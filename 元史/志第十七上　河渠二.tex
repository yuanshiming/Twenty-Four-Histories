\article{志第十七上 河渠二}

\begin{pinyinscope}

 ○黃河



 黃河之水,其源遠而高,其流大而疾,其為患於中國者莫甚焉,前史載河決之患詳矣。



 世祖至元九年七月,衛輝路新鄉縣廣盈倉南河北岸決五十餘步。八月,又崩一百八十三步,其勢未已,去倉止三十步。於是委都水監丞馬良弼與本路官同詣相視,差丁夫並力修完之。二十五年,汴梁路陽武縣諸處河決二十二所,漂蕩麥禾房舍,委宣慰司督本路差夫修治。



 成宗大德三年五月,河南省言:「河決蒲口兒等處,浸歸德府數郡,百姓被災,差官修築計料,合修七堤二十五處,共長三萬九千九十二步,總用葦四十萬四千束,徑尺樁二萬四千七百二十株,役夫七千九百二人。」



 武宗至大三年十一月,河北河南道廉訪司言:



 黃河決溢,千里蒙害,浸城郭,漂室廬壞禾稼,百姓已罹其毒。然後訪求修治之方,而且眾議紛紜,互陳利害,當事者疑惑不決,必須上請朝省,比至議定,其害滋大,所謂不預已然之弊。大抵黃河伏槽之時,水勢似緩,觀之不足為害,一遇霖潦,湍浪迅猛。自孟津以東,土性疏薄,兼帶沙滷,又失導洩之方,崩潰決溢,可翅足而待。



 近歲亳、潁之民,幸河北徙,有司不能遠慮,失於規畫,使陂濼悉為陸地。東至杞縣三水義口,播河為三,分殺其勢,蓋亦有年。往歲歸德、大康建言,相次湮塞南北二水義,遂使三河之水合而為一,下流既不通暢,自然上溢為災。由是觀之,是自奪分洩之利,故其上下決溢,至今莫除。即今水勢趨下,有復鉅野、梁山之意,蓋河性遷徙無常,茍不為遠計預防,不出數年,曹、濮、濟、鄆蒙害必矣。



 今之所謂治水者,徒爾議論紛紜,咸無良策,水監之官,既非精選,知河之利害者百無一二。雖每年累驛而至,名為巡河,徒應故事,問地形之高下,則懵不知;訪水勢之利病,則非所習。既無實才,又不經練。乃或妄興事端,勞民動眾,阻逆水性,翻為後患。為今之計,莫若於汴梁置都水分監,妙選廉幹、深知水利之人,專職其任,量存員數,頻為巡視,謹其防護,可疏者疏之,可堙者堙之,可防者防之。職掌既專,則事功可立。較之河已決溢,民已被害,然後鹵莽修治以勞民者,烏可同日而語哉?



 於是省令都水監議,檢照大德十年正月省臣奏準,昨都水監升正三品,添官二員,鑄分監印,巡視御河,修缺潰,疏淺澀,禁民船越次亂行者,今擬就令分巡提點修治。本監議:「黃河泛漲,止是一事,難與會通河有壩闡漕運分監守治為比。先為御河添官降印,兼提點黃河,若使專一,分監在彼,則有妨御河公事。況黃河已有拘該有司正官提調,自今莫若分監官吏以十月往,與各處官司巡視缺破,會計工物督治,比年終完,來春分監新官至,則一一交割,然後代還,庶不相誤。」



 工部照大德九年黃河決徙,逼近汴梁,幾至浸沒。本處官司權宜開闢董盆口,分入巴河,以殺其勢,遂使正河水緩,並趨支流。緣巴河舊隘不足吞伏,明年急遣蕭都水等閉塞,而其勢愈大,卒無成功,致連年為害,南至歸德諸處,北至濟寧地分,至今不息。本部議:「黃河為害,難同餘水,欲為經遠之計,非用通知古今水利之人專任其事,終無補益。河南憲司所言詳悉,今都水監別無他見,止依舊例議擬未當。如量設官,精選廉干奉公、深知地形水勢者,專任河防之職,往來巡視,以時疏塞,庶可除害。」省準令都水分監官專治河患,任滿交代。



 仁宗延祐元年八月,河南等處行中書省言:「黃河涸露舊水泊污池,多為勢家所據,忽遇泛溢,水無所歸,遂致為害。由此觀之,非河犯人,人自犯之。擬差知水利都水監官,與行省廉訪司同相視,可以疏闢堤障,比至泛溢,先加修治,用力少而成功多。又汴梁路睢州諸處,決破河口數十,內開封縣小黃村計會月堤一道,都水分監修築障水堤堰,所擬不一。宜委請行省官與本道憲司、汴梁路都水分監官及州縣正官,親歷按驗,從長講議。」由是委太常丞郭奉政、前都水監丞邊承務、都水監卿朵兒只、河南行省石右丞、本道廉訪副使站木赤、汴梁判官張承直,上自河陰,下至陳州,與拘該州縣官一同沿河相視。開封縣小黃村河口,測量比舊淺減六尺。陳留、通許、太康舊有蒲葦之地,後因閉塞西河、塔河諸水口,以便種蒔,故他處連年潰決。各官公議:「治水之道,惟當順其性之自然。嘗聞大河自陽武、胙城由白馬河間東北入海,歷年既久,遷徙不常。每歲泛溢兩岸,時有沖決,強為閉塞,正及農忙,科樁梢,發丁夫,動至數萬,所費不可勝紀,其弊多端,郡縣嗷嗷,民不聊生。蓋黃河善遷徙,惟宜順下疏洩。今相視上自河陰,下抵歸德,經夏水漲,甚於常年,以小黃口分洩之故,並無沖決,此其明驗也。詳視陳州,最為低窪,瀕河之地,今歲麥禾不收,民饑特甚,欲為拯救,奈下流無可疏之處。若將小黃村河口閉塞,必移患鄰郡;決上流南岸,則汴梁被害;決下流北岸,則山東可憂。事難兩全,當遺小就大。如免陳村差稅,賑其饑民,陳留、通許、太康縣被災之家,依例取勘賑恤,其小黃村河口仍舊通流外,據修築月堤,並障水堤,閉河口,別難擬議。」於是凡汴梁所轄州縣河堤,或已修治,及當疏通與補築者,條列具備。



 至五年正月,河北河南道廉訪副使奧屯言:「近年河決杞縣小黃村口,滔滔南流,莫能御遏,陳、潁瀕河膏腴之地浸沒,百姓流散。今水迫汴城,遠無數里,儻值霖雨水溢,倉卒何以防禦!方今農隙,宜為講究,使水歸故道,達於江、淮,不惟陳、潁之民得遂其生,竊恐將來浸灌汴城,其害匪輕。」於是大司農司下都水監移文汴梁分監修治,自六年二月十一日興工,至三月九日工畢,總計北至槐疙疸兩舊堤,南至窯務汴堤,通長二十里二百四十三步。創修護城堤一道,長七千四百四十三步,下地修堤,下廣十六步,上廣四步,高一丈,六十尺為一工。堤東二十步外取土,內河溝七處,深淺高下闊狹不一,計工二十五萬三千六百八十,用夫八千四百五十三,除風雨妨工,三十日畢。內流水河溝,南北闊二十步,水深五尺。河內修堤,底闊二十四步,上廣八步,高一丈五尺,積十二萬尺,取土稍遠,四十尺為一工,計三萬工,用夫百人。每步用大樁二,計四十,各長一丈二尺,徑四寸。每步雜草千束,計二萬。每步簽樁四,計八十,各長八尺,徑三寸。水手二十,木匠二,大船二艘,梯钁一副,繩索畢備。



 七年七月,汴梁路言:「滎澤縣六月十一日河決塔海莊東堤十步餘,橫堤兩重,又缺數處。二十三日夜,開封縣蘇村及七裏寺復決二處。」本省平章站馬赤親率本路及都水監官,並工修築,於至治元年正月興工,修堤岸四十六處,該役一百二十五萬六千四百九十四工,凡用夫三萬一千四百一十三人。



 文宗至順元年六月,曹州濟陰縣河防官本縣尹郝承務言:「六月五日,魏家道口黃河舊堤將決,不可修築,以此差募民夫,創修護水月堤,東西長三百九步,下闊六步,高一丈。又緣水勢瀚漫,復於近北築月堤,東西長一千餘步,下廣九步,其功未竟。至二十一日,水忽泛溢,新舊三堤一時咸決,明日外堤復壞,急率民閉塞,而湍流迅猛,有蛇時出沒於中,所下樁土,一掃無遺。又舊堤歲久,多有缺壞,差夫並工築成二十餘步。其魏家道口缺堤,東西五百餘步,深二丈餘,外堤缺口,東西長四百餘步。又磨子口護水堤,低薄不足禦水,東西長一千五百步。魏家道口卒未易修,先差夫補築。磨子口七月十六日興工,二十八日工畢。二十二日,按視至硃從馬頭西,舊堤缺壞,東西長一百七十餘步,計料堤外貼築五步,增高一丈二尺,與舊堤等,上廣二步。於磨子口修堤夫內,摘差三百一十人,於是月二十三日入役,至閏七月四日工畢。」郝承務又言:「魏家道口磚固等村,缺破堤堰,累下樁土,沖洗不存,若復閉築,緣缺堤周回皆泥淖,人不可居,兼無取土之處。又沛郡安樂等保,去歲旱災,今復水澇,漂禾稼,壞室廬,民皆缺食,難於差倩。其不經水害村保民人,先已遍差補築黃家橋、磨子口諸處堤堰,似難重役。如候秋涼水退,倩夫修理,庶蘇民力。今沖破新舊堤七處,共長一萬二千二百二十八步,下廣十二步,上廣四步,高一丈二尺,計用夫六千三百四人,樁九百九十,葦箔一千三百二十,草一萬六千五束。六十尺為一工,無風雨妨工,度五十日可畢。」本縣準言,至八月三十日差夫二千四百二十,關請郝承務督役。郝承務又言:「九月三日興工修築,至十八日大風,十九日雨,二十四日復雨,緣此辛馬頭、孫家道口障水堤堰又壞,計工役倍於元數,移文本縣,添差二千人同築。二十六日,元與成武定、陶二縣分築魏家道口八百二十步修完。十月二日,至辛馬頭、孫家道口,從實又量元缺堤,南北闊一百四十步,內水地五十步,深者至二丈,淺者不下八九尺,依元料用樁箔補築,至七日完。又於本處創築月堤一道,西北東南斜長一千六百二十七步,內成武、定陶分築一百五十步,實築一千四百七十七步,外有元料堌頭魏家道口外堤未築。即欲興工,緣冬寒土凍,擬候來春,並工修理,官民兩便。



 濟州河



 濟州河者,新開以通漕運也。世祖至元十七年七月,耿參政、阿里尚書奏:「為姚演言開河事,令阿合馬與耆舊臣集議,以鈔萬錠為傭直,仍給糧食。」世祖從之。十八年九月,中書丞相火魯火孫等奏:「姚總管等言,請免益都、淄萊、寧海三州一歲賦,入折傭直,以為開河之用。平章阿合馬與諸老臣議,以為一歲民賦雖多,較之官給傭直,行之甚便。」遂從之。十月,火魯火孫等奏:「阿八失所開河,經濟州,而其地又有一河,傍有民田,開之甚便。臣等議,若開此河,阿八失所管一方屯田,宜移之他處,不阻水勢。」世祖令移之。十二月,差奧魯赤、劉都水及精算數者一人,給宣差印,往濟州,定開河夫役,令大名、衛州新附軍亦往助工。



 三十一年,御史臺言:「膠、萊海道淺澀,不能行舟。」臺官玉速帖木兒奏:「阿八失所開河,省遣牙亦速失來,謂漕船泛河則失少,泛海則損多。」既而漕臣囊加鷿、萬戶孫偉又言:「漕海舟疾且便。」右丞麥術丁又奏:「斡奴兀奴鷿凡三移文,言阿八失所開河,益少損多,不便轉漕。水手軍人二萬,舟千艘,見閑不用,如得之,可歲漕百萬石。昨奉旨,候忙古鷿來共議,海道便,則阿八失河可廢。今忙古鷿已自海道運糧回,有一二南人自願運糧萬石,已許之。」囊加鷿、孫萬戶復請用軍驗試海運,省院官暨眾議:「阿八失河揚用水手五千、軍五千、船千艘,畀揚州省教習漕運。今擬以此水手軍人,就用平灤船,從利津海漕運。」世祖從之。阿八失所開河遂廢。



 滏河



 滏河者,引滏水以通洺州城濠者也。



 至元五年十月,洺磁路言:「洺州城中,井泉咸苦,居民食用,多作疾,且死者眾。請疏滌舊渠,置壩閘,引滏水分灌洺州城濠,以濟民用。計會河渠東西長九百步,闊六尺,深三尺,二尺為工,役工四百七十五,民自備用器,歲二次放閘,且不妨漕事。」中書省準其言。



 廣濟渠



 廣濟渠在懷孟路,引沁水以達於河。世祖中統二年,提舉王允中、大使楊端仁奉詔開河渠,凡募夫千六百五十一人,內有相合為夫者,通計使水之家六千七百餘戶,一百三十餘日工畢。所修石堰,長一百餘步,闊三十餘步,高一丈三尺。石斗門橋,高二丈,長十步,闊六步。渠四道,長闊不一,計六百七十七里,經濟源、河內、河陽、溫、武陟五縣,村坊計四百六十三處,渠成甚益於民,名曰廣濟。三年八月,中書省臣忽魯不花等奏:「廣濟渠司言,沁水渠成,今已驗工分水,恐久遠權豪侵奪。」乃下詔依本司所定水分,已後諸人毋得侵奪。



 至文宗天歷三年三月,懷慶路同知阿合馬言:「天久亢旱,夏麥枯槁,秋穀種不入土,民匱於食。近因訪問耆老,咸稱丹水澆溉近山田土,居民深得其利,有沁水亦可溉田,中統間王學土亦為天旱,奉詔開此渠,募自願人戶,於太行山下沁口古跡,置分水渠口,開浚大河四道,歷溫、陟入黃河,約五百餘里,渠成名曰廣濟。設官提調,遇旱則官為斟酌,驗工多寡,分水澆溉,濟源、河內、河陽、溫、武陟五縣民田三千餘頃咸受其賜。二十餘年後,因豪家截河起堰,立碾磨,壅遏水勢,又經霖雨,渠口淤塞,堤堰頹圮。河渠司尋亦革罷,有司不為整治,因致廢壞。今五十餘年,分水渠口及舊渠跡,俱有可考,若蒙依前浚治,引水溉田,於民大便。可令河陽、河內、濟源、溫、武陟五縣,使水人戶自備工力,疏通分水渠口,立閘起堰,仍委諳知水利之人,多方區畫。遇旱,視水緩急,撤閘通流,驗工分水以灌溉;若霖雨泛漲,閉閘退還正流。禁治不得截水置碾磨,栽種稻田。如此,則澇旱有備,民樂趨利。請移文孟州、河內、武陟縣委官講議。」尋據孟州等處申,親詣沁口,咨詢耆老,言舊日沁水正河內築土堰,遮水入廣濟渠,岸北雖有減水河道,不能吞伏,後值霖雨,蕩沒田禾,以此堵閉。今若枋口上連土岸,及於沁水正河置立石堰,與枋口相平,如遇水溢,閉塞閘口,使水漫流石堰,復還本河,又從減水河分殺其勢,如此庶不為害。約會河陽、武陟縣尹與耆老等議,若將舊廣濟渠依前開浚,減水河亦增開深闊,禁安磨碾,設立閘堰,自下使水,遇旱放閘澆田,值澇閉閘退水,公私便益。懷慶路備申工部牒,都水監回文本路,委官相視施行。



 三白渠



 京兆舊有三白渠,自元伐金以來,渠堰缺壞,土地荒蕪。陜西之人雖欲種蒔,不獲水利,賦稅不足,軍興乏用。太宗之十二年,梁泰奏:「請差撥人戶牛具一切種蒔等物,修成渠堰,比之旱地,其收數倍,所得糧米,可以供軍。」太宗準奏,就令梁泰佩元降金牌,充宣差規措三白渠使,郭時中副之,直隸朝廷,置司於雲陽縣;所用種田戶及牛畜,別降旨,付塔海紺不於軍前應副。是月,敕喻塔海紺不:「近梁泰奏修三白渠事,可於汝軍前所獲有妻少壯新民,量撥二千戶,及木工二十人,官牛內選肥腯齒小者一千頭,內乳牛三百,以畀梁泰等。如不敷,於各千戶、百戶內貼補,限今歲十一月內交付數足,趁十二月入工。其耕種之人,所收之米,正為接濟軍糧。如發遣人戶之時,或闕少衣裝,於各千戶、百戶內約量支給,差軍護送出境,沿途經過之處,亦為防送,毋致在逃走逸,驗路程給以行糧,大口一升,小者半之。」



 洪口渠



 洪口渠在奉元路。英宗至治元年十月,陜西屯田府言:



 自秦、漢至唐、宋,年例八月差使水戶,自涇陽縣西仲山下截河築洪堰,改涇水入白渠,下至涇陽縣北白公斗門,分為三限,並平石限,蓋五縣分水之要所。北限入三原、櫟陽、雲陽,中限入高陵,南限入涇陽,澆溉官民田七萬餘畝。近至大三年,陜西行臺御史王承德言,涇陽洪口展修石渠,為萬世之利。由是會集奉元路三原、涇陽、臨潼、高陵諸縣,洎涇陽、渭南、櫟陽諸屯官及耆老議,如準所言,展修石渠八十五步,計四百二十五尺,深二丈,廣一丈五尺,計用石十二萬七千五百尺,人日採石積方一尺,工價二兩五錢,石工二百,丁夫三百,金火匠二,用火焚水淬,日可鑿石五百尺,二百五十五日工畢。官給其糧食用具,丁夫就役使水之家,顧匠傭直使水戶均出。陜西省議,計所用錢糧,不及二年之費,可謂一勞永逸,準所言便。都省準委屯田府達魯花赤只裏赤督工,自延祐元年二月十日發夫匠入役,至六月十九日委官言,石性堅厚,鑿僅一丈,水泉湧出,近前續展一十七步,石積二萬五千五百尺,添夫匠百人,日鑿六百尺,二百四十二日可畢。



 文宗天歷二年三月,屯田總管兼管河渠司事郭嘉議言:「去歲六月三日驟雨,涇水泛漲,無修洪堰及小龍口盡圮,水歸涇,白渠內水淺,為此計用十四萬九千五百十一工,役丁夫一千六百,度九十三日畢。於使水戶內差撥,每夫就持麻一斤,鐵一斤,系囤取泥索各一,長四十尺,草苫一,長七尺,厚二寸。」陜西省準屯田府照,洪口自秦至宋一百二十激,經由三限,自涇陽下至臨潼五縣,分流澆溉民田七萬餘頃,驗田出夫千六百人,自八月一日修堰,至十月放水溉田,以為年例。近因奉元亢旱,五載失稔,人皆相食,流移疫死者十七八。今差夫又令就出用物,實不能辦集。竊詳涇陽水利,雖分三限引水溉田,緣三原等縣地理遙遠,不能依時周遍,涇陽北近,俱在上限,並南限中限,用水最便。今次修堰,除見在戶依例差役,其逃亡之家合出夫數,宜令涇陽縣近限水利戶添差一人,官日給米一升,並工修治。省準出鈔八百錠,委耀州同知李承事,洎本府總管郭嘉議及各處正官,計工役照時直糴米給散。李承事督夫修築,至十一月十六日畢。



 揚州運河



 運河在揚州之北,宋時嘗設軍疏滌,世祖取宋之後,河漸壅塞。至元末年,江淮行省嘗以為言,雖有旨浚治,有司奉行,未見實效。



 仁宗延祐四年十一月,兩淮運司言:「鹽課甚重,運河淺澀無源,止仰天雨,請加修治。」明年二月,中書移文河南省,選官洎運司有司官相視,會計工程費用。於是河南行省委都事張奉政及淮東道宣慰司官、運司官,會州縣倉場官,遍歷巡視,集議:河長二千三百五十里,有司差瀕河有田之家,顧倩丁夫,開修一千八百六十九里;倉場鹽司不妨辦課,協濟有司,開修四百八十二里。運司言:「近歲課額增多,而船灶戶日益貧苦,宜令有司通行修治,省減官錢。」省臣奏準:諸色戶內顧募丁夫萬人,日支鹽糧錢二兩,計用鈔二萬錠,於運司鹽課及減駁船錢內支用。差官與都水監、河南行省、淮東宣慰司官專董其事,廉訪司體察,樞密院遣官鎮遏,乘農隙並工疏治。



 練湖



 練湖在鎮江。元有江南之後,豪勢之家於湖中築堤圍田耕種,侵占既廣,不足受水,遂致泛溢。世祖末年,參政暗都剌奏請依宋例,委人提調疏治,其侵占者驗畝加賦。



 至治三年十二月,省臣奏:「江浙行省言,鎮江運河全藉練湖之水為上源,官司漕運,供億京師,及商賈販載,農民來往,其舟楫莫不由此。宋時專設人夫,以時修浚。練湖瀦蓄潦水,若運河淺阻,開放湖水一寸,則可添河水一尺。近年淤淺,舟楫不通,凡有官物,差民運遞,甚為不便。委官相視,疏治運河,自鎮江路至呂城壩,長百三十一里,計役夫萬五百十三人,六十日可畢。又用三千餘人浚滌練湖,九十日可完,人日支糧三升、中統鈔一兩。行省、行臺分官監督。所用船物,今歲預備,來春興工。合行事宜,依江浙行省所擬。」既得旨,都省移文江浙行省,委參政董中奉率合屬正官親臨督役。於是董中奉言:「所委前都水少監崇明州知州任奉政、鎮江路總管毛中議等議:練湖、運河此非一事,宜依假山諸湖農民取泥之法,用船千艘,船三人,用竹褷撈取淤泥,日可三載,月計九萬載,三月之間,通取二十七萬載,就用所取泥增築湖岸。自鎮江在城程公壩,至常州武進縣呂城壩,河長百三十一里一百四十六步,擬開河面闊五丈,底闊三丈,深四尺,與見有水二尺,可積深六尺。所役夫於平江、鎮江、常州、江陰州及建康路所轄溧陽州田多上戶內差倩。若浚湖開河,二役並興,卒難辦集。宜趁農隙,先開運河,工畢就浚練湖。」省準所言,與都事王徵事等於泰定元年正月至鎮江丹陽縣,洎各監工官沿湖相視,上湖沙岡黃土,下湖茭根叢雜,泥亦堅硬,不可褷取。又議兩役並興,相離三百餘里,往來監督,供給為難,願以所督夫一萬三千五百十二人,先開運河,期四十七日畢,次浚練湖,二十日可完。繼有江南行臺侍御史及浙西廉訪司副使俱至,乃議首事運河,備文咨稟,遂於是月十七日入役。



 二月十八日,省臣奏:「開浚運河、練湖,重役也,宜依行省所議,仍令便宜從事。」後各監工官言:「已分運河作三壩,依元料深闊丈尺開浚,至三月四日工畢。數內平江昆山、嘉定二州,實役二十六日,常熟、吳江二州,長洲、吳縣,實役二十八日,餘皆役三十日,已於三月七日積水行舟。」又監修練湖官言:「任奉議指劃元料,增築堤堰及舊有土基,共增闊一丈二尺,平面至高底灘腳,增築共量斜高二丈五尺。依中堰西石達東舊堤臥羊灘修築,如舊堤高闊已及所料之上者,遇有崩缺,修築令完。中堰西石達至五百婆堤西上增高土一尺,有缺亦補之。五百婆堤至馬林橋堤水勢稍緩,不須修治,其堤底間有滲漏者,窒塞之。三月六日破土,九日入役,至十一日工畢,實役三日。歸勘任少監元料,開運河夫萬五百十三人,六十日畢,浚練湖夫三千人,九十日畢,人日支鈔一兩、米三升,共該鈔萬八千一十四錠二十兩,米二萬七千二十一石六斗,實征夫萬三千五百十二人,共役三十三日,支鈔八千六百七十九錠三十六兩,糧萬三千十九石五斗八升。比附元料,省鈔九千三百三十四錠三十四兩,糧萬四千二石二升。其練湖未畢,相視地形水勢再議。」



 參政董中奉又言:「練湖舊有湖兵四十三人,添補五十七名,共百人,於本路州縣苗糧三石之下、二石之上差充,專任修築湖岸,設提領二員、壕寨二人、司吏三人,於有出身人內選用。」工部議:「練湖所設提領人等印信,即同湖兵,宜咨本省遍行議擬。」又鎮江路言:「河、練湖今已開浚,若不設法關防,徒勞民力。除關本路達魯花赤兀魯失海牙總治其事,同知哈散、知事程郇專管啟閉斗門。」行省從之。



 吳松江



 浙西諸山之水受之太湖,下為吳松江,東匯澱山湖以入海,而潮汐來往,逆湧濁沙,上涇河口,是以宋時設置撩洗軍人,專掌修治。元既平宋,軍士罷散,有司不以為務,勢豪租占為蕩為田,州縣不得其人,輒行許準,以致湮塞不通,公私俱失其利久矣。



 至治三年,江浙省臣方以為言,就委嘉興路治中高朝列、湖州路知事丁將仕同本處正官,體究舊曾疏浚通海故道,及新生沙漲礙水處所,商度開滌圖呈。據丁知事等官按視講究,合開浚河道五十五處。內常熟州九處,十三段,該工百三十二萬一千五百六十二,昆山州十一處,九十五里,用工二萬七千四百,役夫四百五十六,宜於本州有田一頃之上戶內,驗田多寡,算量里步均派,自備糧赴功疏浚。正月上旬興工,限六十日工畢,二年一次舉行。嘉定州三十五處,五百三十八里,該工百二十六萬七千五十九,日支糧一升,計米萬二千六百七十石五斗九升,日役夫二萬一千一百一十七,六十日畢。工程浩大,米糧數多,乞依年例,勸率附河有田用水之家,自備口糧,佃戶傭力開浚。奈本州連年被災,今歲尤甚,力有不逮,宜從上司區處。高治中會集松江府各州縣官按視,議合浚河渠,華亭縣九處,計五百二十八里,該工九百六十八萬四千八百八十二,役夫十六萬一千四百一十四,人日支糧二升,計米十九萬三千六百九十七石六斗四升。上海縣十四處,計四百七十一里,該工千二百三十六萬八千五十二,日役夫二萬六千一百三十四,人日支糧二升,計二十四萬七千三百六十一石四升,六十日工畢。官給之糧,傭民疏治。如下年豐稔,勸率有田之家,五十畝出夫一人,十畝之上驗數合出,止於本保開浚。其權勢之家,置立魚籪並沙塗栽葦者,依上出夫。



 其上海、嘉定連年旱澇,皆緣河口湮塞,旱則無以灌溉,澇則不能疏洩,累致兇歉,官民俱病。至元三十年以後,兩經疏闢,稍得豐稔。比年又復壅閉,勢家愈加租占,雖得征賦,實失大利。上海縣歲收官糧一十七萬石,民糧三萬餘石,略舉似延祐七年災傷五萬八千七百餘石,至治元年災傷四萬九千餘石,二年十萬七千餘石,水旱連年,殆無虛歲,不惟虧欠官糧,復有賑貸之費。近委官相視地形,講議疏浚,其通海大江,未易遽治;舊有河港聯絡官民田土之間、藉以灌溉者,今皆填塞,必須疏通,以利耕種。欲令有田人戶自為開浚,而工役浩繁,民力不能獨成。由是議,上海、嘉定河港,宜令本處所管軍民站灶僧道諸色有田者,以多寡出夫,自備糧修治,州縣正官督役。其豪勢租占蕩田、妨水利者,並與除闢。本處民田稅糧全免一年,官租減半。今秋收成,下年農隙舉行,行省、行臺、廉訪司官巡鎮。外據華亭、昆山、常熟州河港,比上海、嘉定緩急不同,難為一體,從各處勸農正官督有田之家,備糧並工修治。若遽興工,陰陽家言癸亥年動土有忌,預為咨稟可否。



 至泰定元年十月十九日,右丞相旭邁傑等奏:「江浙省言,吳松江等處河道壅塞,宜為疏滌,仍立閘以節水勢。計用四萬餘人,今歲十二月為始,至正月終,六十日可畢,用二萬餘人,二年可畢。其丁夫於旁郡諸色戶內均差,依練湖例,給傭直糧食,行省、行臺、廉訪司並有司官同提調。臣等議,此事官民兩便,宜從其請。若丁夫有餘,止令一年畢。命脫歡答剌罕諸臣同提調,專委左丞朵兒只班及前都水任少監董役。」得旨,移文行省,準擬疏治。江浙省下各路發夫入役,至二年閏正月四日工畢。



 澱山湖



 太湖為浙西巨浸,上受杭、湖諸山之水,瀦蓄之餘,分匯為澱山湖,東流入海。



 世祖末年,參政暗都剌言:「此湖在宋時委官差軍守之,湖旁餘地,不許侵占,常疏其壅塞,以洩水勢。今既無人管領,遂為勢豪絕水築堤,繞湖為田,湖狹不足瀦蓄,每遇霖潦,泛溢為害。昨本省官忙古鷿等興言疏治,因受曹總管金而止。張參議、潘應武等相繼建言,識者咸以為便。臣等議,此事可行無疑。然雖軍民相參,選委廉幹官提督,行省山住子、行院董八都見子、行臺哈剌鷿令親詣相視,會計合用軍夫擬稟。」世祖曰:「利益美事,舉行已晚,其行之。」既而平章鐵哥言:「委官相視,計用夫十二萬,百日可畢。昨奏軍民共役,今民丁數多,不須調軍。」世祖曰:「有損有益,咸令均齊,毋自疑惑,其均科之。」



 至元三十一年,世祖崩,成宗即位。平章鐵哥奏:「太湖、澱山湖昨嘗奏過先帝,差倩民夫二十萬疏掘已畢。今諸河日受兩潮,漸致沙漲,若不依舊宋例,令軍屯守,必致坐隳成功。臣等議,常時工役撥軍,樞府猶且吝惜,屯守河道用軍八千,必辭不遣。澱山湖圍田賦糧二萬石,就以募民夫四千,調軍士四千與同屯守。立都水防田使司,職掌收捕海賊,修治河渠圍田。」命伯顏察兒暨樞密院議畢聞奏。於是樞府言:「嘗奏澱山湖在宋時設軍屯守,範殿帥、硃、張輩必知其故,擬與省官集議定稟奏,有旨從之。乃集樞府官及範殿帥等共議,硃、張言:『宋時屯守河道,用手號軍,大處千人,小處不下三四百,隸巡檢司管領。』範殿帥言:『差夫四千,非動搖四十萬戶不可,若令五千軍屯守,就委萬戶一員提調,事或可行。』臣等亦以為然,與都水巡防萬戶府職名,俾隸行院。」樞府官又言:「若與知源委之人詢其詳,候至都定議。」從之。



 鹽官州海塘



 鹽官州去海岸三十里,舊有捍海塘二,後又添築咸塘,在宋時亦嘗崩陷。成宗大德三年,塘岸崩,都省委禮部郎中游中順,洎本省官相視,虛沙復漲,難於施力。至仁宗延祐己未、庚申間,海汛失度,累壞民居,陷地三十餘里。其時省憲官共議,宜於州後北門添築土塘,然後築石塘,東西長四十三里,後以潮汐沙漲而止。



 至泰定即位之四年二月間,風潮大作,沖捍海小塘,壞州郭四里。杭州路言:「與都水庸田司議,欲於北地築塘四十餘里,而工費浩大,莫若先修咸塘,增其高闊,填塞溝港,且浚深近北備塘濠塹,用樁密釘,庶可護禦。」江浙省準下本路修治。都水庸田司又言:「宜速差丁夫,當水入沖堵閉,其不敷工役,於仁和、錢塘及嘉興附近州縣諸色人戶內斟酌差倩,即日淪沒不已,旦夕誠為可慮。」工部議:「海岸崩摧重事也,宜移文江浙行省,督催庸田使司、鹽運司及有司發丁夫修治,毋致侵犯城郭,貽害居民。」五月五日,平章禿滿迭兒、茶乃、史參政等奏:「江浙省四月內,潮水沖破鹽官州海岸,令庸田司官征夫修堵,又令僧人誦經,復差人令天師致祭。臣等集議,世祖時海岸嘗崩,遣使命天師祈祀,潮即退,今可令直省舍人伯顏奉御香,令天師依前例祈祀。」制曰:「可。」既而杭州路又言:「八月以來,秋潮洶湧,水勢愈大,見築沙地塘岸,東西八十餘步,造木櫃石囤以塞其要處。本省左丞相脫歡等議,安置石囤四千九百六十,抵御鎪嚙,以救其急,擬比浙江立石塘,可為久遠。計工物,用鈔七十九萬四千餘錠,糧四萬六千三百餘石,接續興修。」



 致和元年三月,省臣奏:「江浙省並庸田司官修築海塘,作竹蘧篨,內實以石,鱗次壘疊以御潮勢,今又淪陷入海,見圖修治,倘得堅久之策,移文具報。臣等集議,此重事也,旦夕駕幸上都,分官扈從,不得圓議。今差戶部尚書李家奴、工部尚書李嘉賓、樞密院屬衛指揮青山、副使洪灝、宣政僉院南哥班與行省左丞相脫歡及行臺、行宣政院、庸田使司諸臣,會議修治之方。合用軍夫,除戍守州縣關津外,酌量差撥,從便添支口糧。合役丁力,附近有田之民,及僧、道、也裏可溫、答失蠻等戶內點倩。凡工役之時,諸人毋或沮壞,違者罪之。合行事務,提調官移文稟奏施行。」有旨從之。四月二十八日,朝廷所委官,洎行省臺院及庸田司等官議:「大德、延祐欲建石塘未就。泰定四年春,潮水異常,增築土塘,不能抵御,議置板塘,以水湧難施工,遂作蘧篨木櫃,間有漂沉,欲踵前議,疊石塘以圖久遠。為地脈虛浮,比定海、浙江、海鹽地形水勢不同,由是造石囤於其壞處疊之,以救目前之急。已置石囤二十九里餘,不曾崩陷,略見成效。」庸田司與各路官同議,東西接壘石囤十里,其六十里塘下舊河,就取土築塘,鑿東山之石以備崩損。



 文宗天歷元年十一月,都水庸田司言:「八月十日至十九日,正當大汛,潮勢不高,風平水穩。十四日,祈請天妃入廟,自本州嶽廟東海北護岸鱗鱗相接。十五日至十九日,海岸沙漲,東西長七里餘,南北廣或三十步,或數十百步,漸見南北相接。西至石囤,已及五都,修築捍海塘與鹽塘相連,直抵巖門,障御石囤。東至十一都六十里塘,東至東大尖山嘉興、平湖三路所修處海口。自八月一日至二日,探海二丈五尺;至十九日、二十日探之,先二丈者今一丈五尺,先一丈五尺者今一丈。西自六都仁和縣界赭山、雷山為首,添漲沙塗,已過五都四都,鹽官州廊東西二都,沙土流行,水勢俱淺。二十日,復巡視自東至西岸腳漲沙,比之八月十七日漸增高闊。二十七日至九月四日大汛,本州嶽廟東西,水勢俱淺,漲沙東過錢家橋海岸,元下石囤木植,並無頹圮,水息民安。」於是改鹽官州曰海寧州。



 龍山河道



 龍山河在杭州城外,歲久淤塞。武宗至大元年,江浙省令史裴堅言:「杭州錢塘江,近年以來為沙塗壅漲,潮水遠去,離北岸十五里,舟楫不能到岸。商旅往來,募夫搬運十七八里,使諸物翔湧,生民失所,遞運官物,甚為煩擾。訪問宋時並江岸有南北古河一道,名龍山河,今浙江亭南至龍山閘約一十五里,糞壞填塞,兩岸居民間有侵占。跡其形勢,宜改修運河,開掘沙土,封閘搬載,直抵浙江,轉入兩處市河,免擔負之勞,生民獲惠。」省下杭州路相視,錢塘縣城南上隅龍山河至橫河橋,委系舊河,居民侵占,起建房屋,若疏通以接運河,公私大便。計工十五萬七千五百六十六,日役夫五千二百五十二,度可三十日畢。所役夫於本路錄事司、仁和、錢塘縣富實之家差倩,就持筐簷鍬钁應役。人日支官糧二升,該米三千一百五十一石三斗二升。河長九里三百六十二步,造石橋八,立上下二閘,計用鈔一百六十三錠二十三兩四錢七分七厘。省準咨請丞相脫脫總治其事,於仁宗延祐三年三月七日興工,至四月十八日工畢。



\end{pinyinscope}