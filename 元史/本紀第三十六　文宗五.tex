\article{本紀第三十六 文宗五}

\begin{pinyinscope}

 三年春正月辛未朔,高麗國王楨遣其臣元忠奉表稱賀,貢方物。癸酉,命高麗國王王燾仍為高麗國王,賜金印。初必然真理德國萊布尼茨用語。即理性真理。詳見「理性,燾有疾,命其子楨襲王爵,至是燾疾愈,故復位。甲戌,賜燕鐵木兒妻公主月魯金五百兩、銀五千兩。丁丑,禁冒哀求敘復者。賑糶米五萬石,濟京師貧民。己卯,時享太廟。罷諸建造工役,惟城郭、河渠、橋道、倉庫勿禁。廣西羅偉里叛寇馬武沖等,合龍州嶺北朗龍洞韋大蟲賊兵萬人,攻陷那馬違、那馬安等寨,命廣西宣慰司嚴軍御之。月闕察兒冒請衛士芻束,當坐罪,燕鐵木兒請釋之。壬午,命甘肅行省為豳王不顏帖木兒建居第。封孔子妻鄆國夫人勣官氏為大成至聖文宣王夫人。癸未,給納鄰等十四驛糧及芻粟。賑永昌路流民。慶遠南丹等處溪洞軍民安撫司言,所屬宜山縣饑疫,死者眾,乞以給軍積穀二百八十石賑糶,從之。江西行省言,梅州頻年水旱,民大饑,命發粟七百石以賑糶。丙戌,印造歲額鈔本,至元鈔九十九萬六千錠,中統鈔四千錠。丁亥,幸大承天護聖寺。賜諸王帖木兒及其妃阿剌赤八剌金五百兩、銀萬兩、鈔二萬錠、幣帛各千匹。監察御史劾奏:「翰林學士承旨典哈,其兄野里牙坐誅,當罷。」從之。戊子,萬安軍黎賊王奴羅等,集眾五萬人寇陵水縣。己丑,賑肇慶路高要縣饑民九千五百四十口。四川行省言:「去年九月,左丞帖木兒不花與祿餘賊兵戰被創,賊遂侵境,乞調重慶、敘州兵二千五百人往救之。」順元宣撫司亦言:「賊列行營為十六所,乞調兵分道備御。」詔上都留守司為燕鐵木兒建居第。御史臺言:「選除云南廉訪司官,多托故不行,繼今有如是者,風憲勿復用。」制可。戊戌,命中書省以鈔三千錠、幣帛各三千匹,給皇子古納答剌歲例鷹犬回賜。諸王章吉獻斡羅思百七十人,酬以銀七十二錠、鈔五千錠。己亥,給斡羅思千人衣糧。山南道廉訪副使禿堅董阿劾:「荊湖北道宣慰使別列怯都常貸內府鈔,威逼部民代償,不足則以宣慰使公帑鈔償之。又,副使驢駒,以修治沿江堤岸,縱家奴掊斂民財。二人罪雖遇赦,宜從黜退。」御史臺臣以聞,從之。庚子,封公主不納為鄆安大長公主。夔路忠信寨洞主阿具什用,合洞蠻八百餘人寇施州。



 二月辛丑朔,八番苗蠻駱度來貢方物。癸卯,諸王也先帖木兒薨。甲辰,諸王答兒馬失里、哈兒蠻各遣使來貢蒲萄酒、西馬、金鴉鶻。乙巳,以湖廣行省平章玥璐不華為陜西行臺御史大夫。給豳王及其王傅祿。戊申,雲南行省言:「會通州土官阿賽及河西阿勒等與羅羅賊兵千五百人寇會州路之卜龍村;又,祿餘將引兵與茫部合寇羅羅斯,截大渡河、金沙江以攻東川、會通等州。臣等敢奉先所降詔書招諭之,不奉命則從宜進軍。」制可。己酉,賜怯薛官完者帖木兒及阿昔兒珠衣帽。德寧路去年旱,復值霜雹,民饑,賑以粟三千石。旌晉寧路沁州劉瑋妻張氏志節。祿餘言於四川行省:「自父祖世為烏撒土官宣慰使,佩虎符,素無異心。曩為伯忽誘脅,比聞朝廷招諭,而今期限已過,乞再降詔赦,即率四路土官出降。仍乞改屬四川省,隸永寧路,冀得休息。」四川行省以聞,詔中書、樞密、御史諸大臣雜議之。己未,旌寧夏路趙那海孝行。辛酉,燕鐵木兒兼奎章閣大學士,領奎章閣學士院事。己巳,命燕鐵木兒集翰林、集賢、太禧宗禋院,議立太祖神御殿。詔修曲阜宣聖廟。邛州有二井,宋舊名曰金鳳、茅池,天歷初,九月地震,鹽水湧溢,州民侯坤願作什器煮鹽而輸課於官,詔四川轉運鹽司主之。旌濟州任城縣王德妻秦氏、婺州路金華縣吳塤妻宋氏、廬州路高仁妻張氏、甘州路岳忽南妻失林、蓋州完顏帖哥住妻李氏志節。



 三月庚午朔,帝師至京師。遣使往西域,賜諸王不賽因繡彩幣帛二百四十匹。中書省臣言:「凡遠戍軍官死而歸葬者,宜視民官例,給道里之費。又,四川驛戶比以軍興消乏,宜遣官同行省量濟之。」制可。燕鐵木兒言:「平江、松江澱山湖圩田方五百頃有奇,當入官糧七千七百石,其總田者死,頗為人占耕。今臣願增糧為萬石入官,令人佃種,以所得餘米贍臣弟撒敦。」從之。洛水溢。爪哇國遣其臣僧伽剌等八十三人,奉金書表及方物來朝貢。己卯,詔:「以西寧王速來蠻鎮御有勞,其如安定王朵兒只班例,置王傅官四人,鑄印給之。」庚辰,以安陸府賜並王晃火兒不花為食邑。旌大都良鄉縣韋安妻張氏貞節。丁亥,諸王伯岳兀、完者帖木兒來朝。戊子,占城國遣其臣阿南那那裏沙等四人,奉金書表及方物來朝貢。己丑,復立功德使司。癸巳,皇子古納答剌更名燕帖古思。置興瑞司,掌中宮歲作佛事,秩正三品。乙未,命燕鐵木兒依舊例以鈔萬錠分給蒙古孤寡者。以帝師泛舟於西山高梁河,調衛士三百挽舟。丙申,賜怯薛官紵憐鐵木兒璽書,申飭其所部。賑木憐、苦鹽濼、札哈、掃憐九驛之貧者凡四百五十二戶。丁酉,緬國遣使者阿落等十人,奉方物來朝貢。己亥,賜行樞密院鈔四萬錠,分給征烏撒、烏蒙所調陜西、四川蒙古軍及漸丁萬人。高唐、德、冀諸州,大名、汴梁、廣平諸路,有蟲食桑葉盡。



 夏四月壬寅,中書省臣言:「去歲宿衛士給鈔者萬五千人,今減去千四百人,餘當給者萬三千六百人。又,太府監歲支幣帛二萬匹,不足於用,請再給二百匹。」並從之。四川師壁、散毛、盤速出三洞蠻野王等二十三人來貢方物。戊申,大寧路地震。四川大盤洞謀者什用等十四人來貢方物。丙辰,諸王不別居法郎遣使者要忽難等,及西域諸王不賽因使者也先帖木兒等,皆來貢方物。戊午,命奎章閣學士院以國字譯《貞觀政要》,鐫板模印,以賜百官。四川行省平章汪壽昌辭職,不允。以作佛事祈福,釋御史臺所囚定興劉縣尹及刑部囚二十六人。乙丑,安南國世子陳日阜遣其臣鄧世延等二十四人來貢方物。安西王阿難答之子月魯帖木兒,坐與畏兀僧玉你達八的剌板的、國師必剌忒納失里沙津愛護持謀不軌,命宗王、大臣雜鞫之,獄成,三人皆伏誅,仍籍其家。以必剌忒納失里沙津愛護持妻醜丑賜通政副使伯藍,玉鞍賜撒敦,餘人畜、土田及七寶奩具、金珠、寶玉、鈔幣,並沒入大承天護聖寺。免四川行省境內今年租。命有司為伯顏建生祠,立紀功碑於涿州,伯別建祠、立碑於汴梁。戊辰,免雲南行省田租三年。安州饑,給河間鹽課鈔萬錠賑之。東昌、濟寧二路及曹、濮諸州,皆有蟲食桑。



 五月己巳朔,高昌王藏吉薨,其弟太平奴襲位。壬申,賑木憐、七里等二十三驛,人米二石。癸酉,熒惑犯東井。賜燕鐵木兒宴於流杯池。雲南大理、中慶等路大饑,賑鈔十萬錠。甲戌,升尚舍寺為從三品。撒迪請備錄皇上登極以來固讓大凡、往復奏答,其餘訓敕、辭命及燕鐵木兒等宣力效忠之跡,命朵來續為《蒙古脫不赤顏》一書,置之奎章閣,從之。賜湖廣行省平章政事脫亦納金虎符。旌保定路郭璹孝行、探忒妻靈保賢孝。戊寅,幸大承天護聖寺。京師地震有聲。己卯,命諸王也失班還鎮。浙西道廉訪司劾副使三寶兇惡陰險,紊亂紀綱,詔罷之。壬午,復賑糶米五萬石,濟京城貧民。戊子,唐其勢以疾先住上都,賜藥價鈔千錠。遣使往帝師所居撒思吉牙之地,以珠織制書宣諭其屬,仍給鈔四千錠、幣帛各五千匹,分賜之。賑帖裡乾、不老、也不徹溫等十九驛,人米二石。庚寅,大駕發大都,時巡於上都。置山東益都等處金銀銅鐵提舉司。辛卯,復以司徒印給萬安寺僧嚴吉祥。詔給鈔五萬錠,修帝師巴思八影殿。壬辰,太常博士王瓚言:「各處請加封神廟,濫及淫祠。按《禮經》,以勞定國,以死勤事,能御大災,能捍大患,則祀之。其非祀典之神,今後不許加封。」制可。丁酉,白虹並日出,長竟天。追封顏子父顏無繇為杞國公,謚文裕;母齊姜氏杞國夫人,謚端獻;妻宋戴氏兗國夫人,謚貞素。甘州大雹,揚州之江都、泰興,德安府之雲夢、應城縣水,汴梁之睢州、陳州、開封、蘭陽、封丘諸縣河水溢,滹沱河決,沒河間清州等處屯田四十三頃。常寧州饑,賑糶米二千四百石。杭州火,被災九十一戶,池州火,被災七十三戶,命江浙行省量賑之。六月己亥朔,以月魯帖木兒等罪詔告中外,赦天下。免四川行省今年差稅、陜西行省今年商稅。隸用朵朵、王士熙、脫歡等。己酉,以御史中丞趙世安為中書左丞。癸丑,遣使分祀岳鎮海瀆。戊午,給鈔五萬錠,賜雲南行省為公儲。己未,燕鐵木兒言:「頃伯顏封浚寧王,賜食邑嵩州,今請於瀕汴擇一州賜之。」詔改賜陳州。癸亥,加授知樞密院事也卜倫開府儀同三司。乙丑,御史臺臣劾遼陽行省參政賽甫丁庸鄙不勝任,罷之。監察御史陳思謙言:「內外官非文武全才、出處系天下安危、能拯金革之難者,勿許奪情起復。」制可。禁諸卜筮、陰陽人,毋出入諸王公大臣家。晉寧、冀州桑災,益都、濟寧大雨,無為州、和州水。旌歸德府永城縣民張氏孝節。



 秋七月戊辰朔,諸王答里麻失裡等遣使來貢虎豹。雲南行省言:「本省舊降給驛璽書六十九、金字圓符四,伯忽之亂,散失殆盡,乞更賜為宜。」敕更賜璽書三十二、圓符四,仍究詰所失者。辛未,以車坊官園賜伯顏。賜從征雲南將校三百四十七人鈔幣有差。調軍士修柳林海子橋道。乙亥,命僧於鐵幡竿修佛事,施金百兩、銀千兩、幣帛各五百匹、布二千匹、鈔萬錠。丁丑,賑蒙古軍流離至陜西者四百六十七戶糧三月,遣復其居,戶給鈔五十錠。湖廣行省言:「黎賊勢猖獗,乞益兵三千以備調用。」有旨:「依前詔,促移剌四奴克日進兵。」壬午,江西行省造螺鈿幾榻遺燕鐵木兒,詔賜匠者幣帛各一。甲申,燕鐵木兒獻斡羅思二千五百人。旌裕州民李庭瑞孝行。庚寅,給鈔萬錠,命燕鐵木兒分賜累朝官分嬪御之貧乏者。壬辰,西域諸王不賽因遣哈只怯馬丁以七寶水晶等物來貢。給蒙古民及各部衛士鈔幣有差,仍賑糧五月。甲午,北邊諸王月即別遣南忽裏等來朝貢。燕鐵木兒言:「諸王徹徹禿、沙哥,曩坐罪流南荒,乞賜矜閔,俾還本部。」從之。賑宗仁衛軍士九百戶各鈔一錠。滕州民饑,賑糶米二萬石。慶都縣大饑,以河間鹽課鈔萬錠賑之。



 八月辛丑,諸王阿兒加失里獻斡羅思三十人、漸丁百三人。賑大都寶坻縣饑民,以京畿運司糧萬石。癸卯,吳王木喃子及諸王答都河海、鎖南管卜、帖木兒赤、帖木迭兒等來朝。賜護守上都宮殿衛卒二千二百二十九人,人鈔二十五錠。乙巳,天鼓鳴於東北。丙午,遣官祭社稷。丁未,有事於太廟。海道漕運糧六十九萬餘石至京師。己酉,隴西地震。帝崩,壽二十有九,在位五年。癸丑,靈駕發引,葬起輦穀,從諸帝陵。元統二年正月己酉,太師右丞相伯顏率文武百官等議,上尊謚曰聖明元孝皇帝,廟號文宗,國言謚號曰札牙篤皇帝,請謚於南郊。三月己酉,祔於太廟。後至元六年六月,以帝謀為不軌,使明宗飲恨而崩,詔除其廟主。其略曰:



 昔我皇祖武宗皇帝升遐之後,祖母太皇太后惑於憸慝,俾皇考明宗皇帝出封雲南。英宗遇害,正統浸備,我皇考以武宗之嫡,逃居朔漠,宗王大臣同心翊戴,肇啟大事,於時以地近,先迎文宗,暫總機務。繼知天理人倫之攸當,假讓位之名,以寶璽來上,皇考推誠不疑,即授以皇太子寶。文宗稔惡不悛,當躬迓之際,乃與其臣月魯不花、也裏牙、明裏董阿等謀為不軌,使我皇考飲恨上賓。歸而再御宸極,思欲自解於天下,乃謂夫何數日之間,宮車弗駕。海內聞之,靡不切齒。又私圖傳子,乃構邪言,嫁禍於八不沙皇后,謂朕非明宗之子,遂俾出居遐陬。祖宗大業,幾於不繼。內懷愧慊,則殺也裏牙以杜口。上天不佑,隨降殞罰。叔嬸不答失里,怙其勢焰,不明明考之嗣,而妄立孺稚之弟亦憐真班,奄復不年,諸王大臣以賢以長,扶朕踐位。國之大政,屬不自遂者,詎能枚舉。每念治必本於盡孝,事莫先於正身,賴天之靈,權奸屏黜,盡孝正名,不容復緩,永惟鞠育罔極之恩,忍忘不共戴天之義。既往之罪,不可勝誅,其命太常徹去脫脫木兒在廟之主。不答失裏本朕之嬸,乃陰構奸臣,弗體朕意,僭膺太皇太后之號,跡其閨門之禍,離間骨肉,罪惡尤重,揆之大義,削去鴻名,徙東安州安置。



 放燕帖古思於高麗,未至,月闊察兒害之於中道。



\end{pinyinscope}