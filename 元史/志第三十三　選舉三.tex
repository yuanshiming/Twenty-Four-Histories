\article{志第三十三 選舉三}

\begin{pinyinscope}

 ○銓法中



 至元四年,詔:「諸官品正從分等,職官用廕,各止一名。諸廕官不以居官、去任、致仕、身故,其承廕之人,年及二十五以上者聽。諸用廕者,以嫡長子。若嫡長子有廢疾,立嫡長子之子孫,曾玄同。如無,立嫡長子同母弟,曾玄同。如無,立繼室所生。如無,立次室所生。如無,立婢子。如絕嗣者,傍廕其親兄弟,各及子孫。如無,傍廕伯叔及其子孫。諸用廕者,孫降子、曾孫降孫、婢生子及傍廕者,皆於合敘品從降一等。諸廕子入品職,循其資考,流轉升遷。廉慎幹濟者,依格超升。特恩擢用者,不拘此例。其有不務廉慎,違犯禮法者,依格降罰,重者除名。諸自九品依例遷至正三品,止於本等流轉,二品以上選自特旨。諸職官廕子之後,若有餘子,不得於諸官府自求職事,諸官府亦不許任用。」五年,詔:「諸廕官各具父祖歷仕緣由、去任身故歲月並所受宣敕札付、彩畫宗支,指實該承廕人姓名年甲,本處官司體勘房親,揭照籍冊,別無詐冒,及無廢疾過犯等事,上司審驗相同,保結申覆,令親齎文解赴部。諸廕敘人員,除蒙古及已當禿魯花人數別行定奪外,三品以下、七品以上、年二十五之上者,當儤使一年,並不支俸。滿日,三品至五品子孫量材敘用外,六品七品子準上銓注監當差使,已後通驗各界增虧定奪。」十六年,部擬:「管匠官止於管匠官內遷用。其身故匠官之子,若依管民官品級承廕,緣匠官至正九品以下,止有院長、同院務,例不入流品,似難一例廕用。比附承廕例,量擬正從五品子於九品匠官內敘,六品、七品子於院長內敘。凡儤直曾當怯薛身役,已經歷仕及止有一子,五十以上者,並免。」二十七年,詔:「凡軍民官陣亡,軍官襲父職,民官陣亡者,其子比父職降二等敘,其孫若弟復降一等。」大德四年,省議:「諸職官子孫廕敘,正一品子,正五品敘。從一品子,從五品敘。正二品子,正六品敘。從二品子,從六品敘。正三品子,正七品敘。從三品子,從七品敘。正四品子,正八品敘。從四品子,從八品敘。正五品子,正九品敘。從五品子,從九品敘。正六品子,流官於巡檢內用,雜職於省札錢穀官內用。從六品子,近上錢穀官。正七品子,酌中錢穀官。從七品子,近下錢穀官。諸色目人比漢人優一等廕敘,達魯花赤子孫與民官子孫一體廕敘,傍廕照例降敘。」至大四年,詔:「諸職官子孫承廕,須試一經一史,能通大義者免儤使,不通者發還習學,蒙古、色目願試者聽,仍量進一階。」延祐六年,部呈:「福建、兩廣、海北、海南、左右兩江、雲南、四川、甘肅等處廕敘之人,如父祖始仕本處,止以本地方敘用。據腹裏、江南歷仕升等遷往者,其子孫弟侄承廕,又注遠方,誠可憐憫。今將承廕人等量擬敘用,福建、兩廣、八番官員擬江南廕敘,海北、海南、左右兩江官員擬接連廕敘,雲南官員擬四川廕敘,四川、甘肅官員擬陜西廕敘。」



 凡遷調閩廣、川蜀、雲南官員:每三歲,遣使與行省銓注,而以監察御史往蒞之。至元十九年,省議:「江淮州郡遠近險易不同,似難一體,今量分為三等,若腹裏常調官員遷入兩廣、福建溪洞州郡者,於本等資歷上,例升二等,其餘州郡,例升一等。福建、兩廣官員五品以上,照勘員闕,移咨都省銓注,六品以下,就便委用,開具咨省。」二十年,部擬:「遷敘江淮官員,擬定應得資品,若於接連福建、兩廣溪洞州郡任用,升一等。甘肅、中興行省所轄系西夏邊地,除本處籍貫見任官外,腹里遷去甘肅者,擬升二等,中興府擬升一等。」二十二年,詔:「管民官腹里遷去四川升一等,接連溪洞升二等。四川見任官遷往接連溪洞升一等,若遷去溪洞諸蠻夷,別議定奪。達魯花赤就彼處無軍蒙古軍官內選擬,不為常例。」二十二年,江淮官員遷於龍南、安遠縣地分者,擬升三等,仍以三十月為滿升轉。二十八年,詔:「腹裏官員遷去雲南近裏城邑,擬升二等,若極邊重地,更升一等。行省咨保人員,比依定奪。其蒙古、土人及招附百姓有功之人,不拘此例。」省臣奏準:「福建、兩廣官員多闕,都省差人與彼處行省、行臺官,一同以本土周回相應人員委用。」部議:「雲南六品以下任滿官員,依御史臺所擬,選資品相應人,擬定名闕,具歷仕腳色,咨省奏準,敕牒到日,許令之任。若有急闕,依上選取,權令之任,歷過月日,依上準理。」二十九年,詔:「福建、兩廣官員歷兩任滿者,遷於接連去處,一任滿日,歷江南一任,許入腹裏通行遷轉,願於兩廣、福建者聽,依例升等。」至治元年,省臣奏:「江浙、江西、湖廣、四川、雲南五處行省所轄邊遠地分官員,三年一次差人與行省、行臺官一同遷調。」泰定四年,部擬:「諸職官子孫承廕,已有元定廕敘地方通例,別難議擬,如願於廣海廕敘者,聽其所請,依例升等遷敘。其已咨到都省,應合本省地分廕敘而未受除者,依例咨行省,令差去遷調官就便銓注。廣海闕官,於任滿得代,有由應得路府州縣儒學教授、學正、山長內願充者,借注正九品以下名闕,任回,止理本等月日。廣海應設巡檢,於本省應得常選上等錢穀官選擬,權設,理本等月日。行省自用並不應之人,不許委用,如受敕巡檢到彼,即聽交代。」



 凡遷調循行:各省所轄路府州縣諸司,應合遷調官員,先盡急闕,次及滿任。急闕須憑各官在任解由、依驗月日、應得資品、及解由到行省月日,依次就便遷調。若有急闕,委無相應之人,或員闕不能相就者,於應敘職官內選用,驗合得資品上,雖有超越,不過一等。本管地面,若有遐荒煙瘴險惡重地,除土官外,依例公選銓注,其有超用人員,多者不過二等。軍官、匠官、醫官、站官、各投下人等,例不轉入流品者,雖資品相應,不許銓注。都省已除人員,例應到任,若有違限一年者,聽別行補注。應有合就彼遷敘人員,如在前給由已咨都省聽除,未經遷注照會,不曾咨到本省者,即聽就便開咨。無解由人員,不許銓注。諸犯贓經斷應敘人員,照例銓注。令譯史、奏差人等,須驗實歷月日已滿,方許銓注。邊遠重難去處,如委不可闕官,從差去官與本省官公同選注能干人員,開具歷仕元由,並所注職名,擬咨都省,候回準明文,方許之任。應遷調官員,三品、四品擬定咨呈,五品以下先行照會之任。



 凡文武散官:多採用金制,建官之初,散官例降職事二等。至元二十年,始升官職對品,九品無散官,謂之平頭敕。蒙古、色目,初授散官或降職事,再授職,雖不降,必俟官資合轉,然後升職。漢人初授官,不及職,再授則降職授官。惟封贈廕敘官職,各從一高,必歷官至二品,則官必從職,不復用理算法矣。至治初,稍改之,尋復其舊。此外月日不及者,惟歷繁劇得優,獲功賞則優,由內地入邊遠則優,憲臺舉廉能政跡則優,以選出使絕域則優,然亦各有其格也。



 凡保舉職官:大德二年制:「各廉訪司所按治城邑內,有廉慎幹濟者,歲舉二人。」九年,詔:「臺、院、部五品以上官,各舉廉能識治體者三人,行省臺、宣慰司、廉訪司各舉五人。」



 凡翰林院、國子學官:大德七年議:「文翰師儒難同常調,翰林院宜選通經史、能文辭者,國子學宜選年高德邵、能文辭者,須求資格相應之人,不得預保布衣之士。若果才德素著,必合不次超擢者,別行具聞。」



 凡遷官之法:從七以下屬吏部,正七以上屬中書,三品以上非有司所與奪,由中書取進止。自六品至九品為敕授,則中書牒署之。自一品至五品為宣授,則以制命之。三品以下用金寶,二品以上用玉寶,有特旨者,則有告詞。其理算論月日,遷轉憑散官,內任以三十月為滿,外任以三歲為滿,錢穀典守以二歲為滿。而理考通以三十月為則。內任官率一考升一等,十五月進一階。京官率一考,視外任減一資。外任官或一考進一階,或兩考升一等,或三考升二等。四品則內外考通理。此秋毫不可越。然前任少,則後任足之,或前任多,則後任累之。一考者及二十七月,兩考者及五十七月,三考者及八十一月以上,遇升則借升,而補以後任。此又其權衡也。



 凡選用不拘常格:省參議、都司郎中、員外高第者,拜參預政事、六曹尚書、侍郎,及臺幕官、監察御史出為憲司官。外補官已制授,入朝或用敕除,朝跡秩視六品,外任或為長伯。在朝諸院由判官至使,寺監由丞至卿,館閣由屬官至學士,有遞升之法,用人重於用法如此。又覃官,或準實授,或普減資升等,或內升等,或外減資,或外減內不減,斯則恩數之不常有者,惟四品以下者有之。三品則遞進一階,至正議大夫而止。若夫勛臣世胄、侍中貴人,上命超遷,則不可以選格論。亦有傳敕中書,送部覆奏,或致繳奏者,斯則歷代以來封駁之良法也。



 凡吏部月選:至元十九年議:「到部解由即行照勘,合得七品者呈省,從七以下本部注擬,其餘流外人員,不拘多寡,並以一月一次銓注。」



 凡官吏遷敘:至元十年,議:「舊以三十月遷轉太速,以六十月遷轉太遲。」二十八年,定隨朝以三十月為滿,在外以三周歲為滿,錢穀官以得代為滿,吏員以九十月日出職,職官轉補,與職官同。



 凡覃官:至大二年,詔:「內外官四品以下,普覃散官一等,服色、班次、封廕皆憑散官。三品者遞進一階,至正三品上階而止。其應入流品者,有出身吏員譯史等,考滿加散官一等。」三年,蒙古儒學教授,一體普覃。四年,詔在任官員,普覃散官一等。泰定元年,詔:「內外流官已帶覃官,準理實授。所有軍官及其餘未覃人員,四品以下並覃散官一等,三品遞進一階,至三品上階止,服色、班次、封廕,悉從一高。其有出身應入流品人等,如在恩例之前入役支俸者,考滿亦依上例覃授。」二年,省議:「應覃人員,依例先理月日,後準實授,其正五品任回已歷一百三十五月者,九十月該升從四,餘有四十五月,既循行舊例,覃官三品,擬合準理實授,月日未及者,依驗散官,止於四品內遷用,所有月日,任回,四品內通行理算。」



 凡減資升等:大德九年,詔:「外任流官,升轉甚遲,但歷在外兩任,五品以下並減一資。」部議:「外任五品以下職官,若歷過隨朝及在京倉庫官鹽鐵等職,曾經升等減資外,以後至大德九年格前,歷及在外兩任或一任、六十月之上者,並與優減,未及者不拘此格。」至治二年,太常禮儀院臣奏:「皇帝親祭太廟,恩澤未加。」詔四品以下諸職官,不分內外,普減一資,有出身應入流品者,考滿任回,依上優減。天歷元年,詔:「以兵興,內外官吏供給繁勞,在京者升一等,至三品止,在外者減一資。」



 凡注官守闕:至元八年,議:「已除官員,無問月日遠近,許準守闕外,未奏未注者,許注六月滿闕,六月以上不得預注。」二十二年,詔:「員多闕少,守闕一年,年月滿者照闕注授,餘無闕者令候一年。」大德元年,以員多闕少,宜注二年。



 凡注官避籍:至元五年,議:「各路地里闊遠,若更避路,恐員闕有所礙,止宜斟酌避籍銓選。」



 凡除官照會:至元十年,議:「受除民官,若有守闕人員,當前官任滿,預期一月檢舉照會。錢穀官候見界官任滿,至日行下合屬照會。」二十四年,議:「受除官員省劄到部照勘,急闕任滿者,比之滿期,預先一月照會。」



 凡赴任程限:大德八年,定赴任官在家裝束假限,二千里內三十日,三千里內四十日,遠不過五十日。馬日行七十里,車日行四十里。乘驛者日兩驛,百里以上止一驛。舟行,上水日八十里,下水百二十里。職當急赴者,不拘此例。違限百日外,依例作闕。



 凡赴任公參:至元二年,定散府州縣赴任官,去上司百里之內者公參,百里之外者申到任月日,上司官不得非理勾擾,失誤公事。



 凡官員給假:中統三年,省議:「職官在任病假及緣親病假滿百日,所在官司勘當申部作闕,仍就任所給據,期年後給由求敘,自願休閑者聽。」至元八年,省準:「在任因病求醫並告假侍親者,擬自離職住俸日為始,限一十二月後聽仕。其之任官果因病患事故,不能赴任,自受除日為始,限一十二月後聽仕。」部擬:「凡外任官日久不行赴任,除行程並裝束假限外,違者計日斷罪。」二十七年,議:「祖父母、父母喪亡並遷葬者,許給假限,其限內俸鈔,擬合支給,違例不到,停俸定罪。」二十八年,部議:「官吏遠離鄉土,不幸患病,難議截日住俸,果有患病官吏,百日內給俸,百日外停俸作闕。」大德元年,議:「雲南官員,如遇祖父母、父母喪葬,其家在中原者,並聽解任奔赴。」二年,詔:「凡值喪,除蒙古、色目人員各從本俗外,管軍官並朝廷職不可曠者,不拘此例。」五年,樞密院臣議:「軍官宜限以六月,越限日以他人代之,期年後,授以他職。」七年,議:「已除官員,若有病故及因事不能赴任者,即牒所在官司,否則親鄰主首,呈報上司,別行銓注。」八年,吏部言:「赴任官即將署事月日飛申,以憑標附,有犯贓事故,並仰申聞。」天歷二年,詔:「官吏丁憂,各依本俗,蒙古、色目仿效漢人者,不用。」部議:「蒙古、色目人願丁父母憂者聽。」



 凡官員便養:至大三年,詔:「銓選官員,父母衰老氣力單寒者,得就近遷除,尤為便益。果有親年七十以上,別無以次侍丁,合從元籍官司保勘明白,斟酌定奪。」



 凡遠年求敘:元貞元年,部擬:「自至元二十八年三月為限,於本處官司明具實跡保勘,申覆上司遷敘。」大德七年,議:「求敘人員,具由陳告,州縣體覆相同,明白定奪,依例敘用。」



 銓法下



 凡省部令史、譯史、通事等:至元六年,省議:「舊例一百二十月出職,今案牘繁冗,難同舊日,會量作九十月為滿。其通事、譯史繁劇,合與令史一體。近都省未及兩考省令史譯史授宣,注六品職事,部令史已授省劄,注從七品職事。今擬省令譯史、通事,由六部轉充者,中統四年正月已前,合與直補人員一體,擬九十月考滿,注六品職事,回降正七一任,還入六品。中統四年正月已後,將本司歷過月日,三折二,驗省府月日考滿通理,九十月出職,與正七職事,並免回降。職官充省令譯史,舊例文資右職參注,一考滿,合得從七品,注從六品,未合得從七品,注正七品,如更勒留一考,合同隨朝升一等。一考滿,未得從七注正七品者,回降從七,還入正七。一考滿,合得從七注從六品,合得正七注正六品者,免回降。正從六品人員不合收補省令史、譯史,如有已補人員,合同隨朝一考升一等注授。中統四年正月已前,收補部令史、譯史、通事,擬九十月為考滿,照依已除部令史例,注從七品,回降正八一任,還入從七。中統四年正月已後,充部令譯史、通事人員,亦擬九十月為考滿,依舊例正八品職事,仍免回降。省宣使,舊例無此職名,中統以來,初立中書省,曾受宣命充宣使者,擬出職正七品職,外有非宣授人員,擬九十月為考滿,與正八品。」至元二十年,吏部言:「準內外諸衙門令譯史、通事、知印、宣使、奏差等,病故作闕,未及九十月,並令貼補,值例革者,比至元九年例定奪。」省準:「宣使、各部令史出職同,三考從七。一考之上,驗月日定奪。一考之下,二十月以上者正九,十五月以上者從九,十五月以下擬充巡檢。臺院、大司農司譯史、令史出身同,三考正七。一考之上,驗月日定奪。一考之下,二十月以上從八,十五月以上正九,十五月以下、十月之上從九,添一資,十月以下巡檢。宣使三考正八品。一考之上,驗月日定奪。一考之下,二十月以上從九,十五月以上巡檢,十五月以下酒稅醋使。部令史、譯史、通事三考從七。一考之上,驗月日定奪。一考之下,二十月以上者正九,十五月以上從九,十五月以下令史提控案牘,通事、譯史巡檢。奏差三考從八品。一考之上,驗月日定奪。一考之下,二十月以上巡檢,十五月之上酒稅醋使,十五月之下酒稅醋都監。」大德四年,中書省準:「吏部擬腹裏、江南都吏目、提控案牘升轉通例,凡腹裏提控案牘、都吏目:京畿漕運司令史,元擬六十月考滿,今準九十月考滿,都漕運司令史九十月。諸路寶鈔提舉司司吏,元擬六十月考滿,今準九十月考滿。萬億四庫司吏,元擬六十月考滿,今準九十月考滿。大都路令史,元擬六十月考滿,任回減資升轉,今準六十月考滿,不須減資。大都運司令史,九十月考滿都目。寶鈔總庫司吏,元擬六十月都目,九十月提控案牘,今準九十月都目。富寧庫司吏,元擬六十月提控案牘,今準九十月都目。左右八作司司吏,元擬六十月,今準九十月都目。」又議:「已經改擬出職人員,各路司吏轉充提控案牘、都目,比同升用,其餘直補人數,並循至元二十一年之例遷用。江南提控案牘、都目:至元二十五年呈準,各路司吏六十月吏目,兩考升都目,一考升提控案牘,兩考正九。路司吏九十月吏目,一考轉都目,餘皆依上升轉。江南提控案牘除各路司吏,比腹里路司吏至元二十五年呈準例遷除,其餘已行直補,並自行保舉,自呈準月日立格,實歷案牘兩考者,止依至元二十一年定例,九十月入流。未及兩考者,再添一資遷除。例後違越創補者,雖歷月日不準。」大德十一年,省臣奏:「凡內外諸司令史、譯史、通事、知印、宣使有出身者,一半於職官內選用,依舊一百二十月為滿,外任減一資。」又議:「選補吏員,除都省自行選用外,各部依元設額數,遇闕職官,與籍記內相參發補,合用一半職官,從各部自行選用。通事、知印從長官選用。譯史則從翰林院試發都省書寫典吏考滿人內,挨次上名補用,其有不敷,從翰林發補。奏差亦於職官內選一半,餘於籍記應例人內發補。歲貢人吏,依已擬在役聽候。」省議:「六部令史如正從九品不敷,從八品內亦聽選取。省掾,正從七品得代有解由並見任未滿、已除未任文資流官內選取,考滿於應得資品上升一等,除元任地方,雜職不預。院臺令史如元系七品之人,亦在選補之例。譯史、通事選識蒙古、回回文字,通譯語正從七品流官,考滿驗元資升一等,注元任地方,雜職不預。知印於正從七品流官內選取,考滿並依上例注授,雜職不預。宣使於正從八品流官內選取,仍須色目、漢人相參,歷一考,於應得資品上升一等,除元任地方,雜職不預。」



 凡歲貢吏員:至元十九年,省議:「中書省掾於樞密院、御史臺令史內取,臺、院令史於六部令史內取,六部令史以諸路歲貢人吏補充,內外職官材堪省掾及院、臺、部令史者,亦許擢用。省掾考滿,資品既高,責任亦重,皆自歲貢中出,若不教養銓試,必致人材失真,今擬定例於後:諸州府隸省部者,儒學教授選本管免差儒戶子弟入學讀書習業,非儒戶而願學者聽。遇按察司、本路總管府歲貢之時,於學生內選行義修明、文學優贍、通經史、達時務者,保申解貢。各路司吏有闕,於所屬衙門人吏內選取。委本路長官參佐,同儒學教授考試,習行移算術,字畫謹嚴,語言辯利,《詩》、《書》、《論》《孟》內通一經者為中式,然後補充。按察司書吏有闕,府州司吏內勾補,至歲貢時,本州本路以上,再試貢解。諸歲貢吏,當該官司於見役人內公選,以性行純謹、儒吏兼通者為上,才識明敏、吏事熟閑者次之,月日雖多、才能無取者不許呈貢。」二十二年,省擬:「呈試吏員,先有定立貢法,各道按察司上路總管府凡三年一貢,儒、吏各一人,下路二年貢一人,以次籍記,遇各部令史有闕補用。若隨路司吏及歲貢儒人,先補按察書吏,然後貢之於部,按察書吏依先例選取考試,唯以經史吏業不失章指者為中選。隨路貢舉元額,自至元二十三年為始,各道按察司每歲於書吏內,以次貢二名,儒人一名必諳吏事,吏人一名必知經史者,遇各部令史有闕,以次勾補。」元貞元年,詔:「諸路有儒通吏事、吏通經術、性行修謹者,各路薦舉,廉訪司試選。每道歲貢二人,省臺委官立法考試,必中程式,方許錄用。」大德二年,貢部人吏,擬宣慰司、廉訪司每道歲貢二人儒吏兼通者,自大德三年為始,依例歲貢,應合轉補各部寺監令史,依《至元新格》發遣,到部之日,公座試驗收補。九年,省判:「凡選府州教授,年四十已下,願試吏員程式,許補各部令史。除南人已試者,別無定奪到部,未試之人,依例考試。」至治二年,省準:「各道廉訪司書吏,先盡儒人,不敷者吏員內充貢,各歷一考,依例試貢。」



 凡補用吏員:至元十一年,省議:「有出身人員,遇省掾有闕,擬合於正從七品文資職官並臺、院、六部令史內,從上名轉補。翰林兩院擬同六部令史,有闕於隨路儒學教授通吏事人內選補。樞密院、御史臺令史、省掾有闕,從上轉補,考滿依例除授,又於正從八品文資官及六部令史內轉補。省斷事官令史與六部令史一體三考出身,於部令史內發補。少府監令史,擬於六部並諸衙門考滿典吏內補用。」十三年,省議:「行工部令史與六部令史一體,於應補人內挨次填補。」十四年,詔:「諸站都統領使司令史擬同各部令史,今既改通政院,與臺院令史一體出身,於各部令史內選補。」十五年,部擬:「翰林兼國史院令史同臺令史一體出身,於各部令史內選取。」二十一年,省議:「江淮、江西、荊湖等處行省令史,擬將至元十九年咨發各省貼補人員先行收補,不許自行踏逐,移咨都省,於六部見役令史內補充。或參用職官,則從行省新除正從八品職官內選取,雜職官不預。」二十二年,宣徽院令史,考滿正七品遷敘,於六部請俸令史內選取。總制院與御史臺同品,令譯史、通事一體如之。二十四年,省準:「大都留守司兼少府監令史,依宣徽院、大司農司例遷。」二十八年,省議:「陜西行省令史,於各部及考令史並正從八品流官內選補。」二十九年,大司農司令史,於各部一考之上令史及正從八品職官內選取。省掾有闕,於正七品文資出身人員內選。吏員於樞密院、御史臺令史元系六部令史內發充,歷二十月以上者選,如無,於上名內選。三十一年,省準:「內史府令史,於各部下名令史內選。」大德三年,省準:「遼陽省令史宜從本省選正從八品文資職官補用。復令各部見役令史內,不限歲月,或願充、或籍貫附近、或選到職官,逐旋選解。國子監令譯史,於籍記寺監令史內發補。上都留守司令史,於籍記各部令史內,或於正八品職官內選用,考滿從七品遷用。宣徽院闌遺監令史,準本院依驗元準月日挨補,考滿同,自行踏逐者降等。遇闕如系籍記令史並常調提控案牘內及本院兩考之上典吏內補充者,考滿依例遷敘,自行選用者,止於本衙門就給付身,不入常調。」四年,部擬:「上都留守司令史,仍聽本司於正從八品流官內,或於上都見役寺監令史、河東、山北二道廉訪司上名書吏內,就便選用。上都兵馬司司吏,發補附近隆興、大同、大寧路司吏相應。」部擬:「各處行省令史,除云南、甘肅、征東外,其餘合依至元二十一年定例,於六部見役上名令史、或正從八品流官參補。不敷,聽於各道宣慰司元系廉訪按察司轉補見役兩考之上令史內選充,以宣慰司役過月日,折半準算,通理一百二十月,方許出職。」大德五年,擬:「檀景等處採金鐵冶都提舉司人吏,於附近州縣司吏內遴選。」六年,省擬:「太醫院令史,於各部令史並相應職官內選取。長信寺令史,於元保內選補,考滿降等敘用,有闕於籍記令史內發補。」七年,擬:「刑部人吏,於籍記令史內公選,不許別行差補,考滿離役,依例選取,餘者依次發補。禮部省判,許於籍記部令史內選取儒吏一名,續準一名,於籍記部令史內從上選補。戶部令史,於籍記部令史內從上以通曉書算、練達錢穀者發遣,從本部試驗收補。」八年,省準:「隨路補用吏員,令各路先以州吏入役月日籍為一簿。府吏有闕,從上勾補;州吏有闕,則於本州籍記司縣人吏內從上勾補。各道宣慰司令史,遇闕以籍記部令史下名發補,新除正從九品流官內選取。」九年,省準:「都城所系在京五品衙門司吏,歷兩考轉補京畿都漕運兩司令史。遇闕以倉庫攢典歷一考者選充,及兩考則京畿都漕運兩司籍名,遇闕依次收補。上都寺監令史有闕,先盡省部籍記常調人員發補,仍於正從九品流官內、並應得提控案牘內選取。不敷,就取元由路吏考滿升充都吏目典史準吏目月日及大同、大寧、隆興三路司吏歷兩考之上者參用。」十年,省準:「司縣司吏有闕,於巡尉司吏內依次勾補。巡尉司吏有闕,從本處耆老上戶循眾推舉,仍將祗應月日均以歲為滿。州吏有闕,縣吏內勾補。路吏有闕,州吏內勾補。若無所轄府州,於附近府州吏內勾補,縣吏發補附近府州司吏。戶、刑、禮部合選令史有闕,於籍記令史上十名內、並職官到選正從九品文資流官內試選。」十一年,省準:「縣吏如歷一考,取充庫子一界,再發縣吏,準理州吏月日,路吏有闕,依次勾補。」至大元年,省準:「典寶監令史,就用前典寶署典書蒙古必闍赤一名,例從翰林院試補,知印、通事各一名,從長官選保。」二年,立資國院二品,及司屬衙門令史一十名,半用職官,從本院選,半於上名部令史內發補。譯史二名,內職官一名,從本院選,外一名翰林院發。通事、知印各一名,從本院長官選。宣使八名,半參用職官,餘許本院自用一名,外三名常選相應人內發。典吏六名,從本院選。所轄庫二處,每處司庫六名,本把四名,於常選人內發。泉貨監六處,各設令史八名,於各路上名司吏內選;譯史一名,從翰林院發;通事二名,從本監長官選;奏差六名,各州司吏內選;典吏二名,本監選。以上考滿,同都漕運司吏出身,所轄一十九處,兩提舉司設吏目一人,常選內選,司吏五名,縣司吏內選。三年,省準:「泉貨監令史,於各處行省應得提控案牘人內選,參用正從九品流官。山東、河東二監,從本部於相應人內發補,考滿依例遷用,見役自用之人,考滿降等敘,有闕以相應人補。」四年,省準:「江西等處儒學提舉司司吏,舊從本司公選,事從國子監發補,宜從本司選補。典瑞監首領官、令譯史等,依典寶監例選用,考滿遷敘。」部議:「長信寺通事一名,例從所保。譯史、知印、令史、奏差,從本衙門選一半職官,餘相應人內選,考滿同自用遷敘。典吏二名,就便定奪,其自用者降等敘。」皇慶元年,省準:「群牧監令譯史、知印、怯里馬赤、奏差人等,據諸色譯史例,從翰林院發補。知印、通事,長官選。令史、奏差、典吏俱有發補定例。其已選人,考滿降等敘,有闕於相應人內選發。大都路令史,歷六十月,依至元二十九年例升提控案牘,減一資升轉。有過者,雖貼滿月日,不減資。遇闕於所轄南北兩兵馬司並各州見役上名司吏內勾補,有闕從本路於左右巡院、大興、宛平與其餘縣吏通籍從上挨補,月日雖多,不得無故替罷,違例補用者不準,除已籍記外,有闕依上勾補。覆實司司吏,於諸州見役司吏內選,不敷則以在都倉庫見役上名攢典發充,歷九十月除都目,年四十五之下歷一考之上,亦許轉補京畿都漕運司令史,違例收補,別無定奪。」二年,省準:「中瑞司譯史,從翰林院發,知印長官選保,令史、奏差參取職官一半所選相應,考滿依例遷敘,奉懿旨委用者,考滿本司區用,有闕以相應人補。征東行省令譯史、宣使人等,舊考滿從本省區用,若經省部擬發,相應之人依例遷用,如不應者,雖省發亦從本省區用。」延祐二年,省準:「河間等路都轉鹽運使司所轄場,分二十九處,二處改升從七品,司吏有闕,依各縣人吏,一體於附近各處巡尉捕盜司吏依次以上名勾補,再歷一考,與各場鄰縣吏互相遷調。和林路總管府司吏,以本處兵馬司吏歷一考者轉補,再歷一考,轉稱海宣慰令史,考滿除正八品。補不盡者,六十月受部劄充提控案牘。沙、瓜二州屯儲總管萬戶府邊遠比例,一體出身相應。會福院令譯史、通事、宣使人等,若省部發去者依例遷敘,自用者考滿同二品衙門出身例,降一等添一資升轉。於常選教授儒人職官並見役各部令史內取補,宣使於常選職官內參補,通事、知印從長官選用,仍須參用職官,典吏從本衙門補用。」五年,省準:「詹事院立家令司、府正司,知印、怯里馬赤俱令長官選用。令史六名,內取教授二名,職官二名,廉訪司書吏二名。譯史一名,於蒙古字教授及都省見役蒙古書寫內選補。奏差二名,以相應人補。



 凡宣使、奏差、委差、巡鹽官出身:中書省宣使,至元九年,曾受宣命補充者,九十月考滿正七品。省劄宣使,九十月考滿比依部令史例從七品。其臺院宣使、各部奏差,比例定擬。二十三年,省準:「省部臺院令譯史、通事、宣使、奏差人等,未滿九十月,不許預告遷轉。都省元定六部奏差遷轉格例,應入吏目選充者,三考從八品。應入提控案牘人員選充者,三考從八品,任回減一資升轉。巡檢提控案牘選充者,一考正九品。」二十四年,省準:「大都留守司兼少府監奏差改充宣使,合於各部奏差內選取,改升宣使月日為始,考滿比依宣徽院、大司農司一體出身,自行踏逐者降等遷敘。大司農司所轄各道勸農營田內書吏,於各路司吏內選取,考滿提控案牘內任用。奏差就令本司選委。」二十九年,省準:「各道廉訪司通事、譯史出身,比依書吏一體,考滿正九。奏差考滿,依通事、譯史降二等量擬,於錢穀官並巡檢內任用。」三十年,省準:「延慶司奏差,比依家令司奏差一體,考滿正九品,自行踏逐者降一等。」大德四年,省準:「諸路寶鈔提舉司奏差,改稱委差,九十月為滿,於酌中錢穀官內任用。」五年,部議:「山東運司奏差,九十月近下錢穀官內任用。大都運司,一體定奪。」六年,部擬:「河間運司巡鹽官,依奏差出身,九十月近下錢穀官內任用。」七年,部擬:「凡奏差自改立廉訪司為始,九十月歷巡檢三考,轉從九。」皇慶元年,各道廉訪司奏差出身,於本道所轄上名州司吏內選取,九十月都目內任用。若有路吏並典吏內取充者,歷兩考,比依上例,都目內升轉。



 凡庫藏司吏庫子等出身:至元二十六年,省準:「上都資乘庫庫子、本把,九十月近上錢穀官內任用。衛尉院利器庫、壽武庫庫子,踏逐者九十月近上錢穀官內任用。」二十八年,省擬:「泉府司富藏庫本把、庫子,六十月近下錢穀官內任用。大府監行內藏庫子,三周年為滿,省劄錢穀官內遷敘。備用庫提控三十月,庫子、本把三周歲,近上錢穀官內任用。」三十年,省準:「大都留守司兼少府監器備庫庫子、本把,六十月近下錢穀官內任用。」三十年,省準:「宣徽院生料庫庫子、本把並太醫院所轄御藥局院本把出身,例六十月,近上錢穀官一體遷敘。」大德元年,部擬:「中御府奉宸庫庫子,以三周歲為滿,擬受省劄錢穀官。本把六十月,近上錢穀官內任用。」三年,省擬:「萬億四庫、左右八作司、富寧、寶源等庫,各設色目司庫二名,俱於樞密院各衛色目軍內選差,考滿巡檢內任用,自行踏逐者一考並同,循行如此。又漢人司庫,於院務提領、大使、都監內發補,二周歲滿日,減一界升轉;其色目司庫於到選錢穀官內選發,考滿優減兩界。都提舉萬億庫提控案牘,比常選人員,任回減一資升用。司吏三十五人,除色目四人外,漢人有闕,於大都總管府、轉運司、漕運司下名司吏內選取,三十月擬充吏目,四十五月之上、六十月之下都目,六十月以上轉提控案牘。省擬六十月以上、四十五月以下,願充寺監令史者聽。司庫五十人,除色目一十四人另行定奪外,漢人於大都路人戶內選用,二周歲為滿,院務提領內任用;都監內充司庫,二年為滿,於受省劄錢穀官內任用;務使充司庫,二年為滿,於從九品雜職內任用。秤子五人,於大都人戶內選充,二年為滿,於近下錢穀官內任用。太醫院御藥局本把,六十月近上錢穀官內任用。」四年,受給庫依油磨坊設攢典、庫子,從工部選。會同館收支庫攢典,與長秋庫同。上都廣積、萬盈二倉系正六品,永豐系正七品,比之大都平準庫品級尤高,擬各倉攢典轉寺監本把,並萬億庫司吏相應。提舉廣惠司庫子,考滿近下錢穀官內任用。侍儀司法物庫所設攢典、庫子,依平準行用庫例補用。五年,大都尚食局本把,擬於錢穀官內遷敘,本院自行踏逐者,就給付身,考滿不入常調。都提舉萬億寶源庫色目司庫,擬於巡檢內任用,添一資升轉。京畿都漕運司司倉,於到選錢穀官內選發。六年,部呈:「凡路府諸州提控案牘、都吏目等,諸衙門吏員出身,應得案牘、都吏目,如系路府司吏轉充之人,依舊遷除。其由倉庫攢典雜進者,得提控案牘改省劄錢穀官,都目近上錢穀官,吏目改酌中錢穀官。提控案牘,都吏目月日考滿,於流官內遷用。廣勝庫子,合從武備寺給付身,考滿本衙門定奪。大積等倉典吏,與四庫案牘所掌事同,任回減一資升用。」七年,各路攢典、庫子,部議:「江北及行省所轄路分庫子,依已擬於司縣司吏內差補,周歲發充縣司吏,遇州司吏有闕,挨次勾補。諸倉庫攢典有闕,於各部籍記典吏內發補。左右八作司等五品衙門內司吏有闕,卻於各倉庫上名攢典內發補。若萬億庫四品衙門司吏有闕,亦於上項司吏內從上轉補,將役過五品衙門月日,五折四準算,通理九十月考滿,提控案牘內遷用。如轉補不盡,五品衙門司吏考滿,止於都吏內任用。油磨坊、抄紙坊攢典有闕,並依上例。回回藥物院本把,六十月酌中錢谷內定奪。」九年,省準:「提舉利林倉、昔寶赤八剌哈孫倉、孔古列倉司吏,六十月酌中錢穀官內委用。資成庫庫子出身,部議比依太府、利用、章佩、中尚等監。武備寺庫有闕,如系本衙門典吏請俸一考轉補者,六十月為近上錢穀官,其餘補充之人,九十月依上遷用。和林等處宣慰司都元帥府所轄廣濟庫庫子、攢典,自行踏逐者比依三倉例,六十月於近下錢穀官內定奪。」至大二年,省準:「廣禧庫庫子,依奉宸庫例出身,如系本把一考之上轉充者,四十五月受省劄錢穀官,其餘補充之人,六十月依上例遷用。本把元系本衙門請俸一考典吏轉補者,六十月近上錢穀官,其餘補充者,九十月亦依上例遷用。上都東西萬盈、廣積二倉司倉,與倉官一體,二周歲為滿。」三年,省準:「各路庫子於各處錢穀官內發補,擬不減界,考滿從優定奪。江北庫子,止依舊例。和林設立平準行用庫庫子,宜從本省相應人內量選二名,二周歲為滿,近下錢穀官內定奪。」皇慶元年,部議:「文成、供須、藏珍三庫本把、庫子,依太府監庫子例,常選內委用,考滿比例遷除,有闕於常調人內發補,自行選用者,考滿從本院定奪,若系常選任用者,考滿依例遷敘。」二年,殊祥院所轄萬聖庫庫子、攢典,依崇祥院諸物庫例出身。部議:「如比上例,三十月轉補五品衙門司吏,再歷三十月,於四品衙門司吏內補用,其庫子合於常調籍記倉庫攢典人內發補,六十月為滿,於務都監內任用,自行委用者,考滿本衙門定奪。」延祐元年,省議:「腹里路分司倉庫子,於州縣司吏內勾補,滿日同舊例升轉。」



 凡書寫、銓寫、書吏、典吏轉補:至元二十五年,省準:「通政等二品衙門典吏,九十月補本院宣使。各寺監典吏,比依上例,考滿轉補本衙門奏差。戶部填寫勘合典吏,與管勘合令史一體,考滿從優定奪。參議府、左右司、客省使令史、書寫,四十五月轉補,如補不盡,於提控案牘內任用,於各部銓寫及典吏內收補。會總房、承發司、照磨所、架閣庫典吏,各部銓寫,六十月轉補,已上,都吏內任用。各部典吏並左右部照磨所、架閣庫典吏,於都省參議府、左右司、客省使令史、書寫內以次轉補,如補不盡,六十月轉補各監令史,已上,吏目內任用。樞密院典吏、銓寫,依御史臺典吏一體,六十月轉部,轉補不盡,六十月已上,於都目內任用。御史臺典吏,遇察院書吏有闕,從上挨次轉補,通理六十月,補各道按察司書吏,部令史有闕,亦行收補。」二十六年,省準:「上都留守司兼本路都總管府典吏,九十月補本司宣使,考滿依例定奪。」二十七年,省準:「漕運使司令史,九十月提控案牘內任用,如年四十五以下,願充寺監令史者聽。省院臺部書寫、銓寫、典吏人等出身,與各道宣慰司、按察司、隨路總管府歲貢吏員一體轉部,書寫人等止令轉寺監等衙門令史。」二十八年,省準:「參議府、左右司、客省使令史,各房書寫有闕,擬於都省典吏內選補,五折四令史、書寫月日,通折四十月轉部。及六部銓寫、典吏一考之上選充,三折二令史、書寫月日,通折四十五月轉補各部令史。如已行選用者,四十五月補寺監令史。參議府、左右司、客省使令史,各房書寫有闕,擬於都省典吏內選補,五折四令史、書寫月日,通折四十五月轉部。及六部銓寫、典吏一考之上選充,三折二令史、書寫月日,通折四十五月轉補各部令史。如自行選用者,四十五月補寺監令史。」部議:「執總會總房、照磨所、承發司、架閣庫典吏,一考之上轉補參議府、左右司、客省使令史,補不盡者,四十五月補寺監令史。有闕,於六部銓寫、典吏一考之上選充,三折二省典吏月日,通折六十月轉補各部令史。若轉充參議府、左右司、客省使令史、都省書寫,五折四令史、書寫月日,通折四十五月轉部。如自行選用者,六十月補寺監令史。六部銓寫、典吏並左右部照磨所、架閣庫典吏,一考之上,遇省書寫、典吏月日補不盡者,六十月轉補寺監令史。」省議:「除見役外,後有闕,擬於都省各房寫發人內公舉發補,除轉充參議府、左右司、客省使令史、都省書寫、典吏者,依前例轉補,不盡者六十月充都目。」二十九年,部擬:「御史臺典吏三十月,依廉訪司書吏轉補察院,三十月轉部,補不盡者,考滿從八品遷用外,行臺典吏三十月轉補行臺察院書吏,再歷三十月發補各道宣慰司令史。參議府令史,四十五月轉部令史。光祿寺典吏,考滿轉補本衙門奏差。」元貞元年,省準:「省部見役典吏實歷俸月,名排籍記,遇都省書寫、典吏有闕,從上挨次發補。樞密院銓寫,一考之上補都省書寫,通折月日升轉外,本院銓寫有闕,補請俸上名典吏。」大德元年,省準:「兩淮本道書吏,轉補行臺察院書吏、江南宣慰司令史。雲南、四川、河西三道書吏,在邊遠者三十月為格,依上遷補。江浙行省檢校書吏,於行省請俸典吏內選補,以典吏月日五折四,通折書吏六十月轉各道宣慰司。」四年,省準:「徽政院掌儀、掌膳、掌醫署書吏宜從本院通定名排,若本院典吏有闕,以次轉補。」八年,省議:「院臺以下諸司吏員,俱從吏部發補,據曾經省發並省判籍定典吏、令史,從吏部依次試補,元籍記典吏,見在寫發者,遇各庫攢典試補。省掾每名,設貼書二名,就用已籍記者,呈左右司關吏部籍定,遇部典吏闕收補,歷兩考從上名轉省典吏,除一考外,餘者折省典吏月日,兩考升補參議府、左右司、客省使令史、書寫、檢校、書吏,通折四十五月。補不盡省典吏,六十月,遇寺監令史、宣慰司令史有闕,依次發補。除宣慰司令史,已有貢部定例,寺監令史歷一考,與籍記部令史通籍發補各部令史。寺監見役人等,雖經準設,未曾補闕,不許轉部,考滿依舊例遷敘,其省部典吏、書寫人等轉入寺監、宣慰司,願守考滿者聽。御史臺令史一名,選貼書二名,依次選試相應充架閣庫子,轉補典吏,三十月發充各道廉訪司書吏,再歷一考,依例歲貢。三品衙門典吏,歷三考升宣使,補不盡,本衙門於相應闕內委用。部典吏一考之上,轉省典吏,補不盡者,三考補本衙門奏差,兩考之上發寺監宣慰司奏差外,據六部系名貼書合與都省寫發人相參轉補各部典吏,補不盡者,發各庫攢典。都省寫發人有闕,於六部系名貼書內參選,不盡者依舊發各庫攢典。」九年,省準:「獄典歷一考之上,轉各部典吏。翰林國史院書寫考滿,除從七品,有闕從本院於籍記教授試準應補部令史內指名選用。太常寺典吏,歷九十月注吏目。工部符牌局典吏,三十月轉各部典吏。翰林國史院蒙古書寫,四十五月轉補寺監蒙古必闍赤。宣徽院所轄寺監令史有闕,於到部籍記寺監令史與本院考滿典吏挨次發補。」十年,省準:「陜西諸道行御史臺察院書吏,若系腹裏歲貢廉訪司見役書吏選取人數,須歷一考,以上名貢部,下名轉補察院。總管府獄典轉州司吏,府州者補縣吏,須歷一考,方許轉補。江浙行省運司書吏,九十月升都目,添一資升轉,如非各路散府上州司吏補充,役過月日,別無定奪。」十一年,省準:「左司言照磨所典吏遇闕,宜於左右部照磨所典吏內從上發補。各路府州獄典遇闕,於廉訪司寫發人及各路通曉刑名貼書內參補。」至大元年,省準:「各部蒙古必闍赤,如系翰林院選發之人,四十五月遇各衙門譯史有闕,依次與職官相參補用,不敷從翰林院發補。」三年,省準:「詹事院蒙古書寫,如系翰林院選發之人,四十五月遇典用等監衙門譯史有闕,依次與職官相參補用,不敷從翰林院選發。和林行省典吏,轉理問所令史,四十五月發補稱海宣慰司令史,轉補不盡典吏,須歷六十月依上發補。中瑞司、掌謁司典書,九十月與寺監令史一體除正八品。行臺察院書吏,俱歷九十月依舊出身敘,任回添一資升轉。內臺察院轉部、行臺察院轉江南宣慰司令史,北人貢內臺察院各道廉訪司書吏,先役書吏歷九十月,擬正九品,任回添一資升轉。」省議:「廉訪司書吏,上名貢部,下名轉察院,不盡者通九十月,除正九品。察院書吏三十月轉部,不盡者九十月除從八品,非廉訪司取充則四十五月轉部,不盡者考滿除正九品。」二年,議:「廉訪司書吏、貢察院書吏不盡者九十月除正九品,行臺察院書吏轉補不盡者如之。內臺察院書吏轉部,年高不願轉部者,九十月除從八品。」皇慶元年,部議:「廉訪司職官書吏,合依通例選取,不許遷敘,候書吏考滿,通理敘用。職官先嘗為廉訪司書吏者,避元役道分,並其餘相應職官,歷三十月,減一資。又教授、學正、學錄並府州提控案牘、都吏目內委充職官,各理本等月日,其餘歲貢儒吏,依例選用。又廉訪司奏差、內臺行臺典吏有能者,歷一考之上選充書吏,通儒書者充儒人數,通吏業者充吏員數。參議府、左右司、客省使令史、書寫、檢校書吏,依至元二十八年例,以省典吏選充,五折四令史、書寫、書吏月日,通折五十五月轉部。省典吏系六部銓寫、典吏轉充,三折二省典吏月日,通折六十月轉各部令史。自用之人並轉補不盡省典吏,考滿發補寺監、各道宣慰司令史。」二年,省準:「河東宣慰司選河東山西道廉訪司書吏充令史,合回避按治道分選取,其餘亦合一體。」延祐三年,部擬:「行臺察院書吏、各道廉訪司掌書,元系吏員出身者,並依舊例,以九十月為滿,依漢人吏員降等於散府諸州案牘內選用,任回依例升轉。大宗正府蒙古書寫,四十五月依樞密院轉各衛譯史除正八品例,籍定發補諸寺監譯史。察院書吏與宣慰司令史,皆系八品出身轉部者,宜以五折四理算,宣慰司令史出身正八品,察院從八品,其轉補到部者以五折四準算太優,今三折二。其廉訪司徑發貢部及已除者,難議理算。」天歷元年,臺議:「各道書吏,額設一十六人,有闕宜用終場下第舉子四人,教授四人,各路司吏四人,通吏職官四人,委文資正官試驗相應,方許入部。」



 凡衛翼吏員升轉:皇慶元年,樞密院議:「各處都府並總管高麗、女直、漢軍萬戶府及臨清萬戶府秩三品,本府令史有闕,於一考都目、兩考吏目並各衛三考典吏內,呈院發補,九十月歷提控案牘一任,於各萬戶府知事內選用。」延祐六年,樞密院議:「各衛翼都目得代兩考者,擬受院劄提控案牘內銓注,三考升千戶所知事,月日不及者,各衛翼挨次前後得代日期,於都目內貼補。各衛提控案牘,年過五旬已歷四考者,升千戶所知事。及兩考年四十五以下,發補各衛令史。不及兩考者,止於案牘內銓注,受院劄,通理一百二十月,於千戶所知事內選用。各處蒙古都元帥府額設令史有闕,於本府所轄萬戶府並奧魯府上名司吏年四十以下者選取,呈院準設,歷一百二十月,再歷提控案牘一任,於萬戶府知事內遷用。」泰定三年,樞密院議:「行省所轄萬戶府司吏有闕,於本翼上千戶所上名司吏內取補,須行省準設,九十月充吏目,一考轉都目,一考除千戶所提領案牘,一考升萬戶府提控案牘,歷兩考,通歷省除一百五十月,行省照勘相同,咨院於萬戶府知事內區用。」



 凡各萬戶府司吏:蒙古都萬戶府司吏有闕,於千戶所司吏內選補,歷一百二十月,升千戶所提領案牘,一考萬戶府案牘,通理九十月,轉萬戶府知事。漢軍萬戶府並所轄萬戶府及奧魯府司吏,於千戶所司吏內補用,呈院準設,九十月充吏目,一考都目,一考升千戶所或都千戶所、奧魯府提控案牘,再歷萬戶府或都府、奧魯府提控案牘兩任,於萬戶府知事內用。各處都府令史,於一考都目、兩考吏目並各衛請俸三考典吏內,呈院發補,九十月為滿,再歷提控案牘一任,於各萬戶府知事內選用。各處蒙古軍元帥府令史,大德十年擬於本府所轄萬戶府並奧魯府上名司吏內,年四十以下者選補,呈院準設,歷一百二十月,再歷提控案牘一任,於萬戶府知事內遷用。各省鎮撫司令史,於各萬戶府上名六十月司吏內選取,受行省劄,三十月為滿,再於各萬戶府提控案牘內,歷一百二十月知事內定奪。各衛翼令史,有出身轉補者,九十月正八,無出身者從八內定奪。



 凡提控案牘、都目:至元二十一年三月已後受院劄,九十月為滿,行省、行院劄一百二十月為滿,於萬戶府知事內用。大德四年,案牘年過五旬,已歷四考者,於千戶所知事內定奪外,及兩考四十五以下發補各衛令史,若不及考者,止於案牘內銓注,受院劄,通理一百二十月,於千戶所知事內用。各衛翼都目,延祐六年,請俸兩考者,院劄提控案牘內銓注,歷三考,升千戶所知事,月日不及者,各衛翼都目內貼補。如各衛典吏轉充者,六十月直隸本院萬戶府提控案牘、弩軍屯田千戶所、鎮撫司提控案牘內銓注。無俸人轉充者,二十月依上升轉。鎮撫司、屯田弩軍千戶所都目,依中州例,改設案牘,止請都目俸,三十月為滿,依例注代。



\end{pinyinscope}