\article{志第三十一 選舉一}

\begin{pinyinscope}

 選舉之法尚矣。成周庠序學校,以鄉三物教萬民而賓興之,舉於鄉,升於司徒、司馬論定,而後官之。兩漢有賢良方正、孝弟力田等科,或奉對詔策,事猶近古。隋、唐有秀才、明經、進士、明法、明算等科,或兼用詩賦,士始有棄本而逐末者。宋大興文治,專尚科目,雖當時得人為盛,而其弊遂至文體卑弱,士習委靡,識者病焉。遼、金居北方,俗尚弓馬,遼景宗、道宗亦行貢試,金太宗、世宗屢闢科場,亦粗稱得士。



 元初,太宗始得中原,輒用耶律楚材言,以科舉選士。世祖既定天下,王鶚獻計,許衡立法,事未果行。至仁宗延祐間,始斟酌舊制而行之,取士以德行為本,試藝以經術為先,士褒然舉首應上所求者,皆彬彬輩出矣。然當時仕進有多岐,銓衡無定制,其出身於學校者,有國子監學,有蒙古字學、回回國學,有醫學,有陰陽學。其策名於薦舉者,有遺逸,有茂異,有求言,有進書,有童子。其出於宿衛、勛臣之家者,待以不次。其用於宣徽、中政之屬者,重為內官。又廕敘有循常之格,而超擢有選用之科。由直省、侍儀等入官者,亦名清望。以倉庾、賦稅任事者,例視冗職。捕盜者以功敘,入粟者以貲進,至工匠皆入班資,而輿隸亦躋流品。諸王、公主,寵以投下,俾之保任。遠夷、外徼,授以長官,俾之世襲。凡若此類,殆所謂吏道雜而多端者歟!矧夫儒有歲貢之名,吏有補用之法,曰掾史、令史,曰書寫、銓寫,曰書吏、典吏,所設之名,未易枚舉,曰省、臺、院、部,曰路、府、州、縣,所入之途,難以指計。雖名卿大夫,亦往往由是躋要官,受顯爵;而刀筆下吏,遂致竊權勢,舞文法矣。故其銓選之備,考核之精,曰隨朝、外任,曰省選、部選,曰文官、武官,曰考數,曰資格,一毫不可越。而或援例,或借資,或優升,或回降,其縱情破律,以公濟私,非至明者不能察焉。是皆文繁吏弊之所致也。



 今採摭舊編,載於簡牘,或詳或略,絳分類聚,殆有不勝其紀述者,姑存一代之制,作《選舉志》。



 科目



 太宗始取中原,中書令耶律楚材請用儒術選士,從之。九年秋八月,下詔命斷事官術忽鷿與山西東路課稅所長官劉中,歷諸路考試。以論及經義、詞賦分為三科,作三日程,專治一科,能兼者聽,但以不失文義為中選。其中選者,復其賦役,令與各處長官同署公事,得東平楊奐等凡若干人,皆一時名士,而當世或以為非便,事復中止。



 世祖至元初年,有旨命丞相史天澤絳具當行大事,嘗及科舉,而未果行。四年九月,翰林學士承旨王鶚等,請行選舉法,遠述周制,次及漢、隋、唐取士科目,近舉遼、金選舉用人,與本朝太宗得人之效,以為:「貢舉法廢,士無入仕之階,或習刀筆以為吏胥,或執僕役以事官僚,或作技巧販鬻以為工匠商賈。以今論之,惟科舉取士,最為切務,矧先朝故典,尤宜追述。」奏上,帝曰:「此良法也,其行之。」中書左三部與翰林學士議立程式,又請:「依前代立國學,選蒙古人諸職官子孫百人,專命師儒教習經書,俟其藝成,然後試用,庶幾勛舊之家,人材輩出,以備超擢。」十一年十一月,裕宗在東宮時,省臣復啟,謂「去年奉旨行科舉,今將翰林老臣等所議程式以聞」。奉令旨,準蒙古進士科及漢人進士科,參酌時宜,以立制度,事未施行。至二十一年九月,丞相火魯火孫與留夢炎等言,十一月中書省臣奏,皆以為天下習儒者少,而由刀筆吏得官者多。帝曰:「將若之何?」對曰:「惟貢舉取士為便。凡蒙古之士及儒吏、陰陽、醫術,皆令試舉,則用心為學矣。」帝可其奏。繼而許衡亦議學校科舉之法,罷詩賦,重經學,定為新制。事雖未及行,而選舉之制已立。



 至仁宗皇慶二年十月,中書省臣奏:「科舉事,世祖、裕宗累嘗命行,成宗、武宗尋亦有旨,今不以聞,恐或有沮其事者。夫取士之法,經學實修己治人之道,詞賦乃摛章繪句之學,自隋、唐以來,取人專尚詞賦,故士習浮華。今臣等所擬將律賦省題詩小義皆不用,專立德行明經科,以此取士,庶可得人。」帝然之。十一月,乃下詔曰:「惟我祖宗以神武定天下,世祖皇帝設官分職,徵用儒雅,崇學校為育材之地,議科舉為取士之方,規模宏遠矣。朕以眇躬,獲承丕祚,繼志述事,祖訓是式。若稽三代以來,取士各有科目,要其本末,舉人宜以德行為首,試藝則以經術為先,詞章次之。浮華過實,朕所不取。爰命中書,參酌古今,定其絳制。其以皇慶三年八月,天下郡縣,興其賢者能者,充賦有司,次年二月會試京師,中選者朕將親策焉。具合行事宜於後:科場,每三歲一次開試。舉人從本貫官司於諸色戶內推舉,年及二十五以上,鄉黨稱其孝悌,朋友服其信義,經明行修之士,結罪保舉,以禮敦遣,資諸路府。其或徇私濫舉,並應舉而不舉者,監察御史、肅政廉訪司體察究治。考試程式:蒙古、色目人,第一場經問五絳,《大學》、《論語》、《孟子》、《中庸》內設問,用硃氏章句集注。其義理精明,文辭典雅者為中選。第二場策一道,以時務出題,限五百字以上。漢人、南人,第一場明經經疑二問,《大學》、《論語》、《孟子》、《中庸》內出題,並用硃氏章句集注,復以己意結之,限三百字以上;經義一道,各治一經,《詩》以硃氏為主,《尚書》以蔡氏為主,《周易》以程氏、硃氏為主,已上三經,兼用古注疏,《春秋》許用《三傳》及胡氏《傳》,《禮記》用古注疏,限五百字以上,不拘格律。第二場古賦詔誥章表內科一道,古賦詔誥用古體,章表四六,參用古體。第三場策一道,經史時務內出題,不矜浮藻,惟務直述,限一千字以上成。蒙古、色目人,願試漢人、南人科目,中選者加一等注授。蒙古、色目人作一榜,漢人、南人作一榜。第一名賜進士及第,從六品,第二名以下及第二甲,皆正七品,第三甲以下,皆正八品,兩榜並同。所在官司遲誤開試日期,監察御史、肅政廉訪司糾彈治罪。流官子孫廕敘,並依舊制,願試中選者,優升一等。在官未入流品,願試者聽。若中選之人,已有九品以上資級,比附一高,加一等注授;若無品級,止依試例從優銓注。鄉試處所,並其餘絳目,命中書省議行。於戲!經明行修,庶得真儒之用;風移俗易,益臻至治之隆。咨爾多方,體予至意。」



 中書省所定條目:



 鄉試中選者,各給解據、錄連取中科文,行省移咨都省,送禮部,腹裏宣慰司及各路關申禮部,拘該監察御史、廉訪司,依上錄連科文申臺,轉呈都省,以憑照勘。



 鄉試,八月二十日,蒙古、色目人,試經問五絳;漢人、南人,明經經疑二問,經義一道。二十三日,蒙古、色目人,試策一道;漢人、南人,古賦詔誥章表內科一道。二十六日,漢人、南人,試策一道。



 會試,省部依鄉試例,於次年二月初一日試第一場,初三日第二場,初五日第三場。



 御試,三月初七日,前期奏委考試官二員、監察御史二員、讀卷官二員,入殿廷考試。



 每舉子一名,怯薛歹一人看守。漢人、南人,試策一道,限一千字以上成。蒙古、色目人,時務策一道,限五百字以上成。



 選考試官,行省與宣慰司及腹裏各路,有行臺及廉訪司去處,與臺憲官一同商議選差。上都、大都從省部選差在內監察御史、在外廉訪司官一員監試。每處差考試官、同考試官各一員,並於見任並在閑有德望文學常選官內選差;封彌官一員、謄錄官一員,選廉幹文資正官充之。凡謄錄試卷並行移文字,皆用硃書,仍須設法關防,毋致容私作弊。省部會試,都省選委知貢舉、同知貢舉官各一員,考試官四員,監察御史二員,彌封、謄錄、對讀官、監門等官各一員。



 鄉試,行省一十一:河南,陜西,遼陽,四川,甘肅,雲南,嶺北,征東,江浙,江西,湖廣。宣慰司二:河東,山東。直隸省部路分四:真定,東平,大都,上都。



 天下選合格者三百人赴會試,於內取中選者一百人,內蒙古、色目、漢人、南人分卷考試,各二十五人,蒙古人取合格者七十五人:大都十五人,上都六人,河東五人,真定等五人,東平等五人。山東四人,遼陽五人,河南五人,陜西五人,甘肅三人,嶺北三人,江浙五人,江西三人,湖廣三人,四川一人,雲南一人,征東一人。色目人取合格者七十五人:大都十人,上都四人,河東四人,東平等四人,山東五人,真定等五人,河南五人,四川三人,甘肅二人,陜西三人,嶺北二人,遼陽二人,雲南二人,征東一人,湖廣七人,江浙一十人,江西六人。漢人取合格者七十五人:大都一十人,上都四人,真定等十一人,東平等九人,山東七人,河東七人,河南九人,四川五人,雲南二人,甘肅二人,嶺北一人,陜西五人,遼陽二人,征東一人。南人取合格者七十五人:湖廣一十八人,江浙二十八人,江西二十二人,河南七人。



 鄉試、會試,許將《禮部韻略》外,餘並不許懷挾文字。差搜檢懷挾官一員,每舉人一名差軍一名看守,無軍人處,差巡軍。



 提點擗掠試院,差廉幹官一員,度地安置席舍,務令隔遠,仍自試官入院後,常川妨職,監押外門。



 鄉試、會試,彌封、謄錄、對讀官下吏人,於各衙門從便差設。



 試卷不考格,犯御名廟諱及文理紕繆、塗注乙五十字以上者,不考。謄錄所承受試卷,並用硃書謄錄正文,實計塗注乙字數,標寫對讀無差,將硃卷逐旋送考試所。如硃卷有塗注乙字,亦皆標寫字數,謄錄官書押。候考校合格,中選人數已定,抄錄字號,索上元卷,請監試官、知貢舉官、同試官,對號開拆。



 舉人試卷,各人自備三場文卷並草卷,各一十二幅,於卷首書三代、籍貫、年甲,前期半月於印卷所投納。置簿收附,用印鈐縫訖,各還舉人。



 凡就試之日,日未出入場,黃昏納卷。受卷官送彌封所,撰字號,封彌訖,送謄錄所。



 科舉既行之後,若有各路歲貢及保舉儒人等文字到官,並令還赴本鄉應試。



 倡優之家及患廢疾、若犯十惡奸盜之人,不許應試。



 舉人於試場內,毋得喧嘩,違者治罪,仍殿二舉。



 舉人與考試官有五服內親者,自須回避,仍令同試官考卷。若應避而不自陳者,殿一舉。



 鄉試、會試,若有懷挾及令人代作者,漢人、南人有居父母喪服應舉者,並殿二舉。



 國子監學歲貢生員及伴讀出身,並依舊制,願試者聽。中選者,於監學合得資品上從優銓注。



 別路附籍蒙古、色目、漢人,大都、上都有恆產、住經年深者,從兩都官司,依上例推舉就試,其餘去處冒貫者治罪。



 知貢舉以下官會集至公堂,議擬合行事目云:



 諸輒於彌封所取問舉人試卷封號姓名及漏洩者,治罪。諸試題未出而漏洩者,許人告首。諸對讀試卷官不躬親而輒令人吏對讀,其對讀訖而差誤有礙考校者,有罰。諸謄錄人書寫不慎及錯誤有礙考校者,重事責罰。諸官司故縱舉人私將試卷出院,及祗應人知而為傳送者,許人告首。諸監試官掌試院事,不得干預考校。諸試院官在簾內者,不許與簾外官交語。諸色人無故不得入試。諸舉人謗毀主司,率眾喧競,不服止約者,治罪。諸舉人就試,無故不冠及擅移坐次者,或偶與親姻鄰坐而不自陳者,懷挾代筆傳義者,並扶出。諸拆毀試卷首家狀者,推治。諸舉人於試卷書他語者,駁放;涉謗訕者,推治。諸試日,為舉人傳送文書,及因而受財者,並許人告。諸舉人於別紙上起草者,出榜退落。諸科文內不得自敘苦辛門第,委謄錄所點檢得,如有違犯,更不謄錄,移文考試院出榜退落。諸冒名就試,別立姓名,反受財為人懷挾代筆傳義者,並許人告。諸被黜而妄訴者,治罪。諸監門官譏察出入,其物應入者,拆封點檢。諸巡鋪官及兵級,不得喧擾,及輒視試文,並容縱舉人無故往來,非因公事,不得與舉人私語。諸試卷彌封用印訖,以三不成字為號標寫,仍於塗注乙處用印。



 每舉人一名,給祗應巡軍一人,隔夜入院,分宿席房。試日,擊鐘為節。一次,院官以下皆盥漱。二次,監門官啟鑰,舉人入院,搜檢訖,就將解據呈納。禮生贊曰「舉人再拜」,知貢舉官隔簾受一拜,跪答一拜,試官受一拜,答一拜。鐘三次,頒題,就次。日午,賜膳。其納卷首,赴受卷所揖而退,不得交語。受卷官書舉人姓名於歷,舉人揖而退,取解據出院,巡軍亦出。至晚,鳴鐘一次,鎖院門。第二場,舉人入院,依前搜檢,每十人一甲,序立至公堂下,作揖畢,頒題就次。第三場,如前儀。



 其受卷官具受到試卷,逐旋關發彌封官,將家狀草卷,腰封用印,蒙古、色目、漢人、南人分卷,以三不成字撰號。每名累場同用一號,於卷上親書,及於歷內標附訖,牒送謄錄官置歷,分給吏人,並用硃書謄錄正文,仍具元卷塗注乙及謄錄塗注乙字數,卷末書謄錄人姓名,謄錄官具銜書押,用印鈐縫,牒送對讀所。翰林掾史具謄錄訖試卷總數,呈報監察御史。對讀官以元卷與硃卷躬親對讀無差,具銜書押,呈解貢院,元卷發還彌封所。各所行移,並用硃書,試卷照依元號附簿。



 試官考卷,知貢舉居中,試官相對向坐,公同考校,分作三等,逐等又分上中下,用墨筆批點。考校既定,收掌試卷官於號簿內標寫分數,知貢舉官、同試官、監察御史、彌封官,公同取上元卷對號開拆,知貢舉於試卷家狀上親書省試第幾名。拆號既畢,應有試卷並付禮部架閣,貢舉諸官出院。中書省以中選舉人分為二榜,揭於省門之左右。



 三月初四日,中書省奏準,以初七日御試舉人於翰林國史院,定委監試官及諸執事。初五日,各官入院。初六日,譔策問進呈,俟上採取。初七日,執事者望闕設案於堂前,置策題於上。舉人入院,搜檢訖,蒙古人作一甲,序立,禮生導引至於堂前,望闕兩拜,賜策題,又兩拜,各就次。色目人作一甲,漢人、南人作一甲,如前儀。每進士一人,差蒙古宿衛士一人監視。日午,賜膳。進士納卷畢,出院。監試官同讀卷官,以所對策第其高下,分為三甲進奏。作二榜,用敕黃紙書,揭於內前紅門之左右。



 前一日,禮部告諭中選進士,以次日詣闕前,所司具香案,侍儀舍人唱名,謝恩,放榜。擇日賜恩榮宴於翰林國史院,押宴以中書省官,凡預試官並與宴。預宴官及進士並簪華至所居。擇日恭詣殿廷,上謝恩表。次日,詣中書省參見。又擇日,諸進士詣先聖廟行舍菜禮,第一人具祝文行事,刻石題名於國子監。



 延祐二年春三月,廷試進士,賜護都答兒、張起巖等五十有六人及第、出身有差。五年春三月,廷試進士護都達兒、霍希賢等五十人。



 至治元年春三月,廷試進士達普化、宋本等六十有四人。



 泰定元年春三月,廷試進士捌剌、張益等八十有六人。



 四年春三月,廷試進士阿察赤、李黼等八十有六人。



 天歷三年春三月,廷試進士篤列圖、王文燁等九十有七人。



 元統癸酉科,廷試進士同同、李齊等,復增名額,以及百人之數。稍異其制,左右榜各三人,皆賜進士及第,其餘出身有差。科舉取士,莫盛於斯。後三年,其制遂罷。又七年而復興,遂稍變程式,減蒙古、色目人明經二絳,增本經義;易漢、南人第一場《四書》疑一道為本經疑,增第二場古賦外,於詔誥、章表內又科一道。此有元科目取士之制,大略如此。



 若夫會試下第者,自延祐創設之初,丞相帖木迭兒、阿散及平章李孟等奏:「下第舉人,年七十以上者,與從七品流官致仕;六十以上者,與教授;元有出身者,於應得資品上稍優加之;無出身者,與山長、學正。受省劄,後舉不為例。今有來遲而不及應試者,未曾區用。取旨。」帝曰:「依下第例恩之,勿著為格。」



 泰定元年三月,中書省臣奏:「下第舉人,仁宗延祐間,命中書省各授教官之職,以慰其歸。今當改元之初,恩澤宜溥。蒙古、色目人,年三十以上並兩舉不第者,與教授;以下,與學正、山長。漢人、南人,年五十以上並兩舉不第者,與教授;以下,與學正、山長。先有資品出身者,更優加之;不願仕者,令備國子員。後勿為格。」從之。自餘下第之士,恩例不可常得,間有試補書吏以登仕籍者。惟已廢復興之後,其法始變,下第者悉授以路府學正及書院山長。又增取鄉試備榜,亦授以郡學錄及縣教諭。於是科舉取士,得人為盛焉。



 學校



 世祖至元八年春正月,始下詔立京師蒙古國子學,教習諸生,於隨朝蒙古、漢人百官及怯薛歹官員,選子弟俊秀者入學,然未有員數。以《通鑒節要》用蒙古語言譯寫教之,俟生員習學成效,出題試問,觀其所對精通者,量授官職。成宗大德十年春二月,增生員廩膳,通前三十員為六十員。武宗至大二年,定伴讀員四十人,以在籍上名生員學問優長者補之。仁宗延祐二年冬十月,以所設生員百人,蒙古五十人,色目二十人,漢人三十人,而百官子弟之就學者,常不下二三百人,宜增其廩餼,乃減去庶民子弟一百一十四員,聽陪堂學業,於見供生員一百名外,量增五十名。元置蒙古二十人,漢人三十人,其生員紙札筆墨止給三十人,歲凡二次給之。



 至元六年秋七月,置諸路蒙古字學。十二月,中書省定學制頒行之,命諸路府官子弟入學,上路二人,下路二人,府一人,州一人。餘民間子弟,上路三十人,下路二十五人。願充生徒者,與免一身雜役。以譯寫《通鑒節要》頒行各路,俾肄習之。至成宗大德五年冬十月,又定生員,散府二十人,上、中州十五人,下州十人。元貞元年,命有司割地,給諸路蒙古學生員餼廩。其學官,至元十九年,定擬路府州設教授,以國字在諸字之右,府州教授一任,準從八品,再歷路教授一任,準正八品,任回本等遷轉。大德四年,添設學正一員,上自國學,下及州縣,舉生員高等,從翰林考試,凡學官譯史,取以充焉。



 世祖至元二十六年夏五月,尚書省臣言:「亦思替非文字宜施於用,今翰林院益福的哈魯丁能通其字學,乞授以學士之職,凡公卿大夫與夫富民之子,皆依漢人入學之制,日肄習之。」帝可其奏。是歲八月,始置回回國子學。至仁宗延祐元年四月,復置回回國子監,設監官,以其文字便於關防取會數目,令依舊制,篤意領教。泰定二年春閏正月,以近歲公卿大夫子弟與夫凡民之子入學者眾,其學官及生員五十餘人,已給飲膳者二十七人外,助教一人、生員二十四人廩膳,並令給之。學之建置在於國都,凡百司庶府所設譯史,皆從本學取以充焉。



 太宗六年癸巳,以馮志常為國子學總教,命侍臣子弟十八人入學。世祖至元七年,命侍臣子弟十有一人入學,以長者四人從許衡,童子七人從王恂。至二十四年,立國子學,而定其制。設博士,通掌學事,分教三齋生員,講授經旨,是正音訓,上嚴教導之術,下考肄習之業。復設助教,同掌學事,而專守一齋;正、錄,申明規矩,督習課業。凡讀書必先《孝經》、《小學》、《論語》、《孟子》、《大學》、《中庸》,次及《詩》、《書》、《禮記》、《周禮》、《春秋》、《易》。博士、助教親授句讀、音訓,正、錄、伴讀以次傳習之。講說則依所讀之序,正、錄、伴讀亦以次而傳習之。次日,抽簽,令諸生復說其功課。對屬、詩章、經解、史評,則博士出題,生員具稿,先呈助教,俟博士既定,始錄附課簿,以憑考校。其生員之數,定二百人,先令一百人及伴讀二十人入學。其百人之內,蒙古半之,色目、漢人半之。許衡又著諸生入學雜儀,及日用節目。七年,命生員八十人入學,俾永為定式而遵行之。



 成宗大德八年冬十二月,始定國子生,蒙古、色目、漢人三歲各貢一人。十年冬閏十月,國子學定蒙古、色目、漢人生員二百人,三年各貢二人。



 武宗至大四年秋閏七月,定生員額二百人。冬十二月,復立國子學試貢法,蒙古授官六品,色目正七品,漢人從七品。試蒙古生之法宜從寬,色目生宜稍加密,漢人生則全科場之制。



 仁宗延祐二年秋八月,增置生員百人,陪堂生二十人,用集賢學士趙孟頫、禮部尚書元明善等所議國子學貢試之法更定之。一曰升齋等第。六齋東西相向,下兩齋左曰游藝,右曰依仁,凡誦書講說、小學屬對者隸焉。中兩齋左曰據德,右曰志道,講說《四書》、課肄詩律者隸焉。上兩齋左曰時習,右曰日新,講說《易》、《書》、《詩》、《春秋》科,習明經義等程文者隸焉。每齋員數不等,每季考其所習經書課業,及不違規矩者,以次遞升。二曰私試規矩。漢人驗日新、時習兩齋,蒙古色目取志道、據德兩齋,本學舉實歷坐齋二周歲以上,未嘗犯過者,許令充試。限實歷坐齋三周歲以上,以充貢舉。漢人私試,孟月試經疑一道,仲月試經義一道,季月試策問、表章、詔誥科一道。蒙古、色目人,孟、仲月各試明經一道,季月試策問一道。辭理俱優者為上等,準一分;理優辭平者為中等,準半分。每歲終,通計其年積分,至八分以上者升充高等生員,以四十名為額,內蒙古、色目各十名,漢人二十名。歲終試貢,員不必備,惟取實才。有分同闕少者,以坐齋月日先後多少為定。其未及等,並雖及等無闕未補者,其年積分,並不為用,下年再行積算。每月初二日蚤旦,圓揖後,本學博士、助教公座,面引應試生員,各給印紙,依式出題考試,不許懷挾代筆,各用印紙,真楷書寫,本學正、錄彌封謄錄,餘並依科舉式,助教、博士以次考定。次日,監官覆考,於名簿內籍記各得分數,本學收掌,以俟歲終通考。三曰黜罰科絳。應私試積分生員,其有不事課業及一切違戾規矩者,初犯罰一分,再犯罰二分,三犯除名,從學正、錄糾舉,正、錄知見而不糾舉者,從本監議罰之。應已補高等生員,其有違戾規矩者,初犯殿試一年,再犯除名,從學正、錄糾舉之,正、錄知見而不糾舉者,亦從本監議罰之。應在學生員,歲終實歷坐齋不滿半歲者,並行除名。除月假外,其餘告假,並不準算。學正、錄歲終通行考校應在學生員,除蒙古、色目別議外,其餘漢人生員三年不能通一經及不肯勤學者,勒令出學。其餘責罰,並依舊規。



 泰定三年夏六月,更積分而為貢舉,並依世祖舊制。其貢試之法,從監學所擬,大概與前法略同,而防閑稍加嚴密焉。其本學正、錄各二員,司樂一員,典籍二員,管勾一員,及侍儀舍人,舊例舉積分生員充之,後以積分既革,於上齋舉年三十以上、學行堪範後學者為正、錄,通曉音律、學業優贍者為司樂,幹局通敏者為典籍、管勾。其侍儀舍人,於上、中齋舉禮儀習熟、音吐洪暢、曾掌春秋釋奠、每月告朔明贊、眾與其能者充之。文宗天歷二年春三月,惟伴讀員數,自初二十人歲貢二人,後於大德七年定四十人歲貢六人,至大四年定四十人歲貢四人,延祐四年歲貢八人為淹滯,既額設四十名,宜充部令史者四人、路教授者四人。是後,又命所貢生員,每大比選士,與天下士同試於禮部,策於殿廷,又增至備榜而加選擇焉。



 國初,燕京始平,宣撫王楫請以金樞密院為宣聖廟。太宗六年,設國子總教及提舉官,命貴臣子弟入學受業。憲宗四年,世祖在潛邸,特命修理殿廷;及即位,賜以玉斝,俾永為祭器。至元十三年,授提舉學校官六品印,遂改為大都路學,署曰提舉學校所。二十四年,既遷都北城,立國子學於國城之東,乃以南城國子學為大都路學,自提舉以下,設官有差。仁宗延祐四年,大興府尹馬思忽重修殿門堂廡,建東西兩齋。泰定三年,府尹曹偉增建環廊。文宗天歷二年,復增廣之,提舉郝義恭又增建齋舍。自府尹郝朵而別至曹偉,始定生員凡百人,每名月餼,京畿漕運司及本路給之。泰定四年夏四月,諸生始會食於學焉。



 太宗始定中原,即議建學,設科取士。世祖中統二年,始命置諸路學校官,凡諸生進修者,嚴加訓誨,務使成材,以備選用。至元十九年夏四月,命雲南諸路皆建學以祀先聖。二十三年二月,帝御德興府行宮,詔江南學校舊有學田,復給之以養士。二十八年,令江南諸路學及各縣學內,設立小學,選老成之士教之,或自願招師,或自受家學於父兄者,亦從其便。其他先儒過化之地,名賢經行之所,與好事之家出錢粟贍學者,並立為書院。凡師儒之命於朝廷者,曰教授,路府上中州置之。命於禮部及行省及宣慰司者,曰學正、山長、學錄、教諭,路州縣及書院置之。路設教授、學正、學錄各一員,散府上中州設教授一員,下州設學正一員,縣設教諭一員,書院設山長一員。中原州縣學正、山長、學錄、教諭,並受禮部付身。各省所屬州縣學正、山長、學錄、教諭,並受行省及宣慰司劄付。凡路府州書院,設直學以掌錢穀,從郡守及憲府官試補。直學考滿,又試所業十篇,升為學錄、教諭。凡正、長、學錄、教諭,或由集賢院及臺憲等官舉充之。諭、錄歷兩考,升正、長。正、長一考,升散府上中州教授。上中州教授又歷一考,升路教授。教授之上,各省設提舉二員,正提舉從五品,副提舉從七品,提舉凡學校之事。後改直學考滿為州吏,例以下第舉人充正、長,備榜舉人充諭、錄,有薦舉者,亦參用之。自京學及州縣學以及書院,凡生徒之肄業於是者,守令舉薦之,臺憲考核之,或用為教官,或取為吏屬,往往人材輩出矣。



 世祖中統二年夏五月,太醫院使王猷言:「醫學久廢,後進無所師授。竊恐朝廷一時取人,學非其傳,為害甚大。」乃遣副使王安仁授以金牌,往諸路設立醫學。其生員擬免本身檢醫差占等役,俟其學有所成,每月試以疑難,視其所對優劣,量加勸懲。後又定醫學之制,設諸路提舉綱維之。凡宮壼所需,省臺所用,轉入常調,可任親民,其從太醫院自遷轉者,不得視此例,又以示仕途不可以雜進也。然太醫院官既受宣命,皆同文武正官五品以上遷敘,餘以舊品職遞升,子孫廕用同正班敘。其掌藥,充都監直長,充御藥院副使,升至大使,考滿依舊例於流官銓注。諸教授皆從太醫院定擬,而各路主善亦擬同教授皆從九品。凡隨朝太醫,及醫官子弟,及路府州縣學官,並須試驗。其各處名醫所述醫經文字,悉從考校。其諸藥所產性味真偽,悉從辨驗。其隨路學校,每歲出降十三科疑難題目,具呈太醫院,發下諸路醫學,令生員依式習課醫義,年終置簿解納送本司,以定其優劣焉。



 世祖至元二十八年夏六月,始置諸路陰陽學。其在腹裏、江南,若有通曉陰陽之人,各路官司詳加取勘,依儒學、醫學之例,每路設教授以訓誨之。其有術數精通者,每歲錄呈省府,赴都試驗,果有異能,則於司天臺內許令近侍。延祐初,令陰陽人依儒、醫例,於路府州設教授一員,凡陰陽人皆管轄之,而上屬於太史焉。



 舉遺逸以求隱跡之士,擢茂異以待非常之人。世祖中統間,徵許衡,授懷孟路教官,詔於懷孟等處選子弟之俊秀者教育之。是年,又詔征金進士李冶,授翰林學士。征劉因為集賢學士,不至。又用平章咸寧王野仙薦,徵蕭不起,即授陜西儒學提舉。至元十八年,詔求前代聖賢之後,儒醫卜筮,通曉天文歷數,並山林隱逸之士。二十年,復召拜劉因右贊善大夫,辭,不允。未幾以拋老,乞終養,俸給一無所受。後遣使授命於家,辭疾不起。二十八年,復詔求隱晦之士,俾有司具以名聞。成宗大德六年,徵臨川布衣吳澂,擢應奉翰林文字,拜命即歸。九年,詔求山林間有德行文學、識治道者,遣使徵蕭渼,且曰:「或不樂於仕,可試一來,與朕語而遣歸。」至大三年,復召吳澄,拜國子司業,以病還;延祐三年,召拜集賢直學士,以疾不赴;至治三年,召拜翰林學士。武宗、仁宗累徵蕭渼,授集賢學士、國子司業,未赴,改集賢侍講學士。又以太子右諭德征,始至京師,授集賢學士、國子祭酒,諭德如故。仁宗延祐七年十一月,詔曰:「比歲設立科舉,以取人材,尚慮高尚之士,晦跡丘園,無從可致。各處其有隱居行義、才德高邁、深明治道、不求聞達者,所在官司具姓名,牒報本道廉訪司,覆奏察聞,以備錄用。」又屢詔求言於下,使得進言於上,雖指斥時政,並無譴責,往往採擇其言,任用其人,列諸庶位,以圖治功。其他著書立言、裨益教化、啟迪後人者,亦斟酌錄用,著為常式云。



 童子舉,唐、宋始著於科,然亦無常員。成宗大德三年,舉童子楊山童、海童。五年,大都提舉學校所舉安西路張秦山,江浙行省舉張升甫。武宗至大元年,舉武福安。仁宗延祐三年,江浙行省舉俞傅孫、馮怙哥。六年,河南路舉張答罕,學士完者不花舉丁頑頑。七年,河間縣舉杜山童,大興縣舉陳聃。英宗至治元年,福州路連江縣舉陳元麟。至治三年,河南行省舉張英。泰定四年,福州舉葉留井。文宗天歷二年,舉杜夙靈。至順二年,制舉答不歹子買來的。皆以其天資穎悟,超出兒輩,或能默誦經文,書寫大字,或能綴緝辭章,講說經史,並令入國子學教育之。惟張秦山尤精篆涘,陳元麟能通性理,葉留畊問以《四書》大義,則對曰:「無過事父母能竭其力,事君能致其身。」時人以遠大期之。



\end{pinyinscope}