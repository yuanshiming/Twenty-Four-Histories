\article{志第十一 地理二}

\begin{pinyinscope}

 遼陽等處行中書省,為路七、府一,屬州十二,屬縣十。徒存其名而無城邑者,不在此數。本省計站一百二十處。



 遼陽路,上。唐以前為高句驪及渤海大氏所有。梁貞明中,阿保機以遼陽故城為東平郡。後唐升為南京。石晉改為東京。金置遼陽府,領遼陽、鶴野二縣;後復改為東京,宜豐、澄、復、蓋、沈、貴德州、廣寧府、來遠軍並屬焉。元初廢貴德、澄、復州、來遠軍,以廣寧府、婆娑府、懿州、蓋州作四路,直隸省。至元六年,置東京總管府,降廣寧為散府隸之。十五年,割廣寧仍自行路事,直隸省。十七年,又以婆娑府、懿州、蓋州來屬。二十四年,始立行省。二十五年,改東京為遼陽路,後廢婆娑府為巡檢司。戶三千七百八,口三萬三千二百三十一。壬子年抄籍數。領縣一、州二。



 縣一



 遼陽。下。倚郭。至元六年,以鶴野縣、警巡院入焉。



 州二



 蓋州,下。初為蓋州路。至元六年,並為東京支郡,並熊岳、湯池二縣入建安縣。八年,又並建安縣入本州。



 懿州,下。初為懿州路。至元六年為東京支郡,所領豪州及同昌、靈山二縣省入順安縣,入本州。



 廣寧府路,下。金為廣寧府。元封孛魯古歹為廣寧王,舊立廣寧行帥府事;後以地遠,遷治臨潢,立總管府。至元六年,以戶口單寡,降為東京路總管府屬郡。十五年,復分為路,行總管府事。有醫巫閭山為北鎮,在府城西北二十里。至順錢糧戶數四千五百九十五。領縣二:



 閭陽,下。初立千戶所,至元十五年,以戶口繁夥,復立行千戶所。後復為閭陽縣。望平。至元六年,省鐘秀縣入焉。十五年,為望平軍民千戶所,今復為縣。



 肇州。按《哈剌八都魯傳》至元三十年,世祖謂哈剌八都魯曰:「乃顏故地曰阿八剌忽者產魚,吾今立城,而以兀速、憨哈納思、乞里吉思三部人居之,名其城曰肇州,汝往為宣慰使。」既至,定市里,安民居,得魚九尾皆千斤來獻。又《成宗紀》元貞元年,立肇州屯田萬戶府,以遼陽行省左丞阿散領其事。而《元一統志》與《經世大典》皆不載此州,不知其所屬所領之詳。今以廣寧為乃顏分地,故府注于廣寧府之下。乃顏,孛魯古歹之孫也。



 山北遼東道肅政廉訪司



 大寧路,上。本奚部,唐初其地屬營州,貞觀中奚酋可度內附,乃置饒樂郡。遼為中京大定府。金因之。元初為北京路總管府,領興中府及義、瑞、興、高、錦、利、惠、川、建、和十州。中統三年,割興州及松山縣屬上都路。至元五年,並和州入利州為永和鄉。七年,興中府降為州,仍隸北京,改北京為大寧。二十五年,改為武平路,後復為大寧。戶四萬六千六,口四十四萬八千一百九十三。壬子年數。領司一、縣七、州九。



 錄事司。初置警巡院,至元二年,改置錄事司。



 縣七



 大定,下。中統二年,省長興入焉。龍山,下。初屬大定府。至元四年,屬利州,後復來屬。富庶,下。至元三年,省入興中州,後復置。和眾,下。金源,下。惠和,下。武平。下。



 州九



 義州。下。



 興中州,下。元初因舊為興中府,後省。至元七年,又降府為州。



 瑞州。下。至元二十三年,伯顏奏準以唆都、哈鷿等拘收戶計,種田立屯於瑞州之西,撥瀕海荒閑地及時開耕,設打捕屯田總管府,仍以唆都、哈鷿等為屯田官。



 高州。下。



 錦州。下。



 利州。下。



 惠州。下。



 川州。下。



 建州。下。



 東寧路,本高句驪平壤城,亦曰長安城。漢滅朝鮮,置樂浪、玄菟郡,此樂浪地也。晉義熙後,其王高璉始居平壤城。唐征高麗,拔平壤,其國東徙,在鴨綠水之東南千餘里,非平壤之舊。至王建,以平壤為西京。元至元六年,李延齡、崔垣、玄元烈等以府州縣鎮六十城來歸。八年,改西京為東寧府。十三年,升東寧路總管府,設錄事司,割靜州、義州、麟州、威遠鎮隸婆娑府。本路領司一,餘城堙廢,不設司存,今姑存舊名。



 錄事司。土山縣。中和縣。鐵化鎮。



 都護府,自唐之季,地入高麗,置府州縣鎮六十餘城,此為都護府,雖仍唐舊名,而無都護府之實。至元六年,李延齡等以其地來歸,後城治廢毀,僅存其名,屬東寧路。



 定遠府。郭州。撫州。黃州。領安岳、三和、龍岡、咸從、江西五縣,長命一鎮。靈州。慈州。嘉州。順州。殷州。宿州。德州。領江東、永清、通海、順化四縣,寧遠、柔遠、安戎三鎮。昌州。鐵州。領定戎一鎮。



 泰州。價州。朔州。宣州。領寧朔、席島二鎮。成州。領樹德一鎮。熙州。孟州。領三登一縣,椒島、椴島、寧德三鎮。延州。領陽巖一鎮。雲州。



 沈陽路,本挹婁故地,渤海大氏建定理府,都督沈、定二州,此為沈州地。契丹為興遼軍,金為昭德軍,又更顯德軍,後皆毀於兵火。元初平遼東,高麗國麟州神騎都領洪福源率西京、都護、龜州四十餘城來降,各立鎮守司,設官以撫其民。後高麗復叛,洪福源引眾來歸,授高麗軍民萬戶,徙降民散居遼陽沈州,初創城郭,置司存,僑治遼陽故城。中統二年,改為安撫高麗軍民總管府。及高麗舉國內附,四年,又以質子淳為安撫高麗軍民總管,分領二千餘戶,理沈州。元貞二年,並兩司為沈陽等路安撫高麗軍民總管府,仍治遼陽故城,轄總管五、千戶二十四、百戶二十五。至順錢糧戶數五千一百八十三。



 開元路,古肅慎之地,隋、唐曰黑水靺鞨。唐初,渠長阿固郎始來朝,後乃臣服,以其地為燕州,置黑水府。其後渤海盛,靺鞨皆役屬之。又其後渤海浸弱,為契丹所攻,黑水復擅其地,東瀕海,南界高麗,西北與契丹接壤,即金鼻祖之部落也。初號女真,後避遼興宗諱,改曰女直。太祖烏古打既滅遼,即上京設都,海陵遷都於燕,改為會寧府。金末,其將蒲鮮萬奴據遼東。元初癸巳歲,出師伐之,生禽萬奴,師至開元、率賓,東土悉平。開元之名,始見於此。乙未歲,立開元、南京二萬戶府,治黃龍府。至元四年,更遼東路總管府。二十三年,改為開元路,領咸平府,後割咸平為散府,俱隸遼東道宣慰司。至順錢糧戶數四千三百六十七。



 咸平府,古朝鮮地,箕子所封,漢屬樂浪郡,後高麗侵有其地。唐滅高麗,置安東都護以統之,繼為渤海大氏所據。遼平渤海,以其地多險隘,建城以居流民,號咸州安東軍,領縣曰咸平。金升咸平府,領平郭、安東、新興、慶雲、清安、歸仁六縣,兵亂皆廢。元初因之,隸開元路,後復割出,隸遼東宣慰司。



 合蘭府水達達等路,土地曠闊,人民散居。元初設軍民萬戶府五,撫鎮北邊。一曰桃溫,距上都四千里。一曰胡里改,距上都四千二百里、大都三千八百里。有胡里改江並混同江,又有合蘭河流入於海。一曰斡朵憐。一曰脫斡憐。一曰孛苦江。各有司存,分領混同江南北之地。其居民皆水達達、女直之人,各仍舊俗,無市井城郭,逐水草為居,以射獵為業。故設官牧民,隨俗而治,有合蘭府水達達等路,以相統攝焉。有俊禽曰海東青,由海外飛來,至奴兒乾,土人羅之,以為土貢。至順錢糧戶數二萬九百六。



 河南江北等處行中書省,為路十二、府七、州一,屬州三十四,屬縣一百八十二。本省陸站一百六處,水站九十處。



 河南江北道肅政謙訪司



 汴梁路,上。唐置汴州總管府。石晉為開封府。宋為東京,建都於此。金改南京,宣宗南遷,都焉。金亡,歸附。舊領歸德府,延、許、裕、唐、陳、亳、鄧、汝、潁、徐、邳、嵩、宿、申、鄭、鈞、睢、蔡、息、盧氏行襄樊二十州。至元八年,令歸德自為一府,割亳、徐、邳、宿四州隸之;升申州為南陽府,割裕、唐、汝、鄭、嵩、盧氏行襄樊隸之。九年,廢延州,以所領延津、陽武二縣屬南京路,統蔡、息、鄭、鈞、許、陳、睢、潁八州,開封、祥符倚郭,而屬邑十有五。舊有警巡院,十四年改錄事司。二十五年,改南京路為汴梁路。二十八年,以瀕河而南、大江以北,其地沖要,又新入版圖,置省南京以控治之。三十年,升蔡州為汝寧府,屬行省,割息、潁二州以隸焉。本路戶三萬一十八,口一十八萬四千三百六十七。壬子年數。領司一、縣十七、州五。州領二十一縣。



 錄事司



 縣十七



 開封,下。倚郭。祥符,下。倚郭。中牟,下。原武,下。舊以此縣隸延州,元初隸開封府,後復為延州,縣如舊。至元九年,州廢,後來屬。鄢陵,中。滎澤,下。舊隸鄭州,至元二年來屬。封丘,中。金大定中,河水湮沒,遷治新城。元初,新城又為河水所壞,乃因故城遺址,稍加完葺而遷治焉。扶溝,下。陽武,下。舊隸延州,至元九年,州廢來屬。杞縣,中。元初河決,城之北面為水所圯,遂為大河之道,乃於故城北二里河水北岸,築新城置縣,繼又修故城,號南杞縣。蓋黃河至此分為三,其大河流於二城之間,其一流於新城之北郭睢河中,其一在故城之南,東流,俗稱三叉口。延津,下。舊為延州,隸河南路。至元九年,州廢,以縣來屬。蘭陽,下。通許,下。



 尉氏,下。太康,下。洧川,下。陳留。下。



 州五



 鄭州,下。唐初為鄭州,又改滎陽郡。宋為奉寧軍。金仍為鄭。元初領管城、滎陽、汜水、河陰、原武、新鄭、密、滎澤八縣及司候司,後割新鄭、密屬鈞州,滎澤、原武隸開封府,並司候司入管城。領四縣:



 管城,下。倚郭。滎陽,下。汜水,下。河陰,下。



 許州,下。唐初為許州,後改潁川郡,又仍為許州。宋升潁昌府。金改昌武軍。元初復為許州。領五縣:



 長社,下。長葛,下。郾城,下。襄城,下。臨潁。下。



 陳州,下。唐初為陳州,後改淮陽郡,又仍為陳州。宋升懷德府。金復為陳州。元初因之。舊領宛丘、南頓、項城、商水、西華、清水六縣。至元二年,南頓、項城、清水皆廢,後復置南頓、項城。領五縣:



 宛丘,西華,商水,至元二年,省南頓、項城入焉,後復置。南頓,項城。



 鈞州,下。唐、宋皆不置郡,偽齊置潁順軍。金改潁順州,又改鈞州。元至元二年,又割鄭州密縣來屬。領三縣:



 陽翟,下。新鄭,下。密縣。下。



 睢州,下。唐屬曹州。宋改拱州,又升保慶軍。金改睢州。元因之。領四縣:



 襄邑,下。倚郭。考城,下。儀封,下。柘城。下。



 河南府路,唐初為洛州,後改河南府,又改東京。宋為西京。金為中京金昌府。元初為河南府,府治即周之王城。舊領洛陽、宜陽、永寧、登封、鞏、偃師、孟津、新安、澠池九縣,後割澠池隸陜州。戶九千五百二,口六萬五千七百五十一。壬子年數。領司一、縣八、州一。州領四縣。



 錄事司。



 縣八



 洛陽,宜陽,下。永寧,下。登封,下。中嶽嵩山在焉。鞏縣,下。孟津,下。新安,偃師。下。



 州一



 陜州,下。唐初為陜州,又改陜府,又改陜郡。宋為保義軍。元仍為陜州。領四縣:



 陜縣,下。靈寶,下。至元三年,省入陜縣,八年,廢虢州為虢略,隸陜州。並虢略治靈寶,以虢略為巡檢司,並硃陽縣入焉。閿鄉,下。至元二年,省湖城縣入焉。澠池。下。金升為韶州,置澠池司候司。元至元三年,省司候司。八年,省韶州,復為縣,隸河南府路,後割以來屬。



 南陽府,唐初為宛州,而縣名南陽,後州廢,以縣屬鄧州。歷五代至宋皆為縣。金升為申州。元至元八年,升為南陽府,以唐、鄧、裕、嵩、汝五州隸焉。二十五年,改屬汴梁路,後直隸行省。戶六百九十二,口四千八百九十三。壬子年數。領縣二、州五。州領十一縣。



 縣二



 南陽,下。倚郭。鎮平。下。



 州五



 鄧州,下。唐初為鄧州,後改南陽郡,又仍為鄧州。宋屬京西南路。金屬南京開封府。舊領穰縣、南陽、內鄉、淅川、順陽五縣。元初以淅川、順陽省入內鄉。舊設錄事司,至元二年並入穰縣。領三縣:



 穰縣,下。倚郭。內鄉,下。至元二年,以順陽來屬。新野。下。



 唐州,下。唐初為顯州,後改唐州。宋屬京西南路。金改裕州。元初復為唐州。至元三年,以民力不及,廢湖陽、比陽、桐柏三縣。領一縣:



 泌陽。倚郭。



 嵩州,下。唐為陸渾、伊闕二縣。宋升順州。金改嵩州,領伊陽、福昌二縣。元初以福昌隸河南。至元三年,省伊陽入州。領一縣:



 盧氏。下。至元二年,隸南京路。八年,屬南陽府。十一年來屬。



 汝州,下。唐初為伊州,又改汝州。宋屬京西北路。元至元三年,廢郟城、寶豐二縣入梁縣,後復置郟縣。領三縣:



 梁縣,下。魯山。下。郟縣。下。



 裕州,下。唐初置北澧州,又改魯州,後廢為縣,屬唐州。金升為裕州。舊領方城、舞陽、葉縣。元初即葉縣行隨州事,就置昆陽縣為屬邑。至元三年,罷州,並昆陽、舞陽二縣入葉縣,後復置舞陽。領三縣:



 方城,下。倚郭。葉縣,下。舞陽。下。



 汝寧府,唐蔡州。上蔡、西平、確山、遂平、平輿為屬邑。至元七年,省遂平、平輿入汝陽,隸汴梁路。三十年,河南江北行省平章伯顏言:「蔡州去汴梁地遠,,凡事稽誤,宜升散府。」遂升汝寧府,直隸行省,以息、潁、信陽、光四州隸焉,復置遂平縣。抄籍戶口闕,至順錢糧戶數七千七十五。領縣五、州四。州領十縣。



 縣五



 汝陽,下。元初廢,後置蔡州治此,仍復置縣。上蔡,下。西平,下。



 確山。下。遂平。下。元初省入汝陽,後復置。



 州四



 潁州,下。唐初為信州,後改汝陰郡,又改潁州。宋升順昌府。金復為潁州。舊領汝陰、泰和、沈丘、潁上四縣。元至元二年,省四縣及錄事司入州。後復領三縣:



 太和,下。沈丘,下。潁上。下。



 息州,下。唐初為息州,後為新息縣,隸蔡州。五代至宋皆因之。金復置息州。舊領新息、新蔡、真陽、褒信四縣。元中統三年,以李璮叛,廢州。四年,復置。至元三年,以四縣並入州。後復領二縣:



 新蔡,下。真陽。下。



 光州,下。唐初為光州,後改弋陽郡,又復為光州。宋升光山軍。元至元十二年歸附,屬蘄黃宣慰司。二十二年,同蘄、黃等州,直隸行省。三十年,隸汝寧府。領三縣:



 定城,固始,下。宋末兵亂,徙治無常。至元十一年復舊治。光山。下。兵亂地荒,至元十二年復立舊治。



 信陽州,下。唐初為申州,又改義陽郡。宋改信陽軍,端平間,兵亂地荒,凡四十餘年。元至元十四年,改立信陽府,領羅山、信陽二縣。十五年,改為信陽州。二十年,以羅山縣當驛置要沖,徙州治此,而移縣治於西南,號曰羅山新縣,今州治即舊縣。戶三千四百一十四,口三萬三千七百五十一。至元七年數。領二縣:



 羅山。倚郭。信陽。



 歸德府,唐宋州,又為睢陽郡。後唐為歸德軍。宋升南京。金為歸德府。金亡,宋復取之。舊領宋城、寧陵、下邑、虞城、穀熟、碭山六縣。元初與亳之酂縣同時歸附,置京東行省,未幾罷。歲壬子,又立司府州縣官,以綏定新居之民。中統二年,審民戶多寡,定官吏員數。至元二年,以虞城、碭山二縣在枯黃河北,割屬濟寧府,又並穀熟入睢陽,酂縣入永州,降永州為永城縣,與寧陵、下邑隸本府。八年,以宿、亳、徐、邳並隸焉。壤地平坦,數有河患。府為散郡,設知府、治中、府判各一員,直隸行省。抄籍戶數闕,至順錢糧戶數二萬三千三百一十七。領縣四、州四。州領八縣。



 縣四



 睢陽,下。倚郭。唐曰宋城,亦曰睢陽。金曰睢陽。宋曰宋城。元仍曰睢陽。



 永城,下。下邑,下。寧陵。下。



 州四



 徐州,下。唐初為徐州,又改彭城郡,又升武寧軍。宋因之。金屬山東西路。金亡,宋復之。元初歸附後,凡州縣視民多少設官吏。至元二年,例降為下州。舊領彭城、蕭、永固三縣及錄事司,至是永固並入蕭縣,彭城並錄事司並入州。領一縣:



 蕭縣。下。至元二年,並入徐州,十二年復立。



 宿州,中。唐置,宋升保靜軍,金置防禦使。金亡,宋復之。元初隸歸德府,領臨渙、蘄、靈壁、符離四縣並司候司。至元二年,以四縣一司並入州。四年,以靈壁入泗州,十七年復來屬。領一縣:



 靈壁。下。



 邳州,下。唐初為邳州,後廢屬泗州,又屬徐州。宋置淮陽軍。金復為邳州。金亡,宋暫有之。元初以民少,並三縣入州。至元八年,以州屬歸德府。十二年,復置睢寧、宿遷兩縣,屬淮安。十五年,還來屬。領三縣:



 下邳,下。州治所。宿遷,下。睢寧。下。



 亳州,下。唐初為亳州,後改譙郡,又仍為亳州。宋升集慶軍。金復為亳州。金亡,宋復之。元初領縣六:譙、酂、鹿邑、城父、衛真、穀熟。後以民戶少,並城父入譙,衛真入鹿邑,穀熟入睢陽,酂入永城,其睢陽、永城去隸歸德。後復置城父。領三縣:



 譙縣,下。鹿邑。下。此邑數有水患,歷代民不寧居。城父。下。



 襄陽路,唐初為襄州,後改襄陽郡。宋為襄陽府。元至元十年,兵破樊城,襄陽守臣呂文煥降,罷宋京西安撫司,立河南等路行中書省,更襄陽府為散府,未幾罷省。十一年,改襄陽府為總管府,又立荊湖等路行樞密院。十二年,立荊湖行中書省,後復罷。本府領四縣、一司,十九年割均、房二州,光化、棗陽二縣來屬。抄籍戶口數闕,至順錢糧戶數五千九十。領司一、縣六、州二。州領四縣。



 錄事司。



 縣六



 襄陽,下。倚郭。南漳,下。宜城,下。穀城,下。光化,至元十三年南伐,明年設官置縣,屬南陽,十九年來屬。棗陽。至元十四年,屬南陽,十九年來屬。



 州二



 均州,下。唐初為均州,又為武當郡。宋為武當軍。元至元十二年,江陵歸附,割隸湖北道宣慰司。十九年,還屬襄陽。領二縣:



 武當,下。兵亂遷治無常,至元十四年復置。鄖縣。下。兵後僑治無常,至元十四年復置。



 房州,下。唐初為遷州,後為房州,又改房陵郡。宋置保康軍。德祐中,知州黃思賢納土,命千戶鎮守,仍令思賢領州事。至元十九年,隸襄陽路。領二縣:



 房陵,下。竹山。下。



 蘄州路,下。唐初為蘄州,後改蘄春郡,又仍為蘄州。宋為防禦州。至元十二年,立淮西宣撫司。十四年,改總管府,設錄事司。戶三萬九千一百九十,口二十四萬九千三百二十一。自此以後至德安府,皆用至元二十七年數。領司一、縣五。



 錄事司。



 縣五



 蘄春,中。倚郭。蘄水,中。廣濟,中。宋嘉熙兵亂,徙治大江中洲,歸附後復舊治。黃梅,中。嘉熙兵亂,僑治中洲,後復舊。羅田。下。兵亂縣廢,歸附後始立。



 黃州路,下。唐初為黃州,後改齊安郡,又仍為黃州。宋為團練軍州。元至元十二年歸附。十四年,立總管府。十八年,又為黃蘄州宣慰司治所。二十三年,罷宣慰司,直隸行省。戶一萬四千八百七十八,口三萬六千八百七十九。領司一、縣三。



 錄事司。



 縣三



 黃岡,中。州治所。黃陂,下。兵亂僑治鄂州青山磯,歸附還舊治。麻城。下。兵亂徙治什子山,歸附還舊治。



 淮西江北道肅政廉訪司



 廬州路,上。唐改廬江郡,又仍為廬州。宋為淮南西路。元至元十三年,設淮西總管府。明年,於本路立總管府,隸淮西道。二十八年,以六安軍為縣來屬,後升六安縣為州。戶三萬一千七百四十六,口二十二萬九千四百五十七。領司一、縣三、州三。州領八縣。



 錄事司。



 縣三



 合肥,上。倚郭。梁縣,中。舒城。中。



 州三



 和州,中。唐改歷陽郡,後仍為和州。宋隸淮南西路。元至元十三年,置鎮守萬戶府。明年,改立安撫司。又明年,升和州路。二十八年,降為州,隸廬州路。舊設錄事司,後入州自治。領三縣:



 歷陽,上。倚郭。含山,中。烏江。中。



 無為州,中。唐初隸光州。宋始以城口鎮置無為軍,思與天下安於無事,取「無為而治」之意以名之。元至元十四年,升為路。二十八年,降為州,罷鎮巢州為縣以屬焉。領三縣:



 無為,上。倚郭。廬江,中。巢縣。下。



 六安州,下。唐以霍山縣置霍州,後州廢仍為縣。梁改灊山縣。宋改六安軍。元至元十二年歸附,二十八年降為縣,隸廬州路,後升為州。領二縣:



 六安,中。英山。中。



 安豐路,下。唐初為壽州,後改壽春郡。宋為壽春府,又以安豐縣為安豐軍,繼遷安豐軍於壽春府。元至元十四年,改安豐路總管府。十五年,定為散府,領壽春、安豐、霍丘三縣。二十八年,復升為路,以臨濠府為濠州,與下蔡、蒙城俱來屬。戶一萬七千九百九十二,口九萬七千六百一十一。領司一、縣五、州一。州領三縣。



 錄事司。



 縣五



 壽春,中。倚郭。安豐,下。至元二十一年,江淮行省言:「安豐之芍陂可溉田萬頃,若立屯開耕,實為便益。」從之。於安豐縣立萬戶府,屯戶一萬四千八百有奇。霍丘,下。下蔡,下。至元十三年,隸壽春府。二十八年罷府,與蒙城皆來屬。蒙城。下。



 州一



 濠州,下。唐初為濠州,後改鐘離郡,又仍為濠州。阻淮帶山,與壽陽俱為淮南之險郡,名初從豪,後加水為濠。南唐置定遠軍。宋為團練州,初隸淮南路,後隸淮南西路。元至元十三年歸附,設濠州安撫司。十五年,定為臨濠府。二十八年,復為濠州,革懷遠為下縣來屬。領三縣:



 鐘離,下。倚郭。定遠,下。懷遠。下。宋為懷遠軍,領荊山一縣。至元二十八年,以軍為縣,隸濠州,省荊山入焉。



 安慶路,下。唐初為東安州,又改舒州,又改同安郡,又復為舒州。宋為安慶府。元至元十三年,立安撫司。十四年,改安慶路總管府,屬蘄黃宣慰司。二十三年,罷宣慰司,直隸行省。戶三萬五千一百六,口二十一萬九千四百九十。領司一、縣六。



 錄事司。



 縣六



 懷寧,中。宿松,中。望江,下。太湖,中。桐城,中。灊山。至治三年初立。



 淮東道宣慰使司



 江北淮東道肅政廉訪司



 揚州路,上。唐初改南兗州,又改邗州,又改廣陵郡,又復為揚州。宋為淮南東路。元至元十三年,初建大都督府,置江淮等處行中書省。十四年,改為揚州路總管府。十五年,置淮東道宣慰司,本路屬焉。十九年,省宣慰司,以本路總管府直隸行省。二十一年,行省移杭州,復立淮東道宣慰司,止統本路屬淮安二郡,而本路領高郵府及真、滁、通、泰、崇明五州。二十二年,行省復遷,宣慰司遂廢,所屬如故。後改立河南江北等處行中書省,移治汴梁路,復立淮東道宣慰司,割出高郵府為散府,直隸宣慰司。戶二十四萬九千四百六十六,口一百四十七萬一千一百九十四。領司一、縣二、州五。州領九縣。



 錄事司。



 縣二



 江都,上。倚郭。泰興。上。



 州五



 真州,中。五代以前地屬揚州,宋以迎鑾鎮置建安軍,又升為真州。元至元十三年,初立真州安撫司。十四年,改真州路總管府。二十一年,復為州,隸揚州路。領二縣:



 揚子,上。倚郭。至元二十年,省錄事司入焉。六合。下。



 滁州,下。唐初析揚州地置,又改永陽郡,又復為滁州。元至元十五年,改滁州路總管府。二十年,仍為州,隸揚州路。領三縣:



 清流,中。至元十四年,省錄事司入焉。來安,下。全椒。中。



 泰州,上。唐更海陵縣曰吳陵,置吳州,尋廢。南唐升泰州。元至元十四年,立泰州路總管府。二十一年,改為州,隸揚州路。領二縣:



 海陵,上。倚郭。如皋。上。



 通州,中。唐屬揚州。南唐於海陵東境置靜海鎮。周平淮南,改為通州。宋改靜海郡。元至元十五年,改通州路總管府。二十一年,復為州,隸揚州路。領二縣:



 靜海,上。倚郭。海門。中。



 崇明州,下。本通州海濱之沙洲,宋建炎間有升州句容縣姚、劉姓者,因避兵於沙上,其後稍有人居焉,遂稱姚劉沙。嘉定間置鹽場,屬淮東制司。元至元十四年,升為崇明州。



 淮安路,上。唐楚州,又改臨淮郡,又仍為楚州。宋為淮安州。元至元十三年,行淮東安撫司。十四年,改立總管府,領山陽、鹽城、淮安、淮陰、新城、清河、桃園七縣,設錄事司。二十年,升為淮安府路,並淮安、新城、淮陰三縣入山陽,兼領臨淮府、海寧、泗、安東四郡,其盱眙、天長、臨淮、虹、五河、贛榆、朐山、沐陽各歸所隸。二十七年,革臨淮府,以盱眙、天長隸泗州。戶九萬一千二十二,口五十四萬七千三百七十七。領司一、縣四、州三。州領八縣。至元二十三年,於本路之白水塘、黃家畽等處立洪澤屯田萬戶村。



 錄事司。



 縣四



 山陽,上。至元十二年,安東州歸附,以本縣馬羅軍寨作山陽縣。十三年,淮安路歸附,仍存淮安縣。二十年,省淮安、新城入焉。鹽城,上。桃園,下。清河。下。本泗州之清河口,宋立清河軍,至元十五年為縣。



 州三



 海寧州,下。唐海州。宋隸淮南東路。元至元十五年,升為海州路總管府,復改為海寧府,未幾降為州,隸淮安路。初設錄事司,二十年,與東海縣並入朐山。領三縣:



 朐山,中。沐陽,下。贛榆。下。



 泗州,下。唐改臨淮郡,後復為泗州。宋隸淮南東路。元至元十三年,降為下州。舊領臨淮、淮平、虹、靈壁、睢寧五縣。十六年,割睢寧屬邳州。十七年,割靈壁入宿州,以五河縣來屬。二十一年,並淮平入臨淮。二十七年,廢臨淮府,以盱眙、天長二縣隸焉。領五縣:



 臨淮,下。虹縣,下。五河,下。元隸臨淮府,十七年來屬。盱眙,上。宋招信軍。至元十三年,行招信軍安撫司事,領盱眙、天長、招信、五河四縣。明年,升招信路總管府。十五年,改為臨淮府。十七年,以五河縣在淮之北,改屬泗州。二十年,並招信入盱眙。二十七年,廢臨淮府為盱眙縣。天長。中。



 安東州。下。



 高郵府,唐為縣。宋升為軍。元至元十四年,升為高郵路總管府,領錄事司及高郵、興化二縣。二十年,廢安宜府為寶應縣來屬,又並錄事司,改高郵路為府,屬揚州路。今隸宣慰司。抄籍戶口數闕,至順錢糧戶數五萬九十有八。領縣三:



 高郵,上。興化,中。寶應。上。舊為寶應軍,至元十六年為安宜府。二十年,廢府為縣,來屬本府。



 荊湖北道宣慰司



 山南江北道肅政廉訪司



 中興路,上。唐荊州,復為江陵府。宋為荊南府。元至元十三年,改上路總管府,設錄事司。天歷二年,以文宗潛籓,改為中興路。戶一十七萬六百八十二,口五十九萬九千二百二十四。領司一、縣七。



 錄事司。



 縣七



 江陵,上。公安,中。石首,中。松滋,中。枝江,下。潛江,中。監利。中。宋末兵亂民散,收附後始復舊。



 峽州路,下。唐改夷陵郡,又為峽州。宋隸荊湖北路,後徙治江南。元至元十三年歸附,十七年升為峽州路。戶三萬七千二百九十一,口九萬三千九百四十七。領縣四:



 夷陵,中。宋末隨州遷治不常,歸附後,復歸江北舊治。宜都,下。長陽,下。遠安。下。



 安陸府,唐郢州,又改富水郡,又為郢州。宋隸京西南路。元至元十三年歸附,十五年升為安陸府。戶一萬四千六百六十五,口三萬三千五百五十四。領縣二:



 長壽,中。京山。中。兵亂移治漢濱,至元十三年還舊治。



 沔陽府,唐復州,又改竟陵郡,又為復州。宋端平間,移州治於沔陽鎮。至元十三年歸附,改為復州路,十五年升為沔陽府。戶一萬七千七百六十六,口三萬九百五十五。領縣二:



 玉沙,中。倚郭。景陵。中。兵亂徙治無常,歸附後還舊治。



 荊門州,下。唐為縣。宋升為軍,端平間移治當陽縣。元至元十三年歸附,十四年升為府,十五年遷府治於古城,降為州。戶二萬九千四百七十一,口一十六萬五千四百三十五。領縣二:



 長林,上。當陽。中。



 德安府,唐安州,又改安陸郡,又仍為安州。宋為德安府,咸淳間徙治漢陽。元至元十三年還舊治,隸湖北道宣慰司。十八年罷宣慰司,直隸鄂州行省,為散府,後割以來屬。戶一萬九百二十三,口三萬六千二百一十八。領縣四、州一。州領二縣。



 縣四



 安陸。下。孝感,下。應城,中。雲夢。下。



 州一



 隨州,下。唐初為隨州,又改漢東郡,又復為隨州。宋為崇信軍,又為棗陽軍,後因兵亂遷徙無常。元至元十二年歸附。十三年,即黃仙洞為州治。戶一萬五千九百六十六,口五萬二千六十四。領二縣:



 隨縣,下。應山。下。



\end{pinyinscope}