\article{本紀第十二 世祖九}

\begin{pinyinscope}

 十九年春正月壬戌朔,高麗國王王睶遣其大將軍金子廷來賀。丙寅,罷征東行中書省。丁卯,諸王札剌忽至自軍中。時皇子北平王以軍鎮阿里麻里之地,以御海都。諸王昔里吉與脫脫木兒、棨木忽兒、撒里蠻等謀劫皇子北平王以叛,欲與札剌忽結援於海都,海都不從。撒里蠻悔過,執昔里吉等,北平王遣札剌忽以聞。妖民張圓光伏誅。立太僕院。撥信州民四百八戶隸諸王柏木兒。丙子,車駕畋於近郊。丁丑,高麗國王貢綢布四百匹。丙戌,賜西平王怯薛那懷等鈔一萬一千五百二十一錠。



 二月辛卯朔,車駕幸柳林。饒州總管姚文龍言,江南財賦歲可辦鈔五十萬錠,詔以文龍為江西道宣慰使,兼措置茶法。命司徒阿你哥、行工部尚書納懷制飾銅輪儀表刻漏。敕改給駙馬昌吉印。修宮城、太廟、司天臺。癸巳,調軍一萬五千、馬五千匹,徵也可不薛。遣使代祀岳瀆后土。甲午,甘州逃軍二千二百人自陳願挈家四千九百四十口還戍,敕以鈔一萬六百二十錠、布四千九百四十匹、驢四千九百四十頭給之。議征緬國,以大卜為右丞,也罕的斤為參政,領兵以行。戊戌,給別十八里元帥綦公直軍需。遣使往乾山,造江南戰船千艘。庚子,賜諸王塔剌海籍沒五十戶,願受十二戶。孛羅歡理算未徵糧二十七萬石,詔征之。壬寅,升軍器監秩三品。命軍官陣亡者,其子襲職,以疾卒者,授官降一等,具為令。授溪洞招討使郭昂等九人虎符,仍賞張溫、顏義顯銀各千兩。收晃兀兒塔海民匠九百五十三戶入官。乙巳,立廣東按察司。戊申,車駕還宮。己酉,減省部官冗員。改上都宣課提領為宣課提舉司。立鐵冶總管府,罷提舉司。減大都稅課官十四員為十員。改羅羅斯宣慰司隸雲南省,徙浙東宣慰司於溫州。分軍戍守江南,自歸州以及江陰至三海口,凡二十八所。庚戌,以參知政事唐兀帶等六人,鎮守黃州、建康、江陵、池州、興國。壬子,詔僉亦奚不薛及播、思、敘三州軍征緬國。癸丑,大良平元帥蒲元圭遣其男世能入覲。甲寅,車駕幸上都。申嚴漢人軍器之禁。丁巳,安州張拗驢以詐敕及偽為丞相孛羅署印,伏誅。戊午,賜雲南使臣及陜西簽省八八以下銀鈔、衣服有差。籍福建戶數。



 三月辛酉朔,烏蒙民叛,敕那懷、火魯思迷率蒙古、漢人新附軍討之。賞忽都答兒等戰功牛羊馬。益都千戶王著,以阿合馬蠹國害民,與高和尚合謀殺之。壬午,誅王著、張易、高和尚於市,皆醢之,餘黨悉伏誅。甲申,的斤帖林以己貲充屯田之費,諸王阿只吉以聞,敕酬其直。丙戌,禁益都、東平、沿淮諸郡軍民官捕獵。戊子,立塔兒八合你驛,以烏蒙阿謀歲輸騬馬給之。以領北庭都護阿必失哈為御史大夫,行御史臺事。



 夏四月辛卯,敕和禮霍孫集中書省部、御史臺、樞密院、翰林院等官,議阿合馬所管財賦,先行封籍府庫。丁酉,以和禮霍孫為中書右丞相,降右丞相甕吉剌帶為留守,仍同僉樞密院事。戊戌,征蠻元帥完者都等平陳吊眼巢穴班師,賞其軍鈔,仍令還家休息。遣揚州射士戍泉州。陳吊眼父文桂及兄弟桂龍、滿安納款,命護送赴京師。其黨吳滿、張飛迎敵,就誅之。敕以大都巡軍隸留守司。壬寅,立回易庫。中書左丞耿仁等言:「諸王公主分地所設達魯花赤,例不遷調,百姓苦之。依常調,任滿,從本位下選代為宜。」從之。以留守司兼行工部。敕自今歲用官車,勿賦於民,可即灤河造之,給其糧費。甲辰,以甘州、中興屯田兵逃還太原,誅其拒命者四人,而賞不逃者。乙巳,以阿合馬家奴忽都答兒等久總兵權,令博敦等代之,仍隸大都留守司。弛西山薪炭禁。以阿合馬之子江淮行中書省平章政事忽辛罪重於父,議究勘之。考核諸處平準庫,汰倉庫官。御史臺臣言:「見在贓罰鈔三萬錠,金銀、珠玉、幣帛稱是。」詔留以給貧乏者。丙午,收諸王別帖木兒總軍銀印。敕也裏可溫依僧例給糧。戊申,寧國路太平縣饑,民採竹實為糧,活者三百餘戶。敕出使人還,不即以所給符上,與上而有司不即收者,皆罪之。凡文書並奏可始用御寶。己酉,刊行蒙古畏吾兒字所書《通鑒》。以和禮霍孫為右丞相詔天下。庚戌,行御史臺言:「阿里海牙占降民為奴,而以為征討所得。」有旨降民還之有司,征討所得,籍其數量,賜臣下有功者。以興兵問罪海外,天下供給繁重,詔慰諭軍民,應有逋欠錢糧及官吏侵盜並權停罷。設懷孟路管河渠使、副各一員。拘括江南官豪隱匿逃軍。壬子,罷江南諸司自給驛券。丙辰,敕以妻女姊妹獻阿合馬得仕者,黜之。核阿合馬占據民田,給還其主;庇富強戶輸賦其家者,仍輸之官。北京宣慰使阿老瓦丁濫舉非才為管民官,命選官代之。議設鹽使司賣鹽引法,擇利民者行之,仍令按察司磨刷運司文卷。定民間貸錢取息之法,以三分為率。定內外官以三年為考,滿任者遷敘,未滿者不許超遷。禁吐蕃僧給驛太繁,擾害於民,自今非奉旨勿給。給控鶴人鈔一萬五錠,及其官吏有差。



 五月己未朔,鉤考萬億庫及南京宣慰司。沙汰省部官,阿合馬黨人七百十四人,已革者三十三人,餘五百八十一人並黜之。瀘州管軍總管李從,坐受軍士賄縱其私還,致萬戶爪難等為賊所殺,伏誅。籍阿合馬馬駝牛羊驢等三千七百五十八,追治阿合馬罪,剖棺戮其尸於通玄門外。罷南京宣慰司及江南財賦總管府。丁卯,降各省給驛璽書。戊辰,並江西、福建行省,去江南冗濫官,免福建山縣鎮店宣課,禁當路私人權府、州、司、縣官。招諭畬洞人,免其罪。禁差戍軍防送。禁人匠提舉擅招匠戶。己巳,遣浙西道宣慰司同知劉宣等理算各鹽運司及財賦府茶場都轉運司出納之數。籍阿合馬妻子親屬所營資產,其奴婢縱之為民。罷宣慰使所帶相銜。壬申,鎖系耿仁至大都,命中書省鞫之。庚辰,議於平灤州造船,發軍民合九千人,令探馬赤伯要帶領之,伐木於山,及取於寺觀墳墓,官酬其直,仍命桑哥遣人督之。癸未,給大都拔都兒正軍夏衣。和禮霍孫言:「省部濫官七百十四員,其無過者五百八十一員姑存之。」沿海左副都元帥石國英請以稅戶贍軍,軍逃死者,令其補足;站戶苗稅,貧富不均者,宜均其役。又請行鹽法,汰官吏,罷捕戶。詔中書集議行之。張惠、阿里罷。以甘肅行省左丞麥術丁為中書右丞,行御史臺御史中丞張雄飛參知政事。乙酉,元帥綦公直言:「乞黥逃軍,仍使從軍,及設立冶場於別十八里,鼓鑄農器。」從之。丙戌,別十八里城東三百餘里蝗害麥。



 六月己丑朔,日有食之。芝生眉州。甲午,阿合馬濫設官府二百四所,詔存者三十三,餘皆罷。又江南宣慰司十五道,內四道已立行中書省,罷之。乙未,發六盤山屯田軍七百七十人,以補劉恩之軍。敕宣慰司等官毋役官軍。丙申,發射士百人衛丞相,他人不得援例。戊戌,以占城既服復叛,發淮、浙、福建、湖廣軍五千、海船百艘、戰船二百五十,命唆都為將討之。亡宋軍有手號及無手號者,並聽為民。己亥,命何子志為管軍萬戶,使暹國。辛丑,籍阿合馬妻子婿奴婢財產。癸卯,禁濫保軍功。乙巳,招無籍軍給衣糧。己酉,賞太子府宿衛軍御盜之功,給鈔、馬有差,無妻者以沒官寡婦配之。以阿合馬居第賜和禮霍孫。壬子,申敕中外百官立限決事。癸丑,從和禮霍孫言,罷司徒府及農政院。鎖系忽辛赴揚州鞫治。丁巳,徵亦奚不薛,盡平其地,立三路達魯花赤,留軍鎮守,命藥剌海總之,以也速帶兒為都元帥宣慰使。



 秋七月戊午朔,日有食之。立行樞密院於揚州、鄂州。庚申,命行御史臺揀汰各道按察司官。辛酉,剖郝禎棺,戮其尸。壬戌,命以官錢給戍軍費,而以各奧魯所徵還官。禁諸位下營運錢貨差軍護送。高麗國王請自造船一百五十艘,助征日本。戊辰,徵鴨池回軍屯田安西,以鈔給之。庚午,令蒙古軍守江南者更番還家。壬申,發察罕腦兒軍千人治晉山道。立馬湖路總管府。癸酉,賜高麗王王睶金印。癸酉,宣慰孟慶元、萬戶孫勝夫使瓜哇回,為忙古帶所囚,詔釋之。丁丑,罷汪札剌兒帶總帥,收其制命、虎符。以鞏昌路達魯花赤別速帖木兒為鞏昌平涼等處二十四處軍前便宜都總帥府達魯花赤。以蒙古人孛羅領湖北辰、沅等州淘金事。戊寅,議築阿失答不速皇城,樞密院言:「用木十二萬,地遠難致,依察罕腦兒築土為墻便。」從之。乙酉,賜諸王塔海帖木兒、忽都帖木兒等金銀、幣帛有差。闍婆國貢金佛塔。發米賑乞里吉思貧民。



 八月丁亥朔,給乾山造船軍匠冬衣,及新附軍鈔。庚寅,忙古帶征羅氏鬼國還,仍佩虎符,為管軍萬戶。辛卯,以阿八赤督運糧。癸巳,發羅羅斯等軍助征緬國。辛亥,並淄萊路田、索二鎮,仍於驛臺立新城縣治。大駕駐蹕龍虎臺。江南水,民饑者眾;真定以南旱,民多流移;和禮霍孫請所在官司發廩以賑,從之。申嚴以金飾車馬服御之禁。又禁諸監官不得令人匠私造器物。甲寅,聖誕節,是日還宮。乙卯,御正殿,受皇太子、諸王、百官朝賀。丙辰,謫捏兀迭納戍占城以贖罪。



 九月丁巳朔,賑真定饑民,其流移江南者,官給之糧,使還鄉里。敕中書省窮治阿合馬之黨。別速帶請於羅卜、闍里輝立驛,從之。以阿合馬沒官田產充屯田,籍阿里家。戊午,誅阿合馬第三子阿散,仍剝其皮以徇。庚申,汰冗官。游顯乞罷漣、海州屯田,以其事隸管民官,從其請,仍以顯平章政事,行省揚州。福建宣慰司獲倭國諜者,有旨留之。辛酉,誅耿仁、撒都魯丁及阿合馬第四子忻都。招討使楊庭堅招撫海外,南番皆遣使來貢。俱藍國主遣使奉表,進寶貨、黑猿一。那旺國主忙昂,以其國無識字者,遣使四人,不奉表。蘇木都速國主土漢八的亦遣使二人。蘇木達國相臣那裡八合剌攤赤,因事在俱藍國,聞詔,代其主打古兒遣使奉表,進指環、印花綺段及錦衾二十合。寓俱藍國也裏可溫主兀咱兒撇里馬亦遣使奉表,進七寶項牌一、藥物二瓶。又管領木速蠻馬合馬亦遣使奉表,同日赴闕。壬戌,禁諸人不得沮撓課程。敕:「官吏受賄及倉庫官侵盜,臺察官知而不糾者,驗其輕重罪之。中外官吏贓罪,輕者杖決,重者處死。言官緘默,與受贓者一體論罪。」仍詔諭天下。乙丑,簽亦奚不薛等處軍。丁卯,安南國進貢犀兕、金銀器、香藥等物。增給元帥綦公直軍冬衣鈔。己巳,命軍站戶出錢助民和顧和買。籍云南新附戶。自兀良合帶鎮雲南,凡八籍民戶,四籍民田,民以為病。至是,令已籍者勿動,新附者籍之。定雲南稅賦用金為則,以貝子折納,每金一錢直貝子二十索。罷雲南宣慰司。壬申,敕平灤、高麗、耽羅及揚州、隆興、泉州共造大小船三千艘。亦奚不薛之北,蠻洞向世雄兄弟及散毛諸洞叛,命四川行省就遣亦奚不薛軍前往招撫之,使與其主偕來。癸酉,阿合馬侄宰奴丁伏誅。罷忽辛黨馬璘江淮行省參知政事。丁亥,遣使括雲南所產金,以孛羅為打金洞達魯花赤。戊寅,給新附軍賈祐衣糧。祐言為日本國焦元帥婿,知江南造船,遣其來候動靜,軍馬壓境,願先降附。辛巳,敕各行省止用印一,餘者拘之,及拘諸位下印。發鈔三萬錠,於隆興、德興府、宣德州和糴糧九萬石。壬申,賜諸王阿只吉金五千兩、銀五萬兩。厘正選法,置黑簿以籍阿合馬黨人之名;令諸路歲貢儒、吏各一人,各道提刑按察司舉廉能者升等遷敘。



 冬十月丁亥朔,增兩浙鹽價。詔整治鈔法。己丑,敕河西僧、道、也裏可溫有妻室者,同民納稅。庚寅,以歲事不登,聽諸軍捕獵於汴梁之南。辛卯,以平章軍國重事、監修國史耶律鑄為中書左丞相。壬辰,享於太廟。罷西京宣慰司。丙申,初立詹事院,以完澤為右詹事,賽陽為左詹事。由大都至中灤,中灤至瓜州,設南北兩漕運司。立蘆臺越支三義沽鹽使司,河間滄清、山東濱、樂安及膠萊、莒密鹽使司五。敕籍沒財物精好者及金銀幣帛入內帑,餘付刑部,以待給賜。禁中出納分三庫:御用寶玉、遠方珍異隸內藏,金銀、只孫衣段隸右藏,常課衣段、綺羅、縑布隸左藏。設官吏掌鑰者三十二人,仍以宦者二十二人董其事。減大府監官。癸卯,命崔或等鉤考樞密院文卷。甲辰,占城國納款使回,賜以衣服。乙巳,遣阿耽招降法裏郎、阿魯、乾伯等國。罷屯田總管府,以其事隸樞密院,令管軍萬戶兼之。丙午,以汪惟孝為總帥。丁未,女直六十自請造船運糧赴鬼國贍軍,從之。議徵義巴洞。庚戌,以四川民僅十二萬戶,所設官府二百五十餘,令四川行省議減之。移成都宣慰司於碉門,罷利州及順慶府宣慰司。禁大都及山北州郡酒。詔兩廣、福建五品以下官,從行省就便銓注。耶律鑄言:「有司官吏以採室女,乘時害民,如令大郡歲取三人,小郡二人,擇其可者,厚賜其父母,否則遣還為宜。」從之。籍京畿隱漏田,履畝收稅。命游顯專領江浙行省漕運。乙卯,命堅童專掌奏記。誅阿合馬長子忽辛、第二子抹速忽於揚州,皆醢之。



 十一月戊午,上都建利用庫。賜太常禮樂、籍田等二百六十戶鈔千二百錠。甲子,給欠州屯田軍衣服。丁卯,給和林戍還軍校銀鈔、幣帛。江南襲封衍聖公孔洙入覲,以為國子祭酒,兼提舉浙東道學校事,就給俸祿與護持林廟璽書。詔以阿合馬罪惡頒告中外,凡民間利病即與興除之。壬申,以勢家為商賈者阻遏官民船,立沿河巡禁軍,犯者沒其家。癸酉,分元帥綦公直軍戍曲先。甲戌,中書省臣言:「天下重囚,除謀反大逆,殺祖父母、父母,妻殺夫,奴殺主,因奸殺夫,並正典刑外,餘犯死罪者,令充日本、占城、緬國軍。」從之。改鑄省印。丙子,四川行省招諭大盤洞主向臭友等來朝。戊寅,耶律鑄言:「前奉詔殺人者死,仍徵燒埋銀伍十兩,後止征鈔二錠,其事太輕。臣等議,依蒙古人例,犯者沒一女入仇家,無女者徵鈔四錠。」從之。以袁州、饒州、興國軍復隸隆興省。馬八兒國遣使以金葉書及土物來貢。罷都功德使脫烈,其修設佛事妄費官物,皆徵還之。賜貧乏者合納塔兒、八只等羊馬鈔。



 十二月丁亥,命阿剌海領範文虎等所有海船三百艘。壬寅,中書左丞張文謙為樞密副使。乙未,中書省臣言:「平原郡公趙與芮、瀛國公趙鳷、翰林直學士趙與票,宜並居上都。」帝曰:「與芮老矣,當留大都,餘如所言。」繼有旨,給瀛國公衣糧發遣之,唯與票勿行。以中山薛保住上匿名書告變,殺宋丞相文天祥。癸卯,御史中丞崔彧言:「臺臣於國家政事得失、生民休戚、百官邪正,雖王公將相亦宜糾察。近唯御史有言,臣以為臺官皆當建言,庶於國家有補。選用臺察官,若由中書,必有偏徇之弊。御史宜從本臺選擇,初用漢人十六員,今用蒙古人十六員,相參巡歷為宜。」從之。浚濟川河。降拱衛司復正四品,仍收其虎符。罷湖廣行省金銀鐵冶提舉司,以其事隸各路總管府。以建康淘金總管府隸建康路。中書右丞札散為平章政事。罷解鹽司及諸鹽司,令運司官親行調度鹽引。罷南京屯田總管府,以其事隸南陽府。阿里海牙復鎮遠軍,發軍千人戍守,以其地與西川行省接,就以隸焉。詔立帝師答耳麻八剌剌吉塔,掌玉印,統領諸國釋教。造帝師八合思八舍利塔。免鞏昌等處積年所欠田租稅課。賜皇子北安王位下塔察兒等馬牛羊各有差。



 二十年春正月丙辰朔,高麗國王王睶遣其大將軍俞洪慎來賀。己未,納皇後弘吉剌氏。辛酉,賜諸王出伯印,賞諸王必赤帖木兒、駙馬昌吉軍鈔。敕諸王、公主、駙馬得江南分地者,於一萬戶田租中輸鈔百錠,準中原五戶絲數。癸亥,敕藥剌海領軍征緬國。乙丑,高麗國王王睶遣使兀剌帶貢氎布線綢等物四百段。和禮霍孫言:「去冬中山府奸民薛寶住為匿名書來上,妄效東方朔書,欺罔朝廷,希覬官賞。」敕誅之。又言:「自今應訴事者,必須實書其事,赴省、臺陳告。其敢以匿名書告事,重者處死,輕者流遠方;能發其事者,給犯人妻子,仍以鈔賞之。又阿合馬專政時,衙門太冗,虛費俸祿,宜依劉秉忠、許衡所定,並省為便。」皆從之。設務農司。敕諸事赴省、告訴之,理決不平者,許詣登聞鼓院擊鼓以聞。預備征日本軍糧,令高麗國備二十萬石。以阿塔海依舊為征東行中書省丞相。丙寅,發五衛軍二萬人征日本。發鈔三千錠糴糧於察罕腦兒,以給軍匠。以燕南、河北、山東諸郡去歲旱,稅糧之在民者,權停勿征,仍諭:「自今管民官,凡有災傷,過時不申,及按察司不即行視者,皆罪之。」刑部尚書崔彧言時政十八事,詔中書省與御史大夫玉速帖木兒議行之。罷上都回易庫。丁卯,伯要帶等伐船材於烈堝都山、乾山,凡十四萬二千有奇,起諸軍貼戶年及丁者五千人、民夫三千人運之。己巳,太陰犯軒轅御女。賜諸王也裡乾、塔納合、奴木赤金各五十兩、金衣襖一。庚午,以平灤造船去運木所遠,民疲於役,徙於陽河造之。壬申,御史臺言:「燕南、山東、河北去年旱災,按察司已嘗閱視,而中書不為奏免,民何以堪?請權停稅糧。」制曰:「可。」移鞏昌按察司治甘州。命右丞闍裏帖木兒及萬戶三十五人、蒙古軍習舟師者二千人、探馬赤萬人、習水戰者五百人征日本。丁丑,以招討楊廷璧為宣慰使,賜弓矢鞍勒,使諭俱藍等國。己卯,命諸軍習舟楫,給鈔八千錠於隆興、宣德等處和糴以贍之。庚辰,太陰入南斗。壬午,車駕畋於近郊。以四川歸附官楊文安為荊南道宣慰使。改廣東提刑按察司為海北廣東道,廣西按察司為廣西海北道,福建按察司為福建閩海道,鞏昌按察司為河西隴北道。癸未,撥忽蘭及塔剌不罕等四千戶隸皇太子位下。壬戌,敕於禿烈禿等富戶內貸牛六百頭,給乞里古思之貧乏者。



 二月戊子,定兩廣、四川戍軍二三年一更,廩其家屬,軍官給俸以贍之。賜俱藍國王瓦你金符。賜駙馬阿禿江南民千戶。以春秋仲月上戊日祭社稷及武成王。庚寅,太陰掩昴。癸巳,敕斡脫錢仍其舊。丁酉,給別十八里屯田軍戰襖。庚子,敕權貴所占田土,量給各戶之外,餘者悉以與怯薛帶等耕之。減四川官府,並西川東、西、北三道宣慰司,及潼川等路鎮守萬戶府、新軍總管府,威、灌、茂等州安撫司十四處。是夜太白犯昴。辛丑,定軍官選格,立官吏贓罪法。壬寅,太白犯昴。乙巳,令隆興行省遣軍護送占城糧船。太陰犯心。丁未,定安洞酋長遣其兄弟入覲,敕給驛馬。己酉,升闌遺監秩正五品。癸丑,諭中書省:「大事奏聞,小事便宜行之,毋致稽緩。」甲寅,降太醫院為尚醫監,改給銅印,立江南等處官醫提舉司。賜日本軍官八忽帶及軍士銀鈔有差。敕遣官錄揚州囚徒。



 三月丁巳,諸王勝納合兒設王府官三員。以萬戶不都蠻鎮守金齒。罷女直造日本出征船,罷河西行御史臺,立鞏昌等處行工部。罷福建市舶總管府,存提舉司,並泉州行省入福建行省,免福建歸附後未征苗稅。以闊闊你敦治江淮行省,或言其過,命兀奴忽帶、伯顏佐之。戊午,以新附洞蠻酋長為千戶。己未,歲星犯鍵閉。罷京兆行省,立行工部。御史臺臣言:「平灤造船,五臺山造寺伐木,及南城建新寺,凡役四萬人,乞罷之。」詔:「伐木建寺即罷之,造船一事,其與省臣議。」前後衛軍自願征日本者,命選留五衛漢軍千餘,其新附軍令悉行。庚申,太陰犯井。辛酉,賞諸王合班弟忙兀帶所部軍士戰功,銀鈔、幣帛、衣服各有差。給甘州戍軍鈔。壬戌,太陰犯鬼。乙丑,命兀奴忽魯帶往揚州錄囚,遣江北重囚謫征日本。立雲南按察司,照刷行省文卷。罷淮安等處淘金官,惟計戶取金。以阿合馬綿絹絲線給貧民工匠。給王傅兀訥忽帖只印,給西川、福建、兩廣之任官驛馬。以湖南宣慰使張鼎新、行省參知政事樊楫等嘗阿附阿里海牙,敕罷之。丙寅,車駕幸上都。江西行省參政完顏那懷,坐越例驟升及妄舉一百九十八人入官,罷之。罷河西辦課提舉司。丁卯,增置蒙古監察御史六員。乙巳,歲星犯房。癸酉,歲星掩房。廣州新會縣林桂方、趙良鈐等聚眾,偽號羅平國,稱延康年號,官軍擒之,伏誅,餘黨悉平。乙亥,罷諸處役夫。遣阿塔海戍曲先,漢都魯迷失帥甘州新附軍往斡端。己卯,給各衛軍出征馬價鈔。辛巳,立畏吾兒四處驛及交鈔庫。壬午,祀太一。罷福建道宣慰司,復立行中書省於漳州,以中書右丞張惠為平章政事,御史中丞也先帖木兒為中書左丞,並行中書省事。賜迷裏札蠻、合八失鈔。賑八魯怯薛、八剌合赤等貧乏。賜皇子北平王所部馬牛羊各有差。



 夏四月丙戌,立別十八里、和州等處宣慰司。庚寅,敕藥剌海戍守亦奚不薛。都元帥也速答兒還自亦奚不薛,駐軍成都,求入見,許之,仍遣人屯守險隘。以侍衛親軍二萬人助征日本。辛卯,樞密院臣言:「蒙古侍衛軍於新城等處屯田,砂礫不可種,乞改撥良田。」從之。壬辰,阿塔海求軍官習舟楫者同征日本,命元帥張林、招討張瑄、總管硃清等行。以高麗王就領行省,規畫日本事宜。甲午,減江南諸道醫學提舉司,四省各存其一。免京畿所括豪勢田舊稅三之二、新稅三之一。高麗國王王睶請以蒙古人同行省事。禁近侍為人求官,紊亂選法。申嚴酒禁,有私造者,財產、女子沒官,犯人配役。申私鹽之禁,許按察司糾察鹽司。己亥,太陰犯房。壬寅,太陰犯南斗。癸卯,授高麗國王王睶征東行中書省左丞相,仍駙馬、高麗國王。乙巳,命樞密院集軍官議征日本事宜,程鵬飛請明賞罰,有功者軍前給憑驗,候班師日改授,從之。庚戌,右丞也速帶兒招撫筠連州、定州、阿永、都掌等處蠻,獨山都掌蠻不降,進軍討之,生擒酋長得蘭紐,遂班師。發大都所造回回砲及其匠張林等,付征東行省。辛亥,以征日本,給後衛軍衣甲,及大名、衛輝新附軍鈔。麥術丁等檢核萬億庫,以罪監系者多,請付蒙古人治。有旨:「蒙古人為利所汩,亦異往日矣,其擇可任者使之。」



 五月乙卯,給甘州戍軍夏衣。戊午,丞相伯顏、諸王相吾答兒等言:「征緬國軍宜參用蒙古、新附軍。」從之。己未,免五衛軍征日本,發萬人赴上都。縱平灤造船軍歸耕,撥大都見管軍代役。庚申,減隆興府昌州蓋里泊管鹽官吏九十九人,以其事隸隆興府。定江南民官及轉運司官公田。甲子,徙揚州淘金夫赴益都。立征東行中書省,以高麗國王與阿塔海共事,給高麗國征日本軍衣甲。御史中丞崔彧言:「江南盜賊相繼而起,皆緣拘水手、造海船,民不聊生,日本之役,宜姑止之。江南四省應辦軍需,宜量民力,勿強以土產所無,凡給物價及民者必以實。召募水手,當從所欲。伺民之氣稍蘇,我之力粗備,三二年復東征未晚。」不從。丙寅,太陰掩心東星。免江南稅糧三之二。敕阿里海牙調漢軍七千、新附軍八千,以附唆都從征。辛未,占城行省已破占城,其國主補底遁去,降璽書招徠之。甲戌,發征日本重囚往占城、緬國等處從征。設高麗國勸農官四員。丙子,詔諭諸王相吾答兒:「先是雲南重囚令便宜處決,恐濫及無辜,自今凡大闢罪,仍須待報。」並省江淮、雲南州郡。以耶律老哥為中書參知政事。免戍軍差稅。禁諸王奧魯官科擾軍戶。以西南蠻夷有謀叛未附者,免西川征緬軍,令專守御,支錢令各驛供給。戊寅,諸陳言者從都省集議,可行者以聞,不可則明以諭言者。許按察司官用弓矢。監察御史阿剌渾坐擅免贓錢、不糾私釀等罪罷。用御史中丞崔彧言,罷各路選取室女。頒行宋文思院小口斛。敕以陜西按察司贓罰錢輸於秦王。省北京提刑按察司副使、僉事各一員。立海西遼東提刑按察司,按治女直、水達達部。己卯,酬諸王只必帖木兒給軍羊馬鈔十萬錠。海南四州宣慰使硃國寶請益兵討占城國主,詔以阿里海牙軍萬五千人應之。用王積翁言,詔江南運糧,於阿八赤新開神山河及海道兩道運之。立斡脫總管府。辛巳,給占城行省唆都弓矢甲仗。



 六月丙戌,申嚴私易金銀之禁。以甘州行省參政王椅為中書參知政事。免大都及平灤路今歲絲料。江南遷轉官不之任者杖之,追奪所受宣敕。戊子,以征日本,民間騷動,盜賊竊發,忽都帖木兒、忙古帶乞益兵禦寇,詔以興國、江州軍付之。己丑,增官吏俸給。庚寅,定市舶抽分例,舶貨精者取十之一,粗者十五之一。差五衛軍人修築行殿外垣。命諸王忽牙都設斷事官。丙申,發軍修完大都城。辛丑,發軍修築堤堰。戊申,用伯顏等言,所括宋手號軍八萬三千六百人,立牌甲設官以統之,仍給衣糧。庚戌,流叛賊陳吊眼叔陳桂龍於憨答孫之地。辛亥,四川行省參政曲立吉思等討平九溪十八洞,以其酋長赴闕,定其地,立州縣,聽順元路宣慰司節制。以向世雄等為義巴諸洞安撫大使及安撫使。



 秋七月癸丑朔,蠲建寧路至元十七年前未納苗稅。丙辰,免征骨嵬軍賦。諭阿塔海所造征日本船,宜少緩之;所拘商船,其悉給還。阿里沙坐虛言惑眾誅。太白犯井。丁巳,賜捏古帶等珠衣。庚申,調軍益戍雲南。丙寅,立亦奚不薛宣慰司,益兵戍守。開雲南驛路。分亦奚不薛地為三,設官撫治之。癸亥,太陰犯南斗。乙丑,太白犯井。丁卯,罷淮南淘金司,以其戶還民籍。庚午,熒惑犯司怪。新附官同文英入見,其贄禮銀萬兩、金四十錠,鐵木兒不花匿為己有,詔即其家搜閱,沒入官帑。敕捕阿合馬婦翁尚書蔡仲英,徵償所貸官鈔二十萬錠。阿八赤、姚演以開神山橋渠,侵用官鈔二千四百錠,折閱糧米七十三萬石,詔徵償,仍議其罪。壬申,亦奚不薛軍民千戶宋添富及順元路軍民總管兼宣撫使阿里等來降,班師,以羅鬼酋長阿利及其從者入覲。立亦奚不薛總管府,命阿里為總管。丙子,減江南十道宣慰司官一百四十員為九十三員。敕上都商稅六十分取一。免大都、平灤兩路今歲俸鈔。立總教院,秩正三品。丁丑,命按察司照刷土蕃宣慰司文卷。立鋪軍捕淮西盜賊。淮東宣慰同知宋廷秀私役軍四十人,杖而罷之。庚辰,給忽都帖木兒等軍貧乏。償怯兒合思等羊馬價鈔。



 八月癸未,以明理察平章軍國重事,商議公事。立懷來淘金所。甲午,敕大名、真定、北京、衛輝四路屯駐新附軍,於東京屯田。安南國遣使以方物入貢。丙午,太白犯軒轅。丁未,歲星犯鉤鈐。浙西道宣慰使史弼言:「頃以征日本船五百艘科諸民間,民病之,宜取阿八赤所有船,修理以付阿塔海,庶寬民力,並給鈔於沿海募水手。」從之。濟州新開河成,立都漕運司。庚戌,賞還役宿衛軍。賜皇子北安王所部軍鈔、羊馬。



 九月壬子,太白犯軒轅少女。戊午,合剌帶等招降象山縣海賊尤宗祖等九千五百九十二人,海道以寧。太陰犯鬥。壬戌,調黎兵同征日本。丙寅,古答奴國因商人阿剌畏等來言,自願效順。並占城、荊湖行省為一。徙舊城市肆局院,稅務皆入大都,減稅征四十分之一。賞硃雲龍漕運功,授七品總押,仍以幣帛給之。己巳,太白犯右執法。辛未,以歲登,開諸路酒禁。廣東盜起,遣兵萬人討之。壬申,太陰掩井。癸酉,熒惑犯鬼。甲戌,太陰犯鬼,熒惑犯積尸氣,太白犯左執法。戊寅,史弼陳弭盜之策,為首及同謀者死,餘屯田淮上,帝然其言。詔以其事付弼,賊黨耕種內地,其妻孥送京師以給鷹坊人等。



 冬十月庚寅,給徵日本新附軍鈔三萬錠。壬辰,車駕由古北口路至自上都。癸巳,斡端宣慰使劉恩進嘉禾,同穎九穗、七穗、六穗者各一。甲午,以平章政事札散為樞密副使。詔:「五衛軍,歲以冬十月聽十之五還家備資裝,正月番上代其半還,四月畢入役。」時各衛議先遣七人,而以三人自代,從之。乙未,享於太廟。丙申,太陰犯昴。丁酉,誅占城逃回軍。忙兀帶請增蒙古、漢軍戍邊,從之。以忽都忽總揚州行省唆都新益軍。庚子,許阿速帶軍以兄弟代役。建寧路管軍總管黃華叛,眾幾十萬,號頭陀軍,偽稱宋祥興五年,犯崇安、浦城等縣,圍建寧府。詔卜憐吉帶、史弼等將兵二萬二千人討平之。耶律鑄罷。壬寅,立東阿至御河水陸驛,以便遞運。徙濟州潭口驛於新河魯橋鎮。給甘州納硫黃貧乏戶鈔。癸卯,諸王只必帖木兒請括閱常德府分地民戶,不許。中書省臣言:「阿八赤新開河二處,皆有倉,宜造小船分海運。」從之。中書省臣言:「押亦迷失嘗請諭江南諸郡,募人種淮南田。今乃往各郡轉收民戶,行省官闊闊你敦言其非便,宜令其於治所召募,不可強民。」從之。戊申,給水達達鰥寡孤獨者絹千匹、鈔三百錠。立和林平準庫。遣官檢核益都淘金欺弊。罷中興管課提舉司及北京鹽鐵課程提舉司。己酉,簽河西質子軍年及丁者充軍。庚戌,各道提刑按察司增設判官二員。



 十一月壬子,賞大不花、脫歡等戰功銀幣。癸丑,總管陳義願自備海船三十艘以備征進,詔授義萬戶,佩虎符。義初名五虎,起自海盜,內附後,其兄為招討,義為總管。敕凡盜賊必由管民官鞫問,仍不許私和。丁巳,命各省印《授時歷》。諸王只必帖木兒請於分地二十四城自設管課官,不從。又請立拘榷課稅所,其長從都省所定,次則王府差設,從之。詔:「大都田土,並令輸稅;甘州新括田土,畝輸租三升。」己未,吏部尚書劉好禮以吉利吉思風俗事宜來上。壬戌,復立南京宣慰司。乙丑,罷開成路屯田總管府入開成路,隸京兆宣慰司。戊辰,立司農司,掌官田邸舍人民。給諸王所部撒合兒、兀魯等羊馬,以賙其乏。河西官府參用漢人。徙甘肅沙州民戶復業。大都城門設門尉。丁丑,禁雲南管課官於常額外多取餘錢。戊寅,禁雲南權勢多取債息,仍禁沒人口為奴及黥其面者。太白歲星相犯。己卯,從諸王術白、蒙古帶等請,賞也禿古等銀鈔,以旌戰功。賜皇太子鈔千錠。以御史臺贓罰鈔賜怯憐口。



 十二月庚辰,賜諸王渾都帖木兒衣物,忽都兒所部軍銀鈔幣帛。甲申,賜別速帶所部軍衣服幣帛七千、馬二千。賞西番軍官愛納八斯等戰功。辛卯,以茶忽所管軍六千人備征日本。壬辰,給諸王阿只吉牛價。以中書參議溫迪罕禿魯花廉貧,不阿附權勢,賜鈔百錠。罷女直出產金銀禁。甲午,給鈔四萬錠和糴於上都。給司閽衛士貧者,人鈔二十錠。辛丑,賜諸王昔烈門等銀。以海道運糧招討使硃清為中萬戶,賜虎符;張瑄子文虎為千戶,賜金符。徙新附官仕內郡。以蠡州還隸真定府路。癸卯,發粟賑水達達四十九站。甲辰,太陰掩熒惑。丙午,罷雲南造賣金箔規措所。罷雲南都元帥府及重設官吏。定質子令,凡大官子弟,遣赴京師。戊申,雲南施州子童興兵為亂,敕參知政事阿合八失帥兵,合羅羅斯脫兒世合討之,給布萬匹賑女直饑民一千戶。是歲,斷死罪二百七十八人。



\end{pinyinscope}