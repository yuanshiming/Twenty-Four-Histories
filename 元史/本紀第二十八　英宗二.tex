\article{本紀第二十八 英宗二}

\begin{pinyinscope}

 二年春正月己巳朔,安南、占城各遣使來貢方物。壬申,保定雄州饑,賑之。庚午,廣太廟。甲戌作有《諷刺詩》、《論自然》,現僅存一些殘篇。,禁漢人執兵器出獵及習武藝。丁丑,太陰犯昴。親祀太廟,始陳鹵簿,賜導駕耆老幣帛。戊寅,敕有司存恤孔氏子孫貧乏者。己卯,山東、保定、河南、汴梁、歸德、襄陽、汝寧等處饑,發米三十九萬五千石賑之。庚辰,太白犯建星。公主阿剌忒納八剌下嫁,賜鈔五十萬貫。辛巳,太白犯建星。敕:「臺憲用人,勿拘資格。」儀封縣河溢傷稼,賑之。癸未,流徽政院使羅源於耽羅。建行殿於柳林。封塔察兒為蘭國公。辛卯,太陰犯心。癸巳,以西僧羅藏為司徒。漷州饑,糶米十萬石賑之。甲午,熒惑犯房。丁酉,太白犯牛。



 二月己亥朔,熒惑犯建閉星。庚子,置左、右欽察衛親軍都指揮使司,命拜住總之。罷上都歇山殿及帝師寺役。辛丑,賜鐵失父祖碑。癸卯,以江南行臺御史大夫欽察為中書平章政事,江浙行省參政王居仁為中書參知政事,薛處敬罷為河南行省左丞。丙午,熒惑犯罰星。戊申,祭社稷。順德路九縣水旱,賑之。太陰犯井。庚戌,熒惑犯東咸。辛亥,太陰犯酒旗及軒轅。壬子,太白犯壘壁陣。賜諸王案忒不花鈔七萬五千貫。以徹兀臺禿忽魯死事,賜鈔三萬五千貫。諸王怯伯遣使進文豹。河間路饑,禁釀酒。癸丑,太陰犯明堂。甲寅,以太廟役軍造流杯池行殿。廣海郡邑官曠員,敕願往任者,升秩二等。乙卯,以遼陽行省平章政事買驢為中書平章政事。西僧亦思剌蠻展普疾,詔為釋大闢囚一人、笞罪二十人。戊午,賑真定等路饑。己未,太陰犯天江。括馬賜宗仁衛。壬戌,太白犯壘壁陣。諸王怯伯遣使進海東青鶻。癸亥,遼陽等路饑,免其租,仍賑糧一月。甲子,恩州水,民饑、疫,賑之。



 三月己巳,中書省臣言:「國學廢弛,請令中書平章政事廉恂、參議中書事張養浩、都事孛術魯翀董之。外郡學校,仍命御史臺、翰林院、國子監同議興舉。」從之。敕四宿衛、興聖宮及諸王部勿用南人。斡魯思告訐父母,斬之。辛未,禁捕天鵝,違者籍其家。壬申,復張珪司徒。臨安路河西諸縣饑,賑之。癸酉,河南兩淮諸郡饑,禁釀酒。丙子,延安路饑,賑糧一月。罷京師諸營繕役卒四萬餘人。河間、河南、陜西十二郡春旱秋霖,民饑,免其租之半。戊寅,修都城。庚辰,敕:「江浙僧寺田,除宋故有永業及世祖所賜者,餘悉稅之。」癸未,賑遼陽女直、漢軍等戶饑。乙酉,賑濮州水災。丙戌,以親祀禮成,賜與祭者幣。普減內外官吏一資。萬戶哈剌那海以私粟賑軍,賜銀、幣,仍酬其直。給行通政院印。賜潛邸四宿衛士鈔有差。復置市舶提舉司於泉州、慶元、廣東三路,禁子女、金銀、絲綿下番。丁亥,鳳翔道士王道明妖言伏誅。己丑,有暈貫日如連環。賜諸王斡魯溫孫銀印。命有司建木華黎祠於東平,仍樹碑。以國用匱竭,停諸王賞賚及皇後答里麻失裡等歲賜。庚寅,曹州、滑州饑,賑之。命將作院更制冕旒。辛卯,遣御史錄囚。置甘州八剌哈孫驛。監察御史何守謙坐贓杖免。壬辰,賑上都十一驛。給宗仁衛蒙古子女衣糧。賜諸王脫烈鐵木兒鈔五萬貫。甲午,遼陽哈里賓民饑,賑之。丁酉,幸柳林。駙馬許納之子速怯訴曰:「臣父謀叛,臣母私從人。」帝曰:「人子事親,有隱無犯,今有過不諫,乃復告訐。」命誅之。賑奉元路饑。



 夏四月戊戌朔,車駕幸上都。己亥,嶺北蒙古軍饑,給糧遣還所部。庚子,賑彰德路饑。壬寅,真州火,徽州饑,並賑之。辛亥,涇州雨雹,免被災者租。壬子,公主失憐答里薨,賜鈔五萬貫。甲寅,南陽府西穰等屯風、雹,洪澤、芍陂屯田去年旱、蝗,並免其租。丙辰,恩州饑,禁釀酒。乙丑,中書省臣請節賞賚以紓民力,帝曰:「朕思所出倍於所入,出納之際,卿輩宜慎之,朕當撙節其用。」丙寅,賜邊卒鈔、帛。賑東昌、霸州饑民。松江府上海縣水,仍旱。



 五月己巳,以公主速哥八剌為趙國大長公主。免德安府被災民租,修滹沱河堤。彰德府饑,禁釀酒。庚午,泰符、臨邑二縣民謀逆,其首王驢兒伏誅,餘杖流之。睢、許二州去年水旱,免其租。辛未,駙馬脫脫薨,賜鈔五萬貫。丙子,熒惑退犯東咸。庚辰,賑固安州饑。置營於永平,收養蒙古子女,遣使諭四方,匿者罪之。癸未,以御史大夫脫脫為江南行臺御史大夫。置宗仁蒙古侍衛親軍都指揮使司。甲申,車駕幸五臺山。賑夏津、永清二縣饑。以只兒哈郎為御史大夫。乙酉,以拜住領宗仁蒙古侍衛親軍都指揮使司事,佩三珠虎符。京師饑,發粟二十萬石賑糴。雲南行省平章答失鐵木兒、朵兒只坐贓杖免。戊子,禁民集眾祈神。庚寅,河南、陜西、河間、保定、彰德等路饑,發粟賑之,仍免常賦之半。調各衛漢軍二千,充宗仁衛屯田卒。翽星於五臺山。甲午,賑鞏昌階州饑。丙申,以吳全節為玄教大宗師,特進上卿。閏月戊戌,封諸葛忠武侯為威烈忠武顯靈仁濟王。辛丑,萬戶李英以良民為奴,擅文其面,坐罪。癸卯,禁白蓮佛事。睢陽縣亳社屯大水,饑,賑之。諸王阿馬、承童坐擅徙脫列捏王衛士,並杖流海南。甲辰,御史臺臣請黜監察御史不稱職者,以示懲勸,從之。丙午,嶺北戍卒貧乏,賜鈔三千二百五十萬貫、帛五十萬匹。戊申,奉元路郿縣及成州饑,並賑之。以鐵木迭兒子同知樞密院事班丹知樞密院事。己酉,也不干八禿兒戍邊有功,賜以金、鈔。壬子,作紫檀殿。乙卯,以淮安路去歲大水,遼陽路隕霜殺禾,南康路旱,並免其租。壬戌,安豐屬縣霖雨傷稼,免其租。興元褒城縣饑,賑之。甲子,真定、山東諸路饑,弛其河泊之禁。丙寅,辰州沅陵縣洞蠻為寇,遣兵捕之。敕:「已除不赴任者,奪其官。」封公主速哥八剌乳母為順國夫人。六月丁卯朔,車駕至五臺山,禁扈從宿衛,毋踐民禾。置中慶、大理二路推官各一員。戊辰,揚州屬縣旱,免其租。己巳,廣元路綿谷、昭化二縣饑,官市米賑之。壬申,熒惑犯心。癸酉,申禁日者妄談天象。甲戌,新平、上蔡二縣水,免其租。丙子,修渾河堤。壬午,辰州江水溢,壞民廬舍。丁亥,奉元屬縣水,淮安屬縣旱,並免其租。庚寅,思州風、雹,建德路水,皆賑之。



 秋七月戊戌,淮安路水,民饑,免其租。己亥,熒惑犯天江。丁未,賜拜住平江田萬畝。壬子,遣親王闍闍禿總兵北邊,賜金二百五十兩、銀二千五百兩、鈔五十萬貫。戊午,太陰犯井宿鉞星。車駕次應州,曲赦金城縣囚徒。庚申,升靖州為路。辛酉,次澤源州。中書左丞張思明坐罪杖免,籍其家。甲子,錄京師諸役軍匠病者千人,各賜鈔遣還。南康路大水,廬州六安縣大雨,水暴至,平地深數尺,民饑,命有司賑糧一月。



 八月戊辰,祭社稷。己巳,道州寧遠縣民符翼軫作亂,有司討擒之。壬申,蔚州民獻嘉禾。甲戌,次奉聖州。築宗仁衛營。給廬州流民復業者行糧。戊寅,詔畫《蠶麥圖》於鹿頂殿壁,以時觀之,可知民事也。己卯,廬州路六安、舒城縣水,賑之。庚辰,增壽安山寺役卒七千人。庚寅,鐵木迭兒卒,命給直市其葬地。甲午,瑞州高安縣饑,命有司賑之。九月戊戌,大寧路、水達達等驛水傷禾,賑之。給蒙古子女貧乏者鈔七百五十萬貫。戊申,給壽安山造寺役軍匠死者鈔,人百五十貫。庚戌,申禁江南典雇妻妾。辛亥,幸壽安山寺,賜監役官鈔,人五千貫。甲寅,賑淮東泰興等縣饑。丙辰,太皇太后崩。戊午,賜蒙古子女鈔百五十萬貫。己未,太陰犯明堂。庚申,敕停今年冬祀南郊。癸亥,地震。甲子,臨安河西縣春夏不雨,種不入土,居民流散,命有司賑給,令復業。作層樓於涿州鹿頂殿西。丙寅,西僧班吉疾,賜鈔五萬貫。



 冬十月丁卯,太史院請禁明年興作土功,從之。戊辰,享太廟,以國哀迎香去樂,修廟工役未畢,妨陳宮懸,止用登歌。丙子,押濟思國遣使來貢方物。江南行臺大夫脫脫坐請告未得旨輒去職,杖謫雲南。庚辰至辛巳,太陰犯井。甲申,建太祖神御殿於興教寺。己丑,熒惑犯壘壁陣。以拜住為中書右丞相。南恩州賊潭庚生等降。



 十一月甲午朔,日有食之。己亥,以立右丞相詔天下。流民復業者,免差稅三年。站戶貧乏鬻賣妻子者,官贖還之。凡差役造作,先科商賈末技富實之家,以優農力。免陜西明年差稅十之三,各處官佃田明年租十之二,江淮創科包銀全免之。御史李端言:「近者京師地震,日月薄蝕,皆臣下失職所致。」帝自責曰:「是朕思慮不及致然。」因敕群臣亦當修飭,以謹天戒。罷世祖以後冗置官。括江南僧有妻者為民。安南國遣使來貢方物,回賜金四百五十兩、金幣九,帛如之。癸卯,地震。甲辰,太白犯壘壁陣。罷徽政院。乙巳,熒惑犯壘壁陣。丙午,造龍船三艘。戊申,太陰掩井。岷州旱、疫,賑之,賜戍北邊萬戶、千戶等官金帶。御史李端言:「朝廷雖設起居注,所錄皆臣下聞奏事目。上之言動,宜悉書之,以付史館。世祖以來所定制度,宜著為令,使吏不得為奸,治獄者有所遵守。」並從之。乙卯,遣西僧高主瓦迎帝師。宣德府宣德縣地屢震,賑被災者糧、鈔。己未,太陰犯東咸。定脫脫禾孫入流官選,給印與俸。置八番軍民安撫司,改長官所二十有八為州縣。庚申,太陰犯天江。辛酉,熒惑犯歲星。真人蔡道泰殺人,伏誅;刑部尚書不答失裡坐受其金,範德鬱坐詭隨,並杖免。平江路水,損官民田四萬九千六百三十頃,免其租。



 十二月甲子朔,南康建昌州大水,山崩,死者四十七人,民饑,命賑之。乙丑,太白、歲星、熒惑三星聚於室,太白犯壘壁陣。丁卯,中書平章政事買驢罷為大司農,廉恂罷為集賢大學士,以集賢大學士張珪為中書平章政事。戊辰,以掌道教張嗣成、吳全節、藍道元各三授制命、銀印,敕奪其二。壬申,免回回人戶屯戍河西者銀稅。甲戌,兩江來安路總管岑世興作亂,遣兵討之。鐵木迭兒子宣政院使八思吉思,坐受劉夔冒獻田地伏誅,仍籍其家。乙亥,太陰掩井。丙寅,增鎮南王脫不花戍兵。戊寅,太白犯歲星。庚辰,葛蠻安撫司副使龍仁貴作亂,湖廣行省督兵捕之。以知樞密院事欽察臺為宣政院使,參知政事速速為中書左丞,宗仁侍衛親軍都指揮使馬剌為參知政事。癸未,紹興路柔遠州洞蠻把者為寇,遣兵捕之。以御史大夫只兒哈郎知樞密院事。封闍闍禿為武寧王,授金印。以地震、日食,命中書省、樞密院、御史臺、翰林、集賢院集議國家利害之事以聞。敕兩都營繕仍舊,餘如所議。弛河南、陜西等處酒禁。禁近侍奏取沒入錢物。乙酉,杭州火,賑之。丙戌,定謚太皇太后曰昭獻元聖,遣太常禮儀院使朵臺以謚議告於太廟。升寧昌府為下路,增置一縣。並云南西沙縣入寧州。賜淮安忠武王伯顏祠祭田二十頃。己丑,熒惑犯外屏,太陰犯建星。辛卯,給蒙古流民糧、鈔,遣還本部。張珪足疾免朝賀。西僧灌頂疾,請釋囚,帝曰:「釋囚祈福,豈為師惜。朕思惡人屢赦,反害善良,何福之有?」宣徽院臣言:「世祖時晃吉剌歲輸尚食羊二千,成宗時增為三千,今請增五千。」帝不許,曰:「天下之民,皆朕所有,如有不足,朕當濟之。若加重賦,百姓必致困窮,國亦何益。」命遵世祖舊制。徽州、廬州、濟南、真定、河間、大名、歸德、汝寧、鞏昌諸處及河南芍陂屯田水,大同、衛輝、江陵屬縣及豐贍署大惠屯風,河南及雲南烏蒙等處屯田旱,汴梁、順德、河間、保定、慶元、濟寧、濮州、益都諸屬縣及諸衛屯田蝗。



 三年春正月癸巳朔,暹國及八番洞蠻酋長各遣使來貢。曹州禹城縣去秋霖雨害稼,縣人邢著、程進出粟以賑饑民,命有司旌其門。乙未,享太廟。己亥,思明州盜起,湖廣行省督兵捕之。庚子,刑部尚書烏馬兒坐贓杖免。壬寅,命太僕寺增給牝馬百匹,供世祖、仁宗御容殿祭祀馬湩。和林阿蘭禿等驛戶貧乏,給鈔賑之。以行中書省平章政事復兼總軍政,軍官有罪,重者以聞,輕者就決。罷上都、雲州、興和、宣德、蔚州、奉聖州及雞鳴山、房山、黃蘆、三義諸金銀冶,聽民採煉,以十分之三輸官。授前樞密院副使吳元珪、王約集賢大學士,翰林侍講學士韓從益昭文館大學士,並商議中書省事。拜住言:「前集賢侍講學士趙居信、直學士吳澄,皆有德老儒,請徵用之。」帝喜曰:「卿言適副朕心,更當搜訪山林隱逸之士。」遂以居信為翰林學士承旨,澄為學士。增置上都留守司判官二員,以漢人為之,專掌刑名。置仁宗中宮位提舉司二,秩正五品,隸承徽寺。太陰犯鉞星,又犯井。癸卯,太陰犯井。甲辰,鎮西武寧王部饑,賑之。遣諸王忽剌出往鎮雲南,賜鈔萬五千貫。辛亥,申命鐵失振舉臺綱。壬子,建諸王驛於京師,遣回回砲手萬戶赴汝寧、新蔡,遵世祖舊制,教習砲法。靜江、邕、柳諸郡獠為寇,命湖廣行省督兵捕之。甲寅,以宗仁衛蒙古子女額足萬戶,命罷收之。乙卯,征東末吉地兀者戶,以貂鼠、水獺、海狗皮來獻,詔存恤三歲。丙辰,泉州民留應總作亂,命江浙行省遣兵捕之。丁巳,定封贈官等秩。辛酉,禁故殺子孫誣平民者。增置兵部尚書一員。四川行省平章政事趙世延,為其弟訟不法事,系獄待對,其弟逃去,詔出之。仍著為令:逃者百日不出,則釋待對者。命樞密副使完顏納丹、侍御史曹伯啟、也可扎魯忽赤不顏、集賢學士欽察、翰林直學士曹元用,聽讀仁宗時纂集累朝格例。敕:「常調官外不次銓用者,但升以職,勿升其階。」



 二月癸亥朔,作上都華嚴寺、八思巴帝師寺及拜住第,役軍六千二百人。定軍官襲職,嫡長子孫幼者,令諸兄弟侄攝之,所受制敕書權襲,以息爭訟。是夜,熒惑、太白、填星三星聚於胃。丙寅,翰林國史院進《仁宗實錄》。遣教化等往西番撫初附之民,徵畜牧,治郵傳。戊辰,祭社稷。天壽節,賓丹、爪哇等國遣使來貢。己巳,修通惠河閘十有九所。治野狐、桑乾道。癸酉,畋於柳林,顧謂拜住曰:「近者地道失寧,風雨不時,豈朕纂承大寶行事有闕歟?」對曰:「地震自古有之,陛下自責固宜,良由臣等失職,不能燮理。」帝曰:「朕在位三載,於兆姓萬物,豈無乖戾之事?卿等宜與百官議,有便民利物者,朕即行之。」置鎮遠王也不干王傅官屬。罷播州黃平府長官所一,徙其民隸黃平。是夜,太白犯昴。辛巳,造五輅。司徒劉夔、同僉宣政院事囊加臺,坐妄獻地土、冒取官錢,伏誅。格例成定,凡二千五百三十九條,內斷例七百一十七、條格千一百五十一、詔赦九十四、令類五百七十七,名曰《大元通制》,頒行天下。是夜,太陰犯東咸。癸未,賑北邊軍鈔二十五萬錠、糧二萬石。命宣徽院選蒙古子男四百入宿衛。罷徽政院總管府三:都總管府隸有司,怯憐口及人匠總管府隸陜西行中書省。降開成路為州。丙戌,雨土。京師饑,發粟二萬石賑糶。造五輅旗。丁亥,敕金書《藏經》二部,命拜住等總之。戊午,封鷹師不花為趙國公。辛卯,以太子賓客伯都廉貧,賜鈔十萬貫。諸王月思別遣使來朝。罷稱海宣慰司及萬戶府,改立屯田總管府。諸王怯伯遣使貢蒲萄酒。海漕糧至直沽,遣使祀海神天妃。



 三月壬辰朔,車駕幸上都。賜諸王喃答失鈔二百五十萬貫,復給諸王脫歡歲賜。丁酉,平江路嘉定州饑,發粟六萬石賑之。戊戌,安豐芍陂屯田女直戶饑,賑糧一月。庚子,崇明諸州饑,發米萬八千三百石賑之。甲辰,臺州路黃巖州饑,賑糧兩月。丁未,西番參卜郎諸族叛,敕鎮西武靖王搠思班等發兵討之。戊申,祔太皇太后於順宗廟室,遣攝太尉、中書右丞相拜住奉玉冊、玉寶上尊謚曰昭獻元聖皇后。辛亥,以圓明、王道明之亂,禁僧、道度牒、符錄。丙辰,敕:「醫、卜、匠官,居喪不得去職,七十不聽致仕,子孫無廕敘,能紹其業者,量材錄用。」監察御史拜住、教化,坐舉八思吉思失當,並黜免。諸王火魯灰部軍驛戶饑,賑之。



 夏四月壬戌朔,敕天下諸司命僧誦經十萬部。丙寅,察罕腦兒蒙古軍驛戶饑,賑之。丁卯,旌內黃縣節婦王氏。己巳,浚金水河。甲戌,命張珪及右司員外郎王士熙勉勵國子監學。敕都功德使闊兒魯至京師。釋囚大闢三十一人,杖五十七以上者六十九人。放籠禽十萬,令有司償其直。己卯,詔行助役法,遣使考視稅籍高下,出田若干畝,使應役之人更掌之,收其歲入以助役費,官不得與。北邊軍饑,賑之。蒙古大千戶部,比歲風雪斃畜牧,賑鈔二百萬貫。敕京師萬安、慶壽、聖安、普慶四寺,揚子江金山寺、五臺萬聖祐國寺,作水陸佛事七晝夜。丁亥,故羅羅斯宣慰使述古妻漂末權領司事,遣其子娑住邦來獻方物。戊子,南豐州民及鞏昌蒙古軍饑,賑之。



 五月辛卯,設大理路白鹽城榷稅官,秩正七品;中慶路榷稅官,秩從七品。置安慶灊山縣、雲南寧遠州。戊戌,太白經天。庚子,大風雨雹,拔柳林行宮內外大木二千七百。辛丑,以鐵失獨署御史大夫事。壬寅,雲南行省平章政事忽辛坐贓杖免。詔中外開言路。置慶元路嶧山縣,增尉一員。徙安寨縣於龍安驛。癸卯,太陰犯房。乙巳,嶺北米貴,禁釀酒。戊申,監察御史蓋繼元、宋翼言:「鐵木迭兒奸險貪污,請毀所立碑。」從之,仍追奪官爵及封贈制書。帝御大安閣,見太祖、世祖遺衣皆以縑素木綿為之,重加補綴,嗟嘆良久,謂侍臣曰:「祖宗創業艱難,服用節儉乃如此,朕焉敢頃刻忘之!」太白犯畢。癸丑,荊湖宣慰使脫列受賂,事覺,召至京師,御史臺臣請遣就鞫,不允。乙卯,賜勛舊子撒兒蠻、按灰鐵木兒、也先鐵木兒鈔,人萬五千貫。以鈔千萬貫,市羊馬給嶺北戍卒,人騬馬二、牝馬二、羊十五。禁驛戶無質賣官地。丙辰,東安州水,壞民田千五百六十頃。戊午,真定路武邑縣南水害稼。奉元行宮正殿災。上都利用監庫火,帝令衛士撲滅之。因語群臣曰:「世皇始建宮室,於今安焉。朕嗣登大寶,而值此毀,此朕不能圖治之故也。」欽察衛兵戍邊,有卒累功,請賞以官,帝曰:「名爵豈賞人之物?」命賜鈔三千貫。大名路魏縣霖雨,大同路雁門屯田旱損麥,諸衛屯田及永清縣水,保定路歸信縣蝗。六月,寇圍寧都,州民孫正臣出糧餉軍,旌其門。丁卯,西番參卜郎諸寇未平,遣徽政使醜驢往督師。戊辰,毀鐵木迭兒父祖碑,追收元受制書,告諭中外。贈乳母忽禿臺定襄郡夫人,其夫阿來追封定襄王,謚忠愍。壬申,將作院使哈撒兒不花坐罔上營利,杖流東裔,籍其家。留守司以雨請修都城,有旨:「今歲不宜大興土功,其略完之。」癸酉,置太廟夾室。贈燕赤吉臺太赤為襄安王。諸王別思鐵木兒統兵北部,別頒歲賜。太常請纂修累朝儀禮,從之。癸未,填星犯畢。乙酉,易、安、滄、莫、霸、祁諸州及諸衛屯田水,壞田六千餘頃。諸王怯伯數寇邊,至是遣使來降,帝曰:「朕非欲彼土地人民,但吾民不罹邊患,軍士免於勞役,斯幸矣。今既來降,當厚其賜以安之。」



 秋七月辛卯朔,宣政使欽察臺自傳旨署事,中書以體制非宜,請通行禁止,從之。壬辰,占城國王遣其弟保佑八剌遮奉表來貢方物。真定路驛戶饑,賑糧二千四百石。癸卯,太廟成。班丹坐贓杖免。賜剌禿屯田貧民鈔四十六萬八千貫市牛具。甲辰,諸王帖木兒還自雲南,八宿衛,賜鈔二萬五千貫。乙巳,招諭左右兩江黃勝許、岑世興。己酉,封諸王忽都鐵木兒為威遠王,授金印。減海道歲運糧二十萬石,並免江淮增科糧。甲寅,買馬行宮駕車六百五十匹。丙辰,永寧王卜顏鐵木兒為不法,命宗正府及近侍雜治其傅。籍鐵木迭兒家資。諸王徹徹禿入朝請印,帝以其政績未著,不允,賜鈔二十五萬貫。御史臺請降旨開言路,帝曰:「言路何嘗不開,但卿等選人未當爾。」漷州雨,水害屯田稼。真定州諸路屬縣蝗,冀寧、興和、大同三路屬縣隕霜。東路蒙古萬戶府饑,賑糧兩月。



 八月癸亥,車駕南還,駐蹕南坡。是夕,御史大夫鐵失、知樞密院事也先帖木兒、大司農失禿兒、前平章政事赤斤鐵木兒、前雲南行省平章政事完者、鐵木迭兒子前治書侍御史鎖南、鐵失弟宣徽使鎖南、典瑞院使脫火赤、樞密院副使阿散、僉書樞密院事章臺、衛士禿滿及諸王按梯不花、孛羅、月魯鐵木兒、曲呂不花、兀魯思不花等謀逆,以鐵失所領阿速衛兵為外應,鐵失、赤斤鐵木兒殺丞相拜住,遂弒帝於行幄。年二十一,從葬諸帝陵。泰定元年二月,上尊謚曰睿聖文孝皇帝,廟號英宗。四月,上國語廟號曰格堅。



 英宗性剛明,嘗以地震減膳、徹樂、避正殿,有近臣稱觴以賀,問:「何為賀?朕方修德不暇,汝為大臣,不能匡輔,反為諂耶?」斥出之。拜住進曰:「地震乃臣等失職,宜求賢以代。」曰:「毋多遜,此朕之過也。」嘗戒群臣曰:「卿等居高位,食厚祿,當勉力圖報。茍或貧乏,朕不惜賜汝;若為不法,則必刑無赦。」八思吉思下獄,謂左右曰:「法者,祖宗所制,非朕所得私。八思吉思雖事朕日久,今其有罪,當論如法。」嘗御鹿頂殿,謂拜住曰:「朕以幼沖,嗣承大業,錦衣玉食,何求不得。惟我祖宗櫛風沐雨,戡定萬方,曾有此樂邪?卿元勛之裔,當體朕至懷,毋忝爾祖。」拜住頓首對曰:「創業惟艱,守成不易,陛下睿思及此,億兆之福也。」又謂大臣曰:「中書選人署事未旬日,御史臺即改除之。臺除者,中書亦然。今山林之下,遺逸良多,卿等不能盡心求訪,惟以親戚故舊更相引用邪?」其明斷如此。然以果於刑戮,奸黨畏誅,遂構大變雲。



\end{pinyinscope}