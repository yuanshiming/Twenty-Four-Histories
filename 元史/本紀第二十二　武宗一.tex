\article{本紀第二十二 武宗一}

\begin{pinyinscope}

 武宗仁惠宣孝皇帝,諱海山,順宗答剌麻八剌之長子也。母曰興聖皇太后,弘吉剌氏。至元十八年七月十九日生。



 成宗大德三年,以寧遠王闊闊出總兵北邊,怠於備御,命帝即軍中代之。四年八月,與海都軍戰於闊別列之地,敗之。十二月,軍至按臺山,乃蠻帶部落降。五年八月朔,與海都戰於迭怯里古之地,海都軍潰。越二日,海都悉合其眾以來,大戰於合剌合塔之地,師失利,親出陣力戰,大敗之,盡獲其輜重,悉援諸王、駙馬眾軍以出。明日復戰,軍少卻,海都乘之,帝揮軍力戰,突出敵陣後,全軍而還。海都不得志去,旋亦死。



 八年十月,封帝懷寧王,賜金印,置王傅官,食瑞州六萬五千戶。十年七月,自脫忽思圈之地逾按臺山,追叛王斡羅思,獲其妻孥輜重;執叛王也孫禿阿等及駙馬伯顏。八月,至也裏的失之地,受諸降王禿滿、明里鐵木兒、阿魯灰等降。海都之子察八兒逃於都瓦部,盡俘獲其家屬營帳。駐冬按臺山,降王禿曲滅復叛,與戰敗之,北邊悉平。



 十一年春,聞成宗崩,三月,自按臺山至於和林。諸王勛戚畢會,皆曰今阿難答、明里鐵木兒等熒惑中宮,潛有異議;諸王也只裏昔嘗與叛王通,今亦預謀。既辭服伏誅,乃因闔辭勸進。帝謝曰:「吾母、吾弟在大都,俟宗親畢會,議之。」先是,成宗違豫日久,政出中宮,命仁宗與皇太后出居懷州。至是,仁宗聞訃,以二月辛亥與太后俱至京師。安西王阿難答與諸王明里鐵木兒已於正月庚午先至。左丞相阿忽臺,平章八都馬辛,前中書平章伯顏,中政院使怯烈、道興等潛謀推成宗皇后伯要真氏稱制,阿難答輔之。仁宗以右丞相哈剌哈孫之謀言於太后曰:「太祖、世祖創業艱難,今大行晏駕,德壽已薨,諸王皆疏屬,而懷寧王在朔方,此輩潛有異圖,變在朝夕,俟懷寧王至,恐亂生不測,不若先事而發。」遂定計,誅阿忽臺、怯列等,而遣使迎帝。



 五月,至上都。乙丑,仁宗侍太后來會,左右部諸王畢至會議,乃廢皇后伯要真氏,出居東安州,賜死;執安西王阿難答、諸王明里鐵木兒至上都,亦皆賜死。甲申,皇帝即位於上都,受諸王文武百官朝於大安閣,大赦天下,詔曰:



 昔我太祖皇帝以武功定天下,世祖皇帝以文德洽海內,列聖相承,丕衍無疆之祚。朕自先朝,肅將天威,撫軍朔方,殆將十年,親御甲胄,力戰卻敵者屢矣。方諸籓內附,邊事以寧,遽聞宮車晏駕,乃有宗室諸王、貴戚元勛相與定策於和林,咸以朕為世祖曾孫之嫡,裕宗正派之傳,以功以賢,宜膺大寶。朕謙讓未遑,至於再三。還至上都,宗親大臣復請於朕。間者奸臣乘隙,謀為不軌,賴祖宗之靈,母弟愛育黎拔力八達稟命太后,恭行天罰。內難既平,神器不可久虛,宗祧不可乏祀,合辭勸進,誠意益堅。朕勉徇輿情,於五月二十一日即皇帝位。任大守重,若涉淵冰。屬嗣服之雲初,其與民更始,可大赦天下。存恤征戍軍士及供給繁重州郡,免上都、大都、隆興差稅三年,其餘路分,量重輕優免。雲南八番、田楊地面,免差發一年。其積年逋欠者,蠲之;逃移復業者,免三年。被災之處,山場湖泊課程,權且停罷,聽貧民採取。站赤消乏者,優之。經過軍馬,勿得擾民。諸處鐵冶,許諸人煽辦。勉勵學校,蠲儒戶差役;存問鰥寡孤獨。



 是日,追尊皇考曰皇帝,尊太母元妃曰皇太后。丁亥,升通政院秩正二品,升儀鳳司為玉宸樂院,秩從二品。壬辰,加知樞密院事朵兒朵海太傅,中書右丞相哈剌哈孫答剌罕太保,並錄軍國重事;知樞密院事塔剌海為中書左丞相,預樞密院、宣徽院事;同知徽政院事床兀兒、也可扎魯忽赤阿沙不花、江浙行省平章政事明裏不花,並為中書平章政事;江浙行省左丞劉正為中書左丞;遙授中書左丞欽察、福建道宣慰使也先帖木兒,並為中書參知政事;中書右丞、行御史中丞塔思不花為御史大夫;平章政事床兀兒為知樞密院事。特授乞臺普濟中書平章政事,延慶使抄兒赤中書右丞,同知和林等處宣慰司事塔海中書右丞,阿里中書左丞,脫脫御史大夫。以大都迤北六十二驛驛戶罷乏,給鈔賙之。是月,封皇太子乳母李氏為壽國夫人,其夫燕家奴為壽國公。以中書平章政事合散為遼陽行省平章政事。建州大雨雹,真定、河間、順德、保定等郡蝗。



 六月癸巳朔,詔立母弟愛育黎拔力八達為皇太子,受金寶。升武備寺為武備院,秩從二品。甲午,建行宮於旺兀察都之地,立宮闕為中都。丁酉,中書右丞相哈剌哈孫答剌罕、左丞相塔剌海言:「臣等與翰林、集賢、太常老臣集議,皇帝嗣登寶位,詔追尊皇考為皇帝,皇考大行皇帝同母兄也,大行皇帝祔廟之禮尚未舉行,二帝神主依兄弟次序祔廟為宜。今擬請謚皇考昭聖衍孝皇帝,廟號順宗;大行皇帝曰欽明廣孝皇帝,廟號成宗。太祖之室居中,睿宗西第一室,世祖西第二室,裕宗西第三室,順宗東第一室,成宗東第二室。先元妃弘吉剌氏失憐答里宜謚曰真慈靜懿皇后,祔成宗廟室。」制曰:「可。」又言:「前奉旨命臣等議諸王朝會賜與,臣等議:憲宗、世祖登寶位時賞賜有數,成宗即位,承世祖府庫充富,比先例,賜金五十兩者增至二百五十兩,銀五十兩者增至百五十兩。」有旨:「其遵成宗所賜之數賜之。」戊戌,哈剌哈孫答剌罕言:「比者諸王、駙馬會於和林,已蒙賜與者,今不宜再賜。」帝曰:「和林之會,國事方殷,已賜者,其再賜之。」己亥,御史大夫脫脫、翰林學士承旨三寶奴言:「舊制,皇太子官屬,省、臺參用,請以羅羅斯宣慰使斡羅思任之中書。」詔以為中書右丞。班朝諸司,聽皇太子各置一人。以拱衛直都指揮使馬謀沙角牴屢勝,遙授平章政事。壬寅,塔剌海加太保、錄軍國重事、太子太師。癸卯,置詹事院。甲辰,樞密院請以軍二千五百人繕治上都鷹坊及諸官廨,有旨:「自今非奉旨,軍勿輒役。」以平章政事、行和林等處宣慰使都元帥憨剌合兒,通政使、武備卿鐵木兒不花,並知樞密院事。乙巳,以金二千七百五十兩、銀十二萬九千二百兩、鈔萬錠、幣帛二萬二千二百八十匹奉興聖宮,賜皇太子亦如之。中書省臣言:「中書宰臣十四員,御史大夫四員,前制所無。」詔與翰林、集賢諸老臣議擬以聞。丙午,太陰犯南斗杓星。徽政使頭等言:「拜布哈以私錢建寺,為國祝厘。其父為諸王斡忽所害,請賜以斡忽所得歲賜。」命以五年與之,為銀四千一百餘兩、絲三萬一千二百九十斤、織幣金百兩、絹七百一十匹。戊申,特授尚乘卿孛蘭奚、床兀並平章政事,大同屯儲軍民總管府達嚕噶齊怯裏木丁中書右丞。辛亥,以中書平章政事托克托為江西行省平章政事。壬子,封皇妹祥哥剌吉為魯國大長公主,駙馬周阿不剌為魯王。特穆爾布哈、哈喇格爾等言:「舊制,樞密院銓調軍官,公議以聞。比者近侍自擇名分,從內降旨,恐壞世祖定制,且誤國事。在成宗時嘗有旨,輒奏樞密事者,許本院再陳。臣等以為自今用人,宜一遵世祖成憲。」帝曰:「其遵前制,餘人忽有輒請。」又言:「軍官與民官不同,父子兄弟許其相襲,此世祖定制。比者近侍輒有以萬戶、千戶之職請於上者,內降聖旨,臣等未敢奉行。」帝曰:「其依例行之。」甲寅,敕內郡、江南、高麗、四川、雲南諸寺僧誦《藏經》,為三宮祈福。乙卯,遣伊克扎爾古齊瑪喇勒赴北軍,以印給之。丙辰,御史大夫塔斯布哈言:「殿中司所職:中書而下奏事,必使隨之以入;不在奏事之列者,聽其引退;班朝百官朝會失儀者,得糾劾;病故者,必以告。請如舊制。」又言:「舊制,內外風憲官有所彈劾,諸人勿預。而近有受贓為監察御史所劾者,獄具,夤緣奏請,托言事入覲,以避其罪。臣等以為今後有罪者,勿聽至京,待其對辯事竟,果有所言,方許奏陳。」皆從之。塔思不花又言:「皇太子有旨:有司贓罪,不須刑部定議,受敕者從廉訪司處決,省、臺遣人檢核廉訪司文案,則私意沮格,非便。」平章阿沙不花因言:「此省、臺同議之事,臺臣不宜獨奏。」帝曰:「此御史臺事,阿沙不花勿妄言。臺臣言是也,如所奏行之。」塔思不花、脫脫並遙授左丞相。戊午,進封高麗王王昛為沈陽王,加太子太傅、駙馬都尉。置皇太子家令司、府正司、延慶司、曲寶署、典膳署。己未,封寧遠王闊闊出為寧王,賜金印。庚申,遙授左丞相、行御史大夫塔思不花右丞相。辛酉,汴梁、南陽、歸德、江西、湖廣水,保定屬縣蝗。



 秋七月癸亥朔,封諸王禿剌為越王。諸王出伯言:「瓜州、沙州屯田逋戶漸成丁者,乞拘隸所部。」中書省臣言:「瓜州雖諸王分地,其民役於驛傳,出伯言宜勿從。」升章佩監為章佩院,秩從二品,賜阿剌納八剌鈔萬錠。甲子,命御史臺大夫鐵古迭兒、知樞密院事塔魯忽帶、中書平章政事床兀兒以即位告謝南郊。丙寅,以禮店蒙古萬戶屬土番宣慰司非便,命仍舊隸脫思麻宣慰司,防守陜州。諸王、駙馬入覲者,非奉旨不許給驛。以中書參知政事趙仁榮為太子詹事。以阿保功,授明裏大司徒,封其妻梅仙為順國夫人。賜床兀兒軍士鈔六萬錠、幣帛二萬匹。遣肥兒牙兒迷的裡及鐵胳膽詣西域取佛缽、舍利,肥兒牙兒迷的里遙授宣政使,鐵胳膽遙授平章政事。以並命太傅右丞相哈剌哈孫答剌罕、太保左丞相塔剌海綜理中書庶務,詔諭中外。己巳,太陰犯亢。置宮師府,設太子太師、少師、太傅、少傅、太保、少保,賓客,左、右諭德,贊善,庶子,洗馬,率更令、丞,司經令、丞,中允,文學,通事舍人,校書,正字等官。壬申,命御史大夫鐵古迭兒、中書平章政事床兀兒、樞密副使孛蘭奚,以即位祗謝太廟。以安西、平江、吉州三路為皇太子分地,越州路為越王禿剌分地。賜諸王八不沙鈔萬錠。癸酉,罷和林宣慰司,置行中書省及青海等處宣慰司都元帥府、和林總管府。以太師月赤察兒為和林行省右丞相,中書右丞相哈剌哈孫答剌罕為和林行省左丞相,依前太傅、錄軍國重事。江浙水,民饑,詔賑糧三月,酒醋、門攤、課程悉免一年。乙亥,以永平路為皇妹魯國長公主分地,租賦及土產悉賜之。賜越王禿剌鈔萬錠,諸王兀都思不花所部三萬五千二百二十錠。丙子,以江浙行省平章政事塔失海牙、知樞密院事床兀兒,並為中書平章政事。丁丑,封諸王八不沙為齊王,朵列納為濟王,迭裡哥兒不花為北寧王,太師月赤察兒為淇陽王,加平章政事脫虎脫太尉。以中書左丞相塔剌海為中書右丞相、監修國史,御史大夫塔思不花為中書左丞相,江浙行省平章政事教化、河南江北行省平章政事法忽魯丁並為中書平章政事,平章政事鐵木迭兒為江西行省平章政事。戊寅,以儀鳳司大使火失海牙、鐵木兒不花、教坊司達魯花赤沙的,並遙授平章政事,為玉宸樂院使。己卯,以集賢院使別不花為中書平章政事。庚辰,以御史中丞只兒合郎為御史大夫。辛巳,加封至聖文宣王為大成至聖文宣王。右丞相塔剌海、左丞相塔思不花言:「中書庶務,同僚一二近侍,往往不俟公議,即以上聞,非便。今後事無大小,請共議而後奏。」帝曰:「卿等言是。自今庶政,非公議者勿奏。」置行工部於旺兀察都。以遙授左丞相、同知樞密院事也兒吉絺知樞密院事;御史中丞王壽、江浙行省左丞郝天挺,並為中書左丞。壬午,熒惑犯南斗。命御史大夫鐵古迭兒、知樞密院事塔魯忽帶、中書平章政事床兀兒,以即位告社稷。癸未,升利用監為利用院,秩從二品。甲申,遣贍思丁使西域,遙授福建道宣慰使。乙酉,賜壽寧公主鈔萬錠。丙戌,以內郡歲歉,令諸王衛士還大都者揀汰以入。從和林省臣請,乞如甘肅省例,給鈔二千錠,歲收子錢,以佐供給,仍以網罟賜貧民。御史大夫月兒魯言:「舊制,中書省、樞密院、御史臺、宣政院許得自選其人,他司悉從中書銓擇,近臣不得輒奏,如此則紀綱不紊。」帝嘉納之。以同知宣徽院事孛羅答失為中書左丞,中書參知政事欽察為四川行省左丞。江浙、湖廣、江西屬郡饑,詔行省發粟賑之。丁亥,使完澤偕乞兒乞帶亦難往征乞兒吉思部禿魯花、騬馬、鷹鷂。山東、河北蒙古軍告饑,遣官賑之。賜晉王部貧民鈔五萬錠。己丑,塔剌海、塔思不花言:「前乃顏叛,其系虜之人,奉世祖旨俱隸版籍。比者近臣請以歸之諸王脫脫,彼即遣人拘括。臣等以為此事具有先制,今已歸脫脫所部,宜令遼陽省臣薛闍乾等往諭之,已拘之人悉還其主。」從之。安西等郡旱饑,以糧二萬八千石賑之。庚寅,置延福司,秩正三品。辛卯,詔唐兀禿魯花戶籍已定,其入諸王、駙馬各部避役之人及冒匿者,皆有罪。發卒二千人為晉王也孫鐵木兒治邸舍。是月,江浙、湖廣、江西、河南、兩淮屬郡饑,於鹽茶課鈔內折粟,遣官賑之。詔富家能以私粟賑貸者,量授以官。保定、河間、晉寧等郡水,德州蝗。



 八月甲午,中書省臣言:「內降旨與官者八百八十餘人,已除三百,未議者猶五百餘。請自今越奏者勿與。」帝曰:「卿等言是。自今不由中書奏者,勿與官。」又言:「外任官帶相銜非制也,請勿與。」制可。又言:「以朝會應賜者,為鈔總三百五十萬錠,已給者百七十萬,未給猶百八十萬,兩都所儲已虛。自今特奏乞賞者,宜暫停。」有旨:「自今凡以賞為請者,勿奏。」御史臺臣言:「中書省、樞密院、御史臺、宣政院得自選官,具有成憲。今監察御史、廉訪司官非本臺公選,而從諸臣所請,自內降旨,非祖宗成法。」帝曰:「凡若此者,卿等其勿行。」浙東、浙西、湖北、江東郡縣饑,遣官賑之。賜山後驛戶鈔,每驛五百錠。置掌儀署,秩五品,設令、丞各一員。乙未,賜諸王按灰、阿魯灰、北寧王迭裡哥兒不花金三百五十兩、銀三千七百兩。以治書侍御史烏伯都剌為中書參知政事。戊戌,御史大夫脫脫封秦國公。辛丑,迤北之民新附者,置傳輸粟以賑之。癸卯,改也裏合牙營田司為屯田運糧萬戶府。甲辰,以納蘭不剌所儲糧萬石,賑其旁近饑民。丙午,建佛閣於五臺寺。江南饑,以十道廉訪司所儲贓罰鈔賑之。己酉,從皇太子請,升詹事院從一品,置參議斷事官如樞密院。辛亥,中書右丞孛羅鐵木兒以國字譯《孝經》進,詔曰:「此乃孔子之微言,自王公達於庶民,皆當由是而行。其命中書省刻版模印,諸王而下皆賜之。」癸丑,唐兀禿魯花軍乏食,發粟賑之。丙辰,升闌遺監秩三品。丁巳,以中書左丞王壽為御史中丞。戊午,中書平章政事乞臺普濟、床兀兒、別不花並加太尉,中書右丞塔海加太尉、平章政事,以中書左丞孛羅鐵木兒為中書右丞。東昌、汴梁、唐州、延安、潭、沅、歸、澧、興國諸郡饑,發粟賑之。冀寧路地震,河間、真定等郡蝗,隆平、文水、平遙、祁、霍邑、靖海、容城、束鹿等縣水。九月甲子,車駕至自上都。乙丑,請謚皇考皇帝、大行皇帝於南郊,命中書右丞相塔剌海攝太尉行事。庚午,升御史臺從一品。辛未,加塔剌海、塔思不花並太尉。壬申,命塔剌海奉玉冊、玉寶,上皇考及大行皇帝尊謚、廟號,又上先元妃弘吉烈氏尊謚,祔於成宗廟室。升尚舍監秩正三品。癸酉,太白犯右執法。甲戌,改太常寺為太常禮儀院,秩正二品。升侍儀司秩正三品。丙子,置皇太子位典牧監,秩正三品。中書省臣言:「內外選法,向者有旨一遵世祖成制。兩宮近侍遷敘,為上所命。比有應入常調者,夤緣驟遷;其已仕廢黜及未嘗入仕者,亦復請自內降旨。臣等奏請禁止,蒙賜允從。是後所降內旨復有百餘,臣等已嘗銓擇奉行。第中書政務,他人又得輒請,責以整飭,其效實難。自今銓選、錢穀,請如前制,非由中書議者,毋得越奏。」制從之。又言:「比怯來木丁獻寶貨,敕以鹽萬引與之,仍許市引九萬。臣等竊謂,所市寶貨,既估其直,止宜給鈔,若以引給之,徒壞鹽法。」帝曰:「此朕自言,非臣下所請,其給之,餘勿視為例。」江浙饑,中書省臣言:「請令本省官租,於九月先輸三分之一,以備賑給。又兩淮漕河淤澀,官議疏浚,鹽一引帶收鈔二貫為傭費,計鈔二萬八千錠,今河流已通,宜移以賑饑民。杭州一郡,歲以酒糜米麥二十八萬石,禁之便。河南、益都諸郡,亦宜禁之。」制可。塔剌海言:「比蒙聖恩,賜臣江南田百頃。今諸王、公主、駙馬賜田還官,臣等請還所賜。」從之。仍諭諸人賜田,悉令還官。命張留孫知集賢院事,領諸路道教事。丁丑,中書省臣言:「比議省臣員數,奉旨依舊制定為十二員,右丞相塔剌海,左丞相塔思不花,平章床兀兒、乞臺普濟如故,餘令臣等議。臣等請以阿沙不花、塔失海牙為平章政事,孛羅答失、劉正為右丞,郝天挺、也先鐵木兒為左丞,於璋、兀伯都剌為參知政事。其班朝諸司冗員,並宜揀汰。」從之。己卯,太白犯左執法。壬午,改尚乘寺為衛尉院,秩從二品。甲申,詔立尚書省,分理財用。命塔剌海、塔思不花仍領中書。以脫虎脫、教化、法忽魯丁任尚書省,仍俾其自舉官屬,命鑄尚書省印。敕弛江浙諸郡山澤之禁。丙戌,升掌謁司秩三品。皇太子建佛寺,請買民地益之,給鈔萬七百錠有奇。戊子,升延慶司秩從二品。己丑,遣使錄囚。晉王也孫鐵木兒以詔賜鈔萬錠、止給八千為言,中書省臣言:「帑藏空竭,常賦歲鈔四百萬錠,各省備用之外,入京師者二百八十萬錠,常年所支止二百七十餘萬錠。自陛下即位以來,已支四百二十萬錠,又應求而未支者一百萬錠。臣等慮財用不給,敢以上聞。」帝曰:「卿之言然。自今賜予宜暫停,諸人毋得奏請。可給晉王鈔千錠,餘移陜西省給之。」以中書平章政事別不花為江浙行省平章政事。辛卯,御史臺臣言:「至元中阿合馬綜理財用,立尚書省,三載並入中書。其後桑哥用事,復立尚書省,事敗又並入中書。粵自大德五年以來,四方地震水災,歲仍不登,百姓重困,便民之政,正在今日。頃又聞為總理財用立尚書省,如是則必增置所司,濫設官吏,殆非益民之事也。且綜理財用,在人為之,若止命中書整飭,未見不可。臣等隱而不言,懼將獲罪。」帝曰:「卿言良是。此三臣願任其事,姑聽其行焉。」是月,襄陽霖雨,民饑,敕河南省發粟賑之。十月乙未,升典寶署為典寶監,秩正三品。庚子,中書省奏:「初置中書省時,太保劉秉忠度其地宜,裕宗為中書令,嘗至省署敕。其後桑哥遷立尚書省,不四載而罷。今復遷中書於舊省,乞涓吉徙中書令位,仍請皇太子一至中書。」制可。壬寅,升典瑞監為典瑞院,秩從二品。封知樞密院事床兀兒為容國公。癸卯,以舊制諸王、駙馬事務皆內侍宰臣所領,命中書右丞孛羅鐵木兒領之。乙巳,太白犯亢。敕方士、日者毋游諸王、駙馬之門。丙午,詔整飭臺綱,布告中外。封御史大夫鐵古迭兒為鄆國公,以中衛親軍都指揮使買奴知樞密院事。壬子,從中書省臣言,凡事不由中書,輒遣使並移文者,禁止之。甲寅,太陰犯明堂。升集賢院秩從一品,將作院秩從二品。丙辰,以行省平章總督軍馬,得佩虎符,其左丞等所佩悉追納。中書省奏:「常歲海漕糧百四十五萬石,今江浙歲儉,不能如數,請仍舊例,湖廣、江西各輸五十萬石,並由海道達京師。」從之。己未,塔思不花上疏言政事,且辭太尉職,還所降制書及印。是月,杭州、平江水,民饑,發粟賑之。



 十一月癸亥,封諸王牙忽都為楚王,賜金印,置王傅。建佛寺於五臺山。乙丑,中書省臣言:「宿衛廩給及馬駝芻料,父子兄弟世相襲者給之,不當給者,請令孛可孫汰之。今會是年十月終,馬駝九萬三千餘,至來春二月,闕芻六百萬束、料十五萬石;比又增馬五萬餘匹。此國重務,臣等敢以上聞。」有旨:「不當給者勿給。」丙寅,帝朝隆福宮,上皇太后玉冊、玉寶。丁卯,太白犯房。闊兒伯牙裏言:「更用銀鈔、銅錢便。」命中書與樞密院、御史臺、集賢、翰林諸老臣集議以聞。己巳,中書省臣阿沙不花、孛羅鐵木兒言:「臣等與闊兒伯牙裏面論,折銀鈔、銅錢,非便。」有旨:「卿等以為不便,勿行可也。」詔:「中書省官十二員,脫虎脫仍領宣政院,教化留京師,其餘各任以職。」庚午,盧龍、灤河、遷安、昌黎、撫寧等縣水,民饑,給鈔千錠以賑之。辛未,以塔剌海領中政院事。乙亥,中書省臣言:「大都路供億浩繁,概於屬郡取之。其軍、站、鷹坊、控鶴等戶,恃其雜徭無與,冒占編氓。請降璽書,依祖宗舊制,悉令均當。或輒奏請者,亦宜禁止。」制可。皇太子言:「近蒙恩以安西、吉州、平江為分地,租稅悉以賜臣。臣恐宗親昆弟援例,自五戶絲外,餘請輸之內帑。其陜西運司歲辦鹽十萬引,向給安西王,以此錢斟酌與臣,惟陛下裁之。」中書計會三路租稅及鹽課所入,鈔四十萬錠。有旨:「皇太子所思甚善,歲以十萬錠給之,不足則再賜。」樂工毆人,刑部捕之,玉宸樂院長謂玉宸與刑部秩皆三品,官皆榮祿大夫,留不遣。中書以聞,帝曰:「凡諸司,視其資級,授之散官,不可超越。其閑冗職名官高者,遵舊制降之。」建康路屬州縣饑,詔免今年酒醋課。丙子,太陰犯東井。丁丑,中書省臣言:「前為江南大水,以茶、鹽課折收米,賑饑民。今商人輸米中鹽,以致米價騰湧,百姓雖獲小利,終為無益。臣等議,茶、鹽之課當如舊。」從之。戊寅,授皇太子玉冊。己卯,以皇太子受冊禮成,帝御大明殿,受諸王、百官朝賀。庚辰,中書省臣言:「皇太子謂臣等曰:吾之分地安西、平江、吉州三路,遵舊制,自達魯花赤之外,悉從常選,其常選宜速擇才能。」有旨:「其擇人任之。」乙酉,太陰犯亢。詔:「皇太後軍民人匠等戶租賦徭役,有司勿與,並隸徽政院。」升太僕院秩從二品。丁亥,杭州、平江等處大饑,發糧五十萬一千二百石賑之。庚寅,賜太師月赤察兒江南田四十頃。時賜田悉奪還官,中書省為言,有旨:「月赤察兒自世祖時積有勛勞,非餘人比,宜以前後所賜,合百頃與之。」仍敕行省平章別不花領其歲入。辛卯,辰星犯歲星。從皇太子請,御史臺檢核詹事院文案。



 十二月壬辰朔,中書省臣言:「舊制,金虎符及金銀符典瑞院掌之,給則由中書,事已則復歸典瑞院。今出入多不由中書,下至商人,結托近侍奏請,以致泛濫,出而無歸。臣等請核之,自後除官及奉使應給者,非由中書省勿給。」從之。又言:「今國用甚多,帑藏已乏,用及鈔母,非宜。鹽引向從運司與民為市,今權時制宜,從戶部鬻鹽引八十萬便。」有旨:「今歲姑從所請,後勿復行。」又言:「太府院為內藏,世祖、成宗朝,遇重賜則取給中書,今所賜有逾千錠至萬錠者,皆取之太府。比者太府取五萬錠,已支二萬矣,今復以乏告。請自後內府所用數多者,仍取之中書。」帝曰:「此朕特旨,後當從所奏。」乙未,貴赤塔塔兒等擾檀州民,強取米粟六百餘石,遣官訊之。辛丑,幸大聖壽萬安寺。授吏部尚書察乃平章政事,領工部事。癸卯,以漢軍萬人屯田和林,命留守司以來歲正月十五日起燈山於大明殿後、延春閣前。庚戌,升行泉府司為泉府院,秩正二品。以蒙古萬戶禿監鐵木兒有平內難功,加鎮國上將軍。升皇太子典醫署為典醫監,秩正三品。山東、河南、江浙饑,禁民釀酒。丁巳,以中書省言國用浩穰,民貧歲歉,詔宣政院並省佛事。大都、上都二驛,設敕授官二員,餘驛一員。敕諸王、公主、駙馬、使臣給璽書驛券,不許輒用圓符乘驛。中書省臣言:「驛戶疲乏,宜量事給驛。今經費浩大,其收售寶貨,權宜停罷。又,陛下即位詔書不許越職奏事,比者近侍奏除官丐賞者,皆自內降旨,請今不經中書省勿行。又,刑法者譬之權衡,不可偏重,世祖已有定制,自元真以來,以作佛事之故,放釋有罪,失於太寬,故有司無所遵守。今請凡內外犯法之人,悉歸有司依法裁決。又,各處民饑,除行宮外,工役請悉停罷。」皆從之。又言:「律令者治國之急務,當以時損益。世祖嘗有旨,金《泰和律》勿用,令老臣通法律者,參酌古今,從新定制,至今尚未行。臣等謂律令重事,未可輕議,請自世祖即位以來所行條格,校讎歸一,遵而行之。」制可。庚申,詔曰:



 仰惟祖宗應天撫運,肇啟疆宇,華夏一統,罔不率從。逮朕嗣服丕圖,纘膺景命,遵承詒訓,恪慕洪規,祗揚畏兢,未知攸濟。永思創業艱難之始,煢然軫念;而守成萬事之統,在予一人。故自即位以來,溥從寬大,量能授官,俾勤乃職,夙夜以永康兆民為急務。間者歲比不登,流民未還,官吏並緣侵漁,上下因循,和氣乖戾。是以責任股肱耳目大臣,思所以盡瘁贊襄嘉猷,朝夕入告,朕命惟允,庶事克諧,樂與率土之民,共享治安之化,邇寧遠肅,顧不韙歟。可改大德十二年為至大元年。誕布惟新之令,式孚永固之休。



 存恤征戍蒙古、漢軍,拯治站赤消乏。弛山場、河泊、蘆蕩禁。圍獵飛放毋得搔擾百姓,招誘流移人戶。禁投屬怯薛歹、鷹房避役,濫請錢糧。勸農桑,興學校,議貢舉,旌賞孝弟力田,懲戒游惰。政令得失,許諸人上書陳言。僧、道、也裏可溫、答失蠻,並依舊制納稅。凡選法、錢糧、刑名、造作一切公事,近侍人員毋得隔越聞奏。敕內庭作佛事,毋釋重囚,以輕囚釋之。



 至大元年春正月辛酉朔,曲赦御史臺見系犯贓官吏,罪止征贓、罷職。癸亥,敕樞密院發六衛軍萬八千五百人,供旺兀察都建宮工役。甲子,授中書平章政事阿沙不花右丞相、行御史大夫。丙寅,從江浙行省請,罷行都水監,以其事隸有司。立皇太子位典幄署、承和署,秩並正五品。丁卯,以中書右丞也罕的斤為平章政事,議陜西省事。己巳,紹興、臺州、慶元、廣德、建康、鎮江六路饑,死者甚眾,饑戶四十六萬有奇,戶月給米六斗,以沒入硃清、張瑄物貨隸徽政院者,鬻鈔三十萬錠賑之。特授乳母夫壽國公楊燕家奴開府儀同三司。己巳,緬國進馴象六。辛未,樞密院臣言:「先奉旨以中衛親軍隸皇太子位,皇太子謂臣等曰:世祖立五衛,以應五方,去一不可。宜各翼選漢軍萬人,別立一衛。」帝以為然,敕知院事鐵木兒不花等摘漢軍萬人,別立衛。甲戌,中書省臣言:「進海東青鶻者當乘驛,馬五百不敷,敕遣怯列、應童括民間車馬,兵部請以各驛馬陸續而進,勿括為便。」從之。改徽政院人匠總管府為繕珍司,秩正三品。己卯,升中尚監為中尚院,秩從二品。豳王出伯進玉六百一十五斤,賜金千五百兩、銀二萬兩、鈔萬錠,從人四萬錠;寬闍、也先孛可等,金二千三百兩、銀一萬七百兩、鈔三萬九千一百錠。甲申,敕床兀兒除登極恩例外,特賜金五百兩、銀千兩、鈔二千錠。戊子,皇太子請以阿沙不花復入中書,脫脫復入御史臺。己酉,中書省臣言:「阿失鐵木兒請遣教化的詣河西地採玉,馱攻玉沙需馬四十餘匹,採玉人千餘。臣等以為不急之務勞民,乞罷之。」又言:「近百姓艱食,盜賊充斥,茍不嚴治,將至滋蔓。宜遣使巡行,遇有罪囚,即行決遣,與隨處官吏共議弭盜方略,明立賞罰,或匿盜不聞,或期會不至,或逾期不獲者,官吏連坐。」又言:「江浙行省海賊出沒,殺虜軍民。其已獲者,例合結案待報,宜從中書省、也可扎魯忽赤遣官,同行省、行臺、宣慰司、廉訪司審錄無冤,棄之於市。其未獲者,督責追捕,自首者原罪給粟,能禽其黨者加賞。」有旨:「弭盜安民,事為至重,宜即議行之。」封諸王也先鐵木兒為營王,以乳母夫斡耳朵為司徒。



 二月癸巳,立鷹坊為仁虞院,秩正一品。以右丞相脫脫、遙授左丞相禿剌鐵木兒、也可扎魯忽赤月里赤,並為仁虞院使。汝寧、歸德二路旱、蝗,民饑,給鈔萬錠賑之。甲午,增泉府院副使、同僉各一員。益都、濟寧、般陽、濟南、東平、泰安大饑,遣山東宣慰使王佐同廉訪司核實賑濟,為鈔十萬二千二百三十七錠有奇、糧萬九千三百四十八石。乙未,中書省臣言:「陛下登極以來,錫賞諸王,恤軍力,賑百姓,及殊恩泛賜,帑藏空竭,豫賣鹽引。今和林、甘肅、大同、隆興、兩都軍糧,諸所營繕,及一切供億,合用鈔八百二十餘萬錠。往者或遇匱急,奏支鈔本。臣等固知鈔法非輕,曷敢輒動,然計無所出,今乞權支鈔本七百一十餘萬錠,以周急用,不急之費姑後之。」帝曰:「卿等言是。泛賜者,不以何人,毋得蒙蔽奏請。」升尚舍監為尚舍寺,秩正三品。丙申,立甄用監,秩正三品,隸徽政院。淮安等處饑,從河南行省言,以兩浙鹽引十萬貿粟賑之。戊戌,以上都衛軍三千人,赴旺兀察都行宮工役。壬寅,中書省臣言:「貴赤擾害檀州民,敕遣人往訊,其辭伏者宜加罪,有旨勿問。臣等以為非宜,已辭伏者,先為決遣。」帝曰:「俟其獵畢治之。」從皇太子請,改詹事院使為詹事,副詹事為少詹事,院判為丞。立尚服院,秩從二品。中書省臣言:「陜西行省言,開成路前者地震,民力重困,已免賦二年,請再免今年。」從之。甲辰,賜國王和童金二百五十兩、銀七百五十兩。立皇太子衛率府,發軍千五百人修五臺山佛寺。命有司市邸舍一區,以賜丞相赤因鐵木兒,為鈔萬九千四百錠。丁未,用丞相頭言,設尚冠、尚衣、尚舢鞶、尚沐、尚輦、尚飾六奉御,秩五品,凡四十八員,隸尚服院。甲寅,和林貧民北來者眾,以鈔十萬錠濟之,仍於大同、隆興等處糴糧以賑,就令屯田。請內侍、太醫、陰陽、樂人,毋援常選散官。以網罟給和林饑民。戊午,遣不達達思等送瓜哇使還。己未,以皇太子建佛寺,立營繕署,秩五品。



 三月庚申朔,中書省臣言:「鄃王拙忽難人戶散失,詔有司括索。臣等議,昔阿只吉括索所失人戶,成宗慮其為例,不許。今若括索,未免擾民。且諸王必多援例,乞寢其事。」從之。又莊聖皇后及諸王忽禿禿人戶散入他郡,阿都赤、脫歡降璽書,俾括索。陜西行省及真定等路言:「百姓均在國家版籍,今所遣使,輒奪軍、驛、編民等戶,非宜。」中書省臣以聞,帝曰:「彼奏誤也,卿等速追以還。」賜鎮南王老章金五百兩、銀五千兩、鈔二千錠、幣帛八百匹,也先不花、牙兒昔金各二百五十兩、銀七百五十兩、鈔二千錠。乙丑,太陰犯井。以北來貧民八十六萬八千戶,仰食於官,非久計,給鈔百五十萬錠、幣帛準鈔五十萬錠,命太師月赤察兒、太傅哈剌哈孫分給之,罷其廩給。賜諸王八亦忽金百五十兩、銀七百五十兩。丁卯,建興聖宮,給鈔五萬錠、絲二萬斤。遣使祀五岳四瀆名山大川。賜諸王八不沙金五百兩、銀五千兩。復立白云宗攝所,秩從一品,設官三員。戊寅,車駕幸上都。建佛寺於大都城南。立驥用、資武二庫,秩正五品,隸府正司。升太史院秩從二品,司天臺秩正四品。封中書右丞相、行平章政事阿沙不花為康國公。以甘肅行省右丞脫脫木兒為中書平章政事,加大司徒。賜晉王所部五百四十七人,鈔五萬二千九百六十錠;定王藥木忽兒,金千五百兩、銀三萬兩、鈔萬錠;衛士五十三人,鈔萬六百錠。乙卯,命翰林國史院纂修《順宗》、《成宗實錄》。壬午,嗣漢天師張與材來朝,加金紫光祿大夫,封留國公。



 夏四月戊戌,中書省臣言:「請依元降詔敕,勿超越授官,泛濫賜賚。」帝曰:「卿等言是。朕累有旨止之,又復蒙蔽以請,自今縱有旨,卿等其覆奏罪之。」詔以永平路鹽課賜祥哥剌吉公主,中書省臣執不可,從之。賜諸王木南子金五十兩、銀千兩、鈔千錠,賜皇太子位鷹坊鈔二十萬錠。戊戌,封三寶奴為渤國公,香山為賓國公;加鐵木迭兒右丞相,都護買住中書右丞。立皇太子位人匠總管府,秩正三品。癸卯,加授平章政事教化太子太保、太尉、平章軍國重事、魏國公。甲辰,升典瑞監為典瑞院,秩從二品。知樞密院事也兒吉尼遙授右丞相。辛亥,樞密院臣言:「諸王各用其印符乘驛,使臣旁午,驛戶困乏。宜準舊制,量其馬數,降以璽書。」奏可。乙卯,遣米楫等使蘇魯國。丙辰,高麗國王王昛言:「陛下令臣還國,復設官行征東行省事。高麗歲數不登,百姓乏食,又數百人仰食其土,則民不勝其困,且非世祖舊制。」帝曰:「先請立者以卿言,今請罷亦以卿言,其準世祖舊制,速遣使往罷之。」五月丙寅,降英德路為州,知樞密院事塔魯忽臺遙授左丞相。丁卯,御史臺臣言:「成宗朝建國子監學,迄今未成,皇太子請畢其功。」制可。己巳,管城縣大雨雹。緬國進馴象六。乙亥,知樞密院事憨剌合兒遙授左丞相。丙子,以諸王及西番僧從駕上都,途中擾民,禁之。禁白蓮社,毀其祠宇,以其人還隸民籍。御史臺臣言:「比奉旨罷不急之役,今復為各官營私宅。臣等以為俟旺兀察都行宮及大都、五臺寺畢工,然後從事為宜。」有旨:「除頭、三寶奴所居,餘悉罷之。」授右丞相塔思不花上柱國,監修國史,加左丞相乞臺普濟太子太傅。辛巳,中書省臣言:「舊制,樞密院、御史臺、宣政院得自選官,諸官府必由中書省奏聞遷調,宜申嚴告諭。」制可。癸未,濟南、般陽雨雹。甲申,立大同侍衛親軍都指揮使司,以丞相赤因鐵木兒為使,摘通惠河漕卒九百餘人隸之,漕事如故。渭源縣旱饑,給糧一月。真定、大名、廣平有蟲食桑,寧夏府水,晉寧等處蝗,東平、東昌、益都蝝。



 六月己丑,渤國公三寶奴加錄軍國重事、中書右丞相,應國公、太子詹事、平章軍國重事、大司農曲出加太子太保,左丞相脫脫加上柱國、太尉,遙授參知政事、行詹事丞大慈都加平章軍國重事。甲午,改太子位承和署為典樂司,秩正三品。丁酉,鞏昌府隴西、寧遠縣地震,雲南烏撒、烏蒙三日之中地大震者六。戊戌,大都饑,發官廩減價糶貧民,戶出印帖,委官監臨,以防不均之弊。中書省臣言:「江浙行省管內饑,賑米五十三萬五千石、鈔十五萬四千錠、面四萬斤。又,流民戶百三十三萬九百五十有奇,賑米五十三萬六千石、鈔十九萬七千錠、鹽折直為引五千。」令行省、行臺遣官臨視。內郡、江淮大饑,免今年常賦及夏稅。益都水,民饑,採草根樹皮以食,免今歲差徭,仍以本路稅課及發硃汪、利津兩倉粟賑之。封藥木忽兒為定王,駙馬阿失為昌王,並賜金印。以司徒、平章政事、領大司農李邦寧遙授左丞相。辛丑,以沒入硃清、張瑄田產隸中宮,立江浙財賦總管府、提舉司。己酉,減太常禮儀院官二十七員為八員。河南、山東大饑,有父食其子者,以兩道沒入贓鈔賑之。加乞臺普濟錄軍國重事。是月,保定、真定蝗。



 秋七月庚申,流星起自勾陳,南行,圓若車輪,微有銳,經貫索滅。敕以金銀歲入數少,自今毋問何人,以金銀為請奏及托之奏者,皆抵罪。又,各處行省、宣慰司等官,多以結托來京師,今後非奉朝命毋赴闕。雲南、湖廣、河南、四川盜賊竊發,諭軍民官用心撫治。立廣武康裏侍衛親軍都指揮使司,以中書平章政事阿沙不花為都指揮使。壬戌,皇子和世束請立總管府,領提舉司四,括河南歸德、汝寧境內瀕河荒地約六萬餘頃,歲收其租,令河南省臣高興總其事。中書省臣言:「瀕河之地,出沒無常,遇有退灘,則為之主。先是,有亦馬罕者,妄稱省委括地,蠶食其民,以有主之田俱為荒地,所至騷動。民高榮等六百人,訴於都省,追其驛券,方議其罪,遇赦獲免,今乃獻其地於皇子。且河南連歲水災,人方闕食,若從所請,設立官府,為害不細。」帝曰:「安用多言,其止勿行!」禁鷹坊於大同、隆興等處縱獵擾民。築呼鷹臺于漷州澤中,發軍千五百人助其役。旺兀察都行宮成。立中都留守司兼開寧路都總管府。丙寅,復置泰安州之新泰縣。辛卯,濟寧大水入城,詔遣官以鈔五千錠賑之。己巳,真定淫雨,水溢,入自南門,下及槁城,溺死者百七十七人,發米萬七百石賑之。辛未,立禦香局,秩正五品。壬申,太白犯左執法。香山加太子太傅。遣塔察兒等九人使諸王寬闍,



 月魯等十二人使諸王脫脫。癸酉,詔諭安南國曰:「惟我國家,以武功定天下,文德懷遠人,乃眷安南,自乃祖乃父,世修方貢,朕甚嘉之。邇者先皇帝晏駕,朕方撫軍朔方,為宗室諸王、貴戚、元勛之所推戴,以謂朕乃世祖嫡孫,裕皇正派,宗籓效順於外,臣民屬望於下,人心所共,神器有歸。朕俯徇輿情,大德十一年五月二十一日即皇帝位於上都。今遣少中大夫、禮部尚書阿里灰,朝請大夫、吏部侍郎李京,朝列大夫、兵部侍郎高復禮諭旨。尚體同仁之視,益堅事大之誠,輯寧爾邦,以稱朕意。」又以管祝思監為禮部侍郎、朵兒只為兵部侍郎使緬國。遣脫里不花等二十人使諸王合兒班答。弛上都酒禁。壬午,置皇太子司議郎,秩正五品。封乃蠻帶為壽王。癸未,樞密院臣言:「世祖時樞密臣六員,成宗時增至十三員。今署事者三十二員,乞省之。」敕罷塔思帶等一十一人。甲申,太師淇陽王月赤察兒請置王傅,中書省臣謂異姓王無置傅例,不許。乙酉,以豢虎人徹兒怯思為監察御史。是月,以左丞相塔思不花為中書右丞相,太保乞臺普濟為中書左丞相,內外大小事務並聽中書省區處,諸王、公主、駙馬、勢要人等,毋得攪擾沮壞,近侍臣員及內外諸衙門,毋得隔越聞奏。各處行省、宣慰司及在外諸衙門等官,非奉聖旨並中書省明文,毋得擅自離職,乘驛赴京,營乾私事。江南、江北水旱饑荒,已嘗遣使賑恤者,至大元年差發、官稅並行除免。



 八月戊子,大寧雨雹。丙申,御史臺臣言:「奉敕逮監察御史撒都丁赴上都。世祖、成宗迄於陛下,累有明旨,監察御史乃朝廷耳目,中外臣僚作奸犯科,有不職者,聽其糾劾,治事之際,諸人毋得與焉。邇者鞫問刑部尚書烏剌沙贓罪,蒙玉音獎諭,諸御史皆被錫賚,臺綱益振。今撒都丁被逮,同列皆懼,所系非小,乞寢是命,申明臺憲之制,諸人毋得與聞。」制可。辛丑,以中都行宮成,賞官吏有勞者,工部尚書黑馬而下並升二等,賜塔剌兒銀二百五十兩,同知察乃、通政使塔利赤、同知留守蕭珍、工部侍郎答失蠻金二百兩、銀一千四百兩,軍人金二百兩、銀八百兩,死於木石及病沒者給鈔有差。癸卯,加中書右丞、領將作院呂天麟大司徒。戊申,立中都萬億司。寧夏立河渠司,秩五品,官二員,參以二僧為之。特授頭太師。賜諸王脫歡金二百兩、銀二千五百兩、鈔二千錠,阿里不花金百兩、銀千兩、鈔千錠。己酉,大同隕霜殺禾。甲寅,李邦寧以建香殿成,賜金五十兩、銀四百五十兩。乙卯,中書省臣言:「外臺、行省及諸人應詔言事,未敢一一上煩聖聽。請集朝臣議,擇其切於事者,小則輒行,大則以聞。」從之。揚州、淮安蝗。九月丙辰,以內郡歲不登,諸部人馬之入都城者,減十之五。中書省臣言:「夏秋之間,鞏昌地震,歸德暴風雨,泰安、濟寧、真定大水,廬舍蕩析,人畜俱被其災。江浙饑荒之餘,疫癘大作,死者相枕籍。父賣其子,夫鬻其妻,哭聲震野,有不忍聞。臣等不才,猥當大任,雖欲竭盡心力,而聞見淺狹,思慮不廣,以致政事多舛,有乖陰陽之和,百姓被其災殃,願退位以避賢路。」帝曰:「災害事有由來,非爾所致,汝等但當慎其所行。」立怯憐口提舉司,秩正五品,設官四員。高麗國王王昛卒。命雪尼臺鐵木察使薛迷思干部。己未,升中政院秩從一品。辛酉,遣人使諸王察八兒、寬闍所。壬戌,太尉脫脫奏:「泉州大商合只鐵即剌進異木沉檀可構宮室者。」敕江浙行省驛致之。癸亥,萬戶也列門合散來自薛迷思乾等城,進呈太祖時所造戶口青冊,賜銀鈔幣帛有差。丙寅,蒲縣地震。癸酉,升內史府為內史院,秩正二品。乙亥,車駕至自上都。弛諸路酒禁。戊寅,泉州大商馬合馬丹的進珍異及寶帶、西域馬。庚辰,以高麗國王王章嗣高麗王。諸王禿滿進所藏太宗玉璽,封禿滿為陽翟王,賜金印。中書省臣言:「奉旨:連歲不登,從駕四衛,一衛約四百人,所給芻粟自如常例,給各部者減半。臣等議,大都去歲飼馬九萬四千匹,今請減為五萬匹,外路飼馬十一萬九千餘匹,今請減為六萬匹,自十月十五日為始。」又言:「薛迷思乾、塔剌思、塔失玄等城,三年民賦以輸縣官。今因薛尼臺鐵木察往彼,宜令以二年之賦與寬闍,給與元輸之人,以一年者上進。」並從之。癸未,太陰犯熒惑。立中都虎賁司,特授承務郎、直省舍人藏吉沙資善大夫、行泉府院使。



 冬十月庚寅,為太師頭建第,給鈔二萬錠。癸巳,蒲縣、陵縣地震。甲午,以阿沙不花知樞密院事。丁酉,以大都艱食,復糶米十萬石,減其價以賑之,以其鈔於江南和糴。罷大都榷酤,賜皇太子金千兩。辛丑,太白犯南斗。癸卯,中書省臣請以湖廣米十萬石貯於揚州,江西、江浙海漕三十萬石,內分五萬石貯硃汪、利津二倉,以濟山東饑民,從之。敕:「凡持內降文記買河間鹽及以諸王、駙馬之言至運司者,一切禁之;持內降文記不由中書者,聽運司以聞。」禁奉符、長清、泗水、章丘、沾化、利津、無棣七縣民田獵。甲辰,從帝師請,以釋教都總管朵兒只八兼領囊八地產錢物,為都總管府達魯花赤總其財賦。以西番僧教瓦班為翰林承旨。左丞相、知樞密院事鐵木兒不花加錄軍國重事。中書右丞、司徒禿忽魯,河南江北行省右丞也速,內史脫孛花,並知樞密院事。乙巳,改護國仁王寺昭應規運總管府為會福院,秩從二品。丙午,立興聖宮掌醫監,秩正三品。



 十一月己未,中書省臣言:「世祖時,省、院、臺及諸司皆有定員,後略有增者,成宗已嘗有旨並省。邇者諸司遞升,四品者三品,三品者二品,二品者一品,一司甚至二三十員,事不改舊而官日增。請依大德十年已定員數,冗濫者從各司自與減汰。衙門既升,諸吏止從舊秩出官,果應例者,自如選格。」從之。庚申,太白晝見。以軍五千人供造寺工役。增官吏俸,以至元鈔依中統鈔數給之,止其祿米,歲該四十萬石。吏員以九十月出身,如舊制。詔免紹興、慶元、臺州、建康、廣德田租,紹興被災尤甚,今歲又旱,凡佃戶止輸田主十分之四。山場、河濼、商稅,截日免之。諸路小稔,審被災者免之。乙丑,賜諸王南木忽裏金印。丁卯,中書省臣言:「今銓選、錢糧之法盡壞,廩藏空虛。中都建城,大都建寺,及為諸貴人營私第,軍民不得休息。邇者用度愈廣,每賜一人,輒至萬錠,惟陛下矜察。」又言:「銓選、錢糧,諸司乞毋干預。」帝曰:「已降制書,令諸人毋干中書之政。他日或有乘朕忽忘,持內降文記及傳旨至中書省,其執之以來,朕將加罪。」以也兒吉尼為御史大夫。己巳,以乞臺普濟為右丞相,脫脫為左丞相。既又從脫脫言,以塔思不花與乞臺普濟俱為右丞相。中書省臣言:「國用不給,請沙汰宣徽、太府、利用等院籍,定應給人數,其在上都、行省者,委官裁省。又,行泉院專以守寶貨為任,宜禁私獻寶貨者。又,天下屯田百二十餘所,由所用者多非其人,以致廢弛,除四川、甘州、應昌府、雲南為地絕遠,餘當選習農務者往,與行省、宣慰司親履其地,可興者興,可廢者廢,各具籍以聞。」並從之。詔:「開寧路及宣德、雲州工役,供億浩繁,其賦稅除前詔已免三年外,更免一年。」辛巳,罷益都諸處合剌赤等狩獵。以銀七百五十兩、鈔二千二百錠、幣帛三百匹施昊天寺,為水陸大會。癸未,皇太后造寺五臺山,摘軍六千五百人供其役。閏十一月己丑,以大都米貴,發廩十萬石,減其價以糶賑貧民。北來饑民有鬻子者,命有司為贖之。乙未,賜故中書右丞相完澤妻金五百兩、銀千五百兩。丙申,罷江南進沙糖,止富民輸粟賑饑補官。丁酉,禁江西、湖廣、汴梁私捕駕鵝。己亥,罷遼陽省進雕豹。貴赤衛受烏江縣達魯花赤獻私戶萬,令隸縣官。壬寅,乞臺普濟乞賜固安田二百餘頃,從之。乙巳,中書省臣言:「回回商人持璽書,佩虎符,乘驛馬,名求珍異,既而以一豹上獻,復邀回賜,似此甚眾。臣等議,虎符,國之信器,驛馬,使臣所需,今以畀諸商人,誠非所宜,乞一概追之。」制可。罷順德、廣平鐵冶提舉司,聽民自便,有司稅之如舊。丁未,復立汴梁路之項城縣。以杭州、紹興、建康等路歲比饑饉,今年酒課免十分之三。敕河西僧戶準先朝定制,從軍輸稅,一與民同。甲寅,答剌罕哈剌哈孫卒。



 十二月庚申,封和郎撒為隴王,賜金印。平江路民有隸謹的里部者,依舊制,差賦與民一體均當。雲南畏吾兒一千人居荊襄,雲南省臣言:「世祖有旨使歸雲南,以佐征討。」中書省臣議發還為是,從之。中都立開寧縣,降隆興為源州,升蔚州為蔚昌府。省河東宣慰司,以大同路隸中都留守司,冀寧、晉寧二路隸中書省。甲戌,以平章政事、商議中書省事、太子賓客王太亨行太子詹事,平章軍國重事、太子少詹事大慈都為太子詹事。賜御史臺官及監察御史宴服。



\end{pinyinscope}