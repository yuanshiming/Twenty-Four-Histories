\article{本紀第三十 泰定帝二}

\begin{pinyinscope}

 三年春正月丙午朔,征東行省左丞相、高麗國王王章,遣使奉方物,賀正旦。播州宣慰使楊燕裏不花招諭蠻酋黎平慶等來降。戊申,元江路總管普雙叛真切篤實處便是行了」。其學一反程朱學派傳統,在明代中葉,命雲南行省招捕之。諸王薛徹禿、晃火帖木兒來朝,賜金、銀、鈔、幣有差。壬子,封諸王寬徹不花為威順王,鎮湖廣;買奴為宣靖王,鎮益都;各賜鈔三千錠。以山東、湖廣官田賜民耕墾,人三頃,仍給牛具。諸王不賽因遣使獻西馬。征前翰林學士吳澄,不起。置都水庸田司於松江,掌江南河渠水利。己未,賜武平王帖古思不花部軍民鈔,人十五錠。以湘寧王八剌失里鎮兀魯思部。辛酉,太白犯外屏。癸亥,封朵列捏為國公,以知樞密院事撒忒迷失為嶺北行中書省平章政事。戊辰,緬國亂,其主答里也伯遣使來乞師,獻馴象方物。安南國阮叩寇思明路,命湖廣行省督兵備之。大都路屬縣饑,賑糧六萬石。恩州水,以糧賑之。



 二月丁丑,購能首告謀逆厭魅者給賞,立賞格,諭中外。庚辰,賑魯王阿兒加失里部甕吉剌貧民鈔六萬錠,命諸王魯賓為大宗正。壬午,廣西全茗州土官許文傑率諸徭以叛,寇茗盈州,殺知州事李德卿等,命湖廣行省督兵捕之。以乃馬臺知樞密院事。甲申,祭太祖、太宗、睿宗御容於翰林國史院。丁亥,中書請罷征徭,敕斡耳朵罕等班師,其鎮戍者如故。己丑,禁汴梁路釀酒。甲午,葺真定玉華宮。乙未,修佛事厭雷於崇天門。丙申,建顯宗神御殿於盧師寺,賜額曰大天源延聖寺。敕以金書西番字《藏經》。甲戌,建殊祥寺於五臺山,賜田三百頃。爪哇國遣使貢方物。庚子,以通政院使察乃為中書平章政事。甲辰,車駕幸上都。命諸王也忒古不花及中書省臣兀伯都剌、察乃、善僧、許師敬、朵朵居守。立典醫署,秩從五品,隸詹事院。歸德府屬縣河決,民饑,賑糧五萬六千石。河間、保定、真定三路饑,賑糧四月。建昌路饑,賑糶米三萬石。三月乙巳朔,帝以不雨自責,命審決重囚,遣使分祀五岳四瀆、名山大川及京城寺觀。安南國宮為龍州萬戶趙雄飛等所侵,乞諭還所掠,詔廣西道遣官究之。丙午,填星犯井宿鉞星。丁未,敕百官集議急務。中書省臣等請汰衛士,節濫賞,罷營繕,防徭寇,諸寺官署坑冶等事歸中書,並從之。壬子,翽星於司天監。癸丑,八番巖霞洞蠻來降,願歲輸布二千五百匹,設蠻夷官鎮撫之。乙卯,申禁民間金龍文織幣。丁巳,遣諸王失剌鎮北邊。戊午,詔安撫緬國,賜其主金幣。甲子,命功德使司簡歲修佛事一百二十七。丙寅,翰林承旨阿憐帖木兒、許師敬譯《帝訓》成,更名曰《皇圖大訓》,敕授皇太子。考試國子生。遣僧修佛事於臨洮、鳳翔、星吉兒宗山等處。賜諸王孛羅鐵木兒、阿剌忒納各鈔二千錠。戊辰,熒惑犯壘壁陣,填星犯井。庚午,填星、太白、歲星聚於井。辛未,泉州民阮鳳子作亂,寇陷城邑,軍民官以失討坐罪。永平、衛輝、中山、順德諸路饑,賑鈔六萬六千餘錠。寧夏、奉元、建昌諸路饑,賑糧二月。大都、河間、保定、永平、濟南、常德諸路饑,免其田租之半。



 四月丙戌,鎮安路總管岑修廣為弟修仁所攻,來告,命湖廣行省辨治之。戊戌,太白犯鬼。壬寅,熒惑犯壘壁陣。容米洞蠻田先什用等結十二洞蠻寇長陽縣,湖廣行省遣九姓長官彭忽都不花招之,田先什用等五洞降,餘發兵討之。修夏津、武城河堤三十三所,役丁萬七千五百人。五月甲辰朔,籓王怯別遣使來獻豹。乙巳,修鎮雷佛事三十一所。甘肅行省臣言:「赤斤儲粟,軍士度川遠給不便,請復徙於曲尤之地。」從之。修上都復仁門。涇州饑,禁釀酒。罷造福建歲供蔗餳。以西僧馳驛擾民,禁之。甲寅,八百媳婦蠻招南道遣其子招三聽奉方物來朝。乙卯,以帝師兄鎖南藏卜領西番三道宣慰司事,尚公主,錫王爵。給壽寧公主印,仍賜田百頃、鈔三萬錠。甲子,中書會歲鈔出納之數,請節用以補不足,從之。監察御史劾宣撫使朵兒只班,學士李塔剌海、劉紹祖庸鄙不勝任。中書議:「三人皆勛舊子孫,罪無實狀,乞復其職,仍敕憲臺勿以空言妄劾。」從之。丁卯,岑世興及鎮安路岑修文合山獠、角蠻六萬餘人為寇,命湖廣、雲南行省招諭之。遣指揮使兀都蠻鐫西番咒語於居庸關崖石。庚午,乞住招諭永明縣五洞徭來降。河西加木籠贍部來降,以答兒麻班藏卜領卜剌麻沙朔部,公哥班領古籠羅烏公遠宗蘭宗孛兒間沙加堅部,唆南監藏卜領蘭宗古卜剌卜吉里昔吉林亦木石威石部,朵兒只本剌領籠答吃列八里阿卜魯答思阿答藏部。雄州饑,太平、興化屬縣水,並賑之。廬州、鬱林州及洪澤屯田旱,揚州路屬縣財賦官田水,並免其租。六月癸酉朔,賜籓王怯別七寶束帶。以禿哈帖木兒為四川行省平章政事;請終母喪,從之。癸未,播州蠻黎平愛復叛,合謝烏窮為寇,宣撫使楊燕禮不花招平愛出降。烏窮不附,命湖廣行省討之。丁亥,命湘寧王八剌失裡出鎮阿難答之地。戊子,諸王脫脫等來朝,賜金、銀、鈔、幣有差。乙未,命梁王王禪及諸王徹徹禿鎮撫北軍,賜王禪鈔五千錠、幣帛各二百匹。丁酉,遣道士吳全節修醮事於龍虎、三茆、閣皁三山。戊戌,遣使祀解州鹽池神。中書省臣言:「比郡縣旱蝗,由臣等不能調燮,故災異降戒。今當恐懼儆省,力行善政,亦冀陛下敬慎修德,憫恤生民。」帝嘉納之。賑昌王八剌失里部鈔四萬錠,賜吳王潑皮鈔萬錠。己亥,納皇姊壽寧公主女撒答八剌於中宮。道州路櫟所源徭為寇,命乞住督兵捕之。奉元、鞏昌屬縣大雨雹,峽州旱,東平屬縣蝗,大同屬縣大水,萊蕪等處冶戶饑,賑鈔三萬錠。光州水,中山安喜縣雨雹傷稼,大昌屯河決,大寧、廬州、德安、梧州、中慶諸路屬縣水旱,並蠲其租。



 秋七月甲辰,車駕發上都,禁車騎踐民禾。遼王脫脫請復太母月也倫宮守兵及女直屯戶,不允。增給太祖四大斡耳朵歲賜銀二百錠、鈔八千錠。遣使祀海神天妃。造豢豹氈車三十輛。乙巳,怯憐口屯田霜,賑糧二月。丙午,享太廟。丁未,紹慶酉陽寨冉世昌及何惹洞蠻為寇。詔行宮駝馬及宗戚將校駐冬北邊者,毋輒至京師。辛亥,封阿都赤為綏寧王,賜鈔四千錠,給金印。壬子,皇后受牙蠻答哥戒於水晶殿。甲寅,幸大乾元寺,敕鑄五方佛銅像。乙卯,詔翰林侍講學士阿魯威、直學士燕赤譯《世祖聖訓》,以備經筵進講。戊午,諸王不賽因獻駝馬。遣日本僧瑞興等四十人還國。作別殿於潛邸。敕:「入粟拜官者,準致仕銓格。」己未,禁諸部王妃入京告饑。以月魯帖木兒嗣齊王,給金印。八百媳婦蠻招南通遣使來獻馴象方物。乙丑,發兵修野狐、色澤、桑乾三嶺道。戊辰,太白經天。己巳,大理土官你囊來獻方物。庚申,廣西宣慰副使王瑞請益戍兵,及以土民屯田備蠻,仍置南寧安撫司。河決鄭州、陽武縣,漂民萬六千五百餘家,賑之。永平、大都諸屬縣水,大風,雨雹。龍興、辰州二路火,大名、永平、奉元諸路屬縣旱,汴梁路水,大名、順德、衛輝、淮安等路,睢、趙、涿、霸等州及諸位屯田蝗,大同渾源河溢,檀、順等州兩河決,溫榆水溢,賑永平、奉元鈔七萬錠。賑糶濠州饑民麥三萬九千餘石。命瘞京城外棄骸,死狀不白者,有司究之。



 八月甲戌,兀伯都剌、許師敬並以災變饑歉乞解政柄,不允。乙亥,遣乃馬臺簡閱邊兵,賜鈔千錠。大天源延聖寺神御殿成。戊寅,修澄清石閘。甲申,享太廟。長春宮道士藍道元以罪被黜。詔:「道士有妻者,悉給徭役。」遷黃羊坡民二百五十戶於韃靼部。寧遠州洞蠻刁用為寇,命雲南行省備之。丁亥,遣梁王王禪整飭斡耳朵思邊事。辛卯,雲南行省丞相亦兒吉鷿、廉訪副使散只兀臺,以使酒相詆,狀聞,詔兩釋之。甲午,以災變罷獵。賑河南探馬赤軍,籍其餘丁。罷行宣政院及功德使司,免武備寺逋負兵器。丁酉,籓王不賽因遣使獻玉及獨峰駝。是夜,太白犯軒轅御女。以星變,下詔恤民。辛丑,次中都,畋於汪火察禿之地。賜太師按攤出鈔二千八百錠。鹿頂殿成。罷甘肅札渾倉,徙其軍儲於汪古剌倉。戶部尚書郭良坐贓免。作天妃宮於海津鎮。西番土官撒加布來獻方物,海寇黎三來附。詔諭廉州蜒戶使復業。鹽官州大風,海溢,壞堤防三十餘里,遣使祭海神,不止,徙居民千二百五十家。大都昌平大風,壞民居九百家。龍慶路雨雹一尺,大風損稼。真定蠡州、奉元蒲城等縣及無為州諸處水,河中府、永平、建昌印都、中慶、太平諸路及廣西兩江饑,並發粟賑之。揚州崇明州大風雨,海水溢,溺死者給棺斂之。杭州火,賑糧一月。九月丁未,增置上都留守判官一員,兼推官。辛亥,命帝師還京,修灑凈佛事於大明、興聖、隆福三宮。丁巳,弛大都、上都、興和酒禁。庚申,車駕至大都。壬戌,以察乃領度支事。癸亥,太白犯太微垣右執法。賜大車裏新附蠻官七十五人裘帽靴服。戊辰,命歡赤等使於諸王怯別、月思別、不賽因三部。賑潛邸貧民鈔二十萬錠。湖廣行省太平路總管郭扶、雲南行省威楚路禿剌寨長哀培、景東寨長阿只弄男阿吾、大阿哀寨主弟你刀、木羅寨長哀卜利、茫施路土官阿利、鎮康路土官泥囊弟陀金客、木粘路土官丘羅、大車裏昭哀侄哀用、孟隆甸土官吾仲,並奉方物來獻。以昭哀地置木朵路一、木來州一、甸三,以吾仲地置孟隆路一、甸一,以哀培地置甸一,並降金符、銅印,仍賜幣帛、鞍勒有差。中書省臣言:「今國用不繼,陛下當法世祖之勤儉以為永圖。臣等在職,茍有濫承恩賞者,必當回奏。」帝嘉納之。揚州、寧國、建德諸屬縣水,南恩州旱,民饑,並賑之。汾州平遙縣汾水溢,廬州、懷慶二路蝗。



 冬十月辛未朔,發卒四千治通州道,給鈔千六百錠。甲戌,紐澤升右御史大夫。庚辰,享太廟。奉安顯宗御容於大天源延聖寺。辛巳,太白犯進賢。天壽節,遣道士祠衛輝太一萬壽宮。壬午,帝師以疾還撒思加之地,賜金、銀、鈔、幣萬計,敕中書省遣官從行,備供億。癸酉,河水溢,汴梁路樂利堤壞,役丁夫六萬四千人築之。京師饑,發粟八十萬石,減價糶之。賜大天源延聖寺鈔二萬錠,吉安、臨江二路田千頃。中書省臣言:「養給軍民,必藉地利。世祖建大宣文弘教等寺,賜永業,當時已號虛費,而成宗復構天壽萬寧寺,較之世祖,用增倍半。若武宗之崇恩福元、仁宗之承華普慶,租榷所入,益又甚焉。英宗鑿山開寺,損兵傷農,而卒無益。夫土地祖宗所有,子孫當共惜之。臣恐茲後藉為口實,妄興工役,徼福利以逞私欲,惟陛下察之。」帝嘉納焉。庚子,陜西行臺中丞姚煒請集世祖嘉言善行,以時省覽,從之。沈陽、遼陽、大寧等路及金、復州水,民饑,賑鈔五萬錠。懷慶修武縣旱,免其租。寧夏路萬戶府、慶遠安撫司饑,並賑之。弛寧夏路酒禁。宣撫使馬合某、李讓劾浙西廉訪使完者不花受賂,簿對不服,詔遣刑部郎中唆住鞫其侵辱使者,笞之。籓王不賽因遣使來獻虎。



 十一月癸卯,中書省臣言:「西僧每假元辰疏釋重囚,有乖政典,請罷之。」有旨:「自今當釋者,敕宗正府審覆。」乙巳,梁王王禪往北邊,賜鈔二千錠。己酉,作鹿頂棕樓。辛亥,追復前平章政事李孟官。賜湘寧王八剌失里鈔三千錠。諸王不賽因遣使來獻馬。乙卯,太白犯鍵閉。廣西透江團徭為寇,宣慰使買奴諭降之。扶靈、青溪、櫟頭等源蠻為寇,湖南道宣慰司遣使諭降之。戊午,造中統、至元鈔各十萬錠。封諸王鐵木兒不花為鎮南王,鎮揚州。辛酉,加御史大夫紐澤開府儀同三司。加封廬陵江神曰顯應。弛成都酒禁。播州蠻宋王保來降。己巳,徙上都清寧殿於伯亦兒行宮。弛永平路山澤之禁。階州土蕃為寇,武靖王遣臨洮路元帥盞盞諭降之。廣寧路屬縣霖雨傷稼,賑鈔三萬錠。沔陽府旱,免其稅。永平路大水,免其租,仍賑糧四月。汴梁、建康、太平、池州諸路及甘肅亦集乃路饑,並賑之。錦州水溢,壞田千頃,漂死者百人,人給鈔一錠。崇明州海溢,漂民舍五百家,賑糧一月,給死者鈔二十貫。



 十二月丁丑,諸王月思別獻文豹,賜金、銀、鈔、幣有差。御史哈剌那海請擇正人傅太子,帝嘉納之。壬午,御史賈垕請祔武宗皇后於太廟,不報。敕以來年元夕構燈山於內廷,御史趙師魯以水旱請罷其事,從之。甲申,師魯又請親祀郊廟,帝嘉納之。丙戌,以回回陰陽家言天變,給鈔二千錠,施有道行者及乞人,系囚,以禳之。丁亥,寧夏路地震,有聲如雷,連震者四。庚寅,赦天下。召江浙行省右丞趙簡為集賢大學士,領經筵事。壬辰,賜梁王王禪宴器金銀。以皇子小薛夜啼,賜高年鈔。癸巳,作鹿頂殿。己亥,命帝師修佛事,釋重囚三人。置大承華普慶寺總管府,罷規運提點所。御史言:「比年營繕,以衛軍供役,廢武事不講。請遵世祖舊制,教習五衛親軍,以備扈從。」不報。湖廣屯戍千戶只乾不花招諭扶靈洞蠻劉季等來降。保定路饑,賑米八萬一千五百石。懷慶路饑,賑鈔四萬錠。亳州河溢,漂民舍八百餘家,壞田二千三百頃,免其租。廣西靜江、象州諸路及遼陽路饑,並賑之。大寧路大水,壞田五千五百頃,漂民舍八百餘家,溺死者人給鈔一錠。



 四年春正月甲辰,諸王買奴來朝,賜金一錠、銀十錠、鈔二千錠、幣帛各四十匹。乙巳,御史臺臣請親祀郊廟,帝曰:「朕遵世祖舊制,其命大臣攝之。」己酉,太白犯牛。庚戌,置紹慶路石門十寨巡檢司,御史辛鈞言:「西商鬻寶,動以數十萬錠,今水旱民貧,請節其費。」不報。壬子,以中政院金銀鐵冶歸中書。靖安王闊不花出鎮陜西,賜鈔二千錠。癸丑,賜諸王阿剌忒納失裡等鈔六千錠。甲寅,鷹師脫脫病,賜鈔千錠。戊午,命市珠寶首飾。庚申,皇子允丹藏卜受佛戒於智泉寺。鹽官州海水溢,壞捍海堤二千餘步。甲子,武龍洞蠻寇武緣縣諸堡。丁卯,燕南廉訪司請立真定常平倉,不報。浚會通河,築漷州護倉堤,役丁夫三萬人。初置雲南行省檢校官。遼陽行省諸郡饑,賑鈔十八萬錠。彰德、淮安、揚州諸路饑,並賑之。大寧路水,給溺死者人鈔一錠。二月辛未,祀先農。甲戌,祭太祖、太宗、睿宗御容於大承華普慶寺,以翰林院官執事。乙亥,親王也先鐵木兒出鎮北邊,賜金一錠、銀五錠、鈔五百錠、幣帛各十匹。丙子,命亦烈赤領仁宗神御殿事,大司徒亦憐真乞剌思為大承華普慶寺總管府達魯花赤,仍大司徒。壬午,狩于漷州。諸王火沙、河榮、答里出鎮北邊,賜金、銀、鈔、幣有差。帝師參馬亦思吉思卜長出亦思宅卜卒,命塔失鐵木兒、紐澤監修佛事。丙戌,詔同僉樞密院事燕帖木兒教閱諸衛軍。戊子,進襲封衍聖公孔思晦階嘉議大夫。以馬思忽為雲南行省平章政事,提調烏蒙屯田。庚寅,八百媳婦蠻酋招南通來獻方物。辛卯,白虹貫日,以尚供總管府及雲需總管府隸上都留守司。奉元、廬州、淮安諸路及白登部饑,賑糧有差。永平路饑,賑鈔三萬錠、糧二月。



 三月辛丑,皇子允丹藏卜出鎮北邊。以那海赤為惠國公,商議內史府事。癸卯,和寧地震,有聲如雷。丙午,廷試進士阿察赤、李黼等八十五人,賜進士及第、出身有差。命西僧作止風佛事。潮州路判官錢珍,挑推官梁楫妻劉氏,不從,誣楫下獄殺之。事覺,珍飲藥死,詔戮尸傳首。海北廉訪副使劉安仁,坐受珍賂除名。辛亥,諸王槊思班、不賽亦等,以文豹、西馬、佩刀、珠寶等物來獻,賜金、鈔萬計。庚申,遣使往江南求奇花異果。辛酉,以太傅朵臺為太師,太保禿忽魯為太傅,也可扎魯忽赤伯達沙為太保。敕前太師伯忽與議大事,食其俸終身。召翰林學士承旨蔡國公張珪、集賢大學士廉恂、太子賓客王毅,悉復舊職,陜西行臺中丞敬儼為集賢大學士,並商議中書省事,珪仍預經筵事。賜諸王火沙部鈔四千錠。郡王朵來、兀魯兀等部畜牧災,賑鈔三萬五千錠。中書省臣請酬哈散等累朝售寶價鈔十萬二千錠,從之。壬戌,車駕幸上都。復設武備寺同判六員。命親王八剌失裡出鎮察罕腦兒。封寬徹為國公,以阿散火者知樞密院事。渾河決,發軍民萬人塞之。丁卯,熒惑犯井。復置衛候直都指揮使司,秩正四品。諸王不賽因遣使獻文豹、獅子,賜鈔八千錠。大寧、廣平二路屬縣饑,賑鈔二萬八千錠。河南行省諸州縣及建康屬縣饑,賑糧有差。



 夏四月辛未,盜入太廟,竊武宗金主及祭器。大理鹿甸酋阿你為寇。壬甲,作武宗主。甲戌,作棕毛鹿頂樓。己卯,道州永明縣徭為寇。癸未,鹽官州海水溢,侵地十九里,命都水少監張仲仁及行省官發工匠二萬餘人,以竹落木柵實石塞之,不止。癸巳,高州徭寇電白縣,千戶張額力戰,死之,邑人立祠,敕賜額曰「旌義」。甲午,以西僧公哥列思巴沖納思監藏班藏卜為帝師,賜玉印,仍詔諭天下僧。乙未,以武備寺卿阿昔兒答剌罕為御史大夫。翽星於回回司天臺。湖廣徭寇全州、義寧屬縣,命守將捕之。河南、奉元二路及通、順、檀、薊等州,漁陽、寶坻、香河等縣饑,賜糧兩月。河間、揚州、建康、太平、衢州、常州諸路屬縣及雲南烏撒、武定二路饑,賜糧、鈔有差。永平路饑,免其租,仍賑糧兩月。



 五月辛丑,太尉丑驢卒。癸卯,以鹽官州海溢,命天師張嗣成修醮禳之。乙巳,作成宗神御殿於天壽萬寧寺。己未,占城國遣使貢方物。甲子,以典守宗廟不嚴,罷太常禮儀院官。丁卯,修佛事於賀蘭山及諸行宮。罷諸王分地州縣長官世襲,俾如常調官,以三載為考。元江路總管普雙坐贓免,遂結蠻兵作亂,敕復其舊職。德慶路徭來降,歸所掠男女,悉給其親。河南、江陵屬縣饑,賑糧有差。汴梁屬縣饑,免其租。常州、淮安二路,寧海州大雨雹,睢州河溢,大都、南陽、汝寧、廬州等路屬縣旱蝗,衛輝路大風九日,木盡偃。河南路洛陽縣有蝗可五畝,群烏食之既,數日蝗再集,又食之。六月辛未,翰林侍講學士阿魯威、直學士燕赤等進講,仍命譯《資治通鑒》以進。參知政事史惟良請解職歸養,不允。丁丑,倒剌沙等以災變乞罷,不允。罷兩都營繕工役,錄諸郡系囚。己卯,永興屯被災,免其租。辛巳,造象輿六乘。癸未,遣察乃、伯顏赴大都銓選。甲申,廣西花角蠻為寇,命所部討之。乙未,紹慶路四洞酋阿者等降,並命為蠻夷長官,仍設巡檢司以撫之。發義倉粟,賑鹽官州民。廬州路饑,賑糧七萬九千石。鎮江、興國二路饑,賑糶有差。中山府雨雹,汴梁路河決,汝寧府旱,大都、河間、濟南、大名、峽州屬縣蝗。



 秋七月丁酉,元江路普雙復叛。戊戌,諸王燕只吉臺襲位,遣使來朝。己亥,八兒忽部晃忽來獻方物。御史臺臣言,內郡、江南,旱、蝗薦至,非國細故,丞相塔失帖木兒、倒剌沙,參知政事不花、史惟良,參議買奴,並乞解職。有旨:「毋多辭,朕當自儆,卿等亦宜各欽厥職。」修大明殿。占城國獻馴象二。建橫渠書院於郿縣,祠宋儒張載。辛丑,賜齊王月月魯帖木兒鈔二萬錠。甲辰,播州蠻謝烏窮來獻方物。丙午,享太廟。丁未,敕:「經筵講讀官,非有代不得去職。」詔諭宗正府,決獄遵世祖舊制。戊戌,遣翰林侍讀學士阿魯威還大都,譯《世祖聖訓》。壬子,賜諸王火兒灰、月魯帖木兒、八剌失里及駙馬買住罕鈔一萬五千錠,金、銀、幣、帛有差。甲寅,遣使市旄牛於西域。丁巳,給齊王月魯帖木兒印。伯顏察兒、兀伯都剌以疾乞解政,優詔諭之。戊午,謀粘路土官賽丘羅招諭八百媳婦蠻招三斤來降,銀沙羅土官散怯遮殺賽丘羅,敕云南王遣人諭之。癸亥,賜壽寧公主鈔五千錠。岐王鎖南管卜訴荊王也速也不干侵其分地,命甘肅行省閱籍歸之。乙丑,周王和世束及諸王燕只哥臺等來貢,賜金、銀、鈔、幣有差。遣使祀海神天妃。丙寅,籍僧、道有妻者為民。塞保安鎮渠,役民丁六千人。是月,籍田蝗,雲州黑河水溢。衢州大雨水,發廩賑饑者,給漂死者棺。延安屬縣旱,免其租稅。遼陽遼河、老撒加河溢,右衛率部饑,並賑之。



 八月戊辰,給累朝斡耳朵鈔有差。癸酉,給別乞烈失寧國公印。度支監卿孛羅請辭職奉母,不允。賜皇后乳母鈔千七百錠。滹沱河水溢,發丁浚治河以殺其勢。奉元路治中單鵠言,令民採捕珍禽異獸不便,請罷之,敕:「應獵者其捕以進。」乙亥,賜公主不答昔你媵戶鈔四千錠。苗人祭伯秧寇李陀寨,命湖廣行省捕之。庚辰,運粟十萬石貯瀕河諸倉,備內郡饑。田州洞徭為寇,遣湖廣行省捕之。癸未,賜營王也先帖木兒鈔三千錠。乙酉,伯亦斡耳朵作欽明殿成。壬辰,御史李昌言:「河南行省平章政事童童,世官河南,大為奸利,請徙他鎮。」不報。癸巳,謚武宗皇后曰宣慈惠聖,英宗皇后曰莊靜懿聖,升祔太廟。發衛軍八千,修白浮、甕山河堤。是月,揚州路崇明州、海門縣海水溢,汴梁路扶溝、蘭陽縣河溢,沒民田廬,並賑之。建德、杭州、衢州屬縣水,真定、晉寧、延安、河南等路屯田旱,大都、河間、奉元、懷慶等路蝗,鞏昌府通渭縣山崩。碉門地震,有聲如雷,晝晦。天全道山崩,飛石斃人。鳳翔、興元、成都、峽州、江陵同日地震。九月丙申朔,日有食之。阿察赤的斤獻木綿大行帳。敕:「國子監仍舊制歲貢生員業成者六人。」禁僧道買民田,違者坐罪,沒其直。壬寅,寧夏路地震。壬子,太白犯房。甲寅,湖廣土官宋王保來獻方物。壬戌,遣歡赤等使諸王怯別等部。甲子,御史言:「廣海古流放之地,請以職官贓污者處之,以示懲戒。」從之。保定、真定二路饑,賑糧三萬石、鈔萬五千錠。閏月丁卯,賜諸王徹徹禿、渾都帖木兒鈔各五千錠。己巳,太白經天。車駕至大都。壬申,以災變赦天下。廣西兩江徭為寇,命所部捕之。甲戌,命祀天地,享太廟,致祭五岳四瀆、名山大川。甲午,八百媳婦蠻請官守,置蒙慶宣慰司都元帥府及木安、孟傑二府於其地,以同知烏撒宣慰司事你出公、土官招南通並為宣慰司都元帥,招諭人米德為同知宣慰司事副元帥,南通之子招三斤知木安府,侄混盆知孟傑府,仍賜鈔、幣各有差。建昌、贛州、惠州諸路饑,賑米四萬四千石。土番階州饑,賑鈔千五百錠。奉元、慶遠、延安諸路饑,賑糶有差。



 冬十月丙申,享太廟。戊戌,諸王脫別帖木兒、哈兒蠻等獻玉及蒲萄酒,賜鈔六千錠。己亥,御史德住請擇東宮官。癸卯,命帝師作佛事於大天源延聖寺。甲辰,改封建德路烏龍山神曰忠顯靈澤普佑孚惠王。乙巳,晝有流星。己酉,以治書侍御史王士熙為參知政事。辛亥,監察御史亦怯列臺卜答言,都水庸田使司擾民,請罷之。癸丑,江浙行省左丞相脫歡答剌罕、平章政事高昉,以海溢病民,請解職,不允。雲南沙木寨土官馬愚等來朝。丁巳,以御史中丞趙世延為中書右丞,以中書參議傅巖起為吏部尚書。御史韓鏞言:「尚書三品秩,巖起由吏累官四品,於法不得升。」制可。安南遣使來獻方物。戊午,辰星犯東咸。監察御史馮思忠請命太常纂修累朝禮儀。壬戌,開南州土官阿只弄率蠻兵為寇,雲南行省招捕之。增置肅州、沙州、亦集乃三路推官。大都路諸州縣霖雨,水溢,壞民田廬,賑糧二十四萬九千石。衛輝獲嘉等縣饑,賑鈔六千錠,仍蠲丁地稅。龍興路屬縣旱,免其租。大名、河間二路屬縣饑,並賑之。



 十一月庚午,禁晉寧路釀酒。減價糶京倉米十萬石,以賑貧民。以思州土官田仁為思州宣慰使,召雲南王帖木兒不花赴上都。癸酉,太白犯壘壁陣。乙亥,熒惑犯天江。丙子,賜公主不答昔你鈔千錠。平樂府徭為寇,湖廣行省督兵捕之。辛卯,以降蠻謝烏窮為蠻夷官。雲南蒲蠻來附,置順寧府、寶通州、慶甸縣。緬國主答里必牙請復立行省於迷郎崇城,不允。孛斯來附。給伯亦斡耳朵駝、牛。以歲饑,開內郡山澤之禁。永平路水旱,民饑,蠲其賦三年。諸王塔思不花部衛士饑,賑糧千石。冀寧路陽曲縣地震。



 十二月庚子,發米三十萬石,賑京師饑。絳州太平縣趙氏婦一產三子。定捕盜令,限內不獲者,償其贓。辛丑,敕塔失鐵木兒、倒剌沙領內史府四斡耳朵事。癸卯,安南遣使來貢方物。甲辰,梧州徭為寇,湖廣行省督兵捕之。戊申,諸王孛羅遣使貢岡砂,賜鈔二千錠。癸丑,命趙世延及中書參議韓讓、左司郎中姚庸提調國子監。乙卯,爪哇遣使獻金文豹、白猴、白鸚鵡各一。蔡國公張珪卒。植萬歲山花木八百七十本。丙辰,賜諸王孛羅帖木兒等鈔四千錠。己未,歲星退犯太微西垣上將。靜江路徭兵為寇,湖廣行省督兵捕之。右江諸寨土官岑世忠等來獻方物。大都、保定、真定、東平、濟南、懷慶諸路旱,免田租之半。河南、河間、延安、鳳翔屬縣饑,並賑之。是歲,汴梁、延安、汝寧、峽州旱,濟南、衛輝、濟寧、南陽八路屬縣蝗。汴梁諸屬縣霖雨,河決。揚州路通州、崇明州大風,海溢。



 致和元年春正月乙丑朔,高麗王遣使來朝賀,獻方物。甲戌,享太廟。命繪《蠶麥圖》。乙亥,詔諭百司:「凡不赴任及擅離職者,奪其官;避差遣者,笞之。」御史鄒惟亨言:「時享太廟,三獻官舊皆勛戚大臣,而近以戶部尚書為亞獻,人既疏遠,禮難嚴肅。請仍舊制,以省、臺、樞密、宿衛重臣為之。」丁丑,頒《農桑舊制》十四條於天下,仍詔勵有司以察勤惰。己卯,帝將畋柳林,御史王獻等以歲饑諫,帝曰:「其禁衛士毋擾民家,命御史二人巡察之。諸王星吉班部饑,賑鈔萬錠、米五千石。占城遣使來貢方物,且言為交趾所侵,詔諭解之。禁僧、道匿商稅,給宗仁衛蒙古子女糧六月。辛巳,靜江徭寇靈川、臨桂二縣,命廣西招捕之。甲申,遣使祀海神天妃。戊子,詔優護爪哇國主札牙納哥,仍賜衣物弓矢。罷河南鐵冶提舉司,歸有司。命帝師修佛事於禁中。免陜西撈鹽一年,發卒修京城,罷益都諸屬縣食鹽。加封幸淵龍神福應昭惠公。河間、真定、順德諸路饑,賑鈔萬一千錠。大都路東安州、大名路白馬縣饑,並賑之。



 二月癸卯,弛汴梁路酒禁。乙卯,牙即遣使藏古來貢方物。庚申,詔天下改元致和。免河南自實田糧一年,被災州郡稅糧一年,流民復業者差稅三年,疑獄系三歲不決者咸釋之。賜遼王脫脫鈔五千錠,梁王王禪鈔二千錠。壬戌,太白晝見。癸亥,解州鹽池黑龍堰壞,調番休鹽丁修之。陜西諸路饑,賑鈔五萬錠。河間、汴梁二路屬縣及開城、乾州蒙古軍饑,並賑之。



 三月庚午,阿速衛兵出戍者千人,人給鈔四十錠;貧乏者六千一百人,人給米五石。雲南安隆寨土官岑世忠與其兄世興相攻,籍其民三萬二千戶來附,歲輸布三千匹,請立宣撫司以總之,不允。置州一,以世興知州事,置縣二,聽世忠舉人用之,仍諭其兄弟共處。立萬戶府二,領征西紅胖襖軍。塔失帖木兒、倒剌沙言:「災異未弭,由官吏以罪黜罷者怨誹所致,請量才敘用。」從之。辛未,大天源延聖寺顯宗神御殿成,置總管府以司財賦。壬申,雨霾。甲戌,雅濟國遣使獻方物。乙卯,帝御興聖殿受無量壽佛戒於帝師。庚辰,命僧千人修佛事於鎮國寺。辛巳,賜壽寧公主鹽價鈔萬引。甲申,遣戶部尚書李家奴往鹽官祀海神,仍集議修海岸。丙戌,詔帝師命僧修佛事於鹽官州,仍造浮屠二百一十六,以厭海溢。戊子,車駕幸上都。己丑,以趙世延知經筵事,趙簡預經筵事,阿魯威同知經筵事,曹元用、吳秉道、虞集、段輔、馬祖常、燕赤、孛術魯翀並兼經筵官。雲南土官撒加布降,奉方物來獻,置州一,以撒加布知州事,隸羅羅宣慰司,徵其租賦。壬辰,太平路當塗縣楊氏婦一產三子。晉寧、衛輝二路及泰安州饑,賑鈔四萬八千三百錠。冀寧路平定州饑,賑糶米三萬石。陜西、四川及河南府等處饑,並賑之。



 夏四月丙申,欽州徭黃焱等為寇,命湖廣行省備之。己亥,塔失帖木兒、倒剌沙請凡蒙古、色目人效漢法丁憂者除其名,從之。壬寅,李家奴以作石囤捍海議聞。己酉,御史楊倬等以民饑,請分僧道儲粟濟之,不報。甲寅,改封蒙山神曰嘉惠昭應王,鹽池神曰靈富公,洞庭廟神曰忠惠順利靈濟昭佑王,唐柳州刺史柳宗元曰文惠昭靈公。戊午,禁偽造金銀器皿。大都、東昌、大寧、汴梁、懷慶之屬州縣饑,發粟賑之。保定、冠州、德州、般陽、彰德、濟南屬州縣饑,發鈔賑之。是月,靈州、浚州大雨雹。薊州及岐山、石城二縣蝗。廣寧路大水,崇明州大風,海溢。



 五月甲子,遣官分護流民還鄉,仍禁聚至千人者杖一百。丙寅,廣西普寧縣僧陳慶安作亂,僭建國,改元。己巳,八百媳婦蠻遣子哀招獻馴象。癸酉,籍在京流民廢疾者,給糧遣還。大理怒江甸土官阿哀你寇樂辰諸寨,命雲南行省督兵捕之。庚辰,有流星大如缶,其光燭地。甲申,安南國及八洞蠻酋遣使獻方物。戊子,以嶺北行省平章政事塔失帖木兒為中書平章政事。是月,燕南、山東東道及奉元、大同、河間、河南、東平、濮州等處饑,賑鈔十四萬三千餘錠。峽州屬縣饑,賑糶糧五千石。冀寧、廣平、真定諸路屬縣大雨雹,汝寧府潁州、衛輝路汲縣蝗,涇州靈臺縣旱。六月,高麗世子完者禿訴取其印,遣平章政事買閭往諭高麗王,俾還之。丙午,遣使祀世祖神御殿。是月,諸王喃答失、徹徹禿、火沙、乃馬臺諸郡風雪斃畜牧,士卒饑,賑糧五萬石、鈔四十萬錠。奉元、延安二路饑,賑鈔四千八百九十錠。彰德屬縣大雨雹,南寧、開元、永平諸路水,江陵路屬縣旱,河南德安屯蠖食桑。



 秋七月辛酉朔,寧夏地震。庚午,帝崩,壽三十六,葬起輦穀。己卯,大寧路地震。癸未,修佛事於欽明殿。乙酉,皇后、皇太子降旨諭安百姓。丙戌,太白犯軒轅大星。



 九月,倒剌沙立皇太子為皇帝,改元天順,詔天下。



 泰定之世,災異數見,君臣之間,亦未見其引咎責躬之實,然能知守祖宗之法以行,天下無事,號稱治平,茲其所以為足稱也。



\end{pinyinscope}