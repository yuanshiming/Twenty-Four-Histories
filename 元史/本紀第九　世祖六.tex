\article{本紀第九 世祖六}

\begin{pinyinscope}

 十三年春正月丁卯朔,克潭州,宋安撫使李芾盡室自焚死。阿里海牙分遣官屬招徠未附者,旬日間,湖南州郡相繼悉降安豐場(今江蘇東臺)人。出身鹽戶,早年為灶叮自修典,得府一、州六、軍二、縣四十,戶五十六萬一千一百一十二,口百五十三萬七千七百四十。伯顏軍次嘉興府,安撫劉漢傑以城降。董文炳軍次乍浦,宋統制官劉英以本軍降。辛未,董文炳軍至海鹽,知縣事王與賢及澉浦鎮統制胡全、福建路馬步軍總沈世隆皆降。壬申,改都統領司為通政院,以兀良合帶等領之。立回易庫於諸路,凡十有一,掌市易幣帛諸物。敕大都路總管府和顧和買,權豪與民均輸。癸酉,宋相陳宜中遣軍器監劉庭瑞齎宋主稱籓表章,詣軍前稟議,又致宜中等書於伯顏,伯顏以書答之。乙亥,詔諭四川制置使趙定應來朝。徙大都等路獵戶戍大洪山之東,符寶郎董文忠請貧病者勿徙,從之。宋復遣監察御史劉岊齎宋主稱籓表至軍前,且致書伯顏,為宗社生靈請命。丙子,賞合兒魯帶所部將士徵建都功銀鈔錦衣。丁丑,宋遣都統洪模齎陳宜中、吳堅等書,請俟宗長福王至,同詣軍前。戊寅,伯顏以軍出嘉興府,留萬戶忽都虎、千戶王禿林察戍之。劉漢傑仍為其府安撫使。辛巳,命雲南行省給建都屯軍弓矢。軍次崇德縣,宋遣侍郎劉庭瑞、都統洪模來迓。行都元帥府宋都帶言:「江西隆興、建昌、撫州等郡雖附,而閩、廣諸州尚阻兵,乞增兵進討。」敕以襄漢軍四千俾將之。壬午,軍次長安鎮,董文炳以兵來會。宋陳宜中、吳堅等違約不至。癸未,軍次臨平鎮。甲申,次皋亭山,阿剌罕以兵來會。宋主遣其保康軍承宣使尹甫、和州防禦使吉甫等,齎傳國玉璽及降表詣軍前。其辭曰:「大宋國主鳷,謹百拜奉表於大元仁明神武皇帝陛下:臣昨嘗遣侍郎柳岳、正言洪雷震捧表馳詣闕庭,敬伸卑悃,伏計已徹聖聽。臣眇焉幼沖,遭家多難,權奸似道,背盟誤國,臣不及知,至於興師問罪,宗社阽危,生靈可念。臣與太皇日夕憂懼,非不欲遷闢以求兩全,實以百萬生民之命寄臣之身,今天命有歸,臣將焉往?惟是世傳之鎮寶,不敢愛惜,謹奉太皇命戒,痛自貶損,削帝號,以兩浙、福建、江東西、湖南北、二廣、四川見在州郡,謹悉奉上聖朝,為宗社生靈祈哀請命。欲望聖慈垂哀,祖母太后耄及,臥病數載,臣煢煢在疚,情有足矜,不忍臣祖宗三百年宗社遽至殞絕,曲賜裁處,特與存全,大元皇帝再生之德,則趙氏子孫世世有賴,不敢弭忘。臣無任感天望聖,激切屏營之至。」伯顏既受降表、玉璽,復遣囊加帶以趙尹甫、賈餘慶等還臨安,召宰相出議降事。乙酉,師次臨安北十五里,囊加帶、洪模以總管殷俊來報,宋陳宜中、張世傑、蘇劉義、劉師勇等挾益、廣二王出嘉會門,渡浙江遁去,惟太皇太后、嗣君在宮。伯顏亟使諭阿剌罕、董文炳、範文虎率諸軍先據守錢塘口,以勁兵五千人追陳宜中等,過浙江不及而還。丙戌,伯顏下令禁軍士入城,違者以軍法從事。遣呂文煥齎黃榜安諭臨安中外軍民,俾按堵如故。時宋三司衛兵白晝殺人,張世傑部曲尤橫閭里,小民乘時剽殺。令下,民大悅。伯顏又遣宣撫程鵬飛,計議孫鼎亨、囊加帶、洪君祥入宮,安諭太皇謝氏。丁亥,雲南行省賽典赤,以改定雲南諸路名號來上。又言雲南貿易與中州不同,鈔法實所未諳,莫若以交會、子公私通行,庶為民便。並從之。戊子,中書省臣言:「王孝忠等以罪命往八答山採寶玉自效,道經沙州,值火忽叛,孝忠等自拔來歸,令於瓜、沙等處屯田。」從之。大名路達魯花赤小鈐部坐奸贓伏誅,沒其家。宋主祖母謝氏遣其丞相吳堅、文天祥,樞密謝堂,安撫賈餘慶,中貴鄧惟善來見伯顏於明因寺。伯顏顧文天祥舉動不常,疑有異志,遂令萬戶忙古帶、宣撫唆都羈留軍中。且以其降表不稱臣,仍書宋號,遣程鵬飛、洪君祥偕來使賈餘慶復往易之。己丑,軍次湖州市。遣千戶囊加帶、省掾王祐,齎傳國玉璽赴闕。敕高麗國以有官子弟為質。中書省臣言:「賦民舊籍已有定額,至元七年新括協濟合並戶,為數凡二十萬五千一百八十。」敕減今歲絲賦之半。庚寅,伯顏建大將旗鼓,率左右翼萬戶巡臨安城,觀潮浙江,於是宋宗室大臣以次來見,暮還湖州市。辛卯,張弘範、孟祺、程鵬飛齎所易宋主稱臣降表至軍前。甲午,復薊州平谷縣。立隨路都轉運司,仍詔諭諸處管民官,以甕吉剌帶丑漢所部軍五百戍哈答城,不吉帶所部軍六百移戍建都,其兀兒禿、唐忽軍前在建都者,並遣還翼。穿濟州漕渠。以真定總管昔班為中書右丞。



 二月丁酉,詔劉頡、程德輝招淮西制置使夏貴。己亥,克臨江軍。庚子,宋主鳷率文武百僚詣祥曦殿,望闕上表,乞為籓輔;遣右丞相兼樞密使賈餘慶、樞密使謝堂、端明殿學士簽樞密院事家鉉翁、端明殿學士同簽樞密院事劉岊奉表以聞。宋主祖母太皇太后亦奉表及箋。是日,宋文武百司出臨安府,詣行中書省,各以其職來見。行省承制以臨安為兩浙大都督府,都督忙古帶、範文虎入城視事。辛丑,伯顏令張惠、阿剌罕、董文炳、左右司官石天麟、楊晦等入城,取軍民錢穀之數,閱實倉庫,收百官誥命符印,悉罷宋官府,散免侍衛禁軍。宋主鳷遣其右丞相賈餘慶等充祈請使,詣闕請命,右丞相命吳堅、文天祥同行。行中書省右丞相伯顏等,以宋主鳷舉國內附,具表稱賀,兩浙路得府八、州六、軍一、縣八十一,戶二百九十八萬三千六百七十二,口五百六十九萬二千六百五十。丁未,詔諭臨安新附府州司縣官吏士民軍卒人等曰:



 間者行中書省右丞相伯顏遣使來奏,宋母后、幼主暨諸大臣百官,已於正月十八日齎璽綬奉表降附。朕惟自古降王必有朝覲之禮,已遣使特往迎致。爾等各守職業,其勿妄生疑畏。凡歸附前犯罪,悉從原免;公私逋欠,不得征理。應抗拒王師及逃亡嘯聚者,並赦其罪。百官有司、諸王邸第、三學、寺、監、秘省、史館及禁衛諸司,各宜安居。所在山林河泊,除巨木花果外,餘物權免征稅。秘書省圖書,太常寺祭器、樂器、法服、樂工、鹵簿、儀衛,宗正譜牒,天文地理圖冊,凡典故文字,並戶口版籍,盡仰收拾。前代聖賢之後,高尚儒、醫、僧、道、卜筮,通曉天文歷數,並山林隱逸名士,仰所在官司,具以名聞。名山大川,寺觀廟宇,並前代名人遺跡,不許拆毀。鰥寡孤獨不能自存之人,量加贍給。



 伯顏就遣宋內侍王野入宮,收宋國袞冕、圭璧、符璽及宮中圖籍、寶玩、車輅、輦乘、鹵簿、麾仗等物。戊申,立浙東西宣慰司於臨安,以戶部尚書麥歸、秘書監焦友直為宣慰使,吏部侍郎楊居寬同知宣慰司事,並兼知臨安府事。乙卯,詔諭淮東制置使李庭芝、淮西制置使夏貴及所轄州軍縣鎮官吏軍民。丁巳,命焦友直括宋秘書省禁書圖籍。戊午,祀先農東郊。淮西制置夏貴以淮西諸郡來降,唯鎮巢軍復叛,貴遣使招之,守將洪福殺其使,貴親至城下,福始降,阿術斬之軍中。淮西路得府二、州六、軍四、縣三十四,戶五十一萬三千八百二十七,口一百二萬一千三百四十九。庚申,召伯顏偕宋君臣入朝。辛酉,車駕幸上都。設資戒大會於順德府開元寺。伯顏遣不伯、周青招泉州蒲壽庚、壽晟兄弟。甲子,董文炳、唆都發宋隨朝文士劉褒然及三學諸生赴京師,太學生徐應鑣父子四人同赴井死。帝既平宋,召宋諸將問曰:「爾等何降之易耶?」對曰:「宋有強臣賈似道擅國柄,每優禮文士,而獨輕武官。臣等久積不平,心離體解,所以望風而送款也。」帝命董文忠答之曰:「借使似道實輕汝曹,特似道一人之過耳,且汝主何負焉?正如所言,則似道之輕汝也固宜。」



 三月丁卯,命樞密副使張易兼知秘書監事。伯顏入臨安,遣郎中孟祺籍宋太廟四祖殿,景靈宮禮樂器、冊寶暨郊天儀仗,及秘書省、國子監、國史院、學士院、太常寺圖書祭器樂器等物。戊辰,括江南已附州郡軍器。甲戌,阿術遣使報廬州夏貴已降,文天祥自鎮江遁去,追之弗獲。荊湖南路行中書省言:「潭州既定,湖南州郡降者相繼,即分命諸將鎮守其地。」從之。宋福王與芮自浙東至伯顏軍中。以獨松關守將張濡嘗殺奉使廉希賢,斬之,籍其家。乙亥,伯顏等發臨安。丁丑,阿塔海、阿剌罕、董文炳詣宋主宮,趣宋主鳷同太后入覲。郎中孟祺奉詔宣讀,至「免系頸牽羊」之語,太后全氏聞之泣,謂宋主鳷曰:「荷天子聖慈活汝,當望闕拜謝。」宋主鳷拜畢,子母皆肩輿出宮,唯太皇太后謝氏以疾留。戊寅,敕諸路儒戶通文學者三千八百九十,並免其徭役,其富實以儒戶避役者為民,貧乏者五百戶,隸太常寺。敕淮西廬州置總管萬戶府,以中書右丞、河南等路宣慰使合剌合孫、襄陽管軍萬戶邸浹並行府事。庚辰,囊加帶以宋玉璽來上。乙酉,贛、吉、袁、南安四郡內附。庚寅,賜郡王瓜都銀印。敕上都和顧和買並依大都例。以中書右丞昔班為戶部尚書。



 閏月丙申,置宣慰司於濟寧路,掌印造交鈔,供給江南軍儲。以前西夏中興簽行中書省事暗都剌即思、大都路總管張守智並為宣慰使。東川行樞密院總帥汪惟正略地涪州,克山寨溪洞凡二十有三所。丁酉,詔湖廣阿里海牙、忽都帖木兒赴闕,令脫撥忽魯禿花、崔斌並留後鄂州。辛亥,命副樞張易遣宋降臣吳堅、夏貴等赴上都。戊午,淮西萬戶府招降方山等六寨。甲子,禁西番僧持軍器。以中書省左右司郎中郝禎參知政事。



 夏四月乙丑朔,阿術以宋高郵、寶應嘗饋餉揚州,遣蒙古軍將苫徹及史弼等守之,別遣都元帥孛魯歡等攻泰州之新城。丁卯,賜諸王都魯金印。戊辰,以河南兵事未息,開元路民饑,並弛正月五月屠殺之禁。庚午,敕南商貿易京師者毋禁。辛未,行江西都元帥宋都帶以應詔儒生醫卜士鄭夢得等六人進,敕隸秘書監。丙子,省東川行樞密院及成都經略司,以其事入西川行院。復石人山寨居民於信陽軍。免大都醫戶至元十二年絲銀。己卯,以侍衛親軍征戍歲久,放令還家,期六月,各歸其軍。庚辰,以水達達分地歲輸皮革,自今並入上都。壬午,召嗣漢天師張宗演赴闕。乙酉,召昭文館大學士姚樞、翰林學士王磐、翰林侍講學士徒單公履赴上都。庚寅,修太廟。以北京行中書省廉希憲為中書右丞,行中書省事於荊南府。



 五月乙未朔,伯顏以宋主鳷至上都,制授鳷開府儀同三司、檢校大司徒,封瀛國公。以平宋,遣官告天地、祖宗於上都之近郊。遣使代祀岳瀆。己亥,伯顏請罷兩浙宣慰司,以忙古帶、範文虎仍行兩浙大都督府事,從之。庚子,定度量。壬寅,宋三學生四十六人至京師。癸卯,復沂、莒、膠、密、寧海五州所括民為防城軍者為民,免其租徭二年。乙巳,賜伯顏所部有功將校銀二萬四千六百兩。阿術遣總管陳傑攻拔泰州之新城,遣萬戶烏馬兒守之,以逼泰州。丁未,宋揚州都統姜才攻灣頭堡,阿里別擊走之,殺其步騎四百人,右衛親軍千戶董士元戰死。戊申,宋馮都統等自真州率兵二千、戰船百艘襲瓜洲,阿術遣萬戶昔里罕、阿塔赤等出戰,大敗之,追至珠金沙,得船七十七艘,馮都統等赴水死。改博州為東昌路。己酉,括獵戶、鷹坊戶為兵。乙卯,靖州張州判及李信、李發焚其城,退保飛山新城,行中書省發兵攻殺之,徙其黨及家屬於大都。宋江西制置黃萬石率其軍來附,敕令入覲。辛酉,安西王相府請頒詔招合州張玨,不從。癸亥,升異樣局為總管府,秩三品。



 六月甲子朔,敕新附三衛兵之老弱者,放還其家。己巳,以孔子五十三世孫曲阜縣尹孔治兼權主祀事。命東征元帥府選襄陽生券軍五百,充侍衛軍。置行戶部於大名府,掌印造交鈔,通江南貿易。庚午,敕西京僧、道、也裏可溫、答失蠻等有室家者,與民一體輸賦。辛未,命阿里海牙出征廣西,請益兵,選軍三萬俾將之。壬申,罷兩浙大都督府,立行尚書省於鄂州、臨安。設諸路宣慰司,以行省官為之,並帶相銜,其立行省者,不立宣慰司。甲戌,以《大明歷》浸差,命太子贊善王恂與江南日官置局更造新歷,以樞密副使張易董其事。易、恂奏:「今之歷家,徒知歷術,罕明歷理,宜得耆儒如許衡者商訂。」詔衡赴京師。宋揚州姜才夜率步騎數千趨丁村堡,守將史弼、苫徹出戰,斬首百餘級,獲馬四十匹。詰旦,阿里、都督陳巖以灣頭堡兵邀其後,伯顏察兒踵至,所將皆阿術麾下兵,姜才軍遙望旗幟,亟走,遂大破之,獲米五千餘石。阿術又以宋人高郵水路不通,必由陸路饋運,千戶也先忽都以千騎邀之,數日米運果來,殺負米卒數千,獲米三千石。戊寅,詔作《平金》、《平宋錄》,及諸國臣服傳記,仍命平章軍國重事耶律鑄監修國史。戊子,樞密院上言:「陳宜中、張世傑聚兵福建以攻我師,江西都元帥宋都帶求援。」命以安慶、蘄、黃等郡宿兵,付宋都帶將之。己丑,宋都帶言福建魏天祐、游義榮棄家來附,以天祐為管軍總管兼知邵武軍事,義榮遙授建寧路同知,充管軍千戶。壬辰,下詔招諭宋揚州制置李庭芝以次軍官,及通、泰、真、滁、高郵大小官員。又詔諭陳宜中、張世傑、蘇劉義、劉師勇等使降。李庭芝留硃煥守揚州,與姜才率步騎五千東走,阿術親率百餘騎馳去,督右丞阿里、萬戶劉國傑分道追及泰州西,殺步卒千人,庭芝等僅得入,遂築長圍塹而守之,阿術獨當東南面,斷其走路。以戶部尚書張澍參知政事,行中書省事於北京。



 秋七月乙未,行中書省左右司郎中孟祺,以亡宋金玉寶及牌印來上,命太府監收之。丙申,淮安、寶應民流寓邳州者萬餘口,聽還其家。丁酉,宋涪州觀察楊立子嗣榮,請降詔招諭其父,從之。戊戌,升閬州為保寧府。敕山丹城直隸省部,以達魯花赤行者仍領之。壬寅,以李庭出征,賞其部將李承慶等鈔、馬、衣服、甲仗有差。乙巳,硃煥以揚州降。丁未,詔諭廣西路靜江府等大小州城官吏使降。甲寅,賜諸王孛羅印。以楊村至浮雞泊漕渠洄遠,改從孫家務。乙卯,宋泰州守將孫良臣與李庭芝帳下卒劉發、鄭俊開北門以降,執李庭芝、姜才,系揚州獄。丙辰,阿術以總管烏馬兒等守泰州,其通、滁、高郵等處相繼來附。淮東路得州十六、縣三十三,戶五十四萬二千六百二十四,口一百八萬三千二百一十七。遣使持香幣祠岳瀆后土。以中書右丞阿里海牙為平章政事,簽書樞密院事、淮東行樞密院別乞裏迷失為中書右丞,參知政事董文炳為中書左丞,淮東左副都元帥塔出、兩浙大都督範文虎、江東江西大都督知江州呂師夔、淮東淮西左副都元帥陳巖並參知政事。



 八月己巳,穿武清蒙村漕渠。敕漢軍都元帥闊闊帶、李庭將侍衛軍二千人西征。升漷陰縣為漷州。乙亥,斬宋淮東制置使李庭芝、都統姜才於揚州市。庚辰,罷襄陽統軍司。車駕至自上都。遣太常卿脫忽思以銅爵一、豆二,獻於太廟。以四萬戶總管奧魯赤參知政事。



 九月壬辰朔,命國師益憐真作佛事於太廟。己亥,享於太廟,常饌外,益野豕、鹿、羊、蒲萄酒。庚子,命姚樞、王磐選宋三學生之有實學者留京師,余聽還家。辛丑,遣廬州屯田軍四千,轉漕重慶。癸卯,以平宋赦天下。乙巳,高麗國王王愖上參議中贊金方慶功,授虎符。丙午,敕常德府歲貢包茅。丁未,諭西川行樞密院移檄重慶,俾內附。命有司隳沿淮城壘。辛亥,太白犯南斗。甲寅,太白入南斗。乙卯,以吐番合答城為寧遠府。辛酉,召宋宗臣鄂州教授趙與票赴闕。設資戒會於京師。阿術入覲。江淮及浙東西、湖南北等路,得府三十七、州一百二十八、關一、監一、縣七百三十三,戶九百三十七萬四百七十二,口千九百七十二萬一千一十五。



 冬十月甲子,以陳巖拔新城、丁村功,賜金五十兩,部將劉忠等賜銀有差。乙亥,賜皇子北平王出征軍士貧乏者羊馬幣帛有差。申明以良為娼之禁。丁亥,兩浙宣撫使焦友直以臨安經籍、圖畫、陰陽秘書來上。戊子,淮西安撫使夏貴請入覲,乞令其孫貽孫權領宣撫司事,從之。以淮東左副都元帥阿里為平章政事,河南等路宣慰使合剌合孫為中書右丞,兵部尚書王儀、吏部尚書兼臨安府安撫使楊鎮、河南河北道提刑按察使迷裏忽辛並參知政事。參知政事陳巖行中書省事於淮東。



 十一月癸巳,安西王所部軍克萬州。丙午,賜阿術所部有功將士二百三十九人各銀二百五十兩。西川行院忽敦言:「所部軍士久圍重慶,逃亡者眾,乞益軍一萬,並降詔招誘逋民之在大良平者。」並從之。壬子,賜龍答溫軍有功及死事者銀鈔有差。癸丑,並省內外諸司。丁卯,太陰犯填星。庚申,敕管民及理財之官由中書銓調,軍官由樞密院定議。隳襄漢、荊湖諸城。南平招撫使兼知峽州事趙真,請降詔招諭夔州安撫張起巖,從之。高麗國王王愖遣其臣判秘書寺硃悅,來告更名睶。



 十二月辛酉朔,熒惑掩鉤鈐。以十四年歷日賜高麗。丁卯,改雲南蘿葡甸為元江府路。辛未,賜塔海所部戰士及死事者銀鈔有差。賜忽不來等戰功十九人銀千二百兩。壬申,李思敬告運使姜毅所言悖妄,指毅妻子為證。帝曰:「妻子豈證者耶?」詔勿問。乙亥,定江南所設官府。辛巳,以軍士圍守崇慶勞苦,賜鈔六千錠。庚寅,詔諭浙東西、江東西、淮東西、湖南北府州軍縣官吏軍民:「昔以萬戶、千戶漁奪其民,致令逃散,今悉以人民歸之元籍州縣。凡管軍將校及宋官吏,有以勢力奪民田廬產業者,俾各歸其主,無主則以給附近人民之無生產者。其田租商稅、茶鹽酒醋、金銀鐵冶、竹貨湖泊課程,從實辦之。凡故宋繁冗科差、聖節上供、經總制錢等百有餘件,悉除免之。」伯顏言:「張惠守宋府庫,不俟命擅啟管鑰。」詔阿術詰其事,仍諭江之東西、浙之東西、淮之東西官吏等,檢核新舊錢穀。除浙西、浙東、江西、江東、湖北五道宣慰使。升江陵為上路,瑞安府仍為溫州,隴州為散府,薊州復置豐閏縣,升臨洮渭源堡為縣。賜諸王金、銀、幣、帛如歲例。賜諸王乃蠻帶等羊馬價。賞阿術等戰功,及賜降臣吳堅、夏貴等銀、鈔、幣、帛各有差,賜伯顏、阿術等青鼠、銀鼠、黃犬由只孫衣,餘功臣賜豹裘、麞裘及皮衣帽各有差。是歲,東平、濟南、泰安、德州、漣海、清河、平灤、西京西三州以水旱缺食,賑軍民站戶米二十二萬五千五百六十石,粟四萬七千七百十二石,鈔四千二百八十二錠有奇。平陽路旱,濟寧路及高麗瀋州水,並免今年田租。斷死罪三十四人。



 十四年春正月癸巳,行都元帥府軍次廣東,知循州劉興以城降。丙申,以江南平,百姓疲於供軍,免諸路今歲所納絲銀。賜嗣漢天師張宗演道靈應沖和真人,領江南諸路道教。戊戌,高麗金方慶等為亂,命高麗王治之,仍命忻都、洪茶丘飭兵禦備。癸卯,復立諸道提刑按察司。甲辰,命阿術選銳軍萬人赴闕。丁未,知梅州錢榮之以城降。戊申,賜三衛軍士之貧乏者八千三百五十二人各鈔二錠、幣十匹。己酉,賜耶律鑄鈔千錠。甲寅,敕宋福王趙與芮家貲之在杭、越者,有司輦至京師,付其家。丙辰,立建都、羅羅斯四路,守戍烏木等處,並置官屬。己未,以白玉碧玉水晶爵六,獻於太廟。括上都、隆興、北京、西京四路獵戶二千為兵。置江淮等路都轉運鹽使司,及江淮榷茶都轉運使司。命嗣漢天師張宗演修周天醮於長春宮,宗演還江南,以其弟子張留孫留京師。



 二月辛酉,命征東都元帥洪茶丘將兵二千赴上都。壬戌,瑞州安撫姚文龍率張文顯來降,其家屬為宋人所害,賜文龍、文顯等鈔有差。癸亥,彗星出東北,長四尺餘。甲子,遣使代祀岳瀆后土。丙寅,改安西王傅銅印為銀印。立永昌路山丹城等驛,仍給鈔千錠為本,俾取息以給驛傳之須。諸王只必鐵木兒言:「永昌路驛百二十戶,疲於供給,質妻孥以應役。」詔賜鈔百八十錠贖還之。丁卯,荊湖北道宣慰使塔海拔歸州山寨四十七所。戊辰,祀先農東郊。甲戌,西川行院不花率眾數萬至重慶,營浮屠關,造梯沖將攻之,其夜都統趙安以城降。張玨艤船江中,與其妻妾順流走涪州,元帥張德潤以舟師邀之,玨遂降。車駕幸上都。辛巳,命北京選福住所統軍三百赴上都。壬午,隳吉、撫二州城,隆興濱西江,姑存之。仍選汀州軍馬守禦瑞金縣。丙戌,連州守過元龍已降復叛,塔海將兵討之,元龍棄城遁。丁亥,知南恩州陳堯道、僉判林叔虎以城降。詔以僧亢吉益、憐真加加瓦並為江南總攝,掌釋教,除僧租賦,禁擾寺宇者。以大司農、御史大夫、宣慰使兼領侍儀司事孛羅為樞密副使,兼宣徽使,領侍儀司事。



 三月庚寅朔,以冬無雨雪,春澤未繼,遣使問便民之事於翰林國史院,耶律鑄、姚樞、王磐、竇默等對曰:「足食之道,唯節浮費,靡穀之多,無逾醪醴曲糵。況自周、漢以來,嘗有明禁。祈賽神社,費亦不貲,宜一切禁止。」從之。辛卯,湖廣行中書省言:「廣西二十四郡並已內附,議復行中書省於潭州,置廣南西路宣撫司於靜江。」詔鄭鼎所將侍衛軍萬人還京師,崔斌、阿里海牙同駐靜江,忽都鐵木兒、鄭鼎同駐鄂漢,賈居貞、脫博忽魯禿花同駐潭州。癸巳,以行都水監兼行漕運司事。甲午,以鄭鼎所部軍士撫定靜江之勞,命還家少休,期六月赴上都。乙未,福建漳、泉二郡蒲壽庚、印德傅、李玨、李公度皆以城降。丁酉,括馬三萬二千二百六匹,孕駒者還其主。壬寅,廣東肇慶府新封等州皆來降。癸卯,壽昌府張之綱以從叛棄市。乙巳,命中外軍民官所佩金銀符,以色組系於肩腋,庶無褻瀆,具為令。庚戌,建寧府通判郭纘以城降。黃州歸附官史勝入覲,以所部將校於躍等三十一人戰功聞,命官之。簽書東西川行樞密院事昝順言:「比遣同知隆州事趙孟烯齎詔招諭南平軍都掌蠻、羅計蠻及鳳凰、中壟、羅韋、高崖等四寨皆降。田、楊二家、豕鵝夷民,亦各遣使納款。」壬子,寶應軍人施福殺其守將,降於淮東都元帥府,詔以福為千戶,佩金符。癸丑,命汪惟正自東川移鎮鞏昌。行中書省承制,以閩浙溫、處、臺、福、泉、汀、漳、劍、建寧、邵武、興化等郡降官,各治其郡。潭州行省遣使上言:「廣南西路慶遠、鬱林、昭、賀、藤、梧、融、賓、柳、象、邕、廉、容、貴、潯皆降,得府一、州十四。」復立襄陽府襄陽縣。平章政事、浙西道宣慰使阿塔海為平章政事,行中書省事於江淮;郡王合答為平章政事,行中書省事於北京。



 夏四月甲子,宋特磨道將軍農士貴、知安平州李惟屏、知來安州岑從毅等,以所屬州縣溪洞百四十七、戶二十五萬六千來附。癸酉,省各路轉運司,事入總管府。設鹽轉運司四,置榷場於碉門、黎州,與吐蕃貿易。丙子,召安撫趙與可、宣撫陳巖入覲。丙戌,禁江南行用銅錢。均州復立南漳縣。



 五月癸巳,申嚴大都酒禁,犯者籍其家貲,散之貧民。辛丑,千戶合剌合孫死於渾都海之戰,命其子忽都帶兒襲職。癸卯,改廣南西路宣撫司為宣慰司,廣西欽、橫二州改立安撫司。各道提刑按察司兼勸農事。敕江南歸附官,三品以上者遣質子一人入侍。西番長阿立丁寧占等三十一族來附,得戶四萬七百。丙子,融州安撫使譚昌謀為不軌,伏誅。辛亥,以河南、山東水旱,除河泊課,聽民自漁。乙卯,選蒙古、漢軍相參宿衛。詔諭思州安撫使田景賢。又詔諭瀘州西南番蠻王阿永,筠連、騰串等處諸族蠻夷,使其來附。命真人李德和代祀濟瀆。



 六月丙寅,涪州安撫陽立及其子嗣榮相繼來附,命立為夔路安撫使,嗣榮為管軍總管,並佩虎符,仍賜鈔百錠。壬寅,賞徵廣戰死之家銀各五十兩。丁丑,置尚膳院,秩三品,以提點尚食、尚藥局忽林失為尚膳使,其屬司有七。庚辰,賞陽立所部戰士鈔千錠。甲申,荊湖北道宣慰使黑的得諜者,言夔府將出兵攻荊南。諭陽立等與塔海會兵御之。丁亥,升崇明沙為崇明州。以行省參政、行江東道宣慰使阿剌罕為中書左丞、行江東道宣慰使,湖北道宣慰使奧魯赤參知政事、行湖北道宣慰使。



 秋七月戊子朔,罷大名、濟寧印鈔局。壬辰,敕犯盜者皆棄市。符寶郎董文忠言:「盜有強竊,贓有多寡,似難悉置於法。」帝然其言,遽命止之。丁酉,敕自今非佩符使臣及軍情急速,不聽乘傳。戊戌,申禁羊馬群之在北者,八月內毋縱出北口諸隘踐食京畿之禾,犯者沒其畜。癸卯,諸王昔里吉劫北平王於阿力麻里之地,械系右丞相安童,誘脅諸王以叛,使通好於海都。海都弗納,東道諸王亦弗從,遂率西道諸王至和林城北。詔右丞相伯顏帥軍往御之。諸王忽魯帶率其屬來歸,與右丞相伯顏等軍合。丙午,置行御史臺於揚州,以都元帥相威為御史大夫。置八道提刑按察司。戊申,東川都元帥張德潤等攻取涪州,大敗之,擒安撫程聰、程廣。置行中書省於江西,以參知政事、行江西宣慰使塔出為右丞,參知政事、行江西宣慰使麥術丁為左丞,淮東宣慰使徹里帖木兒、江東宣慰使張榮實、江西宣慰使李恆、招討使也的迷失、萬戶昔裏門、荊湖路宣撫使程鵬飛、閩廣大都督兵馬招討使蒲壽庚並參知政事,行江西省事。壬子,榷大都商稅。丁巳,湖北宣慰司調兵攻司空山,復壽昌、黃州二郡。賜平宋將帥軍士及簡州軍士廣西死事者銀鈔各有差。回水窩淵聖廣源王加封善佑,常山靈濟昭應王加封廣惠,安丘雹泉靈霈侯追封靈霈公。以參知政事、行江東道宣慰使呂文煥為中書左丞。



 八月戊午朔,詔不花行院西川。丁卯,成都路倉收羨餘五千石,按察司已治其罪,命以其米就給西川兵。辛未,常德府總管魯希文與李三俊結構為亂,事覺,命行省誅之。車駕畋於上都之北。



 九月壬申,制鑌鐵海青圓符。丙申,廣南東路廣、連、韶、德慶、惠、潮、南雄、英德等郡皆內附。甲辰,福建行省以宋二王在其疆境,調都督忙兀帶、招討高興領兵討之。昂吉兒、忻都、唐兀帶等引兵攻司空山寨,破之,殺張德興,執其三子以歸。壬子,福建路宣慰使、行征南都元帥唆都,遣招討使百家奴、丁廣取建寧之崇安等縣及南劍州。



 冬十月丙辰朔,日有食之。己未,享於太廟。庚申,湖北宣慰使塔海略地至夔府之太原坪,禽其將,誅之。辛酉,弛蓋州獵禁。乙亥,以宋張世傑、文天祥猶未降,命阿塔海選銳兵防遏隆興諸城。禁無籍軍隨大軍剽掠者,勿過關渡。己卯,降臣郭曉、魏象祖入覲,賜幣帛有差。壬午,置宣慰司於黃州。甲申,播州安撫使楊邦憲言:「本族自唐至宋,世守此土,將五百年。昨奉旨許令仍舊,乞降璽書。」從之。以行省參政忽都帖木兒、脫博忽魯禿花、崔斌並為中書左丞,鄂州總管府達魯花赤張鼎、湖北道宣慰使賈居貞並參知政事。



 十一月戊子,樞密院臣言:「宋文天祥與其徒趙孟瀯同起兵,行中書發兵攻之,殺孟瀯,天祥僅以身免。」詔以其妻孥赴京師。右副都元帥張德潤上涪州功,賜鈔千錠。乙未,凡偽造寶鈔,同情者並處死,分用者減死杖之,具為令。庚子,命中書省檄諭中外,江南既平,宋宜曰亡宋,行在宜曰杭州。以吏部尚書別都魯丁參知政事。



 十二月丙辰,置中灤、唐村、淇門驛。丁卯,以大都物價翔踴,發官廩萬石,賑糶貧民。庚午,梁山軍袁世安以其城及金石城軍民來降。壬申,潭州行省復祁陽縣。斬首賊羅飛,餘黨悉平。乙亥,都元帥楊文安攻咸淳府,克之。以十五年歷日賜高麗國。以參議中書省事耿仁參知政事。冠州及永年縣水,免今年田租。導任河,復民田三千餘頃。賜諸王金、銀、幣、帛等物如歲例。賜諸王也不干、燕帖木兒等五百二十九人羊馬價,鈔八千四百五十二錠。賞拜答兒等千三百五十五人戰功,金百兩、銀萬五千一百兩、鈔百三十錠及納失失、金素幣帛、貂鼠豹裘、衣帽有差。是歲,賑東平、濟南等郡饑民,米二萬一千六百十七石、粟二萬八千六百十三石、鈔萬一百十二錠。斷死罪三十二人。



\end{pinyinscope}