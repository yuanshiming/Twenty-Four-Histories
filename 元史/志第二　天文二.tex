\article{志第二 天文二}

\begin{pinyinscope}

 月五星凌犯及星變下



 順帝元統元年正月癸酉,太白晝見。二月戊戌,亦如之。己亥,填星退犯太微東垣上相。丙辰,太陰犯天江下星。三月戊寅,太陰犯太微東垣上相。五月丁酉,熒惑犯太微垣右執法。六月丁丑,太陰犯壘壁陣西第二星。七月己亥,太陰犯房宿北第二星。九月甲午,太陰犯東咸西第一星,填星犯進賢。乙未,太陰犯天江下星。丁巳,太陰犯填星。己未,太陰犯氐宿距星。十月甲子,太陰入犯鬥宿魁東北星。十一月甲午,太陰犯壘壁陣西方第二星。辛亥,太陰犯太微東垣上相。壬子,太陰犯填星。癸丑,太陰犯亢宿南第一星。十二月癸酉,太陰犯鬼宿東北星。乙亥,太白犯壘壁陣西第八星,太陰犯軒轅夫人星。己卯,太陰犯進賢。癸未,太陰犯東咸西第二星。



 二年正月壬寅,太陰犯軒轅夫人星。庚戌,太陰犯房宿北第二星。二月癸酉,太陰犯太微東垣上相。丁亥,太白經天。三月辛丑,太陰犯進賢,又犯填星。四月丁丑,太白經天。戊寅,太白晝見。辛巳、壬午,皆如之。壬午夜,太白犯鬼宿積尸氣。七月己亥,太白經天。甲辰,亦如之。丙午,復如之。己酉,太白晝見。夜,流星如酒杯大,色赤,尾跡約長五尺餘,光明燭地,起自天津之側,沒於離宮之南。庚戌,太白經天。壬子,熒惑入犯鬼宿積尸氣。癸丑,太白經天。甲寅,亦如之。八月丙辰朔,太白經天。丁巳、戊午、己未,亦如之。癸亥、丙寅、戊辰、辛未、壬申、癸酉、甲戌、丁丑、己卯,皆如之。己卯夜,太白犯軒轅御女星。庚辰,太白經天。壬午,亦如之。九月庚寅,太白經天。壬辰,太陰入南斗魁。癸巳,太陰犯狗國東星,太白犯靈臺中星。甲午,太白經天。乙未,亦如之。己亥、壬寅,皆如之。乙巳,太白犯太微垣右執法。壬子,太白犯太微垣左執法。十月癸亥,熒惑犯太微西垣上將,太白犯進賢。乙亥,太陰犯軒轅夫人星,太白犯填星。十一月乙未,填星犯亢宿距星。庚戌,熒惑犯太微東垣上相。



 仍改至元元年二月甲戌,熒惑逆行入太微垣。四月壬戌,太陰犯太微垣左執法。五月癸卯,太陰犯壘壁陣東方第四星。六月壬戌,太陰犯心宿大星。七月乙未,太陰犯壘壁陣西方第二星。八月辛亥,熒惑犯氐宿東南星。九月丁亥,太陰入魁,犯鬥宿東南星。庚寅,太陰犯壘壁陣西方第二星。十月甲寅,熒惑犯鬥宿西第二星。庚申,太陰犯壘壁陣東方東第二星。甲子,太陰犯昴宿西第二星。丁卯,太白犯鬥宿魁第三星。戊辰,太白晝見。十一月甲申,太白經天。丙戌,亦如之。己丑,辰星犯房宿上星及鉤鈐星。丙申,太陰犯鬼宿東北星。己亥,太陰犯太微西垣上將。庚子,太陰犯太微垣左執法。十二月壬子,太陰犯壘壁陣西方第二星。辛酉,太白犯壘壁陣東方第六星。甲子,太白經天。乙丑,太陰犯軒轅夫人星。丙寅,太白經天。丁卯,亦如之。太陰犯太微垣右執法。庚午,太白經天。壬申,亦如之。癸酉,歲星晝見。乙亥,太白、歲星皆晝見。戊寅,太白經天,歲星晝見。閏十二月乙酉,熒惑犯壘壁陣西第八星。庚子,太陰犯心宿大星。壬寅,太陰犯箕宿距星。癸卯,太陰犯鬥宿魁東南星。



 二年正月壬戌,太陰犯太微垣右執法。甲子,太陰犯角宿距星。丁卯,太陰犯房宿距星。二月辛巳,太陰犯昴宿距星。甲申,太白經天。己丑,太陰犯太微西垣右執法。三月壬戌,太陰犯心宿距星。甲子,太陰犯箕宿距星。乙丑,太陰犯鬥宿東南星。四月丙戌,太陰犯角宿距星。五月庚戌,太陰犯靈臺西第一星。丙辰,太白晝見。丁巳,亦如之。六月戊子,太白犯井宿東扇北第二星。七月己酉,太白犯鬼宿東南星。乙卯,太白犯熒惑。八月己卯,太陰犯心宿東第一星。辛巳,太陰犯箕宿東北星。九月庚戌,熒惑犯太微西垣上將。十月丙子,熒惑犯太微垣左執法。丁亥,太陰犯昴宿。己亥,熒惑犯進賢。十一月己酉,太陰犯壘壁陣西第八星。己未,太陰犯鬼宿積尸氣。丁卯,太陰犯房宿距星。



 三年三月辛亥,太陰犯靈臺上星。四月辛卯,太陰犯壘壁陣西方第五星。庚子,太白晝見。五月壬寅,太白犯鬼宿東北星。乙巳,太陰犯軒轅左角。戊申,太白晝見。壬子,太陰犯心宿後星。戊午,太白晝見。己未,太陰犯壘壁陣西方第六星。辛酉,太白晝見。丁卯,彗星見於東北,如天船星大,色白,約長尺餘,彗指西南,測在昴五度。六月庚午,太白經天。辛未,亦如之。甲戌,復如之。乙亥,太白犯靈臺上星。己卯,太白經天。夜,太白犯太微西垣上將。壬午,太白晝見,太陰犯鬥宿魁尖星。丁亥,太白犯太微垣右執法。己丑,太白晝見。庚寅,亦如之。七月癸卯,太白經天。乙巳,亦如之。丙午,復如之。庚戌,太白晝見。甲寅,太白經天。辛酉,太白晝見。壬戌,太白經天。癸亥、甲子,皆如之。八月庚午,彗星不見。其星自五月丁卯始見,戊辰往西南行,日益漸速,至六月辛未,芒彗愈長,約二尺餘,丁丑掃上丞,己卯光芒愈甚,約長三尺餘,入圜衛,壬午掃華蓋、杠星,乙酉掃鉤陳大星及天皇大帝,丙戌貫四輔,經樞心,甲午出圜衛,丁酉出紫微垣,戊戌犯貫索,掃天紀,七月庚子掃河間,癸卯經鄭、晉,入天市垣,丙午掃列肆,己酉太陰光盛,微辨芒彗,出天市垣,掃梁星,至辛酉,光芒微小,瞻在房宿鍵閉之上、罰星中星正西,難測,日漸南行,至是凡見六十有三日,自昴至房,凡歷一十五宿而滅。甲戌,太陰犯心宿後星。九月己亥,熒惑犯鬥宿西第二星。甲辰,太陰犯鬥宿魁第二星。丁未,太陰犯壘壁陣西第一星。己酉,太陰犯壘壁陣西第八星。辛酉,太陰犯軒轅大星。十月庚午,太白晝見。丙子,太陰犯壘壁陣西方第七星。壬子,太陰犯昴宿上行星。丁亥,太白晝見,太陰犯鬼宿積尸氣。庚寅,太白晝見。辛卯,亦如之。丙申,復如之。十一月丁酉,太白經天,戊戌,太白犯亢宿距星。己亥,太白經天。壬寅,太陰犯熒惑。癸卯,太陰犯壘壁陣西第六星。丁未,填星犯鍵閉。辛亥,太陰犯五車東南星。甲寅,太陰犯鬼宿西北星。丙辰,太陰犯軒轅左角。丁巳,太白經天,太陰犯太微垣三公東南星。戊午,太白經天。癸亥,亦如之。甲子、乙丑,皆如之。十二月己巳,歲星退犯天罇東北星,填星犯罰星南第一星。甲戌,熒惑犯壘壁陣東第五星,太白犯東咸上星。



 四年正月癸卯,太白犯建星西第二星。甲辰,太白犯建星西第三星。丙午,太陰犯五車東南星。辛亥,太陰犯軒轅左角。己未,填星犯東咸上星。庚申,太陰入斗魁,太白犯牛宿。二月戊寅,太陰犯軒轅大星。己卯,太陰犯靈臺中星。三月戊申,填星退犯東咸上星。六月辛巳,填星退犯鍵閉星。閏八月己亥,填星犯罰星南第一星,太陰犯鬥宿南第二星。庚戌,太陰犯昴宿南第二星。乙卯,太陰犯鬼宿東南星。九月丙寅,太陰犯鬥宿距星。戊辰,太白犯東咸上第二星。癸酉,奔星如酒杯大,色白,起自右旗之下,西南行,沒於近濁。甲申,太陰犯軒轅御女。乙酉,太陰犯靈臺南第一星。庚寅,太白犯鬥宿北第二星。十月辛亥,太陰犯酒旗上星。十一月辛未,熒惑犯氐宿距星。丁丑,太陰犯鬼宿東南星。戊寅,太白犯壘壁陣西第六星。十二月庚子,熒惑犯房宿上星。癸卯,太白經天。己酉、庚戌、辛亥,皆如之。壬子,熒惑犯東咸上第二星。乙卯,太白犯外屏西第二星,太陰犯鬥宿距星。丙辰,太白經天。



 五年正月庚午,太陰犯井宿東扇上星。乙亥,熒惑犯天江上星。二月甲午,太陰犯昴宿上西第一星。壬寅,太陰犯靈臺下星。四月壬寅,太陰犯日星及犯房宿距星。五月庚午,太陰犯心宿後星。壬申,太陰犯鬥宿西第四星。丙子,太白犯畢宿右股西第三星。



 六月甲辰,熒惑退入南斗魁內。七月辛酉,熒惑犯南斗魁尖星。壬戌,亦如之。甲子,復如之。太陰犯房宿距星。甲戌,太白經天。乙亥、丙子,亦如之。戊寅、乙酉、丙戌,皆如之。八月戊子,太白經天。己丑、庚寅、辛卯,皆如之。甲午,太陰犯鬥宿西第四星。丁酉,太白犯軒轅大星。戊戌,太白經天。己亥,亦如之。壬寅、甲辰,皆如之。乙巳,太陰犯昴宿上行西第三星。九月戊午,太白經天。己未,亦如之。十月己亥,熒惑犯壘壁陣西方第六星。十一月丁巳,熒惑犯壘壁陣東方第五星。十二月甲午,太陰犯昴宿距星。癸卯,熒惑犯外屏西第三星。



 六年正月丁卯,太陰犯鬼宿距星。乙亥,太陰犯房宿距星。二月己丑,太陰犯昴宿。丙申,太陰犯太微西垣上將。癸卯,太陰犯心宿大星。丁未,太陰犯羅堰南第一星。戊申,熒惑犯月星。己酉,彗星如房星大,色白,狀如粉絮,尾跡約長五寸餘,彗指西南,測在房七度,漸往西北行。太陰犯虛梁南第二星。三月癸亥,太陰犯軒轅右角。庚午,太陰犯房宿距星。壬申,太陰犯南斗杓第二星。丙子,太陰犯虛梁南第一星。戊寅,太白犯月星。辛巳,是夜彗星不見。自二月己酉至三月庚辰,凡見三十二日。四月乙巳,太陰犯雲雨西北星。五月丁卯,太陰犯鬥宿西第二星。辛未,太陰犯虛梁西第二星。六月癸卯,太白晝見。己酉,亦如之。辛亥,復如之。辛亥夜,太白犯歲星。又,太白、歲星犯右執法。七月甲寅,太白晝見。丁巳,亦如之。庚申,太陰犯心宿距星,又犯心中央大星。壬戌,太白晝見。癸亥,亦如之。甲子,太陰犯羅堰。乙丑,太白晝見。丙寅,亦如之。癸酉,復如之。九月辛酉,太陰犯虛梁北第一星。丁卯,太陰犯昴宿距星,熒惑犯歲星。甲戌,太陰犯軒轅右角。十月丁酉,太白入南斗魁。己亥,太白犯鬥宿中央東星。十一月乙卯,太陰犯虛梁西第一星。戊午,熒惑犯氐宿距星。丙寅,辰星犯東咸上第一星。戊寅,辰星犯天江北第一星。十二月癸未,太陰犯虛梁北第一星。乙酉,太陰犯土公東星。丁亥,熒惑犯鉤鈐南星。乙未,熒惑犯東咸北第二星。戊戌,太陰犯明堂星。



 至正元年正月甲寅,熒惑犯天江上星。庚申,太陰犯井宿東扇北第二星。辛未,太陰犯心宿距星。癸酉,太陰犯鬥宿北第二星。甲戌,太白晝見。乙亥、丙子、丁丑,皆如之。二月己卯,太白晝見。庚辰,亦如之。丙戌,復如之。癸巳,太陰犯明堂東南星。



 三月癸酉,太陰犯雲雨西北星。六月庚午,太陰犯井宿距星。七月乙酉,太陰犯填星。庚寅,太陰犯雲雨西北星。九月庚辰,太陰犯建星南第二星。壬辰,太陰犯鉞星,又犯井宿距星。十月乙卯,歲星犯氐宿距星。丁巳,太陰犯月星。十一月己亥,太陰犯東咸南第一星。庚子,太陰犯天江北第二星。十二月丁巳,太白犯壘壁陣東方第五星。



 二年正月戊子,太陰犯明堂北第二星。甲午,熒惑犯月星。三月戊子,太陰犯房宿北第二星。四月庚申,太陰犯羅堰上星。五月甲申,太白經天。七月乙未,太陰掩太白。丁酉,太白晝見。八月丙午,太白晝見。九月丁丑,太陰犯羅堰北第一星。戊子,太陰犯井宿東扇南第一星。十月癸卯,太陰犯建星北第三星。甲寅,太陰犯天關。十一月辛卯,歲星、熒惑、太白聚於尾宿。



 三年二月甲辰,太陰犯井宿西扇北第二星,填星犯牛宿南第一星,熒惑犯羅堰南第一星。乙卯,太陰犯氐宿東南星。三月壬午,太陰犯氐宿東南星。七月庚辰,太白犯右執法。



 四年十二月壬戌,太陰犯外屏西第二星。



 七年七月丙辰,太陰犯壘壁陣東第四星。十一月庚戌,太陰犯天廩西北星。



 八年二月庚辰,太陰犯軒轅左角。癸未,太陰犯平道東星。三月丙辰,太陰犯建星西第一星。八月丙子,太陰犯壘壁陣西方第五星。九月己未,太陰犯靈臺東北星。



 九年正月庚戌,太白犯建星東第三星。辛亥,太陰犯平道西星。二月甲申,太陰犯建星西第二星。三月己亥,太白犯壘壁陣東方第六星。七月丙午,太陰犯壘壁陣東方南第一星。癸丑,太陰犯天關。九月丙戌,熒惑犯靈臺上星。十一月戊辰,太陰犯畢宿左股北第三星。庚辰,太白犯壘壁陣西方第二星。十二月戊戌,太白犯壘壁陣東方第五星。



 十年正月壬申,太陰犯熒惑。二月辛丑,太陰犯平道東星。甲辰,太陰犯鍵閉。三月己卯,熒惑犯太微西垣上將。四月丙午,太白犯鬼宿西北星。七月辛酉,太陰犯房宿北第一星。辛未,太白晝見。壬申、丁丑、壬午,皆如之。八月癸未朔,太白晝見。丁酉,亦如之。九月癸丑朔,太白晝見。壬戌,熒惑犯天江南第二星。十月癸巳,歲星犯軒轅大星。丙申,太陰犯昴宿右股東第二星。十一月戊辰,太陰犯鬼宿東北星。十二月乙未,太陰犯鬼宿西北星。



 十一年正月丙辰,辰星犯牛宿西南星。二月庚寅,太陰犯鬼宿東北星。乙未,太陰犯太微東垣上相。丁酉,太陰犯亢宿距星。三月丁卯,太陰犯東咸第二星。戊辰,太陰犯天江西第一星。七月己未,太陰犯鬥宿東第三星。壬戌,太白犯右執法。甲子,太陰犯壘壁陣東方第一星。己巳,太白犯太微垣左執法,熒惑入犯鬼宿積尸氣。八月乙酉,太陰犯天江南第二星。九月乙卯,辰星犯太微垣左執法。丁巳,太白犯房宿第二星。戊辰,太陰犯鬼宿東北星。十月戊寅,熒惑犯太微西垣上將。辛巳,太陰犯鬥宿距星。乙酉,太白犯鬥宿西第二星。己丑,太白晝見,熒惑犯歲星。辛卯,太白犯鬥宿西第四星。癸巳,歲星犯右執法。丙午,熒惑犯太微垣左執法。十一月辛亥,孛星見於奎宿。癸丑,孛星見於婁宿。甲寅,孛星見於胃宿。乙卯,亦如之。丙辰,孛星見於昴宿。丁巳,太陰犯填星,孛星微見於畢宿。丁卯,太白晝見。庚午,歲星晝見。十二月丙子,太白晝見。丁丑,太白經天。庚辰,亦如之。夜,太白犯壘壁陣西第六星。甲申,太陰犯填星。丙戌,太白經天。夜,太白犯壘壁陣西第七星。辛卯,太白經天。壬辰,亦如之。甲午,復如之。丁酉,太白晝見,太陰犯熒惑。庚子,太白經天,辰星犯天江西第二星。辛丑,太白經天。壬寅,太白晝見。



 十二年正月乙丑,太陰犯熒惑。己巳,歲星犯右執法。二月庚寅,太陰犯太微東垣上相。癸巳,太陰犯氐宿距星。三月戊午,太陰犯進賢。壬戌,太陰犯東咸西第一星。戊辰,太白晝見。五月癸酉,太白犯填星。六月辛亥,太白犯井宿東第二星。七月丁酉,辰星犯靈臺北第二星。八月丁卯,太白犯歲星。九月壬辰,太陰犯軒轅南第三星。十月戊午,太陰犯鬼宿東北星。甲子,太陰犯歲星。乙丑,太陰犯亢宿南第一星。十一月庚寅,太陰犯太微東垣上相。



 十三年正月乙酉,太陰犯太微東垣上相。戊戌,熒惑、太白、辰星聚於奎宿。二月己酉,太陰犯軒轅南第三星。庚戌,太白犯熒惑。壬子,太陰犯太微東垣上相。四月辛丑,太白犯井宿東扇北第一星。辛亥,太陰犯房宿北第二星。五月乙亥,太陰犯歲星。



 七月戊辰,太白晝見。九月庚寅,太陰犯熒惑。壬辰,太白經天,熒惑犯左執法。十月庚子,太白經天。甲辰,歲星犯氐宿距星。癸亥,太白犯亢宿距星。十一月壬申,太陰犯壘壁陣東方第四星。十二月丁酉,太白犯東咸北第一星。庚子,熒惑入氐宿。丁巳,太陰犯心宿距星。



 十四年正月乙丑,熒惑犯歲星。丁卯,太白犯建星西第二星。癸酉,熒惑犯房宿北第一星。二月戊午,太白犯壘壁陣西第八星。六月甲辰,太陰入斗宿南第一星。七月乙丑,太陰犯角宿距星。壬午,太陰犯昴宿距星。十月壬子,太陰犯太微垣右執法。十一月丙子,太陰犯鬼宿東北星。十二月己亥,太陰掩昴宿。



 十五年正月戊辰,太陰犯五車東南星。辛未,太陰犯鬼宿東北星。閏正月丁未,太陰犯心宿後星。丙辰,太白經天。三月庚寅,太陰犯五車東南星。五月丙申,太陰犯房宿距星。癸丑,太白經天。六月癸亥,太白經天。八月戊寅,太白晝見。九月己丑,太白晝見。夜,太白入犯太微垣左執法。庚寅,太白晝見。十月己未,太陰犯壘壁陣西方第二星。癸酉,太陰犯軒轅大星。十一月乙酉,熒惑犯氐宿距星。庚寅,填星退犯井宿東扇北第二星。己亥,太陰犯鬼宿東北星。十二月癸丑,熒惑犯房宿北第一星。



 十六年正月己丑,太陰犯昴宿西第一星。四月癸亥,熒惑犯壘壁陣西方第四星。五月壬辰,太白犯鬼宿西北星。癸巳,太白犯鬼宿積尸氣。甲午,太陰入犯鬥宿南第二星。丁酉,太陰犯壘壁陣西方第一星。八月丁卯,太陰犯昴宿西北星。甲戌,彗星見於正東,如軒轅左角大,色青白,彗指西南,約長尺餘,測在張宿十七度一十分,至十月戊午滅跡,西北行四十餘日。十一月丁亥,流星如酒杯大,色青白,尾跡約長五尺餘,光明燭地,起自西北,東南行,沒於近濁,有聲如雷。壬辰,太陰犯井宿東扇上星。



 十七年二月癸丑,太陰犯五車東南星。三月甲申,太陰入犯鬼宿積尸氣,又犯東南星。壬辰,歲星犯壘壁陣西南第六星。七月癸未,太白入犯鬼宿積尸氣。甲申,太陰入犯鬥宿距星。丁亥,填星入犯鬼宿距星。八月癸卯,填星犯鬼宿東南星,太白犯軒轅大星。己酉,歲星犯壘壁陣西方第六星。甲子,太陰犯五車尖星。閏九月癸卯,飛星如酒盂大,色青白,光明燭地,尾跡約長尺餘,起自王良,沒於勾陳之下。丙午,太陰犯鬥宿南第三星。庚申,太陰犯井宿東扇北第一星。十月乙亥,熒惑犯氐宿距星。甲申,太陰掩昴宿。十二月庚午朔,熒惑犯天江北第一星。戊寅,太白犯歲星。庚辰,太白犯壘壁陣東方第五星。甲申,太陰犯鬼宿距星。丁亥,歲星犯壘壁陣東方第五星。癸巳,太陰犯心宿後星。己亥,申時流星如金星大,尾跡約長三尺餘,起自太陰近東,往南行,沒後化為青白氣。



 十八年正月辛丑,填星退入犯鬼宿積尸氣。丙午,太陰犯昴宿。二月乙亥,填星入守鬼宿積尸氣。三月丁卯,太白在井宿,失行於北,生芒角。熒惑犯壘壁陣東方第六星。四月辛卯,太白入犯鬼宿積尸氣。五月壬寅,太白犯填星。壬子,太陰犯鬥宿東第三星。



 七月丁未,太陰犯鬥宿南第三星。戊申,太白晝見。八月壬申,太陰掩心宿大星。甲申,太陰掩昴宿距星。十月己卯,太陰犯昴宿距星。十一月丙午,太陰犯昴宿距星,太白犯房宿上第一星。辛酉,太陰掩心宿大星。十二月戊寅,太白生黑芒,環繞太白,乍東乍西,乍動乍靜。癸未,太白生黑芒,忽明忽暗,乍東乍西。戊子,太陰犯房宿南第二星。



 十九年正月辛丑,太陰犯昴宿東第一星。癸丑,流星如酒盂大,色赤,尾跡約長五尺餘,起自南河,沒於騰蛇,其星將沒,迸散隨落處有聲如雷。三月庚戌,太陰犯房宿距星。五月丙申,熒惑入犯鬼宿積尸氣。丙午,太陰犯天江南第一星。丁未,太陰犯鬥宿北第二星。七月丁酉,太白犯上將。甲辰,太白犯右執法。己酉,太白犯左執法。九月甲寅,太白入犯天江南第一星。十月壬申,太白入犯鬥宿南第三星。辛巳,流星如桃大,色黃潤,後離一尺又一小星相隨,色赤,尾跡通約長三尺餘,起自危宿之東,緩緩東行,沒於畢宿之西。十二月戊辰,太白犯壘壁陣西方第七星。



 二十年正月己亥,太陰犯井宿東扇北第二星。丙辰,熒惑犯牛宿東角星。四月丁卯,太陰犯明堂中星。癸酉,太陰犯東咸西第一星。五月癸卯,太陰犯建星西第二星。閏五月乙亥,流星如桃大,色赤,尾跡約長丈餘,起自房宿之側,緩緩西行,沒於近濁。六月癸巳,太白犯井宿東扇北第二星。戊戌,太陰犯建星西第三星。七月丁丑,太陰犯井宿距星。八月辛卯,太陰犯天江北第二星。壬寅,填星犯太微西垣上將。甲辰,太陰犯井宿鉞星。十月戊子,熒惑犯井宿東扇北第一星。



 二十一年正月庚申,太陰犯歲星。二月癸未,填星退犯太微西垣上將。壬寅,太陰犯天江北第一星。三月丙辰,太陰犯井宿西扇第二星。庚辰,熒惑入犯鬼宿西北星。五月壬戌,太陰犯房宿北第二星。癸酉,太白犯軒轅左角。甲戌,熒惑犯太白。六月乙未,熒惑、歲星、太白聚於翼宿。戊戌,太陰犯雲雨上二星。甲辰,太白晝見。七月丙辰,太陰犯氐宿東南星。十月甲申,太陰犯牛宿距星。十一月庚戌,太陰犯建星西第四星。癸亥,太陰犯井宿東扇北第四星。壬申,太陰犯氐宿東南星。



 二十二年正月戊申朔,太白犯建星西第二星。乙卯,填星退犯左執法。二月己卯,太白犯壘壁陣西方第二星。乙酉,彗星見,光芒約長尺餘,色青白,測在危七度二十分。丁酉,彗星犯離宮西星,至二月終,光芒約長二丈餘。三月戊申,彗星不見星形,惟有白氣,形曲竟天,西指,掃大角。壬子,彗星行過太陽前,惟有星形,無芒,如酒杯大,昏蒙,色白,測在昴宿六度,至戊午始滅跡焉。四月丁亥,熒惑離太陽三十九度,不見,當出不出。五月辛酉,太陰犯建星西第四星。六月辛巳,彗星見於紫微垣,測在牛二度九十分,色白,光芒約長尺餘,東南指,西南行。戊子,彗星光芒掃上宰。七月乙卯,彗星滅跡。八月癸巳,太陰犯畢宿右股第二星。九月丁未,太白犯亢宿南第一星。己酉,太陰犯鬥宿北第一星。癸亥,歲星犯軒轅大星。丙寅,熒惑犯鬼宿西北星。己巳,流星如酒杯大,色青白,光明燭地。熒惑入犯鬼宿積尸氣。十月己卯,太陰犯牛宿距星。丁亥,辰星犯亢宿南第一星。戊子,太陰犯畢宿距星。十二月壬辰,太陰犯角宿距星。



 二十三年正月庚戌,歲星退犯軒轅大星。二月戊戌,太白晝見。庚子,亦如之。三月丙辰,太陰犯氐宿距星。四月辛丑,熒惑犯歲星。庚申,歲星犯軒轅大星。五月壬午,太白晝見。甲午,亦如之。乙未,熒惑犯右執法。六月乙卯,太白犯井宿西扇北第二星。壬戌,太白晝見。夜,太白入犯井宿東扇南第二星。七月乙酉,太白晝見。丙戌、辛卯,皆如之。八月壬寅,太白入犯軒轅大星。乙巳,太陰犯建星東第二星。丁未,太白犯軒轅左角。己酉,太白晝見。壬子,亦如之。丙辰,太陰犯畢宿右股北第二星。己未,太白晝見。辛酉,太白犯歲星。乙丑,太白入犯右執法。九月辛未,太白入犯左執法。乙亥,歲星入犯右執法。丁丑,辰星犯填星,丁亥,太白犯填星。辰星犯亢宿南第一星。十月癸卯,太白犯氐宿距星。戊午,太白犯房宿北第一星。十一月癸未,太陰犯軒轅右角,歲星犯太微垣左執法。



 二十四年正月癸酉,太陰犯畢宿大星。戊寅,太陰犯軒轅右角。二月壬子,歲星自去年出右掖門,又犯右執法。太陰犯西咸南第一星。四月丁未,太陰犯西咸南第一星。癸丑,太白入犯井宿東扇北第一星。五月甲戌,太白犯鬼宿西北星。乙亥,又犯積尸氣。歲星入犯右執法。六月丁巳,太白犯右執法。七月癸亥,太白與歲星相合於翼宿,二星相去八寸餘。甲子,歲星犯左執法。八月丁未,熒惑入犯鬼宿積尸氣。九月乙丑,太白晝見。甲申,太陰犯軒轅右角。戊子,熒惑入犯軒轅大星。十月丙午,太陰犯畢宿大星。己酉,太陰犯井宿東扇南第一星。丙辰,太白犯鬥宿西第二星。十二月乙卯,太陰犯太白。



 二十五年正月丁卯,太白晝見。戊辰,亦如之。太陰犯畢宿右股東第四星。甲戌,太白犯建星西第四星。二月丙午,太陰犯填星。三月戊辰,太白犯壘壁陣東方第五星。四月壬子,熒惑犯靈臺東北星。五月辛酉,熒惑犯太微西垣上將。流星如酒杯大,色青白,光明燭地,起自房宿之側,緩緩西行,沒於太微垣右執法之下。七月丁丑,填星、歲星、熒惑聚於角、亢。己卯,太陰犯畢宿左股北第二星。八月乙未,太陰犯建星東第三星。己亥,太陰犯壘壁陣東方第六星。九月丁丑,太陰犯井宿東扇南第一星。十月辛卯,熒惑犯天江東第二星。己酉,熒惑犯鬥宿杓星西第二星,太陰犯右執法。庚戌,太陰犯太微東垣上相。閏十月戊辰,太白、辰星、熒惑聚於斗宿。太陰犯畢宿右股北第四星,又犯左股北第三星。壬申,太白犯辰星。十一月己丑,太白犯熒惑。太陰犯壘壁陣東方第五星。丙申,太陰犯畢宿大星。癸卯,太陰犯太微西垣上將。十二月丙辰,太陰犯太白。癸亥,太陰犯畢宿右股第二星。庚午,歲星掩房宿北第一星。辛未,太陰犯太微垣右執法。



 二十六年正月戊戌,太陰犯太微西垣上將。辛丑,太陰犯亢宿距星。二月戊午,太陰犯畢宿大星。丁丑,歲星退行,犯房宿北第一星。歲星守鉤鈐。三月甲午,太陰犯左執法。四月己未,太陰犯軒轅大星。乙丑,太陰犯西咸西第一星。丙子,太白入犯鬼宿積尸氣。六月癸酉,流星如酒杯大,色青白,尾跡約長尺餘,起自心宿之側,東南行,光明燭地,沒於近濁。七月丁酉,熒惑犯鬼宿積尸氣。甲辰,太白晝見。丙午、丁未、戊申,皆如之。八月辛亥,太白晝見。己未,太陰掩牛宿南三星。庚午,歲星犯鉤鈐。乙亥,太陰掩軒轅大星。九月壬辰,太白犯太微垣右執法。庚子,孛星見於紫微垣北斗權星之側,色如粉絮,約斗大,往東南行,過犯天棓星。辛丑,孛星測在尾十八度五十分。壬寅,孛星測在女二度五十分。癸卯,孛星測在女九度九十分。甲辰,孛星測在虛初度八十分。太陰犯太微西垣上將。乙巳,孛星出紫微垣北斗權星、玉衡之間,在於軫宿,東南行,過犯天棓,經漸臺、輦道,去虛宿、壘壁陣西方星,始消滅焉。丙午,熒惑犯太微西垣上將。十一月乙酉,太白犯填星。丁亥,太白犯房宿北第一星。戊子,熒惑犯太微東垣上相,太白犯鍵閉。己丑,流星如酒杯大,分為三星,緊相隨,前星色青明,後二星色赤,尾跡約長二丈餘,起自東北,緩緩往西南行,沒於近濁。庚寅,太陰犯畢宿右股北第四星。丙申,太白、歲星、辰星聚於尾宿。庚子,太陰犯太微東垣上相。辛丑,填星犯房宿北第一星。甲辰,太白犯歲星。十二月戊午,太陰犯畢宿大星。庚申,太陰犯井宿西扇北第二星。乙丑,太陰犯軒轅左角。丙寅,太陰犯太微西垣上將。辛未,太陰犯西咸西第一星。甲戌,太陰犯建星西第三星。



 二十七年正月癸巳,太陰犯太微西垣上將。二月乙卯,太陰犯井宿西扇北第二星。三月辛巳,填星退犯鍵閉星。四月丙寅,太陰犯壘壁陣西方第四星。六月乙卯,太陰犯氐宿東北星。辛未,太陰犯井宿西扇北第二星。七月壬辰,熒惑犯氐宿東南星。丙申,太陰犯畢宿大星。己亥,太陰犯井宿東扇南第二星。八月庚戌,熒惑犯房宿北第二星。癸丑,太陰犯建星西第二星。九月丁丑,填星犯房宿北第一星,熒惑犯天江南第二星。乙酉,太陰犯壘壁陣東方第六星。辛卯,填星犯鍵閉,太陰犯畢大星。癸巳,太陰犯井宿西扇北第二星。丁酉,熒惑犯鬥宿西第二星。十月戊午,太陰犯畢宿右股西第二星。辛酉,太陰犯井宿東扇南第三星。癸亥,太陰犯鬼宿西南星。丁卯,歲星、太白、熒惑聚於斗宿。十一月戊寅,太白晝見。庚辰,太陰犯壘壁陣東方南東第一星。



 餘見本紀。



\end{pinyinscope}