\article{列傳第一 後妃一}

\begin{pinyinscope}

 太祖光獻翼聖皇后,名孛兒臺旭真,弘吉剌氏,特薛禪之女也。特薛禪與子按陳從太祖征伐有功,賜號國舅工夫」的認識方法。主要著作《花潭集》。,封王爵,以統其部族。有旨:「生女為後,生男尚公主,世世不絕。」世祖至元二年十二月,追謚光獻翼聖皇后。冊文曰:「尊祖宗,致誠孝,實王政之攸先;法天地,建鴻名,亦母儀之克稱。肆先虔於太室,庸昭示於後昆,體茲至公,節以大惠。欽惟光獻皇后,宅收淵靜,稟德柔嘉,當聖神創業之初,有夙夜求賢之助。功施社稷,垂慈訓於景襄;慶衍宮闈,流徽音於莊聖。協贊龍飛之運,永詒燕翼之謀。惟周人著稱《思齊》,亦推本興王之跡;在漢世始謚光烈,蓋篤申追遠之情。是用稽迪舊章,增崇遺美。謹遣攝太尉某,奉玉冊玉寶,加上尊謚曰光獻翼聖皇后。伏惟淑靈降格,典禮備膺,於億萬年,茂隆丕祚。」升祔太祖廟。其餘后妃,有四斡耳朵四十餘人,不記氏族,其名悉見於《表》。後皆仿此。



 太宗昭慈皇后,名脫列哥那,乃馬真氏,生定宗。歲辛丑十一月,太宗崩,後稱制攝國者五年。丙午,會諸王百官,議立定宗。朝政多出於後。至元二年崩,追謚詔慈皇后,升祔太宗廟。



 定宗欽淑皇后,名斡兀立海迷失。定宗崩,後抱子失列門垂簾聽政者六月。至元二年,追謚欽淑皇后。



 憲宗貞節皇后,名忽都臺,弘吉剌氏,特薛禪孫忙哥陳之女也,蚤崩,後妹也速兒繼為妃。至元二年,追謚貞節皇后,升祔憲宗廟。



 世祖昭睿順聖皇后,名察必,弘吉剌氏,濟寧忠武王按陳之女也。生裕宗。中統初,立為皇后。至元十年三月,授冊寶,上尊號貞懿昭聖順天睿文光應皇后。一日,四怯薛官奏割京城外近地牧馬,帝既允,方以圖進。後至帝前,將諫,先陽責太保劉秉忠曰:「汝漢人聰明者,言則帝聽,汝何為不諫?向初到定都時,若以地牧馬則可,今軍蘸俱分業已定,奪之可乎?」帝默然,命寢其事。後嘗於太府監支繒帛表裏各一,帝謂後曰:「此軍國所需,非私家物,後何可得支?」後自是率宮人親執女工,拘諸舊弓弦練之,緝為綢,以為衣,其韌密比綾綺。宣徽院羊臑皮置不用,後取之,合縫為地毯。其勸儉有節而無棄物,類如此。十三年,平宋,幼主朝於上都。大宴,眾皆歡甚,唯後不樂。帝曰:「我今平江南,自此不用兵甲,眾人皆喜,爾獨不樂,何耶?」後跪奏曰:「妾聞自古無千歲之國,毋使吾子孫及此,則幸矣。」帝以宋府庫故物各聚置殿庭上,召後視之,後遍視即去。帝遣宦者追問後,欲何所取。後曰:「宋人貯蓄以遺其子孫,子孫不能守,而歸於我,我何忍取一物耶!」時宋太后全氏至京,不習北方風土,後為奏令回江南。帝不允,至三奏,帝乃答曰:「爾婦人無遠慮,若使之南還,或浮言一動,即廢其家,非所以愛之也。茍能愛之,時加存恤,使之便安可也。」後退,益厚待之。胡帽舊無前簷,帝因射日色炫目,以語後,後即益前簷。帝大喜,遂命為式。又制一衣,前有裳無衽,後長倍於前,亦無領袖,綴以兩襻,名曰比甲,以便弓馬,時皆仿之。後性明敏,達於事機,國家初政,左右匡正,當時與有力焉。



 十八年二月崩。三十一年,成宗即位,五月,追謚昭睿順聖皇后,其冊文曰:「奉先思孝,臣子之至情;節惠易名,古今之大典。惟殷娥有明德之號,而周任著《思齊》之稱。爰考舊章,式崇尊謚。恭惟先皇后,厚德載物,正位承天。隆內治於公宮,綱大倫於天下。曩事龍潛之邸,及乘虎變之秋。鄂渚班師,洞識事機之會;上都踐祚,居多輔佐之謀。先物之明,獨斷於衷;進賢之志,允葉於上。左右我聖祖,建帝王之極功;撫育我前人,嗣社稷之重托。臣下之勸勞灼見,生民之疾苦周知。儷宸極二十年,垂慈範千萬世。惟全美聖而益聖,宜顯冊書而屢書。不勝惓藐懇懇之誠,敬展尊尊親親之義,以揚盛烈,以對耿光。謹遣某官某奉玉冊玉寶,上尊謚曰昭睿順聖皇后。欽惟淑靈在天,明鑒逮下。增輝煒管,茂揚徽懿之音;合饗太宮,益衍壽昌之福。」升祔世祖廟。



 南必皇后,弘吉剌氏,納陳孫仙童之女也。至元二十年,納為皇后,繼守正宮。時世祖春秋高,頗預政,相臣常不得見帝,輒因後奏事焉。有子一人,名鐵蔑赤。



 成宗貞慈靜懿皇后,名失憐答里,弘吉剌氏,斡羅陳之女也。大德三年十月,立為後。生皇子德壽,早薨。武宗至大三年十月,追尊謚貞慈靜懿皇后,其冊文曰:「宗祧定位,象天地之有陰陽;今古同符,通幽明以行典禮。哀榮斯備,孝敬兼陳。恭惟先元妃弘吉剌氏,慶毓仙源,德昭彤史。春宮主饋,共瞻採翟之輝;椒掖正名,莫際飛龍之會。惟貞協在中之美,而慈推成物之仁。靜既合夫坤元,懿益彰於壼則。雖小星之逮下,豈眾曜之敢齊。嗣服雲初,追懷曷已。是用究成先志,式闡徽稱。謹遣某官某,上尊謚曰貞慈靜懿皇后,升祔於成宗皇帝殿室。伏惟淑靈,永伸配侑,介以景福,佑我無疆。」



 卜魯罕皇后,伯岳吾氏,駙馬脫里思之女。元貞初,立為皇后。大德三年十月,授冊寶。成宗多疾,後居中用事,信任相臣哈剌哈孫,大德之政,人稱平允,皆後處決。京師創建萬寧寺,中塑秘密佛像,其形醜怪,後以手帕蒙覆其面,尋傳旨毀之。省院臺臣奏上尊號,帝不允。車駕幸上都,後方自奏請。帝曰:「我病日久,國家大事多廢不舉,尚寧理此等事耶!」事遂寢。大德十年,後嘗謀貶順宗妃答吉與其子仁宗往懷州。明年,成宗崩。時武宗在北邊,恐其歸,必報前怨,後乃命取安西王阿難答失裡來京師,謀立之。仁宗自懷州入清宮禁,既誅安西王,並構後以私通事,出居東安州。



 武宗宣慈惠聖皇后,名真哥,弘吉剌氏,脫憐子迸不剌之女。至大三年四月,冊為皇后,其文曰:「乾為天,坤為地,四時由是以相成,日宗陽,月宗陰,萬象以之而並著。後職有關於世教,先猷具載於邦彞。惟慈旨之親承,亦僉言之允若。咨爾皇後弘吉剌氏,睿聰淑哲,端懿誠莊。寶婺分輝,源天潢之自出;纓徽迪慶,系紱組以相仍。後逸」皇慶二年,立長秋寺,掌皇后宮政,秩三品。泰定四年十一月崩,上尊謚曰宣慈惠聖皇后,升祔武宗廟。



 速哥失里皇后,按陳從孫哈兒只之女、真哥皇后之從妹也。



 妃二人:亦乞烈氏,奴兀倫公主之女,實生明宗,天歷二年追謚仁獻章聖皇后;唐兀氏,生文宗,天歷二年追謚文獻昭聖皇后。



 仁宗莊懿慈聖皇后,名阿納失失里,弘吉剌氏,生英宗。皇慶二年三月,冊為皇后,上冊寶,遣官祭告天地於南郊及太廟。改典內院為中政院,秩正二品。



 英宗即位,上尊號皇太后,其冊文曰:「坤承乾德,所以著兩儀之稱;母統父尊,所以崇一體之號。故因親而立愛,宜考禮以正名。恭惟聖母,溫慈惠和,淑哲端懿。上以奉宗祧之重,下以敘倫紀之常。恢王化於二南,嗣徽音於三母。輔佐先考,憂勤警戒之慮深;擁佑眇躬,撫育提攜之恩至。迨於今日,紹我丕基。規摹一出於慈闈,付托益彰於祖訓。致天下之養以為樂,未足盡於孝心;極域中之大以為尊,庶可稱其懿美。式遵貴貴之義,用罄親親之情。謹遣某官某奉冊,上尊號曰皇太后。伏惟周宗綿綿,長信穆穆,備《洛書》之錫福,粲坤極之儀天。啟佑後人,永錫胤祚。」明日,受百官朝賀於興聖宮。至治二年崩,上謚莊懿慈聖皇后,其冊文曰:「致孝所以揚親,易名所以表行。矧為天下母而養弗逮,履天子位而報則豐。曷勝孺慕之心,必盡欽崇之禮。欽惟先皇太后,夙明壼則,克嗣徽音。輔佐先朝,有恭儉節用之實;誕育眇質,有劬勞顧復之恩。九族咸育於仁,四海仰遵其化。昊天不吊,景命靡融。愴聖善之長違,念風猷之未泯。是用揄揚於彤史,正宜敷繹於寶慈。爰據彞經,追嚴徽號。謹遣攝太慰某官某奉玉冊玉寶,上尊號曰莊懿慈聖皇后。伏惟淑靈如在,合饗太宮。鑒格孔昭,膺茲巨典。陰相丕祚,億萬斯年。」升祔仁宗廟。



 英宗莊靜懿聖皇后,名速哥八剌,亦啟烈氏,昌國公主益里海涯女也。至治元年,冊為皇后。泰定四年六月崩,謚曰莊靜懿聖皇后。



 泰定帝八不罕皇后,弘吉剌氏,按陳孫斡留察兒之女。泰定元年,冊為皇后。



 妃二人:一曰必罕,一曰速哥答里,皆弘吉剌氏,兗王買住罕之女也。文宗天歷初,俱安置東安州。



 明宗貞裕徽聖皇后,名邁來迪,生順帝而崩。文宗立,謚貞裕徽聖皇后。



 八不沙皇后,成宗甥壽寧公主之女也。侍明宗潛邸,生寧宗。天歷二年,立寧徽寺,掌明宗皇后宮事,以鈔萬錠、幣帛二千匹,供後宮費用。十一月,後請為明宗資冥福,命帝師率諸僧作佛事七日於大天源延聖寺,道士建醮於玉虛、天寶、太乙、萬壽四宮,及武當、龍虎二山。至順元年,敕有司供明宗後宮幣帛二百匹。是年四月崩。



 文宗卜答失里皇后,弘吉剌氏,父駙馬魯王雕阿不剌,母魯國公主桑哥剌吉。文宗居建業,後亦在行。天歷元年,文宗即位,立為皇后。二年,授冊寶。十一月,後以銀五萬兩助建大承天護聖寺。至順元年,以籍沒張珪家田四百頃,賜護聖寺為永業。後與宦者拜住謀殺明宗後八不沙。三年八月,文宗崩於上都,後導揚末命,申帝初志,遂立明宗次子懿璘質班,是為寧宗。十一月,奉玉冊玉寶尊皇后為皇太后。十二月,御興聖殿受朝賀。寧宗崩,大臣請立太子燕貼古思。後曰:「天位至重,吾子尚幼,明宗長子妥歡帖睦爾在廣西,今十三歲矣,理當立之。」於是奉旨迎至京師,以明年六月即位,是為順帝。元統元年,尊為太皇太后,仍稱制臨朝。至元六年六月,詔去尊號,安置東安州,尋崩。



 寧宗答里也忒迷失皇后,弘吉剌氏。至順三年十月,立為後。至正二十八年崩,升祔寧宗廟。



 順帝答納失里皇后,欽察氏,師太平王燕鐵木兒之女。至順四年,立為後。元統二年,授冊寶,其冊文曰:「天之元統二氣,配莫厚於坤儀;月之道循右行,明周貞於乾曜。若昔帝王之宅後,居多輔相之世勛。蓋選德於亢宗,亦疇庸於先正。造周資任、姒之化,興漢表馬、鄧之功。咨爾皇后欽察氏,雍肅惠慈,謙裕靜淑。乃祖乃父,夙堅翼亮之心;於國於家,實獲修齊之助。朕纘丕圖之初載,親承太后之睿謨。眷我元臣,簡茲碩媛。相嚴禘而率典,奉慈極以愉顏。用彰禕翟之華,式著旗常之舊。令攝太尉某官授以玉冊寶章,命爾為皇后。備成嘉禮,宏賁太猷。於戲!嵩高生賢,予篤懷於良佐;《關雎》正始,爾勉嗣於徽音。永錫壽康,昭示悠久。」三年,後兄御史大夫唐其勢以謀逆誅,弟塔剌海走匿後宮,後以衣蔽之,因遷後出宮,丞相伯顏鴆后於開平民舍。



 伯顏忽都皇后,弘吉剌氏,宣慈惠聖皇后真哥侄毓德王孛羅帖木兒之女也。至元三年三月,立為皇后。其冊文曰:「帝王之道,齊其家而天下平;風教所基,正乎位而人倫厚。爰擇配以承宗事,若稽古以率典常。咨爾弘吉剌氏,淑哲溫恭,齊莊貞一。屬選賢於中壼,躬受命於慈闈。勖帥來嬪,蹈矩儀之有度;動容中禮,謹夙夜以無違。茲表式於宮庭,宜推崇其位號。乃蠲吉旦,庸舉彞章,遣攝太尉某持節授以玉冊寶章,命爾為皇后。於戲!乾施坤承,克順成於四序;日明月儷,久照臨於萬方。朕欲躋世於乂安,爾其助予之德化,共御亨嘉之運,益延昌熾之期。勉爾徽音,聿修內治。」生皇子真金,二歲而夭。後性節儉,不妒忌,動以禮法自持。第二皇后奇氏素有寵,居興聖西宮,帝希幸東內。後左右以為言,後無幾微怨望意。從帝時巡上京,次中道,帝遣內官傳旨,欲臨幸,後辭曰:「暮夜非至尊往來之時。」內官往復者三,竟拒不納,帝益賢之。帝嘗問後:「中政院所支錢糧,皆傳汝旨,汝還記之否?」後對曰:「妾當用則支。關防出入,必己選人司之,妾豈能盡記耶?」居坤德殿,終日端坐,未嘗妄逾戶閾。至正二十五年八月崩,年四十二。奇氏後見其所遺衣服弊壞,大笑曰:「正宮皇后,何至服此等衣耶!」其樸素可知。逾月,皇太子自冀寧歸,哭之甚哀。



 完者忽都皇后奇氏,高麗人,生皇太子愛猷識理達臘。家微,用後貴,三世皆追封王爵。初,徽政院使禿滿迭兒進為宮女,主供茗飲,以事順帝。後性穎黠,日見寵幸。後答納失里皇后方驕妒,數箠辱之。答納失裏既遇害,帝欲立之,丞相伯顏爭不可。伯顏罷相,沙剌班遂請立為第二皇后,居興聖宮,改徽政院為資正院。後無事,則取《女孝經》、史書,訪問歷代皇后之有賢行者為法。四方貢獻,或有珍味,輒先遣使薦太廟,然後敢食。至正十八年,京城大饑,後命官為粥食之。又出金銀粟帛,命資正院使樸不花於京都十一門置塚,葬死者遺骼十餘萬,復命僧建水陸大會度之。時帝頗怠於政治,後與皇太子愛猷識理達臘遽謀內禪,遣樸不花諭意丞相太平,太平不答。復召太平至宮,舉酒賜之,自申前請,太平依違而已,由是後與太子銜之。而帝亦知後意,怒而疏之,兩月不見。樸不花因後而寵幸,既被劾黜,後諷御史大夫佛家奴為之辯明。佛家奴乃謀再劾樸不花,後知之,反嗾御史劾佛家奴,謫居潮河。初,奇氏之族在高麗者,怙勢驕橫,高麗王怒,盡殺之。二十三年,後謂皇太子曰:「汝何不為我復讎耶?」遂立高麗王族人留京師者為王,以奇族之子三寶奴為元子。遣同知樞密院事崔帖木兒為丞相,用兵一萬,並招倭兵,共往納之。過鴨綠水,伏兵四起,乃大敗,餘十七騎而還,後大慚。二十四年七月,孛羅帖木兒稱兵犯闕,皇太子出奔冀寧,下令討孛羅帖木兒。孛羅帖木兒怒,嗾監察御史武起宗言後外撓國政,奏帝宜遷後出於外,帝不答。二十五年三月,遂矯制幽於諸色總管府,令其黨姚伯顏不花守之。四月庚寅,孛羅帖木兒逼後還宮,取印章,偽為後書召太子。後仍回幽所,後又數納美女於孛羅帖木兒,至百日,始還宮。及孛羅帖木兒死,召皇太子還京師,後傳旨令廓擴帖木兒以兵擁皇太子入城,欲脅帝禪位。廓擴帖木兒知其意,至京城三十里外,即遣軍還營,皇太子復銜之。事見《擴廓帖木兒傳》。會伯顏忽都皇后崩,十二月,中書省臣奏言,後宜正位中宮,帝不答。又奏改資正院為崇政院,而中政院亦兼主之,帝乃授之冊寶,其冊文曰:「坤以承乾元,人道莫先於夫婦;後以母天下,王化實始於家邦。典禮之常,古今攸重。咨爾肅良合氏,篤生名族,來事朕躬。儆戒相成,每勤於夙夜;恭儉率下,多歷於歲年。既發祥元子於儲闈,復流慶孫枝於甲觀。眷若中宮之位,允宜淑配之賢。宗戚大臣,況僉言而敷請;掖庭諸御,咸傾望以推尊。乃屢遜辭,尤可嘉尚。今遣攝太尉某持節授以玉冊玉寶,命爾為皇后。於戲!慎修壼政,益勉爾輔佐之心;昭嗣徽音,同保我延洪之福。其欽寵命,以衍壽祺。」二十八年,從帝北奔。



\end{pinyinscope}