\article{列傳第八}

\begin{pinyinscope}

 ○速不臺



 速不臺,蒙古兀良合人。其先世獵於斡難河上,遇敦必乃皇帝,因相結納,至太祖時,已五世矣。捍里必者生孛忽都,眾目為折里麻。折里麻者,漢言有謀略人也。三世孫合赤溫,生哈班。哈班二子,長忽魯渾,次速不臺,俱驍勇善騎射。太祖在班硃尼河時,哈班嘗驅群羊以進,遇盜,被執。忽魯渾與速不臺繼至,以槍剌之,人馬皆倒,餘黨逸去,遂免父難,羊得達於行在所。忽魯渾以百戶從帝與乃蠻部主戰於長城之南,忽魯渾射卻之,其眾奔闊赤檀山而潰。



 速不臺以質子事帝,為百戶。歲壬申,攻金桓州,先登,拔其城。帝命賜金帛一車。滅里吉部強盛不附,丙子,帝會諸將於禿兀剌河之黑林,問:「誰能為我征滅里吉者?」速不臺請行,帝壯而許之。乃選裨將阿里出領百人先行,覘其虛實。速不臺繼進。速不臺戒阿里出曰:「汝止宿,必載嬰兒具以行,去則遺之,使若挈家而逃者。」滅里吉見之,果以為逃者,遂不為備。己卯,大軍至蟾河,與滅里吉遇,一戰而獲其二將,盡降其眾。其部主霍都奔欽察,速不臺追之,與欽察戰於玉峪,敗之。壬午,帝徵回回國,其主滅里委國而去。命速不臺與只別追之,及於灰裏河,只別戰不利,速不臺駐軍河東,戒其眾人爇三炬以張軍勢,其王夜遁。復命統兵萬人由不罕川必里罕城追之,凡所經歷,皆無水之地。既度川,先發千人為游騎,繼以大軍晝夜兼行。比至,滅裏逃入海,不月餘,病死,盡獲其所棄珍寶以獻。帝曰:「速不臺枕乾血戰,為我家宣勞,朕甚嘉之。」賜以大珠、銀罌。癸未,速不臺上奏,請討欽察。許之。遂引兵繞寬定吉思海,展轉至太和嶺,鑿石開道,出其不意。至則遇其酋長玉里吉及塔塔哈兒方聚於不租河,縱兵奮擊,其眾潰走。矢及玉里吉之子,逃於林間,其奴來告而執之,餘眾悉降,遂收其境。又至阿里吉河,與斡羅思部大、小密赤思老遇,一戰降之,略阿速部而還。欽察之奴來告其主者,速不臺縱為民。還,以聞。帝曰:「奴不忠其主,肯忠他人乎?」遂戮之。又奏以滅里吉、乃蠻、怯烈、杭斤、欽察諸部千戶,通立一軍,從之。略也迷裏霍只部,獲馬萬匹以獻。帝欲征河西,以速不臺比年在外,恐父母思之,遣令歸省。速不臺奏,願從西征。帝命度大磧以往。丙戌,攻下撒里畏吾、特勤、赤閔等部,及德順、鎮戎、蘭、會、洮、河諸州,得牝馬五千匹,悉獻於朝。丁亥,聞太祖崩,乃還。己丑,太宗即位,以禿滅幹公主妻之。從攻潼關,軍失利,帝責之。睿宗時在籓邸,言兵家勝負不常,請令立功自效。遂命引兵從睿宗經理河南。道出牛頭關,遇金將合達帥步騎數十萬待戰。睿宗問以方略,速不臺曰:「城居之人不耐勞苦,數挑以勞之,戰乃可勝也。」師集三峰山,金兵圍之數匝。會風雪大作,其士卒殭僕,師乘之,殺戮殆盡。自是金軍不能復振。壬辰夏,睿宗還駐官山,留速不臺統諸道兵圍汴。癸巳,金主渡河北走,追敗之於黃龍岡,斬首萬餘級。金主復南走歸德府,未幾,復走蔡州。汴降,俘其後妃及寶器以獻,進圍蔡州。甲午,蔡州破,金主自焚死。時汴梁受兵日久,歲饑,人相食,速不臺下令縱其民北渡以就食。乙未,太宗命諸王拔都西征八赤蠻,且曰:「聞八赤蠻有膽勇,速不臺亦有膽勇,可以勝之。」遂命為先鋒,與八赤蠻戰。繼又令統大軍,遂虜八赤蠻妻子於寬田吉思海。八赤蠻聞速不臺至,大懼,逃入海中。辛丑,太宗命諸王拔都等討兀魯思部主也烈班,為其所敗,圍禿里思哥城,不克。拔都奏遣速不臺督戰,速不臺選哈必赤軍怯憐口等五十人赴之,一戰獲也烈班。進攻禿里思哥城,三日克之,盡取兀魯思所部而還。經哈咂里山,攻馬札兒部主怯憐。速不臺為先鋒,與諸王拔都、籲里兀、昔班、哈丹五道分進。眾曰:「怯憐軍勢盛,未可輕進。」速不臺出奇計,誘其軍至水郭寧河。諸王軍於上流,水淺,馬可涉,中復有橋。下流水深,速不臺欲結筏潛渡,繞出敵後。未渡,諸王先涉河與戰。拔都軍爭橋,反為所乘,沒甲士三十人,並亡其麾下將八哈禿。既渡,諸王以敵尚眾,欲要速不臺還,徐圖之。速不臺曰:「王欲歸自歸,我不至禿納河馬茶城,不還也。」及馳至馬茶城,諸王亦至,遂攻拔之而還。諸王來會,拔都曰:「漷寧河戰時,速不臺救遲,殺我八哈禿。」速不臺曰:「諸王惟知上流水淺,且有橋,遂渡而與戰,不知我於下流結筏未成,今但言我遲,當思其故。」於是拔都亦悟。後大會,飲以馬乳及蒲萄酒。言征怯憐時事,曰:「當時所獲,皆速不臺功也。」壬寅,太宗崩。癸卯,諸王大會,拔都欲不往。速不臺曰:「大王於族屬為兄,安得不往?」甲辰,遂會於也只裏河。丙午,定宗即位,既朝會,還家於禿剌河上。戊申卒,年七十三。贈效忠宣力佐命功臣、開府儀同三司、上柱國,追封河南王,謚忠定。子兀良合臺。



 兀良合臺,初事太祖。時憲宗為皇孫,尚幼,以兀良合臺世為功臣家,使護育之。憲宗在潛邸,遂分掌宿衛。歲癸巳,領兵從定宗征女真國,破萬奴於遼東。繼從諸王拔都征欽察、兀魯思、阿速、孛烈兒諸部。丙午,又從拔都討孛烈兒乃、捏迷思部,平之。己酉,定宗崩。拔都與宗室大臣議立憲宗,事久未決。四月,諸王大會,定宗皇后問所宜立,皆惶惑,莫敢對。兀良合臺對曰:「此議已先定矣,不可復變。」拔都曰:「兀良合臺言是也。」議遂定。憲宗即位之明年,世祖以皇弟總兵討西南夷烏蠻、白蠻、鬼蠻諸國,以兀良合臺總督軍事。其鬼蠻,即赤禿哥國也。癸丑秋,大軍自旦當嶺入雲南境。摩些二部猷長唆火脫因、塔裹馬來迎降,遂至金沙江。兀良合臺分兵入察罕章,蓋白蠻也,所在寨柵,以次攻下之。獨阿塔剌所居半空和寨,依山枕江,牢不可拔。使人覘之,言當先絕其汲道。兀良合臺率精銳立砲攻之。阿塔剌遣人來拒,兀良合臺遣其子阿術迎擊之,寨兵退走。遂並其弟阿叔城俱拔之。進師取龍首關,翊世祖入大理國城。甲寅秋,復分兵取附都善闡,轉攻合剌章水城,屠之。合剌章。蓋烏蠻也。前次羅部府,大酋高升集諸部兵拒戰,大破之於洟可浪山下,遂進至烏蠻所都押赤城。城際滇池,三面皆水,既險且堅,選驍勇以砲摧其北門,縱火攻之,皆不克。乃大震鼓鉦,進而作,作而止,使不知所為,如是者七日,伺其困乏,夜五鼓,遣其子阿術潛師躍入,亂斫之,遂大潰。至昆澤,擒其國王段興智及其渠帥馬合剌昔以獻。餘眾依阻山谷者,分命裨將也里、脫伯、押真掩其右,合臺護尉掩其左,約三日卷而內向。及圍合,與阿術引善射者二百騎,期以三日,四面進擊。兀良合臺陷陣鏖戰,又攻纖寨,拔之。至乾德哥城,兀良合臺病,委軍事於阿術。環城立砲,以草填塹,眾軍始集,阿術已率所部搏戰城上,城遂破。乙卯,攻不花合因、阿合阿因等城,阿術先登,取其三城。又攻赤禿哥山寨,阿術緣嶺而戰,遂拔之。乘勝擊破魯廝國塔渾城,又取忽蘭城。魯魯廝國大懼,請降。阿伯國有兵四萬,不降。阿術攻之,入其城,舉國請降。復攻阿魯山寨,進攻阿魯城,克之。乃搜捕未降者,遇赤禿哥軍於合打臺山,追赴臨崖,盡殺之。自出師至此,凡二年,平大理五城八府四郡,洎烏、白等蠻三十七部。兵威所加,無不款附。丙辰,征白蠻國、波麗國,阿術生擒其驍將,獻俘闕下。詔以便宜取道,與鐵哥帶兒兵合,遂出烏蒙,趨瀘江,刬禿剌蠻三城,卻宋將張都統兵三萬,奪其船二百艘於馬湖江,斬獲不可勝計。遂通道於嘉定、重慶,抵合州,濟蜀江,與鐵哥帶兒會。丁巳,以雲南平,遣使獻捷於朝,且請依漢故事,以西南夷悉為郡縣,從之。賜其軍銀五千兩、彩幣二萬四千匹,授銀印,加大元帥。還鎮大理,遂經六盤山至臨洮府,與大營合。月餘,復西征烏蠻。秋九月,遣使招降交趾,不報。冬十月,進兵壓境。其國主陳日煚,隔江列象騎、步卒甚盛。兀良合臺分軍為三隊濟江,徹徹都從下流先濟,大帥居中,駙馬懷都與阿術在後。仍授徹徹都方略曰:「汝軍既濟,勿與之戰,彼必來逆我,駙馬隨斷其後,汝伺便奪其船。蠻若潰走,至江無船,必為我擒矣。」師既登岸,即縱與戰,徹徹都違命,蠻雖大敗,得駕舟逸去。兀良合臺怒曰:「先鋒違我節度,軍有常刑。」徹徹都懼,飲藥死。兀良合臺入交趾,為久駐計,軍令嚴肅,秋毫無犯。越七日,日煚請內附,於是置酒大饗軍士。還軍柙赤城。戊午,引兵入宋境,其地炎瘴,軍士皆病,遇敵少卻,亡軍士四人。阿術還戰,擒其卒十二人,其援復至,阿術以三十騎,阿馬禿繼以五十騎擊走之。時兀良合臺亦病,將旋師,阿術戰馬五十匹夜為禿剌蠻所掠,入告兀良合臺曰:「吾馬盡為盜掠去,將何以行?」即分軍搜訪,知有三寨藏馬山顛。阿術親率將士攀崖而上,破其諸寨,生擒賊酋,盡得前後所盜馬千七百匹,乃屠柙赤城。憲宗遣使諭旨,約明年正月會軍長沙,乃率四王騎兵三千,蠻、僰萬人,破橫山寨,闢老蒼關,徇宋內地。宋陳兵六萬以俟。遣阿術與四王潛自間道沖其中堅,大敗之,盡殺其眾。乘勝擊逐,蹴貴州,蹂象州,入靜江府,連破辰、沅二州,直抵潭州城下。潭州出兵二十萬,斷我歸路。兀良合臺遣阿術與大納、玉龍帖木兒軍其前,而自與四王軍其後,夾擊破之。兵自入敵境,轉鬥千里,未嘗敗北。大小十三戰,殺宋兵四十餘萬,擒其將大小三人。其州又遣兵來攻,追至門濠,掩溺殆盡,乃不敢復出。壁城下月餘。時世祖已渡江駐鄂州,遣也里蒙古領兵二千人來援,且加勞問。遂自鄂州之滸黃洲北渡,與大軍合。庚申,世祖即位。夏四月,兀良合臺至上都。後十二年卒,年七十二。子阿術,自有傳。



 ○按竺邇



 按竺邇,雍古氏。其先居雲中塞上,父公,為金群牧使。歲辛未,驅所牧馬來歸太祖,終其官。按竺邇幼鞠於外祖術要甲家,訛言為趙家,因姓趙氏。年十四,隸皇子察合臺部。嘗從大獵,射獲數麋,有二虎突出,射之皆死。由是以善射名,皇子深器愛之。甲戌,太祖西征尋思乾、阿里麻裏等國,以功為千戶。丁亥,從征積石州,先登,拔其城。圍河州,斬首四十級。破臨洮,攻德順,斬首百餘級。攻鞏昌,駐兵秦州。



 太宗即位,尊察合臺為皇兄,以按竺邇為元帥。戊子,鎮刪丹州,自郭煌置驛抵玉關,通西域,從定關隴。辛卯,從圍鳳翔,按竺邇分兵攻西南隅,城上礌石亂下,選死士先登,拔其城,斬金將劉興哥。分兵攻西和州,宋將強俊領眾數萬,堅壁清野,以老我師。按竺邇率死士罵城下,挑戰。俊怒,悉眾出陣,按竺邇佯走,俊追之,因以奇兵奪其城。伏兵要其歸,轉戰數十里,斬首數千級,擒俊。餘眾退保仇池,進擊拔之,從拔平涼,慶陽、邠、原、寧皆降。涇州復叛,殺守將郭元恕,眾議屠之,按竺邇但誅首惡。師還原州,降民棄老幼,夜亡走。眾曰:「此必反也,宜誅之以警其餘。」按竺邇曰:「此輩懼吾驅之北徙耳。」遣人諭之曰:「汝等若走,以軍法治罪,父母妻子並誅矣。汝歸,保無他。明年草青,具牛酒迎師於此州。」民皆復歸。豪民陳茍集數千人潛新寨諸洞,眾議以火攻之。按竺邇曰:「招諭不出,攻之未晚。」遂偕數騎抵寨,縱馬解弓矢,召茍遙語,折矢與為誓。茍即相呼羅拜,謝更生之恩,皆降。



 金人守潼關,攻之,戰於扇車回,不克。睿宗分兵由山南入金境,按竺邇為先鋒,趣散關。宋人已燒絕棧道,復由兩當縣出魚關,軍沔州。宋制置使桂如淵守興元。按竺邇假道於如淵曰:「宋讎金久矣,何不從我兵鋒,一洗國恥。今欲假道南鄭,由金、洋達唐、鄧,會大兵以滅金,豈獨為吾之利?亦宋之利也。」如淵度我軍壓境,勢不徒還,遂遣人導我師由武休關東抵鄧州,西破小關。金人大駭,謂我軍自天而下。其平章完顏合達、樞密使移剌蒲阿帥十七都尉,兵數十萬,相拒於鄧。我師不與戰,直趣鈞州,與親王按赤臺等兵合,陳三峰山下。會天大雪,金兵成列。按竺邇先率所部精兵迎擊於前,諸軍乘之,金師敗績。癸巳,金主奔蔡。十二月,從圍蔡。甲午,金亡。初,金將郭斌自鳳翔突圍出,保金、蘭、定、會四州。至是命按竺邇往取之,圍斌於會州。食盡將走,敗之於城門。兵入城巷戰,死傷甚眾。斌手劍驅其妻子聚一室,焚之。已而自投火中。有女奴自火中抱兒出,泣授人曰:「將軍盡忠,忍使絕嗣,此其兒也,幸哀而收之。」言畢,復赴火死。按竺邇聞之惻然,命保其孤。遂定四州。金將汪世顯守鞏州,皇子闊端圍之,未下。遣按竺邇等往招之,世顯率眾來降。皇兄嘉其材勇,賞賚甚厚,賜名拔都,拜征行大元帥。



 丙申,大軍伐蜀,皇子出大散關,分兵令宗王穆直等出陰平郡,期會於成都。按竺邇領砲手兵為先鋒,破宕昌,殘階州。攻文州,守將劉祿,數月不下,諜知城中無井,乃奪其汲道,率勇士梯城先登,殺守陴者數十人,遂拔其城,祿死之。因招徠吐蕃酋長勘拖孟迦等十族,皆賜以銀符。略定龍州。遂與大散軍合,進克成都。師還,而成都復叛。丁酉,按竺邇言於宗王曰:「隴州縣方平,人心猶貳,西漢陽當隴蜀之沖,宋及吐蕃利於入寇,宜得良將以鎮之。」宗王曰:「安反側,制寇賊,此上策也,然無以易汝。」遂分蒙古千戶五人,隸麾下以往。按竺邇命侯和尚南戍沔州之石門,術魯西戍階州之兩水,謹斥堠,嚴巡邏,西南諸州不敢犯之。戊戌,從元帥塔海率諸翼兵伐蜀,克隆慶。己亥,攻重慶。庚子,圖萬州。宋人將舟師數百艘逆流迎戰。按竺邇順流率勁兵,乘巨筏,浮革舟於其間,弓弩兩射,宋人不能敵,敗諸夔門。辛丑,伐西川,破二十餘城。成都守將田顯開北門以納師。宋制置使陳隆之出奔,追獲之,縛至漢州,令誘降守將王夔。夔不降,進兵攻之。夔夜驅火牛,突圍出奔,遂斬隆之。壬寅,會大軍破遂寧、瀘、敘等州。癸卯,破資州。庚戌,按竺邇安輯涇、邠二州。宋制置使餘玠攻興元,文州降將王德新乘隙自階州叛,執扈、牛二鎮將,領眾千餘走江油。憲宗召按竺邇還舊鎮。按竺邇遣將直搗江油,奪扈、牛以歸。



 中統元年,世祖即位,親王有異謀者,其將阿藍答兒、渾都海圖據關隴。時按竺邇以老,委軍於其子。帝遣宗王哈丹、哈必赤、阿曷馬西討。按竺邇曰:「今內難方殷,浸亂關隴,豈臣子安臥之時耶?吾雖老,尚能破賊。」遂引兵出刪丹之耀碑谷,從阿曷馬,與之合戰。會大風,晝晦,戰至晡,大敗之,斬馘無算。按竺邇與總帥汪良臣獲阿藍答兒、渾都海等。捷聞,帝錫璽書褒美,賜弓矢錦衣。四年,卒,年六十九。延祐元年,贈推忠佐運功臣、太保、儀同三司、上柱國,封秦國公,謚武宣。



 子十人,徹理、國寶最知名。徹理襲職為元帥。丁巳,從父攻瀘州,降宋將劉整。宋將姚德壁雲頂山,戊午,大軍圍之。徹理率部兵由水門先登,破其壁,德降。後以病廢,卒。



 國寶一名黑梓,少擊劍學書,倜儻好義,有謀略。父為元帥,軍務悉以委之,故所至多捷。從攻重慶,降宋都統張實,並掠合州以歸。中統元年,從攻阿藍答兒有功。阿藍答兒叛將火都據吐蕃之點西嶺。國寶攝帥事,討之。眾欲速戰,國寶曰:「此窮寇也,宜少緩,以計破之。」遂以精兵襲其後。火都欲西走,國寶據險要之,挑戰則斂兵自固。相持兩月,潛兵出其不意,擒殺之。捷聞,賜弓矢、金綺。初,按竺邇之告老,制命徹理襲征行元帥。徹理以病不視事,國寶乃謂諸弟曰:「昔我先人,耀兵西陲,大功既集,關隴雖寧,而西戎未靖,此吾輩立功之秋也。」乃遣謝鼎與弟國能,持金帛說降吐蕃,酋長勘陀孟迦從國寶入覲。國寶奏曰:「文州山川險厄,控庸蜀,拒吐蕃,宜城文州,屯兵鎮之。」從之,授國寶三品印,為蒙古漢軍元帥,兼文州吐蕃萬戶府達魯花赤,與勘陀孟迦皆賜金符。時扶州諸羌未附,國寶宣上威德,於是呵哩禪波哩揭諸酋長皆歸款,從國寶入覲。國寶圖山川形勢以獻,詔授呵哩禪波哩揭為萬戶,賜金虎符,諸酋長為千戶,皆賜金符。賜國寶金幣。國寶治文州有善政。至元四年卒。延祐元年,贈推誠佐理功臣、光祿大夫、平章政事、柱國,封梁國公,謚忠定。



 子世榮、世延。初,國寶將卒,以世榮幼,命弟國安襲其職。國安既襲蒙古漢軍元帥,兼文州吐蕃萬戶府達魯花赤,後以其兄國寶安邊功,賜金虎符,進昭勇大將軍。十五年,討叛王吐魯於六盤,獲之,請解職授世榮。帝曰:「人爭而汝讓,可以敦薄俗。」錄其六盤功,進昭毅大將軍、招討使。世榮,襲懷遠大將軍、蒙古漢軍元帥,兼文州吐蕃萬戶府達魯花赤。後以功進安遠大將軍、吐蕃宣慰使議事都元帥,佩三珠虎符。世延,中書平章政事。



 ○畏答兒



 畏答兒,忙兀人。其先剌真八都兒,有二子,次名忙兀兒,始別為忙兀氏。畏答兒其六世孫也。與兄畏翼俱事太祖。時大疇強盛,畏翼率其屬歸之,畏答兒力止之,不聽,追之,又不肯還,畏答兒乃還事太祖。太祖曰:「汝兄既去,汝獨留此何為?」畏答兒無以自明,取矢折而誓曰:「所不終事主者,有如此矢。」太祖察其誠,更名為薛禪,約為按達。薛禪者,聰明之謂也;按達者,定交不易之謂也。太祖與克烈王罕對陳於哈剌真,師少不敵。帝命兀魯一軍先發,其將術徹臺橫鞭馬鬣不應。畏答兒奮然曰:「我猶鑿也,諸君斧也,鑿匪斧不入,我請先入,諸軍繼之,萬一不還,有三黃頭兒在,唯上念之。」遂先出陷陣,大敗之,至晡時,猶追逐不已,敕使止之,乃還。腦中流矢,創甚,帝親傅以善藥,留處帳中,月餘卒,帝深惜之。



 及王罕滅,帝以其將只里吉實抗畏答兒,乃分只里吉民百戶隸其子,且使世世歲賜不絕。仍令收完忙兀人民之散亡者。太宗思其功,復以北方萬戶封其子忙哥為郡王。歲丙申,忽都忽大料漢民,分城邑以封功臣,授忙哥泰安州民萬戶。帝訝其少,忽都忽對曰:「臣今差次,惟視舊數多寡,忙哥舊才八百戶。」帝曰:「不然,畏答兒封戶雖少,戰功則多,其增封為二萬戶,與十功臣同。為諸侯者,封戶皆異其籍。」兀魯爭曰:「忙哥舊兵不及臣之半,今封顧多於臣。」帝曰:「汝忘而先橫鞭馬鬣時耶?」兀魯遂不敢言。忙哥卒,孫只裏瓦鷿、乞答鷿,曾孫忽都忽、兀乃忽裏、哈赤,俱襲封為郡王。



 ○博羅歡伯都



 博羅歡,畏答兒幼子蘸木曷之孫,瑣魯火都之子也。時諸侯王及十功臣各有斷事官,博羅歡年十六,為本部斷事官。從世祖討阿里不哥,數有功,帝喜而賜馬四十匹,金幣稱之。中統三年,李璮叛。命帥忙兀一軍圍濟南,分兵掠益都、萊州,悉平之。詔錄燕南獄,讞決明允,賜衣一襲。皇子雲南王忽哥赤為其省臣寶合丁毒死,事覺,中書擇可治其獄者四人,奏上,皆不稱旨。丞相糸泉真以博羅歡聞,帝可其奏。博羅歡辭曰:「臣不敢愛死,第年少不知書,恐誤事耳。」帝乃以吏部尚書別帖木兒輔其行。未至雲南,寶合丁密以金六籝迎饋,祈勿究其事。博羅歡慮其握兵徼外,拒之恐致變,陽諾曰:「吾橐不能容,可且持歸,待我取之。」博羅歡至,則竟其獄,誅毒王者,而歸其金於省。陛見,帝顧謂糸泉真曰:「卿舉得其人矣。」賜黃金五十兩,詔忙兀事無大小,悉統於博羅歡。授昭勇大將軍、右衛親軍都指揮使,大都則專右衛,上都則兼三衛。



 會伐宋,授金吾衛上將軍、中書右丞。詔分大軍為二,右軍受伯顏、阿術節度,左軍受博羅歡節度。俄兼淮東都元帥,罷山東經略司,而以其軍悉隸焉。遂軍於下邳,召將佐謀曰:「清河城小而固,與昭信、淮安、泗州為掎角,猝未易拔。海州、東海、石秋,遠在數百里之外,必不嚴備。吾頓大兵為疑兵,以輕騎倍道襲之,其守將可擒也。」師至,三城果皆下,清河亦降。宋主以國內附,而淮東諸城猶為之守。詔博羅歡進軍,拔淮安南堡,戰白馬湖及寶應,掠高郵,自西小河入漕河,據灣頭,斷通、泰援兵,遂下揚州,淮東平。益封桂陽、德慶二萬一千戶。十四年,討叛臣只裏斡臺於應昌,平之。賜玉帶文綺,與博羅同署樞密院事,拜中書右丞,行省北京。未幾,召還。時江南新附,尚多反側,詔募民能從大軍進討者,使自為一軍,聽節度於其長,而毋役於他軍,制命符節,皆與正同。會博羅歡寢疾,乃附樞密董文忠奏曰:「今疆土浸廣,勝兵百萬,指揮可集,何假此無藉之徒。彼一踐南土,則掠人貨財,俘人妻孥,仇怨益滋,而叛者將愈眾矣。」奏上,召輿疾賜坐,與語,帝大悟,遂可其奏。而常德入訴唐兀一軍殘暴其境內,敕斬以徇。凡所募軍皆罷。



 十六年,以哈剌斯、博羅思、斡羅罕諸部不相統,命博羅歡監之。十八年,以中書右丞行省甘肅。二十年,拜御史大夫,行御史臺事,以疾歸。諸王乃顏叛,帝將親征。博羅歡諫曰:「昔太祖分封東諸侯,其地與戶,臣皆知之,以二十為率,乃顏得其九,忙兀、兀魯、扎剌兒、弘吉剌、亦其烈思五諸侯得其十一,惟征五諸侯兵,自足當之,何至上煩乘輿哉?臣疾且愈,請事東征。」帝乃賜鎧甲弓矢鞍勒,命督五諸侯兵,與乃顏戰,敗之。其黨塔不帶以兵來拒,會久雨,軍乏食,諸將欲退。博羅歡曰:「今兩陣相對,豈容先動?」俄塔不帶引兵退。博羅歡以其師乘之,轉戰二日,身中三矢,大破之,斬其駙馬忽倫。適太師月魯那演大軍來會,遂平乃顏,擒塔不帶。既而其黨哈丹復叛,詔與諸侯王乃馬帶討之。哈丹游騎猝至,博羅歡從三騎返走,抵絕澗,可二丈許,追騎垂及,博羅歡策其馬一躍而過,三從騎皆沒,人以為有神助云。哈丹死,斬其子老的於陣。往返凡四歲。凱旋,俘哈丹二妃以獻,敕以一賜乃馬帶,一賜博羅歡。陳其金銀器於延春閣,上召諸侯王將帥分賜之。博羅歡辭,帝曰:「卿可謂能讓。」乃賜金銀器五百兩以旌之。



 河南宣慰改行中書省,拜平章政事,有詔括馬毋及勛臣之家。博羅歡曰:「吾馬成群,所治地方三千里,不先出馬,何以為吏民之倡?」乃先入善馬十有八。汴南諸州,漭為巨浸,博羅歡躬行決口,督有司繕完之。三十一年,成宗立,遷陜西行中書省平章政事。未行,留鎮河南。入朝,請以泰安州所入五戶絲四千斤易內庫繒帛,分給忙兀一軍。帝為敕遞車送軍中,賜以銀百五十兩。陛辭,帝諭之曰:「卿今白須,世祖德言,實多聞之,宜加慎護。」因以世祖所佩弓矢鞶帶賜之。有頃,近臣奏:「伐宋時,右軍分屬伯顏、阿術,左軍分屬博羅歡。今伯顏、阿術皆受分地,而博羅歡未及,惟帝裁之。」帝曰:「何久不言,豈彼恥自請耶?」乃益封高郵五百戶。



 大德元年,叛王藥木忽兒、兀魯速不花來歸。博羅歡聞之,遣使馳奏曰:「諸王之叛,皆由其父,此輩幼弱,無所與知。今茲來歸,宜棄其前惡,以勸未至。」帝深以為然,賜金鞍勒,命以平章政事行省湖廣。會並福建行省入江浙,拜光祿大夫、上柱國、江浙等處行中書省平章政事。居歲餘,卒,年六十三。



 博羅歡勇有智略,戰常以身先之,所獲財物悉與將士,故得其死力。平居常以國事為憂,聞變即請行,至終其事乃止。其忠義蓋天性然也。累贈推忠宣力贊運功臣、太師、開府儀同三司、上柱國,加封泰安王,謚武穆。



 子渾都、伯都、野先帖木兒、博羅。渾都,山東宣慰使,遙授中書平章政事。野先帖木兒,河南江北等處行中書省左丞相。卒官開府儀同三司、翰林學士承旨。博羅,陜西等處行中書省平章政事。野先帖木兒子尼摩星吉,襲郡王;亦思剌瓦性吉,中政使。



 伯都幼穎異,不以家世自矜,長嗜書史。大德五年,擢江東道廉訪副使,拜江南行臺侍御史。未幾,召入僉樞密院事,領舍兒別赤。至大二年,出為江南行臺御史中丞,遷陜西行臺御史大夫。延祐元年,拜甘肅行省平章政事。時米價騰水勇,陸挽一石,費二百緡,乃為經畫計,所省至四百餘萬緡,自是諸倉俱充溢。甘州氣寒地瘠,少稔歲。民饑,則發粟賑之,春闕種,則貸之。於是兵餉既足,民食亦給。詔賜名鷹、甲胄、弓矢及鈔五千緡以勞焉。四年,移江浙行省平章政事,入為太子賓客。上書陳古先聖王正心修身之道,帝嘉納之。遷江南行臺御史大夫。皇太后謂東宮官不宜使外,止其行。遂以疾辭去,寓居高郵。英宗即位,復命為江南行臺御史大夫。陛見,以疾固辭。帝慰諭久之,命以平章之祿歸養於家,復賜鈔十萬緡。所服藥須空青,詔遣使江南訪求之。伯都辭謝曰:「臣曩膺重寄,深懼弗稱,今已病廢,況敢叨濫厚祿以受重賜乎?」並以所給平章政事祿歸有司。泰定元年,還京師,卒。朝廷知其貧,賻鈔二萬五千貫。御史臺奏賻三萬五千貫,仍還所辭祿,妻弘吉剌氏弗受,曰:「始伯都仕於朝,不敢虛受廩祿。今歿矣,茍受是祿,非其意也。」卒辭之。子篤爾只,將作院判官。



 ○抄思



 抄思,乃蠻部人。又號曰答祿。其先泰陽,為乃蠻部主。祖曲書律。父敞溫。太祖舉兵討不庭,曲書律失其部落,敞溫奔契丹卒。抄思尚幼,與其母跋涉間行,歸太祖,奉中宮旨侍宮掖。抄思年二十五,即從征伐,破代、石二州,不避矢石,每先登焉。雁門之戰,屢捷。會太宗命睿宗平金,抄思執銳以從,與金兵戰,所向無前。壬辰,兵次鈞州,金兵壘於三峰山,抄思察其營壁不堅,夜領精騎襲之,金兵驚擾,遂乘擊之,拔三峰山。睿宗以抄思功聞於朝,有旨以湯陰縣黃招撫等一百一十七戶賜之。抄思力辭不受。復賜以男女五十口,宅一區,黃金鞶帶、酒壺、杯盂各一。辭弗許,乃受之。制授萬戶,與內侍胡都虎、留乞簽起西京等處軍人征行及鎮守隨州。招集民戶,每千人以官一員領之。丁酉秋七月,奉旨調軍,得西京、大名、濱、棣、懷、孟、真定、河間、邢、名、磁、威、新、衛、保等府州軍四千六十餘人,統之。後移鎮潁,以疾歸大名。歲戊申正月卒,年四十四。子別的因。



 別的因在襁褓時,父抄思方領兵平金,與其祖母康里氏在三皇后宮庭。戊申,父抄思卒,母張氏迎別的因以歸。祖母康里氏卒。張嘗從容訓之曰:「人有三成人,知畏懼成人,知羞恥成人,知艱難成人。否則禽獸而已。」別的因受教唯謹。甲寅,世祖以宗王鎮黑水,有旨諭察罕那顏,命別的因襲抄思職,為副萬戶,鎮守隨、潁等處。丙辰冬十有二月,世祖復諭征鎮軍士悉聽別的因等號令。別的因身長七尺餘,肩豐多力,善刀舞,尤精騎射,士卒咸畏服之。明年,庚申,世祖即位,委任尤專。癸亥正月,召赴行在所。冬十一月,謁見世祖於行在所,世祖賜金符,以別的因為壽潁二州屯田府達魯花赤。時二州地多荒蕪,有虎食民妻,其夫來告,別的因默然良久,曰:「此易治耳。」乃立檻設機,縛羔羊檻中以誘虎。夜半,虎果至,機發,虎墮檻中,因取射之,虎遂死。自是虎害頓息。至元十三年,授明威將軍、信陽府達魯花赤,佩金符。時信陽亦多虎,別的因至未久,一日,以馬裼置鞍上出獵,命左右燔山,虎出走,別的因以裼擲虎,虎搏裼,據地而吼,別的因旋馬視虎射之,虎立死。十六年,進宣威將軍、常德路副達魯花赤。會同知李明秀作亂,別的因請以單騎往招之,直抵賊壘,賊輕之,不設備。別的因諭以朝廷恩德,使為自新計,明秀素畏服,遂與俱來。別的因聞於朝,明秀伏誅,賊遂平。三十一年,進懷遠大將軍,遷池州路達魯花赤。之官,道經潁上。潁近荊山,有野豕時出害民禾稼,民莫能制。聞別的因至,迎拜境上,告以其故。別的因曰:「毋慮也。」遂至荊山,以狼牙箭射之,豕走數里。大德十三年,進昭勇大將軍、臺州路達魯花赤。卒,年八十一。



 子不花,僉嶺南廣西道肅政廉訪司事;文圭,有隱德,贈秘書監著作郎;延壽,湯陰縣達魯花赤。孫守恭,曾孫與權,皆讀書登進士科,人多稱之。



\end{pinyinscope}