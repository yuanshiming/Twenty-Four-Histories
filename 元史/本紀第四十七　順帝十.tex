\article{本紀第四十七 順帝十}

\begin{pinyinscope}

 二十六年春正月己酉,以崇政院使孛羅沙為御史大夫。壬子,以完者木知樞密院事。是月,以沙藍答里為中書左丞相。命燕南、河南、山東、陜西、河東等處舉人會試者,增其額數制商賈事業。著作存《荀子》。參見「文學」、「教育」中的,進士及第以下遞升官一級。



 二月癸丑朔,立河淮水軍元帥府於孟津縣。甲戌,詔天下,以比者逆臣孛羅帖木兒、禿堅帖木兒、老的沙等干紀亂倫,內外之民經值軍馬,致使困乏,與免一切雜泛差徭。是月,擴廓帖木兒還河南,分立省部以自隨,尋居懷慶,又居彰德,調度各處軍馬,陜西張良弼拒命。



 三月癸未朔,罷洛陽嵩縣宣慰司。丁亥,白虹五道亙天,其第三道貫日,又有氣橫貫東南,良久始滅。甲午,擴廓帖木兒遣關保、虎林赤以兵西攻張良弼於鹿臺。李思齊、脫烈伯、孔興等兵皆與良弼合。以蠻子、脫脫木兒知樞密院事。乙未,廷試進士七十二人,賜赫德溥化、張棟進士及第,餘出身有差。監察御史玉倫普建言八事:一曰用賢,二曰申嚴宿衛,三曰保全臣子,四曰八衛屯田,五曰禁止奏請,六曰培養人才,七曰罪人不孥,八曰重惜名爵。帝嘉納之。是月,大明兵取高郵府。



 夏四月辛酉,詔立皇太子妃瓦只剌孫答里氏。是月,大明兵取淮安路、徐州、宿州、濠州、泗州、潁州、安豐路。五月壬午朔,洛陽瑞麥生,一莖四穗。甲辰,以脫脫不花為御史大夫。六月壬子朔,汾州介休縣地震。平遙縣大雨雹。紹興路山陰縣臥龍山裂。己未,命知樞密院事買閭以兵守直沽,命河間鹽運使拜住、曹履亨撫諭沿海灶戶,俾出丁夫從買閭征討。丙寅,詔:「英宗時謀為不軌之臣,其子孫或成丁者,可安置舊地,幼者隨母居草地,終身不得入京城及不得授官,止許於本愛馬應役。」皇后肅良合氏生日,百官進箋,皇后諭沙藍答里等曰:「自世祖以來,正宮皇后壽日,不曾進箋,近年雖行,不合典故。」卻之。



 秋七月辛巳朔,日有食之。徐溝縣地震,介休縣大水,石州大星如斗自西南而落。甲申,以李思齊為太尉。甲午,太白經天。丙申,擴廓帖木兒遣硃珍、盧旺屯兵河中,遣關保、虎林赤合兵渡河,會竹貞、商暠,且約李思齊以攻張良弼。良弼遣子弟質於思齊,與良弼拒守。關保等不利,思齊請詔和解之。丙午,太白經天。八月戊寅,以李國鳳為中書左丞,陳有定為福建行省平章政事。九月甲申,李思齊兵下鹽井,獲川賊餘繼隆,誅之。禮部侍郎滿尚賓、吏部侍郎掩篤剌哈自鳳翔還京師。先是,尚賓等持詔諭思齊開通川蜀道路,思齊方兵爭,不奉詔,尚賓等留鳳翔一年,至是始還。丙戌,以方國珍為江浙行省左丞相,弟國瑛、國氏,侄明善,並為江浙行省平章政事。己亥,以中書平章政事失列門為御史大夫。辛丑,孛星見東北方。



 冬十月甲子,擴廓帖木兒遣其弟脫因帖木兒及貊高、完哲等駐兵濟南,以控制山東。



 十一月甲申,大明兵取湖州路。丙申,大明兵取杭州路及紹興路。辛丑,大明兵取嘉興路。時湖州、杭州、紹興、嘉興、松江、平江諸路及無錫州皆張士誠所據。



 十二月庚午,蒲城洛水和順崖崩。



 二十七年春正月乙未,絳州夜聞天鼓鳴,將旦復鳴,其聲如空中戰鬥者。庚子,大明兵取松江府。癸卯,大明兵取沅州路。是月,李思齊、張良弼、脫列伯自會於含元殿基,推李思齊為盟主,同拒擴廓帖木兒。



 二月庚申,以買住為雲國公,七十為中書平章政事,月魯不花為御史大夫。乙丑,以詹事月魯帖木兒為御史大夫。



 三月丁丑朔,萊州大風,有大鳥至,其翅如席。擴廓帖木兒遣兵屯滕州以御王信。庚子,京師大風自西北起,飛砂揚礫,白日昏暗。



 夏四月丙子朔,白氣二道亙天。以去歲水潦霜災,嚴酒禁。戊寅,以空名宣敕遣付福建行省,命平章政事曲出、陳有定同驗有功者給之。辛巳,大同隕霜殺麥。癸未,福建行宣政院以廢寺錢糧由海道送京師。乙酉,以完者帖木兒為中書右丞相,辭以老病,不許。辛卯,以知樞密院事失列門為嶺北行省左丞相,提調分通政院。己亥,以俺普為中書平章政事。辛丑,擴廓帖木兒定擬其所屬官員二千六百一十人,從之。是月,山東地震,雨白氂。李思齊遣張良弼部將郭謙等守黃連寨,擴廓帖木兒部將關保、虎林赤、商暠、竹貞引兵拔其寨,郭謙走;會貊高等為變,關保、虎林赤夜遁,李思齊遂解而西。



 月丙午朔,日有食之,晝晦。丁巳,皇太子寢殿後新甃井中有龍出,光焰爍人,宮人震懾僕地。又長慶寺有龍纏繞槐樹飛去,樹皮皆剝。丁卯,沂州山崩。是月,知樞密院事壽安,奉空名宣敕與侯伯顏達世,令其以兵援擴廓帖木兒。時李思齊據長安,與商暠拒戰,侯伯顏達世進兵攻李思齊,秦州守將蕭公達降思齊。思齊知關保等兵退,遣蔡琳等破其營,侯伯顏達世奔潰。



 秋七月甲申,命也速提調武備寺。丁酉,絳州星隕,光耀如晝。是月,李思齊遣許國佐、薛穆飛會張良弼、脫列伯兵屯於華陰。時命禿魯為陜西行省左丞相,思齊不悅,遣其部將鄭應祥守陜西,而自還鳳翔。龍見於臨朐龍山,大石起立。八月丙午,詔命皇太子總天下兵馬,其略曰:「元良重任,職在撫軍,稽古征今,卓有成憲。曩者障塞決河,本以拯民昏墊,豈期妖盜橫造訛言,簧鼓愚頑,塗炭郡邑,殆遍海內,茲逾一紀。故察罕帖木兒仗義興師,獻功敵愾,汛掃汴洛,克平青齊,為國捐軀,深可哀悼。其子擴廓帖木兒克繼先志,用成駿功。愛猷識理達臘計安宗社,累請出師。朕以國本至重,詎宜輕出,遂授擴廓帖木兒總戎重寄,畀以王爵,俾代其行。李思齊、張良弼等,各懷異見,構兵不已,以致盜賊愈熾,深遺朕憂。況全齊密邇輦轂,儻失早計,恐生異圖,詢諸眾謀,僉謂皇太子聰明仁孝,文武兼資,聿遵舊典,爰命以中書令、樞密使,悉總天下兵馬,諸王、駙馬、各道總兵、將吏,一應軍機政務,生殺予奪,事無輕重,如出朕裁。其擴廓帖木兒,總領本部軍馬,自潼關以東,肅清江淮;李思齊總統本部軍馬,自鳳翔以西,與侯伯顏達世進取川蜀;以少保禿魯為陜西行中書省左丞相,本省駐札,總本部及張良弼、孔興、脫列伯各枝軍馬,進取襄樊;王信本部軍馬,固守信地,別聽調遣。詔書到日,汝等悉宜洗心滌慮,同濟時艱。」庚戌,貊高殺衛輝守御官餘仁輔、彰德守御官範國英,引軍至清化,聞懷慶有備,遂還彰德,上疏言:「人臣以尊君為本,以盡忠為心,以愛民為務。今總兵官擴廓帖木兒,歲與官軍仇殺,臣等乃朝廷培養之人,素知忠義,焉能俯首聽命。乞降明詔,別選重臣,以總大兵。」詔以擴廓帖木兒不遵君命,宜黜其兵權,就命貊高討之。辛亥,帖木兒不花進封淮王,賜金印,設王傅等官。壬子,為皇太子立大撫軍院,秩從一品,知院四員,同知二員,副使、同僉各一員,經歷、都事各二員,管勾一員。癸丑,封太師伯撒里永平王。甲寅,以右丞相完者帖木兒、翰林承旨答爾麻、平章政事完者帖木兒並知大撫軍院事。丙辰,完者帖木兒言:「大撫軍院專掌軍機,今後迤北軍務,仍舊制樞密院管,其餘內外諸王、駙馬、各處總兵、統兵、行省、行院、宣慰司一應軍情,不許隔越,徑行移大撫軍院。」詹事院同知李國鳳同知大撫軍院事,參政完者帖木兒為副使,左司員外郎咬住、樞密參議王弘遠為經歷。庚申,完者帖木兒言:「諸軍將士有能用命效力建立奇功者,請所賞宣敕依常制外,加以忠義功臣之號。」從之。辛酉,以完者帖木兒仍前少師、知樞密院事,也速仍前太保、中書右丞相,帖裏帖木兒以太尉、添設中書左丞相。丙寅,立行樞密院於阿難答察罕腦兒,命陜西行省左丞相禿魯仍前少保兼知行樞密院事。壬申,命帖裏帖木兒仍前太尉、左丞相,為知大撫軍院事;中書右丞陳敬伯為中書平章政事。九月甲戌朔,義士戴晉生上皇太子書,言治亂之由。命右丞相也速以兵往山東,命參知政事法都忽剌分戶部官,一同供給。丁亥,以兵興,迤南百姓供給繁重,其真定、河南、陜西、山東、冀寧等處,除軍人自耕自食外,與免民間今年田租之半,其餘雜泛一切住罷。辛巳,大明兵取平江路,執張士誠。乙酉,大明兵取通州。丁亥,大明兵取無錫州。己丑,詔也速以中書右丞相分省山東,沙藍答里以中書左丞相分省大同。丙申,太師汪家奴追封兗王,謚忠靖。己亥,命帖裏帖木兒提調端本堂及領經筵事。辛丑,大明兵取臺州路。時臺州、溫州、慶元三路皆方國珍所據。



 冬十月甲辰朔,貊高以兵入山西,定孟州、忻州,下漷州,遂攻真定。詔也速自河間以兵會貊高取真定,已而不克,命也速還河間,貊高還彰德。乙巳,皇太子奏以淮南行省平章政事王信為山東行省平章政事兼知行樞密院事。立中書分省於真定路。丙午,加司徒、淮南行省平章政事王宣為沂國公。丁未,享於太廟。壬子,詔擴廓帖木兒落太傅、中書左丞相並諸兼領職事,仍前河南王,錫以汝州為其食邑;其弟脫因帖木兒以集賢學士同擴廓帖木兒於河南府居。其帳前諸軍,命瑣住、虎林赤一同統之。其河南諸軍,命中書平章政事、內史李克彞統之。關保本部諸軍仍舊統之。山東諸軍,命太保、中書右丞相也速統之。山西諸軍,命少保、中書左丞相沙藍答里統之。河北諸軍,命知樞密院事貊高統之。赦天下。甲寅,以火里赤為中書平章政事。乙丑,命集賢大學士丁好禮為中書添設平章政事。丙寅,平章、內史關保封許國公。己巳,大明兵取溫州。



 十一月壬午,大明兵取沂州,守臣王信遁,其父宣被執。癸未,大明兵取慶元路。丙戌,以平章政事月魯帖木兒,知樞密院事完者帖木兒,平章政事伯顏帖木兒、帖林沙並知撫軍院事。戊子,大明兵取嶧州。乙未,以知樞密院事貊高為中書平章政事。命太尉、中書左丞相帖裏帖木兒為撫軍院使。丁酉,命帖裏帖木兒同監修國史。命關保分省於晉寧。辛丑,大明兵取益都路,平章政事保保降,宣慰使普顏不花、總管胡濬、知院張俊皆死之。



 十二月癸卯朔,日有食之。丁未,大明兵取般陽路。戊申,大明兵取濟寧路,陳秉直遁。己酉,大明兵取萊州,遂取濟南及東平路。丁巳,大明兵入杉關,取邵武路。時邵武、建寧、延平、福州、興化、泉、漳、汀、潮諸路,皆陳友定所據。庚申,以楊誠、陳秉直並為國公、中書平章政事。甲子,命右丞相也速,太尉知院脫火赤,中書平章政事忽林臺,平章政事貊高,知樞密院事小章、典堅帖木兒、江文清、驢兒等會楊誠、陳秉直、伯顏不花、俞勝各部諸軍同守禦山東,又命關保往援山東。丙寅,以莊家為中書參知政事。庚午,大明兵由海道取福州,守臣平章政事曲出遁,行宣政院使朵耳死之。是月,方國珍歸於大明。詔命陜西行省左丞相禿魯總統張良弼、脫列伯、孔興各枝軍馬,以李思齊為副總統,御關中,撫安軍民。脫列伯、孔興等出潼關,及取順便山路,渡黃河合勢東行,共勤王事。思齊等皆不奉命。是歲,詔分潼關以西屬李思齊,以東屬擴廓帖木兒,各罷兵還鎮。於是關保退屯潞州,商暠留屯潼關。



 二十八年春正月壬申朔,皇太子命關保固守晉寧,總統諸軍,如擴廓帖木兒拒命,當以大義相裁,就便擒擊。以中書平章政事不顏帖木兒為御史大夫。辛巳,詔諭擴廓帖木兒曰:「比者也速上奏,卿以書陳情,深自悔悟,及省來意,良用惻然。朕視卿猶子,卿何惑於憸言,不體朕心,隳其先業!卿今能自悔,固朕所望。卿其思昔委任肅清江淮之意,即將冀寧、真定諸軍,就行統制渡河,直搗徐沂,以康靖齊魯,則職任之隆,當悉還汝。衛輝、彰德、順德,皆為王城,卿無以貊高為名,縱軍侵暴。其晉寧諸軍,已命關保總制策應,戡定山東,將帥各宜悉心。」庚寅,彗星見於昴、畢之間。是月,大明兵取建寧、延平二路,陳有定被執。



 二月壬寅朔,詔削擴廓帖木兒爵邑,命禿魯、李思齊等討之,詔曰:「擴廓帖木兒本非察罕帖木兒之宗,俾嗣職任,冀承遺烈,畀以相位,陟以師垣,崇以王爵,授以兵柄,顧乃憑藉寵靈,遂肆跋扈,構兵關陜,專事吞並。貊高倡明大義,首發奸謀,關保弗信邪言,乃心王室,陳其罪惡,請正邦典。今禿魯、李思齊,其率兵東下,共行天討。」癸卯,武庫災。癸丑,大明兵取東昌路,守將申榮、王輔元死之。丙辰,擴廓帖木兒自澤州退守晉寧,關保守澤、潞二州,與貊高軍合。己未,大明兵取寶慶路。甲子,汀州路總管陳穀珍以城降於大明。丙寅,大明兵取棣州。是月,大明兵至河南,李思齊、張良弼等解兵西還。詔命知樞密院事脫火赤、平章政事魏賽因不花進兵攻晉寧。李思齊次渭南,張良弼次櫟陽。興化、泉州、漳州、潮州四路皆降於大明。



 三月庚寅,彗星見於西北。壬辰,翰林學士承旨王時、太常院使陳祖仁上章,乞撫諭擴廓帖木兒,以兵勤王赴難。是月,有星流於東北,眾小星隨之,其聲大震。大明兵取河南。李思齊、張良弼會兵駐潼關,火焚良弼營,思齊移軍葫蘆灘,調其所部張德斂、穆薛飛守潼關。大明兵入潼關,攻李思齊營,思齊棄輜重,奔於鳳翔。是月,大明兵取永州路,又取惠州路。



 夏四月辛丑朔,大明兵取英德州。丙午,隕霜殺菽。戊申,大明兵取廣州路,又取嵩、陜、汝等州。五月庚午朔,大明兵取道州。李克彞棄河南城,奔陜西,推李思齊為總兵,駐兵岐山。是月,李思齊部將忽林赤、脫脫、張意據盩啡,高暠據武功,李克彞據岐山,任從政據隴州。六月庚子朔,徐溝縣地震。癸丑,大明兵取全州、郴州、梧州、藤州、潯州、貴、象、鬱林等郡。甲寅,雷雨中有火自天墜,焚大聖壽萬安寺。壬戌,臨州、保德州地震,五日不止。大明兵取靜江路。是月,廣西諸郡縣皆附於大明。



 秋七月癸酉,京城紅氣滿空,如火照人,自旦至辰方息。乙亥,京城黑氣起,百步內不見人,從寅至巳方消。貊高、關保以兵攻晉寧。是月,李思齊會李克彞、商暠、張意、脫列伯等於鳳翔。海南、海北諸郡縣皆降於大明。閏月己亥朔,擴廓帖木兒與貊高、關保戰,敗之,擒關保、貊高,遣其斷事官以聞。詔:「關保、貊高,間諜構兵,可依軍法處治。」關保、貊高皆被殺。辛丑,大明兵取衛輝路。癸卯,大明兵取彰德路。乙巳,左江、右江諸路皆降於大明。丁未,大明兵取廣平路。丁巳,詔罷大撫軍院,誅知大撫軍院事伯顏帖木兒等。詔復命擴廓帖木兒仍前河南王、太傅、中書左丞相,統領見部軍馬,由中道直抵彰德、衛輝;太保、中書右丞相也速統率大軍,經由東道,水陸並進;少保、陜西行省左丞相禿魯統率關陜諸軍,東出潼關,攻取河洛;太尉、平章政事李思齊統率軍馬,南出七盤、金、商,克復汴洛。四道進兵,掎角剿捕,毋分彼此。秦國公、平章、知院俺普,平章瑣住等軍,東西布列,乘機掃殄。太尉、遼陽左丞相也先不花,郡王、知院厚孫等軍,捍御海口,籓屏畿輔。皇太子愛猷識理達臘悉總天下兵馬,裁決庶務,具如前詔。壬戌,白虹貫日。癸亥,罷內府河役。甲子,擴廓帖木兒自晉寧退守冀寧。大明兵至通州。知樞密院事卜顏帖木兒力戰,被擒死之。左丞相失列門傳旨,令太常禮儀院使阿魯渾等,奉太廟列室神主與皇太子同北行。阿魯渾等即至太廟,與署令王嗣宗、太祝哈剌不華襲護神主畢,仍留室內。乙丑,白虹貫日。罷內府興造。詔淮王帖木兒不花監國,慶童為中書左丞相,同守京城。丙寅,帝御清寧殿,集三宮後妃、皇太子、皇太子妃,同議避兵北行。失列門及知樞密院事黑廝、宦者趙伯顏不花等諫,以為不可行,不聽。伯顏不花慟哭諫曰:「天下者,世祖之天下,陛下當以死守,奈何棄之!臣等願率軍民及諸怯薛歹出城拒戰,願陛下固守京城。」卒不聽。至夜半,開健德門北奔。八月庚午,大明兵入京城,國亡。



 後一年,帝駐於應昌府,又一年,四月丙戌,帝因痢疾殂於應昌,壽五十一,在位三十六年。太尉完者、院使觀音奴奉梓宮北葬。五月癸卯,大明兵襲應昌府,皇孫買的裡八剌及后妃並寶玉皆被獲,皇太子愛猷識禮達臘從十數騎遁。大明皇帝以帝知順天命,退避而去,特加其號曰順帝,而封買的裡八剌為崇禮侯。



\end{pinyinscope}