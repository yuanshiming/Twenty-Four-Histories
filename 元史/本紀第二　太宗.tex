\article{本紀第二 太宗}

\begin{pinyinscope}

 太宗英文皇帝,諱窩闊臺,太祖第三子。母曰光獻皇后,弘吉剌氏。太祖伐金、定西域,帝攻城略地之功居多。太祖崩,自霍博之地來會喪。



 元年己丑夏,至忽魯班雪不只之地,皇弟拖雷來見。秋八月己未,諸王百官大會於怯綠連河曲雕阿蘭之地,以太祖遺詔即皇帝位於庫鐵烏阿剌里。始立朝儀,皇族尊屬皆拜。頒大札撒。華言大法令也。金遣阿虎帶來歸太祖之賵,帝曰:「汝主久不降,使先帝老於兵間,吾豈能忘也,賵何為哉!」卻之,遂議伐金。敕蒙古民有馬百者輸牝馬一,牛百者輸牸牛一,羊百者輸羒羊一,為永制。始置倉廩,立驛傳。命河北漢民以戶計,出賦調,耶律楚材主之;西域人以丁計,出賦調,麻合沒的滑剌西迷主之。印度國主、木羅夷國主來朝。西域伊思八剌納城酋長來降。是歲,金復遣使來聘,不受。



 二年庚寅春正月,詔自今以前事勿問。定諸路課稅,酒課驗實息十取一,雜稅三十取一。是春,帝與拖雷獵於斡兒寒河,遂遣兵圍京兆。金主率師來援,敗之,尋拔其城。夏,避暑於塔密兒河。朵忽魯及金兵戰,敗績,命速不臺援之。



 秋月,帝自將南伐,皇弟拖雷、皇侄蒙哥率師從,拔天成等堡,遂渡河攻鳳翔。冬十一月,始置十路征收課稅使,以陳時可、趙昉使燕京,劉中、劉桓使宣德,周立和、王貞使西京,呂振、劉子振使太原,楊簡、高廷英使平陽,王晉、賈從使真定,張瑜、王銳使東平,王德亨、侯顯使北京,夾谷永、程泰使平州,田木西、李天翼使濟南。是月,師攻潼關、藍關,不克。十二月,拔天勝寨及韓城、蒲城。



 三年辛卯春二月,克鳳翔,攻洛陽、河中諸城,下之。夏五月,避暑於九十九泉。命拖雷出師寶雞。遣搠不罕使宋假道,宋殺之。復遣李國昌使宋需糧。秋八月,幸雲中。始立中書省,改侍從官名,以耶律楚材為中書令,粘合重山為左丞相,鎮海為右丞相。是月,以高麗殺使者,命撒禮塔率師討之,取四十餘城。高麗王皞遣其弟懷安公請降,撒禮塔承制設官分鎮其地,乃還。冬十月乙卯,帝圍河中。十二月己未,拔之。



 四年壬辰春正月戊子,帝由白坡渡河。庚寅,拖雷渡漢江,遣使來報,即詔諸軍進發。甲午,次鄭州。金防城提控馬伯堅降,授伯堅金符,使守之。丙申,大雪。丁酉,又雪。次新鄭。是日,拖雷及金師戰於鈞州之三峰,大敗之,獲金將蒲阿。戊戌,帝至三峰。壬寅,攻鈞州,克之,獲金將合達,遂下商、虢、嵩、汝、陜、洛、許、鄭、陳、亳、潁、壽、睢、永等州。三月,命速不臺等圍南京,金主遣其弟曹王訛可入質。帝還,留速不臺守河南。夏四月,出居庸,避暑官山。高麗叛,殺所置官吏,徙居江華島。秋七月,遣唐慶使金諭降,金殺之。八月,撒禮塔復征高麗,中矢卒。金參政完顏思烈、恆山公武仙救南京,諸軍與戰,敗之。九月,拖雷薨,帝還龍庭。冬十一月,獵於納蘭赤剌溫之野。十二月,如太祖行宮。



 五年癸巳春正月庚申,金主奔歸德。戊辰,金西面元帥崔立殺留守完顏奴申、完顏習捏阿不,以南京降。二月,幸鐵列都之地。詔諸王議伐萬奴,遂命皇子貴由及諸王按赤帶將左翼軍討之。夏四月,速不臺進至青城,崔立以金太后王氏、後徒單氏及梁王從恪、荊王守純等至軍中,速不臺遣送行在,遂入南京。六月,金主奔蔡,塔察兒率師圍之。詔以孔子五十一世孫元措襲封衍聖公。秋八月,獵於兀必思地。以阿同葛等充宣差勘事官,括中州戶,得戶七十三萬餘。九月,擒萬奴。冬十一月,宋遣荊鄂都統孟珙以兵糧來助。十二月,諸軍與宋兵合攻蔡,敗武仙於息州,金人以海、沂、萊、濰等州降。是冬,帝至阿魯兀忽可吾行宮。大風霾七晝夜。敕修孔子廟及渾天儀。



 六年甲午春正月,金主傳位於宗室子承麟,遂自經而焚。城拔,獲承麟,殺之。宋兵取金主餘骨以歸,金亡。是春,會諸王,宴射於斡兒寒河。夏五月,帝在達蘭達葩之地,大會諸王百僚,諭條令曰:「凡當會不赴而私宴者,斬。諸出入宮禁,各有從者,男女止以十人為朋,出入毋得相雜。軍中凡十人置甲長,聽其指揮,專擅者論罪。其甲長以事來宮中,即置權攝一人、甲外一人,二人不得擅自往來,違者罪之。諸公事非當言而言者,拳其耳;再犯,笞;三犯,杖;四犯,論死。諸千戶越萬戶前行者,隨以木鏃射之。百戶、甲長、諸軍有犯,其罪同。不遵此法者,斥罷。今後來會諸軍,甲內數不足,於近翼抽補足之。諸人或居室,或在軍,毋敢喧呼。凡來會,用善馬五十匹為一羈,守者五人,飼羸馬三人,守乞烈思三人。但盜馬一二者,即論死。諸人馬不應絆於乞烈思內者,輒沒與畜虎豹人。諸婦人制質孫燕服不如法者,及妒者,乘以驏牛徇部中,論罪,即聚財為更娶。」秋七月,以胡土虎那顏為中州斷事官。遣達海紺卜征蜀。是秋,帝在八里裡答闌答八思之地,議自將伐宋,國王查老溫請行,遂遣之。冬,獵於脫卜寒地。



 七年乙未春,城和林,作萬安宮。遣諸王拔都及皇子貴由、皇侄蒙哥征西域,皇子闊端徵秦、鞏,皇子曲出及胡土虎伐宋,唐古征高麗。秋九月,諸王口溫不花獲宋何太尉。冬十月,曲出圍棗陽,拔之,遂徇襄、鄧,入郢,虜人民牛馬數萬而還。十一月,闊端攻石門,金便宜都總帥汪世顯降。中書省臣請契勘《大明歷》,從之。



 八年丙申春正月,諸王各治具來會宴。萬安宮落成。詔印造交鈔行之。二月,命應州郭勝、鈞州孛術魯九住、鄧州趙祥從曲出充先鋒伐宋。三月,復修孔子廟及司天臺。夏六月,復括中州戶口,得續戶一百一十餘萬。耶律楚材請立編修所於燕京,經籍所於平陽,編集經史,召儒士梁陟充長官,以王萬慶、趙著副之。



 秋月,命陳時可閱刑名、科差、課稅等案,赴闕磨照。詔以真定民戶奉太后湯沐,中原諸州民戶分賜諸王、貴戚、斡魯朵:拔都,平陽府;茶合帶,太原府;古與,大名府;孛魯帶,邢州;果魯干,河間府;孛魯古帶,廣寧府;野苦,益都、濟南二府戶內撥賜;按赤帶,濱、棣州;斡陳那顏,平、灤州;皇子闊端、駙馬赤苦、公主阿剌海、公主果真、國王查剌溫、茶合帶、鍛真、蒙古寒札、按赤那顏、圻那顏、火斜、術思,並於東平府戶內撥賜有差。耶律楚材言非便,遂命各位止設達魯花赤,朝廷置官吏收其租頒之,非奉詔不得徵兵賦。闊端率汪世顯等入蜀,取宋關外數州,斬蜀將曹友聞。冬十月,闊端入成都。詔招諭秦、鞏等二十餘州,皆降。皇子曲出薨。張柔等攻郢州,拔之。襄陽府來附,以游顯領襄陽、樊城事。



 九年丁酉春,獵於揭揭察哈之澤。蒙哥徵欽察部,破之,擒其酋八赤蠻。夏四月,築掃鄰城,作迦堅茶寒殿。六月,左翼諸部訛言括民女,帝怒,因括以賜麾下。秋八月,命術虎乃、劉中試諸路儒士,中選者除本貫議事官,得四千三十人。冬十月,獵於野馬川,幸龍庭,遂至行宮。是冬,口溫不花等圍光州,命張柔、鞏彥暉、史天澤攻下之,遂別攻蘄州,降隨州,略地至黃州,宋懼請和,乃還。



 十年戊戌春,塔思軍至北峽關,宋將汪統制降。夏,襄陽別將劉義叛,執游顯等降宋。宋兵復取襄、樊。帝獵於揭揭察哈之澤。築圖蘇湖城,作迎駕殿。秋八月,陳時可、高慶民等言諸路旱蝗,詔免今年田租,仍停舊未輸納者,俟豐歲議之。



 十一年己亥春,復獵於揭揭察哈之澤。皇子闊端軍至自西川。秋七月,游顯自宋逃歸。以山東諸路災,免其稅糧。冬十一月,蒙哥率師圍阿速蔑怯思城,閱三月,拔之。十二月,商人奧都剌合蠻買撲中原銀課二萬二千錠,以四萬四千錠為額,從之。



 十二年庚子春正月,以奧都剌合蠻充提領諸路課稅所官。皇子貴由克西域未下諸部,遣使奏捷。命張柔等八萬戶伐宋。冬十二月,詔貴由班師。敕州郡失盜不獲者,以官物償之。國初,令民代償,民多亡命,至是罷之。是歲,以官民貸回鶻金償官者歲加倍,名羊羔息,其害為甚,詔以官物代還,凡七萬六千錠。仍命凡假貸歲久,惟子本相侔而止,著為令。籍諸王大臣所俘男女為民。



 十三年辛丑春二月,獵於揭揭察哈之澤。帝有疾,詔赦天下囚徒。帝瘳。秋,高麗國王王皞以族子綧入質。冬十月,命牙老瓦赤主管漢民公事。十一月丁亥,大獵。庚寅,還至鈋鐵金辜胡蘭山。奧都剌合蠻進酒,帝歡飲,極夜乃罷。辛卯遲明,帝崩於行殿。在位十三年,壽五十有六。葬起輦穀。追謚英文皇帝,廟號太宗。



 帝有寬弘之量,忠恕之心,量時度力,舉無過事,華夏富庶,羊馬成群,旅不齎糧,時稱治平。



 壬寅年春,六皇后乃馬真氏始稱制。秋七月,張柔自五河口渡淮,攻宋揚、滁、和等州。



 癸卯年春正月,張柔分兵屯田於襄城。夏五月,熒惑犯房星。秋,後命張柔總兵戍杞。



 甲辰年夏五月,中書令耶律楚材薨。



 乙巳年秋,後命馬步軍都元帥察罕等率騎三萬與張柔掠淮西,攻壽州,拔之,遂攻泗州、盱眙及揚州。宋制置趙蔡請和,乃還。



 ◎定宗



 定宗簡平皇帝,諱貴由,太宗長子也。母曰六皇后,乃馬真氏,以丙寅年生帝。太宗嘗命諸王按只帶伐金,帝以皇子從,虜其親王而歸。又從諸王拔都西征,次阿速境,攻圍木柵山寨,以三十餘人與戰,帝及憲宗與焉。太宗嘗有旨以皇孫失烈門為嗣。太宗崩,皇后臨朝,會諸王百官於答蘭答八思之地,遂議立帝。



 元年丙午春正月,張柔入覲於和林。秋七月,即皇帝位於汪吉宿滅禿里之地。帝雖御極,而朝政猶出於六皇后云。冬,獵黃羊於野馬川。權萬戶史權等耀兵淮南,攻虎頭關寨,拔之,進圍黃州。



 二年丁未春,張柔攻泗州。夏,避暑於曲律淮黑哈速之地。秋,西巡。八月,命野裡知吉帶率搠思蠻部兵征西。是月,詔蒙古人戶每百以一名充拔都魯。九月,取太宗宿衛之半,以也曲門答兒領之。冬十月,括人戶。



 三年戊申春三月,帝崩於橫相乙兒之地。在位三年,壽四十有三。葬起輦穀。追謚簡平皇帝,廟號定宗。是歲大旱,河水盡涸,野草自焚,牛馬十死八九,人不聊生。諸王及各部又遣使於燕京迤南諸郡,徵求貨財、弓矢、鞍轡之物,或於西域回鶻索取珠璣,或於海東樓取鷹鶻,馹騎絡繹,晝夜不絕,民力益困。然自壬寅以來,法度不一,內外離心,而太宗之政衰矣。



 己酉年。



 庚戌年。



 定宗崩後,議所立未決。當是時,已三歲無君,其行事之詳,簡策失書,無從考也。



\end{pinyinscope}