\article{本紀第十九 成宗二}

\begin{pinyinscope}

 二年春正月丙子,詔蠲兩都站戶和顧和市。己卯,詔江南毋捕天鵝。以忽剌出千戶所部屯夫貧乏,免其所輸租。上思州叛賊黃勝許攻剽水口思光寨,湖廣行省調兵擊破之存」。後收入《嵇康集》。,獲其黨黃法安等,賊遁入上牙六羅。壬午,太陰犯輿鬼。詔凡戶隸貴赤者,諸人毋爭。甲申,命西平王奧魯赤今夏居上都。丙戌,太白晝見。安西王傅鐵赤、脫鐵木而等復請立王相府,帝曰:「去歲阿難答已嘗面陳,朕以世祖定制諭之,今復奏請,豈欲以四川、京兆悉為彼有耶?賦稅、軍站,皆朝廷所司,今姑從汝請,置王相府,惟行王傅事。」丁亥,太陰犯平道。己丑,御史臺臣言:「漢人為同寮者,嘗為奸人捃摭其罪,由是不敢盡言。請於近侍昔寶赤、速古而赤中,擇人用之。」帝曰:「安用此曹?其選漢人識達事體者為之。」以御史中丞禿赤為御史大夫。庚寅,太陰犯鉤鈐。辛卯,令月赤察而也可及合剌赤所部衛士自運軍糧,給其行費。甲午,授嗣漢三十八代天師張與材太素凝神廣道真人,管領江南諸路道教。乙未,詔諸王、公主、駙馬非奉旨毋罪官吏,賜諸王合班妃鈔千二百錠、雜幣帛千匹,駙馬塔海鐵木而鈔三千錠。回紇不剌罕獻獅、豹、藥物,賜鈔千三百餘錠。



 二月乙亥朔,中書省臣言:「陛下自御極以來,所賜諸王、公主、駙馬、勛臣,為數不輕,向之所儲,散之殆盡。今繼請者尚多,臣等乞甄別貧匱及赴邊者賜之,其餘宜悉止。」從之。分江浙行省軍萬人戍湖廣。給稱海屯田軍農具。詔奉使及軍官歿而子弟未襲職者,其所佩金銀符歸於官,違者罪之。辛丑,立中御府,以脫忽伯、唐兀並為中御卿。丙午,禁軍將擅易侍衛軍、蒙古軍,以家奴代役者罪之,仍令其奴別入兵籍,以其主資產之半畀之,軍將敢有縱之者,罷其職。括蒙古戶漸丁,以充行伍。丁未,太陰犯井。庚戌,詔軍卒擅更代及逃歸者死。給禿禿合所部屯田農器。丙辰,詔江南道士貿易、田者,輸田、商稅。庚申,命札剌而忽都虎所部戶居於奉聖、雲州者,與民均供徭役。自六盤山至黃河立屯田,置軍萬人。丙寅,以大都留守司達魯花赤段貞為中書平章政事。遣使代祀岳瀆。賜安西王米三千石,以賑饑民。



 三月壬申,以中書平章政事不忽木為昭文館大學士,平章軍國事。罷太原、平陽路釀進蒲萄酒,其蒲萄園民恃為業者,皆還之。諸王出伯言所部探馬赤軍懦弱者三千餘人,乞代以強壯,從之,仍命出伯非奉旨毋擅徵發。以怯魯剌駐夏民饑,戶給糧六月。郡王慶童有疾,以其子也裏不花代之。賜八撒、火而忽答孫、禿剌三人鈔各千錠。治書侍御史萬僧受贓,命御史臺與宣政院使答失蠻雜治之。癸酉,增駐夏軍為四萬人。忻都言晉王甘麻剌,朵兒帶言月兒魯,皆有異圖,詔樞密院鞫之,無驗。帝命言晉王者死,言月兒魯者謫從軍自效。詔雲南行臺檢劾亦乞不薛宣慰司案牘。甲戌,遣諸王亦只里、八不沙、亦憐真、也裏慳、甕吉剌帶並駐夏於晉王怯魯剌之地。丙子,車駕幸上都。丁丑,以完顏邦義、納速丁、劉季安妄議朝政,杖之,徒二年,籍其家財之半。甲申,次大口。乙酉,太陰犯鉤鈐。辛卯,賜遼陽行省糧三萬石。壬辰,詔駙馬亦都護括流散畏吾而戶。癸巳,湖廣行省以叛賊黃勝許黨魯萬丑、王獻於京師。賜諸王鐵木兒金二百五十兩、銀二千五百兩、鈔五千錠,以旌其戰功。以合伯及塔塔剌所部民饑,賑米各千石。



 夏四月己亥朔,命撒的迷失招集其祖忙兀臺所部流散人戶。賜諸王八卜沙鈔四萬錠,也真所部六萬錠。平陽之絳州、臺州路之黃巖州饑,杭州火,並賑之。



 五月戊辰朔,免兩都徭役。辛未,安西王遣使來告貧乏,帝語之曰:「世祖以分賚之難,嘗有聖訓,阿難答亦知之矣。若言貧乏,豈獨汝耶?去歲賜鈔二十萬錠,又給以糧,今與,則諸王以為不均;不與,則汝言人多饑死。其給糧萬石,擇貧者賑之。」甲戌,詔民間馬牛羊,百取其一,羊不滿百者亦取之,惟色目人及數乃取。丁丑,太陰犯平道。庚辰,土蕃叛,殺掠階州軍民,遣脫脫會諸王鐵木而不花、只列等合兵討之。甲申,命也真、薛闍罕駐夏於合亦而之地。禁諸王、公主、駙馬招戶。己丑,詔諸徒役者,限一年釋之,毋杖。庚寅,罷四川馬湖進獨本蔥。詔諸王、駙馬及有分地功臣戶。居上都、大都、隆興者,與民均納供需。丁酉,命諸行省非奉旨毋擅調軍。安南國遣人招誘叛賊黃勝許。也黑迷失進紫檀,賜鈔四千錠。是月,野蠶成繭。河中府之猗氏雹;太原之平晉,獻州之交河、樂壽,莫州之莫亭、任丘,及湖南醴陵州皆水;濟寧之濟州螟。六月己亥,給出伯軍馬七千二百餘匹。詔晉王所部衣糧,糧以歲給,衣則三年賜之。給瓜州、沙州站戶牛種田具。御史臺臣言:「官吏受賂,初既辭伏,繼以審核,而有司徇情致令異辭者,乞加等論罪。」從之。乙巳,太白犯天關。以調兵妨農,免廣西容州等處田租一年。丙午,叛賊黃勝許遁入交趾。甲寅,降官吏受贓條格,凡十有三等。丁巳,太白犯填星。癸亥,太陰犯井。丙寅,詔行省、行臺,凡硃清有所陳列,毋輒止之。賜西平王奧魯赤銀二百五十兩、鈔六千錠,所部六萬錠,諸王亦憐真所部二十萬錠,兀魯思駐冬軍三萬錠。是月,大都、真定、保定、太平、常州、鎮江、紹興、建康、澧州、岳州、廬州、汝寧、龍陽州、漢陽、濟寧、東平、大名、滑州、德州蝗,大同、隆興、順德、太原雹。海南民饑,發粟賑之。



 秋七月庚午,肇州萬戶府立屯田,給以農具、種、食。辛未,以鈔十一萬八千錠給西蕃諸驛。甘、肅兩州驛戶饑,給糧有差。賜諸王完澤印。癸酉,詔茶鹽轉運司、印鈔提舉司、運糧漕運司官,仍舊以三年為代;雲南、福建官吏滿任者,給驛以歸。壬午,填星犯井,太白犯輿鬼。括伯顏、阿術、阿里海牙等所據江南田及權豪匿隱者,令輸租。河泊官歲入五百錠者敕授。增江西、河南省參政一員,以硃清、張瑄為之。授特進上柱國高麗王世子王謜為儀同三司、領都僉議司事。乙酉,遣雲南省逃軍戍亦乞不薛,命湖廣、江西兩省擇駐夏軍牧地。丙戌,遣岳樂也奴等使馬八兒國。己丑,命行臺監察御史鉤校隨省理問所案牘,以虎賁三百人戍應昌。諸提調錢正官,其部凡有逋欠者,勿遷敘。廣西賊陳飛、雷通、藍青、謝發寇昭、梧、藤、容等州,湖廣左丞八都馬辛擊平之。辛巳,賜貴由赤戍軍鈔三萬九千餘錠。是月,平陽、大名,歸德、真定蝗,彰德、真定、曹州、濱州水,懷孟、大名、河間旱,太原、懷孟雹。福建、廣西兩江道饑,賑粟有差。



 八月丁酉朔,禁舶商毋以金銀過海,諸使海外國者不得為商。庚子,太陰犯亢,太白犯軒轅。壬寅,命江浙行省以船五十艘、水工千三百人,沿海巡禁私鹽。癸卯,太陰犯天江。乙巳,詔諸人告捕盜賊者,強盜一名賞鈔五十貫,竊盜半之,應捕者又半之,皆徵諸犯人,無可徵者官給。乙卯,太陰犯天街,太白犯上將。給諸王亦憐真軍糧三月。是月,德州、彰德、太原蝗,咸寧縣,金、復州,隆興路隕霜殺禾,寧海州大雨,大名路水。九月戊辰,太白犯左執法。辛未,聖誕節,帝駐蹕安同泊,受諸王百官賀。壬申,太陰掩南斗。甲戌,增鹽價鈔一引為六十五貫,鹽戶造鹽錢為十貫,獨廣西如故。征浙東、福建、湖廣夏稅。罷民間鹽鐵爐灶。給襄陽府合剌魯軍未賜田者糧兩月。罷淮西諸巡禁打捕人員。丁丑,太陰犯壘壁陣。戊寅,元江賊舍資殺掠邊境,梁王命怯薛丹等討降之。甲申,雲南省臣也先不花征乞藍,拔瓦農、開陽兩寨,其黨答剌率諸蠻來降,乞藍悉平,以其地為雲遠路軍民總管府。己丑,太陰犯軒轅。辛卯,諸王出伯言汪總帥等部軍貧乏,帝以其久戍,命留五千駐冬,餘悉遣還,至明年四月赴軍。甲午,令廣海、左右兩江戍軍,以二年三年更戍;海都兀魯思不花部給出伯所部軍米萬石。是月,常德之沅江縣水,免其田租。河間之莫州、獻州旱。河決河南杞、封丘、祥符、寧陵、襄邑五縣。



 冬十月丁酉,有事於太廟。壬寅,發米十萬石賑糶京師,以宣德、奉聖、懷來、縉山等處牧宿衛馬。甲辰,修大都城。壬子,車駕至自上都。職官坐贓,經斷再犯者,加本罪三等。贛州賊劉六十攻掠吉州,江西行省左丞董士選討平之。是月,廣備屯及寧海之文登水。



 十一月丁卯,以蠻洞將領彭安國父子討田知州有功,賜安國金符,子為蠻夷官。答馬剌一本王遣其子進象十六。戊辰,以廣西戍軍悉隸兩江宣慰司都元帥府。己巳,兀都帶等進所譯《太宗》、《憲宗》、《世祖實錄》,帝曰:「忽都魯迷失非昭睿順聖太后所生,何為亦曰公主?順聖太后崩時,裕宗已還自軍中,所計月日先後差錯。又別馬里思丹砲手亦思馬因、泉府司,皆小事,何足書耶?」辛未,徙江浙行省拔都軍萬人戍潭州,潭州以南軍移戍郴州。以洪澤、芍陂屯田軍萬人修大都城。遣樞密院官整飭江南諸鎮戍軍,凡將校勤怠者,列實以聞。增海運明年糧為六十萬石。丁丑,太陰犯月星,又犯天街。庚辰,太陰犯井。丁亥,太陰犯上相。乙酉,樞密院臣言:「江南近邊州縣,宜擇險要之地,合群戍為一屯,卒有警急,易於徵發。」詔行省圖地形、核軍實以聞。戊子,太陰犯平道。增大都巡防漢軍。壬辰,太陰犯天江。緬王遣其子僧伽巴叔撒邦巴來貢方物。罷雲南柏興府入德昌路,賜太常禮樂戶鈔五千餘錠。是月,象食屯水,免其田租。



 十二月戊戌,立徹里軍民總管府。雲南行省臣言:「大徹里地與八百媳婦犬牙相錯,今大徹里胡念已降,小徹里復占扼地利,多相殺掠。胡念遣其弟胡倫乞別置一司,擇通習蠻夷情狀者為之帥,招其來附,以為進取之地。」詔復立蒙樣剛等甸軍民官。癸卯,定諸王朝會賜與:太祖位,金千兩、銀七萬五千兩;世祖位,金各五百兩、銀二萬五千兩;餘各有差。丁未,太陰犯井。詔諸行省征補逃亡軍。復司天臺觀星戶。乙卯,太陰犯進賢。癸亥,釋在京囚百人;增置侍御史二員;賜金齒、羅斛來朝人衣。是歲,大都、保定、汴梁、江陵、沔陽、淮安水,金、復州風損禾,太原、開元、河南、芍陂旱,蠲其田租。是歲,斷大闢二十四人。



 大德元年春正月庚午,增諸王要木忽而、兀魯而不花歲賜各鈔千錠。辛未,諸王亦憐真來朝,薨於道,賜幣帛五百匹。乙亥,給月兒魯匠者田,人百畝。乙酉,以邊地乏芻,給出伯征行馬粟四月。丙戌,以鈔十二萬錠、鹽引三萬給甘肅行省。昔寶赤等為叛寇所掠,仰食於官,賜以農具牛種,俾耕種自給。己丑,以藥木忽而等所部貧乏,摘和林漢軍置屯田於五條河,以歲入之租資之。辛卯,以張斯立為中書省參知政事。諸王阿只吉駐太原,河東之民困於供億,詔詰問之,仍歲給鈔三萬錠、糧萬石。給晉王所部屯田農器千具。建五福太乙神壇畤。汴梁、歸德水,木鄰等九站饑,以米六百餘石賑之。給可溫種田戶耕牛。



 二月甲午朔,賜晉王甘麻剌鈔七萬錠,安西王阿難答三萬錠。丙申,蒙陽甸酋長納款,遣其弟阿不剌等來獻方物,且請歲貢銀千兩及置驛傳,詔即其地立通西軍民府,秩正四品。戊戌,升全州為全寧府。庚子,詔東部諸王分地蒙古戍軍,死者補之,不勝役者易之。癸卯,徙揚州萬戶鄧新軍屯蘄、黃,以闍里臺所隸新附高麗、女直、漢軍居沈州。甲申,諸軍民相訟者,命軍民官同聽之。丁未,省打捕鷹房府入東京路。戊午,羅羅斯酋長來朝。己未,改福建省為福建平海等處行中書省,徙治泉州。平章政事高興言泉州與琉求相近,或招或取,易得其情,故徙之。減福建提舉司歲織段三千匹,其所織者加文繡,增其歲輸衲服二百,其車渠帶工別立提舉司掌之。封的立普哇拿阿迪提牙為緬國王,且詔之曰:「我國家自祖宗肇造以來,萬邦黎獻,莫不畏威懷德。向先朝臨御之日,爾國使人稟命入覲,詔允其請。爾乃遽食前言,是以我帥閫之臣加兵於彼。比者爾遣子信合八的奉表來朝,宜示含弘,特加恩渥,今封的立普哇拿阿迪提牙為緬國王,賜之銀印;子信合八的為緬國世子,錫以虎符。仍戒飭云南等處邊將,毋擅興兵甲。爾國官民,各宜安業。」又賜緬王弟撒邦巴一珠虎符,酋領阿散三珠虎符,從者金符及金幣,遣之。以新附軍三千屯田漳州。庚申,升寧都、會昌縣為州,並隸贛州路;寧陽鎮為縣,隸濟寧路;隩州巡檢司為河曲縣,隸保德州。安豐路設錄事司。以行徽政院副使王慶端為中書右丞。詔改元赦天下。免上都、大都、隆興差稅三年,給也只所部六千戶糧三月。



 三月戊辰,熒惑犯井。己巳,完澤等奏定銓調選法。庚午,以陜西行省平章也先鐵木而為中書平章政事,中書省左丞梁暗都剌為中書省右丞。癸酉,太陰掩軒轅大星。畋於柳林。免武當山新附軍徭賦。甲戌,西蕃寇階州,陜西行省平章脫列伯以兵進討,其黨悉平,留軍五百人戍之。詔各省合並鎮守軍,福建所置者合為五十三所,江浙所置者合為二百二十七所。丙子,車駕幸上都。丁丑,封諸王鐵木而不花為鎮西武靖王,賜駝紐印。以江西省左丞八都馬辛為中書左丞。庚辰,札魯忽赤脫而速受賂,為其奴所告,毒殺其奴,坐棄市。乙酉,遣阿里以鈔八萬錠糴糧和林。丁亥,禁正月至七月捕獵,大都八百里內亦如之。庚寅,立江淮等處財賦總管府及提舉司。賜諸王岳木忽而及兀魯思不花金各百兩,兀魯思不花母阿不察等金五百兩,銀鈔有差。賜稱海匠戶市農具鈔二萬二千九百餘錠,及牙忽都所部貧戶萬錠,別吉韂匠萬九百餘錠。五臺山佛寺成,皇太后將親往祈祝,監察御史李元禮上封事止之。歸德、徐、邳、汴梁諸縣水,免其田租。道州旱,遼陽饑,並發粟賑之。岳木忽而及兀魯思不花所部民饑,以乳牛牡馬濟之。



 夏四月癸巳朔,日有食之。丙申,中書省、御史臺臣言:「阿老瓦丁及崔彧條陳臺憲諸事,臣等議,乞依舊例。御史臺不立選,其用人則於常調官選之,惟監察御史首領官,令御史臺自選。各道廉訪司必擇蒙古人為使,或闕,則以色目世臣子孫為之,其次參以色目、漢人。又合剌赤、阿速各舉監察御史非便,亦宜止於常選擇人。各省文案,行臺差官檢核。宿衛近侍,奉特旨令臺憲擢用者,必須明奏,然後任之。行臺御史秩滿而有效績者,或遷內臺,或呈中書省遷調,廉訪司亦如之;其不稱職者,省、臺擇人代之。未歷有司者,授以牧民之職;經省、臺同選者,聽御史臺自調。中書省或用臺察之人,亦宜與御史臺同議,各官府憲司官,毋得輒入體察。今擬除轉運鹽使司外,其餘官府悉依舊例。」制曰:「可。」壬寅,賜兀魯思不花圓符。賜暹國、羅斛來朝者衣服有差。賜牙忽都部鈔萬錠,給岳木忽而所部和林屯田種,以米二千石賑應昌府。



 五月丙寅,河決汴梁,發民三萬餘人塞之。戊辰,安南國遣使來朝。追收諸位下為商者制書、驛券。命回回人在內郡輸商稅。給鈔千錠建臨洮佛寺。詔強盜奸傷事主者,首從悉誅;不傷事主,止誅為首者,從者剌配,再犯亦誅。給葛蠻安撫司驛券一。辛未,遂寧州軍戶任福妻一產三男,給復三歲。癸酉,太白犯鬼積尸氣。乙亥,太陰犯房。丁丑,禁民間捕鬻鷹鷂。庚寅,平伐酋領內附,乞隸於亦乞不薛,從之。各路平準行用庫,舊制選部民富有力者為副,命自今以常調官為之,隸行省者從行省署用。上思州叛賊黃勝許遣其子志寶來降。漳河溢,損民禾稼。饒州鄱陽、樂平及隆興路水。亦乞列等二站饑,賑米一百五十石。六月甲午,諸王也裡乾遣使乘驛祀五岳、四瀆,命追其驛券,仍切責之。以湖廣行省參政崔良知廉貧,特賜鹽課鈔千錠。給和林軍需鈔十萬錠。乙未,太白晝見。戊戌,平伐九寨來降,立長官司。己酉,令各部宿衛士輸上都、隆興糧各萬五千石於北地。甲寅,罷亦奚不薛歲貢馬及氈衣。丙辰,監察御史斡羅失剌言:「中丞崔彧兄在先朝嘗有罪,還其所籍家產非宜。又買僧寺水碾違制。」帝以其妄言,笞之。詔僧道犯奸盜重罪者,聽有司鞫問。賜諸王也裡乾等從者鈔二萬錠,朵思麻一十三站貧民五千餘錠。是月,平灤路蟲食桑,歸德徐、邳州蝗,太原風、雹,河間、大名路旱,和州歷陽縣江漲,漂沒廬舍萬八千五百餘家。以糧四千餘石賑廣平路饑民,萬五千石賑江西被水之家,二百九十餘石賑鐵裡乾等四站饑戶。



 秋七月庚午,太陰犯房。辛未,賜諸王脫脫、孛羅赤、沙禿而鈔二千錠,所部八萬四千餘錠,撒都失裏千錠,所部二萬餘錠。罷蒙古軍萬戶府入曲先塔林都元帥府。癸未,增晉王所部屯田戶。甲申,增中御府官一員。賜馬八兒國塔喜二珠虎符。詔出使招諭者授以招諭使、副;諸取藥物者,授以會同館使、副,但降旨差遣,不給制命。丙戌,以八兒思禿倉糧隸上都留守司,招籍宋兩江鎮守軍。丁亥,免上都酒課三年。賜諸王不顏鐵木而及其弟伯真孛羅鈔四千錠,所部八萬四千八百餘錠,仍給糧一年。寧海州饑,以米九千四百餘石賑之。河決杞縣蒲口。郴州路、耒陽州、衡州之酃縣大水山崩,溺死三百餘人。懷州武陟縣旱。



 八月庚子,詔合伯留軍五千屯守,令孛來統其餘眾以歸。丁未,命諸王阿只吉自今出獵,悉自供具,毋傷民力。丁巳,妖星出奎。揚州、淮安、寧海州旱,真定、順德、河間旱、疫,池州、南康、寧國、太平水。九月辛酉朔,妖星復犯奎。壬戌,八番、順元等處初隸湖廣,後改隸雲南,雲南戍兵不至,其屯駐舊軍逃亡者眾,仍命湖廣行省遣軍代之。甲子,八百媳婦叛,寇徹里,遣也先不花將兵討之。丙寅,詔恤諸郡水旱疾疫之家,罷括兩淮民田。汰諸王來大都者及宿衛士冗員。丁卯,命平章伯顏專領給賜孤老衣糧。壬午,車駕還大都。己丑,增海漕為六十五萬石。罷南丹州安撫司,立慶遠南丹溪洞等處軍民安撫司。詔邊遠官已嘗優升品級而托他事不起者,奪其所升官。平珠、六洞蠻及十部洞蠻皆來降,命以蠻夷官授之。給衛士牧馬外郡者糧,令毋仰食於民。以札魯忽赤所追贓物輸中書省。衛輝路旱、疫,澧州、常德、饒州、臨江等路,溫之平陽、瑞安二州大水,鎮江之丹陽、金壇旱,並以糧給之。



 冬十月甲午,詔諸遷轉官注闕二年。丁酉,有事於太廟。辛丑,減上都商稅歲額為三千錠。溫州陳空崖等以妖言伏誅。癸丑,免陜西鹽戶差稅,罷其所給米。乙卯,瓜哇遣失剌班直木達奉表來降。戊午,太白經天。增吏部尚書一員。以朵甘思十九站貧乏,賜馬牛羊有差。廬州路無為州江潮泛溢,漂沒廬舍。歷陽、合肥、梁縣及安豐之蒙城、霍丘自春及秋不雨,揚州、淮安路饑,韶州、南雄、建德、溫州皆大水,並賑之。



 十一月壬戌,禁權豪、僧、道及各位下擅據礦炭山場。罷順德、彰德、廣平等路五提舉司,立都提舉司二,升正四品,設官四員,直隸中書戶部。衛輝路提舉司隸廣平彰德都提舉司,真定鐵冶隸順德都提舉司。罷保定紫荊關鐵治提舉司,還其戶八百為民。癸亥,詔自今田獵始自九月。高麗王王昛告老,乞以爵與其子謜。福建行省遣人覘琉求國,俘其傍近百人以歸。戊辰,增太廟牲用馬。庚午,籍唐兀軍入樞密院。辛未,曹州禹城進嘉禾,一莖九穗。丁丑,詔以高麗王世子謜為開府儀同三司、征東行中書省左丞相、駙馬、上柱國、高麗國王,仍加授王昛為推忠宣力定遠保節功臣、開府儀同三司、太尉、駙馬、上柱國、逸壽王。增烏撒烏蒙等處宣慰使一員,以孛羅歡為之。賜諸王兀魯德不花金千兩、銀萬五千兩、鈔萬錠。徙大同路軍儲所於紅城。以河南行省經用不足,命江浙行省運米二十萬石給之。總帥汪惟和以所部軍屯田沙州、瓜州,給中統鈔二萬三千二百餘錠置種、牛、田具。大都路總管沙的坐贓當罷,帝以故臣子,特減其罪,俾仍舊職。崔彧言不可復任,帝曰:「卿等與中書省臣戒之,若後復然,則置爾死地矣。」戊子,太白經天。增晉王內史一員,尚乘寺卿一員。賜藥木忽而金一千二百五十兩、銀一萬五千兩、鈔一萬二千錠。常德路大水,常州路及宜興州旱,並賑之。



 十二月癸巳,令也速帶而、藥樂罕將兵出征。丙申,徙襄賜屯田合剌魯軍於南陽,戶受田百五十畝,給種、牛、田具。戊戌,中書省臣同河南平章孛羅歡等言:「世祖撫定江南,沿江上下置戍兵三十一翼,今無一二,懼有不虞。外郡戍卒封樁錢,軍官遷延不以時取,而以己錢貸之,徵其倍息。逃亡者各處鎮守官及萬戶府並遣人追捕,皆非所宜。又富戶規避差稅冒為僧道,且僧道作商賈有妻子與編氓無異,請汰為民。宋時為僧道者,必先輸錢縣官,始給度牒,今不定制,僥幸必多。無為礬課,初歲入為鈔止一百六錠,續增至二千四百錠,大率斂富民、刻吏俸、停灶戶工本以足之,亦宜減其數。」帝曰:「礬課遣人核實,汰僧道之制,卿等議擬以聞。軍政與樞密院議之。」諸王也只里部忽剌帶於濟南商河縣侵擾居民,蹂踐禾稼,帝命詰之,走歸其部。帝曰:「彼宗戚也,有是理耶?其令也只裏罪之。」禁諸王、駙馬並權豪毋奪民田,其獻田者有刑。復立芍陂、洪澤屯田。壬寅,朝洞蠻內附,立長官司二,命楊漢英領之。甲辰,太白經天,又犯東咸。丙午,太陰犯軒轅。丁未,旌表烈婦漳州招討司知事闞文興妻王氏。戊申,增給雲南廉訪司驛券十二。甲寅,太陰犯心。乙卯,免上都至大都並宣德等十三站戶和顧和買。賜諸王忽剌出鈔千錠,所部四萬四千五百餘錠;諸王阿術、速哥鐵木而所部二萬八千九百餘錠。閏十二月壬戌,太陰犯壘壁陣。命也速帶而等出征;詔諸軍戶賣田者,由所隸官給文券。甲子,福建平章高興言:「漳州漳浦縣大梁山產水晶,乞割民百戶採之。」帝曰:「不勞民則可,勞民勿取。」壬申,徙乃顏民戶於內地。定燕禿忽思所隸戶差稅,以三分之一輸官。賜忽剌出所部鈔萬錠。癸酉至丙子,太白犯建星。己卯,賜不思塔伯千戶等鈔約九萬錠。淮東饑,遣參議中書省事於章發廩賑之,弛湖泊之禁,仍聽正月捕獵。平伐等蠻未附,播州宣撫使楊漢英請以己力討之,命湖廣省答剌罕從宜收撫。瓜州屯田軍萬人貧乏,命減一千,以張萬戶所領兵補之。甲申,增兩淮屯田軍為二萬人。賜諸王阿牙赤鈔千錠,所部一萬一千餘錠,藥樂罕等所部七萬錠,暗都剌火者所部四萬餘錠。般陽路饑疫,給糧兩月。是歲,濟南及金、復州水、旱,大都之檀州、順州,遼陽、沈陽、廣寧水,順德、河間、大名、平陽旱。河間之樂壽、交河疫,死六千五百餘人。斷大闢百七十五人。



 二年春正月壬辰,詔以水旱減郡縣田租十分之三,傷甚者盡免之,老病單弱者差稅並免三年。禁諸王、公主、駙馬受諸人呈獻公私田地及擅招戶者。丙申,遣使閱諸省兵。丁酉,置汀州屯田。辛丑,御史臺臣言:「諸轉運司案牘,例以歲終檢覆。金穀事繁,稽照難盡,奸偽無從知之。其未終者,宜聽憲司於明年檢覆。」從之。乙巳,以糧十萬石賑北邊內附貧民。己酉,建康、龍興、臨江、寧國、太平、廣德、饒池等處水,發臨江路糧三萬石以賑,仍弛澤梁之禁,聽民漁採。遣所俘琉求人歸諭其國,使之效順。並土蕃、碉門安撫司、運司,改為碉門魚通黎雅長河西寧遠軍民宣撫司。以翰林王惲、閻復、王構、趙與票、王之綱、楊文鬱、王德淵,集賢王顒、宋渤、盧摯、耶律有尚、李泰、郝採、楊麟,皆耆德舊臣,清貧守職,特賜鈔二千一百餘錠。給西平王奧魯赤部民糧三月,晉王秫米五百石,所部鈔十二萬錠,戍和林高麗、女直、漢軍三萬錠。



 二月戊午朔,詔樞密院合並貧難軍戶。辛酉,歲星、熒惑、太白聚危,熒惑犯歲星。壬戌,徙重慶宣慰司都元帥府於成都,立軍民宣慰司都元帥府於福建。乙丑,立浙西都水庸田司,專主水利。以中書右丞、徽政院副使張九思為平章政事,與中書省事。丁卯,改泉州為泉寧府。己巳,畋于漷州。辛未,太陰犯左執法。並江西省元分置軍為六十四所。丙子,太陰犯心。帝諭中書省臣曰:「每歲天下金銀鈔幣所入幾何,諸王、駙馬賜與及一切營建所出幾何,其會計以聞。」右丞相完澤言:「歲入之數,金一萬九千兩,銀六萬兩,鈔三百六十萬錠,然猶不足於用,又於至元鈔本中借二十萬錠,自今敢以節用為請。」帝嘉納焉。罷中外土木之役。癸未,詔諸王、駙馬毋擅祀岳鎮海瀆;申禁諸路軍及豪右人等,毋縱畜牧損農。乙酉,車駕幸上都。罷建康金銀銅冶轉運司,還淘金戶於元籍,歲辦金悉責有司。詔廉訪司作成人材,以備選舉。禁諸王從者假控鶴佩帶擾民。詔諸郡凡民播種怠惰及有司勸課不至者,命各道廉訪司治之。減行省平章為二員。丙子,以梁德珪為中書平章政事,楊炎龍為中書右丞。賜瓜忽而所部鈔三十萬錠,近侍伯顏鐵木而等三萬錠,也先鐵木而等市馬價三萬四千四百餘錠,鎮南王脫歡六萬錠。浙西嘉興、江陰,江東建康溧陽、池州水、旱,並賑恤之。湖廣省漢陽、漢川水,免其田租。甘肅省沙州鼠傷禾稼,大都檀州雨雹,歸德等處蝗。



 三月丁亥朔,罷大名路故河堤堰歲入隆福宮租鈔七百五十錠。申禁官吏受賂詣諸司首者,不得輒受。戊子,詔僧人犯奸盜詐偽,聽有司專決,輕者與僧官約斷,約不至者罪之。庚寅,命各萬戶出征者,其印令副貳掌之,不得付其子弟,違法行事。以兩淮閑田給蒙古軍。壬子,御史臺臣言:「道州路達魯花赤阿林不花、總管周克敬虛申麥熟,不賑饑民,雖經赦宥,宜降職一等。」從之。壬子,詔加封東鎮沂山為元德東安王,南鎮會稽山為昭德順應王,西鎮吳山為成德永靖王,北鎮醫巫閭山為貞德廣寧王,歲時與岳瀆同祀,著為令式。



 夏四月戊午,遣征不剌壇軍還本部。庚申,以也速帶而擅調甘州戍軍,遣伯顏等笞之。賜大都守門合赤剌等鈔九萬錠,織工四萬四千錠。發慶元糧五萬石,減其直以賑饑民。江南、山東、江浙、兩淮、燕南屬縣百五十處蝗。



 五月辛卯,罷海南黎兵萬戶府及黎蠻屯田萬戶府,以其事入瓊州路軍民安撫司。罷蕁麻林酒稅羨餘。壬辰,以中書右丞何榮祖為平章政事,與中書省事,湖廣左丞八都馬辛為中書右丞。淮西諸郡饑,漕江西米二十萬石以備賑貸。命中書省遣使監云南、四川、海北海南、廣西兩江、廣東、福建等處六品以下選。戊戌,太陰犯心。壬寅,平灤路旱,發米五百石,減其直賑之。己酉,諸王念不列妃扎忽真詐增所部貧戶,冒支鈔一萬六百餘錠,遣扎魯忽赤同王府官追之。衛輝、順德旱,大風損麥,免其田租一年。詔總帥汪惟正所轄二十四城,有安西王、諸王等並朵思麻來寓者,與編戶均當賦役。耽羅國以方物來貢。撫州之崇仁星隕為石。復致用院。置和林宣慰司都元帥府,以忽剌出、耶律希周、納鄰合剌並為宣慰使都元帥,佩虎符。給兩都八剌合赤鈔各三萬錠。六月庚申,御史臺臣言:「江南宋時行兩稅法,自阿里海牙改為門攤,增課錢至五萬錠。今宣慰張國紀請復科夏稅,與門攤並徵,以圖升進,湖、湘重罹其害。」帝命中書趣罷之。禁權豪、斡脫括大都漕河舟楫。西臺侍御史脫歡以受賂不法罷。禁諸王擅行令旨,其越例開讀者,並所遣使拘執以聞。壬戌,太陰犯角。詔陜西諸色戶與民均當徭役,申嚴陜西運司私鹽之禁。置奉宸庫。賜諸王岳木忽而金一千二百五十兩,兀魯思不花並其母一千兩,銀、鈔有差。山東、河南、燕南、山北五十處蝗,山北遼東道大寧路金源縣蝗。



 秋七月癸巳,太陰犯心。汴梁等處大雨,河決壞堤防,漂沒歸德數縣禾稼、廬舍,免其田租一年。遣尚書那懷、御史劉賡等塞之,自蒲口首事,凡築九十六所。壬寅,詔諸王、駙馬及諸近侍,自今奏事不經中書,輒傳旨付外者,罪之。高麗王王謜擅命妄殺,詔遣中書右丞楊炎龍、僉樞密院事洪君祥召其入侍,以其父昛仍統國政。賜諸王亦憐真等金、銀、鈔有差。江西、江浙水,賑饑民二萬四千九百有奇。



 八月壬戌,太陰犯箕。癸未,給四川出征蒙古軍馬萬匹。九月己丑,聖誕節,駐蹕阻媯之地,受諸王百官賀。交趾、瓜哇、金齒國各貢方物。給和林更戍軍牛、車。丙申,車駕還大都。辛丑,太陰犯五車南星。命廣海、左右江戍軍依舊制以二年或三年更代。癸卯,太陰犯五諸侯。樞密副使塔剌忽帶犯贓罪,命御史臺鞫之。己酉,太陰犯左執法。庚戌,吉、贛立屯田;減中外冗員。



 冬十月甲寅朔,增海漕米為七十萬石。壬戌,太白犯牽牛。置蒙古都萬戶府於鳳翔,立平珠、六洞蠻夷長官司二,設土官四十四員。戊寅,太陰犯角距星。令御史臺檢劾樞密院案牘。賜諸王岳木忽而、兀魯忽不花所部糧五萬石;控鶴七百人,賜鈔五百錠。



 十一月庚寅,安南貢方物。丙申,知樞密院那懷言:「常例文移,乞令副樞以下署行。」從之。罷雲南行御史臺,置肅政廉訪司。己亥,太陰犯輿鬼。辛丑,辰星犯牽牛。罷徐、邳爐冶所進息錢。壬寅,太陰犯右執法。以中書右丞王慶端為平章政事。賜和林軍校幣六千匹,衣帽等物有差。



 十二月戊午,太白經天。己未,填星犯輿鬼。乙丑,太白犯歲星,太陰犯熒惑。括諸路馬,除牝孕攜駒者,齒三歲以上並拘之。賜朵而朵海所部鈔八十五萬錠。庚午,鎮星入輿鬼,太陰犯上將。辛未,增置各路推官,專掌刑獄,上路二員,下路一員。詔諸逃軍復業者免役三年。江浙行省平章政事答剌罕升左丞相。甲戌,彗出子孫星下。己卯,太陰犯南斗。辛巳,命廉訪司歲舉所部廉幹者各二人。詔和市價直隨給其主,違者罪之。定諸稅錢三十取一,歲額之上勿增。揚州、淮安兩路旱、蝗,以糧十萬石賑之。給陣亡軍妻子衣糧。免內郡賦稅。諸王小薛所部三百餘戶散處鳳翔,以潞州田二千八百頃賜之。釋在京囚二百一十九人。



\end{pinyinscope}