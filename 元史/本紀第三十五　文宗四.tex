\article{本紀第三十五 文宗四}

\begin{pinyinscope}

 二年春正月己卯,禦制《奎章閣記》。行樞密臣言:「十一月,仁德府權達魯花赤曲術,糾集兵眾以討雲南,首敗伯忽賊兵於馬龍州是極端自私利己的,生活本身就是痛苦。參見「文學」中的,以是月十一日殺伯忽弟拜延,獻馘於豫王。十三日。戰於馬金山,獲伯忽及其弟伯顏察兒、其黨拜不花、卜顏帖木兒等十餘人,誅之,餘兵皆潰,獨祿余猶據金沙江。」有旨趣進兵討之。庚辰,住持大承天護聖寺僧寶峰加司徒。辛巳,大名魏縣民曹革輸粟賑陜西饑,旌其門。癸未,立侍正府以總近侍,秩從二品。乙酉,時享太廟。丙戌,伯顏、月魯帖木兒、玥璐不花、阿卜海牙等十四人,並以本官兼侍正。旌大都大興縣郭仲安妻李氏貞節。丁亥,以壽安山英宗所建寺未成,詔中書省給鈔十萬錠供其費,仍命燕鐵木兒、撒迪等總督其工役。命後衛指揮使史塤往四川行省調軍官選。戊子,命奴都赤阿里火者按行北邊牧地。以晉邸部民劉元良等二萬四千餘戶隸壽安山大昭孝寺為永業戶。中書省臣言:「四川省臣塔出、脫帖木兒等討雲南,以十一月九日領兵至烏撒周泥驛。明日,祿餘、阿奴、阿答等賊兵萬餘,自山後間道潛出。塔出、脫帖木兒等進擊,屢戰敗之。十五日,又戰七星關,六日凡十七戰,賊大敗潰去。」詔遣使以銀、幣賞塔出、脫帖木兒等。造歲額鈔本至元鈔八十九萬五十錠,中統鈔五千錠。給鈔五千錠,賑寧海州饑民。罷益都等處廣農提舉司,改立田賦總管府,秩從三品,仍令隆祥總管府統之。命興和路建燕鐵木兒鷹棚。樞密院臣言:「四川行省地鄰烏撒,而雲南未平,今戍卒單少,宜增兵防遏。請調夔路怯憐口戶丁七百、重慶河東五路兩營兵三百,同往戍之。俟征進軍還日,悉罷遣。」從之。庚寅,改東路蒙古軍萬戶府為東路蒙古侍衛親軍指揮使司。諸王哈兒蠻遣使來貢蒲萄酒。國制,累朝行帳設衛士、給事如在位時。近嘗汰其冗濫,武宗、仁宗兩朝,各定為八百人,英宗七百人。中書省臣言,舊給事人有失職者,詔復其百人。辛卯,皇太子阿剌忒納答剌薨。壬辰,命宮相法里及給事者五十八人護靈輿北祔葬於山陵,仍令法裡等守之。御史臺臣劾奏:「福建宣慰副使哈只,前為廣東廉訪副使,貪污狼籍,宜罷黜。」從之。己亥,遣吏部尚書撒里瓦,佩虎符,禮部郎中趙期頤,佩金符,齎即位詔告安南國,且賜以《授時歷》。賜武寧王徹徹禿金百兩、銀五百兩,以淮安路之海寧州為其食邑。癸卯,以皇子古納答剌疹疾愈,賜燕鐵木兒及公主察吉兒各金百兩、銀五百兩、鈔二千錠,撒敦等金、銀、鈔各有差;又賜醫巫、乳媼、宦官、衛士六百人金三百五十兩、銀三千四百兩、鈔五千三百四十錠。甲辰,敕每歲四祭五福太一星。建孔子廟於後衛。至元末,討諸王乃顏之叛,獲其部蒙古軍,分置河南、江浙、湖廣、江西諸省,命樞密院遣使括其數,得二千六百人。乙巳,封蒙古巫者所奉神為靈感昭應護國忠順王,號其廟曰靈祐。給衛士萬人歲例鈔,人八十錠,內以他物及粟折五之一。鎮西武靖王搠思班、豫王阿剌忒納失里及行省、行院官同討雲南,兵十餘萬,以去年十一月十一日,搠思班師次羅羅斯,期躍里鐵木兒俱至三泊郎,仍趣小雲失會於曲靖馬龍等州,同進兵。躍里鐵木兒倍道兼進,奪金沙江。十二月十七日,大兵與阿禾蒙古軍相值,戰敗之,阿禾偽降,明日,率其兵三千為三隊來襲我營,搠思班、躍里鐵木兒等分十三隊又擊敗之,阿禾竄走,大兵直趨中慶。二十六日,遇賊黨蒙古軍於安寧州,與再戰,又大敗之。二十八日,阿禾來逆戰,遂就禽,斬於軍前。三十日,將抵中慶,賊兵七千猶拒戰於伽橋、古壁口,兵交,躍里鐵木兒左頰中流矢,洞耳後,拔矢復與戰,大捷,遂復行省治。諸軍皆會,駐於城中,分兵追捕殘賊於嵩明州。樞密院臣以捷聞,詔總兵官量度緩急,從宜區處。新添安撫司壅河寨主,訴他部徭、獠蹂其禾,民饑,命湖廣行省發鈔二千錠,市米賑之。



 二月丙戌,以上都留守乃馬臺行嶺北行樞密院事,太禧宗禋使謹只兒、答鄰答里、馬烈捏四人並知院事,遙授平章政事。戊申,立廣教總管府,以掌僧尼之政,凡十六所:曰京畿山後道,曰河東山右道,曰遼東山北道,曰河南荊北道,曰兩淮江北道,曰湖北湖南道,曰浙西江東道,曰浙東福建道,曰江西廣東道,曰廣西兩海道,曰燕南諸路,曰山東諸路,曰陜西諸路,曰甘肅諸路,曰四川諸路,曰雲南諸路。秩正三品,府設達魯花赤、總管、同知府事、判官各一員,宣政院選流內官擬注以聞,總管則僧為之。四川行省招諭懷德府驢穀什用等四洞及生蠻十二洞,皆內附,詔升懷德府為宣撫司以鎮之。諸洞各設長官司及巡檢司,且命各還所掠生口。湖廣參政徹里帖木兒與速速、班丹俱坐出怨言,鞫問得實,刑部議當徹里帖木兒、班丹杖一百七,速速處死,會赦,徹里帖木兒流廣東,班丹廣西,速速徙海南,皆置荒僻州郡。有旨:「此輩怨望於朕,向非赦原,俱當置之極刑,可俱籍其家,速速禁錮終身。」己酉,白虹貫日。旌鞏昌金州民杜祖隆妻張氏志節。樞密院臣言:「徹里鐵木兒、孛羅以正月戊寅敗烏撒蠻兵,射中祿餘,降其民,烏蒙、東川、易良州蠻兵、夷獠等俱款附。鎮西武靖王搠思班等駐中慶,復行省事;豫王阿剌忒納失裡等至當當驛,安輯其人民。」又言:「澂江路蠻官郡容報賊古剌忽及禿堅之弟必剌都迷失等偽降於豫王而反圍之,至易龍驛,古剌忽等兵掩襲官軍。四川行省平章塔出頓兵不進,平章乞住妻子孳畜為賊所掠。諜知禿堅方修城堡,布兵拒守,無出降意。」詔速進兵討之。敕探馬赤軍士歲以五月十日遷處山後諸州。辛亥,建燕鐵木兒居第於興聖宮之西南,詔撒迪及留守司董其役。壬子,太白晝見。中書平章政事亦列赤兼沈陽等路安撫使,燕王宮相伯撒里為中書平章政事,陜西行臺中丞朵兒只班為中書參知政事,戶部尚書高履亨、兩淮都轉運鹽使許有壬並參議中書省事。甲寅,燕鐵木兒言:「賽因怯列木丁,英宗時嘗獻寶貨於昭獻元聖太后,議給價鈔十二萬錠,故相拜住奏酬七萬錠,未給,泰定間以鹽引萬六百六十道折鈔給之,今有司以詔書奪之還官。臣等議,以為寶貨太后既已用之,以鹽引還之為宜。」從之。燕鐵木兒又言:「安慶萬戶鎖住,坐令家人殺人系獄,久未款伏,宜若無罪,乞釋之。」制曰:「可。」乙卯,太白犯昴。祀太祖、太宗、睿宗御容。雲南統兵官來報捷,諸蠻悉降,唯祿餘追捕未獲。命番休各衛漢軍,十之二以三月一日放遣。丁巳,駙馬不顏帖木兒自北邊從武寧王徹徹禿來朝。己未,命西僧為皇子古納答剌作佛事一周歲。壬戌,改封武寧王徹徹禿為郯王,賜以金印。甲子,中書省臣言:「國家錢穀,歲入有額,而所費浩繁,是以不足。天歷二年,嘗以鹽賦十分之一折銀納之,凡得銀二千餘錠。今請以銀易官帑鈔本,給宿衛士卒。」又言:「陛下不用經費,不勞人民,創建大承天護聖寺。臣等願上向所易鈔本十萬錠、銀六百錠助建寺之需。」從之。丙寅,以太祖四大行帳世留朔方不遷者,其馬駝孳畜多死損,發鈔萬錠,命內史府市以給之。行樞密院都事阿里火者來報雲南之捷。庚午,給宿衛士歲例鈔,詔毋出定額萬人之外。占城國遣其臣高暗都剌來朝貢。創建五福太一宮於京城乾隅,修上都洪禧、崇壽等殿。諸王徹徹禿、沙哥坐妄言不道,詔安置徹徹禿廣州,沙哥雷州。壬申,命遼陽行省發粟賑國王朵兒只及納忽答兒等六部蒙古軍民萬五千戶。旌大都民劉德仁妻王氏貞節。甲戌,給宣讓王王傅印。荊王也速也不干貢BX牛。命田賦總管府稅礦銀輸大承天護聖寺。命興和路為玥璐不花作鷹棚。雲南景東甸蠻官阿只弄遣子罕旺來朝,獻馴象,乞升甸為景東軍民府,阿只弄知府事,罕旺為千戶,常賦外歲增輸金五千兩、銀七百兩,許之。以山東鹽課鈔萬錠,賑膠州饑。命龍翊衛以屯田歲入粟贍衛卒孤貧者。是月,深、冀二州有蟲食桑為災。



 三月丙子朔,熒惑犯鬼宿。辛巳,御史臺臣劾奏:「燕南廉訪使卜咱兒,前為閩海廉訪使,受贓計鈔二萬二千餘錠、金五百餘兩、銀三千餘兩、男女生口二十二人及它寶貨無算,難遇赦原,乞追奪制命,籍沒流竄。」詔如所言,仍暴其罪示天下。壬午,賜南郊侍祠文武官金、幣有差。特令沙津愛護持必剌忒納失里為三藏國師,賜玉印。以陜西鹽課鈔萬錠,賑察罕腦兒蒙古饑民。癸未,割外府幣、帛各千匹輸之中宮,以供需用。甲申,繪皇太子真容,奉安慶壽寺之東鹿頂殿,祀之如累朝神御殿儀。鞫宦者拜住侍皇太子疹疾,飲食不時進,以酥拭其眼鼻,又為禳咒,杖一百七,斥出京城。冠州有蟲食桑四十餘萬株。御史臺臣言:「奎章閣參書雅琥,阿媚奸臣,所為不法,宜罷其職。」從之。丙戌,雨土,霾。伯撒里辭所兼儲政使,不允。伯顏娶諸王女,賜金二百兩、銀千兩。賜上都死事者不顏帖木兒等十一家鈔各百錠,分賜燕鐵木兒鷹坊百人。中書省臣言:「宣課提舉司歲榷商稅,為鈔十萬餘錠,比歲數不登,乞凡僧道為商者,仍徵其稅。」有旨:「誠為僧者,其仍免之。」司徒香山言:「陶弘景《胡笳曲》,有『負扆飛天歷,終是甲辰君』之語,今陛下生年、紀號,實與之合,此實受命之符,乞錄付史館,頒告中外。」詔令翰林、集賢、奎章、禮部雜議之。翰林諸臣議以謂:「唐開無間,太子賓客薛讓進武后鼎銘云『上玄降鑒,方建隆基』,為玄宗受命之符。姚崇表賀,請宣示史官,頒告中外。而宋儒司馬光斥其採偶合之文以為符瑞,乃小臣之諂,而宰相實之,是侮其君也。今弘景之曲,雜於生年、紀號若偶合者,然陛下應天順人,紹隆正統,於今四年,薄海內外,罔不歸心,固無待於旁引曲說以為符命。從其所言,恐啟讖緯之端,非所以定民志。」事遂寢。趙王不魯納食邑沙、凈、德寧等處蒙古部民萬六千餘戶饑,命河東宣慰發近倉糧萬石賑之。又發山東鹽課鈔、硃王倉粟賑登、萊饑民,興和倉粟賑保昌饑民。戊子,以西僧旭你迭八答剌班的為三藏國師,賜金印。以龍慶州之流杯園池、水磑、土田賜燕鐵木兒。命諸王阿魯出鎮陜西行省。以籍入速速、班丹、徹理帖木兒貲產賜大承天護聖寺為永業。浙西諸路比歲水旱,饑民八十五萬餘戶,中書省臣請令官私、儒學、寺觀諸田佃民,從其主假貸錢穀自賑,餘則勸分富家及入粟補官,仍益以本省鈔十萬錠,並給僧道度牒一萬道,從之。旌同知大都府事忙兀禿魯迷失妻海迷失貞節。己丑,賑云內州饑民及察忽涼樓戍兵共七千戶。庚寅,命威順王寬徹不花還鎮湖廣。癸巳,詔累朝神御殿之在諸寺者,各制名以冠之:世祖曰元壽,昭睿順聖皇后曰睿壽,南必皇后曰懿壽,裕宗曰明壽,成宗曰廣壽,順宗曰衍壽,武宗曰仁壽,文獻昭聖皇后曰昭壽,仁宗曰文壽,英宗曰宣壽,明宗曰景壽。召亳州太清宮道士馬道逸、汴梁朝天宮道士李若訥、河南嵩山道士趙亦然,各率其徒赴闕,修普天大醮。賑浙西鹽丁五千餘戶。命玥璐不花作佛事於德興府。監察御史劾江浙行省平章童童荒泆宴安,才非輔佐,詔免其官。豫王阿剌忒納失里、鎮西武靖王搠思班等禽雲南諸賊也木干、羅羅、脫脫木兒、板不、阿居、澂江路總管羅羅不花、伯忽之叔怯得該、偽署萬戶哈剌答兒及諸將校,悉斬之,磔尸以徇。賑遼陽境內蒙古饑民萬四千餘戶。旌山丹州郝榮妻李閏貞節。陜州諸縣蝗。八番軍從征雲南者俱屯貴州,樞密院臣請遣使發粟給之。己亥,御史臺臣劾奏:「大都總管劉原仁稱疾,久不視事,及遷同知儲政院事,即就職,僥幸巧官,避難就易。」有旨罷之。庚子,以將幸上都,命西僧作佛事於乘輿次舍之所。壬寅,以欽察衛軍士增多,析為左右二衛。給雲南行省鈔十萬錠,以備軍資民食。癸卯,御史臺臣劾奏工部尚書蘇炳性行貪邪,詔罷之。大同路累歲水旱,民大饑。裁節衛士馬芻粟,自四月一日始。壽王脫裡出、陽翟王帖木兒赤、西平王管不八、昌王八剌失裡等七部之民居遼陽境者萬四千五百餘戶告饑,命遼陽行省發近境倉糧賑兩月。命宣靖王買奴置王傅等官。立宮相都總管府,秩正三品,給銀印。以儒學教授在選數多,凡仕由內郡、江淮者,注江西、江浙、湖廣;由陜西、兩廣者,注福建;由甘肅、四川、雲南、福建者,注兩廣。敕河南行省右丞那海提督境內屯田。中書省臣言:「嘉興、平江、松江、江陰蘆場、蕩山、沙塗、沙田等地之籍於官者,嘗賜他人,今請改賜燕鐵木兒。」有旨:「燕鐵木兒非他臣比,其令所在有司如數給付。」發通州官糧賑檀、順、昌平等處饑民九萬餘戶,以山東鹽課鈔三千五百錠賑益都三萬餘戶。是月,陜西行省遣官分給復業饑民七萬餘口行糧,賑諸王伯顏也不干部內蒙古饑民千餘口。真定、汴梁二路,恩、冠、晉、冀、深、蠡、景、獻等八州,俱有蟲食桑為災。旌故戶部主事趙野妻柳氏貞節。



 夏四月丙午朔,全寧民王脫歡獻銀礦。詔設銀場提舉司,隸中政院。中書、樞密臣言:「天歷兵興,諸領軍與敵戰者,宜定功賞。臣等議,諸王各金百兩、銀五百兩、金腰帶一、織金等幣各十八匹,諸臣四戰以上者同,三戰及一戰者各有差。」有旨:「賞格具如卿等議。燕鐵木兒首倡大義,躬擐甲胄,伯顏在河南先誅攜貳,使朕道路無虞,兩人功無與比,其賞不可與眾同,其賜燕鐵木兒七寶腰帶一、金四百兩、銀九百兩,伯顏金腰帶一、金二百兩、銀七百兩。」受賞者凡九十六人,用金二千四百兩、銀萬五千六百兩、金腰帶九十一副、幣帛千三百餘匹。命西僧於五臺及霧靈山作佛事各一月,為皇太子古納答剌祈福。以糧五萬石賑糶京師貧民。戊申,皇姑魯國大長公主薨。以宮中高麗女子不顏帖你賜燕鐵木兒,高麗國王請割國中田為資送,詔遣使往受之。發衛卒三千助大承天護聖寺工役。庚戌,詔建燕鐵木兒生祠於紅橋南,樹碑以紀其勛。御史臺臣言:「平章政事曹立,累任江浙,今雖閑廢,猶與富民交納,宜遣還其本籍大同路。又,監察御史萬家閭嘗薦中丞和尚,脫脫嘗舉廉訪使卜咱兒,今和尚、卜咱兒俱以贓罪除名,萬家閭、脫脫難任臺省之職。」並從之。真定武陟縣地震,逾月不止。壬子,命燕鐵木兒總制宮相都總管府事,也不倫、伯撒裏俱以本官兼宮相都總管府都達魯花赤。諸王哈兒蠻遣使來朝貢。甲寅,改宣忠扈衛親軍都萬戶府為宣忠斡羅思扈衛親軍諸指揮使司,賜銀印。中書省臣言:「越王禿剌在武宗時以紹興路為食邑,歲割賜本路租賦鈔四萬錠,今其子阿剌忒納失里襲王號,宜歲給其半。」從之。乙卯,時享太廟。鎮西武靖王搠思班等已平雲南,各遣使來報捷。諸王朵列捏鎮雲南品甸,自以貲力給軍,協力討賊,詔以襲衣賜之。丙辰,葺太祖所御大行帳。戊午,以集慶路玄妙觀為大元興崇壽宮。命興和建屋居海青,上都建屋居鷹鶻。庚申,特命河南儒士吳炳為藝文監典簿,仍予對品階。寧國路涇縣民張道殺人為盜,道弟吉從而不加功,居囚七年不決。吉母老,無他子孫,中書省臣以聞,敕免死,杖而黜之,俾養其母。辛酉,以山東鹽課鈔五千錠賑博興州饑民九千戶,一千錠賑信陽等場鹽丁。御史臺臣言:「儲政使哈撒兒不花侍陛下潛邸時,受馬七十九匹,又盜用官庫物。天歷初,領兵蘆溝橋,迎敵即逃,擅閉城門,驚惑民庶。度支卿納哈出嘗匿官馬,又矯增制命,又受諸王斡即七寶帶一、鈔百六十錠。臣等議,其罪宜杖一百七,除名,斥還鄉里。」從之。壬戌,樞密院臣言:「雲南事已平,鎮西武靖王搠思班言,蒙古軍及哈剌章、羅羅斯諸種人叛者,或誅或降,雖已略定,其餘黨逃竄山谷,不能必其不反側,今請留荊王也速也不干及諸王鎖南等各領所部屯駐一二歲,以示威重。」從之。仍命豫王阿剌忒納失里分兵,給探馬赤三百、乞赤伯三百,共守一歲,以鎮輯之,餘軍皆遣還所部,統兵官召赴闕。時已命探馬赤為雲南行省平章政事,遂命總制境內軍事。潞州潞城縣大水。癸亥,諸王完者也不干所部蒙古民二百八十餘戶告饑,命河東宣慰司發官粟賑之。甲子,陜西行省言終南屯田去年大水,損禾稼四十餘頃,詔蠲其租。鎮寧王那海部曲二百,以風雪損孳畜,命嶺北行省賑糧兩月。欽察臺以名園為獻,命御史臺給贓罰鈔千錠酬其直。諸王乞八言:「臣每歲扈從時巡,為費甚廣,臣兄豫王阿剌忒納失里、弟亦失班,歲給鈔五百錠、幣帛各五千匹。敢視其例以請。」制可。詔:「故尚書省丞相脫脫,可視三寶奴例,以所籍家貲還其家。」御史臺臣言:「同僉中政院事殷仲容,奸貪邪佞,冒哀居官。」詔黜之。楊州泰興縣饑民萬三千餘戶,河南行省先賑以糧一月後以聞,許之。命遼陽行省發粟賑孛羅部內蒙古饑民。戊辰,奎章閣以纂修《經世大典》,請從翰林國史院取《脫卜赤顏》一書以紀太祖以來事跡,紹以命翰林學士承旨押不花、塔失海牙。押不花言:「《脫卜赤顏》事關秘禁,非可令外人傳寫,臣等不敢奉詔。」從之。增置拱衛司儀仗。命武備寺諸匠官避元籍。遣使召趙世延於集慶。詔以泥金畏兀字書《無量壽佛經》千部。壬申,散遣宣忠扈衛新籍軍士六百人還鄉里,期以七月一日還營。衡州路屬縣比歲旱蝗,仍大水,民食草木殆盡,又疫癘,死者十九,湖南道宣慰司請賑糧米萬石,從之。河中府蝗,晉寧、冀寧、大同、河間諸路屬縣,皆以旱不能種告饑。甘州阿兒思蘭免古妻忽都的斤以貞節旌其門。



 五月丙子,皇太子影殿造祭器如裕宗故事。敕建宮相都總管府公廨。丁丑,熒惑犯軒轅左角。賜宮相都總管府給驛璽書。調衛兵浚金水河。己卯,安南世子陳日阜遣其臣段子貞來朝貢。安慶之望江縣、淮安之山陽縣去歲皆水災,免其田租。丙戌,太禧宗禋院臣言:「累朝所建大萬安等十二寺,舊額僧三千一百五十人,歲例給糧,今其徒猥多,請汰去九百四十三人。」制可。常德府桃源州去歲水災,免其租。丁亥,復立怯憐口提舉司,仍隸中政院。命樞密院調軍士修京城。己丑,置雲南等處宣慰司都元帥府,以土官昭練為宣慰使都元帥。又置臨安元江等處宣慰司兼管軍萬戶府。孟定路、孟璟路並為軍民總管府,秩從三品。者線、蒙慶甸、銀沙羅等甸並為軍民府,秩從四品。孟並、孟廣、者樣等甸並設軍民長官司,秩從五品。益都路宋德讓、趙仁各輸米三百石賑膠州饑民九千戶,中書省臣請依輸粟補官例予官,從之。賑駐冬衛士二萬一千五百戶糧四月。庚寅,立雲南省蘆傳路軍民總管府,以土官為之,制授者各給金符。癸巳,雲南威楚路之蒲蠻猛吾來朝貢,願入銀為歲賦,詔為置散府一及土官三十三所,皆賜金銀符。甲午,太白犯畢宿。封宣政使脫因為薊國公。以平江官田五百頃立稻田提舉司,隸宮相都總管府。乙未,以陜西行臺御史大夫脫別臺知樞密院事。御史大夫玥璐不花累辭職,江西行省平章朵兒只以疾辭新任,並許之。脫忽思娘子繼主明宗幄殿,詔賜湘潭州民戶四萬為湯沐。奎章閣學士院纂修《皇朝經世大典》成。詔以泥金書佛經一藏。丙申,大駕幸上都。四川行省平章汪壽昌辭職,不允。敕在京百司日集公署,自晨及暮毋廢事。賑灤陽、桓州、李陵臺、昔寶赤、失八兒禿五驛鈔各二百錠。桓州民以所種麥獻,詔賜幣帛二匹,慰遣之。戊戌,次紅橋,臨視燕鐵木兒生祠。以太禧宗禋院所隸昭孝營繕司隸崇禧總管府。賑遼陽東路蒙古萬戶府饑民三千五百戶糧兩月。己亥,也兒吉尼知行樞密院事。八番西蠻官阿馬路奉方物入貢。高郵、寶應等縣去歲水,免其租。庚子,太陰犯太白。辛丑,太白經天。改阿速萬戶府為宣毅萬戶府,賜銀印,命伯顏領之。旌濟南章丘縣馬萬妻晉氏志節。癸卯,加也兒吉尼太尉,賜銀印。以河間鹽課鈔四千錠賑河間屬縣饑民四千一百戶。甲辰,詔通政院整治內外水陸驛傳。宣政院臣言:「舊制,列聖神御殿及諸寺所作佛事,每歲計二百十六,今汰其十六為定式。」制可。東昌、保定二路,濮、唐二州,有蟲食桑。寧夏、紹慶、保定、德安、河間諸路屬縣大水。六月乙巳朔,徵儲政院鈔三萬錠,給中宮道路之用。敕河南行省立阿不海牙政跡碑。監察御史韓元善言:「歷代國學皆盛,獨本朝國學生僅四百員,又復分辨蒙古、色目、漢人之額。請凡蒙古、色目、漢人,不限員額,皆得入學。」又監察御史陳守中言:「請凡仕者親老,別無侍丁奉養,不限地方名次,宜從優附近遷調,庶廣忠孝之道。」皆不報。發米五千石賑興和屬縣饑民。丁未,太白晝見。乙卯,監察御史陳良劾浙東廉訪使脫脫赤顏阿附權奸倒剌沙,其生母何氏本父之妾,而兄妻之,欺誑朝廷,封溫國夫人,請黜罷憲職,追還贈恩為宜。御史臺臣以聞,從之。旌大都右警巡院胡德妻曹氏貞節。壬戌,以鈔萬五千錠賑國王朵兒只等九部蒙古饑民三萬三百六十二戶。癸亥,詔:「諸官吏在職役或守代未任,為人行賕關說,即有所取者,官如十二章論贓,吏罷不敘終其身;雖無所取,訟起滅由己者,罪加常人一等。」甲子,太府監頒宮嬪、閹宦及宿衛士行帳資裝。免控鶴衛士當驛戶。丙寅,雲南出征軍悉還,烏撒羅羅蠻復殺戍軍黃海潮等,撒加伯又殺掠良民為亂,命雲南行省及行樞密院:「凡境上諸關戍兵,未可輕撤,宜視緩急以制其變。」丁卯,太陰犯畢,太白犯井。庚午,以揚州泰興、江都二縣去歲雨害稼,免今年租。樞密院臣言:「征西萬戶府軍七百人,自泰定以來,累經優恤,放還者四百五十人,今邊防軍少,例當追使還營。」從之。是月,晉寧、亦集乃二路旱,濟寧路蟲食桑,河南、晉寧二路諸屬縣蝗,大都、保定、真定、河間、東昌諸路屬州縣及諸屯水,彰德路臨漳縣漳水決。



 秋七月甲戌朔,賜野馬川等處駐冬衛士衣。藝文少監歐陽玄言:「先聖五十四代孫襲封衍聖公,爵最五等,秩登三品,而用四品銅印,於爵秩不稱。」詔鑄從三品印給之。德安府去年水,免今年田租。旌德安應山縣高可燾孝行。己卯,以雲南既平,惟祿餘等懼罪竄伏,降詔曲赦之。辛巳,只兒哈答兒坐罪當流遠,以唐其勢舅氏故釋之。壬午,祀太祖、太宗、睿宗御容於翰林國史院。監察御史張益等言:「欽察臺在英宗朝,陰與中政使咬住造謀,誣告脫歡察兒將構異圖,辭連潛邸,致出居海南。及天歷初,倒剌沙據上都,遣欽察臺以兵拒命,倒剌沙疑其有異志,復禽以歸,即追言昔日咬住之謀以自解。皇上即位,不念舊惡,擢居中書,而又自貽厥咎,以致奪官籍產。旋復釋宥,以為四川平章。今雲南未平,與蜀接境,其人反覆,不可信任,宜削官遠竄,仍沒入其家產。」臺臣以聞,詔奪其制命、金符,同妻孥禁錮於廣東,毋籍其家。仍詔諭御史:「凡憸人如欽察臺者,其極言之,毋隱。」鐵木兒補化辭御史大夫職,不允。乙酉,遣使代祀護國庇民廣濟福惠明著天妃。命西僧於大都萬歲山憫忠閣作佛事,起八月八日,至車駕還大都日止。丁亥,海南黎賊作亂,詔江西、湖廣兩省合兵捕之。諸王搠思吉亦兒甘卜、哈兒蠻,駙馬完者帖木兒遣使來獻蒲萄酒。壬辰,以知樞密院事脫別臺為御史大夫。癸巳,辰州、興國二路蟲傷稼,免今年租。甲午,歸德府雨傷稼,免今年租。給諸衛士及蒙古戶糧四月。乙未,立閔子書院於濟南。杭州火,賑被災民百九十戶。丁酉,調甘州兵千人、撒里畏兀兵五百人守參卜郎,以防土番。戊戌,封伯顏為浚寧王,賜金印,仍前太保、知樞密院事。高郵府去歲水災,免今年租。湖州安吉縣大水暴漲,漂死百九十人,人給鈔二十貫瘞之,存者賑糧兩月。庚子,廣西徭賊平,召諸王云都思帖木兒還。辛丑,懷德府洞蠻二十一洞田先什用等以方物來貢,還所虜生口八百餘人給其家。癸卯,知行樞密院事徹里帖木兒以兵討叛蠻鎖力哈迷失,戮其黨七百餘人。是月,河南、奉元屬縣蝗,大都、河間、漢陽屬縣水,冀寧屬縣雨雹傷稼,廬州去年水,寧夏霜為災,並免今年田租,賑寧夏鳴沙、蘭山二驛戶二百九十,定西州新軍戶千二百,應理州民戶千三百糧各一月,又賑龍興路饑民九百戶糧一月。大寧和眾縣何千妻柏都賽兒,夫亡以身殉葬,旌其門。



 八月甲辰朔,日有食之。封脫憐忽禿魯為靖恭王,沙藍朵兒只為懿德王,並給以塗金銀印。西域諸王卜賽因遣使忽都不丁來朝。灤陽驛戶增置馬牛各一,免其和市雜役。賜上都孔子廟碑。御史臺臣劾奏:「宣徽副使桑哥,比奉旨給宿衛士錢糧,稽緩九日,玩法欺公,罪當黜罷。」從之。己酉,以銀符二十八賜拱衛直百戶,命燕鐵木兒以鈔萬錠分賜蒙古孤寡者。辛亥,大駕南還大都。壬子,西域諸王答兒麻失里襲朵列帖木兒之位,遣諸王孛兒只吉臺等來朝貢。甲寅,雪別臺之孫月魯帖木兒、買閭也先來獻失剌奴,賜以金百兩、銀千五百兩、鈔五百錠、金帶一。命宣課提舉司毋收燕鐵木兒邸舍商貨稅。斡兒朵思之地頻年災,畜牧多死,民戶萬七千一百六十,命內史府給鈔二萬錠賑之。乙卯,太白犯軒轅大星。丙辰,封內史怯列該為豐國公。以星變,令群臣議赦。丁巳,命邠王不顏帖木兒圍獵於撫州。己未,立鎮寧王總管府於撫州。公主脫脫灰來朝。以汴梁路尉氏縣賜伯顏為食邑。詔刑部鞫內侍撒里不花巫蠱事,凡當死者杖一百七,流廣東、西。中書省臣言:「明年海運糧二百四十萬石,已令江浙運二百二十萬,河南二十萬。今請令江浙復增二十萬,本省參政杜貞督領。」從之。復命賑糶米五萬石濟京城貧民。旌濟寧路魏鐸孝行、揚州路呂天麟妻韋氏貞節。庚申,太白犯軒轅左角。中書、樞密臣言:「西域諸王不賽因,其臣怯列木丁矯王命來朝,不賽因遣使來言,請執以歸。臣等議,宗籓之國,行人往來,執以付之,不可,宜令乘驛歸國以自辨。」制可。壬申,升侍正府秩正二品。是月,江浙諸路水潦害稼,計田十八萬八千七百三十八頃。景州自六月至是月不雨。澧州、泗州等縣去年水,免今年租。沅州饑,賑糶米二千石。金州及西和州頻年旱災,民饑,賑以陜西鹽課鈔五千錠。九月癸酉朔,市阿魯渾撒里宅,命燕鐵木兒奉皇子古納答剌居之。中書省臣言:「今歲當飼馬駝十四萬八千四百匹,京城飼六萬匹,餘令外郡分飼,每匹給芻粟價鈔四錠。」從之。乙亥,命留守司發軍士築駐蹕臺於大承天護聖寺東。御史臺臣劾奏:「四川行省參政馬鎔,發糧六千石餉雲南軍,中道輒還,預借俸鈔一十九錠以娶妾,又詬罵平章汪壽昌,罪雖蒙宥,難任宰輔。」帝曰:「綱常之理,尊卑之分,懵無所知,其何以居上而臨下!亟罷之。」丙子,太白犯填星。樞密院臣言:「雲南東川路總管普折兄那具,會祿餘兵,殺烏撒宣慰使月魯、東川路府判教化的二十餘人;又會伯忽侄阿福,領蒙古兵將擊羅羅斯。臣等與燕鐵木兒議,遣西域指揮使鎖住等發陜西都萬戶府兵,直抵羅羅斯,發碉門安撫司兵,絕大渡河,直抵邛部州,巡守關隘。」詔宣政院亦遣使同往督之。海南賊王周糾率十九洞黎蠻二萬餘人作亂,命調廣東、福建兵,隸湖廣行省左丞移剌四奴統領討捕之。阿速及斡羅思新戍邊者,命遼陽行省給其牛具糧食。己卯,發粟五千石賑興和路鷹坊。庚辰,樞密院臣言:「六月中,行樞密院官以兵與烏撒賊兵五戰,破之,惟祿餘竄伏未獲。」命四川行省給其軍餉。賑興和寶昌州饑民米二千石。御史臺臣言:「大聖壽萬安寺壇主司徒嚴吉祥,盜公物,畜妻孥,宜免其司徒、壇主之職。」從之。禁諸驛毋畜竄行馬,免控鶴戶雜役。湖州安吉縣久雨,太湖溢,漂民居二千八百九十戶,溺死男女百五十七人,命江浙行省賑恤之。丁亥,御史臺臣言:「江西行省參政李允中,乃故內侍李邦寧養子,器質庸下,誤叨重選,宜黜罷。」從之。庚寅,幸大承天護聖寺。以鈔五萬錠及預貸四川明年鹽課鈔五萬錠,給行樞密院軍需。祿餘寇順元路。癸巳,罷供需府覆實司,置廣誼司,秩正三品,以右丞撒迪領其務。御史臺臣劾太禧宗禋使童童淫侈不潔,不可以奉明禋;又,奎章閣監書博士柯九思,性非純良,行極矯譎,挾其末技,趨附權門,請罷黜之。乙未,以金虎符賜中書平章政事亦列赤。思州鎮遠府饑,賑米五百石。丁酉,雲南行省遣都事那海、鎮撫欒智等奉詔往諭祿餘及授以參政制命,至撒家關,祿餘拒不受,俄而賊大至,那海因與力戰,賊乃退。及晚,烏撒兵入順元境,左丞帖木兒不花禦戰,那海復就陣宣詔招之,遂遇害,帖木兒不花等斂兵還。壬寅,改隆祥總管府為隆祥使司,秩從二品。



 冬十月甲辰,遣秘書太監王珪等代祀岳鎮、海瀆、后土。乙巳,召行樞密院徹里鐵木兒、小雲失還朝。以前東川路總管普折子安樂襲其父職。己酉,時享於太廟。為皇子古納答剌作佛事,釋在京囚,死罪者二人,杖罪者四十七人。辛亥,召江南行臺御史大夫阿兒思蘭海牙赴闕。癸丑,幸大承天護聖寺。蒙古都元帥怯烈引兵擊阿禾賊黨於澂江路海中山,為雲梯登山,破其柵,殺賊五百餘人。禿堅之弟必剌都古彖失舉家赴海死。又獲禿堅弟二人、子三人,誅之。甲寅,杭州火,命江浙行省賑其不能自存者。丁巳,中書省臣言:「江浙平江、湖州等路水傷稼,明年海漕米二百六十萬石,恐不足,若令運百九十萬,而命河南發三十萬,江西發十萬為宜。又,遣官齎鈔十萬錠、鹽引三萬五千道,於通、漷、陵、滄四州,優價和糴米三十萬石。又,以鈔二萬五千錠、鹽引萬五千道,於通、漷二州,和糴粟豆十五萬石;以鈔三十萬錠,往遼陽懿、錦二州,和糴粟豆十萬石。」並從之。燒在京積年還倒昏鈔二百七十餘萬錠。戊午,詔還平江路大玉清昭應宮田百頃,官勿徵其租。己未,給宿衛士有官者芻豆。諸王卜賽因使者還西域,詔酬其所貢藥物價直。辛酉,命西僧作佛事於興聖宮,十有五日乃罷。吳江州大風雨,太湖溢,漂沒廬舍孳畜千九百七十家,命江浙行省給鈔千五百錠賑之。乙丑,立昭功萬戶都總使府,伯顏、鐵木兒補化並兼昭功萬戶都總使。丙寅,命大都路定時估,每月朔望送廣誼司,以酬物價。燕鐵木兒取BX牛五千於西域來獻。



 十一月壬申朔,日有食之。雲南行省言:「亦乞不薛之地所牧國馬,歲給鹽,以每月上寅日啖之,則馬健無病。比因伯忽叛亂,去南鹽不可到,馬多病死。」詔令四川行省以鹽給之。乙亥,李彥通、蕭不蘭奚等謀反,伏誅。丙子,封諸王斡即為保寧王,賜以印,以其先所受印賜諸王渾禿帖木兒之子庚兀臺。詔給移剌四奴分行省印。丁丑,興和路鷹坊及蒙古民萬一千一百餘戶,大雪畜牧凍死,賑米五千石。戊寅,樞密院臣言:「天歷兵興,以揚州重鎮,嘗假淮東宣慰司以兵權,今事已寧,宜以所部兵復隸河南行省。又,征西元帥府自泰定初調兵四千一百人戍龍剌、亦集乃,期以五年為代,今已十年,逃亡者眾,宜加優恤,期以來歲五月代還。」並從之。己卯,封蘸班為豳國公。庚辰,左、右欽察衛軍士千四百九十戶饑,命上都留守司賑之。辛巳,以戶部尚書耿煥為中書參知政事。癸未,詔養燕鐵木兒之子塔剌海為子,賜居第及所籍李彥通貲產。荊王也速也不干獻BX牛四百。詔:「每歲樞密院、宗正府遣官,與遼陽行省官巡歷諸郡,毋令諸王所部擾民。」隆祥司使晃忽兒不花言:「海南所建大興龍普明寺,工費浩穰,黎人不勝其擾,以故為亂。」詔湖廣行省臣玥璐不花及宣慰、宣撫二司領其役,仍命廉訪司蒞之。辛卯,諸王撒兒蠻遣使者七十四人來。賑左欽察衛撒敦等翼頂也兒古駐冬軍千五百八十戶。諸鹽課鈔以十分之一折收銀,銀每錠五十兩,折鈔二十五錠。乙未,敕宮相都總管府勿隸昭功都總使府。丁酉,以南陽府之嵩州,更賜伯顏為食邑。



 十二月戊申,陜西行臺御史捏古伯、高坦等劾奏:「本臺監察御史陳良,恃勢肆毒,徇私破法,請罷職籍贓,還歸田里。」有旨:「雖會赦,其準風憲例,追奪敕命,餘如所奏。」以黃金符鐫文曰「翊忠徇義迪節同勛」,賜西域親軍副都指揮使欽察,以旌其天歷初紅橋戰功。壬子,復命諸王忽剌出還鎮雲南。癸丑,撒敦獻斡羅思十六戶,酬以銀百七錠、鈔五千錠。以河間路清池、南皮縣牧地賜斡羅思駐冬,仍以忽里所牧官羊給之。河南河北道廉訪副使僧家奴言:「自古求忠臣必於孝子之門。今官於朝者,十年不省覲者有之,非無思親之心,實由朝廷無給假省親之制,而有擅離官次之禁。古律,諸職官父母在三百里,於三年聽一給定省假二十日;無父母者,五年聽一給拜墓假十日。以此推之,父母在三百里以至萬里,宜計道里遠近,定立假期,其應省覲匿而不省覲者,坐以罪。若詐冒假期,規避以掩其罪,與詐奔喪者同科。」御史臺臣以聞,命中書省、禮部、刑部及翰林、集賢、奎章閣議之。丁巳,雨木冰。戊午,西域諸王禿列帖木兒遣使獻西馬及蒲萄酒。預給四宿衛及諸潛邸衛士歲賜鈔,人二十錠。庚申,遣集賢直學士答失蠻詣真定玉華宮,祀睿宗及顯懿莊聖皇后神御殿。辛酉,遣兵部尚書也速不花、同僉通政院事忽納不花迎帝師。詔中書省、御史臺遣官詣各道,同廉訪司錄囚。癸亥,雨木冰。給征東元帥府兵仗。丁卯,御史臺臣言:「甘肅行省平章月魯帖木兒,既非蒙古族姓,且暗於事機,使總兵柄,恐非所宜。」詔樞密院勿令提調軍馬。己巳,御史臺臣言:「河東道廉訪副使忽哥兒不花,僉燕南道廉訪司事不顏忽都、王士元、郝志善,憲綱不振,宜免官。」從之。旌寧海州崔惟孝孝行。是歲,真定路屬州水,冀寧、河南二路旱,大饑。



\end{pinyinscope}