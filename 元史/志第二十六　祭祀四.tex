\article{志第二十六 祭祀四}

\begin{pinyinscope}

 ○宗廟下



 親謝儀,其目有八:



 一曰齋戒。前享三日,皇帝散齋二日於別殿,致齋一日於大次。應享官員受誓戒於中書省,如常儀。



 二曰陳設,如前親祀儀。



 三曰車駕出宮。前享一日,所司備儀從、內外仗,與應享之官兩行序立於崇天門外,太僕卿控御馬立於大明門外,諸侍臣及導駕官二十四人,俱於齋殿前左右分班立候。通事舍人引侍中跪奏請中嚴,俯伏興。少頃,侍中版奏外辦,皇帝即御座。四品以上應享執事官起居訖,侍中奏請升輿。皇帝出齋殿,降自正階,乘輿,華蓋傘扇如常儀。導駕官前導至大明門外,侍中進當輿前,奏請降輿,乘馬訖,導駕官分左右步導。門下侍郎跪奏請進發,俯伏興,前稱警蹕。至崇天門,門下侍郎奏請權停,敕眾官上馬。侍中承旨退,稱制可,門下侍郎退傳制,稱眾官上馬,贊者承傳,眾官出欞星門外,上馬訖,門下侍郎奏請進發,前稱警蹕,華蓋傘扇儀仗與眾官左右前引,教坊樂鼓吹不振作。至太廟欞星門外,紅橋南,贊者承傳眾官下馬。下馬訖,自卑而尊與儀仗倒卷而北,兩行駐立。駕至廟門,侍中奏請皇帝下馬,步入廟門。入廟門訖,侍中奏請升輿,尚輦奉輿,華蓋傘扇如常儀。導駕官前導,皇帝乘輿至大次前,侍中奏請降輿。皇帝降輿入就位,簾降,侍衛如式。尚食進膳,如常儀。禮儀使以祝冊奏御署訖,奉出,太廟令受之,各奠於坫,置各室祝案上。通事舍人承旨,敕眾官各還齋次。



 四曰省牲器,見前親祀儀。



 五曰晨祼。享日醜前五刻,光祿卿、良醖令、太官令入實籩豆簠簋尊罍,各如常儀。太樂令率工人二舞,以次入就位。禮直官引御史及執事者以次入就位。禮直官引太常卿、御史升殿點視陳設,退復位。禮直官引司徒等官詣各室,分香設酒如常儀。禮直官復引太常卿及御史、太廟令、太祝、宮闈令升殿、奉出帝後神主訖,各退降就拜位,立定。奉禮於殿上贊奉神主訖,奉禮贊曰「再拜」,贊者承傳,御史以下皆再拜訖,各就位。禮直官引攝太尉由南神門東偏門入就位,立定。協律郎跪俯伏,舉麾興,工鼓柷,宮縣樂作《思成之曲》,以黃鐘為宮,大呂為角,太簇為征,應鐘為羽,作文舞九成止。太尉以下皆再拜訖,禮直官引太尉詣盥洗位,宮縣樂作《肅寧之曲》,至位樂止,北向立,搢笏、盥手、帨手,執笏詣爵洗位,北向立,搢笏、洗瓚、拭瓚,以瓚授執事者。執笏升殿,宮縣樂作,至阼階下,樂止。升自阼階,登歌樂作,詣太祖尊彞所,西向立,樂止。執事者以瓚奉太尉,太尉搢笏執瓚。司尊者舉冪酌鬱鬯訖,太尉以瓚授執事者,執笏詣太祖神位前,搢笏跪,三上香,執事者以瓚奉太尉,太尉執瓚以鬯祼地訖,以虛瓚授執事者。執笏俯伏興,退出戶外,北向再拜訖,次詣各室,並如上儀。禮畢,降自阼階,復位。



 六曰進饌。太尉祼將畢,進饌如前儀。



 七曰酌獻。太尉既升祼,禮直官引博士,博士引禮儀使至大次前,北向立。通事舍人引侍中詣大次前,版奏請中嚴,皇帝服袞冕。坐少頃,侍中奏外辦,禮儀使跪奏請皇帝行禮,俯伏興。簾卷出次,禮儀使前導至西神門,華蓋傘扇停於門外,近侍從入,太禮使後從。殿中監跪進鎮圭,皇帝執圭入門,協律郎跪,俯伏興,舉麾,宮縣《順成之樂》作,至版位東向立,樂止。引禮官分左右侍立,禮儀使奏請皇帝再拜。奉禮曰「眾官再拜」,贊者承傳,凡在位者皆再拜。禮儀使奏請皇帝詣盥洗位,宮縣樂作,至位樂止。內侍跪取醿,興,沃水,又內侍跪取盤,承水。禮儀使奏請搢鎮圭,皇帝搢圭盥手。內侍跪取巾於篚,興,進。帨手訖,奉爵官以爵跪進。皇帝受爵,內侍奉醿沃水,又內侍奉盤承水。皇帝洗爵訖,內侍奉巾跪進。皇帝拭爵訖,內侍奠盤醿,又奠巾於篚,奉爵官受爵。禮儀使奏請執鎮圭,導升殿,宮縣樂作,至西階下,樂止。升自西階,登歌樂作。禮儀使前導詣太祖室尊彞所,東向立,樂止。宮縣樂作,奏《開成之曲》,奉爵官以爵蒞尊,執事者舉冪,侍中跪酌犧尊之泛齊,以爵授執事者。禮儀使前導,入詣太祖神位前,北向立。禮儀使奏請搢鎮圭,跪,又奏請三上香。上香訖,奉爵官以爵授進酒官,進酒官東向以爵跪進,禮儀使奏請執爵祭酒。執爵三祭酒於茅苴訖,以虛爵授進酒官,進酒官受爵以授奉爵官,退立尊彞所。進酒官進徹神案上所奠玉爵馬湩,東向跪進,禮儀使奏請執爵祭馬湩。祭訖以虛爵授進酒官,進酒官進奠神案上訖,退。禮儀使奏請執圭,俯伏興,司徒搢笏跪俎前,舉牲盤西向以進。禮儀使奏請搢鎮圭,皇帝搢圭,俯受牲盤,北向跪,奠神案上訖,禮儀奏請執圭興,前導出戶外褥位,北向立,樂止。舉祝官搢笏跪,對舉祝版。讀祝官北向跪,讀祝文訖,俯伏興。舉祝官奠祝版訖。先詣次室。次蒙古祝史詣室前致辭訖,禮儀使奏請再拜。拜訖,禮儀使前導詣各室,奏各室之樂。其酌獻、進牲體、祭馬湩,並如第一室之儀。既畢,禮儀使奏請詣飲福位。登歌樂作,至位,西向立,樂止。宮縣《厘成之樂》作,禮直官引司徒立於飲福位側,太祝以爵酌上尊福酒,合置一爵,以奉侍中,侍中受爵奉以立。禮儀使奏請皇帝再拜。拜訖,奏搢鎮圭跪,侍中東向以爵跪進。禮儀使奏請執爵三祭酒,又奏請啐酒。啐訖,以爵授侍中。禮儀使奏請受胙,太祝以黍稷飯籩授司徒,司徒東向跪進,皇帝受,以授左右。太祝又以胙肉俎跪授司徒,司徒跪進,皇帝受,以授左右,禮直官引司徒退立。侍中再以爵酒跪進,禮儀使奏請皇帝受爵,飲福酒訖,侍中受虛爵興,以授太祝。禮儀使奏請執鎮圭,俯伏興,又奏請再拜。拜訖,樂止。禮儀使前導還版位。登歌樂作,降自西階,樂止。宮縣樂作,至位樂止。奉禮於殿上唱太祝徹籩豆。宮縣《豐寧之樂》作,卒徹,樂止。奉禮曰「賜胙」,贊者唱「眾官再拜」,在位者皆再拜。送神樂作,《保成之曲》作,一成止。禮儀使奏請皇帝再拜,贊者承傳,在位者皆再拜。拜訖,禮儀使前奏禮畢,皇帝還大次。宮縣《昌寧之樂》作,出門,樂止。禮儀使奏請釋鎮圭,殿中監跪受,華蓋傘扇如常儀。入次,簾降。禮直官引太常卿、御史、太廟令、太祝、宮闈令升殿納神主訖,各降就位。贊者於殿上唱升納神主訖,奉禮曰「再拜」,御史以下諸執事者皆再拜訖,以次出。通事舍人、禮直官各引享官以次出,太樂令率工人二舞以次出,太廟令闔戶訖降乃退。祝版藏於匱。



 八曰車駕還宮。皇帝既還大次,侍中奏請解嚴。皇帝釋袞冕,停大次。五刻頃,尚食進膳,如常儀。所司備儀從、內外仗,與從祀諸執事官兩行序立太廟欞星門外。侍中版奏外辦,皇帝出次升輿,導駕官前導,華蓋傘扇如常儀。至廟門,太僕卿進御馬,侍中奏請皇帝降輿乘馬。乘馬訖,門下侍郎奏請進發,俯伏興退,前稱警蹕。至欞星門外,門下侍郎奏請權停,敕眾官上馬。侍中承旨退稱曰「制可」,門下侍郎退傳制,贊者承傳,眾官上馬畢,導駕官及華蓋傘扇分左右前導,稱警蹕,教坊樂鼓吹振作。至崇天門欞星門外,門下侍郎奏請權停,敕眾官下馬。贊者承傳,眾官下馬訖,左右前引入內石橋北,與儀仗倒卷而北,駐立。駕入崇天門,至大明門外降馬,升輿以入,駕既入,通事舍人承旨敕眾官皆退,宿衛官率衛士宿衛如式。



 攝祀儀,其目有九:



 一曰齋戒。享前三日,三獻官以下凡與祭員,皆公服受誓戒於中書省。是日質明,有司設金椅於省庭,一人執紅羅傘立於其左。奉禮郎率儀鸞局陳設版位,獻官諸執事位,俱藉以席,仍加紫綾褥。設初獻太尉位於省階少西,南向;大禮使位於其東,少南,西向;監祭御史位二,於通道之西,東向;監禮博士位二,於通道之東,西向;俱北上。設司徒亞終獻位於其南,北向,西上。次助奠七祀獻官,次太常卿、光祿卿、光祿丞、書祝官、讀祝官、太官令、良醖令、廩犧令、司尊彞、舉祝官、太官丞、廩犧丞、奉爵官、奉瓚官、盥爵官二、巾篚官、蒙古太祝、巫祝、點視儀衛、清道官及與祭官,依品級陳設,皆異位重行。太廟令、太樂令、郊社令、太祝位於通道之西,北向,東上。太廟丞、太樂丞、郊社丞、奉禮郎、協律郎、司天生位於通道之東,北向,西上。齋郎位於其後。贊者引行事等官,各就位,立定。次引初獻官立定。禮直官搢笏,讀誓文曰「某年某月某日,享於太廟,各揚其職,其或不敬,國有常刑」。散齋二日宿於正寢,致齋一日宿於祠所。散齋日治事如故,不吊喪問病,不作樂,不判署刑殺文字,不決罰罪人,不與穢惡事。致齋日惟享事得行,餘悉禁。凡與享之官,已齋而闕者,通攝行事。七品以下官先退,餘官再拜。禮直官贊「鞠躬」,「拜」,「興」,「拜」,「興」,「平立」,「禮畢」。守廟兵衛與太樂工人,俱清齋一宿。赴祝所之日,官給酒饌。



 二曰陳設。享前二日,所司設兵衛於廟門,禁斷行人。儀鸞局設幄幔於饌殿,所司設三獻官以下行事執事官次於齋房之所。前一日,太樂令率其屬設宮縣之樂於庭中。東方西方磬虡起北,鐘虡次之;南主北方磬虡起西,鐘虡次之。設十二鎛鐘於編縣之間。各依辰位。樹建鼓於四隅,置柷敔於北縣之內。柷一在道東,敔一在道西。路鼓一在柷之東南,晉鼓一在其後,又路鼓一在柷之西南。諸工人各於其後。東方西方,以北為上;南方北方,以西為上。文舞在北,武舞在南,立舞表於酂綴之間。又設登歌之樂於殿上前楹間。玉磬一虡在西,金鐘一虡在東,柷一在金鐘北稍西,敔一在玉磬北稍東。搏拊二,一在敔北,一在柷北,東西相向。歌工次之,餘工各位於縣後。其匏竹者立於階間,重行北向,相對為首。



 享前一日,太廟令率其屬掃除廟庭之內外;樞密院軍官一員,率軍人鏟除草穢,平治道路。又設七祀燎柴於廟門之外。又於室內鋪設神位於北牖下,當戶南向。每位設黼扆一,紫綾厚褥一,薄褥一,莞席一,繅席二,虎皮次席二。時暄則用桃枝竹席,幾在筵上。又設三獻官拜跪褥位二,一在室內,一在室外。學士院定撰祝冊訖,書祝官於饌幕具公服書祝訖,請初獻官署御名訖,以授太廟令。又設祝案於室戶外之右。又設三獻官位於殿下橫街之南,稍西,東向;亞獻終獻位稍卻,助奠七祀獻官又於其南;書祝官、讀祝官、舉祝官、太廟令、太官令、良醖令、廩犧令、太廟丞、太官丞位,又於其南;司尊彞、奉瓚官、奉爵官、盥洗巾篚、爵洗巾篚、蒙古太祝、蒙古巫祝、太祝、宮闈令及七祀司尊彞、盥洗巾篚,以次而南。又設齋郎位於其後。每等異位,重行,東向,北上。又設大禮使位於南神門東偏門稍北,北向。又設司徒、太常卿等位於橫街之南,稍東,西向,與亞終獻相對,司徒位在北,太常卿稍卻;太常同知、光祿卿、僉院、同僉院判、光祿丞、拱衛使,以次而南。又設監祭御史位二、監禮博士位二於橫街之北,西向,以北為上。又設協律郎位在宮縣樂虡西北,東向,大樂丞在樂虡之間。又設大樂令、協律郎位於登歌樂虡之間。又設牲榜於東神門外,南向。設太常卿位於牲位,南向。監祭御史位在太常卿之左,太官令次之,光祿丞、太官丞又次之,廩犧令位在牲西南,廩犧丞稍卻,俱北向,以右為上。又設諸太祝位於牲東,西向,以北為上。又設蒙古巫祝位於牲東南,北向。又設省饌位於省饌殿前,太常卿、光祿卿、光祿丞、太官令位於東,西向;監祭、監禮位於西,東向;皆北上。太廟令陳祝版於室右之祝案,又率祠祭局設籩豆簠簋。每室左十有二籩,右十有二豆,俱為四行。登三在籩豆之間,鉶三次之,簠二、簋二又次之,簠左簋右,俎七在簠簋之南,香案一次之,沙池又次之。又設每室尊罍於通廊,斝彞、黃彞各一,春夏用雞彞、鳥彞、犧尊二、象尊二,秋冬用著尊、壺尊,著尊二、山罍二,以次在本室南之左,皆加勺冪。為酌尊所,北向,西上。彞有舟坫冪。又設壺尊二、太尊二、山罍四,在殿下階間,俱北向,望室戶之左,皆有坫加冪,設而不酌。凡祭器,皆藉以席。又設七祀位於橫階之南道東,西向,以北為上。席皆以莞。設神版位,各於座首。又設祭器,每位左二籩,右二豆,簠一、簋一在籩豆間,俎一在籩前,爵坫一次之,壺尊二在神位之西,東向以北為上,皆有坫勺冪。又設三獻盥洗、爵洗在通街之西,橫街之南,北向。罍在洗西加勺,篚在洗東,皆實以巾。爵洗仍實以瓚,爵加盤坫。執罍篚者各位於後。又設七祀獻官盥洗位於七祀神位前,稍北。罍在洗西,篚在洗東,實以巾。又實爵於坫。執罍篚者各位於後。



 三曰習儀。享前二日,三獻以下諸執事官員赴太廟習儀。次日早,各具公服乘馬赴東華門,迎接禦香至廟省牲。



 四曰迎香。享前一日,有司告諭坊市,灑掃經行衢路,祗備香案。享前一日質明,三獻官以下及諸執事官,各具公服,六品以下官皆借紫服,詣崇天門下。太常禮儀院官一員奉御香,一員奉酒,二員奉馬湩,自內出;監祭、監禮、奉禮郎、太祝,分兩班前導;控鶴五人,一人執傘,從者四人,執儀仗在前行。至大明門,由正門出,教坊大樂作。至崇天門外,奉香、酒、馬湩者安置於輿,導引如前。行至外垣欞星門外,百官上馬,分兩班行於儀仗之外,清道官行於儀衛之先,兵馬司之兵夾道次之,金鼓又次之,京尹儀從又次之,教坊大樂為一隊次之。控鶴弩手各服其服,執儀仗左右成列次之,拱衛使居其中,儀鳳司細樂又次之。太常卿與博士、御史導於輿前,獻官、司徒、大禮使、助奠官從入至殿下。獻官奉香酒馬湩升自東階,入殿內通廊正位安置。禮直官引獻官降自東階,由東神門北偏門出,釋服。



 五曰省牲器,見親祀儀。



 六曰晨祼。祀日醜前五刻,太常卿、光祿卿、太廟令率其屬設燭於神位,遂同三獻官、司徒、大禮使等每室一人,分設御香酒醴,以金玉爵斝,酌馬湩、葡萄尚醖酒奠於神案。又陳籩豆之實。籩四行,以右為上。第一行,魚鱐在前,糗餌、粉糍次之。第二行,乾尞在前,乾棗、形鹽次之。第三行,鹿脯在前,榛實、乾桃次之。第四行,菱在前,芡、慄次之。豆四行,以左為上。第一行,芹菹在前,筍菹、葵菹次之。第二行,菁菹在前,韭菹、厓食次之。第三行,魚醢在前,兔醢、豚拍次之。第四行,鹿MZ在前,醓醢、糝食次之。簠實以稻粱,簋實以黍稷,登實以太羹,鉶實以和羹,尊彞、斝彞實以明水,黃彞實以鬱鬯,犧尊實以泛齊,象尊實以醴齊,著尊實以盎齊,山罍實以三酒,壺尊實以醍齊,太尊實以沈齊。凡齊之上尊實以明水,酒之上尊實以玄酒,其酒齊皆以上醖代之。又實七祀之祭器,每位左二籩,慄在前,鹿脯次之;右二豆,菁菹在前,鹿MZ次之。簠實以黍,簋實以稷,壺尊實以醍齊,其酒齊亦以上配代之。陳設訖,獻官以下行事執事官,各服其服,會於齊班。禮直官引太常卿、監祭、監禮、太廟令、太祝、宮闈令、諸執事官、齋郎,自南神門東偏門入就位,東西相向立定。候監祭、監禮按視殿之上下,徹去蓋冪,糾察不如儀者,退復位。禮直官引太常卿、監祭、監禮、太廟令、太祝、宮闈令升自東階,詣太祖室。蒙古太祝起帝主神冪,宮闈令起後主神冪。次詣每室,並如常儀畢,禮直官引太常卿以下諸執事官,當橫街間,重行,以西為上,北向立定。奉禮郎贊曰「奉神主訖,再拜」。禮直承傳,太常卿以下皆再拜訖,奉禮郎又贊曰「各就位」。禮直官引諸執事官各就位,次引太官令率齋郎由南神門東偏門以次出。贊者引三獻官、司徒、大禮使、七祀獻官、諸行事官,由南神門東偏門入,各就位,立定。禮直官進於初獻官之左,贊曰「有司謹具,請行事」,退復位。協律郎跪,俯伏興,舉麾興工鼓柷,宮縣樂奏《思成之曲》九成,文舞九變。奉禮郎贊再拜,在位者皆再拜。奉禮又贊諸執事者各就位,禮直官引奉瓚、奉爵、盥爵、洗巾篚執事官各就位,立定。禮直官引初獻官詣盥洗位,宮縣樂作無射宮《肅寧之曲》,至位北向立定;搢笏、盥手、帨手,執笏詣爵洗位,至位北向立定;搢笏、執瓚、洗瓚、拭瓚,以瓚授執事者。執笏,樂止。登歌樂作,奏夾鐘宮《肅寧之曲》,升自東階,樂止。詣太祖酌尊所,西向立,搢笏,執事者以瓚授初獻官,執瓚。司尊彞跪舉冪,良醖令跪酌黃彞鬱鬯,初獻以瓚授執事者,執笏詣太祖神位前,北向立,搢笏跪,三上香。執事者以瓚授初獻,初獻執瓚以鬯灌於沙池,以瓚授執事者,執笏,俯伏興,出室戶外,北向立。再拜訖,詣每室祼鬯如上儀。俱畢,禮直官引初獻降自東階,登歌樂作,奏夾鐘宮《肅寧之曲》。復位,樂止。



 七曰饋食。初獻既祼,如前進饌儀。



 八曰酌獻。太祝立茅苴於盤。禮直官引初獻詣盥洗位,宮縣樂作,奏無射宮《肅寧之曲》,至位北向立;搢笏、盥手、帨手,執笏詣爵洗位;至位,搢笏、執爵、洗爵、拭爵、以爵授執事者,執笏,樂止。登歌樂作,奏夾鐘宮《肅寧之曲》。升自東階,樂止。詣太祖酒尊所,西向立,搢笏執爵。司尊彞搢笏跪舉冪,良醖令搢笏跪酌犧尊之泛齊,以爵授執事者,執笏。宮縣樂作,奏無射宮《開成之曲》。詣太祖神座前,北向立,稍前,搢笏跪,三上香。執爵,三祭酒於茅苴,以爵授執事者,執笏,俯伏興,平立。請出室戶外,北向立,樂止,俟讀祝。舉祝官搢笏跪,對舉祝版,讀祝官跪讀祝文。讀訖,舉祝官奠祝版於案,執笏興,讀祝官俯伏興。禮直官贊再拜訖,次詣每室,酌獻如上儀,各奏本室之樂。獻畢,宮縣樂止。降自東階,登歌樂作,奏夾鐘宮《肅寧之曲》。初獻復位,立定。文舞退,武舞進,宮縣樂作,奏無射宮《肅寧之曲》。舞者立定,樂止。禮直官引亞獻詣盥洗位,至位北向立,搢笏、執爵、洗爵、拭爵,以爵授執事者。升自東階,詣太祖酌尊所,西向立,搢笏,執爵。司尊彞搢笏跪舉冪,良醖令搢笏跪酌象尊之醴齊,以爵授執事者,執笏。宮縣樂作,奏無射宮《肅寧之曲》。詣太祖神座前,北向立,稍前,搢笏跪,三上香,執爵三祭酒於茅苴,以爵授執事者,執笏俯伏興,平立,請出室戶外,北向立。再拜訖,次詣每室,酌獻並如上儀。獻畢,樂止。降自東階,復位立定。禮直官引終獻,如亞獻之儀,唯酌著尊之盎齊。禮畢,降復位。初終獻將行,贊者引七祀獻官詣盥洗位,搢笏、盥手、帨手訖,執笏詣酒尊所,搢笏、執爵、酌酒,以爵授執事者,執笏詣首位神座前,東向立,稍前,搢笏跪執爵,三祭酒於沙池,奠爵於案,執笏俯伏興,少退立,再拜訖,每位並如上儀。俱畢,七祀獻官俟終獻官降復位,立定。



 九曰祭馬湩。終獻酌獻將畢,禮直官分引初獻亞獻官、司徒、太禮使、助奠官、七祀獻官、太常卿、監祭、監禮、太廟令丞、蒙古庖人、巫祝等升殿。每室獻官一員,各立於戶外,太常卿、監祭、監禮以下立於其後。禮直官引獻官詣神座前,蒙古庖人割牲體以授獻官。獻官搢笏跪奠於帝主神位前,次奠於後主神位前訖,出笏退就拜位,搢笏跪。太廟令取案上先設金玉爵斝馬湩,蒲萄尚醖酒,以次授獻官,獻官皆祭於沙池。蒙古巫祝致詞訖,宮縣樂作同進饌之曲。初獻出笏就拜興,請出室戶外,北向立。俟眾獻官畢立,禮直官通贊曰「拜」,「興」,凡四拜。監祭、監禮以下從拜。皆作本朝跪禮。拜訖退,登歌樂作,降階,樂止。太祝徹籩豆,登歌樂作,奏夾鐘宮《豐寧之曲》。奉禮贊賜胙,贊者承傳,眾官再拜興。送神樂作,奏黃鐘宮《保成之曲》,一成而止。太祝各奉每室祝版,降自太階望瘞位,禮直官引三獻、司徒、大禮使、助奠、七祀獻官、太常卿、光祿卿、監祭、監禮視燔祝版,至位坎北南向跪,以祝版奠於柴,就拜興。俟半燎,禮直官贊可瘞。禮直官引三獻以下及諸執事者齋郎等,由南神門東偏門出至揖位,圓揖。樂工二舞以次從出。三獻之出也,禮直官分引太常卿、太廟令、監祭、監禮、蒙古太祝、宮闈令及各室太祝,升自東階,詣太祖神座前,升納神主,每室如儀。俱畢,降自東階,至橫街南,北向西上立定。奉禮贊曰「升納神主訖,再拜」。贊者承傳,再拜訖,以次出。禮畢,三獻官、司徒、大禮使、太常禮儀院使、光祿卿等官,奉胙進於闕庭。駕幸上都,則以驛赴奉進。



 攝行告謝儀:告前三日,三獻官以下諸執事官,各具公服赴中書省受誓戒。告前一日未正二刻,省牲器。至期質明,三獻官以下諸執事者各服法服,禮直官引太常卿、監祭御史、監禮博士、五令諸執事官先入就位。禮直官引監祭、監禮點視陳設畢,復位。禮直官引太常卿、監祭、監禮、太廟令、太祝、宮闈令奉遷各室神主訖,降自橫街,北向立定。奉禮郎贊再拜,在位官皆再拜訖,奉禮郎贊各就位訖,太官令、齋郎出。禮直官引三獻、司徒、光祿卿、捧瓚、爵盥、爵洗官入就位,立定。禮直官贊「有司謹具,請行事」,降神樂作,九成止。奉禮郎贊再拜,三獻以下再拜訖,奉禮郎贊諸執事者各就位,立定。禮直官引初獻詣盥洗位,盥手,詣爵洗位,洗瓚。詣第一室酒尊所,酌鬱鬯。詣神座前北向跪,搢笏三上香,奠幣執瓚,以鬯灌於沙池,執笏俯伏興。出室戶外,再拜訖,次詣各室,並如上儀。俱畢,降復位。司徒率齋郎進饌,如常儀。奠畢,降復位。禮直官引初獻詣盥洗位,盥手,詣爵洗位,洗爵。詣第一室酒尊所,酌酒。詣神座前,北向搢笏跪,三上香,執爵三祭酒於茅苴,以爵授執事者,執笏俯伏興,出室戶外,北向立。俟讀祝官讀祝文訖,再拜。詣每室,並如上儀。俱畢,降復位。禮直官引亞獻官盥手、洗爵、酌獻,並如初獻儀,惟不讀祝。俱畢,降復位。禮直官引終獻,並如亞獻儀。俱畢,復位。太祝徹籩豆,奉禮郎贊賜胙,眾官再拜。在位官皆再拜訖,禮直官引三獻官、司徒、太常卿、監祭、監禮視焚祝版幣帛,禮直官贊可瘞。禮畢,太常卿、監祭、監禮升納神主訖,降自橫階。奉禮郎贊再拜,在位官皆再拜訖,退。



 薦新儀:至日質明,太常禮儀院官屬赴廟所,皆公服俟於次。太廟令率其屬升殿,開室戶,不出神主,設籩豆俎、酒醴、馬湩及室戶內外褥位。又設盥洗位於階下,少東,西向。奉禮郎率儀鸞局設席褥版位於橫街南,又設盥盆巾帨二所於齊班幕前。凡與祭執事官皆盥手訖,太常官詣神廚點視神饌。執事者奉所薦饌物,各陳饌幕內。太常官以下入就位,東西重行,北向立定。禮直官贊「皆再拜」,「鞠躬」,「拜」,「興」,「拜」,「興」,「平立」,「各就位」。禮直官引太常次官一員,率執事者出詣饌所,奉饌入自正門,升自太階,奠各室神位前。執事者進時食,院官搢笏受而奠之。禮直官引太常禮儀使詣盥洗位,盥手帨手。升殿詣第一室神位前,搢笏,執事者注酒於杯,三祭酒,又注馬湩於杯,亦三祭之,奠杯於案。出笏,就拜興,出室戶外,北向立,再拜。每室俱畢,降復位,執事者皆降。禮直官贊「再拜」,「鞠躬」,「拜」,「興」,「拜」,「興」,「平立」,餘官率執事者升徹饌,出殿闔戶。禮直官引太常官以下俱出東神門外,圓揖。



 神御殿



 神御殿,舊稱影堂。所奉祖宗御容,皆紋綺局織錦為之。影堂所在:世祖帝後大聖壽萬安寺,裕宗帝後亦在焉;順宗帝後大普慶寺,仁宗帝後亦在焉;成宗帝後大天壽萬寧寺;武宗及二後大崇恩福元寺,為東西二殿;明宗帝後大天源延聖寺;英宗帝後大永福寺;也可皇后大護國仁王寺。世祖、武宗影堂,皆藏玉冊十有二牒,玉寶一鈕。仁守影堂,藏皇太子玉冊十有二牒,皇后玉冊十有二牒,玉寶一鈕。英宗影堂,藏皇帝玉冊十有二牒,玉寶一鈕,皇太子玉冊十有二牒。凡帝後冊寶,以匣匱金鎖鑰藏於太廟,此其分置者。



 其祭器,則黃金瓶斝盤盂之屬以十數,黃金塗銀香合碗楪之屬以百數,銀壺釜杯匜之屬稱是。玉器、水晶、瑪瑙之器為數不同,有玻璃瓶、琥珀勺。世祖影堂有真珠簾,又皆有珊瑚樹、碧甸子山之屬。



 其祭之日,常祭每月初一日、初八日、十五日、二十三日,節祭元日、清明、蕤賓、重陽、冬至、忌辰。其祭物,常祭以蔬果,節祭忌辰用牲。祭官便服,行三獻禮。加薦用羊羔、炙魚、饅頭、食其子、西域湯餅、圜米粥、砂糖飯羹。



 泰定二年,亦作顯宗影堂於大天源延聖寺,天歷元年廢。舊有崇福、殊祥二院,奉影堂祀事,乃改為泰禧院。二年,又改為太禧宗禋院,秩二品。既而復以祖宗所御殿尚稱影堂,更號神御殿。殿皆制名以冠之:世祖曰元壽,昭睿順聖皇后曰睿壽,南必皇后曰懿壽,裕宗曰明壽,成宗曰廣壽,順宗曰衍壽,武宗曰仁壽,文獻昭聖皇后曰昭壽,仁宗曰文壽,英宗曰宣壽,明宗曰景壽。且命學士擬其祭祀儀注,今闕。



 又有玉華宮孝思殿在真定,世祖所立。以忌日享祀太上皇、皇太后御容。本路官吏祭奠,太常博士按《宋會要》定其儀。所司前期置辦茶飯、香果。質明,禮直官、引獻官與陪位官以下,並公服入廟庭,西向立。俱再拜訖,引獻官詣殿正階下再拜,升階至案前褥位,三上香,三奠酒訖,就拜興。又再拜訖,引獻官復位,與陪位官以下俱再拜,退。仁宗皇慶二年秋八月庚辰,命大司徒田忠良詣真定致祭,依歲例給御香酒並犧牲祭物錢中統鈔一百錠。延祐四年,始用登歌樂,行三獻禮。七年,太常博士言影堂用太常禮樂非是,制罷之,歲時本處依舊禮致祭。



 其太祖、太宗、睿宗御容在翰林者,至元十五年十一月,命承旨和禮霍孫寫太祖御容。十六年二月,復命寫太上皇御容,與太宗舊御容,俱置翰林院,院官春秋致祭。二十四年二月,翰林院言舊院屋敝,新院屋才六間,三朝御容宜於太常寺安奉,後仍遷新院。至大四年,翰林院移署舊尚書省,有旨月祭。中書平章完澤等言:「祭祀非小事,太廟歲一祭,執事諸臣受戒誓三日乃行事,今此輕易非宜。舊置翰林院御容,春秋二祭,不必增益。」制若曰「可」。至治三年遷置普慶寺,祀禮廢。泰定二年八月,中書省臣言當祭如故,乃命承旨斡赤齎香酒至大都,同省臣祭於寺。四年,造影堂於石佛寺,未及遷。至順元年七月,即普慶寺祭如故事。二年,復祀於翰林國史院。重改至元之六年,翰林院言三朝御容祭所甚隘,兼歲久屋漏,於石佛寺新影堂奉安為宜。中書省臣奏,此世祖定制,當仍其舊,制可。



\end{pinyinscope}