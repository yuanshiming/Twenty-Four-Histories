\article{志第十九 禮樂二}

\begin{pinyinscope}

 ○制樂始末



 太祖初年,以河西高智耀言,徵用西夏舊樂。太宗十年十一月,宣聖五十一代孫衍聖公元措來朝,言於帝曰:「今禮樂散失,燕京、南京等處,亡金太常故臣及禮冊、樂器多存者,乞降旨收錄。」於是降旨,令各處管民官,如有亡金知禮樂舊人,可並其家屬徙赴東平,令元措領之,於本路稅課所給其食。十一年,元措奉旨至燕京,得金掌樂許政、掌禮王節及樂工翟剛等九十二人。十二年夏四月,始命制登歌樂,肄習於曲阜宣聖廟。十六年,太常用許政所舉大樂令苗蘭詣東平,指授工人,造琴十張,一弦、三弦、五弦、七弦、九弦者各二。



 憲宗二年三月五日,命東平萬戶嚴忠濟立局,制冠冕、法服、鐘磬、筍虡、儀物肄習。五月十三日,召太常禮樂人赴日月山。八月七日,學士魏祥卿、徐世隆,郎中姚樞等,以樂工李明昌、許政、吳德、段楫、寇忠、杜延年、趙德等五十餘人,見於行宮。帝問制作禮樂之始,世隆對曰:「堯、舜之世,禮樂興焉。」時明昌等各執鐘、磬、笛、簫、篪、塤、巢笙,於帝前奏之,曲終,復合奏之,凡三終。十一日,始用登歌樂祀昊天上帝於日月山。祭畢,命驛送樂工還東平。



 三年,時世祖居潛邸,命勾當東平府公事宋周臣兼領大樂禮官、樂工人等,常令肄習,仍令萬戶嚴忠濟依已降旨存恤。六年夏五月,世祖以潛邸次灤州,下教命嚴忠濟督宋周臣以所得禮樂舊人肄習,宜如故事勉行之,毋忽。冬十有一月,敕樂工老不堪任事者,以子孫代之,不足者,以他戶補之。



 中統元年春正月,命宣撫廉希憲等,召太常禮樂人至燕京。夏六月,命許唐臣等制樂器、公服、法服,秋七月七日,工畢。十一日,用新制雅樂,享祖宗於中書省。禮畢,賜預祭官及禮樂人百四十九人鈔有差。八月,命太常禮樂人復還東平。二年秋九月,敕太常少卿王鏞領東平樂工,常加督視肄習,以備朝廷之用。



 五年,太常寺言:「自古帝王功成作樂,樂各有名,盛德形容,於是乎在。伏睹皇上踐阼以來,留心至治,聲名文物,思復承平之舊,首敕有司,修完登歌、宮縣、八佾樂舞,以備郊廟之用。若稽古典,宜有徽稱。謹案歷代樂名,黃帝曰《咸池》、《龍門》、《大卷》、少昊《大淵》,顓頊《六莖》,高辛《五英》,唐堯《大咸》、《大章》,虞舜《大韶》,夏禹《大夏》,商湯《大濩》,周武《大武》。降及近代,咸有厥名,宋總名曰《大晟》,金總名曰《大和》。今採輿議,權以數名,伏乞詳定。曰《大成》,按《尚書》『簫韶九成,鳳凰來儀』。《樂記》曰『王者功成作樂』,《詩》云『展也大成』。曰《大明》,按《白虎通》言『如唐堯之德,能大明天人之道』。曰《大順》,《易》曰『天之所助者順』,又曰『順乎天而應乎人』。曰《大同》,《樂記》曰『樂者為同,禮者為異』。《禮運》曰『大道之行也,故人不獨親其親,不獨子其子,是之謂大同』。曰《大豫》,《易》曰『豫順以動,故天地如之』。《象》曰『雷出地奮,豫。先王以作樂崇德,殷薦之上帝,以配祖考』。」中書省遂定名曰《大成之樂》,乃上表稱賀。表曰:「離日中天,已睹文明之化;豫雷出地,又聞正大之音。神人以和,祖考來格。欽惟皇帝陛下,潤色洪業,游意太平,爰從龍邸之潛,久敬鳳儀之奏。及登寶位,申命鼎司,謂雖陳堂上之登歌,而尚闕庭前之佾舞。方嚴禋祀,當備聲容。屬天語之一宣,乃春官之畢會。臣等素無學術,徒有汗顏。聿求舊署之師工,仍討累朝之典故。按圖索器,永言和聲,較鐘律於積黍之中,續琴調於絕弦之後。金而模,石而琢,虡斯豎,筍斯橫,合八音而克諧,閱三歲而始就。列文武兩階之干羽,象帝王四面之宮庭,一洗哇淫之聲,可謂盛大之舉。既完雅器,未錫嘉名。蓋聞軒、昊以來,俱有《咸》、《云》之號,《莖》、《英》、《章》、《韶》以象德,《夏》、《濩》、《武》、《勺》以表功。洪惟國朝,誕受天命,地大物巨,人和歲豐。宜符古記之文,稱曰《大成之樂》。漢庭聚議,作章敢望於一夔;舜殿鳴弦,率舞願觀於百獸。」



 至元元年冬十有一月,括金樂器散在寺觀民家者。先是,括到燕京鐘、磬等器,凡三百九十有九事,下翟剛辨驗給價。至是,大興府又以所括鐘、磬樂器十事來進。太常因言:「亡金散失樂器,若止於燕京拘括,似為未盡,合於各路各觀民家括之,庶省鑄造。」於是奏檄各道宣慰司,括到鐘三百六十有七,磬十有七,錞一,送於太常。又中都、宣德、平灤、順天、河東、真定、西京、大名、濟南、北京、東平等處,括到大小鐘、磬五百六十有九,其完者,景鐘二,鎛鐘十六,大聲鐘十,中聲鐘一,小聲鐘二十有七,編鐘百五十有五,編磬七;其不完者,景鐘四,鎛鐘二十有三,大聲鐘十有三,中聲鐘一,小聲鐘四十有五,編鐘二百五十有一,編磬十有四。



 三年,初用宮縣、登歌樂、文武二舞於太廟。先是,東平萬戶嚴忠範奏:「太常登歌樂器樂工已完,宮縣樂、文武二舞未備,凡用人四百一十二,請以東平漏籍戶充之,合用樂器,官為置備。」制可,命中書省臣議行。於是中書命左三部、太常寺、少府監,於興禪寺置局,委官楊天祐、太祝郭敏董其事,大樂正翟剛辨驗音律,充收受樂器官。丞相耶律鑄又言:「今制宮縣大樂,內編磬十有二虡,宜於諸處選石材為之。」太常寺以新撥宮縣樂工、文武二舞四百一十二人,未習其藝,遣大樂令許政往東平教之。大樂署言:「堂上下樂舞官員及樂工,合用衣服、冠冕、靴履等物,乞行制造。」中書禮部移準太常博士,議定制度,下所屬制造。宮縣樂器既成,大樂署郭敏開坐名數以上:編鐘、磬三十有六虡,樹鼓四,建鞞、應同一座。晉鼓一,路鼓二,鞀鼓二,相鼓二,雅鼓二,柷一,敔一,笙二十有七,巢和竽。塤八,篪、簫、籥、笛各十,琴二十有七,瑟十有四,單鐸、雙鐸、鐃、錞、鉦、麾、旌、纛各二,補鑄編鐘百九十有二,靈壁石磬如其數。省臣言:「太廟殿室向成,宮縣樂器咸備,請征東平樂工,赴京師肄習,以俟享廟。」制可。秋七月,新樂服成,樂工至自東平,敕翰林院定撰八室樂章,大樂署編運舞節,俾肄習之。



 冬十有一月,有事於太廟,宮縣、登歌樂、文武二舞咸備。其迎送神曲曰《來成之曲》,烈祖曰《開成之曲》,太祖曰《武成之曲》,太宗曰《文成之曲》,皇伯考術赤曰《弼成之曲》,皇伯考察合帶曰《協成之曲》,睿宗曰《明成之曲》,定宗曰《熙成之曲》,憲宗曰《威成之曲》。初獻、升降曰《肅成之曲》,司徒奉俎曰《嘉成之曲》,文舞退、武舞進曰《和成之曲》,亞終獻、酌獻曰《順成之曲》,徹豆曰《豐成之曲》。文舞曰《武定文綏之舞》,武舞曰《內平外成之舞》。第一成象滅王罕,二成破西夏,三成克金,四成收西域、定河南,五成取西蜀、平南詔,六成臣高麗、服交趾。詳見《樂舞篇》。



 十有二月,籍近畿儒戶三百八十四人為樂工。先是,召用東平樂工凡四百一十二人。中書以東平地遠,惟留其戶九十有二,餘盡遣還,復入民籍。



 十一年秋八月,制內庭曲舞。中書以上皇帝冊寶,下太常太樂署編運無射宮《大寧》等曲,及上壽曲譜。當時議殿庭用雅樂,後不果用。



 十三年,以近畿樂戶多逃亡,僅得四十有二,復徵用東平樂工。十六年冬十月,命太常卿忽都於思召太常樂工。是月十一日,大樂令完顏椿等以樂工見於香閣,文郎魏英舞迎神黃鐘宮曲,武郎安仁舞亞獻無射宮曲。十八年冬十月,昭睿順聖皇后將祔廟,制昭睿順聖皇后室曲舞。



 十九年,王積翁奏請徵亡宋雅樂器至京師,置於八作司。二十一年,大樂署言「宜付本署收掌」,中書命八作司與之。鎛鐘二十有七,編鐘七百二十有三,持磬二十有二,編磬二十有八,鐃六,單鐸、雙鐸各五,鉦、錞各八。二十二年冬閏十有一月,太常卿忽都於思奏:「大樂見用石磬,聲律不協。稽諸古典,磬石莫善於泗濱,女直未嘗得此。今泗在封疆之內,宜取其石以制磬。」從之。選審聽音律大樂正趙榮祖及識辨磬材石工牛全,詣泗州採之,得磬璞九十,制編磬二百三十。命大樂令陳革等料簡,應律者百有五。二十三年,忽都於思又奏:「太廟樂器,編鐘、笙匏,歲久就壞,音律不協。」遂補鑄編鐘八十有一,合律者五十,造笙匏三十有四。二十九年四月,太常太卿香山請採石增制編磬,遣孔鑄馳驛往泗州,得磬璞五十八,制磬九十。大樂令毛莊等審聽之,得應律磬五十有八,於是編磬始備。



 三十年夏六月,初立社稷,命大樂許德良運制曲譜,翰林國史院撰樂章,其降送神曰《鎮寧之曲》,初獻、盥洗、升壇、降壇、望瘞位皆《肅寧之曲》,正配位奠玉幣曰《億寧之曲》,司徒奉俎徹豆曰《豐寧之曲》,正配位酌獻曰《保寧之曲》,亞終獻曰《咸寧之曲》。按祭社稷、先農及大德六年祀天地五方帝,樂章皆用金舊名。釋奠宣聖,亦因宋不改。詳《樂章篇》。三十一年,世祖、裕宗祔廟,命大樂署編運曲譜舞節,翰林定撰樂章。世祖室曰《混成之曲》,裕宗室曰《昭成之曲》。



 成宗大德九年,新建郊壇既成,命大樂署編運曲譜舞節,翰林撰樂章。十一月二十八日,祀圜丘用之。其迎送神曰《天成之曲》,初獻奠玉幣曰《欽成之曲》,酌獻曰《明成之曲》,登降曰《隆成之曲》,亞終酌獻曰《和成之曲》,奉饌徹豆曰《寧成之曲》,望燎如登降,惟用黃鐘宮。文舞曰《崇德之舞》,武舞曰《定功之舞》。十年,命江浙行省制造宣聖廟樂器,以宋舊樂工施德仲審較應律,運至京師。秋八月,用於廟祀宣聖。先令翰林新撰樂章,命樂工習之,降送神曰《凝安之曲》,初獻、盥洗、升殿、降殿、望瘞皆《同安之曲》,奠幣曰《明安之曲》,奉俎曰《豐安之曲》,酌獻曰《成安之曲》,亞終獻曰《文安之曲》,徹豆曰《娛安之曲》。蓋舊曲也,新樂章不果用。



 十一年,武宗即位,祭告天地,命大樂署編運皇地祗酌獻大呂宮一曲及舞節,翰林撰樂章。無曲名。九月,順宗、成宗二室祔廟,下大樂署編運曲譜舞節,翰林撰樂章,順宗室曰《慶成之曲》,成宗室曰《守成之曲》。



 至大二年,親享太廟。皇帝入門奏《順成之曲》,盥洗、升殿用至元中初獻升降《肅成之曲》,亦曰《順成之曲》,出入小次奏《昌寧之曲》,迎神用至元中《來成之曲》,改曰《思成》,初獻、攝太尉盥洗、升殿奏《肅寧之曲》,酌獻太祖室仍用舊曲,改名《開成》,《開成》本至元中烈祖曲名,其詞則太祖舊曲也。



 睿宗室仍用舊曲,改名《武成》,此亦至元中太祖曲名,其詞則「神祖創業」以下仍舊。皇帝飲福、登歌奏《成之曲》,新制曲。文舞退、武舞進仍用舊曲,改名《肅寧》,舊名《和成》,其詞「天生五材,孰能去兵」以下是也。



 亞終獻、酌獻仍用舊曲,改名《肅寧》,舊名《順成》,其詞「幽明精禋」以下是也。徹豆曰《豐寧之曲》,舊名《豐成》,詞語亦異。送神曰《保成之曲》,皇帝出廟廷亦曰《昌寧之曲》。《太常集禮》曰:「樂章據孔思逮本錄之。國朝樂章皆用成字,凡用寧字者,金曲也。國初禮樂之事,悉用前代舊工,循習故常,遂有用其舊者。亦有不用其詞,而冒以舊號者,如郊祀先農等樂是也。」



 冬十有二月,始制先農樂章,以太常登歌樂祀之。先是,有命祀先農以登歌樂,如祭社稷之制。大樂署言「《禮》祀先農如社」,遂錄祭社林鐘宮《鎮寧》等曲以上,蓋金曲也。三年冬十月,置曲阜宣聖廟登歌樂。初,宣聖五十四代孫左三部照磨思逮言:「闕里宣聖祖廟,釋奠行禮久闕,祭服登歌之樂,未蒙寵賜。如蒙移咨江浙行省,於各處贍學祭餘子粒內,制造登歌樂器及祭服,以備祭祀,庶盡事神之禮。」中書允其請,移文江浙制造。至是,樂器成,運赴闕里用之。十有一月,敕以二十三日冬至,祀昊天上帝於南郊,配以太祖,令大樂署運制配位及親祀曲譜舞節,翰林撰樂章。皇帝出入中壝黃鐘宮曲二,盥洗黃鐘宮曲一,升殿登歌大呂宮曲一,酌獻黃鐘宮曲一,飲福登歌大呂宮曲一,出入小次黃鐘宮曲一。皆無曲名。四年夏六月,武宗祔廟,命樂正謝世寧等編曲譜舞節,翰林侍講學士張士觀撰樂章,曲名《威成之曲》。



 仁宗皇慶二年秋九月,用登歌樂祀太上皇睿宗。於真定玉華宮。自是歲用之,至延祐七年春三月奏罷。延祐五年,命各路府宣聖廟罷雅樂,選擇習古樂師教肄生徒,以供春秋祭祀。六年秋八月,議置三皇廟樂,不果行。七年,仁宗祔廟,命樂正劉瓊等編運酌獻樂譜舞節,翰林撰樂章,曲名曰《歆成之曲》。



 英宗至治二年冬十月,用登歌樂於太廟。是月,英宗祔廟,下大樂署編運樂譜舞節,翰林撰樂章,曲曰《獻成之曲》。文宗天歷二年春三月,明宗祔廟,下大樂署編運樂譜舞節,翰林定撰樂章,曲曰《永成之曲》。



 登歌樂器



 金部



 編鐘一虡,鐘十有六,範金為之。筍虡橫曰筍,植曰虡。皆雕繪樹羽,塗金雙鳳五,中列博山,崇牙十有六,縣以紅絨組。虡趺青龍籍地,以綠油臥梯二,加兩跗焉。筍兩端金螭首,銜鍮石璧翣,五色銷金流蘇,絳以紅絨維之。鐵杙者四,所以備欹側。在太室以礙地甓,因易以石麟。虡額識以金飾篆字。擊鐘者以茱萸木為之,合竹為柄。凡鐘,未奏,覆以黃羅;雨,覆以油絹。磬亦然。元初,鐘用宋、金舊器,其識曰「大晟」、「大和」、「景定」者是也。後增制,兼用之。



 石部



 編磬一虡,磬十有六,石為之。縣以紅絨紃,虡跗狻猊。拊磬者,以牛角為之。餘筍虡、崇牙、樹羽、璧翣、流蘇之制,並與鐘同。元初,磬亦用宋、金舊器。至元中,始採泗濱靈壁石為之。



 絲部



 琴十,一弦、三弦、五弦、七弦、九弦者各二。斫桐為面,梓為底,冰弦,木軫,漆質,金徽,長三尺九寸。首闊五寸二分,通足中高二寸七分,旁各高二寸;尾闊四寸一分,通足中高二寸,旁各高一寸五分。俱以黃綺夾囊貯之。琴卓髹以綠。



 瑟四,其制,底面皆用梓木,面施採色,兩端繪錦,長七尺。首闊尺有一寸九分,通足中高四寸,旁各高三寸;尾闊尺有一寸七分,通足中高五寸,旁各高三寸五分。硃絲為弦,凡二十有五,各設柱,兩頭有孔,疏通相連,以黃綺夾囊貯之。架四,髹以綠,金飾鳳首八。



 竹部



 簫二,編竹為之,每架十有六管,闊尺有六分。黑槍金鸞鳳為飾,鍮石釘鉸。以黃絨紃維於人項,左右復垂紅絨絳結。架以木為之,高尺有二寸,亦號排簫,韜以黃囊。



 笛一,斷竹為之,長尺有四寸,七孔,亦號長笛。纏以硃絲,垂以紅絨絳結,韜以黃囊。



 籥二,制如笛,三孔。纏以硃絲,垂以紅絨絳結,韜以黃囊。



 篪二,髹色如桐葉,七孔。纏以硃絲,垂以紅絨絳結,韜以黃囊。



 匏部



 巢笙四,和笙四,七星匏一,九曜匏一,閏餘匏一,皆以斑竹為之。玄髹底,置管匏中,施簧管端,參差如鳥翼。大者曰巢笙,次曰和笙,管皆十九,簧如之。十三簧者曰閏餘匏,九簧者曰九曜匏,七簧者曰七星匏,皆韜以黃囊。



 土部



 塤二,陶土為之,圍五寸半,長三寸四分,形如稱錘,六孔,上一,前二,後三,韜以黃囊。



 革部



 搏拊二,制如鼓而小,中實以糠,外髹以硃,繪以綠雲,系以青絨絳。兩手用之,或搏或拊,以節登歌之樂。



 木部



 柷一,以桐木為之,狀如方桶,繪山於上,髹以粉,旁為圓孔,納椎於中。椎以杞木為之,撞之以作樂。



 敔一,制以桐木,狀如伏虎,彩繪為飾,背有二十七鉏鋙刻,下承以盤。用竹長二尺四寸,破為十莖,其名曰籈,櫟其背以止樂。



 宮縣樂器



 金部



 鎛鐘十有二虡,虡一鐘,制視編鐘而大,依十二辰位特縣之,亦號辰鐘。筍虡硃髹、塗金,彩繪飛龍,跗東青龍,西白虎,南赤豸,北玄麟,素羅五色流蘇。餘制並與編鐘同。



 編鐘十有二虡,虡十有六鐘,制見《登歌》。此下樂器制與《登歌》同者,皆不重載。



 石部



 編磬十有六虡,虡十有二磬,制見《登歌》。筍虡與鎛鐘同。



 絲部



 琴二十有七,一弦者三,三弦、五弦、七弦、九弦者各六。



 瑟十有二。



 竹部



 簫十,籥十,篪十,笛十。



 匏部



 巢笙十。



 竽十,竹為之。與巢笙皆十九簧,惟指法各異。



 七星匏一,九曜匏一,閏餘匏一。



 土部



 塤八。



 革部



 晉鼓一,長六尺六寸,面徑四尺,圍丈有二尺,穹隆者居鼓面三之一,穹徑六尺六寸三分寸之一,面繪雲龍為飾,其皋陶以硃髹之,下承以彩繪趺座,並鼓高丈餘。在郊祀者,鞔以馬革。



 樹鼓四,每樹三鼓。其制高六尺六寸,中植以柱,曰建鼓。柱末為翔鷺,下施小圓輪。又為重斗,方蓋,並繚以彩繪。四角有竿,各垂璧翣流蘇,下以青狻猊四為趺。建旁挾二小鼓,曰鞞、曰應,樹樂縣之四隅。踏床、鼓桴,並髹以硃。



 雷鼓二,制如鼓而小,鞔以馬革,持其柄播之,旁耳自擊,郊祀用之。



 雷鞀二,亦以馬革鞔之,為大小鼓三,交午貫之以柄,郊祀用之。



 路鼓二,制如雷鼓,惟非馬革,祀宗廟用之。



 路鞀二,其制為大小二鼓,午貫之,旁各有耳,以柄搖之,耳往還自擊,不以馬革,祀宗廟用之。



 木部



 柷一,敔一。



 節樂之器



 麾一,制以絳繒,長七尺,畫升龍於上,以塗金龍首硃杠縣之。樂長執之,舉以作樂,偃以止樂。



 照燭二,以長竿置絳羅籠於其末,然燭於中。夜暗,麾遠難辨,樂正執之,舉以作樂,偃以止樂。



 文舞器



 纛二,制若旌幢,高七尺,杠首刻象牛首,下施硃繒蓋為三重,以導文舞。



 籥六十有四,木為之。象籥之制,舞人所執。



 翟六十有四,木柄,端刻龍首,飾以雉羽,綴以流蘇,舞人所執。



 武舞器



 旌二,制如纛,杠首棲以鳳,以導武舞。



 乾六十有四,木為之,加以彩繪,舞人所執。



 戚六十有四,制若劍然,舞人所執。《禮記注》:「戚,斧也。」今制與古異。



 金錞二,範銅為之,中虛,鼻象狻猊,木方趺。二人舉錞,築於趺上。



 金鉦二,制如銅盤,縣而擊之,以節樂。



 金鐃二,制如火斗,有柄,以銅為匡,疏其上如鈴,中有丸。執其柄而搖之,其聲鐃鐃然,用以止鼓。



 單鐸、雙鐸各二,制如小鐘,上有柄,以金為舌,用以振武舞。兩鐸通一柄者,號曰雙鐸。



 雅鼓二,制如漆筒,鞔以羊革,旁有兩紐。工人持之,築地以節舞。



 相鼓二,制如搏拊,以韋為表,實之以糠。拊其兩端,以相樂舞節。



 鞀鼓二。



 舞表



 表四,木桿,鑿方石樹之,用以識舞人之兆綴。



\end{pinyinscope}