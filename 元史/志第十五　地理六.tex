\article{志第十五 地理六}

\begin{pinyinscope}

 湖廣等處行中書省,為路三十、州十三、府三、安撫司十五、軍三,屬府三,屬州十七,屬縣一百五十,管番民總管一。本省陸站一百處,水站七十三處。



 江南湖北道肅政廉訪司



 武昌路,上。唐初為鄂州,又改江夏郡,又升武昌軍。宋為荊湖北路。元憲宗末年,世祖南伐,自黃州陽羅洑,橫橋梁,貫鐵鎖,至鄂州之白鹿磯,大兵畢渡,進薄城下,圍之數月,既而解去,歸即大位。至元十一年,丞相伯顏從陽羅洑南渡,權州事張晏然以城降,自是湖北州郡悉下。是年,立荊湖等路行中書省,並本道安撫司。十三年,設錄事司。十四年,立湖北宣慰司,改安撫司為鄂州路總管府,並鄂州行省入潭州行省。十八年,遷潭州行省於鄂州,移宣慰司於潭州。十九年,隨省處例罷宣慰司,本路隸行省。大德五年,以鄂州首來歸附,又世祖親征之地,改武昌路。戶一十一萬四千六百三十二,口六十一萬七千一百一十八。至元二十七年抄籍數。領司一、縣七。



 錄事司。



 縣七



 江夏,中。倚郭。咸寧,下。嘉魚,下。蒲圻,中。崇陽,中。通城,中。武昌。下。宋升壽昌軍,以其為江西沖要地也。元因之。至元十四年,升散府,治本縣。後革府,以縣屬本路。戶一萬五千八百五,口六萬四千五百九十八。



 岳州路,上。唐巴州,又改嶽州。宋為岳陽軍。元至元十二年歸附。十三年,立岳州路總管府。戶一十三萬七千五百八,口七十八萬七千七百四十三。領司一、縣三、州一。



 錄事司。



 縣三



 巴陵,上。倚郭。臨湘,中。華容。中。



 州一



 平江州,下。唐平江縣,宋因之。元元貞元年升州。



 常德路,上。唐朗州。宋常德府。元至元十二年,置常德府安撫司。十四年,改為總管府。戶二十萬六千四百二十五,口一百二萬六千四十二。領司一、縣一、州二。州領一縣。



 錄事司。



 縣一



 武陵。上。



 州二



 桃源州,中。宋置縣,元元貞元年升州。



 龍陽州,下。宋辰陽縣,元元貞元年升州。領一縣:



 沅江。下。本屬朗州。後來屬。



 澧州路,上。唐改澧陽郡,復改澧州。元至元十二年,立安撫司。十四年,改澧州路總管府。戶一十萬九千九百八十九,口一百一十一萬一千五百四十三。領司一、縣三、州二。



 錄事司。



 縣三



 澧陽,上。倚郭。石門,上。安鄉。下。



 州二



 慈利州,中。唐、宋皆為縣,元元貞元年升州。



 柿溪州。下。



 辰州路,下。唐改盧溪郡,復改辰州。宋因之。元改辰州路。戶八萬三千二百二十三,口一十一萬五千九百四十五。領縣四:



 沅陵,中。辰溪,下。盧溪。下。敘浦。下。



 沅州路,下。唐巫州,又改沅州,又為潭陽郡,又改敘州。宋為鎮遠州。元至元十二年,立沅州安撫司。十四年,改沅州路總管府。戶四萬八千六百三十二,口七萬九千五百四十五。領縣三:



 盧陽,下。黔陽,下。麻陽。下。



 興國路,下。本隋永興縣。宋置永興軍,又改興國軍。元至元十四年,升興國路總管府,舊隸江西。三十年,自江西割隸湖廣。戶五萬九百五十二,口四十萬七千六百一十六。領司一、縣三。



 錄事司。至元十七年立。



 縣三



 永興,下。倚郭。大冶,下。通山。下。



 漢陽府,唐初為沔州,又改沔陽郡。宋為漢陽軍。咸淳十年,郡守孟琦以城來歸。元至元十四年,升漢陽府。戶一萬四千四百八十六,口四萬八百六十六。領縣二:



 漢陽,至元二十二年,升中縣。漢川。下。



 歸州,下。唐初為歸州,又改巴東郡,又復為歸州。宋端平三年,元兵至江北,遂遷郡治於江南曲沱,次新灘,又次白沙南浦,今州治是也。德祐初歸附。元至元十二年,立安撫司。十四年,改歸州路總管府。十六年,降為州。戶七千四百九十二,口一萬九百六十四。領縣三:



 秭歸,下。倚郭。巴東,下。興山。



 靖州路,下。唐為夷、播、敘三州之境。宋為誠州,復改靖州。元至元十二年,立安撫司,明年,改靖州路總管府。戶二萬六千五百九十四,口六萬五千九百五十五。領縣三:



 永平,下。會同,下。通道。下。



 湖南道宣慰司



 嶺北湖南道肅政廉訪司



 天臨路,上。唐為潭州長沙郡。宋為湖南安撫司。元至元十三年,立安撫司。十四年,立行省,改潭州路總管府。十八年,遷行省於鄂州,徙湖南道宣慰司治潭州。天歷二年,以潛邸所幸,改天臨路。戶六十萬三千五百一,口一百八萬一千一十。領司一、縣五、州七。



 錄事司。宋有兵馬司,都監領之。元至元十四年改置。



 縣五



 長沙,上。倚郭。善化,倚郭。衡山,上。南岳衡山在焉。寧鄉,上。安化。下。



 州七



 醴陵州,中。唐、宋皆為縣。元元貞元年升州。



 瀏陽州,中。唐、宋皆為縣。元元貞元年升州。



 攸州,中。唐為縣,屬南雲州。宋屬潭州。元元貞元年升州。



 湘鄉州,下。唐、宋皆為縣。元元貞元年升州。



 湘潭州,中。唐、宋皆為縣。元元貞元年升州。



 益陽州,中。唐新康縣。宋安化縣。元元貞元年,升為益陽州。



 湘陰州,下。唐、宋皆為縣。元元貞元年升州。



 衡州路,上。唐初為衡州,又改衡陽郡,又仍為衡州。宋因之。元至元十三年,置安撫司。十四年,改衡州路總管府。十五年,置湖南宣慰司,以衡州為治所。十八年,移司於潭,衡州隸焉。戶一十一萬三千三百七十三,口二十萬七千五百二十三。領司一、縣三。本路屯田一百二十頃。



 錄事司。宋立兵馬司,分在城民戶為五廂。元至元十三年改立。



 縣三



 衡陽,上。倚郭。安仁,下。酃縣。下。



 道州路,下。唐為南營州,復改道州,復為江華郡。宋仍為道州。元至元十三年,置安撫司。十四年,改道州路總管府。戶七萬八千一十八,口一十萬九百八十九。領司一、縣四。



 錄事司。



 縣四



 營道,中。倚郭。寧遠,中。江華,中。永明。下。



 永州路,下。唐改零陵郡為永州,宋因之。元至元十三年,置安撫司。十四年,改永州路總管府。戶五萬五千六百六十六,口一十萬五千八百六十四。領司一、縣三。本路屯田一百三頃。



 錄事司。



 縣三



 零陵,上。倚郭。東安,上。祁陽。中。



 郴州路,下。唐改桂陽郡為郴州,宋因之。元至元十三年,置安撫司。十四年,改郴州路總管府。戶六萬一千二百五十九,口九萬五千一百一十九。領司一、縣六。



 錄事司。舊有兵馬司,至元十四年改立。



 縣六



 郴陽,中。倚郭。舊為敦化縣,至元十三年,改今名。宜章,中。永興,中。興寧,下。桂陽,下。桂東。下。



 全州路,下。石晉於清湘縣置全州,宋因之。元至元十三年,置安撫司。十四年,改全州路總管府。戶四萬一千六百四十五,口二十四萬五百一十九。領司一、縣二。



 錄事司。舊有兵馬司,至元十五年改立。



 縣二



 清湘,上。倚郭。灌陽。下。



 寶慶路,下。唐邵州,又為邵陽郡。宋仍為邵州,又升寶慶府。元至元十二年,立安撫司。十四年,改寶慶路總管府。戶七萬二千三百九,口一十二萬六千一百五。領司一、縣二。



 錄事司。



 縣二



 邵陽,上。倚郭。新化。中。



 武岡路,下。唐武岡縣。宋升為軍。元至元十三年,置安撫司。十四年,升武岡路總管府。戶七萬七千二百七,口三十五萬六千八百六十三。領司一、縣三。本路屯田八十六頃。



 錄事司。舊有兵馬司,領四廂,至元十五年改立。



 縣三



 武岡,上。倚郭。新寧,下。綏寧。下。



 桂陽路,下。唐郴州。宋升桂陽軍。元至元十二年,置安撫司。十四年,升桂陽路總管府。戶六萬五千五十七,口一十萬二千二百四。領司一、縣三。



 錄事司。



 縣三



 平陽,上。臨武,中。藍山。下。



 茶陵州,下。唐為縣,隸南雲州。宋隸衡州,升為軍,復為縣。元至元十九年,升為州。戶三萬六千六百四十二,口一十七萬七千二百二。



 耒陽州,下。唐、宋皆為縣,隸湘東郡。元至元十九年,升為州。戶二萬五千三百一十一,口一十一萬一十。



 常寧州,下。唐為縣,隸衡州。宋因之。元至元十九年,升為州。戶一萬八千四百三十一,口六萬九千四百二。



 廣西兩江道宣慰使司都元帥府大德二年,廣西兩江道宣慰司都元帥府言:「比者黃聖許叛亂,逃竄交趾,遺棄水田五百四十五頃,請募溪洞徭、獞民丁,於上浪、忠州諸處開屯耕種,緩急則令擊賊,深為便益。」從之。



 嶺南廣西道肅政廉訪司



 靜江路,上。唐初為桂州,又改始安郡,又改建陵郡,又置桂管,又升靜江軍。宋仍為靜江軍。元至元十三年,立廣西道宣撫司。十四年,改宣慰司。十五年,為靜江路總管府。元貞元年,並左右兩江宣慰司都元帥府為廣西兩江道宣慰司都元帥府,仍分司邕州。戶二十一萬八百五十二,口一百三十五萬二千六百七十八。領司一、縣十。



 錄事司。



 縣十



 臨桂,上。倚郭。興安,下。靈川,下。理定,下。義寧,下。修仁,下。荔浦,下。陽朔,下。永福,下。古縣。下。



 南寧路,下。唐初為南晉州,又改邕州,又為永寧郡。元至元十三年,立安撫司。十六年,改為邕州路總管府兼左右兩江溪洞鎮撫。泰定元年,改為南寧路。戶一萬五百四十二,口二萬四千五百二十。領司一、縣二。



 錄事司。



 縣二



 宣化,下。武緣。下。



 梧州路,下。唐改蒼梧郡,又仍為梧州。宋因之。元至元十四年,置安撫司。十六年,改梧州路總管府。戶五千二百,口一萬九百一十。領縣一:



 蒼梧。下。



 潯州路,下。唐改潯江郡,又仍為潯州。元至元十三年,置安撫司。十六年,改為總管府。戶九千二百四十八,口三萬八十九。領縣二:



 桂平,下。平南。下。



 柳州路,下。唐改龍城郡,又改柳州。元至元十三年,置安撫司。十六年,改柳州路總管府。戶一萬九千一百四十三,口三萬六百九十四。領縣三:



 柳城,下。倚郭。馬平,下。洛容。下。



 慶遠南丹溪洞等處軍民安撫司,唐為龍水郡,又改粵州。宋為慶遠府。元至元十三年,置安撫司。十六年,改慶遠路總管府。大德元年,中書省臣言:「南丹州安撫司及慶遠路相去為近,所隸戶少,請省之。」遂立慶遠南丹溪洞等處軍民安撫司。戶二萬六千五百三十七,口五萬二百五十三。領縣五:



 宜山,下。忻城,下。天河,下。思恩,下。河池。下。



 平樂府,唐以平樂縣置樂州,復改昭州,又為平樂郡,又仍為昭州。宋因之。元改為平樂府。戶七千六十七,口三萬三千八百二十。領縣四:



 平樂,下。倚郭。恭城,下。立山,下。龍平。下。



 鬱林州,下。唐為南尹州,又改貴州,又為鬱林州。宋因之。元至元十四年,仍行州事。戶九千五十三,口五萬一千五百二十八。領縣三:



 南流,下。興業,下。博白。下。



 容州,下。唐改銅州為容州,又改普寧郡,又置管內經略使。宋為寧遠軍。至元十三年,改安撫司。十六年,改容州路總管府。戶二千九百九十九,口七千八百五十四。領縣三:



 普寧,下。北流,下。陸川。下。



 象州,下。唐改為象郡,又改象州。元至元十三年,立安撫司。十五年,改象州路總管府。戶一萬九千五百五十八,口九萬二千一百二十六。領縣三:



 陽壽,下。來賓,下。武仙。下。



 賓州,下。唐以嶺方縣地置南方州,又為賓州,又改安城郡,又改嶺方郡,又仍為賓州。元至元十三年,置安撫司。十六年,改下路總管府。戶六千一百四十八,口三萬八千八百七十九。領縣三:



 嶺方,下。倚郭。上林,下。遷江。下。



 橫州,下。唐初為簡州,又改南簡州,又改橫州,又為寧浦郡。元至元十四年,立安撫司。十六年,改總管府。戶四千九十八,口三萬一千四百七十六。領縣二:



 寧浦,下。倚郭。永淳。下。



 融州,下。唐初為融州,又改融水郡,後仍為融州。宋為清遠軍。元至元十四年,置安撫司。十六年,改融州路總管府。二十二年,改散州。戶二萬一千三百九十三,口三萬九千三百三十四。領縣二:



 融水,下。懷遠。下。



 藤州,下。唐改感義郡,後仍為藤州。宋徙州治於大江西岸。元至元十三年,仍行州事。戶四千二百九十五,口一萬一千二百一十八。領縣二:



 鐔津,下。岑溪。下。



 賀州,下。唐改臨賀郡,後仍為賀州。宋因之。元至元十三年,仍行州事。戶八千六百七十六,口三萬九千二百三十五。領縣四:



 臨賀,下。倚郭。富川,下。桂嶺,下。懷集,下。宋屬廣州,至元十五年,以隸本州。



 貴州,下。唐改懷澤郡,後仍為貴州。元至元十四年,領鬱林縣。大德九年,省縣,止行州事。戶八千八百九十一,口二萬八百一十一。貴州地接八番,與播州相去二百餘里,乃湖廣、四川、雲南喉衿之地。大德六年,雲南行省右丞劉深徵八百媳婦,至貴州科夫,致宋隆濟等糾合諸蠻為亂,水東、水西、羅鬼諸蠻皆叛,劉深伏誅。



 左江。左江出源州界,至合江鎮與右江水合為一,流入橫州號鬱江。



 思明路,戶四千二百二十九,口一萬八千五百一十。



 太平路,戶五千三百一十九,口二萬二千一百八十六。



 右江。右江源出峨利州,與大理大盤水通。大盤在大理之威楚州。



 田州路軍民總管府,戶二千九百九十一,口一萬六千九百一。



 來安路軍民總管府。



 鎮安路。以上並闕。



 海北海南道宣慰司



 海北海南道肅政廉訪司至元三十年立。



 雷州路,下。唐初為南合州,又更名東合州,又為海康郡,又改雷州。元至元十五年,平章政事阿里海牙南征海外四州,雷州歸附,初置安撫司。十七年,即此州為海北海南道宣慰司治所,改安撫司為總管府,隸宣慰司。戶八萬九千五百三十五,口一十二萬五千三百一十。本路屯田一百六十五頃有奇。領縣三:



 海康,中。徐聞,下。遂溪。下。



 化州路,下。唐置羅州、辯州。宋廢羅州入辯州。復改辯州曰化州。元至元十五年,立安撫司。十七年,改總管府。戶一萬九千七百四十九,口五萬二千三百一十七。本路屯田五十五頃有奇。領縣三:



 石龍,下。吳川,下。石城。下。



 高州路,下。唐為高涼郡,又為高州。宋廢高州入竇州,後復置。元至元十五年,置安撫司。十七年,改總管府。戶一萬四千六百七十五,口四萬三千四百九十三。本路屯田四十五頃。領縣三:



 電白,下。茂名,下。信宜。下。



 欽州路,下。唐為寧越郡,又為欽州。宋因之。元至元十五年,置安撫司。十七年,改總管府。戶一萬三千五百五十九,口六萬一千三百九十三。領縣二:



 安遠,下。靈山。下。



 廉州路,下。唐為合浦郡,又改廉州。元至元十七年,設總管府。戶五千九百九十八,口一萬一千六百八十六。本路屯田四頃有奇。領縣二:



 合浦,下。倚郭。石康。下。



 乾寧軍民安撫司,唐以崖州之瓊山置瓊州,又為瓊山郡。宋為瓊管安撫都監。元至元十五年,隸海北海南道宣慰司。天歷二年,以潛邸所幸,改乾寧軍民安撫司。戶七萬五千八百三十七,口一十二萬八千一百八十四。本路屯田二百九十餘頃。領縣七:



 瓊山,下。倚郭。澄邁,下。臨高,下。文昌,下。樂會,下。會同,下。定安。下。



 南寧軍,唐儋州,改昌化郡。宋改昌化軍,又改南寧軍。元至元十五年,隸海北海南道宣慰司。戶九千六百二十七,口二萬三千六百五十二。領縣三:



 宜倫,下。昌化,下。感恩。下。



 萬安軍,唐萬安州。宋更為軍。元至元十五年,隸海北海南道宣慰司。戶五千三百四十一,口八千六百八十六。領縣二:



 萬安,下。倚郭。陵水。下。



 吉陽軍,唐振州。宋改崖州,又為硃崖郡,又改吉陽軍。元至元收附後,隸海北海南道宣慰司。戶一千四百三十九,口五千七百三十五。領縣一:



 寧遠。下。



 八番順元蠻夷官。至元十六年,潭州行省遣兩淮招討司經歷劉繼昌招降西南諸番,以龍方零為小龍番靜蠻軍安撫使,龍文求臥龍番南寧州安撫使,龍延三大龍番應天府安撫使,程延隨程番武盛軍安撫使,洪延暢洪番永盛軍安撫使,韋昌盛方番河中府安撫使,石延異石番太平軍安撫使,盧延陵盧番靜海軍安撫使,羅阿資羅甸國遏蠻軍安撫使,並懷遠大將軍、虎符,仍以兵三千戍之。是年,宣慰使塔海以西南八番、羅氏等國已歸附者,具以來上,洞寨凡千六百二十有六,戶凡十萬一千一百六十有八。西南五番千一百八十六寨,戶八萬九千四百。西南番三百一十五寨,大龍番三百六十寨。二十八年,從楊勝請,割八番洞蠻,自四川行省隸湖廣行省。三十年,四川行省官言:「思、播州元隸四川,近改入湖廣,今土人願仍其舊。」有旨遣問,還云,田氏、楊氏言,昨赴闕廷,取道湖廣甚便,況百姓相鄰,驛傳已立,願隸平章答剌罕。



 羅番遏蠻軍安撫司。



 程番武盛軍安撫司。



 金石番太平軍安撫司。



 臥龍番南寧州安撫司。



 小龍番靜蠻軍安撫司。



 大龍番應天府安撫司。



 木瓜犵狫蠻夷軍民長官。



 韋番蠻夷長官。



 洪番永盛軍安撫司。



 方番河中府安撫司。



 盧番靜海軍安撫司。



 盧番蠻夷軍民長官。



 定遠府。



 桑州。



 章龍州。



 必化州。



 小羅州。



 下思同州。



 〓朝宗縣。〓上橋縣。〓新安縣。〓麻峽縣。〓甕蓬縣。〓小羅縣。〓章龍縣。〓〓烏山縣。〓華山縣。〓都雲縣。〓羅博縣。



 管番民總管。



 小程番。以下各設蠻夷軍民長官。



 中嶆百納等處。



 底窩紫江等處。



 甕眼納八等處。



 獨塔等處。



 客當刻地等處。



 天臺等處。



 梯下。



 黨兀等處。



 勇都硃砂古瓦等處。



 大小化等處。



 洛甲洛屯等處。



 低當低界等處。



 獨石寨。



 百眼佐等處。



 羅來州。



 那歷州。



 重州。



 阿孟州。



 上龍州。



 峽江州。



 羅賴州。



 桑州。



 白州。



 北島州。



 羅那州。



 龍裡等寨。



 六寨等處。



 帖犵狫等處。



 本當三寨等處。



 山齋等處。



 羨塘帶夾等處。



 都雲桑林獨立等處。



 六洞柔遠等處。



 竹古弄等處。



 中都雲板水等處。



 金竹府。古瓦縣。



 都云軍民府。



 萬平等處。



 南寧。



 丹竹等處。



 陳蒙。



 李稍李殿等處。



 陽安等處。



 八千蠻。



 恭焦溪等處。



 都鎮。



 平溪等處。



 平月。



 李崖等處。



 陽並等處。



 盧山等處。



 乖西軍民府。皇慶元年立,以土官阿馬知府事,佩金符。



 順元等路軍民安撫司。至元二十年,四川行省討平九溪十八洞,以其酋長赴闕,定其地之可以設官者與其人之可以入官者,大處為州,小處為縣,並立總管府,聽順元路宣慰司節制。



 雍真乖西葛蠻等處。



 葛蠻雍真等處。



 曾竹等處。大德七年,順元同知宣撫事阿重嘗為曾竹蠻夷長官,以其叔父宋隆濟結諸蠻為亂,棄家朝京師,陳其事宜,深入烏撒、烏蒙,至於水東,招諭木樓苗、狫,生獲隆濟以獻。



 龍平寨。



 骨龍等處。



 底寨等處。



 茶山百納等處。



 納壩紫江等處。



 磨坡雷波等處。



 漕泥等處。



 青山遠地等處。



 木窩普沖普得等處。



 武當等處。



 養龍坑宿徵等處。



 骨龍龍里清江水樓雍眼等處。



 高橋青塘鴨水等處。



 落邦札佐等處。



 平遲安德等處。



 六廣等處。



 貴州等處。



 施溪樣頭。



 朵泥等處。



 水東。



 市北洞。



 思州軍民安撫司。婺川縣。



 鎮遠府。



 楠木洞。



 古州八萬洞。



 偏橋中寨。



 野雞平。



 德勝寨偏橋四甲等處。



 思印江等處。



 石千等處。



 曉愛瀘洞赤溪等處。



 卑帶洞大小田等處。



 黃道溪。



 省溪壩場等處。



 金容金達等處。



 臺蓬若洞住溪等處。



 洪安等處。



 葛章葛商等處。



 平頭著可通達等處。



 溶江芝子平茶等處。



 亮寨。



 沿河。



 龍泉平。思州舊治龍泉,及火其城,即移治清江。至元十七年,敕徙安撫司還舊治。



 佑溪。



 水特姜。



 楊溪公俄等處。



 麻勇洞。



 恩勒洞。



 大萬山蘇葛辦等處。



 五寨銅人等處。



 銅人大小江等處。



 德明洞。



 鳥羅龍幹等處。



 西山大洞等處。



 禿羅。



 浦口。



 高丹。



 福州。



 永州。



 乃州。



 鑾州。



 程州。



 三旺州。



 地州。



 忠州。



 天州。



 文州。



 合鳳州。



 芝山州。



 安習州。



 茆惸等團。



 荔枝。



 安化上中下蠻。



 曹滴等洞。



 洛卜寨。



 麥著土村。



 衙迪洞。



 會溪施容等處。



 感化州等處。



 契鋤洞。



 臘惹洞。



 勞巖洞。



 驢遲洞。



 來化州。



 客團等處。



 中古州樂墩洞。



 上里坪。



 洪州泊李等洞。



 張家洞。



 沿邊溪洞宣慰使司。至元二十八年,播州楊賽因不花言:「洞民近因籍戶,懷疑竄匿,乞降詔招集。」又言:「向所授安撫職任,隸順元宣慰司,其所管地,於四川行省為近,乞改為軍民宣撫司,直隸四川行省。」從之。以播州等處管軍萬戶楊漢英為紹慶珍州南平等處沿邊宣慰使,行播州軍民宣撫使、播州等處管軍萬戶,仍虎符。漢英即賽因不花也。仍頒所請詔旨,詔曰:「爰自前宋歸附,十五餘年,閱實戶數,乃有司當知之事,諸郡皆然,非獨爾播。自今以往,咸奠厥居,流移失所者,招諭復業,有司常加存恤,毋致煩擾,重困吾民。」



 播州軍民安撫司。



 黃平府。



 平溪上塘羅駱家等處。



 水軍等處。



 石粉羅家永安等處。



 六洞柔遠等處。



 錫樂平等處。



 白泥等處。



 南平綦江等處。



 珍州思寧等處。



 水煙等處。



 溱洞涪洞等處。



 洞天觀等處。



 葛浪洞等處。



 賽壩埡黎焦溪等處。



 小姑單張。



 倒柞等處。



 烏江等處。



 舊州草堂等處。



 恭溪杳洞。



 水囤等處。



 平伐月石等處。



 下壩。



 寨章。



 橫坡。



 平地寨。



 寨勞。



 寨勇。



 上塘。



 寨坦。



 岑奔。



 平莫。



 林種密秀。



 沿河佑溪等處。



 新添葛蠻安撫司。大德元年,授葛蠻安撫驛券一。



 南渭州。



 落葛谷鵝羅椿等處。



 昔不梁駱杯密約等處。



 乾溪吳地等處。



 噥聳古平等處。



 甕城都桑等處。



 都鎮馬乃等處。



 平普樂重墺等處。



 落同當等處。



 平族等處。



 獨祿。



 三陂地蓬等處。



 小葛龍洛邦到駱豆虎等處。



 羅月和。



 麥傲。



 大小田陂帶等處。



 都雲洞。



 洪安畫劑等處。



 谷霞寨。



 刺客寨。



 吾狂寨。



 割利寨。



 必郎寨。



 谷底寨。



 都谷郎寨。



 犵狫寨。



 平伐等處。大德元年,平伐酋領內附,乞隸於亦奚不薛,從之。



 安剌速。



 思樓寨。



 落暮寨。



 梅求望懷寨。



 甘長寨。



 桑州郎寨。



 永縣寨。



 平里縣寨。



 鎖州寨。



 雙隆。



 思母。



 歸仁。



 各丹。



 木當。



 雍郎客都等處。



 雍門犵狫等處。



 棲求等處仲家蠻。



 婁木等處。



 樂賴蒙囊吉利等處。



 華山谷津等處。



 青塘望懷甘長不列獨娘等處。



 光州。



 者者寨。



 安化思雲等洞。



 北遐洞。



 茅難思風北郡都變等處。



 必際縣。



 上黎平。



 潘樂盈等處。



 誠州富盈等處。



 赤畬洞。



 羅章特團等處。



 福水州。



 允州等處。



 欽村。



 硬頭三寨等處。



 顏村。



 水歷吾洞等處。



 順東。



 六龍圖。



 推寨。



 橘叩寨。



 黃頂寨。



 金竹等寨。



 格慢等寨。



 客蘆寨。



 地省等寨。



 平魏。



 白崖。



 雍門客當樂賴蒙囊大化木瓜等處。



 嘉州。



 分州。



 平硃。



 洛河洛腦等處。



 寧溪。



 甕除。



 麥穰。



 孤頂得同等處。



 甕包。



 三陂。



 控州。



 南平。



 獨山州。



 木洞。



 瓢洞。



 窖洞。



 大青山骨記等處。



 百佐等處。



 九十九寨蠻。



 當橋山齊硃穀列等處。



 虎列谷當等處。



 真滁杜珂等處。



 楊坪楊安等處。



 棣甫都城等處。



 楊友閬。



 百也客等處。



 阿落傅等寨。



 蒙楚。



 公洞龍木。



 三寨貓犵剌等處。



 黑土石。



 洛賓洛咸。



 益輪沿邊蠻。



 割和寨。



 王都谷浪寨。



 王大寨。



 只蛙寨。



 黃平下寨。



 林拱章秀拱江等處。



 密秀丹張。



 林種拱幫。



 西羅剖盆。



 杉木箐。



 各郎西。



 恭溪望成崖嶺等處。



 孤把。



 焦溪篤住等處。



 草堂等處。



 上桑直。



 下桑直。



 米坪。



 令其平尾等處。



 保靖州。



 特團等處。



 征東等處行中書省,領府二、司一、勸課使五。大德三年,立征東行省,未幾罷。至治元年復立,命高麗國王為左丞相。



 高麗國。事跡見《高麗傳》。至元十八年,王睶言:「本國置站凡四十,民畜凋弊。」敕並為二十站。三十年,沿海立水驛,自耽羅至鴨綠江並楊村、海口凡三十所。



 沈陽等路高麗軍民總管府。



 征東招討司。



 各道勸課使。



 慶尚州道。



 東界交州道。



 全羅州道。



 忠清州道。



 西海道。



 耽羅軍民總管府。大德五年立。



 河源附錄



 河源古無所見。《禹貢》導河,止自積石。漢使張騫持節,道西域,度玉門,見二水交流,發蔥嶺,趨於闐,匯鹽澤,伏流千里,至積石而再出。唐薛元鼎使吐蕃,訪河源,得之於悶磨黎山。然皆歷歲月,涉艱難,而其所得不過如此。世之論河源者,又皆推本二家。其說怪迂,總其實,皆非本真。意者漢、唐之時,外夷未盡臣服,而道未盡通,故其所往,每迂回艱阻,不能直抵其處而究其極也。



 元有天下,薄海內外,人跡所及,皆置驛傳,使驛往來,如行國中。至元十七年,命都實為招討使,佩金虎符,往求河源。都實既受命,是歲至河州。州之東六十里,有寧河驛。驛西南六十里,有山曰殺馬關,林麓穹隘,舉足浸高,行一日至巔。西去愈高,四閱月,始抵河源。是冬還報,並圖其城傳位置以聞。其後翰林學士潘昂霄從都實之弟闊闊出得其說,撰為《河源志》。臨川硃思本又從八里吉思家得帝師所藏梵字圖書,而以華文譯之,與昂霄所志,互有詳略。今取二家之書,考定其說,有不同者,附注于下。按河源在土蕃朵甘思西鄙,有泉百餘泓,沮洳散渙,弗可逼視,方可七八十里,履高山下瞰,燦若列星,以故名火敦腦兒。火敦,譯言星宿也。思本曰:「河源在中州西南,直四川馬湖蠻部之正西三千餘里,雲南麗江宣撫司之西北一千五百餘里,帝師撒思加地之西南二千餘里。水從地湧出如井。其井百餘,東北流百餘里,匯為大澤,曰火敦腦兒。」群流奔輳,近五七里,匯二巨澤,名阿剌腦兒。自西而東,連屬吞噬,行一日,迤邐東騖成川,號赤賓河。又二三日,水西南來,名亦裡出,與赤賓河合。又三四日,水南來,名忽闌。又水東南來,名也裏術,合流入赤賓,其流浸大,始名黃河,然水猶清,人可涉。思本曰:「忽闌河源,出自南山。其地大山峻嶺,綿亙千里,水流五百餘里,注也裏出河。也裏出河源,亦出自南山。西北流五百餘里,始與黃河合。」又一二日,歧為八九股,名也孫斡論,譯言九渡,通廣五七里,可度馬。又四五日,水渾濁,土人抱革囊,騎過之。聚落糾木乾象舟,傅髦革以濟,僅容兩人。自是兩山峽束,廣可一里、二里或半里,其深叵測。朵甘思東北有大雪山,名亦耳麻不莫剌,其山最高,譯言騰乞里塔,即昆侖也。山腹至頂皆雪,冬夏不消。土人言,遠年成冰時,六月見之。自八九股水至昆侖,行二十日。思本曰:「自渾水東北流二百餘里,與懷裡火禿河合。懷裡火禿河源自南山,水正北偏西流八百餘里,與黃河合,又東北流一百餘里,過郎麻哈地。又正北流一百餘里,乃折而西北流二百餘里,又折而正北流一百餘里,又折而東流,過昆侖山下,番名亦耳麻不莫剌。其山高峻非常,山麓綿亙五百餘里,河隨山足東流,過撒思加闊即、闊提地。」



 河行昆侖南半日,又四五日,至地名闊即及闊提,二地相屬。又三日,地名哈剌別里赤兒,四達之沖也,多寇盜,有官兵鎮之。近北二日,河水過之。思本曰:「河過闊提,與亦西八思今河合。亦西八思今河源自鐵豹嶺之北,正北流凡五百餘里,而與黃河合。」昆侖以西,人簡少,多處山南。山皆不穹峻,水亦散漫,獸有髦牛、野馬、狼、包、羱羊之類。其東,山益高,地亦漸下,岸狹隘,有狐可一躍而越之處。行五六日,有水西南來,名納鄰哈剌,譯言細黃河也。思本曰:「哈剌河自白狗嶺之北,水西北流五百餘里,與黃河合。」又兩日,水南來,名乞兒馬出。二水合流入河。思本曰:「自哈剌河與黃河合,正北流二百餘里,過阿以伯站,折而西北流,經昆侖之北二百餘里,與乞里馬出河合。乞里馬出河源自威、茂州之西北,岷山之北,水北流,即古當州境,正北流四百餘里,折而西北流,又五百餘里,與黃河合。」



 河水北行,轉西流,過昆侖北,一向東北流,約行半月,至貴德州,地名必赤裏,始有州治官府。州隸吐蕃等處宣慰司,司治河州。又四五日,至積石州,即《禹貢》積石。五日,至河州安鄉關。一日,至打羅坑。東北行一日,洮河水南來入河。思本曰:「自乞里馬出河與黃河合,又西北流,與鵬拶河合。鵬拶河源自鵬拶山之西北,水正西流七百餘里,過札塞塔失地,與黃河合。折而西北流三百餘里,又折而東北流,過西寧州、貴德州、馬嶺凡八百餘里,與邈水合。邈水源自青唐宿軍穀,正東流五百餘里,過二巴站與黃河合,又東北流,過土橋站古積石州來羌城、廓州構米站界都城凡五百餘里,過河州與野龐河合。野龐河源自西傾山之北,水東北流凡五百餘里,與黃河合。又東北流一百餘里,過踏白城銀川站與湟水、浩亹河合。湟水源自祁連山下,正東流一千餘里,注浩亹河。浩亹河源自刪丹州之南刪丹山下,水東南流七百餘里,注湟水,然後與黃河合。又東北流一百餘里,與洮河合。洮河源自羊撒嶺北,東北流,過臨洮府凡八百餘里,與黃河合。」又一日,至蘭州,過北卜渡。至鳴沙州,過應吉里,正東行。至寧夏府南,東行,即東勝州,隸大同路。自發源至漢地,南北澗溪,細流傍貫,莫知紀極。山皆草石,至積石方林木暢茂。世言河九折,彼地有二折,蓋乞兒馬出及貴德必赤裏也。思本曰:「自洮水與河合,又東北流,過達達地,凡八百餘里。過豐州西受降城,折而正東流,過達達地古天德軍中受降城、東受降城凡七百餘里。折而正南流,過大同路雲內州、東勝州與黑河合。黑河源自漁陽嶺之南,水正西流,凡五百餘里,與黃河合。又正南流,過保德州、葭州及興州境,又過臨州,凡一千餘里,與吃那河合。吃那河源自古宥州,東南流,過陜西省綏德州,凡七百餘里,與黃河合。又南流三百里,與延安河合。延安河源自陜西蘆子關亂山中,南流三百餘里,過延安府,折而正東流三百里,與黃河合。又南流三百里,與汾河合。汾河源自河東朔、武州之南亂山中,西南流,過管州,冀寧路汾州,霍州,晉寧路絳州,又西流,至龍門,凡一千二百餘里,始與黃河合。又南流二百里,過河中府,遇潼關與太華大山綿亙,水勢不可復南,乃折而東流。大概河源東北流,所歷皆西番地,至蘭州凡四千五百餘里,始入中國。又東北流,過達達地,凡二千五百餘里,始入河東境內。又南流至河中,凡一千八百餘里。通計九千餘里。」



 西北地附錄



 篤來帖木兒



 途魯吉。



 柯耳魯地。



 畏兀兒地。至元二十年,立畏兀兒四處站及交鈔庫。



 哥疾寧。



 可不里。



 巴達哈傷。



 途思。



 忒耳迷。



 不花剌。



 那黑沙不。



 的裡安。



 撒麻耳幹。



 忽氈。



 麻耳亦囊。



 可失哈耳。



 忽炭。



 柯提。



 兀提剌耳。



 巴補。



 訛跡邗。



 倭赤。



 苦叉。



 柯散。



 阿忒八失。



 八里茫。



 察赤。



 也云赤。



 亦剌八里。



 普剌。



 也迷失。



 阿里麻里。諸王海都行營於阿力麻裏等處,蓋其分地也。自上都西北行六千里,至回鶻五城,唐號北庭,置都護府。又西北行四五千里,至阿力麻里。至元五年,海都叛,舉兵南來,世祖逆敗之於北庭,又追至阿力麻裏,則又遠遁二千餘里。上令勿追,以皇子北平王統諸軍於阿力麻里以鎮之,命丞相安童往輔之。



 合剌火者。



 魯古塵。



 別失八里。至元十五年,授八撒察里虎符,掌別失八里畏兀城子里軍站事。十七年,以萬戶綦公直戍別失八里。十八年,從諸王阿只吉請,自大和嶺至別失八里置新站三十。二十年,立別失八里和州等處宣慰司。二十一年,阿只吉使來言:「元隸只必帖木兒二十四城之中,有察帶二城置達魯花赤,就付闊端,遂不隸省。」至是奉旨:「誠如所言,其還正之。」二十三年,遣侍衛新附兵千人屯田別失八里,置元帥府,即其地以總之。



 他古新。



 仰吉八里。



 古塔巴。



 彰八里。至元十五年,授朵魯知金符,掌彰八里軍站事。



 月祖伯



 撒耳柯思。



 阿蘭阿思。



 欽察。太宗甲午年,命諸王拔都征西域欽義、阿速、斡羅思等國。歲乙未,亦命憲宗往焉。歲丁酉,師至寬田吉思海傍,欽義酋長八赤蠻逃避海島中,適值大風,吹海水去而干,生禽八赤蠻,遂與諸王拔都征斡羅思,至也列贊城,七日破之。歲丁巳,出師南征,以駙馬剌真之子乞反為達魯花赤,鎮守斡羅思、阿思。歲癸丑,括斡羅思、阿思戶口。



 阿羅思。



 不里阿耳。



 撒吉剌。



 花剌子模。



 賽蘭。



 巴耳赤邗。



 氈的。



 不賽因。



 八哈剌因。



 怯失。



 八吉打。



 孫丹尼牙。



 忽裏模子。



 可咱隆。



 設剌子。



 洩剌失。



 苦法。



 瓦夕的。



 兀乞八剌。



 毛夕里。



 設里汪。



 羅耳。



 乞里茫沙杭。



 蘭巴撒耳。



 那哈完的。



 亦思法杭。



 撒瓦。



 柯傷。



 低廉。



 胡瓦耳。



 西模娘。



 阿剌模忒。



 可疾云。



 阿模里。



 撒里牙。



 塔米設。



 贊章。



 阿八哈耳。



 撒里茫。



 硃里章。



 的希思丹。



 巴耳打阿。



 打耳班。



 巴某。



 塔八辛。



 不思忒。



 法因。



 乃沙不耳。



 撒剌哈歹。



 巴瓦兒的。



 麻裏兀。



 塔裡幹。



 巴里黑。



 吉利吉思、撼合納、謙州、益蘭州等處。吉利吉思者,初以漢地女四十人,與烏斯之男結婚,取此義以名其地。南去大都萬有餘里。相傳乃滿部始居此,及元朝析其民為九千戶。其境長一千四百里,廣半之,謙河經其中,西北流。又西南有水曰阿浦,東北有水曰玉須,皆巨浸也,會於謙,而注于昴可剌河,北入於海。俗與諸國異。其語言則與畏吾兒同。廬帳而居,隨水草畜牧,頗知田作,遇雪則跨木馬逐獵。土產名馬、白黑海東青。昴可剌者,因水為名,附庸於吉利吉思,去大都二萬五千餘里。其語言與吉利吉思特異。晝長夜短,日沒時炙羊肋熟,東方已曙矣,即《唐史》所載骨利斡國也。烏斯亦因水為名,在吉利吉思東,謙河之北。其俗每歲六月上旬,刑白馬牛羊,灑馬湩,咸就烏斯沐漣以祭河神,謂其始祖所從出故也。撼合納猶言布囊也,蓋口小腹巨,地形類此,因以為名。在烏斯東,謙河之源所從出也。其境上惟有二山口可出入,山水林樾,險阻為甚,野獸多而畜字少。貧民無恆產者,皆以樺皮作廬帳,以白鹿負其行裝,取鹿乳,採松實,及劚山丹、芍藥等根為食。



 月亦乘木馬出獵。謙州亦以河為名,去大都九千里,在吉利吉思東南,謙河西南,唐麓嶺之北,居民數千家,悉蒙古、回紇人。有工匠數局,蓋國初所徙漢人也。地沃衍宜稼,夏種秋成,不煩耘耔。或云汪罕始居此地。益蘭者,蛇之稱也。初,州境山中居人,見一巨蛇,長數十步,從穴中出飲河水,腥聞數里,因以名州。至元七年,詔遣劉好禮為吉利吉思撼合納謙州益蘭州等處斷事官,即於此州修庫廩,置傳舍,以為治所。先是,數部民俗,皆以杞柳為杯皿,刳木為槽以濟水,不解鑄作農器,好禮聞諸朝,乃遣工匠,教為陶冶舟楫,土人便之。



 安南郡縣附錄



 安南,古交趾也。陳氏叛服之跡,已見本傳,今取其城邑之可紀者,錄於左方。



 大羅城路,漢交趾郡。唐置安南都護府。宋時郡人李公蘊立國於此。及陳氏立,以其屬地置龍興、天長、長安府。



 龍興府,本多岡鄉。陳氏有國,置龍興府。



 天長府,本多墨鄉,陳氏祖父所生之地。建行宮於此,歲一至,示不忘本,故改曰天長。



 長安府,本華閭洞,丁部領所生之地。五代末,部領立國於此。



 歸化江路,地接雲南。



 宣化江路,地接特磨道。



 沱江路,地接金齒。



 諒州江路,地接左右兩江。



 北江路,在羅城東岸,瀘江水分入北江,江有六橋。



 如月江路。



 南冊江路。



 大黃江路。



 烘路。



 快路。



 國威州,在羅城南。此以下州,多接雲南、廣西界,雖名州,其實洞也。



 古州,在北江。



 仙州,古龍編。



 富良。



 司農。一云楊舍。



 定邊。一云明媚。



 萬涯。一云明黃。



 文周。一云門州。



 七源。



 思浪。



 大原。一云黃源。



 通農。



 羅順。一云來神。



 梁舍。一云梁個。



 平源。



 光州。一云明蘇。



 渭龍。一云乙舍。



 道黃。即平林場。



 武寧。此以下縣,接雲南、廣西界,雖名縣,其實洞也。



 萬載。



 丘溫。



 新立。



 恍縣。



 紙縣。



 歷縣。



 闌橋。



 烏延。



 古勇。



 供縣。



 窟縣。



 上坡。



 門縣。



 清化府路,漢九真。隋、唐為愛州。其屬邑更號曰江、曰場、曰甲、曰社。



 梁江。



 波龍江。



 古農江。



 宋舍江。



 茶江。



 安暹江。



 分場。古文場。



 古藤甲。



 支明甲。



 古弘甲。



 古戰甲。



 緣甲。



 安府路,漢日南。隋、唐為驩州。



 倍江。



 惡江。



 偈江。



 尚路社。



 唐舍社。



 張舍社。



 演州路,本日南屬縣,曰扶演、安仁。唐改演州。



 孝江。



 多壁場。



 巨賴社。



 他袁社。



 布政府路,本日南郡象林縣,東濱海,西際真蠟,南接扶南,北連九德。東漢末,區連殺象林令,自立國,稱林邑。唐時有環王者,徙國於占,曰占城。今布政乃林邑故地。



 自安南大羅城至燕京,約一百一十五驛,計七千七百餘里。



 邊氓服役



 占城。



 王琴。



 蒲伽。



 道覽。



 淥淮。



 稔婆邏。



 獠。



\end{pinyinscope}