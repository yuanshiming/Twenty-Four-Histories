\article{志第五十 刑法一}

\begin{pinyinscope}

 自古有天下者,雖聖帝明王,不能去刑法以為治,是故道之以德義,而民弗從有機的世界整體。,則必律之以法,法復違焉,則刑闢之施,誠有不得已者。是以先王制刑,非以立威,乃所以輔治也。故《書》曰:「士制百姓於刑之中,以教祗德。」後世專務黷刑任法以為治者,無乃昧於本末輕重之義乎!歷代得失,考諸史可見已。



 元興,其初未有法守,百司斷理獄訟,循用金律,頗傷嚴刻。及世祖平宋,疆理混一,由是簡除繁苛,始定新律,頒之有司,號曰《至元新格》。仁宗之時,又以格例條畫有關於風紀者,類集成書,號曰《風憲宏綱》。至英宗時,復命宰執儒臣取前書而加損益焉,書成,號曰《大元通制》。其書之大綱有三:一曰詔制,二曰條格,三曰斷例。凡詔制為條九十有四,條格為條一千一百五十有一,斷例為條七百十有七,大概纂集世祖以來法制事例而已。其五刑之目:凡七下至五十七,謂之笞刑;凡六十七至一百七,謂之杖刑;其徒法,年數杖數,相附麗為加減,鹽徒盜賊既決而又鐐之;流則南人遷於遼陽迤北之地,北人遷於南方湖廣之鄉;死刑,則有斬而無絞,惡逆之極者,又有凌遲處死之法焉。蓋古者以墨、劓、剕、宮、大闢為五刑,後世除肉刑,乃以笞、杖、徒、流、死備五刑之數。元因之,更用輕典,蓋亦仁矣。世祖謂宰臣曰:「朕或怒,有罪者使汝殺,汝勿殺,必遲回一二日乃覆奏。」斯言也,雖古仁君,何以過之。自後繼體之君,惟刑之恤,凡郡國有疑獄,必遣官覆讞而從輕,死罪審錄無冤者,亦必待報,然後加刑。而大德間,王約復上言:「國朝之制,笞杖十減為七,今之杖一百者,宜止九十七,不當又加十也。」此其君臣之間,唯知輕典之為尚,百年之間,天下乂寧,亦豈偶然而致哉!然其弊也,南北異制,事類繁瑣,挾情之吏,舞弄文法,出入比附,用譎行私,而兇頑不法之徒,又數以赦宥獲免;至於西僧歲作佛事,或恣意縱囚,以售其奸宄,俾善良者喑啞而飲恨,識者病之。然而元之刑法,其得在仁厚,其失在乎緩弛而不知檢也。今按其實,條列而次第之,使後世有以考其得失,作《刑法志》。



 ◎名例



 ○五刑



 笞刑:



 七下,十七,二十七,三十七,四十七,五十七。



 杖刑:



 六十七,七十七,八十七,九十七,一百七。



 徒刑:



 一年,杖六十七;一年半,杖七十七;二年,杖八十七;二年半,杖九十七;三年,杖一百七。



 流刑:



 遼陽,湖廣,迤北。



 死刑:



 斬,凌遲處死。



 五服



 斬衰:三年。



 子為父、婦為夫之父之類。



 齊衰:三年,杖期,期,五月,三月。



 子為母,婦為夫之母之類。



 大功:九月,長殤九月,中殤七月。



 為同堂兄弟、為姑姊妹適人者之類。



 小功:五月,殤。



 為伯叔祖父母、為再從兄弟之類。



 緦麻:三月,殤。



 為族兄弟、為族曾祖父母之類。



 十惡



 謀反:



 謂謀危社稷。



 謀大逆:



 謂謀毀宗廟、山陵及宮闕。



 謀叛:



 謂謀背國從偽。



 惡逆:



 謂毆及謀殺祖父母、父母,殺伯叔父母、姑、兄、姊、外祖父母、夫、夫之祖父母、父母者。



 不道:



 謂殺一家非死罪三人,及支解人、造畜蟲毒、魘魅。



 大不敬:



 謂盜大祀神御之物、乘輿服御物;盜及偽造御寶;合和御藥,誤不如本方,及封題誤;若造御膳,誤犯食禁;御幸舟船,誤不牢固;指斥乘輿,情理切害,及對捍制使,而無人臣之禮禮。



 不孝:



 謂告言詛詈祖父母、父母,及祖父母、父母在,別籍異財,若供養有闕;居父母喪,身自嫁娶,若作樂釋服從吉;聞祖父母、父母喪,匿不舉哀;許稱祖父母、父母死。



 不睦:



 謂謀殺及賣緦麻以上親,毆告夫及大功以上尊長、小功尊屬。



 不義:



 謂殺本屬府主、剌史、縣令、見受業師,吏卒殺本部五品以上官長,及聞夫喪匿不舉哀,若作樂釋服從吉及改嫁。



 內亂:



 謂奸小功以上親、父祖妾,及與和者。



 八議



 議親:



 謂皇帝袒免以上親,及太皇太后、皇太后緦麻以上親,皇后小功以上親。



 議故:



 謂故舊。



 議賢:



 謂有大德行。



 議能:



 謂有大才業。



 議功:



 謂有大功勛。



 議貴:



 謂職事官三品以上,散官二品以上,及爵一品者。



 議勤:



 謂有大勤勞。



 議賓:



 謂承先代之後,為國賓者。



 贖刑附



 諸牧民官,公罪之輕者,許罰贖。



 諸職官犯夜者,贖。



 諸年老七十以上,年幼十五以下,不任杖責者,贖。



 諸罪人癃篤殘疾,有妨科決者,贖。



 衛禁



 諸掌宿衛,三日一更直,掌四門之鑰,昏閉晨啟,毋敢不慎。諸欲言事人,闌入宮殿,呼冀上聞,杖一百七,發元籍。諸擅帶刀闌入殿庭者,杖八十七,流遠。諸登皇城角樓,因為盜者,處死。諸闌入禁衛,盜金玉寶器者,處死。諸輒入禁苑,盜殺官獸者,為首杖八十七,徒二年,為從減一等,並刺字;知見不首者,笞四十七;掌門衛受財從放者,五十七;坐鋪守把軍人不訶問,二十七。諸漢人、南人投充宿衛士,總宿衛官輒收納之,並坐罪。諸大都、上都諸城門,夜有急務須出入者,遣官以夜行象牙圓符及織成聖旨啟門,門尉辯驗明白,乃許啟。雖有牙符而無織成聖旨者,不論何人,並勿啟,違者處死。



 職制上



 諸官府印章,長官掌收,次官封之,差故即以牒發次官,次其下者第封之,不得付其私人。諸郡縣城門鎖鑰,並從有司掌之。諸有司,凡薦舉刑名出納等文字,非有故,並須圓署行之。諸職官到任,距上司百里之內者公參,百里之外者免;上司輒非理徵會,稽失公務者,禁之。諸內外百司呈署文字,並須由下而上論定而後行之。諸省府以下百司,凡行公務,置硃銷簿,按治官以時考之。諸職官公坐,同職者以先到任居上,輒越次而坐者,正之。諸有司公事,各官連銜申稟其上司者,並自書其名。有故,從對讀首領官代書之,具述其故於名下,曹吏輒代書其名者,罪之。諸職官受代職除之處,從所便,具載解由。私赴都者,禁之。諸有司案牘籍帳,編次架閣。各路,提控案牘兼架閣庫官與經歷、知事同掌之;散府州縣,知事、提控案牘、都史目、典史掌之。任滿相沿交割,毋敢不慎。諸樞密院行省文卷,除軍數及邊關兵機不在考閱,餘並從監察御史考閱之。諸職官承上司他委,所治闕官者,許回申。不得擅令首領官吏攝事。諸職官押運官物赴都,除常所不差者,餘並置籍輪差。徇私不均者,罪其上司。諸吏員遷調,廉訪司書吏,奏差避道,路府州縣吏避貫。諸有司遺失印信,隨即尋獲者,罰俸一月;追尋不獲者,具申禮部別鑄。元掌印官解職坐罪,非獲元印,不得給由求敘。諸毀匿邊關文字者,流。諸蒙古人居官犯法論罪既定,必擇蒙古官斷之,行杖亦如之。諸四怯薛及諸王、駙馬、蒙古、色目之人,犯奸盜詐偽,從大宗正府治之。諸以親女獻當路權貴求進用,已得者追奪所受命,仍沒入其家。諸官吏在任,與親戚故舊及禮應追往之人追往者聽,餘並禁之。



 諸職官到任,輒受所部贄見儀物,比受贓減等論。諸職官受部民事後致謝食用之物者,笞二十七,記過。諸上司及出使官,於使所受其燕饗餽遺者,準不枉法減二等論,經過而受者各減一等,從臺憲察之。諸職官及有出身人,因事受財枉法者,除名不敘;不枉法者,殿三年,再犯不敘,無祿者減一等。以至元鈔為則,枉法:一貫至十貫,笞四十七,不滿貫者,量情斷罪,依例除名;一十貫以上至二十貫,五十七;二十貫以上至五十貫,杖七十七;五十貫以上至一百貫,八十七;一百貫之上,一百七。不枉法:一貫至二十貫,笞四十七,本等敘,不滿貫者,量情斷罪,解見任,別行求仕;二十貫以上至五十貫,五十七,注邊遠一任;五十貫以上至一百貫,杖六十七,降一等;一百貫以上至一百五十貫,七十七,降二等;一百五十貫以上至二百貫,八十七,降三等;二百貫以上至三百貫,九十七,降四等;三百貫以上,一百七,除名不敘。諸內外百司官吏,受贓悔過自首,無不盡不實者免罪,有不盡不實,止坐不盡之贓。若知人欲告而首及以贓還主,並減罪二等。聞知他處事發首者,計其日程雖不知,亦以知人欲告而首論。詭名代首者勿聽。犯人實有病故,許親屬代首。臺憲官吏受贓,不在準首之限。有司受人首告者,罪之。諸職官恐嚇有罪人求賂,未得財者,笞二十七。諸告官吏贓,有實取之者,有為過度人所諱而官吏初不知者,有官吏已知而姑付過度之家、事畢而後取之者,有本未嘗言而故以錢物置人家、指作過度而誣陷人者,止以錢物所在坐之,與錢人俱坐。諸職官但犯贓私,有罪狀明白者,停職聽斷。諸奴賤為官,但犯贓罪,除名。諸職官犯贓,生前贓狀明白,雖死猶責家屬納贓。諸官吏犯贓罪,遇原免,或自首免罪,過錢人即因人致罪,不坐。諸官吏贓罰,臺官問者歸臺,省官問者歸省。諸職官犯賊,罪狀已明,反誣告臨問官者,斷後仍徒。諸官吏家人受贓,減官吏法二等坐。官吏初不知,及知即首,官吏家人俱免;不即首,官吏減家人法二等坐,家人依本法。若官吏知情,故令家人受財,官吏依本法,家人免坐。官吏實不知者,止坐家人。諸職官受除未任,因承差而犯贓者,同見任論。邊遠遷轉官,已任而未受文憑犯贓者,亦如之。吏未出職受贓,既出職事發,罷所受職。諸錢穀官吏受贓,不枉法者,止計贓論罪,不殿年敘。諸職官受贓,聞知事發,回付到主,同知人欲告自首論,減二等科罪。枉法者降先職三等敘,不枉法者解職別敘。諸職官侵用官錢者,以枉法論,雖會赦,仍除名不敘。諸職官在任犯贓,被問贓狀已明而稱疾者,停其職歸對。諸職官所將親屬兼從,受所部財而無入己之贓,會赦還職。諸外任牧守受贓,被問垂成,近臣奏徵入朝者,執付元問官。諸職官犯贓在逃者,同獄成。諸職官受贓,丁憂,終制日究問。軍官不丁憂者,不在終制之限。諸職官犯贓,已承伏會赦者,免罪徵贓,黜降如條;未承伏者勿論。諸職官受贓,即改悔還主,其主猶執告者勿論。諸職官受財為人請托者,計贓論罪。諸小吏犯贓,並斷罪除名。諸庫子等職,已有出身,無添給祿米者,不與小吏犯贓同論。諸掾吏出身應入流,或以職官轉補,但犯贓,並同吏員坐除名。府州縣首領官非朝命者,同吏員。諸吏員取受非真犯者,不除名。



 諸流外官越受民詞者,笞一十七,首領官二十七,記過。諸臨民官於無職田州縣,虛徵其入於民者,斷罪解職,記過。諸職官頻入茶酒市肆及倡優之家者,斷罪罷職。諸監臨官私役弓手,笞二十七,三名已上加一等。占騎弓手馬,笞一十七,並記過名。本管官吏輒應付者,各減一等。諸內外官吏疾病滿百日者,作闕,期年後仕。諸職官連犯二罪,輕罪已斷,重罪始發,罪從已斷,殿降從後發。諸有過被問,詐死逃罪者,杖六十七,有官者罷職不敘,贓多者從重論。諸行省以下大小司存長官,非理折辱其首領官者,禁之。首領官有過失,聽申上司,不得擅問。長官處決不公,首領官執覆不從,許直申上司。諸隨朝官無故不公聚者,坐罪選待。



 諸職官已受宣敕,以地遠官卑,輒稱故不赴者,奪所受命,謫種田。或在任詐稱病而去者,三年後降二等敘,其同僚徇私與文書者,降一等敘。諸受命職官,闕期已及,或有辨證勾稽喪葬疾病公私諸務,妨阻不能之任者,許具始末詣本處有司自陳,保勘給據再敘,並任元注地方。有司保勘不實者,並坐之。諸受除官員,闕次未及,輒先往任所居住守代者,從本管上司究之。諸各衙門,輒將聽除及罷閑無祿私己之人差遣者,禁。諸職官親死不奔喪,杖六十七,降先職二等,雜職敘。未終喪赴官,笞四十七,降一等,終制日敘。若有罪詐稱親喪,杖八十七,除名不敘。親久沒稱始死,笞五十七,解見任,雜職敘。凡不丁父母憂者,罪與不奔喪同。諸官吏私罪被逮,無問已招未招,罹父母大故者,聽其奔赴丁憂,終制日追問,公罪並矜恕之。諸職官父母亡,匿喪縱宴樂,遇國哀,私家設音樂,並罷不敘。諸外任官員謁告,應有假故,具曹狀報所屬,仍置籍以記之。有托故者,風憲官糾而罪之。諸官吏遷葬祖父母、父母,給假二十日,並除馬程日七十里,限內俸錢仍給之,違限不至者勒停。諸職官任滿解由,應給而不給,不應給而給,及有過而不開寫者,罪及有司。解由到部,增損功罪不以實者,亦如之。諸罷免官吏,敘復給由而匿其過名者,罪及初給由有司。諸匿過求仕,已除事覺者,笞四十七,追奪不敘。諸職官年及致仕而不知止者,廉訪司糾黜之。諸職官被罪,理算殿年,以被問停職月日為始。諸遠方官員親年七十以上者,許元籍有司保勘,量注近闕便養,冒濫者坐罪。諸職官沒於王事者,其應繼之人,降二等廕敘。



 諸內外百司五品以上進上表章,並以蒙古字書,毋敢不敬,仍以漢字書其副。諸內外百司,凡進賀表箋,繕寫謄籍印識各以式,其輒犯廟諱御名者,禁之。諸內外百司應出給劄付,有額設譯史者,並以蒙古字書寫。諸內外百司有兼設蒙古、回回譯史者,每遇行移及勘合文字,標譯關防,仍兼用之。諸內外百司公移,尊卑有序,各守定制,惟執政出典外郡,申部公文,書姓不書名。諸人臣口傳聖旨行事者,禁之。



 諸大小機務,必由中書,惟樞密院、御史臺、徽政、宣政諸院許自言所職,其餘不由中書而輒上聞,既上聞而又不由中書徑下所司行之者,以違制論。所司亦不稟白而輒受以行之者,從監察御史、廉訪司糾之。諸中書機務,有洩其議者,量所洩事,聞奏論罪。諸省部官名隸宿衛者,晝出治事,夜入番直。諸檢校官勾檢中書及六曹之務,其有稽違,省掾呈省論罰,部吏就錄罪名開呈。



 諸行省擅役軍人營繕,雖公NZ,不奏請,猶議罪。諸行省差使軍官,非軍情者,禁之。諸行省長官二員,給金虎符典軍,惟雲南行省官皆給府。諸各處行省所轄軍官,軍情怠慢,從提調軍馬長官斷遣。其餘雜犯,受宣官以上咨稟,受敕官以下就斷。諸行省歲支錢糧,各處正官季一照勘,歲終會其成於行省,以式稽考,濫者征之,實者籍之,總其概,咨都省臺憲官閱實之。諸方面大臣,受金縱賊成亂者斬,僚佐受金,或阿順不能匡正,並坐罪,會赦仍除名。諸樞密院及各省所部軍官,其麾下征者、戍者、出者、處者、饑寒不贍,役使不均,代以私人,舉債倍息,在家曰逃,有力曰乏,惟單窮是使,惟貨賄是圖,以苦士卒,以耗兵籍,百戶有罪,罪及千戶,千戶有罪,罪及萬戶。萬戶有罪,從樞密院及行省帥府以其狀聞,隨事論罪。諸宣徽院所抽分馬牛羊,官嚴其程期,制其供億,謹其鈐束之法,以譏察之。其有欺官擾民者,廉訪司糾之。諸翰林院應譯寫制書,必呈中書省,共議其稿。其文卷非邊遠軍情重事,並從監察御史考閱之。諸宣政院文卷,除修佛事不在照刷外,其餘文卷及所隸內外司存,並照刷之。諸徽政院及怯憐口人匠,舊設諸府司文卷,並從臺憲照刷。



 諸臺官職掌,飭官箴,稽吏課,內秩群祀,外察行人,與聞軍國奏議,理達民庶冤辭,凡有司刑名、賦役、銓選、會計、調度、徵收、營繕、鞫勘、審讞、勾稽、及庶官廉貪,厲禁張弛,編民煢獨流移,強暴兼並,悉糾舉之。諸行臺官,主察行省宣慰司已下諸軍民官吏之作奸犯科者,窮民之流離失業者,豪強家之奪民利者,按察官之不稱職任者,餘視內臺立法同。諸御史臺所轄各道憲司,民有冤滯赴訴於臺者,咸著於籍,歲終則會以考其各道之殿最,而黜陟之。諸臺憲所察天下官吏贓污、欺詐、稽違,罪入於刑書者,歲會其數及其罪狀上之,藏於中書。諸內外臺,歲遣監察御史刷磨各省文卷,並察各道廉訪司官吏臧否,官弗稱者呈臺黜罰,吏弗稱者就罷之。諸風憲,薦舉必考其最績,彈劾必著其罪狀,舉劾失當,並坐之。諸殿中侍御史,凡遇廷臣奏事,必隨入內,在廷有不可與聞之人,即糾斥之;朝會祭祀,一切行禮,失儀越次及托故不至者,即糾罰之;文武百官謁假事故,三日以外者,以曹狀報之。凡官府創置,百官禮任,及被差往還,報曹狀並同。諸廉訪分司官,每季孟夏初旬,出錄囚,仲秋中旬,出按治,明年孟夏中旬還。其憚遠違期、托故避事者,從監察御史劾之。諸廉訪司分巡各路軍民,官吏有過,得罪狀明白者,六品以下牒總司論罪,五品以上申臺聞奏。諸廉訪司官,擅封點軍器庫者,笞三十七,解職別敘。諸官吏受贓,事主雖不告言,監察御史廉訪司察之,實者糾之。諸行省官及首領官受賂,隨省廉訪司察知者,上之臺,已下就問。諸行省理問所見問公事,廉訪司輒逮問者,禁止。諸職官受贓,廉訪司必親臨聽決,有必不能親臨者,摘敵品有司老成廉能正官問之。諸被按官吏,有冤抑者,詣御史臺陳理。所言實,罪被告,所言虛,罪告者,仍加等。其有故摭按問官吏以事者,禁之。諸按問職官贓,毋遽施刑,惟眾證已明而不款伏者,加刑問之,軍官則先奪所佩符而問之。諸風憲官吏但犯贓,加等斷罪,雖不枉法亦除名。諸方面之臣入覲,輒斂所部官吏俸錢備禮物者,禁之。違者罪之。



 諸湖南北、江西、兩廣接境溪洞蠻獠竊發,諸監臨禁治不嚴及故縱者,軍官笞三十七,管民官不坐。諸軍民官鎮撫邊陲,三年無嘯聚之盜者,民官減一資,軍官升散官一階;五年無者,軍民官各升散官一等。諸郡縣版籍,所司謹庋置之,正官相沿掌之。



 諸勸農官,每歲終則上其所治農桑水利之成績於本屬上司,本屬上司會所部之成績,以上於大司農若部,部考其勤惰成否,以上於省而殿最之。其在官怠其事隳其法者,罪之。諸職官行田,受民戶齊斂錢者,以多科斷。諸受財占民差徭者,以枉法論。諸額課所在,管民正官董其事,若以他故出,次官通攝之。諸額收錢糧,各處計吏,歲一詣省會之。有齊斂者,從按治官舉劾。諸郡縣歲以三限徵收稅糧,初限十月終,中限十一月終,末限十二月終。違者初限笞四十,再犯杖八十,但結攬及自願與結攬人等,並沒入其家財,仍依元科之數倍征之。若不差正官部糧,而以權官部之,或致失陷及輸不足者,達魯花赤管民官同坐。諸州縣義倉糧數不實,監臨失舉察者,罪之。



 諸職官於禁刑之日決斷公事者,罰俸一月,吏笞二十七,記過。諸有司斷諸小罪,輒以杖頭非法杖人致死,罪坐判署官吏。諸曾訴官吏之人有罪,其被訴官吏勿推。諸有司輒憑妄言帷薄私事逮系人者,笞四十七,解職,期年後敘。諸職官得代及休致,凡有追會,並同見任。其婚姻田債諸事,止令子孫弟侄陳訴,有司輒相侵陵者究之。諸職官告吏民毀罵,非親聞者勿問,違者罪之。諸職官聽訟者,事關有服之親並婚姻之家及曾受業之師與所仇嫌之人,應回避而不回避者,各以其所犯坐之。有輒以官法臨決尊長者,雖會赦,仍解職降敘。



 諸有司事關蒙古軍者,與管軍官約會問。諸管軍官、奧魯官及鹽運司、打捕鷹坊軍匠、各投下管領諸色人等,但犯強竊盜賊、偽造寶鈔、略賣人口、發塚放火、犯奸及諸死罪,並從有司歸問。其鬥訟、婚田、良賤、錢債、財產、宗從繼絕及科差不公自相告言者,從本管理問。若事關民戶者,從有司約會歸問,並從有司追逮,三約不至者,有司就便歸斷。諸州縣鄰境軍民相關詞訟,元告就被論官司歸斷,不在約會之例。斷不當理,許赴上司陳訴,罪及元斷官吏。諸僧、道、儒人有爭,有司勿問,止令三家所掌會問。諸哈的大師,止令掌教念經,回回人應有刑名、戶婚、錢糧、詞訟並從有司問之。諸僧人但犯奸盜詐偽,致傷人命及諸重罪,有司歸問。其自相爭告,從各寺院住持本管頭目歸問。若僧俗相爭田土,與有司約會;約會不至,有司就便歸問。諸各寺院稅糧,除前宋所有常住及世祖所賜田土免納稅糧外,已後諸人布施並己力典買者,依例納糧。諸管民官以公事攝所部,並用信牌,其差人擾眾者,禁之。



 諸掩骼埋胔,有司之職。或饑歲流莩,或中路暴死,無親屬收認,應聞有司檢覆者,檢覆既畢,就付地主鄰人收葬;不須檢覆者,亦就收葬。諸救災恤患,鄰邑之禮。歲饑輒閉糴者,罪之。諸郡縣災傷,過時而不申,或申不以實,及按治官不以時檢踏,皆罪之。諸蟲蝗為災,有司失捕,路官各罰俸一月,州官各笞一十七,縣官各二十七,並記過。諸水旱為災,人民艱食,有司不以時申報賑恤,以致轉徙饑莩者,正官笞三十七,佐官二十七,各解見任,降先職一等敘。諸有司檢覆災傷,或以熟作荒,或以可救為不可救,一頃已上者罰俸,二十頃者笞一十七,二百頃已上者笞二十七,五百頃已上笞三十七,惟以荒作熟,抑民納糧者,笞四十七,罷之。托故不行,妨誤檢覆者,笞三十七。



 諸義夫、節婦、孝子、順孫,其節行卓異,應旌表者,從所屬有司舉之,監察御史廉訪司察之,但有冒濫,罪及元舉。諸賜高年帛,應受賜而有司不以實報者,正官笞四十七,解職別敘。諸州縣舉茂異秀才,非經監察御史廉訪司體察者,不得開申。



 諸民犯弒逆,有司稱故不聽理者,杖六十七,解見任,殿三年,雜職敘。諸檢尸,有司故遷延及檢覆牒到不受,以致尸變者,正官笞三十七,首領官吏各四十七。其不親臨或使人代之,以致增減不實,移易輕重,及初覆檢官相符同者,正官隨事輕重論罪黜降,首領官吏各笞五十七罷之,仵作行人杖七十七,受財者以枉法論。諸有司,在監囚人因病而死,虛立檢尸文案及關覆檢官者,正官笞三十七,解職別敘。已代會赦者,仍記其過。諸職官覆檢尸傷,尸已焚瘞,止傅會初檢申報者,解職別敘。若已改除,仍記其過。



 諸籓王及軍馬經過,郡縣委積館勞,並許於應給官物內支遣,隨申行省知會,或擅移易齊斂者,禁之。諸郡縣非遇聖旨令旨,諸王駙馬大臣經過,官吏並免郊迎,妨奪公務,仍不得贐以錢物,按治官常糾察之。諸職官但犯軍情違誤,受敕官各路就斷,受宣官從都省行省處分。其餘公罪,各路並不得輒斷。



 諸部送囚徒,中路所次州縣,不寄囚於獄而監收旅舍,以致反禁而亡者,部送官笞二十七,還職本處,防護官笞四十七,就責捕賊,仍通記過名。諸有司各處遞至流囚,輒主意故縱者,杖六十七,解職,降先品一等敘,刑部記過。



 諸和顧和買,依時置估,對物給價。官吏權家,因緣結攬,營私害公者,罪之。諸有司和買諸物,多餘估計,分受其價者,準盜官錢論,不分受,以冒估多寡論。監臨及當該官吏詭名中納者,物價全沒之。克落價鈔者,準不枉法贓論。不即支價者,臺憲官糾之。諸職官輒以親故人事之物,為散之民,鳩斂錢財者,計其時直,以餘利為坐,減不枉法贓二等科罪,錢物各歸其主。諸職官私用民力者,笞二十七,記過,追顧直給其民。諸克除所屬官吏俸錢,為公用及備進上禮物,既去職者,並勿論。諸在任官斂屬吏俸贈去官者,笞四十七,還職。諸職官輒借騎所部內驛馬者,笞三十七,降先職一等敘,記過。諸職官於所部非親故及理應往復之家,輒行慶吊之禮者,禁之。違者罪之。



\end{pinyinscope}