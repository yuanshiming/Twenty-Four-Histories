\article{本紀第二十一 成宗四}

\begin{pinyinscope}

 七年春正月戊戌,太陰犯昴。甲辰,太陰犯軒轅。丙午,定諸改補鈔罪例,為首者杖一百有七,從者減二等;再犯,從者杖與首同,為首者流。己酉,以歲不登,禁河北、甘肅、陜西等郡釀酒。益都諸處牧馬之地為民所墾者,畝輸租一斗太重,減為四升。弛饑荒所在山澤河泊之禁一年,賑那海貧乏戶米八千石。壬子,罷歸德府括田。乙卯,詔凡為匿名書,辭語重者誅之,輕者流配,首告人賞鈔有差,皆籍沒其妻子充賞。命御史臺、宗正府委官遣發硃清、張瑄妻子來京師,仍封籍其家貲,拘收其軍器、海舶等。丁巳,令樞密院選軍士習農業者十人教軍前屯田。賜也梯忽而的合金五十兩、銀千兩、鈔千錠、幣帛百匹。



 二月壬戌,詔中書省汰諸有司冗員,仍令諭樞密院,除出征將帥外,掌署院事者,定其員數以聞。辛未,以平章政事、行上都留守木八剌沙、陜西行省平章阿老瓦丁並為中書平章政事,江南行臺御史中丞尚文為中書左丞,江浙行省參知政事董士珍為中書參知政事。壬申,詔:「樞密院、宗正府等,自今每事與中書共議,然後奏聞;諸司不得擅奏遷調,官員雖經特旨用之,而於例未允者,亦聽覆奏。」甲戌,減杭州稅課提舉司冗員。丙子,詔和林軍以六年更戍,仍給鈔以周其乏。命西京也速迭而軍及大都所起軍,皆以四月至上都,五月赴北。丁丑,命諸王出伯非急務者勿遣乘驛。詔中書省設官自左右丞相以下,平章二員,左右丞各一員,參知政事二員,定為八府。戊寅,太陰犯心。己卯,盡除內郡饑荒所在差稅,仍令河南省賑恤流民,給北師鈔三十八萬錠。以安南陳益稷久居鄂州,賜鈔千錠。以侍御史朵臺為中書參知政事。御史臺臣言:「江浙行省平章阿里,左丞高翥、安祐,僉省張祐等,詭名買鹽萬五千引,增價轉市於人,乞遣省、臺官按問。」從之。太原、大同、平灤路饑,並減直糶糧以賑之。庚辰,命陜西、甘肅行省賑鳳翔、秦、鞏、甘州、合迷裏貧乏戶。監察御史杜肯構等言太傅、右丞相完澤受硃清、張瑄賄賂事,不報。壬午,帝語中書省臣曰:「比有以歲課增羨希求爵賞者,此非掊刻於民,何從而出?自今除元額外,勿以增羨作正數。」罷江南財賦總管府及提舉司。禁內外中書省戶部轉運司官,不得私買鹽引。罷致用院。禁諸人毋以金銀絲線等物下番。罷江南都水庸田司、行通政院。並大都鹽運司入河間運司,其所掌京師酒稅課,令戶部領之。禁諸人非奉旨毋得以寶貨進獻。汰諸色人冒充宿衛及諸王、駙馬、妃主部屬濫請錢糧者。真定路饑,賑鈔五萬錠。仍諭諸王小薛及鷹師等,毋於真定近地縱獵擾民。丙戌,詔除征邊軍士及兩都站戶外,其餘人戶均當徭役。丁亥,詔自今除樞密院、御史臺、宣政院依舊奏選,諸司毋得擅奏,其舉用人員,並經中書省。



 三月己丑朔,保定路饑,賑鈔四萬錠。庚寅,詔遣奉使宣撫循行諸道:以郝天挺、塔出往江南、江北,石珪往燕南、山東,耶律希逸、劉賡往河東、陜西,鐵裏脫歡、戎益往兩浙、江東,趙仁榮、岳叔謨往江南、湖廣,木八剌、陳英往江西、福建,塔赤海牙、劉敏中往山北、遼東,並給二品銀印,仍降詔戒飭之。江浙行省平章脫脫遣發硃清、張瑄家屬,其家以金、珠重賂之,脫脫以聞。帝諭之曰:「朕以江南任卿,果能爾,真男子事也。其益恪勤乃事。」賜以黃金五十兩。都城火,命中書省與樞密院議增巡防兵。甘肅行省供軍錢糧多弊,詔徙廉訪司於甘州。壬辰,定大都南北兵馬司奸盜等罪,六十七以下付本路,七十七以上付也可札魯忽赤。河間路禾稼不登,命罷修建僧寺工役。乙未,真定路饑,賑鈔六百六十餘錠。中書平章伯顏、梁德珪、段真、阿里渾撒里,右丞八都馬辛,左丞月古不花,參政迷而火者、張斯立等,受硃清、張瑄賄賂,治罪有差,詔皆罷之。以洪君祥為中書右丞,監察御史言其曩居宥密,以貪賄罷黜,乞別選賢能代之,不報。甲辰,詔定贓罪為十二章。京朝官月俸外,增給祿米;外任官無公田者,亦量給之。乙巳,以徵八百媳婦喪師,誅劉深,笞合剌帶、鄭祐,罷雲南征緬分省。戊申,小蘭禧、岳鉉等進《大一統志》,賜賚有差。己酉,追收元降除免和顧和市璽書。以脫歡誣告諸王脫脫,謫置湖廣省軍前自效。罷甘肅行省差調民兵及取勘軍民站戶家屬孳畜之數。庚戌,以鐵哥察而所收愛牙合赤戶仍隸諸王脫脫。癸丑,樞密院臣及監察御史言:「中丞董士選貸硃清、張瑄鈔,非義。」帝曰:「臺臣稱貸不必問也,若言者不已,後當杖之。」甲寅,車駕幸上都。丙辰,賜諸王小薛所部等鈔六萬錠,賑李陵臺等五站戶鈔一千四百餘錠。遼陽等路饑,賑鈔萬錠。



 夏四月癸亥,太陰犯東井。詔省、臺、樞密院、通政院,凡呼召大都總管府官吏,必用印帖,其餘諸司不得輒召。征籓臣陳天祥、張孔孫、郭筠至京師,以天祥、孔孫為集賢大學士,筠為昭文館大學士,皆同議中書省事。丙寅,太陰犯軒轅。庚午,以中書文移太繁,其二品諸司當呈省者,命止關六部。中書左丞相答剌罕言:「僧人修佛事畢,必釋重囚。有殺人及妻妾殺夫者,皆指名釋之。生者茍免,死者負冤,於福何有?」帝嘉納之。辛未,流硃清、張瑄子孫於遠方,仍給行費。乙亥,歲星犯輿鬼,太陰犯南斗。庚辰,蛇節降,令海剌孫將兵五千守之,餘眾悉遣還各戍。撥碉門四川軍人一千人鎮羅羅斯,其土軍修治道路者,悉令放還。甲申,熒惑犯太微垣右執法。丁亥,歲星犯輿鬼。誅蛇節。衛輝路、辰州螟,濟南路隕霜殺麥。



 五月己丑,給和林軍鈔三十八萬錠。開上都、大都酒禁,其所隸兩都州縣及山後、河東、山西、河南嘗告饑者,仍悉禁之。詔雲南行省整飭錢糧。壬辰,辰星犯東井。以大德五年戰功,賞北師銀二十萬兩、鈔二十萬錠、幣帛各五萬九千匹。賜皇侄海山及安西王阿難答,諸王脫脫、八不沙,駙馬蠻子臺等各金五十兩、銀珠錦幣等物有差。丙申,遣征緬回軍萬四千人還各戍。癸卯,詔和林軍糧,除歲支十二萬石,其餘非奉旨不得擅支。丁未,床兀兒來朝,以戰功賜金五十兩、銀四百兩,仍給其萬戶所隸貧乏軍鈔六十九萬餘錠。辛亥,奉使宣撫耶律希逸、劉賡言:「平陽僧察力威犯法非一,有司憚其豪強,不敢詰問,聞臣等至,潛逃京師。」中書省臣言:「宜捕送其所,令省、臺、宣政院遣官雜治。」從之。甲寅,浚上都灤河。乙卯,以昌童王五戶絲分給諸王塔失鐵木而。令甘州站戶為僧人、禿魯花等隱藏者,依例還役。詔中外官吏無職田者,驗俸給米有差,其上都、甘肅、和林諸處非產米地,惟給其價。禁諸王八不沙部於般陽等處圍獵擾民。詔諸宿衛士,除官員子弟曾經奏準者留,餘悉革去。禁諸王、駙馬毋輒杖州縣官吏,違者罪王府官。立和林宣慰司都元帥府,以忽剌出遙授中書省左丞,為宣慰使都元帥。賜諸王納忽里鈔千錠、幣三十匹。濟寧、東昌、濟南、般陽、益都蟲食麥;太原、龍興、南康、袁、瑞、撫等路,高唐、南豐等州饑,減直糶糧五萬五千石;東平、益都、濟南等路蝗;般陽路隕霜。



 閏五月戊午朔,日有食之。以也奴鐵木而、闊闊出、晃兀沒於軍,賜其家鈔有差。壬戌,詔禁犯曲阜林廟者。丁卯,平江等十五路民饑,減直糶糧三十五萬四千石。戊辰,太陰犯心。己巳,以諸王孛羅、真童皆討賊有功,徵詣京師。完澤薨。庚辰,雲南行省平章也速帶而入朝,以所獲軍中金五百兩為獻。帝曰:「是金卿效死所獲者。」賜鈔千錠。丁丑,禁諸王、駙馬等征北諸軍以奴為代者罪之。辛巳,詔僧人與民均當差役。癸未,各道奉使宣撫言:』去歲被災人戶未經賑濟者,宜免其差役。」從之。命江浙行省右丞董士選發所籍硃清、張瑄貨財赴京師,其海外未還商舶,至則依例籍沒。甘肅行省平章合散等侵盜官錢十六萬三千餘錠、鹽引五千餘道,命省、臺官征之。詔上都路、應昌府、亦乞列思、和林等處依內郡禁酒。丙戌,罷營田提舉司。汴梁開封縣蟲食麥。



 六月己丑,御史臺臣言:「瓜、沙二州,自昔為邊鎮重地,今大軍屯駐甘州,使官民反居邊外,非宜。乞以蒙古軍萬人分鎮險隘,立屯田以供軍實為便。」從之。罷四川宣慰司,立四川行中書省,以雲南行省平章脫脫、湖廣行省議事平章程鵬飛並為平章政事。壬辰,武岡路饑,減價糶糧萬石以賑之。給欽察千戶等貧乏者鈔三萬七千八百餘錠。癸巳,叛賊雄挫來降。乙未,以亦乞不薛就平,留探馬赤軍二千人討阿永叛蠻,餘悉放還。庚子,西京道宣慰使法忽魯丁以瑟瑟二千五百餘斤鬻於官,為鈔一萬一千九百餘錠。有旨除御榻所用外,餘未用者,宜悉還之。命阿伯、阿忽臺等整飭河西軍事。癸卯,詔凡軍官子弟年及二十者,與民官子孫同,儤直一年方許襲職,萬戶於樞密院,千戶於行省,百戶於本萬戶。乙巳,罷行省僉省。浙西淫雨,民饑者十四萬,賑糧一月,仍免今年夏稅並各戶酒醋課。命甘肅行省修阿合潭、曲尤壕以通漕運。大寧路蝗。



 秋七月辛酉,常德路饑,減直糶糧萬石以賑之。壬戌,御史臺臣言:「前河間路達魯花赤忽賽因、轉運使術甲德壽皆坐贓罷。今忽賽因以獻鷹犬,復除大寧路達魯花赤;術甲德壽以迭裏迷失妄奏其被誣,復除福寧知州;並宜改正不敘,以戢奸貪。」從之。禁僧人以修建寺宇為名,齎諸王令旨乘傳擾民。汰宿衛士。丙寅,答剌罕哈剌哈孫為中書右丞相、知樞密院事。戊寅,歲星犯軒轅。丙子,給四川行省驛券十二道。詔除集賢、翰林老臣預議朝政,其餘三品以下,年七十者,各升散官一等致仕。立和林兵馬司,罷遼東宣慰司。丁丑,中書省臣言:「大同稅課,比奉旨賜乳母楊氏,其家掊斂過數,擾民為甚。」敕賜鈔五百錠,其稅課依例輸官。御史臺臣言:「湖南輸糧百石者,出驛馬一匹,廣海地狹,所輸不及百石者,所出亦如之,故官以鹽引助其不給。每馬一匹,貴州以北給鹽十七引,以南二十引。近立榷鹽提舉司,官價增五之三,元給二十引者,宜與鈔十七錠,十七引者十五錠。」從之。罷江南白云宗攝所,其田令依例輸租。都哇、察八而、滅里鐵木而等遣使請息兵,帝命安西王慎飭軍士,安置驛傳,以俟其來。戊寅,賜諸王奴倫、伯顏、也不乾等鈔九萬錠,罷諸王所設總管府。叛賊麻你降,貢金五百兩、童男女二百人及馬牛羊,卻之。己卯,太陰犯井。乙酉,熒惑犯房。賜諸王曲而魯等部鈔幣有差。



 八月己丑,罷護國仁王寺元設江南營田提舉司,給安西王所部貧民米二萬石。辛卯,夜地震,平陽、太原尤甚,村堡移徙,地裂成渠,人民壓死不可勝計,遣使分道賑濟,為鈔九萬六千五百餘錠,仍免太原、平陽今年差稅,山場河泊聽民採捕。癸巳,太白犯氐。月里不花將甕吉里軍赴雲南,道卒,以其子普而耶代之。甲午,熒惑犯東咸,太陰犯牽牛。庚子,中書省臣言:「法忽魯丁輸運和林軍糧,其負欠計二十五萬餘石,近監察御史亦言其侵匿官錢十三萬餘錠。臣等議:遣官征之,不足,則籍沒其財產。」從之。乙巳,歲星犯軒轅。庚戌,緬王遣使獻馴象四。辛亥,熒惑犯天江。賜諸王脫鐵木而之子也先博怯所部等鈔六千九百餘錠。九月戊午,車駕還大都。丙寅,太白晝見。以太原、平陽地震,禁諸王阿只吉、小薛所部擾民,仍減太原歲飼馬之半。遣刑部尚書塔察而、翰林直學士王約使高麗,以其國相吳祈專權,徵詣闕問罪。辛未,熒惑犯南斗。詔諭諸司賑恤平陽、太原。甲戌,太陰犯東井。乙亥,太白犯南斗。丙子,罷僧官有妻者。壬午,辰星犯氐。復木八剌沙平章政事。



 冬十月丁亥,太白經天。御史臺臣劾言江浙行省平章阿里不法,帝曰:「阿里朕所信任,臺臣屢以為言,非所以勸大臣也。後有言者,朕當不恕。」戊子,弛太原、平陽酒禁。以江浙年穀不登,減海運糧四十萬石。己丑,詔從軍醫工止復其妻子,戶如故。辛卯,復立陜西行御史臺。癸巳,御史臺臣及諸道奉使言:「行省官久任,與所隸編氓聯姻,害政。」詔互遷之。以只而合忽知樞密院事,給大都文宣王廟灑掃戶五。乙未,發雲南叛寇餘黨未革心者來京師,留蛇節養子阿闕於本境,以撫其民。改平灤為永平路,升甘州為上路。設刑部獄吏一員,以掌囚徒。安西轉運司於常課外增算五萬七千四百錠,人賜衣一襲,以勸其功。詔諸司凡錢糧不經中書省議者,勿奏。庚子,改普定府為路,隸曲靖宣慰司,以故知府容苴妻適姑為總管,佩虎符。以敘州宣慰司為敘南等處諸部蠻夷宣撫司。辛丑,太陰犯東井。庚戌,翰林國史院進太祖、太宗、定宗、睿宗、憲宗五朝《實錄》。辛亥,詔軍戶貧乏者,存恤六年。增蒙古國子生百員。



 十一月甲寅朔,賜諸王阿只吉所部鈔二十萬錠、糧萬石。命鷹師圍獵毋得擾民。以順元隸湖廣省,並海道運糧萬戶府為海道都漕運萬戶府,給印二。亦乞不薛賊黨魏傑等降,人賜衣一襲,遣還,俾招其首亂者。丁巳,詔大同、靜州、隆興等路運糧五萬石入和林。己未,太白經天。辛酉,木冰。甲子,命依十二章斷僧官罪。丙寅,鎮星犯進賢。戊辰,太陰犯井。辛未,升全寧府為路。己卯,太陰犯東咸。遣諸王滅怯禿、玉龍鐵木而使察八而。



 十二月甲申朔,詔內郡比歲不登,其民已免差者,並蠲免其田租。乙酉,弛京師酒課,許貧民釀酒。丙戌,太白經天。熒惑犯壘壁陣。戊子,以平宋隆濟功,增諸將秩,賜銀、鈔等物有差,其軍士各賜鈔十錠,放歸存恤一年。丙申,太陰犯東井。辛丑,太陰犯明堂。詔撫諭順元諸司,免大德七年民間逋稅。命江南、浙西官田奉特旨賜賚者,許中書省回奏。賜皇姑魯國大長公主鈔一萬五千錠、幣帛各三百匹。加封真武為元聖仁威玄天上帝。丁未,太陰犯天江。以轉輸軍餉勞,免思、播二州及潭、衡、辰、沅等路稅糧一年,常、澧三分之一,淘金、站戶無種佃者,免雜役一年。七道奉使宣撫所罷贓污官吏凡一萬八千四百七十三人,贓四萬五千八百六十五錠,審冤獄五千一百七十六事。是歲,斷大闢十人。



 八年春正月己未,以災異故,詔天下恤民隱,省刑罰。雜犯之罪,當杖者減輕,當笞者並免;私鹽徒役者減一年。平陽、太原免差稅三年;隆興、延安及上都、大同、懷孟、衛輝、彰德、真定、河南、安西等路被災人戶,免二年;大都、保定、河間路免一年。江南佃戶私租太重,以十分為率減二分,永為定例。仍弛山場河泊之禁,聽民採捕。庚申,以雲南順元同知宣撫事宋阿重生獲其叔隆濟來獻,特升其官,賜衣一襲。置掌薪司,以供尚食,令宣徽院掌其事。癸亥,禁錮硃清、張瑄族屬。乙丑,復置遂平、新蔡、真陽、太和、沈丘、潁上、柘城、城父、郟、舞陽十縣。丙寅,以御史中丞、太僕卿塔思不花為中書右丞,江南行臺中丞趙仁榮為中書參知政事,升教坊司三品。庚午,以輦真監藏為帝師。辛巳,詔:「諸王、妃主及諸路有馬者,十取其一,諸王、駙馬往遼東捕海東鶻者,毋給驛。」自榮澤至睢州,築河防十有八所,給其夫鈔人十貫。駙馬也列幹住所部民饑,以糧二千石賑之。是月,平陽地震不止,已修民屋復壞。



 二月丙戌,增置國子生二百員,選宿衛大臣子孫充之。降莊浪路為州,並隴干縣入德順州。辛卯,命諸王出伯所部軍屯田於薛出合出穀。甲午,詔父子兄弟有才者,許並居風憲。徙江東建康道廉訪司治於寧國,其建康路簿書,命監察御史鉤考。丙申,分軍千人戍嘉定州。甲辰,翰林學士承旨撒里蠻進金書《世祖實錄節文》一冊、漢字《實錄》八十冊。減宿衛繁冗者。丙午,車駕幸上都。敕軍人奸盜詐偽悉歸有司。賜太祖位怯憐口戶鈔萬八千二百錠、布帛萬匹。賜禿赤及塔剌海以所籍硃清、張瑄田,人六十頃,近侍鷹坊怯憐口鈔二萬七千三百錠、布帛萬二千匹。賜平章政事王慶端玉帶,半俸終身。



 三月丁巳,詔:「軍民官已除,以地遠官卑不赴者,奪其官不敘。軍官擅離所部者,悉遣還翼,違者論如律。軍人不告所部私歸者,杖而還之。」乙丑,去歲十二月庚戌,彗星見,約盈尺,在室十一度,入紫微垣,至是滅,凡七十四日。戊辰,中書左丞尚文以疾辭,不允。詔:「諸王、駙馬所分郡邑,達魯花赤惟用蒙古人,三年依例遷代,其漢人、女直、契丹名為蒙古者皆罷之。」敕軍民逃奴有獲者即付其主,主在他所者,赴所在官司給之,仍追逃奴鈔充獲者賞;逃及誘匿者,論罪有差。詔諸路牧羊及百至三十者,官取其一,不及數者勿取。中書省臣言:「自內降旨除官者,果為近侍宿衛,踐履年深,依已除敘。嘗宿衛未官者,視散官敘,始歷一考,準為初階。無資濫進,降官二級,官高者量降。各位下再任者,從所隸用,三任之上,聽入常調。蒙古人不在此限。」從之。雲南黎州盜劫也速帶而家屬貲產,命宣政院督其郡邑捕之。給諸王出伯所部馬萬三千五百匹。庚辰,詔內外使以軍務行者,至其地有司給饋十五日,自餘重事八日,細事三日。命凡為衙兵者,皆半隸屯田,仍諭各衛屯官及屯田者,視其勤惰,以為賞罰。升分寧縣為寧州,罷廬州路榷茶提舉司。灤城、濟陽等縣隕霜殺桑。



 夏四月丙戌,置千戶所,戍定海,以防歲至倭船。永寧路叛寇雄挫來降。命僧道為商者輸稅。凡諸王、駙馬徵索,有司非奉旨輒給者,罪且罷之。詔諸路畏吾兒、合迷裏自相訟者,歸都護府,與民交訟者,聽有司專決。甲午,詔:「朝廷、諸王、駙馬進捕鷹鷂皆有定戶,自今非鷹師而乘傳冒進者,罪之。」庚子,以永平、清、滄、柳林屯田被水,其逋租及民貸食者皆勿征。丁未,分教國子生於上都。賜西平王奧魯赤、合帶等部民鈔萬錠,朵耳思等站戶鈔二千二百錠、銀三百九十兩有奇。益都臨朐、德州齊河蝗。



 五月癸未朔,日有食之。辛酉,以所籍硃清、張瑄江南財產隸中政院。己巳,以平宋隆濟功,賜諸王脫脫、亦吉裡帶,平章床兀而等銀、鈔、金、幣、玉帶,及大理金齒、曲靖、烏撒、烏蒙宣慰等官銀、鈔各有差。壬申,罷福建都轉運鹽使司,以其歲課並隸宣慰司,中書省臣言:「吳江、松江實海口故道,潮水久淤,凡湮塞良田百有餘里,況海運亦由是而出,宜於租戶役萬五千人浚治,歲免租人十五石,仍設行都水監以董其程。」從之。追收諸王驛券。癸酉,定館陶等十七倉官品級:諸糧十萬石以上者從七品,五萬石以上者正八品。不及五萬者從八品,庚辰,以去歲平陽、太原地震,宮觀摧圮者千四百餘區,道士死傷者千餘人,命賑恤之。是月,蔚州之靈仙,太原之陽曲,隆興之天城、懷安,大同之白登,大風雨雹傷稼,人有死者。大名之浚、滑,德州之齊河霖雨,汴梁之祥符、太康,衛輝之獲嘉,太原之陽武河溢。六月癸未,開和林酒禁,立酒課提舉司。丁酉,汝寧妖人李曹驢等妄言得天書惑眾,事覺伏誅。益津蝗,汴梁祥符、開封、陳州霖雨,蠲其田租。扶風、岐山、寶雞諸縣旱,烏撒、烏蒙、益州、忙部、東川等路饑、疫,並賑恤之。



 秋七月辛酉,罷江淮等處財賦總管府。癸亥,諸王合贊自西域遣使來貢珍物。賜諸王也孫鐵木而等鈔二十萬錠,戍北千戶十五萬錠,怯憐口等九萬餘錠,西平王奧魯赤二萬錠。以順德、恩州去歲霖雨,免其民租四千餘石。



 八月,太原之交城、陽曲、管州、嵐州,大同之懷仁雨雹隕霜殺禾,杭州火,發粟賑之。以大名、高唐去歲霖雨,免其田租二萬四千餘石。九月癸丑,車駕至自上都。庚申,伯顏、梁德珪並復為中書平章政事,八都馬辛復為中書右丞,迷而火者復為中書參知政事,以江浙行省平章阿里為中書平章政事。庚午,以戶部尚書張祐為中書參知政事。癸酉,諸王察八而、朵瓦等遣使來附,以幣帛六百匹給之。詔諸王凡泉府規營錢,非奉旨毋輒支貸。給諸王出伯所部帛四百匹。四川、雲南鎮戍軍家居太原、平陽被災者,給鈔有差。潮州颶風起,海溢,漂民廬舍,溺死者眾,給其被災戶糧兩月。以冀、孟、輝、雲內諸州去歲霖雨,免其田租二萬二千一百石。



 冬十月辛卯,有事於太廟。辛巳,給諸王阿只吉所部馬料價鈔三千九百錠。以宣徽使、大都護長壽為中書右丞,陜西行省右丞脫歡為中書參知政事。丁亥,安南遣使入貢。詔諸王、駙馬毋乘驛以獵。庚寅,封皇侄海山為懷寧王,賜金印,仍割瑞州戶六萬五千隸之,歲給五戶絲直鈔二千六百錠、幣帛各千匹。戊戌,命省、臺、院官鞫高麗國相吳祈及千戶石天輔等,以祈離間王父子,天輔謀歸日本,皆笞之,徙安西。



 十一月壬子,詔:「內郡、江南人凡為盜黥三次者,謫戍遼陽;諸色人及高麗三次免黥,謫戍湖廣;盜禁騑馬者,初犯謫戍,再犯者死。」以平陽、太原去歲地大震,免其稅課一年。遣制用院使忽鄰、翰林直學士林元撫慰高麗。放遼陽民樂亦等三百九十戶為兵者還民籍。丁卯,復免僧人租。戊辰,以武備卿鐵古迭而為御史大夫。壬申,詔凡僧奸盜殺人者,聽有司專決。寧遠王闊闊出以馬萬五百餘匹給軍,命以鈔五萬二千五百餘錠償其直。增海漕米為百七十萬石。



 十二月庚子,復立益都淘金總管府。辛丑,封諸王出伯為威武西寧王,賜金印。賜安西王阿難答,諸王阿只吉、也速不乾等鈔一萬四千錠。



 九年春正月丁巳,太陰犯天關。戊午,帝師輦真監藏卒,賻金五百兩、銀千兩、幣帛萬匹、鈔三千錠,仍建塔寺。甲子,太陰犯明堂。以甕吉剌部民張道奴等舊權為軍者復隸民籍。己巳,太陰犯東咸。壬申,弛大都酒禁。甲戌,賜諸王完澤、撒都失里、別不花等所部鈔五萬六千九百錠、幣帛有差,鷹師等百五十萬錠。



 二月癸未,敕軍匠等戶元隸東宮者,有司毋得奪之。中書省臣言:「近侍自內傳旨,凡除授賞罰皆無文記,懼有差違,乞自今傳旨者,悉以文記付中書。」從之。甲午,免天下道士賦稅。乙未,建大天壽萬寧寺。丁酉,封諸王完澤為衛安王,定遠王岳木忽而為威定王,並賜金印。升翰林國史院為正二品,賜朵瓦使者幣帛五百匹。庚子,命中書議行郊祀禮。辛丑,詔赦天下。令御史臺、翰林、集賢院、六部,於五品以上,各舉廉能識治體者三人,行省、行臺、宣慰司、廉訪司各舉五人。免大都、上都、隆興差稅、內郡包銀俸鈔一年。江淮以南租稅及佃種官田者,均免十分之二。致仕官止有一子應承廕者,其儤使並免之,家貧者給半俸終其身。丙午,賜宿衛怯憐口鈔一百萬錠。以歸德頻歲被水民饑,給糧兩月。平陽、太原地震,站戶被災,給鈔一萬二千五百錠。



 三月丁未朔,車駕幸上都。給還安西王積年所減歲賜金五百兩、絲一萬一千九百斤,仍賜其所部鈔萬錠。敕遼陽行省毋專決大闢,以和林所貯幣帛給懷寧王所部軍。庚戌,以吃剌思八斡節兒侄相加班為帝師。詔梁王勿與雲南行省事,賜鈔千錠。甲寅,熒惑犯氐。戊午,歲星犯左執法。以樞密副使高興為平章政事,仍樞密副使。賜親王脫脫鈔二千錠,奴兀倫、孛羅等金五百兩、銀千兩、鈔二萬錠。以濟寧去歲霖雨傷稼,常寧州饑,並賑恤之。河間、益都、般陽屬縣隕霜殺桑,撫之。宜黃、興國之大冶等縣火,給被災者糧一月。



 夏四月庚辰,太陰犯井。雲南行省請益戍兵,不許,遣使詣諸路閱其當戍者遣之。乙酉,大同路地震,有聲如雷,壞官民廬舍五千餘間,壓死二千餘人。懷仁縣地裂二所,湧水盡黑,漂出松柏朽木,遣使以鈔四千錠、米二萬五千餘石賑之。是年租賦稅課徭役一切除免。戊子,賜察八而、朵瓦所遣使者銀千四百兩、鈔七千八百餘錠。己丑,東川路蠻官阿葵以馬二百五十匹、金二百五十兩及方物來獻。壬辰,太白犯井。中書省臣言:「前代郊祀,以祖宗配享。臣等議:今始行郊禮,專祀昊天為宜。」詔依所議行之。以汴梁、歸德、安豐去歲被災,潭州、郴州、桂陽、東平等路饑,並賑恤之。



 五月丁未,詔諸王、駙馬部屬及各投下,凡市傭徭役與民均輸。遣官調雲南、四川、福建、兩廣官。大都旱,遣使持香禱雨。戊申,徵陜西儒學提舉蕭赴闕,命有司給以安車。戊午,改各道肅政廉訪司為詳刑觀察司,聽省、臺闢人用之。立衍慶司,正二品。癸亥,歲星掩左執法。以地震,改平陽為晉寧,太原為冀寧。復立洪澤、芍陂屯田,令河南行省平章阿散領其事。省鬱林縣入貴州。以晉寧、冀寧累歲被災,給鈔三萬五千錠。寶慶路饑,發粟五千石賑之。以陜西渭南、櫟陽諸縣去歲旱,蠲其田租。道州旱。六月丙子朔,以立皇太子,遣中書右丞相答剌罕哈剌哈孫告昊天上帝,御史大夫鐵古迭而告太廟。庚辰,立皇子德壽為皇太子,詔告天下。賜高年帛,八十者一匹,九十者二匹。孝子順孫堪從政者,量才任之。親年七十別無侍丁者,從近遷除;外任官五品以下並減一資。諸處罪囚淹系五年以上,除惡逆外,疑不能決者釋之。流竄遠方之人,量移內地。甲午,潼川霖雨江溢,漂沒民居,溺死者眾,敕有司給糧一月,免其田租。以瓊州屢經叛寇,隆興、撫州、臨江等路水,汴梁霖雨為災,並給糧一月。桓州、宣德雨雹,鳳翔、扶風旱,通、泰、靜海、武清蝗。



 秋七月乙巳朔,禁晉寧、冀寧、大同釀酒,蠲晉寧、冀寧今年商稅之半。丙午,熒惑犯氐。辛亥,築郊壇於麗正、文明門之南丙位,設郊祀署,令、丞各一員,太祝三員,奉禮郎二員,協律郎一員,法物庫官二員。癸丑,以黑水新城為靖安路。升秘書監、拱衛司並正三品,罷福建蒙古字提舉司及醫學提舉司。賜安西王阿難答子月魯鐵木而鈔二千錠。甲寅,太白經天。庚申,升太府監為太府院。壬戌,以金千兩、銀七萬五千兩、鈔十三萬錠,賜興聖太后及宿衛臣,出居懷州。復置懷寧王王府官。賜威遠王岳木忽而鈔萬錠,給大都至上都十二驛鈔一萬一千二百錠。丁卯,熒惑犯房。以大司徒段貞、中書右丞八都馬辛並為中書平章政事,參知政事合剌蠻子為右丞,參知政事迷而火者為左丞,參議中書省事也先伯為參知政事。給脫脫所部乞而吉思民糧五月。沔陽之玉沙江溢,陳州之西華河溢,嶧州水,賑米四千石。揚州之泰興、江都,淮安之山陽水,蠲其田租九千餘石。潭、郴、衡、雷、峽、滕、沂、寧海諸郡饑,減直糶糧五萬一千六百石。



 八月乙亥朔,省孛可孫冗員。孛可孫專治芻粟,初惟數人,後以各位增入,遂至繁冗。至是存十二員,餘盡革之。丙子,給大都車站戶粟千四百七十餘石。丁丑,給曲阜林廟灑掃戶,以尚珍署田五十頃供歲祀。己卯,以冀寧歲復不登,弛山澤之禁,聽民採捕。命太常卿丑問、昭文館大學士靳德進祭星於司天臺。辛巳,太陰犯東咸。丙戌,商胡塔乞以寶貨來獻,以鈔六萬錠給其直。癸巳,復立制用院。乙未,熒惑犯天江。賜寧遠王闊闊出鈔萬錠,及其所部三萬錠。是月,涿州、東安州、河間、嘉興蝗,象州、融州、柳州旱,歸德、陳州河溢,大名大水,揚州饑。九月戊申,聖誕節,帝駐蹕於壽寧宮,受朝賀。丁巳,熒惑犯鬥。庚申,車駕至自上都。賜威武西寧王出伯所部鈔三萬錠。



 冬十月丁丑朔,升都水監正三品。辛巳,有事於太廟。丙戌,太白經天。己丑,命兩廣以南軍與土人同戍。庚寅,駙馬按替不花來自朵瓦,賜銀五十兩、鈔二百錠。乙未,帝諭中書省、樞密院、御史臺臣曰:「省中政事,聽右丞相哈剌哈孫答剌罕總裁,自今用人,非與答剌罕共議者,悉罷之。」戊戌,詔芍陂、洪澤等屯田為豪右占據者,悉令輸租。辛丑,復以詳刑觀察司為廉訪司。常州僧錄林起祐以官田二百八十頃冒為己業施河西寺,敕募民耕種,輸其租於官。御史臺臣請增官吏俸,命與中書省共議以聞。括兩淮地為豪民所占者,令輸租賦。賜安南王陳益稷湖廣地五百頃。諸王忽剌出及昔而吉思來賀立皇太子,賜鈔及衣服、弓矢等有差。



 十一月丁未,以鈔萬錠給雲南行省,命與貝參用,其貝非出本土者同偽鈔論。拘收諸王、妃主驛券。置大都南城警巡院。黃勝許遣其屬來獻方物,請復其子官,帝不允,曰:「勝許反側不足信,如其悔罪自至,則官可得。」命賜衣服遣之。以去年冀寧地震,站戶貧乏,詔諸王、駙馬毋妄遣使乘驛。復立雲南屯田,命伯顏察而董其事。給四川征戍軍士其家居大同為地震壓死者戶鈔五錠。庚戌,歲星、太白、鎮星聚於亢。癸丑,歲星犯亢。丙寅,歲星晝見。庚午,祀昊天上帝於南郊,牲用馬一、蒼犢一、羊豕鹿各九,其文舞曰《崇德之舞》,武舞曰《定功之舞》。以攝太尉、右丞相哈剌哈孫、左丞相阿忽臺、御史大夫鐵古迭而為三獻官。壬申,太白經天。



 十二月乙亥,賜冀寧路鈔萬錠、鹽引萬紙,以給歲費。丙子,太白犯西咸。地震。庚寅,熒惑犯壘壁陣。皇太子德壽薨。己亥,辰星犯建星。



 十年春正月壬寅朔,高麗王王昛遣使來獻方物。甲辰,詔詢訪莊聖皇后、昭睿順聖皇后、徽仁裕聖皇后儀範中外之政,以備紀錄。丙午,浚吳松江等處漕河。四川行省臣言:「所在驛傳,舊制以各路達魯花赤兼督,今沿江水驛迂遠,宜令所隸州縣官統治之。」從之。增置甘肅行省王渾木敦等處驛傳,立福建鹽課提舉司,隸宣慰司。庚戌,浚真、揚等州漕河,令鹽商每引輸鈔二貫,以為傭工之費。丁巳,太白犯建星。戊午,罷江南白云宗都僧錄司,汰其民歸州縣,僧歸各寺,田悉輸租。壬戌,發河南民十萬築河防。丙寅,以沙都而所部貧乏,給糧兩月。丁卯,命諸王、駙馬、妃主奏請錢穀者,與中書議行之。升巡檢為九品,命近侍無輒驛召外郡官,弛大同路酒禁,封駙馬合伯為昭武郡王,營國子學於文宣王廟西偏。詔各道禁沮擾鹽法,以京畿雷家站戶貧乏,給鈔五百錠。奉聖州懷來縣民饑,給鈔九百錠。



 閏正月癸酉,太白犯牽牛。甲戌,賑合民所部留處鳳翔者糧三月。壬午,給諸王也先鐵木而所部米二千石,賑暗伯拔突軍屯東地者糧兩月。丁亥,免大都今年租賦。己丑,太白犯壘壁陣。甲午,以前中書平章政事鐵哥、江浙行省平章闍里、河南行省平章阿散,並為中書平章政事;行宣政院使張閭、四川行省左丞杜思敬,並為中書左丞;參議中書省事劉源為參知政事。是月,以曹之禹城去歲霖雨害稼,民饑,發陵州糧二千餘石賑之。晉寧、冀寧地震不止。



 二月壬寅,賑金蘭站戶不能自贍者糧兩月,賑遼陽千戶小薛干所部貧匱者糧三月。辛亥,中書省臣言:「近侍傳旨以文記至省者,凡一百五十餘人,令臣擢用。其中犯法妄進者實多,宜加遴選。」制曰:「可。」升行都水監為正三品,諸路提控案牘為九品。駙馬濟寧王蠻子帶以所部用度不足,乞預貸歲得五戶絲,從之。遣六衛漢軍貧乏者還家休息一年。丙辰,封孛羅為鎮寧王,錫以金印。朵瓦遣使來朝,賜衣幣遣之。戊午,太陰犯氐。己未,江西福建道奉使宣撫塔不帶坐贓遇赦,釋其罪,終身不敘。丁卯,以月古不花為中書左丞。戊辰,車駕幸上都。賜安西王阿難答,西平王奧魯赤、不里亦鈔三萬錠,南哥班萬錠,從者三萬二千錠。鎮西武靖王搠思班所部民饑,發甘肅糧賑之。是月,大同路暴風大雪,壞民廬舍,明日雨沙陰霾,馬牛多斃,人亦有死者。



 三月戊寅,歲星犯亢。己卯,崆古王遣使來貢方物。乙未,慮大都囚,釋上都死囚三人。賜駙馬蠻子帶鈔萬錠。道州營道等處暴雨,江溢山裂,漂蕩民廬,溺死者眾,復其田租。以濟州任城縣民饑,賑米萬石。給千家木思答伯部糧三月。柳州民饑,給糧一月。河間民王天下奴弒父,磔裂於市。



 夏四月庚子朔,詔凡匿鷹犬者,沒家貲之半,笞三十,來獻者給之以賞。甲辰,樞密院臣言:「太和嶺屯田,舊置屯儲總管府,專督其程。人給地五十畝,歲輸糧三十石,或佗役不及耕作者,悉如數征之,人致重困。乞令軍官統治,以宣慰使玉龍失不花總其事,視軍民所收多寡以為賞罰。」從之。丁未,命威武西寧王出伯領甘肅等地軍站事。辛酉,填星犯亢。壬戌,雲南羅雄州軍火主阿邦龍少結豆溫匡虜、普定路諸蠻為寇,右丞汪惟能進討,賊退據越州,諭之不服,遣平章也速帶而率兵萬人往捕之。兵至曲靖,與惟能合,從諸王昔寶赤、亦吉裡帶等進壓賊境,獲阿邦龍少,斬之,餘眾皆潰。命也速帶而留軍二千戍之,其從軍有功者皆加賞賚。癸亥,置昆山、嘉定等處水軍上萬戶府。甲子,倭商有慶等抵慶元貿易,以金鎧甲為獻,命江浙行省平章阿老瓦丁等備之。賜梁王松山鈔千錠。是月,以廣東諸郡、吉州、龍興、道州、柳州、漢陽、淮安民饑,贛縣暴雨水溢,賑糧有差。鄭州暴風雨雹,大若雞卵,麥及桑棗皆損,蠲今年田租。真定、河間、保定、河南蝗。



 五月辛未,大都旱,遣使持香禱雨。壬午,增河間、山東、兩浙、兩淮、福建、廣海鹽運司歲煮鹽二十五萬餘引。癸未,詔西番僧往還者不許馳驛,給以舟車。禁御史臺、宣慰司、廉訪司官毋買鹽引。乙酉,以同知樞密院事塔魯忽臺、塔剌海並知樞密院事。遣高麗國王王昛還國,仍署行省以鎮撫之,其國僉議、密直司等官並授以宣敕。封駙馬脫鐵木而為濮陽王,賜以金印,公主忙哥臺為鄄國大長公主。丁亥,詔命右丞相哈剌哈孫答剌罕、左丞相阿忽臺等整飭庶務,凡銓選錢穀等事,一聽中書裁決,百司勤怠者悉以名聞。賜威武西寧王出伯鈔三萬錠。遼陽、益都民饑,賑貸有差。大都、真定、河間蝗,平江、嘉興諸郡水傷稼。六月癸卯,御史臺臣言:「江南行臺監察御史教化劾江浙行省宣使李元不法,行省亦遣人摭拾,教化不令檢核案牘。中書省臣復言教化等不循法度,擅遣軍士守衛其門,搒掠李元,誣指行省等官,實溫省事。」詔省、臺及也可札魯忽赤同訊之。癸丑,太陰犯羅堰上星。己未,歲星犯亢。壬戌,來安路總管岑雄叛,湖廣行省遣宣慰副使忽都魯鐵木而招諭之,雄令其子世堅來降,賜衣物遣之。復淮西道廉訪司。大名、益都、易州大水,景州霖雨,龍興、南康諸郡蝗。



 秋七月庚辰,太陰犯牽牛。辛巳,釋諸路罪囚,常赦所不原者不與。宣德等處雨雹害稼,大同之渾源隕霜殺禾,平江大風,海溢漂民廬舍。道州、武昌、永州、興國、黃州、沅州饑,減直賑糶米七萬七千八百石。



 八月壬寅,歲星犯氐,熒惑犯太微垣上將。開成路地震,王宮及官民廬舍皆壞,壓死故秦王妃也裡完等五千餘人,以鈔萬三千六百餘錠、糧四萬四千一百餘石賑之。辛亥,賜王侄阿木哥鈔三千錠。丁巳,京師文宣王廟成,行釋奠禮,牲用太牢,樂用登歌,制法服三襲。命翰林院定樂名、樂章。成都等縣饑,減直賑糶米七千餘石。九月己巳,熒惑犯太微垣右執法。壬申,以聖誕節,朵瓦遣款徹等來賀。壬午,熒惑犯太微垣左執法。



 冬十月甲辰,太白犯鬥。丁未,有事於太廟。辛亥,太陰犯畢。甲寅,太陰犯井。丁卯,安南國遣黎亢宗來貢方物。青山叛蠻紅犵獠等來附,仍貢方物,賜金幣各一。吳江州大水,民乏食,發米萬石賑之。



 十一月己巳,車駕還大都。辛未,歲星犯房。壬申,太陰犯虛。甲戌,熒惑犯亢。丁亥,武昌路火,給被災者糧一月。戊子,熒惑犯氐。辛卯,太陰犯熒惑。丙申,安西王阿難答、西平王奧魯赤所部皆乏食,給米有差。益都、揚州、辰州歲饑,減直賑糶米二萬一千餘石。



 十二月壬寅,太白晝見。乙巳,歲星犯東咸。壬子,速哥察而等十三站乏食,給糧三月。乙卯,帝有疾,禁天下屠宰四十二日。丙辰,遣宣政院使沙的等禱於太廟。諸王合而班答部民潰散,詔諭所在敢匿者罪之。戊午,太陰犯氐。癸亥,瓊州臨高縣那蓬洞主王文何等作亂伏誅。磁州民田雲童弒母,磔裂於市。是歲,斷大闢四十四人。



 十一年春正月丙寅朔,帝大漸,免朝賀。癸酉,崩於玉德殿,在位十有三年,壽四十有二。乙亥,靈駕發引,葬起輦穀,從諸帝陵。是年九月乙丑,謚曰欽明廣孝皇帝,廟號成宗,國語曰完澤篤皇帝。



 成宗承天下混壹之後,垂拱而治,可謂善於守成者矣。惟其末年,連歲寢疾,凡國家政事,內則決於宮壼,外則委於宰臣;然其不致於廢墜者,則以去世祖為未遠,成憲具在故也。



\end{pinyinscope}