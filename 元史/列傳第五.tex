\article{列傳第五}

\begin{pinyinscope}

 ○特薛禪



 特薛禪,姓孛思忽兒,弘吉剌氏,世居朔漠。本名特,因從太祖起兵有功,賜名薛禪,故兼稱曰特薛禪。女曰孛兒臺,太祖光獻翼聖皇后。子曰按陳,從太祖征伐,凡三十二戰,平西夏,斷潼關道,取回紇尋斯干城,皆與有功。歲丁亥,賜號國舅按陳那顏。壬辰,賜銀印,封河西王,以統其國族。丁酉,賜錢二十萬緡,有旨:「弘吉剌氏生女世以為後,生男世尚公主,每歲四時孟月,聽讀所賜旨,世世不絕。」又賜所俘獲軍民五千二百,仍授萬戶以領之。按陳薨,葬官人山。元貞元年二月,追封濟寧王,謚忠武;妻哈真,追封濟寧王妃。



 子斡陳,歲戊戌授萬戶,尚睿宗女也速不花公主。斡陳薨,葬不海韓。



 弟納陳,歲丁巳襲萬戶,奉旨伐宋,攻釣魚山。又從世祖南涉淮甸,下大清口,獲船百餘艘。又率兵平山東濟、兗、單等州。及阿里不哥叛,中統二年與諸王北伐,以其子哈海、脫歡、斡羅陳等十人自從,至於莽來,由失木魯與阿里不哥之黨八兒哈八兒思等戰,追北至孛羅克禿,復戰,自旦及夕,斬首萬級,殭尸被野。薨,葬末懷禿。斡羅陳襲萬戶,尚完澤公主。完澤公主薨,繼尚囊加真公主。至元十四年薨,葬拓剌里。無子。



 弟曰帖木兒,至元十八年襲萬戶。二十四年,乃顏叛,從帝親征,以功封濟寧郡王,賜白傘蓋以寵之。二十五年,諸王哈丹禿魯幹叛,與諸王及統兵官玉速帖木兒等率兵討之,由龜剌兒河與哈丹等遇,轉戰至惱木連河,殲其眾。帝賜名按答兒禿那顏,以旌其功。薨,葬末懷禿。



 子二人:長曰雕阿不剌,次曰桑哥不剌,皆幼。至元二十七年,以其弟蠻子臺襲萬戶,亦尚囊加真公主。成宗即位,封皇姑魯國大長公主,以金印封蠻子臺為濟寧王。奉旨率本部兵討叛王海都、篤哇,既與之遇,方約戰,行伍未定,單騎突入陣中,往復數四,敵兵大擾,一戰遂大捷。時武宗在籓邸,統大軍以鎮朔方,有旨令蠻子臺總領蒙古軍民官,輔武宗守莽來,以遏北方。囊加真公主薨,繼尚裕宗女喃哥不剌公主。蠻子臺薨,年五十有二。



 大德十一年三月,按答兒禿長子雕阿不剌襲萬戶,尚祥哥剌吉公主,六月,封大長公主,賜雕阿不剌金印,加封魯王。至大二年,賜平江稻田一千五百頃。皇慶間,加封皇姊大長公主。天歷間,加號皇姑徽文懿福貞壽大長公主。至大三年,雕阿不剌薨,葬末懷禿。



 阿里嘉室利,雕阿不剌嫡子也。至大三年,甫八歲,襲萬戶。四年七月,襲封魯王,尚朵兒只班公主。元統元年,阿里嘉失利薨。至順間,封朵兒只班號肅雍賢寧公主。



 桑哥不剌者,魯王雕阿不剌之弟、阿里嘉室利之叔也。自幼奉世皇旨,養於斡可珍公主所,是為不只兒駙馬,後襲統其本部民四百戶。成宗時,奉旨尚普納公主;至順間,封鄆安大長公主,授桑哥不剌金印,封鄆安王,職千戶。元統元年,授萬戶。二年三月,加封鄆安公主號皇姑大長公主;加封桑哥不剌魯王。以疾薨,年六十一。此皆以駙馬襲封王爵者也。



 唆兒火都者,亦按陳之子,以從征功,在太祖朝遙授左丞相,為千戶,仍賜以塗金銀章,及金銀字海青圓符五、驛馬券六。其子曰阿哈駙馬,當憲宗朝嘗率兵破徐州,以功受賞黃金一鋌、白金十鋌及銀鞍勒,仍命襲父官。至世祖時,有昭「弘吉剌萬戶所受驛券、圓符皆仍其舊,凡唆兒火都所受者,宜皆收之」,而唆兒火都之諸孫若孛羅沙、伯顏、蠻子、添壽不花、大都不花、掌吉等,及阿哈千戶之孫曰也速達兒與按陳之弟名冊者,在太祖世授官本籓蒙古軍站千戶。冊之子曰哈兒哈孫,以平金功,賜號拔都兒。哈兒哈孫之孫曰都羅兒,至元四年,授光祿大夫,以銀章封懿國公。



 有脫憐者,亦按陳之裔孫也,世祖授本籓千戶,仍賜驛券、圓符各四,令以兵守朔土之怯魯連。二十四年,從族父按答兒禿征叛王乃顏有功,亦賜號拔都兒。脫憐卒,子迸不剌嗣。迸不剌卒,子買住罕嗣。買住罕尚拜答沙公主。卒,弟孛羅帖木兒嗣,以金章封毓德王。孛羅帖木兒薨,買住罕孫阿失襲千戶。



 有名醜漢者,按陳次子必哥之裔孫,尚臺忽魯都公主。仁宗朝,封安遠王,以兵守莽來。



 有答兒罕者,亦特薛禪之裔孫也,以從軍功,世祖亦賜以拔都兒之號,加賜黃金一鋌。其子曰不只兒,從征乃顏,禽其黨金家奴,帝賞以金帶。其後有曰伯奢者,即其孫也。



 又按陳之孫納合,尚太宗唆兒哈罕公主。火忽之孫不只兒,尚斡可真公主。又特薛禪諸孫有名脫羅禾者,尚不魯罕公主,繼尚闊闊倫公主。此皆尚公主為駙馬者也。



 凡其女之為後者,自光獻翼聖皇后以降,憲宗貞節皇后諱忽都臺,及後妹也速兒,皆按陳從孫忙哥陳之女。世祖昭睿順聖皇后諱察必,濟寧忠武王按陳之女;其諱帖古倫者,按陳孫脫憐之女;諱喃必冊繼守正宮者,納陳孫仙童之女。成宗貞慈靜懿皇后諱實憐答里,斡羅陳之女也。順宗昭獻元聖皇后諱答吉,大德十一年十一月,武宗冊上皇太后,至大三年十月,加上尊號曰儀天興聖慈仁昭懿壽元皇太后,仁宗延祐二年,加上尊號曰儀天興聖慈仁昭懿壽元全德泰寧福慶皇太后,延祐七年,又加徽文崇祐四字,尊號太皇太后,則按陳孫渾都帖木兒之女。武宗宣慈惠聖皇后諱真哥,脫憐子迸不剌之女;其諱速哥失里者,按陳從孫哈兒只之女。泰定皇后諱八不罕,按陳孫斡留察兒之女;其諱必罕、諱速哥答里者,皆脫憐孫買住罕之女。文宗皇后諱不答失里,雕阿不剌魯王之女。此則弘吉剌氏之為後者也。



 初,弘吉剌氏族居於苦烈兒、溫都兒斤、迭烈木兒、也裏古納河之地。歲甲戌,太祖在迭蔑可兒時,有旨分賜按陳及其弟火忽、冊等農土,農土猶言經界也。若曰「是苦烈兒、溫都兒斤,以與按陳及哈撒兒為農土」。申諭按陳曰:「可木兒溫都兒、答兒腦兒、迭蔑可兒等地,汝則居之。」諭冊曰:「阿剌忽馬乞迤東,蒜吉納禿山、木兒速拓、哈海斡連直至阿只兒哈溫都、哈老哥魯等地,汝則居之。當以胡盧忽兒河北為鄰,按赤臺為界。」又諭火忽曰:「哈老溫迤東,塗河、潢河之間,火兒赤納慶州之地,與亦乞列思為鄰,汝則居之。」又諭按陳之子唆魯火都曰:「以汝父子能輸忠於國,可木兒溫都兒迤東,絡馬河至於赤山,塗河迤南,與國民為鄰,汝則居之。」至至元七年,斡羅陳萬戶及其妃囊加真公主請於朝曰:「本籓所受農土,在上都東北三百里答兒海子,實本籓駐夏之地,可建城邑以居。」帝從之。遂名其城為應昌府。二十二年,改為應昌路。元貞元年,濟寧王蠻子臺亦尚囊加真公主,復與公主請於帝,以應昌路東七百里駐冬之地創建城邑,復從之。大德元年,名其城為全寧路。



 弘吉剌之分邑,得任其陪臣為達魯花赤者,有濟寧路及濟、兗、單三州,巨野、鄆城、金鄉、虞城、碭山、豐縣、肥城、任城、魚臺、沛縣、單父、嘉祥、磁陽、寧陽、曲阜、泗水一十六縣。此丙申歲之所賜也。至元六年,升古濟州為濟寧府,十八年始升為路,而濟、兗、單三州隸焉。又汀州路長汀、寧化、清流、武平、上杭、連城六縣,此至元十三年之所賜也。又有永平路灤州、盧龍、遷安、撫寧、昌黎、石城、樂亭六縣,此至大元年之所賜也。若平江稻田一千五百頃,則至大二年所賜也。其應昌、全寧等路則自達魯花赤總管以下諸官屬,皆得專任其陪臣,而王人不與焉。



 此外,復有王傅府,自王傅六人而下,其群屬有錢糧、人匠、鷹房、軍民、軍站、營田、稻田、煙粉千戶、總管、提舉等官,以署計者四十餘,以員計者七百餘,此可得而稽考者也。其五戶絲、金鈔之數,則丙申歲所賜濟寧路之三萬戶,至元十八年所賜汀州路之四萬戶,絲以斤計者,歲二千二百有奇;鈔以錠計者,歲一千六百有奇。此則所謂歲賜者也。



 ○孛禿



 孛禿,亦乞列思氏,善騎射。太祖嘗潛遣術兒徹丹出使,至也兒古納河。孛禿知其為帝所遣,值日暮,因留止宿,殺羊以享之。術兒徹丹馬疲乏,復假以良馬。及還,孛禿待之有加。術兒徹丹具以白帝,帝大喜,許妻以皇妹帖木倫。孛禿宗族乃遣也不堅歹等詣太祖,因致言曰:「臣聞威德所加,若雲開見日、春風解凍,喜不自勝。」帝問:「孛禿孳畜幾何?」也不堅歹對曰:「有馬三十匹,請以馬之半為聘禮。」帝怒曰:「婚姻而論財,殆若商賈矣。昔人有言,同心實難,朕方欲取天下,汝亦乞列思之民,從孛禿效忠於我可也,何以財為!」竟以皇妹妻之。既而札赤剌歹札木哈、脫也等以兵三萬入寇。孛禿聞之,遣波欒歹、磨里禿禿來告,乃與哈剌里、札剌兀、塔兒哈泥等討脫也等,掠其輜重,降其民。乃蠻叛,帝召孛禿以兵至,大戰敗之。皇妹薨,復妻以皇女火臣別吉,而命哈兒八臺之子也可忽林圖帶弓矢為之侍。哈兒八臺曰:「吾兒豈能為人臣僕,寧死不為也!」帝令孛禿與之敵,哈兒八臺令月列等拒戰於碗圖河。孛禿直前擒月列,剌殺也可忽林圖,哈兒八臺走渡拙赤河,又擒之,盡殺其眾。從太師國王木華黎略地遼東、西,以功封冠懿二州。從征西夏,病薨。贈推忠宣力佐命功臣、太師、開府儀同三司、駙馬都尉、上柱國,進封昌王,謚忠武。子鎖兒哈襲爵。



 鎖兒哈,事太宗。與木華黎取嘉州,降其民,遣伯禿兒哈拙赤碣來獻捷,帝曰:「若父宣力國家,朕昔見之。今鎖兒哈克光前烈。」賜以金錦、金帶、七寶鞍,召至中都,以疾薨。鎖兒哈娶皇子斡赤女安禿公主,生女,是為憲宗皇后。



 子札忽兒臣,從定宗出討萬奴有功,太宗命親王安赤臺以女也孫真公主妻之。薨,贈推誠靖宣佐運贊治功臣、太師、開府儀同三司、駙馬都尉、上柱國,襲封昌王,謚忠靖。



 札忽兒臣有子二人:長月列臺,娶皇子賽因主卜女哈答罕公主,生脫別臺,與乃顏戰,有功。次忽憐。



 忽憐,尚憲宗女伯牙魯罕公主。後脫黑帖木兒叛,世祖命忽憐與失列及等討之,大戰終日,脫黑帖木兒敗走,帝嘉之,復令尚憲宗孫女不蘭奚公主。宋平,封以廣州。乃顏、聲剌哈兒叛,世祖新徵,薛徹堅等與哈答罕屢戰,帝召忽憐至。值薛徹堅等戰於程火失溫之地,哈答罕眾甚盛,忽憐以兵二百迎敵,敗之。哈答罕等走度猱河,還其巢穴。逾年夏,帝命忽憐復徵之。至曲列兒、塔兀兒二河之間,大戰,其眾皆度塔兀河遁去。餘百人逃匿山谷,忽憐即率兵二百徒步追之。薛徹堅止之曰:「彼亡命者,安得徒行。」忽憐不聽,往殺其眾。薛徹堅以聞,賜金一鋌、銀五鋌。又逾年,復往征之,與哈答罕遇於兀剌河。忽憐夜率千人潛入其軍,盡殺之。帝賜鈔五萬貫、金一鋌、銀十鋌。忽憐薨,贈效忠保德輔運佐理功臣、太師、開府儀同三司、駙馬都尉、上柱國,追封昌王,謚忠宣。



 子阿失,事成宗。篤哇叛於海都,帝遣晉王甘麻剌並武宗帥師討之。大德五年,戰哈剌答山,阿失射篤哇,中其膝,擒殺甚多,篤哇號哭而遁,武宗賜之衣。成宗加賜珠衣,封為昌王,置王府官屬。仁宗朝,復賜以寧昌縣稅入。阿失尚成宗女亦里哈牙公主,復尚憲宗曾孫女買的公主。阿失薨,子八剌失里襲封昌王。忽憐從弟不花,尚世祖女兀魯真公主;其弟鎖郎哈,娶皇子忙哥剌女奴兀倫公主,生女,是為武宗仁獻章聖皇后,實生明宗。



 ○阿剌兀思剔吉忽裏



 阿剌兀思剔吉忽裏,汪古部人,系出沙陀雁門之後。遠祖卜國,世為部長。金源氏塹山為界,以限南北,阿剌兀思剔吉忽里以一軍守其沖要。時西北有國曰乃蠻,其主太陽可汗遣使來約,欲相親附,以同據朔方。部眾有欲從之者,阿剌兀思剔吉忽里弗從,乃執其使,奉酒六尊,具以其謀來告太祖。時朔方未有酒,太祖飲三爵而止,曰:「是物少則發性,多則亂性。」使還,酬以馬五百、羊一千,遂約同攻太陽可汗。阿剌兀思剔吉忽裏先期而至。既平乃蠻,從下中原,復為向導,南出界垣。太祖留阿剌兀思剔吉忽裏歸鎮本部,為其部眾昔之異議者所殺,長子不顏昔班並死之。



 其妻阿里黑攜幼子孛要合與侄鎮國逃難,夜遁至界垣,告守者,縋城以登,因避地雲中。太祖既定雲中,購求得之,賜與甚厚,乃追封阿剌兀思剔吉忽里為高唐王,阿里黑為高唐王妃,以其子孛要合尚幼,先封其侄鎮國為北平王。鎮國薨,子聶古臺襲爵,尚睿宗女獨木干公主,略地江淮,薨於軍,賜興州民千餘戶,給其葬。



 孛要合幼從攻西域,還封北平王,尚阿剌海別吉公主。公主明睿有智略,車駕征伐四出,嘗使留守,軍國大政,諮稟而後行,師出無內顧之憂,公主之力居多。孛要合未有子,公主為進姬妾,以廣嗣續,生三子:曰君不花,曰愛不花,曰拙里不花。公主視之,皆如己出。孛要合薨,追封高唐王,謚武毅。後加贈宣忠協力翊衛果毅功臣、太傅、儀同三司、上柱國、駙馬都尉,追封趙王。公主阿剌海別吉追封皇祖姑齊國大長公主,加封趙國。



 子君不花,尚定宗長女葉裏迷失公主。愛不花,尚世祖季女月烈公主。中統初,總兵討阿里不哥,敗闊不花於按檀火爾歡之地。三年,圍李璮於濟南,獨當一面。事平,又從征西北,敗叛王之黨撒里蠻於孔古烈。愛不花卒。子闊里吉思。



 闊里吉思,性勇毅,習武事,尤篤於儒術,築萬卷堂於私第,日與諸儒討論經史,性理、陰陽、術數,靡不該貫。尚忽答的迷失公主,繼尚愛牙失里公主。宗王也不幹叛,率精騎千餘,晝夜兼行,旬日追及之。時方暑,將戰,北風大起,左右請待之,闊里吉思曰:「當暑得風,天贊我也。」策馬赴戰,騎士隨之,大殺其眾,也不乾以數騎遁去。闊里吉思身中三矢,斷其發。凱還,詔賜黃金三斤、白金千五百斤。成宗即位,封高唐王。西北不安,請於帝,願往平之,再三請,帝乃許。及行,且誓曰:「若不平定西北,吾馬首不南。」大德元年夏,遇敵於伯牙思之地,眾謂當俟大軍畢至,與戰未晚,闊里吉思曰:「大丈夫報國,而待人耶!」即整眾鼓躁以進,大敗之,擒其將卒百數以獻。詔賜世祖所服貂裘、寶鞍,及繒錦七百、介胄、戈戟、弓矢等物。二年秋,諸王將帥共議防邊,咸曰:「敵往歲不冬出,且可休兵於境。」闊里吉思曰:「不然,今秋候騎來者甚少,所謂鷙鳥將擊,必匿其形,備不可緩也。」眾不以為然,闊里吉思獨嚴兵以待之。是冬,敵兵果大至,三戰三克,闊里吉思乘勝逐北,深入險地,後騎不繼,馬躓陷敵,遂為所執。敵誘使降,惟正言不屈,又欲以女妻之,闊里吉思毅然曰:「我帝婿也,非帝後面命,而再娶可乎!」敵不敢逼。帝嘗遣其家臣阿昔思特使敵境,見於人眾中,闊里吉思一見輒問兩宮安否,次問嗣子何如,言未畢,左右即引其去。明日,遣使者還,不復再見,竟不屈死焉。九年,追封高唐忠獻王,加贈推忠宣力崇文守正亮節保德功臣、太師、開府儀同三司、上柱國、駙馬都尉,追封趙王。公主忽答的迷失追封齊國長公主,愛牙失里封齊國公主,並加封趙國。



 子術安幼,詔以弟術忽難襲高唐王。術忽難才識英偉,謹守成業,撫民御眾,境內乂安。痛其兄死節,遣使如京師,表請恤典,又請翰林承旨閻復銘諸石。教養術安過於己子,命家臣之謹厚者掌其兄之珍服秘玩,待術安成立,悉以付之。至大二年,術忽難加封趙王,即以讓術安。三年,術安襲趙王,尚晉王女阿剌的納八剌公主。一日,召王傅脫歡、司馬阿昔思謂曰:「先王旅殯卜羅,荒遠之地,神靈將何依,吾痛心欲無生,若請於上,得歸葬先塋,瞑目無憾矣。」二人言之知樞密院事也里吉尼以聞,帝嗟悼久之,曰:「術安孝子也。」即賜阿昔思黃金一瓶,得脫歡之子失忽都魯、王傅術忽難之子阿魯忽都、斷事官也先等一十九人,乘驛以往,復賜從者鈔五百貫。淇陽王月赤察兒、丞相脫禾出八都魯差兵五百人,護其行至殯所,奠告啟視,尸體如生,遂得歸葬。



\end{pinyinscope}