\article{本紀第十六 世祖十三}

\begin{pinyinscope}

 二十七年春正月戊申,改大都路總管府為都總管府。庚戌,太白犯牛。改儲人待提舉司為軍儲所,秩從三品。以河東山西道宣慰使阿里火者為尚書右丞,宣慰使如故。癸丑,太陰犯井。敕從臣子弟入國子學。安南國王陳日烜遣其中大夫陳克用來貢方物。乙卯,造祀天幄殿。高麗國王王睶遣使來貢方物。丁巳學之一。,遣使代祀岳瀆、海神、后土。戊午,遼陽自乃顏之叛,民甚疲敝,發鈔五千八十錠賑之。己未,賜鎮遠王牙忽都、靖遠王合帶塗金銀印各一。章吉寇甘木里,諸王術伯、拜答寒、亦憐真擊走之。庚申,賑馬站戶饑。給滕竭兒回回屯田三千戶牛、種。辛酉,營懿州倉。壬戌,造長甲給北征軍。乙丑,伸思、八兒術答兒、移剌四十、石抹蠻忒四人,以謀不軌伏誅。丙寅,合丹餘寇未平,命高麗國發耽羅戍兵千人討之。賜河西質子軍五百人馬。丁卯,熒惑犯房。高麗國王王睶言:「臣昔宿衛京師,遭林衍之叛,國內大亂,高麗民居大同者皆籍之,臣願復以還高麗為民。」從之。己巳,改西南番總管府為永寧路。辛未,賜也速帶兒所部萬人鈔萬錠。豐閏署田戶饑,給六十日糧。無為路大水,免今年田租。癸酉,忻都所部別笳兒田戶饑,給九十日糧。降臨淮府為盱眙縣,隸泗州。復立興文署,掌經籍板及江南學田錢穀。合丹寇遼東海陽。



 二月乙亥朔,立全羅州道萬戶府。江西諸郡盜未平,詔江淮行省分兵一千益之。命太僕寺毋隸宣徽院。丙子,新附屯田戶饑,給六十日糧。順州僧、道士四百九十一人饑,給九十日糧。戊寅,太陰犯畢。開元路寧遠等縣饑,民、站戶逃徙,發鈔二千錠賑之。播州安撫使楊漢英進雨氈千,駙馬鐵別赤進羅羅斯雨氈六十、刀五十、弓二十。己卯,興州興安饑,給九十日糧。庚辰,伯答罕民戶饑,給六十日糧。辛巳,括河間昔寶赤戶口。癸未,泉州地震。乙酉,賑新附民居昌平者。丙戌,改奉先縣為房山縣。泉州地震。己丑,江西群盜鐘明亮等復降,詔徙為首者至京師,而給其餘黨糧。浙東諸郡饑,給糧九十日。庚寅,太陰犯亢。辛卯,復立南康、興國榷茶提舉司,秩從五品。發虎賁更休士二千人赴上都修城。河間路任丘饑,給九十日糧。癸巳,晉陵、無錫二縣霖雨害稼,並免其田租。江西賊華大老、黃大老等掠樂昌諸郡,行樞密院討平之。闍兀所部闌遺戶饑,給六十日糧。常寧州民遭群盜之亂,免其田租。己亥,保定路定興饑,發粟五千二百六十四石賑之。辛丑,唆歡禾稼不登,給九十日糧。



 三月乙巳,中山畋戶饑,給六十日糧。戊申,廣濟署饑,給粟二千二百五十石以為種。壬子,熒惑犯鉤鈐。薊州漁陽等處稻戶饑,給三十日糧。戊午,出忙安倉米,賑燕八撒兒所屬四百二十人。己未,改雲南蒙憐甸為蒙憐路軍民總管府,蒙萊甸為蒙萊路。放罷福建獵戶、沙魚皮戶為民,以其事付有司總之。發雲州民夫鑿銀洞。永昌站戶饑,賣子及奴產者甚眾,命甘肅省贖還,給米賑之。並福、泉二州人匠提舉司為一,仍放無役者為民。庚申,升御史臺侍御史正四品,治書侍御史正五品,增蒙古經歷一員,從五品。罷行司農司及各道勸農營田司;增提刑按察司僉事二員,總勸農事。四川行省舊移重慶,成都之民苦於供給,詔復徙治成都。立江南營田提舉司,秩從五品,掌僧寺貲產。放壽、潁屯田軍千九百五十九戶為民,撤江南戍兵代之。凡工匠隸呂合剌、阿尼哥、段貞無役者,皆區別為民。詔風憲之選仍歸御史臺,如舊制。置金竹府大隘等四十二寨蠻夷長官。癸亥,建昌賊丘元等稱大老,集眾千餘人,掠南豐諸郡,建昌副萬戶擒斬之。甲子,楊震龍餘眾剽浙東,總兵官討賊者,多俘掠良民,敕行御史臺分揀之,凡為民者千六百九十五人。庚午,以建昌路廣昌縣經鐘明亮之亂,免其田租九千四百四十七石。辛未,太平縣賊葉大五集眾百餘人寇寧國,皆擒斬之。



 夏四月癸酉朔,大駕幸上都。婺州螟害稼,雷雨大作,螟盡死。丙子,太陰犯井。辛巳,命大都路以粟六萬二千五百六十四石賑通州、河西務等處流民。芍陂屯田以霖雨河溢,害稼二萬二千四百八十畝有奇,免其租。癸未,罷海道運糧萬戶府。江淮行省言:「近朝廷遣白絜矩來,與沙不丁議,令發兼並戶偕宋宗族赴京,人心必致動搖,江南之民方患增課、料民、括馬之苦,宜俟它日行之。」從之。阿速敦等二百九十五人乏食,命驗其實給糧賑之。改利津海道運糧萬戶府為臨清禦河運糧上萬戶府。諸王小薛部曲萬二千六十一戶饑,給六十日糧。發六衛漢軍萬人伐木為修城具。甲申,以薦饑免今歲銀俸鈔,其在上都、大都、保定、河間、平灤者萬一百八十錠,在遼陽省者千三百四十八錠有奇。丙戌,遣桑吉剌失等詣馬八兒國訪求方伎士。壬辰,熒惑守氐十餘日。癸巳,河北十七郡蝗。千戶也先、小闊闊所部民及喜魯、不別等民戶並饑,敕河東諸郡量賑之。千戶也不干所部乏食,敕發粟賑之。太傅玉呂魯言:「招集斡者所屬亦乞烈,今已得六百二十一人,令與高麗民屯田,宜給其食。」敕遼陽行省驗實給之。平山、真定、棗強二縣旱,靈壽、元氏二縣大雨雹,並免其租。丁酉,以鈔二千五百錠賑昌平至上都站戶貧乏者。定興站戶饑,給三十日糧。己亥,命考大都路貧病之民在籍者二千八百三十七人,發粟二百石賑之。庚子,合丹復寇海陽。復立安和署,從六品。



 五月乙巳,罷秦王典藏司,收其印。括江南闌遺人雜畜、錢帛。合丹寇開元。戊申,江西行省管如德、江西行院月的迷失合兵討反寇鐘明亮,明亮降,詔縛致闕下,如德等留不遣,明亮復率眾寇贛州。樞密院以如德等違詔縱賊,請詰之,從之。詔罷江西行樞密院。庚戌,陜西南市屯田隕霜殺稼,免其租。壬子,賜諸王鐵木兒等軍一萬七百人糧,一人一從者五石,二人一從者七石五斗。丙辰,發粟賑御河船戶。敘州等處諸部蠻夷進雨氈八百。戊午,移江西行省於吉州,以便捕盜。尚書省遣人行視雲南銀洞,獲銀四千四十八兩。奏立銀場官,秩從七品。出魯等千一百一十五戶饑,給六十日糧。癸亥,敕:「諸王分地之民有訟,王傅與所置監郡同治,無監郡者,王傅聽之。」平灤民萬五千四百六十五戶饑,賑粟五千石。徽州績溪賊胡發、饒必成伏誅。乙丑,太陰犯填星。丙寅,罷奉宸庫。遷江西行尚書省參政楊文璨為左丞,文璨逾歲不之官,詔以外剌帶代之。外剌帶至,文璨復署事,桑哥乃奏文璨升右丞。江西行省言:「吉、贛、湖南、廣東、福建以禁弓矢,賊益發,乞依內郡例,許尉兵持弓矢。」從之。己巳,立雲南行御史臺。命徹里鐵木兒所部女直、高麗、契丹、漢軍輸地稅外,並免他徭。江陰大水,免田租萬七百九十石。庚午,復置諸王也只裏王傅,秩正四品。尚珍署廣備等屯大水,免其租。伯要民乏食,命撒的迷失以車五百輛運米千石賑之。婺州永康、東陽,處州縉云賊呂重二、楊元六等反,浙東宣慰使史弼擒斬之。泉州南安賊陳七師反,討平之。括天下陰陽戶口,仍立各路教官,有精於藝者,歲貢各一人。



 六月壬申朔,升閏鹽州為柏興府,降普樂州為閏鹽縣,金州為金縣。河溢太康,沒民田三十一萬九千八百餘畝,免其租八千九百二十八石。納鄰等站戶饑,給九十日糧。甲戌,桑州總管黃布蓬、那州長羅光寨、安郡州長閉光過率蠻民萬餘戶內附。丙子,放保定工匠楚通等三百四十一戶為民。庚辰,從江淮行省請,升廣濟庫為提舉司,秩從五品。用江淮省平章沙不丁言,以參政王巨濟鉤考錢穀有能,賞鈔五百錠。繕寫金字《藏經》,凡糜金三千二百四十四兩。廣州增城、韶州樂昌以遭畬賊之亂,並免其田租。杭州賊唐珍等伏誅。己丑,熒惑犯房。辛卯,敕應昌府以米千二百石給諸王亦只里部曲。壬辰,別給江西行省印,以便分省討賊。泉州大水。丙申,發侍衛兵萬人完都城。丁酉,大司徒撒里蠻、翰林學士承旨兀魯帶進《定宗實錄》。己亥,棣州厭次、濟陽大風雹,害稼,免其租。庚子,從江西省請,發各省戍兵討賊。辛丑,免河間、保定、平灤歲賦絲之半。懷孟路武陟縣、汴梁路祥符縣皆大水,蠲田租八千八百二十八石。



 秋七月,終南等屯霖雨害稼萬九千六百餘畝,免其租。丙午,禁平地、忙安倉釀酒,犯者死。戊申,江西霖雨,贛、吉、袁、瑞、建昌、撫水皆溢,龍興城幾沒。癸丑,罷緬中行尚書省。江淮省平章沙不丁,以倉庫官盜欺錢糧,請依宋法黥而斷其腕,帝曰:「此回回法也。」不允。免大都路歲賦絲。戊午,貴州貓蠻三十餘人作亂,劫順元路,入其城,遂攻阿牙寨,殺傷官吏,其眾遂盛。湖廣省檄八番蔡州、均州二萬戶府及八番羅甸宣慰司合兵討之。鳳翔屯田霖雨害稼,免其租。建平賊王靜照伏誅。辛酉,熒惑犯天江。壬申,駐蹕老鼠山西。乙丑,蕪湖賊徐汝安、孫惟俊等伏誅。丙寅,雲南闍力白衣甸酋長凡十一甸內附。丁卯,用桑哥言,詔遣慶元路總管毛文豹搜括宋時民間金銀諸物,已而罷之。滄州樂陵旱,免田租三萬三百五十六石。江夏水溢,害稼六千四百七十餘畝,免其租。魏縣御河溢,害稼五千八百餘畝,免其租百七十五石。



 八月辛未朔,日有食之。並廣東道真陽、洽光二縣為英德州。沁水溢,害冀氏民田,免其租。禁諸人毋沮平陽、太原、大同宣課。丁丑,廣州清遠大水,免其租。庚辰,免大都、平灤、河間、保定四路流民租賦及酒醋課。丁亥,復徙四川南道宣慰司於重慶府。以南安、贛、建昌、豐州嘗罹鐘明亮之亂,悉免其田租。癸巳,地大震,武平尤甚,壓死按察司官及總管府官王連等及民七千二百二十人,壞倉庫局四百八十間,民居不可勝計。己亥,帝聞武平地震,慮乃顏黨入寇,遣平章政事鐵木兒、樞密院官塔魯忽帶引兵五百人往視。



 九月壬寅,河東山西道饑,敕宣慰使阿里火者炒米賑之。癸卯,歲星犯鬼。申嚴漢人田獵之禁。乙巳,禁諸王遣僧建寺擾民。敕河東山西道宣慰使阿里火者發大同鈔本二十萬錠,糴米賑饑民。平章政事闍里鐵木兒帥師與合丹戰於瓦法,大破之。丁未,御河決高唐,沒民田,命有司塞之。戊申,武平地震,盜賊乘隙剽劫,民愈憂恐。平章政事鐵木兒以便宜蠲租賦,罷商稅,弛酒禁,斬為盜者;發鈔八百四十錠,轉海運米萬石以賑之。金竹府知府掃閭貢馬及雨氈,且言:「金竹府雖內附,蠻民多未服。近與趙堅招降竹古弄、古魯花等三十餘寨,乞立縣,設長官、總把,參用土人。」從之。己酉,福建省以管內盜賊蜂起,請益戍兵,命江淮省調下萬戶一軍赴之。發蒙古都萬戶府探馬赤軍五百人戍鄂州。辛亥,修東海廣德王廟。丙辰,赦天下。丁卯,命江淮行省鉤考行教坊司所總江南樂工租賦,置四巡檢司於宿遷之北。以所罷陸運夫為兵,護送會通河上供之物,禁發民挽舟。



 冬十月壬申,封皇孫甘麻剌為梁王,賜金印,出鎮雲南。癸酉,享於太廟。甲戌,立會通汶泗河道提舉司,從四品。丁丑,尚書省臣言:「江陰、寧國等路大水,民流移者四十五萬八千四百七十八戶。」帝曰:「此亦何待上聞,當速賑之!」凡出粟五十八萬二千八百八十九石。己卯,增上都留守司副留守、判官各一員。從甘肅行省請,簽管內民千三百人為兵,以戍其境。辛巳,太白犯鬥。只深所部八魯剌思等饑,命寧夏路給米三千石賑之。禁大同路釀酒。乙酉,門答占自行御史臺入覲。梁洞梁宮朝、吳曲洞吳湯暖等凡二十洞,以二千餘戶內附。丁亥,賜北邊幣帛十萬匹。己丑,新作太廟登歌、宮懸樂。以昔寶赤歲取鸕鶿成都擾民,罷之。



 十一月辛丑,廣濟署洪濟屯大水,免租萬三千一百四十一石。興、松二州隕霜殺禾,免其租。隆興苦鹽濼等驛饑,發鈔七千錠賑之。丁未,大同路蒙古多冒名支糧,置千戶、百戶十員,以達魯花赤總之,食糧戶以富為貧者,籍家貲之半。戊申,太陰掩鎮星。桑哥言:「向奉詔,內外官受命不赴及受代官居五年不赴銓者,罷不復敘。臣謂茍無大故,不可終棄。」帝復允其請。江淮行省平章不憐吉帶言:「福建盜賊已平,惟浙東一道,地極邊惡,賊所巢穴。復還三萬戶,以合剌帶一軍戍沿海明、臺,亦怯烈一軍戍溫、處,札忽帶一軍戍紹興、婺。共寧國、徽,初用土兵,後皆與賊通,今以高郵、泰兩萬戶漢軍易地而戍。楊州、建康、鎮江三城,跨據大江,人民繁會,置七萬戶府。杭州行省諸司府庫所在,置四萬戶府。水戰之法,舊止十所,今擇瀕海沿江要害二十二所,分兵閱習,伺察諸盜。錢塘控扼海口,舊置戰船二十艘,故海賊時出,奪船殺人,今增置戰船百艘、海船二十艘,故盜賊不敢發。」從之。庚戌,罷雲南會川路採碧甸子。甲寅,禁上都釀酒。乙卯,貴赤三百三十戶乏食,發粟賑之。己未,禁山後釀酒。庚申,賜伯顏所將兵,幣帛各萬三千四百匹、綿三千四百斤。辛酉,太陰掩左執法。隆興路隕霜殺稼,免其田租五千七百二十三石。壬戌,大司徒撒里蠻、翰林學士承旨兀魯帶進《太宗實錄》。癸亥,河決祥符義唐灣,太康、通許,陳、潁二州大被其患。甲子,御史臺言:「江南盜起,討賊官利其剽掠,復以生口充贈遺,請給還其家。」帝嘉納之。徙河北河南道提刑按察司治許州。罷大都東西二驛脫脫禾孫,以通政院總之。乙丑,易水溢,雄、莫、任丘、新安田廬漂沒無遺,命有司築堤障之。丙寅,括遼陽馬六千匹,擇肥者給闍里鐵木兒所部軍。丁卯,立新城榷場、平地脫脫禾孫,遣使鉤考延安屯田。降南雄州為保昌縣,韶州為曲江縣。



 十二月辛未,以衛尉院為太僕寺。戊寅,免大都、平灤、保定、河間自至元二十四年至二十六年逋租十三萬五百六十二石。己卯,命樞密院括江南民間兵器及將士習武,如戊子歲詔。甲申,遣兵部侍郎靳榮等閱實安西、鳳翔、延安三道軍戶,元籍四千外,復得三萬三千二百八十丁,樞密院欲以為兵,桑哥不可,帝從之。丙戌,興化路仙游賊硃三十五集眾寇青山,萬戶李綱討平之。京兆省上屯田所出羊價鈔六百九錠,敕以賜札散、暗伯民貧乏者。辛卯,太陰犯亢。乙未,初分萬億為四庫,以金銀輸內府,至是立提舉富寧庫,秩從五品,以掌之。大同路民多流移,免其田租二萬一千五百八石。洪贊、灤陽驛饑,給六十日糧。不耳答失所部滅乞裡饑,給九十日糧。詔諸王乃蠻帶、遼陽行省平章政事薛闍干、右丞洪察忽,摘蒙古軍萬人分戍雙城及婆娑府諸城,以防合丹兵。己亥,省溧陽路為縣,入建康。湖廣省上二年宣課珠九萬五百一十五兩。處州青田賊劉甲乙等集眾萬餘人寇溫州平陽。是歲,賜諸王、公主、駙馬金、銀、鈔、幣如歲例。命帝師西僧遞作佛事坐靜於萬壽山厚載門、茶罕腦兒、聖壽萬安寺、桓州南屏庵、雙泉等所,凡七十二會。斷死罪七十二人。



 二十八年春正月壬寅,太白、熒惑、鎮星聚奎。癸卯,給諸王愛牙赤印。命玄教宗師張留孫置醮祠星三日。上都民仰食於官者眾,詔傭民運米十萬石致上都,官價石四十兩,命留守木八剌沙總其事。辛亥,罷汴梁至正陽、杞縣、睢州、中牟、鄭、唐、鄧十二站站戶為民。癸丑,高麗國遣使來貢方物。丁巳,遣貴由赤四百人北征。辛酉,罷江淮漕運司,並於海船萬戶府,由海道漕運。並浙西金玉人匠提舉司入浙西道金玉人匠總管府。降無為、和州二路、六安軍為州,巢州為縣,入無為,並隸廬州路。升安豐府為路,降壽春府、懷遠軍為縣,懷遠入濠州,並隸安豐路。升各處行省理問所為四品。免江淮貧民至元十二年至二十五年所逋田租二百九十七萬六千餘石,及二十六年未輸田租十三萬石、鈔千一百五十錠、絲五千四百斤、綿千四百三十餘斤。罷淘金提舉司,立江東兩浙都轉運使司。壬戌,以札散、禿禿合總兵於甕古之地,命有司供其軍需,敕大同路發米賑甕古饑民。尚書省臣桑哥等以罪罷。



 二月辛未,賜也速帶兒所部兵騬馬萬匹。徙萬億庫金銀入禁中富寧庫。尚書省言:「大同仰食於官者七萬人,歲用米八十萬石,遣使覆驗,不當給者萬三千五百人,乞征還官。」從之。癸酉,以隴西四川總攝輦真術納思為諸路釋教都總統。改福建行省為宣慰司,隸江西行省。詔:「行御史臺勿聽行省節度。」雲南行省言:「敘州、烏蒙水路險惡,舟多破溺,宜自葉稍水站出陸,經中慶,又經鹽井、土老、必撒諸蠻,至敘州慶符,可治為驛路,凡立五站。」從之。也速帶兒、汪總帥言:「近制,和顧和買不及軍家,今一切與民同。」詔自今軍勿輸。丙子,罷征理司。上都、太原饑,免至元十二年至二十六年民間所逋田租三萬八千五百餘石,遣使同按察司賑大同、太原饑民,口給糧兩月或三月。以桑哥黨與,罷楊州路達魯花赤唆羅兀思。遣官覆驗水達達、咸平貧民,賑之。丁丑,以太子右詹事完澤為尚書右丞相,翰林學士承旨不忽木平章政事,詔告天下。以列兀難粳米賑給貧民。己卯,遣官持香詣中嶽、南海、淮瀆致禱。立金齒等處宣慰司都元帥府。以上都虎賁士二千人屯田,官給牛具農器,用鈔二萬錠。以雲南曲靖路宣撫司所轄地廣,民心未安,改立曲靖等處宣慰司、管軍萬戶府以鎮之,辛巳,以湖廣行省八番羅甸司復隸四川省。壬午,以桑哥沮抑臺綱,又箠監察御史,命御史大夫月兒魯辨之。癸未,太陰犯左執法。大駕幸上都,是日次大口,復召御史臺及中書、尚書兩省官辨論桑哥之罪。復以闌遺監隸宣徽院。詔毋沮擾山東轉運使司課程。甲申,太白犯昴。命江淮行省鉤考沙不丁所總詹事院江南錢穀。乙酉,立江淮、湖廣、江西、四川等處行樞密院,詔諭中外,江淮治廣德軍,湖廣治岳州,江西治汀州,四川治嘉定。丙戌,詔:「改提刑按察司為肅政廉訪司,每道仍設官八員,除二使留司以總制一道,餘六人分臨所部,如民事、錢穀、官吏奸弊,一切委之,俟歲終,省、臺遣官考其功效。」以集賢大學士何榮祖為尚書右丞,集賢學士賀勝為尚書省參知政事。詔江淮行省遣蒙古軍五百、漢兵千人,從皇子鎮南王鎮楊州。執河間都轉運使張庸,仍遣官鉤考其事。丁亥,營建宮城南面周廬,以居宿衛之士。執湖廣要束木詣京師,戊子,籍要束木家貲,金凡四千兩。辛卯,封諸王鐵木兒不花為肅遠王,賜之印。壬辰,雨壞太廟第一室,奉遷神主別殿。癸巳,籍桑哥家貲。遣行省、行臺官發粟,賑徽之績溪,杭之臨安、餘杭、於潛、昌化、新城等縣饑民。命江淮行省參政燕公楠整治鹽法之弊。丁酉,詔加岳、瀆、四海封號,各遣官詣祠致告。



 三月己亥朔,真定、河間、保定、平灤饑,平陽、太原尤甚,民流移就食者六萬七千戶,饑而死者三百七十一人。桑哥妻弟八吉由為燕南宣慰使,以受賂積贓伏誅。僕桑哥輔政碑。太原饑,嚴酒禁。丁未,太陰犯御女。己酉,大陰犯右執法。庚戌,太陰犯太微東垣上相。甲寅,常德路水,免田租二萬三千九百石。乙卯,太白犯五車。乃顏所屬牙兒馬兀等同女直兵五百人追殺內附民餘千人,遣塔海將千人平之。辛酉,呂連站木赤五十戶饑,賑三月糧。發侍衛兵營紫檀殿。壬戌,以甘肅行省右丞崔彧為中書右丞。南丹州莫國麟入覲,授國麟安撫使、三珠虎符。杭州、平江等五路饑,發粟賑之,仍弛湖泊蒲、魚之禁。溧陽、太平、徽州、廣德、鎮江五路亦饑,賑之如杭州。武平路饑,百姓困於盜賊軍旅,免其去年田租。凡州郡田嘗被災者,悉免其租,不被災者免十之五。罷甘州轉運司。江淮豪家多行賄權貴,為府縣卒史,容庇門戶,遇有差賦,惟及貧民,詔江淮行省嚴禁之。賑遼陽、武平饑民,仍弛捕獵之禁。



 夏四月己巳,禁屠宰牝羊。甲戌,詔各路府、州、司、縣長次官兼管諸軍奧魯。以地震故,免侍衛兵籍武平者今歲徭役。增置欽察衛經歷一員,用漢人為之,餘不得為例。庚辰,弛杭州西湖禽魚禁,聽民網罟。丙戌,詔凡負斡脫銀者,入還皆以鈔為則。乙未,歲星犯輿鬼。以沙不丁等米賑江南饑民。召硃清、張瑄詣闕。庚寅,並總制院入宣政院。以鈔法故,召葉李還京師。乙未,徙湖廣行樞密院治鄂州。丙申,以米三千石賑闊里吉思饑民。



 五月戊戌,召江西行樞密院副使阿里詣闕,升章佩監秩三品。遣脫脫、塔剌海、忽辛三人追究僧官江淮總攝楊璉真伽等盜用官物。以參知政事廉希恕為湖廣等處行省右丞,行海北海南道宣慰使都元帥,瓊州安撫使陳仲達海北海南道宣慰使都元帥,湖廣行省左右司郎中不顏於思、別十八里副元帥王信並同知海北海南道宣慰司事副元帥,並佩虎符,將二千二百人以征黎蠻,僚屬皆從仲達闢置。立左右兩江宣慰司都元帥府。壬寅,太陰犯少民。徙江淮行樞密院治建康。甲辰,中書省臣麥術丁、崔彧言:「桑哥當國四年,諸臣多以賄進,親舊皆授要官,唯以欺蔽九重、朘削百姓為事,宜令兩省嚴加考核,並除名為民。」從之。要束木以桑哥妻黨為湖廣行省平章,至是坐不法者數十事,詔械致湖廣省誅之。辛亥,以太原及杭州饑,免今歲田租。增河東道宣慰使一員。徵太子贊善劉因;因前為太子贊善,以繼母病去,至是母亡,以集賢學士徵之,不起。罷脫脫、塔剌海、忽辛等理算僧官錢穀。罷江南六提舉司歲輸木綿。鞏昌舊惟總帥府,桑哥特升為宣慰司,以其弟答麻剌答思為使,桑哥敗,懼誅自殺,至是復總帥府。增置異珍、御帶二庫,秩從五品,並設提點、使、副各一員。減中外冗官三十七員。宮城中建蒲萄酒室及女工室。詔以桑哥罪惡系獄按問,誅其黨要束木、八吉等。發兵塞晃火兒月連地河渠,修城堡,令蒙古戍兵屯田川中以禦寇。癸丑,罷尚書省事皆入中書,改尚書右丞相、右詹事完澤為中書右丞相,平章政事麥術丁、不忽木並中書平章政事,尚書右丞何榮祖中書右丞,尚書左丞馬紹中書左丞,參知政事賀勝、高翥並參知中書政事;征東行尚書省左丞相、駙馬高麗國王王睶為征東行中書省左丞相。罷大都燒鈔庫,仍舊制,各路昏鈔令行省官監燒。增置戶部司計、工部司程,正七品。甲寅,太陰犯牛。賑上都、桓州、榆林、昌平、武平、寬河、宣德、西站、女直等站饑民。乙卯,以政事悉委中書,仍遣使布告中外。詔禁失陷錢糧者托故詣京師。丁巳,建白塔二,各高一丈一尺,以居咒師朵四的性吉等七人。何榮祖以公規、治民、御盜、理財等十事緝為一書,名曰《至元新格》,命刻版頒行,使百司遵守。桑哥嘗以劉秉忠無子,收其田土,其妻竇氏言秉忠嘗鞠猶子蘭章為嗣,敕以地百頃還之。己未,以門答占復為御史大夫,行御史臺事。高麗國王王睶乞以其子謜為世子,詔立謜為高麗王世子,授特進上柱國,賜銀印。



 六月丁卯朔,禁蒙古人往回回地為商賈者。湖廣饑,敕以剌里海牙米七萬石賑之。辛巳,洞蠻鎮遠立黃平府。乙酉,以雲南諸路行省參知政事兀難為梁王傅。洗國王洞主、市備什王弟同來朝。益江淮行院兵二萬擊郴州、桂陽、寶慶、武岡四路盜賊。以汴梁逃人男女配偶成家,給農具耕種。丙戌,敕:「屯田官以三歲為滿,互於各屯內調用。」宣諭江淮民恃總統璉真加力不輸租者,依例徵輸。辛卯,太陰犯畢。癸巳,以漣、海二州隸山東宣慰司。



 秋七月丙申朔,雲南省參政怯剌言:「建都地多產金,可置冶,令旁近民煉之以輸官。」從之。己亥,太白犯井。詔諭尚州等處諸洞蠻夷。庚子,徙江西行樞密院治贛州。乙巳,大都饑,出米二十五萬四千八百石賑之。戊申,揚州路學正李淦上言:「人皆知桑哥用群小之罪,而不知尚書右丞葉李妄舉桑哥之罪,宜斬葉李以謝天下。」有旨驛召淦詣京師,淦至而李卒,除淦江陰路教授,以旌直言。給還行臺監察御史周祚妻子。祚嘗劾行尚書省官,桑哥誣以他罪,流祚於憨答孫,妻子家貲入官,及是還之。禁屠宰馬牛。敕:「江南重囚,依舊制聞奏處決。」罷江南諸省買銀提舉司。遣官招集宋時涅手軍可充兵者八萬三千六百人,以蒙古、漢人、宋人參為萬戶、千戶、百戶領之。遼陽諸路連歲荒,加以軍旅,民苦饑,發米二萬石賑之。己酉,召交趾王弟陳益稷、右丞陳巖、鄭鼎子那懷並詣京師。癸丑,賜師壁洞安撫司、師壁鎮撫所、師羅千戶所印,安撫司從三品,餘皆五品。丁巳,桑哥伏誅。募民耕江南曠土,戶不過五頃,官授之券,俾為永業,三年後征租。遣憨散總兵討平江南盜賊。己未,降江陰路為州,宜興府為縣,並隸常州路。移揚子縣治新城,分華亭之上海為縣,松江府隸行省。罷淘金提舉司、江淮人匠提舉司凡五,以其事並隸有司。雨壞都城,發兵二萬人築之。增置各衛經歷一員,俾漢人為之。壬戌,弛畿內秋耕禁。



 八月乙丑朔,平陽地震,壞民廬舍萬有八百二十六區,壓死者百五十人。丙寅,太白犯輿鬼。己巳,置中書省檢校二員,秩正七品,俾考核戶、工部文案疏緩者。罷江西等處行泉府司、大都甲匠總管府、廣州人匠提舉司、廣德路錄事司,罷泉州至杭州海中水站十五所。撫州路饑,免去歲未輸田租四千五百石。馬八兒國遣使進花牛二、水牛土彪各一。丙子,太陰犯牽牛。大名之清河、南樂諸縣霖雨害稼,免田租萬六千六百六十九石。己卯,詔諭思州提省溪洞官楊都要招安叛蠻,悔過來歸者,與免本罪。罷雲南四州,立東川府。癸未,歲星犯軒轅大星。乙酉,遣麻速忽、阿散乘傳詣雲南,捕黑虎。戊子,太白犯軒轅大星,並犯歲星。咀喃籓邦遣馬不剌罕丁進金書、寶塔及黑獅子、番布、藥物。婺州水,免田租四萬一千六百五十石。辛卯,命工部造飛車五輛。癸巳,太陰掩熒惑。



 九月辛丑,以平章政事麥術丁商議中書省事,復以咱喜魯丁平章政事代之。乙巳,景州、河間等縣霖雨害稼,免田租五萬六千五百九十五石。丙午,立行宣政院,治杭州。己酉,設安西、延安、鳳翔三路屯田總管府。庚戌,太白犯右執法。襄陽南漳縣民李氏妻黃一產三男。辛亥,安南王陳日烜遣使上表貢方物,且謝不朝之罪。徽州績溪縣賊未平,免二十七年田租。禁宣德府田獵。壬子,酒醋課不兼隸茶鹽運司,仍隸各府縣。立乞里吉思至外剌等六驛。命海船副萬戶楊祥、合迷、張文虎並為都元帥,將兵征琉求。置左右兩萬戶府,官屬皆從祥選闢。既又用福建吳志斗言「祥不可信,宜先招諭之」,乃以祥為宣撫使,佩虎符,阮監兵部員外郎,志斗禮部員外郎,並銀符,齎詔往琉求。明年,楊祥、阮監果不能達琉求而還,志鬥死於行,時人疑為祥所殺,詔福建行省按問,會赦不治。乙卯,以歲荒,免平灤屯田二十七年田租三萬六千石有奇。丙辰,熒惑犯左執法。戊午,太白犯熒惑。徙四川行樞密院治成都。以八忽答兒、禿魯歡、唆不闌、脫兒赤四翼蒙古兵復隸蒙古都萬戶府。庚申,以鐵里為禮部尚書,佩虎符,阿老瓦丁、不剌並為侍郎,遣使俱藍。辛酉,歲星犯少民。免大都今歲田租。保定、河間、平灤三路大水,被災者全免,收成者半之。以別鐵木兒、亦列失金為禮部侍郎,使馬八兒國;陜西脫西為禮部侍郎,佩金符,使於馬都。尚衣局職無縫衣。



 冬十月乙丑朔,賜薛徹溫都兒等九驛貧民三月糧。己巳,修太廟在真定傾壞者。壬申,以前緬中行尚書省平章政事雪雪的斤為中書省平章政事。癸酉,享太廟。遣使發倉,賑大同屯田兵及教化的所部軍士之饑者。江淮行省言:「鹽課不足,由私鬻者多,乞付兵五千巡捕。」從之。塔剌海、張忽辛、崔同知並坐理算錢穀受贓論誅。辛巳,召高麗國王王睶、公主忽都魯揭裏迷失詣闕。癸未,羅斛國王遣使上表,以金書字,仍貢黃金、象齒、丹頂鶴、五色鸚鵡、翠毛、犀角、篤縟、龍腦等物。高麗國饑,給以米一十萬斛。罷各處行樞密院,事入行省。割八番洞蠻自四川隸湖廣行省。丙戌,太陰犯軒轅大星並御女。丁亥,洞蠻爛土立定雲府,改陳蒙洞為陳蒙州,合江為合江州。嚴山後酒禁。中書省臣言:「洞蠻請歲進馬五十匹、雨氈五十被、刀五十握,丹砂、雌雄黃等物,率二歲一上。」有詔從其所為。己丑,太陰犯太微東垣上相。敕沒入璉真加、沙不丁、烏馬兒妻,並遣詣京師。召行省轉運司官赴京師,集議治賦法。辛卯,諸王出伯部曲饑,給米賑之。癸巳,武平路饑,免今歲田租。以武平路總管張立道為禮部尚書,使交趾。免衛輝種仙茅戶徭役。從遼陽行省言,以乃顏、合丹相繼叛,詔給蒙古人內附者及開元、南京、水達達等三萬人牛畜、田器。詔嚴益都、般陽、泰安、寧海、東平、濟寧畋獵之禁,犯者沒其家貲之半。



 十一月丙申,以甘肅曠土賜昔寶赤合散等,俾耕之。壬寅,遣左吉奉使新合剌的音。甲辰,太白犯房。減太府監冗員三十一人,罷器備、行內藏二庫。詔:「回回以答納珠充獻及求售者還之,留其估以濟貧者。」塔義兒、塔帶民饑,發米賑之。給按答兒民戶四月糧,罷海道運糧鎮撫司。丙午,熒惑犯亢。丁未,太陰犯畢。耽羅遣使貢東紵百匹。太史院靈臺上修祀事三晝夜。甲寅,太陰犯歲星。郴州路達魯花赤曲列有罪論誅。復置會同館,禁沮擾益都淘金。乙卯,新添葛蠻宋安撫率洞官阿汾、青貴來貢方物。監察御史言:「沙不丁、納速剌丁滅里、烏里兒、王巨濟、璉真加、沙的、教化的皆桑哥黨與,受贓肆虐,使江淮之民愁怨載路,今或系獄,或釋之,此臣下所未能喻。」帝曰:「桑哥已誅,納速剌丁滅里在獄,唯沙不丁朕姑釋之耳。」武平、平灤諸州饑,弛獵禁,其孕字之時勿捕。諭中書議增中外官吏俸。戊午,金齒國遣阿腮入覲。庚申,熒惑犯氐。辛酉,升宣德龍門鎮為望雲縣,割隸雲州,置望雲銀冶。



 十二月乙丑,復都水監,秩從三品。遣官迓雲南鴨池所遣使。遼陽洪寬女直部民饑,借高麗粟賑給之。籍探馬赤八忽帶兒等六萬戶成丁者為兵。丁卯,高麗國鴨綠江西十九驛,經乃顏反,掠其馬畜,給以牛各四十。大都饑,下其價糶米二十萬石賑之。己巳,詔罷遣官招集畏兀氏。改辰、沅、靖州轉運司為湖北湖南道轉運司,立葛蠻軍民安撫司。宣政院臣言:「宋全太后、瀛國公母子以為僧、尼,有地三百六十頃,乞如例免徵其租。」從之。辛未,以鐵滅為兵部尚書,佩虎符,明思昔答失為兵部侍郎,佩金符,使於羅孛卜兒。御史臺臣言:「鉤考錢穀,自中統初至今餘三十年,更阿合馬、桑哥當國,設法已極,而其餘黨公取賄賂,民不堪命,不如罷之。」有旨:「議擬以聞。」壬申,立河南江北行中書省,治汴梁。撒里蠻、老壽並為大司徒,領太常寺。中書省臣言:「江南在宋時,差徭為名七十有餘,歸附後一切未征,今分隸諸王城邑,歲賜之物,仰給京師,又中外官吏俸少,似宜量添,可令江南依宋時諸名征賦盡輸之。」何榮祖言:「宜召各省官任錢穀者詣京師,集議科取之法以聞。」從之。甲戌,詔:「罷鉤考錢穀,應昔年逋負錢穀文卷,聚置一室,非朕命而視之者有罪。」仍遣使布告中外。庚辰,太陰犯御女。江北州郡割隸河南江北行中書省,改江淮行省為江浙等處行中書省,治杭州。賑闊闊出饑民米。闍裡帶言:「乃顏餘黨竄女直之地,臣與月兒魯議,乞益兵千五百人,可平之。」從之。癸未,太陰犯東垣上相。廣濟署大昌等屯水,免田租萬九千五百石。平灤路及豐贍、濟民二署饑,出米萬五千石賑之。別都兒丁前以桑哥專恣,不肯仕,命仍為中書左丞。丙戌,八番洞官吳金叔等以所部二百五十寨民二萬有奇內附,詣闕貢方物。戊子,詔釋天下囚非殺人抵罪者。己丑,熒惑犯房。庚寅,熒惑犯鉤鈐。升營田提舉司為規運提點所,正四品。辛卯,浚運糧壩河,築堤防。授吃剌思八斡節兒為帝師,統領諸國僧尼釋教事。賜親王、公主、駙馬金、銀、鈔、幣如歲例。令僧羅藏等遞作佛事坐靜於聖壽萬安、涿州寺等所,凡五十度。遣真人張志仙持香詣東北海岳、濟瀆致禱。戶部上天下戶數,內郡百九十九萬九千四百四十四,江淮、四川一千一百四十三萬八百七十八,口五千九百八十四萬八千九百六十四,游食者四十二萬九千一百一十八。司農司上諸路所設學校二萬一千三百餘,墾地千九百八十三頃有奇,植桑棗諸樹二千二百五十二萬七千七百餘株,義糧九萬九千九百六十石。宣政院上天下寺宇四萬二千三百一十八區,僧、尼二十一萬三千一百四十八人。斷死刑五十五人。



\end{pinyinscope}