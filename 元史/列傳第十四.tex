\article{列傳第十四}

\begin{pinyinscope}

 ○伯顏



 伯顏,蒙古八鄰部人。曾祖述律哥圖,事太祖,為八鄰部左千戶。祖阿剌,襲父職,兼斷事官,平忽禪有功,得食其地。父曉古臺世其官,從宗王旭烈兀開西域。伯顏長於西域。至元初,旭烈兀遣入奏事,世祖見其貌偉,聽其言厲,曰:「非諸侯王臣也,其留事朕。」與謀國事,恆出廷臣右,世祖益賢之,敕以中書右丞相安童女弟妻之,若曰「為伯顏婦,不慚爾氏矣」。二年七月,拜光祿大夫、中書左丞相。諸曹白事,有難決者,徐以一二語決之。眾服曰:「真宰輔也。」四年,改中書右丞。七年,遷同知樞密院事。十年春,持節奉玉冊立燕王真金為皇太子。



 十一年,大舉伐宋,與史天澤並拜中書左丞相,行省荊湖。時荊湖、淮西各建行省,天澤言,號令不一,或致敗事。詔改淮西行省為行樞密院。天澤又以病,表請專任伯顏,乃以伯顏領河南等路行中書省,所屬並聽節制。秋七月,陛辭,世祖諭之曰:「昔曹彬以不嗜殺平江南,汝其體朕心,為吾曹彬可也。」



 九月甲戌朔,會師於襄陽,分軍為三道並進。丙戌,伯顏與平章阿術由中道循漢江趨郢州。萬戶武秀為前鋒,遇水濼,霖雨水溢,無舟不能涉。伯顏曰:「吾且飛渡大江,而憚此潢潦耶!」乃召一壯士,負甲仗,騎而前導,麾諸軍畢濟。癸巳,次鹽山,距郢州二十里。郢在漢水北,以石為城,宋人又於漢水南築新郢,橫鐵繩,鎖戰艦,密樹樁木水中。下流黃家灣堡,亦設守禦之具,堡之西有溝,南通藤湖,至江僅數里。乃遣總管李庭、劉國傑攻黃家灣堡,拔之,破竹席地,蕩舟由藤湖入漢江。諸將請曰:「郢城,我之喉襟,不取,恐為後患。」伯顏曰:「用兵緩急,我則知之。攻城,下策也,大軍之出,豈為此一城哉!」遂舍郢,順流下。伯顏、阿術殿後,不滿百騎。十月戊午,行大澤中,郢將趙文義、範興以騎二千來襲,伯顏、阿術未及介胄,亟還軍迎擊之。伯顏手殺文義,擒範興殺之,其士卒死者五百人,生獲數十人。



 甲子,次沙洋。乙丑,命斷事官楊仁風招之,不應。復使一俘持黃榜、檄文,傳趙文義首,入城,招其守將王虎臣、王大用。虎臣等斬俘,焚黃榜。裨將傅益以水軍十七人來降,虎臣等又斬其軍之欲降者。伯顏復命呂文煥招之,又不應。日暮,風大起,伯顏命順風掣金汁砲,焚其廬舍,煙焰漲天,城遂破。萬戶忙古歹生擒虎臣、大用等四人,餘悉屠之。丙寅,次新城,令萬戶帖木兒、史弼列沙洋所馘於城下,射黃榜、檄文於城中以招之。其守將邊居誼,邀呂文煥與語。丁卯,文煥至城下,飛矢中右臂,奔還。戊辰,其總制黃順逾城出降,即授招討使,佩以金符,令呼城上軍,其部曲即縋城下,居誼邀入城,悉斬之。己巳,其副都統制任寧亦降,居誼終不出,乃令總管李庭攻破其外堡,諸軍蟻附而登,拔之。餘眾三千,猶力戰而死,居誼舉家自焚。遂並誅王虎臣、王大用等四人。



 十一月丙戌,次復州,知州翟貴以城降。諸將請點視其倉庫軍籍,遣官鎮撫,伯顏不聽,諭諸將不得入城,違者以軍法論。阿術使右丞阿里海牙來言渡江之期,伯顏不答。明日又來,又不答。阿術乃自來,伯顏曰:「此大事也,主上以付吾二人,可使餘人知吾實乎?」潛刻期而去。乙未,軍次蔡店。丁酉,往觀漢口形勢。宋淮西制置使夏貴等,以戰艦萬艘,分據要害,都統王達守陽邏堡,京湖宣撫硃示異孫以游擊軍扼中流,兵不得進。千戶馬福建言,淪河口可通沙蕪入江,伯顏使覘沙蕪口,夏貴亦以精兵守之。乃圍漢陽軍,聲言由漢口渡江,貴果移兵援漢陽。



 十二月丙午,軍次漢口。辛亥,諸將自漢口開壩,引船入淪河,先遣萬戶阿剌罕以兵拒沙蕪口,逼近武磯,巡視陽羅城堡,徑趨沙蕪,遂入大江。壬子,伯顏戰艦萬計,相踵而至,以數千艘泊於淪河灣口,屯布蒙古、漢軍數十萬騎於江北。諸將言:「沙蕪南岸,彼戰船在焉,可攻而取。」伯顏曰:「吾亦知其可必取,慮汝輩貪小功,失大事;一舉渡江,收其全功可也。」遂令修攻具,進軍陽羅堡。癸丑,遣人招之,不應。甲寅,再遣人招之,其將士皆曰:「我輩受宋厚恩,戮力死戰,此其時也,安有叛逆歸降之理。備吾甲兵,決之今日,我宋天下,猶賭博孤注,輸贏在此一擲爾。」伯顏麾諸將攻之,三日不克。有術者來言:「天道南行,金、木相犯,若二星交過,則江可渡。」伯顏卻之,使勿言。乃密謀於阿術曰:「彼謂我必拔此堡,方能渡江。此堡甚堅,攻之徒勞。汝今夜以鐵騎三千,泛舟直趨上流,為搗虛之計,詰旦渡江襲南岸。已過,則速遣人報我。」乙卯,分遣右丞阿里海牙督萬戶張弘範、忽失海牙、折的迷失等,先以步騎攻陽羅堡,夏貴來援。遂俾阿術出其不意,率萬戶晏徹兒、忙古歹、史格、賈文備四翼軍,溯流西上四十里,對青山磯而泊。是夜,雪大作,遙見南岸多露沙洲,阿術登舟,指示諸將,令徑趨是洲,載馬後隨。萬戶史格一軍先渡,為其都統程鵬飛所卻。阿術橫身蕩決,血戰中流,擒其將高邦顯等,死者無算,鵬飛被七創,敗走,得船千餘艘,遂得南岸。阿術與鎮撫何瑋等數十人,攀岸步斗,開而復合者數四。南軍阻水,不得相薄,遂起浮橋,成列而渡。阿里海牙繼遣張榮實、解汝楫等四翼軍,舳艫相銜,直抵夏貴。貴引麾下軍數千先遁,諸軍乘之,斬溺不可數計,追至鄂州東門而還。丙辰,阿術遣使來報,伯顏大喜,揮諸將急攻破陽羅堡,斬王達。宋軍大潰,數十萬眾死傷幾盡。夏貴僅以身免,走至白虎山。諸將謂貴大將,不可使逸去,請追之。伯顏曰;「陽羅之捷,吾欲遣使前告宋人,而貴走代吾使,不必追也。」丁巳,伯顏登武磯山,大江南北,皆我軍也,諸將稱賀,伯顏辭謝之。



 阿術還渡江,議兵所向,或欲先取蘄、黃,阿術曰:「若赴下流,退無所據,先取鄂、漢,雖遲旬日,可為萬全計。」伯顏從之。己未,師次鄂州,遣呂文煥、楊仁風等諭之曰:「汝國所恃者,江、淮而已,今我大兵飛渡長江,如履平地,汝輩何不速降。」鄂恃漢陽,將戰,乃焚其戰艦三千艘,火照城中,兩城大恐。庚申,知鄂州張晏然、知漢陽軍王儀、知德安府來興國,皆以城降,程鵬飛以其軍降。壬戌,定新附官品級,撤宋兵,分隸諸將。先是,邊民戍卒陷入宋境者,悉縱遣之。丁卯,遣萬戶也的哥、總管忽都歹,入奏渡江之捷。分命阿剌罕先鋒黃頭取壽昌糧四十萬斛,以充軍餉。留右丞阿里海牙等,以兵四萬分省於鄂,規取荊湖。己巳,伯顏與阿術以大軍水陸東下,俾阿術先據黃州。



 十二年春正月癸酉朔,至黃州。甲戌,沿江制置副使、知黃州陳奕降,伯顏承制授奕沿江大都督。奕遣書至漣水招其子巖,巖降。遣呂文煥、陳奕以書招蘄州安撫使管宗模,復遣阿術以舟師造其城下。癸未,伯顏至蘄州,宗模出降,即承制授以淮西宣撫使,留萬戶帶塔兒守之。阿術復以舟師先趨江州,兵部尚書呂師夔在江州,與知州錢真孫遣人來迎降。丙戌,伯顏至江州,即以師夔為江州守。師夔設宴庾公樓,選宋宗室女二人,盛飾以獻,伯顏怒曰:「吾奉聖天子明命,興仁義之師,問罪於宋,豈以女色移吾志乎!「斥遣之。知南康軍葉閶來降,殿前都指揮使、知安慶府範文虎亦奉書納款,阿術遂率舟師造安慶,文虎出降。伯顏至湖口,遣千戶寧玉系浮橋以渡,風迅水駛,橋不能成,乃禱於大孤山神,有頃,風息橋成,大軍畢渡。二月壬寅朔,伯顏至安慶,承制授文虎兩浙大都督,文虎以其從子友信知安慶府事,命萬戶喬珪戍之。丁未,次池州,都統制張林以城降;戊申,通判權州事趙昴發與其妻自經死,伯顏入城,見而憐之,令具衣衾葬焉。



 宋宰臣賈似道遣宋京致書,請還已降州郡,約貢歲幣。伯顏遣武略將軍囊加歹同其介阮思聰報命,止京以待,且使謂似道曰:「未渡江,議和入貢則可,今沿江諸郡皆內附,欲和,則當來面議也。」囊加歹還,乃釋宋京。庚申,發池州,壬戌,次丁家洲。賈似道都督諸路軍馬十三萬,號百萬,步軍指揮使孫虎臣為前鋒,淮西制置使夏貴以戰艦二千五百艘橫亙江中,似道將後軍。伯顏命左右翼萬戶率騎兵夾江而進,砲聲震百里。宋軍陣動,貴先遁,以扁舟掠似道船,呼曰:「彼眾我寡,勢不支矣!」似道聞之,倉皇失措,遽鳴金收軍,軍潰。眾軍大呼曰:「宋軍敗矣!」諸戰艦居後者,阿術促騎召之,挺身登舟,手柁沖敵船,舳艫相蕩,乍分乍合。阿術以小旗麾何瑋、李庭等並舟深入,伯顏命步騎左右掎之,追殺百五十餘里,溺死無算,得船二千餘艘,及其軍資器仗、圖籍符印。似道東走揚州,貴走廬州,虎臣走泰州。甲子,攻太平州。丁卯,知州孟之縉及知無為軍劉權、知鎮巢軍曹旺、知和州王喜,俱以城降。庚午,師次建康之龍灣,大賚將士。三月癸酉,宋沿江制置趙溍遁,溍兄淮起兵溧陽,就執而死。都統徐王榮、翁福等以城降,命招討使唆都守之。知鎮江府洪起畏遁,總管石祖忠以城降。知寧國府趙與可遁,知饒州唐震死,而江東諸郡皆下。淮西滁州諸郡亦相繼降。



 丙子,國信使廉希賢至建康,傳旨令諸將各守營壘,毋得妄有侵掠。希賢與嚴忠範等奉命使宋,請兵自衛,伯顏曰:「行人以言不以兵,兵多,徒為累使事。」希賢固請,與之。丙戌,至獨松嶺,果為宋人所殺。庚寅,伯顏遣左右司員外郎石天麟詣闕奏事,世祖大悅,悉可其奏。伯顏以行中書省駐建康,阿塔海、董文炳以行樞密院駐鎮江,阿術別奉詔攻揚州。江東歲饑,民大疫,伯顏隨賑救之,民賴以安。宋人遣都統洪模移書徐王榮等,言殺使之事太皇太后及嗣君實不知,皆邊將之罪,當按誅之,願輸幣,請罷兵通好。伯顏曰:「彼為譎詐之計,以視我之虛實。當擇人以同往,觀其事體,宣布威德,令彼速降。」乃命議事官張羽等持王榮答書,至平江驛,宋人又殺之。



 四月乙丑,有詔以時暑方熾,不利行師,俟秋再舉。伯顏奏曰:「宋人之據江海,如獸保險,今已扼其吭,少縱之則逸而逝矣。」世祖語使者曰:「將在軍,不從中制,後法也。宜從丞相言。」五月丁亥,復命奉御愛先傳旨,召伯顏赴闕,以阿剌罕為參政,留治省事。伯顏至鎮江,會諸將計事,令各還鎮,乃渡江北行,入見於上都。七月癸未,進中書右丞相,讓功於阿術,遂以阿術為左丞相。八月癸卯,受命還行省,付以詔書,俾諭宋主。乃取道益都,行視沂州等軍壘,調淮東都元帥孛魯歡、副都元帥阿里伯,以所部兵沂淮而進。九月戊寅,會師淮安城下,遣新附官孫嗣武叩城大呼,又射書城中,諭守將使降,皆不應。庚辰,招討別吉裏迷失拒北城西門,伯顏與孛魯歡、阿里伯親臨南城堡,揮諸將長驅而登,拔之,潰兵欲奔大城,追襲至城門,斬首數百級,遂平其南堡。丙戌,次寶應軍。戊子,次高郵。十月庚戌,圍揚州。召諸將指授方略,留孛魯歡、阿里伯守灣頭新堡,眾軍南行。壬戌,至鎮江,罷行院,以阿塔海、董文炳同署事。



 十一月乙亥,伯顏分軍為三道,期會於臨安。參政阿剌罕等為右軍,以步騎自建康出四安,趨獨松嶺;參政董文炳等為左軍,以舟師自江陰循海趨澉浦、華亭;伯顏及右丞阿塔海由中道,節制諸軍,水陸並進。壬午,伯顏軍至常州。先是,常州守王宗洙遁,通判王虎臣以城降,其都統制劉師勇與張彥、王安節等復拒之,推姚為守,固拒數月不下。伯顏遣人至城下,射書城中招諭:勿以已降復叛為疑,勿以拒敵我師為懼。皆不應。乃親督帳前軍臨南城,又多建火砲,張弓弩,晝夜攻之。浙西制置文天祥遣尹玉、麻士龍來援,皆戰死。甲申,伯顏叱帳前軍先登,豎赤旗城上,諸軍見而大呼曰:「丞相登矣。」師畢登。宋兵大潰,拔之,屠其城,姚及通判陳炤等死之,生獲王安節,斬之。劉師勇變服單騎奔平江,諸將請追之,伯顏曰:「勿追,師勇所過,城守者膽落矣。」以行省都事馬恕為常州尹。遣蒙古軍都元帥闍裏帖木兒、萬戶懷都先據無錫州,萬戶忙古歹、晏徹兒巡太湖,遣監戰亦乞裡歹、招討使唆都、宣撫使游顯,會闍裏帖木兒先趨平江。



 庚寅,遣降人游介實奉詔書副本使於宋,仍以書諭宋大臣。十二月辛丑,次無錫,宋將作監柳岳等奉其國主及太皇太后書,並宋之大臣與伯顏書來見,垂泣而言曰:「太皇太后年高,嗣君幼沖,且在衰絰中。自古禮不伐喪,望哀恕班師,敢不每年進奉修好。今日事至此者,皆奸臣賈似道失信誤國耳。」伯顏曰:「主上即位之初,奉國書修好,汝國執我行人一十六年,所以興師問罪。去歲又無故殺害廉奉使等,誰之過歟?如欲我師不進,將效錢王納土乎?李主出降乎?爾宋昔得天下於小兒之手,今亦失於小兒之手,蓋天道也,不必多言。」岳頓首泣不已。遣招討使抄兒赤,以柳岳來使事,及嚴奉使所齎國書入奏。



 先是,平江守潛說友遁,通判胡玉等既以城降,而復為宋人所據。甲辰,眾軍次平江,都統王邦傑、通判王矩之率眾出降。庚戌,遣囊加歹同其使柳嶽還臨安。以忙古歹、範文虎行兩浙大都督事。遣寧玉修吳江長橋,不旬日而成。庚申,囊加歹同宋尚書夏士林、侍郎呂師孟、宗正少卿陸秀夫以書來,請尊世祖為伯父,而世修子侄之禮,且約歲幣銀二十五萬兩,帛二十五萬匹。癸亥,遣囊加歹同師孟等還臨安。遣忙古歹、範文虎會阿剌罕、昔里伯取湖州,知州趙良淳死之。丙寅,趙與可以城降。伯顏發平江,留游顯、懷都、忽都不花屯兵鎮守。別遣寧玉守長橋。



 十三年正月己巳,次嘉興,安撫劉漢傑以城降,留萬戶忽都虎等戍之。癸酉,宋軍器監劉庭瑞以其宰臣陳宜中等書來,即遣回。乙亥,宜中遣御史劉岊奉宋主稱臣表文副本,及致書伯顏,約會長安鎮。辛巳,眾軍至崇德。宜中又令都統洪模,持書同囊加歹來見。壬午,次長安鎮,宜中等不至。癸未,進軍臨平鎮。甲申,次阜亭山,宋主遣知臨安府賈餘慶,同宗室保康軍承宣使尹甫、和州防禦使吉甫,奉傳國璽及降表詣軍前。伯顏受訖,遣囊加歹以餘慶等還臨安,召宋宰臣出議降事。時宜中已遁,以文天祥代為丞相,不拜,自請至軍前。乙酉,進軍至臨安北十五里,分遣董文炳、呂文煥、範文虎巡視城堡,安諭軍民。囊加歹、洪模來報,宜中與張世傑、蘇劉義、劉師勇等,挾益王、廣王下浙江,航海而南,惟謝太后及幼主在宮中。伯顏亟遣使諭右軍阿剌罕、奧魯赤,左軍董文炳、範文虎,據守浙江,以勁兵五千人追之,不及而還。



 丙戌,禁軍士毋入城,遣呂文煥持黃榜諭臨安中外軍民,俾安堵如故。先是,三衙衛士,白晝殺人,閭里小民,乘亂剽掠,至是民皆安之。丁亥,遣程鵬飛、洪雙壽等入宮,慰諭謝後。戊子,謝後遣丞相吳堅、文天祥,樞密謝堂,安撫賈餘慶,內官鄧惟善來見,伯顏慰遣之,顧天祥舉動不常,疑有異志,留之軍中。天祥數請歸,伯顏笑而不答。天祥怒曰:「我此來為兩國大事,彼皆遣歸,何故留我?」伯顏曰:「勿怒。汝為宋大臣,責任非輕,今日之事,政當與我共之。」令忙古歹、唆都館伴羈縻之。令程鵬飛、洪雙壽同賈餘慶易宋主削帝號降表。己丑,駐軍臨安城北之湖州市。遣千戶囊加歹等以宋傳國璽入獻。



 庚寅,伯顏建大將旗鼓,率左右翼萬戶,巡臨安城,觀潮於浙江。暮還湖州市,宋宗室大臣皆來見。辛卯,萬戶張弘範、郎中孟祺同程鵬飛,以所易降表及宋主、謝後諭未附州郡手詔至軍前。令鎮撫唐古歹罷文天祥所招募義兵二萬餘人。壬辰,伯顏登獅子峰,觀臨安形勢。命唆都撫諭軍民,部分諸將,共守其城,護其宮。癸巳,謝後復使人來勞問,仍以溫言慰遣之。甲午,分置其三衙諸司兵於各翼,以俟調遣;其生募等軍,願歸者聽。分遣蕭鬱、王世英等,招諭衢、信諸州。二月丁酉,遣劉頡等往淮西招夏貴,仍遣別將徇地浙東、西,於是知嚴州方回、知婺州劉怡、知臺州楊必大、知處州梁椅,並以城降。



 命右丞張惠,參政阿剌罕、董文炳、呂文煥入見謝後,宣布德意,以慰諭之。辛丑,宋主率文武百僚,望闕拜發降表。伯顏承制,以臨安為兩浙大都督府,忙古歹、範文虎入治府事。復命張惠、阿剌罕、董文炳、呂文煥等入城,籍其軍民錢穀之數,閱實倉庫,收百官誥命、符印圖籍,悉罷宋官府。取宋主居之別室。分遣新附官招諭湖南北、兩廣、四川未下州郡。部分諸將,分屯要害,仍禁人不得侵壞宋氏山陵。是日,進軍浙江之滸,潮不至者三日,人以為天助。



 癸卯,謝後命吳堅、賈餘慶、謝堂、家鉉翁、劉岊與文天祥並為祈請使,楊應奎、趙若秀為奉表押璽官,赴闕請命。伯顏拜表稱賀曰:



 臣伯顏言:國家之業大一統,海岳必明主之歸;帝王之兵出萬全,蠻夷敢天威之抗。始干戈之爰及,迄文軌之會同。區宇一清,普天均慶。臣伯顏等誠歡誠忭,頓首頓首,恭惟皇帝陛下,道光五葉,統接千齡。梯航日出之邦,冠帶月支之域;際丹崖而述職,奄瀚海而為家。獨此島夷,弗遵聲教,謂江湖可以保逆命,舟楫可以敵王師。連兵負固,逾四十年,背德食言,難一二計。當聖主飛渡江南之日,遣行人乞為城下之盟。逮凱奏之言旋,輒詐謀之復肆。拘囚我信使,忘乾坤再造之恩;招納我叛臣,盜漣海三城之地。我是以有六載襄樊之討,彼居然無一介行李之來。禍既出於自求,怒致聞於斯赫。臣伯顏等,肅將禁旅,恭行天誅。爰從襄漢之上流,復出武昌之故渡。籓屏一空於江表,烽煙直接於錢塘。尚無度德量力之心,薦有殺使毀書之事。屬廟謨之親廩,謂根本之宜先。乃命阿剌罕取道於獨松,董文炳進師於海渚,臣與阿塔海忝司中閫,直指偽都。掎角之勢既成,水陸之師並進。常州已下,列郡傳檄而悉平;臨安為期,諸將連營而畢會。彼知窮蹙,迭致哀鳴。始則有為侄納幣之祈,次則有稱籓奉璽之請。顧甘言何益於實事,率銳卒直抵於近郊。召來用事之大臣,放散思歸之衛士。崛強心在,四郊之橫草都無;飛走計窮,一片之降幡始豎。其宋國主已於二月初五日,望闕拜伏歸附訖。所有倉廩府庫,封籍待命外,臣奉揚寬大,撫戢吏民,九衢之市肆不移,一代之繁華如故。茲惟睿算,卓冠前王,視萬里如目前,運天下於掌上。致令臣等,獲對明時,歌《七德》以告成,深切龍庭之想,上萬年而為壽,敬陳虎拜之詞。臣伯顏等無任瞻天望聖激切屏營之至,謹奉表稱賀以聞。



 戊申,堅等發臨安,堂不行。癸丑,宋福王與芮奉書於伯顏,辭甚懇切,伯顏曰:「爾國既以歸降,南北共為一家,王勿疑,宜速來,同預大事。」且遣迓之。戊午,夏貴以淮西降。庚申,命囊加歹傳旨,召伯顏偕宋君臣入朝。三月丁卯,伯顏入臨安,俾郎中孟祺籍其禮樂祭器、冊寶、儀仗、圖書。庚午,囊加歹至。甲戌,與芮來。伯顏議以阿剌罕、董文炳留治行省事,以經略閩、粵;忙古歹以都督鎮浙西;唆都以宣撫使鎮浙東;唐兀歹、李庭護送宋君臣北上。乙亥,伯顏發臨安。丁丑,阿塔海等宣詔,趣宋主、母後入覲,聽詔畢,即日俱出宮,惟謝後以疾獨留,隆國夫人黃氏、宮人從行者百餘人,福王與芮、沂王乃猷、謝堂、楊鎮而下,官屬從行者數千人,三學之士數百人。宋主求見,伯顏曰:「未入朝,無相見之禮。」



 五月乙未,伯顏以宋主至上都,世祖御大安閣受朝,降授宋主鳷開府儀同三司、檢校大司徒,封瀛國公。宋平,得府三十七、州百二十八、關監二、縣七百三十三。命伯顏告於天地宗廟,大赦天下。帝勞伯顏,伯顏再拜謝曰:「奉陛下成算,阿術效力,臣何功之有。」復拜同知樞密院,賜銀鼠青鼠只孫二十襲。裨校有功者百二十三人,賞銀有差。



 初,海都稱兵內向,詔以右丞相安童佐皇子北平王那木罕,統諸軍於阿力麻裏備之。十四年,諸王昔里吉劫北平王,拘安童,脅宗王以叛,命伯顏率師討之,與其眾遇於斡魯歡河,夾水而陣,相持終日,俟其懈,麾軍為兩隊,掩其不備,破之,昔里吉走死。十八年二月,世祖命燕王撫軍北邊,以伯顏從,仍諭之曰:「伯顏才兼將相,忠於所事,故俾從汝,不可以常人遇之。」燕王每與論事,尊禮有加。是歲,頒群臣食邑,詔益以藤州等處四千九百七十七戶。



 伯顏之取宋而還也,詔百官郊迎以勞之,平章阿合馬先百官半舍道謁,伯顏解所服玉鉤絳遺之,且曰:「宋寶玉固多,吾實無所取,勿以此為薄也。」阿合馬謂其輕己,思中傷之,乃誣以平宋時取其玉桃盞,帝命按之,無驗,遂釋之,復其任。阿合馬既死,有獻此盞者,帝愕然曰:「幾陷我忠良!」別吉裏迷失嘗誣伯顏以死罪,未幾,以它罪誅,敕伯顏臨視,伯顏與之酒,愴然不顧而返。世祖問其故,對曰:「彼自有罪,以臣臨之,人將不知天誅之公也。」



 二十二年秋,宗王阿只吉失律,詔伯顏代總其軍。先是,邊兵嘗乏食,伯顏令軍中採蔑怯葉兒及蓿敦之根貯之,人四斛,草粒稱是,盛冬雨雪,人馬賴以不饑。又令軍士有捕塔剌不歡之獸而食者,積其皮至萬,人莫知其意。既而遣使輦至京師,帝笑曰:「伯顏以邊地寒,軍士無衣,欲易吾繒帛耳。」遂賜以衣。二十四年春二月,或告乃顏反,詔伯顏窺覘之,乃多載衣裘入其境,輒以與驛人。既至,乃顏為設宴,謀執之,伯顏覺,與其從者趨出,分三道逸去。驛人以得衣裘故,爭獻健馬,遂得脫,馳還白狀。夏四月,乃顏反,從世祖親征。奏李庭、董士選將漢軍,得以漢法戰。乃顏之黨金家奴、塔不歹進逼乘輿,漢軍力戰,乃皆潰,卒擒乃顏。二十六年,進金紫光祿大夫、知樞密院事,出鎮和林,和林置知院,自伯顏始。



 二十九年秋,宗王明理鐵木兒挾海都以叛,詔伯顏討之,相值於阿撒忽禿嶺,矢下如雨,眾軍莫敢登,伯顏令之曰:「汝寒君衣之,汝饑君食之,政欲效力於此時爾。於此不勉,將何以報!」麾諸軍進,後者斬,伯顏先登陷陣,諸軍望風爭奮,大破之。明里鐵木兒挺身走,命速哥、梯迷禿兒等追之。伯顏引軍夜還,至必失禿,卒遇伏兵,伯顏堅壁不動,黎明,遂引去。伯顏輕騎追至別竭兒,速哥、梯迷禿兒等兵亦至,乃夾擊之,斬首二千級,俘其餘眾以歸。諸將言:古禮,兵勝必祃旗於所征之地。欲用囚虜為牲,伯顏不可,眾皆嘆服。軍中獲諜者,忻都欲殺之,伯顏不許,厚賜之,遣齎書諭明里鐵木兒以禍福,明里鐵木兒得書感泣,以眾來歸。未幾,海都復犯邊,伯顏留拒之。廷臣有譖伯顏久居北邊,與海都通好,仍保守,無尺寸之獲者,詔以御史大夫玉昔帖木兒代之,居伯顏於大同,以俟後命。玉昔帖木兒未至三驛,會海都兵復至,伯顏遣人語玉昔帖木兒曰:「公姑止,待我翦此寇而來,未晚也。」伯顏與海都兵交,且戰且卻,凡七日,諸將以為怯,憤曰:「果懼戰,何不授軍於大夫!」伯顏曰:「海都縣軍涉吾地,邀之則遁,誘其深入,一戰可擒也。諸軍必欲速戰,若失海都,誰任其咎?」諸將曰:「請任之。」即還軍擊敗之,海都果脫去。乃召玉昔帖木兒至軍,授以印而行。時成宗以皇孫奉詔撫軍北邊,舉酒以餞曰:「公去,將何以教我?」伯顏舉所酌酒曰:「可慎者,惟此與女色耳。軍中固當嚴紀律,而恩德不可偏廢。冬夏營駐,循舊為便。」成宗悉從之。



 三十年冬十二月,驛召至自大同,世祖不豫。明年正月,世祖崩,伯顏總百官以聽。兵馬司請日出鳴晨鐘,日入鳴昏鐘,以防變故,伯顏呵之曰:「汝將為賊邪!其一如平日。」適有盜內府銀者,宰執以其幸赦而盜,欲誅之,伯顏曰:「何時無盜,今以誰命而誅之?」人皆服其有識。成宗即位於上都之大安閣,親王有違言,伯顏握劍立殿陛,陳祖宗寶訓,宣揚顧命,述所以立成宗之意,辭色俱厲,諸王股慄,趨殿下拜。五月,拜開府儀同三司、太傅、錄軍國重事,依前知樞密院事,賜金銀各有差。時相有忌之者,伯顏語之曰:「幸送我兩罌美酒,與諸王飲於宮前,餘非所知也。」江南三省累請罷行樞密院,成宗問於伯顏,時已屬疾,張目對曰:「內而省、院各置為宜,外而軍、民分隸不便。」成宗是之,三院遂罷。冬十二月丙申,有大星隕於東北。己亥,雨木冰。庚子,伯顏薨,年五十九。



 伯顏深略善斷,將二十萬眾伐宋,若將一人,諸帥仰之若神明。畢事還朝,歸裝惟衣被而已,未嘗言功也。大德八年,特贈宣忠佐命開濟功臣、太師、開府儀同三司,追封淮安王,謚忠武。至正四年,加贈宣忠佐命開濟翊戴功臣,進封淮王,餘如故。



 子買的,僉樞密院事;囊加歹,樞密副使。孫相嘉失禮,同僉樞密院事、集賢學士。至治末,省先塋於白只剌山,聞有變,赴上都,或勸少避之,曰:「我與國同休戚,今有難,可避乎!」至上都,果見囚。久之得釋,尋拜河南江北行省平章政事,遷江南行臺御史大夫。曾孫普達失理,皆能世其家。



\end{pinyinscope}