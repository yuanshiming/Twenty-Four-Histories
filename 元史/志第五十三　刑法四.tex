\article{志第五十三 刑法四}

\begin{pinyinscope}

 ○詐偽



 諸主謀偽造符寶,及受財鑄造者,皆處死。同情轉募工匠,及受募刻字者,杖一百七。偽造制敕者,與符寶同。諸妄增減制書者,處死。諸近侍官輒詐傳上旨者,杖一百七,除名不敘。諸偽造省府印信文字,但犯制敕者處死。若偽造省府劄付者,杖一百七,再犯流遠。知情不首者,八十七。其文理訛謬不堪行用者,九十七。若偽造司縣印信文字,追呼平民,勒取財物者,初犯杖七十七,累犯不悛者一百七。諸偽造宣慰司印信契本,及商稅務青由欺冒商賈者,杖一百七。諸赦前偽造省印,赦後不曾銷毀,杖七十七,有官者奪所受宣敕,除名不敘。諸掾屬輒造省官押字,盜用省印,賣放官職者,雖會赦,流遠。諸偽造稅物雜印,私熬顏色,偽稅物貨者,杖八十七。告捕得實者,徵中統鈔一百貫充賞。物主知情,減犯人罪一等,其匿稅之物,一半沒官,於沒官物內一半付告人充賞;不知情者不坐,物給元主。其捕獲人擅自脫放者,減犯人罪二等,受財者與犯人同罪。諸省部小史,為人誤毀行移檢扎,輒自刻印信,偽補署押,求蓋本罪,無他情弊者,杖七十七,發元籍。諸僧道偽造諸王印信及令旨抄題者,處死。諸盤獲偽造印信之人,同獲強盜給賞。諸告獲私造歷日者,賞銀一百兩。如無太史院歷日印信,便同私歷造者,以違制論。諸受財賣他人敕牒,及收買轉賣者,杖一百七,刺面發元籍,買者杖八十七,發元籍。諸職官被差,以疾輒令人代乘驛傳而往者,杖六十七,代者笞五十七。諸公差,於官船夾帶從人,冒支分例者,笞一十七,記過,支過分例來,追徵還官。



 諸詐稱使臣,偽寫給驛文字,起馬匹舟船者,杖一百七。有司失覺察,輒憑無印信關牒倒給者,判署官笞三十七,首領官吏四十七。諸職官詐傳上司言語,擅起驛馬者,杖六十七。脫脫禾孫依隨擅給驛馬者,笞五十七,並解職別敘,記過;驛官二十七,還職。諸詐稱按部官,恐嚇官吏者,杖六十七。諸詐稱監臨長官署置差遣,欺取錢物者,杖八十七,錢物沒官。諸詐稱奉使所委官,聽理民訟者,杖九十七。詐稱隨行令史者,笞五十七。



 諸偽造寶鈔,首謀起意,並雕板抄紙,收買顏料,書填字號,窩藏印造,但同情者皆處死,仍沒其家產。兩鄰知而不首者,杖七十七。坊里正、主首、社長失覺察,並巡捕軍兵各笞四十七。捕盜官及鎮守巡捕軍官各三十七,未獲賊徒,依強盜立限緝捕。買使偽鈔者,初犯杖一百七,再犯加徒一年,三犯科斷流遠。諸捕獲偽鈔,賞銀五錠,給銀不給鈔。諸父子同造偽鈔者,皆處死。諸父造偽鈔,子聽給使,不與父同坐;子造偽鈔,父不同造,不與子同坐。諸夫偽造寶鈔者,妻不坐。諸偽造寶鈔,印板不全者,杖一百七。諸偽造寶鈔,沒其家產,不及其妻子。諸赦前收藏偽鈔,赦後行使者,杖一百七。不曾行使而不首者,減一等。諸偽造鈔罪應死者,雖親老無兼丁,不聽上請。諸捕獲偽造寶鈔之人,雖已身故,其應得賞錢,仍給其親屬。諸奴婢買使偽鈔,其主陳首者,不在理賞之例。諸挑剜裨輳寶鈔者,不分首從,杖一百七,徒一年,再犯流遠。年七十以上者,呈稟定奪,毋輒聽贖。買使者減一等。諸燒造偽銀者,徒。諸造賣偽銀,買主不知情,價錢給主,偽銀內銷,提真銀沒官,依本犯科罪。諸偽造各倉支發糧籌者,笞五十七,已支出官糧者,準盜系官錢物科罪。倉官人等有犯者,依監主自盜法,贓重者從重論。諸冒支官錢,計贓以枉法論,並除名不敘。



 諸冒名入仕者,杖六十七,奪所受命,追俸發元籍,會赦不首,笞四十七,仍追奪之。諸奴受主命冒充職官者,杖九十七。其主及同僚相容隱者,八十七。諸子冒父官居職任事者,杖七十七,犯在革前,革後不出首者,笞四十七,並追回所受宣敕,及支過俸祿還官。諸邊臣,輒以子婿詐稱招徠蠻獠,保充土官者,除名不敘,拘奪所授官。諸軍官承襲,偽增年者,監察御史廉訪司糾察之,濫保官吏,並坐罪。諸職官妄報出身履歷者,除名不敘。諸譯史、令史,有過不敘。詐稱作闕,別處補用者,笞五十七,罷役不敘。



 諸輸納官物,輒增改硃鈔者,杖六十七,罷之。諸有司長官,輒以追到盜贓支使,卻虛立給主文案者,雖會赦,解職,降先職二等敘。承吏,除名不敘。諸帥府上功文字,詐添有功軍人名數,主謀者杖八十七,除名不敘,隨從書寫者笞五十七。諸詐以軍功受舉入仕者,罷之,仍奪所受命。諸擅改已奏官員選目姓名者,雖會赦,除名發元籍。諸曹吏輒於公版改易年月,圖逭罪責者,笞五十七,罷役別敘,記過。諸嘩強之人,輒為人偽增籍面者,杖八十七,紅泥粉壁識過其門。諸蒙古譯史,能辨出詐偽文字二起以上者,減一資升轉。



 ○訴訟



 諸告人罪者,須明注年月,指陳實事,不得稱疑。誣告者抵罪反坐,越訴者笞五十七。本屬官司有過,及有冤抑,屢告不理,或理斷偏屈,並應合回避者,許赴上司陳之。諸訴訟本爭事外,別生餘事者,禁。其本爭事畢,別訴者聽。諸軍民風憲官有罪,各從其所屬上司訴之。諸民間雜犯,赴有司陳首者聽。諸告言重事實,輕事虛,免坐;輕事實,重事虛,反坐。諸中外有司,發人家錄私書,輒興獄訟者,禁之。若本宗事須引用證驗者,仍聽追照。其構飾傅會,以文致人罪者,審辨之。除本宗外,餘事並勿聽理。諸教令人告緦麻以上親,及奴婢告主者,各減告者罪一等。若教令人告子孫,各減所告罪二等。其教令人告事虛應反坐,或得實應賞者,皆以告者為首,教令為從。諸老廢篤疾,事須爭訴,止令同屬親屬深知本末者代之。若謀反大逆,子孫不孝,為同居所侵侮,必須自陳者聽。諸致仕得代官,不得已與齊民訟,許其親屬家人代訴,所司毋侵撓之。諸婦人輒代男子告辨爭訟者,禁之。若果寡居,及雖有子男,為他故所妨,事須爭訟者,不在禁例。諸子證其父,奴訐其主,及妻妾弟侄不相容隱,凡幹名犯義,為風化之玷者,並禁止之。諸親屬相告,並同自首。諸妻訐夫惡,比同自首原免。凡夫有罪,非惡逆重事,妻得相容隱,而輒告訐其夫者,笞四十七。諸妻曾背夫而逃,被斷復誣告其夫以重罪者,抵罪反坐,從其夫嫁賣。諸職官同僚相言者,並解職別敘,記過。諸告人罪者,自下而上,不得越訴。諸府州司縣應受理而不受理,雖受理而聽斷偏屈,或遷延不決者,隨輕重而罪罰之。諸訴官吏受賂不法,徑赴憲司者,不以越訴論。諸陳訴有理,路府州縣不行,訴之省部臺院,省部臺院不行,經乘輿訴之。未訴省部臺院,輒經乘輿訴者,罪之。諸職官誣告人枉法贓者,以其罪罪之,除名不敘。諸奴婢誣告其主者處死,本主求免者,職減一等。諸以奴告主私事,主同自首,奴杖七十七。



 ○鬥毆



 諸鬥毆,以手足擊人傷者,笞二十七,以他物者三十七。傷及拔發方寸以上,四十七。若血從耳目出及內損吐血者,加一等。折齒、毀缺耳鼻、眇一目及折手足指,若破骨及湯火傷人者,杖六十七。折二齒二指以上,及髡發,並刃傷、折人肋、眇人兩目、墮人胎,七十七。以穢物污人頭面者,罪亦如之。折跌人肢體,及瞎其目者,九十七。辜內平復者,各減二等。即損二事以上,及因舊患,令至篤疾,若斷舌及毀敗人陰陽者,一百七。諸訴毆詈,有闌告者勿聽,違者究之。諸保辜者,手足毆傷人,限十日。以他物毆傷者,二十日。以刃及湯火傷人者,三十日,折跌肢體及破骨者,五十日。毆傷不相須,餘條毆傷,及殺傷者準此。限內死者,各依殺人論。其在限外,及雖在限內,以他故死者,各依本毆傷法。他故,謂別增餘患而死者。諸倡女鬥傷良人,辜限之外死者,杖七十七,單衣受刑。諸毆傷人,辜限外死者,杖七十七。諸以非理毆傷妻妾者,罪以本毆傷論,並離之。若妻不為父母悅,以致非理毆傷者,罪減三等,仍離之。諸職官毆妻墮胎者,笞三十七,解職,期年後降先品一等,注邊遠一任,妻離之。諸以非理苦虐未成婚男婦者,笞四十七,婦歸宗,不追聘財。諸舅姑非理陵虐無罪男婦者,笞四十七,男婦歸宗,不追聘財。諸蒙古人與漢人爭,毆漢人,漢人勿還報,許訴於有司。諸蒙古人斫傷他人奴,知罪願休和者聽。諸以他物傷人,致成廢疾者,杖七十七,仍追中統鈔一十錠,付被傷人,充養濟之資。諸因鬥毆,斫傷人成廢疾者,杖八十七,徵中統鈔一十錠,付被傷人,充養濟之資。為父還毆致傷者,徵其鈔之半。諸豪橫輒誣平人為盜,捕其夫婦男女,於私家拷訊監禁,非理陵虐者,杖一百七,流遠。其被害有致殘廢者,人征中統鈔二十錠,充養贍之貲。諸職官輒將養男去勢,以充閹官進納者,杖一百七,除名不敘,記過,義男歸宗。諸以微故殘傷義男肢體廢疾者,加凡人折跌肢體一等論,義男歸宗,仍徵中統鈔五百貫,充養贍之貲。諸尊長輒以微罪刺傷弟侄雙目者,與常人同罪,杖一百七,追徵贍養鈔二十錠給苦主,免流,識過於門;無罪者,仍流。諸弟雖聽其兄之仇,同謀剜其兄之眼,即以弟為首,各杖一百七,流遠,而弟加遠。諸卑幼挾仇,輒刺傷尊長雙目成廢疾者,杖一百七,流遠。諸以刃刺破人兩目成篤疾者,杖一百七,流遠,仍徵中統鈔二十錠,充養贍之貲,主使者亦如之。諸挾仇傷人之目者,若一目元損,又傷其一目,與傷兩目同論,雖會赦,仍流。諸因爭誤瞎人一目者,杖七十七,徵中統鈔五十兩,充醫藥之貲。



 諸脫脫禾孫輒毆傷往來使臣者,笞四十七,解職記過。諸職官輒以他物毆傷使臣者,杖六十七。諸司屬官輒毆本管上司幕官者,笞四十七,解職記過。諸方鎮僚屬輒以他物毆傷主帥者,杖六十七,幕官使酒罵長官者,笞四十七,並解職別敘,記過。諸按部官因爭辯,輒毆有司官,有司官還毆者,各笞三十七,解職。諸監臨官挾怨,當扯捽屬官,屬官輒毆之者,笞四十七,解職。諸方面大臣,不能以正率下,輒與幕屬公堂鬥爭,雖會赦,並罷免記過,赦前無招者還職。諸職官輒毆傷所監臨,以所毆傷法論罪,記過。諸職官毆傷同署長官者,笞五十七,解見任,降先品一等敘,仍記過名。諸有司長官,輒毆同位正官者,笞三十七,毆佐貳官者,二十七,並解職記過。諸同僚改除,復以私忿相毆詈者,皆罷其所受新命。諸在閑職官,輒毆詈本籍在任長官者,杖六十七。諸職官相毆,其官等,從所傷輕重論罪。諸軍官縱酒,因戲而怒,故毆傷有司官者,笞三十七,記過。諸幕僚因公輒以惡言詈長官者,笞四十七,長官輒還毆者,笞一十七,並記過名。諸職官乘醉當街毆傷平人者,笞四十七,記過。諸職官閑居與庶民相毆者,職官減一等,聽罰贖。諸以他物毆傷職官者,加一等,笞五十七。諸小民恃年老毆詈所屬官長者,杖六十七,不聽贖。諸惡少無賴輒毆傷禁近之人者,杖七十七。



 ○殺傷



 諸殺人者死,仍於家屬徵燒埋銀五十兩給苦主,無銀者徵中統鈔一十錠,會赦免罪者倍之。諸部民毆死官長,主謀及下手者皆處死,同毆傷非致命者,杖一百七,流遠,均徵燒埋銀。諸殺人還自殺不死者,仍處死。諸殺人從而加功,無故殺之情者,會赦仍釋之。諸鬥毆殺人,先誤後故者,即以故殺論。諸因鬥毆,以刃殺人,及他物毆死人者,並同故殺。諸因爭以刃傷人,幸獲生免者,杖一百七。諸持刃方殺人,人覺而逃,卻移怒殺所解勸者,與故殺同。諸有司徵科急,民弗堪,致殺其徵科者,仍以故殺論。諸醉中欲殺其妻不得,移怒殺死其解紛之人者,處死。諸欲誘倡女逃,不從輒殺之者,與殺常人同。諸鬥毆殺人者,結案待報。諸人殺死其父,子毆之死者,不坐,仍於殺父者之家征燒埋銀五十兩。諸蒙古人因爭及乘醉毆死漢人者,斷罰出征,並全徵燒埋銀。諸因閧爭,一人誤蹂死小兒,一人毆人致死,毆者結案,蹂者杖一百七,並徵燒埋銀。諸有人戲調其妻,夫遇而毆之,因傷而死者,減死一等論罪,仍徵燒埋銀。諸毆死應捕殺惡逆之人者,免罪,不征燒埋銀。諸以他物傷人,傷毒流注而死,雖在辜限之外,仍減殺人罪三等坐之。諸因爭,以頭觸人,與人俱僕,肘抵其心,邂逅致死者,杖一百七,全徵燒埋銀。諸出使從人,毆死館夫者,以毆殺論。諸因戲言相毆,致傷人命者,杖一百七。諸父亡,母復納他人為夫,即為義父。若逐其子出居於外,即同凡人,其有所鬥毆殺傷,即以凡人鬥毆殺傷論。諸彼此有罪之人,相格致死者,與殺常人同。



 諸職官以微故毆死齊民者,處死,諸職官受贓,為民所告,輒毆死告者,以故殺論。諸軍官因公乘怒,輒命麾下毆人致死者,杖八十七,解職,期年後降先品一等敘,徵燒埋銀給苦主,若會赦,仍殿降徵銀。諸閫帥侵盜系官錢糧,怒吏發其奸,輒令人毆死者,以故殺論,雖會大赦,仍追奪不敘,倍徵燒埋銀。諸局院官輒以微故毆死匠人者,處死。



 諸父無故以刃殺其子者,杖七十七。諸子不孝,父與弟侄同謀置之死地者,父不坐,弟侄杖一百七。諸女已嫁,聞女有過,輒殺其女者,笞五十七,追還元受聘財,給夫別娶。諸父有故毆其子女,邂逅致死者,免罪。諸後夫毆死前夫之子者,處死。諸妻故殺妾子者,杖九十七,從其夫嫁賣。諸男婦雖有過,舅姑輒加殘虐致死者,杖一百七。諸子不孝,父殺其子,因及其婦者,杖七十七,婦元有妝奩之物,盡歸其父母。諸以細故殺其弟者,處死。諸兄以立繼之子,主謀殺其嫡弟者,主謀下手皆處死,其田宅人口財物盡歸死者妻子,其子歸宗。諸弟先毆其兄,兄還殺其弟,即兄殺有罪之弟,不以凡人鬥殺論。諸因爭誤毆死異居弟者,杖七十七,徵燒埋銀之半。諸因爭故殺族弟者,與殺常人同。諸妹為尼與人私,兄聞而諫之,不從,反詬詈扯捽其兄,兄殺之,即兄殺有罪之妹,不以凡人鬥殺論。諸兄毆弟妻,因傷而死者,杖一百七,徵燒埋銀。諸嫂溺死其小姑者,以故殺論。諸因爭毆死族兄弟之子者,杖一百七;故以刃殺之者,處死,並徵燒埋銀。諸毆死兄弟之子而圖其財者,處死。諸夫婦同謀,殺其兄弟之子者,皆處死。諸尊長誤毆卑幼致死者,杖七十七,異居者仍徵燒埋銀。諸以微過輒殺其妻者,處死。諸因夫妻反目,輒藥死其妻者,與故殺常人同。諸妻悖慢其舅姑,其夫毆之致死者,杖七十七。諸夫臥疾,妻不侍湯藥,又詬詈其舅姑,以傷其夫之心,夫毆之,邂逅致死者,不坐。諸夫惡妻而愛妾,輒求妻微罪而殺之者,處死。諸風聞涉疑,故殺定婚妻者,與殺凡人同論。諸妻以殘酷毆死其妾者,杖一百七,去衣受刑。諸舅以無實之罪故殺其甥者,與殺常人同論。諸因爭挾仇毆死其婿者,與殺常人同。



 諸奴毆詈其主,主毆傷奴致死者,免罪。諸故殺無罪奴婢,杖八十七,因醉殺之者,減一等。諸毆死擬放良奴婢者,杖七十七。諸謀殺已放良奴婢者,與故殺常人同。諸良人以鬥毆殺人奴,杖一百七,徵燒埋銀五十兩。諸良人戲殺他人奴者,杖七十七,徵燒埋銀五十兩。諸奴毆死其弟,弟亦為同主奴,主乞貸死者聽。諸異主奴婢相犯死者,同常人;同主相犯至重刑者,仍依例結案。諸地主毆死佃客者,杖一百七,徵燒埋銀五十兩。



 諸醉中誤認他人為仇人,故殺致命者,雖誤同故。諸奴受本主命,執仇殺人者,減死流遠。諸挾仇殺人會赦,為首下手者不赦,為從不曾下手者免死,徒一年,諸以老病殺人者,不以老病免。諸謀故殺人年七十以上,並枷禁歸勘結案。諸兩家之子,昏暮奔還,中路相迎,撞僕於地,因傷致死者,不坐,仍徵鈔五十兩給苦主。諸十五以下小兒,過失殺人者,免罪,徵燒埋銀。諸十五以下小兒,因爭毀傷人致死者,聽贖,徵燒埋銀給苦主。諸瞽者毆人,因傷致死,杖一百七,徵燒埋銀給苦主。諸病風狂,毆傷人致死,免罪,徵燒埋銀。諸庸醫以針藥殺人者,杖一百七,徵燒埋銀。諸颺磚石剝鄰之果,誤傷人致死者,杖八十七,徵燒埋銀。諸軍士習射,招箭者不謹,致被傷而死,射者不坐,仍徵燒埋銀。諸過誤踏死小兒,杖七十七,徵燒埋銀給苦主。諸昏夜馳馬,誤觸人死,杖七十七,徵燒埋銀。諸驅車走馬,致傷人命者,杖七十七,徵燒埋銀。諸昏夜行車,不知有人在地,誤致轢死者,笞三十七,徵燒埋銀之半給苦主。諸幼小自相作戲,誤傷致死者,不坐。諸戲傷人命,自願休和者聽。諸兩人作戲爭物,一人放手,一人失勢跌死,放者不坐。諸以物戲驚小兒,成疾而死者,杖六十七,追徵燒埋銀五十兩。諸以戲與人相逐,致人跌傷而死者,其罪徒,仍徵燒埋銀給苦主。諸駱駝在牧,嚙人而死者,牧人笞一十七,以駱駝給苦主。諸驛馬在野,嚙人而死者,以其馬給苦主,馬主別買當役。諸奴故殺其子女,以誣其主者,杖一百七。諸因爭,以妻前夫男女溺死,誣賴人者,以故殺論。諸後夫置毒飲食,與前夫子女食而死者,與藥死常人同。諸故殺無罪子孫,以誣賴仇人者,以故殺常人論。諸殺人無苦主者,免征燒埋銀,犯人財產人口並付其妻子,仍為民當差。諸殺有罪之人,免征燒埋銀。諸圖財謀故殺人多者,皆凌遲處死,驗各賊所殺人數,於家屬均徵燒埋銀。諸同居相毆而死,及殺人罪未結正而死者,並不征燒埋銀。諸殺人者,被殺之人或家住他所,官征燒埋銀移本籍,得其家屬給之。諸鬥毆殺人,應徵燒埋銀,而犯人貧窶,不能出備,並其餘親屬無應徵之人,官與支給。諸致傷人命,應徵燒埋銀者,止征銀價中統鈔一十錠。諸因爭同毆死人,會赦應倍徵燒埋銀者,為首致命征中統鈔一十錠,為從均徵一十錠。諸毆死人,雖不見尸,招證明白者,仍徵燒埋銀。諸僧道殺人,燒埋銀於常住追徵。諸庸作毆傷人命,徵燒埋銀,不及庸作之家。諸奴毆人致死,犯在主家,於本主征燒埋銀;不犯在主家,燒埋銀無可徵者,不征於其主。



 ○禁令



 諸度量權衡不同者,犯人笞五十七。司縣正官,初犯罰俸一月,再犯笞二十七,三犯別議,仍記過名。路府州縣達魯花赤長官提調失職,初犯罰俸二十日,再犯別議。諸奏目及官府公文,並用國字,其有襲用畏兀字者,禁之。諸但降詔旨條畫,民間輒刻小本賣於市者,禁之。諸內外應佩符職官,輒以符付其傔從佩服者,禁之。諸官員朝會,服其朝服,私致敬於人臣者罰。諸隨朝文武百官,朝賀不至者,罰中統鈔十貫,失儀者罰中統鈔八貫。諸宰相出入,輒敢沖犯者,罪之。



 諸章服,惟蒙古人及宿衛之士,不許服龍鳳文,餘並不禁。謂龍,五爪二角者。職官一品、二品許服渾金花,三品服金答子,四品、五品服云袖帶襴,六品、七品服六花,八品、九品服四花,職事散官從一高。命婦一品至三品服渾金,四品、五品服金答子,六品以下惟服銷金並金紗答子。首飾,一品至三品許用金珠寶玉,四品、五品用金玉真珠,六品以下用金,惟耳環用珠玉。同籍者,不限親疏,期親雖別籍並出嫁同。車輿並不得用龍鳳文,一品至三品許用間金妝飾、銀螭頭、繡帶、青幔,四品、五品用素獅頭、繡帶、青幔,六品至九品用素雲頭、素帶、青幔。內外有出身考滿應入流見役人員,服用與九品同。受各投下令旨鈞旨,有印信見任人員,亦與九品同。庶人惟許服暗花紵絲、絲綢綾羅、毛毳,不許用赭黃,冒笠不得飾以金玉,靴不得裁置花樣。首飾許用翠花金釵篦各一事,惟耳環許用金珠碧甸,餘並用銀。車輿,黑油齊頭平頂皁幔。諸色目人,除行營帳外,餘並與庶人同。職官致仕與見任同,解降者依應得品級;不敘者與庶人同。父祖有官,既歿年深,非犯除名不敘,其命婦及子孫與見任同。諸樂人工藝人等服用,與庶人同,凡承應妝扮之物,不拘上例。皁隸公使人,惟許服綢絹。倡家出入,止服皁褙,不許乘坐車馬。應服色等第,上得兼下,下不得僭上,違者,職官解見任,期年後降一等敘,餘人笞五十七,違禁之物,付告捉人充賞。御賜之物,不在禁限。諸官員以黃金飾甲者禁之,違者甲匠同罪。諸常人鞍韂,畫虎兔者聽,畫雲龍犀牛者,禁之。諸段匹織造周身大龍者,禁之,胸背小龍者勿禁。諸市造鞍轡箭鏃靴履及諸雜帶,用金為飾者,禁之。



 諸郡縣達魯花赤及諸投下,擅造軍器者,禁之。諸神廟儀仗,止以土木紙彩代之,用真兵器者禁。諸都城小民,造彈弓及執者,杖七十七,沒其家財之半,在外郡縣不在禁限。諸打捕及捕盜巡馬弓手、巡鹽弓手,許執弓箭,餘悉禁之。諸漢人持兵器者,禁之;漢人為軍者不禁。諸賣軍器者,賣與應執把之人者不禁。諸民間有藏鐵尺、鐵骨朵,及含刀鐵拄杖者,禁之。諸私藏甲全副者,處死;不成副者,笞五十七,徒一年;零散甲片下堪穿系禦敵者,笞三十七。槍若刀若弩私有十件者,處死;五件以上,杖九十七,徒三年;四件以上,七十七,徒二年;不堪使用,笞五十七。弓箭私有十副者,處死;五副以上,杖九十七,徒三年;四副以下,七十七,徒二年;不成副,笞五十七。凡弓一,箭三十,為一副。



 諸岳瀆祠廟,輒敢觸犯作踐者,禁之。諸伏羲、媧皇、堯、舜、禹、湯、后土等廟,軍馬使臣敢沮壞者,禁之。諸名山大川寺觀祠廟,並前代名人遺跡,敢拆毀者,禁之。諸改寺為觀,改觀為寺者,禁之。諸祠廟寺觀,模勒御寶聖旨及諸王令旨者,禁之。



 諸為子行孝,輒以割肝、刲股、埋兒之屬為孝者,並禁止之。諸民間喪葬,以紙為屋室,金銀為馬,雜彩衣服帷帳者,悉禁之。諸墳墓以磚瓦為屋其上者,禁之。諸家廟春秋祭祀,輒用公服行禮者,禁之。諸民間祖宗神主,稱皇字者,禁之。諸小民房屋,安置鵝項銜脊,有鱗爪瓦獸者,笞三十七,陶人二十七。諸職官居見任,雖有善政,不許立碑,已立而犯贓污者毀之,無治狀以虛譽立碑者毀之。



 諸夜禁,一更三點,鐘聲絕,禁人行。五更三點,鐘聲動,聽人行。違者笞二十七,有官者聽贖。其公務急速,及疾病死喪產育之類不禁。諸有司曉鐘未動,寺觀輒鳴鐘者,禁之。諸江南之地,每夜禁鐘以前,市井點燈買賣,曉鐘之後,人家點燈讀書工作者,並不禁。其集眾祠禱者,禁之。諸犯夜拒捕,斫傷徼巡者,杖一百七。



 諸城郭人民,鄰甲相保,門置水徹,積水常盈,家設火具,每物須備,大風時作,則傳呼以徇於路。有司不時點視,凡救火之具不備者,罪之。諸遺火延燒系官房舍,杖七十七;延燒民房舍,笞五十七;因致傷人命者,杖八十七;所毀房舍財畜,公私俱免征償。燒自己房舍者,笞二十七,止坐失火之人。諸煎鹽草地,輒縱野火延燒者,杖八十七,因致闕用者,奏取聖裁。鄰接管民官,專一關防禁治。諸縱火圍獵,延燒民房舍錢穀者,斷罪勒償,償未盡而會赦者,免征。諸故燒太子諸王房舍者,處死。諸故燒官府廨宇,及有人居止宅舍,無問舍宇大小,財物多寡,比同強盜,免刺,杖一百七,徒三年;因傷人命,同殺人。其無人居止空房,並損壞財物,及田場積聚之物,同竊盜,免刺,計贓斷罪。因盜取財物者,同強盜,刺斷,並追陪所燒物價;傷人命者,仍徵燒埋銀。再犯者決配,役滿,徙千里之外。諸挾仇放火,隨時撲滅,不曾延燎者,比強盜不曾傷人不得財,杖七十七,徒一年半,免刺,雖親屬相犯,比同常人。



 諸每月朔望二弦,凡有生之物,殺者禁之。諸郡縣歲正月五月,各禁宰殺十日,其饑饉去處,自朔日為始,禁殺三日。諸每歲,自十二月至來歲正月,殺母羊者,禁之。諸宴會,雖達官,殺馬為禮者,禁之。其有老病不任鞍勒者,亦必與眾驗而後殺之。諸私宰牛馬者,杖一百,徵鈔二十五兩,付告人充賞。兩鄰知而不首者,笞二十七。本管頭目失覺察者,笞五十七。有見殺不告,因脅取錢物者,杖七十七。若老病不任用者,從有司辨驗,方許宰殺。已病死者,申驗開剝,其筋角即付官,皮肉若不自用,須投稅貨賣,違者同匿稅法。有司禁治不嚴者,糾之。諸私宰官馬牛,為首杖一百七,為從八十七。諸助力私宰馬牛者,減正犯人二等論罪。諸牛馬驢騾死,而筋角不盡實輸官者,一副以上,笞二十七;五副以上,四十七;十副以上,杖六十七,仍徵所犯物價,付告人充賞。



 諸毀傷體膚以行丐於市者,禁之。諸城郭內外放鴿帶鈴者,禁之。諸諸王駙馬及諸權貴豪右,侵占山場,阻民樵採者,罪之。諸關譏不嚴,受財故縱者,罪之。諸江河津渡,或明知潮信已到,及風濤將起,貪索渡錢,淹延不渡,以致中流覆溺,傷害人命者,為首處死,為從減一等。



 諸棄俗出家,不從有司體覆,輒度為僧道者,其師笞五十七,受度者四十七,發元籍。諸以白衣善友為名,聚眾結社者,禁之。諸色目僧尼女冠,輒入民家強行抄化者,禁之。諸僧道偽造經文,犯上惑眾,為首者斬,為從者各以輕重論刑。諸以非理迎賽祈禱,惑眾亂民者,禁之。諸俗人集眾鳴鐃作佛事者,禁之。諸軍官鳩財聚眾,張設儀衛,鳴鑼擊鼓,迎賽神社,以為民倡者,笞五十七,其副二十七,並記過。諸陰陽家天文圖讖應禁之書,敢私藏者罪之。諸陰陽家偽造圖讖,釋老家私撰經文,凡以邪說左道誣民惑眾者,禁之,違者重罪之。在寺觀者,罪及主守,居外者,所在有司察之。諸妄言禁書者,徒。諸陰陽家者流,輒為人燃燈祭星,蠱惑人心者,禁之。諸妄言星變災祥,杖一百七。諸陰陽法師,輒入諸王公主駙馬家者,禁之。諸以陰陽相法書符咒水,凡異端之,惑亂人聽,希求仕進者,禁之,違者罪之。



 諸寫匿名文書,所言重者處死,輕者流,沒其妻子,與捕獲人充賞。事主自獲者不賞。諸寫匿名文字,訐人私罪,不涉官事者,杖七十七。諸投匿名文字於人家,脅取錢物者,杖八十七,發元籍。諸見匿名文書,非隨時敗獲者,即與燒毀;輒以聞官者,減犯人二等論罪。凡匿名文字,其言不及官府,止欲訐人罪者,如所訐論。



 諸民間子弟,不務生業,輒於城市坊鎮演唱詞話,教習雜戲,聚眾淫謔,並禁治之。諸弄禽蛇、傀儡,藏擫撇鈸、倒花錢、擊魚鼓,惑人集眾,以賣偽藥者,禁之,違者重罪之。諸棄本逐末,習用角牴之戲,學攻刺之者,師弟子並杖七十七。諸亂制詞曲為譏議者,流。



 諸賭博錢物,杖七十七,錢物沒官,有官者罷見任,期年後雜職內敘。開張博房之家,罪亦如之,再犯加徒一年。應捕故縱,笞四十七,受財者同罪。有司縱令攀指平人,及在前同賭人,罪及官吏。賭飲食者,不坐。諸賭博錢物,同賭之人自首者,勿論。諸賭博,因事發露,追到攤場,賭具贓證明白者,即以本法科論,不以展轉攀指革撥。



 諸故縱牛馬食踐田禾者,禁之。諸所在鎮守蒙古、漢軍,各立營所。無故輒入人家,求索酒食,及縱頭匹食踐田禾桑果,罪及主將。諸籓王無都省文書,輒於各處徵收差發,強取飲食草料,為民害者,禁之。



 諸有虎豹為害之處,有司嚴勒官兵及打捕之人,多方捕之。其中不應捕之人,自能設機捕獲者,皮肉不須納官,就以充賞。諸職官違例放鷹,追奪當日所服用鞍馬衣物沒官。諸所撥各官圍獵山場,並毋禁民樵採,違者治之。諸年穀不登,人民愁困,諸王達官應出圍獵者,並禁止之。諸田禾未收,毋縱圍獵,於迤北不耕種之地圍獵者聽。諸軍人受財,偽造火印,將所管官馬盜換與人者,杖九十七,追贓沒官。諸年穀不登,百姓饑乏,遇禁地野獸,搏而食之者,毋輒沒入。諸打捕鷹坊官,以合進御膳野物賣價自私者,計贓以枉法論,除名不敘。諸舟車之靡、器服之奇,方面大臣非錫貢不得擅進。



 諸闌遺人口到監,即移所稱籍貫,召主識認。半年之上無主識認者,匹配為戶,付有司當差。殘疾老病,給以文引,而縱遣之。頭匹有主識認者,徵還已有草料價錢,然後給主;無主識認,則籍其毛齒而收養之。諸闌遺奴婢,私相配合,雖生育子女,有主識認者,各歸其主,無本主者官與收系。諸隱藏闌遺鷹犬者,笞三十七,沒其家財之半。其收拾闌遺鷹犬之人,因以為民害者,罪之。



 諸鋤獲宿藏之物,在他人地內者,與地主中分,在官地內者一半納官,在己地內者即同業主。得古器珍寶之物者,聞官進獻,約量給價,若有詐偽隱匿,斷罪追沒。



 諸監臨官輒舉貸於民者,取與俱罪之。諸稱貸錢穀,年月雖多,不過一本一息,有輒取贏於人,或轉換契券,息上加息,或占人牛馬財產,奪人子女以為奴婢者,重加之罪,仍償多取之息,其本息沒官。諸典質,不設正庫,不立信帖,違例取息者,禁之。



 諸關廂店戶,居停各旅,非所知識,必問其所奉官府文引,但有可疑者,不得容止,違者罪之。諸官戶行錢商船,輒豎旗號,置弓箭鑼鼓,揭錢主衙門職名,往來江河者,禁之。諸經商及因事出外,必從有司會問鄰保,出給文引,違者究治。諸投下並其餘有印信衙門,並不得濫給文引。



 諸有毒之藥,非醫人輒相賣買,致傷人命者,買者賣者皆處死。不曾傷人者,各杖六十七,仍追至元鈔一百兩,與告人充賞。不通醫,制合偽藥,於市井貸賣者,禁之。



 諸下海使臣及舶商,輒以中國生口、寶貨、戎器、馬匹遺外番者,從廉訪司察之。諸商賈收買金銀下番者,禁之,違者罪之。諸海濱豪民,輒與番商交通貿易銅錢下海者,杖一百七。



 諸倡妓之家,所生男女,每季不過次月十日,會其數以上於中書省。有未生墮其胎、已生輒殘其命者,禁之。諸倡妓之家,輒買良人為倡,而有司不審,濫給公據,稅務無憑,輒與印稅,並嚴禁之,違者痛繩之。



 ○雜犯



 諸鬥爭折辨,輒提大名字者,罪之。諸職官因公失口亂言者,笞二十七。諸快意中或酒後及害風狂疾,失口亂言,別無情理者,免罪。



 諸惡少無賴,結聚朋黨,陵轢善良,故行鬥爭,相與羅織者,與木偶連鎖,巡行街衢,得後犯人代之,然後決遣。諸惡少白晝持刀劍於都市中,欲殺本部官長者,杖九十七。諸無賴軍人,輒受財毆人,因奪取錢物者,杖八十七,紅泥粉壁識過其門,免徒。諸先作過犯,曾經紅泥粉壁,後犯未應遷徙者,於元置紅泥粉壁添錄過名。



 諸豪右權移官府,威行鄉井,淫暴貪虐,累犯不悛者,徙遠惡之地屯種。諸頻犯過惡,累斷不改者,流遠。諸兇人殘害良善,強將男子去勢,絕滅人後,幸獲生免者,杖一百七,流遠。諸貴勢之家,奴隸有犯,輒私置鐵枷,釘項禁錮,及擅刺其面者,禁之。諸獲逃奴,輒刺面劓鼻,非理殘苦者,禁之。諸無故擅刺其奴者,杖六十七。諸囉哩、回回為民害者,從所在有司禁治。



 ○捕亡



 諸失盜,捕盜官不立限捕盜,卻令他戶陪償事主財物者,罰俸兩月,仍立限追捕。諸強盜殺人,三限不獲,會赦,捕盜官合得罪罰革撥,仍令捕盜,任滿不獲,解由內通行開寫,依例黜降。諸他境盜入境逃藏,捕盜官輒分彼疆此界,不即捕捉者,笞四十七,解職別敘,記過。



 諸已斷流囚,在禁未發,反獄毆傷禁子,已逃復獲者,處死;未出禁者杖一百七,發已擬流所。諸解發囚徒,經過州縣止宿,不寄收牢房,輒於逆旅監系,以致脫監在逃者,長押官笞二十七,還役;防送官四十七,記過。諸囚徒反獄而逃,主守減犯人罪二等,提牢官又減主守四等。隨時捉獲及半以上者,罰俸一月。



 諸奴婢背主而逃,杖七十七;誘引窩藏者,六十七。鄰人、社長、坊里正知不首捕者,笞三十七;關譏應捕人受贓脫放者,以枉法論。寺觀、軍營、勢家影蔽,及投下冒收為戶者,依藏匿論,自首者免罪。諸告獲逃奴者,於所將財物內,三分取一,付告獲人充賞。諸逃奴拒捕,不曾致傷人命者,仗一百七。



 ○恤刑



 諸獄囚,必輕重異處,男女異室,毋或參雜。司獄致其慎,獄卒去其虐,提牢官盡其誠。諸在禁囚徒,無親屬供給,或有親屬而貧不能給者,日給倉米一升,三升之中,給粟一升,以食有疾者。凡油炭席薦之屬,各以時具。其饑寒而衣糧不繼,疾患而醫療不時,致非理死損者,坐有司罪。諸各處司獄司看守囚徒,夜支清油一斤。諸路府州縣,但停囚去處,於鼠耗糧內放支囚糧。諸在禁無家屬囚徒,歲十二月至於正月,給羊皮為披蓋,褲襪及薪草為暖匣熏炕之用。諸獄訟,有必聽候歸對之人,召保知在,如無保識,有司給糧養濟,勿寄養於民家。諸流囚在路,有司日給米一升,有疾命良醫治之,疾愈隨時發遣。諸獄醫,囚之司命,必試而後用之,若有弗稱,坐掌醫及提調官之罪。諸獄囚病至二分,申報漸增至九分,為死證,若以重為輕,以急為緩,誤傷人命者,究之。諸獄囚有病,主司驗實,給醫藥,病重者去枷鎖杻,聽家人入侍。職事散官五品以上,聽二人入侍。犯惡逆以上,及強盜至死,奴婢殺主者,給醫藥而已。諸有司,在禁囚徒饑寒,衣食不時,病不督醫看候,不脫枷杻,不令親人入侍,一歲之內死至十人以上者,正官笞二十七,次官三十七,還職;首領官四十七,罷職別敘,記過。諸孕婦有罪,產後百日決遣,臨產之月,聽令召保,產後二十日,復追入禁。無保及犯死罪者,產時令婦人入侍。諸犯死罪,有親年七十以上,無兼丁侍養者,許陳請奏裁。諸有罪年七十以上、十五以下,及篤廢殘疾罰贖者,每笞杖一,罰中統鈔一貫。諸疑獄,在禁五年之上不能明者,遇赦釋免。



 ○平反



 諸官吏平反冤獄,應賞者,從有司保勘,廉訪司體覆,而後議之。其有冒濫不實者,罪及保勘體覆官吏。諸路府軍民長官,因收捕反叛,輒羅織平民,強奸室女,殺虜人口財產,並覆人之家,其同僚能理平民之冤,正犯人之罪,歸其俘虜,活其死命者,於本官上優升一等遷用。凡職官能平反重刑一起以上,升等同。諸職官能平反冤獄一起之上,與減一資。諸路府曹吏,能平反冤獄者,於各道宣慰司部令史補用。



\end{pinyinscope}