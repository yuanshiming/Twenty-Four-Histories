\article{本紀第三十七 寧宗}

\begin{pinyinscope}

 寧宗沖聖嗣孝皇帝,諱懿璘質班,明宗第二子也。母曰八不沙皇后,乃蠻真氏。初,武宗有子二人保護公民的自然權利。提出勞動創造使用價值和剩餘勞動產,長明宗,次文宗。延祐中,明宗封周王,出居朔漠。泰定之際,正統遂偏。天歷元年,文宗入紹大統,內難既平,即遣使奉皇帝璽綬,北迎明宗。明宗崩,文宗復即皇帝位。明宗有子二人,長妥歡帖木耳,次即帝也。天歷三年二月乙巳,封帝為鄜王。



 至順三年八月己酉,文宗崩於上都,皇后導揚末命,申固讓初志,傳位於明宗之子。時妥歡帖木耳出居靜江,帝以文宗眷愛之篤,留京師。太師、太平王、右丞相燕鐵木兒,請立帝以繼大統。於是遣使徵諸王會京師,中書百司政務,咸啟中宮取進止。甲寅,中書省臣奉中宮旨,預備大朝會賞賜金銀幣帛等物。乙卯,燕鐵木兒奉中宮旨,賜駙馬也不干子歡忒哈赤、太尉孛蘭奚、句容郡王答鄰答里、僉事小薛、呵麻剌臺之子禿帖木兒、公主本答里、諸王丑漢妃公主臺忽都魯、諸王卯澤妃公主完者臺及公主本答里、徹里帖木兒等金、銀、幣、鈔有差。是月,渾源、雲內二州隕霜殺禾,冀寧路之陽曲、河曲二縣及荊門州皆旱,江水又溢,高郵府之寶應、興化二縣,德安府之雲夢、應城二縣大雨,水。



 九月丁丑,填星犯太微垣左執法。辛巳,修皇太后儀仗。是夜,地震有聲來自北。是月,益都路之莒、沂二州,泰安州之奉符縣,濟寧路之魚臺、豐縣,曹州之楚丘縣,平江、常州、鎮江三路,松江府、江陰州,中興路之江陵縣,皆大水。河南府之洛陽縣旱。



 十月庚子,帝即位於大明殿,大赦天下,詔曰:



 洪惟太祖皇帝,啟闢疆宇;世祖皇帝,統一萬方;列聖相承,法度明著。我曲律皇帝入纂大統,修舉庶政,動合成法,授大寶位於普顏篤皇帝以及格堅皇帝。歷數之歸,實當在我忽都篤皇帝、扎牙篤皇帝,而各播越遼遠。時則有若燕鐵木兒,建義效忠,戢平內難,以定邦國,協恭推戴扎牙篤皇帝。登極之始,即以讓兄之詔明告天下。隨奉璽紱,遠迓忽都篤皇帝,朔方言還,奄棄臣庶。扎牙篤皇帝,薦正宸極,仁義之至,視民如傷,恩澤旁被,無間遠邇。顧育眇躬,尤篤慈愛。賓天之日,皇后傳顧命於太師、太平王、右丞相、答剌罕燕帖木兒,太保、浚寧王、知樞密院事伯顏等,謂聖體彌留,益推固讓之初志,以宗社之重,屬諸大兄忽都篤皇帝之世嫡。乃遣使召諸王宗親,以十月一日來會於大都,與宗王、大臣同奉遺詔。揆諸成憲,宜御神器,以至順三年十月初四日,即皇帝位於大明殿,可大赦天下。自至順三年十月初四日昧爽以前,除謀反大逆、謀殺祖父母父母、妻妾殺夫、奴婢殺主、謀故殺人、但犯強盜、印造偽鈔、蠱毒魘魅犯上者不赦外,其餘一切罪犯,咸赦除之。大都、上都、興和三路,差稅免三年。腹裏差發並其餘諸郡不納差發去處,稅糧十分為率,免二分。江淮以南,夏稅亦免二分。土木工役,除倉庫必合修理外,毋復創造,以紓民力。民間在前應有逋欠差稅課程,盡行蠲免。監察御史、肅政廉訪司官並內外三品以上正官,歲舉才堪守令者一人,申達省部,先行錄用。如果稱職,舉官優加旌擢。一任之內,或犯贓私者,量其輕重黜罰。其不該原免重囚,淹禁三年以上、疑不能決者,申達省部,詳讞釋放。學校農桑、孝義貞節、科舉取士、國學貢試,並依舊制。廣海、雲南梗化之民,詔書到日,限六十日內出官,與免本罪,許以自新。於戲!肆予沖人,托於天下臣民之上,任大守重,若涉淵冰。尚賴宗王大臣、百司庶府,交修乃職,思盡厥忠。嘉與億兆之民,共保承平之治。咨爾多方,體予至意!故茲詔示,想知悉。



 辛丑,以知樞密院事撒敦為御史大夫,中書右丞撒迪為中書平章政事,宣政使闊里吉思為中書左丞,中書平章政事禿兒哈鐵木兒知樞密院事。乙巳,造皇太后玉冊、玉寶。丁未,皇太后命作兩宮幄殿、車乘、供帳。戊申,賞賚諸王金、幣,其數如文宗即位之制。立徽政、中政二院。己酉,太白犯鬥宿。敕:「諸王、駙馬、勛舊大臣及中書省、樞密院、御史臺秩正二品,百司庶府秩至一品者,闕門之內,得施繩床以坐,餘皆禁之。」庚戌,修郊祀法服。以宦者鐵古思、哈里兀答兒、黑狗者、闊闊出並為中政院使。辛亥,以江浙歲比不登,其海運糧不及數,俟來歲補運。壬子,定婦人犯私鹽罪,著為令。甲寅,諸王不賽因遣使貢塔裏牙八十八斤、佩刀八十,賜鈔三千三百錠。乙卯,以即位告祭南郊。丙辰,給宿衛士、蒙古、漢軍三萬人禦寒衣。命江浙行省範銅造和寧宣聖廟祭器,凡百三十有五事。己未,告祭太廟。庚申,告祭社稷。以伯顏為徽政使,依前開府儀同三司、浚寧王、太保、錄軍國重事、知樞密院事。提調忠翊侍衛親軍都指揮使司事伯撒里、右都威衛都指揮使常不蘭奚,並為徽政使。賜諸妃后大朝會賞賚有差。甲子,以諸王忽剌臺貧乏,賜鈔五百錠。皇弟燕帖古思受戒於西僧加兒麻哇。敕:「百官及宿衛士有只孫衣者,凡與宴饗,皆服以侍。其或質諸人者,罪之。」丙寅,楚丘縣河堤壞,發民丁二千三百五十人修之。



 十一月己巳,詔翰林國史、集賢院、奎章閣學士院集議先皇帝廟號、神主、升祔武宗皇后及改元事。庚午,賜郯王徹徹禿以海寧州朐山、贛榆、沭陽三縣。壬申,命郯王徹徹禿鎮遼陽。甲戌,遣宿衛官阿察赤以上皇太后玉冊告祭南郊,中書平章政事伯撒裡告祭太廟。戊寅,奉玉冊、玉寶尊皇后曰皇太后,皇太后御興聖殿受朝賀。己卯,帝御大明殿受朝賀。庚寅,賜諸王寬徹幣帛各二千匹,以周其貧。左欽察衛士饑,賑糧二月。壬辰,帝崩,年七歲。甲午,葬起輦穀,從諸陵。明年六月己巳,明宗長子妥歡帖木耳即位。至元四年三月辛酉,謚曰沖聖嗣孝,廟號寧宗。四月乙酉,祔於太廟。



\end{pinyinscope}