\article{志第三下 五行二}

\begin{pinyinscope}

 ○水不潤下



 元統元年五月,汴梁陽武縣河溢害稼。六月,京畿大霖雨,水平地丈餘。涇河溢,關中水災。黃河大溢,河南水災。泉州霖雨,溪水暴漲,漂民居數百家。七月,潮州大水。二年正月,東平須城縣、濟寧濟州、曹州濟陰縣水災。二月,灤河、漆河溢,永平路屬縣皆水。瑞州路水。三月,山東霖雨,水湧。四月,東平、益都水。五月,鎮江路水,宣德府大水。六月,淮河漲,漂山陽縣境內民畜房舍。九月,吉安路水。至元元年,河決汴梁封丘縣。二年五月,南陽鄧州大水。六月,涇水溢。八月,大都至通州霖雨,大水。三年二月,紹興大水。五月,廣西賀州大水害稼。六月,衛輝淫雨至七月,丹、沁二河泛漲,與城西御河通流,平地深二丈餘,漂沒人民房舍田禾甚眾,民皆棲於樹木,郡守僧家奴以舟載飯食之,移老弱居城頭,日給糧餉,月餘水方退。汴梁蘭陽、尉氏二縣,歸德府皆河水泛溢。黃州及衢州常山縣皆大水。四年五月,吉安永豐縣大水。六月,邵武大水,城市皆洪流,漂沿溪民居殆盡。五年六月庚戌,汀州路長汀縣大水,平地深三丈許,損民居八百家,壞民田二百頃,溺死者八千餘人。七月,沂州沂、沭二河暴漲,決堤防,害田稼。邵武光澤縣大水。常州宜興縣山水出,勢高一丈,壞民居。六年二月,京畿五州十一縣及福州路福寧州大水。五月甲子,慶元奉化州山崩,水湧出平地,溺死人甚眾。六月,衢州西安、龍游二縣大水。庚戌,處州松陽、龍泉二縣積雨,水漲入城中,深丈餘,溺死五百餘人;遂昌縣尤甚,平地三丈餘。桃源鄉山崩,壓溺民居五十三家,死者三百六十餘人。七月壬子,延平南平縣淫雨,水泛漲,溺死百餘人,損民居三百餘家,壞民田二頃七十餘畝。乙卯,奉元路盩啡縣河水溢,漂溺居民。八月甲午,衛輝大水,漂民居一千餘家。十月,河南府宜陽縣大水,漂民居,溺死者眾。至正元年,汴梁鈞州大水,揚州路崇明、通、泰等州海潮湧溢,溺死一千六百餘人。二年四月,睢州儀封縣大水害稼。六月癸丑夜,濟南山水暴漲,沖東西二關,流入小清河,黑山、天麻、石固等寨及臥龍山水通流入大清河,漂沒上下民居千餘家,溺死者無算。三年二月,鞏昌寧遠、伏羌、成紀三縣山崩水湧,溺死者無算。五月,黃河決白茅口。七月,汴梁中牟、扶溝、尉氏、洧川四縣,鄭州滎陽、汜水、河陰三縣大水。四年五月,霸州大水。六月,河南鞏縣大雨,伊、洛水溢,漂民居數百家。濟寧路兗州,汴梁鄢陵、通許、陳留、臨潁等縣大水害稼,人相食。七月,灤河水溢,出平地丈餘,永平路禾稼廬舍漂沒甚眾。東平路東阿、陽谷、汶上、平陰四縣,衢州西安縣大水。溫州颶風大作,海水溢,漂民居,溺死者甚眾。五年七月,河決濟陰,漂官民亭舍殆盡。十月,黃河泛溢。七年五月,黃州大水。八月壬午,杭州、上海浦中午潮退而復至。八年正月辛亥,河決,陷濟寧路。四月,平江、松江大水。五月庚子,廣西山崩水湧,漓江溢,平地水深二丈餘,屋宇人畜漂沒。壬子,寶慶大水。乙卯,錢塘江潮比之八月中高數丈餘,沿江民皆遷居以避之。六月己丑,中興路松滋縣驟雨,水暴漲,平地深丈有五尺餘,漂沒六十餘里,死者一千五百人。是月,膠州大水。七月,高密縣大水。九年七月,中興路公安、石首、潛江、監利等縣及沔陽府大水。夏秋,蘄州大水傷稼。十年五月,龍興瑞州大水。六月乙未,霍州靈石縣雨水暴漲,決堤堰,漂民居甚眾。七月,汾州平遙縣汾水溢,靜江荔浦縣大水害稼。十一年夏,龍興南昌、新建二縣大水。安慶桐城縣雨水泛漲,花崖、龍源二山崩,沖決縣東大河,漂民居四百餘家。七月,冀寧路平晉、文水二縣大水,汾河泛溢東西兩岸,漂沒田禾數百頃。河決歸德府永城縣,壞黃陵岡岸。靜江路大水,決南北二陡渠。十二年六月,中興路松滋縣驟雨,水暴漲,漂民居千餘家,溺死七百人。七月,衢州西安縣大水。十三年夏,薊州豐潤、玉田、遵化、平谷四縣大水。七月丁卯,泉州海水日三潮。十四年六月,河南府鞏縣大雨,伊、洛水溢,漂沒民居,溺死三百餘人。秋,薊州大水。十五年六月,荊州大水。十六年,河決鄭州河陰縣,官署民居盡廢,遂成中流。山東大水。十七年六月,暑雨,漳河溢,廣平郡邑皆水。秋,薊州四縣皆大水。十八年秋,京師及薊州、廣東惠州、廣西四縣、賀州皆大水。十九年九月,濟州任城縣河決。二十年七月,通州大水。二十二年三月,邵武光澤縣大水。二十三年,孟州濟源、溫縣水。七月,河決東平壽張縣,圮城墻,漂屋廬,人溺死甚眾。二十四年三月,益都縣井水溢而黃。懷慶路孟州、河內、武陟縣水。七月,益都路壽光縣、膠州高密縣水。二十五年秋,薊州大水。東平須城、東阿、平陰三縣河決小流口,達於清河,壞民居,傷禾稼。二十六年二月,河北徙,上自東明、曹、濮,下及濟寧,皆被其害。六月,河南府大霖雨,籞水溢,深四丈許,漂東關居民數百家。秋七月,汾州介休縣汾水溢。薊州四縣、衛輝、汴梁鈞州大水害稼。



 八月,棣州大清河決,濱、棣二州之界,民居漂流無遺。濟寧路肥城縣西黃水泛溢,漂沒田禾民居百有餘里,德州齊河縣境七十里亦如之。



 至正二十年十一月,汴梁原武、滎澤二縣黃河清三日。二十一年十一月,河南孟津縣至絳州垣曲縣二百里河清七日,新安縣亦如之。十二月,冀寧路石州河水清,至明年春冰泮,始如故。二十四年夏,衛輝路黃河清。



 至正六年九月,彰德雨雪,結凍如琉璃。七年八月,衛輝隕霜殺稼。九年三月,溫州大雪。十年春,彰德大寒,近清明節,雨雪三尺,民多凍餒而死。十一年三月,汴梁路鈞州大雷雨雪,密縣平地雪深三尺餘。十三年秋,邵武光澤縣隕霜殺稼。二十三年三月,東平路須城、東河、陽穀三縣隕霜殺桑,廢蠶事。八月,鈞州密縣隕霜殺菽。二十七年三月,彰德大雪,寒甚於冬,民多凍死。五月辛巳,大同隕霜殺麥。秋,冀寧路徐溝、介休二縣雨雪。十二月,奉元路咸寧縣井水冰。二十八年四月,奉元隕霜殺菽。



 元統元年三月戊子,紹興蕭山縣大風雨雹,拔木僕屋,殺麻麥,斃傷人民。二年二月甲子,塞北東涼亭雨雹。至元元年七月,西和州、徽州雨雹。二年八月甲戌朔,高郵寶應縣大雨雹。是時,淮、浙皆旱,唯本縣瀕河,田禾可刈,悉為雹所害,凡田之旱者無一雹及之。四年四月癸巳,清州八里塘雨雹,大過於拳,其狀有如龜者,有如小兒形者,有如獅象者,有如環玦者,或橢如卵,或圓如彈,玲瓏有竅,色白而堅,長老云:「大者固常見之,未有奇狀若是也。」至正二年五月,東平路東阿縣雨雹,大者如馬首。三年六月,東平陽谷縣雨雹。六年二月辛未,興國路雨雹,大如馬首,小者如雞子,斃禽畜甚眾。五月辛卯,絳州雨雹,大者二尺餘。八年四月庚辰,鈞州密縣雨雹,大如雞子,傷麥禾。龍興奉新縣大雨雹,傷禾折木。八月己卯,益都臨淄縣雨雹,大如杯盂,野無青草,赤地如赭。九年二月,龍興大雨雹。十年五月,汾州平遙縣雨雹。十一年四月乙巳,彰德雨雹,大者如斧,時麥熟將刈,頃刻亡失,田疇堅如築場,無稭粒遺留者,地廣三十里,長百有餘里,樹木皆如斧所劈,傷行人、斃禽畜甚眾。五月癸丑,文水縣雨雹。十三年四月,益都高苑縣雨雹,傷麥禾及桑。十四年六月,薊州雨雹。十七年四月,濟南大風雨雹。十九年四月,莒州蒙陰縣雨雹。五月,通州及益都臨朐縣雨雹害稼。二十年五月,薊州遵化縣雨雹終日。二十一年五月,東平雨雹害稼。二十二年八月,南雄雨雹如桃李實。二十三年五月,鄜州宜君縣雨雹,大如雞子,損豆麥。七月,京師及隰州永和縣大雨雹害稼。二十五年五月,東昌聊城縣雨雹,大如拳,小者如雞子,二麥不登。二十六年六月,汾州平遙縣雨雹。二十七年二月乙丑,永州城中晝晦,雞棲於塒,人舉燈而食,既而大雨雹,逾時方明。五月,益都大雷雨雹。七月,冀寧徐溝縣大風雨雹,拔木害稼。二十八年六月,慶陽府雨雹,大如盂,小者如彈丸,平地厚尺餘,殺苗稼,斃禽獸。



 至正三年秋,興國路永興縣雷,擊死糧房貼書尹章於縣治。時方大旱,有硃書在其背云:「有旱卻言無旱,無災卻道有災,未庸殲厥渠魁,且擊庭前小吏。」七年五月庚戌,臺州路黃巖州海濱無雲而雷。冬,衛輝路天鼓鳴。十年六月戊申,廣西臨桂縣無雲而雷,震死邑民廖黃達。十二月庚子,汾州孝義縣雷雨。十一年十二月,臺州大雨震電。十二年三月丙午,寧國路無雲而雷。十三年十二月庚戌,京師無雲而雷,少頃有火墜於東南。懷慶路河內縣及河南府天鼓鳴於西北。是日懷慶之修武、潞州之襄垣縣皆無雲而雷,聲震天地。是月,汾州雷雨。十四年十二月,孝義縣雷雨。十九年十二月,臺州大雷電。二十一年十一月戊申,溫州樂清縣雷。二十七年正月乙未夜,晉寧路絳州天鼓鳴空中,如聞戰鬥之聲。十月,奉元路雷電。



 至正二十五年六月戊申,京師大雨,有魚隨雨而落,長尺許,人取而食之。



 至元五年六月庚戌,汀州長汀縣山蛟出,大雨驟至,平地湧水,深三丈餘,漂沒民居八百餘家,壞田二百餘頃。至正十七年六月癸酉,溫州有龍鬥於樂清江中,颶風大作,所至有光如球,死者萬餘人。八月癸丑,祥符縣西北有青白二龍見,若相鬥之勢,良久而散。二十三年正月甲辰,廣西貴州江中有物登岸,蛇首四足而青色,長四尺許,軍民聚觀而殺之。二十四年六月,保德州有黃龍見於咸寧井中。二十七年六月丁巳,皇太子寢殿新甃井成,有龍自井而出,光焰爍人,宮人震懾僕地。又宮墻外長慶寺所掌成宗斡耳朵內大槐樹,有龍纏繞其上,良久飛去,樹皮皆剝。七月,益都臨朐縣有龍見於龍山,巨石重千斤,浮空而起。二十八年十一月,大同路懷仁縣河岸崩,有蛇大小相綰結,可載數車。



 至正三年秋,建寧浦城縣民家豕生豚,二尾八足。十五年,鎮江民家豕生豚如象形。二十四年正月,保德州民家豕生豚,一首二身八蹄二尾。



 至元元年正月,廣西師宗州軿生妻適和,一產三男。汴梁祥符縣市中一乞丐婦人,忽生髭須。至正九年四月,棗陽民張氏婦生男,甫及周歲,長四尺許,容貌異常,皤腹擁腫,見人輒嬉笑,如世俗所畫布袋和尚云。二十三年五月,霸州民王馬駒妻趙氏,一產三男。六月,毫家務李閏妻張氏,一產三男。



 至正元年四月戊寅,彰德有赤風自西北來,忽變為黑,晝晦如夜。十三年冬,袁州路每日暮,有黑氣環繞郡城。十七年正月己丑,杭州降黑雨,河池水皆黑。二十八年七月乙亥,京師黑霧,昏暝不辨人物,自旦近午始消,如是者旬有五日。



 火不炎上



 元統元年六月甲申,杭州火。至正元年四月辛卯,臺州火。乙未,杭州火,燔官舍民居公廨寺觀,凡一萬五千七百餘間,死者七十有四人。二年四月,杭州又火。六年八月己巳,延平路火,燔官舍民居八百餘區,死者五人。十年,興國路自春及夏,城中火災不絕,日數十起。二十年,惠州路城中火災屢見。二十三年正月乙卯夜,廣西貴州火,同知州事韓帖木不花、判官高萬章及家人九口俱死焉,居民死者三百餘人,牛五十頭,馬九匹,公署、倉庫、案牘焚燒皆盡。二十八年二月癸卯,京師武器庫災。己巳,陜西有飛火自華山下,流入張良弼營中,焚兵庫器仗。六月甲寅,大都大聖壽萬安寺災。是日未時,雷雨中有火自空而下,其殿脊東鰲魚口火焰出,佛身上亦火起。帝聞之泣下,亟命百官救護,唯東西二影堂神主及寶玩器物得免,餘皆焚毀。此寺舊名白塔,自世祖以來,為百官習儀之所,其殿陛闌楯一如內庭之制。成宗時,置世祖影堂於殿之西,裕宗影堂於殿之東,月遣大臣致祭。



 至元六年冬,京師無雪。至正八年九月,奉元路桃杏花。十四年八月,冀寧路榆次縣桃李花。十五年十一月,汾州介休縣桃杏花。十七年十一月,汾州桃杏花。



 至正十一年十月,衢州東北雨米如黍。十一月,建寧浦城縣雨黑子如稗實;邵武大雨震電,雨黑黍如蘆穄,信州雨黑黍;鄱陽縣雨菽豆。郡邑多有,民皆取而食之。十六年六月,彰德路葦葉順次倚疊而生,自編成若旗幟,上尖葉聚粘如槍,民謠云:「葦生成旗,民皆流離;葦生成槍,殺伐遭殃。」又有黍自生成文,紅稭黑字,其上節云「天下太平」,其下節云「天下刀兵」。十八年,處州山谷中小竹結實如小麥,饑民採食之。二十一年,明州象山縣竹穗生實如小米,可食。



 至正十一年,廣西慶遠府有異禽雙飛,見於述昆鄉,飛鳥千百隨之,蓋鳳凰云。其一飛去,其一留止者,為僮人射死,首長尺許,毛羽五色,有藏之以獻於帥府者,久而其色鮮明如生云。五月,興國有大鳥百餘,飛至郡西白朗山顛,狀如人立,去而復至者數次。十九年,京師鴟鴞夜鳴達旦,連月乃止,有杜鵑啼於城中,居庸關亦如之。二十七年三月丁丑朔,萊州招遠縣大社裏黑風大起,有大鳥自南飛至,其色蒼白,展翅如席,狀類鶴,俄頃飛去,遺下粟、黍、稻、麥、黃黑豆、蕎麥於張家屋上,約數升許,是歲大稔。



 元統二年正月庚寅朔,河南省雨血。是日眾官晨集,忽聞燔柴煙氣,既而黑霧四塞,咫尺不辨,腥穢逼人,逾時方息。及行禮畢,日過午,驟雨隨至,沾灑堊墻及裳衣皆赤。至元四年四月辛未,京師雨紅沙,晝晦。至正五年四月,鎮江丹陽縣雨紅霧,草木葉及行人裳衣皆濡成紅色。十三年三月丙戌,彰德路西南,有火自天而下,如在城外,覓之無有。十二月庚戌,潞州襄垣縣有火墜於東南。十四年,衛輝路有天光見於西方。十二月辛卯,絳州有紅氣,起自北方,蔽天幾半,移時方散。十五年春,薊州雨血。十八年三月辛丑夜,大同路有黑氣蔽於西方,聲如雷然。俄頃,有雲如火,交射中天,遍地俱見火光,以物觸地,輒有火起,至夜半,空中如有兵戈相擊之聲。二十一年七月己巳,冀寧路忻州西北,有赤氣蔽空如血,逾時方散。八月壬午,棣州夜半有赤氣亙天,起西北至於東北。癸未,彰德西北,夜有紅氣亙天,至明方息。乙酉,大同路北方,夜有赤氣蔽天,直過天庭,自東而西,移時方散,如是者三。十月癸巳昧爽,絳州有紅氣見於北方,如火。二十三年三月壬戌,大同路夜有赤氣亙天,中侵北斗。六月丁巳,絳州日暮有紅光見於北方,如火,中有黑氣相雜,又有白虹二,直沖北斗,逾時方散。庚申,晉寧路北方,日暮天赤,中有白氣如虹者三,一貫北斗,一貫北極,一貫天潢,至夜分方滅。八月丙辰,忻州東北,夜有赤氣亙天,中有白色如蛇形,徐徐而行,逾時方散。



 十月丙申朔,大名路向青、齊一方,有赤氣照耀千里。二十四年九月癸酉,冀寧平晉縣西北方,至夜天紅半壁,有頃,從東而散。二十八年六月壬寅,彰德路天寧寺塔忽變紅色,自頂至踵,表裏透徹,如煆鐵初出於爐,頂上有光焰迸發,自二更至五更乃止。癸卯、甲辰,亦如之。先是,河北有童謠云:「塔兒黑,北人作主南人客;塔兒紅,硃衣人作主人公。」七月癸酉,京師赤氣滿天,如火照人,自寅至辰,氣焰方息。



 至元元年十二月,芝草生於荊門州當陽縣覆船山,一本五干,高尺有二寸,一本二干,高五寸有半,干皆兩岐;二本相依附,扶疏瑰奇,如珊瑚枝,其高者結為華蓋慶雲之狀。五年秋,芝草生於中書工部之屋梁,一本七幹。



 木不曲直



 至元五年十一月癸酉,瑞州路新昌州雨木冰,至明年二月壬寅冰始解。至正四年正月,汴梁路鄭州尉氏、洧川、河陰三縣及龍興靖安縣雨木冰。十一月,東平雨木冰。十二年九月壬午,冀寧保德州雨木冰。十四年冬,龍興雨木冰。二十五年二月辛亥,汴梁雨木冰,狀如樓閣、人物、冠帶、鳥獸、花卉,百態具備,羽幢珠葆,彌望不絕,凡五日始解。



 至正三年夏,上都、大都桑果葉,皆有黃色龍文。九年秋,奉元桃杏實。十二年五月,汴梁祥符縣椿樹結實如木瓜。十六年七月,彰德李樹結實如小黃瓜。民謠云:「李生黃瓜,民皆無家。」二十一年,明州松樹結實,其大有盈尺者。八月,汴梁祥符縣邑中樹木,一夕皆有濕泥塗之。



 至元二年五月乙卯,南陽鄧州大霖雨,自是日至於六月甲申乃止。三年六月,衛輝路淫雨。至正二年秋,彰德路霖雨。三年四月至七月,汴梁路滎澤縣,鈞州新鄭、密縣霖雨害稼。四年夏,汴梁蘭陽縣,許州長葛、郾城、襄城,睢州,歸德府亳州之鹿邑,濟寧之虞城淫雨害蠶麥,禾皆不登。八月,益都霖雨,饑民有相食者。五年夏秋,汴梁祥符、尉氏、洧川,鄭州、鈞州、亳州久雨害稼,二麥禾豆俱不登。河間路淫雨,妨害鹽課。八年五月,京師大霖雨,都城崩圮。鈞州新鄭縣淫雨害麥。九年七月,高唐州大霖雨,壞官署民居。歸德府淫雨浹十旬。十年二月,彰德路大雨害麥。二十年七月,益都高苑縣、陜州黽池縣大雨害稼。二十三年七月,懷慶路河內、修武、武陟三縣及孟州淫雨害稼。二十四年秋,密州安丘縣大雨。二十五年秋,密州安丘縣,潞州,汴梁許州及鈞州之密縣淫雨害稼。二十七年秋,彰德路淫雨。



 至正六年八月,龍興進賢縣甘露降。二十年十月,國子學大成殿松柏樹有甘露降其上。



 至正十年春,麗正門樓斗栱內,有人伏其中,不知何自而至,遠近聚觀之。門尉以白留守,達於都堂,上聞,有旨令取付法司鞫問。但云薊州人,問其姓名,詰其所從來,皆惘若無知,唯妄言禍福而已,乃以不應之罪笞之,忽不知所在。



 至正二十年八月,慶陽,延安,寧、安等州野鼠食稼,初由鶉卵化生,既成牝牡,生育日滋,百畝之田,一夕俱盡。二十六年,泗州瀕淮兩岸,有灰黑色鼠,暮夜出穴,成群覆地食禾。



 金不從革



 至正十年正月甲戌,棣州白晝空中有聲自西北而來,距州二十里隕於地,化為石,其色黑,微有金星散布其上。有司以進,遂藏之司天監。十一月冬至夜,陜西耀州有星墜於西原,光耀燭地,聲如雷鳴者三,化為石,形如斧,一面如鐵,一面如錫,削之有屑,擊之有聲。十六年冬十一月,大名路大名縣有星如火,自東南流,尾如曳篲,墜入於地,化為石,青黑光瑩,狀如狗頭,其斷處類新割者。有司以進,太史驗視云「天狗也」,命藏於庫。十九年四月己丑,建寧路甌寧縣有星墜於營山前,其聲如雷,化為石。二十三年六月庚戌,益都臨朐縣龍山有星墜入於地,掘之深五尺,得石如磚,褐色,上有星如銀,破碎不完。



 至正九年,龍興靖安縣山石迸裂,湧水,人多死者。十年三月,慶元奉化州南山石突開,其碎而大者,有山川人物禽鳥草木之文。二十七年六月丁卯,沂州東蒼山有巨石,大如屋,崩裂墜地,聲震如雷。七月丙戌,廣西靈川縣臨江石崖崩。



 元統元年夏,紹興旱,自四月不雨至於七月。淮東、淮西皆旱。二年三月,湖廣旱,自是月不雨至於八月。四月,河南旱,自是月不雨至於八月。秋,南康旱。至元元年夏,河南及邵武大旱。二年,蘄州、黃州、浙東衢州、婺州、紹興、江東信州、江西瑞州等路及陜西皆旱。是年四月,黃州黃岡縣周氏婦產一男即死,狗頭人身,咸以為旱魃云。六年夏,廣東南雄路旱,自二月不雨至於五月,種不入土。至正二年,彰德、大同二郡及冀寧平晉、榆次、徐溝縣,汾州孝義縣,忻州皆大旱,自春至秋不雨,人有相食者。秋,衛輝大旱。三年秋,興國大旱。四年,福州大旱,自三月不雨至於八月。興化、邵武、鎮江及湖南之桂陽皆旱。五年,曹州禹城縣大旱。夏,膠州高密縣旱。六年,鎮江及慶元奉化州旱。七年,懷慶、衛輝、河東及鳳翔之岐山、汴梁之祥符、河南之孟津皆大旱。八年三月,益都臨淄縣大旱。五月,四川旱。十年夏秋,彰德旱。十一年,鎮江旱。十二年,蘄州、黃州大旱,人相食。浙東紹興旱。臺州自四月不雨至於七月。十三年,蘄州、黃州及浙東慶元、衢州、婺州,江東饒州,江西龍興、瑞州、建昌、吉安,廣東南雄,湖南永州、桂陽皆大旱。十四年,懷慶河內縣、孟州,汴梁祥符縣,福建泉州,湖南永州、寶慶,廣西梧州皆大旱。祥符旱魃再見,泉州種不入土,人相食。十五年,衛輝大旱。十六年,婺州、處州皆大旱。十八年春,薊州旱。莒州、濱州、般陽淄川縣、霍州、鄜州、鳳翔岐山縣春夏皆大旱。莒州家人自相食,岐山人相食。十九年,晉寧、鳳翔,廣西梧州、象州皆大旱。二十年,通州旱。汾州介休縣自四月至秋不雨。廣西賓州大旱,自閏五月不雨至於八月。二十二年,河南洛陽、孟津、偃師三縣大旱,人相食。二十三年,山東濟南、廣西賀州皆大旱。



 至元五年八月,京師童謠云:「白雁望南飛,馬札望北跳。」至正五年,淮、楚間童謠云:「富漢莫起樓,窮漢莫起屋,但看羊兒年,便是吳家國。」十年,河南、北童謠云:「石人一隻眼,挑動黃河天下反。」十五年,京師童謠云:「一陣黃風一陣沙,千里萬里無人家,回頭雪消不堪看,三眼和尚弄瞎馬。」此皆詩妖也。至元三年,郡邑皆相傳朝廷欲括童男女,於是市井鄉里競相嫁娶,倉卒成言,貧富長幼多不得其宜者,此民訛也。



 至正十年,彰德境內狼狽為害,夜如人形,入人家哭,就人懷抱中取小兒食之。二十三年正月,福州連江縣有虎入於縣治。二十四年七月,福州白晝獲虎於城西。



 至元二年七月,黃州蝗。三年六月,懷慶、溫州、汴梁陽武縣蝗。五年七月,膠州即墨縣蝗。至正四年,歸德府永城縣及亳州蝗。十七年,東昌茌平縣蝗。十八年夏,薊州、遼州、濰州昌邑縣、膠州高密縣蝗。秋,大都、廣平、順德及濰州之北海、莒州之蒙陰、汴梁之陳留、歸德之永城皆蝗。順德九縣民食蝗,廣平人相食。十九年,大都霸州、通州,真定,彰德,懷慶,東昌,衛輝,河間之臨邑,東平之須城、東阿、陽穀三縣,山東益都、臨淄二縣,濰州、膠州、博興州,大同、冀寧二郡,文水、榆次、壽陽、徐溝四縣,沂、汾二州,及孝義、平遙、介休三縣,晉寧潞州及壺關、潞城、襄垣三縣,霍州趙城、靈石二縣,隰之永和,沁之武鄉,遼之榆社、奉元,及汴梁之祥符、原武、鄢陵、扶溝、杞、尉氏、洧川七縣,鄭之滎陽、汜水,許之長葛、郾城、襄城、臨潁,鈞之新鄭、密縣,皆蝗,食禾稼草木俱盡,所至蔽日,礙人馬不能行,填坑塹皆盈。饑民捕蝗以為食,或曝乾而積之。又罄,則人相食。七月,淮安清河縣飛蝗蔽天,自西北來,凡經七日,禾稼俱盡。二十年,益都臨朐、壽光二縣,鳳翔岐山縣蝗。二十一年六月,河南鞏縣蝗,食稼俱盡。七月,衛輝及汴梁滎澤縣、鄭州蝗。二十二年秋,衛輝及汴梁開封、扶溝、洧川三縣,許州及鈞之新鄭、密二縣蝗。二十五年,鳳翔岐山縣蝗。



 元統二年六月,彰德雨白毛,俗呼云「老君髯」。民謠曰:「天雨氂,事不齊。」至元三年三月,彰德雨毛,如線而綠,俗呼云「菩薩線」。民謠云:「天雨線,民起怨,中原地,事必變。」六年七月,延安路鄜州雨白毛,如馬鬃,所屬邑亦如之。至正十三年四月,冀寧榆次縣雨白毛,如馬鬃。七月,泉州路雨白絲。十八年五月,益都雨白氂。十九年三月,興化路連日雨氂。二十五年五月甲子,京師雨氂,長尺許,如馬鬃。二十七年五月,益都雨白氂。



 至元四年八月丁丑,京師白虹亙天。至正二十二年,京師有白氣如小索,起危宿,長五百丈,掃太微。二十四年六月癸卯,冀寧路保德州三星晝見,有白氣橫突其中。二十六年三月丁亥,白虹五道亙天,其第三道貫日。又氣橫貫東南,良久乃滅。二十七年五月,大名路有白氣二道。二十八年閏七月乙丑,冀寧文水縣有白虹貫日,自東北直繞西南,雲影中似日非日,如鏡者三,色青白,逾時方沒。



 稼穡不成



 元統元年夏,兩淮大饑。二年春,淮西饑。七月,池州饑。十一月,濟南、萊蕪縣饑。至元元年春,益者路沂水、日照、蒙陰、莒四縣及龍興路饑。夏,京師饑。是歲,沅州、道州、寶慶及邵武、建寧饑。二年,順州及淮西安豐,浙西松江,浙東臺州,江西江、撫、袁、瑞,湖北沅州盧陽縣饑。三年,大都及濟南、蘄州、杭州、平江、紹興、溧陽、瑞州、臨江饑。五年,上都開平縣、桓州,興和寶昌州,濮州之鄄城,冀寧之交城,益都之膠、密、莒、濰四州,遼東沈陽路,湖南衡州,江西袁州,八番順元等處皆饑。六年,順德之邢臺,濟南之歷城,大名之元城,德州之清平,泰安之奉符、長清,淮安之山陽等縣,歸德邳州,益都、般陽、處州、婺州四郡皆饑。至正元年春,京畿州縣、真定、河間、濟南及湖南饑。夏,彰德及溫州饑。二年,保德州大饑。三年,衛輝、冀寧、忻州大饑,人相食。四年,霸州大饑,人相食。東平路東阿、陽谷、汶上、平陰四縣皆大饑。冬,保定、河南饑。五年春,東平路須城、東阿、陽穀三縣及徐州大饑,人相食。夏,濟南、汴梁、河南、邠州、瑞州、溫州、邵武饑。六年五月,陜西饑。七年,彰德、懷慶、東平、東昌、晉寧等處饑。九年春,膠州大饑,人相食。鈞州新鄭、密縣饑。十四年春,浙東臺州,江東饒,閩海福州、邵武、汀州,江西龍興、建昌、吉安、臨江,廣西靜江等郡皆大饑,人相食。十七年,河南大饑。十八年春,莒州蒙陰縣大饑,斗米金一斤。冬,京師大饑,人相食,彰德、山東亦如之。十九年正月至五月,京師大饑,銀一錠得米僅八斗,死者無算。通州民劉五殺其子而食之。保定路莩死盈道,軍士掠孱弱以為食。濟南及益都之高苑,莒之蒙陰,河南之孟津、新安、黽池等縣皆大饑,人相食。二十一年,霸州饑,民多莩死。



 至正四年,福州、邵武、延平、汀州四郡,夏秋大疫。五年春夏,濟南大疫。十二年正月,冀寧保德州大疫。夏,龍興大疫。十三年,黃州、饒州大疫。十二月,大同路大疫。十六年春,河南大疫。十七年六月,莒州蒙陰縣大疫。十八年夏,汾州大疫。十九年春夏,鄜州並原縣,莒州沂水、日照二縣及廣東南雄路大疫。二十年夏,紹興山陰、會稽二縣大疫。二十二年,又大疫。



 至正元年七月,廣西雷州颶風大作,湧潮水,拔木害稼。二年十月,海州颶風作,海水漲,溺死人民。十三年五月乙丑,潯州颶風大作,壞官舍民居,屋瓦門扉皆飄揚七里之外。十四年七月甲子,潞州襄垣縣大風拔木偃禾。二十一年正月癸酉,石州大風拔木,六畜皆鳴,人持槍矛,忽生火焰,抹之即無,搖之即有。二十四年,臺州路黃巖州海溢,颶風拔木,禾盡偃。二十七年三月庚子,京師有大風,起自西北,飛砂揚礫,昏塵蔽天,逾時,風勢八面俱至,終夜不止,如是者連日。自後,每日寅時風起,萬竅爭鳴,戌時方息,至五月癸未乃止。



 至正三年六月,梧州青蟲食稼。十年七月,同州蟲食稼,郡守石亨祖禱於玄妙觀,寒雨三日,蟲盡死。十九年五月,濟南章丘、鄒平二縣蝻,五穀不登。二十二年春,衛輝路螟。六月,萊州膠水縣虸蚄生。七月,掖縣虸蚄生,害稼。二十三年六月,寧海文登縣虸蚄生。七月,萊州招遠、萊陽二縣及登州、寧海州虸蚄生。



 至正九年三月,陳州楊家莊上牛生黃犢,火光滿室,麻頂綠角,間生綠毛,不食乳,二日而死。十年秋,襄陽車城民家牛生犢,五足,前三後二。十六年春,汴梁祥符縣牛生犢,雙首,不及二日死。二十八年五月,東昌聊城縣錢鎮撫家牛生黃犢,六足,前二後四。



 至元五年二月,信州雨土。至正三年三月至四月,忻州風霾盡晦。二十六年四月乙丑,奉元路黃霧四塞。



 元統元年八月,鞏昌、徽州山崩。九月庚申,秦州山崩。十月丙寅,鳳州山崩。十一月丙申,鞏昌成紀縣地裂山崩。癸卯,安慶灊山縣地震。辛亥,秦州地裂山崩。十二月,饒州德興縣,餘干、樂平二州地震。二年五月,信州地震。八月辛未,京師地震。雞鳴山崩,陷為池,方百里,人死者眾。至元元年十一月壬寅,興國路地震。十二月丙子,安慶路地震,所屬宿松、太湖、灊山三縣同時俱震。廬州、蘄州、黃州亦如之。是月,饒州亦地震。二年正月乙丑,宿松地震。五月壬申,秦州山崩。三年八月辛巳夜,京師地震。壬午,又大震,損太廟神主;西湖寺神御殿壁僕,祭器皆壞。順州、龍慶州及懷來縣皆以辛巳夜地震,壞官民房舍,傷人及畜牧。宣德府亦如之,遂改為順寧云。四年春,保安州及瑞州路新昌州地震。六月,信州路靈山裂。七月己酉,保安州地大震。丙辰,鞏昌府山崩。八月丙子,京師地震,日凡二三,至乙酉乃止。密州安丘縣地震。六年六月己亥,秦州成紀縣山崩地裂。至正元年二月,汴梁路地震。二年四月辛丑,冀寧路平晉縣地震,聲如雷鳴,裂地尺餘,民居皆傾僕。七月,惠州雨水,羅浮山崩,凡二十七處,壞民居,塞田澗。十二月己酉,京師地震。三年二月,鈞州新鄭、密縣地震。六月乙巳,秦州秦安縣南坡崩裂,壓死人畜。七月戊辰,鞏昌山崩,人畜死者眾。十二月,膠州及屬邑高密地震。四年八月,莒州蒙陰縣地震。十二月,東平路東阿、陽谷、平陰三縣及漢陽地震。五年春,薊州地震,所領四縣及東平汶上縣亦如之。十二月乙丑,鎮江地震。六年二月,益都路益都、昌樂、壽光三縣,濰州北海縣、膠州即墨縣地震。三月,高苑縣地震,壞民居。六月,廣州增城縣羅浮山崩,水湧溢,溺死百餘人。九月戊午,邵武地震。翌日,地中有聲如鼓,夜復如之。七年二月,益都臨淄、臨朐,濰州之昌邑、膠州之高密、濟南之棣州地震。三月,東平路東阿、陽谷、平陰三縣地震,河水動搖。五月,臨淄地又震,七日乃止。河東地坼泉湧,崩城陷屋,傷人民。十一月,鎮江丹陽縣地震。九年六月,臺州地震。七月庚寅,泉州大風雨。永春縣南象山崩,壓死者甚眾。十年,冀寧徐溝縣地震。五月甲子,龍興寧州大雨,山崩數十處。丙寅,瑞州上高縣蒙山崩。十月乙酉,泉州安溪縣侯山鳴。十一年四月,冀寧路汾、忻二州,文水、平晉、榆次、壽陽四縣,晉寧遼州之榆社,懷慶河內、修武二縣及孟州皆地震,聲如雷霆,圮房屋,壓死者甚眾。八月丁丑,中興路公安、松滋、枝江三縣,峽、荊門二州地震。十二年二月丙戌,霍州靈石縣地震。閏三月丁丑,陜西地震,莊浪、定西、靜寧、會州尤甚,移山湮谷,陷沒廬舍,有不見其跡者。會州公廨墻圮,得弩五百餘張,長丈餘,短者九尺,人莫能開挽。十月丙午,霍州趙城縣霍山崩,湧石數里,前三日,山鳴如雷,禽獸驚散。十三年三月,莊浪、定西、靜寧、會州地震。七月,汾州白彪山坼。十四年四月,汾州介休縣地震,泉湧。七月,孝義縣地震。十一月,寧國路地震,所領寧國、旌德二縣亦如之。淮安路海州地震。十二月己酉,紹興地震。十五年四月,寧國敬亭、麻姑、華陽諸山崩。六月丁丑,冀寧保德州地震。十六年春,薊州地震,凡十日,所領四縣亦如之。六月,雷州地大震。十七年十月,靜江路東門地陷,城東石山崩。十二月丁酉,慶元路象山縣鵝鼻山崩,有聲如雷。十八年二月乙亥,冀寧臨州地震。五月,益都地震。十九年正月甲午,慶元地震。二十年二月,延平順昌縣地震。二十二年三月,南雄路地震。二十三年十二月丁巳,臺州地震。二十五年十月壬申,興化路地震,有聲如雷。二十六年三月,海州地震如雷,贛榆縣吳山崩。六月,汾州介休縣地震。紹興山陰縣臥龍山裂。七月辛亥,冀寧路徐溝縣,石、忻、臨三州,汾之孝義、平遙二縣同日地震,有壓死者。丙辰,泉州同安縣大雷雨,三秀山崩。是月,河南府鞏縣大霖雨,地震山崩。十一月辛丑,華州蒲城縣洛岸崩,壅水,絕流三日。十二月庚午,華州之蒲城縣洛水和順崖崩,其崖戴石,有巖穴可居,是日壓死闢亂者七十餘人。二十七年五月,山東地震。六月,沂州山石崩裂,有聲如雷。七月丙戌,靜江靈川縣大藏山石崖崩。十月丙辰,福州雷雨,地震。十二月庚午,又震,有聲如雷。二十八年六月,冀寧文水、徐溝二縣,汾州孝義、介休二縣,臨州、保德州,隰之石樓縣及陜西皆地震。十月辛巳,陜西地又震。



 至元四年五月,彰德臨彰縣麥秀兩岐,有三穗者。至正元年,延平順昌縣嘉禾生,一莖五穗。冀寧太原縣有嘉禾,異畝同穎。三年八月,晉寧臨汾縣嘉禾生,有五穗至八穗者。十年,彰德路穀麥雙穗。十六年,大同路秦城鄉嘉禾生,一莖二穗五穗,有九穗者,有異莖而同穗者。二十六年五月,洛陽縣康家莊有瑞麥,一莖四穗雙穗三穗者甚眾。



\end{pinyinscope}