\article{志第四十一下 百官八}

\begin{pinyinscope}

 元之官制,其大要具見於前,自元統、至元以來,頗有沿革增損之異。至正兵興,四郊多壘,中書、樞密,俱有分省、分院;而行中書省、行樞密院增置之外,亦有分省、分院。自省院以及郡縣,又各有添設之員。而各處總兵官以便宜行事者,承制擬授,具姓名以軍功奏聞,則宣命敕牒隨所索而給之,無有考核其實者。於是名爵日濫,紀綱日紊,疆宇日蹙,而遂至於亡矣。惜其掌故之文,缺軼不完,令據有司所送上者,緝而載之,以附前志,庶覽者得以參考其得失治亂之概云。



 中書省。元統三年七月,中書省奏請自今不置左丞相。十月,命伯顏獨長臺司,詔天下。至元五年十月,加右丞相伯顏為大丞相。六年十月,命脫脫為右丞相,復置左丞相。至正七年,置議事平章四人。十二年二月,以賈魯為添設左丞。三月,以悟良哈臺為添設參知政事。七月,又以杜秉彞為添設參政。八月,以哈麻為添設右丞。十三年六月,命皇太子領中書令,如舊制。十四年九月,以呂思誠為添設左丞。二十七年八月,以樞密知院蠻子為添設第三平章,以太尉帖裏帖木兒為添設左丞相。



 中書分省。至正十一年,置中書分省於濟寧,以松壽為參知政事。十二年二月,中書右丞玉樞虎兒吐華、左丞韓大雅開分省於彰德。十四年,升濟寧分省參政帖裏帖木兒為平章政事,是後嘗置右丞以守御焉。十五年四月,彰德分省除右丞、左丞各一員。十七年七月,以平章答蘭,參政俺普、崔敬分省陵州。十一月,平章臧卜分省冀寧。十八年三月,掃地王、沙劉陷冀寧,臧卜遁。五月,王、劉北行,總兵官察罕帖木兒遣瑣住院判來冀寧鎮守,臧卜復回。十九年,臧卜卒。二十年正月,以右丞不花、參政王時分省冀寧。三月,鐵甲韓至,分省官皆遁。二十一年,以平章答蘭鎮守。二十二年,答蘭還京師,以左丞剌馬乞剌、參政脫禾兒領分省事。二十三年三月,又以平章愛不花鎮之。八月,擴廓帖木兒兵至,冀寧分省遂罷。二十七年八月,以添設平章蠻子兼知院,分省保定。九月,命太保、右丞相也速統領軍馬,分省山東;沙藍答里仍中書左丞相、知樞密院,分省大同。以哈剌那海為大同分省平章,阿剌不花為參知政事。又置分省於冀寧,升冀寧總管為參政,鑄印與之,凡事必咨大同分省而後行之。十月,又置分省於真定。



 六部。至元三年十二月,伯顏太師等奏準,吏部考功郎中、員外郎、主事各設一員。至正元年四月,吏部置司績一員,正七品,掌百官行止,以憑敘用廕襲。六月,中書奏準,戶部事繁,見設司計四員,宜依前至元二十八年例,添設二員。十一月,吏、禮、兵、刑分為二庫,戶、工二部分二庫,各設管勾一員。十二年正月,刑部添設尚書、侍郎、郎中、員外郎各一員。十五年十月,濟寧分省置兵、刑、工、戶四部。



 樞密院。至正七年,知樞密院阿吉剌奏:「樞密院故事,亦設議事平章二人。」有旨令復置。十三年六月,令皇太子領樞密使,如舊制。十五年四月,添設僉院一員、院判二員。



 樞密分院。至正十五年三月,置樞密分院於衛輝。四月,彰德分院添設同知、副樞各一員,都事一員。直沽分院添設副樞一員、都事一員。十六年,又置分樞密院於沂州,以指揮使司隸焉。



 大宗正府。至元元年閏十二月,中書省奏準,世祖時立大宗正府,至仁宗時減去大字,今宜遵世祖舊制,仍為大宗正府。至正十年十二月,大宗正府添設掌判二員。



 宣文閣。至元六年十一月,罷奎章閣學士院。至正元年九月,立宣文閣,不置學士,唯授經郎及監書博士以宣文閣系銜雲。



 崇文監。至元六年十二月,改藝文監為崇文監。至正元年三月,奉旨,令翰林國史院領之。



 詳定使司。至正十七年七月,置四方獻言詳定使司,正三品,掌考其所陳之言,擇其善者以聞於上而舉行之。詳定使二員,正三品;副使二員,正四品;掌書記二員,正七品。中書官提調之。



 司禋監。至正元年十二月,奉旨,依世祖故事,復立司禋監,給四品印,掌師翁祭祀祈禳之事。置內監、少監、監丞各二員,知事一員,譯史、令史、奏差各二名。自後復升為三品。



 延徽寺。至元六年二月,中書省奉旨,依累朝故事,起蓋懿璘質班皇帝斡耳朵,置延徽寺以掌之。



 規運提點所。至元六年十一月,罷太禧宗禋院隆祥使司。十二月,中書奏以宗禋院所轄會福、崇祥、隆禧、壽福四總管府,並隆祥使司,俱改為規運提點所,正五品,仍添置萬寧提點所一處,並隸宣政院。



 諸路寶泉都提舉司,至正十年十月置。其屬有鼓鑄局,正七品;永利庫,從七品。掌鼓鑄至正銅錢,印造交鈔。



 徽政院。元統元年十二月,依太皇太后故事,為皇太后置徽政院,設立官屬三百六十有六員。



 資正院。至元六年十二月,中書省奉旨為完者忽都皇后置資正院,正二品。院使六員,同知、僉院、同僉、院判各二員。首領官:經歷、都事各二員,管勾、照磨各一員。將昭功萬戶府司屬,除已罷繕工司外,集慶路錢糧並入,有司每年驗數,撥付資正院。其餘司屬,並付資正院領之。自後正宮皇后崩,冊立完者忽都為皇后,改置崇政院。



 東宮官屬。至正六年四月,立皇太子宮傅府,以長吉等為宮傅官,時太子猶未受冊寶。至九年冬,立端本堂為皇太子學宮。置諭德一員,正二品;贊善二員,正三品;文學二員,正五品;正字二員,正七品;司經二員,正七品。十三年六月,冊立皇太子,定置皇太子賓客二員,正二品;左、右諭德各一員,從二品;左、右贊善各一員,從三品;文學二員,從五品;中庶子、中允各一員,從六品。



 詹事院。至正十三年六月,立詹事院,罷宮傅府。置詹事三員,從一品;同知詹事二員,正二品;副詹事二員,從二品;詹事丞二員,正三品;首領官四員,中議二員,從五品;長史二員,從六品;管勾、照磨各一員,正八品;蒙古必霝赤六人,回回掾史二人,掾史十人,知印二人,怯里馬赤二人,宣使十人。其屬有家令司,家令二員,正三品,二員,正四品;家丞二員,正五品;典簿二員,從七品;照磨一員,正九品。有府正司,府正二員,正三品;府丞二員,正五品;典簿二員,從七品;照磨一員,正九品。有典寶監,典寶卿二員,正三品;太監二員,從三品;少監二員,從四品;監丞二員,正五品;經歷一員,從七品;知事一員,從八品;照磨一員,正九品。有儀衛司,指揮二員,從四品;副二員,從五品;知事一員,從八品。十一月,置典藏庫,從五品,掌收皇太子錢帛。十七年十月,置分詹事院,詹事一員,同知、副使各一員,詹事丞二員,經歷一員,都事二員,照磨兼架閣一員,斷事官二員,知事一員。



 大撫軍院。至正二十七年八月乙巳,命皇太子總天下軍馬。九月,皇太子置大撫軍院,從一品。知院四員,同知二員,副使一員,同僉一員。首領官:經歷、都事各二員,照磨兼管勾一員。二十八年閏七月,詔罷之。



 大都分府。至正十八年三月,東安、漷州、柳林日有警報,京師備御四隅,俱立大都分府。其官吏數,視都府減半。



 警巡院。至正十一年七月,升左、右兩巡院為正五品。十八年,又於大都在城四隅各立警巡分院,官吏視本院減半。



 行中書省。至正十二年正月,江西、江浙行省皆除添設平章,陜西行省除添設右丞。閏三月,置淮南江北等處行中書省於揚州,以淮西宣慰司、兩淮鹽運司、揚州、淮安、徐州、唐州、安豐、蘄、黃皆隸焉。除平章二員,右丞、左丞各一員,參政二員,及首領官、屬官共二十五員。為頭平章,兼提調鎮南王傅府事。至十一月,始鑄淮南江北等處行中書省印給之。是年,江浙行省添設右丞、參政,四川行省添設參政。十六年五月,置福建等處行中書省於福州,鑄印設官,一如各處行省之制。以江浙行中書省平章左答納失里、南臺中丞阿魯溫沙為福建行中書省平章政事,福建閩海道廉訪使莊嘉為右丞,福建元帥吳鐸為左丞,司農丞訥都赤、益都路總管卓思誠為參政。以九月至福州,罷帥府,開省署。十七年九月,置山東行省,以大司農哈剌章為平章政事,鑄印與之。十八年,福建行省右丞朵歹分省建寧,參政訥都赤分省泉州。二十三年三月,置廣西行中書省,以廉訪使也兒吉尼為平章政事。又置膠東行省於萊陽,總制東方事。二十六年八月,置福建江西等處行中書省。



 行樞密院。至元三年,伯顏右丞相奏準,於四川及湖廣、江西之境,及江浙,凡三處,各置行樞密院,以鎮遏好亂之民。每處設知院一員,同知、僉院、院判各一員。湖廣、江西二省所轄地里險遠,添設同僉一員。各院經歷一員,都事二員,照磨一員,客省副使一員,斷事官二員,蒙古必闍赤二人,掾史六人,宣使六人,知印、怯里馬赤各一人,斷事官譯史一人,令史二人,怯里馬赤、知印各一人,奏差二人。至四年二月,遂罷之。至正十三年五月,嶺北行樞密院添設斷事官二員,先已設四員,共六員。又立鎮撫司,除鎮撫二員。立管勾所,置管勾一員,兼照磨。後又添設僉院二員、都事一員。十五年十月,置淮南江北等處行樞密院於揚州。十二月,河南行樞密院添設院判一員。十六年三月,置江浙行樞密院於杭州,知院二員,同知二員,副樞二員,僉院二員,同僉二員,院判二員。首領官:經歷、知事各一員,斷事官二員,經歷一員。十八年,以參政崔敬為山東等處行樞密院副使,分院於漷州,兼領屯田事。十九年八月,以察罕帖木兒為河南行省平章政事,兼河南山東等處行樞密院知院。二十六年八月,置福建江西等處行樞密院。



 行御史臺。至正十六年九月二十八日,命太尉納麟為江南諸道行御史臺御史大夫,以次官員,各依等第選用。是日,御史臺奉旨,移置行臺於紹興。十二月,合臺官屬,開臺署事。是年,置河南廉訪司於沂州。十八年,御史臺奏準,江西湖東道肅政廉訪司,權於建寧路開司署事。二十二年九月,權置山北廉訪司於惠州。二十三年六月,濟南路復置肅政廉訪司。二十五年閏十月,御史大夫完者帖木兒奏:「江南諸道行御史臺衙門,嘗奉旨於紹興路開設,近因道梗,湖南、湖北、廣東、廣西、海北、江西、福建等處,凡有文書,北至南臺,風信不便,徑申內臺,未委事情虛實。宜於福建置分臺,給降印信,俾湖南、湖北、廣東、廣西、海北、江西、福建各道文書,由分臺以達內臺,於事體為便。」有旨從之。十一月,仍置河東廉訪司於冀寧。



 行宣政院。元統二年正月,革罷廣教總管府一十六處,置行宣政院於杭州。除院使二員,同知二員,副使二員,同僉、院判各一員。首領官:經歷二員,都事、知事、照磨各一員,令史八人,譯史二人,宣使八人。至元二年五月,西番寇起,置行宣政院,以也先帖木兒為院使往討之。至正二年,江浙行宣政院設崇教所,擬行中書省理問官,秩四品,以理僧民之事。



 河南山東都水監。至正六年五月,以連年河決為患,置都水監,以專疏塞之任。



 行都水監。至正八年二月,河水為患,詔於濟寧鄆城立行都水監。九年,又立山東河南等處行都水監。十一年十二月,立河防提舉司,隸行都水監,掌巡視河道,從五品。十二年正月,行都水監添設判官二員。十六年正月,又添設少監、監丞、知事各一員。



 都水庸田使司。至元二年正月,置都水庸田使司於平江,既而罷之。至五年,復立。至正十二年,因海運不通,京師闕食,詔河南窪下水泊之地,置屯田八處,於汴梁添都水庸田使司,正三品,掌種植稻田之事。庸田使二員,副使二員,僉事二員。首領官:經歷、知事、照磨各一員,司吏十二人,譯史二人。



 都總制庸田使司。至正十年,置河南江北等處都總制庸田使司,定置都總制庸田使二員,從二品;副使二員,從三品;僉司六員,從四品。首領官:經歷二員,從六品;都事二員,從七品;照磨兼管勾承發架閣一員,從八品;蒙古必BX赤、回回令使、怯里馬赤、知印各一人,令史十八人,宣使十八人,壕寨十八人,典吏四人。其屬官,則有軍民屯田總管府,凡五處,置達魯花赤各一員,從三品;總管各一員,正五品;同知各一員,正六品;府判各一員,從七品。首領官:經歷各一員,從八品;知事各一員,從九品;提控案牘兼管勾承發架閣各一員,蒙古譯史各一人,司吏各六人,典吏各二人。又有農政司,置農政一員,正五品;農丞一員,正六品;提控一員,司吏二人。又有豐盈庫,置提領一員,正八品;大使、副使各一員,正九品。



 分司農司。至正十三年正月,命中書右丞悟良哈臺、左丞烏古孫良楨兼大司農卿,給分司農司印。西自西山,南至保定、河間,北至檀、順州,東至遷民鎮,凡系官地,及元管各處屯田,悉從分司農司立法募民佃種之。



 大兵農司。至正十五年,詔有水田去處,置大兵農司,招誘夫丁,有事則乘機招討,無事則栽植播種。所置司之處,曰保定等處大兵農使司、河間等處大兵農使司、武清等處大兵農使司、景薊等處大兵農使司。其屬,有兵農千戶所,共二十四處;百戶所,共四十八處;鎮撫司各一。



 大都督兵農司。至正十九年二月,置大都督兵農司於西京,以孛羅帖木兒領之,從其所請也。仍置分司十道,專掌屯種之事。



 茶運司。元統元年十一月,復置湖廣江西榷茶都轉運司。



 鹽運司。至正二年十一月,中書省奉旨講究鹽法,奏準於杭州、嘉興、紹興、溫臺四處,各置檢校批驗所,直隸運司,專掌批驗鹽商引目,均平袋法稱盤等事。每所置檢校批驗官一員,從六品;相副官一員,正七品。



 漕運司。至元二年五月,京畿都漕運司添設提調官、運副、運判各一員。至正九年,添設海道巡防官,給降正七品印信,掌統領軍人水手,防護糧船。巡防官二員,相副官二員。



 防御海道運糧萬戶府。至正十五年七月,升臺州海道巡防千戶所為防御海道運糧萬戶府。九月,置分府於平江。



 添設兵馬司。至正十年十月,中書省奏:「東南千里外,妖氣見,合立兵馬司四處,掌防禦之職。」遂置大名兵馬司、東平兵馬司、濟南兵馬司、徐州兵馬司。每司置都指揮、指揮各二員,副指揮各四員,經歷、知事、提控案牘各一員,譯史各二人,司吏各十二人,奏差各八人,貼書各二十四人,忽剌罕赤各三十人,司獄各一員,獄丞各一員。十一年,罷沂州分元帥府,改立兵馬指揮使司。十五年十月,濟寧兵馬司添設副指揮二員。



 各處寶泉提舉司。至正十一年十月,置寶泉提舉司於河南行省及濟南、冀寧等處,凡九所。江浙、江西、湖廣行省各一所。十二年三月,置銅冶場於饒州路德興縣、信州路鉛山州、韶州岑水,凡三處。每所置提領一員,正八品;大使一員,從八品;副使一員,正九品;流官內銓注。直隸寶泉提舉司,掌浸銅事。



 湖南道宣慰使司都元帥府。至元元年六月奏準,湖南道宣慰使司兼都元帥府,總領所轄路分鎮守萬戶軍馬。



 邦牙等處宣慰使司都元帥府,至元四年十二月置。先是,以緬地處雲南極邊,就立其酋長為帥,三年一貢方物。至是來貢,故改立官府以獎異之。



 永昌等處宣慰使司都元帥府。至正三年七月,中書省奏:「闊端阿哈所分地方,接連西番,自脫脫木兒既沒之後,無人承嗣。達達人口頭匹,時被西番劫奪殺傷,深為未便。」遂定置永昌等處宣慰使司都元帥府以治之,置宣慰使三員、同知二員、副使二員。首領官:經歷、知事、照磨各一員,令史十人,蒙古譯史四人,知印二人,怯里馬赤一人,奏差八人,典吏二人。



 山東東西道宣慰使司都元帥府,至正六年十二月改立,掌開設屯田、屯駐軍馬之事。



 荊湖北道宣慰使司都元帥府。至正十一年十一月奏準,荊湖北道宣慰使司兼都元帥府。



 浙東宣慰司。至正十二年正月,添設宣慰使一員、同知一員、都事二員。



 淮東等處宣慰使司都元帥府,至正十五年二月置。統率濠泗義兵萬戶府,並洪澤等處義兵。招誘富民,出丁壯五千名者為萬戶,五百名者為千戶,一百名者為百戶,降宣敕牌面與之,命置司於泗州天長縣。



 興元等處宣慰使司都元帥府,至正十五年十二月置。



 江州等處宣慰使司都元帥府,至正十六年九月奏準,宣慰使都元帥廷授,佐貳僚屬,命江西行省平章政事道童、火你赤承制署之。



 河南宣慰司。至正十九年十月,罷洛陽招討軍民萬戶府,置宣慰司,以張俊為宣慰使。



 東路都蒙古軍都元帥府,至正八年正月置。



 分元帥府。至正八年十二月,以福建盜起,詔汀、漳二州立分元帥府,以討捕之。十一月,命買列的開分元帥府於沂州,以鎮御東海群盜。十一年正月,湖南寶慶路置分元帥府,又置寶武分元帥府。三月,置山東分元帥府於登州,提調登、萊、寧海三州三十六處海口事。十二年二月,置安東、安豐二處分元帥府。



 水軍元帥府。至正二十六年二月,置河淮水軍元帥府於孟津縣。



 紹熙軍民宣撫司。至元四年,因監察御史言:「四川在宋時,有紹熙一府,統六州、二十縣、一百五十二鎮。近年雍、梁、淮甸人民,見彼中田疇廣闊,開墾成業者,凡二十餘萬戶。」省部議定,遂奏準置紹熙等處軍民宣撫司。正官六員,宣撫使、同知、副使各二員。首領官三員,經歷、知事、提控案牘各一員。司獄一員,蒙古、儒學教授各一員,令史八人,譯吏、知印、怯里馬赤各一人,奏差四人。所隸資、普、昌、隆下州四處,盤石、內江、安岳、昌元、貴平下縣五處,巡檢司一十三處,各設官如制。又置都總使司,命御史大夫脫脫兼都總使,治書侍御史吉當普為副都總使。至元六年十一月,中書又因臺臣言裁減冗官事,遂罷紹熙軍民宣撫司。



 永順宣撫司。至正十一年四月,改升永順安撫司為宣撫司。



 平緬宣撫司。至正十五年八月,以雲南死可伐等降,令其子莽三入貢方物,乃置平緬宣撫司以羈縻之。



 忠孝軍民安撫司。至正十一年七月,革罷四川省所轄大奴管勾等洞長官司,立忠孝軍民府。至十五年四月,詔改為忠孝軍民安撫司。



 忠義軍民安撫司。至正十五年四月,罷四川羊母甲洞、臭南王洞長官司,置忠義軍民安撫司。又罷盤順府,置盤順軍民安撫司。



 宣化鎮南五路軍民府。至正十五年四月,命於四川置立提調軍民鎮撫所、蠻夷軍民千戶所。



 團練安撫勸農使司。至正十八年九月,置奉元延安等處團練安撫勸農使司於耀州,鞏昌等處團練安撫勸農使司於邠州,以行省丞相朵朵、行臺大夫完者帖木兒領之,各設參謀一人。每道置使二人,同知、副使各二人,檢督六人,經歷、知事、照磨各一人。



 防禦使。至正十七年正月,準山東分省咨,團結義兵,每州添設州判一員,每縣添設主簿一員,詔有司正官俱兼防禦使事,聽宣慰使司節制。



 屯田使司。至正十五年十二月,置軍民屯田使司於沛縣,正三品。



 屯田打捕總管府。至元四年五月,升兩淮屯田打捕總管府為正三品。



 黎兵萬戶府。元統二年十月,湖廣行省咨:「海南僻在極邊,南接占城,西鄰交趾,環海四千餘里,中盤百洞,黎、獠雜居,宜立萬戶府以鎮之。」中書省奏準,依廣西屯田萬戶府例,置黎兵萬戶府。萬戶三員,正三品。千戶所一十三處,正五品。每所領百戶所八處,正七品。



 水軍萬戶府。至正十三年十月,置水軍都萬戶府於昆山州,以浙東宣慰使納麟哈剌為正萬戶,宣慰使董摶霄為副萬戶。十四年二月,立鎮江水軍萬戶府,命江浙行省右丞佛家閭領之。十五年十月,置水軍萬戶府於黃河小清口。



 義兵萬戶府。至正十四年二月,詔河南、淮南兩省並立義兵萬戶府。五月,置南陽、鄧州等處毛胡蘆義兵萬戶府,募土人為軍,免其差役,令討賊自效。先是,鄉人自相團結,號毛胡蘆,故因以名之。十五年四月,置汴梁等處義兵萬戶府。十二月,置忠義、忠勤萬戶府於宿州及武安州。



 招討軍民萬戶府。至正二十年,以鞏縣為招討軍民萬戶府。二十六年三月,置嵩州軍民招討萬戶府。



 義兵千戶所。至正十年七月中書奏準,於廣西平樂等古城竹山院、桑江隘、尊化鄉、剌場嶺,湖南道州路、武岡路、湖北靖州路等處,置義兵千戶所,每所置千戶一員、彈壓一員、百戶十員。仍於義兵內推選才勇功能,充千戶、彈壓、百戶之職。首領官、都目各一員,於本省都吏目選內注授,並從本道帥府節制。湖南道州二處千戶所,於帥府分司處設立,本司調遣。湖北靖州一處,從本省摽撥鎮守調遣。總定九十六員,給降宣敕牌面印信。十三年十一月,立義兵千戶水軍千戶所於江西。



 奉使宣撫。至正五年十月,遣官分道奉使宣撫,布宣德意,詢民疾苦,疏滌冤滯,蠲除煩苛,體察官吏賢否,明加黜陟。有罪者,四品以上停職申請,五品以下就便處決,民間一切興利除害之事,悉聽舉行。其餘必合上聞者,條具入告。兩浙江東道,以江西行省左丞忽都不丁、吏部尚書何執禮為之,宣政院都事吳密為首領官。江西福建道,以雲南行省右丞散散、將作院使王士弘為之,國子典簿孟昉為首領官。江南湖廣道,以大都路達魯花赤拔實、江浙參政秦從德為之,留守司都事月忽難為首領官。海北廣東道,以平江路達魯花赤左答納失理、都水使賈惟貞為之,都水照磨楊文在為首領官。燕南山東道,以資正院使蠻子、兵部尚書李獻為之,太醫院都事賈魯為首領官。河東陜西道,以兵部尚書不花、樞密院判官靳義為之,翰林應奉王繼善為首領官。山北遼東道,以宣政院同知伯家奴、宣徽僉院王也速迭兒為之,工部主事明理不花為首領官。雲南省,以荊湖宣慰阿乞剌、兩浙鹽運使杜德遠為之,通政院都事楊矩為首領官。甘肅永昌道,以上都留守阿牙赤、陜西行省左丞王紳為之,沁源縣尹喬遜為首領官。四川省,以大都留守答爾麻失里、河南參政王守誠為之,宣政院都事武祺為首領官。京畿道,以西臺中丞定定、集賢侍講學士蘇天爵為之,太史院都事留思誠為首領官。河南江北道,以吏部尚書定僧、宣政院僉院魏景道為之,中書檢校哈爾丹為首領官。至正十七年九月,詔以中書右丞也先不花、御史中丞成遵奉使宣撫彰德、大名、廣平、東昌、東平、曹、濮等處,獎厲將帥。



 經略使。至正十八年九月初六日,命經略使問民疾苦,招諭叛逆,果有怙終不悛,總督一應大小官吏,治兵裒粟,精練士卒,審用成算,申明紀律。先定江西、湖廣、江浙、福建諸處,並力掎角,務收平復之效,不尚屠戮之威。江南各省民義,忠君親上,姓名不能上達者,優加撫存,量才驗功,授以官爵。旌表孝子順孫、義夫節婦、高年耆德,常令有司存恤鰥寡孤獨。選官二員為經略使參謀官,闢名士一人掌案牘。設行軍司馬一員,秩正五品,掌軍律。



 ◎選舉附錄



 ○科目



 元以科目取士,自延祐至元統凡七科,具見前志。既罷復興之後,至正二年三月戊寅,廷試舉人,賜拜住、陳祖仁等進士及第、進士出身、同進士出身有差,凡七十有八人。國子生員十有八人:蒙古人六名,從六品出身;色目人六名,正七品出身;漢人、南人共六名,從七品出身。五年三月辛卯,廷試舉人,賜普顏不花、張士堅等進士及第、進士出身、同進士出身有差,如前科之數。國子生員亦如之。八年三月癸卯,廷試舉人,賜阿魯輝帖穆而、王宗哲等進士及第、進士出身、同進士出身有差,如前科之數。國子生員亦如之。是年四月,中書省奏準,監學生員每歲取及分生員四十人,三年應貢會試者,凡一百二十人。除例取十八人外,今後再取副榜二十人,於內蒙古、色目各四名,前二名充司鑰,下二名充侍儀舍人。漢人取一十二人,前三名充學正、司樂,次四名充學錄、典籍管勾,以下五名充舍人。不願者,聽其還齋。十一年三月丙辰,廷試舉人,賜朵列圖、文允中等進士及第、進士出身、同進士出身有差,凡八十有三人。國子生員如舊制。



 十二年三月,有旨:「省院臺不用南人,似有偏負。天下四海之內,莫非吾民,宜依世祖時用人之法,南人有才學者,皆令用之。」自是累科南方之進士,始有為御史,為憲司官,為尚書者矣。十四年三月己巳,廷試舉人,賜薛朝晤、牛繼志等進士及第、進士出身、同進士出身有差,凡六十有二人。國子生員如舊制。十七年三月,廷試舉人,賜侻徵、王宗嗣等進士及第、進士出身、同進士出身有差,凡五十有一人。國子生員如舊制。



 十九年,中書左丞成遵建言:「宋自景祐以來,百五十年,雖無兵禍,常設寓試名額,以待四方游士。今淮南、河南、山東、四川、遼陽等處,及江南各省所屬州縣,避兵士民,會集京師。如依前代故事,別設流寓鄉試之科,令避兵士民就試,許在京官員及請俸掾譯史人等,系其鄉里親戚者,結罪保舉,行移大都路印卷,驗其人數,添差試官,別為考校,依各處元額,選合格者充之,則國有得人之效,野無遺賢之嘆矣。」既而監察御史亦建言此事,中書送禮部定擬:「曾經殘破處所,其鄉試元額,蒙古、色目、漢人、南人總計一百三十有二人。如今流寓儒人,應試名數,難同全盛之時,其寓試解額,合照依元額減半量擬,取合格蒙古、色目各十五名,漢人二十名,南人十五名,通六十有五名。」中書省奏準,如所擬行之。而是歲福建行中書省初設鄉試,定取七人為額,而江西流寓福建者亦與試焉,通取十有五人,充貢於京師。而陜西行省平章政事察罕帖木兒又請:「今歲八月鄉試,河南舉人及避兵儒士,不拘籍貫,依河南省元額數,就陜州置貢院應試。」詔亦從之。二十年三月,廷試舉人,賜買住、魏元禮等進士及第、進士出身、同進士出身有差,凡三十有五人。國子生員如舊制。二十三年三月丁未,廷試舉人,賜寶寶、楊輗等進士及第、進士出身、同進士出身有差,凡六十有二人。國子生員如舊制。是年六月,中書省奏:「江浙、福建舉人,涉海道以赴京,有六人者,已後會試之期,宜授以教授之職;其下第三人,亦以教授之職授之。非徒慰其跋涉險阻之勞,亦及激勸遠方忠義之士。」



 二十五年,皇太子撫軍河東,適當大比之歲,擴廓帖木兒以江南、四川等處皆阻於兵,其鄉試不廢者,唯燕南、河南、山東、陜西、河東數道而已,乃啟皇太子倍增鄉貢之額。二十六年三月,廷試舉人,賜赫德溥化、張棟等進士及第、進士出身、同進士出身有差,凡七十有三人,優其品秩,第一甲,授承直郎,正六品,第二甲,授承務郎,從六品;第三甲,授從仕郎,從七品。國子生員:蒙古七名,正六品;色目六名,從六品;漢人七名,正七品;通二十人。兵興已後,科目取士,莫盛於斯;而元之設科,亦止於是歲雲。



\end{pinyinscope}