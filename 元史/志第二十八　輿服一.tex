\article{志第二十八 輿服一}

\begin{pinyinscope}

 若稽往古,黃帝、堯、舜垂衣裳而天下治,蓋取諸乾坤;服牛乘馬,引重致遠,蓋取諸大壯。冕服車輿之制,其來尚矣。《虞書》舜作十二章,五服以命有德,車服以賞有功。《禮記》虞鸞車,夏鉤車,商大輅。至周,損益前代,弁師掌王之五冕,巾車掌王之五輅,而儀文始備。然孔子論治天下之大法,於殷輅取其質而得中,周冕取其文而得中也。至秦並天下,兼收六國車旗服御,窮極侈靡,有大駕、法駕以及鹵簿。漢承秦後,多因其舊。由唐及宋,亦效秦法,以為盛典。於文質適中之義,君子或得而議焉。



 元初立國,庶事草創,冠服車輿,並從舊俗。世祖混一天下,近取金、宋,遠法漢、唐。至英宗親祀太廟,復置鹵簿。今考之當時,上而天子之冕服,皇太子冠服,天子之質孫,天子之五輅與腰輿、象轎,以及儀衛隊仗,下而百官祭服、朝服,與百官之質孫,以及於士庶人之服色,粲然其有章,秩然其有序。大抵參酌古今,隨時損益,兼存國制,用備儀文。於是朝廷之盛,宗廟之美,百官之富,有以成一代之制作矣。作《輿服志》,而儀衛附見於後云。



 冕服



 天子冕服:袞冕,制以漆紗,上覆曰綖,青表硃里。綖之四周,匝以雲龍。冠之口圍,縈以珍珠。綖之前後,旒各十二,以珍珠為之。綖之左右,系黈纊二,系以玄紞,承以玉瑱,纊色黃,絡以珠。冠之周圍,珠雲龍網結,通翠柳調珠。綖上橫天河帶一,左右至地。珠鈿窠網結,翠柳硃絲組二,屬諸笄,為纓絡,以翠柳調珠。簪以玉為之,橫貫於冠。



 袞龍服,制以青羅,飾以生色銷金帝星一、日一、月一、升龍四、復身龍四、山三十八、火四十八、華蟲四十八、虎蜼四十八。



 裳,制以緋羅,其狀如裙,飾以文繡,凡一十六行,每行藻二、粉米一、黼二、黻二。



 中單,制以白紗,絳緣,黃勒帛副之。



 蔽膝,制以緋羅,有褾。緋絹為里,其形如衣詹,袍上著之,繡復身龍。



 玉佩,珩一、琚一、瑀一、沖牙一、璜二。沖牙以系璜,珩下有銀獸面,塗以黃金,雙璜夾之。次又有衡,下有沖牙。傍別施雙的以鳴,用玉。



 大帶,制以緋白二色羅,合縫為之。



 玉環綬,制以納石失。金錦也。上有三小玉環,下有青絲織網。



 紅羅靴,制以紅羅為之,高靿。



 履,制以納石失,有雙耳二,帶鉤,飾以珠。



 襪,制以紅綾。



 右按《太常集禮》,至元十二年十一月,博士議擬:冕天版長一尺六寸,廣八寸,前高八寸五分,後高九寸五分,身圍一尺八寸三分,並納言,用青羅為表,紅羅為里,周回緣以黃金。天版下四面,珠網結子,花素墜子,前後共二十有四旒,以珍珠為之。青碧線織天河帶,兩頭各有珍珠金翠旒三節,玉滴子節花全。紅線組帶二,上有珍珠金翠旒,玉滴子,下有金鐸二。梅紅繡款幔帶一,黈纊二,珍珠垂系,上用金萼子二。簪窠款幔組帶鈿窠各二,內組帶窠四,並鏤玉為之。玉簪一,頂面鏤雲龍。袞衣,用青羅夾制,五採間金,繪日、月、星辰、山、龍、華蟲、宗彞。正面日一,月一,升龍四,山十二,上下襟華蟲、火各六對,虎隹各闕對,背星一,升龍四,山十二,華蟲、火各十二對,虎蜼各六對。中單,用白羅單制,羅領褾襈。裳一,帶褾襈全,紅羅八幅夾造。上繡藻、粉米、黼、黻,藻三十二,粉米十六,黼三十二,黻三十二。蔽膝一,帶衣票襈,紅羅夾造八幅,上繡升龍二。綬一幅,六採織造,紅羅托里。小綬三,色同大綬,銷金黃羅綬頭全,上間施二玉環,並碾雲龍。緋白大帶一,銷金黃帶頭,鈿窠二十有四。紅羅勒帛一,青羅抹帶一。佩二,玉上、中、下璜各一,半月各二,並碾玉為雲龍文。玉滴子各二,並珍珠穿造。金篦鉤,獸面,水葉環釘全。涼帶一,紅羅里,鏤金為之;上為玉鵝七,撻尾束各一,金攀龍口,玳瑁襯釘。舄一,重底,紅羅面,白綾托裏,如意頭,銷金黃羅緣口,玉鼻,純飾以珍珠。金緋羅錦襪一兩。



 大德十一年九月,博士議:唐制,天子袞冕,垂白珠十有二旒,以組為纓,色如其綬,黈纊充耳,玉簪導。玄衣纁裳,凡十二章。八章在衣,日、月、星辰、山、龍、華蟲、火、宗彞;四章在裳,藻、粉米、黼、黻。褾領為升龍,皆織成之。龍章以下,每章一行,每行十二。白紗中單,黼領,青褾襈裾,黻加龍、山、火三章。毳冕以上,火、山二章。絺冕,山一章。玄冕無章。革帶、大帶、玉佩、綬、襪,與上同。舄加金飾。享廟、謁廟及朝遣上將、徵還飲至、踐阼加元服、納后、元日受朝及臨軒冊拜王公則服之。又宋制,天子服有袞冕,廣尺二寸,長二尺四寸,前後十有二旒,二纊,並貫珍珠。又有翠旒十二,碧鳳銜之,在珠旒外。冕板,以龍鱗錦表,上綴玉為七星,傍施琥珀瓶、犀瓶各二十四,周綴金絲網鈿,以珍珠雜寶玉,加紫雲白鶴錦里。四柱飾以七寶,紅綾里。金飾玉簪導,紅絲絳組帶。亦謂之平天冠。袞服青色,日、月、星、山、龍、雉、虎蜼七章。紅裙,藻、火、粉米、黼、黻五章。紅蔽膝,升龍二,並織成,間以雲彩,飾以金鈒花鈿窠,裝以珍珠、琥珀、雜寶玉。紅羅襦裙,繡五章,青褾襈裾。六採綬一,小綬三,結三,玉環三。素大帶,硃里。青羅四神帶二,繡四神盤結。綬帶飾並同袞服。白羅中單,青羅抹帶,紅羅勒帛,鹿盧玉具劍,玉鏢首鏤白玉雙佩,金飾,貫珍珠。金龍鳳革帶,紅襪赤舄,金鈒花,四神玉鼻。祭天地宗廟、受冊尊號、元日受朝、冊皇太子則服之。事未果行。



 至延祐七年七月,英宗命禮儀院使八思吉斯傳旨,令省臣與太常禮儀院速制法服。八月,中書省會集翰林、集賢、太常禮儀院官講議,依秘書監所藏前代帝王袞冕法服圖本,命有司制如其式。



 鎮圭,制以玉,長一尺二寸,有袋副之。



 皇太子冠服:袞冕,玄衣,纁裳,中單,蔽膝,玉佩,大綬,硃襪,赤舄。



 按《太常集禮》,至元十二年,博士擬袞冕制,用白珠九旒,紅絲組為纓,青纊充耳,犀簪導。青衣、硃裳,九章。五章在衣,山、龍、華蟲、火、宗彞;四章在裳,藻、粉米、黼、黻。白紗中單,青示票襈裾。革帶,塗金銀鉤颻。蔽膝,隨裳色,為火、山二章。瑜玉雙佩,四採織成大綬,間施玉環三。白襪硃舄,舄加金塗銀扣。



 大德十一年九月,照擬前代制度。唐制,皇太子袞冕,垂白珠九旒,紅絲組為纓,青纊充耳,犀簪導。玄衣、纁裳,九章。五章在衣,山、龍、華蟲、火、宗彞;四章在裳,藻、粉米、黼、黻,織成之,每行一章,黼、黻重以為等,每行九。白紗中單,黼領,青褾襈裾。革帶,金鉤颻,大帶。蔽膝,隨裳色,火、山二章。玉具劍,金寶飾玉鏢首,瑜玉雙佩。硃組帶大綬,四採赤白縹紺,純硃質,長丈八尺,首廣九寸。小雙綬,長二尺六寸,色同大綬,而首半之,間施玉環三。硃襪赤舄,加金飾。侍從祭祀及謁廟、加元服、納妃服之。宋制,皇太子袞冕,垂白珠九旒,紅絲組為纓,青纊充耳,犀簪導。青衣、硃裳,九章。五章在衣,山、龍、華蟲、火、宗彞;四章在裳,藻、粉米、黼、黻。白紗中單,青褾襈裾。革帶,塗金銀鉤颻。蔽膝,隨裳色,火、山二章。瑜玉雙佩,四採織成大綬,間施玉環三。白襪、硃舄,舄加塗金銀飾。加元服、從祀、受冊、謁廟、朝會服之。已擬其制,未果造。



 三獻官及司徒、大禮使祭服:籠巾貂蟬冠五,青羅服五,領、袖、襴俱用皁綾。



 紅羅裙五,皁綾為襴。紅羅蔽膝五,其羅花樣俱系牡丹。白紗中單五,黃綾帶。紅組金綬紳五,紅組金譯語言納石失,各佩玉環二。象笏五,銀束帶五,玉佩五,白羅方心曲領五,赤革履五對,白綾襪五對。



 助奠以下諸執事官冠服:貂蟬冠、獬豸冠、七梁冠、六梁冠、五梁冠、四梁冠、三梁冠、二梁冠二百,青羅服二百,領、袖、襴俱用皁綾。紅綾裙二百,皁綾為襴。紅羅蔽膝二百,紫羅公服二百,用梅花羅。白紗中單二百,黃綾帶。



 織金綬紳二百,紅一百九十八,青二,各佩銅環二。銅束帶二百,白羅方心曲領二百,銅佩二百,展角襆頭二百,塗金荔枝帶三十,烏角帶一百七十,皁靴二百對,赤革履二百對,白綾襪二百對,象笏三十,銀杏木笏一百七十。



 凡獻官諸執事行禮,俱衣法服。惟監察御史二,冠獬豸,服青綬。凡迎香、讀祝及祀日遇陰雨,俱衣紫羅公服。六品以下,皆得借紫。



 都監庫、祠祭局、儀鸞局、神廚局頭目長行人等:交角襆頭五十,窄袖紫羅服五十,塗金束帶五十,皁靴五十對。



 初憲宗壬子年秋八月,祭天於日月山,用冕服自此始。成宗大德六年春三月,祭天於麗正門外丙地,命獻官以下諸執事,各具公服行禮。是時,大都未有郊壇,大禮用公服自此始。九年冬至祭享,用冠服,依宗廟見用者制。其後節次祭祀,或合祀天地,增配位從祀,獻攝職事,續置冠服,於法服庫收掌。法服二百九十有九,公服二百八十,窄紫二百九十有五。至大間,太常博士李之紹、王天祐疏陳,親祀冕無旒,服大裘而加袞,裘以黑羔皮為之。臣下從祀冠服,歷代所尚,其制不同。集議得依宗廟見用冠服制度。



 社稷祭服:青羅袍一百二十三,白紗中單一百三十三,紅梅花羅裙一百二十三,藍織錦銅環綬紳二,紅織錦銅環綬紳一百一十七,紅織錦玉環綬紳四,紅梅花羅蔽膝一百二十三,革履一百二十三,白綾襪一百二十三,白羅方心曲領一百二十三,黃綾帶一百二十三,佩一百二十三,銅珩璜者一百一十九,玉珩璜者四,藍素苧絲帶一百二十三,銀帶四,銅帶一百一十九,冠一百二十三,水角簪金梁冠一百七,紗冠一十,獬豸冠二,籠巾紗冠四,木笏一百二十三,紫羅公服一百二十三,黑漆襆頭一百二十三,展角全二色羅插領一百二十三,鍍金銅荔枝帶一十,角帶一百一十三,象笏一十三枝,木笏一百一十枝,黃絹單包復一百二十三,紫苧絲抹口青氈襪一百一十三,皁靴一百二十三,窄紫羅衫三十,黑漆襆頭三十,銅束帶三十,黃絹單包復三十,皁靴三十,紫苧絲抹口青氈襪三十。



 宣聖廟祭服:獻官法服,七梁冠三,簪全。鴉青袍三,絨錦綬紳三,各帶青絨網並銅環二。方心曲領三,藍結帶三,銅佩三,紅羅裙三,白絹中單三,紅羅蔽膝三,革履三。白絹襪全。



 執事儒服,軟角唐巾,白襴插領,黃鞓角帶,皁靴,各九十有八。



 曲阜祭服,連蟬冠四十有三,七梁冠三,五梁冠三十有六,三梁冠四,皁苧絲鞋三十有六兩,舒角襆頭二,軟角唐巾四十,角簪四十有三,冠纓四十有三副,凡八十有六條。象牙笏七,木笏三十有八,玉佩七,凡十有四系。銅佩三十有六,凡七十有二系。帶八十有五,藍鞓帶七,紅鞋帶三十有六,烏角帶二,黃鞓帶、烏角偏帶四十,大紅金綬結帶七,上用玉環十有四。青羅大袖夾衣七,紫羅公服二,褐羅大袖衣三十有六,白羅衫四十,白絹中單三十有六,白紗中單七,大紅羅夾蔽膝七,大紅夾裳、緋紅羅夾蔽膝三十有六,緋紅夾裳四,黃羅夾裳三十有六,黃羅大帶七,白羅方心曲領七,紅羅綬帶七,黃絹大帶三十有六,皁靴、白羊毳襪各四十有二對,大紅羅鞋七兩,白絹夾襪四十有三兩。



 質孫,漢言一色服也,內庭大宴則服之。冬夏之服不同,然無定制。凡勛戚大臣近侍,賜則服之。下至於樂工衛士,皆有其服。精粗之制,上下之別,雖不同,總謂之質孫云。



 天子質孫,冬之服凡十有一等,服納石失、金錦也。怯綿里,翦茸也。則冠金錦暖帽。服大紅、桃紅、紫藍、綠寶裏,寶裏,服之有襴者也。則冠七寶重頂冠。服紅黃粉皮,則冠紅金褡子暖帽。服白粉皮,則冠白金褡子暖帽。服銀鼠,則冠銀鼠暖帽,其上並加銀鼠比肩。俗稱曰襻子答忽。夏之服凡十有五等,服答納都納石失,綴大珠於金錦。則冠寶頂金鳳鈸笠。服速不都納石失,綴小珠於金錦。則冠珠子卷雲冠。服納石失,則帽亦如之。服大紅珠寶裏紅毛子答納,則冠珠緣邊鈸笠。服白毛子金絲寶裏,則冠白藤寶貝帽。服駝褐毛子,則帽亦如之。服大紅、綠、藍、銀褐、棗褐、金繡龍五色羅,則冠金鳳頂笠,各隨其服之色。服金龍青羅,則冠金鳳頂漆紗冠。服珠子褐七寶珠龍褡子,則冠黃牙忽寶貝珠子帶後簷帽。服青速夫金絲襴子,速夫,回回毛布之精者也。則冠七寶漆紗帶後簷帽。



 百官質孫,冬之服凡九等,大紅納石失一,大紅怯綿里一,大紅冠素一,桃紅、藍、綠官素各一,紫、黃、鴉青各一。夏之服凡十有四等,素納石失一,聚線寶里納石失一,棗褐渾金間絲蛤珠一,大紅官素帶寶裏一,大紅明珠褡子一,桃紅、藍、綠、銀褐各一,高麗鴉青雲袖羅一,駝褐、茜紅、白毛子各一,鴉青官素帶寶裏一。



 百官公服:



 公服,制以羅,大袖,盤領,俱右衽。一品紫,大獨科花,徑五寸。二品小獨科花,徑三寸。三品散答花,徑二寸,無枝葉。四品、五品小雜花,徑一寸五分。六品、七品緋羅小雜花,徑一寸。八品、九品綠羅,無文。



 襆頭,漆紗為之,展其角。



 笏,制以牙,上圓下方。或以銀杏木為之。



 偏帶,正從一品以玉,或花,或素。二品以花犀。三品、四品以黃金為荔枝。五品以下以烏犀。並八胯,鞓用硃革。



 靴,以皁皮為之。



 儀衛服色:



 交角襆頭,其制,巾後交折其角。



 鳳翅襆頭,制如唐巾,兩角上曲,而作雲頭,兩旁覆以兩金鳳翅。



 學士帽,制如唐巾,兩角如匙頭下垂。



 唐巾,制如襆頭,而撱其角,兩角上曲作雲頭。



 控鶴襆頭,制如交角,金鏤其額。



 花角襆頭,制如控鶴襆頭,兩角及額上,簇象生雜花。



 錦帽,制以漆紗,後幅兩旁,前拱而高,中下,後畫連錢錦,前額作聚文。



 平巾幘,黑漆革為之,形如進賢冠之籠巾,或以青,或以白。



 武弁,制以皮,加漆。



 甲騎冠,制以皮,加黑漆,雌黃為緣。



 抹額,制以緋羅,繡寶花。



 巾,制以絁,五色,畫寶相花。



 兜鍪,制以皮,金塗五色,各隨其甲。



 襯甲,制如云肩,青錦質,緣以白錦,衷以氈,里以白絹。



 雲肩,制如四垂雲,青緣,黃羅五色,嵌金為之。



 裲襠,制如衫。



 襯袍,制用緋錦,武士所以衣易裲襠。



 士卒袍,制以絹絁,繪寶相花。



 窄袖袍,制以羅或絁。



 辮線襖,制如窄袖衫,腰作辮線細折。



 控鶴襖,制以青緋二色錦,圓答寶相花。



 窄袖襖,長行輿士所服,紺緅色。



 樂工襖,制以緋錦,明珠琵琶窄袖,辮線細折。



 甲,覆膊、掩心、捍背、捍股,制以皮,或為虎文、獅子文,或施金鎧銷子文。



 臂韝,制以錦,綠絹為里,有雙帶。



 錦螣蛇,束麻長一丈一尺,裹以紅錦。



 束帶,紅鞓雙獺尾,黃金塗銅胯,餘同腰帶而狹小。



 絳環,制以銅,黃金塗之。



 汗胯,制以青錦,緣以銀褐錦,或繡撲獸,間以雲氣。



 行縢,以絹為之。



 鞋,制以麻。



 翁鞋,制以皮為履,而長其靿,縛於行縢之內。



 雲頭靴,制以皮,幫嵌雲朵,頭作云象,翁束於脛。



 服色等第:仁宗延祐元年冬十有二月,定服色等第,詔曰:「比年以來,所在士民,靡麗相尚,尊卑混淆,僭禮費財,朕所不取。貴賤有章,益明國制,儉奢中節,可阜民財。」命中書省定立服色等第於後。



 一,蒙古人不在禁限,及見當怯薛諸色人等,亦不在禁限,惟不許服龍鳳文。龍謂五爪二角者。



 一,職官除龍鳳文外,一品、二品服渾金花,三品服金褡子,四品、五品服云袖帶襴,六品、七品服六花,八品、九品服四花。職事散官從一高。系腰,五品以下許用銀,並減鐵。



 一,命婦衣服,一品至三品服渾金,四品、五品服金褡子,六品以下惟服銷金,並金紗褡子。首飾,一品至三品許用金珠寶玉,四品、五品用金玉珍珠,六品以下用金,惟耳環用珠玉。同籍不限親疏,期親雖別籍,並出嫁同。



 一,器皿,謂茶酒器。除鈒造龍鳳文不得使用外,一品至三品許用金玉,四品、五品惟臺盞用金,六品以下臺盞用鍍金,餘並用銀。



 一,帳幕,除不得用赭黃龍鳳文外,一品至三品許用金花刺繡紗羅,四品、五品用刺繡紗羅,六品以下用素紗羅。



 一,車輿,除不得用龍鳳文外,一品至三品許用間金妝飾銀螭頭、繡帶,青幔,四品、五品用素獅頭、繡帶、青幔,六品至九品用素雲頭、素帶、青幔。



 一,鞍轡,一品許飾以金玉,二品、三品飾以金,四品、五品飾以銀,六品以下並飾以鍮石銅鐵。



 一,內外有出身,考滿應入流,見役人員服用,與九品同。



 一,授各投下令旨、鈞旨,有印信,見任勾當人員,亦與九品同。



 一,庶人除不得服赭黃,惟許服暗花苧絲紬綾羅毛毳,帽笠不許飾用金玉,靴不得裁制花樣。首飾許用翠花,並金釵錍各一事,惟耳環用金珠碧甸,餘並用銀。酒器許用銀壺瓶臺盞盂鏇,餘並禁止。帳幕用紗絹,不得赭黃,車輿黑油,齊頭平頂皁幔。



 一,諸色目人,除行營帳外,其餘並與庶人同。



 一,諸職官致仕,與見任同。解降者,依應得品級,不敘者,與庶人同。



 一,父祖有官,既沒年深,非犯除名不敘之限,其命婦及子孫與見任同。



 一,諸樂藝人等服用,與庶人同。凡承應妝扮之物,不拘上例。



 一,皁隸公使人,惟許服紬絹。



 一,娼家出入,止服皁褙子,不得乘坐車馬,餘依舊例。



 一,今後漢人、高麗、南人等投充怯薛者,並在禁限。



 一,服色等第,上得兼下,下不得僭上。違者,職官解見任,期年後降一等敘,餘人決五十七下。違禁之物,付告捉人充賞。有司禁治不嚴,從監察御史、廉訪司究治。



 御賜之物,不在禁限。



 輿輅



 玉輅。青質,金裝,青綠藻井,栲栳輪蓋。外施金裝雕木雲龍,內盤碾玉福海圓龍一,頂上匝以金塗鍮石耀葉八十一。上圍九者二,中圍九者三,下圍九者四。頂輪衣三重,上二重青繡雲龍瑞草,下一重無文。輪衣內黃屋一,黃素苧絲瀝水,下周垂硃絲結網,青苧絲繡小帶四十八,帶頭綴金塗小銅鈴,青苧絲繡絡帶二。頂輪平素面夾用青苧絲。蓋四周垂流蘇八,飾以五色茸線結網五重,金塗銅鈸五,金塗木珠二十有五。又系玉雜佩八,珩璜沖瑀全,金塗



 鍮石鉤掛十六,黃茸貫頂天心直下十字繩二,各長三丈。蓋下立硃漆柱四。柱下直平盤,虛櫃,中欞三十,下外桄二。漆繪犀、象、鸚鵡、錦雉、孔雀,隔窠嵌裝花板。櫃周硃漆勾闌,雲拱地霞葉百七十有九,下垂牙護泥虛板,並硃漆畫瑞草。勾闌上玉行龍十,碾玉蹲龍十,孔雀羽臺九,水精面火珠七,金圈焰銅照八。輿下周垂硃絲結網,飾以金塗鍮石鐸三百,彩畫鍮石梅萼嵌網眼中。輿之長轅三,界轅勾心各三,上下龍頭六。前轅引手玉螭頭三,並系以蹲龍。後轅方罨頭三,桄頭十六,絟以蹲龍三。轅頭衡一,兩端玉龍頭二,上列金塗銅鳳十二,含以金塗銅鈴。輿之軸一,輪二。軸之挲羅二,明轄蹲龍絟,並青漆。輪之輻各二十四,轂首壓貼金塗銅轂葉八十一,金塗鍮石擎耳戀攀四。櫃之前,硃漆金裝雲龍輅牌一,牌字以玉裝綴。輅之箱,四壁雕鎪漆畫填心隔窠龜文華板。上層左畫青龍,右畫白虎,前畫硃雀,後畫玄武。輅之前額,玉行龍二,奉一水精珠,後額如之。前兩柱青茸鈴索五,貼金鸞和大響銅鈴十,金塗鍮石雙魚五。下硃漆軾櫃一,櫃上金香球、金香寶、金香合、銀灰盤各一,並黃絲綬帶。輅之後,硃漆後轛一,金塗曲戌,黃苧絲銷金雲龍門簾一,緋苧絲繡雲龍帶二。輅之中,金塗鍮石鉸碾玉龍椅一,靠背上金塗圈焰玉明珠一。右建太常旗,十有二斿,青羅繡日、月、五星、升龍。右建闒戟一,九斿,青羅繡雲龍。中央黃羅繡青黑黼文兩旗,綢杠,並青羅,旗首金塗鍮石龍頭二,金塗銅鈴二,金塗鍮石鈸青纓緌十二重,金塗木珠流蘇十二重。龍椅上,方坐一,綠褥一,皆錦。銷金黃羅夾帕一,方輿地褥二,勾闌內褥八,皆用雜錦綺。青漆金塗鍮石鉸葉踏道一,小褥五重。青漆雕木塗金龍頭行馬一,小青漆梯一,青漆柄金塗長托叉二,短托叉二,金塗首青漆推竿一,青茸引輅索二,各長六丈餘,金塗銅環二,黃茸綏一。輅馬、誕馬,並青色。鞍轡鞦勒纓拂靷,並青韋,金飾。誕馬青織金苧絲屜四副。青羅銷金絹里籠鞍六。蓋輅黃絹大蒙帕一,黃油絹帕一。駕士平巾大袖,並青繪苧絲為之。



 至治元年,英宗親祀太廟,詔中書及太常禮儀院、禮部定擬制鹵簿五輅。以平章政事張珪、留守王伯勝、將作院使明裏董阿、侍儀使乙剌徒滿董其事。是年,玉輅成。明年,親祀御之。後復命造四輅,工未成而罷。



 金輅。赤質,金妝,青綠藻井,栲栳輪蓋。外施金妝雕木雲龍,內盤真金福海圓龍一,頂上匝以金塗鍮石耀葉八十一。上圍九者二,中圍九者三,下圍九者四。頂輪衣三重,上二重大紅繡雲龍瑞草,下一重無文。輪衣內黃屋一,黃素紵絲瀝水,下垂硃絲結網一周,大紅紵絲繡小帶四十八,帶頭綴金塗小銅鈴三百,大紅紵絲繡絡帶二。頂輪平素面夾用緋紵絲。蓋之四周垂流蘇八,飾以五色茸線結網五重,金塗鍮石雜佩八,珩璜沖瑀全,金塗鍮石鉤掛十有六,黃絨貫頂天心直下十字繩二。蓋下立硃漆柱四,柱下直平盤,虛櫃,中欞三十,其下外桄二,漆繪犀、象、鸚鵡、錦雉、孔雀,隔窠嵌妝花板。櫃上周遭硃漆勾闌,雲拱地霞葉一百七十有九,下垂牙護泥虛板,並硃漆畫瑞草。勾闌上金塗鍮石行龍十二,金塗鍮石蹲龍十,孔雀羽臺九,水精面火珠七,金圈焰銅照八。輿下垂硃絲結網一遭,飾以金塗鍮石鐸子三百,彩畫鍮石梅萼嵌網眼中。輿之長轅三,界轅勾心各三,上下龍頭六。前轅引手金塗鍮石螭頭三,並系以蹲龍。後轅方罨頭三,桄頭十六,系以蹲龍三。轅頭衡一,兩端金塗鍮石龍頭二,上列金塗銅鳳十二,含以金塗銅鈴。輿之軸一,輪二。軸之挲羅二,明轄蹲龍絟,並漆以赤。輪之輻各二十有四,轂首壓貼金塗銅轂葉八十有一,金塗鍮石擎耳戀攀四。櫃之前,硃漆金妝雲龍輅牌一,金塗鐵曲戌。輅之箱,四壁雕鎪漆畫填心隔窠龜文花板,上層左畫青龍,右畫白虎,前畫硃雀,後畫玄武。輅之前額,金行龍二,奉一水精珠,後額亦如之。前兩柱緋絨鈴索五,貼金鸞和大響銅鈴十,金塗鍮石雙魚五。下硃漆軾櫃一,櫃上金香球一,金香寶一,金香合,銀灰盤一,並黃紵絲綬帶。輅之後,硃漆後轛一,金塗曲戌,黃紵絲銷金雲龍門簾一,緋紵絲繡雲龍帶二。輅之中,黃金妝鉸龍椅一,靠背上金塗圈焰玉明珠一。左建太常旗,十有二斿,緋羅繡日、月、五星、升龍。右建闒戟一,九斿,緋羅繡雲龍。中央黃羅繡青黑黼文兩旗,綢杠,並大紅羅。旗首金塗鍮石龍頭二,金塗銅鈴二,金塗鍮石鈸硃纓緌十二重,金塗木珠流蘇十二重。龍椅上,金錦方坐子一,綠可貼金錦也。褥一,銷金黃羅夾帕一,方輿地金錦褥一,綠可貼褥一。勾闌內,可貼條褥四,藍紵絲條褥四,硃漆金塗鍮石鉸葉踏道一,小可貼條褥五重。硃漆雕木塗金龍頭行馬一,小硃漆梯一,硃漆柄金塗長托叉二,短托叉二,金塗首硃漆推竿一,紅絨引輅索二,金塗銅環二,黃絨執綏一。輅馬、誕馬,並赤色。鞍轡鞦勒纓拂套項,並赤韋,金妝。誕馬紅織金苧絲屜四副,紅羅銷金紅絹里籠鞍六。蓋輅黃絹大蒙帕一,黃油絹帕一。駕士平巾大袖,並緋繡紵絲為之。



 象輅。黃質,金妝,青綠藻井,栲栳輪蓋。外施金妝雕木雲龍,內盤描金象牙雕福海圓龍一,頂上匝以金塗鍮石耀葉八十有一。上圍九者二,中圍九者三,下圍九者四。頂輪衣三重,上二重黃繡雲龍瑞草,下一重無文。輪衣內黃屋一,黃素紵絲瀝水,下垂硃絲結網一遭,黃紵絲繡小帶四十有八,帶頭綴金塗小銅鈴三百,黃紵絲繡絡帶二。頂輪平素面夾用黃紵絲。蓋之四周垂流蘇八,飾以五色茸線結網五重,金塗銅鈸五,金塗木珠二十有五。又系金塗鍮石雜佩八,珩璜沖瑀全,金塗鍮石鉤掛十有六,黃絨貫頂天心直下十字繩二。蓋下立硃漆柱四,柱下直平盤,虛櫃,中欞三十,下外桄二,漆繪犀、象、鸚鵡、錦雉、孔雀,隔窠嵌妝花板。櫃上周遭硃漆勾闌,雲拱地霞葉百七十有九,下垂牙護泥虛板,並硃漆畫瑞草。勾闌上描金象牙雕行龍十,蹲龍十,孔雀羽臺九,水精面火珠七,金圈焰銅照八。輿下垂硃絲結網一遭,飾以金塗鍮石鐸子三百,採畫鍮石梅萼嵌網眼中。輿之長轅三,界轅勾心各三,上下龍頭六。前轅引手描金象牙雕螭頭三,並系以蹲龍。後轅方罨頭三,桄頭十有六,系以蹲龍三。轅頭衡一,兩端描金象牙雕龍頭二,上列金塗銅鳳十二,含以金塗銅鈴。輿之軸一,輪二。軸之挲羅二,明轄蹲龍絟,並漆以黃。輪之輻各二十有四,轂首壓貼金塗銅轂葉八十有一,金塗鍮石擎耳戀攀四。櫃之前,硃漆金妝雲龍輅牌一,金塗鐵曲戌。輅之箱,四傍雕鎪漆畫填心隔窠龜文花板,上層左畫青龍,右畫白虎,前畫硃雀,後畫玄武。輅之前額,描金象牙雕行龍二,奉一水精珠,後額如之。前兩柱黃絨鈴索五,貼金鸞和大響銅鈴十,金塗鍮石雙魚五。下硃漆軾櫃一,櫃上金香球一,金香寶一,金香合一,銀灰盤一,並黃紵絲綬帶。輅之後,硃漆後轛一,金塗曲戌,黃紵絲銷金雲龍門簾一,緋紵絲繡雲龍帶二。輅之中,黃金妝鉸描金象牙雕龍椅一,靠背上金塗圈焰玉明珠一。左建太常旗一,十有二斿,黃羅繡日、月、五星、升龍。右建闒戟一,九斿,黃羅繡雲龍。中央黃羅繡青黑黼文兩旗,綢杠,並黃羅。旗首金塗鍮石龍頭二,金塗銅鈴二,金塗鍮石鈸黃纓緌十二重,金塗木珠流蘇十二重。龍椅上,金錦方坐一,綠可貼褥一。勾闌內,可貼條褥四,藍紵絲條褥四,黃漆金塗鍮石鉸葉踏道一,小可貼條褥五重。黃漆木塗金龍頭行馬一,小黃漆梯一,黃漆柄金塗長托叉二,短托叉二,金塗首黃漆推竿一,黃絨引輅索二,金塗銅環二,黃絨執綏一。輅馬、誕馬,皆黃色。鞍轡鞦勒纓拂套項,並金妝,黃韋。誕馬銀褐織金紵絲屜四副,黃羅銷金黃絹里籠鞍六。蓋輅黃絹大蒙帕一,黃油絹帕一。駕士平巾大袖,並黃繡紵絲為之。



 革輅。白質,金妝,青綠藻井,栲栳輪蓋。外施金妝雕木雲龍,內盤描金白檀雕福海圓龍一,頂上匝以金塗鍮石耀葉八十有一。上圍九者二,中圍九者三,下圍九者四。頂輪衣三重,上二重素白繡雲龍瑞草,下一重無文。輪衣內黃屋一,黃素紵絲瀝水,下垂硃絲結網一遭,素白紵絲繡小帶四十有八,帶頭綴金塗小銅鈴三百,素白紵絲繡絡帶二。頂輪平素面夾用白素紵絲。蓋之四周垂流蘇八,飾以五色絨線結網五重,金塗銅鈸五,金塗木珠二十有五。又系金塗鍮石雜佩八,珩璜沖瑀全,金塗鍮石鉤掛十有六,黃絨貫頂天心直下十字繩二。蓋下立硃漆柱四,柱下直平盤,虛櫃,中欞三十,下外桄二,漆繪革鞔犀、象、鸚鵡、錦雉、孔雀,隔窠嵌妝花板。櫃上周遭硃漆勾闌,雲拱地霞葉百七十有九,下垂牙護泥虛板,並硃漆畫瑞草。勾闌上描金白檀行龍十,擺白蹲龍十,孔雀羽臺九,水精面火珠七,金圈焰銅照八。輿下垂硃絲結網一遭,飾以金塗鍮石鐸子三百,彩畫鍮石梅萼嵌網眼中。輿之長轅三,界轅勾心各三,上下龍頭六。前轅引手擺白螭頭三,並系以蹲龍。後轅方罨頭三,桄頭十有六,系以蹲龍三。轅頭衡一,兩端擺白龍頭二,上列金塗銅鳳十二,含以金塗銅鈴。輿之軸一,輪二。軸之挲羅二,明轄蹲龍絟,皆漆以白。其輪之輻各二十有四,轂首壓貼金塗銅轂葉八十有一,金塗鍮石擎耳戀攀四。櫃之前,硃漆金妝雲龍輅牌一,金塗鐵曲戌。輅箱之四傍,雕鎪革鞔漆畫填心,隔窠龜文花板,上層左畫青龍,右畫白虎,前畫硃雀,後畫玄武。輅之前額,白檀行龍二,奉一水精珠,後額如之。前兩柱素白絨鈴索五,帖金鸞和大響銅鈴十,金塗鍮石雙魚五。下硃漆革鞔軾櫃一,櫃上金香球一,金香寶一,金香合一,銀灰盤一,皆黃紵絲綬帶。輅之後,硃漆革鞔後轛一,金塗曲戌,黃紵絲銷金雲龍門簾一,緋紵絲繡雲龍帶二。輅之中,金妝鉸白檀雕龍椅一,靠背上金塗圈焰玉明珠一。右建太常旗一,十有二斿,白羅繡日、月、五星、升龍。右建闒戟一,九斿,素白羅繡雲龍。中央黃羅繡青黑黼文兩旗,綢杠,並素白羅,旗首金塗鍮石龍頭二,金塗銅鈴二,金塗鍮石鈸素白纓緌十有二重,金塗木珠流蘇十有二重。龍椅上,金錦方座一,綠可貼褥一,銷金黃羅夾帕一,方輿地金錦褥一,綠可貼褥一。勾闌內,可貼條褥五重。素白漆雕木塗金龍頭行馬一,小白漆梯一,白漆柄金塗長托叉二,短托叉二,金塗首白漆推竿一,白絨引輅索二,金塗銅環二,黃絨執綏一。輅馬、誕馬,皆白色。鞍轡鞦勒纓拂套項,皆白圍,金妝。誕馬白織金紵絲屜四副,白羅銷金白絹里籠鞍六。蓋輅黃絹大蒙帕一,黃油絹帕一。駕士平巾大袖,皆白繡紵絲為之。



 木輅。黑質,金妝,青綠藻井,栲栳輪蓋。外施金妝雕木雲龍,內盤描金紫檀雕福海圓龍一,頂上匝以金塗鍮石耀葉八十有一。上圍九者二,中圍九者三,下圍九者四。頂輪衣三重,上二重皁繡雲龍瑞草,下一重無文。輪衣內黃屋一,黃素紵絲瀝水,下垂硃絲結網一遭,皁紵絲繡小帶四十有八,帶頭綴金塗小銅鈴三百,皁紵絲繡絡帶二。頂輪平素面夾用檀褐紵絲。蓋之四周垂流蘇八,飾以五色絨線結網五重,金塗銅鈸五,金塗木珠二十五。又系金塗鍮石雜佩八,珩璜沖瑀全,金塗鍮石鉤掛十有六,黃絨貫頂天心直下十字繩二。蓋下立硃漆柱四,柱下直平盤,虛櫃,中欞三十,下外桄二,漆繪犀、象、鸚鵡、錦雉、孔雀,隔窠嵌妝花板,櫃上周遭硃漆勾闌,雲拱地霞葉百七十有九,下垂牙護泥虛板,皆硃漆畫瑞草。勾闌上金嵌鑌鐵行龍十,蹲龍十,孔雀羽臺九,水精面火珠七,金圈焰銅照八。輿下垂硃絲結網一遭,飾以金塗鍮石鐸子三百,彩畫鍮石梅萼嵌網眼中。輿之長轅三,界轅勾心各三,上下龍頭六。前轅引手金嵌鑌鐵螭頭三,皆糸全以蹲龍。後轅方罨頭三,桄頭十有六,系以蹲龍三。轅頭衡一,兩端金嵌鑌鐵龍頭二,上列金塗銅鳳十二,含以金塗銅鈴。輿之軸一,輪二。軸之挲羅二,明轄蹲龍絟,並漆以黑。輪之輻各二十有四,轂首壓貼金塗銅轂葉八十有一,金塗鍮石擎耳戀攀四。櫃之前,硃漆金妝雲龍輅牌一,金塗鐵曲戌。輅之箱,四傍雕鎪漆畫填心,隔窠龜文花板,上層左畫青龍,右畫白虎,前畫硃雀,後畫玄武。輅之前額,金嵌鑌鐵行龍二,奉一水精珠,後額如之。前兩柱皁絨鈴索五,貼金鸞和大響銅鈴十,金塗鍮石雙魚五。下硃漆軾櫃一,櫃上金香球一,金香寶一,金香合一,銀灰盤一,皆黃紵絲綬帶。輅之後,硃漆後轛一,金塗曲戌,黃紵絲銷金雲龍門簾一,緋紵絲繡雲龍帶二。輅之中,金妝烏木雕龍椅一,靠背上金塗圈焰玉明珠一。左建太常旗一,十有二斿,皁羅繡日、月、五星、升龍。右建闒戟一,九斿,皁羅繡雲龍。中央黃羅繡青黑黼文兩旗,綢杠,並皁羅,旗首金塗鍮石鈸紫纓緌十有二重,金塗流蘇十有二重。龍椅上,金錦方座一,綠可貼褥一,銷金黃罷夾帕一,方輿地金錦褥一,綠可貼褥一。勾闌內,可貼條褥四,藍紵絲條褥四,黑漆金塗鍮石鉸葉踏道一,小可貼條褥五重。黑漆雕木塗金龍頭行馬一,小黑漆梯一,黑漆柄金塗長托叉二,短托叉二,金塗首黑漆推竿一,皁絨引輅索二,金塗銅環二,黃絨執綏一。輅馬、誕馬,並黑色。鞍轡鞦勒纓拂套項,皆以淺黑韋,金妝。誕馬紫織金紵絲屜四副,紫羅銷金紫絹里籠鞍六。蓋輅黃絹大蒙帕一,黃油絹帕一。駕士平巾大袖,皆紫繡紵絲為之。



 腰輿。制以香木,後背作山字牙,嵌七寶妝雲龍屏風,上施金圈焰明珠,兩傍引手。屏風下施雕鏤雲龍床。坐前有踏床,可貼錦褥一。坐上貂鼠緣金錦條褥,綠可貼方坐。



 象轎。駕以象,凡巡幸則御之。



\end{pinyinscope}