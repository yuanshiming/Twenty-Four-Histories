\article{本紀第四十二 順帝五}

\begin{pinyinscope}

 九年春正月丁酉,享於太廟。癸卯,立山東河南等處行都水監,專治河患。乙巳,廣西徭賊復陷道州車爾尼雪夫斯基(dIWMVKXOKSLIVMSITeRLNfF,萬戶鄭均擊走之。丙午,命中書平章政事太不花提調會同館。庚戌,太白犯建星。辛亥,太白犯平道。



 二月戊辰,祭社稷。辛巳,太不花辭職,不允。甲申,太陰犯建星。



 三月丁酉,壩河淺澀,以軍士、民夫各一萬浚之。己亥,太白犯壘壁陣。己巳,命大司農達識帖睦邇為湖廣行省平章政事。是月,河北潰。陳州麒麟生,不乳而死。徭賊吳天保復寇沅州。



 夏四月丁卯,享於太廟。丁丑,以知樞密院事欽察臺為中書平章政事。己卯,以燕南廉訪使韓元善為中書左丞。立鎮撫司於直沽海津鎮。壬午,以河間鹽運司水災,住煎鹽三萬引。是月,車駕時巡上都。



 五月戊戌,命太傅脫脫提調斡耳朵內史府。庚子,詔修黃河金堤,民夫日給鈔三貫。辛丑,罷瑞州路上高縣長官司。庚戌,命翰林國史院等官薦舉守令。丙辰,定守令督攝之法,路督攝府,府督攝州,州督攝縣。是月,白茅河東注沛縣,遂成巨浸。蜀江大溢,浸漢陽城,民大饑。六月丙子,刻小玉印,以「至正珍秘」為文,凡秘書監所掌書,盡皆識之。



 秋七月庚寅,監察御史斡勒海壽劾奏殿中侍御史哈麻及其弟雪雪罪惡,御史大夫韓嘉訥以聞,不省,章三上,詔奪哈麻、雪雪官,出海壽為陜西廉訪副使,韓家訥為宣政院使。壬辰,詔命太子愛猷識理達臘習學漢人文書,以李好文為諭德,歸暘為贊善,張仲為文學。李好文等上書辭,不許。賜公主不答昔你平江田五十頃。甲午,以也先帖木兒為御史大夫。乙未,以湖廣行省左丞相亦憐真班知樞密院事。丙午,太陰犯壘壁陣。癸丑,太陰犯天關。甲寅,以柏顏為集賢大學士。乙卯,罷右丞相朵兒只,依前為國王,左丞相太平為翰林學士承旨。是月,大霖雨,水沒高唐州城;江、漢溢,漂沒民居、禾稼。



 閏月辛酉,詔脫脫為中書左丞相,仍太傅;韓家訥為江浙行省平章政事。庚午,以也可札魯忽赤搠思監為中書右丞,同知樞密院事,玉樞虎兒吐華為中書參知政事。辛巳,詔赦湖廣徭賊詿誤者。戊子,命岐王阿剌乞鎮西番。八月甲辰,以集賢大學士柏顏為中書平章政事,河南行省平章政事月魯不花為宣政院使。庚戌,以司徒雅普化提調太史院、知經筵事。是月,車駕還自上都。九月甲子,凡建言中外利害者,詔委官選其可行之事以聞。丙寅,命平章政事柏顏提調留守司。丙子,中書平章政事定住以疾辭職,不允。辛巳,命知樞密院事亦憐真班提調武備寺。丙戌,熒惑犯靈臺。是月,遣御史中丞李獻代祀河瀆。



 冬十月辛卯,享於太廟。丁酉,命皇太子愛猷識理達臘自是日為始入端本堂肄業。命脫脫領端本堂事,司徒雅普化知端本堂事。端本堂虛中座,以俟至尊臨幸,太子與師傅分東西向坐授書,其下僚屬以次列坐。



 十一月戊午朔,日有食之。戊辰,太陰犯畢宿。庚辰,太白犯壘壁陣。



 十二月戊戌,太白復犯壘壁陣。丁未,徭賊吳天保陷辰州。是歲,詔汰冗官,均俸祿,賜致仕官及高年帛。漕運使賈魯建言便益二十餘事,從其八事:其一曰京畿和糴,二曰優恤漕司舊領漕戶,三曰接運委官,四曰通州總治豫定委官,五曰船戶困於壩夫、海糧壞於壩主戶,六曰疏浚運河,七曰臨清運糧萬戶府當隸漕司,八曰宜以宣忠船戶付本司節制。冀寧平遙等縣曹七七反,命刑部郎中八十、兵馬指揮沙不丁討平之。



 十年春正月丙辰朔,以中書右丞搠思監為平章政事,玉樞虎兒吐華為中書右丞。壬戌,立四川容美洞軍民總管府。壬申,太陰犯熒惑。甲戌,隕石棣州,色黑,中微有金星;先有聲自西北來,至州北二十里乃隕。



 二月丙戌,詔加封天妃父種德積慶侯,母育聖顯慶夫人。辛丑,太陰犯平道。甲辰,太陰犯鍵閉。



 三月己卯,熒惑犯太微垣。是月,奉化州山石裂,有禽鳥、草木、山川、人物之形。



 夏四月己丑,左司都事武祺建言更鈔法。丁酉,赦天下,其略曰:「朕纂承洪業,撫臨萬邦,夙夜厲精,靡遑暇逸。比緣倚注失當,治理乖方,是用圖任一相,俾贊萬機。爰命脫脫為中書右丞相,統正百官,允厘庶績,曾未期月,百度具舉,中外協望,朕甚嘉焉。尚慮軍國之重,民物之繁,政令有未孚,生意有未遂,可赦天下。」丙午,太白犯鬼宿。是月,車駕時巡上都。



 六月壬子,星大如月,入北斗,震聲若雷,三日復還。



 秋七月辛酉,太陰犯房宿。癸亥,以大護國仁王寺昭應宮財用規運總管府仍屬宣政院。辛未,太白晝見。丁丑,太白復晝見。



 八月壬寅,車駕還自上都。九月癸丑朔,太白晝見。辛酉,祭三皇,如祭孔子禮。先是,歲祀以醫官行事,江西廉訪使文殊訥建言,禮有未備,乃敕工部具祭器,江浙行省造雅樂,太常定儀式,翰林撰樂章,至是用之。壬戌,熒惑犯天江。庚午,命樞密院以軍士五百修築白河堤。壬午,脫脫以吏部選格條目繁多,莫適據依,銓選者得以高下之,請編類為成書,從之。



 冬十月癸巳,歲星犯軒轅。乙未,吏部尚書偰哲篤建言更鈔法,命中書省、御史臺、集賢、翰林兩院之臣集議之。丙申,太陰犯昴宿。辛丑,置諸路寶泉都提舉司於京城。是月,大名、東平、濟南、徐州各立兵馬指揮司以捕上馬賊。



 十一月壬子朔,日有食之。丙辰,以高麗沈王之孫脫脫不花等為東宮怯薛官。辛酉,罷遼陽濱海民煎熬野鹽。戊辰,太陰犯鬼宿。己巳,詔天下以中統交鈔一貫文權銅錢一千文,準至元寶鈔貳貫,仍鑄至正通寶錢並用,以實鈔法,至元寶鈔通行如故。是月,三星隕於耀州,化為石,如斧形,削之有屑,擊之有聲。



 十二月壬午朔,修大都城。辛卯,以大司農禿魯等兼領都水監,集河防正官議黃河便益事。命前同知樞密院事不顏不花等討廣西徭賊。乙未,太陰犯鬼宿。己酉,方國珍攻溫州。是歲,京師麗正門樓上忽有人妄言災禍,鞫問之,自稱薊州人,已而不知所往。



 十一年春正月乙卯,享於太廟。丙辰,辰星犯牛宿。庚申,命江浙行省左丞孛羅帖木兒討方國珍。丁卯,蘭陽縣有紅星大如斗,自東南墜西北,其聲如雷。己卯,命搠思監提調大都留守司。



 二月庚寅,太陰犯鬼宿。乙未,太陰犯太微。丁酉,太陰犯亢宿。是月,命游皇城,中書省臣諫止之,不聽。立湖南元帥府分府於寶慶路。



 三月庚戌,立山東分元帥府於登州。丙辰,親策進士八十三人,賜朵烈圖、文允中進士及第,其餘賜出身有差。壬戌,徵建寧處士彭炳為端本堂說書,不至。丁卯,太陰犯東咸。戊辰,太陰犯天江。是月,遣使賑湖南、北被寇人民,死者鈔五錠,傷者三錠,毀所居屋者一錠。



 夏四月壬午,詔開黃河故道,命賈魯以工部尚書為總治河防使,發汴梁、大名十三路民十五萬,廬州等戍十八翼軍二萬,自黃陵岡南達白茅,放於黃固、哈只等口,又自黃陵西至陽青村,合於故道,凡二百八十里有奇,仍命中書右丞玉樞虎兒吐華、同知樞密院事黑廝以兵鎮之。冀寧路屬縣多地震,半月乃止。乙酉,享於太廟。詔加封河瀆神為靈源神佑弘濟王,仍重建河瀆及西海神廟。改永順安撫司為宣撫司。丁酉,孟州地震。庚子,罷海西遼東道巡防捕盜所,立鎮寧州。辛丑,師壁安撫司土官田驢什用、盤順府土官墨奴什用降,立長官司四、巡檢司七。乙巳,彰德路雨雹,形如斧,傷人畜。是月,罷沂州分元帥府,改立兵馬指揮使司,復分司於膠州。車駕時巡上都。



 五月己酉朔,日有食之。辛亥,潁州妖人劉福通為亂,以紅巾為號,陷潁州。初,欒城人韓山童祖父,以白蓮會燒香惑眾,謫徙廣平永年縣。至山童,倡言天下大亂,彌勒佛下生,河南及江淮愚民皆翕然信之。福通與杜遵道、羅文素、盛文鬱、王顯忠、韓咬兒復鼓妖言,謂山童實宋徽宗八世孫,當為中國主。福通等殺白馬、黑牛,誓告天地,欲同起兵為亂,事覺,縣官捕之急,福通遂反。山童就擒,其妻楊氏,其子韓林兒,逃之武安。癸丑,文水縣雨雹。壬申,命同知樞密院事禿赤以兵討劉福通,授以分樞密院印。丙子,命大都至汴梁二十四驛,凡馬一匹助給鈔五錠。六月,發軍一千,從直沽至通州,疏浚河道。是月,劉福通據硃皋,攻破羅山、真陽、確山,遂犯武陽、葉縣等處。江浙左丞孛羅帖木兒為方國珍所敗。



 秋七月丙辰,廣西大水。丁巳,罷四川大奴管勾洞長官司,改立忠孝軍民府。己未,太陰犯鬥宿。壬戌,太白犯右執法,己巳,太白犯左執法。熒惑入鬼宿。是月,開河功成,乃議塞決河。命大司農達識帖睦邇及江浙行省參知政事樊執敬、浙東廉訪使董守愨同招諭方國珍。



 八月丁丑朔,中興地震。戊寅,祭社稷。乙酉,太陰犯天江。丙戌,蕭縣李二及老彭、趙君用攻陷徐州。李二號芝麻李,與其黨亦以燒香聚眾而反。是月,車駕還自上都。蘄州羅田縣人徐貞一,名壽輝,與黃州麻城人鄒普勝等,以妖術陰謀聚眾,遂舉兵為亂,以紅巾為號。九月戊申,以中書平章政事朵兒直班提調宣文閣、知經筵事,平章政事定住提調會同館事。壬子,命御史大夫也先帖木兒知樞密院事,及衛王寬徹哥總率大軍出征河南妖寇,各賜鈔一千錠,從征者賜予有差。乙卯,辰星犯左執法。丁巳,太白犯房宿。壬戌,詔以高麗國王不答失里之弟伯顏帖木兒襲其王封,不答失里之子遂廢。戊辰,太陰犯鬼宿。是月,劉福通陷汝寧府及息州、光州,眾至十萬。徐壽輝陷蘄水縣及黃州路。



 冬十月戊寅,熒惑犯太微垣。己卯,享於太廟。辛巳,太陰犯鬥宿。癸未,立寶泉提舉司於河南行省及濟南、冀寧等路凡九,江浙、江西、湖廣行省等處凡三。命知樞密院事老章以兵同也先帖木兒討河南妖寇。乙酉,太白犯鬥宿。己丑,太白晝見,熒惑犯歲星。辛卯,太白犯鬥宿。立中書分省於濟寧。癸巳,歲星犯右執法。癸卯,以宗王神保克復睢寧、虹縣有功,賜金帶一,從征者賞銀有差。丙午,熒惑犯左執法。是月,天雨黑子於饒州,大如黍菽。徐壽輝據蘄水為都,國號天完,僭稱皇帝,改元治平,以鄒普勝為太師。



 十一月癸酉,有星孛于婁宿。甲寅,孛星見於胃宿。乙卯、丙辰,亦如之。丁巳,太陰犯填星,孛星微見於畢宿。黃河堤成,散軍民役夫。庚午,監察御史徹徹帖木兒等言,右丞相脫脫治河功成,宜有異數以旌其勞。甲戌,江西妖人鄧南二作亂,攻瑞州,總管禹蘇福擒斬之。是月,遣使以治河功成告祭河伯,召賈魯還朝。超授榮祿大夫、集賢大學士,賜金系腰一、銀十錠、鈔千錠、幣帛各二十匹。都水監並有司官有功者三十七員,皆升遷其職。詔賜脫脫答剌罕之號,俾世襲之,以淮安路為其食邑。命立《河平碑》。



 十二月丙子朔,太白晝見。丁丑,太白經天。己卯,立河防提舉司,隸行都水監。庚辰,太白經天,是夜,犯壘壁陣。甲申,太陰犯填星。丙戌,太白復經天,是夜,復犯壘壁陣。以治書侍御史烏古孫良楨為中書參知政事。辛卯,太白經天,壬辰,復如之。丁酉,太白晝見,太陰犯熒惑。命脫脫於淮安立諸路打捕鷹房民匠錢糧總管府,秩從三品。庚子,太白經天,辰星犯天江。辛丑,太白經天。也先帖木兒復上蔡縣,擒韓咬兒等至京師,誅之。壬寅,太白晝見。是歲,括馬。



 十二年春正月丙午朔,詔印造中統元寶交鈔一百九十萬錠、至元鈔十萬錠。戊申,竹山縣賊陷襄陽路,總管柴肅死之。是日,荊門州亦陷。己酉,時享太廟。庚戌,以宣政院使月魯不花為中書平章政事。壬子,中書省臣言:「河南、陜西、腹裏諸路,供給繁重,調兵討賊,正當春首耕作之時,恐農民不能安於田畝,守令有失勸課,宜委通曉農事官員,分道巡視,督勒守令,親詣鄉都,省諭農民,依時播種,務要人盡其力,地盡其利。其有曾經盜賊、水患、供給之處,貧民不能自備牛、種者,所在有司給之。仍令總兵官,禁止屯駐軍馬,毋得踏踐,以致農事廢弛。」從之。乙卯,淮東宣慰司添設同知宣慰司事及都事各一員。丙辰,徐壽輝遣偽將丁普郎、徐明遠陷漢陽。丁巳,陷興國府。己未,徐壽輝遣鄒普勝陷武昌,威順王寬徹普化、湖廣行省平章政事和尚棄城走。刑部尚書阿魯收捕山東賊,給敕牒十一道,使分賞有功者。辛酉,徐壽輝偽將魯法興陷安陸府,知府丑驢戰不勝,死之。癸亥,刑部添設尚書、侍郎、郎中、員外郎各一員,五愛馬添設忽剌罕赤一百名。乙丑,太陰犯熒惑。丙寅,以河復故道,大赦天下。己巳,歲星犯右執法。辛未,徐壽輝兵陷沔陽府。壬申,中興路陷,山南宣慰司同知月古輪失領兵出戰,眾潰,宣慰使錦州不花、山南廉訪使卜禮月敦皆遁走。是月,命逯魯曾為淮東添設元帥,統領兩淮所募鹽丁五千討徐州。拘刷河南、陜西、遼陽三省及上都、大都、腹裏等處漢人馬。命四川行省平章政事月魯帖木兒為總兵官,與四川行省右丞長吉討興元、金州等處賊;宣政院同知桑哥率領亦都護畏吾兒軍與荊湖北道宣慰使朵兒只班同守襄陽;濟寧兵馬指揮使寶童統領右都衛軍,從知樞密院事月闊察兒討徐州。



 二月乙亥朔,詔許溪洞蠻徭自新。丁丑,以集賢大學士賈魯為中書添設左丞;以河南廉訪使哈藍朵兒只為荊湖北道宣慰使都元帥,守襄陽。癸未,命諸王禿堅領從官百人,馳驛守揚州,賜金一錠、鈔一千錠。命西寧王牙安沙鎮四川。賜鎮南王孛羅不花鈔一萬錠。甲申,鄒平縣馬子昭為亂,捕斬之。乙酉,徐壽輝兵陷江州,總管李黼死之,遂陷南康路。丙戌,霍州靈石縣地震。徐壽輝兵陷岳州,房州賊陷歸州。戊子,詔:「徐州內外群聚之眾,限二十日,不分首從,並與赦原。」置安東、安豐分元帥府。己丑,游皇城。庚寅,太陰犯太微垣。癸巳,太陰犯氐宿。辛丑,鄧州賊王權、張椿陷澧州,龍鎮衛指揮使俺都剌哈蠻等帥師復之。褒贈伏節死義宣徽使帖木兒等二十七人。壬寅,以御史大夫納麟為江南行臺御史大夫,仍太尉。命翰林學士承旨八剌與諸王孛蘭奚領軍守大名。癸卯,命中書平章政事月魯不花知經筵事,左丞賈魯、參知政事帖理帖木兒、烏古孫良楨並同知經筵事。是月,賊侵滑、浚,命德住為河南右丞,守東明。德住時致仕於家,聞命,馳至東明,浚城隍,嚴備御,賊不敢犯。徐壽輝偽將歐普祥陷袁州。命帖理帖木兒以中書參知政事分省濟寧。



 三月乙巳朔,追封太師、忠王馬扎兒臺為德王。丁未,徐壽輝偽將許甲攻衡州,洞官黃安撫敗之。徐壽輝偽將陶九陷瑞州,總管禹蘇福、萬戶張岳敗之。壬子,河南左丞相太不花克復南陽等處。癸丑,中書省臣請行納粟補官之令:「凡各處士庶,果能為國宣力,自備糧米供給軍儲者,照依定擬地方實授常選流官,依例升轉、封廕;及已除茶鹽錢穀官有能再備錢糧供給軍儲者,驗見授品級,改授常流。」從之。戊午,太陰犯進賢。辛酉,命親王阿兒麻以兵討商州等處賊。以鞏卜班知行樞密院事。壬戌,太陰犯東咸。甲子,徐壽輝偽將項普略陷饒州路,遂陷徽州、信州。四川未附生蠻向亞甲洞主墨得什用出降,立盤順府。丁卯,江南行臺御史大夫帖木哥乞致仕,不允,以為甘肅行省平章政事。以出征馬少,出幣帛各一十萬匹,於迤北萬戶、千戶所易馬。戊辰,太白晝見。詔:「南人有才學者,依世祖舊制,中書省、樞密院、御史臺皆用之。」中書省臣言:「張理獻言,饒州德興三處,膽水浸鐵,可以成銅,宜即其地各立銅冶場,直隸寶泉提舉司,宜以張理就為銅冶場官。」從之。以江浙行省左丞相亦憐真班為江西行省左丞相,領兵收捕饒、信賊。庚午,詔:「隨朝一品職事及省、臺、院、六部、翰林、集賢、司農、太常、宣政、宣徽、中政、資正、國子、秘書、崇文、都水諸正官,各舉循良材幹、智勇兼全、堪充守令者二人。知人多者,不限員數。各處試用守令,並授兼管義兵防禦諸軍奧魯勸農事,所在上司不許擅差。守令既已優升,其佐貳官員,比依入廣例,量升二等。任滿,驗守令全治者,與真授;不治者,全削二等,依本等敘;半治者,減一等敘。雜職人員,其有知勇之士,並依上例。凡除常選官於殘破郡縣及迫近賊境之處,升四等;稍近賊境,升二等。」是月,方國珍復劫其黨下海,入黃巖港,臺州路達魯花赤泰不花率官軍與戰,死之。隴西地震百餘日,城郭頹夷,陵穀遷變,定西、會州、靜寧、莊浪尤甚。會州公宇中墻崩,獲弩五百餘張,長者丈餘,短者九尺,人莫能挽。改定西為安定州,會州為會寧州。詔定軍民官不守城池之罪。



 閏三月辛巳,以臺州路達魯花赤泰不花為江浙行省參知政事,行臺州路事,命下,泰不花已死。壬午,以大理宣慰使答失八都魯為四川行省添設參知政事,與本省平章政事咬住討山南、湖廣等處賊。乙酉,徐壽輝偽將陳普文陷吉安路,鄉民羅明遠起義兵復之。命工部尚書朵來、兵部侍郎馬某火者,分詣上都、察罕腦兒、集寧等處,給散出征河南達達軍口糧。立淮南江北等處行中書省,治揚州,轄揚州、高郵、淮安、滁州、和州、廬州、安豐、安慶、蘄州、黃州。壬辰,以大都留守兀忽失為江浙行省添設右丞,討饒、信賊。丙申,阿速愛馬里納忽臺擒滑州、開州賊韓兀奴罕有功,授資用庫大使。丁酉,湖廣行省參知政事鐵傑,以湖南兵復岳州。戊戌,詔淮南行省設官二十五員,以翰林學士承旨晃火兒不花、湖廣平章政事失列門並為平章政事,淮東元帥蠻子為右丞,燕南廉訪使秦從德為左丞,陜西行臺侍御史答失禿、山北廉訪使趙璉並為參知政事。庚子,以樞密副使悟良哈臺為中書添設參知政事、同知經筵事。辛丑,命淮南行省平章政事晃火兒不花提調鎮南王傅事。是月,詔四川行省平章政事咬住以兵東討荊襄賊,克復忠、萬、夔、雲陽等州;命江西行省左丞相亦憐真班以兵守江東、西關隘;命諸王亦憐真班、愛因班,參知政事也先帖木兒與陜西行省平章政事月魯帖木兒討南陽、襄陽賊,刑部尚書阿魯討海寧賊,江西行省右丞火你赤與參知政事朵鷿討江西賊;以浙東宣慰使恩寧普代江浙行省左丞左答納失裡守蕪湖。命江西行省右丞兀忽失、江浙行省左丞老老與星吉、不顏帖木兒、蠻子海牙同討饒、信等處賊。方國珍不受招安之命,命江浙左丞左答納失裡討之。命典瑞院給淮南行省銀字圓牌三面、驛券五十道。詔江西行省左丞相亦憐真班、淮南行省平章政事晃火兒不花、江浙行省左丞左答納失里、湖廣行省平章政事也先帖木兒、四川行省平章政事八失忽都及江南行臺御史大夫納麟與江浙行省官,並以便宜行事。也先帖木兒駐軍沙河,軍中夜驚,軍潰,退屯硃仙鎮。詔以中書平章政事蠻子代總其兵,也先帖木兒還京師,仍命為御史大夫。



 夏四月癸卯朔,日有食之。江西臨川賊鄧忠陷建昌路。己酉,時享太廟。甲寅,以御史大夫搠思監為中書平章政事,提調留守司。乙卯,鐵傑及萬戶陶夢楨復武昌、漢陽,尋再陷。丙辰,江西宜黃賊塗佑與邵武建寧賊應必達等攻陷邵武路,總管吳按攤不花以兵討之,千戶魏淳以計擒塗佑、應必達,復其城。辛酉,翰林學士承旨渾都海牙乞致仕,不允,以為中書平章政事。四川行省參知政事桑哥失里復渠州。甲子,翰林學士承旨歐陽玄以湖廣行省右丞致仕,賜玉帶及鈔一百錠,給全俸終其身。戊辰,諸王禿堅帖木兒、平章政事也先帖木兒討和州有功,各賜金系腰並鈔一千錠。辛未,荊門知州聶炳復荊門州。平章政事忽都海牙年老有疾,詔免其朝賀。是月,大駕時巡上都。永懷縣賊陷桂陽。咬住復歸州,進攻峽州,與峽州總管趙餘褫大破賊兵,誅賊將李太素等,遂平之。詔天下完城郭,築堤防。命亦都護月魯帖木兒領畏吾兒軍馬,同豫王阿剌忒納失里、知樞密院事老章討襄陽、南陽、鄧州賊。陜西行臺監察御史蒙古魯海牙、範文等糾言也先帖木兒喪師辱國,乞明正其罪,詔不允。左遷西臺御史大夫朵爾直班為湖廣行省平章政事,蒙古魯海牙十二人為各路添設佐貳官。



 五月癸酉朔,太白犯鎮星。戊寅,命龍虎山張嗣德為三十九代天師,給印章。海道萬戶李世安建言權停夏運,從之。命江南行臺御史大夫納麟給宣敕與臺州民陳子由、楊恕卿、趙士正、戴甲,令其集民丁夾攻方國珍。己卯,咬住復中興路。庚辰,監察御史徹徹帖木兒等言:「河南諸處群盜,輒引亡宋故號以為口實,宜以瀛國公子和尚趙完普及親屬徙沙州安置,禁勿與人交通。」從之。罷幹兒棚等處金銀場課。癸未,建昌民戴良起鄉兵克復建昌路。乙酉,命留守帖木哥與諸王朵兒只守口北龍慶州。是月,答失八都魯至荊門,增募兵,趨襄陽,與賊戰,大敗克之。命左答納失里仍守蕪湖險隘。六月丙午,中書省臣言,大名路開、滑、浚三州、元城十一縣水旱蟲蝗,饑民七十一萬六千九百八十口,給鈔十萬錠賑之。戊申,命治書侍御史杜秉彞、中書參議李稷並兼經筵官。辛亥,太白犯井宿。河南行省左丞匝納祿、參知政事王也速迭兒,並以失誤軍需,左遷添設淮西宣慰使,隨軍供給。命河南行省平章政事禿魯、參知政事李猷供給汝寧軍需。丁巳,賜中書參知政事悟良哈臺珠衣並帽。乙丑,宣讓王帖木兒不花,諸王乞塔歹、曲憐帖木兒及淮南廉訪使班祝兒並平賊有功,賜金系腰、銀、鈔有差。紹慶宣慰使楊延禮不花遙授湖廣左丞,楊伯顏卜花為紹慶宣慰使,換文資;楊城為沿邊溪洞招討使兼征行萬戶,回賜先所拘收牌面。丙寅,紅巾周伯顏陷道州。修太廟西神門。



 秋七月丁丑,時享太廟。庚辰,饒、徽賊犯昱嶺關,陷杭州路。辛巳,命通政院使答兒麻失里與樞密副使禿堅不花討徐州賊,給敕牒三十道以賞功。己丑,湘鄉賊陷寶慶路。庚寅,以殺獲西番首賊功,賜岐王阿剌乞巴鈔一千錠,邠王嵬厘、諸王班的失監、平章政事鎖南班各金系腰一。以征西元帥斡羅為章佩添設少監,討徐州。脫脫請親出師討徐州,詔許之。辛卯,命脫脫臺為行樞密院使,提調二十萬戶,賜金系腰一、銀鈔幣帛有差。丁酉,辰星犯靈臺。以杜秉彞為中書添設參知政事。湖南元帥副使小雲失海牙、總管兀顏思忠復寶慶路。是月,徐壽輝偽將王善、康壽四、江二蠻等陷福安、寧德等縣。



 八月癸卯,命中書參知政事帖理帖木爾、淮南行省右丞蠻子供給脫脫行軍一應所需。方國珍率其眾攻臺州城,浙東元帥也忒迷失、福建元帥黑的兒擊退之。甲辰,以同知樞密院事哈麻為中書添設右丞。齊王失列門獻馬一萬五千匹於京師。賜脫脫金三錠,銀三十錠,鈔一萬錠,幣、帛各一千匹。丁未,日本國白高麗賊過海剽掠,身稱島居民,高麗國王伯顏帖木兒調兵剿捕之,賜金系腰一、鈔二千錠。己酉,命知樞密院事咬咬、中書平章政事搠思監、也可扎魯忽赤福壽,並從脫脫出師征徐州,賜金系腰及銀、鈔、幣、帛有差。翰林學士承旨闊怯鎮遏五投下百姓,賜金系腰一。壬子,以扎撒溫孫為河南行省右丞,偰哲篤為淮南行省左丞,各賜鈔五十錠。丙辰,以禿思迷失為淮南行省平章政事。丁巳,命中書平章政事普化知經筵事。脫脫將出師,六部尚書密邇麻和謨等上言:「大臣天子之股肱,中書庶政之根本,不可以一日離。乞詔留賢相,弼亮天工,如此則內外有兼治之宜,社稷有倚重之寄。」不報。脫脫言,皇后斡耳朵思支用不敷,自今為始,每年宜給金一十錠、銀五十錠。以同知樞密院事雪雪出軍南陽,同知樞密院事禿赤出軍河南,皆有功,各進階榮祿大夫。中書左丞哈麻進階榮祿大夫。庚申,命哈麻等提調各怯薛、各愛馬口糧。丁卯,太白犯歲星。詔:「脫脫以答剌罕、太傅、中書右丞相分省於外,督制諸處軍馬,討徐州。中書省、樞密院、御史臺分官屬從行,稟受節制,爵賞有功,誅殺有罪,綏順討逆,悉聽便宜從事。」是日,發京師。是月,大駕還大都。安陸賊將俞君正復陷荊門州,知州聶炳死之。賊將黨仲達復陷岳州。九月乙亥,俞君正復陷中興,咬住領兵與戰於樓臺,敗績,奔松滋,本路判官上都死之。己卯,監察御史及河南分御史臺、行樞密院、河南廉訪司、鞏昌總帥府、陜西都府、義兵萬戶府等官,交章言御史大夫也先帖木兒出征河南功績。庚辰,賜也先帖木兒金系腰一、金一錠、銀一十錠、鈔五千錠、幣帛各一百匹。癸未,中興義士範中,偕荊門僧李智率義兵復中興路,俞君正敗走,龍鎮衛指揮使俺都剌哈蠻領兵入城,咬住自松滋還,屯兵於石馬。乙酉,脫脫至徐州。丁亥,命知行樞密院事阿剌吉從脫脫討徐州,賜金系腰一,金一錠,銀五錠,鈔、幣有差。辛卯,脫脫復徐州,屠其城,芝麻李等遁走。壬辰,太陰犯軒轅。戊戌,賜哈麻李三百錠買玉帶。己亥,賊攻辰州,達魯花赤和尚擊走之。庚子,詔加脫脫為太師,班師還京。



 冬十月丁未,時享太廟。庚戌,知樞密院事老章進階金紫光祿大夫。命平章定住、右丞哈麻同知經筵事。癸丑,命和糴粟豆五十萬石於遼陽。甲寅,拜知行樞密院事阿乞剌為太尉、淮南行省平章政事。戊午,太陰犯鬼宿。甲子,太陰犯歲星。乙丑,太陰犯亢宿。



 十一月辛未,命江浙行省平章政事慶童收捕常州賊。乙亥,以星吉為江西行省平章政事,出師湖廣。丙子,中書省臣請為脫脫立《徐州平寇碑》及加封王爵。癸未,命江浙行省右丞帖裏帖木兒總兵討方國珍。己丑,以脫脫平徐功,賜金一十錠、銀一百錠、鈔五萬錠、幣帛各三千匹,上表辭,從之。庚寅,太陰犯太微垣。



 十二月壬寅,答失八都魯復襄陽。辛亥,詔以杭、常、湖、信、廣德諸路皆克復,赦詿誤者,蠲其夏稅、秋糧,命有司撫恤其民。辛酉,以湖廣行省參知政事卜顏不花、右丞阿兒灰討徭賊,復湖南潭、岳等處有功,卜顏不花升散階從一品,阿兒灰升正二品。癸未,脫脫言:「京畿近地水利,召募江南人耕種,歲可得粟麥百萬餘石,不煩海運而京師足食。」帝曰:「此事有利於國家,其議行之。」是歲,海運不通。立都水庸田使司於汴梁,掌種植之事。潁州沈丘人察罕帖木兒與信陽州羅山人李思齊同起義兵,破賊有功,授察罕帖木兒中順大夫、汝寧府達魯花赤,李思齊知汝寧府。



\end{pinyinscope}