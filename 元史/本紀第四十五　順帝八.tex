\article{本紀第四十五 順帝八}

\begin{pinyinscope}

 十七年春正月丙子朔,日有食之。以伯顏禿古思為大司徒。辛卯,命山東分省團結義兵,每州添設判官一員,每縣添設主簿一員出《禮記·禮運》:「大道之行也,天下為公,選賢與能,講,專率義兵以事守御,仍命各路達魯花赤提調,聽宣慰使司節制。丙申,監察御史哈剌章言:「淮東道廉訪使褚不華,徇忠盡節,宜加褒贈,優恤其家。」從之。



 二月壬子,賊犯七盤、藍田,命察罕帖木兒以軍會答兒麻亦兒守陜州、潼關;哈剌不花由潼關抵陜西,會豫王阿剌忒納失里及定住等同進討。癸丑,太陰犯五車。以征河南許、亳、太康、嵩、汝大捷,詔赦天下。戊辰,知樞密院事脫脫復邳州,調客省使撒兒答溫等攻黃河南岸賊,大破之。壬申,劉福通遣其黨毛貴陷膠州,僉樞密院事脫歡死之。甲戌,倪文俊陷峽州。是月,李武、崔德陷商州,察罕帖木兒與李思齊以兵自陜、虢援陜西,以察罕帖木兒為陜西行省左丞,李思齊為四川行省左丞。詔以高寶為四川行省參知政事,將兵取中興,不克,賊遂破轆轤關。



 三月乙亥朔,義兵萬戶賽甫丁、阿迷裏丁叛據泉州。庚辰,毛貴陷萊州,守臣山東宣慰副使釋嘉訥死之。壬午,大明兵取常州路。甲申,太陰犯鬼宿。壬辰,歲星犯壘壁陣。甲午,毛貴陷益都路,益王買奴遁,自是山東郡邑皆陷。乙未,以江淮行樞密院副使董摶霄為山東宣慰使。丁酉,毛貴陷濱州。戊戌,以中書平章政事帖裏帖木兒為御史大夫,悟良哈臺、斡欒並為中書平章政事。



 夏四月丙午,監察御史五十九言:「今京師周圍,雖設二十四營,軍卒疲弱,素不訓練,誠為虛設,儻有不測,誠可寒心。宜速選擇驍勇精銳,衛護大駕,鎮守京師,實當今奠安根本、固堅人心之急務。況武備莫重於兵,而養兵莫先於食。今朝廷撥降鈔錠,措置農具,命總兵官於河南克復州郡,且耕且戰,甚合寓兵於農之意。為今之計,權命總兵官從宜於軍官內選委能撫字軍民者,兼路府州縣之職,務要農事有成,軍民得所,則擾民之害亦除,而匱乏之憂亦釋矣。」帝嘉納之。乙卯,毛貴陷莒州。丙辰,京師立便民六庫,倒易昏鈔。辛酉,以咬咬為甘肅行省左丞相。答失八都魯加太尉、四川行省左丞相。漢中道廉訪司糾陜西行省左丞蕭家奴遇賊逃竄,失陷所守郡邑,詔正其罪。是月,車駕時巡上都。封江西行省平章政事火你赤為營國公。大明兵取寧國路。



 五月乙亥朔,命知樞密院事孛蘭奚進兵討山東。戊寅,平章政事亦老溫帖木兒復武安州等三十餘城。丙申,命搠思監為右丞相,太平為左丞相,詔天下。免民今歲稅糧之半。詔以永昌宣慰司屬詹事院。六月甲辰朔,以實理門為中書分省右丞,守濟寧。丙辰,監察御史脫脫穆而言:「去歲河南之賊窺伺河北,惟河南與山東互相策應,為害尤大。為今之計,中書當遴選能將,就太不花、答失八都魯、阿魯三處軍馬內,擇其精銳,以守河北,進可以制河南之侵,退可以攻山東之寇,庶幾無虞。」從之。己未,以帖裏帖木兒、老的沙並為御史大夫。庚申,大明兵取江陰州。壬申,帖裏帖木兒糾陜西知行樞密院事也先帖木兒,遂命罷陜西行樞密院,令也先帖木兒居於草地。癸酉,溫州路樂清江中龍起,颶風作,有火光如球。是月,劉福通犯汴梁,其軍分三道,關先生、破頭潘、馮長舅、沙劉二、王士誠寇晉、冀,白不信、大刀敖、李喜喜趨關中,毛貴據山東,其勢大振。



 秋七月己卯,帖裏帖木兒奏續集《風憲宏綱》。庚辰,大明兵取徽州路。癸未,太白犯鬼宿。甲申,太陰犯鬥宿。乙酉,命右丞相搠思監領宣政院事,平章政事臧卜知經筵事,參知政事李稷同知經筵事,參知政事完者帖木兒兼太府卿。丁亥,填星犯鬼宿。戊子,以李稷為御史中丞。中書省臣言:「山東般陽、益都相次而陷,濟南日危,宜選將練卒,信賞必罰,為保燕、趙計,以衛京師。」不報。己丑,鎮守黃河義兵萬戶田豐叛,陷濟寧路,分省右丞實理門遁,義兵萬戶孟本周攻之,田豐敗走,本周還守濟寧。甲午,以御史中丞完者帖木兒為中書右丞,河南廉訪使俺普為中書參知政事。監察御史迭里彌實、劉傑言:「疆域日蹙,兵律不嚴,陜西、汴梁、淮潁、山東之寇有窺伺燕、趙之志,宜俯詢大臣,共圖克復之宜,預定守備之策。」不報。是月,立四方獻言詳定使司,秩正三品。歸德府知府林茂、萬戶時公權叛,以城降於賊,歸德府及曹州皆陷。



 八月癸卯朔,填星犯鬼宿,太白犯軒轅。癸丑,劉福通兵陷大名路,遂自曹、濮陷衛輝路,答失八都魯之子孛羅帖木兒與萬戶方脫脫擊之。甲子,太陰犯五車。乙丑,以陜西行臺御史中丞伯嘉訥為陜西行省平章政事;淮南行省參知政事余闕為淮南行省左丞;江浙行省參知政事楊完者升左丞;方國珍為江浙行省參知政事,海道運糧萬戶如故。丙寅,慶陽府鎮原州大雹。是月,大駕還自上都。薊州大水。詔知樞密院事紐的該進討山東。大明兵取揚州路。平江路張士誠,俾前江南行臺御史中丞蠻子海牙為書請降,江浙左丞相達識帖睦邇承制令參知政事周伯琦等至平江撫諭之,詔以士誠為太尉,士德為淮南行省平章政事,時士德已為大明兵所擒。九月丙子,命同知樞密院事壽童以兵討冠州。以老的沙為中書省平章政事兼兀良海牙指揮使。甲午,澤州陵川縣陷,縣尹張輔死之。戊戌,太不花復大名路並所屬郡縣。辛丑,詔中書右丞也先不花、御史中丞成遵奉使宣撫彰德、大名、廣平、東昌、東平、曹、濮等處,獎厲將帥。是月,命紐的該加太尉,總諸軍守禦東昌。時田豐據濟、濮,率眾來寇,擊走之。倪文俊謀殺其主徐壽輝,不果,自漢陽奔黃州,壽輝偽將陳友諒襲殺之,友諒遂自稱平章。



 閏九月癸卯,有飛星如盂,青色,光燭地,尾約長尺餘,起自王良,沒於勾陳。監察御史朵兒只等劾奏知樞密院使哈剌八禿兒失陷所守郡縣,詔正其罪。丙午,太陰犯鬥宿。右丞相搠思監、左丞相太平並加開府儀同三司。平章政事完者不花兼大司農。庚申,太陰犯井宿。乙丑,潞州陷。丙寅,賊攻冀寧,察罕帖木兒以兵擊走之。



 冬十月乙亥,熒惑犯氐宿。戊寅,設分詹事院。甲申,太陰掩昴宿。戊戌,曹州賊入太行山。是月,白不信、大刀敖、李喜喜陷興元,遂入鳳翔,察罕帖木兒、李思齊屢擊破之,其黨走入蜀。答失八都魯與知樞密院事答里麻失里以軍討曹州賊,官軍敗潰,答里麻失里死之。靜江路山崩,地陷,大水。



 十一月辛丑朔,山東道宣慰使董摶霄建言:「請令江淮等處各枝官軍,分布連珠營寨於隘口,屯駐守御,宜廣屯田,以足軍食。」從之。汾州桃杏花。壬寅,賊侵壺關,察罕帖木兒大破之。戊午,以河南行省平章政事答蘭為中書平章政事,御史中丞李獻為中書左丞,陜西行臺中丞卜顏帖木兒、樞密院副使哈剌那海、司農少卿崔敬、侍御史陳敬伯皆為參知政事。癸亥,豫王阿剌忒納失里與陜西行省左丞相朵朵、陜西行臺御史中丞伯嘉訥,分道攻討關陜。己巳,以中書參知政事八都麻失里為右丞。



 十二月庚午朔,熒惑犯天江。辛未,山東道廉訪使伯顏不花建言:嚴保伍,集勇健,汰冗官。戊寅,太白犯歲星。甲申,太陰犯鬼宿。丁亥,歲星犯壘壁陣。庚寅,太白犯壘壁陣。癸巳,太陰犯心宿。丁酉,慶元路象山縣鵝鼻山崩。己亥,流星如金星大,尾約長三尺餘,起自太陰,近東而沒,化為青白氣。庚子,答失八都魯卒於軍中。是歲,詔天下團結義兵,路、府、州、縣正官俱兼防御事。詔淮南知行樞密院事脫脫領兵討淮南。詔諭濟寧李秉彞、田豐等,令其出降,敘復元任;嘯亂士卒,仍給資糧,欲還鄉者聽。倪文俊陷川蜀諸郡,命偽元帥明玉珍守據之。趙君用及彭大之子早住同據淮安,趙僭稱永義王,彭僭稱魯淮王。義兵千戶餘寶殺其知樞密院事寶童以叛,降於毛貴。餘寶遂據棣州。河南大饑。



 十八年春正月辛丑,填星犯鬼宿。乙巳,察罕帖木兒、李思齊合兵於鳳翔。丙午,太陰犯昴宿。陳友諒陷安慶路,守將余闕死之。庚戌,大明兵取婺源州。甲子,以不蘭奚知樞密院事。乙丑,大風起自西北,益都土門萬歲碑僕而碎。丙寅,田豐陷東平路。丁卯,不蘭奚與毛貴戰於好石橋,敗績,走濟南。是月,詔答失八都魯子孛羅帖木兒為河南行省平章政事,總領其父元管軍馬。詔察罕帖木兒屯陜西,李思齊屯鳳翔。



 二月己巳朔,議團結西山寨大小十一處以為保障,命中書右丞塔失帖木兒、左丞烏古孫良楨等總行提調,設萬夫長、千夫長、百夫長,編立牌甲,分守要害,互相策應。毛貴陷清、滄州,遂據長蘆鎮。中書省臣奏以陜西軍旅事劇務殷,去京師道遠,供費艱難,請就陜西印造寶鈔為便,遂分戶部寶鈔庫等官,置局印造。仍命諸路撥降鈔本,畀平準行用庫倒易昏幣,布於民間。癸酉,毛貴陷濟南路,守將愛的戰死。毛貴立賓興院,選用故官,以姬宗周等分守諸路;又於萊州立三百六十屯田,每屯相去三十里,造大車百輛,以挽運糧儲,官民田十止收二分,冬則陸運,夏則水運。乙亥,填星犯鬼宿。辛巳,詔以太不花為中書右丞相,總兵山東。壬午,田豐復陷濟寧路。甲申,輝州陷。丙戌,紐的該聞田豐逼近東昌,棄城走。丁亥,察罕帖木兒調兵復涇州、平涼,保鞏昌。戊子,田豐陷東昌路。庚寅,王士誠自益都犯懷慶路,周全擊敗之。辛卯,以安童為中書參知政事。丁酉,興元路陷。



 三月己亥朔,日色如血。加右丞相搠思監太保。庚子,毛貴陷般陽路。辛丑,大同路夜黑氣蔽西方,有聲如雷;少頃,東北方有雲如火,交射中天,遍地俱見火,空中有兵戈之聲。癸卯,王士誠陷晉寧路,總管杜賽因不花死之。甲辰,察罕帖木兒遣賽因赤等復晉寧路。己酉,劉福通遣兵犯衛輝,孛羅帖木兒擊走之。庚戌,毛貴陷薊州,詔征四方兵入衛。乙卯,毛貴犯漷州,至棗林,樞密副使達國珍戰死,遂略柳林,同知樞密院事劉哈剌不花以兵擊敗之,貴走據濟南。丙辰,大明兵取建德路。以周全為湖廣行省參知政事,統奧魯等軍,移鎮嵩州白龍寨。冀寧路陷。丁巳,田豐陷益都路。辛酉,大同諸縣陷,察罕帖木兒遣關保等往擊之。是時賊分二道犯晉、冀,一出沁州,一侵絳州。乙丑,以老章為太子少保。



 夏四月甲申,陳友諒陷龍興路,省臣道童、火你赤棄城遁。壬午,田豐陷廣平路,大掠,退保東昌。詔令元帥方脫脫以兵復廣平。癸未,以諸處捷音屢至,詔頒軍民事宜十一條。庚寅,以翰林學士承旨蠻子為嶺北行省平章政事。辛卯,太白犯鬼宿。甲午,陳友諒遣王奉國陷瑞州路。是月,車駕時巡上都。察罕帖木兒、李思齊會宣慰張良弼、郎中郭擇善、宣慰同知拜帖木兒、平章政事定住、總帥汪長生奴,各以所部兵討李喜喜於鞏昌,李喜喜敗入蜀。察罕帖木兒駐清湫,李思齊駐斜坡,張良弼駐秦州,郭擇善駐崇信,拜帖木兒等駐通渭,定住駐臨洮,各自除路府州縣官,徵納軍需。李思齊、張良弼又同襲殺拜帖木兒,分總其兵。



 五月戊戌朔,察罕帖木兒遣董克昌等以兵復冀寧。以方國珍為江浙行省左丞,兼海道運糧萬戶。詔察罕帖木兒還兵鎮冀寧。李思齊殺同僉樞密院事郭擇善。庚子,賊兵逾太行,察罕帖木兒部將關保擊敗之。以察罕帖木兒為陜西行省右丞兼陜西行臺侍御史、同知河南行樞密院事。劉福通攻汴梁。壬寅,太白犯填星。汴梁守將竹貞棄城遁,福通等遂入城,乃自安豐迎其偽主居之以為都。陳友諒遣康泰、趙琮、鄧克明等以兵寇邵武路。甲辰,命太尉阿吉剌為甘肅行省左丞相。乙巳,關保與賊戰於高平,大敗之。庚戌,陳友諒陷吉安路。壬子,太陰犯鬥宿。癸丑,監察御史七十等,糾劾太保、中書右丞相太不花。乙卯,詔削太不花官爵,安置蓋州。時太不花總兵山東,以知行樞密院悟良哈臺代之。命悟良哈臺節制河北諸軍,河南行省平章政事周全節制河南諸軍。辛酉,陳友諒兵陷撫州路。甲子,監察御史七十、燕赤不花等劾中書參知政事燕只不花。是月,遼州蝗。山東地震,天雨白毛。察罕帖木兒自以劉尚質為冀寧路總管。六月戊辰朔,太不花伏誅。察罕帖木兒調虎林赤、關保同守潞州。拜察罕帖木兒陜西行省平章政事,便宜行事。庚辰,關先生、破頭潘等陷遼州,虎林赤以兵擊走之,關先生等遂陷冀寧路。乙酉,命左丞相太平督諸軍守禦京城,便宜行事。是月,汾州大疫。



 秋七月丁酉朔,周全據懷慶路以叛,附於劉福通。時察罕帖木兒駐軍洛陽,遣伯帖木兒以兵守碗子城。周全來戰,伯帖木兒為其所殺,周全遂盡驅懷慶民渡河,入汴梁。丁未,太陰犯鬥宿。不蘭奚以兵復般陽路,已而復陷。戊申,太白晝見。癸丑,有賊兵犯京城,刑部郎中不花守西門,夜,開門擊退之。己未,劉福通遣周全引兵攻洛陽,守將登城,以大義責全,全愧謝退兵,劉福通殺之。丙寅,以完卜花、脫脫帖木兒為中書平章政事。是月,京師大水,蝗,民大饑。



 八月丁卯朔,江浙行省平章政事三旦八遁於福建。先是,三旦八討饒州,貪財玩寇,久而無功,遂妄稱遷職福建行省。至福建,為廉訪僉事般若帖木兒所劾,拘之興化路。壬申,太陰掩心宿。庚辰,陳友諒兵陷建昌路。辛巳,義兵萬戶王信以滕州叛,降於毛貴。甲申,太陰掩昴宿。庚寅,以老的沙為御史大夫。詔作新風紀。九月丁酉朔,詔授昔班帖木兒同知河東宣慰司事,其妻剌八哈敦雲中郡夫人,子觀音奴贈同知大同路事,仍旌表其門閭。先是,昔班帖木兒為趙王位下同知怯憐口總管府事,其妻嘗保育趙王,及是部落滅里叛,欲殺王,昔班帖木兒與妻謀,以其子觀音奴服王平日衣冠居王宮,夜半,夫妻衛趙王微服遁去。比賊至,遂殺觀音奴,趙王得免。事聞,故旌其忠焉。褒封唐贈諫議大夫劉騕為文節昌平侯。關先生攻保定路,不克,遂陷完州,掠大同、興和塞外諸郡。中書左丞張沖請立團練安撫勸農使司二道,一奉元延安等處,一鞏昌等處,從之。壬寅,詔命中書參知政事普顏不花、治書侍御史李國鳳經略江南。癸卯,詔以福建行中書省平章政事慶童為江南行臺御史大夫。丙午,賊兵攻大同路。壬戌,平定州陷。乙丑,陳友諒陷贛州路,江西行省參知政事全普庵撒里及總管哈海赤死之。



 冬十月丙寅朔,詔豫王阿剌忒納失里徙居白海,尋遷六盤。壬申,大明兵取蘭溪州。己卯,太陰犯昴宿。壬午,監察御史燕赤不花劾右丞相搠思監罪狀,詔收其印綬。乙酉,監察御史答兒麻失里、王彞等復劾之,請正其罪,帝不聽。壬辰,大同路陷,達魯花赤完者帖木兒棄城遁。



 十一月乙未朔,以普化帖木兒為福建行省平章政事。癸卯,陳友諒陷汀州路。丙午,太陰犯昴宿,太白犯房宿。丁未,田豐陷順德路。先是,樞密院判官劉起祖守順德,糧絕,劫民財,掠牛馬,民強壯者令充軍,弱者殺而食之。至是城陷,起祖遂盡驅其民走於廣平。辛酉,太陰掩心宿。



 十二月乙丑朔,日有食之。癸酉,關先生、破頭潘等陷上都,焚宮闕,留七日,轉略往遼陽,遂至高麗。戊寅,太白經天。庚辰,察罕帖木兒遣樞密院判官瑣住進兵於遼陽。癸未,太白經天。甲申,大明兵取婺州路,達魯花赤僧住、浙東廉訪使楊惠死之。戊子,太陰犯房宿。



 十九年春正月甲午朔,陳友諒兵陷信州路,守臣江東廉訪副使伯顏不花的斤力戰死之。大明兵取諸暨州。辛丑,太陰犯昴宿。乙巳,以朵兒只班為中書平章政事。丙午,遼陽行省陷,懿州路總管呂震死之,贈震河南行省左丞,追封東平郡公。察罕帖木兒遣樞密院判官陳秉直、八不沙將兵二萬守冀寧。癸丑,流星如酒杯大,有聲如雷。



 二月辛巳,樞密副使朵兒只以賊犯順寧,命張立將精銳由紫荊關出討,命鴉鶻由北口出迎敵。甲申,叛將梁炳攻辰州,守將和尚擊敗之,以和尚為湖廣行省參知政事。賊由飛狐、靈丘犯蔚州。庚寅,御史臺臣言:「先是召募義兵,費用銀鈔一百四十萬錠,多近侍、權幸冒名關支,率為虛數。乞令軍士,凡已領官錢者,立限出征。」詔從之,已而復止不行。是月,詔孛羅帖木兒移兵鎮大同,以為京師捍蔽。置大都督兵農司,仍置分司十道,專督屯種,以孛羅帖木兒領之,所在侵奪民田,不勝其擾。太不花潰散之兵數萬鈔掠山西,察罕帖木兒遣陳秉直分兵駐榆次招撫之,其首領悉送河南屯種。



 三月癸巳朔,陳友諒遣兵由信州略衢州,復遣兵陷襄陽路。辛丑,京城北兵馬司指揮周哈剌歹與林智和等謀叛,事覺,伏誅。庚戌,太陰犯房宿。壬戌,詔定科舉流寓人名額,蒙古、色目、南人各十五名,漢人二十名。



 夏四月癸亥朔,汾水暴漲。賊陷金、復等州,司徒、知樞密院事佛家奴調兵平之。甲子,毛貴為趙君用所殺。帝以天下多故,卻天壽節朝賀,詔群臣曰:「朕方今宜敬天地,法祖宗,以自修省。朕初度之日,群臣毋賀。」庚午,左丞相太平暨文武百官奏曰:「天壽節朝賀,乃臣子報本,實合禮典。今謙讓不受,固陛下盛德,然今軍旅征進,君臣名分,正宜舉行。」不允。壬申,皇太子復率群臣上奏曰:「朝賀祝壽,是祖宗以來舊行典故,今不行,有乖於禮。」帝曰:「今盜賊未息,萬姓荼毒,正朕恐懼、修省、敬天之時,奈何受賀以自樂!」乙亥,御史大夫帖裏帖木兒復奏曰:「天壽朝賀之禮,蓋出臣子之誠,伏望陛下曲徇所請。若朝賀之後,內庭燕集,特賜除免,亦古者人君減膳之意,仍乞宣示中書,使內外知聖天子憂勤惕厲至於如此。」帝曰:「為朕缺於修省,以致萬姓塗炭,今復朝賀燕集,是重朕之不德。當候天下安寧,行之未晚。卿等其毋復言。」卒不聽。己丑,賊陷寧夏路,遂略靈武等處。



 五月壬辰朔,以陜西行臺御史大夫完者帖木兒為陜西行省左丞相,便宜行事。丙申,熒惑犯鬼宿。丁酉,皇太子奏請巡北邊以撫綏軍民,御史臺臣上疏固留,詔從之。壬寅,察罕帖木兒請今歲八月鄉試河南舉人及避兵儒士,不拘籍貫,依河南省元定額數,就陜州置貢院應試,詔從之。丙午,太陰犯天江。丁未,太陰犯鬥宿。是月,察罕帖木兒大發秦、晉諸軍討汴梁,圍其城。山東、河東、河南、關中等處蝗飛蔽天,人馬不能行,所落溝塹盡平,民大饑。六月辛巳,詔以宣徽使燕古兒為御史大夫。



 秋七月壬辰朔,出搠思監為遼陽行省左丞相,便宜行事。丁酉,太白犯上將。庚子,詔以察罕腦兒宣慰司之地屬資正院,有司毋得差占。察罕腦兒之地,在世祖時隸忙哥歹太子四千戶,今從皇后奇氏請,故以屬之資正院。甲辰,太白犯右執法。戊申,命國王囊加歹、中書平章政事佛家奴、也先不花、知樞密院事黑驢等,統領探馬赤軍進征遼陽。己酉,太白犯左執法。丙辰,趙君用既殺毛貴,其黨續繼祖自遼陽入益都,殺君用,遂與其所部自相仇敵。是月,霸州及介休、靈石縣蝗。



 八月辛酉朔,倪文俊餘黨陷歸州。戊寅,察罕帖木兒督諸將閻思孝、李克彞、虎林赤、賽因赤、答忽、脫因不花、呂文、完哲、賀宗哲、孫翥等攻破汴梁城,劉福通奉其偽主遁,退據安豐。己卯,蝗自河北飛渡汴梁,食田禾一空。詔以察罕帖木兒為河南行省平章政事,兼同知河南行樞密院事、陜西行臺御史中丞,依前便宜行事,仍賜御衣、七寶腰帶,以旌其功。是月,大同路蝗,襄垣縣螟蝝。九月癸巳,以中書平章政事帖裏帖木兒為陜西行省左丞相,便宜行事。乙巳,以湖南、北,江東、西四道廉訪司所治之地皆陷,詔任其所便之地置司。丙午,夜,白虹貫天。丁未,禁軍人不得私殺牛馬。甲寅,太白犯天江。是月,大明兵取衢州路。詔遣兵部尚書伯顏帖木兒、戶部尚書曹履亨,以御酒、龍衣賜張士誠,征海運糧。



 冬十月庚申朔,詔京師十一門皆築甕城,造吊橋。以方國珍為江浙行省平章政事。壬申,太白犯鬥宿。辛巳,流星大如桃。



 十一月癸卯,大明兵取處州路。戊申,陳友諒兵陷杉關。



 十二月戊辰,太白犯壘壁陣。是月,知樞密院事兀良哈臺領太不花軍,其所部方脫脫與弟方伯帖木兒時保遼州,兀良哈臺同唐琰、高脫因等屯孟州,與察罕帖木兒部將八不沙等交兵。已而兀良哈臺獨引達達軍還京師,方脫脫等乃從孛羅帖木兒。皇太子憾太平忤己,以中書左丞成遵、參知政事趙中皆太平所用,使監察御史誣成遵、趙中以贓罪,杖殺之。是歲以後,因上都宮闕盡廢,大駕不復時巡。陳友諒以江州為都,迎偽主徐壽輝居之,自稱漢王。



 二十年春正月己丑朔,察罕帖木兒請以鞏縣改立軍州萬戶府,招民屯種,從之。御史大夫老的沙、御史中丞咬住奏:「今後各處從宜行事官員,毋得陰挾私仇,明為舉索,輒將風憲官吏擅自遷除,侵擾行事,沮壞臺綱。」從之。己亥,太陰犯井宿。癸卯,大寧路陷。壬子,以危素為參知政事。乙卯,會試舉人,知貢舉平章政事八都麻失里、同知貢舉翰林學士承旨李好文、禮部尚書許從宗、考試官國子祭酒張翥、同考官太常博士傅亨等奏:「舊例,各處鄉試舉人,三年一次,取三百名,會試取一百名。今歲鄉試所取,比前數少,止有八十八名,會試三分內取一分,合取三十名,如於三十名外,添取五名為宜。」從之。丙辰,五色雲見移時。



 二月戊午朔,左丞相太平罷為太保,守上都。



 三月戊子朔,田豐陷保定路。彗星見東方。甲午,廷試進士三十五人,賜買住、魏元禮進士及第,其餘出身有差。乙巳,冀寧路陷。壬子,以搠思監為中書右丞相。



 夏四月庚申,命大司農司都事樂元臣招諭田豐,至其軍,為豐所害。丁卯,太陰犯明堂。辛未,僉行樞密院事張居敬復興中州。癸酉,太陰犯東咸。



 五月丁亥朔,日有食之。雨雹。陳友諒殺其偽主徐壽輝於太平路,遂稱皇帝,國號大漢,改元大義,已而回駐於江州。乙未,陳友諒遣羅忠顯陷辰州。己亥,以絆住馬為中書平章政事。壬寅,太陰犯建星。是月,張士誠海運糧十一萬石至京師。閏月己未,以太尉也先帖木兒知經筵事,以甘肅行省左丞相阿吉剌為太尉。乙亥,流星大如桃。六月己丑,命孛羅帖木兒部將方脫脫守禦嵐、興、保德州等處。詔:「今後察罕帖木兒與孛羅帖木兒部將,毋得互相越境,侵犯所守信地,因而仇殺,方脫脫不得出嵐、興州境界,察罕帖木兒亦不得侵其地。」癸巳,太白犯井宿。戊戌,太陰犯建星。是月,大明兵取信州路。



 秋七月辛酉,命遼陽行省參知政事張居敬討義州賊。孛羅帖木兒敗賊王士誠於臺州。乙丑,太陰犯井宿。乙亥,詔孛羅帖木兒總領達達、漢兒軍馬,為總兵官,仍便宜行事。



 八月戊子,命孛羅帖木兒守石嶺關以北,察罕帖木兒守石嶺關以南。辛卯,太陰犯天江。壬辰,加封福建鎮閩王為護國英仁武烈忠正福德鎮閩尊王。乙未,永平路陷。壬寅,填星犯太微。甲辰,太陰犯井宿。詔:「諸處所在權攝官員,專務漁獵百姓,今後非朝廷允許,不得之任。」庚戌,詔江浙行省左丞相達識帖睦邇加太尉兼知江浙行樞密院事,提調行宣政院事,便宜行事。九月乙卯朔,詔遣參知政事也先不花往諭孛羅帖木兒、察罕帖木兒,令講和。時孛羅帖木兒調兵自石嶺關直抵冀寧,圍其城三日,復退屯交城。察罕帖木兒調參政閻奉先引兵與戰,已而各於石嶺關南北守御。壬戌,賊陷孟州,又陷趙州,攻真定路。癸未,賊復犯上都,右丞忙哥帖木兒引兵擊之,敗績。



 冬十月甲申朔,甘露降於國子監大成殿前柏木。以張良弼為湖廣行省參知政事,討南陽、襄樊。詔孛羅帖木兒守冀寧,孛羅帖木兒遣保保、殷興祖、高脫因倍道趨冀寧,守者不納。丙戌,命迭兒必失為太尉,守衛大斡耳朵思。戊子,熒惑犯井宿。己亥,察罕帖木兒遣陳秉直、瑣住等,以兵攻孛羅帖木兒之軍於冀寧,與孛羅帖木兒部將脫列伯戰,敗之。時帝有旨以冀寧畀孛羅帖木兒,察罕帖木兒以為用兵數年,惟藉冀、晉以給其軍,而致盛強,茍奉旨與之,則彼得以足其兵食,乃托言用師汴梁,尋渡河就屯澤、潞拒之,調延安軍交戰於東勝州等處,再遣八不沙以兵援之。八不沙謂彼軍奉旨而來,我何敢抗王命,察罕帖木兒怒,殺之。



 十一月甲寅朔,黃河清,凡三日。孛羅帖木兒以兵侵汾州,察罕帖木兒以兵拒之。癸酉,賊犯易州。



 十二月丙戌,詔:「太廟、影堂祭祀,乃子孫報本重事。近兵興歲歉,品物不能豐備,累朝四祭,減為春秋二祭,今宜復四祭。」後竟不行。辛卯,廣平路陷。是歲,陽翟王阿魯輝帖木兒擁兵數十萬,屯於木兒古徹兀之地,將犯京畿,使來言曰:「祖宗以天下付汝,汝已失其太半;若以國璽付我,我當自為之。」帝遣報之曰:「天命有在,汝欲為則為之。」命知樞密院事禿堅帖木兒等將兵擊之,不克,軍士皆潰,禿堅帖木兒走上都。



\end{pinyinscope}