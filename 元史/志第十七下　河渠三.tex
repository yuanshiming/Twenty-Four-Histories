\article{志第十七下 河渠三}

\begin{pinyinscope}

 ○黃河



 至正四年夏五月,大雨二十餘日,黃河暴溢,水平地深二丈許,北決白茅堤。六月,又北決金堤,並河郡邑濟寧、單州、虞城、碭山、金鄉、魚臺、豐、沛、定陶、楚丘、武城,以至曹州、東明、鉅野、鄆城、嘉祥、汶上、任城等處皆罹水患,民老弱昏墊,壯者流離四方。水勢北侵安山,沿入會通、運河,延袤濟南、河間,將壞兩漕司鹽場,妨國計甚重。省臣以聞,朝廷患之,遣使體量,仍督大臣訪求治河方略。



 九年冬,脫脫既復為丞相,慨然有志於事功,論及河決,即言於帝,請躬任其事,帝嘉納之。乃命集群臣議廷中,而言人人殊,唯都漕運使賈魯,昌言必當治。先是,魯嘗為山東道奉使宣撫首領官,循行被水郡邑,具得修捍成策;後又為都水使者,奉旨詣河上相視,驗狀為圖,以二策進獻:一議修築北堤以制橫潰,其用功省;一議疏塞並舉,挽河使東行以復故道,其功費甚大。至是復以二策封,脫脫韙其後策。議定,乃薦魯於帝,大稱旨。



 十一年四月初四日,下詔中外,命魯以工部尚書為總治河防使,進秩二品,授以銀印。發汴梁、大名十有三路民十五萬人,廬州等戍十有八翼軍二萬人供役,一切從事大小軍民,咸稟節度,便宜興繕。是月二十二日鳩工,七月疏鑿成,八月決水故河,九月舟楫通行,十一月水土工畢,諸掃諸堤成。河乃復故道,南匯於淮,又東入於海。帝遣貴臣報祭河伯,召魯還京師,論功超拜榮祿大夫、集賢大學士,其宣力諸臣遷賞有差,賜丞相脫脫世襲答剌罕之號,特命翰林學士承旨歐陽玄制河平碑文,以旌勞績。



 玄既為河平之碑,又自以為司馬遷、班固記河渠溝洫,僅載治水之道,不言其方,使後世任斯事者無所考則,乃從魯訪問方略,及詢過客,質吏牘,作《至正河防記》,欲使來世罹河患者按而求之。其言曰:



 治河一也,有疏、有浚、有塞,三者異焉。釃河之流,因而導之,謂之疏。去河之淤,因而深之,謂之浚。抑河之暴,因而扼之,謂之塞。疏浚之別有四:曰生地,曰故道,曰河身,曰減水河。生地有直有紆,因直而鑿之,可就故道。故道有高有卑,高者平之以趨卑,高卑相就,則高不壅,卑不瀦,慮夫壅生潰,瀦生堙也。河身者,水雖通行,身有廣狹,狹難受水,水益悍,故狹者以計闢之;廣難為岸,岸善崩,故廣者以計御之。減水河者,水放曠則以制其狂,水隳突則以殺其怒。



 治堤一也,有創築、修築、補築之名,有剌水堤,有截河堤,有護岸堤,有縷水堤,有石船堤。



 治掃一也,有岸掃、水掃,有龍尾、欄頭、馬頭等掃。其為掃臺及推卷、牽制、珣掛之法,有用土、用石、用鐵、用草、用木、用杙、用糸亙之方。



 塞河一也,有缺口,有豁口,有龍口。缺口者,已成川。豁口者,舊常為水所豁,水退則口下於堤,水漲則溢出於口。龍口者,水之所會,自新河入故道之水眾也。



 此外不能悉書,因其用功之次第,而就述於其下焉。



 其浚故道,深廣不等,通長二百八十里百五十四步而強。功始自白茅,長百八十二里。繼自黃陵岡至南白茅,闢生地十里。口初受,廣百八十步,深二丈有二尺,已下停廣百步,高下不等,相折深二丈及泉。曰停、曰折者,用古算法,因此推彼,知其勢之低昂,相準折而取勻停也。南白茅至劉莊村,接入故道十里,通折墾廣八十步,深九尺。劉莊至專固,百有二里二百八十步,通折停廣六十步,深五尺。專固至黃固,墾生地八里,面廣百步,底廣九十步,高下相折,深丈有五尺。黃固至哈只口,長五十一里八十步,相折停廣墾六十步,深五尺。乃浚凹裡減水河,通長九十八里百五十四步。凹裏村缺河口生地,長三里四十步,面廣六十步,底廣四十步,深一丈四尺。自凹裏生地以下舊河身至張贊店,長八十二里五十四步。上三十六里,墾廣二十步,深五尺;中三十五里,墾廣二十八步,深五尺;下十里二百四十步,墾廣二十六步,深五尺。張贊店至楊青村,接入故道,墾生地十有三里六十步,面廣六十步,底廣四十步,深一丈四尺。



 其塞專固缺口,修堤三重,並補築凹裡減水河南岸豁口,通長二十里三百十有七步。其創築河口前第一重西堤,南北長三百三十步,面廣二十五步,底廣三十三步,樹置樁橛,實以土牛、草葦、雜梢相兼,高丈有三尺,堤前置龍尾大掃。言龍尾者,伐大樹連梢系之堤旁,隨水上下,以破嚙岸浪者也。築第二重正堤,並補兩端舊堤,通長十有一里三百步。缺口正堤長四里,兩堤相接舊堤,置樁堵閉河身,長百四十五步,用土牛、草葦、梢土相兼修築,底廣三十步,修高二丈。其岸上土工修築者,長三里二百十有五步有奇,高廣不等,通高一丈五尺。補築舊堤者,長七里三百步,表裏倍薄七步,增卑六尺,計高一丈。築第三重東后堤,並接修舊堤,高廣不等,通長八里。補築凹裡減水河南岸豁口四處,置樁木,草土相兼,長四十七步。



 於是塞黃陵全河,水中及岸上修堤長三十六里百三十六步。其修大堤刺水者二,長十有四里七十步。其西復作大堤刺水者一,長十有二里百三十步。內創築岸上土堤,西北起李八宅西堤,東南至舊河岸,長十里百五十步,顛廣四步,趾廣三之,高丈有五尺。仍築舊河岸至入水堤,長四百三十步,趾廣三十步,顛殺其六之一,接修入水。



 兩岸掃堤並行。作西掃者夏人水工,徵自靈武;作東掃者漢人水工,徵自近畿。其法以竹絡實以小石,每掃不等,以蒲葦綿腰索徑寸許者從鋪,廣可一二十步,長可二三十步。又以曳掃索綯徑三寸或四寸、長二百餘尺者衡鋪之。相間復以竹葦麻釭大纖,長三百尺者為管心索,就系綿腰索之端於其上,以草數千束,多至萬餘,勻布厚鋪於綿腰索之上,褲而納之,丁夫數千,以足蹈實,推卷稍高,即以水工二人立其上,而號於眾,眾聲力舉,用小大推梯,推卷成掃,高下長短不等,大者高二丈,小者不下丈餘。又用大索或互為腰索,轉致河濱,選健丁操管心索,順掃臺立踏,或掛之臺中鐵貓大橛之上,以漸縋之下水。掃後掘地為渠,陷管心索渠中,以散草厚覆,築之以土,其上復以土牛、雜草、小掃梢土,多寡厚薄,先後隨宜。修疊為掃臺,務使牽制上下,縝密堅壯,互為掎角,掃不動搖。日力不足,火以繼之。積累既畢,復施前法,卷掃以壓先下之掃,量水淺深,制掃厚薄,疊之多至四掃而止。兩掃之間置竹絡,高二丈或三丈,圍四丈五尺,實以小石、土牛。既滿,系以竹纜,其兩旁並掃,密下大樁,就以竹絡上大竹腰索系於樁上。東西兩掃及其中竹絡之上,以草土等物築為掃臺,約長五十步或百步,再下掃,即以竹索或麻索長八百尺或五百尺者一二,雜廁其餘管心索之間,俟掃入水之後,其餘管心索如前珣掛,隨以管心長索,遠置五七十步之外,或鐵貓,或大樁,曳而系之,通管束累日所下之掃,再以草土等物通修成堤,又以龍尾大掃密掛於護堤大樁,分析水勢。其堤長二百七十步,北廣四十二步,中廣五十五步,南廣四十二步,自顛至趾,通高三丈八尺。



 其截河大堤,高廣不等,長十有九里百七十七步。其在黃陵北岸者,長十里四十一步。築岸上土堤,西北起東西故堤,東南至河口,長七里九十七步,顛廣六步,趾倍之而強二步,高丈有五尺,接修入水。施土牛、小掃梢草雜土,多寡厚薄隨宜修疊,及下竹絡,安大樁,系龍尾掃,如前兩堤法。唯修疊掃臺,增用白闌小石。並掃上及前幾修掃堤一,長百餘步,直抵龍口。稍北,欄頭三掃並行,掃大堤廣與刺水二堤不同,通前列四掃,間以竹絡,成一大堤,長二百八十步,北廣百一十步,其顛至水面高丈有五尺,水面至澤腹高二丈五尺,通高三丈五尺;中流廣八十步,其顛至水面高丈有五尺,水面至澤腹高五丈五尺,通高七丈。並創築縷水橫堤一,東起北截河大堤,西抵西刺水大堤。又一堤東起中刺水大堤,西抵西刺水大堤,通長二里四十二步,亦顛廣四步,趾三之,高丈有二尺。修黃陵南岸,長九里百六十步,內創岸土堤,東北起新補白茅故堤,西南至舊河口,高廣不等,長八里二百五十步。



 乃入水作石船大堤,蓋由是秋八月二十九日乙巳道故河流,先所修北岸西中刺水及截河三堤猶短,約水尚少,力未足恃。決河勢大,南北廣四百餘步,中流深三丈餘,益以秋漲,水多故河十之八。兩河爭流,近故河口,水刷岸北行,洄漩湍激,難以下掃。且掃行或遲,恐水盡湧入決河,因淤故河,前功遂隳。魯乃精思障水入故河之方,以九月七日癸丑,逆流排大船二十七艘,前後連以大桅或長樁,用大麻索、竹糸亙絞縛,綴為方舟。又用大麻索、竹糸亙用船身繳繞上下,令牢不可破,乃以鐵貓於上流垂之水中。又以竹糸亙絕長七八百尺者,系兩岸大橛上,每糸亙或垂二舟或三舟,使不得下,船腹略鋪散草,滿貯小石,以合子板釘合之,復以掃密布合子板上,或二重,或三重,以大麻索縛之急,復縛橫木三道於頭桅,皆以索維之,用竹編笆,夾以草石,立之桅前,約長丈餘,名曰水簾桅。復以木耆拄,使簾不偃僕,然後選水工便捷者,每船各二人,執斧鑿,立船首尾,岸上搥鼓為號,鼓鳴,一時齊鑿,須臾舟穴,水入,舟沉,遏決河。水怒溢,故河水暴增,即重樹水簾,令後復布小掃土牛白闌長梢,雜以草土等物,隨以填垛以繼之。石船下詣實地,出水基趾漸高,復卷大掃以壓之。前船勢略定,尋用前法,沉餘船以竟後功。昏曉百刻,役夫分番甚勞,無少間斷。船堤之後,草掃三道並舉,中置竹絡盛石,並掃置樁,系纜四掃及絡,一如修北截水堤之法。第以中流水深數丈,用物之多,施功之大,數倍他堤。船堤距北岸才四五十步,勢迫東河,流峻若自天降,深淺叵測。於是先卷下大掃約高二丈者,或四或五,始出水面。修至河口一二十步,用工尤艱。薄龍口,喧豗猛疾,勢撼掃基,陷裂欹傾,俄遠故所,觀者股弁,眾議騰沸,以為難合,然勢不容已。魯神色不動,機解捷出,進官吏工徒十餘萬人,日加獎諭,辭旨懇至,眾皆感激赴功。十一月十一日丁巳,龍口遂合,決河絕流,故道復通。又於堤前通卷欄頭掃各一道,多者或三或四,前掃出水,管心大索系前掃,垂後闌頭掃之後,後掃管心大索亦系小掃,垂前闌頭掃之前,後先羈縻,以錮其勢。又於所交索上及兩掃之間,壓以小石白闌土牛,草土相半,厚薄多寡,相勢措置。



 掃堤之後,自南岸復修一堤,抵已閉之龍口,長二百七十步。船堤四道成堤,用農家場圃之具曰轆軸者,穴石立木如比櫛,珣前掃之旁,每步置一轆軸,以橫木貫其後,又穴石,以徑二寸餘麻索貫之,系橫木上,密掛龍尾大掃,使夏秋潦水、冬春凌筼,不得肆力於岸。此堤接北岸截河大堤,長二百七十步,南廣百二十步,顛至水面高丈有七尺,水面至澤腹高四丈二尺;中流廣八十步,顛至水面高丈有五尺,水面至澤腹高五丈五尺;通高七丈。仍治南岸護堤掃一道,通長百三十步,南岸護岸馬頭掃三道,通長九十五步。修築北岸堤防,高廣不等,通長二百五十四里七十一步。白茅河口至板城,補築舊堤,長二十五里二百八十五步。曹州板城至英賢村等處,高廣不等,長一百三十三里二百步。梢岡至碭山縣,增培舊堤,長八十五里二十步。歸德府哈只口至徐州路三百餘里,修完缺口一百七處,高廣不等,積修計三里二百五十六步。亦思剌店縷水月堤,高廣不等,長六里三十步。



 其用物之凡,樁木大者二萬七千,榆柳雜梢六十六萬六千,帶梢連根株者三千六百,槁秸蒲葦雜草以束計者七百三十三萬五千有奇,竹竿六十二萬五千,葦席十有七萬二千,小石二千艘,繩索小大不等五萬七千,所沉大船百有二十,鐵纜三十有二,鐵貓三百三十有四,竹篾以斤計者十有五萬,垂石三千塊,鐵鉆萬四千二百有奇,大釘三萬三千二百三十有二。其餘若木龍、蠶椽木、麥稭、扶樁、鐵叉、鐵吊、枝麻、搭火鉤、汲水、貯水等具皆有成數。官吏俸給,軍民衣糧工錢,醫藥、祭祀、賑恤、驛置馬乘及運竹木、沉船、渡船、下樁等工,鐵、石、竹、木、繩索等匠傭貲,兼以和買民地為河,並應用雜物等價,通計中統鈔百八十四萬五千六百三十六錠有奇。



 魯嘗有言:「水工之功,視土工之功為難;中流之功,視河濱之功為難;決河口視中流又難;北岸之功視南岸為難。用物之效,草雖至柔,柔能狎水,水漬之生泥,泥與草並,力重如碇。然維持夾輔,纜索之功實多。」蓋由魯習知河事,故其功之所就如此。



 玄之言曰:「是役也,朝廷不惜重費,不吝高爵,為民闢害。脫脫能體上意,不憚焦勞,不恤浮議,為國拯民。魯能竭其心思智計之巧,乘其精神膽氣之壯,不惜劬瘁,不畏譏評,以報君相知人之明。宜悉書之,使職史氏者有所考證也。」



 先是歲庚寅,河南北童謠云:「石人一隻眼,挑動黃河天下反。」及魯治河,果於黃陵岡得石人一眼,而汝、潁之妖寇乘時而起。議者往往以謂天下之亂,皆由賈魯治河之役,勞民動眾之所致。殊不知元之所以亡者,實基於上下因循,狃於宴安之習,紀綱廢弛,風俗偷薄,其致亂之階,非一朝一夕之故,所由來久矣。不此之察,乃獨歸咎於是役,是徒以成敗論事,非通論也。設使賈魯不興是役,天下之亂,詎無從而起乎?今故具錄玄所記,庶來者得以詳焉。



 蜀堰



 江水出蜀西南徼外,東至於岷山,而禹導之。秦昭王時,蜀太守李冰鑿離堆,分其江以灌川蜀,民用以饒。歷千數百年,所過沖薄蕩嚙,又大為民患。有司以故事,歲治堤防,凡一百三十有三所,役兵民多者萬餘人,少者千人,其下猶數百人。役凡七十日,不及七十日,雖事治,不得休息。不役者,日出三緡為庸錢。由是富者屈於貲,貧者屈於力,上下交病,會其費,歲不下七萬緡。大抵出於民者,十九藏於吏,而利之所及,不足以償其費矣。



 元統二年,僉四川肅政廉訪司事吉當普巡行周視,得要害之處三十有二,餘悉罷之。召灌州判官張弘,計曰:「若甃之以石,則歲役可罷,民力可蘇矣。」弘曰:「公慮及此,生民之福,國家之幸,萬世之利也。」弘遂出私錢,試為小堰,堰成,水暴漲而堰不動。乃具文書,會行省及蒙古軍七翼之長、郡縣守宰,下及鄉里之老,各陳利害,咸以為便。復禱於冰祠,卜之吉。於是征工發徒,以仍改至元元年十有一月朔日,肇事於都江堰,即禹鑿之處,分水之源也。鹽井關限其西北,水西關據其西南,江南北皆東行。北舊無江,冰鑿以闢沫水之害,中為都江堰,少東為大、小釣魚,又東跨二江為石門,以節北江之水,又東為利民臺,臺之東南為侍郎、楊柳二堰,其水自離堆分流入於南江。



 南江東至鹿角,又東至金馬口,又東道大安橋,入於成都,俗稱大皁江,江之正源也。北江少東為虎頭山,為鬥雞臺。臺有水則,以尺畫之,凡十有一。水及其九,其民喜,過則憂,沒其則則困。又書「深淘灘,高作堰」六字其旁,為治水之法,皆冰所為也。又東為離堆,又東過凌虛、步雲二橋,又東至三石洞,釃為二渠。其一自上馬騎東流,過郫,入於成都,古謂之內江,今府江是也;其一自三石洞北流,過將軍橋,又北過四石洞,折而東流,過新繁,入於成都,古謂之外江。此冰所穿二江也。



 南江自利民臺有支流,東南出萬工堰,又東為駱駝,又東為碓口,繞青城而東,鹿角之北涯,有渠曰馬壩,東流至成都,入於南江。渠東行二十餘里,水決其南涯四十有九,每歲疲民力以塞之。乃自其北涯鑿二渠,與楊柳渠合,東行數十里,復與馬壩渠會,而渠成安流。自金馬口之西鑿二渠,合金馬渠,東南入於新津江,罷藍澱、黃水、千金、白水、新興至三利十二堰。



 北江三石洞之東為外應、顏上、五鬥諸堰,外應、顏上之水皆東北流,入於外江。五斗之水,南入馬壩渠,皆內江之支流也。外江東至崇寧,亦為萬工堰。堰之支流,自北而東,為三十六洞,過清白堰東入於彭、漢之間。而清白堰水潰其南涯,延袤三里餘,有司因潰以為堰。堰輒壞,乃疏其北涯舊渠,直流而東,罷其堰及三十六洞之役。



 嘉定之青神,有堰曰鴻化,則授成其長吏,應期而功畢。若成都之九里堤,崇寧之萬工堰,彰之堋口、豐潤、千江、石洞、濟民、羅江、馬腳諸堰,工未及施,則召長吏免諭,使及農隙為之。諸堰都江及利民臺之役最大,侍郎、楊柳、外應、顏上、五斗次之,鹿角、萬工、駱駝、碓口、三利又次之。而都江又居大江中流,故以鐵萬六千斤,鑄為大龜,貫以鐵柱,而鎮其源,然後即工。



 諸堰皆甃以石,範鐵以關其中,取桐實之油,和石灰,雜麻絲,而搗之使熟,以苴罅漏。岸善崩者,密築江石以護之,上植楊柳,旁種蔓荊,櫛比鱗次,賴以為固,蓋以數百萬計。所至或疏舊渠以導其流,或鑿新渠以殺其勢。遇水之會,則為石門,以時啟閉而洩蓄之,用以節民力而資民利,凡智力所及,無不為也。初,郡縣及兵家共掌都江之政,延祐七年,其兵官奏請獨任郡縣,民不堪其役‖至是復合焉。常歲獲水之利僅數月,堰輒壞,至是,雖緣渠所置碓磑紡績之處以千萬計,四時流轉而無窮。



 其始至都江,水深廣莫可測,忽有大洲湧出其西南,方可數里,人得用事其間。入山伐石,崩石已滿,隨取而足。蜀故多雨,自初役至工畢,無雨雪,故力省而功倍,若有相之者。五越月,功告成,而吉當普以監察御史召,省臺上其功,詔揭手奚斯制文立碑以旌之。



 是役也,凡石工、金工皆七百人,木工二百五十人,役徒三千九百人,而蒙古軍居其二千。糧為石千有奇,石之材取於山者百萬有奇,石之灰以斤計者六萬有奇,油半之,鐵六萬五千斤,麻五千斤。撮其工之直、物之價,以緡計者四萬九千有奇,皆出於民之庸,而在官之積者,尚餘二十萬一千八百緡,責灌守以貸於民,歲取其息,以備祭祀及淘灘修堰之費。仍蠲灌之兵民所常徭役,俾專其力於堰事。



 涇渠



 涇渠者,在秦時韓使水工鄭國說秦,鑿涇水,自仲山西抵瓠口為渠,並北山,東注于洛三百餘里以溉田,蓋欲以罷秦之力,使無東伐。秦覺其謀,欲殺之,鄭曰:「臣為韓延數年之命,而為秦建萬世之利。」秦以為然,使迄成之,號鄭渠。漢時有白公者,奏穿渠引涇水,起谷口,入櫟陽,注渭中,袤二百里,溉田四千五百餘頃,因名曰白渠。歷代因之,皆享其利。至宋時,水沖嚙,失其故跡。熙寧間,詔賜常平息錢,助民興作,自仲山旁開鑿石渠,從高瀉水,名豐利渠。



 元至元間,立屯田府督治之。大統八年,涇水暴漲,毀堰塞渠,陜西行省命屯田府總管夾谷伯顏帖木兒及涇陽尹王琚疏道之,起涇陽、高陵、三原、櫟陽用水人戶及渭南、櫟陽、涇陽三屯所人夫,共三千餘人興作,水通流如舊。其制編荊為囤,貯之以石,復填以草以土為堰,歲時葺理,未嘗廢止。



 至大元年,王琚為西臺御史,建言於豐利渠上更開石渠五十一丈,闊一丈,深五尺,積一十五萬三千工,每方一尺為一工。自延祐元年興工,至五年渠成。是年秋,改堰至新口。泰定間,言者謂石渠歲久,水流漸穿逾下,去岸益高。至正三年,御史宋秉亮相視其堰,謂渠積年坎取淤土,疊壘於岸,極為高崇,力難送土於上,因請就岸高處開通鹿巷,以便夫行,廷議允可。四年,屯田同知牙八胡、涇尹李克忠發丁夫開鹿巷八十四處,削平土壘四百五十餘步。二十年,陜西行省左丞相帖裏帖木兒遣都事楊欽修治,凡溉農田四萬五千餘頃。



 金口河



 至正二年正月,中書參議孛羅帖木兒、都水傅佐建言,起自通州南高麗莊,直至西山石峽鐵板開水古金口一百二十餘里,創開新河一道,深五丈,廣二十丈,放西山金口水東流至高麗莊,合御河,接引海運至大都城內輸納。是時,脫脫為中書右丞相,以其言奏而行之。廷臣多言其不可,而左丞許有壬言尤力,脫脫排群議不納,務於必行。有壬因條陳其利害,略曰:



 大德二年,渾河水發為民害,大都路都水監將金口下閉閘板。五年間,渾河水勢浩大,郭太史恐沖沒田薛二村、南北二城,又將金口已上河身,用砂石雜土盡行堵閉。至順元年,因行都水監郭道壽言,金口引水過京城至通州,其利無窮,工部官並河道提舉司、大都路及合屬官員耆老等相視議擬,水由二城中間窒礙。又盧溝河自橋至合流處,自來未嘗有漁舟上下,此乃不可行船之明驗也。且通州去京城四十里,盧溝止二十里,此時若可行船,當時何不於盧溝立馬頭,百事近便,卻於四十里外通州為之?又西山水勢高峻,亡金時,在都城之北流入郊野,縱有沖決,為害亦輕。今則在都城西南,與昔不同。此水性本湍急,若加以夏秋霖潦漲溢,則不敢必其無虞,宗廟社稷之所在,豈容僥幸於萬一?若一時成功,亦不能保其永無沖決之患。且亡金時此河未必通行,今所有河道遺跡,安知非作而復輟之地乎?又地形高下不同,若不作閘,必致走水淺澀,若作閘以節之,則沙泥渾濁,必致淤塞,每年每月專人挑洗,蓋無窮盡之時也。且郭太史初作通惠河時,何不用此水,而遠取白浮之水,引入都城,以供閘壩之用?蓋白浮之水澄清,而此水渾濁不可用也。此議方興,傳聞於外,萬口一辭,以為不可。若以為成大功者不謀於眾,人言不足聽,則是商鞅、王安石之法,當今不宜有此。



 議既上,丞相終不從,遂以正月興工,至四月功畢。起閘放金口水,流湍勢急,沙泥壅塞,船不可行,而開挑之際,毀民廬舍墳塋,夫丁死傷甚眾,又費用不貲,卒以無功。繼而御史糾劾建言者,孛羅帖木兒、傅佐俱伏誅。今附載其事於此,用為妄言水利者之戒。



\end{pinyinscope}