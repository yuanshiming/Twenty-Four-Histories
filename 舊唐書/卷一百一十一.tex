\article{卷一百一十一}

\begin{pinyinscope}

 ○靖德
 太子琮庶人瑛棣王琰庶人瑤靖恭太子琬庶人琚夏悼王一儀王璲潁王璬
 懷哀王敏永王璘壽王瑁延王玢盛王琦濟王環信王瑝義王玭陳王珪豐王珙恆王瑱涼王璿汴哀王璥



 玄宗三十子:元獻楊皇后生肅宗,劉華妃生奉天皇帝琮、靖恭太子琬、儀王璲,趙麗妃生廢太子瑛,錢妃生棣王琰,皇甫德儀生鄂王瑤,劉才人生光王琚,貞順武皇
 后生夏悼王一、懷哀王敏、壽王瑁、盛王琦,高婕妤生潁王璬、郭順儀生永王璘,柳婕妤生延王玢,鐘美人生濟王環,盧美人生信王瑝,閻才人生義王玭,王美人生陳王珪,陳美人生豐王珙,鄭才人生恆王瑱,武賢儀生涼王璿,汴哀王璥,餘七王早夭。



 奉天皇帝琮,玄宗長子也,本名嗣直。景雲元年九月,封許昌郡王。先天元年八月,進封郯王。開元四年正月,遙領安西大都護,仍充安撫河東、關內、隴右諸蕃大使。十三
 年,改封慶王,仍改名潭。十五年,遙領涼州都督,兼河西諸軍節度大使。二十一年,加太子太師,改名琮。二十四年,拜司徒。天寶元年,兼太原牧。十一載薨,贈靖德太子,葬於渭水之南細柳原,仍於啟夏門內置廟祔享焉。肅宗元年建寅月九日,詔追冊為奉天皇帝,妃竇氏為恭應皇后,備禮改葬於華清宮北齊陵,以尚書右僕射、冀國公裴冕為其使。初,開元二十五年,太子瑛得罪廢,令琮養其子,及天寶十一載琮薨,以瑛子俅為嗣慶王,除秘
 書監同正員。



 廢太子瑛,玄宗第二子也,本名嗣謙。景雲元年九月,封真定郡王。先天元年八月,進封郢王。開元三年正月,立為皇太子。七年正月,加元服。其年,玄宗又令太子詣國子學行齒胄之禮,仍敕右散騎常侍褚無量升筵講論,學官及文武百官節級加賜。十三年,改名鴻,納妃薛氏,禮畢,曲赦京城之內,侍講潘肅等並加級改職,中書令蕭嵩親迎,特封徐國公。二十五年七月,改名瑛。



 瑛母趙
 麗妃,本伎人,有才貌,善歌舞,玄宗在潞州得幸。及景雲升儲之後,其父元禮、兄常奴擢為京職,開元初皆至大官。及武惠妃寵幸,麗妃恩乃漸弛。時鄂王瑤母皇甫德儀、光王琚母劉才人,皆玄宗在臨淄邸以容色見顧,出子朗秀而母加愛焉。及惠妃承恩,鄂、光之母亦漸疏薄,惠妃之子壽王瑁,鐘愛非諸子所比。瑛於內第與鄂、光王等自謂母氏失職,嘗有怨望。惠妃女咸宜公主出降於楊洄,洄希惠妃之旨,規利於己,日求其短,譖於惠妃。
 妃泣訴於玄宗,以太子結黨,將害於妾母子,亦指斥於至尊。玄宗惑其言,震怒,謀於宰相,意將廢黜。中書張九齡奏曰:「陛下纂嗣鴻業,將三十年,太子已下,常不離深宮,日受聖訓。今天下之人,皆慶陛下享國日久,子孫蕃育,不聞有過,陛下奈何以一日之間廢棄三子?伏惟陛下思之。且太子國本,難於動搖。昔晉獻公惑寵嬖之言,太子申生憂死,國乃大亂。漢武威加六合,受江充巫蠱之事,將禍及太子,遂至城中流血。晉惠帝有賢
 子為太子,容賈後之譖,以至喪亡。隋文帝取寵婦之言,廢太子勇而立晉王廣,遂失天下。由此而論之,不可不慎。今太子既長無過,二王又賢,臣待罪左右,敢不詳悉。」玄宗默然,事且寢。其年,駕幸西京,以李林甫代張九齡為中書令,希惠妃之旨,托意於中貴人,揚壽王瑁之美,惠妃深德之。二十五年四月,楊洄又構於惠妃,言瑛兄弟三人與太子妃兄駙馬薛鏽常構異謀。玄宗遽召宰相籌之,林甫曰:「此蓋陛下家事,臣不合參知。」玄宗意乃決矣。使
 中官宣詔於宮中,並廢為庶人,鏽配流,俄賜死於城東驛。天下之人不見其過,咸惜之。其年,武惠妃數見三庶人為崇,怖而成疾,巫者祈請彌月,不痊而殞。



 瑛有六男:儼、伸、倩、俅、備、人敬。慶王琮先無子,瑛得罪後,玄宗遣鞫之。天寶中,儼為新平郡王、光祿卿同正員,伸為平原郡王、宗正卿同正員,俅為嗣慶王。寶應元年,詔雪瑤、瑛、琚之罪,贈瑛為皇太子,瑤、琚復贈為王。



 棣王琰,玄宗第四子也,初名嗣真。開元二年十二月,封
 為鄫王。十二年三月,改封棣王,仍改名洽。十五年,遙領太原牧、太原已北諸軍節度大使。二十二年,加太子太傅,餘如故。二十四年,改名琰。天寶元年六月,遙領兼武威郡都督、河西隴右經略節度大使。



 先是,琰妃韋氏有過,琰怒之,不敢奏聞,乃斥於別室。寵二孺人,孺人又不相協。至十一載,孺人乃密求巫者,書符置於琰履中以求媚。琰與監院中官有隙,中官聞其事,密奏於玄宗,云琰厭魅聖躬。玄宗使人掩其履而獲之。玄宗大怒,引琰
 詰責之。琰頓首謝曰:「臣之罪合死矣,請一言以就鼎鑊。然臣與新婦,情義絕者,二年於茲,臣有二孺人,又皆爭長。臣實不知有符,恐此三人所為也。惟三哥辯其罪人。」及推問之,竟孺人也。玄宗猶疑琰知情,怒未解,太子已下皆為請,命囚於鷹狗坊中,絕朝請,憂懼而死。琰妃即少師韋滔女,無子,琰死後,妃得還其父。琰男女繁衍,至五十五人。天寶中封為王者三人:僎為汝南郡王、秘書監同正員,僑為宜都王、衛尉卿同正員,雋為濟南王、光
 祿卿同正員。寶應元年五月,代宗即位,舍琰罪,贈其王位。



 鄂王瑤,玄宗第五子也,初名嗣初。開元二年五月,封為鄂王。十二年,改名涓,遙領幽州都督、河北道節度大使。二十一年四月,加太子太保,兼幽州都督,餘如故。二十三年,改名瑤。二十五年,得罪廢。寶應元年五月追復。



 靖恭太子琬,玄宗第六子也,初名嗣玄。開元二年三月,封為甄王。十二年三月,改名滉,封為榮王。十五年,授京
 兆牧,又遙領隴右節度大使。二十三年,加開府儀同三司,餘如故。二十五年,改名琬。天寶元年六月,授單于大都護。十四年十一月,安祿山反於範陽,其月制以琬為征討元帥,高仙芝為副,令仙芝征河、隴兵募屯於陜郡以御之。數日,琬薨。琬素有雅稱,風格秀整,時士庶冀琬有所成功,忽然殂謝,遠近咸失望焉。贈靖恭太子,葬於見子西原。琬諸子尤繁衍,男女五十八人。天寶中封為郡王者二:俯為濟陰王、太僕卿同正員,偕為北平王、國
 子祭酒同正員。



 光王琚,玄宗第八子也。開元十二年,封為光王。十五年,遙領廣州都督、五府經略大使。二十三年七月,光王琚、儀王濰、潁王澐、壽王清、延王洇、盛王沐、信王沔、義王漼等十王,並授開府儀同三司;皇子珪封為陳王,澄封為翌王,潓封為恆王,滔封為汴王。陳王已下第四王,幼未授官,並置府官僚屬。其日,光、儀等十人同於東宮尚書省上,詔宰臣及文武百僚送,儀注甚盛。俄除十五王府
 元僚,並未有府幕,同於禮院上,亦無精選。其時,琚兼廣州都督,餘如故。琚與鄂王瑤,皇子中有學尚才識,同居內宅,最相愛狎。琚有才力,善騎射。初封甚善,玄宗愛之。以母見疏薄,嘗有怨言,為人所構得罪,人用憐之。寶應元年五月,追復官爵。無子。



 夏悼王一,玄宗第九子也。母貞順皇后為惠妃,見寵。一生而美秀,上鐘愛無比,名之為一。開元五年,孩孺而薨,玄宗追封謚。時車駕在東都,葬於城南龍門東岑,欲宮
 中舉目見之。



 儀王璲,玄宗第十二子也,初名濰。開元十三年五月,封為儀王。十五年,授河南牧。二十三年,加開府儀同三司,兼河南牧,其年改名璲。永泰元年二月薨,廢朝三日,贈太傅。天寶中有子封王者二人:侁為鐘陵郡王、光祿卿同正員,僆為廣陵王、國子祭酒同正員。



 潁王璬,玄宗第十三子也。讀書有文詞。初名
 澐。開元十三年,封潁王。十五年,遙領安東都護、平盧軍節度大使。二十三年,加開府儀同三司,改名璬。安祿山反,除蜀郡大都督、劍南節度大使,楊國忠為之副。玄宗幸蜀,令御史大夫魏方進充置頓使,先移牒至蜀,托以潁王之籓,令設儲供。玄宗至馬嵬,方進被殺,乃令璬先赴本郡,以蜀郡長史崔圓為副。璬性儉率,將渡綿州江,登舟見彩緣席為藉者,顧曰:「此可以為寢處,奈何踐之?」命撤去。璬初奉命之籓,卒遽不遑受節,綿州司馬史賁進說曰:「王,帝子也,且為節度大使。今之籓而不持節,單騎徑進,人
 何所贍?請建大槊,蒙之油囊,為旌節狀,先驅道路,足以威眾。」璬笑曰:「但為真王,何用假旌節乎?」將至成都,崔圓迓之,拜於馬前,璬不止之,圓頗怒。玄宗至,璬視事兩月,人甚安之。為圓所奏,罷居內宅。後令宣慰肅宗於彭原,遂從歸京師。建中四年薨。年六十六,輟朝三日。子伸,天寶中封滎陽郡王,授衛尉卿同正員。



 懷哀王敏,玄宗第十五子也。幼而豐秀,以母惠妃之寵,玄宗特加顧念。才晬,開元八年二月薨,追封謚,權窆於景
 龍觀。天寶十三載,改葬京城南,以祔其母敬陵也。



 永王璘,玄宗第十六子也。母曰郭順儀,劍南節度尚書虛己之妹。璘數歲失母,肅宗收養,夜自抱眠之。少聰敏好學,貌陋,視物不正。開元十三年三月,封為永王。十五年五月,遙領荊州大都督。二十年七月,加開府儀同三司,改名璘。



 天寶十四載十一月,安祿山反範陽。十五載六月,玄宗幸蜀,至漢中郡,下詔以璘為山南東路及嶺南黔中江南西路四道節度採訪等使、江陵郡大都督,
 餘如故。璘七月至襄陽,九月至江陵,召募士將數萬人,恣情補署,江淮租賦,山積於江陵,破用鉅億。以薛鏐、李臺卿、蔡坰為謀主,因有異志。肅宗聞之,詔令歸觀於蜀,璘不從命。十二月,擅領舟師東下,甲仗五千人趨廣陵,以季廣琛、渾惟明、高仙琦為將。璘生於宮中,不更人事,其子襄城王人易又勇而有力,馭兵權,為左右眩惑,遂謀狂悖。璘雖有窺江左之心,而未露其事。吳郡採訪使李希言乃平牒璘,大署其名,璘遂激怒,牒報曰:「寡人上皇
 天屬,皇帝友于,地尊侯王,禮絕僚品,簡書來往,應有常儀,今乃平牒抗威,落筆署字,漢儀隳紊,一至於斯!」乃使渾惟明取希言,季廣琛趣廣陵攻採訪李成式。璘進至當塗,希言在丹陽,令元景曜、閻敬之等以兵拒之,身走吳郡,李成式使將李承慶拒之。先是,肅宗以璘不受命,先使中官啖廷瑤、段喬福招討之。中官至廣陵,反式括得馬數百匹。時河北招討判官、司虞郎中李銑在廣陵,瑤等結銑為兄弟,求之將兵。銑麾下有騎一百八十人,
 遂率所領屯於楊子,成式使判官評事裴茂以廣陵步卒三千同拒於瓜步洲伊婁埭。希言將元景曜及成式將李神慶並以其眾迎降於璘,璘又殺丹徒太守閻敬之以徇。江左大駭。



 裴茂至瓜步洲,廣張旗幟,耀於江津。璘與人易登陴望之竟日,始有懼色。季廣琛召諸將割臂而盟,以貳於璘。是日,渾惟明走於江寧,馮季康、康謙投於廣陵之白沙。廣琛以步卒六千趨廣陵,璘使騎追之,廣琛曰:「我感王恩,是以不能決戰,逃而歸國。若逼我,我則
 不擇地而回戰矣。」使者返報。其夕,銑等多燃火,人執兩炬以疑之,隔江望者,兼水中之影,一皆為二矣。璘軍又以火應之。璘懼,以官軍悉濟矣,遂以兒女及麾下宵遁。遲明,不見濟者,遂入城具舟楫,使襄城王驅其眾以奔晉陵。宵諜曰:「王走矣。」於是江北之軍齊進,募敢死士趙侃、庫狄岫、趙連城等共二十人,先鋒游弈於新豐,皆因醉而寐。璘聞官軍之至,乃使襄城王、高仙琦逆擊之。驛騎奔告,侃等介馬而出,襄城王已隨而至,銑等奔救,張
 左右翼擊之,射中襄城王首,人易軍遂敗。高仙琦等四騎與璘南奔,至鄱陽郡,司馬陶備閉城拒之。璘怒,命焚其城。至餘干,及大庾嶺,將南投嶺外,為江西採訪使皇甫侁下防禦兵所擒,因中矢而薨。子人易等為亂兵所害。肅宗以璘愛弟,隱而不言。



 壽王瑁,玄宗第十八子也,初名清。初,瑁母武惠妃,開元元年見幸,寵傾後宮,頻產夏悼王、懷哀王、上仙公主,皆端麗,襁褓不育。及瑁之初生,讓帝妃元氏請瑁在於邸
 中收養,妃自乳之,名為己子。十餘年在寧邸,故封建之事晚於諸王。宮中常呼為十八郎。十三年三月,封為壽王,始入宮中。十五年,遙領益州大都督、劍南節度大使。二十三年,加開府儀同三司,改名瑁。二十五年,惠妃薨,葬以後禮。二十九年,讓帝薨,瑁請制服,以報乳養之恩,玄宗從之。瑁,天寶中有子封為王者二人:懷為濟陽郡王,偡為廣陽郡王、鴻臚卿同正員。



 唐法,親王食封八百戶,有至一千戶;公主三百戶,長公主加三百戶,有至六百
 戶。高宗朝以沛、英、豫王、太平公主武后所生,食逾於制。垂拱中,太平至一千二百戶。聖歷初,皇嗣封為相王,食封與太平同三千戶。長安中,壽春王兄弟五人,並賜實封三百戶。神龍初,相府與太平同至五千戶,衛王三千戶,溫王二千戶,成王七百戶。壽春王加四百戶,通前七百戶;嗣雍、衡陽、臨淄、巴陵、中山各加二百戶,通前五百戶。安樂初封二千戶,長寧一千五百戶,宣城、宜城、宣安各一千戶,相王女為縣主者各三百戶。衛王尋升儲位,相
 府增至七千戶,太平至五千戶,安樂三千戶,長寧二千五百戶,宣城已下各二千戶。相府、太平、長寧、安樂皆以七千為限,雖水旱亦不破損免,以正租庸充數。唐隆元年,遺制以嗣雍王守禮、壽春王成器封為親王,各賜實封一千戶。開元之後,朝恩睦親,以寧府最長,封至五千五百戶;岐、薛愛弟著勛,五千戶;申府以外家微,至四千戶;邠府以外枝,至一千八百戶。皇妹為公主者,食封一千戶,中宗女亦同。其後,皇子封王者賜封二千戶,皇女
 為公主者賜封五百戶。咸宜賜湯沐,以母惠妃封至一千戶,諸皇女為公主者,例加至一千戶。其封自開元已來,皆約以三千為限。



 延王玢,玄宗第二十子也,初名洄。玢母即尚書右丞柳範孫也,最為名家,玄宗深重之。玢亦仁愛,有學問。開元十三年,封為延王。十五年,遙領安西大都護、磧西節度大使。二十三年七月,加開府儀同三司,餘如故,改名玢。天寶十五載,玄宗幸蜀,玢男女三十六人,不忍棄於道
 路,數日不及行在所,玄宗怒之;賴漢中王瑀抗疏救之,聽歸於靈武。興元元年薨。天寶末,封子倬彭城郡王、秘書監同正員,人延平陽郡王、殿中監同正員。



 盛王琦,玄宗第二十一子也。壽王母弟,初名沐。十三年三月,封為盛王。十五年,領揚州大都督。二十年,加開府儀同三司,餘如故,改名琦。天寶十五年六月,玄宗幸蜀,在路除琦為廣陵大都督,仍領江南東路及淮南河南等路節度支度採訪等使,以前江陵大
 都督府長史劉匯為之副,以廣陵長史李成式為副大使、兼御史中丞。琦竟不行。廣德二年四月薨,贈太傅。天寶末有子封王者二人:償真定郡王、太常卿同正員,佩封武都郡王、殿中監同正員。



 濟王環,玄宗第二十二子也,初名溢。開元十三年三月,封濟王。二十三年七月,授開府儀同三司,其月改名環。天寶末有子封為王者二人,傃為永嘉郡王、衛尉卿同正員,俛為平樂郡王、光祿卿同正員。



 信王瑝,玄宗第二十三子也,初名沔。開元十三年三月,封為信王。二十三年七月,授開府儀同三司,仍改名瑝。



 天寶末有子封為王者二人:人冬為新安郡王、太常卿同正員,倜為晉陵郡王、光祿卿同正員。



 義王玭,玄宗第二十四子也,初名漼。開元十三年三月,封為義王。二十三年七月,授開府儀同三司,仍改名玭。天寶末有子封為王者二人:儀為舞陽郡王、太僕卿同正員,僇為高密郡王、宗正卿同正員。



 陳王珪,玄宗第二十五子也,初名渙。開元二十三年七月,封為陳王。二十四年三月改名珪。天寶末男女二十一人,封為王者二人:佗為臨淮郡王、太常卿同正員,佼為安陽王、殿中監同正員。



 豐王珙,玄宗第二十六子也,初名澄。開元二十三年七月,封為豐王。二十四年二月改名珙。天寶十五年六月,玄宗幸蜀,至扶風郡,授珙武威郡都督,仍領河西隴右安西北庭等路節度支度採訪使;以隴右太守鄧景山
 為之副,兼武威長史、御史中丞,充都副大使。珙竟不行。



 廣德元年十月,吐蕃凌逼上都,上將幸陜州,自苑中而出,騎從半渡滻水。將軍王懷忠遂閉苑門,橫截五百餘騎,擁十宅諸王西投吐蕃。至城西,適遇元帥郭子儀,懷忠謂子儀曰:「主上東遷,社稷無主,萬國顒顒,何所瞻仰!今僕奉諸王等西奔,以副天下之望。令公身為元帥,廢置在手,何不行冊立之事乎?」子儀未及對,珙遂越次而言曰:「令公作何語,何不言也?」行軍司馬王延昌責之曰:「
 主上雖蒙塵於外,聖德欽明,王身為籓翰,何乃發狂悖之詞也?延昌當奏聞於上。」子儀又數讓之,命軍士領之盡赴行在。潼關謁見,上不之責,珙歸幕次,詞又不順。群臣恐遂為亂,請除之,遂賜死。天寶中有子二人為王:佻齊安郡王、宗正卿同正員,伷宜春郡王、鴻臚卿同正員。



 恆王瑱,玄宗第二十七子也,初名潓。開元二十三年七月,封為恆王。性好道,常服道士衣。授右衛大將軍,加開府儀同三司。二十四年二月改名瑱。天寶十五載,從幸
 巴蜀,不復衣道士衣矣。



 涼王璿,玄宗第二十九子也,初名漎。母武賢儀,則天時高平王重規女也,開元中入宮中,號為「小武妃」。二十三年七月,封為涼王。二十四年二月,改名璿。



 初,貞觀中,高宗為晉王,以文德皇后最少子,後崩後累年,太宗憐之,不令出閣,至立為太子。高宗朝,睿宗為豫王,雖成長,亦以則天最小子,不令出閣。及至聖歷初,封為相王,始出閣。中宗時,以譙王重福失愛,出遷外籓,衛王重俊為太
 子,入與成王千里等起兵,將誅韋後,故溫王重茂雖年十六七,竟亦居中。先天之後,皇子幼則居內,東封年,以漸成長,乃於安國寺東附苑城同為大宅,分院居,為十王宅。令中官押之,於夾城中起居,每日家令進膳。又引詞學工書之人入教,謂之侍讀。十王,謂慶、忠、棣、鄂、榮、光、儀、潁、永、延、濟,蓋舉全數。其後,盛、儀、壽、陳、豐、恆、涼六王又就封,入內宅。二十五年,鄂、光得罪,忠繼大統,天寶中,慶、棣又歿,唯榮、儀等十四王居院,而府幕列於外坊,時通
 名起居而已。外諸孫成長,又於十宅外置百孫院。每歲幸華清宮,宮側亦有十王院、百孫院。宮人每院四百,百孫院三四十人。又於宮中置維城庫,諸王月俸物,約之而給用。諸孫納妃嫁女,亦就十宅中。太子不居於東宮,但居於乘輿所幸之別院。太子亦分院而居,婚嫁則同親王、公主,在於崇仁之禮院。



 天寶十五載六月,玄宗幸蜀,儀王已下十三王從。至漢中郡,遣永王璘出鎮荊州。至德二年十月,從還京。廣德元年十二月五日,上都失
 守,有儀、潁、壽、延、盛、濟、信、義、陳、恆、涼十一王扈從,幸陜州。十二月,從還上都。璿之子,天寶中封為王者一人:仂,瀘陽郡王、殿中監同正員。



 汴哀王璥,玄宗第三十子也,初名滔。開元二十五年七月,封為汴王。二十四年二月,改名璥,以其月薨。



 史臣曰:前史有云:「母愛者子抱」,太子瑛之廢,有由然矣。琬為元帥,不幸遽薨,豈天啟亂階,何失眾望之速也!永王璘,父在蜀城,兄居靈武,不能立忠孝之節,為社稷
 之謀,而乃聚兵江上,規為己利,不義不暱,以災其身,《書》所謂「自作孽,不可逭」也。豐王珙因緣厄運,竊有覬覦,不慎樞機,自貽伊咎,悲矣!



 贊曰:《螽斯》之詠,樂有子孫。用建籓屏,以崇本根。讒勝瑛廢,恩移至尊。盜熾琬卒,情乖萬民。口禍豐珙,自災永璘。惜乎二胤,不如仁人。



\end{pinyinscope}