\article{卷一百二十九}

\begin{pinyinscope}

 ○張鎰馮河清附劉從一蕭復柳渾



 張鎰,蘇州人,朔方節度使齊丘之子也。以門廕授左衛兵曹參軍。郭子儀為關內副元帥,以嘗伏事齊丘,闢鎰
 為判官。授大理評事,遷殿中侍御史。乾元初,華原令盧樅以公事呵責邑人內侍齊令詵,令詵銜之,構誣。外發鎰按驗,樅當降官,及下有司,樅當杖死。鎰其公服白其母曰:「上疏理樅,樅必免死,鎰必坐貶。若以私則鎰負於當官,貶則以太夫人為憂,敢問所安?」母曰:爾無累於道,吾所安也。」遂執奏正罪,樅獲配流,鎰貶撫州司戶。量移晉陵令,未之官,洪吉觀察張鎬闢為判官,奏授殿中侍御史。遷屯田員外郎,轉祠部、右司二員外。母憂居喪有
 聞,免喪,除司勛員外。交游不雜,與楊綰、崔祐甫相善。大歷五年,除濠州刺史,為政清凈,州事大理。乃招經術之士,講訓生徒,比去郡,升明經者四十餘人。撰《三禮圖》九卷、《五經微旨》十四卷、《孟子音義》三卷。李靈曜反於汴州,鎰訓練鄉兵,嚴守禦之備,詔書褒異,加侍御史、沿淮鎮守使。尋遷壽州刺史,使如故。德宗即位,除江南西道都團練觀察使、洪州刺史、兼御史中丞,徵拜吏部侍郎,尋除河中晉絳都防禦觀察使。到官數日,改汴滑節度觀
 察使、汴州刺史、兼御史大夫,以疾辭,逗留於中路,徵入,養疾私第。未幾,拜中書侍郎、平章事、集賢殿學士,修國史。



 建中三年正月,太僕卿趙縱為奴當千發其陰事,縱下御史臺,貶循州司馬,留當千於內侍省。鎰上疏論之曰:



 伏見趙縱為奴所告下獄,人皆震懼,未測聖情。貞觀二年,太宗謂侍臣曰:比有奴告其主謀逆,此極弊法,特須禁斷。假令有謀反者,必不獨成,自有他人論之,豈藉其奴告也。自今已後,奴告主者皆不受,盡令斬決。」由是
 賤不得干貴,下不得陵上,教化之本既正,悖亂之漸不生。為國之經,百代難改,欲全其事體,實在防微。頃者長安令李濟得罪因奴,萬年令霍晏得罪因婢,愚賤之輩,悖慢成風,主反畏之,動遭誣告,充溢府縣,莫能斷決。建中元年五月二十八日,詔曰:準鬥競律,諸奴婢告主,非謀叛已上者,同自首法,並準律處分。」自此奴婢復順,獄訴稍息。今趙縱非叛逆,奴實奸兇,奴在禁中,縱獨下獄,考之於法,或恐未正。將帥之功,莫大於子儀;人臣之位,
 莫大於尚父。歿身未幾,墳土僅乾,兩婿先已當辜,趙縱今又下獄。設令縱實抵法,所告非奴,才經數月,連罪三婿。錄勛念舊,猶或可容,況在章程,本宜宥免。陛下方誅群賊,大用武臣,雖見寵於當時,恐息望於他日。太宗之令典尚在,陛下之明詔始行,一朝偕違,不與眾守,於教化恐失,於刑法恐煩,所益悉無,所傷至廣。臣非私趙縱,非惡此奴,叨居股肱,職在匡弼,斯昌大體,敢不極言。伏乞聖慈,納臣愚懇。



 上深納之,縱於是左貶而已,當千杖
 殺之。鎰乃令召子儀家僮數百人,以死奴示之。



 盧杞忌鎰名重道直,無以陷之,以方用兵西邊,杞乃偽請行,上固以不可,因薦鎰以中書侍郎為鳳翔隴右節度使代硃泚,與吐蕃相尚結贊等盟於清水。將盟,鎰與結贊約各以二千人赴壇所,執兵者半之,列於壇外二百步;散從者半之,分立壇下。鎰與賓佐齊映、齊抗及盟官崔漢衡、樊澤、常魯、于頔等七人,皆朝服;結贊與其本國將相論悉頰藏、論臧熱、論利陁、斯官者、論力徐等亦七人,俱升
 壇為盟。初,約漢以牛,蕃以馬為牲,鎰恥與之盟,將殺其禮,乃請結贊曰:「漢非牛不田,蕃非馬不行,今請以羊豕犬三物代之。」結贊許諾。時塞外無豕,結贊請以羝羊,鎰出犬、白羊,乃坎於壇北刑之,雜血一器而歃,盟文曰:



 唐有天下,恢奄禹跡,舟車所至,莫不率俾。以累聖重光,卜年惟永,恢王者之丕業,被四海以聲教。與吐蕃贊普,代為婚姻,因結鄰好,安危同體,甥舅之國,將二百年。其間或因小忿,棄惠為仇,封疆騷然,靡有寧歲。皇帝踐阼,愍
 茲黎元,乃釋俘囚悉歸蕃落。二國展禮,同茲協和,行人往復,累布成命。是必詐謀不起,兵革不用矣。彼猶以兩國之要,求之永久,古有結盟,今請用之。國家務息邊人,外其故地,棄利蹈義,堅盟從約。今國家所守界:涇州西至彈箏峽西口,隴州西至清水縣,鳳州西至同谷縣,暨劍南西山、大渡河東,為漢界。蕃國守鎮在蘭、渭、原、會,西至臨洮,又東至成州,抵劍南西界磨些諸蠻、大渡水西南,為蕃界。其兵馬鎮守之處州縣見有居人,彼此兩
 邊見屬漢諸蠻,以今所分見住處依前所有不載者,蕃有兵馬處蕃守,漢有兵馬處漢守,不得侵越。其先未有兵馬處,不得雜置並築城堡耕種。今二國將相受辭而會,齋戒將事,告天地山川之神,惟神昭臨,無得衍墜。其盟文藏於郊廟,副在有司,二國之誠,其永保之。



 結贊亦出盟文,不加於坎,但埋牲而已。盟畢,結贊請鎰就壇之西南隅佛幄中焚香為
 誓,誓畢,復升壇飲酒。獻酬之禮,各用其物,以將厚意而歸。



 德宗將幸奉天,鎰竊知之,將迎鑾駕,具財貨服用獻行在。李楚琳者,嘗事硃泚,得其心。軍司馬齊映等密謀曰:「楚琳不去,必為亂。」乃遣楚琳屯於隴州。楚琳知其謀,乃托故不時發。鎰始以迎駕心憂惑,以楚琳承命去矣,殊不促其行。鎰修飾邊幅,不為軍士所悅。是夜,楚琳遂與其黨王汾、李卓、牛僧伽等作亂。鎰夜縋而走,判官齊映自水竇出,齊抗為傭保負荷而逃,皆獲免。鎰出鳳翔
 三十里,及二子皆為候騎所得,楚琳俱殺之;判官王沼、張元度、柳遇、李漵被殺。尋贈太子太傅,葬事官給。



 馮河清者,京兆人也。初以武藝從軍,隸朔方節度郭子儀,以戰功授左衛大將軍同正;隸涇原節度馬璘,頻以偏師禦吐蕃,甚有殺獲之功。歷試太子詹事、兼御史中丞,充兵馬使。建中四年,節度使姚令言奉詔率兵赴關東,以河清知兵馬留後,判官、殿中侍御史姚況知州事。及令言至京師,所統兵叛,上幸奉天,河清與況聞之,乃
 集三軍大哭,因共激勵將吏,誓敦誠節,眾頗義之。即時發甲仗、器械、車百餘輛,連夜送行在所。時駕初遷幸,六軍雖集,蒼黃之際,都無戎器,及涇州甲仗至,軍士大振,特詔褒其誠效,拜四鎮北庭行軍涇原節度使、兼御史大夫;姚況兼御史中丞、行軍司馬。俄加河清檢校工部尚書。賊泚及姚令言累遣間諜招誘,河清輒拘而戮焉。及駕幸梁州,其將田希鑒潛通泚,使結兇黨害河清。尋贈尚書左僕射,葬事官給。興元元年,贈太子少傅。



 劉從一,中書侍郎林甫之玄孫也。祖令植,禮部侍郎。父孺之,京兆府少尹。從一少舉進士,大歷中宏詞,授秘書省校書郎,以調中第,補渭南尉,雅為常袞所推重。及袞為相,遷監察御史。居無何,丁母憂。服除,宰相盧杞薦之,超遷侍御史。居數月,以親避除刑部員外郎。建中末,普王之為元帥也,遷吏部郎中、兼御史中丞,為元帥判官。德宗居奉天,拜刑部侍郎、平章事,從幸梁州。明年六月,改中書侍郎、平章事。歲中,加集賢殿大學士、修史。上遇
 之甚厚,以容身遠罪而已,不能有所匡輔。無幾,以疾請告,至是,病甚辭位,章疏六上,乃許,除戶部尚書。尋卒,年四十四,輟朝三日,贈太子太傅。初,林甫生祥道,麟德初為右相,祥道即從一曾伯祖也。令植從父兄齊賢,弘道初為侍中。自祥道至從一,劉氏凡三相。



 蕭復,字履初,太子太師嵩之孫,新昌公主之子。父衡,太僕卿、駙馬都尉。少秉清操,其群從兄弟,競飾輿馬,以侈靡相尚,復衣浣濯之衣,獨居一室,習學不倦,非詞人儒
 士不與之游。伯華每嘆異之。以主廕,初為宮門郎,累至太子僕。



 廣德中,連歲不稔,穀價翔貴,家貧,將鬻昭應別業。時宰相王縉聞其林泉之美,心欲之,乃使弟竑誘焉,曰:「足下之才,固宜居右職,如以別業奉家兄,當以要地處矣。」復對曰:「僕以家貧而鬻舊業,將以拯濟孀幼耳,倘以易美職於身,令門內凍餒,非鄙夫之心也。」縉憾之,乃罷復官。沉廢數年,復處之自若。後累至尚書郎。大歷十四年,自常州刺史為潭州刺史、湖南觀察使。及為同州
 刺史,州人阻饑,有京畿觀察使儲廩在境內,復輒以賑貸,為有司所劾,削階。朋友唁之,復怡然曰:「茍利於人,敢憚薄罰。」尋為兵部侍郎。建中末,普王為襄漢元帥,以復為戶部尚書、統軍長史,以復父名衡,特詔避之,未行。扈駕奉天,拜吏部尚書、平章事。復嘗奏曰:「宦者自艱難已來,初為監軍,自爾恩幸過重。此輩只合委宮掖之寄,不可參兵機政事之權。」上不悅,又請別對,奏云:「陛下臨御之初,聖德光被,自用楊炎、盧杞秉政,惛瀆皇猷,以致今
 日。今雖危急,伏願陛下深革睿思,微臣敢當此任。若令臣依阿偷免,臣不敢曠職。」盧杞奏對於上前,阿諛順旨,復正色曰:「杞之詞不正。」德宗愕然,退謂左右曰:「蕭復頗輕朕。」遂令往江南宣撫。



 先時,淮南節度陳少游首稱臣於李希烈,鳳翔將李楚琳殺節度使張鎰以應硃泚,鎰判官韋皋先知隴州留後,首殺豳叛卒數百人,不應楚琳。復江南使回,與宰相同對訖,復獨留,奏曰:「陛下自返宮闕,勛臣已蒙官爵,唯旌善懲惡,未有區分。陳少游
 將相之寄最崇,首敗臣節;韋皋名宦最卑,特建忠義。請令韋皋代少游,則天下明然知逆順之理。」上許之。復出,宰相李勉、盧翰、劉從一方同歸中書,中使馬欽緒至,揖從一,附耳語而退,諸相各歸閣。從一詣復曰:「適欽緒宣旨,令與公商量朝來所奏便進,勿令李勉、盧翰知。」復曰:「適來奏對,亦聞斯旨,然未諭聖心,已面陳述,上意尚爾,復未敢言其事。」復又曰:「唐、虞有僉曰之論,朝廷有事,尚合與公卿同議。今勉、翰不可在相位,即去之;既在相位,
 合同商量,何故獨避此之一節?且與公行之無爽,但恐浸以成俗,此政之大弊也。」竟不言於從一。從一奏之,上浸不悅。復累表辭疾,請罷知政事,從之,守太子左庶子。三年,坐郜國公主親累,檢校左庶子,於饒州安置。四年,終於饒州,時年五十七。



 復門望高華,志礪名節,與流俗不甚通狎。及登臺輔,臨事不茍,頗為同列所嫉,以故居位不久。性孝友,居家甚睦,為族子所累,晏然屏退,口未嘗言。



 郜國公主者,肅宗之女也,出降駙馬蕭升,升於復
 為從兄弟,升早卒。貞元中,蜀州別駕蕭鼎、商州豐陽令韋恪、前彭州司馬李萬、太子詹事李升等出入主第,穢聲流聞。德宗怒,幽主於別第,李萬決殺,升貶嶺南,蕭鼎、韋恪決四十,長流嶺表。又言公主行厭禱,其子位為禱文,位弟佩、儒、偲及異父兄駙馬都尉裴液,並長流端州。公主女為皇太子妃,即順宗也。太子懼,亦請與妃離婚。六年,郜國薨,位兄弟及液詔還京師。液父徽,初尚郜國;徽卒,降蕭升。



 柳渾,字夷曠,襄州人,其先自河東徙焉。六代祖惔,梁僕射。渾少孤,父慶休,官至渤海丞,而志學棲貧。天寶初,舉進士,補單父尉。至德中,為江西採訪使皇甫侁判官,累除衢州司馬。未至,召拜監察御史。臺中執法之地,動限儀矩,渾性放,不甚檢束,僚長拘局,忿其疏縱。渾不樂,乞外任,執政惜其才,奏為左補闕。明年,除殿中侍御史,知江西租庸院事。



 大歷初,魏少游鎮江西,奏署判官,累授檢校司封郎中。州理有開元寺僧與徒夜飲,醉而延
 火,歸罪於守門瘖奴,軍候亦受財,同上其狀,少游信焉。人知奴冤,莫肯言。渾與崔祐甫遽入白,少游驚問,醉僧首伏。既而謝曰:「微二君子,幾成老夫暗劣矣。」自此以公正聞。及路嗣恭領鎮,復以為都團練副使。十二年,拜袁州刺史。居二年,崔祐甫入相,薦為諫議大夫、浙江東西黜陟使,累遷尚書左丞。及駕在奉天,微服徒行,遁終南山谷,逾旬方達行在。扈從至梁州,改左散騎常侍。初,渾之歸行在,賊泚籍其名甚,願以致之,猶疑匿在閭里,乃
 加宰相。及克復,渾尚名載,乃上言:「頃為狂賊點穢,臣實恥稱舊名,矧字或帶戈,時當偃武,請改名渾。」



 貞元二年,拜兵部侍郎,封宜城縣伯。三年正月,加同平章事,仍判門下省。時上命玉工為帶,墜壞一銙,乃私市以補;及獻,上指曰:「此何不相類?」工人伏罪,上命決死。詔至中書,渾執曰:「陛下若便殺則已,若下有司,即須議讞。且方春行刑,容臣條奏定罪。」以誤傷乘輿器服,杖六十,餘工釋放,詔從之。復奏:「故尚書左丞田季羔,公忠正直,先朝名臣。
 其祖、父皆以孝行旌表門閭,京城隋朝舊第,季羔一家而已。今被堂侄伯強進狀,請貨宅召市人馬,以討吐蕃。一開此門,恐滋不逞。討賊自有國計,豈資僥幸之徒?且毀棄義門,虧損風教,望少責罰,亦可懲勸。」上可其奏。



 先時,韓滉自浙西入覲,朝廷委政待之,至於調兵食,籠鹽鐵,勾官吏贓罰,鋤豪強兼並,上悉仗焉。每奏事,或日旰,他相充位而已,公卿救過不能暇,無敢枝梧者。渾雖滉所引,心惡其專政,正色讓之曰:「先相公以狷察為相,不
 滿歲而罷;今相公搒吏於省中至死,且非刑人之地,奈何蹈前非而又甚焉?專立威福,豈尊主卑臣之禮!」滉感悟愧悔,為霽威焉。及白志貞除浙西觀察使,渾奏曰:「志貞一末吏憸人,縱稱廉謹,不當頓居重職。」適遇渾以疾稱告,即日詔下。疾間,因乞骸骨,優詔不許。其判門下,主吏白當過官,渾愀然曰:「列官分職,復更撓之,非禮法也。千里辭家,以干微祿,邑主辭辦,豈慮無能,矧旌善進賢,事不在此。」故其年注擬,無退量者。



 及渾瑊與吐蕃會盟
 之日,上御便殿謂宰相曰:「和戎息師,國之大計,今日將士與卿同歡。」馬燧前賀曰:今之一盟,百年內更無蕃寇。」渾曰:「五帝無誥誓之盟,皆在季末。今盛明之代,豈又行於夷狄!人面獸心,難以信結,今日盟約,臣竊憂之。」李晟繼言曰:「臣生長邊城,知蕃戎心,今日之事,誠如渾言。」上變色曰:「柳渾書生,未達邊事;大臣智略,果亦有斯言乎!」皆頓首俯伏,遽令歸中書。其夜三更,邠寧節度韓游瑰飛驛叩苑門,奏盟會不成,將校覆沒,兵臨近鎮,上驚嘆,
 即遞其表以示渾。詰旦,臨軒慰勉渾曰:「卿文儒之士,而萬里知軍戎之情。」自此驟加禮異。時張延賞與渾同列,延賞怙權矜己,而嫉渾守正,俾其所厚謂渾曰:「相公舊德,但節言於廟堂,則重位可久。」。渾曰:「為吾謝張相公,柳渾頭可斷,而舌不可禁也。」自是為其所擠,尋除常侍,罷知政事。貞元五年二月,以疾終,年七十五。有文集十卷。



 渾母兄識,,篤意文章,有重名於開元、天寶間,與蕭穎士、元德秀、劉迅相亞。其練理創端往往詣極,當時作者,咸
 伏其簡拔,而趣尚辨博。渾亦善為文,然趨時向功,非沉思之所及。渾警辯,好諧謔放達,與人交,豁然無隱。性節儉,不治產業,官至丞相,假宅而居。罷相數日,則命親族尋勝,宴醉方歸,陶陶然忘其黜免。時李勉、盧翰皆退罷居第,相謂曰:「吾輩方柳宜城,悉為拘俗之人也。」



 史臣曰:張鎰、蕭復、柳渾,節行才能訏謨亮直,皆足相明主,平泰階,而盧杞忌之於前,延賞排之於後,管仲有言:「任君子,使小人間之,害霸也。」德宗黜賢相,位奸臣,致硃
 泚、懷光之亂,是失其人也,豈尤其時哉!河清歿於王事,乃顯忠貞;從一舉自奸人,固宜循默。



 贊曰:得人則興,失人則亡。鎰、復、渾去,宗社其殃。



\end{pinyinscope}