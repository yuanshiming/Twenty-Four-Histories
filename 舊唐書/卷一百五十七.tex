\article{卷一百五十七}

\begin{pinyinscope}

 ○姚南仲劉乃子伯芻孫寬夫端夫曾孫允章附袁高段平仲薛存誠子廷老廷老子保遜保遜子昭緯盧坦



 姚南仲,華州下邽人。乾元初,制科登第,授太子校書,歷高陵、昭應、萬年三縣尉。遷右拾遺,轉右補闕。大歷十三
 年,貞懿皇后獨孤氏崩,代宗悼惜不已,令於近城為陵墓,冀朝夕臨望於目前。南仲上疏諫曰:



 伏聞貞懿皇后今於城東章敬寺北以起陵廟,臣不知有司之請乎,陛下之意乎,陰陽家流希旨乎?臣愚以為非所宜也。謹具疏陳論,伏願暫留天睠而省察焉。



 臣聞人臣宅於家,君上宅於國。長安城,是陛下皇居也,其可穿鑿興動,建陵墓於其側乎?此非宜一也。



 夫葬者藏也,欲人之不得見也。是以古帝前王葬後妃,莫不憑丘原,遠郊郭。今則西
 臨宮闕,南迫康莊,若使近而可見,死而復生,雖在西宮待之可也。如骨肉歸土,魂無不之,章敬之北,竟何所益?視之兆庶,則彰溺愛;垂之萬代,則累明德,此非所宜二也。



 夫帝王者,居高明,燭幽滯。先皇所以因龍首建望春,蓋為此也。今若起陵目前,動傷宸慮,天心一傷,數日不平。且匹夫向隅,滿堂為之不樂;萬乘不樂,人其可歡心乎?又暇日起歌,動鐘於內,此地皆聞,此非宜三也。



 伏以貞懿皇后,坤德合天,母慈逮下,陛下以切軫旒扆,久俟
 蓍龜。始謚之以貞懿,終待之以褻近,臣竊惑焉,非所以稱述後德,光被下泉也。今國人皆曰:「貞懿皇后之陵邇於城下者,主上將日省而時望焉。」斯有損於聖德,無益於貞懿。將欲寵之,而反辱之,此非宜四也。



 凡此數事,實玷大猷,天下咸知,伏惟陛下熟計而取其長也。陛下方將偃武靖人,一誤於此,其傷實多。臣恐君子是非,史官褒貶,大明忽虧於掩蝕,至德翻後於堯、舜,不其惜哉!今指日尚遙,改卜何害?抑皇情之殊眷,成貞懿之美號。



 疏
 奏,帝甚嘉之,賜緋魚袋,特加五品階,宣付史館。



 與宰相常袞善,袞貶官,南仲坐出為海鹽縣令。浙江東、西道觀察使韓滉闢為推官,奏授殿中侍御史、內供奉,充支使。尋徵還,歷左司兵部員外,轉郎中,遷御史中丞、給事中、同州刺史、陜虢觀察使。



 貞元十五年,代李復為鄭滑節度使。監軍薛盈珍恃勢奪軍政,南仲數為盈珍讒毀,德宗頗疑之。十六年,盈珍遣小使程務盈馳驛奉表,誣奏南仲陰事。南仲裨將曹文洽亦入奏事京師,伺知盈珍
 表中語。文洽私懷憤怒,遂晨夜兼道追務盈,至長樂驛及之,與同舍宿;中夜殺務盈,沉盈珍表於廁中,乃自殺。日旰,驛吏闢門,見血流塗地,旁得文洽二緘,一告於南仲,一表理南仲之冤,且陳首殺務盈。上聞其事,頗駭異之。南仲慮釁深,遂乞入朝。德宗曰:「盈珍擾軍政耶?」南仲對曰:「勇珍不擾軍政,臣自隳陛下法耳。如盈珍輩所在有之,雖羊、杜復生,撫百姓,御三軍,必不能成愷悌父母之政,師律善陣之制矣。」上默然久之。授尚書右僕射。貞
 元十九年七月,終於位,年七十四,贈太子太保,謚曰貞。



 劉乃,字永夷,洺州廣平人。高祖武干,武德初拜侍中,即中書侍郎林甫從祖兄子也。父如璠,昫山丞,以乃貴,贈民部郎中。乃少聰穎志學,暗記《六經》,日數千言。及長,文章清雅,為當時推重。天寶中,舉進士,尋丁父艱,居喪以孝聞。既終制,從調選曹。乃常以文部選才未為盡善,遂致書於知銓舍人宋昱曰:



 《虞書》稱:「知人則哲,能官人則惠。」巍巍唐、虞,舉以為難。今夫文部,既始之以掄材,終之
 以授位,是則知人官人,斯為重任。昔在禹、稷、皋陶之眾聖,猶曰載採有九德,考績以九載。近代主司,獨委一二小塚宰,察言於一幅之判,觀行於一揖之內,古今遲速,何不侔之甚哉!夫判者,以狹詞短韻,語有定規為體,亦猶以一小冶而鼓眾金,雖欲為鼎為鏞,不可得也。故曰:判之在文,至局促者。夫銓者,必以崇衣冠,自媒耀為賢,斯又士之醜行,君子所病。若引文公、尼父登之於銓廷,則雖圖書《易象》之大訓,以判體挫之,曾不及徐、庾。雖有
 至德,以喋喋取之,曾不若嗇夫。嗚呼!彼幹霄蔽日,誠巨樹也,當求尺寸之材,必後於椓杙。龍吟武嘯,誠希聲也,若尚頰舌之感,必下於蛙黽。觀察之際,猶不悲夫!執事慮過龜策,文合雅誥,豈拘以瑣瑣故事,曲折因循哉?誠能先資以政事,次征以文學,退觀其理家,進察其臨節,則厖鴻深沉之事,亦可以窺其門戶矣!



 其載,補剡縣尉,改會稽尉。宣州觀察使殷日用奏為判官,宣慰使李季卿又以表薦,連授大理評事、兼監察御史。轉運使劉晏
 奏令巡覆江西,多所蠲免。改殿中侍御史、檢校倉部員外、民部郎中,並充浙西留後。佐晏征賦,頗有裨益,晏甚任之。



 大歷十二年,元載既誅,以乃久在職,召拜司門員外郎。十四年,崔祐甫秉政,素與乃友善。會加郭子儀尚父,以冊禮久廢,至是復行之。祐甫令兩省官撰冊文,未稱旨;召乃至閣草之,立就。詞義典裁,祐甫嘆賞久之。數日,擢為給事中,尋遷權知兵部侍郎。及楊炎、盧杞為相,意多醜正,以故五歲不遷。建中四年夏,但真拜而已。



 其
 冬,涇師作亂,駕幸奉天。乃臥疾在私第,賊泚遣使以甘言誘之,乃稱疾篤。又令其偽宰相蔣鎮自來招誘,乃托喑疾,灸灼遍身。鎮再至,知不可劫脅,乃嘆息曰:「鎮亦嘗忝列曹郎,茍不能死,以至於斯,寧以自辱膻腥,復欲污穢賢哲乎?」歔欷而退。及聞輿駕再幸梁州,乃自投於床,搏膺呼天,因是危惙,絕食數日而卒,時年六十。德宗還京,聞乃之忠烈,追贈禮部尚書。子伯芻。



 伯芻,字素芝,登進士第,志行修謹。淮南杜佑闢為從事,府罷,屏居吳中。
 久之,徵拜右補闕,遷主客員外郎。以過從友人飲噱,為韋執誼密奏,貶虔州掾曹,復為考功員外郎裴垍善其應對機捷,遷考功郎中、集賢院學士,轉給事中。裴垍罷相,為太子賓客,未幾而卒。李吉甫復入相,與垍宿嫌,不加贈官;伯芻上疏論之,贈垍太子少傅。伯芻妻,垍從姨也。或讒於吉甫,此以論奏。伯芻懼,亟請散地,因出為虢州刺史。吉甫卒,裴度擢為刑部侍郎,俄知吏部選事。元和十年,以左常侍致仕,卒,年六十一,贈工部尚書。伯芻
 風姿古雅,涉學,善談笑,而動與時適,論者稍薄之。



 子寬夫,登進士第,歷諸府從事。寶歷中,入為監察御史。嘗上言曰:「近日攝祭多差王府官僚,位望既輕,有乖嚴敬。伏請今後攝太尉,差尚書省三品已上及保傅賓詹等官;如人少,即令丞郎通攝之。」俄轉左補闕。少列陳岵進注《維摩經》,得濠州刺史。寬夫與同列,因對論之,言岵因供奉僧進經以圖郡牧。敬宗怒謂宰相曰:「陳岵不因僧得郡,諫官安得此言,須推排頭首來。」寬夫奏曰:「昨論陳岵
 之時,不記發言前後,唯握筆草狀,即是微臣。今論事不當,臣合當罪。若尋究推排,恐傷事體。」帝嘉其引過,欣然釋之。



 寬夫弟端夫,為太常博士,駁韋綬謚議知名。寬夫子允章、煥章。



 允章登進士第,累官至翰林學士承旨、禮部侍郎。咸通九年,知貢舉,出為鄂州觀察使、檢校工部尚書,後遷東都留守。黃巢犯洛陽,允章不能拒,賊不之害,坐是廢於家。以疾卒。



 袁高,字公頤,恕己之孫。少慷慨,慕名節。登進士第,累闢
 使府,有贊佐裨益之譽。代宗登極,徵入朝,累官至給事中、御史中丞。建中二年,擢為京畿觀察使。以論事失旨,貶韶州長史,復拜為給事中。



 貞元元年,德宗復用吉州長史盧杞為饒州刺史,令高草詔書。高執詞頭以謁宰相盧翰、劉從一曰:「盧杞作相三年,矯詐陰賊,退斥忠良。朋附者咳唾立至青雲、睚眥者顧盼已擠溝壑。傲很明德,反易天常,播越鑾輿,瘡痍天下,皆杞之為也。爰免族戮,雖示貶黜,尋已稍遷近地,若更授大郡,恐失天下之
 望。惟相公執奏之,事尚可救。」翰、從一不悅,改命舍人草之。詔出,執之不下,仍上奏曰:「盧杞為政,窮極兇惡。三軍將校,願食其肉;百闢卿士,嫉之若讎。」遺補陳京、趙需、裴佶、宇文炫、盧景亮、張薦等上疏論奏。次日,又上疏。高又於正殿奏云:「陛下用盧杞獨秉鈞軸,前後三年,棄斥忠良,附下罔上,使陛下越在草莽,皆杞之過。且漢時三光失序,雨旱不時,皆宰相請罪,小者免官,大者刑戮。杞罪合至死,陛下好生惡殺,赦杞萬死,唯貶新州司馬,旋復
 遷移。今除刺史,是失天下之望。伏惟聖意裁擇。」上謂曰:「盧杞有不逮,是朕之過。」復奏曰;「盧杞奸臣,常懷詭詐,非是不逮。」上曰:「朕已有赦。高曰:「赦乃赦其罪,不宜授刺史。且赦文至優黎民,今饒州大郡,若命奸臣作牧,是一州蒼生,獨受其弊。望引常參官顧問,並擇謹厚中官,令採聽於眾。若億兆之人異臣之言,臣當萬死。」於是,諫官爭論於上前,上良久謂曰:「若與盧杞刺史太優,與上佐可乎?」曰:「可矣!」遂追饒州制。翌日,遣使宣慰高云:「朕思卿言
 深理切,當依卿所奏。」太子少保韋倫、太府卿張獻恭等奏:「袁高所奏至當,高是陛下一良臣,望加優異。」



 貞元二年,上以關輔祿山之後,百姓貧乏,田疇荒穢,詔諸道進耕牛,待諸道觀察使各選揀牛進貢,委京兆府勸課民戶,勘責有地無牛百姓,量其地著,以牛均給之。其田五十畝已下人,不在給限。高上疏論之:「聖慈所憂,切在貧下。有田不滿五十畝者尤是貧人,請量三兩家共給牛一頭,以濟農事。」疏奏,從之。尋卒於官,年六十,中外嘆惜。
 憲宗朝,宰臣李吉甫嘗言高之忠鯁,詔贈禮部尚書。



 段平仲,字秉庸,武威人。隋人部尚書段達六代孫也。登進士第。杜佑、李復相繼鎮淮南,皆表平仲為掌書記。復移鎮華州、滑州,仍為從事。入朝為監察御史。平仲磊落尚氣節,嗜酒傲言。時德宗春秋高,多自聽斷。由是庶務壅隔,事或不理,中外畏上嚴察,無敢言者。平仲嘗謂人曰:「主上聰明神武,臣下畏懼不言,自循默耳。如平仲一得召見,必當大有開悟。」貞元十四年,京師旱,詔擇御史、
 郎官各一人,發廩賑恤。平仲與考功員外陳歸當奉使,因辭得對,乃入近御座,粗陳本事。上察平仲意有所蓄,以歸在側不言。及奏事畢退,平仲獨不退,欲有奏啟;上因兼留歸問之,聲色甚厲,雜以他語。平仲錯愕,都不得言因誤稱其名。上怒,叱出之。平仲蒼黃,又誤趨御障後,歸下階連呼,乃得出。由是坐廢七年,然亦因此名顯。



 後除屯田膳部二員外郎、東都留守判官,累拜右司郎中。元和初,遷諫議大夫。內官吐突承璀為招討使,征鎮州,
 無功而還。平仲與呂元膺抗疏論列,請加黜責。轉給事中。自在要近,朝廷有得失,未嘗不論奏,時人推其狷直。轉尚書左丞,以疾改太子左庶子卒。



 薛存誠,字資明,河東人。父勝,能文,嘗作《拔河賦》,詞致瀏亮,為時所稱。存誠進士擢第,累闢使府,入朝為監察御史,知館驛。元和初,王師討劉闢,郵傳多事,上特令中官為館驛使。存誠密表論奏,以為有傷公體。會諫官亦論奏,上乃罷之。轉殿中侍御史,遷度支員外郎。裴垍作相,
 用為起居郎,轉司勛員外、刑部郎中、兼侍御史、知雜事,改兵部郎中、給事中。瓊林庫使奏占工徒太廣,存誠以為此皆奸人竄名,以避征役,不可許。咸陽縣尉袁儋與軍鎮相競,軍人無理,遂肆侵誣,儋反受罰。二敕繼至,存誠皆執之。上聞甚悅,命中使嘉慰之,由是擢拜御史中丞。



 僧鑒虛者,自貞元中交結權倖,招懷賂遺,倚中人為城社,吏不敢繩。會于頔、杜黃裳家私事發,連逮鑒虛下獄。存誠案鞫得奸贓數十萬,獄成,當大闢。中外權要,更
 於上前保救,上宣令釋放,存誠不奉詔。明日,又令中使詣臺宣旨曰:「朕要此僧面詰之,非赦之也。」存誠附中使奏曰:「鑒虛罪款已具,陛下若召而赦之,請先殺臣,然後可取。不然,臣期不奉詔。」上嘉其有守,從之,鑒虛竟笞死。洪州監軍高重昌誣奏信州刺史李位謀大逆,追赴京師。上令付仗內鞫問。存誠一日三表,請付位於御史臺。及推案無狀,位竟得雪。



 未幾,再授給事中。數月,中丞闕,上思存誠前效,謂宰相持憲無以易存誠,遂復為御史
 中丞。未視事,暴卒。憲宗深惜之,贈刑部侍郎。存誠性和易,於人無所不容,及當官御事,即確乎不拔,士友以是稱重之。子廷老。



 廷老謹正有父風,而性通銳。寶歷中為右拾遺。敬宗荒恣,宮中造清思院新殿,用銅鏡三千片、黃白金薄十萬番。廷老與同僚入閣奏事曰:「臣伏見近日除拜,往往不由中書進擬,或是宣出。伏恐綱紀漸壞,奸邪恣行。」敬宗厲聲曰:「更諫何事?」舒元褒對曰:「近日宮中修造太多。」上色變曰:「何處修造?」元褒不能對,廷老進
 曰:「臣等職是諫官,凡有所聞,即合論奏。莫知修造之所,但見運瓦木絕多,即知有用。乞陛下勿罪臣言。」帝曰:「所奏已知。」尋加史館修撰。



 時李逢吉秉權,惡廷老言太切直。鄭權因鄭注得廣州節度,權至鎮,盡以公家珍寶赴京師以酬恩地。廷老上疏請按權罪,中人由是切齒。又論逢吉黨人張權輿、程昔範不宜居諫列,逢吉大怒。廷老告滿十旬,逢吉乃出廷老為臨晉縣令。



 文宗即位,入為殿中侍御史。太和四年,以本官充翰林學士,與同職
 李讓夷相善。廷老之入內署,讓夷薦挈之。廷老性放逸嗜酒,不持檢操,終日酣醉,文宗知之不悅。五年,罷職,守本官,讓夷亦坐廷老罷職,守職方員外郎。廷老尋拜刑部員外郎,轉郎中,遷給事中。開成三年卒。廷老當官舉職,不求虛譽,侃侃於公卿之間,甚有正人風望。贈刑部侍郎。



 子保遜,登進士第,位亦至給事中。



 保遜子昭緯,乾寧中為禮部侍郎,貢舉得人,文章秀麗。為崔胤所惡,出為磎州刺史,卒。



 盧坦,字保衡,河南洛陽人,其先自範陽徙焉。父巒,贈鄭州刺史。坦嘗為義成軍判官,節度使李復疾篤,監軍使薛盈珍慮變,遽封府庫,入其麾下五百人於使牙,軍中恟々;坦密言於盈珍促收之。及復卒,坦護喪歸東都。後為壽安令。



 時河南尹征賦限窮,而縣人訴以機織未就;坦請延十日,府不許。坦令戶人但織而輸,勿顧限也,違之不過罰令俸耳。既成而輸,坦亦坐罰,由是知名。累遷至庫部員外郎、兼侍御史、知雜事。會李錡反,有司請毀
 錡祖父廟墓。坦常為錡從事,乃上言曰:「淮安王神通有功於草昧。且古之父子兄弟,罪不相及,況以錡故累五代祖乎?」乃不毀。因賜神通墓五戶,以備灑掃。及武元衡為宰相,以坦為中丞,李元素為大夫,命坦分司東都,未幾歸臺。裴均為僕射,在班逾位,坦請退之,均不受。坦曰:「姚南仲為僕射,例如此。」均曰:「南仲何人?」坦曰:「南仲是守正而不交權幸者也。」尋罷為右庶子,時人歸咎於均。旬月,出為宣歙池觀察使。三年,入為刑部侍郎、鹽鐵轉運
 使,改戶部侍郎、判度支。



 元和八年,西受降城為河徙浸毀,宰相李吉甫請移兵於天德故城。坦與李絳葉議,以為:「西城張仁願所築,制匈奴上策。城當磧口,居虜要沖,美水豐草,邊防所利。今河流之決,不過退就二三里,奈何舍萬代安永之策,徇一時省費之謀?況天德故城僻處確瘠,其北枕山,與河絕遠,烽候警備,不相統接。虜之唐突,勢無由知,是無故而蹙國二百里,非所利也。」及城使周懷義奏利害,與坦議同。事竟不行。未幾,出為劍南
 東川節度使。在鎮累年,後請收閏月軍吏糧料,以助軍行營,人多非之。貞元十二年九月卒,年六十九,贈禮部尚書。



 史臣曰:古之諍臣,有死於言者。其次,引裾折檻,不改其操,亦難矣哉!袁高之執盧杞,存誠之戮鑒虛,有古人之遺風焉!平仲觸鱗之氣,糾其謬歟?文洽奪章,以攄府憤;永夷絕食,不飲盜泉,節義之士也。南仲非葬之言,盧坦西城之議,量之深也。如數子,道為時無君子,乃是厚誣。



 贊曰:靈草指佞,諫臣匡失。惟袁與薛,人中屈軼。寬夫雀躍,廷老鴻軒。姚、盧啟奏,君子之言。



\end{pinyinscope}