\article{卷九十}

\begin{pinyinscope}

 ○燕王忠原王孝澤王上金許王素節孝敬皇帝弘裴居道附章懷太子賢賢子邠王守
 禮懿德太子重潤庶人重福節愍太子重俊殤帝重茂



 高宗八男:則天順聖皇后生中宗、睿宗及孝敬皇帝弘、章懷太子賢,後宮劉氏生燕王忠,鄭氏生原王孝,楊氏生澤王上金,蕭淑妃生許王素節。



 燕王忠,字正本,高宗長子也。高宗初入東宮而生忠,宴宮僚於弘教殿。太宗幸宮,顧謂宮臣曰:「頃來王業稍可,非無酒食,而唐突卿等宴會者,朕初有此孫,故相就為
 樂耳。」太宗酒酣起舞,以屬群臣,在位於是遍舞,盡日而罷,賜物有差。



 貞觀二十年,封為陳王。永徽元年,拜雍州牧。時王皇后無子,其舅中書令柳奭說後謀立忠為皇太子,以忠母賤,冀其親己,後然之。奭與尚書右僕射褚遂良、侍中韓瑗諷、太尉長孫無忌、右僕射於志寧等,固請立忠為儲後,高宗許之。三年,立忠為皇太子,大赦天下,五品已上子為父後者賜勛一級。六年,加元服,制大闢罪已下並降一等,大酺三日。其年,王皇后被廢,武昭
 儀所生皇子弘年三歲。禮部尚書許敬宗希旨上疏曰:「伏惟陛下憲章千古,含育萬邦,爰立聖慈,母儀天下。既而皇后生子,合處少陽。出自塗山,是謂吾君之胤;夙聞胎教,宜展問豎之心。乃復為孽奪宗,降居籓邸,是使前星匿彩,瑤岳韜峰。臣以愚誠,竊所未喻。且今之守器,素非皇嫡,永徽爰始,國本未生,權引彗星,越升明兩。近者元妃載誕,正胤降神,重光日融,爝暉宜息。安可以茲傍統,叨據溫文?國有諍臣,孰逃其責!竊惟息姑克讓,可以
 思齊;劉強守籓,宜遵往軌。追跡太伯,不亦休哉?踵武延陵,故常安矣。寧可反植枝幹,久易位於天庭;倒襲衣裳,使違方於震位?蠢爾黎庶,云誰系心?垂裕後昆,將何播美?」高宗從之。顯慶元年,廢忠為梁王,授梁州都督,賜實封二千戶,物二萬段,甲第一區。其年,轉房州刺史。



 忠年漸長大,常恐不自安,或私衣婦人之服,以備刺客。又數有妖夢,常自占卜。事發,五年,廢為庶人,徙居黔州,囚於承乾之故宅。麟德元年,又誣忠與西臺侍御上官儀、宦
 者王伏勝謀反,賜死於流所,年二十二,無子。儀等伏誅。明年,皇太子弘表請收葬,許之。神龍初,追封燕王,贈太尉、揚州大都督。



 原王孝,高宗第二子也。永徽元年,封許王。三年,拜並州都督。顯慶三年,累除遂州刺史。麟德元年薨,贈益州大都督,謚曰悼。神龍初,追贈原王、司徒、益州大都督。



 澤王上金,高宗第三子也。永徽元年,封巳王。三年,遙授益州大都督。乾封元年,累轉壽州刺史,有罪免官,削封
 邑,仍於澧州安置。上金既為則天所惡,所司希旨,求索罪失以奏之,故有此黜。永隆二年二月,則天矯抗表巳王上金、鄱陽王素節許同朝集之例,義陽、宣城二公主緣母蕭氏獲譴,從夫外官,請授官職。以上金為沔州刺史,素節為岳州刺史,仍不預朝集。嗣聖元年,上金、素節,義陽、宣城二公主聽赴哀。文明元年,上金封畢王,素節封為葛王。又改上金封為澤王、蘇州刺史,素節許王、隆州刺史。垂拱元年,改陳州刺史。永昌元年,授太子左衛
 率,出為隨州刺史。載初元年,武承嗣使酷吏周興誣告上金、素節謀反,召至都,系於御史臺。舒州刺史、許王素節見殺於都城南驛,因害其支黨。上金恐懼,自縊死。子義珍、義玫、義璋、義環、義瑾、義璲七人並配流顯州而死。神龍初,追復上金官爵,封庶子義珣為嗣澤王。



 先是,義珣竄在嶺外,匿於傭保之間。及紹封無幾,有人告義珣非上金子,假冒襲爵。義珣不能自明,復流於嶺外。開元初,封素節子璆為嗣澤王,繼上金後。十二年,玉真公主
 表稱義珣實上金遺胤,被嗣許王瓘兄弟利其封爵,謀構廢之。今上由是削璆王爵,復召義珣為嗣澤王,拜率更令。因是,諸宗室非本宗襲爵,自中興已後繼為嗣王者,皆令歸宗,削其爵邑也。



 許王素節,高宗第四子也。年六歲,永徽二年,封雍王,尋授雍州牧。素節能日誦古詩賦五百餘言,受業於學十徐齊聃,精勤不倦,高宗甚愛之。又轉岐州刺史。年十二,改封郇王。



 初,則天未為皇后也,與素節母蕭淑妃爭寵,
 遞相譖毀。六年,則天立為皇后後,淑妃竟為則天所譖毀,幽辱而殺之。素節尤被讒嫉,出為申州刺史。乾封初,下敕曰:「素節既舊疾患,宜不須入朝。」而素節實無疾。素節自以久乖朝覲,遂著《忠孝論》以見意,詞多不載。時王府倉曹參軍張柬之因使潛封此論以進,則天見之,逾不悅,誣以贓賄,降封鄱陽郡王,仍於袁州安置。儀鳳二年,禁錮終身,又改於岳州安置。永隆元年,轉岳州刺史,後改封葛王。則天稱制,又進封許王,累除舒州刺史。天
 授中,與上金同被誣告,追赴都。臨發州,聞有遭喪哭者,謂左右曰:「病死何由可得,更何須哭!」行至都城南龍門驛,被縊死,年四十三,則天令以庶人禮葬之。中宗即位,追封許王,贈開府儀同三司、許州刺史,仍以禮改葬,陪於乾陵。



 素節被殺之時,子瑛、琬、璣、易等九人並為則天所殺,惟少子琳、瓘、璆、欽古以年小,特令長禁雷州。神龍初,封瓘為嗣許王。開元初,封琳為嗣越王,以紹越王貞之後。璆為嗣澤王,以繼伯父澤王上金之後。琳,官至右
 監門將軍,卒。瓘,開元十一年為衛慰卿。以抑伯上金男不得承襲,以弟璆繼之,遽譴瓘為鄂州別駕。於是下詔絕其外繼,乃以故澤王上金男義珣為嗣澤王,江王禕為信安郡王,嗣蜀王褕為廣漢郡王,嗣密王徹為濮陽郡王,嗣曹王臻為濟國公,嗣趙王琚為中山郡王,武陽郡王繼宗為澧國公。瓘累遷邠州刺史、秘書監、守太子詹事。璆性仁厚謹願,居家邕睦,朝廷重之。天寶六載卒,贈蜀郡大都督。瓘晚有子,命璆子益為嗣。及卒,有
 解、需二子,皆幼孺。十一載,益襲封許王。十四載,解娶楊銛女,乃襲許王。璆初為嗣澤王,降為郢國公、宗王卿同正員,特封褒信郡王。進《龍池皇德頌》,遷宗正卿、光祿卿、殿中監。天寶初,重拜宗五卿,加金紫光祿大夫。璆友弟聰敏,聞善若驚,宗子中有一善,無不薦拔,故宗枝居省闥者,多是璆之所舉。九載卒,贈江陵大都督。



 孝敬皇帝弘,高宗第五子也。永徽四年,封代王。顯慶元年,立為皇太子,大赦改元。弘嘗受《春秋左氏傳》於率更
 令郭瑜,至楚子商臣之事,廢卷而嘆曰:「此事臣子所不忍聞,經籍聖人垂訓,何故書此?」瑜對曰:「孔子修《春秋》,義薦褒貶,故善惡必書。褒善以示代,貶惡以誡後,故使商臣之惡,顯於千載。」太子曰:「非唯口不可道,故亦耳不忍聞,請改讀餘書。」瑜再拜賀曰:「里名勝母,曾子不入;邑號朝歌,墨子回車。殿下誠孝冥資,睿情天發,兇悖之跡,黜於視聽。循奉德音,實深慶躍。臣聞安上理人,莫善於禮,非禮無以事天地之神,非禮無以辨君臣之位,故先王
 重焉。孔子曰:『不學《禮》,無以立。』請停《春秋》而讀《禮記》。」太子從之。龍朔元年,命中書令、太子賓客許敬宗,侍中兼太子右庶子許圉師,中書侍郎上官儀,太子中舍人楊思儉等於文思殿博採古今文集,摘其英詞麗句,以類相從,勒成五百卷,名曰《瑤山玉彩》,表上之。制賜物三萬段,敬宗已下加級、賜帛有差。總章元年二月,親釋菜司成館,因請贈顏回太子少師,曾參太子少保,高宗並從之。



 時有敕,征邊遼軍人逃亡限內不首及更有逃亡者,身
 並處斬,家口沒官。太子上表諫曰:「竊聞所司以背軍之人,身久不出,家口皆擬沒官。亦有限外出首,未經斷罪,諸州囚禁,人數至多。或臨時遇病,不及軍伍,緣茲怖懼,遂即逃亡;或因樵採,被賊抄掠;或渡海來去,漂沒滄波;或深入賊庭,有被傷殺。軍法嚴重,皆須相傔。若不及傔,及不因戰亡,即同隊之人,兼合有罪。遂有無故死失,多注為逃。軍旅之中,不暇勘當,直據隊司通狀,將作真逃,家口令總沒官,論情實可哀愍。《書》曰:『與其殺不辜,寧失
 不經。』伏願逃亡之家,免其配沒。」制從之。



 咸亨二年,駕幸東都,留太子於京師監國。時屬大旱,關中饑乏,令取廓下兵士糧視之,見有食榆皮蓬實者,乃令家令等各給米使足。是時戴至德、張文瓘兼左庶子,與右庶子蕭德昭同為輔弼,太子多疾病,庶政皆決於至德等。時義陽、宣城二公主以母得罪,幽於掖庭,太子見之驚惻,遽奏請令出降。又請以同州沙苑地分借貧人。詔並許之。又召詣東都,納右衛將軍裴居道女為妃。所司奏以白雁
 為贄,適會苑中獲白雁,高宗喜曰:「漢獲硃雁,遂為樂府;今獲白雁,得為婚贄。彼禮但成謠頌,此禮便首人倫,異代相望,我無慚德也。」裴氏甚有婦禮,高宗嘗謂侍臣曰:「東宮內政,吾無憂矣。」



 上元二年,太子從幸合璧宮,尋薨,年二十四。制曰:「皇太子弘,生知誕質,惟幾毓性。直城趨賀,肅敬著於三朝;中寢問安,仁孝聞於四海。自琰圭在手,沉瘵嬰身,顧惟耀掌之珍,特切鐘心之念,庶其痊復,以禪鴻名。及腠理微和,將遜於位,而弘天資仁厚,孝心
 純確,既承朕命,掩欻不言,因茲感結,舊疾增甚。億兆攸系,方崇下武之基;五福無徵,俄遷上賓之駕。昔周文至愛,遂延慶於九齡;朕之不慈,遽永訣於千古。天性之重,追懷哽咽,宜申往命,加以尊名。夫謚者,行之跡也;號者,事之表也。慈惠愛親曰『孝』,死不忘君曰『敬』,謚為孝敬皇帝。」其年,葬於緱氏縣景山之恭陵。制度一準天子之禮,百官從權制三十六日降服。高宗親為制《睿德紀》,並自書之於石,樹於陵側。初,將營築恭陵,功費鉅億,萬姓厭
 役,呼嗟滿道,遂亂投磚瓦而散。



 太子無子,長壽中,制令楚王諱繼其後。中宗踐祚,制祔於太廟,號曰義宗,又追贈妃裴氏為哀皇后。景雲元年,中書令姚元之、吏部尚書宋璟奏言:「準禮,大行皇帝山陵事終,即合祔廟。其太廟第七室,先祔皇昆義宗孝敬皇帝、哀皇后裴氏神主。伏以義宗未登大位,崩後追尊,至神龍之初,乃特令升祔。《春秋》之義,國君即位未逾年者,不合列昭穆。又古者祖宗各別立廟,孝敬皇帝恭陵既在洛州,望於東都別
 立義宗之廟,遷祔孝敬皇帝、哀皇后神主,命有司以時享祭,則不違先旨,又協古訓,人神允穆,進退得宜。在此神主,望入夾室安置,伏願陛下以禮斷恩。」詔從之。開元六年,有司上言:「孝敬皇帝今別廟將建,亨祔有期,準禮,不合更以義宗為廟號,請以本謚孝敬為廟稱。」於是始停義宗之號。



 裴居道,絳州聞喜人,隋兵部侍郎鏡民孫也。父熙載,貞觀中為尚書左丞。居道以女為太子妃,則天時,歷位納
 言、內史、太子少保,封翼國公。載初元年春,為酷吏所陷,下獄死。



 章懷太子賢,字明允,高宗第六子也。永徽六年,封潞王。顯慶元年,遷授岐州刺史。其年,加雍州牧、幽州都督。時始出閣,容止端雅,深為高宗所嗟賞。高宗嘗謂司空李勛曰:「此兒已讀得《尚書》、《禮記》、《論語》,誦古詩賦復十餘篇,暫經領覽,遂即不忘。我曾遣讀《論語》,至『賢賢易色』,遂再三覆誦。我問何為如此,乃言性愛此言。方知夙成聰敏,
 出自天性。」龍朔元年,徙封沛王,加揚州都督、兼左武衛大將軍,雍州牧如故。二年,加揚州大都督。麟德二年,加右衛大將軍。咸亨三年,改名德,徙封雍王,授涼州大都督,雍州牧、右衛大將軍如故,食實封一千戶。上元元年,又依舊名賢。



 上元二上,孝敬皇帝薨。其年六月,立為皇太子,大赦天下,尋令監國。賢處事明審,為時論所稱。儀鳳元年,手敕褒之曰:「皇太子賢自頃監國,留心政要。撫字之道,既盡於哀矜;刑綱所施,務存於審察。加以聽覽
 餘暇,專精墳典。往聖遺編,咸窺壺奧;先王策府,備討菁華。好善載彰,作貞斯在,家國之寄,深副所懷。可賜物五百段。」賢又招集當時學者太子左庶子張大安、洗馬劉訥言、洛州司戶格希玄、學士許叔牙成玄一史藏諸周寶寧等,注範曄《後漢書》,表上之,賜物三萬段,仍以其書付秘閣。



 時正議大夫明崇儼以符劾之術為則天所任使,密稱「英王狀類太宗」。又宮人潛議云「賢是后姊韓國夫人所生」,賢亦自疑懼。則天又嘗為賢撰《少陽政範》及《
 孝子傳》以賜之,仍數作書以責讓賢,賢逾不自安。調露二年,崇儼為盜所殺,則天疑賢所為。俄使人發其陰謀事,詔令中書侍郎薛元超、黃門侍郎裴炎、御史大夫高智周與法官推鞫之,於東宮馬坊搜得皁甲數百領,乃廢賢為庶人,幽於別所。永淳二年,遷於巴州。文明元年,則天臨朝,令左金吾將軍丘神勣往巴州檢校賢宅,以備外虞。神勣遂閉於別室,逼令自殺,年三十二。則天舉哀於顯福門,貶神勣為疊州刺史,追封賢為雍王。神龍
 初,追贈司徒,仍遣使迎其喪柩,陪葬於乾陵。睿宗踐祚,又追贈皇太子,謚曰章懷。有三子:光順、守禮、守義。



 光順,大授中封安樂郡王,尋被誅。



 守義,文明年封犍為郡王。垂拱四年,徙封永安郡王,病卒。



 守禮本名光仁,垂拱初改名守禮,授太子洗馬,封嗣雍王。時中宗遷於房陵,睿宗雖居帝位,絕人朝謁,諸武贊成革命之計,深嫉宗枝。守禮以父得罪,與睿宗諸子同處於宮中,凡十餘年不出庭院。至聖歷元年,睿宗自皇嗣封為相王,許出外邸。
 睿宗諸子五子皆封郡王,與守禮始居於外。神龍元年,中宗纂位,授守禮光祿卿同正員。神龍中,遺詔進封邠王,賜實封五百戶。景雲二年,帶光祿卿,兼幽州刺史,轉左金吾衛大將軍,遙領單于大都護。先天二年,遷司空。開元初,歷虢、隴、襄、晉、滑六州刺史,非奏事及大事,並上佐知州。時寧、申、岐、薛、邠同為刺史,皆擇首僚以持綱紀。源乾曜、袁嘉祚、潘好禮皆為邠府長史兼州佐,守禮唯弋獵、伎樂、飲謔而已。九年已後,諸王並徵還京師。



 守禮
 以外枝為王,才識猥下,尤不逮岐、薛。多寵嬖,不修風教,男女六十餘人,男無中才,女負貞稱,守禮居之自若,高歌擊鼓。常帶數千貫錢債,或有諫之者曰:「王年漸高,家累甚眾,須有愛惜。」守禮曰:「豈有天子兄沒人葬?」諸王因內宴言之,以為歡笑。時積陰累日,守禮白於諸王曰:「欲晴。」果晴。愆陽涉旬,守禮曰:「即雨。」果連澍。岐王等奏之,云:「邠哥有術。」守禮曰:「臣無術也。則天時以章懷遷謫,臣幽閉宮中十餘年,每歲被敕杖數頓,見瘢痕甚厚。欲雨,臣
 脊上即沉悶,欲晴,即輕健,臣以此知之,非有術也。」涕泗沾襟,玄宗亦憫然。二十九年薨,年七十餘,贈太尉。



 子承宏,開元初封廣武郡王,歷秘書員外監,又為宗正卿同正員。廣德元年,吐蕃凌犯上都,乘輿幸陜。蕃、渾之眾入城,吐蕃宰相馬重英立承宏為帝,以於可封、霍環等為宰相,補署百餘人。旬餘日,賊退,郭子儀率眾入城,送承宏於行在,上不之責,止於虢州。尋死。承寧,天寶初,授率更令同正員,嗣邠王。承寀,至德二載,燉封為煌郡王,加
 開府儀同三司。與僕固懷恩使回紇和親,因納其女為妃,冊為毗伽公主。回紇著勛,承寀甚遇恩寵。乾元元年六月卒,贈司空。



 唐法,嗣郡王但加四品階,親王子例著緋。開元中,張九齡為中書令,奏請寧、薛王男並賜紫,邠王三男衣紫,餘二十人衣緋,官亦不越六局郎,王府掾屬仍員外置。十五載,扈從至巴蜀,依例著紫。



 中宗四男:章庶人生懿德太子重潤,後宮生庶人重福、節愍太子重俊、殤帝重茂。



 懿德太子重潤,中宗長子也。本名重照,以避則天諱,故改焉。開耀二年,中宗為皇太子,生重潤於東宮內殿,高宗甚悅。及月滿,大赦天下,改元為永淳。是歲,立為皇太孫,開府置官屬。及中宗遷於房州,其府坐廢。聖歷初,中宗為皇太子,封為邵王。大足元年,為人所構,與其妹永泰郡主、婿魏王武延基等竊議張易之兄弟何得恣入宮中,則天令杖殺,時年十九。重潤風神俊朗,早以孝友知名,既死非其罪,大為當時所悼惜。中宗即位,追贈皇
 太子,謚曰懿德,陪葬乾陵。仍為聘國子監丞裴粹亡女為冥婚,與之合葬。又贈永泰郡主為公主,令備禮改葬,仍號其墓為陵焉。



 庶人重福,中宗第二子也。初封唐昌王,聖歷三年,徙封平恩王。長安四年,進封譙王,歷遷國子祭酒、左散騎常侍。神龍初,為韋庶人所譖,雲與張易之兄弟潛構成重潤之罪,由是左授濮州員外刺史,轉均州,司防守,不許視事。景龍三年,中宗親祀南郊,大赦天下,流人並放還。
 重福不得歸京師,尤深鬱怏,上表自陳曰:「臣聞功同賞異,則勞臣疑;罪均刑殊,則百姓惑。伏惟陛下德侔造化,明齊日月,恩及飛鳥,惠加走獸。近者焚柴展禮,郊祀上玄,萬物沾愷悌之仁,六合承曠蕩之澤。事無輕重,咸赦除之。蒼生並得赦除,赤子偏加擯棄,皇天平分之道;固此乎?天下之人,聞者為臣流涕。況陛下慈念,豈不愍臣恓惶?伏望舍臣罪愆,許臣朝謁。儻得一仰雲陛,再睹陛聖顏,雖沒九泉,實為萬足。重投荒徼,亦所甘心。」表奏不
 報。



 及韋庶人臨朝,遽令左屯衛大將軍趙承恩以兵五百人就均州守衛重福。俄而韋氏伏誅,睿宗即位,又轉集州刺史。未及行,洛陽人張靈均進計於重福曰:「大王地居嫡長,自合繼為天子。相王雖有討平韋氏功,安可越次而居大位!昔漢誅諸呂,猶迎代王,今東都百官士庶,皆願王來。王若潛行直詣洛陽,亦是從天上落,遣人襲殺留守,即擁兵西據陜州,東下河北,此天下可圖也。」初,景龍三年,鄭愔自吏部侍郎出為江州司馬,便道詣
 重福陰相結托。至是又與靈均通傳動靜,亦密遣使勸重福構逆,預推尊重福為天子,溫王重茂為皇太弟,自署為左丞相。重福乃遣家臣王道先赴東都,潛募勇敢之士,重福遽自均州詐乘驛與靈均繼進。



 王道始至東都,俄有洩其謀者,洛州司馬崔日知捕獲其黨數十人。經聞重福至,王道等率眾隨重福徑取左右屯營兵作亂,將至天津橋,願從者已數百人,皆執持器仗,助其威勢。侍御史李邕先詣左掖門,令閉關拒守。又至右屯營
 號令云:「重福雖先帝之子,已得罪於先帝,今者無故入城,必是作亂。君等皆委質聖朝,宜盡誠節,立功立事,以取富貴。」有頃,重福果來奪右屯營,堅壁不動,營中矢射如雨。便趣大臣掖門,擬取留守,遇門閉,遂縱火以燒城門。左屯營兵又來逼之,重福度數窮,出自上東門而遁,匿於山谷間。明日,東都留守裴談等大出兵搜索,重福窘迫,自投漕河而死,磔尸三日,時年三十一。詔曰:「集州刺史譙王重福,幼則兇頑,長而險詖。幸托體於先聖,嘗通
 交於巨逆。子而不子,自絕於天。有國有家,莫容於代。往者頗不含忍,長令幽縶。自大行晏駕,韋氏臨朝,將肆屠滅,尤加防衛。洎天有成命,集於朕躬,永懷猶子之情,庶協先親之義。所以開置僚屬,任隆刺舉,冀其悛改,以怙恩榮。而詿誤有徒,狂狡未息。便即私出均州,詐乘驛騎,至於都下,遂逞其謀。先犯屯兵,次燒左掖,計窮力屈,投河而斃。雖人所共棄,邦有常刑,我非不慈,爾自招咎。且聞其故,有惻於懷。昔劉長既歿,楚英遂殞,以禮收葬,抑
 惟舊章,屈法申恩,宜仍舊寵。可以三品禮葬。」



 節愍太子重俊,中宗第三子也。聖歷元年,封義興郡王。長安中,累授衛尉員外少卿。神龍初,封衛王,拜洛州牧,賜實封千戶,尋遷左衛大將軍,兼遙授揚州大都督。二年秋,立為皇太子。重俊性雖明果,未有賢師傅,舉事多不法。俄以秘書監楊璬、太常卿武崇訓並為太子賓客。璬等皆主婿年少,唯以蹴鞠猥戲取狎於重俊,竟無調護之意。左庶子姚珽數上疏諫諍,右庶子平貞慎又獻《
 孝經議》、《養德傳》以諷,重俊皆優納焉。



 時武三思得幸中宮,深忌重俊。三思子崇訓尚安樂公主,常教公主凌忽重俊,以其非韋氏所生,常呼之為奴。或勸公主請廢重俊為王,自立為皇太女,重俊不勝忿恨。三年七月,率左羽林大將軍李多祚、右羽林將軍李思沖、李承況、獨孤禕之、沙吒忠義等,矯制發左右羽林兵及千騎三百餘人,殺三思及崇訓於其第,並殺黨與十餘人。又令左金吾大將軍成王千里分兵守宮城諸門,自率兵趨肅章
 門,斬關而入,求韋庶人及安樂公主所在。又以昭容上官氏素與三思奸通,扣閤索之。韋庶人及公主遽擁帝馳赴玄武門樓,召左羽林將軍劉仁景等,令率留軍飛騎及百餘人於樓下列守。俄而多祚等兵至,欲突玄武門樓,宿衛者拒之;不得進。帝據檻呼多祚等所將千騎,謂曰:「汝並是我爪牙,何故作逆?若能歸順,斬多祚等,與汝富貴。」於是千騎王歡喜等倒戈,斬多祚及李承況、獨孤禕之、沙吒忠義等於樓下,餘黨遂潰散。重俊既敗,率
 其屬百餘騎趨肅章門,奔終南山。帝令長上果毅趙思慎率輕騎追之。重俊至雩縣西十餘里,騎不能屬,唯從奴數人。會日暮憩林下,為左右所殺。制今梟首於朝,又獻之於太廟,並以祭三思、崇訓尸柩。



 睿宗即位,下制曰:「朕聞曾氏之孝也,慈親惑於疑聽;趙虜之族也,明主哀而望思。歷考前聞,率由舊典。重俊,大行之子,元良守器。往罹構間,困於讒嫉。莫顧鈇鉞,輕盜甲兵,有此誅夷,無不悲惋。今四兇咸服,十起何追,方申赤軍之冤,以紓黃
 泉之痛。可贈皇太子。」謚曰節愍,陪葬定陵。一子宗暉,開元初封湖陽郡王。初,重俊被害,宮府僚吏莫敢近者,永和丞甯嘉勖解衣裹重俊首號哭,時人義之。宗楚客聞而大怒,收付制獄,貶為平興丞,尋卒。睿宗踐祚,下制曰:「寧嘉勖能重名節,事高欒、向,幽塗已往,生氣凜然。靜言忠義,追存褒寵。可贈永和縣令。」宗暉,天寶中為衛尉員外卿。十一載,王鉷反,宗暉以賣宅與鉷,貶涪川郡長史,量移盧陽長史。至德元年,追赴行在所,授特進、鴻臚卿。
 宗暉無他才,以外族之親,受恩顧轉隆。太常員外卿卒。



 殤皇帝重茂,中宗第四子也。聖歷三年,封北海王。神龍初,進封溫王,授右衛大將軍,兼遙領並州大都督,未出閤。景龍四年,中宗崩,韋庶人立重茂為帝,而自臨朝稱制。及韋氏敗,重茂遂遜位,讓叔父相王,退居別所。景雲二年,改封襄王,遷於集州,令中郎將率兵五百人守衛。開元二年,轉房州刺史。尋薨,時年十七,謚曰殤皇帝,葬於武功西原。



 史臣曰:前代以嬖婦孽子破國亡家者多矣,然未如大帝、孝和之甚也。高宗八子,二王早世,為武后所斃者四人,章懷以母子之愛,穎悟之賢,猶不免於虎口。況燕、澤、素節異腹之胤乎!覆載胡心,產茲鴆毒,悲夫!孝和母囂,婦傲女暴,如置身群魅之中,安有保其終吉哉!天將滌蕩昏氛,非重茂所能枝也。



 贊曰:父子天性,嬖能害正。宜臼、申生,翻為不令。唐年鈞德,章懷最仁。兇母畏明,取樂於身。



\end{pinyinscope}