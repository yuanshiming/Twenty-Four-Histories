\article{卷一百五十}

\begin{pinyinscope}

 ○薛播鮑防李自良李說嚴綬蕭昕杜亞王緯李若初于頎盧徵楊憑鄭元杜兼裴玢薛伾



 薛播,河中寶鼎人,中書舍人文思曾孫也。父元暉,什邡令,以播贈工部郎中。播,天寶中舉進士,補校書郎,累授萬年縣丞、武功令、殿中侍御史、刑部員外郎、萬年令。播溫敏,善與人交,李棲筠、常袞、崔祐甫皆引擢之。及祐甫輔政,用為中書舍人。出汝州刺史,以公事貶泉州刺史。尋除晉州刺史,河南尹,遷尚書左丞,轉禮部侍郎。遇疾,貞元三年卒,贈禮部尚書。



 初,播伯父元曖終於隰城丞,其妻濟南林氏,丹陽太守洋之妹,有母儀令德,博涉《五
 經》,善屬文,所為篇章,時人多諷詠之。元曖卒後,其子彥輔、彥國、彥偉、彥云及播兄據、手並早孤幼,悉為林氏所訓導,以至成立,咸致文學之名。開元、天寶中二十年間,彥輔、據等七人並舉進士,連中科名,衣冠榮之。



 鮑防,襄州人。幼孤貧,篤志好學,善屬文。天寶末舉進士,為漸東觀察使薛兼訓從事,累至殿中侍御史。入為職方員外郎,改太原少尹,正拜節度使。入為御史大夫,歷福建、江西觀察使,徵拜左散騎常侍。扈從奉天,除禮部
 侍郎,尋遷工部尚書致仕。



 防歷洪、福、京兆,皆有政聲,唯總戎非所宜,而謬執兵柄。以太原革車胡騎雄雜,而回鶻深入寇,防出拒戰,為虜所敗。為禮部侍郎時,嘗遇知雜侍御史竇參於通衢,導騎不時引避,僕人為參所鞭;及參秉政,遽令致仕。防謂親友曰:「吾與蕭昕之子齒,而與昕同日懸車,非朽邁之致,以餘忿見廢。」防文學舊人,歷職中外,不因罪戾,而為俗吏所擯,竟以憤終。眾頗憫防而咎參,故參之敗不旋踵,非不幸也。



 李自良,兗州泗水人。初,祿山之亂,自良從兗鄆節度使能元皓,以戰功累授右衛率。後從袁傪討袁晁陳莊賊,積功至試殿中監,隸浙江東道節度使薛兼訓。兼訓移鎮太原,自良從行,授河東軍節度押衙。兼訓卒,鮑防代,又事防為牙將。會回鶻入寇,防令大將焦伯瑜、杜榮國將兵擊之。自良謂防曰:「回鶻遠來求戰,未可與爭鋒。但於歸路築二壘,以兵守之,堅壁不動,虜求戰不得,師老自旋。俟其返昪,即乘之,縱不甚捷,虜必狼狽矣。二壘
 厄其歸路,策之上也。」防不從,促伯瑜等逆戰,遇虜於百井。伯瑜等大敗而還,由是稍知名。馬燧代防為帥,署奏自良代州刺史、兼御史大夫,仍為軍候。自良勤恪有謀,燧深委信之。建中年,田悅叛,燧與抱真東討;自良常為河東大將,摧鋒陷陣,破田悅。及討李懷光於河中,自良專河東軍都將,前後戰績居多。燧之立功名,由自良協輔之力也。



 貞元三年,從燧入朝,罷燧兵權,德宗欲以自良代燧。自良懇辭事燧久,不欲代為軍帥,物議多之,乃授
 右龍武大將軍。德宗以河東密邇胡戎,難於擇帥,翌日,自良謝,上謂之曰:「卿於馬燧存軍中事分,誠為得禮,然北門之寄,無易於卿。」即日拜檢校工部尚書、兼御史大夫、太原尹、北都留守、河東節度支度營田觀察使。在鎮九年,以簡儉守職,軍民胥悅。雖出身戎伍,動必循法,略不以暴戾加人。十一年五月,卒於軍,年六十三,上甚嗟惜之,廢朝一日,贈左僕射,賻布帛米粟有差。



 李說,淮安王神通之裔也。父遇,天寶中為御史中丞。說
 以門廕歷仕,累佐使幕。馬燧為河陽三城、太原節度,皆闢為從事。累轉御史郎官,御史中丞,太原少尹,出為汾州刺史。節度使李自良復奏為太原少尹、檢校庶子、兼中丞。



 貞元十一年五月,自良病,凡六日而卒。匿喪,陽言病甚,數日發喪。先是,都虞候張瑤久在軍,素得士心,嘗請假遷葬,自良未許。至是,說與監軍王定遠謀,乃給瑤假,以大將毛朝陽代瑤,然後遣使告自良病。中使第五國珍自雲、朔使還,過太原,聞自良病,中使遲留信宿。自
 良卒,國珍急馳至京,先說使至。乃下制以通王領河東節度大使,以說為行軍司馬,充節度留後、北都副留守;仍令國珍齎說官告及軍府將吏部內刺史等敕書三十餘通往太原宣賜,軍中始定。



 定遠恃立說之功,頗恣縱橫,軍政皆自專決,仍請賜印。監軍有印,自定遠始也。定遠既得印,益暴,將吏輒自補授,說浸不歡,遂成嫌隙。是歲七月,定遠署虞候田宏為列將,以代彭令茵。令茵不伏,揚言曰:「超補列將,非功不可,宏有何功,敢代予任!」
 定遠聞而含怒,召令茵斬之,埋於馬糞之中。家人請尸,不與,三軍皆怨。說具以事聞。德宗以定遠有奉天扈從之功,恕死停任。制未至,定遠怒說奏聞,趨府謀殺說,升堂未坐,抽刀刺說,說走而獲免。定遠馳至府門,召集將吏,於箱中陳敕牒官告二十餘軸,示諸將曰:「有敕,令李景略知留後,遣說赴京,公等皆有恩命。」指箱中示之,諸將方拜抃,大將馬良輔呼而麾眾曰:「箱中皆監軍舊官告,非恩命也,不可受,但備急變爾。」定遠知事敗,走登乾
 陽樓,召其部下將卒,多不之應。比夜,定遠墜城下槎枿,傷而不死。尋有詔削奪,長流崖州。大將高迪等同其謀,說皆斬之。尋正拜河東節度使,檢校禮部尚書。



 說在鎮六年,初勤心吏職,後遇疾,言語行步蹇澀,不能錄軍府之政,悉監軍主之。又為孔目吏宋季等欺誑,軍政事多隳紊,如此累年。十六年十月卒,年六十一,廢朝一日,贈左僕射。



 是月,制以河東節度行軍司馬鄭儋檢校工部尚書,兼太原尹、御史大夫、河東節度度支營田觀察等
 使、北都留守,在任不期年而卒。



 嚴綬,蜀人。曾祖方約,利州司功。祖挹之,符離尉。父丹,殿中侍御史。綬,大歷中登進士第,累佐使府。貞元中,由侍御史充宣翕團練副使,深為其使劉贊委遇,政事多所咨訪。十二年,贊卒,綬掌宣歙留務,傾府藏以進獻,由是有恩,召為尚書刑部員外郎。天下賓佐進獻,自綬始也。



 未幾,河東節度使李說嬰疾,事多曠弛,行軍司馬鄭儋代綜軍政;既而說卒,因授儋河東節度使。是時姑息四
 方諸侯,未嘗特命帥守,物故即用行軍司馬為帥,冀軍情厭伏。儋既為帥,德宗選朝士可以代儋為行軍司馬者。因綬前日進獻,上頗記之,故命檢校司封郎中,充河東行軍司馬。不周歲,儋卒,遷綬銀青光祿大夫、檢校工部尚書,兼太原尹、御中大夫、北都留守,充河東節度支度營田觀察處置等使。元和元年,楊惠琳叛於夏州,劉闢叛於成都,綬表請出師討伐。綬悉選精甲,付牙將李光顏兄弟,光顏累立戰功。蜀、夏平,加綬檢校尚書左僕
 射。尋拜司空,進階金紫,封扶風郡公。綬在鎮九年,以寬惠為政,士馬蕃息,境內稱治。



 四年,入拜尚書右僕射。綬雖名家子,為吏有方略,然銳於勢利,不存名節,人士以此薄之。嘗預百僚廊下食,上令中使馬江朝賜櫻桃。綬居兩班之首,在方鎮時識江朝,敘語次,不覺屈膝而拜,御史大夫高郢亦從而拜。是日,為御史所劾,綬待罪於朝,命釋之。翌日,責江朝,降官一等。尋出鎮荊南,進封鄭國公。有漵州蠻首張伯靖者,殺長吏,據辰、錦等州,連九
 洞以自固,詔綬出兵討之。綬遣部將李忠烈齎書曉諭,盡招降之。



 九年,吳元濟叛,朝議加兵,以綬有弘恕之稱,可委以戎柄,乃授山南東道節度使,尋加淮西招撫使。綬自帥師壓賊境,無威略以制寇;到軍日,遽發公藏以賞士卒,累年蓄積,一旦而盡。又厚賂中貴人以招聲援。師徒萬餘,閉壁而已,經年無尺寸功。裴度見上,屢言綬非將帥之才,不可責以戎事,乃拜太子少保代歸。尋檢校司空。久之,進位太傅,食封至三千戶。長慶二年五月
 卒,年七十七,詔贈太保。



 綬材器不逾常品,事兄嫂過謹,為時所稱。常以寬柔自持,位躋上公,年至大耋,前後統臨三鎮,皆號雄籓,所親士親睹為將相者凡九人,其貴壽如此。



 蕭昕,河南人。少補崇文進士。開元十九年,首舉博學宏辭,授陽武縣主簿。天寶初,復舉宏辭,授壽安尉,再遷左拾遺。昕嘗與布衣張鎬友善,館而禮之,表薦之曰:「如鎬者,用之則為王者師,不用則幽谷一叟爾。」玄宗擢鎬拾
 遺,不數年,出入將相。及安祿山反,昕舉贊善大夫來瑱堪任將帥;思明之亂,瑱功居多。累遷憲部員外郎,為副元帥哥舒翰掌書記。潼關敗,間道入蜀,遷司門郎中。尋兼安陸長史,為河南等道都統判官。遷中書舍人,兼揚府司馬,佐軍仍舊,入拜本官,累遷秘書監。代宗幸陜,昕出武關詣行在,轉國子祭酒。大歷初,持節吊回鶻。時回鶻恃功,廷詰昕曰:「祿山、思明之亂,非我無以平定,唐國奈何市馬而失信,不時歸價?」眾皆失色。昕答曰:「國家自
 平寇難,賞功無絲毫之遺,況鄰國乎!且僕固懷恩,我之叛臣,乃者爾助為亂,聯西戎而犯郊畿;及吐蕃敗走,回紇悔懼,啟顙乞和。非大唐存念舊功,則當匹馬不得出塞矣!是回紇自絕,非我失信。」回紇慚退,加禮以歸,為常侍。十二年。硃泚之亂,徒步出城,泚急求之,亡竄山谷間。至奉天,遷太子少傅。貞元初,兼禮部尚書,尋復知貢舉。五年,致仕。七年,卒於家,年九十,廢朝,謚曰懿。



 杜亞,字次公,自云京兆人也。少頗涉學,善言物理及歷
 代成敗之事。至德初,於靈武獻封章,言政事,授校書郎。其年,杜鴻漸為河西節度,闢為從事,累授評事、御史。後入朝,歷工、戶、兵、吏四員外郎。永泰末,劍南叛亂,鴻漸以宰相出領山、劍副元帥,以亞及楊炎並為判官。使還,授吏部郎中、諫議大夫。炎為禮部郎中、知制誥、中書舍人。亞自以才用合當柄任,雖為諫議大夫,而心不悅。李棲筠承恩,眾望必為宰相,亞厚結之。元載得罪,亞與劉晏、李涵等七人同鞫訊之。載死之翌日,亞遷給事中、河北
 宣慰使。宰相常袞亦不悅亞,歲餘,出為洪州刺史、兼御史中丞、江西都團練觀察使。



 德宗初嗣位,勵精求賢,令中使召亞。亞自揣必以宰輔見征,乃促程而進,累路與人言議,語及行宰相事方面,或以公事諮祈,亞皆納之。既至,帝微知之,不悅;又奏對辭旨疏闊,出為陜州觀察使兼轉運使。尋遷河中、晉、絳等州防禦觀察使。楊炎作相,劉晏得罪,亞坐貶睦州刺史。



 興元初,召拜刑部侍郎。出為揚州長史、兼御史大夫、淮南節度觀察使。時承陳
 少游征稅煩重,奢侈僭濫之後,又新遭王紹亂兵剽掠;淮南之人,望亞之至,革刬舊弊,冀以康寧。亞自以材當公輔之選,而聯出外職,志頗不適,政事多委參佐,招引賓客,談論而已。揚州官河填淤,漕輓堙塞,又僑寄衣冠及工商等多侵衢造宅,行旅擁弊。亞乃開拓疏啟,公私悅賴,而盛為奢侈。江南風俗,春中有競渡之戲,方舟並進,以急趨疾進者為勝。亞乃令以漆塗船底,貴其速進;又為綺羅之服,塗之以油,令舟子衣之,入水而不濡。亞
 本書生,奢縱如此,朝廷亟聞之。



 貞元五年,以戶部侍郎竇覦為淮南節度代亞。亞猶以舊望,竇覦甚畏之。改檢校吏部尚書,判東都尚書省事,充東都留守、都防禦使。既病風,尚建利以固寵,奏請開苑內地為營田,以資軍糧;減度支每年所給,從之。亞不躬親部署,但委判官張薦、楊晪。初,奏請取荒地營田,其苑內地堪耕食者,先為留司中官及軍人等開墾已盡。晉計急,乃取軍中雜錢舉息與畿內百姓,每至田收之際,多令軍人車牛散入
 村鄉,收斂百姓所得菽粟將還軍。民家略盡,無可輸稅,人多艱食,由是大致流散。乃厚賂中官,令奏河南尹無政,亞自此亦規求兼領河南尹,事不果。帝漸知虛誕,乃以禮部尚書董晉代為東都留守,召亞還京師。既風疾漸深,又患腳膝,不任朝謁。貞元十四年卒於家,年七十四,贈太子少傅。



 王緯,字文卿,太原人也。祖景,司門員外、萊州刺史。父之咸,長安尉;與昆弟之賁、之渙皆善屬文。之咸以緯貴,故
 累贈刺史。緯舉明經,又書判入等,歷長安尉,出佐使府,授御史郎官,入朝為金部員外郎、劍南租庸使、檢校司封郎中、彭州刺史、檢校庶子、兼御史中丞、西川節度營田副使。初,大歷中,路嗣恭為江西觀察使,陷害判官李泌,將誅之;緯亦為路嗣恭判官,說諭救解,獲免。貞元三年,泌為相,擢授緯給事中。未數日,又擢為潤州刺史、兼御史中丞、浙江西道都團練觀察使。十年,加御史大夫,兼諸道鹽鐵轉運使。三歲,加檢校工部尚書。緯性勤儉,
 歷官清潔,而傷於苛碎,多用削刻之吏,督察巡屬,人不聊生。貞元十四年卒,年七十一,廢朝一日,贈太子少保。



 李若初,趙郡人。貞觀中並州長史、工部侍郎弘節之曾孫也。祖道謙,太府卿。若初少孤貧,初為轉運使劉晏下微冗散職;晏判官包佶重其勤幹,以女妻之。歷陳州太康令。刺史李芃初蒞官,若初獻計,請收斂羨餘錢物,交結權貴,芃厚遇之。累歲,芃遷河陽三城使,奏若初為從事,軍中之事,多以委之。累授檢校郎中、兼中丞、懷州刺
 史。轉虢州刺史,坐公事為觀察使劾奏,免歸。久之,出為衢州刺史,遷福州刺史、兼御史中丞、福建都團練使。尋遷越州刺史、浙江東道都團練觀察使。十四年秋,代王緯為潤州刺史、兼御史大夫、浙江都團練觀察、諸道鹽鐵轉運使。善於吏道,性嚴強,力束斂下,吏人甚畏服。方整理鹽法,頗有次敘。貞元十五年,遇疾卒,廢朝一日,贈禮部尚書。



 于頎,字休明,河南人也。父庭謂,濟王府倉曹,累贈尚書
 左僕射。頎少以吏事聞,累授京兆府士曹,為尹史翽所賞重。翽出鎮襄、漢,奏為御史,充判官。翽為亂兵所殺,頎挺出收葬遺骸,時人義之。度支使第五琦署為河東租庸使,累授鳳翔少尹、度支郎中、兼御史中丞、轉運租庸糧料鹽鐵等使。頎因奏移轉運汴州院於河陰,以汴州累遇兵亂,散失錢帛故也。元載為諸道營田使,又署為郎官,令於東都、汝州開置屯田。歷戶部侍郎、秘書少監、京兆尹、太府卿,代杜濟為京兆尹。



 及為大官,好任機數,
 專候權要,朝列中無勢利者,視之蔑如也。曲事元載,親暱之。而為政苛細無大體;丁所生母憂罷。及載得罪後,出為鄭州刺史,遷河南尹,以無政績代還。時征汾州刺史劉暹。暹剛腸嫉惡,歷典數州,皆為廉使畏懼。宰相盧杞恐暹為御史大夫,虧沮己之所見,遽稱薦頎為御史大夫,以其柔佞易制也。從幸奉天,改左散騎常侍,歷左千牛上將軍,徙大理卿、太子少保、工部尚書。因入朝僕地,為金吾仗衛掖起,改太子少師致仕。貞元十五年卒,
 時年七十四。



 盧徵,範陽人也,家於鄭之中牟。少涉獵書記。永泰中,江淮轉運使劉晏闢為從事,委以腹心之任,累授殿中侍御史。晏得罪,貶珍州司戶。元琇亦晏之門人,興元中,為戶部侍郎、判度支,薦徵為京兆司錄、度支員外。琇得罪,坐貶為信州長史。遷信州刺史。入為右司郎中,驟遷給事中。戶部侍郎竇參深遇之,方倚以自代。貞元八年春,同州刺史闕,參請以尚書左丞趙憬補之,特詔用徵,以
 間參腹心也。數歲,轉華州刺史。徵冀復入用,深結托中貴,厚遺之。故事,同、華以近地人貧,每正至端午降誕,所獻甚薄;徵遂竭其財賦,每有所進獻,輒加常數,人不堪命。疾病臥理者數年,貞元十六年卒,時年六十四。



 楊憑,字虛受,弘農人。舉進士,累佐使府。徵為監察御史,不樂檢束,遂求免。累遷起居舍人、左司員外郎、禮部兵部郎中、太常少卿、湖南江西觀察使,入為左散騎常侍、刑部侍郎、京兆尹。憑工文辭,少負氣節;與母弟凝、凌相
 友愛,皆有時名。重交游,尚然諾,與穆質、許孟容、李鄘、王仲舒為友,故時人稱楊、穆、許、李之友,仲舒以後進慕而入焉。性尚簡傲,不能接下,以此人多怨之。及歷二鎮,尤事奢侈。



 元和四年,拜京兆尹,為御史中丞李夷簡劾奏憑前為江西觀察使贓罪及他不法事,敕付御史臺覆按,刑部尚書李鄘、大理卿趙昌同鞫問臺中。又捕得憑前江西判官、監察御史楊瑗系於臺,復命大理少卿胡珦、左司員外郎胡證、侍御史韋顗同推鞫之。詔曰:「楊憑
 頃在先朝,委以籓鎮,累更選用,位列大官。近者憲司奏劾,暴揚前事,計錢累萬,曾不報聞,蒙蔽之罪,於何逃責?又營建居室,制度過差,侈靡之風,傷我儉德。以其自尹京邑,人頗懷之,將議刑書,是加愍惻。宜從遐譴,以誡百僚,可守賀州臨賀縣尉同正,仍馳驛發遣。」先是,憑在江西,夷簡自御史出,官在巡屬。憑頗疏縱,不顧接之。夷簡常切齒。及憑歸朝,修第於永寧里,功作並興,又廣蓄妓妾於永樂里之別宅,時人大以為言。夷簡乘眾議,舉劾
 前事,且言修營之僭,將欲殺之。及下獄,置對數日,未得其事。夷簡持之益急,上聞,且貶焉,追舊從事以驗。自貞元以來居方鎮者,為德宗所姑息,故窮極僭奢,無所畏忌。及憲宗即位,以法制臨下,夷簡首舉憑罪,故時議以為宜;然繩之太過,物論又譏其深切矣。



 鄭元,舉進士第,累遷御史中丞。貞元中為河中節度使杜確行軍司馬。確卒,遂繼為節度使,入拜尚書左丞。元和二年,轉戶部侍郎、兼御史大夫、判度支。三年春,遷刑
 部尚書,兼京兆尹。九月,復判度支,依前刑部尚書、兼御史大夫。元性嚴毅,有威斷,更踐劇任,時稱其能。元和四年,以疾辭職,守本官,逾月卒。



 杜兼,京兆人,貞觀中宰相杜正倫五代孫。舉進士,累闢諸府從事,拜濠州刺史。兼性浮險,豪侈矜氣。屬貞元中德宗厭兵革,姑息戎鎮,至軍郡刺史,亦難於更代。兼探上情,遂練卒修武,占召勁勇三千人以上聞,乃恣兇威。錄事參軍韋賞、團練判官陸楚,皆以守職論事忤兼,兼
 密誣奏二人通謀,扇動軍中。忽有制使至,兼率官吏迎於驛中,前呼韋賞、陸楚出,宣制杖殺之。賞進士擢第,楚兗公象先之孫,皆名家,有士林之譽;一朝以無罪受戮,郡中股慄,天下冤嘆之。又誣奏李籓,將殺之,語在籓事中。故兼所至,人側目焉。元和初,入為刑部、吏部郎中,拜給事中,除金商防禦使,旋授河南少尹、知府事,尋正拜河南尹。皆杜佑在相位所借護也。元和四年,卒於官。



 裴玢,京兆人。五代祖疏勒國王綽,武德中來朝,授鷹揚
 大將軍,封天郡公,因留闕下,遂為京兆人。玢初為金吾將軍論惟明



 傔,德宗幸奉天,以戰功封忠義郡王。惟明鎮鄜坊,累署玢為都虞候。後節度王棲曜卒,中軍將何朝宗謀作亂,中夜縱火,玢匿身不救火,遲明而擒朝宗。德宗發三司使按問,竟斬朝宗及行軍司馬崔輅,以同州刺史劉公濟為節度使,以玢為坊州長史、兼侍御史,充行軍司馬。明年,公濟卒,拜玢鄜州刺史、兼御史大夫,充節度觀察等使。三年,改授山南西道節度觀察等
 使。



 玢歷二鎮,頗以公清苦節為政;不交權幸,不務貢獻,蔬食敝衣,居處才避風雨,而廩庫饒實,三軍百姓安業,近代將帥無比焉。及綿疾辭位,請歸長安。元和七年卒,年六十五,贈尚書左僕射,謚曰節。



 薛伾,勝州刺史渙之子。尚父汾陽王召置麾下,著名於諸將間。左僕射李揆使西蕃,伾為將從役。時賊泚之難,昆夷赴義,伾馳騎鄉導,至於武功,擢授左威衛將軍。使絕域者前後數四,累遷左金吾衛大將軍、檢校工部尚
 書、兼將作監,出為鄜坊觀察使。元和八年,卒於官,贈潞州大都督。



 史臣曰:薛播溫敏有文,鮑防董戎無術,李、嚴太原之政,可謂美矣。蕭昕抱則哲之知,杜亞懷非次之望。王緯清潔而傷苛碎,若初善理而性剛嚴。於頎好任機權,趨附勢利。盧徵厚斂貨賄,結托中人。楊憑好奢,鄭元有斷。杜兼殺戮端士,怙亂邀君。裴玢發奸謀,安民和眾。而玢敝衣糲食,不交權幸,帑庾咸實,郡邑以寧。若夫君子無求
 備於人,舍短從長,彰善癉惡,則裴玢之善,抑之更揚;杜兼之惡,欲蓋而彰耳。



\end{pinyinscope}