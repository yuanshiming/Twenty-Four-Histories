\article{卷一百三十三}

\begin{pinyinscope}

 ○韓滉子皋弟洄張延賞子弘靖弘靖子文規次
 宗



 韓滉,字太沖,太子少師休之子也。少貞介好學,以廕解褐
 左威衛騎曹參軍,出為同官
 主簿。至德初,青齊節度鄧景山闢為判官,授監察御史、兼北
 海郡司馬,以道路阻絕,因避地山南。採訪使李承昭奏充判官,授通州長史、彭王府諮議參軍。鄧景山移鎮淮南,又表為賓佐,未行,除殿中侍御史,追赴京師。先是,滉兄法知制誥,草王璵拜官之詞,不加虛美,璵頗銜之。及其秉政,諸使奏滉兄弟者,必以冗官授之。璵免相,群議稱其屈,累遷至祠部、孝功、吏部三員外郎。



 滉公潔強直,明於吏道,判南曹凡五年,詳究簿書,無遺纖隱。大歷中,改吏部郎中、給事中。時盜殺富平令韋當,縣吏捕獲賊黨,而名隸北軍,監
 軍魚朝恩以有武材,請詔原其罪,滉密疏駁奏,賊遂伏辜。遷尚書右丞。五年,知兵部選。六年,改戶部侍郎、判度支。自至德、乾元已後,所在軍興,賦稅無度,帑藏給納,多務因循。滉既掌司計,清勤檢轄,不容奸妄,下吏及四方行綱過犯者,必痛繩之。又屬大歷五年已後,蕃戎罕侵,連歲豐稔,故滉能儲積穀帛,帑藏稍實。然苛克頗甚,覆治案牘,勾剝深文,人多咨怨。



 大歷十二年秋,霖雨害稼,京兆尹黎幹奏畿縣損田,滉執雲干奏不實。乃命御史
 巡覆,回奏諸縣凡損三萬一千一百九十五頃。時渭南令劉藻曲附滉,言所部無損,白於府及戶部。分巡御史趙計復檢行,奏與藻合。代宗覽奏,以為水旱咸均,不宜渭南獨免,申命御史硃敖再檢,渭南損田三千餘頃。上謂敖曰:「縣令職在字人,不損猶宜稱損,損而不問,豈有恤隱之意耶!卿之此行,可謂稱職。」下有司訊鞫,藻、計皆伏罪,藻貶萬州南浦員外尉,計貶豐州員外司戶。滉弄權樹黨,皆此類也。俄改太常卿,議未息,又出為晉州刺史。數月,拜
 蘇州刺史、浙江東西都團練觀察使。尋加檢校禮部尚書、兼御史大夫、潤州刺史、鎮軍節度使。滉既移鎮,安輯百姓,均其租稅,未及逾年,境內稱理。及建中年冬,涇師之亂,德宗出幸,河、汴騷然,滉訓練士卒,鍛礪戈甲,稱為精勁。李希烈既陷汴州,滉乃擇其銳卒,令裨將李長榮、王棲曜與宣武軍節度劉玄佐掎角討襲,解寧陵之圍,復宋、汴之路,滉功居多。



 然自關中多難,滉即於所部閉關梁,築石頭五城,自京口至玉山,禁馬牛出境;造樓
 船戰艦三十餘艘,以舟師五千人由海門揚威武,至申浦而還;毀撤上元縣佛寺道觀四十餘所,修塢壁,建業抵京峴,樓雉相屬,以佛殿材於石頭城繕置館第數十。時滉以國家多難,恐有永嘉渡江之事,以為備預,以迎鑾駕,亦申儆自守也。城中穿深井十丈近百所,下與江平,俾偏將丘涔督其役。涔酷虐士卒,日役千人,朝令夕辦,去城數十里內先賢丘墓,多令毀廢。明年正月,追李長榮等戍軍還,以其所親吏盧復為宣州刺史、採石軍
 使,增營壘,教習長兵。以佛寺銅鐘鑄弩牙兵器。陳少游時鎮揚州,以甲士三千人臨江大閱,滉亦以兵三千人臨金山,與少游相應,樓船於江中,以金銀繒彩互相聘賚。而自德宗出居,及歸京師,軍用既繁,道路又阻,關中饑饉,加之以災蝗,江南、兩浙轉輸粟帛,府無虛月,朝廷賴焉。



 興元元年,就加檢校吏部尚書。數月,又加檢校右僕射。貞元元年七月,拜檢校左僕射、同平章事,使並如故。二年春,特封晉國公。其年十一月,來朝京師。時右丞
 元琇判度支,以關輔旱儉,請運江淮租米以給京師。上以滉浙江東西節度,素著威名,加江淮轉運使,欲令專督運務。琇以滉性剛愎,難與集事,乃條奏滉督運江南米至揚子,凡一十八里,揚子以北,皆元琇主之。滉深怒於琇。琇以京師錢重貨輕,切疾之,乃於江東監院收獲見錢四十餘萬貫,令轉送入關。滉不許,乃誣奏云:「運千錢至京師,費錢至萬,於國有害。」請罷之。上以問琇,琇奏曰:「一千之重,約與一斗米均。自江南水路至京,一千之所
 運,費三百耳,豈至萬乎?」上然之,遣中使賚手詔令運錢。滉堅執以為不可。其年十二月,加滉度支諸道轉運鹽鐵等使,遂逞宿怒,累誣奏琇,貶雷州司戶。其責既重,舉朝以為非罪,多竊議者。尚書左丞董晉謂宰臣劉滋、齊映曰:「元左丞忽有貶責,未知罪名,用刑一濫,誰不危懼?假有權臣騁志,相公何不奏請三司詳斷之。去年關輔用兵,時方蝗旱,琇總國計,夙夜憂勤,以贍給師旅,不增一賦,軍國皆濟,斯可謂之勞臣也。今見播逐,恐失
 人心,人心一搖,則有聞雞起舞者矣。竊為相公痛惜之。」滋、映但引過而已。給事袁高又抗疏申理之,滉誣以朋黨,寢而不行。



 時兩河罷兵,中土寧乂,滉上言:「吐蕃盜有河湟,為日已久。大歷已前,中國多難,所以肆其侵軼。臣聞其近歲已來,兵眾浸弱,西迫大食之強,北病回紇之眾,東有南詔之防,計其分鎮之外,戰兵在河、隴五六萬而已。國家第令三數良將,長驅十萬眾,於涼、鄯、洮、渭並修堅城,各置二萬人,足當守禦之要。臣請以當道所貯蓄
 財賦為饋運之資,以充三年之費。然後營田積粟。且耕且戰,收復河、隴二十餘州,可翹足而待也。」上甚納其言。滉之入朝也,路由汴州,厚結劉玄佐,將薦其可任邊事,玄佐納其賂,因許之。及來覲,上訪問焉,初頗稟命,及滉以疾歸第,玄佐意怠,遂辭邊任,盛陳犬戎未衰,不可輕進。滉貞元三年二月,以疾薨,遂寢其事,年六十五。上震悼久之,廢朝三日,贈太傅,賻布帛米粟有差。



 滉,宰相子,幼有美名,其所結交,皆時之俊彥,非公直者不與之親密。
 性持節儉,志在奉公,衣裘茵衽,十年一易,居處陋薄,才蔽風雨。弟洄常於里宅增修廊宇,滉自江南至,即命撤去之,曰:「先公容焉,吾輩奉之,常恐失墜,所有摧圮,葺之則已,豈敢改作,以傷儉德。」自居重位,愈清儉嫉惡,彌縫闕漏,知無不為,家人盜產,未嘗在意。入仕之初,以至卿相,凡四十年,相繼乘馬五匹,皆及敝帷。尤工書,兼善丹青,以繪事非急務,自晦其能,未嘗傳之。好《易象》及《春秋》,著《春秋通例》及《天文事序議》各一卷。然以前輩早達,稍
 薄後進。晚歲至京師,丞郎卿佐,接之頗倨,眾不能平。其在浙右也,政令明察,未年傷於嚴急,巡內婺州傍縣有犯其令者,誅及鄰伍,死者數十百人。又俾推覆官分察境內,情涉疑似,必置極法,誅殺殘忍,一判即剿數十人,且無虛日。雖令行禁止,而冤濫相尋。議者以滉統制一方,頗著勤績,自幼立名貞廉,晚途政甚苛慘,身未達則飾情以進,得其志則本質遂彰。子群、皋。群,官至考功員外郎。



 皋字仲聞,夙負令名,而器質重厚,有大臣之度。由
 雲陽尉擢賢良科,拜右拾遺,轉左補闕,累遷起居郎、考功員外郎。俄丁父艱,德宗遣中人就第慰問,仍宣令論譔滉之事業,皋號泣承命,立草數千言,德宗嘉之。及免喪,執政者擬考功郎中,御筆加知制誥。遷中書舍人、御史中丞、尚書右丞、兵部侍郎,皆稱職。改京兆尹,奏鄭鋒為倉曹,專掌錢穀。鋒苛刻剝下為事,人皆咨怨。又勸皋搜索府中雜錢,折糴百姓粟麥等三十萬石進奉,以圖恩寵。皋納其計。尋奏鋒為興平縣令。



 及貞元十四年,春
 夏大旱,粟麥枯槁,畿內百姓,累經皋陳訴,以府中倉庫虛竭,憂迫惶惑,不敢實奏。會唐安公主女出適右庶子李愬,內官中使於愬家往來,百姓遮道投狀,內官繼以事上聞。德宗下詔曰:「京邑為四方之則,長吏受親人之寄,實系邦本,以分朕憂,茍非其才,是紊於理。正議大夫、守京兆尹、賜紫金魚袋韓皋,比踐清貫,頗聞謹恪,委之尹正,冀效公忠。乃者邦畿之間,粟麥不稔,朕念茲黎庶,方議蠲除,自宜悉心,以副勤恤。皋奏報失實,處理
 無方,致令閭井不安,囂然上訴。及令覆視,皆涉虛詞,壅蔽頗深,罔惑斯甚。宜加懲誡,以勖守官。可撫州司馬,員外置同正員,馳驛發遣。」。鋒亦尋出為汀州司馬。皋無幾移杭州刺史,復拜尚書右丞。



 皋恃前輩,頗以簡倨自處。順宗時,王叔文黨盛,皋嫉之,謂人曰:「吾不能事新貴。」皋從弟曄,幸於叔文,以告之,因出為鄂州刺史、岳鄂蘄沔等州觀察使。入為東都留守。元和八年六月,加檢校吏部尚書,兼許州刺史,充忠武軍節度等使。以陳、許二州水潦
 之後,賜皋綾絹布葛十萬端疋,以助軍資宴賞。所理以簡儉稱。入為吏部尚書,兼太子少傅,判太常卿事。元和十一年三月,皇太后王氏崩,以皋充大明宮使。十五年閏正月,充憲宗山陵禮儀使。三月,穆宗以師保之舊,加檢校右僕射。十二月,以銓司考科目人失實,與刑部侍郎知選事李建罰一月俸料。長慶元年正月,正拜尚書右僕射。二年四月,轉左僕射,赴尚書省上事,命中使宣賜酒饌,及宰臣百僚送上,皆如近式。其年,以本官東
 都留守,行及戲源驛暴卒,年七十九。贈太子太保。大和元年,謚曰貞。



 皋生知音律,嘗觀彈琴,至《止息》,嘆曰:「妙哉!嵇生之為是曲也,其當晉、魏之際乎!其音主商,商為秋聲。秋也者,天將搖落肅殺,其歲之晏乎!又晉乘金運,商,金聲,此所以知魏之季而晉將代也。慢其商弦,與宮同音,是臣奪君之義也,所以知司馬氏之將篡也。司馬懿受魏明帝顧托後嗣,反有篡奪之心,自誅曹爽,逆節彌露。王陵都督揚州,謀立荊王彪;毋丘儉、文欽、諸葛誕前後
 相繼為揚州都督,咸有匡復魏室之謀,皆為懿父子所殺。叔夜以揚州故廣陵之地,彼四人者,皆魏室文武大臣,咸敗散於廣陵,《散》言魏氏散亡,自廣陵始也。《止息》者,晉雖暴興,終止息於此也。其哀憤躁蹙,憯痛迫脅之旨,盡在於是矣。永嘉之亂,其應乎!叔夜撰此,將貽後代之知音者,且避晉、魏之禍,所以托之神鬼也。」



 洄以廕緒受任,劉晏判鹽鐵度支,闢為屬吏,累官至諫議大夫、知制誥。與元載善,載誅,以累貶邵州司戶同正員。建中元年二月,復諫議大夫。先以
 劉晏兼領度支,晏既罷黜,令天下錢穀各歸尚書省。本司廢職罷事,久無綱紀,徒收其名而莫綜其任,國用出入,未有所統,故轉洄戶部侍郎、判度支。洄上言:「江淮七監,歲鑄錢四萬五千貫,輸於京師,度工用轉送之費,每貫計錢二千,是本倍利也。今商州有紅崖冶,出銅益多,又有洛源監,久廢不理。請增工鑿山以取銅,興洛源故監,置十爐鑄之。歲計出錢七萬二千貫,度工用轉送之費,貫計錢九百,則利浮本矣。其江淮七監,請皆罷之。」復
 以「天下銅鐵之冶,是曰山澤之利,當歸於王者,非諸侯方岳所有。今諸道節度都團練使皆占之,非宜也,總隸鹽鐵使」。皆從之。



 洄與楊炎善,炎得罪,常不自安。無何,兄子皋抗疏理炎罪,德宗意洄令為之,尋貶蜀州刺史。興元元年三月,入為兵部侍郎。六月,為京兆尹。七月,加御史大夫。貞元二年正月,刑部侍郎劉太真黨於宰相盧杞得罪,以洄代太真為刑部侍郎,尋復兵部侍郎。貞元七年十一月,為國子祭酒。



 張延賞,中書令嘉貞之子。幼孤,本名寶符,開元末,玄宗召見,賜名延賞,取「賞延於世」之義,特授左司禦率府兵曹參軍。博涉經史,達於政事,侍中、韓國公苗晉卿見而奇之,以女妻焉。肅宗在鳳翔,擢拜監察御史,賜緋魚袋,轉殿中侍御史。關內節度使王思禮請為從事,思禮領河東,又為太原少尹,兼行軍司馬、北都副留守。



 代宗幸陜,除給事中,轉御史中丞、中書舍人。大歷二年,拜河南尹,充諸道營田副使。河洛久當兵沖,閭井丘墟,延賞勤
 身率下,政尚簡約,疏導河渠,修築宮廟,數年間流庸歸附,邦畿復完,詔書褒美焉。時罷河南、淮西、山南副元帥,以其兵鎮東都,延賞權知東都留守以領之,理行第一,入朝拜御史大夫。初,上封人李少良潛以元載陰事聞,載黨知之,奏少良狂妄,下御史臺訊鞫,欲有所屬。延賞不承其意,尋出為揚州刺史、淮南節度觀察等使。屬歲旱歉,人有亡去他境者,吏或拘之。延賞曰:「夫食,人之所恃而生也,此居而坐斃,適彼而可生,得存吾人,又何限
 於彼也。」乃具舟楫而遣之,俾吏修其廬室,已其逋債,而歸者增於其舊。邊江之瓜洲,舟航湊會,而縣屬江南,延賞奏請以江為界,人甚為便。尋以母憂去職,終制授授檢校禮部尚書、江陵尹、兼御史大夫、荊南節度觀察使。



 數年,改檢校兵部尚書、成都尹、劍南西川節度觀察使,依前兼御史大夫,尋就加吏部尚書。建中四年十一月,部將西山兵馬使張朏以兵入成都為亂,延賞奔漢州鹿頭,戍將叱干遂等討之。其月,斬朏及同惡者,復歸成都。
 先是兵革屢擾,自天寶末楊國忠用事南蠻,三蜀疲弊,屬車駕遷幸;其後郭英乂淫崔寧之室,遂縱崔寧、楊琳交亂;及崔寧得志,復極侈靡,故蜀土殘弊,蕩然無制度。延賞薄賦約事,動遵法度,僅至庶富焉。建中末,駕在山南,延賞貢奉供億,頗竭忠力焉。駕在梁州,倚劍南蜀川為根本。



 貞元元年,以宰相劉從一有疾,詔徵延賞為中書侍郎、同中書門下平章事。與鳳翔節度使李晟不協,晟表論延賞過惡,德宗重違晟意,延賞至興元,改授
 左僕射。初,大歷末,吐蕃寇劍南,李晟領神策軍戍之,及旋師,以成都官妓高氏歸。延賞聞而大怒,即使將吏令追還焉。晟頗銜之,形於詞色。三年正月,晟入朝,詔晟與延賞釋憾,德宗注意於延賞,將用之。會浙西觀察使韓滉來朝,嘗有德於晟,因會宴說晟使釋憾,遂同飲極歡,且請晟表薦為相,晟然之,於是復加同中書門下平章事。及延賞當國用事,晟請一子聘其女,固情好焉,延賞拒而不許。晟謂人曰:「武人性快,若釋舊惡於杯酒之間,終
 歡可解。文士難犯,雖修睦於外,而蓄怒於內,今不許婚,釁未忘也,得無懼焉!」無幾,延賞果謀罷晟兵權。初,吐蕃尚結贊興兵入隴州,抵鳳翔,無所虜掠,且曰:「召我來,何不持牛酒勞軍?」徐乃引去,持是以間晟。晟令牙將王佖選銳兵三千設伏汧陽,大敗吐蕃,結贊僅免,自是數遣使乞和。晟朝於京師,奏曰:「戎狄無信,不可許。」宰相韓滉又扶晟議,請調軍食以繼之,上意將帥生事邀功。會滉卒,延賞揣上意,遂行其志,奏令給事中鄭雲逵代之。上
 不許,且曰:「晟有社稷之功,令自舉代己者。」於是始用邢君牙焉。拜晟太尉、兼中書令,奉朝請而已。是年五月,吐蕃果背約以劫渾瑊。及冊晟太尉,故事,臨軒冊拜三公,中書令讀冊,侍中奉禮,如闕,即以宰相攝之。延賞欲輕其禮,始令兵部尚書崔漢衡攝中書令讀冊,時議非之。



 延賞奏議請省官員,曰:「為政之本,必先命官。舊制官員繁而且費,州縣殘破,職此之由。臣在荊南、劍南,所管州縣闕官員者,少不下十數年,吏部未嘗補授,但令一官假
 攝,公事亦理。以此言之,員可減無疑也。請減官員,收其祿俸,資幕職戰士,俾劉玄佐復河湟,軍用不乏矣。」上然之。初,韓滉入朝,至汴州,厚結劉玄佐,將薦其可委邊任,玄佐亦欲自效,初稟命,及滉卒,玄佐以疾辭,上遣中官勞問,臥以受命。延賞知不可用,奏用李抱真,抱真亦辭不行。時抱真判官陳曇奏事京師,延賞俾曇勸抱真,竟拒絕之。蓋以延賞挾怨罷李晟兵柄,由是武臣不附。自建議減員之後,物議不平。延賞懼,量留其官,下詔曰:「諸
 州府停減及所留官,並合厘務。其中有先考滿及充職掌,遇停減或恐公務有闕,宜委長吏於合停官中取考淺人清白乾舉者,留填闕官,差攝訖聞奏。但取才堪,不限資序。如當州官少,任以鄰州官充。其州縣諸色部送,準舊例以當州官及本土寄客有資產乾了者差遣。」及減員人眾,道路怨嘆,日聞於上。侍中馬燧奏減員太甚,恐不可行;太子少保韋倫及常參官等各抗疏以減員招怨,並請復之;浙西觀察使白志貞亦以疏論。時廷賞
 疾甚,在私第;李泌初為相,採於群情,由是官員悉復。



 貞元三年七月薨,年六十一,廢朝三日,贈太保,賻禮加等,謚曰成肅。



 子弘靖,字元理,雅厚信直。少以門廕授河南府參軍,調補藍田尉。東都留守杜亞闢為從事,奏改監察御史裏行,轉殿中侍御史、內供奉。留守將令狐運逐賊出郊,其日有劫轉運絹於道者,亞以運豪家子,意其為之,乃令判官穆員及弘靖同鞫其事。員與弘靖皆以運職在牙門,必不為盜,堅請不按。亞不聽,遂以獄聞,仍
 斥員及弘靖出幕府,有詔令三司使雜治之,後果於河南界得賊。無何,德陽公主下嫁,治第將侵弘靖家廟。弘靖拜表陳情,具述祖考之德,德宗慰撫之,不令毀廟。又獻賦美二京之制,德宗嘉其文,擢授監察御史。轉殿中侍御史、禮部員外郎;遷兵部郎中、知制誥、中書舍人、知東都選事;拜工部侍郎,轉戶部侍郎、陜州觀察、河中節度使;拜刑部尚書、同中書門下平章事。



 吳少陽死,其子元濟擅主留務,憲宗怒,欲下詔誅之。弘靖請先命吊贈
 使,待其不恭,然後加兵,憲宗從其議。尋加中書侍郎平章事。盜殺宰相武光衡,京師索賊未得。時王承宗邸中有鎮卒張晏輩數人,行止無狀,人多意之,詔錄付御史陳中師按之,皆附致其罪,如京中所說。弘靖疑其不直,驟於上前言之,憲宗不聽,竟殺張晏輩。及田弘正入鄆,按簿書,亦有殺元衡者,但事暖味,互有所說,卒未得其實。又殺張晏後,憲宗欲遂伐承宗。弘靖以為戎事並興,鮮有濟者,不若人並攻元濟,待淮西平,然後悉師河朔。憲
 宗業已北討,不為之止,然亦重違其言。弘靖知終不聽用,遂自陳乞罷政事。俄檢校吏部尚書、同中書門下平章事,充太原節度使。行未及鎮,果下詔誅承宗。弘靖以驟諫不行,宜用自效,大閱軍實,請躬討承宗。詔許出軍,不許自往。俄而魏博、澤潞悉為承宗所敗,有詔賞其前言。弘靖即間道發使懇喻承宗,承宗因亦款附。旋征拜吏部尚書,遷檢校右僕射、宣武軍節度使,時韓弘入覲之後也。弘靖用政寬緩,代弘之理。俄以劉總累求歸闕,
 且請弘靖代己,制加檢校司空平章事,充幽州、盧龍等軍節度使。



 弘靖之入幽州也,薊人無老幼男女,皆夾道而觀焉。河朔軍帥冒寒暑,多與士卒同,無張蓋安輿之別。弘靖久富貴,又不知風土,入燕之時,肩輿於三軍之中,薊人頗駭之。弘靖以祿山、思明之亂,始自幽州,欲於事初盡革其俗,乃發祿山墓,毀其棺柩,人尤失望。從事有韋雍、張宗厚數輩,復輕肆嗜酒,常夜飲醉歸,燭火滿街,前後呵叱,薊人所不習之事。又雍等詬責吏卒,多以
 反虜名之,謂軍士曰:「今天下無事,汝輩挽得兩石力弓,不如識一丁字。」軍中以意氣自負,深恨之。劉總歸朝,以錢一百萬貫賜軍士,弘靖留二十萬貫充軍府雜用。薊人不勝其憤,遂相率以叛,囚弘靖於薊門館,執韋雍、張宗厚輩數人,皆殺之。續有張徹者,自遠使回,軍人以其無過,不欲加害,將引置館中。徹不知其心,遂索弘靖所在,大罵軍人,亦為亂兵所殺。明日,吏卒稍稍自悔,悉詣館,請弘靖為帥,願改心事之。凡三請,弘靖卒不對。軍人
 乃相謂曰:「相公無言,是不赦吾曹必矣,軍中豈可一日無帥!」遂取硃洄為兵馬留後。朝廷既除洄子克融為幽州節度使,乃貶弘靖為撫州刺史。未幾,遷太子賓客、少保、少師。長慶四年六月卒,年六十五。



 元和初,王承宗阻兵,劉總父濟備陳征討之術,請身先之。及出軍,累拔城邑。總既繼父,願述先志,且欲盡更河朔舊風。長慶初,累表求入朝,兼請分割理之地,然後歸朝。其意欲以幽、涿、營州一道,請弘靖理之;瀛州為一道,盧士玫理之;平、
 薊、媯、檀為一道,請薛平理之。仍籍軍中宿將,盡薦於闕下,因望朝廷升獎,使幽、薊之人,皆有希美爵祿之意。及疏上,穆宗且欲速得範陽,宰臣崔植、杜元穎又不為遠大經略,但欲重弘靖所授而省其事局。唯瀛、莫兩州許置觀察使,其他郡縣悉命弘靖統之。時總所薦將校俱在京師旅舍中,久而不問,硃克融輩僅至假衣丐食,日詣中書求官,不勝其困。及除弘靖,命悉還本軍。克融輩雖得復歸,皆深懷觖望,其後因為叛亂。初,總以平、薊、媯、檀請
 薛平,於分裂之中尤為上策,而朝廷不能行之,竟致後患,人到於今惜之。



 子文規、景初、嗣慶、次宗。



 文規,歷拾遺、補闕、吏部員外郎。開成三年十一月,右丞韋溫彈劾文規:長慶中父弘靖陷在幽州,文規徘徊京師,不尋赴難,不宜塵汙南宮,乃出為安州刺史。累遷右散騎常侍、兼御史中丞、桂管都防禦觀察使。



 景初,歷職使府,官止殿中侍御史。



 嗣慶,位終河南少尹。



 次宗最有文學,稽古履行。開成中,為起居舍人。文宗復故事,每入閣,左右史執
 筆立於螭頭之下,宰相奏事,得以備錄。宰臣既退,上召左右史更質證所奏是非,故開成政事,詳於史氏,次宗尤稱奉職。改禮部員外郎,以兄文規為韋溫不放入省出官,次宗堅辭省秩,改國子博士兼史館修撰。出為舒州刺史,卒。



 文規子彥遠,大中初由左補闕為尚書祠部員外郎。景初子天保,嗣慶子彥修,次宗子曼容。延賞東都舊第在思順里,亭館之麗,甲於都城,子孫五代,無所加工,時號「三相張氏」云。



 史臣曰:君民足則國富,將相和則國安,反是道焉非得人者。滉殺元琇,奏瑞鹽,逞斡運之能,非貞純之士,刻下罔上,以為己功。幸逢多事之朝,例在姑息之地,幸而獲免,餘無可稱。延賞以私害公,罷李晟兵柄,使武臣不陳其力矣;惡直醜正,擠柳渾相位,致賢者不進其才矣。象恭僝功,皆四兇之跡也,雖以廕繼世,以才進身,蹈非道者,實小人哉!延賞歷典名籓,皆稱善政,及登大位,乃彰飾情。皋迭處大僚,徒稱舊德;弘靖輕傲邊事,欺減軍資;
 洄附元載、楊炎,繼及累貶,俱非守正中立者也。《書》云:「世祿之家,鮮克由禮。」不其是歟!



 贊曰:韓滉刻下,延賞害公。皋、洄繼世,弘靖興戎。



\end{pinyinscope}