\article{卷二十八 志第八 音樂一}

\begin{pinyinscope}

 樂者,太古聖人治情之具也。人有血氣生知之性,喜怒哀樂之
 情。情感物而動於中,聲成文而應於外。聖王乃調之以律度,文之以歌頌,蕩之以鐘石,播之以弦管,然後可以滌精靈,可以祛怨思。施之於邦國則朝廷序,施之於天下則神祇格,施之於賓宴則君臣和,施之於戰陣則士民勇。



 三五之代,世有厥官,故虞廷振干羽之容,周人立弦誦之教。洎蒼精道喪,戰國塵飛,禮樂出於諸侯,《雅》、《頌》淪於衰俗。齊竽燕築,俱非皦繹之音;東缶西琴,各寫哇淫之狀。乃至播鞀入漢,師摯寢弦。延陵有自鄶之譏,孔子起聞《韶》之嘆。及始皇一統,傲視百王。鐘鼓滿於秦宮,無非鄭、衛;歌舞陳於
 漢廟,並匪《咸》、《韶》。而九成、六變之容,八佾、四懸之制,但存其數,罕達其情。而制氏所傳,形容而已。武、宣之世,天子弘儒,
 採夜誦之詩,考從臣之賦,朝吟蘭殿,暮奏竹宮,乃命協律之官,始制禮神之曲。屬河間好古,遺籍充庭,乃約《詩頌》而制樂章,體《周官》而為舞節。自茲相襲,代易其辭,雖流管磬之音,恐異《莖》、《英》之旨。其後臥聽桑、濮,雜以《兜離》,孤竹、空桑,無復旋宮之義;崇牙樹羽,惟陳備物之儀。煩手即多,知音蓋寡。自永嘉之後,咸、洛為墟,禮壞樂崩,典章殆盡。江左掇其遺
 散,尚有治世之音。而元魏、宇文,代雄朔漠,地不傳於清樂,人各習其舊風。雖得兩京工胥,亦置四廂金奏。殊非入耳之玩,空有作樂之名。隋文帝家世士人,銳興禮樂,踐祚之始,詔太常卿牛弘、祭酒辛彥之增修雅樂。弘集伶官,措思歷載無成,而郊廟侑神,黃鐘一調而已。開皇九年平陳,始獲江左舊工及四懸樂器,帝令廷奏之,嘆曰:「此華夏正聲也,非吾此舉,世何得聞。」乃調五音為五夏、二舞、登歌、房中等十四調,賓、祭用之。隋氏始有雅樂,
 因置清商署以掌之。既而協律郎祖孝孫依京房舊法,推五音十二律為六十音,又六之,有三百六十音,旋相為宮,因定廟樂。諸儒論難,竟不施用。隋世雅音,惟清樂十四調而已。隋末大亂,其樂猶全。



 高祖受禪,擢祖孝孫為吏部郎中,轉太常少卿,漸見親委。孝孫由是奏請作樂。時軍國多務,未遑改創,樂府尚用隋氏舊文。武德九年,始命孝孫修定雅樂,至貞觀二年六月奏之。太宗曰:「禮樂之作,蓋聖人緣物設教,以為撙節,治之隆替,豈此
 之由?」御史大夫杜淹對曰:「前代興亡,實由於樂。陳將亡也,為《玉樹後庭花》;齊將亡也,而為《伴侶曲》。行路聞之,莫不悲泣,所謂亡國之音也。以是觀之,蓋樂之由也。」太宗曰:「不然,夫音聲能感人,自然之道也。故歡者聞之則悅,憂者聽之則悲,悲歡之情,在於人心,非由樂也。將亡之政,其民必苦,然苦心所感,故聞之則悲耳,何有樂聲哀怨,能使悅者悲乎?今《玉樹》、《伴侶》之曲,其聲具存,朕當為公奏之,知公必不悲矣。」尚書右丞魏徵進曰:「古人稱:『禮
 云禮云,玉帛云乎哉!樂云樂云,鐘鼓云乎哉!』樂在人和,不由音調。」太宗然之。孝孫又奏:陳、梁舊樂,雜用吳、楚之音;周、齊舊樂,多涉胡戎之伎。於是斟酌南北,考以古音,作為大唐雅樂。以十二律各順其月,旋相為宮。按《禮記》云,「大樂與天地同和」,故制十二和之樂,合三十一曲,八十四調。祭圓丘以黃鐘為宮,方澤以林鐘為宮,宗廟以太簇為宮。五郊、朝賀、饗宴,則隨月用律為宮。初,隋但用黃鐘一宮,惟扣七鐘,餘五鐘虛懸而不扣。及孝孫建旋
 宮之法,皆遍扣鐘,無復虛懸者矣。祭天神奏《豫和》之樂,地祇奏《順和》,宗廟奏《永和》。天地、宗廟登歌,俱奏《肅和》。皇帝臨軒,奏《太和》。王公出入,奏《舒和》。皇帝食舉及飲酒,奏《休和》。皇帝受朝,奏《政和》。皇太子軒懸出入,奏《承和》。元日,冬至皇帝禮會登歌,奏《昭和》。郊廟俎入,奏《雍和》。皇帝祭享酌酒、讀祝文及飲福、受胙,奏《壽和》。五郊迎氣,各以月律而奏其音。又郊廟祭享,奏《化康》、《凱安》之舞。《周禮》旋宮之義,亡絕已久,時莫能知,一朝復古,自此始也。及孝孫
 卒後,協律郎張文收復採《三禮》,言孝孫雖創其端,至於郊禋用樂,事未周備。詔文收與太常掌禮樂官等更加厘改。於是依《周禮》,祭昊天上帝以圓鐘為宮,黃鐘為角,太簇為徵,姑洗為羽,奏《豫和》之舞。若封太山,同用此樂。若地祇方丘,以函鐘為宮,太簇為角,姑洗為徵,南呂為羽,奏《順和》之舞。禪梁甫,同用此樂。祫禘宗廟,以黃鐘為宮,大呂為角,太簇為徵,應鐘為羽,奏《永和》之舞。五郊、日月星辰及類於上帝,黃鐘為宮,奏《豫和》之曲。大蠟、大報,
 以黃鐘、太簇、姑洗、蕤賓、夷則、無射等調奏《豫和》、《順和》、《永和》之曲。明堂、雩,以黃鐘為宮,奏《豫和》之曲。神州、社稷、藉田,宜以太簇為宮,雨師以姑洗為宮,山川以蕤賓為宮,並奏《順和》之曲。饗先妣,以夷則為宮,奏《永和》之舞。大饗宴,奏姑洗、蕤賓二調。皇帝郊廟、食舉,以月律為宮,並奏《休和》之曲。皇帝郊廟出入,奏《太和》之樂,臨軒出入,奏《舒和》之樂,並以姑洗為宮。皇帝大射,姑洗為宮,奏《騶虞》之曲。皇太子奏《貍首》之曲。皇太子軒懸,姑洗為宮,奏《永和》
 之曲。凡奏黃鐘,歌大呂;奏太簇,歌應鐘;奏姑洗,歌南呂;奏蕤賓,歌林鐘;奏夷則,歌中呂;奏無射,歌夾鐘。黃鐘蕤賓為宮,其樂九變;大呂、林鐘為宮,其樂八變。太簇、夷則為宮,其樂七變。夾鐘、南呂為宮,其樂六變。姑洗、無射為宮,其樂五變。中呂、應鐘為宮,其樂四變。天子十二鐘,上公九,侯伯七,子男五,卿六,大夫四,士三。及成,奏之。太宗稱善,於是加級頒賜各有差。



 十四年,敕曰:「殷薦祖考,以崇功德,比雖加以誠潔,而廟樂未稱。宜令所司詳諸故
 實,制定奏聞。」八座議曰:「七廟觀德,義冠於宗祀;三祖在天,式章於嚴配。致敬之情允洽,大孝之道宜宣。是以八佾具陳,肅儀形於綴兆;四懸備展,被鴻徽於雅音。考作樂之明義,擇皇王之令典,前聖所履,莫大於茲。伏惟皇帝陛下,天縱感通,率由冥極。孝理昭懿,光被於八埏;愛敬純深,追崇於百葉。永言錫祚,斯弘頌聲。鐘律革音,播鏗鏘於饗薦;羽籥成列,申蹈厲於烝嘗。爰詔典司,乃加隆稱,循聲核實,敬闡尊名。竊以皇靈滋慶,浚源長委,邁
 吞燕之生商,軼擾龍之肇漢,盛韜光於九二,漸發跡於三分。高祖縮地補天,重張區宇,反魂肉骨,再造生靈。恢恢帝圖,與二儀而合大;赫赫皇道,共七曜以齊明。雖復聖跡神功,不可得而窺測;經文緯武,敢有寄於名言。敬備樂章,式昭彞範。皇祖弘農府君、宣簡公、懿王三廟樂,請同奏《長發》之舞。太祖景皇帝廟樂,請奏《大基》之舞。世祖元皇帝廟樂,請奏《大成》之舞。高祖大武皇帝廟樂,請奏《大明》之舞。文德皇后廟樂,請奏《光大》之舞。七廟登歌,
 請每室別奏。」制可之。二十三年,太尉長孫無忌、侍中於志寧議太宗廟樂曰:「《易》曰:『先王作樂崇德,殷薦之上帝,以配祖考。』請樂名《崇德》之舞。」制可之。後文德皇后廟,有司據禮停《光大》之舞,惟進《崇德》之舞。



 光宅元年九月,高宗廟樂,以《鈞天》為名。中宗廟樂,奏《太和》之舞。開元六年十月敕,睿宗廟奏《景雲》之舞。二十九年六月,太常奏:「準十二年東封太山日所定雅樂,其樂曰《元和》六變,以降天神。《順和》八變,以降地祇。皇帝行,用《太和》之樂。其封太
 山也,登歌、奠玉幣,用《肅和》之樂;迎俎,用《雍和》之樂;酌獻、飲福,用《壽和》之樂;送文、迎武,用《舒和》之樂;亞獻、終獻,用《凱安》之樂;送神,用夾鐘宮《元和》之樂。神社首也,送神用林鐘宮《順和》之樂。享太廟也,迎神用《永和》之樂;獻祖宣皇帝酌獻用《光大》之舞,懿祖光皇帝酌獻用《長發》之舞,太祖景皇帝酌獻用《大政》之舞,世祖元皇帝酌獻用《大成》之舞,高祖神堯皇帝酌獻用《大明》之舞,太宗文皇帝酌獻用《崇德》之舞,高宗天皇大帝酌獻用《鈞天》之舞,中
 宗孝和皇帝酌獻用《太和》之舞,睿宗大聖貞皇帝酌獻用《景雲》之舞;徹豆,用《雍和》之舞;送神,用黃鐘宮《永和》之樂。臣以樂章殘缺,積有歲時。自有事東巡,親謁九廟,聖情慎禮,精祈感通,皆祠前累日考定音律,請編入史冊,萬代施行。」下制曰:「王公卿士,爰及有司,頻詣闕上言,請以『唐樂』為名者,斯至公之事,朕安得而辭焉。然則《大咸》、《大韶》、《大濩》、《大夏》,皆以大字表其樂章,今之所定,宜曰《大唐樂》。」皇祖弘農府君至高祖大武皇帝六廟,貞觀中已
 詔顏師古等定樂章舞號。洎今太常寺又奏有司所定獻祖宣皇帝至睿宗聖貞皇帝九廟酌獻用舞之號。



 天寶元年四月,命有司定玄元皇帝廟告享所奏樂,降神用《混成》之樂,送神用《太一》之樂。寶應二年六月,有司奏:玄宗廟樂請奏《廣運》之舞,肅宗廟樂請奏《惟新》之舞。大歷十四年,代宗廟樂請奏《保大》之舞。永貞元年十月,德宗廟樂請奏《文明》之舞。元和元年,順宗廟樂請奏《大順》之舞。元和十五年,憲宗廟樂請奏《象德》之舞。穆宗廟樂
 請奏《和寧》之舞。敬宗廟樂請奏《大鈞》之舞。文宗廟樂請奏《文成》之舞。武宗廟樂請奏《大定》之舞。



 貞觀元年,宴群臣,始奏秦王破陣之曲。太宗謂侍臣曰:「朕昔在籓,屢有征討,世間遂有此樂,豈意今日登於雅樂。然其發揚蹈厲,雖異文容,功業由之,致有今日,所以被於樂章,示不忘於本也。」尚書右僕射封德彞進曰:「陛下以聖武戡難,立極安人,功成化定,陳樂象德,實弘濟之盛烈,為將來之壯觀。文容習儀,豈得為比。」太宗曰:「朕雖以武功定天
 下,終當以文德綏海內。文武之道,各隨其時,公謂文容不如蹈厲,斯為過矣。」德彞頓首曰:「臣不敏,不足以知之。」其後令魏徵、虞世南、褚亮、李百藥改制歌辭,更名《七德》之舞,增舞者至百二十人,被甲執戟,以象戰陣之法焉。六年,太宗行幸慶善宮,宴從臣於渭水之濱,賦詩十韻。其宮即太宗降誕之所。車駕臨幸,每特感慶,賞賜閭里,有同漢之宛、沛焉。於是起居郎呂才以禦制詩等於樂府,被之管弦,名為《功成慶善樂》之曲,令童兒八佾,皆進
 德冠、紫褲褶,為《九功》之舞。冬至享宴,及國有大慶,與《七德》之舞皆奏於庭。七年,太宗制《破陣舞圖》:左圓右方,先偏後伍,魚麗鵝貫,箕張翼舒,交錯屈伸,首尾回互,以象戰陣之形。令呂才依圖教樂工百二十人,被甲執戟而習之。凡為三變,每變為四陣,有來往疾徐擊刺之象,以應歌節,數日而就,更名《七德》之舞。癸巳,奏《七德》、《九功》之舞,觀者見其抑揚蹈厲,莫不扼腕踴躍,凜然震竦。武臣列將咸上壽云:「此舞皆是陛下百戰百勝之形容。」群臣
 咸稱萬歲。蠻夷十餘種自請率舞,詔許之,久而乃罷。十四年,有景雲見,河水清。張文收採古《硃雁》、《天馬》之義,制《景雲河清歌》,名曰宴樂,奏之管弦,為諸樂之首,元會第一奏者是也。



 永徽二年十一月,高宗親祀南郊,黃門侍郎宇文節奏言:「依儀,明日朝群臣,除樂懸,請奏《九部樂》。」上因曰:「《破陣樂舞》者,情不忍觀,所司更不宜設。」言畢,慘愴久之。顯慶元年正月,改《破陣樂舞》為《神功破陣樂》。二年,太常奏《白雪》琴曲。先是,上以琴中雅曲,古人歌之,近
 代已來,此聲頓絕,雖有傳習,又失宮商,令所司簡樂工解琴笙者修習舊曲。至是太常上言曰:「臣謹按《禮記》、《家語》云:舜彈五弦之琴,歌《南風》之詩。是知琴操曲弄,皆合於歌。又張華《博物志》云:『《白雪》是大帝使素女鼓五十弦瑟曲名。』又楚大夫宋玉對襄王云:『有客於郢中歌《陽春白雪》,國中和者數十人。』是知《白雪》琴曲,本宜合歌,以其調高,人和遂寡。自宋玉以後,迄今千祀,未有能歌《白雪曲》者。臣今準敕,依於琴中舊曲,定其宮商,然後教習,並
 合於歌。輒以禦制《雪詩》為《白雪》歌辭。又按古今樂府,奏正曲之後,皆別有送聲,君唱臣和,事彰前史。輒取侍臣等奉和雪詩以為送聲,各十六節,今悉教訖,並皆諧韻。」上善之,乃付太常編於樂府。六年二月,太常丞呂才造琴歌《白雪》等曲,上制歌辭十六首,編入樂府。六年三月,上欲伐遼,於屯營教舞,召李義府、任雅相、許敬宗、許圉師、張延師、蘇定方、阿史那忠、于闐王伏闍、上官儀等,赴洛城門觀樂。樂名《一戎大定樂》。賜觀樂者雜彩有差。麟德
 二年十月,制曰:「國家平定天下,革命創制,紀功旌德,久被樂章。今郊祀四懸,猶用干戚之舞,先朝作樂,韜而未伸。其郊廟享宴等所奏宮懸,文舞宜用《功成慶善》之樂,皆著履執拂,依舊服褲褶、童子冠。其武舞宜用《神功破陣》之樂,皆被甲持戟,其執纛之人,亦著金甲。人數並依八佾,仍量加簫、笛、歌鼓等,並於懸南列坐,若舞即與宮懸合奏。其宴樂內二色舞者,仍依舊別設。」上元三年十一月敕:「供祠祭《上元舞》,前令大祠享皆將陳設。自今已後,圓丘
 方澤,大廟祠享,然後用此舞,餘祭並停。」



 儀鳳二年十一月六日,太常少卿韋萬石奏曰:「據《貞觀禮》,郊享日文舞奏《豫和》、《順和》、《永和》等樂,其舞人著委貌冠服,並手執籥翟。其武舞奏《凱安》,其舞人並著平冕,手執干戚。奉麟德二年十月敕,文舞改用《功成慶善樂》,武舞改用《神功破陣樂》,並改器服等。自奉敕以來,為《慶善樂》不可降神,《神功破陣樂》未入雅樂,雖改用器服,其舞猶依舊,迄今不改。事既不安,恐須別有處分者。」以今月六日錄奏,奉敕:「
 舊文舞、武舞既不可廢,並器服總宜依舊。若懸作《上元舞》日,仍奏《神功破陣樂》及《功成慶善樂》,並殿庭用舞,並須引出懸外作。其安置舞曲,宜更商量作安穩法。並錄《凱安》六變法象奏聞。」萬石又與刊正官等奏曰:



 謹按《凱安舞》是貞觀中所造武舞,準《貞觀禮》及今禮,但郊廟祭享奏武舞之樂即用之。凡有六變:一變象龍興參野,二變象克靖關中,三變象東夏賓服,四變象江淮寧謐,五變象獫狁讋伏,六變復位以崇,象兵還振旅。謹按《貞觀
 禮》,祭享日武舞惟作六變,亦如周之《大武》,六成樂止。按樂有因人而作者,則因人而止。如著成數者,數終即止,不得取行事賒促為樂終早晚,即禮云三闋、六成、八變、九變是也。今禮奏武舞六成,而數終未止,既非師古,不可依行。其武舞《凱安》,望請依古禮及《貞觀禮》,六成樂止。



 立部伎內《破陣樂》五十二遍,修入雅樂,只有兩遍,名曰《七德》。立部伎內《慶善樂》七遍,修入雅樂,只有一遍,名曰《九功》。《上元舞》二十九遍,今入雅樂,一無所減。每見祭享
 日三獻已終,《上元舞》猶自未畢,今更加《破陣樂》、《慶善樂》,兼恐酌獻已後,歌舞更長。其雅樂內《破陣樂》、《慶善樂》及《上元舞》三曲,並望修改通融,令長短與禮相稱,冀望久長安穩。《破陣樂》有象武事,《慶善樂》有象文事。按古六代舞,有《雲門》、《大咸》、《大夏》、《大韶》,是古之文舞;殷之《大濩》,周之《大武》是古之武舞。依古義,先儒相傳,國家以揖讓得天下,則先奏文舞。若以征伐得天下,則先奏武舞。望請應用二舞日,先奏《神功破陣樂》,次奏《功成慶善樂》。



 先奉敕於圓丘、
 方澤、太廟祠享日,則用《上元》之舞。臣據見行禮,欲於天皇酌獻降復位已後,即作《凱安》,六變樂止。其《神功破陣樂》、《功成慶善樂》、《上元》之舞三曲,待修改訖,以次通融作之,即得與舊樂前後不相妨破。若有司攝行事日,亦請據行事通融。



 從之。



 三年七月,上在九成宮咸亨殿宴集,有韓王元嘉、霍王元軌及南北軍將軍等。樂作,太常少卿韋萬石奏稱:「《破陣樂舞》者,是皇祚發跡所由,宣揚宗祖盛烈,傳之於後,永永無窮。自天皇臨馭四海,寢而不作,
 既緣聖情感愴,群下無敢關言。臣忝職樂司,廢缺是懼。依禮,祭之日,天子親總干戚以舞先祖之樂,與天下同樂之也,今《破陣樂》久廢,群下無所稱述,將何以發孝思之情?」上矍然改容,俯遂所請,有制令奏樂舞。既畢,上欷歔感咽,涕泗交流,臣下悲淚,莫能仰視。久之,顧謂兩王曰:「不見此樂,垂三十年,乍此觀聽,實深哀感。追思往日,王業艱難勤苦若此,朕今嗣守洪業,可忘武功?古人云:『富貴不與驕奢期,驕奢自至。』朕謂時見此舞,以自誡勖,
 冀無盈滿之過,非為歡樂奏陳之耳。」侍宴群臣咸呼萬歲。



 調露二年正月二十一日,則天禦洛城南樓賜宴,太常奏《六合還淳》之舞。長壽二年正月,則天親享萬象神宮。先是,上自制《神宮大樂》,舞用九百人,至是舞於神宮之庭。景龍二年,皇后上言:「自妃主及五品以上母妻,並不因夫子封者,請自今遷葬之日,特給鼓吹。宮官亦準此。」侍御史唐紹上諫曰:「竊聞鼓吹之作,本為軍容,昔黃帝涿鹿有功,以為警衛。故㧏鼓曲有《靈夔吼》、《雕鶚爭》、《石
 墜崖》、《壯士怒》之類。自昔功臣備禮,適得用之。丈夫有四方之功,所以恩加寵錫。假如郊祀天地,誠是重儀,惟有宮懸,本無案架。故知軍樂所備,尚不洽於神祇;鉦鼓之音,豈得接於閨閫。準式,公主王妃已下葬禮,惟有團扇、方扇、彩帷、錦障之色。加至鼓吹,歷代未聞。又準令,五品官婚葬,先無鼓吹,惟京官五品,得借四品鼓吹為儀。令特給五品已上母妻,五品官則不當給限。便是班秩本因夫子,儀飾乃復過之,事非倫次,難為定制,參詳義理,
 不可常行。請停前敕,各依常典。」上不納。延載元年正月二十三日,制《越古長年樂》一曲。



 玄宗在位多年,善音樂,若宴設酺會,即御勤政樓。先一日,金吾引駕仗北衙四軍甲士,未明陳仗,衛尉張設,光祿造食。候明,百僚朝,侍中進中嚴外辦,中官素扇,天子開簾受朝。禮畢,又素扇垂簾,百僚常參供奉官、貴戚、二王后、諸蕃酋長,謝食就坐。太常大鼓,藻繪如錦,樂工齊擊,聲震城闕。太常卿引雅樂,每色數十人,自南魚貫而進,列於樓下。鼓笛雞婁,
 充庭考擊。太常樂立部伎、坐部伎依點鼓舞,間以胡夷之伎。日旰,即內閑廄引蹀馬三十匹,為《傾杯樂曲》,奮首鼓尾,縱橫應節。又施三層板床,乘馬而上,抃轉如飛。又令宮女數百人自帷出擊雷鼓,為《破陣樂》、《太平樂》、《上元樂》。雖太常積習,皆不如其妙也。若《聖壽樂》,則回身換衣,作字如畫。又五坊使引大象入場,或拜或舞,動容鼓振,中於音律,竟日而退。玄宗又於聽政之暇,教太常樂工子弟三百人為絲竹之戲,音響齊發,有一聲誤,玄宗必覺
 而正之。號為皇帝弟子,又云梨園弟子,以置院近於禁苑之梨園。太常又有別教院,教供奉新曲。太常每凌晨,鼓笛亂發於太樂署。別教院廩食常千人,宮中居宜春院。玄宗又制新曲四十餘,又新制樂譜。每初年望夜,又御勤政樓,觀燈作樂,貴臣戚里,借看樓觀望。夜闌,太常樂府縣散樂畢,即遣宮女於樓前縛架出眺,歌舞以娛之。若繩戲竿木,詭異巧妙,固無其比。天寶十五載,玄宗西幸,祿山遣其逆黨載京師樂器樂伎衣盡入洛城。尋
 而肅宗克復兩京,將行大體,禮物盡闕。命禮儀使太常少卿於休烈使屬吏與東京留臺領,赴於朝廷。詔給錢,使休烈造伎衣及大舞等服,於是樂工二舞始備矣。



 乾元元年三月十九日,上以太常舊鐘磬,自隋已來所傳五聲,或有差錯,謂於休烈曰:「古者聖人作樂,以應天地之和,以合陰陽之序。和則人不夭札,物不疵癘。且金石絲竹,樂之器也。比親享郊廟,每聽樂聲,或宮商不倫,或鐘磬失度。可盡供鐘磬,朕當於內自定。」太常進入,上集
 樂工考試數日,審知差錯,然後令再造及磨刻。二十五日,一部先畢,召太常樂工,上臨三殿親觀考擊,皆合五音,送太常。二十八日,又於內造樂章三十一章,送太常,郊廟歌之。貞元三年四月,河東節度使馬燧獻《定難曲》。御麟德殿,命閱試之。十二年十二月,昭義軍節度使王虔休獻《繼天誕聖樂》。十四年二月,德宗自制《中和舞》,又奏《九部樂》及禁中歌舞。伎者十數人,布列在庭,上御麟德殿會百僚觀新樂詩,仍令太子書示百官。貞元十六
 年正月,南詔異牟尋作《奉聖樂舞》,因韋皋以進。十八年正月,驃國王來獻本國樂。



 大和八年十月,宣太常寺,準《雲韶樂》舊用人數,令於本寺閱習進來者。至開成元年十月,教成。三年,武德司奉宣索《雲韶樂縣圖》二軸,進之。



 大和三年八月,太常禮院奏:



 謹按凱樂,鼓吹之歌曲也。《周官大司樂》:「王師大獻,則奏凱樂。」注云:「獻功之樂也。」又《大司馬》之職,「師有功,則凱樂獻於社。」注云:「兵樂曰凱。」《司馬法》曰:「得意則凱樂,所以示喜也。」《左氏傳》載晉文公
 勝楚,振旅凱入。魏、晉已來鼓吹曲章,多述當時戰功,是則歷代獻捷,必有凱歌。太宗平東都,破宋金剛,其後蘇定方執賀魯,李勣平高麗,皆備軍容凱歌入京師。謹檢《貞觀》、《顯慶》、《開元禮》書,並無儀注。今參酌今古,備其陳設及奏歌曲之儀如後。



 凡命將征討,有大功獻俘馘者,其日備神策兵衛於東門外,如獻俘常儀。其凱樂用鐃吹二部,笛、篳篥、簫、笳、鐃、鼓,每色二人,歌工二十四人。樂工等乘馬執樂器,次第陳列,如鹵簿之式。鼓吹令丞前導,
 分行於兵馬俘馘之前。將入都門,鼓吹振作,迭奏《破陣樂》等四曲。《破陣樂》、《應聖期》兩曲,太常舊有辭。《賀朝歡》、《君臣同慶樂》,今撰補之。《破陣樂》「受律辭元首,相將討叛臣。咸歌《破陣樂》,共賞太平人。」《應聖期》:「聖德期昌運,雍熙萬宇清。乾坤資化育,海嶽共休明。闢土忻耕稼,銷戈遂偃兵。殊方歌帝澤,執贄賀升平。」《賀朝歡》:「四海皇風被,千年德水清。戎衣更不著,今日告功成。」《君臣同慶樂》:「主聖開昌歷,臣忠奏大猷。君看偃革後,便是太平秋。」候行至太
 社及太廟門,工人下馬,陳列於門外。按《周禮大司樂》注云:「獻於祖。」《大司馬》云:「先凱樂獻於社。」謹詳禮儀,則社廟之中,似合奏樂。伏以尊嚴之地,鐃吹嘩歡,既無明文,或乖肅敬。今請並於門外陳設,不奏歌曲。候告獻禮畢,復導引奏曲如儀。至皇帝所御樓前兵仗旌門外二十步,樂工皆下馬徐行前進。兵部尚書介胄執鉞,於旌門內中路前導。《周禮》:「師有功,則大司馬左執律,右秉鉞,以先凱樂。」注云:「律所以聽軍聲,鉞所以為將威。」今吹律聽聲,其術久廢,惟請秉鉞,以存禮文。次協律郎二人,公服執麾,亦於門下分導。鼓吹令、丞引樂工等至位立定。太常卿於樂工之前跪,具官臣某奏事,請奏凱
 樂。協律郎舉麾,鼓吹大振作,遍奏《破陣樂》等四曲。樂闋,協律郎偃麾,太常卿又跪奏凱樂畢。兵部尚書、太常卿退。樂工等並出旌門外訖,然後引俘馘入獻及稱賀如別儀。別有獻俘馘儀注。俟俘囚引出方退。



 請宣付當司,編入新禮,仍令樂工教習。



 依奏。



\end{pinyinscope}