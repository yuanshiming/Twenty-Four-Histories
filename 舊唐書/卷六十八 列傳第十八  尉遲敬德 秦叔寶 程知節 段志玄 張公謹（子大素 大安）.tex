\article{卷六十八 列傳第十八  尉遲敬德 秦叔寶 程知節 段志玄 張公謹(子大素 大安)}

\begin{pinyinscope}

 ○尉遲敬德秦叔寶程知節段志玄張公謹子大素大安



 尉遲敬德,朔州善陽人。大業末,從軍於高陽,討捕群賊,以武勇稱,累授朝散大夫。劉武周起,以為偏將,與宋金
 剛南侵,陷晉、澮二州。敬德深入,至夏縣,應接呂崇茂,襲破永安王孝基,執獨孤懷恩、唐儉等。武德三年,太宗討武周於柏壁,武周令敬德與宋金剛來拒王師於介休。金剛戰敗,奔於突厥;敬德收其餘眾,城守介休。太宗遣任城王道宗、宇文士及往諭之。敬德與尋相舉城來降。太宗大悅,賜以曲宴,引為右一府統軍,從擊王世充於東都。既而尋相與武周下降將皆叛,諸將疑敬德必叛,囚於軍中。行臺左僕射屈突通、尚書殷開山咸言:「敬德
 初歸國家,情志未附。此人勇健非常,縶之又久,既被猜貳,怨望必生。留之恐貽後悔,請即殺之。」太宗曰:「寡人所見,有異於此。敬德若懷翻背之計,豈在尋相之後耶?」遽命釋之,引入臥內,賜以金寶,謂曰:「丈夫以意氣相期,勿以小疑介意。寡人終不聽讒言以害忠良,公宜體之。必應欲去,今以此物相資,表一時共事之情也。」是日,因從獵於榆窠,遇王世充領步騎數萬來戰。世充驍將單雄信領騎直趨太宗,敬德躍馬大呼,橫刺雄信墜馬。賊徒
 稍卻,敬德翼太宗以出賊圍。更率騎兵與世充交戰,數合,其眾大潰,擒偽將陳智略,獲排槊兵六千人。太宗謂敬德曰:「比眾人證公必叛,天誘我意,獨保明之,福善有征,何相報之速也!」特賜金銀一篋,此後恩眄日隆。敬德善解避槊,每單騎入賊陣,賊槊攢刺,終不能傷,又能奪取賊槊,還以刺之。是日,出入重圍,往返無礙。齊王元吉亦善馬槊,聞而輕之,欲親自試,命去槊刃,以竿相刺。敬德曰:「縱使加刃,終不能傷。請勿除之,敬德槊謹當卻刃。」
 元吉竟不能中。太宗問曰:「奪槊、避槊,何者難易?」對曰:「奪槊難。」乃命敬德奪元吉槊。元吉執槊躍馬,志在刺之,敬德俄頃三奪其槊。元吉素驍勇,雖相嘆異,甚以為恥。及竇建德營於板渚,太宗將挑戰,先伏李勣、程知節、秦叔寶等兵。太宗持弓矢,敬德執槊,造建德壘下大呼致師。賊眾大驚擾,出兵數千騎,太宗逡巡漸卻,前後射殺數人,敬德所殺亦十數人,遂引賊以入伏內。於是與勣等奮擊,大破之。王世充兄子偽代王琬,使於建德軍中,乘
 隋煬帝所御驄馬,鎧甲甚鮮,迥出軍前以誇眾。太宗曰:「彼之所乘,真良馬也。」敬德請往取之,乃與高甑生、梁建方三騎直入賊軍,擒琬,引其馬以歸,賊眾無敢當者。又從討劉黑闥於臨洺,黑闥軍來襲李世勣,太宗勒兵掩賊,復以救之。既而黑闥眾至,其軍四合,敬德率壯士犯圍而入,大破賊陣,太宗與江夏王道宗乘之以出。又從破徐圓朗。累有戰功,授秦王府左二副護軍。



 隱太子、巢剌王元吉將謀害太宗,密致書以招敬德曰:「願迂長者
 之眷,敦布衣之交,幸副所望也。」仍贈以金銀器物一車。敬德辭曰:「敬德起自幽賤,逢遇隋亡,天下土崩,竄身無所,久淪逆地,罪不容誅。實荷秦王惠以生命,今又隸名籓邸,唯當以身報恩。於殿下無功,不敢謬當重賜。若私許殿下,便是二心,徇利忘忠,殿下亦何所用?」建成怒,是後遂絕。敬德尋以啟聞,太宗曰:「公之素心,鬱如山嶽,積金至斗,知公情不可移。送來但取,寧須慮也。若不然,恐公身不安。且知彼陰計,足為良策。」元吉等深忌敬德,令
 壯士往刺之。敬德知其計,乃重門洞開,安臥不動,賊頻至其庭,終不敢入。元吉乃譖敬德於高祖,下詔獄訊驗,將殺之,太宗固諫得釋。會突厥侵擾烏城,建成舉元吉為將,密謀請太宗同送於昆明池,將加屠害。敬德聞其謀,與長孫無忌遽啟太宗曰:「大王若不速正之,則恐被其所害,社稷危矣。」太宗嘆曰:「今二宮離阻骨肉,滅棄君親,危亡之機,共所知委。寡人雖深被猜忌,禍在須臾,然同氣之情,終所未忍。欲待其先起,然後以義討之,公意
 以為何如?」敬德曰:「人情畏死,眾人以死奉王,此天授也。若天與不取,反受其咎。雖存仁愛之小情,忘社稷之大計,禍至而不恐,將亡而自安,失人臣臨難不避之節,乏先賢大義滅親之事,非所聞也。以臣愚誠,請先誅之。王若不從,敬德言請奔逃亡命,不能交手受戮。且因敗成功,明賢之高見;轉禍為福,智士之先機。敬德今若逃亡,無忌亦欲同去。」太宗猶豫未決,無忌曰:「王今不從敬德之言,必知敬德等非王所有。事今敗矣,其若之何?」太宗
 曰:「寡人所言,未可全棄,公更圖之。」敬德曰:「王今處事有疑,非智;臨難不決,非勇。王縱不從敬德言,請自決計,其如家國何?其如身命何?且在外勇士八百餘人,今悉入宮,控弦被甲,事勢已就,王何得辭!」敬德又與侯君集日夜進勸,然後計定。時房玄齡、杜如晦皆被高祖斥出秦府,不得復入。太宗令長孫無忌密召之,玄齡等報曰:「有敕不許更事王,今若私謁,必至誅滅,不敢奉命。」太宗大怒,謂敬德曰:「玄齡、如晦豈背我耶?」取所佩刀授敬德曰:「
 公且往,觀其無來心,可並斬其首持來也。」敬德又與無忌喻曰:「王已決計克日平賊,公宜即入籌之。我等四人不宜群行在道。」於是玄齡、如晦著道士服隨無忌入,敬德別道亦至。



 六月四日,建成既死,敬德領七十騎躡踵繼至,元吉走馬東奔,左右射之墜馬。太宗所乘馬又逸於林下,橫被所繣,墜不能興。元吉遽來奪弓,垂欲相扼,敬德躍馬叱之,於是步走,欲歸武德殿,敬德奔逐射殺之。其宮府諸將薛萬徹、謝叔方、馮立等率兵大至,屯於玄
 武門,殺屯營將軍。敬德持建成、元吉首以示之,宮府兵遂散。是時,高祖泛舟於海池。太宗命敬德侍衛高祖。敬德擐甲持矛,直至高祖所。高祖大驚,問曰:「今日作亂是誰?卿來此何也?」對曰:「秦王以太子、齊王作亂,舉兵誅之,恐陛下驚動,遣臣來宿衛。」高祖意乃安。南衙、北門兵馬及二宮左右猶相拒戰,敬德奏請降手敕,令諸軍兵並受秦王處分,於是內外遂定。高祖勞敬德曰:「卿於國有安社稷之功。」賜珍物甚眾。太宗升春宮,授太子左衛率。
 時議者以建成等左右百餘人,並合從坐籍沒,唯敬德執之不聽,曰:「為罪者二兇,今已誅訖,若更及支黨,非取安之策。」由是獲免。及論功,敬德與長孫無忌為第一,各賜絹萬匹;齊王府財幣器物,封其全邸,盡賜敬德。



 貞觀元年,拜右武候大將軍,賜爵吳國公,與長孫無忌、房玄齡、杜如晦四人並食實封千三百戶。會突厥來入寇,授涇州道行軍總管以擊之。賊至涇陽,敬德輕騎與之挑戰,殺其名將,賊遂敗。敬德好訐直,負其功,每見無忌、玄齡、如晦等
 短長,必面折廷辯,由是與執政不平。三年,出為襄州都督。八年,累遷同州刺史。嘗侍宴慶善宮,時有班在其上者,敬德怒曰:「汝有何功,合坐我上?」任城王道宗次其下,因解喻之。敬德勃然,拳毆道宗目,幾至眇。太宗不懌而罷,謂敬德曰:「朕覽漢史,見高祖功臣獲全者少,意常尤之。及居大位以來,常欲保全功臣,令子孫無絕。然卿居官輒犯憲法,方知韓、彭夷戮,非漢祖之愆。國家大事,唯賞與罰,非分之恩,不可數行,勉自修飭,無貽後悔也。」十一年,封建功
 臣為代襲刺史,冊拜敬德宣州刺史,改封鄂國公。後歷鄜、夏二州都督。十七年,抗表乞骸骨,授開府儀同三司,令朝朔望。尋與長孫無忌等二十四人圖形於凌煙閣。及太宗將征高麗,敬德奏言:「車駕若自往遼左,皇太子又在定州,東西二京,府庫所在,雖有鎮守,終是空虛。遼東路遙,恐有玄感之變。且邊隅小國,不足親勞萬乘,伏請委之良將,自可應時摧滅。」太宗不納,令以本官行太常卿,為左一馬軍總管,從破高麗於駐蹕山。軍還,依舊
 致仕。敬德末年篤信仙方,飛煉金石,服食雲母粉,穿築池臺,崇飾羅綺,嘗奏清商樂以自奉養,不與外人交通,凡十六年。顯慶三年,高宗以敬德功,追贈其父為幽州都督。其年薨,年七十四。高宗為之舉哀,廢朝三日,令京官五品以上及朝集使赴宅哭,冊贈司徒、並州都督,謚曰忠武,賜東園秘器,陪葬於昭陵。子寶琳嗣,官至衛尉卿。



 秦叔寶,名瓊,齊州歷城人。大業中,為隋將來護兒帳內。叔寶喪母,護兒遣使吊之,軍吏怪曰:「士卒死亡及遭喪
 者多矣,將軍未嘗降問,獨吊叔寶何也?」答曰:「此人勇悍,加有志節,必當自取富貴,豈得以卑賤處之?」隋末群盜起,從通守張須陀擊賊帥盧明月於下邳。賊眾十餘萬,須陀所統才萬人,力勢不敵,去賊六七里立柵,相持十餘日,糧盡將退,謂諸將士曰:「賊見兵卻,必輕來追我。其眾既出,營內即虛,若以千人襲營,可有大利。此誠危險,誰能去者?」人皆莫對,唯叔寶與羅士信請行。於是須陀委柵遁,使二人分領千兵伏於蘆葦間。既而明月果悉
 兵追之,叔寶與士信馳至其柵,柵門閉不得入,二人超升其樓,拔賊旗幟,各殺數人,營中大亂。叔寶、士信又斬關以納外兵,因縱火焚其三十餘柵,煙焰漲天。明月奔還,須陀又回軍奮擊,大破賊眾。明月以數百騎遁去,餘皆虜之。由是勇氣聞於遠近。又擊孫宣雅於海曲,先登破之。以前後累勛授建節尉。從須陀進擊李密於滎陽,軍敗,須陀死之,叔寶以餘眾附裴仁基。會仁基以武牢降於李密,密得叔寶大喜,以為帳內驃騎,待之甚厚。密
 與化及大戰於黎陽童山,為流矢所中,墮馬悶絕。左右奔散,追兵且至,唯叔寶獨捍衛之,密遂獲免。叔寶又收兵與之力戰,化及乃退。後密敗,又為王世充所得,署龍驤大將軍。叔寶薄世充之多詐,因其出抗官軍,至於九曲,與程咬金、吳黑闥、牛進達等數十騎西馳百許步,下馬拜世充曰:「雖蒙殊禮,不能仰事,請從此辭。」世充不敢逼,於是來降。高祖令事秦府,太宗素聞其勇,厚加禮遇。從鎮長春宮,拜馬軍總管。又從征於美良川,破尉遲敬
 德,功最居多。高祖遣使賜以金瓶,勞之曰:「卿不顧妻子,遠來投我,又立功效。朕肉可為卿用者,當割以賜卿,況子女玉帛乎?卿當勉之。」尋授秦王右三統軍。又從破宋金剛於介休。錄前後勛,賜黃金百斤、雜彩六千段,授上柱國。從討王世充,每為前鋒。太宗將拒竇建德於武牢,叔寶以精騎數十先陷其陣。世充平,進封翼國公,賜黃金百斤、帛七千段。從平劉黑闥,賞物千段。叔寶每從太宗征伐,敵中有驍將銳卒,炫耀人馬,出入來去者,太宗
 頗怒之,輒命叔寶往取。叔寶應命,躍馬負槍而進,必刺之萬眾之中,人馬闢易,太宗以是益重之,叔寶亦以此頗自矜尚。



 六月四日,從誅建成、元吉。事寧,拜左武衛大將軍,食實封七百戶。其後每多疾病,因謂人曰:「吾少長戎馬,所經二百餘陣,屢中重瘡。計吾前後出血亦數斛矣,安得不病乎?」十二年卒,贈徐州都督,陪葬昭陵。太宗特令所司就其塋內立石人馬,以旌戰陣之功焉。十三年,改封胡國公。十七年,與長孫無忌等圖形於凌煙閣。



 程知節,本名咬金,濟州東阿人也。少驍勇,善用馬槊。大業末,聚徒數百,共保鄉里,以備他盜。後依李密,署為內軍驃騎。時密於軍中簡勇士尤異者八千人,隸四驃騎,分為左右以自衛,號為內軍。自云:「此八千人可當百萬。」知節既領其一,甚被恩遇。及王世充出城決戰,知節領內馬軍,與密同營在北邙山上,單雄信領外馬軍,營在偃師城北。世充來襲雄信營,密遣知節及裴行儼助之。行儼先馳赴敵,為流矢所中,墜於地。知節救之,殺數人,
 世充軍披靡,乃抱行儼重騎而還。為世充騎所逐,刺槊洞過,知節回身捩折其槊,兼斬獲追者,於是與行儼俱免。及密敗,世充得之,接遇甚厚。知節謂秦叔寶曰:「世充器度淺狹,而多妄語,好為咒誓,乃巫師老嫗耳,豈是撥亂主乎?」及世充拒王師於九曲,知節領兵在其陣,與秦叔寶等馬上揖世充曰:「荷公接待,極欲報恩。公性猜貳,傍多扇惑,非僕托身之所,今謹奉辭。」於是躍馬與左右數十人歸國,世充懼,不敢追之。授秦王府左三統軍。破
 宋金剛,擒竇建德,降王世充,並領左一馬軍總管。每陣先登,以功封宿國公。武德七年,建成忌之,構之於高祖,除康州刺史。知節白太宗曰:「大王手臂今並翦除,身必不久。知節以死不去,願速自全。」六月四日,從太宗討建成、元吉。事定,拜太子右衛率,遷右武衛大將軍,賜實封七百戶。貞觀中,歷瀘州都督、左領軍大將軍。與長孫無忌等代襲刺史,改封盧國公,授普州刺史。十七年,累轉左屯衛大將軍,檢校北門屯兵,加鎮軍大將軍。永徽六
 年,遷左衛大將軍。顯慶二年,授蔥山道行軍大總管以討賀魯。師次怛篤城,有胡人數千家開門出降,知節屠城而去,賀魯遂即遠遁。軍還,坐免官。未幾,授岐州刺史。表請乞骸骨,許之。麟德二年卒,贈驃騎大將軍、益州大都督,陪葬昭陵。子處默,襲爵盧國公。處亮,以功臣子尚太宗女清河長公主,授駙馬都尉、左衛中郎將。少子處弼,官至右金吾將軍。處弼子伯獻,開元中,左金吾大將軍。



 段志玄,齊州臨淄人也。父偃師,隋末為太原郡司法書佐,從高祖起義,官至郢州刺史。志玄從父在太原,甚為太宗所接待。義兵起,志玄募得千餘人,授右領大都督府軍頭。從平霍邑,下絳郡,攻永豐倉,皆為先鋒,歷遷左光祿大夫。從劉文靜拒屈突通於潼關,文靜為通將桑顯和所襲,軍營已潰,志玄率二十騎赴擊,殺數十人而還。為流矢中足,慮眾心動,忍而不言,更入賊陣者再三。顯和軍亂,大軍因此復振,擊,大破之。及屈突通之遁,志
 玄與諸將追而擒之,以功授樂游府驃騎將軍。後從討王世充,深入陷陣,馬倒,為賊所擒。兩騎夾持其髻,將渡洛水,志玄踴身而奮,二人俱墜馬,馳歸,追者數百騎,不敢逼。及破竇建德,平東都,功又居多。遷秦王府右二護軍,賞物二千段。隱太子建成、巢剌王元吉競以金帛誘之,志玄拒而不納,密以白太宗,竟與尉遲敬德等同誅建成、元吉。太宗即位,累遷左驍衛大將軍,封樊國公,食實封九百戶。文德皇后之葬也,志玄與宇文士及分統
 士馬出肅章門。太宗夜使宮官至二將軍所,士及開營內使者,志玄閉門不納,曰:「軍門不可夜開。」使者曰:「此有手敕。」志玄曰:「夜中不辯真偽。」竟停使者至曉。太宗聞而嘆曰:「此真將軍也,周亞夫無以加焉。」十一年,定世封之制,授金州刺史,改封褒國公。十二年,拜右衛大將軍。十四年,加鎮軍大將軍。十六年,寢疾,太宗親自臨視,涕泣而別,顧謂曰:「當與卿子五品。」志玄頓首固請回授母弟志感,太宗遂授志感左衛郎將。及卒,上為發哀,哭之甚
 慟,贈輔國將軍、揚州都督,陪葬昭陵,謚曰忠壯。十七年正月,詔圖形於凌煙閣。子瓚,襲爵褒國公,武太后時,官至左屯衛大將軍。子懷簡,襲爵,開元中,官至太子詹事。



 張公謹,字弘慎,魏州繁水人也。初為王世充洧州長史。武德元年,與王世充所署洧州刺史崔樞以州城歸國,授鄒州別駕,累除右武候長史。初未知名,李勣尉遲敬德亦言之,乃引入幕府。時太宗為隱太子建成、巢王元吉所忌,因召公謹,問以自安之策,對甚合
 旨,漸見親遇。及太宗將討建成、元吉,遣卜者灼龜占之,公謹自外來見,遽投於地而進曰:「凡卜筮者,將以決嫌疑,定猶豫,今既事在不疑,何卜之有?縱卜之不吉,勢不可已。願大王思之。」太宗深然其言。六月四日,公謹與長孫無忌等九人伏於玄武門以俟變。及斬建成、元吉,其黨來攻玄武門,兵鋒甚盛。公謹有勇力,獨閉門以拒之。以功累授左武候將軍,封定遠郡公,賜實封一千戶。貞觀元年,拜代州都督,上表請置屯田以省轉運,又前後
 言時政得失十餘事,並見納用。後遣李靖經略突厥,以公謹為副,公謹因言突厥可取之狀曰:「頡利縱欲肆情,窮兇極暴,誅害良善,暱近小人,此主昏於上,其可取一也。又其別部同羅、僕骨、回紇、延陀之類,並自立君長,將圖反噬,此則眾叛於下,其可取二也。突厥被疑,輕騎自免;拓設出討,匹馬不歸;欲谷喪師,立足無地,此則兵挫將敗,其可取三也。塞北霜早,糧餱乏絕,其可取四也。頡利疏其突厥,親委諸胡,胡人翻覆,是其常性,大軍一
 臨,內必生變,其可取五也。華人入北,其類實多,比聞自相嘯聚,保據山險,師出塞垣,自然有應,其可取六也。」太宗深納之。破定襄,敗頡利,璽書慰勞,進封鄒國公。



 轉襄州都督,甚有惠政。卒官,年三十九。太宗聞而嗟悼,出次發哀,有司奏言:「準《陰陽書》,日子在辰,不可哭泣,又為流俗所忌。」太宗曰:「君臣之義,同於父子,情發於衷,安避辰日?」遂哭之。贈左驍衛大將軍,謚曰襄。十三年,追思舊功,改封郯國公。十七年,圖形於凌煙閣。永徽中,又贈荊州都
 督。長子大象嗣,官至戶部侍郎。次子大素、大安,並知名。大素,龍朔中歷位東臺舍人,兼修國史,卒於懷州長史,撰《後魏書》一百卷、《隋書》三十卷。大安,上元中歷太子庶子、同中書門下三品。時章懷太子在春宮,令大安與太子洗馬劉訥言等注範曄《後漢書》。宮廢,左授普州刺史。光宅中,卒於橫州司馬。大安子涚,開元中為國子祭酒。



 史臣曰:敬德奪槊陷陣,鼓勇王師,卻賂報恩,竭忠霸主。然而奮拳負氣,非自全之道;文皇告誡之言,可為功臣
 藥石。叔寶善用馬槊,拔賊壘則以寡敵眾,可謂勇矣。知節志平國難,拜隼籞則致命輔君,可謂忠矣。而並曉世充之猜貳,識唐代之霸圖,可謂見幾君子矣。志玄中鏑不言,竟安師旅。公謹投龜定議,志助儲君。皆所謂猛將謀臣,知機識變。有唐之盛,斯實賴焉。



 贊曰:太宗經綸,實賴虎臣。胡、鄂諸將,奮不顧身。圖形凌煙,配食嚴禋。光諸簡冊,為報君親。



\end{pinyinscope}