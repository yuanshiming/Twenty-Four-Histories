\article{卷二百三}

\begin{pinyinscope}

 ○李
 德武妻裴氏楊慶妻王氏獨孤師仁乳母王氏附楊三安妻李氏魏衡妻王氏樊會仁母敬氏絳州孝女衛氏
 濮州孝女賈氏鄭義宗妻盧氏劉寂妻夏侯氏楚王靈龜妃上官氏楊紹宗妻王氏於敏直妻張氏冀州女子王氏樊彥琛妻魏氏鄒保英妻奚氏古玄應妻高氏附宋庭瑜妻魏氏崔繪妻盧氏奉天縣竇氏二女盧甫妻李氏王泛妻裴氏附鄒待徵妻薄氏李湍妻
 董昌齡母楊氏韋雍妻蘭陵縣君蕭氏衡方厚妻武昌縣君程氏女道士李玄真孝女王和子鄭神佐女附



 女子稟陰柔之質,有從人之義。前代志貞婦烈女,蓋善其能以禮自防。至若失身賊庭,不污非義;臨白刃而慷慨,誓丹衷而激發;粉身不顧,視死如歸,雖在壯夫,恐難守節,窈窕之操,不其賢乎!其次梁鴻之妻,無辭偕隱,共姜之誓,不踐二庭,婦道母儀,克彰圖史,又其長也。末代
 風靡,貞行寂寥,聊播椒蘭,以貽閨壺,彤管之職,幸無忽焉!



 李德武妻裴氏,字淑英,戶部尚書、安邑公矩之女也。性婉順有容德,事父母以孝聞。適德武,經一年而德武坐從父金才事徙嶺表。矩時為黃門侍郎,奏請德武離婚,煬帝許之。德武將與裴別,謂曰:「燕婉始爾,便事分離,方遠投瘴癘,恐無還理。尊君奏留,必欲改嫁耳,於此即事長訣矣!」裴泣而對曰:「婦人事夫,無再醮之禮。夫者,天也,
 何可背乎!守之以死,必無他志!」因操刀欲割耳自誓,保者禁之,乃止。



 裴與德武別後,容貌毀悴,常讀佛經,不御膏澤。李氏之姊妹在都邑者,歲時朔望,必命左右致敬而省焉。裴又嘗讀《烈女傳》,見稱述不改嫁者,乃謂所親曰:「不踐二庭,婦人常理,何為以此載於記傳乎?」後十餘年間,與德武音信斷絕。矩欲奪其志。時有柳直求婚,許之。期有定日,乃以翦刀斷其發,悲泣絕粒。矩不可奪,乃止。德武已於嶺表娶爾硃氏為妻,及遇赦得還,至襄州,聞
 裴守節,乃出其後妻,重與裴合。生三男四女。貞觀中,德武終於鹿城令,裴歲餘亦卒。



 楊慶妻王氏,世充兄之女也。慶即隋河間王弘之子。大業末,封郇王,為滎陽太守。後陷於世充。世充以兄女妻之,授管州刺史。及太宗攻圍洛陽,慶謀背世充,欲與其妻俱來歸國。妻謂慶曰:「鄭國以妾奉箕帚於公者,所以結公心耳。今既二三其行,負恩背義,自為身謀,妾將奈何?若至長安,則公家之婢耳!願送至東都,公之惠也。」慶
 不聽。伺慶出後,謂侍者曰:「唐兵若勝,我家則戚。鄭國無危,吾夫又死,進退維谷。何以生焉?」乃飲藥而卒。慶既入朝,官至宜州刺史。



 時又有獨孤武都,謀叛王世充歸國,事覺誅死。武都子師仁,年始三歲,世充以其年幼不殺,使禁掌之。乳母王氏,號蘭英,請髡鉗,求入保養,世充許之。蘭英撫育提攜,備盡筋力。時喪亂年饑,人多鋨死,蘭英扶路乞丐捃拾。遇有所得,便歸與師仁;蘭英唯啖土飲水而已。後詐採拾,乃竊師仁歸於京師。高祖嘉其義,
 下詔曰:「師仁乳母王氏,慈惠有聞,撫鞠無倦,提攜遺幼,背逆歸朝。宜有褒隆,以錫其號。可封永壽郡君」。



 楊三安妻李氏,雍州涇陽人也。事舅姑以孝聞。及舅姑亡沒,三安亦死,二子孩童,家至貧窶。李晝則力田,夜紡緝,數年間葬舅姑及夫之叔侄兄弟者七喪,深為遠近所嗟尚。太宗聞而異之,賜帛二百段,遣州縣所在存恤之。



 魏衡妻王氏,梓州郪人也。武德初,薛仁杲舊將房企地
 侵掠梁郡,因獲王氏,逼而妻之。後企地漸強盛,衡謀以城應賊。企地領眾將趨梁州,未至數十里,飲酒醉臥。王氏取其佩刀斬之,攜其首入城,賊眾乃散。高祖大悅,封為崇義夫人,舍衡同賊之罪。



 樊會仁母敬氏,字像子,蒲州河東人也。年十五,適樊氏,生會仁而夫喪,事舅姑姊姒以謹順聞。及服終,母兄以其盛年,將奪其志。微加諷諭,便悲恨嗚咽,如此者數四。母兄乃潛許人為婚,矯稱母患以召之。凡所營具,皆寄
 之鄰里。像子既至,省母無疾,鄰家復具肴善,像子知為所欺,佯為不悟者。其嫂復請像子沐浴。像子私謂會仁曰:「吾不幸孀居,誓與汝父同穴。所以不死者,徒以我母羸老,汝身幼弱。今汝舅欲奪吾志,將加逼迫,於汝何如!」會仁失聲啼泣。像子撫之曰:「汝勿啼。吾向偽不覺者,令汝舅不我為意。聞汝啼,知吾覺悟,必加妨備,則吾難為計矣!」會仁便佯睡,像子於是伺隙攜之遁歸。中路,兄使追及之,將逼與俱返。像子誓以必死,辭情甚切,其兄感
 嘆而止。後會仁年十八病卒,時像子母已終。既葬,像子謂其所親曰:「吾老母不幸,又夫死子亡,義無久活。」於是號慟不食,數日而死。



 絳州孝女衛氏,字無忌,夏縣人也。初,其父為鄉人衛長則所殺。無忌年六歲,母又改嫁,無兄弟。及長,常思復仇。無忌從伯常設宴為樂,長則時亦預坐,無忌以磚擊殺之。既而詣吏,稱父仇既報,請就刑戮。巡察大使、黃門侍郎褚遂良以聞,太宗嘉其孝烈,特令免罪,給傳乘徙於
 雍州,並給田宅,仍令州縣以禮嫁之。



 孝女賈氏,濮州鄄城人也。年始十五,其父為宗人玄基所害。其弟強仁年幼,賈氏撫育之,誓以不嫁。及強仁成童,思共報復,乃俟玄基殺之;取其心肝,以祭父墓。遣強仁自列於縣,司斷以極刑。賈氏詣闕自陳己為,請代強仁死。高宗哀之,特下制賈氏及強仁免罪,移其家於洛陽。



 鄭義宗妻盧氏,幽州範陽人,盧彥衡之女也。略涉書史,
 事舅姑甚得婦道。嘗夜有強盜數十人,持杖鼓噪,逾垣而入,家人悉奔竄,唯有姑獨在室。盧冒白刃往至姑側,為賊捶擊之,幾至於死。賊去後,家人問曰:「群兇擾橫,人盡奔逃,何獨不懼?」答曰:「人所以異於禽獸者,以其仁義也。昔宋伯姬守義赴火,流稱至今。吾雖不敏,安敢忘義!且鄰里有急,尚相赴救,況在於姑,而可委棄!若萬一危禍,豈宜獨生!」其姑每嘆云:「古人稱歲寒然後知松柏之後凋也,吾今乃知盧新婦之心矣!」貞觀中卒。



 劉寂妻夏侯氏,滑州胙城人,字碎金。父長雲,為鹽城縣丞,因疾喪明。碎金乃求離其夫,以終侍養。經十五年,兼事後母,以至孝聞。及父卒,毀瘠殆不勝喪,被發徒跣,負土成墳,廬於墓側,每日一食,如此者積年。貞觀中,有制表其門閭,賜以粟帛。



 楚王靈龜妃上官氏,秦州上邽人。父懷仁,右金吾將軍。上官年十八,歸於靈龜,繼楚哀王後。本生具存,朝夕侍奉,恭謹彌甚。凡有新味,非舅姑啖訖,未曾先嘗。經數載,
 靈龜薨。及將葬,其前妃閻氏,嫁不逾年而卒,又無近族,眾議欲不舉之。上官氏曰:「必神而靈,寧可使孤魂無托!」於是備禮同葬,聞者莫不嘉嘆。服終,諸兄姊謂曰:「妃年尚少,又無所生,改醮異門,禮儀常範,妃可思之。」妃掩泣對曰:「丈夫以義烈標名,婦人以守節為行。未能即先犬馬,以殉溝壑,寧可復飾妝服,有他志乎!」遽將刀截鼻割耳以自誓,諸兄姊知其志不可奪,嘆息而止。尋卒。



 楊紹宗妻王氏,華州華陰人也。初,年二歲,所生母亡,為
 繼母鞠養。至年十五,父又征遼而歿。繼母尋亦卒。王乃收所生及繼母尸柩,並立父形像,招魂遷葬訖,廬於墓側,陪其祖父母及父母墳。永徽中,詔曰:「故楊紹宗妻王氏,因心為孝,率性成道。年迫桑榆,筋力衰謝。以往在隋朝,父歿遼左,招魂遷葬,負土成墳,又葬其祖父母等,遏此老年,親加板築。痛結晨昏,哀感行路。永言志行,嘉尚良深。宜標其門閭,用旌敏德。」賜物三十段、粟五十石。



 於敏直妻張氏,營州都督、皖城公儉之女也。數歲時父
 母權有疾,即觀察顏色,不離左右,晝夜省侍,宛若成人。及稍成長,恭順彌甚。適延壽公於欽明子敏直。初聞儉有疾,便即號踴自傷,期於必死。儉卒後,兇問至,號哭一慟而絕。高宗下詔,賜物百段,仍令史官錄之。



 冀州鹿城女子王阿足者,早孤,無兄弟,唯姊一人。阿足初適同縣李氏,未有子而夫亡。時年尚少,人多聘之。為姊年老孤寡,不能舍去,乃誓不嫁,以養其姊。每晝營田業,夜便紡績,衣食所須,無非阿足出者,如此二十餘年。
 及姊喪,葬送以禮。鄉人莫不稱其節行,競令妻女求與相識。後數歲,竟終於家。



 樊彥琛妻魏氏,楚州淮陰人。彥琛病篤,將卒,魏泣而言曰:「幸以愚陋,托身明德,奉侍衣裳,二十餘載。豈意釁妨所招,遽見此禍。同入黃泉,是其願也。」彥琛答曰:「死生常道,無所多恨。君宜勉勵,養諸孤,使其成立。若相從而死,適足貽累,非吾所取也。」彥琛卒後,屬李敬業之亂,乃為賊所獲。賊黨知其素解絲竹,逼令彈箏。魏氏嘆曰:「我夫
 不幸亡歿,未能自盡,茍復偷生。今爾見逼管弦,豈非禍從手發耶?」乃引刀斬指,棄之於地。賊黨又欲妻之,魏以必死自固。賊等忿怒,以刃加頸,語云:「若不從我,即當殞命。」乃厲聲罵曰:「爾等狗盜,乃欲污辱好人,今得速死,會我本志。」賊乃斬之,聞者莫不傷惜。



 鄒保英妻奚氏,不知何許人也。萬歲通天年,契丹賊李盡忠來寇平州。保英時任刺史,領兵討擊。既而城孤援寡,勢將欲陷。奚氏乃率家僮及城內女丁相助固守。賊
 退,所司以聞,優制封為誠節夫人。



 時有古玄應妻高氏,亦能固守飛狐縣城,卒免為突厥所陷。下詔曰:「頃屬默啜攻城,咸憂陷沒。丈夫固守,猶不能堅,婦人懷忠,不憚流矢;由茲感激,危城重安。如不褒升,何以獎勸!古玄應妻可封為徇忠縣君。」



 宋庭瑜妻魏氏,定州鼓城人,隋著作郎彥泉之後也。世為山東士族。父克己,有詞學,則天時為天官侍郎。魏氏善屬文。先天中,庭瑜自司農少卿左遷涪州別駕。魏氏
 隨夫之任,中路作《南征賦》以敘志,詞甚典美。開元中,庭瑜累遷慶州都督。初,中書令張說年少時為克己所重,魏氏恨其夫為外職,乃作書與說,敘亡父疇昔之事,並為庭瑜申理,乃錄《南征賦》寄說。說嘆曰:「曹大家《東征》之流也。」庭瑜尋轉廣州都督,道病卒。魏氏旬日亦殞,時人莫不傷之。



 崔繪妻盧氏,幽州範陽人也,為山東著姓。祖幼孫,常州刺史。父獻,有美名,則天時歷鸞臺侍郎、文昌左丞。天授
 中為酷吏來俊臣所陷,左遷西鄉令而卒。



 繪早終,盧既年少,諸兄常欲嫁之。盧輒稱病固辭。盧亡姊之夫李思沖,神龍初為工部侍郎,又求續親。時思沖當朝美職,諸兄不之拒。將婚之夕,方以告盧;盧又固辭不可,仍令人防其門。盧謂左右曰:「吾自誓久已定矣!」乃夜中出自竇中,奔歸崔氏,發面盡為糞穢所污。宗族見者皆為之垂淚。因出家為尼,諸尼欽其操行,皆尊事之。開元中,以老病而卒。



 奉天縣竇氏二女伯娘、仲娘,雖長於村野,而幼有志操。住與邠州接界。永泰中,草賊數千人,持兵刃入其村落行剽劫,聞二女有容色,姊年十九,妹年十六,藏於巖窟間。賊徒擬為逼辱,乃先曳伯娘出,行數十步,又曳仲娘出,賊相顧自慰。行臨深谷,伯娘曰:「我豈受賊污辱!」乃投之於穀。賊方驚駭,仲娘又投於穀。谷深數百尺,姊尋卒;仲娘腳折面破,血流被體,氣絕良久而蘇,賊義之而去。京兆尹第五琦感其貞烈,奏之;詔旌表門閭,長免丁役,
 二女葬事官給。京兆尹曹陸海著賦以美之。



 原武尉盧甫妻李氏,隴西成紀人也。父瀾,永泰元年春任蘄縣令。界內先有草賊二千餘人。瀾挺身入賊,結以誠信,賊並降附,百姓復業者二百餘家。時曹升任徐州刺史,知賊降,領兵掩襲。賊得脫後,入縣殺瀾。瀾將被殺,從父弟渤,詣賊救瀾,請代兄死。瀾又請留弟,弟兄爭死。瀾女盧甫妻,又泣請代父死。並為賊所害,宣慰使、吏部侍郎李季卿以節義聞。



 又有尉氏尉王泛妻裴氏,儀王
 傅巨卿之女也。素有容範,為賊所俘,賊逼之。裴曰:「吾衣冠之子,當死即死,終不茍全一命,受污於賊。」賊脅之以兵,逼之以罵,裴堅力抗之。賊怒,乃支解裴氏,至死不屈。季卿亦以狀跡聞。



 詔曰:「鄭州原武縣尉盧甫亡妻李氏、汴州尉氏縣尉王泛亡妻裴氏等,懿範傳家,柔明植性;頃因寇難,克彰義烈。或請代父死,表因心之孝;或誓逐夫亡,標難奪之節。宜膺贈律,俾光休美。李氏可贈孝昌縣君,裴氏可贈河東縣君,仍編入史冊。」瀾、渤亦贈官秩。



 鄒待徵妻簿氏。待徵,大歷中為常州江陰縣尉,其妻為海賊所掠。薄氏守節,出待征官誥於懷中,托付村人,使謂待徵曰:「義不受辱。」乃投江而死。賊退潮落,待徵於江岸得妻尸焉。江左文士,多著節婦文以紀之。



 李湍妻。湍,吳元濟之軍人也。元和中,淮南未平,湍心懷向順,乃急渡水殷河,東降烏重胤。其妻遂為賊束縛在樹,臠而食之,至死,叫其夫曰:「善事烏僕射。」觀者義之。至是,重胤以其事請列史冊。十三年,憲宗下詔從之。



 董昌齡母楊氏。昌齡常為泗州長史,世居於蔡。少孤,受訓於母。累事吳少誠、少陽,至元濟時,為吳房令。楊氏潛誡曰:「逆順之理,成敗可知,汝宜圖之。」昌齡志未果,元濟又署為郾城令。楊氏復誡曰:「逆黨欺天,天所不福。汝當速降,無以前敗為慮,無以老母為念。汝為忠臣,吾雖歿無恨矣!」及王師逼郾城,昌齡乃以城降,且說賊將鄧懷金歸款於李光顏。憲宗聞之喜,急召昌齡至闕,直授郾城令、兼監察御史,仍賜緋魚。昌齡泣謝曰:「此皆老母之
 訓。」憲宗嗟嘆良久。元濟囚楊氏,欲殺之,而止者數矣。蔡平,楊氏幸無恙。元和十五年,陳許節度使李遜疏楊氏之強明節義以聞,乃封北平郡太君。



 韋雍妻蕭氏。雍,故太子賓客。張弘靖鎮幽州日,奏授觀察判官,攝監察御史。時屬朝廷制置未備,幽州俗本兇悍,尤不樂文儒為主帥,賓佐習於常態,忿其變通,議論不密,卒然起亂。雍時家亦從劫,蕭氏聞難號呼,專執夫袂,左右格去,以死不從。及雍臨刃,蕭氏涕而告曰:「妾不
 幸年少,義不茍活;今日之事,願先就死!」執刃者斷其臂而殺雍,蕭氏詞氣不撓,雖兇悍圜視,無不嗟嘆。其夕,蕭氏亦卒。太和六年,節度使楊志誠表明其事,因降敕追封蘭陵縣君。



 衡方厚妻程氏。方厚,太和中任邕州都督府錄事參軍,為招討使董昌齡誣枉殺之。程氏力不能免,乃抑其哀,如非冤者。昌齡雅不疑慮,聽其歸葬。程氏故得以徒行詣闕,截耳於右銀臺門,告夫被殺之冤。御史臺鞫之,得
 實,諫官亦有章疏,故昌齡再受譴逐。



 程氏,開成元年降敕曰:「乃者吏為不道,虐殺爾夫,詣闕申冤,徒行萬里,崎嶇逼畏,濱於危亡。血誠即昭,幽憤果雪,雖古之烈婦,何以加焉!如聞孤孀無依,晝哭待盡,俾榮祿養,仍賜疏封。可封武昌縣君,仍賜一子九品正員官。」



 女道士李玄真,越王貞之玄孫。曾祖珍子,越王張六男也。先天中得罪,配流嶺南。玄真祖、父,皆亡歿於嶺外。雖曾經恩赦,而未昭雪。玄真進狀曰:「去開成三年十二月
 內得嶺南節度使盧鈞出俸錢接措,哀妾三代旅櫬暴露,各在一方,特與發遣,歸就大塋合祔。今護四喪,已到長樂旅店權下,未委故越王墳所在,伏乞天恩,允妾所奏,許歸大塋。妾年已六十三,孤露家貧,更無依倚。」詔曰:「越王事跡,國史著明,枉陷非辜,尋已洗雪。其珍子他事配流,數代漂零,不還京國。玄真弱女,孝節卓然,啟護四喪,綿歷萬里;況是近族,必可加恩。行路猶或嗟稱,朝廷固須恤助。委宗正寺、京兆府與訪越王墳墓報知。如不
 是陪陵,任祔塋次卜葬。其葬事仍令京兆府接措,必使備禮。葬畢,玄真如願住京城,便配咸宜觀安置。」



 孝女王和子者,徐州人。其父及兄為防秋卒,戍涇州。元和中,吐蕃寇邊,父兄戰死,無子,母先亡。和子時年十七,聞父兄歿於邊上,被發徒跣縗裳,獨往涇州。行丐取父兄之喪,歸徐營葬。手植松柏,剪發壞形,廬於墓所。節度使王智興以狀聞,詔旌表之。



 又大中五年,兗州瑕丘縣人鄭神佐女,年二十四,先許適馳雄牙官李玄慶。神佐
 亦為官健,戍慶州。時黨項叛,神佐戰死,其母先亡,無子。女以父戰歿邊城,無由得還,乃剪發壞形,自往慶州護父喪還,至瑕丘縣進賢鄉馬青村,與母合葬。便廬於墳所,手植松檜,誓不適人。節度使蕭椒以狀奏之曰:「伏以閭里之中,罕知禮教,女子之性,尤昧義方。鄭氏女痛結窮泉,哀深《陟岵》,投身沙磧,歸父遺骸,遠自邊陲,得還閭里。感《蓼莪》以積恨,守丘墓以誓心。克彰孝理之仁,足厲貞方之節。」詔旌表門閭。



 贊曰:政教隆平,男忠女貞。禮以自防,義不茍生。彤管有煒,蘭閨振聲。《關雎》合《雅》,始號文明。



\end{pinyinscope}