\article{卷一百七十八}

\begin{pinyinscope}

 ○李德裕



 李
 德裕,字文饒,趙郡人。祖棲筠,御史大夫。父吉甫,趙國忠公,元和初宰相。祖、父自有傳。德裕幼有壯志,苦心力學,尤精《西漢書》、《左氏春秋》。恥與諸生同鄉賦,不喜科試。
 年才及冠,志業大成。貞元中,以父譴逐蠻方,隨侍左右,不求仕進。元和初,以父再秉國鈞,避嫌不仕臺省,累闢諸府從事。十一年,張弘靖罷相,鎮太原,闢為掌書記。由大理評事得殿中侍御史。十四年府罷,從弘靖入朝,真拜監察御史。明年正月,穆宗即位,召入翰林,充學士。帝在東宮,素聞吉甫之名,既見德裕,尤重之。禁中書詔大手筆,多詔德裕草之。是月,召對思政殿,賜金紫之服。逾月,改屯田員外郎。



 穆宗不持政道,多所恩貸,戚
 里諸親,邪謀請謁;傳導中人之旨,與權臣往來,德裕嫉之。長慶元年正月,上疏論之曰:「伏見國朝故事,駙馬緣是親密,不合與朝廷要官往來。玄宗開元中,禁止尤切。訪聞近日駙馬輒至宰相及要官私第,此輩無他才伎可以延接,唯是洩漏禁密;交通中外,群情所知,以為甚弊。其朝官素是雜流,則不妨來往。若職在清列,豈可知聞?伏乞宣示宰臣,其駙馬諸親,今後公事即於中書見宰相,請不令詣私第。」上然之。尋轉考功郎中、知制誥。二年二月,
 轉中書舍人,學士如故。



 初,吉甫在相位時,牛僧孺、李宗閔應制舉直言極諫科。二人對詔,深詆時政之失,吉甫泣訴於上前。由是,考策官皆貶,事在《李宗閔傳》。元和初,用兵伐叛,始於杜黃裳誅蜀。吉甫經畫,欲定兩河,方欲出師而卒。繼之元衡、裴度。而韋貫之、李逢吉沮議,深以用兵為非。而韋、李相次罷相,故逢吉常怒吉甫、裴度。而德裕於元和時,久之不調,而逢吉、僧孺、宗閔以私怨恆排擯之。



 時德裕與李紳、元稹俱在翰林,以學識才名相
 類,情頗款密。而逢吉之黨深惡之。其月,罷學士,出為御史中丞。其元稹自禁中出,拜工部侍郎、平章事。三月,輩度自太原復輔政。是月,李逢吉亦自襄陽入朝,乃密賂纖人,構成於方獄。六月,元稹、裴度俱罷相。稹出為同州刺史。逢吉代裴度為門下侍郎、平章事。既得權位,銳意報怨。時德裕與牛僧孺俱有相望,逢吉欲引僧孺,懼紳與德裕禁中沮之;九月,出德裕為浙西觀察使,尋引僧孺同平章事。由是交怨愈深。



 潤州承王國清兵亂之後,
 前使竇易直傾府藏賞給,軍旋浸驕,財用殫竭。德裕儉於自奉,留州所得,盡以贍軍,雖施與不豐,將卒無怨。二年之後,賦輿復集。



 德裕壯年得位,銳於布政,凡舊俗之害民者,悉革其弊。江、嶺之間信巫祝,惑鬼怪,有父母兄弟厲疾者,舉室棄之而去。德裕欲變其風,擇鄉人之有識者,諭之以言,繩之以法,數年之間,弊風頓革。屬郡祠廟,按方志,前代名臣賢後則祠之。四郡之內,除淫祠一千一十所。又罷私邑山房一千四百六十,以清寇盜。人
 樂其政,優詔嘉之。



 昭愍皇帝童年纘歷,頗事奢靡。即位之年七月,詔浙西造銀盝子妝具二十事進內。德裕奏曰:



 臣百生多幸,獲遇昌期。受寄名籓,常憂曠職,孜孜夙夜,上報國恩。數年已來,災旱相繼,罄竭微慮,粗免流亡,物力之間,尚未完復。臣伏準今年三月三日赦文,常貢之外,不令進獻。此則陛下至聖至明,細微洞照,一恐聚斂之吏緣以成奸,一恐凋瘵之人不勝其弊。上弘儉約之德,下敷惻憫之心。萬國群氓,鼓舞未息。昨奉五月二
 十三日詔書,令訪茅山真隱,將欲師處謙守約之道,發務實去華之美。雖無人上塞丹詔,實率土已偃玄風,豈止微臣,獨懷抃賀。



 況進獻之事,臣子常心,雖有敕文不許,亦合竭力上貢。唯臣當道,素號富饒,近年已來,比舊即異。貞元中,李錡任觀察使日,職兼鹽鐵。百姓除隨貫出榷酒錢外,更置官酤,一兩重納榷,獲利至厚。又訪聞當時進奉,亦兼用鹽鐵羨餘,貢獻繁多,自後莫及。至薛蘋任觀察使時,又奏置榷酒。上供之外,頗有餘財,軍用之
 間,實為優足。自元和十四年七月三日敕,卻停榷酤。又準元和十五年五月七日赦文,諸州羨餘,不令送使,唯有留使錢五十萬貫。每年支用,猶欠十三萬貫不足,常須是事節儉,百計補填,經費之中,未免懸欠。至於綾紗等物,猶是本州所出,易於方圓。金銀不出當州,皆須外處回市。



 去二月中奉宣令進盝子,計用銀九千四百餘兩。其時貯備,都無二三百兩,乃諸頭收市,方獲制造上供。昨又奉宣旨,今進妝具二十件,計用銀一萬三千兩,
 金一百三十兩。尋令並合四節進奉金銀,造成兩具進納訖。今差人於淮南收買,旋到旋造,星夜不輟;雖力營求,深憂不迨。臣若因循不奏,則負陛下任使之恩;若分外誅求,又累陛下慈儉之德。伏乞陛下覽前件榷酤及諸州羨餘之目,則知臣軍用褊短,本末有由。伏料陛下見臣奏論,必賜詳悉,知臣竭愛君守事之節,盡納忠罄直之心。伏乞聖慈,宣令宰臣商議,何以遣臣上不違宣索,下不闕軍儲,不困疲人,不斂物怨,前後詔敕,並可遵
 承。輒冒宸嚴,不勝戰汗之至。



 時準赦不許進獻。逾月之後,徵貢之使,道路相繼。故德裕因訴而諷之。事奏,不報。



 又詔進可幅盤條繚綾一千匹,德裕又論曰:



 臣昨緣宣索,已具軍資歲計及近年物力聞奏,伏料聖慈,必垂省覽。又奉詔旨,令織定羅紗袍段及可幅盤條繚綾一千匹。伏讀詔書,倍增惶灼。



 臣伏見太宗朝,臺使至涼州,見名鷹諷李大亮獻之。大亮密表陳誠。太宗賜詔云:「使遣獻之,遂不曲順。」再三嘉嘆,載在史書。又玄宗命中使於
 江南採鵁鶄諸鳥,汴州刺史倪若水陳論,玄宗亦賜詔嘉納,其鳥即時皆放。又令皇甫詢於益州織半臂背子、琵琶捍撥、鏤牙合子等,蘇頲不奉詔書,輒自停織。太宗、玄宗皆不加罪,欣納所陳。臣竊以鵁鶄、鏤牙,至為微細,若水等尚以勞人損德,瀝款效忠。當聖祖之朝,有臣如此,豈明王之代,獨無其人?蓋有位者蔽而不言,必非陛下拒而不納。



 又伏睹四月二十三日德音云:「方、召侯伯有位之士,無或棄吾謂不可教。其有違道傷理,徇欲懷
 安,面刺廷攻,無有隱諱。」則是陛下納誨從善,道光祖宗,不盡忠規,過在臣下。況玄鵝天馬,椈豹盤絳,文彩珍奇,只合聖躬自服。今所織千匹,費用至多,在臣愚誠,亦所未諭。昔漢文帝衣弋綈之衣,元帝罷輕纖之服,仁德慈儉,至今稱之。伏乞陛下,近覽太宗、玄宗之容納,遠思漢文、孝元之恭己;以臣前表宣示群臣,酌臣當道物力所宜,更賜節減。則海隅蒼生,無不受賜。臣不勝懇切兢惶之至。



 優詔報之。其繚綾罷進。



 元和已來,累敕天下州府,
 不得私度僧尼。徐州節度使王智興聚貨無厭,以敬宗誕月,請於泗州置僧壇,度人資福,以邀厚利。江、淮之民,皆群黨渡淮。德裕奏論曰:



 「王智興於所屬泗州置僧尼戒壇,自去冬於江、淮已南,所在懸榜招置。江、淮自元和二年後,不敢私度。自聞泗州有壇,戶有三丁,必令一丁落發,意在規避王徭,影庇資產。自正月已來,落發者無算。臣今於蒜山渡點其過者,一日一百餘人,勘問唯十四人是舊日沙彌,餘是蘇、常百姓,亦無本州文憑,尋已
 勒還本貫。訪聞泗州置壇次第,凡僧徒到者,人納二緡,給牒即回,別無法事。若不特行禁止,比到誕節,計江、淮已南,失卻六十萬丁壯。此事非細,系於朝廷法度。」狀奏,即日詔徐州罷之。



 敬宗荒僻日甚,游幸無恆;疏遠賢能,暱比群小。坐朝月不二三度,大臣罕得進言。海內憂危,慮移宗社。德裕身居廉鎮,傾心王室,遣使獻《丹扆箴》六首,曰:「臣聞『心乎愛矣,遐不謂矣』,此古之賢人所以篤於事君者也。夫跡疏而言親者危,地遠而意忠者忤。然臣
 竊念拔自先聖,偏荷寵光,若不愛君以忠,則是上負靈鑒。臣頃事先朝,屬多陰沴,嘗獻《大明賦》以諷,頗蒙先朝嘉納。臣今日盡節明主,亦由是心。昔張敞之守遠郡,梅福之在遐徼,尚竭誠盡忠,不避尤悔。況臣嘗學舊史,頗知箴諷,雖在疏遠,猶思獻替。謹獻《丹扆箴》六首,仰塵睿鑒,伏積兢惶。」



 其《宵衣箴》曰:「先王聽政,昧爽以俟。雞鳴既盈,日出而視。伯禹大聖,寸陰為貴。光武至仁,反支不忌。無俾姜后,獨去簪珥。彤管記言,克念前志。」



 其《正服箴》曰:「
 聖人作服,法象可觀。雖在宴游,尚不懷安。汲黯莊色,能正不冠。楊阜毅然,亦譏縹紈。四時所御,各有其官。非此勿服,惟闢所難。」



 其《罷獻箴》曰:「漢文罷獻,詔還騄耳。鑾輅徐驅,焉用千里?厥後令王,亦能恭己。翟裘既焚,筒布則毀。道德為麗,慈仁為美。不過天道,斯為至理。」



 其《納誨箴》曰:「惟後納誨,以求厥中。從善如流,乃能成功。漢驁流湎,舉白浮鐘。魏睿侈汰,凌霄作宮。忠雖不忤,善亦不從。以規為瑱,是謂塞聰。」



 其《辯邪箴》曰:「居上處深,在察微萌。雖
 有讒慝,不能蔽明。漢之有昭,德過周成。上書知偽,照奸得情。燕、蓋既折,王猷洽平。百代之後,乃流淑聲。」



 其《防微箴》曰:「天子之孝,敬遵王度。安必思危,乃無遺慮。亂臣猖蹶,非可遽數。玄黃莫辨,觸瑟始僕。柏谷微行,豺豕塞路。睹貌獻飧,斯可誡懼。」



 帝手詔答曰:「卿文雅大臣,方隅重寄。表率諸部,肅清全吳。化洽行春,風澄坐嘯,眷言善政,想嘆在懷。卿之宗門,累著聲績,冠內廷者兩代,襲侯伯者六朝。果能激愛君之誠,喻詩人之旨。在遠而不忘忠
 告,諷上而常深慮微。博我以端躬,約予以循禮。三復規諫,累夕稱嗟。置之座隅,用比韋弘之益;銘諸心腑,何啻藥石之功?卿既以投誠,朕每懷開諫,茍有過舉,無忘密陳。山川既遐,睠屬何已,必當克己,以副乃誠。」



 德裕意在切諫,不欲斥言,托箴以盡意。《宵衣》,諷坐朝稀晚也;《正服》,諷服御乖異也;《罷獻》,諷徵求玩好也;《納誨》,諷侮棄讜言也;《辨邪》,諷信任群小也;《防微》,諷輕出游幸也。帝雖不能盡用其言,命學士韋處厚殷勤答詔,頗嘉納其心焉。德
 裕久留江介,心戀闕廷,因事寄情,望回聖獎。而逢吉當軸,枳棘其塗,竟不得內徙。



 寶歷二年,亳州言出聖水,飲之者愈疾。德裕奏曰:「臣訪聞此水,本因妖僧誑惑,狡計丐錢。數月已來,江南之人,奔走塞路。每三二十家,都顧一人取水。擬取之時,疾者斷食葷血,既飲之後,又二七日蔬飧,危疾之人,俟之愈病。其水鬥價三貫,而取者益之他水,沿路轉以市人,老疾飲之,多至危篤。昨點兩浙、福建百姓渡江者,日三五十人。臣於蒜山渡已加捉搦。
 若不絕其根本,終無益黎氓。昔吳時有聖水,宋、齊有聖火,事皆妖妄,古人所非。乞下本道觀察使令狐楚,速令填塞,以絕妖源。」從之。



 敬宗為兩街道士趙歸真說以神仙之術,宜訪求異人以師其道。僧惟貞、齊賢、正簡說以祠禱修福,以致長年。四人皆出入禁中,日進邪說。山人杜景先進狀,請於江南求訪異人。至浙西,言有隱士周息元,壽數百歲。帝即令高品、薛季棱往潤州迎之。仍詔德裕給公乘遣之。德裕因中使還,獻疏曰:



 臣聞道之高
 者,莫如廣成、玄元,人之聖者,莫若軒黃、孔子。昔軒黃問廣成子:理身之要,何以長久?對曰:「無視無聽,抱神以靜。形將自正,神必自清。無勞子形,無搖子精,乃可長生。慎守其一,以處其和。故我修身千二百歲矣,吾形未嘗衰。」又云:「得吾道者,上為皇而下為王。」玄元語孔子曰:「去子之驕氣與多欲,態色與淫志,是皆無益於子之身。吾所告子者是已。」故軒黃發謂天之嘆,孔子興猶龍之感。前聖於道,不其至乎?



 伏惟文武大聖廣孝皇帝陛下,用玄
 祖之訓,修軒黃之術;凝神閑館,物色異人;將以覿冰雪之姿,屈順風之請。恭惟聖感,必降真仙。若使廣成、玄元混跡而至,語陛下之道,授陛下之言,以臣度思,無出於此。臣所慮赴召者,必迂怪之士,茍合之徒,使物淖冰,以為小術,炫耀邪僻,蔽欺聰明。如文成、五利,一無可驗。臣所以三年之內,四奉詔書,未敢以一人塞詔,實有所懼。



 臣又聞前代帝王,雖好方士,未有服其藥者。故《漢書》稱黃金可成,以為飲食器則益壽。又高宗朝劉道合、玄宗
 朝孫甑生,皆成黃金,二祖竟不敢服。豈不以宗廟社稷之重,不可輕易!此事炳然載於國史。以臣微見,倘陛下睿慮精求,必致真隱,唯問保和之術,不求餌藥之功,縱使必成黃金,止可充於玩好。則九廟靈鑒,必當慰悅;寰海兆庶,誰不歡心?臣思竭愚衷,以裨玄化,無任兢憂之至。



 息元至京,帝館之於山亭,問以道術。自言識張果、葉靜能,詔寫真待詔李士昉問其形狀,圖之以進。息元山野常人,本無道學,言事誕妄,不近人情。及昭愍遇盜而
 殂,文宗放還江左。德裕深識守正,皆此類也。



 文宗即位,就加檢校禮部尚書。太和三年八月,召為兵部侍郎,裴度薦以為相。而吏部侍郎李宗閔有中人之助,是月拜平章事,懼德裕大用。九月,檢校禮部尚書,出為鄭滑節度使。德裕為逢吉所擯,在浙西八年。雖遠闕庭,每上章言事。文宗素知忠藎,採朝論征之。到未旬時,又為宗閔所逐,中懷於悒,無以自申。賴鄭覃侍講禁中,時稱其善;雖朋黨流言,帝乃心未已。宗閔尋引牛僧孺同知政事,
 二憾相結,凡德裕之善者,皆斥之於外。四年十月,以德裕檢校兵部尚書、成都尹、劍南西川節度副大使、知節度事、管內觀察處置、西山八國雲南招撫等使。裴度於宗閔有恩。度征淮西時,請宗閔為彰義觀察判官,自後名位日進。至是恨度援德裕,罷度相位,出為興元節度使,牛、李權赫於天下。



 西川承蠻寇剽虜之後,郭釗撫理無術,人不聊生。德裕乃復葺關防,繕完兵守。又遣人入南詔,求其所俘工匠,得僧道工巧四千餘人,復歸成都。
 五年九月,吐蕃維州守將悉怛謀請以城降。其州南界江陽,岷山連嶺而西,不知其極;北望隴山,積雪如玉;東望成都,若在井底。一面孤峰,三面臨江,是西蜀控吐蕃之要地。至德後,河、隴陷蕃,唯此州尚存。吐蕃利險要,將婦人嫁於此州閽者。二十年後,婦人生二子成長。及蕃兵攻城,二子內應,其州遂陷。吐蕃得之,號曰「無憂城」。貞元中,韋皋鎮蜀,經略西山八國,萬計取之不獲,至是悉怛謀遣人送款。德裕疑其詐,遣人送錦袍金帶與之,
 托云候取進止,悉怛謀乃盡率郡人歸成都。德裕乃發兵鎮守,因陳出攻之利害。時牛僧孺沮議,言新與吐蕃結盟,不宜敗約,語在《僧孺傳》。乃詔德裕卻送悉怛謀一部之人還維州,贊普得之,皆加虐刑。德裕六年復修邛峽關,移巂州於臺登城以捍蠻。



 德裕所歷征鎮,以政績聞。其在蜀也,西拒吐蕃,南平蠻、蜒。數年之內,夜犬不驚;瘡痏之民,粗以完復。會監軍王踐言入朝知樞密,嘗於上前言悉怛謀縛送以快戎心,絕歸降之義,上頗尤僧
 孺。其年冬,召德裕為兵部尚書。僧孺罷相,出為淮南節度使。七年二月,德裕以本官平章事,進封贊皇伯,食邑七百戶。六月,宗閔亦罷,德裕代為中書侍郎、集賢大學士。



 其年十二月,文宗暴風恙,不能言者月餘。八年正月十六日,始力疾御紫宸見百僚。宰臣退問安否,上嘆醫無名工者久之。由是王守澄進鄭注。初,注構宋申錫事,帝深惡之,欲令京兆尹杖殺之。至是以藥稍效,始善遇之。守澄復進李訓,善《易》。其年秋,上欲授訓諫官。德裕奏
 曰:「李訓小人,不可在陛下左右。頃年惡積,天下皆知;無故用之,必駭視聽。」上曰:「人誰無過,俟其悛改。朕以逢吉所托,不忍負言。」德裕曰:「聖人有改過之義。訓天性奸邪,無悛改之理。」上顧王涯曰:「商量別與一官。」遂授四門助教。制出,給事中鄭肅、韓佽封之不下。王涯召肅面喻令下。俄而鄭注亦自絳州至。訓、注惡德裕排己,九月十日,復召宗閔於興元,授中書侍郎、平章事,代德裕。出德裕為興元節度使。德裕中謝日,自陳戀闕,不願出籓,追敕
 守兵部尚書。宗閔奏制命已行,不宜自便,尋改檢校尚書左僕射、潤州刺史、鎮海軍節度、蘇常杭潤觀察等使,代王璠。



 德裕至鎮,奉詔安排宮人杜仲陽於道觀,與之供給。仲陽者,漳王養母,王得罪,放仲陽於潤州故也。九年三月,左丞王璠、戶部侍郎李漢進狀,論德裕在鎮,厚賂仲陽,結托漳王,圖為不軌。四月,帝於蓬萊殿召王涯、李固言、路隨、王璠、李漢、鄭注等,面證其事。璠、漢加誣構結,語甚切至。路隨奏曰:「德裕實不至此。誠如璠、漢之言,
 徼臣亦合得罪。」群論稍息。尋授德裕太子賓客,分懷東都。其月,又貶袁州長史。路隨坐證德裕,罷相,出鎮浙西。其年七月,宗閔坐救楊虞卿,貶處州。李漢坐黨宗閔,貶汾州。十一月,王璠與李訓造亂伏誅,而文宗深悟前事,知德裕為朋黨所誣。明年三月,授德裕銀青光祿大夫,量移滁州刺史。七月,遷太子賓客。十一月,檢校戶部尚書,復浙西觀察使。德裕凡三鎮浙西,前後十餘年。



 開成二年五月,授揚州大都督府長史、淮南節度副大使、知
 節度使事,代牛僧孺。初,僧孺聞德裕代己,乃以軍府事交付副使張鷺,即時入朝。時揚州府藏錢帛八十萬貫匹,及德裕至鎮,奏領得止四十萬,半為張鷺支用訖。僧孺上章訟其事,詔德裕重檢括,果如僧孺之數。德裕稱初到鎮疾病,為吏隱欺,請罰。詔釋之。補闕王績、魏謨,崔黨韋有翼、拾遺令狐綯書左僕射。五年正月,武宗即位。七月,召德裕於淮南。九
 月,授門下侍郎、同平章事。



 初,德裕父吉甫,年五十一出鎮淮南,五十四自淮南復相。今德裕鎮淮南,復入相,一如父之年,亦為異事。



 會昌元年,兼左僕射。開成末,回紇為黠戛斯所攻。戰敗,部族離散。烏介可汗奉太和公主南來。會昌二年二月,牙於塞上,遣使求助兵糧,收復本國,權借天德軍以安公主。時天德軍使田牟,請以沙陁、退渾諸部落兵擊之。上意未決,下百僚商議,議者多雲如牟之奏。德裕曰:「頃者國家艱難之際,回紇繼立大功。
 今國破家亡,竄投無所,自居塞上,未至侵淫。以窮來歸,遽行殺伐,非漢宣待呼韓邪之道也。不如聊濟資糧,徐觀其變。」宰相陳夷行曰:「此借寇兵而資盜糧,非計也,不如擊之便。」德裕曰:「田牟、韋仲平言沙陀、退渾並願擊賊,此緩急不可恃也。夫見利則進,遇敵則散,是雜虜之常態,必不肯為國家捍禦邊境。天德一城,戍兵寡弱,而欲與勁虜結讎,陷之必矣。不如以理恤之,俟其越軼,用兵為便。」帝以為然,許借米三萬石。



 俄而回紇宰相霡沒斯
 殺赤心宰相,以其眾來降。赤心部族又投幽州。烏介勢孤,而不與之米,其眾饑乏,漸近振武保大柵、杷頭峰,突入朔州州界。沙陁、退渾皆以其家保山險;雲州張獻節嬰城自固。虜大縱掠,卒無拒者。上憂之,與宰臣計事。德裕曰:「杷頭峰北,便是沙磧,彼中野戰,須用騎兵。若以步卒敵之,理難必勝。今烏介所恃者公主,如令勇將出奇奪得公主,虜自敗矣。」上然之,即令德裕草制處分代北諸軍,固關防,以出奇形勢授劉沔。沔令大將石雄急擊
 可汗於殺胡山;敗之,迎公主還宮,語在《石雄傳》。尋進位司空。



 三年二月,趙蕃奏黠戛斯攻安西、北庭都護府,宜出師應援。德裕奏曰:



 據地志,安西去京七千一百里,北庭去京五千二百里。承平時,向西路自河西、隴右出玉門關,迤邐是國家州縣,所在皆有重兵。其安西、北庭要兵,便於側近徵發。自艱難已後,河、隴盡陷吐蕃,若通安西、北庭,須取回紇路去。今回紇破滅,又不知的屬黠戛斯否。縱令救得,便須卻置都護,須以漢兵鎮守。每處不
 下萬人,萬人從何徵發?饋運取何道路?今天德、振武去京至近,兵力常苦不足。無事時貯糧不支得三年,朝廷力猶不及,況保七千里安西哉!臣所以謂縱令得之,實昔無用也。昔漢宣帝時,魏相請罷車師之田;漢元帝時,賈捐之請棄珠崖郡;國朝賢相狄仁傑亦請棄四鎮,立斛瑟羅為可汗,又請棄安東,卻立高氏。蓋不欲貪外虛內,耗竭生靈。此三臣者,當自有之時,尚欲棄之,以肥中國,況隔越萬里,安能救之哉!臣恐蕃戎多計,知國力不及,
 偽且許之,邀求中國金帛。陛下不可中悔,此則將實費以換虛事,即是滅一回紇而又生之,恐計非便。



 乃止。



 德裕又以太和五年,吐蕃維州守將以城降,為牛僧孺所沮,終失維州,奏論之曰:



 臣在先朝,出鎮西蜀。其時吐蕃維州首領悉怛謀,雖是雜虜,久樂皇風,將彼堅城,降臣本道。臣尋差兵馬,入據其城,飛章以聞,先帝驚嘆。其時與臣不足者,望風嫉臣,遽獻疑言,上罔宸聽,以為與吐蕃盟約,不可背之,必恐將此為辭,侵犯郊境。詔臣還卻
 此城,兼執送悉怛謀等,令彼自戮。復降中使,迫促送還。昔白起殺降,終於杜郵致禍;陳湯見徙,是為郅支報讎。感嘆前事,愧心終日。今者幸逢英主,忝備臺司,輒敢追論,伏希省察。



 且維州據高山絕頂,三面臨江,在戎虜平川之沖,是漢地入兵之路。初,河、隴盡沒,此州獨存。吐蕃潛將婦人嫁與此州門子。二十年後,兩男長成,竊開壘門,引兵夜入,因茲陷沒,號曰「無憂」。因並力於西邊,遂無虞於南路,憑凌近甸,宵旰累朝。貞元中,韋皋欲經略河
 湟,須以此城為始,盡銳萬旅,急攻累年。吐蕃愛惜既甚,遂遣舅論莽熱來援。雉堞高峻,臨沖難及於層霄;鳥逕屈盤,猛士多糜於礧石。莫展公輸之巧,空擒莽熱而還。



 及南蠻負恩,掃地驅劫。臣初到西蜀,眾心未安,外揚國威,中緝邊備。其維州執臣信令,乃送款與臣。臣告以須俟奏聞,所冀探其情偽。其悉怛謀尋率一城之兵眾,並州印甲仗,塞途相繼,空壁歸臣。臣大出牙兵,受其降禮。南蠻在列,莫敢仰視。況西山八國,隔在此州,比帶使名,
 都成虛語。諸羌久苦蕃中徵役,願作大國王人。自維州降後,皆云但得臣信牒帽子,便相率內屬。其蕃界合水、棲雞等城,既失險厄,自須抽歸,可減八處鎮兵,坐收千里舊地。臣見莫大之利,乃為恢復之基。繼具奏聞,請以酬賞。臣自與錦袍金帶,顒俟詔書。且吐蕃維州未降已前一年,猶圍魯州。以此言之,豈守盟約?況臣未嘗用兵攻取,彼自感化來降。又沮議之人,不知事實。犬戎遲鈍,土曠人稀,每欲乘秋犯邊,皆須數歲就食。臣得維州逾
 月,未有一使入疆。自此之後,方應破膽,豈有慮其後怨,鼓此游詞。



 臣受降之時,指天為誓,寧忍將三百餘人性命,棄信偷安。累表上陳,乞垂矜赦。答詔嚴切,竟令執還,加以體披桎梏,舁於竹畚。及將就路,冤叫呼天。將吏對臣,無不流涕。其部送者,使遭蕃帥譏誚,曰:「既已降彼,何須送來?」乃卻將此降人,戮於漢界之上,恣行殘害,用固攜離。乃至擲其嬰孩,承以槍槊。臣聞楚靈誘殺蠻子,《春秋》明譏;周文外送鄧叔,簡冊深鄙。況乎大國,負此異類,
 絕忠款之路,快兇虐之情,從古以來,未有此事。臣實痛悉怛謀舉城受酷,由臣陷此無辜,乞慰忠魂,特加褒贈。



 帝意傷之,尋賜贈官。



 其年,德裕兼守司徒。四月,澤潞節度使劉從諫卒,軍人以其侄稹擅總留後,三軍請降旄鉞。帝與宰臣議可否,德裕曰:「澤潞國家內地,不同河朔。前後命帥,皆用儒臣。頃者李抱真成立此軍,身歿之後,德宗尚不許繼襲,令李緘護喪歸洛。洎劉悟作鎮,長慶中頗亦自專。屬敬宗因循,遂許從諫繼襲。



 開成初,於長
 子屯軍,欲興晉陽之甲,以除君側;與鄭注、李訓交結至深,外托效忠,實懷窺伺。自疾病之初,便令劉稹管兵馬。若不加討伐,何以號令四方?若因循授之,則籓鎮相效,自茲威令去矣!」帝曰:「卿算用兵必克否?」對曰:「劉稹所恃者,河朔三鎮耳。但得魏鎮不與稹同,破之必矣。請遣重臣一人,傳達聖旨,言澤潞命帥,不同三鎮。自艱難已來,列聖皆許三鎮嗣襲,已成故事。今國家欲加兵誅稹,禁軍不欲出山東。其山東三州,委鎮魏出兵攻取。」上然之,
 乃令御史中丞李回使三鎮諭旨,賜魏鎮詔書云:「卿勿為子孫之謀,欲存輔車之勢。」何弘敬、王元逵承詔,聳然從命。初議出兵,朝官上疏相繼,請依從諫例,許之繼襲,而宰臣四人,亦有以出師非便者。德裕奏曰:「如師出無功,臣請自當罪戾,請不累李紳、讓夷等。及弘敬、元逵出兵,德裕又奏曰:「貞元、太和之間,朝廷伐叛,詔諸道會兵,才出界便費度支供餉,遲留逗撓,以困國力。或密與賊商量,取一縣一柵以為勝捷,所以師出無功。今請處分
 元逵、弘敬,只令收州,勿攻縣邑。」帝然之。及王宰、石雄進討,經年未拔澤潞。及弘敬、元逵收邢、洺、磁三州,稹黨遂離,以至平殄,皆如其算。



 時王師方討澤潞。三年十二月,太原橫水戍兵因移戍榆社。乃倒戈入太原城,逐節度使李石,推其都將楊弁為留後。武宗以賊稹未殄,又起太原之亂,心頗憂之。遣中使馬元貫往太原宣諭,覘其所為。元貫受楊弁賂,欲保祐之。四年正月,使還,奏曰:「楊弁兵馬極多,自牙門列隊至柳子,十五餘里,明光甲曳
 地。」德裕奏曰:「李石比以城內無兵,抽橫水兵一千五百人赴榆社,安能朝夕間便致十五里兵甲耶?」元貫曰:「晉人驍敢,盡可為兵,重賞招致耳。」德裕曰:「招召須財,昨橫水兵亂,止為欠絹一匹。李石無處得,楊弁從何致耶?又太原有一聯甲,並在行營,安致十五里明光耶?」元貫詞屈。德裕奏曰:「楊弁微賊,決不可恕!如國力不及,寧舍劉稹。」即時請降詔,令王逢起榆社軍,又令王元逵兵自土門入,會於太原。河東監軍呂義忠聞之,即日召榆社本
 道兵,誅楊弁以聞。



 自開成五年冬回紇至天德,至會昌四年八月平澤潞,首尾五年,其籌度機宜,選用將帥,軍中書詔,奏請雲合,起草指蹤,皆獨決於德裕,諸相無預焉。以功兼守太尉,進封衛國公,三千戶。五年,武宗上徽號後,累表乞骸,不許。德裕病月餘,堅請解機務,乃以本官平章事兼江陵尹、荊南節度使。數月追還,復知政事。宣宗即位,罷相,出為東都留守、東畿汝都防禦使。



 德裕特承武宗恩顧,委以樞衡。決策論兵,舉無遺悔,以身捍
 難,功流社稷。及昭肅棄天下,不逞之伍,咸害其功。白敏中、令狐綯,在會昌中德裕不以朋黨疑之,置之臺閣,顧待甚優。及德裕失勢,抵掌戟手,同謀斥逐,而崔鉉亦以會昌末罷相怨德裕。



 大中初,敏中復薦鉉在中書,乃相與掎摭構致,令其黨人李咸者,訟德裕輔政時陰事。乃罷德裕留守,以太子少保分司東都,時大中元年秋。尋再貶潮州司馬。敏中等又令前永寧縣尉吳汝納進狀,訟李紳鎮揚州時謬斷刑獄。明年冬,又貶潮州司戶。德
 裕既貶,大中二年,自洛陽水路經江、淮赴潮州。其年冬,至潮陽,又貶崖州司戶。至三年正月,方達珠崖郡。十二月卒,時年六十三。



 德裕以器業自負,特達不群。好著書為文,獎善嫉惡,雖位極臺輔,而讀書不輟。有劉三復者,長於章奏,尤奇待之。自德裕始鎮浙西,迄於淮甸,皆參佐賓筵。軍政之餘,與之吟詠終日。在長安私第,別構起草院。院有精思亭;每朝廷用兵,詔令制置,而獨處亭中,凝然握管,左右侍者無能預焉。東都於伊闕南置平泉
 別墅,清流翠,樹石幽奇。初未仕時,講學其中。及從官籓服,出將入相,三十年不復重游,而題寄歌詩,皆銘之於石。今有《花木記》、《歌詩篇錄》二石存焉。有文集二十卷。記述舊事,則有《次柳氏舊書》、《御臣要略》、《代叛志》、《獻替錄》行於世。



 初貶潮州,雖蒼黃顛沛之中,猶留心著述,雜序數十篇,號曰《窮愁志》。其《論冥數》曰:



 仲尼罕言命,不語神,非謂無也。欲人嚴三綱之道,奉五常之教,修天爵而致人爵,不欲信富貴於天命,委福祿於冥數。昔衛卜協於
 沙兵,為謚已久;秦塞屬於臨洮,名子不悟;朝歌未滅,而國流丹烏;白帝尚在,而漢斷素蛇。皆兆發於先,而符應於後,不可以智測也。周、孔與天地合德,與神明合契,將來之數,無所遁情。而狼跋于周,鳳衰於楚,豈親戚之義,不可去也,人倫之教,不可廢也。條侯之貴,鄧通之富,死於兵革可也,死於女室可也,唯不宜以餒終,此又不可以理得也。命偶時來,盜有名器者,謂禍福出於胸懷,榮枯生於口吻,沛然而安,溘然而笑,曾不知黃雀游於茂
 樹,而挾彈者在其後也。



 乙丑歲,予自荊楚,保厘東周,路出方城間,有隱者困於泥塗,不知其所如,謂方城長曰:「此官人居守後二年,南行萬里。」則知憾予者必因天譴,譖予者乃自鬼謀。雖抱至冤,不為恨。予嘗三遇異人,非卜祝之流,皆遁世者。初掌記北門,管涔隱者謂予曰:「君明年當在人君左右,為文翰之職,須值少主。」予聞之,愕然變色,隱者亦悔失言,避席求去。予問曰:「何為事少主?」對曰:「君與少主已有宿緣。」其年秋登朝,至明年正月,
 穆宗纘緒,召入禁苑。及為中丞,閩中隱者叩門請見,予下榻與語,曰:「時事非久,公不早去,冬必作相,禍將至矣。若亟請居外,則代公者受患。公後十年終當作相,自西而入。」是秋,出鎮吳門,時年三十六歲。經八稔,尋又仗鉞南燕。秋暮,有邑子於生引鄴郡道士至。才升階,未及命席,謂予曰:「公當為西南節制,孟冬望舒前,符節至矣。」三者皆與之協,不差歲月。自憲闈竟十年居相位,由西蜀而入,代予持憲者,俄亦竄逐。唯再謫南荒,未嘗有前知
 之士為予言之。豈禍患不可移者,神道所秘,莫得預聞。



 其自序如此。斯論可以警夫躁競者,故書於事末。



 德裕三子。燁,檢校祠部員外郎、汴宋亳觀察判官。大中二年,坐父貶象州立山尉。二子幼,從父歿於崖州。燁咸通初量移郴州郴縣尉,卒於桂陽。子延古。



 史臣曰:臣總角時,亟聞耆德言衛公故事。是時天子神武,明於聽斷;公亦以身犯難,酬特達之遇。言行計從,功成事遂,君臣之分,千載一時。觀其禁掖彌綸,巖廊啟奏,
 料敵制勝,襟靈獨斷,如由基命中,罔有虛發,實奇才也。語文章,則嚴、馬扶輪;論政事,則蕭、曹避席。罪其竊位,即太深文。所可議者,不能釋憾解仇,以德報怨,泯是非於度外,齊彼我於環中。與夫市井之徒,力戰錐刀之末,淪身瘴海,可為傷心。古所謂攫金都下,忽於市人,離婁不見於眉睫。才則才矣,語道則難。



 贊曰:公之智決,利若青萍。破虜誅叛,摧枯建瓴。功成北闕,骨葬南溟。嗚呼煙閣,誰上丹青?



\end{pinyinscope}