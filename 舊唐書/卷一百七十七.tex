\article{卷一百七十七}

\begin{pinyinscope}

 ○鄭覃弟朗
 陳夷行李紳吳汝納李回李玨李固言



 鄭覃,故相珣瑜之子。以父廕補弘文
 校
 理,歷拾遺、補闕、考功員外郎、刑部郎中。元和十四年二月,遷諫議大
 夫。憲宗用內官五人為京西北和糴使,覃上疏論罷。穆宗不恤政事,喜游宴;即位之始,吐蕃寇邊,覃與同職崔玄亮等廷奏曰:「陛下即位已來,宴樂過多,畋游無度。今蕃寇在境,緩急奏報,不知乘輿所在。臣等忝備諫官,不勝憂惕,伏願稍減游縱,留心政道。伏聞陛下晨夜暱狎倡優;近習之徒,賞賜太厚。凡金銀貨幣,皆出自生靈膏血,不可使無功之人,濫沾賜與。縱內藏有餘,亦乞用之有節,如邊上警急,即支用無闕。免令有司重斂百姓,實天
 下幸甚。」帝初不悅其言,顧宰相蕭俛曰:「此輩何人?」俛對曰:「諫官也。」帝意稍解,乃曰:「朕之過失,臣下盡規,忠也。」乃謂覃曰:「閣中奏事,殊不從容。今後有事面陳,朕與卿延英相見。」時久無閣中奏事,覃等抗論,人皆相賀。



 鎮冀節度使王承宗死,其弟承元聽朝旨,移授鄭滑節度。鎮之三軍留承元,以難不能赴鎮;承元乞重臣宣諭,乃以覃為宣諭使,起居舍人王璠副之。



 初,鎮卒辭語不遜,覃至宣詔,諭以大義,軍人釋然聽命。長慶元年十一月,轉給
 事中。四年,遷御史中丞,十一月,權知工部侍郎。寶歷元年,拜京兆尹。文宗即位,改左散騎常侍。三年,以本官充翰林侍講學士。四年四月,拜工部侍郎。



 覃長於經學,稽古守正,帝尤重之。覃從容奏曰:「經籍訛謬,博士相沿,難為改正。請召宿儒奧學,校定六籍;準後漢故事,勒石於太學,永代作則,以正其闕。」從之。



 五年,李宗閔、牛僧孺輔政。宗閔以覃與李德裕相善,薄之。時德裕自浙西入朝,復為閔、孺所排,出鎮蜀川。宗閔惡覃禁中言事,奏為工
 部尚書,罷侍講學士。文宗好經義,心頗思之。六年二月,復召為侍講學士。七年春,德裕作相。五月,以覃為御史大夫。文宗嘗於延英謂宰相曰:「殷侑通經學,為人頗似鄭覃。」宗閔曰:「覃、侑誠有經學,於議論不足聽覽。」李德裕對曰:「殷、鄭之言,他人不欲聞,唯陛下切欲聞之。」覃嘗嫉人朋黨,為宗閔所薄故也。八年,遷戶部尚書。其年,德裕罷相,宗閔復知政,與李訓、鄭注同排斥李德裕、李紳。二人貶黜,覃亦左授秘書監。九年六月,楊虞卿、李宗閔得
 罪長流,復以覃為刑部尚書。十月,遷尚書右僕射,兼判國子祭酒。訓、注伏誅,召覃入禁中草制敕,明日以本官同平章事,封滎陽郡公,食邑二千戶。



 覃雖精經義,不能為文。嫉進士浮華。開成初,奏禮部貢院宜罷進士科。初,紫宸對,上語及選士,覃曰:「南北朝多用文華,所以不治。士以才堪即用,何必文辭?」帝曰:「進士及第人已曾為州縣官者,方鎮奏署即可之,餘即否。」覃曰:「此科率多輕薄,不必盡用。」帝曰:「輕薄敦厚,色色有之,未必獨在進士。此
 科置已二百年,亦不可遽改。」覃曰:「亦不可過有崇樹。」帝嘗謂宰臣曰:「百司弛慢,要重條舉。」因指香爐曰:「此爐始亦華好,用之既久,乃無光彩。若不加飾,何由復初?」覃對曰:「丕變風俗,當考實效。自三十年已來,多不務實,取於顏情。如嵇、阮之流,不攝職事。」李石云:「此本因治平,人人無事,安逸所致。今之人俗亦慕王夷甫,恥不能及之。」上曰:「卿等輔朕,在振舉法度而已。」



 時太學勒石經,覃奏起居郎周墀、水部員外郎崔球、監察御史張次宗、禮部
 員外郎溫業等,校定《九經》文字,旋令上石。加門下侍郎、弘文館大學士、監修國史。上嘗於延英論古今詩句工拙,覃曰:「孔子所刪,三百篇是也。降此五言七言,辭非雅正,不足帝王賞詠。夫《詩》之《雅》、《頌》,皆下刺上所為,非上化下而作。王者採詩,以考風俗得失。仲尼刪定,以為世規。近代陳後主、隋煬帝皆能章句,不知王者大端,終有季年之失。章句小道,願陛下不取也。」覃以宰相兼判國子祭酒,奏太學置五經博士各一人,緣無職田,請依王府
 官例,賜祿粟。從之。又進《石壁九經》一百六十卷。



 其年,李固言復為宰相。固言與李宗閔、楊嗣復善,覃憎之。因起居郎闕,固言奏曰:「周敬復、崔球、張次宗等三人,皆堪此任。」覃曰:「崔球游宗閔之門,且赤墀下秉筆,為千古法,不可朋黨。如裴中孺、李讓夷,臣不敢有纖芥異論。」乃止。三年,楊嗣復自西川入拜平章事,與覃尤相矛盾;加之以固言、李玨,入對之際,是非蜂起。二月,覃進位太子太師。



 文宗以旱放系囚,出宮人劉好奴等五百餘人,送兩街
 寺觀,任歸親戚。紫宸對,李玨曰:「陛下放宮女數多,德邁千古。漢制,八月選人,晉武平吳,亦多採擇。仲尼所謂『未見好德如好色』。今陛下以為無益放之,微臣敢賀。」覃曰:「晉武帝以採擇之失,中原化為左衽;陛下以為殷鑒,放去攸宜。」其年十二月,三上章求罷,詔落太子太師,餘如故。仍三五日一入中書,商量政事。四年五月,罷相,守左僕射。



 武宗即位,李德裕用事,欲援為宰相。固以足疾不任朝謁。會昌二年,守司徒致仕,卒。



 子裔綽,以廕授渭南
 尉,直弘文館。



 覃少清苦貞退,不造次與人款狎。位至相國,所居未嘗增飾,才庇風雨。家無媵妾,人皆仰其素風。然嫉惡太過,多所不容,眾憚而惡之。



 覃弟朗、潛。



 朗,字有融。長慶元年,登進士甲科,再遷右拾遺。開成中,為起居郎。初,太和末風俗稍奢,文宗恭勤節儉,冀革其風。宰臣等言曰:「陛下節儉省用,風俗已移,長裾大袂,漸以減損。若更令戚屬絕其侈靡,不慮下不從教。」帝曰:「此事亦難戶曉,但去其泰甚,自以儉德化之。朕聞前時內庫唯二
 錦袍,飾以金鳥,一袍玄宗幸溫湯御之,一即與貴妃。當時貴重如此,如今奢靡,豈復貴之?料今富家往往皆有。左衛副使張元昌便用金唾壺,昨因李訓,已誅之矣。」時朗執筆螭頭下,宰臣退,上謂朗曰:「適所議論,卿記錄未?吾試觀之。」朗對曰:「臣執筆所記,便名為史。伏準故事,帝王不可取觀。昔太宗欲覽國史,諫議大夫硃子奢云:『史官所述,不隱善惡。或主非上智,飾非護失,見之則致怨,所以義不可觀。』又褚遂良曰:『今之起居郎,古之左右史
 也;記人君言行,善惡必書,庶幾不為非法,不聞帝王躬自觀史。』」帝曰:「適來所記,無可否臧,見亦何爽?」乃宣謂宰臣曰:「鄭朗引故事,不欲脫見起居注。夫人君之言,善惡必書。朕恐平常閑話,不關理體,垂諸將來,竊以為恥。異日臨朝,庶幾稍改,何妨一見,以誡醜言。」朗遂進之。朗轉考功郎中。四年,遷諫議大夫。



 會昌初,為給事中。出為華州刺史,入為御史中丞、戶部侍郎,判本司事。大中朝,出為定州刺史、義武軍節度、易定觀察、北平軍等使。尋遷
 檢校戶部尚書、汴州刺史、宣武軍節度、宋亳汴潁觀察等使。入為工部尚書,判度支。遷御史大夫,改禮部尚書。以本官同平章事,加中書侍郎、集賢殿大學士,修國史。



 大中十年,以疾辭位。進加檢校右僕射、守太子少師。十一年十月卒。詔曰:



 故通議大夫、檢校尚書右僕射、兼太子少師、上柱國、賜紫金魚袋鄭朗,植操端方,稟氣莊重;藹若瑞玉,淡如澄川。智略合乎蓍龜,誠信服於僚友。自膺寵寄,頗負全才,竭匪躬於諫垣,彰盡瘁於瑣闥。載踐
 方岳,亟登師壇。觀風推惠愛之心,訓士得撫循之術。政溢聞聽,念茲征還,位冠冬卿,職重邦計。經費有節,財用不虧。繄彼休功,明我推擇。爰嘉峭峻,俾總紀綱。公望益隆,典彞具舉;式諧注意,且沃深衷。俄參化源,以提政柄。三事仰清廉之節,百度見損益之能。近煦和風,遠浹膏雨。方俟坐鎮雅俗,表率庶官,頤養或乖,腠理生疾,屢陳章疏,乞遂退閑。既堅乃誠,式允其請。每圖懿績,唯冀有瘳。何竟至於彌留,而遽聞於捐代。閱奏興悼,臨軒載懷。
 將輟視朝之儀,兼列上公之秩。慰茲幽壞,期爾有知,可贈司空。



 潛,字無悶,亦登進士第。



 陳夷行,字周道,潁川人。祖忠,父邑。夷行,元和七年登進士第,累闢使府。寶歷末,由侍御史改虞部員外郎,皆分務東都。太和三年,入為起居郎、史館修撰,預修《憲宗實錄》。四年獻上,轉司封員外郎。五年,遷吏部郎中。四月,召充翰林學士。八年,兼充皇太子侍讀,詔五日一度入長生院侍太子講經。上召對,面賜緋衣牙笏,遷諫議大夫、
 知制誥,餘職如故。九年八月,改太常少卿,知制誥、學士侍講如故。



 開成二年四月,以本官同平章事。三年,楊嗣復、李玨繼入輔政。夷行介特,素惡其所為,每上前議政,語侵嗣復,遂至往復。性不能堪,上表稱足疾辭位;不許,詔中使就第宣勞。七月,以王彥威為忠武節度使,史孝章為邠寧節度使,皆嗣復擬議。因延英對,上問夷行曰:「昨除二鎮,當否?」夷行對曰:「但出自聖心即當。」楊嗣復曰:「若出自聖心當,即人情皆愜。如事或過當,臣下安得無
 言?」帝曰:「誠如此,朕固無私也。」夷行曰:「自三數年來,奸臣竊權,陛下不可倒持太阿,授人金尊柄。」嗣復曰:「齊桓用管仲於讎虜,豈有太阿之慮乎?」上不悅。



 仙韶院樂官尉遲璋授王府率,右拾遺竇洵直當衙論曰:「伶人自有本色官,不合授之清秩。」鄭覃曰:「此小事,何足當衙論列!王府率是六品雜官,謂之清秩,與洵直得否?此近名也。」嗣復曰:「嘗聞洵直幽,今當衙論一樂官,幽則有之,亦不足怪。」夷行曰:「諫官當衙,只合論宰相得失,不合論樂官。然業
 已陳論,須與處置。今後樂人每七八年與轉一官,不然,則加手力課三數人。」帝曰:「別與一官。」乃授光州長史,賜洵直絹百疋。夷行尋轉門下侍郎。



 上紫宸議政,因曰:「天寶中政事,實不甚佳。當時姚、宋在否?」李玨曰:「姚亡而宋罷。」玨因言:「人君明哲,終始尤難。玄宗嘗云:『自即位已來,未嘗殺一不辜。』而任林甫陷害破人家族,不亦惑乎?」夷行曰:「陛下不可移權與人。」嗣復曰:「夷行之言容易,且太宗用房玄齡十六年、魏徵十五年,何嘗失道?臣以為用
 房、魏多時不為不理,用邪佞一日便足。」夷行之言,皆指嗣復專權。



 文宗用郭薳為坊州刺史,右拾遺,宋邧論列,以為不可。既而薳坐贓。帝謂宰相曰:「宋邧論事可嘉,邧授官來幾時?」嗣復曰:「去年。」因曰:「諫官論事,陛下但記其姓名,稍加優獎。如不當,亦須令知。」夷行曰:「諫官論事,是其本職。若論一事即加一官,則官何由得,不免有情。」帝曰:「情固不免,理平之時,亦不可免。」上竟以夷行議論太過,恩禮漸薄。尋罷知政事,守吏部尚書。



 四年九月,檢校
 禮部尚書,出為華州刺史。五年,武宗即位,李德裕秉政。七月自華召入,復為中書侍郎、平章事。



 會昌三年十一月,檢校司空、平章事、河中尹、河中晉絳節度使。卒,贈司徒。



 弟玄錫、夷實,皆進士擢第。玄賜又制策登科。



 李紳,字公垂,潤州無錫人。本山東著姓。高祖敬玄,則天朝中書令,封趙國文憲公,自有傳。祖守一,成都郫縣令。父晤,歷金壇、烏程、晉陵三縣令,因家無錫。



 紳六歲而孤,母盧氏教以經義。紳形狀眇小而精悍,能為歌詩。鄉賦
 之年,諷誦多在人口。元和初,登進士第,釋褐國子助教,非其好也。東歸金陵,觀察使李錡愛其才,闢為從事。紳以錡所為專恣,不受其書幣;錡怒,將殺紳,遁而獲免。錡誅,朝廷嘉之,召拜右拾遺。



 歲餘,穆宗召為翰林學士,與李德裕、元稹同在禁署,時稱「三俊」,情意相善。尋轉右補闕。長慶元年三月,改司勛員外郎、知制誥。二年二月,超拜中書舍人,內職如故。



 俄而稹作相,尋為李逢吉教人告稹陰事;稹罷相,出為同州刺史。時德裕與牛僧孺俱
 有相望,德裕恩顧稍深。逢吉欲用僧孺,懼紳與德裕沮於禁中。二年九月,出德裕為浙西觀察使,乃用僧孺為平章事,以紳為御史中丞,冀離內職,易掎摭而逐之。乃以吏部侍郎韓愈為京兆尹,兼御史大夫,放臺參。知紳剛褊,必與韓愈忿爭。制出,紳果移牒往來,論臺府事體。而愈復性訐,言辭不遜,大喧物議,由是兩罷之。愈改兵部侍郎,紳為江西觀察使。天子待紳素厚,不悟逢吉之嫁禍,為其心希外任,乃令中使就第宣勞,賜之玉帶。紳
 對中使泣訴其事,言為逢吉所排,戀闕之情無已。及中謝日,面自陳訴,帝方省悟,乃改授戶部侍郎。



 中尉王守澄用事,逢吉令門生故吏結托守澄為援以傾紳,晝夜計畫。會紳族子虞,文學知名,隱居華陽,自言不樂仕進,時來京師省紳。虞與從伯耆、進士程昔範,皆依紳。及耆拜左拾遺,虞在華陽寓書與耆求薦,書誤達於紳。紳以其進退二三,以書誚之。虞大怨望。及來京師,盡以紳嘗所密話言逢吉奸邪附會之語告逢吉。逢吉大怒,問計
 於門人張又新、李續之,咸曰:「搢紳皆自惜毛羽,孰肯為相公搏擊!須得非常奇士出死力者。有前鄧州司倉劉棲楚者,嘗為吏。鎮州王承宗以事繩之。棲楚以首觸地固爭,而承宗竟不能奪,其果銳如此。若相公取之為諫官,令伺紳之失,一旦於上前暴揚其過,恩寵必替。事茍不行,過在棲楚,亦不足惜也。」逢吉乃用李虞、程昔範、劉棲楚,皆擢為拾遺,以伺紳隙。



 俄而穆宗晏駕。敬宗初即位,逢吉快紳失勢,慮嗣君復用之。張又新等謀逐紳。會
 荊州刺史蘇遇入朝,遇能決陰事,眾問計於遇。遇曰:「上聽政後,當開延英,必有次對,官欲拔本塞源,先以次對為慮,餘不足恃。」群黨深然之。逢吉乃以遇為左常侍。王守澄每從容謂敬宗曰:「陛下登九五,逢吉之助也。先朝初定儲貳,唯臣備知。時翰林學士杜元穎、李紳勸立深王,而逢吉固請立陛下,而李續之、李虞繼獻章疏。」帝雖沖年,亦疑其事。會逢吉進擬,進李紳在內署時,嘗不利於陛下,請行貶逐。帝初即位,方倚大臣,不能自執,乃貶
 紳端州司馬。貶制既行,百僚中書賀宰相,唯右拾遺吳思不賀。逢吉怒,改為殿中侍御史,充入吐蕃告哀使。



 紳之貶也,正人腹誹,無敢有言。唯翰林學士韋處厚上疏,極言逢吉奸邪,誣摭紳罪,語在《處厚傳》。天子亦稍開悟。會禁中檢尋舊書,得穆宗時封書一篋。發之,得裴度、杜元穎與紳三人所獻疏,請立敬宗為太子。帝感悟興嘆,悉命焚逢吉黨所上謗書,由是讒言稍息,紳黨得保全。



 及寶歷改元大赦,逢吉定赦書節文,不欲紳量移,但云
 左降官已經量移者與量移,不言左降官與量移。韋處厚復上疏論之,語在《處厚傳》。帝特追赦書,添節文云「左降官與量移」,紳方移為江州長史。再遷太子賓客,分司東都。



 太和七年,李德裕作相。七月,檢校左常侍、越州刺史、浙東觀察使。九年,李訓用事,李宗閔復相,與李訓、鄭注連衡排擯德裕罷相,紳與德裕俱以太子賓客分司。



 開成元年,鄭覃輔政,起德裕為浙西觀察使,紳為河南尹。六月,檢校戶部尚書、汴州刺史、宣武節度、宋亳汴潁
 觀察等使。二年,夏秋旱,大蝗,獨不入汴、宋之境,詔書褒美。又於州置利潤樓店。四年,就加檢校兵部尚書。



 武宗即位,加檢校尚書右僕射、揚州大都督府長史,知淮南節度大使事。會昌元年,入為兵部侍郎、同平章事,改中書侍郎,累遷守右僕射、門下侍郎、監修國史、上柱國、趙國公,食邑二千戶。四年,暴中風恙,足緩不任朝謁,拜章求罷。十一月,守僕射、平章事,出為淮南節度使。六年,卒。



 紳始以文藝節操進用,受顧禁中。後為朋黨所擠,濱於
 禍患。賴正人匡救,得以功名始終。歿後,宣宗即位,李德裕失勢罷相,歸洛陽;而宗閔、嗣復之黨崔鉉、白敏中、令狐綯欲置德裕深罪。大中初,教人發紳鎮揚州時舊事,以傾德裕。



 初,會昌五年,揚州江都縣尉吳湘坐贓下獄,準法當死,具事上聞。諫官疑其冤,論之。遣御史崔元藻覆推,與揚州所奏多同,湘竟伏法。及德裕罷相,群怨方構,湘兄進士汝納,詣闕訴冤,言紳在淮南恃德裕之勢,枉殺臣弟。德裕既貶,紳亦追削三任官告。



 吳汝納者,澧州人,故韶州刺史武陵兄之子。武陵進士登第,有史學,與劉軻並以史才直史館。武陵撰《十三代史駁議》二十卷。自尚書員外郎出為忠州刺史,改韶州。坐贓貶潘州司戶卒。



 汝納亦進士擢第,以季父贓罪,久之不調。會昌中,為河南府永寧縣尉。初,武陵坐贓時,李德裕作相,貶之。故汝納以不調挾怨,而附宗閔、嗣復之黨,同作謗言。會汝納弟湘為江都尉,為部人所訟贓罪,兼娶百姓顏悅女為妻,有逾格律。李紳令觀察判官魏
 鉶鞫之,贓狀明白,伏法。湘妻顏,顏繼母焦,皆笞而釋之。仍令江都令張弘思以船監送湘妻顏及兒女送澧州。



 及揚州上具獄,物議以德裕素憎吳氏,疑李紳織成其罪。諫官論之,乃差御史崔元藻為制使,覆吳湘獄。,據款伏妄破程糧錢,計贓準法。其恃官娶百姓顏悅女為妻,則稱悅是前青州衙推。悅先娶王氏,是衣冠女,非繼室焦所生,與揚州案小有不同。德裕以元藻無定奪,奏貶崖州司戶。及汝納進狀,追元藻覆問。元藻既恨德裕,陰
 為崔鉉、白敏中、令狐綯所利誘,即言湘雖坐贓,罪不至死。又云,顏悅實非百姓,此獄是鄭亞首唱,元壽協李恪鍛成,李回便奏。遂下三司詳鞫。故德裕再貶,李回、鄭亞等皆竄逐。吳汝納、崔元藻為崔、白、令狐所獎,數年並至顯官。



 李回,字昭度,宗室郇王禕之後。父如仙。回本名躔,以避武宗廟諱。長慶初,進士擢第,又登賢良方正制科。釋褐滑臺從事,揚州掌書記,得監察御史。入為京兆府戶曹,
 轉司錄參軍。合朝為正補闕、起居郎,尤為宰相李德裕所知。回強幹有吏才,遇事通敏,官曹無不理。授職方員外郎,判戶部案,歷吏部員外郎,判南曹。以刑部員外郎知臺雜,賜緋。開成初,以庫部郎中知制誥,拜中書舍人,賜金紫服。武宗即位,拜工部侍郎,轉戶部侍郎,判本司事。三年,兼御史中丞。



 會昌三年,劉稹據潞州,邀求旄鉞,朝議不允,加兵問罪。武宗懼稹陰附河朔三鎮,以沮王師,乃命回奉使河朔。魏博何弘敬、鎮冀王元逵皆具櫜
 鞬郊迎。回喻以朝旨,言澤潞密邇王畿,不同河北,自艱難已來,唯魏、鎮兩籓,列聖皆許襲,而稹無功,欲效河朔故事,理即太悖。聖上但以山東三郡,境連魏、鎮,用軍便近,王師不欲輕出山東,請魏、鎮兩籓只收山東三郡。弘敬、元逵俯僂從命。幽州張仲武與太原劉沔攻回鶻。時兩人不協,朝廷方用兵,不欲籓帥不和。回至幽州,喻以和協之旨,仲武欣然釋憾。乃移劉沔鎮滑臺,命仲武領太原軍攻潞。賊平,以本官同平章事,累加中書侍郎,轉
 門下,歷戶、吏二尚書。



 武宗崩,回充山陵使,祔廟竟,出為成都尹、劍南西川節度。大中元年冬,坐與李德裕親善,改潭州刺史、湖南觀察使,再貶撫州刺史。白敏中、令狐綯罷相,入朝為兵部尚書,復出為成都尹、劍南西川節度使。卒,贈司徒,謚曰文懿。



 李玨,字待價,趙郡人。父仲朝。玨進士擢弟,又登書判拔萃科,累官至右拾遺。穆宗荒於酒色,才終易月之制,即與勛臣飲宴。玨與同列上疏論之曰:



 臣聞人臣之節,本
 於忠盡,茍有所見,即宜上陳。況為陛下諫官,食陛下厚祿,豈敢腹誹巷議,辜負恩榮?臣等聞諸道路,不知信否,皆云有詔追李光顏、李醖,欲於重陽節日,合宴群臣。倘誠有之,乃陛下念群臣,敷惠澤之慈旨也。然元朔未改,園陵尚新。雖陛下執易月之期,俯從人欲;而禮經著三年之制,猶服心喪。今遵同軌之會,適去於中邦;告遠夷之使,未復其來命。遏密弛禁,蓋為齊人,合宴內廷,事將未可。夫明王之舉動,為天下法;王言既降,其出如綸。茍
 玷皇猷,徒章直諫,臣等是以昧死上聞。且光顏、李愬,久立忠勞,今方盛秋,務拓邊境。如或召見,詔以謀猷,褒其宿勛,付以疆事,則與歌鐘合宴,酒食邀歡,不得同年而語也。陛下自纘嗣以來,發號施令,無非孝理因心,形於詔敕,固以感動於人倫。更在敬慎威儀,保持聖德而已。



 上雖不用其言,慰勞遣之。



 長慶元年,鹽鐵使王播增茶稅,初稅一百,增之五十,玨上疏論之曰:



 榷率救弊,起自干戈,天下無事,即宜蠲省。況稅茶之事,尤出近年,在貞
 元元年中,不得不爾。今四海鏡清,八方砥平,厚斂於人,殊傷國體。其不可一也。



 茶為食物,無異米鹽,於人所資,遠近同俗。既袪竭乏,難舍斯須,田閭之間,嗜好尤切。今增稅既重,時估必增,流弊於民,先及貧弱。其不可二也。



 且山澤之饒,出無定數,量斤論稅,所冀售多。價高則市者稀,價賤則市者廣,歲終上計,其利幾何?未見阜財,徒聞斂怨。其不可三也。



 臣不敢遠征故事,直以目前所見陳之。伏望暫留聰明,稍垂念慮,特追成命,更賜商量。陛
 下即位之初,已懲聚斂,外官押貫,旋有詔停,洋洋德音,千古不朽。今若榷茶加稅,頗失人情。臣忝諫司,不敢緘默。



 時禁中造百尺樓,國計不充。王播希恩增稅,奉帝嗜欲,疏奏不省。遷吏部員外郎,轉司勛員外郎、知制誥。



 太和五年,李宗閔、牛僧孺在相,與玨親厚,改度支郎中、知制誥,遂入翰林充學士。七年三月,正拜中書舍人。九年五月,轉戶部侍郎充職。七月,宗閔得罪,玨坐累,出為江州刺史。開成元年四月,以太子賓客分司東都,遷河南
 尹。二年五月,李固言入相,召玨復為戶部侍郎,判本司事。三年,楊嗣復輔政,薦玨以本官同平章事。玨與固言、嗣復相善,自固言得位,相繼援引;居大政,以傾鄭覃、陳夷行、李德裕三人。凡有奏議,必以朋黨為謀,屢為覃所廷折之。玨自朝議郎進階正議大夫,其年十二月,上疏求罷,不許。



 四年三月,文宗謂宰臣曰:「朕在位十四年,屬天下無事,雖未至理,亦少有如今日之無事也。」玨對曰:「邦國安危,亦知人之身。當四體和平之時,長宜調適,以
 順寒暄之節。如恃安自忽,則疾患旋生。朝廷當無事之時,思省闕失而補之,則禍難不作矣。」



 文宗以杜悰領度支稱職,欲加戶部尚書,因紫宸言之。陳夷行曰:「一切恩權,合歸君上。陛下自看可否?」玨對曰:「太宗用宰臣,天下事皆先平章,謂之平章事。代天理物,上下無疑,所以致太平者也。若拜一官,命一職,事事皆決於君上,即焉用彼相?昔隋文帝一切自勞心力,臣下發論則疑,凡臣下用之則宰相,不用是常僚,豈可自保?陛下常語臣云:『竇
 易直勸我,宰相進擬,但五人留三人、兩人,勾一人。渠即合勸我擇宰相,不合勸我疑宰相。』」帝曰:「易直此言甚鄙。」又曰:「韋處厚作相,三日薦六度師,亦大可怪。」玨曰:「處厚淫於奉佛,不悟其是非也。」



 其年五月,上謂宰臣曰:「貞元政事,初年至好。」玨曰:「德宗中年好貨,方鎮進奉,即加恩澤。租賦出自百姓,更令貪吏剝削,聚貨以希恩,理道故不可也。」上曰:「人君聚斂,猶自不可。但輕賦節用可也。」玨又曰:「貞觀中,房、杜、王、魏啟告文皇,意只在此,請不易初
 心。自古好事,克終實難。」上曰:「朕心終不改也。」尋封贊皇男,食邑三百戶。



 武宗即位之年九月,與楊嗣復俱罷相,出為桂州刺史、桂管觀察使。三年,長流驩州。大中二年,崔鉉、白敏中逐李德裕,徵入朝為戶部尚書。出為河陽節度使。入為吏部尚書,累遷金紫光祿大夫、檢校尚書右僕射、揚州大都督府長史、淮南節度使、上柱國、贊皇郡開國公、食邑一千五百戶。大中七年卒,贈司空。



 李固言,趙郡人。祖並,父現。固言,元和七年登進士甲科。
 太和初,累官至賀部郎中、知臺雜。四年,李宗閔作相,用為給事中。五年,宋申錫為王守澄誣陷,固言與同列伏閣論之。將作監王堪修奉太廟弛慢,罰俸,仍改官為太子賓客。制出,固言封還曰:「東宮調護之地,不可令弛慢被罰之人處之。」改為均王傅。六年,遷工部侍郎。七年四月,轉尚書左丞,奉詔定左右僕射上事儀注。八年,李德裕輔政,出為華州刺史。



 其年十月,宗閔復入,召拜吏部侍郎。九年五月,遷御史大夫。六月,宗閔得罪,固言代為
 門下侍郎、平章事,尋加崇文館大學士。時李訓、鄭注用事,自欲竊輔相之權。宗閔既逐,外示公體,爰立固言,其實惡與宗閔朋黨。九月,以兵部尚書出為興元節度使。李訓自代固言為平章事。訓、注誅,文宗思其讜正,開成元年四月,復召為平章事,判戶部事。



 二年,君臣上徽號,上紫宸言曰:「中外上章,請加徽號。朕思理道猶鬱,實愧岳牧之請。如聞州郡甚有無政處?」固言曰:「人言鄧州王堪衰老,隋州鄭襄無政。」帝曰:「堪是貞元時御史,只有此
 一人。」鄭覃曰:「臣以王堪舊人,舉為刺史。鄭襄比來守官,亦無敗事。若言外郡不理,何止二人?」帝曰:「濟濟多士,文王以寧。德宗時,班行多閑員,豈時乏才耶?」李石對曰:「十室之邑,必有忠信。安有大國無人?蓋貞元中仕進路塞,所以有才之人或托跡他所,此乃不敘進人才之過也。」固言曰:「求才之道,有人保任,便宜獎用。隨其稱職與否升黜之。」上曰:「宰相薦人,莫計親疏。竇易直作相,未嘗論用親情。若己非相才,自宜引退。若是公舉,親亦何嫌?人
 鮮全才,但用其所長爾。」



 尋進階金紫,判戶部事。其年十月,以門下侍郎平章事出為成都尹、劍南西川節度使,代楊嗣復。上表讓門下侍郎,乃檢校左僕射。會昌初入朝,歷兵、戶二部尚書。宣宗即位,累授檢校司徒、東都留守、東畿汝都防禦使。大中末,以太常卿孫簡代之,拜太子太傅,分司東都,卒。



 史臣曰:陳、鄭諸公,章疏議論,綽有端士之風。天子待以賢能,付之以鼎職。延英獻納,罕聞康濟之謨;文陛敷揚,
 莫副具瞻之望。加以互生傾奪,競起愛憎。惟回奉使命而喻籓臣,救危邦而除宿憾。況昭獻文章可以為世範,德行可以為人師,有啟、誦之上才,非桓、靈之失道,詎可不思己過,只務面欺。輔弼之宜,安可垂訓?若俾韓非之言進矣,子輩安可逃乎?土運之衰,斯為魍魎,悲夫!



 贊曰:愛而知惡,憎不忘善。平心救非,可居鼎鉉。吠聲濟惡,結黨專朝。謀身壞國。何名燮調?



\end{pinyinscope}