\article{卷八十一}

\begin{pinyinscope}

 ○韋挺
 子待價弟萬石楊
 纂族子弘禮弘武武子元亨元禧元禕劉德威子審禮孫易從審禮從弟延嗣閻立德弟立本柳亨族子範兄子奭亨孫渙澤崔義玄子神慶



 韋挺,雍州萬年人,隋民部尚書沖子也。少與隱太子相
 善,及高祖平京城,引為隴西公府祭酒。武德中,累遷太子左衛驃騎,檢校左率。太子遇之甚厚,宮臣罕與為比。七年,高祖避暑仁智宮,會有上書言事者,稱太子與宮臣潛構異端。時慶州刺史楊文乾構逆伏誅,辭涉東宮,挺與杜淹、王珪等並坐流於越巂。及太宗在東宮,徵拜主爵郎中。貞觀初,王珪數舉之,由是遷尚書右丞。俄授吏部侍郎,轉黃門侍郎,進拜御史大夫,封扶陽縣男。太宗以挺女為齊王祐妃。常與房玄齡、王珪、魏徵、戴胄等
 俱承顧問,議以政事。又與高士廉、令狐德棻等同修《氏族志》,累承賞賚。太宗嘗謂挺曰:「卿之任御史大夫,獨朕意耳,左右大臣無為卿地者,卿勉之哉!」挺陳謝曰:「臣駑下,不足以辱陛下高位。且臣非勛非舊,而超處籓邸故僚之上,臣願後之,以勸立功者。」太宗不許。尋改授銀青光祿大夫,行黃門侍郎,兼魏王泰府事。時泰有寵,太子承乾多過失,太宗微有廢立之意。中書侍郎杜正倫以漏洩禁中語左遷,時挺亦預泰事,太宗謂曰:「朕已罪正
 倫,不忍更置卿於法。」特原之。尋遷太常卿。初,挺為大夫時,馬周為監察御史,挺以周寒士,殊不禮之。至是,周為中書令,太宗嘗復欲用挺在門下,周密陳挺傲狠,非宰相器,遂寢。十九年,將有事於遼東,擇人運糧,周又奏挺才堪粗使,太宗從之。挺以父在隋為營州總管,有經略高麗遺文。因此奏之。太宗甚悅,謂挺曰:「幽州以北,遼水二千餘里無州縣,軍行資糧無所取給,卿宜為此使。但得軍用不乏,功不細矣」。以人部侍郎崔仁師為副使,任
 自擇文武官四品十人為子使,以幽、易、平三州驍勇二百人,官馬二百匹為從。詔河北諸州皆取挺節度,許以便宜行事。太宗親解貂裘及中廄馬二匹賜之。挺至幽州,令燕州司馬王安德巡渠通塞。先出幽州庫物,市木造船,運米而進。自桑乾河下至盧思臺,去幽州八百里,逢安德還曰:「自此之外,漕渠壅塞。」挺以北方寒雪,不可更進,遂下米於臺側權貯之,待開歲發春,方事轉運,度大兵至,軍糧必足,仍馳以聞。太宗不悅,詔挺曰:「兵尚拙
 速,不貴工遲。朕欲十九年春大舉,今言二十年運漕,甚無謂也。」乃遣繁畤令韋懷質往挺所支度軍糧,檢覆渠水。懷質還奏曰:「挺不先視漕渠,輒集工匠造船,運米即下。至盧思臺,方知渠閉,欲進不得,還復水涸,乃便貯之無通平夷之區。又挺在幽州,日致飲會,實乖至公。陛下明年出師,以臣度之,恐未符聖策。」太宗大怒,令將作少監李道裕代之,仍令治書侍御史唐臨馳傳械挺赴洛陽,依議除名,仍令白衣散從。及前軍破蓋牟城,詔挺統
 兵士鎮蓋牟,示漸用之也。挺城守去大軍懸遠,與高麗新城鄰接,日夜戰鬥,鼓噪之聲不絕。挺不堪其憂,且不平於失職,素與術士公孫常善,乃與常書以敘所懷。會常以他事被拘,自縊而死,索其囊中,得挺書,論城中危蹙,兼有嘆悵之辭。太宗以挺怨望,謫為象州刺史。歲餘卒,年五十八。



 子待價,初為左千牛備身。永徽中,江夏王道宗得罪,待價即道宗之婿也,緣坐左遷盧龍府果毅。時將軍辛文陵率兵招慰高麗,行至吐護真水,高麗掩
 其不備,襲擊敗之。待價與中郎將薛仁貴受詔經略東蕃,因率所部救之。文陵苦戰,賊漸退,軍始獲全。待價被重瘡,流矢中其左足,竟不言其功,以足疾免官而歸。後累授蘭州刺史。時吐蕃屢為邊患,高宗以沛王賢為涼州大都督,以待價為司馬。俄又遷蕭州刺史,頻有守禦之功,徵拜右武衛將軍,兼檢校右羽林軍事。儀鳳三年,吐蕃又犯塞,待價復以本官檢校涼州都督,兼知鎮守兵馬事。俄又徵還舊職,復封扶陽侯。則天臨朝,拜吏部
 尚書,攝司空。營高宗山陵,功畢,加金紫光祿大夫,改為天官尚書、同鳳閣鸞臺三品,賜物一千段,仍與一子五品。待價素無藻鑒之才,自武職而起,居選部,既銓綜無敘,甚為當時所嗤。垂拱元年十月,復為燕然道行軍大總管,以御突厥。明年春還。六月,拜文昌右相,依舊同鳳閣鸞臺三品。既累登非據,頗不自安,頻上表辭職,則天每降優制不許之。又表請削官秩,回恩贈父,於是贈挺潤州刺史。明年,上疏請自效戎旅之用,於是拜安息道
 行軍大總管,督三十六總管以討吐蕃,進封扶陽郡公。軍至寅識迦河,與吐蕃合戰,初勝後敗。又屬天寒凍雪,師人多死,糧饋又不支給,乃旋師弓月,頓於高昌。則天大怒,副將閻溫古以逗留伏法,待價坐除名,配流繡州,尋卒。



 弟萬石,頗有學業,而特善音律。上元中,自吏部郎中遷太常少卿。當時郊廟樂調及宴會雜樂,皆萬石與太史令姚玄辯增損之,時人以為稱職。尋又兼知吏部選事,卒官。挺從祖兄子安石,別有傳。



 楊纂,華州華陰人也。祖儉,周東雍州刺史。父文偉,隋溫州刺史。纂略涉經史,尤明時務。少與瑯邪顏師古、燉煌令狐德棻友善。大業中,進士舉,授朔方郡司法書佐,坐楊玄感近屬除名,乃家於蒲城。義軍渡河,於長春宮謁見。累授侍御史。數上書言事,因被召問,擢為考功郎中。貞觀初,長安令,賜爵長安縣男。有婦人袁氏妖逆,為人所告,纂究問之,不得其狀。袁氏後又事發伏誅,太宗以纂為不忠,將殺之。中書令溫彥博以纂過誤,罪不至死,
 固諫,乃赦之。三遷吏部侍郎。八年,副特進蕭瑀為河南道巡察大使,與瑀情有不協,屢相表奏,瑀因以獲罪。纂尋拜尚書左丞。纂既長於吏道,所在皆有聲績。俄又除吏部侍郎。前後典選十餘載,銓敘人倫,稱為允當。然而抑文雅,進酷吏,觀時任數,頗為時論所譏。後歷太常少卿、雍州別駕,加銀青光祿大夫。復為尚書左丞,遷太僕卿,檢校雍州別駕。遷戶部尚書。永徽初卒,贈幽州都督,謚曰敬。子守愚,則天時官至雍州長史;守挹,岐州刺史。
 族子弘禮。



 弘禮,隋尚書令素弟之子也。父岳,大業中為萬年令,與素子玄感不協,嘗密上表稱玄感必為亂。及玄感被誅,岳在長安系獄,帝遽使赦之。比使至,岳已為留守所殺,弘禮等遂免從坐。高祖受禪,以楊素隋代有勛業,詔弘禮襲封清河郡公,拜太子通事舍人。貞觀中,歷兵部員外郎,仍為西河道行軍大總管府長史,三遷中書舍人。太宗有事遼東,以弘禮有文武材,擢拜兵部侍郎,專典兵機之務。弘禮每入參謀議,出則統眾攻戰。
 駐蹕之陣,領馬步二十四軍,出其不意以擊之,所向摧破。太宗自山下見弘禮所統之眾,人皆盡力,殺獲居多,甚壯之。謂許敬宗等曰:「越公兒郎,故有家風矣。」時諸宰相並在定州留輔皇太子,唯有褚遂良、許敬宗及弘禮在行在所,掌知機務。二十年,拜中書侍郎。明年,加銀青光祿大夫,尋遷司農卿,兼充昆丘道副大總管,諸道軍將咸受節度。於是破處月,降處密,殺焉耆王,降馺支部,獲龜茲、于闐王。凱旋,未及行賞,太宗晏駕。弘禮頗忤大
 臣之旨,由是出為涇州刺史。永徽初,論昆丘之功,改授勝州都督。尋遷太府卿。四年卒,贈蘭州都督,謚曰質。弟弘武。



 弘武少修謹,武德初,拜左千牛備身。永徽中,為吏部郎中。孝敬初,為皇太子精擇僚寀,以弘武為中舍人。麟德中,將有事於東嶽,弘武自荊州司馬擢拜司戎少常伯。從駕還,高宗特令弘武補授吏部選人五品已上官,由是漸見親委。後母榮國夫人楊氏,以與弘武同宗,又稱薦之,俄遷西臺侍郎。乾封二年,與戴至德、李安
 期等同東西臺三品。及在政事,頗以清簡見稱。總章元年,卒於官,贈汴州刺史,謚曰恭。



 子元亨,則天時為司府少卿;元禧,尚食奉御。元禧頗有醫術,為則天所任。嘗忤張易之之意,易之密奏元禧是楊素兄弟之後,素父子在隋有逆節,子孫不合供奉。則天乃下制曰:「隋尚書令楊素,昔在本朝,早荷殊遇。稟兇邪之德,懷諂佞之才,惑亂君上,離間骨肉。搖動塚嫡,寧唯掘蠱之禍?誘扇後主,卒成請蹯之釁。隋室喪亡,蓋惟多僻,究其萌兆,實此之由。
 生為不忠之人,死為不義之鬼,身雖幸免,子竟族誅。斯則奸逆之謀,是其庭訓;險薄之行,遂成門風。刑戮雖加,枝胤仍在,豈可復肩隨近侍,齒跡朝行?朕接統百王,恭臨四海,上嘉賢佐,下捍賊臣,常欲從容於萬機之餘,褒貶於千載之外,況年代未遠,耳目所存者乎?其楊素及兄弟子孫,並不得令任京官及侍衛。」於是左貶元亨為睦州刺史,元禧為資州長史,元禧弟緱氏令元禕為梓州司馬。張易之誅後,元亨等皆復任京職,元亨至齊州
 刺史,元禧臺州刺史,元禕宣州刺史。



 劉德威,徐州彭城人也。父子將,隋毗陵郡通守。德威姿貌魁偉,頗以幹略見稱。大業末,從左光祿大夫裴仁基討賊淮左,手斬賊帥李青珪,傳首於行在所。後與仁基同歸李密,密素聞其名,與麾下兵,令於懷州鎮守。武德元年,密與王世充戰敗入朝,德威亦率所部隨密歸款。高祖嘉之,授左武候將軍,封滕縣公。及劉武周南侵,詔德威統兵擊之,又判並州總管府司馬。俄而裴寂失律
 於介州,齊王元吉棄並州還朝,留德威總知留府事。元吉才出,武周已至城下,百姓相率投賊。武周獲德威,令率其本兵往浩州招慰。德威自拔歸朝,高祖親勞問之,兼陳賊中虛實及晉、絳諸部利害,高祖皆嘉納之。改封彭城縣公。未幾,檢校大理少卿。從擒建德,平世充,皆有功,轉刑部侍郎,加散騎常侍,妻以平壽縣主。貞觀初,歷大理、太僕二卿,加金紫光祿大夫。俄出為綿州刺史,以廉平著稱,百姓為之立碑。尋檢校益州大都督府長史。十
 一年,復授大理卿。太宗嘗問之曰:「近來刑網稍密,其過安在?」德威奏言:「誠在主上,不由臣下。人主好寬則寬,好急則急,律文失入減三等,失出減五等。今則反是,失入則無辜,失出便獲大罪。所以吏各自愛,競執深文,非有教使之然,畏罪之所致耳。陛下但舍所急,則『寧失不經』復行於今日矣。」太宗深然之。數歲,遷刑部尚書,兼檢校雍州別駕。十七年,馳驛往濟州推齊王祐還,至濮州,聞祐殺長史權萬紀,德威入據濟州,遣使以聞。詔德威便
 發河南兵馬,以申經略,會遭母憂而罷。十八年,起為遂州刺史,三遷同州刺史。永徽三年卒,年七十一。贈禮部尚書、幽州都督,謚曰襄,陪葬獻陵。德威閨門友穆,接物寬平,所得財貨,多以分贍宗親。子審禮襲爵。



 審禮,少喪母,為祖母元氏所養。隋末,德威從裴仁基討擊,道路不通。審禮年未弱冠,自鄉里負載元氏,渡江避亂。及天下定,始西入長安。元氏若有疾,審禮必親嘗湯藥,元氏顧謂孫曰:「我兒孝順,貫徹幽微,吾一顧念,宿疾頓輕。」貞觀
 中,歷左驍衛郎將。丁父憂去職。及葬,跣足隨車,流血灑地,行路稱之。服闋當襲爵,累表讓弟,朝議不許。永徽中,累遷將作大匠,兼檢校燕然都護,襲封彭城郡公。審禮父歿雖久,猶悲慕不已,每見父時僚舊,必嗚咽流涕。母鄭氏早亡,事繼母平壽縣主,稍疾輒憂懼形於容色,終夕不寐。撫繼母男延景,友愛甚篤。所得祿俸,皆送母處,以資延景之費;而審禮妻子處饑寒,晏然未嘗介意。再從同居,家無異爨,合門二百餘口,人無間言。稍遷工部
 尚書,兼檢校左衛大將軍。儀鳳二年,吐蕃寇涼州,命審禮為行軍總管,與中書令李敬玄合勢討擊。遇賊於青海,敬玄後期不至,審禮事敗,為賊所執。永隆二年,卒於蕃中。贈工部尚書,謚曰僖。延景,官至陜州刺史,睿宗初,以後父追贈尚書右僕射。



 審禮子易從,歷位岐州司兵參軍。審禮之沒吐蕃,詔許易從入蕃省之。及審禮卒,易從號哭,晝夜不止,毀瘠過禮。吐蕃哀其志行,還其父尸柩,易從徒跣萬里,扶護歸彭城,為朝野之所嗟賞。後歷
 彭州長史、任城男。永昌中,坐為徐敬貞所誣構遇害。易從在官仁恕,及將刑,人吏無遠近奔走,競解衣相率造功德,以為長史祈福,州人從之者十餘萬。其為人所愛如此。易從子升,開元中,為中書舍人、太子右庶子。



 審禮從父弟延嗣,文明年為潤州司馬,屬徐敬業作亂,率眾攻潤州,延嗣與刺史李思文固守不降。俄而城陷,敬業執延嗣,邀之令降,辭曰:「延嗣世蒙國恩,當思效命,州城不守,多負朝廷。終不能茍免偷生,以累宗族,豈以
 一身之故,為千載之辱?今日之事,得死為幸。」敬業大怒,將斬之,其黨魏思溫救之獲免,乃囚之於江都獄。俄而賊敗,竟以裴炎近親,不得敘功,遷為梓州長史,再轉汾州刺史卒。宗族至刺史者二十餘人。



 閻立德,雍州萬年人,隋殿內少監毗之子也。其先自馬邑徙關中。毗初以工藝知名,立德與弟立本,早傳家業。武德中,累除尚衣奉御,立德所造袞冕大裘等六服並腰輿傘扇,咸依典式,時人稱之。貞觀初,歷遷將作少匠,
 封太安縣男。高祖崩,立德以營山陵功,擢為將作大匠。貞觀十年,文德皇后崩,又令攝司空,營昭陵。坐怠慢解職。俄起為博州刺史。十三年,復為將作大匠。十八年,從征高麗,及師旅至遼澤,東西二百餘里泥淖,人馬不通。立德填道造橋,兵無留礙。太宗甚悅。尋受詔造翠微宮及玉華宮,咸稱旨,賞賜甚厚。俄遷工部尚書。二十三年,攝司空,營護太宗山陵。事畢,進封為公。顯慶元年卒,贈吏部尚書、並州都督。玄邃子,官至司農少卿。玄邃子知
 微,聖歷初,歷位右豹韜衛將軍。時突厥默啜有女請和親,則天令淮陽王武延秀往納其女,命知微攝春官尚書送赴虜廷。默啜以延秀非皇室諸王,大怒,遂拘之別所,與知微率眾自恆岳道攻陷趙、定二州。知微經歲餘自突厥所還,則天以其隨賊入寇,令百官臠割,然後斬之,並夷其三族。



 立本,顯慶中累遷將作大匠,後代立德為工部尚書,兄弟相代為八座,時論榮之。總章元年,遷右相,賜爵博陵縣男。立本雖有應務之才,而尤善圖畫,
 工於寫真。《秦府十八學士圖》及貞觀中《凌煙閣功臣圖》,並立本之跡也,時人咸稱其妙。太宗嘗與侍臣學士泛舟於春苑,池中有異鳥,隨波容與。太宗擊賞,數詔座者為詠,召立本令寫焉。時閣外傳呼云:「畫師閻立本。」時已為主爵郎中,奔走流汗,俯伏池側,手揮丹粉,瞻望座賓,不勝愧赧。退誡其子曰:「吾少好讀書,幸免面墻,緣情染翰,頗及儕流。唯以丹青見知,躬廝役之務,辱莫大焉!汝宜深誡,勿習此末伎。」立本為性所好,欲罷不能也。及
 為右相,與左相姜恪對掌樞密。恪既歷任將軍,立功塞外;立本唯善於圖畫,非宰輔之器。故時人以《千字文》為語曰:「左相宣威沙漠,右相馳譽丹青。」咸亨元年,百司復舊名,改為中書令。四年卒。



 柳亨,蒲州解人,魏尚書左僕射慶之孫也。父旦,隋太常少卿、新城縣公。亨,隋末歷熊耳、王屋二縣長,陷於李密。密敗歸國,累授駕部郎中。亨容貌魁偉,高祖甚愛重之,特以殿中監竇誕之女妻焉,即帝之外孫也。三遷左衛
 中郎將,封壽陵縣男。未幾,以譴出為邛州刺史。加散騎常侍,被代還,數年不調。因兄葬,遇太宗游於南山,召見與語,頗哀矜之。數日,北門引見,深加誨獎,拜銀青光祿大夫,行光祿少卿。太宗每誡之曰:「與卿舊親,情素兼宿,卿為人交游過多,今授此職,宜存簡靜。」亨性好射獵,有饕湎之名。此後頗自勖勵,杜絕賓客,約身節儉,勤於職事。太宗亦以此稱之。二十三年,以修太廟功,加金紫光祿大夫。久之,拜太常卿,從幸萬年宮,檢校岐州刺史。永
 徽六年卒,贈禮部尚書、幽州都督,謚曰敬。



 亨族子範,貞觀中為侍御史。時吳王恪好畋獵,損居人,範奏彈之。太宗因謂侍臣:「權萬紀事我兒,不能匡正,其罪合死。」範進曰:「房玄齡事陛下,猶不能諫止畋獵,豈可獨罪萬紀?」太宗大怒,拂衣而入。久之,獨引範謂曰:「何得逆折我?」範曰:「臣聞主聖臣直,陛下仁明,臣敢不盡愚直。」太宗意乃解。範,高宗時歷位尚書右丞、揚州大都督府長史。



 亨兄子奭。奭父則,隋左衛騎曹,因使卒於高麗。奭入蕃迎喪柩,
 哀號逾禮,深為夷人所慕。貞觀中,累遷中書舍人。後以外生女為皇太子妃,擢拜兵部侍郎。妃為皇后,奭又遷中書侍郎。永徽三年,代褚遂良為中書令,仍監修國史。俄而後漸見疏忌,奭憂懼,頻上疏請辭樞密之任,轉為吏部尚書。及後廢,累貶愛州刺史。尋為許敬宗、李義府所構,云奭潛通宮掖,謀行鴆毒,又與褚遂良等朋黨構扇,罪當大逆。高宗遣使就愛州殺之,籍沒其家。奭既死非其罪,甚為當時之所傷痛。神龍初,則天遺制,與褚遂
 良、韓瑗等並還官爵。子孫親屬當時緣坐者,咸從曠蕩。



 開元初,亨孫渙為中書舍人,表曰:「臣堂伯祖奭,去明慶三年,與褚遂良等五家同被譴戮。雖蒙遺制蕩雪,而子孫亡沒並盡。唯有曾孫無忝,見貫龔州,蒙雪多年,猶同遠竄。陛下自臨宇縣,優政必被,鴻恩及於泉壤,大造加於亡絕。先天已後,頻降絲綸,曾任宰相之家,並許收其淪滯。況臣伯祖往叨執政,無犯受誅,槁窆尚隔故鄉,後嗣遂編蠻服。臣不申號訴,義所難安。伏乞許臣伯祖還
 葬鄉里,其曾孫無忝放歸本貫。」疏奏,敕令奭歸葬,官造靈輿遞還。無忝後歷位潭州都督。



 渙弟澤,景雲中為右率府鎧曹參軍。先是,姚元之、宋璟知政事,奏請停中宗朝斜封官數千員。及元之等出為刺史,太平公主又特為之言,有敕總令復舊職。澤上疏諫曰:



 臣聞藥不毒,不可以蠲疾;詞不切,不可以補過。是以習甘旨者,非攝養之方;邇諛佞者,積危殆之本。臣實愚樸,志懷剛勵,或聞政之不當,事之不直,常慷慨關心,夢寐懷憤。每願殉身
 以諫,伏死而爭。但利於社稷,有便於君上,雖蒙禍被難,殺身不悔也。竊見神龍以來,群邪作孽,法網不振,綱維大紊,實由內寵專命,外嬖擅權,因貴憑寵,賣官鬻爵。硃紫之榮,出於僕妾之口;賞罰之命,乖於章程之典。妃主之門,有同商賈;舉選之署,實均闤闠。屠販之子,悉由邪而忝官;黜斥之人,咸因奸而冒進。天下為亂,社稷幾危,賴陛下聰明神武,拯其將墜。此陛下耳目之所親擊,固可永為炯誡者也。臣聞作法於理,猶恐其亂,作法於亂,
 誰能救之?只如斜封授官,皆是僕妾汲引,迷謬先帝,昧目前朝,豈是孝和情之所憐,心之所愛?陛下初即位時,納姚元之、宋璟之計,所以咸令黜之。頃日已來,又令敘之。將謂為斜封之人不忍棄也,以為先帝之意不可違也?若斜封之人不忍棄也,是韋月將、燕欽融之流亦不可褒贈也,李多祚、鄭克義之徒亦不可清雪也。陛下何不能忍於此而獨能忍於彼?使善惡不定,反覆相攻,使君子道消,小人道長,為邪者獲利,為正者銜冤,奈何導
 人以為非,勸人以為僻?將何以懲風俗,將何以止奸邪?今海內咸稱太平公主令胡僧慧範曲引此輩,將有誤於陛下矣。謗議盈耳,咨嗟滿衢,故語曰:「姚、宋為相,邪不如正。太平用事,正不如邪。」《書》曰:「無偏無陂,遵王之義,無反無側,王道正直。」臣恐因循,流近致遠,積小為大,累微起高。勿謂何傷,其禍將長;勿謂何害,其禍將大。又賞罰之典,紀綱不謬,天秩有禮,君爵有功,不可因怒以妄罰,不可因喜以妄賞。伏見尚醫奉御彭君慶,以邪巫小道,
 超授三品,奈何輕用名器,加非其才?昔公主為子求郎,明帝不許;今聖朝私愛,賞及憸人。董狐不亡,豈有所隱?臣聞賞一人而千萬人悅者賞之,罰一人而千萬人勸者罰之。臣雖未睹聖朝之妄罰,已睹聖朝之妄賞矣,《書》曰:「官不及私暱,惟其能;爵罔及惡德,惟其賢。」臣恐近習之人為其先容,有謬於陛下也。惟陛下熟思而察之。雖往者不可諫,而來者猶可追。願杜請謁之路,塞恩幸之門,鑒誡前非,無累後悔。申畫一之法,明不二之刑,不詢
 之謀勿庸,無稽之言勿應,則天下之化,人無間焉,日新之德,天鑒不遠。



 澤後參選,會有敕令選人上書陳事,將加收擢,澤又上書曰:



 頃者韋氏險詖,奸臣同惡。賞罰紊弛,綱紀紛綸,政以賄成,官因寵進,言正者獲戾,行殊者見疑,海內寒心,實將莫救。賴神明佑德,宗廟降靈,天討有罪,人用不保,陛下睿謀神聖,勇智聰明,安宗廟於已危,拯黎庶於將溺。今尨眉鮐背,歡欣踴躍,望聖朝之撫輯,聽聖朝之德音。今陛下蠲煩省徭,法明德舉,萬邦愷
 樂,室家胥慶。臣又聞危者保其存也,亂者有其理也。伏惟陛下安不忘危,理不忘亂,存不忘亡,則克享天心,國家長保矣。《詩》曰:「罔不有初,鮮克有終。」伏惟陛下慎厥終,修其初,非禮勿視,非禮勿動。《書》曰:「惟德罔小,萬邦惟慶,惟不德罔大,墜厥宗。」甚可畏也,甚可懼也,伏惟陛下慎之哉!夫驕奢起於親貴,綱紀亂於寵幸。願陛下禁之於親貴,則天下隨風矣;制之於寵幸,則天下法明矣。《詩》曰:「刑於寡妻,至於兄弟,以御於家邦。」若親貴為之而不禁,
 寵幸撓之而見從,是政之不常,令之不一,則奸詐斯起,暴亂生焉。雖嚴刑峻制,朝施暮戮,而法不行矣。縱陛下親之愛之,莫若安之福之。寵祿之過,罪之漸也,非安之也;驕奢之淫,危之本也,非福之也。前事不忘,後之師也,伏願陛下精求俊哲,朝夕納誨。縱有逆於耳、謬於心者,無速之罰,姑籌之以道,省於厥躬。雖木樸忌忤,願恕之以直,開諫諍之路也。或有順於耳、便於身者,無急之賞,當求諸非道,稽之典訓。其不協於德,必置之以法,用杜
 側媚之行也。有羞淫巧於陛下者,遽黜之,則淫巧息矣;有進忠讜於陛下者,遽賞之,則忠讜進矣。臣又聞生於富者驕,生於貴者傲。石碏曰:「臣聞愛子,教之以義方,不納於邪,驕奢淫逸,所自邪也。」《書》曰:「罔淫於逸,罔游於樂。」穆王有命,「實賴前後左右有位之士,繩愆糾謬,格其非心。」今儲宮肇建,王府初啟,至於僚友,必惟妙擇。今驕奢之後,流波未變;慢游之樂,餘風或存。夫小人幸臣,易合於意;奇伎淫巧,多適於心。臣恐狎於非德,茲為愈怠。《書》
 曰:「慎簡乃僚,無以巧言令色,其惟吉士。僕臣正,厥後克正;僕臣諛,厥後自聖。」伏願採溫良博聞之士,恭儉忠鯁之人,任以東宮及諸王府官,仍請東宮量署拾遺補闕之職。令朝夕講論,出入時從,授以訓誥,交修不迨。臣又聞馳騁畋獵,令人發狂。名教之中,自有樂地。承前貴戚,鮮克由禮。或打球擊鼓,比周伎術;或飛鷹奔犬,盤游藪澤。此甚為不道,非進德修業之本也。《書》曰:「內作色荒,外作禽荒。」又曰:「無若丹硃傲,惟慢游是好。朋淫於家,用殄
 厥世。」伏惟陛下誕降謀訓,敦勤學業,示之以好惡,陳之以成敗,以義制事,以禮制心,圖之於未萌,慮之於未有,則福祿長享,與國並休矣。臣又聞富不與驕期而驕自至,驕不與罪期而罪自至,罪不與死期而死自至。信矣斯語,明哉至誡!頃韋庶人、安樂公主、武延秀等可謂貴矣,可謂寵矣,權侔人主,威震天下。然怙侈滅德,神怒人棄。豈不謂愛之太極,富之太多,不節之以禮,不防之以法,終轉吉為兇,變福為禍。諺曰:「千人所指,無病自死。」不
 其然歟?《書》曰:「殷鑒不遠,在彼夏王。」今陛下何勸,豈非皇祖謀訓之則也?今陛下何懲,豈非孝和寵任之甚也?《禮》曰:「愛而知其惡,憎而知其善。」可不慎哉!夫寵愛之心則不免,去其太甚,閑之禮節,適則可矣。今諸王、公主、駙馬,亦陛下之所親愛也。矯枉之道,在於厥初,鑒誡之義,其取不遠。使觀過務善,居寵思危,庶夙夜惟寅,聿修厥德。《經》曰:「在上不驕,高而不危,所以長守貴也;制節謹度,滿而不溢,所以長守富也。富貴不離其身,然後能保其社
 稷。」《書》曰:「制於官刑,警於有位。敢有常舞於宮,酣歌於室,時謂巫風;敢有徇於貨色,常於游畋,時謂淫風;敢有侮聖言,逆忠直,遠耆德,比頑童,時謂亂風。惟茲三風十愆,卿士有一於身,家必喪;邦君有一於身,國必亡。」甚可畏也,甚可懼也!伏惟陛下必察而明之,必信而勸之。有奢僭驕怠者,削其祿封;樸素修業者,錫以紳服。以勖其非心,使其奉命,無使久而忽之,無使遠而墜之。臣聞非知之艱,行之惟艱。又曰:「常厥德,保厥位,厥德匪常,九有以
 亡。」伏惟陛下慎之哉!前車之覆,實惟明證;先王之誡,可以終吉。若陛下奉伊尹之訓,崇傅說之命,不作無益,不啟私門,刑不差,賞不濫,則惟德是輔,惟人之懷,天祿永終,景福是集。儻陛下忘精一之德,開恩幸之門,爵賞有差,刑罰不當,則忠臣正士,亦不復談矣。



 睿宗覽而善之,令中書省重詳議,擢拜監察御史。開元中,累遷太子右庶子。出為鄭州刺史,未行病卒,贈兵部侍郎。



 崔義玄,貝州武城人也。大業末,往依李密,初不見用。義
 玄見群鼠渡洛,又槊刃有花文,謂所親曰:「此王敦敗亡之兆也。」時黃君漢守據柏崖,義玄往說之曰:「見機而作,不俟終日。今群盜蜂起,九州幅裂,神器所歸,必在有德。唐公據有秦京,名應符籙,此真主也。足下孤城獨立,宜遵寇恂、竇融之策,及時歸誠,以取封侯也。」君漢然之,即與義玄歸國。拜懷州總管府司馬。世充遣將高毗侵掠河內,義玄擊敗之,多下城堡。君漢將分子女金帛與之,義玄皆拒而不受,以功封清丘縣公。後從太宗討世充,
 屢獻籌策,太宗頗納用之。東都平,轉隰州都督府長史。貞觀初,歷左司郎中,兼韓王府長史,行州府事。與友人孟神慶雖志好不同,各以介直匡正府幕,王並委任之。永徽初,累遷婺州刺史。屬睦州女子陳碩真舉兵反,遣其黨童文寶領徒四千人掩襲婺州。義玄將督軍拒戰,時百姓訛言碩真嘗升天,犯其兵馬者無不滅門,眾皆兇懼。司功參軍崔玄籍言於義玄曰:「起兵仗順,猶且不成,此乃妖誑,豈能得久?」義玄以為然,因命玄籍為先鋒,
 義玄率兵繼進,至下淮戌,擒其間諜二十餘人。夜有流星墜賊營,義玄曰:「此賊滅之徵也。」詰朝進擊,身先士卒,左右以盾蔽箭,義玄曰:「刺史尚欲避箭,誰肯致死?」由是士卒戮力,斬首數百級,餘悉許其歸首。進兵至睦州界,歸降萬計。及碩真平,義玄以功拜御史大夫。義玄少愛章句之學,《五經》大義,先儒所疑及音韻不明者,兼採眾家,皆為解釋,傍引證據,各有條疏。至是,高宗令義玄討論《五經》正義,與諸博士等詳定是非,事竟不就。高宗
 之立皇后武氏,義玄協贊其謀。及長孫無忌等得罪,皆義玄承中旨繩之。顯慶元年,出為蒲州刺史。尋卒,年七十一,贈幽州都督,謚曰貞。則天時思其功,重贈揚州大都督,賜其家實封二百戶。子神基襲爵。長壽中,為司賓卿、同鳳閣鸞臺平章事。為相月餘,為酷吏所陷,減死配流。後漸錄用,中宗初,為大理卿。神基弟神慶。



 神慶,明經舉,則天時,累遷萊州刺史。因入朝,待制於億歲殿,奏事稱旨。則天以神慶歷職皆有美政,又其父嘗有翊贊之勛,
 甚賞慰之,擢拜並州長史。因謂曰:「並州,朕之枌榆,又有軍馬,比日簡擇,無如卿者。前後長史,皆從尚書為之,以其委重,所以授卿也。」因自為按行圖,擇日而遣之。神慶到州,有豪富偽作改錢文敕,文書下州,穀麥踴貴,百姓驚擾。神慶執奏,以為不便,則天下制褒賞之。先是,並州有東西二城,隔汾水,神慶始築城相接,每歲省防禦兵數千人,邊州甚以為便。尋而兄神基下獄當死,神慶馳赴都告事,得召見。則天出神基推狀以示之,神慶據狀
 申理,神基竟得減死,神慶亦緣坐貶授歙州司馬。長安中,累轉禮部侍郎,數上疏陳時政利害,則天每嘉納之。轉太子右庶子,賜爵魏縣子。時有突厥使入朝,準儀注,太子合預朝參,先降敕書。神慶上疏曰:「伏以五品已上所以佩龜者,比為別敕徵召,恐有詐妄,內出龜合,然後應命。況太子元良國本,萬方所瞻,古來徵召皆用玉契,此誠重慎之極,防萌之慮。昨緣突厥使見,太子合預朝參,直有文符下宮,曾不降敕處分。今人稟淳化,內外同
 心,然古人慮事於未萌之前,所以長無悔吝之咎。況太子至重,不可不深為誡慎。以臣愚見,太子既與陛下異宮,伏望每召太子,預報來日,非朔望朝參,應須別喚,望降墨敕及玉契。」則天甚然之。尋令神慶與詹事祝欽明更日於東宮侍讀。俄歷司刑、司禮二卿。神慶嘗受詔推張昌宗,而竟寬其罪。神龍初,昌宗等伏誅,神慶坐流於欽州。尋卒,年七十餘。明年,敬暉等得罪,緣昌宗被流貶者例皆雪免,贈神慶幽州都督。



 開元中,神慶子琳等
 皆至大官,群從數十人,趨奏省闥。每歲時家宴,組珮輝映,以一榻置笏,重疊於其上。開元、天寶間,中外族屬無緦麻之喪,其福履昌盛如此。東都私第門,琳與弟太子詹事珪、光祿卿瑤,俱列棨戟,時號「三戟崔家」。琳位終太子少保。



 史臣曰:周、隋已來,韋氏世有令人,鬱為冠族,而安石嗣立,竟大其門。挺恃才傲物,固虧長者之風,賓王報之以不仁,難與議乎君子矣!議者以堯、舜有溢美,桀、紂有溢
 惡,蓋以一為兇德,則群惡所歸。楊素父子,傾覆隋祚,醜聲流聞,雖弘禮、弘武之正士,而元亨兄弟,竟以兇族竄逐。古人守死善道,不無為也。德威奏議,練刑名之要,俾長秋卿,美哉!審禮仁孝,治行可為世範,卒與禍會,悲夫!二閻曲學甚工,措思精巧,藝成而下,垂誡宜然。柳氏世稱謇諤,奭、澤有正人風彩,忠規獻納,抑有人焉。義玄附麗武後,神慶寬縱穢臣,弈世纖邪,以至傾敗,宜哉!



 贊曰:韋子驕矜,終損功名。楊家積惡,宗門擯落。閻以
 藝辱,劉以孝愆。二崔能吏,行無取焉。



\end{pinyinscope}