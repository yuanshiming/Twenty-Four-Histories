\article{卷五十五 列傳第五 薛舉(子仁杲) 李軌 劉武周(苑君璋附) 高開道 劉黑闥(徐圓朗)}

\begin{pinyinscope}

 ○薛舉子仁杲李軌劉武周苑君璋附高開道劉黑闥徐圓朗



 薛舉,河東汾陰人也。其父汪,徙居金城。舉容貌瑰偉,兇悍善射,驍武絕倫,家產巨萬,交結豪猾,雄於邊朔。初,為金城府校尉。大業末,隴西群盜蜂起,百姓饑餒,金城令郝瑗,募得數千人,使舉討捕。授甲於郡中,吏人咸集,置酒以饗士。舉與其子仁杲及同謀者十三人,於座中劫瑗,矯稱收捕反者,因發兵囚郡縣官,開倉以賑貧乏。自稱西秦霸王,建元為秦興,封仁杲為齊公,少子仁越為晉公。有宗羅者,先聚黨為群盜,至是帥眾會之,封為義興公,餘皆以次封拜。掠官收馬,招集群盜,兵鋒甚銳,所至皆下。隋將皇甫綰屯兵一萬在枹罕,舉選精銳二
 千人襲之,與綰軍遇於赤岸,陳兵未戰,俄而風雨暴至。初,風逆舉陣,而綰不擊之;忽返風,正逆綰陣,氣色昏昧,軍中擾亂。舉策馬先登,眾軍從之,隋軍大潰,遂陷枹罕。時羌首鐘利俗擁兵二萬在岷山界,盡以眾降舉,兵遂大振。進仁杲為齊王,授東道行軍元帥;仁越為晉王,兼河州刺史;羅為義興王,以副仁杲。總兵略地,又克鄯、廓二州,數日間,盡有隴西之地,眾至十三萬。



 十三年秋七月,舉僭號於蘭州,以妻鞠氏為皇后,母為皇太后,起
 墳塋,置陵邑,立廟於城南。其月,舉陳兵數萬,出拜墓,禮畢大會。仁杲進兵圍秦州。仁越兵趨劍口,至河池郡,太守蕭瑀拒退之。舉命其將常仲興渡河擊李軌,與軌將李贇大戰於昌松,仲興敗績,全軍陷於軌。及仁杲克秦州,舉自蘭州遷都之。遣仁杲引軍寇扶風郡,汧源賊帥唐弼率眾拒之,兵不得進。初,弼起扶風,立隴西李弘芝為天子,有徒十萬。舉遣使招弼,弼殺弘芝,引軍從舉。仁杲因弼弛備,襲破之,並有其眾,弼以數百騎遁免。舉勢
 益張,軍號三十萬,將圖京師。會義兵定關中,遂留攻扶風。太宗帥師討敗之,斬首數千級,追奔至隴坻而還。舉又懼太宗逾隴追之,乃問其眾曰:「古來天子有降事否?」偽黃門侍郎褚亮曰:「昔越帝趙佗卒歸漢祖,蜀主劉禪亦仕晉朝,近代蕭琮,至今猶貴。轉禍為福,自古有之。」其衛尉卿郝瑗趨而進曰:「皇帝失問。褚亮之言,又何悖也!昔漢祖屢經敗績,蜀先主亟亡妻子,戰之利害,何代無之?安得一戰不捷,而為亡國之計也!」舉亦悔之,答曰:「聊
 發此問,試君等耳。」乃厚賞瑗,引為謀主。瑗又勸舉連結梁師都,共為聲勢,厚賂突厥,餌其戎馬,合從並力,進逼京師。舉從其言,與突厥莫賀咄設謀取京師。莫賀咄設許以兵隨之,期有日矣。會都水監宇文歆使於突厥,歆說莫賀咄設,止其出兵,故舉謀不行。



 武德元年,豐州總管張長遜進擊宗羅,舉悉眾來援,軍屯高墌,縱兵虜掠,至於豳、岐之地。太宗又率眾擊之,軍次高墌城,度其糧少,意在速戰,乃命深溝堅壁,以老其師。未及與戰,會
 太宗不豫,行軍長史劉文靜、殷開山請觀兵於高墌西南,恃眾不設備,為舉兵掩乘其後。太宗聞之,知其必敗,遽與書責之。未至,兩軍合戰,竟為舉所敗,死者十五六,大將慕容羅、李安遠、劉弘基皆陷於陣。太宗歸於京師,舉軍取高墌,又遣仁杲進圍寧州。郝瑗言於舉曰:「今唐兵新破,將帥並擒,京師騷動,可乘勝直取長安。」舉然之。臨發而舉疾,召巫視之,巫言唐兵為祟,舉惡之,未幾而死。舉每破陣,所獲士卒皆殺之,殺人多斷舌、割鼻,或
 碓搗之。其妻性又酷暴,好鞭撻其下,見人不勝痛而宛轉於地,則埋其足,才露腹背而捶之。由是人心不附。仁杲代董其眾,偽謚舉為武皇帝,未葬而仁杲滅。



 仁杲,舉長子也,多力善騎射,軍中號為萬人敵。然所至多殺人,納其妻妾。獲庾信子立,怒其不降,磔於猛火之上,漸割以啖軍士。初,拔秦州,悉召富人倒懸之,以醋灌鼻,或杙其下竅,以求金寶。舉每誡之曰:「汝智略縱橫,足辦我家事,而傷於苛虐,與物無恩,終當覆我宗社。」舉死,仁杲立
 於折墌城,與諸將帥素多有隙,及嗣位,眾咸猜懼。郝瑗哭舉悲思,因病不起,自此兵勢日衰。



 自劉文靜為舉所敗後,高祖命太宗率諸軍以擊仁杲,師次高墌,而堅壁不動。諸將咸請戰,太宗曰:「我士卒新敗,銳氣猶少。賊以勝自驕,必輕敵好鬥,故且閉壁以折之。待其氣衰而後奮擊,可一戰而破,此萬全計也。」乃令軍中曰:「敢言戰者斬。」相持者久之。仁杲勇而無謀,兼糧饋不屬,將士稍離,其內史令翟長孫以其眾來降,仁杲妹夫偽左僕射鐘
 俱仇以河州歸國。太宗知其可擊,遣將軍龐玉擊賊將宗羅於淺水原。兩軍酣戰,太宗以勁兵出賊不意,奮擊大破之。乘勝進薄其折墌城,仁杲窮蹙,率偽百官開門降,太宗納之。王師振旅,以仁杲歸於京師,及其首帥數十人皆斬之。舉父子相繼偽位至滅,凡五年,隴西平。



 李軌,字處則,武威姑臧人也。有機辯,頗窺書籍,家富於財,賑窮濟乏,人亦稱之。大業末,為鷹揚府司馬。時薛舉作亂於金城,軌與同郡曹珍、關謹、梁碩、李贇、安修仁等
 謀曰:「薛舉殘暴,必來侵擾,郡官庸怯,無以御之。今宜同心戮力,保據河右,以觀天下之事,豈可束手於人,妻子分散!」乃謀共舉兵,皆相讓,莫肯為主。曹珍曰:「常聞圖讖云『李氏當王』。今軌在謀中,豈非天命也?」遂拜賀之,推以為主。軌令修仁夜率諸胡入內苑城,建旗大呼,軌於郭下聚眾應之,執縛隋虎賁郎將謝統師、郡丞韋士政。軌自稱河西大涼王,建元安樂,署置官屬,並擬開皇故事。初,突厥曷娑那可汗率眾內屬,遣弟闕達度闕設領部
 落在會寧川中,有二千餘騎,至是自稱可汗,來降於軌。



 武德元年冬,軌僭稱尊號,以其子伯玉為皇太子,長史曹珍為左僕射。謹等議欲盡殺隋官,分其家產,軌曰:「諸人見逼為主,便須稟吾處分。義兵之起,意在救焚,今殺人取物,是為狂賊。立計如此,何以求濟乎!」乃署統師太僕卿,士政太府卿,薛舉遣兵侵軌,軌遣其將李贇擊敗於昌松,斬首二千級,盡虜其眾,復議放還之。贇言於軌曰:「今竭力戰勝,俘虜賊兵,又縱放之,還使資敵,不如盡
 坑之。」軌曰:「不然。若有天命,自擒其主,此輩士卒,終為我有。若事不成,留此何益?」遂遣之。未幾,攻陷張掖、燉煌、西平、枹罕,盡有河西五郡之地。



 其年,軌殺其吏部尚書梁碩。初,軌之起也,碩為謀主,甚有智略,眾咸憚之。碩見諸胡種落繁盛,乃陰勸軌宜加防察,與其戶部尚書安修仁由是有隙。又軌子仲琰懷恨,形於辭色,修仁因之構成碩罪,更譖毀之,云其欲反,軌令齎鴆就宅殺焉。是後,故人多疑懼之,心膂從此稍離。時高祖方圖薛舉,遣使
 潛往涼州與之相結,下璽書,謂之為從弟。軌大悅,遣其弟懋入朝,獻方物。高祖授懋大將軍,遣還涼州。又令鴻臚少卿張侯德持節,冊拜為涼州總管,封涼王,給羽葆鼓吹一部。軌召群僚廷議曰:「今吾從兄膺受圖籙,據有京邑,天命可知,一姓不宜競立,今去帝號受冊可乎?」曹珍進曰:「隋失天下,英雄競逐,稱王號帝,鼎峙瓜分。唐國自據關中,大涼自處河右,己為天子,奈何受人官爵?若欲以小事大,宜依蕭察故事,自稱梁帝而稱臣於周。」軌
 從之。



 二年,遣其尚書左丞鄧曉隨使者入朝,表稱皇從弟大涼皇帝臣軌而不受官。時有胡巫惑之曰:「上帝當遣玉女從天而降。」遂徵兵築臺以候玉女,多所糜費,百姓患之。又屬年饑,人相食,軌傾家賑之,私家罄盡,不能周遍。又欲開倉給粟,召眾議之。珍等對曰:「國以人為本,本既不立,國將傾危,安可惜此倉粟,而坐觀百姓之死乎?」其故人皆云,給粟為便。謝統師等隋舊官人,為軌所獲,雖被任使,情猶不附。每與群胡相結,引進朋黨,排軌
 舊人,因其大餒,欲離其眾。乃詬珍曰:「百姓餓者自是弱人,勇壯之士終不肯困,國家倉粟須備不虞,豈可散之以供小弱?僕射茍悅人情,殊非國計。」軌以為然,由是士庶怨憤,多欲叛之。



 初,安修仁之兄興貴先在長安,表請詣涼州招慰軌。高祖謂曰:「李軌據河西之地,連好吐谷渾,結援於突厥,興兵討擊,尚以為難,豈單使所能致也?」興貴對曰:「李軌兇強,誠如聖旨。今若諭之以逆順,曉之以禍福,彼則憑固負遠,必不見從。何則?臣於涼州,奕代
 豪望,凡厥士庶,靡不依附。臣之弟為軌所信任,職典樞密者數十人,以此候隙圖之,易於反掌,無不濟矣。」高祖從之。興貴至涼州,軌授以左右衛大將軍,又問以自安之術,興貴諭之曰:「涼州僻遠,人物凋殘,勝兵雖餘十萬,開地不過千里,既無險固,又接蕃戎,戎狄豺狼,非我族類,此而可久,實用為疑。今大唐據有京邑,略定中原,攻必取,戰必勝,是天所啟,非人力焉。今若舉河西之地委質事之,即漢家竇融,未足為比。」軌默然不答,久之,謂興
 貴曰:「昔吳濞以江左之兵,猶稱己為『東帝』;我今以河右之眾,豈得不為『西帝』?彼雖強大,其如帝何?君與唐為計,誘引於我,酬彼恩遇耳。」興貴懼,乃偽謝曰:「竊聞富貴不在故鄉,有如衣錦夜行。今合家子弟並蒙信任,榮慶實在一門,豈敢興心,更懷他志?」興貴知軌不可動,乃與修仁等潛謀,引諸胡眾起兵圖軌,將圍其城,軌率步騎千餘出城拒戰。先時,有薛舉柱國奚道宜,率羌兵三百人亡奔於軌,既許其刺史而不授之,禮遇又薄,深懷憤怨。
 道宜率所部共修仁擊軌,軌敗入城,引兵登陴,冀有外救。興貴宣言曰:「大唐使我來殺李軌,不從者誅及三族!」於是諸城老幼皆出詣修仁。軌嘆曰:「人心去矣,天亡我乎!」攜妻子上玉女臺,置酒為別,修仁執之以聞。時鄧曉尚在長安,聞軌敗,舞蹈稱慶。高祖數之曰:「汝委質於人,為使來此,聞軌淪陷,曾無蹙容,茍悅朕情,妄為慶躍。既不能留心於李軌,何能盡節於朕乎?」竟廢而不齒。軌尋伏誅,自起至滅三載,河西悉平。詔授興貴右武候大將
 軍、上柱國,封涼國公,食實封六百戶,賜帛萬段;修仁左武候大將軍,封申國公,並給田宅,食實封六百戶。



 劉武周,河間景城人。父匡,徙家馬邑。匡嘗與妻趙氏夜坐庭中,忽見一物,狀如雄雞,流光燭地,飛入趙氏懷,振衣無所見,因而有娠,遂生武周。驍勇善射,交通豪俠。其兄山伯每誡之曰:「汝不擇交游,終當滅吾族也。」數詈辱之。武周因去家入洛,為太僕楊義臣帳內,募征遼東,以軍功授建節校尉。還家,為鷹揚府校尉。太守王仁恭以
 其州里之雄,甚見親遇,每令率虞候屯於閣下。因與仁恭侍兒私通,恐事洩,又見天下已亂,陰懷異計,乃宣言於郡中曰:「今百姓饑餓,死人相枕於野,王府尹閉倉不恤,豈憂百姓之意乎!」以此激怒眾人,皆發憤怨。武周知眾心搖動,因稱疾不起,鄉閭豪傑多來候問,遂椎牛縱酒大言曰:「盜賊若此,壯士守志,並死溝壑。今倉內積粟皆爛,誰能與我取之?」諸豪傑皆許諾。與同郡張萬歲等十餘人候仁恭視事,武周上謁,萬歲自後而入,斬仁恭
 於郡,持其首出徇郡中,無敢動者。於是開廩以賑窮乏,馳檄境內,其屬城皆歸之,得兵萬餘人。



 武周自稱太守,遣使附於突厥。隋雁門郡丞陳孝意、虎賁將王智辯合兵討之,圍其桑乾鎮。會突厥大至,與武周共擊智辯,隋師敗績。孝意奔還雁門,部人殺之,以城降於武周。於是襲破樓煩郡,進取汾陽宮,獲隋宮人以賂突厥,始畢可汗以馬報之,兵威益振。及攻陷定襄,復歸於馬邑。突厥立武周為定楊可汗,遺以狼頭纛。因僭稱皇帝,以妻
 沮氏為皇后,建元為天興。以衛士楊伏念為左僕射,妹婿同縣人苑君璋為內史令。先是,上谷人宋金剛有眾萬餘人,在易州界為群盜,定州賊帥魏刀兒與相表裏。後刀兒為竇建德所滅,金剛救之,戰敗,率餘眾四千人奔武周。武周聞金剛善用兵,得之甚喜,號為宋王,委以軍事,中分家產遺之。金剛亦深自結納,遂出其妻,請聘武周之妹。又說武周入圖晉陽,南向以爭天下。武周授金剛西南道大行臺,令率兵二萬人侵並州,軍黃虵鎮。
 又引突厥之眾,兵鋒甚盛,襲破榆次縣,進陷介州。高祖遣太常少卿李仲文率眾討之,為賊所執,一軍全沒。仲文後得逃還。復遣右僕射裴寂拒之,戰又敗績。武周進逼,總管齊王元吉委城遁走,武周遂據太原。遣金剛進攻晉州,六日,城陷,右驍衛大將軍劉弘基沒於賊。進取澮州,屬縣悉下。



 夏縣人呂崇茂殺縣令,自號魏王,以應賊。河東賊帥王行本又密與金剛連和,關中大駭。高祖命太宗益兵進討,屯於柏壁,相持者久之。又命永安王
 孝基、陜州總管於筠、工部尚書獨孤懷恩、內史侍郎唐儉進取夏縣,不能克,軍於城南。崇茂與賊將尉遲敬德襲破孝基營,諸軍並陷,四將俱沒。敬德還澮州,太宗邀擊於美良川,大破之。敬德與賊將尋相又援王行本於蒲州,太宗復破之於蒲州。高祖親幸蒲津關,太宗自柏壁輕騎謁高祖於行在所。宋金剛遂圍絳州。及太宗還,金剛懼而引退。武周復攻李仲文於浩州,頻戰皆敗,又饋運不屬,賊眾大餒,於是金剛遂遁。太宗復追及金剛
 於雀鼠谷,一日八戰,皆破之,俘斬數萬人,獲輜重千餘兩。金剛走入介州,王師逼之。金剛尚有眾二萬,出其西門,背城而陣,太宗與諸將力戰破之,金剛輕騎遁走。其驍將尉遲敬德、尋相、張萬歲收其精兵,舉介州及永安來降。武周大懼,率五百騎棄並州北走,自乾燭穀亡奔突厥。金剛復收其亡散以拒官軍,人莫之從,與百餘騎復奔突厥。太宗進平並州,悉復故地。未幾,金剛背突厥而亡,將還上谷,為追騎所獲,腰斬之。武周又欲謀歸馬
 邑,事洩,為突厥所殺。武周自初起至死,凡六載。初,武周引兵南侵,苑君璋說曰:「唐主舉一州之兵,定三輔之地,郡縣影附,所向風靡,此固天命,豈曰人謀?且並州已南,地形險阻,若懸軍深入,恐後無所繼,不如連和突厥,結援唐朝,南面稱孤,足為上策。」武周不聽,遣君璋守朔州,遂侵汾、晉。及敗,泣謂君璋曰:「恨不用君言,乃至於此!」



 武周既死,突厥又以君璋為大行臺,統其餘眾,仍令鬱射設督兵助鎮。高祖遣諭之,君璋部將高滿政謂君璋曰:「
 夷狄無禮,本非人類,豈可北面事之?不如盡殺突厥以歸唐朝。」君璋不從,滿政因人心夜逼君璋,君璋亡奔突厥。滿政遂以城來降,拜朔州總管,封榮國公。



 明年,君璋復引突厥來攻馬邑,滿政死之,君璋盡殺其黨而去,退保恆安。君璋所部稍稍離散,勢蹙請降,高祖許之,遣使賜以金券。會突厥頡利可汗復遣召之,君璋猶豫未決。其子孝政曰:「劉武周足為殷鑒。今既降唐,又歸頡利,取滅之道也。糧儲已盡,人情悉離,如更遲留,變生肘腋。」恆
 安人郭子威說君璋曰:「恆安之地,王者舊都,山川形勝,足為險固。突厥方強,為我脣齒。據此堅城,足觀天下之變,何乃欲降於人也?」君璋然其計,乃執我行人送於突厥,與突厥合軍寇太原之北境。君璋復見頡利政亂,竟率所部來降,拜安州都督,封芮國公,賜實封五百戶。



 高開道,滄州陽信人也。少以煮鹽自給,有勇力,走及奔馬。隋大業末,河間人格謙擁兵於豆子,開道往從之,署為將軍。後謙為隋師所滅,開道與其黨百餘人亡匿
 海曲。復出掠滄州,招集得數百人,北掠城鎮,臨渝至於懷遠,皆破之,悉有其眾。武德元年,隋將李景守北平郡,開道引兵圍之,連年不能克。景自度不能支,拔城而去。開道又取其地,進陷漁陽郡,有馬數千匹,眾且萬人,自立為燕王,都於漁陽。先是,有懷戎沙門高曇晟者,因縣令設齋,士女大集,曇晟與其僧徒五十人擁齋眾而反,殺縣令及鎮將,自稱大乘皇帝,立尼靜宣為耶輸皇后,建元為法輪。至夜,遣人招誘開道,結為兄弟,改封齊王。
 開道以眾五千人歸之,居數月,襲殺曇晟,悉並其眾。



 三年,復稱燕王,建元,署置百官。羅藝在幽州,為竇建德所圍,告急於開道,乃率二千騎援之。建德懼其驍銳,於是引去。開道因藝遣使來降,詔封北平郡王,賜姓李氏,授蔚州總管。時幽州大饑,開道許給之粟,藝遣老弱就食,開道皆厚遇之。藝甚悅,不以為虞,乃發兵三千人、車數百乘、驢馬千餘匹,請粟於開道。悉留之,北連突厥,告絕於藝,復稱燕國。



 是歲,劉黑闥入寇山東,開道與之連和,
 引兵攻易州,不克而退。又遣其將謝稜詐降於藝,請兵援接,藝出兵應之,將至懷戎,稜襲破藝兵。開道又引突厥頻來為寇,恆、定、幽、易等州皆罹其患。突厥頡利可汗攻馬邑,以開道兵善為攻具,引之陷馬邑而去。時天下大定,開道欲降,自以數翻復,終恐致罪,又北恃突厥之眾。其將士多山東人,思還本士,人心頗離。先是,劉黑闥亡將張君立奔於開道,因與其將張金樹潛相結連。時開道親兵數百人,皆勇敢士也,號為「義兒」,常在閣內。金
 樹每督兵於閣下。金樹將圍開道,潛令數人入其閣內,與諸義兒陽為游戲,至日將夕,陰斷其弓弦,又藏其刀,伏聚其槊於床下。迨暝,金樹以其徒大呼來攻閣下,向所遣人抱義兒槊一時而出,諸義兒遽將出戰,而弓弦皆絕,刀仗已失。君立於外城舉火相應,表裏驚擾。義兒窮蹙,爭歸金樹。開道知不免,於是擐甲持兵坐堂上,與其妻妾樂酣宴。金樹之黨憚其勇,不敢逼。天將曉,開道先縊其妻妾及諸子而後自殺。金樹陳兵,執其義兒,皆
 斬之。又殺張君立,死者五百餘人,遂歸國。開道自初起至滅,凡八歲。以其地為媯州。



 劉黑闥,貝州漳南人。無賴,嗜酒,好博弈,不治產業,父兄患之。與竇建德少相友善,家貧,無以自給,建德每資之。隋末亡命,從郝孝德為群盜,後歸李密為裨將。密敗,為王世充所虜。世充素聞其勇,以為騎將。見世充所為而竊笑之,乃亡歸建德,建德署為將軍,封漢東郡公,
 令將奇兵東西掩襲。黑闥既遍游諸賊,善觀時變,素驍勇,多奸詐。建德有所經略,必令專知斥候,常間入敵中覘視虛實,或出其不意,乘機奮擊,多所克獲,軍中號為神勇。及建德敗,黑闥自匿於漳南,杜門不出。會高祖征建德故將,範願、董康買、曹湛、高雅賢等將赴長安,願等相與謀曰:「王世充以洛陽降,其下驍將公卿、單雄信之徒皆被夷滅,我輩若至長安,必無保全之理。且夏王往日擒獲淮安王,全其性命,遣送還之。唐家今得夏王,即
 加殺害,我輩殘命,若不起兵報仇,實亦恥見天下人物。」於是相率復謀反叛。卜以劉氏為主吉,共往漳南,見建德故將劉雅告之,且請。雅曰:「天下已平,樂在丘園為農夫耳。起兵之事,非所願也。」眾怒,殺雅而去。範願曰:「漢東公劉黑闥果敢多奇略,寬仁容眾,恩結於士卒。吾久常聞劉氏當有王者,今舉大事,欲收夏王之眾,非其人莫可。」遂往詣黑闥,以告其意。黑闥大悅,殺牛會眾,舉兵得百餘人,襲破漳南縣。貝州刺史戴元詳、魏州刺史權威
 合兵擊之,並為黑闥所敗,元詳及威皆沒於陣。黑闥盡收其器械及餘眾千餘人,於是範願、高雅賢等宿舊左右漸來歸附,眾至二千人。



 武德四年七月,設壇於漳南,祭建德,告以舉兵之意,自稱大將軍。淮安王神通、將軍秦武通、王行敏前後討之,皆為所敗。於是移書趙、魏,其建德將士,往往殺官吏以應。黑闥北連懷戎賊帥高開道,兵鋒甚銳,進至宗城,有眾數萬。黎州總管李世勣不能拒,棄城走保洺州。黑闥追擊破之,步卒五千人,皆歿
 於陣,世勣與武通僅以身免。黑闥又徵王琮為中書令,劉斌為中書侍郎,以掌文翰。遣使北連突厥,頡利可汗遣俟斤宋耶那,率胡騎從之。黑闥軍大振,進陷相州。半歲,悉復建德故地。兗州賊帥徐圓朗舉齊、兗之地以附於黑闥,其勢益張。



 五年正月,黑闥至相州,僭稱漢東王,建元為天造。以範願為左僕射,董康買為兵部尚書,高雅賢為右領軍,又引建德時文武悉復本位,都於洺州。其設法行政,皆師建德而攻戰勇決過之。於是太宗又
 自請統兵討之,師次衛州,黑闥數以兵挑戰,輒為官軍所挫。黑闥懼,委相州,而退保於列人營。時洺水縣人請為內應,太宗遣總管羅士信入城據守,黑闥又攻陷其城,士信死之,遂據洺州。三月,太宗阻洺水列營以逼之,分遣奇兵,斷其糧道。黑闥又數挑戰,太宗堅壁不應,以挫其鋒。黑闥城中糧盡,太宗度其必來決戰,預擁洺水上流,謂守堤吏曰:「我擊賊之日,候賊半度而決堰。」黑闥果率步騎二萬渡洺水而陣,與官軍大戰,賊眾大潰,水
 又大至,黑闥眾不得渡,斬首萬餘級,溺死者數千人。黑闥與範願等以千餘人奔於突厥,山東悉定。太宗遂引軍於河南以討徐圓朗。



 六月,黑闥復借兵於突厥,來寇山東。七月,至定州,其舊將曹湛、董康買先亡在鮮虞,復聚兵以應黑闥。高祖遣淮陽王道玄、原國公史萬寶討之,戰於下博,王師敗績,道玄死於陣,萬寶輕騎逃還。由是河北諸州盡叛,又降於黑闥,旬日間悉復故城,復都洺州。十一月,高祖遣齊王元吉擊之,遲留不進。又令隱
 太子建成督兵進討,頻戰大捷。六年二月,又大破之於館陶,黑闥引軍北走。建成與元吉合千餘騎屯於永濟渠,縱騎擊之,黑闥敗走,命騎將劉弘基追之。黑闥為王師所蹙,不得休息,道遠兵疲,比至饒陽,從者才百餘人,眾皆餒,入城求食。黑闥所署饒州刺史葛德威出門迎拜,延之入城。黑闥初不許,德威謬為誠敬,涕泣固請。黑闥乃進,至城傍,德威勒兵執之,送於建成,斬於洺州,山東復定。



 徐圓朗者,兗州人也。隋末,亡命為群盜,據本郡,縱兵略地,自瑯邪巳西,北至東平,盡有之,勝兵二萬餘人。初附於李密,密敗,歸王世充。及洛陽平,歸國,拜兗州總管,封魯郡公。高祖令葛國公盛彥師安輯河南,行至任城。會劉黑闥作亂,潛結於圓朗,因執彥師舉兵應黑闥,自稱魯王。黑闥以圓朗為大行臺元帥,兗、鄆、陳、杞、伊、洛、曹、戴等八州豪猾,皆殺其長吏以應之。太宗平黑闥,進師曹州,遣淮安王神通及李世勣攻之。圓朗數出戰,不利,城
 內百姓爭逾城降。圓朗窮蹙,與數騎棄城夜遁,為野人所殺,其地悉平。



 史臣曰:薛舉父子勇悍絕倫,性皆好殺,仁杲尤甚,無恩眾叛,雖猛何為?李軌竊據鷹揚,僭號河西,安隋朝官屬,不奪其財;破李贇甲兵,放還其眾,是其興也。及殺害謀主,崇信妖巫,眾叛親離,其亡也,宜哉!武周始為鼠竊,偶恣鴟張,不用君璋之謀,竟為突厥所殺。苑君璋及總餘眾,別生異圖,見頡利歸朝,亦是見機者也。黑闥、開道,勇而無謀,顧其行師,祗是狂賊,皆為麾下所殺,馭眾之道謬哉。



 贊曰:國無紀綱,盜興草澤。不有隋亂,焉知唐德?



\end{pinyinscope}