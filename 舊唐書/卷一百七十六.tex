\article{卷一百七十六}

\begin{pinyinscope}

 ○令狐楚弟定子緒綯綯抃子滈渙牛僧孺子蔚蔚子徽蕭俛弟傑俶從弟人放人放子廩李石弟福



 令孤楚,字殼士,自言國初十八學士德棻之裔。祖崇亮,綿州昌明縣令。父承簡,太原府功曹。家世儒素。楚兒童
 時已學屬文,弱冠應進士,貞元七年登第。桂管觀察使王拱愛其才,欲以禮闢召,懼楚不從,乃先聞奏而後致聘。楚以父掾太原,有庭闈之戀,又感拱厚意,登第後徑往桂林謝拱。不預宴游,乞歸奉養,即還太原,人皆義之。李說、嚴綬、鄭儋相繼鎮太原,高其行義,皆闢為從事。自掌書記至節度判官,歷殿中侍御史。



 楚才思俊麗。德宗好文,每太原奏至,能辨楚之所為,頗稱之。鄭儋在鎮暴卒,不及處分後事,軍中喧嘩,將有急變。中夜十數騎持
 刃迫楚至軍門,諸將環之,令草遺表。楚在白刃之中,搦管即成,讀示三軍,無不感泣,軍情乃安。自是聲名益重。丁父憂,以孝聞。免喪,徵拜右拾遺,改太常博士、禮部員外郎。母憂去官。服闋,以刑部員外郎征,轉職方員外郎、知制誥。



 楚與皇甫鎛、蕭俛同年登進士第。元和九年,鎛初以財賦得幸,薦俛、楚俱入翰林,充學士,遷職方郎中、中書舍人,皆居內職。時用兵淮西,言事者以師久無功,宜宥賊罷兵,唯裴度與憲宗志在殄寇。十二年夏,度自
 宰相兼彰義軍節度、淮西招撫宣慰處置使。宰相李逢吉與度不協,與楚相善。楚草度淮西招撫使制,不合度旨,度請改制內三數句語。憲宗方責度用兵,乃罷逢吉相任,亦罷楚內職,守中書舍人。



 元和十三年四月,出為華州刺史。其年十月,皇甫鎛作相,其月以楚為河陽懷節度使。十四年四月,裴度出鎮太原。七月,皇甫鎛薦楚入朝,自朝議郎授朝議大夫、中書侍郎、同平章事,與鎛同處臺衡,深承顧待。



 十五年正月,憲宗崩,詔楚為山陵
 使,仍撰哀冊文。時天下怒皇甫鎛之奸邪。穆宗即位之四日,群臣素服,班於月華門外,宣詔貶鎛,將殺之。會蕭俛作相,托中官救解,方貶崖州。物議以楚因鎛作相而逐裴度,群情共怒。以蕭俛之故,無敢措言。



 其年六月,山陵畢,會有告楚親吏贓污事發,出為宣歙觀察使。楚充奉山陵時,親吏韋正牧、奉天令於翬、翰林陰陽官等同隱官錢,不給工徒價錢,移為羨餘十五萬貫上獻。怨訴盈路,正牧等下獄伏罪,皆誅。楚再貶衡州刺史。



 時元稹
 初得幸,為學士,素惡楚與鎛膠固希寵,稹草楚衡州制,略曰:「楚早以文藝,得踐班資,憲宗念才,擢居禁近。異端斯害,獨見不明,密隳討伐之謀,潛附奸邪之黨。因緣得地,進取多門,遂忝臺階,實妨賢路。」楚深恨稹。



 長慶元年四月,量移郢州刺史,遷太子賓客,分司東都。二年十一月,授陜州大都督府長史、兼御史大夫、陜虢觀察使。制下旬日,諫官論奏,言楚所犯非輕,未合居廉察之任。上知之,遽令追制。時楚已至陜州,視事一日矣。復授賓客,
 歸東都。時年逢吉作相,極力援楚,以李紳在禁密沮之,未能擅柄。敬宗即位,逢吉逐李紳,尋用楚為河南尹、兼御史大夫。



 其年九月,檢校禮部尚書、汴州刺史、宣武軍節度、汴宋亳觀察等使。汴軍素驕,累逐主帥;前後韓弘兄弟,率以峻法繩之,人皆偷生,未能革志。楚長於撫理,前鎮河陽,代烏重胤移鎮滄州,以河陽軍三千人為牙卒,卒咸不願從,中路叛歸,又不敢歸州,聚於境上。楚初赴任,聞之,乃疾驅赴懷州,潰卒亦至,楚單騎喻之,咸令
 橐弓解甲,用為前驅,卒不敢亂。及蒞汴州,解其酷法,以仁惠為治,去其太甚,軍民咸悅,翕然從化,後竟為善地。汴帥前例,始至率以錢二百萬實其私藏,楚獨不取,以其羨財治廨舍數百間。



 太和二年九月,徵為戶部尚書。三年三月,檢校兵部尚書、東都留守、東畿汝都防禦使。其年十一月,進位檢校右僕射、鄆州刺史、天平軍節度、鄆曹濮觀察等使。奏故東平縣為天平縣。屬歲旱儉,人至相食,楚均富贍貧,而無流亡者。六年二月,改太原尹、
 北都留守、河東節度等使。楚久在並州,練其風俗,因人所利而利之,雖屬歲旱,人無轉徙。楚始自書生,隨計成名,皆在太原,實如故里。及是垂旄作鎮,邑老歡迎。楚綏撫有方,軍民胥悅。七年六月,入為吏部尚書,仍檢校右僕射。故事,檢校高官者,便從其班。楚以正官三品不宜從二品之列,請從本班,優詔嘉之。九年六月,轉太常卿。十月,守尚書左僕射,進封彭陽郡開國公。十一月,李訓兆亂,京師大擾。訓亂之夜,文宗召右僕射鄭覃與楚宿
 於禁中,商量制敕,上皆欲用為宰相。楚以王涯、賈餗冤死,敘其罪狀浮泛,仇士良等不悅,故輔弼之命移於李石。乃以本官領鹽鐵轉運等使。



 先是,鄭注上封置榷茶使額,鹽鐵使兼領之,楚奏罷之,曰:



 伏以江、淮數年已來,水旱疾疫,凋傷頗甚,愁嘆未平。今夏及秋,稍校豐稔,方須惠恤,各使安存。昨者忽奏榷茶,實為蠹政。蓋是王涯破滅將至,怨怒合歸,豈有令百姓移茶樹於官場中栽植,摘茶葉於官場中造作,有同兒戲,不近人情。方在恩
 權,孰敢沮議?朝班相顧而失色,道路以目而吞聲。今宗社降靈,奸兇盡戮,聖明垂祐,黎庶合安。微臣蒙恩,兼領使務,官銜之內,猶帶此名。俯仰若驚,夙宵知懼。伏乞特回聖聽,下鑒愚誠,速委宰臣,除此使額。緣軍國之用或闕,山澤之利有遺,許臣條疏,續具聞奏。採造將及,妨廢為虞。



 前月二十一日,內殿奏對之次,鄭覃與臣同陳論訖。伏望聖慈早賜處分,一依舊法,不用新條。唯納榷之時,須節級加價,商人轉賣,必校稍貴,即是錢出萬國,利歸
 有司。既不害茶商,又不擾茶戶,上以彰陛下愛人之德,下以竭微臣憂國之心。遠近傳聞,必當感悅。



 從之。



 先是,元和十年,出內庫弓箭陌刀賜左右街使,充宰相入朝以為翼衛,及建福門而止。至是,因訓、注之亂,悉罷之。楚又奏:「諸道新授方鎮節度使等,具帑抹,帶器仗,就尚書省兵部參辭。伏以軍國異容,古今定制,若不由舊,斯為改常。未聞省閣之門,忽內弓刀之器。鄭注外蒙恩寵,內蓄兇狂,首創奸謀,將興亂兆。致王璠、郭行餘之輩,敢驅
 將吏,直詣闕庭。震驚乘輿,騷動京國,血濺朝路,尸殭禁街。史冊所書,人神共憤,既往不咎,其源尚開。前件事宜,伏乞速令停罷,如須參謝,即具公服。」從之。又奏請罷修曲江亭絹一萬三千七百匹,回修尚書省,從之。



 開成元年上巳,賜百僚曲江亭宴。楚以新誅大臣,不宜賞宴,獨稱疾不赴,論者美之。以權在內官,累上疏乞解使務。其年四月,檢校左僕射、興元尹,充山南西道節度使。二年十一月,卒於鎮,年七十二,冊贈司空,謚曰文。



 楚風儀嚴
 重,若不可犯;然寬厚有禮,門無雜賓。嘗與從事宴語方酣,有非類偶至,立命徹席,毅然色變。累居重任,貞操如初。未終前三日,猶吟詠自若。疾甚,諸子進藥,未賞入口,曰:「修短之期,分以定矣,何須此物?」前一日,召從事李商隱曰:「吾氣魄已殫,情思俱盡,然所懷未已,強欲自寫聞天,恐辭語乖舛,子當助我成之。」即秉筆自書曰:



 臣永惟際會,受國深恩。以祖以父,皆蒙褒贈;有弟有子,並列班行。全腰領以從先人,委體魄而事先帝,此不自達,誠為
 甚愚。但以永去泉扃,長辭云陛,更陳尸諫,猶進瞽言。雖號叫而不能,豈誠明之敢忘?今陛下春秋鼎盛,寰海鏡清,是修教化之初,當復理平之始。然自前年夏秋已來,貶譴者至多,誅戮者不少,望普加鴻造,稍霽皇威。歿者昭洗以雲雷,存者沾濡以雨露,使五穀嘉熟,兆人安康。納臣將盡之苦言,慰臣永蟄之幽魄。



 書訖,謂其子緒、綯曰:「吾生無益於人,勿請謚號。葬日,勿請鼓吹,唯以布車一乘,餘勿加飾。銘志但志宗門,秉筆者無擇高位。」當
 歿之夕,有大星隕於寢室之上,其光燭廷。楚端坐與家人告訣,言已而終。嗣子奉行遺旨。詔曰:「生為名臣,歿有理命。終始之分,可謂兩全。鹵簿哀榮之末節,難違往意;誄謚國家之大典,須守彞章。鹵簿宜停,易名須準舊例。」後綯貴,累贈至太尉。有文集一百卷,行於時。所撰《憲宗哀冊文》,辭情典鬱,為文士所重。



 楚弟定,字履常。元和十一年進士及第,累闢使府。太和九年,累遷至職方員外郎、弘文館直學士、檢校右散騎常侍、桂州刺史、桂管都防
 禦觀察等使。卒,贈禮部尚書。



 緒以廕授官,歷隨、壽、汝三郡刺史。在汝州日,有能政,郡人請立碑頌德。緒以弟襜在輔弼,上言曰:「臣先父元和中特承恩顧,弟綯官不因人,出自宸衷。臣伏睹詔書,以臣刺汝州日,粗立政勞,吏民求立碑頌,尋乞追罷。臣任隨州日,郡人乞留,得上下考。及轉河南少尹,加金紫。此名已聞於日下,不必更立碑頌,乞賜寢停。」宣宗嘉其意,從之。



 綯字子直。太和四年登進士第,釋褐弘文館校書郎。開成初為左拾遺。二年,
 丁父喪。服闋,授本官,尋改左補闕、史館修撰,累遷庫部、戶部員外郎。會昌五年,出為湖州刺史。大中二年,召拜考功郎中,尋知制誥。其年,召入充翰林學士。三年,拜中書舍人,襲封彭陽男,食邑三百戶,尋拜御史中丞。四年,轉戶部侍郎,判本司事。其年,改兵部侍郎、同中書門下平章事。綯以舊事帶尚書省官,合先省上。上日同列集於少府監。時白敏中、崔龜從曾為太常博士,至相位,欲榮其舊署,乃改集於太常禮院,龜從手筆志其事於壁。



 綯輔政十年,累官至吏部尚書、右僕射、涼國公,食邑二千戶。十三年,罷相,檢校司空、同中書門下平章事、河中尹、河中晉絳等節度使。



 咸通二年,改汴州刺史、宣武軍節度使。三年冬,遷揚州大都督府長史、淮南節度副大使、知節度事。累加開府儀同三司、檢校司徒,進食邑至三千戶。



 九年,徐州戍兵龐勛自桂州擅還。七月至浙西,沿江自白沙入濁河,剽奪舟船而進。綯聞勛至,遣使慰撫,供給芻米。都押衙李湘白綯曰:「徐兵擅還,必無好意。
 雖無詔命除討,權變制在籓方。昨其黨來投,言其數不逾二千,而虛張舟航旗幟,恐人見其實。涉境已來,心頗憂惴。計其水路,須出高郵縣界,河岸斗峻而水深狹。若出奇兵邀之,俾荻船縱火於前,勁兵奮擊於後,敗走必矣。若不於此誅鋤,俟濟淮、泗,合徐人負怨之徒,不下十萬,則禍亂非細也。」綯性懦緩,又以不奉詔命,謂湘曰:「長淮已南,他不為暴。從他過去,餘非吾事也。」



 其年冬,龐勛殺崔彥曾,據徐州,聚眾六七萬。徐無兵食,乃分遣賊帥
 攻剽淮南諸郡,滁、和、楚、壽繼陷。穀食既盡,淮南之民多為賊所啖。時兩淮郡縣多陷,唯杜慆守泗州,賊攻之經年,不能下。初,詔綯為徐州南面招討使。賊攻泗州急,綯令李湘將兵五千人援之。賊聞湘來援,遣人致書於綯,辭情遜順,言:「朝廷累有詔赦宥,但抗拒者三兩人耳,旦夕圖去之,即束身請命,願相公保任之。」綯即奏聞,請賜勛節鉞,仍誡李湘但戍淮口,賊已招降,不得立異。由是湘軍解甲安寢,去警徹備,日與賊軍相對,歡笑交言。一日,
 賊軍乘間,步騎徑入湘壘,淮卒五千人皆被生縶送徐州,為賊蒸而食之。湘與監軍郭厚本為龐勛斷手足,以徇於康承訓軍。時浙西杜審權發軍千人,與李湘約會兵,大將翟行約勇敢知名。浙軍未至而湘軍敗。賊乃分兵,立淮南旗幟,為交鬥之狀。行約軍望見,急趨之,千人並為賊所縛。送徐州。



 綯既喪師,朝廷以左衛大將軍、徐州西南面招討使馬舉代綯為淮南節度使。十二年八月,授檢校司徒、太子太保,分司東都。十三年,以本官
 為鳳翔尹、鳳翔隴節度使,進封趙國公,食邑三千戶,卒。子滈、渙、渢。



 滈,少舉進士,以父在內職而止。及綯輔政十年,滈以鄭顥之親,驕縱不法,日事游宴,貨賄盈門,中外為之側目。以綯黨援方盛,無敢措言。及懿宗即位,訟者不一,故綯罷權軸。既至河中,上言曰:「臣男滈,爰自孩提,便從師訓,至於詞藝,頗及輩流。會昌二年,臣任戶部員外郎時,已令應舉,至大中二年猶未成名。臣自湖州刺史蒙先帝擢授考功郎中、知制誥,尋充學士。繼叨渥澤,
 遂忝樞衡,事體有妨,因令罷舉,自當廢絕,一十九年。每遣退藏,更令勤勵。臣以祿位逾分,齒發已衰。男滈年過長成,未沾一第,犬馬私愛,實切憫傷。臣二三年來,頻乞罷免,每年取得文解,意待才離中書,便令赴舉。昨蒙恩制,寵以近籓。伏緣已逼禮部試期,便令就試。至於與奪,出自主司,臣固不敢撓其衡柄。臣初離機務,合具上聞。昨延英奉辭,本擬面奏,伏以戀恩方切,陳誠至難。伏冀宸慈,察臣丹懇。」詔令就試。



 是歲,中書舍人裴坦權知貢
 舉,登第者三十人。有鄭羲者,故戶部尚書浣之孫,裴弘餘,故相休之子,魏綯故相扶之子,及滈,皆名臣子第,言無實才。諫議大夫崔瑄上疏論之曰:「令狐滈昨以父居相位,權在一門。求請者詭黨風趨,妄動者群邪雲集。每歲貢闈登第,在朝清列除官,事望雖出於綯,取舍全由於滈。喧然如市,旁若無人,權動寰中,勢傾天下。及綯罷相作鎮之日,便令滈納卷貢闈。豈可以父在樞衡,獨撓文柄?請下御史臺按問文解日月者。」奏疏不下。



 滈既及
 第,釋褐長安尉、集賢校理。咸通二年,遷右拾遺、史館修撰。制出,左拾遺劉蛻、起居郎張云,各上疏極論滈云:「恃父秉權,恣受貨賂。取李琢錢,除琢安南都護,遂致蠻陷交州。」張云言:「大中十年,襜以諫議大夫豆盧籍、刑部郎中李鄴為夔王已下侍讀,欲立夔王為東宮,欲亂先朝子弟之序。滈內倚鄭顥,人誰敢言?」時襜在淮南,累表自雪。懿宗重傷大臣意,貶云為興元少尹,蛻為華陰令,改滈詹事府司直。滈為眾所非,宦名不達。



 渙、渢俱登進士
 第。渙位至中書舍人。定子緘,緘子澄、湘。澄亦以進士登第,累闢使府。



 牛僧孺,字思黯,隋僕射奇章公弘之後。祖紹。父幼簡,官卑。僧孺進士擢第,登賢良方正制科,釋褐伊闕尉,遷監察御史,轉殿中,歷禮部員外郎。元和中,改都官,知臺雜,尋換考功員外郎,充集賢直學士。



 穆宗即位,以庫部郎中知制誥。其年十一月,改御史中丞。以州府刑獄淹滯,人多冤抑,僧孺條疏奏請,按劾相繼,中外肅然。



 長慶元
 年,宿州刺史李直臣坐贓當死,直臣賂中貴人為之申理,僧孺堅執不回。穆宗面喻之曰:「直臣事雖僭失,然此人有經度才,可委之邊任,朕欲貸其法。」僧孺對曰:「凡人不才,止於持祿取容耳。帝王立法,束縛奸雄,正為才多者。祿山、硃泚以才過人,濁亂天下,況直臣小才,又何屈法哉?」上嘉其守法,面賜金紫。二年正月,拜戶部侍郎。三年三月,以本官同平章事。



 初,韓弘入朝,以宣武舊事,人多流言,其子公武以家財厚賂權幸及多言者,班列之
 中,悉受其遺。俄而父子俱卒,孤孫幼小,穆宗恐為廝養竊盜,乃命中使至其家,閱其宅簿,以付家老。而簿上具有納賂之所,唯於僧孺官側硃書曰:「某月日,送牛侍郎物若干,不受,卻付訖。」穆宗按簿甚悅。居無何,議命相,帝首可僧孺之名。



 敬宗即位,加中書侍郎、銀青光祿大夫,封奇章子,邑五百戶。十二月,加金紫階,進封郡公、集賢殿大學士、監修國史。



 寶歷中,朝廷政事出於邪幸,大臣朋比。僧孺不奈群小,拜章求罷者數四。帝曰:「俟予郊禮
 畢放卿。」及穆宗祔廟郊報後,又拜章陳退,乃於鄂州置武昌軍額,以僧孺檢校禮部尚書、同中書門下平章事、鄂州刺史、武昌軍節度、鄂岳蘄黃觀察等使。江夏城風土散惡,難立垣墉,每年加板築,賦青茆以覆之。吏緣為奸,蠹弊綿歲。僧孺至,計茆苫板築之費,歲十餘萬,即賦之以專,以當苫築之價。凡五年,墉皆甃葺,蠹弊永除。屬郡沔州與鄂隔江相對,虛張吏員,乃奏廢之,以其所管漢陽、汶川兩縣隸鄂州。文宗即位,就加檢校吏部尚書,
 凡鎮江夏五年。



 太和三年,李宗閔輔政,屢薦僧孺有才,不宜居外。四年正月,召還,守兵部尚書、同平章事。



 五年正月,幽州軍亂,逐其帥李載義。文宗以載義輸忠於國,遽聞失帥,駭然,急召宰臣謂之曰:「範陽之變奈何?」僧孺對曰:「此不足煩聖慮。且範陽得失,不系國家休戚,自安、史已來,翻覆如此。前時劉總以土地歸國,朝廷耗費百萬,終不得範陽尺帛斗粟入於天府,尋復為梗。至今志誠,亦由前載義也,但因而撫之,俾捍奚、契丹不令入寇,
 朝廷所賴也。假以節旄,必自陳力,不足以逆順治之。」帝曰:「吾初不祥,思卿言是也。」即日命中使宣慰。尋加門下侍郎、弘文館大學士。



 六年,吐蕃遣使論董勃義入朝修好。俄而西川節度李德裕奏,吐蕃維州守將悉怛謀以城降。德裕又上利害云:「若以生羌三千,出戎不意,燒十三橋,搗戎之腹心,可以得志矣。」上惑其事,下尚書省議,眾狀請如德裕之策。僧孺奏曰:「此議非也。吐蕃疆土,四面萬里,失一維州,無損其勢。況論董勃義才還,劉元鼎
 未到,比來修好,約罷戍兵。中國御戎,守信為上,應敵次之,今一朝失信,戎醜得以為詞。聞贊普牧馬茹川,俯於秦、隴。若東襲隴阪,徑走回中,不三日抵咸陽橋,而發兵枝梧,駭動京國。事或及此,雖得百維州,亦何補也。」上曰:「然。」遂詔西川不內維州降將。僧孺素與德裕仇怨,雖議邊公體,而怙德裕者以僧孺害其功,謗論沸然,帝亦以為不直。其年十二月,檢校左僕射、兼平章事、揚州大都督府長史、淮南節度副大使、知節度事。



 時中尉王守澄
 用事,多納纖人,竊議時政,禁中事密,莫知其說。一日,延英對宰相,文宗曰:「天下何由太平,卿等有意於此乎?」僧孺奏曰:「臣等待罪輔弼,無能康濟,然臣思太平亦無象。今四夷不至交侵,百姓不至流散;上無淫虐,下無怨讟;私室無強家,公議無壅滯。雖未及至理,亦謂小康。陛下若別求太平,非臣等所及。」既退至中書,謂同列曰:「吾輩為宰相,天子責成如是,安可久處茲地耶?」旬日間,三上章請退,不許。



 會德裕黨盛,垂將入朝,僧孺故得請。上既
 受左右邪說,急於太平,奸人伺其銳意,故訓、注見用。數年之間,幾危宗社,而僧孺進退以道,議者稱之。



 開成初,搢紳道喪,閽寺弄權,僧孺嫌處重籓,求歸散地,累拜章不允,凡在淮甸六年。



 開成二年五月,加檢校司空,食邑二千戶,判東都尚書省事、東都留守、東畿汝都防禦使。



 僧孺識量弘遠,心居事外,不以細故介懷。洛都築第於歸仁里。任淮南時,嘉木怪石,置之階廷,館宇清華,竹木幽邃。常與詩人白居易吟詠其間,無復進取之懷。



 三年
 九月,徵拜左僕射,仍令左軍副使王元直齎告身宣賜。舊例,留守入朝,無中使賜詔例,恐僧孺退讓,促令赴闕。僧孺不獲已入朝。屬莊恪太子初薨,延英中謝日,語及太子,乃懇陳父子君臣之義,人倫大經,不可輕移國本。上為之流涕。是時宰輔皆僧孺僚舊,未嘗造其門。上頻宣召,托以足疾。久之,上謂楊嗣復曰:「僧孺稱疾,不任趨朝,未可即令自便。」四年八月,復檢校司空、兼平章事、襄州刺史、山南東道節度使,加食邑至三千戶。辭日,賜觚、
 散、樽、杓等金銀古器,令中使喻之曰:「以卿正人,賜此古器,卿且少留。」僧孺奏曰:「漢南水旱之後,流民待理,不宜淹留。」再三請行,方允。



 武宗即位,就加檢校司徒。會昌二年,李德裕用事,罷僧孺兵權,徵為太子少保,累加太子少師。大中初卒,贈太子太師,謚曰文貞。



 僧孺少與李宗閔同門生,尤為德裕所惡。會昌中,宗閔棄斥,不為生還。僧孺數為德裕掎摭,欲加之罪,但以僧孺貞方有素,人望式瞻,無以伺其隙。德裕南遷,所著《窮愁志》,引里俗犢
 子之讖以斥僧孺。又目為「太牢公」,其相憎恨如此。僧孺二子:蔚、。



 蔚,字大章,十五應兩經舉。太和九年,復登進士第。三府闢署為從事,入朝為監察御史。大中初,為右補闕,屢陳章疏,指斥時病。宣宗嘉之,曰:「牛氏子有父風,差慰人意。」尋改司門員外郎,出為金州刺史,入拜禮、吏二郎中。以祀事準禮,天官司所掌班列,有恃權越職者,蔚奏正之,為時權所忌,左授國子博士,分司東都。逾月,權臣罷免,復徵為吏部郎中,兼史館修撰,遷左諫議大
 夫。咸通中,為給事中,延英謝日,面賜金紫。蔚封駁無避,帝嘉之。逾歲,遷戶部侍郎,襲封奇章侯,以公事免。歲中復本官,歷工、禮、刑三尚書。咸通末,檢校兵部尚書、興元尹、山南西道節度使。在鎮三年。時中官用事,急於賄賂。屬徐方用兵,兩中尉諷諸籓貢奉助軍,蔚盡索軍府之有三萬端匹,隨表進納。中官怒,即以神策將吳行魯代還。及黃巢犯闕,乃自京師奔遁,避地山南,拜章請老,以尚書左僕射致仕。卒,累贈太尉。子循、徽。



 徽,咸通八年登
 進士第,三佐諸侯府,得殿中侍御史,賜緋魚。入朝為右補闕,再遷吏部員外郎。乾符中,選曹猥濫,吏為奸弊,每歲選人四千餘員。徽性貞剛,特為奏請。由是銓敘稍正,能否旌別,物議稱之。



 巢賊犯京師,父蔚方病,徽與其子自扶籃輿,投竄山南。閣路險狹,盜賊縱橫,谷中遇盜,擊徽破首,流血被體,而捉輿不輟。盜苦迫之,徽拜之曰:「父年高疾甚,不欲駭動。人皆有父,幸相垂恤。」盜感之而止。及前谷,又逢前盜,相告語曰:「此孝子也。」即同舉輿,延於
 其家,以帛封創,饘飲奉蔚。留之信宿,得達梁州。故吏感恩,爭來奔問。時僖宗已幸成都,徽至行朝拜章,乞歸侍疾。已除諫議大夫,不拜。謂宰相杜讓能曰:「願留兄循在朝,以當門戶,乞侍醫藥。」時循為給事中,丞相許之。



 其年鐘家艱,執喪梁、漢。既除,以中書舍人征,未赴,疾作。以舍人綸制之地,不可曠官,請授散秩,改給事中。從駕還京,至陳倉,疾甚,經年方間。



 宰相張浚為招討使,奏徽為判官,檢校左散騎常侍。詔下鳳翔,促令赴闕。徽謂所親曰:「
 國步方艱,皇居初復,帑廩皆虛,正賴群臣協力,同心王室。而於破敗之餘,圖雄霸之舉,俾諸侯離心,必貽後悔也。以吾衰疾之年,安能為之捍難。」辭疾不起。明年,浚敗,召徽為給事中。



 楊復恭叛歸山南,李茂貞上表,請自出兵糧問罪,但授臣詔討使。奏不待報,茂貞與王行瑜軍已出疆。上怒其專,不時可之,茂貞恃強,章疏不已。昭宗延英召諫官宰相議可否。以邠、鳳皆有中人內應,不敢極言,相顧辭遜,上情不悅。徽奏曰:「兩朝多艱,茂貞實有
 翼衛之功,惡諸楊阻兵,意在嫉惡。所造次者,不俟命而出師也。近聞兩鎮兵入界,多有殺傷,陛下若不處分,梁、漢之民盡矣。須授以使名,明行約束,則軍中爭不畏法。」帝曰:「此言極是。」乃以招討之命授之。及茂貞平賊,自恃浸驕,多撓國政,命杜讓能料兵討之。徽諫曰:「岐是國門,茂貞倔強,不顧禍患。萬一蹉跌,挫國威也,不若漸以制之。」及師出,復召徽謂之曰:「卿能斟酌時事,岐軍烏合,朕料必平,卿以為捷在何日?」徽對曰:「臣忝侍從諫諍之列,
 所言軍國,據理陳聞。如破賊之期,在陛下考蓍龜,責將帥,非臣之職也。」而王師果衄,大臣被害。



 徽尋改中書舍人。歲中,遷刑部侍郎,封奇章男。崔胤連結汴州,惡徽言事,改散騎常侍。不拜,換太子賓客。天復初,賊臣用事,朝政不綱,拜章請罷。詔以刑部尚書致仕,乃歸樊川別墅。病卒,贈吏部尚書。



 ,字表齡,開成二年登進士第,出佐使府,歷踐臺省。乾符中,位至劍南西川節度使。黃巢之亂,從幸西川,拜太常卿。以病求為巴州刺史,不許。駕還,
 拜吏部尚書。襄王之亂,避地太原,卒。子蟜,位至尚書郎。



 蕭俛,字思謙。曾祖太師徐國公嵩,開元中宰相。祖華,襲徐國公,肅宗朝宰相。父恆,贈吏部尚書。皆自有傳。俛,貞元七年進士擢第。元和初,復登賢良方正制科,拜右拾遺,遷右補闕。元和六年,召充翰林學士。七年,轉司封員外郎。九年,改駕部郎中、知制誥,內職如故。坐與張仲方善,仲方駁李吉甫謚議,言用兵徵發之弊,由吉甫而生。憲宗怒,貶仲方。俛亦罷學士,左授太僕少卿。



 十三年,皇
 甫鎛用事,言於憲宗,拜俛御史中丞。俛與鎛及令狐楚,同年登進士第。明年,鎛援楚作相,二人雙薦俛於上。自是,顧眄日隆,進階朝議郎、飛騎尉,襲徐國公,賜緋魚袋。穆宗即位之月,議命宰相,令狐楚援之,拜中書侍郎、平章事,仍賜金紫之服。八月,轉門下侍郎。



 十月,吐蕃寇涇原,命中使以禁軍援之。穆宗謂宰臣曰:「用兵有必勝之法乎?」俛對曰:「兵者兇器,戰者危事,聖主不得已而用之。以仁討不仁,以義討不義,先務招懷,不為掩襲。古之用
 兵,不斬祀,不殺厲,不擒二毛,不犯田稼。安人禁暴,師之上也。如救之甚於水火。故王者之師,有征無戰,此必勝之道也。如或縱肆小忿,輕動干戈,使敵人怨結,師出無名,非惟不勝,乃自危之道也。固宜深慎!」帝然之。



 時令狐楚左遷西川節度使,王播廣以貨幣賂中人權幸,求為宰相。而宰相段文昌復左右之。俛性嫉惡,延英面言播之纖邪納賄,喧於中外,不可以污臺司。事已垂成,帝不之省,俛三上章求罷相任。長慶元年正月,守左僕射,進
 封徐國公,罷知政事。俛居相位,孜孜正道,重慎名器。每除一官,常慮乖當,故鮮有簡拔而涉克深,然志嫉奸邪,脫屣重位,時論稱之。



 穆宗乘章武恢復之餘,即位之始,兩河廓定,四鄙無虞。而俛與段文昌屢獻太平之策,以為兵以靜亂,時已治矣,不宜黷武,勸穆宗休兵偃武。又以兵不可頓去,請密語天下軍鎮有兵處,每年百人之中,限八人逃死,謂之「消兵」。帝既荒縱,不能深料,遂詔天下,如其策而行之。而籓籍之卒,合而為盜,伏於山林。明
 年,硃克融、王廷湊復亂河朔,一呼而遺卒皆至。朝廷方徵兵諸籓,籍既不充,尋行招募。烏合之徒,動為賊敗,由是復失河朔,蓋「消兵」之失也。



 俛性介獨,持法守正。以己輔政日淺,超擢太驟,三上章懇辭僕射,不拜。詔曰:「蕭俛以勤事國,以疾退身,本末初終,不失其道,既罷樞務,俾居端揆。朕欲加恩超等,復吾前言。而繼有讓章,至於三四,敦諭頗切,陳乞彌堅。成爾謙光,移之選部,可吏部尚書。」俛又以選曹簿書煩雜,非攝生之道,乞換散秩。其年
 十月,改兵部尚書。二年,以疾表求分司,不許。三月,改太子少保,尋授同州刺史。寶歷二年,復以少保分司東都。



 文宗即位,授檢校左僕射、守太子少師。俛稱疾篤,不任赴闕,乞罷所授官。詔曰:「新除太子少師蕭俛,代炳臺耀,躬茂天爵。文可以經緯邦俗,行可以感動神祇。夷澹粹和,精深敏直,進退由道,周旋令名。近以師傅之崇,疇於舊德,俾從優逸,冀保養頤。而抗疏懇辭,勇退知止,嘗亦敦諭,確乎難拔。遂茲牢讓,以厚時風,可銀青光祿大夫、
 守尚書左僕射致仕。」



 俛趣尚簡潔,不以聲利自污。在相位時,穆宗詔撰《故成德軍節度使王士真神道碑》,對曰:「臣器褊狹,此不能強。王承宗先朝阻命,事無可觀,如臣秉筆,不能溢美。或撰進之後,例行貺遺。臣若公然阻絕,則違陛下撫納之宜;僶俛受之,則非微臣平生之志。臣不願為之秉筆。」帝嘉而免之。



 俛家行尤孝。母韋氏,賢明有禮,理家甚嚴。俛雖為宰相,侍母左右,不異褐衣時。丁母喪,毀瘠逾制。免喪,文宗征詔,懇以疾辭。既致仕於家,
 以洛都官屬賓友,避歲時請謁之煩,乃歸濟源別墅,逍遙山野,嘯詠窮年。



 八年,以莊恪太子在東宮,上欲以耆德輔導,復以少師征之。俛令弟傑奉表京師,復納制書,堅辭痼疾。詔曰:「不待年而求謝,於理身之道則至矣,其如朝廷之望何?朕以肇建元良,精求師傅,遐想漢朝故事,玄成、石慶,當時重德,咸歷此官。吾以元子幼沖,切於師訓,欲以敕汝發明古今,冀忠孝之規,日聞於耳。特遣左右,至於林園。而卿高蹈翛然,屏絕趨進,復遣令弟還
 召詔書。天爵自優,冥鴻方遠,不轉之志,其堅若山。循省來章,致煩為愧。終以呂尚之秩,遂其疏曠之心。勵俗激貪,所補多矣。有益於政,寄聲以聞,亦有望於舊臣矣。可太子太傅致仕。」



 開成二年,俛弟俶授楚州刺史。辭日,文宗謂俶曰:「蕭俛先朝名相,觔力未衰,可一來京國。朕賜俛詔書匹帛,卿便齎至濟源,道吾此意。」詔曰:「卿道冠時髦,業高儒行。著作礪濟川之效,弘致君匡國之規,留芳巖廊,逸老林壑。累降褒詔,亟加崇秩,而志不可奪,情見
 乎辭。鴻飛入冥,吟想增嘆。今賜絹三百匹,便令蕭俶宣示。」俛竟不起,卒。



 傑,字豪士。元和十二年登進士第。累官侍御史,遷主客員外郎。太和九年十月,鄭注為鳳翔節度使,慎選參佐。李訓以傑檢校工部郎中,充鳳翔隴觀察判官。其年十一月,鄭注誅,傑為鳳翔監軍使所害。



 俶以廕授官。太和中,累遷至河南少尹。九年五月,拜諫議大夫。開成二年,出為楚州刺史。四年三月,遷越州刺史、御史中丞、浙東都團練觀察使。會昌中,入為左散騎常
 侍,遷檢校刑部尚書、華州刺史、潼關防禦等使。大中初,坐在華州時斷獄不法,授太子賓客分司。四年,檢校戶部尚書、兗州刺史、兗沂海節度使。復入為太子賓客。大中十二年,以太子少保分司東都,卒。俛從父弟仿。



 人放,父悟,恆之弟也。悟,仕至大理司直。人放,太和元年登進士第。大中朝,歷諫議大夫、給事中。咸通初,遷左散騎常侍。



 懿宗怠臨朝政,僻於奉佛,內結道場,聚僧念誦。又數幸諸寺,施與過當。人放上疏論之曰:



 臣聞玄祖之道,由慈儉為
 先;而素王之風,以仁義為首。相沿百代,作則千年,至聖至明,不可易也。如佛者,生於天竺,去彼王宮,割愛中之至難,取滅後之殊勝,名歸象外,理絕塵中,非為帝王之所能慕也。昔貞觀中,高宗在東宮,以長孫皇后疾亟,嘗上言曰:「欲請度僧,以資福事。」後曰:「為善有征,吾未為惡,善或無報,求福非宜。且佛者,異方之教,所可存而勿論。豈以一女子而紊王道乎?」故謚為文德。且母後之論,尚能如斯,哲王之謨,安可反是?



 伏睹陛下留神天竺,屬意
 桑門,內設道場,中開講會,或手錄梵策,或口揚佛音。雖時啟於延英,從容四輔;慮稍稀於聽政,廢失萬機。居安思危,不可忽也。夫從容者,君也,必疇咨於臣,盡忠匡救,外逆其耳,內沃其心;陳皋陶之謨,述仲虺之誥;發揮王道,恢益帝圖,非賜對之間,徒侍坐而已。夫廢失者,上拒其諫,下希其旨,言則狎玩,意在順從。漢重神仙,東方朔著《十洲》之記;梁崇佛法,劉孝儀詠《七覺》之詩。致祠禱無休,講誦不已,以至大空海內,中輟江東。以此言之,是廢
 失也。然佛者,當可以悟取,不可以相求。漢、晉已來,互興寶剎;姚、石之際,亦有高僧。或問以苦空,究其不滅,止聞有性,多曰忘言。執著貪緣,非其旨也。必乞陛下力求民瘼,虔奉宗祧。思繆賞與濫刑,其殃立至;俟勝殘而去殺,得福甚多。幸罷講筵,頻親政事。昔年韓愈已得罪於憲宗,今日微臣固甘心於遐徼。



 疏奏,帝甚嘉之。



 四年,本官權知貢舉,遷禮部侍郎,轉戶部。以檢校工部尚書,出為滑州刺史,充義成軍節度、鄭滑潁觀察處置等使。在鎮
 四年,滑臨黃河,頻年水潦,河流泛溢,壞西北堤。人放奏移河四里,兩月畢功,畫圖以進。懿宗嘉之,就加刑部尚書,入為兵部尚書、判度支,轉吏部尚書,選序平允。咸通末,復為兵部尚書、判度支。尋以本官同平章事,累遷中書、門下二侍郎,兼戶部、兵部尚書。遷左右僕射,改司空、弘文館大學士、蘭陵郡開國侯。



 俄而盜起河南,內官握兵,王室濁亂。人放氣勁論直,同列忌之;罷知政事,出為廣州刺史、嶺南節度使。



 人放性公廉,南海雖富珍奇,月俸之外,
 不入其門。家人疾病,醫工治藥,須烏梅,左右於公廚取之;人放知而命還,促買於市。遇亂,不至京師而卒。



 子廩,咸通三年進士擢第,累遷尚書郎。乾符中,以父出鎮南海,免官侍行。中和中,徵為中書舍人,再遷京兆尹。僖宗再幸山南,廩以疾不能從。襄王僭竊,廩宗人遘受偽署;廩懼,自洛避地河朔,鎮冀節度使王鎔館之於深州。光化三年卒。



 廩貞退寡合,綽有家法。初從父南海,地多穀紙,人放敕子弟繕寫缺落文史。廩白曰:「家書缺者,誠宜補葺。
 然此去京師,水際萬里,不可露齎,當須篋笥。人觀兼乘,謂是貨財,古人薏苡之嫌,得為深誡。」人放曰:「吾不之思也。」故濁亂之際,克保令名。



 子頎,亦登進士第,後官位顯達。



 李石,字中玉,隴西人。祖堅,父明。石,元和十三年進士擢第,從涼國公李聽歷四鎮從事。石機辯有方略,尤精吏術,籓府稱之。自聽征伐,常司留使務,事無不辦。太和三年,為鄭滑行軍司馬。時聽握兵河北,令石入朝奏事,占對明辯,文宗目而嘉之。府罷,入為工部郎中,判鹽鐵案。
 五年,改刑部郎中。由兵部郎中令狐楚請為太原節度副使。七年,拜給事中。九年七月,權知京兆尹事。十月,遷戶部侍郎,判度支事。



 文宗自德裕、宗閔朋黨相傾。太和七年以後,宿素大臣,穎而不用。意在擢用新進孤立,庶幾無黨,以革前弊,故賈餗、舒元輿驟階大用。及訓、注伏誅,欲用令狐楚,尋而中輟。石自朝議郎加朝議大夫,以本官同平章事,判使如故。石器度豁如,當官不撓。自京師變亂之後,宦者氣盛,凌轢南司,延英議事,中貴語必
 引訓以折文臣。石與鄭覃嘗謂之曰:「京師之亂,始自訓、注;而訓、注之起,始自何人?」仇士良等不能對。其勢稍抑,縉紳賴之。是時,逾月,人情不安。帝謂侍臣曰:「如聞人心尚未安帖,比日何如?」石對曰:「比日苦寒,蓋刑殺太過,致此陰沴。昨聞鄭注到鳳翔,招募士卒不至,捕索誅夷不已,臣恐邊上聞之,乘此生事。宜降詔安喻其心。」從之。



 江西、湖南兩道觀察使以新經訓、注之亂,吏卒多死,進官健衣糧一百二十分,充宰相募召從人。石奏曰:「宰相上
 弼聖政,下理群司。若忠正無私,宗社所祐,縱逢盜賊,兵不能傷;若事涉隱欺,心懷矯妄,雖有防衛,鬼得而誅。臣等願推赤心以答聖獎。孟軻知非臧氏,孔子不畏匡人。其兩道所進衣糧,並望停寢,依從前制置,只以金吾手力引從。」可之。帝又曰:「宰相之任,在選賢任能。」石曰:「臣與鄭覃常以此事為切,但以人各有求,茍遂所欲則美譽至,稍不如意則謗議生。只宜各委所司薦用,臣等擇可授之,則物議息矣。」



 其年十二月,中使田全操、劉行深巡
 邊回,走馬入金光門。從者訛言兵至,百官朝退,倉惶駭散。有不及束帶、襪而乘者。市人叫噪,塵坌四起。二相在中書,人吏稍散。鄭覃曰:「耳目頗異,且宜出去。」石曰:「事勢不可知,但宜堅坐鎮之,冀將寧息。若宰相亦走,則中外亂矣。必若繼亂,走亦何逃?任重官崇,人心所屬,不可忽也。」石視簿書,沛然自若。京城無賴之徒,皆戎服兵仗,北望闕門以俟變。內使連催閉皇城門,金吾大將軍陳君賞率其徒立望仙門下,謂中使曰:「假如有賊,閉門不晚。
 請徐觀其變,無宜自弱。」晡晚方定。是日,茍非石之鎮靜,君賞之禦侮,幾將亂矣。



 開成元年,改元,大赦。石等商量節文,放京一年租稅。及正、至、端午進奉,並停三年,其錢代充百姓紐配錢。諸道除藥物、口味、茶果外,不得進獻。諸司宣索制造,並停三年。赦後,紫宸宣對。鄭覃曰:「陛下改元御殿,全放京畿一年租稅,又停天下節鎮進奉。恩澤所該,實當要切。近年赦令,皆不及此。」上曰:「朕務行其實,不欲崇長空文。」石對曰:「赦書須內置一本,陛下時
 省覽之。十道黜陟使發日,付與公事根本,令與長吏詳擇施行,方盡利害之要。」石以從前德音雖降,人君不能守,奸吏從而違之,故有內置之奏以諷之。



 尋加中書侍郎、集賢殿大學士,領鹽鐵轉運使。上御紫宸論政曰:「為國之道,致治甚難。」石對曰:「朝廷法令行則易。臣聞文王陟降在上,陛下推赤誠,上達於天,何憂不治?」上又曰:「治亂由人邪正,由時運耶?」鄭覃對曰:「由聖帝,由忠臣,是由人也。」石曰:「亦由時運。九廟聖靈,鐘德於陛下,時也;陛下
 行己之道,則是由人。而前代帝王甚有德者,當亂離無奈何之際,又安得不推運耶?」帝曰:「卿言是也。」石又奏:「咸陽令韓遼請開興成渠。舊漕在咸陽縣西十八里,東達永豐倉,自秦、漢已來疏鑿,其後堙廢。昨遼計度,用功不多。此漕若成,自咸陽抵潼關,三百里內無車挽之勤,則轅下牛盡得歸耕,永利秦中矣。」李固言曰:「王涯已前已曾陳奏,實秦中之利,但恐征役今非其時。」上曰:「莫有陰陽拘忌否?茍利於人,朕無所慮也。」石辭領使務。八月,罷
 鹽鐵轉運使。石用金部員外郎韓益判度支案,益坐贓系臺。石奏曰:「臣以韓益曉錢穀錄用之,不謂貪猥如此!」帝曰:「宰相但知人則用,有過則懲。卿所用人,且不掩其惡,可謂至公。從前宰相用人,有過曲為蔽之,不欲人彈劾,此大謬也。但知能則舉,舉不失職則獎之,自然易得其人,何必容隱。」



 三年正月五日,石自親仁裏將曙入朝,盜發於故郭尚父宅;引弓追及,矢才破膚,馬逸而回。盜已伏坊門,揮刀斫石,斷馬尾,竟以馬逸得還私第。上聞
 之駭愕,遣中使撫問,賜金瘡藥,因差六軍兵士三十人衛從宰相。是日,京師大恐,常參官入朝者,九人而已,旬日方安。石拜章辭位者三。乃加金紫光祿大夫、中書侍郎、同平章事、江陵尹、荊南節度使。



 李訓之亂,人情危迫,天子起石於常僚之中,付以衡柄。石以身徇國,不顧患難,振舉朝綱,國威再復。而中官仇士良切齒惡之,而伏戎加害。天子深知其故,畏逼而不能理,乃至罷免。及石赴鎮,賜宴之儀並闕,人士傷之,恥君子之道消也。石至
 鎮,表讓中書侍郎,乃加檢校兵部尚書、兼平章事。



 武宗即位,就加檢校尚書右僕射。會昌三年十月,加檢校司空、平章事、隴西郡開國伯、食邑七百戶、太原尹、北都留守、河東節度觀察等使。時澤潞劉稹阻兵,以石嘗為太原副使,諳練北門軍政,故代劉沔鎮之。



 初,沔以兵三千人戍橫水。王師之討澤潞也,王逢軍於榆社,訴兵少,請益之,詔石以太原之卒赴榆社。石乃割橫水戍卒一千五百人,令別將楊弁率之,以赴王逢。舊例發軍,人給二
 縑。石以支計不足,量減一匹,軍人聚怨。又將及歲除,促令上路,眾愈不悅。楊弁乘其釁謀亂,出言激動軍人。



 四年正月,軍亂逐石,朝廷乃以晉絳觀察使崔元式代還。五年,檢校司徒、東都留守、判東都尚書省事、畿汝都防禦使。以太子少保分司卒。



 石弟福,字能之。太和七年登進士第,累闢使府。石為宰相,自薦弟於延英,言福才堪理人,授監察御史。累遷尚書郎,出為商、鄭、汝、潁四州刺史。大中時,檢校工部尚書、滑州刺史、兼御史大夫,充義
 成軍節度、鄭滑潁觀察使。入為刑部侍郎,累遷刑部、戶部尚書。乾符初,以檢校右僕射、襄州刺史、兼御史大夫充山南東道節度。



 四年,草賊王仙芝徒黨數萬寇掠山南。福團練鄉兵,屯集要路,賊不敢犯。其秋,賊陷岳、鄂、饒、信等州。十二月,逼江陵,節度使楊知溫求援於福;福即自率州兵及沙阤五百騎赴援。時賊已陷江陵之郛,聞福兵至,乃退去。僖宗嘉之,就加檢校司空、同平章事。歸朝,終於太子太傅。



 史臣曰:彭陽奇章,起徒步而升臺鼎。觀其人文彪炳,潤色邦典,射策命中,橫絕一時,誠俊賢也。而峨冠曳組,論道於皋、夔之伍,孰曰不然?如能蹈道匪躬,中立無黨,則其善盡矣。蕭太師貞獨嫉惡,不為利回,不以夷、惠儗之,俾之經綸,則其道至矣。開成之始,帝道方淪,石於此時欲振頹緒,幾嬰戕賊,可為咄嗟。多僻之時,止堪太息。



 贊曰:喬松孤立,蘿蔦夤緣。柔附凌雲,豈曰能賢?嗚呼楚、孺,道喪曲全!蕭、李相才,致之外篇。



\end{pinyinscope}