\article{卷一百六十}

\begin{pinyinscope}

 ○于頔韓弘子公武弘弟充李質附王智興子晏
 平晏宰



 于頔,字允元,河南人也,周太師燕文公謹之後也。始以廕補千牛,調授華陰尉,黜陟使劉灣闢為判
 官。又以櫟
 陽主簿攝監察御史,充入蕃使判官。再遷司門員外郎,兼侍御史,賜紫。充入西蕃計會使,將命稱旨,時論以為有出疆專對之能。歷長安縣令、駕部郎中。



 出為湖州刺史。因行縣至長城方山,其下有水曰西湖,南朝疏鑿,溉田三千頃,久堙廢。頔命設堤塘以復之,歲獲粳稻蒲魚之利,人賴以濟。州境陸地褊狹,其送終者往往不掩其棺槥,頔葬朽骨凡十餘所。改蘇州刺史,浚溝瀆,整街衢,至今賴之。吳俗事鬼,頔疾其淫祀廢生業,神宇皆撤去,
 唯吳太伯、伍員等三數廟存焉。雖為政有績,然橫暴已甚,追憾湖州舊尉,封杖以計強決之。觀察使王緯奏其事,德宗不省。及後頔累遷,乃與緯書曰:「一蒙惡奏,三度改官。」由大理卿遷陜虢觀察使。自以為得志,益恣威虐。官吏日加科罰,其惴恐重足一跡。掾姚峴不勝其虐,與其弟泛舟於河,遂自投而死。



 貞元十四年,為襄州刺史,充山南東道節度觀察。地與蔡州鄰。吳少誠之叛,頔率兵赴唐州,收吳房、朗山縣,又破賊於濯神溝。於是廣軍
 籍,募戰士,器甲犀利,僴然專有漢南之地。小失意者,皆以軍法從事。因請升襄州為大都督府,府比鄆、魏。時德宗方姑息方鎮,聞頔事狀,亦無可奈何,但允順而已。頔奏請無不從。於是公然聚斂,恣意虐殺,專以凌上威下為務。鄧州刺史元洪,頔誣以贓罪奏聞,朝旨不得已為流端州,命中使監焉。至隋州棗陽縣,頔命部將領士卒數百人,劫洪至襄州,拘留之。中使奔歸京師。德宗怒,笞之數十。頔又表洪其責太重,復降中使景忠信宣旨慰
 諭。遂除洪吉州長史,然後洪獲赴謫所。又怒判官薛正倫,奏貶峽州長史。及敕下,頔怒已解,復奏請為判官,德宗皆從之。正倫卒,未殯,頔以兵圍其宅,令孽男逼娶其嫡女。頔累遷至左僕射、平章事、燕國公。俄而不奉詔旨,擅總兵據南陽,朝廷幾為之旰食。



 及憲宗即位,威肅四方,頔稍戒懼。以第四子季友求尚主。憲宗以長女永昌公主降焉。其第二子方,屢諷其父歸朝入覲,冊拜司空、平章事。



 元和中,內官梁守謙掌樞密,頗招權利。有梁正
 言者,勇於射利,自言與守謙宗盟情厚,頔子敏與之游處。正言取頔財賄,言賂守謙,以求出鎮。久之無效,敏責其貨於正言。乃誘正言之僮,支解棄於溷中。八年春,敏奴王再榮詣銀臺門告其事,即日捕頔孔目官沈璧、家僮十餘人,於內侍獄鞫問。尋出付臺獄,詔御史中丞薛存誠、刑部侍郎王播、大理卿武少儀為三司使按問,乃搜死奴於其第,獲之。頔率其男贊善大夫正、駙馬都尉季友,素服單騎,將赴闕下,待罪於建福門。門司不納,退
 於街南,負墻而立,遣人進表。閣門使以無引不受,日沒方歸。明日,復待罪於建福門。宰相喻令還第,貶為恩王傅。敏長流雷州,錮身發遣。殿中少監、駙馬都尉季友,追奪兩任官階,令其家循省。左贊善大夫正、秘書丞方並停見任。孔目官沈璧決四十,配流封州。奴犀牛與劉乾同手殺人,宜付京兆府決殺。敏行至商山賜死。梁正言、僧鑒虛並付京兆府決殺。頔其年十月,改授太子賓客。



 十年,王師討淮、蔡,諸侯貢財助軍。頔進銀七千兩、金五
 百兩、玉帶二,詔不納,復還之。十三年,頔表求致仕。宰臣擬授太子少保,御筆改為太子賓客。其年八月卒,贈太保,謚曰「厲」。其子季友從獵苑中,訴於穆宗,賜謚曰「思」。右丞張正甫封敕,請還本謚。



 右補闕高鉞上疏論之曰:



 夫謚者,所以懲惡勸善,激濁揚清,使忠臣義士知勸,亂臣賊子知懼。雖竊位於當時,死加惡謚者,所以懲暴戾,垂沮勸。孔子修《春秋》,亂臣賊子懼,蓋為此也。垂範如此而不能救,況又隳其典法乎?



 臣風聞此事是徐泗節度使
 李愬奏請。李愬勛臣節將,陛下寵其勛勞,賜其爵祿、車服、第宅則可,若亂朝廷典法,將何以沮勸?仲尼曰:「唯名與器,不可以假人。」名器,君之所司,若以假人,與之政也,政亡則國家從之。頔頃鎮襄、漢,殺戮不辜,恣行兇暴。移軍襄、鄧,迫脅朝廷,擅留逐臣,徼遮天使。當先朝嗣位之始,貴安反側,以靖四方。幸免鈇鉞之誅,得全腰領而斃,誠宜謚之「繆厲」,以沮兇邪,豈可曲加美名,以惠奸宄。如此,則是于頔生為奸臣,死獲美謚,竊恐天下有識之士,
 謂聖朝無人,有此倒置。伏請速追前詔,卻依太常謚為厲,使朝典無虧,國章不濫。



 太常博士王彥威又疏曰:



 古之聖王立謚法者,所以彰善惡、垂勸誡。使一字之褒,賞逾紱冕;一言之貶,辱過朝市。此有國之典禮,陛下勸懲之大柄也。頔頃擁節旄,肆行暴虐,人神共憤,法令不容。擅興全師,僭為正樂,侵辱中使,擅止制囚,殺戮不辜,誅求無度,臣故定謚為厲。今陛下不忍,改賜為「思」,誠出聖慈,實害聖政。伏以陛下自臨宸扆,懋建大中,聞善若驚,
 從諫不倦。況當統天立極之始,所謂執法慎名之時,一垂恩光,大啟僥幸。且如頔之不法,然而陛下不忍加懲,臣恐今後不逞之徒如頔者眾矣!死援頔例,陛下何以處之?是恩曲於前而弊生於後。若以李吉甫有賜謚之例,則甫之為相也,有犯上殺人之罪乎?以頔況之,恐非倫類。如以頔常入財助國,改過來覲,兩使絕域,可以贖論,夫傷物害人,剝下奉上,納賄求幸,尤不可長其漸焉。



 自兩河宿兵,垂七十年,王師憓征,瘡磐未息。及張茂昭
 以易定入覲,程權以滄景歸朝,故恩禮殊尤,以勸來者。而于頔以文吏之職,居腹心之地,而倔強犯命,不獲已而入朝,豈茂昭之比乎!縱有入財使遠之勤,何以掩其惡跡!伏望陛下恩由義斷,澤以禮成,褒貶道存,僥幸路絕,則天下幸甚。



 疏奏不報,竟謚為思。



 長慶中,以戚里勛家諸貴引用於方,復至和王傅,家富於財。方交結游俠,務於速進。元稹作相,欲以其策平河朔群盜,方以策畫乾稹。而李逢吉之黨欲傾裴度,乃令人告稹欲結客刺
 度。事下法司,按鞫無狀,而方竟坐誅。



 韓弘,潁川人。其祖、父無聞,世居滑之匡城。少孤,依母族。劉玄佐即其舅也。事玄佐為州掾,累奏試大理評事。玄佐卒,子士寧被逐。弘出汴州,為宋州南城將。劉全諒時為都知兵馬使。貞元十五年,全諒卒,汴軍懷玄佐之惠,又以弘長厚,共請為留後,環監軍使請表其事,朝廷亦以玄佐故許之。自試大理評事檢校工部尚書、汴州刺史,兼御史大夫、宣武軍節度副大使知節度事、宋亳汴
 潁觀察等使。



 時吳少誠遣人至汴,密與劉全諒謀,因曲環卒襲陳許。會全諒卒,其人在傳舍,弘喜獲節鉞,即斬其人以聞。立出軍三千,助禁軍共討少誠。汴州自劉士寧之後,軍益驕恣,及陸長源遇害,頗輕主帥。其為亂魁黨數十百人。弘視事數月,皆知其人。有部將劉鍔者,兇卒之魁也。弘欲大振威望。一日,引短兵於衙門,召鍔與其黨三百,數其罪,盡斬之以徇,血流道中。弘對賓僚言笑自若。自是訖弘入朝,二十餘年,軍眾十萬,無敢怙亂
 者。累授檢校左右僕射、司空。憲宗即位,加同平章事。時王鍔檢校司空、平章事。致書於宰臣武元衡,恥在王鍔之下。憲宗方欲用形勢以臨淮西,乃授以司徒、平章事,班在鍔上。及用嚴綬為招討,為賊所敗,弘方鎮汴州,當兩河賊之沖要,朝廷慮其異志,欲以兵柄授之,而令李光顏、烏重胤實當旗鼓。乃授弘淮西諸軍行營都統,令兵部郎中、知制誥李程宣賜官告。弘實不離理所,唯令其子公武率師三千隸李光顏軍。弘雖居統帥,常不欲
 諸軍立功,陰為逗撓之計。每聞獻捷,輒數日不怡,其危國邀功如是。吳元濟誅,以統帥功,加檢校司徒、兼侍中,封許國公,罷行營都統。



 十四年,誅李師道,收復河南二州,弘大懼。其年七月,盡攜汴之牙校千餘人入覲。對於便殿,拜舞之際,以其足疾,命中使掖之。宴賜加等,預冊徽號大禮。進絹三十五萬匹、騑三萬匹、銀器二百七十件。三上章堅辭戎務,願留京師奉朝請。詔曰:



 納大忠,樹嘉績,為臣所以明極節;錫殊寵,進高秩,有國所以待元
 臣。況乎邦教誕敷,王言總會,百闢攸憲,四方式瞻。永念於懷,久虛其位,載揚成命,僉曰休哉。



 宣武軍節度副大使知節度事、汴宋亳潁等州觀察處置等使、開府儀同三司、守司徒、兼侍中、使持節汴州諸軍事、汴州刺史、上柱國、許國公、食邑三千戶韓弘,降神挺材,積厚成器;中蘊深閎之量,外標嚴重之姿。有匡國濟時之心,推誠不耀;有夷兇禁暴之略,仗義益彰。自鎮浚郊,二十餘載,師徒稟訓而咸肅,吏士奉法而愈明。俗臻和平,人用庶富,
 威聲之重,隱若山崇。



 屬者,淮濆濯征,命統群帥,克殄殘孽,惟乃有指蹤之功。及齊境興妖,分師進討,遂梟元惡,惟乃有略地之效。既聞旋旆,俄請執珪,深陳魏闕之誠,遠繼韓侯之志。朝天有慶,就日方伸。又抗表章,固辭戎旅,三加敦諭,所守彌堅。於蕃於宣,諒切於注意;我弼我輔,難違其衷懇。式遂良願,載兼上司。論道之榮,因之以齊八政;中樞之長,升之以贊萬務。玄袞赤舄,備於寵光;不有其人,孰膺斯任?可守司徒、兼中書令。



 乃以吏部尚
 書張弘靖兼平章事,代弘鎮宣武。



 憲宗崩,以弘攝塚宰。十五年六月,以本官兼河中尹、河中晉絳節度觀察等使。時弘弟充為鄭滑節度使,子公武為鄜坊節度使。父子兄弟,皆秉節鉞,人臣之寵,冠絕一時。二年,請老乞罷戎鎮,三表從之。依前守司徒、中書令。其年十二月病卒,時年五十八。贈太尉,賻絹二千匹、布七百端、米粟千碩。



 初,弘鎮大梁二十餘載,四州征賦皆為己有,未嘗上供。有私錢百萬貫、粟三百萬斛、馬七千匹,兵械稱是。專務
 聚財積粟,峻法樹威。而莊重寡言,沉謀勇斷,鄰封如吳少誠、李師道輩皆憚之。詔使宣諭,弘多倨待。及齊、蔡賊平,勢屈入覲,兩朝寵待加等,弘竟以名位始終,人臣之幸也。時公武已卒,弘孫紹宗嗣。



 公武自宣武馬步都虞候將兵誅蔡,賊平,檢校右散騎常侍、鄜州刺史、鄜坊等州節度使。丁所生憂,起復金吾將軍,仍舊職。十四年,父弘入朝,公武乞罷節度,入為右金吾將軍。既而弘出鎮河中,季父充移鎮宣武,公武嘆曰:「二父聯居重鎮,吾以
 孺子當執金吾職,家門之盛,懼不克勝。」堅辭宿衛,改右驍衛將軍。性頗恭遜,不以富貴自處。弘罷河中,居崇里第;公武居宣陽里之北門,因省父,無疾暴卒,贈戶部尚書。



 充依舅劉玄佐,歷河陽、昭義牙將。及兄弘節制宣武,召歸主親兵,奏授御史大夫。弘頗酷法,人人不自保。充獨謙恭執禮,未嘗懈怠,由是遍得士心。然以親逼權重,常不自安。元和六年,因獵近郊,單騎歸於洛陽。時朝廷方姑息弘,亦憐充之無異志,擢拜右金吾衛將軍。十
 二月,轉大將軍,歷少府監。十五年,代侄公武為鄜坊節度使、檢校工部尚書。



 長慶二年,幽、鎮、魏復亂。朝廷以王承元有冀卒數千在滑州,恐封疆相接,復相勸誘。命充與承元更換所守,檢校左僕射。是歲,汴州節度使李願被三軍所逐,立都將李絺既為留後。朝廷以充久在汴州,從心悅附,命充為宣武節度使,兼統義成之師往討絺。會絺疽發腦,屬兵於紀綱李質。質以計誅首亂,送絺歸京師。充遂不戰而入大梁。時陳許李光顏亦奉詔討絺,
 軍於尉氏,意欲必先收汴,因大肆俘掠。汴州監軍使姚文壽亦欲招許下之師。充在中牟聞其謀,率眾徑至城下。汴人素懷充來,皆踴躍相賀,無復疑貳。詔加檢校司空。詔割潁州隸滑州。充既安堵,密籍部伍間,得嘗構惡者千餘人。一日下令,並父母妻子立出之,敢逡巡境內者斬!自是軍政大理,汴人無不愛戴。



 四年八月,例加司徒。詔未至,暴疾卒,時年五十五。贈司徒,謚曰肅。充雖內外皆將家,素不事豪侈,常以簡約自持。臨機決策,動無
 遺悔,善將者多之。



 李質者,汴之牙將。李絺既為留後,倚質為心腹。及朝廷以絺為郡守,志邀節鉞,質勸喻不從。會絺疽發首,乃與監軍姚文壽謀,斬絺傳首京師。有詔以韓充鎮汴。充未至,質權知軍州事。使衙牙兵二千人,皆日給酒食,物力為之損屈。充將至,質曰:「若韓公始至,頓去二千人日膳,人情必大去;若不除之,後當無繼。不可留此弊以遺吾帥。」遂處分停日膳而後迎充。召為金吾將軍,長慶三年
 四月卒。



 王智興,字匡諫,懷州溫縣人也。曾祖靖,左武衛將軍。祖瑰,右金吾衛將軍。父縉,太子詹事。



 智興少驍銳,為徐州衙卒,事刺史李洧。及李納謀叛,欲害洧。洧遂以徐州歸國。納怒,以兵攻徐甚急。智興健行,不四五日齎表京師求援。德宗發朔方軍五千人隨智興赴之,淄青圍解。自是,智興常以徐軍抗納,累歷滕、豐、沛、狄四鎮將。自是二十餘年為徐將。



 元和中,王師誅吳元濟,李師道與蔡賊
 謀撓沮王師,頻出軍侵徐,徐帥李願以所部步騎悉委智興以抗之。鄆將王朝晏以兵攻沛,智興擊敗之。賊又令姚海率勁兵二萬圍豐,攻城甚急。智興復擊敗之。於賊壁獲美妾,智興懼軍士爭之,乃曰:「軍中有女子,安得不敗?此雖無罪,違軍法也。」即斬之以徇。累官至侍御史、本軍都押衙。



 十三年,王師誅李師道,智興率徐軍八千會諸道之師進擊。與陳許之軍大破賊於金鄉,拔魚臺,俘斬萬計,以功遷御史中丞。賊平,授沂州刺史。



 長慶初,
 河朔復亂,徵兵進討。穆宗素知智興善將,遷檢校左散騎常侍、兼御史大夫,充武寧軍節度副使、河北行營都知兵馬使。



 初,召智興以徐軍三千渡河,徐之勁卒皆在部下。節度使崔群慮其旋軍難制,密表請追赴闕,授以他官。事未行,會赦王廷湊,諸道班師。智興先期入境,群頗憂疑,令府僚迎勞,且誡之曰:「兵士悉輸甲仗於外,副使以十騎入城。」智興既首處,賓僚聞之心動,率歸師斬關而入,殺軍中異己者十餘人。然後詣衛謝群曰:「此軍
 情也。」群治裝赴闕,智興遣兵士援送群家屬至埇橋。遂掠鹽鐵院緡幣及汴路進奉物,商旅貲貨,率十取七八。逐濠州刺史侯弘度。弘度棄城走。朝廷以罷兵,力不能加討,遂授智興檢校工部尚書、徐州刺史、御史大夫,充武寧軍節度、徐泗濠觀察使。自是智興務積財賄,以賂權勢,賈其聲譽,用度不足,稅泗口以裒益之。累加至檢校僕射、司空。



 太和初,李同捷據滄德叛,智興上章,請躬督士卒討賊。從之。乃出全軍三萬,自備五月糧餉,朝廷
 嘉之。加檢校司徒、同平章事,兼滄德行營招撫使。初,同捷狂桀犯命,濟之以王廷湊,王師經年無功。及智興拔棣州,賊大懼,諸軍稍務進取。以智興首功,加守太傅,封雁門郡王。賊平入朝,上賜宴麟德殿,賞賜珍玩名馬,進位侍中,改許州刺史、忠武軍節度、陳許蔡等州觀察使。



 太和七年,改授河中尹、河中節度、晉磁隰觀察等使。智興因入朝。九年五月,改汴州刺史、宣武軍節度、宋亳汴潁觀察等使。



 開成元年七月卒,年七十九。贈太尉,不視
 朝三日。葬於洛陽榆林之北原,四鎮將校會葬者千人。



 智興九子:晏平、晏宰、晏皋、晏實、晏恭、晏逸、晏深、晏斌、晏韜,而晏平、晏宰最知名。



 晏平幼從父征伐,以討李同捷功,授檢校右散騎常侍、靈州大都督府長史、朔方靈鹽節度。丁父憂,奔歸洛陽。晏平居官貪黷,去鎮日,擅將征馬四百餘匹及兵仗七千事自衛,為憲司所糾。減死,長流康州。以父喪,未赴流所,告於河北三鎮。三帥上表救解,請從昭雪,改授撫州司馬。給事中韋溫、薛廷老、盧弘
 宣封還制書,改永州司戶。韋溫又執不下,文宗令中使宣諭方行。



 晏宰於昆仲間最稱偉器,大中後,歷上黨、太原節度使。捍回鶻、黨項,屢立邊功。



 晏皋仕至左威衛將軍。



 史臣曰:於燕公以儒家子,逢時擾攘,不持士範,非義非俠,健者不為,末塗淪躓,固其宜矣。韓、王二帥,乘險蹈利,犯上無君,豺狼噬人,鵂鶹幸夜,爵祿過當,其可已乎?謂之功臣,恐多慚色。



 贊曰:於子清狂,輕犯彞章。韓虐王剽,專恣一方。元和赫斯,揮劍披攘。擇肉之倫,爪距摧藏。



\end{pinyinscope}