\article{卷十六 本紀第十六 穆宗}

\begin{pinyinscope}

 穆宗睿聖文惠孝皇帝諱恆,憲宗第三子,母曰懿安皇后郭氏。貞元十一年七月,生於大明宮之別殿。初名宥,元和元年八月,進封遂王。五年三月,領彰
 義軍節度大使。七年十月,冊為皇太了,改今諱。



 十五年正月庚子,憲宗崩。丙午,即皇帝位於太極殿東序。是日,召翰林學士段文昌杜元穎沈傳師李肇、侍讀薛放丁公著對於思政殿,並賜金紫。丁未,集群臣班於月華門外。貶門下侍郎、平章事皇甫鎛為崖州司戶。戊申,上見宰臣於紫宸門外。辛亥,以朝議郎、守御史中丞、飛騎尉、襲徐國公、賜緋魚袋蕭俛為朝散大夫、守中書;舍人、翰林學士、武騎尉、賜紫金魚段文昌為中書侍郎:同
 平章事。上始御延英對宰臣。詔曰:「山人柳泌輕懷左道。上惑先朝。固求牧人,貴欲疑眾,自知虛誕,仍更遁逃。僧大通醫方不精,藥術皆妄。既延禍釁,俱是奸邪。邦國固有常刑,人神所宜共棄,付京兆府決杖處死。」金吾將軍李道古貶循州司馬。憲宗未年,銳於服餌,皇甫鎛與李道古薦術人柳泌、僧大通待詔翰林。泌於臺州為上鍊神丹,上服之,日加躁渴,遽棄萬國。甲寅,二王後介國公宇文仲達卒,有司與舊典葬祭之。以監察御史李德裕、
 右拾遺李紳、禮部員外郎庾敬休並守本官,充翰林學士。丁巳,以劍南東川節度使李逢吉為襄州刺史,充山南東道節度使;以吏部侍郎王涯檢校禮部尚書、梓州刺史,充劍南東川節度使。己未,改恆岳為鎮岳,恆州來鎮州,定州恆陽縣為曲陽縣。恆王房子孫改為泜王房。丙寅,以右神策大將軍張維清為單于大都護,充振武麟勝節度使。丁卯,上及群臣皆釋服從吉。戊辰,群臣始朝於宣政衙。是夜地震。庚午,冊大行皇帝貴妃郭氏為
 皇太后。貶諫議大夫李景儉為建州刺史。



 二月癸酉朔。丁丑,御丹鳳樓,大赦天下。宣制畢,陳俳優百戲於丹鳳門內,上縱觀之。丁亥,幸左神策軍觀角抵及雜戲,日昃而罷。癸巳,罷邕管經略使,所管州縣隸邕府。甲午,以桂管觀察使裴行立為安南都護,充本管經略使。乙未,以太僕卿杜式方為桂州刺史,充桂管觀察使。丙申,丹王逾薨。丁酉。敕入回紇使宜與私覿正員官十三員,入吐蕃使與八員。庚子,太子賓客呂元膺卒。辛丑,以戶部侍
 郎楊於陵為戶部尚書。壬寅,敕舉賢良方正、直言極諫等科目人,宜令中書門下尚書省四品已上於尚書省同試。



 三月癸卯朔,贈皇太后父郭曖太傅,母虢國大長公主贈齊國大長公主。壬子,召侍講學士韋處厚、路隨於太液亭講《毛詩關雎》、《尚書洪範》等篇。既罷,並賜緋魚袋。左右軍中尉馬進潭、梁守謙、魏弘簡等請立門戟,從之。以太子詹事分司東都韋貫之為河南尹。丁巳,御史中丞崔植奏:「元和十二年敕,御史臺三院御史據除拜
 上日為先後,未上日不得計月數。又準其年九月十七日敕,逾一個月不在此限,行立班次,即宜以敕內先後為定。臣觀此後敕未便事宜,請自今後三院御史職事行立,一切依敕文先後為定,除拜上日便為月數。」從之。戊午,吏部尚書趙宗儒奏:「先奉敕,先朝所放制科舉人,令與中書門下四品已上官同於尚書省就試者。臣伏以制科所設,本在親臨,南省策試,亦非舊典。今覃恩既畢,庶政惟新,況山陵日近,公務繁迫,待問之士,就試非多。
 臣等商量,恐須停罷。」從之。罷申州歲貢茶。乙丑,以皇太后兄司農卿郭釗為刑部尚書兼司農卿,右金吾衛大將軍郭鏦檢校工部尚書。丁卯,貶太子賓客留司東都孟簡為吉州員外司馬。戊辰夜,大風雹。



 夏四月壬申朔。丁丑,澧王薨。乙酉,三恪酅國公楊造卒。丁亥,敕:「內侍省見管高品官白身,都四千六百一十八人,除官員一千六百九十六人外,其餘單貧,無屋室居止,宜每人加衣糧半分。」五月壬寅朔。癸卯,詔:「以國用不足,應天下兩
 稅、鹽利、榷酒、稅茶及戶部闕官、除陌等錢,兼諸道雜榷稅等,應合送上都及留州、留使、諸道支用、諸司使職掌人課料等錢,並每貫除舊墊陌外,量抽五十文。仍委本道、本司、本使據數逐季收計。其諸道錢便差綱部送付度支收管,待國用稍充,即依舊制。其京百司俸料,文官已抽修國學,不可重有抽取;武官所給校薄,亦不在抽取之限。」壬子,詔:「入景陵玄宮合供千味食,魚肉肥鮮,恐致薰穢,宜令尚藥局以香藥代食。」庚申,葬憲宗於景陵。



 六月辛未朔。,丁丑,以司徒、兼中書令韓弘為河中尹,充河中晉絳慈隰等州節度使。安南都護桂仲武奏誅賊首楊清,收復安南府戊寅,以金吾將軍李祐檢校左散騎常侍,兼夏州刺史,充夏綏銀宥節度使,代李聽。以聽為靈州大都督府長史,充朔方靈鹽節度使。以中書舍人王仲舒為洪州刺史、御史中丞,充江西觀察使。己卯,放京兆府今年夏青苗錢八萬三千五百六十貫,宜委令狐楚,以楚山陵用不盡綾絹,準實估付京兆府,代所
 放青苗錢。庚辰,加邠、寧、慶節度使李光顏特進,以城鹽州之功也。以考功員外郎、史館修撰李翱為朗州刺史,坐與李景儉相善故也。癸未,並兗州萊蕪縣入乾封縣。己丑,工部尚書歸登卒。壬辰,詔:「帝王所重者國體,所切者人情。茍得其體,必臻於大和;如失其情,是由於小利。況設官求理,頒祿責功,教既有常,寧宜就減。近以每歲經費,量入數少,外官俸料,據數收貫。朕再三思度,終所未安。今則歲屬豐登,兵方偃息,自宜克己以足用,何得
 剝下以為謀。臨軒載懷,實所增愧。其今年五月敕應給用錢每貫抽五十文,都計一百五十萬貫,宜並停抽。」仍出內庫錢三十七萬五千貫,付度支給用。初,憲宗用兵,擢皇甫鎛為相,苛斂剝下,人皆咎之,以至譴逐。至是宰臣創抽貫之利,制下,人情不悅,故罷之。癸巳,皇太后移居興慶宮,皇帝與六宮侍從大合宴於南內,回幸右軍,頒賜中尉等有差。自是凡三日一幸左右軍及御宸暉、九仙等門,觀角抵、雜戲。



 秋七月辛朔。壬寅,以河中、晉、
 絳觀察使李絳為兵部尚書。甲辰,以大理卿孔戢為潭州刺史、湖南觀察使。乙巳,詔:「皇太后就安長樂,朝夕承顏,慈訓所加,慶感兼極。今月六日是朕載誕之辰,奉迎皇太后於宮中上壽。朕既深歡慰,欲與臣下同之。其日,百僚、命婦宜於光順門進名參賀,朕於光順門內殿與百僚相見,永為常式。」非典也。鄆曹濮等州節度賜號天平軍,從馬總奏也。丙午,敕:乙巳詔書載誕受賀儀宜停。先是,左丞韋綬奏行之,宰臣以古無降誕受賀之禮,奏
 罷之。丁未,苑內假山毀。壓死役者七人。自五月五雨,至此月壬子始雨。甲寅,御新成永安殿觀百戲,極歡而罷。乙卯,敕自今後新除節度、觀察使到任日,具見在錢帛、斛斗、器械數目分析以聞。安南都護行立卒。是日,上幸安國寺觀盂蘭盆。邕管經略使楊旻卒。平盧軍新加押新羅、渤海兩蕃使,賜印一面,許置巡官一人。新作寶慶殿。庚申夜,熒惑入羽林。壬戌,盛飾安國、慈恩、千福、開業、章敬等寺,縱吐蕃使者觀之。丙寅,以新成永安殿,與
 中宮貴主密宴以樂之,嬪妃皆預。丁卯,以門下待郎、平章事令狐楚為宣州刺史、兼御史大夫,充宣、歙、池觀察使。楚為山陵使,縱吏於翬刻下,不給工徒價錢,積留錢十五萬貫,為羨餘以獻,故及於貶。



 八月庚午朔。辛未,兵部尚書楊於陵總百僚錢貨輕重之議,取天下兩稅、榷酒、鹽利等,悉以布帛任土所產物充稅,並不徵見錢,則物漸重,錢漸輕,農人見免賤賣匹段。請中書門下、御史臺諸官長重議施行。從之。癸酉,太子少傅致仕李鄘
 卒。甲戌,安南都護桂仲武斬叛將楊清首以獻,收復安南府。乙亥,賜教坊錢五千貫,充息利本錢。御勤政樓,問人疾苦。前江西觀察使裴次元卒。己卯,月掩牽牛。同州雨雪,害秋稼。京兆府戶曹參軍韋正牧專知景陵工作,刻削廚料充私用,計贓八千七百貫文;石作專知官奉仙縣令於翬刻削,計贓一萬三千貫,並宜決重杖處死。壬辰,幸魚藻池,發神策軍二千人浚魚藻池。戊戌,以朝議郎、守御史中丞、武騎尉、賜紫金魚袋崔植為朝散大
 夫、守中書侍郎、同中書門下平章事。己亥,宣歙觀察使令狐楚再貶衡州刺史。



 九月庚子朔,改河北稅鹽使為榷鹽使。辛丑,大合樂於魚藻宮,觀競渡。又召李醖、李光顏入朝,欲於重陽日宴群臣。拾遺李玨等上疏諫云:「元朔未改,園陵尚新。雖易月之期,俯從人欲;而三年之制,猶服心喪。夫遏密弛禁,蓋為齊人;合樂內庭,事將未可。」不聽。乙巳,以駕部郎中、知制誥李宗閔為中書舍人。宋州大水,損田六千頃。戊申,以重陽節曲宴郭釗兄弟、貴
 戚、主婿等於宣和殿。己酉,大酺三日,至是雨雪,樹木無風而摧僕者十五六。以吏部侍郎崔群為御史大夫。滄、景水,損田。戊午,加河東節度使、金紫光錄大夫、檢校尚書右僕射、兼門下侍郎、同平章事、太原尹、北都留守、上柱國、晉國公、食邑三千戶裴度守司空、門下侍郎、同平章事。以邠寧節度使、檢校司空、邠州刺史、上柱國、武威郡開國公、食邑二千戶李光顏並同中書門下平章事。又武寧軍節度使、徐泗濠等州觀察等使、檢校尚書左
 僕射、徐州刺史、上柱國、涼國公、食邑三千戶李醖為同中書門下平章事、潞州大都督府長史,充昭義軍節度、澤潞磁邢洺觀察處置等使。夏州奏移宥州於長澤縣置。辛酉,宴李光顏、李醖於麟德殿,頒賜優厚。以袁州刺史韓愈為朝散大夫、守國子祭酒,復賜金紫。丙寅,以御史大夫崔群檢校兵部尚書、徐州刺史,充武寧軍節度、徐泗宿濠觀察等使;以將作監崔能為廣州刺史,充嶺南節度使。丁卯以兵部尚書李絳為御史大夫。戊辰,以
 前嶺南節度使孔戣為吏部侍郎。



 冬十月庚午朔,闍婆國遣使朝貢。庚辰,宰相與吐蕃使於中書議事。京百司共賜錢一萬貫,仰御史臺據司額大小、公事閑劇均之。成德軍節度使王承宗卒,其弟承元上表請朝廷命帥,遣起居舍人柏耆宣慰之。辛巳,金公亮修成指南車、記里鼓車。壬午,吐蕃寇涇州,命中尉梁守謙將神策軍四千人及八鎮兵赴援。乙酉,以魏博等州節度觀察等使、光祿大夫、檢校司徒、兼侍中、魏博大都督府長史、上柱
 國、沂國公、食邑三千戶、實封三百戶田弘正可檢校司徒、兼中書令、鎮州大都督府長史、成德軍節度、鎮冀深趙等州觀察處置等使。以鎮冀深趙等觀察度支使、朝議郎、試金吾左衛胄曹參軍兼監察御史王承元可銀青光祿大夫、檢校工部尚書、使持節滑州諸軍事、守滑州刺史、御史大夫,充義成軍節度、鄭滑等州觀察等使。以昭義節度使、檢校尚書左僕射、同中書門下平章事李醖可本官,為魏州大都督府長史,充魏博等州節度、
 觀察等使。以義成軍節度使劉悟依前檢校右僕射、兼潞州大都督府長史,充昭義節度、澤潞邢洺磁等州觀察等使。以左金吾將軍田布為檢校左散騎常侍、兼懷州刺史、御史大夫,充河陽三城懷孟節度使。乙酉,涇州奏吐蕃退去。時夏州節度使田縉貪猥,侵刻黨項羌,羌引西蕃入寇,賴郝玼、李光顏奮命拒之,方退。丁亥,西川奏吐蕃侵雅州,令發兵鎮守。東川節度使王涯陳破吐蕃策,言以厚賂北蕃,俾入西蕃,據地得人多少賞之。



 十
 一月乙亥朔。癸卯,制:「朕聞帝王丕宅四海,子育群生,如天無不覆,如日無不燭。乃睠冀方,初喪戎帥,念乎三軍之事,洎於四州之人。或懷忠積誠,而思用莫展;或災荒兵役,而望恤何階。今則昌運一開,誠節咸著。王承元首陳章疏,願赴闕庭。永念父兄之忠,克固君臣之義,已加殊獎,別委重籓。又念成德軍將士等,葉謀向義,丹款載申,咸欲效其器能,各宜列之爵秩。大將史重歸、牛元翼已超授寵榮,今更都加厚賜。宜令諫議大夫鄭覃往鎮
 州宣慰,賜錢一百萬貫。王澤所洽,天綱方恢,宥過釋冤,與人休泰。其管內見禁囚徒,罪無輕重,並宜釋放。朕以武俊之勛勞,光於彞鼎;士真之恭恪,繼被節旄。承宗感恩,亦克立效。永言十代之宥,俾賜一門之榮。承宗兄弟已授官爵,其承宗葬事亦差官監視,務令周厚。」丁未,封王承宗祖母李氏為晉國太夫人。辛亥,田弘正奏王承元以今月九日領兵二千人赴鎮滑州。成德軍徵賞錢頗急,乃命柏耆先往諭之。以華州刺史衛中行為陜州
 長史,充陜虢觀察使;以宗正卿李翱為華州刺史、潼關防禦、鎮國軍使。乙卯,上幸金吾將軍郭鏦城南莊,鏦以莊為獻。戊午,詔曰:「朕來日暫往華清宮,至暮卻還。」御史大夫李絳、常侍崔元略已下伏延英門切諫。上曰:「朕已成行,不煩章疏。」諫官再三論列。是日,田弘正奏今月十六日入鎮州訖。己未,上由復道出城幸華清宮,左右中尉擗仗,六軍諸使、諸王、駙馬千餘人從,至晚還宮。癸亥,檢校司徒、兼太子少少師鄭餘慶卒。以渭州刺史、涇原行
 營兵馬使、保定郡王郝玼為慶州刺史。將,深入吐蕃接戰,朝廷恐失勇將,故移之內地。十二月己巳朔。戊寅,召故女學士宋若華妹若昭入宮掌文奏。壬午,幸右軍擊鞠,遂畋於城西。丙戌,前昭義軍節度使辛秘卒。己丑,以庫部郎中、知制誥牛僧孺為御史中丞。嶺南奏崖州司戶參軍皇甫鎛卒。丙申,以司門員外郎白居易為主客郎中、知制誥。是歲,計戶帳,戶總二百三十七萬五千四百,口總一千五百七十六萬。定、鹽、夏、劍南東西川、
 嶺南、黔中、邕管、容管、安南合九十七州不申戶帳。



 長慶元年正月己亥朔,上親薦獻太清宮、太廟。是日,法駕赴南郊。日抱珥,宰臣賀於前。辛丑,祀昊天上帝於圓丘,即日還宮,御丹鳳樓,大赦天下。改元長慶。內外文武及致仕官三品已上賜爵一及,四品已下加一階,陪位白身人賜勛兩轉,應緣大禮移仗宿衛御樓兵仗將士,普恩之外,賜勛爵有差。仍準舊例,賜錢物二十萬四千九百六十端匹。禮畢,群臣於樓前稱賀。仗退,上朝太后
 於興慶宮。壬寅,夏州節度使奏浙東、湖南等道防秋兵不習邊事,請留其兵甲,歸其人。靈武節度使李聽奏請於淮南、忠武、武寧等道防秋兵中取三千人衣賜月糧,賜當道自募一千五百人馬驍勇者以備邊。仍令五十人為一社,每一馬死,社人共補之,馬永無闕。從之。癸卯,以河陽、懷節度使。田布為涇州刺史,充四鎮北庭行營、涇原節度使;以刑部尚書兼司農卿郭釗檢校戶部尚書、懷州刺史,充河陽三城、懷節度使。以涇原節度使
 王潛檢校兵部尚書、江陵尹,充荊南節度使。乙巳,鄜坊節度使韓璀改名充。己酉,以前檢校大理少卿、駙馬都尉劉士涇為太僕卿。給事中韋弘景、薛存慶封還詔書,上諭之曰:「士涇父昌有邊功,久為少列十餘年,又以尚雲安公主,朕欲加恩,制官敕下。」制命始行。翰林學士、司勛員外郎李德裕上疏曰:「臣見國朝故事,駙馬國之親密,不合與朝廷要官往來,開元中止尤切。近日駙馬多至宰相及要官宅,此輩無他才可以延接,唯是漏洩
 禁密、交通中外。伏望宣示駙馬等,今後有事任至中書見宰臣,此外不得至宰臣及臺省官私第。」從之。戊午夜,星孛于翼。壬戌,制朝議大夫、守門下侍郎、同中書門下平章事徐國公蕭俯為尚書右僕射,累表乞罷政事故也。癸亥,以左散騎常侍崔元略為黔州刺史,充黔中觀察使。丁卯,星孛於長辰,近太微西垣南第一星。



 二月戊辰朔。癸酉,以尚書右僕射蕭俯為吏部尚。甲戌,以檢校右僕射兼吏部尚書韓皋守右僕射。乙亥夜,太白犯昴。
 丙子,上觀雜伎樂於麟德殿,歡甚,顧謂給事中丁公著曰:「比聞外間公卿士庶時為歡宴,蓋時和民安,甚慰予心。」公著對曰:「誠有此事。然臣之愚見,風俗如此,亦不足嘉。百司庶務,漸恐勞煩聖慮。」上曰:「何至於是?」對曰:「夫賓宴之禮,務達誠敬,不繼以淫。故詩人美『樂且有儀』。譏其屢舞。前代名士良辰宴聚,或清談賦詩,投壺雅歌,以杯酌獻酬,不至於亂。國家自天寶已後,風俗奢靡,宴席以喧嘩沉湎為樂。而居重位、秉大權者,優雜倨肆於公吏
 之間,曾無愧恥。公私相效,漸以成俗。則是物務多廢。獨聖心求理,安得不勞宸慮乎!陛下宜頒訓令,禁其過差,則天下幸甚。」時上荒於酒樂,公著因對諷之,頗深嘉納。己卯,幽州節度使劉總奏請去位落發為僧。又請分割幽州所管郡縣為三道,請支三軍賞設錢一百萬貫。壬申,以中書侍郎、平章事段文昌檢校刑部尚書、同平章事、成都尹,充劍南西川節度等使,以朝散大夫、尚書戶部侍郎、知制誥、翰林學士、上柱國、建安縣開國男杜元
 穎守本官、同中書門下平章事。以劍南西川節度使王播為刑部尚書,充鹽鐵轉運使。乙酉,天平軍節度使馬總奏:「當道見管軍士三萬三千五百人,從去年正月巳後,情願居農者放,逃戶者不捕。」先是,平定河南,及王承元去鎮州,宰臣蕭俯等不顧遠圖,乃獻銷兵之議,請密詔天下軍鎮,每年限百人內破八人逃死,故總有是奏。丁亥夜,月犯歲星,在尾十三度。辛卯,寒食節,宴群臣於麟德殿,頒賜有差。壬辰,刑部侍郎李建卒。癸巳,九姓回
 紇毗伽保義可汗卒。



 三月丁酉朔,浙東奏移明州於鄮縣置。劉總進馬一萬五千匹。甲辰,鄭滑節度使王承元祖母晉國太夫人李氏來朝,既見上,令朝太后於南內。丁未,宗正寺奏:「準貞元二十一年敕,宗子陪位,放五百七十人出身。準今年敕放三百人。伏緣人數至多,不沾恩澤,乞降特恩,更放二百人出身。」從之。平盧薛平奏:海賊掠賣新羅人口於緣海郡縣,請嚴加禁絕,俾異俗懷恩。從之。戊申,罷京西、京北和糴使,擾人故也。罷河北榷
 鹽法,許維計課利都數付榷鹽院。庚戌,以左丞韋綬為禮部尚書。是夜,太白近五車。辛亥,命給事中韋弘慶充幽州宣慰使,左拾遣狄兼謨副之。鹽鐵使王播奏江淮鹽估每斗加五十文,兼舊三百文。癸丑,以幽州盧龍軍節度副大使、知節度事、押奚、契丹兩蕃經略等使、檢校司空、同中書門下平章事、楚國公劉總可檢校司徒、兼侍中、天平軍節度、鄆曹濮等州觀察等使。以宣武軍節度使。檢校右僕射、同平章事張弘靖為檢校司空、同平
 章事、兼幽州大都督府長史,充幽州盧龍軍節度使。從劉總所奏故也。以鳳翔節度使李願檢校司空、汴州刺史,充宣武軍節度使;以邠寧節度使李光顏為鳳翔尹,依前檢校司空、平章事,充鳳翔隴右節度使。以右衛大將軍高霞寓檢校工部尚書、邠州刺史、充邠寧節度使。諫官上疏論霞寓敗軍左謫,未宜拜方鎮。不從。乙卯,以權知京兆尹盧士玫為瀛州刺史,充瀛莫等州團練觀察使。從劉總奏析置也。丁巳,制:「劉總已極上臺,仍移
 重鎮,兄弟子侄,各授官榮,大將賓僚,亦宜超擢。幽州百姓給復一年,賜三軍賞設錢一百萬貫。令宣慰使薛存慶與弘靖計會支給。」戊午,封皇弟憬為鄜王,悅為瓊王,恂為沔王,懌為婺王,愔為茂王,怡為光王,協為淄王,憺為衢王,心充為澶王;皇子湛為景王,涵為江王,湊為漳王,溶為安王,<注厘>為潁王。以兵部侍郎柳公綽為京兆尹、兼御史大夫。己未,以屯田員外郎李德裕為考功郎中,左補闕李紳為司勛員外郎,並依前知制誥、翰林學士。敕
 今年錢徽下進士及第鄭朗等一十四人,宜令中書舍人王起、主客郎中知制誥白居易等重試以聞。甲子,劉總請以私第為佛寺,乃遣中使賜寺額曰「報恩」。幽州奏劉總堅請為僧,又賜以僧衣,賜號大覺。總是夜遁去,幽州人不知所之。乙丑,以漳州刺史韓泰為郴州刺史,汀州刺史韓曄為永州刺史,循州刺史陳諫為道州刺史,量移也。



 夏四月丙寅朔,授劉總弟約及總男等一十一人官,內五人為刺史,餘朝班環衛。庚午,易定奏劉總已為僧,三
 月二十七日卒於當道界,贈太尉。甲戌,秘書監蔣乂卒。丙子,以前天平軍節度使馬總復為天平節度使。丁丑,詔:「國家設文學之科,本求才實,茍容僥幸,則異至公。訪聞近日浮薄之徒,扇為朋黨,謂之關節,干擾主司,每歲策名,無不先定。永言敗俗,深用興懷。鄭朗等昨令重試,意在精覆藝能,不於異常之中,固求深僻題目,貴令所試成就,以觀學藝淺深。孤竹管是祭天之樂,出於《周禮》正經,閱其呈試之文,都不知其本事。辭律鄙淺,
 蕪累何多。亦令宣示錢徵,庶其深自懷愧。誠宜盡棄,以警將來。但以四海無虞,人心方泰,用弘寬假,式示殊恩。孔溫業、趙存約、竇洵直所試粗通,與及第;盧公亮等十一人可落下。自今後禮部舉人,宜準開元二十五年敕,及第人所試雜文並策,送中書門下詳覆。」貶禮部侍郎錢徽為江州刺史,中書舍人李宗閔為劍州刺史,右補闕楊汝士為開州開江令。戊寅,宰臣崔植、杜元穎奏請,坐日所有群臣獻替,事關禮體,便隨日撰錄,號為《聖政
 紀》,歲終付史館。從之。事亦不行。丙戌,正衙命使冊九姓回紇為登羅羽錄沒密施句主錄毗伽可汗。辛卯,以衡州刺史令狐楚為郢州刺史,吉州司馬孟簡為睦州刺史。壬辰,詔百闢卿士宜各徇公,勿為朋黨。甲午,以張弘靖入幽州,受朝賀。中書門下奏燕、薊八州平,準禮宜告陵廟,從之。



 五月丙申朔。戊戌,以刑獄淹滯,立程:凡大事,大理寺三十五日詳斷訖,申刑部,三十日聞奏;中事,大理寺三十日,刑部二十五日;小事,大理寺二十五日,刑
 部二十日。所斷罪二十件已上為大,十件已上為中,十件已下為小。刑部四覆官、大理六丞每月常須二十日入省寺,其廚料令戶部加給。從中丞牛僧孺奏也。己亥,貶考功員外郎李渤為虔州刺史,以前書宰相考辭太過,宰相杜元穎等奏貶之。癸卯,幽州大將李參已下十八人並為刺史及諸衛將軍。己酉,右散騎常侍致仕柳登卒。辛亥,造百尺樓於宮中。壬子,加茶榷,舊額百文,更加五十文,從王播奏。拾遺李玨上參論其不可,疏奏不
 報。丙辰,建王審薨。丁巳,滄州先置景州於弓高縣,置歸化縣於福城草市,並宜停廢。壬戌,幽州宣慰使給事中薛存慶卒於鎮州。癸亥,敕先置溵州於郾城,宜廢;其郾城上蔡、西平、遂平兩縣復隸蔡州。皇妹太和公主出降回紇登羅骨沒施合毗伽可汗。甲子,命金吾大將軍胡證充送公主入回紇使,兼冊可汗。又以太府卿李銳為入回紇婚禮使。



 六月乙丑朔。辛未,吐蕃犯青塞堡。甲申,賜御史中丞牛僧孺金紫。



 秋七月乙未朔。壬寅,月掩房
 次相。壬子,群臣上尊號曰文武孝德皇帝。是日,上受冊於宣政殿,禮畢,御丹鳳樓,大赦天下。甲寅,幽州監軍使奏:「今月十日軍亂,囚節度使張弘靖別館。害判官韋雍、張宗元、崔仲卿、鄭塤。軍人取硃滔子洄為留後。」丁巳,貶張弘靖為太子賓客分司。己未,再貶弘靖為吉州刺史。硃洄自以年老,令軍人立其子無融為留後。初,劉總歸朝,籍其軍中素難制者送歸闕庭,克融在籍中。宰相崔植、杜元穎素不知兵,心無遠慮,謂兩河無虞,不復禍亂
 矣,遂奏劉總所籍大將並勒還幽州,故克融為亂,復失河北矣。庚申,以昭義軍節度使劉悟檢校司空,兼幽州大都督府長史,充幽州盧龍軍節度副大使、知節度事。以國子祭酒韓愈為兵部侍郎。辛酉,太和長色主發赴回紇,上以半仗禦通化門臨送,群臣班於章敬寺前。



 八月甲子朔。己巳,鎮州監軍宋惟澄奏:七月二十八日夜軍亂,節度使田弘正並家屬將佐三百餘口並遇害。軍人推衙將王廷湊為留後。辛未,以左金吾將軍楊元卿
 為涇州刺史,充四鎮北庭行軍、涇原節度使。敕公卿大臣至中書議幽、鎮討伐之謀。癸酉,王廷湊遣盜殺冀州刺史王進岌,據其郡。乙亥,以前涇原節度使田布起復檢校工部尚書,兼魏州大都督府長史,充魏博節度使。己卯,以深州刺史、本州團練使牛元翼充深冀節度使。辛巳夜,太白近軒轅左角。冀州刺史吳暐潛為幽州兵所逐。瀛州兵亂,囚觀察使盧士玫。瀛州尋為幽州兵所據。乙丑,以河東節度裴度充幽、鎮兩道招撫使。庚寅,以
 建州刺史李景儉為諫議大夫。壬辰夜,太白近太微西垣。癸巳,鎮州出兵圍深州。



 九月甲午朔。丁酉,廢興州鳴水縣。戊戌夜,太白近太微右執法。壬寅,大雨震霆。乙巳,相州兵亂,殺刺史邢楚。丙午,令內常侍段文政監領鄭滑、河東、許三道兵,救援深州。吐蕃請盟,許之。辛亥夜,月近天關。壬子,幽州賊掠易州淶水、遂城、滿城。癸丑,以前魏博節度使李醖為太子少保。癸酉,魏博節度使田布奏,出師五千赴貝州行營。



 冬十月甲子朔。丙寅,太中大
 夫、守刑部尚書、騎都尉王播可中書侍郎、同中書門下平章事,依前充鹽鐵轉運使。以河東節度使裴度充鎮州四面行營都招討使。以左領軍衛大將軍杜叔良充深、冀諸道行營節度使。戊辰,以深、冀節度使牛元翼為鎮州大都督府長史,充成德軍節度、鎮冀深趙等州節度使。辛未,以中書舍人、知貢舉王起為禮部侍郎,兵部郎中楊嗣復為庫部郎中、知制誥。壬申,以東都留守鄭絪為吏部尚書。以吏部尚書李絳檢校右僕射,判東都
 尚書省事、東都留守、都畿防禦使。以工部尚書丁公著檢校左散騎常侍,兼越州刺史、御史中丞,充浙東觀察使。乙亥,沂州刺史王智興為武寧軍節度副使。丁丑,裴度奏,自將兵取故關路進討。硃克融兵寇蔚州。戊寅,王廷湊兵寇貝州。易州刺史柳公濟奏於白石嶺破燕軍三千。滄州烏重胤奏,於饒陽破賊。工部尚書韋貫之卒。壬午,以尚書主客郎中、知制誥白居易為中書舍人。河東節度使裴度三上章,論翰林學士元稹與中官知樞
 密魏弘簡交通,傾亂朝政。以稹為工部侍郎,罷學士。弘簡為弓箭庫使。甲申,以京兆尹、御史大夫柳公綽為吏部侍郎。丙戌,以深冀行營節度使杜叔良為滄州刺史、橫海軍節度使,以代烏重胤;授重胤檢校司徒、興元尹,充山南西道節度使。時上急於誅賊,杜叔良出征日面辭,奏云:「臣必旦夕破賊。」重胤善將知兵,以賊勢未可卒平,用兵稍緩,故有是拜。丁亥,前浙東觀察使薛戎卒。戊子,魏博田布奏,自率全師進討。太子少保李醖卒。己丑,
 以戶部侍郎、判度支崔為工部尚書、判度支。以山南西道節度使崔從為尚書左丞;以秘書監許季同為華州刺史,充潼關防禦、鎮國軍使。辛卯,昭義劉悟奏,自將兵次臨城。



 十一月甲午朔,裴度奏破賊於會星鎮。硃克融兵大寇定州,節度使陳楚出師拒戰,破賊二萬。乙巳,徐州崔群奏,遣節度副使王智興率師赴行營。戊申,以司農卿裴武為鎮州行營供軍使。戊午,上御宣政殿,試制科舉人。辛酉,淄青牙將馬延崟謀逆,節度使薛平覺其
 謀而誅之。詔中書舍人白居易、繕部郎中陳岵、考功員外郎賈餗同考制策。十二月甲子朔。丙寅,以前容管經略使留後嚴公素為容州刺史、容管經略使。丁卯,貶諫議大夫李景儉為楚州刺史。庚午,杜叔良之軍與賊戰於博野,為賊所敗,七千人陷賊,叔良僅免。乙亥,敕諸道除上供外,留州留使錢內每貫割二百文以助軍用,賊平後仍舊。定州陳楚破硃克融賊二萬於望都。戊寅,以鳳翔節度使李光顏為忠武軍節度使,代李遜,仍兼深、
 冀行營節度。以李遜為鳳翔節度使。貶員外郎獨孤朗韶州刺史,起居舍人溫造朗州刺史,司勛員外郎李肇澧州刺史,刑部員外郎王鎰郢州刺史,坐與李景儉於史館同飲,景儉乘醉見宰相謾罵故也。兵部郎中知制誥馮宿、庫部郎中知制誥楊嗣復各罰一季俸料,亦坐與景儉同飲,然先起,不貶官。辛巳,李光顏赴鎮,百僚餞於章敬寺。上禦通化門臨送,賜玉帶名馬。仍敕神策副使楊承和充深、冀行營都監押。壬午,出內庫錢五萬貫
 以助軍。乙酉,以幽州都知兵馬使硃克融檢校左散騎常侍,充幽州盧龍軍節度使,其拘囚張弘靖、殺害府僚之罪,一切釋放。時朝議以克融能保全弘靖,王廷湊殺害弘正,可赦燕而誅趙,故有是詔。是歲,天下戶計二百三十七萬五千八百五,口一千五百七十六萬二千四百三十二,元不進戶軍州不在此內。



 二年春正月癸巳朔,以用兵罷元會。乙未,以夔州刺史王承弁為安南都護、本管經略招討使。丁酉,硃克融陷
 滄州弓高縣,賊攻下博,兼邀餉道車六百乘而去。庚子,魏博兵自潰於南宮縣。戊申,魏博牙將史憲誠奪師,田布伏劍而卒。己酉,以魏博中軍先鋒兵馬使史憲誠檢校工部尚書,兼魏州大都督府長史魏博節度使。是日,大風霾。庚戌,以德州刺史王日簡為滄州刺史,充橫海軍節度、滄德棣觀察等使,以代叔良。壬子,貶叔良為歸州刺史,以獻計誅幽鎮無功,而兵敗喪所持旌節也。甲寅,以工部尚書、判度支崔倰檢校禮部尚書,兼鳳翔尹,充鳳
 翔隴節度使。以鴻臚卿、兼御史大夫張平叔判度支。復以弓高縣為景州。青州奏海凍二百里。乙卯,以前鳳翔節度使李遜為刑部尚書。己未,刑部尚書李遜卒。庚子,以兗、沂、密觀察使曹華為節度使;以天德軍防禦使李進誠兼靈州刺史,充朔方、靈、鹽定遠城等州節度使;以晉州刺史李岵為豐州刺史,充天德軍、豐州、東西受降城都防禦使。內出繒帛八萬匹以助軍。權停嶺南、黔中今年選補。



 二月癸亥朔。甲子,詔雪王廷湊,仍授鎮州大都督府
 長史、御史大夫,充成德軍節度、鎮冀深趙等州觀察等使。三軍將士,待之如初。仍令兵部侍郎韓愈往彼宣諭。以前吉州刺史張弘靖為撫州刺史。弘靖初貶官,尚在幽州,拘留半歲,克融授節,始得還,故有是命。丙寅,以前成德軍節度使牛元翼檢校工部尚書、襄州刺史,充山南東道節度觀察、臨漢監牧等使。丁卯,以考功郎中、知制誥李德裕為書舍人,依前翰林學士。癸酉,以鄜坊節度使韓充為義成軍節度使,以代王承元。以承元為鄜坊
 節度使。甲戌夜,火、木星相近。滄州節度使王日簡賜姓名李全略。辛巳,以正議大夫、守中書侍郎、同中書門下平章事、武騎尉、賜紫金魚袋崔植為刑部尚書,罷知政事。以工部侍郎元稹守本官、同平章事。以翰林學士、中書舍人李德裕為御史中丞。司勛員外郎、知制誥李紳為中書舍人,依前翰林學士。右庶子王仲周以奉使緩命,貶臺州刺史。癸未,以深、冀行營諸軍節度、忠武軍節度使李光顏為滄州刺史、橫海軍節度使,兼忠武
 軍節度、深冀行營並如故;以橫海軍節度使李全略為德州刺史、德棣等州節度。丙戌,以兵部郎中、知制誥馮宿檢校左庶子,充山南道節度副使,權知襄州軍府事,以牛元翼在深州重圍故也。丁亥,以河東節度使、司空、兼門下侍郎、平章事裴度守司徒、平章事,充東都留守,判東都尚書省事、都畿汝防禦使、太微宮等使;以前靈武節度使李聽為太原尹、北都留守、河東節度使。



 三月壬辰朔,詔曰:「武班之中,淹滯頗久。又諸薦送大將,或
 隨節度使歸朝。自今已後,宜令神策六軍軍使及南衙常參武官,各具歷任送中書門下,素立大功及有才器者,量加獎擢。常參官依月限改轉,諸道軍府帶監察已上官者,限三周年即與改轉。軍士死王事者,三周年內不得停衣糧。先於留州留使錢內每貫割二百文助軍,今後不用抽取。」上於馭軍之道,未得其要,常云宜姑息其臣。故即位之初,傾府庫頒賞之,長行所獲,人至巨萬,非時賜與,不可勝紀。故軍旅益驕,法令益弛,戰則不克,
 國祚日危。洎頒此詔,方鎮多以大將文符鬻之富賈,曲為論奏,以取朝秩者,疊委於中書矣。名臣扼腕,無如之何,癸巳,以兵部尚書蕭俛為太子少保,以前山南東道節度使李逢吉為兵部尚書。壬寅,左驍衛上將軍張奉國卒。以鴻臚卿、判度支張平叔為戶部侍郎唷職。平叔以曲承恩顧,上疏請官自賣鹽,可以富國強兵,陳利害十八條。詔下其疏,令公卿詳議。中書舍人韋處厚隨條詰難,固言不可,事遂不行。硃克融、王廷湊合兵攻深州,
 不解。裴度與書諭之,克融還鎮,廷湊攻城亦緩,乃並加檢校工部尚書。戊申,裴度來朝,對於麟德殿,伏奏龍墀,因敘河北用兵,嗚咽流涕,上改容慰勞之。壬子,以新授東都留守裴度為揚州大都督府長史,充淮南節度使。癸丑,徐州節度使崔群為其副使王智興所逐,智興自專軍務。甲寅,以右僕射韓皋為左僕射,以前淮南節度使李夷簡為右僕射。前東都留守李絳復拜舊官。丙辰,守司徒裴度正衙受冊訖,謁太廟,赴尚書省上,宰臣百
 僚皆送。丁己,以左丞崔從檢校禮部尚書、鄜州刺史、鄜坊節度使,以代王承元。以承元為鳳翔、隴節度使。戊午,司徒裴度復入中書知政事。以中書侍郎、平章事王播檢校右僕射,兼揚州大都督府長史,充淮南節度使,依前兼諸道鹽鐵轉運使。以鳳翔節茺使崔俛為河南尹。牛元翼率十餘騎突圍出深州來朝,深州大將臧平等一百八十人皆為王廷湊所殺。己未,以武寧軍節度副使王智興檢校工部尚書,兼徐州刺史,充武寧軍節度使。
 以德、棣節度使李全略復為滄州節度使,仍合滄、景、德、棣為一鎮。李光顏還鎮許州。



 夏四月辛酉朔,日有蝕之。甲子,左僕射韓皋赴省上,中使賜酒饌,宰臣百僚送,一如近式。雲陽縣角抵力人張蒞負羽林官騎康憲錢。憲往徵之。蒞乘醉打憲將殞,憲男買德年十四,持木鐘擊蒞首破,三日而卒。刑部奏覆,敕曰:「買德尚在童年,能知子道。雖殺人當死,為父可哀。若從沉命之科,恐失原情之意。可減死罪一等。」忻州刺史李寰守博野,王廷湊玫
 之不下。其李寰所領兵宜割屬右神策,以寰為軍使,仍以忻州軍為名。庚辰,桂管觀察使杜式方卒。癸未,以武寧軍節度使崔群為秘書監,分司東都。翰林侍講學士韋處厚、路隨進所撰《六經法言》二十卷,賜錦彩二百匹、銀器二百事,處厚改中書舍人,隨改諫議大夫,並賜金紫。丁亥,以秘書監嚴譽為桂管觀察使。是夜,東北有流星,光彩燭地,殷殷有聲,擊天市垣,至郎位滅。



 五月辛卯朔。以德州刺史李景儉為諫議大夫。癸丑,太子少傅嚴
 綬卒。戊午,幽州硃克融上表進馬萬匹、羊十萬口,先請其價賞軍。隴山有異獸如猴,腰尾皆長,色青赤而猛鷙,見蕃人則躍而食之,遇漢人則否。



 六月庚申朔。甲子,司徒、平章事裴度守尚書右僕射,工部侍郎、平章事元稹為同州刺史史。以正議大夫、守兵部尚書、輕車都尉李逢吉為門下侍郎、同中書門下平章事。乙丑,大風震電,墜太廟鴟吻,霹御史臺樹。丁卯,以易州刺史柳公濟為定州刺史、義武節度使。壬申,諫官論責裴度太重,元稹
 太輕,乃追稹制書,削長春宮使。戊寅,以前右僕射李夷簡為太子少保,分司東都。戊子,復置邕管,以安南副使崔結為邕管經略使。秋七月己丑朔。丙申,宋王結薨,廢朝。戊戌,汴州軍亂,逐節度使李願,立牙將李翽為留後。好畤縣山水漂溺居人三百家。陳、許、蔡等州水。壬寅,出中書舍人白居易為杭州刺史。乙巳,詔南北省五品已上官議討李翽。丙午,貶李願為隨州刺史。以鄭、滑節度使韓充為汴州刺史、宣武軍節蓄使、汴宋亳潁觀察等
 使,鄭、滑如故;以宣武軍節度押衙李為右金吾衛將軍。丁未,內出綾絹五十萬匹付度支,以供軍用。陳、許水災,賑粟五萬石。己酉,中使楊瑞昌使鎮州。王廷湊奏:「奉詔取牛元翼家族,請至秋末發遣。其田弘正骸骨,尋訪不知所在。」辛亥,以贈司徒、忠烈公李心妻子源為諫議大夫,賜緋魚袋。乙卯,敕:「員外郎知刺誥二年後轉郎中,又二年後轉前行郎中,又一年即正除;諫議大夫知同前郎中;給事中並翰林學士別宣知者,不在此限。」以前義
 武軍節度使陳楚為東都留守、判尚書省事、東畿汝防禦使。本朝故事,東都留守罕用武臣,今用楚,以李翽擾汴、宋故也。八月己未朔,以絳州刺史崔弘禮為河南尹,兼東畿防禦副使。給事中韋穎以弘禮望輕,封還詔書,上遣中使諭之,乃下。詔陳、許李光顏將兵收汴州。戊辰,以左僕射韓皋為東都留守、判尚書省事、東畿汝防禦使。以東都留守陳楚為河陽懷節度使。癸酉,韓充奏今月六日發軍入汴州界,營於千塔。丙子,汴州監軍姚文
 壽與兵馬使李質同謀斬李翽及其黨薛志忠、秦鄰等。丁丑,韓充入汴州。以前東都留守李絳為華州刺史,充潼關防禦、鎮國軍等使。浙東處州大水,溺居民。以兗、海沂密節度使曹華為滑州刺史,充義成軍節度、鄭滑潁等州觀察等使;以宋州刺史高承簡為兗州刺史、兗海沂密等州節度使;以汴州防城兵馬使李質為右金吾衛將軍。潁州棣鄭、滑觀察使。鹽鐵轉運使王播進《開潁口圖》。



 九月戊子朔,浙西大將王國清謀叛,觀察使竇易
 直討平之,同惡二百餘人並誅之。韓充送李翽男道源、道樞、道瀹等三人,斬於西市;翽妻馬氏、小男道本、女汴娘配於掖庭。壬子,太子少師李夷簡卒贈太子太保。癸卯,以前河陽節度使郭釗為河中尹,兼河中、絳、隰等州節度使。御使中丞李德裕為潤州刺史、兼御史大夫、浙江西道都團練觀察處置等使,以代竇易直。以易直為吏部會議郎。加晉州刺史李寰為晉、慈等州都團練觀察使。乙巳,敕團練防禦州置判官一員,其副使推巡並停。
 辛亥,以吏部侍郎柳公綽為御史大夫。先有詔廣芙蓉苑南面,居人廬舍墳墓並移之,群情駭擾。癸丑,降敕罷之。德州軍亂,害剌史王稷,盡剽其家財奴僕。丁巳,以萬州刺史李元喜為安南都護。陰山府沙陀突厥兵馬使硃耶執宜來朝貢,賜官誥、錦彩、銀器。



 冬十月戊午朔。壬戌,前河中晉、絳、慈、隰等州節度使、開府儀同三司、守司徒、中書令、河中尹、上柱國、許國公韓弘可守司徒、兼中書令。甲子夜,月掩牽牛中星。戊辰,興元節度使以色列重胤來
 朝,移授天平軍節度使。己卯,以工部侍郎鄭權為工部尚書,以前華州刺史許季同為工部侍郎。是日,上由復道幸咸陽,止於善因佛寺,施僧錢百萬,咸陽令絹百匹。閏十月戊子朔,入回紇使金吾大將軍胡證、副使光祿卿李憲、婚禮使衛尉卿李銳、副使宗正少卿李子鴻等,送太和公主自蕃中回。庚寅,以吏部尚書鄭絪為太子少傅;以太常卿趙宗儒為吏部尚書;韋綬為興元尹,充山南西道節度使。壬辰,右驍衛大將軍韓公武卒,廢朝。
 以戶部尚書楊於陵為太常卿。丙申,回紇可汗遣使獻國信四床、女口六人、葛祿口四人。己亥,敕翰林侍講學士諫議大夫路隨、中書舍人韋處厚,兼充史館修撰《憲宗實錄》,仍更日入史館。《實錄》未成,且許不入內署,仍放朝參。甲寅,詔:「江淮諸州旱損多,所在米價不免踴貴,眷言疲困,須議優矜。宜委淮南、浙西東、宣歙、江西、福建等道觀察使,各於當道有水旱處,取常平義倉斛斗,據時估減半價出糶,以惠貧民。」丙辰,以太子賓客令狐楚
 為陜、虢觀察使。十一月丁巳朔。丁卯,尚書左丞庾承宣為陜、虢觀察使。令狐楚復為太子賓客,分司東都。楚已至陜州視事一日,追改之。庚午,命景王率禁軍五百騎,侍從皇太后幸華清宮,又幸石甕寺。辛未,以前安南都護桂仲武為邕管經略使。癸酉,上幸華清宮迎太后,巡狩於驪山下,即日馳還,太后翌日方還。丙子,集王緗薨。庚辰,上與內官擊鞠禁中,有內官欻然墜馬,如物所擊。上恐,罷鞠升殿,足不能履地,風眩就床。自是外不聞
 上起居者三日。是夜,月近房。十二月丁亥朔,詔五坊鷹隼並解放,獵具皆毀之。庚寅,宰臣李逢吉率百僚至延英門請見,上不許。中外與度等三上疏,請立皇太子。是夜,司徒、中書令韓弘卒。辛卯,上於紫宸殿御大繩床見百官,李逢吉奏景王成長,請立為皇太子,左僕射裴度又極言之。癸已,詔景王為皇太子。淮南奏和州饑,烏江百姓殺縣令以取官米。甲午,內出絹二百匹,賑兩市癃殘窮者。己未,兩軍容內司公主戚屬之家,並以上疾痊
 平,諸寺為僧齋。仍敕在京諸司疏放系囚。丙午,上御宣政殿冊皇太子。受冊畢,百僚謁太子於東宮,太子舉簾,執笏答拜,宮僚拜則受之。丁未,判度支、戶部侍郎張平叔貶通州刺史。是夜,月掩左角。己酉,以前天平軍節度使馬總檢校左僕射、守戶部尚書。庚戌,以吏部侍郎竇易直為戶部侍郎、判度支。癸丑,以太子冊禮畢,宣制赦囚徒。以前黔中觀察使崔元略為鄂、岳、蘄、黃、安等州觀察使。太子賓客孟簡卒。乙卯,以前陜虢觀察使衛中行
 為尚書右丞。是冬十月頻雪,其後恆燠,水不冰凍,草木萌發,如正二月之後。



 三年正月丁已朔,上以疾不受朝賀。是日大風,昏翳竟日。嗣郢王佐宜於崖州安置,坐妄傳禁中語也。敕不得買新羅人為奴婢,已在中國者即放歸其國。禮部侍郎王起奏:當司所試貢舉人,試訖申送中書,候覆訖下當司,然後大字放榜。從之。



 二月,天平軍監軍奏:節度使烏重胤病,牙將王贄割股肉以療,河陽節度使陳楚奏:移
 使府於三城,未有門戟,欲移懷州門戟於河陽。從之。廣東省議大夫殷侑奏禮部貢舉請置《三傳》、《三史》科,從之。戶部尚書。崔倰卒。



 三月丁已,宰臣百僚賜宴於曲江亭。敕應御服及器用在淮南、兩浙、宣歙等道合供進者,並端午誕節常例進獻者,一切權停。其鷹犬之類,除備蒐狩外,並令解放。以牛僧孺同中書門下平章事。日晡晚後,有賊入通化門,鬥死者一人,傷者六人。賜宣徽院從奉官錢一百二十貫文已下有差。



 五月,山南西道奏移成
 州於寶井堡。山南東道節度使牛元翼卒。秘書少監李隨奏請造當司圖書印一面,從之。



 六月,宰相監修圖史杜元穎奏:史官沈傳師除鎮湖南,其本分修史,便令將赴本任修撰。從之。敕京兆尹、御史大夫韓愈宜放臺參,後不得為例。七月,國子祭酒韋乾慶卒。



 八月,鄭、滑節度使曹華卒。檢校尚書右僕射、戶部尚書馬總卒。興元節度使韋綬卒。上由復道幸興慶宮,至通化門,賜持盂僧絹二百匹。因幸五方,賜從官金銀鋌有差。



 九月,澤潞節
 度使劉悟進位平章事。賜宰臣百僚重九宴於曲江亭。南詔王丘佺進金碧文絲十有六品。十月,以京兆尹韓愈為兵部侍郎,以御史中丞李紳為江西觀察使。宰相李逢吉與李紳不協,紳有時望,恐用為相。及紳為中丞,乃除韓愈為京兆尹、兼御史大夫,仍放臺參。紳性峭直,屢上疏論其事,遂與愈辭理往復,逢吉乃兩罷之。然紳出而愈留。宰相杜元穎罷知政事,除成都尹、劍南西川節度使。龍武統軍陳楚卒。以兵部侍郎韓愈為吏部侍
 郎,新除江西觀察使李紳為戶部侍郎。紳既罷除江西,上令中使就第賜玉帶,紳因除敘泣而請留,中使具奏,故與愈俱改官。召翰林學士龐嚴對,因賜金紫。賜內園使公廨本錢一萬貫,軍器使三千貫。杜元穎赴鎮蜀,上御安福門餞,因賜皇城留守及金吾衛率等帛有差。



 十一月,上禦通化門,觀作毗沙門神,因賜絹五百匹。停浙東貢甜菜、海蚶。十二月,浙西觀察使李德裕奏去管內淫祠一千一十五所。



 四年正月辛亥朔,上御殿受朝如常儀。上餌金石之藥,處士張皋上疏切諫,上悅,召之,求皋不獲。澤、潞判官賈直言新授諫議大夫,劉悟上表乞留,從之。禮部尚書致仕孔戣卒。辛未,上大漸,詔皇太子監國。壬申,上崩於寢殿,時年三十。群臣上謚曰睿聖文惠孝皇帝,廟號穆宗。十一月庚申,葬於光陵。



 史臣曰:臣觀五運之推遷,百王之隆替,亦無常治,亦無常亂,在人而已,匪降自天。當軒黃御宇之秋,則百年無
 事;及商辛握圖之日,則四海橫流。昔章武皇帝國命之不行,惜朝綱之將墜,乃求賢俊,總攬英雄,果能扼大盜之喉,制奸臣之命。五十載已終之土,復入提封;百萬戶受弊之氓,重蘇景化。元和之政,幾致升平。鴟梟方革於好音,龍鼎俄傷於短祚。茍或時有平、勃之佐,繼以文、景之才,則延湊、克融,自縮螳螂之臂;智興、李,敢萌狗鼠之謀?強盜寧窺孟賁之金,餓隸不拾嬰兒之餌。觀夫孱主,可謂痛心,不知創業之艱難,不恤黎元之疾苦。謂
 威權在手,可以力制萬方;謂旒冕在躬,可以坐馳九有。曾不知聚則萬乘,散則獨夫,朝作股肱,暮為仇敵。仲長子所謂「至於運徙勢去,獨不覺悟者,豈非富貴生不仁,沉溺致愚疾。存亡以之迭代,治亂從此周復。」誠哉是言也!贊曰:惠王不令,敗度亂政。驕僻偶全,實賴遺慶。皇皇上帝,為民立正。此何人哉,遽主鼎命。



\end{pinyinscope}