\article{卷一百七十二}

\begin{pinyinscope}

 ○韋溫蕭祐附獨孤鬱弟朗錢徽子可復高釴弟銖鍇馮宿弟定審封敖



 韋溫,字弘育,京兆人。祖肇,吏部侍郎。父綬,德宗朝翰林
 學士,以散騎常侍致仕。綬弟貫之,憲宗朝宰相,自有傳。溫七歲時,日念《毛詩》一卷。年十一歲,應兩經舉登第。釋褐太常寺奉禮郎。以書判拔萃,調補秘書省校書郎。時綬致仕田園,聞溫登第,愕然曰:「判入高等,在群士之上,得非交結權幸而致耶?」令設席於廷,自出判目試兩節。溫命筆即成,綬喜曰:「此無愧也!」調授咸陽尉。入為監察御史,以父在田里,憲府禮拘,難於省謁,不拜。換著作郎,一謝即還。侍省父疾,溫侍醫藥,衣不解帶,垂二十年。父
 憂,毀瘠逾制。免喪,久之為右補闕,忠鯁救時。宋申錫被誣,溫倡言曰:「宋公履行有素,身居臺輔,不當有此,是奸人陷害也。吾輩諫官,豈避一時之雷電,而致聖君賢相蒙蔽惑之咎耶?」因率同列伏閣切爭之,由是知名。



 太和五年,太廟第四、第六室缺漏,上怒,罰宗正卿李銳、將作王堪,乃詔中使鳩工補葺之。溫上疏曰:「臣聞吏舉其職,國家所以治;事歸於正,朝廷所以尊。夫設制度,立官司,事存典故,國有經費,而最重者,奉宗廟也。伏以太廟當
 修,詔下逾月,有司弛墮,曾不加誡。宜黜慢官,以懲不恪之罪;擇可任者,責以繕完之功。此則事歸於正,吏舉其職也。而聖思不勞,百職無曠。今慢官不恪,止於罰俸,宗廟所切,便委內臣,是許百司之官,公然廢職,以宗廟之重,為陛下所私,群官有司,便同委棄。此臣竊為聖朝惜此事也。事關宗廟,皆書史策,茍非舊典,不可率然。伏乞更下詔書,得委所司營繕,則制度不紊,官業交修。」上乃止內使。



 群臣上尊號,溫上疏曰:「德如三皇止稱皇,功如
 五帝止稱帝。徽號之來,乃聖王之末事。今歲三川水災,江淮旱歉,恐非崇飾徽稱之時。」帝深嘉之,乃止。改侍御史。



 李德裕作相,遷禮部員外郎。或以溫厚於牛僧孺,言於德裕。德裕曰:「此人堅正中立,君子也。」鄭注鎮鳳翔,自知不為所齒,求德門弟子為參佐,請溫為副使。或以為理不可拒,拒則生患。溫曰:「擇禍莫若輕。拒之止於遠貶,從之有不測之禍。」鄭注誅,轉考功員外郎。尋知制誥,召入翰林為學士。以父職禁廷,憂畏成病,遺誡不令居禁
 職,懇辭不拜。



 俄兼太子侍讀,每晨至少陽院,午見莊恪太子。溫曰:「殿下盛年,宜早起,學周文王為太子,雞鳴時問安西宮。」太子幼,不能行其言。稱疾。上不悅,改太常少卿。未幾,拜給事中。王晏平為靈武,刻削軍士,贓罪發,帝以智興之故,減死,貶官。溫三封詔書,文宗深獎之。莊恪得罪,召百僚諭之。溫曰:「太子年幼,陛下訓之不早,到此非獨太子之過。」遷尚書右丞。



 吏部員外郎張文規父弘靖,長慶初在幽州為硃克融所囚;文規不時省赴,人士
 喧然罪之。溫居綱轄,首糾其事,出文規為安州刺史。鹽鐵判官姚勖知河陰院,嘗雪冤獄。鹽鐵使崔珙奏加酬獎,乃令權知職方員外郎。制出,令勖上省。溫執奏曰:「國朝已來,郎官最為清選,不可以賞能吏。」上令中使宣諭,言勖能官,且放入省。溫堅執不奉詔,乃改勖檢校禮部郎中。翌日,帝謂楊嗣復曰:「韋溫不放姚勖入省,有故事否?」嗣復對曰:「韋溫志在銓擇清流。然姚勖士行無玷,梁公元崇之孫,自殿中判鹽鐵案,陛下獎之,宜也。若人有
 吏能,不入清流,孰為陛下當煩劇者?此衰晉之風也。」上素重溫,亦不奪其操,出為陜虢觀察使。



 武宗即位,李德裕用事,召拜吏部侍郎,欲引以為相。時李漢以家行不謹,貶汾州司馬。溫從容白德裕曰:「李漢不為相公所知,昨以不孝之罪絀免,乞加按問。」德裕曰:「親情耶?」溫曰:「雖非親暱,久相知耳。」德裕不悅。居無何,出溫為宣歙觀察使,闢鄭處誨為觀察判官,德裕愈不悅。池州人訟郡守,溫按之無狀,杖殺之。



 明年,瘍生於首,謂愛婿張復魯曰:「
 予任校書郎時,夢二黃衣人齎符來追,及滻,將渡,一人續至曰:『彼墳至大,功須萬日。』遂不涉而寤。計今萬日矣,與公訣矣。」明日卒,贈工部尚書,謚曰孝。



 溫在朝時,與李玨、楊嗣復周旋。及楊、李禍作,嘆曰:「楊三、李七若取我語,豈至是耶!」初溫以楊、李與德裕交怨,及居位,溫勸楊、李徵用德裕,釋憾解慍。二人不能用,故及禍。溫無子,女適薛蒙,善著文,續曹大家《女訓》十二章,士族傳寫,行於時。溫剛腸寡合,人多疏簡,唯與常侍蕭祐善。



 蕭祐者,蘭陵人。少孤貧。耿介苦學,事親以孝聞。自處士徵拜左拾遺,累遷至考功郎中。祐博雅好古,尤喜圖畫。前代鐘、王遺法,蕭、張筆勢,編序真偽,為二十卷,元和末進御,優詔嘉之,授兵部郎中。出為虢州刺史,入為太常少卿,轉諫議大夫。逾月為桂州刺史、御史中丞、桂管防禦觀察使。太和二年八月,卒於官,贈右散騎常侍。



 祐閑淡貞退,善鼓琴賦詩,書畫盡妙。游心林壑,嘯詠終日,而名人高士,多與之游。給事中韋溫尤重之,結為林泉之
 友。



 獨孤鬱,河南人。父及,天寶末與李華、蕭潁士等齊名。善為文,所著《仙掌銘》,大為時流所賞,位終常州刺史。鬱,貞元十四年登進士第,文學有父風,尤為舍人權德輿所稱,以子妻之。貞元末,為監察御史。



 元和初,應制舉才識兼茂、明於體用,策入第四等,拜左拾遺。太子司議郎杜從鬱拜左補闕,鬱與同列,論之曰:「從鬱是宰臣佑之子,父居宰執,從鬱不宜居諫列。」乃改為左拾遺,又論曰:「補
 闕之與拾遺,資品雖殊,同是諫官,若時政或有得失,不可令子論父。」從鬱竟改他官。



 四年,轉右補闕,又與同列拜章論中官吐突承璀不宜為河北招討使,乃改招撫宣慰使。



 五年,兼史館修撰。尋召充翰林學士,遷起居郎。權德輿作相,鬱以婦公辭內職。憲宗曰:「德輿乃有此佳婿。」因詔宰相於士族之家,選尚公主者。遷鬱考功員外郎,充史館修撰、判館事,預修《德宗實錄》。



 七年,以本官復知制誥。八年,轉駕部郎中。其年十月,復召為翰林學士。
 九年,以疾辭內職。十一月,改秘書少監,卒。



 鬱弟朗,嘗居諫官,請罷淮西用兵,不協旨,貶興元戶曹。入為監察御史,轉殿中。十五年,兼充史館修撰,遷都官員外郎。



 長慶初,諫議大夫李景儉於史館飲酒,憑醉謁宰相,語辭侵侮;朗坐同飲,出為漳州刺史。入為左司員外郎,遷諫議大夫。揚州節度使王播罷兼鹽鐵使,行賂於中人,求復領銅鹽。朗上章論之。



 寶歷元年十一月,拜御史中丞。二年六月,賜金紫之服。侍御史李道樞乘醉謁朗;朗劾之,左授司議郎。憲府故事,三院御史由大夫、中丞自闢,請命於朝。時崔晃、鄭居中不由憲長而除,皆丞相之僚舊也,敕命雖行,朗拒而不納,晃竟改太常博士,居中分司東臺。其年十月,高少逸入閣失儀,朗不彈奏,宰相銜阻崔晃事,左授少逸贊善大夫,
 朗亦罰俸。朗稱執法不稱,乞罷中丞,敬宗令中使諭之,不允其讓。文宗即位,改工部侍郎。太和元年八月,出為福州刺史、御史中丞、福建觀察使。是月赴官,暴卒於路,贈右散騎常侍。



 鬱子庠,亦登進士第。大中後官達,亦至侍郎。



 錢徽,字蔚章,吳郡人。父起,天寶十年登進士第。起能五言詩。初從鄉薦,寄家江湖,嘗於客舍月夜獨吟,遽聞人吟於庭曰:「曲終人不見,江上數峰青。」起愕然,攝衣視之,無所見矣,以為鬼怪,而志其一十字。起就試之年,李暐所試《湘靈鼓瑟詩》題中有「青」字,起即以
 鬼謠十字為落句,暐深嘉之,稱為絕唱。是歲登第,釋褐秘書省校書郎。大歷中,與韓翃、李端輩十人,俱以能詩,出入貴游之門,時號「十才子」,形於圖畫。起位終尚書郎。



 徽,貞元初進士擢第,從事戎幕。元和初入朝,三遷祠部員外郎,召充翰林學士。六年,轉祠部郎中、知制誥。八年,改司封郎中、賜緋魚袋,職如故。九年,拜中書舍人。十一年,王師討淮西,詔朝臣議兵,徽上疏言用兵累歲,供饋力殫,宜罷淮西之徵。憲宗不悅,罷徽學士之職,守本官。



 長慶元年,為禮部侍郎。時宰相段文昌出鎮蜀川。文昌好學,尤喜圖書古畫。故刑部侍郎楊憑兄弟,以文學知名,家多書畫,鐘、王、張、鄭之跡在《書斷》、《畫呂》者,兼而有之。憑子渾之求進,盡以家藏書畫獻文昌,求致進士第。文昌將發,面托錢徽,繼以私書保薦。翰林學士李紳亦托舉子周漢賓於徽。及榜出,渾之、漢賓皆不中選。李宗閔與元稹素相厚
 善。初稹以直道譴逐久之,及得還朝,大改前志。由逕以徼進達,宗閔亦急於進取,二人遂有嫌隙。楊汝士與徽有舊。是歲,宗閔子婿蘇巢及汝士季弟殷士俱及第。故文昌、李紳大怒。文昌赴鎮。辭日,內殿面奏,言徽所放進士鄭朗等十四人,皆子弟藝薄,不當在選中。穆宗以其事訪於學士元稹、李紳,二人對與文昌同。遂命中書舍人王起、主客郎中知制誥白居易,於子亭重試,內出題目《孤竹管賦》、《鳥散餘花落》詩,而十人不中選。詔曰:



 國家設文學之科,本求才實,茍容僥幸,則異至公。訪聞近日浮薄之徒,扇為朋黨,謂之關節,干撓主司。每歲策名,無不先定,永言敗俗,深用興懷。鄭朗等昨令重試,意在精核藝能,不於異書之中,固求深僻題目,責令所試成就,以觀學藝淺深。孤竹管是祭天之樂,出於《周禮》正經;閱其呈試之文,都不知其本事,辭律鄙淺,蕪累亦多。比令宣示錢徽,庶其深自懷愧,誠宜盡棄,以警將來。但以四海無虞,人心方泰,用弘寧撫,式示殊恩,特掩爾瑕,庶明予志。孔溫業、趙存約、竇洵直所試粗通,與及第;裴撰特賜及第;鄭朗等十人並落下。自今後禮部舉人,宜準開元二十五年敕,及第訖,所試雜文並策,送中書門下詳覆。



 尋貶徽為江州刺史,中書舍人李宗閔劍州刺史,右補闕楊汝士開江令。初議貶徽,宗閔、汝士令徽以文昌、李紳私書進呈,上必開悟。徽曰:「不然。茍無愧心,得喪一致,修身慎行,安可以私書相證耶?」令子弟焚之,人士稱徽長者。



 既而穆宗知其朋比之端,乃下詔曰:



 昔者,卿大夫相與讓於朝,士庶人相與讓於列;周成王刑措不用,漢文帝恥言人過,真理古也,朕甚慕焉。中代已還,爭端斯起,掩抑其言則專蔽,誘掖其說則侵誣。自非責實循名,不能彰善癉惡,故孝宣必有告訐及下,光武不以單辭遽行。《語》稱訕上之非,律有匿名之禁,皆以防三至之毀,重兩造之明。是以爵人於朝則皆勸,刑人於市則皆懼,罪有歸而賞當事也。



 末代偷巧,內荏外剛。卿大夫無進思盡忠之誠,多退有後言之謗;士庶人無切磋琢磨之益,多鑠浸潤之讒。進則諛言諂笑以相求,退則群居州處以相議。留中不出之請,蓋發其陰私;公論不容之誅,是生於朋黨。擢一官,則曰恩皆自我;黜一職,則曰事出他門。比周之跡已彰,尚矜介特;由徑之蹤盡露,自謂
 貞方。居省寺者不以勤恪蒞官,而曰務從簡易;提紀綱者不以準繩檢下,而曰密奏風聞。獻章疏者更相是非,備顧問者互有憎愛。茍非秦鏡照膽,堯羊觸邪,時君聽之,安可不惑?參斷一謬,俗化益訛。禍發齒牙,言生枝葉,率是道也,朕甚憫焉。



 我國家貞觀、開元,同符三代,風俗歸厚,禮讓皆行。兵興已來,人散久矣。始欲導之以德,不欲驅之以刑。然而信有未孚,理有未至,曾無恥格,益用雕元刂。小則綜核之權,見侵於下輩;大則樞機之重,旁撓
 於薄徒。尚念因而化之,亦冀去其尤者。而宰臣懼其浸染,未克澄清。備引祖宗之書,願垂勸誡之詔,遂伸告諭,頗用殷勤。各當自省厥躬,與我同底於道。



 元稹之辭也。制出,朋比之徒,如撻於市,咸睚眥於紳、稹。



 徽明年遷華州刺史、潼關防禦、鎮國軍等使。文宗即位,徵拜尚書左丞。太和元年十二月,復授華州刺史。二年秋,以疾辭位,授吏部尚書致仕。三年三月卒,時年七十五。子可復、可及,皆登進士第。



 可復累官至禮部郎中。太和九年,鄭注
 出鎮鳳翔,李訓選名家子以為賓佐,授可復檢校兵部郎中、兼御史中丞,充鳳翔節度副使。其年十一月,李訓敗,鄭注誅,可復為鳳翔監軍使所害。



 高釴,字翹之。祖鄭賓,宋州寧陵令。父去疾,攝監察御史。釴,元和初進士及第,判入等,補秘書省校書郎,累遷至右補闕,充史館修撰。十四年,上疏請不以內官為京西北和糴使。十五年,轉起居郎,依前充職。



 釴孤貞無黨,而能累陳時政得失。長慶元年,穆宗憐之,面賜緋於思政
 殿,仍命以本官充翰林學士。二年,遷兵部員外郎,依前充職。四年四月,禁中有張韶之變,敬宗幸左軍。是夜,釴從帝宿於左軍。翌日賊平,賞從臣,賜釴錦彩七十匹,轉戶部郎中、知制誥。十二月,正拜中書舍人,充職如故。謝恩於思政殿,因諫敬宗,以求理莫若躬親,用示憂勤之旨也。帝深納其言,又賜錦彩五十匹。



 寶歷二年三月,罷學士,守本官。太和三年七月,授刑部侍郎。四年冬,遷吏部侍郎。銓綜之司,官業振舉。七年,出為同州刺史、兼御
 史中丞。八年六月卒,贈兵部尚書,遺命薄葬。釴少時孤貧,潔己力行,與弟銖、鍇皆以檢靜自立,致位崇顯,居家友睦,為搢紳所重。



 銖,元和六年登進士第。穆宗即位,入朝為監察御史,累遷員外郎、吏部郎中。太和五年,拜給事中。七年,為外官監考使。八年十月,文宗用國子助教李仲言為侍講,銖率諫官伏閣論曰:「仲言素行纖邪,若聽用,必亂國經。」上令中使宣諭曰:「朕要仲言講書,非有聽用也。」是歲,先旱後水,京師穀價騰踴;彗星為變,舉選
 皆停,人情雜然流議。鄭注奸謀,日聞於外。銖等犯難論諍,冀上省悟。既奉宣傳,相顧失色,以其危亡可翹足而待也。明年,訓、注竊權,惡銖不附己,五月,出為越州刺史、御史中丞、浙東觀察使。開成三年,就加檢校左散騎常侍,尋入為刑部侍郎。四年七月,出為河南尹。會昌末,為吏部侍郎。



 鍇,元和九年登進士第,升宏辭科,累遷吏部員外。太和三年,準敕試別頭進士明經鄭齊之等十八人。榜出之後,語辭紛競。監察御史姚中立以聞,詔鍇審
 定。乃升李景、王淑等,人以為公。六年二月,自司勛郎中轉諫議大夫。七年,遷中書舍人。九年十月,以本官權知禮部貢舉。開成元年春,試畢,進呈及第人名,文宗謂侍臣曰:「從前文格非佳,昨出進士題目,是朕出之,所試似勝去年。」鄭覃曰:「陛下改詩賦格調,以正頹俗,然高鍇亦能勵精選士,仰副聖旨。」帝又曰:「近日諸侯章奏,語太浮華,有乖典實。宜罰掌書記,以誡其流。」李石曰:「古人因事為文,今人以文害事,懲弊抑末,實在盛時。」乃以鍇為禮
 部侍郎。凡掌貢部三年,每歲登第者四十人。三年,榜出後,敕曰:「進士每歲四十人,其數過多,則乖精選。官途填委,要窒其源,宜改每年限放三十人,如不登其數,亦聽。」然鍇選擢雖多,頗得實才,抑豪華,擢孤進,至今稱之。尋轉吏部侍郎。其年九月,出為鄂州刺史、御史大夫、鄂岳觀察使,卒。



 釴子湜,鍇子湘,偕登進士第。湜,咸通十二年為禮部侍郎。湘自員外郎知制誥,正拜中書舍人。咸通年,改諫議大夫。坐宰相劉瞻親厚,貶高州司馬。乾符初,
 復為中書舍人。三年,遷禮部侍郎,選士得人。出為潞州大都督府長史、昭義節度、澤潞觀察等使,卒。



 馮宿,東陽人。丱歲隨父子華廬祖墓,有靈芝、白兔之祥。宿昆弟二人,皆幼有文學。宿登進士第,徐州節度張建封闢為掌書記。後建封卒,其子愔為軍士所立,李師古欲乘喪襲取。時王武俊且觀其釁,愔恐懼,計無所出。宿乃以檄書招師古,而說武俊曰:「張公與君為兄弟,欲同力驅兩河歸天子,眾所知也。今張公歿,幼子為亂兵所
 脅,內則誠款隔絕於朝廷,外則境土侵逼於強寇。孤危若此,公安得坐視哉!誠能奏天子,念先僕射之忠勛,舍其子之迫脅,使得束身自歸,則公於朝廷有靖亂之功,於張氏有繼絕之德矣!」武俊大悅,即以表聞。由是朝廷賜愔節鉞,仍贈建封司徒。



 宿以嘗從建封,不樂與其子處,乃從浙東觀察使賈全府闢。愔恨其去己,奏貶泉州司戶。徵為太常博士。王士真死,以其子承宗不順,不加謚。宿以為懷柔之義,不可遺其忠勞,乃加之美謚。轉虞
 部、都官二員外郎。



 元和十二年,從裴度東征,為彰義軍節度判官。淮西平,拜比部郎中。會韓愈論佛骨,時宰疑宿草疏,出為歙州刺史。入為刑部郎中。十五年,權判考功。宿以宰臣及三品已下官,故事內校考,別封以進;翰林學士,職居內署,事莫能知,請依前書上考;諫官御史亦請仍舊,並書中上考。



 長慶元年,以本官知制誥。二年,轉兵部郎中,依前充職。牛元翼以深州不從王庭湊,詔授襄州節度使。元翼未出,深州為庭湊所圍。二年,以宿檢
 校右庶子、兼御史中丞,賜紫金魚袋,往總留務。監軍使周進榮不遵詔命,宿以狀聞。元翼既至,宿歸朝,拜中書舍人,轉太常少卿。



 敬宗即位,宿常導引乘輿,出為華州刺史。以父名拜章乞罷,改左散騎常侍,兼集賢殿學士,充考制策官。



 太和二年,拜河南尹。時洛苑使姚文壽縱部下侵欺百姓,吏不敢捕。一日,遇大會,嘗所捕者傲睨於文壽之側,宿知而掩之,杖死。



 太和四年,入為工部侍郎。六年,遷刑部侍郎,修《格後敕》三十卷,遷兵部侍郎。九
 年,出為劍南東川節度使,檢校禮部尚書。



 開成元年十二月卒,廢朝,贈吏部尚書,謚曰懿。有文集四十卷。子圖、陶、韜,三人皆登進士,揚歷清顯。



 宿弟定,字介夫。儀貌壯偉,與宿俱有文學,而定過之。貞元中皆舉進士,時人比之漢朝二馮君。于頔牧姑蘇也,定寓焉,頔友於布衣間。後頔帥襄陽,定乘驢詣軍門;吏不時白,定不留而去。頔慚,笞軍吏,馳載錢五十萬,及境謝之。定飯逆旅,復書責以貴傲而返其遺,頔深以為恨。權德輿掌貢士,擢居上
 第,後於澗州佐薛蘋幕,得校書郎,尋為鄠縣尉,充集賢校理。定先時居父憂,因號毀得肺病,趨府或不及時,大學士疑其恃才簡怠,乃奪其職,俾為大理評事。登朝為大常博士,轉祠部員外郎。



 寶歷二年,出為郢州刺史。長壽縣尉馬洪沼告定強奪人妻,及將闕官職田祿粟入己費用,詔監察御史李顧行鞫之。獄具上聞,制曰:「馮定經使臣推問,無入己贓私,所告罰錢,又皆公用。然長吏之體,頗涉無儀,刑賞或乖,宴游不節。緣經恩赦,難更科書,
 猶持郡符,公議不可,宜停見任。」尋除國子司業、河南少尹。



 太和九年八月,為太常少卿。文宗每聽樂,鄙鄭、衛聲,詔奉常習開元中《霓裳羽衣舞》,以《雲韶樂》和之。舞曲成,定總樂工閱於庭,定立於其間。文宗以其端凝若植,問其姓氏。翰林學士李玨對曰:「此馮定也。」文宗喜,問曰:「豈非能為古章句者耶?」乃召升階。文宗自吟定《送客西江詩》,吟罷益喜,因錫禁中瑞錦,仍令大錄所著古體詩以獻。尋遷諫議大夫、知匭事。



 是歲,李訓事敗伏誅,衣冠橫
 罹其禍,中外危疑。及改元御殿,中尉仇士良請用神策仗衛在殿門;定抗疏論罷,人情危之。又請許左右史隨宰臣入延英記事,宰臣不樂。二年,改太子詹事。三年,宰臣鄭覃拜太子太師,欲於尚書省上事。定奏曰:「據《六典》,太師居詹事府,不合於都省禮上。」乃詔於本司上事,人推美之。四年,遷衛尉卿。是歲,上章請老,詔以左散騎常侍致仕。會昌六年,改工部尚書而卒。



 先長慶中,源寂使新羅國,見其國人傳寫諷念定所為《黑水碑》、《畫鶴記》。韋
 休符之使西番也,見其國人寫定《商山記》於屏障。其文名馳於戎夷如此。



 子袞、顓、軒、巖四人,皆進士登第。咸通中,歷任臺省。宿從弟審、寬。



 審父子鬱。審,貞元十二年登進士第,累闢使府。入為監察御史,累遷至兵部郎中。開成三年,遷諫議大夫。四年九月,出為桂州刺史、桂管觀察使。入為國子祭酒。國子監有《孔子碑》,睿宗篆額,加「大周」兩字,蓋武后時篆也。審請琢去偽號,復「大唐」字,從之。咸通中,卒於秘書監。



 審弟寬,子緘,皆進士擢第,知名於
 時。



 封敖,字碩夫,其先渤海蓚人。祖希奭。父諒,官卑。敖,元和十年登進士第,累闢諸侯府。太和中,入朝為右拾遺。會昌初,以員外郎知制誥,召入翰林為學士,拜中書舍人。



 敖構思敏速,語近而理勝,不務奇澀,武宗深重之。嘗草《賜陣傷邊將詔》,警句云:「傷居爾體,痛在朕躬。」帝覽而善之,賜之宮錦。李德裕在相位,定策破回鶻,誅劉稹。議兵之際,同列或有不可之言,唯德裕籌計指畫,竟立奇
 功。武宗賞之,封衛國公,守太尉。其制語有:「遏橫議於風波,定奇謀於掌握。逆稹盜兵,壺關晝鎖,造膝嘉話,開懷靜思,意皆我同,言不他惑。」制出,敖往慶之,德裕口誦此數句,撫敖曰:「陸生有言,所恨文不迨意。如卿此語,秉筆者不易措言。」座中解其所賜玉帶以遺敖,深禮重之。



 然敖不持士範,人重其才而輕其所為,德裕不能大用之。德裕罷相,敖亦罷內職。宣宗即位,遷禮部侍郎。大中二年,典貢部,多擢文士。轉吏部侍郎、渤海男、食邑七百戶。四
 年,出為興元尹、御史大夫、山南西道節度使,歷左散騎常侍。十一年,拜太常卿,出為淄青節度使,入為戶部尚書,卒。



 子彥卿、望卿,從子特卿,皆進士及第,咸通後,歷位清顯。



 史臣曰:韋公鯁亮守官,犯而得禮。蕭子恬於吏隱,抑亦名賢。蔚章操韻非高,而從容長者。鬱、朗襟概,鬱有世風。三高並秀於一時,二馮爭驅於千里,威以摛英掞藻,華國揚名。潤色之能,封無與讓,壽考垂慶,儒何負哉!



 贊曰:伏蒲進諫,染翰為文。獨孤、韋氏,志在匡君。馮、高諸子,綺繡繽紛。禁垣擅美,渤海凌雲。



\end{pinyinscope}