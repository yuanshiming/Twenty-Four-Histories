\article{卷二十上 本紀第二十上 昭宗}

\begin{pinyinscope}

 昭宗聖穆景文孝皇帝諱曄,懿宗第七子,母曰惠安太后王氏。以咸通八年二月二十二日生於東內。十三年四月,
 封壽王,名傑。乾符四年,授開府儀同三司、幽州大都督、幽州盧龍等軍節度、押奚契丹、管內觀察處置等使。帝於僖宗,母弟也,尤相親睦。自艱難播越,嘗隨侍左右,握兵中要,皆奇而愛之。文德元年二月,僖宗暴不
 豫。時初復宮闈,人心傾矚,遽聞被疾,軍民駭愕。及大漸之夕,而未知所立。群臣以吉王最賢,又在壽王之上,將立之,唯軍容楊復恭請以壽王監國。三月六日,宣遺詔立為皇太弟。八日,柩前即位,時年二十二。以司空韋昭度攝塚宰。己丑,見群臣,始聽政。帝攻書好文,尤重儒術,神氣雄俊,有會昌之遺風。以先朝威武不振,國命浸微而尊禮大臣,詳延道術,意在恢張舊業,號令天下。即位之始,中外稱之。四月戊辰朔。庚午,追謚聖母惠安太后曰恭獻。乙亥,河南尹彥張全義以兵襲李罕之於河陽,罕之出據澤州。魏博衙軍殺其帥樂禎於龍興寺,又擊樂從訓,敗之。從訓以殘眾保洹水,為羅宗弁陷其城而殺之。壬午,蔡賊孫儒陷揚州,楊行密潰圍而出,
 據宣州。孫儒自稱淮南節度,仍率其眾攻宣州。



 五月丁酉朔,制以宣武軍節度使、檢校侍中、沛郡王硃全忠為蔡州四面行營兵馬都統。自秦賢、石璠敗後,蔡賊漸弱,時溥方為全忠所攻,故移溥都統之命授全忠。壬寅,蔡賊將偽署荊襄節度使趙德諲遣使歸朝,願討賊自效,乃以德諲為蔡州四面行營副都統,德諲遂以荊襄之兵屬全忠。



 六月丁卯朔,以川賊王建大亂,劍南陳敬瑄告難,制以開府儀同三司、守司空、門下侍郎、
 同平章事、太清宮使、弘文館大學士、延資庫使、上柱國、扶陽郡開國公,食邑二千戶韋昭度檢校司徒、門下侍郎、
 平
 章事,兼成都尹,充劍南西川節度副
 大使、知節度事,兼兩川招撫制置等使。蔡州行營奏大破賊於龍陂,進軍以逼賊城。七月丙申
 朔,澤州
 刺史李罕之引太原之師攻河陽,為汴將丁會所敗,退還高平。



 九月乙未,汴將硃珍敗時溥之師於埇橋,遂陷宿州,自是溥嬰城不敢復出。汴將胡元琮急攻蔡州。十二月甲子朔,蔡州牙將申叢執秦宗權,撾折其足,乞降。詔中使宣諭,便以叢權知留後。比中使
 至,別將郭璠殺申叢,篡宗權,縶送汴州。蔡、申、光等州平。詔賜蔡州行營兵士錢二十五萬貫,令度支逐近支給。是月,葬僖宗於靖陵。



 龍紀元年春正月癸巳朔,上御武德殿受朝賀,宣制大赦,改元。中外文武臣僚進秩頒爵有差。以劍南西川節度、兩川招撫制置使韋昭度檢校司空,為東都留守;以翰林學士承旨、兵部侍郎、知制誥劉崇望本官同平章事;以刑部侍郎孫揆為京兆尹。



 二月癸亥朔。己丑,汴州
 行軍司馬李璠監送逆賊秦宗權並妻趙氏以獻,上御延喜門受俘,百僚稱賀,以之徇市,告廟社,斬於獨柳。趙氏笞死。初,自諸侯收長安,黃巢東出關,與宗權合。巢賊雖平,而宗權之兇徒大集,西至金、商、陜、虢,南極荊、襄,東過淮甸,北侵徐、兗、汴、鄭,幅員數十州。五六年間,民無耕織,千室之邑,不存一二,歲既兇荒,皆膾人而食,喪亂之酷,未之前聞。宗權既平,而硃全忠連兵十萬,吞噬河南,兗、鄆、青、徐之間,血戰不解,唐祚以至於亡。中書奏請以
 二月二十二日為嘉會節,從之。



 三月壬辰朔,以右僕射、門下侍郎、同平章事孔緯守司空、太清宮使、弘文館大學士、延資庫使、領諸道鹽鐵轉運等使,以右僕射、門下侍郎、集賢殿大學士杜讓能為左僕射、監修國史、判度支,以中書侍郎、戶部尚書、同平章事張浚為集賢殿大學士、判戶部事。四月壬戌朔,以宣武淮南等節度副大使、知節度事、管內營田觀察處置等使、開府儀同三司、檢校太傅、兼侍中、揚州大都督府長史、汴州刺史、充蔡
 州四面行營都統、上柱國、沛郡王、食邑四千戶硃全忠為檢校太尉、中書令,進封東平王,仍賜賞軍錢十萬貫。



 五月壬辰朔,漢州刺史王建陷成都府,遷陳敬瑄於雅州,建自稱西川兵馬留後。復用田令孜為監軍。



 六月辛酉朔,邢洺節度使孟方立卒,三軍推其弟洺州刺史遷為留後,太原李克用出軍攻之。杭州刺史錢鏐攻宣州,下之,擒劉浩,剖心以祭周寶。七月,詔於杭州置武勝軍,以鏐為本軍防禦觀察等使。十月己未朔,青州節度使
 王敬武卒。制以特進、太子少師、博陵郡開國侯、食邑一千戶崔安潛檢校太傅、兼侍中、青州刺史、平盧軍節度觀察、押新羅渤海兩蕃等使。青州三軍以敬武子師範權知兵馬事。



 十一月己丑朔,將有事於圓丘。改御名曰曄。辛亥,上宿齋於武德殿,宰相百僚朝服於位。時兩軍中尉楊復恭及兩樞密皆朝服侍上,太常博士錢珝、李綽等奏論之曰:「皇帝赴齋宮,內臣皆服朝服。臣檢國朝故事及近代禮令,並無內官朝服助祭之文。伏惟皇帝
 陛下承天御歷,聖祚中興,祗見宗祧,克陳大禮。皆稟高祖、太宗之成制,必循虞、夏、商、周之舊經,置冕服章,式遵彞憲。禮院先準大禮使牒稱得內侍省牒,要知內臣朝服品秩,禮院已準禮令報訖。今參詳近朝事例,若內官及諸衛將軍必須制冠服,即各依所兼正官,隨資品依令式服本官之服。事存傳聽,且可俯從,然亦不分明著在禮令。乞聖慈允臣所奏。」狀入,至晚不報。錢珝又進狀曰:「臣今日巳時進狀,論內官冠服制度,未奉聖旨。伏以
 陛下虔事郊禋,式遵彞範,凡關典禮,必守憲章。今陛下行先王之大禮,而內臣遂服先王之法服。來日朝獻大聖祖,臣贊導皇帝行事,若侍臣服章有違制度,是為非禮,上瀆祖宗,臣期不奉敕。臣謬當聖代,叨備禮官,獲正朝儀,死且不朽,脂膏泥滓,是所甘心。」狀入,降硃書御札曰:「卿等所論至當,事可從權。勿以小瑕,遂妨大禮。」於是內四臣遂以法服侍祠。甲寅,圓丘禮畢,御承天門,大赦。十二月戊午,宰臣杜讓能兼司空。



 大順元年春正月戊子朔,御武德殿受朝賀。宰臣百僚上徽號曰聖文睿德光武弘孝皇帝,禮畢,大赦,改元大順。



 二月丁巳,宰臣兼國子祭酒孔緯以孔子廟經兵火,有司釋奠無所,請內外文臣自觀察使、制使下及令佐,於本官料錢上緡抽十文,助修國學,從之。宣武節度使硃全忠進位守中書令,加食邑千戶,餘如故。太原都將安金俊攻圍邢州歷年,城中食盡,邢洺觀察使孟遷以城降,乃以孟遷之族歸太原。克用以大將安建為邢洺
 留後。



 三月丁亥朔,硃全忠上表:「關東籓鎮,請除用朝廷名德為節度觀察使。如籓臣固位不受代,臣請以兵誅之。如王徽、裴璩、孔晦、崔安潛等皆縉紳名族,踐歷素高,宜用為徐鄆青兗等道節度使。」從之。昭義節度使李克修卒,太原帥克用之弟也,三軍推克修弟克恭知留後事。四月丙辰朔,李克用遣大將安金俊率師攻雲州。赫連鐸求援於幽州,李匡威出兵援之,戰於蔚州,太原軍大敗,燕軍執安金俊,獻之於朝。李匡威、赫連鐸、硃全忠
 等上表:「請因沙陀敗亡,臣與河北三鎮及臣所鎮汴滑河陽之兵平定太原,願朝廷命重臣一人都總戎事。」昭宗以太原于艱難時立興復大功,心疑其事,下兩省、御史臺、尚書省四品已上官議。唯黨全忠者言其可伐,不可者十之七,宰臣杜讓能、劉崇望深以為不可。惟張浚議曰:「先朝再幸興元,實沙陀之罪。比慮河北諸侯與之膠固,無以滌除。今兩河大籓皆願誅討,不因其離貳而除之,是當斷失斷也。」孔緯曰:「浚言是也。」軍容楊復恭曰:「
 先朝蒙犯霜露,播越草莽,七八年間,寢不安席,雖賊臣搖蕩於外,亦由失制於中。陛下纘承,人心忻戴,不宜輕舉干戈,為國生事。望優詔報全忠,且以柔服為辭。」上然之。全忠密遣浚之親黨賂浚,浚恃全忠之援,論奏不已,天子黽勉從之。



 五月,制特進、中書侍郎、兵部尚書、同平章事、集賢殿大學士、上柱國、河間郡開國伯、食邑七百戶張浚為太原四面行營兵馬都統,京兆尹孫揆副之。以華州節度使韓建為北面行營招討都虞候、供軍等
 使;以宣武節度使硃全忠為太原東南面招討使;成德軍節度使王鎔為太原東面招討使;幽州節度使李匡威為太原北面招討使,雲州防禦使赫連鐸副之。丙午,潞州軍亂,殺其帥李克恭。監軍使薛繢本函克恭首獻之於朝,浚方起兵,朝廷稱賀。壬子,都招討使張浚、孫揆率諸策神軍三千赴行營,昭宗御安喜門臨送,誡誓之。



 六月乙卯,李克用大將權知邢洺兵馬留後安建上表,請以三州歸順,遣中使往勞之。制以德州刺史、權知滄
 州兵馬留後盧彥威檢校尚書右僕射,兼滄州刺史、御史大夫,充義昌軍節度、滄德觀察處置等使。彥威,光啟初逐其帥楊全玫,求旄節,朝廷以扈蹕都將曹誠為滄德節度使,誠雖不至任,而彥威之請不行。至是,王鎔、羅弘信因張浚用兵,為彥威論請,故有斯授。以京兆尹、行營兵馬副招討孫揆檢校兵部尚書,兼潞州大都督府長史,充昭義節度副大使、知節度事。張浚會諸軍於晉州,硃全忠選汴卒三千為張浚牙隊。



 秋七月乙酉朔,王
 師屯於陰地,太原大將康君立以兵拒戰。硃全忠遣大將葛從周率千騎入潞州,從周權充兵馬留後。硃全忠奏已差兵士守潞州,請節度使孫揆赴鎮。時中使韓歸範押揆旌節、官告送至行營。丙申,揆建節,率兵二千,自晉州赴鎮昭義。戊申,至長子縣山谷中。太原騎將李存孝伏兵執揆與韓歸範牙兵五百,俘送太原,餘兵悉為存孝所殺。太原將康君立率兵二萬攻潞州。



 九月甲申,幽州、雲州蕃、漢兵三萬攻雁門,太原將李存信、薛阿檀
 擊敗之。汴將葛從周棄上黨,康君立入據之,克用以君立為澤潞兵馬留後。



 十一月癸丑朔,太原將邢州刺史李存孝自恃擒孫揆功,合為昭義帥,怨克用授康君立。存孝自晉州率行營兵歸邢州,據城上表歸朝,仍致書與張浚、王鎔求援。克用遣大將李存信、薛阿檀拒王師於陰地,三戰三捷,由是河西鄜、夏、邠、岐之軍渡河西歸。韓建以諸軍保平陽,存信追之,建軍又敗,建退保絳州。張浚以汴卒、禁軍萬人在晉州,存信攻之三日,相與謀
 曰:「張浚宰相,俘之無益,天子禁兵,不宜加害。如得平陽,於我無利。」遂退舍五十里而軍。十二月壬午朔,張浚、韓建拔晉、絳遁去,李存信收晉、絳,大掠河中四郡。丙寅,制特進、中書侍郎、平章事、太原四面行營都統張浚可檢校兵部尚書,兼鄂州刺史、御史大夫,充鄂岳觀察使。以開府儀同三司、守司徒、門下侍郎、同平章事、上柱國、魯國公、食邑三千戶、充諸道鹽鐵轉運等使孔緯檢校司徒,兼江陵尹、荊南節度觀察處置使。庚午,新除鄂岳觀
 察使張浚責授連州刺史,新除荊南節度使孔緯責授均州刺史,並馳驛赴任。太原軍屯晉州,李克用遣中使韓歸範還朝,因上表訴冤,言:「被賊臣張浚依倚硃全忠離間功臣,致削奪臣官爵。」朝廷欲令釋憾,下群臣議其可否。左僕射韋昭度等議曰:



 賞功罰否,前聖之令猷;含垢匿瑕,百王之垂訓。是以雷解而羲文象德,網開而湯化歸仁,用彼懷柔,式存彞範。上自軒農之代,下臻文武之朝,罔不允洽寬弘,以流霈澤。況國家德祖守成之日,
 憲宗致理之時,車軌一同,桑麻萬里。燭龍外野,悉在梯航;火鼠窮郊,咸歸正朔。然猶王承宗擁兵鎮、冀,詔範希朝討之,仍歲無功,卒行赦宥。而又硃滔以幽州之眾,結田悅、李納、王武俊之強,遣馬燧等征之不克,旋又寬之。以累聖之典謀睿哲,大朝之紀律文明,非不欲厲彼風驅,快其電掃。然且考《春秋》之義,稽楚、鄭之文,或退而許平,或服而更舍,存於舊史,載彼新書。



 李克用代漠強宗,陰山貴胤,呼吸而風雲作氣,指麾而草樹成形。仰天指
 心,誓獻失訾之首;伏弢歐血,屢親都護之營。所謂勇多上人,自匪窮來歸我。及陛下聖考懿宗皇帝之朝,彭門失守,親驅銳卒,首建殊功。而先帝即位之初,渚宮大擾,復提義旅,克靜妖氛。其後封豕長蛇,薦食上國,繼以子朝之亂,皆因重耳之盟,保大朝之宗祧,垂中興於簡冊。蓋聖王之御天下也,有勛可書,有績可載,宥過不忘於十代,念功豈止於一時。天高聽卑,請事斯語。且四海之內,創磐猶殷,九貢之邦,綱條未理。昨者遽起邠、岐之眾,
 尋已退還;又徵燕、薊之師,倏聞內變。出於饟饋失職,資扉絕供,致此投戈,是乖借箸。蓋下計之未熟,非聖謀之不臧。儻宸斷重離,天機間出,錄茲成款,散彼師徒,虛其念舊之懷,待以如初之禮。臣等所議,實以在斯。



 抑又聞往者漢將趙充國欲因邊境衰弱,出兵擊之,於是魏相上書,畫陳利害,且曰:「恃國家之大,矜人物之眾,欲見威於敵者,謂之驕兵。兵驕者滅,非但人事,乃天道也。」又曰:「臣不知此兵何名者也。」兵出無名,事乃不成,漢宣納之,
 竟罷其伐。伏惟皇帝陛下鑒往古用師之難,採列聖遷善之美,恩加區宇,信及豚魚,則臣等不勝懇願。況今汴、魏猶艱,幽、定方困,縱遣之調發,豈能集事!虛行號令,徒召寇讎,將以剿人,非唯辱國。且黠斯舉勤王之眾,推效命之誠,未能虜騎獨攻,所望漢兵同力。令茲數鎮,奔命不遑,難致濟師,恐又生事。諭其漸當暑熱,非利戎旃,悉力頒沾,遣還蕃部。重盈陳五郡之卒,益謹關防;王珙振兩河之雄,更嚴旗鼓。然後獎其上表,哀以自陳,錄彼
 前勞,責之後效。徵神爵之往典,還日逐之故封。諭其已斥王恭,不使更疑晉帝,凡百臣子,實切乃誠。其克用在身官爵,並請卻還,仍依前編入屬籍。



 從之。以翰林學士承旨、兵部侍郎崔昭緯本官同平章事,御史中丞徐彥若為戶部侍郎、同平章事。尚書右僕射王徽卒,贈司空,謚曰貞。



 二年春正月壬子朔,李克用急攻邢州。李存考求援於王鎔,鎔出軍援之,屯於堯山。克用自太原至,擊敗之,進
 圍邢州。司徒、門下侍郎、平章事杜讓能進位太尉、太清宮使、弘文館大學士、延資庫使領諸道鹽鐵轉運等使。以中書侍郎、吏部尚書、平章事劉望為門下侍郎、監修國史、判度支事,工部侍郎、平章事崔昭緯判戶部事。



 二月辛巳,李克用復檢校太師、中書令、太原尹、北都留守、河東節度觀察處置等使。時張浚、韓建兵敗後,為太原將李存信等所追,至是方自含山逾王偓,出河清,達於河陽。屬河溢,無舟楫,建壞人盧舍,為木罌數百,方獲
 渡,人多覆溺,休其徒於司徒廟。是役也,朝廷倚硃全忠及三鎮兵。全忠方連兵徐鄆,乃求兵糧於鎮、魏,全忠終不至行營。鎮、魏倚太原為捍蔽,如破太原郡,恐危鎮、魏,王鎔、羅弘信亦不出師。唯邠、岐、華、鄜、夏烏合之眾會晉州。兵未交而孫揆擒,燕卒敗,所以河西、岐下之師望風潰散,而浚、建至敗。全忠以鎮、魏不助兵糧觀望,遣龐師古將兵討魏,陷十縣,羅弘信乞盟,乃退。棣州刺史張蟾為青州將王師範所敗。新授平盧節度使崔安潛自棣
 州歸朝,復授太子少師。



 三月辛亥朔,以青州權知兵馬留後王師範檢校兵部尚書,兼青州刺史、御史大夫,充平盧軍。三月辛亥朔,以青州權知兵馬留後王師節度觀察、押新羅渤海兩蕃等使。淮南節度孫儒為宣州觀察使楊行密所殺。初,行密揚州失守,據宣州,孫儒以兵攻圍三年。是春,淮南大饑,軍中疫癘死者十三四。是月,孫儒亦病,為帳下所執,降行密。行密乃並孫儒之眾,復據廣陵。



 六月,王鎔出軍援李存孝,克用大舉討鎮州。七月,太原軍出井陘,屯於常山鎮,大掠鎮、趙、
 深諸郡。幽州節度使李匡威自率步騎三萬援王鎔。



 八月,克用班師。



 九月丁未朔。乙卯,天子賜左軍中尉楊復恭幾杖,以大將軍致仕。復恭怒,稱病不受詔。十月丁丑朔。甲申,天威軍使李順節率禁兵討楊復恭,復恭假子玉山軍使楊守信以兵拒之,列陣於昌化里。昭宗登延喜樓,陳兵自衛以俟變。相持至晚,不戰而退。是夜,守信乃擁其眾衛復恭出京師,且戰且行,出通化門,由七盤路之商州,又令義兒張綰為後殿。永安都頭安權追及
 綰,擒之而還。



 十一月,硃全忠上表,請移時溥節鎮。是月,汴軍陷宿州,乃授溥太子太師。溥將劉知俊降汴軍。鎮州王鎔、幽州李匡威復謀攻定州以分其地,王處存求援於太原。十二月丙子朔,以光祿大夫、門下侍郎、右僕射、平章事、監修國史、判度支、上柱國、彭城縣開國男劉崇望檢校司空、同平章事,兼徐州刺史,充武寧軍節度、徐宿觀察制置使。時李順節恃恩恣橫,出入以兵仗自隨,兩軍中尉劉景宣、西門君遂懼其窺圖非望。丁亥,兩
 中尉傳詔召順節,順節以甲士三百自隨,至銀臺門,門司傳詔止從者。兩中尉在仗舍邀順節,坐次,令部將嗣光審斫順節,頭隨劍落。其部下知順節死,大噪出延喜門。是日,天威、捧日、登封三都亂,剽永寧里,至晚方定。戶部尚書鄭延昌為中書侍郎、平章事、判度支。



 景福元年春正月丙午朔,上御武德殿受朝賀,大赦,改元景福。鳳翔李茂貞、邠州王行瑜、華州韓建、同州王行約、秦州李茂莊等上表疏興元楊守亮納叛臣楊復恭,
 請同出本軍討伐,兼自備供軍糧料,不取給於度支,只請加茂貞山南招討使名。內臣皆不可其奏,昭宗亦以茂貞得山南之後有問鼎之志,詔久之不下。茂貞怒,與王行瑜不俟進止,發兵攻興元。累請招討之命,兼與宰相杜讓能、中尉西門君遂書,詞語詬詈,凌蔑王室,昭宗心不能容。二月丙子朔。庚寅,太原、易定之兵合勢攻鎮州,王鎔復告難於幽州,李匡威率步騎三萬赴之。時太原之眾軍於常山鎮,易定之眾軍堅固鎮,燕、趙之卒分
 拒之。



 三月,克用、處存斂軍而退。四月乙亥,左軍中尉西門君遂殺天威軍使賈德晟,時德晟與李順節俱掌天威軍,順節死,中尉惡德晟,誣奏殺之。是日,德晟部下千餘騎出奔鳳翔,自是岐軍益盛。



 五月甲辰,制以河南尹張全義檢校司徒、同平章事,兼孟州刺史,充河陽三城節度、孟懷澤觀察等使。七月,燕、趙之卒合勢援邢州,太原大將李存信率軍拒於堯山,王鎔大敗而還。



 十一月辛丑,鳳翔、邠寧之眾攻興元府,陷之。山南西道節度使
 楊守亮與前左軍中尉楊復恭、判官李巨川突圍而遁,將奔太原。李茂貞表其子繼密權知興元府事。十二月辛未朔,華州節度使韓建奏於乾元縣遇興元潰散兵士,擊敗之。其楊守亮、楊復恭並已處斬訖,皆傳首京師。



 二年春正月辛丑朔,制以權知劍南東川兵馬留後顧彥暉檢校尚書右僕射,兼梓州刺史、御史大夫,充劍南東川節度觀察等使。時王建連年攻彥暉,李茂貞欲與建爭東川,故表請彥暉正授旄鉞,示修好也。



 二月庚午
 朔,太原李克用以兵攻鎮州,師出井陘,王鎔懼,再求救於幽州。甲申,李匡威復來赴援,太原之軍還邢州。



 三月庚子,制以捧日都頭陳珮為廣州刺史、嶺南東道節度使,扈蹕都頭曹誠為黔州刺史、黔中節度使,耀德都頭李鋋為潤州刺史、鎮海軍節度使,宣威都頭孫惟晟為江陵尹、荊南節度使,並加特進、同平章事。各令赴鎮,並落軍權。時朝議以茂貞傲侮王命,武臣難制,欲用杜讓能及親王典禁兵,故罷五將之權,兼以平章事悅其心。
 太尉杜讓能冊拜,加食邑至六千戶。是月,幽州節度使李匡威弟匡籌據幽州,自稱留後,以符追行營兵,兵皆還幽州。匡威既無歸路,遣判官李貞抱入奏,請朝覲。王鎔感匡威援助之惠,乃築第於恆州,迎匡威處之。四月己巳,汴將王重師、牛存節陷徐州,節度使時溥舉家自燔而死。硃全忠遣將龐師古守徐州。



 六月丁酉朔。乙卯,幽州節度使李匡威謀害王鎔而奪其帥,恆州三軍攻匡威,殺之。戊午,制太尉、門下侍郎、平章事、晉國公杜讓
 能加食邑至九千戶。門下侍郎、吏部尚書、平間事崔昭緯進階光祿大夫,中書侍郎、平章事鄭延昌兼刑部尚書,並加食邑至千戶。以祠部郎中、知制誥陸扆為中書舍人,仍前翰林學士。幽州節度使李匡籌遣使檄王鎔,訊殺匡威之罪。二籓結怨,硃全忠遣判官韋震使幽州和解之。七月,李克用興兵攻鎮州,敗王鎔軍於平山。鎔懼,乞盟,請以兵糧助攻邢州,許之,克用遂旋軍襄國。癸未,制以鳳翔隴州節度使、檢校太尉、中書令、鳳翔尹、上
 柱國、岐王、食邑四千五百戶李茂貞為興元尹、山南西道節度等使。以中書侍郎、同平章事徐彥若檢校尚書左僕射、同平章事,兼鳳翔尹,充鳳翔隴州節度使。時茂貞恃兵求兼領山南節度,昭宗久之不行,茂貞表章不遜,深詆時政,上不能容,將加兵問罪,故以彥若代之。



 八月丙申朔,以嗣覃王為京西招討使,神策大將軍李金歲副之。



 九月丙寅朔,以武勝軍防禦使錢鏐為鎮海軍節度、浙江西道觀察處置等使,仍移鎮海軍額於杭州。乙
 亥,覃王率扈駕五十四軍進攻岐陽,屯於興平。李茂貞以兵逆戰,屯於盭厔。壬午,岐軍進迫興平,王師自潰。茂貞乘勝逼京師,進屯三橋。甲申,昭宗御安福門,斬觀軍容使西門君遂、內樞密使李周潼,遣中使賜茂貞詔,令收兵歸鎮。茂貞陳兵臨皋驛,數宰臣杜讓能之罪,請誅之。制貶太尉、平章事、晉國公杜讓能為雷州司戶。十月乙未,賜杜讓能自盡,其弟戶部侍郎弘徽坐讓能賜死。



 十一月,制以鳳翔節度使李茂貞守中書令,進封秦王,
 兼興元尹、山南西道節度使。邠州節度使王行瑜賜號「尚父」,賜鐵券。以門下侍郎、吏部尚書、平章事、監修國史崔昭緯兼尚書左僕射,充諸道鹽鐵轉運等使;以特進、行右僕射韋昭度為司空、門下侍郎、同平章事、弘文館大學士、太清宮使、延資庫使。中書侍郎、刑部尚書、平章事、判度支鄭延昌罷知政事,守尚書左僕射,以病求罷故也。以新除鳳翔節度使徐彥若復知政事。戶部侍郎、判戶部事王搏本官同平章事。



 乾寧元年春正月乙丑朔,上御武德殿受朝,宣制大赦,改元乾寧。鳳翔李茂貞來朝,大陳兵衛,獻妓女三十人,宴之內殿,數日還籓。時茂貞有山南梁、洋、興、鳳、岐、隴、秦、涇、原等十五餘郡,甲兵雄盛,凌弱王室,頗有問鼎之志。



 二月,汴人大敗兗、鄆之軍於東阿,瑄、瑾勢蹙,求援於太原,李克用出師援之。



 三月甲子朔,太原軍攻邢州,陷之,執其逆將李存孝,檻送太原,裂之。克用以大將馬師素權知邢洺團練事。



 五月,蔡賊孫儒部將劉建鋒攻陷潭
 州,自稱湖南節度使。以翰林學士、中書舍人陸扆為戶部侍郎、知制誥,充職。



 六月壬辰,李克用攻陷雲州,執大同防禦使赫連鐸,以其牙將薛志勤守雲中。十月庚寅,以中書侍郎、平章事王搏為湖南節度使。以翰林學士承旨、禮部尚書、知制誥李磎為戶部侍郎、同平章事。宣制之日,水部郎中、知制誥劉崇魯出班而泣,言磎奸邪,黨附內官,不可居輔弼之地,由是制命不行。戊申,制御史中丞崔胤為兵部侍郎、同平章事。是月,李克用以太
 原之眾進攻幽州。十二月,幽州節度使李匡籌潰圍而遁。克用陷幽州,以李匡威故將劉仁恭為幽州兵為留後。是月,李匡籌南奔赴關,至景城,為滄州節度使盧彥威所殺。



 二年春正月己未朔,河中節度使、檢校太師、中書令、河中尹、上柱國、瑯邪郡王王重盈卒,三軍立重榮子行軍司馬珂知留後事。



 二月己丑朔,王重盈子陜州節度使珙、絳州刺史瑤舉兵討王珂,兼上章訴珂冒姓,非重榮
 子。珂、珙爭為蒲帥,上遣中使慰勞。



 三月,制以中書侍郎、同平章事崔胤檢校尚書左僕射、同平章事、河中尹,充河中節度、晉絳慈隰觀察處置等使。浙東節度使董昌僭號稱羅平國,年稱大聖,用婺州刺史蔣瑰為宰相,仍偽署官員。鎮海軍節度使錢鏐請以本軍進討,從之。以翰林學士承旨、兵部侍郎、知制誥趙光逢為尚書左丞,依前充職。太原李克用上章言王重榮有功於國,其子珂宜承襲,請賜節鉞。邠州王行瑜、鳳翔李茂貞、華州韓
 建各上章,言珂螟蛉,不宜纘襲,請以王珂為陜州,王珙為河中。天子以先允克用之奏,久之不下。



 五月丁巳朔。甲子,李茂貞、王行瑜、韓建等各率精甲數千人入覲,京師大恐,人皆亡竄,吏不能止。昭宗御安福門以俟之,三帥既至,拜舞樓下,昭宗臨軒自諭之曰:「卿等籓侯,宜存臣節,稱兵入朝,不由奏請,意在何也?」茂貞、行瑜汗流洽背,不能對,唯韓建陳敘入覲之由。上並召升樓,賜之卮酒,宴之於同文殿。茂貞、行瑜極言南北司相傾,深蠹時
 政,請誅其太甚者。乃貶宰相韋昭度、李磎,尋殺之於都亭驛,殺內官數人而去。王行瑜留弟行約,茂貞留假子閻圭,各以兵二千人宿衛。時三帥同謀廢昭宗立吉王,聞太原起軍乃止,留兵宿衛而還。壬申,以責授均州司戶孔緯、繡州司戶張浚並為太子賓客。以翰林學士、戶部侍郎、知制誥陸扆為兵部侍郎,充職。



 六月丁亥朔,以京兆尹、嗣薛王知柔兼戶部尚書、判度支,兼諸道鹽鐵轉運等使。壬辰,以太子賓客孔緯為吏部尚書,尋復開
 府儀同三司、守司空、門下侍郎、同平章事、弘文館大學士、太清宮延資庫使、上柱國、魯郡開國公,食邑四千戶、食實封二百戶,仍號「持危啟運保乂功臣。」時緯在華州,尋屬太原軍至而止。以太子賓客張浚復光祿大夫、行兵部尚書、上柱國、河間郡開國侯、食邑二千戶。浚在長水,亦不至京師。復以王搏為中書侍郎、平章事。七月丙辰朔,李克用舉軍渡河,以討王行瑜、李茂貞、韓建等稱兵詣闕之罪。庚申,同州節度使王行實棄郡入京師,謂
 兩軍中尉駱全瓘、劉景宣曰:「沙陀十萬至矣!請奉車駕幸邠州,且有城守。」時景宣附鳳翔,癸亥夜,閻圭與劉景宣子繼晟、同州王行實縱火剽東市,請上出幸。上聞亂,登承天門,遣諸王率禁兵御之。捧日都頭李筠率本軍侍衛樓上。閻圭以鳳翔之卒攻李筠,矢及御座之樓扉。上懼,下樓與親王、公主、內人數百幸永興坊李筠營。扈蹕都頭李君實以兵繼至,乃與筠兩都兵士侍衛出啟夏門,憩於華嚴寺,以候內人繼至。其日晚,幸莎城鎮。京師
 士庶從幸者數十萬,比至南山谷口,暍死者三之一。至暮,為盜寇掠,慟哭之聲,殷動山谷。權令京兆尹知柔中書事及隨駕置頓使。信宿,宰相徐彥若、王摶、崔胤三人至,乃移石門鎮之佛宮。仍令知樞密劉光裕、薛王知柔歸京師制置,合禁軍以備宮禁。丙寅,李克用遣牙將閻諤奉表奔問,奏屯軍河中,候進止發赴邠州。丁卯,上遣內官張承業傳詔克用軍,便令監太原行營兵馬,發赴新平。又令內官卻廷立傳詔涇州,令張鐇起涇原之師
 會克用軍。上在南山半月餘,克用仍在河中,未至渭北。上懼鳳翔兵士劫遷,乃令延王將御服、鞍馬、玉器等至河中,宣諭曰:「朕以景宣、全瓘、行實、繼鵬為表里之奸謀,縱干戈於雙闕,煙塵倏忽,劫殺縱橫。朕偶脫鋒鋩,遂移輦輅,所為巡幸,止在近郊。蓋知卿統領雄師,駐臨蒲阪,累飛書詔,繼遣使人。期卿以社稷為憂,君親在念,必思響應,速議龔行。豈謂將涉兩旬,未有來表,憂虞是切,寢食不遑。豈忠義不切疚懷,而道途或有阻滯?今則專令
 親信,懇托勛賢,故遣延王戒丕、丹王允與供奉官王魯紆等宣示。卿宜便董貔貅,徑臨邠鳳,蕩平妖穴,以拯阽危,是所望也。」八月乙酉朔,延王至河中,克用已發前鋒至渭北,又令史儼率五百騎赴行在侍衛。己丑,克用自至渭橋砦。癸巳,行梨園殺邠軍數千,獲其大將王令陶以獻。又詔鄜州節度使李思孝本軍進討。丁酉,制以河東節度使、開府儀同三司、守太師、中書令,兼太原尹、北都留守、上柱國、隴西郡王李克用為邠寧四面行營都
 招討使。夏州節度使李思諫充邠寧東北面招討使,涇原節度使張鐇充邠寧西面招討使,河中節度使王珂充行營供軍糧料使。李茂貞聞之懼,斬閻圭、武禿子,傳首行在,上章請罪。辛丑,制削奪王行瑜在身官爵。改授李克用邠寧四面行營都統。其大將蓋寓李存信閻鍔、判官王讓李襲吉等,並降詔錫賚。又以河中都監袁季貞充邠寧四面行營兵馬都監押。壬寅,李克用遣子存貞奉表行在,請車駕還宮。答詔曰:「昨延王回,言卿憂時
 體國,執禮輸忠,接遇之間,周旋盡節。備知肺腑,識我恩榮,靜惟尊主之心,果契知臣之分。朕欲取今月十二四日卻復都城,冀寧兆庶,倚我勛德,有若長城,速伸翦蕩之謀,以慰黔黎之望。」癸卯,又令延王傳詔,令克用發騎軍三千赴三橋屯駐,以備回鑾。辛亥,車駕還宮。壬子,司空、門下侍郎、平章事、臨修國史、諸道鹽鐵轉運使崔昭緯罷知政事,為太子賓客。以河中兵馬留後王珂檢校司空,兼河中尹、御史大夫,充護國軍節度、河中晉絳慈
 隰觀察等使;以幽州兵馬留後劉仁恭檢校司空,兼幽州大都督府長史,充幽州盧龍軍節度、押奚契丹等使;以故左軍中尉楊復恭開府、魏國公:並從克用奏請也。



 九月甲寅朔。丙辰,制光祿大夫、守尚書左僕射、門下侍郎、同平章事、監修國史、上柱國、東莞郡公徐彥若為司空、門下侍郎、同平章事、太清宮修奉太廟等使、弘文館大學士、延資庫使,充諸道鹽鐵轉運等使。正議大夫、中書侍郎、同平章事王摶為金紫光祿大夫、戶部尚書門
 下侍郎、監修國史、判度支;正議大夫、中書侍郎、同平章事崔胤為金紫光祿大夫、戶部兼禮部尚書、集賢殿大學士、判戶部事。並賜號「扶危匡國致理功臣。」癸亥,司空、門下侍郎、平章事、太清宮修奉太廟等使、弘文館大學士、延資庫使、上柱國、魯郡開國公孔緯卒,贈太尉。十月甲申朔,王師破賊梨園砦,俘斬萬計,行瑜由是嬰城自固。丁亥,制赦系囚,其節文曰:「其有任崇柱石,位重臺衡,或委以軍權,或參諸宥密。竟因連謗,終至禍名,鬱我好生,嗟乎
 強死。應大順已來,有非罪而加削奪者,並復官資。其杜讓能、西門君遂、李周潼已下,並與昭雪,還其爵秩。韋昭度頃處臺司,每伸相業,王行瑜求尚書令,獨能抑之,致於沉冤,諒由此事。李磎文章宏贍,迥出輩流,竟以朋黨之間,擠於死地,凡在有識,孰不咨嗟。宜並與昭洗,仍復官爵。」又敕:「太子賓客崔昭緯責授梧州司馬,水部郎中、知制誥劉崇魯貶崖州司戶。又詔邠州行營都統曰:「邠州節度副使崔鋋,破賊之時,勿令漏網。鋋與昭緯去年
 朋黨,交結行瑜,構合禍胎,原由此賊。付四面行營知委。」是月,四面行營大集邠州。



 十一月癸未朔。壬寅,王行瑜與其妻子部曲五百餘人潰圍出奔,至慶州,行瑜為部下所殺,並其家二百口,並詣行營乞降,李克用遣牙將閻鍔獻於京師。十二月甲申朔,昭宗御延喜門受俘馘,百僚樓前稱賀。制以李克用守太師、中書令,進封晉王,食邑九千戶,改賜「忠貞平難功臣。」是月,克用班師太原。制:皇第三子祤封棣王,第五子禊封虔王,第六子禋封
 沂王,第七子禕封遂王。三年春正月癸丑朔,制以特進、戶部尚書、兼京兆尹、嗣薛王知柔檢校司徒,兼廣州刺史、御史大夫,充清海軍節度、嶺南東道觀察處置等使。以尚書右丞崔澤為鳳州刺史。魏博羅弘信擊敗太原軍於莘縣。初,兗鄆示援於太原,克用令蕃將史完府、何懷寶等千騎赴之。至是又令大將李存信屯於莘縣,魏人常假其道,存信戢軍不謹,或侵撓魏民。弘信怒,伏兵擊之,其軍宵潰。自是弘
 信南結於梁,與太原絕,兗鄆已至俱陷。



 二月壬子朔,制以通王滋為開府儀同三司,判侍衛諸道軍事。以銀青光祿大夫、戶部尚書、嘉興縣子、食邑五百戶陸扆為兵部尚書。



 三月壬子朔,以考功員外郎、集賢殿學士杜德祥為工部郎中、知制誥。四月壬午朔,湖南軍亂,殺其帥劉建鋒,三軍立其部將權知邵州刺史馬殷為兵馬留後。鎮海軍節度使錢鏐攻越州,下之,斬董昌,平浙東。制加錢鏐檢校太尉、中書令。



 五月辛巳,責授梧州司馬崔
 昭緯賜自盡。制金紫光祿大夫、戶部尚書、門下侍郎、平章事、監修國史、上柱國、太原郡開國公王摶為檢校尚書左僕射、同平章事,兼越州刺史,充鎮東軍節度、浙江東道觀察處置等使。



 六月庚戌,李克用率沙陀、並、汾之眾五萬攻魏州,及其郛,大掠於其六郡,陷成安、洹水、臨漳十餘邑,報莘之怨也。鳳翔李茂貞怨國家有硃玫之討,絕朝貢,謀將犯闕,天子命覃王治兵以俟變。是月,茂貞上章,請以兵師入覲。上令通王、覃王、延王分統安聖、
 捧宸、保寧、宣化等四軍,以衛近畿。丙寅,鳳翔軍犯京畿,覃王拒之於婁館,接戰不利。



 秋七月庚辰朔。壬辰,岐軍逼京師,諸王率禁兵奉車駕將幸太原。癸巳,次渭北,華州韓建遣子充奉表起居,請駐蹕華州,乃授建京畿都指揮、安撫制置、催促諸道綱運等使。詔謂建曰:「啟途之行,已在河東,今且幸鄜畤。」甲午,次富平。韓建來朝,泣奏曰:「籓臣倔強,非止茂貞。雖太原勤王,無宜巡幸。臣之鎮守,控扼關畿,兵力雖微,足以自固。陛下若輕舍近畿,遠
 巡極塞,去園陵宗廟,寧不痛心,失魏闕金湯,又非良算。若輿駕渡河,必難再復,謀茍不臧,悔之寧及。願陛下且駐三峰,以圖恢復。」上亦泣下曰:「朕難奈茂貞,忿不思難。卿言是也。」乙未,次下邽丙申,駐蹕華州,以衙城為行宮。時岐軍犯京師,宮室廛閭,鞠為灰燼,自中和已來葺構之功,掃地盡矣。乙巳,制以金紫光祿大夫、中書侍郎,兼禮部尚書、同平章事、集賢殿大學士、判戶部事、上柱國、博陵縣開國伯崔胤檢校尚書左僕射,兼廣州刺史、御
 史大夫,充清海軍節度、嶺南東道觀察處置等使。丙午,制以翰林學士承旨、尚書左丞、知制誥、嘉興縣開國子、食邑五百戶陸扆為戶部侍郎、同平章事。八月己酉朔。甲寅,新除鎮東軍節度使錢鏐權領浙江東道軍州事。戊午,制以戶部侍郎、平章事陸扆為中書侍郎,兼判戶部事。



 九月己卯朔,汴州硃全忠、河南尹張全義與關東諸侯俱上表,言秦中有災,請車駕遷都洛陽。全忠、全義言臣已表率諸籓,繕治洛陽宮室。優詔答之。乙未,制新
 除清海軍節度使崔胤復知政事。胤之出鎮,硃全忠再表請論奏,言胤不宜去相位,故有是命。丁酉,制中書侍郎、集賢殿大學士、判戶部事陸扆責授硤州刺史,崔胤怒扆代己,誣奏扆黨庇茂貞故也。丙午,制以鎮國軍節度使韓建檢校太尉,兼中書令,充修復宮闕、京畿制置、催促諸道綱運等使。以京兆尹孫偓為兵部侍郎、同平章事。十月戊申朔,以中書舍人、權知禮部貢舉薛昭緯為禮部侍郎。壬子,制以兵部侍郎、平章事孫偓為中書
 侍郎,充鳳翔行營招討使。甲寅,偓於驛舍會諸將,以議進軍。戊午,李茂貞上表章請罪,願改事君之禮,繼修職貢,仍獻錢十五萬,助修京闕。韓建左右之,師遂不行。



 十一月丁丑朔,以韓建兼領京兆尹、京城把截使。十二月丁未,李克用縱兵俘剽魏博諸郡邑。以前翰林學士承旨、尚書左丞、知制誥趙光遠為御史中丞。太常禮院奏權立行廟,以備告饗,從之。



 四年春正月丁丑朔,車駕在華州行宮,受群臣朝賀。癸
 未,汴將龐師古陷鄆州,節度使硃瑄與妻榮氏潰圍,瑄至中都,為野人所殺,榮氏俘於汴軍。硃全忠署龐師古為鄆州兵馬留後。宰相孫偓罷知政事,守兵部尚書。



 二月丙午朔。戊申,汴將葛從周攻兗州,陷之,節度使硃瑾奔楊行密,其將康懷貞降從周,硃全忠署從周為兗州兵馬留後。自是鄆、齊、曹、棣、兗、沂、密、徐、宿、陳、許、鄭、滑、濮等州皆沒於全忠,唯王師範守青州,亦納款於汴。己未,制朝議大夫、守右散騎常侍、上柱國、滎陽縣男鄭綮為禮
 部侍郎、同平章事。癸丑,責授硤州刺史陸扆為工部尚書。甲寅,華州防城將花重武告睦王已下八王欲謀殺韓建,移車駕幸河中。帝聞之駭然,召韓建諭之,建辭疾不敢行。帝即令通王已下詣建治所自陳。建奏曰:「今日未時,睦王、濟王、韶王、通王、彭王、韓王、儀王、陳王等八人到臣治所,不測事由。臣酌量事體,不合與諸王相見,兼恐久在臣所,於事非宜。況睦王等與臣中外事殊,尊卑禮隔,至於事柄,未有相侵,忽然及門,意不可測。」又引晉
 室八王撓亂天下事,「請依舊制,令諸王在十六宅,不合典兵。其殿後捧日、扈蹕等軍人,皆坊市無賴之徒,不堪侍衛,伏乞放散,以寧眾心。」昭宗不得已,皆從之。是日,囚八王於別第,殿後侍衛四軍二萬餘人皆放散,殺捧日都頭李筠於大雲橋下,自是天子之衛士盡矣。丙辰,韓建上表,請封拜皇太子、親王,以為維城之計。己未,制德王裕宜冊為皇太子。辛酉,制第八男秘可封景王,第九男祚可封輝王,第十男祺可封祁王,第十一男禛可封
 雅王,第十二男祥可封瓊王。



 三月丙子朔。戊寅,制韓建進封昌黎郡王,改賜「資忠靖國功臣」。以光祿大夫、兵部尚書、上柱國、河間郡開國侯、食邑二千戶張浚為尚書左僕射,依前充租庸使。四月丙午朔,就加福建節度使王潮檢校尚書右僕射。韓建獻封事十條,其三,太子、諸王請置師傅教導。乃以太子賓客王牘為諸王侍讀。宰相鄭綮以病乞骸,乃罷知政事。



 五月乙亥朔,以國子博士硃樸為右諫議大夫、同平章事。七月甲戌,帝與學士、
 親王登齊雲樓,西望長安,令樂工唱禦制《菩薩蠻》詞,奏畢,皆泣下沾襟,覃王已下並有屬和。



 八月甲辰朔,以工部尚書陸扆為兵部尚書。韓建與邠、岐三鎮素有無君之跡,及李克用誅行瑜,心常切齒。去歲車駕將幸河東,乃令延王戒丕使太原,見克用,陳省方之意。是月,延王自太原還。韓建奏曰:「自陛下即位已來,與近輔交惡,皆因諸王典兵,兇徒樂禍,遂致輿駕不安。比者臣奏罷兵權,實慮有不測之變。今聞延王、覃王尚苞陰計,願陛下
 宸斷不疑,制於未亂,即社稷之福也。」上曰:「豈至是耶!」居數日,以上無報,乃與知樞密劉季述矯制發兵,圍十六宅。諸王懼,披發沿垣而呼曰:「官家救兒命!或登屋沿樹。是日,通王、覃王已下十一王並其侍者,皆為建兵所擁,至石堤谷,無長少皆殺之,而建以謀逆聞。尋殺太子詹事馬道殷、將作監許巖士,貶平章事硃樸,皆上所寵暱者。



 九月癸酉朔,以御史中丞狄歸昌為尚書右丞。以刑部侍郎楊涉為吏部侍郎。制以鎮海軍節度使錢鏐為
 鎮海軍節度、漸江東西道觀察處置等使、杭州越州刺史、上柱國、吳王。



 冬十月癸卯朔,以華州節度使韓建兼同州刺史、匡國軍節度使。硃全忠遣其將權徐州兵馬留後龐師古、兗州留後葛從周率兗、鄆、曹、濮、徐、宿、滑等兵士七萬渡淮討楊行密。制以太中大夫、前御史中丞裴贄為禮部尚書、知貢舉。幽州節度使劉仁恭大敗沙陀於安塞,李克用單騎僅免。



 十一月壬申朔。癸酉,淮南大將硃瑾潛出舟師襲汴軍於清口,龐師古舉軍皆沒,
 師古被執。進葛從周自霍丘渡淮,至濠州,聞師古敗,乃退軍,信宿至渒河,方渡而硃瑾至。是日殺傷溺死殆盡,還者不滿千人,唯牛存節一軍先渡獲免。比至潁州,大雪寒凍,死者十五六。自古喪師之甚,無如此也。由是行密據有江、淮之間。以檢校司空、權知兗州兵馬事葛從周為兗州刺史,充泰寧軍節度使;以潁州刺史王敬蕘檢校尚書左僕射、兼徐州刺史,充武寧軍節度使:從全忠奏也。



 光化元年春正月辛未朔,車駕在華州。以兵部侍郎崔遠為戶部侍郎、同平章事。諸道貢修宮闕錢,命京兆尹韓建入京城計度。硃全忠遣判官韋震奏事,求兼領鄆州。時全忠軍敗之後,欲自大其權,以扼鄰籓之變。幽州節度使劉仁恭恃安塞之捷,欲吞噬河朔,是月遣其子守文將兵襲滄州,節度使盧彥威棄城而循,守文遂據之,自稱留後。四月庚子,制淑妃何氏宜冊為皇后。上幸陟屺寺,宴從官於韓建所獻御莊。



 五月己巳朔,以立後
 大赦。汴將葛從周率眾攻李克用邢、洺、磁等州,陷之。全忠署從周為三州兵馬留後。



 六月己亥,帝幸西溪觀競渡。天下籓牧、文武百僚上表,請車駕還京。七月,汴將氏叔琮陷趙匡凝之隨、唐、鄧等州。敕升華州為興德府,剌史為尹,左右司馬為少尹,鄭縣為次赤,官員資望一同五府。封華嶽廟為佑順侯。



 八月戊戌朔。己未,車駕自華還京師。甲子,御端門,大赦,改元光化。



 九月戊辰朔,以御史中丞狄歸昌為尚書左丞。制以鎮國、匡國等軍節度
 使韓建守太傅、中書令、興德尹,封潁川郡王,賜鐵券,並御寫「忠貞」以遺之。建累上表辭王爵,乃改封許國公。魏博節度使羅弘信進封臨清郡王。是月,弘信卒,贈太師,謚曰莊肅。衙軍立其子副大使紹威知兵馬事,尋賜之節鉞。十月丁酉朔,河南尹張全義就加侍中。汴將硃友恭自江西行營還,過安州,殺刺史武渝,遣部將守之。汴將張存敬以兵襲蔡州,刺史崔洪納款,請以弟賢質於汴,許之。十二月丙寅,李克用將潞州節度使薛志勤死,
 澤州刺史李罕之乘其無帥,襲潞取之,遣其子顥乞降於汴,全忠表罕之為節度使。



 二年春正月乙未朔。丁未,以兵部尚書陸扆為兵部侍郎、同平章事。



 二月,蔡州刺史崔洪為衙兵所迫,同竄淮南。時洪以弟賢質於汴,汴人遣賢還蔡,徵兵三千出征。蔡兵亂,殺賢,遂擁洪度淮。硃全忠令其子友裕守蔡州。幽州節度使劉仁恭驅燕軍十萬,將兼趙、魏。是月陷貝州,人無少長皆屠之,投尸清水,為之不流。遂進攻魏州。
 羅紹威求救於汴。



 三月,硃全忠遣大將張存敬率師援之,屯於內黃。葛從周自邢、洺率勁騎八百入魏州。燕將劉守文、單可及聞汴軍在內黃,引軍往擊之。存敬設伏內黃東,大敗燕軍,俘斬三萬,生擒單可及。劉守文以餘眾還魏州,為存敬、從周所乘,燕軍復敗,仁恭父子僅免。汴、魏合兵躡之,趙人復邀之東境,自魏至滄五百里間,殭尸相枕。是春,有白氣竟天如練,自西南徹東北,而旋有燕卒之敗。四月,汴將氏叔琮由上黨進軍攻太原,出
 石會,為沙陀擒其前鋒將陳章,叔琮乃退去。



 六月,制以昭義節度使、檢校太尉、兼太師、侍中、潞州大都督府長史、隴西郡開國公、食邑三千戶李罕之為孟州刺史,充河陽三城節度、孟懷觀察等使;以檢校司徒、孟州刺史、河陽節度使丁會為澤、潞等節度使:從全忠奏也。丁丑,李罕之至懷州,卒於傳舍。陜州軍亂,殺其帥王珙,立都將軍李璠為留後。丁亥,制以前太常卿劉崇望為吏部尚書,兵部侍郎裴樞為吏部侍郎,戶部侍郎薛昭緯為
 兵部侍郎。七月,青州守海州將牛從毅擁郡人投淮南,行密遂有海州。



 十一月,陜州衙將硃簡殺李璠,自稱留後,降汴,全忠表簡為帥守。



 三年春正月庚子朔,以禮部尚書裴贄為刑部尚書。癸卯,硃全忠奏:「本貫宋州碭山縣,蒙恩升為輝州,其地卑濕,難葺廬舍,請移輝州治所於單父縣。」從之,仍賜號為崇德軍。四月戊午,汴、魏合軍攻滄州,以報入郛之役,葛從周連陷滄德郡邑,王鎔遣使和解於全忠,令劉仁恭
 修好,汴、魏班師。辛未,皇后、太子謁九廟。



 六月丁巳,硃全忠表陜州兵馬留後硃簡鄉里同宗,改名友謙,乞真授節鉞。從之。戊辰,特進、司空、門下侍郎、平章事、監修國史王摶貶崖州司戶,尋賜死於藍田驛,樞密使宋道弼、景務修並死。為崔胤所誣,言三人中外相結也。七月丁亥朔,兵部尚書劉崇望卒,贈司空。甲午,兵部郎中薛正表為右諫議大夫。以許州刺史硃友恭檢校司徒,為潁州刺史;以左武衛將軍趙霖檢左僕射,為許州刺史;
 宣武押衙劉知俊檢校右僕射,為鄭州刺史:從全忠奏也。戊申,制以武貞軍節度、澧朗敘等州觀察處置等使、開府儀同三司、檢校司徒、同平章事、朗州刺史、上柱國、馮翊郡開國侯、食邑一千五百戶雷滿檢校太保,封馮翊郡王,餘如故。以武泰軍節度、黔中觀察處置等使、光祿大夫、檢校尚書左僕射、黔州刺史、御史大夫、上柱國趙崇封天水縣開國子,食邑五百戶。庚戌,制昭義節度留後、光祿大夫、檢校司空、上柱國孟遷為檢校司徒,兼潞
 州大都府長史,充昭義節度副大使、知節度事、潞磁邢洺等州觀察處置使,仍封平昌縣男,食邑三百戶,從李克用奏也。以金紫光祿大夫、守兵部尚書、上柱國、樂安郡開國公、食邑一千五百戶孫儲守兵部尚書,兼京兆尹。乙卯,制忠烈衛聖鎮國功臣、劍南西川節度副大使、知節度事、管內營田觀察處置統押近界諸蠻兼西山八國雲南安撫制置等使、開府儀同三司、檢校太尉、中書令、成都尹、上柱國、瑯邪郡王、食邑三千戶、實封一百
 戶王建可兼劍南東川、武信軍兩道都指揮制置等使,加食邑一千戶,餘如故。時建攻下梓州顧彥暉,兼有東川洋、果、閬等州故也。又以忠義軍節度、山南東道管內觀察處置三司水陸發運等使、開府儀同三司、檢校太尉、中書令、兼襄州刺史、上柱國、南平王、食邑三千戶趙匡凝可檢校太師、兼中書令,加實封一百戶。



 八月丙辰朔,硃全忠奏:「先割汝州隸許州,請卻還東都。河陽先管澤州,今緣蕃戎占據,得失不常,請權割河南府王屋、清
 河、鞏三縣隸河陽。」從之。癸亥,制忠貞平難功臣、河東節度、管內觀察處置等使、開府儀同三司、守太師、兼中書令、北都留守、太原尹、上柱國、晉王、食邑九千戶、食實封七百戶李克用加實封一百戶。丁卯,以朝請大夫、虞部郎中、知制誥、上柱國、賜紫金魚袋顏蕘為中書舍人。己巳,制前歸義軍節度副使、權知兵馬留後、銀青光祿大夫、檢校國子祭酒、監察御史、上柱國張承奉為檢校左散騎常侍,兼沙州刺史、御史大夫,充歸義節度、瓜沙伊
 西等州觀察處置押蕃落等使。庚辰,太原大將李嗣昭攻洺州、下之,執汴將硃紹宗。汴將葛從周率師赴之,嗣昭棄城而去。從周邀之於青山口,晉軍大敗,從周乘勝攻鎮州。壬午,制荊南節度、忠萬歸夔涪峽等州觀察處置水陸催運等使、開府儀同三司、檢校太尉、兼中書令、江陵尹、上柱國、上谷郡王、食邑三千戶成汭可檢校太師、中書令,餘如故。甲申,制扶危匡國致理功臣、特進、行尚書左僕射、兼門下侍郎、同平章事、監修國史、判度支、
 上柱國、清河郡開國公、食邑二千戶崔胤可開府儀同三司,進封魏國公,加食邑一千戶,餘如故。



 九月丙戌朔,硃全忠引三鎮之師攻鎮州,王鎔懼,遣判官周式、副大使王昭祚、主事梁公儒子弟為質於汴,出犒師絹十五萬匹求盟,許之。張存敬遂自深、冀進軍,攻瀛、莫,下郡邑二十,阻雨泥濘,不及幽州。遂西行陷祁州,大敗中山將王處直軍於沙河北,進屯懷德驛。遂攻定州,節度使王郜奔太原,衙將王處直斬孔目官梁汶,出縑二十萬乞
 盟,許之。全忠遂署王處直為義武軍留後。乙巳,制扶危匡國致理功臣、開府儀同三司、守太保、兼門下侍郎、平章事,充太清宮使、修奉太廟使、弘文館大學士、延資庫使、諸道鹽鐵轉運等使、上柱國、齊國公、食邑五千戶、食實封一百戶徐彥若可檢校太尉、同平章事,充清海軍節度、嶺南東道管內觀察處置供軍糧料等使。丙午,制光祿大夫、中書侍郎、兼吏部尚書、同平章事、充集賢殿大學士、判戶部事、博陵郡開國公、食邑二千戶崔遠罷
 知政事,守本官。戊申,制左僕射、門下侍郎、平章事、監修國史、判度支崔胤充太清宮使、修奉太廟使、弘文館大學士、延資庫使,依前判度支,兼充諸道鹽鐵運等使。光祿大夫、中書侍郎,兼戶部尚書、同平章事、上柱國、吳郡開國公、食邑一千五百戶陸扆為門下侍郎、戶部尚書、監修國史。以正議大夫、守刑部尚書、上柱國、河東縣開國男、食邑三百戶、賜紫金魚袋裴贄為中書侍郎,兼刑部尚書、同平章事,充集賢殿大學士。以銀青光祿大
 夫、行尚書吏部侍郎、上柱國裴樞為中書侍郎、同平章事,判戶部事。辛亥,以光祿大夫、尚書右僕射、租庸使張浚罷租庸使,守本官。十月丙辰朔。辛酉,以前清海軍節度副使、朝散大夫、檢校左散騎常侍、御史大夫、上柱國王溥守左散騎常侍,充鹽鐵副使。癸未,制以保義軍節度留後、銀青光祿大夫、檢校戶部尚書、兼御史大夫、上柱國硃友謙為金紫光祿大夫、檢校尚書右僕射,兼陜州大都督府長史、御史大夫,充保義軍節度、陜虢觀察
 處置等使。



 十一月乙酉朔。庚寅,左右軍中尉劉季述、王仲先廢昭宗,幽於東內問安宮,請皇太子裕監國。時昭宗委崔胤以執政,胤恃全忠之助,稍抑宦官。而帝自華還宮後,頗以禽酒肆志,喜怒不常,自宋道弼等得罪,黃門尤懼。至是,上獵苑中,醉甚,是夜,手殺黃門、侍女數人。庚寅,日及辰巳,內門不開。劉季述詣中書謂宰相崔胤曰:「宮中必有不測之事,人臣安得坐觀?我等內臣也,可以便宜從事。」即以禁兵千人破關而入,問訊中人,具知
 其故。即出與宰臣謀曰:「主上所為如此,非社稷之主也。廢昏立明,具有故事,國家大計,非逆亂也。」即召百官署狀,崔胤等不獲已署之。季述、仲先與汴州進奏官程巖等十三人請對,對訖,季述上殿待罪次。左右軍將士齊唱萬歲聲,遂突入宣化門,行至思政殿,便行殺戮,徑至乞巧樓下。帝遽見兵士,驚墮床下,起而將去,季述、仲先掖而令坐。何皇后遽出拜曰:「軍容長官護官家,勿至驚恐,有事取軍容商量。」季述即出百官合同狀,曰:「陛下倦臨
 寶位,中外群情,願太子監國,請陛下頤養於東宮。」帝曰:「吾昨與卿等歡飲,不覺太過,何至此耶!」皇后曰:「聖人依他軍容語。」即於御前取國寶付季述,即時帝與皇后共一輦,並常所侍從十餘內人赴東宮。入後,季述手自扃鎖院門,日於窗中通食器。是日,迎皇太子監國,矯宣昭宗命稱上皇。甲午,宣上皇制,太子登皇帝位,宰臣、百僚、方鎮加爵進秩,又賜百僚銀一千五百兩、絹千匹、綿萬兩充救接,皆季述求媚於朝也。時硃全忠在定州行營,
 崔胤與前左僕射張浚告難於全忠,請以兵問罪,全忠自行營還大梁。十二月乙卯朔。癸未夜。護駕鹽州都將孫德昭、周承誨、董彥弼以兵攻劉季述、王仲先,殺仲先,攜其首詣東宮門,呼曰:「逆賊王仲先已斬首訖,請陛下出宮慰諭兵士。」宮人破鑰,帝與皇后方得出。



 天復元年春正月甲申朔,昭宗反正,登長樂門樓,受朝賀。班未退,孫德昭執劉季述至樓前,上方詰責,已為亂棒擊死,乃尸之於市。乙酉,制以孫德昭檢校司空,充靜
 海軍節度使。丙戌,宰相崔胤進位司空。己丑,硃全忠械程巖,折足檻送京師,戮之於市。制皇太子裕降為德王,改名祐。庚寅,制以孫德昭為安南節度、檢校太保。以周承誨為邕州刺史、邕管節度經略使,以董彥弼為容州刺史、容管節度等使,並檢校太保、同平章事。殺神策軍使李師虔、徐彥回。敕曰:「朕臨御已來,十有四載,常慕好生之德,固無樂殺之心。昨季述等幽辱朕躬,迫脅太子。李師虔是逆賊親厚,選來東內主持,動息之間,俾其偵
 伺。每有須索,皆不供承。要紙筆則恐作詔書,索錐刀則慮為利器,凌辱萬狀,出入搜羅。朕所御之衣,晝服夜濯,凝冽之際,寒苦難勝。嬪嬙公主,衾裯皆闕。緡錢則貫百不入,繒帛則尺寸難求。六輩同其主張,五人權其威勢。若言狀罪,翰墨難窮,若許生全,是為貸法,宜並處斬。」時硃全忠既服河朔三鎮,欲窺圖王室篡代之謀,以李克用在太原,懼其角逐。是月,全忠令大將張存敬率兵三萬,由含山襲河中王珂。晉州刺史張漢瑜、絳州刺史、陶
 建不意賊至,城守無備,皆以郡降。存敬移兵圍河中,王珂求救於太原,克用不能救,乃嬰城謂存敬曰:「吾與汴王有舊,俟王至即降。」二月甲寅朔。戊辰,硃全忠至河中,遂移王珂及兄璘、弟瓚舉室徙於汴,以張存敬守河中。是月,制以全忠檢校太師、守中書令,進封梁王。



 三月癸未朔,全忠引軍歸汴,奏:「河中節度使歲貢課鹽三千車,臣今代領池場,請加二千車,歲貢五千車。候五池完葺,則依平時供訂額。」從之。四月癸丑朔,汴軍大舉攻太原,
 氏叔琮以兵三萬由天井關進攻澤潞,節度使孟遷以上黨降。叔琮長驅出圍柏,營於洞渦驛。葛從周率趙、魏、中山之兵由土門入,陷承天軍,與叔琮會。時屬大雨,芻糧不給,汴將保眾而還。甲戌,天子有事於宗廟。是日,禦長樂門,大赦天下,改元天復。李茂貞自鎮來朝,賜宴於壽春殿,進錢數萬緡。時中尉韓全誨及北司與茂貞相善,宰相崔胤與硃全忠相善,四人各為表裏。全忠欲遷都洛陽,茂貞欲迎駕鳳翔,各有挾天子令諸侯之意。



 五
 月壬午朔。庚子,制門下侍郎、戶部尚書、平章事陸扆加兵部尚書,進階特進。壬寅,制以硃全忠兼河中尹、河中節度、晉絳慈隰觀察處置、安邑解縣兩池榷鹽制置等使。閏六月辛巳朔,制以河陽節度丁會依前檢校司徒,兼潞州大都督府長史、昭義節度等使,代孟遷;以遷檢校司徒,為河陽節度。全忠奏也。仍請於昭義節度官階內落下邢、洺、磁三州,卻以澤州為屬郡,其河陽節度只以懷州為屬郡,從之。全忠又奏請以齊州隸鄆州,從之。
 十月己卯朔。戊戌,全忠引四鎮之師七萬赴河中,京師聞之大恐,豪民皆亡竄山谷。



 十一月己酉朔。壬子,中尉韓全誨與鳳翔護駕都將李繼誨奉車駕出幸鳳翔。是日,汴軍陷同州,執州將司馬鄴,華州節度使韓建遣判官李巨川送款。甲寅,汴軍駐靈口。乙卯,全忠知帝出幸,乃回兵攻華州。大軍駐赤水,全忠以親兵駐西溪。韓建出降,乃署為忠武軍節度使,以陳州為理所。丁巳,宰相崔胤令戶部侍郎王溥至赤水砦,促全忠以兵迎駕。戊
 午,全忠自赤水趨長安,崔胤率文武百僚太子太師盧知猷已下迎全忠於坡頭。庚申,汴軍趨鳳翔。戊辰,至岐下。全忠令判官李擇、裴鑄入城奏事,言:「臣在河中,得崔胤書,言奉密詔令臣以兵士迎駕,臣不敢擅自迎鑾。」昭宗怒胤矯命,連詔全忠以兵士還鎮。辛未,全忠引軍離鳳翔,退攻邠州。甲戌,制扶危致理功臣、開府儀同三司、守司空、門下侍郎、平章事、充太清宮使、弘文館大學士、延資庫使、諸道鹽鐵轉運等使、判度支、上柱國、魏國公、
 食邑五千戶、食實封二百戶崔胤可責授朝散大夫、守工部尚書。乙亥,邠州節度使李繼徽以城降,全忠乃舍其孥于河中,以繼徽從軍。以汴軍營於三原。十二月己卯,崔胤自長安至三原砦,與全忠謀攻鳳翔。



 二年春正月戊申朔,車駕在鳳翔。全忠在三原,李克用遣大將周德威攻慈、隰、晉等州。全忠歸河中,令其將硃友寧率眾五萬屯絳州,大敗太原軍於蒲縣西北,友寧乘勝追奔,陷汾州,進圍太原。天子遣諫議大夫張顗至
 晉州諭全忠,令與太原通和。屬友寧再戰不利,乃還關西。四月丁丑,硃友寧總大軍屯於興平。



 五月,岐軍出戰,大敗於武功南之漢谷。全忠聞捷,自引汴軍五萬西征。



 六月,進營虢縣。丁亥,進圍鳳翔,遣判官入城迎駕。



 九月,岐軍出戰,又敗。



 十一月,鄜州節度使李周彞率眾救鳳翔。十二月癸酉,汴將孔勍乘虛襲下鄜州,獲周彞妻子,周彞即以兵士來降。於是邠、寧、鄜、坊等州皆陷於汴軍。茂貞懼,謀誅內官以解。



 三年春正月癸卯朔,車駕在鳳翔。甲辰,天子遣中使到全忠軍,茂貞亦令軍將郭啟奇來達上欲還京之旨。丙午,青州牙將劉鄩陷全忠之兗州,又令牙將張厚入奏,是日,亦竊發於華州,殺州將婁敬思。上又令戶部侍郎韓偓、趙國夫人寵顏宣諭於全忠軍。辛亥,全忠令判官李振入奏,上令翰林學士姚洎傳宣,令全忠喚崔胤令率文武百僚來迎駕。癸丑,上令禮部尚書蘇循傳詔,賜全忠玉帶,仍令全忠處分蔣玄暉侍帝左右。丁巳,蔣玄
 暉與中使同押送中尉韓全誨、張弘彥已下二十人首級,告諭四鎮兵士回鑾之期。戊午,遣中使走馬華州,追崔胤,胤托疾不至。甲子巳時,車駕出鳳翔,幸全忠軍。全忠素服待罪,泣下不自勝,上親解玉帶賜之。乙丑,次扶風,令硃友倫總兵侍衛。丙寅,次武功。丁卯,次興平,宰臣崔胤率百官迎謁。即日降制,以崔胤守司空、門下侍郎、平章事,復太清宮使、弘文館大學士、延資庫使、諸道鹽鐵轉運使、判度支,魏國公封邑如故。戊辰,次咸陽。己巳,入
 京師。天子素服哭於太廟,改服冕旒,謁九廟。禮畢,禦長樂樓,大赦,百僚稱賀。全忠處左軍。辛未,宴全忠於內殿,內第子奏樂。是日,制內官第五可範已下七百人並賜死於內侍省,其諸道監軍及小使,仰本道節度使處斬訖奏,從全忠、崔胤所奏也。帝悲惜之,自為奠文祭之。



 二月壬申朔。甲戌,制賜全忠「回天再造竭忠守正功臣」名。己卯,制以輝王祚充諸道兵馬元帥。又制以回天再造竭忠守正功臣、宣武宣義天平護國等軍節度使、汴宋
 亳輝河中晉絳慈隰鄭滑潁鄆齊曹等州觀察處置等使、太清宮修葺宮闕制置度支解縣池場等使、開府儀同三司、檢校太師、守中書令、河中尹、汴滑鄆等州刺史、上柱國、梁王、食邑九千戶、食實封六百戶硃全忠守司徒,兼侍中、判六軍十二衛。以吏部尚書、平章事裴樞檢校右僕射,同平章事,兼廣州刺史、可守太尉、中書令,充諸道兵馬副元帥,進邑三千戶。以宰臣崔胤清海軍節度、嶺南東道觀察等使。甲戌,制以門下侍郎、兵部尚書、
 同平章事、監修國史陸扆責授沂王傅分司。己丑,上宴全忠於壽春殿。又令全忠與茂貞書,取平原公主。同州節度使趙翊、陜州節度使硃友謙來朝。制以硃友裕為華州刺史,充感化軍節度使。乙未,會鞠於保寧殿,全忠得頭籌,令內弟子送酒,仍面賜副元帥官告。以新除廣州節度使裴樞為門下侍郎、吏部尚書、平章事、監修國史;以戶部侍郎王溥同平章事。戊戌,全忠歸大梁,上宴之內殿,置酒於延喜門。是日,全忠與四鎮判官皆預席,
 上臨軒泣別,又令中使走送禦制《楊柳枝》詞五首賜之。辛丑,平原公主至京師。



 三月壬寅朔,全忠引四鎮之兵征王師範。先是,大將硃友寧、楊師厚前軍臨淄、青,師範求援於淮南,楊行密遣將王景仁帥眾萬人赴之。四月辛未朔,西川王建以兵攻秦、隴,乘茂貞之弱也,仍遣判官韋莊入貢,修好於全忠。



 五月,制鳳翔隴右四鎮北庭行軍、彰義軍節度、涇原渭武觀察處置押蕃落等使、開府儀同三司、守尚書令、兼侍中、鳳翔尹、上柱國、秦王李
 茂貞可檢校太師、守中書令。初,茂貞凌弱王室,朝廷姑息,加尚書令,及是全忠方守太尉,茂貞懼,乞罷尚書令故也。崔胤奏:「六軍十二衛名額空存,實無兵士。京師侍衛,亦藉親軍。請每軍量召募一千一百人,共置六千六百人。」從之。乃令六軍諸衛副使、京兆尹鄭元規立格招收於市。制以潁州刺史硃友恭檢校司空,兼徐州刺史,充武寧軍節度使,從全忠奏也。



 六月,青州、淮南軍與汴人戰於臨淄,汴軍大敗,硃友寧戰死,傳首淮南。



 九月,汴
 將楊師厚大敗青州軍於臨朐。荊南節度使成汭以舟師赴援鄂州,澧朗雷彥恭承虛襲陷江陵。汭軍士聞之潰歸,汭憤怒投水而死。趙匡凝遂以兵襲荊州,據之。辛巳,汴州護駕都將硃友倫擊鞠墜馬卒,全忠怒,殺同鞠將校數人。



 十一月丁酉朔,王師範以青州降楊師厚,全忠復令師範知青州事。邠州、鳳翔兵士逼京畿。汴軍屯河中。青州牙將劉鄩以兗州降葛從周,稟師範命也。全忠嘉之,署為元帥府都押衙,權知鄜州留後事。十二月
 丁卯朔。辛巳,制以禮部尚書獨孤損為兵部侍郎、同平章事。丙申,制守司徒、侍中、太清宮使、弘文館大學士、延資庫使、判六軍十二衛事、諸道鹽鐵轉運使、判度支、上柱國、魏國公、食邑四千五百戶崔胤責授太子賓客,守刑部尚書、兼京兆尹、六軍諸衛副使鄭元規責授循州司戶。是日,汴州扈駕指揮使硃友諒殺胤及元規、皇城使王建勛、飛龍使陳班、閣門使王建襲、客省使王建乂、前左僕射上柱國河間郡公張浚。全忠將逼車駕幸洛
 陽,懼胤、浚立異也。



 天祐元年春正月丁酉朔,以翰林學士、左拾遺柳璨為右諫議大夫、同平章事,賜紫金魚袋。己亥,制以兵部尚書崔遠為中書侍郎、同平章事、集賢殿大學士。己酉,全忠率師屯河中,遣牙將寇彥卿奉表請車駕遷都洛陽。全忠令長安居人按籍遷居,徹屋木,自渭浮河而下,連甍號哭,月餘不息。秦人大罵於路曰:「國賊崔胤,召硃溫傾覆社稷,俾我及此,天乎!天乎!」丁巳,車駕發京師。癸亥,
 次陜州,全忠迎謁於路。



 二月丙寅朔。乙亥,全忠辭赴洛陽,親督工作。四月丙寅朔。癸巳,帝遣晉國夫人可證傳詔諭全忠,言中宮誕蓐未安,取十月入洛陽宮。全忠意上遲留俟變,怒甚,謂牙將寇彥卿曰:「亟往陜州,到日便促官家發來!」閏四月乙未朔。丁酉,車駕發陜州。壬寅,次穀水行宮。時崔胤所募六軍兵士,胤死後亡散並盡,從上東遷者,唯諸王、小黃門十數,打球代奉內園小兒共二百餘人。全忠在陜,仍慮此輩為變,欲盡去之,以汴卒
 為侍衛。至谷水頓,全忠令醫官許昭遠告內園等謀變,因會設幄,酒食次並坑之,乃以謀逆聞。由是帝左右前後侍衛職掌,皆汴人也。甲辰,車駕由徽安門入,硃全忠、張全義、宰相裴樞獨孤損前導。是日大風雨土,跬步不辨物色,日暝稍止。上謁太廟,禮畢還宮,御正殿宣勞從官衛士,受駕。乙巳,上禦光政門,大赦,制曰:



 乃睠中州,便侯伯會朝之路;運逢百六,順古今禳避之宜。況建鼎舊京,我家二宅,轘轅通其左,郟、鄏引其前。周平王之東遷,
 更延姬姓;漢光武之定業,克茂劉宗。肇葺新都,祈天永命,皆因否運,復啟昌期。或西避於戎狄,或載殲於妖孽。朕遭家不造,布德不明,十載已來,三罹播越。亦屬災纏秦、雍,叛起邠、岐。始幸石門,以避衛兵之亂;載行華嶽,仍驚畿邑之侵。憂危則矢及車輿,凌脅則火延宮廟。迨至逆連宮豎,構結奸兇,致劉季述幽朕於下宮,韓全誨劫予於右輔。莫匪兵圍內殿,焰亙九重,皆思假武以容身,唯效指鹿而威眾。矯宣天憲,欺蔑外籓,行書詔以任情,
 欲忠良而獲罪。雖群方岳牧,協力匡扶,拘戎律於阻修,報朝恩而隔越。副元帥、梁王全忠以兼鎮近輔,總兵四籓,遠赴岐陽,躬迎大駕。辛勤百戰,盡剿兇渠,營野三年,竟回鑾輅。咸、鎬載新其宮闕,讓、珪絕類於閹徒,方崇再造之功,以正中興之運。又邠岐結釁,巴蜀連兵,上負國恩,下隳鄰好。焚宮烈火,更延熱於親鄰;卻駕兇鋒,復延侵於禁苑。抑又太一游處,並集六宮,罰星熒惑,久纏東井,玄象薦災於秦分,地形無過於洛陽。爰有一二藎臣,
 洎四方同志,竭心王室,共誓嘉謀。魏鎮定燕,航大河而畢至;陳徐潞蔡,輦巨軸以偕來。披荊棘而立朝廷,劃灰燼而化輪奐。左郊祧而右社稷,肅爾崇嚴;前廣殿而後重廊,藹然華邃。公卿僉議,龜筮協從。甲子令年,孟夏初吉,備法駕而離陜分,列百官而入洛郊,觀此殷繁,良多嘉慰。謝罪太廟,憂惕驚懷;登御端門,軫惻興感。蓋以一人寡祐,至萬姓靡寧,工役艱疲,忠良盡瘁,克建再遷之業,冀延八百之基。宜覃渙汗之恩,俟此雍熙之慶,滌瑕
 蕩垢,咸與惟新。可大赦天下,改天復四年為天祐元年,於戲!肆眚閶闔,即安宮闈。雖九廟幾筵,已閟於新室;而諸陵松柏,遙隔於舊都。將務乂寧,難申綣慕。文武百闢,執事具僚,從我千里而來,端爾一心蒞政。恩覃既往,效責從新,方當開國之初,必舉慢官之罰。



 戊申,敕今後除留宣徽兩院、小馬坊、豐德廟、御廚、客省、閣門、飛龍、莊宅九使外,其餘並停。內園冰井公事委河南尹,仍不差內夫人傳宣。殺醫官閻祐之、國子博士歐陽特,言星讖也。
 宰相裴樞兼右僕射、諸道鹽鐵轉運等使、監修國史,戶部尚書、門下侍郎、平章事獨孤損判度支,中書侍郎、平章事柳璨判戶部事。



 五月乙丑朔。丙寅,制河陽節度使張漢瑜同平章事。宴百僚於崇勛殿,上贊述全忠之功業,因言御樓前一日所司亡失赦書,賴元帥府收得副本施行,幾失事矣,中書不得無過。裴樞等起待罪。中飲,帝更衣,召全忠曲宴閣中,全忠懇辭。帝曰:「朕以全忠功業崇高,欲齋中款曲,以表庇賴耳。全忠既不欲來,即令
 敬翔來,朕與之言。」全忠令敬翔私退,奏曰:「敬翔亦醉而出矣。」己巳,全忠辭赴大梁,宴於崇勛殿,是日雨甚。乙酉,翰林學士、左諫議大夫、知制誥沈棲遠守本官,以病陳乞故也。丁亥,敕河南府畿縣先減尉一員,可準京兆府例,復置縣尉一員。癸巳,中書奏:準今年四月十一日赦文,陜州都督府改為興唐府,其都督府長史宜改為尹,左右司馬為少尹,錄事為司錄,陜縣為次赤,餘為次畿。從之。



 六月甲午朔,邠州楊崇本侵掠關內,全忠遣硃友
 裕屯軍於百仁村。丙申,通議大夫、中書舍人、賜紫金魚袋楊注可充翰林學士。庚子,三佛齊國入朝使薄訶粟可寧遠將軍。丁未,制金紫光祿大夫、太子少傅盧紹可太子太保致仕。銀青光祿大夫、太子少師、天水男、食邑三百戶趙崇可檢校右僕射。甲寅,以京兆少尹鄭韜光為太常少卿,前侍御史韋說為右司員外郎,前進士姚顗為校書郎,前進士趙頎、劉明濟、竇專並可秘書省校書郎正字,從柳璨奏也。荊南襄州忠義軍節度、開府儀
 同三司、檢校太師、中書令、江陵尹、襄州刺史、上柱國、楚王、食邑六千戶趙匡凝宜備禮冊命。七月癸亥朔,全忠率師討邠、鳳。甲子,自汴至洛陽,宴於文思球場。全忠入,百官或坐於廊下,全忠怒,笞通引官何凝。丙寅,制金紫光祿大夫、行御史中丞、上柱國韓儀責授棣州司馬,侍御史歸藹責授登州司戶,坐百官傲全忠也。甲戌,制以中大夫、中書舍人、上柱國、賜紫金魚袋杜彥林為太中大夫、守御史中丞。丁丑,制以兵部郎中蕭頎為吏部郎
 中,戶部郎中徐綰為兵部郎中,司勛員外郎張茂樞為禮部郎中,監察御史卻殷象為右補闕。己卯,制武昌軍節度、鄂岳蘄黃等州觀察處置兼三司水陸發運淮南西面行營招討等使、開府儀同三司、檢校太師、中書令、西平王、食邑三千戶杜洪加食邑一千戶,實封二百戶。庚寅,中書奏:「西京舊有凌煙閣,圖畫功臣,今遷都洛陽,合議修建。副元帥梁王勛庸冠世,請凌煙閣之側別創一閣,以表殊勛。」從之。



 八月壬辰朔。壬寅夜,硃全忠令左
 龍武統軍硃友恭、右龍武統軍氏叔琮、樞密使蔣玄暉弒昭宗於椒殿。自帝遷洛,李克用、李茂貞、西川王建、襄陽趙匡凝知全忠篡奪之謀,連盟舉義,以興復為辭。而帝英傑不群,全忠方事西討,慮變起於中,故害帝以絕人望。帝自離長安、日憂不測,與皇后、內人唯沉飲自寬。是月壬寅,全忠令判官李振自河中至洛陽,與友恭等圖之。是夜二鼓,蔣玄暉選龍武衙官史太等百人叩內門,言軍前有急奏面見上。內門開,玄暉每門留卒十人,
 至椒殿院,貞一夫人啟關,謂玄暉曰:「急奏不應以卒來。」史太執貞一殺之,急趨殿下。玄暉曰:「至尊何在?」昭儀李漸榮臨軒謂玄暉曰:「院使莫傷官家,寧殺我輩。」帝方醉,聞之遽起。史太持劍入椒殿,帝單衣旋柱而走,太追而弒之。漸榮以身護帝,亦為太所殺。復執何皇后,將害之。後求哀於玄暉,玄暉以全忠止令害帝,釋後而去。帝殂,年三十八,群臣上謚曰聖穆景文孝皇帝,廟號昭宗。二年二月二十日,葬於和陵。



\end{pinyinscope}