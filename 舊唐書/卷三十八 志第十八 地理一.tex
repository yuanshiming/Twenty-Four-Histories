\article{卷三十八 志第十八 地理一}

\begin{pinyinscope}

 王者司牧黎元,方制天下。列井田而底職貢,分縣道以控華夷。雖《皇墳》、《帝典》之殊塗,《禹貢》、《周官》之異制,其於建侯胙土,頒瑞剖符,外湊百
 蠻,內親九牧,古之元首,咸有意焉。然子弟受封,周室竟貽於衰削;郡縣為理,秦人不免於敗亡。蓋德業有淺深,制置無工拙。殷、周未為得,秦、漢未為非。摭實而言,在哲後守成而已。謹詳前代隆平之時,校今日耗登之數,存諸戶籍,以志休期。



 昔秦並天下,裂地為四十九郡,郡置守尉,以御史監之。其地西臨洮,而北沙漠,東縈南帶,皆際海濱。漢興,以秦郡稍大,析置郡國。武帝斥越攘胡,土宇彌廣。哀、平之季,凡郡國百有三,縣千三百一十四,道三十二,侯國二百四十一,而諸郡置十三部刺史分統之。謂司隸、並、荊、兗、豫、揚、冀、青、徐、益、交、涼、幽等十三州。漢地東西九千三百二里,南北一萬二千三百六十八里。後漢郡國,百有五,縣道侯國千
 一百八十六。亦如西京之制,置十三州刺史以充郡守。其地廣袤,亦如前制。



 曹魏之時,三分鼎峙,淮、漢之間,鞠為斗壤。洎太康混一,尋陷胡戎。南北分爭,何暇疆理?三百年間,廢置不一。及隋氏平陳,寰區一統。大業三年,改州為郡,亦如漢制,置司隸、刺史,以糾郡守。大凡隋簿,郡百九十,縣一千二百五十五,戶八百九十萬七千五百
 三十六,口四千六百一萬九千九百五
 十六。其地
 東西九千三百里,南北一萬四千八百一十五里。東、南皆際大海,西至且末,北至五原,隋氏之極盛也。及大業季年,群盜蜂起,郡
 縣
 淪陷,戶口減耗。高祖受命之初,改郡為州,太守並稱刺史。其緣邊鎮守及襟帶之地,置總管府,以統軍戎。至武德七年,改總管府為都督府。



 自隋季喪亂,群盜初附,權置州郡,倍於開皇、大業之間,貞觀元年,悉令並省。始於山河
 形便,分為十道:一曰關內道,二曰河南道,三曰河東道,四曰河北道,五曰山南道,六曰隴右道,七曰淮南道,八曰江南道,九曰劍南道,十曰嶺南道。至十三年定簿,凡州府三百五十八,縣一千五百五十一。至十四年平高昌,又增二州六縣。自北殄突厥頡利,西平高昌,北逾陰山,西抵大漠。其地東極海,西至焉耆,南盡林州南境,北接薛延陀界。凡東西九千五百一十里,南北萬六千九百一十八里。高宗時,平高麗、百濟,遼海已東,皆為州,俄
 而復叛,不入提封。景雲二年,分天下郡縣,置二十四都督府以統之。議者以權重不便,尋亦罷之。



 開元二十一年,分天下為十五道,每道置採訪使,檢察非法,如漢刺史之職:京畿採訪使、理京師城內都畿、河東理蒲卅理東都城內關內、以京官遙領河南、理汴州河北、理魏州隴右、理鄯州山南東道、理襄州山南西道、理梁州劍南、理益州淮南、理揚州江南東道、理蘇州江南西道、理洪州黔中、理黔州嶺南理廣州。又於邊境置節度、經略使,
 式遏四夷。凡節度使十,經略守捉使三。大凡鎮兵四十九萬人,戎馬八萬餘疋。每歲經費:衣賜則千二十萬疋段,軍食則百九十萬石,大凡千二百一十萬。開元已前,每年邊用不過二百萬,天寶中至於是數。



 安西節度使,撫寧西域,統龜茲、焉耆、于闐、疏勒四國。安西都護府治所,在龜茲國城內,管戍兵二萬四千人,馬二千七百疋,衣賜六十二萬疋段。焉耆治所,在安西府東八百里。於闐,在安西府南二千里。疏勒,在安西府西二千餘里。



 北庭節度使,防制突騎施、堅昆、斬啜,管瀚海、天山、伊吾三軍。北庭節度使所治,在北庭都護府,管兵二萬人,馬五千疋,衣賜四十八萬疋段。突騎施牙帳,在北庭府西北三千餘里。堅昆,在北庭府北七千里。
 東北去斬啜千七百里。瀚海軍,在北庭府城內,管兵萬二千人,馬四千二百疋。天山軍,在西州城內,管兵五千人,馬五百疋。伊吾軍,在伊州西北三百里甘露川,管兵三千人,馬三百疋。



 河西節度使,斷隔羌胡。統赤水、大斗、建康、寧寇、玉門、墨離、豆盧、新泉等八軍,張掖、交城、白亭三守捉。河西節度使治,在涼州,管兵七萬三千人,馬萬九千四百疋,衣賜歲百八十萬疋段。赤水軍,在涼州城內,管兵三萬三千人,馬萬三千疋。大斗軍,在涼州西二百餘里,管兵七千五百人,馬二千四百疋。建康軍,在甘州西二百里,管兵五千三百人,馬五百疋。寧寇軍,在涼州東北千餘里。玉門軍,在肅州西二百里,管兵五千二百人,馬六百疋。墨離軍,在瓜州西北千里,管兵五千人,馬四百疋。豆盧軍,在沙州城內,管兵四千三百人,馬四百疋。新泉軍,在會州西北二百餘里,管兵千人。張掖守捉,在涼州
 南二里,管兵五百人。交城守捉,在涼州西二百里,管兵千人。白亭守捉,在涼州西北五百里,管兵千七百人。



 朔方節度使,捍禦北狄,統經略、豐安、定遠、西受降城、東受降城、安北都護、振武等七軍府。朔方節度使,治靈州,管兵六萬四千七百人,馬四千三百疋,衣賜二百萬疋段。經略軍,理靈州城內,管兵二萬七百人,馬三千疋。豐安軍,在靈州西黃河外百八十里,管兵八千人,馬千三百疋。安遠城,在靈州東北二百里黃河外,管兵七千人,馬三千疋。西受降城,在豐州北黃河外八十里,管兵七千人,馬千七百疋。安北都護府治,在中受降城黃河北岸,管兵六千人,馬二千疋。東受降城,在勝州東北二百里,管兵七千人,馬千七百疋。振武軍,在單于東都護府城內,管兵九千人,馬千六百疋。



 河東節度使,掎角朔方,以禦北狄,統天兵、大同、橫
 野、岢嵐等四軍,忻、代、嵐三州,雲中守捉。河東節度使,治太原府,管兵五萬五千人,馬萬四千疋,衣賜歲百二十六萬疋段,軍糧五十萬石。天兵軍,理太原府城內,管兵三萬人,馬五千五百疋。雲中守捉,在單于府西北二百七十里,管兵七千七百人,馬二千疋。大同軍,在代州北三百里,管兵九千五百人,馬五千五百疋。橫野軍,在蔚州東北一百四十里,管兵三千人,馬千八百疋。忻州,在太原府北百八十里,管兵七千八百人。代州,至太原府五百里,管兵四千人。嵐州,在太原府西北二百五十里,管兵三千人。岢嵐軍,在嵐州北百里,管兵一千人。



 範陽節度使,臨制奚、契丹,統經略、威武、清夷、靜塞、恆陽、北平、高陽、唐興、橫海等九軍。範陽節度使,理幽州,管兵九萬一千四百人,馬六千五百疋,衣賜八十萬疋段,軍糧五十萬石。經略軍,在幽州城內,管軍三萬
 人,馬五千四百疋。威武軍,在檀州城內,管兵萬人,馬三百疋。清夷軍,在媯州城內,管兵萬人,馬三百疋。靜塞軍,在薊州城內,管兵萬六千人,馬五百疋。恆陽軍,在恆州城東,管兵三千五百人。北平軍,在定州城西,管兵六千人。高陽軍,在易州城內,管兵六千人。唐興軍,在莫州城內,管兵六千人。橫海軍,在滄州城內,管兵六千人。



 平盧軍節度使,鎮撫室韋、靺鞨,統平盧、盧龍二軍,榆關守捉,安東都護府。平盧軍節度使治,在營州,管兵萬七千五百人,馬五千五百疋。平盧軍,在營州城內,管兵萬六千人,馬四千二百疋。盧龍軍,在平州城內,管兵萬人,馬三百疋。榆關守捉,在營州城西四百八十里,管兵三百人,馬百疋。安東都護府,在營州東二百七十里,管兵八千五百人,馬七百疋。



 隴右節度使,以備羌戎,統臨洮、河源、白水、安人、振威、威戎、莫門、
 寧塞、積石、鎮西等十軍,綏和、合川、平夷三守捉。隴右節度使,在鄯州,管兵七萬人,馬六百疋,衣賜二百五十萬疋段。臨洮軍,在鄯州城內,管兵萬五千人,馬八千疋。河源軍,在鄯州西百二十里,管兵四千人,馬六百五十疋。白水軍,在鄯州西北二百三十里,管兵四千人,馬五百疋。安人軍,在鄯州界星宿川西,兵萬人,馬三百五十疋。振威軍,在鄯州西三百里,管兵千人,馬五百疋。威戎軍,在鄯州西北三百五十里,管兵千人,馬五十疋。綏和守捉,在鄯州西南二百五十里,管兵千人。合川守捉,在鄯州南百八十里,管兵千人。莫門軍,在洮州城內,管兵五千五百人,馬二百疋。寧塞軍,在廓州城內,管兵五百人,馬五十疋。積石軍,在廓州西百八十里,管兵七千人,馬三百疋。鎮西軍,在河州城內,管兵萬一千人,馬三百疋。平夷守捉,在河州西南四十里,管兵三千人。



 劍南節度使,西抗吐蕃,南撫蠻獠,
 統團結營及松、維、蓬、恭、雅、黎、姚、悉等八州兵馬,天寶、平戎、昆明、寧遠、澄川、南江等六軍鎮。劍南節度使治,在成都府,管兵三萬九百人,馬二千疋,衣賜八十萬疋段,軍糧七十萬石。團結營,在成都府城內,管兵萬四千人,馬千八百疋。翼州,管兵五百人。茂州,管兵三百人。維州,管兵五百人。天寶軍,在恭州東南九十里,管兵千人。柘州,管兵五百人。松州,管兵二千八百人。平戎城,在恭州南八十里,管兵千人。雅州,管兵四百人。當州,管兵五百人。黎州,管兵千人。昆明軍,在巂州南,管兵五千一百人,馬二百疋。寧遠城,在管兵二千人。悉州,管兵五千人。南江郡,管兵三百人。



 嶺南五府經略使,綏靜夷獠,統經略、清海二軍,桂管、容管、安南、邕管四經略使。
 五府經略使治,在廣州,管兵萬五千四百人,輕稅本鎮以自給。經略軍,在廣州城內,管兵五千四百人。清海軍,在恩州城內,管兵二千人。桂管經略使,治桂州,管兵千人。容管經略使,治容州,管兵千一百人。安南經略使,治安南都護府,即交州,管兵四千二百人。邕管經略使,管兵七百人。



 長樂經略使,福州刺史領之,管兵千五百人。



 東萊守捉、萊州刺史領之,管兵千人。東牟守捉。登州刺史領之,管兵千人。



 至德之後,中原用兵,刺史皆治軍戎,遂有防禦、團練、制置之名。要沖大郡,皆有節度之額;寇盜稍息,則易以觀察之號。



 東都畿汝防禦觀察使。領汝州,東都留守兼之。



 河陽三城節度使。治孟州,領孟、懷二州。



 宣武軍節度使。治汴州,管汴、宋、亳、潁四州。



 義成
 軍節度使。治滑州,管滑、鄭、濮三州。



 忠武軍節度使。治許州,管許、陳、蔡三州。



 天平軍節度使。治鄆州,管鄆、齊、曹、棣四州。



 兗海節度使。治兗州,管兗、海、沂、密四州。



 武寧軍節度使。治徐州,管徐、泗、濠、宿四州。



 平盧軍節度使。治青州,管淄、青、登、萊四州。



 陜州節度使。治陜州,管陜、虢二州。



 潼關防禦鎮國軍使。華州刺史領之。



 同州防禦長春宮使。同州刺史領之。



 鳳翔隴節度使。治鳳翔府,管鳳翔府、隴州。



 邠寧節度使。治邠州,管邠、寧、慶、鄜、坊、丹、延、衍等州。



 涇原節度使。治涇州,管涇、原、渭、武四州。



 朔方節度使。治靈州,管鹽、夏、綏、銀、宥、豐、會、麟、勝、單于府等州。



 河中節度使。治河中府,管蒲、晉、絳、慈、隰等州。



 昭義軍節度使。治潞州,領潞、澤、邢、洺、磁五州。



 河東節度使。
 治太原府,管汾、遼、沁、嵐、石、忻、憲等州。



 大同軍防禦使。雲州刺史領之,管雲、蔚、朔三州。



 魏博節度使。治魏州,管魏、貝、博、相、澶、衛六州。



 義昌軍節度使。治滄州,管滄、景、德三州。



 成德軍節度使。治恆州,領恆、趙、冀、深四州。



 義武軍節度使。治定州,領易、祁二州。



 幽州節度使。治幽州,管幽、涿、瀛、莫、檀、薊、平、營、媯、順等十州。



 山南西道節度使。治興元府,管開、通、渠、興、集、鳳、洋、蓬、利、璧、巴、閬、果、金、商等州。



 山南東道節度使。治襄州,管襄、復、均、房、鄧、唐、隨、郢等州。元和中,淮、蔡用兵,析鄧、唐二州別立一節度。



 荊南節度使。治江陵府,管歸、夔、峽、忠、萬、灃、朗等州,使親王領之。



 劍南西川節度使。治成都府,管彭、蜀、漢、眉、嘉、資、簡、維、茂、黎、雅、松、扶、文、龍、戎、翼、邛、巂、姚、柘、恭、當、悉、奉、疊、靜等州,使親王領之。



 劍南東川節度使。治梓州,管梓、綿、
 劍、普、榮、遂、合、渝、瀘等州。



 武昌軍節度使。治鄂州,管鄂、岳、蘄、黃、安、申、光等州。



 淮南節度使。治揚州,管揚、楚、滁、和、舒、壽、廬等州,使親王領之。



 浙江西道節度使。治潤州,管潤、蘇、常、杭、湖等州。或為觀察使。



 浙江東道節度使。治越州,管越、衢、婺、溫、臺、明等州。或為觀察使。



 福建觀察使。治福州,管福、建、泉、汀、漳等州。



 宣州觀察使。治宣州,管宣、歙、池等州。



 江南西道觀察使。治洪州,管洪、饒、吉、江、袁、信、虔、撫等州。喪亂後,時升為節度使。



 湖南觀察使。治潭州,管潭、衡、郴、、連、道、永、邵等州。



 黔中觀察使。治黔州,管涪、溪、思、費、辰、錦、播、施、珍、夷、業、溱、南、巫等州。



 嶺南東道節度使。治廣州,管廣、韶、循、崗、恩、春、賀、潮、端、藤、康、封、瀧、高、義、新、勤、竇等州。



 嶺南西道桂管經略觀察使。治桂州,管桂、昭、蒙、富、梧、潯、龔、鬱林、平琴、賓、澄、繡、
 象、柳、融等州。



 邕管經略使。治邕州,管邕、貴、黨、橫、田、嚴、山、巒、羅、潘等州。



 容管經略使。治容州,管容、辯、白、牢、欽、巖、禺、湯、瀼、古等州。



 安南都護節度使。治安南府,管交、武峨、粵、芝、愛、福祿、長、峰、陸、廉、雷、籠、環、崖、儋、振、瓊、萬安等州。



 上元年後,河西、隴右州郡,悉陷吐蕃。大中、咸通之間,隴右遺黎,始以地圖歸國,又析置節度。



 秦州節度使。治秦州,管秦、成、階等州。



 涼州節度使。治涼州,管西、洮、鄯、臨、河等州。



 瓜沙節度使。治沙州,管沙、瓜、甘、肅、蘭、伊、岷、廓等州。乾符之後,天下亂離。禮樂征伐,不自朝廷。禹跡九州,瓜分臠剖,或並或析,不可備書。



 今舉天寶十一載地理。唐土東至安東府,
 西至安西府,南至日南郡,北至單于府。南北如前漢之盛,東則不及,西則過之。漢地東至樂浪、玄菟、,今高麗、渤海是也。今在遼東,非唐土也。漢境西至燉煌郡,今沙州,是唐土。又龜茲,是西過漢之盛也。開元二十八年,戶部計帳,凡郡府三百二十有八,縣千五百七十有三。羈縻州郡,不在此數。戶八百四十一萬二千八百七十一,口四千八百一十四萬三千六百九,應受田一千四百四十萬三千八百六十二頃一十三畝。雖未盈兩漢之數,晉、魏以來,斯為盛矣。永泰之後,河朔、隴西,淪於寇盜。元和掌
 計之臣,嘗為版簿,二方不進戶口,莫可詳知。今但自武德已來,備書廢置年月。其前代沿革,略載郡邑之端。俾職方之臣,不殆於顧問耳。



 十道郡國



 關內道一河南道二



 關內道



 京師秦之咸陽,漢之長安也。隋開皇二年,自漢長安故城東南移二十里置新都,今京師是也。城東西十八
 里一百五十步,南北十五里一百七十五步。皇城在西北隅,謂之西內。正門曰承天,正殿曰太極。太極之後殿曰兩儀。內別殿、亭、觀三十五所。京師西有大明、興慶二宮,謂之三內。有東西兩市。都內,南北十四街,東西十一街。街分一百八坊。坊之廣長,皆三百餘步。皇城之南大街曰硃雀之街,東五十四坊,萬年縣領之。街西五十四坊,長安縣領之。京兆尹總其事。東內曰大明宮,在西內之東北,高宗龍朔二年置。正門曰丹鳳,正殿曰含元,含
 元之後曰宣政。宣政左右,有中書門下二省、弘文史二館。高宗已後,天子常居東內,別殿、亭、觀三十餘所。南內曰興慶宮,在東內之南隆慶坊,本玄宗在籓時宅也。自東內達南內,有夾城復道,經通化門達南內。人主往來兩宮,人莫知之。宮之西南隅,有花萼相輝、勤政務本之樓。禁苑在皇城之北。苑城東西二十七里,南北三十里,東至灞水,西連故長安城,南連京城,北枕渭水。苑內離宮、亭、觀二十四所。漢長安故城東西十三里,亦隸入苑中。
 苑置西南監及總監,以掌種植。



 京兆府隋京兆郡,領大興、長安、新豐、渭南、鄭、華陰、藍田、鄠、盩厔、始平、武功、上宜、醴泉、涇陽、雲陽、三原、宜君、同官、華原、富平、萬年、高陵二十二縣。武德元年,改為雍州。改大興為萬年,萬年為櫟陽,分櫟陽置平陵,以渭南縣屬華州,分醴泉置溫秀縣,分雲陽置石門縣。二年,分萬年置芷陽縣,分藍田置白鹿縣,分涇陽、始平置咸陽縣,分高陵置鹿苑縣,改平陵為粟邑縣,分醴泉置好畤縣,分
 盩厔置終南縣。三年,改白鹿為寧人縣,分藍田置玉山縣,分始平置醴泉縣。仍分武功、好畤、盩厔、扶風四縣置稷州,分溫秀、石門二縣置泉州。四年,改三原為池陽。五年,復以華州之渭南來屬。六年,改池陽為華池縣。七年,廢芷陽入萬年縣。貞觀元年,廢鹿苑入高陵縣,廢寧人、玉山入藍田縣,改雲陽為池陽縣,改華池為三原縣。廢稷州,以武功、好畤、盩厔三縣來屬。八年,廢粟邑入櫟陽縣,廢終南入盩厔縣,廢雲陽入池陽縣。仍改池陽為雲
 陽縣。廢上宜入岐州之岐陽縣。十七年,罷宜州,以華原、同官二縣來屬。二十年,又置宜君縣。永徽二年,廢宜君縣。乾封元年,置明堂、乾封二縣。咸亨元年,置美原縣。文明元年,置奉天縣。天授元年,改雍州為京兆郡,其年復舊。二年,分始平、武功、奉天、盩厔、好畤等縣置稷州;雲陽、涇陽、醴泉、三原、富平、美原等縣置宜州。大足元年罷,以鴻、宜、鼎、稷四州依舊為縣,以始平等十七縣還隸雍州。長安二年,廢乾封、明堂二縣。景龍三年,以邠州之永壽、
 商州之安業二縣來屬。景雲元年,復以永壽屬邠州,安業隸商州。開元元年,改雍州為京兆府,復隋舊名。四年,改同州蒲城縣為奉先縣,仍隸京兆府。天寶元年,以京師為西京。七載,置貞符縣。十一年廢。舊領縣十八,戶二十萬七千六百五十,口九十二萬三千三百二十。天寶領縣二十三,戶三十六萬二千九百二十一,口一百九十六萬七千一百。八十八府。理京城之光德坊。去東京八百里。



 萬年隋大興縣。武德元年,改為萬年。乾封元年,分置明堂縣,治永樂坊。長安三年廢,復並萬年。天寶七載,改為咸寧,乾元復舊也。



 長安隋縣。乾封元年,分為乾封縣,治懷直坊。長安三年廢,復並長安



 藍田隋縣



 渭南隋縣。武德元年屬華州,五年復隸雍州。天授二年置鴻州,分渭南置鴻門縣,凡領渭南、慶山、高陵、櫟陽、鴻門五縣。尋廢鴻門縣,還入渭南。大足元年,廢鴻州入雍州也



 昭應隋新豐縣,治古新豐城北。垂拱二
 年,改為慶山縣。神龍元年,復為新豐。天寶二年,分新豐、萬年置會昌縣。七載,省新豐縣,改會昌為昭應,治溫泉宮之西北



 三原隋縣。武德四年,移治清谷南故任城,改為池陽縣。六年,又移故所,改為華池縣,仍分置三原縣,屬北泉州。貞觀元年,廢三原縣,仍改華池縣為三原縣,屬雍州。九年,置高祖獻陵於縣之東南。天授元年,改隸鼎州。大足元年,隸京兆府



 富平隋縣。天授二年,隸宜州。大足元年州廢,還隸雍州。景雲二年,置中宗
 定陵於縣界



 櫟陽陳萬年縣。武德元年,改為櫟陽。二年,分置粟邑縣。貞觀八年,廢粟邑並櫟陽。天授三年,隸鴻州。大足元年,還隸雍州



 咸陽隋廢縣。武德二年,復分涇陽置。初治鮑橋,其年,移治杜郵。天授二年,則天以其母順陵在其界,升為赤,神龍初復



 高陵隋縣。天授二年,隸鴻州。大足元年,還雍州



 涇陽隋縣。天授二年,隸鼎州。大足元年,還雍州



 醴泉隋寧夷縣,後廢。貞觀十年,置昭陵於九嵕山,因析雲陽、咸陽二
 縣置醴泉縣。天授元年,改隸鼎州。大足元年,還雍州。寶應二年,又置肅宗建陵,在縣北之憑山



 雲陽隋縣。武德元年,分置石門縣。三年,於石門縣置泉州,領石門、溫秀二縣。貞觀元年,廢泉州,改石門為雲陽,改雲陽為池陽,並屬雍州。八年,廢雲陽,改池陽復名雲陽



 興平隋始平縣。天授二年,隸稷州。大足元年,還雍州。景龍四年,中宗送金城公主入蕃,別於此,因改金城縣。至德二年十月,改興平縣



 鄠隋縣



 武功隋縣。武德
 三年,分雍州之武功、好畤、盩厔、扶風四縣置稷州,因後稷封邰為名。其年,割郇州之郿、鳳泉二縣來屬。四年,又割岐州之圍川、鳳泉屬岐州,以盩厔、好畤、武功三縣屬雍州。天授二年,置稷州,領武功、始平、奉天、盩厔、好畤五縣。大足元年,還屬雍州



 好畤武德二年,分醴泉縣置,因漢舊名,屬雍州。三年,改隸稷州。貞觀元年,復屬雍州。天授二年,復隸稷州。大足元年,還屬雍州



 盩厔隋縣。武德三年,屬稷州。貞觀三年,還雍州。天授二年,屬
 稷州。大足元年,還雍州。天寶元年,改為宜壽縣。至德二年三月十八日,復為盩厔



 奉先舊蒲城縣,屬同州。開元四年,以管橋陵,改京兆府,仍改為奉先縣。十七年,制官員同赤縣。寶應二年,又置玄宗泰陵於縣東北



 奉天文明元年,以管乾陵,分醴泉置。天授二年,隸稷州。大足元年,還雍州



 華原舊宜州,領華原、宜君、同官、土門四縣。貞觀十七年,省宜州及土門縣,以華原、同官屬雍州。宜君屬坊州。垂拱二年,改華原為永安縣。天
 授二年,又置宜州,領永安、同官、富平、美原四縣。大足元年,廢宜州,縣還雍州。神龍元年,復為華原縣



 美原舊宜州土門縣,貞觀十七年廢。咸亨二年,又割富平、華原及同州之蒲城縣置,改為美原縣。天授二年,又屬宜州。大足元年,還雍州



 同官屬宜州,貞觀十七年,改屬雍州。天授二年,改屬宜州。大足元年,還屬雍州。



 華州上輔隋京兆郡之鄭縣。義寧元年,割京兆之鄭縣、華陰二縣置華山郡,因後魏郡名。武德元年,改為華州,
 割雍州之渭南來屬。五年,改渭南還雍州。垂拱元年,割同州之下邽來屬。二年,改為太州。神龍元年,復舊名。天寶元年,改為華陰郡。乾元元年,復為華州。上元元年十二月,改為太州,華山為太山。寶應元年,復為華州。舊領縣二,戶一萬八千八百二十三,口八萬八千八百三十。天寶領縣三,戶三萬三千一百八十七,口二十一萬三千六百一十三。在京師東一百八十里,去東都六百七十里。



 鄭隋縣。華陰隋縣。垂拱二年,改為仙掌縣。天授二年,分置同津縣於關口,長安中廢。神龍元年,復為華陰。上元元年,改為太陰縣。寶應元年復舊。下邽隋縣。舊屬同州,垂拱元年來屬。



 同州上輔隋馮翊郡。武德元年,改為同州,領馮翊、下邽、蒲城、朝邑、澄城、白水、郃陽、韓城八縣。三年,分朝邑置河濱縣,分郃陽置河西縣,分澄城置長寧縣。仍割河西、韓城、郃陽三縣,於河西置西韓州。九年,分馮翊置臨沮
 縣。貞觀元年,省河濱、臨沮二縣。八年,省長寧縣,廢西韓州,以郃陽、河西二縣來屬。垂拱元年,割下邽屬華州。開元四年,割蒲城縣屬京兆府。天寶元年,改同州為馮翊郡。乾元元年,復為同州。乾元三年,以蒲州為河中府;割朝邑縣入河中府,改河西縣為夏陽縣,又屬河中府。舊領縣九,戶五萬三千三百一十五,口二十三萬二千一十六。天寶領縣六,戶六萬九百二十八,口四十萬八千七百五。在京師東北二百五十五里,至東都六百二里。



 馮翊隋縣



 郃陽隋縣。武德三年,割屬西韓州。貞觀八年,復屬同州



 白水隋縣



 澄城隋縣



 韓城隋縣。武德七年,割屬西韓州。八年,自河西縣移西韓州理於此,領韓城、郃陽、河西三縣。貞觀八年,廢西韓州,以韓城等三縣復還屬同州也



 夏陽武德三年,分郃陽於此置河西縣。乾元三年,為夏陽。



 坊州上隋上郡之內部縣。周天和七年,元皇帝作牧鄜州,於此置馬坊。武德二年,分鄜州置坊州,以馬坊為
 名。天寶元年,改為中部。乾元元年,復為坊州。舊領縣二,戶七千五百七,口一萬一千六百七十一。天寶領縣四,戶二萬二千四百五十八,口十二萬二百八。在京師東北三百四十七里,去東都九百四十八里。



 鄜城隋縣。武德元年,屬鄜州。二年,改屬坊州



 中部隋曰內部。武德元年,屬鄜州。二年,改為中部,屬坊州



 宜君舊屬宜州。貞觀十七年廢,二十年復置,屬雍州,管玉華宮。永徽二年,復廢。龍朔三年,又割中部、同
 官兩縣地復置宜君縣,理古示殳祤城北,屬坊州



 昇平天寶十二年,分宜君縣置。



 丹州下隋延安郡之義川縣。義寧元年,於義川置丹陽郡。武德元年,改為丹州,領縣五。二年,於州置總管府,北連、北廣二州。貞觀元年,罷都督府。天寶元年,改為咸寧郡。乾元元年,復為丹州。舊領縣五,戶三千一百九十四,口一萬七千二十。天寶,戶一萬五千一百五,口八萬七千六百二十五。在京師東北六百一十一里,去東都
 九百二十里。



 義川隋縣



 汾川隋縣,治土壁堡。開元二十二年,移於今所



 咸寧隋縣,治白水川。景龍二年,移治長松川



 雲巖隋廢縣,武德元年,復分義川縣置,理回城堡。咸亨四年,移治今所



 門山隋廢縣,武德三年,分汾川縣置,治宋斯堡。總章二年,移治庫利川。



 鳳翔府隋扶風郡。武德元年,改為岐州,領雍、陳倉、郿、虢、岐山、鳳泉等六縣。又割雍等三縣,置圍川縣。其年,割
 圍川屬稷州。貞觀元年,廢稷州,以圍川及鄜州之麟游、普潤等三縣來屬。七年,又置岐陽縣。八年,改圍川為扶風縣,省虢縣及鳳泉。天授二年,復置虢縣。天寶元年,改為扶風郡。至德二年,肅宗自順化郡幸扶風郡,置天興縣,改雍縣為鳳翔縣,並治郭下。初以陳倉為鳳翔縣,乃改為寶雞縣。其年十月,克復兩京。十二月,置鳳翔府,號為西京,與成都、京兆、河南、太原為五京。寶應元年,並鳳翔縣入天興縣,後罷京名。舊領縣八,戶二萬七千二百
 八十二,口十萬八千三百二十四。天寶領縣九,戶五萬八千四百八十六,口三十八萬四百六十三。在京師西三百一十五里,去東都一千一百七十里。



 天興隋雍縣。至德二年分雍縣置天興縣。寶應元年廢雍縣,並入天興



 扶風武德三年,分岐山縣置圍川縣,取湋川為名,俗訛改為「圍」。四年,以圍川隸稷州。貞觀元年,為扶風縣,復屬岐州。



 寶雞隋陳倉縣。至德二年二月十五日,改為鳳翔縣,其月十八日,改為寶雞。



 岐陽貞觀七年,割扶風、岐山二縣置,至二十一年廢,永徽五年復置。



 岐山隋縣。武德元年,移治張堡。七年,移治龍尾城。貞觀八年,移治豬驛南,即今治所是。仍省虢縣並入。郿隋縣。義寧二年,於縣界置郿城郡,領郿、鳳泉二縣。武德元年,罷郡,置郇州,領郿縣。三年,廢郇州,改屬稷州。七年,改屬岐州。麟游義寧元年,於仁壽宮置鳳棲郡及麟游縣。其郡領麟游、上宜、普潤三縣。二年,改為麟游郡及靈臺縣,仍割安定郡之鶉觚
 來屬。武德元年,改麟游郡為麟州。貞觀元年,省靈臺縣入麟游,又廢麟州,以普潤、麟游二縣隸岐州,上宜隸雍州,鶉觚隸涇州。太宗改仁壽宮為九成宮。普潤隋縣。本屬麟州,貞觀元年來屬。虢隋縣。貞觀八年,廢入岐山縣。天授二年,復分岐山置虢縣。



 邠州上隋北地郡之新平縣。義寧二年,割北地郡之新平、三水二縣置新平郡。武德元年,改為豳州。二年,分新平置永壽縣。貞觀二年,又分新平置宜祿縣。開元十
 三年,改豳為邠。天寶元年,改為新平郡。乾元元年,復為邠州。舊領縣四,戶一萬五千五百三十四,口六萬四千八百一十九。天寶,戶二萬二千九百七十七,口十三萬五千二百五十。去京師西北四百九十三里,至東都一千一百三十二里。



 新平隋縣



 三水隋縣



 永壽武德二年,分新平置。神龍三年,改屬雍州。景龍元年,復屬邠州。宜祿貞觀二年,分新平置宜祿縣,後魏廢縣名。



 涇州上隋安定郡。武德元年,討平薛仁杲,改名涇州。天寶元年,復為安定郡。乾元元年,復為涇州。舊領縣五,戶八千七百七十三,口三萬五千九百二十一。天寶,戶三萬一千三百六十五,口十八萬六千八百四十九。在京師西北四百九十三里,至東都一千三百八十七里。



 安定隋縣。靈臺隋鶉觚縣。天寶元年,改為靈臺。良原



 潘原隋陰盤縣。天寶元年,改為潘原,縣界有潘原廢縣。臨涇隋縣。



 隴州上隋扶風郡之汧源縣。義寧二年,置隴東郡,領縣五。武德元年,改為隴州,以南由縣屬含州。四年,廢含州,復以南由來屬。天寶元年,改為汧陽郡。乾元元年,復為隴州。舊領縣五,戶四千五百七十一,口一萬八千六百三。天寶,戶二萬四千六百五十二,口十萬一百四十八。在京師西四百九十六里,去東都一千三百二十五里。



 汧源隋縣



 汧陽隋縣



 南由隋縣。武德元年,
 置含州於此,領南由一縣。四年,廢含州,以縣屬隴州。



 吳山隋長蛇縣。貞觀元年,改為吳山縣,治槐衙堡。上元元年,移治龍盤城。



 華亭隋縣。垂拱二年,改亭州。神龍元年,復舊。



 寧州上隋北地郡。義寧元年,領安定、羅川、襄樂、彭原、新平、三水六縣。二年,分定安置歸義縣,以新平、三水屬新平郡。武德元年,改北地郡為寧州。其年,以彭原縣屬彭州。三年,分彭原置豐義縣,躭彭州。又分定安置定平
 縣。貞觀元年,廢彭州,以彭原、豐義二縣來屬。仍於寧州置都督府。四年,罷都督府。十七年,廢歸義縣。天寶元年,改為彭原郡。乾元元年,復為寧州。舊領縣七,戶一萬五千四百九十一,口六萬六千一百三十五。天寶,領縣六,戶三萬七千一百二十一,口二十二萬四千八百三十七。在京師西北四百四十六里,至東都一千三百二十四里。



 定安隋縣



 彭原隋縣。武德元年,置彭州,領彭原
 一縣。二年,分置豐義縣。貞觀元年,廢彭州,以縣來屬寧州



 真寧隋羅川縣。天寶元年,改為真寧



 定平武德二年,分定安縣置。貞觀十七年,廢歸義縣,並入定平



 襄樂隋縣



 豐義武德二年,分彭原縣置,屬彭州。貞觀元年廢彭州,來屬。



 原州中都督府隋平涼郡。武德元年,平薛仁杲,置原州。貞觀五年,置都督府,管原、慶、會、銀、亭、達、要等七州。十年,省亭、達、要三州,唯督四州。天寶元年,改為平涼郡。乾
 元元年,復為原州。舊領縣三,戶二千四百四十三,口一萬五百一十二。天寶領縣四,戶七千三百四十九,口三萬三千一百四十六。在京師西北八百里,至東都一千六百四十五里。



 平高隋縣



 平涼隋縣,治陽晉川。開元五年,移治古塞城



 百泉隋縣



 蕭關貞觀六年,置緣州,領突厥降戶,寄治於平高縣界他樓城。高宗時,於蕭關置他樓縣。神龍元年,廢他樓縣,置蕭關縣。大中五年,置武
 州。



 慶州中都督府隋弘化郡。武德元年,改為慶州,領合水、樂蟠、三泉、馬嶺、弘化五縣。三年,改三泉為同川縣。六年,置總管府,改合水為合川縣,又置白馬、蟠交二縣。七年,改總管為都督府。貞觀元年,廢都督府及合川縣,仍割林州之華池縣來屬。二年,置洛源縣。四年,復置都督府及北永州,以洛源屬北永州。五年,又罷都督府,以慶州隸原州都督府。八年,又以廢北永州之洛源縣來屬。開
 元四年,復置都督府。二十六年,昇為中都督府。天寶元年,改為安化郡。至德元年,改為順化郡。乾元元年,改為慶州。舊領縣八,戶七千九百一十七,口三萬五千一十九。天寶領縣十,戶二萬三千九百四十九,口一十二萬四千三百三十六。在京師西北五百七十二里,至東都一千四百一十里。



 安化隋弘化縣,治弘州故城。武德六年,移治今所,與合水縣俱在州治。其年,改合水為合川縣。貞觀元年,省
 合川縣並入。神龍元年,改為安化縣



 樂蟠義寧元年,分合水縣置



 合水武德六年,分合水置蟠交縣。天寶元年廢,並入合水。馬嶺隋縣,治天家堡。貞觀八年,移理新城。以縣西有馬嶺阪。方渠景龍元年,分馬嶺置。同川義寧二年,廢北永州,分寧州彭原置於三泉縣故城。武德三年,復治同川城,改為同川縣



 洛源隋縣。大業十三年,為胡賊所破,因廢。貞觀二年,復置。又自延州金城縣移北永州治於此。八年,北永州廢,復以洛
 源縣屬慶州



 延慶武德六年,分合水縣置白馬縣。天寶元年,改為延慶縣



 華池隋舊縣。大業十三年,為胡賊所破,縣廢。武德四年復置,又於此置林州總管府,管永州。其林州領華池一縣。五年,改永州為北永州。七年,罷林州總管府。貞觀元年,廢林州,華池隸慶州



 懷安開元十年,檢括逃戶置,因名懷安。



 芳池州都督府,寄在慶州懷安縣界,管小州十:靜、獯、王、濮、林、尹、位、長、寶、寧,並黨項野利氏種落。



 安定州都督府寄在慶州界,管小州七:黨、橋、烏、西戎州、野利州、米州、還州。



 安化州都督府寄在慶州界,管小州七:永利州、威州、旭州、莫州、西滄州、儒州、琮州。



 鄜州上隋上郡。武德元年,改為鄜州,領洛交、洛川、三川、伏陸、內部、鄜城六縣。二年,以內部、鄜城隸坊州。三年,置直羅縣。貞觀二年,置都督府。六年,又改為大都督府。九年,復為都督府。天寶元年,改為洛交郡。乾元元年,復
 為鄜州。舊領縣五,戶一千七百三,口五萬一千二百一十六。天寶,戶二萬三千四百八十三,口十五萬三千七百十四。在京師東北五百里,至東都九百二十五里。



 洛交隋縣



 洛川隋縣



 三川隋縣。以華池水、黑水、洛水三水會同,因名



 直羅武德三年,分三川、洛交於直羅城置,以城枕羅水,其川平直故也



 甘泉武德元年,分洛交縣置伏陸縣。天寶元年,改為甘泉
 縣。



 延州中都督府隋延安郡。武德元年,改為延州總管府,領膚施、豐林、延川三縣,管南平、北武、東夏三州。四年,又管丹、廣、達三州。貞觀元年,罷都督府。開元二年,復置都督府,領丹、綏、渾等州。天寶元年,改為延安郡。乾元元年,復為延州。舊領縣九,戶九千三百四,口一萬四千一百七十六。天寶,戶一萬八千九百五十四,口十萬四十。在京師東北六百三十一里,至東都一千一百五十一里。



 膚施隋縣。分豐林、金明二縣置。延長隋廢縣。武德二年,復於此置北連州,領義鄉、齊明二縣。貞觀二年,廢北連州及義鄉、齊明二縣,並入延安。廣德二年,改為延長縣



 臨真隋縣。武德初,屬東夏州。貞觀二年,州廢來屬



 敷政隋因城縣。武德二年,移治於金城鎮,改為金城縣。又於界內置永州,領金城、洛盤、新昌、土塠四縣。貞觀四年,移永州於洛源縣。八年,廢洛盤等三縣,並入金城,屬延州。天寶元年,改金城為敷政



 金明
 隋廢縣。武德二年,置北武州,領開遠、金義、崇德、永定、安義五縣。復分膚施置金明縣。貞觀二年,廢北武州,以開遠等五縣並入金明縣



 豐林隋舊縣。武德四年,於此僑置雲州及雲中、榆林、龍泉三縣。八年,廢雲州及三縣,以龍泉並入臨真,以雲中、榆林並入豐林



 延水武德二年,分延川縣置西和州,領安人、修文、桑原三縣。貞觀二年,廢西和州,以修文、桑原並入安人,屬北基州。八年,廢北基州入延川。二十三年,改為弘風縣。神龍元
 年,改為延水



 延川隋舊縣。武德二年,置南平州,領義門縣。四年,廢南平州及縣,並入延川



 延昌武德二年,置北平州。貞觀三年廢,十年於廢州置罷交縣。天寶元年,改名為延昌縣



 渾州寄治延安郡界,隸延州節度使。



 綏州下隋雕陰郡。武德三年,於延州豐林縣置綏州總管府,領西和、南平、北基、銀、雲、貞、上、殄、北吉、匡、龍等十一州。其綏州領上、大斌、城平、綏德、延福五縣。六年,移治所
 於延州延川縣界。七年,又移治城平縣界魏平廢城。貞觀二年,平梁師都,罷都督府,移州治上縣。天寶元年,改為上郡。乾元元年,復為綏州。舊領縣五,戶三千一百六十三,口一萬六千一百二十九。天寶,戶一萬八百六十七,口八萬九千一百一十一。在京師東北一千里,至東都一千八百一十九里。



 龍泉隋曰上縣。天寶元年,改為龍泉



 延福隋縣。武德六年,置北吉州,領歸義、洛陽二縣,羅州領石羅、開
 善、萬福三縣;匡州領安定、源泉二縣。貞觀二年,三州及縣並廢,地並入延福



 綏德隋廢縣。武德二年,復置。六年,又分置雲州,領信義、淳義二縣;龍州領風鄉、義良二縣。貞觀二年,二州及縣俱廢,地並入綏德



 城平隋舊縣。武德三年,又置魏平縣,屬南平州。又置魏州,領安故、安泉二縣。七年,又於魏平城中置綏州總管府並大斌縣。貞觀二年,廢南平州、魏州及魏平、安故、安泉三縣,移綏州治於上縣,大斌治於今所



 大斌武德七
 年置,治魏平。貞觀二年,移治今所。



 銀州下隋雕陰郡之儒林縣。貞觀二年,平梁師都置銀州,隋舊名。天寶元年,改為銀川郡。乾元元年,復為銀州。舊領縣四,戶一千四百九十五,口七千七百二。天寶,戶七千六百二,口四萬五千五百二十七。在京師東北一千一百三十里,至東都一千五百七十九里。



 儒林隋舊縣



 撫寧隋縣。貞觀二年,屬綏州。八年,改屬銀州,治龍泉川。開元二年,移於今所



 真鄉隋
 縣



 開光隋縣。貞觀二年,屬綏州。八年,改屬柘州。十三年,柘州廢,來屬銀州



 靜邊州都督府舊治銀川郡界內,管小州十八



 歸德州寄治銀州界,處降黨項羌。



 夏州都督府隋朔方郡。貞觀二年,討平梁師都,改為夏州都督府,領夏、綏、銀三州。其夏州,領德靜、巖綠、寧朔、長澤四縣。其年,改巖綠為朔方縣。七年,於德靜縣置長州都督府。八年,改北開州為化州。十三年,廢化州及長
 州,以德靜、長澤二縣來屬。天寶元年,改為朔方郡。乾元元年,復為夏州。舊領縣四,戶二千三百二十三,口一萬二百八十六。天寶,戶九千二百一十三,口五萬三千一百四。在京師東北一千一百一十里,至東都一千六百八十里。



 朔方隋巖綠縣。貞觀二年,改為朔方縣。永徽五年,分置寧朔縣,長安二年廢。開元四年又置,九年又廢,還並入朔方



 德靜隋縣。貞觀七年,屬北開州。八年,改北
 開州為化州。十三年,廢化州,以縣屬夏州



 寧朔隋縣。武德六年,於此置南夏州。貞觀二年廢



 長澤隋縣。貞觀七年,置長州都督府。十三年,廢長州,縣還夏州



 雲中都督府黨項部落,寄在朔方縣界,管小州五:舍利、思璧州、阿史那州、綽部州、白登州。戶一千四百三十,口五千六百八十一



 呼延州都督府黨項部落,寄在朔方縣界,管小州三:賀魯州、那吉州、𧾷夾跌州。戶一百五十五,口六百五



 桑乾都督府寄朔方縣界,管小州四:鬱射州、藝失州、畢失州、叱略州。戶二百七十四,口一千三百二十三



 定襄都督府寄治寧朔縣界,管小州四:阿德州、執失州、蘇農州、拔延州。戶四百六十,口一千四百六十三



 達渾都督府延陀部落,寄在寧朔縣界,管小州五:姑衍州、步訖若州、嵠彈州、鶻州、低粟州。戶一百二十四,口四百九十五



 安化州都督府寄在朔方縣界。戶四百八十三,口二
 千五十三



 寧朔州都督府寄在朔方縣界。戶三百七十四,口二千二十七



 僕固州都督府寄在朔方縣界。戶一百二十二,口六百七十三。



 靈州大都督府隋靈武郡。武德元年,改為靈州總管府,領回樂、弘靜、懷遠、靈武、鳴沙五縣。二年,以鳴沙縣屬西會州。貞觀四年,於回樂縣置回、環二州,並屬靈武都
 督府。十三年,廢回、環二州,靈州都督入靈、填二州。二十年,鐵勒歸附,於州界置皋蘭、高麗、祁連三州,並屬靈州都督府。永徽元年,廢皋蘭等三州。調露元年,又置魯、麗、塞、含、依、契等六州,總為六胡州。開元初廢,復置東皋蘭、燕然、燕山、雞田、雞鹿、燭龍等六州,並寄靈州界,屬靈州都督府。天寶元年,改靈州為靈武郡。至德元年七月,肅宗即位於靈武,升為大都督府。乾元元年,復為靈州。舊領縣五,戶四千六百四十,口二萬一千四百六十二。天
 寶領縣六,戶一萬一千四百五十六,口五萬三千一百六十三。在京師西北一千二百五十里,至東都二千里。



 回樂隋縣,在郭下。武德四年,分置豐安縣,屬回州。十三年,州廢,並入回東



 鳴沙隋縣。武德二年,置西會州,以縣屬焉。貞觀六年,廢西會州,置環州。九年,廢環州,縣屬靈州。神龍二年,移治廢豐安城



 靈武隋縣



 懷元隋縣。界有隋五原郡。武德元年,改為豐州,領九原縣。六年,州縣俱省入懷遠縣。儀鳳中,再築新城。縣
 有鹽池三所



 保靜隋弘靜縣。神龍元年,改為安靜。至德元年,改為保靜



 溫池神龍元年置



 燕然州寄在回樂縣界,突厥九姓部落所處。戶一百九十,口九百七十八



 雞鹿州寄在回樂縣界,突厥九姓部落所處。戶一百三十二,口五百五十六



 雞田州寄在回樂縣界,突厥九姓部落所處。戶一百四,口四百六十九



 東皋蘭州寄在鳴沙界,九姓所處。戶一千三百四十二,口五千一百八十二



 燕山州在溫池縣界,亦九姓所處。戶四百三十,口二千一百七十六



 燭龍州在溫池界,亦九姓所處。戶一百一十七,口三百五十三。



 鹽州下隋鹽川郡。武德元年,改為鹽州,領五原、興寧二縣。其年,移州及縣寄治靈州。四年,省興寧入五原縣。
 貞觀元年,廢鹽州五原縣入靈州。二年,平梁師都,復於舊城置鹽州及五原、興寧二縣,隸夏州都督府。其年,改為靈州都督府。天寶元年,改為五原郡。乾元元年,改為鹽州。永泰元年十一月,升為都督府。元和八年,隸夏州。舊領縣二,戶九百三十二,口三千九百六十九。天寶,戶二千九百二十九,口一萬六千六百六十五。在京師西北一千一百里,至東都二千一十里。



 五原隋縣。武德元年,寄治靈州。貞觀元年省,二年
 復置



 興寧龍朔三年置。



 豐州下隋文帝置,後廢。貞觀四年,以突厥降附,置豐州都督府,不領縣,唯領蕃戶。十一年廢,地入靈州。二十三年,又改豐州。天寶元年,改為九原郡。乾元元年,復為豐州。領縣二,戶二千八百一十三,口九千六百四十一。在京師北二千二百六里,至東都三千四十四里。



 九原永徽四年置



 永豐隋縣。武德六年省,永徽元年復置。



 會州上隋會寧鎮。武德二年,討平李軌,置西會州。天寶元年,改為會寧郡。乾元元年,復為會州。永泰元年,昇為上州,領縣二,戶四千五百九十四,口二萬六千六百六十二。去京師一千一百里,至東都二千一百里。



 會寧隋涼川縣。武德二年,改為會寧



 烏蘭後周縣,置在會寧關東南四里。天授二年,移於關東北七里。



 宥州調露初,六胡州也。長安四年,並為匡、長二州。神龍三年,置蘭池都督府。仍置六縣以隸之。開元十年,復
 分為魯、麗、契、塞四州。十一年,克定康待賓後,遷其人於河南、江淮之地。十八年,又為匡、長二州。二十六年,自江淮放回胡戶,於此置宥州及延恩、懷德、歸仁三縣。天寶元年,改為寧朔郡。至德二年,又改為懷德郡都督府。乾元元年,復為宥州。寶應後廢。元和九年,復於經略軍置宥州,郭下置延恩縣。十五年,移治長澤縣,為吐蕃所破。長慶四年,夏州節度使李祐復置。領縣三,戶七千八十三,口三萬二千六百五十二。去京師二千一百里,去東
 都三千一百九十里。



 延恩開元二十六年,以廢匡州置,後隨州移徙



 歸仁舊蘭池州之長泉縣。開元二十六年,置歸仁縣



 懷德開元二十六年,以廢塞門縣置。



 勝州下都督府隋置勝州,大業為榆林郡。武德中,平梁師都,復置勝州。天寶元年,復為榆林郡。乾元元年,復為勝州。領縣二,戶四千一百八十七,口二萬九百五十二。去京師一千八百三十里,至東都一千九百五里。



 榆林隋舊



 河濱隋榆林郡地。貞觀三年,置雲州於河濱,因置河濱縣。四年,改為威州。八年廢,河濱屬勝州。



 麟州下天寶元年,王忠嗣奏請割勝州連谷、銀城兩縣置麟州,其年改為新秦郡。乾元元年,復為麟州,領縣三,戶二千四百二十八,口一萬九百三。去京師一千四百四十里,至東都一千九百五里。



 新秦天寶元年,分連谷、銀城二縣地置



 連穀舊
 屬勝州,天寶元年來屬



 銀城舊屬勝州,天寶元年來屬。



 安北大都護府開元十年,分豐、勝二州界置瀚海都護府。總章中,改為安北大都護府。北至陰山七十里,至回紇界七百里。舊領縣一,戶二千六,口七千四百九十八。去京師二千七百里,至東都二千九百里。在黃河之北。



 陰山天寶元年置。



 河南道



 東都周之王城,平王東遷所都也。故城在今苑內東北隅,自赧王已後及東漢、魏文、晉武,皆都於今故洛城。隋大業元年,自故洛城西移十八里置新都,今都城是也。北據邙山,南對伊闕,洛水貫都,有河漢之象。都城南北十五里二百八十步,東西十五里七十步,周圍六十九里三百二十步。都內縱橫各十街,街分一百三坊、二市。每坊縱橫三百步,開東西二門。



 宮城,在都城之西北
 隅。城東西四里一百八十步,南北二里一十五步。宮城有隔城四重。正門曰應天,正殿曰明堂。明堂之西有武成殿,即正衙聽政之所也。宮內別殿、臺、館三十五所。上陽宮,在宮城之西南隅。南臨洛水,西拒谷水,東即宮城,北連禁苑。宮內正門正殿皆東向,正門曰提象,正殿曰觀風。其內別殿、亭、觀九所。上陽之西,隔谷水有西上陽宮,虹梁跨谷,行幸往來。皆高宗龍朔後置。禁苑,在都城之西。東抵宮城,西臨九曲,北背邙阜,南距飛仙。苑城東
 面十七里,南面三十九里,西面五十里,北面二十里。苑內離宮、亭、觀一十四所。



 河南府隋河南郡。武德四年,討平王世充,置洛州總管府,領洛、鄭、熊、穀、嵩、管、伊、汝、魯九州。洛州領河南、洛陽、偃師、鞏、陽城、緱氏、嵩陽、陸渾、伊闕等九縣。其年十一月,罷總管府,置陜東道大行臺。九年,罷行臺,置洛州都督府,領洛、懷、鄭、汝等四州,權於府置尚書省。貞觀元年,割穀州之新安來屬。七年,又割穀州之壽安來屬。八年,移
 治所於河南縣之宣範坊。十八年,廢都督府,省緱氏、嵩陽二縣。顯慶二年,置東都,官員準雍州。是年,廢穀州,以福昌、長水、永寧、澠池等四縣,懷州之河陽、濟源、溫、王屋,鄭州之汜水來屬。龍朔二年,又以許州之陽翟,鄭州之密縣,絳州之垣縣來屬。乾封元年,以垣縣隸絳州。咸亨四年,又置柏崖、大基二縣。其年,省柏崖縣。上元元年,復置緱氏縣。永淳元年,復置嵩陽縣。光宅元年,改東都為神都。垂拱四年,置永昌縣。載初元年,置武臨縣。天授元
 年,置武泰縣,尋廢。仍改鄭州之滎陽、武泰來屬。三年,置來廷縣。神龍元年,改神都復為東都;廢永昌、來廷三縣;改武泰、滎陽還鄭州。先天元年,置伊闕縣。開元元年,改洛州為河南府。二十二年,置河陰縣。天寶元年,改東都為東京也。天寶,領縣二十六,戶十九萬四千七百四十六,口一百一十八萬三千九十三。在西京之東八百五十里。



 河南隋舊。武德四年,權治司隸臺。貞觀元年,移治所
 於大理寺。貞觀二年,徙理金墉城。六年,移治都內之毓德坊。垂拱四年,分河南、洛陽置永昌縣,治於都內之道德坊。永昌元年,改河南為合宮縣。神龍元年,復為河南縣,廢永昌縣。三年,復為合宮縣。景龍元年,復為河南縣



 洛陽隋舊。武德四年,權治大理寺。貞觀元年,徙治金墉城。六年,移治都內之毓德坊。垂拱四年,分置永昌縣。天授三年,又分置來廷縣,治於都內之從善坊。龍朔元年,廢來廷縣。神龍二年十一月,改洛陽為永昌縣。唐
 隆元年七月,復為洛陽



 偃師隋縣



 鞏隋縣



 緱氏隋縣。貞觀十八年省。上元二年七月復置,管孝敬陵,舊縣治西北澗南。上元中,復置治所於通谷北,今治是



 告成隋陽城縣。武德四年,割陽城、嵩陽、陽翟置康城縣,又置嵩州,治陽城。貞觀元年,割陽翟隸許州。三年,省嵩州及康城縣,以陽城、嵩陽屬洛州。登封元年,將有事嵩山,改為告成縣



 登封隋嵩陽縣。貞觀十七年省。永淳元年七月,復置。二年,又廢。光宅元年,又
 置。登封元年十二月,改為登封縣。神龍元年二月,改為嵩陽。二年十一月,復為登封



 陸渾隋縣



 伊闕隋縣



 伊陽先天元年十二月,割陸渾縣置



 壽安隋縣。義寧元年,移治九曲城,屬熊州。貞觀七年,移今治,屬洛州。長安四年,立興泰宮,分置興泰縣。神龍元年廢,並入壽安



 新安隋縣。義寧二年,置新安郡。武德元年,改為穀州,領新安、澠池、東垣三縣。四年,省東垣入新安。貞觀元年,移穀州治澠池,新安移入廢州城,改屬
 洛州。顯慶二年十二月,廢穀州,以福昌、新安、澠池、永寧,並懷州之河陽、濟源、溫、王屋,鄭州氾水,並隸洛州



 福昌隋宜陽縣。義寧二年,置宜陽郡,領宜陽、澠池、永寧三縣;又於新安縣置新安郡,領新安一縣。武德元年,改宜陽郡為熊州,新安為穀州,割熊州之澠池又置東垣縣屬之,仍改熊州之宜陽為福昌縣。三年,割熊州永寧置函州。四年,省東垣縣。八年,廢函州,復以永寧屬熊州。貞觀元年,省熊州,以永寧屬穀州,壽安屬洛州。顯慶二
 年,廢穀州,福昌隸洛州也



 澠池隋舊,治大塢城。貞觀元年,移穀州治所於此,領福昌、澠池、永寧三縣。三年,縣南移於雙橋。其年,穀州又移治雙橋。六年,又移理於福昌。顯慶二年十二月,廢穀州,澠池隸洛州



 長水隋長澤縣。義寧元年,改為長水。武德元年,屬虢州。貞觀元年,屬穀州。顯慶二年,隸洛州



 永寧隋熊耳縣所治。義寧二年,置永寧縣,治永固城,屬宜陽郡。武德元年,改屬熊州。三年,移治同軌城,改屬函州。八年,復屬熊州。
 貞觀元年,改屬穀州。十四年,移於今所。十七年,移治鹿橋。顯慶元年,穀州廢,改隸洛州



 密隋縣。武德三年,置密州。四年廢,縣屬鄭州。龍朔二年,割屬洛州



 河清咸亨四年,分河南、洛陽、新安、王屋、濟源、河陽置大基縣。先天元年,改為河清



 潁陽載初元年,析河南、伊闕、嵩陽三縣置武臨縣。開元十五年,改為潁陽。



 河陽氾水溫河陰已上縣會昌三年割屬孟州,陽翟還許州,濟源還懷州,王屋還懷州。



 孟州上本河南府之河陽縣,本屬懷州。顯慶二年,割屬河南府。以城臨大河,長橋架水,古稱設險。乾元中,史思明再陷洛陽,太尉李光弼以重兵守河陽。及雍王平賊,留觀軍容使魚朝恩守河陽,乃以河南府之河陽、河清、濟源、溫四縣租稅入河陽三城使。河南尹但總領其縣額。尋又以氾水軍賦隸之。會昌三年九月,中書門下奏:「河陽五縣,自艱難已來,割屬河陽三城使。其租賦色役,盡歸河陽,河南尹但總管名額而已,使歸一統,便為定制。
 既是雄鎮,足壯三城,其河陽望昇為孟州,仍為望,河陽等五縣改為望縣。」尋有敕,割河陰隸孟州,河清還河南府。時河陽節度,以懷州為理所。會昌四年,又割澤州隸河陽節度使,仍移治所於孟州,戶口籍帳入河南府。



 河陽隋縣。武德四年,於隋河陽宮置盟州,領河陽、集城、溫三縣。八年,廢盟州,省集城入河陽縣,以河陽、溫屬懷州。顯慶二年,以河陽、溫屬洛州



 氾水隋縣。武德四年,分置成皋縣。貞觀元年,省入氾水,屬鄭州。顯慶二年,
 割屬洛州,仍移治武牢城。垂拱四年,改為廣武。神龍元年,復為氾水。開元二十九年,移治所於武牢。成皋府在縣北



 河陰開元二十年,割氾水、滎澤二縣置,管河陰倉



 溫舊屬懷州。顯慶二年,割屬洛州



 濟源隋舊縣。武德二年,置西濟州,又分置飀陽、蒸川、邵原三縣。四年,廢西濟州及邵原、蒸川、頠陽三縣入濟源,改隸懷州。



 鄭州隋滎陽郡。武德四年,平王世充,置鄭州於武牢,
 領氾水、滎陽、滎澤、成皋、密五縣。其年,又於管城縣置管州,領管城、須水、圃田、清池四縣。貞觀元年,廢管州及須水、清池二縣,以廢管州之陽武、新鄭四縣屬鄭州。七年,自武牢移鄭州理所於管城。舊領縣八,戶一萬八千七百九十三,口九萬三千九百三十七。天寶領縣七,戶七萬六千六百九十四,口三十六萬七千八百八十一。至京師一千一百五里,至東都二百七十里。



 管城郭下,隋舊



 滎陽隋縣。天授二年,分置武泰
 縣,隸洛州,又改滎陽為武泰。萬歲通天元年,復為滎陽,尋又為武泰。神龍復



 滎澤隋舊



 新鄭隋舊



 中牟隋圃田縣。武德元年,改為中牟,屬汴州。龍朔二年,改屬鄭州



 原武隋舊。



 陜州大都督府隋河南郡之陜縣。義寧元年,置弘農郡,領陜、崤、桃林、長水四縣。二年,省崤縣。武德元年,改為陜州總管府,管陜、鼎、熊、函、穀五州,仍割長水屬虢州。其年,復立崤縣。二年,復割崤縣屬函州。三年,又置南韓州、
 嵩州,並屬陜府。四年,東都平,割熊、穀、嵩三州屬洛州總管府。其年,罷洛州總官,復以熊、穀、嵩三州來屬;仍省南韓州入洛州。八年,廢函州,以崤縣來屬。貞觀元年,罷都督府,又以廢芮州芮城、河北二縣來屬。十四年,改崤縣為峽石縣。大足元年,割絳州之夏縣來屬,尋卻還絳州。天寶元年,改為陜郡,置軍。至德二載十月,收兩京。乾元元年,復為陜州,因割蒲州之解、安邑,絳州之夏縣來屬;仍改安邑為虞邑。廣德元年十月,吐蕃犯京師,車駕幸陜州,
 仍以陜為大都督府。天祐初,昭宗遷都洛陽,駐蹕陜州,改為興德府,縣次畿赤。哀帝即位,省,復為大都督府。舊領縣五,戶二萬一千一百七十一,口八萬一千九百一十九。天寶領縣七,戶三萬九百五十,口十七萬二百三十八。在京師東四百九十里,東至東都三百三十里。



 陜郭下。隋縣



 峽石隋崤縣。義寧二年省。武德元年,復置。二年,割屬函州。三年,自石隖移治鴨橋。八年,改屬陜州。十四年,移治峽石隖,因改為峽石縣



 靈寶
 隋桃林縣。天寶元年,以掘得寶符,改為靈寶縣



 芮城隋縣。武德二年,置芮州,領芮城、河北二縣。貞觀元年,罷芮州,以芮城、河北屬陜州



 平陸隋河北縣。義寧元年,置安邑郡,縣屬焉。天寶三載,太守李齊物開三門,石下得戟,大刃,有「平陸」篆字,因改為平陸縣



 安邑隋為虞州,郭下置安邑縣,領安邑、解、夏、桐鄉四縣。貞觀十七年,廢虞州及桐鄉縣以安邑、解縣屬蒲川,夏縣屬絳州。乾元元年,割屬陜州,改安邑為虞邑。大歷四年,復為安
 邑縣



 夏縣舊屬虞州。貞觀十七年,改隸絳州。乾元元年,改屬陜州。



 安邑、夏縣,天寶後,加管戶一萬八千五百。



 虢州望漢弘農郡。隋廢郡為弘農縣,屬陜州。隋末復置郡。義寧元年,改為鳳林郡,仍於盧氏置虢郡。武德元年,改為虢州,改鳳林為鼎州。貞觀八年,廢鼎州,移虢州於今治,屬河南道。開元初,以巡按所便,屬河東道。天寶元年,改為弘農郡。乾元元年,復為虢州,以弘農為緊縣,
 盧氏、硃陽、玉城為望縣。天寶領縣六,戶二萬八千二百四十九,口八萬八千四十五。西至京師四百三十里,東至東都五百五十三里。



 弘農漢縣,隋廢。大業三年,於今湖城縣西一里置,尋隨郡移於弘農川。神龍元年,改「弘」為「恆」。開元十六年,復為弘農,州所治也。



 閿鄉隋縣



 湖城漢湖縣,後加「城」字。乾元元年,改為天平縣。大歷四年,復為湖城。



 硃陽隋縣



 玉城隋縣,分盧氏置。



 盧氏隋縣。



 汝州望隋襄城郡。武德四年,平王世充,改為伊州,領承休、梁、郟城三縣。貞觀元年,以廢魯州魯山縣來屬。其年,省梁縣,仍改承休為梁縣。八年,改伊州為汝州,領梁、郟城、魯山三縣。證聖元年,置武興縣。先天元年,置臨汝縣。開元二十六年,以仙州之葉縣來屬。天寶元年,以許州之襄城來屬,仍改為臨汝郡。乾元元年,復為汝州也。舊領縣三,戶三千八百八十四,口一萬七千五百三十四。天寶領縣七,戶六萬九千三百七十四,口二十七萬
 三千七百五十六。在京師東九百八十二里,至東都一百八十里。



 梁隋承休縣。貞觀元年,改為梁縣



 郟城隋舊縣



 魯山隋舊。武德四年,於縣置魯州,領魯山、LM陽二縣。貞觀元年,州廢,仍置LM陽縣,以魯山縣屬伊州。八年,改伊州為汝州



 葉隋縣。武德四年,置葉州。五年廢,縣屬許州。開元四年,置仙州,領葉、襄城、方城、西平、舞陽五縣。二十六年,廢仙州,以葉屬汝州,襄城、舞陽屬許州,
 方城還唐州,西平屬豫州



 襄城隋舊縣。武德元年,於此置汝州,領襄城、汝墳、期城三縣。貞觀元年,廢汝州及汝墳、期城二縣,以襄城屬許州。開元四年,屬仙州。二十六年,還屬許州。其年,改屬汝州也



 龍興證聖元年,分郟城、魯山置武興縣。神龍元年,改為中興縣。其年,又改為龍興



 臨汝先天元年置。貞元八年,以梁縣西界二鄉益之,兼移縣於石壕驛。



 許州望隋潁川郡。武德四年,平王世充,改為許州,領
 長社、長葛、許昌、繁昌、黃臺、水隱強、臨潁七縣。貞觀元年,廢黃臺、繁昌、水隱強三縣,以洧州之扶溝、鄢陵,汝州之襄城,嵩州之陽翟,北灃之葉縣來屬。十三年,改置都督府,管許、唐、陳、潁四州,而許州領長社、長葛、許昌、鄢陵、扶溝、臨潁、襄城、陽翟、葉九縣。十六年,罷都督府。顯慶二年,割陽翟屬洛州。開元四年,割葉、襄城置仙州。二十六年,仙州廢,以葉、襄城、陽翟來屬。其年,又以葉、襄城屬汝州。二十八年,又以襄城來屬。是歲,又以葉屬汝州。天寶元年,改為
 潁川郡。乾元元年,復為許州。長慶三年,廢溵州為郾城縣,屬許州。舊領縣九,戶一萬五千七百一十五,口七萬二千二百二十九。天寶領縣七,戶七萬三千二百四十七,口四十八萬七千八百六十四。在京師東一千二百里,至東都四百里。



 長社郭下。隋潁川縣。武德四年,改為長社,取舊名



 長葛隋分許昌縣置,取舊名



 許昌舊縣



 鄢陵隋置洧州,後廢為縣,屬許州



 扶溝隋縣。武德四年,置北
 陳州。其年,州廢,縣屬洧州。九年,洧州廢,來屬



 臨潁隋舊縣。建中二年,隸溵州。貞元元年,州廢來屬



 舞陽漢縣,治所在古城內,屬仙州。開元二十六年,隸許州。元和十三年,移治於吳城鎮



 郾城本屬豫州。長慶元年來屬。



 汴州上隋滎陽郡之浚儀縣也。武德四年,平王世充,置汴州總管府,管汴、洧、杞、陳四州,領浚儀、新里、小黃、開封、封丘等五縣。七年,改為都督府。廢開封、小黃、新里三縣入
 浚儀,復以廢杞州之雍丘、陳留,管州之中牟,洧州之尉氏來屬。龍朔二年,以中牟隸鄭州。延和元年,復置開封縣。天寶元年,改汴州為陳留郡。乾元元年,復為汴州。建中二年,築其羅城。舊領縣五:浚儀、雍丘、陳留、中牟、尉氏,戶五萬七千七百一,口八萬二千八百七十九。天寶領縣六,戶十萬九千八百七十六,口五十七萬七千五百七。在京師東一千三百五十里,東都四百一里。



 浚儀古縣,隋置,在今縣北三十里,為李密所陷。縣人
 王要漢率豪族置縣於汴州之內,要漢自為縣令。義寧元年,於縣復置汴州,以要漢為刺史。武德四年,移縣於州北羅城內。貞觀元年,移於州西一里,延和元年六月,割浚儀十四鄉分置開封縣



 開封漢縣,在今縣南五十里。貞觀元年省,並入浚儀。延和元年六年,析浚儀復置,並在郭下



 尉氏隋縣,屬潁川郡。武德四年,於縣置洧州,領尉氏、扶溝、康陰、新汲、鄢陵、宛陵、歸化七縣。貞觀元年,廢洧州及康陰、宛陵、新汲、歸化四縣,以扶溝、
 鄢陵屬許州,尉氏屬汴州



 陳留隋縣,屬汴州。武德四年,屬杞州。貞觀元年,廢杞州,陳留屬汴州



 封丘隋縣



 雍丘隋縣。武德四年,於縣置杞州,領雍丘、陳留、圉城、襄邑、外黃、濟陽六縣,權於州內以倉院置。貞觀元年,廢杞州及濟陽、圍城、外黃三縣,以襄邑屬宋州,陳留、雍丘屬汴州,而移縣入廢杞州。



 蔡州上隋汝南郡。武德四年四月,平王世充,置豫州總管府,管豫、道、輿、息、舒五州。豫州領安陽、平輿、真陽、
 吳房、上蔡五縣。七年,改為都督府,廢輿、道、舒、息四州。貞觀元年,罷都督府,廢平輿、新蔡二縣,復以道州之郾城,息州之新息,朗州之朗山,舒州之褒信、新蔡五縣來屬。天授三年,又置平輿、西平兩縣。開元四年,以西平屬仙州。二十六年,省仙州,復以西平來屬。天寶元年,改為汝南郡。乾元元年,復為豫州。寶應元年,改為蔡州。舊領縣十,戶一萬二千一百八十二,口六萬四百一十五。天寶領縣十一,戶八萬七百六十一,口四十六萬二百五。去
 京師一千五百四十里,至東都六百七十里。



 汝陽隋舊縣。治郭下



 朗山漢安昌縣,隋改為朗山



 遂平隋吳房縣。元和十二年,討吳元濟於文城柵,置行吳房縣,權隸溵州。賊平,改為遂平縣,隸唐州。長慶元年,復隸蔡州



 郾城隋舊。武德四年,於此置道州,領郾城、邵陵北武、西平四縣。貞觀元年,廢道州及北武、邵陵、西平三縣,以郾城屬豫州。本治溵水南。開元一十年,因大水,移治溵水北。元和十二年,於縣置溵州。長
 慶元年,廢溵州,以郾城隸許州



 上蔡隋縣



 新蔡隋舊。武德四年,於此置舒州,領新蔡、褒信二縣。貞觀元年,廢舒州,新蔡屬豫州



 褒信後漢縣



 新息隋縣。武德四年,於縣置息州,領新息、淮川、長陵三縣。貞觀元年,廢息州及淮川、長陵二縣,以新息屬豫州



 平輿隋置。貞觀元年廢,天授二年復置



 西平漢縣。貞觀元年廢。天授二年復置。元和十二年,隸溵州。州廢,隸蔡州



 真陽漢慎陽縣,隋為真陽。載初元年,改為
 淮陽。神龍元年復。



 滑州望,隋東郡。武德元年,改為滑州,以城有古滑臺也。二年,陷賊。及平王世充,復置,領白馬、衛南、韋城、匡城、靈昌、長垣七縣。八年,廢長垣縣入匡城,以廢東梁州之酸棗縣來屬。天寶元年,改為靈昌郡。乾元元年,復為滑州。舊領縣七,戶一萬三千七百三十八,口六萬四千九百六十。天寶,戶七萬一千九百八十三,口四十二萬二千七百九十。去京師一千四百四十里,至東都五百三
 十里。



 白馬郭下。漢縣。衛南隋楚丘縣。後以曹有楚丘,乃改為衛南縣,治古楚丘城。儀鳳元年,移治西北濱河之新城。永昌元年,又移於楚丘之城南



 韋城隋分白馬縣置於古城韋氏之國城



 匡城漢長垣縣,隋改為匡城



 胙城漢南燕縣,隋改為胙城,隸滑州



 酸棗漢縣



 靈昌隋分酸棗縣置。靈昌者,河津之名。



 陳州上,隋淮陽郡。武德元年,討平房憲伯,改為陳州,領宛丘、箕城、扶樂、太康、新平五縣。貞觀元年,廢扶樂、箕城、新平三縣,復以沈州之項城、溵水二縣來屬。長壽元年,置武城縣。證聖元年,置光武縣。天寶元年,改陳州為淮陽郡。乾元元年,復為陳州。舊領縣四,戶六千三百六十七,口三萬九百六十一。天寶領縣六,戶六萬六千四百四十二,口四十萬二千四百八十六。在京師一千五百二十里,至東都七百一十七里。



 宛丘郭下。隋縣



 太康漢陽夏縣,隋改太康,以縣東有太康城



 項城隋舊。武德四年,於此置沈州,領項城、潁東、鮦陽、南頓、溵水五縣。貞觀元年,廢沈州,以縣屬陳州



 溵水漢汝陽縣。改為溵水。建中二年,隸溵州。興元元年,廢溵州,縣隸陳州



 南頓隋縣。武德六年,省入項城。證聖元年,割項城置光武縣,以縣有光武廟故也。景雲元年,改為南頓,復古名也



 西華漢縣。武德元年,改為箕城縣。貞觀元年,省入宛丘。長壽元年,割宛丘置武城縣,
 以縣本楚武王所築故也。神龍元年,復為箕城。景雲元年,改為西華,復古名也。



 亳州望隋譙郡。武德四年,平王世充,改為亳州,領譙、城父、穀陽、鹿邑、酂五縣。五年,置總管府,管譙、亳、宋、北荊、潁、沈六州。七年,改為都督府。貞觀元年,罷都督府,亳州不改。十七年,廢譙州,以臨渙、永城、山桑三縣來屬。天寶元年,改為譙郡。乾元元年,復為亳州也。舊領縣八,戶五千七百九十,口三萬三千一百七十七。天寶,戶八萬
 八千九百六十,口六十七萬五千一百二十一。至京師一千七百里,至東都八百九十八里。



 譙郭下。貞觀十七年,自古譙城移入州城置



 酂漢縣。隋屬沛郡。武德四年,改屬亳州。開元二十六年,移於汴城垣陽驛置



 城父隋舊



 鹿邑隋舊



 真源漢苦縣。隋為穀陽。乾封元年,改為真源。載初元年,改為仙源。神龍元年,復為真源。有老子祠



 臨渙隋置譙州,領縣四。貞觀十七年省,以臨渙、永城、山桑屬亳
 州,蘄縣屬徐州。縣本治銍城,十七年移治所於廢譙州。元和九年,割屬宿州



 永城隋縣,屬譙州。貞觀十七年廢,屬亳州。舊治於馬浦城東北三里。武德五年,移置於馬浦城



 蒙城隋山桑縣,屬譙州。州廢,隸亳州。天寶元年,改為蒙城。



 潁州中漢汝南郡。隋為汝陰郡。武德四年,平王世充,於汝陰縣西北十里置信州,領汝陰、清丘、永安、高唐、永東等六縣。六年,改為潁州,移於今治,省高唐、永樂、永安
 三縣。貞觀元年,省清丘縣。八年,又以廢渦州之下蔡縣來NP。天寶元年,改為汝陰郡。乾元元年,復為潁州。長慶二年,以潁州隸滑鄭節度使。舊領縣三,戶二千九百五,口一萬四千一百八十五。天寶領縣四,戶三萬七百七,口二十萬二千八百九十。至京師一千八百二十里,至東都九百六十里。



 汝陰郭下。漢縣



 潁上隋置治所於古鄭城。武德四年,移於今治



 下蔡隋舊。武德四年,於縣置渦州,
 下蔡隸之。八年,州廢,縣屬潁州也



 沈丘古曰寢丘,至隋不改。神龍二年,改為沈丘。



 宋州望隋之梁郡。武德四年,平王世充,置宋州,領宋城、寧陵、柘城、穀熟、下邑、碭山、虞城七縣。其年,以虞城屬東虞州。五年,廢東虞州,仍以虞城來屬。貞觀元年,廢杞州,以襄邑縣來屬,仍省柘城縣。十七年,以廢戴州之單父、楚丘來屬。永淳元年,又置柘城縣。天寶元年,改宋州為睢陽郡。乾元元年,復為宋州。舊領縣七,戶一
 萬一千三百三,口六萬一千七百二十。天寶領縣十,戶一十二萬四千二百六十八,口八十九萬七千四十一。去京師一千五百四十里,至東都七百八十里。



 宋城郭下。治古睢陽城。漢睢陽縣,隋改為宋城



 襄邑隋置。武德二年,屬杞州。貞觀元年,屬宋州



 寧陵漢縣,久廢。隋特置。貞觀元年,並柘城縣入



 虞城隋分下邑縣置。武德四年,屬宋州。其年,於縣置東虞州。五年,州廢,縣屬宋州。



 碭山舊安陽縣,隋改為碭山,
 屬宋州



 下邑漢縣



 穀熟漢縣。武德二年,於縣置南穀州。四年,州廢,縣屬宋州



 單父古邑。隋於縣置戴州,大業廢。武德五年,復置戴州。貞觀十七年,戴州廢,縣屬宋州



 楚丘治古巳氏城,屬戴州。貞觀十七年,屬宋州



 柘城秦縣,久廢。隋復置。貞觀初廢。永淳元年,析穀熟、寧陵復置。



 曹州上隋濟陰郡。武德四年,改為曹州,領濟陰、定陶、冤句、離狐、乘氏,並置蒙澤、普陽等七縣。其年,省普陽縣。
 五年,以廢梁州之考城來屬。貞觀元年,省定陶、蒙澤二縣入濟陰。十七年,以廢載州之成武來屬。天寶元年,改曹州為濟陰郡。乾元元年,復為曹州。舊領縣五,戶九千二百四十四,口五萬四千九百八十一。天寶領縣六,戶十萬三百五十二,口七十一萬六千八百四十八。在京師東北一千四百五十三里,至東都東北六百五十七里。



 濟陰郭下。隋縣



 考城隋舊。武德四年,於縣置梁州,領考城縣。五年,州廢,以縣屬曹州



 冤句漢縣。武
 德四年,分縣西界置濟陽縣,屬杞州。貞觀元年,廢濟陽,並入冤句



 乘氏漢縣,春秋之重丘地也



 南華漢離狐縣,累代不改。天寶元年,改為南華



 成武漢縣。隋屬戴州。州廢,屬曹州。



 濮州上隋東平郡之鄄城縣也。武德四年,置濮州,領鄄城、廩城、雷澤、臨濮、昆吾、濮陽、永定、安丘、長城九縣。五年,廢安丘、長城二縣。八年,廢昆吾、永定、廩城三縣。貞觀八年,割濟州之範縣來屬。天寶元年,改為濮陽郡。乾元元
 年,復為濮州。舊領縣五,戶八千六百二十八,口四萬四千一百三十五。天寶,戶五萬七千七百八十一,口四十萬六百四十八。在京師東北一千五百七十里,至東都七百三十五里。



 鄄城古縣。後漢於縣置兗州。武德四年,分置永定縣。八年,並入鄄城。



 濮陽隋舊。武德四年,分置昆吾縣。八年省,並入濮陽



 範漢縣。武德二年,置範州,治昆吾城。五年,州廢,縣屬濟州。貞觀八年,改屬濮州。



 雷澤
 漢縣。武德四年,分置廩城縣。貞觀八年,省入雷澤



 臨濮武德四年,分雷澤置。五年,省長城縣並入。



 鄆州上隋東平郡之須昌縣。武德四年,平徐圓朗,於鄆城置鄆州,領鄆城、須昌、宿城、鉅野、乘丘五縣。又以廢壽州之壽張來屬。其年,置總管府,管鄆、濮、兗、戴、曹五州。貞觀元年,罷都督府,仍以鉅野屬戴州。又廢宿城、乘丘二縣。八年,自鄆城移治須昌。景龍元年,又置宿城縣。天寶元年,改鄆州為東平郡。乾元元年,復為鄆州。舊領縣
 三:須昌、鄆城、壽張;戶四千一百四十一,口二萬一千六百九十二。天寶領縣五,戶四萬四千二百九十九,口二十八萬四千五百三十。天寶十三載,廢濟州,其所管五縣,並入鄆州。濟州舊領縣五,戶六千九百五,口三萬四千五百一十。天寶,領戶三萬八千七百四十九,口二十一萬六千九百七十九,並入鄆州。在京師東北一千六百九十七里,去東都東北九百七十三里。今領縣十。



 壽張隋縣。武德四年,於縣置壽州,領壽張、壽良二縣。
 五年,廢壽州,省壽良入壽張,屬鄆州



 鄆城漢壽良縣。隋改為萬安縣,仍於縣置鄆州,尋改萬安為鄆城。貞觀八年,移鄆州治所於須昌縣



 鉅野漢縣。隋縣升為州。尋廢,屬戴州。貞觀十七年,載州廢,鉅野來屬



 須昌郭下。漢縣,故城在今鄆州東南三十二里。隋於故城置宿城縣,仍置須昌縣於今所。貞觀八年,州自鄆城移於須昌縣。後廢宿城縣。景雲三年十二月,復分須昌置宿城縣。貞元四年,改宿城為東平縣,移就郭下。大和四
 年,改為天平縣。六年七月,廢天平縣入須昌縣



 盧縣漢舊。隋置濟北郡。武德四年,改濟州,領盧、平陰、長清、東阿、陽谷、範六縣。又置昌城、濟北、穀城、孝感、冀丘、美政六縣。六年,廢美政、孝感、穀城、冀丘、昌城五縣。八年,割範縣屬濮州。貞觀元年,又廢濟北縣入長清。天寶元年,改為濟陽郡。乾元元年,復為濟州。十三載六月一日,廢濟州,盧、長清、平陰、東阿、陽谷等五縣並入鄆州



 平陰漢肥城縣。隋為平陰,屬濟州。天寶十三載,州廢,縣屬鄆
 州。大和六年,並入東阿縣。開成二年七月,節度使王源中奏置平陰縣



 東阿漢縣。隋屬濟州。州廢,屬鄆州



 陽谷隋置,取縣界陽谷臺為名,屬濟州。州廢,屬鄆州



 中都漢平陸縣,本治殷密城,在今治西三十九里。天寶元年,改為中都,移於今治。



 泗州中,隋下邳郡。武德四年,置泗州,領宿預、徐城、淮陽三縣。貞觀元年,省淮陽縣入宿預,以廢邳州之下邳,廢連州之漣水來屬。八年,又以廢仁州之虹縣來屬。總
 章元年,割海州沐陽來屬。咸亨五年,沐陽還海州。長安四年,置臨淮縣。開元二十三年,自宿預移治所於臨淮。天寶元年,改為臨淮郡。乾元元年,復為泗州,舊領縣五,戶二千二百五十,口二萬六千九百二十。領宿豫、漣水、徐城、虹、下邳。天寶領縣六,戶三萬七千五百二十六,口二十萬五千九百五十九。今領縣三:臨淮、漣水、徐城。其虹縣割隸宿州,宿預、下邳隸徐州。



 臨淮、長安四年,割徐城南界兩鄉於沙熟淮口置臨
 淮縣。開元二十三年,移治郭下



 漣水隋縣。武德四年,置漣州,仍分置金城縣。貞觀元年,廢漣州,並省金城縣,以縣屬泗州。總章元年,改為楚州。咸亨五年,還屬泗州



 徐城漢徐縣。隋為徐城縣,屬泗州,治於大徐城。開元二十五年,移就臨淮縣。



 海州中隋東海郡。武德四年,置海州總管府,領海、漣、環、東楚四州。海州領朐山、龍沮、新樂、曲陽、沐陽、厚丘、懷仁、利城、祝其九縣。六年,改新樂為祝其。七年,以東楚州屬
 揚府,又以沂州來屬。八年,廢環州及龍沮、祝其、曲陽、厚丘、利城六縣,仍以廢環州之東海來屬。九年,廢漣州。貞觀元年,罷都督府。天寶元年,以海州為東海郡。乾元元年,復為海州。舊領縣四:朐山、東海、沐陽、懷仁,戶八千九百九十九,口四萬三千六百九十三。天寶,戶二萬八千五百四十九,口十八萬四千九。在京師東二千五百七十里,至東都一千七百五十四里。



 朐山郭下。漢朐縣,後加「山」字



 東海漢贛榆縣。武
 德四年,置環州,領東海、青山、石城、贛榆四縣。八年,廢環州,仍廢青山等三縣入東海縣,隸海州。縣治鬱州,四面環海



 沐陽漢厚丘縣。後魏改沐陽



 懷仁後魏置。



 兗州上都督府隋魯郡。武德五年,平徐圓朗,置兗州,領任城、瑕丘、平陸、龔丘、曲阜、鄒、泗水七縣。貞觀元年,省曲阜縣。其年,又省東泰州,以博城縣來屬。八年,復置曲阜縣。十四年,置都督府,管兗、泰、沂三州。十七年,以廢戴州
 之金鄉、方輿來屬。長安四年,置萊蕪縣。天寶元年,改兗州為魯郡。乾元元年,復為兗州。舊領縣八,戶九千三百六十六,口一萬五千四百二十八。天寶領縣十一,戶八萬八千九百八十七,口五十八萬六百八。中都割屬鄆州。在京師東一千八百四十三里,去東都一千七十里。



 瑕丘郭下。宋置兗州於魯瑕邑故治,隋因置瑕丘縣



 曲阜隋縣。貞觀元年省,八年復置



 乾封隋博城縣。武德五年,於縣置東泰州,領博城、梁父、嬴、肥
 城、岱六縣,貞觀元年,罷東泰州,省梁父、嬴二縣入博城。仍以博城屬兗州,兼省肥城。乾封元年,高宗封泰山,改為乾封縣。總章元年,復為博城。神龍元年,又為乾封



 泗水漢卞縣。隋分汶陽縣於卞縣古城置泗水縣



 鄒古邾國,魯穆公改為鄒



 任城漢縣。北齊於縣置高平郡。隋廢,縣屬兗州



 龔丘北齊平原縣,隋改為龔丘



 金鄉後漢縣。武德四年,於縣置金州,領方輿、金鄉二縣。五年,改金州為戴州。貞觀十七年,州廢,以金
 鄉、方輿屬兗州,以單父、楚丘屬宋州,成武屬曹州,鉅野屬鄆州



 魚臺漢方輿縣。隋屬戴州。貞觀十七年,戴州廢,縣入兗州。寶應元年,改為魚臺,以城北有魯公觀魚臺



 萊蕪漢縣,晉廢。後魏於古城置嬴縣。貞觀初,廢入博城縣。長安四年,於廢嬴縣置萊蕪縣。元和十五年,並入乾封縣,尋卻置,屬兗州。



 徐州上隋彭城郡。武德四年,平王世充,置徐州總管府,管徐、邳、泗、鄫、沂、仁六州。徐州領彭城、蕭、沛、豐、滕、符離、
 諸陽七縣。貞觀元年,廢諸陽縣入符離。二年,省鄫、邳二州,仍以譙州來屬。七年,以沂州屬海州都督。八年,廢仁州入譙州。其徐州都督,管徐、泗、譙三州。十七年,罷都督府。以廢譙州之蘄縣來屬。天寶元年,改徐州為彭城郡。乾元元年,復為徐州。舊領縣六,戶八千一百六十二,口四萬五千五百三十七。天寶領縣七,戶六萬五千一百七十,口四十七萬八千六百七十六。在京師東二千六百里,至東都一千二百五十七里。



 彭城漢彭城郡治也



 蕭漢縣。隋為龍城縣,尋改為蕭



 豐漢縣。北齊置永昌郡,尋省為豐縣



 沛漢縣,隋廢。武德復置



 滕縣古滕國,隋置縣



 宿遷晉宿預縣,元魏於縣置徐州。州移彭城縣,隸泗州,寶應元年,以犯代宗諱,改「預」為「遷」,仍隸徐州



 下邳漢下邳郡。元魏置東徐州,周改邳州,隋廢。武德四年,復邳州,領下邳、郯、良城三縣。貞觀元年,廢邳州,仍省郯、良城二縣,以下邳屬泗州。元和中,復屬徐州。



 宿州上徐州之符離縣也。元和四年正月敕,以徐州之符離置宿州,仍割徐州之蘄、泗州之虹。九年,又割亳州之臨渙等三縣屬宿州。大和三年,徐泗觀察使崔群奏罷宿州,四縣各歸本屬。至七年敕,宜準元和四年正月敕,復置宿州於埇橋,在徐之南界汴水上,當舟車之要。其舊割四縣,仍舊來躭。州新置,元和已來,未計戶口。



 符離漢縣。隋治朝解城。貞觀元年,移治竹邑城。元和四年正月,置宿州,仍為上州



 虹漢縣。隋曰夏丘縣,
 武德四年,屬仁州。其年,分置虹縣於古虹城,屬仁州。六年,廢夏丘縣。貞觀八年,廢仁州,以虹縣屬泗州,移治夏丘故城。元和四年,割屬宿州



 蘄漢縣。後魏加「城」,曰蘄城縣。隋去「城」字,屬北譙州。貞觀十七年,廢譙州,屬徐州。舊治穀城,顯慶元年,移於今所。元和四年,割屬宿州也



 臨渙隋舊。屬譙州。州廢,隸亳州。大和元年,割屬宿州。



 沂州中漢東海郡之瑯邪縣。武德四年,平徐圓朗,置
 沂州,領費、臨沂、顓臾三縣。又置蘭山、臨沐、昌樂三縣。六年,省蘭山、臨沐、昌樂三縣入臨沂。貞觀元年,省顓臾入費縣。其年,省鄫州,以承縣來屬。八年,又省莒州,以新泰、沂水二縣來屬。天寶元年,改為瑯邪郡。乾元元年,復為沂州。舊領縣五,戶四千六百五十二,口二萬三千九百。天寶,戶三萬三千五百一十,口十九萬五千七百三十七。在京師東二千二百五十四里,至東都一千四百三十里。



 臨沂漢縣,州所治。後魏置郯郡,又改為北徐州,並在
 此縣。後周置沂州



 承漢縣,隋蘭陵縣。武德四年,置鄫州,以蘭陵隸之,仍改為承縣,別置蘭陵、鄫城二縣,屬鄫州。貞觀元年,鄫州與二縣俱廢,以承縣來屬沂州。



 費漢縣。春秋時費國。



 新泰漢東新泰縣,晉去「東」字。武德五年,屬莒州。貞觀八年,莒州廢,縣屬沂州。



 沂水漢東莞縣。隋改為東安縣,尋改為沂水。武德五年,於縣置莒州,領沂水、新泰、莒三縣。貞觀八年,省莒州,縣屬密州,沂水、新泰屬沂州。



 密州中隋高密郡。武德五年,改為密州,領諸城、安丘、高密三縣。貞觀八年,省莒州,以莒來屬。天寶元年,改為高密郡。乾元元年,復為密州。舊領縣四,戶三千五百八十,口二萬八千五百九十三。天寶,戶二萬八千二百九十二,口十四萬六千五百二十四。在京師東南二千五百三十里,至東都東一千八百六十九里。



 諸城州所治,本漢東武縣城也。隋移入廢高密郡城,因改為諸城



 輔唐漢安丘縣,屬北海郡。乾元二年,
 刺史殷仲卿奏請治於故昌安城,因改為輔唐



 高密、漢縣。隋末大亂,廢之。武德三年,於義城堡置高密縣。六年,並高密、膠西兩縣,移就故夷安城。城,舊高密縣也。仍廢膠西縣



 莒漢縣,屬東海郡。武德五年,於縣置莒州。州廢,以縣屬密州。



 齊州上漢濟南郡,隋為齊郡。武德元年,改為齊州,領歷城、山茌、祝阿、源陽、臨邑五縣。二年,置總管府,管齊、鄒、東泰、譚、淄、濟六州。貞觀元年,廢都督府及譚州,省源陽
 縣。又以廢譚州之平陵、臨濟、亭山、章丘四縣來屬。七年,又置都督府,管齊、青、淄、萊、密五州。天寶元年,改為臨淄郡。五載,為濟南郡。乾元元年,復為齊州。舊領縣八,戶一萬一千五百九十三,口六萬一千七百七十一。天寶,戶六萬二千四百八十五,口三十六萬五千九百七十二。在京師東北二千六十九里,至東都東北一千二百四十四里。今管縣六,並三縣也。



 歷城漢縣,屬濟南郡。舊志有平陵縣。貞觀十七年,齊王祐起兵,平陵人不從
 順,遂改為全節。元和十年正月,以戶口凋殘,並全節入歷城縣



 章丘漢陽丘縣。隋為章丘。武德二年,於平陵縣置譚州,領平陵、亭山、章丘、營城四縣。八年,廢營城入平陵,又以廢鄒州之臨濟來屬。貞觀元年,廢譚州為平陵縣,屬齊州,章丘亦來屬



 亭山隋縣。元和十五年,以戶口凋殘,並入章丘縣,因廢亭山



 臨邑漢縣。武德元年,屬譚州。州廢來屬



 長清隋置,屬濟州。貞觀十七年,屬齊州。舊志有豐齊縣,古山茌邑也。天寶元
 年改為豐齊。元和十五年,以戶口凋殘,並入長清縣



 禹城漢祝阿縣。天寶元年,以為禹城,以縣西有禹息故城



 臨濟漢之菅縣。隋為朝陽縣,尋改為臨濟縣。武德元年,於縣置鄒州,領臨濟、蒲臺、高苑、長山、鄒平五縣。八年,廢鄒州,縣屬譚州。州廢,屬齊州。



 青州上隋北海郡。武德四年,置青州總管府,管青、濰、登、牟、莒、密、萊、乘八州。青州領益都、臨朐、臨淄、般陽、樂安、時水、安平等七縣。八年,省乘、濰、牟、登四州,以廢濰州之
 北海,廢乘州之千乘、壽光、博昌來屬,省般陽、樂安、時水、安平四縣。貞觀元年,罷都督府。天寶元年,改青州為北海郡。乾元元年,復為青州。舊領縣七,戶一萬六百五十八,口五萬六千三百一十七。天寶,戶七萬三千一百四十八,口四十萬二千七百四。在京師東北二千二百五十里,至東都一千五百七里。



 益都漢縣。在今壽光縣南十里故益都城是也。北齊移入青州城北門外為治所。



 臨淄漢縣,治古齊國
 城。久廢,隋復置。



 博昌漢縣,治故郡城。樂安,隋縣。武德二年,屬乘州。州廢,屬青州。總章二年,移治於今所



 壽光漢縣。隋移治所於博昌縣。初屬乘州,州廢來屬。



 千乘漢千乘國,後漢改為樂安郡。宋、齊廢,隋置千乘縣。武德二年,於縣置乘州,領千乘、博昌、壽光、新河五縣。六年,廢新河縣。八年,乘州廢,千乘等縣隸青州。



 臨朐漢縣。隋為逢山縣,尋復為臨朐,屬北海郡。



 北海漢平壽縣。隋置北海郡。開皇三年罷郡,置下密
 縣於廢郡城。大業二年,改為北海縣。武德二年,於縣置濰州,領北海、連水、平壽、華池、城都、下密、東陽、寒水、訾亭、濰水、汶陽、膠東、營丘、華宛、昌安、都昌、城平等十七縣。六年,唯留北海、營丘、下密三縣,餘十四縣並廢。八年,廢濰州,仍省營丘、下密二縣,以北海屬青州。



 淄州上隋齊郡之淄川縣。武德四年,置淄州,領淄川、長白、萊蕪三縣。六年,廢長白、萊蕪二縣。八年,又以廢鄒州之長山、高苑、蒲臺三縣來屬。天寶元年,復為淄川郡。
 乾元元年,復為淄州。景龍元年,分高苑置濟陽縣,又並高苑。又割蒲臺隸之,後割屬棣州。舊領縣五,戶六千三百二十三,口三萬四千四百二十五。天寶,戶四萬二千七百三十七,口二十萬三千八百二十一。在京師東北二千一百三十三里,東都東北一千四百二十五里。今管縣四,並濟陽入高苑。



 淄川郭下。漢般陽縣。武德初,屬淄州



 長山漢於陵縣。武德初,屬鄒州。州廢,屬淄州



 高苑隋置。初屬
 鄒州,州廢來屬。景龍元年,分置濟陽縣。元和十五年,並入高苑



 鄒平漢縣。北齊為平原縣。隋移治漢鄒平故城,因改為鄒平。初屬譚州,州廢來屬。



 棣州上後漢樂安郡。隋渤海郡之厭次縣。武德四年,置棣州,領陽信、樂陵、滳河、厭次四縣,治陽信。六年,並入滄州。貞觀十七年,復置棣州於樂陵縣,領厭次、滳河、陽信三縣,又割淄州之蒲臺隸焉。而樂陵屬滄州。天寶元年,改為樂安郡。上元元年,復為棣州。領縣五,戶三萬九
 千一百五十,口二十三萬八千一百五十九。在京師東北二千二百一十里,東都東北一千三百七十里。



 厭次郭下。漢富平縣。隋屬滄州。武德四年,改屬棣州。六年,省棣州,復隸滄州。貞觀十七年,復置棣州,厭次還屬



 滳河隋縣



 陽信漢縣,屬渤海郡。貞觀十七年,改屬棣州



 蒲臺漢漯沃縣。隸淄州。割屬棣州



 渤海垂拱四年,析蒲臺、厭次置。



 萊州中漢東萊郡,隋因之。武德四年,討平綦順,置萊
 州,領掖、膠水、即墨、盧鄉、昌陽、曲城、當利、曲臺、膠東九縣。六年,廢曲城、當利、曲臺、膠東四縣。貞觀元年,廢盧鄉,割登州之文登、廢牟州之黃來屬。麟德元年,置牟平縣。如意元年,割黃縣、文登、牟平置登州。天寶元年,改萊州為東萊郡。乾元元年,復為萊州。舊領縣六:掖、黃、文登、昌陽、即墨、膠水,戶一萬一千五百六十八,口六萬三千三百九十六。天寶領縣四,戶二萬六千九百九十八,口七萬一千五百。在京師東北二千五百九十九里,去東都
 一千八百五十二里。



 掖州治漢東萊郡也。隋置掖縣,屬萊州



 昌陽漢縣,置於古昌陽城。永徽元年,移古縣西北二十三里



 膠水漢膠東國地。隋置縣於古光州,因改名膠水



 即墨漢不其邑也。隋置即墨縣。



 登州漢東萊郡之黃縣。如意元年,分置登州,領文登、牟平、黃三縣,以牟平為治所。神龍三年,改黃縣為蓬萊縣,移州治於蓬萊。天寶元年,以登州為東牟郡。乾元元
 年,復為登州。天寶領縣四,戶二萬二百九十八,口一十萬八千九百。在京師東三千一百五十里,至東都二千七十一里。



 蓬萊漢黃縣,屬萊州。如意元年,於縣置登州。神龍三年,改為蓬萊,移於今所



 牟平麟德二年,分文登置,屬萊州。如意元年,置登州,治牟平。神龍三年,移治所於蓬萊縣



 文登隋舊縣。武德四年,置登州,領文登、觀陽二縣。六年,以觀陽屬牟州,又置清陽、廓定二縣,屬登
 州。貞觀元年,登州及清陽、廓定二縣並廢,地入文登縣



 黃漢舊縣。神龍三年,改為蓬萊縣,屬登州,以為州治。先天元年,又割蓬萊置黃縣。



\end{pinyinscope}