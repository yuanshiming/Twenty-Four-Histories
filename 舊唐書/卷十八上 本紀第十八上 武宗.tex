\article{卷十八上 本紀第十八上 武宗}

\begin{pinyinscope}

 武宗至道昭肅孝皇帝諱炎,穆宗第五子,母曰宣懿皇
 后韋氏。元和九年六月十
 二日生於東宮。長慶元年三月,封潁王,本名瀍。開成中加開府儀同三司、檢校吏部尚書,依百官例,逐月給俸料。初,文宗追悔莊恪太子殂不由道,乃以敬宗子陳王成美為皇太子,開成四年冬十月宣制,未遑冊禮。五年正月二日,文宗暴疾,宰相李玨、知樞密劉弘逸奉密旨,以皇太子監國。兩軍中尉仇士良、魚弘志矯詔迎潁王於十六宅,曰:「朕自嬰疾疹,有加無瘳,懼不能躬總萬機,日厘庶政。稽於謨訓,謀及大臣,用建親賢,以貳神器。親弟潁王瀍昔在籓邸,與朕常同師訓,動成儀矩,性稟寬仁。俾奉昌圖,必諧人欲。可立為皇太弟,應軍國政事,便令權勾當。百闢卿士,中外庶臣,宜竭乃心,輔成予志。陳王成美先立為皇太子,
 以其年尚沖幼,未漸師資,比日重難,不遑冊命,回踐硃邸,式協至公,可復封陳王。」是夜,士良統兵士於十六宅迎太弟赴少陽院,百官謁見於東宮思賢殿。三日,仇士良收捕仙韶院副使尉遲璋殺之,屠其家。四日,文宗崩,宣遺詔:皇太弟宜於柩前即皇帝位,宰相楊嗣復攝塚
 宰。十四日,受冊於正殿,時年二十七。陳王成美、安王溶殂於邸第。初,楊賢妃有寵於文宗,而莊恪太子母王妃失寵怨望,為楊妃所譖,王妃死,太子廢。及開成未年,帝多疾無嗣,賢妃請以安王溶嗣,帝謀於宰臣李玨,玨非之,乃立陳王。至是,仇士良立武宗,欲歸功於己,乃發安王舊事,故二王與賢妃皆死。



 二月,制穆宗妃韋氏追謚宣懿皇太后,帝之母也。
 上御正殿,降德音,以開府、右軍中尉仇士良封楚國公,左軍中尉魚弘志為韓國公,太常卿崔鄲、戶部尚書判度支崔珙並本官同中書門下平章
 事。敕二月十五日玄元皇帝降生日宜為降聖節,休假一日。



 三月,詔宮人劉氏、王氏並為妃。敕朔望入閣對刑法官,是日非便,宜停。



 五月,中書奏:六月十二日,皇帝載誕之辰,請以其日為慶陽節。祔宣懿太后於太廟。初,武宗欲啟穆宗陵祔葬,中書門下奏曰:「園陵已安,神道貴靜。光陵二十餘載,福陵則近又修崇。竊惟孝思,足彰嚴奉。今若再因合祔,須啟二陵,或慮聖靈不安,未合先旨。又以陰陽避忌,亦有所疑。不移福陵,實協典禮。」乃止。就
 舊墳增築,名曰福陵。又奏:「準今年二月八日赦文,應京諸司勒留官,令本處克留手力雜給與攝官者。臣等檢詳,諸道正官料錢絕少,雜給手力即多,今正官勒留,亦管公事,料錢少於雜給,刻下事未得中。臣等商量,其正官料錢雜給等錢,望每貫留二百文與攝官,餘並如舊。」從之。



 秋七月,制檢校禮部尚書、華州刺史陳夷行復為中書侍郎、同平章事。



 八月十七日,葬文宗皇帝於章陵。知樞密劉弘逸、薛季稜率禁軍護靈駕至陵所,二人
 素為文宗獎過,仇士良惡之,心不自安,因是掌兵,欲倒戈誅士良、弘志。鹵簿使兵部尚書王起、山陵使崔稜覺其謀,先諭鹵簿諸軍。是日弘逸、季稜伏誅。門下侍郎、同平章事楊嗣復檢校吏部尚書、潭州刺史,充湖南都團練觀察使;中書侍郎、同平章事李玨檢校兵部尚書、桂州刺史,充桂管防禦觀察等使;御史中裴夷真為杭州刺史:皆坐弘逸、季稜黨也。易定軍亂,逐節度使陳君賞。君賞鳩合豪傑數百人,復入城,盡誅謀亂兵士,軍城
 復安。



 九月,以淮南節度使、檢校尚書左僕射李德裕為吏部尚書、同中書門下平章事,尋兼門下侍郎;以宣武軍節度使、檢校吏部尚書、汴州刺史李紳代德裕鎮淮南。帝在籓時,頗好道術修攝之事,是秋。召道士越歸真等八十一人入禁中,於三殿修金籙道場。帝幸三殿,於九天壇親受法籙。右拾遺王哲上疏,言王業之初,不宜崇信過當,疏奏不省。



 十一月,鹽鐵轉運使奏江淮已南請復稅茶,從之。魏博節度使何進滔卒,三軍推其子重
 霸知留後事。



 會昌元年正月壬寅朔。庚戌,有事於郊廟,禮畢,御丹鳳樓,大赦,改元。



 二月壬寅,以淮南節度使、檢校吏部尚書李紳為中書侍郎、同平章事。中書奏:「南宮六曹皆有職分,各責官業,即事不因循。近者戶部度支多是諸軍奏請,本司郎吏束手閑居。今後請祗令本行分判,委中書門下簡擇公幹才器相當者轉授。」從之。車駕幸昆明池。賜仇士良紀功碑,詔右僕射李程為其文。



 三月,貶湖南
 觀察使楊嗣復潮州司馬,桂管觀察使李玨端州司馬,杭州刺史裴夷直驩州司戶。宰臣李德裕進位司空。三月壬申,宰相李德裕、陳夷行、崔珙、李紳等奏:「憲宗皇帝有恢復中興之功,請為百代不遷之廟。」帝曰:「所論至當。」續議之,事竟不行。贈故中書令、晉國公裴度太師。山南東道蝗害稼。造靈符應聖院於龍首池。四月辛丑,敕:「《憲宗實錄》舊本未備,宜令史官重修進內。其舊本不得注破,候撰成同進。」時李德裕先請不遷憲宗廟,為議者
 沮之,復恐或書其父不善之事,故復請改撰實錄,朝野非之。



 五月辛未,中書門下奏:「據《六典》,隋置諫議大夫七人,從四品上。大歷二年,升門下侍郎為正三品,兩省遂闕四品。建官之道,有所未周。詩云『袞職有闕,仲山甫補之』。周、漢大臣,願入禁闥,補過拾遺。張衡為侍郎,常居帷幄,從容諷諫。此皆大臣之任,故其秩峻,其任重,則敬其言而行其道。況蹇諤之地,宜老成之人,秩未優崇,則難用耆德。其諫議大夫望依隋氏舊制,升為從四品,分為左
 右,以備兩省四品之闕。向後與丞出入迭用,以重其選。又御史中丞為大夫之貳,緣大夫秩崇,官不常置,中丞為憲臺之長。今寺監、少卿、少監、司業、少尹並為寺署之貳,皆為四品。中丞官名至重,見秩未崇,望升為從四品。」從之。



 六月,有禿鶖鳥集於禁苑。庚子夜五更,小流星五十餘旁午流散。制以魏博兵馬留後何重霸檢校工部尚書、魏州大都督府長史,充天雄軍節度使,仍賜名重順。中書奏請依姚璹故事,宰相每月修時政記送史
 館,從之。以衡山道士劉玄靖為銀青光祿大夫,充崇玄館學士,賜號廣成先生,令與道士趙歸真於禁中修法籙。左補闕劉彥謨上疏切諫,貶彥謨為河南府戶曹。敕:「自前中外上封論事,有所糾舉,則請留中。今後並云『請付御史臺』,不得云『留中不下』。如事關軍國,理須宥密,不在此限。如臺司勘當後,若得事實,必獎奉公。茍涉加誣,必當反問。告示中外,明知此意。」七月己巳,北方有流星,經天良久。關東大蝗傷稼。襄、郢、江左大水。彗復出室壁
 之間。



 八月,回鶻烏介可汗遣使告難,言本國為黠戛斯所攻,故可汗死,今部人推為可汗。緣本國破散,今奉太和公主南投大國。時烏介至塞上,大首領嗢沒斯與赤心宰相相攻,殺赤心,率其部下數千帳近西城。天德防禦使田牟以聞。烏介又令其相頡乾迦斯上表,借天德城以安公主,仍乞糧儲牛羊供給。詔金吾大將軍王會、宗正少卿李師偃往其牙宣慰,令放公主入朝,賑粟二萬石。



 九月,幽州軍亂,逐其帥史元忠,推牙將陳行泰為
 留後。三軍上章請符節,朝旨未許。十月,幽州雄武軍使張絳遣軍吏吳仲舒入朝,言行泰慘虐,不可處將帥之任,請以鎮軍加討,許之。十月,誅行泰,遂以絳知兵馬使。車駕校獵咸陽。



 十一月丁酉朔。壬寅夜,大星東北流,其光燭地,有聲如雷,山崩石隕。其彗起於室,凡五十六日而滅。太和公主遣使入朝,言烏介自稱可汗,乞行策命,緣初至漠南,乞降使宣慰,從之。十二月,中書門下奏修實錄體例:「舊錄有載禁中之言。伏以君上與宰臣、公卿
 言事,皆須眾所聞見,方可書於史冊。且禁中之語,在外何知,或得之傳聞,多涉於浮妄,便形史筆,實累鴻猷。今後實錄中如有此色,並請刊削。又宰臣與公卿論事,行與不行,須有明據。或奏請允愜,必見褒稱;或所論乖僻,因有懲責。在籓鎮上表,必有批答,居要官啟事者,自有著明,並須昭然在人耳目。或取手舍存於堂案,或與奪形於詔敕,前代史書所載奏義,罔不由此。近見實錄多載密疏,言不彰於朝聽,事不顯於當時,得自其家,未足為
 信。今後實錄所載章奏,並須朝廷共知者,方得紀述,密疏並請不載。如此則理必可法,人皆向公,愛憎之志不行,褒貶之言必信。」從之。李德裕奏改修《憲宗實錄》所載吉甫不善之述,鄭亞希旨削之。德裕更此條奏,以掩其跡。搢紳謗議,武宗頗知之。



 二年春正月丙申朔,以撫王紘為開府儀同三司、幽州大都督府長史,充幽州盧龍節度大使。以雄武軍使張絳檢校左散騎常侍,兼幽州左司馬,知兩使留後,仍賜
 名仲武。中書奏百官議九宮壇本大祠,請降為中祠。宰相崔珙、陳夷行奏定左右僕射上事儀注。



 二月丙寅,中書奏:「準元和七年敕,河東、鳳翔、鄜坊、邠寧等道州縣官,令戶部加給課料錢歲六萬二千五百貫。吏部出得平留官數百員,時以為當。自後戶部支給零碎不時,觀察使乃別將破用,徒有加給,不及官人,所以選人憚遠,不樂注受。伏望令部都與實物,及時支遣。諸道委觀察判官知給受,專判此案,隨月支給,年終計帳申戶部。又
 赴選官人多京債,到任填還,致其貪求,罔不由此。今年三銓,於前件州府得官者,許連狀相保,戶部各借兩月加給料錢,至支時折下。所冀初官到任,不帶息債,衣食稍足,可責清廉。」從之。太子太師致仕蕭俯卒。牂柯、南詔蠻遣使入朝。



 三月,遣使冊回紇烏介可汗。以振武麟勝節度使、銀青光祿大夫、檢校尚書右僕射、單于大都護、兼御史大夫、彭城郡開國公、食邑二千戶劉沔可檢校右僕射,兼太原尹、北京留守。充河東節度、管內觀察處置
 等使,代苻澈。時回紇在天德,命沔以太原之師討之。四月乙丑朔,光祿大夫、守司空、兼門下侍郎、平章事李德裕,銀青光祿大夫、守右僕射、門下侍郎、平章事崔珙,銀青光祿大夫、中書侍郎、同平章事李紳,金紫光祿大夫、檢校司徒、兼太子太保牛僧孺等上章,請加尊號曰仁聖文武至神大孝皇帝。戊寅,御宣政殿受冊。是月九日雨,至十四日轉甚,乃改用二十三日。時有纖人告中尉仇士良,言宰相作赦書,欲減削禁軍衣糧馬草料。士良
 怒曰:「必若有此,軍人須至樓前作鬧。」宰相李德裕等知之,請開延英訴其事。帝曰:「奸人之詞也。」召兩軍中尉諭之曰:「赦書出自朕意,不由宰相,況未施行,公等安得此言?」士良惶恐謝之。是日晴霽。中書奏:「元日御含元殿,百官就列,唯宰相及兩省官皆未開扇前立於欄檻之內,及扇開,便侍立於御前。三朝大慶,萬邦稱賀,唯宰相侍臣同介胄武夫,竟不拜至尊而退,酌於禮意,事未得中。臣等請御殿日昧爽,宰相、兩省官鬥班於香案前,俟扇
 開,通事贊兩省官再拜,拜訖,升殿侍立。」從之。天德奏,回紇族帳侵擾部內。敕:「勸課種桑,比有敕命,如能增數,每歲申聞。比知並無遵行,恣加翦伐,列於鄽市,賣作薪蒸。自今州縣甩由,切宜禁斷。」五月,敕慶陽節百官率醵外,別賜錢三百貫,以備素食合宴,仍令京兆府供帳,不用追集坊市樂人。天德軍使田牟奏:回紇大將嗢沒斯與多覽將軍將吏二千六百人請降,遣中人齎詔慰勞之。宰相李德裕兼守司徒。太子太師致仕鄭覃卒。



 六月甲子
 朔,火星犯木。丙寅,太白犯東井。回紇降將嗢沒斯將吏二千六百餘人至京師。制以嗢沒斯檢校工部尚書,充歸義軍使,封懷化郡王,仍賜姓名曰李思忠;以回紇宰相受耶勿為歸義軍副使、檢校右散騎常侍,賜姓名曰李弘順。七月,嵐州人田滿川據郡叛,劉沔誅之。



 八月,回紇烏介可汗過天德,至杷頭烽北,俘掠雲、朔北川,詔劉沔出師守雁門諸關。回紇首領屈武降幽州,授左武衛將軍同正。詔以回紇犯邊,漸浸內地,或攻或守,於理何
 安?令少師牛僧孺、陳夷行與公卿集議可否以聞。僧孺曰:「今百僚議狀,以固守關防,伺其可擊則用兵。」宰相李德裕議:「以回匕所恃者嗢沒、赤心耳,今已離叛,其強弱之勢可見。戎人獷悍,不顧成敗,以失二將,乘忿入侵,出師急擊,破之必矣。守險示弱,虜無由退。擊之為便。」天子以為然。乃徵發許、蔡、汴、滑等六鎮之師,以太原節度使劉沔為回紇南面招討使;以張仲武為幽州盧龍節度使、檢校工部尚書,封蘭陵郡王,充回紇東面招討使;以
 李思忠為河西黨項都將,回紇西南面招討使:皆會軍於太原。制以皇子峴為益王,岐為兗王,皇長女為昌樂公主,第二女為壽春公主,第三女永寧公主。上御麟德殿,見室韋首領督熱論等十五人。太原奏回紇移帳近南四十里,索叛將嗢沒斯,日昨至橫水俘虜,兼公主上表言食盡,乞賜牛羊事。賜烏介詔曰:



 朕自臨寰區,為人父母,唯好生為德,不願黷武為名。故自彼國不幸為黠戛斯所破,來投邊境,已歷歲年,撫納之間,無此不到。初
 則念其饑歉,給以糧儲;旋則知其破傷,盡還馬價。前後遣使勞問,交馳道途。小小侵擾,亦盡不計。今可汗尚此近塞,未議還蕃。朝廷大臣,四方節鎮,皆懷疑忿,盡請興師,雖朕切務含弘,亦所未諭。一昨數使回來。皆言可汗只待馬價,及令付之次,又聞所止屢遷,或侵掠雲、朔等州,或劫奪羌、渾諸部,未知此意,終欲如何?若以未交馬價,須近塞垣,行止之間,亦宜先告邊將。豈有倏來忽往,遷徙不常。雖雲隨逐水草,動皆逼近城柵。遙揣深意,似
 恃姻好之情;每睹蹤由,實為弛突之計。況到橫水柵下,殺戮至多。蕃、渾牛羊,豈吝馳掠;黎庶何罪,皆被傷夷。所以中朝大臣皆云:「回紇近塞,已是違盟;更戮邊人,實背大義。」咸願因此翦逐,以雪殂謝之冤。然朕志在懷柔,情深屈己,寧可汗之負德,終未忍於幸災。石戒直久在京城,備知人實憤惋,發於誠懇。固請自行。嘉其深見事機,不能違阻。可汗審自問遂,速擇良圖,無至不悛,以貽後悔。



 詔太原起室韋沙陀三部落、吐渾諸部,委石雄為前
 鋒。易定兵千人守大同軍,契苾通、何清朝領沙陀、吐渾六千騎趨天德,李思忠率回紇、黨項之師屯保大柵。十月,吐蕃贊普卒,遣使論普熱入朝告哀,詔將作少監李璟入蕃吊祭。帝幸涇陽,校獵白鹿原。諫議大夫高少逸、鄭朗等於閣內論:「陛下校獵太頻,出城稍遠,萬機廢馳,星出夜歸,方今用兵,且宜停止。」上優勞之。諫官出,謂宰相曰:「諫官甚要,朕時聞其言,庶幾減過。」



 三年春正月,以宿師於野,罷元會。敕新授銀州刺史、本
 州押蕃落、銀川監牧使何清朝可檢校太子賓客、左龍武大將軍,令分領沙陀、吐渾、黨項之眾赴振武,取劉沔處分。



 二月,先詔百官之家不得於京城置私廟者,其皇城南向六坊不得置,其閑僻坊曲即許依舊置。太原劉沔奏:「昨率諸道之師至大同軍,遣石雄襲回鶻牙帳,雄大敗回鶻於殺胡山,烏介可汗被創而走。已迎得太和公主至雲州。」是日,御宣政殿,百僚稱賀。制曰:



 夫天之所廢,難施繼絕之恩;人之所棄,當用侮亡之道。朕每思前
 訓,豈忘格言。回鶻比者自恃兵強,久為桀驁,凌虐諸部,結怨近鄰。黠戛斯潛師彗掃,穹居瓦解,種族盡膏於原野,區落遂至於荊榛。今可汗逃走失國,竊號自立,遠逾沙漠,寄命邊陲。朕念其衰殘,尋加賑颻。每陳章表,多詐諛之詞;接我使臣,如全盛之日。無傷禽哀鳴之意,有因獸猶鬥之心。去歲潛入朔川,大掠牛馬;今春掩襲振武,逼近城池。可汗皆自率兵,首為寇盜,不恥破敗,莫顧姻親。河東節度使劉沔料敵伐謀,乘機制勝,發胡貉之騎
 以為前鋒,搴翎侯之旗伐彼在穴。短兵鏖於帳下,元惡抶於彀中。況乘匪六飛,眾才一旅,儲備已竭,計日可擒。太和公主居處不同,情義久絕。懷土多思,亟聞黃鵠之歌;失位自傷,寧免《綠衣》之嘆。念其羈苦,常軫朕心。今已脫於豺狼,再見宮闕,上以攄宗廟之宿憤,次以慰太皇太后之深慈,永言歸寧,良用欣感。其回紇既以破滅,義在翦除,宜令諸道兵馬使同進討。河東立功將士已下,優厚賞給,續條疏處分。應在京外宅及東都修功德回
 紇,並勒冠帶,各配諸道收管。其回紇及摩尼寺莊宅、錢物等,並委功德使與御史臺及京兆府各差官點檢收抽,不得容諸色人影占。如犯者並處極法,錢物納官。摩尼寺僧委中書門下條疏聞奏。



 以麟州刺史、天德行營副使石雄為銀青光祿大夫、檢校左散騎常侍、豐州刺史、御史大夫,充豐州西城中城都防禦、本管押蕃落等使。劉沔檢校尚書左僕射,張仲武檢校尚書右僕射,餘並如故。黠戛斯使注吾合素入朝,獻名馬二匹,言可汗
 已破回鶻,迎得太和公主歸國,差人送公主入朝,愁回鶻殘眾奪之於路。帝遂遣中使送注吾合素往太原迎公主。時烏介可汗中箭,走投黑車子,詔黠戛斯出兵攻之。



 三月,太和公主至京師,百官班於章敬寺迎謁,仍令所司告憲宗、穆宗二室。四月,昭義節度使劉從諫卒,三軍以從諫侄稹為兵馬留後,上表請授節鉞。尋遣使齎詔潞府,令稹護從諫之喪歸洛陽。稹拒朝旨。詔中書門下兩省尚書御史臺四品已上、武官三品已上,會議劉
 稹可誅可宥之狀以聞。



 五月,敕諸道節度使置隨身不過六十人察使不得過四十人,經略、都護不得過三十人。築望仙觀於禁中。宰臣百僚進議狀:「以昆戎未殄,塞上用兵,不宜中原生事,潞府請以親王遙領,令稹權知兵馬事,以俟邊上罷兵。」獨李德裕以為澤潞內地,前時從諫許襲,已是失斷,自後跋扈難制,規脅朝廷。以稹豎子,不可復踐前車,討之必殄。武宗性雄俊,曰:「吾與德裕同之,保無後悔。」自是諫官上疏言不可用兵相繼。



 六月,西內神龍寺災。左軍中尉楚國公仇士良卒。



 秋七月戊子,宰相奏:「秋色已至,將議進軍,幽州須早平回鶻,鎮、魏須速誅劉稹,各須遣使諭旨,兼偵三鎮軍情。今日延英面奉聖旨,欲遣張賈充使。臣等續更商量,張賈幹濟有才,甚諳軍中體勢,然性剛負氣,慮不安和,不如且命李回。若以臺綱闕人,即兵部侍郎鄭涯久為征鎮判官,情甚精敏,雖無詞辯,言事分明,官重事閑,最似相稱。」上曰:「不如令李回去。」即遣回奉使三鎮。



 八月壬戌,火星
 自七月蒼赤色,動搖井中,至是月十六日犯輿鬼。萬年縣東市火。黠戛斯使諦德伊斯難珠入朝。以右僕射、平章事陳夷行檢校司空,兼河中尹、御史大夫,充河中節度、晉絳慈隰觀察等使。



 九月,制:



 定天下者,致風俗於大同;安生人者,齊法度於畫一。雖晉之欒、趙,家有舊勛;漢之韓、黥,身為佐命。至於干亂紀律,罔不梟夷,禁暴除殘,古今大義。



 故昭義節度劉悟,頃居海岱,嘗列爪牙。屬師道阻兵,王師問罪,三面開綱,一境離心,乘此危機,遂能
 歸命。憲宗嘉其誠款,授以南燕;穆宗待以腹心,委之上黨。招致死士,固護一方,迨於未年,已虧臣節。劉從諫生稟戾氣,幼習亂風。因跋扈之資,以專封上;恃紀綱之力,以襲兵符。暫展執圭之儀,終無上綬之請。隙駒為喻,魏豹姑務於絕河;井蛙自居,孫述頗聞於恃險。誘受戶命,妄作妖言,中罔朝廷,潛圖左道。接壤戎帥,屢奏陰謀,顧髫齔之所矜,豈淵魚之是察。洎乎沈痼,曾靡哀鳴,猶駐將盡之魂,恣行邪僻之志,罔或奮拔,自樹狡童。中使授
 醫,莫睹其朝服;近臣銜命,不入於壘門。逆節甚明,人神共棄。其贈官及先所授官爵、並劉稹在身官爵,宜並削奪。成德軍節度使王元逵、魏博節度使何弘敬,或姻連王室,或任重籓維,懇陳一至之誠,願揚九伐之命。吳漢任職,受詔而初無辦嚴;卜式樸忠,未戰而義形於色。況成德軍嘗以梟騎橫陳,首破硃滔。戰氣方酣,再回魯陽之日;鼓音不息,三周不注之山。魏博軍頃以大旆涉河,竟殲師道。建十二郡之旗鼓,以列降人;削六十年之歷
 階,盡歸皇化。士傳餘勇,軍有雄名,必能稟酂侯之指縱,成葛亮之心伐。咨爾二帥,朕所注懷,元逵可本官充北面招討澤潞使,弘敬充東面招討澤潞使。



 曩者列祖在籓,先天啟聖。符瑞昭晰,彩繪煥於泗亭;鑾輅巡游,金石刻於代邸。實謂可封之俗,久為仁壽之鄉。寇難以來,頗著誠節,必非同惡,咸許自新。其昭義舊將士及百姓等,如保初心,並赦而不問。如能手舍逆效順,以州郡兵眾歸降者,必厚加封賞。如能擒送劉稹者,別授土地,以報勛
 庸。頃隨劉悟鄆州舊將校子孫,既有義心,宜思改悔。如能感喻劉稹,束身歸朝,必當待之如初,特與洗雪。爾等舊校,亦並酬勞。仍委夷行、沔、王茂元各進兵同力攻討。其諸道進軍,並不得焚燒廬舍,發掘墳墓,擒執百姓以為俘囚。桑麻田苗,各許本戶為主。罪止元惡,務拯生靈。



 於戲!蕃維大臣,抗疏於外;髦俊舊老,昌言於朝。戒朕以祖宗之法,不可私一族;弄賞之柄,新以正萬邦。宜用甲兵,陳於原野。雖朕以恩不聽,而群臣以義固爭,詢自
 僉謀,諒非獲已。布告中外,明體朕懷。



 仍以徐泗節度使李彥佐為澤潞西南面招討使。河陽節度使王茂元以本軍屯萬善。彥佐制下後逾月未出師,朝廷疑其持重,乃以天德軍石雄為彥佐之副。劉稹牙將李丕降,用為忻州刺史。以陳許節度使王宰充澤潞南面招討使。河陽節度使王茂元卒,贈司徒。王宰代茂元總萬善之師。十月,宰相監修國史李紳、兵部郎中史館修撰判館事鄭亞進重修《憲宗實錄》四十卷,頒賜有差。晉絳行營副
 招討石雄奏收賊砦五。以河東節度使劉沔檢校司空,兼滑州刺史、御史大夫,充義成軍節度、鄭滑濮觀察等使。以荊南節度使、檢校右僕射、同平章事李石可檢校司空、平章事,兼太原尹、北都留守,充河東節度、管內觀察等使。



 十一月,敕:「中外官員,過為繁冗,量宜減省,以便軍民。宜令吏部條疏合減員數以聞。」十二月,王宰奏收天井關。榆社行營都將王逢奏兵少,乞濟師,詔太原軍二千人赴之。初,劉沔破回鶻,留三千人戍橫水,至是,李
 石以太原無兵,抽橫水戍卒一千五百人以赴王逢。是月二十八日,橫水軍至太原,請出軍優給。舊例第一軍絹二疋,時劉沔交代後,軍庫無絹。石以己絹益之。方可人給一疋,便催上路。軍人以歲將除,欲候過歲,期既速,軍情不悅。都頭楊弁乘士卒流怨,激之為亂。



 四年春正月乙酉朔,以澤潞用兵,罷元會。其日,楊弁逐太原節度使李石。敕:「齋月斷屠,出於釋氏,國家創業,猶近梁、隋、卿相大臣,或沿茲弊。鼓刀者既獲厚利,糾察者
 潛受請求。正月以萬物生植之初,宜斷三日。列聖忌斷一日。仍準開元二十二年敕,三元日各斷三日,餘月不禁。」壬子,河東監軍使呂義忠收復太原,生擒楊弁,盡斬其亂卒,百僚稱賀。



 二月甲寅朔。丁巳,制河中晉、絳、慈、隰等州節度觀察等使、中散大夫、檢校左散騎常侍、河中尹、御史大夫、上柱國、博陵縣開國男、食邑三百戶崔元式可檢校禮部尚書,兼太原尹、北都留守,充河東節度觀察等使。戊午夜,太白犯鎮星。辛酉,太原送楊弁與其
 同惡五十四人來獻,斬於狗脊嶺。



 三月,以晉絳副招討石雄為澤潞西面招討,以汾州刺史李丕為副。以道士趙歸真為左右街道門教授先生。時帝志學神仙,師歸真。歸真乘寵,每對,排毀釋氏,言非中國之教,蠹耗生靈,盡宜除去,帝頗信之。四月,王宰進軍攻澤州。



 五月,以司農卿薛元賞為京兆尹。



 六月,金紫光祿大夫、尚書右僕射、中書侍郎、同平事、判度支崔珙貶澧州刺史。癸丑,敕:「諫官論事,所見不同,連狀署名,事同糾率。此後凡論
 公事,各隨己見,不得連署姓名。如有大政奏論,即可連署。」制追削故左軍中尉仇士良先授官及贈官,其家財並籍沒。士良死後,中人於其家得兵仗數千件,兼發士良宿罪故也。敕責授官銀青光祿大夫、澧州刺史、上柱國、安平郡開國公、食邑二千永崔珙再貶恩州司馬員外置,以珙領鹽鐵時欠宋滑院鹽鐵九十萬貫。帝令度支、鹽鐵、轉運合為一使。七月,以淮南節度使、檢校司空杜忭守尚書右僕射、兼門下侍郎、同平章事,仍判度支,
 充鹽鐵轉運等使。又制銀青光祿大夫、守尚書右僕射、兼門下侍郎、同平章事、監修國史、上柱國、趙郡開國公、食邑二千戶李紳可檢校司空、平章事、揚州大都督府長史、淮南節度副大使、知節度事。吏部條奏中外合減官員一千一百一十四員。王元逵奏邢州刺史裴問、別將高元武以城降。洺州刺史王釗、磁州刺史安玉以城降何弘敬。山東三州平。潞州大將郭誼、張谷、陳揚廷遣人至王宰軍,請殺稹以自贖。王宰以聞,乃詔石雄率軍
 七千入潞州,誼斬劉稹首以迎雄,澤、潞等五州平。



 八月戊戌,王宰傳稹首與大將郭誼等一百五十人,露布獻於京,上御安福門受俘,百僚樓前稱賀。以魏博節度使、檢校尚書右僕射、同平間事何弘敬進封廬江郡開國公,食邑二千戶;以成德軍節度使王元逵檢校司空、兼太子太師、同平章事,進封太原郡開國公,食邑二千戶。宰相李德裕守太尉,進封衛國公,加食邑一千戶。以兵部侍郎、翰林學士承旨崔鉉為中書侍郎、同平章事。
 河東節度使陳夷行卒。



 九月,以天德軍使、晉絳行營招討使石雄檢校兵部尚書、河中嚴、兼御史大夫、河中晉絳慈隰等州節度使。以前山南東道節度使盧鈞檢校尚書左僕射、潞州大都督府長史,充昭義軍節度使、澤潞邢洺觀察等使。以忠武軍節度、陳許蔡等州觀察處置等使、河陽行營諸軍招討使、金紫光祿大夫、檢校尚書右僕射、兼御史大夫、上柱國、太原郡開國公、食邑二千戶王宰檢校司空、太原尹、北都留守,充河東節度、管
 內觀察處置等使。制曰:「逆賊郭誼等,狐鼠之妖,依丘穴而作固;牛羊之力,得水草而逾兇。久從叛臣,皆負逆氣。劉從諫背德反義,掩賊藏奸,積其怙亂之謀,無非親吏之計。劉公直、安全慶等各憑地地險,屢抗王師,每肆悖言,罔懷革面。況郭誼、王協聞邢、洺歸款,懼義旅覆巢,賣孽童以圖全,據堅城而請命。昔伍被詣吏,不免就誅;延岑出降,終亦夷族。致之大闢,無所愧懷。」郭誼、劉公直、王協、安全慶、李道德、李佐堯、劉稹、稹母阿裴、稹弟曹九滿郎
 君郎、妹四娘五娘、從兄洪卿漢卿周卿魯卿匡堯、張穀男涯、解愁、陳揚廷弟宣、男醜奴、張溢男歡郎三寶、門客甄戈、伎術人郭諗蔣黨、李訓兄仲京、王涯侄孫羽、韓約男茂章茂寶、王璠男圭等,並處斬於獨柳。敕以河陽三城鎮遏使為孟州,割澤州隸焉,與懷、孟、澤為節度,號河陽。制以皇子愕為開府儀同三司、夏州刺史、朔方軍節度大使。時黨項叛,命親王以制之。十月,車駕幸鄠縣。



 十一月,幸雲陽。十二月,敕:「郊禮日近,獄囚數多,案款已成,
 多有翻覆。其兩京天下州府見系囚,已結正及兩度翻案伏款者,並令先事結斷訖申。」時左僕射王起頻年知貢舉,每貢院考試訖,上榜後,更呈宰相取可否。後人數不多,宰相延英論言:「主司試藝,不合取宰相與奪。比來貢舉艱難,放入絕少,恐非弘訪之道。」帝曰:「貢院不會我意。不放子弟,即太過,無論子弟、寒門,但取實藝耳。」李德裕對曰:「鄭肅、封敖有好子弟,不敢應舉。」帝曰:「我比聞楊虞卿弟朋比貴勢,妨平人道路。昨楊知至、鄭樸之徒,
 並令落下,抑其太甚耳。」德裕曰:「臣無名第,不合言進士之非。然臣祖天寶末以仕進無他伎,勉強隨計,一舉登第。自後不於私家置《文選》,蓋惡其祖尚浮華,不根藝實。然朝廷顯官,須是公卿子弟。何者?自小便習舉業,自熟朝廷間事,臺閣儀範,班行準則,不教而自成。寒士縱有出人之才,登第之後,始得一班一級,固不能熟習也。則子弟成名,不可輕矣。」



 五年春正月己酉朔,敕造望仙臺於南郊壇。時道士趙
 歸真特承恩禮,諫官上疏,論之延英。帝謂宰臣曰:「諫官論趙歸真,此意要卿等知。朕宮中無事,屏去聲技,但要此人道話耳。」李德裕對曰:「臣不敢言前代得失,只緣歸真於敬宗朝出入宮掖,以此人情不願陛下復親近之。」帝曰:「我爾時已識此道人,不知名歸真,只呼趙鏈師。在敬宗時亦無甚過。我與之言,滌煩爾。至於軍國政事,唯卿等與次對官論,何須問道士。非直一歸真,百歸真亦不能相惑。」歸真自以涉物論,遂舉羅浮道士鄧元起有
 長年之術,帝遣中使迎之。由是與衡山道士劉玄靖及歸真膠固,排毀釋氏,而拆寺之請行焉。宰臣李德裕社忭李讓夷崔鉉、太常卿孫簡等率文武百僚上徽號曰仁聖文武章天成功神德明道皇帝。辛亥,有事於郊廟,禮畢,御承天門,大赦天下。庚申,義安太后崩,敬宗之母也。遺令皇帝三日聽政,十三日小祥,二十五日大祥,二十七日釋服。兵部尚書歸融奏:「事貴得中,禮從順變,配祔之禮,宜有等差。請服期,以日易月,十二日釋服。內
 外臣僚,亦請以其日釋服。陵園制度,請無降殺。」從之。以前太原節度使、檢校司空李石以本官充東都留守。



 二月戊寅朔,太白掩昴之北側。諫議大夫、權知禮部貢舉陳商選士三十人中第,物論以為請托,令翰林學士白敏中覆試,落張瀆、李玗、薛忱、張覿崔凜、王諶、劉伯芻等七人。



 三月,崔鉉罷知政事,出為陜虢觀察使。以御史中丞、兼兵部侍郎李回本官同平章事。



 夏四月,皇第四女封延慶公主,第五女封靖樂公主。敕祠部檢括天下
 寺及僧尼人數。大凡寺四千六百,蘭若四萬,僧尼二十六萬五百。宰相杜忭罷知政事。以戶部侍郎、判戶部崔元式同平章事。



 六月丙子,敕:「漢、魏已來,朝廷大政,必下公卿詳議,博求理道,以盡群情。所以政必有經,人皆向道。此後事關禮法,群情有疑者,令本司申尚書都省,下禮官參議。如是刑獄,亦先令法官詳議,然後申刑部參覆。如郎官、御史有能駁難,或據經史故事,議論精當,即擢授遷改以獎之。如言涉浮華,都無經據,不在申聞。」神
 策奏修望仙樓及廊舍五百三十九間功畢。



 秋七月庚子,敕並省天下佛寺。中書門下條疏聞奏:「據令式,諸上州國忌日官吏行香於寺,其上州望各留寺一所,有列聖尊容,便令移於寺內;其下州寺並廢。其上都、東都兩街請留十寺,寺僧十人。」敕曰:「上州合留寺,工作精妙者留之;如破落,亦宜廢毀。其合行香日,官吏宜於道觀。其上都、下都每街留寺兩所,寺留僧三十人。上都左街留慈恩、薦福,右街留西明、莊嚴。」中書又奏:「天下廢寺,銅像、
 鐘磬委鹽鐵使鑄錢,其鐵像委本州鑄為農器,金、銀、鍮石等像銷付度支。衣冠士庶之家所有金、銀、銅、鐵之像,敕出後限一月納官,如違,委鹽鐵使依禁銅法處分。其土、木、石等像合留寺內依舊。」又奏:「僧尼不合隸祠部,請隸鴻臚寺。如外國人,送還本處收管。」八月,制:



 朕聞三代已前,未嘗言佛,漢魏之後,像教浸興。是由季時,傳此異俗,因緣染習,蔓衍滋多。以
 至於蠹耗國風而漸不覺。誘惑人意,而眾益迷。洎於九州山原,兩京關,僧徒日廣,佛寺日崇。勞人力於土木之功,奪人利於金寶之飾,遺君親於師資之際,違配偶於戒律之間。壞法害人,無逾此道。且一夫不田,有受其饑者;一婦不蠶,有受其寒者。今天下僧尼,不可勝數,皆待農而食,待蠶而衣。寺宇招提,莫知紀極,皆云構藻飾,僭擬宮居。晉、宋、齊、梁,物力凋瘵,風俗澆詐,莫不由是而致也。況我高祖、太宗,以武定禍亂,以文理華夏,執此二
 柄,足以經邦,豈可以區區西方之教,與我抗衡哉!貞觀、開元,亦嘗厘革,剷除不盡,流衍轉滋。朕博覽前言,旁求輿議,弊之可革,斷在不疑。而中外誠臣,協予至意,條疏至當,宜在必行。懲千古之蠹源,成百王之典法,濟人利眾,予何讓焉。其天下所拆寺四千六百餘所,還俗僧尼二十六萬五百人,收充兩稅戶,拆招堤、蘭若四萬餘所,收膏腴上田數千萬頃,收奴婢為兩稅戶十五萬人。隸僧尼屬主客,顯明外國之教。勒大秦穆護、襖三千餘人
 還俗,不雜中華之風。於戲!前古未行,似將有待;及今盡去,豈謂無時。驅游惰不業之徒,已逾十萬;廢丹雘無用之室,何啻億千。自此清凈訓人,慕無為之理;簡易齊政,成一俗之功。將使六合黔黎,同歸皇化。尚以革弊之始,日用不知,下制明廷,宜體予意。



 制第六女封樂溫公主,第七女封長寧公主。中書奏:「伏見公主上表稱『妾某者』,伏以臣妾之義,取其賤稱;家人之禮,即宜區別。臣等商量,公主上表,請如長公主之例,並云『某邑公主幾女上
 表』,郡、縣主亦望依此例稱謂。」從之。



 九月,火星犯上將。十月乙亥,中書奏:「氾水縣武牢關是太宗擒王世充、竇建德之地,關城東峰有二聖朔容,在一堂之內。伏以山河如舊,城壘猶存,威靈皆盛於軒臺,風雲疑還於豐沛。誠宜百代嚴奉,萬邦式瞻。西漢故事,祖宗嘗行幸處,皆令邦國立廟。今緣定覺寺例合毀拆。望取寺中大殿材木,於東峰以造一殿,四面置宮墻,伏望名為昭武廟,以昭聖祖武功之盛。委懷孟節度使差判官一人勾當。緣驛
 像年代已久,望令李石於東都揀好畫手,就增嚴飾。初興功日,望令東都差分司官一員薦告。」從之。



 十一月甲辰,敕:「悲田養病坊,緣僧尼還俗,無人主持,恐殘疾無以取給,兩京量給寺田賑濟。諸州府七頃至十頃,各於本管選耆壽一人勾當,以充粥料。」十二月,車駕幸咸陽。給事中韋弘質上疏,論中書權重,三司錢穀不合宰相府兼領。相奏論之曰:



 臣等昨於延英對,恭聞對旨常欲朝廷尊,臣下肅,此是陛下深究理本也。臣按《管子》云:「凡國
 之重器,莫重於令。令重則群尊,君尊則國安。故國安在於奠君,尊君在於行令。君人之理,本莫要於出令。故曰:虧令者死,益令者死,不得令者死,不從令者死。又曰:令行於上,而下論不可,是上失其威,下系於人也。」自大和已來,其風大弊,令出於上,非之於下。此弊不除,無以理國也。



 昨韋弘質所論宰相不合兼領錢穀。臣等輒以事體陳聞。昔匡衡所以云:「大臣者,國家之股肱,萬姓所瞻仰,明王所慎擇。」《傳》曰:「下輕其上,賤人圖柄,則國家搖
 動,而人不靜。」弘質受人教導,輒獻封章,是則賤人圖柄矣。蕭望之漢朝名儒重德,為御史大夫,奏云:「今首歲日月少光,罪在臣等,」上以望之意輕丞相,乃下侍中御史詰問。貞觀中,監察御史陳師合上書云:「人之思慮有限,一人不可兼總數職。」太宗曰:「此人妄有毀謗,欲離間我君臣。」流師合於嶺外。賈誼云:「人主如堂,群臣如陛,陛高則堂高。」亦由將相重則君尊,其勢然也。如宰相奸謀隱匿,則人人皆得上論。至於制置職業,固是人主之柄,非
 小人所得乾議。古者朝廷之上,各守其官。思不出位。弘質賤人,豈得以非所宜言上瀆明主,此是輕宰相撓時政也。昔東漢處士橫議,遂有黨錮事起,此事深要懲絕。伏望陛下詳其奸詐,去其朋徒,則朝廷安靜,制令肅然。臣等不勝感憤之至。



 弘質坐貶官。又奏曰:「天寶已前,中書除機密遷授之外,其他政事皆與中書舍人同商量。自艱難已來,務從權便,政頗去於臺閣,事多系於軍期,決遣萬機,不暇博議。臣等商量,今後除機密公事外,諸
 候表疏、百僚奏事、錢穀刑獄等事,望令中書舍人六人,依故事先參詳可否,臣等議而奏聞。」從之。李德裕在相位日久,朝臣為其所抑者皆怨之。自崔鉉、杜忭罷相後,中貴人上前言德裕太專,上意不悅,而白敏中之徒,教弘質論之,故有此奏。而德裕結怨之深,由此言也。



 六年春正月癸卯朔。丁巳,左散騎常侍致仕馮定卒,贈工部尚書。己未,南詔、契丹、室韋、渤海、牂柯、昆明等國遣使入朝,對於麟德殿。兵部侍郎、判度支盧商奏:「諸道兵
 討伐黨項,今差度支郎官一人往所在有糧料州郡,先計度支給。」從之。己丑,渤海王子大之萼入朝。東都太微宮修成玄元皇帝、玄宗、肅宗三聖容,遣右散騎常侍裴章往東都薦獻。監察元壽奏前彭州刺史李鐵買本州龍興寺婢為乳母,違法,貶隨州長史。



 二月壬申朔。癸酉,以時雨愆候,詔:「京城天下系囚,除官典犯贓、持仗劫殺、忤逆十惡外,餘罪遞減一等,犯輕罪者並釋放。征黨項行營兵士,不得濫有殺傷。」丁丑,左拾遺王龜以父興元節度使起年
 高,乞休官侍養,從之。是夜,月犯畢大星,相去三寸。庚辰,以夏州節度使米暨充東北道招討黨項使。壬午,右庶子呂讓進狀:「亡兄溫女,大和七年嫁左衛兵曹蕭敏,生二男。開成三年,敏心疾乖忤,因而離婚。今敏日愈,卻乞與臣侄女配合。」從之。乙酉,前太子少保劉沔可太子太保致仕。前壽州刺史王鎮貶潞州長史。丁亥夜,月色少光,至一更一點,犯熒惑,相去四寸。後良久,其光燭地,在軫七度。壬辰,以翰林學士、起居郎孫穀為兵部員外郎
 充職。以旱,停上巳曲江賜宴。敕:「比緣錢重幣輕,生人轉困,今新加鼓鑄,必在流行,通變救時,莫切於此。宜申先甲之令,以儆居貨之徒。京城諸道,宜起來年正月已後,公私行用,並取新錢。其舊錢權停三數年。如有違犯,同用鉛錫錢例科斷。其舊錢並沒納。」又敕:「諸道鑄錢,已有次第,須令舊錢流布,絹價值增。文武百僚俸料,起三月一日,並給見錢一半。先給疋段,對估時價,皆給見錢。」貶舒州刺史蘇滌為連州刺史。滌李宗閔黨,前自
 給事中為德裕所斥,累年郡守,至是李紳言其無政故也。以邠寧節度使高承恭充西南面討黨項使。丙申夜,月掩牛南星,又犯歲星。丁酉,新羅使金國連入朝。辛丑夜,東北流星如桃,色赤,其光燭地,尾跡入大角,西流穿紫微垣。



 三月壬寅,上不豫,制改御名炎。帝重方士,頗服食修攝,親受法籙。至是藥躁,喜怒失常,疾既篤,旬日不能言。宰相李德裕等請見,不許。中外莫知安否,人情危懼。是月二十三日,宣遺詔,以皇太叔光王柩前即們。
 是日崩,時年三十三。謚曰至道昭肅孝皇帝,廟號武宗,其年八月,葬於端陵,德妃王氏祔焉。



 史臣曰:開成中,王室浸卑,政由閽寺。及綴衣將變,儲位遽移。昭肅以孤立維城,副茲當璧。而能雄謀勇斷,振已去之威權;運策勵精,拔非常之俊傑。屬天驕失國,潞孽阻兵,不惑盈庭之言,獨納大臣之計。戎車既駕,亂略底寧,紀律再張,聲名復振,足以章武出師之跡,繼元和戡亂之功。然後迂訪道之車,築禮神之館,棲心玄牝,物
 色幽人,將致俗於大庭,欲希蹤於姑射。於是削浮圖之法,懲游隋之民,志欲矯步丹梯,求珠赤水。徒見蕭衍、姚興之謬學,不悟秦王、漢武之非求,蓋惑於左道之言,偏斥異方之說。況身毒西來之教,向欲千祀,蚩蚩之民,習以成俗,畏其教甚於國法,樂其徒不異登仙。如文身祝發之鄉,久習而莫知其醜;以吐火吞刀之戲,乍觀而便以為神。安可正以《咸》《韶》,律之以章甫。加以笮融、何充之佞,代不乏人,非荀卿、孟子之賢,誰興正論。一朝隳殘
 金狄,燔棄胡書,結怨於膜拜之流,犯怒於鄙夫之口。哲王之舉,不駭物情,前代存而勿論,實為中道。欲革斯弊,以俟河清,昭肅明照,聽斯弊矣。



\end{pinyinscope}