\article{卷十 本紀第十 肅宗}

\begin{pinyinscope}

 肅宗文明武德大聖大宣孝皇帝諱亨,玄宗第三子,母曰元獻皇后楊氏。景雲二年乙亥生。初名嗣升,二歲封陜王,五歲拜安西大都護、河西四鎮諸蕃落大使。上仁
 愛英悟,得之天然;及長,聰敏強記,屬辭典麗,耳目之所聽覽,不復遺忘。



 開元十五年正月,封忠王,改名浚。五月,領朔方大使、單于大都護。十八年,奚、契丹犯塞,以上為河北道元帥,信安王禕為副,帥御史大夫李朝隱、京兆尹裴伷先等八總管兵以討之。仍命百僚設次於光順門,與上相見。左丞相張說退謂學士孫逖、韋述曰:「嘗見太宗寫真圖,忠王英姿穎發,儀表非常,雅類聖祖,此社稷之福也。」二十年,諸將大破奚、契丹,以上遙統之功,加
 司徒。二十三年,改名璵。二十五年,皇太子瑛得罪。二十六年六月庚子,立上為皇太子,改名紹。後有言事者云:紹與宋太子名同,改今名。初,太子瑛得罪,上召李林甫議立儲貳,時壽王瑁母武惠妃方承恩寵,林甫希旨,以瑁對。及立上為太子,林甫懼不利己,乃起韋堅、柳勣之獄,上幾危者數四。後又楊國忠依倚妃家,恣為褻穢,懼上英武,潛謀不利,為患久之。



 天寶十三載正月,安祿山來朝,上嘗密奏,雲祿山有反相。玄宗不聽。十四載十一
 月,祿山果叛,稱兵詣闕。十二月丁未,陷東京。辛丑,制太子監國,仍遣上親總諸軍進討。時祿山以誅楊國忠為名,由是軍民切齒於楊氏。國忠懼,乃與貴妃謀間其事,上遂不行。乃召河西節度使哥舒翰為皇太子前鋒兵馬元帥,令率眾二十萬守潼關。



 明年六月,哥舒翰為賊所敗,關門不守,國忠諷玄宗幸蜀。丁酉,至馬嵬頓,六軍不進,請誅楊氏。於是誅國忠,賜貴妃自盡。車駕將發,留上在後宣諭百姓。眾泣而言曰:「逆胡背恩,主上播越,臣
 等生於聖代,世為唐民,願戮力一心,為國討賊,請從太子收復長安。」玄宗聞之曰:「此天啟也。」乃令高力士與壽王瑁送太子內人及服御等物,留後軍廄馬從上。令力士口宣曰:「汝好去!百姓屬望,慎勿違之。莫以吾為意。且西戎北狄,吾嘗厚之,今國步艱難,必得其用,汝其勉之!」上回至渭北,便橋已斷,水暴漲,無舟楫;上號令水濱百姓,歸者三千餘人。渭水可涉,又遇潼關散卒,誤以為賊,與之戰,士眾多傷。乃收其餘眾北上,軍既濟,其後皆溺,
 上喜,以為天之佑。時從上惟廣平、建寧二王及四軍將士,才二千人。自奉天而北,夕次永壽,百姓遮道獻牛酒。有白雲起西北,長數丈,如樓閣之狀,議者以為天子之氣。戊戌,至新平郡。時晝夜奔馳三百餘里,士眾器械亡失過半,所存之眾,不過一旅。己亥,至安定郡,斬新平太守薛羽、保定太守徐,以其棄郡也。庚子,至烏氏驛,彭原太守李遵謁見,率兵士奉迎,仍進衣服糧糗。上至彭原,又募得甲士四百,率私馬以助軍。辛丑,至平涼郡,蒐
 閱監牧公私馬,得數萬疋,官軍益振。時賊據長安,知上治兵河西。三輔百姓皆曰:「吾太子大軍即至!」賊望西北塵起,有時奔走。戊申,扶風人康景龍殺賊宣慰使薛總等二百餘人,陳倉令薛景仙率眾收扶風郡守之。由是關輔豪右皆謀殺賊,賊故不敢侵軼。



 上在平涼,數日之間未知所適,會朔方留後杜鴻漸、魏少游、崔漪等遣判官李涵奉箋迎上,備陳兵馬招集之勢,倉儲庫甲之數,上大悅。鴻漸又發朔方步騎數千人於白草頓奉迎。時
 河西行軍司馬裴冕新授御史中丞赴闕,遇上於平涼,亦勸上治兵於靈武以圖進取,上然之。上初發平涼,有彩雲浮空,白鶴前引,出軍之後,有黃龍自上所憩屋騰空而去。上行至豐寧南,見黃河天塹之固,欲整軍北渡,以保豐寧,忽大風飛沙,跬步之間,不辨人物,及回軍趨靈武,風沙頓止,天地廓清。



 七月辛酉,上至靈武,時魏少游預備供帳,無不畢備。裴冕、杜鴻漸等從容進曰:「今寇逆亂常,毒流函谷,主上倦勤大位,移幸蜀川。江山阻險,
 奏請路絕,宗社神器,須有所歸。萬姓顒顒,思崇明聖,天意人事,不可固違。伏願殿下順其樂推,以安社稷,王者之大孝也。」上曰:「俟平寇逆,奉迎鑾輿,從容儲闈,侍膳左右,豈不樂哉!公等何急也?」冕等凡六上箋。辭情激切,上不獲已,乃從。是月甲子,上即皇帝位於靈武。禮畢,冕等跪進曰:「自逆賊恁陵,兩京失守,聖皇傳位陛下,再安區宇,臣稽首上千萬歲壽。」群臣舞蹈稱萬歲。上流涕歔欷,感動左右。即日奏其事於上皇。是日,御靈武南門,下制
 曰:



 朕聞聖人畏天命,帝者奉天時。知皇靈睠命,不敢違而去之;知歷數所歸,不獲已而當之。在昔帝王,靡不由斯而有天下者也。乃者羯胡亂常,京闕失守,天未悔禍,群兇尚扇。聖皇久厭大位,思傳眇身,軍興之初,已有成命,予恐不德,罔敢祗承。今群工卿士僉曰:「孝莫大於繼德,功莫盛於中興。」朕所以治兵朔方,將殄寇逆,務以大者,本其孝乎。須安兆庶之心,敬順群臣之請,乃以七月甲子,即皇帝位於靈武。敬崇徽號,上尊聖皇曰上皇天
 帝,所司擇日昭告上帝。朕以薄德,謬當重位,既展承天之禮,宜覃率士之澤,可大赦天下,改元曰至德。內外文武官九品已上加兩階、賜兩轉,三品已上賜爵一級。



 以朔方度支副使、大理司直杜鴻漸為兵部郎中,朔方節度判官崔漪為吏部郎中,並知中書舍人。以御史中丞裴冕為中書侍郎、同中書門下平章事。河西兵馬使周佖為河西節度使,隴右兵馬使彭元暉為隴右節度使,前蒲州刺史呂崇賁為關內節度使兼順化郡太守。以
 陳倉縣令薛景仙為扶風太守,以隴右節度使郭英乂為天水郡太守。改靈武郡為大都督府,上縣為望,中縣為上。丁卯,逆胡害霍國長公主、永王妃侯莫陳氏、義王妃閻氏、陳王妃韋氏、信王妃任氏、駙馬楊朏等八十餘人於崇仁之街。甲戌,賊黨同羅部五千餘人自西京出降朔方軍。己卯,京兆尹崔光遠、長安令蘇震等率府縣官吏大呼於西市,殺賊數千級,然後來赴行在。詔改扶風為鳳翔郡。



 八月壬午,朔方節度使郭子儀、範陽節度使
 李光弼破賊於常山郡之嘉山。上以治兵收京城,詔子儀等旋師,子儀、光弼率所統步騎五萬至自河北。詔以子儀為兵部尚書,依前靈州大都督府長史;光弼為戶部尚書,兼太原尹、北京留守:同中書門下平章事。回紇、吐蕃遣使繼至,請和親,願助國討賊,皆宴賜遣之。是日,上皇至成都,大赦。癸巳,上所奉表始達成都。丁酉,上皇遜位稱誥,遣左相韋見素、文部尚書房琯、門下侍郎崔渙等奉冊書赴靈武。



 九月戊辰,上南幸彭原郡。封故邠王
 守禮男承寀為燉煌王,令使回紇和親,冊回紇可汗女為毗伽公主,仍令僕固懷恩送承寀至回紇部。內官邊令誠背上皇投賊,至是復來見,上命斬之。丙子,至順化郡,韋見素、房琯、崔渙等自蜀郡賚上冊書及傳國寶等至。己卯,斬潼關敗將李承光於纛下。



 十月辛巳朔,日有蝕之,既。癸未,彭原郡以軍興用度不足,權賣官爵及度僧尼。上素知房琯名,至是琯請為兵馬元帥收復兩京,許之,仍令兵部尚書王思禮為副。分兵為三軍,楊希文、
 劉貴哲、李光進等各將一軍,其眾五萬。辛丑,琯與賊將安守忠戰於陳濤斜,官軍敗績,楊希文、劉貴哲等降於賊,琯亦奔還。平原太守顏真卿以食盡援絕,棄城渡河,於是河北郡縣盡陷於賊。十一月辛亥,河西地震有聲,圮裂廬舍,張掖、酒泉尤甚。戊子,回紇引軍來赴難,與郭子儀同破賊黨同羅部三千餘眾於河上。詔宰相崔渙巡撫江南,補授官吏。



 十二月戊子,以王思禮為關內節度。彭原郡百姓給復二載,郡同六雄,縣升緊、望。以秦州
 都督郭英乂為鳳翔太守,諫議大夫高適為廣陵長史、淮南節度兼採訪使。賊將阿史那承慶攻陷潁川郡,執太守薛願、長史龐堅。甲辰,江陵大都督府永王璘擅領舟師下廣陵。



 二載春正月庚戌朔,上在彭原受朝賀。是日通表入蜀賀上皇。上皇在蜀,每得上表疏,訊其使者,知上涕戀晨省,乃下誥曰:「至和育物,大孝安親,古之哲王,必由斯道。朕往在春宮,嘗事先後,問安靡闕,視膳無違。及同氣天
 倫,聯華棣萼,居嘗共被,食必分甘。今皇帝奉而行之,未嘗失墜,每有銜命而來,戒途將發,必肅恭拜跪,涕泗漣洏,左右侍臣,罔不感動。間者抱戴、赤雀、白狼之瑞,接武薦臻,此皆皇帝聖敬之符,孝友之感也。故能誕敷德教,橫於四海,信可以光宅寰宇,永綏黎元者哉!其天下有至孝友悌行著鄉閭堪旌表者,郡縣長官採聽聞奏,庶孝子順孫沐於玄化也。」甲寅,以襄陽太守李峘為蜀郡長史、劍南節度使,將作少監魏仲犀為襄陽、山南道節
 度使,永王傅劉匯為丹陽太守兼防禦使。以憲部尚書李麟同中書門下平章事。上皇遣平章事崔圓奉誥赴彭原。乙卯,逆胡安祿山為其子慶緒所殺。辛酉,於江寧縣置金陵郡,仍置軍,分人以鎮之。甲子,幸保定郡。丙寅,武威郡九姓商胡安門物等叛,殺節度使周佖,判官崔稱率眾討平之。是日,蜀郡健兒賈秀等五千人謀逆,上皇禦蜀郡南樓,將軍席元慶等討平之。



 二月戊子,幸鳳翔郡。文城太守武威郡九姓齊莊破賊五千餘眾。上議
 大舉收復兩京,盡括公私馬以助軍。給事中李暠署云「無馬」,大夫崔光遠劾之,貶暠江華太守。節度使李光弼大破賊將蔡希德之眾於城下,斬虜七萬,軍資器杖稱是。朔方節度使郭子儀大破賊將崔乾祐於潼關,收河東郡。永王璘兵敗,奔於嶺外,至大庾嶺,為洪州刺史皇甫侁所殺。三月癸亥,河西自去冬地震,至是方止。辛酉,以左相韋見素、平章事裴冕為左右僕射,並罷知政事。以前憲部尚書致仕苗晉卿為左相。吐蕃遣使和親,遣
 給事中南巨川報命。癸亥大雨,至癸酉不止,詔疏理刑獄,甲戌方止。夏四月戊寅朔,以郭子儀為司空,兼副元帥,統諸節度;李光弼為司徒。乙酉,太史奏歲星、太白、熒惑集於東井。



 五月癸丑,郭子儀與賊將安守忠戰於清渠,官軍敗績,子儀退保武功。丁巳,房琯為太子少師,罷知政事。以諫議大夫張鎬為中書侍郎、同中書門下平章事。以武部侍郎杜鴻漸為河西節度。庚申,誥追贈故妃楊氏為元獻皇太后,上母也。甲子,郭子儀以失律讓
 司空,許之。七月庚戌夜,蜀郡軍人郭千仞謀逆,上皇御玄英樓,節度使李峘討平之。丁巳,賊將安武臣陷陜郡,民無遺類。八月甲申,以黃門侍郎崔渙為餘杭太守、江東採訪防禦使。己丑,以平章事張鎬兼河南節度、採訪處置等使。靈昌太守許叔冀為賊所攻,援兵不至,拔眾投睢陽郡。癸巳,大閱諸軍,上御城樓以觀之。丁酉,改雍縣為鳳翔縣,陳倉為寶雞縣。閏八月辛未,賊將遽寇鳳翔,崔光遠行軍司馬王伯倫、判官李椿率眾捍賊。賊退,
 乘勝至中渭橋,殺賊守橋眾千人,追擊入苑中。時賊大軍屯武功,聞之燒營而去。伯倫與賊血戰而死,李椿力窮被執,然自是賊不敢西侵。



 九月丁丑,上黨節度使程千里與賊挑戰,為賊將蔡希德所擒。燉煌王承寀自回紇使還,拜宗正卿;納回紇公主為妃,回紇封為葉護,持四節,與回紇葉護太子率兵四千助國討賊。葉護入見,宴賜加等。丁亥,元帥廣平王統朔方、安西、回紇、南蠻、大食之眾二十萬,東向討賊。壬寅,與賊將安守忠、李歸仁
 等戰於香積寺西北,賊軍大敗,斬首六萬級,賊帥張通儒棄京城東走。癸卯,廣平王收西京。甲辰,捷書至行在,百僚稱賀,即日告捷於蜀。上皇遣裴冕入京,啟告郊廟社稷。冬十月乙巳朔,以崔光遠為京兆尹。詔曰:「緣京城初收,要安百姓,又灑掃宮闕,奉迎上皇。以今月十九日還京,應緣供頓,務從減省。」吐蕃寇陷西平郡。癸丑,賊將尹子奇陷睢陽,害張巡、姚訚、許遠。賊自香積之敗,悉眾保陜郡,廣平王統郭子儀等進攻,與賊戰於陜西之新
 店,賊眾大敗,斬首十萬級,橫尸三十里。庚申,安慶緒與其黨奔河北。壬戌,廣平王入東京,陳兵天津橋南,士庶歡呼路側。陷賊官偽署侍中陳希烈、中書令張垍等三百餘人素服待罪。癸亥,上自鳳翔還京,仍遣太子太師韋見素入蜀迎上皇,鳳翔郡給復五載。丙寅,至望賢宮,得東京捷書至,上大喜。丁卯,入長安。士庶涕泣拜忭曰:「不圖復見吾君!」上亦為之感惻。九廟為賊所焚,上素服哭於廟三日,入居大明宮。是日,上皇發蜀郡。己巳,文武
 脅從官免冠徒跣,朝堂待罪,禁之府獄,命中丞崔器劾之。回紇葉護自東京還,宴之於宣政殿,便辭還蕃。乃封葉護為忠義王,約每年送絹二萬疋,至朔方王便交授。



 十一月壬申朔,上御丹鳳樓,下制曰:「我國家出震乘乾,立極開統。謳歌歷數,啟聖千齡;文物聲名,握圖六葉。安祿山夷羯賤類,粗立邊功,遂肆兇殘,變起倉卒,而毒流四海,塗炭萬靈。朕興言痛憤,,提戈問罪,靈武聚一旅之眾,至鳳翔合百萬之師,親總元戎,掃清群孽。廣平王俶
 受委元帥,能振天聲;郭子儀決勝無前,克成大業。兼回紇葉護、雲南子弟、諸蕃兵馬,力戰平兇,勢若摧枯,易同破竹。朕早承聖訓,嘗讀禮經,義切奉先,恐不克荷。今復宗廟於函洛,迎上皇於巴蜀;導鑾輿而反正,朝寢門而問安;寰宇載寧,朕願畢矣。且復人將有主,敬當天地之心;興豈在予,實憑社稷之祐。今兩京無虞,三靈通慶,可以昭事,宜在覃恩,待上皇到日,當取處分。」是時河南、河東諸郡縣皆平。宮省門帶「安」字者改之。偽御史大夫嚴
 莊來降。新成九廟神主,上親告享。



 十二月丙午,上皇至自蜀,上至望賢宮奉迎。上皇御宮南樓,上望樓闢易,下馬趨進樓前,再拜蹈舞稱慶。上皇下樓,上匍匐捧上皇足,涕泗嗚咽,不能自勝。遂扶侍上皇御殿,親自進食;自御馬以進,上皇上馬,又躬攬轡而行,止之後退。上皇曰:「吾享國長久,吾不知貴,見吾子為天子,吾知貴矣。」上乘馬前導,自開遠門至丹鳳門,旗幟燭天,彩棚夾道。士庶舞忭路側,皆曰:「不圖今日再見二聖!」百僚班於含元殿
 庭,上皇御殿,左相苗晉卿率百闢稱賀,人人無不感咽。禮畢,上皇詣長樂殿謁九廟神主,即日幸興慶宮。上請歸東宮,上皇遣高力士再三尉譬而止。受賊偽署左相陳希烈、達奚珣等二百餘人並禁於楊國忠宅鞫問。甲寅,以左相苗晉卿為中書侍郎、同中書門下平章事。十二月戊午朔,上御丹鳳門,下制大赦。蜀郡靈武元從功臣太子太師、豳國公韋見素,內侍、齊國公高力士,右龍武大將軍陳玄禮,各加實封三百戶。田長文、張崇俊、杜
 休祥各加二百戶。右僕射裴冕冀國公,殿中監李輔國成國公,宗正卿李遵鄭國公,兼進封邑。廣平王俶封楚王,加實封二千戶。左僕射、朔方節度郭子儀加司徒,進封代國公,實封一千戶。兵馬使僕固懷恩封豐國公,右金吾將軍李嗣業封虢國公,司徒兼太原尹李光弼薊國公,關內節度王思禮霍國公,淮南節度來瑱潁國公,南陽太守魯炅岐國公,仍並加實封。京兆尹崔光遠鄴國公,開府李光進範陽郡公,左相苗晉卿為侍中、封韓
 國公,憲部尚書、平章事李麟褒國公,中書侍郎崔圓為中書令、趙國公,中書侍郎張鎬南陽縣公。近日所改百司額及郡名官名,一依故事。改蜀郡為南京,鳳翔府為西京,西京改為中京,蜀郡改為成都府。鳳翔府官僚並同三京名號。其李憕、盧弈、顏杲卿、袁履謙、許遠、張巡、張介然、蔣清、龐堅等即與追贈,訪其子孫,厚其官爵。文武三品已上賜爵一級,四品已下加一階。賜酺五日。進封南陽王系為趙王,新城王僅為彭王,潁川王僴為兗
 王。第七男侹為涇王,第九男僙封襄王,第十男佋封興王,第十一男倕封杞王,第十二男侗封定王。甲子,上皇御宣政殿,授上傳國璽,上於殿下涕泣而受之。己丑,賊將偽範陽節度使史思明以其兵眾八萬之籍,與偽河東節度使高秀巖並表送降。庚午,制:「人臣之節,有死無二;為國之體,叛而必誅。況乎委質賊廷,宴安逆命,耽受寵祿,淹延歲時,不顧思義,助其效用,此其可宥,法將何施?達奚珣等或受任臺輔,位極人臣;或累葉寵榮,姻聯
 戚里;或歷踐臺閣,或職通中外。夫以犬馬微賤之畜,猶知戀主;龜蛇蠢動之類,皆能報恩。豈曰人臣,曾無感激?自逆胡作亂,傾覆邦家,凡在黎元,皆含怨憤,殺身殉國者,不可勝數。此等黔首,猶不背國恩。受任於梟獍之間,咨謀於豺虺之輩,靜言此情,何可放宥。達奚珣等一十八人,並宜處斬;陳希烈等七人,並賜自盡;前大理卿張均特宜免死,配流合浦郡。」是日斬達奚珣等於子城西南隅獨柳樹,仍集百僚往觀之。



 三載正月甲戌朔。戊寅,上皇御宣政殿,冊皇帝尊號曰光天文武大聖孝感皇帝。上以徽號中有「大聖」二字,上表固讓,不允。乙酉,敕:「因亂所失庫物,先差使搜檢,如聞下吏因便擾人,其搜檢使一切並停,務令安輯。」內出宮女三千人。庚寅,大閱諸軍於含元殿庭,上御棲鸞閣觀之。庚子,冊良娣張氏為淑妃。



 二月癸卯朔,賊將偽淄青節度能元皓以其地請降,用為河北招討使,並其子昱並授官爵。乙巳,上御興慶宮,奉冊上皇徽號曰太上至
 道聖皇大帝。丁未,御明鳳門,大赦天下,改至德三載為乾元元年。成都、靈武扈從功臣三品已上與一子官,五品已下與一子出身,六品已下量與改轉。死王事、陷賊不受偽命而死者,並與追贈。陷賊官先推鞫者,例減罪一等。今後醫卜入仕者,同明法例處分。三月癸酉朔。甲戌,元帥楚王俶改封成王。乙亥,山南東道、河南、淮南、江南皆置節度使。辛卯,以歲饑,禁酤酒,麥熟之後,任依常式。太史監為司天臺,取承寧坊張守珪宅置,仍補官員六十人。夏
 四月癸卯,以太子少師、嗣虢王巨為東京留守、河南尹,充京畿採訪處置使。己酉,冊淑妃張氏為皇后。辛亥,九廟成,備法駕自長安殿迎九廟神主入新廟。甲寅,上親享九廟,遂有事於圓丘,即日還宮。翌日,御明鳳門,大赦天下。戊辰,上進煉石英金灶於興慶宮。五月壬申朔,回紇、黑衣大食各遣使朝貢,至閣門爭長,詔其使各從左右門入。壬午,詔:「近緣狂寇亂常,諸道分置節度,蓋總管內徵發、文牒往來,仍加採訪,轉滋煩擾。其諸道先置採
 訪、黜陟二使宜停。」癸未夜,月掩心前星。戊子,以河南節度、中書侍郎、平章事張鎬為荊州大都督府長史、本州防禦使,以禮部尚書崔光遠為河南節度。庚寅,立成王俶為皇太子。以荊州長史季廣琛赴河南行營會計討賊於河北。已未,中書令崔圓為太子少師,刑部尚書、同平章事李麟為太子少傅,並罷知政事。以太常少卿、知禮儀事王璵為中書侍郎、同中書門下平章事。丙申,敦煌王承寀薨。



 六月辛丑朔,吐火羅、康國遣使朝貢。己酉,
 初置太一神壇於圓丘東。是日,命宰相王璵攝行祠事。癸丑夜,月入南斗魁。戊午,詔:「三司所推劾受賊偽官等,恩澤頻加,科條遞減,原其事狀,稍近平人,所推問者,並宜釋放。」秋七月辛未朔,吐火羅葉護烏利多並九國首領來朝,助國討賊,上令赴朔方行營。丙戌,初鑄新錢,文曰「乾元重寶」,用一當十,與開元通寶同行用。丁亥,制上第二女寧國公主出降回紇英武威遠毗伽可汗。



 八月壬寅,以青徐等五州節度使季廣琛兼許州刺史,河
 南節度使崔光遠兼汴州刺史。以青州刺史許叔冀兼滑州刺史,充青滑六州節度使。甲辰,上皇誕節,上皇宴百官於金明門樓。朔方節度使郭子儀、河東節度使李光弼、關內節度使王思禮來朝,加子儀中書令,光弼侍中,思禮兵部尚書,餘如故。



 九月庚午朔,右羽林大將軍趙泚為蒲州刺史、蒲同虢三州節度使,貝州刺史能元皓為齊州刺史、齊兗鄆等州防禦使。庚寅,大舉討安慶緒於相州。命朔方節度郭子儀、河東節度李光弼、關內
 潞州節度使王思禮、淮西襄陽節度魯炅、興平節度李奐、滑濮節度許叔冀、平盧兵馬使董秦、北庭行營節度使李嗣業、鄭蔡節度使季廣琛等九節度之師,步騎二十萬,以開府魚朝恩為觀軍容使。癸巳,廣州奏大食國、波斯國兵眾攻城,刺史韋利見棄城而遁。十月乙未,以鳳翔尹李齊物為刑部尚書,以濮州刺史張方須為廣州都督、五府節度使。郭子儀奏破賊十萬於衛州,獲安慶緒弟慶和,進收衛州。甲寅,上皇幸華清宮,上送於灞
 上。許叔冀奏:「衛州婦人侯四娘、滑州婦人唐四娘、某州婦人王二娘相與歃血,請赴行營討賊。」皆補果毅。壬申,王思禮破賊二萬於相州。



 十一月丁丑,郭子儀收魏州,得偽署刺史蕭華於州獄,詔復以華為刺史。是日,上皇至自華清宮,上迎於灞上。上自控上皇馬轡百餘步,誥止之,乃已。十二月癸卯,以河南節度崔光遠為魏州刺史,遣蕭華赴相州行營。甲辰,以升州刺史韋黃裳為蘇州刺史、浙西節度使。庚戌,以戶部尚書李峘充淮南、浙
 西觀察使、處置節度使。丙寅,立春,上御宣政殿,讀時令,常參官五品已上升殿序坐而聽之。時王師圍相州,慶緒食盡,求於史思明,率眾來援。丁卯,思明復陷魏州,刺史崔光遠出奔。



 二年春正月己巳朔,上御含元殿,受尊號曰乾元大聖光天文武孝感皇帝。是日,史思明自稱燕王於魏州,僭立年號。丁丑,上親祀九宮貴神,齋宿於壇所。戊寅,有事於籍田,上行九推,禮官奏太過,上曰:「朕勸農率下,所恨
 不終千畝耳。」癸未夜,月掩歲星。乙丑,以御史中丞崔寓都統浙江、淮南節度處置使。丙申,開府儀同三司、衛尉卿、懷州北庭行營節度使、虢國公李嗣業卒於相州行營。庚子,以太子少師崔圓充東京留守,判尚書省事。



 二月壬子望,月蝕既。百官請加皇后張氏尊號曰「翊聖」,上以月蝕陰德不修而止。貶東京留守、嗣虢王巨以遂州刺史,苛政也。丙辰,月犯心大星。壬戌,遣侍中苗晉卿、王璵分錄囚徒。三月丁卯朔。己巳,皇后祀先蠶於苑中。壬
 申,相州行營郭子儀等與賊史思明戰,王師不利,九節度兵潰,子儀斷河陽橋,以餘眾保東京。辛卯,以衛尉卿荔非元禮為懷州刺史,權鎮西、北庭行營節度使;以滑州刺史許叔冀充滑、汴、曹、宋等州節度使;以鄆州刺史尚衡為徐州刺史,充亳、潁等州節度使。甲午,以兵部侍郎呂諲同中書門下平章事,以太子賓客薛景仙為鳳翔尹、本府防禦使。乙未,侍中苗晉卿為太子太傅,平章事王璵為刑部尚書,並罷知政事。以京兆尹李峴為吏
 部尚書,禮部侍郎李揆為中書侍郎,與戶部侍郎第五琦等並同中書門下平章事。丙申,以郭子儀為東畿、山南東、河南等道節度、防禦兵馬元帥,權東京留守,判尚書省事。以河西節度副使來瑱為陜州刺史,充虢華節度、潼關防禦團練等使。四月丁酉朔,王思禮奏於潞城縣東直千嶺破賊萬人。壬寅,詔以寇孽未平,務懷捴挹,「自今以後,朕常膳及服御等物,並從節減,諸作坊造坊並停」。「比緣軍國務殷,或宣口敕處分。今後非正宣,並不得
 行用,中外諸務,各歸有司。英武軍及六軍諸使,比因論竟便行追攝。今後須經臺府,如處斷不平,具狀聞奏。自文武五品已上正官各舉賢良方正、直言極諫一人,任自封進。兩省官十日一上封事。御史臺欲彈事,不須進狀,仍服豸冠。殘妖未殄,國步猶難,共體至公,以康庶政。朕推誠御物,與眾共之,思與蒼生,臻夫至道。宣示中外,知朕意焉。」甲辰,以鄧州刺史魯炅為鄭州刺史,充陳、鄭、潁、亳節度使;以徐州刺史尚衡為青州刺史,充青、淄、密、
 登、萊、沂、海等州節度使;以商州刺史、興平軍節度李奐兼豫、許、汝等州節度使。乙巳,第五琦依舊判度支、租庸等使。史思明僭號於魏州。貶季廣琛宣州刺史。崔光遠為太子少保。癸亥,以久旱徙市,雩祈雨。五月辛巳,貶宰相李峴蜀州刺史。丁亥,上御宣政殿試文經邦國等四科舉人。乃以汝州刺史劉展為滑州刺史,以平盧軍節度都知兵馬使董秦為濮州刺史。六月乙未朔,以右僕射裴冕為御史大夫、成都尹,持節充劍南節度副大使、
 本道觀察使;以邠州刺史房琯為太子賓客;以饒州刺史顏真卿為升州刺史,充浙江西道節度使。已巳,以明州刺史呂延之為越州刺史,充浙江東道節度使;以右羽林大將軍彭元曜為鄭州刺史,充陳、鄭、申、光、壽等州節度使。秋七月乙丑朔,以禮部尚書韋陟充東京留守。太子少傅、兗國公李麟卒。辛巳,制以趙王系為天下兵馬元帥,司空兼侍中李光弼為副。丁亥,以兵部尚書、潞州大都督府長史、潞沁節度、霍國公王思禮兼太原尹,
 充北京留守、河東節度副大使。刑部尚書王璵為蒲州刺史,充蒲、同、絳三州節度使。八月乙亥,襄州偏將康楚元逐刺史王政,據城自守。丙辰,寧國公主自回紇還宮。副元帥李光弼兼幽州大都督府長史、河北節度等使。九月甲午,襄州賊張嘉延襲破荊州,澧、朗、復、郢、硤、歸等州官吏皆棄城奔竄。戊辰,新鑄大錢,文如乾元重寶,而重其輪,用一當五十,以二十二斤成貫。丁亥,以太子少保崔光遠充荊、襄等州招討使,右羽林大將軍王仲升
 充申、安、沔等州節度使,右羽林將軍李抱玉為鄭州刺史、鄭陳潁亳四州節度使。庚寅,逆胡史思明陷洛陽,副元帥李光弼守河陽,汝、鄭、滑等州陷賊。冬十月丁酉,制親征史思明,竟不行。乙巳,李光弼奏破賊於城下。壬戌,宰相呂諲起復,依前平章事。十一月甲子朔,商州刺史韋倫破康楚元,荊襄平。庚午,戶部侍郎、同平章事第五琦貶忠州長史,御史大夫賀蘭進明貶溱州司馬。十二月癸巳朔,神策將軍衛伯玉破賊於陜東強子阪。甲寅,
 以御史大夫史翽為襄州刺史,充山南東道節度、觀察處置等使。



 三年春正月癸亥朔。辛巳,李光弼進位太尉、兼中書令,餘如故。以杭州刺史侯令儀為升州刺史,充浙江西道節度兼江寧軍使。戊子,以朔方節度使郭子儀兼邠寧、鄜坊兩道節度使。二月癸巳朔,以右丞崔寓為蒲州刺史,充蒲、同、晉、絳等州節度使。庚戌,第五琦除名,長流夷州。癸丑,以太子少保崔光遠為鳳翔尹、秦隴節度使。



 三
 月壬申,以京兆尹李若幽為成都尹、劍南節度使。甲申,以蒲州為河中府,其州縣官吏所置,同京兆、河南二府。四月甲午,李光弼奏破賊於懷州、河陽。甲辰,以禮部尚書、東京留守韋陟為吏部尚書,太子賓客房琯為禮部尚書。以太子賓客、平章事張鎬為左散騎常侍,太子賓客崔渙為大理卿。是歲饑,米斗至一千五百文。戊申,襄州軍亂,殺節度使史翽,部將張維瑾據州叛。丁巳夜,彗出東方,在婁、胃間,長四尺許。戊午,以右丞蕭華為河中
 尹、兼御史中丞,充同、晉、絳等州節度、觀察處置使。己未,以陜州刺史來瑱為襄州刺史,充山南東道襄鄧等十州節度、觀察處置等使。庚申,以右羽林大將軍郭英乂為陜州刺史、陜西節度、潼關防禦等使。閏四月辛酉朔,彗出西方,其長數丈。壬戌,以禮部尚書房琯為晉州刺史。甲子,制彭王僅充河西節度大使,兗王僴北庭節度大使,涇王侹隴右節度大使,杞王倕陜西節度大使,興王佋鳳翔節度大使,蜀王偲邠寧節度大使,並不出閣。
 丁卯,太原尹王思禮進位司空。甲戌,天下兵馬元帥、趙王系改封越王。己卯,以星文變異,上御明鳳門,大赦天下,改乾元為上元。追封周太公望為武成王,依文宣王例置廟。時大霧,自四月雨至閏月末不止。米價翔貴,人相食,餓死者委骸於路。壬午,以刑部尚書王璵為太常卿,右散騎常侍韓擇木為禮部尚書。



 五月庚寅朔。丙午,以太子太傅、韓國公苗晉卿為侍中。壬子,黃門侍郎、同中書門下三品呂諲為太子賓客,罷知政事。癸丑,以河
 南尹劉晏為戶部侍郎,勾當度支、鑄錢、鹽鐵等使。是夜,月掩昴。



 六月乙丑,詔先鑄重棱錢一當五十,宜減當三十文;開元宜一當十。七月己丑朔。丁未,上皇自興慶宮移居西內。丙辰,開府高力士配流巫州;內侍王承恩流播州,魏悅流溱州;左龍武大將軍陳玄禮致仕。丙辰,御史大夫崔器卒。八月辛未,吏部尚書韋陟卒。丁丑,以太子賓客呂諲為荊州大都督府長史、澧朗硤忠五州節度觀察處置等使。己卯,以將作監王昂為河中尹、本府
 晉絳等州節度使。丁亥,贈故興王佋為恭懿太子。



 九月甲午,以荊州為南都,州曰江陵府,官吏制置同京兆。其蜀郡先為南京,宜復為蜀郡。十月壬申,以廬州刺史趙良弼為越州刺史,充浙江東道節度使;青州刺史殷仲卿為淄州刺史、淄沂滄德棣等州節度使。甲申,以兵部侍郎尚衡為青州刺史、青登等州節度使。十一月乙巳,李光弼奏收懷州。宋州刺史劉展赴鎮揚州,揚州長史鄧景山以兵拒之,為展所敗,展進陷揚、潤、升等州。十二
 月庚辰,以右羽林軍大將軍李鼎為鳳翔尹、興鳳隴等州節度使。癸未夜,歲星掩房。



 二年春正月丁亥朔。辛卯,溫州刺史季廣琛為宣州刺史,充浙江西道節度使。甲午,上不康,皇后張氏刺血寫佛經。甲寅,詔府縣、御史臺、大理疏理系囚,死罪降從流,流已下並釋放。乙卯,平盧軍兵馬使田神功生擒劉展,揚、潤平。



 二月己未,黨項寇寶雞,入散關,陷鳳州,殺刺史蕭心曳,鳳翔李鼎邀擊之。癸亥,以鳳翔尹崔光遠為成都
 尹、劍南節度、度支營田觀察處置等使,以太子詹事、趙國公崔圓為揚州大都督府長史、淮南節度觀察等使。辛未夜,月有蝕之,既。戊寅,李光弼率河陽之軍五萬,與史思明之眾戰於北邙,官軍敗績。光弼、僕固懷恩走保聞喜,魚朝恩、衛伯玉走保陜州,河陽、懷州共陷賊,京師戒嚴。癸未,中書侍郎、同中書門下三品李揆貶為袁州長史。以前河中尹蕭華為中書侍郎、同平章事、集賢殿崇文館大學士,兼修國史。



 三月甲子,史朝義率眾夜襲
 我陜州,衛伯玉逆擊,敗之。戊戌,史思明為其子朝義所殺。李光弼以失律讓太尉、中書令,許之,授侍中、河中尹、晉絳等州節度觀察使。夏四月乙亥朔,嗣岐王珍得罪,廢為庶人,於溱州安置。連坐竇如玢、崔昌處斬,駙馬都尉楊洄、薛履謙賜自盡,左散騎常侍張鎬貶辰州司戶長任。己未,以吏部侍郎裴遵慶為黃門侍郎、同中書門下平章事。青州刺史尚衡、兗州刺史能元皓並奏破賊。壬午,梓州刺史段子璋叛,襲破遂州,殺刺史嗣虢王巨。
 東川節度使李奐戰敗,奔成都。



 五月甲午,思明偽將滑州刺史令狐彰以滑州歸朝,授彰御史中丞,依前滑州刺史、滑魏德貝相六州節度使。乙未,劍南節度使崔光遠率師與李奐擊敗段子璋於綿州,擒子璋殺之。綿州平。李光弼來朝,進位太尉、兼侍中,充河南副元帥,都統河南、淮南、山南東道五道行營節度,鎮臨淮。北京留守、守司空、太原尹、河東節度副大使、霍國公王思禮卒。辛丑,以鴻臚卿、趙國公管崇嗣為太原尹、兼御史大夫,充
 北京留守、河東節度副大使。壬子,太子少傅、宗正卿李齊物卒。六月癸丑朔。己卯,以鳳翔尹李鼎為鄯州刺史、隴右節度營田等使。秋七月癸未朔,日有蝕之,既。大星皆見。甲辰,延英殿御座梁上生玉芝,一莖三花,上制《玉靈芝詩》。



 八月癸丑朔,以中官李輔國守兵部尚書,於尚書省上,命宰臣百官送之,酣宴竟日。自七月霖雨,至是方止,墻宇多壞,漉魚道中。辛巳,以殿中監李若幽為戶部尚書,充朔方鎮西北庭陳鄭等州節度使,鎮絳州,賜
 名國貞。九月壬午朔。壬辰,以太子賓客、集賢殿學士、昌黎伯韓擇木為禮部尚書。壬寅,制:朕獲守丕業,敢忘謙沖,欲垂範而自我,亦去華而就實。其「乾元大聖光天文武孝感」等尊崇之稱,何德以當之?欽若昊天,定時成歲,《春秋》五始,義在體元,惟以紀年,更無潤色。至於漢武,飾以浮華,非前王之茂典,豈永代而作則。自今已後,朕號唯稱皇帝,其年號但稱元年,去上元之號。其以今北庭潞儀隰等州行營、本管節度觀察等事,移鎮絳州。壬申,
 嗣寧王棣薨。癸酉,河南副元帥李光弼破賊於許州城下,收復許州。建辰月庚辰朔。壬午,詔天下見禁系囚,無輕重一切釋放。丙戌夜,月有白冠。癸巳,以襄州刺史來瑱為安州刺史,充淮西申、安、蘄、黃、沔等十六州節度使。甲午,黨項奴剌寇梁州,刺史李勉棄郡走。丙申,黨項寇奉天。上不康,百僚於佛寺齋僧。丁未,詔左降官、流人一切放還。戊申,中書侍郎、平章事、徐國公蕭華為禮部尚書,罷知政事。以尚書戶部侍郎元載同中書門下平章
 事,以禮部尚書韓擇木為太子太保。建巳月庚戌朔。壬子,楚州刺史崔侁獻定國寶玉十三枚:一曰玄黃天符,如笏,長八寸,闊三寸,上圓下方,近圓有孔,黃玉也。二曰玉雞,毛文悉備,白玉也。三曰穀璧,白玉也,徑可五六寸,其文粟粒無雕鐫之跡。四曰西王母白環,二枚,白玉也,徑六七寸。五曰碧色寶,圓而有光。六曰如意寶珠,形圓如雞卵,光如月。七曰紅靺鞨,大如巨慄,赤如櫻桃。八曰瑯玕珠,二枚,長一寸二分。九曰玉玦,形如玉環,四分缺
 一。十曰玉印,大如半手,斜長,理如鹿形,陷入印中,以印物則鹿形著焉。十一曰皇后採桑鉤,長五六寸,細如箸,屈其末,似真金,又似銀。十二曰雷公石斧,長四寸,闊二寸,無孔,細致如青玉。十三寶置於日中,皆白氣連天。人先表云:「楚州寺尼真如者,恍惚上升,見天帝。帝授以十三寶,曰:『中國有災,宜以第二寶鎮之。』」甲寅,太上至道聖皇天帝崩於西內神龍殿。上自仲春不豫,聞上皇登遐,不勝哀悸,因茲大漸。乙丑,詔皇太子監國。又曰:「上天降寶,
 獻自楚州,因以體元,葉乎五紀。其元年宜改為寶應,建巳月為四月,餘月並依常數,仍依舊以正月一日為歲首。」丁卯,宣遺詔。是日,上崩於長生殿,年五十二。群臣上謚曰文明武德大聖大宣孝皇帝,廟號肅宗。寶應二年三月庚午,葬於建陵。



 史臣曰:臣每讀《詩》至許穆夫人聞宗國之顛覆,周大夫傷宮室之黍離,其辭情於邑,賦諭勤懇,未嘗不廢書興嘆。及觀天寶失馭,流離奔播,又甚於詩人之於邑也。當
 其戎羯負恩,奄為豨突,豺豕遽興於轂下,胡越寧慮於舟中,借人之戈,持之反刺,變生於不意也。所幸太王去國,豳人不忘於周君;新莽據圖,黔首仍思於漢德。是以宣皇帝蒙六聖之遺業,因百姓之樂推。號令朔方,旬日而車徒雲合;旋師右輔,期月而關、隴砥平。故兩都再復於鑾輿,九廟復歆於黍稷。觀其迎上皇於蜀道,陳拜慶於望賢,父子於是感傷,行路為之隕涕。昔太公迎子,或從家令之言;而西伯事親,靡怠寢門之問。曾參、孝己,足
 以擬倫。然而道屈知幾,志微遠略。殘妖未殄,宜先恢復之謀;餘燼才收,何暇升平之禮。方聽王璵伏奏,輔國贊成,紺轅躬籍於春郊,翠AW先蠶於繭館、或御殿曉宣時令,或登壇宿禮貴神。禮即宜然,時何暇給。鐘懸未移於簨虡,思明已陷於洛陽,是知祝史疇人,安能及遠。猶賴大臣宣力,諸將效忠,旄頭終隕於三川,杲日重明於六合。比平王之遷洛,我則英雄;論元帝之渡江,彼誠麼麼。寧親復國,肅乃休哉!



 贊曰:犬羊犯順,輦輅播遷。兇徒竟斃,景祚重延。星馳蜀道,雨泣望賢。孝宣之謚,誰曰不然?



\end{pinyinscope}