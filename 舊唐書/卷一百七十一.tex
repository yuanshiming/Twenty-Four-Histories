\article{卷一百七十一}

\begin{pinyinscope}

 ○趙宗儒竇易直李
 逢吉段文昌子成式宋申錫李程



 趙宗儒,字秉文。八代祖彤,仕後魏為征南將軍。父驊,為秘書少監。宗儒舉進士,初授弘文館校書郎。滿歲,又以
 書判入高等,補陸渾主簿。數月,徵拜右拾遺,充翰林學士。時父驊秘書少監,與父並命,出於一日,當時榮之。建中四年,轉屯田員外郎,內職如故。居父憂,免喪,授司門、司勛二員外郎。



 貞元六年,領考功事,定百吏考績,黜陟公當,無所畏避。右司郎中獨孤良器、殿中侍御史杜倫,各以過黜之。尚書左丞裴鬱、御名中丞盧紹,比皆考中上,宗儒貶之中中。又秘書少監鄭雲逵考其同官孫昌裔入上下,宗儒復入中上。凡考之中上者,不過五十人,
 餘多減入中中。德宗聞而善之,遷考功郎中。



 丁夏憂,終喪,授吏部郎中。十一年,遷給事中。十二年,與諫議大夫崔損同日以本官同中書門下平章事,俱賜紫金魚袋。十四年,罷相,為右庶子。



 宗儒端居守道,勤奉朝請而已,德宗聞而嘉之。二十年,遷吏部侍郎,召見,勞之曰:「知卿閉關六年,故有此拜。曩者與先臣並命,尚念之耶?」宗儒因俯伏流涕。德宗崩,順宗命為德宗哀冊文,辭頗淒惋。



 元和初,檢校禮部尚書,判東都尚書省事、兼御史大夫,
 充東都留守、畿汝都防禦使。入為禮部、戶部二尚書,尋檢校吏部尚書,守江陵尹、兼御史大夫、荊南節度營田觀察等使。散冗食之戍二千人。六年,又入為刑部尚書。八年,轉檢校吏部尚書、興元尹、兼御史大夫,充山南西道節度觀察等使。九年,召拜御史大夫,俄遷檢校右僕射、河中尹、兼御史大夫、晉絳慈隰節度觀察等使。赴鎮後,擅用供軍錢八千餘貫,坐罰一月俸。十一年七月,入為兵部尚書。九月,改太子少傅,權知吏部尚書銓事。十
 四年九月,拜吏部尚書。



 穆宗即位,以初釋服,令尚書省官試先朝所徵集應制舉人。宗儒奏曰:「準今月十五日敕:比者先朝徵集應制人等,已及時限,恐皆來自遠方,難於久住,酌宜審事,遂委有司定日就試。如聞所集之人多已分散,須知審的,然後裁定,宜令所司商量聞奏者。伏以制科所設,本在親臨,南省試人,亦非舊典。今覃恩既畢,庶政惟新。況山陵日近,公務繁迫,待問之士,就試非多。臣等商量,恐須權罷。」從之。復拜太子少傅,判太常
 卿事。



 長慶元年二月,檢校右僕射,守太常卿。太常有《師子樂》,備五方之色,非會朝聘享不作,幼君荒誕,伶官縱肆,中人掌教坊者移牒取之。宗儒不敢違,以狀白宰相。宰相以為事在有司執守,不合關白。以宗儒怯不任事,改太子少師。



 寶歷元年,遷太子太保。昭肅晏駕,為大明宮留守。太和四年,拜檢校司空、兼太子太傅。文宗召見,諮以理道。對曰:「堯、舜之化,慈儉而已。願陛下守而勿失。」文宗嘉納之。五年,宋申錫被誣,上召師保已下議其刑。
 上以宗儒高年,宣令不拜。尋拜疏請老。六年,詔以司空致仕。是歲九月卒,年八十七,廢朝,冊贈司徒。



 宗儒以文學進,前後三鎮方任,八領選部,略於儀矩,切於治生,時論以此少之。



 竇易直,字宗玄,京兆人。祖元昌,彭州九瀧縣令。父彧,廬州刺史。易直舉明經,為秘書省校書郎,再以判入等,授藍田尉。累歷右司、兵部、吏部三郎中。元和六年,遷御史中丞。謝日,賜緋魚袋。八年,改給事中。九月,出為陜虢都
 防禦觀察使,仍賜紫。入為京兆尹。萬年尉韓晤奸贓事發,易直令曹官韋正晤訊之,得贓三十萬。上意其未盡,詔重鞫,坐贓三百萬,貶易直金州刺史,正晤長流昭州。十三年六月,遷宣州刺史、宣歙池都團練觀察等使。



 長慶二年七月,汴州將李絺逐其帥李願,易直聞之,欲出官物以賞軍。或謂易直曰:「賞給無名,卻恐生患。」乃已。軍士已聞之。時江、淮旱,水淺,轉運司錢帛委積不能漕,州將王國清指以為賞,激諷州兵謀亂。先事有告者,乃收國
 清下獄。其黨數千,大呼入獄中,篡取國清而出之,因欲大剽。易直登樓謂將吏曰:「能誅為亂者,每獲一人,賞十萬。」眾喜,倒戈擊亂黨,並擒之。國清等三百餘人,皆斬之。九月,以李德裕代還,為吏部侍郎。十一月,改戶部,兼御史大夫,判度支。四年五月,以本官同平章事,判使如故。改門下侍郎,封晉陽郡公。



 寶歷元年七月,罷判度支。大和二年十月罷相,檢校左僕射、平章事、襄州刺史、山南東道節度使。五年,入為左僕射,判太常卿事。十一月,檢
 校司空、鳳翔尹、鳳翔隴節度使。六年,以疾求還京師。七年四月卒,贈司徒,謚曰恭惠。



 易直自入仕十年餘,常居散秩,不應請闢;及居方任,亦以公廉聞。在相位,未嘗論用親黨,凡於公舉,即無所避。然元和中,吏部尚書鄭餘慶議僕射上日儀制,不與隔品官亢禮。易直時為御史中丞,奏駁餘慶所議。及易直為左僕射,卻行隔品致敬之禮,時論非之。



 李逢吉,字虛舟,隴西人。貞觀中學士李玄道曾孫。祖顏,
 父歸期。逢吉登進士第,釋褐授振武節度掌書記。入朝為左拾遺、左補闕,改侍御史,充入吐蕃冊命副使、工部員外郎,又充入南詔副使。元和四年,使還,拜祠部郎中,轉右司。六年,遷給事中。七年,與司勛員外郎李巨並為太子諸王侍讀。九年,改中書舍人。十一年二月,權知禮部貢舉、騎都尉,賜緋。四月,加朝議大夫、門下侍郎、同平章事,賜金紫。其貢院事,仍委禮部尚書王播署榜。



 逢吉天與奸回,妒賢傷善。時用兵討淮、蔡,憲宗以兵機委裴
 度,逢吉慮其成功,密沮之,由是相惡。及度親征,學士令孤楚為度制辭,言不合旨,楚與逢吉相善,帝皆黜之;罷楚學士,罷逢吉政事,出為劍南東川節度使、檢校兵部尚書。



 穆宗即位,移襄州刺史、山南東道節度使。逢吉於帝有侍讀之恩,遣人密結幸臣,求還京師。長慶二年三月,召為兵部尚書。時裴度亦自太原入朝。以度招懷河朔功,復留度,與工部侍郎元稹相次拜平章事。度在太原時,嘗上表論稹奸邪。及同居相位,逢吉以為勢必相
 傾,乃遣人告和王傅於方結客,欲為元稹刺裴度。及捕於方,鞫之無狀,稹、度俱罷相位,逢吉代度為門下侍郎平章事。自是浸以恩澤結朝臣之不逞者,造作謗言,百端中傷裴度。賴學士李紳、韋處厚等顯於上前,言度為逢吉排斥,而度於國有功,不宜擯棄,故得以僕射在朝。時已失河朔,而王智興擅據徐州,李絺+據汴州。國威不振,天下延頸俟度再秉國鈞,以攘暴亂。及為逢吉嫁禍,奪其權,四海為之側目,朝士上疏論列者十餘人。屬時
 君荒淫,政出群小,而度竟逐外籓。



 學士李紳有寵,逢吉惡之,乃除為中丞,又欲出於外。乃以吏部侍郎韓愈為京兆尹,兼御史大夫,放臺參。以紳褊直,必與愈爭。及制出,紳果移牒往來。愈性木強,遂至語辭不遜,喧論於朝。逢吉乃罷愈為兵部侍郎,紳為江西觀察使。紳中謝日,帝留而不遣。



 翼城人鄭注以醫藥得幸於中尉王守澄,逢吉令其從子仲言賂注,求結於守澄。仲言辯譎多端,守澄見之甚悅。自是,逢吉有助,事無違者。



 敬宗初即位,
 年方童丱,守澄從容奏曰:「陛下得為太子,逢吉之力也。是時,杜元穎、李紳堅請立深王為太子。」乃貶紳端州司馬。朝士代逢吉鳴吠者,張又新、李續之、張權輿、劉棲楚、李虞、程昔範、姜洽、李仲言,時號「八關十六子」。又新等八人居要劇,而胥附者又八人,有求於逢吉者,必先經此八人納賂,無不如意者。逢吉尋封涼國公,邑千戶,兼右僕射。



 昭愍即位,左右屢言裴度之賢,曾立大勛,帝甚嘉之。因中使往興元,即令問訊。



 寶歷初,度連上章請入覲。
 逢吉之黨坐不安席,如矢攢身,乃相與為謀,欲沮其來。張權輿撰「非衣小兒」之謠,傳於閭巷。言度相有天分,應謠讖。而韋處厚於上前解析,言權輿所撰之言。既不能沮,又令衛尉卿劉遵古從人安再榮告武昭謀害逢吉。武昭者,有才力,裴度破淮、蔡時獎用之,累奏為刺史。及度被斥,昭以門吏久不見用,客於京師,途窮頗有怨言。逢吉冀法司鞫昭行止,則顯裴度任用,以沮入朝之行。逢吉又與同列李程不協。太學博士李涉、金吾兵曹茅
 匯者,於京師貴游間以氣俠相許,二人出入程及逢吉之門。水部郎中李仍叔,程之族,知武昭鬱鬱恨不得官,仍叔謂昭曰:「程欲與公官,但逢吉阻之。」昭愈憤怒,因酒與京師人劉審、張少騰說刺逢吉之言。審以昭言告張權輿,乃聞於逢吉,即令茅匯召昭相見,逢吉厚相結托,自是疑怨之言稍息。逢吉待茅匯尤厚,嘗與匯書云:「足下當字僕為『自求』,僕當字足下為『利見』」。文字往來,其間甚密。及裴度求覲,無計沮之,即令訐武昭事,以暴揚其
 跡。再榮既告,李仲言誡匯曰:「言武昭與李程同謀則活,否則爾死。」匯曰:「冤死甘心。誣人以自免,予不為也。」及昭下獄,逢吉之醜跡皆彰。昭死,仲言流象州,茅匯流巂州,李涉流康州,李虞自拾遺為河南士曹。敬宗待裴度益厚,乃自漢中召還,復知政事。



 逢吉檢校司空、平章事、襄州刺史、山南東道節度使,仍請張又新、李續之為參佐。太和二年,改汴州刺史、宣武軍節度使。五年八月,入為太子太師、東都留守、東畿汝防禦使,加開府儀同三司。
 八年,李訓用事。三月,徵拜左僕射,兼守司徒。時逢吉已老,病足,不任朝謁,即以司徒致仕。九年正月卒,時年七十八。贈太尉,謚曰成。



 段文昌,字墨卿,西河人。高祖志玄,陪葬昭陵,圖形凌煙閣。祖德皎,贈給事中。父諤,循州刺史,贈左僕射。文昌家於荊州,倜儻有氣義,節度使裴胄知之而不能用。韋皋在蜀,表授校書郎。李吉甫刺忠州,文昌嘗以文乾之。及吉甫居相位,與裴垍同加獎擢,授登封尉、集賢校理。俄
 拜監察御史,遷補闕,改祠部員外郎。元和十一年,守本官,充翰林學士。



 文昌,武元衡之子婿也。元衡與宰相韋貫之不協,憲宗欲召文昌為學士,貫之奏曰:「文昌志尚不修,不可擢居近密。」至是貫之罷相,李逢吉乃用文昌為學士,轉祠部郎中,賜緋,依前充職。十四年,加知制誥。十五年,穆宗即位,正拜中書舍人,尋拜中書侍郎、平章事。



 長慶元年,拜章請退。朝廷以文昌少在西蜀,詔授西川節度使、同中書門下平章事。文昌素洽蜀人之情,至
 是以寬政為治,嚴靜有斷,蠻夷畏服。二年,雲南入寇,黔中觀察使崔元略上言,朝廷憂之,乃詔文昌御備。文昌走一介之使以喻之,蠻寇即退。



 敬宗即位,徵拜邢部尚書,轉兵部,兼判左丞事。



 文宗即位,遷御史大夫,尋檢校尚書右僕射、揚州大都督府長史、同平章事、淮南節度使。太和四年,移鎮荊南。



 文昌於荊、蜀皆有先祖故第,至是贖為浮圖祠。又以先人墳墓在荊州,別營居第,以置祖禰影堂,歲時伏臘,良辰美景享薦之。徹祭,即以音聲
 歌舞繼之,如事生者,搢紳非焉。



 六年,復為劍南西川節度。九年三月,賜春衣中使至,受宣畢,無疾而卒,年六十三,贈太尉。有文集三十卷。



 文昌布素之時,所向不偶。及其達也,揚歷顯重,出入將相,洎二十年。其服飾玩好、歌童妓女,茍悅於心,無所愛惜,乃至奢侈過度,物議貶之。子成式。



 成式,字柯古,以廕入官,為秘書省校書郎。研精苦學,秘閣書籍,披閱皆遍。累遷尚書郎。咸通初,出為江州刺史。解印,寓居襄陽,以閑放自適。家多書史,用以自
 娛,尤深於佛書。所著《酉陽雜俎》傳於時。



 宋申錫,字慶臣。祖素,父叔夜。申錫少孤貧,有文學。登進士第,釋褐秘書省校書郎。韋貫之罷相,出湖南,闢為從事。其後累佐使府。長慶初,拜監察御史。二年,遷起居舍人。寶歷二年,轉禮部員外郎,尋充翰林侍講學士。



 申錫始自策名,及在朝行,清慎介潔,不趨黨與。當長慶、寶歷之間,時風囂薄,朋比大扇。及申錫被用,時論以為激勸。



 文宗即位,拜戶部郎中、知制誥。太和二年,正拜中書舍
 人,復為翰林學士。



 初,文宗常患中人權柄太盛,自元和、寶歷,比致宮禁之禍。及王守澄之領禁兵,恃其宿舊,跋扈尤甚。有鄭注者,依恃守澄為奸利,出入禁軍,賣官販權,中外咸扼腕視之。文宗雅知之,不能堪。申錫時居內廷,文宗察其忠厚,可任以事。嘗因召對,與申錫從容言及守澄,無可奈何,令與外廷朝臣謀去之,且約命為宰相。申錫頓首謝之。未幾,拜左丞。逾月,加平章事。申錫素能謹直,寵遇超輩,時情大為屬望。及到中書,剖斷循常,
 望實頗不相副。



 太和五年,忽降中人召宰相入赴延英。路隨、李宗閔、牛僧孺等既至中書東門,中人云:「所召無宋申錫。」申錫始知被罪,望延英以笏叩頭而退。隨等至,文宗以神策軍中尉王守澄所奏,得本軍虞候豆盧著狀,告宋申錫與漳王謀反,隨等相顧愕然。初,守澄於浴堂以鄭注所構告於文宗,守澄即時於市肆追捕,又將以二百騎就靖恭里屠申錫之家。會內官馬存亮同入,諍於文宗曰:「謀反者適宋申錫耳,何不召南司會議。今
 卒然如此,京師企足自為亂矣。」守澄不能難,乃止。乃召三相告之。又遣右軍差人於申錫宅捕孔目官張全真、家人買子緣信等。又於十六宅及市肆追捕胥吏,以成其獄。文宗又召師保、僕射、尚書丞郎、常侍、給事、諫議、舍人、御史中丞、京兆尹、大理卿,同於中書及集賢院參驗其事。



 翌日,開延英,召宰臣及議事官,帝自詢問。左常侍崔玄亮,給事中李固言,諫議大夫王質,補闕盧鈞、舒元褒、羅泰、蔣系、裴休、竇宗直、韋溫,拾遺李群、韋端符、丁居
 晦、袁都等一十四人,皆伏玉階下奏以申錫獄付外,請不於禁中訊鞫。文宗曰:「吾已謀於公卿大僚,卿等且出。」玄亮固言,援引今古,辭理懇切。玄亮泣涕久之,文宗意稍解,貶申錫為右庶子,漳王為巢縣公。再貶申錫為開州司馬。



 初,申錫既得密旨,乃除王璠為京兆尹,以密旨喻之。璠不能謀,而注與守澄知之,潛為其備。漳王湊,文宗之愛弟也,賢而有人望。豆盧著者,職屬禁軍,與注親表。文宗不省其詐,乃罷申錫為庶子。時京城恟々,眾庶
 嘩言,以為宰相真連十宅謀反,百僚震駭。居一二日,方審其詐。諫官伏閣懇論,文宗震怒,叱諫官令出者數四。時中外屬望大僚三數人廷辯其事。僕射竇易直曰:「人臣無將,將而必誅。」聞者愕然。唯京兆尹崔琯、大理卿王正雅連上疏請出內獄,且曰:「王師文未獲,即獄未具,請出豆盧著與申錫同付外廷勘。」當時人情翕然推重。初議申錫抵死,顧物論不可,又將投於嶺表。文宗終悟外廷之言,乃有開州之命。



 初,申錫既被罪,怡然不以為意,
 自中書歸私第,止於外,素服以俟命。其妻出謂之曰:「公為宰相,人臣位極於此,何負天子反乎?」申錫曰:「吾生被厚恩,擢相位,不能鋤去奸亂,反為所羅織,夫人察申錫,豈反者乎?」因相與泣下。



 申錫自居內廷,及為宰相,以時風侈靡,居要位者尤納賄賂,遂成風俗,不暇更方遠害,且與貞元時甚相背矣。申錫至此,約身謹潔,尤以公廉為己任,四方問遺,悉無所受。既被罪,為有司驗劾,多獲其四方受領所還問遺之狀,朝野為之嘆息。



 七年七
 月,卒於開州。詔曰:「申錫雖不能周慎,自抵憲章,聞其亡歿遐荒,良用悲惻。宜許其歸葬鄉里,以示寬恩。」開成元年九月,詔復申錫正議大夫、尚書左丞、同中書門下平章事、上柱國,賜紫,兼贈兵部尚書。仍以其子慎微為城固縣尉。



 李程,字表臣,隴西人。父鷫伯。程,貞元十二年進士擢第,又登宏辭科,累闢使府。二十年,入朝為監察御史。其年秋,召充翰林學士。



 順宗即位,為王叔文所排,罷學士。三
 遷為員外郎。元和中,出為劍南西川節度行軍司馬。十年,入為兵部郎中,尋知制誥。韓弘為淮西都統,詔程銜命宣諭。明年,拜中書舍人,權知京兆尹事。十二年,權知禮部貢舉。十三年四月,拜禮部侍郎。六月,出為鄂州刺史、鄂岳觀察使。入為吏部侍郎,封渭源男,食邑三百戶。敬宗即位之五月,以本官同平章事。



 敬宗沖幼,好治宮室,畋游無度,欲於宮中營新殿。程諫曰:「自古聖帝明王,以恭儉化天下。陛下在諒闇之中,不宜興作,願以瓦木
 回奉園陵。」上欣然從之。程又奏請置侍講學士,數陳經義。程辯給多智算,能移人主之意。尋加中書侍郎,進封彭原郡公。寶歷二年,罷相,檢校兵部尚書、同平章事、太原尹、北京留守、河東節度使。太和四年三月,檢校尚書左僕射、平章事、河中尹、河中晉絳節度使。



 六年,就加檢校司空。七月,徵為左僕射。中謝日奏曰:「臣所忝官上禮,前後儀注不同。在元和、長慶中,僕射數人上日,不受四品已下官拜。近日再定儀注,四品已下官悉許受拜,王
 涯、竇易直已行之於前。今御史臺云:『已聞奏,太常侍定取十五日上』。臣進退未知所據。」時中丞李漢以為受四品已下拜太重。敕曰:「僕射上儀,近已詳定。所緣拜禮,皆約令文,已經施行,不合更改。宜準太和四年十一月六日敕處分。」



 程藝學優深,然性放蕩,不修儀檢,滑稽好戲,而居師長之地,物議輕之。七年六月,檢校司空、汴州刺史、宣武軍節度使。九年,復為河中晉絳節度使,就加檢校司徒。開成元年五月,復入為右僕射,兼判太常卿事。
 十一月,兼判吏部尚書銓事。二年三月,檢校司徒,出為襄州刺史、山南東道節度使。卒,有司謚曰繆。子廓。



 廓進士登第,以詩名聞於時。大中末,累官至潁州刺史,再為觀察使。廓子晝,亦登進士第。



 史臣曰:宗儒、易直,以寬柔養望,坐致公臺;與時沉浮,壽考終吉,可謂能奉身矣。逢吉起徒步而至鼎司,欺蔽幼君,依憑內豎,蛇虺其腹,毒害正人,而不與李訓同誅,天道福淫明矣。申錫小器大謀,貶死為幸。程不持士範,歿
 獲醜名。君子操修,豈宜容易!



 贊曰:趙、竇優柔,坐享公侯。蝮蛇野葛,逢吉之流。豈無令人?主輔謨猷。程、錫弼諧,於道難周。



\end{pinyinscope}