\article{卷十九下 本紀第十九下 僖宗}

\begin{pinyinscope}

 僖宗惠聖恭定孝皇
 帝諱儇,懿宗第五子,母曰惠安皇后王氏。咸通三年五月八日生於東內。初封普王,名儼。十四年七月,懿宗大漸。其月十八日,制曰:「朕守大器之重,居兆人之上,日慎一日,如履如臨。旰昃勞懷,寢興思理,涉道猶淺,導代未孚。而攝養乖方,寒暑成癘,實有慮於闕政,且無暇於怡神。恙未少瘳,日加浸劇,萬務凡總,須有主張。考思舊章,謀於卿士,思闡鴻業,式建皇儲。第五男普王儼改名儇,孝敬溫恭,寬和博厚,日新令德,天假英姿,言皆中規,動必由禮。俾崇邦本,允協人心,宜立為皇太子,權勾當軍國政事。咨爾中外卿士,洎於腹心之臣,敬保予胤,輔成予志,各竭乃心,以安黎庶。布告中外,知朕意焉。」是日,懿宗崩。二十日,即皇帝位於柩前,時年十二。左軍中尉劉行深、右軍中尉韓文約居中執政,並封國公。



 八月,皇帝釋服。冊聖母王氏為皇太后。河南大水,自七月雨不止,至釋服後方霽。



 九月,守司空、門下侍郎、平章事韋保衡貶賀州刺史。以岳州刺史於琮為太子少傅,緣琮貶逐者並放還。循州司戶崔沆復為中書舍人,前戶部侍郎、知制誥、翰林學士承旨鄭畋為左散騎常侍,前兵部侍郎、知制誥、
 翰林學士張裼為太子賓客,前諫議大夫高湘復為諫議大夫,前宣歙觀察使楊嚴復為給事中。十月,左僕射、門下侍郎、平章事劉鄴檢校左僕射、同平章事,兼揚州大都督府長史,充淮
 南節度觀察副大使、知節度事。



 十一月,以光祿大夫、守太子少傅、駙馬都尉於琮檢校尚書左僕射,兼襄州刺史、御史大夫,充山南東道節度觀察等使。十二月,雷震。義成軍節度使、檢校刑部尚書杜慆就加兵部尚書。



 乾符元年春正月辛酉朔。乙丑,左僕射、門下侍郎、平章事蕭仿兼右僕射。門下侍郎、吏部尚書、平章事王鐸檢校吏部尚書、同平章事,兼汴州
 刺史,充宣武軍節度、宋亳觀察等使。二月,葬懿宗於簡陵。



 三月,以河東節度使、檢校尚書右僕射崔彥昭為尚書兵部侍即,充諸道鹽鐵轉運等使。以銀青光祿大夫、京兆尹、上柱國、岐山郡開國公、食邑三千戶竇浣檢校戶部尚書、太原尹、北都留守、御史大夫,充河東節度管內觀察處置等使。以中書侍郎、刑部尚書、同平章事趙隱檢校吏部尚書、潤州刺史、浙江西道都團練觀察等使。四月,崔彥昭本官
 同平章事,領使
 如故。以前淮南節度使李蔚為吏部尚書。以天平
 軍節度使、檢校尚書右僕射、兼鄆州刺史高駢檢校司空,兼成都尹,充劍南西川節度副大使、知節度事。以右散騎常侍韋荷為吏部侍郎。前同州刺史崔璞為右散騎常侍。右領軍衛上將軍渾僖檢校吏部尚書、左千牛衛上將軍。以侍御史盧胤徵為司封員外郎,判戶部案。



 五月,以吏部侍郎鄭畋為兵部侍郎、同平章事,戶部侍郎、知制誥、翰林學士、賜紫金魚袋
 盧攜本官同平章事。太子右庶子李嶧為太僕卿,侍御史裴渥為起居郎。以嶺南東道節度使、檢校刑部尚書鄭從讜為刑部尚書,以吏部侍郎韋荷
 檢校
 禮部尚書、廣州刺史、嶺南東道節度使。七月,以禮部侍郎裴瓚為檢校左散騎常侍、潭州刺史、御史大夫、湖南觀察使;故湖南觀察使李庾贈禮部尚書。十月,以中書舍人崔沆為中書侍郎,右諫議大夫崔胤為給事中。



 十一月丙戌朔。庚寅,上有
 事於宗廟,禮畢,御丹鳳門,大赦,改元為乾符。宰相蕭仿兼司空、弘文館大學士、太清宮使,兵部侍郎崔彥昭為中書侍郎,兵部侍郎鄭畋為集賢殿大學士。以宣慰沙陀六州部落、檢校兵部尚書李鈞為靈武節度,制曰:「朕以沙陀驍勇,重累戰功,六州蕃、渾,沐浴王化。念其出於猜貳,互有傷殘,而克璋報仇,其意未已。被我君臨之德,軫吾子育之心,爰擇良能,俾之宣撫。惟爾先正,嘗鎮北門,待國昌以雄傑之才,置國昌於濟活之地。既藉奕葉
 之舊,又懷任土之觀。是用付以封疆,委之軍旅,必集王事,無墜家聲。」初鈞父業鎮太原,能安集代北部落。時李國昌父子據大同、振武,吐渾、契苾、幽州諸道之軍攻之不利,故假鈞靈武節鉞,率師招諭之。以長安令李壁為諫議大夫,以吏部員外郎徐彥若為長安令。兵部郎中盧鄯為楚州刺史。十二月,黨項、回鶻寇邊。以左司郎中崔原為兵部郎中,江州刺史李可仁為右司郎中。權知工部尚書牛蔚為禮部尚書,太子賓客於浱為工部尚
 書。是冬,南詔蠻寇蜀,詔河西、河東、山南西道、東川征兵赴援。西川節度使高駢奏:「奉敕抽發長武、鄜州、河東等道兵士赴劍南行營者。伏以西川新軍舊軍差到已眾,況蠻蜒小醜,必可枝梧。今以道路崎嶇,館驛窮困,更有軍頓,立見流移,所謂望一處完全而百處俱破。且兵不在眾而在於和,其左右神策長武鎮、鄜州、河東所抽甲馬兵士,人數不少,況備辦軍食,費損尤多。又緣三道籓鎮,盡扼羌戎,邊鄙未寧,望不差發。如已在道路,並請降
 敕勒回。」詔答曰:「蠻蜒如尚憑凌,固須倍兵禦敵;若已奔退,即要並力追擒。方藉北軍,助平南寇,其三處兵士,宜委高駢候到蜀日分布驅使。具務多多之辦,寧亂整整之師。其河東一千二百人,令竇浣不要差發。」時駢捍蠻已退,長武兵士竟至蜀而還,議者惜其勞費而虛邀出入之賞也。右軍中尉韓文約以疾乞休致,從之。



 二年春正月乙酉朔。己丑,宰相崔彥昭率文武百僚上尊號,上御正殿受冊。以知內樞密田令孜為右軍中尉。
 南蠻驃信遣使乞盟,許之。以鳳州刺史郭弘業為左金吾衛將軍。庫部郎中韋岫為泗州刺史,都官員外郎李頻為建州刺史。



 二月,以兵部侍即、充諸道鹽鐵轉運使王凝為秘書監,以所補吏職罪也。以吏部侍郎裴坦為兵部侍郎,充諸道鹽鐵轉運使。以翰林學士崔澹為中書舍人;翰林學士徐仁嗣為司封郎中,學士如故。以容管經略招討使高秦檢校戶部尚書,太府卿李嶧為宗正卿,湖州刺史張搏為盧州刺史,庫部員外郎楊堪為
 吏部員外郎。



 三月,以右補闕鄭勤為起居郎,度支推官牛徽為右補闕。以戶部郎中崔彥融為長安令,都官郎中楊知退為戶部郎中。左司員外郎唐嶠為刑部郎中,刑部員外郎畢紹顏為左司員外郎,侍御史鄭頊為刑部員外郎。四月,海賊王郢攻剽浙西郡邑。以殿中侍御史李燭為禮部員外郎。以太子賓客張裼為吏部侍郎。前淮南節度使李蔚為太常卿,成德軍節度使王景崇加開府儀同三司。秘書監蕭峴為國子祭酒。汝州刺史
 崔彥沖為太子賓客分司。新除吏部侍郎張裼為京兆尹。東川點檢兵馬使吳行魯可金紫光祿大夫、檢校兵部尚書,兼梓州刺史、御史大夫,充劍南東川節度等使。以東川節度使、檢校戶部尚書崔充為河南尹;河南尹李晦檢校左散騎常侍,兼福州刺史、福建都團練觀察使。以鳳翔隴西節度使、檢校司徒、同平章事、上柱國、涼國公、食邑三千戶令狐綯進封趙國公。



 五月,濮州賊首王仙芝聚於長垣縣,其眾三千,剽掠閭井,進陷濮州,俘
 丁壯萬人。鄆州節度使李種出兵擊之,為賊所敗。以殿中少監薛璫為衛州刺史,國子司業裴拙為洋州刺史,中書舍人崔沆為禮部侍郎,兵部郎中裴虔餘為太常少卿。六月,以司勛員外郎薛邁為兵部郎中,戶部員外郎鄭就為司勛員外郎,倉部員外郎鄭綮為戶部員外郎,主客員外郎王鐐為倉部員外郎。



 秋七月,以大理卿蔡行為豐州刺史、天德軍都防禦使,大理卿張彥遠為大理卿。以京兆尹張裼檢校戶部尚書,兼鄆州刺史、御
 史大夫,充天平軍節度、鄆曹濮觀察等使。以左司勛員外郎杜貞符為都官郎中,吏部員外郎牛循為金州刺史,司封員外郎盧胤徵為吏部員外郎。十月,以秘書少監李貺為諫議大夫。以前大同軍及雲朔都防御營田供軍等使李璫檢校左散騎常侍、豐州刺史,充天德軍豐州西城中城都防禦使、本管押蕃落等使。以考功員外郎趙蘊為吏部員外郎,戶部員外郎盧莊為起居員外郎,禮部員外郎蕭遘為考功員外郎。



 十一月,以起居
 郎劉崇龜為禮部員外郎,殿中侍御史孔綸為戶部員外郎。是月,雷震電。左僕射王鐸兼門下侍郎、同平章事,復輔政。



 三年春正月己卯朔,司空、門下侍郎、同平章事蕭仿以病求免,罷為太子太傅。浙西奏誅王郢徒黨。以左金吾衛大將軍、右街使齊克讓檢校兵部尚書,兼袞兗沂海等州節度使。



 三月,以吏部尚書歸仁晦、吏部侍郎孔晦、吏部侍郎崔蕘試宏詞選人,考功郎中崔庾、考功員外郎
 周仁舉為考官。以太常卿李蔚本官同平章事。奉天鎮上言金龍晝見,自河升天。門下侍郎崔彥昭太清宮使、弘文館大學士,中書侍郎、刑部尚書、平章事鄭畋監修國史。以右武衛大將軍墨沖謙為左金吾衛大將軍,以黎州刺史杜岡為雅州刺史。



 五月,以江西觀察使獨孤雲為太子少傅,金州刺史束鄉勵為嘉州刺史。



 六月,敕福建觀察使李播、荊州刺史楊權古、蔚州刺史王龜範、璧州刺史張贄、濮州刺史韋浦、施州刺史婁傅會、刑州
 刺史王回、撫州刺史崔理、黃州刺史計信卿等:「刺史親人之官,茍不諳詳,豈宜除授。比為朕養百姓,非獨榮爾一身,每念疲羸,實所傷嘆。李播等九人授官之時,眾詞不可;王回等三人到郡無政,惟務貪求。實污方州,並宜停任。」以檢校右散騎常侍、衛尉卿李鐸為太府卿,以涼王傅分司裴思謙為衛尉卿,撫王府長史劉允章涼王傅。主客郎中崔福為汾州刺史,荊南節度副使王慥為主客郎中。六月,以門下侍郎、刑部尚書、平章事、太清宮使、
 弘文館大學士、判度支崔彥昭兼左僕射,中書侍郎鄭畋兼門下侍郎,太常卿、平章事李蔚為中書侍郎。以歙州刺史蕭騫為右司員外郎,右司員外郎崔潼為歙州刺史。七月,草賊王仙芝寇掠河南十五州,其眾數萬。是月,賊逼潁、許,攻汝州,下之,虜刺史王鐐。刑部侍郎劉承雍在郡,為賊所害。賊遂南攻唐、鄧、安、黃等州。時關東諸州府兵不能討賊,但守城而已。以戶部郎中李節為駕部郎中,金部郎中王慥為戶部郎中,主客郎中鄭諴為
 金部郎中,金部員外郎張譙為主客郎中,屯田員外郎竇珝為金部員外郎,京兆司錄趙曄為屯田員外郎。工部侍郎崔朗為同州刺史,左軍擗仗使、左監門衛上將軍西門思恭為右威衛上將軍。以右諫議大夫、知制誥魏簹為中書舍人。



 九月,以右丞崔蕘權知吏部侍郎,禮部侍郎崔沆為尚書右丞,中書舍人高湘權知禮部侍郎,京兆尹楊知至為工部侍郎。兵部尚書、兼太常卿李榼檢校尚書右僕射、太常;衛尉卿蕭寬為鴻臚卿,充閑
 廄使。以宰相崔彥昭男保謙為秘書省校書郎。右僕射、門下侍郎、平章事崔彥昭加特進;門下侍郎、禮部尚書、平章事鄭畋可特進。太中大夫、平章事盧攜可銀青光祿大夫;銀青光祿大夫、平章事李蔚可金紫光祿大夫。以太府卿李嶧檢校工部尚書、滑州刺史、御史大夫,充義成軍節度、鄭滑潁觀察處置等使。雅州自六月地震至七月未止,壓傷人頗眾。詔河南籓鎮舉兵討賊。以刑部郎中李磎為戶部郎中,分司東都;戶部郎中鄭諴為
 刑部郎中。戶部郎中、知制誥、翰林學士王徽為中書舍人,戶部員外郎、翰林學士蕭遘為戶部郎中,學士並如故。諫議大夫趙蒙為給事中,商州刺史張同為諫議大夫。



 十一月,以司門員外郎鄭蕘為池州刺史,水部員外郎樊充為工部員外郎,汴宋度支使杜孺休為水部員外郎。太常少卿崔渾貶康州刺史,揚州左司馬鄭祥為澧州刺史,度支分巡院使李仲章為建州刺史。十二月,以右金吾衛將軍張簡會為左金吾大將軍,充右街使;右
 龍武將軍李弢為右金吾將軍。前陜西虢觀察使陸墉為太子賓客。



 四年春正月癸酉朔。丁丑,降制赦天下系囚及徒流人放還。以諫議大夫李湯為給事中,以兵部郎中崔厚為諫議大夫。大理少卿王承顏為鹽州刺史,明州刺史殷僧辯為大理卿。以吏部尚書鄭從讜、吏部侍郎孔晦、吏部侍郎崔蕘考宏詞選人。



 三月,以開府、行內侍監致仕劉行深為內侍省觀軍容、守內侍監致仕。以判鹽鐵案、
 檢校考功郎中鄭溵為司封員外郎,充轉運判官。兵部員外郎裴渥為蘄州刺史,職方員外郎盧澄為兵部員外郎。以草賊大寇河南、山南,詔曰:



 亂常幹紀,天地所不容;伐罪吊人,帝王之大典。歷觀往代,遍數前朝,其有怙眾稱兵,憑兇構孽,或疑迷於郡縣,或殘害於生靈。初則狐假鴟張,自謂驍雄莫敵;旋則鳥焚魚爛,無非破敗而終。蓋以逆順相懸,幽明共怒。近者龐勛拒命,王郢挻災,結聚至多,猖狂頗甚,尋則身膏原野,家受誅夷。亦有方
 從叛亂,能自徊翔,移吉兇於反掌之間,變禍福於立談之際。則諸葛爽今為刺史,硃實見存將軍,弘霸郎受職於禁營,宋再雄策名於淮海,莫不身名光顯,家族輝榮。近準諸道奏報,草賊稍多,江西、淮南,宋、亳、曹、潁,或攻郡縣,或掠鄉村。雖命兵師,且令招撫。朕以寬弘為理,慈愍居心,每念蒼生,皆同赤子。恨不能均其衣食,令致荒饑,寧忍迫以鋒芒,斷其身首。如王仙芝及諸賊頭領能洗心悔過,散卒休兵,所在州府投降,便令具名聞奏,朝廷
 當議獎升。如諸賊頑傲不悛,兇強自恃,即宜令諸道兵師掎角誅剪。若諸軍全捕得一火草賊數至三百人已上者,超授將軍,賞錢一千貫。如鄉材有干勇才略,而能率合義徒,驅除草寇者,本處以聞,亦與重賞。如鄭鎰、湯群之輩,已為刺史,朝廷故不食言。敕到,宜令諸道明行宣諭,令知朕意。青州節度使宋威上表:「請步騎五千,特為一使,兼率本道兵士,所在討賊,必立微功以酬聖獎。」優詔嘉之,乃授威諸道招討草賊使,仍給禁兵三千,甲
 馬五百匹。仍諭河南方鎮曰:「王仙芝本為鹽賊,自號草軍,南至壽、盧,北經曹、宋。半年燒劫,僅十五州;兩火轉鬥,逾七千眾。諸道發遣將士,同共討除,日月漸深,煙塵未息。蓋以遞相觀望,虛費餱糧,州縣罄於供承,鄉材泣於侵暴。今平盧軍節度使宋威深憤萑蒲,請行誅討。朕以威前時蜀部,破南詔之全軍;比歲徐州,摧龐勛之大陣。官階甚貴,可以統諸道之都頭;驍勇素彰,足以破伏戎之草寇。今已授指揮諸道兵馬招討草賊使,侯宋威到
 本道日,供給犒設,並取上供錢支給。仍命指揮都頭,凡攻討進退,取宋威處分。」時賊渠王仙芝、尚君長在安州,宋威自青州與副使曹全晸進軍攻討,所在破賊。是月,冤朐賊黃巢聚萬人攻鄆州,陷之,逐節度使薛崇。



 五月,幽州節度使李茂勛上表乞致仕,以其男可舉權知兵馬事。制以壽王傑為開府儀同三司、幽州經略盧龍等軍節度觀察押奚契丹等使;以幽州節度副使、權知兵馬事李可舉檢校左散騎常侍、幽州大都督府左司馬,
 充幽州兵馬留後。制以幽州盧龍節度使、檢校工部尚書李茂勛守尚書左僕射致仕。以前綿州刺史皇甫鏞為秘書少監,以陳州刺史許珂為睦州刺史,以右衛將軍程可復為左衛大將軍。黃巢賊陷沂州。



 六月,以宣歙觀察使高駢檢校司空,兼潤州刺史、鎮海軍節度、蘇常杭潤觀察處置、江淮鹽鐵轉運、江西招討等使。以汝州防禦使李鈞檢校尚書右僕射、潞州大都督府長史,充昭義軍節度、潞刑洺磁觀察等使。幽州留後李可舉請
 以本軍討沙陀三部落,從之。七月,黃巢自沂、海,其徒數萬,趨潁、蔡,入查牙山,遂與王仙芝合。



 八月,賊陷隨州,執刺史崔休徵。群賊屯於白洑。是月,江州賊首柳彥璋聚徒陷江州,殺刺史陶祥。



 九月,以中書舍人崔澹權知貢舉。沙陀大寇雲、朔。十月,詔昭義節度李鈞、幽州李可舉、吐渾赫連鐸白義誠、沙陀安慶薛葛部落合兵討李國昌父子於蔚州。



 十一月,賊王仙芝率眾渡漢,攻江陵,節度使楊知溫嬰城拒守。知溫本非禦侮之才,城無宿備,
 賊急攻之。十二月,賊陷江陵之郛,知溫窮蹙,求援於襄陽,山南東道節度使李福悉其師援之。時沙陀軍五百騎在襄陽,軍次荊門,騎軍擊賊,敗之。賊盡焚荊南郛郭而去。



 五年春正月丁酉朔,沙陀首領李盡忠陷遮虜軍。太原節度使竇浣遣都押衙康傳圭率河東土團二千人屯代州,將發,求賞呼噪,殺馬步軍使鄧虔。竇浣自入軍中安慰,仍借率富戶錢五萬貫以賞之。朝廷以浣非禦侮
 才,以前昭義節度使曹翔檢校尚書右僕射,兼太原尹、北都留守、河東節度使;又以左散騎常侍支謨為河東節度副使。



 二月,王仙芝餘黨攻江西,招討使宋威出軍屢敗之,仍宣詔書諭仙芝。仙芝致書於威,求節鉞,威偽許之。仙芝令其大將尚君長、蔡溫玉奉表入朝,威乃斬君長、溫玉以徇。仙芝怒,急攻洪州,陷其郛。宋威赴援,與賊戰,大敗之,殺仙芝,傳首京師。尚君長弟尚讓為黃巢黨,以兄遇害,乃大驅河南、山南之民,其眾十萬,大掠淮
 南,其鋒甚銳。侍中、晉國公王鐸請自督眾討賊,天子以宋威失策殺君長,乃以王鐸檢校司徒、兼侍中、門下侍郎、江陵尹、荊南節度使,充諸道兵馬都統。



 三月,王鐸奏兗州節度使李系為統府左司馬,兼潭州刺史,充湖南都團練觀察使。黃巢之眾再攻江西,陷虔、吉、饒、信等州,自宣州渡江,由浙東欲趨福建,以無舟船,乃開山洞五百里,由陸趨建州,遂陷閩中諸州。以吏部尚書鄭從讜、吏部侍郎崔沆考宏詞選人。七月,滑州、忠武、昭義諸道
 之師會於太原,大同軍副使支謨為前鋒,先趨行營。



 八月,沙陀陷岢嵐軍,曹翔自率軍赴忻州。翔至軍,中風而卒,諸軍皆退。太原大懼,閉城門,昭義兵士為亂,劫坊市。



 九月,門下侍郎、吏部尚書、平章事李蔚檢校尚書左僕射,充東都留守;以吏部尚書鄭從讜本官同平章事。十月,司空、平章事崔彥昭罷為太子太傅。



 十一月,制以河東宣尉使、權知代北行營招討崔季康檢校戶部尚書,兼太原尹、北都留守,充河東節度、代北行營招討使。沙
 陀攻右州,崔季康救之。十二月,季康與北面行營招討使李鈞,與沙陀李克用戰於岢嵐軍之洪谷,王師大敗,鈞中流矢而卒。戊戌,至代州,昭義軍亂,為代州百姓所殺殆盡。以中書舍人張讀權知禮部貢舉。



 六年春正月辛卯朔,河東節度使崔季康自靜樂縣收合餘眾回軍,軍亂,殺孔目官石裕。季康委眾遁歸行營,衙將張鍇、郭朏率其眾歸太原,兵士鼓噪,攻東陽門,入使衙,季康父子皆被害。



 三月,以吏部侍郎崔沆、崔澹試
 宏詞選人,駕部郎中盧蕰、刑部郎中鄭頊為考官。制以邠寧節度使李偘檢校戶部尚書,兼太原尹、北都留守,充河東節度等使。四月,黃巢陷桂管。



 五月,賊圍廣州,仍與廣南節度使李巖、浙東觀察使崔璆書,求保薦,乞天平節鉞。璆巖上表論之,詔公卿議其可否。宰相鄭畋、盧攜爭論於中書,詞語不遜,俱罷為太子賓客,分司東都。以吏部侍郎崔沆為戶部侍郎,戶部侍郎、翰林學士豆盧彖為兵部侍郎,並本官同平章事。黃巢陷廣州,大掠嶺南郡邑。



 八
 月,制以特進、檢校司空、東都留守李蔚為檢校司徒、同平章事,兼太原尹、北都留守、河東節度觀察,兼代北行營招討供軍等使。十月,制以鎮海軍節度、浙江西道觀察處置等使高駢檢校司徒、同平章事、揚州大都督府長史,充淮南節度副大使、知節度事、江淮鹽鐵轉運、江南行營招討等使,進封燕國公,食邑三千戶。初,駢在浙西,遣大將張璘、梁績等大破黃巢於浙東,賊進寇福建,逾嶺表,故移鎮揚州。時賊北逾大庾嶺,朝廷授駢諸道
 行營兵馬都統。太原節度使李蔚卒。以禮部侍郎張讀權知左丞事。



 十一月,制以銀青光祿大夫、檢校右散騎常侍、河東行軍司馬、雁門代北制置等使、石嶺鎮北兵馬、代北軍等使、上柱國康傳圭檢校工部尚書,兼太原尹、北都留守、河東節度使。時傳圭已率兵在代州,是月自行營赴任,兩都虞候張鍇、郭朏迎於鳥城驛,並殺之,軍中震悚。又制以神策大將軍周寶檢校尚書左僕射,兼潤州刺史、鎮海軍節度、浙江西道觀察等使。以定州
 已來制置內閑廄宮苑等使、金紫光祿大夫、檢校刑部尚書、上柱國、太原縣開國伯、食邑七百戶王處存檢校戶部尚書,兼定州刺史,充義武軍節度、易定觀察處置、北平軍等使。十二月,制以河東馬步軍都虞候硃玫為代州刺史。以太子賓客分司盧攜為兵部尚書、同平章事;太子賓客鄭畋檢校左僕射、鳳翔尹,充鳳翔節度使。



 廣明元年春正月乙卯朔,上御宣政殿,制曰:



 朕祗膺寶祚,嗣守宗祧,夙夜一心,勤勞八載,實欲驅黎元於仁壽,
 致華夏之升平。而國步猶艱,群生寡遂,災迍薦起,寇孽仍臻。竊弄干戈,連攻郡邑,雖輸降款,未息狂謀。江右、海南,瘡痍既甚,湖湘荊漢,耕織屢空。言念疲贏,良深軫惻。我心未濟,天道如何。賴近者嚴敕師徒,稍聞勝捷,皆明聖之潛祐,寧菲德以言功。屬節變三陽,日當首歲,乃御正殿,爰命改元,況及發生,是宜在宥。自古繼業守文之主,握圖禦宇之君,必自正月吉辰,發號施令。所以垂千年之懿範,固萬代之洪基,莫不由斯道也。可改乾符七
 年為廣明元年。近日東南州府,頻奏草賊結連。本是平人,迫於饑饉,驅之為盜,情不願為。委所在長吏子細曉諭,如自首歸降,保非詐偽,便須撫納,不要勘問。如未倒戈,即登時剪撲。東南州府遭賊之處,農桑失業,耕種不時。就中廣州、荊南、湖南,資賊留駐,人戶逃亡,傷夷最甚,自廣明已前諸色稅賦,宜令十分減四。其河中府、太原府遭賊寇掠處,亦宜準此。吏部選人粟錯及除駁放者,除身名渝濫欠考外,並以比遠殘闕收注。入仕之門,兵
 部最濫,全無根本,頗壞紀綱。近者武官多轉入文官,依資除授,宜懲僭幸,以辨品流。自今後武官不得轉入文官選改,所冀輪轅各適,秩序區分,其內司不在此限。沙陀部落逾鷹門關,進逼忻州。



 二月,沙陀逼太原,陷大谷。康傳圭遣大將伊釗、張彥球、蘇弘軫分兵拒之於秦城驛,為沙陀所敗。傳圭怒,斬蘇弘軫。張彥球部下兵士為亂,倒戈攻太原,殺傳圭,監軍使周從寓安慰方定。是月,制以開府儀同三司、門下侍郎、兼兵部尚書、同平章事、
 充太清宮使、弘文館大學士、延資庫使、上柱國、滎陽郡開國公、食邑三千戶鄭從讜檢校司空、同平章事,兼太原尹、北都留守,充河東節度、管內觀察處置兼行營招討供軍等使。黃巢賊軍自衡、永州下,頻陷湖南、江西屬郡。時都統王鐸前鋒都將李系守潭州,有眾五萬,並諸團結軍號十萬。賊自桂陽編木為筏數千,其眾乘暴水沿湘而下,徑至潭州,急攻其城,一日而陷。李系僅以身免,兵士五萬皆為賊所殺,流尸塞江。賊將尚讓乘勝沿
 流而下,進逼江陵。王鐸聞系軍敗,乃棄城奔襄陽。別將劉漢宏大掠江陵之民,剽肅不勝其酷,士民亡竄山谷,江陵焚剽殆盡。半月餘,賊眾方至江陵。



 三月,賊悉眾欲寇襄陽,江西招討使曹全晸與襄陽節度使劉巨容謀拒之。時營於荊門,賊軍一萬屯於團林驛。全晸命巨容悉以精甲陣於林薄之中,自以騎軍挑戰,偽不勝而遁。賊大乘之,比至荊門,其徒不成列,巨容發伏擊之,賊大潰而走。全晸鐵騎急追之,比至江陵,十俘七八。黃巢、尚
 讓以餘眾徒濟江。全晸方渡江襲賊,遽詔至,以段彥謨為江西節度使,全晸乃還。賊遂率舟軍東下,攻鄂州,陷其郛。全晸救至,賊遂轉戰江西,陷江西饒、信、杭、衢、宣、歙、池等十五州。全晸在江西。朝廷以王鐸統眾無功,乃授淮南節度使高駢為諸道兵馬行營都統。駢令大將張璘渡江討賊,屢捷。賊眾疫癘,其將李罕之以一軍投淮南,其眾稍沮。是月,沙陀寇忻、代,詔以汝州防禦使諸葛爽為北面行營副招討,率東都防禦兵士赴代州。四月
 甲申朔,大雨雹,大風拔兩京街樹十二三,東都長夏門內古槐十拔七八,宮殿鴟尾皆落。丁酉,制以檢校吏部尚書、前太常卿、上柱國、隴西郡開國公、食邑三千戶李琢為光祿大夫、檢校尚書右僕射、御史大夫,充蔚朔等州諸道行營都招討使;應東北面行營李孝昌、李元禮、諸葛爽、王重盈、硃玫等兵馬及忻、代州土團,並取琢處分。以內常侍張存禮充都糧料使,判官崔鋋充制置副使。



 六月,代北行營招討使李琢、幽州節度使李可舉、吐
 渾首領赫連鐸等軍討李克用於雲州。時克用令其大將軍傅文達守蔚州,高文集守朔州。吐渾赫連鐸遣人說高文集令歸國,文集與沙陀首領李友金、薩葛都督米海萬、安慶都督史敬存以前蔚州歸款於李琢。時克用率眾御燕軍於雄武軍。七月,沙陀三部落李友金等開門迎大軍,克用聞之,亟來赴援,為李可舉之追擊,大敗於藥兒嶺。李琢、赫連鐸又擊敗於蔚州,降文達,李克用部下皆潰,獨與國昌及諸兄弟北入達靼部。乃以
 吐渾都督赫連鐸為雲州刺史、大同軍防禦使,吐渾白義誠為蔚州刺史,薩葛米海萬為朔州刺史,加李可舉檢校司徒、同平章事。



 八月,黃巢之眾渡江寇淮南。是歲春末,賊在信州疫癘,其徒多喪。淮南將張璘急擊之,賊懼,以金啖璘,仍致書高駢乞保命歸國。駢信之,厚待其使,許求節鉞。時昭義、武寧、義武等軍兵馬數萬赴淮南,駢欲收功於己,乃奏賊已將殄,不假諸道之師,並遣還北。賊知諸軍已退,以求節鉞不獲,暴怒,與駢絕,請戰。駢
 怒,令張璘整軍擊之,為賊所敗,臨陣殺璘。賊遂乘勝渡江,攻天長、六合等縣,駢不能拒,但決陳登水自固而已。朝廷聞賊復振,大恐,詔河南諸道之師屯於溵水。官軍大集,賊未北渡。時兗州節度使齊克讓屯汝州。



 九月,徐州兵三千人赴溵水,途經許。許州節度使薛能前為徐帥,得軍民情。徐軍吏至,請館,能以徐軍懷惠,令館於州內。許軍懼徐人見襲,許州大將周岌自溵水以其戍卒還,逐薛能,自據其城。徐軍已至河陰,聞許軍亂,徐將時
 溥亦以戍兵還徐,逐節度使支詳。齊克讓懼兵見襲,亦還兗州。溵水諸軍皆散。賊聞之,十月,乃悉眾渡淮。黃巢自號率土大將軍,其眾富足,自淮已北整眾而行,不剽財貨,惟驅丁壯為兵耳。



 十一月辛亥朔。己巳,賊陷東都,留守劉允章率分司官屬迎謁之,賊供頓而去,坊市晏然。壬申,陷虢州。丙子,攻潼關,守關諸將望風自潰。十二月庚辰朔。辛巳,賊據潼關。時左軍中尉田令孜專政,宰相盧攜曲事之,相與誤謀,以至傾敗。令孜恐眾罪加己,
 請貶攜官,命學士王徽、裴徹為相。甲申,宣制以戶部侍郎、翰林學士王徽、裴徹本官同平章事。貶右僕射、門下侍郎、平章事盧攜為太子賓客。攜聞賊至,仰藥而死。是日,上與諸王、妃、後數百騎,自子城由含光殿金光門出幸山南,文武百官僚不之知,並無從行者,京城晏然。是日晡晚,賊入京城,時右驍衛大將張直方率武官十餘迎黃巢於坡頭。壬辰,黃巢據大內,僭號大齊,稱年號金統。悉陳文物,據丹鳳門偽赦。以太常博士皮日休、進士
 沈雲翔為學士。為偽赦書云:「揖讓之儀,廢已久矣,竄遁之跡,良用憮然。朝臣三品已上並停見任,四品已下宜復舊位。」以趙章為中書令,尚讓為太尉,崔璆為中書侍郎、平章事。時宰相豆盧彖崔沆、故相左僕射劉鄴、太子少師裴諗、御史中丞趙蒙、刑部侍郎李溥、故相於琮皆從駕不及,匿於閭里,為賊所捕,皆遇害。將作監鄭綦、庫部郎中鄭系義不臣賊,舉家雉經而死。



 中和元年春正月庚戌朔,車駕在興元。以翰林學士承
 旨、尚書戶部侍郎、知制誥蕭遘為兵部侍郎,充諸道鹽鐵轉運等使;尋以本官同平章事,領使如故。以宿州刺史劉漢宏為越州刺史、鎮東軍節度、浙江東道觀察處置等使。詔太原節度使鄭從讜發本道之師,與北面行營招討副使諸葛爽、代州刺史北面行營馬步都虞候硃玫、夏州將李思恭等行營諸軍,並赴京師討賊。河中馬步都虞候王重榮逐其帥李都,自稱留後。



 二月,代州北面行營都監押陳景思率沙陀、薩葛、安慶等三部落
 與吐渾之眾三萬赴援關中,次絳州。沙陀首領翟稽俘掠絳州叛還,景思知不可用,遣使詣行在,請赦李國昌父子,令討賊以贖罪,從之。



 三月,陳景思齎詔入達靼,召李克用軍屯蔚州,克用因大掠雁門已北軍鎮。以鳳翔節度使鄭畋守司空、門下侍郎、同平章事,充京西諸道行營都統,與涇原節度使程宗楚、秦州經略使仇公遇、鄜延節度使李孝昌、夏州節度使拓拔思恭等同盟起兵,傳檄天下。黃巢遣大將林言、尚讓率眾數萬寇鳳翔,
 鄭畋率師逆擊,大敗賊眾於龍尾陂。四月,以前大同軍防禦使李克用檢校工部尚書,兼代州刺史、雁門已北行營兵馬節度等使。五月,李克用赴代州,遂率蕃、漢兵萬人南出石嶺關,稱準詔赴難長安。丁巳,沙陀軍至太原,鄭從讜供給糧料。辛酉,沙陀求發軍賞錢,從讜與錢千貫,米千石。克用怒,縱兵大掠。從讜求援於振武,契苾通自率兵來赴,與沙陀戰於晉王嶺。沙陀敗走,陷榆次、陽曲而退。是日大風,天雨土。特進、尚書右僕射趙隱卒,
 贈司空。



 六月,沙陀退還代州。車駕幸成都府,西川節度使陣敬瑄自來迎奉。七月丁未朔。乙卯,車駕至西蜀。丁巳,御成都府廨,改廣明二年為中和元年,大赦天下。以兵部侍郎、判度支韋昭度本官同平章事。以侍中王鐸檢校太尉、中書令,兼滑州刺史、義成軍節度、鄭滑觀察處置,兼充京城四面行營都統;以太子太保崔安潛為副。觀軍容使西門思恭為天下行營兵馬都監押;中書侍郎、平章事、諸道鹽鐵轉運等使韋昭度為供軍使。時
 淮南節度使高駢為諸道行營都統,自車駕出幸,中使相繼促駢起軍,駢托以周寶、劉漢宏不利於己,遷延半歲,竟不出軍,乃以鐸為都統。以河中節度使王重榮為京城北面都統,義武軍節度使王處存為京城東面都統,鄜延節度使李孝昌為京城西面都統,朔方軍節度使拓拔思恭為京城南面都統。以忠武監軍使楊復光為天下行營兵馬都監,代西門思恭。許王鐸以便宜從事。遣郎官、御史分行天下,徵兵赴關內。



 八月,代北行營
 兵馬使諸葛爽、硃玫、拓拔思恭等軍屯渭橋。硃玫屯興平,為賊將王璠所擊,退保奉天。諸葛爽降賊,偽署爽河陽節度使。許州牙將秦宗權奏破賊於汝州,乃授宗權察州防禦使。昭義節度使高潯與賊將李詳戰於石橋,為賊所敗,退歸河中。賊乘勝陷同州。



 九月,澤潞高潯牙將劉廣擅還據潞州。是月,潯天井關戍將孟方立率戍卒攻劉廣,殺之。方立遂自稱留後,仍移軍鎮於邢州。制以京城四面催陣使、守兵部尚書王徽檢校左僕射,兼
 潞州大都督府長史、昭義節度、潞邢洺磁觀察等使。貶高潯端州刺史。楊復光、王重榮以河西、昭義、忠武、義成之師屯武功。鳳翔節度使鄭畋以病徵還行在,以鳳翔大將李昌言代畋為節度使,兼京城西面行營都統。十月,青州軍亂,逐節度使安師儒,立其行營將王敬武為留後。十二月,行營都統王鐸率禁軍、山南東川之師三萬至京畿,屯於盩厔。



 二年春正月甲辰朔,天下勤王之師,雲會京畿,京師食
 盡。賊食樹皮,以金玉買人於行營之師,人獲數百萬。山谷避亂百姓,多為諸軍之所執賣。



 二月,涇原大將唐弘夫大敗賊將林言於興平,俘斬萬計。王處存率軍二萬,徑入京城,賊偽遁去。京師百姓迎處存,歡呼叫噪。是日軍士無部伍,分占第宅,俘掠妓妾。賊自灞上分門復入,處存之眾蒼黃潰亂,為賊所敗。黃巢怒百姓歡迎處存,凡丁壯皆殺之,坊市為之流血。自是諸軍退舍,賊鋒愈熾。



 三月,前蔚州刺史蘇祐為沙陀所敗,棄郡投鎮州,至
 靈壽,部人為盜,祐為王景崇所殺。七月辛丑朔。丙午夜,西北方赤氣,如絳虹竟天。賊將尚讓攻宜君砦,雨雪盈尺,甚寒,賊兵凍死者十二三。



 八月庚子,賊同州防禦使硃溫殺其監軍嚴實,與大將胡真、謝瞳等來降,王鐸承制拜華州刺史、潼關防禦、鎮國軍等使。魏博節度韓簡自率軍三萬攻河陽,偽署節度使諸葛爽棄城而去,簡遣大將守河橋而還。



 九月,賊以黃鄴為華州刺史。初,賊以李詳守華州,詳與硃溫素善,及溫歸河中,黃巢遣閹
 官後冗率功臣馬千匹至華殺詳,以鄴代歸。太原諸山桃杏有花實。十月,西北方無雲而雷,名「天狗墜」。以嵐州刺史湯群為懷州刺史,時群倚沙陀為援,朝廷疑而易之。鄭從讜遣人傳官告授群,群怒,殺使者,據城,內沙陀。魏博節度使韓簡以兵攻鄆州,節度使曹全晸拒之,為簡所敗,執而殺之。全晸大將硃瑄以餘眾保鄆州,乞和於簡,簡舍之而去。



 十一月,沙陀李克用監軍陳景思以部落之眾一萬七千騎自嵐石州路赴河中。賊將李詳
 下牙隊斬華州守將歸明,王鐸用其部將王遇為華州刺史。十二月己亥朔。庚戌,成德軍節度、鎮冀深趙觀察處置等使、開府儀同三司、檢校太尉、中書令、上柱國、常山郡王、食邑六千戶王景崇卒,贈太傅,謚曰忠穆。遺表請以子鎔纘繼戎事,遂以鎔為兵馬留後。



 三年春正月戊辰朔,車駕在成都府。雁門節度使、檢校工部尚書李克用率師至河中。己巳,沙陀軍進屯沙苑之乾坑。



 二月,沙陀攻華州,刺史黃鄴出奔至石堤谷,追
 擒之。魏博節度使韓簡再興兵討河陽,諸葛爽遣大將李罕之拒之於武陟,逆擊之,魏軍大敗而還。大將樂彥禎先據魏州,韓簡為部下所殺,推彥禎為留後。就加李克用檢校尚書左僕射、忻代雲蔚等州觀察處置等使。



 三月丁卯朔。壬申,沙陀軍與賊將趙章、尚讓戰於成店,賊軍大敗,追奔至良天坡,橫尸三十里;王重榮築尸為京觀。四月丁酉朔。庚子,沙陀、忠武、義成、義武等軍趨長安,賊悉眾拒之於渭橋,大敗而還;李克用乘勝追之。己
 卯,黃巢收其殘眾,由藍田關而遁。庚辰,收復京城。天下行營兵馬都監楊復光上章告捷行在,曰:



 頃者妖興霧市,嘯聚叢祠,而岳牧籓侯,備盜不謹。謂大同之運,常可容奸;謂無事之秋,縱其長惡。賊首黃巢,因得充盈窟穴,蔓延萑蒲,驅我蒸黎,徇其兇逆。展鉏鶴以成鋒刃,殺耕牛以恣燔砲,魑魅晝行,虺蜴夜噬。自南海失守,湖外喪師,養虎災深,馴梟逆大。物無不害,惡靡不為,豺狼貽朝市之憂,瘡磐及腹心之痛。遂至毒流萬姓,盜污兩京,衣
 冠銜塗炭之悲,郡邑起丘墟之嘆。萬方共怒,十道齊攻,仗九廟之威靈,殄積年之兇醜。河中節度使王重榮神資壯烈,天賦機謀,誓立功名,志安家國。至於屯田待敵,率士當沖,收百姓十萬餘家,降賊黨三萬餘眾。法能持重,功遂晚成,久稽原野之刑,未決雷霆之怒。自收同、華,進逼京師,夕烽高照於國門,游騎頻臨於灞岸。既知四隅斷絕,百計奔沖,如窮鳥觸籠,似飛蛾赴焰。雁門節度使李克用神傳將略,天付忠貞,機謀與武藝皆優,臣節
 共本心相稱。殺賊無非手刃,入陣率以身先,可謂雄才,得名飛將。統領本軍南下,與臣同力前驅,雖在寢興,不忘寇孽。今月八日,遣衙隊將前鋒楊守宗、河中騎將白志遷、橫野軍使滿存、躡雲都將丁行存、朝邑鎮將康師貞、忠武黃頭軍使龐從等三十二都,隨李克用自光泰門先入京師,力摧兇逆。又遣河中將劉讓王瑰冀君武孫珙、忠武大將喬從遇、鄭滑將韓從威、荊南大將申屠忭、滄州大將賈滔、易定大將張仲慶、壽州大將張行方、天
 德大將顧彥朗、左神策弩手甄君楚公孫佐、橫沖軍使楊守亮、躡雲都將高周彞、忠順都將胡貞、絳州監軍毛宣伯聶弘裕等七十都繼進。賊尚為堅陣,來抗官軍。李克用率勵驍雄,整齊金革,叫噪而聲將動瓦,喑嗚而氣欲吞沙。寬列戈矛,麾軍夾擊,自卯至申,兇徒大敗。自望春宮蹙殺,至升陽殿合圍,戈不濫揮,矢無虛發。其賊即時奔遁,散入商山,徒延漏刃之生,佇作飲頭之器。伏自收平京國,三面皆立大功,若破敵摧鋒,雁門實居其首。
 其餘將佐,同效驅馳,兼臣所部二萬餘人,數歲櫛風沐雨,既茲蕩定,並錄以聞。報至,從官稱賀。



 五月,制以河中節度使、檢校尚書右僕射王重榮檢校司空、同平章事,餘如故。雁門已北行營節度、忻代蔚朔等州觀察處置等使、檢校尚書左僕射、代州刺史、上柱國、食邑七百戶李克用檢校司空、同平章事,兼太原尹、北京留守,充河東節度、管內觀察處置等使。義武軍節度使、檢校司空王處存檢校司徒、同平章事,餘如故。以檢校尚書右僕
 射、華州刺史、潼關防禦等使硃溫檢校司空,兼汴州刺史、御史大夫,充宣武節度觀察等使,仍賜名全忠。京城西北面行營都統、金紫光祿大夫、檢校司空、邠寧節度使硃玫就加同平章事,進封吳興縣侯,食邑一千戶。鄜坊節度使、金紫光祿大夫、檢校尚書右僕射東方逵就加同平章事。王鐸罷行營都統,依前檢校太師、中書令,進封晉國公,加食邑二千戶,節度觀察使如故。時中尉田令孜用事,自負帷幄之功,以鐸用兵無功,而
 由楊復光建策召沙陀成破賊之效,欲權歸北司,乃黜王而悅復光也。就加諸道行營兵馬都監楊復光開府儀同三司、弘農郡開國公,食邑三千戶,充同華等州管內制置使,仍賜號「資忠耀武匡國平難功臣。」六月乙未朔。甲子,楊復光卒於河中,其部下忠武八都都頭鹿晏弘、晉暉、王建、韓建等各以其眾散去。時復光兄復恭知內樞密,田令孜以復光立破賊功,憚而惡之,故賊平賞薄。及聞復光死,甚悅,復擯復恭,罷樞密為飛龍使。是
 月,黃巢圍陳州,營於州北五里。初,賊出藍田關,遣前鋒將孟楷攻蔡州,刺史秦宗權以兵逆戰,為楷所敗,宗權勢窘,與賊通和。孟楷移兵攻陳州,刺史趙犨示弱,伏兵擊之,臨陣斬楷。楷,賊之愛將,深惜之。黃巢怒,悉眾攻陳州。時黃巢與宗權合從,縱兵四掠,遠近皆罹其酷。時仍歲大饑,民無積聚,賊俘人為食,其砲炙處謂之「舂磨寨」,白骨山積,喪亂之極,無甚於斯。賊攻城急,徐州節度使時溥、許州周岌、汴州硃全忠皆出師護援之。七月,制以
 西川節度使、開府儀同三司、守太尉、同平章事、成都尹、上柱國、潁川郡王、食邑三千戶、實封四百戶陳敬瑄賜鐵券。詔鄭從讜赴行在。



 八月,李克用赴鎮太原。制以前振武節度、檢校司空、兼單于都護、御史大夫李國昌為檢校司徒、代州刺史、雁門已北行營節度、蔚朔等州觀察等使。十月,李國昌卒。



 十一月,蔡賊秦宗權圍許州。十二月,詔河東李克用赴援陳許。忠武大將鹿晏弘陷興元,逐節度使牛勖,自為留後。



 四年春正月癸亥朔,車駕在成都府。



 二月,河東節度使李克用將出師援陳許,河陽節度使諸葛爽以兵屯澤州拒之。



 三月壬戌朔。甲戌,克用移軍自河中南渡,東下洛陽。四月辛卯朔。甲寅,沙陀軍次許州,節度使周岌、監軍田從異以兵會戰。賊將尚讓屯太康,黃鄴屯西華,稍有芻粟。己未,沙陀分兵攻太康、西華賊砦。庚申,尚讓、黃鄴遁去,官軍得其芻粟,黃巢亦退保郾城。以兵部侍郎、判度支鄭昌圖以本官同平章事。



 五月辛酉朔。癸亥,沙
 陀追黃巢而北。丁卯,次尉氏。戊辰,大雨,平地水深三尺,溝河漲溢。賊至中牟,臨汴河欲渡,沙陀遽至,賊大駭,其黨分潰,殺傷溺死殆半。尚讓一軍降時溥,別將楊能、李讜、霍存、葛從周、張歸霸等降硃全忠,李周、楊景彪以殘眾走封丘。己巳,沙陀渡汴河,趨封丘,黃巢兄弟悉力拒戰,李克用擊敗之。獲所俘男女五萬口,牛馬萬餘,並偽乘輿、法物、符印、寶貨、戎仗等三萬計。得巢幼子,年六歲。黃巢既敗,以其殘眾東走。庚午,李克用急躡黃巢,一
 日夜行二百里,馬疲乏死者殆半。宿冤朐,糧運不及,騎軍至寡,乃與忠武監軍田從異班師。甲戌,次汴州,節度使硃全忠館克用於上源驛。全忠以克用兵力寡弱,大軍在遠,乃圖之。是夜,置酒郵舍,克用既醉,全忠以兵圍驛,縱火燒之。雷雨驟作,平地水深尺餘,克用逾垣僅免。其部下三百餘人及監軍使史敬思、書記任珪皆被害。丙子,克用至許州,率本軍還太原。庚辰,徐州將李師悅、陳景思率兵萬人追黃巢於兗州。



 六月,鄆州節度使硃瑄
 奏大敗賊於合鄉。



 秋七月己未朔。癸酉,賊將林言斬黃巢、黃揆、黃秉三人首級降時溥。初,徐將李師悅與賊戰於瑕丘,賊殊死戰,其眾殆盡。林言與巢走至太山狼虎谷之襄王村,懼追至並命,乃斬賊降師悅。壬午,捷書至行在,從官稱賀。河東節度使李克用累表訴屈,請討汴州。天子優詔和解之,就加克用階特進,封隴西郡王以悅之。自是全忠、克用有尋戈之怨。



 九月,山南西道節度使鹿晏弘為禁軍所討,棄城擁眾東出襄、鄧,大掠許州。
 晏弘大將王建、韓建、張造、晉暉、李師泰各率本軍歸朝,田令孜以建等楊復光故將,薄之,皆授諸衛將軍,惟以王建為壁州刺史。十月,關東諸鎮上章請車駕還京。



 十一月,鹿晏弘陷許州,殺周岌,自稱留後,尋為秦宗權所攻。制以義成軍節度、檢校太師、中書令、上柱國、晉國公王鐸為滄州刺史、義昌軍節度、滄德觀察處置等使。十二月丁亥朔,大明宮留守、權知京兆尹、御史大夫、京畿制置等使王徽與留司百官上表,請車駕還宮。詔以來
 年正月還京。新除滄德節度使王鐸,為魏博節度使樂彥禎害之於漳南縣之高雞泊,行從三百餘人皆遇害。



 光啟元年春正月丁巳朔,車駕在成都府。己卯,僖宗自蜀還京。



 二月丁亥朔。丙申,車駕次鳳翔。



 三月丙辰朔。丁卯,車駕至京師。己巳,御宣政殿,大赦,改元光啟。時李昌符據鳳翔,王重榮據蒲、陜,諸葛爽據河陽、洛陽,孟方立據邢、洺,李克用據太原、上黨,硃全忠據汴、滑,秦宗權據許、蔡,時溥據徐、泗,硃宣據鄆、齊曹、濮,王敬武據淄、青,高
 駢據淮南八州,秦彥據宣、歙,劉漢宏據浙東,皆自擅兵賦,迭相吞噬,朝廷不能制。江淮轉運路絕,兩河、江淮賦不上供,但歲時獻奉而已。國命所能制者,河西、山南、劍南、嶺南四道數十州。大約郡將自擅,常賦殆絕,籓侯廢置,不自朝廷,王業於是蕩然。蔡賊秦宗權侵寇籓鄰,制以徐州節度使時溥為鉅鹿王,充蔡州四面行營兵馬都統。宗權將秦賢攻汴、鄭不已,以汴州刺史硃全忠為沛郡王,充蔡州西北面行營都統。杭州刺史董昌大敗
 劉漢宏之眾,進攻越、婺、臺、明等州,下之。遂以昌為越州刺史、鎮東軍節度、浙江東道觀察等使,以杭州大將錢鏐為杭州刺史。閏三月,鎮冀節度使王鎔獻耕牛千頭,農具九千,兵仗十萬。四月乙卯朔,以開府儀同三司、右金吾衛上將軍、左街功德使、齊國公田令孜為左右神策十軍使。時自蜀中護駕,令孜招募新軍五十四都,都千人,左右神策各二十七都,分為五軍,令孜總領其權。時軍旅既眾,南衙北司官屬萬餘,三司轉運無調發之
 所,度支惟以關畿稅賦,支給不充,賞勞不時,軍情咨怨。舊日安邑、解縣兩池榷鹽稅課,鹽鐵使特置鹽官以總其事。自黃巢亂離,河中節度使王重榮兼領榷務,歲出課鹽三千車以獻朝廷。至是令孜以親軍闕供,計無從出,乃舉廣明前舊事,請以兩池榷務歸鹽鐵使,收利以贍禁軍。詔下,重榮上章論訴,言河中地窘,悉籍鹽課供軍。



 五月,制以河中節度使、檢校司徒、同平章事、河中尹、上柱國、瑯邪郡王王重榮為檢校太傅、同平章事,兼兗
 州刺史、兗沂海節度觀察處置等使,代齊克讓。以克讓檢校司徒,兼定州刺史御史大夫,充義武節度觀察、北平軍等使,代王處存。以處存依前檢校太傅、同平章事、河中尹、河中晉慈隰節度觀察等使。是月,宰臣蕭遘率文武百僚上徽號曰至德光烈孝皇帝,御宣政殿受冊,大赦。



 六月甲寅朔。丙辰,定州王處存奏:「幽州節度使李可舉、鎮州節度使王鎔各令大將率領兵士侵攻當道,臣並已殺退。」時李可舉乘天子播越,中原大亂,以河朔
 三鎮,休戚事同,惟易、定二郡為朝廷所有,乃同議攻處存以分其地。會燕將李全忠有奪帥之志,軍情相疑。全忠方圍易州,處存出奇騎以擊之,燕軍大敗。是月,全忠收合殘眾攻幽州,李可舉舉室登樓自焚而死,全忠自稱留後。滄州軍亂,逐其帥楊全玫,立衙將盧彥威為留後。制以保鑾都將、檢校司徒,兼黔州刺史、黔中節度觀察等使曹誠檢校太保,兼滄州刺史,充義昌軍節度、滄德觀察等使。河中王重榮累表論列,數令孜離間方鎮,
 令孜遣邠寧節度使硃玫會合鄜、延、靈、夏之師討河中。



 九月,硃玫屯沙苑。王重榮求援於太原。十月,李克用率太原軍南出陰地關。



 十一月,河中、太原之師與禁軍對壘於沙苑。十二月辛亥朔。癸酉,官軍合戰,為沙陀所敗,硃玫走還邠州。神策軍潰散,遂入京師肆掠。乙亥,沙陀逼京師,田令孜奉僖宗出幸鳳翔。初,黃巢據京師,九衢三內,宮室宛然。及諸道兵破賊,爭貨相攻,縱火焚剽,宮室居市閭里,十焚六七。賊平之後,令京兆尹王徽經年
 補葺,僅復安堵。至是,亂兵復焚,宮闕蕭條,鞠為茂草矣。



 二年春正月辛巳朔,車駕在鳳翔。李克用旋師河中,與硃玫、王重榮同上表,請駕駐蹕鳳翔,仍數田令孜之罪。乃以飛龍使楊復恭復知內樞密事。戊子,田令孜迫乘輿請幸興元。庚寅,車駕次寶雞。授刑部尚書孔緯兼御史大夫,令率從官赴行在。時車駕夜出,宰相蕭遘、裴徹、鄭昌圖及文武百僚不之知,扈從不及,故令孔緯促之。蕭遘惡令孜弄權,再亂京國,因邠州奏事判官李松年
 至鳳翔,乃令亟召硃玫迎奉。癸巳,硃玫引步騎五千至鳳翔。令孜聞邠州軍至,奉帝入散關,令禁軍守靈璧。玫至,禁軍潰散,遂長驅追駕至尊途驛。嗣襄王煴疾,為玫所得。時興元節度使石君涉聞車駕入關,乃毀徹棧道,柵絕險要,車駕由他道僅達,為邠州軍踵後,崎嶇危殆者數四。



 二月辛亥朔,以十軍觀軍容使、開府田令孜為劍南西川節度監軍,以內樞密使楊復恭為神策左軍中尉。



 三月庚辰朔。壬午,興元李度使石君涉棄城入硃
 玫軍內。丙申,車駕至興元。戊辰,以翰林學士承旨、兵部尚書、知制誥杜讓能為兵部侍郎;刑部尚書、御只大夫孔緯為兵部侍郎,充諸道鹽鐵轉運等使:並以本官同平章事。保鑾都將李鋋、楊守亮、楊守宗等敗邠州軍於鳳州。四月庚戌朔,是夜熒惑犯月角。壬子,硃玫、李昌符迫宰相蕭遘等於鳳翔驛舍,請嗣襄王煴權監軍國事。玫自為大丞相,兼左右神策十軍使。遂驅率文武百僚奉襄王還京師。



 五月己卯朔。庚辰,襄王僭即皇帝位,年
 號建貞。以蕭遘初沮襄王監國之命,罷知政事,為太子少師。以硃玫為侍中、諸道鹽鐵轉運使。以裴徹為門下侍郎、右僕射、同平章事、判度支。中書侍郎、刑部尚書、平章事鄭昌圖判戶部事。蕭遘移疾歸河中之永樂。偽制加諸侯官爵。以淮南節度使、檢校太尉、兼侍中高駢為太師、中書令、江淮鹽鐵轉運、諸道行營兵馬都統。又以淮南右都押衙、和州刺史呂用之檢校兵部尚書,兼廣州刺史、嶺南東道節度使。令戶部侍郎柳涉往江淮宣
 諭,戶部侍郎夏侯潭河北宣諭,諸籓節將多授其偽署,惟定州、太原、宣武、河中拒而不受。是月,星孛於箕尾,歷北斗攝提。荊南、襄陽仍歲蝗旱,米斗三十千,人多相食。楊復恭兄弟於河中、太原有破賊連衡之舊,乃奏遣諫議大夫劉崇望齎詔宣諭,達復恭之旨。王重榮、李克用欣然聽命,尋遣使貢奉,獻縑十萬匹,願殺硃玫自贖。崇望使還,君臣相賀。



 六月己酉朔,以扈蹕都將楊守亮為金州刺史、金商節度、京畿制置使。守亮率師二萬趨金
 州,與王重榮、李克用掎角進軍。時硃玫遣將王行瑜率邠寧、河西之師五萬屯鳳州,保鑾都將李鋋、李茂貞、陳珮等抗之於大唐峰。七月戊寅朔,蔡賊秦宗權陷許州,殺鹿晏弘。以金商節度使楊守亮檢校司徒,兼興元尹,充山南西道節度等使。王行瑜急攻興州,守亮出師擊敗之。



 八月,幽州節度使李全忠卒,三軍立其子匡威為留後。



 九月,楊守亮復敗邠州軍於鳳州,軍容楊復恭密遣人說王行瑜,令謀歸國。十月壬子朔,滑州軍亂,逐其
 帥安師儒,推衙將張驍主留後軍務。師儒奔汴州,硃全忠殺之,遂以兵攻滑,斬張驍以告行在,朝廷以汴帥全忠兼領義成軍節度使。壬辰夜,白虹見西方。



 十一月,蔡賊孫儒陷鄭州,刺史李璠遁免。儒引軍攻河陽。十二月乙巳朔。是月,硃玫愛將王行瑜受密詔,自鳳州率眾還長安。辛酉,行瑜斬硃玫及其黨與數百人,縱兵大掠。是冬苦寒,九衢積雪,兵入之夜,寒冽尤劇,民吏剽剝之後,殭凍而死蔽地。裴徹、鄭昌圖及百官奉襄王奔河中,王
 重榮紿稱迎奉,執李煴斬之,械裴徹、鄭昌圖於獄,文武官僚遭戮者殆半。重榮函襄王首赴行在。刑部奏請御興元城南門,閱俘馘受賀,下禮院定儀注。博士殷盈孫奏曰:



 伏以偽煴違背宗社,僭竊乘輿,欺天之禍既盈,盜國之罪斯重,果至覆敗,以就誅夷。九重之妖既除,萬國之生靈共慶,宜陳賀禮,以顯皇猷。然物議之間,有所未允。臣按禮經,公族有罪,獄既具,有司聞於公曰:「某之罪在大闢。」君曰:「赦之。」如是者三,有司走出致刑,君復使
 謂之曰:「雖然,固當赦之。」有司曰:「不及矣!」君為之素服不樂三月。《左傳》:衛君在晉,衛臣元咺立衛君之弟叔武,衛君入國,叔武為前驅所殺,衛君哭之,左氏書焉。今偽煴,皇族也,雖犯殊死之罪,宜就屠戮,其可以朝群臣而受賀乎?臣以為煴胤系金枝,名標玉牒,迫脅之際,不能守節效死,而乃甘心逆謀,罪實滔天,刑不可赦。已為軍前處置,宜即黜為庶人,絕其屬籍,其首級仍委所在以庶禮收葬。大捷之慶,當以硃玫首級到日稱賀,為得其
 宜。上不軫於宸衷,下無傷於物體,協禮經之旨,祛中外之疑。



 遂罷賀禮。及硃玫傳首至,乃御樓受俘馘。是月,蔡賊孫儒陷河陽,諸葛仲方奔歸汴州,別將李罕之出據澤州,張全義據懷州。



 三年春正月乙亥朔,車駕在興元府。制以邠州都將王行瑜檢校刑部尚書,兼邠州刺史、邠寧慶節度使。保鑾都將李鋋檢校司空、黔州刺史、黔中節度觀察使;扈蹕都頭李茂貞為檢校尚書左僕射、洋州刺史、武定軍節
 度使;扈蹕都頭楊守宗為金州刺史、金商節度等使;保鑾都將陳珮檢校尚書右僕射,為宣州刺史、宣歙觀察使。兵部侍郎、諸道租庸使張浚本官同平章事。



 二月乙巳朔,潤州牙將劉浩、度支使薛朗同謀逐其帥周寶,劉浩自稱留後。



 三月乙亥朔。甲申,車駕還京,次鳳翔。以宮室未完,節度使李昌符請駐蹕,以俟畢工。河中械送偽宰相裴徹、鄭昌圖,命斬之於岐山縣。太子少師致仕蕭遘賜死於永樂縣。以特進、監修國史、門下侍郎、吏部尚
 書、平章事孔緯領諸道鹽鐵轉運使。以集賢殿大學士、中書侍郎、兵部尚書、平章事杜讓能進封襄陽郡公,增食邑三千戶。四月甲辰朔,揚州牙將畢師鐸自高郵率戍兵攻揚州,下之,囚高駢於別室,自總軍政。蔡賊秦賢攻汴州,周列三十六砦。硃全忠乞師於兗鄆,硃瑾率師來赴,屯封禪寺,硃瑄屯靜戎鎮。



 五月甲戌朔。乙亥,秦宗權自率眾來應秦賢。壬午,鄆、兗、汴三鎮之師大破蔡賊於邊孝村,宗權退走。孫儒聞秦賢敗,盡驅河陽之人殺
 之,投尸於河,焚燒閭井而去。王師收孟、洛、許、汝、懷、鄭、陜、虢等州。詔以扈駕都頭楊守宗權知許州事,汴將孟從益權知鄭州事。諸葛爽舊將李罕之自澤州收河陽,懷州刺史張全義收洛陽。揚州牙將畢師鐸召宣州觀察使秦彥入揚州,推為節度使。



 六月癸卯朔。戊申,天威軍都頭楊守立與李昌符爭道,麾下相歐。上命中使諭之,不止,是夜嚴兵為備。己酉,守立以兵攻昌符,戰於通衢。昌符兵敗,出保隴州,命扈駕都將李茂貞攻之。甲寅,河
 中牙將常行儒殺其帥王重榮,推重榮兄重盈為兵馬留後。丙辰,太常禮院奏:「太廟十一室,並祧廟八室,孝明太后等別廟三室,自車駕再幸山南,並經焚毀,神主失墜。今大駕還京,宜先葺宗廟神主,然後還宮。」遂詔修奉太廟使宰相鄭延昌修奉。是時,宮室未完,國力方困,未暇舉行舊制,延昌請權以少府監大為太廟。太廟凡十一室,二十三間,間十一架,今監五間,請添造成十一間,以備十一室之數。敕曰:「敬依典禮。」七月壬申朔,隴州
 刺史薛知籌以城降李茂貞,遂拔隴州,斬李昌符、昌仁等,傳首獻於行在。丙子,制以武定軍節度使、檢校尚書左僕射,兼洋州刺史、御史大夫、上柱國、隴西郡公、食邑一千五百戶李茂貞檢校司空、同平章事,兼鳳翔尹、鳳翔隴右節度等使。



 九月辛未朔,淮南節度使高駢為其牙將畢師鐸所殺。楊行密急攻廣陵,蔡賊秦宗權遣其將孫儒將兵三萬渡淮,爭揚州,城中食盡。



 十一月,秦彥、畢師鐸潰圍奔於孫儒軍,行密入據揚州。秦彥引孫儒
 之兵攻廣陵,行密遣使求援於硃全忠。制授全忠檢校太尉、侍中,兼揚州大都督府長史,充淮南節度觀察等使、行營兵馬都統。汴將李璠率師至淮口以援之。十二月己巳朔,東川節度使顧彥朗、壁州刺史王建連兵五萬攻成都,陳敬瑄告難於朝,詔中使諭之。



 文德元年春正月己亥朔,車駕在鳳翔。制故鳳翔隴右節度觀察處置等使、檢校司徒、同平章事,兼鳳翔尹、上柱國、滎陽郡開國公、食邑三千戶鄭畋贈司徒,謚曰文
 昭。蔡賊孫儒斬秦彥、畢師鐸於高郵。



 二月己巳朔。壬午,車駕在鳳翔至京師。魏博軍亂。逐其帥樂彥禎。彥禎子相州刺史從訓率眾攻魏州,牙軍立其小校羅宗弁為留後,出兵拒之。從訓求援於汴,硃全忠遣將硃珍渡河赴之。戊子,上御承天門,大赦,改元文德。宰相韋昭度兼司空,孔緯、杜讓能加左右僕射,進階開府儀同三司,並賜號「持危啟運保乂功臣」。張浚兼兵部尚書,進階開府儀同三司。左右神策十軍觀軍容使、左金吾衛上將軍、
 左右街功德使、上柱國、弘農郡開國公楊復恭進封魏國公,加食邑七千戶,賜號「忠貞啟聖定國功臣。」以保鑾都將、黔中節度使李鋋檢校司徒、平章事,保鑾都將陳珮檢校司空、廣州刺史、嶺南東道節度使。籓鎮諸侯,進秩有差。宰臣韋昭度率文武百僚上徽號曰聖文睿德光武弘孝皇帝。



 三月戊戌朔,正殿受冊。庚子,上暴疾。壬寅,大漸。癸卯,宣制立弟壽王傑為皇太弟,勾當軍國事。是夕,崩於武德殿,聖壽二十七,群臣上謚曰惠聖恭定
 孝皇帝,廟號僖宗。其年十二月,葬於靖陵。



 史臣曰:恭帝沖年纘歷,政在宦臣,惕勵虔恭,殷憂重慎。屬世道交喪,海縣橫流,赤眉搖蕩於中原,黃屋流離於遐徼,黔黎塗炭,宗社丘墟。而猶籓垣多仗義之臣,心腹有盡忠之輔,驅駕豪傑,號令軍戎,終誅伏莽之徒,大雪失邦之恥。而令孜一為謬計,幾喪丕圖,雖如線之僅存,固棼絲之莫救。茫茫禹跡,空悲文命之艱難;赫赫宗周,竟墜文王之基業。非僖皇失道之過,其土運之窮歟?悲
 夫!



 贊曰:運歷將窮,人君幼沖。塵飛巨盜,波駭群雄。天既降喪,人罕輸忠。回鑾返正,禁旅之功。



\end{pinyinscope}