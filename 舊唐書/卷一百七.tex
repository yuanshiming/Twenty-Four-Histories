\article{卷一百七}

\begin{pinyinscope}

 ○郭虔瓘張嵩郭知運
 子英傑
 王君賈師順附張守珪牛仙客王忠嗣



 郭虔瓘,齊州歷城人也。開元初,累遷右驍衛將軍,兼北庭都護。二年春,突厥默啜遣其子移江可汗及同俄特
 勒率精騎圍逼北庭,虔瓘率眾固守。同俄特勒單騎親逼城下,虔瓘使勇士伏於路左,突起斬之。賊眾既至,失同俄,相率於城下乞降,請盡軍中衣資器杖以贖同俄。及聞其死,三軍慟哭,便引退。默啜女婿火拔頡利發石阿失畢時與同俄特勒同領兵,以同俄之死,懼不敢歸,遂將其妻歸降。虔瓘以破賊之功,拜冠軍大將軍,行右驍衛大將軍。又下制曰:



 朕聞賞有功、報有德者,政之急也。若功不賞,德不報,則人何謂哉。雲麾將軍、檢校右驍衛將軍,兼北庭都
 護、翰海軍經略使、金山道副大總管、招慰營田等使、上柱國、太原縣開國子郭虔瓘,宣威將軍、守右驍衛翊府中郎將、檢校伊州刺史兼伊吾軍使、借紫金魚袋、上柱國郭知運等,早負名節,見稱義勇。頃者柳中、金滿,偏師禦敵,蕭條窮漠之外,奔迫孤城之下。強寇益侵,援兵不至,既守而戰,自秋涉冬,櫪馬長嘶,戍人遠望。謀以十勝,成其九拒。遂能摧日逐之遺種,斬天驕之愛息。豈耿恭、班超,獨高前史;將廉頗、李牧,與朕同時。眷言茂勛,是所
 嘉嘆。信可以疇其井邑,昭示遐邇,俾勞臣觀而懦夫立焉。虔瓘可進封太原郡開國公,知運可封介休縣開國公。



 虔瓘俄轉安西副大都護、攝御史大夫、四鎮經略安撫使,進封潞國公,賜實封一百戶虔瓘及奏請募關中兵一萬人往安西討擊,皆給公乘,兼供熟食,敕許之。將作大匠韋湊上疏曰:



 臣聞兵者兇器,不護己而用之。今西域諸蕃,莫不順軌。縱鼠竊狗盜,有戍卒鎮兵,足宣式遏之威,非降赫斯之怒。此師之出,未見其名。臣又聞安
 不忘危,理必資備。自近及遠,強幹弱枝,是以漢實關中,徙諸豪族。今關輔戶口,積久逋逃,承前先虛,見猶未實。屬北虜犯塞,西戎駭邊,凡在丁壯,征行略盡。豈宜更募驍勇,遠資荒服。又一萬行人,詣六千餘里,咸給遞馱,並供熟食,道次州縣,將何以供?秦、隴之西,人戶漸少,涼州已去,沙磧悠然。遣彼居人,如何得濟?又萬人賞賜,費用極多;萬里資糧,破損尤廣。縱令必克,其獲幾何?儻稽天誅,無乃甚損!請令計議所用所得,校其多少,即知利害。
 況用者必賞,獲者未量,何要此行,頓空畿甸。且上古之時,大同之化,不獨子子,不獨親親,何隔華戎,務均安靖。洎皇道謝古,帝德慚皇,猶尚綏懷,不從征伐,有占風覘雨之客,無越海逾山之師。其後漢武膺圖,志恢土宇,西通絕域,北擊匈奴。雖廣獲珍奇,多斬首級,而中國疲耗,殆至危亡。是以俗號昇平君稱盛德者,咸指唐堯之代,不歸漢武之年。其要功不成者,復焉足比議?惟陛下圖之。



 虔瓘竟無克獲之功。尋遷右威衛大將軍,以疾卒。



 其
 後,又以張嵩為安西都護以代虔瓘。嵩身長七尺,偉姿儀。初進士舉,常以邊任自許。及在安西,務農重戰,安西府庫,遂為充實。十年,轉太原尹,卒官。俄又以黃門侍郎杜暹代嵩為安西都護。



 郭知運字逢時,瓜州常樂人。壯勇善射,頗有膽略。初為秦州三度府果毅,以戰功累除左驍衛中郎將、瀚海軍經略使,又轉檢校伊州刺史,兼伊吾軍使。開元二年春,副郭虔瓘破突厥於北庭,以功封介休縣公,加雲麾將
 軍,擢拜右武衛將軍。其秋,吐蕃入寇隴右,掠監牧馬而去,詔知運率眾擊之。知運與薛訥、王皎等掎角擊敗之,拜知運鄯州都督、隴右諸軍節度大使。四年冬,突厥降戶阿悉爛、𧾷夾跌思太等率眾反叛,單于副都護張知運為賊所執,詔薛訥領兵討之。叛賊至綏州界,詔知運領朔方兵募橫擊之,大破賊眾於黑山呼延谷,賊舍甲仗並棄張知運走。六年,知運又率兵入討吐蕃,賊徒無備,遂掩至九曲,獲鎖及甲馬耗牛等數萬計。知運獻捷,遂分賜京文武
 五品已上清官及朝集使,拜知運為兼鴻臚卿、攝御史中丞,加封太原郡公。八年,六州胡康待賓等反,詔知運與王皎討平之,拜左武衛大將軍,授一子官,賜金銀器百事、雜彩千段。九年,卒於軍,贈涼州都督,錫米粟五百斛、絹帛五百段,仍令中書令張說為其碑文。知運自居西陲,甚為蕃夷所憚,其後王君亦號勇將,時人稱王、郭焉。子英傑、英乂。



 英傑官至左衛將軍。開元二十一年,幽州長史薛楚玉遣英傑及裨將吳克勤、烏知義、羅守
 忠等率精騎萬人及降奚之眾以討契丹,屯兵於榆關之外;契丹首領可突幹引突厥之眾拒戰於都山之下。官軍不利,知義、守忠率麾下便道遁歸。英傑與克勤逢賊力戰,皆沒於陣。其下精銳六千餘人仍與賊苦戰,賊以英傑之首示之,竟不降,盡為賊所殺。英乂,劍南西川節度使,自有傳。



 王君,瓜州常樂人也。初,為郭知運別奏,驍勇善騎射,以戰功累除右衛副率。及知運卒,遂代知運為河西、隴右節度使,遷
 右羽林軍將軍,判涼州都督事。開元十六年冬,吐蕃大將悉諾邏率眾入寇大斗谷,又移攻甘州,焚燒市里而去。君以其兵疲,整士馬以掩其後。會大雪,賊徒凍死者甚眾,賊遂取積石軍西路而還。君令副使馬元慶、裨將車蒙追之,不及。君先令人潛入賊境,於歸路燒草。番諾邏還至大非川,將息甲牧馬,而野草皆盡,馬死過半。君襲其後,入至青海之西,時海水冰合,君與秦州都督張景順等率將士並乘冰而渡。會悉諾邏
 已度大非山,輜重及疲兵尚在青海之側,君縱兵盡俘獲之,及羊馬萬數。君以功遷右羽林軍大將軍,攝御史中丞,依舊判涼州都督,封晉昌伯。拜其父壽為少府監,仍聽致仕。上又嘗於廣達樓引君及妻夏氏設宴,賜以金帛。夏氏亦有戰功,故特賞之,封為武威郡夫人。其冬,吐蕃寇陷瓜州,執刺史田仁獻及君父壽,殺掠人戶,並取軍資及倉糧。又進攻玉門軍及常樂縣。仍縱僧徒使歸涼州,謂君曰:「將軍常欲以忠勇報國,今日
 何不一戰?」君聞父被執,登陴西向而哭,竟不敢出兵。



 初,涼州界有回紇、契苾、思結、渾四部落,代為酋長,君微時往來涼府,為回紇等所輕。及君為河西節度使,回紇等怏怏,恥在其麾下。君以法繩之,回紇等積怨,密使人詣東都自陳枉狀。君遽發驛奏「回紇部落難制,潛有叛謀。」上使中使往按問之,回紇等竟不得理。由是瀚海大都督回紇承宗長流瀼州,渾大德長流吉州,賀蘭都督契苾承明長流藤州,盧山都督思結歸國長流
 瓊州。右散騎常侍李令問、特進契苾嵩以與回紇等結婚,貶令問為撫州別駕,嵩為連州別駕。於是承宗之黨瀚海州司馬護輸糾合黨與,謀殺君,以復其怨。會吐蕃使間道往突厥,君率精騎往肅州掩之,還至甘州南鞏幰驛,護輸伏兵突起,奪君旌節,先殺其左右宗貞,剖其心,云是其始謀也。君從數十人與賊力戰,自朝至晡,左右盡死。遂殺君,馱其尸以奔吐蕃。追及之,護輸遂棄君尸而走。上甚痛惜之,制贈特進、荊州大都
 督,給靈輿遞歸京師,葬於京城之東,官供喪事。仍令張說為其碑文,上自書石以寵異之。



 吐蕃之寇瓜州也,分遣副將莽布支攻常樂縣,縣令賈師順嬰城固守。及瓜州城陷,大將悉諾邏又盡引其眾乘勢以攻之,數日不陷。賊中有分得漢口為妻者,其妻弟在常樂城中,悉諾邏使夜就城下詐為私見,謂師順曰:「瓜州已破,吐蕃盡眾來此,豈有拒守之理?小人妻弟在城,情有所念,明府何不早降,以全城中之眾。」師順答曰:「漢法,降賊者九族為
 戮,吾受國官爵,祗可以死拒寇,豈得背恩降賊!」悉諾邏知師順不降,又攻城八日,復令前使謂師順曰:「明府既不肯降,吾眾欲還,城中豈無財物以相贈耶?」師順請脫士卒衣裳以為賂。悉諾邏知城中無財帛,夜燒死人,收營而去,引眾毀瓜州城。師順遽開門收器械,更修守備。吐蕃果使精騎回襲,而巡城知有備,始去。



 賈師順者,岐州人也。以守城之功,累遷鄯州都督、隴右節度使。入為左領軍將軍,病卒。



 張守珪,陜州河北人也。初以戰功授平樂府別駕,從郭虔瓘於北庭鎮,遣守珪率眾救援,在路逢賊甚眾,守珪身先士卒,與之苦戰,斬首千餘級,生擒賊率頡斤一人。開元初,突厥又寇北庭,虔瓘令守珪間道入京奏事,守珪因上書陳利害,請引兵自蒲昌、輪臺翼而擊之。及賊敗,守珪以功特加游擊將軍,再轉幽州良社府果毅。守珪儀形瑰壯,善騎射,性慷慨,有節義。時盧齊卿為幽州刺史,深禮遇之,常共榻而坐,謂曰:「足下數年外必節度
 幽、涼,為國之良將,方以子孫相托,豈得以僚屬常禮相期耶!」守珪後累轉左金吾員外將軍,為建康軍使。



 十五年,吐蕃寇陷瓜州,王君死,河西恟懼。以守珪為瓜州刺史、墨離軍使,領餘眾修築州城。板堞才立,賊又暴至城下,城中人相顧失色,雖相率登陴,略無守禦之意。守珪曰:「彼眾我寡,又創痍之後,不可以矢石相持,須以權道制之也。」乃於城上置酒作樂,以會將士。賊疑城中有備,竟不敢攻城而退。守珪縱兵擊敗之。於是修復廨宇,
 收合流亡,皆復舊業。守珪以戰功加銀青光祿大夫,仍以瓜州為都督府,以守珪為都督。瓜州地多沙磧,不宜稼穡,每年少雨,以雪水溉田。至是渠堰盡為賊所毀,既地少林木,難為修葺。守珪設祭祈禱,經宿而山水暴至,大漂材木,塞澗而流,直至城下。守珪使取充堰,於是水道復舊,州人刻石以紀其事。明年,遷鄯州都督,仍充隴右節度。



 二十一年,轉幽州長史、兼御史中丞、營州都督、河北節度副大使,俄又加河北採訪處置使。先是,契丹
 及奚連年為邊患,契丹衙官可突幹驍勇有謀略,頗為夷人所伏。趙含章、薛楚玉等前後為幽州長史,竟不能拒。及守珪到官,頻出擊之,每戰皆捷。契丹首領屈剌與可突干恐懼,遣使詐降。守珪察知其偽,遣管記右衛騎曹王悔詣其部落就謀之。悔至屈剌帳,賊徒初無降意,乃移其營帳漸向西北,密遣使引突厥,將殺悔以叛。會契丹別帥李過折與可突幹爭權不葉,悔潛誘之,斬屈剌可突幹,盡誅其黨,率餘眾以降。守珪因出師次於紫
 蒙川,大閱軍實,宴賞將士,傳屈剌、可突幹等首於東都,梟於天津橋之南。詔封李過折為北平王,使統其眾,尋為可突乾餘黨所殺。二十三年春,守珪詣東都獻捷,會籍田禮畢酺宴,便為守珪飲至之禮,上賦詩以褒美之。遂拜守珪為輔國大將軍、右羽林大將軍、兼御史大夫,餘官並如故。仍賜雜彩一千匹及金銀器物等,與二子官,仍詔於幽州立碑以紀功賞。



 二十六年,守珪裨將趙堪、白真陁羅等假以守珪之命,逼平盧軍使烏知義
 令率騎邀叛奚餘眾於湟水之北,將踐其禾稼。知義初猶固辭,真陁羅又詐稱詔命以迫之,知義不得已而行。及逢賊,初勝後敗,守珪隱其敗狀而妄奏克獲之功。事頗洩,上令謁者牛仙童往按之。守珪厚賂仙童,遂附會其事,但歸罪於白真陁羅,逼令自縊而死。二十七年,仙童事露伏法,守珪以舊功減罪,左遷括州刺史,到官無幾,疽發背而卒。



 弟守琦,左驍衛將軍;守瑜,金吾將軍。守珪子獻城、守瑜子獻恭、守琦子獻甫,三人皆為興元節度
 使,各自有傳。



 牛仙客,涇州鶉觚人也。初為縣小吏,縣令傅文靜甚重之。文靜後為隴右營田使,引仙客參預其事,遂以軍功累轉洮州司馬。開元初,王君為河西節度使,以仙客為判官,甚委信之。時又有判官宋貞,與仙客俱為腹心之任。及君死,宋貞亦為回紇所殺,仙客以不從獲免。俄而蕭嵩代君為河西節度,又以軍政委於仙客。仙客清勤不倦,接待上下,必以誠信。及嵩入知政事,數稱
 薦之。稍遷太僕少卿,判涼州別駕事,仍知節度留後事。竟代嵩為河西節度使,判涼州事。歷太僕卿、殿中監,軍使如故。



 開元二十四年秋,代信安王禕為朔方行軍大總管,右散騎常侍崔希逸代仙客知河西節度事。初,仙客在河西節度時,省用所積鉅萬,希逸以其事奏聞,上令刑部員外郎張利貞馳傳往覆視之。仙客所積倉庫盈滿,器械精勁,皆如希逸之狀。上大悅,以仙客為尚書。中書令張九齡執奏以為不可,乃加實封二百戶。其年
 十一月,九齡等罷知政事,遂以仙客為工部尚書、同中書門下三品,仍知門下事。時有監察御史周子諒竊言於御史大夫李適之曰:「牛仙客不才,濫登相位,大夫國之懿親,豈得坐觀其事?」適之遽奏子諒之言,上大怒,廷詰之,子諒辭窮,於朝堂決配流瀼州,行至藍田而死。



 仙客既居相位,獨善其身,唯諾而已。所有錫齎,皆緘封不啟。百司有所諮決,仙客曰:「但依令式可也』,不敢措手裁決。明年,特封豳國公,贈其父意為禮部尚書,祖會為涇
 州刺史。俄又進拜侍中,兼兵部尚書。天寶年,改易官名,拜左相,尚書如故。其年七月卒,年六十八。內出絹一千匹、布五百端,遣中使送至宅以賻之,贈尚書左丞,謚曰貞簡。



 初,仙客為朔方軍使,以姚崇孫閎為判官。及知政事,閎累遷侍御史,自云能通鬼道,預知休咎。仙客頗信惑之。及疾甚,閎請為仙客祈禱,在其門下,遂逼仙客令作遺表薦閎叔尚書右丞弈及兵部侍郎盧奐堪代己,閎為起草。仙客時既危殆,署字不成,其妻因中使來吊,
 以其表上。玄宗覽而怒之,左遷弈為永陽太守,盧奐為臨淄太守,賜閎死。



 王忠嗣,太原祁人也,家於華州之鄭縣。父海賓,太子右衛率、豐安軍使、太谷男,以驍勇聞隴上。開元二年七月,吐蕃入寇,朝廷起薛訥攝左羽林將軍,為隴右防禦使,率杜賓客、郭知運、王晙、安思順以御之,以海賓為先鋒。及賊於渭州西界武階驛,苦戰勝之,殺獲甚眾。諸將嫉其功,按兵不救,海賓以眾寡不敵,歿於陣。大軍乘其勢
 擊之,斬首一萬七千級,獲馬七萬五千匹,羊牛十四萬頭。玄宗聞而憐之,詔贈左金吾大將軍。



 忠嗣初名訓,年九歲,以父死王事,起復拜朝散大夫、尚輦奉御,賜名忠嗣,養於禁中累年。肅宗在忠邸,與之游處。及長,雄毅寡言,嚴重有武略。玄宗以其兵家子,與之論兵,應對縱橫,皆出意表。玄宗謂之曰:「爾後必為良將。」十八年,又贈其父安西大都護。



 其後,遂從河西節度、兵部尚書蕭嵩,河東副元帥、信安王禕,並引為兵馬使。二十一年再轉左
 領軍衛郎將、河西討擊副使、左威衛將軍、賜紫金魚袋、清源男,兼檢校代州都督。嘗短皇甫惟明義弟王昱,憾焉,遂為所陷,貶東陽府左果毅。屬河西節度使杜希望謀拔新城,或言忠嗣之材足以輯事,必欲取勝,非其人不可。希望即奏聞,詔追忠嗣赴河西。既下新城,忠嗣之功居多,因授左威衛郎將,專知行軍兵馬。是秋,吐蕃大下,報新城之役,晨壓官軍,眾寡不敵。,師人皆懼焉。忠嗣乃以所部策馬而前,左右馳突,當者無不闢易,出而復
 合,殺數百人,賊眾遂亂。三軍翼而擊之,吐蕃大敗。以功最,詔拜左金吾衛將軍同正員,尋又兼左羽林軍上將軍、河東節度副使,兼大同軍使。二十八年,以本官兼代州都督,攝御史大夫,兼充河東節度,又加雲麾將軍。二十九年,代韋光乘為朔方節度使,仍加權知河東節度事。其月,以田仁琬充河東節度使,忠嗣依舊朔方節度。



 天寶元年,兼靈州都督。是歲北伐,與奚怒皆戰於桑乾河,三敗之,大虜其眾,耀武漠北,高會而旋。時突厥葉護
 新有內難,忠嗣盛兵磧口以威振之。烏蘇米施可汗懼而請降,竟遷延不至。忠嗣乃縱反間於拔悉密與葛邏祿、回紇三部落,攻米施可汗走之。忠嗣因出兵伐之,取其右廂而歸,其西葉護及毗伽可敦、男殺葛臘哆率其部落千餘帳入朝,因加左武衛大將軍。明年,又再破怒皆及突厥之眾。自是塞外晏然,虜不敢入。天寶三載,突厥九姓拔悉密葉等竟攻殺烏蘇米施可汗,傳首京師。四載,加攝御史大夫,充河東節度採訪使。五月,進封清
 源縣公。



 忠嗣少以勇敢自負,及居節將,以持重安邊為務。嘗謂人云:「國家昇平之時,為將者在撫其眾而已。吾不欲疲中國之力,以徼功名耳。」但訓練士馬,缺則補之。有漆弓百五十斤,嘗貯之袋中,示無所用。軍中皆日夜思戰,因多縱間諜以伺虜之隙,時以奇兵襲之,故士樂為用,師出必勝。每軍出,即各召本將付其兵器,令給士卒,雖一弓一箭,必書其名姓於上以記之,軍罷卻納。若遺失,即驗其名罪之。故人人自勸,甲仗充牣矣。



 四載,又
 兼河東節度採訪使。自朔方至雲中,緣邊數千里,當要害地開拓舊城,或自創制,斥地各數百里。自張仁亶之後四十餘年,忠嗣繼之,北塞之人,復罷戰矣。五年正月,河隴以皇甫惟明敗衄之後,因忠嗣以持節充西平郡太守,判武威郡事,充河西、隴右節度使。其月,又權知朔方、河東節度使事。忠嗣佩四將印,控制萬里,勁兵重鎮,皆歸掌握,自國初已來,未之有也。尋遷鴻臚卿,餘如故,又加金紫光祿大夫,仍授一子五品官。後頻戰青海、積
 石,皆大克捷。尋又伐吐谷渾於墨離,虜其全國而歸。初,忠嗣在河東、朔方日久,備諳邊事,得士卒心。及至河、隴,頗不習其物情,又以功名富貴自處,望減於往日矣。其載四月,固讓朔方、河東節度,許之。



 玄宗方事石堡城,詔問以攻取之略,忠嗣奏云:「石堡險固,吐蕃舉國而守之。若頓兵堅城之下,必死者數萬,然後事可圖也。臣恐所得不如所失,請休兵秣馬,觀釁而取之,計之上者。」玄宗因不快。李林甫尤忌忠嗣,日求其過。六載,會董延光獻
 策請下石堡城,詔忠嗣分兵應接之。忠嗣僶俯而從,延光不悅。河西兵馬使李光弼危之,遽而入告。將及於庭,忠嗣曰:「李將軍有何事乎?」光弼進而言曰:「請議軍。」忠嗣曰:「何也?」對曰:「向者大夫以士卒為心,有拒董延光之色,雖曰受詔,實奪其謀。何者?大夫以數萬眾付之,而不懸重賞,則何以賈三軍之勇乎?大夫財帛盈庫,何惜數萬段之賞以杜其讒口乎!彼如不捷,歸罪於大夫矣。」忠嗣曰:「李將軍,忠嗣計已決矣。平生始望,豈及貴乎?今爭一
 城,得之未制於敵,不得之未害於國,忠嗣豈以數萬人之命易一官哉?假如明主見責,豈失一金吾羽林將軍,歸朝宿衛乎!其次,豈失一黔中上佐乎?此所甘心也。雖然,公實愛我。」光弼謝曰:「向者恐累大夫,敢以衷告。大夫能行古人之事,非光弼所及也。」遂趨而出。及延光過期不克,訴忠嗣緩師,故師出無功。李林甫又令濟陽別駕魏林告忠嗣,稱往任朔州刺史,忠嗣為河東節度,云「早與忠王同養宮中,我欲尊奉太子。」玄宗大怒,因
 徵入朝,令三司推訊之,幾陷極刑。會哥舒翰代忠嗣為隴右節度,特承恩顧,因奏忠嗣之枉,詞甚懇切,請以己官爵贖罪。玄宗怒稍解。十一月,貶漢陽太守。七載,量移漢東郡太守。明年,暴卒,年四十五。子震,天寶中秘書丞。



 其後哥舒翰大舉兵伐石堡城,拔之,死者大半,竟如忠嗣之言,當代稱為名將。先是,忠嗣之在朔方也,每至互市時,即高估馬價以誘之,諸蕃聞之,競來求市,來輒買之。故蕃馬益少,而漢軍益壯。及至河、隴,又奏請徙朔方、河東戎
 馬九千匹以實之,其軍又壯。迄於天寶末,戰馬蕃息。寶應元年,追贈兵部尚書。



 史臣曰:郭虔瓘、郭知運、王君、張守珪、牛仙客、王忠嗣,立功邊域,為世虎臣,班超、傅介子之流也。然虔瓘以萬人征西,請給公乘、熟食,可謂謀之不臧矣。君以父執登陴,兵竟不出,此則不知門外之事,義斷恩也。守珪以至誠感神,取材成堰,與夫耿恭拜井,有何異焉?仙客爰自方隅,驟登廊廟,顯招物議,獨善其身,蓋才有不周,昧
 於陳力就列。忠嗣因青蠅之點,幾危其身,讒人之言,誠可畏也!



 贊曰:隴山之西,幽陵之北,爰有戎夷,世為殘賊。二郭、二王,守珪、仙客,御寇之功,存乎方策。



\end{pinyinscope}