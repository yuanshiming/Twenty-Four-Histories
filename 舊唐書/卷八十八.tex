\article{卷八十八}

\begin{pinyinscope}

 ○劉仁軌郝處俊裴行儉子光庭



 劉仁軌,汴州尉氏人也。少恭謹好學,遇隋末喪亂,不遑專習,每行坐所在,輒書空地,由是博涉文史。武德初,河南道大使、管國公任瑰將上表論事,仁軌見其起草,因
 為改定數字。瑰甚異之,遂赤牒補息州參軍。稍除陳倉尉。部人有折沖都尉魯寧者,恃其高班,豪縱無禮,歷政莫能禁止。仁軌特加誡喻,期不可再犯,寧又暴橫尤甚,竟杖殺之。州司以聞,太宗怒曰:「是何縣尉,輒殺吾折沖!」遽追入,與語,奇其剛正,擢授櫟陽丞。貞觀十四年,太宗將幸同州校獵,屬收獲未畢,仁軌上表諫曰:「臣聞屋漏在上,知之者在下;愚夫之計,擇之者聖人。是以周王詢於芻蕘,殷後謀於板築,故得享國彌久,傳祚無疆,功宣清
 廟,慶流後葉。伏惟陛下天性仁愛,躬親節儉,朝夕克念,百姓為心,一物失所,納隍軫慮。臣伏聞大駕欲幸同州教習,臣伏知四時搜狩,前王恆典,事有沿革,未必因循。今年甘雨應時,秋稼極盛,玄黃亙野,十分才收一二;盡力刈獲,月半猶未訖功;貧家無力,禾下始擬種麥。直據尋常科喚,田家已有所妨。今既供承獵事,兼之修理橋道,縱大簡略,動費一二萬工,百姓收斂,實為狼狽。臣願陛下少留萬乘之恩,垂聽一介之言,退近旬日,收刈
 總了,則人盡暇豫,家得康寧。輿輪徐動,公私交泰。」太宗特降璽書勞曰:「卿職任雖卑,竭誠奉國,所陳之事,朕甚嘉之。」尋拜新安令,累遷給事中。



 顯慶四年,出為青州刺史。五年,高宗征遼,令仁軌監統水軍,以後期坐免,特令以白衣隨軍自效。時蘇定方既平百濟,留郎將劉仁願於百濟府城鎮守,又以左衛中郎將王文度為熊津都督,安撫其餘眾。文度濟海病卒。百濟為僧道琛、舊將福信率眾復叛,立故王子扶餘豐為王,引兵圍仁願於府
 城。詔仁軌檢校帶方州刺史,代文度統眾,便道發新羅兵合勢以救仁願。轉鬥而前,仁軌軍容整肅,所向皆下。道琛等乃釋仁願之圍,退保任存城。尋而福信殺道琛,並其兵馬,招誘亡叛,其勢益張。仁軌乃與仁願合軍休息。時蘇定方奉詔伐高麗,進圍平壤,不克而還。高宗敕書與仁軌曰:「平壤軍回,一城不可獨固,宜拔就新羅,共其屯守。若金法敏藉卿等留鎮,宜且停彼;若其不須,即宜泛海還也。」將士咸欲西歸,仁軌曰:「《春秋》之義,大夫出
 疆,有可以安社稷、便國家、專之可也。況在滄海之外,密邇豺狼者哉!且人臣進思盡忠,有死無貳,公家之利,知無不為。主上欲吞滅高麗,先誅百濟,留兵鎮守,制其心腹。雖妖孽充斥,而備預甚嚴,宜礪戈秣馬,擊其不意。彼既無備,何攻不克?戰而有勝,士卒自安。然後分兵據險,開張形勢,飛表聞上,更請兵船。朝廷知其有成,必當出師命將,聲援才接,兇逆自殲。非直不棄成功,實亦永清海外。今平壤之軍既回,熊津又拔,則百濟餘燼,不日更
 興,高麗逋藪,何時可滅?且今以一城之地,居賊中心,如其失腳,即為亡虜。拔入新羅,又是坐客,脫不如意,悔不可追。況福信兇暴,殘虐過甚,餘豐猜惑,外合內離,鴟張共處,勢必相害。唯宜堅守觀變,乘便取之,不可動也。」眾從之。時扶餘豐及福信等以真峴城臨江高險,又當沖要,加兵守之。仁軌引新羅之兵,乘夜薄城。四面攀草而上,比明而入據其城,遂通新羅運糧之路。俄而餘豐襲殺福信,又遣使往高麗及倭國請兵,以拒官軍。詔右威
 衛將軍孫仁師率兵浮海以為之援。仁師既與仁軌等相合,兵士大振。於是諸將會議,或曰:「加林城水陸之沖,請先擊之。」仁軌曰:「加林險固,急攻則傷損戰士,固守則用日持久,不如先攻周留城。周留,賊之巢穴,群兇所聚,除惡務本,須拔其源。若克周留,則諸城自下。」於是仁師、仁願及新羅王金法敏帥陸軍以進。仁軌乃別率杜爽、扶餘隆率水軍及糧船,自熊津江往白江,會陸軍同趣周留城。仁軌遇倭兵於白江之口,四戰捷,焚其舟四百
 艘,煙焰漲天,海水皆赤,賊眾大潰。餘豐脫身而走,獲其寶劍。偽王子扶餘忠勝、忠志等,率士女及倭眾並耽羅國使,一時並降。百濟諸城,皆復歸順。賊帥遲受信據任存城不降。



 先是,百濟首領沙吒相如、黑齒常之自蘇定方軍回後,鳩集亡散,各據險以應福信,至是率其眾降。仁軌諭以恩信,令自領子弟以取任存城,又欲分兵助之。孫仁師曰:「相如等獸心難信,若授以甲仗,是資寇兵也。」仁軌曰:「吾觀相如、常之皆忠勇有謀,感恩之士,從我
 則成,背我必滅,因機立效,在於茲日,不須疑也。」於是給其糧仗,分兵隨之,遂拔任存城。遲受信棄其妻子走投高麗,於是百濟之餘燼悉平。孫仁師與劉仁願振旅而還,詔留仁軌勒兵鎮守。初,百濟經福信之亂,合境凋殘,殭尸相屬。仁軌始令收斂骸骨,瘞埋吊祭之。修錄戶口,署置官長,開通途路,整理村落,建立橋梁,補葺堤堰,修復陂塘,勸課耕種,賑貸貧乏,存問孤老。頒宗廟忌諱,立皇家社稷。百濟餘眾,各安其業。於是漸營屯田,積糧撫
 士,以經略高麗。仁願既至京師,上謂曰:「卿在海東,前後奏請,皆合事宜,而雅有文理。卿本武將,何得然也?」對曰:「劉仁軌之詞,非臣所及也。」上深嘆賞之,因超加仁軌六階,正授帶方州刺史,並賜京城宅一區,厚賚其妻子,遣使降璽書勞勉之。仁軌又上表曰:



 臣蒙陛下曲垂天獎,棄瑕錄用,授之刺舉,又加連率。材輕職重,憂責更深,常思報效,冀酬萬一,智力淺短,淹滯無成。久在海外,每從征役,軍旅之事,實有所聞。具狀封奏,伏願詳察。臣看見
 在兵募,手腳沉重者多,勇健奮發者少,兼有老弱,衣服單寒,唯望西歸,無心展效。臣問:「往在海西,見百姓人人投募,爭欲征行,乃有不用官物,請自辦衣糧,投名義征。何因今日募兵,如此佇弱?」皆報臣云:「今日官府,與往日不同,人心又別。貞觀、永徽年中,東西征役,身死王事者,並蒙敕使吊祭,追贈官職,亦有回亡者官爵與其子弟。從顯慶五年以後,征役身死,更不借問。往前渡遼海者,即得一轉勛官;從顯慶五年以後,頻經渡海,不被記錄。
 州縣發遣兵募,人身少壯、家有錢財、參逐官府者,東西藏避,並即得脫;無錢參逐者,雖是老弱,推背即來。顯慶五年,破百濟勛,及向平壤苦戰勛,當時軍將號令,並言與高官重賞,百方購募,無種不道。洎到西岸,唯聞枷鎖推禁,奪賜破勛,州縣追呼,求住不得,公私困弊,不可言盡。發海西之日,已有自害逃走,非獨海外始逃。又為征役,蒙授勛級,將為榮寵,頻年征役,唯取勛官,牽挽辛苦,與白丁無別。百姓不願征行,特由於此。」陛下再興兵馬,
 平定百濟,留兵鎮守,經略高麗。百姓有如此議論,若為成就功業?臣聞琴瑟不調,改而更張,布政施化,隨時取適。自非重賞明罰,何以成功?臣又問:「見在兵募,舊留鎮五年,尚得支濟;爾等始經一年,何因如此單露?」並報臣道:「發家來日,唯遣作一年裝束,自從離家,已經二年。在朝陽甕津,又遣來去運糧,涉海遭風,多有漂失。」臣勘責見在兵募,衣裳單露,不堪度冬者,給大軍還日所留衣裳,且得一冬充事。來年秋後,更無準擬。陛下若欲殄滅
 高麗,不可棄百濟土地。餘豐在北,餘勇在南,百濟、高麗,舊相黨援,倭人雖遠,亦相影響,若無兵馬,還成一國。既須鎮壓,又置屯田,事藉兵士,同心同德。兵士既有此議,不可膠柱因循,須還其渡海官勛及平百濟向平壤功效。除此之外,更相褒賞,明敕慰勞,以起兵募之心。若依今日以前布置,臣恐師老且疲,無所成就。臣又見晉代平吳,史籍具載。內有武帝、張華,外有羊祜、杜預,籌謀策畫,經緯諮詢。王浚之徒,折沖萬里,樓船戰艦,已到石頭。
 賈充、王渾之輩,猶欲斬張華以謝天下。武帝報云:「平吳之計,出自朕意,張華同朕見耳,非其本心。」是非不同,乖亂如此。平吳之後,猶欲苦繩王浚,賴武帝擁護,始得保全。不逢武帝聖明,王浚不存首領。臣每讀其書,未嘗不撫心長嘆。伏惟陛下既得百濟,欲取高麗,須外內同心,上下齊奮,舉無遺策,始可成功。百姓既有此議,更宜改調。臣恐是逆耳之事,無人為陛下盡言。自顧老病日侵,殘生詎幾?奄忽長逝,銜恨九泉,所以披露肝膽,昧死聞
 奏。



 上深納其言。又遣劉仁願率兵渡海,與舊鎮兵交代,仍授扶餘隆熊津都督,遣以招輯其餘眾。扶餘勇者,扶餘隆之弟也,是時走在倭國,以為扶餘豐之應,故仁軌表言之。於是仁軌浮海西還。初,仁軌將發帶方州,謂人曰:「天將富貴此翁耳!」於州司請歷日一卷,並七廟諱,人怪其故,答曰:「擬削平遼海,頒示國家正朔,使夷俗遵奉焉。」至是皆如其言。



 麟德二年,封泰山,仁軌領新羅及百濟、耽羅、倭四國酋長赴會,高宗甚悅,擢拜大司憲。乾封
 元年,遷右相,兼檢校太子左中護,累前後戰功,封樂城縣男。三年,為熊津道安撫大使,兼浿江道總管,副司空李勣討平高麗。總章二年,軍回,以疾辭職,加金紫光祿大夫,聽致仕。咸亨元年,復授隴州刺史。三年,徵拜太子左庶子、同中書門下三品,監修國史。五年,為雞林道大總管,東伐新羅。仁軌率兵徑度瓠盧河,破其北方大鎮七重城。以功進爵為公,並子侄三人,並授上柱國。州黨榮之,號其所居為樂城鄉三柱里。上元二年,拜尚書左
 僕射、同中書門下三品,兼太子賓客,依舊監修國史。儀鳳二年,以吐蕃入寇,命仁軌為洮河道行軍鎮守大使。仁軌每有奏請,多被中書令李敬玄抑之,由是與敬玄不協。仁軌知敬玄素非邊將才,冀欲中傷之,上言西蕃鎮守事非敬玄莫可。高宗遽命敬玄代之。敬玄至洮河軍,尋為吐蕃所敗。永隆二年,兼太子太傅。未幾,以老乞骸骨,聽解尚書左僕射,以太子太傅依舊知政事。永淳元年,高宗幸東都,皇太子京師監國,遣仁軌與侍中裴
 炎、中書令薛元超留輔太子。二年,太子赴東都,又令太孫重照京師留守,仍令仁軌為副。則天臨朝,加授特進,復拜尚書左僕射、同中書門下三品,專知留守事。仁軌復上疏辭以衰老,請罷居守之任,因陳呂後禍敗之事,以申規諫。則天使武承嗣齎璽書往京慰喻之曰:「今日以皇帝諒暗不言,眇身且代親政。遠勞勸誡,復表辭衰疾,怪望既多,徊徨失據。又云『呂后見嗤於後代,祿、產貽禍於漢朝』,引喻良深,愧慰交集。公忠貞之操,終始不渝;
 勁直之風,古今罕比。初聞此語,能不罔然;靜而思之,是為龜鏡。且端揆之任,儀刑百闢,況公先朝舊德,遐邇具瞻。願以匡救為懷,無以暮年致請。」尋進封郡公。垂拱元年,從新令改為文昌左相、同鳳閣鸞臺三品。尋薨,年八十四。則天廢朝三日,令在京百官以次赴吊,冊贈開府儀同三司、並州大都督,陪葬乾陵,賜其家實封三百戶。仁軌雖位居端揆,不自矜倨。每見貧賤時故人,不改布衣之舊。初為陳倉尉,相工袁天綱謂曰:「君終當位鄰臺
 輔,年將九十。」後果如其言。仁軌身經隋末之亂,輯其見聞,著《行年記》,行於代。



 子浚,官至太子中舍人。垂拱二年,為酷吏所陷,被殺,妻子籍沒。中宗即位,以仁軌春宮舊僚,追贈太尉。浚子冕,開元中,為秘書省少監,表請為仁軌立碑,謚曰文獻。



 史臣韋述曰:世稱劉樂城與戴至德同為端揆,劉則甘言接人,以收物譽;戴則正色拒下,推美於君。故樂城之善於今未弭,而戴氏之勣無所聞焉。嗚呼!高名美稱,或因邀飾而致遠;深仁至行,或以韜晦
 而莫傳。豈唯劉、戴而然,蓋自古有之矣。故孔子曰:「眾好之,必察焉;眾惡之,必察焉。」非夫聖智,鮮不惑也。且劉公逞其私忿,陷人之所不能,覆徒貽國之恥,忠恕之道,豈其然乎?



 郝處俊,安州安陸人也。父相貴,隋末,與妻父許紹據硤州歸國,以功授滁州刺史,封甑山縣公。處俊年十歲餘,其父卒於滁州,父之故吏賻送甚厚,僅滿千餘匹,悉辭不受。及長,好讀《漢書》,略能暗誦。貞觀中,本州進士舉,吏
 部尚書高士廉甚奇之,解褐授著作佐郎,襲爵甑山縣公。兄弟篤睦,事諸舅甚謹。再轉滕王友,恥為王官,遂棄官歸耕。久之,召拜太子司議郎,五遷吏部侍郎。乾封二年,改為司列少常伯。屬高麗反叛,詔司空李勣為浿江道大總管,以處俊為副。嘗次賊城,未遑置陣,賊徒奄至,軍中大駭。處俊獨據胡床,方餐乾糧,乃潛簡精銳擊敗之,將士多服其膽略。總章二年,拜東臺侍郎,尋同東西臺三品。咸亨初,高宗幸東都,皇太子於京師監國,盡留
 侍臣戴至德、張文瓘等以輔太子,獨以處俊從。時東州道總管高侃破高麗餘眾於安市城,奏稱有高麗僧言中國災異,請誅之。上謂處俊曰:「朕聞為君上者,以天下之目而視,以天下之耳而聽,蓋欲廣聞見也。且天降災異,所以警悟人君。其變茍實,言之者何罪?其事必虛,聞之者足以自戒。舜立謗木,良有以也。欲箝天下之口,其可得乎?此不足以加罪。」特令赦之。因謂處俊曰:「王者無外,何藉於守御。雖然,重門擊柝,蓋備不虞,方知禁衛在
 於謹肅。朕嘗以秦法猶為太寬,荊軻匹夫耳,而匕首竊發,始皇駭懼,莫有拒者,豈不由積習寬慢使其然乎?」處俊對曰:「此由法急所致,非寬慢也。」上曰:「何以知之?」對曰:「秦法:輒升殿者,夷三族。人皆懼族,安有敢拒者?逮乎魏武,法尚峻。臣見《魏令》云:『京城有變,九卿各居其府。』其後嚴才作亂,與其徒屬數十人攻左掖門,魏武登銅雀臺遠望,無敢救者。時王修為奉常,聞變召車馬,未至,便將官屬步至宮門。魏武望見之,曰:『彼來者必王修乎!』此由
 王修察變知機,違法赴難。向各守法,遂成其禍。故王者設法敷化,不可以太急。夫政寬則人慢,政急則人無所措手足。聖王之道,寬猛相濟。《詩》曰『不懈於位,人之攸塈』,謂仁政也;又曰『式遏寇虐,無俾作慝』,謂威刑也。《洪範》曰『高明柔克,沉潛剛克』,謂中道也。」上曰:「善。」又有胡僧盧伽阿逸多受詔合長年藥,高宗將餌之。處俊諫曰:「修短有命,未聞萬乘之主,輕服蕃夷之藥。昔貞觀末年,先帝令婆羅門僧那羅邇娑寐依其本國舊方合長生藥。胡人
 有異術,徵求靈草秘石,歷年而成。先帝服之,竟無異效,大漸之際,名醫莫知所為。時議者歸罪於胡人,將申顯戮,又恐取笑夷狄,法遂不行。龜鏡若是,惟陛下深察。」高宗納之,但加盧伽為懷化大將軍,不服其藥。尋而官名復舊。處俊授黃門侍郎。三年,加銀青光祿大夫,轉中書侍郎。四年,監修國史。上元元年,高宗御含元殿東翔鸞閣觀大酺。時京城四縣及太常音樂分為東西兩朋,帝令雍王賢為東朋,周王諱為西朋,務以角勝為樂。處俊
 諫曰::「臣聞禮所以示童子無誑者,恐其欺詐之心生也。伏以二王春秋尚少,意趣未定,當須推多讓美,相敬如一。今忽分為二朋,遞相誇競。且俳優小人,言辭無度,酣樂之後,難為禁止,恐其交爭勝負,譏誚失禮。非所以導仁義,示和睦也。」高宗矍然曰:「卿之遠識,非眾人所及也。」遽令止之。尋代閻立本為中書令。歲餘,兼太子賓客、檢校兵部尚書。



 三年,高宗以風疹欲遜位,令天后攝知國事,與宰相議之。處俊對曰:「嘗聞禮經云:『天子理陽道,後
 理陰德。』則帝之與後,猶日之與月,陽之與陰,各有所主守也。陛下今欲違反此道,臣恐上則謫見於天,下則取怪於人。昔魏文帝著令,身崩後尚不許皇后臨朝,今陛下奈何遂欲躬自傳位於天後?況天下者,高祖、太宗二聖之天下,非陛下之天下也。陛下正合謹守宗廟,傳之子孫,誠不可持國與人,有私於後族。伏乞特垂詳納。」中書侍郎李義琰進曰:「處俊所引經旨,足可依憑,惟聖慮無疑,則蒼生幸甚。」帝曰:「是。」遂止。儀鳳二年,加金紫光祿
 大夫,行太子左庶子,並依舊知政事,監修國史。四年,代張文瓘為侍中。處俊性儉素,土木形骸,自參綜朝政,每與上言議,必引經籍以應對,多有匡益,甚得大臣之體。侍中、平恩公許圉師,即處俊之舅,早同州里,俱宦達於時。又其鄉人田氏、彭氏,以殖貨見稱。有彭志筠,顯慶中,上表請以家絹布二萬段助軍,詔受其絹萬匹,特授奉議郎,仍布告天下。故江、淮間語曰:「貴如許、郝,富若田、彭。」處俊遷太子少保。開耀元年薨,年七十五,贈開府儀同
 三司、荊州大都督。高宗甚傷悼之,顧謂侍臣曰:「處俊志存忠正,兼有學識。至於雕飾服玩,雖極知無益,然常人不能抑情棄舍,皆好尚奢侈,處俊嘗保其質素,終始不渝。雖非元勛佐命,固亦多時驅使。又見遺表,憂國忘家,今既云亡,深可傷惜。」即於光順門舉哀一日,不視事,終祭以少牢,贈絹布八百段、米粟八百碩。令百官赴哭,給靈輿,並家口遞還鄉,官供葬事。其子秘書郎北叟上表辭所贈賜及葬遞之事,高宗不許。侍中裴炎曰:「處俊臨
 亡,臣往見之,屬臣曰:『生既無益明時,死後何宜煩費。瞑目之後,儻有恩賜贈物,及歸鄉遞送,葬日營造,不欲勞官司供給。』」高宗深嘉嘆之,從其遺意,唯加贈物而已。處俊孫象賢,垂拱中為太子通事舍人,坐事伏誅,臨刑言多不順。則天大怒,令斬訖,仍支解其體,發其父母墳墓,焚爇尸體,處俊亦坐斫棺毀柩。自此法司每將殺人,必先以木丸塞其口,然後加刑,訖於則天之代。



 裴行儉,絳州聞喜人。曾祖伯鳳,周驃騎大將軍、汾州刺
 史、瑯邪郡公。祖定高,馮翊郡守,襲封瑯邪公。父仁基,隋左光祿大夫,陷於王世充,後謀歸國,事洩遇害;武德中,贈原州都督,謚曰忠。行儉幼以門廕補弘文生。貞觀中,舉明經,拜左屯衛倉曹參軍。時蘇定方為大將軍,甚奇之,盡以用兵奇術授行儉。顯慶二年,六遷長安令。時高宗將廢皇后王氏而立武昭儀,行儉以為國家憂患必從此始,與太尉長孫無忌、尚書左僕射褚遂良私議其事,大理袁公瑜於昭儀母榮國夫人譖之,由是左授西
 州都督府長史。麟德二年,累拜安西大都護,西域諸國多慕義歸降,徵拜司文少卿。總章中,遷司列少常伯。咸亨初,官名復舊,改為吏部侍郎,與李敬玄為貳,同時典選十餘年,甚有能名,時人稱為裴、李。行儉始設長名姓歷榜,引銓注等法,又定州縣升降、官資高下,以為故事。上元二年,加銀青光祿大夫。高宗以行儉工於草書。嘗以絹素百卷,令行儉草書《文選》一部,帝覽之稱善,賜帛五百段。行儉嘗謂人曰:「褚遂良非精筆佳墨,未嘗輒書,
 不擇筆墨而妍捷者,唯餘及虞世南耳。」三年,吐蕃背叛,詔行儉為洮州道左二軍總管。尋又為泰州鎮撫右軍總管,並受元帥周王節度。儀鳳二年,十姓可汗阿史那匐延都支及李遮匐扇動蕃落,侵逼安西,連和吐蕃,議者欲發兵討之。行儉建議曰:「吐蕃叛渙,干戈未息,敬玄、審禮,失律喪元,安可更為西方生事?今波斯王身沒,其子泥涅師師充質在京,望差使往波斯冊立,即路由二蕃部落,便宜從事,必可有功。」高宗從之,因命行儉冊送
 波斯王,仍為安撫大食使。途經莫賀延磧,屬風沙晦暝,導者益迷。行儉命下營,虔誠致祭,令告將吏,泉井非遙。俄而雲收風靜,行數百步,水草甚豐,後來之人,莫知其處。眾皆悅服,比之貳師將軍。至西州,人吏郊迎,行儉召其豪傑子弟千餘人隨己而西。乃揚言紿其下曰:「今正炎蒸,熱阪難冒,涼秋之後,方可漸行。」都支覘知之,遂不設備。行儉仍召四鎮諸蕃酋長豪傑謂曰:「憶昔此游,未嘗厭倦,雖還京輦,無時暫忘。今因是行,欲尋舊賞,
 誰能從吾獵也?」是時蕃酋子弟投募者僅萬人。行儉假為畋游,教試部伍,數日,遂倍道而進。去都支部落十餘里,先遣都支所親問其安否,外示閑暇,似非討襲,續又使人趣召相見。都支先與遮匐通謀,秋中擬拒漢使,卒聞軍到,計無所出,自率兒侄首領等五百餘騎就營來謁,遂擒之。是日,傳其契箭,諸部酋長悉來請命,並執送碎葉城。簡其精騎,輕齎曉夜前進,將虜遮匐。途中果獲都支還使,與遮匐使同來。行儉釋遮匐行人,令先往曉喻其
 主,兼述都支已擒,遮匐尋復來降。於是將吏已下立碑於碎葉城以紀其功,擒都支、遮匐而還。高宗廷勞之曰:「比以西服未寧,遣卿總兵討逐,孤軍深入,經途萬里。卿權略有聞,誠節夙著,兵不血刃,而兇黨殄滅。伐叛柔服,深副朕委。」尋又賜宴。謂行儉曰:「卿文武兼資,今故授卿二職。」即日拜禮部尚書,兼檢校右衛大將軍。



 調露元年,突厥阿史德溫傅反,單于管內二十四州並叛應之,眾數十萬。單于都護蕭嗣業率兵討之,反為所敗。於是以
 行儉為定襄道行軍大總管,率太僕少卿李思文、營州都督周道務等部兵十八萬,並西軍程務挺、東軍李文暕等總三十餘萬,連亙數千里,並受行儉節度。唐世出師之盛,未之有也。行儉行至朔州,知蕭嗣業以運糧被掠,兵多餒死,遂詐為糧車三百乘,每車伏壯士五人,各齎陌刀、勁弩,以羸兵數百人援車,兼伏精兵,令居險以待之。賊果大下,羸兵棄車散走。賊驅車就泉水。解鞍牧馬,方擬取糧,車中壯士齊發,伏兵亦至,殺獲殆盡,餘眾
 奔潰。自是續遣糧車,無敢近之者。及軍至單于之北,際晚下營,壕塹方周,遽令移就崇岡。將士皆以士眾方就安堵,不可勞擾,行儉不從,更令促之。比夜,風雨暴至,前設營所水深丈餘,將士莫不嘆伏。賊眾於黑山拒戰,行儉頻戰皆捷,前後殺虜不可勝數。偽可汗泥熟匐為其下所殺,以其首來降,又擒其大首領奉職而還。餘黨走依狼山。行儉既回,阿史那伏念又偽稱可汗,與溫傅合勢,鳩集餘眾。明年,行儉復總諸軍討之。頓軍於代州之
 陘口,縱反間說伏念與溫傅,令相猜貳。伏念恐懼,密送降款,仍請自效。行儉不洩其事,而密表以聞。數日,有煙塵漲天而至,斥候惶惑來白,行儉召三軍謂曰:「此是伏念執溫傅來降,非他。然受降如受敵,但須嚴備。」更遣單使迎前勞之。少間,伏念果率其屬縛溫傅詣軍門請罪,盡平突厥餘黨。高宗大悅,遣戶部尚書崔知悌赴軍勞之。侍中裴炎害行儉之功,總管程務挺、張虔勖上言:「伏念為子營逼逐,又磧北回紇等同向南逼之,窘急而降。」
 由是行儉之功不錄,斬伏念及溫傅於都市。行儉嘆曰:「渾、浚前事,古今恥之。但恐殺降之後,無復來者。」因稱疾不出,以勛封聞喜縣公。永淳元年,十姓偽可汗車薄反叛,詔復以行儉為金牙道大總管,率十將軍以討之。師未行。其年四月,行儉病卒,年六十四,贈幽州都督,謚曰獻。特詔令皇太子差六品京官一人檢校家事,五六年間,待兒孫稍成長日停。中宗即位,追贈揚州大都督。



 有集二十卷,撰《草字雜體》數萬言,並傳於代。又撰《選譜》十
 卷,安置軍營、行陣部統、克料勝負、甄別器能等四十六訣,則天令秘書監武承嗣詣宅,並密收入內。行儉尤曉陰陽、算術,兼有人倫之鑒。自掌選及為大總管,凡遇賢俊,無不甄採,每制敵摧兇,必先期捷日。時有後進楊炯、王勃、盧照鄰、駱賓王並以文章見稱,吏部侍郎李敬玄盛為延譽,引以示行儉,行儉曰:「才名有之,爵祿蓋寡。楊應至令長,餘並鮮能令終。」是時,蘇味道、王劇未知名,因調選,行儉一見,深禮異之。仍謂曰:「有晚年子息,恨不見
 其成長。二公十數年當居衡石,願記識此輩。」其後相繼為吏部。皆如其言。行儉嘗所引偏裨,有程務挺、張虔勖、崔智辯、王方翼、黨金毗、劉敬同、郭待封、李多祚、黑齒常之,盡為名將,至刺史、將軍者數十人。其所知賞,多此類也。行儉嘗令醫人合藥,請犀角、麝香,送者誤遺失,已而惶懼潛竄。又有敕賜馬及新鞍,令史輒馳驟,馬倒鞍破,令史亦逃。行儉並委所親招到,謂曰:「爾曹豈相輕耶?皆錯誤耳。」待之如故。初,平都支、遮匐,大獲瑰寶,蕃酋將士
 願觀之,行儉因宴設,遍出歷示。有馬腦盤,廣二尺餘,文彩殊絕。軍吏王休烈捧盤,歷階趨進,誤躡衣,足跌便倒,盤亦隨碎,休烈驚惶,叩頭流血,行儉笑而謂曰:「爾非故也,何至於是!」更不形顏色。詔賜都支等資產金器皿三千餘事,駝馬稱是,並分給親故並副使已下,數日便盡。少子光庭,開元中為侍中,以恩例贈行儉為太尉。



 光庭早孤。母庫狄氏,則天時召入宮,甚見親待,光庭由是累遷太常丞。後以武三思之婿緣坐,左遷郢州司馬。開元
 初,六遷右率府中郎將,擢授司門郎中。歲餘,轉兵部郎中。光庭沉靜少言,寡於交游,既歷清要,時人初未許之。及在職,公務修整,眾方嘆伏焉。十三年,將有事於岱嶽,中書令張說以大駕東巡,京師空虛,恐夷狄乘間竊發,議欲加兵守邊,以備不虞,召光庭謀兵事。光庭曰:「封禪者,所以告成功也。夫成功者,恩德無不及,百姓無不安,萬國無不懷。今將告成而懼夷狄,何以昭德也?大興力役,用備不虞,且非安人也。方謀會同而阻戎心,又非懷
 遠也。有此三者,則名實乖矣。且諸蕃之國,突厥為大,贄幣往來,願修恩好有年矣。今茲遣一使徵其大臣赴會,必欣然應命。突厥受詔,則諸蕃君長必相率而來。雖偃旗息鼓,高枕有餘矣。」說曰:「善。吾所不及矣。」因奏而行之,尋轉鴻臚少卿。東封還,遷兵部侍郎。十七年,拜中書侍郎,同中書門下平章事,尋兼御史大夫。無幾,遷黃門侍郎,依舊知政事。從巡五陵回,拜侍中,兼吏部尚書,又加弘文館學士。光庭乃撰《瑤山往則》及《維城前軌》各壹卷,
 上表獻之。手制褒美,賜絹五百匹,上令皇太子已下於光順門與光庭相見,以重其諷誡之意。光庭又引壽安丞李融、拾遺張琪、著作左郎司馬利賓等,令直弘文館,撰《續春秋傳》。上表請以經為御撰,而光庭等依左氏之體為之作傳,上又手制褒賞之。光庭委筆削於李融,書竟不就。時有上書請以皇室為金德者,中書令蕭嵩奏請集百僚詳議。光庭以國家符命久著史策,若有改易,恐貽後學之誚,密奏請依舊為定,乃下詔停百僚集議
 之事。二十年,扈從祠后土,加光祿大夫,封正平男。尋卒,年五十八,優制贈太師,輟朝三日。初,光庭與蕭嵩爭權不協。及為吏部,奏用循資格,並促選限至正月三十日令畢,其流外行署,亦令門下省之。光庭卒後,嵩又奏請一切罷之,光庭所引進者盡出為外職。時有門下主事閻麟之,為光庭腹心,專知吏部選官,每麟之裁定,光庭隨而下筆,時人語曰:「麟之口,光庭手。」太常博士孫琬將議光庭謚,以其用循資格,非獎勸之道,建議謚為「克」。時
 人以為希嵩意旨。上聞而特下詔,賜謚曰忠獻,仍令中書令張九齡為其碑文。史官韋述以改謚為非,論之曰:《春秋》之義,諸侯死王事者,葬之加一等,嘉其有功而不及其賞也。爰至漢、魏,則襚之印綬,寵被窀穸,唯德是褒,豈虛授也!近代已來,寵贈無紀,或以職位崇顯,一切優錫;或以子孫榮貴,恩例所加,賢愚虛實,為一貫矣。裴光庭以守法之吏,驟登相位,踐歷機衡,豈不多愧?贈以師範,何其濫歟!張燕公有扶翊之勛,居講諷之舊,秩躋九
 命,官歷二端,議者猶謂贈之過當,況光庭去斯猶遠,何妄竊之甚哉!蓋名器假人,昔賢之所惋也。



 史臣曰:昔晉侯選任將帥,取其說《禮》《樂》而敦《詩》《書》,良有以也。夫權謀方略,兵家之大經,邦國系之以存亡,政令因之而強弱,則馮眾怙力,豨勇虎暴者,安可輕言推轂授任哉!故王猛、諸葛亮振起窮巷,驅駕豪傑,左指右顧,廓定霸圖,非他道也,蓋智力權變,適當其用耳。劉樂城、裴聞喜,文雅方略,無謝昔賢,治戎安邊,綽有心術,儒將
 之雄者也。天后預政之時,刑峻如壑,多以諛佞希恩,而樂城、甑山,昌言規正,若時無君子,安及此言?正平銓藻吏能,文學政事,頗有深識。而前史譏其謬謚,有涉陳壽短武侯應變之論乎!非通論也。



 贊曰:殷禮阿衡,周師呂尚。王者之兵,儒者之將。樂城、聞喜,當仁不讓。管、葛
 之譚,
 是吾心匠。



\end{pinyinscope}