\article{卷二百四}

\begin{pinyinscope}

 ○突厥之始,啟民之前,《隋書》載之備矣,只以入國之事而述之。



 始畢可汗咄吉者,啟民可汗子也。隋大業中嗣位,值天
 下大亂,中國人奔之者眾。其族強盛,東自契丹、室韋,西盡吐谷渾、高昌諸國,皆臣屬焉。控弦百餘萬,北狄之盛,未之有也。高視陰山,有輕中夏之志。



 可汗者,猶古之單于;妻號可賀敦,猶古之閼氏也。其子弟謂之特勒,別部領兵者皆謂之設。其大官屈律啜,次阿波,次頡利發,次吐屯,次俟斤,並代居其官而無員數,父兄死則子弟承襲。



 高祖起義太原,遣大將軍府司馬劉文靜聘於始畢,引以為援。始畢遣其特勒康稍利等獻馬千匹,會於絳
 郡。又遣二千騎助軍,從平京城。及高祖即位,前後賞賜,不可勝紀。始畢自恃其功,益驕踞;每遣使者至長安,頗多橫恣。高祖以中原未定,每優容之。



 武德元年,始畢使骨咄祿特勒來朝,宴於太極殿,奏《九部樂》,賚錦彩布絹各有差。二年二月,始畢帥兵渡河,至夏州,賊帥梁師都出兵會之,謀入抄掠。授馬邑賊帥劉武周兵五百餘騎,遣入句注,又追兵大集,欲侵太原。是月,始畢卒,其子什缽苾以年幼不堪嗣位,立為泥步設,使居東偏,直幽州
 之北。立其弟俟利弗設,是為處羅可汗。



 處羅可汗嗣位,又以隋義成公主為妻,遣使入朝告喪。高祖為之舉哀,廢朝三日,詔百官就館吊其使者,又遣內史舍人鄭德挺往吊處羅,賜物三萬段。處羅此後頻遣使朝貢。先是,隋煬帝蕭後及齊王暕之子政道,陷於竇建德。三年二月,處羅迎之,至於牙所,立政道為隋王。隋末中國人在虜庭者,悉隸於政道,行隋正朔,置百官,居於定襄城,有徒一萬。時太宗在籓,受詔討劉武周,師
 次太原,處羅遣其弟步利設率二千騎與官軍會。六月,處羅至並州,總管李仲文出迎勞之。留三日,城中美婦人多為所掠,仲文不能制。俄而,處羅卒,義成公主以其子奧射設醜弱,廢不立之,遂立處羅之弟咄苾,是為頡利可汗。



 頡利可汗者,啟民可汗第三子也。初為莫賀咄設,牙直五原之北。高祖入長安,薛舉猶據隴右,遣其將宗羅攻陷平涼郡,北與頡利連結。高祖患之,遣光祿卿宇文
 歆齎金帛以賂頡利。歆說之,令絕交於薛舉。初,隋五原太守張長遜,因亂以其所部五原城隸於突厥。歆又說頡利遣長遜入朝,以五原地歸於我。頡利並從之,因發突厥兵及長遜之眾,並會於太宗軍所。武德三年,頡利又納義城公主為妻,以始畢之子什缽苾為突利可汗,遣使入朝,告處羅死。高祖為之罷朝一日,詔百官就館吊其使。



 頡利初嗣立,承父兄之資,兵馬強盛。有憑陵中國之志。高祖以中原初定,不遑外略,每優容之,賜與不
 可勝計。頡利言辭悖傲,求請無厭。四年四月,頡利自率萬餘騎,與馬邑賊苑君璋將兵六千人共攻雁門。定襄王李大恩擊走之。先是漢陽公蘇瑰、太常卿鄭元璹、左驍衛大將軍長孫順德等各使於突厥,頡利並拘之。我亦留其使,前後數輩。至是為大恩所挫,於是乃懼,仍放順德還,更請和好。獻魚膠數十斤,欲充二國同於此膠。高祖嘉之,放其使者特勒熱寒、阿史德等還蕃,賜以金帛。



 五年春,李大恩奏言突厥饑荒,馬邑可圖。詔大恩與
 殿內少監獨孤晟帥師討苑君璋,期以二月會於馬邑。晟後期不至,大恩不能獨進,頓兵新城以待之。頡利遣數萬騎與劉黑闥合軍,進圍大恩。王師敗績,大恩歿於陣,死者數千人。六月,劉黑闥又引突厥萬餘騎入抄河北。頡利復自率五萬騎南侵,至於汾州。又遣數千騎西入靈、原等州,詔隱太子出豳州道,太宗出蒲州道以討之。時頡利攻圍並州,又分兵入汾、潞等州,掠男女五千餘口,聞太宗兵至蒲州,乃引兵出塞。



 七年八月,頡利、突
 利二可汗舉國入寇,道自原州,連營南上。太宗受詔北討,齊王元吉隸焉。初,關中霖雨,糧運阻絕,太宗頗患之,諸將憂見於色,頓兵於豳州。頡利、突利率萬餘騎奄至城西,乘高而陣,將士大駭。太宗乃親率百騎馳詣虜陣,告之曰:「國家與可汗誓不相負,何為背約深入吾地?我秦王也,故來一決。可汗若自來,我當與可汗兩人獨戰;若欲兵馬總來,我唯百騎相御耳。」頡利弗之測,笑而不對。太宗又前,令騎告突利曰:「爾往與我盟,急難相救;爾
 今將兵來,何無香火之情也?亦宜早出,一決勝負。」突利亦不對。太宗前,將渡溝水,頡利見太宗輕出,又聞香火之言,乃陰猜突利。因遣使曰:「王不須渡,我無惡意,更欲共王自斷當耳。」於是稍引卻,各斂軍而退。太過因縱反間於突利,突利悅而歸心焉,遂不欲戰。其叔侄內離,頡利欲戰不可,因遣突利及夾畢特勒阿史那思摩奉見請和,許之。突利因自托於太宗,願結為兄弟。思摩初奉見,高祖引升御榻,頓顙固辭。高祖謂曰:「頡利誠心遣特
 勒朝拜,今見特勒,如見頡利。」固引之,乃就坐,尋封思摩為和順王。



 八年七月,頡利集兵十餘萬,大掠朔州,又襲將軍張瑾於太原。瑾全軍並沒,脫身奔於李靖。出師拒戰,頡利不得進,屯於並州。太宗帥師討之,次蒲州;頡利引兵而去,太宗旋師。



 九年七月,頡利自率十餘萬騎進寇武功,京師戒嚴。己卯,進寇高陵,行軍總管左武候大將軍尉遲敬德與之戰於涇陽,大破之,獲俟斤阿史德烏沒啜,斬首千餘級。癸未,頡利遣其腹心執失思力入
 朝為覘,自張形勢云:「二可汗總兵百萬,今已至矣。」太宗謂之曰:「我與突厥面自和親,汝則背之,我實無愧。又義軍入京之初,爾父子並親從,我賜汝玉帛,前後極多,何故輒將兵入我畿縣?爾雖突厥,亦須頗有人心,何故全忘大恩,自誇強盛?我當先戮爾矣!」思力懼而請命,太宗不許,縶之於門下省。



 太宗與侍中高士廉、中書令房玄齡、將軍周範馳六騎幸渭水之上,與頡利隔津而語,責以負約。其酋帥大驚,皆下馬羅拜。俄而,眾軍繼至,頡利
 見軍容大盛,又知思力就拘,由是大懼。太宗獨與頡利臨水交言,麾諸軍卻而陣焉。蕭瑀以輕敵固諫於馬前,上曰:「吾已籌之,非卿所知也。突厥所以掃其境內,直入渭濱,應是聞我國家初有內難,朕又新登九五,將謂不敢拒之。朕若閉門,虜必大掠,強弱之勢,在今一舉。朕故獨出,以示輕之;又耀軍容,使知必戰。事出不意,乖其本圖,虜入既深,理當自懼。與戰則必克,與和則必固,制服匈奴,自茲始矣!」是日,頡利請和,詔許焉。車駕即日還
 宮。乙酉,又幸城西,刑白馬與頡利同盟於便橋之上,頡利引兵而退。蕭瑀進曰:「初,頡利之未和也,謀臣猛將多請戰,而陛下不納,臣以為疑。既而虜自退,其策安在?」上曰:「我觀突厥之兵,雖眾而不整,君臣之計,唯財利是視。可汗獨在水西,酋帥皆來謁我,我因而襲擊其眾,勢同拉朽。然我已令無忌、李靖設伏於幽州以待之,虜若奔還,伏兵邀其前,大軍躡其後,覆之如反掌矣!我所以不戰者,即位日淺,為國之道,安靜為務,一與虜戰,必有死
 傷;又匈虜一敗,或當懼而修德,結怨於我,為患不細。我今卷甲韜戈,陷以玉帛,頑虜驕恣,必自此始,破亡之漸,其在茲乎!將欲取之,必固與之,此之謂也!」九月,頡利獻馬三千匹,羊萬口,上不受;詔頡利所掠中國戶口者悉令歸之。



 貞觀元年,陰山已北薛延陀、回紇、拔也古等部皆相率背叛,擊走其欲谷設。頡利遣突利討之,師又敗績,輕騎奔還。頡利怒,拘之十餘日;突利由是怨望,內欲背之。其國大雪,平地數尺,羊馬皆死,人大饑,乃懼我師
 出乘其弊。引兵入朔州,揚言會獵,實設備焉。侍臣咸曰:「夷狄無信,先自猜疑,盟後將兵,忽踐疆境。可乘其便,數以背約,因而討之。」太宗曰:「匹夫一言,尚須存信,何況天下主乎!豈有親與之和,利其災禍而乘危迫險以滅之耶?諸公為可,朕不為也。縱突厥部落叛盡,六畜皆死,朕終示以信,不妄討之;待其無禮,方擒取耳。」



 二年,突利遣使奏言與頡利有隙,奏請擊之,詔秦武通以並州兵馬隨便應接。三年,薛延陀自稱可汗於漠北,遣使來貢方
 物。頡利始稱臣,尚公主,請修婿禮。頡利每委任諸胡,疏遠族類,胡人貪冒,性多翻覆,以故法令滋彰,兵革歲動,國人患之,諸部攜貳。頻年大雪,六畜多死,國中大餒,頡利用度不給,復重斂諸部,由是下不堪命,內外多叛之。上以其請和,後復援梁師都,詔兵部尚書李靖、代州都督張公謹出定襄道。並州都督李勣、右武衛將軍丘行恭出通漢道,左武衛大將軍柴紹出金河道,衛孝節出恆安道,薛萬徹出暢武道,並受靖節度以討之。十二月,
 突利可汗及鬱射設、廕奈特勒等,並帥所部來奔。



 四年正月,李靖進屯惡陽嶺,夜襲定襄,頡利驚擾,因徙牙於磧口。胡酋康蘇密等遂以隋蕭後及楊政道來降。二月,頡利計窘,竄於鐵山,兵尚數萬,使執失思力入朝謝罪,請舉國內附。太宗遣鴻臚卿唐儉、將軍安修仁持節安撫之,頡利稍自安。靖乘間襲擊,大破之,遂滅其國。頡利乘千里馬,獨騎奔於從侄沙缽羅部落。三月,行軍副總管張寶相率眾奄至沙缽羅營,生擒頡利送於京師。太
 宗謂曰:「凡有功於我者,必不能忘,有惡於我者,終亦不記。論爾之罪狀,誠為不小,但自渭水曾面為盟,從此以來,未有深犯,所以錄此,不相責耳!」仍詔還其家口,館於太僕,稟食之。頡利鬱鬱不得志,與其家人或相對悲歌而泣。帝見羸憊,授虢州刺史,以彼土多麞鹿,縱其畋獵,庶不失物性。頡利辭不願往,遂授右衛大將軍,賜以田宅。



 五年,太宗謂侍臣曰:「天道福善禍淫,事猶影響。昔啟民亡國奔隋,文帝不吝粟帛,大興士眾,營衛安置,乃得
 存立。既而強盛,當須子子孫孫思念報德。才至始畢,即起兵圍煬帝於雁門,及隋國將亂,又恃強深入,遂使昔安立其家國者,身及子孫,並為頡利兄弟之所屠戮。今頡利破亡,豈非背恩忘義所致也!」



 八年卒,詔其國人葬之,從其俗禮,焚尸於灞水之東,贈歸義王,謚曰荒。其舊臣胡祿達官吐谷渾邪自刎以殉。



 渾邪者,頡利之母婆施氏之媵臣也。頡利初誕,以付渾邪,至是哀慟而死。太宗聞而異之,贈中郎將,仍葬於頡利墓側,樹碑以紀之。



 突利可汗什缽苾者,始畢可法之嫡子,頡利之侄也。隋大業中,突利年數歲,始畢遣領其東牙之兵,號為泥步設。隋淮南公主之北也,遂妻之。頡利嗣位,以為突利可汗,牙直幽州之北。突利在東偏,管奚、霫等數十部,征稅無度,諸部多怨之。貞觀初,奚,霫等並來歸附,頡利怒其失眾,遣北徵延陀,又喪師,遂囚而撻焉。



 突利初自武德時,深自結於太宗;太宗亦以恩義撫之,結為兄弟,與盟而去。後頡利政亂,驟徵兵於突利,拒之不與,由是有隙。
 貞觀三年,表請入朝。上謂侍臣曰:「朕觀前代為國者,勞心以憂萬姓,世祚乃長;役人以奉其身,社稷必滅。今北蕃百姓喪亡。誠由其君不君之故也。至使突利情願入朝,若非困迫,何能至此?夷狄弱則邊境無虞,亦甚為慰。然見其顛狽,又不能不懼,所以然者,慮己有不逮,恐禍變亦爾。朕今視不能遠見,聽不能遠聞,唯藉公等盡忠匡弼,無得惰於諫諍也。」突利尋為頡利所攻,遣使來乞師。太宗謂近臣曰:「朕與突利結為兄弟,不可以不救。」杜
 如晦進曰:「夷狄無信,其來自久,國家雖為守約,彼必背之。不若因其亂而取之,所謂取亂侮亡之道。」太宗然之。因令將軍周範屯太原,以圖進取。突利乃率其眾來奔,太宗禮之甚厚,頻賜以御膳。



 四年,授右衛大將軍,封北平郡王,食邑封七百戶,以其下兵眾置順祐等州,帥部落還蕃。太宗謂曰:「昔爾祖啟民亡失兵馬,一身投隋,隋家翌立,遂至強盛,荷隋之恩,未嘗報德。至爾父始畢,乃為隋家之患,自爾已後,無歲不侵擾中國。天實禍淫,大
 降災變,爾眾散亂,死亡略盡。既事窮後,乃來投我,我所以不立爾為可汗者,正為啟民前事故也。改變前法,欲中國久安,爾宗族永固,是以授爾都督。當須依我國法,整齊所部,不得妄相侵掠,如有所違,當獲重罪。」



 五年,徵入朝,至並州,道病卒,年二十九。太宗為之舉哀,詔中書侍郎岑文本為其碑文。子賀邏鶻嗣。



 突利弟結社率,貞觀初入朝,歷位中郎將。十三年,從幸九成宮,陰結部落得四十餘人,並擁賀邏鶻,相與夜犯御營,逾第四重幕,
 引弓亂發,殺衛士數十人。折沖孫武開率兵奮擊,乃退。北走渡渭水,欲奔其部落。尋皆捕而斬之,詔原賀邏鶻,流於嶺外。



 頡利之敗也,其部落或走薛延陀,或走西域,而來降者甚眾。詔議安邊之術。朝士多言突厥恃強,擾亂中國,為日久矣。今天實喪之,窮來歸我,本非慕義之心。因其歸命,分其種落,俘之河南兗、豫之地,散居州縣,各使耕織,百萬胡虜可得化為百姓,則中國有加戶之利,塞北可常空矣。唯中書令溫彥博議請準漢建武時
 置降匈奴於五原塞下。全其部落,得為捍蔽,又不離其土俗,因而撫之,一則實空虛之地,二則示無猜心。若遣向河南兗、豫,則乖物性,故非含育之道。太宗將從之。秘書監魏徵奏言:「突厥自古至今,未有如斯之破敗者也,此是上天剿絕,宗廟神武。且其世寇中國,百姓冤仇,陛下以其降伏,不能誅滅,即宜遣還河北,居其故土。匈奴人面獸心,非我族類,強必寇盜,弱則卑服,不顧恩義,其天性也。秦、漢患其若是,故發猛將以擊之,收取河南,以
 為郡縣,陛下奈何以內地居之!且今降者幾至十萬,數年之間,孳息百倍,居我肘腋,密邇王畿,心腹之疾,將為後患,尤不可河南處也。」溫彥博奏曰:「天子之於物也,天覆地載,有歸我者,則必養之。今突厥破滅之餘,歸心降附,陛下不加憐愍,棄而不納,非天地之道,阻四夷之意,臣愚甚謂不可。遣居河南,所謂死而生之,亡而存之,懷我德惠,終無叛逆。」魏徵又曰:「晉代有魏時胡落,分居近郡,平吳已後,郭欽、江統勸武帝逐出塞外;不用欽等言,
 數年之後,遂傾瀍、洛。前代覆車,殷鑒不遠,陛下必用彥博之言,遣居河南,所謂養獸自遺患也!」彥博又曰:「聞聖人之道,無所不通,古先哲王,有教無類。突厥餘魂,以命歸我,我援護之,收居內地,稟我指麾,教以禮法,數年之後,盡為農民。選其酋首,遣居宿衛,畏威懷德,何患之有?光武居南單于內郡,為漢籓翰,終乎一代,不有叛逆。」彥博既口給,引類百端,太宗遂用其計,於朔方之地,自幽州至靈州置順、祐、化、長四州都督府,又分頡利之地
 六州,左置定襄都督府,右置雲中都督府,以統其部眾。其酋首至者,皆拜為將軍、中郎將等官,布列朝廷,五品以上百餘人,因而入居長安者數千家。自結社率之反也,太宗始患之。又上書者多雲處突厥於中國,殊謂非便,乃徙於河北,立右武候大將軍、化州都督、懷化郡王思摩為乙彌泥孰侯利苾可汗,賜姓李氏,率所部建牙於河北。



 思摩者,頡利族人也。始畢、處羅以其貌似胡人,不類突
 厥,疑非阿史那族類,故歷處羅、頡利世,常為夾畢特勒,終不得典兵為設。武德初,數來朝貢,高祖封為和順郡王。及其國亂,諸部多歸中國,唯思摩隨逐頡利,竟與同擒。太宗嘉其忠,除右武候大將軍、化州都督,令統頡利舊部落於河南之地,尋改封懷化郡王。



 及將徙於白道之北,思摩等咸憚薛延陀,不肯出塞。太宗遣司農卿郭嗣本賜延陀璽書曰:



 突厥頡利可汗未破已前,自恃強盛,抄掠中國,百姓被其殺者,不可勝紀。我發兵擊破之,
 諸部落悉歸化。我略其舊過,嘉其從善,並授官爵,同我百僚,所有部落,愛之如子,與我百姓不異。但中國禮義,不滅爾國,前破突厥,止為頡利一人為百姓之害,所以廢而黜之,實不貪其土地,利其人馬也。自黜廢頡利以後,恆欲更立可汗,是以所降部落等並置河南,任其放牧,今戶口羊馬日向滋多。元許冊立,不可失信,即欲遣突厥渡河,復其國土。我策爾延陀,日月在前,今突厥居後,後者為小,前者為大。爾在磧北,突厥居磧南,各守土
 境,鎮撫部落。若其逾越,故相抄掠,我即將兵各問其罪。此約既定,非但有便爾身,貽厥子孫,長守富貴也。」



 於是命禮部尚書趙郡王孝恭齎書就思摩部落,築壇於河上以拜之,並賜之鼓纛。突厥及胡在諸州安置者,並令渡河北,還其舊部。又以左屯衛將軍阿史那忠為左賢王,左武衛將軍阿史那泥孰為右賢王。以貳之。



 薛延陀聞太宗遣思摩渡河北,慮其部落翻附磧北,預蓄輕騎,伺至而擊之。太宗遣敕之曰:「擅相侵者,國有常刑。」延陀
 曰:「至尊遣莫相侵掠,敢不奉詔。然突厥翻覆難信,其未破前,連年殺中國人,動以千萬計。至尊破突厥,須收為奴婢,將與百姓,而反養之如子,結社率竟反,此輩獸心,不可信也。臣荷恩甚深,請為至尊誅之。」時思摩下部眾渡河者凡十萬,勝兵四萬人,思摩不能撫其眾,皆不愜服。至十七年,相率叛之,南渡河,請分處於勝、夏二州之間,詔許之。思摩遂輕騎入朝,尋授右武衛將軍,從征遼東,為流矢所中;太宗親為吮血,其見顧遇如此。未幾,卒
 於京師。贈兵部尚書、夏州都督,陪葬昭陵,立墳以象白道山,詔為立碑於化州。



 先是,貞觀中,突厥別部有車鼻者,亦阿史那之族也。代為小可汗,牙於金山之北。頡利可汗之敗,北荒諸部將推為大可汗,遇薛延陀為可汗,車鼻不敢當,遂率所部歸於延陀。為人勇烈,有謀略,頗為眾附。延陀惡而將誅之,車鼻密知其謀,竄歸於舊所,其地去京師萬里,勝兵三萬人,自稱乙注車鼻可汗。西有歌羅祿,北有結骨,皆附隸之。自延陀破後,遣其子沙
 缽羅特勒來朝,貢方物,又請身入朝。太宗遣將軍郭廣敬征之,竟不至。太宗大怒。貞觀二十三年,遣右驍衛郎將高侃潛引回紇、僕骨等兵眾襲擊之,其酋長歌邏祿泥孰闕俟利發及拔塞匐處木昆莫賀咄俟斤等,率部落背車鼻,相繼來降。永徽元年,侃軍次阿息山。車鼻聞王師至,召所部兵,皆不赴,遂攜其妻子從數百騎而遁,其眾盡降。侃率精騎追車鼻,獲之,送於京師。仍獻於社廟,又獻於昭陵。高宗數其罪而赦之,拜左武衛將軍,賜
 宅於長安,處其餘眾於鬱督軍山,置狼山都督以統之。車鼻長子羯漫陀,先統拔悉密部。車鼻未敗前,遣其子庵鑠入朝,太宗嘉之,拜左屯衛將軍,更置新黎州,以統其眾。



 車鼻既破之後,突厥盡為封疆之臣,於是分置單于、瀚海二都護府。單于都護領狼山、雲中、桑乾三都督、蘇農等一十四州;瀚海都護領瀚海、金微、新黎等七都督、仙萼、賀蘭等八州,各以其首領為都督、刺史。高宗東封泰山,狼山都督葛邏祿社利等首領三十餘人,並扈從至
 嶽下,勒名於封禪之碑。自永徽已後,殆三十年,北鄙無事。



 調露元年,單于管內突厥首領阿史德溫傅、奉職二部落,始相率反叛,立泥孰匐為可汗,二十四州並叛應之。高宗遣鴻臚卿蕭嗣業、右千牛將軍李景嘉率眾討之,反為溫傅所敗,兵士死者萬餘人。又詔禮部尚書裴行儉為定襄道行軍大總管,率太僕少卿李思文、營州都督周道務等統眾三十餘萬,討擊溫傅,大破之。泥孰匐為其下所殺,並擒奉職而還。



 永隆元年,突厥又迎頡
 利從兄之子阿史那伏念於夏州,將渡河立為可汗,諸部落復響應從之。又詔裴行儉率將軍曹繼叔、程務挺、李崇直、李文暕等討之。伏念窘急,詣行儉降。行儉遂虜伏念詣京師,斬於東市。永淳二年,突厥阿史那骨咄祿復反叛。



 骨咄祿者,頡利之疏屬,亦姓阿史那氏。其祖父本是單于右雲中都督舍利元英下首領也,世襲吐屯啜。伏念既破,骨咄祿鳩集亡散,入總材山,聚為群盜,有眾五千
 餘人。又抄掠九姓,得羊馬甚多,漸至強盛,乃自立為可汗。以其弟默啜為設,咄悉匐為葉護。時有阿史德元珍,在單于檢校降戶部落,嘗坐事為單于長史王本立所拘縶,會骨咄祿入寇,元珍請依舊檢校部落,本立許之,因而便投骨咄祿。骨咄祿得之,甚喜,立為阿波達干,令專統兵馬事。



 永淳二年,進寇蔚州。豐州都督崔智辯擊之,反為賊所殺。文明元年,又寇朔州,殺掠人吏,則天詔左武威衛大將軍程務挺為單于道安撫大使以備之。
 垂拱二年,骨咄祿又寇朔、代等州,左玉鈐衛中郎將淳于處平為陽曲道總管,與副將中郎將蒲英節率兵赴援,行至忻州,與賊戰,大敗,死者五千餘人。三年,骨咄祿及元珍又寇昌平,詔左鷹揚衛大將軍黑齒常之擊卻之。其年八月,又寇朔州,復以常之為燕然道大總管,擊賊於黃花堆,大破之。追奔四十餘里,賊眾遂散走磧北。右監門衛中郎將爨寶璧又率精兵一萬三千人出塞窮追,反為骨咄祿所敗,全軍盡沒,寶璧輕騎遁歸。初,寶
 璧見常之破賊,遽表請窮其餘黨,則天詔常之與寶璧計議,遙為聲援。寶璧以為破賊在朝夕,貪功先行。又令人出塞二千餘里覘候,見元珍等部落皆不設備,遂率眾掩襲之。既至,又遣人報賊,令得設備出戰,遂為賊所覆,寶璧坐此伏誅。則天大怒,因改骨咄祿為不卒祿。元珍後率兵討突騎施,臨陣戰死。骨咄祿,天授中病卒。



 默啜者,骨咄祿之弟也。骨咄祿死時,其子尚幼,默啜遂篡其位,自立為可汗。長壽二年,率眾寇靈州,殺掠人吏。
 則天遣白馬寺僧薛懷義為代北道行軍大總管,領十大將軍以討之,既不遇賊,尋班師焉。默啜俄遣使來朝,則天大悅,冊授左衛大將軍,封歸國公,賜物五千段。明年,復遣使請和,又加授遷善可汗。



 萬歲通天元年,契丹首領李盡忠、孫萬榮反叛,攻陷營府。默啜遣使上言:「請還河西降戶,即率部落兵馬為國家討擊契丹。」制許之。默啜遂攻討契丹,部眾大潰,盡獲其家口,默啜自此兵眾漸盛。則天尋遣使冊立默啜為特進、頡跌利施大單于、
 立功報國可汗。聖歷元年,默啜表請與則天為子,並言有女,請和親。初,咸亨中,突厥諸部落來降附者,多處之豐、勝、靈、夏、朔、代等六州,謂之降戶。默啜至是,又索此降戶及單于都護府之地,兼請農器、種子。則天初不許,默啜大怨怒,言辭甚慢,拘我使人司賓卿田歸道,將害之。時朝廷懼其兵勢,納言姚璹、鸞臺侍郎楊再思建議請許其和親,遂盡驅六州降戶數千帳,並種子四萬餘碩、農器三千事以與之,默啜浸強由此也。



 其年,則天令魏
 王武承嗣、男淮陽王延秀就納其女為妃,遣右豹韜衛大將軍閻知微攝春官尚書,右武威衛郎將楊鸞莊攝司賓卿,大齎金帛,送赴虜庭。行至黑沙南庭,默啜謂知微等曰:「我女擬嫁與李家天子兒,你今將武家兒來,此是天子兒否?我突厥積代已來,降附李家,今聞李家天子種末總盡,唯有兩兒在,我今將兵助立。」遂收延秀等,拘之別所。偽號知微為可汗,與之率眾十餘萬,襲我靜難及平狄、清夷等軍。靜難軍使左正鋒衛將軍慕容玄
 皦以兵五千人降之。俄進寇媯、檀等州,則天令司屬卿武重規為天兵中道大總管,右武威衛將軍沙吒忠義為天兵西道前軍總管,幽州都督張仁亶為天兵東道總管,率兵三十萬擊之。右羽林衛大將軍閻敬容為天兵西道後軍總管,統兵十五萬以為後援。默啜又出自恆岳道,寇蔚州,陷飛狐縣。俄進攻定州,殺刺史孫彥高,焚燒百姓廬舍,虜掠男女,無少長皆殺之。則天大怒,購斬默啜者,封王,改默啜號為斬啜。尋又圍逼趙州,長史
 唐波若翻城應之,刺史高睿抗節不從,遂遇害。則天乃立廬陵王為皇太子,令充河北道行軍大元帥。軍未發而默啜盡抄掠趙、定等州男女八九萬人,從五回道而去,所過殘殺,不可勝紀。沙吒忠義及後軍總管李多祚等皆持重兵,與賊相望,不敢戰。河北道元帥、納言狄仁傑總兵十萬追之,無所及。



 二年,默啜立其弟咄悉匐為左廂察,骨咄祿子默矩為右廂察,各主兵馬二萬餘人。又立其子匐俱為小可汗,位在兩察之上,仍主處木昆
 等十姓兵馬四萬餘人。又號為拓西可汗,自是連歲寇邊。久視元年,掠隴右諸監馬萬餘匹而去。制右肅政御史大夫魏元忠為靈武道行軍大總管,以備之,又命安北大都督相王旦為天兵道元帥,統諸軍討擊,竟未行而賊退。



 長安三年,默啜遣使莫賀達干,請以女妻皇太子之子。則天令太子男平恩王重俊、義興王重明廷立見之。默啜遣大酋移力貪汗入朝,獻馬千匹,及方物以謝許親之意。則天宴之於宿羽亭,太子、相王及朝集使
 三品以上並預會,重賜以遣之。



 中宗即位,默啜又寇靈州鳴沙縣。靈武軍大總管沙吒忠義拒戰久之,官軍敗績,死者六千餘人。賊遂進寇原、會等州,掠隴右群牧馬萬餘匹而去,忠義坐免。中宗下制絕其請婚,仍購募能斬獲默啜者封國王,授諸衛大將軍,賞物二千段。又命內外官各進破突厥諸策。右補闕盧俌上疏曰:



 臣聞有虞咸熙,苗人逆命,殷宗大化,鬼方不賓,則戎狄交侵,其來遠矣。漢高帝納婁敬之議,與匈奴和親,妻以宗女,賂
 以鉅萬,冒頓益驕,邊寇不止。則遠荒之地,兇悍之俗,難以德綏,可以威制,而降自三代,無聞上策。今匈奴不臣,擾我亭障,皇赫斯怒,將整元戎。臣聞方叔帥師,功歌周《雅》;去病耀武,勛勒燕山,則萬里折沖,在於擇將。《春秋》謀元帥,取其說《禮樂》、敦《詩書》。晉臣杜預,射不穿札,而建平吳之勛,是知中權制謀,不在一夫之勇。其蕃將沙吒忠義等身雖驍悍,志無遠圖,此乃騎將之材,本不可當大任。且師出以律,將軍死綏。秦克長平,趙括受戮;胡去馬
 邑,王恢坐誅,則棄軍有刑,古之常典。近者鳴沙之役,主將先逃,輕挫國威,須正邦憲。又其中軍既敗,陳亂矢窮,義勇之士,猶能死戰,功合紀錄,以勸戎行,賞罰既明,將士盡節,此擒敵之術也。



 臣聞以蠻夷攻蠻夷,中國之長算。故陳湯統西域而郅支滅,常惠用烏孫而匈奴敗。請購辯勇之士,班、傅之儔,旁結諸蕃,與圖攻取,此又掎角之勢也。



 臣聞昔置新秦以實塞下,宜因古法,募人徙邊,選其勝兵,免其行役,次廬伍,明教令,則狃習戎事,究識
 夷情,其所虜獲,因而賞之。近戰則守家,遠戰則利貨,趨赴鋒鏑,不勞訓誓,朝賦「楊柳」,夕歌《杕杜》,十年之後,可以久安。



 臣聞漢拜郅都,匈奴避境;趙命李牧,林胡遠竄。則朔方之安危,邊城之勝負,地方千里,制在一賢。其邊州刺史不可不慎擇,得其人而任之。蒐乘訓兵,屯田積粟,謹設烽燧,精飾戈矛,來則懲而御之,去則備而守之,此又古之善經也。去歲亢陽,天下不稔,利在保境,不可窮兵。使內郡黔黎,各安其業,擇共宰牧,輕其賦徭,事無過
 舉,爵不以私。愛人之財,節其徭役;惜人之力,不廣臺榭。察地利天時以趨耕獲,命秋獮冬狩以教戰陣。則數年之後,有勇知方,帑藏山積,金革犀利。然後整六軍,絕大漠,雷擊萬里,風掃二庭,斬蹛林之酋,懸槁街之邸,使百蠻震怖,五兵載戢,則上合天時,下順人事。理內以及外,綏近以來遠,以惠中國,以靜四方。臣少慕文儒,不習軍旅,奇正之術,多愧前良,獻替是司,輕陳瞽議。



 上覽而善之。默啜於是殺我行人假鴻臚卿臧思言。思言對賊不
 屈節,特贈鴻臚卿,仍命左屯衛大將軍張仁亶攝右御史臺大夫,充朔方道大總管以御之。仁亶始於河外築三受降城,絕其南寇之路。



 睿宗踐祚,默啜又遣使請和親。制以宋王成器女為金山公主許嫁之。默啜乃遣其男楊我支特勒來朝,授右驍衛員外大將軍。俄而睿宗傳位,親竟不成。



 初,默啜景雲中率兵西擊娑葛,破滅之。契丹及奚,自神功之後,常受其征役,其地東西萬餘里,控弦四十萬,自頡利之後最為強盛。自恃兵威,虐用其
 眾。默啜既老,部落漸多逃散。開元二年,遣其子移涅可汗及同俄特勒、妹婿火拔頡利發石阿失畢率精騎圍逼北庭。右驍衛將軍郭虔瓘嬰城固守,俄而出兵擒同俄特勒於城下,斬之。虜因退縮,火拔懼不敢歸,攜其妻來奔,制授左衛大將軍,封燕北郡王,封其妻為金山公主,賜宅一區,奴婢十人,馬十匹,物千段。明年,十姓部落左廂五咄六啜、右廂五弩失畢五俟斤及子婿高麗莫離支高文簡、睟跌都督崿跌思泰等各率其眾,相繼來
 降,前後總萬餘帳。制令居河南之舊地。授高文簡左衛員外大將軍,封遼西郡王;睟跌思泰為特進、右衛員外大將軍兼睟跌都督,封樓煩郡公。自餘首領,封拜賜物各有差。默啜女婿阿史德胡祿,俄又歸朝,授以特進。其秋,默啜與九姓首領阿布思等戰於磧北。九姓大潰,人畜多死,阿布思率眾來降。



 四年,默啜又北討九姓拔曳固,戰於獨樂河,拔曳固大敗。默啜負勝輕歸,而不設備。遇拔曳固迸卒頡質略於柳林中,突出擊默啜,斬之。便
 與入蕃使郝靈荃傳默啜首至京師。骨咄祿之子闕特勒鳩合舊部,殺默啜子小可汗及諸弟並親信略盡,立其兄左賢王默棘連,是為毗伽可汗。



 毗伽可汗以開元四年即位,本蕃號為小殺。性仁友,自以得國是闕特勒之功,固讓之。闕特勒不受,遂以為左賢王,專掌兵馬。是時奚、契丹相率款塞,突騎施蘇祿自立為可汗,突厥部落頗多攜貳,乃召默啜時衙官暾欲谷為謀主。初,默啜下衙官盡為闕特勒所殺,暾欲谷以
 女為小殺可敦,遂免死。廢歸部落,乃復用,年已七十餘,蕃人甚敬伏之。



 俄而降戶阿悉爛、睟跌思泰等復自河曲叛歸。初,降戶南至單于,左衛大將軍單于副都護張知運,盡收其器仗,令渡河而南,蕃人怨怒。御史中丞姜晦為巡邊使,蕃人訴無弓矢。不得射獵,晦悉給還之。故有抗敵之具。張知運既不設備,與降戶戰於青剛嶺,為降戶所敗。臨陣生擒知運,擬送與突厥。朔方總管薛納率兵追討之。賊至大斌縣,又為將軍郭知運所擊。賊眾
 大潰散,投黑山呼延谷,釋張知運而去。上以張知運喪師,斬之以徇。小殺既得降戶,謀欲南入為寇。暾欲谷曰:「唐主英武,人和年豐,未有間隙,不可動也。我眾新集,猶尚疲羸,須且息養三數年,始可觀變而舉。」小殺又欲修築城壁,造立寺觀。暾欲谷曰:「不可。突厥人戶寡少,不敵唐家百分之一,所以常能抗拒者,正以隨逐水草,居處無常,射獵為業,又皆習武。強則進兵抄掠,弱則竄伏山林,唐兵雖多,無所施用。若築城而居,改變舊俗,一朝失
 利,必將為唐所並。且寺觀之法,教人仁弱,本非用武爭強之道,不可置也。」小殺等深然其策。



 八年冬,御史大夫王晙俊為朔方大總管,奏請西征拔悉密,東發奚、契丹兩蕃,期以明年秋初,引朔方兵數道俱入,掩突厥衙帳於稽落河上。小殺聞之,大恐。暾欲谷曰:「拔悉密今在北庭,與兩蕃東、西相去極遠,勢必不合。王晙兵馬,計亦無能至此。必若能來,候其臨到,即移衙帳向北三日,唐兵糧盡,自然去矣。且拔悉密輕而好利,聞命必是先來,王晙
 與張嘉貞不協,奏請有所不愜,必不敢動。若王晙兵馬不來,拔悉密獨至,即須擊取之,勢易為也!」



 九年秋,拔悉密果臨突厥衙帳,而王晙兵及兩蕃不至。拔悉密懼而引退。突厥欲擊之,暾欲谷曰:「此眾去家千里,必將死戰,未可擊也,不如以兵躡之。」去北庭二百里,暾欲谷分兵間道先掩北庭,因縱卒擊拔悉密之還眾。遂散走投北庭,而城陷不得入,盡為突厥所擒,並虜其男女而還。暾欲谷回兵,因而出赤亭以掠涼州羊馬。時楊敬述為涼
 州都督,遣副將盧公利、判官元澄,出兵邀擊之。暾欲谷曰:「敬述若守城自固,即與連和;若出兵相當,即須決戰。我今乘勝,必有功矣!」公利等兵至刪丹,遇賊,元澄令兵士揎臂持滿,仍急結其袖,會風雪凍烈,盡墜弓矢。由是官軍大敗,元澄脫身而走。敬述坐削除官爵,白衣檢校涼州事。小殺由是大振,盡有默啜之眾。俄又遣使請和,乞與玄宗為子,上許之。仍請尚公主,上但厚賜而遣之。



 十三年,玄宗將東巡,中書令張說謀欲加兵以備突厥。
 兵部郎中裴光庭曰:「封禪者,告成之事,忽此徵發,豈非名實相乖?」說曰:「突厥比雖請和,獸心難測。且小殺者仁而愛人,眾為之用;闕特勒驍武善戰,所向無前;暾欲谷深沉有謀,老而益智,李靖、徐勣之流也。三虜協心,動無遺策,知我舉國東巡,萬一窺邊,何以御之?」光庭請遣使徵其大臣扈從,則突厥不敢不從,又亦難為舉動。說然其言,乃遣中書直省袁振攝鴻臚卿,往突厥以告其意。小殺與其妻及闕特勒、暾欲谷等環坐帳中設宴,謂振
 曰:「吐蕃狗種,唐國與之為婚;奚及契丹,舊是突厥之奴,亦尚唐家公主;突厥前後請結和親,獨不蒙許,何也?」袁振曰:「可汗既與皇帝為子,父子豈合為婚姻?」小殺等曰:「兩蕃亦蒙賜姓,猶得尚主,但依此例,有何不可?且聞入蕃公主,皆非天子之女,今之所求,豈問真假,頻請不得,實亦羞見諸蕃。」振許為奏請。小殺乃遣其大臣阿史德頡利發入朝貢獻,因扈從東巡。



 玄宗發都,至嘉會頓,引頡利發及諸蕃酋長入仗,仍與之弓箭。時有兔起於御
 馬之前,上引弓傍射,一發獲之。頡利發便下嘛捧兔蹈舞曰:「聖人神武超絕,若天上則不知,人間無也。」上因令問饑否。對曰:「仰觀聖武如此。十日不食,猶為飽也!」自是常令突厥入仗馳射。起居舍人呂向上疏曰:



 臣聞鴟梟不鳴,未為瑞鳥,猛虎雖伏,豈齊仁獸,是由醜性毒行,久務常積故也。今夫突厥者,正與此類,安忍殘賊,莫顧君親!陛下持武義臨之,修文德來之,既懾威靈,又沐聲教;以力以勢,不得不庭。故稽顙稱臣,奔命遣使。陛下乃能
 收其傾效,雜以從官,赴封禪之禮,參玉帛之會,此德業自盛,固不可名焉。因復詔許侍游,召入禁仗。仰英姿之四照,送神藝之百發,恩意俱極,誠無得逾焉。乃更賜以馳逐,使操弓矢,競飛鏃於前,同獲獸之樂,是屑略太過,未敢取也。雖聖胸豁達,與物無猜,而愚心徘徊,與時加慄。儻此等各懷犬吠,交肆盜憎,荊卿詭動,何羅竊至,暫逼嚴蹕,稍冒清塵,縱即殪玄方,墟幽土,單于為醢,穹廬為污,何塞過責?特願陛下勿復親近,使知分限。待不失
 常,歸於得所,以謂回兩曜之鑒,祛九宇之憂,孰不幸甚!



 上納其言,遂令諸蕃先發。東封回,上為頡利發設宴,厚賜而遣之,竟不許其和親。



 十五年,小殺使其大臣梅錄啜來朝,獻名馬三十匹。時吐蕃與小殺書,將計議同時入寇,小殺並獻其書。上嘉其誠,引梅錄啜宴於紫宸殿,厚加賞賚,仍許於朔方軍西受降城為互市之所,每年齎縑帛數十萬匹就邊以遺之。



 二十年,闕特勒死,詔金吾將軍張去逸、都官郎中呂向,齎璽書入蕃吊祭,並為
 立碑。上自為碑文,仍立祠廟,刻石為像,四壁畫其戰陣之狀。



 二十年,小殺為其大臣梅錄啜所毒,藥發,未死,先討斬梅錄啜,盡滅其黨。既卒,國人立其子為伊然可汗。詔宗正卿李佺往申吊祭,並冊立伊然,為立碑廟。仍令史官起居舍人李融為其碑文。無幾,伊然病卒,又立其弟為登利可汗。



 登利者,猶華言果報也。登利年幼,其母即暾欲谷之女,與其小臣飲斯達干奸通,干預國政,不為蕃人所伏。登
 利從叔父二人分掌兵馬,在東者號為左殺,在西者號為右殺,其精銳皆分在兩殺之下。二十八年,上遣右金吾將軍李質齎璽書,又冊立登利為可汗。俄而登利與其母誘斬西殺,盡並其眾。而左殺懼禍及己,勒兵攻登利,殺之。自立,號烏蘇米施可汗。左殺又不為國人所附,拔悉密部落起兵擊之。左殺大敗,脫身遁走,國中大亂。西殺妻子及默啜之孫勃德支特勒、毗伽可汗女大洛公主、伊然可汗小妻餘塞匐、登利可汗女餘燭公主及
 阿布思頡利發等,並率其部眾相次來降。天寶元年八月,降虜至京師,上令先謁太廟,仍於殿庭引見,御華萼樓以宴之,上賦詩以紀其事。



\end{pinyinscope}