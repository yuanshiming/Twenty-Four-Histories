\article{卷十一 本紀第十一 代宗}

\begin{pinyinscope}

 代宗睿文孝武皇帝諱豫,肅宗長子,母曰章敬皇太后吳氏。以開元十四年十二月十三日生於東都上陽宮。初名俶,年十五封廣平王。玄宗諸孫百餘,上為嫡皇孫。宇量弘深,寬而能斷。喜懼不形於色。仁孝溫恭,動必由禮。幼而好學,尤專《禮》、《易》,玄宗鐘愛之。



 祿山之亂,京城陷賊,從肅宗搜兵靈武,以上為天下兵馬元帥。時朝廷草創,兵募寡弱,上推心示信,招懷流散,比至彭原,兵眾數萬。及肅宗回幸鳳翔,時房琯、郭子儀繼戰不利,賊鋒方銳,屢來寇襲。上選求勇干,頻挫其鋒,聖慮遑寧,士心大振。及師進討,百官辭送,步出闕門,方始乘馬。回紇葉護王子率兵入助,勇冠諸蕃,上接以優恩,結為兄弟,故香
 積之戰,賊徒大敗,遂委西京而遁。雖子儀、嗣業之奮命,由上恩信結於士心,故人思自效。既收京城,令行禁止,民庶按堵,秋毫不犯,耆老歡迎,對之歔欷。聞賊殘眾猶保陜郊,即日長驅,東趨虢洛。新店之役,一戰大捷,慶緒之黨,十殲七八。數旬之間,河南底定,兩都恢復,二聖回鑾,統率之功,推而不受。肅宗還京,大赦,改封楚王。



 乾元元年三月,改封成王。四月庚寅,立為皇太子,改名豫。上元末年,兩宮不豫,太子往來侍疾,躬嘗藥膳,衣不角帶
 者久之,及承監國命,流涕從之。



 寶應元年四月,肅宗大漸,所幸張皇后無子,後懼上功高難制,陰引越王系於宮中,將圖廢立。乙丑,皇后矯詔召太子。中官李輔國、程元振素知之,乃勒兵於凌霄門,俟太子至,即衛從太子入飛龍廄以俟其變。是夕,勒兵於三殿,收捕越王系及內官硃光輝、馬英俊等禁錮之,幽皇后於別殿。丁犯,肅宗崩,元振等始迎上於九仙門,見群臣,行監國之禮。己巳,即皇帝位於柩前。甲戌,詔:「國之大事,戎馬為先,朝
 有舊章,親賢是屬。故求諸必當,用制於中權;存乎至公,豈慚於內舉。特進、奉節郡王適可天下兵馬元帥。」乙亥,以兵部尚書、判元帥行軍、閑廄等使李輔國進號尚父,飛龍閑廄副使程元振為右監門將軍。流宦官硃光輝、啖庭瑤、陳仙甫等於黔中。



 五月己卯朔,以李輔國為司空兼中書令,餘如故。辛卯,制曰:「三年之喪,天下達禮,茍或變革,何以教人?朕遘此閔兇,攀號罔極,公卿固請,俾聽朝務,斬焉縗絰,痛貫心靈,豈可便議公除,遽移諒陰。
 昨見所司儀注,今月十三日大祥,十五日從吉。仰憑遺制,又欲抑予,竅惟哀思,深謂未可。其百僚並以此釋服,朕將繼武丁之道,《素冠》之詩,恭默再周,不忍權奪。凡庶在位,宜悉哀懷。」宰臣苗晉卿等三上表請依遺制,方聽政。丙戌,嗣魯王宇改封鄒王,奉節郡王適進封魯王,李光弼進封臨淮王。貶禮部尚書蕭華為陜州司馬。改行乾元錢。重棱小錢一當二,重棱大錢一當三。丙申,以戶部侍郎元載同中書門下平章事,充度支轉運使。改乾
 元大小錢一當一。丁酉,御丹鳳樓,大赦。子儀、光弼、李光進諸道節度使並加實封。四月十七日立功人並號「寶應功臣」。內外文武官三品已上進爵,四品已下加階。諸州防禦使並停。內外官三考一轉。益昌郡王邈進封鄭王,延慶郡王迥進封韓王。故庶人皇后王氏、故誣人太子瑛、鄂王瑤、光王琚並宜復封號。棣王琰、永王璘並與昭雪。建昌王追封齊王,崇恩王追封衛王,靈昌王追封鄆王。壬寅,以來瑱復為襄州刺史、山南東道節度使。
 六月己酉朔,百僚臨於西宮,上不視朝。自是每朔望皆如之,迄於山陵。凡人臣有事辭見,先臨西宮,然後詣朝。改豫州為蔡州,避上名也。待中苗晉卿以老疾,請三日一入中書,從之。己未,罷尚父李輔國判元帥行軍及兵部尚書、閑廄等使。輔國請遜位。辛酉,以輔國為博陸王,罷中書令,許朝朔望。壬申,以通州刺史劉晏為戶部侍郎、兼御史大夫、京兆尹,充度支轉運鹽鐵諸道鑄錢等使。秋七月己卯朔。辛巳,觀軍容使魚朝恩封馮翊郡開
 國公,宦官程元振為鎮軍大將軍、保定郡開國公。乙酉,襄州剌史裴長流費州,賜死於藍田驛。庚寅,詔不許匭使閱投匭人文狀,賜道州司馬敬羽自盡。來瑱自襄州來朝。郭子儀自河中來朝。八月己酉朔。自七月不雨,至此月癸丑方雨。庚午夜,西北有赤光亙天,貫紫微,漸移東北,彌漫半天。貶太子少傅李遵為袁州刺史。臺州賊袁晁陷臺州,連陷浙東州縣。九月丁丑朔,魯王適改封雍王。以山南東道節度使來瑱為兵部尚書,同中書
 門下平章事,節度如故。程元振進封邠國公。丙申,右僕射、山陵使裴冕貶施州刺史。戊戌,回紇登里可汗率眾來助國討逆,令御史大夫尚衡宣慰之。甲午,太州至陜州二百餘里黃河清,澄澈見底。甲午,秘書監韓穎、中書舍人劉烜配流嶺表,尋賜死,坐狎暱李輔國也。



 冬十月辛酉,詔天下兵馬元帥雍王統河東、朔方及諸道行營、回紇等兵十餘萬討史朝義,會軍於陜州。加朔方行營節度使、大寧郡王僕固懷恩同中書門下平章事。丁卯
 夜,盜殺李輔國於其第,竊首而去。戊辰,元帥雍王率諸軍進發,留郭英乂、魚朝恩鎮陜州。壬申,王師次洛陽北郊。甲戌,戰於橫水,賊大敗,俘斬六萬計。史朝義奔冀州。乙亥,雍王奏收東京、河陽、汴、鄭、滑、相、魏等州。乙酉,陜西節度使郭英乂權知東京留守。丁酉,偽恆州節度使張忠志以趙、定、深、恆、易五州歸順,以忠志檢校禮部尚書、恆州刺史,充成德軍節度使,賜姓名曰李寶臣。於是河北州郡悉平。賊範陽尹李懷仙斬史朝義首來獻,請降。
 十二月庚戌,太子太師,邠國公韋見素薨。辛未,僕固懷恩為尚書左僕射、兼中書令,靈州大都督府長史、河北副元帥。邛州新置鎮南軍。是歲,江東大疫,死者過半。吐蕃陷我臨、洮、成、渭等州。



 二年春正月丁亥朔。甲午,戶部尚書、兼御史大夫、都統淮南節度觀察等使、越國公李峘卒。國子祭酒、兼御史大夫、京兆尹劉晏為吏部尚書、同中書門下平章事,度支諸使如故。壬寅,制開府儀同三司、行兵部尚書、同中
 書門下平章事、充山南東道節度觀察處置等使、上柱國、潁國公來瑱削在身官爵,長流播州,尋賜死於路。閏月戊申,以史朝義下降將李寶臣為檢校禮部尚書、兼御史大夫、恆州刺史、清河郡王,充成德軍節度使;薛嵩為檢校刑部尚書、相州刺史、相衛等州節度使;李懷仙檢校兵部尚書、兼侍中、武威郡王、幽州節度使;田承嗣檢校戶部尚書、魏州刺史、雁門郡王、魏博等州都防禦使。



 二月甲午,回紇登里可汗辭歸蕃。三月甲辰朔,襄州右
 兵馬使梁崇義殺大將李昭,據城自固,仍授崇義襄州刺史、山南東道節度使。丁未,袁傪破袁晁之眾於浙東。玄宗、肅宗歸祔山陵。自三月一日廢朝,至於晦日,百僚素服詣延英門通名起居。



 四月戊寅朔,太州依舊為華州,太陰縣為華陰縣。庚辰,河南副元帥李光弼奏生擒袁晁,浙東州縣盡平。辛巳,屬臣請上尊號。五月癸卯朔。丙寅,尚書省試制舉人,命左右丞、侍郎對試,賜食如舊儀。太常卿杜鴻漸奏:「婚葬合給鹵簿,望於國立大功及
 二等已上親則給,餘不在給限。」從之。六月癸酉朔。癸未,以陳鄭澤潞節度使李抱玉檢校司空,封武威郡王;河中節度使王昂檢校刑部尚書,封云阜國公;同華節度使李讓檢校工部尚書。同日入省,宰相送上。甲申,以前淮西節度使王仲升為右羽林大將軍,兼御史大夫。六軍將軍兼大夫,自仲升始也。甲午,觀軍容使魚朝恩自陜州入朝。上御達禮門,命公卿百僚觀兵馬。同華節度使李懷讓自殺,為程元振所構。



 秋七月壬寅朔。戊申,群
 臣上尊號曰寶應元聖文武皇帝,御含元殿受冊。壬子,御宣政殿宣制,改元曰廣德,大赦天下,常赦不原者咸赦除之。安祿山、史思明親族應在諸道,一切原免不問。民戶三丁免一丁庸,租稅依舊每畝二升。男子二十成丁,五十入老。元帥雍王兼尚書令,河北副元帥僕固懷恩加太保,回紇登里可汗進徽號。功臣皆賜鐵券,藏名太廟,畫像凌煙閣。刺史、縣令自今後改轉,刺史以三年為限,縣令四年為限,員外及攝試,不得厘務。丁巳,僕固
 瑒兼御史大夫,充朔方行營節度。是月,吐蕃大寇河、隴,陷我秦成、渭三州,入大震關,陷蘭、廓、河、鄯、洮、岷等州,盜有隴右之地。八月,以荊南節度使李峴為宗正卿。九月壬戌朔,僕固懷恩拒命於汾州,遣宰臣裴遵慶往宣撫之。已丑,吐蕃寇涇州,刺史高暉以城降,因為吐蕃鄉導。



 冬十月庚午朔。辛未,高暉引吐蕃犯京畿,寇奉天、武功、盩厔等縣。蕃軍自司竹園渡渭,循南山而東。丙子,駕幸陜州,上出苑門,射生將王獻忠率四百騎叛,脅豐王已
 下十王歸京。從官多由南山諸穀赴行在。郭子儀收合散卒,屯於商州。丁丑,次華州,官吏藏竄,無復儲擬。會魚朝恩領神策軍自陜來迎駕,乃幸朝恩軍。戊寅,吐蕃入京師,立廣武王承宏為帝,仍逼前翰林學士於可封為制封拜。辛巳,車駕至陜州。子儀在商州會六軍使張知節,烏崇福、長孫全緒等率兵繼至,軍威遂振。舊將王甫誘聚京城惡少,齊擊街鼓於硃雀街,蕃軍震懾,狼狽奔潰。庚寅,子儀收京城。壬辰,以宰臣元載判天下元帥行
 軍司馬,京兆尹、兼吏部侍郎嚴武為黃門侍郎,朗州刺史第五琦為京兆尹、兼御史大夫。癸巳,以郭子儀為京留守。高暉聞吐蕃潰,以三百騎東奔至潼關,為關守李伯越所殺。十一月辛丑朔,太常博士柳伉上疏,以蕃寇犯京師,罪由程元振,請斬之以謝天下。上甚嘉納,以元振有保護之功,削在身官爵,放歸田里。



 十二月甲辰,宦官市舶使呂太一逐廣南節度使張休,縱下大掠廣州。丁亥,車駕發陜郡還京。辛卯,鄂州大風,火發江中,焚船
 三千艘,焚居人盧舍二千家。甲午,上至自陜州。乙未,以侍中苗晉卿為太保,黃門侍郎、同平章事裴遵慶為太子少傅,並罷知政事;宗正卿、梁國公李峴為黃門侍郎、同中書門下平章事。丙申,放廣武王承宏於華州,一切不問。丁酉,朔方行營節度使僕固瑒為帳下梟首來獻。懷恩聞瑒死,燒營遁入吐蕃。朝臣稱賀,上不悅,曰:「朕之涼德,信不及人,致勛臣顛覆,用增愧恧,何至賀焉!」程元振自三原縣衣婦人服入京城,京兆府擒之以聞,乃下
 御史臺鞫問。吐蕃陷松州、維州、雲山城、籠城。



 二年春正月己亥朔。壬寅,御史臺以程元振獄狀聞,配流溱州。既行,追念舊勛,特矜遐裔,令於江陵府安置。甲辰,復置京畿觀察使,以御史中丞領之。癸卯,尚書右丞顏真卿為刑部尚書、兼御史大夫,充朔方宣慰使。癸亥,吏部尚書、同平章事、度支轉運使劉晏為太子賓客,黃門侍郎、同平章事李峴為太子詹事,並罷知政事。以前右散騎常侍王縉為黃門侍郎,太常卿杜鴻漸為兵部
 侍郎,並同中書門下平章事。罷度支使,以戶部侍郎第五琦專判度支及諸道鹽鐵、轉運鑄錢等使。甲子,元帥,尚書令雍王三上章讓皇太子。第五琦奏諸道置常平倉使司,量置本錢和糴,許之。丁卯,司徒、兼中書令郭子儀充河東副元帥、河中等處觀察,兼雲州大都督、單于鎮北大都護。



 二月己巳朔,冊天下兵馬元帥,尚書令,雍王適為皇太子。癸酉,上親薦獻太清宮、太廟。乙亥,祀昊天上帝於圓丘,即日還宮。戊寅,以灃州刺史裴冕為左
 僕射兼御史大夫,充東都、河南、江南、淮南轉運使。乙未,第五琦開決汴河。五月丁酉朔。戊午,敕中書、門下兩省加置散騎常侍四員,官為正三品。庚申,罷歲貢孝悌力田、童子等科。甲子,禁鈿作珠翠等,委所司切加捉搦。癸未,制:「太保、兼中書令、靈州大都督府長史、單于鎮北副大都護、充朔方節度、關內度支營田鹽池押諸蕃部落副大使、知節度事、六城水運使、河北副元帥、上柱國、大寧郡王僕固懷恩,先任靈州大都督府長史、單于鎮北
 副元帥、朔方節度使宜並停,其太保、兼尚書令、大寧郡王如故。七月己酉,河南副元帥、太尉、兼侍中、臨淮王李光弼薨於徐州,廢朝三日。判度支第五琦兼京兆尹、御史大夫。八月丁卯,宰臣王縉為侍中,持節都統河南、淮西、淮南、山南東道節度行營事,進封太原郡公。固讓侍中,從之。宰相杜鴻漸判門下省事。癸巳,王縉兼領東京留守。



 九月乙未朔。丙申,詔征河中兵討吐蕃,將發,是夜軍眾喧噪,劫節度使崔寓家財及民家財產殆盡,皆
 重裝而行,吏不能禁。自七月大雨未止,京城米斗值一千文。蝗食田。丙午,河東節度使辛云京檢校尚書右僕射、同中書門下平章事、太原尹、北京留守。己酉,江南西道觀察,洪州刺史張鎬卒。辛亥,河東副元帥、中書令、汾陽郡王郭子儀加太尉,充北道邠寧、涇原、河西已東通和吐蕃及朔方招撫使;陳鄭、澤潞節度使李抱玉進位司徒,充南道通和吐蕃使、鳳翔秦隴臨洮已東觀察使。子儀三表懇讓太尉,許之。己未,劍南節度嚴武攻拔吐蕃
 當狗城,破蕃軍七萬。尚書左丞楊綰知東京選,禮部侍郎賈至知東都舉。兩都分舉選,自此始也。辛酉,以太子詹事李峴為吏部尚書、兼御史大夫,知江南東西及福建道選,並觀農宣慰使。仍命洪州刺史李勉副知選事。是秋,蝗食田殆盡,關輔尤甚。米斗千錢。



 冬十月丙寅,僕固懷恩引吐蕃二萬寇邠州,節度使白孝德閉城拒守。丁卯,寇奉天,京師戒嚴。先鋒郭晞斬賊營於邠州西,俘斬數百計。子儀屯涇陽,蕃軍挑戰,子儀不出。甲申,河
 南尹蘇震卒。劍南嚴武奏收吐蕃鹽川城。十一月乙未,懷恩與蕃軍自潰,京師解嚴。丁未,子儀自涇陽入覲,詔宰臣百僚迎之於開遠門,上御安福寺待之。十二月乙丑,加子儀關內、河中副元帥兼尚書令,吏部侍郎暢璀為左散騎常侍、河中尹。子儀三表讓尚書令,詞情懇切,優詔從之。丁卯夜,星流如雨。戊辰,子儀於都省領副元帥事,宰臣百僚送,仍令射生五百戎服自光範門送至省門。右僕射郭英乂以樂迎之。是日便赴奉天。是歲,
 戶部計帳,管戶二百九十三萬三千一百二十五,口一千六百九十二萬三百八十六。



 永泰元年正月癸巳朔,制曰:



 葉五紀者,建號以體元;授四時者,布和而須氣。天心可見,人欲是從,爰立大中之道,式受惟新之命。朕嗣膺下武,獲主萬方,顧以薄德,乘茲艱運,戎麾問罪,今已十年。飲至策勛,惟兇渠之授首;勞師黷武,豈人主之用心。軍役屢興,干戈未戢,茫茫士庶,斃於鋒鏑。皇穹以朕為子,蒼生以朕為父,至德不
 能被物,精誠不能動天。俾我生靈,淪於溝壑,非朕之咎,孰之過歟?朕所以馭朽懸旌,坐而待曙,勞懷罪己之念,延想安人之策。亦惟群公卿士,百闢庶僚,咸聽朕命,協宣乃力,履清白之道,還淳素之風。率是黎元,歸於仁壽,君臣一德。何以尚茲。乃者刑政不修,惠化未洽,既盡財力,良多抵犯,靜惟哀矜,實軫於懷。今將大振綱維,益明懲勸,肇舉改元之典,弘敷在宥之澤,可大赦天下,改廣德三年為永泰元年。



 是日雪盈尺。戊申,澤潞李抱玉兼
 鳳翔隴右節度使,兼南道通和吐蕃、鳳翔「秦隴,臨洮已東」,觀察處置等使。仍命四鎮行營節度使馬璘為副和吐蕃使。癸丑,罷岐州之鳳翔縣,並入天興縣。乙卯,左散騎常侍高適卒。戊午,劍南節度使嚴武加檢校吏部尚書,山南節度使張獻誠加檢校工部尚書。以前袁州刺史李遵為太子少保,聽朝朔望。



 二月甲子夜。雷霆震擊。丁丑,內出宮女千人,品官六百人守洛陽宮。戊寅,黨項羌寇富平;焚定陵寢殿。
 庚辰,儀王璲薨。諸陵署復隸太常寺。戊子,河西黨項永、定等十二州部落內屬,請置宜、芳等十五州,許之。三月壬辰朔,詔左僕射裴冕、右僕射郭英乂、太子少傅裴遵慶、檢校太子少保白志貞、太子詹事臧希讓、左散騎常侍暢璀、檢校刑部尚書王昂高升、檢校工部尚書崔渙、吏部侍郎李季卿王延昌、禮部侍郎賈至、涇王傅吳令瑤等十三人,並集賢院待詔。上以另臣罷節制者,京師無職事,仍合於禁門書院,間以文儒公卿,寵之也。仍特
 給飧本錢三千貫。庚子夜,降霜,木有冰。歲饑,米斗千錢,諸穀皆貴。丙午,鳳翔李抱玉讓司徒,從之,授左僕射、同平章事。庚戌,吐蕃請和。詔宰臣元載、杜鴻漸與蕃使同盟於興唐寺。辛亥,大風拔木。是春大旱,京師米貴,斛至萬錢。



 夏四月己巳,乃雨。戊子,太保致仕苗晉卿薨。庚寅,劍南節度使、檢校吏部尚書嚴武卒。五月癸丑,以尚書右僕射、定襄郡王郭英乂為成都尹、御史大夫,充劍南節度使。是月麥稔。判度支第五琦奏請十畝稅一畝,效
 古什一而征,從之。



 六月癸亥,吏部尚書李峴南選回,至江陵,貶衢州刺史。自春無雷,至此月甲申,大風而雷。代州置代北軍,平州置柳城,析通州石鼓縣置巴渠縣。秋七月辛卯朔,淄青節度使侯希逸為副將李懷玉所逐。制以鄭王邈為平盧、淄青節度大使,令懷玉權知留後事。以久旱,遣近臣分錄京城諸獄系囚。甲午,升平公主出降駙馬都尉郭曖。庚子,雨。時久旱,京師米斗一千四百,他穀食稱是。



 八月乙亥,河南道副元帥、涇原節度使
 馬璘封扶風郡王。九月辛卯,太白經天。丁酉,僕固懷恩死於靈州之鳴沙縣。時懷恩誘吐蕃數十萬寇邠州,客將尚品息贊磨、尚悉東贊等寇奉天、醴泉,黨項羌、渾、奴剌寇同州及奉天,逼鳳翔府、盩厔縣,京師戒嚴。時以星變,羌虜入寇,內出《仁王佛經》兩輿付資聖、西明二佛寺,置百尺高座講之。及奴虜寇逼京畿,方罷講。己酉,郭子儀自河中至,進屯涇陽,李忠臣屯東渭橋,李光進屯雲陽,馬璘、郝玉屯便橋,駱奉仙、李伯越屯盩厔,李抱玉屯
 鳳翔,周智光屯同州,杜冕屯坊州。上親率六軍屯苑內。庚戌,下詔親征。內官魚朝恩上言,請括私馬,京城男子悉皁衣團結,塞京城二門之一。士庶大駭,有逾垣鑿竇出城者,吏不能禁。自丙午至甲寅大雨,平地水流。丁巳,吐蕃大掠京畿男婦數萬計,焚廬舍而去。同華節度周智光以兵追擊於澄城,破賊萬計。



 冬十月己未,復講《仁王經》於資聖寺。吐蕃至邠州,與回紇相遇,復合從入寇。辛酉,逼奉天。癸亥,黨項攻同州,焚州民廬舍。丁丑,郭子
 儀說諭回紇,令與吐蕃疑貳。庚辰,子儀先鋒將白元光合回紇軍擊吐蕃之眾於靈臺縣之西原,斬首五萬級,俘獲人畜凡三百里不絕。辛巳,京師解嚴。壬午,僕固懷恩大將僕固名臣以千騎來降。詔稅百官錢;市絹十萬以賞回紇。乙酉,回紇首領胡祿都督來朝。癸卯,朔方將李回方奏收靈武郡。丁亥,分宣饒、歙戶口於秋浦縣置池州,分信州弋陽置貴溪縣。閏十月辛卯,以京兆少尹黎幹為京兆尹。丙午,封朔方大將孫守亮等九人為異
 姓王,李國臣等十三人為同姓王。丁未,百僚上表,以軍興急於糧,請納職田以助費,從之。戊申,進封渭北節度使李光進為武威郡王。以刑部侍郎路嗣恭檢校工部尚書、兼御史大夫、靈州大都督府長史,充關內副元帥,兼知朔方節度等使。劍南節度使郭英乂為其檢校西山兵馬使崔旰所殺,邛州柏茂林、瀘州楊子琳、劍南李昌巙皆起兵討旰,蜀中亂。十一月,宰臣河南都統王縉請減諸道軍資錢四十萬貫修洛陽宮,從之。十二月
 己酉,敕:「如聞諸州承本道節度、觀察使牒,科役百姓,致戶口凋弊,此後委轉運使察訪以聞。」



 二年春正月丁巳朔,大雪平地二尺。壬申,減子孫襲實封者半租,永為常式。乙酉,制:



 治道同歸,師氏為上,化人成俗,必務於學。俊造之士,皆從此途,國之貴游,罔不受業。修文行忠信之教,崇祗庸孝友之德,盡其師道,乃謂成人。然後揚於王庭,敷以政事,徵之以理,任之以官,置於周行,莫匪邦彥,樂得賢也,其在茲乎!朕志承理體,尤
 重儒術,先王設教,敢不虔行。頃以戎狄多虞;急於經略,太學空設,諸生蓋寡。弦誦之地,寂寥無聲,函丈之間,殆將不掃,上庠及此,甚用閔焉。今宇縣乂寧,文武並備,方投戈而講藝,俾釋菜以行禮。使四科咸進,六藝復興,神人以和,風化浸美,日用此道,將無間然。其諸道節度、觀察、都防禦等使,朕之腹心,久鎮方面,眷其子弟,為奉義方,修德立知,是資藝業。恐干戈之後,學校尚微,僻居遠方,無所咨稟,負經來學,宜集京師。其宰相朝官、六軍諸
 將子弟,欲得習學,可並補國子學生。其中身雖有官,欲附學讀書者亦聽,其學官委中書門下選行業堪為師範者充。其學生員數,所習經業,供承糧料,增修學館,委本司條奏以聞。



 丙戌,以戶部尚書劉晏充東都京畿、河南、淮南、江南東西道、湖南、荊南、山南東道轉運、常平、鑄錢、鹽鐵等使,以戶部侍郎第五琦充京畿、關內、河東、劍南西轉運、常平、鑄錢、鹽鐵等使。至是天下財賦,始分理焉。



 二月丁亥朔,釋奠於國學,賜宰臣百官飧錢五百
 貫,於國學食。壬辰,鎮南都護依舊為安南都護府。乙未,貶刑部尚書顏真卿為峽州員外別駕,以不附元載,載陷之於罪也。壬子,命黃門侍郎、同平章事杜鴻漸兼成都尹,持節充山南西道、劍南東川等道副元帥,仍充劍南西川節度使,以平郭英乂之亂也。以四鎮行營節度使馬璘兼邠州刺史。癸丑,以山南西道節度使、梁州刺史張獻誠兼充劍南東川節度觀察使,邛州刺史柏茂林充邛南防禦使,劍南西山兵馬使崔旰為茂州刺史、
 充劍南西山防禦使,從杜鴻漸請也。三月辛未,張獻誠與崔旰戰於梓州,為旰所敗,僅以身免。



 夏四月辛亥,詔尚書省郎中授中州刺史,員外郎授下州刺史,為定制。五月丙辰,稅青苗地錢使、殿中侍御韋光裔諸道稅地回。是歲得錢四百九十萬貫。自乾元已來,天下用兵,百官俸錢折,乃議於天下地畝青苗上量配稅錢,命御史府差使徵之,以充百官俸料,每年據數均給之,歲以為常式。六月戊戌,以淮南節度使崔圓檢校尚書右僕射。
 自春旱,此月庚子始雨。丁未,日重輪。其夜,月重輪。



 秋七月辛酉,檢校兵部尚書、衢州刺史李峴卒。自五月大雨,洛水泛溢,漂溺居人廬舍二十坊。河南諸州水。加荊南節度使衛伯玉檢校工部尚書。癸未,太廟二室芝草生。八月丁亥,國子監釋奠復用牲牢。上元二年,詔諸祠獻熟,至是魚朝恩請復舊制。壬寅,以茂州刺史崔旰為成都尹、兼御史大夫、劍南西川節度行軍司馬,邛南防禦使、邛州刺史柏茂林為邛南節度使,從杜鴻漸所請也。
 癸卯,太子少保裴遵慶為吏部尚書,吏部尚書崔寓為太子少傅。甲辰,以開府儀同三司、右監衛大將軍、觀軍容宣慰處置使、神策軍兵馬使、上柱國、馮翊郡開國公魚朝恩加內侍監、判國子監事,充鴻臚禮賓等使,進封鄭國公。辛亥,以檢校禮部尚書裴士淹充禮儀。九月庚申,京兆尹黎幹以京城薪炭不給。奏開漕渠,自南山谷口入京城,至薦福寺東街,北抵景風、延喜門入苑,闊八尺,深一丈。渠成,是日上幸安福門以觀之。丙子,宣州
 刺史李佚坐贓二十四萬貫,集眾杖死,籍沒其家。



 冬十月癸未朔。己丑,宗正卿吳王祗奏上《皇室永泰新論》二十卷,太常博士柳芳撰。和蕃使楊漳與蕃使論位藏等來朝。丙申,令宰臣宴論位藏於中書省。



 十一月甲寅,乾陵令於陵署得赤兔以獻。丙辰,詔:



 古者量其國用,而立稅典,必於經費,則之重輕。公田之籍,可謂通制;履畝而稅,斯誠弊法。所期折中,以便於時。億兆不康,君孰與足?故愛人之體,先以博施;富國之源,必均節用。朕自臨宸
 極,比屬艱難,嘗欲闡淳樸之風,守沖儉之道,每念黎庶,思致和平。而邊事猶殷,戎車屢駕,軍與取給,皆出邦畿。九伐之師,尚勤王略;千金之費,重困吾人。乃者遵冉有之言,守周公之制,什而稅一,務於行古。今則編戶流亡,而墾田減稅,計量入之數,甚倍征之法。納隍之懼,當寧軫懷。慮失三農,憂深萬姓,務從省約,稍冀蠲除。用申勤恤之懷,以救惸嫠之弊。京兆府今年合徵八十二萬五千石數內,宜減放一十七萬五千石。青苗地頭錢宜三
 分取一。在京諸司官員久不請俸,頗聞艱辛。其諸州府縣官,及折沖府官職田,據苗子多少,三分取一,隨處糶貨,市輕貨以送上都,納青苗錢厙,以助均給百官。



 甲子,日長至,上御含元殿,下制大赦天下,改永泰二年為大歷元年。十二月己亥,彗起匏瓜,其長尺餘,犯宦者星。癸卯,同華節度使周智光專殺陜州監軍張志斌、前虢州刺史龐充,據華州謀叛。是冬無雪。



 、二年春正月壬子朔。丁巳,密詔關內、河東副元帥郭子
 儀治兵討周智光。壬戌,貶智光為灃州剌史。甲子,以兵部侍郎張仲光為華州刺史、潼關防禦使,大理卿敬括為同州刺史、長春宮等使。是日,周智光帳下將斬智光並子元耀、元乾三首,傳之以獻。己巳,詔潼關置兵三千。癸酉,詔:



 天文著象,職在於疇人;讖緯不經,蠹深於疑眾。蓋有國之禁,非私家所藏。雖裨灶明徽,子產尚推之人事;王彤必驗,景略猶置於典刑。況動皆訛謬,率是矯誣者乎!故聖人以經籍之義,資理化之本,側言曲學,實紊
 大猷,去左道之亂政,俾彞倫而攸敘。自四方多故,一紀於茲,或有妄庸,輒陳休咎,假造符命,私習星歷。共肆穹鄉之辯,相傳委巷之談,作偽多端,順非僥澤。熒惑州縣,詿誤閭閻,壞紀挾邪,莫逾於此。其玄象器局、天文圖書、《七曜歷》、《太一雷公式》等,私家不合輒有。今後天下諸州府,切宜禁斷。本處分明榜示,嚴加捉搦,先藏蓄此等書者,敕到十日內送官,本處長吏帶領集眾焚毀。限外隱藏為人所告者,先決一百,留禁奏聞。所告人有官即與超資
 注擬,無官者給賞錢五百貫。兩京委御史臺處分。各州方面勛臣,洎百僚庶尹,罔不誠亮王室,簡於朕心,無近憸人,慎乃有位,端本靜末,其誡之哉!



 丁丑,升魏州為大都督府,戊寅,敕:「同、華兩州,頃因盜據,民力凋殘,宜給復二年,一切蠲免。」庚辰,禁王公、宗子、郡縣主之家,不得與軍將婚姻交好,委御史臺察訪彈奏。



 二月壬午,幸昆明池踏青。丙戌,封華州牙將姚懷為感義郡王,李延俊為承化郡王,以斬智光之功也。郭子儀自河中來朝。癸卯,
 宰臣元載王縉、左僕射裴冕、戶部侍郎第五琦、京兆尹黎幹各出錢三十萬,置宴於子儀之第。三月辛亥夜,大風。丁巳,河中府獻玄狐。汴宋節度使田神功來朝。戊辰,貶太子少保李遵永州司馬,坐贓也。甲戌,魚朝恩宴子儀、宰相、節度、度支使、京兆尹於私第。乙亥,子儀亦置宴於其第。戊寅,田神功宴於其第。時以子儀元臣,寇難漸平,蹈舞王化,乃置酒連宴。酒酣,皆起舞。公卿大臣列坐於席者百人。子儀、朝恩、神功一宴費至十萬貫。



 夏四月
 巳亥,以江面西道都團練觀察等使、洪州刺史李勉為京兆尹,刑部侍郎魏少游為洪州刺史、兼御史大夫、江西觀察團練等使。庚子,宰臣內侍魚朝恩與吐蕃同盟於興唐寺。丙午,加田神功檢校右僕射。癸酉,以工部侍郎徐浩為廣州刺史、嶺南節度觀察使。



 六月戊戌,山南、劍南副元帥杜鴻漸自蜀入朝。壬寅,荊南節度使衛伯玉封城陽郡王。癸卯,御史大夫王翊卒。秋七月戊申朔,以右散騎常侍於休烈為檢校工部尚書、知省事。時方
 面勛臣升八座者多非正員。朝命正員者以知省事為名。以中書舍人張延賞檢校河南尹。丙寅,以劍南西川節度行軍司馬崔旰為劍南西川節度觀察等使,遂州刺史杜濟為劍南東川節度觀察等使。以杭州刺史張伯儀為安南都護。癸酉,析道州延唐縣置大歷縣。甲戌酉時,有白氣竟天。八月庚辰,鳳翔節度使李抱玉來朝。壬午,月入氐。丙戌,渤海朝貢。辛卯,潭、衡水災。丙申,月犯畢。壬寅,太常卿、駙馬都尉姜慶初得罪,賜自盡。敕陵廟署復隸
 宗正寺。九月戊申朔,歲星守東井七日。甲寅,吐蕃寇靈州,進寇邠州。詔子儀率師三萬,自河中鎮涇陽,京師戒嚴。戊午夜,白霧起西北竟天。子儀移鎮奉天。乙丑晝,有大流星出於午,沒於亥。命左丞李涵宣慰河北。熒惑犯南斗。辛未,靺鞨使來朝。桂州山獠陷州城,刺史李良遁去。十月戊寅,靈州奏破吐蕃二萬,京師解嚴。甲申,減京官職田三分之一,給軍糧。乙酉,醴泉出於櫟陽,飲之愈疾。回紇、黨項使來朝。癸卯,上御紫宸殿。策試茂才異行、
 安貧樂道、孝悌力田、高蹈不仕等四科舉人。十一月庚申,改黃門侍郎依舊為門下侍郎。詔曰:「春秋以九命作上公。而謂之宰臣者,三公之職。漢制:中書令出納詔命,典司樞密;侍中上殿稱制,參議政事。魏、晉已還益重其任。職有關於公府,事不系於尚書,雖陳啟沃之謀,未專宰臣之稱,所以委遇斯大,品秩非崇。至於國朝,實執其政,當左輔右弼之寄,總代天理物之名,典領百僚,陶鎔景化。豈可具瞻之地,命數不加。固當進以等威,副其僉
 屬。其侍中、中書令宜升入正二品,門下、中書侍郎升入正三品。」壬戌夜,月暈南北河、東井,鎮星入輿鬼,久之方散。甲子,月去軒轅一尺。己丑,率百官京城士庶出錢以助軍。壬申,京師地震,自東北來,其聲如雷。十二月甲申,鳳翔李抱玉來朝。丁酉,太原節度使辛云京來朝。熒惑入壁壘。戊戌黑氣如塵,竟北方。是秋,河東、河南、淮南、浙江東西、福建等道五十五州奏水災。



 三年春正月丙午朔。辛亥,劍南西山置乾州,管招武、寧
 遠二縣。壬子夜,月掩畢。甲子,冊新羅國王金乾運母為太妃。甲戌,以工部侍郎蔣渙為尚書左丞,浙西團練觀察使、蘇州刺史韋元甫為尚書右丞。左丞李涵、右丞賈至並為兵部侍郎。乙亥,永和公主薨。二月己卯,以常州刺史李棲筠為蘇州刺史、兼御史中丞、浙西團練觀察使。壬午,邠寧節度使馬璘來朝。三月乙巳朔,日有蝕之。壬申,割恆州行唐縣置泜州,以靈壽、恆陽隸之。



 夏四月戊寅,以山南西道節度使、鄧國公張獻誠為檢校戶部尚
 書。以疾辭位也。右羽林將軍張獻恭為梁州刺史、兼御史中丞,充山南西道節度觀察使。兄獻誠所薦也。壬寅,滑亳節度使令狐彰加檢校工部尚書。劍南西川節度使、兼御史大夫崔旰來朝。



 五月戊申,加崔旰檢校右散騎常待。乙卯,追謚故齊王倓為承天皇帝,興信公主亡女張氏為恭順皇后,祔葬。辛酉,改桂州臨源縣為全義縣。癸酉,以左散騎常侍崔昭為京兆尹。是日地震。戊辰,以劍南西川節度使崔旰檢校工部尚書,改名寧。寧為
 柏茂林、楊子琳所攻,寧既入朝,子琳乘虛襲據成都府。朝廷憂之,即日詔寧還成都。庚午,以邛州剌史鮮於叔明為梓州刺史,充劍南東川節度使。



 六月戊子,承天皇帝祔奉天皇帝廟,同殿異室,庚寅,太子少師王璵卒。壬辰,幽州節度使、檢校侍中、幽州大都督府長史李懷仙為麾下兵馬使硃希彩所殺。庚子,淮南節度使檢校尚書左僕射、知省事、揚州大都督府長史、趙國公崔圓卒。閏月己酉,郭子儀加司徒。庚申,宰臣充河南副元帥王
 縉兼幽州節度使。以尚書右丞韋元甫揚州大都督府長史,兼御史大夫,充淮南節度觀察等使。西卯,以幽州節度副使、試太常卿硃希彩知幽州留後。遣兵部侍郎李涵兼御史大夫,使河北宣慰,以幽州亂故也。庚午,相州薛嵩、魏州田承嗣、恆州李寶臣並加左右僕射。七月壬申,崔寧弟寬攻破楊子琳,收復成都府。是月,五星並聚於東井,占曰:中國之利也。乙亥,王縉赴鎮州。



 八月己未,月掩畢。辛酉,月入東井。壬戌,吐蕃十萬寇靈武。熒或
 犯太微垣。丁卯,吐蕃寇邠寧,節度使馬璘破吐蕃二萬於邠州。御史大夫崔渙為稅地青青錢使。給百官俸錢不平,詔尚書左丞蔣渙按鞫,貶崔渙為道州刺史。庚午,河東節度使、檢校左僕射、太原尹、同中書門下平章事辛云京卒。門下侍郎、同中書門下平章事、兼幽州長史、持節、河南副元帥、都統河南淮西山南東道諸節度行營、兼幽州盧龍等軍節度使、太微宮使、弘文館大學士、兼東都留守、齊國公王縉兼太原
 尹、北都留守,充河東軍節度,餘官使並如故。辛未,以門下侍郎、同中書門下平章事、山劍副元帥、太清宮使、崇玄館大學士杜鴻漸兼東都留守。



 九月壬申。郭子儀自河中移鎮奉天。歲星入輿鬼。丁丑,濟王環薨。熒惑入太微垣。壬午,吐蕃寇靈州。甲申,以尚書左丞蔣渙為華州刺史,充鎮國軍潼關防禦使。丙戌,檢校戶部尚書、知省事、鄧國公張獻誠卒。丁亥。工部尚書趙國珍卒。庚寅,以華州刺史張重光為尚書左丞。壬辰,靈州將白元
 光破吐蕃二萬於靈武。戊戌,靈武奏破吐蕃六萬,百僚稱賀,京師解嚴。



 冬十月甲寅,朔方留後、靈武大都督府長史常廉光加檢校工部尚書。乙未,以京兆尹李勉為廣州刺史,充嶺南節度使。丁卯,子儀自奉天來朝。十一月丁亥,幽州留後硃希彩為幽州長史,充幽州盧龍節度使,癸巳,加廊下百官廚料,增舊五分之一。十二月壬寅,道州刺史崔渙卒。己酉,以邠寧節度使馬璘為涇原節度,移鎮涇州,其邠寧割隸朔方軍。邠州將吏以燒馬
 坊為亂,兵馬使段秀實斬其兇首八人,方定。



 四年春正月庚午朔。甲戌,大風。乙亥,大雪,平地盈尺。甲申,日有蝕之。子儀回河中。戊子,敕有司定王公士庶每戶稅錢,分上、中、下三等。宗室潁州刺史李岵專殺,法司以議親,宜賜自盡。乙未,福建觀察使李承昭請徙汀州於長汀縣之白石村,從之。黑衣大食國使朝貢。二月乙巳,以瀘州刺史楊子琳為陜州刺史。乙卯,宰臣杜鴻漸讓山劍副元帥,從之。丙辰夜,地震,有聲如雷者三。辛酉,
 以湖南都團練觀察使、衡州刺史韋之晉為潭州刺史。因是徙湖南軍於潭州。江西團練使魏少游來朝。三月壬申,詔:



 夫計人而置官,度事而賦任,因時立制,損益在焉。吏足以理人,人足以奉吏,則官稱其祿,祿當其秩,然後上下相樂,公私不匱。昔漢光武時及魏太和中,並減吏員,兼省鄉邑,致理之道,此其一隅。今連歲治戎,天下凋瘵,京師近甸,煩苦尤重,比屋流散,念之惻然。人寡吏多,困於供費,欲其蘇息,不可得也,設令廉恥守分,以奉
 科條,猶有錄廩之煩,役使之弊;而況貪猾縱欲,而動逾典章,作威以虐下,厚斂以潤已者乎!古者縣置大夫一員,足以為治,奚必貳佐分掌而後治耶?且京畿戶口,減耗大半,職員如舊,何以堪之?豈可以重困之人,供不給之費。使人不倦,其在變通,制事之宜,式從省便。其京兆府長安、萬年宜各減丞一員、尉兩員,餘縣各減丞、尉一員。餘委吏部條件處分。



 吏部尚書裴遵慶為右僕射,劉晏改吏部尚書。庚寅,江西團練使魏少游封趙國公。丙
 申復置仙州。



 夏四月壬寅,陜州虞邑縣復為安邑縣,虢州天平縣復為湖城縣。五月丙戌,京師地震。辛卯,以僕固懷恩女為崇徽公主,嫁回紇可汗,仍命兵部侍郎李涵往冊命。六月丁酉,以太子詹事臧希讓檢校工部尚書,充渭北節度;以渭北節度李光進為太子太保。辛亥,升辰州為都督府,析辰、巫、溪、錦、業等州置團練觀察使。秋七月己巳,以灃州刺史崔瓘為潭州刺史、湖南都團練觀察使。癸未,以天下刑官濫刑,詔:



 至理之代,先德後
 刑,上歡然以臨下,下欣然而奉上,禍亂不作,法令可施。去聖久遠,薄於教化,簡書填委,獄訟煩興。苛吏舞文,冤人致闢,思欲刷恥改行,厥路無由,豈天地父母慈愛之意也!朕主三靈之重,托群后之上,夕惕若厲,不敢荒寧。內訪卿士,外咨方岳,日不暇給,八年於茲,而大道淳風,鬱而不振。四郊多壘,連歲備邊,師旅在外,役費尤廣,賦役轉輸,疾耗吾人,困竭無聊,窮期濫矣。下庶暗昧,不見刑綱,戎士在軍,未習法令,犯禁抵罪,其徒實繁。狴犴之
 間,未詳事實,吏議不決,動淹時月,傷沮和氣,屢彰咎徵。此皆朕之不明,教之未至。上失其道而繩下以刑,敢不罪己以答災眚。人者君之支體,害之則君有所傷;刑者教之輔助,失之則人無所措。慮有冤濫,慘然憂傷,用明慎罰之典,俾弘在宥之澤。其天下見禁囚,死罪降從流,流已下釋放。左降、流人、移隸等,委所司奏聽進止。如聞州縣官比來率意恣行粗杖,不依格令,致使殞斃,深可哀傷。頻有處分,仍聞乖越。自今已後,非灼然蠹害,不得輒
 加非理,所司嚴加糾察以聞。



 先是,皇姨弟薛華因酒色之急,手刃三人,棄尸於井,事發系獄,賜自盡,故有是詔。八月丙申朔。自夏四月連雨至此月,京城米斗八百文。官出米二萬石,減估而糶,以惠貧民。己卯,虎入長壽坊元載家廟,射生將周皓引弩斃之。



 冬十月乙卯,以汝州刺史孟皞為京兆尹。十一月辛未,禁畿內弋獵。乙亥,門下侍郎、同中書門下平章事、衛國公杜鴻漸卒。丙子以左僕射、冀國公裴冕同中書門下平章事,充東都留守、
 河南淮南淮西山南東道副元帥。十二月乙未,敕左右補闕、拾遺、內供奉員左右各置兩員,餘罷之。戊戌,裴冕卒。辛酉,敕京兆府稅宜分作兩等,上等每畝稅一斗,下等稅六升,能耕墾荒地者稅二升。



 五年春正月乙丑朔。辛卯,以陜州節度使皇甫溫判鳳翔尹,充鳳翔、河隴節度使;鳳翔節度使李抱玉判梁州事,充山南西道節度使。壬申,河南尹張延賞兼御史大夫,充東都留守。罷河南、淮西、山南東道副元帥,所
 管軍隸東都留守。



 二月戊戌,李抱玉移鎮



 ,鳳翔軍仇,縱兵大掠,數日乃止。己亥。廢仙州,以襄城、葉縣隸汝州。詔罷魚朝恩觀軍容使。己巳,朝恩自縊而死。戊寅,詔定京兆府戶稅。夏稅,上田畝稅六升,下田四升。秋稅,上田畝五升,下田三升。荒田開墾者二升。己丑,敕:



 唐虞之際,內有百揆,庶政惟和。至於宗周,六卿分職,以倡九牧。《書》曰:「龍作納言,帝命惟允。《詩》云:仲山甫,王之喉舌。皆尚書之任也。雖西漢以二府分理,東京以三公總務;至於領
 錄天下之綱,綜核萬事之要,邦國善否,出納之由,莫不處正於會府也。令、僕以綜詳朝政,丞、郎以彌綸國典,法天地而分四序,配星辰而統五行,元本於是乎在。九卿之職,亦中臺之輔助,小大之政,多所關決。自王室多難,一紀於茲,東征西伐,略無寧歲。內外薦費,徵求調發,皆迫於國計,切於軍期,率於權便裁之,新書從事,且求當時之急,殊非致理之道。今外虞既平,罔不率俾,天時人事,表裹相符。將明畫一之法,大布惟新之命,陶甄化源,
 去末歸本。



 魏、晉有度支尚書,校計軍國之用,國朝但以郎官署領,辦集有餘。時艱之後,方立使額,參佐既眾,簿書轉煩,終無弘益,又失事體。其度支使及關內、河東、山南西道、劍南西川轉運常平鹽鐵等使宜停。禮儀之本,職在奉常,往年置使,因循未改,有乖舊制,實曠司存。委太常卿自舉本職,其使宜停。漢朝丞相與公卿已下五日一決事,帝親斷可否。且國之安危,不獨注于將相;考之理亂,固亦在於庶官。尚書、侍郎、左右丞及九卿,參領
 要重,朕所親倚,固當朝夕進見,以之匡益也。並宜詳校所掌,具陳損益,如非時宜,須有奏議,亦聽詣閣請對。當親覽其意,擇善而從。



 朕受昊天之成命,承累聖之鴻業,齊心滌慮,夙夜憂勞。顧以不敏不明,薄於德化,致使舊章多廢,至理未弘,其心愧恥,終食三嘆。雖詔書屢下,以申振恤,且朝典未舉,猶深鬱悼。思與百闢卿士,勵精於理,俾國經王道,可舉而行,各宜承式,以恭爾位。諸州置屯亦宜停。



 於是悉以度支之務委於宰相。辛卯,以兵部
 侍郎賈至為京兆尹。以京西兵馬使李忠臣為鳳翔尹,代皇甫溫。溫移鎮陜州。



 夏四月庚子,湖南都團練使崔旰為其兵馬使臧玠所殺,玠據潭州為亂,灃州刺史楊子琳、道州刺史裴虯、衡州刺史楊漳出軍討玠。乙巳夜,歲星入軒轅。丙午,復置先農、馬祖壇,祀之。丁未,封幽州節度使硃希彩為高密郡王。己未夜,彗起五車,長三丈。庚申,宰臣太原尹王縉入朝。五月辛未,刑部侍郎黎幹為桂州刺史、桂管防禦經略招討觀察等使。己卯夜,彗起
 北方,其色白。庚辰,貶禮儀使、禮部尚書裴士淹為虔州刺史,戶部侍郎、判度支第五琦為饒州刺史。皆魚朝恩黨也。元載既誅朝恩,下制罷使,仍放黜之。癸未,以羽林大將軍辛京杲為潭州刺史、湖南觀察使。甲申,西北白氣竟天。徙置當、悉、柘、靜、恭五州於山險要害地,備吐蕃也。



 六月己未彗星始滅,赦天下見禁囚徒。秋七月丁卯,以浙東觀察使、越州刺史、御史大夫薛兼訓為檢校工部尚書、太原尹、北都留守,充河東節度使。是月,京城斗米千文。八
 月辛卯,宰臣元載上疏請置中都於河中府,秋杪行幸,春中還京,以避蕃戎侵寇之患。疏入不報。載疏大旨以關輔、河東等十州戶稅入奉京師,創置精兵五萬,以威四方。辭多捭闔,欲權歸於己也。



 九月丁丑,以宣、歙、池等州都團練觀察使、宣州刺史、兼御史中丞陳少游充浙江東道團練觀察使。吐蕃寇永壽。汴州田神功來朝。十二月乙未,改巫州為漵州,業州為蔣州。



 六年春正月己未朔。戊寅,於鄜州之鄜城置肅戎軍。二
 月乙酉御史大夫敬括卒。夏四月丁巳,上御宣政殿試制舉人,至夕,策未成者,令太官給燭,俾盡其才。己未,灃州刺史楊子琳來朝,賜名猷。丁丑,改果州為充州。戊寅,詔:「纂組文繡,正害女紅。今師旅未息,黎元空虛,豈可使淫巧之風,有虧常制。其綾錦花文所織盤龍、對鳳、麒麟、獅子、天馬、闢邪、孔雀、仙鶴、芝草、萬字、雙勝、透背、及大繝綿、竭鑿、六破已上、並窒禁斷。其長行高麗白錦、大小花綾錦,任依舊例織造。有司明行曉諭。」五月癸卯,以河南
 尹張延賞為御史大夫。秋七月乙巳,月掩畢。



 八月乙卯,淮南節度使韋元甫卒。丙辰,以東都副留守常休明為檢校左散騎常侍、河陽三城使。夏旱,此月己未始雨。庚午,以御史大夫張延賞為揚州大都督府長史、淮南節度使。丙午、以蘇州刺史、浙江觀察使李棲筠為御史大夫。丁丑,獲白兔於太極殿之內廊。庚辰夜,月入紫微垣。九月壬辰夜,熒惑犯哭星。自八月連雨,害秋稼。戊申,於輪臺置靜塞軍。辛亥,熒惑入壁壘。



 冬十月壬午,滄州置
 橫海軍。十一月己亥,文單國王婆彌來朝,獻馴象一十一。壬寅夜,月入太微,又掩氐。十二月己未,江西觀察使、檢校刑部尚書魏少游卒。庚午,制以文單王婆彌開府儀同三司、試殿中監。是歲春旱,米斛至萬錢。



 七年春正月癸未朔。戊子,於魏州頓邱縣置澶州。以頓邱縣之觀城店置觀城縣,以張之清豐店置清豐縣,並割魏州之臨黃縣,並隸澶州。以貝州臨清縣之張橋店置永濟縣。乙未,月犯軒轅。庚子,以檢校戶部尚書路嗣
 恭為洪州刺史、兼御史大夫、江西觀察使。辛丑,太常卿楊綰兼充禮儀使。甲辰,回紇使出鴻臚寺劫坊市,吏不能禁止,復三百騎犯金光、硃雀等門。是日皇城諸門皆閉,慰諭之方止。二月甲寅,以兵部侍郎李涵為蘇州剌史、兼御史中丞,充浙西觀察使。鎮星臨太微。戊午夜,月掩天關。三月壬辰,詔諫議大夫置四員為定。



 夏四月甲寅,回紇王子李秉義卒,歸國宿衛賜名也。五月乙酉,雨雹,大風折樹。丙戌夜,月入太微。辛卯,徙忻州之七聖容
 於太原府之紫極宮。乙未,詔:



 躋於道者,化淳而刑措;善於理者,網舉而綱疏。朕涉道未弘,燭理多昧,常亦遐想太古,高挹玄風,保合太和,在宥天下,蓋德薄而未臻也。是用因時以設教,便俗以立防,務盡平恕,用申哀恤,又化淺而多犯也。加以邊虞未戢,徭賦適繁,荒廢之際,寇攘期起。遂令圓土嘉石之下,積有系囚;竹章牙簡之中,困於法吏。屬盛陽之候,大暑方蒸,仍念狴牢,何堪鬱灼?所以汨傷和氣,感致咎徵,天道人事,豈相遠也!如聞天
 下諸州,或愆時雨,首種不入,宿麥未登。哀我矜人,何時不恐?皆由朕過,益用懼焉。惕然憂嗟,深自咎責。所以減膳徹樂,別居齋宮,禱於神明,冀獲嘉應。仲夏之月,靜事無為,以助晏陰,以弘長養。斷薄決小,已過於麥秋;繼長增高,宜順乎天意。可大赦天下,見禁囚徒,罪無輕重,一切釋放。



 癸亥,以檢校禮部尚書蔣渙充東都留守。六月庚戌,有司言日蝕,陰雲不見。丁丑,詔誡薄葬,不得造假花果及金手脫寶鈿等物。秋七月癸巳,回紇蕃客奪
 長安縣令邵說所乘馬,人吏不能禁。八月庚戌,賜北庭都護曹令忠姓名曰李元忠。



 九月乙未,工部尚書於休烈卒。



 冬十月壬子,上畋於苑中,矢一發貫二兔,從臣皆賀。辛未,以權知幽州盧龍節度留後硃泚檢校左散騎常侍,充幽州盧龍節度使。丙子,以太府卿呂崇賁為廣州都督,充嶺南節度使。十一月庚辰,詔:自頃蕃戎入寇,巴南屢多征役。其巴、蓬、渠、集、壁、充、通、開等州,宜放二年租庸。甲申,以福建觀察使李承昭為禮部尚書,華州刺
 史李琦為福州刺史、福建都團練觀察使。辛卯,以嶺南節度李勉為工部尚書。十二月丙寅,雨士。是夜,長星出於參。辛未,滑州置永平軍。壬子,禁鑄銅器。癸酉,大雪。是秋稔。回紇、吐蕃、大食、渤海、室韋、靺鞨、契丹、奚、牂柯、康國、石國並遣使朝貢。



 八年春正月丁丑朔,壬午,昭義軍節度、檢校右僕射、相州刺史薛嵩卒。癸卯,敕天下青苗地頭錢每畝十五文,率京畿三十文,自今一例十五文。京官三品已上郎官
 御史,每年各舉一人堪為刺史縣令者。二月甲子,御史大夫李棲筠彈吏部侍郎徐浩。丁卯,幽州節度使硃泚加檢校戶部尚書,封懷寧郡王。徐浩、薛邕違格,並停知選事。壬申,永平軍節度使、檢校右僕射、滑州刺史、霍國公令狐彰卒,遣表薦劉晏、李勉代己。三月丙子,以工部尚書李勉兼御史大夫、滑州刺史,充永平軍節度、滑亳觀察等使。



 夏四月戊申,乾陵上仙觀天尊殿有雙鵲銜紫泥補殿之隙缺,凡十五處。戊午,以太僕卿吳仲孺為鄂州刺史、
 鄂岳沔等州團練觀察使。五月乙酉,貶吏部侍郎徐浩明州別駕,薛邕歙州刺史,京兆尹杜濟杭州刺史,皆坐典選也。以太府卿於頎為京兆尹。辛卯,鄭王邈薨,贈昭靜太子。壬辰,曲赦京城系囚。癸卯,詔赦天下系囚,死罪降從流,流已下並放。六月隴州華亭縣置義寧軍。癸亥,戶部侍郎、判度支韓滉奏安邑鹽池生乳鹽。是夏,城奉天以備蕃寇。秋七月己卯,太白入東井。乙未,月掩畢。



 八月甲寅,詔吏部尚書劉晏知三銓選事。己未,吐蕃寇靈
 武。庚午,靈武奏蕃軍退去。辛未,幽州節度使硃泚弟滔率五千騎來朝,請河西防秋。詔千騎迓於國門,許自皇城南面出開遠門,赴涇州行營。九月癸酉,臨晉公主薨。壬午,嶺南節度使、廣州刺史呂崇賁為部將哥舒晃所殺。癸未,晉州男子郇謨以麻辮發,持竹筐及葦席,哭於東市,請進三十字,如不稱旨,請裹尸於席筐。上召見,賜衣館之禁中。內二字曰「監團」,欲去諸道監軍、團練使也。丁亥,貶左巡使、殿中侍御史楊護,以其抑郇謨而不上
 聞也,戊子,詔京官五品以上各上封事,議政得失。己丑夜,太白入太微。甲午,東都留守蔣瓊兼知東都貢舉。戊戌,以辰、錦觀察使李昌巙為桂州刺史、桂管防禦觀察使。大鳥見武功,肉翅狐首,四足有爪,爪長四尺三寸,毛赤如蝙蝠,群鳥隨而噪之。神策將張日芬射斃以獻。



 冬十月癸卯,魏博田承嗣加同平章事。丁巳夜,月掩畢。吐蕃寇涇州、邠州。甲子,子儀先鋒將琿瑊與吐蕃戰於宜錄,我師不利。瑊與涇原馬璘極力追躡,蕃軍潰去。乙丑,
 以江西觀察使路嗣恭為廣州刺史,充嶺南節度使,封翼國公。以浙東觀察使、越州刺史陳少游為揚州大都督府長史,充淮南節度使。戊辰,郭子儀奏破吐蕃十萬,百僚稱賀。己卯夜,月入羽林。癸巳,月入太微。十一月壬寅朔。庚戌,汴宋節度使田神功來朝。辛酉,淮西節度使李忠臣來朝。十二月癸酉,月入羽林。是冬無雪。是歲大有年。



 九年正月庚子朔。壬寅,汴宋節度使、太子少師、檢校
 尚書右僕射、兼御史大夫、汴州刺史田神功卒。灃朗兩州鎮遏使、灃州刺史楊猷擅浮江而下,至鄂州。詔許赴汝州,遂溯漢而上,復、郢、襄等州皆閉城拒之。二月己丑,以田神功弟神玉權知汴宋留後。癸巳,郭子儀自邠州來朝,李抱玉自鳳翔來朝。三月丙午,禁畿內漁獵採捕,自正月至五月晦,永為常式。戊子,以灃州刺史楊猷為洮州刺史。



 夏四月丁丑,月入太微。己卯,以桂管觀察使黎幹為京兆尹、兼御史大夫。甲申,中書舍人常袞率兩省
 官一十八人詣閣請論事,詔三人各盡所懷。乙酉,詔郭子儀等大閱兵師以備吐蕃。壬辰,詔赦大闢以下系囚,無輕重釋放。乙未,華陽公主薨,上悲惜之,累日不聽朝,宰臣抗疏陳請。五月庚戌,廢泜州。庚申,詔度支使支七十萬貫、轉運使五十萬貫和糴,歲豐穀賤也。乙丑,詔:



 四海之內,方協大寧,西戎無厭,獨阻王命,不可忘戰,尚勞邊事。朕頃以兵革之後,軍國空耗,躬率節儉,務勤農桑。上玄儲休,仍歲大稔,益用多愧,不知其然。雖屬此人和,近於家給,
 而邊穀未實,戎備猶虛。因其天時,思致豐積,將設平糴,以之餽軍。然以中都所供,內府不足,粗充常入之數,豈齊倍餘之收。其在方面藎臣,成茲大計。共佐公家之急,以資塞下之儲。每道歲有防秋兵馬,其淮南四千人,浙西三千人,魏博四千人,昭義二千人,成德三千人,山南東道三千人,荊南二千人,湖南三千人,山南西道二千人,劍南西川三千人,東川二千人,鄂岳一千五百人。宣歙三千人,福建一千五百人。其嶺南、浙東、浙西,亦合準
 例。恐路遠往來增費,各委本道每年取當使諸色雜錢及回易利潤、贓贖錢等,每人計二十貫。每道據合配防秋人數多少,都計錢數,市輕貨送納上都,以備和糴,仍以秋收送畢。



 涇原節度使馬璘來朝。丙寅,加馬璘尚書左僕射、知省事。璘諷將士進狀求宰相,故有是授。幽州節度使硃泚遣弟滔奉表請自入朝,兼自率五千騎防秋。許之,詔所司築第待之。六月己卯,月掩南斗。庚辰,月入太微。秋七月甲辰,月掩房,又入羽林。久旱,京兆尹黎幹歷
 禱諸祠,未雨。又請禱文宣廟,上曰「丘之禱久矣。」八月辛未,以虢州刺史宋晦為同州刺史,充長春宮營田等使。戊寅,以陜州大都督府長史皇甫溫為越州剌史,充浙東觀察使。辛卯,月掩軒轅。九月庚子,幽州節度使硃泚來朝。乙巳,渭北節度使、坊州刺史臧希讓卒。是秋大雨。



 冬十月壬申,信王瑝薨。乙亥,梁王璿薨。以前宣州刺史季廣琛為右散騎常侍。十一月戊戌,大雪。平地盈尺。庚子,以商州刺史李國清為陜州大都督府長史,充陜
 州觀察使。十二月庚寅,以中書舍人楊炎、秘書少監韋肇並為吏部侍郎,中書舍人常袞為禮部待郎。壬辰,赦京系囚,死罪降從流,流已下並釋放。



 十年春正月乙未朔。己酉,昭義牙將裴志清逐其帥薛摐,薛摐奔洺州,上章待罪。志清率眾歸田承嗣。壬寅,壽王瑁薨。乙未,硃泚抗表乞留京師,西征吐蕃,請以弟滔權為幽州留後,許之。以昭義將薛擇為相州刺史,薛雄為衛州刺史,薛堅為洺州刺史,皆嵩之族人也。戊申,遣
 使慰諭田承嗣,令各守封疆,承嗣不奉詔。壬子,充州復為果州。癸丑。田承嗣盜取洺州,又破衛州。二月乙丑,盜殺衛州刺史薛雄。丙寅,罷辰、錦、溪、獎、漵五州經略使,復隸黔中。辛未,制第四子述封睦王,充嶺南節度度支營田、王府經略觀察處置等大使。第五子逾可封郴王,充渭北鄜坊等州節度大使。第六子連封恩王。第七子韓王迥可充汴宋節度大使。第八子遘可封鄜王。第十三子造封忻王,充昭義節度大使。第十四子暹封韶王。十五子
 運封嘉王。十六子遇封端王。十七子遹封循王。十八子通封恭王。十九子達封原王。二十子逸封雅王。並可開府儀同三司,不出閣。丙子,以華州刺史李承昭為相州刺史。知昭義兵馬留後。時田承嗣盡盜入相、衛所管四州之地,自署長吏。是日河陽軍亂,逐城使常休明,迫牙將王惟恭為留後,軍士大掠數日,休明奔東都。甲申,以平盧淄青節度觀察海運押新羅渤海兩蕃等使、檢校工部尚書、青州刺史李正己檢校尚書左僕射;前隴
 右節度副使、隴州刺史馬燧為商州刺史,充本州防禦使。



 三月甲午,陜州軍亂,逐觀察使李國清,縱兵大掠。國清卑詞遍拜將士,方免禍,一夕而定。乙巳,薛崿、常休明至闕下,素服待罪。丁未,以左散騎常侍孟皞為華州刺史,充潼關防禦使。庚戌,熒惑入壁壘。四月,太常寺奏:諸州府所用斗秤,當寺給銅斗秤,州府依樣製造而行,從之。乙丑,制:魏博節度使、開府儀岵司、太尉、檢校尚書左僕射、同中書門下平章事、魏州大都督府長史、上柱
 國、雁門郡王田承嗣可貶永州刺史。仍詔河東、鎮冀、幽州、淄青、淮西、滑毫、汴宋、澤潞、河陽道出師進討。甲申,大雨雹,暴風拔樹,飄屋瓦,落鴟吻,人震死者十之二,京畿損稼者七縣。五月乙未,田承嗣部將霍榮國以磁州歸。癸卯,劍南置昌州。罷兩都貢舉,都集上都,停童子科。



 六月辛未,田承承嗣遣其黨裴志清攻圍冀州,為李寶臣所敗。秋七月己未。戶部尚書暢璀卒。杭州大風,海水翻潮,溺州民五千家,船千艘。八月丁卯,田承嗣上表請束身
 歸朝。戊子夜,月入太微。己丑,田承嗣將盧子期攻磁州。九月戊戌,荊南節度使衛伯玉來朝。壬寅,宥京城系囚。戊申,回紇白晝殺人於市,吏捕之,拘於萬年獄。其首領赤心持兵入縣,劫囚而出,斫傷獄吏。月暈,熒惑犯昴、五車、參、東井等星。癸丑,吐蕃寇隴州,鳳翔李抱玉擊之。戊午,幽州節度使硃泚鎮奉天。



 冬十月辛酉日,日有蝕之。癸亥,以商州刺史馬燧檢校左散騎常侍、河陽三城使。甲子,昭義節度使李承昭與盧子期戰於磁州清水縣,
 大破之,生擒子期以獻。丙寅,貴妃獨孤氏薨,追贈曰貞懿皇后。己丑,尚書右僕射裴遵慶卒。十一月辛卯,新平公主薨。丁酉,田承嗣所署瀛州刺史吳希光以城降。丁未,路嗣恭攻破廣州,擒哥舒晃,斬首以獻。



 十一年春正月庚寅朔,田承嗣上表請罪。壬辰,遣諫議大夫杜亞使魏州宣慰,許其自新。辛亥。劍南節度使崔寧奏大破吐蕃二十萬,斬首萬級,生擒首領一千一百五十人,獻於闕下。二月癸亥,荊南節度使衛伯玉卒於京
 師。戊子,河陽軍復亂,大掠三日,監軍使冉廷蘭率兵斬其亂首。方定。戊申,昌樂公主薨。辛亥,御史大夫李棲筠卒。



 夏四月戊午朔。丙子,以浙西觀察使、蘇州刺史、御史大夫李涵知臺事,充京畿觀察使。己卯,以前淮南節度使、陽州大都督府長史、御史大夫張延賞為江陵尹、兼御史大夫充荊南節度使。五月癸巳,以永平軍節度使李勉為汴州刺史,充汴宋等八州節度觀察留後,時汴將李靈耀專殺濮州刺史孟鑒,北連田承嗣。故命勉兼
 領汴州。授靈耀濮州刺史,靈耀不受詔。



 六月戊戌,以李靈耀為汴州刺史,充節度留後。秋七月戊子夜,暴澍雨,平地水深盈尺,溝渠漲溢,壞坊民千二百家。庚寅,田承嗣兵寇滑州,李勉拒戰而敗。八月丙寅,幽州節度使硃泚加同中書門下平章事。李靈耀據汴叛。甲申,命淮西李忠臣、滑州李勉、河陽馬燧三鎮兵討之。閏月丁酉,太白經天。九月乙丑,李忠臣等兵進營鄭州,靈耀之眾來薄戰。淮西兵亂,乃退軍於滎澤。戊辰,淄青李正己奏
 取鄆、濮二州。



 冬十月乙酉,忠臣等宮破賊於中牟,進軍,又敗賊於汴州郭外,乃攻之。乙丑,承嗣遣侄悅率兵三萬援靈耀。丙午,淮西、河陽之師合擊田悅營,其眾大敗,悅脫身北走。靈耀聞悅之敗,棄城遁走。汴州平。丁未,滑將杜如江生擒靈耀而獻。十二月丁亥,加平盧淄青節度使、檢校尚書左僕射、青州刺史、饒陽王李正己為檢校司空、同中書門下平章事,成德軍節度使、太子太傅、檢校尚書左僕射、隴西郡王李寶臣檢校司空、同中書
 門下平章事。庚寅,涇原節度使、檢校尚書左僕射、知省事、扶風郡王馬璘卒。丁酉,以涇原節度副使、試太常卿、張掖郡王段秀實權知河東節度留後,北都留守薛兼訓病故也。昭義節度使李承昭搞表稱疾,以澤潞行軍司馬李抱真權知磁、邢兵馬留後。庚戌,加淮西節度、檢校右僕射、安州刺史、西平郡王李忠臣檢校司空、同中書門下平章事,仍兼汴州刺史。



 十二年春正月甲寅朔。辛酉,以四鎮北庭涇原節度副
 使、知節度使事、張掖郡王段秀實為涇州刺史、兼御史大夫,充本州團練使。月掩軒轅。渤海使獻日本國舞女十一人。癸酉夜,月掩心前大星,又入南斗魁。京師旱,分命祈禱。二月戊子,淄青節度使李正己之子納為青州刺史,充淄青節度留後。丁未,以朗州刺史李國清為黔州刺史、經略招討觀察使。



 三月乙卯,河西隴右副元帥、鳳翔懷澤潞秦隴等州節度觀察等使、兵部尚書門下平章事、潞州大都督府長史、知鳳翔府事、上柱
 國、涼國公李抱玉卒。壬戌,月入太微。癸亥,以太原少尹、河東節度行軍司馬、權知河留後鮑防為太原尹、御史大夫,充北都留守、河東節度使。戊辰夜,月逼心前星。庚午,左降官永州刺史田承嗣復授魏博節度使,餘官並如故。承嗣侄悅、子綰緒綸並復舊官。庚辰,宰相元載、王縉得罪下獄,命吏部尚書劉晏訊鞫之。辛巳,制:中書侍郎、平章事元載賜自盡,門下侍郎、平章事王縉貶括州刺史。



 夏四月壬午,以朝議大夫、守太常卿、兼修國史楊
 綰為中書侍郎,尚書禮部郎、集賢院學士常袞為門下侍郎,並同中書門下平章事。癸未,以右庶子潘炎為禮部侍郎。貶吏部侍郎楊炎為道州司馬,元載黨也。諫議大夫、知制誥韓洄王定包佶徐璜,戶部侍郎趙縱,大理少卿裴翼,太常少卿王紞,起居舍人韓會等十餘人,皆坐元載貶官也。給事中杜亞使魏州,賜田承嗣鐵券。癸巳,以前秘書監李揆為睦州刺史。揆故宰相,為元載所忌,二十年流落丐食江湖間,載誅,方得為郡。又召顏
 真卿於湖州,亦載所忌斥外也。乙未,月掩心前星。丁酉,西川破吐蕃於望漢城,擒蕃將大籠官論器然以獻。壬寅,以前商州刺史烏崇福為安南都護、本營經略使。渤海、奚、契丹、室韋、靺鞨並遣使朝貢。己酉,加京官料錢,文武班諸司共二千七百九十六員,文官一千八五十四員,武官九百四十二員,歲加給一十五萬六千貫,並舊給凡二十六萬貫。以關內副元帥、兵馬使渾瑊兼邠州刺史。



 五月辛亥。罷天下州團練守捉使名。甲寅,諸道
 邸務在上都名曰留後,改為進奏院。丙辰夜,月入太微。辛酉,貶刑部尚書王昂連州刺史,昂至萬州卒。庚午,敕毀元載祖、父墳,剖棺棄骸,焚毀私廟主於大寧里。甲戌,以前安南都護張伯儀為廣州刺史、兼御史大夫,充嶺南節度使。六月癸巳,時小旱,上齋居祈禱,聖體不康,是日不視朝。秋七月戊午,罷潤州丹陽軍、蘇州長洲軍。己巳,中書侍郎、同中書門下平章事、集賢殿崇文館大學士、兼修國史楊綰卒。八月癸巳,賜東川節度使鮮於叔
 明姓李氏。癸卯,宰臣讓賜食。先是元載、王縉輔政,每日賜食,因為故事。至是,常袞等上表云:「飧錢已多,更頒御膳,胡顏自安,乞停賜食。」從之。甲辰,以湖州刺史顏真卿為刑部尚書。乙巳,以久雨宥常參百僚,不許御史點班。九月乙卯,許以庶人禮葬元載。辛酉,以涇原節度副使段秀實為四鎮北庭行營、涇原鄭潁等節度使。庚午,吐蕃寇坊州,掠黨項羊馬而去。是秋,宋、亳、陳、滑等州水。



 冬十月丁亥,戶部待郎、判度支韓滉言解縣兩池生瑞鹽,
 乃置祠,號寶應靈慶池。壬寅夜,月掩昴,又入太微。乙巳,以滑州牙將劉洽為宋州刺史。京兆尹黎幹奏水損田三萬一千頃。度支使韓滉奏所損不多。兼渭南令劉藻曲附滉,亦云部內田不損。差御史趙計檢渭南田,亦附滉云不損。上曰:「水旱咸均,不宜渭南獨免。」復命御史硃敖檢之,渭南損田三千頃。上嘆息曰:「縣令職在字人,不損亦宜稱損,損而不聞,豈有恤隱之意耶!劉藻、趙計皆貶官。十一月癸丑,太白臨哭星。乙卯夜,月入羽林。癸酉,
 以右散騎常侍蕭昕為工閱尚書。刑部尚書顏真卿獻所著《韻海鏡源》三百六十卷。十二月丁亥,西川崔寧奏於西山破吐蕃十萬,斬首八千,生擒九百人。己亥,天下仙洞靈跡禁樵捕。庚子,以幽州節度使硃泚兼隴右節度副大使,權知河西、澤潞行營兵馬事。京兆尹請修六門堰,許之。



 十三年春正月戊申朔。辛酉,壞白渠碾磑八十餘所,以奪農溉田也。壬戌,刑部尚書、魯郡公顏真卿三抗章乞
 致仁,不允。淄青節度使李正己請附屬籍,從之。戊辰,回紇寇太原,鮑防與之戰,我師不利。硃泚徙封遂寧郡王。二月庚辰,代州都督張光晟擊回紇,戰於羊武谷,破之,北人乃安。己亥,吐蕃寇靈武。甲辰,太僕寺佛堂有小脫空金剛右臂忽有黑汁滴下,以紙承之,色類血。



 三月甲戌,河陽將士劫回紇輜重,因與相鬥,縱兵大掠,久之方定。四月丁亥,以浙西觀察留後李道昌為蘇州刺史、兼御史中丞,充浙西都團練觀察使。己丑,以前浙西觀察
 使李涵為御史大夫。甲辰,吐蕃寇靈州,朔方留後常謙光擊敗之。五月戊午,宦官劉清潭賜名忠翼。六月戊戌,隴右節度使硃泚於軍士趙貴家得貓鼠同乳不相害,籠而獻之。秋七月壬子,中書舍人崔祐甫知吏部選事。癸丑,劍南節度使崔寧加檢校司空,東川李叔明加檢校工部尚書。辛未,吐蕃寇鹽州、慶州。八月甲戌朔,成德軍節度使李寶臣抗章請復本姓張氏,從之。



 冬十月丁酉,葬貞懿皇后於莊陵。十一月丁卯,日長至,有司禮昊
 天上帝於南郊,上不視朝故也。十二月丙戌,以吏部尚書劉晏為左僕射,判使如故。以給事中杜亞為洪州刺史、兼御史中丞,充江西觀察使。以江西觀察使路嗣恭為兵部尚書。是歲,郴州黃芩山崩,壓死者有數百人。



 十四年春正月壬寅朔。壬戌,以楚州刺史李泌為澧州刺史。二月癸未,魏博七州節度使、太尉、檢校尚書左僕射、同中書門下平章事、魏州大都督府長史田承嗣卒。甲申,以魏博中軍兵馬使、左司馬田悅兼御史中丞,充
 魏博節度留後。



 三月丁未,汴宋節度使李忠臣為麾下將族侄李希烈所逐,忠臣狼狽歸朝。上以忠臣立功於國,乃授檢校司空、同平章事。庚戌,以河南尹嚴郢為京兆尹,河中少尹、知府事趙惠伯為河南尹。辛酉,以前容管經略使、容州刺史王翃為河中少尹、知府事。



 夏四月癸未,成德軍節度使張寶臣復請姓李,從之。五月癸卯,上不康,至辛亥,不視朝。北都留守鮑防以北庭歸朝。辛酉,詔皇太子監國。是夕,上崩於紫宸之內殿。遺詔皇太
 子柩前即位。壬戌,遷神柩於太極殿,發喪。八月庚申,群臣上尊謚曰睿文孝武皇帝,廟號代宗。十月己酉,葬於元陵。十二月丁酉,祔於太廟。



 史臣曰:嗚呼,治道之失也,若河決金堤,火炎昆崗,雖神禹之乘四載,玄冥之灑八瀛,亦不能堙洪濤而撲烈焰者,何也?良以勢既壞而不能遽救也。觀夫開元之治也,則橫制六合,駿奔百蠻;及天寶之亂也,天子不能守兩都,諸侯不能安九牧。是知有天下者,治道其可忽乎!明
 皇之失馭也,則思明再陷於河洛;大歷之失馭也,則懷恩鄉導於犬戎。自三盜合從,九州羹沸,軍士膏於原野,民力殫於轉輸,室家相吊,人不聊生,而子儀號泣於用兵,元載殷憂於避狄。然而代宗皇帝少屬亂離,老於軍旅,識人間之情偽,知稼穡之艱難,內有李、郭之效忠,外有昆戎之幸利。遂得兇渠傳首,叛黨革心,關輔載寧,獯戎漸弭。至如稔輔國之惡,議元振之罪,去朝恩之權,不以酷刑,俾之自咎,
 亦立法念功之旨也。罪己以傷僕固,徹樂而悼神功,懲縉、載之奸回,重袞、綰之儒雅,修己以禳星變,側身以謝咎徵,古之賢君,未能及此。而猶有李靈耀作梗,田承嗣負恩,命將出軍,勞師弊賦者,蓋陽九之未泰,豈君道之過歟!



 贊曰:群盜方梗,諸戎競侵。猛士嘗膽,忠臣痛心。掃除沴氣,敷衍德音。延洪納祉,帝慮何深。



\end{pinyinscope}