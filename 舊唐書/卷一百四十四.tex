\article{卷一百四十四}

\begin{pinyinscope}

 ○韋皋
 劉闢附張建封盧群



 韋皋,字城武,京兆人。大歷初,以建陵挽郎調補華州參軍,累授使府監察御史。宰相張鎰出為鳳翊隴右節度使,奏皋為營田判官,得殿中侍御史,權知隴州行營留
 後事。



 建中四年,涇師犯闕,德宗幸奉天,鳳翔兵馬使李楚琳殺張鎰,以府城叛歸於硃泚,隴州刺史郝通奔於楚琳。先是,硃泚自範陽入朝,以甲士自隨;後泚為鳳翔節度使,既罷,留範陽五百人戍隴州,而泚舊將牛雲光督之。時泚既以逆徒圍奉天,雲光因稱疾,請皋為帥,將謀亂,擒皋以赴泚。皋將翟曄伺知之,白皋為備;雲光知事洩,遂率其兵以奔泚。行及汧陽,遇泚家僮蘇玉將使於皋所,蘇玉謂雲光曰:「太尉已登寶位,使我持詔以韋
 皋為御史中丞,君可以兵歸隴州。皋若承命,即為吾人;如不受詔,彼書生,可以圖之,事無不濟矣。」乃反昪疾趨隴州。皋迎勞之,先納蘇玉,受其偽命,乃問雲光曰:「始不告而去,今又來,何也?」雲光曰:「前未知公心,故潛去;知公有新命,今乃復還。願與公戮力定功,同其生死。」皋曰:「善。」又謂雲光曰:「大使茍不懷詐,請納器甲,使城中無所危疑,乃可入。」雲光以書生待皋,且以為信然,乃盡付弓矢戈甲。皋既受之,乃內其兵。明日,皋犒宴蘇玉、雲光之卒
 於郡舍,伏甲於兩廊。酒既行,伏發,盡誅之,斬雲光、蘇玉首以徇。泚又使家僮劉海廣以皋為鳳翔節度使,皋斬海廣及從者三人,生一人,使報泚。於是詔以皋為御史大夫、隴州刺史,置奉義軍節度以旌之。皋遣從兄平及弇繼入奉天城,城中聞皋有備,士氣增倍。



 皋乃築壇於廷,血牲與將士等盟曰:「上天不吊,國家多難,逆臣乘間,盜據宮闈。而李楚琳亦扇兇徒,傾陷城邑,酷虐所加,爰及本使,既不事上,安能恤下。皋是用激心憤氣,不遑底
 寧,誓與群公,竭誠王室。凡我同盟,一心協力,仗順除兇,先祖之靈,必當幽贊。言誠則志合,義感則心齊;粉骨糜軀,決無所顧。有渝此志,明神殛之,迨於子孫,亦罔遺類。皇天后土,當兆斯言。」又遣使入吐蕃求援。十一月,加檢校禮部尚書。興元元年,德宗還京,徵為左金吾衛將軍,尋遷大將軍。



 貞元元年,拜檢校戶部尚書,兼成都尹、御史大夫、劍南西川節度使,代張延賞。皋以雲南蠻眾數十萬,與吐蕃和好,蕃人入寇,必以蠻為前鋒。四年,皋遣
 判官崔佐時入南詔蠻,說令向化,以離吐蕃之助。佐時至蠻國羊咀咩城,其王異牟尋忻然接遇,請絕吐蕃,遣使朝貢。其年,遣東蠻鬼主驃傍、苴夢沖、苴烏等相率入朝。南蠻自巂州陷沒,臣屬吐蕃,絕朝貢者二十餘年,至是復通。



 五年,皋遣大將王有道簡習精卒以入蕃界,與東蠻於故巂州臺登北穀大破吐蕃青海、臘城二節度,斬首二千級,生擒籠官四十五人,其投崖谷而死者不可勝計。蕃將乞臧遮遮者,蕃之驍將也,久為邊患。自擒
 遮遮,城柵無不降,數年之內,終復巂州,以功加吏部尚書。九年,朝廷築鹽州城,慮為吐蕃掩襲,詔皋出兵牽維之。乃命大將董勔、張芬出西山及南道,破峨和城、通鶴軍。吐蕃南道元帥論莽熱率眾來援,又破之,殺傷數千人,焚定廉城。凡平堡柵五十餘所,以功進位檢校右僕射。皋又招撫西山羌女、訶陵、白狗、逋租、弱水、南王等八國酋長,入貢闕廷。十一年九月,加統押近界諸蠻、西山八國兼雲南安撫等使。十二年二月,就加同中書門下
 平章事。十三年,收復巂州城。十六年,皋命將出軍,累破吐蕃於黎、巂二州。吐蕃怒,遂大搜閱,築壘造舟,欲謀入寇,皋悉挫之。於是吐蕃酋帥兼監統曩貢、臘城等九節度嬰、籠官馬定德與其大將八十七人舉部落來降。定德有計略,習知兵法及山川地形,吐蕃每用兵,定德常乘驛計事,蕃中諸將稟其成算。至是,自以捍邊失律,懼得罪而歸心焉。



 十七年,吐蕃昆明城管些蠻千餘戶又降。贊普以其眾外潰,遂北寇靈、朔,陷麟州。德宗遣使至
 成都府,令皋出兵深入蕃界。皋乃令鎮靜軍使陳洎等統兵萬人出三奇路,威戎軍使崔堯臣兵千人出龍溪石門路南,維保二州兵馬使仇冕、保霸二州刺史董振等兵二千趨吐蕃維州城中,北路兵馬使邢玼等四千趨吐蕃棲雞、老翁城,都將高倜、王英俊兵二千趨故松州,隴東兵馬使元膺兵八千人出南道雅、邛、黎、巂路。又令鎮南軍使韋良金兵一千三百續進,雅州經略使路惟明等兵三千趨吐蕃租、松等城,黎州經略使王有道
 兵二千人過大渡河,深入蕃界,巂州經略使陳孝陽、兵馬使何大海、韋義等及磨些蠻、東蠻二部落主苴那時等兵四千進攻昆明城、諾濟城。自八月出軍齊入,至十月破蕃兵十六萬,拔城七、軍鎮五、戶三千,擒生六千,斬首萬餘級,遂進攻維州。救軍再至,轉戰千里,蕃軍連敗。於是寇靈、朔之眾引而南下,贊普遣論莽熱以內大相兼東境五道節度兵馬都群牧大使,率雜虜十萬而來解維州之圍。蜀師萬人據險設伏以待之,先出千人挑
 戰。莽熱見我師之少,悉眾追之。發伏掩擊,鼓噪雷駭,蕃兵自潰,生擒論莽熱,虜眾十萬,殲夷者半。是歲十月,遣使獻論莽熱於朝;德宗數而釋之,賜第於崇仁里。皋以功加檢校司徒,兼中書令,封南康郡王。



 順宗即位,加檢校太尉。順宗久疾,不能臨朝聽政,宦者李忠言、侍棋待詔王叔文、侍書待詔王伾等三人頗干國政,高下在心。皋乃遣支度副使劉闢使於京師,闢私謁王叔文曰:「太尉使致誠於足下,若能致某都領劍南三川,必有以相
 酬;如不留意,亦有以奉報。」叔文大怒,將斬闢以徇;韋執誼固止之,闢乃私去。皋知王叔文人情不附,又知與韋執誼有隙,自以大臣可議社稷大計,乃上表請皇太子監國,曰:「臣聞上承宗廟,下鎮黎元,永固無疆,莫先儲兩。伏聞聖明以山陵未祔,哀毀逾制,心勞萬幾,伏計旬月之間,未甚痊復。皇太子睿質已長,淑問日彰,四海之心,實所倚賴。伏望權令皇太子監撫庶政,以俟聖躬痊平,一日萬幾,免令壅滯。」又上皇太子箋曰:



 殿下體重離之
 德,當儲貳之重,所以克昌九廟,式固萬方,天下安危,系於殿下。皋位居將相,志切匡扶,先朝獎知,早承恩顧。人臣之分,知無不為,願上答眷私,罄輸肝鬲。伏以聖上嗣膺鴻業,睿哲英明,攀感先朝,志存孝理。諒闇之際,方委大臣,但付托偶失於善人,而參決多虧於公政。今群小得志,隳紊紀綱,官以勢遷,政由情改,朋黨交構,熒惑宸聰。樹置腹心,遍於貴位;潛結左右,難在蕭墻。國賦散於權門,王稅不入天府,褻慢無忌,高下在心。貨賄流聞,遷
 轉失敘,先聖屏黜贓犯之類,咸擢居省寺之間。至令忠臣隕涕,正人結舌,遐邇痛心,人知不可。伏恐奸雄乘便,因此謀動干戈,危殿下之家邦,傾太宗之王業。伏惟太宗櫛沐風雨,經營廟朝,將垂二百年,欲及千萬祀;而一朝使叔文奸佞之徒,侮弄朝政,恣其胸臆,坐致傾危。臣每思之,痛心疾首!伏望殿下斥逐群小,委任賢良,心妻々血誠,輸寫於此!



 太子優令答之。而裴均、嚴綬箋表繼至,由是政歸太子,盡逐伾文之黨。是歲,暴疾卒,時年六十
 一,贈太師,廢朝五日。



 皋在蜀二十一年,重賦斂以事月進,卒致蜀土虛竭,時論非之。其從事累官稍崇者,則奏為屬郡刺史,或又署在府幕,多不令還朝,蓋不欲洩所為於闕下故也。故劉闢因皋故態,圖不軌以求三川,歷階之作,蓋有由然。



 皋兄聿,時為國子司業,劉闢與盧文若據西川叛,皋侄行式,先娶文若妹,而聿不奏。既收行式,以其妻沒官,詔御史臺按聿,聿下獄。有司以行式妻在遠,不與兄同情,不當連坐,詔歸行式妻而釋聿。



 劉闢者,貞元中進士擢第,宏詞登科,韋皋闢為從事,累遷至御史中丞、支度副使。永貞元年八月,韋皋卒,闢自為西川節度留後,率成都將校上表請降節鉞。朝廷不許,除給事中,便令赴闕。闢不奉詔。時憲宗初即位,以無事息人為務,遂授闢檢校工部尚書,充劍南西川節度使。闢益兇悖,出不臣之言,而求都統三川,與同幕盧文若相善,欲以文若為東川節度使,遂舉兵圍梓州。憲宗難於用兵,宰相杜黃裳奏:「劉闢一狂蹶書生耳,王師鼓
 行而俘之,兵不血刃。臣知神策軍使高崇文,驍果可任,舉必成功。」帝數日方從之。於是令高崇文、李元奕將神策京西行營兵相續進發,令與嚴礪、李康掎角相應以討之,仍許其自新。



 元和元年正月,崇文出師。三月,收復東川。乃下詔曰:



 朕聞皇祖玄元之誡曰:「兵者,兇器也,不得已而用之。」恭惟聖謨,常所祗服。故惟文誥有所不至,誠信有所未孚,始務安人,必能忍恥,朕之此志,亦可明徵。近者德宗皇帝舉柔服之規,授宰衡之傑,弘我廟勝,
 遂康巴、庸,故得南詔入貢,西戎寢患。成績始究,元臣喪亡,劉闢乘此變故,坐邀符節。朕以成狂命者雖乖於理體,從權便者所冀於輯寧,竟乖卿士之謀,遂允幸求之志。朕之於闢,恩亦弘矣。曾不知恩,負牛羊之力,飽則逾兇;畜梟獍之心,馴之益悖。誑惑士伍,圍逼梓州;誘陷戎臣,塞絕劍路。師徒所至,燒劫無遺,干紀之辜,擢發難數。朕為人司牧,字彼黎元,如闢之罪,非朕敢舍,可削奪在身官爵。



 六月,崇文破鹿頭關,進收漢州。九月,崇文收成
 都府。劉闢以數十騎遁走,投水不死;騎將酈定進入水,擒闢於成都府西洋灌田。盧文若先自刃其妻子,然後縋石投江,失其尸。闢檻送京師,在路飲食自若,以為不當死。及至京西臨皋驛,左右神策兵士迎之,以帛系首及手足,曳而入,乃驚曰:「何至於是?」或紿之曰:「國法當爾,無憂也。」是日,詔曰:「劉闢生於士族,敢蓄梟心,驅劫蜀人,拒捍王命。肆其狂逆,詿誤一州,俾我黎元,肝腦塗地。賊將崔綱等同惡相扇,至死不回,咸宜伏辜,以正刑典。劉
 闢男超郎等九人,並處斬。」闢入京城,上御興安樓受俘馘,令中使於樓下詰闢反狀。闢曰:「臣不敢反,五院子弟為惡,臣不能制。」又遣詰之曰:「朕遣中使送旌節官告,何故不受?」闢乃伏罪。令獻太廟、郊社,徇於市,即日戮於子城西南隅。



 初,闢嘗病,見諸問疾者來,皆以手據地,倒行入闢口,闢因礫裂食之;惟盧文若至,則如平常。故尤與文若厚,竟以同惡俱赤族,不其怪歟!



 張建封,字本立,兗州人。祖仁範,洪州南昌縣令,貞元初
 贈鄭州刺史。父玠,少豪俠,輕財重士。安祿山反,令偽將李庭偉率蕃兵脅下城邑,至魯郡;太守韓擇木具禮郊迎,置於郵館。玠率鄉豪張貴、孫邑、段絳等集兵將殺之。擇木怯懦,大懼;唯員外司兵張孚然其計,遂殺庭偉並其黨數十人,擇木方遣使奏聞。擇木、張孚俱受官賞,玠因游蕩江南,不言其功。以建封貴,贈秘書監。



 建封少頗屬文,好談論,慷慨負氣,以功名為己任。寶應中,李光弼鎮河南,時蘇、常等州草賊,寇掠郡邑,代宗遣中使馬日
 新與光弼將兵馬同征討之。建封乃見日新,自請說喻賊徒。日新從之,遂入虎窟、蒸裡等賊營,以利害禍福喻之。一夕,賊黨數千人並詣日新請降,遂悉放歸田里。



 大歷初,道州刺史裴虯薦建封於觀察使韋之晉,闢為參謀,奏授左清道兵曹,不樂吏役而去。滑亳節度使令狐彰聞其名,闢之;彰既未曾朝覲,建封心不悅之,遂投刺於轉運使劉晏,自述其志,不願仕於彰也。晏奏試大理評事,勾當軍務。歲餘,復罷歸。



 建封素與馬燧友善,大歷
 十年,燧為河陽三城鎮遏使,闢為判官,奏授監察御史,賜緋魚袋。李靈曜反於梁、宋間,與田悅掎角,同為叛逆,燧與李忠臣同討平之,軍務多咨於建封。及燧為河東節度使,復奏建封為判官,特拜侍御史。建中初,燧薦之於朝,楊炎將用為度支郎中,盧杞惡之,出為岳州刺史。



 時淮西節度使李希烈乘破滅梁崇義之勢,漸縱恣跋扈,壽州刺史崔昭數書疏往來。淮南節度使陳少游奏之,上遽召宰相令選壽州刺史。盧杞本惡建封,是日蒼
 黃,遂薦建封以代崔昭牧壽陽。李希烈稱兵,寇陷汝州,擒李元平,擊走胡德信、唐漢臣等,又摧破哥舒曜於襄城,連陷鄭、汴等州,李勉棄城而遁。涇師內逆,駕幸奉天,賊鋒益盛。淮南陳少游潛通希烈,尋稱偽號,改元,遣將楊豐齎偽赦書二道,令送少游及建封。至壽州,建封縛楊豐徇於軍中。適會中使自行在及使江南回者同至,建封集眾對中使斬豐於通衢,封偽赦書送行在,遠近震駭。陳少游聞之,既怒且懼。建封乃具奏少游與希烈
 往來事狀。希烈又偽署其黨杜少誠為淮南節度使,令先平壽州,趣江都。建封令其將賀蘭元均、邵怡等守霍丘秋柵。少誠竟不能侵軼,乃南掠蘄、黃等州,又為伊慎所挫衄。尋加建封兼御史中丞、本州團練使。車駕還京,陳少游憂憤而卒。



 興元元年十二月,乃加兼御史大夫,充濠壽廬三州都團練觀察使。於是大修緝城池,悉心綏撫,遠近悅附,自是威望益重。李希烈選兇黨精悍者率勁卒以攻建封,曠日持久,無所克獲而去。及希烈平,
 進階封,賜一子正員官。



 初,建中年,李涓以徐州歸附。涓尋卒,其後高承宗父子、獨孤華相繼為刺史。為賊侵削,貧困不能自存;又咽喉要地,據江淮運路,朝廷思擇重臣以鎮者久之。貞元四年,以建封為徐州刺史,兼御史大夫、徐泗濠節度、支度營田觀察使。既創置軍伍,建封觸事躬親;性寬厚,容納人過誤,而按據綱紀,不妄曲法貸人。每言事,忠義感激,人皆畏悅。七年,進位檢校禮部尚書。十二年,加檢校右僕射。十三年冬,入覲京師,德宗
 禮遇加等,特以雙日開延英召對,又令朝參入大夫班,以示殊寵。建封賦《朝天行》一章上獻,賜名馬珍玩頗厚。



 時宦者主宮中市買,謂之宮市,抑買人物,稍不如本估。末年不復行文書,置白望數十百人於兩市及要鬧坊曲,閱人所賣物;但稱宮市,則斂手付與,真偽不復可辨,無敢問所從來及論價之高下者。率用直百錢物買人直數千物,仍索進奉門戶及腳價銀。人將物詣市,至有空手而歸者,名為宮市,其實奪之。嘗有農夫以驢馱柴,
 宦者市之,與絹數尺,又就索門戶,仍邀驢送柴至內。農夫啼泣,以所得絹與之,不肯受,曰:「須得爾驢。」農夫曰:「我有父母妻子,待此而後食;今與汝柴,而不取直而歸,汝尚不肯,我有死而已。」遂毆宦者。街使擒之以聞,乃黜宦者,賜農夫絹十匹。然宮市不為之改,諫宮御史表疏論列,皆不聽。吳湊以戚里為京兆尹,深言其弊。建封入覲,具奏之,德宗頗深嘉納;而戶部侍郎、判度支蘇弁希宦者之旨,因入奏事,上問之,弁對曰:「京師游手墮業者數
 千萬家,無土著生業,仰宮市取給。」上信之,凡言宮市者皆不聽用。詔書矜免百姓諸色逋賦,上問建封,對曰:「凡逋賦殘欠,皆是累積年月,無可徵收,雖蒙陛下憂恤,百姓亦無所裨益。」時河東節度使李說、華州刺史盧微,皆中風疾,口不能言,足不能行,但信任左右胥吏決遣之。建封皆悉聞奏,上深嘉納。又金吾大將軍李翰好伺察城中細事,加諸聞奏,冀求恩寵,人畏而惡之。建封亦奏之,乃下詔曰:「比來朝官或諸處過從,金吾皆有上聞。其
 間如素是親故,或曾同僚友,伏臘歲序,時有還往,亦是常禮,人情所通。自今以後,金吾不須聞。」



 十四年春上巳,賜宰臣百僚宴於曲江亭,特令建封與宰相同座而食。貞元已後,籓帥入朝及還鎮,如馬燧、渾瑊、劉玄佐、李抱真、曲環之崇秩鴻勛,未有獲禦制詩以送者,建封將還鎮,特賜詩曰:「牧守寄所重,才賢生為時。宣風自淮甸,授鉞膺籓維。入覲展遐戀,臨軒慰來思。忠誠在方寸,感激陳清詞。報國爾所尚,恤人予是資。歡宴不盡懷,車馬當
 還期。穀雨將應候,行春猶未遲。勿以千里遙,而云無已知。」又令高品中使齎常所執鞭以賜之,曰:「以卿忠貞節義,歲寒不移,此鞭朕久執用,故以賜卿,表卿忠節也。」建封又獻詩一篇,以自警勵。



 建封在彭城十年,軍州稱理。復又禮賢下士,無賢不肖,游其門者,皆禮遇之,天下名士向風延頸,其往如歸。貞元時,文人如許孟容、韓愈諸公,皆為之從事。



 十六年,遇疾,連上表請速除代,方用韋夏卿為徐泗行軍司馬。未至而建封卒,時年六十六,冊
 贈司徒。子愔。



 愔以廕授虢州參軍。初,建封卒,判官鄭通誠權知留後事。通誠懼軍士謀亂,適遇浙西兵遷鎮,通誠欲引入州城為援。事洩,三軍怒,五六千人斫甲仗庫取戈甲,執帶環繞衙城,請愔為留後。乃殺通誠、楊德宗、大將段伯熊、吉遂、曲澄、張秀等。軍眾請於朝廷,乞授愔旄節。初不之許,乃割濠、泗二州隸淮南,加杜佑同平章事以討徐州。既而泗州刺史張伾以兵攻埇橋,與徐軍接戰,伾大敗而還。朝廷不獲已,乃授愔起復右驍衛將
 軍同正,兼徐州刺史、御史中丞,充本州團練使,知徐州留後。仍以泗州刺史張伾為泗州留後,濠州刺史杜兼為濠州留後。正授武寧軍節度、檢校工部尚書。元和元年,被疾,上表請代,徵為兵部尚書,以東都留守王紹為武寧軍節度代愔,復隸濠、泗二州於徐。徐軍喜復得二州,不敢為亂,而愔遂赴京師,未出界卒。愔在徐州七年,百姓稱理,詔贈右僕射。



 盧群,字載初,範陽人。少好讀書,初學於太安山。淮南節
 度使陳少游聞其名,闢為從事。建中末,薦於朝廷,會李希烈反叛,詔諸將討之。以群為監察御史、江西行營糧料使。興元元年,江西節度、嗣曹王皋奏為判官。曹王移鎮江陵、襄陽,群皆從之,幕府之事,委以咨決,以正直聞。



 貞元六年,入拜侍御史。有人誣告故尚父子儀嬖人張氏宅中有寶玉者,張氏兄弟又與尚父家子孫相告訴,詔促按其獄。群奏曰:「張氏以子儀在時分財,子弟不合爭奪。然張氏宅與子儀親仁宅,皆子儀家事。子儀有大
 勛,伏望陛下特赦而勿問,俾私自引退。」德宗從其言,時人嘉其識大體。累轉左司、職方、兵部三員外郎中。



 淮西節度使吳少誠擅開決司、洧等水漕輓溉田,遣中使止之,少誠不奉詔。令群使蔡州詰之,少誠曰:「開大渠,大利於人。」群曰:「為臣之道,不合自專,雖便於人,須俟君命。且人臣須以恭恪為事,若事君不盡恭恪,即責下吏恭恪,固亦難矣。」凡數百千言,諭以君臣之分,忠順之義,少誠乃從命,即停工役。



 群博涉,有口辨,好談論,與少誠言
 古今成敗之事,無不聳聽。又與唱和賦詩,自言以反側,常蒙隔在恩外,群於筵中醉而歌曰:「祥瑞不在鳳凰、麒麟,太平須得邊將、忠臣。衛、霍真誠奉主,貔虎十萬一身。江、河潛注息浪,蠻貊款塞無塵。但得百僚師長肝膽,不用三軍羅綺金銀。」少誠大感悅。群以奉使稱旨,俄遷檢校秘書監,兼御史中丞、義成軍節度行軍司馬。



 貞元十六年四月,節度姚南仲歸朝,拜群義成軍節度、鄭滑觀察等使。先寓居鄭州,典質良田數頃;及為節度使至鎮,各
 與本地契書,分付所管令長,令召還本主,時論稱美。尋遇疾,其年十月卒,時年五十九,廢朝一日,贈工部尚書,賵賻布帛、米粟有差。



 史臣曰:韋南康、張徐州,慷慨下位之中,橫身喪亂之際,力扶衰運,氣激壯圖,義風凜凜,聳動群醜,舂盜之喉,折賊之角,可謂忠矣!而韋公季年,惑賊闢之奸說,欲兼巴、益,則志未可量。徐州請覲,頗有規諫之言,所謂以道匡君,能以功名始終者。盧載初喻少誠,還地券,君子哉!三
 子之賢,不可多得。



 贊曰:南康英壯,力匡交喪。張侯義烈,志平亂象。見危能振,蹈利無謗。韋德不周,張心可亮。



\end{pinyinscope}