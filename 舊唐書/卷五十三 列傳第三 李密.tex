\article{卷五十三 列傳第三 李密}

\begin{pinyinscope}

 ○李密



 李密,字玄邃,本遼東襄平人。魏司徒弼曾孫,
 後周賜弼姓徒何氏。祖曜,周太保、魏國公;父寬,隋上柱國、蒲山公,皆知名當代。徙為京兆長安人。密以父廕為左親侍,嘗在仗下,煬帝顧見之,退謂許公宇文述曰:「向者左仗下黑色小兒為誰?」許公對曰:「故蒲山公李寬子密也。」帝曰:「個小兒視瞻異常,勿令宿衛。」他日,述謂密曰:「弟聰令如此,當以才學取官,三衛叢脞,非養賢之所。」密大喜,因謝病,專以讀
 書為
 事,
 時人希見其面。嘗欲尋包愷,乘一黃牛,被以蒲韉,仍將《漢書》一帙掛於角上,一手捉牛靷,一手翻卷書讀之。尚書令、越國公楊素見於道,從後按轡躡之,既及,問曰:「何處書生,耽學若此?」密識越公,乃下牛
 再拜,自言姓名。又問所讀書,答曰《項羽傳》。越公奇之,與語,大悅,謂其子玄感等曰:「吾觀李密識度,汝等不及。」於是玄感傾心結托。



 大業九年,煬帝伐高麗,使玄感於黎陽監運。時天下騷動,玄感將謀舉兵,潛遣人入關迎密,以為謀主。密至,謂玄感曰:「今天子出征,遠在遼外,地去幽州,懸隔千里,南有巨海之限,北有胡戎之患,中間一道,理極艱危。今公擁兵出其不意,長驅入薊,直扼其喉。前有高麗,退無歸路,不過旬朔,齎糧必盡。舉麾一召,其
 眾自降,不戰而擒,此計之上也。關中四塞,天府之國,有衛文升,不足為意。若經城勿攻,西入長安,掩其無備,天子雖還,失其襟帶。據險臨之,固當必克,萬全之勢,此計之中也。若隨近逐便,先向東都,頓堅城之下,勝負殊未可知,此計之下也。」玄感曰:「公之下計,乃上策也。今百官家口,並在東都,若不取之,安能動物?且經城不拔,何以示威?」密計遂不行。玄感既至東都,頻戰皆捷,自謂天下響應,功在朝夕。及獲內史舍人韋福嗣,又委以腹心,是
 以軍旅之事,不專歸密。福嗣既非同謀,因戰被執,每設籌畫,皆持兩端。玄感后使作檄文,福嗣固辭不肯,密揣其情,因謂玄感曰:「福嗣既非同盟,實懷觀望。明公初起大事,而奸人在側,必為所誤,請斬之以謝眾,方可安輯。」玄感曰:「何至於此!」密知言之不用,退謂所親曰:「楚公好反而不圖勝,如何?吾屬今為虜矣!」後玄感將西入,福嗣竟亡歸東都。



 隋左武衛大將軍李子雄坐事被收,系送行在所,於路殺使者,亡投玄感,乃勸玄感速稱尊號。玄
 感問於密,密曰:「昔陳勝自欲稱王,張耳諫而被外;魏武將求九錫,荀彧止而見疏。今者密若正言,還恐追蹤二子;阿諛順意,又非密之本圖。何者?兵起已來,雖復頻捷,至於郡縣,未有從者。東都守禦尚強,天下救兵益至。公當身先士眾,早定關中,乃欲急自尊崇,何示人不廣也!」玄感笑而止。及隋將宇文述、來護兒等率軍且至,玄感謂曰:「計將安出?」密曰:「元弘嗣統強兵於隴右,今可陽言其反,遣使迎公,因此入關,可得紿眾。」因引軍西入。至陜
 縣,欲圍弘農宮,密諫之曰:「公今詐眾西入,事宜在速,況乃追兵將至,安可稽留!若前不得據關,退無所守,大眾一散,何以自全?」玄感不從,遂圍之,三日不拔,方引而西。至於晙鄉,追兵遂及,玄感敗。密乃間行入關,為捕者所獲。



 時煬帝在高陽,密與其黨俱送帝所,謂其徒曰:「吾等之命,同於朝露,若至高陽,必為俎醢。今在道中,猶可為計,安得行就鼎鑊,不規逃避也!」眾然之。其多有金者,密令出示使者曰:「吾等死日,幸用相瘞,其餘即皆報德。」使
 者利其金,許之。及出關外,防禁漸弛,密請市酒食,每夜宴飲,喧嘩竟夕,使者不以為意。行至邯鄲,密等七人穿墻而遁。抵平原賊帥郝孝德,孝德不甚禮之。密又舍去,詣淮陽,隱姓名,自稱劉智遠,聚徒教授。經數月,鬱鬱不得志,為五言詩曰:「金風蕩初節,玉露凋晚林。此夕窮途士,鬱陶傷寸心。野平葭葦合,村荒藜藿深。眺聽良多感,徙倚獨沾襟。沾襟何所為?悵然懷古意。秦俗猶未平,漢道將何冀?樊噲市井徒,蕭何刀筆吏。一朝時運會,千古
 傳名謚。寄言世上雄,虛生真可愧。」詩成而泣下數行。時人有怪之者,以告太守趙佗,下縣捕之,密又亡去。會東郡賊帥翟讓聚黨萬餘人,密往歸之。或有知密是玄感亡將,潛勸讓害之,讓囚密於營外。密因王伯當以策於讓曰:「當今主昏於上,人怨於下,銳兵盡於遼東,和親絕於突厥,方乃巡游揚、越,委棄京都,此亦劉、項奮起之會,以足下之雄才大略,士馬精勇,席卷二京,誅暴滅虐,則隋氏之不足亡也。」讓深加敬慕,遽釋之。遣說諸小賊,所
 至皆降。密又說讓曰:「今兵眾既多,糧無所出,若曠日持久,則人馬困弊,大敵一臨,死亡無日矣!未若直取滎陽,休兵館穀,待士勇馬肥,然後與人爭利。」讓以為然。自是破金堤關,掠滎陽諸縣城堡,多下之。滎陽太宗楊慶及通守張須陀以兵討讓,讓曾為須陀所敗,聞其來,大懼,將遠避之。密曰:「須陀勇而無謀,兵又驟勝,既驕且狠,可一戰而擒之。公但列陣以待,為公破之。」讓不得已,勒兵將戰,密分兵千餘人於木林間設伏。讓與戰不利,稍卻,
 密發伏自後掩之,須陀眾潰,與讓合擊,大破之,遂斬須陀於陣。讓於是令密別統所部。密軍陣整肅,凡號令兵士,雖盛夏皆若背負霜雪。躬服儉素,所得金寶皆頒賜麾下,由是人為之用。尋復說讓曰:「昏主蒙塵,播蕩吳、越,群兵競起,海內饑荒。明公以英傑之才,而統驍雄之旅,宜當廓清天下,誅剪群兇,豈可求食草間,常為小盜而已!今東都士庶,中外離心,留守諸官,政令不一。明公親率大眾,直掩興洛倉,發粟以賑窮乏,遠近孰不歸附?百
 萬之眾,一朝可集,先發制人,此機不可失也!」讓曰:「僕起隴畝之間,望不至此,必如所圖,請君先發,僕領諸軍便為後殿。得倉之日,當別議之。」大業十三年春,密與讓領精兵千人出陽城北,逾方山,自羅口襲興洛倉,破之。開倉恣人所取,老弱襁負,道路不絕,眾至數十萬。隋越王侗遣虎賁郎將劉長恭率步騎二萬五千討密,密一戰破之,長恭僅以身免。讓於是推密為主,號為魏公。二月,於鞏南設壇場,即位,稱元年,其文書行下稱行軍元帥
 魏公府。以房彥藻為左長史,邴元真為右長史,楊得方為左司馬,鄭德韜為右司馬。拜翟讓為司徒,封東郡公。單雄信為左武候大將軍,徐世勣為右武候大將軍,祖君彥為記室,其餘封拜各有差。於是城洛口周回四十里以居之。



 長白山賊孟讓率所部歸密,鞏縣長柴孝和、侍御史鄭頤以鞏縣降密。隋虎賁郎將裴仁基率其子行儼以武牢歸密,拜為上柱國,封河東郡公。因遣仁基與孟讓率兵三萬餘人襲回洛倉,破之,入東都,俘掠居
 人,燒天津,東都出兵乘之,仁基等大敗,僅以身免。密復親率兵三萬逼東都,將軍段達、虎賁郎將高毗、劉長林等出兵七萬拒之,戰於故都城,隋軍敗走。密復下回洛倉而據之,大修營塹,以逼東都,仍作書以移郡縣曰:



 自元氣肇闢,厥初生人,樹之帝王,以為司牧。是以羲、農、軒、頊之後,堯、舜、禹、湯之君,靡不祗畏上玄,愛育黔首,乾乾終日,翼翼小心,馭朽索而同危,履春冰而是懼。故一物失所,若納隍而愧之;一夫有罪,遂下車而泣之。謙德軫
 於責躬,憂勞切於罪己。普天之下,率土之濱,蟠木距於流沙,瀚海窮於丹穴,莫不鼓腹擊壤,鑿井耕田,治致升平,驅之仁壽。是以愛之如父母,敬之若神明,用能享國多年,祚延長世。未有暴虐臨人,克終天位者也。



 隋氏往因周末,預奉綴衣,狐媚而圖聖寶,胠篋以取神器。及纘承負扆,狼虎其心,始曀明兩之暉,終乾少陽之位。先皇大漸,侍疾禁中,遂為梟獍,便行鴆毒。禍深於莒僕,釁酷於商臣,天地難容,人神嗟憤!州籲安忍,閼伯日尋,劍閣
 所以懷兇,晉陽所以興亂,甸人為罄,淫刑斯逞。夫九族既睦,唐帝闡其欽明;百世本枝,文王表其光大。況復隳壞盤石,剿絕維城,脣亡齒寒,寧止虞、虢?欲其長久,其可得乎!其罪一也。



 禽獸之行,在於聚麀,人倫之體,別於內外。而蘭陵公主逼幸告終,誰謂敤首之賢,翻見齊襄之恥。逮於先皇嬪御,並進銀環;諸王子女,咸貯金屋。牝雞鳴於詰旦,雄雉恣其群飛,衵衣戲陳侯之朝,穹廬同冒頓之寢。爵賞之出,女謁遂成,公卿宣淫,無復綱紀。其罪
 二也。



 平章百姓,一日萬機,未曉求衣,昃晷不食。大禹不貴於尺壁,光武不隔於支體,以是憂勤,深慮幽枉。而荒湎於酒,俾晝作夜,式號且呼,甘嗜聲伎,常居窟室,每藉糟丘。朝謁罕見其身,群臣希睹其面,斷決自此不行,敷奏於是停擁。中山千日之飲,酩酊無名;襄陽三雅之杯,留連詎比?又廣召良家,充選宮掖,潛為九市,親駕四驢,自比商人,見要逆旅。殷辛之譴為小,漢靈之罪更輕,內外驚心,遐邇失望。其罪三也。



 上棟下宇,著在《易》爻;茅茨
 採椽,陳諸史籍。聖人本意,惟避風雨,詎待硃玉之華,寧須綈錦之麗!故璇室崇構,商辛以之滅亡;阿房崛起,二世是以傾覆。而不遵古典,不念前章,廣立池臺,多營宮觀,金鋪玉戶,青瑣丹墀,蔽虧日月,隔閡寒暑。窮生人之筋力,罄天下之資財,使鬼尚難為之,勞人固其不可。其罪四也。



 公田所徹,不過十畝;人力所供,才止三日。是以輕徭薄賦,不奪農時,寧積於人,無藏於府。而科稅繁猥,不知紀極;猛火屢燒,漏卮難滿。頭會箕斂,逆折十年之
 租;杼軸其空,日損千金之費。父母不保其赤子,夫妻相棄於匡床。萬戶則城郭空虛,千里則煙火斷滅。西蜀王孫之室,翻同原憲之貧;東海糜竺之家,俄成鄧通之鬼。其罪五也。



 古先哲王,卜征巡狩,唐、虞五載,周則一紀。本欲親問疾苦,觀省風謠,乃復廣積薪芻,多備饔餼。年年歷覽,處處登臨,從臣疲弊,供頓辛苦。飄風凍雨,聊竊比於先驅;車轍馬跡,遂周行於天下。秦皇之心未已,周穆之意難窮。宴西母而歌云,浮東海而觀日。家苦納秸之
 勤,人阻來蘇之望。且夫天下有道,守在海外,夷不亂華,在德非險。長城之役,戰國所為,乃是狙詐之風,非關稽古之法。而追蹤秦代,板築更興,襲其基墟,延袤萬里,尸骸蔽野,血流成河,積怨滿於山川,號哭動於天地。其罪六也。



 遼水之東,朝鮮之地,《禹貢》以為荒服,周王棄而不臣,示以羈縻,達其聲教,茍欲愛人,非求拓土。又強弩末矢,理無穿於魯縞;沖風餘力,詎能動於鴻毛?石田得而無堪,雞肋啖而何用?而恃眾怙力,強兵黷武,惟在並吞,
 不思長策。夫兵,猶火也;不戢,將自焚,遂令億兆夷人,只輪莫返。夫差喪國,實為黃池之盟;苻堅滅身,良由壽春之役。欲捕鳴蟬於前,不知挾彈在後。復矢相顧,髽而成行,義夫切齒,壯士扼腕。其罪七也。



 直言啟沃,王臣匪躬,惟木從繩,若金須礪。唐堯建鼓,思聞獻替之言;夏禹懸鞀,時聽箴規之美。而愎諫違卜,蠹賢嫉能,直士正人,皆由屠害。左僕射、齊國公高穎,上柱國、宋國公賀若弼,或文昌上相,或細柳功臣,暫吐良藥之言,翻加屬鏤之賜。
 龍逢無罪,便遭夏癸之誅;王子何辜?濫被商辛之戮。遂令君子結舌,賢人緘口。指白日而比盛,射蒼天而敢欺,不悟國之將亡,不知死之將至。其罪八也。



 設官分職,貴在銓衡;察獄問刑,無聞販鬻。而錢神起論,銅臭為公,梁冀受黃金之蛇,孟佗薦蒲萄之酒。遂使彞倫攸篸,政以賄成,君子在野,小人在位。積薪居上,同汲黯之言;囊錢不如,傷趙壹之賦。其罪九也。



 宣尼有言,無信不立,用命賞祖,義豈食言?自昏主嗣位,每歲行幸,南北巡狩,東西
 征伐。至如浩亹陪蹕,東都守固,閿鄉野戰,雁門解圍。自外征夫,不可勝紀。既立功勛,須酬官爵。而志懷翻覆,言行浮詭,危急則勛賞懸授,克定則絲綸不行,異商鞅之頒金,同項王之剚印。芳餌之下,必有懸魚,惜其重賞,求人死力,走丸逆坡,匹此非難。凡百驍雄,誰不仇怨。至於匹夫蕞爾,宿諾不虧,既在乘輿,二三其德。其罪十也。



 有一於此,未或不亡。況四維不張,三靈總瘁,無小無大,愚夫愚婦,共識殷亡,咸知夏滅。罄南山之竹,書罪未窮;決
 東海之波,流惡難盡。是以窮奇災於上國,猰暴於中原。三河縱封豕之貪,四海被長蛇之毒,百姓殲亡,殆無遺類,十分為計,才一而已。蒼生懍懍,咸憂杞國之崩;赤子嗷嗷,但愁歷陽之陷。且國祚將改,必有常期,六百殷亡之年,三十姬終之世。故讖籙云:「隋氏三十六年而滅。」此則厭德之象已彰,代終之兆先見。皇天無親,惟德是輔。況乃攙搶竟天,申繻謂之除舊;歲星入井,甘公以為義興。兼硃雀門燒,正陽日蝕,狐鳴鬼哭,川竭山崩。並是
 宗廟為墟之妖,荊棘旅庭之事。夏氏則災釁非多,殷人則咎徵更少。牽牛入漢,方知大亂之期;王良策馬,始驗兵車之會。



 今者順人將革,先天不違,大誓孟津,陳命景亳,三千列國,八百諸侯,不謀而同辭,不召而自至。轟轟隱隱,如霆如雷,彪虎嘯而谷風生,應龍驤而景雲起。我魏公聰明神武,齊聖廣淵,總七德而在躬,包九功而挺出。周太保、魏公之孫,上柱國、蒲山公之子。家傳盛德,武王承季歷之基;地啟元勛,世祖嗣元皇之業。篤生白水,
 日角之相便彰;載誕丹陵,大寶之文斯著。加以姓符圖緯,名協歌謠,六合所以歸心,三靈所以改卜。文王厄於羑里,赤雀方來;高祖隱於碭山,彤雲自起。兵誅不道,《赤伏》至自長安;鋒銳難當,黃星出於梁、宋。九五龍飛之始,天人豹變之初,歷試諸難,大敵彌勇。上柱國、司徒、東郡公翟讓功宣締構,翼亮經綸,伊尹之佐成湯,蕭何之輔高帝。上柱國、總管、齊國公孟讓,柱國、歷城公孟暢,柱國、絳郡公裴行儼,大將軍、左長史邴元真等,並運籌千里,
 勇冠三軍,擊劍則截蛟斷鰲,彎弧則吟猿落雁。韓、彭、絳、灌,成沛公之基;寇、賈、吳、馮,奉蕭王之業。復有蒙輪挾輈之士,拔距投石之夫,驥馬追風,吳戈照日。魏公屬當期運,伏茲億兆。躬擐甲胄,跋涉山川,櫛風沐雨,豈辭勞倦,遂起西伯之師,將問南巢之罪。百萬成旅,四七為名,呼吸則河、渭絕流,叱吒則嵩、華自拔。以此攻城,何城不陷;以此擊陣,何陣不摧!譬猶瀉滄海而灌殘熒,舉昆侖而壓小卵。鼓行而進,百道俱前,以今月二十一日屆于東
 都。而昏朝文武、留守段達等,昆吾惡稔,飛廉奸佞,久迷天數,敢拒義兵,驅率醜徒,眾有十萬,回洛倉北,遂來舉斧。於是熊羆角逐,貔虎爭先,因其倒戈之心,乘我破竹之勢,曾未旋踵,瓦解冰銷,坑卒則長平未多,積甲則熊耳為小。達等助桀為虐,嬰城自固,梯沖亂舞,徒設九拒之謀;鼓角將鳴,空憑百樓之險。燕巢衛幕,魚游宋池,殄滅之期,匪朝伊暮。然興洛、虎牢,國家儲積,我已先據,為日久矣。既得回洛,又取黎陽,天下之倉,盡非隋有。四方
 起義,足食足兵,無前無敵。裴光祿仁基,雄才上將,受脤專征,遐邇攸憑,安危是托,乃識機知變,遷殷事夏。袁謙擒自藍水,張須陀獲在滎陽,竇慶戰沒於淮南,郭詢授首於河北,隋之亡候,聊可知也。清河公房彥藻,近秉戎律,略地東南,師之所臨,風行電擊。安陸、汝南,隨機蕩定;淮安、濟陽,俄然送款。徐圓朗已平魯郡,孟海公又破濟陽,海內英雄,咸來響應。封民贍取平原之境,郝孝德據黎陽之倉,李士雄虎視於長平,王德仁鷹揚於上黨。滑
 公李景、考功郎中房山基發自臨渝,劉興祖起於白朔,崔白駒在潁川起,方獻伯以譙郡來,各擁數萬之兵,俱期牧野之會。滄溟之右,函谷以東,牛酒獻於軍前,壺漿盈於道路。諸君等並衣冠世胄,杞梓良才,神鼎靈繹之秋,裂地封侯之始,豹變鵲起,今也其時,鼉鳴鱉應,見機而作,宜各鳩率子弟,共建功名。耿弇之赴光武,蕭何之奉高帝,豈止金章紫綬,華蓋硃輪,富貴以重當年,忠貞以傳奕葉,豈不盛哉!



 若隋代官人,同吠堯之犬,尚荷王
 莽之恩,仍懷蒯聵之祿。審配死於袁氏,不如張郃歸曹;範增困於項王,未若陳平從漢。魏公推以赤心,當加好爵,擇木而處,令不自疑。脫猛虎猶豫,舟中敵國,夙沙之人共縛其主,彭寵之僕自殺其君,高官上賞,即以相授。如暗於成事,守迷不反,昆山縱火,玉石俱焚,爾等噬臍,悔將何及!黃河帶地,明餘旦旦之言;皎日麗天,知我勤勤之意。布告海內,咸使聞知。



 祖君彥之辭也。



 俄而德韜、德方俱死,復以鄭頲為左司馬,鄭虔象為右司馬。柴孝
 和說密曰:「秦地阻山帶河,西楚背之而亡,漢高都之而霸。如愚意者,令仁基守回洛,翟讓守洛口,明公親簡精銳,西襲長安,百姓孰不郊迎,必當有征無戰。既克京邑,業固兵強,方更長驅崤函,掃蕩東洛,傳檄指捴,天下可定。但今英雄競起,實恐他人我先,一朝失之,噬臍何及!」密曰:「君之所圖,僕亦思之久矣,誠乃上策。但昏主尚存,從兵猶眾,我之所部,並是山東人,既見未下洛陽,何肯相隨西入?諸將出於群盜,留之各競雄雌。若然者,殆將
 敗矣!」密將兵鋒甚銳,每入苑與隋軍連戰。會密為流矢所中,臥於營內,東都復出兵乘之,密眾大潰,棄回洛倉,歸於洛口。煬帝遣王世充率勁卒五萬擊之,密與戰,不利,孝和溺死於洛水,密哭之甚慟。世充營於洛西,與密相拒百餘日,大小六十餘戰。武陽郡丞元寶藏、黎陽賊帥李文柏、洹水賊帥張升、清河賊帥趙君德、平原賊帥郝孝德,並歸於密,共襲破黎陽倉,據之。永安大族周法明舉江、黃之地以附密,齊郡賊帥徐圓朗、任城大俠徐
 師仁、淮陽太守趙佗皆歸之。



 翟讓部將王儒信勸讓為大塚宰,總統眾務,以奪密之權。讓兄寬復謂讓曰:「天子止可自作,安得與人!汝若不能作,我當為之。」密聞其言,陰有圖讓之計。會世充列陣而至,讓出拒之,為世充所擊,讓軍少失利,密與單雄信等率精銳赴之,世充敗走。明日,讓徑至密所,欲為宴樂,密具饌以待之,其所將左右,各分令就食。密引讓入坐,以良弓示讓,讓方引滿,密遣壯士自後斬之,並殺其兄寬及王儒信。讓部將徐世
 勣為亂兵所斫,中重瘡,密遽止之,得免,單雄信等頓首求哀,密並釋而慰諭之。於是詣讓連營,諭其將士,無敢動者。乃命徐世勣、單雄信、王伯當分統其眾。未幾,世充襲倉城,密復破之。世充復移營洛北,造浮橋,悉眾以擊密,密與千餘騎拒之,不利而退。世充因薄其城下,密簡銳卒數百人以邀之,世充大潰,爭趣浮橋,溺死者數萬。虎賁郎將楊威、王辯、霍舉、劉長恭、梁德、董智皆沒於陣,世充僅而獲免。其夜,大雨雪,士卒凍死者殆盡。密乘勝
 陷偃師,於是修金墉城居之,有眾三十餘萬。留守韋津又與密戰於上春門,津大敗,執於陣。將作大匠宇文愷叛東都,降於密。東至海、岱,南至江、淮郡縣,莫不遣使歸密。竇建德、硃粲、楊士林、孟海公、徐圓朗、盧祖尚、周法明等並隨使通表於密勸進,於是密下官屬咸勸密即尊號,密曰:「東都未平,不可議此。」



 及義旗建,密負其強盛,欲自為盟主,乃致書呼高祖為兄,請合從以滅隋,大略云欲與高祖為盟津之會,殪商辛於牧野,執子嬰於咸陽,
 其旨以弒後主執代王為意。高祖覽書笑曰:「李密陸梁放肆,不可以折簡致之。吾方安輯京師,未遑東討,即相阻絕,便是更生一秦。密今適所以為吾拒東都之兵,守成皋之扼,更求韓、彭,莫如用密。宜卑辭推獎,以驕其志,使其不虞於我。我得入關,據蒲津而屯永豐,阻崤函而臨伊、洛,吾大事濟矣。」令記室溫大雅作書報密曰:



 頃者,昆山火烈,海水群飛,赤縣丘墟,黔黎塗炭。布衣戎卒,鋤櫌棘矜,爭霸圖王,狐鳴蜂起。翼翼京洛,強弩圍城;AS
 幛周原,殭尸滿路。主上南巡,泛膠舟而忘返;匈奴北熾,將被發於伊川。輦上無虞,群下結舌,大盜移國,莫之敢指。忽焉至此,自貽伊戚,七百之基,窮於二世。周、齊以往,書契以還,邦國淪胥,未有如斯之酷者也。天生蒸民,必有司牧,當今為牧,非子而誰?老夫年餘知命,願不及此,欣戴大弟,攀鱗附翼。惟冀早應圖籙,以寧兆庶。宗盟之長,屬籍見容;復封於唐,斯榮足矣!殪商辛於牧野,所不忍言;執子嬰於咸陽,非敢聞命。汾、晉左右,尚須安輯,盟津
 之會,未暇卜期,今日鑾輿南幸,恐同永嘉之勢。顧此中原,鞠為茂草,興言感嘆,實疚於懷。脫知動靜,數遲貽報,未面靈襟,用增勞軫。名利之地,鋒鏑縱橫,深慎垂堂,勉茲鴻業。



 密得書甚悅,示其部下曰:「唐公見推,天下不足定也!」於是不虞義師而專意於世充。俄而宇文化及率眾自江都北指黎陽,兵十餘萬,密乃自將步騎二萬拒之。隋越王侗稱尊號,遣使授密太尉、尚書令、東南道大行臺行軍元帥、魏國公,令先平化及,然後入朝輔政。密
 將與化及相抗,恐前後受敵,因卑辭以報謝焉。化及至黎陽,與密相遇,密知其軍少食,利在急戰,故不與交鋒,又遏其歸路。密遣徐世勣守倉城,化及攻之不能下。密知化及糧且盡,因偽與和,以弊其眾。化及弗之悟,大喜,恣其兵食,冀密饋之。後知其計,化及怒,與密大戰於衛州之童山下,密為流矢所中,頓於汲縣。化及力竭糧盡,眾多叛之,掠汲縣,北趣魏縣。其將陳智略、張童仁等率所部兵歸於密者,前後相繼。初,化及留輜重於東郡,遣
 其所署刑部尚書王軌守之,至是軌舉郡降密。密引兵而西,遣使朝於東都,執弒煬帝人於弘達獻越王侗。侗召密入朝,至溫縣,聞世充作難而止,乃歸金墉城。



 時密兵少衣,世充兵乏食,乃請交易,密初難之,邴元真好求私利,屢勸密,密遂許焉。初,東都絕糧,兵士歸密者日有數百,至此得食,而降人益少,密方悔而止。密雖據倉而無府庫,兵數戰皆不獲賞,又厚撫初附之兵,由是眾心漸怨。武德元年九月,世充以其眾五千來決戰,密留王
 伯當守金墉,自引精兵就偃師,北阻邙山以待之。世充軍至,密遂敗績,裴仁基、祖君彥並為世充所虜,密與萬餘人馳向洛口。世充圍偃師,守將鄭頲之下兵士劫叛,以城降世充。密將入洛口倉城,邴元真已遣人潛引世充,密陰知之,不發其事,欲待世充兵半渡洛水,然後擊之。及世充軍至,密候騎不時覺,比將出戰,世充軍已濟矣。密自度不能支,引騎而遁,徑赴武牢,元真竟以城降於世充。



 密將如黎陽,或謂密曰:「殺翟讓之際,徐世勣幾
 至於死,今向其所,安可保乎?」時王伯當棄金墉,保河陽,密以輕騎自武牢歸之,謂伯當曰:「兵敗矣,久苦諸君!我今自刎,請以謝眾。」伯當抱密,號叫慟絕,眾皆泣,莫能仰視。密復曰:「諸軍幸不相棄,當共歸關中,密身雖愧無功,諸君必保富貴。」其府掾柳奭對曰:「昔盆子歸漢,尚食均輸。明公與唐公同族,兼有疇昔之遇,雖不陪從起義,然而阻東都,斷隋歸路,使唐公不戰而據京師,此亦公之功也。」眾咸曰:「然。」密又謂王伯當曰:「將軍室家重大,豈復
 與孤俱行哉!」伯當曰:「昔漢高誅項,蕭何率子弟以從,伯當恨不昆季盡從,以此為愧耳。豈以公今日失利,遂輕去就?縱身分原野,亦所甘心。」左右莫不感激,於是從入關者尚二萬人。高祖遣使迎勞,相望於道,密大喜,謂其徒曰「我有眾百萬,一朝至此,命也。今事敗歸國,幸蒙殊遇,當思竭忠以事所奉耳!且山東連城數百,知吾至此,遣使招之,盡當歸國。比於竇融,勛亦不細,豈不以一臺司見處乎?」及至京師,禮數益薄,執政者又來求賄,意甚
 不平。尋拜光祿卿,封邢國公。



 未幾,聞其所部將帥皆不附世充,高祖使密領本兵往黎陽,招集故時將士,經略世充。時王伯當為左武衛將軍,亦令為副。密行至桃林,高祖復徵之,密大懼,謀將叛。伯當頗止之,密不從,因謂密曰:「義士之立志也,不以存亡易心。伯當荷公恩禮,期以性命相報。公必不聽,今祗可同去,死生以之,然終恐無益也。」乃簡驍勇數千人,著婦人衣,戴幕離,藏刀裙下,詐為妻妾,自率之入桃林縣舍。須臾,變服突出,因據縣
 城,驅掠畜產,直趣南山,乘險而東,遣人馳告張善相,令以兵應接。時右翊衛將軍史萬寶留鎮熊州,遣副將盛彥師率步騎數千追躡,至陸渾縣南七十里,與密相及。彥師伏兵山谷,密軍半度,橫出擊,敗之,遂斬密,時年三十七。王伯當亦死之,與密俱傳首京師。時李勣為黎陽總管,高祖以勣舊經事密,遣使報其反狀。勣表請收葬,詔許之。高祖歸其尸,勣發喪行服,備君臣之禮。大具威儀,三軍皆縞素,葬於黎陽山南五里。故人哭之,多有歐
 血者。邴元真之降世充也,以為行臺僕射,鎮滑州。密故將杜才幹恨元真背密,詐與之會,伏甲斬之,以其首祭於密塚。



 單雄信者,曹州人也。翟讓與之友善。少驍健,尤能馬上用槍,密軍號為「飛將」。密偃師失利,遂降於王世充,署為大將軍。太宗圍逼東都,雄信出軍拒戰,援槍而至,幾及太宗,徐世勣呵止之,曰:「此秦王也。」雄信惶懼,遂退,太宗由是獲免。東都平,斬於洛陽。



 史臣曰:當隋政板蕩,煬帝荒淫,搖動中原,遠征遼海。內無賢臣以匡國,外乏良吏以理民,兩京空虛,兆庶疲弊。李密因民不忍,首為亂階,心斷機謀,身臨陣敵,據鞏、洛之口,號百萬之師,竇建德輩皆效樂推,唐公紿以欣戴,不亦偉哉!及偃師失律,猶存麾下數萬眾,茍去猜忌,疾趣黎陽,任世勣為將臣,信魏徵為謀主,成敗之勢,或未可知。至於天命有歸,大事已去,比陳涉有餘矣。始則稱首舉兵,終乃甘心為降虜,其為計也,不亦危乎!又不能
 委質為臣,竭誠事上,竟為叛者,終是狂夫,不取伯當之言,遂及桃林之禍。或以項羽擬之,文武器度即有餘,壯勇斷果則不及。楊素既知密之才幹,合為王之爪牙,委之癡兒,卒為謀主,覆族之禍,其宜也哉!



 贊曰:烏陽既升,爝火不息。狂哉李密,始亂終逆。



\end{pinyinscope}