\article{卷四十九 志第二十九 食貨下}

\begin{pinyinscope}

 武
 德八年十二月,水部郎中姜行本請於隴州開五節堰,引水通運,許之。永徽元年,薛大鼎為滄州刺史,界內有無棣河,隋末填廢。大鼎奏開之,引魚鹽於海。百姓歌
 之曰:「新河得通舟楫利,直達滄海魚鹽至。昔日徒行今騁駟,美哉薛公德滂被!」咸亨三年,關中饑,監察御史王師順奏請運晉、絳州倉粟以贍之。上委以運職。河、渭之間,舟楫相繼,會於渭南,自師順始之也。大足元年六月,於東都立德坊南穿新潭,安置諸州租船。神龍三年,滄州刺史姜師度於薊州之北,漲水為溝,以備奚、契丹之寇。又約舊渠,傍海穿漕,號為平虜渠,以避海難運糧。



 開元二年,河南尹李傑奏,汴州東有梁公堰,年久堰破,江
 淮曹運不通。發汴、鄭丁夫以浚之。省功速就,公私深以為利。十五年正月,令將作大匠範安及檢行鄭州河口斗門。先是,洛陽人劉宗器上言,請塞汜水舊汴河口,於下流滎澤界開梁公堰,置斗門,以通淮、汴,擢拜左衛率府胄曹。至是,新漕塞,行舟不通,貶宗器焉。安及遂發河南府、懷、鄭、汴、滑三萬人疏決開舊河口,旬日而畢。



 十八年,宣州刺史裴耀卿上便宜事條曰:「江南戶口稍廣,倉庫所資,惟出租庸,更無徵防。緣水陸遙遠,轉運艱辛,功
 力雖勞,倉儲不益。竊見每州所送租及庸調等,本州正二月上道,至揚州入斗門,即逢水淺,已有阻礙,須留一月已上。至四月已後,始渡淮入汴,多屬汴河乾淺,又般運停留,至六七月始至河口。即逢黃河水漲,不得入河。又須停一兩月,待河水小,始得上河。入洛即漕路乾淺,船艘隘鬧,般載停滯,備極艱辛。計從江南至東都,停滯日多,得行日少,糧食既皆不足,欠折因此而生。又江南百姓不習河水,皆轉雇河師水手,更為損費。伏見國家
 舊法,往代成規,擇制便宜,以垂長久。河口元置武牢倉,江南船不入黃河,即於倉內便貯。鞏縣置洛口倉,從黃河不入漕洛,即於倉內安置。爰及河陽倉、柏崖倉、太原倉、永豐倉、渭南倉,節級取便,例皆如此。水通則隨近運轉,不通即且納在倉,不滯遠船,不憂久耗,比於曠年長運,利便一倍有餘。今若且置武牢、洛口等倉,江南船至河口,即卻還本州,更得其船充運。並取所減腳錢,更運江淮變造義倉,每年剩得一二百萬石。即望數年之外,
 倉廩轉加。其江淮義倉,下濕不堪久貯,若無船可運,三兩年色變,即給貸費散,公私無益。」疏奏不省。至二十一年,耀卿為京兆尹,京師雨水害稼,穀價踴貴,玄宗以問耀卿,奏稱:「昔貞觀、永徽之際,祿廩未廣,每歲轉運,不過二十方石便足。今國用漸廣,漕運數倍,猶不能支。從都至陜,河路艱險,既用陸運,無由廣致。若能兼河漕,變陸為水,則所支有餘,動盈萬計。且江南租船,候水始進,吳人不便漕挽,由是所在停留。日月既淹,遂生竊盜。臣望
 於河口置一倉,納江東租米,便放船歸。從河口即分入河、洛,官自雇船載運。三門之東,置一倉。三門既水險,即於河岸開山,車運十數里。三門之西,又置一倉,每運至倉,即般下貯納。水通即運,水細便止。自太原倉溯河,更無停留,所省鉅萬。前漢都關中,年月稍久,及隋亦在京師,緣河皆有舊倉,所以國用常贍。」上深然其言。至二十二年八月,置河陰縣及河陰倉、河西柏崖倉、三門東集津倉、三門西鹽倉。開三門山十八里,以避湍險。自江淮
 而溯鴻溝,悉納河陰倉。自河陰送納含嘉倉,又送納太原倉,謂之北運。自太原倉浮於渭,以實關中。上大悅。尋以耀卿為黃門侍郎、同中書門下平章事,充江淮、河南轉運都使。以鄭州刺史崔希逸、河南少尹蕭炅為副。凡三年,運七百萬石,省陸運之傭四十萬貫。舊制,東都含嘉倉積江淮之米,載以大輿而西,至於陜三百里,率兩斛計傭錢千。此耀卿所省之數也。明年,耀卿拜侍中,而蕭炅代焉。二十五年,運米一百萬石。二十九年,陜郡太
 守李濟物,鑿三門山以通運,闢三門巔,逾巖險之地,俾負索引艦,升於安流,自齊物始也。



 天寶三載,韋堅代蕭炅,以滻水作廣運潭於望春樓之東,而藏舟焉。是年,楊釗以殿中侍御史為水陸運使,以代韋堅。先是,米至京師,或砂礫糠紕,雜乎其間。開元初,詔使揚擲而較其虛實,「揚擲」之名,自此始也。十四載八月,詔水陸運宜停一年。



 天寶以來,楊國忠、王鉷皆兼重使以權天下。肅宗初,第五琦始以錢穀得見。請於江、淮分置租庸使,市輕貨以
 救軍食,遂拜監察御史,為之使。乾元元年,加度支郎中,尋兼中丞,為鹽鐵使。於是始大鹽法,就山海井灶,收榷其鹽,立監院官吏。其舊業戶洎浮人欲以鹽為業者,免其雜役,隸鹽鐵使。常戶自租庸外無橫賦。人不益稅,而國用以饒。明年,琦以戶部侍郎同平章事,詔兵部侍郎呂諲代之。寶應元年五月,元載以中書侍郎代呂諲。是時淮、河阻兵,飛挽路絕,鹽鐵租賦,皆溯漢而上。以侍御史穆寧為河南道轉運租庸鹽鐵使,尋加戶部員外,遷
 鄂州刺史,以總東南貢賦。是時朝議以寇盜未戢,關東漕運,宜有倚辦,遂以通州刺史劉晏為戶部侍郎、京兆尹、度支鹽鐵轉運使。鹽鐵兼漕運,自晏始也。二年,拜吏部尚書、同平章事,依前充使。晏始以鹽利為漕傭,自江淮至渭橋,率十萬斛傭七千緡,補綱吏督之。不發丁男,不勞郡縣,蓋自古未之有也。自此歲運米數千萬石,自淮北列置巡院,搜擇能吏以主之,廣牢盆以來商賈。凡所制置,皆自晏始。廣德二年正月,復以第五琦專判度
 支鑄錢鹽鐵事。而晏以檢校戶部尚書為河南及江淮已來轉運使,及與河南副元帥計會開決汴河。永泰二年,晏為東道轉運常平鑄錢鹽鐵使,琦為關內、河東,劍南三川轉運常平鑄錢鹽鐵使。大歷五年,詔停關內、河東、三川轉運常平鹽鐵使。自此晏與戶部侍郎韓滉分領關內、河東、山、劍租庸青苗使。至十四年,天下財賦,皆以晏掌之。



 建中初,宰相楊炎用事,尤惡劉晏。炎乃奪其權。詔曰:「朕以征稅多門,郡邑凋耗,聽於群議,思有變更,
 將致時雍,宜遵古制。其江淮米準旨轉運入京者,及諸軍糧儲,宜令庫部郎中崔河圖權領之。今年夏稅以前,諸道財賦多輸京者,及鹽鐵財貨,委江州刺史包佶權領之。天下錢穀,皆歸金部、倉部。委中書門下簡兩司郎官,準格式條理。」尋貶晏為忠州刺史。晏既罷黜,天下錢穀歸尚書省。既而出納無所統,乃復置使領之。其年三月,以韓洄為戶部侍郎,判度支;金部郎中杜佑權勾當江淮水陸運使。炎尋殺晏於忠州。自兵興已來,兇荒相
 屬,京師米斛萬錢,官廚無兼時之食。百姓在畿甸者,拔谷挼穗,以供禁軍。洎晏掌國計,復江淮轉運之制,歲入米數十萬斛以濟關中。代第五琦領鹽務,其法益密。初年入錢六十萬,季年則十倍其初。大歷末,通天下之財,而計其所入,總一千二百萬貫,而鹽利過半。李靈耀之亂,河南皆為盜據,不奉法制,賦稅不上供,州縣益減。晏以羨餘相補,人不加賦,所入仍舊,議者稱之。其相與商榷財用之術者,必一時之選。故晏沒後二十年,韓洄、元
 琇、裴腆、包佶、盧徵、李衡相繼分掌財賦,出晏門下。屬吏在千里外,奉教如目前。四方水旱,及軍府纖芥,莫不先知焉。其年詔曰:「天下山澤之利,當歸王者,宜總榷鹽鐵使。」



 三年,以包佶為左庶子、汴東水陸運鹽鐵租庸使,崔縱為右庶子、汴西水陸運鹽鐵租庸使。四年,度支侍郎趙贊議常平事,竹、木、茶、漆盡稅之。茶之有稅,肇於此矣。貞元元年,元琇以御史大夫為鹽鐵水陸運使。其年七月,以尚書右僕射韓滉統之。滉歿,宰相竇參代之。五年
 十二月,度支轉運鹽鐵奏:「比年自揚子運米,皆分配緣路觀察使差長綱發遣。運路既遠,實謂勞人。今請當使諸院,自差綱節級般運,以救邊食。」從之。八年,詔:東南兩稅財賦,自河南、江淮、嶺南、山南東道至於渭橋,以戶部侍郎張滂主之;河東、劍南、山南西道,以戶部尚書度支使班宏主之。今戶部所領三川鹽鐵轉運,自此始也。其後宏、滂互有短長。宰相趙憬、陸贄以其事上聞,由是遵大歷故事,如劉晏、韓滉所分焉。



 九年,張滂奏立稅茶法。自後
 裴延齡專判度支,與鹽鐵益殊塗而理矣。十年,潤州刺史王緯代之,理於硃方。數年而李錡代之,鹽院津堰,改張侵剝,不知紀極。私路小堰,厚斂行人,多自錡始。時鹽鐵轉運有上都留後,以副使潘孟陽主之。王叔文權傾朝野,亦以鹽鐵副使兼學士為留後。



 順宗即位,有司重奏鹽法,以杜佑判鹽鐵轉運使,理於揚州。元和二年三月,以李巽代之。先是,李錡判使,天下榷酤漕運,由其操割,專事貢獻,牢其寵渥。中朝柄事者悉以利積於私室,
 而國用日耗。巽既為鹽鐵使,大正其事。其堰埭先隸浙西觀察使者,悉歸之;因循權置者,盡罷之;增置河陰敖倉;置桂陽監,鑄平陽銅山為錢。又奏:「江淮、河南、峽內、兗鄆、嶺南鹽法監院,去年收鹽價緡錢七百二十七萬,比舊法張其估一千七百八十餘萬,非實數也。今請以其數,除煮之外,付度支收其數。」鹽鐵使煮鹽利系度支,自此始也。又以程異為揚子留後。四月五日,巽卒。自榷筦之興,惟劉晏得其術,而巽次之。然初年之利,類晏之季
 年;季年之利,則三倍於晏矣。舊制,每歲運江淮米五十萬斛,至河陰留十萬,四十萬送渭倉。晏歿,久不登其數,惟巽秉使三載,無升斗之闕焉。六月,以河東節度使李鄘代之。



 五年,李鄘為淮南節度使,以宣州觀察使盧坦代之。六年,坦奏,每年江淮運米四十萬石到渭橋,近日欠闕太半,請旋收糴,遞年貯備。從之。坦改戶部侍郎,以京兆尹王播代之。播遂奏:「元和五年,江淮、河南、嶺南、峽中、兗鄆等鹽利錢六百九十八萬貫。比量改法已前舊
 鹽利,時價四倍虛估,即此錢為一千七百四十餘萬貫矣,請付度支收管。」從之。其年詔曰:「兩稅之法,悉委郡國,初極便人。但緣約法之時,不定物估。今度支鹽鐵,泉貨是司,各有分巡,置於都會。爰命帖職,周視四方,簡而易從,庶葉權便。政有所弊,事有所宜,皆得舉聞,副我憂寄。以揚子鹽鐵留後為江淮已南兩稅使,江陵留後為荊衡漢沔東界、彭蠡已南兩稅使,度支山南西道分巡院官充三川兩稅使。峽內煎鹽五監先屬鹽鐵使,今宜割
 屬度支,便委山南西道兩稅使兼知糶賣。」峽內鹽屬度支,自此始也。七年,王播奏去年鹽利除割峽內鹽,收錢六百八十五萬,從實估也。又奏,商人於戶部、度支、鹽鐵三司飛錢,謂之「便換」。八年,以崔倰為揚子留後、淮嶺已來兩稅使;崔祝為江陵留後,為荊南已來兩稅使。十三年正月,播又奏,以「軍興之時,財用是切。頃者劉晏領使,皆自按置租庸,至於州縣否臧,錢谷利病之物,虛實皆得而知。今臣守務在城,不得自往。請令臣副使程異出
 巡江淮,其州府上供錢穀,一切勘問。」從之。閏五月,異至江淮,得錢一百八十五萬貫以進。其年,以播守禮部尚書,以衛尉卿程異代之。十四年,異卒,以刑部侍郎柳公綽代之。長慶初,王播復代公綽。四年,王涯以戶部侍郎代播。敬宗初,播復以鹽鐵使為揚州節度使。文宗即位,入覲,以宰相判使。其後,王涯復判二使,表請使茶山之人移植根本,舊有貯積,皆使焚棄。天下怨之。九年,涯以事誅。而令狐楚以戶部尚書右僕射主之,以是年茶法
 大壞,奏請付州縣而入其租於戶部,人人悅焉。開成元年,李石以中書侍郎判收茶法,復貞元之制也。三年,以戶部尚書同平章事楊嗣復主之,多革前監院之陳事。開成三年至大中壬申,凡一十五年,多任以元臣,以集其務。崔珙自刑部尚書拜,杜忭以淮南節度領之,既而皆踐公臺。薛元賞、李執方、盧弘正、馬植、敬晦五人,於九年之中,相踵理之,植亦自是居相位。



 大中五年二月,以戶部侍郎裴休為鹽鐵轉運使。明年八月,以本官平章
 事,依前判使。始者,漕米歲四十萬斛,其能至渭倉者,十不三四。漕吏狡蠹,敗溺百端,官舟之沉,多者歲至七十餘只。緣河奸犯,大紊晏法。休使僚屬按之,委河次縣令董之。自江津達渭,以四十萬斛之傭,計緡二十八萬,悉使歸諸漕吏。巡院胥吏,無得侵牟。舉之為法,凡十事,奏之。六年五月,又立稅茶之法,凡十二條,陳奏。上大悅。詔曰:「裴休興利除害,深見奉公。」盡可其奏。由是三歲漕米至渭濱,積一百二十萬斛,無升合沉棄焉。



 武德元年九
 月四日,置社倉。其月二十二日詔曰:「特建農圃,本督耕耘,思俾齊民,既康且富。鐘庾之量,冀同水火。宜置常平監官,以均天下之貨。市肆騰踴,則減價而出;田穡豐羨,則增糴而收。庶使公私俱濟,家給人足,抑止兼並,宣通壅滯。」至五年十二月,廢常平監官。貞觀二年四月,尚書左丞戴胄上言曰:「水旱兇災,前聖之所不免。國無九年儲畜,《禮經》之所明誡。今喪亂之後,戶口凋殘,每歲納租,未實倉廩。隨時出給,才供當年,若有兇災,將何賑恤?故
 隋開皇立制,天下之人,節級輸粟,多為社倉,終於文皇,得無饑饉。及大業中年,國用不足,並貸社倉之物,以充官費,故至末塗,無以支給。今請自王公已下,爰及眾庶,計所墾田稼穡頃畝,至秋熟,準其見在苗以理勸課,盡令出粟。稻麥之鄉,亦同此稅。各納所在,為言義倉。若年穀不登,百姓饑饉,當所州縣,隨便取給。」太宗曰:「既為百姓預作儲貯,官為舉掌,以備兇年,非朕所須,橫生賦斂。利人之事,深是可嘉。宜下所司,議立條制。」戶部尚書韓
 仲良奏:「王公已下墾田,畝納二升。其粟麥粳稻之屬,各依土地。貯之州縣,以備兇年。」可之。自是天下州縣,始置義倉,每有饑饉,則開倉賑給。以至高宗、則天,數十年間,義倉不許雜用。其後公私窘迫,漸貸義倉支用。自中宗神龍之後,天下義倉費用向盡。



 高宗永徽二年六月,敕:「義倉據地收稅,實是勞煩。宜令率戶出粟,上上戶五石,餘各有差。」六年,京東西二市置常平倉。明慶二年十二月,京常平倉置常平署官員。開元二年九月,敕:「天下諸
 州,今年稍熟,穀價全賤,或慮傷農。常平之法,行之自古,宜令諸州加時價三兩錢糴,不得抑斂。仍交相付領,勿許懸欠。蠶麥時熟,穀米必貴,即令減價出糶。豆穀等堪貯者,熟亦準此。以時出入,務在利人。其常平所須錢物,宜令所司支料奏聞。」四年五月二十一日,詔:「諸州縣義倉,本備饑年賑給。近年已來,每三年一度,以百姓義倉糙米,遠赴京納,仍勒百姓私出腳錢。自今已後,更不得義倉變造。」七年六月,敕:「關內,隴右、河南、河北五道,及荊、
 揚、襄、夔、綿、益、彭、蜀,漢、劍、茂等州,並置常平倉。其本上州三千貫,中州二千貫,下州一千貫。」十六年十月,敕:「自今歲普熟,穀價至賤,必恐傷農。加錢收糴,以實倉廩,縱逢水旱,不慮阻饑,公私之間,或亦為便。宜令所在以常平本錢及當處物,各於時價上量加三錢,百姓有糶易者,為收糴。事須兩和,不得限數。配糴訖,具所用錢物及所糴物數,申所司。仍令上佐一人專勾當。」



 天寶六載三月,太府少卿張瑄奏:「準四載五月並五載三月敕節文,至
 貴時賤價出糶,賤時加價收糴。若百姓未辦錢物者,任準開元二十年七月敕,量事賒糶,至粟麥熟時徵納。臣使司商量,且糶舊糴新,不同別用。其賒糶者,至納錢日若粟麥雜種等時價甚賤,恐更回易艱辛,請加價便與折納。」廣德二年正月,第五琦奏:「每州常平倉及庫使司,商量置本錢,隨當處米物時價,賤則加價收糴,貴則減價糶賣。」



 建中元年七月,敕:「夫常平者,常使穀價如一,大豐不為之減,大儉不為之加。雖遇災荒,人無菜色。自今
 已後,忽米價貴時,宜量出官米十萬石,麥十萬石,每日量付兩市行人下價糶貨。」三年九月,戶部侍郎趙贊上言曰:「伏以舊制,置倉儲粟,名曰常平。軍興已來,此事闕廢,或因兇荒流散,餓死相食者,不可勝紀。古者平準之法,使萬室之邑,必有萬鐘之藏,千室之邑,必有千鐘之藏,春以奉耕,夏以奉耘,雖有大賈富家,不得豪奪吾人者,蓋謂能行輕重之法也。自陛下登極以來,許京城兩市置常平,官糴鹽米,雖經頻年少雨,米價未騰貴,此乃即
 自明驗,實要推而廣之。當軍興之時,與承平或異,事須兼儲布帛,以備時須。臣今商量,請於兩都並江陵、成都、揚、汴、蘇、洪等州府,各置常平,輕重本錢,上至百萬貫,下至數十萬貫,隨其所宜,量定多少。唯貯斛斗疋段絲麻等,候物貴則下價出賣,物賤則加價收糴。權其輕重,以利疲人。」從之。贊於是條奏諸道津要都會之所,皆置吏,閱商人財貨。計錢每貫稅二十,天下所出竹、木、茶、漆,皆十一稅之,以充常平本。時國用稍廣,常賦不足,所稅亦隨
 時而盡,終不能為常平本。



 貞元八年十月,敕:「諸軍鎮和糴貯備,共三十三萬石,價之外,更量與優饒。其粟及麻,據米數準折虛價,直委度支,以停江淮運腳錢充,並支綾絹、糸、綿,勿令折估。所糴粟等,委本道節度使監軍同勾當別貯,非承特敕,不得給用。」十四年六月,詔以米價稍貴,令度支出官米十萬石,於兩街賤糶。其年九月,以歲饑,出太倉粟三十萬石出糶。是歲冬,河南府穀貴人流,令以含嘉倉粟七萬石出糶。十五年二月,以久旱歲饑,
 出太倉粟十八萬石,於諸縣賤糶。元和元年正月,制:「歲時有豐歉,穀價有重輕,將備水旱之虞,在權聚斂之術。應天下州府每年所稅地子數內,宜十分取二分,均充常平倉及義倉,仍各逐穩便收貯,以時出糶,務在救人,賑貸所宜,速奏。」六年二月,制:「如聞京畿之內,舊穀已盡,宿麥未登,宜以常平、義倉粟二十四萬石貸借百姓。諸道州府有乏少糧種處,亦委所在官長,用常平、義倉米借貸。淮南、浙西、宣歙等道,元和二年四月賑貸,並且停
 征。容至豐年,然後填納。」九年四月,詔出太倉粟七十萬石,開六場糶之,並賑貸外縣百姓。至秋熟徵納,便於外縣收貯,以防水旱。十二年四月,詔出粟二十五萬石,分兩街降估出糶。其年九月,詔諸道應遭水州府,河中、澤潞、河東、幽州、江陵府等管內,及鄭、滑、滄、景、易、定、陳、許、晉、顯、蘇、襄、復、臺、越、唐、隨、鄧等州人戶,宜令本州厚加優恤。仍各以當處義倉斛斗,據所損多少,量事賑給。十三年正月,戶部侍郎孟簡奏:「天下州府常平、義倉等斛斗,請
 準舊例減估出糶,但以石數奏申,有司更不收管,州縣得專達以利百姓。」從之。



 長慶四年二月,敕出太倉陳粟三十萬石,於兩街出糶。其年三月制曰:「義倉之制,其來日久。近歲所在盜用沒入,致使小有水旱,生人坐委溝壑。永言其弊,職此之由。宜令諸州錄事參軍,專主勾當。茍為長吏迫制,即許驛表上聞。考滿之日,戶部差官交割。如無欠負,與減一選。如欠少者,量加一選。欠數過多,戶部奏聞,節級科處。」大和四年八月,敕:「今年秋稼似熟,
 宜於關內七州府及鳳翔府和糴一百萬石。」大中六年四月,戶部奏:「諸州府常平、義倉斛斗,本防水旱,賑貸百姓。其有災沴州府地遠,申奏往復,已至流亡。自今已後,諸道遭災旱,請委所在長吏,差清強官審勘,如實有水旱處,便任先從貧下不支濟戶給貸。」從之。



 建中四年六月,戶部侍郎趙贊請置大田:天下田計其頃畝,官收十分之一。擇其上腴,樹桑環之,曰公桑。自王公至於匹庶,差借其力,得穀絲以給國用。詔從其說。贊熟計之,自以
 為非便,皆寢不下。復請行常平稅茶之法。又以軍須迫蹙,常平利不時集,乃請稅屋間架、算除陌錢。間架法:凡屋兩架為一間,至有貴賤,約價三等,上價間出錢二千,中價一千,下價五百。所由吏秉算執籌,入人之廬舍而計其數。衣冠士族,或貧無他財,獨守故業,坐多屋出算者,動數十萬。人不勝其苦。凡沒一間者,仗六十,告者賞錢五十貫,取於其家。除陌法:天下公私給與貨易,率一貫舊算二十,益加算為五十。給與他物或兩換者,約錢
 為率算之。市牙各給印紙,人有買賣,隨自署記,翌日合算之。有自貿易不用市牙者,驗其私簿。無私簿者,投狀自集。其有隱錢百者沒入,二千杖六十,告者賞十千,取其家資。法既行,而主人市牙得專其柄,率多隱盜。公家所入,曾不得半,而怨惸之聲,囂然滿於天下。至興元二年正月一日赦,悉停罷。



 貞元九年正月,初稅茶。先是,諸道鹽鐵使張滂奏曰:「伏以去歲水災,詔令減稅。今之國用,須有供儲。伏請於出茶州縣,及茶山外商人要路,委
 所由定三等時估,每十稅一,充所放兩稅。其明年以後所得稅,外貯之。若諸州遭水旱,賦稅不辦,以此代之。」詔可之,仍委滂具處置條奏。自此每歲得錢四十萬貫。然稅無虛歲,遭水旱處亦未嘗以錢拯贍。



 大和七年,御史臺奏:「伏準大和三年十一月十八日赦文,天下除兩稅外,不得妄有科配,其擅加雜榷率,一切宜停,令御史臺嚴加察訪者。臣昨因嶺南道擅置竹綀場,稅法至重,害人頗深。伏請起今已後,應諸道自大和三年準赦文所
 停兩稅處科配雜榷率等復卻置者,仰敕至後十日內,具卻置事由聞奏,仍申臺司。每有出使郎官御史,便令嚴加察訪。茍有此色,本判官重加懲責,長吏奏聽進止。」從之。九年十二月,左僕射令狐楚奏新置榷茶使額:「伏以江淮間數年以來,水旱疾疫,凋傷頗甚,愁嘆未平。今夏及秋,稍較豐稔。方須惠恤,各使安存。昨者忽奏榷茶,實為蠹政。蓋是王涯破滅將至,怨怒合歸。豈有令百姓移茶樹就官場中栽,摘茶葉於官場中造?有同兒戲,不
 近人情。方有恩權,無敢沮議,朝班相顧而失色,道路以目而吞聲。今宗社降靈,奸兇盡戮,聖明垂佑,黎庶各安。微臣伏蒙天恩,兼授使務,官銜之內,猶帶此名,俯仰若驚,夙宵知愧。伏乞特回聖聽,下鑒愚誠,速委宰臣,除此使額。緣國家之用或闕,山澤之利有遺,許臣條流,續具奏聞。採造欲及,妨廢為虞。前月二十一日內殿奏封之次,鄭覃與臣同陳論訖。伏望聖慈早賜處分,一依舊法,不用新條。惟納榷之時,須節級加價,商人轉抬,必較
 稍貴,即是錢出萬國,利歸有司,既無害茶商,又不擾茶戶。上以彰陛下愛人之德,下以竭微臣憂國之心。遠近傳聞,必當咸悅。」詔可之。先是,鹽鐵使王涯表請使茶山之人,移植根本,舊有貯積,皆使焚棄,天下怨之。及是楚主之,故奏罷焉。



 開成二年十二月,武寧軍節度使薛元賞奏:「泗口稅場,應是經過衣冠商客金銀、羊馬、斛斗、見錢、茶鹽、綾絹等,一物已上並稅。今商量,其雜稅並請停絕。」詔許之。



 大中六年正月,鹽鐵轉運使裴休奏:「諸道節
 度、觀察使,置店停上茶商,每斤收搨地錢,並稅經過商人,頗乖法理。今請厘革橫稅,以通舟船,商旅既安,課利自厚。今又正稅茶商,多被私販茶人侵奪其利。今請強幹官吏,先於出茶山口,及廬、壽、淮南界內,布置把捉,曉諭招收,量加半稅,給陳首帖子,令其所在公行,從此通流,更無苛奪。所冀招恤窮困,下絕奸欺,使私販者免犯法之憂,正稅者無失利之嘆。欲尋究根本,須舉綱條。」敕旨依奏。其年四月,淮南及天平軍節度使並浙西觀察
 使,皆奏軍用困竭,伏乞且賜依舊稅茶。敕旨:「裴休條流茶法,事極精詳,制置之初,理須畫一。並宜準今年正月二十六日敕處分。」



 建中三年,初榷酒,天下悉令官釀。斛收直三千。米雖賤,不得減二千。委州縣綜領。醨薄私釀,罪有差。以京師王者都,特免其榷。元和六年六月,京兆府奏:「榷酒錢除出正酒戶外,一切隨兩稅青苗,據貫均率。」從之。會昌六年九月敕:「揚州等八道州府,置榷麴,並置官店沽酒,代百姓納榷酒錢,並充資助軍用,各有榷許
 限。揚州、陳許、汴州、襄州、河東五處榷麴,浙西、浙東、鄂岳三處置官沽酒。如聞禁止私酤,過於嚴酷,一人違犯,連累數家,閭里之間,不免咨怨。宜從今以後如有人私沽酒及置私麴者,但許罪止一身,並所由容縱,任據罪處分。鄉井之內,如不知情,並不得追擾。其所犯之人,任用重典,兼不得沒入家產。」



\end{pinyinscope}