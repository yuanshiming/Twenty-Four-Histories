\article{卷一百四十六}

\begin{pinyinscope}

 ○李寶臣子惟岳惟誠惟簡惟簡子元本王武俊子士真士平士則士真子承宗承元王廷湊子元逵元逵子紹鼎紹懿紹鼎子景崇景崇子鎔



 李寶臣,範陽城旁奚族也。故範陽將張金巢高之
 假子,故
 姓張,名忠志。幼善騎射,節度使安祿山選為射生官。天寶中,隨祿山入朝,玄宗留為射生子弟,出入禁中。及祿山叛,忠志遁歸範陽;祿山喜,錄為假子,姓安,常給事帳中。祿山兵將指闕,使忠志領驍騎八千人入太原,劫太原尹楊光翽。忠志挾光翽出太原,萬兵追之不敢近。祿山使董精甲,扼井陘路,軍於土門。安慶緒偽署為恆州刺史。九節度之師圍慶緒於相州,忠志懼,獻章歸國,肅宗因授恆州刺史。及史思明復渡河,偽授忠志工部尚
 書、恆州刺史、恆趙節度使,統眾三萬守常山。及思明敗,不受朝義之命,乃開土門路以內王師。河朔平定,忠志與李懷仙、薛嵩、田承嗣各舉其地歸國,皆賜鐵券,誓以不死。因授忠志開府儀同三司、檢校禮部尚書、恆州刺史,實封二百戶,仍舊為節度使。乃以恆州為成德軍,賜姓名曰李寶臣。



 時寶臣有恆、定、易、趙、深、冀六州之地,後又得滄州,步卒五萬、馬五千匹,當時勇冠河朔諸帥。寶臣以七州自給,軍用殷積,招集亡命之徒,繕閱兵仗,與
 薛嵩、田承嗣、李正己、梁崇義等連結姻婭,互為表裏,意在以土地傳付子孫,不稟朝旨,自補官吏,不輸王賦。初,天寶中,天下州郡皆鑄銅為玄宗真容,擬佛之制。及安、史之亂,賊之所部,悉熔毀之,而恆州獨存,由是實封百戶。



 初,寶臣、正己皆為承嗣所易。寶臣弟寶正娶承嗣女,在魏州與承嗣子維擊鞠,寶正馬馳駭,觸殺維。承嗣怒,縶寶正以告。寶臣謝為教不謹,緘杖令承嗣以示責,承嗣遂鞭殺之,由是交惡。



 大歷十年,寶臣、正己更言承嗣
 之罪,請討之。代宗欲因其相圖,乃從其請。時幽州節度留後硃滔方恭順朝廷,詔滔與寶臣及太原之師攻其北,正己與滑亳、河陽、江淮之師攻其南。寶臣、正己會軍於棗強,椎牛釃酒,犒勞將士,仍頒優賞。寶臣軍賞厚,正己軍賞薄。既罷會,正己軍中咄咄有辭,正己聞,懼有變,即時引退。由是寶臣、硃滔共攻承嗣之滄州,連年未下。時承嗣使腹心將盧子期攻邢州,城將陷,寶臣發精卒赴救,擊敗之,擒子期來獻。河南諸將又大破田悅於陳
 留,正己收承嗣之德州,以重兵臨其境,指期進討。承嗣大懾,遂求解於寶臣,寶臣不許。



 初,正己將發兵,使人至魏,承嗣囚之。及是,乃厚禮遣歸,發使與俱,具列境內戶口兵糧之數,悉以奉正己。且告曰:「承嗣老矣,今年八十有六,形體支離,無日月焉。己子不令,悅亦孱弱,不足保其後業。今之所有,為公守耳,曷足辱公師旅焉!」立使者於廷,南向,拜而授書。又圖正己形,焚香事之如神,謂人曰:「真聖人也!」正己聞之,且得其歡,乃止諸軍,莫敢進者。



 承嗣止正己,無南軍之虞。又知範陽寶臣故里,生長其間,心常欲得之;乃勒石為讖,密瘞寶臣境內,使望氣者云:「此中有玉氣。」寶臣掘地得之,有文曰:「二帝同功勢萬全,將田作伴入幽、燕。」二帝,指寶臣、正己也。承嗣又使客諷之曰:「公與硃滔共舉,取吾滄州,設得之,當歸國,非公所有。誠能舍承嗣之罪,請以滄州奉獻,可不勞師而致,願取範陽以自效。公將騎為前驅,承嗣率步卒從,此萬全之勢。」寶臣喜,以為事合符命,遂與承嗣通謀,割州與
 之。寶臣乃密圖範陽,承嗣亦陳兵境上。寶臣謂硃滔使曰:「吾聞硃公貌如神,安得而識之?願因繪事而觀,可乎?」滔乃圖其形以示之。寶臣懸於射堂,命諸將熟視之,曰:「硃公信神人也!」他日,滔出軍,寶臣密選精卒劫之,戒其將曰:「取彼貌如射堂所懸者。」是時,二軍不相虞有變,滔與戰於瓦橋。滔適衣他服,以不識免。承嗣聞與滔交鋒,其釁已成,乃旋軍,使告寶臣曰:「河內有警急,不暇從公。石上讖文,吾戲為之耳!」寶臣慚怒而退。



 遷左僕射,封隴
 西郡王、檢校司空、同中書門下平章事。德宗即位,拜司空,兼太子太傅。寶臣名位既高,自擅一方,專貯異志。妖人偽為讖語,言寶臣終有天位。寶臣乃為符瑞及靈芝硃草,作硃書符。又於深室齋戒築壇,上置金匜、玉斝,云「甘露神酒自出」。又偽刻玉為印,金填文字,告境內云:「天降靈瑞,非予所求,不祈而至。」將吏無敢言者。妖輩慮其詐發,乃曰:「相公須飲甘露湯,即天神降。」寶臣然之。妖人置堇湯中,飲之,三日而卒。



 寶臣暮年,益多猜忌,以惟岳
 暗懦,諸將不服,即殺大將辛忠義、盧俶、定州刺史張南容、趙州刺史張彭老、許崇俊等二十餘人,家口沒入,自是諸將離心。建中二年春卒,時年六十四,廢朝三日,冊贈太保。子惟岳、惟誠、惟簡。



 寶臣卒時,惟岳為行軍司馬,三軍推為留後,仍遣使上表求襲父任,朝旨不允。魏博節度使田悅上章保薦,請賜旄節,不許。惟岳乃與田悅、李正己同謀拒命,判官邵真泣諫,以為不可。惟岳暗懦,初雖聽從,終為左右所惑而止。而所與圖議,皆奸吏胡
 震、家人王他奴等,唯勸拒逆為事。



 惟岳舅穀從政者,有智略。為寶臣所忌,稱病不出,至是知惟岳之謀,慮其覆宗,乃出諫惟岳曰:「今天下無事,遠方朝貢,主上神武,必致太平。如至不允,必至加兵。雖大夫恩及三軍,萬一不捷,孰為大夫用命者?又先朝相公與幽帥不協,今國家致討,必命硃滔為帥。彼嘗切齒,今遂復讎,可不懼乎!又頃者相公誅滅軍中將校,其子弟存者,口雖不言,心寧無憤?兵猶火也,不戢自焚。往者田承嗣佐安祿山、史思
 明謀亂天下,千征百戰。及頃年侵擾



 洺、相等州,為官軍所敗,及貶永州,仰天垂泣。賴先相公佐佑保援,方獲赦宥,若雷霆不收,承嗣豈有生理!今田悅兇狂,何如承嗣名望?茍欲坐邀富貴,不料破家覆族。而況今之將校,罕有義心,因利乘便,必相傾陷。為大夫畫久長之計,莫若令惟誠知留後,大夫自速入朝。國家念先相公之功,見大夫順命,何求而不得?今與群逆為自危之計,非保家之道也。」惟嶽亦素忌從政,皆不聽,竟與魏、齊謀叛。



 既而
 惟岳大將張孝忠以郡歸國,朝廷以孝忠為成德軍節度使,仍詔硃滔與孝忠合勢討之。惟岳以精甲屯束鹿以抗之,田悅遣大將孟佑率兵五千助惟岳。建中三年正月,硃滔、孝忠大破恆州軍於束鹿,惟岳燒營而遁。惟岳大將趙州刺史康日知以郡歸國,惟岳乃令衙將衛常寧率士卒五千,兵馬使王武俊率騎軍八百同討日知。武俊既出恆州,謂常寧曰:「武俊盡心於本使,大夫信讒,頗相猜忌,所謂朝不謀夕,豈圖生路!且趙州用兵,捷
 與不捷,武俊不復入恆州矣!妻子任從屠滅,且以殘生往定州事張尚書去也,孰能持頸就戮!」常寧曰:「中丞以大夫不可事,且有詔書云,斬大夫首者,以其官爵授。自大夫拒命已來,張尚書以易州歸國得節度使。今聞日知已得官爵。觀大夫事勢,終為硃滔所滅。此際轉禍為福,莫若倒戈入使府,誅大夫以取富貴也。況大夫暗昧,左右誑惑,其實易圖。事茍不捷,歸張尚書非晚。」武俊然之。三年閏正月,武俊與常寧自趙州回戈,達明至恆,武
 俊子士真應於內。武俊兵突入府署,遣虞任越劫擒惟岳,縊死於戟門外。又誅惟岳妻父鄭華及長慶、王他奴等二十餘人,傳首京師。



 惟誠,惟嶽異母兄,以父廕為殿中丞,累遷至檢校戶部員外郎。好儒書理道,寶臣愛之,委以軍事;性謙厚,以惟岳嫡嗣,讓而不受。同母妹嫁李正己子納。寶臣以其宗姓,請惟誠歸本姓,又令入仕於鄆州,為李納營田副使。歷兗、淄、濟、淮四州刺史,竟客死東平。



 惟簡,寶臣第三子。初,王武俊既誅惟岳,又械惟簡
 送京師。德宗拘於客省,防伺甚峻。硃泚之亂,惟簡斬關而出,赴奉天。德宗嘉之,用為禁軍將。從渾瑊率師討賊,頻戰屢捷,加御史中丞。從幸山南,得「元從功臣」之號,封武安郡王。後授左神威大將軍,轉天威統軍。元和初,檢校戶部尚書、左金吾衛大將軍,充街使;俄拜鳳翔隴右節度使。元和十三年正月卒,贈尚書右僕射。



 子元本,生於貴族,輕薄無行。初,張茂昭子克禮尚襄陽公主。長慶中,主縱恣不法,常游行市里。有士族子薛樞、薛渾者,俱
 得幸於主。尤愛渾,每詣渾家,謁渾母行事姑之禮。有吏誰何者,即以厚賂啖之。渾與元本皆少年,遂相誘掖;元本亦得幸於主,出入主第。張克禮不勝其忿,上表陳聞,乃召主幽於禁中。以元本功臣之後,得減死,杖六十,流象州。樞、渾以元本之故,亦從輕杖八十,長流崖州。



 王武俊,契丹怒皆部落也。祖可訥干,父路俱。開元中,饒樂府都督李詩率其部落五千帳,與路俱南河襲冠帶,有詔褒美,從居薊。武俊初號沒諾干,年十五,能騎射。上
 元中,為史思明恆州刺史李寶臣裨將。寶應元年,王師入井陘。將平河朔,武俊謂寶臣曰:「以寡敵眾,以曲遇直,戰則離,守則潰,銳師遠鬥,庸可御乎?」寶臣遂徹警備,以恆、定、深、趙、易,充本軍先鋒兵馬使。



 大歷十年,田承嗣因薛嵩死,兼有相、衛、磁、邢、洺五州。承嗣遣將盧子期寇磁州,詔令寶臣與李正己、李勉、李承昭、田神玉、硃滔、李抱真各出兵討之。諸軍
 與子期戰於清水,大破之。寶臣將有節生擒子期以獻。代宗嘉其功,使中貴人馬承倩齎詔宣勞。承倩將歸,止傳舍,寶臣親遺百縑。承倩詬詈,擲出道中;寶臣顧左右有愧色。還休府中,諸將散歸,寶臣潛伺屏間,獨武俊佩刀立於門下。召入,解刀與語曰:「見向者頑豎乎?」武俊曰:「今閣下有功尚爾,寇平後,天子以幅紙之詔召置京下,一匹夫耳,可乎?」寶臣曰:「為之若何?」武俊曰:「不如玩養承嗣,以為己資。」寶臣曰:「今與承嗣有釁矣,可推腹心哉?」武
 俊曰:「勢同患均,轉寇仇為父子,亥唾間。若傳虛言,無益也。今中貴人劉清譚在驛,斬首送承嗣,立質妻孥矣!」寶臣曰:「恐不能如此。」武俊曰:「硃滔為國屯兵滄州,請擒送承嗣以取信。」許之。立選士二千,皆乘駿馬,通夜馳三百里,晨至滔營,掩其不備。滔軍出戰,大敗,擒類滔者,滔故得脫。自此寶臣與田承嗣、李正己更相為援,皆武俊萌之。



 寶臣死,其子惟岳謀襲父位。寶臣舊將易州刺史張孝忠以州順命,遂以孝忠代寶臣。俾惟岳護喪歸京,
 惟岳不受命。建中三年正月,詔硃滔、張孝忠合軍討之。惟岳與武俊復統萬餘眾戰於束鹿。武俊率三千騎先進,為滔所敗,惟岳遁走。趙州刺史康日知遂以州順命,惟嶽令武俊統兵擊之。日知遣人謂武俊曰:「惟岳孱微而無謀,何足同反!我城堅眾一,未可以歲月下。且惟岳恃田悅為援,前歲悅之丁男甲卒塗地於邢州城下,猶不能陷,況此城乎!」復給偽手詔招武俊,信之;遂倒兵入恆州,率數百騎入衙門。使謂惟岳曰:「大夫舉兵與魏、齊
 同惡,今田尚書已喪敗,李尚書為趙州所間,軍士自束鹿之役,傷痛軫心。硃僕射強兵宿境內,張尚書已授定州,三軍俱懼殞首喪家。聞有詔征大夫,宜亟赴命,不爾,禍在漏刻。」惟岳怖,遽睢盱。武俊子士真斬惟岳,持首而出。武俊殺不同己者十數人,遂定。傳首上聞,授武俊檢校秘書少監、兼御史大夫、恆州刺史、恆冀都團練觀察使,實封五百戶,以康日知為深趙團練觀察使。



 時惟岳偽定州刺史楊政義以州順命,深州刺史楊榮國降硃
 滔,分兵鎮之。朝廷既以定州屬張孝忠。深州屬康日知。武俊怒失趙、定二州,且名位不滿其志。硃滔怒失深州,因誘武俊謀反,斥言朝廷,遂連率勁兵救田悅。時馬燧、李抱真、李芃、李晟方討田悅,敗悅於洹水。後連歲暴兵,然悅勢已蹙。至是武俊、硃滔復振起之。悅勢益張。



 十一月,武俊使大將張鐘葵寇趙州,康日知擊敗之,斬首上獻。是日,武俊僭建國,稱趙王,又恆州為真定府,偽命官秩。硃滔、田悅、李納一同僭號,分據所部,各遣使勸誘蔡
 州李希烈同僭位號。四年三月,希烈既為周曾謀潰其腹心,或傳希烈已死,馬燧等四節度軍中聞之,歡聲震外。



 六月,李抱真使辯客賈林詐降武俊。林至武俊壁曰:「是來傳詔,非降也。」武俊色動,徵其說。林曰:「天子知大夫宿誠,及登壇建國之日,撫膺顧左右曰:『我本忠義,天子不省。』是後諸軍曾同表論列大夫。天子覽表動容,語使者曰:『朕前事誤,追無及已。朋友間失意尚可謝,朕四海主,毫芒安可復念哉!』」武俊曰:「僕虜將,尚知存撫百姓,天
 子固不專務殺人以安天下。今山東大兵者五,比戰勝,骨盡暴野,雖勝與誰守?今不憚歸國,以與諸侯盟約,虜性直,不欲曲在己。朝廷能降恩滌蕩之,僕首倡歸國,不從者,於以奉辭,則上不負天子,下不負朋友。此謀既行,河朔不五旬可定。」



 十月,涇原兵犯闕,上幸奉天。京師問至,諸將退軍。李抱真將還潞澤,田悅說武俊與硃滔襲擊之。賈林復說武俊曰:「今退軍前輜重,後銳師,人心固一,不可圖也。且勝而得地,則利歸魏博;喪師,即成德大
 傷。大夫本部易、定、滄、趙四州,何不先復故地?」武俊遂北馬首,背田悅約。賈林復說武俊曰:「大夫冀邦豪族,不合謀據中華。且滔心幽險,王室強即藉大夫援之,卑即思有並吞。且河朔無冀國,唯趙、魏、燕耳!今硃滔稱冀,則窺大夫冀州,其兆已形矣。若滔力制山東,大夫須整臣禮;不從,即為所攻奪,此時臣滔乎?」武俊投袂作色曰:「二百年宗社,我尚不能臣,誰能臣田舍漢!」由此計定,遂南修好抱真,西連盟馬燧。會興元元年德宗罪己,大赦反側。
 二月,武俊集三軍,削偽國號。詔國子祭酒兼御史大夫董晉、中使王進傑,自行在至恆州宣命,授武俊檢校兵部尚書、成德軍節度使。三月,加司空、同中書門下平章事,兼幽州、盧龍兩道節度使、瑯邪郡王。



 時硃泚偽冊滔為皇太弟,滔率幽、檀勁卒,誘回紇二千騎,已圍貝州數十日,將絕白馬津,南盜洛都,與泚合勢。時李懷光反,據河中;李希烈已陷大梁,南逼江、漢;李納尚反於齊,田緒未為用;李晟孤軍壁渭上。天子羽書所制者,天下才十
 二三,海內蕩析,人心失歸。賈林又說武俊與抱真合軍,同救魏博,為武俊陳利害曰:「硃滔此行,欲先平魏博,更逢田悅被害,人心不安。旬日不救,魏、貝必下,滔益數萬。張孝忠見魏、貝已拔,必臣硃滔。三道連衡,兼統回紇,長驅至此,家族可得免乎?常山不守,則昭義退保山西,河朔地盡入滔。今乘魏、貝未下,孝忠未附,公與昭義合軍破之,如掇遺耳!此計就,則聲振關中,京邑可坐復,鑾輿反正自公,則勛業無二也。」武俊歡然許之。兩軍議定,卜
 日同征。五月,武俊、抱真會軍於鉅鹿東。兩軍既交,滔震恐。抱真為方陣,武俊用奇兵,硃滔傾壘出戰。武俊不擐甲而馳之。滔望風奔潰,自相蹂踐,死者十四五。收其輜重、器甲、馬牛不可勝計,滔夜奔還幽州。武俊班師,表讓幽州盧龍節度使,許之。乃升恆州為大都督府,以武俊為長史,加檢校司徒,實封七百戶,餘如故。



 車駕還京,寵之逾厚。子尚貴主,子弟在孩稚者,皆賜官名。尋丁母憂,起復加左金吾上將軍同正;免喪,加開府儀同三司。十
 二年,上念舊勛,加檢校太尉,兼中書令。



 十七年六月卒,時年六十七,廢朝五日,群臣詣延英門奉慰,如渾瑊故事。詔左庶子上公持節冊贈太師,賻絹三千匹、布千端、米粟三千碩。太常謚曰威烈,德宗曰:「武俊竭忠奉國,宜賜謚忠烈。」子士真、士清、士平、士則。士真嗣。



 士真,武俊長子。少驍悍,冠於軍中,沉謀有斷。事李寶臣為帳中親將,仍以女妻之。寶臣末年,慮身後諸子暗弱,為諸將所奪,屢行誅戮,諸將離心。武俊官位雖卑,而勇略邁世;寶臣
 惜其才,不忍誅之。而士真密結寶臣左右,保護其父,以是獲免。



 惟岳之世,尤加委任,武俊亦盡心匡佐。既兵敗束鹿,張孝忠、康日知以地歸國,受官賞;惟岳稍貯防疑,武俊謀自貶損,出入不過三兩人。左右謂惟岳曰:「先相公委任武俊,以遺大夫,兼有治命。今披肝膽為大夫者,武俊耳。又士真即大夫妹婿,保無異志。今勢危急,若不坦懷待之,若更如康日知,即大事去矣!」惟岳曰:「我待武俊自厚,不獨先公遺旨。」由是無疑,即令將兵攻趙州。士
 真更宿於府衙,與同職謀事。及武俊倒戈,士真等數人擒惟岳出衙,縊死之。武俊領節鉞,以士真為副大使。



 建中年,武俊僭稱趙王於魏縣,以士真為司空、真定府留守,充元帥。及武俊破硃滔順命,以武俊兼幽州盧龍軍節度使,仍以士真為副使、檢校工部尚書。德宗還京,進位檢校兵部尚書,充德州刺史、德棣觀察使,封清河郡王。十七年,武俊卒,起復授左金吾衛大將軍同正、恆州大都督府長史,充成德軍節度、恆冀深趙德棣等州觀
 察等使。尋檢校尚書左僕射。順宗即位,進位檢校司空。



 士真佐父立功,備歷艱苦;得位之後,恬然守善,雖自補屬吏,賦不上供,然歲貢貨財,名為進奉者,亦數十萬,比幽、魏二鎮,最為承順。元和元年,就加同中書門下平章事。四年三月卒。子承宗、承元、承通、承迪、承榮。



 士清,以父勛累加官至殿中少監同正。元和初,為冀州刺史、御史大夫,封北海郡王,早卒。



 士平,以父勛補原王府咨議。貞元二年,選尚義陽公主,加秘書少監同正、附馬都尉。元
 和中,累遷至安州刺史。時公主縱恣不法,士平與之爭忿;憲宗怒,幽公主於禁中,士平幽於私第,不令出入。後釋之,出為安州刺史。坐與中貴交結,貶賀州司戶。時輕薄文士蔡南、獨孤申叔為義陽主歌詞,曰《團雪》、《散雪》等曲,言其游處離異之狀,往往歌於酒席。憲宗聞而惡之,欲廢進士科,令所司網捉搦,得南、申叔貶之,由是稍止。及盜殺宰相武元衡,旬日捕賊未獲。士平與兄士則庭奏盜主於承宗,既獲張晏等誅之,乃以士平為左金吾
 衛大將軍。及奪承宗官爵,仍以士平襲父實封。



 士則,士平異母兄。承宗既立為節度使,不容諸父,乃奔於京師,用為神策大將軍。及承宗叛逆,盜殺宰相,士則請移貫京兆府。諸鎮兵討承宗,裴度言士則武俊子,其軍中必有懷之者,乃用士則為邢州刺史,兼本州團練使,從昭義節度使郗士美討賊,冀攜離承宗之黨,且許以節制。士則恃此,頗不受士美節制,行止以兵自衛;雖謁士美,而衛兵如故。吏呵止之,士則不能平,見於辭氣。士美惡
 之,密以狀聞,乃以張遵代還。



 承宗,士真長子。河朔三鎮自置副大使,以嫡長為之。承宗累奏至鎮州大都督府右司馬、知州事、御史大夫,充都知兵馬使、副大使。



 元和四年三月,士真卒;三軍推為留後,朝廷伺其變,累月不問。承宗懼,累上表陳謝。至八月,上令京兆少尹裴武往宣諭,承宗奉詔甚恭,且曰:「三軍見迫,不候朝旨,今請割德、棣二州上獻,以表丹懇。」由是起復雲麾將軍、左金吾衛大將軍同正、檢校工部尚書、鎮州大都督府長史、御
 史大夫、成德軍節度、鎮冀深趙等州觀察等使。又以德州刺史薛昌朝檢校右散騎常侍、德州刺史、御史大夫,充保信軍節度、德棣觀察等使。



 昌朝,故昭義節度使嵩之子,婚姻於王氏,入仕於成德軍,故為刺史。



 承宗既獻二州,朝廷不欲別命將帥,且授其親將。保信旌節未至德州,承宗遣數百騎馳往德州,虜昌朝歸真定囚之。朝廷又加棣州刺史田渙充本州團練守捉使,冀漸離之。令中使景忠信往諭旨,令遣昌朝還鎮,承宗不奉詔。憲
 宗怒,下詔曰:「枉承宗頃在苫廬,潛窺戎鎮;而內外以事君之禮,逆而必誅,分土之儀,專則有闢。朕念其先祖嘗有茂勛,貸以私恩,抑於公議。使臣旁午以告諭,孽童俯伏以陳誠,願獻兩州,期無二事。朕欲收其後效,用以曲全,授節制於舊疆,齒勛賢於列位。況德、棣本非成德所管,昌朝又是承宗懿親,俾撫近鄰,斯誠厚渥,外雖兩鎮,中實一家。而承宗象恭懷奸,肖貌稔禍。欺裴武於得位之後,縲昌朝於受命之中。豺狼之心,飽之而愈發;梟獍
 之性,養之而益兇。加以表疏之中,悖慢斯甚。式遏亂略,期於無刑;恭行天誅,干於有制。可削承宗在身官爵。」詔左神策護軍中尉吐突承璀為左右神策、河中、河陽、浙西、宣歙等道赴鎮州行營兵馬招討處置等使,會諸道軍進討。神策兵馬使趙萬敵者,王武俊之騎將也,驍悍聞於燕、趙,具言進討必捷。承璀因得兵柄,與萬敵偕行。承璀至行營,威令不振,禁軍屢挫衄。都將酈定進前擒劉闢有功,號為驍將,又陷於賊。唯範陽節度使劉濟、易
 定節度使張茂昭至效忠赤,戰賊屢捷。而昭義節度使盧從史反復難制,陰附於賊;憲宗密詔承璀擒之,送於京師。



 五年七月,承宗遣巡官崔遂上表三封,乞自陳首,且歸過於盧從史。其略曰:



 臣頃在苫廬,綿歷時序,恭守朝旨,罔敢闕違。復奉詔書,令獻州郡,迫以三軍之勢,不從孤臣之心。今天兵四臨,王命久絕,白刃之下,難避國刑;殷憂之中,轉積釁隙。中由盧從史首為亂階,興天下之兵,生海內之亂,既不忠於國,又不孝於家。當其聞父
 之喪,已變為臣之節,迫脅天使,瀆紊朝經。而乃幸臣居喪,敗臣求利,上敢欺於聖主,下不顧其死親;矯情徒見於封章,邪妄素萌於胸臆。今構禍者已就擒獲,抱冤者實冀辯明。況臣之一軍,素守忠義,橫被從史離間君臣,哀號轅門,痛隔恩外。伏冀陛下以天地之德,容納為心;弘好生之仁,許自新之路。順陽和而布澤,因雷雨以覃恩。追念祖父之前勞,俯觀臣子之來效,特開湯網,使樂堯年。



 時朝廷以承璀宿師無功,國威日沮,頗憂。會承宗
 使至,宰臣商量,請行赦宥,乃全以六郡付之。承宗送薛昌朝入朝,授以右武衛將軍。



 承宗以國家加兵不勝,誣從史奸計得行,雖上章表謙恭,而心無忌憚。十年,王師討吳元濟,承宗與李師道繼獻章表,請宥元濟。其牙將尹少卿奏事,因為元濟游說。少卿至中書,見宰相論列,語意不遜;武元衡怒,叱出之,承宗益不順。自是與李師道奸計百端,以沮用兵。四月,遣盜燒河陰倉。六月,遣盜伏於靖安里,殺宰相武元衡,京師震恐,大索旬
 日,天子為之旰食。是時,承宗、師道之盜,所在竊發,焚襄州佛寺,斬建陵門戟,燒獻陵寢宮,欲伏甲屠洛陽。憲宗赫怒,命田弘正出師臨其境,並鄰道六節度之眾討之。時方淮西用兵,國用虛竭,河北諸軍多觀望不進。獨昭義節度使郗士美率精兵壓賊壘,欲乘釁而取之,軍威甚盛。承宗懼,不敢犯。俄詔權罷河北用兵,並力淮西。



 十二年十月,誅吳元濟,承宗始懼,求救於田弘正。十三年三月,弘正遣人送承宗男知感、知信及其牙將石汛等詣闕請
 命,令於客舍安置;又獻德、棣二州圖印,兼請入管內租稅,除補官吏。上以弘正表疏相繼,重違其意,乃下詔曰:



 帝者承天子人,下臨萬國。觀乾坤覆載之施,常務其曲全;用德刑撫御之方,每先其弘貸。叛則必伐,服而舍之,訪於典謨,亦尚斯道。朕祗符前訓,纘嗣丕圖,底寧方隅,蕩滌氛祲。上以攄祖宗之宿憤,下以致黎庶之阜康,思厚者生,務去者殺。至於包荒藏慝,屈法伸恩,茍衷誠之可矜,則宥過而無大。



 王承宗頃居喪紀,見賣於鄰封;後
 鄰籓城,受疑於朝野。國恩雖厚,時憲不容。戚實自貽,寵非我絕。百闢卿士,昌言在廷;四方諸侯,飛奏盈篋,競請致討,爭先出軍。尚復廣示招懷,務存容納,至於動眾,事豈願然!開境愍罹其殺傷,退舍為伏其士伍,取陷救溺,能無慘嗟!以其先祖武俊,有勞王室,書於甲令,銘在景鐘;雖再駕王師,再從人欲,而十代之宥,常切朕懷。



 近以三朝稱慶,八表流澤,廣此鴻霈,開其自新。而承宗果能翻然改圖,披露忠懇,遠遣二子,進陳表章,緘圖印以上
 聞,獻德、棣之名部,發囷奉粟,並灶貢鹽,地願帥於職方,物請歸於司會。且天子所臨,莫非王土;析茲舊服,將表爾誠,諒由效順之心,悉見納忠之志,抑而不撫,何以示懷。朕念此方,亦猶赤子,一物失所,寢興靡寧;忍驅樂土之人,竟就陳原之戮!既克翦暴,常思止戈,予之此心,天地臨鑒。況常山師旅,舊有功勞,將改往以修來,誓酬恩而遷善,鑒精誠之俱切,俾渙汗而再敷。曠滌乃愆,斷於朕志;復此殊渥,當懷永圖。承宗可依前銀青光祿大夫、
 檢校吏部尚書、鎮州大都督府長史、御史大夫,充成德軍節度、鎮冀深趙觀察等使。



 仍令右丞崔從往鎮州宣慰。承宗素服俟命,乃以華州刺史鄭權為德州刺史,充橫海軍節度、德棣滄景觀察等使。明年,加金紫光祿大夫、檢校尚書左僕射。是歲,李師道平,承宗奉法逾謹,請當管四州,每州置錄事參軍一員、判司三員,每縣令一員、主簿一員,吏補授皆聽朝旨。十五年十一月卒,贈侍中。子知感、知信在朝。



 承元,士真第二子。兄承宗既領節
 鉞,奏承元為觀察支使、朝議郎、左金吾衛胄曹參軍,兼監察御史,年始十六。勸承宗以二千騎佐王師平李師道,承宗不能用其言。



 元和十五年冬,承宗卒,秘不發喪,大將謀取帥於旁郡。時參謀崔燧密與握兵者謀,乃以祖母涼國夫人之命,告親兵及諸將,使拜承元。承元拜泣不受,諸將請之不已。承元曰:「天子使中貴人監軍,有事盍先與議。」及監軍至,因以諸將意贊之。承元謂諸將曰:「諸公未忘先德,不以承元齒幼,欲使領事。承元欲效
 忠於國,以奉先志,諸公能從之乎?」諸將許諾。遂於衙門都將所理視事,約左右不得呼留後,事無巨細,決之參佐。密疏請帥,天子嘉之,授銀青光祿大夫、檢校工部尚書,兼滑州刺史、義成軍節度、鄭滑觀察等使。鄰鎮以兩河近事諷之,承元不聽,諸將亦悔。及起居舍人柏耆齎詔宣諭滑州之命,兵士或拜或泣。承元與柏耆於館驛召諸將諭之,諸將號哭喧嘩。承元詰之曰:「諸公以先世之故,不欲承元失此,意甚隆厚;然奉詔遲留,其罪大矣!
 前者李師道未敗時,議赦其罪,時師道欲行,諸將止之,他日殺師道,亦諸將也!今公輩辛勿為師道之事,敢以拜請。」遂拜諸將,泣涕不自勝。承元乃盡出家財,籍其人以散之,酌其勤者擢之。牙將李寂等十數人固留承元,斬寂等,軍中始定。承元出鎮州,時年十八,所從將吏,有具器用貨幣而行者,承元悉命留之。承元昆弟及從父昆弟,授郡守者四人,登朝者四人,從事將校有勞者,亦皆擢用。祖母涼國夫人入朝,穆宗命內宮筵待,錫賚甚
 厚。



 俄而王廷湊殺田弘正,據鎮州叛。移鎮鄜坊丹延節度使,便道請覲,穆宗器之,數召顧問。未幾,改鳳翔節度使。鳳翔西北界接涇原,無山谷之險,吐蕃由是徑往入寇。承元於要沖築壘,分兵千人守之,賜名曰臨汧城。詔襲岐國公,累加檢校左僕射。鳳翔城東,商旅所集,居人多以烽火相警,承元奏益城以環之。居鎮十年,加檢校司空、御史大夫,移授平盧軍節度、淄青登萊觀察等使。時均輸鹽法未嘗行於兩河,承元首請鹽法,歸之有司,
 自是兗、鄆諸鎮,皆稟均輸之法。承元寬惠有制,所理稱治。太和七年十二月,卒於平盧,時年三十三,冊贈司徒。



 王廷湊,本回鶻阿布思之種族,世隸安東都護府。曾祖曰五哥之,事李寶臣父子。王武俊養為假子,驍果善鬥,武俊愛之。以軍功累授左武衛將軍同正,贈越州都督。祖末怛活,贈左散騎常侍。父升朝,贈禮部尚書。皆以廷湊貴加贈典。祖父世為王氏騎將,累遷右職。



 廷湊沉勇寡言,雄猜有斷,為王承元衙內兵馬使。初,承元上稟朝
 旨,田弘正帥成德軍,國家賞錢一百萬貫,度支輦運不時至,軍情不悅。廷湊每抉其細故,激怒眾心。會弘正以魏兵二千為衙隊,左右有備不能間。長慶元年六月,魏軍還鎮。七月二十八日夜,廷湊乃結衙兵噪於府署;遲明,盡誅弘正與將吏家族三百餘人。廷湊自稱留後、知兵馬使,將吏逼監軍宋惟澄上章請授廷湊節鉞。穆宗怒,下詔徵鄰道兵,仍以河東節度裴度充幽、鎮兩道招撫使,仍以弘正子涇原節度使布代李醖為魏博節度
 使,令率魏軍進討。又以承宗故將深州刺史牛元翼為成德軍節度使,下詔購誅廷湊。是月,鎮州大將王位等謀殺廷湊事洩,坐死者二千餘人。



 時硃克融囚張弘靖,廷湊殺弘正,合從構逆謀,拒王命。兩鎮並力,討除慮難應接,詔朝臣議其可否。東川節度使王涯獻狀曰:「幽、鎮兩州,悖亂天紀,迷亭育之厚德,肆狼虎之非心。囚縶鼎臣,戕賊戎帥,毒流州郡,釁及賓僚。凡在有情,孰不痛憤?伏以國家文德誕敷,武功繼立,遠無不伏,邇無不安,矧茲
 二方,敢逆天理。臣竊料詔書朝下,諸鎮夕驅,以貔貅問罪之師,當猖狂失節之寇,傾山壓卵,決海灌熒,勢之相懸,不是過也。但常山、薊郡,虞、虢相依,一時興師,恐費財力。罪有輕重,事有後先,譬之攻堅,宜從易者。如聞範陽肇亂,出自一時,事非宿謀,跡亦可驗。鎮州構禍,殊匪偶然,扇諸屬城,以兵拒境。如此,則幽薊之眾,可示寬刑;鎮冀之戎,可資先討。況廷湊闒茸,不席父祖之資;成德分離,又多迫脅之勢。今以魏博思復仇之眾,昭義願盡敵
 之師,參之晉陽,輔以滄德,掎角而進,實若建瓴。盡屠其城,然後北首燕路,在朝廷不為失信,於軍勢實得機宜。臣之愚誠,切在於此。臣又聞用兵若鬥,先扼其喉。今瀛鄚、易定,兩賊之咽喉也。誠宜假之威柄,戍以重兵,俾其死生不相知,間諜無所入;而以大軍先進冀、趙,次臨井陘,此一舉萬全之勢也。」



 於是命易定節度使開境以抗克融,諸軍三面進討。初,以滄德烏重胤獨當一面,重胤宿將,知不可進,頗遲留,乃以杜叔良代重胤。叔良有中
 官之援,朝辭日,大言云:「賊不足破。」時廷湊合幽薊之兵圍深州,梯沖雲合,牛元翼嬰城拒守。十一月,杜叔良為賊所敗,眾皆陷沒,僅以身免,乃以德州王日簡代之。裴度率眾屯承天軍,諸將挫敗,深州危急。乃以鳳翔節度使李光顏為忠武節度使,兼深冀節度,救深州,仍以中官楊永和監光顏軍。



 國家自憲宗誅除群盜,帑藏虛竭;穆宗即位,賞賜過當;及幽、鎮共起,徵發百端,財力殫竭。時諸鎮兵十五萬餘,才出其境,便仰給度支,置南北供
 軍院。既深入賊境,輦運艱阻,芻薪不繼,諸軍多分番樵採。俄而度支轉運車六百乘,盡為廷湊邀而虜之,兵食益困。賊圍深州數重,雖光顏之善將,亦無以施其方略。其供軍院布帛衣賜,往往不得至院,在途為諸軍強奪,而懸軍深鬥者,率無支給。復又每軍遣內官一人監軍,悉選驍健者自衛,羸懦者即戰,以是屢多奔北。而廷湊、克融之眾,不過萬餘,而抗官軍十五萬者,良以統制不一,玩寇邀利故也。宰相崔祐甫不曉兵家,膠柱於常態,
 以至復失河朔。既無如之何,遂議休兵而赦廷湊。



 二年正月,魏府牙將史憲誠誘其軍謀叛,田布不能止,其眾自潰於南宮。二月,詔赦廷湊,仍授檢校右散騎常侍、鎮州大都督府長史、成德軍節度、鎮冀深趙等州觀察等使;以牛元冀為山南東道節度使。遣兵部侍郎韓愈至鎮州宣慰,又遣中使銜命入深州,監元翼赴鎮。廷湊雖受命,而深州之圍不解。招撫使裴度與幽、鎮書,以大義責之;硃克融解圍而去,廷湊亦退舍。朝廷欲其稟命,並
 加克融檢校工部尚書。三月,元翼率十餘騎突圍出深州赴闕,深州將校臧平以城降。廷湊責其固守,殺將吏一百八十餘人。五月,遣中使楊再昌至鎮州,取牛元翼家族及田弘正骸骨。廷湊曰:「弘正骸骨,不知所在;元翼家族,請至秋發遣。」俄而元翼卒,廷湊乃盡屠其家,其酷毒如此。自獲赦宥,遂與硃克融、史憲誠連衡相應,謀拒朝廷。



 太和初,滄州李全略死,其子同捷欲效河朔事,求代父任。文宗授以兗海節度使;同捷不奉詔,據郡構逆,
 以珍玩器幣妓女子弟投款於廷湊及幽州李載義。時載義初代克融,輸誠效順,盡送同捷所遣赴闕,詔徵幽、魏、徐、兗之師進討。廷湊出兵撓魏北境,以援同捷。二年,下詔絕廷湊進奉。既魏博將丌志治以行營兵叛,倒戈攻魏州,諸軍擊志治,廷湊出兵應之。史憲誠危急,詔義武軍節度使李聽擊敗之,志治奔於廷湊。三年六月,誅李同捷。尋又何進滔殺史憲誠,據魏州。朝廷厭兵,誅之不果,遂授進滔魏博節度。八月,廷湊遣使詣闕請罪,朝
 廷因而赦之;依前檢校司徒、成德軍節度使。



 鎮冀自李寶臣已來,雖惟岳、承宗繼叛,而猶親鄰畏法,期自新之路。而兇毒好亂,無君不仁,未如廷湊之甚也!又就加太子太傅、太原郡開國公,食邑二千戶。八年十一月卒,冊贈太尉,累贈至太師。



 子元逵,為鎮州右司馬,兼都知兵馬使。廷湊卒,三軍推主軍事,請命於朝。乃起復檢校工部尚書、鎮州大都督府長史、成德軍節度使,累遷檢校左僕射。元逵素懷忠順,頓革父風。及領籓垣,頗輸誠款,
 歲時貢奉,結轍於途,文宗嘉之。開成二年,詔以壽安公主出降,加駙馬都尉。元逵遣段氏姑詣闕納聘禮。段氏進食二千盤,並御衣戰馬、公主妝奩及私白身女口等,其從如云,朝野榮之。會昌中,昭義節度使劉從諫卒,其子稹擅領軍政;武宗怒,誅之。命鄰籓分地而進討,以元逵為北面招討使。詔至之日,出師次趙州,與魏博何弘敬同收山東三州。元逵進攻邢州,俄而賊將裴問、高元武降元逵,王釗、安玉降何弘敬,並拔三郡。累遷檢校司
 徒、同中書門下平章事。以破劉稹功,加太傅、太原郡開國公,食邑二千戶,食實封二百戶。太中十一年二月卒,冊贈太師,謚曰忠。子紹鼎、紹懿。



 紹鼎,時為鎮州大都督府左司馬、知府事、節度副使、都知兵馬使。起復授檢校工部尚書、鎮府長史、成德軍節度、鎮深冀趙觀察等使,累加光祿大夫、尚書左僕射。其年七月卒,贈司空,賻布帛三百段、米粟二百碩,累贈司徒、太尉,又贈太傅。



 子景胤、景崇、景敔;景崇為嫡,時年幼。



 紹鼎卒,宣宗以昭王汭
 為鎮州大都督、成德軍節度副使,都知兵馬使、檢校右散騎常侍、鎮府左司馬、知府事、兼御史中丞,王紹懿本官充成德軍節度、觀察留後,仍賜紫金魚袋。尋正授節度使、檢校工部尚書。累加檢校右僕射、兼御史大夫、太原縣開國伯,食邑七百戶,又加檢校司空。卒,贈司徒。



 景胤,初為成德軍中軍兵馬使、銀青光祿大夫、檢校太子賓客、監察御史。紹鼎卒,出為深州刺史、兼殿中侍御史,充本州團練守捉使。



 景崇,於季父紹懿時為鎮州大都
 督府左司馬、知府事、都知兵馬使。紹鼎卒,三軍立紹懿。數月,疾篤。召景崇謂之曰:「亡兄以軍政托予,以俟汝成立。今危惙如此,殆將不救。汝雖少年,勉自負荷,下禮籓鄰,上奉朝旨,俾吾兄家業不墜,惟汝之才也!」言訖而卒。時監軍在席,奏其治命,上嘉之,詔起復忠武將軍、守左金吾衛將軍同正、檢校右散騎常侍,充成德軍節度觀察留後,仍賜上柱國,賜紫金魚袋。尋正授節度使、檢校工部尚書。



 咸通中,景崇以公主嫡孫,特承恩渥。季年,盜
 起徐方,王師進討,景崇令大將從諸軍。徐寇平,以功授檢校右僕射,封太原縣男,食邑三百戶。祖母章惠長公主薨,景崇居喪得禮,朝野稱之。起復左金吾衛上將軍同正,進位檢校司空。明年,同中書門下平章事,累加檢校太尉、趙國公,食邑三千戶,食實封二百戶,尋進封常山王。丁母秦國夫人憂,起復本官。乾符末,盜起河南,黃巢犯闕,駕幸劍南;景崇與定州節度使王處存馳檄籓鄰,以兵附處存入關討賊,奔問行在,貢輸相繼。關輔平
 定,以功真拜太尉。中和二年十二月卒。



 子鎔,時年十歲,三軍推為留後,朝廷因授旄鉞,檢校工部尚書。時天子蒙塵,九州鼎沸,河東節度李克用虎視山東,方謀吞據;鎔以重賂結納,以修和好。晉軍討孟方立於邢州,鎔常奉以芻糧。及方立平,晉將李存孝侵鎔南部,鎔求援於幽州。幽帥李匡威率眾三萬赴之,存孝退去。景福元年,鎔乘存孝有間於其師,乃出兵攻堯山。晉帥遣大將李存質來援,大敗鎮人於堯山,死者萬計。晉人乘勝至趙
 州,鎔復求援於燕。二年,匡威率眾數萬來援。會邢州節度使李存孝背其帥據城自固,存孝單騎入鎮州,與鎔面相盟約。俄而李克用自率全師攻存孝,時匡威離鎮後,其弟匡籌奪據其位,匡威退無歸路。鎔感其援助之恩,乃迎入府城,築第以居之,事之如父;匡威亦盡心裨益,軍中之事,皆為訓練。是年五月,鎔過匡威第,陰遣部下伏甲劫鎔;鎔抱持之,鎔曰:「公誡止人勿倉卒!吾為晉人所困,賴公獲濟,猶吾父也,軍政請公帥之。」即並轡歸
 府署,鎮軍拒之,竟殺匡威。晉人知匡威死,克用自率師至城下;鎔出練二十萬犒勞,修好而退。



 及汴宋節度使硃全忠領鄆、青三鎮,兵強天下,遣將葛從周、張存敬寇陷邢、洺二州,乘勝北掠燕、趙。俄而全忠率親兵薄於城下。鎔倉卒無備,謂賓佐曰:「勢危矣,計將安出?」判官周式者,率先而對曰:「敵人迫我,兵不能抗,此可以理說耳,請見梁帥圖之!」式即時出見全忠,全忠逆謂式曰:「爾不必言。王令朋附並汾,違盟爽信,敝賦業已及此,期於無舍!」
 式曰:「公言過矣!且公為唐室之桓、文,當以禮義而成霸業。乃欲窮兵黷武,困人於險難,天下其謂公何!」全忠喜,引式袂而慰之曰:「前言戲之耳!且君為王令計如何?」式曰:「但修好耳!」即復見鎔,請出牛酒貨幣以犒軍;仍以鎔子昭祚及牙將梁公儒、李弘規子各一人,從昭祚入官於大梁,全忠以女妻昭祚。



 及全忠僭,天下無主;鎔不獲已,行其正朔。鎔累遷至開府儀同三司,守太師、中書令,仍賜「敦睦保定大功臣」、上柱國、趙王,食邑一萬五千戶,
 食實封一千戶,襲食實封二百五十戶。偽梁加尚書令,及唐室中興,去偽尚書令之號。天祐七年,母魏國太夫人何氏卒,起復本官。十八年,為其大將王德明所殺,至於赤族。其後事在中興雲。



 史臣曰:土運中微,群盜孔熾。寶臣附麗安、史,流毒中原,終竊土疆,為國蟊賊。加以武俊之狠狡,為其腹心,或叛或臣,見利忘義,蛇吞蝮吐,垂二百年。哀哉,王政不綱,以至於此。若使明皇不懈於開元之政,姚崇久握于阿衡,
 詎有柳城一胡,敢窺佐伯,況其下者哉!觀此無君,可為太息。



 贊曰:鵂鶹為怪,必取其昏。人君失政,為盜啟門。牙旂金鉞,虎子狼孫。茫茫黔首,於何叫閽?



\end{pinyinscope}