\article{卷一百五十九}

\begin{pinyinscope}

 ○穆寧子贊質員賞崔邠弟鄯郾鄲竇群兄常牟弟庠鞏李遜弟建薛戎弟放



 穆寧,懷州河內人也。父元休,以文學著。撰《洪範外傳》十
 篇,開元中獻之。玄宗賜帛,授偃師縣丞、安陽令。



 寧清慎剛正,重交游,以氣節自任。少以明經調授鹽山尉。是時,安祿山始叛,偽署劉道玄為景城守,寧唱義起兵,斬道玄首。傳檄郡邑,多有應者。賊將史思明來寇郡,寧以攝東光令將兵御之。思明遣使說誘,寧立斬之。郡懼賊怨深,後大兵至,奪寧兵及攝縣。初,寧佐採訪使巡按,常過平原,與太守顏真卿密揣祿山必叛。至是,真卿亦唱義,舉郡兵以拒祿山。會間使持書遺真卿曰:「夫子為衛君
 乎?」更無他詞。真卿得書大喜,因奏署大理評事、河北採訪支使。寧以長子屬母弟曰:「惟爾所適,茍不乏嗣,吾無累矣。」因往平原,謂真卿曰:「先人有嗣矣!古所謂死有輕於鴻毛者,寧是也。願佐公以定危難。」真卿深然之。其後,寧計或不行,真卿迫蹙,棄郡,夜渡河而南,見肅宗於鳳翔。帝問拒賊之狀,真卿曰:「臣不用穆寧之言,功業不成。」帝奇之,發驛召寧,將以右職待之。會真卿以抗直失旨,事遂止。



 上元二年,累官至殿中侍御史,佐鹽鐵轉運使。
 副元帥李光弼以餉運不繼,或惡寧者,誣譖於光弼,光弼揚言欲殺寧。寧直抵徐州見光弼,喻以大義,不為撓折。光弼深重之,寧得行其職。寶應初,轉侍御史,為河南轉運租庸鹽鐵等副使。明年,遷戶部員外郎。無幾,加兼御史中丞,為河南、江南轉運使。廣德初,加庫部郎中。是時河運不通,漕挽由漢、沔自商山達京師。選鎮夏口者,詔以寧為鄂州刺史、鄂岳沔都團練使,及淮西鄂岳租庸鹽鐵沿江轉運使,賜金紫。時淮西節度使李忠臣貪
 暴不奉法,設防戍以稅商賈,又縱兵士剽劫,行人殆絕。與寧夾淮為理,憚寧威名,寇盜輒止。沔州別駕薛彥偉坐事忤旨,寧仗之致死。寧坐貶虔州司馬,重貶昭州平集尉。



 大歷四年,起授監察御史,領轉運留後事於淄青。間一年,改檢校司封郎中、兼侍御史,領轉運留後事於江西。明年,拜檢校秘書少監,兼和州刺史,理有善政。居無何,官罷。代寧者以天寶版籍校見戶,誣以逋亡多,坐貶泉州司戶。寧子贊,守闕三年告冤。詔遣御史按覆,而
 人戶增倍,詔書召寧,除右諭德。寧強毅,不能事權貴。執政者以為不附己,且憚其難制,故處之散位。寧默默不得志,且曰:「時不我容,我不時殉,則非吾之進也,在於退乎!」辭病居家,請告幾十旬者數矣。親友強之,復一朝請。上居奉天,寧詣行在,拜秘書少監。興元初,改右庶子。德宗還京師,寧曰:「可以行吾志矣。」因移病,罷歸東都。貞元六年,就拜秘書監致仕。



 寧好學,善教諸子,家道以嚴稱。事寡姊以悌聞。通達體命,未嘗服藥。每誡諸子曰:「吾聞
 君子之事親,養志為大,直道而已。慎無為諂,吾之志也。」貞元十年十月卒,時年七十九。四子:贊、質、員、賞。



 贊,字相明,釋褐為濟源主簿。時父寧為和州刺史,以剛直不屈於廉使,遂被誣奏,貶泉州司戶參軍。贊奔赴闕庭,號泣上訴。詔御史覆問,寧方得雪。詔曰:「令子申父之冤,憲臣奉君之命,楚劍不沖於牛斗,秦臺自洗於塵埃。」由是知名。累遷京兆兵曹參軍、殿中侍御史,轉侍御史,分司東都。



 時陜州觀察使盧嶽妾裴氏,以有子,岳妻分財不及,
 訴於官,贊鞫其事。御史中丞盧佋佐之,令深繩裴罪。贊持平不許。宰臣竇參與佋善,參、佋俱持權,怒贊以小事不受指使,遂下贊獄。侍御史杜倫希其意,誣贊受裴之金,鞭其使以成其獄,甚急。贊弟賞,馳詣闕,撾登聞鼓。詔三司使覆理無驗,出為郴州刺史。參敗,徵拜刑部郎中。因次對,德宗嘉其才,擢為御史中丞。時裴延齡判度支,以奸巧承恩。屬吏有贓犯,贊鞫理承伏。延齡請曲法出之,贊三執不許,以款狀聞。延齡誣贊不平,貶饒州別駕。
 丁母憂,再轉虔、常二州刺史。



 憲宗即位,拜宣州刺史、御史中丞,充宣歙觀察使,所蒞皆有政聲。永貞元年十一月卒,時年五十八,贈工部尚書。



 贊與弟質、員、賞以家行人材為搢紳所仰。贊官達,父母尚無恙,家法清嚴。贊兄弟奉指使,笞責如僮僕,贊最孝謹。



 質強直,應制策入第三等。其所條對,至今傳之。自補闕至給事中,時政得失,未嘗不先論諫。元和初,掌賦使院多擅禁系戶人,而有笞掠至死者。質乃論奏鹽鐵轉運司應決私鹽,系囚須
 與州府長吏監決。自是刑名畫一。憲宗以王承宗叛,用內官吐突承璀為招討使。質率同列伏閣論奏,言自古無以中官為將帥者。上雖改其名,心頗不悅,尋改質為太子左庶子。五年,坐與楊憑善,出為開州刺史。未幾卒。



 員工文辭,尚節義。杜亞為東都留守,闢為從事、檢校員外郎。早卒,有文集十卷。



 質兄弟俱有令譽而和粹,世以「滋味」目之:贊俗而有格,為酪;質美而多入,為酥;員為醍醐;賞為乳腐。近代士大夫言家法者,以穆氏為高。



 崔邠,字處仁,清河武城人。祖結,父倕,官卑。邠少舉進士,又登賢良方正科。貞元中授渭南尉。遷拾遺、補闕。常疏論裴延齡,為時所知。以兵部員外郎知制誥至中書舍人,凡七年。又權知吏部選事。明年,為禮部侍郎,轉吏部侍郎,賜以金紫。



 邠溫裕沉密,尤敦清儉。上亦器重之。裴垍將引為相,病難於承答,事竟寢。兄弟同時奉朝請者四人,頗以孝敬怡睦聞。後改太常卿,知吏部尚書銓事。故事,太常卿初上,大閱《四部樂》於署,觀者縱焉。邠自私
 第去帽,親導母輿,公卿逢者回騎避之,衢路以為榮。居母憂,歲餘卒,元和十年三月也,時年六十二。贈吏部尚書,謚曰文簡。



 弟鄯、郾、鄲等六人。子璀、璜,璀子彥融,皆登進士第,歷位臺閣。



 鄯少有文學,舉進士。元和中,歷監察御史。太和元年十月,自太子詹事拜左金吾衛大將軍。鄯昆弟六人,仕官皆至三品。邠、郾、鄲三人,知貢舉,掌銓衡。冠族聞望,為時名德。



 鄯太和九年冬,為左金吾大將軍,無病暴亡。不旬日有訓、注之亂,其亂始自金吾。君子
 乃知鄯之亡,崔氏積善之徵也。贈禮部尚書。子瑄。



 郾,字廣略。舉進士,平判入等,授集賢殿校書郎。三命升朝,為監察御史、刑部員外郎。資質秀偉,神情重雅,人望而愛之,終不可舍,不知者以為事高簡,拘靜默耳。居內憂,釋服為吏部員外。奸吏不敢欺,孤寒無援者未嘗留滯,銓敘之美,為時所稱。再遷左司郎中。



 元和十三年,鄭餘慶為禮儀詳定使,選時有禮學者共事,以郾為詳定判官、吏部郎中。十五年,遷諫議大夫。



 穆宗即位,荒於禽酒,坐
 朝常晚。郾與同列鄭覃等延英切諫。穆宗甚嘉之,畋游稍簡。長慶中,轉給事中。



 昭愍即位,選侍講學士,轉中書舍人。入思政殿謝恩,奏曰:「陛下用臣為侍講,半歲有餘,未嘗問臣經義。今蒙轉改,實慚尸素,有愧厚恩。」帝曰:「朕機務稍閑,即當請益。」高鉞曰:「陛下意雖樂善,既未延接儒生,天下之人,寧知重道?」帝深引咎,賜之錦彩。郾退,與同列高重抄撮《六經》嘉言要道,區分事類,凡十卷,名曰《諸經纂要》,冀人主易於省覽。上嘉之,賜錦彩二百匹、銀
 器等。



 其年轉禮部侍郎,東都試舉人。凡兩歲掌貢士,平心閱試,賞拔藝能,所擢者無非名士,至大中、咸通之代,為輔相名卿者十數人。出為陜州觀察使。舊弊有上供不足,奪吏俸以益之,歲八十萬,郾以廉使常用之直代之。居二年,政績聞於朝。遷鄂岳安黃等州觀察使。又五年,移浙西道都團練觀察使,至,用寬政安疲人。及居鄂渚,則峻法嚴刑,未常貰一死罪。江湖之間,萑蒲是叢,因造蒙沖小艦,上下千里,期月而盡獲群盜。凡三按廉車,
 率由清簡少事,財用有餘,遂寧泰。開成元年卒,年六十九,贈吏部尚書,謚曰德。



 郾與兄邠、弟鄲等皆有令譽。而郾疏財恢廓,昆仲所不及。子瑤、瑰、瑾、珮、璆。



 瑤,太和三年登進士第,出佐籓方,入升朝列,累至中書舍人。大中六年,知貢舉,旋拜禮部侍郎。出為浙西觀察使,又遷鄂州刺史、鄂岳觀察使,終於位。瑰、珮、璆官至郎署給諫。



 謹,大中十年登進士第,累居使府,歷尚書郎、知制誥。咸通十三年,知貢舉,選拔頗為得人。尋拜禮部侍郎,出為湖
 南觀察使。



 鄲,登進士第,累遷監察御史,三遷考功郎中。太和三年,以本官充翰林學士,轉中書舍人。六年,罷學士。八年,為工部侍郎、集賢殿學士,權知禮部,真拜兵部侍郎,本官判吏部東銓事。



 文宗勤於政道,每苦選曹訛弊,延英謂宰臣曰:「吏部殊不選才,安得摭實無濫,可厘革否?」李石對曰:「令錄可以商量,他官且宜循舊。」上曰:「循舊如配官耳,賢不肖安能甄別?」帝召三銓謂之曰:「卿等比選令錄,如何注擬?」鄲對曰:「資敘相當,問其為治之術,
 視可否而擬之。」帝曰:「依資合得,而才劣者何授?」對曰:「與邊遠慢官。」帝曰:「如以不肖之才治邊民,則疾苦可知也。凡朝廷求理,遠近皆須得人。茍非其才,人受其弊矣。」尋拜吏部侍郎。



 開成二年,出為宣州刺史,兼御史中丞、宣歙觀察使。四年,入為太常卿。七月,以本官同中書門下平章事,尋加中書侍郎、銀青光祿大夫。會昌初,李德裕用事,與鄲弟兄素善。鄲在相位累年,歷方鎮、太子師保卒。



 竇群,字丹列,扶風平陵人。祖亶,同昌郡司馬。父叔向,以工詩稱,代宗朝,官至左拾遺。群兄常、牟,弟鞏,皆登進士第,唯群獨為處士,隱居毗陵,以節操聞。及母卒,嚙一指置棺中,因廬墓次終喪。後學《春秋》於啖助之門人盧庇者,著書三十四卷,號《史記名臣疏》。貞元中,蘇州刺史韋夏卿以丘園茂異薦,兼獻其書,不報。及夏卿入為吏部侍郎,改京兆尹,中謝日,因對復薦群。徵拜左拾遺,遷侍御史,充入蕃使秘書監張薦判官。群因入對,奏曰:「陛下
 即位二十年,始自草澤擢臣為拾遺,是難其進也。今陛下以二十年難進之臣,用為和蕃判官,一何易也?」德宗異其言,留之,復為侍御史。



 王叔文之黨柳宗元、劉禹錫皆慢群,群不附之。其黨議欲貶群官,韋執誼止之。群嘗謁王叔文,叔文命撤榻而進。群揖之曰:「夫事有不可知者。」叔文曰:「如何?」群曰:「去年李實伐恩恃貴,傾動一時,此時公逡巡路旁,乃江南一吏耳。今公已處實形勢,又安得不慮路旁有公者乎?」叔文雖異其言,竟不之用。



 憲宗
 即位,轉膳部員外,兼侍御史知雜,出為唐州刺史。節度使于頔素聞其名,既謁見,群危言激切,頔甚悅。奏留充山南東道節度副使、檢校兵部郎中,兼御史中丞,賜紫金魚袋。宰相武元衡、李吉甫皆愛重之,召入為吏部郎中。元衡輔政,舉群代己為中丞。群奏刑部郎中呂溫、羊士諤為御史。吉甫以羊、呂險躁,持之數日不下,群等怒怨吉甫。



 三年八月,吉甫罷相,出鎮淮南,群等欲因失恩傾之。吉甫嘗召術士陳登宿於安邑里第。翌日,群令吏
 捕登考劾,偽構吉甫陰事,密以上聞。帝召登面訊之,立辯其偽。憲宗怒,將誅群等,吉甫救之,出為湖南觀察使。數日,改黔州刺史、黔州觀察使。在黔中,屬大水壞其城郭,復築其城,徵督溪洞諸蠻。程作頗急,於是,辰、錦生蠻乘險作亂,群討之不能定。六年九月,貶開州刺史。在郡二年,改容州刺史、容管經略觀察使。九年,詔還朝,至衡州病卒,時年五十。



 群性狠戾,頗復恩讎,臨事不顧生死。是時徵入,雲欲大用,人皆懼駭,聞其卒方安。二子:謙餘、
 審餘。



 兄常,字中行,大歷十四年登進士第,居廣陵之柳楊。結廬種樹,不求茍進,以講學著書為事,凡二十年不出。貞元十四年,鎮州節度使王武俊聞其賢,遣人致聘,闢為掌書記,不就。其年,杜佑鎮淮南,奏授校書郎,為節度參謀。元和六年,自湖南判官入為侍御史,轉水部員外郎。出為朗州刺史,歷固陵、潯陽、臨川三郡守。入為國子祭酒,求致仕。寶歷元年卒,時年七十。子弘餘,會昌中為黃州刺史。



 牟,字貽周。貞元二年登進士第,試秘書省
 校書郎、東都留守巡官。歷河陽、昭義從事,檢校水部郎中,賜緋,再為留守判官。入為都官郎中,出為澤州刺史,入為國子祭酒。長慶二年卒,時年七十四。子周餘,大中年秘書監。



 牟弟庠,字胃卿,釋褐國子主簿。吏部侍郎韓皋出鎮武昌,闢為推官。皋移鎮浙西,奏庠為節度副使、殿中侍御史,遷澤州刺史。又為宣歙副使,除奉天令、登州刺史、東都留守判官,歷信、婺二州刺史。卒年六十三。子繇、載。



 鞏,字友封,元和二年登進士第。袁滋鎮滑州,闢
 為從事。滋改荊、襄二鎮,皆從之,掌管記之任。平盧薛平又闢為副使。入朝,拜侍御史,歷司勛員外、刑部郎中。元稹觀察浙東,奉為副使、檢校秘書少監,兼御史中丞,賜金紫。稹移鎮武昌,鞏又從之。鞏能五言詩,昆仲之間,與牟詩俱為時所賞重。性溫雅,多不能持論,士友言議之際,吻動而不發,白居易等目為「囁嚅翁」。終於鄂渚,時年六十。子六人,景餘、師裕最知名。



 李遜,字友道,後魏申公發之後,於趙郡謂之申公房。曾
 祖進德,太子中允。祖珍玉,昌明令。父震,雅州別駕。世寓於荊州之石首。



 遜登進士第,闢襄陽掌書記。復從事於湖南,主其留務,頗有聲績,累拜池、濠二州刺史。先是,濠州之都將楊騰,削刻士卒,州兵三千人謀殺騰。騰覺之,走揚州,家屬皆死。濠兵不自戢,因行攘剽。及遜至郡,餘亂未殄。徐驅其間,為陳逆順利害之勢,眾皆釋甲請罪,因以寧息。觀察使旨限外征役,皆不從。入拜虞部郎中。



 元和初,出為衢州刺史。以政績殊尤,遷越州刺史,兼御
 史大夫、浙東都團練觀察使。先是,貞元初,皇甫政鎮浙東,嘗福建兵亂,逐觀察使吳詵。政以所鎮實壓閩境,請權益兵三千,俟賊平而罷。賊平向三十年,而所益兵仍舊。遜視事數日,舉奏停之。遜為政以均一貧富、扶弱抑強為己任,故所至稱理。



 九年,入為給事中。遜以舊制只日視事對群臣,遜奏論曰:「事君之義,有犯無隱。陳誠啟沃,不必擇辰。今群臣敷奏,乃候只日,是畢歲臣下睹天顏、獻可否能幾何?」憲宗嘉之,乃許不擇時奏對。俄遷戶
 部侍郎。



 元和十年,拜襄州刺史,充山南東道節度、觀察等使。襄陽前領八郡,唐、鄧、隋在焉。是時方討吳元濟,朝議以唐、蔡鄰接,遂以鄧隸唐州,三郡別為節制,命高霞寓領之,專俟攻討。遜以五州賦餉之。



 時孫代嚴綬鎮襄陽。綬以八州兵討賊在唐州。既而綬以無功罷兵柄,命高霞寓代綬將兵於唐州,其襄陽軍隸於霞寓。軍士家口在襄州者,遜厚撫之,士卒多舍霞寓亡歸。既而霞寓為賊所敗,乃移過於遜,言供饋不時。霞寓本出禁軍,內
 官皆佐之。既貶官,中人皆言遜撓霞寓軍,所以致敗。上令中使至襄州聽察曲直,奏言遜不直,乃左授太子賓客分司,又降為恩王傅。



 十三年,李師道效順,命遜為左散騎常侍,馳赴東平諭之。師道得詔意動,即請效順,旋為其下所惑而止。遜還,未幾,除京兆尹,改國子祭酒。



 十四年,拜許州刺史,充忠武節度、陳許溵蔡等州觀察處置等使。是時,新罹兵戰,難遽完緝。及遜至,集大軍與之約束,嚴具示賞罰必信,號令數百言,士皆感悅。



 長慶元
 年,幽、鎮繼亂。遜請身先討賊,不許。但命以兵一萬,會於行營。遜奉詔,即日發兵,故先諸軍而至,由是進位檢校吏部尚書。尋改鳳翔節度使,行至京師,以疾陳乞,改刑部尚書。長慶三年正月卒,年六十三,廢朝一日,贈右僕射。



 遜幼孤,寓居江陵。與其弟建,皆安貧苦,易衣並食,講習不倦。遜兄造,知二弟賢,日為營丐,成其志業。建先遜一年卒。兄弟同致休顯,士君子多之。謚曰恭肅。造早卒。



 建,字杓直,家素清貧,無舊業。與兄造、遜於荊南躬耕致
 養,嗜學力文。舉進士,選授秘書省校書郎。德宗聞其名,用為右拾遺、翰林學士。元和六年,坐事罷職,降詹事府司直。高郢為御史大夫,奏為殿中侍御史,遷兵部郎中、知制誥。自以草詔思遲,不願司文翰,改京兆尹。與宰相韋貫之友善。貫之罷相,建亦出為澧州刺史。徵拜太常少卿,尋以本官知禮部貢舉。建取舍非其人,又惑於請托,故其年選士不精,坐罰俸料。明年,除禮部侍郎,竟以人情不洽,改為刑部。



 建名位雖顯,以廉儉自處,家不理
 垣屋,士友推之。長慶二年二月卒,贈工部尚書。三子:訥、恪、樸。訥最知名,官至華州刺史、檢校尚書右僕射。



 薛戎,字元夫,河中寶鼎人。少有學術,不求聞達,居於毗陵之陽羨山。年餘四十,不易其操。江西觀察使李衡闢為從事,使者三返方應。故相齊映代衡,又留署職,府罷歸山。福建觀察使柳冕表為從事,累月,轉殿中侍御史。會泉州闕刺史,冕署戎權領州事。



 是時,姚南仲節制鄭滑,從事馬總以其道直為監軍使誣奏,貶泉州別駕。冕
 附會權勢,欲構成總罪,使戎按問曲成之。戎以總無辜,不從冕意,別白其狀。戎還自泉州,冕盛氣據衙而見賓客。戎遂歷東廂從容而入。冕度勢未可屈,徐起以見,一揖而退。又構其罪以狀聞,置戎於佛寺,環以武夫,恣其侵辱,如是累月,誘令成總之罪。操心如一,竟不動搖。杜佑鎮淮南,知戎之冤,乃上其表,發書諭冕,戎難方解,遂辭職寓居於江湖間。



 後閻濟美為福建觀察使,備聞其事,奏充副使。又隨濟美移鎮浙東,改侍御史,入拜刑部
 員外郎。出為河南令,累改衢、湖、常三州刺史,遷浙東觀察使。所蒞皆以政績聞。居數歲,以疾辭官。長慶元年十月卒,贈左散騎常侍。



 戎檢身處約,不務虛名。俸入之餘,散於宗族。身歿之後,人無譏焉。兄弟五人,季弟放最知名。



 放登進士第,性端厚寡言,於是非不甚系意。累佐籓府,蒞事幹敏。官至試大理評事,擢拜右拾遺,轉補闕,歷水部、兵部二員外,遷兵部郎中。



 遇憲宗以儲皇好書,求端士輔導經義,選充皇太子侍讀。及穆宗嗣位,未聽政
 間,放多在左右,密參機命。穆宗常謂放曰:「小子初承大寶,懼不克荷,先生宜為相,以匡不逮。」放叩頭曰:「臣實庸淺,獲侍冕旒,固不足猥塵大位。輔弼之任,自有賢能。」其言無矯飾,皆此類也。穆宗深嘉其誠,因召對思政殿,賜以金紫之服。轉工部侍郎、集賢學士。雖任非峻切,而恩顧轉隆。轉刑部侍郎,職如故。



 穆宗常謂侍臣曰:「朕欲習學經史,何先?」放對曰:「經者,先聖之至言,仲尼之所發明,皆天人之極致,誠萬代不刊之典也。史記前代成敗得
 失之跡,亦足鑒其興亡。然得失相參,是非無準的,固不可為經典比也。」帝曰:「《六經》所尚不一,志學之士,白首不能盡通,如何得其要?」對曰:「《論語》者《六經》之菁華,《孝經》者人倫之本。窮理執要,真可謂聖人至言。是以漢朝《論語》首列學官,光武令虎賁之士皆習《孝經》,玄宗親為《孝經》注解,皆使當時大理,四海乂寧。蓋人知孝慈,氣感和樂之所致也。」上曰:「聖人以孝為至德要道,其信然乎!」轉兵部侍郎、禮部尚書,判院事。



 放閨門之內,尤推孝睦,孤孀
 百口,家貧每不給贍,常苦俸薄。放因召對,懇求外任。其時偶以節制無闕,乃授以廉問。及鎮江西,惟用清潔為理,一方之人,至今思之。寶歷元年,卒於江西觀察使,廢朝一日。



 史臣曰:穆秘監之剛正不奪,如寒松倚巖,千丈勁節。而竇容州之敢決,如鷙鳥逐雀,英氣動人,巖穴之流,罕能及此。然矯激過當,君子不為。如塤如篪,不通不介,士行之美,崔氏諸子有焉。建、遜之貞方,戎、放之道義,元和已
 來,稱為令族,宜哉!



 贊曰:穆之贊、質,竇之常、群,跡參時傑,氣爽人文。二李英英,四崔濟濟。薛氏三門,難兄難弟。



\end{pinyinscope}