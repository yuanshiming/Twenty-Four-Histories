\article{卷一百三}

\begin{pinyinscope}

 ○崔日用從兄日知張嘉貞弟嘉祐
 蕭嵩子華張九齡仲方李適之子季卿嚴挺之



 崔日用,滑州靈昌人,其先自博陵徙家焉。進士舉,初為芮城尉。大足元年,則天幸長安,路次陜州。宗楚客時為
 刺史,日用支供頓事,廣求珍味,稱楚客之命,遍饋從官。楚客知而大加賞嘆,盛稱薦之,由是擢為新豐尉。無幾,拜監察御史。



 神龍中,秘書監鄭普思納女後宮,潛謀左道,日用遽奏劾之。普思方承恩,中宗不之省。日用廷爭懇至,詞甚抗直,普思竟伏其罪。時宗楚客、武三思、武延秀等遞為朋黨,日用潛皆附之,驟遷兵部侍郎兼修文館學士。中宗暴崩,韋庶人稱制,日用恐禍及己。知玄宗將圖義舉,乃因沙門普潤、道士王曄密詣籓邸,深自結
 納,潛謀翼戴。玄宗嘗謂曰:「今謀此舉,直為親,不為身。」日用曰:「此乃孝感動天,事必克捷。望速發,出其不意,若少遲延,或恐生變。」及討平韋氏,其夜,令權知雍州長史事。以功授銀青光祿大夫、黃門侍郎,參知機務,封齊國公,食實封二百戶。



 為相月餘,與中書侍郎薛稷不協,於中書忿競,由是轉雍州長史,停知政事。尋出為揚州長史,歷婺、汴二州刺史、兗州都督、荊州長史。因入奏事,言:「太平公主謀逆有期,陛下往在宮府,欲有討捕,猶是子道
 臣道,須用謀用力。今既光臨大寶,但須下一制,誰敢不從?忽奸宄得志,則禍亂不小。」上曰:「誠如此,直恐驚動太上皇,卿宜更思之。」日用曰:「臣聞天子孝與庶人孝全別。庶人孝,謹身節用,承順顏色;天子孝,安國家,定社稷。今若逆黨竊發,即大業都棄,豈得成天子之孝乎!伏請先定北軍,次收逆黨,即不驚動太上皇。」玄宗從其議。及討蕭至忠、竇懷貞之際,又令權檢校雍州長史,加實封通前滿四百戶。尋拜吏部尚書。



 日用嘗採《毛詩》、《大雅》、《小雅》
 二十篇及司馬相如《封禪書》,因上生日表上之,以申規諷,並述告成之事。手詔答曰:「夫詩者,動天地,感鬼神,厚於人,美於教矣。朕志之所尚,思與之齊,庶乎採詩之官,補朕之闕。且古者封禪,升中告成,朕以菲德,未明於至道。竦然以聽,頗壯相如之詞;惕然載懷,復慚夷吾之語。卿洽聞殫見,溫故知新,逮此發揮,益彰忠懇。豈非討蓬山之籍,心不忘於起予;因蘭殿之祥,言固深於啟沃,朕循環覽諷,用慰於懷。今賜卿衣裳一副、物五十段,以示
 無言不酬之信也。」



 尋出為常州刺史,削實封三百戶,轉汝州刺史。開元七年,差降口賦,特下敕曰:「唐元之際,逆黨構兇,崔日用當時潛論其事,及於戡翦,實預元謀,而所食之封,後以例減。功既居多,特宜準初食之封,與二百戶。」十年,轉並州大都督長史。尋卒,時年五十,贈吏部尚書,謚曰昭。後又贈荊州大都督,子宗之襲。



 日用才辯過人,見事敏速,每朝廷有事,轉禍為福,以取富貴。及先天已後,復求入相,竟亦不遂。常謂人曰:「吾一生行事,皆
 臨時制變,不必重專守始謀。每一念之,不覺芒刺在於背也。」



 日用從父兄日知,亦有吏乾。景雲中為洛州司馬。會譙王重福入東都作亂,群臣皆避難逃匿,日知獨督率人吏赴留守,與屯營合勢討賊。重福既死,以功加銀青光祿大夫,累遷京兆尹。坐贓為御史李如璧所劾,左遷歙縣丞,俄又歷遷殿中監。日知素與張說友善,說薦之,奏請授御史大夫,上不許。遂以為左羽林懷大將軍,而以河南尹崔隱甫為御史大夫,隱甫由是與說不葉。
 日知俄遷太常卿。自以歷任年久,每朝士參集,常與尚書同列,時人號為「尚書裏行」,遂為口實。開元十六年,出為潞州大都督府長史。尋以年老致仕,卒,謚曰襄。



 張嘉貞,蒲州猗氏人也。弱冠應五經舉,拜平鄉尉,坐事免歸鄉里。長安中,侍御史張循憲為河東採訪使,薦嘉貞材堪憲官,請以己之官秩授之。則天召見,垂簾與之言,嘉貞奏曰:「以臣草萊而得入謁九重,是千載一遇也。咫尺之間,如隔雲務,竟不睹日月,恐君臣之道有所未
 盡。」則天遽令卷簾,與語大悅,擢拜監察御史。累遷中書舍人,歷秦州都督、並州長史,為政嚴肅,甚為人吏所畏。



 開元初,因奏事至京師,上聞其善政,數加賞慰。嘉貞因奏曰:「臣少孤,兄弟相依以至今。臣弟嘉祐,今授鄯州別駕,與臣各在一方,同心離居,魂絕萬里。乞移就臣側近,臣兄弟盡力報國,死無所恨。」上嘉其友愛,特改嘉祐為忻州刺史。



 時突厥九姓新來內附,散居太原以北,嘉貞奏請置軍以鎮之,於是始於並州置天兵軍,以嘉貞為
 使。六年春,嘉貞又入朝。俄有告其在軍奢僭及贓賄者,御史大夫王晙因而劾奏之,按驗無狀,上將加告者反坐之罪。嘉貞奏曰:「昔者天子聽政於上,瞍賦矇誦,百工諫,庶人謗,而後天子斟酌焉。今反坐此輩,是塞言者之路,則天下之事無由上達。特望免此罪,以廣謗誦之道。」從之,遂令減死,自是帝以嘉貞為忠。嘉貞又嘗奏曰:「今志力方壯,是效命之秋,更三數年,即衰老無能為也。惟陛下早垂任使,死且不憚。」上以其明辯,尤重之。八年春,
 宋璟、蘇頲罷知政事,擢嘉貞為中書侍郎、同中書門下平章事。數月,加銀青光祿大夫,遷中書令。



 嘉貞斷決敏速,善於敷奏,然性強躁自用,頗為時論所譏。時中書舍人苗延嗣、呂太一、考功員外郎員嘉靜、殿中侍御史崔訓,皆嘉貞所引,位列清要,常在嘉貞門下共議朝政,時人為之語曰:「令公四俊,苗、呂、崔、員。」



 開元十年,車駕幸東都。有洛陽主簿王鈞為嘉貞修宅,將以求御史,因受贓事發,上特令朝堂集眾決殺之。嘉貞促所由速其刑以
 滅口,乃歸罪於御史大夫韋抗、中丞韋虛心,皆貶黜之。其冬,秘書監姜皎犯罪,嘉貞又附會王守一奏請杖之,皎遂死於路。俄而廣州都督裴伷先下獄,上召侍臣問當何罪,嘉貞又請杖之。兵部尚書張說進曰:「臣聞刑不上大夫,以其近於君也。故曰:『士可殺,不可辱。』臣今秋受詔巡邊,中途聞姜皎以罪於朝堂決杖,配流而死。皎官是三品,亦有微功。若其有犯,應死即殺,應流即流,不宜決杖廷辱,以卒伍待之。且律有八議,勛貴在焉。皎事已
 往,不可追悔。伷先只宜據狀流貶,不可輕又決罰。」上然其言。嘉貞不悅,退謂說曰:「何言事之深也?」說曰:「宰相者,時來即為,豈能長據?若貴臣盡當可杖,但恐吾等行當及之。此言非為伷先,乃為天下士君子也。」初,嘉貞為兵部員外郎,時張說為侍郎。及是,說位在嘉貞下,既無所推讓,說頗不平,因以此言激怒嘉貞,由是與說不葉。上又以嘉貞弟嘉祐為金吾將軍,兄弟並居將相之位,甚為時人之所畏憚。十一年,上幸太原行在所,嘉祐贓污
 事發。張說勸嘉貞素服待罪,不得入謁,因出為幽州刺史,說遂代為中書令。嘉貞惋恨,謂人曰:「中書令幸有二員,何相迫之甚也!」明年,復拜戶部尚書,兼益州長史,判都督事。敕嘉貞就中書省與宰相會宴,嘉貞既恨張說擠己,因攘袂勃罵,源乾曜、王晙共和解之。



 明年,坐與王守一交往,左轉臺州刺史。復代盧從願為工部尚書、定州刺史,知北平軍事,累封河東侯。將行,上自賦詩,詔百僚於上東門外餞之。至州,於恆岳廟中立頌,嘉貞自為
 其文,乃書於石,其碑用白石為之,素質黑文,甚為奇麗。先是,岳祠為遠近祈賽,有錢數百萬,嘉貞自以為頌文之功,納其數萬。十七年,嘉貞以疾請就醫東都,制從之。至都,目瞑無所見,上令醫人內直郎田休裕、郎將呂弘泰馳傳往省療之。其秋卒,年六十四,贈益州大都督。謚曰恭肅。



 嘉貞雖久歷清要,然不立田園。及在定州,所親有勸植田業者,嘉貞曰:「吾忝歷官榮,曾任國相,未死之際,豈憂饑餒?若負譴責,雖富田莊,亦無用也。比見朝士
 廣占良田,及身沒後,皆為無賴子弟作酒色之資,甚無謂也。」聞者皆嘆伏。



 初,嘉貞作相,薦萬年縣主簿韓朝宗,擢為監察御史。及嘉貞卒後十數歲,朝宗為京兆尹,因奏曰:「自陛下臨御已來,所用宰相,皆進退以禮,善始令終,身雖已沒,子孫咸在朝廷。唯張嘉貞晚年一子,今猶未登官序。」上亦惘然,遽令召之,賜名延賞,特拜左內率府兵曹參軍。德宗朝,位至宰輔,自有傳。



 嘉祐,有幹略,自右金吾將軍貶浦陽府折沖,至二十五年,為相州刺史。
 相州自開元已來,刺史死貶者十數人,嘉祐訪知尉遲迥周末為相州總管,身死國難,乃立其神祠以邀福。經三考,改左金吾將軍。後吳兢為鄴郡守,又加尉遲神冕服。自後郡守無患。



 蕭嵩,貞觀初左僕射、宋國公瑀之曾侄孫。祖鈞,中書舍人,有名於時。嵩美須髯,儀形偉麗。初,娶會稽賀晦女,與吳郡陸象先為僚婿。象先時為洛陽尉,宰相子,門望甚高。嵩尚未入仕,宣州人夏榮稱有相術,謂象先曰:「陸郎
 十年內位極人臣,然不及蕭郎一門盡貴,官位高而有壽。」時人未之許。



 神龍元年,嵩調補洺州參軍。尋而侍中、扶陽王桓彥範出為洺州刺史,見之推重,待以殊禮。景雲元年,為醴泉尉。時陸象先已為中書侍郎,引為監察御史。及象先知政事,嵩又驟遷殿中侍御史。開元初,為中書舍人。與崔琳、王丘、齊澣同列,皆以嵩寡學術,未異之,而紫微令姚崇許其致遠,眷之特深。歷宋州刺史,三遷為尚書左丞、兵部侍郎。



 十五年,涼州刺史、河西節度
 王君恃眾每歲攻擊吐蕃。吐蕃大將悉諾邏恭祿及燭龍莽布支攻陷瓜州城,執刺史田元獻及君父壽,盡取城中軍資及倉糧,仍毀其城而去。又攻玉門軍及常樂縣,縣令賈師順嬰城固守,賊遂引退。無何,君又為回紇諸部殺之於鞏筆驛,河、隴震駭。玄宗以君勇將無謀,果及於難,擇堪邊任者,乃以嵩為兵部尚書、河西節度使,判涼州事。嵩乃請以裴寬、郭虛己、牛仙客在其幕下,又請以建康軍使、左金吾將軍張守珪為瓜州
 刺史,修築州城,招輯百姓,令其復業。又加嵩銀青光祿大夫。時悉諾邏恭祿威名甚振,嵩乃縱反間於吐蕃,言其與中國潛通,贊普遂召而誅之。明年秋,吐蕃大下,悉末明復率眾攻瓜州,守珪出兵擊走之。隴右節度使、鄯州都督張志亮引兵至青海西南馮波谷,與吐蕃接戰,大破之。八月,嵩又遣副將杜賓客率弩手四千人,與吐蕃戰於祁連城下,自晨至暮,散而復合,賊徒大潰,臨陣斬其副將一人,散走山谷,哭聲四合。露布至,玄宗大悅,
 乃加嵩同中書門下三品,恩顧莫比。



 十七年,授宇文融、裴光庭宰相,又加嵩兼中書令。自十四年燕國公張說罷中書令後,缺此位四年,而嵩居之。常帶河西節度,遙領之。加集賢殿學士、知院事,兼修國史,進位金紫光祿大夫。子衡,尚新昌公主,嵩夫人賀氏入覲拜席,玄宗呼為親家母,禮儀甚盛。尋又進封徐國公。二十一年二月,侍中裴光庭卒。光庭與嵩同位數年,情頗不協,及是,玄宗遣嵩擇相,嵩以右丞韓休長者,舉之。及休入相,嵩舉
 事,休峭直,輒不相假,互於玄宗前論曲直,因讓位。玄宗眷嵩厚,乃許嵩授尚書右丞相,令罷相,以休為工部尚書。尋又以嵩子華為給事中。



 二十四年,拜太子太師。及幽州節度使張守珪坐賂遺中官牛仙童,貶為括州刺史,嵩嘗賄仙童,李林甫發之,貶青州刺史。尋又追拜太子太師,嵩又請老。嵩性好服餌,及罷相,於林園植藥,合煉自適。華時為工部侍郎,衡以主婿三品,嵩皤然就養十餘年,家財豐贍,衣冠榮之。天寶八年薨,年八十餘,贈
 開府儀同三司。



 子華,天寶末轉兵部侍郎。祿山之亂,從駕不及,陷賊,偽署魏州刺史。乾元元年,郭子儀與九節度之師渡河攻安慶緒於相州,華潛通表疏,俟官軍至為內應。賊伺知之,禁錮華於獄。崔光遠收魏州,破械出華。魏人美華之惠政,詣光遠請留,朝廷正授魏州刺史。既而史思明率眾南下,子儀懼華復陷,乃表崔光遠代華,召至軍中。及相州兵潰,華歸京,仍以偽命所污,降授試秘書少監。華謹重方雅,綽有家法,人士稱之。尋遷尚
 書右丞。乾元二年,出為河中尹、河中晉絳節度使。



 上元元年十二月,制曰:「弼予之選,審象是求,天步未平,廟謨尤切。必資明表,佇以佐時,畫一之才,取則不遠。正議大夫、前河中尹、兼御史中丞、充本府晉絳等州節度觀察等使、上柱國、嗣徐國公、賜紫金魚袋蕭華,公輔成名,承家繼業,詞標麗則,德蘊謨明。再履宮坊,尤知至行,致君望美,閱相求能。且推伊陟之賢,更啟漢臣之閤,還依日月,佐理陰陽。俾參政於紫宸,用建中於皇極。可中書侍
 郎、同中書門下平章事、集賢殿崇文館大學士,監修國史。」



 時中官李輔國專典禁兵,怙寵用事,求為宰相,諷宰臣裴冕等薦己,華頗拒之,輔國怒。肅宗方寢疾,輔國矯命罷華相位,守禮部尚書,仍引元載代華。肅宗崩,代宗在諒暗,元載希輔國旨,貶華為硤州員外司馬,卒於貶所。



 衡子復,德宗朝位亦至宰輔。華子恆、悟。恆子俯,大和中宰輔。悟子仿,咸通中宰輔,皆自有傳。



 張九齡,字子壽,一名博物。曾祖君政,韶州別駕,因家於
 始興,今為曲江人。父弘愈,以九齡貴,贈廣州刺史。九齡幼聰敏,善屬文。年十三,以書乾廣州刺史王方慶,大嗟賞之,曰:「此子必能致遠。」登進士第,應舉登乙第,拜校書郎。玄宗在東宮,舉天下文藻之士,親加策問,九齡對策高第,遷右拾遺。時帝未行親郊之禮,九齡上疏曰:



 伏以天才者,百神之君,而王者之所由受命也。自古繼統之主,必有郊配之義,蓋以敬天命以報所受。故於郊之義,則不以德澤未洽,年穀不登,凡事之故,而闕其禮。《孝經》云:「
 昔者周公郊祀后稷以配天。」斯謂成王幼沖,周公居攝,猶用其禮,明不暫廢。漢丞相匡衡亦云:「帝王之事,莫重乎郊祀。」董仲舒又云:「不郊而祭山川,失祭之序,逆於禮正,故《春秋》非之。」臣愚以為匡衡、仲舒,古之知禮者,皆謂郊之為祭所宜先也。伏惟陛下紹休聖緒,其命惟新,御極已來,於今五載,既光太平之業,未行大報之禮,竊考經傳,義或未通。今百穀嘉生,鳥獸咸若,夷狄內附,兵革用寧。將欲鑄劍為農,泥金封禪,用彰功德之美,允答神只
 之心。能事畢行,光耀帝載。況郊祀常典,猶闕其儀,有若怠於事天,臣恐不可以訓。伏望以迎日之至,展焚柴之禮,升紫壇,陳採席,定天位,明天道,則聖朝典則,可謂無遺矣。



 九齡以才鑒見推,當時吏部試拔萃選人及應舉者,咸令九齡與右拾遺趙冬曦考其等第,前後數四,每稱平允。開元十年,三遷司勛員外郎。時張說為中書令,與九齡同姓,敘為昭穆,尤親重之,常謂人曰:「後來詞人稱首也。」九齡既欣知己,亦依附焉。十一年,拜中書舍
 人。



 十三年,車駕東巡,行封禪之禮。說自定侍從升中之官,多引兩省錄事主書及己之所親攝官而上,遂加特進階,超授五品。初,令九齡草詔,九齡言於說曰:「官爵者,天下之公器,德望為先,勞舊次焉。若顛倒衣裳,則譏謗起矣。今登封霈澤,千載一遇。清流高品,不沐殊恩。胥吏末班,先加章紱。但恐制出之後,四方失望。今進草之際,事猶可改,唯令公審籌之,無貽後悔也。」說曰:「事已決矣,悠悠之談,何足慮也!」竟不從。及制出,內外甚咎於說。時
 御史中丞宇文融方知田戶之事,每有所奏,說多建議違之,融亦以此不平於說。九齡復勸說為備,說又不從其言。無幾,說果為融所劾,罷知政事,九齡亦改太常少卿,尋出為冀暫刺史。九齡以母老在鄉,而河北道里遼遠,上疏固請換江南一州,望得數承母音耗,優制許之,改為洪州都督。俄轉桂州都督,仍充嶺南道按察使。上又以其弟九章、九皋為嶺南道刺史,令歲時伏臘,皆得寧覲。



 初,張說知集賢院事,常薦九齡堪為學士,以備顧
 問。說卒後,上思其言,召拜九齡為秘書少監、集賢院學士,副知院事。再遷中書侍郎。常密有陳奏,多見納用。尋丁母喪歸鄉里。二十一年十二月,起復拜中書侍郎、同中書門下平章事。明年,遷中書令,兼修國史。時範陽節度使張守珪以裨將安祿山討奚、契丹敗衄,執送京師,請行朝典。九齡奏劾曰:「穰苴出軍,必誅莊賈;孫武教戰,亦斬宮嬪。守珪軍令必行,祿山不宜免死。」上特舍之。九齡奏曰:「祿山狼子野心,面有逆相,臣請因罪戮之,冀絕
 後患。」上曰:「卿勿以王夷甫知石勒故事,誤害忠良。」遂放歸籓。



 二十三年,加金紫光祿大夫,累封始興縣伯。李林甫自無學術,以九齡文行為上所知,心頗忌之。乃引牛仙客知政事,九齡屢言不可,帝不悅。二十四年,遷尚書右丞相,罷知政事。後宰執每薦引公卿,上必問:「風度得如九齡否?」故事皆搢笏於帶,而後乘馬,九齡體羸,常使人持之,因設笏囊。笏囊之設,自九齡始也。



 初,九齡為相,薦長安尉周子諒為監察御史。至是,子諒以妄陳休咎,
 上親加詰問,令於朝堂決殺之。九齡坐引非其人,左遷荊州大都督府長史。俄請歸拜墓,因遇疾卒,年六十八,贈荊州大都督,謚曰文獻。九齡在相位時,建議復置十道採訪使,又教河南數州水種稻,以廣屯田。議置屯田,費功無利,竟不能就,罷之。性頗躁急,動輒忿詈,議者以此少之。



 子拯,伊闕令。祿山之亂陷賊,不受偽命。兩京克復,詔加太子右贊善。弟九皋,自尚書郎歷唐、徐、宋、襄、廣五州刺史。九章,歷吉、明、曹三州刺史,鴻臚卿。



 九齡為中書
 令時,天長節百僚上壽,多獻珍異,唯九齡進《金鏡錄》五卷,言前古興廢之道,上賞異之。又與中書侍郎嚴挺之、尚書左丞袁仁敬、右庶子梁升卿、御史中丞盧怡結交友善。挺之等有才幹,而交道終始不渝,甚為當時之所稱。至德初,上皇在蜀,思九齡之先覺,下詔褒贈,曰:「正大廈者柱石之力,昌帝業者輔相之臣。生則保其榮名,歿乃稱其盛德,節終未允於人望,加贈實存乎國章。故中書令張九齡,維嶽降神,濟川作相,開元之際,寅亮成功。
 讜言定其社稷,先覺合於蓍策,永懷賢弼,可謂大臣。竹帛猶存,樵蘇必禁,爰從八命之秩,更進三臺之位。可贈司徒,仍遣使就韶州致祭。」有集二十卷。



 九皋曾孫仲方,少朗秀。為兒童時,父友高郢見而奇之,曰;「此子非常,必為國器,吾獲高位,必振發之。」後郢為御史大夫,首請仲方為御史。歷金州刺史。郡人有田產為中人所奪,仲方三疏奏聞,竟理其冤。入為度支郎中,駁李吉甫謚,吉甫之黨惡之,出為遂州司馬。稍遷復、曹、鄭三郡守。為諫議
 大夫。時鄠縣令崔發因辱小黃門,敬宗赫怒,付臺推鞫。及元日大赦,獨發不得宥。仲方上疏,其略曰:「鴻恩將布於天下,而不行御前;霈澤始被於昆蟲,而獨遺崔發。」由是發得不死,時論美之。大和九年,為京兆尹,將相從累者皆大戮,仲方密令識之。旋詔下許令收葬,得認遺骸,實仲方之力也。是時軍人橫恣,仲方脂韋,坐不稱職,出為華州刺史,改秘書監。開成二年卒,年七十二,贈禮部尚書,謚曰成。



 李適之,一名昌,恆山王承乾之孫也。父象,官至懷州別駕。適之神龍初起家拜左衛郎將。開元中,累遷通州刺史,以強幹見稱。時給事中韓朝宗為按察使,特表薦之,擢拜秦州都督。俄轉陜州刺史,入為河南尹。適之性簡率,不務苛細,人吏便之。歲餘,拜御史大夫。開元二十七年,兼幽州大都督府長史,知節度事。適之以祖得罪見廢,父又遭則天所黜,葬禮有闕,上疏請歸葬昭陵之闕內。於是下詔追贈承乾為恆山愍王,象為越州都督、郇
 國公,伯父厥及亡兄數人並有褒贈。數喪同至京師,葬禮甚盛,仍刊石於墳所。俄拜刑部尚書。適之雅好賓友,飲酒一斗不亂,夜則宴賞,晝決公務,庭無留事。



 天寶元年,代牛仙客為左相,累封清和縣公。與李林甫爭權不葉,適之性疏,為其陰中。林甫嘗謂適之曰:「華山有金鑛,採之可以富國,上未之知。」適之心善其言,他日從容奏之。玄宗大悅,顧問林甫,對曰:「臣知之久矣。然華山陛下本命,王氣所在,不可穿鑿,臣故不敢上言。」帝以為愛己,
 薄適之言疏。隴右節度皇甫惟明、刑部尚書韋堅、戶部尚書裴寬、京兆尹韓朝宗,悉與適之善,林甫皆中傷之,構成其罪,相繼放逐。適之懼不自安,求為散職。五載,罷知政事,守太子少保。遽命親故歡會,賦詩曰:「避賢初罷相,樂聖且銜杯。為問門前客,今朝幾個來?」竟坐與韋堅等相善,貶宜春太守。後御史羅希奭奉使殺韋堅、盧幼臨、裴敦復、李邕等於貶所,州縣且聞希奭到,無不惶駭。希奭過宜春郡,適之聞其來,仰藥而死。



 子季卿,弱冠舉
 明經,頗工文詞。應制舉,登博學宏詞科,再遷京兆府鄠縣尉。肅宗朝,累遷中書舍人,以公事坐貶通州別駕。代宗即位,大舉淹抑,自通州徵為京兆少尹。尋復中書舍人,拜吏部侍郎。俄兼御史大夫,奉使河南、江淮宣慰,振拔幽滯,進用忠廉,時人稱之。在銓衡數年,轉右散騎常侍。季卿有宇量,性識博達,善與人交,襟懷豁如。其在朝以進賢為務,士以此多之。大歷二年卒,贈禮部尚書。



 孫融,立性嚴整,善吏事。貞元十年,歷官至渭州節度使卒。



 嚴挺之,華州華陰人。叔父方嶷,景雲中戶部郎中。挺之少好學,舉進士。神龍元年,制舉擢第,授義興尉。遇姚崇為常州刺史,見其體質昂藏,雅有吏乾,深器異之。及崇再入為中書令,引挺之為右拾遺。



 睿宗好樂,聽之忘倦,玄宗又善音律。先天二年正月望,胡僧婆陀請夜開門燃百千燈,睿宗御延喜門觀樂,凡經四日。又追作先天元年大酺,睿宗御安福門樓觀百司酺宴,以夜繼晝,經月餘日。挺之上疏諫曰:



 微臣竊惟陛下應天順人,發號
 施令,躬親大禮,昭布鴻澤,孜孜庶政,業業萬幾。蓋以天下心為心,深戒安危之理,此誠堯、舜、禹、湯之德教也。奈何親御城門,以觀大酺,累日兼夜,臣愚竊所未諭。



 夫酺者,因人所利,合醵為歡,無相奪倫,不至糜弊。且臣卜其晝,史冊攸存,君舉必書,帝王重慎。今乃暴衣冠於上路,羅妓樂於中宵。雜鄭、衛之音,縱倡優之樂。陛下還淳復古,宵衣旰食,不矜細行,恐非聖德所宜。臣以為一不可也。誰何警夜,伐鼓通晨,以備非常,存之善教。今陛下不
 深惟戒慎,輕違動息,重門弛禁,巨猾多徒。倘有躍馬奔車,流言駭叫,一塵聽覽,有累宸衷。臣以為二不可也。且一人向隅,滿堂不樂;一物失所,納隍增慮。陛下北宮多暇,西墉暫臨。青春日長,已積埃塵之弊;紫微漏永,重窮歌舞之樂。倘令有司跛倚,下人饑倦,以陛下近猶不恤,而況於遠乎!聖情攸聞,豈不懍然只畏。臣以為三不可也。且元正首祚,大禮頻光,百姓顒顒,咸謂業盛配天,功垂曠代。今陛下恩似薄於眾望,酺即過於往年。王公貴
 人,各承微旨;州縣坊曲,競為課稅。籲嗟道路,貿易家產,損萬人之力,營百戲之資。適欲同其歡,而乃遺其患,復令兼夜,人何以堪?臣以為四不可也。



 《書》曰:「罔咈百姓,以從己之欲。」況自去夏霪霖,經今亢旱,農乏收成,市有騰貴。損其實,崇其虛,馳不急之務,擾方春之業。前代聖主明王,忽於細微而成過患多矣,陛下可效之哉?伏望晝則歡娛,暮令休息,要令兼夜,恐無益於聖朝。



 上納其言而止。



 時侍御史任知古恃憲威,於朝行詬詈衣冠,挺之
 深讓之,以為不敬,乃為臺司所劾,左遷萬州員外參軍。開元中,為考功員外郎。典舉二年,大稱平允,登科者頓減二分之一。遷考功郎中,特敕又令知考功貢舉事,稍遷給事中。時黃門侍郎杜暹、中書侍郎李元紘同列為相,不葉。暹與挺之善,元紘素重宋遙,引為中書舍人。及與起居舍人張咺等同考吏部等第判,遙復與挺之好尚不同,遙言於元紘。元紘詰譙挺之,挺之曰:「明公位尊國相,情溺小人,乃有憎惡,甚為不取也。」詞色俱厲。元紘
 曰:「小人為誰?」挺之曰:「即宋遙也。」因出為登州刺史、太原少尹。殿中監王毛仲使太原、朔方、幽州,計會兵馬,事隔數年,乃牒太原索器仗。挺之以不挾敕,毛仲寵幸久,恐有變故,密奏之。尋遷濮、汴二州刺史。挺之所歷皆嚴整,吏不敢犯,及蒞大郡,人乃重足側息。



 二十年,毛仲得罪賜死,玄宗思曩日之奏,擢為刑部侍郎,深見恩遇,改太府卿。與張九齡相善,九齡入相,用挺之為尚書左丞,知吏部選,陸景融知兵部選,皆為一時精選。時侍中裴耀
 卿、禮部尚書李林甫與九齡同在相位,九齡以詞學進,入視草翰林,又為中書令,甚承恩顧。耀卿與九齡素善,林甫巧密,知九齡方承恩遇,善事之,意未相與。林甫引蕭炅為戶部侍郎,嘗與挺之同行慶吊,客次有《禮記》,蕭炅讀之曰:「蒸嘗伏獵。」炅早從官,無學術,不識「伏臘」之意,誤讀之。挺之戲問,炅對如初。挺之白九齡曰:「省中豈有『伏獵侍郎。』」由是出為岐州刺史,林甫深恨之。九齡嘗欲引挺之同居相位,謂之曰:「李尚書深承聖恩,足下宜一
 造門款狎。」挺之素負氣,薄其為人,三年,非公事竟不私造其門,以此彌為林甫所嫉。及挺之囑蔚州刺史王元琰,林甫使人詰於禁中,以此九齡罷相,挺之出為洺州刺史,二十九年,移絳郡太守。



 天寶元年,玄宗嘗謂林甫曰:「嚴挺之何在?此人亦堪進用。」林甫乃召其弟損之至門敘故,云「當授子員外郎」,因謂之曰:「聖人視賢兄極深,要須作一計,入城對見,當有大用。」令損之取絳郡一狀,云:「有少風氣,請入京就醫。」林甫將狀奏云:「挺之年高,近
 患風,且須授閑官就醫。」玄宗嘆叱久之。林甫奏授員外詹事,便令東京養疾。



 挺之素歸心釋典,事僧惠義。及至東都,鬱鬱不得志,成疾。自為墓志曰:「天寶元年,嚴挺之自絳郡太守抗疏陳乞,天恩允請,許養疾歸閑,兼授太子詹事。前後歷任二十五官,每承聖恩,嘗忝獎擢,不盡驅策,駑蹇何階,仰答鴻造?春秋七十,無所展用,為人士所悲。其年九月,寢疾,終於洛陽某里之私第。十一月,葬於大照和尚塔次西原,禮也。盡忠事君,叨載國史,勉拙
 從仕,或布人謠。陵穀可以自紀,文章焉用為飾。遺文薄葬,斂以時服。」挺之與裴寬皆奉佛。開元末,惠義卒,挺之服縗麻送於龕所。寬為河南尹,僧普寂卒,寬與妻子皆服縗絰,設次哭臨,妻子送喪至嵩山。故挺之志文云「葬於大照塔側」,祈其靈祐也。挺之素重交結,有許與,凡舊交先歿者,厚撫其妻子,凡嫁孤女數十人,時人重之。



 子武,廣德中黃門侍郎、成都尹、劍南節度使。



 史臣曰:崔日用附會三思,以取高位,預討韋氏,遂握重
 權。自言「吾一生行事,皆臨時制變,不必專守始謀」,信矣。與夫守死善道者,不可同年而語也。張嘉貞雖不立田園,奈急於勢利,朋比近習,杖姜皎、伷先,非中立之士也。蕭嵩位極中令,異政無聞,樹破虜之勛,真致遠之器。九齡文學政事,咸有所稱,一時之選也。適之臨下雖簡,在公克勤,惜乎不得其死也!挺之才略器識,不下諸公,恥近權門,為人所惡,不登臺輔,養疾宮僚。雖富貴在天,窮達有命,彼林甫者,誠可投畀豺虎也。



 贊曰:開元之代,多士盈庭。日用無守,嘉貞近名。嵩、齡、適、挺,各有度程。大位俱極,半慚德馨。



\end{pinyinscope}