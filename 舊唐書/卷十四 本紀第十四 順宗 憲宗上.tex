\article{卷十四 本紀第十四 順宗 憲宗上}

\begin{pinyinscope}

 順
 宗至德大聖
 大安孝丘帝諱誦,德
 宗長子,母昭德皇后王氏。上元二年正月生於長安之東內。大歷十四年六月,封宣王。建中元年正月丁卯,立為皇太子。



 上元二
 十一年正月癸巳,德宗崩,丙申,即位於太極殿。上自二十年九月風病,不能言,暨德宗不豫,諸王親戚皆侍醫藥,獨上臥病不能侍。德宗彌留,思見太子,涕咽久之。大行發喪,人情震懼。上力疾衰服,見百僚於九仙門。既即位,知社稷有奉,中外始安。庚子,群臣上書請聽政。



 二月辛丑朔。甲申,以河陽三城行軍司馬元韶為懷州刺史、河陽懷州節度使。丙午,罷翰林醫工、相工、占星、射覆、冗食者四十二人。己酉,以易定張茂昭兼同平章事,以來
 朝,故寵之。是夜,太白犯昴。辛卯,以吏部郎中韋執誼為尚書左丞、同中書門下平章事。辛酉,貶京兆尹李實通州長史,尋卒。壬子,淄青李師古以兵寇滑之東鄙,聞國喪也。甲寅,釋仗內囚嚴懷志、呂溫等一十六人。平涼之盟陷蕃。久之得還,以習蕃中事,不欲令出外,故囚之仗內,至是方釋之。日本國王並妻還蕃,賜物遣之。壬寅,以太子侍書、翰林待詔王伾為左散騎常侍,充翰林學士。以前司功參軍、翰林待詔王叔文為起居舍人,充翰
 林學士。以鴻臚卿王權為京兆尹。甲子,御丹鳳樓,大赦天下。諸道除正敕率稅外,諸色榷稅並宜禁斷;除上供外,不得別有進奉。百姓九十已上,賜米二石,絹兩匹,版授上佐、縣君,仍令本部長吏就家存問;百歲已上,賜米五石,絹二匹,綿一屯,羊酒,版授下州刺史、郡君。戊辰,以開府儀同三司、檢校太尉、使持節、大都督雞林州諸軍事、雞林州刺史、上柱國、新羅王金重熙兼寧海軍使,以重熙母和氏為太妃,妻樸氏為妃。



 三月庚午,出宮女三
 百人於安國寺,又出掖庭教坊女樂六百人於九仙門,召其親族歸之。戊寅,以韋皋兼檢校太尉,李師古、劉濟兼檢校司空。張茂昭司徒。丙戌,檢校司空、同平章事杜祐為度支鹽鐵使。戊子,徐州節度賜名武寧軍。蔡州吳少誠兼同平章事。以翰林學士王叔文為度支鹽鐵轉運副使。杜祐雖領使名,其實叔文專總。宰相賈耽兼檢校司空,鄭瑜吏部尚書,高郢刑部尚書,韋執誼中書侍郎,鎮冀王士真、淮南王鍔、魏博田季安皆檢校司空。癸
 巳,詔冊廣陵郡王淳為皇太子,改名純。



 夏四月壬寅,制第十弟諤封欽王,第十一弟諴封珍王。男建康郡王渙封郯王,改名經;洋川郡王沔封均王,改名緯;臨淮郡王洵封漵王,改名縱;弘農王浼封莒王,改名紓;漢東郡王泳封密王,改名綢;晉陵郡王堤封郇王,改名總;高平郡王漵封邵王,改名約;雲安郡王滋封宋王,改名結;宣城郡王淮封集王,改名緗;德陽郡王湑封冀王,改名絿;河東郡王浥封和王,改名綺。十七男絢封衡王,十九男纁封會王,二十男綰封福王,二
 十一紘男封撫王,二十三男緄封嶽王,二十四男紳封袁王,二十五男綸封桂王,二十七男繟封翼王,彌臣國嗣王道勿禮封彌臣國王。西平郡王晟男左羽林大將軍願襲封岐公,食邑三千戶。戊申,詔以冊太子禮畢,赦京城系囚,大闢降從流,流以下減一等。以給事中陸質、中書舍人崔樞並為太子侍讀。庚戌,封太子男寧、寬、宥、察、寰、寮等六人為郡王,並食邑三千戶。癸丑,贈吐蕃使、工部侍郎、兼御史大夫張薦禮部尚書。丙寅,罷萬安
 監牧。戊辰,以杭州刺史韓皋為尚書右丞。



 五月己巳,以右金吾衛大將軍範希朝為右神策統軍,充左右神策、京西諸城鎮行營兵馬節度使。丁丑,以邕管經略使韋丹為河南少尹,以萬年縣令房啟為容管經略招討使。癸未,以郴州司馬鄭餘慶為尚書左丞。甲辰,以檢校司空、忽汗州都督、渤海國王大嵩璘檢校司徒。承徽王氏、趙氏可昭儀,崔氏、楊氏可充儀,王氏可昭媛,王氏可昭容,牛氏可修儀,張氏可美人。以右丞韓皋為鄂岳沔蘄
 都團練觀察使。丁亥,升襄州為大都督府。臨漢縣仍徙於鄧城。辛卯,以鹽鐵轉運使副王叔文為戶部侍郎。



 六月丙申,詔二十一年十月已前百姓所欠諸色課利、租賦、錢帛,共五十二萬六千八百四十一貫、石、匹、束,並宜除免。七月戊辰朔,吐蕃使論悉諾來朝貢。丙子,鄆州李師古加檢校侍中。贈故忠州別駕陸贄兵部尚書,謚曰宣;贈故道州刺史陽城為左散騎常侍。戊寅,以戶部侍郎潘孟陽為度支鹽鐵轉運使副。丙戌,關東蝗食田稼。
 癸已,橫海軍節度伎、滄州刺史程懷信卒,以其子副使執恭起復滄州刺史、橫海軍節度使。甲午,度支使杜祐奏:「太倉見米八十萬石,貯來十五年,東渭橋米四十五萬石,支諸軍皆不悅。今歲豐阜,請權停北河轉運,於濱河州府和糴二十萬石,以救農傷之弊。」乃下百僚議,議者同異,不決而止。乙未,詔:「朕承九聖之烈,荷萬邦之重。顧以寡德,涉道未明,虔恭寅畏,懼不克荷。恐上墜祖宗之訓,下貽卿士之憂,夙夜祗勤,如臨淵谷。而積疾未復,
 至於經時,怡神保和,常所不暇。永惟四方之大,萬務之殷,不躬不親,慮有曠廢。加以山陵有日,霖潦逾旬,是用儆于朕心,以答天戒。其軍國政事,宜令皇太子勾當。」時上久疾,不復延納宰臣共論大政。事無巨細皆決於李忠言、王人丕、王叔文。物論喧雜,以為不可。籓鎮屢上箋於皇太子,指三豎之撓政,故有是詔。以太常卿杜黃裳為門下侍郎,左金吾衛大軍袁滋為中書侍郎,並同中書門下平章事;鄭珣瑜為吏部尚書,高郢刑部尚書,並
 罷知政事。皇太子見百僚於朝堂。丙申,皇太子於麟德殿西亭見奏事官。



 八月丁酉朔。庚子,詔:「惟皇天佑命烈祖,誕受方國,九聖儲祉,萬邦咸休。肆予一人,獲績丕業,嚴恭守位,不遑暇逸。而天佑不降,疾恙無瘳,將何以奉宗廟之靈,展郊、禋之禮!疇咨庶尹,對越上玄,內愧於朕心,上畏於天命。夙夜祗心慄,深惟永圖。一日萬機,不可以久曠;天工人代,不可以久違。皇太子純睿哲溫文,寬和仁惠,孝友之德,愛敬之誠,通乎神明,格於上下。是用法皇
 王至公之道,遵父子傳歸之制,付之重器,以撫兆人。必能宣祖宗之重光,荷天地之休命,奉若成憲,永綏四方。宜令皇太子即皇帝位,朕稱太上皇,居興慶宮,制稱誥。」辛丑,誥:「有天下傳歸於子,前王之制也。欽若大典,斯為至公,式揚耿光,用體文德。朕獲奉宗廟,臨御萬方,降疾不瘳,庶政多闕。乃命元子,代予守邦,爰以令辰,光膺冊禮,宜以今月九日冊皇帝於宣政殿。國有大命,恩俾惟新,宜因紀元之慶,用覃在宥之澤。宜改貞元二十一年
 為永貞元年。自貞元二十一年八月五日已前,天下死罪降從流,流以下遞減一等。」誥立良娣王氏為太上皇后,良媛董氏為太上皇德妃。壬寅,貶右散騎常侍王伾為開州司馬,前戶部侍郎、度支鹽鐵轉運使王叔文為渝州司戶。



 元和元年正月丙寅朔,皇帝率百僚上太上皇尊號曰應乾聖壽。甲申,太上皇崩於興慶宮之咸寧殿,享年四十六歲。六月乙卯,皇帝率群臣上大行太上皇謚曰至德大聖大安孝皇帝,廟號順宗。秋七月壬申,
 葬於豐陵。



 史臣韓愈曰:順宗之為太子也,留心藝術,善隸書。德宗工為詩,每賜大臣方鎮詩制,必命書之。性寬仁有斷,禮重師傅,必先致拜。從幸奉天,賊泚逼迫,常身先禁旅,乘城拒戰,督勵將士,無不奮激。德宗在位歲久,稍不假權宰相。左右幸臣如裴延齡、李齊運、韋渠牟等,因間用事,刻下取功,而排陷陸贄、張滂輩,人不敢言,太子從容論爭,故卒不任延齡、渠牟為相。嘗侍宴魚藻宮。張水嬉,彩
 監雕靡,宮人引舟為棹歌,絲竹間發,德宗歡甚,太子引詩人「好樂無荒」為對。每於敷奏,未嘗以顏色假借宦官。居儲位二十年,天下陰受其賜。惜乎寢疾踐祚,近習弄權;而能傳政元良,克昌運祚,賢哉!



 憲宗聖神章武孝皇帝諱純,順宗長子也,母曰莊憲王太后。大歷十三年二月生於長安之東內。六七歲時,德宗抱置膝上,問曰:「汝誰子,在吾懷?」對曰:「是第三天子。」德
 宗異而憐之。貞元四年六月,封廣陵王。順宗即位之年四月,冊為皇太子。七月乙未,權勾當軍國政事。



 八月丁酉朔,受內禪。乙巳,即皇帝位於宣政殿。先是,連月霖雨,上即位之日晴霽,人情欣悅。丙午,升平公主進女口十五人,上曰:「太上皇不受獻,朕何敢違!其還郭氏。」丁未,始御紫宸對百僚。己酉,以道州刺史路怒為邕管經略使。庚戌,荊南獻龜二,詔曰:「朕以寡昧,纂承丕業,永思理本,所寶惟賢。至如嘉禾神芝,奇禽異獸,蓋王化之虛美也。
 所以光武形於詔令,《春秋》不書祥瑞,朕誠薄德,思及前人。自今已後,所有祥瑞,但令準式申報有司,不得上聞;其奇禽異獸,亦宜停進。」癸丑,劍南西川節度使、檢校太尉、中書令、南康郡王韋皋薨。甲寅,以常州刺史穆贊為宣歙池觀察使,以前宣歙觀察使崔衍為工部尚書。己未,以中書侍郎、平章事袁滋為劍南東西兩川、山南西道安撫大使,時昌韋皋卒,劉闢據蜀邀節鉞故也。辛酉,太上皇誥冊良娣王氏為太上皇后。癸亥,以朝請大夫、守
 尚書左丞、輕車都尉、賜紫金魚袋鄭餘慶同中書門下平章事。丙寅,以饒州刺史李吉甫為考功郎中,夔州刺史唐次為吏部郎中。並知制誥。



 九月丁卯朔。己巳,罷教坊樂人授正員官之制。辛未,河陽三城節度使元韶卒。癸酉,以陳州刺史孟元陽為懷州刺史、河陽三城孟懷節度使。丙子,敕申光蔡、陳許兩道比遭亢旱,宜加賑恤,申、光、蔡賑米十萬石,陳、許五萬石。丁丑,前戶部侍郎蔡弁卒。襄州于頔進鷹,詔還之。己卯,京西神策行營節度
 行軍司馬韓泰貶撫州刺史,司封郎中韓曄貶池州刺史,禮部員外郎柳宗元貶邵州刺史,屯田員外郎劉禹錫貶連州刺史,坐交王叔文也。辛巳,給事中陸質卒。



 冬十月丙申朔。丁酉,集百僚發曾太皇太后沈氏哀於肅章門外。檢校司空兼右僕射、同中書門下平章事、魏國公賈耽卒。戊戌,以宰臣劍南安撫使袁滋檢校吏部尚書、同中書門下平章事、成都尹、劍南西川節度觀察等使,以西川行軍司馬齊闢為給事中。舒王誼薨。庚子,南
 詔使趙迦寬來赴山陵。浙東觀察使賈全卒。辛丑,吐蕃使論乞縷貢助山陵金銀衣服。太常上大行曾太皇太后沈氏謚曰睿真皇后。丙午,以華州刺史楊於陵為越州刺史、浙東觀察使。丁未,改桂州純化縣為慕化縣,蒙州純義縣為正義縣。乙酉,葬德宗皇帝於崇陵。甲寅,以刑部尚書高郢為華州刺史、潼關防禦、鎮國軍使,御史中丞李鄘為京兆尹。貶京兆尹王權為雅王傅。久雨,京師鹽貴,出庫鹽二萬石,糶以惠民。乙巳,祔睿真皇后神
 主、德宗皇帝神主於太廟。壬申,貶正議大夫、中書侍郎、平章事韋執誼為崖州司馬,以交王叔文也。潤、池、揚、楚、湖、杭、睦、江等州旱。貶劍南西川節度使袁滋為吉州刺史,以其慰撫三川逗留不進故也。以左驍衛將軍李演為夏州刺史、夏綏銀等州節度使,以右庶子武元衡為御史中丞。己卯,再貶撫州刺史韓泰為虔州司馬,河中少尹陳諫臺州司馬,邵州刺史柳宗元為永州司馬,連州刺史劉禹錫朗州司馬,池州刺史韓曄饒州司馬,和
 州刺史凌準連州司馬,岳州刺史程異郴州司馬,皆坐交王叔文。初貶刺史,物議罪之,故再加貶竄。辛巳,宣、撫、和、郴、郢、袁、衢七州旱。壬午,吏部尚書鄭珣瑜卒。甲申,以湖南觀察使楊憑為洪州刺史、江西觀察使,以虢州刺史薛蘋為潭州刺史、湖南觀察使。鄂、岳、婺、衡等州旱。癸巳,宣歙觀察使穆贊卒。



 十二月丙申朔。庚子,以東都留守韋夏卿為太子少保,以兵部尚書王紹為東都留守。壬寅,改淳縣為清溪縣,
 姓淳于者改姓于。甲辰,襄陽于頔加平章事。丙申,月犯畢。己酉,以新除給事中、西川行軍司馬劉闢為成都尹、劍南西川節度使。歲星犯太微西垣。庚戌,金州復析漢陰縣置石泉縣。壬子,以右諫議大夫韋丹為梓州刺史,充劍南東川節度使,以常州刺史路應為宣州刺史、宣歙池觀察使。壬戌,以朝請大夫、守中書舍人、翰林學士、上柱國鄭絪為中書侍郎、同平章事、集賢殿學士。以考功郎中、知制誥李吉甫為中書舍人,以考功員外郎
 裴<注自>為考功郎中、知制誥,並充翰林學士。



 元和元年春正月丙寅朔,皇帝率群臣於興慶宮奉上太上皇號曰應乾聖壽太上皇。丁卯,御含元殿受朝賀。禮畢,御丹鳳樓,大赦天下,改元曰元和。自正月二日昧爽已前,大闢罪已下,常赦不原者,咸赦除之。辛未,以鄂岳沔觀察使韓皋為鄂、岳、蘄、安、黃等州節度使。丁丑,太子少保韋夏卿卒。辛巳,以興元元從功臣、右神策護軍使副薛盈珍為右神策護軍中尉。壬午,成德軍節度
 使、檢校司空王士真同中書門下平章事。癸未,詔以太上皇舊恙愆和,親侍藥膳,起今月十六日已後,權不聽政。以左神策長武城防秋都知兵馬使高崇文檢校工部尚書,充神策行營節度使。甲申,太上皇崩於興慶宮,遷殯於太極殿,發喪。乙酉,宰相杜祐攝塚宰,杜黃裳為禮儀使,右僕射伊慎大明宮留守,視事於尚書省。壬辰,復置斜谷路館驛。戊子,制:「劍南西川,疆界素定,籓鎮守備,各有區分。頃因元臣薨謝,鄰籓不睦,劉闢乃因虛構
 隙,以忿結仇,遂勞王軍,兼害百姓。朕志存含垢,務欲安人,遣使諭宣,委之旄鉞。如聞道路擁塞,未息干戈,輕肆攻圍,擬圖吞並。為君之體,義在勝殘,命將興師,蓋非獲已。宜令興元嚴礪、東川李康掎角應接,神策行營節度使高崇文、神策兵馬使李元奕率步騎之師,與東川、興元之師類會進討。其糧料供餉,委度支使差官以聞。」甲午,高崇文之師由斜谷路,李元奕之師由駱谷路,俱會於梓潼。辛卯,群臣請聽政。



 二月乙未朔,以度支郎中
 敬寬為山劍行營糧料使。嚴礪奏收劍州。乙丑,入朝奚王梅落可銀青光錄大夫、檢校司空,封饒樂郡王,放還蕃。癸卯,贈宣武軍節度使陸長源為右僕射,贈故吉州刺史姜公輔禮部尚書。甲辰,以錢少,禁用銅器。癸丑,以魏博田季安同平章事。戊戌,謂宰臣曰:「前代帝王,或怠於聽政,或躬決繁務,其道如何。」杜黃裳對曰:「帝王之務,在於修己簡易,擇賢委任,宵旰以求民瘼,舍己從人以厚下,固不宜怠肆安逸。然事有綱領小大,當務知其遠者
 大者;至如簿書訟獄,百吏能否,本非人主所自任也。昔秦始皇自程決事。見嗤前代;諸葛亮王霸之佐,二十罰以上皆自省之,亦為敵國所誚,知不久堪;魏明帝欲省尚書擬事,陳矯言其不可;隋文帝日旰聽政,令衛士傳餐,文皇帝亦笑其煩察。為人主之體固不可代下司職,但擇人委任,責其成效,賞罰必信,誰不盡心。《傳》稱帝舜之德曰:『夫何為哉?恭己南面而已!』誠以能舉十六相,去四兇也。豈與勞神疲體自任耳目之主同年而語哉!但
 人主常勢。患在不能推誠,人臣之弊,患在不能自竭。由是上疑下詐,禮貌或虧,欲求致理,自然難致。茍無此弊,何患不至於理。』上稱善久之。以京兆尹李鄘為尚書右丞,以金吾大將軍鄭雲逵為京兆尹。



 三月乙丑朔。戊辰,詔常參官寒食拜墓,在畿內聽假日往還,他州府奏取進止。辛未,御史中丞武元衡奏:「中書門下御史臺五品已上官、尚書省四品已上、諸司正三品已上、從三品職事官、東都留守、轉運鹽鐵節度觀察使、團練防禦招討
 經略等使、河南尹、同華州刺史、諸衛將軍三品已上官除授,皆入合謝,其餘官許於宣政南班拜訖便退。」詔曰:「如此例中有加使及職掌並準此。」又「兵部、吏部、禮部貢院官員,每舉選限內,有十月至二月不奉朝參。若稱事繁,則中書門下、御史臺、度支、京兆府公事至重,朝謁如常。況旬節已賜歸休,又許分日,一月之內,才奉十日朝參,甚暑甚寒,又蒙矜放。臣求故實,以為王顏任中丞日嘗論其事,舉對甚詳。伏請準貞元十二年四月二十七
 日敕,永為常式。」從之。丙子,嚴礪收梓州。丁丑,制削奪劉闢在身官爵。先是韓全義入朝,令其甥楊惠琳知留後,俄有詔除李演為節度,代全義。演赴任,惠琳據城叛,詔發河東、天德兵誅之。辛巳,夏州兵馬使張承金斬惠琳,傳首以獻。壬辰,大行太上皇德妃董氏卒。以右神策行營節度高崇文檢校兵部尚書、梓州刺史、劍南東川節度。戊戌,以安南經略副使張舟為安南都護、本管經略使。己亥,以前劍南東川節度使韋丹為晉絳觀察使。壬
 寅,以前安南經略使趙昌為廣州刺史、嶺南節度使。癸卯,前嶺南節度使徐申卒。丙午,命宰臣監試制舉人於尚書省,以制舉人先朝所征,不欲親試也。丁未,以檢校司空、平章事杜祐為司徒,所司備禮冊拜,平章事如故;罷領度支、鹽鐵、轉運等使,從其讓也,仍以兵部侍即李巽代領其任。戊申,以隴右經略使、秦州經略使、秦州刺史劉澭為保義軍節度使。賑浙東米十萬石。己未,武元衡奏,常參官兼御史大夫、中丞者,準檢校省官例,立在本品同類之
 上。壬戌,邵王約薨。武元衡奏:「正衙待制官,本置此官以備問。比來正衙多不奏事。自今後請以尚書省六品以上職事官、東宮師傅賓詹、王傅等,每坐日令兩人待制,退朝,詔於延英候對。」從之。



 五月甲子朔。丁卯,京兆尹鄭雲逵卒。辛未,以兵部侍郎韋武為京兆尹兼御史大夫。壬申,貶劍南東川節度使李康為雷州司馬。陳、許、蔡等州旱。以橫海軍留後程執恭橫海軍節度使。庚辰,左丞、同平章事鄭餘慶為太子賓客,罷知政事。辛卯,冊太上皇
 后王氏為皇太后。



 六月癸己朔,以冊太后禮畢,赦天下系囚,死罪降從流,流以下遞減一等。文武內外官加母邑號,太后諸親,量與優給。丙申,冊德宗充容武氏為崇陵德妃。大風折樹。丁酉,高崇文破賊萬人於鹿頭關。加幽州劉濟侍中,淄青李師古檢校司徒。癸卯,高崇文收漢州。閏六月壬子朔,淄青李師古卒。戊辰,以秘書監董叔經為京兆尹。壬午,諫議大夫去左、右字,只置四員。以前司封員外郎韋況為諫議大夫。甲申,吐蕃論勃藏來
 朝貢。



 秋七月壬辰朔。壬寅,葬順宗於豐陵。己酉,太子少保致仕韓全義卒。八月辛酉朔。癸亥,以左衛大將軍李願檢校禮部尚書、夏州刺史,充夏、綏、銀節度使。甲子,郇王母王昭儀、宋王母趙昭儀、郯王母張昭訓、衡王母閻昭訓等,各以其王並為太妃。以許氏為美人,尹氏、段氏為才人。潯陽公主母崔昭訓為太妃。韓全義子進女樂八人,詔還之。丁卯,封王子平原郡王寧為鄧王,同安郡王寬為澧王,建安郡王宥為遂王,彭城郡王察為深王,
 高密郡王寰為洋王,文安郡王寮為絳王,第十男審為建王。己巳,以建王審為鄆州大都督、平盧淄青節度使;以節度副使李師道權知鄆州事,充節度留後。乙亥,冊妃郭氏為貴妃。靈武李欒奏,黃河岸塌處得古錢三千三百,其形小,方孔,三足。壬午,左降官韋執誼,韓泰、陳諫、柳宗元、劉禹錫、韓曄、凌準、程異等八人,縱逢恩赦,不在量移之限癸未。,京兆尹董叔經卒。甲申,御史臺奏,常參官在城未上及在外未到、假故等,在外未到,計水陸程
 外,滿百日,並停解,從之。丙戌,以尚書右丞李鄘為京兆尹。



 九月辛卯朔。癸卯,詔自今後兩省官每坐日一人對。丙午,以太子賓客鄭餘慶為國子祭酒。辛亥,高崇文奏收成都,擒劉闢以獻。癸丑,以山人李渤為左拾遺,征不至。甲子,易定張茂昭來朝。丙寅,以劍南東川節度使、檢校兵部尚書、梓州刺史、封渤海郡王高崇文檢校司空,兼成都尹、御史大夫,充劍南西川節度副大使、知節度事、管內度支營田觀察使、處置統押近界諸蠻及西山
 八國雲南安撫等使,仍改封南平郡王,食邑三千戶。戊戌,以山南西道節度使嚴礪為梓州刺史、劍南東川節度使;以將作監柳晟檢校工部尚書,兼興元尹,充山南西道節度使。庚辰,以吉州刺史袁滋為御史大夫,充義成軍節度使。壬午,以淄青節度使留後李師道檢校工部尚書,兼鄆卅大都督府長史,充平盧淄青節度副大使、知節度事。丙戌,以渤海國王大嵩璘檢校太尉。戊子,斬劉闢並子超郎等九人於獨柳樹下。



 十一月庚寅
 朔。己巳,以簡王傅王權為河南尹。丁未,以司農卿李上公為陜州大都督府長史,充陜虢觀察使。甲申,以武寧軍節度張愔為工部尚書,以東都留守王紹檢校右僕射,兼徐州刺史、武寧軍節度使、徐泗濠等州觀察等使。庚戌,以吏部侍郎趙宗儒為東都留守、東畿汝防禦使,以國子祭酒鄭餘慶為河南尹。甲寅,以給事中劉宗經為華州刺史、潼關防禦、鎮國軍等使。丙辰,以內常侍吐突承璀為神策護軍中尉。十二月丙申朔,太常奏隱太
 子、章懷、懿德、節愍、惠莊、惠文、惠宣、請恭、昭靖以下九太子陵,代數已遠,官額空存,今請陵戶外並停。乙亥,工部尚書張愔卒。丙戌,新羅、渤海、牂柯、回紇各遣使朝貢。



 二年春正月己丑朔,上親獻太清宮、太廟。辛卯,祀昊天上帝於郊丘,是日還宮,御丹鳳樓,大赦天下。先是,將及大禮,陰晦浹辰,宰臣請改日,上曰:「郊廟事重,齋戒有日,不可遽更。」享獻之辰,景物晴霽,人情欣悅。丁酉,司徒杜祐辭知政事。詔令每月三度入朝,便於中書商量政事。
 庚子,回紇請於河南府、太原府置摩尼寺,許之。乙巳,以門下侍郎、同平章事、南陽郡開國公杜黃裳檢校司空、同平章事,兼河中尹、河中晉絳等州節度使。停諸陵留守。己卯,以戶部侍郎、賜緋魚袋武元衡為門下侍郎、同平章事、賜紫金魚袋,以中書舍人、翰林學士李吉甫為中書侍郎、同平章事。丁巳,停中和、重陽二節賜宴;其上巳宴,仍舊賜之。



 二月辛酉,詔僧尼道士全隸左右街功德使,自是祠部司封不復關奏。丙寅,左右羽林軍應管
 月飜騎總五千六百一十三人,並停。己巳,起居舍人鄭隨次對,面受進止;令宣與兩省供奉官,自今已後,有事即進狀,次對官宜停。庚午,司天造新歷成,詔題為《元和觀象歷》。壬申夜,月掩歲星。丁丑,寒食節,宴群臣於麟德殿,賜物有差。壬午,以第五國軫為右神策軍中尉。



 三月辛卯,賜群臣宴於曲江亭。癸卯,判度支李巽為兵部尚書,依前判度支鹽鐵轉運使。



 夏四月甲子,禁鉛錫錢。以右金吾衛大將軍範希朝為檢校司空、靈州長史、朔方
 靈鹽節度使。戊寅,近置英武軍額,宜停。庚辰,嶺南節度使趙昌進瓊、管、儋、振、萬安六州《六十二洞歸降圖》。



 六月丁巳朔,始置百官待漏院於建福門外。故事,建福、望仙等門,昏而閉,五更而啟,與諸坊門同時。至德中有吐蕃囚自金吾仗亡命,因敕晚開門,宰相待漏於太僕寺車坊。至是始令有司據班品置院。戊午,鳳翔節度使張敬則卒。乙丑,五坊色役戶及中書門下兩省納課陪廚戶及捉錢人,並歸府縣色役。己巳,停舒、廬、滁、和四州團練
 使額。癸酉,東都莊宅使織造戶,並委府縣收管,乙亥,停潤州丹陽軍額。丙子,左神策軍新築夾城,置玄化門晨耀樓。辛巳,以京兆尹李鄘為鳳翔尹、鳳翔隴右節度使。蔡州水,平地深七八尺。



 秋七月丙戌朔,敕刑部侍郎許孟容等刪定《開元格後敕》。丁亥,敕外命婦朝謁皇太后,多有前卻,今後諸親委宗正寺,百官母妻委臺司,如有違越者,夫子在一月俸,頻不到,有司具狀奏聞。戊子,錄配享功臣之後,得蘇瑰孫系,用為京兆府司錄;崔玄暐
 孫元方、張說孫騑,並為監察御史;狄仁傑後玄範,為右拾遺;敬暉孫元亮、袁恕己孫德師,相次敘用。癸巳,太僕寺丞令狐丕進亡父亙所撰《代宗實錄》四十卷,詔贈峘工部尚書。



 八月丙辰朔。辛酉,宰相武元衡兼判戶部事。壬戌,刑部奏改《律》卷第八普《鬥競律》。甲子,以職方員外郎王潔為嶺南選補使,監察御史崔元方監之。甲戌,中書奏:「先停諸道奏祥瑞。伏以所獻祥瑞,皆緣臘饗、告廟、元會奏聞,今後諸大瑞隨表聞奏,中瑞、下瑞申有司,其
 元日奏祥瑞,請依令式。」從之。辛巳,封杜黃裳為邠國公,于頔為燕國公。沒蕃僧惟良闡等四百五十人自蕃中還。九月乙酉,密王綢薨。



 十月己酉,以浙西節度使李錡為左僕射;以御史大夫李元素為潤州刺史,鎮海軍、浙西節度使。庚申,李錡據潤州反,殺判官王澹、大將趙琦。時錡詐請入朝,署澹為留後,因諷詔:「李錡屬列宗枝,任居方伯,赫奕之跺,飽綢繆
 之恩。待以親賢,報之以逆節;授其師旅,用元以亂常。屢獻表章,亟請朝會,初則詐疾,後萬縱兵。僚佐以獻規受屠,王臣以傳命見脅。朕切於含垢,未忍發明,累降中人,令遵前旨。無軺車之戒路,有沴氣之滔天。加以日逞淫刑,月興暴賦。朕為人父母,聞甚惻然,顧惟紀綱,焉敢廢墜!李錡在身官爵,並宜削奪。」以淮南節度使王鍔充諸道行營招討使,內官薛尚衍為監軍,率汴、徐、鄂、淮南、宣歙之師,取宣州路進討。丁卯,以門下侍郎、平章事武元
 衡檢校吏部尚書、兼門下侍郎、平章事、成都尹、充劍南西川節度使,仍封臨淮郡公。將行,上御安福門慰勞之。癸酉,潤州大將張子閬、李奉獨等執李錡以獻。辛巳,錡從父弟宋州刺史銛、通事舍人銑坐貶嶺外。



 十一月甲申,斬李錡於獨柳樹下,削錡屬籍。丙戌,以擒李錡潤州牙將張子良為左金吾衛將軍,封南陽郡王;田少卿、李奉仙等為羽林將軍,並封公。甲辰,詔司徒杜祐筋力未衰,起今後每日入中書視事。十二月甲寅,宰相李吉甫
 封贊皇侯。丙辰,上謂宰臣曰:「朕覽國書,見文皇帝行事,少有過差,諫臣論諍,往復數四。況朕之寡昧,涉道未明,今後事或未當,卿等每事十論,不可一二而止。」丁巳,東都國子監增置學生一百人。癸亥,御史臺奏:「文武常參官準乾元元年三月十四日敕,如有朝堂相吊慰及跪拜、待漏行立失序,語笑喧嘩;入衙入閣,執笏不端,行立遲慢;立班不正,趨拜失儀,言語微喧嘩穿班穿仗,出入閣門,無故離位;廊下飲食,行坐失儀喧鬧;入朝及退朝不
 從正衙出入;非公事入中書等:每犯奪一月俸。班列不肅,所由指摘,猶或飾非,即具聞奏貶責。臣等商量,於舊條每罰各減一半,所貴有犯必舉。」從之。丙寅,以劍南西川節度使高崇文檢校司空、同平章事,兼邠州刺史、邠寧慶節度使,充京西諸軍都統。壬申,禮部舉人,罷試口義,試墨義十條,五經通五,明經通六,即放進士。舉人曾為官司科罰,曾任州縣小吏,雖有辭藝,長吏不得舉送,違者舉送官停任,考試官貶黜。丙子,令宰臣宣敕:百僚
 游宴過從餞別,此後所由不得奏報,務從歡泰。保義軍節度使劉澭卒。己卯,史官李吉甫撰《元和國計簿》,總計天下方鎮凡四十八,管州府二百九十五,縣一千四百五十三,戶二百四十四萬二百五十四,其鳳翔、鄜坊、邠寧、振武、涇原、銀夏、靈鹽、河東、易定、魏博、鎮冀、範陽、滄景、淮西、淄青十五道,凡七十一州,不申戶口。每歲賦入倚辦,止於浙江東西、宣歙、淮南、江西、鄂岳、福建、湖南等八道,合四十九州,一百四十四萬戶。比量天寶供稅之戶,
 則四分有一。天下兵戎仰給縣官者八十三萬餘人,比量天寶士馬,則三分加一,率以兩戶資一兵。其他水旱所損,徵科發斂,又在常役之外。吉甫都纂其事,成書十卷。是歲,吐蕃、回紇、奚、契丹、渤海、牂柯、南詔並朝貢。



 三年春正月癸未朔。癸巳,群臣上尊號曰睿聖文武皇帝。御宣政殿受冊,禮畢,移仗禦丹鳳樓,大赦天下。庚子,涇原段祐請修臨涇城,在涇州北九十里,扼犬戎之沖要,詔從之。戊申,罷左右神威軍,合為一,號天威軍。



 二月
 丙申,宰相李吉甫進封趙國公。己丑,以武昌軍節度使韓皋為潤州刺史、鎮海軍節度、浙西觀察使。辛未,贈故布衣崔善真睦州司馬,忠諫而死於李錡也。癸丑,以鄜坊節度使裴玢為興元尹、山南西道節度使。丙子,以右金吾衛大將軍路恕為鄜州刺史、鄜坊節度使。戊寅,咸安大長公主卒於回紇。



 三月癸巳,郇王總薨。庚子,以定平鎮兵馬使硃士明為四鎮、北庭、涇原等州節度使。乙巳,御宣政殿試制科舉人。



 夏四月癸丑,中使郭裏旻酒
 醉犯夜,杖殺之,金吾薛伾、巡使韋纁皆貶逐。賜硃士明名曰忠亮。乙丑,貶翰林學士王涯虢州司馬,時涯甥皇甫湜與牛僧孺、李宗閔並登賢良方正科第三等,策語太切,權幸惡之,故涯坐親累貶之。壬申,大風毀含元殿欄檻二十七間。乙亥,以嶺南節度使趙昌為江陵尹、荊南節度使,以戶部侍郎楊於陵為廣州刺史、嶺南節度使。丁丑,以荊南節度使裴均為左僕射、判度支。敕五月一日御殿受朝賀禮宜停。己卯,裴均於尚書省都堂上
 僕射。其送印及呈孔目唱案授案,皆尚書郎為之,文武三品已上升階列坐,四品五品及郎官、御史拜於下,然後召御史中丞、左右丞、侍郎升階答拜。雖修故事行之,議者論其太過。



 五月壬辰,兵部請復武舉,從之。甲午,敕東都畿、汝州都防禦使及副使宜停,所管將士三千七百三十人,隨畿、汝界分留守及汝州防禦使分掌之。辛丑,右僕射裴均請取荊南雜錢萬貫修尚書省,從之。丙午,正衙冊九姓回紇可汗為登囉里汨蜜施合毗伽
 保義可汗。六月戊辰,詔以錢少,欲設畜錢之令,先告諭天下商貢畜錢者,並令逐便市易,不得畜錢。天下銀坑,不得私手採。癸亥,以邕管將黃少卿為歸順州刺史,弟少高、少溫並授官,西原蠻酋也,貞元中屢寇邕管,至是歸款。乙丑,罷江淮私堰埭二十二,從轉運使奏也。甲戌,以河南尹關除慶為東都留守。丁丑,沙陀、突厥七百人攜其親屬歸振武節度伎範希朝,乃授其大首領曷勒河波陰山府都督。



 秋七月辛巳朔,日有蝕之。己亥,復以度
 支安邑、解縣兩池留後為榷鹽使。丁未,涪州復隸黔中道。八月庚申,復置東都防禦兵七百人。九月己丑,淮南節度使王鍔來朝。庚寅,以山南東道節度使于頔守司空、同平章事;以右僕射裴均檢校左僕射、同平章事、襄州長史,充山南東道節度使;加宣武韓弘同平章事。丙申,以戶部侍郎裴洎為中書侍郎、同平章事。戊戌,以中書侍郎、平章事李吉甫檢校兵部尚書、兼中書侍郎、平章事、揚州大都督府長史、淮南節度使。以淮南節度使
 王鍔檢校司徒、河中尹、河中晉絳慈隰節度使。河中節度使、檢校司空、同平章事邠國公杜黃裳卒。是秋,京師大雨。



 十月己酉朔。癸亥,以太常卿高郢為御史大夫。甲子,以御史中丞竇群為湖南觀察使,既行,改為黔中觀察使。群初為李吉甫所擢用,及持憲,反傾吉甫,吉甫謐其陰事,故貶之。丁卯,度支使下判案官,以四員為定。



 十一月甲午,橫海軍節度使程執恭來朝。十二月庚戌,以臨涇縣為行原州,命鎮將郝玼為刺史。自玼鎮臨涇,西戎
 不敢犯塞。甲子,南詔異牟尋卒。辛未,以諫議大夫段平仲使南詔吊祭,仍立其子驃信苴蒙閣勸為南詔王。是歲,淮南、江南、江西、湖南、山南東道旱。



 夏四月丙子朔。戊寅,國子祭酒馮伉卒。壬午,裴均進銀器一千五百兩,以違敕,付左藏庫。甲申,令皇太子居少陽院。武功人張英奴撰《回波辭》惑眾,杖殺之。丙申,撫州山人張洪騎牛冠履,獻書於光順門,書不足採,遣之。庚子,制故太尉、西平郡王李晟宜編附屬籍。以太常卿李元素為戶部尚書、判
 度支,以商州刺史元義方為福建觀察使。甲辰,以兵部侍郎權德輿為太常卿,仍賜金紫。以御史大夫高郢為兵部尚書,以刑部郎中、侍御史知雜李夷簡為御史中丞。五月丙午朔。辛酉,刑部尚書鄭元卒。丁卯,鹽鐵使、吏部尚書李巽卒。六月乙亥朔。丁丑,以河東節度使李鄘為刑部尚書以充諸道鹽鐵轉運使;以靈鹽節度使範希朝為太原尹、北都留守、河東節度使;以右衛上將軍王佖為靈州大都督府長史、靈鹽節度使。辛丑,五嶺已北
 銀坑任人開採,禁錢不過嶺南。



 秋七月乙巳朔,禦制《前代君臣事跡》十四篇,書於六扇屏風。是月,出書屏以示宰臣,李籓等表謝之。丁未,渭南暴水,壞廬舍二百餘戶,溺死六百人,命府司賑給。乙卯,右羽林統軍高固卒。壬戌,御史中丞李夷簡彈京兆尹楊憑前為江西觀察使時贓罪,貶憑臨賀尉。戊辰,以尚書右丞許孟容為京兆尹,賜金紫。八月甲戌朔。癸未,兗州魚臺縣移置於黃臺市。丙申,安南都護張舟奏破環王國三萬餘人,獲戰象、
 兵械,並王子五十九人。癸卯,贈太師裴冕宜配享代宗廟庭,贈太師李晟、贈太尉段秀實宜配享德宗廟庭。



 九月甲辰朔。庚戌,以成德軍都知兵馬使、鎮府右司馬王承宗起復檢校工部尚書,充成德軍節度使;以德州刺史薛昌朝檢校左常侍,充保信軍節度、德棣等州觀察等使。昌朝,薛嵩之子,婚於王氏,時為德州刺史。朝廷以承宗難制,乃割二州為節度,以授昌朝。制才下,承宗以兵虜昌朝歸鎮州。丁卯,邠寧節度使、檢校司空、同平章
 事高崇文卒。



 冬十月癸酉朔,以右羽林統軍閻巨源為邠州刺史、邠寧慶節度使,以少府監崔頲為同州刺史、本州防禦、長春宮等使、癸未,詔:「成德軍節度使王承宗頃在苫廬,潛窺戎鎮。而內外以事君之禮,叛而必誅;分土之儀,專則有闢。朕念其先祖嘗有茂勛,貸以私恩,抑於公議。使臣旁午以告諭,孽童俯伏以陳誠,願獻兩州,期無二事。朕亦收其後效,用以曲全,授節制於舊疆,齒勛賢於列位。況德、棣本非成德所管,昌朝又是承宗懿
 親,俾撫近鄰,斯誠厚澤,外雖兩鎮,內是一家。而承宗象恭懷奸,肖貌稔惡,欺裴武於得位之後,囚昌朝於授命之中。加以表疏之章,悖慢斯甚,義士之所興嘆,天地之所不容。恭行天誅,蓋示朝典,承宗在身官爵,並宜削奪。」以神策左軍中尉吐突承璀為鎮州行營招討處置等使,以龍武將軍趙萬敵為神策先鋒將,內官宋惟澄、曹進玉、馬朝江等為行營館驛糧料等使。京兆尹許孟容與諫官面論,征伐大事,不可以內官為將帥,補闕獨孤
 鬱其言激切。詔旨只改處置為宣慰,猶存招討之名。己丑,詔軍進討,其王武俊、士真墳墓,軍士不得樵採,其士平、士則各守本官,仍令士則各襲武俊之封。庚寅,冊鄧王寧為皇太子。癸巳,以冊儲,肆赦系囚,死罪降從流,流以下遞降一等。文武常參官、外州府長官子為父後者,賜勛兩轉。工部侍郎歸登、給事中呂元膺為皇太子諸王侍讀。己亥,吐突承璀軍發京師,上禦通化門勞遣之。



 十一月癸卯朔,浙西蘇、潤、常州旱儉,賑米二萬石。甲子,
 河南尹杜兼卒。己巳,彰義軍節度使、檢校司空、同平章事吳少誠卒。十二月壬申朔,以戶部侍郎張弘靖為陜府長史、陜虢觀察陸運等使,賜金紫。以陜虢觀察使房式為河南尹。中丞李夷簡奏:「諸州府於兩稅外違格科率,請諸道鹽鐵、轉運、度支、巡院察訪報臺司,以憑舉奏。」從之。



 五年春正月壬寅朔,己巳,浙西觀察使韓皋以杖決安吉令孫澥致死,有乖典法,罰一月俸料。



 二月辛未朔。戊
 子,禮院奏東宮殿閣名及宮臣姓名,與太子名同者改之,其上臺官列、王官爵土無例輒改,從之。東臺監察御史元稹攝河南尹房式於臺,擅令停務,貶江陵府士曹參軍。



 三月辛丑朔,宰相杜祐與同列宴於樊川別墅,上遣中使賜酒饌。乙巳,以御史中丞李夷簡為戶部侍郎、判度支,以兵部侍郎王播為御史中丞。癸巳,以太子賓客鄭絪檢校禮部尚書、廣州刺史、嶺南節度使。己未,制以遂王宥為彰義軍節度使,以申州刺史吳少陽為申
 光蔡節度留後。甲子,大風折木。丁卯,宰相于頔請依杜祐例一月三朝,從之。



 夏四月庚午朔。癸酉,戶部尚書李元素卒。甲申,鎮州行營招討使吐突承璀執昭義節度使盧從史,載從史送京師。丁亥,河東範希朝奏破賊於木刀溝。福州復置侯官、長樂二縣,建州置將樂縣。壬申,以昭義都知兵馬使、潞州左司馬烏重胤為懷州刺史、河陽三城懷州節度使,以河陽節度使孟元陽為潞州長史、昭義軍節度、澤潞磁邢洺觀察使。戊戌,貶前昭義
 節度使盧從史為驩州司馬。



 五月庚子朔。乙巳,昭義軍三千人夜潰奔魏州。右神策軍使段祐卒。庚申,吐蕃使論思即熱朝貢,並歸鄭叔矩、路泌之柩。六月庚午朔。戊寅,以太府卿李少和為洪州刺史、江西觀察使。奚、回紇、室韋寇振武。癸巳,應給食實封例,節度使兼宰相,每食實封百戶,歲給八百端匹,若是絹,加給綿六百兩;節度使不兼宰相,每百戶給四百端匹;軍使諸衛大將軍,每百戶給三百五十端匹。



 秋七月己亥朔。庚子,承宗遣判官
 崔遂上表自首,請輸常賦,朝廷除授官吏。丁未,詔昭洗王承宗,復其官爵,待之如初。諸道行營將士,共賜物二十八萬四百三十端匹。時招討非其人,諸軍解體,而籓鄰觀望養寇,空為逗撓,以弊國賦。而李師道、劉濟亟請昭雪,乃歸罪盧從史而宥承宗。不得已而行之也。幽州劉濟加中書令,魏博田季安加司徒,淄青李師道加僕射,並以罷兵加賞也。乙卯,幽州節度使劉濟為其子總鴆死。庚申,以虔州刺史馬總為安南都護、本管經略使。
 八月乙巳朔。乙亥,上顧謂宰臣曰:「神仙之事信乎?李籓對曰:「神仙之說,出於道家;所宗《老子》五千文為本。《老子》指歸,與經無異。後代好怪之流,假托老子神仙之說。故秦始皇遣方士載男女入海求仙,漢武帝嫁女與方士求不死藥,二主受惑,卒無所得。文皇帝服胡僧長生藥,遂致暴疾不救。古詩云:『服食求神仙,多為藥所誤。』誠哉是言也。君人者,但務求理,四海樂推,社稷延永,自然長年也。」上深然之。以浙東觀察使薛蘋為潤州刺史、浙西
 觀察使,以常州刺史李遜為越州刺史、浙東觀察使。以都官郎中韋貫之為中書舍人,起居舍人裴度為司封員外郎、知制誥。癸巳,以鄧州刺史崔詠為邕州刺史、本管經略使。



 九月戊戌朔。辛亥,以吐突承璀復為左軍中尉。諫官以承璀建謀討伐無功,請行朝典。上宥之,降承璀為軍器使。乃以內官程文乾為左軍中尉。壬戌,以瀛州刺史劉總起復受幽州長史,充幽州盧龍軍節度使。癸亥,以兵部尚書高郢為右僕射致仕。丙寅,制以正議
 大夫、守太常卿、上柱國、襄武縣開國侯、賜紫金魚袋權德輿可守禮部尚書、同中書門下平章事。丁卯,翰林學士獨孤鬱守本官起居,以妻父權德輿在中書,避嫌也。



 冬十月戊辰朔,以京兆尹許孟容為兵部侍郎,以中丞王播代容,又以呂元膺代播。升平大長公主薨。庚辰,宰相裴垍進所撰《德宗實錄》五十卷,賜垍錦彩三百匹、銀器等,史官蔣武、韋處厚等頒賜有差。辛巳,定州將楊伯玉誘三軍為亂,拘行軍司馬任迪簡。別將張佐元殺
 伯玉,迪簡謀歸朝,三軍懼,乃殺佐元。壬辰,制以迪簡檢校工部尚書、定州長史,充義武軍節度觀察、北平軍等使。甲午,以前義武軍節度、檢校太尉、兼太子太傅、同平章事張茂昭檢校太尉、兼中書令、河中尹,充河中晉絳慈隰節度使。



 十一月戊戌朔,浙西奏當鎮舊有丹陽軍,今請並為鎮海軍,從之。庚子,右金吾衛大將軍伊慎降為右衛將軍,以行賂三十萬與中尉第五從直,求為河東節度故也。甲辰,會王纁薨。庚戌,以前河中節度使王
 鍔檢校司空、兼太子太傅、太原尹、北都留守、河東節度使。以代州刺史阿跌光進為單于大都護、振武麟勝節度度支營田觀察押蕃落等使。庚申,以中書侍郎、平章事裴垍為兵部尚書。以前保信軍節度使、德州刺史薛昌朝為右武衛將軍。前為王承宗虜之,囚於鎮州,至是歸朝故也。丙寅,吏部郎中柳公綽獻《太醫箴》,上深喜納,遣中使勞之。



 十二月丁卯朔。癸酉,諸道鹽鐵轉運使、刑部尚書李鄘檢校吏部尚書,兼揚府長史,充淮南節
 度使。以河南尹房式為宣州刺史、宣歙池觀察、採石軍等使。以前宣歙觀察使盧坦為刑部侍郎,充諸道鹽鐵轉運使。壬午,以吏部郎中柳公綽為御史中丞。以前御史中丞呂元膺為鄂州刺史、鄂黃岳沔蘄安黃等州觀察使。以鄂岳察使郗士美為河南尹。新授諫議大夫蔣武請改名乂。以吏部侍郎崔邠為太常卿。



 六年春正月丙寅朔。丙申,以彰義軍留後吳少陽檢校工部尚書,充彰義軍節度、申光蔡等州觀察使。敕諫議
 大夫孟簡、給事中劉伯芻、工部侍郎歸登、右補闕蕭俯等於豐泉寺翻譯《大乘本生心地觀音經》。庚申,以淮南節度使、中書侍郎、同平章事、趙國公李吉甫復知政事、集賢殿大學士、監修國史。



 二月丙寅朔。壬申,門下侍郎、同平章事李籓為太子詹事。籓與吉甫不葉,吉甫既用事,故罷籓相位。丙子,河中節度使、檢校太尉、中書令張茂昭卒。以太府卿裴次元為福建觀察使。己丑,訴王造薨。癸巳,以陜虢觀察使絳弘靖檢校禮部尚書、河中尹、
 晉絳慈等州節度使,以右丞衛次公為陜府長史、陜虢觀察使。以中書舍人、翰林學士李張為戶部侍郎。以京畿民貧,貸常平義倉粟二十四萬石,諸道州府依此賑貸。



 三月乙未朔,以河南尹郗士美檢校工部尚書,兼潞府長史、昭義軍節度使。丁未,以檢校右僕射嚴綬為江陵尹荊南節度使。河東舊使錫錢,民頗為弊,宜於蔚州置五爐鑄錢。乙卯,畿內軍鎮牧放,駙馬貴族略獲,並不得帶兵仗,恐雜盜也。



 夏四月乙丑朔。戊辰,兵部尚書裴
 垍為太子賓客,以諫議大夫裴堪為同州防禦從事使。庚午,以戶部侍郎、判度支李夷簡檢校禮部尚書、襄州大都督府長史、山南東道節度使;以刑部侍郎、鹽鐵轉運使盧坦為戶部侍郎、判度支;京兆尹王播為刑部侍郎,充諸道鹽鐵轉運使;以福建觀察使元義方為京兆尹。癸酉,以張茂昭家妓四十七人歸定州。己卯,月近房。以前荊南節度使趙宗儒為刑部尚書。東都留守鄭餘慶為兵部尚書,依前留守。王播奏:江淮河嶺已南、兗鄆等鹽院,
 元和五年都收賣鹽價錢六百九十八萬五千五百貫。校量未改法已前四倍抬估,虛錢一千七百四十六萬三千七百貫。除鹽本外,付度支收管。從之。辛卯,戶部奏置巡官。



 五月甲午朔,取受王承宗錢物人品官王伯恭杖死。庚子,以左金吾衛將軍李惟簡檢校戶部尚書、鳳翔尹、隴右節度使。丙午,前山南東道節度使、檢校左僕射、平章事裴均卒。壬子,以振武節度阿跌光進夙彰誠節,久立茂勛,宜賜姓李氏。弟洺州刺史光顏,已從別敕
 處分。



 六月甲子朔,減教坊樂人衣糧。丁卯,中書門下奏:



 官省則事省,事省則人清;官煩則事煩,事煩則人濁。清濁之由,在官之煩省。國家自天寶已後,中原宿兵,見在軍士可使者八十餘萬。其餘浮為商販,度為僧道,雜入色役,不歸農桑者,又十有五六。則是天下常以三分勞筋苦骨之人,奉七分坐衣待食之輩。今內外官給俸料者不下一萬餘員,其間有職出異名,奉離本局,府寺曠廢,簪組因循者甚眾。況斂財日寡而授祿至我,設官有
 限而入色無數,九流安得不雜,萬物安得不煩。漢初置郡不過六十,文景醲化,百官莫先,則官少不必政紊,郡多不必事理。今天下三百郡,一千四百縣。故有一邑之地,虛設群司,一鄉之甿,徒分縣職,所費至廣,所制全輕。伏請敕吏兵部侍郎、郎中、給事中、中書舍人各一人,錯綜利病,詳定廢置,吏員可並省者並省之,州縣可並合者並合之,每年入仕者可停減者停減之。此則利廣而易求,官少而易理,稍減冗食,足寬疲甿。又國家舊章,依
 品制俸,官一品月俸三十千,其餘職田祿米,大約不過千石,自一品以下,多少可知。艱難已來,禁綱漸弛,於是增置使額,厚請俸錢。故大歷中權臣月俸有至九千貫者,列郡刺史無大小給皆千貫。常袞為相,始立限約,至李泌又量其閑劇,隨事增加,時謂通濟,理難減削。然猶有名存職廢,額去俸存,閑劇之間,厚薄頓異。將為永式,須立常規。



 從之。乃命給事中段平仲、中書舍人韋貫之、兵部侍郎許孟容、戶部侍郎李絳等詳定減省。甲申,以
 御史中丞柳公綽為湖南觀察使。丁亥,太白近右執法。戊子,賜御史中丞竇易直緋魚袋。



 秋七月癸巳朔,尚書右僕射致仕高郢卒。庚申,贈銀青光祿大夫、太子賓客裴垍太子少傅。八月癸亥朔,戶部侍郎李絳奏:「諸州闕官職田祿米,及見任官抽一分職田,請所在收貯,以備水旱賑貸。」從之。乙丑,以天德軍防禦使張煦為夏州刺史、夏綏銀等州節度使。丁卯,荊南先置永安軍,宜停。辛巳,以常州刺史崔芃為洪州刺史、江西觀察使。九月癸
 巳朔,以蜀州刺史崔能為黔中觀察使。戊戌,富平縣人梁悅為父復仇,殺秦杲,投獄請罪。特敕免死,決杖一百,配流循州。職方員外郎韓愈獻議執奏之。減諸司流外總一千七百六十九人。貶黔中觀察使竇群為開州刺史,以為政煩苛,辰、錦二州蠻叛故也。



 冬十月,以前夏州節度使李願檢校兵部尚書、徐州刺史,充武寧軍節度使。戊辰,以戶部尚書韓皋為東都留守,判東都尚書省事。以太子詹事李籓為華州刺史、潼關防禦、鎮國軍使。
 以東都留守鄭餘慶為吏部尚書。己巳,詔:「朕於百執事、群有司,方澄源流,以責實效。轉運重務,專委使臣,每道有院,分督其任;今陜路漕引悉歸中都,而尹守職名尚仍舊貫。又諸道都團練使,足修武備以靖一方;而別置軍額,因加吏祿,亦既虛設,頗為浮費。思去煩以循本,期省事以使人。其河水陸運、陜府陸運、潤州鎮海軍、宣州採石軍、越州義勝軍、洪州南昌軍、福州靖海軍等使額,並宜停。所收使已下俸料一事已來,委本道充代百
 姓闕額兩稅,仍具數奏聞。」戊寅,詔:「王者之牧黎元也,愛之如子,視之如傷。茍或風雨不時,稼穡不稔,則必除煩就簡,惜力重勞,以圖便安,以阜生業。況邦畿之內,百役所業,雖勤恤之令亟行,而供億之制猶廣。重以經夏炎又,自秋霖澤,南畝虧播植之功,西成失豐登之望。內管口食,外牽王徭,豈惟轉輸之虞,慮有餒殍之患。斯蓋理道猶鬱,和氣未通,永言於茲,良所咎嘆。京兆府每年所配折糶粟二十五萬石宜放。於百姓有粟情願折納者,
 時估外特加優饒。今春所貸義倉粟,方屬歲饑,容至豐熟歲送納。元和五年已前諸色逋租並放。百官職田,其數甚廣,今緣水潦,諸處道路不通,宜令所在貯納,度支支用,令百官據數於太倉請受。遭水旱處,通計所損,便與除破,不得檢覆。為理之本,在乎安人。咨爾尹京宰邑之臣,實為親人阜俗之寄,必當詢其疾苦,奉我詔條,恤隱為心,無怠於事,罔或徇利以剝下,吐剛而茹柔,使閭井咸安,惸嫠獲濟。各勉忠孝,宜悉朕懷。」丙戌,以諫議大
 夫孔戣為皇太子諸王侍讀。



 十一月壬辰朔。癸巳,新授華州刺史李籓卒。乙巳,以工部尚書趙昌檢校兵部尚書,兼華州刺史,充潼關防禦、鎮國軍等使。



 十二月癸亥朔。壬申,詔委宗正卿選人門嫁十六宅諸王女,仍封為縣主。甲申,京兆尹元義方、戶部侍郎判度支盧坦以違令立戟,罰一月俸,收奪所請門戟。己丑,制以朝義郎、守尚書戶部侍郎、驍騎尉、賜紫金魚袋李絳為朝議大夫、守中書侍郎、同中書門下平章事。閏十二月辛卯朔,右
 衛上將軍伊慎卒。辛亥,皇太子寧薨,謚曰惠昭,廢朝三日。國典無太子薨禮,國子司業裴苣精禮學,特賜於西內定儀。



\end{pinyinscope}