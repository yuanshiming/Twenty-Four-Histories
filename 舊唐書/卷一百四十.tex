\article{卷一百四十}

\begin{pinyinscope}

 ○竇參從子申附齊映劉滋從兄贊附盧邁崔損齊抗



 竇參,字時中,工部尚書誕之玄孫。父審言,聞喜尉,以參貴贈吏部尚書。參習法令,通政術,性矜嚴,強直而果斷。
 少以門廕,累官至萬年尉。時同僚有直官曹者,將夕,聞親疾,請參代之。會獄囚亡走,京兆尹按直簿,將奏,參遽請曰:「彼以不及狀謁,參實代之,宜當罪。」坐貶江夏尉,人多義之。



 累遷奉先尉。縣人曹芬,名隸北軍,芬素兇暴,因醉毆其女弟,其父救之不得,遂投井死。參捕理芬兄弟當死,眾官皆請俟免喪,參曰:「子因父生,父由子死,若以喪延罪,是殺父不坐也。」皆正其罪而杖殺之,一縣畏伏。轉大理司直。按獄江淮,次揚州,節度使陳少游驕蹇,不
 郊迎,令軍吏傳問,參正辭讓之,少游悔懼,促詣參,參不俟濟江。還奏合旨。時婺州刺史鄧珽坐贓八千貫,珽與執政有舊,以會赦,欲免贓。詔百僚於尚書省雜議,多希執政意,參獨堅執正之於法,竟徵贓。明年,除監察御史,奉使按湖南判官馬彞獄。時彞舉屬令贓罪至千貫,為得罪者之子因權幸誣奏彞,參竟白彞無罪。彞實能吏,後累佐曹王皋,以正直強幹聞。



 參轉殿中侍御史,改金部員外郎、刑部郎中、侍御史、知雜事。無幾,遷御史中丞,
 不避權貴,理獄以嚴稱。數蒙召見,論天下事,又與執政多異同,上深器之,或參決大政。時宰頗忌之,多所排抑,亦無以傷參。然多率情壞法。初定百官俸料,以嘗為司直,黨其官,故給俸多於本寺丞;又定百官班秩,初令太常少卿在左右庶子之上;又惡詹事李昇,遂移詹事班退居諸府尹之下,甚為有識所嗤。尋兼戶部侍郎。時京師人家豕生兩首四足,有司欲奏;參曰:「此為豕禍,安可上聞!」命棄之。是時,郊牛生犢有六足者,太僕卿周皓白
 宰相請奏,李泌亦戲答以遣之。



 故淮南節度使陳少游子正儀請襲封,參大署尚書省門曰:「陳少游位兼將相之崇,節變艱危之際,君上含垢,未能發明,愚子何心,輒求傳襲。」正儀懼,不敢求封而去。時神策將軍孟華有戰功,為大將軍所誣奏,稱華謀反;有右龍武將軍李建玉,前陷吐蕃,久之自拔,為部曲誣告潛通吐蕃,皆當死,無以自白,參悉理出之,由是人皆屬望。明年,拜中書侍郎、同平章事,領度支、鹽鐵轉運使。每宰相間日於延英召
 對,諸相皆出,參必居後久之,以度支為辭,實專大政。參無學術,但多引用親黨,使居要職,以為耳目,四方籓帥,皆畏懼之。李納既憚參,饋遺畢至,外示敬參,實陰間之。上所親信,多非毀參。竇申又與吳通玄通犯事覺,參任情好惡,恃權貪利,不知紀極,終以此敗。貶參郴州別駕,貞元八年四月也。



 參至郴州,汴州節度使劉士寧遺參絹五千匹。湖南觀察使李巽與參有隙,遂具以聞;又中使逢士寧使於路,亦奏其事。德宗大怒,欲殺參。宰相陛
 贄曰:「竇參與臣無分,因事報怨,人之常情。然臣參宰衡,合存公體,以參罪犯,置之於死,恐用刑太過。」於是且止。尋又遣中使謂贄等曰:「卿等所奏,於大體雖好,然此人交結中外,其意難測,朕尋情狀,其事灼然。又竇參在彼,與諸戎帥交通,社稷事重,卿等速進文書處分。」贄奏曰:「臣面承德音,幸奉密旨,皆以社稷為言,又知根尋已審,敢不上同憂憤,內絕狐疑,豈願遲回,更貽念慮?但以參常經重任,斯謂大臣,進退之間,猶宜有禮,誅戮之際,不
 可無名。劉晏久掌貨財,當時亦招怨讟,及加罪責,事不分明,叛者既得以為辭,眾人亦為之懷愍。用刑曖昧,損累不輕,事例未遙,所宜重慎。竇參頃司鈞軸,頗怙恩私,貪受貨財,引縱親黨,此則朝廷同議,天下共傳。至於潛懷異圖,將起大惡,跡既未露,人皆莫知。臣等新奉天顏,議加刑闢,但聞兇險之意,尚昧結構之由。況在眾流,何由備悉,忽行峻罰,必謂冤誣,群情震驚,事亦非細。若不付外推鞫,則恐難定罪名,乞留睿聰,更少詳度。竇參於
 臣,素亦無分,陛下固已明知,有何顧懷,輒欲營救?良以事關國體,義絕私嫌,所冀典刑不濫於清時,君道免虧於聖德。」乃再貶為驩州司馬。男景伯,配泉州;女尼真如,隸郴州;其財物婢妾,傳送京師。參時為左右中官深怒,謗沮不已,未至驩州,賜死於邕州武經鎮,時年六十。



 竇申者,參之族子。累遷至京兆少尹,轉給事中。參特愛之,每議除授,多訪於申,申或洩之,以招權受賂。申所至,人目之為喜鵲。德宗頗聞其事,數誡參曰:「卿他日必為申
 所累,不如出之以掩物議。」參曰:「臣無強子侄,申雖疏屬,臣素親之,不忍遠出,請保無他犯。」帝曰:「卿雖自保,如眾人何?」參固如前對。申亦不悛。



 兵部侍郎陸贄與參有隙。吳通微弟兄與贄同在翰林,俱承德宗顧遇,亦爭寵不協。金吾大將軍、嗣虢王則之與申及通微、通玄善,遂相與傾。贄考貢舉,言贄考貢不實。吳通玄取宗室女為外婦,德宗知其毀贄,且令察視,具得其奸狀,乃貶則之為昭州司馬,吳通玄為泉州司馬,竇申為道州司馬。不旬
 日貶參郴州別駕,即日以陸贄為宰相。明年,竇參再貶驩州。德宗謂陸贄曰:「竇申、竇榮、李則之首末同惡,無所不至,又並細微,不比竇參,便宜商量處置,所有親密,並發遣於遠惡處。」贄奏曰:



 竇參罪犯,誠合誅夷,聖德含弘,務全事體,特寬嚴憲,俯貸餘生。始終之恩,實足感於庶品;仁煦之惠,不獨幸於斯人。所議貶官,謹具別狀。其竇申、竇榮、李則之等,即皆同惡,固亦難容;然以得罪相因,法有首從,首當居重,從合從輕。參既蒙恩矜全,申等亦
 宜減降。又於黨與之內,亦有淑慝之殊,稍示區分,足彰沮勸。竇榮與參雖非近屬,亦甚相親,然于款密之中,都無邪僻之事。仍聞激憤,屢有直言,因此漸構猜嫌,晚年頗見疏忌。若論今者陰事,則尚未究端由,如據比來所行,應不至兇險,恐須差異,以表詳明。臣等商量,竇榮更貶遠官,竇申、則之並除名配流,庶允從輕之典,以洽好生之恩。夫趨勢附權,時俗常態,茍無高節出眾,何能特立不群?竇參久塵鈞衡,特承寵渥,君之所任,孰敢不從?
 或游於門庭,或序以中表,或偏被接引,或驟與薦延,如此之徒,十常八九。若聽流議,皆謂黨私,自非甚與交親,安可悉從貶累?況竇參罷黜,殆欲周星,應是私黨近親,當時並已連坐,人心久定,不可復搖。臣等商量,除與竇參陰謀邪事處,一切不問。



 詔從之,由是申等得配流嶺南。既賜參死,乃杖殺申,諸竇皆貶,榮得免死。



 齊映,瀛州高陽人。父圮,試太常少卿,兼檢校工部郎中。映登進士第,應博學宏辭,授河南府參軍。滑亳節度使
 令狐彰闢為掌書記,累授監察御史。彰疾甚,映草遺表,因與謀後事,映說彰令上表請代,令子建歸京師,彰皆從之,因妻以女。彰卒後兵亂,映脫身歸東都,河陽三城使馬燧闢為判官,奏殿中侍御史。建中初,盧杞為宰相,薦之,遷刑部員外郎,會張鎰出鎮鳳翔,奏為判官。映口辯,頗更軍事,數以論奏合旨,尋轉行軍司馬、兼御史中丞。德宗在奉天,鳳翔逼於賊泚。鎰懦緩不曉兵家事,部將有李楚琳者,慓悍兇暴,軍中畏之,乘間將謀亂。先數
 日,映與同列齊抗覺其謀,乃言於鎰,請早圖之。鎰不從映言,乃示其寬大,召楚琳語之曰:「欲令公使於外。」楚琳恐,是夜作亂,乃殺鎰以應泚;軍中多為映指道,故得免。因赴奉天行在,除御史中丞。



 興元初,從幸梁州,每過險,映常執轡。會御馬遽駭,奔跳頗甚,帝懼傷映,令舍轡,映堅執久之,乃止。帝問其故,曰:「馬奔蹶,不過傷臣;如舍之,或犯清塵,雖臣萬死,何以塞責?」上嘉獎無已。在梁州,拜給事中。映白晰長大,言音高朗。上自山南還京,常令映
 侍左右,或令前馬,至城邑州鎮,俾映宣詔令,帝益親信之。其年冬,轉中書舍人。



 貞元二年,以本官與左散騎常侍劉滋、給事中崔造同拜平章事。滋以端默雅重寡言,映謙和美言悅下,無所是非,政事多決於造。無幾,造疾病,映當國政,乘間亦敢言事。時吐蕃數入寇,人情搖動,且言帝欲行幸避狄。映奏曰:「戎狄亂華,臣之罪也。今人情恟懼,謂陛下理裝具糗糧,臣聞大福不再,奈何不與臣等熟計之?」因俯伏流涕,上亦為之感動。時給事中袁
 高忤旨,映連請為左丞、御史大夫。



 映於東都舉進士及宏詞時,張延賞為河南尹、東都留守,厚映。及映為相,延賞罷相為左僕射,數畫時事令映行之,及為所親求官,映多不應。延賞怒,言映非宰相器。三年正月,貶映夔州刺史,又轉衡州。七年,授御史中丞、桂管觀察使,又改洪州刺史、江西觀察使。映常以頃為相輔,無大過而罷,冀其復入用,乃掊斂貢奉,及大為金銀器以希旨。先是,銀瓶高者五尺餘,李兼為江西觀察使,乃進六尺者,至是,
 因帝誕日端午,映為瓶高八尺者以獻。貞元十一年七月卒,時年四十八,贈禮部尚書。



 劉滋,字公茂,左散騎常侍子玄之孫。父貺,開元初為左拾遺,父子仍代為史官。貺依劉向《說苑》撰《續說苑》一十卷以獻,玄宗嘉之。滋少以門廕,調授太子正字,歷漣水令。吏部侍郎楊綰薦滋堪為諫官,拜左補闕,改太常卿,復為左補闕。辭官侍親還東都,河南尹李廙署奏功曹參軍。無幾,丁母喪,服除,遷屯田員外郎,轉司勛員外郎,
 判南曹,勤於吏職,孜孜奉法。遷司勛郎中,累拜給事中。從幸奉天,轉太常少卿,掌禮儀。興元元年,改吏部侍郎,往洪州知選事。時京師寇盜之後,天下蝗旱,穀價翔貴,選人不能赴調,乃命滋江南典選,以便江、嶺之人,時稱舉職。



 貞元二年,遷左散騎常侍、同中書門下平章事,在相位無所啟奏,但多謙退,廉謹畏慎而已。三年正月,守本官,罷知政事。四年,復為吏部侍郎。六年,遷吏部尚書。竇參以宰相為吏部尚書,換刑部尚書。無何,御史臺劾
 奏滋前在吏部選人渝濫,詔奪金紫階。滋有經學,善持論,性廉潔刻苦,嫉惡,掌選多所發手適更代,詐偽者尤畏之。十年十月卒,時年六十六,贈陜州大都督。



 滋從兄贊,大歷中左散騎常侍匯之子。少以資廕補吏,累授鄠縣丞,宰相杜鴻漸自南還朝,途出於鄠,贊儲供精辦。鴻漸判官楊炎以贊名儒之子,薦之,累授侍御史、浙江觀察判官。楊炎作相,擢為歙州刺史,以勤幹聞。有老婦人手君拾榛間,猛獸將噬之,幼女號呼搏獸而救之,母子
 俱免。宣歙觀察使韓滉表其異行,加金紫之服,再遷常州刺史。韓滉入相,分舊所統為三道,以贊為宣州刺史、兼御史中丞、宣歙池都團練觀察使。贊在宣州十餘年。



 贊祖子玄,開元朝一代名儒,父匯博涉經史,唯贊不知書,但以強猛立威,官吏畏之,重足一跡。宣為天下沃饒,贊久為廉察,厚斂殖貨,蝢貢奉以希恩。子弟皆虧庭訓,雖童年稚齒,便能侮易驕人,人士鄙之。貞元十二年卒,時年七十,贈吏部尚書。



 盧邁,字子玄,範陽人。少以孝友謹厚稱,深為叔舅崔祐甫所親重。兩經及第,歷太子正字、藍田尉。以書判拔萃,授河南主簿,充集賢校理。朝臣薦其文行,遷右補闕、侍御史、刑部吏部員外郎。邁以叔父兄弟姊妹悉在江介,屬蝗蟲歲饑,懇求江南上佐,由是授滁州刺史。入為司門郎中,遷右諫議大夫,累上表言時政得失。轉給事中,屬校定考課,邁固讓,以授官日近,未有政績,不敢當上考,時人重之。遷尚書右丞。



 將作監元亙當攝太尉享昭
 德皇后廟,以私忌日不受誓誡,為御史劾奏,詔尚書省與禮官、法官集議。邁奏狀曰:「臣按《禮記》,大夫士將祭於公,既視濯而父母死,猶奉祭。又按唐禮,散齋有大功之喪,致齋有周親喪,齋中疾病,即還家不奉祭事,皆無忌日不受誓誡之文。雖假寧令忌日給假一日,《春秋》之義,不以家事辭王事。今亙以假寧常式,而違攝祭新命,酌其輕重,誓誡則祀事之嚴,校其禮式,忌日乃尋常之制,詳求典據,事緣薦獻,不宜以忌日為辭。」由是亙坐罰俸。



 邁九年以本官同中書門下平章事;歲餘,遷中書侍郎。時大政決在陸贄、趙憬,邁謹身中立,守文奉法而已。而友愛恭儉。邁從父弟MH,為劍南西川判官,卒於成都,歸葬於洛陽,路由京師,邁奏請至城東哭於其柩,許之。近代宰臣多自以為崇重,三服之親,或不過從而吊臨;而邁獨振薄俗,請臨弟喪,士君子是之。十二年九月,邁於政事堂中風,肩輿而歸,上表請罷官,不許,詔宰臣就第問疾。自是凡五上表,堅乞骸骨,詔曰:「卿操履貞方,器識
 淹茂,自居臺輔,益見忠清。方藉謀猷,遽嬰疾疹,歲月滋久,章表屢聞,陳請再三,捴謙難奪。且備養賢之禮,宜遂優閑之秩,告免之誠,雖為懇至,俯從來奏,良用憮然。」乃除太子賓客。貞元十四年卒,時年六十,贈太子太傅,賻以布帛。邁再娶無子,以從父弟子紀為嗣。



 崔損,字至無,博陵人。高祖行功已後,名位卑替。損大歷末進士擢第,登博學宏詞科,授秘書省校書郎,再授咸陽尉。外舅王翃為京兆尹,改大理評事,累遷兵部郎中。
 貞元十一年,遷右諫議大夫。會門下侍郎平章事趙憬卒,中書侍朗平章事盧邁風病請告,戶部尚書裴延齡素與損善,乃薦之於德宗。十二年,以本官同中書門下平章事,與給事中趙宗儒同日知政事,並賜金紫。初,二相有故,旬日中外顒望名德,損比無聲實,及制下之日,中外失望。性齷齪謹慎,每延英論事,未嘗有言。十四年秋,轉門下侍郎平章事。是歲,以昭陵舊宮為野火所焚,所司請修奉。「昭陵舊宮在山上,置來歲久,曾經野火燒
 爇,摧毀略盡,其宮尋移在瑤臺寺左側。今屬通年,欲議修置,緣供水稍遠,百姓勞弊,今欲於見住行宮處修創,冀久遠便人。又為移改舊制,恐禮意未周,宜令宰臣百僚集議。」議者多云:「舊宮既焚,宜移就山下。」上意不欲遷移,只於山上重造,命損為八陵修奉使。於是獻、昭、乾、定、泰五陵造屋五百七十間,橋陵一百四十間,元陵三十間,唯建陵仍舊,但修葺而已。所緣陵寢中床蓐帷幄一事以上,帝親自閱視,然後授損送於陵所。



 損以久疾在
 家,賜絹二百匹以為醫藥。南北兩省清要,損皆歷踐之,在位無稱於人者。身居宰相,母野殯,不言展墓,不議遷祔;姊為尼,沒於近寺,終喪不臨,士君子罪之。加以過為恭遜,接見便僻,不止於容身而已。自建中以後,宰相罕有久在位者,數歲罪黜;損用此中上意,竊大任者八年。上亦知物議鄙其持祿取容,然憐而厚之。貞元十九年卒,贈太子太傅,賻布帛五百端、米粟四百石。



 齊抗,字遐舉,天寶中平陽太守浣之孫。父翱,一命卑官
 卒,以抗貴,累贈國子祭酒。抗少隱會稽剡中讀書,為文長於箋奏。大歷中,壽州刺史張鎰闢為判官,明閑吏事,敏於文學,鎰甚重之。建中初,鎰為江西觀察使,抗亦隨在幕府。三年,鎰自中書侍郎平章事出鎮鳳翔,奏抗為監察御史,仍為賓佐,幕中籌畫,多出於抗。



 德宗在奉天,鎰為李楚琳所害。抗奔赴行在,拜侍御史,旬日改戶部員外郎。宰相蕭復為江淮宣慰使,以抗為判官。德宗還京,大盜之後,天下旱蝗,國用盡竭。鹽鐵轉運使元琇以
 抗有才用,奏授倉部郎中,條理江淮鹽務。貞元初,為水陸運副使,督江淮漕運以給京師。遷諫議大夫。歷處州刺史,轉潭州刺史、湖南都團練觀察使。入為給事中,又為河南尹,歷秘書監、太常卿,代鄭餘慶為中書侍郎、同中書門下平章事。



 先是每年吏部選人試判,別奏官考覆,第其上下;既考,中書門下復奏擇官覆定,浸以為便。抗乃奏曰:「吏部尚書、侍郎,已是朝廷精選,不宜別差考官重覆。」其年他官考判訖,俾吏部侍郎自覆,一歲遂除
 考判官,蓋抗所論奏也。故事,禮部侍郎掌貢舉,其親故即試於考功。謂之「別頭舉人」,抗亦奏罷之。尋奏省諸州府別駕、田曹、司田官及判司之雙曹者,復省中書省驅使官及諸胥吏。尋加修國史。抗雖讀書,無遠智大略,凡為官,必求至精,末乃滋彰,物論薄其隘刻。遇疾,上表請罷,改太子賓客,竟不任朝謝。貞元二十年卒,時年六十五,贈戶部尚書,又賜其家絹二百匹。



 史臣曰:竇參朋黨,不顧君上之誡,斯為悖矣。齊映曲貢
 希用甚謬,而愛君蒞事,往往有長者之言。滋、邁家行修謹,臨事可稱,器雖齷齪,無廢為君子矣。而損、抗之比,夫何足雲,遽汙臺槐,蓋時主之容易耳。



 贊曰:物之同器,貴於弘通。竇阿齊佞,偏詖斯同。滋、邁之行,可以飾躬。康濟蒸民,胡為厥中。



\end{pinyinscope}