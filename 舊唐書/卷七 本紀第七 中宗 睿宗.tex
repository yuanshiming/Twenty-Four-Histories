\article{卷七 本紀第七 中宗 睿宗}

\begin{pinyinscope}

 中宗大和聖
 昭孝皇帝諱顯,高宗第七子,母曰則天順聖皇后。顯慶元年十一月乙丑,生於長安。明年封周王,授洛州牧。儀鳳二年,徙封英王,改名哲,授雍州牧。永隆元年,章懷太子廢,其年立為皇太子。弘道元年十二月,高宗崩,遺詔皇
 太子
 柩前即帝位。皇太后臨朝稱制,改元嗣聖。元年二月,皇太后廢帝為廬陵王,幽於別所。其年五月,遷於均州,尋徙居房陵。聖歷元年,召還東都,立為皇太子,依舊名顯。時張易之與弟昌宗潛圖逆亂。神龍元年正月,鳳閣侍郎張柬之、鸞臺侍郎崔玄暐、左羽林將軍敬暉、右羽林將軍桓彥範、司刑少卿袁恕己等定策率羽林兵誅易之、昌宗,迎皇太子監國,總司庶政。
 大赦天下。鳳閣侍郎韋承慶、正諫大夫房融、司禮卿崔神慶等下獄。甲辰,命地官侍郎樊忱往京師告廟陵。司刑少卿兼相王府司馬袁恕己為鳳閣鸞臺平章事。



 乙巳,則天傳位於皇太子。丙午,即皇帝位於通天宮,大赦天下,唯易之黨與不在原限。為周興、來俊臣所枉陷者,咸令雪免。內外文武官加兩階,三品已上加爵二等,入五品等特減四考。大酺五日。以並州牧相王旦及太平公主有誅易之兄弟功,相王加號安國相王,進拜太尉、同鳳
 閣鸞臺三品;公主加號鎮國太平公主,仍賜實封,通前滿五千戶。皇親先被配沒者,子孫令復屬籍,仍量敘官爵。出宮女三千。丁未,天後徙居上陽宮。庚戌,鳳閣侍郎同鳳閣鸞臺平章事張柬之為夏官尚書、同鳳閣鸞臺三品,封漢陽郡公;鸞臺侍郎兼檢校太子右庶子、同鳳閣鸞臺平章事崔玄暐為守內史,封博陵郡公;袁恕己同鳳閣鸞臺三品,封南陽郡公;敬暉為納言、平陽郡公;桓彥範為納言、譙郡公:並加銀青光祿大夫,賜實封五
 百戶。右羽林大將軍、遼國公李多祚進封遼陽郡王,賜實封六百戶;內直郎、駙馬都尉王同皎為雲麾將軍、右千牛將軍、瑯邪郡公,食實封五百戶。並賞誅張易之兄弟功。其餘封各有差。上天后尊號為則天大聖皇帝。



 二月甲寅,復國號,依舊為唐。社稷、宗廟、陵寢、郊祀、行軍旗幟、服色、天地、日月、寺宇、臺閣、官名,並依永淳已前故事。神都依舊為東都,北都為並州大都督府,老君依舊為玄元皇帝。諸州百姓免今年租稅,房州百姓給復三年。
 改左右肅政臺為左右御史臺。韋承慶貶高要尉,房融配流欽州。中書令楊再思為戶部尚書、同中書門下三品、京留守;太僕卿、同中書門下三品姚元之出為亳州刺史。己未,封堂兄左金吾將軍、鬱林郡公千里為成紀郡王、左金吾衛大將軍,實封五百戶。令貢舉人停習《臣軌》,依舊習《老子》。甲子,立妃韋氏為皇后,大赦天下,內外官陪位者賜勛一轉,大酺三日。後父故豫州刺史玄貞為上洛郡王,後母崔氏贈上洛郡王妃。初,韓王元嘉、霍
 王元軌等自垂拱以來皆遭非命,是日追復官爵,令備禮改葬,有胤嗣者即令承襲,無胤嗣者聽取親為後。詔九品已上及朝集使極言朝政得失,兼舉賢良方正直言極諫之士。丙寅,左散騎常侍、譙王重福貶濮州員外刺史,不知州事。特進、太子賓客、梁王武三思為司空、同中書門下三品,加實封五百戶,通前一千五百戶。丁卯,右散騎常侍、定安郡王、駙馬都尉武攸暨封定王,為司徒,更加實封四百戶,通前一千戶。辛未,上往觀風殿朝
 天後。太尉、安國相王旦固讓太尉及知政事,遂從其請。甲戌,國子祭酒祝欽明同中書門下三品。黃門侍郎、知侍中事韋安石為刑部尚書,罷知政事。丙子,諸州置寺、觀一所,以「中興」為名。丁丑,武三思固讓司空、同中書門下三品,武攸暨固讓司徒、封王,許之。改封義興郡王重俊為衛王,北海郡王重茂為溫王。



 三月辛巳,追復故司空、英國公李勣官爵,令所司為起墳改葬。甲申,制文明已來破家臣僚所有子孫,並還資廕。其揚州構逆徒黨,
 唯徐敬業一房不在免限,餘並原宥。丁亥,廢左右司員外郎。其酷吏劉光業、王德壽、王處貞、屈貞筠、劉景陽等五人,雖已身死,官爵並宜追奪;景陽見在,貶祿州樂單尉。丘神勣、來子珣,萬國俊、周興、來俊臣、魚承曄、王景昭、索元禮、傅游藝、王弘義、張知默、裴籍、焦仁亶、侯思立、郭霸、李敬仁、皇甫文備、陳嘉言等雖已身死,並宜除名。唐奉一配流,李秦授、曹仁哲並改與嶺南遠惡處。己丑,中書侍郎兼檢校相王府長史、南陽郡公袁恕己為中書
 令,兼檢校安國相王府長史。詔曰:「君臣朝序,貴賤之禮斯殊;兄弟大倫,先後之儀亦異。聖人之制,率由斯道。朕臨茲寶極,位在崇高。負扆當陽,雖受宗枝之敬;退朝私謁,仍用家人之禮。近代以來,罕遵軌度,王及公主,曲致私情,姑叔之尊,拜於子侄,違法背禮,情用惻然。自今已後,宜從革弊。安國相王某及鎮國太平公主更不得輒拜衛王重俊兄弟及長寧公主姊妹等。宜告宗屬,知朕意焉。」先是,諸王及公主皆以親為貴,天子之子,諸姑叔
 見之必先致拜,若致書則稱為啟事。上志欲敦睦親族,故下制革之。庚寅,衛王重俊上洛州牧。王乘駟馬車,鹵簿從;諸王公已下、中書門下五品已上及諸親並祖送,禮儀甚盛。事畢,賜物有差。辛卯,以故司僕少卿徐有功執法平恕,追贈越州都督,並授一子官。戊戌,左右千牛衛各置大將軍一員。罷奉宸府官員。以安北大都護、安國相王旦為左右千牛大將軍,每大朝會內供奉。丙午,改秋社依舊用仲秋。戊申,相王旦於太常上。王公諸
 親祖送,衛尉張設,光祿造食。禮畢,賜物如衛王上洛州牧之儀。



 夏四月乙丑,端州尉魏元忠為衛尉卿、同中書門下三品。甲戌,左庶子韋安石為吏部尚書,太子賓客李懷遠為右散騎常侍,右庶子唐休璟為輔國大將軍,右庶子崔玄暐為特進、檢校益州大都督府長史、判都督事,右庶子、西留守、戶部尚書、弘農郡公楊再思為檢校揚州大都督府長史、判都督事,少詹事兼侍讀、國子祭酒祝欽明為刑部尚書:並依前知政事,以上在春宮
 故僚也。乙亥,張柬之為中書令。戊寅,追贈邵王重潤為懿德太子。同官縣大雨雹,燕雀多死,漂溺居人四百家,遣使賑給。五月壬午,遷武氏七廟神主於西京崇尊廟。東都創置太廟社稷。戊子,制依舊以周、隋為二王後。壬辰,封成紀郡王千里為成王。癸巳,侍中敬暉封為平陽郡王;侍中桓彥範扶陽郡王,賜姓韋氏;中書令張柬之漢陽郡王;中書令袁恕己南陽郡王;特進崔玄暐海陵郡王;並加授特進,罷知政事。吏部尚書韋安石為兼中
 書令,兵部尚書魏元忠為兼侍中。丙申,皇后表請天下士庶為出母為三年服,年二十二成丁,五十九免役。癸卯,降梁王武三思為德靜郡王,定王武攸暨為樂壽郡王,河內王武懿宗等十餘人並降為國公。甲辰,特進、芮國公豆盧欽望為尚書左僕射,輔國大將軍、酒泉郡公唐休璟為尚書右僕射:依舊同中書門下三品。丙午,制以鄒魯之邑百戶為太師、隆道公宣尼採邑,用供薦享。又授裔孫褒聖侯崇基朝散大夫,仍許子孫傳襲。



 六月
 丁巳,河北十七州大水,漂沒人居。癸亥,尚書左僕射豆盧欽望,軍國重事中書門下可共平章;檢校中書令韋安石中書令,兼檢校吏部尚書;檢校侍中魏元忠兼檢校兵部尚書;楊再思兼戶部尚書,兼檢校中書令。丁卯,祔孝敬皇帝神主於太廟。廟號義宗,非禮也。戊辰,洛水暴漲,壞廬舍二千餘家,溺死者甚眾。秋七月辛巳,太子賓客韋巨源同中書門下三品。乙未,以特進、漢陽郡王張柬之為襄州刺史,仍不知州事。八月戊申,以水災,令
 文武官九品以上直言極諫。河南洛陽百姓被水兼損者給復一年。甲子,追冊故妃趙氏為恭皇后,尊孝敬妃裴氏為哀皇后。乙亥,上親祔太祖景皇帝、獻祖光皇帝、世祖元皇帝、高祖神堯皇帝、皇祖太宗文武皇帝、皇孝高宗天皇大帝、皇兄義宗孝敬皇帝神主於太廟。皇后廟見。丁丑,御洛城南門觀斗象。九月壬午,親祀明堂,大赦天下。禁《化胡經》及婚娶之家父母親亡停喪成禮。天下大酺三日。戊戌,太子賓客韋巨源為禮部尚書,依舊知政
 事。冬十月癸亥,幸龍門香山寺。乙丑,幸新安。改弘文館為修文館。辛未,魏元忠為中書令,楊再思為侍中。



 十一月戊寅,加皇帝尊號曰應天,皇后尊號曰順天。壬午,皇帝、皇后親謁太廟,告受徽號之意,大赦天下,賜酺三日。己丑,御洛城南門樓觀潑寒胡戲。辛丑,衛王重俊為左衛大將軍,遙領揚州大都督;溫王重茂為右衛大將軍,遙領並州大都督。十二月壬寅,則天皇太后崩。



 二年春正月丙申,護則天靈駕還京。戊戌,吏部尚書李
 嶠同中書門下三品,中書侍郎於惟謙同中書門下平章事。閏月丙午朔,置公主府官員。乙卯,以特進敬暉、桓彥範、袁恕己等三人為滑、洺、豫刺史。二月乙未,刑部尚書韋巨源同中書門下三品。遣十使巡察風俗。丙申,僧會範、道士史崇玄等十餘人授官封公,以常賞聖善寺功也。三月甲辰,中書令韋安石為戶部尚書,罷知政事。戶部尚書蘇瑰為侍中、京留守。乙巳,黃霧四塞。唐休璟請致仕,許之。庚戌,殺光祿卿、駙馬都尉王同皎。壬子,洛
 陽城東七里許,地色如水,側近樹木、往來車馬歷歷影見水中,經月餘乃滅。是月,大置員外官,自京諸司及諸州佐凡二千餘人,超授閹官七品已上及員外者千餘人。壬戌,贈後父韋玄貞太師、益州都督。



 夏四月甲戌,又贈玄貞為酆王,玄貞弟四人並贈郡王。己卯,左散騎常侍、同中書門下三品李懷遠請致仕,許之。辛巳,洛水暴漲,壞天津橋。六月戊寅,特進、朗州刺史、平陽郡王敬暉貶崖州司馬,特進、毫州刺史、扶陽郡王桓彥範瀧州司
 馬,特進、郢州刺史袁恕己竇州司馬,特進、均州刺史、博陵郡王崔玄暐白州司馬,特進、襄州刺史、漢陽郡王張柬之新州司馬,並員外置,長任,舊官封爵並追奪。秋七月丙午,立衛王重俊為皇太子。丙寅,中書令兼檢校兵部尚書齊國公魏元忠為尚書右僕射兼中書令,仍知兵部事;吏部尚書李嶠為中書令;刑部尚書韋巨源為吏部尚書,依舊同中書門下三品。庚午,禮部尚書祝欽明為中丞蕭至忠所劾。前左散騎常侍李懷遠為左散
 騎常侍、同中書門下三品、東都留守。



 九月,祝欽明貶青州刺史。壬寅,幸白馬寺。戊午,左散騎常侍李懷遠卒。壬寅,置戶部侍郎一員。



 冬十月己卯,車駕還京師。戊戌,至自東都。十一月乙巳,大赦天下,行從文武官賜勛一轉。改河南為合宮,洛陽為永昌,嵩陽為登封,陽城為告成。戊午,兼秘書鄭普思坐妖逆配流儋州,其黨與皆伏誅。十二月己卯,突厥默啜寇靈州鳴沙縣,靈武軍大總管沙吒忠義逆擊之,官軍敗績,死者三萬。丁巳,突厥進寇
 原、會等州,掠隴右牧馬萬餘而去。甲申,募能斬默啜者,封授諸大衛大將軍。丙戌,以突厥犯邊,京師亢旱,令減膳徹樂。河北水,大饑,命侍中蘇瑰存撫賑給。丙申,特進、尚書左僕射、兼安國相王府長史、芮國公豆盧欽望為開府儀同三司,依舊平章軍國重事;尚書右僕射兼中書令、知兵部事、齊國公魏元忠為尚書左僕射兼中書令,仍兼知兵部事。是冬,牛大疫。



 三年春正月庚子朔,不受朝會,喪未再期也。庚戌,以默
 啜寇邊,制募猛士武藝超絕者,各令自舉,內外群官各進破滅突厥之策。丙辰,以旱,親錄囚徒。己巳,遣武攸暨、武三思往乾陵祈雨於則天皇后,既而雨降,上大感悅。二月辛未,制武氏崇恩廟依舊享祭,仍置五品令、七品丞,其昊陵、順陵置令、丞如廟。壬午,贈太師、酆王廟號褒德,陵號榮先,置六品令、八品丞。庚寅,改中興寺、觀為龍興,內外不得言「中興」。辛卯,幸安樂公主宅。



 三月丙子,吐蕃贊普遣大臣悉董熱獻方物。是春,自京師至山東疾
 疫,民死者眾。河北、河南大旱。夏四月辛巳,以嗣雍王守禮女為金城公主,出降吐蕃贊普。庚寅,幸薦福寺,曲赦雍州。五月戊戌,左屯衛大將軍兼檢校洛州長史張仁亶為朔方道大總管,以備突厥。丙午,突厥默啜殺我行人臧思言。六月丁卯逆,日有蝕之。戊子,姚雋道討擊使、侍御史唐九徵擊姚州叛蠻,破之,俘虜三千計,遂於其處勒石紀功焉。是夏,山東、河北二十餘州旱,饑饉疾疫死者數千計,遣使賑恤之。秋七月庚子,皇太子重俊與
 羽林將軍李多祚等,率羽林千騎兵三百餘人,誅武三思、武崇訓,遂引兵自肅章門斬關而入。帝惶遽登玄武樓,重俊引兵至下,上自臨軒諭之,眾遂散去,殺李多祚。重俊出奔至鄠縣,為部下所殺。癸卯,大赦天下。八月丙子,改玄武門為神武門,樓為制勝樓。丙戌,左僕射兼中書令魏元忠請致仕,授特進。



 九月丁酉,兵部尚書、郢國公宗楚客,左衛將軍兼太府卿紀處訥並同中書門下三品;吏部侍郎兼左御史臺中丞蕭至忠為黃門侍郎
 兼左御史中丞、同中書門下三品;中書侍郎、東海郡公於惟謙國子祭酒,罷知政事。庚子,上皇帝尊號曰應天神龍,皇后尊號曰順天翊聖,大赦天下,改元為景龍。兩京文武官,三品已上賜爵一級,四品已下加一階,外官賜勛一轉。景龍元年九月甲辰,特進魏元忠左授務川尉,言與重俊通謀也。庚辰,侍中兼左御史臺大夫楊再思為中書令,吏部尚書韋巨源、太府卿紀處訥並為侍中,侍中蘇瑰為吏部尚書。壬戌,改左右羽林衛千騎為
 萬騎,仍分為左右。



 冬十月壬午,彗見於西,月餘而滅。壬午,皇后上《神武頌》,令兩京及四大都督府皆刻之於石。十二月乙丑朔,日有蝕之。丁丑,京師雨土。



 二年春正月丙申,滄州雨雹,大如雞卵。二月辛未,幸左金吾大將軍、陳國公陸頌宅。皇后自言衣箱中裙上有五色雲起,令畫工圖之,以示百僚,乃大赦天下。癸未夜,天保星墜西南,有聲如雷,野雉皆雊。乙酉,帝以後服有慶雲之瑞,大赦天下。內外五品已上母妻各加邑號一
 等,無妻者聽授女;天下婦人八十已上,版授鄉、縣、郡等君。三月丙子,朔方道大總管張仁亶築受降城於河上。



 夏四月庚午,左散騎常侍、樂壽郡王、駙馬都尉武攸暨讓郡王,改封楚國公。癸未,修文館增置大學士八員,直學士十二員。己丑,幸長樂公主莊,即日還宮。



 六月丁亥,改太史局為太史監,罷隸秘書省。秋七月辛卯,臺州地震。癸巳,左屯衛大將軍、攝右御史臺大夫、朔方道行營大總管、韓國公張仁亶同中書門下三品。有赤氣竟天,
 其光燭地,經三日乃止。冬十一月庚申,突厥首領娑葛叛,自立為可汗,遣弟遮弩率眾犯塞。己卯,以安樂公主出降,假皇后仗出於禁中,以盛其儀,帝及後御安福樓以觀之。禮畢,大赦天下,賜酺三日。癸未,安西都護牛師獎與娑葛戰於火燒城,師獎敗績,沒於陣。是冬,西京吏部置兩侍郎銓試,東都又置兩銓,恣行囑請。又有斜封授官,預用秋闕。



 三年春正月丁卯,黃霧四塞。癸酉,幸薦福寺。乙亥,宴侍
 臣及近親於梨園亭。二月己丑,幸玄武門,與近臣觀宮女大酺,既而左右分曹,共爭勝負。上又遣宮女為市肆,鬻賣眾物,令宰臣及公卿為商賈,與之交易,因為忿爭,言辭猥褻。上與後觀之,以為笑樂。壬寅,侍中、舒國公韋巨源為尚書左僕射,並同中書門下三品。戊午,兵部尚書、郢國公宗楚客中書令,中書侍郎、酂國公蕭至忠為侍中,太府卿韋嗣立為兵部尚書、同中書門下三品,中書侍郎、檢校吏部侍郎崔湜同中書門下平章事,兵部
 侍郎趙彥昭為中書侍郎、同中書門下平章事。庚申,日赤紫色,無光。戊寅,禮部尚書兼揚州大都督、曹國公韋溫為太子少保兼揚州大都督、同中書門下三品。太常少卿兼檢校吏部侍郎鄭愔同中書門下平章事。夏五月丙戌,崔湜、鄭愔坐贓,湜貶襄州刺史,愔貶江州司馬。



 六月癸丑,太白晝見於東井。庚子,以經籍多缺,使天下搜括。壬寅,以旱,避正殿,減膳,親錄囚徒。癸卯,尚書右僕射楊再思薨。秋七月乙卯朔,鎮軍大將軍、右驍衛將軍、兼知
 太史事迦業至忠配流柳州。丙辰,娑葛遣使來降。辛酉,幸梨園亭,宴侍臣學士。皇后表請諸婦人不因夫子而加邑號者,許同見任職事官,聽子孫用廕,從之。壬戌,安福門外設無遮齋,三品已上行香。癸亥,御承慶殿,錄囚徒。壬午,遣使冊驍衛大將軍、兼衛尉卿、金河王突騎施守忠為歸化可汗。八月乙酉,特進、行中書令、趙國公李嶠為特進、同中書門下三品,侍中、酂國公蕭至忠為中書令,特進、鄖國公韋安石為侍中。庚寅,諸州各置司田
 參軍一員。吐蕃贊普遣使勃祿星奉進國信、贊普祖娑進物,及上中宮、安國相王、太平公主有差。壬辰,遣十使巡察天下。有星孛於紫宮。令特進佩魚。散職佩魚,自此始也。乙未,親送朔方軍總管、韓國公張仁亶於通化門外,上制序賦詩。乙巳,幸安樂公主山亭,宴侍臣、學士,賜繒帛有差。九月壬戌,幸九曲亭子,宴侍臣、學士。戊辰,吏部尚書、懷縣公蘇瑰為尚書右僕射、同中書門下三品。



 冬十月庚寅,幸安樂公主金城新宅,宴侍臣、學士。十一
 月乙丑,親祀南郊,皇后登壇亞獻,左僕射舒國公韋巨源為終獻。大赦天下,見系囚徒及十惡咸赦除之,雜犯流人並放還。京文武三品已上賜爵一等,四品已下加一階,京官及應襲岳牧入三品五品減考,高年版授。大酺三日。壬申,幸見子陵。甲戌,開府儀同三司、芮國公豆盧欽望薨。吐蕃贊普遣其大臣尚贊吐來逆女。十二月壬戌,前尚書右僕射、宋國公唐休璟為太子少師、同中書門下三品。甲子,上幸新豐之溫湯。庚子,幸兵部尚書
 韋嗣立莊,封嗣立為逍遙公,上親制序賦詩,便游白鹿觀。甲辰,曲賜新豐縣,百姓給復一年,行從官賜勛一轉。是日幸驪山。乙巳,至自溫湯。乙酉,令諸司長官向醴泉坊看潑胡王乞寒戲。



 四年春正月乙卯,於化度寺門設無遮大齋。丙寅上元夜,帝與皇后微行觀燈,因幸中書令蕭至忠之第。是夜,放宮女數千人看燈,因此多有亡逸者。丁卯夜,又微行看燈。丁丑,命左驍衛大將軍、河源軍使楊矩為送金城
 公主入吐蕃使。己卯,幸始平,送金城公主歸吐蕃。



 二月壬午,曲赦咸陽、始平,改始平為金城縣。便幸長安令王光輔馬嵬北原莊。癸未,至自金城。庚戌,令中書門下供奉官五品已上、文武三品已上並諸學士等,自芳林門入,集於梨園球場,分朋拔河,帝與皇后、公主親往觀之。三月甲寅,幸臨渭亭修禊飲,賜群官柳棬以闢惡。丙辰,游宴桃花園。庚申,京師雨木冰,井溢。壬戌,賜宰臣已下內樣巾子。夏四月丁亥,上游櫻桃園,引中書門下五品已上諸
 司長官學士等入芳林園嘗櫻桃,便令馬上口摘,置酒為樂。乙未,幸隆慶池,結彩為樓,宴侍臣,泛舟戲樂,因幸禮部尚書竇希宅。五月辛酉,秘書監、賜虢王邕改封汴王。乙丑,皇后請加嗣王三品。丁卯,前許州司兵參軍燕欽融上書,言皇后干預國政,安樂公主、武延秀、宗楚客等同危宗社。帝怒,召欽融廷見,撲殺之。時安樂公主志欲皇后臨朝稱制,而求立為皇太女,自是與後合謀進鴆。



 六月壬午,帝遇毒,崩於神龍殿,年五十五。秘不發喪,皇后
 親總庶政。癸未,以刑部尚書裴談、工部尚書張錫並同中書門下三品,依舊東都留守。吏部尚書張喜福、中書侍郎岑羲、吏部侍郎崔湜並同中書門下平章事。又命左右金吾衛大將軍趙承恩、右監門大將軍薛簡帥兵五百人往均州,備譙王重福。立溫王重茂為皇太子。甲申,發喪於太極殿,宣遺制。皇太后臨朝,大赦天下,改元為唐隆。見系囚徒常赦所不免者咸赦除之,長流任放歸田里,負犯痕瘕咸從洗滌。內外官三品已上賜爵一
 級,四品已下加一階。以安國相王旦為太子太師。進封雍王守禮為邠王,壽春郡王成器為宋王,宗正卿晉封新興王。丁亥,皇太子即帝位於柩前,時年十六。皇太后韋氏臨朝稱制,大赦天下,常赦所不原者咸赦除之。內外兵馬諸親掌,仍令韋溫總知。時召諸府折沖兵五萬人分屯京城,列為左右營,諸韋子侄分統之。壬辰,遣使諸道巡撫,紀處訥關內道,張嘉福河北道,岑羲河南道。庚子夜,臨淄王諱舉兵誅諸韋、武,皆梟首於安福門外,
 韋太后為亂兵所殺。九月丁卯,百官上謚曰孝和皇帝,廟號中宗。十一月己酉,葬於定陵。天寶十三載二月,改謚曰大和大聖大昭孝皇帝。



 史臣曰:廉士可以律貪夫,賢臣不能輔孱主。誠以志昏近習,心無遠圖,不知創業之難,唯取當年之樂。孝和皇帝越自負扆,遷於房陵,崎嶇瘴癘之鄉,契闊幽囚之地。所以張漢陽徘徊於克復,狄梁公哽咽以奏論,遂得生還,庸非己力。洎滌除金虎,再握璇衡,不能罪己以謝萬
 方,而更漫游以隳八政。縱艷妻之煽黨,則棸、楀爭衡;信妖女以撓權,則彞倫失序。桓、敬由之覆族,節愍所以興戈,竟以元首之尊,不免齊眉之禍。比漢、晉之惠、盈輩為優,茍非繼以命世之才,則土德去也。



 睿宗玄真大聖大興孝皇帝,諱旦,高宗第八子,中宗母弟。龍朔二年六月己未,生於長安。其年封殷王,遙領冀州大都督、單于大都護、右金吾衛大將軍。及長,謙恭孝友,好學,工草隸,尤愛文字訓詁之書。乾封元年,徙封豫
 王。總章二年,徙封冀王。上初名旭輪,至是去「旭」字。上元二年,徙封相王,拜右衛大將軍。儀鳳三年,遷洛牧;改名旦,徙封豫王。嗣聖元年,則天臨朝,廢中宗為廬陵王,立豫王為皇帝,仍臨朝稱制。及革命,改國號為周,降帝為皇嗣,令依舊名輪,徙居東宮,其具儀一比皇太子。聖歷元年,中宗自房陵還。帝數稱疾不朝,請讓位於中宗。則天遂立中宗為皇太子,封帝為相王,又改名旦,授太子右衛率。長安中,拜司徒、右羽林衛大將軍。自則天初臨
 朝及革命之際,王室屢有變故,帝每恭儉退讓,竟免於禍。神龍元年,以誅張易之昆弟功,進號安國相王,遷太尉,加實封。其年立為皇太弟,固辭不受。



 景龍四年夏六月,中宗崩,韋庶人臨朝,引用其黨,分握政柄,忌帝望實素高,潛謀危害。庚子夜,臨淄王諱與太平公主子薛崇簡、前朝邑尉劉幽求、長上果毅麻嗣宗、苑總監鐘紹京等率兵入北軍,誅韋溫、紀處訥、宗楚客、武延秀、馬秦客、葉靜能、趙履溫、楊均等,諸韋、武黨與皆誅之。辛丑,帝挾
 少帝御安福門樓慰諭百姓,大赦天下,見系囚徒常赦所不免者咸赦除之。內外文武官三品已上賜爵一級,四品已下加一階,親皇三等已上加兩階,四等已下及諸親賜勛三轉,天下百姓免今年田租之半。進封臨淄王為平王,以薛崇簡為立節郡王。鐘紹京為中書侍郎,劉幽求為中書舍人,並參知機務,加實封。其餘封賞有差。遣使分行諸道宣諭,仍令往均州慰勞譙王。壬寅,左千牛中郎將、宋王成器為左衛大將軍,司農少卿同正
 員、衡陽王成義為右衛大將軍,太府少卿同正員、巴陵王隆範為左羽林衛大將軍,太僕少卿同正員、彭城王隆業為右羽林衛大將軍。黃門侍郎李日知同中書門下三品。癸卯,殿中兼知內外閑廄、檢校龍武右軍、仍押左右廂萬騎平王諱同中書門下三品。中書侍郎、潁川郡公鐘紹京為中書令。中書令、酂國公蕭至忠為許州刺史,兵部尚書、逍遙公韋嗣立為宋州刺史,中書侍郎趙彥昭為絳州刺史,蕭、韋、趙特置位。誅吏部尚書張嘉
 福於懷州。其日,王公百僚上表,咸以國家多難,宜立長君,以帝眾望所歸,請即尊位。



 甲辰,少帝詔曰:「自古帝王,必有符命,兄弟相及,存諸典禮。朕以孤藐,遭家艱難,顧茲蒙識,未洽治途。茫茫四海,將何所屬,累聖丕基,若墜於地。王室多難,義擇長君,思與群公,推崇明聖。叔父相王,高宗之子,昔以天下,讓於先帝,孝友寬簡,彰信兆人。神龍之初,己有明旨,將立太弟,以為副君。為王懇辭,未行冊命,所以東宮虛位,至於歷年。徹綴在辰,禍變倉卒,
 後掖稱制,計立沖人。欽奉前懷,願遵理命。上申天聖之旨,下遂蒼生之心;俯稽圖緯之文,仰跂祖宗之烈。擇今日,請叔父相王即皇帝位。朕退守本籓,歸於舊邸。凡百卿士,敬承朕言,克贊我天人之休期,光我有唐之勛業。布告遐邇,咸使聞知。」相王上表,讓曰:「臣以宗社事重,家國情深,誅鋤巨逆,奉戴嗣主。今承制旨,猥推宸極。在臣虛薄,不敢祗膺。循環震驚,無任感哽!」制答曰:「皇極大寶,天下至公,王者臨之,蓋非獲已。王先聖舊意,蒼生推仰,
 龍光紫宸,貴允系望。請遵前旨,勿或推讓。」於是少帝遜於別宮。是日即皇帝位,御承天門樓,大赦天下,常赦所不免並原之。內外官四品已上加一階,相王府官吏加兩階。流人長流、長任未還者並放還。立功人王承曄已下千餘人,賜爵秩有差。封少帝為溫王。其日,景雲見。乙巳,中書令鐘紹京為戶部尚書、越國公,實封五百戶;中書舍人劉幽求為尚書左丞、徐國公,實封五百戶:並依前知政事。左衛大將軍、宋王成器為太子太師、雍州牧、揚
 州大都督,加實封二百戶。宮人比來取百姓子女入宮者,放還其家。丙午,新除太常少卿薛稷為黃門侍郎,參知機務。丁未,許州刺史、梁縣侯姚元之為兵部尚書、同中書門下三品,兵部尚書韋嗣立為中書令。追削武三思、武崇訓官爵。戊申,蕭至忠、韋嗣立、趙彥昭、崔湜並停刺史。衡陽王成義封申王,巴陵王降範封岐王,彭城王隆業封薛王。己酉,鎮國太平公主加實封五百戶,通前一萬戶。



 秋七月癸丑,兵部侍郎兼知雍州長史崔日用
 為黃門侍郎,參知機務。丙辰,則天大聖皇后依舊號為天後。追謚雍王賢為章懷太子,庶人重俊曰節愍太子。復敬暉、桓彥範、崔玄暐、張柬之、袁恕己、成王千里、李多祚等官爵。丁巳,河南、洛陽、華州並依舊名。以洛州長史宋璟為檢校吏部尚書、同中書門下三品,中書侍郎岑羲為右散騎常侍。壬戌,以蕭至忠為晉州刺史,韋嗣立為許州刺史,趙彥昭為宋州刺史,兵部尚書姚元之兼太子右庶子,吏部尚書宋璟兼太子左庶子。癸亥,吏部
 侍郎崔湜為尚書右丞,罷知政事。甲子,右僕射許國公蘇瑰、兵部尚書姚元之、吏部尚書宋璟、右常侍判刑部尚書岑羲並充使冊定陵。丙寅,姚元之兼中書令。丁卯,蘇瑰為尚書左僕射,仍舊同中書同下三品。宋國公唐休璟致仕。右武衛大將軍、攝右御史大夫、同中書門下三品、韓國公張仁亶右衛大將軍。戊辰,崔日用為雍州長史,薛稷為右散騎常侍,並停知機務。特進、同中書門下三品、趙國公李嶠為懷州刺史。廢司田參軍。己巳,冊
 平王為皇太子。大赦天下,改元為景雲。內外官九品已上及子為父後者各加勛一轉,自神龍以來直諫枉遭非命者咸令式墓,天下州縣名目天授以來改為「武」字者並令復舊。廢武氏崇恩廟,其昊陵、順陵並去陵名。



 景雲元年七月己巳,制自今授左右僕射、侍中、中書令、六尚書已上官聽讓,其餘停讓。追廢皇后韋氏為庶人,安樂公主為悖逆庶人。丁丑,改太史監為太史局,隸秘書省。八月癸巳,新除集州刺史、譙王重福潛入東都構
 逆,州縣討平之。先是,中宗時官爵渝濫,因依妃、主墨敕而授官者,謂之斜封,至是並令罷免。癸卯,改門下坊為左春坊,典書坊為右春坊,左右羽林衛依舊為左右羽林軍。九月庚戌,封皇太子男嗣貞為許昌郡王,嗣謙為真定郡王。冬十月甲申,詔孝敬皇帝神主先祔太廟,有違古義,於東都別立義宗廟。丁未,姚元之為中書令,兼檢校兵部尚書。十一月己酉,葬孝和皇帝於定陵。辛亥,太子太師、宋王成器為尚書左僕射。蘇瑰為太子少傅,
 侍中、鄖國公韋安石為太子少保,改封郇國公,並罷知政事。戊辰,宋王成器為司徒,兼領揚州大都督。庚午,太子少傅蘇瑰薨。



 是歲,韋庶人、悖逆庶人並以禮改葬,武三思父子剖棺戮尸。二年春正月丁未朔,以山陵日近,不受朝賀。癸丑,改泉州為閩州,置都督府,改武榮州為泉州。突厥默啜遣使請和親,許之。己未,太僕卿郭元振、中書侍郎張說並同中書門下平章事。甲子,改封溫王重茂為襄王,遷於集州。乙丑,
 追尊皇后劉氏為肅明皇后,墓曰惠陵;德妃竇氏為昭成皇后,墓曰靖陵。



 二月丁丑,令皇太子監國。甲辰,姚元之左授申州刺史,宋璟左授楚州刺史。韋安石為侍中。丙戌,劉幽求為戶部尚書,罷知政事。戊子,詔中宗時斜封官並許依舊。庚申,復置太子左右諭德、太子左右贊善,各置兩員。戊戌,郭元振為兵部尚書,仍依舊同中書門下平章事。己未,改修文館為昭文館。黃門侍郎李日知為左臺御史大夫,依舊同中書門下三品。



 夏四月庚
 辰,張說為兵部侍郎,依舊同中書門下平章事。癸未,分瀛州置鄭州。詔以釋典玄宗,理均跡異,拯人化俗,教別功齊。自今每緣法事集會,僧尼、道士、女冠等宜齊行道集。甲申,韋安石為中書令;宋王成器為太子賓客,仍依舊遙領揚州大都督。丙申,李日知為侍中。壬寅,大赦天下,重福徒黨放雪。京官四品已下加一階,外官賜勛一轉,三品已上各賜爵一級。天下濫度僧尼、道士、女冠並依舊。又令內外官依上元元年九品已上文武官,咸帶
 手巾算袋,武官咸帶七事占鞢並足。其腰帶一品至五品並用金,六品七品並用銀,八品九品並用鍮石。魚袋著紫者金裝,著緋者銀裝。景龍三年已前逋懸並放免。天下大酺五日。



 五月庚戌,復武氏昊陵、順陵,仍量置官屬,太平公主為武攸暨請也。庚申,韋安石加開府儀同三司。辛丑,改西城公主為金仙公主,昌隆公主為玉真公主,仍置金仙、玉真兩觀。壬戌,殿中監竇懷貞為左臺御史大夫、同中書門下平章事。



 六月壬午,依漢代故事,
 分置二十四都督府。閏六月,初置十道按察使。



 秋七月,新置都督府並停。唯雍洛州長史、揚益荊並四大都督府長史階為三品。八月乙卯,詔以興聖寺是高祖舊宅,有柿樹,天授中枯死,至是重生,大赦天下。其謀殺、劫殺、造偽頭首並免死配流嶺南,官典受贓者特從放免。天下大酺三日。丁巳,皇太子釋奠於太學。己巳,韋安石為尚書右僕射、同中書門下三品兼太子賓客,禮部尚書竇希玠為太子少傅。庚午,改左右屯衛為左右威衛,左
 右宗衛率府為左右司御府,渾儀監為太史監。九月丁卯,竇懷貞為侍中。



 冬十月甲辰,吏部尚書劉幽求為侍中,散騎常侍魏知古同中書門下三品,太子詹事崔湜為中書侍郎、同中書門下三品,中書侍郎陸象先同中書門下平章事。韋安石為尚書左僕射、東都留守,侍中李日知為戶部尚書,兵部尚書郭元振為吏部尚書,侍中兼檢校左臺御史大夫竇懷貞為左臺御史大夫,兵部侍郎兼左庶子張說為尚書左丞:罷知政事。十一月
 戊寅,改太史監為太史局,依舊隸秘書省。改王師為傅。



 三年春正月辛未朔,親謁太廟。癸酉,上始釋慘服,御正殿受朝賀。甲戌,並、汾、絳三州地震,壞人廬舍。辛巳,南郊。戊子,躬耕籍田。己丑,大赦天下,改元為太極。內外官四品已下加一階,三品已上加爵一級。孔宣父祠廟,本州取側近三十戶以供灑掃。天下大酺五日,特賜老人九十已上緋衫牙笏,八十已上綠衫木笏。乙未,戶部尚書岑羲、左臺御史大夫竇懷貞並同中書門下三品。二月丁
 酉,秘書增少監一員,光祿、大理、鴻臚、太府、衛尉、宗正各增置少卿一員,少府監、將作監增置少監一員,國子監增置司業一員,左右臺各增置中丞一員。雍洛二州、並益荊揚四大都督府各增置司馬一員,仍分為左右司馬。丁亥,皇太子釋奠於國學。追贈顏回為太子太師,曾參為太子太保。每年春秋釋奠,以四科弟子、曾參從祀,列於二十二賢之上。辛酉,廢右御史臺官員。己巳,頒新格式於天下。



 夏四月辛丑,制曰:



 朕聞措刑由於用刑,去
 殺存乎必殺。明罰峻典,自古而然;立制齊人,於是乎在。自我朝建國,僅將百年,天下和平,其來已久。往承隋季,守法頗專;比襲時安,持綱日緩。況朕薄德,甚莫逮先;惟人難理,遠不如昔。粵從守位,三載於茲,庶務煩勞,不損晷景。嘗謂自我作則,感而成化;痛乎迷俗忘返,不威罔懲。將至純風,先歸重典。比者贓賄不息,渝濫公行,放心未寧,禁犯無懼。此焉暫革,期於承平,遂割小慈,以崇大體。自今已後,造偽頭首者斬,仍沒一房資財,同用廕者
 並停奪。非頭首者絞。其承前造偽人,限十日內首使盡。官典主司枉法受贓一匹已上,先決杖一百。其緣贓及惡狀被解及與替者,非選時不得輒入京城。縱家貫在京,不得輒至朝堂,妄有披訴。如有此色,並決杖仍加貶斥。其先在京城者,限三日內勒還。上下官僚輒緣私情相囑者,其受囑人宜封狀奏聞。成器已下,朕自決罰。其餘王公已下,並解見任官,三五年間不須齒錄。其進狀人別加褒賞。御史宜令分察諸司。



 五月戊寅,親祀北郊。
 辛未,大赦天下,改元為延和。桓彥範、敬暉、崔玄暐、張柬之、袁恕己等,特還其子孫實封二百戶。天下大酺五日。六月癸丑,戶部尚書岑羲為侍中。乙卯,追奠則天皇后曰天后聖帝。庚申,幽州都督孫儉率左驍衛將軍李楷洛、左威衛將軍周以悌等,將兵三萬,與奚首領李大輔戰於硎山,為賊所敗,儉沒於陣。壬戌,魏知古為戶部尚書,仍依舊同中書門下三品。秋七月庚午,竇懷貞為尚書右僕射,平章軍國重事。己卯,上觀樂於安福門,以燭
 繼晝,經日乃止。



 八月庚子,帝傳位於皇太子,自稱太上皇帝,五日一度受朝於太極殿,自稱曰朕,三品已上除授及大刑獄,並自決之,其處分事稱誥、令。皇帝每日受朝於武德殿,自稱曰予,三品已下除授及徒罪並令決之,其處分事稱制、敕。甲辰,大赦天下,改元為先天。八月戊申,皇帝子許昌王嗣直改封郯王,真定王嗣謙為郢王。己酉,以宋王成器為司空,依舊遙領揚州大都督。庚戌,竇懷貞為尚書左僕射、同中書門下三品,仍兼御史
 大夫;劉幽求為尚書右僕射,依舊同中書門下三品;魏知古為侍中;崔湜為中書令;並監修國史。丁巳,立皇帝妃王氏為皇后。癸亥,劉幽求配流封州。九月丁卯朔,日有蝕之。甲申,封皇帝子嗣升為陜王。冬十月庚子,皇帝親謁太廟,禮畢,御延喜門,大赦天下。壬寅,祔昭成皇后、肅明皇后神主於儀坤廟。癸卯,皇帝幸新豐之溫湯,校獵於渭川。十二月丁未,誥禁人屠殺犬雞。戊午,改箕州為儀州。



 二年春正月,敕河北諸州團結兵馬,皆令本州刺史押掌。乙亥,吏部尚書兼太子右諭德、酂國公蕭至忠為中書令。上元日夜,上皇御安福門觀燈,出內人連袂踏歌,縱百僚觀之,一夜方罷。二月丙申,改隆州為閬州,始州為劍州。分冀州置深州。初,有僧婆陀請夜開門然燈百千炬,三日三夜。皇帝御延喜門觀燈縱樂,凡三日夜。左拾遺嚴挺之上疏諫之,乃止。



 三月辛卯,皇后祀先蠶。癸巳,制敕表狀、書奏、箋牒年月等數,作一十、三十、四十字。
 夏六月丙辰,兵部尚書、朔方道行軍大總管郭元振加同中書門下三品。秋七月甲子,太平公主與僕射竇懷貞、侍中岑羲、中書令蕭至忠、左羽林大將軍常元楷等謀逆,事覺,皇帝率兵誅之。窮其黨與,太子少保薛稷、左散騎常侍賈膺福、右羽林將軍李慈李欽、中書舍人李猷、中書令崔湜、尚書左丞盧藏用、太史令傅孝忠、僧惠範等皆誅之。兵部尚書郭元振從上御承天門樓,大赦天下,自大闢罪已下,無輕重咸赦除之。翌日,太上皇誥
 曰:』朕將高居無為,自今後軍國刑政一事以上,並取皇帝處分。」



 開元四年夏六月甲子,太上皇帝崩於百福殿,時年五十五。秋七月己亥,上尊謚曰大聖貞皇帝,廟曰睿宗。冬十月庚午,葬於橋陵。天寶十三載二月,改謚曰玄真大聖大興孝皇帝。



 史臣曰:法不一則奸偽起,政不一則朋黨生,上既啟其泉源,下胡息於奔競。觀夫天后之時,雲委於二張之第;孝和之世,波注于三王之門。獻奇則除設盈庭,納賄則
 斜封滿路,咸以進趨相軌,奸利是圖,如火投泉,安得無敗?洎景龍繼統,污俗廓清,然猶投杼於乘輿之間,抵掌於太平之日。以至書頻告變,上不自安,宮臣致禦魅之科,天子慊巡邊之詔。彼既彎弓而射我,我則號泣以行刑。此雖鎮國之尤,亦是臨軒之失。夫君人孝愛,錫之以典刑,納之於軌物,俾無僭逼,下絕覬覦,自然治道惟新,亂階不作。孝和既已失之,玄真亦未為得。



 贊曰:孝和、玄真,皆肖先人。率情背禮,取樂於身。夷塗不
 履,覆轍攸遵。扶持聖嗣,賴有賢臣。



\end{pinyinscope}