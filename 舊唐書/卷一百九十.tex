\article{卷一百九十}

\begin{pinyinscope}

 ○裴懷古張知謇兄知玄知晦弟知泰知默楊元琰倪若水李浚陽嶠宋慶禮姜師度
 強循和逢堯潘好禮楊茂謙楊軿崔隱甫李尚隱呂厓蕭定蔣沇薛玨李惠登任迪簡範傳正袁滋薛蘋閻濟美



 裴懷古,壽州壽春人也。儀鳳中,詣闕上書,授下邽主簿。長壽中,累轉監察御史。時姚、巂蠻首反叛,詔懷古往招
 輯之。懷古申明賞罰,賊徒歸附者日以千數,乃俘其魁首,處其居人而還。蠻夷荷恩,立碑頌德。時恆州鹿泉寺僧凈滿為弟子所謀,密畫女人居高樓,仍作凈滿引弓而射之,藏於經笥。已而詣闕上言僧咒詛,大逆不道。則天命懷古按問誅之。懷古究其辭狀,釋凈滿以聞。則天大怒。懷古奏曰:「陛下法無親疏,當與天下畫一。豈使臣誅無辜之人,以希聖旨?向使凈滿有不臣之狀,臣復何顏能寬之乎?臣今慎守平典,雖死無恨也。」則天意乃解。



 聖歷中,閻知微充使往突厥,懷古監其軍。至虜庭,默啜立知微為南面可汗。將授懷古偽職,懷古不從,將殺之。懷古抗辭曰:「寧守忠以就死,不毀節以求生,請就斬,所不避也!」乃禁錮隨軍,因挺身奔竄以歸,拜祠部員外郎。



 時姚、巂蠻首相率詣闕頌懷古綏撫之狀,請為牧守以撫之。遂授姚州都督。以疾不行,轉司封郎中。時始安賊歐陽倩擁徒數萬,剽陷州縣,授懷古桂州都督,仍充招慰討擊使。才及嶺,飛書招誘,示以禍福,賊徒迎降,自陳
 為吏人侵逼,乃舉兵耳。懷古知其誠懇,乃輕騎以赴之。左右曰:「夷獠難親,未可信也。」懷古曰:「吾仗忠信,可通於神明,況於人乎!」因造其營以慰諭之。群賊喜悅,歸其所掠財貨,納於公府。諸洞酋長素持兩端者,盡來款附,嶺外悉定。



 復歷相州刺史、並州大都督府長史,所在為人吏所慕。神龍中,遷左羽林大將軍,行未達都,復授並州長史。吏人聞懷古還,老幼相攜,郊野歡迎。時崔宣道代懷古為並州,下車而罷,出郊以候懷古。懷古恐傷宣道
 之意,命官吏驅逐出迎之人,而百姓奔赴愈眾,其為人所思如此。俄轉幽州都督,徵為左威衛大將軍。尋卒。



 張知謇,蒲州河東人也,徙家於岐。少與兄知玄、知晦,弟知泰、知默五人,勵志讀書,皆以明經擢第。儀質瑰偉,眉目疏朗,曉於玄理,清介自守,故當時名公爭引薦之,遞歷畿赤。知謇、知泰、知默,調露後又歷臺省。



 知謇,天授後歷房、和、舒、延、德、定、稷、晉、洺、宣、貝十一州刺史,所涖有威嚴,人不敢犯。通天中,知泰為洛州司馬,知默為秋官郎
 中。知謇自德州入計,則天重其才幹,又目其狀貌過人,命畫工寫之,以賜其本。曰:「人或有才,未必有貌,卿家昆弟,可謂兩絕。」時人稱之。尋以知泰為夏官、地官侍郎,益州長史,中臺右丞。



 初,知謇為房州時,中宗以廬陵王安置房州,制約甚急。知謇與董玄質、崔敬嗣相次為刺史,皆保護,供擬豐贍,中宗德之。及神龍元年,中宗踐極,自貝州追知謇為左衛將軍,加雲麾將軍,封範陽郡公。知泰自兵部侍郎授右御史大夫,加銀青光祿大夫,進封
 漁陽郡公。須發華皓,同貴於朝,時望甚美之。



 知泰以忤武三思,出並州刺史、天平軍使,仍帶本官。尋又為魏州刺史。景龍二年卒,優詔褒贈,謚曰定。時知謇為洛州長史、東都副留守。又歷左、右羽林大將軍,同、華州刺史,大理卿致仕。開元中卒,年八十。



 知謇敏於從政,性亮直,不喜有請托求進、無才而冒位者。故子侄經義不精,不許論舉。知默嘗與來俊臣、周興等同掌詔獄,陷於酷吏,子孫禁錮。知泰,開元中累贈刑部尚書、特進。



 知玄子景
 升,知泰子景佚,開元中皆至大官,門列棨戟。



 楊元琰,虢州閿鄉人,隋禮部尚書希曾孫也。初生時,數歲不能言,相者曰:「語遲者神定,此必成大器也。」及長,偉姿儀,以器局見稱。初為平棘令,號為善政。載初中,累遷安南副都護,又歷蘄、蒲、晉、魏、宣、許六州刺史,涼、梁二都督,荊府長史。前後九度清白升進,累降璽書褒美。



 長安中,張柬之代元琰為荊州長史,與元琰泛江中流,言及則天革命,議諸武擅權之狀,元琰發言慷慨,有匡復之
 意。及柬之知政事,奏引元琰為右羽林將軍。至都,柬之謂曰:「記昔江中之言乎?今日之授,意不細也。」乃結元琰與李多祚等,定計誅張易之兄弟。及事成,加雲麾將軍,封弘農郡公,食實封五百戶,仍賜鐵券,恕十死。



 俄而張柬之、敬暉等為武三思所構,元琰覺變,奏請削發出家,仍辭官爵實封。中宗不許。敬暉聞而笑曰:「向不知奏請出家,合贊成其事,剃卻胡頭,豈不妙也。」元琰多須類胡,暉以此言戲之。元琰曰:「功成名遂,不退將危。此由衷之
 請,不徒然也。」暉知其意,瞿然不悅。



 及暉等得罪,元琰竟以先覺獲全。尋加金紫光祿大夫,轉衛尉卿。明年,李多祚等被誅,元琰以曾與多祚同立功,亦被系獄問狀。賴中書侍郎蕭至忠保明之,竟得免罪,又轉光祿卿。景雲中,抗疏請削在身官爵,回贈父官。中宗許之,乃追贈其父越州長史。睿宗即位,三遷刑部尚書,改封魏國公。開元初,拜太子賓客致仕。六年,卒於家,年七十九。



 子仲嗣,密州刺史;仲昌,吏部郎中。



 倪若水,恆州稾城人也。開元初,歷遷中書舍人、尚書右丞,出為汴州刺史。政尚清靜,人吏安之。又增修孔子廟堂及州縣學舍,勸勵生徒,儒教甚盛,河、汴間稱詠不已。



 四年,玄宗令宦官往江南採鵁鶄等諸鳥,路由汴州。若水知之,上表諫曰:「方今九夏時忙,三農作苦,田夫擁耒,蠶婦持桑。而以此時採捕奇禽異鳥,供園池之玩,遠自江、嶺,達於京師,水備舟船,陸倦擔負,飯之以魚肉,間之以稻粱。道路觀者,豈不以陛下賤人貴鳥也!陛下方當
 以鳳皇為凡鳥,麒麟為凡獸,即鵁鶄、鸂鶒,曷足貴也?陛下昔潛龍籓邸,備歷艱虞。今氛昆廓清,高居九五,玉帛子女,充於後庭,職貢珍奇,盈於內府,過此之外,復何求哉?臣承國厚恩,超居重任。草芥賤命,常欲殺身以效忠;葵藿微心,常願隳肝以報主。瞻望庭闕,敢布腹心,直言忤旨,甘從鼎鑊。」手詔答曰:「朕先使人取少雜鳥,其使不識朕意,採鳥稍多。卿具奏其事,辭誠忠懇,深稱朕意。卿達識周材,義方敬直,故輟綱轄之重,委以方面之權。果
 能閑邪存誠,守節彌固,骨鯁忠烈,遇事無隱。言念忠讜,深用嘉慰。使人朕已量事決罰,禽鳥並令放訖。今賜卿物四十段,用答至言。」



 尋入拜戶部侍郎。七年,復授尚書右丞,卒。



 李浚,隴西人,祖世武。睿宗即位,加銀青光祿大夫。上在東宮,選為太子中允。又出為麟州刺史,政有能名。開元初,置諸道按察使,盛選能吏,授浚潤州刺史、江東按察使,累封真源縣子。州人孫處玄以學行著名,浚特加禮
 異,累表薦之,仍令子麟與之結交。處玄竟稱疾不起。浚尋拜虢、潞二州刺史,又拜益州長史、劍南節度使,攝御史大夫。所歷皆以誠信待物,稱為良吏。及去職,咸有遺愛。八年卒官,贈戶部尚書,謚曰成。子麟,自有傳。



 陽嶠,河南洛陽人,其先自北平徙焉,北齊右僕射休之玄孫也。儀鳳中應八科舉,授將陵尉,累遷詹事司直。長安中,桓彥範為左御史中丞,袁恕己為右御史中丞,爭薦嶠,請引為御史。內史楊再思素與嶠善,知嶠不樂搏
 擊之任,謂彥範等曰:「聞其不情願,如何?」彥範曰:「為官擇人,豈待情願。唯不情願者,尤須與之,所以長難進之風,抑躁求之路。」再思然其言,擢為右臺侍御史。景龍末,累轉國子司業。嶠恭謹好學,有儒者之風。又勤於政理,循循善誘。及在學司,時人以為稱職。奏修先聖廟及講堂,因建碑前庭,以紀崇儒之事。



 睿宗即位,拜尚書右丞。時分建都督府以統外臺,精擇良吏,以嶠為涇州都督府,尋停不行。又歷魏州刺史,充袞州都督、荊州長史,為本
 道按察使,所在以清白聞。魏州人詣闕割耳,請嶠重臨其郡,又除魏州刺史。入為國子祭酒,累封北平伯,薦尹知章、範行恭、趙玄默等為學官,皆稱名儒。時學徒漸弛,嶠課率經業,稍行鞭箠,學生怨之,頗有喧謗,乃相率乘夜於街中毆之。上聞而令所由杖殺無理者,由是始息。



 嶠素友悌,撫孤侄如己子。常謂人曰:「吾雖位登方伯,而心不異於曩時一尉耳。」識者甚稱嘆之。尋以年老致仕,卒於家,謚曰敬。



 宋慶禮,洺州永年人。舉明經,授衛縣尉。則天時,侍御史桓彥範受詔於河北斷塞居庸、岳嶺、五回等路,以備突厥,特召慶禮以謀其事。慶禮雅有方略,彥範甚禮之。尋遷大理評事,仍充嶺南採訪使。時崖、振等五州首領,更相侵掠,荒俗不安,承前使人,懼其炎瘴,莫有到者。慶禮躬至其境,詢問風俗,示以禍福。於是安堵,遂罷鎮兵五千人。開元中,累遷貝州刺史,仍為河北支度營田使。



 初,營州都督府置在柳城,控帶奚、契丹。則天時,都督趙文
 翽政理乖方,兩蕃反叛,攻陷州城,其後移於幽州東二百里漁陽城安置。開元五年,奚、契丹各款塞歸附,玄宗欲復營州於舊城。侍中宋璟固爭以為不可,獨慶禮甚陳其利。乃詔慶禮及太子詹事姜師度、左驍衛將軍邵宏等充使,更於柳城築營州城,興役三旬而畢。俄拜慶禮御史中丞,兼檢校營州都督。開屯田八十餘所,追拔幽州及漁陽、淄青等戶,並招輯商胡,為立店肆。數年間,營州倉廩頗實,居人漸殷。



 慶禮為政清嚴,而勤於聽理,
 所歷之處,人吏不敢犯。然好興功役,多所改更。嘗於邊險置阱立槍,以邀賊路,議者頗嗤其不切事也。七年卒,贈工部尚書。太常博士張星議曰:「宋慶禮大剛則折,至察無徒,有事東北,所亡萬計,所謂害於而家,兇於而國。案謚法,好巧自是曰『專』,請謚曰『專』。」禮部員外郎張九齡駁曰:



 慶禮在人苦節,為國勞臣,一行邊陲,三十年所。戶庭可樂,彼獨安於傳遞;稼穡為艱,又能實於軍廩。莫不服勞辱之事而匪懈其心,守貞堅之規而自盡其力,有
 一於此,人之所難。況營州者,鎮彼戎夷,扼喉斷臂,逆則制其死命,順則為其主人,是稱樂都,其來尚矣。往緣趙翽作牧,馭之非才,自經隳廢,便長寇孽。故二十年間,有事東鄙,殭尸暴骨,敗將覆軍,蓋不可勝紀。



 大明臨下,聖謀獨斷,恢祖宗之舊,復大禹之跡。以數千之役徒,無甲兵之強衛,指期遂往,稟命而行。於是量畚築,執沴鼓,親總其役,不愆所慮。俾柳城為金湯之險,林胡生腹心之疾,蓋為此也。尋而罷海運,收歲儲,邊亭晏然,河朔無擾。
 與夫興師之費,轉輸之勞,較其優劣,孰為利害?而云「所亡萬計」,一何謬哉!及契丹背誕之日,懼我掎角之勢,雖鼠穴自固,而駒牧無侵,蓋張皇彼都系賴之力也!安有踐其跡以制其實,貶其謚以徇其虛,採慮始之謗聲,忘經遠之權利,義非得所,孰謂其可?請以所議,更下太常,庶素行之跡可尋,易名之典不墜者也。



 星復執前議,慶禮兄子辭玉又詣闕稱冤,乃謚曰敬。



 姜師度,魏人也。明經舉。神龍初,累遷易州刺史、兼御史
 中丞,為河北道監察兼支度營田使。師度勤於為政,又有巧思,頗知溝洫之利。始於薊門之北,漲水為溝,以備奚、契丹之寇。又約魏武舊渠,傍海穿漕,號為平虜渠,以避海艱,糧運者至今利焉。尋加銀青光祿大夫,累遷大理卿。景雲二年,轉司農卿。



 開元初,遷陜州刺史。州西太原倉控兩京水陸二運,常自倉車載米至河際,然後登舟。師度遂鑿地道,自上注之,便至水次,所省萬計。六年,以蒲州為河中府,拜師度為河中尹,令其繕緝府寺。



 先
 是,安邑鹽池漸涸,師度發卒開拓,疏決水道,置為鹽屯,公私大收其利。再遷同州刺史,又於朝邑、河西二縣界,就古通靈陂,擇地引雒水及堰黃河灌之,以種稻田,凡二千餘頃,內置屯十餘所,收獲萬計。特加金紫光祿大夫,尋遷將作大匠。



 明年,左拾遺劉彤上言:「請置鹽鐵之官,收利以供國用,則免重賦貧人,使窮困者獲濟。」疏奏,令宰相議其可否,咸以為鹽鐵之利,甚裨國用。遂令師度與戶部侍郎強循並攝御史中丞,與諸道按察使計
 會,以收海內鹽鐵。其後頗多沮議者,事竟不行。



 師度以十一年病卒,年七十餘。師度既好溝洫,所在必發眾穿鑿,雖時有不利,而成功亦多。先是,太史令傅孝忠善占星緯,時人為之語曰:「傅孝忠兩眼看天,姜師度一心穿地。」傳之以為口實。



 強循者,鳳州人。亦以吏乾知名,官至大理卿。



 又有和逢堯者,岐州岐山人。性詭譎,有辭辯。睿宗時,突厥默啜請尚公主,許之。逢堯以御史中丞攝鴻臚卿充使報命。既
 至虜庭,默啜遣其大臣謂逢堯曰:「敕書送金鏤鞍,檢乃銀胎金塗,豈是天子意,為是使人換卻。如此虛假,公主必應非實。請還信物,罷和親之事。」遂策馬而去。逢堯大呼,命左右引馬回,謂曰:「漢法重女婿,令送鞍者,只取平安長久之義,何必以金銀為升降耶?若爾,乃是可汗貪金而輕銀,豈是重人而貴信?」默啜聞之,曰:「承前漢使,不敢如此,不可輕也。」遂設宴備禮。逢堯又說默啜令裹頭著紫衫,南面再拜,遣子隨逢堯入朝。



 逢堯以奉使功,驟
 遷戶部侍郎。尋以附會太平公主,左遷朗州司馬。開元中,累轉柘州刺史,卒於官。



 潘好禮,貝州宗城人。少與鄉人孟溫禮、楊茂謙為莫逆之友。好禮舉明經,累授上蔡令,理有異績,擢為監察御史。開元三年,累轉邠王府長史。俄而邠王出為滑州刺史,以好禮兼邠王府司馬,知滑州事。王欲有所游觀,好禮輒諫止之。後王將鷹犬與家人出獵,好禮聞而遮道請還。王初不從,好禮遂臥於馬前,呼曰:「今正是農月,王
 何得非時將此惡少狗馬踐暴禾稼,縱樂以損於人!請先蹋殺司馬,然後聽王所為也!」王慚懼,謝之而還。



 好禮尋遷豫州刺史,為政孜孜,而繁於細事,人吏雖憚其清嚴,亦厭其苛察。其子請歸鄉預明經舉,好禮謂曰:「國法須平,汝若經業未精,則不可妄求也。」乃自試其子。經義未通,好禮大怒,集州僚笞而枷之,立於州門以徇於眾。俄坐事左遷溫州別駕卒。好禮常自以直道,不附於人。又未嘗敘累階勛,服用粗陋,形骸土木,議者亦嫌其邀
 名。



 楊茂謙者,清河人。竇懷貞初為清河令,甚重之。起家應制舉,拜左拾遺,出為臨洺令。時洺州稱茂謙與清漳令馮元淑、肥鄉令韋景駿,皆有政理之聲。茂謙以清白聞,擢為秘書郎。時竇懷貞為相,數稱薦之,由是歷遷大理正、御史中丞。開元初,出為魏州刺史、河北道按察使,與司馬張懷玉本同鄉曲,初善而末隙,遂相糾訐,坐貶桂州都督。尋轉廣州都督,以疾卒。



 楊諲,華陰人。高祖縉,陳中書舍人,以辭學知名。陳亡,始自江左徙關中。祖琮,絳州刺史。諲初為麟游令,時御史大夫竇懷貞檢校造金仙、玉真二觀,移牒近縣,徵百姓所隱逆人資財,以充觀用。諲拒而不受,懷貞怒曰:「焉有縣令卑微,敢拒大夫之命乎?」諲曰:「所論為人冤抑,不知計位高卑。」懷貞壯其對。又中宗時,韋庶人上表請以年二十二為丁限。及韋氏敗,省司舉征租調。諲執曰:「韋庶人臨朝當國,制書非一,或進階卿士,或赦宥罪人,何獨
 於已役中男,重征丁課,恐非保人之術。」省司遂依軿所執,一切免之。諲由是知名,擢拜殿中侍御史。



 開元初,遷侍御史。時崔日知為京兆尹,貪暴犯法。諲與御史大夫李傑將糾劾之。傑反為日知所構,諲廷奏曰:「糾彈之司,若遭恐脅,以成奸人之謀,御史臺固可廢矣。」上以其言切直,遽令傑依舊視事,貶日知為歙縣丞。諲歷遷御史中丞、戶部侍郎。上曾於延英殿召中書門下與諸司尚書及瑒議戶口之事,諲因奏人間損益,甚見嗟賞。時御
 史中丞宇文融奏括戶口,議者或以為不便,敕百僚省中集議。時融方在權要,公卿已下,多雷同融議,諲獨與盡理爭之。尋出為華州刺史。



 十六年,遷國子祭酒,表薦:「滄州人王迥質、瀛州人尹子路、汴州人白履忠,皆經學優長,德行純茂,堪為後生師範,請追授學官,令其教授,以獎儒學之路。」及追至,迥質起家拜諫議大夫,仍為皇太子侍讀;履忠以年老,不任職事,拜朝散大夫,放歸家;子路直弘文館教授。諲又奏曰:「竊見今之舉明經者,主
 司不詳其述作之意,曲求其文句之難,每至帖試,必取年頭月日,孤經絕句。且今之明經,習《左傳》者十無二三。若此久行,臣恐左氏之學,廢無日矣。臣望請自今已後,考試者盡帖平文,以存大典。又《儀禮》及《公羊》、《穀梁》,殆將廢絕,若無甄異,恐後代便棄。望請能通《周》、《儀禮》、《公羊》、《穀梁》者,亦量加優獎。」於是下制:「明經習左氏及通《周禮》等四經者,出身免任散官。」遂著於式。由是生徒為諲立頌於學門之外。再遷大理卿,以老疾辭職。二十三年,拜左
 散騎常侍。尋卒。贈戶部尚書,謚曰貞。



 瑒常嘆《儀禮》廢絕,雖士大夫不能行之。其家子女婚冠及有吉兇之會,皆按據舊文,更為儀注,使長幼遵行焉。



 崔隱甫,貝州武城人,散騎侍郎人鹿之曾孫也。祖濟,太子洗馬。父元彥,太平令。隱甫,開元初再遷洛陽令,理有威名。九年,自華州刺史轉太原尹,人吏刊石頌其美政。十二年,入為河南尹。十四年,代程行諶為御史大夫。時中書令張說當朝用事,隱甫與御史中丞宇文融、李林甫
 劾其犯狀,說遂罷知政事。



 隱甫在職強正,無所回避。自貞觀年李乾祐為御史大夫,別置臺獄,有所鞫訊,便輒系之。由是自中丞、侍御史已下,各自禁人,牢扉常滿。隱甫引故事,奏以為不便,遂掘去之。又憲司故事,大夫已下至監察御史,競為官政,略無承稟。隱甫一切督責,事無大小,悉令諮決;稍有忤意者,便列上其罪,前後貶黜者殆半,群僚側目。是冬,敕隱甫校外官考。舊例皆委細參問,經春未定。隱甫召天下朝集使,一時集省中,一日
 校考便畢,時人伏其敏斷。帝嘗謂曰:「卿為御史大夫,海內咸云稱職,甚副朕之所委也。」



 隱甫既與張說有隙,俄又遞為朋黨,帝聞而惡之,特免官,令歸侍母。歲餘,復授御史大夫。遷刑部尚書,母憂去官。二十一年,起復太原尹,仍為河東採訪處置使。復為刑部尚書,兼河南尹。二十四年,車駕還京,以隱甫為東都留守,為政嚴肅,甚為人吏之所嘆服。尋卒。



 李尚隱,其先趙郡人,世居潞州之銅鞮,近又徙家京兆
 之萬年。弱冠明經累舉,補下邽主簿。時姚珽為同州刺史,甚禮之。景龍中,為左臺監察御史。時中書侍郎、知吏部選事崔湜及吏部侍郎鄭愔同時典選,傾附勢要,逆用三年員闕,士庶嗟怨。尋而相次知政事,尚隱與同列御史李懷讓於殿廷劾之,湜等遂下獄推究,竟貶黜之。時又有睦州刺史馮昭泰,誣奏桐廬令李師等二百餘家,稱其妖逆,詔御史按覆之。諸御史憚昭泰剛愎,皆稱病不敢往。尚隱嘆曰:「豈可使良善陷枉刑而不為申明
 哉!」遂越次請往,竟推雪李師等,奏免之。俄而崔湜、鄭愔等復用,尚隱自殿中侍御史出為伊闕令,懷讓為魏縣令。湜等既死,尚隱又自定州司馬擢拜吏部員外郎,懷讓自河陽令擢拜兵部員外郎。尚隱累遷御史中丞。時御史王旭頗用威權,為士庶所患。會為仇者所訟,尚隱按之,無所容貸,獲其奸贓鉅萬,旭遂得罪。尚隱尋轉兵部侍郎,再遷河南尹。



 尚隱性率剛直,言無所隱,處事明斷。其御下,豁如也。又詳練故事,近年制敕,皆暗記之,所
 在稱為良吏。



 十三年夏,妖賊劉定高夜犯通洛門,尚隱坐不能覺察所部,左遷桂州都督。臨行,帝使謂之曰:「知卿公忠,然國法須爾。」因賜雜彩百匹以慰之。俄又遷廣州都督,仍充五府經略使。及去任,有懷金以贈尚隱者,尚隱固辭之,曰:「吾自性分,不可改易,非為慎四知也。」竟不受之。累轉京兆尹,歷蒲、華二州刺史,加銀青光祿大夫,賜爵高邑伯,入為大理卿,代王鉷為御史大夫。



 時司農卿陳思問多引小人為其屬吏,隱盜錢穀,積至累萬。
 尚隱又舉按之,思問遂流嶺南而死。尚隱三為憲官,輒去朝廷之所惡者,時議甚以此稱之。二十四年,拜戶部尚書、東都留守。二十八年,轉太子賓客。尋卒,年七十五,謚曰貞。



 呂諲,蒲州河東人。志行修整,勤於學業。少孤貧,不能自振。里人程楚賓家富於財,諲娶其女,楚賓及子震皆重其才,厚與資給,遂游京師。天寶初,進士及第,調授寧陵尉,本道採訪使韋陟嘉其才,闢為支使。隴右、河西節度
 使哥舒翰奏充度支判官,累兼衛佐、太子通事舍人。諲性謹守,勤於吏職,雖同僚追賞,而塊然視事,不離案簿,翰益親之,累兼虞部員外郎、侍御史。



 祿山之亂,哥舒翰敗,肅宗即位於靈武,諲馳赴行在。內官硃光輝、李遵驟薦有才,帝深遇之,超拜御史中丞,進奏無不允從。幸鳳翔,遷武部侍郎,賜金紫之服。十月,克復兩京,詔厓與三司官詳定陷賊官陳希烈已下數百人罪戾輕重。諲用法太深,君子薄之。



 乾元二年三月,以本官同中書門下
 平章事,知門下省事。七月,丁母憂免。十月,起復授本官,兼充度支使,遷黃門侍郎。上元元年正月,加同中書門下三品,賜門戟。既立於第門,或謂諲曰:「吉慶之事,不宜兇服受之。」諲遂權釋縗麻,當中而拜,人皆笑其失禮。累加銀青光祿大夫,東平男。



 諲既為相,用妻父程楚賓為衛尉少卿,子震為員外郎。中官馬上言出納詔命,諲暱之。有納賂於上言求官者,諲補之藍田尉。五月,上言事洩笞死,以其肉令從官食之,諲坐貶太子賓客。



 七月,授
 諲荊州大都督府長史、兼御史大夫,充澧、朗、忠、硤五州節度觀察處置等使。諲至治所,上言請於江陵置南都。九月,敕改荊州為江陵府,永平軍團練三千人,以遏吳、蜀之沖。又析江陵置長寧縣。又請割潭、衡、連、道、邵、柳、涪等七州隸江陵府。



 先是,張惟一為荊州長史,己為防禦使,陳希昂為司馬。希昂,衡州酋帥,家兵千人在部下,自為籓衛。有牟遂金仕至將軍,為惟一親將,與希昂積憾。率兵入惟一衙,索遂金之首,惟一懼,即令斬首與之。自
 是軍政歸於希昂。及諲至,奏追希昂赴上都,除侍御史,出為常州刺史、本州防禦使。希昂路由江陵,諲伏甲擊殺之,部下皆斬,積尸於府門。府中懾服,始奏其罪。



 又妖人申奉芝以左道事李輔國,擢為諫議大夫。輔國奏於道州界置軍,令奉芝為軍校,誘引群蠻,納其金帛,賞以緋紫,用囊中敕書賜衣以示之,人用聽信。軍人例衣硃紫,作剽溪洞,吏不敢制,已積年矣。潭州刺史龐承鼎忿之,因奉芝入奏,至長沙,縶之。首贓巨萬,及左道文記,一
 時搜獲,遣使奏聞。輔國黨奉芝,奏召奉芝赴闕。既得召見,具言承鼎曲加誣陷。詔鞫承鼎誣罔之罪,令荊南府按問。諲令判官、監察御史嚴郢鞫之。諲上疏論其事,肅宗怒,流郢於建州。承鼎竟得雪,後奉芝竟以贓敗流死。人重諲之守正,其剛斷不撓,皆此類也。



 初諲作相,與同列李揆不協。及諲被斥二年,以善政聞,揆惡之,因言置軍湖南不便,又使人往荊、湖,密伺諲過。諲知之,乃上疏論揆,揆坐貶袁州長史。



 諲素羸疾,元年建卯月卒,贈吏
 部尚書,有司謚曰肅。故吏度支員外郎嚴郢請以二字曰「忠肅」,博士獨孤及堅議以「肅」為當,從之。諲在臺司無異稱,及理江陵三年,號為良守。初郡人立祠,諲歿後歲餘,江陵將吏合錢十萬,於府西爽塏地大立祠宇,四時祠禱之。



 蕭定,字梅臣,江南蘭陵人,左僕射、宋國公瑀曾孫也。父恕,虢州刺史,以定贈工部尚書。定以廕授陜州參軍、金城丞,以吏事清幹聞。給事中裴遵慶奏為選補黜陟使
 判官。回改萬年主簿,累遷侍御史、考功員外郎、左右司二郎中。為元載所擠,出為秘書少監,兼袁州刺史,歷信、湖、宋、睦、潤五州刺史,所涖有政聲。



 大歷中,有司條天下牧守課績,唯定與常州刺史蕭復、豪州刺史張鎰為理行第一。其勤農桑,均賦稅,逋亡歸復,戶口增加,定又冠焉。尋遷戶部侍郎、太常卿。硃泚之逆,變姓名藏匿裏閭間。京師平,首蒙旌擢,除太子少師。興元元年卒,年七十七,加贈太子太師。



 蔣沇,萊州膠水人,吏部侍郎欽緒之子也。性介獨好學,早有名稱。以孝廉累授洛陽尉、監察御史。與兄演、溶,弟清,俱以幹局吏事擅能名於天寶中。長史韓朝宗、裴迥咸以推覆檢勾之任委之,處事平允,剖斷精當,動為群僚楷式。乾元後,授陸渾、盩厔、咸陽、高陵四縣令。當軍旅之後,瘡痍未平,沇竭心綏撫,所至安輯。副元帥郭子儀每統兵由其縣,必誡軍吏曰:「蔣沇令清而嚴幹,供億故當有素,士眾得蔬飯見饋則足,無撓清政。」其為名人所
 知如此。



 稍遷長安令、刑部郎中、兼侍御史,領渭橋河運出納使。時元載秉政,廉潔守道者多不更職,沇以故滯於郎位,久不徙官。



 大歷十二年,常袞以群議稱沇屈,擢拜御史中丞、東都副留守。尋遷刑部侍郎、刪定副使。改大理卿,持法明審,號為稱職。



 建中元年冬,鑾駕幸奉天,沇奔行在,為賊候騎所拘執,欲以偽職誘之,因絕食稱病,潛竄裏閭間。京師平,首蒙旌擢,拜右散騎常侍。尋以疾終,年七十四,追贈工部尚書。



 薛玨,字溫如,河中寶鼎人。祖寶胤,邠州刺史。父紘,蒲州刺史。玨少以門廕授懿德太子廟令,累授乾陵臺令。無幾,拜試太子中允,兼渭南尉,奏課第一。間歲,復以清名尤異聞,遷昭德令。縣人請立碑紀政,玨固讓不受。遷楚州刺史、本州營田使。



 先是,州營田宰相遙領使,刺史得專達,俸錢及他給百餘萬,田官數百員,奉廝役者三千戶,歲以優授官者復十餘人。玨皆條去之,十留一二,而租入有贏。為觀察使誣奏,左授硤州刺史,遷陳州刺史。



 建中初,上分命使臣黜陟官吏,使淮南李承以玨楚州之去煩政簡,使山南趙贊以玨硤州之廉清,使河南盧翰以玨之肅物,皆以陟狀聞,加中散大夫,賜紫。宣武軍節度使劉玄佐署奏兼御史大夫、汴宋都統行軍司馬。無幾,李希烈自汴州走,除玨汴州刺史,遷河南尹,入為司農卿。



 當是時,詔天下舉可任刺史、縣令者,殆有百人。有詔令與群官詢考,及延問人間疾苦,及胥吏得失,取其有惻隱、通達事理者條舉,什才一二。宰相將以辭策校
 之。玨曰:「求良吏不可兼責以文學,宜以聖君愛人之本為心。」執政卒無難之,皆敘進官,頗多稱職。



 貞元五年,拜京兆尹。玨剛嚴明察,練達法理,以勤身率下,失於纖巧,無文學大體。八年,坐竇參改太子賓客。無幾,除嶺南節度觀察使。以疾卒,年七十四,廢朝一日,贈工部尚書。有子存慶,自有傳。



 李惠登,平盧人也。少為平盧裨將。安祿山反,遂從兵馬使董秦海轉收滄、棣等州,輕師遠鬥,賊不能支。史思明
 反,復陷於賊。脫身投山南節度使來瑱,奏授試金吾衛將軍。



 李希烈反,授惠登兵二千,鎮隋州。貞元初,舉州歸順,授隋州刺史、兼御史中丞。遭李忠臣、希烈殲殘之後,野曠無人。惠登樸素不知學,居官無拔萃,率心為政,皆與理順。利人者因行之,病人者因去之,二十年間,田疇闢,戶口加。諸州奏吏入其境,無不歌謠其能。及于頔為山南東道節度,以其績上聞,加御史大夫,升其州為上。尋加檢校國子祭酒。及卒,加贈洪州都督。



 任迪簡,京兆萬年人。舉進士。初為天德軍使李景略判官。性重厚,嘗有軍宴,行酒者誤以醯進。迪簡知誤,以景略性嚴,慮坐主酒者,乃勉飲盡之,而偽容其過,以酒薄白景略,請換之,於是軍中皆感悅。及景略卒,眾以迪簡長者,議請為帥。監軍使聞之,拘迪簡於別室,軍眾連呼而至,發戶扃取之。表聞,德宗使察焉,具以軍情奏,除豐州刺史、天德軍使,自殿中授兼御史大夫,再加常侍。追入,拜太常少卿、汝州刺史、左庶子。



 及張茂昭去易定,以
 迪簡為行軍司馬。既至,屬虞候楊伯玉以府城叛,俄而眾殺之。迪簡兵馬使張佐元又叛,迪簡政殺之,乃得入。尋加檢校工部尚書,充節度使。



 初,茂昭奢蕩不節,公私殫罄。迪簡至,欲饗士,無所取給,乃以糲食與士同之。身居戟門下凡周月,軍吏感之,請歸堂寢,迪簡乃安其位。三年,以疾代,除工部侍郎,至京,竟不能朝謝。改太子賓客卒,贈刑部尚書。



 範傳正,字西老,南陽順陽人也。父倫,戶部員外郎,與郡
 人李華敦交友之契。傳正舉進士,又以博學宏辭及書判皆登甲科,授集賢殿校書郎、渭南尉,拜監察、殿中侍御史。自比部員外郎出為歙州刺史,轉湖州刺史,歷三郡,以政事修理聞。擢為宣歙觀察使,受代至京師,憲宗聞其里第過侈,薄之,因拜光祿卿。以風恙卒,贈左散騎常侍。



 傳正精悍有立,好古自飭。及為廉察,頗事奢侈,厚以財貨問遺權貴,視公蓄如私藏,幸而不至甚敗。褐衣時游西邊,著《西陲要略》三卷。



 袁滋,字德深,陳郡汝南人也。弱歲強學,以外兄道州刺史元結有重名,往來依焉。每讀書,玄解旨奧,結甚重之。無何,黜陟使趙贊以處士薦,授試校書郎。何士干鎮武昌,闢為從事,累官詹事府司直。部有邑長,下吏誣以盜金,滋察其冤,竟出之。御史中丞韋縚聞之,薦為侍御史,轉工部員外郎。



 貞元十九年,韋皋始通西南蠻夷,酋長異牟尋貢琛請使,朝廷方命撫諭,選郎吏可行者,皆以西南遐遠憚之。滋獨不辭,德宗甚嘉之,以本官兼御史
 中丞,持節充入南詔使。未行,遷祠部郎中,使如故。來年夏,使還,擢為諫議大夫。俄拜尚書右丞,知吏部選事。出為華州刺史、兼御史中丞、潼關防禦使、鎮國軍使。以寬易清簡為政。百姓有至自他境者,皆給地以居,名其居曰義合里。專以慈惠為本,人甚愛之。然百姓有過犯者,皆縱而不理。擒盜輒舍,或以物償之。徵拜金吾衛大將軍,耆耋鰥寡遮道不得進。楊於陵代其任,宣言謂百姓曰:「於陵不敢易袁公之政。」然後羅拜而訣。



 上始監國,與
 杜黃裳俱為相,拜中書侍郎、平章事。會韋皋歿,劉闢擁兵擅命,滋持節安撫。行及中路,拜檢校吏部尚書、平章事、劍南西川節度使,百姓立生祠禱之。徵拜戶部尚書,連為荊襄二帥,改彰義軍節度、隨唐鄧申光等州觀察使。逆賊吳元濟與官軍對壘者數年,滋竟以淹留無功,貶撫州刺史。未幾,遷湖南觀察使卒,年七十,贈太子少保。



 滋工篆籀書,雅有古法。因使行,著《雲南記》五卷。嘗讀劉暉《悲甘陵賦》,嘆其褒善懲惡雖失《春秋》之旨,然其文
 不可廢,因著《甘陵賦後序》。



 子都,仕至翰林學士。



 薛蘋,河東寶鼎人也。少以吏事進,累官至長安令,拜虢州刺史。朝廷以尤課擢為湖南觀察使,又遷浙江東道觀察使,以理行遷浙江西道觀察使。廉風俗,守法度,人甚安之。理身儉薄,嘗衣一綠袍,十餘年不易,因加賜硃紱,然後解去。



 蘋歷三鎮,凡十餘年,家無聲樂,俸祿悉以散諸親族故人子弟。除左散騎常侍致仕。時有年過懸車而不知止者,唯蘋年至而無疾請告,角巾東洛,時甚
 高之。卒,年七十四,贈工部尚書。



 閻濟美,登進士第。累歷臺省,有長者之譽。自婺州刺史為福建觀察使,復為潤州刺史、浙西觀察使。所至以簡淡為理,兩地之人,常賦之外,不知其他。入拜右散騎常侍。華州刺史、潼關防禦、鎮國軍使,入為秘書監。以年及懸車,上表乞骸骨,以工部尚書致仕。後以恩例,累有進改。及歿於家,年九十餘。



 贊曰:聖人造世,才傑濟時。在理致治,無為而為。坑鷫非
 議,簡易從規。樂只君子,邦家之基。



\end{pinyinscope}