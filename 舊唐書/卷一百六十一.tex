\article{卷一百六十一}

\begin{pinyinscope}

 ○王翃兄翊郗士美李鄘子柱柱子磎辛秘馬手韋弘景王彥威



 王翃,太原晉陽人也。兄翊,乾元中累官至京兆少尹。性
 謙柔,淡於聲利。自商州刺史遷襄州刺史、山南東道節度觀察等使。入朝,充北蕃宣慰使,稱職。代宗素重之,及即位,目為純臣。遷刑部侍郎、御史中丞。居憲司,雖不能舉振綱條,然以謹重知名。大歷二年卒。



 翊為侍郎時,翃自折沖授辰州刺史,遷朗州,有威望智術,所蒞立名。大歷五年遷容州刺史、容管經略使。



 自安、史之亂,頻詔徵發嶺南兵募,隸南陽魯炅軍。炅與賊戰於葉縣,大敗,餘眾離散。嶺南溪洞夷獠,乘此相恐為亂,其首領梁崇牽,
 自號「平南十道大都統」。及其黨覃問等,誘西原賊張侯、夏永攻陷城邑,據容州。前後經略使陳仁琇、李抗、侯令儀、耿慎惑、元結、長孫全緒等,雖容州刺史,皆寄理藤州,或寄梧州。



 及翃至藤州,言於眾曰:「吾為容州刺史,安得寄理他邑!」乃出私財募將健,許奏以好爵,以是人各盡力。不數月,斬賊魁歐陽珪。馳於廣州,見節度使李勉,求兵為援。勉曰:「容州陷賊已久,群獠方強,卒難圖也。若務速攻,只自敗耳,郡不可復也。」翃請曰:「大夫如未暇出師,
 但請移牒諸州,揚言出千兵援助,冀藉聲勢,成萬一之功。」勉然之。翃乃以手札告諭義州刺史陳仁璀、藤州刺史李曉庭等,同盟約討賊。翃復募三千餘人。力戰,日數合。節度使牒止翃用兵。翃慮惑將士,匿其牒,奮起士卒,大破賊數萬眾,擒其帥梁崇牽。賊遁數百里外,盡復容州故境。翃發使以聞,奏置順州,以遏餘寇。前後大小百餘戰,生擒賊帥上獻者七十餘人。累加銀青光祿大夫、兼御史中丞,充招討處置使。



 翃又令其將張利用、李實
 等分兵討襲西原。遂收復鬱林諸州,部內漸安。後因哥舒晃殺節度使呂崇賁,嶺南復亂。翃遣大將李實悉所管兵赴援廣州。西原賊率覃問復招合夷獠曰:「容州兵馬盡赴廣州,郡可圖也。」於是悉眾來襲。翃知其來,伏兵御之,生擒覃問,其眾大敗。代宗聞而壯之,遣中使慰勞,加金紫光祿大夫。



 時西蕃入寇河中,元帥郭子儀統兵備之。乃徵翃為河中少尹,充節度留後,領子儀之務。有悍將凌正者,橫暴擾軍政,約其徒夜噪斬關以逐翃。有
 告者,翃縮夜漏數刻,以差其期。賊驚而遁,卒誅正,軍城乂安。



 歷汾州刺史、京兆尹。屬發涇原兵討李希烈,軍次滻水。翃備供頓,肉敗糧臭,眾怒以叛。翃奔至奉天,加御史大夫,改將作監,從幸山南。車駕還京,改大理卿。出為福州刺史、福建觀察使,入為太子賓客。



 貞元十二年,檢校禮部尚書,代董晉為東都留守,判尚書省事、東畿汝防禦使。凡開置二十餘屯,市勁筋良鐵以為兵器,簡練士卒,軍政頗修。無何,吳少誠阻命,翃賦車籍甲,不待完
 繕,東畿之人賴之。十八年卒,時七十餘,贈禮部尚書。



 郗士美,字和夫,高平金鄉人也。父純,字高卿,為李邕、張九齡等知遇,尤以詞學見推。與顏真卿、蕭穎士、李華皆相友善。舉進士,繼以書判制策,三中高第,登朝歷拾遺、補闕、員外、郎中、諫議大夫、中書舍人。處事不回,為元載所忌。魚朝恩署牙將李琮為兩街功德使。琮暴橫,於銀臺門毀辱京兆尹崔昭。純詣元載抗論,以為國恥,請速論奏。載不從,遂以疾辭。退歸東洛凡十年,自號「伊川田
 父。」清名高節,稱於天下。及德宗即位,崔祐甫作相,召拜左庶子、集賢學士。到京,以年老乞身,表三上。除太子詹事致仕,東歸洛陽。德宗召見,屢加褒嘆,賜以金紫。公卿大夫皆賦詩祖送於都門,搢紳以為美談。有文集六十卷行於世。



 士美少好學,善記覽。父友顏真卿、蕭潁士輩嘗與之討論經傳,應對如流。既而相謂曰:「吾曹異日,當交於二郗之間矣。」未冠,為陽翟丞。李抱真鎮潞州,闢為從事,雅有參贊之績。其後易二帥,皆詔士美佐之。



 由坊
 州刺史為黔州刺史、兼御史大夫、持節黔中經略招討觀察鹽鐵等使。時溪州賊帥向子琪連結夷獠,控據山洞,眾號七八千。士美設奇略討平之。詔書勞慰,加檢校右散騎堂侍,封高平郡公,再遷京兆尹。每別殿延問,必咨訪大政。出為鄂州觀察使。



 貞元十八年,伊慎有功,特授安黃節度。二十年,慎來朝,其子宥主留事,朝廷未能去。會宥母卒於京師,利主軍權,不時發喪。士美命從事托以他故過其境。宥果迎之,告以兇問,先備肩籃,即日
 遣之。



 元和五年,拜河南尹。明年三月,檢校工部尚書、潞州大都督府長史,充昭義節度。前政之豐給浮費,至皆減損,號令嚴肅。



 及朝廷討王承宗,士美遣兵馬使王獻領勁兵一萬為先鋒。獻兇惡恃亂,逗撓不進;遽令召至,數其罪斬之。下令曰:「敢後出者斬!」士美親鼓之。兵既合,而賊軍大敗,下三營,環柏鄉,屢以捷聞。上大悅曰:「吾故知士美能辦吾事。」於時四面七、八鎮兵共十餘萬,以環鎮、冀,未有首功,多犯法。士美兵士勇敢畏法,威聲甚振。
 承宗大懼,指期有破亡之勢,會詔班師,至今兩河間稱之。



 十二年,以疾徵為工部尚書。稍間,拜忠武節度使、檢校刑部尚書。至鎮逾月,寢疾。元和十四年九月卒,年六十四。贈尚書左僕射,謚曰景。



 士美善與人交,然諾之際豁如也,當時名稱翕然。



 李鄘,字建侯,江夏人。北海太守邕之侄孫。父暄,官至起居舍人。鄘大歷中舉進士,又以書判高等,授秘書正字。為李懷光所闢,累遷監察御史。及懷光據蒲津叛,鄘與
 母、妻陷賊中。恐禍及親,因偽白懷光曰:「兄病在洛,請母往視之。」懷光許焉,且戒妻子無得從。鄘皆遣行。後懷光知,責之。對曰:「鄘名隸軍籍,不得隨侍老母,奈何不使婦隨姑行也。」懷光無以罪之。時與故相高郢同在賊廷,乃密奏賊軍虛實及攻取之勢。德宗賜手詔以勞之。後事洩,懷光嚴兵召郢與鄘詰責。鄘詞激氣壯,三軍義之。懷光不敢殺,囚之獄中。懷光死,馬燧就獄致禮,表為河東從事。尋以言不行,歸養洛中。襄州節度使嗣曹王皋致
 禮延闢,署從事,奏兼殿中侍御史。入為吏部員外郎。



 徐州張建封卒,其子愔為將校所迫,俾領軍務。詔擇臨難不懾者,即其軍以諭之,遂命鄘為徐州宣慰使。鄘直抵其軍,召將士,傳朝旨,陳禍福,脫監軍使桎梏,令復其位。兇黨不敢犯。及愔上表稱兵馬留後,鄘以為非詔令所加,不宜稱號,立使削去,方受其表。遷吏部郎中。



 順宗登極,拜御史中丞,遷京兆尹、尚書右丞。元和初,以京師多盜,復選為京兆尹,擒奸禁暴,威望甚著。尋拜檢校禮部
 尚書、鳳翔尹、鳳翔隴右節度使。是鎮承前命帥,多用武將,有「神策行營」之號。初受命,必詣軍修謁。鄘既受命,表陳其不可,詔遂去「神策行營」字,但為鳳翔隴右節度。未幾,遷鎮太原,入為刑部尚書、兼御史大夫、諸道鹽鐵轉運使。



 五年冬,出為揚州大都督府長史、淮南節度使。鄘前在兩鎮,皆以剛嚴操下,遽變舊制,人情不安,故未幾即改去。至淮南數歲,就加檢校左僕射,政嚴事理,府廩充積。



 及王師征淮夷,鄆寇李師道表裏相援。鄘發楚、壽
 等州二萬餘兵,分壓賊境,日費甚廣,未嘗請於有司。時憲宗以兵興,國用不足,命鹽鐵副使程異乘驛諭江淮諸道,俾助軍用。鄘以境內富實,乃大籍府庫,一年所蓄之外,咸貢於朝廷。諸道以鄘為倡首,悉索以獻,自此王師無匱乏之憂。



 先是,吐突承璀監淮南軍,貴寵莫貳。鄘亦以剛嚴素著,而差相敬憚,未嘗稍失。承璀歸,遂引以為相。十二年,徵拜門下侍郎、同平章事。鄘出入顯重,素不以公輔自許,年侵勢過,頗安外鎮。登祖筵,聞樂而泣
 下,曰:「宰相之任,非吾所長也。」行頗緩,至京師,又辭疾歸第。既未朝謁,亦不領政事,竟以疾辭,改授戶部尚書。俄換檢校左僕射,兼太子賓客,分司東都。尋以太子少傅致仕。元和十五年八月卒,贈太子太保,謚曰肅。



 鄘強直無私飾,與楊憑、穆質、許孟容、王仲舒友善,皆任氣自負。然鄘當官嚴重,為吏以峻法立操,所至稱理,而剛決少恩。鎮揚州七年,令行禁止。擒擿生殺,一委軍吏,參佐束手,居人頗陷非法,物議以此少之。子柱,官至浙東觀察
 使。



 柱子磎,字景望,博學多通,文章秀絕。大中十三年,一舉登進士第。歸仁晦鎮大梁,穆仁裕鎮河陽,自監察、殿中相次奏為從事。入為尚書水部員外郎,累遷吏部郎中,兼史館修撰,拜翰林學士、中書舍人。廣明中,分司洛下。遇巢、讓之亂,逃於河橋。光啟中,避亂淮海,有偽襄王詔命,磎皆不從。



 王鐸鎮滑臺,杖策詣之。鐸表薦於朝。昭宗雅重之,復召入翰林為學士,拜戶部侍郎,遷禮部尚書。



 景福二年十月,與韋昭度並命中書門下平章事。宣
 制日,水部郎中、知制誥劉崇魯掠其麻哭之,奏云:「李磎奸邪,挾附權幸,以忝學士,不合為相。」時宰臣崔昭緯與昭度及磎素不相協,密遣崇魯沮之也,乃左授太子少師。磎因上十章及《納諫論》三篇自雪,且數崇魯之惡。議者重其才而鄙其訟。昭宗素愛其才,而急於大用。至乾寧初,又上第十一表,乃復命為相。數月,與昭度同為王行瑜等所殺。



 磎自在臺省,聚書至多,手不釋卷,時人號曰「李書樓」。所撰文章及注解書傳之闕疑,僅百餘卷,經
 亂悉亡。王行瑜死,德音昭雪,贈司徒,謚曰文。



 子沇,字東濟,有俊才。與父同日遇害,詔贈禮部員外郎。



 辛秘,隴西人。少嗜學。貞元年中,累登《五經》、《開元禮》科,選授華原尉,判入高等,調補長安尉。高郢為太常卿,嘉其禮學,奏授太常博士。遷祠部、兵部員外郎,仍兼博士。山陵及郊丘二禮儀使,皆署為判官。當時推其達禮。



 元和初,拜湖州刺史。未幾,屬李錡命,將收支郡,遂令大將監守五郡。蘇常杭睦四州刺史,或以戰敗,或被拘執。賊黨
 以秘儒者,甚易之。秘密遣衙門將丘知二勒兵數百人,候賊將動,逆戰大破之。知二中流矢墜馬,起而復戰,斬其將,焚其營,一州遂安。賊平,以功賜金紫,由是僉以秘材堪將帥。



 及太原節度範希朝領全師出討王承宗,徵秘為河東行軍司馬,委以留務。尋召拜左司郎中,出為汝州刺史。



 九年,徵拜諫議大夫,改常州刺史,選為河南尹。蒞職修政,有可稱者。



 十二年,拜檢校工部尚書,代郗士美為潞州大都督府長史、御史大夫,充昭義軍節度、
 澤潞磁洺邢等州觀察使。是時以再討王承宗,澤潞壓境,凋費尤甚。朝議以兵革之後,思能完復者,遂以命秘。凡四歲,府庫積錢七十萬貫,餱糧器械稱是。



 及歸,道病,先自為墓志。將歿,又為書一通,命緘致幾上。其家發之,皆送終遵儉之旨。久歷重任,無豐財厚產,為時所稱。元和十五年十二月卒,年六十四。贈左僕射,謚曰昭。



 馬手,字會元,扶風人。少孤貧好學。性剛直,不妄交游。貞元中,姚南仲鎮滑臺,闢為從事。南仲與監軍使不葉,監
 軍誣奏南仲不法。及罷免,手坐貶泉州別駕,監軍入掌樞密。福建觀察使柳冕希旨欲殺手,從事穆贊鞫手,贊稱無罪,手方免死。後量移恩王傅。



 元和初,遷虔州刺史。四年,兼御史中丞,充嶺南都護、本管經略使。手敦儒學,長於政術。在南海累年,清廉不撓,夷獠便之。於漢所立銅柱之處,以銅一千五百斤特鑄二柱,刻書唐德,以繼伏波之跡。以綏蠻功,就加金紫。



 八年,轉桂州刺史、桂管經略觀察使,入為刑部侍郎。裴度宣慰淮西,奏為制置
 副使。吳元濟誅,度留手蔡州,知彰義軍留後。尋檢校工部尚書、蔡州刺史、兼御史大夫,充淮西節度使。手以申、光、蔡等州久陷賊寇,人不知法,威刑勸導,咸令率化。奏改彰義軍曰淮西,賊之偽跡,一皆削蕩。



 十三年,轉許州刺史、忠武軍節度、陳許溵等州觀察處置等使。明年,改華州刺史、潼關防禦、鎮國軍等使。



 十四年,遷檢校刑部尚書、鄆州刺史、天平軍節度、鄆曹濮等州觀察等使,就加檢校尚書左僕射。入為戶部尚書。長慶三年卒,贈右
 僕射。



 手理道素優,軍政多暇,公務之餘,手不釋卷。所著《奏議集》、《年歷》、《通歷》、《子鈔》等書百餘卷,行於世。



 韋弘景,京兆人,後周逍遙公夐之後。祖嗣立,終宣州司戶。父堯,終洋州興道令。弘景貞元中始舉進士,為汴州、浙東從事。



 元和三年,拜左拾遺,充集賢殿學士,轉左補闕。尋召入翰林為學士。普潤鎮使蘇光榮為涇原節度使,弘景草麻,漏敘光榮之功,罷學士,改司門員外郎,轉吏部員外、左司郎中,改吏部度支郎中。張仲方貶李吉
 甫謚,上怒,貶仲方。弘景坐與仲方善,出為綿州刺史。宰相李夷簡出鎮淮南,奏為副使,賜以金紫。入為京兆少尹,遷給事中。



 劉士涇以駙馬交通邪幸,穆宗用為太僕卿。弘景與給事薛存慶封還詔書,諭士涇曰:「伏以司僕正卿,位居九列。在周之命,伯冏其人,所以惟月膺名,象河稱重。漢朝亦以石慶之謹願,陳萬年之行潔,皆踐斯職,謂之大僚。今士涇戚里常人,班敘散秩,以父任將帥,家富貲財,聲名不在於士林,行義無聞於朝野,忽長卿
 寺,有瀆官常。以親則人物未賢,以勛則寵待常厚,今叨顯任,誠謂謬官。《傳》曰:『惟名與器,不可假人。』蓋士涇之謂。臣等職司違失,實在守官。其劉士涇新除太僕卿敕,未敢行下。」穆宗遣宰臣宣諭,弘景等固執如前。宰臣不得已,改衛尉少卿。穆宗復遣諭弘景曰:「士涇父昌有邊功,士涇為少列十餘年,又尚雲安公主,宜有加恩。朕思賞勞睦親之意,竟行前命。」穆宗怒,乃令弘景使安南、邕、容宣慰,時認翕然推重。



 時蕭俛以清直在位,弘景議論,常
 所輔助。遷刑部侍郎,轉吏部侍郎,銓綜平允,權邪憚其嚴勁,不敢干以非道。掌選二歲,改陜虢觀察使。歲滿,徵拜尚書左丞,駁吏部授官不當者六十人。弘景素以鯁亮稱,及居綱轄之地,郎吏望風修整。會吏部員外郎楊虞卿以公事為下吏所訕,獄未能辨,詔下弘景與憲司就尚書省詳讞。虞卿多朋游,人多向附之。弘景素所不悅,時已請告在第,及準詔就召,以公服來謁。弘景謂之曰:「有敕推公。」虞卿失容自退。轉禮部尚書,充東都留守,
 判東都尚書省事。繕完宮室,至今賴之。



 太和五年五月卒,年六十六,贈尚書左僕射。弘景歷官行事,始終以直道自立,議論操持,無所阿附,當時風教,尤為倚賴。自長慶已來,目為名卿。



 王彥威,太原人。世儒家,少孤貧苦學,尤通《三禮》。無由自達,元和中游京師,求為太常散吏。卿知其書生,補充檢討官。彥威於禮閣掇拾自隋已來朝廷沿革、吉兇五禮,以類區分,成三十卷獻之,號曰《元和新禮》,由是知名,特
 授太常博士。



 憲宗晏駕,未定謚。淮南節度使李夷簡以憲宗功高列聖,宜特稱祖,穆宗下禮官議。彥威奏曰:「據禮經,三代之制,始封之君,謂之太祖。太祖之外,又祖有功而宗有德,故夏后氏祖顓頊而宗禹,殷人祖契而宗湯,周人郊祀后稷,祖文王而宗武王。自東漢魏晉,漸違經意,沿革不一。子孫以推美為先,自始祖已下並有建祖之制。蓋非典訓,不可法也。國朝祖宗制度,本於《周禮》,以景皇帝為太祖,又祖神堯而宗太宗。自高宗已降,但
 稱宗。謂之尊名,可為成法。不然,則太宗造有區夏,理致升平;玄宗掃清內難,翊戴聖父;肅宗龍飛靈武,收復兩都,此者應天順人,撥亂返正,至於廟號,亦但稱宗。謹按經義,祖者始也,宗者尊也,故《傳》曰:『始封必為祖。』《書》曰:『德高可宗,故號高宗。』今宜本三代之定制,去魏、晉之亂法,守貞觀、開元之憲章,而擬議大名,垂以為訓。大行廟號,宜稱宗。」制從之。



 故事,祔廟之禮,先告於太極殿,然後奉神主赴太廟。祔禮畢,不再告於太極殿。時憲宗祔廟禮
 畢,執政詳舊典,令有司再告祔享禮畢於太極殿。彥威執議以為不可,執政怒。會宗正寺進祝版,誤以憲宗為睿宗。執政銜其強,奏祝版參差,博士之罪,彥威坐削一階,奪兩季俸。彥威殊不低回,每議禮事,守正不阿附,君子稱之。累轉司封員外郎中。弘文館舊不置學士,文宗特置一員以待彥威。尋使魏博宣慰,特賜金紫。五年,遷諫議大夫。朝廷自誅李師道,收復淄青十二州,未定戶籍。乃命彥威充十二州勘定兩稅使。朝法振舉,人不以
 為煩。以本官兼史館修撰。



 彥威通悉典故,宿儒碩學皆讓之。時以僕射上事儀注,前後不定,中丞李漢奏定,朝議未以為允。中書門下奏請依元和七年已前儀注,左右僕射上日,請受諸司四品六品丞郎已下拜。彥威奏論曰:「臣謹按《開元禮》:凡受冊官,並與卑官答拜。國朝官品,令三師三公正一品,尚書令正二品,並是冊拜授官。上之日,亦無受朝官再拜之文。僕射班次三公,又是尚書令副貳之職,雖端揆之重,有異百寮,然與群官比肩
 事主。《禮》曰『非其臣即答拜之』。又曰『大夫之臣不稽首』。非尊家臣,以避君也。即僕射上日受常參官拜,事頗非儀。況元和七年已經奏議,酌為定制,編在國章。近年上儀,又有受拜之禮,禮文乍變,物論未安,請依元和七年敕為定。」時李程為左僕射,宰執難於改革,雖不從其議,論者稱之。



 興平縣人上官興,因醉殺人亡竄,吏執其父下獄,興自首請罪,以出其父。京兆尹杜悰、御史中丞宇文鼎,以其首罪免父,有光孝義,請減死配流。彥威與諫官
 上言曰:「殺人者死,百王共守。若許殺人不死,是教殺人。興雖免父,不合減死。」詔竟許決流。彥威詣中書投宰相面論,語訐氣盛。執政怒,左授河南少尹。未幾,改司農卿。李宗閔重之。既秉政,授青州刺史、兼御史大夫,充平盧軍節度、淄青等觀察使。開成元年,召拜戶部侍郎,尋判度支。



 彥威儒學雖優,亦勤吏事,然貨泉之柄,素非所長,性既剛訐,自恃有餘。嘗紫宸廷奏曰:「臣自計司按見管錢穀文簿,皆量入以為出,使經費必足,無所刻削。且百
 口之家,猶有歲蓄,而軍用錢物,一切通用,悉隨色額占定,終歲支給,無毫厘之差。倘臣一旦愚迷,欲自欺竊,亦不可得也。」名曰《度支占額圖》。既而又進《供軍圖》曰:「起至德、乾元之際,迄於永貞、元和之初,天下有觀察者十,節度二十有九,防御者四,經略者三。掎角之師,犬牙相制,大都通邑,無不有兵,都計中外各額,至八十餘萬。長慶戶口凡三百三十五萬,而兵額約九十九萬,通計三戶資一兵。今計天下租賦,一歲所入,總不過三千五百餘
 萬,而上供之數三之一焉。三萬之中,二給衣賜。自留州留使兵士衣賜之外,其餘四十萬眾,仰給度支。伏以時逢理安,運屬神聖,然而兵不可弭,食哉惟時。憂勤之端,兵食是切。臣謬司邦計,虔奉睿圖,輒纂事功,庶裨聖覽。」又纂集國初已來至貞元帝代功臣,如《左氏傳》體敘事,號曰《唐典》,進之。



 彥威既掌利權,心希大用。時內官仇士良、魚弘志禁中用事。先是左右神策軍多以所賜衣物於度支中估,判使多曲從,厚給其價。開成初,有詔禁止,
 然趨利者猶希意從其請托。至是,彥威大結私恩,凡內官請托,無不如意,物議鄙其躁妄。復修王播舊事,貢奉羨餘,殆無虛日。會邊軍上訴衣賜不時,兼之朽故。宰臣惡其所為,令攝度支人吏付臺推訊。彥威略不介懷,入司視事。及人吏受罰,左授衛尉卿,停務,方還私第。



 三年七月,檢校禮部尚書,代殷侑為許州刺史,充忠武軍節度、陳許溵觀察等使。會昌中,入為兵部侍郎,歷方鎮,檢校兵部尚書。卒,贈僕射,謚曰靖。



 史臣曰:世以治軍戎,決權變,非儒者之事。而王翃、郗士美釋衣逢掖之儒衣,奮將軍之旗鼓,俾士赴湯火,威振籓籬,何其壯也!所謂非秦無人,吾謀適不用也。二子遭遇英主,伸其效用,宜哉!李建侯不屈於賊庭,馬會元見伸於貝錦,臨危挺操,所謂貞臣,將相之榮,固其宜矣。辛潞州之特達,韋僕射之峻整,王尚書之果敢,皆一時之偉器也。若以道自牧,求福不回,即能臣也。而彥威欲為巧宦,不亦疏乎?



 贊曰:見危致命,臨難不恐。士美、建侯,仁者之勇。弘景陸離,駁正黃扉。貪名喪道,狂哉彥威。



\end{pinyinscope}