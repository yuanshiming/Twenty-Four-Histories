\article{卷一百八十三}

\begin{pinyinscope}

 ○蕭遘孔緯韋昭度崔昭緯張濬硃樸鄭綮劉崇望兄崇龜弟崇魯崇謨徐彥若陸扆柳璨



 蕭遘,蘭陵人。開元朝宰相太師徐國公嵩之四代孫。嵩
 生衡。衡生復,德宗朝宰相。復生湛。湛生寘,咸通中宰相。寘生遘,以咸通五年登進士第,釋褐秘書省校書郎、太原從事。入朝為右拾遺,再遷起居舍人。與韋保衡同年登進士第,保衡以幸進無藝,同年門生皆薄之。



 遘形神秀偉,志操不群。自比李德裕,同年皆戲呼「太尉」,保衡心銜之。及保衡作相,掎遘之失,貶為播州司馬。途經三峽,維舟月夜賦詩自悼。慮保衡見害,遽有神人謂之曰:「相公勿憂,予當禦侮奉衛。」遘心異之。過峽州,經白帝祠,即
 所睹之神人也。



 保衡誅,以禮部員外郎徵還,轉考功員外郎、知制誥。乾符初,召充翰林學士,正拜中書舍人,累遷戶部侍郎、翰林承旨。



 黃巢犯闕,僖宗出幸,以供饋不給,須近臣掌計,改兵部侍郎、判度支。中和元年三月,自褒中幸成都,次綿州。以本官同平章事,加中書侍郎,累兼吏部尚書、監修國史。



 遘少負大節,以經濟為己任。洎處臺司,風望尤峻,奏對朗拔,天子器之。光啟初,王綱不振。是時天下諸侯,半出群盜;強弱相噬,怙眾邀寵,國法
 莫能制。



 有李凝古者,從支詳為徐州從事。詳為衙將時溥所逐,而賓佐陷於徐。及溥為節度使,因食中毒。而惡凝古者譖之,云為支詳報讎行鴆。溥收凝古殺之。凝古父損,時為右常侍,溥上章披訴,言損與凝古同謀。內官田令孜受溥厚賂,曲奏請收損下獄。中丞盧渥附令孜,鍛煉其獄。侍御史王華嫉惡,堅執奏證損無罪。令孜怒,奏移損付神策獄按問,王華拒不奉詔,奏曰:「李損位居近侍,當死即死,安可取辱於黃門之手?」遘非時進狀,請
 開延英,奏曰:「李凝古行鴆之謀,其事暖昧,已遭屠害,今不復論。李損父子相別三四年,音問斷絕,安得誣罔同謀?時溥恃勛壞法,凌蔑朝廷,而抗表請按侍臣,悖戾何甚?厚誣良善,人皆痛心。若李損羅織而誅,行當便及臣等。」帝為之改容,損得免,止於停任。



 時田令孜專總禁軍,公卿僚庶,無不候其顏色,唯遘以道自處,未嘗屈降。是年冬,令孜奏安邑兩池鹽利,請直屬禁軍。王重榮上章論列。乃奏移重榮別鎮。重榮不受,令孜請率禁軍討之。
 重榮求援於太原,李克用引軍赴之,拒戰沙苑,禁軍大敗,逼京城。僖宗懼,出幸鳳翔。諸籓上章抗論令孜生事,離間方面。遘素惡令孜,乃與裴澈致書召硃玫。玫以邠州之軍五千迎駕,仍與河中、太原修睦,請同匡王室。由是,諸鎮繼上章,請駕還京。令孜聞玫軍至,迫脅天子幸陳倉。時僖宗倉卒出城,夜中百官不及扈從。玫怒令孜弄權,又以天子不諒其忠,語辭怨望,乃訴於遘曰:「主上六年奔播,百端艱險。中原士庶,與賊血戰,肝腦塗地,十
 室九空。比至收復京都,十亡七八。殘民遺老,方喜車駕歸宮。主上不念生靈轉輸之勞,甲士血戰之效,將勤王之功業,為敕使之寵榮;而更志在亂邦,與國生事,召戎結怨,不自他人。昨奉指蹤,徑來奔問,不蒙見信,翻類脅君。古者忠而獲罪,正如此也!吾等報國之心極矣!戰賊之力殫矣!安能垂頭疊翼,喘喘於閽寺之手哉!《春秋》之義,喪君有君。相國徐思其宜,改圖可也。」遘曰:「主上臨御十餘年,未聞過行。比來喪亂播越,失於授任非才。近年
 令孜掣肘,動不如意,上每言之,流涕不已。昨去陳倉,上無行意,令孜陳兵帳下,列卒階前,造次迫行,不容俟旦。靜言此賊,罪不容誅。至尊之心,孰不深鑒?足下乃心王室,止有歸兵還鎮,拜表迎鑾,德業功名,益光圖史。舍此已往;理或未安。改圖之言,未敢聞命。」玫曰:「李家王子極多,有天下者,豈一王哉?」遘曰:「廢立危事,雖有伊尹、霍光之賢,尚貽後悔。古人云:『勿為福始,勿為禍先。』如公矢謀,未見其利。」玫退而宣言曰:「我冊個王子為主,不從者斬。」
 及立襄王,請遘為冊文。遘曰:「少嬰衰疾,文思減落。比來禁署,未免倩人,請命能者。」竟不措筆。乃命鄭昌圖為之,玫滋不悅。及還長安,以昌圖代遘為相,署遘太子太保。乃移疾,滿百日,退居河中之永樂縣。



 遘在相位五年,累兼尚書右僕射,進封楚國公。僖宗再遷京,宰相孔緯與遘不協,以其受偽命,奏貶官。尋賜死於永樂。咸通中,王鐸掌貢籍,遘與韋保衡俱以進士中選,而保衡暴貴,與鐸同在中書。及僖宗在蜀,遘又與鐸並居相位。帝嘗召
 宰臣,鐸年高,升階足跌,踣勾陳中,遘旁掖起,帝目之,喜曰:「輔弼之臣和,予之幸也。」謂遘曰:「適見卿扶王鐸,予喜卿善事長矣。」遘對曰:「臣扶王鐸不獨司長。臣應舉歲,鐸為主司,以臣中選門生也。」上笑曰:「王鐸選進士,朕選宰相,於卿無負矣。」遘謝之而退。



 遘為大臣,士行無缺。逢時不幸,為偽襜所污,不以令終,人士惜之。



 弟蘧,時為永樂令。



 孔緯,字化文,魯曲阜人,宣尼之裔。曾祖岑父,位終秘書
 省著作佐郎,諫議大夫巢父兄也。祖戣,位終禮部尚書,自有傳。父遵孺,終華陰縣丞。



 緯少孤,依諸父溫裕、溫業,皆居方鎮,與名公交,故緯聲籍早達。大中十三年,進士擢第,釋褐秘書省校書郎。崔慎由鎮梓州,闢為從事。又從崔鉉為揚州支使,得協律郎。崔慎由鎮華州、河中,緯皆從之,歷觀察判官。宰相楊收奏授長安尉,直弘文館。御史中丞王鐸奏為監察御史,轉禮部員外郎。宰相徐商奏兼集賢直學士,改考功員外郎。丁內憂免。服闋,以
 右司員外郎入朝。宰臣趙隱嘉其能文,薦為翰林學士,轉考功郎中、知制誥,賜緋。正拜中書舍人,累遷戶部侍郎。謝日,面賜金紫之服。乾符中,罷學士,出為御史中丞。



 緯器志方雅,嫉惡如仇。既總憲綱,中外不繩而自肅。歷戶部、兵部、吏部三侍郎。居選曹,動循格令。權要有所托,私書盈幾,不之省。執政怒之,改太常卿。



 黃巢之亂,從僖宗幸蜀,改刑部尚書,判戶部事。宰臣蕭遘在翰林時,與緯情旨不協。至是,因戶部取給不充,移之散秩,改太子
 少保。光啟元年,從駕還京。



 是時,田令孜軍敗,沙陁逼京師,帝移幸鳳翔,邠帥硃玫引兵來迎駕。令孜挾帝幸山南。時中夜出幸,百官不及扈從,而隨駕者黃門衛士數百人而已。帝駐寶雞,侯百官,詔授緯御史大夫,遣中使傳詔,令緯率百僚赴行在。時京師急變,從駕官屬至盩厔,並為亂兵所剽,資裝殆盡。緯承命見宰相論事,蕭遘、裴澈以田令孜在帝左右,意不欲行,辭疾不見緯。緯遣臺吏促百官上路,皆以袍笏不具為詞。緯無如之何,乃
 召三院御史謂之曰:「吾輩世荷國恩,身居憲秩。雖六飛奔迫而咫尺天顏,累詔追徵,皆無承稟,非臣子之義也。凡布衣交舊,緩急猶相救恤,況在君親?策名委質,安可背也!」言竟泣下。三院曰:「夫豈不懷,但盩厔剽剝之餘,乞食不給。今若首途,聊營一日之費,俟信宿紀行可也。」緯拂衣起曰:「吾妻危疾,旦不保夕,丈夫豈以妻子之故,怠君父之急乎?公輩善自為謀,吾行決矣。」



 即日見李昌符告曰:「主上再有詔命,令促百僚前進。觀群公立意,未有
 發期。僕忝憲闈,不宜居後。道途多梗,明公幸假五十騎,送至陳倉。」昌符嘉之,謂緯曰:「路無頓遞,裹糧辦耶?」乃送錢五十緡,令騎士援緯達散關。緯知硃玫必蓄異志,奏曰:「關城小邑,不足以駐六師,請速幸梁州。」翌日,車駕離陳倉,才入關而邠、岐之兵圍寶雞,攻散關。微緯之言幾危矣!



 至褒中,改兵部侍郎、同中書門下平章事,尋改中書侍郎、集賢殿大學士。王行瑜斬硃玫,平定京城,遷門下侍郎、監修國史。從駕還京,駐蹕岐陽,進階特進,兼吏
 部尚書,領諸道鹽鐵轉運使。車駕還宮,進位左僕射,賜「持危啟運保乂功臣」,食邑四千戶,食實封二百戶,賜鐵券,恕十死罪,賜天興縣莊、善和里宅各一區,兼領京畿營田使。



 僖宗晏駕,充山陵使。僖宗祔廟,緯準故事,不入朝。昭宗遣中使召赴延英,令緯依舊視事,進加司空。以國學盜火所焚,令緯完葺,仍兼領國子祭酒。蔡賊秦宗權伏誅,進階開府儀同三司,進位司徒,封魯國公。



 十一月,昭宗謁郊廟,兩中尉、內樞密請朝服。所司申前例,中
 貴人無朝服助祭之禮,少府監亦無素制冠服。中尉怒,立令制造,下太常禮院。禮官舉故事,亦稱無中尉朝服助祭之文,諫官亦論之。緯奏曰:「中貴不衣朝服助祭,國典也。陛下欲以權道寵內臣,則請依所兼之官而為之服。」天子召諫官謂之曰:「大禮日近,無宜立異,為朕容之。」於是內官以朝服助祭。郊禮畢,進位兼太保。



 大順元年夏,幽州、汴州請討太原。宰臣張浚請自率禁軍為招討。上持疑未決,問計於緯。緯以討之為便,語在《浚傳》。其
 年秋,浚軍為太原所擊,大敗而還。浚罷相貶官,緯坐附浚,以檢校太保、江陵尹、荊南節度觀察等使,未離闕下,再貶均州刺史。緯、浚密遣人求援於汴州,硃全忠上章論救。緯至商州,有詔俾令就便,遂寓居華州。



 乾守二年五月,三鎮入京師,殺宰相韋昭度、李谿。帝以大臣朋黨,外交方鎮,思用骨鯁正人,遣中使趨華州召緯入朝,以疾未任上路。六月,授太子賓客。其日之夕,改吏部尚書。翌日,拜司空,兼門下侍郎、同平章事、太清宮使,修奉太
 廟、弘文館大學士、延資庫使。階爵、功臣名、食邑並如故。旬日之內,驛騎敦促,相望於路,扶疾至京師。



 延英中謝,奏曰:「臣前時待罪宰相,智術短淺,有負弼諧。陛下特貸刑書,曲全腰領。臣期於死報泉壤,不望生叩玉階。復拜龍顏,實臣榮幸。然臣比嬰衰疾,伏枕累年,形骸雖存,生意都盡。平居勉強,御事猶疏。況比尪羸,寧勝重委?國祚方泰,英彥盈庭,豈以朽腐之人,再塵機務!臣力疾一拜殿庭,乞陛下許臣自便。」因鳴咽流涕。緯久疾,拜蹈艱難,
 上令中使止之,改容軫念。令閣門使送緯中書視事。不旬日,沙陁次河中,同州王行約入京師謀亂,天子出幸石門。緯從駕至莎城,疾漸危篤,先還京城。九月,卒於光德里第,贈太尉。



 緯家尚節義,挺然不屈。雖權勢燻灼,未嘗假以恩禮。大順初,天武都頭李順節恃恩頗橫,不期年領浙西節度使,俄加平章事。謝日,臺吏申中書,稱天武相公衙謝,準例班見百僚。緯判曰:「不用立班。」順節粗暴小人,不閑朝法,盛飾趨中書,既見無班,心甚怏怏。他
 日因會,順節微言之。緯曰:「必知公慊也。夫百闢卿士,天子庭臣也,比來班見宰相,以輔臣居班列之首,奉長之義也。公握天武健兒,而於政事受百僚班見,意自安乎?必若須此儀,俟去『都頭』二字可也。」順節不敢復言。其秉禮不回,多此類也。



 孔氏自元和後,昆仲貴盛,至正卿、方鎮者六七人,未有為宰輔者,至緯始在鼎司。



 子崇弼,亦登進士第,仕至散騎常侍。



 韋昭度,字正紀,京兆人。祖縃,父逢。昭度,咸通八年進士
 擢第。乾符中,累遷尚書郎、知制誥,正拜中書舍人。從僖宗幸蜀,拜戶部侍郎。中和元年,權知禮部貢舉。明年,以本官同平章事,兼吏部尚書。



 昭宗即位,閬州刺史王建攻陳敬瑄於成都,隔絕貢奉。乃以昭度檢校司空、同平章事、成都尹、劍南西川節度招撫宣慰等使。昭度赴鎮,敬瑄不受代。詔東川顧彥朗與王建合勢討之。昭度為行營招討。卒歲,止拔漢州。王建謂昭度曰:「相公勞師弊眾,遠事蠻夷。訪聞京洛以東,群侯相噬,禍難未已。朝廷
 不治,腹心之疾也。相公宜亟還京師,咨謀匡合,平定兩河,國家之利也。敬瑄小醜,以日月制之,擒之必矣!此事責建可辦。」昭度然之,奏請還都。昭度未及京師,建以重兵守劍門,急攻成都下之。殺敬瑄,自稱留後。昭度還,以檢校司空充東都留守。召還,為右僕射。



 景福二年冬,宰相杜讓能為鳳翔所殺,復委昭度知政事,與李谿並命。時宰相崔昭緯專政,惡李谿之為人。降制日,令知制誥劉崇魯哭麻以沮之。谿上表論列,天子待谿益厚。明年
 春,復命谿同平章事,昭緯不勝其忿。



 先是,邠州王行瑜求為尚書令,昭度奏議云:「國朝已來,功如郭子儀,未省曾兼此官。」乃賜號「尚父」。崔昭緯宗人鋌,曾為行瑜從事,朝廷每降制敕,不便於昭緯者,即令鋌訴於行瑜,俾上章論列。朝旨小有依違,即表章不遜。至是李谿入拜。昭緯謂鋌曰:「前時尚父之命已行,而昭度沮之,今又引谿同列。此人奸纖,惑上視聽,宗社不寧。恐復有杜太尉之事。」行瑜與李茂貞上章言:「命相非其人,懼危宗社。」天子
 優詔曉諭,言谿有才。其年五月,行瑜、茂貞、華州韓建以兵入覲,面奏昭度、李谿之奸邪,請加譴逐。制敕未行,三鎮兵害昭度於都亭驛。及行瑜誅,降制復其官爵,令其家收葬。



 崔昭緯,清河人也。祖庇,滑州酸棗縣尉。父巘,鄂州觀察使。昭緯進士及第。昭宗朝,歷中書舍人、翰林學士、戶部侍郎、同平章事。性奸纖,忌前達。內結中人,外連籓閫。屬朝廷微弱,每托援以凌人主。昭宗明察,心不能堪。以誘
 召三鎮將兵詣闕,賊殺宰輔內臣,帝深切齒。會太原之師誅行瑜,罷相,授右僕射。後又以托附汴州,再貶梧州司馬。尋降制曰:



 崔昭緯頃居內署,粗著微勞。擢於侍從之司,委以燮調之任。不能忠貞報國,端慎處身。潛交結於奸臣,致漏洩於機事。星霜累換,匡輔蔑聞。爾罪一也。



 又快其私忿,輒恣陰謀。托崔鋌之險巇,連行瑜之計畫,遂致稱兵向闕,怙眾脅君。故宰臣韋昭度、李谿並以無辜見害,幾危宗社,顯辱君親。爾罪二也。



 及行瑜敗滅,京
 國甫安,而乃自懼欺誣,別謀托附。又於籓閫,潛請薦論,不唯茍免罪愆,兼亦再希任用。貪榮冒寵,僭濫無厭,敗俗傷風,賢愚共鄙。爾罪三也。



 又將厚賂,欲結諸王,輕侮我憲章,玷瀆我骨肉。貨財之數,文字具存。賴諸王作朕腹心,嫉其蠹害,盡將昭緯情款,兼其親吏姓名,直具奏聞,拒其求托。昭緯曾居宰輔,久歷清崇,但欲逞其回邪,都不顧其事體。觀其識見,實駭聽聞。爾罪四也。



 自奸邪既露,情狀難容。尚示寬刑,未行嚴憲,投於荒裔,冀其自
 新。而不能退省過尤,恭承制命,速赴貶所,用守常規。而猶自務宴安,尋聞所在留駐;攪擾籓鎮,侮慢朝章。曾無稟畏之心,可驗苞藏之計。罔知愆咎,唯謗朝廷。爾罪五也。



 朕以恩澤者,帝王之雨露,弄法者,邦國之雷霆;無雨露則庶物不榮,無雷霆則萬邦不肅。朕體天道以化育,遵王度以澄清,罪既昭彰,理難含垢。凡百多士,宜體予懷。宜所在賜自盡。



 時昭緯行次至荊南,中使至,斬之。



 兄昭符,仕至禮部尚書。昭願,太子少保。昭矩,給事中。昭遠,
 考功員外郎。



 張濬,字禹川,河間人。祖仲素,位至中書舍人。父鐐,官卑,家寓州。濬倜儻不羈,涉獵文史,好大言,為士友之所擯棄。初從鄉賦隨計,咸薄其為人。濬憤憤不得志,乃田衣野服,隱於金鳳山,學鬼谷縱橫之術,欲以捭闔取貴仕。乾符中,樞密使楊復恭因使遇之,自處士薦為太常博士,累轉度支員外郎。



 黃巢將逼關輔,濬托疾請告,侍其母,挈族避亂商州。賊犯京師,僖宗出幸,途無供頓,衛軍
 不得食。漢陰令李康獻糗餌數百騾綱,軍士始得食。僖宗召康問曰:「卿為縣令,安操心及此?」康對曰:「臣為塵吏,敢有此進獻?張濬員外教臣也。」帝異之,急召至行在,拜兵部郎中。未幾,拜諫議大夫。



 其年冬,宰相王鐸至滑臺,兼充天下行營都統。方徵兵諸侯,奏用濬為都統判官。時王敬武初破弘霸郎,軍威大振,累詔徵平盧兵,敬武獨不赴援。鐸遣濬往說之,敬武已受偽命,復怙強不迎詔使。濬至,謁見,責之曰:「公為天子守籓,王臣齎詔宣諭,
 而侮慢詔使。既未識君臣禮分,復何顏以御軍民哉?」敬武愕然謝咎。既宣詔,軍士按兵默然,濬並召將佐集於鞠場面諭之曰:「人生效忠仗義,所冀粗分順逆,懸知利害。黃巢前日販鹽虜耳,公等舍累葉天子而臣販鹽白丁,何利害之可論耶?今諸侯勤王,天下響應,公等獨據一州,坐觀成敗。賊平之後,去就何安?若能此際排難解紛,陳師鞠旅,共誅寇盜,迎奉鑾輿,則富貴功名,指掌可取。吾惜公輩舍安而即危也!」諸將改容引過,謂敬武曰:「
 諫議之言是也。」即時出軍,從濬入援京師。賊平,累遷戶部侍郎。僖宗再幸山南,拜平章事、判度支。



 濬初發跡,依楊復恭。及復恭失勢,乃依田令孜,以至重位,而反薄復恭。及再幸山南,復恭代令孜為中尉,罷濬知政事。昭宗初在籓邸,深嫉宦官,復恭有援立大勛,恃恩任事,上心不平之。當時趨向者,多言濬有方略,能畫大計,復用為宰相、判度支。上嘗問濬,致理何事最急?對曰:「莫若強兵。兵強而天下服。」上由是專務搜補兵甲,欲以武功勝天
 下。後延英論前代為治得失,濬曰:「不必遠征漢、晉之弊。臣竊見陛下春秋鼎盛,英睿如此,內外逼於強臣。臣每思之,實痛心而泣血也。」



 會硃全忠誅秦宗權,安居受殺李克恭,以潞州降全忠。幽州李匡威、雲州赫連鐸等奏請出軍討太原。詔四品以上官議,皆言:「國祚未安,不宜生事。假如得太原,亦非國家所有。」濬議曰:「先帝頻至播越,王室不寧。原其亂階,由克用、全忠之矛盾也。請因其奏,乘全忠立功,可斷兩雄之勢。」上曰:「收復之功,克用第
 一。今乘其危困而加兵,諸侯其謂我何?」濬懇論用兵之利害,蓋欲示外勢而擠復恭也。上旨未決。宰臣孔緯曰:「張濬所陳,萬代之利也。陛下所惜,即日之利也。以臣所料,師渡河而賊必自破。昨計度軍中轉餉犒勞,一二年間,必無闕事,陛下斷意行之。」



 既二相俱論,乃以濬為河東行營兵馬都招討宣慰使,以京兆尹孫揆副之。仍授揆昭義節度使,華州韓建為供軍使,硃全忠為太原西南面招討使,李匡威、赫連鐸為太原東北面招討使。全忠以
 汴軍三千為濬牙隊。大順元年六月,濬率軍五十二都,兼邠寧、鄜、夏雜虜共五萬人騎,發自京師。昭宗御安喜樓臨送,濬酒酣泣奏曰:「陛下動為賊臣掣肘,臣所以誓死憤惋,為陛下除其僭逼。」楊復恭聞之不悅。中尉內使餞於長樂,復恭奉卮酒屬濬,濬辭曰:「聖人賜酒,已醉矣。」復恭戲曰:「相公握禁兵,擁大蒐,獨當一面,不領復恭意作面子耶!」濬笑曰:「賊平之後,方見面子。」復恭銜之。



 時汴、華、邠、岐之師渡河,會濬於晉州。汴將硃崇節權知潞州
 事,太原將李存孝攻之。濬慮賊平汴人據昭義,乃令孫揆分兵赴鎮,中使韓歸範送旌節至軍。八月,揆與歸範赴潞州。至潞,並為存孝擒送太原。九月,汴將葛從周棄潞州。十月,濬軍至陰地,邠、岐、華三鎮之師營平陽。李存孝擊之,一戰而敗,委兵仗潰散。進攻晉州。數日,中夜濬斂眾遁走。比曙,喪師殆半。存孝進收晉、絳、慈、隰等州。濬狼狽由含山逾王屋,出河清,拆屋木縛筏濟河,部下離散將盡。李克用上章論訴曰:



 晉州長寧關使張承暉於
 當道錄到張濬榜並詔曰,張濬充招討制置使,令率師討臣,兼削臣屬籍官爵者。臣誠冤誠憤,頓首,頓首!伏以宰臣張濬欺天蔽日,廊廟不容。讒臣於君,奪臣之位。憑燕帥妄奏,與汴賊結恩;矯托皇威,擅宣王命,徵集師旅,撓亂乾坤。誤陛下中興之謀,資黔黎重傷之困。臣實何罪,而陛下伐之?此則宰臣持權,面欺陛下。



 況臣父子三代,受恩四朝,破徐方,救荊楚,收鳳闕,碎梟巢,致陛下今日冠通天之冠,佩白玉之璽。臣之屬籍,懿皇所賜;臣之
 師律,先帝所命。臣無逆節,濬討何名?陛下若厭逐功臣,欲用文吏,自可遷臣封邑,以侯就第。奈何加諸其罪,孰肯無詞?若以臣雲中之伐,獲罪於時,則拓拔思恭取鄜、延,硃全忠侵徐、鄆,陛下何不討之?假令李孝德不忠於主,伐之為是,則硃瑄、時溥有何罪耶?此乃同坐而異名,賞彼而誅此,使天下籓服,強者扼腕,弱者自動,流言竊議,為臣怨嗟,固非中興之術也。



 且陛下阽危之秋,則獎臣為韓、彭、伊、霍;既安之後,罵臣曰戎、羯、蕃、夷。海內握兵
 立事如臣者眾矣,寧不懼陛下他時之罵哉?臣昨遇燕軍,以禮退舍。匡威淺昧,厚自矜誇,乃言臣中矢石,覆士卒。致內外吠聲一發,短謀競陳,誤陛下君臣之分。況命官選將,自有典刑,不必幸臣之弱而後取之。倘臣延期挺命,尚固一方,彼實何顏以見陛下。此則奸邪朋黨,輕弄邦典,陛下凝旒端扆,何由知之?今張濬既以出軍,微臣固難束手。臣便欲叫閽,輕騎面叩玉階,訴邪佞於陛下之彤墀,納詔命於先皇之宗廟,然後束身司敗,甘處
 憲章。



 時克用令所擒中使奉表,表至而濬敗,朝廷聳震,制曰:



 漢武因恭儉富庶之後,建置朔方,孫弘沮之,十不得一。而良史以弘有宰相體者,誠以愛人治國為先,拓境開疆為末。及孝宣值雄才削平之餘,將議北征,魏相爭之,五將尋罷。果致中興,號為賢輔。況朕承天厭兵戈之後,人思休息之時。敢望皋、夔,共成堯日;庶幾孫、魏,粗及漢年。茍易於斯,如何倚注!



 光祿大夫、門下侍郎、兼戶部尚書、同中書門下平章事、上柱國、清河郡開國伯、食邑一
 千二百戶、充河東行營諸道兵馬招討制置等使張濬,早以盛名,稱為奇士,由是再加徵用,委以鈞衡,謂其必致小康,克勝大任。而乃罔思守道,但欲邀功,用不詭之詢謀,起無名之兵革。自云一舉,止在旬時,堅請抗論,勢莫能奪。輕葛亮渭濱之役,小裴度淮右之行。經功寒暄,耗費百萬。虛誕彰於朝野,詐詭布於華夷,橫草蔑聞,燎原愈急。俾擁旄乘驛之使,囚在虜庭;勤王奉國之軍,懷歸本土。忘廊廟之威重,結籓屏之仇讎。欲使海內
 生靈,竭其貢賦;不獨河中郡邑,蕩為丘墟。潛生厲階,欲誰歸咎?



 於戲!征晁錯之故事,思王恢之舊章,國有明文,爾當何逭?尚以愛人以禮,理體宜然。廉鎮劇權,武昌善地,宜罷樞軸之務,仍停支度之司。勉自思惟,以逃後命。可檢校戶部尚書、鄂州刺史、武昌軍節度觀察等使。



 尋貶連州刺史,馳驛發遣。行至藍田關不行,留華州依韓建。時朝廷微弱,竟不能詰。



 乾寧二年,三鎮殺韋昭度。帝召孔緯欲大用,亦以濬為兵部尚書,又領天下租庸使。
 三年,天子幸華州,罷濬使務,守尚書右僕射。上疏乞致仕,授左僕射致仕。乃還洛陽,居於長水縣別墅。濬雖退居山墅,朝廷或有得失,必章疏上言。德王廢立之際,濬致書諸籓,請圖匡復。王師範青州起兵,欲取濬為謀主。事雖不果,其跡頗洩。硃全忠將圖篡代,懼濬構亂四方,不欲顯誅,密諷張全義令圖之。乃令牙將楊麟率健卒五十人,有如劫盜,圍其墅而殺之,天復三年十二月晦夜也。



 永寧縣吏葉彥者,張氏待之素厚。楊麟之來,彥知
 之,告濬第二子格曰:「相公之禍不可免,郎君宜自為謀。」格、濬父子號咷而已。濬謂格曰:「留則並命,去或可免。汝自圖之,勿以吾為累,冀存後祀也。」格拜辭而去。葉彥率義士三十人,送渡漢江而旋。格由荊江上峽入蜀。王建僭號,用為宰相。中興平蜀,任圜攜格而還。格感葉彥之惠,訪之,身已歿,而厚報其家。濬第三子竄於楊行密。



 自乾寧之後,賊臣內侮,王室浸微。昭宗不堪凌弱,欲簡拔奇材以為相。然採於群小之論,未嘗獲一名人。登用之
 徒,無不為時嗤誚。



 硃樸者,乾寧中為國子博士。腐儒木強,無他才伎。道士許巖士出入禁中,嘗依樸為奸利,從容上前薦僕有經濟才。昭宗召見,對以經義,甚悅,即日拜諫議大夫、平章事。在中書與名公齒,筆札議論,動為笑端。數月,巖士事敗,俱為韓建所殺。



 鄭綮者,以進士登第,歷監察、殿中,倉、戶二員外,金、刑、右司三郎中。家貧求郡,出為廬州刺史。黃巢自嶺表還,經
 淮南剽掠。綮移黃巢文牒,請不犯郡界。巢笑而從之,一郡獨不被寇。天子嘉之,賜緋魚袋。罷郡,有錢千緡,寄州帑。後郡數陷,盜不犯鄭使君寄庫錢。至楊行密為刺史,送所寄於京師還綮。



 綮善為詩,多侮劇刺時,故落格調,時號鄭五歇後體。初去廬江,與郡人別云:「唯有兩行公廨淚,一時灑向渡頭風。」滑稽皆此類也。



 王徽為御史大夫,奏綮為兵部郎中、知臺雜,遷給事中,賜金紫。僖宗自山南還,以宰相杜讓能弟弘徽為中書舍人。綮以弘徽
 兄在中書,弟不宜同居禁近,封還制書。天子不報,綮即移病休官。無幾,以左散騎常侍徵還。朝政有闕,無不上章論列。事雖不行,喧傳都下,執政惡之,改國子祭酒。物議以綮匡諫而置之散地,不可,執政懼,復用為常侍。



 光化初,昭宗還宮,庶政未愜。綮每形於詩什而嘲之,中人或誦其語於上前。昭宗見其激訐,謂有蘊蓄,就常奏班簿側注云:「鄭綮可禮部侍郎、平章事。」中書胥吏詣其家參謁,綮笑而問之曰:「諸君大誤,俾天下人並不識字,宰
 相不及鄭五也。」胥吏曰:「出自聖旨特恩,來日制下。」抗其手曰:「萬一如此,笑殺他人。」明日果制下,親賓來賀,搔首言曰:「歇後鄭五作宰相,時事可知矣。」累表遜讓,不獲。既入視事,侃然守道,無復詼諧。終以物望非宜,自求引退。三月餘,移疾乞骸,以太子少保致仕。光化二年卒。



 時議以昭宗命臺臣濬、樸、綮三人尤謬,季末之妖也。



 劉崇望,字希徒。其先代郡人,隨元魏孝文帝徙洛陽,遂為河南人。八代祖隋大理卿坦,生政會,輔太宗起義晉
 陽,官至戶部尚書,封渝國公,圖形凌煙閣。政會生玄意,尚太宗女南平公主,歷洪、饒八州採訪使。玄意生奇,位至吏部侍郎。奇生慎知,仕至獲嘉令。慎知生褧,仕至東阿令。褧生藻,位終秘書郎。藻生符,進士登第,咸通中位終蔡州刺史,生八子:崇龜、崇望、崇魯、崇謨最知名。



 崇龜,咸通六年進士擢第,累遷起居舍人,禮部、兵部二員外。丁母憂免。廣明元年春,鄭從讜罷相,鎮太原,奏崇龜為度支判官、檢校吏部郎中、御史中丞,賜金紫。中和三年
 入朝,為兵部郎中,拜給事中。大順中,遷左散騎常侍、集賢殿學士、判院事,改戶部侍郎,檢校戶部尚書。出為廣州刺史、清海軍節度、嶺南東道觀察處置等使,卒。



 崇望,咸通十五年登進士科。王凝廉問宣歙,闢為轉運巡官。戶部侍郎裴坦領鹽鐵,闢為參佐。崔安潛鎮許昌、成都,崇望昆仲四人,皆在安潛幕下。入為長安尉,直弘文館,遷監察御史、右補闕、起居郎、弘文館學士,轉司勛、吏部二員外郎。崔安潛為吏部尚書,崇望判南曹,滌除宿弊,
 復清選部。田令孜干政,籓鎮怨望,河中尤甚,不修職貢。僖宗在山南,以蒲阪近關,欲其效用,選使諭旨,以崇望為諫議大夫。既至,諭以大義,重榮奉詔恭順,誓心匡復,請殺硃玫自贖。使還,上悅,召入翰林充學士,累遷戶部侍郎、承旨,轉兵部,在禁署四年。



 昭宗即位,拜中書侍郎、同平章事,累兼兵部、吏部尚書。大順初,同列張濬畫策討太原,崇望以為不可,濬果敗。濬黜,崇望代為門下侍郎、監修國史、判度支。



 明年,玉山都頭楊守信協楊復恭
 稱兵闕下,陣於通化門。上陳兵於延嘉門。是夜,命崇望守度支庫。明日曉,入含光門。未開,門內禁軍列於左右,俟門開即劫掠兩市。及聞傳呼宰相來,門方啟,崇望駐馬慰諭之曰:「聖上在街東親總戎事。公等禁軍,何不樓前殺賊,立取功名。切不可剽掠街市,圖小利以成惡名也。」將士唯唯,從崇望至長樂門。守信見兵來,即遁去,軍士呼萬歲。是日庫市獲全,軍人不亂,繄崇望之方略也。尋加左僕射。



 時溥與硃全忠爭衡,全忠謀兼徐、泗,上表
 請以重臣鎮徐,乃以崇望守本官,充武寧軍節度使。溥不受代,行至華陰而還,拜太常卿。王重盈死,王珂、王珙爭河中節鉞,朝廷以宰相崔胤為河中節度使。珂,李克用之子婿也。河東進奏官薛志勤揚言曰:「崔相雖重德,如作鎮河中代王珂,不如光德劉公,於我公事素也。」及三鎮以兵入朝,殺害大臣,以志勤之言,責授崇望昭州司馬。及王行瑜誅,太原上表言崇望無辜放逐。時已至荊南,有詔召還,拜吏部尚書。未至,王溥再知政事,兼吏
 部尚書,乃改崇望兵部尚書。



 時西川侵寇顧彥暉,欲並東川,以崇望檢校右僕射、平章事、梓州刺史、劍南東川節度使。未至鎮,召還,復為兵部尚書。光化二年卒,時年六十二,冊贈司空。



 崇魯,廣明元年登進士第,鄭從讜奏充太原推官。時兄崇龜為節度判官,昆仲同居幕府,尋轉掌書記。中和二年入朝,拜右拾遺、左補闕。景福初,以水部員外郎知制誥。二年,杜讓能得罪,昭宗復命韋昭度為相,翰林學士李谿同平章事。崇魯與崔昭緯相善。
 昭緯恃邠、岐之援。讓能既誅之後,權歸於己,昭宗師李谿為文,懼居位得寵則恩顧漸衰,乃私與崇魯謀沮之。及谿宣制之日,出班而哭,謂昭緯曰:「朝廷雖乏賢,不可用纖人為宰輔。谿比依復恭、重遂居內職。前日杜太尉狼籍,為朝廷深恥。今則削弱如此,安可更遵覆轍乎?」由是谿命不行。谿自十一月初至歲暮,聯上十表訴冤,其詞詆毀,所不忍聞。明年春,復命谿為平章事。昭緯召李茂貞、王行瑜、韓建稱兵入朝,殺昭度與谿。其年,太原誅
 王行瑜,昭緯貶官,崇魯坐貶崖州司戶。初崇龜在外,聞崇魯哭麻,大恚,數日不食,謂所親曰:「吾家兄弟進身有素,未嘗以聲利敗名。吾門不幸,生此等兒。」



 崇謨,中和三年進士及第。乾寧末,為太常少卿、弘文館直學士。



 徐彥若,天后朝大理卿有功之裔。曾祖宰,祖陶,父商,三世繼登進士科。商,字義聲,大中十三年及第,釋褐秘書省校書郎。累遷侍御史,改禮部員外郎。尋知制誥,轉郎中,召充翰林學士,拜中書舍人、戶部侍郎判本司事,檢
 校戶部尚書、襄州刺史、山南東道節度等使。入為御史大夫。咸通初,加刑部尚書,充諸道鹽鐵轉運使,遷兵部尚書、東莞子、食邑五百戶。四年,以本官同平章事。六年罷相,檢校右僕射、江陵尹、荊南節度觀察等使。入為吏部尚書,累遷太子太保,卒。



 彥若,咸通十二年進士擢第。乾符末,以尚書郎知制誥,正拜中書舍人。昭宗即位,遷御史中丞,轉吏部侍郎,檢校戶部尚書,代李茂貞為鳳翔隴節度使。茂貞不受代,復拜中丞,改兵部侍郎、同平
 章事,進加中書侍郎,累兼左僕射、監修國史。扈昭宗石門還宮,加開府儀同三司、守司空,進封齊國公,太清宮、修奉太廟等使,加弘文館大學士,賜「扶危匡國致理功臣」名。昭宗自華還宮,進位太保、門下侍郎。時崔胤專權,以彥若在己上,欲事權萃於其門。二年九月,以彥若檢校太尉、同平章事、廣州刺史、清海軍節度、嶺南東道節度等使。卒於鎮。



 弟彥樞,位至太常少卿。



 子綰,天祐初歷司勛、兵部二員外,戶部、兵部二郎中。



 陸扆,字祥文,本名允迪,吳郡人。徙家於陜,今為陜州人。曾祖澧,位終殿中侍御史。祖師德,淮南觀察支使。父鄯,陜州法曹參軍。扆,興啟二年登進士第,其年從僖宗幸興元。九月,宰相韋昭度領鹽鐵,奏為巡官。明年,宰相孔緯奏直史館,得校書郎,尋丁母憂免。龍紀元年冬,召授藍田尉,直弘文館,遷左拾遺,兼集賢學士。中丞柳玭奏改監察御史。大順二年三月,召充翰林學士,改屯田員外郎,賜緋。景福元年,加祠部郎中、知制誥,二年元日朝
 賀,面賜金紫之服。五月,拜中書舍人。



 扆文思敏速,初無思慮,揮翰如飛,文理俱愜,同舍服其能。天子顧待特異。嘗金鑾作賦,命學士和,扆先成。帝覽而嗟挹之,曰:「朕聞貞元時有陸贄、吳通玄兄弟,能作內庭文書,後來絕不相繼。今吾得卿,斯文不墜矣。」



 乾寧初,轉戶部侍郎。二年,改兵部,進階銀青光祿大夫、嘉興男、三百戶。三年正月,宣授學士承旨,尋改左丞。其年七月,改戶部侍郎、同平章事。故事,三署除拜,有光署錢以宴舊僚,內署即無斯
 例。扆拜輔相之月,送學士光院錢五百貫,特舉新例,內署榮之。八月,加中書侍郎、集賢殿大學士、判戶部事。



 九月,覃王率師送徐彥若赴鳳翔。師之起也,扆堅請曰:「播越之後,國步初集,不宜與近輔交惡,必為他盜所窺。加以親王統兵,物議騰口,無益於事,只貽後患。」昭宗已發兵,怒扆沮議,是月十九日,責授硤州刺史。師出果敗,車駕出幸。四年二月,復授扆工部尚書。八月,轉兵部尚書,從昭宗自華還宮。



 明年正月,復拜中書侍郎、同平章事。
 光化三年四月,兼戶部尚書,進封吳郡開國公,食邑一千戶。九月,轉門下侍郎、監修國史。天復元年五月,進階特進,兼兵部尚書,加食邑五百戶。車駕自鳳翔還京,赦後諸道皆降詔書,獨鳳翔無詔。扆奏曰:「鳳翔近在國門,責其心跡,罪實難容。然比來職貢無虧,朝廷未與之絕。一朝獨無詔命,示人不廣也。」崔胤怒,奏貶扆沂王傅,分司東都,削階至正議大夫。居無何,崔胤誅,復授吏部尚書,階封如故。從昭宗還洛。其年秋,昭宗遇弒。明年正月,責
 授濮州司戶,與裴樞、崔遠、獨孤損等被害於滑州白馬驛,時年五十九。



 子璪,後為緱氏令。



 柳璨,河東人。曾祖子華。祖公器,僕射公綽之再從弟也。父遵。璨少孤貧好學,僻居林泉。晝則採樵,夜則燃木葉以照書。性謇直,無緣飾。宗人壁、玭,貴仕於朝,鄙璨樸鈍,不以諸宗齒之。光化中,登進士第。尤精《漢史》,魯國顏蕘深重之。蕘為中書舍人,判史館,引為直學士。璨以劉子玄所撰《史通》譏駁經史過當,璨紀子玄之失,別為十卷,
 號《柳氏釋史》,學者伏其優贍。遷左拾遺。公卿朝野,托為箋奏,時譽日洽。以其博奧,目為「柳篋子」。



 昭宗好文,初寵待李谿頗學。洎谿不得其死,心常惜之,求文士似谿者。或薦璨高才,召見,試以詩什,甚喜。無幾,召為翰林學士。崔胤得罪前一日,召璨入內殿草制敕。胤死之日,既夕,璨自內出,前驅傳呼相公來。人未見制敕,莫測所以。翌日對學士,上謂之曰:「朕以柳璨奇特,似可獎任。若令預政事,宜授何官?」承旨張文蔚曰:「陛下拔用賢能,固不拘
 資級。恩命高下,出自聖懷。若循兩省遷轉,拾遺超等入起居郎,臨大位,非宜也。」帝曰:「超至諫議大夫可乎?」文蔚曰:「此命甚愜。」即以諫議大夫平章事,改中書侍郎。任人之速,古無茲例。



 同列裴樞、獨孤損、崔遠皆宿素名德,遽與璨同列,意微輕之,璨深蓄怨。昭宗遷洛,諸司內使、宿衛將佐,皆硃全忠腹心也,璨皆將迎,接之以恩,厚相交結,故當時權任皆歸之。



 二年五月,西北長星竟天,掃太微、文昌、帝座諸宿,全忠方謀篡代。而妖星謫見,占者云:「
 君臣俱災,宜刑殺以應天變。」蔣玄暉、張廷範謀殺衣冠宿望難制者,璨即首疏素所不快者三十餘人,相次誅殺。班行為之一空,冤聲載路。傷害既甚,硃全忠心惡之。會全忠授九錫,蔣玄暉等別陳意見。王殷至大梁,誣玄暉等通導宮掖,欲興復李氏。全忠怒,捕廷範,令河南聚眾,五軍分裂之,兼誅璨,臨刑呼曰:「負國賊柳璨,死其宜矣!」初,璨遷洛後,累兼戶部尚書、守司空,進階光祿大夫、鹽鐵轉運使。



 其弟瑀、瑊坐璨笞死。



 史臣曰:嗚呼!李氏之失馭也,孛沴之氣紛如,仁義之徒殆盡。狐鳴鴟嘯,瓦解土崩。帶河礪岳之門,寂無琨、逖;奮挺揭竿之類,唯效敦、玄。手未舍於棘矜,心已萌於問鼎。加以囂浮士子,闒茸鯫儒。昧管、葛濟時之才,無王、謝扶顛之業,邀功射利,陷族喪邦。濬、緯養虎於前,胤、璨剝廬於後。逐徐、薛於瘴海,置綮、樸於巖廊。殿廷有哭制之夫,輔弼走破輿之黨。九疇既紊,百怪斯呈。木將朽而蠹蠍生,厲既篤而夔魖見。妖徒若此,亡國宜然。何必長星,更
 臨衰運?



 贊曰:蕭召、硃玫,孔符、張濬,身世罹殃,邦家起釁。如木斯蠹,自潰於中。抵巇侮亂,安責伏戎。



\end{pinyinscope}