\article{卷一百八}

\begin{pinyinscope}

 ○高仙芝封常清哥舒翰



 高仙芝,本高麗人也。父舍雞,初從河西軍,累勞至四鎮十將、諸衛將軍。仙芝美姿容,善騎射,勇決驍果。少隨父至安西,以父有功授游擊將軍。年二十餘即拜將軍,與
 父同班秩。事節度使田仁琬、蓋嘉運,未甚任用,後夫蒙靈察累拔擢之。開元末,為安西副都護、四鎮都知兵馬使。



 小勃律國王為吐蕃所招,妻以公主,西北二十餘國皆為吐蕃所制,貢獻不通。後節度使田仁琬、蓋嘉運並靈察累討之,不捷,玄宗特敕仙芝以馬步萬人為行營節度使往討之。時步軍皆有私馬,自安西行十五日至撥換城,又十餘日至握瑟德,又十餘日至疏勒,又二十餘日至蔥嶺守捉,又行二十餘日至播密川,又二十餘
 日至特勒滿川,即五識匿國也。仙芝乃分為三軍:使疏勒守捉使趙崇玭統三千騎趣吐蕃連雲堡,自北穀入;使撥換守捉使賈崇瓘自赤佛堂路入;仙芝與中使邊令誠自護密國入,約七月十三日辰時會於吐蕃連雲堡。堡中有兵千人,又城南十五里因山為柵,有兵八九千人。城下有婆勒川,水漲不可渡。仙芝以三牲祭河,命諸將選兵馬,人齎三日乾糧,早集河次。水既難渡,將士皆以為狂。既至,人不濕旗,馬不濕韉,已濟而成列矣。仙
 芝喜謂令誠曰:「向吾半渡賊來,吾屬敗矣,今既濟成列,是天以此賊賜我也。」遂登山挑擊,從辰至巳,大破之。至夜奔逐,殺五千人,生擒千人,餘並走散。得馬千餘匹,軍資器械不可勝數。



 玄宗使術士韓履冰往視日,懼不欲行,邊令誠亦懼。仙芝留令誠等以羸病尪弱三千餘人守其城,仙芝遂進。三日,至坦駒嶺,直下峭峻四十餘里,仙芝料之曰:「阿弩越胡若速迎,即是好心。」又恐兵士不下,乃先令二十餘騎詐作阿弩越城胡服上嶺來迎。既
 至坦駒嶺,兵士果不肯下,云:「大使將我欲何處去?」言未畢,其先使二十人來迎,云:「阿弩越城胡並好心奉迎,娑夷河藤橋已斫訖。」仙芝陽喜以號令,兵士盡下。娑夷河,即古之弱水也,不勝草芥毛發。下嶺三日,越胡果來迎。明日,至阿弩越城,當日令將軍席元慶、賀婁餘潤先修橋路。仙芝明日進軍,又令元慶以一千騎先謂小勃律王曰:「不取汝城,亦不斫汝橋,但借汝路過,向大勃律去。」城中有首領五六人,皆赤心為吐蕃。仙芝先約元
 慶云:「軍到,首領百姓必走入山谷,招呼取以敕命賜彩物等,首領至,齊縛之以待我。」元慶既至,一如仙芝之所教,縛諸首領。王及公主走入石窟,取不可得。仙芝至,斬其為吐蕃者五六人。急令元慶斫藤橋,去勃律猶六十里,及暮,才斫了,吐蕃兵馬大至,已無及矣。藤橋闊一箭道,修之一年方成。勃律先為吐蕃所詐借路,遂成此橋。至是,仙芝徐自招諭勃律及公主出降,並平其國。



 天寶六載八月,仙芝虜勃律王及公主趣赤佛堂路班師。九月,復至
 婆勒川連雲堡,與邊令誠等相見。其月末,還播密川,令劉單草告捷書,遣中使判官王廷芳告捷。仙芝軍還至河西,夫蒙靈察都不使人迎勞,罵仙芝曰:「啖狗腸高麗奴!啖狗屎高麗奴!於闐使誰與汝奏得?」仙芝曰:「中丞。」「焉耆鎮守使誰邊得?」曰:「中丞。」「安西副都護使誰邊得?」曰:「中丞。」「安西都知兵馬使誰邊得?」曰:「中丞。」靈察曰:「此既皆我所奏,安得不待我處分懸奏捷書!據高麗奴此罪,合當斬,但緣新立大功,不欲處置。」又謂劉單曰:「聞爾能作捷
 書。」單恐懼請罪。令誠具奏其狀曰:「仙芝立奇功,今將憂死。」其年六月,制授仙芝鴻臚卿、攝御史中丞,代夫蒙靈察為四鎮節度使,徵靈察入朝。靈察大懼,仙芝每日見之,趨走如故,靈察益不自安。將軍程千里時為副都護,大將軍畢思琛為靈察押衙,行官王滔、康懷順、陳奉忠等,嘗構譖仙芝於靈察。仙芝既領節度事,謂程千里曰:「公面似男兒,心如婦人,何也?」又謂思琛曰:「此胡敢來!我城東一千石種子莊被汝將去,憶之乎?」對曰:「此是中丞知
 思琛辛苦見乞。」仙芝曰:「吾此時懼汝作威福,豈是憐汝與之!我欲不言,恐汝懷憂,言了無事矣。」又呼王滔等至,捽下將笞,良久皆釋之,由是軍情不懼。



 八載,入朝,加特進,兼左金吾衛大將軍同正員,仍與一子五品官。九載,將兵討石國,平之,獲其國王以歸。仙芝性貪,獲石國大塊瑟瑟十餘石、真金五六馲駝、名馬寶玉稱是。初,舍雞以仙芝為懦緩,恐其不能自存,至是立功,家財鉅萬,頗能散施,人有所求,言無不應。其載,入朝,拜開府儀同三司,尋
 除武威太守、河西節度使,代安思順。思順諷群胡割耳捴面請留,監察御史裴周南奏之,制復留思順,以仙芝為右羽林大將軍。十四載,封密雲郡公。



 十一月,安祿山據範陽叛。是日,以京兆牧、榮王琬為討賊元帥,仙芝為副。命仙芝領飛騎、彍騎及朔方、河西、隴右應赴京兵馬,並召募關輔五萬人,繼封常清出潼關進討,仍以仙芝兼御史大夫。十二月,師發,玄宗御望春亭慰勞遣之,仍令監門將軍邊令誠監其軍,屯於陜州。是月十一日,封
 常清兵敗於汜水。十三日,祿山陷東京,常清以餘眾奔陜州,謂仙芝曰:「累日血戰,賊鋒不可當。且潼關無兵,若狂寇奔突,則京師危矣。宜棄此守,急保潼關。」常清、仙芝乃率見兵取太原倉錢絹,分給將士,餘皆焚之。俄而賊騎繼至,諸軍惶駭,棄甲而走,無復隊伍。仙芝至關,繕修守具,又令索承光守善和戍。賊騎至關,已有備矣,不能攻而去,仙芝之力也。



 封常清,蒲州猗氏人也。外祖犯罪流安西效力,守胡城
 南門,頗讀書,每坐常清於城門樓上,教其讀書,多所歷覽。外祖死,常清孤貧,年三十餘,屬夫蒙靈察為四鎮節度使,將軍高仙芝為都知兵馬使,頗有材能,每出軍,奏傔從三十餘人,衣服鮮明。常清慨然發憤,投牒請預一傔。常清細瘦目類腳短而跛,仙芝見其貌寢,不納。明日又投牒,仙芝謂曰:「吾奏傔已足,何煩復來!」常清怒,倨謂仙芝曰:「常清慕公高義,願事鞭轡,所以無媒而前,何見拒之深乎?公若方圓取人,則士大夫所望;若以貌取人,恐
 失之子羽矣!」仙芝猶未納。常清自爾候仙芝出入,晨夕不離其門,凡數十日,仙芝不得已,補為傔。



 開元末,會達奚部落背叛,自黑山北向,西趣碎葉,玄宗敕靈察邀擊之。靈察使仙芝以二千騎自副城向北至綾嶺下,遇賊擊之。達奚行遠,人馬皆疲,斬殺略盡。常清於幕中潛作捷書,具言次舍井泉,遇賊形勢,克獲謀略,事頗精審。仙芝所欲言,無不周悉,仙芝大駭異之。仙芝軍回,靈察賞勞,仙芝去奴襪帶刀見。判官劉眺、獨孤峻等逆問之曰:「前者捷書,誰之所
 作?副大使幕下何得有如此人」仙芝曰:「即仙芝傔人封常清也。」眺等揖仙芝,命常清進坐,與語如舊相識,眾人方異之。以破達奚功,授疊州地下戍主,便以為判官。累以軍功授鎮將、果毅、折沖。



 天寶六年,從仙芝破小勃律。十二月,仙芝代夫蒙靈察為安西節度使,便奏常清為慶王府錄事參軍,充節度判官,賜紫金魚袋。尋加朝散大夫,專知四鎮倉庫、屯田、甲仗、支度、營田事。仙芝每出征討,常令常清知留後事。常清有才學,果決。知留後時,仙
 芝乳母子鄭德詮已為郎將,德銓母在宅內,仙芝視之如兄弟,家事皆令知之,威望動三軍。常清出回,諸將皆引前,德詮見常清出其門,素易之,自後走馬突常清而去。常清至使院,命左右密引至,連節度使宅院,凡經數重門,德詮既過,命隨後閉之。德詮至,常清離席謂之曰:「常清起自細微,預中丞兵馬使傔,中丞再不納,郎將豈不知乎?今中丞過聽,以常清為留後使,郎將何得無禮,對中使相凌!」因叱之曰:「郎將須暫死以肅軍容。」因令
 勒回,杖六十,面僕地,曳出。仙芝妻及乳母於門外號哭救之,不得,因以其狀上仙芝。仙芝覽之,驚曰:「已死矣!」及見常清,遂無一言,常清亦不之謝。諸大將有罪者,擊殺二人,於是軍中股心慄。



 十載,仙芝改西節度使,奏常清為判官。王正見為安西節度,奏常清為四鎮支度營田副使、行軍司馬。十一載,正見死,乃以常清為安西副大都護,攝御史中丞,持節充安西四鎮節度、經略、支度、營田副大使,知節度事。十三載入朝,攝御史大夫,仍與一
 子五品官,賜第一區,亡父母皆贈封爵。俄而北庭都護程千里入為右金吾大將軍,仍令常清權知北庭都護,持節充伊西節度等使。常清性勤儉,每出征或乘驛,私馬不過一兩匹,賞罰嚴明。



 十四載,入朝,十一月,謁玄宗於華清宮。時祿山已叛,玄宗言兇胡負恩之狀,何方誅討?常清奏曰:「祿山領兇徒十萬,徑犯中原,太平斯久,人不知戰。然事有逆順,勢有奇變,臣請走馬赴東京,開府庫,募驍勇,挑馬箠渡河,計日取逆胡之首懸於闕下。」玄宗
 方憂,壯其言。翌日,以常清為範陽節度,俾募兵東討。其日,常清乘驛赴東京召募,旬日得兵六萬,皆傭保市井之流。乃斫斷河陽橋,於東京為固守之備。十二月,祿山渡河,陷陳留,入罌子穀,兇威轉熾,先鋒至葵園。常清使驍騎與柘羯逆戰,殺賊數十百人。賊大軍繼至,常清退入上東門,又戰不利,賊鼓噪於四城門入,殺掠人吏。常清又戰於都亭驛,不勝。退守宣仁門,又敗。乃從提象門入,倒樹以礙之。至谷水,西奔至陜郡,遇高仙芝,具以
 賊勢告之。恐賊難與爭鋒,仙芝遂退守潼關。



 玄宗聞常清敗,削其官爵,令白衣於仙芝軍效力。仙芝令常清監巡左右廂諸軍,常清衣皁衣以從事。監軍邊令誠每事干之。仙芝多不從。令誠入奏事,具言仙芝、常清逗撓奔敗之狀。玄宗怒,遣令誠齎敕至軍並誅之。



 令誠至潼關,引常清於驛南西街,宣敕示之。常清曰:「常清所以不死者,不忍污國家旌麾,受戮賊手,討逆無效,死乃甘心。」初,常清兵敗入關,欲馳赴闕庭,至渭南,有敕令卻赴潼關,
 自草表待罪。是日臨刑,托令誠上之。其表曰:



 中使駱奉仙至,奉宣口敕,恕臣萬死之罪,收臣一朝之效,令臣卻赴陜州,隨高仙芝行營,。負斧縲囚,忽焉解縛,敗軍之將,更許增修。臣常清誠歡誠喜,頓首頓首。臣自城陷已來,前後三度遣使奉表,具述赤心,竟不蒙引對。臣之此來,非求茍活,實欲陳社稷之計,破虎狼之謀。冀拜首闕庭,吐心陛下,論逆胡之兵勢,陳討捍之別謀。酬萬死之恩,以報一生之寵。豈料長安日遠,謁見無由;函谷關遙,
 陳情不暇!臣讀《春秋》,見狼瞫稱未獲死所,臣今獲矣。



 昨者與羯胡接戰,自今月七日交兵,至於十三日不已。臣所將之兵,皆是烏合之徒,素未訓習。率周南市人之眾,當漁陽突騎之師,尚猶殺敵塞路,血流滿野。臣欲挺身刃下,死節軍前,恐長逆胡之威,以挫王師之勢。是以馳御就日,將命歸天。一期陛下斬臣於都市之下,以誡諸將;二期陛下問臣以逆賊之勢,將誡諸軍;三期陛下知臣非惜死之徒,許臣竭露。臣今將死抗表,陛下或以臣
 失律之後,誑妄為辭;陛下或以臣欲盡所忠,肝膽見察。臣死之後,望陛下不輕此賊,無忘臣言,則冀社稷復安,逆胡敗覆,臣之所願畢矣。仰天飲鴆,向日封章,即為尸諫之臣,死作聖朝之鬼。若使歿而有知,必結草軍前。回風陣上,引王師之旗鼓,平寇賊之戈鋋。生死酬恩,不任感激,臣常清無任永辭聖代悲戀之至。



 常清既刑,陳其尸於蘧蒢上。仙芝歸至,令誠索陌刀手百餘人隨而從之,曰:「大夫亦有恩命。」仙芝遽下,遂至常清所刑處。仙
 芝曰:「我退,罪也,死不辭;然以我為減截兵糧及賜物等,則誣我也。」謂令誠曰:「上是天,下是地,兵士皆在,足下豈不知乎!」其召募兵排列在外,素愛仙芝,仙芝呼謂之曰:「我於京中召兒郎輩,雖得少許物,裝束亦未能足,方與君輩破賊,然後取高官重賞。不謂賊勢憑陵,引軍至此,亦欲固守潼關故也。我若實有此,君輩即言實;我若實無之,君輩當言枉。」兵齊呼曰:「枉」,其聲殷地。仙芝又目常清之尸,謂之曰:「封二,子從微至著,我則引拔子為我
 判官,俄又代我為節度使,今日又與子同死於此,豈命也夫!」遂斬之。



 哥舒翰,突騎施首領哥舒部落之裔也。蕃人多以部落稱姓,因以為氏。祖沮,左清道率。父道元,安西副都護,世居安西。翰家富於財,倜儻任俠,好然諾,縱蒲酒。年四十,遭父喪,三年客居京師,為長安尉不禮,慨然發憤折節,仗劍之河西。初事節度使王倕,倕攻新城,使翰經略,三軍無不震懾。後節度使王忠嗣補為衙將。翰好讀《左氏
 春秋傳》及《漢書》,疏財重氣,士多歸之。忠嗣以為大斗軍副使,嘗使翰討吐蕃於新城,有同列為副者,見翰禮倨,不為用,翰怒,撾殺之,軍中股怵。遷左衛郎將。後吐蕃寇邊,翰拒之於苦拔海,其眾三行,從山差池而下,翰持半段槍當其鋒擊之,三行皆敗,無不摧靡,由是知名。



 天寶六載,擢授右武衛員外將軍,充隴西節度副使、都知關西兵馬使、河源軍使。先是,吐蕃每至麥熟時,即率部眾至積石軍獲取之,共呼為「吐蕃麥莊」,前後無敢拒之
 者。至是,翰使王難得、楊景暉等潛引兵至積石軍,設伏以待之。吐蕃以五千騎至,翰於城中率驍勇馳擊,殺之略盡,餘或挺走,伏兵邀擊,匹馬不還。翰有家奴曰左軍,年十五六,亦有膂力。翰善使槍,追賊及之,以槍搭其肩而喝之,賊驚顧,翰從而刺其喉,皆剔高三五尺而墮,無不死者。左車輒下馬斬首,率以為常。



 其冬,玄宗在華清宮,王忠嗣被劾。敕召翰至,與語悅之,遂以為鴻臚卿,兼西平郡太守,攝御史中丞,代忠嗣為隴右節度支度營田
 副大使,知節度事。仍極言救忠嗣,上起入禁中,翰叩頭隨之而前,言詞慷慨,聲淚俱下,帝感而寬之,貶忠嗣為漢陽太守,朝廷義而壯之。



 明年,築神威軍於青海上,吐蕃至,攻破之;又築城於青海中龍駒島,有白龍見,遂名為應龍城,吐蕃屏跡不敢近青海。吐蕃保石堡城,路遠而險,久不拔。八載,以朔方、河東群牧十萬眾委翰總統攻石堡城。翰使麾下將高秀巖、張守瑜進攻,不旬日而拔之。上錄其功,拜特進、鴻臚員外卿,與一子五品官,賜
 物千匹、莊宅各一所,加攝御史大夫。十一載,加開府儀同三司。



 翰素與祿山、思順不協,上每和解之為兄弟。其冬,祿山、思順、翰並來朝,上使內侍高力士及中貴人於京城東駙馬崔惠童池亭宴會。翰母尉遲氏,於闐之族也。祿山以思順惡翰,嘗銜之,至是忽謂翰曰:「我父是胡,母是突厥;公父是突厥,母是胡。與公族類同,何不相親乎?」翰應之曰:「古人云,野狐向窟嗥,不祥,以其忘本也。敢不盡心焉!」祿山以為譏其胡也,大怒,罵翰曰:「突厥敢如
 此耶!」翰欲應之,高力士目翰,翰遂止。



 十二載,進封涼國公,食實封三百戶,加河西節度使,尋封西平郡王。時楊國忠有隙於祿山,頻奏其反狀,故厚賞翰以親結之。十三載,拜太子太保,更加實封三百戶,又兼御史大夫。



 翰好飲酒,頗恣聲色。至土門軍,入浴室,遘風疾,絕倒良久乃蘇。因入京,廢疾於家。



 及安祿山反,上以封常清、高仙芝喪敗,召翰入,拜為皇太子先鋒兵馬元帥,以田良丘為御史中丞,充行軍司馬,以王思禮、鉗耳大福、李承光、
 蘇法鼎、管崇嗣及蕃將火拔歸仁、李武定、渾萼、契苾寧等為裨將,河隴、朔方兵及蕃兵與高仙芝舊卒共二十萬,拒賊於潼關。上御勤政樓勞遣之,百僚出餞於郊。十五載,加翰尚書左僕射、同中書門下平章事。



 翰至潼關,或勸翰曰:「祿山阻兵,以誅楊國忠為名,公若留兵三萬守關,悉以精銳回誅國忠,此漢挫七國之計也,公以為何如?」翰心許之,未發。有客洩其謀於國忠,國忠大懼,及奏曰:「兵法『安不忘危』,今潼關兵眾雖盛,而無後殿,萬一
 不利,京師得無恐乎!請選監牧小兒三千人訓練於苑中。」詔從之,遂遣劍南軍將李福、劉光庭分統焉。又奏召募一萬人,屯於灞上,令其腹心杜乾運將之。翰慮為所圖,乃上表請乾運兵隸於潼關,遂召乾運赴潼關計事,因斬之。自是,翰心不自安。又素有風疾,至是頗甚,軍中之務,不復躬親,委政於行軍司馬田良丘。良丘復不敢專斷,教令不一,頗無部伍。其將王思禮、李承光又爭長不葉,人無鬥志。



 先是,翰數奏祿山雖竊河朔,而不得人
 心,請持重以弊之,彼自離心,因而翦滅之,可不傷兵擒茲寇矣。賊將崔乾祐於陜郡潛鋒蓄銳,而覘者奏云「賊殊無備」,上然之,命悉眾速討之。翰奏曰:「賊既始為AT逆,祿山久習用兵,必不肯無備,是陰計也。且賊兵遠來,利在速戰。今王師自戰其地,利在堅守,不利輕出;若輕出關,是入其算。乞更觀事勢。」楊國忠恐其謀己,屢奏使出兵。上久處太平,不練軍事,既為國忠眩惑,中使相繼督責。翰不得已,引師出關。



 六月四日,次於靈寶縣之西原。
 八日,與賊交戰,官軍南迫險峭,北臨黃河;崔乾祐以數千人先據險要。翰及良丘浮船中流以觀進退,謂乾祐兵少,輕之,遂促將士令進,爭路擁塞,無復隊伍。午後,東風急,乾祐以草車數十乘縱火焚之,煙焰亙天。將士掩面,開目不得,因為AT徒所乘,王師自相排擠,墜於河。其後者見前軍陷敗,悉潰,填委於河,死者數萬人,號叫之聲振天地,縛器械,以槍為楫,投北岸,十不存一二。軍既敗,翰與數百騎馳而西歸,為火
 拔歸仁執降於賊。祿山謂之曰:「汝常輕我,今日如何?」翰懼,俯伏稱:「肉眼不識陛下,遂至於此。陛下為撥亂主,今天下未平,李光弼在土門,來填在河南,魯炅在南陽,但留臣,臣以尺書招之,不日平矣。」祿山大喜,遂偽署翰司空。作書招光弼等,諸將報書皆讓翰不死節。祿山知事不諧,遂閉翰於苑中,潛殺之。



 翰之守潼關也,主天下兵權,肆志報怨,誣奏戶部尚書安思順與祿山潛通,偽令人為祿山遺思順書,於關門擒之以獻。其年三月,思順及弟太僕卿元貞坐
 誅,徙其家屬於嶺外,天下冤之。



 史臣曰:大盜作梗,祿山亂常,詞雖欲誅國忠,志則謀危社稷。於時承平日久,金革道消,封常清、高仙芝相次率不教之兵,募市人之眾,以抗AT寇,失律喪師。哥舒翰廢疾於家,起專兵柄,二十萬眾拒賊關門,軍中之務不親,委任又非其所。及遇羯賊,旋致敗亡,天子以之播遷,自身以之拘執,此皆命帥而不得其人也。《禮》曰:「大夫死眾。」又曰:「謀人之軍師敗則死之。」翰受署賊庭,茍延視息,忠
 義之道,即可知也,豈不愧於顏杲卿乎!抑又聞之,古之命將者,推轂而謂之曰:「閫外之事,將軍裁之。」觀楊國忠之奏事,邊令誠之護戎,又掣肘於軍政者也,未可偏責三帥,不尤伊人。後之君子,得不深鑒!



 贊曰:羯賊犯順,戎車啟行。委任失所,封、高敗亡。虔劉圻甸,僭竊衣裳。丑哉舒翰,不能死王。



\end{pinyinscope}