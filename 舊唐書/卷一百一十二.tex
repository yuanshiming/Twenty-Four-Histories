\article{卷一百一十二}

\begin{pinyinscope}

 ○韋見素子諤益益子顗崔圓崔渙子縱杜鴻漸



 韋見素,字會微,京兆萬年人。父湊,開元中太原尹。見素學科登第。景龍中,解褐相王府參軍,歷衛佐、河南府倉
 曹。丁父憂,服闋,起為大理寺丞,襲爵彭城郡公。坐事出為坊州司馬。入為庫部員外郎,加朝散大夫,歷右司兵部二員外,左司兵部二郎中,遷諫議大夫。天寶五年,充江西、山南、黔中、嶺南等黜陟使,觀省風俗,彈糾長吏,所至肅然。使還,拜給事中,駁正繩違,頗振臺閣舊典。尋檢校尚書工部侍郎,改右丞。九載,遷吏部侍郎,加銀青光祿大夫。見素仁恕長者,意不忤物,及典選累年,銓敘平允,人士稱之。時右相楊國忠用事,左相陳希烈畏其權
 寵,凡事唯諾,無敢發明,玄宗頗知之,聖情不悅。天寶十三年秋,霖雨六十餘日,京師廬舍垣墉頹毀殆盡,凡一十九坊汙潦。天子以宰輔或未稱職,見此咎徵,命楊國忠精求端士,時兵部侍郎吉溫方承寵遇,上意用之。國忠以溫祿山賓佐,懼其威權,奏寢其事。國忠訪於中書舍人竇華、宋昱等,華、昱言見素方雅,柔而易制。上亦以經事相王府,有舊恩,可之。其年八月,拜武部尚書、同中書門下平章事,充集賢院學士,知門下省事,代陳希烈。
 見素既為國忠引用,心德之。時祿山與國忠爭寵,兩相猜嫌,見素亦無所是非,署字而已,遂至兇胡犯順,不措一言。



 十五年六月,哥舒翰兵敗桃林,潼關不守。是月,玄宗蒼黃出幸,莫知所詣。楊國忠以身領劍南旄鉞,請幸成都。見素與國忠、御史大夫魏方進遇上於延秋門,便扈從之咸陽。翌日,次馬嵬驛,軍士不得食,流言不遜。龍武將軍陳玄禮懼其亂,乃與飛龍馬家李護國謀於皇太子,請誅國忠,以慰士心。是日,玄禮等禁軍圍行宮,盡
 誅楊氏。見素遁走,為亂兵所傷,眾呼曰:「勿傷韋相!」識者救之,獲免。上聞之,令壽王瑁宣慰,賜藥傅瘡。魏方進為亂兵所殺。是日,朝士獨見素一人。是夜宿馬嵬,上命見素子京兆府司錄參軍諤為御史中丞,充置頓使。凌晨將發,六軍將士曰:「國忠反叛,不可更往蜀川,請之河、隴。」或言靈武、太原,或云還京,議者不一。上意在劍南,慮違士心,無所言。諤曰:「還京須有捍賊之備。今兵馬數少,恐非萬全,不如且至扶風,徐圖去就。」上詢於眾,眾以為然,
 乃令皇太子後殿。



 上至扶風郡,從駕諸軍各圖去就,頗出醜言。陳玄禮不能制,上聞之憂懼。會益州貢春彩十萬疋,乃以其綱使濛陽尉劉景溫為監察御史,其彩悉陳於廷,召六軍將士等入,上謂之曰:「卿等皆國之功臣,勛勞素著,朕之優賞,常亦不輕。逆胡負恩,事須回避,甚知卿等不得別父母妻子,朕亦不及辭九廟。」言發涕流。又曰:「朕今須幸蜀,蜀路險狹,人若多往,恐難祗供。今有此彩,卿等即宜分取,各自圖去就。朕自有子弟、中官等
 相隨,便與卿等訣別。」眾咸俯伏號泣,曰:「死生從陛下。」上良久曰:「去住聽卿自便。」自是醜言方息。七月,至巴西郡,以見素兼左相、武部尚書。數日,至蜀郡,加金紫光祿大夫,進封豳國公,與一子五品官。



 是月,皇太子即位於靈武,道路艱澀,音驛未通。八月,肅宗使至,始知靈武即位。尋命見素與宰臣房琯齎傳國寶玉冊奉使靈武,宣傳詔命,便行冊禮。將行,上皇謂見素等曰:「皇帝自幼仁孝,與諸子有異,朕豈不知。往十三年,已有傳位之意,屬其
 歲水旱,左右勸朕且俟豐年。爾來便屬祿山構逆,方隅震擾,未遂此心。昨發馬嵬,亦有處分。今皇帝受命,朕心頓如釋負。勞卿等遠去,勉輔佐之。多難興王,自古皆有,卿等乃心王室,以宗社為念,早定中原,吾之望也。」見素等悲泣不自勝。仍以見素子諤及中書舍人賈至充冊禮使判官。時肅宗已回幸順化郡。九月,見素等至,冊禮畢,從幸彭原郡。肅宗在東宮,素聞房琯名重,故虛懷以待;以見素常附國忠,禮遇稍薄。明年,至鳳翔。三月,除左
 僕射,罷知政事,以憲部尚書致仕。苗晉卿代為左相。



 初,肅宗在鳳翔,喪亂之後,綱紀未立,兵吏三銓,簿籍煨燼,南曹選人,文符悉多偽濫。上以兇醜未滅,且示招懷,據到注擬,一無檢括。見素曰:「臣典選歲久,周知此弊。今寰區未復,員闕不多。若總無條綱,恐難持久。」上然之,未暇厘革。及還京,選人數千,補授無所,喧訴於朝,由是行見素之言。及房琯以敗軍左降,崔圓、崔渙等皆罷知政事,上皇所命宰臣,無知政事者。五月,遷見素太子太師。十一
 月,肅宗自右輔還京,詔見素入蜀奉迎太上皇。十二月,上皇至京師,肅宗御樓大赦。見素以奉上皇幸蜀功,加開府儀同三司,食實封三百戶。上元中,以足疾上表請致仕,許之。寶應元年十二月卒,年七十六,贈司空,謚曰忠貞,喪事官給。子倜、諤、益、丱。倜、諤皆位至給事中,益終刑部員外郎,丱終秘書丞。倜子頌。



 益子顗,字周人,生一歲而孤,事姊稱為恭孝。性嗜學,尤精陰陽、象緯、經略、風俗之書。善持論,有清譽。少以門廕補
 千牛備身,自鄠縣尉判入等,授萬年尉,歷御史、補闕、尚書郎,累遷給事中、尚書左丞、戶部侍郎、中丞、吏部侍郎。其在諫垣,與李約、李正辭迭申裨諷,頗回大政。宰相裴垍、李絳、崔群輩多與友善,而後進之有浮名者,亦游其門,以是稱有時望。及李逢吉殲朋黨以專政柄,而顗附麗之跡尤密,頗為時人所譏。然處身儉約,有足多者。著《易蘊解》,推演潛亢終始之義,甚有奧旨。寶歷元年七月卒,贈禮部尚書。



 崔圓,清河東武城人也。後魏左僕射亮之後。父景晊,官
 至大理評事。圓少孤貧,志尚閎博,好讀兵書,有經濟宇宙之心。開元中,詔搜訪遺逸,圓以鈐謀射策甲科,授執戟。自負文藝,獲武職,頗不得意。蕭炅為京兆尹,薦為會昌丞,累遷司勛員外郎。宰臣楊國忠遙制劍南節度使,引圓佐理,乃奏授尚書郎,兼蜀郡大都督府左司馬,知節度留後。天寶末,玄宗幸蜀郡,特遷蜀郡大都督府長史、劍南節度。圓素懷功名,初聞國難,潛使人探國忠深旨,知有行幸之計,乃增修城池,建置館宇,儲備什器。及
 乘輿至,殿宇牙帳咸如宿設,玄宗甚嗟賞之,即日拜中書侍郎、同中書門下平章事、劍南節度,餘如故。



 肅宗即位,玄宗命圓同房琯、韋見素並赴肅宗行在所,玄宗親制遺愛碑於蜀以寵之。從肅宗還京,以功拜中書令,封趙國公,賜實封五百戶。明年,罷知政事,遷太子少師,留守東都。會官軍不利於相州,軍回過洛陽,所在剽掠。圓棄城南奔襄陽,詔削除階封。尋起為濟王傅。李光弼用為懷州刺史,除太子詹事,改汾州刺史,皆以理行稱。拜
 揚州大都督府長史、淮南節度觀察使,加檢校右僕射、兼御史大夫,轉檢校左僕射知省事。大歷三年六月薨,年六十四,輟朝三日,贈太子太師,謚曰昭襄。



 崔渙,祖玄暐,神龍功臣,封博陵郡王。父璩,文學知名,位至禮部侍郎。渙少以士行聞,博綜經籍,尤善談論,累遷尚書司門員外郎。天寶末,楊國忠出不附己者,渙出為劍州刺史。天寶十五載七月,玄宗幸蜀,渙迎謁於路,抗詞忠懇,皆究理體,玄宗嘉之,以為得渙晚。宰臣房琯又
 薦之,即日拜黃門侍郎、同中書門下平章事,扈從成都府。



 肅宗靈武即位。八月,與左相韋見素、同平章事房琯、崔圓同齎冊赴行在。時未復京師,舉選路絕,詔渙充江淮宣諭選補使,以收遺逸。惑於聽受,為下吏所鬻,濫進者非一,以不稱職聞。乃罷知政事,除左散騎常侍,兼餘杭太守、江東採訪防禦使。旋授正議大夫、太子賓客。乾元三年正月,轉大理卿。再遷吏部侍郎、檢校工部尚書、集賢院待詔。性尚簡澹,不交世務,頗為時望所歸。
 遷御史大夫,加稅地青苗錢物使。時以此錢充給京百官料,渙為屬吏希中,以下估為使料,上估為百官料。其時為皇城副留守張清發之,詔下有司訊鞫,渙無詞以對,坐是貶道州刺史。大歷三年十二月壬寅,以疾終。



 子縱,初以廕補協律郎,三遷為監察御史。詔擇令長於臺省,除藍田令,寬明勤幹,德化大行,縣人為之立碑頌德。轉京兆府司錄,累遷金部員外郎。以父貶道州刺史,棄官就養。丁父憂,終制,六遷大理卿、兼御史中丞、汴西水陸運
 兩稅鹽鐵等使。田悅連敗,走魏州,嬰城自守,諸道兵圍之,屢乏食,詔縱兼魏州四節度糧料使,軍儲稍給。德宗幸奉天,四方握兵,未有至者。縱先知之,潛告李懷光勸令奔命,懷光從之。縱乃悉斂軍財與懷光俱來,調給具備。懷光兵士久戰河外,及次河中,將遷延。縱之貨幣先已渡河,縱謂眾曰:「若濟,悉以分賜。」眾利之,乃西。至奉天,加右庶子,充使。無幾,拜京兆尹、兼御史大夫。數奏懷光剛愎反覆,宜陰備之。及行幸梁州,左右或短之曰:「縱素
 善懷光,今不來矣。」上曰:「他人不知縱,吾可保其心。」不數日,縱至,拜御史大夫。嘗議其大體,不親細事,獄訴儀制,皆付之僚吏。



 貞元元年,親祠南郊,為大禮使。屬兵旱之後,賦入尚少,縱裁定文物,儉而中禮。無何,萬年丞源邃為京兆尹李齊運所抑捽至死,縱劾奏不行。數月,除吏部侍郎,尋檢校禮部尚書、東畿唐汝鄧都觀察使、河南尹。是時兵革甫定,民耗六七,縱悉心求瘼,為理簡易。先是,戍邊之師由洛陽者,儲餼取辦於編戶。縱始官備,不
 征於人,令五家相保,俾自占告發斂,以絕胥吏之私。又引伊、洛水以通裏閈,都中灌溉濟不逮為十一二,人甚安之。徵拜太常卿。貞元七年六月卒官,年六十二,謚曰忠,贈吏部尚書。



 縱孝悌,修飭自立,以父為元載排抑,居退十餘年,左宦外府,訖載得罪,不求聞達。初,渙有寵妾鄭氏,縱以母事之。鄭氏性剛戾,待縱不以理,雖為大僚,每加笞詬。縱率妻子候顏,敬順不懈,時以為難。



 杜鴻漸,故相暹之族子。祖慎行,益州長史。父鵬舉,官至
 王友。鴻漸敏悟好學,舉進士,解褐王府參軍。天寶末,累遷大理司直,朔方留後、支度副使。



 肅宗北幸,至平涼,未知所適。鴻漸與六城水運使魏少游、節度判官崔漪、支度判官盧簡金、關內鹽池判官李涵謀曰:「今胡羯亂常,二京陷沒,主上南幸於巴蜀,皇太子理兵於平涼。然平涼散地,非聚兵之處,必欲制勝,非朔方不可。若奉殿下,旬日之間,西收河、隴,回紇方強,與國通好,北征勁騎,南集諸城,大兵一舉,可復二京。雪社稷之恥,上報明主,下
 安蒼生,亦臣子之用心,國家之大計也。」鴻漸即日草箋具陳兵馬招集之勢,錄軍資、器械、倉儲、庫物之數,令李涵齎赴平涼,肅宗大悅。鴻漸知肅宗發平涼,於北界白草頓迎謁,因勞諸使及兵士,進言曰:「朔方天下勁兵,靈州用武之處。今回紇請和,吐蕃內附,天下郡邑,人皆堅守,以待制命。其中雖為賊所據,亦望不日收復,殿下整理軍戎,長驅一舉,則逆胡不足滅也。」肅宗然之。及至靈武,鴻漸與裴冕等勸即皇帝位,以歸中外之望,五上表,
 乃從。鴻漸素習帝王陳布之儀,君臣朝見之禮,遂採摭舊儀,綿蕝其事。城南設壇壝,先一日具儀注草奏。肅宗曰:「聖君在遠,寇逆未平,宜罷壇場。」餘可其奏。肅宗即位,授兵部郎中,知中書舍人事,尋轉武部侍郎。至德二年,兼御史大夫,為河西節度使、涼州都督。兩京平,遷荊州大都督府長史、荊南節度使。



 襄州大將康楚元、張嘉延盜所管兵,據襄州城叛,刺史王政遁走。嘉延南襲荊州,鴻漸聞之,棄城而遁。澧、朗、硤、歸等州聞鴻漸出奔,皆惶
 駭,潛竄山谷。歲餘,徵拜尚書右丞、吏部侍郎、太常卿,充禮儀使。二聖晏駕,鴻漸監護儀制,山陵畢,加光祿大夫,封衛國公。廣德二年,代宗將享郊廟,拜鴻漸兵部侍郎、同中書門下平章事,尋轉中書侍郎。



 永泰元年十月,劍南西川兵馬使崔旰殺節度使郭英乂,據成都,自稱留後。邛州衙將柏貞節、瀘州衙將楊子琳、劍州衙將李昌巙等興兵討旰,西蜀大亂。明年二月,命鴻漸以宰相兼充山、劍副元帥、劍南西川節度使,以平蜀亂。鴻漸心無
 遠圖,志氣怯懦,又酷好浮圖道,不喜軍戎。既至成都,懼旰雄武,不復問罪,乃以劍南節制表讓於旰。時西戎寇邊,關中多事,鴻漸孤軍陷險,兵威不振,代宗不獲已,從之。仍以旰為劍南西川行軍司馬,柏貞節為邛州刺史,楊子琳為瀘州刺史,各罷兵。尋請入覲,仍表崔旰為西川兵馬留後。大歷二年,詔以旰為成都尹、劍南西川節度使,召鴻漸還京。鴻漸仍率旰同入覲,代宗嘉之。後知政事,轉門下侍郎,讓山南副元帥。三年八月,
 代王縉為東都留守,充河南、淮西、山南東道副元帥,平章事如故。以疾上表乞骸骨,從之,竟不之任。四年十一月卒,贈太尉,謚曰文憲。輟朝三日,賜物五百疋,粟五百石。



 鴻漸晚年樂於退靜,私第在長興里,館宇華靡,賓僚宴集。鴻漸悠然賦詩曰:「常願追禪理,安能挹化源。」朝士多屬和之。及休致後病,令僧剃頂發,及卒,遺命其子依胡法塔葬,不為封樹,冀類緇流,物議哂之。



 史臣曰:祿山狂悖已顯,玄宗寵任無疑,見素知國危,陳
 廟算,直言極諫,而君不從,獨正犯難,而人不咎,出生入死,善始令終者鮮矣。時論以見素取容於國忠,無言匡大政。且國忠恃內戚,弄重權,沮林甫奸豪,取其大位,若見素之孤直,豈許取容?蓋禍胎已成,政柄久紊,見素入相餘年,言不從而難作,雖有周、孔之才,其能匡救者乎?諤才辯,顗儉約,雅符積善之慶矣。圓守文之士,非禦侮之才。渙才兼行聞,命與時會。發言上沃主意,遽致顯榮;當官屢為吏欺,終及竄逐。所謂可與適道,未可與權。縱
 忠於國,能於官,孝於家,三者備矣,孰能繼之?鴻漸有衛社之功,非干城之責,時以任崔旰為非,則不然矣。且旰南拒貞節,北敗獻誠,宜以懷來,未可力制,終致歸國,豈非臧謀?向討之,即為劇賊矣。然事佛徼福,朋勢取容,非君子之道焉。



 贊曰:玄宗失德,祿山肆逆。見素竭節,諸公協力。



\end{pinyinscope}