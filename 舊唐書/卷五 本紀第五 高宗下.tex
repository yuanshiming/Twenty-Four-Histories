\article{卷五 本紀第五 高宗下}

\begin{pinyinscope}

 麟德三年春正月戊辰朔,車駕至泰山頓。是日親祀昊天上帝於封
 祀壇,以高祖、太宗配饗。己巳,帝升山行封禪之禮。庚午,禪於社首,祭皇地祇,以太穆太皇太后、文德皇太后配饗;皇后為亞獻,越國太妃燕氏為終獻。辛未,御降禪壇。



 壬申,御朝覲壇受朝賀。改麟德三年為乾封元年,諸行從文武官及朝覲華戎岳牧、致仕老人朝朔望
 者,三
 品已上賜爵二等,四品已下、七品以上加階,八品已下加一階,勛一轉。諸老人百歲已上版授下州刺史,婦人郡君;九十、八十節級。齊州給復一年半,管岳縣二年。所歷之處,無出今年租賦。乾封元年正月五日已前,大赦天下,賜酺七日。癸酉,宴群臣,陳《九部樂》,賜物有
 差,日失而罷。丙子,皇太子弘設會。丁丑,以前恩薄,普進爵及階勛等。男子賜古爵。兗州界置紫雲。仙鶴、萬歲觀,封巒、非煙、重輪三寺。天下諸州署觀、寺一所。丙戌,發自泰山。甲午,次曲阜縣,幸孔子廟。追贈太師,增修祠宇,以少牢致祭。其褒聖侯德倫子孫,並免賦役。



 二月己未,次亳州。幸老君廟,追號曰太上玄元皇帝,創造祠堂。其廟置令、丞各一員。改穀陽縣為真源縣,縣內宗姓特給復一年。夏四月甲辰,車駕至自泰山,先謁太廟而後入。五
 月庚寅,改鑄乾封泉寶錢。六月壬寅,高麗莫離支蓋蘇文死。其子男生繼其父位,為其弟男建所逐,使其子獻誠詣闕請降,詔左驍衛大將軍契苾何力率兵以應接之。秋七月乙丑,徙封殷王旭輪為豫王。庚午,左侍極、檢校右相、嘉興子陸敦信緣老病乞辭機揆,拜大司成,兼知左侍極。大司憲兼檢校右中護劉仁軌兼右相、檢校右中護。八月辛丑,兼司元太常伯、兼檢校左相、鉅鹿男竇德玄卒。丁未,殺司衛少卿武惟良、淄州刺史武懷運,仍改
 姓蝮氏。冬十月己酉,命司空、英國公勣為遼東道行軍大總管,以伐高麗。



 二年春正月丁丑,以去冬至於是月無雨雪,避正殿,減膳,親錄囚徒。罷乾封錢,復行開元通寶錢。二月戊戌,涪陵郡王愔薨。辛丑,改萬年宮依舊名九成宮。夏六月乙卯,西臺侍郎楊武,西臺侍郎、道國公、檢校太子左中護戴至德,正諫議大夫、檢校東臺侍郎、安平郡公李安期,東臺侍郎張文瓘,並同東西臺三品。秋八月己丑朔,日
 有蝕之。丙辰,東臺侍郎李安期出為荊州大都督府長史。



 三年春正月庚寅,詔繕工大監兼瀚海都護劉審禮為西域道安撫大使。壬子,以右相劉仁軌為遼東道副大總管。二月戊午,遼東道破薛賀水五萬人,陣斬首五千餘級,獲生口三萬餘人,器械牛馬不可勝計。丙寅,以明堂制度歷代不同,漢、魏以還,彌更訛舛,遂增損古今,新制其圖。下詔大赦,改元為總章元年。二月戊寅,幸九成宮。己卯,分長安、萬年置乾封、明堂二縣,分理於京城之中。
 癸未,皇太子弘釋奠於國學。贈顏回太子少師,曾參太子少保。



 夏四月丙辰,有彗星見於畢、昴之間。乙丑,上避正殿,減膳,詔內外群官各上封事,極言過失。於是群臣上言:「星雖孛而光芒小,此非國眚,不足上勞聖慮,請御正殿,復常饌。」帝曰:「朕獲奉宗廟,撫臨億兆,謫見於天,誡朕之不德也,當責躬修德以禳之。」群臣復進曰:「星孛於東北,此高麗將滅之徵。」帝曰:「高麗百姓,即朕之百姓也。既為萬國之主,豈可推過於小蕃!」竟不從所請。乙亥,彗
 星滅。辛巳,西臺侍郎楊武卒。秋八月癸酉,至自九成宮。九月癸巳,司空、英國公勣破高麗,拔平壤城,擒其王高藏及其大臣男建等以歸。境內盡降,其城一百七十,戶六十九萬七千,以其地為安東都護府,分置四十二州。



 二年春正月,封諸王嫡子皆為郡王。二月,東臺侍郎、同東西臺三品兼知左史事張文瓘署位,始入銜。三月,東臺侍郎郝處俊同東西臺三品。癸酉,皇后親祀先蠶。夏四月乙酉,幸九成宮。置司列少常伯、司戎少常伯各兩
 員。



 五月庚子,移高麗戶二萬八千二百,車一千八十乘,牛三千三百頭,馬二千九百匹,駝六十頭,將入內地,萊、營二州般次發遣,量配於江、淮以南及山南、並、涼以西諸州空閑處安置。六月戊申朔,日有蝕之。括州大風雨,海水泛溢永嘉、安固二縣城郭,漂百姓宅六千八百四十三區,溺殺人九千七十、牛五百頭,損田苗四千一百五十頃。冀州大水,漂壞居人廬舍數千家。並遣使賑給。秋七月,劍南益、瀘、巂、茂、陵、邛、雅、綿、翼、維、始、簡、資、榮、隆、果、
 梓、普、遂等一十九州旱,百姓乏絕,總三十六萬七千六百九十戶,遣司珍大夫路勵行存問賑貸,癸巳,冀州大都督府奏,自六月十三日夜降雨,至二十日水深五尺,其夜暴水深一丈已上,壞屋一萬四千三百九十區,害田四千四百九十六頃。遣右衛大將軍、涼國公契苾何力為駕海道行軍大總管。秋八月甲戌,改瀚海都護府為安北都護府。



 九月己亥,發自九成宮。壬寅,停華林頓,大蒐於岐。乙巳,至岐州。高祖初仕隋為扶風太守,故曲
 赦岐州管內。高祖時胥徒隨材擢用,賜高年衣物粟帛各有差。冬十月丁巳,至自九成宮。十一月庚辰,發九州人夫,轉發太原倉米粟入京。丁亥,徙封豫王旭輪為冀王,仍令單名輪。十二月戊申,司空、太子太師、英國公勣薨。是冬無雪。



 三年春正月丁丑,右相、樂成男劉仁軌致仕。辛卯,列遼東地為州縣。二月戊申,以旱,親錄囚徒,祈禱名山大川。癸丑,日色出如赭。三月甲戌朔,大赦天下,改元為咸亨
 元年。三月丁丑,改蓬萊宮為含元殿。壬辰,太子少師、同東西臺三品許敬宗致仕。



 夏四月,吐蕃寇陷白州等一十八州,又與於闐合眾襲龜茲撥換城,陷之。罷安西四鎮。辛亥,以右威衛大將軍薛仁貴為邏娑道行軍大總管,右衛員外大將軍阿史那道真、左衛將軍郭待封為副,領兵五萬以擊吐蕃。庚午,幸九成宮。雍州大雨雹。



 五月丙戌,詔曰:「諸州縣孔子廟堂及學館有破壞並先來未造者,遂使生徒無肄業之所,先師闕奠祭之儀,久致
 飄露,深非敬本。宜令所司速事營造。」六月壬寅朔,日有蝕之。秋七月戊子,前西臺侍郎李敬玄起復本職,仍依舊同東西臺三品。薛仁貴、郭待封至大非川,為吐蕃大將論欽陵所襲,大敗,仁貴等並坐除名。吐谷渾全國盡沒,唯慕容諾曷缽及其親信數千帳內屬,仍徙於靈州界。八月甲子,至自九成宮。梁州都督、趙王福薨。丙寅,以久旱,避正殿,尚食減膳。九月甲申,衛國夫人楊氏薨,贈魯國夫人,謚曰忠烈。閏月壬子,故贈司徒、周忠孝公士
 鷿贈太尉、太子太師、太原郡王,贈魯國忠烈太夫人贈太原王妃。甲寅,葬太原王妃,京官文武九品已上及外命婦,送至便橋宿次。



 冬十月癸酉,大雪,平地三尺餘,行人凍死者贈帛給棺木。令雍、同、華州貧窶之家,有年十五已下不能存活者,聽一切任人收養為男女,充驅使,皆不得將為奴婢。丙申,太子右中護兼攝正諫大夫、同東西臺三品趙仁本為左肅機,罷知政事。十二月庚寅,諸司及百官各復舊名。是歲,天下四十餘州旱及霜蟲,
 百姓饑乏,關中尤甚。詔令任往諸州逐食,仍轉江南租米以賑給之。



 二年春正月乙巳,幸東都。留皇太子弘於京監國,令侍臣戴至德、張文瓘、李敬玄等輔之。唯以閻立本、郝處俊從。甲子,至東都。二月丁亥,雍州人梁金柱請出錢三千貫賑濟貧人。夏四月戊子,大風折木。六月戊寅,左散騎常侍兼檢校秘書、太子賓客、周國公武敏之以罪復本姓賀蘭氏,除名,流雷州。丁亥,以旱,親錄囚徒。秋九月,地
 震。司徒、潞州刺史、徐王元禮薨。冬十月,搜揚明達禮樂之士。十一月甲午朔,日有蝕之。庚戌,幸許、汝等州教習。癸酉,冬狩,校獵於許州葉縣昆水之陽。十二月丙戌,還東都。



 三年春正月辛丑,發梁、益等一十八州兵,募五千三百人,遣右衛副率梁積壽往姚州擊叛蠻。辛未,制雍、洛二州人聽任本州官。二月己卯,侍中、永安郡公姜恪卒於河西鎮守。



 夏四月戊寅,幸合璧宮。壬午,於水南教旗。上
 問中書令閻立本、黃門侍郎郝處俊:「伊尹負鼎俎於湯,應是補緝時政,不知鑄鼎所緣,復在何國?將為國之重器,歷代傳寶?」閻立本以古義對。五月乙未,五品已上改賜新魚袋,並飾以銀;三品已上各賜金裝刀子、礪石一具。六月丙子,於洛州柏崖置倉。八月壬子,特進、高陽郡公許敬宗卒。九月乙卯,冀州大都督府復為魏州,魏州復為冀州。壬寅,沛王賢徙封雍王。



 冬十月己未,皇太子監國。壬戌,車駕還京師。乙亥,中書侍郎、同中書門下三
 品、道國公戴至德加兼戶部尚書,黃門侍郎、同中書門下三品張文瓘檢校大理卿,黃門侍郎、甑山縣公、同中書門下三品郝處俊為中書侍郎,兼檢校吏部侍郎、同中書門下三品李敬玄為吏部侍郎,並依舊同中書門下三品。十一月戊子朔,日有蝕之。甲辰,至自東都。十二月癸卯,太子左庶子劉仁軌同中書門下三品。是冬,左監門大將軍高侃大敗新羅之眾於橫水。



 四年春正月甲午,詔咸亨初收養為男女及驅使者,聽
 量酬衣食之直,放還本處。丙辰,絳州刺史、鄭王元懿薨。二月壬午,以左金吾將軍裴居道女為皇太子弘妃。夏四月丙子,幸九成宮。閏五月丁卯,燕山道總管李謹行破高麗叛黨於瓠盧河之西,高麗平壤餘眾遁入新羅。秋七月庚午,九成宮太子新宮成,上召五品已上諸親宴太子宮,極歡而罷。辛巳,婺州暴雨,水泛溢,漂溺居民六百家,詔令賑給。八月辛丑,上痁疾,令太子受諸司啟事。己酉,大風毀太廟鴟吻。



 冬十月壬午,中書令、博陵縣
 子閻立本卒。乙未,皇太子弘納妃畢,曲赦岐州,大酺三日。庚子,還京師。乙巳,至自九成宮。十一月丙寅,上制樂章,有《上元》、《二儀》、《三才》、《四時》、《五行》、《六律》、《七政》、《八風》、《九宮》、《十洲》、《得一》、《慶雲》之曲,詔有司諸大祠享即奏之。十二月丙午,弓月、疏勒二國王入朝請降。



 五年春二月壬午,遣太子左庶子、同中書門下三品劉仁軌為雞林道大總管,以討新羅,仍令衛尉卿李弼、右領大將軍李謹行副之。三月辛亥朔,日有蝕之。己巳,皇
 后祀先蠶。



 夏四月辛卯,以尚輦奉御、周國公武承嗣為宗正卿。五月己未,詔:「春秋二社,本以祈農,如聞此外別為邑會。此後除二社外,不得聚集,有司嚴加禁止。」六月壬寅,太白入東井。秋八月壬辰,追尊宣簡公為宣皇帝,懿王為光皇帝,太祖武皇帝為高祖神堯皇帝,太宗文皇帝為文武聖皇帝,太穆皇后為太穆神皇后,文德皇后為文德聖皇后。皇帝稱天皇,皇后稱天後。改咸亨五年為上元元年,大赦。戊戌,敕文武官三品己上服紫,金
 玉帶;四品深緋,五品淺緋,並金帶;六品深綠,七品淺綠,並銀帶;八品深青,九品淺青,鍮石帶;庶人服黃,銅鐵帶。一品已下文官,並帶手巾、算袋、刀子、礪石,武官欲帶亦聽之。



 九月辛亥,百僚具新服,上宴之於麟德殿。癸丑,追復長孫無忌官爵,仍以其曾孫翼襲封趙國公,許歸葬於昭陵先造之墳。十一月丙午朔,幸東都。己酉,狩於華山之曲武原。戊辰,至東都。十二月,蔣王惲薨。戊子,於闐王伏闍雄來朝。辛卯,波斯王卑路斯來朝。壬寅,天後上
 意見十二條,請王公百僚皆習《老子》,每歲明經一準《孝經》、《論語》例試於有司。又請子父在為母服三年。虢王鳳薨。



 二年春正月甲寅,熒惑犯房。壬戌,支汗郡王獻碧玻璃。丙寅,以於闐為毗沙都督府,以尉遲伏闍雄為毗沙都督,分其境內為十州,以伏闍雄有擊吐蕃功故也。庚午,龜茲王白素稽獻銀頗羅。辛未,吐蕃遣其大臣論吐渾彌來請和,不許。



 二月,雞林道行軍大總管大破新羅之
 眾於七重城,斬獲甚眾。新羅遣使入朝獻方物,伏罪。赦之,復其王金法敏官爵。三月丁未,日色如赭。丁巳,天后親蠶於邙山之陽。時帝風疹不能聽朝,政事皆決於天後。自誅上官儀後,上每視朝,天後垂簾於御座後,政事大小皆預聞之,內外稱為「二聖」。帝欲下詔令天后攝國政,中書侍郎郝處俊諫止之。



 夏四月,分括州永嘉、永固二縣置溫州,析臨海縣為樂安、永寧二縣。辛巳,周王顯妃趙氏以罪幽死。己亥,皇太子弘薨於合璧宮之綺雲
 殿。時帝幸合璧宮,是日還東都。五月己亥,追謚太子弘為孝敬皇帝。



 六月戊寅,以雍王賢為皇太子,大赦。秋七月辛亥,洛州復置緱氏縣,以管孝敬皇帝恭陵。慈州刺史、杞王上金坐事,於澧州安置。八月庚子,太子左庶子、同中書門下三品、樂成侯劉仁軌為左僕射,依舊監修國史。中書門下三品、大理卿張文瓘為侍中。中書侍郎、同三品、甑山公郝處俊為中書令,監修國史如故。吏部侍郎、檢校太子左庶子、監修國史李敬玄吏部尚書兼
 太子左庶子、同中書門下三品,依前監修國史。左丞許圉師為戶部尚書。九月丙午,宰相劉仁軌、戴至德、張文瓘、郝處俊並兼太子賓客。



 冬十月,析永州營道、江華、唐興三縣置道州。壬午,星孛於角、亢之南,長五尺。十二月丁亥,龜茲王白素稽獻名馬。



 三年春正月戊戌,徙封冀王輪為相王。二月甲戌,移安東都護府於遼東。乙亥,堅昆獻名馬。丁亥,幸汝州之溫湯。三月癸卯,黃門侍郎來恆、中書侍郎薛元超並同中
 書門下三品。甲辰,還東都。閏三月己巳朔,吐蕃入寇鄯、廓、河、芳等四州。乙酉,洛州牧、周王顯為洮州道行軍元帥,領工部尚書劉審禮等十二總管;並州都督、相王輪為涼州道行軍元帥,領左衛將軍契苾何力等軍,以討吐蕃。二王竟不行。戊午,敕制比用白紙,多為蟲蠹,今後尚書省下諸司、州、縣,宜並用黃紙。其承制敕之司,量為卷軸,以備披檢。庚寅,車駕還京。



 夏四月戊申,至自東都。甲寅,中書侍郎李義琰同中書門下三品。戊午,幸九成宮。
 六月癸丑,黃門侍郎高智周同中書門下三品。秋七月,彗起東井,指北河,漸東北,長三丈,掃中臺,指文昌宮,五十八日方滅。八月乙未,吐蕃寇疊州。庚子,以星變,避殿,減膳,放京城系囚,令文武官各上封事言得失。壬寅,置南選使,簡補廣、交、黔等州官吏。青、齊等州海泛溢,又大雨,漂溺居人五千家,遣使賑恤之。



 九月甲子朔,車駕還京。丙申,郇王素節削戶三分之二,於袁州安置。癸丑,於北京置金鄰州。十一月丁卯,敕新造《上元舞》,圓丘、方澤、
 享太廟用之,餘祭則停。壬申,以陳州言鳳凰見於宛丘,改上元三年曰儀鳳元年,大赦。庚寅,吏部尚書李敬玄為中書令。十二月丙申,皇太子賢上所注《後漢書》,賜物三萬段。戊午,遣使分道巡撫:宰相來恆河南道,薛元超河北道,左丞崔知悌等江南道。



 二年春正月乙亥,上躬籍田於東郊。庚辰,京師地震。壬辰,幸司竹園,即日還宮。二月丁巳,工部尚書高藏授遼東都督,封朝鮮郡王,遣歸安東府,安輯高麗餘眾;司農
 卿扶餘隆熊津州都督,封帶方郡王,令往安輯百濟餘眾。仍移安東都護府於新城以統之。



 夏四月,以河南、河北旱,遣使賑給。八月,徙封周王顯為英王,改名哲。乙巳,太白犯軒轅。十二月乙卯,敕關內、河東諸州召募勇敢,以討吐蕃。詔京文武職事官三品已上,每年各舉文武才能堪任將帥牧守者一人。是冬無雪。三年四月丁亥朔,以旱,避正殿,親錄囚徒,悉原之。戊申,大赦,改來年正月一日為通乾。癸丑,涇州獻二小兒,連
 心異體,年四歲。五月壬戌,幸九成宮。以相王輪為洛州牧。秋七月丁巳,宴近臣諸親於咸亨殿。上謂霍王元軌曰:「去冬無雪,今春少雨,自避暑此宮,甘雨頻降,夏麥豐熟,秋稼滋榮。又得敬玄表奏,吐蕃入龍支,張虔勖與之戰,一日兩陣,斬馘極多。又太史奏,七月朔,太陽合虧而不虧。此蓋上天垂祐,宗社降靈,豈虛薄所能致此!又男輪最小,特所留愛,比來與選新婦,多不稱情;近納劉延景女,觀其極有孝行,復是私衷一喜。思與叔等同為此
 歡,各宜盡醉。」上因賦七言詩效柏梁體,侍臣並和。



 九月丁巳,還京師。辛酉,至自九成宮。癸亥,侍中張文瓘卒。丙寅,洮河道行軍大總管中書令李敬玄、左衛大將軍劉審禮等與吐蕃戰於青海之上,王師敗績,審禮被俘。上以蕃寇為患,問計於侍臣中書舍人郭正一等,咸以備邊不深討為上策。十月丙午,徐州刺史、密王元曉薨。閏十月戊寅,熒惑犯鉤鈐。十一月乙未,昏霧四塞,連夜不解。丙申,雨木冰。壬子,黃門侍郎、同中書門下三品來恆
 卒。十二月,詔停明年「通乾」之號,以反語不善故也。



 四年正月辛未,戶部尚書、平恩縣公許圉師卒。己酉,幸東都。庚戌,尚書右僕射、道國公戴至德薨。二月壬戌,吐蕃贊普卒,遣使吊祭之。乙丑,東都饑,官出糙米以救饑人。



 夏四月戊午,熒惑入羽林星。左丞崔知悌為戶部尚書,中書令郝處俊為侍中。五月壬午,盜殺正諫大夫明崇儼。丙戌,皇太子賢監國。戊戌,造紫桂宮於沔池之西。六月辛亥,制大赦天下,改儀鳳四年為調露元年。秋七
 月己卯朔,詔以今年冬至有事嵩岳,禮官學士詳定儀注。



 八月丁巳,侍中郝處俊、左庶子高智周、黃門侍郎崔知溫、給事中劉景先兼修國史。九月壬午,吏部侍郎裴行儉討西突厥,擒其十姓可汗阿史那都支及別帥李遮匐以歸。冬十月,單于大都護府突厥阿史德溫傅及奉職二部相率反叛,立阿史那泥熟匐為可汗,二十四州首領並叛。遣單于大都護長史蕭嗣業,將軍花大智、李景嘉等討之。與突厥戰,為賊所敗。嗣業配流桂州。壬
 子,令將軍曹懷舜率兵往恆州守井陘,崔獻往絳州守龍門,以備突厥。庚申,前詔封嵩山,宜停。癸亥,吐蕃文成公主遣其大臣論塞調傍來告喪,請和親,不許。遣郎將宋令文使吐蕃,會贊普之葬。十一月戊寅朔,左庶子、同三品高智周罷知政事。癸未,以吏部侍郎裴行儉為禮部尚書,賞擒都支、遮匐之功也。甲辰,裴行儉為定襄道大總管,與營州都督周道務等兵十八萬,並西軍程務挺、東軍李文暕等,總三十萬以討突厥。甲寅,臨軒試應
 岳牧舉人。



 二年春正月乙酉,宴諸王、諸司三品已上、諸州都督刺史於洛城南門樓,奏新造《六合還淳》之舞。二月丙午,詔曰:「故符璽郎李延壽撰《正典》一部,辭殫雅正,雖已淪亡,功猶可錄,宜賜其家絹五十疋。」壬子,霍王元軌率文武百僚,請出一月俸料助軍,以討突厥。癸丑,幸汝州溫湯。丁巳,至少室山。戊午,親謁少姨廟。賜故玉清觀道士王遠知謚曰升真先生,贈太中大夫。又幸隱士田游巖所
 居。己未,幸嵩陽觀及啟母廟,並命立碑。又幸逍遙谷道士潘師正所居。甲子,自溫湯還東都。



 三月,裴行儉大破突厥於黑山,擒其首領奉職。偽可汗泥熟匐為其部下所殺,傳首來降。夏四月乙丑,幸紫桂宮。戊辰,黃門侍郎裴炎崔知溫、中書侍郎王德真並同中書門下三品。五月癸未,熒惑犯輿鬼。丁酉,太白經天。秋七月,吐蕃寇河源,屯於良非川。河西鎮撫大使李敬玄與吐蕃將贊婆戰於湟中,官軍敗績。時左武衛將軍黑齒常之力戰,大
 破蕃軍,遂擢為河源軍經略大使;令李敬玄鎮鄯州,為之援。丙申,江王元祥薨。是月,突厥餘眾圍雲州,中郎將程務挺擊破之。八月丁未,自紫桂宮還東都。丁巳,鄯州都督李敬玄左遷衡州刺史。甲子,廢皇太子賢為庶人,幽於別所。乙丑,立英王哲為皇太子。改調露二年為永隆元年,赦天下,大酺三日。太子左庶子、同中書門下三品張大安坐庶人左遷普州刺史。九月,河南、河北諸州大水,遣使賑恤,溺死者官給棺槥,其家賜物七段。



 冬十
 月壬寅,蘇州刺史曹王明封零陵郡王,於黔州安置,坐附庶人賢也。己酉,自東都還京。十一月朔,日有蝕之。洛州饑,減價官糶,以救饑人。



 二年春正月,突厥寇原、慶等州。乙亥,命將軍李知十、王杲等分兵御之。癸巳,遣禮部尚書裴行儉為定襄道大總管,率師討突厥溫傅部落。己亥,詔雍、岐、華、同民戶宜免兩年地稅,河南、河北遭水處一年。上詔雍州長史李義玄曰:「朕思還淳返樸,示天下以質素。如聞游手墮業,
 此類極多,時稍不豐,便致饑饉。其異色綾錦,並花間裙衣等,糜費既廣,俱害女工。天後,我之匹敵,常著七破間裙,豈不知更有靡麗服飾?務遵節儉也。其紫服赤衣,閭閻公然服用。兼商賈富人,厚葬越禮。卿可嚴加捉搦,勿使更然。」二月丙午,皇太子親行釋奠禮。



 三月辛卯,左僕射、同三品劉仁軌兼太子少傅。侍中郝處俊為太子少保,罷知政事。五月丙戌,定襄道總管曹懷舜與突厥史伏念戰於橫水,官軍大敗。懷舜減死,配流嶺南。六月壬
 子,故江王元祥男晫以犯名教,斬於大理寺後園。七月,太平公主出降薛紹,赦京城系囚。閏七月丁未,黃門侍郎裴炎為侍中,黃門侍郎崔知溫、中書侍郎薛元超並為中書令。庚申,上以服餌,令皇太子監國。丙寅,雍州大風害稼,米價騰踴。是月,裴行儉大破突厥史伏念之眾,伏念為程務挺急追,遂執溫傅來降,行儉於是盡平突厥餘黨。行儉執伏念、溫傅,振旅凱旋。



 八月丁卯朔,河南、河北大水,許遭水處往江、淮已南就食。丁亥,戶部尚書
 崔知悌卒。辛卯,改交州為安南都護府。九月丙申,彗星見於天市,長五尺。冬十月丙寅朔,日有蝕之。乙丑,改永隆二年為開耀元年。曲赦定襄軍及緣征突厥官吏兵募等。丙寅,斬阿史那伏念及溫傅等五十四人於都市。丁亥,新羅王金法敏薨,仍以其子政襲位。十一月癸卯,徙庶人賢於巴州。十二月,吐火羅獻金衣一領,上不受。辛未,太子少保、甑山縣公郝處俊薨。



 永淳元年正月乙未朔,以年饑,罷朝會。關內諸府兵,令
 於鄧、綏等州就穀。



 二月癸未,以太子誕皇孫滿月,大赦。改開耀二年為永淳元年,大酺三日。戊午,立皇孫重照為皇太孫,欲開府置僚屬。吏部郎中王方慶曰:「按周禮,有嫡子無嫡孫。漢、魏已來,皇太子在,不立太孫,但封王耳。晉立愍懷太子子彧為太孫,齊立文惠太子子昭業為太孫,便居東宮;而皇太子在而立太孫,未有前例。」上曰:「自我作古,可乎?」曰:「可。」然竟不立府僚。是春,關內旱,日色如赭。四月甲子朔,日有蝕之。丙寅,幸東都。皇太子京
 師留守,命劉仁軌、裴炎、薛元超等輔之。上以穀貴,減扈從兵,士庶從者多殍踣於路。辛未,以裴行儉為金牙道行軍大總管,與將軍閻懷旦等三總管兵分道討十姓突厥阿史那車薄。行儉未行而卒。安西副都護王方翼破車薄、咽面,西域平。戊寅,次澠池之紫桂宮。乙酉,至東都。丁亥,黃門侍郎郭待舉、兵部侍郎岑長倩、中書侍郎郭正一、吏部侍郎魏玄同並同中書門下同承受進止平章事。上謂參知政事崔知溫曰:「待舉等歷任尚淺,且
 令預聞政事,未可即與卿等同名稱。」自是外司四品已下知政事者,遂以平章為名。



 五月壬寅,置東都苑總監。自丙午連日澍雨,洛水溢,壞天津及中橋、立德、弘教、景行諸坊,溺居民千餘家。六月,關中初雨,麥苗澇損,後旱,京兆、岐、隴螟蝗食苗並盡,加以民多疫癘,死者枕藉於路,詔所在官司埋瘞。丁丑,以岐州刺史蘇良嗣為雍州長史。京師人相食,寇盜縱橫。秋七月己亥,造奉天宮於嵩山之陽,仍置嵩陽縣。又於藍田造萬全宮。庚申,零陵
 王明薨。是秋,山東大水,民饑。吐蕃寇柘、松、翼等州。冬十月甲子,京師地震。丙寅,黃門侍郎劉景先同平章事。十二月,南天竺、于闐各獻方物。突厥餘黨阿史那骨篤祿等招合殘眾,據黑沙城,入寇並州北境。



 二年春正月甲午朔,幸奉天宮,遣使祭嵩岳、少室、箕山、具茨等山,西王母、啟母、巢父、許由等祠。二月甲午,洛州長史李仲玄為宗正卿。庚午,突厥寇定州、媯州之境。己卯,左領軍衛大將軍薛仁貴卒。三月庚寅,突厥阿史那
 骨篤祿、阿史德元珍等圍單于都護府。丙午,彗見五車北,二十五日而滅。癸丑,中書令崔知溫卒。夏四月己巳,還東都。甲申,綏州部落稽白鐵餘據城平縣反,命將軍程務挺將兵討之。



 五月庚寅,幸芳桂宮,陰雨,還東都。突厥寇蔚州,殺刺史李思儉,豐州都督崔智辨率師出朝那山掩擊之,為賊所敗,遂寇嵐州。秋七月已丑,封皇孫重福為唐昌郡王。甲辰,相王輪改封豫王,更名旦。己丑,令唐昌郡王重福為京留守,劉仁軌副之。召皇太子至東都。
 己巳,河水溢,壞河陽城,水面高於城內五尺,北至鹽坎,居人廬舍漂沒皆盡,南北並壞。庚戌,熒惑入輿鬼,犯質星。十一月,皇太子來朝。癸亥,幸奉天宮。時天后自封岱之後,勸上封中嶽。每下詔草儀注,即歲饑、邊事警急而止。至是復行封中嶽禮,上疾而止。上苦頭重不可忍,侍醫秦鳴鶴曰:「刺頭微出血,可愈。」天後帷中言曰:「此可斬,欲刺血於人主首耶!」上曰:「吾苦頭重,出血未必不佳。」即刺百會,上曰:「吾眼明矣。」戊戌,命將軍程務挺為單于道
 安撫大使,以招討總管討山賊元珍、骨篤祿、賀魯等。詔皇太子監國,裴炎、劉齊賢、郭正一等於東宮同平章事。丁未,自奉天宮還東都。上疾甚,宰臣已下並不得謁見。十二月己酉,詔改永淳二年為弘道元年。將宣赦書,上欲親御則天門樓,氣逆不能上馬,遂召百姓於殿前宣之。禮畢,上問侍臣曰:「民庶喜否?」曰:「百姓蒙赦,無不感悅。」上曰:「蒼生雖喜,我命危篤。天地神祇若延吾一兩月之命,得還長安,死亦無恨。」是夕,帝崩於真觀殿,時年五十
 六。宣遺詔:「七日而殯,皇太子即位於柩前。園陵制度,務從節儉。軍國大事有不決者,取天后處分。」群臣上謚曰天皇大帝,廟號高宗。文明元年八月庚寅,葬於乾陵。天寶十三載,改謚曰天皇大弘孝皇帝。



 史臣曰:大帝往在籓儲,見稱長者;暨升旒扆,頓異明哉。虛襟似納於觸鱗,下詔無殊於扇暍。既蕩情於帷薄,遂忽怠於基扃。惑麥斛之佞言,中宮被毒;聽趙師之誣說,元舅銜冤。忠良自是脅肩,奸佞於焉得志。卒致盤維盡
 戮,宗社為墟。古所謂一國為一人興,前賢為後愚廢,信矣哉!



 贊曰:藉文鴻業,僅保餘位。封岱禮天,其德不類。伏戎於寢,構堂終墜。自蘊禍胎,邦家殄瘁。



\end{pinyinscope}