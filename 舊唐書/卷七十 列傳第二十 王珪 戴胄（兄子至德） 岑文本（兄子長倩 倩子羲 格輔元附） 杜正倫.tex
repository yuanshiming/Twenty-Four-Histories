\article{卷七十 列傳第二十 王珪 戴胄(兄子至德) 岑文本(兄子長倩 倩子羲 格輔元附) 杜正倫}

\begin{pinyinscope}

 ○王
 珪戴胄兄子至德岑文本兄子長倩倩子羲格輔元附杜正倫



 王珪,字叔玠,太原祁人也。在魏為烏丸氏,曾祖神念,自魏奔梁,復姓王氏。祖僧辯,梁太尉、尚書令。父顗,北齊樂
 陵太守。珪幼孤,性雅澹,少嗜欲,志量沉深,能安於貧賤,體道履正,交不茍合。季叔頗,當時通儒,有人倫之鑒,嘗謂所親曰:「門戶所寄,唯在此兒耳。」開皇末,為奉禮郎。及頗坐漢王諒反事被誅,珪當從坐,遂亡命於南山,積十餘歲。高祖入關,丞相府司錄李綱薦珪貞諒有器識,引為世子府諮議參軍。及東宮建,除太子中舍人;尋轉中允,甚為太子所禮。後以連其陰謀事,流於巂州。建成誅後,太宗素知其才,召拜諫議大夫。貞觀元年,太宗嘗謂
 侍臣曰:「正主御邪臣,不能致理;正臣事邪主,亦不能致理,唯君臣相遇,有同魚水,則海內可安也。昔漢高祖,田舍翁耳。提三尺劍定天下,既而規模弘遠,慶流子孫者,此蓋任得賢臣所致也。朕雖不明,幸諸公數相匡救,冀憑嘉謀,致天下於太平耳。」珪對曰:「臣聞木從繩則正,後從諫則聖。故古者聖主,必有諍臣七人,言而不用,則相繼以死。陛下開聖慮,納芻蕘,臣處不諱之朝,實願罄其狂瞽。」太宗稱善,敕自今後中書門下及三品以上入閣,
 必遣諫官隨之。珪每推誠納忠,多所獻替,太宗顧待益厚,賜爵永寧縣男,遷黃門侍郎,兼太子右庶子。二年,代高士廉為侍中。太宗嘗閑居,與珪宴語,時有美人侍側,本廬江王瑗之姬,瑗敗籍沒入宮,太宗指示之曰:「廬江不道,賊殺其夫而納其室。暴虐之甚,何有不亡者乎!」珪避席曰:「陛下以廬江取此婦人為是耶,為非耶?」太宗曰:「殺人而取其妻,卿乃問朕是非,何也?」對曰:「臣聞於管子曰:『齊桓公之郭,問其父老曰:『郭何故亡?』父老曰:『以其善
 善而惡惡也。』桓公曰:『若子之言,乃賢君也,何至於亡?』父老曰:『不然,郭君善善而不能用,惡惡而不能去,所以亡也。』今此婦人尚在左右,竊以聖心為是之,陛下若以為非,此謂知惡而不去也。」太宗雖不出此美人,而甚重其言。時太常少卿祖孝孫以教宮人聲樂不稱旨,為太宗所讓。珪及溫彥博諫曰:「孝孫妙解音律,非不用心,但恐陛下顧問不得其人,以惑陛下視聽。且孝孫雅士,陛下忽為教女樂而怪之,臣恐天下怪愕。」太宗怒曰:「卿皆我
 之腹心,當進忠獻直,何乃附下罔上,反為孝孫言也!」彥博拜謝,珪獨不拜。曰:「臣本事前宮,罪已當死。陛下矜恕性命,不以不肖,置之樞近,責以忠直。今臣所言,豈是為私?不意陛下忽以疑事誚臣,是陛下負臣,臣不負陛下。」帝默然而罷。翌日,帝謂房玄齡曰:「自古帝王,能納諫者固難矣。昔周武王尚不用伯夷、叔齊,宣王賢主,杜伯猶以無罪見殺,吾夙夜庶幾前聖,恨不能仰及古人。昨責彥博、王珪,朕甚悔之。公等勿以此而不進直言也。」



 時房
 玄齡、李靖、溫彥博、戴胄、魏徵與珪同知國政。後嘗侍宴,太宗謂珪曰:「卿識鑒清通,尤善談論,自房玄齡等,咸宜品藻,又可自量,孰與諸子賢?」對曰:「孜孜奉國,知無不為,臣不如玄齡;才兼文武,出將入相,臣不如李靖;敷奏詳明,出納惟允,臣不如溫彥博;處繁理劇,眾務必舉,臣不如戴胄;以諫諍為心,恥君不及於堯、舜,臣不如魏徵。至如激濁揚清,嫉惡好善,臣於數子,亦有一日之長。」太宗深然其言,群公亦各以為盡己所懷,謂之確論。後進爵
 為郡公。七年,坐漏洩禁中語,左遷同州刺史。明年,召拜禮部尚書。十一年,與諸儒正定《五禮》,書成,賜帛三百段,封一子為縣男。是歲,兼魏王師。既而上問黃門侍郎韋挺曰:「王珪為魏王泰師,與其相見,若為禮節?」挺對曰:「見師之禮,拜答如禮。」王問珪以忠孝,珪答曰:「陛下,王之君也,事君思盡忠;陛下,王之父也,事父思盡孝。忠孝之道,可以立身,可以成名,當年可以享天祐,餘芳可以垂後葉。」王曰:「忠孝之道,已聞教矣,願聞所習。」珪答曰:「漢東平
 王蒼云:『為善最樂。』」上謂侍臣曰:「古來帝子,生於宮闥,及其成人,無不驕逸,是以傾覆相踵,少能自濟。我今嚴教子弟,欲令皆得安全。王珪我久驅使,是所諳悉,以其意存忠孝,選為子師。爾宜語泰:『汝之待珪,如事我也,可以無過。』」泰每為之先拜,珪亦以師道自居,物議善之。時珪子敬直尚南平公主。禮有婦見舅姑之儀,自近代公主出降,此禮皆廢。珪曰:「今主上欽明,動循法制。吾受公主謁見,豈為身榮,所以成國家之美耳。」遂與其妻就席而
 坐,令公主親執笄行盥饋之道,禮成而退。是後公主下降有舅姑者,皆備婦禮,自珪始也。珪少時貧寒,人或遺之,初不辭謝;及貴,皆厚報之,雖其人已亡,必賑贍其妻子。事寡嫂盡禮,撫孤侄恩義極隆,宗姻困匱者,亦多所周恤。珪通貴漸久,而不營私廟,四時蒸嘗,猶祭於寢。坐為法司所劾,太宗優容,弗之譴也,因為立廟,以愧其心。珪既儉不中禮,時論以是少之。十三年,遇疾,敕公主就第省視,又遣民部尚書唐儉增損藥膳。尋卒,年六十九。
 太宗素服舉哀於別次,悼惜久之。詔魏王泰率百官親往臨哭,贈吏部尚書,謚曰懿。



 長子崇基,襲爵,官至主爵郎中。少子敬直,以尚主拜附馬都尉,坐與太子承乾交結,徙於嶺外。崇基孫旭,開元初,為左司郎中,兼侍御史。時光祿少卿盧崇道犯罪配流嶺南,逃歸匿於東都,為讎家所發。玄宗令旭究其獄,旭欲擅其威權,因捕系崇道親黨數十人,皆極其楚毒,然後結成其罪,崇道及其三子並坐死,親友皆決杖流貶。時得罪多是知名之士,
 四海冤之。旭又與御史大夫李傑不協,遞相糾訐,傑竟坐左遷衢州刺史。旭既得志,擅行威福,由是朝廷畏而鄙之。俄以贓罪黜為龍川尉,憤恚而死,甚為時之所快。



 戴胄,字玄胤,相州安陽人也。性貞正,有幹局。明習律令,尤曉文簿。隋大業末,為門下錄事,納言蘇威、黃門侍郎裴矩甚禮之。越王侗以為給事郎。王世充將篡侗位,胄言於世充曰:「君臣之分,情均父子,理須同其休戚,勖以終始。明公以文武之才,當社稷之寄,與存與亡,在於今
 日。所願推誠王室,擬跡伊、周,使國有泰山之安,家傳代祿之盛,則率土之濱,莫不幸甚。」世充詭辭稱善,勞而遣之。世充後逼越王加其九錫,胄又抗言切諫。世充不納,由是出為鄭州長史,令與兄子行本鎮武牢。太宗克武牢而得之,引為秦府士曹參軍。及即位,除兵部郎中,封武昌縣男。



 貞觀元年,遷大理少卿。時吏部尚書長孫無忌嘗被召,不解佩刀入東上閣。尚書右僕射封德彞議以監門校尉不覺,罪當死;無忌誤帶入,罰銅二十斤。上
 從之。胄駁曰:「校尉不覺與無忌帶入,同為誤耳。臣子之於尊極,不得稱誤,準律云:『供御湯藥、飲食、舟船,誤不知者,皆死。』陛下若錄其功,非憲司所決;若當據法,罰銅未為得衷。」太宗曰:「法者,非朕一人之法,乃天下之法也。何得以無忌國之親戚,便欲阿之?」更令定議。德彞執議如初,太宗將從其議,胄又曰:「校尉緣無忌以致罪,於法當輕。若論其誤,則為情一也,而生死頓殊,敢以固請。」上嘉之,竟免校尉之死。於時朝廷盛開選舉,或有詐偽資廕
 者,帝令其自首,不首者罪至於死。俄有詐偽者事洩,胄據法斷流以奏之。帝曰:「朕下敕不首者死,今斷從流,是示天下以不信。卿欲賣獄乎?」胄曰:「陛下當即殺之,非臣所及。既付所司,臣不敢虧法。」帝曰:「卿自守法,而令我失信邪?」胄曰:「法者,國家所以布大信於天下;言者,當時喜怒之所發耳。陛下發一朝之忿而許殺之,既知不可而置之於法,此乃忍小忿而存大信也。若順忿違信,臣竊為陛下惜之。」帝曰:「法有所失,公能正之,朕何憂也!」胄前
 後犯顏執法多此類。所論刑獄,皆事無冤濫,隨方指手適,言如泉湧。其年,轉尚書右丞,尋遷左丞。先是,每歲水旱,皆以正倉出給,無倉之處,就食他州,百姓多致饑乏。二年,胄上言:「水旱兇災,前聖之所不免。國無九年儲蓄,禮經之所明誡。今喪亂已後,戶口凋殘,每歲納租,未實倉稟。隨即出給,才供當年,若有兇災,將何賑恤?故隋開皇立制,天下之人,節級輸粟,名為社倉,終文皇代,得無饑饉。及大業中年,國用不足,並取社倉之物以充官費,故
 至末途,無以支給。自王公已下,爰及眾庶,計所墾田稼穡頃畝,每至秋熟,準其苗以理勸課,盡令出粟。稻麥之鄉,亦同此稅,各納所在,立為義倉。」太宗從其議。以其家貧,齎錢十萬。



 時尚書左僕射蕭瑀免官,僕射封德彞又卒,太宗謂胄曰:「尚書省天下綱維,百司所稟,若一事有失,天下必有受其弊者。今以令、僕系之於卿,當稱朕所望也。」胄性明敏,達於從政,處斷明速。議者以為左右丞稱職,武德已來,一人而已。又領諫議大夫,令與魏徵更
 日供奉。三年,進拜民部尚書,兼檢校太子左庶子。先是,右僕射杜如晦專掌選舉,臨終請以選事委胄,由是詔令兼攝吏部尚書,其民部、庶子、諫議並如故。胄雖有幹局,而無學術。居吏部,抑文雅而獎法吏,甚為時論所譏。四年,罷吏部尚書,以本官參預朝政,尋進爵為郡公。五年,太宗將修復洛陽宮,胄上表諫曰:



 陛下當百王之弊,屬暴隋之後,拯餘燼於塗炭,救遺黎於倒懸。遠至邇安,率土清謐,大功大德,豈臣之所稱贊。臣誠小人,才識非
 遠,唯知耳目之近,不達長久之策,敢竭區區之誠,論臣職司之事。比見關中、河外,盡置軍團,富室強丁,並從戎旅。重以九成作役,餘丁向盡,去京二千里內,先配司農將作。假有遺餘,勢何足紀?亂離甫爾,戶口單弱,一人就役,舉家便廢。入軍者督其戎仗,從役者責其餱糧,盡室經營,多不能濟。以臣愚慮,恐致怨嗟。七月已來,霖潦過度,河南、河北,厥田洿下,時豐歲稔,猶未可量。加以軍國所須,皆資府庫,布絹所出,歲過百萬。丁既役盡,賦調不
 減,費用不止,帑藏其虛。且洛陽宮殿,足蔽風雨,數年功畢,亦謂非晚。若頓修營,恐傷勞擾。



 太宗甚嘉之,因謂侍臣曰:「戴胄於我無骨肉之親,但以忠直勵行,情深體國,事有機要,無不以聞。所進官爵,以酬厥誠耳。」七年卒,太宗為之舉哀,廢朝三日。贈尚書右僕射,追封道國公,謚曰忠,詔虞世南撰為碑文。又以胄宅宇弊陋,祭享無所,令有司特為造廟。房玄齡、魏徵並美胄才用,俱與之親善,及胄卒後,嘗見其游處之地,數為之流涕。胄無子,以
 兄子至德為後。



 至德,乾封中累遷西臺侍郎、同東西臺三品。尋轉戶部尚書,依舊知政事。父子十數年間相繼為尚書,預知國政,時以為榮。咸亨中,高宗為飛白書以賜侍臣,賜至德曰「泛洪源,俟舟楫」;賜郝處俊曰「飛九霄,假六翮」;賜李敬玄曰「資啟沃,罄丹誠」;又賜中書侍郎崔知悌曰「竭忠節,贊皇猷」,其辭皆有興比。俄遷尚書右僕射。時劉仁軌為左僕射,每遇申訴冤滯者,輒美言許之;而至德先據理難詰,未嘗與奪,若有理者,密為奏之,終
 不顯己之斷決,由是時譽歸於仁軌。或以問至德,答曰:「夫慶賞刑罪,人主之權柄,凡為人臣,豈得與人主爭權柄哉!」其慎密如此。後高宗知而深嘆美之。儀鳳四年薨,輟朝三日,使百官以次赴宅哭之,贈開府儀同三司、並州大都督,謚曰恭。



 岑文本,字景仁,南陽棘陽人。祖善方,仕蕭察吏部尚書。父之象,隋末為邯鄲令,嘗被人所訟,理不得申。文本性沈敏,有姿儀,博考經史,多所貫綜,美談論,善屬文。時年
 十四,詣司隸稱冤,辭情慨切,召對明辯,眾頗異之。試令作《蓮花賦》,下筆便成,屬意甚佳,合臺莫不嘆賞。其父冤雪,由是知名。其後,郡舉秀才,以時亂不應。蕭銑僭號於荊州,召署中書侍郎,專典文翰。及河間王孝恭定荊州,軍中將士咸欲大掠,文本進說孝恭曰:「自隋室無道,群雄鼎沸,四海延頸以望真主。今蕭氏君臣、江陵父老,決計歸降者,實望去危就安耳。王必欲縱兵虜掠,誠非鄙州來蘇之意,亦恐江、嶺以南,向化之心沮矣。」孝恭稱善,
 遂止之。署文本荊州別駕。孝恭進擊輔公祏,召典軍書,復署行臺考功郎中。貞觀元年,除秘書郎,兼直中書省。遇太宗行藉田之禮,文本上《藉田頌》。及元日臨軒宴百僚,文本復上《三元頌》,其辭甚美。文本才名既著,李靖復稱薦之,擢拜中書舍人,漸蒙親顧。初,武德中詔誥及軍國大事,文皆出於顏師古。至是,文本所草詔誥。或眾務繁湊,即命書僮六七人隨口並寫,須臾悉成,亦殆盡其妙。時中書侍郎顏師古以譴免職,頃之,溫彥博奏曰:「師
 古諳練時事,長於文法,時無及者,冀蒙復用。」太宗曰:「我自舉一人,公勿憂也。」於是以文本為中書侍郎,專典機密。又先與令狐德棻撰《周史》,其史論多出於文本。至十年史成,封江陵縣子。十一年,從至洛陽宮,會谷、洛泛溢,文本上封事曰:



 臣聞創撥亂之業,其功既難;守已成之基,其道不易。故居安思危,所以定其業也;有始有卒,所以隆其基也。今雖億兆乂安,方隅寧謐,既承喪亂之後,又接凋弊之餘,戶口減損尚多,田疇墾闢猶少。覆燾之
 恩著矣,而瘡痍未復;德教之風被矣,而資產屢空。是以古人譬之種樹,年祀綿遠,則枝葉扶疏;若種之日淺,根本未固,雖壅之以黑墳,暖之以春日,一人搖之,必致枯槁。今之百姓,頗類於此。常加含養,則日就滋息;暫有征役,則隨而凋耗。凋耗既甚,則人不卿生;人不卿生,則怨氣充塞;怨氣充塞,則離叛之心生矣。故帝舜曰:「可愛非君,可畏非人。」孔安國曰:「人以君為命,故可愛;君失道,人叛之,故可畏。」仲尼曰:「君猶舟也,人猶水也;水所以載舟,
 亦所以覆舟。」是以古之哲王,雖休勿休,日慎一日者,良為此也。伏惟陛下覽古今之事,察安危之機,上以社稷為重,下以億兆為念。明選舉,慎賞罰,進賢才,退不肖。聞過即改,從諫如流。為善在於不疑,出令期於必信。頤神養性,省畋游之娛;去奢從儉,減工役之費。務靜方內而不求闢土;載橐弓矢而無忘武備。凡此數者,雖為國之常道,陛下之所常行,臣之愚心,唯願陛下思之而不倦,行之而不怠。則至道之美,與三、五比隆;億載之祚,隨天
 地長久。雖使桑穀為妖,龍蛇作孽,雉雊於鼎耳,石言於晉地,猶當轉禍為福,變咎為祥。況水雨之患,陰陽常理,豈可謂之天譴而系聖心哉?臣聞古人有言:「農夫勞而君子養焉,愚者言而智者擇焉。」輒陳狂瞽,伏待斧鉞。



 是時魏王泰寵冠諸王,盛修第宅,文本以為侈不可長,上疏盛陳節儉之義,言泰宜有抑損,太宗並嘉之,賜帛三百段。十七年,加銀青光祿大夫。



 文本自以出自書生,每懷捴損。平生故人,雖微賤必與之抗禮。居處卑陋,室無
 茵褥帷帳之飾。事母以孝聞,撫弟侄恩義甚篤。太宗每言其「弘厚忠謹,吾親之信之。」是時,新立晉王為皇太子,名士多兼領宮官,太宗欲令文本兼攝。文本再拜曰:「臣以庸才,久逾涯分,守此一職,猶懼滿盈,豈宜更忝春坊,以速時謗。臣請一心以事陛下,不願更希東宮恩澤。」太宗乃止。仍令五日一參東宮,皇太子執賓友之禮,與之答拜。其見待如此。俄拜中書令,歸家有憂色,其母怪而問之,文本曰:「非勛非舊,濫荷寵榮,責重位高,所以憂懼。」
 親賓有來慶賀,輒曰:「今受吊,不受賀也。」又有勸其營產業者,文本嘆曰:「南方一布衣,徒步入關,疇昔之望,不過秘書郎、一縣令耳。而無汗馬之勞,徒以文墨致位中書令,斯亦極矣。荷俸祿之重,為懼已多,何得更言產業乎?」言者嘆息而退。



 文本既久在樞揆,當塗任事,賞錫稠疊,凡有財物出入,皆委季弟文昭,一無所問。文昭時任校書郎,多與時人游款,太宗聞而不悅,嘗從容謂文本曰:「卿弟過多交結,恐累卿,朕將出之為外官,如何?」文本泣
 曰:「臣弟少孤,老母特所鐘念,不欲信宿離於左右。若今外出,母必憂悴,儻無此弟,亦無老母也。」歔欷嗚咽,太宗愍其意而止。唯召見文昭,嚴加誡約,亦卒無愆過。及將伐遼,凡所籌度,一皆委之。文本受委既深,神情頓竭,言辭舉措,頗異平常。太宗見而憂之,謂左右曰:「文本今與我同行,恐不與我同返。」及至幽州,遇暴疾,太宗親自臨視,撫之流涕。尋卒,年五十一。其夕,太宗聞嚴鼓之聲,曰:「文本殞逝,情深惻怛。今宵夜警,所不忍聞。」命停之。贈侍
 中、廣州都督,謚曰憲,賜東園秘器,陪葬昭陵。有集六十卷行於代。



 文本兄文叔。文叔子長倩,少為文本所鞠,同於己子。永淳中,累轉兵部侍郎、同中書門下平章事。垂拱初,自夏官尚書遷內史,知夏官事,俄拜文昌右相,封鄧國公。則天初革命,尤好符瑞,長倩懼罪,頗有陳奏,又上疏請改皇嗣姓為武氏,以為周室儲貳,則天許之,實封五百戶。天授二年,加特進、輔國大將軍。其年,鳳閣舍人張嘉福與洛州人王慶之等列名上表,請立武承嗣
 為皇太子。長倩以皇嗣在東宮,不可更立承嗣,與地官尚書格輔元竟不署名,仍奏請切責上書者。由是大忤諸武意,乃斥令西征吐蕃,充武威道行軍大總管。中路召還,下制獄,被誅,仍發掘其父祖墳墓。來俊臣又脅迫長倩子靈源,令誣納言歐陽通及格輔元等數十人,皆陷以同反之罪,並誅死。



 長倩子羲,長安中為廣武令,有能名。則天嘗令宰相各舉堪為員外郎者,鳳閣侍郎韋嗣立薦羲,且奏曰:「恨其從父長倩犯逆為累。」則天曰:「茍
 有材幹,何恨微累?」遂拜天官員外郎。由是緣坐近親,相次入省,登封令劉守悌為司門員外郎,渭南令裴惓為地官員外郎。先是,羲為金壇令,守悌及惓稱為清德。羲以文吏著名,俱為巡察使所薦,皆授畿縣令,又同為尚書郎,悉有美譽。守悌後至陜州刺史,惓至杭州刺史。羲,神龍初為中書舍人。時武三思用事,侍中敬暉欲上表請削諸武之為王者,募為疏者。眾畏三思,皆辭托不敢為之,羲便操筆,辭甚切直。由是忤三思意,轉秘書少監,
 再遷吏部侍郎。時吏部侍郎崔湜、太常少卿鄭愔、大理少卿李元恭分掌選事,皆以贓貨聞,羲最守正,時議美之。尋加銀青光祿大夫、右散騎常侍、同中書門下三品。睿宗即位,出為陜州刺史。復歷刑部、戶部二尚書,門下三品,監修國史,刪定格令,仍修《氏族錄》。初,中宗時,侍御史冉祖雍誣奏睿宗及太平公主與節愍太子連謀,請加推究,羲與中書侍郎蕭至忠密申保護。及羲監修《中宗實錄》,自書其事,睿宗覽而大加賞嘆,賜物三百段、良
 馬一匹,仍下制書褒美之。時羲兄獻為國子司業,弟翔為陜州刺史,休為商州刺史,從族兄弟子侄,因羲引用登清要者數十人。羲嘆曰:「物極則返,可以懼矣!」然竟不能有所抑退。尋遷侍中。先天元年,坐預太平公主謀逆伏誅,籍沒其家。



 格輔元者,汴州浚儀人也。伯父德仁,隋剡縣丞,與同郡人齊王文學王孝逸、文林郎繁師玄、羅川郡戶曹靖君亮、司隸從事鄭祖咸、宣城縣長鄭師善、王世充中書舍
 人李行簡、處士盧協等八人,以辭學擅名,當時號為「陳留八俊」。輔元弱冠舉明經,歷遷御史大夫、地官尚書、同鳳閣鸞臺平章事。初,張嘉福等請立武承嗣也,則天以問輔元,固稱不可,遂為承嗣所譖而死,海內冤之。輔元兄希元,高宗時洛州司法參軍,章懷太子召令與洗馬劉訥言等注解範曄《後漢書》,行於代。先輔元卒。



 杜正倫,相州洹水人也。隋仁壽中,與兄正玄、正藏俱以秀才擢第。隋代舉秀才止十餘人,正倫一家有三秀才,
 甚為當時稱美。正倫善屬文,深明釋典。仕隋為羽騎尉。武德中,歷遷齊州總管府錄事參軍。太宗聞其名,令直秦府文學館。貞觀元年,尚書右丞魏徵表薦正倫,以為古今難匹,遂擢授兵部員外郎。太宗謂曰:「朕今令舉行能之人,非朕獨私於行能者,以其能益於百姓也。朕於宗親及以勛舊無行能者,終不任之。以卿忠直,朕今舉卿,卿宜勉稱所舉。」二年,拜給事中,兼知起居注。太宗嘗謂侍臣曰:「朕每日坐朝,欲出一言,即思此言於百姓有
 利益否,所以不能多言。」正倫進曰:「君舉必書,言存左右史。臣職當修起居注,不敢不盡愚直。陛下若一言乖於道理,則千載累於聖德,非直當今損於百姓,願陛下慎之。」太宗大悅,賜絹二百段。



 四年,累遷中書侍郎。六年,正倫與御史大夫韋挺、秘書少監虞世南、著作郎姚思廉等咸上封事稱旨,太宗為之設宴,因謂曰:「朕歷觀自古人臣立忠之事,若值明王,便得盡誠規諫,至如龍逢、比干,竟不免孥戮。為君不易,為臣極難。我又聞龍可擾而
 馴,然喉下有逆鱗,觸之則殺人。人主亦有逆鱗,卿等遂不避犯觸,各進封事。常能如此,朕豈慮有危亡哉!我思卿等此意,豈能暫忘?故聊設宴樂也。」仍並賜帛有差。尋加散騎常侍,行太子右庶子,兼崇賢館學士。太宗謂曰:「國之儲副,自古所重,必擇善人為之輔佐。今太子年在幼沖,志意未定,朕若朝夕見之,可得隨事誡約。今既委以監國,不在目前,知卿志懷貞愨,能敦直道,故輒輟卿於朕,以匡太子,宜知委任輕重也。」十年,復授中書侍郎,
 賜爵南陽縣侯,仍兼太子左庶子。正倫出入兩宮,參典機密,甚以干理稱。時太子承乾有足疾,不能朝謁,好暱近群小。太宗謂正倫曰:「我兒疾病,乃可事也。但全無令譽,不聞愛賢好善,私所引接,多是小人,卿可察之。若教示不得,須來告我。」正倫數諫不納,乃以太宗語告之,承乾抗表聞奏。太宗謂正倫曰:「何故漏洩我語?」對曰:「開導不入,故以陛下語嚇之,冀其有懼,或當反善。」帝怒,出為穀州刺史,又左授交州都督。後承乾構逆,事與侯君集
 相連,稱遣君集將金帶遺正倫,由是配流驩州。顯慶元年,累授黃門侍郎,兼崇賢館學士,尋同中書門下三品。二年,兼度支尚書,仍依舊知政事。俄拜中書令,兼太子賓客、弘文館學士,進封襄陽縣公。三年,坐與中書令李義府不協,出為橫州刺史,仍削其封邑。尋卒。有集十卷行於代。



 史臣曰:王珪履正不回,忠讜無比,君臣時命,胥會於茲。《易》曰:「自天祐之,吉無不利。」叔玠有焉。戴胄兩朝仕官,一
 乃心力,刑無僭濫,事有箴規。雖學術不能求備,而匡益自可濟時,亦所謂巧於任大矣。文本文傾江海,忠貫雪霜,申慈父之冤,匡明主之業,及委繁劇,俄致暴終。《書》曰:「小心翼翼,昭事上帝。」所謂憂能傷人,不復永年矣。洎羲而下,登清要者數十人。積善之道,焉可忽諸?正倫以能文被舉,以直道見委,參典機密,出入兩宮,斯謂得時。然被承乾金帶之譏,孰與夫薏苡之謗,士大夫慎之。



 贊曰:五靈嘉瑞,出系汙隆。人中麟鳳,王、戴諸公。動必由
 禮,言皆匡躬。獻規納諫,貞觀之風。



\end{pinyinscope}