\article{卷六十四 列傳第十四 高祖二十二子}

\begin{pinyinscope}

 ○隱太子建成衛王玄霸巢王元吉楚王智云荊王元景漢王元昌酆王元亨周王元方徐王元禮
 韓王元嘉彭王元則鄭王元懿霍王元軌虢王鳳道王元慶鄧王元裕舒王元名魯王靈夔江王元祥密王元曉滕王元嬰



 高祖二十二男:太穆皇后生隱太子建成及太宗、衛王玄霸、巢王元吉,萬貴妃生楚王智云,尹德妃生酆王元亨,莫嬪生荊王元景,孫嬪生漢王元昌,宇文昭儀生韓王元嘉、魯王靈夔,崔嬪生鄧王元裕,楊嬪生江王元祥,
 小楊嬪生舒王元名,郭婕妤生徐王元禮,劉婕妤生道王元慶,楊美人生虢王鳳,張美人生霍王元軌,張寶林生鄭王元懿,柳寶林生滕王元嬰,王才人生彭王元則,魯才人生密王元曉,張氏生周王元方。



 隱太子建成,高祖長子也。大業末,高祖捕賊汾、晉,建成攜家屬寄於河東。義旗初建,遣使密召之,建成與巢王元吉間行赴太原。建成至,高祖大喜,拜左領軍大都督,封隴西郡公。引兵略西河郡,從平長安。義寧元年冬,隋
 恭帝拜唐國世子,開府,置僚屬。二年,授撫軍大將軍、東討元帥,將兵十萬徇洛陽。及還,恭帝授尚書令。武德元年,立為皇太子。二年,司竹群盜祝山海有眾一千,自稱護鄉公,詔建成率將軍桑顯和進擊山海,平之。時涼州人安興貴殺賊帥李軌,以眾來降,令建成往原州應接之。時甚暑,而馳獵無度,士卒不堪其勞,逃者過半。高祖憂其不閑政術,每令習時事,自非軍國大務,悉委決之。又遣禮部尚書李綱、民部尚書鄭善果俱為宮官,與參
 謀議。四年,稽胡酋帥劉GC成擁部落數萬人為邊害,又詔建成率師討之。軍次鄜州,與GC成軍遇,擊,大破之,斬首數百級,虜獲千餘人。建成設詐放其渠帥數十人,並授官爵,令還本所招慰群胡,GC成與胡中大帥亦請降。建成以胡兵尚眾,恐有變,將盡殺之。乃揚言增置州縣,須有城邑,悉課群胡執板築之具,會築城所,陰勒兵士,皆執之。GC成聞有變,奔於梁師都。竟誅降胡六千餘人。時太宗功業日盛,高祖私許立為太子,建成密知之,乃
 與齊王元吉潛謀作亂。及劉黑闥重反,王珪、魏徵謂建成曰:「殿下但以地居嫡長,爰踐元良,功績既無可稱,仁聲又未遐布。而秦王勛業克隆,威震四海,人心所向,殿下何以自安?今黑闥率破亡之餘,眾不盈萬,加以糧運限絕,瘡痍未瘳,若大軍一臨,可不戰而擒也。願請討之,且以立功,深自封植,因結山東英俊。」建成從其計,遂請討劉黑闥,擒之而旋。



 時高祖晚生諸王,諸母擅寵椒房,親戚並分事宮府,競求恩惠。太宗每總戎律,惟以撫接
 才賢為務,至於參請妃媛,素所不行。初平洛陽,高祖遣貴妃等馳往東都選閱宮人及府庫珍物,因私有求索,兼為親族請官。太宗以財簿先已封奏,官爵皆酬有功,並不允許,因此銜恨彌切。時太宗為陜東道行臺,詔於管內得專處分。淮安王神通有功,太宗乃給田數十頃。後婕妤張氏之父令婕妤私奏以乞其地,高祖手詔賜焉。神道以教給在前,遂不肯與。婕妤矯奏曰:「敕賜妾父地,秦王奪之以與神通。」高祖大怒,攘袂責太宗曰:「我詔
 敕不行,爾之教命,州縣即受。」他日,高祖呼太宗小名謂裴寂等:「此兒典兵既久,在外專制,為讀書漢所教,非復我昔日子也。」又德妃之父尹阿鼠所為橫恣,秦王府屬杜如晦行經其門,阿鼠家僮數人牽如晦墜馬毆擊之,罵云:「汝是何人,敢經我門而不下馬!」阿鼠或慮上聞,乃令德妃奏言:「秦王左右兇暴,凌轢妾父。」高祖又怒謂太宗曰:「爾之左右,欺我妃嬪之家一至於此,況凡人百姓乎!」太宗深自辯明,卒不被納。妃嬪等因奏言:「至尊萬歲
 後,秦王得志,母子定無孑遺。」因悲泣哽咽。又云:「東宮慈厚,必能養育妾母子。」高祖惻愴久之。自是於太宗恩禮漸薄,廢立之心亦以此定,建成、元吉轉蒙恩寵。



 自武德初,高祖令太宗居西宮之承乾殿,元吉居武德殿後院,與上臺、東宮晝夜並通,更無限隔。皇太子及二王出入上臺,皆乘馬攜弓刀雜用之物,相遇則如家人之禮。由是皇太子令及秦、齊二王教與詔敕並行,百姓惶惑,莫知準的。建成、元吉又外結小人,內連嬖幸,高祖所寵張
 婕妤、尹德妃皆與之淫亂。復與諸公主及六宮親戚驕恣縱橫,並兼田宅,侵奪犬馬。同惡相濟,掩蔽聰明,茍行己志,惟以甘言諛辭承候顏色。建成乃私召四方驍勇,並募長安惡少年二千餘人,畜為宮甲,分屯左、右長林門,號為長林兵。及高祖幸仁智宮,留建成居守,建成先令慶州總管楊文乾募健兒送京師,欲以為變。又遣郎將爾硃煥、校尉橋公山齎甲以賜文乾,令起兵共相應接。公山、煥等行至豳鄉,懼罪馳告其事。高祖托以他事,
 手詔追建成詣行在所。既至,高祖大怒,建成叩頭謝罪,奮身自投於地,幾至於絕。其夜,置之幕中,令殿中監陳萬福防禦,而文干遂舉兵反。高祖馳使召太宗以謀之,太宗曰:「文干小豎狂悖,起兵州府,官司已應擒剿。縱其假息時刻,但須遣一將耳。」高祖曰:「文幹事連建成,恐應之者眾,汝宜自行,還,立汝為太子。吾不能效隋文帝誅殺骨肉,廢建成封作蜀王,地既僻小易制。若不能事汝,亦易取耳。」太宗既行,元吉及四妃更為建成內請,封倫
 又外為游說,高祖意便頓改,遂寢不行,復令建成還京居守。惟責以兄弟不能相容,歸罪於中允王珪、左衛率韋挺及天策兵曹杜淹等,並流之巂州。後又與元吉謀行鴆毒,引太宗入宮夜宴,既而太宗心中暴痛,吐血數升,淮安王神通狼狽扶還西宮。高祖幸第問疾,因敕建成:「秦王素不能飲,更勿夜聚。」乃謂太宗曰:「發跡晉陽,本是汝計;克平宇內,是汝大功。欲升儲位,汝固讓不受,以成汝美志。建成自居東宮,多歷年所,今復不忍奪之。
 觀汝兄弟是不和,同在京邑,必有忿競。汝還行臺,居於洛陽,自陜已東,悉宜主之。仍令汝建天子旌旗,如梁孝王故事。」太宗泣而奏曰:「今日之授,實非所願,不能遠離膝下。」言訖嗚咽,悲不自勝。高祖曰:「昔陸賈漢臣,尚有遞過之事,況吾四方之主,天下為家。東西兩宮,途路咫尺,憶汝即往,無勞悲也。」及將行,建成、元吉相與謀曰:「秦王今往洛陽,既得土地甲兵,必為後患。留在京師,制之一匹夫耳。」密令數人上封事曰:「秦王左右多是東人,聞往洛
 陽,非常欣躍,觀其情狀,自今一去,不作來意。」高祖於是遂停。是後,日夜陰與元吉連結後宮,譖訴愈切,高祖惑之。太宗懼,不知所為。李靖、李勣等數言:「大王以功高被疑,靖等請申犬馬之力。」封倫亦潛勸太宗圖之,並不許。倫反言於高祖曰:「秦王恃有大勛,不服居太子之下。若不立之,願早為之所。」又說建成作亂,曰:「夫為四海者,不顧其親。漢高乞羹,此之謂矣。」



 九年,突厥犯邊,詔元吉率師拒之,元吉因兵集,將與建成克期舉事。長孫無忌、房
 玄齡、杜如晦、尉遲敬德、侯君集等日夜固爭曰:「事急矣!若不行權道,社稷必危。周公聖人,豈無情於骨肉?為存社稷,大義滅親。今大王臨機不斷,坐受屠戮,於義何成?若不見聽,無忌等將竄身草澤,不得居王左右。」太宗然其計。六月三日,密奏建成、元吉淫亂後宮,因自陳曰:「臣於兄弟無絲毫所負,今欲殺臣,似為世充、建德報仇。臣今枉死,永違君親,魂歸地下,實亦恥見諸賊。」高祖省之愕然,報曰:「明日當勘問,汝宜早參。」四日,太宗將左右九
 人至玄武門自衛。高祖已召裴寂、蕭瑀、陳叔達、封倫、宇文士及、竇誕、顏師古等,欲令窮覆其事。建成、元吉行至臨湖殿,覺變,即回馬,將東歸宮府。太宗隨而呼之,元吉馬上張弓,再三不彀。太宗乃射之,建成應弦而斃,元吉中流矢而走,尉遲敬德殺之。俄而東宮及齊府精兵二千人結陣馳攻玄武門,守門兵仗拒之,不得入,良久接戰,流矢及於內殿。太宗左右數百騎來赴難,建成等兵遂敗散。高祖大驚,謂裴寂等曰:「今日之事如何?」蕭瑀、陳
 叔達進曰:「臣聞內外無限,父子不親,當斷不斷,反受其亂。建成、元吉,義旗草創之際,並不預謀;建立已來,又無功德,常自懷憂,相濟為惡,釁起蕭墻,遂有今日之事。秦王功蓋天下,率土歸心,若處以元良,委之國務,陛下如釋重負,蒼生自然乂安。」高祖曰:「善!此亦吾之夙志也。」乃命召太宗而撫之曰:「近日已來,幾有投杼之惑。」太宗哀號久之。建成死時年三十八。長子太原王承宗早卒。次子安陸王承道、河東王承德、武安王承訓、汝南王承明、
 鉅鹿王承義並坐誅。太宗即位,追封建成為息王,謚曰隱,以禮改葬。葬日,太宗於宜秋門哭之甚哀,仍以皇子趙王福為建成嗣。十六年五月,又追贈皇太子,謚仍依舊。



 衛王玄霸,高祖第三子也。早薨無子。武德元年,追贈衛王,謚曰懷。四年,封太宗子泰為宜都王以奉其祀,以禮改葬,太子以下送於郭外。泰後徙封於越,又以宗室贈西平王瓊之子保定為嗣。貞觀五年薨,無子,國除。



 巢王元吉,高祖第四子也。義師起,授太原郡守,封姑臧郡公。尋進封齊國公,授十五郡諸軍事、鎮北大將軍,留鎮太原,許以便宜行事。武德元年,進爵為王,授並州總管。二年,劉武周南侵汾、晉,詔遣右衛將軍宇文歆助元吉守並州。元吉性好畋獵,載網罟三十餘兩,嘗言「我寧三日不食,不能一日不獵」,又縱其左右攘奪百姓。歆頻諫不納,乃上表曰:「王在州之日,多出微行,常共竇誕游獵,蹂踐穀稼,放縱親暱,公行攘奪,境內六畜,因之殆盡。
 當衢而射,觀人避箭以為笑樂。分遣左右,戲為攻戰,至相擊刺毀傷至死。夜開府門,宣淫他室。百姓怨毒,各懷憤嘆。以此守城,安能自保!」元吉竟坐免。又諷父老詣闕請之,尋令復職。時劉武周率五千騎至黃蛇嶺,元吉遣車騎將軍張達以步卒百人先嘗之。達以步卒少,固請不行。元吉強遣之,至則盡沒於賊。達憤怒,因引武周攻陷榆次,進逼並州。元吉大懼,紿其司馬劉德威曰:「卿以老弱守城,吾以強兵出戰。」因夜出兵,攜其妻妾棄軍奔
 還京師,並州遂陷。高祖怒甚,謂禮部尚書李綱曰:「元吉幼小,未習時事,故遣竇誕、宇文歆輔之。強兵數萬,食支十年,起義興運之基,一朝而棄。宇文歆首畫此計,我當斬之。」綱曰:「賴歆令陛下不失愛子,臣以為有功。」高祖問其故,綱封曰:「罪由竇誕不能規諷,致令軍人怨憤。又齊王年少,肆行驕逸,放縱左右,侵漁百姓。誕曾無諫止,乃隨順掩藏,以成其釁,此誕之罪。宇文歆論情則疏,向彼又淺,王之過失,悉以聞奏。且父子之際,人所難言,而歆
 言之,豈非忠懇?今欲誅罪,不錄其心,臣愚竊以為過。」翌日,高祖召綱入,升御坐,謂曰:「今我有公,遂使刑罰不濫。元吉自惡,結怨於人。歆既曾以表聞,誕亦焉能禁制?皆非其罪也。」尋加授元吉侍中、襄州道行臺尚書令、稷州刺史。四年,太宗征竇建德,留元吉與屈突通圍王世充於東都。世充出兵拒戰,元吉設伏擊破之,斬首八百級,生擒其大將樂仁昉、甲士千餘人。世充平,拜司空,餘官如故,加賜袞冕之服、前後部鼓吹樂二部、班劍二十人、
 黃金二千斤,與太宗各聽三爐鑄錢以自給。六年,加授隰州總管。及與建成連謀,各募壯士,多匿罪人。復內結宮掖,遞加稱譽,又厚賂中書令封倫以為黨助。由是高祖頗疏太宗而加愛元吉。太宗嘗從高祖幸其第,元吉伏其護軍宇文寶於寢內,將以刺太宗。建成恐事不果而止之,元吉慍曰:「為兄計耳,於我何害!」九年,轉左衛大將軍,尋進位司徒、兼侍中,並州大都督、隰州都督、稷州刺史並如故。



 高祖將避暑太和宮,二王當從,元吉謂建
 成曰:「待至宮所,當興精兵襲取之。置土窟中,唯開一孔以通飲食耳。」會突厥鬱射設屯軍河南,入圍烏城。建成乃薦元吉代太宗督軍北討,仍令秦府驍將秦叔寶、尉遲敬德、程知節、段志玄等並與同行。又追秦府兵帳,簡閱驍勇,將奪太宗兵以益其府。又譖杜如晦、房玄齡,逐令歸第。高祖知其謀而不制。元吉因密請加害太宗,高祖曰:「是有定四海之功,罪跡未見,一旦欲殺,何以為辭?」元吉曰:「秦王常違詔敕,初平東都之日,偃蹇顧望,不急還
 京,分散錢帛,以樹私惠。違戾如此,豈非反逆?但須速殺,何患無辭!」高祖不對,元吉遂退。建成謂元吉曰:「既得秦王精兵,統數萬之眾,吾與秦王至昆明池,於彼宴別,令壯士拉之於幕下,因云暴卒,主上諒無不信。吾當使人進說,令付吾國務。正位已後,以汝為太弟。敬德等既入汝手,一時坑之,孰敢不服?」率更丞王晊聞其謀,密告太宗。太宗召府僚以告之,皆曰:「大王若不正斷,社稷非唐所有。若使建成、元吉肆其毒心,群小得志,元吉狼戾,終
 亦不事其兄。往者護軍薛寶上齊王符籙云:『元吉合成唐字。』齊王得之喜曰:『但除秦王,取東宮如反掌耳。』為亂未成,預懷相奪。以大王之威,襲二人如拾地芥。」太宗遲疑未決,眾又曰:「大王以舜為何如人也?」曰:「浚哲文明,溫恭允塞,為子孝,為君聖,焉可議之乎?」府僚曰:「向使舜浚井不出,自同魚鱉之斃,焉得為孝子乎?塗廩不下,便成煨燼之餘,焉得為聖君乎?小杖受,大杖避,良有以也。」太宗於是定計誅建成及元吉。元吉死時年二十四。有五
 子:梁郡王承業、漁陽王承鸞、普安王承獎、江夏王承裕、義陽王承度,並坐誅。尋詔絕建成、元吉屬籍。太宗踐祚,追封元吉為海陵郡王,謚曰剌,以禮改葬。貞觀十六年,又追封巢王,謚如故,復以曹王明為元吉後。



 楚王智云,高祖第五子也。母曰萬貴妃,性恭順,特蒙高祖親禮。宮中之事,皆諮稟之,諸王妃主,莫不推敬。後授楚國太妃,薨,陪葬獻陵。智云本名稚詮,大業末,從高祖於河東。及義師將起,隱太子建成潛歸太原,以智云年
 小,委之而去。因為吏所捕,送於長安,為陰世師所害,年十四。義寧元年,贈尚書左僕射、楚國公。武德元年,追封楚王,謚曰哀。無子,三年,以太宗子寬為嗣。寬薨,貞觀二年,復以濟南公世都子靈龜嗣焉。靈龜,永徽中歷魏州刺史,政尚清嚴,奸盜屏跡。又開永濟渠入於新市,以控引商旅,百姓利之。卒官。子福嗣,嗣降爵為公。儀鳳中,卒於右威衛將軍。子承況,神龍中為右羽林將軍,與節愍太子同舉兵,入玄武門,為亂兵所殺。



 荊王元景,高祖第六子也。武德三年,封為趙王。八年,授安州都督。貞觀初,歷遷雍州牧、右驍衛大將軍。十年,徙封荊王,授荊州都督。十一年,定制元景等為代襲刺史。詔曰:



 皇王受命,步驟之跡以殊;經籍所紀,質文之道匪一。雖治亂不同,損益或異,至於設官司以制海內,建籓屏以輔王室,莫不明其典章,義存於致治;崇其賢戚,志在於無疆。朕以寡昧,丕承鴻緒,寅畏三靈,憂勤百姓,考明哲之餘論,求經邦之長策。帝業之重,獨任難以成務;
 天下之曠,因人易以獲安。然則侯伯肇於自昔,州郡始於中代,聖賢異術,沿革隨時,復古則義難頓從,尋今則事不盡理。遂規模周、漢,斟酌曹、馬,採按部之嘉名,參建侯之舊制,共治之職重矣,分土之實存焉。已有制書,陳其至理。繼世垂範,貽厥後昆;維城作固,同符前烈。荊州都督荊王元景、梁州都督漢王元昌、徐州都督徐王元禮、潞州都督韓王元嘉、遂州都督彭王元則、鄭州刺史鄭王元懿、絳州刺史霍王元軌、虢州刺史虢王鳳、豫州
 刺史道王元慶、鄧州刺史鄧王元裕、壽州刺史舒王元名、幽州都督燕王靈夔、蘇州刺史許王元祥、安州都督吳王恪、相州都督魏王泰、齊州都督齊王裕、益州都督蜀王愔、襄州刺史蔣王惲、揚州都督越王貞、並州都督晉王某、秦州都督紀王慎等,或地居旦、奭,夙聞《詩》、《禮》;或望及間、平,早稱才藝,並爵隆土宇,寵兼車服。誠孝之心,無忘於造次;風政之舉,克著於期月。宜冠恆冊,祚以休命。其所任刺史,咸令子孫代代承襲。



 尋又罷代襲之制。
 元景久之轉鄜州刺史。高宗即位,進位司徒,加實封通前滿一千五百戶。永徽四年,坐與房遺愛謀反賜死,國除。後追封沉黎王,備禮改葬。以渤海王奉慈子長沙為嗣,降爵為侯。神龍初,追復爵土,並封其孫逖為嗣荊王,尋薨,國除。



 漢王元昌,高祖第七子也。少好學,善隸書。武德三年,封為魯王。貞觀五年,授華州刺史,轉梁州都督。十年,改封漢王。元昌在州,頗違憲法,太宗手敕責之。初不自咎,更
 懷怨望。知太子承乾嫉魏王泰之寵,乃相附托,圖為不軌。十六年,元昌來朝京師,承乾頻召入東宮夜宿,因謂承乾曰:「願陛下早為天子。近見御側,有一宮人,善彈琵琶,事平之後,當望垂賜。」承乾許諾。又刻臂出血,以帛拭之,燒作灰,和酒同飲,共為信誓,潛伺間隙。十七年,事發,太宗弗忍加誅,特敕免死。大臣高士廉、李世勣等奏言:「王者以四海為家,以萬姓為子,公行天下,情無獨親。元昌苞藏兇惡,圖謀逆亂,觀其指趣,察其心府,罪深燕旦,
 釁甚楚英。天地之所不容,人臣之所切齒,五刑不足申其罰,九死無以當其愆。而陛下情屈至公,恩加梟獍,欲開疏網,漏此鯨鯢。臣等有司,期不奉制,伏願敦師憲典,誅此兇慝。順群臣之願,奪鷹鸇之心,則吳、楚七君,不幽嘆於往漢;管、蔡二叔,不沉恨於有周。」太宗事不獲已,乃賜元昌自盡於家,妻子籍沒,國除。



 酆王元亨,高祖第八子也。武德三年受封。貞觀二年,授散騎常侍,拜金州刺史。及之籓,太宗以其幼小,甚思之,
 中路賜以金盞,遣使為之設宴。六年薨,無子,國除。



 周王元方,高祖第九子也。武德四年受封。貞觀二年,授散騎常侍。三年薨,贈左光祿大夫,無子,國除。



 徐王元禮,高祖第十子也。少恭謹,善騎射。武德四年,封鄭王。貞觀六年,賜實封七百戶,授鄭州刺史,徙封徐王,遷徐州都督。十七年,轉絳州刺史,以善政聞,太宗降璽書勞勉,賜以錦彩。二十三年,加實封千戶。永徽四年,加授司徒,兼潞州刺史。咸亨三年薨,贈太尉、冀州大都督,
 陪葬獻陵。



 子淮南王茂嗣。茂險薄無行,元禮姬趙氏有美色,及元禮遇疾,茂遂逼之,元禮知而切加責讓。茂乃屏斥元禮侍衛,斷其藥膳,仍云:「既得五十年為王,更何煩服藥?」竟以餒終。上元中,事洩,配流振州而死。神龍初,又封茂子璀為嗣徐王。景龍四年,加銀青光祿大夫。開元中,除宗正員外卿,卒。子延年嗣。開元二十六年,封嗣徐王,除員外洗馬。天寶初,拔汗那王入朝,延年將嫁女與之,為右相李林甫所奏,貶文安郡別駕、彭城長史,坐
 贓貶永嘉司士。至德初,餘杭郡司馬,卒。永泰元年,女婿黔中觀察使趙國珍入朝,請以延年子前施州刺史諷為嗣,因封嗣徐王。



 韓王元嘉,高祖第十一子也。母宇文昭儀,隋左武衛大將軍述之女也。早有寵於高祖,高祖初即位,便欲立為皇后,固辭不受。元嘉少以母寵,特為高祖所愛,自登極晚生皇子,無及之者。武德四年,封宋王,徙封徐王。貞觀六年,賜實封七百戶,授潞州刺史,時年十五。在州聞太
 妃有疾,便涕泣不食。及京師發喪,哀毀過禮,太宗嗟其至性,屢慰勉之。九年,授右領軍大將軍。十年,改封韓王,授潞州都督。二十三年,加實封滿千戶。元嘉少好學,聚書至萬卷,又採碑文古跡,多得異本。閨門修整,有類寒素士大夫。與其弟靈夔甚相友愛,兄弟集見,如布衣之禮。其修身潔己,內外如一,當代諸王莫能及者,唯霍王元軌抑其次焉。高宗末,元嘉轉澤州刺史。及天后臨朝攝政,欲順物情,乃進授元嘉為太尉,定州刺史、霍王元
 軌為司徒,青州刺史、舒王元名為司空,隆州刺史、魯王靈夔為太子太師,蘇州刺史、越王貞為太子太傅,安州都督、紀王慎為太子太保,並外示尊崇,實無所綜理。其後漸將誅戮宗室諸王不附己者,元嘉大懼,與其子通州刺史、黃公譔及越王貞父子謀起兵,於是皇宗國戚內外相連者甚廣。遣使報貞及貞子瑯邪王沖曰:「四面同來,事無不濟。」沖與諸道計料未審而先發兵,倉卒唯貞應之,諸道莫有赴者,故其事不成。元嘉坐誅。譔少以
 文才見知,諸王子中,與瑯邪王沖為一時之秀,凡所交結皆當代名士。時天下犯罪籍沒者甚眾,唯沖與譔父子書籍最多,皆文句詳定,秘閣所不及。神龍初,追復元嘉爵土,並封其第五子訥為嗣韓王,官至員外祭酒。開元十七年卒。元嘉長子訓,高祖時封潁川王,早卒。次子誼,封武陵王,官至濮州刺史。開元中,封訥子叔璇為嗣韓王、國子員外司業。



 彭王元則,高祖第十二子也。武德四年,封荊王。貞觀
 七年,授豫州刺史。十年,改封彭王,除遂州都督,尋坐章服奢僭免官。十七年,拜澧州刺史,更折節勵行,頗著聲譽。永徽二年薨,高宗為之廢朝三日,贈司徒、荊州都督,陪葬獻陵,謚曰思。發引之日,高宗登望春宮望其靈車,哭之甚慟。無子,以霍王元軌子絢嗣,龍朔中封南昌王。子志暕,神龍初封嗣彭王。景龍初,加銀青光祿大夫。開元中,宗正卿同正員,卒。



 鄭王元懿,高祖第十三子也。頗好學。武德四年,封滕王。
 貞觀七年,授兗州刺史,賜實封六百戶。十年,改封鄭王,歷鄭、潞二州刺史。二十三年,加實封滿千戶。總章中,累授絳州刺史。數斷大獄,甚有平允之譽。高宗嘉之,降璽書褒美,賜物三百段。咸亨四年薨,贈司徒、荊州大都督,謚曰惠,陪葬獻陵。子璥,上元初,封為嗣鄭王,官至鄂州刺史。神龍初,又封璥嫡子希言為嗣鄭王。景龍四年,嗣鄭王希言等共一十四人,並加銀青光祿大夫。開元中,右金吾大將軍。天寶初,再為太子詹事同正員,卒。



 霍王元軌,高祖第十四子也。少多才藝,高祖甚奇之。武德六年,封蜀王。八年,徙封吳王。貞觀初,太宗嘗問群臣曰:「朕子弟孰賢?」侍中魏徵對曰:「臣愚暗,不盡知其能。唯吳王數與臣言,未嘗不自失。」上曰:「朕亦器之,卿以為前代誰比?」徵曰:「經學文雅,亦漢之間、平也。」由是寵遇彌厚,因令娶征女焉。從太宗游獵,遇群獸,命元軌射之,矢不虛發,太宗撫其背曰:「汝武藝過人,悵今無所施耳。當天下未定,我得汝豈不美乎!」七年,拜壽州刺史,賜實封六
 百戶。高祖崩,去職,毀瘠過禮,自後常衣布,示有終身之戚焉。每至忌辰,輒數日不食。十年,改封霍王,授絳州刺史,尋轉徐州刺史。元軌前後為刺史,至州,唯閉閣讀書,吏事責成於長史、司馬,謹慎自守,與物無忤,為人不妄。在徐州,唯與處士劉玄平為布衣之交。人或問玄平王之長,玄平答曰:「無長。」問者怪而復問之,玄平曰:「夫人有短,所以見其長。至於霍王,無所不備,吾何以稱之哉?」二十三年,加實封滿千戶,為定州刺史。突厥來寇,元軌令
 開門偃旗,虜疑有伏,懼而宵遁。州人李嘉運與賊連謀,事洩,高宗令收按其黨。元軌以強寇在境,人心不安,惟殺嘉運,餘無所及,因自劾違制。上覽表大悅,謂使曰:「朕亦悔之,向無王,則失定州矣。」有王文操遇賊,而二子鳳、賢遂以身蔽捍,文操獲全,二子皆死。縣司抑而不申,元軌察知,遣使吊祭,表上其事,詔並贈朝散大夫,令加旌表。其禮賢愛善如此。後因入朝,屢上疏陳時政得失,多所匡益,高宗甚尊重之。及在外籓,朝廷每有大事,或
 密制問焉。高宗崩,與侍中劉齊賢等知山陵葬事,齊賢服其識練故事,每謂人曰:「非我輩所及也。」元軌嘗使國令征封,令白:「請依諸國賦物貿易取利。」元軌曰:「汝為國令,當正吾失,反說吾以利耶!」拒而不納。垂拱元年,加位司徒,尋出為襄州刺史,轉青州。四年,坐與越王貞連謀起兵,事覺,徙居黔州,仍令載以檻車,行至陳倉而死。有子七人。長子緒,最有才藝。上元中,封江都王,累除金州刺史。重拱中,坐與裴承光交通被殺。神龍初,與元軌並追
 復爵位,仍封緒孫暉為嗣霍王。景龍四年,加銀青光祿大夫。開元中,左千牛員外將軍。



 虢王鳳,高祖第十五子也。武德六年,封豳王。貞觀七年,授鄧州刺史,賜實封六百戶。十年,徙封虢王,歷虢、豫二州刺史。二十三年,加實封滿千戶。麟德初,累授青州刺史。上元元年薨,年五十二,贈司徒、揚州大都督,陪葬獻陵,謚曰莊。子平陽郡王翼嗣,官至光州刺史。永隆二年卒。子寓嗣,則天時失爵。鳳第三子定襄郡公宏,則天初
 為曹州刺史。第五子東莞郡公融,少以武勇見知。垂拱中,為申州刺史。初,黃公譔將與越王貞通謀,深倚仗融,以為外助。時詔追諸親赴都,融私使問其所親成均助教高子貢曰:「可入朝以否?」子貢報曰:「來必取死。」融乃稱疾不朝,以俟諸籓期。及得越王貞起兵書,倉卒不能相應,為僚吏所逼,不獲已而奏之,於是擢授銀青光祿大夫,行太子右贊善大夫。未幾,為支黨所引,被誅。子徹,神龍元年襲封東莞郡公。開元五年,繼密王元曉,改為嗣
 密王。十二年,改封濮陽郡王,歷宗正卿、金紫光祿大夫,卒。神龍初,封鳳嫡孫邕為嗣虢王。邕娶韋庶人妹為妻,由是中宗時特承寵異,轉秘書監,俄又改封汴王,開府置僚屬。月餘而韋氏敗,邕揮刃截其妻首,以至於朝,深為物議所鄙。貶沁州刺史,不知州事,削封邑。景雲二年,復嗣虢王,還封二百戶。累遷衛尉卿。開元十五年卒。子巨嗣,別有傳。



 道王元慶,高祖第十六子也。武德六年,封漢王。八年,改
 封陳王。貞觀九年,拜趙州刺史,賜實封八百戶。十年,改封道王,授豫州刺史。二十三年,加實封滿千戶。永徽四年,歷滑州刺史,以政績聞,賜物二百段。後歷徐、沁、衛三州刺史。元慶事母甚謹,及母薨,又請躬修墳墓,優詔不許。麟德元年薨,贈司徒、益州都督,陪葬獻陵,謚曰孝。子臨淮王誘嗣,官至澧州刺史。永淳中,坐贓削爵。次子詢,壽州刺史。詢子微,神龍初,封為嗣道王。景龍四年,加銀青光祿大夫。景雲元年,宗正卿,卒。子鏈,開元二十五年,
 襲封嗣道王。廣德中,官至宗正卿。



 鄧王元裕,高祖第十七子也。貞觀五年,封鄶王。十一年,改封鄧王,賜實封八百戶,歷鄧、梁、黃三州刺史。元裕好學,善談名理,與典簽盧照鄰為布衣之交。二十三年,加實封通前一千五百戶。高宗時,又歷壽、襄二州刺史、兗州都督。麟德二年薨,贈司徒、益州大都督,陪葬獻陵,謚曰康。無子,以弟江王元祥子廣平公炅嗣。神龍初,封炅子孝先為嗣鄧王。開元十三年,右監門衛大將軍、冠軍
 大將軍,卒。



 舒王元名,高祖第十八子也。年十歲時,高祖在大安宮,太宗晨夕使尚宮起居送珍饌,元名保傅等謂元名曰:「尚宮品秩高者,見宜拜之。」元名曰:「此我二哥家婢也,何用拜為?」太宗聞而壯之,曰:「此真我弟也。」貞觀五年,封譙王。十一年,徙封舒王,賜實封八百戶,拜壽州刺史。後歷滑、許、鄭三州刺史。二十三年,加實封滿千戶,轉石州刺史。



 元名性高潔,罕問家人產業,朝夕矜莊,門庭清肅,常
 誡其子豫章王亶等曰:「籓王所乏者,不慮無錢財官職,但勉行善事,忠孝持身,此吾志也。」及亶為江州刺史,以善政聞,高宗手敕褒美元名,以賞其義方之訓。高宗每欲授元名大州刺史,固辭曰:「忝預籓戚,豈以州郡戶口為仕進之資?」辭情懇到,故在石州二十年,賞玩林泉,有塵外之意。垂拱年,除青州刺史,又除鄭州刺史。州境鄰接都畿,諸王及帝戚蒞官者,或有不檢攝家人,為百姓所苦。及元名到,大革其弊。轉滑州刺史,政理如在鄭州。
 尋加授司空。永昌年,與子亶俱為丘神勣所陷,被殺。神龍初,贈司徒,復其官爵,仍令以禮改葬。亶子津為嗣舒王。景龍四年,加銀青光祿大夫。開元中,左威衛將軍,卒。子萬嗣,天寶二年卒。子藻嗣,天寶九載,封嗣舒王。



 魯王靈夔,高祖第十九子也。少有美譽,善音律,好學,工草隸,與同母兄韓王元嘉特相友愛。貞觀五年,封魏王。十年,改封燕王,賜實封八百戶,授幽州都督。十四年,改封魯王,授兗州都督。二十三年,加實封滿千戶。永徽六
 年,轉隆州刺史,後歷絳、滑、定等州刺史,太子太師。垂拱元年,授邢州刺史。四年,與兄元嘉子黃公譔結謀,欲起兵應接越王貞父子,事洩,配流振州,自縊而死。有二子:長子銑,封清河王。次子藹,封範陽王,歷右散騎常侍,為酷吏所陷。神龍初,追復靈夔官爵,仍令以禮改葬。封藹子道堅為嗣魯王。性嚴整,雖在閨門,造次必於莊敬。少年佐郡,聲實已彰。景龍四年,加銀青光祿大夫,歷果、隴、吉、冀、洺、汾、滄等七州刺史,國子祭酒。開元二十二年,兼
 檢校魏州刺史,未行,改汴州刺史、河南道採訪使。此州都會,水陸輻湊,實曰膏腴,道堅特以清毅聞。入為宗正卿,卒。子宇嗣,二十九年,封嗣魯王。至德元年,從幸巴蜀,為右金吾將軍。寶應元年,皇太子封為魯王,改宇嗣鄒王。道堅弟道邃,中興初,封戴國公。以恭默自守,修山東婚姻故事,頻任清列。天寶中為右丞,大理、宗正二卿,卒。



 江王元祥,高祖第二十子也。貞觀五年,封許王。十一年,徙封江王,授蘇州刺史,賜實封八百戶。二十三年,加實
 封滿千戶。高宗時,又歷金、鄜、鄭三州刺史。性貪鄙,多聚金寶,營求無厭,為人吏所患。時滕王元嬰、蔣王惲、虢王鳳亦稱貪暴,有授得其府官者,以比嶺南惡處,為之語曰:「寧向儋、崖、振、白,不事江、滕、蔣、虢。」元祥體質洪大,腰帶十圍,飲啖亦兼數人,其時韓王元嘉、虢王鳳、魏王恭狀貌亦偉,不逮於元祥。又眇一目。永隆元年薨,贈司徒、並州大都督,陪葬獻陵,謚曰安。子永嘉王晫,永隆中,為復州刺史。以禽獸其行,賜死於家。中興初,元祥子鉅鹿郡
 公晃子欽嗣江王。景龍四年,加銀青光祿大夫,娶王仁皎女,至千牛將軍,卒。



 密王元曉,高祖第二十一子也。貞觀五年受封。九年,授虢州刺史。十四年,賜實封八百戶。二十三年,加滿千戶,轉澤州刺史。永徽四年,除宣州刺史,後歷徐州刺史。上元三年薨,贈司徒、揚州都督,陪葬獻陵,謚曰貞。子南安王穎嗣。神龍初,封穎弟亮子曇為嗣密王。



 滕王元嬰,高祖第二十二子也。貞觀十三年受封。十五
 年,賜實封八百戶,授金州刺史。二十三年,加實封滿千戶。永徽中,元嬰頗驕縱逸游,動作失度,高宗與書誡之曰:「王地在宗枝,寄深磐石,幼聞《詩》、《禮》,夙承義訓。實冀孜孜無怠,漸以成德,豈謂不遵軌轍,逾越典章。且城池作固,以備不虞,關鑰閉開,須有常準。鳩合散樂,並集府僚,嚴關夜開,非復一度。遏密之悲,尚纏比屋,王以此情事,何遽紛紜?又巡省百姓,本觀風問俗,遂乃驅率老幼,借狗求置,志從禽之娛,忽黎元之重。時方農要,屢出畋游,
 以彈彈人,將為笑樂。取適之方,亦應多緒,何必此事,方得為娛?晉靈虐主,未可取則。趙孝文趨走小人,張四又倡優賤隸,王親與博戲,極為輕脫,一府官僚,何所瞻望?凝寒方甚,以雪埋人,虐物既深,何以為樂?家人奴僕,侮弄官人,至於此事,彌不可長。朕以王骨肉至親,不能致王於法,令與王下上考,以愧王心。人之有過,貴在能改,國有憲章,私恩難再。興言及此,慚嘆盈懷。」三年,遷蘇州刺史,尋轉洪州都督。又數犯憲章,削邑戶及親事帳內
 之半,於滁州安置。後起授壽州刺史,轉隆州刺史。弘道元年,加開府儀同三司,兼梁州都督。文明元年薨,贈司徒、冀州都督,陪葬獻陵。子長樂王循琦嗣。兄弟六人,垂拱中並陷詔獄。神龍初,以循琦弟循琣子涉嗣滕王,本名茂宗,狀貌類胡而豐碩。開元十二年,加銀青光祿大夫,左驍衛將軍。天寶初,淮安郡別駕,卒。子湛然嗣。十一載,封滕王。十五載,從幸蜀,除左金吾將軍。



 史臣曰:一人元良,萬國以貞。若明異重離,道非出震,雖
 居嫡長,寧固金其鎡!況當開創之初,未見太平之兆。建成殘忍,豈主鬯之才;元吉兇狂,有覆巢之跡。若非太宗逆取順守,積德累功,何以致三百年之延洪、二十帝之纂嗣?或堅持小節,必虧大猷,欲比秦二世、隋煬帝,亦不及矣。元嘉修身,元軌無短,元裕名理,元名高潔,靈夔嚴整,皆有封冊之名,而無磐石之固。武氏之亂,或連頸被刑;奸臣擅權,則束手為制。其望本枝百世也,不亦難乎?



 贊曰:有功曰祖,有德曰宗。建成、元吉,實為二兇。中外交
 構,人神不容。用晦而明,殷憂啟聖。運屬文皇,功成守正。善惡既分,社稷乃定。盤維封建,本枝茂盛。元嘉、元軌,修身慎行。元裕、元名,行簡居正。犬牙不固,武氏易姓。既無兵民,若拘陷井。敢告後人,無或失政。



\end{pinyinscope}