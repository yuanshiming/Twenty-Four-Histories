\article{卷二十九 志第九 音樂二}

\begin{pinyinscope}

 高
 祖登極之後,享宴因隋舊制,用九部之樂,其後分為立坐二部。今立部伎有《安樂》、《太平樂》、《破陣樂》、《慶善樂》、《大定樂》、《上元樂》、《聖壽樂》、《樂聖樂》,凡八部。《安樂》者,後周武帝
 平齊所作也。行列方正,象城郭,周世謂之城舞。舞者八十人。刻木為面,狗喙獸耳,以金飾之,垂線為發,畫猰皮帽。舞蹈姿制,猶作羌胡狀。《太平樂》,亦謂之五方師子舞。師子鷙獸,出於西南夷天竺、師子等國。綴毛為之,人居其中,像其俯仰馴狎之容。二人持繩秉拂,為習弄之狀。五師子各立其方色。百四十人歌《太平樂》,舞以足,持繩者服飾作昆侖象。《破陣樂》,太宗所造也。太宗為秦王之時,征伐四方,人間歌謠《秦王破陣樂》之曲。及即位,使呂
 才協音律,李百藥、虞世南、褚亮、魏徵等制歌辭。百二十人披甲持戟,甲以銀飾之。發揚蹈厲,聲韻慷慨。享宴奏之,天子避位,坐宴者皆興。《慶善樂》,太宗所造也。太宗生於武功之慶善宮,既貴,宴宮中,賦詩,被以管弦。舞者六十四人。衣紫大袖裾襦,漆髻皮履。舞蹈安徐,以象文德洽而天下安樂也。《大定樂》,出自《破陣樂》。舞者百四十人。被五彩文甲,持槊。歌和雲,「八紘同軌樂」,以象平遼東而邊隅大定也。《上元樂》,高宗所造。舞者百八十人。畫雲衣,備
 五色,以象元氣,故曰「上元」。《聖壽樂》,高宗武后所作也。舞者百四十人。金銅冠,五色畫衣。舞之行列必成字,十六變而畢。有「聖超千古,道泰百王,皇帝萬年,寶祚彌昌」字。《光聖樂》,玄宗所造也。舞者八十人。烏冠,五彩畫衣,兼以《上元》、《聖壽》之容,以歌王跡所興。



 自《破陣舞》以下,皆雷大鼓,雜以龜茲之樂,聲振百里,動蕩山谷。《大定樂》加金鉦。惟《慶善舞》獨用西涼樂,最為閑雅。《破陣》、《上元》、《慶善》三舞,皆易其衣冠,合之鐘磬,以享郊廟。以《破陣》為武舞,謂之《
 七德》;《慶善》為文舞,謂之《九功》。自武后稱制,毀唐太廟,此禮遂有名而亡實。《安樂》等八舞,聲樂皆立奏之,樂府謂之立部伎。其餘總謂之坐部伎。則天、中宗之代,大增造坐立諸舞,尋以廢寢。



 坐部伎有《宴樂》、《長壽樂》、《天授樂》、《鳥歌萬壽樂》、《龍池樂》、《破陣樂》,凡六部。《宴樂》,張文收所造也。工人緋綾袍,絲布褲。舞二十人,分為四部:《景雲樂》,舞八人,花錦袍,五色綾褲,雲冠烏皮靴;《慶善樂》,舞四人,紫綾袍,大袖,絲布褲,假髻。《破陣樂》,舞四人,緋綾袍,錦衿褾,
 緋綾褲。《承天樂》,舞四人,紫袍,進德冠,並銅帶。樂用玉磬一架,大方響一架,NN箏一,臥箜篌一,小箜篌一,大琵琶一,大五弦琵琶一,小五弦琵琶一,大笙一,小笙一,大篳篥一,小篳篥一,大簫一,小律一,正銅拔一,和銅拔一,長笛一,短笛一,楷鼓一,連鼓一,鞉鼓一,桴鼓一,工歌二。此樂惟《景雲舞》僅存,餘並亡。《長壽樂》,武太后長壽年所造也。舞十有二人。畫衣冠。《天授樂》,武太后天授年所造也。舞四人。畫衣五採,鳳冠。《鳥歌萬歲樂》,武太后所造也。武
 太后時,宮中養鳥能人言,又常稱萬歲,為樂以象之。舞三人。緋大袖,並畫鸜鵒,冠作鳥像。今案嶺南有鳥,似鸜鵒而稍大,乍視之,不相分辨。籠養久則能言,無不通,南人謂之吉了,亦云料。開元初,廣州獻之,言音雄重如丈夫,委曲識人情,慧於鸚鵡遠矣,疑即此鳥也。《漢書·武帝本紀》書南越獻馴象、能言鳥。注《漢書》者,皆謂鳥為鸚鵡。若是鸚鵡,不得不舉其名,而謂之能言鳥。鸚鵡秦、隴尤多,亦不足重。所謂能言鳥,即吉了也。北方常言鸜鵒逾
 嶺乃能言,傳者誤矣。嶺南甚多鸜鵒,能言者非鸜鵒也。《龍池樂》,玄宗所作也。玄宗龍潛之時,宅在隆慶坊,宅南坊人所居,變為池,望氣者亦異焉。故中宗季年,泛舟池中。玄宗正位,以坊為宮,池水逾大,彌漫數里,為此樂以歌其祥也。舞十有二人,人冠飾以芙蓉。《破陣樂》,玄宗所造也。生於立部伎《破陣樂》。舞四人,金甲胄。自《長壽樂》已下皆用龜茲樂,舞人皆著靴。惟《龍池》備用雅樂,而無鐘磬,舞人躡履。



 《清樂》者,南朝舊樂也。永嘉之亂,五都淪覆,
 遺聲舊制,散落江左。宋、梁之間,南朝文物,號為最盛;人謠國俗,亦世有新聲。後魏孝文、宣武,用師淮、漢,收其所獲南音,謂之《清商樂》。隨平陳,因置清商署,總謂之《清樂》。遭梁、陳亡亂,所存蓋鮮。隋室已來,日益淪缺。武太后之時,猶有六十三曲,今其辭存者,惟有《白雪》、《公莫舞》、《巴渝》、《明君》、《鳳將雛》、《明之君》、《鐸舞》、《白鳩》、《白糸寧》、《子夜》、《吳聲四時歌》、《前溪》、《阿子》及《歡聞》、《團扇》、《懊憹》、《長史》、《督護》、《讀曲》、《烏夜啼》、《石城》、《莫愁》、《襄陽》、《棲烏夜飛》、《估客》、《楊伴》、《雅歌》、《驍壺》、《常林歡》、《三
 洲》、《採桑》、《春江花月夜》、《玉樹後庭花》、《堂堂》、《泛龍舟》等三十二曲,《明之君》、《雅歌》各二首,《四時歌》四首,合三十七首。又七曲有聲無辭:《上林》、《鳳雛》、《平調》、《清調》、《瑟調》、《平折》、《命嘯》,通前為四十四曲存焉。



 《白雪》,周曲也。《平調》、《清調》、《瑟調》,皆周房中曲之遺聲也。漢世謂之三調。《公莫舞》,晉、宋謂之巾舞。其說云:「漢高祖與項籍會於鴻門,項莊劍舞,將殺高祖。項伯亦舞,以袖隔之,且云公莫害沛公也。漢人德之,故舞用巾,以象項伯衣袖之遺式也。《巴渝》,漢高帝所作
 也。帝自蜀漢伐楚,以版盾蠻為前鋒,其人勇而善鬥,好為歌舞,高帝觀之曰:「武王伐紂歌也。」使工習之,號曰《巴渝》。渝,美也。亦云巴有渝水,故名之。魏、晉改其名,梁復號《巴渝》,隋文廢之。《明君》,漢元帝時,匈奴單于入朝,詔王嬙配之,即昭君也。及將去,入辭。光彩射人,聳動左右,天子悔焉。漢人憐其遠嫁,為作此歌。晉石崇妓綠珠善舞,以此曲教之,而自制新歌曰:「我本漢家子,將適單于庭,昔為匣中玉,今為糞土英。」晉文王諱昭,故晉人謂之《明君》。
 此中朝舊曲,今為吳聲,蓋吳人傳受訛變使然。《鳳將雛》,漢世舊歌曲也。《明之君》,本漢世《鞞舞曲》也。梁武時,改其辭以歌君德。《鐸舞》,漢曲也。《白鳩》,吳朝《拂舞曲》也。楊泓《拂舞序》曰:「自到江南,見《白符舞》,或言《白鳧鳩》,云有此來數十年。察其辭旨,乃是吳人患孫皓虐政,思屬晉也。」隋牛弘請以鞞、鐸、巾、拂等舞陳之殿庭。帝從之,而去其所持巾拂等。《白紵》,沈約云:本吳地所出,疑是吳舞也。梁帝又令約改其辭。其《四時白紵》之歌,約集所載是也。今中
 原有《白紵曲》,辭旨與此全殊。《子夜》,晉曲也。晉有女子夜造此聲,聲過哀苦,晉日常有鬼歌之。《前溪》,晉車騎將軍沈珫所制。《阿子》及《歡聞》,晉穆帝升平初。歌畢,輒呼「阿子汝聞否」,後人演其聲以為此曲。《團扇》,晉中書令王氏與嫂婢有情,愛好甚篤。嫂捶撻婢過苦,婢素善歌,而氏好捉白團扇,故云:「團扇復團扇,持許自遮面。憔悴無復理,羞與郎相見。」《懊憹》,晉隆安初民間訛謠之曲。歌云:「春草可攬結,女兒可攬擷。」齊太祖常謂之《中朝歌》。《長史變》,晉
 司徒左長史王廞臨敗所制。《督護》,晉、宋間曲也。彭城內史徐達之為魯軌所殺。徐,宋高祖長婿也。使府內直督護丁旿殯斂之。其妻呼旿至閣下,自問斂達之事,每問輒嘆息曰:「丁督護!」其聲哀切,後人因其聲廣其曲焉。今歌是宋孝武帝所制,云:「督護上征去,儂亦惡聞許。願作石尤風,四面斷行旅。」《讀曲》,宋人為彭城王義康所制也,有死罪之辭。《烏夜啼》,宋臨川王義慶所作也。元嘉十七年,徙彭城王義康於豫章。義慶時為江州,至鎮,相見而
 哭,為帝所怪,徵還宅,大懼。妓妾夜聞烏啼聲,扣齋閣云:「明日應有赦。」其年更為南兗州刺史,作此歌。故其和云:「籠窗窗不開,烏夜啼,夜夜望郎來。」今所傳歌似非義慶本旨。辭曰:「歌舞諸少年,娉婷無種跡。菖蒲花可憐,聞名不相識。」《石城》,宋臧質所作也。石城在竟陵。質嘗為竟陵郡,於城上眺矚,見群少年歌謠通暢,因作此曲。歌云:「生長石城下,開門對城樓。城中美年少,出入見依投。」《莫愁樂》,出於《石城樂》。石城有女子名莫愁,善歌謠。《石城樂》和中
 復有「莫愁」聲,故歌云:「莫愁在何處?莫愁石城西。艇子打兩槳,催送莫愁來。」《襄陽樂》,宋隨王誕之所作也。誕始為襄陽郡,元嘉二十六年,仍為雍州,夜聞諸女歌謠,因作之。故歌和云「襄陽來夜樂。」其歌曰:「朝發襄陽來,暮至大堤宿。大堤諸女兒,花艷驚郎目。」裴子野《宋略》稱:「晉安侯劉道彥為雍州刺史,有惠化,百姓歌之,號《襄陽樂》。」其辭旨非也。《棲烏夜飛》,沈攸之元徽五年所作也。攸之未敗之前,思歸京師,故歌和云:「日落西山還去來!」《估客
 樂》,齊武帝之制也。布衣時常游樊、鄧,追憶往事而作。歌曰:「昔經樊、鄧役,阻潮梅根渚。感憶追往事,意滿情不敘。」使太樂令劉瑤教習,百日無成。或啟釋寶月善音律,帝使寶月奏之,便就。敕歌者常重為感憶之聲。梁改其名為《商旅行》。《楊伴》,本童謠歌也。齊隆昌時,女巫之子曰楊旻,旻隨母入內,及長,為後所寵。童謠云:「楊婆兒,共戲來。」而歌語訛,遂成楊伴兒。歌云:「暫出白門前,楊柳可藏烏。歡作沉水香,儂作博山爐。」《驍壺》,疑是投壺樂也。投壺者
 謂壺中躍矢為驍壺,今謂之驍壺者是也。《常林歡》,疑是宋、梁間曲。宋、梁世,荊、雍為南方重鎮,皆皇子為之牧,江左辭詠,莫不稱之,以為樂土,故隨王作《襄陽》之歌,齊武帝追憶樊、鄧。梁簡文樂府歌云:「分手桃林岸,送別峴山頭。若欲寄音信,漢水向東流。」又曰:「宜城投音豆酒今行熟,停鞍系馬暫棲宿。」桃林在漢水上,宜城在荊州北。荊州有長林縣。江南謂情人為歡。「常」「長」聲相近,蓋樂人誤謂「長」為「常」。《三洲》,商人歌也。商人數行巴陵三江之間,因作
 此歌。《採桑》,因《三洲曲》而生此聲也。《春江花月夜》、《玉樹後庭花》、《堂堂》,並陳後主所作。叔寶常與宮中女學士及朝臣相和為詩,太樂令何胥又善於文詠,採其尤艷麗者以為此曲。《泛龍舟》,隋煬帝江都宮作。餘五曲,不知誰所作也。其辭類皆淺俗,而綿世不易。惜其古曲,是以備論之。其他集錄所不見,亦闕而不載。



 當江南之時,《巾舞》、《白紵》、《巴渝》等衣服各異。梁以前舞人並二八,梁舞省之,咸用八人而巳。令工人平巾幘,緋褲褶。舞四人,碧輕紗衣,
 裙襦大袖,畫雲鳳之狀。漆鬟髻,飾以金銅雜花,狀如雀釵;錦履。舞容閑婉,曲有姿態。沈約《宋書》志江左諸曲哇淫,至今其聲調猶然。觀其政已亂,其俗已淫,既怨且思矣。而從容雅緩,猶有古士君子之遺風。他樂則莫與為比。樂用鐘一架,磬一架,琴一,三弦琴一,擊琴一,瑟一,秦琵琶一,臥箜篌一,築一,箏一,節鼓一,笙二,笛二,簫二,篪二,葉二,歌二。



 自長安已後,朝廷不重古曲,工伎轉缺,能合於管弦者,唯《明君》、《楊伴》、《驍壺》、《春歌》、《秋歌》、《白雪》、《堂堂》、《春
 江花月》等八曲。舊樂章多或數百言。武太后時,《明君》尚能四十言,今所傳二十六言,就之訛失,與吳音轉遠。劉貺以為宜取吳人使之傳習。以問歌工李郎子,李郎子北人,聲調已失,雲學於俞才生。才生,江都人也。今郎子逃,《清樂》之歌闕焉。又聞《清樂》唯《雅歌》一曲,辭典而音雅,閱舊記,其辭信典。漢有《盤舞》,今隸《散樂》部中。又有《幡舞》、《扇舞》,並亡。自周、隋已來,管弦雜曲將數百曲,多用西涼樂,鼓舞曲多用龜茲樂,其曲度皆時俗所知也。惟彈琴家
 猶傳楚、漢舊聲。及《清調》、《瑟調》,蔡邕雜弄,非朝廷郊廟所用,故不載。



 《西涼樂》者,後魏平沮渠氏所得也。晉、宋末,中原喪亂,張軌據有河西,苻秦通涼州,旋復隔絕。其樂具有鐘磬,蓋涼人所傳中國舊樂,而雜以羌胡之聲也。魏世共隋咸重之。工人平巾幘,緋褶。白舞一人,方舞四人。白舞今闕。方舞四人,假髻,玉支釵,紫絲布褶,白大口褲,五彩接袖,烏皮靴。樂用鐘一架,磬一架,彈箏一,NN箏一,臥箜篌一,豎箜篌一,琵琶一,五弦琵琶
 一,笙一,簫一,篳篥一,小篳篥一,笛一,橫笛一,腰鼓一,齊鼓一,簷鼓一,銅拔一,貝一。編鐘今亡。



 《周官》:「韎師掌教《韎樂》,祭祀則帥其屬而舞之,大享亦如之。」《韎》,東夷之樂名也。舉東方,則三方可知矣。又有「鞮鞻氏掌四夷之樂,與其聲歌,祭祀則歙而歌之,宴亦如之。」作先王樂者,貴能包而用之。納四夷之樂者,美德廣之所及也。東夷之樂曰《韎離》,南蠻之樂曰《任》,西戎之樂曰《禁》,北狄之樂曰《昧》。《離》,言陽氣始通,萬物離地而生也。《任》,言陽氣用事,萬物懷任也。《禁》,言陰
 氣始通,禁止萬物之生長也。《昧》,言陰氣用事,萬物眾形暗昧也。其聲不正,作之四門之外,各持其方兵,獻其聲而已。自周之衰,此禮尋廢。



 後魏有曹婆羅門,受龜茲琵琶於商人,世傳其業。至孫妙達,尤為北齊高洋所重,常自擊胡鼓以和之。周武帝聘虜女為後,西域諸國來媵,於是龜茲、疏勒、安國、康國之樂,大聚長安。胡兒令羯人白智通教習,頗雜以新聲。張重華時,天竺重譯貢樂伎,後其國王子為沙門來游,又傳其方音。宋世有高麗、百
 濟伎樂。魏平拓跋,亦得之而未具。周師滅齊,二國獻其樂。隋文帝平陳,得《清樂》及《文康禮畢曲》,列九部伎,百濟伎不預焉。煬帝平林邑國,獲扶南工人及其匏琴,陋不可用,但以《天竺樂》轉寫其聲,而不齒樂部。西魏與高昌通,始有高昌伎。我太宗平高昌,盡收其樂,又造《宴樂》,而去《禮畢曲》。今著令者,惟此十部。雖不著令,聲節存者,樂府猶隸之。德宗朝,又有驃國亦遣使獻樂。



 《高麗樂》,工人紫羅帽,飾以鳥羽,黃大袖,紫羅帶,大口褲,赤皮靴,五色
 絳繩。舞者四人,椎髻於後,以絳抹額,飾以金璫。二人黃裙襦,赤黃褲,極長其袖,烏皮靴,雙雙並立而舞。樂用彈箏一,搊箏一,臥箜篌一,豎箜篌一,琵琶一,義觜笛一,笙一,簫一,小篳篥一,大篳篥一,桃皮篳篥一,腰鼓一,齊鼓一,簷鼓一,貝一。武太后時尚二十五曲,今惟習一曲,衣服亦浸衰敗,失其本風。《百濟樂》,中宗之代,工人死散。岐王範為太常卿,復奏置之,是以音伎多闕。舞二人,紫大袖裙襦,章甫冠,皮履。樂之存者,箏、笛、桃皮篳篥、箜篌、歌。此
 二國,東夷之樂也。



 《扶南樂》,舞二人,朝霞行纏,赤皮靴。隋世全用《天竺樂》,今其存者,有羯鼓、都曇鼓、毛員鼓、簫、笛、篳篥、銅拔、貝。《天竺樂》,工人皁絲布頭巾,白練襦,紫綾褲,緋帔。舞二人,辮發,朝霞袈裟,行纏,碧麻鞋。袈裟,今僧衣是也。樂用銅鼓、羯鼓、毛員鼓、都曇鼓、篳篥、橫笛、鳳首箜篌、琵琶、銅拔、貝。毛員鼓、都曇鼓今亡。《驃國樂》,貞元中,其王來獻本國樂,凡一十二曲,以樂工三十五人來朝。樂曲皆演釋氏經論之辭。此三國,南蠻之樂。



 《高昌樂》,舞二人,
 白襖錦袖,赤皮靴,赤皮帶,紅抹額。樂用答臘鼓一腰鼓一,雞婁鼓一,羯鼓一,簫二,橫笛二,篳篥二,琵琶二,五弦琵琶二,銅角一,箜篌一。箜篌今亡。《龜茲樂》,工人皁絲布頭巾,緋絲布袍,錦袖,緋布褲。舞者四人,紅抹額,緋襖,白褲帑,烏皮靴。樂用豎箜篌一,琵琶一,五弦琵琶一,笙一,橫笛一,簫一,篳篥一,毛員鼓一,都曇鼓一,答臘鼓一,腰鼓一,羯鼓一,雞婁鼓一,銅拔一,貝一。毛員鼓今亡。《疏勒樂》,工人皁絲布頭巾,白絲布褲,錦襟褾,舞二人,白襖,錦袖,赤
 皮靴,赤皮帶。樂用豎箜篌、琵琶、五弦琵琶、橫笛、簫、篳篥、答臘鼓、腰鼓、羯鼓、雞婁鼓。《康國樂》,工人皁絲布頭巾,緋絲布袍,錦領。舞二人,緋襖,錦領袖,綠綾渾襠褲,赤皮靴,白褲帑。舞急轉如風,俗謂之胡旋。樂用笛二,正鼓一,和鼓一,銅拔一。《安國樂》,工人皁絲布頭巾,錦褾領,紫袖褲。舞二人,紫襖,白褲帑,赤皮靴。樂用琵琶、五弦琵琶、豎箜篌、簫、橫笛、篳篥、正鼓、和鼓、銅拔、箜篌。五弦琵琶今亡。此五國,西戎之樂也。



 南蠻、北狄國俗,皆隨發際斷其發,今舞者
 咸用繩圍首,反約發杪,內於繩下。又有新聲河西至者,號胡音聲,與《龜茲樂》、《散樂》俱為時重,諸樂咸為之少寢。



 《北狄樂》,其可知者鮮卑、吐谷渾、部落稽三國,皆馬上樂也。鼓吹本軍旅之音,馬上奏之,故自漢以來,《北狄樂》總歸鼓吹署。後魏樂府始有北歌,即《魏史》所謂《真人代歌》是也。代都時,命掖庭宮女晨夕歌之。周、隋世,與《西涼樂》雜奏。今存者五十三章,其名目可解者六章;《慕容可汗》、《吐谷渾》、《部落稽》、《鉅鹿公主》、《白凈王》、《太子企喻》也。其不可解
 者,咸多「可汗」之辭。按今大角,此即後魏世所謂《簸邏回》者是也,其曲亦多「可汗」之辭。北虜之俗,呼主為可汗。吐谷渾又慕容別種,知此歌是燕、魏之際鮮卑歌。歌辭虜音,竟不可曉。梁有《鉅鹿公主歌辭》,似是姚萇時歌,其辭華音,與北歌不同。梁樂府鼓吹又有《大白凈皇太子》、《小白凈皇太子》、《企喻》等曲。隋鼓吹有《白凈皇太子》曲,與北歌校之,其音皆異。開元初,以問歌工長孫元忠,雲自高祖以來,代傳其業。元忠之祖,受業於侯將軍,名貴昌,並州人也,亦世
 習北歌。貞觀中,有詔令貴昌以其聲教樂府。元忠之家世相傳如此。雖譯者亦不能通知其辭,蓋年歲久遠,失其真矣。絲桐,惟琴曲有胡笳聲大角,金吾所掌。



 《散樂》者,歷代有之,非部伍之聲,俳優歌舞雜奏。漢天子臨軒設樂,舍利獸從西方來,戲於殿前,激水成比目魚,跳躍嗽水,作霧翳日,化成黃龍,修八丈,出水游戲,輝耀日光。繩系兩柱,相去數丈,二倡女對舞繩上,切肩而不傾。如是雜變,總名百戲。江左猶有《高祇紫鹿》、《跂行鱉食》、《齊王卷衣》、《
 綍鼠》、《夏育扛鼎》、《臣象行乳》、《神龜抃戲背負靈岳》、《桂樹白雪》、《畫地成川》之伎。晉成帝咸康七年,散騎侍郎顧臻表曰:「末世之樂,設外方之觀,逆行連倒。四海朝覲帝庭,而足以蹈天,頭以履地,反天地之順,傷彞倫之大。」乃命太常悉罷之。其後復《高祇紫鹿》。後魏、北齊,亦有《魚龍闢邪》、《鹿馬仙車》、《吞刀吐火》、《剝車剝驢》、《種瓜拔井》之戲。周宣帝征齊樂並會關中。開皇初,散遣之。大業二年,突厥單于來朝洛陽宮,煬帝為之大合樂,盡通漢、晉、周、齊之術。胡人大
 駭。帝命樂署肄習,常以歲首縱觀端門內。大抵《散樂》雜戲多幻術,幻術皆出西域,天竺尤甚。漢武帝通西域,始以善幻人至中國。安帝時,天竺獻伎,能自斷手足,刳剔腸胃,自是歷代有之。我高宗惡其驚俗,敕西域關令不令入中國。苻堅嘗得西域倒舞伎。睿宗時,婆羅門獻樂,舞人倒行,而以足舞於極銛刀鋒,倒植於地,低目就刃,以歷臉中,又植於背下,吹篳篥者立其腹上,終曲而亦無傷。又伏伸其手,兩人躡之,施身繞手,百轉無已。漢世
 有橦木伎,又有盤舞。晉世加之以柸,謂之《柸盤舞》。樂府詩云,「妍袖陵七盤」,言舞用盤七枚也。梁謂之《舞盤伎》。梁有《長蹻伎》、《擲倒伎》、《跳劍伎》、《吞劍伎》,今並存。又有《舞輪伎》,蓋今戲車輪者。《透三峽伎》,蓋今《透飛梯》之類也。《高祇伎》,蓋今之戲繩者是也。梁有《獼猴幢伎》,今有《緣竿》,又有《獼猴緣竿》,未審何者為是。又有《弄碗珠伎》、《丹珠伎》。



 歌舞戲,有《大面》、《撥頭》、《踏搖娘》、《窟壘子》等戲。玄宗以其非正聲,置教坊於禁中以處之。《婆羅門樂》,與四夷同列。《婆羅門樂》用
 漆篳篥二,齊鼓一。《散樂》,用橫笛一,拍板一,腰鼓三。其餘雜戲,變態多端,皆不足稱。《大面》出於北齊。北齊蘭陵王長恭,才武而面美,常著假面以對敵。嘗擊周師金墉城下,勇冠三軍,齊人壯之,為此舞以效其指麾擊刺之容,謂之《蘭陵王入陣曲》。《撥頭》出西域。胡人為猛獸所噬,其子求獸殺之,為此舞以像之也。《踏搖娘》,生於隋末。隋末河內有人貌惡而嗜酒,常自號郎中,醉歸必毆其妻。其妻美色善歌,為怨苦之辭。河朔演其曲而被之弦管,因
 寫其妻之容。妻悲訴,每搖頓其身,故號《踏搖娘》。近代優人頗改其制度,非舊旨也。《窟壘子》,亦云《魁壘子》,作偶人以戲,善歌舞。本喪家樂也。漢末始用之於嘉會。齊後主高緯尤所好。高麗國亦有之。



 八音之屬,協於八節。匏,瓠也,女媧氏造。列管於匏上,內簧其中,《爾雅》謂之巢。大者曰竽,小者曰和。竽,煦也,立春之音,煦生萬物也。竽管三十六,宮管在左。和管十三,宮管居中。今之竽、笙,並以木代匏而漆之,無復音矣。荊、梁之南,尚存古制云。



 管三孔
 曰龠,春分之音,萬物振躍而動也。簫,舜所造也。《爾雅》謂之茭。音交大曰絪,二十三管,修尺四寸。笛,漢武帝工丘仲所造也。其元出於羌中。短笛,修尺有咫。長笛、短笛之間,謂之中管。篪,吹孔有觜如酸棗。橫笛,小篪也。漢靈帝好胡笛。五胡亂華,石遵玩之不絕音。《宋書》云:有胡篪出於胡吹,則謂此。梁胡吹歌云:「快馬不須鞭,反插楊柳枝。下馬吹橫笛,愁殺路傍兒。」此歌辭元出北國。之橫笛皆去觜,其加觜者謂之義觜笛。篳篥,本名悲篥,出於胡中,其
 聲悲。亦云:胡人吹之以驚中國馬云。柷,眾也。立夏之音,萬物眾皆成也。方面各二尺餘,旁開員孔,內手於中,擊之以舉樂。敔,如伏虎,背皆有鬣二十七,碎竹以擊其首而逆刮之,以止樂也。舂牘,虛中如桶,無底,舉以頓地如舂杵,亦謂之頓相。相,助也,以節樂也。或謂梁孝王築睢陽城,擊鼓為下杵之節。《睢陽操》用舂牘,後世因之。拍板,長闊如手,厚寸餘,以韋連之,擊以代抃。



 琴,伏羲所造。琴,禁也,夏至之音,陰氣初動,禁物之淫心。五弦以備五聲,
 武王加之為七弦。琴十有二柱,如琵琶。擊琴,柳惲所造。惲嘗為文詠,思有所屬,搖筆誤中琴弦,因為此樂。以管承弦,又以片竹約而束之,使弦急而聲亮,舉竹擊之,以為節曲。瑟,昔者大帝使素女鼓五十弦瑟,悲不能自止,破之為二十五弦。大帝,太昊也。箏,本秦聲也。相傳云蒙恬所造,非也。制與瑟同而弦少。案京房造五音準,如瑟,十三弦,此乃箏也。雜樂箏並十有二弦,他樂皆十有三弦。軋箏,以片竹潤其端而軋之。築,如箏,細頸,以竹擊之,
 如擊琴。《清樂》箏,用骨爪長寸餘以代指。琵琶,四弦,漢樂也。初,秦長城之役,有鞀而鼓之者。及漢武帝嫁宗女於烏孫,乃裁箏、築為馬上樂,以慰其鄉國之思。推而遠之曰琵,引而近之曰琶,言其便於事也。今《清樂》奏琵琶,俗謂之「秦漢子」,圓體修頸而小,疑是弦鞀之遺制。其他皆充上銳下,曲項,形制稍大,疑此是漢制。兼似兩制者,謂之「秦漢」,蓋謂通用秦、漢之法。《梁史》稱侯景之將害簡文也,使太樂令彭雋齎曲項琵琶就帝飲,則南朝似無。
 曲項者,亦本出胡中。五弦琵琶,稍小,蓋北國所出。《風俗通》云:以手琵琶之,因為名。案舊琵琶皆以木撥彈之,太宗貞觀中始有手彈之法,今所謂搊琵琶者是也。《風俗通》所謂以手琵琶之。乃非用撥之義,豈上世固有搊之者耶?阮咸,亦秦琵琶也,而項長過於今制,列十有三柱。武太后時,蜀人蒯朗於古墓中得之。晉《竹林七賢圖》阮咸所彈與此類,因謂之阮咸。咸,晉世實以善琵琶知音律稱。箜篌,漢武帝使樂人侯調所作,以祠太一。或云侯輝
 所作,其聲坎坎應節,謂之坎侯,聲訛為箜篌。或謂師延靡靡樂,非也。舊說亦依琴制。今按其形,似瑟而小,七弦,用撥彈之,如琵琶。豎箜篌,胡樂也,漢靈帝好之。體曲而長,二十有二弦,豎抱於懷,用兩手齊奏,俗謂之擘箜篌。鳳首箜篌,有項如軫。七弦,鄭善子作,開元中進。形如阮咸,其下缺少而身大,旁有少缺,取其身便也。弦十三隔,孤柱一,合散聲七,隔聲九十一,柱聲一,總九十九聲,隨調應律。太一,司馬糸舀開元中進。十二弦,六隔,合散聲十
 二,隔聲七十二。弦散聲應律呂,以隔聲旋相為宮,合八十四調。今編入雅樂宮縣內用之。六弦,史盛作,天寶中進,形如琵琶而長。六弦,四隔,孤柱一,合散聲六,隔聲二十四,柱聲一,總三十一聲,隔調應律。天寶樂,任偃作,天寶中進。類石幢,十四弦,六柱。黃鐘一均足倍七聲,移柱作調應律。



 塤,曛也,立秋之音,萬物將曛黃也。埏土為之,如鵝卵,凡六孔,銳上豐下。大者《爾雅》謂之曰LT。缶,如足盆,古西戎之樂,秦俗應而用之。其形似覆盆,以四杖擊
 之。秦、趙會於澠池,秦王擊缶而歌。八缶,唐永泰初司馬縚進《廣平樂》,蓋八缶具黃鐘一均聲。鐘,黃帝之工垂所造。鐘,種也,立秋之音,萬物種成也。大曰鎛,鎛亦大鐘也。《爾雅》謂之鏞。小而編之曰編鐘,中曰剽,小曰棧。錞于,圓如碓頭,大上小下,縣以籠床,芒渼將之以和鼓。沈約《宋書》云,「今人間時有之」,則宋日非廟庭所用。後周平蜀獲之,斛斯徵觀曰:「錞于也。」依干寶《周禮注》試之,如其言。鐃,木舌,搖之以和鼓。梁有銅磬,蓋今方響之類。方響,以鐵
 為之,修八寸,廣二寸,圓上方下。架如磬而不設業,倚於架上以代鐘磬。人間所用者才三四寸。銅拔,亦謂之銅盤,出西戎及南蠻。其圓數寸,隱起若浮漚,貫之以韋皮,相擊以和樂也。南蠻國大者圓數尺。或謂南齊穆士素所造,非也。鉦,如大銅疊,縣而擊之,節鼓。銅鼓,鑄銅為之,虛其一面,覆而擊其上。南夷扶南、天竺類皆如此。嶺南豪家則有之,大者廣丈餘。磬,叔所造也。磬,勁也,立冬之音,萬物皆堅勁。《書》云,「泗濱浮磬」,言泗濱石可為磬。今磬
 石皆出華原,非泗濱也。登歌磬,以玉為之,《爾雅》謂之芃。鼓,動也,冬至之音,萬物皆含陽氣而動。雷鼓八面以祀天,靈鼓六面以祀地,路鼓四面以祀鬼神。夏後加之以足,謂之足鼓。殷人貫之以柱,謂之楹鼓。周人縣之,謂之縣鼓。後世從殷制建之,謂之建鼓。晉鼓六尺六寸,金奏則鼓之。傍有鼓謂之應鼓,以和大鼓。小鼓有柄曰鞞,搖之以和鼓。大曰鞉。腰鼓,大者瓦,小者木,皆廣首而纖腹,本胡鼓也。石遵好之,與橫笛不去左右。齊鼓,如漆桶,大
 一頭,設齊於鼓面如麝臍,故曰齊鼓。簷鼓,如小甕,先冒以革而漆之。羯鼓,正如漆桶,兩手具擊,以其出羯中,故號羯鼓,亦謂之兩杖鼓。都曇鼓,似腰鼓而小,以槌擊之。毛員鼓,似都曇鼓而稍大。答臘鼓,制廣羯鼓而短,以指揩之,其聲甚震,俗謂之揩鼓。雞婁鼓,正圓,兩手所擊之處,平可數寸。正鼓、和鼓者,一以正,一以和,皆腰鼓也。節鼓,狀如博局,中間員孔,適容其鼓,擊之節樂也。撫拍,以韋為之,實之以糠,撫之節樂也。



 金、石、絲、竹、匏、土、革、木,謂
 之八音。金木之音,擊而成樂。今東夷有管木者,桃皮是也。西戎有吹金者,銅角是也。長二尺,形如牛角。貝,蠡也,容可數升,並吹之以節樂,亦出南蠻。桃皮,卷之以為篳篥。嘯葉,銜葉而嘯,其聲清震,橘柚尤善。四夷絲竹之量,國異其制,不可詳盡。《爾雅》:琴二十弦曰離,瑟二十七弦曰灑。漢世有洞簫,又有管,長尺圍寸而並漆之。宋世有繞梁,似臥箜篌。今並亡矣。今世又有篪,其長盈尋,曰七星,如箏稍小,曰雲和,樂府所不用。



 周天子宮縣,諸侯軒
 縣,大夫曲縣,士特縣。故孔子之堂,聞金石之音;魏絳之家,有鐘磬之聲。秦、漢之際,斯禮無聞。漢丞相田蚡,前庭羅鐘磬,置曲旃。光武又賜東海恭王鐘之樂。即漢世人臣,尚有金石。漢樂歌云,「高張四縣,神來宴饗」,謂宮縣也。制氏在太樂,能記鏗鏘鼓舞。河間王著《樂記》,八佾之舞與制氏不甚相遠,又舞八佾之明文也。《漢儀》云,高廟撞千石之鐘十枚,即《上林賦》所謂「撞千石之鐘,立萬石之鋌鉅」者也。鐘當十二,而此十枚,未識其義。議者皆云漢世不知用宮縣。今案漢章、和世用旋宮,漢世群儒,備言其義,牛弘、祖
 孝孫所由準的也。又河間王博採經籍,與制氏不殊,知漢世之樂,為最備矣。魏、晉已來,但云四廂金石,而不言其禮,或八架,或十架,或十六架。梁武始用二十六架。貞觀初增三十六架,加鼓吹熊羆桉十二於四隅。後魏、周、齊皆二十六架。建德中,復梁三十六架。隋文省。煬帝又復之。



 樂縣,橫曰簨,豎曰。飾簨以飛龍,飾趺以飛廉,鐘以摯獸,磬以摯鳥,上列樹羽,旁垂流蘇,周制也。縣以崇牙,殷制也。飾以博山,後世所加也。宮縣每架金博
 山五,軒縣三。鼓,承以花趺,覆以華蓋,上集翔鷺。隋氏二十架,先置建鼓於四隅,鎛鐘方面各三,依其辰位,雜列編鐘、磬各四架於其間。二十六架,則編鐘十二架,磬亦如之。軒縣九架,鎛鐘三架,在辰、丑、申地,編鐘、磬皆三架。設路鼓二於縣內戌、巳地之北。設柷敔於四隅,舞人立於其中。錞于、鐃、鐸、撫拍、舂牘,列於舞人間。唐禮,天子朝廟用三十六架。高宗成蓬萊宮,充庭七十二架。武後遷都,乃省之。皇后廟及郊祭並二十架,同舞八佾。先聖廟及皇太子廟並九架,舞六佾。縣間設柷敔各
 一,柷於左,敔於右。錞于、撫拍、頓相、鐃、鐸,次列於路鼓南。舞人列於縣北。登歌二架,登於堂上兩楹之前。編鐘在東,編磬在西。登歌工人坐堂上,竹人立堂下,所謂「琴瑟在堂,竽笙在庭」也。殿庭加設鼓吹於四隅。



 宴享陳《清樂》、《西涼樂》。架對列於左右廂,設舞筵於其間。舊皇后庭但設絲管,大業尚侈,始置鐘磬,猶不設鎛鐘,以鎛磬代。武太后稱制,用鐘,因而莫革。樂縣,庭廟以五彩雜飾,軒縣以硃,五郊則各從其方色。每先奏樂三日,太樂令宿設縣於庭,
 其日率工人入居其次。協律郎舉麾,樂作;僕麾,樂止。文舞退,武舞進。若常享會,先一日具坐、立部樂名封上,請所奏御注而下。及會,先奏坐部伎,次奏立部伎,次奏蹀馬,次奏《散樂》而畢矣。



 廣明初,巢賊干紀,輿駕播遷,兩都覆圮,宗廟悉為煨燼,樂工淪散,金奏幾亡。及僖宗還宮,購募鐘縣之器,一無存者。昭宗即位,將親謁郊廟,有司請造樂縣,詢於舊工,皆莫知其制度。修奉樂縣使宰相張浚悉集太常樂胥詳酌,竟不得其法。時太常博士殷
 盈孫深於典故,乃案《周官考工記》之文,究其欒、銑、于、鼓、鉦、舞、甬之法,沉思三四夕,用算法乘除,鎛鐘之輕重高低乃定。懸下編鐘,正黃鐘九寸五分,下至登歌倍應鐘三寸三分半,凡四十八等。口項之量,徑衡之圍,悉為圖,遣金工依法鑄之,凡二百四十口。鑄成,張浚求知聲者處士蕭承訓、梨園樂工陳敬言與太樂令李從周,令先校定石磬,合而擊拊之,八音克諧,觀者聳聽。浚既進呈,昭宗陳於殿庭以試之。時以宗廟焚毀之後,修奉不及,
 乃權以少府監為太廟。其庭甚狹,議者論縣樂之架不同。浚奏議曰:



 臣伏準舊制,太廟含元殿並設宮縣三十六架,太清宮、南北郊、社稷及諸殿庭,並二十架。今修奉樂懸,太廟合造三十六架,臣今參議,請依古禮用二十架。伏自兵興已來,雅樂淪缺,將為修奉,事實重難。變通宜務於酌中,損益當循於寧儉。臣聞諸舊史,昔武王定天下,至周公相成王,始暇制樂。魏初無樂器及伶人,後稍得登歌食舉之樂。明帝太寧末,詔增益之。咸和中,
 鳩集遺逸,尚未有金石之音。至孝武太元中,四廂金石始備,郊祀猶不舉樂。宋文帝元嘉九年,初調金石。二十四年,南郊始設登歌,廟舞猶闕。孝武孝建中,有司奏郊廟宜設備樂,始為詳定。故後魏孝文太和初,司樂上書,陳樂章有闕,求集群官議定,廣修器數,正立名品。詔雖行之,仍有殘缺。隋文踐祚,太常議正雅樂,九年之後,惟奏黃鐘一宮,郊廟止用一調。據禮文,每一代之樂,二調並奏,六代之樂,凡十二調。其餘聲律,皆不復通。高祖受隋禪,軍國多務,未遑改創,
 樂府尚用隋氏舊文。武德九年,命太常考正雅樂。貞觀二年,考畢上奏。蓋其事體大,故歷代不能速成。



 伏以俯逼郊天,式修雅樂,必將集事,須務相時。今者帑藏未充,貢奉多闕,凡闕貨力,不易方圓,制度之間,亦宜撙節。臣伏惟《儀禮》宮懸之制,陳鎛鐘二十架,當十二辰之位。甲、丙、庚、壬,各設編鐘一架;乙、丁、辛、癸,各設編磬一架,合為二十架。樹建鼓於四隅。當乾、坤、艮、巽之位,以象二十四氣。宗廟、殿庭、郊丘、社稷,皆用此制,無聞異同。周、漢、魏、晉、
 宋、齊六朝,並只用二十架。隋氏平陳,檢梁故事,乃設三十六架。國初因之不改。高宗皇帝初成蓬萊宮,充庭七十二架,尋乃省之。則簨架數太多,本近於侈。止於二十架,正協禮經。兼今太廟之中,地位甚狹,百官在列,萬舞充庭,雖三十六架具存,亦施為不得。廟庭難容,未易開廣,樂架不可重沓鋪陳。今請依周、漢、魏、晉、宋、齊六代故事,用二十架。



 從之。



 古制,雅樂宮縣之下,編鐘四架,十六口。近代用二十四口,正聲十二,倍聲十二,各有律呂,
 凡二十四聲。登歌一架,亦二十四鐘。雅樂淪滅,至是復全。



\end{pinyinscope}