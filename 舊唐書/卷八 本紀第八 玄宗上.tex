\article{卷八 本紀第八 玄宗上}

\begin{pinyinscope}

 玄
 宗至道大聖大明孝皇帝諱
 隆
 基,睿宗第三子也,母曰昭成順聖皇后竇氏。垂拱元年秋八月戊寅,生於東都。性英斷多藝,尤知音律,善八分書。儀範偉麗,有非常
 之表。



 三年閏七月丁卯,封楚王。天授三年十月戊戌,出閣,開府置官屬,年始七歲。朔望車騎至朝堂,金吾將軍武懿宗忌上嚴整,訶排儀仗,因欲折之。上叱之曰:「吾家朝堂,干汝何事?敢迫吾騎從!」則天聞而特加寵異之。尋卻入閣。長壽二年臘月丁卯,改封臨淄郡王。聖歷元年,出閣,賜第於東都積善坊。大足元年,從幸西京,賜宅於興慶坊。長安中,歷右衛郎將、尚輦奉御。



 神龍元年,遷衛尉少卿。景龍二年四月,兼潞州別駕。十二月,加銀青光
 祿大夫。州境有黃龍白日升天。嘗出畋,有紫雲在其上,後從者望而得之。前後符瑞凡一十九事。四年,中宗將祀南郊,來朝京師。將行,使術士韓禮筮之,蓍一莖孑然獨立。禮驚曰:「蓍立,奇瑞非常也,不可言。」屬中宗末年,王室多故,上常陰引材力之士以自助。上所居宅外有水池,浸溢頃餘,望氣者以為龍氣。四年四月,中宗幸其第,因游其池,結彩為樓船,令巨象踏之。



 至六月,中宗暴崩,韋后臨朝稱制。韋溫、宗楚客、紀處訥等謀傾宗社,以睿
 宗介弟之重,先謀不利。道士馮道力、處士劉承祖皆善於占兆,詣上布誠款。上所居里名隆慶,時人語訛以「隆」為「龍」;韋庶人稱制,改元又為唐隆,皆符御名。上益自負,乃與太平公主謀之,公主喜,以子崇簡從。上乃與崇簡、朝邑尉劉幽求、長上折沖麻嗣宗、押萬騎果毅葛福順李仙鳧、寶昌寺僧普潤等定策誅之。或曰:「先啟大王。」上曰:「我拯社稷之危,赴君父之急,事成福歸於宗社,不成身死於忠孝,安可先請,憂怖大王乎!若請而從,是王與
 危事;請而不從,則吾計失矣。」遂以庚子夜率幽求等數十人自苑南入,總監鐘紹京又率丁匠百餘以從。分遣萬騎往玄武門殺羽林將軍韋播、高嵩,持首而至,眾歡叫大集。攻白獸、玄德等門,斬關而進,左萬騎自左入,右萬騎自右入,合於凌煙閣前。時太極殿前有宿衛梓宮萬騎,聞噪聲,皆披甲應之。韋庶人惶惑走入飛騎營,為亂兵所害。於是分遣誅韋氏之黨,比明,內外討捕,皆斬之。乃馳謁睿宗,謝不先啟請之罪。睿宗遽前抱上而泣曰:「
 宗社禍難,由汝安定,神祇萬姓,賴汝之力也。」拜殿中監、同中書門下三品,兼押左右萬騎,進封平王。



 睿宗即位,與侍臣議立皇太子,僉曰:「除天下之禍者,享天下之福;拯天下之危者,受天下之安。平王有聖德,定天下,又聞成器已下咸有推讓,宜膺主鬯,以副群心。」睿宗從之。丙午,制曰:



 舜去四兇而功格天地,武有七德而戡定黎人,故知有大勛者必受神明之福,仗高義者必為匕鬯之主。朕恭臨寶位,亭育寰區,以萬物之心為心,以兆人之
 命為命。雖承繼之道,咸以塚嫡居尊;而無私之懷,必推功業為首。然後可保安社稷,永奉宗祧。第三子平王基孝而克忠,義而能勇。比以朕居籓邸,虔守國彞,貴戚中人,都無引接。群邪害正,兇黨實繁,利口巧言,讒說罔極。韋溫、延秀,朋黨競起;晉卿、楚客,交構其間。潛結回邪,排擠端善,潛貯兵甲,將害朕躬。基密聞其期,先難奮發,推身鞠弭,眾應如歸,呼吸之間,兇渠殄滅。安七廟於幾墜,拯群臣於將殞。方舜之功過四,比武之德逾七。靈祇望
 在,昆弟樂推。一人元良,萬邦以定。為副君者,非此而誰?可立為皇太子。有司擇日,備禮冊命。



 七月己巳,睿宗御承天門,皇太子詣朝堂受冊。是日有景雲之瑞,改元為景雲,大赦天下。



 二年,又制曰:「惟天生丞人,牧以元後;維皇立國,副以儲君。將以保綏家邦,安固後嗣者也。朕纂承洪業,欽奉寶圖,夜分不寢,日昃忘倦。茫茫四海,懼一人之未周;蒸蒸萬姓,恐一物之失所。雖卿士竭誠,守宰宣化,緬懷庶域,仍未小康。是以求下人之變風,遵先朝
 之故事。皇太子基仁孝因心,溫恭成德,深達禮體,能辨皇猷,宜令監國,俾爾為政。其六品以下除授及徒罪已下,並取基處分。」延和元年六月,兇黨因術人聞睿宗曰:「據玄象,帝座及前星有災,皇太子合作天子,不合更居東宮矣。」睿宗曰:「傳德避災,吾意決矣。」七月壬午,制曰:



 朕以寡昧,虔奉鴻休,本殊王季之賢,早達延陵之節。昔在聖歷,已讓皇嗣之尊;爰暨神龍,終辭太弟之授。豈唯衣冠所睹,抑亦兆庶咸知。頃屬國步不夷,時艱主幼,大業
 有綴旒之懼,寶位深墜地之憂,議迫公卿,遂司契篆,日慎一日,以至於今。一紀之勞,勤亦至矣;萬方之俗,化漸行矣。將成宿願,脫屣寰區。昔堯之禪舜,唯能是與,禹以命啟,匪私其親,神器之重,允歸公授。皇太子基有大功於天地,定阽危於社稷,溫文既習,聖敬克躋。委之監國,已移歲年,時政益明,庶工惟序。朕之知子,庶不負時,歷數在躬,宜陟元後。可令即皇帝位,有司擇日授冊。朕方比跡洪古,希風太皇,神與化游,思與道合,無為無事,豈
 不美歟!王公百僚,宜識朕意。



 上意惶懼,馳見叩頭,請所以傳位之旨。睿宗曰:「吾因汝功業得宗社。今帝座有眚,思欲遜避,唯聖德大勛,始轉禍為福。易位於汝,吾知晚矣。」上始居武德殿視事,三品以下除授及徒罪皆自決之。



 先天二年七月三日,尚書左僕射竇懷貞、侍中岑羲、中書令蕭至忠崔湜、雍州長史李晉、左羽林大將軍常元楷、右羽林將軍李慈等與太平公主同謀,期以其月四日以羽林軍作亂。上密知之,因以中旨告岐王範、薛
 王業、兵部尚書郭元振、將軍王毛仲,取閑廄馬及家人三百餘人,率太僕少卿李令問、王守一、內侍高力士、果毅李守德等親信十數人,出武德殿,入虔化門。梟常元楷、李慈於北闕。擒賈膺福、李猷於內客省以出。執蕭至忠、岑羲於朝,皆斬之。睿宗明日下詔曰:「朕將高居無為,自今軍國政刑一事已上,並取皇帝處分。」上御承天門樓,下制曰:



 朕承累聖之洪休,荷重光之積慶。昔因多難,內屬構屯,寶位深墜地之憂,神器有綴旒之懼。事殷家
 國,義感神祇,吟嘯風雲,龔行雷電,致君親於堯、舜,濟黔首於休和。遂以孟秋,允升儲貳;旋承內禪,繼體宸居。拜首之請空勤,讓立之誠莫展,恭臨億兆,二載於茲。上稟聖謨,下凝庶績,八荒同軌,瀛海無波。不謂奸慝潛謀,蕭墻竊發。逆賊竇懷貞等並以庸妄,權齒朝廷,毫發之效未申,丘山之釁乃積,共成梟獍,將肆奸回。太上皇聖斷宏通,英謀獨運,命朕率岐王範、薛王業等躬事誅鋤。齊斧一麾,兇渠盡殪。太陽朗耀,澄氛靄於天衢;高風順時,
 厲肅殺於秋序。神靈協贊,夷夏相歡,四族之慝既清,七百之祚方永。爰承後命,載闡休期,總軍國之大猷,施雲雨之鴻澤。承乾之道,既光被於無垠;作解之恩,思式覃於品物。當與億兆,同此惟新。可大赦天下,大闢罪已下咸赦除之。加邠王守禮實封三百戶,宋王成器、申王成義各加實封一千戶,岐王範、薛王業各加實封七百戶。文武官三品以下賜爵一級,四品已下各加一階。內外官人被諸道按察使及御史所摘伏,咸宜洗滌;選日依
 次敘用。



 丁卯,崔湜、盧藏用除名,長流嶺表。壬申,王琚為銀青光祿大夫、戶部尚書,封趙國公,實封三百戶;姜皎銀青光祿大夫、工部尚書,封楚國公,實封五百戶;李令問銀青光祿大夫、殿中監,實封三百戶;王毛仲輔國大將軍、左武衛大將軍、檢校內外閑廊兼知監牧使、霍國公,實封五百戶;王守一銀青光祿大夫、太常卿同正員,進封晉國公,實封五百戶:並賞其定策功。琚、皎、令問固讓。癸丑,中書侍郎陸象先為益州大都督府長史兼劍
 南道按察兵馬使,尚書左丞張說為檢校中書令。甲戌,令毀天樞,取其銅鐵充軍國雜用。庚辰,王琚為中書侍郎,加實封二百戶;姜皎殿中監,仍充內外閑廄使,加實封二百戶;李令問殿中少監、知尚食事,加實封二百戶。己丑,周孝明高皇帝依舊追贈太原王,宜去帝號;孝明皇后宜稱太原王妃;昊陵、順陵並稱太原王及妃墓。



 八月壬辰,封州流人劉幽求為尚書左僕射、知軍國重事、徐國公,仍依舊實封七百戶。制曰:「凡有刑人,國家常法。
 掩骼埋胔,王者用心。自今已後,輒有屠割刑人骨肉者,依法科殘害之罪。」九月,司空兼揚州大都督、宋王成器為太尉兼揚州大都督,益州大都督兼右金吾大將軍、申王成義為司徒兼益州大都督,單于大都護兼左金吾大將軍、邠王守禮為司空。癸丑,封華嶽神為金天王。



 九月丁卯,宋王成器為開府儀同三司,尚書左僕射劉幽求同中書門下三品,檢校中書令、燕國公張說為中書令,特進王仁皎為開府儀同三司。己卯,宴王公百僚
 於承天門,令左右於樓下撒金錢,許中書門下五品已上官及諸司三品已上官爭拾之,仍賜物有差。郭元振兼御史大夫。丙戌,又置右御史臺。冬十一月甲申,幸新豐之溫湯。癸卯,講武於驪山。兵部尚書、代國公郭元振坐虧失軍容,配流新州;給事中、攝太常少卿唐紹以軍禮有失,斬於纛下。甲辰,畋獵於渭川。同州刺史、梁國公姚元之為兵部尚書、同中書門下三品。乙巳,至自溫湯。十一月乙丑,幽求兼知侍中。戊子,上加尊號為開元神
 武皇帝。十二月庚寅朔,大赦天下,改元為開元,內外官賜勛一轉。改尚書左、右僕射為左、右丞相,中書省為紫微省,門下省為黃門省,侍中為監。雍州為京兆府,洛州為河南府,長史為尹,司馬為少尹。國初以來宰相及食實封功臣子孫,一應沉翳未承恩者,令量才擢用。開元元年十二月己亥,禁斷潑寒胡戲。癸丑,尚書左丞相兼黃門監劉幽求為太子少保,罷知政事;紫微令張說為相州刺史。甲寅,門下侍郎盧懷慎同紫微黃門平章事。



 二年春正月,關中自去秋至於是月不雨,人多饑乏,遣使賑給。制求直諫昌言弘益政理者。名山大川,並令祈祭。丙寅,紫微令姚崇上言請檢責天下僧尼,以偽濫還俗者二萬餘人。甲申,並州大都督府長史兼檢校左衛大將軍薛訥同紫微黃門三品,仍總兵以討奚、契丹。二月,突厥默啜遣其子同俄特勤率眾寇北庭都護府,右驍衛將軍郭虔瓘擊敗之,斬同俄於城下。己酉,以旱,親錄囚徒。改太史監罷隸秘書省。閏月癸亥,令道士、女冠、僧
 尼致拜父母。丁卯,復置十道按察使。己未,突厥默啜妹婿火拔頡利發石失畢與其妻來奔,封燕山郡王,授左衛員外大將軍。紫微侍郎、趙國公王琚左授澤州刺史,賜實封一百戶,餘並停。丁亥,劉幽求為睦州刺史。



 三月甲辰,青州刺史、郇國公韋安石為沔州別駕;太子賓客、逍遙公韋嗣立為岳州別駕;特進致仕李嶠先隨子在袁州,又貶滁州別駕:並員外置。去年九月有詔毀天樞,至今春始。夏五月辛亥,黃門監魏知古工部尚書,罷知
 政事。六月丁巳,開府儀同三司、宋王成器為岐州刺史,司徒、申王成義為豳州刺史,司空、邠王守禮為虢州刺史:委務於上佐。內出珠玉錦繡等服玩,又令於正殿前焚之。乙丑,兵部尚書致仕、韓國公張仁願卒。



 七月,薛訥與副將杜賓客、崔宣道等總兵六萬自檀州道遇賊於灤河,為賊所敗。訥等屏甲遁歸,減死,除名為庶人。辛未,光祿卿竇希瑊為太子太傅。房州刺史、襄王重茂薨於梁州,謚曰殤帝。丙午,昭文館學士柳沖、太子左庶子劉
 子玄刊定《姓族系錄》二百卷,上之。以興慶裏舊邸為興慶宮。諸王傅並停。京官所帶跨巾算袋,每朝參日著,外官衙日著,餘日停。吐蕃寇臨洮軍,又游寇蘭州、渭州,掠群牧,起薛訥攝左羽林將軍、隴右防禦使,率杜賓客、郭知運、王晙、安思順以御之。太常卿、岐王範為華州刺史,秘書監、薛王業為同州刺史。



 八月戊午,西天竺國遣使獻方物。九月戊申,幸新豐之溫泉。甲寅,制曰:「自古帝王皆以厚葬為誡,以其無益亡者,有損生業故也。近代以來,共行奢靡,遞
 相仿效,浸成風俗,既竭家產,多至凋弊。然則魂魄歸天,明精誠之已遠;卜宅於地,蓋思慕之所存。古者不封,未為非達。且墓為真宅,自便有房,今乃別造田園,名為下帳,又冥器等物,皆競驕侈。失禮違令,殊非所宜;戮尸暴骸,實由於此。承前雖有約束,所司曾不申明,喪葬之家,無所依準。宜令所司據品令高下,明為節制:冥器等物,仍定色數及長短大小;園宅下帳,並宜禁絕;墳墓塋域,務遵簡儉;凡諸送終之具,並不得以金銀為飾。如有違
 者,先決杖一百。州縣長官不能舉察,並貶授遠官。」冬十月戊午,至自溫泉。薛訥破吐蕃於渭州西界武階驛,斬首一萬七十級,馬七萬七匹,牛羊四萬頭。豐安軍使郎將、判將軍王海賓先鋒力戰,死之。十一月庚寅,葬殤帝於武功西原。十二月乙丑,封皇子嗣真為鄫王,嗣初為鄂王,嗣玄為鄄王。時右威衛中郎將周慶立為安南市舶使,與波斯僧廣造奇巧,將以進內。監選使、殿中侍御史柳澤上書諫,上嘉訥之。



 三年春正月丁亥,立郢王嗣謙為皇太子,降死罪已下,大酺三日。癸卯,黃門侍郎盧懷慎為檢校黃門監。甲辰,工部尚書魏知古卒。二月,禁斷天下採捕鯉魚。十姓部落左廂五咄六啜、右廂五弩失畢五俟斤,及高麗莫離支高文簡、都督鳷跌思太等,各率其眾自突厥相繼來奔,前後總二千餘帳。析許州、唐州置仙州。



 夏四月,岐王範兼虢州刺史,薛王業兼幽州刺史。六月,山東諸州大蝗,飛則蔽景,下則食苗稼,聲如風雨。紫微令姚崇奏請
 差御史下諸道,促官吏遣人驅撲焚瘞,以救秋稼,從之。是歲,田收有獲,人不甚饑。秋七月,刑部尚書李日知卒。冬十月甲寅,制曰:「朕聽政之暇,常覽史籍,事關理道,實所留心,中有闕疑,時須質問。宜選耆儒博學一人,每日入內侍讀。」以光祿卿馬懷素為左散騎常侍,與右散騎常侍褚無量並充侍讀。甲子,幸郿縣之鳳泉湯。十一月已卯,至自鳳泉湯。乙酉,幸新豐之溫湯。丁亥,妖賊崔子巖等入相州作亂。戊子,州司討平之。甲午,至自溫湯。十二月庚午,以軍
 器使為軍器監,置官員。是冬無雪。



 四年春正月癸未,尚衣奉御長孫昕恃以皇后妹婿,與其妹夫楊仙玉毆擊御史大夫李傑,上令朝堂斬昕以謝百官。以陽和之月不可行刑,累表陳請,乃命杖殺之。丁亥,宋王成器、申王成義以「成」字犯昭成皇后謚號,於是成器改名憲,成義改為捴。刑部尚書、中山郡公李乂卒。



 二月丙辰,幸新豐之溫湯。丁卯,至自溫湯。以關中旱,遣使祈雨於驪山,應時澍雨。令以少牢致祭,仍禁斷樵
 採。夏六月庚寅,月蝕既。癸亥,太上皇崩於百福殿。辛未,京師、華、陜三州大風拔木。癸酉,突厥可汗默啜為九姓拔曳固所殺,斬其首送於京師。默啜兄子小殺繼立為可汗。是夏,山東、河南、河北蝗蟲大起,遣使分捕而瘞之。其回紇、同羅、霫、勃曳固、僕固五部落來附,於大武軍北安置。秋七月丙申,分巂、雅二州置黎州。



 冬十月癸丑,戶部尚書、新除太子詹事畢構卒。庚午,葬睿宗大聖貞皇帝於橋陵。以同州蒲城縣為奉先縣,隸京兆府。十一月
 丁亥,徙中宗神主於西廟。甲午,尚書左丞源乾曜為黃門侍郎、同紫微黃門平章事。辛丑,黃門監兼吏部尚書盧懷慎卒。十二月乙卯,幸新豐之溫湯。其夜,定陵寢殿災。乙丑,至自溫湯。尚書、廣平郡公宋璟為吏部尚書兼黃門監,紫微侍郎、許國公蘇頲同紫微黃門平章事。兵部尚書兼紫微令、梁國公姚崇為開府儀同三司,黃門侍郎、安陽男源乾曜守京兆尹,並罷知政事。停十道採訪使。



 五年春正月壬寅朔,上以喪制不受朝賀。癸卯寅時,太廟屋壞,移神主於太極殿,上素服避正殿,輟朝五日,日躬親祭享。辛亥,幸東都。戊辰,昏霧四塞。



 二月甲戌,至自東都,大赦天下,唯謀反大逆不在赦限,餘並宥之。河南百姓給復一年,河南、河北遭澇及蝗蟲處,無出今年地租。武德、貞觀以來勛臣子孫無位者,訪求其後奏聞;有嘉遁幽棲養高不仕者,州牧各以名薦。三月庚戌,於柳城依舊置營州都督府。丁巳,以辛景初女封為固安縣
 主,妻於奚首領饒樂郡王大酺。



 夏四月己丑,皇帝第九子嗣一薨,追封夏王,謚曰悼。甲午,以則天拜洛受圖壇及碑文並顯聖侯廟,初因唐同泰偽造瑞石文所建,令即廢毀。六月壬午,鞏縣暴雨連月,山水泛濫,毀郭邑廬舍七百餘家,人死者七十二。汜水同日漂壞近河百姓二百餘家。秋七月甲子,詔曰:「古者操皇綱執大象者,何嘗不上稽天道,下順人極,或變通以隨時,爰損益以成務。且衢室創制,度堂以筵。因之以禮神,是光孝德;用之
 以布政,蓋稱視朔,先王所以厚人倫感天地者也。少陽有位,上帝斯歆,此則神貴於不黷,禮殷於至敬。今之明堂,俯鄰宮掖,比之嚴祝,有異肅恭,茍非憲章,將何軌物?由是禮官博士公卿大臣廣參群議,欽若前古,宜存露寢之式,用罷闢雍之號。可改為乾元殿,每臨御依正殿禮。」九月壬寅,改紫微省依舊為中書省,黃門省為門下省,黃門監為侍中。



 冬十月丙子,京師修太廟成。丁丑,詔以故越王貞死非其罪,封故許王男琳為嗣越王,以繼
 其後。戊寅,祔神主於太廟。十一月己亥,契丹首領松漠郡王李失活來朝,以宗女為永樂公主以妻之。司徒兼鄧州刺史、申王捴兼虢州刺史。



 六年春正月丙辰朔,以未經大祥,不受朝賀。辛酉,禁斷天下諸州惡錢,行二銖四分已上好錢,不堪用者並即銷破復鑄。將作大匠韋湊上疏,請遷孝敬神主,別立義宗廟。以太子少師兼許州刺史、岐王範兼鄭州刺史。二月甲戌,禮幣徵嵩山隱士盧鴻。夏五月乙未,孝敬哀皇
 后祔於恭陵。契丹松漠郡王李失活卒。六月甲申,瀍水暴漲,壞人廬舍,溺殺千餘人。乙酉,制以故侍中桓彥範敬暉、故中書令兼吏部尚書張柬之、故特進崔玄暐、故中書令袁恕己配饗中宗廟庭,故司空蘇瑰、故左丞相太子少保郴州刺史劉幽求配饗睿宗廟庭。秋七月已未,秘書監馬懷素卒。九月乙未,遣工部尚書劉知柔持節往河南道存問。冬十月丙申,車駕還京師。



 十一月辛卯,至自東都。丙申,親謁太廟,回御承天門,詔:「七廟元皇
 帝已上三祖枝孫有失官序者,各與一人五品京官。內外官三品已上有廟者,各賜物三十匹,以備修祭服及俎豆。」賜文武官有差。乙巳,傳國八璽依舊改稱寶,符璽郎為符寶郎。十二月,以開府儀同三司兼澤州刺史、宋王憲為涇州刺史,司徒兼虢州刺史、申王捴為絳州刺史,以太子少師兼鄭州刺史、岐王範為岐州刺史,以太子少保兼衛州刺史、薛王業為虢州刺史。



 七年春正月,吐蕃遣使朝貢。三月丁酉,左武衛大將軍、
 霍國公王毛仲加特進。渤海靺鞨郡王大祚榮死,其子武藝嗣位。夏四月癸酉,開府儀同三司王仁皎薨。五月已丑朔,日有蝕之。秋七月丙辰,制以亢陽日久,上親錄囚徙,多所原免。諸州委州牧、縣宰量事處置。八月癸丑,敕:「周公制禮,歷代不刊;子夏為傳,孔門所受。逮及諸家,或變例。與其改作,不如好古。諸服紀宜一依舊文。」九月甲子,改昭文館依舊為弘文館。宋王憲徙封寧王。冬十月,於東都來庭縣廨置義宗廟。辛卯,幸新豐之溫湯。癸
 卯,至自溫湯。戊寅,皇太子詣國學行齒胄禮,陪位官及學生賜物有差。十二月丙戌,置弘文、崇文兩館讎校書郎官員。



 八年春正月甲子朔,皇太子加元服。乙丑,皇太子謁太廟。丙寅,會百官於太極殿,賜物有差。壬申,右散騎常侍、舒國公褚元量卒。己卯,侍中宋璟為開府儀同三司,中書侍郎蘇頲為禮部尚書,並罷知政事。京兆尹源乾曜為黃門侍郎,並州大督府長史張嘉貞為中書侍郎,
 並同中書門下平章事。二月丁酉,皇子敏薨,追封懷王,謚曰哀。



 夏五月丁卯,源乾曜為侍中,張嘉貞為中書令。南天竺國遣使獻五色鸚鵡。



 六月壬寅夜,東都暴雨,穀水泛漲。新安、澠池、河南、壽安、鞏縣等廬舍蕩盡,共九百六十一戶,溺死者八百一十五人。許、衛等州掌閑番兵溺者千一百四十八人。秋九月,突厥欲谷寇甘、涼等州,涼州都督楊敬述為所敗,掠契苾部落而歸。以御史大夫王晙為兵部尚書兼幽州都督,黃門侍郎韋抗為
 御史大夫、朔方總管以御之。甲子,太子少師兼岐州刺史、岐王範兼太子太傅,太子少保兼虢州刺史、薛王業為太子太保,餘並如故。



 冬十月辛巳,幸長春宮。壬午,畋於下邽。十一月乙丑,至自長春宮。辛未,突厥寇涼州,殺人掠羊馬數萬計而去。



 九年春正月丙辰,改蒲州為河中府,置中都。丙寅,幸新豐之溫湯。



 夏四月庚寅,蘭池州叛胡顯首偽稱葉護康待賓、安慕容,為多覽殺大將軍何黑奴,偽將軍石神奴、
 康鐵頭等,據長泉縣,攻陷六胡州。兵部尚書王晙發隴右諸軍及河東九姓掩討之。甲戌,上親策試應制舉人於含元殿,謂曰:「古有三道,今減二策。近無甲科,朕將存其上第,務收賢俊,用寧軍國。」仍令有司設食。



 秋七月戊申,罷中都,依舊為蒲州。己酉,王晙破蘭池州叛胡,殺三萬五千騎。丙辰,揚、潤等州暴風,發屋拔樹,漂損公私船舫一千餘只。辛酉,集諸酋長,斬康待賓。先天中,重修三九射禮,至是,給事中許景先抗疏罷之。



 九月己巳朔,日
 有蝕之。丁未,開府儀同三司、梁國公姚崇薨。丁巳,御丹鳳樓,宴突厥首領。庚申,幸中書省。癸亥,右羽林將軍、權檢校並州大都督府長史、燕國公張說為兵部尚書、同中書門下三品。冬十一月丙辰,左散騎常侍元行沖上《群書目錄》二百卷,藏之內府。庚午冬至,大赦天下,內外官九品己上加一階,三品已上加爵一等。自六月二十日、七月三日匡衛社稷食實封功臣,坐事削除官爵,中間有生有死,並量加收贈。致仕官合佩魚者聽其終身。賜
 酺三日。十二月乙酉,幸新豐之溫湯。壬午,至自溫湯。是冬無雪。



 十年春正月丁巳,幸東都。甲子,省王公已下視品官參佐及京三品已上官伏身職員。乙丑,停天下公廨錢,其官人料以稅戶錢充,每月準舊分例數給。戊申,內外官職田,除公廨田園外,並官收,給還逃戶及貧下戶欠丁田。二月戊寅,至東都。三月戊申,詔自今內外官有犯贓至解免已上,縱逢赦免,並終身勿齒。



 夏四月丁酉,封契
 丹首領松漠都督李鬱於為松漠郡王,奚首領饒樂都督李魯蘇為饒樂郡王。五月,東都大雨,伊、汝等水泛漲,漂壞河南府及許、汝、仙、陳等州廬舍數千家,溺死者甚眾。閏五月壬申,兵部尚書張說往朔方軍巡邊。戊寅,敕諸番充質宿衛子弟,並放還國。六月辛丑,上訓注《孝經》,頒於天下。癸卯,以餘姚縣主女慕容氏為燕郡公主,出降奚首領饒樂郡王李魯蘇。己巳,增置京師太廟為九室,移孝和皇帝神主以就正廟。秋八月丙戌,嶺南按察
 使裴伷先上言安南賊帥梅叔鸞等攻圍州縣,遣驃騎將軍兼內侍楊思勖討之。丁亥,遣戶部尚書陸象先往汝、許等州存撫賑給。丙申,博、棣等州黃河堤破,漂損田稼。



 九月,張說擒康願子於木盤山。詔移河曲六州殘胡五萬餘口於許、汝、唐、鄧、仙、豫等州,始空河南朔方千里之地。甲戌,秘書監、楚國公姜皎坐事,詔杖之六十,配流欽州,死於路。都水使者劉承祖配流雷州。乙亥,制曰:「朕君臨宇內,子育黎元。內修睦親,以敘九族;外協庶政,以
 濟兆人。勛戚加優厚之恩,兄弟盡友于之至。務崇敦本,克慎明德。今小人作孽,已伏憲章,恐不逞之徒,猶未能息。凡在宗屬,用申懲誡:自今已後,諸王、公主、駙馬、外戚家,除非至親以外,不得出入門庭,妄說言語。所以共存至公之道,永協和平之義,克固籓翰,以保厥休。貴戚懿親,宜書座右。」又下制,約百官不得與卜祝之人交游來往。乙卯夜,京兆人權梁山偽稱襄王男,自號光帝,與其黨權楚璧,以屯營兵數百人,自景風、長樂等門斬關入宮城構
 逆。至曉兵敗,斬梁山,傳首東都。廢河陽柏崖倉。



 冬十月癸丑,乾元殿依舊題為明堂。甲寅,幸壽安之故興泰宮。畋獵於土宜川。庚申,至自興泰宮。波斯國遣使獻獅子。十一月乙未,初令宰相共食實封三百戶。十二月,停按察使。



 十一年春正月丁卯,降都城見楚囚徒,流、死罪減一等,餘並原之。己巳,北都巡狩,敕所至處存問高年、鰥寡惸獨、征人之家;減流、死罪一等,徒以下放免。庚辰,幸並州、
 潞州,宴父老,曲赦大闢罪已下,給復五年。別改其舊宅為飛龍宮。辛卯,改並州為太原府,官吏補授,一準京兆、河南兩府。百姓給復一年,貧戶復二年,元從戶復五年。武德功臣及元從子孫,有才堪文武未有官者,委府縣搜揚,具以名薦。上親制《起義堂頌》及書,刻石紀功於太原府之南街。戊申,次晉州。壇場使、中書令張嘉貞貶為幽州刺史。壬子,祠后土於汾陰之脽上,升壇行事官三品已上加一爵,四品已上加一階,陪位官賜勛一轉。改
 汾陰為寶鼎縣。癸亥,兵部尚書張說兼中書令。三月庚午,車駕至京師,制所經州、府、縣無出今年地稅,京城見禁囚徒並原免之。



 夏四月丙辰,遷祔中宗神主於太廟。癸亥,張說正除中書令,吏部尚書、中山公王晙為兵部尚書、同中書門下三品。五月己巳,北都置軍器監官員。王晙為朔方節度使,兼知河北郡、隴右、河西兵馬使。六月,王晙赴朔方軍。秋八月戊申,尊八代祖宣皇帝廟號獻祖,光皇帝廟號懿祖,始祔於太廟之九廟。九月己巳,
 頒上撰《廣濟方》於天下,仍令諸州各置醫博士一人。春秋二時釋奠,諸州宜依舊用牲牢,其屬縣用酒酺而已。



 冬十月丁酉,幸新豐之溫泉宮。甲寅,至自溫泉。十一月戊寅,親祀南郊,大赦天下,見禁囚徒死罪至徒流已下免除之。升壇行事及供奉官三品已上賜爵一級,四品轉一階。武德以來實封功臣、知政宰輔淪屈者,所司具以狀聞。賜酺三日,京城五日。是月,自京師至於山東、淮南大雪,平地三尺餘。丁亥,廢軍器監官員,少府監加置
 少監一人以充之。十二月甲午,幸鳳泉湯。戊申,至自鳳泉湯。庚申,王晙授蘄州刺史。



 十二年春正月。



 夏四月,封故澤王上金男義珣為嗣澤王。嗣許王瓘左授鄂州別駕,以弟璆為上金嗣故也。癸卯,嗣江王禕降為信安郡王,嗣蜀王示俞為廣漢郡王,嗣密王徹為濮陽郡王,嗣曹王臻為濟國公,嗣趙王琚為中山郡王,武陽郡王堪為澧國公。禕等並自神龍之後外繼為王,以瓘利澤王之封,盡令歸宗改封焉。秋七月
 壬申,月蝕既。己卯,廢皇后王氏為庶人。後弟太子少保、駙馬都尉守一貶為澤州別駕,至藍田,賜死。戶部尚書、河東伯張嘉貞貶臺州刺史。冬十一月庚申,幸東都,至華陰,上制嶽廟文,勒之於石,立於祠南之道周。戊寅,至自東都。庚辰,司徒、申王捴薨,追謚曰惠莊太子。五溪首領覃行璋反,遣鎮軍大將軍兼內侍楊思勖討平之。閏十二月丙辰朔,日有蝕之。



 十三年春正月乙酉,以幽州都督府為大都督府。戊子,
 降死罪從流,流已下罪悉原之。分遣御史中丞蔣欽緒等往十道疏決囚徒。二月戊午,幸龍門,即日還宮。乙亥,初置彍騎,分隸十二司。丙子,改豳州為邠州,鄚州為莫州,梁州為褒州,沅州為巫州,舞州為鶴州,泉州為福州,以避文相類及聲相近者。三月甲午,皇太子嗣謙改名鴻;郯王嗣直改名潭,徙封慶王;陜王嗣升改名浚,徙封忠王;鄫王嗣真改名洽,徙封棣王;鄂王嗣初改名涓,徙封郎王;嗣玄改名滉,封榮王。又第八子涺封為光王,第
 十二男濰封為儀王,第十三男沄封為潁王,第十六男澤封為永王,第十八男清封為壽王,第二十男洄封為延王,第二十一男沐封為盛王,第二十二男溢封為濟王。丙申,御史大夫程行諶奏:「周朝酷吏來子珣、萬國俊、王弘義、侯思止、郭霸、焦仁亶、張知默、李敬仁、唐奉一、來俊臣、周興、丘神勣、索元禮、曹仁哲、王景昭、裴籍、李秦授、劉光業、王德壽、屈貞筠、鮑思恭、劉景陽、王處貞等二十三人,殘害宗枝,毒陷良善,情狀尤重,子孫不許仕宦。陳
 嘉言、魚承曄、皇甫文備、傅游藝四人,情狀雖輕,子孫不許近任。請依開元二年二月五日敕。」



 夏四月丁巳,改集仙殿為集賢殿,麗正殿書院改集賢殿書院;內五品已上為學士,六品已下為直學士。癸酉,令朝集使各舉所部孝悌文武,集於泰山之下。五月庚寅,妖賊劉定高率其黨夜犯通洛門,盡擒斬之。六月乙亥,廢都西市。



 冬十月癸丑,新造銅儀成,置於景運門內,以示百官。辛酉,東封泰山,發自東都。



 十一月丙戌,至兗州岱宗頓。丁亥,致
 齋於行宮。己丑,日南至,備法駕登山,仗衛羅列嶽下百餘里。詔行從留於谷口,上與宰臣、禮官升山。庚寅,祀昊天上帝於上壇,有司祀五帝百神於下壇。禮畢,藏玉冊於封祀壇之石感,然後燔柴。燎發,群臣稱萬歲,傳呼自山頂至嶽下,震動山谷。上還齋宮,慶雲見,日抱戴。辛卯,祀皇地祇於社首,藏玉冊於石感,如封祀壇之禮。壬辰,御帳殿受朝賀,大赦天下,流人未還者放還。內外官三品已上賜爵一等,四品已下賜一階,登山官封賜一階,
 褒聖侯量才與處分。封泰山神為天齊王,禮秩加三公一等,近山十里,禁其樵採。賜酺七日。侍中源乾曜為尚書左丞相兼侍中,中書令張說為尚書右丞相兼中書令。甲午,發岱嶽。丙申,幸孔子宅,親設奠祭。十二月己巳,至東都。時累歲豐稔,東都米斗十錢,青、齊米斗五錢。是冬,分吏部為十銓,敕禮部尚書蘇頲、刑部尚書韋抗、工部尚書戶從願等分掌選事。



 十四年春正月癸亥,改封契丹松漠郡王李召固為廣
 化王,奚饒樂郡王李魯蘇為奉誠王,封宗室外甥女二人為公主,各以妻之。二月庚戌朔,邕州獠首領梁大海、周光等據賓、橫等州叛,遣驃騎大將軍兼內侍楊思勖討之。三月壬寅,以國甥東華公主降於契丹李召固。



 夏四月癸丑,御史中丞宇文融與御史大夫崔隱甫彈尚書右丞相、兼中書令張說,鞫於尚書省。丁巳,戶部侍郎李元紘同中書門下平章事。庚申,張說停兼中書令。丁卯,太子少師、岐王範薨,冊贈惠文太子。辛丑,於定、恆、莫、易、
 滄等五州置軍以備突厥。五月癸卯,戶部進計帳,今年管戶七百六萬九千五百六十五,管口四千一百四十一萬九千七百一十二。



 六月戊午,大風,拔木發屋,毀端門鴟吻,都城門等及寺觀鴟吻落者殆半。上以旱、暴風雨,命中外群官上封事,指言時政得失,無有所隱。秋七月癸丑夜,瀍水暴漲入漕,漂沒諸州租船數百艘,溺者甚眾。九月己丑,檢校黃門侍郎兼磧西副大都護杜暹同中書門下平章事。是秋,十五州言旱及霜,五十州言
 水,河南、河北尤甚,蘇、同、常、福四州漂壞廬舍,遣御史中丞宇文融檢覆賑給之。



 冬十月,廢麟州。庚申,幸汝州廣成湯。己巳,還東都。十一月甲戌,突厥遣使來朝。辛丑,渤海靺鞨遣其子義信來朝,並獻方物。十二月丁巳,幸壽安之方秀川。己未,日色赤如赭。壬戌,還東都。



 十五年春正月戊寅,制草澤有文武高才,令詣闕自舉。庚子,太史監復為太史局,依舊隸秘書省。辛丑,涼州都督王君掞破吐蕃於青海之西,虜輜車、馬羊而還。二月,
 遣左監門將軍黎敬仁往河北賑給貧乏,時河北牛畜大疫。己巳,尚書右丞相張說、御史大夫崔隱甫、中丞宇文融以朋黨相構,制說致仕,隱甫免官侍母,融左遷魏州刺史。夏五月,晉州大水,漂損居人廬舍。癸酉,以慶王潭為涼州都督兼河西諸軍節度大使,忠王浚為單于大都護、朔方節度大使,棣王洽為太原冀北牧、河北諸軍節度大使,鄂王涓為幽州都督、河北節度大使,榮王滉為京兆牧、隴右節度大使,光王涺為廣州都督、五府
 節度大使,儀王濰為河南牧,潁王潭為安東都護、平盧軍節度大使,永王澤為荊州大都督,壽王清為益州大都督、劍南節度大使,延王洄為安西大都護、磧西節度大使,盛王沐為揚州大都督,並不出閣。秋七月甲戌,雷震興教門樓兩鴟吻,欄檻及柱災。禮部尚書蘇頲卒。庚寅,鄜州洛水泛漲,壞人廬舍。辛卯,又壞同州馮翊縣廨宇,及溺死者甚眾。丙申,改武臨縣為潁陽縣。己亥,赦都城系囚,死罪降從流,徒已下罪悉免之。



 九月丙子,吐蕃
 寇瓜州,執刺史田元獻及王君掞父壽,殺掠人吏,盡取軍資倉糧而去。丙戌,突厥毗伽可汗使其大臣梅錄啜來朝。閏月庚子,突騎施蘇祿、吐蕃贊普圍安西,副大都護趙頤貞擊走之。庚申,車駕發東都,還京師。回紇部落殺王君掞於甘州之鞏筆驛。制檢校兵部尚書蕭嵩兼判涼州事,總兵以御吐蕃。是秋,六十三州水,十七州霜旱;河北饑,轉江淮之南租米百萬石以賑給之。



 冬十月己卯,至自東都。十二月乙亥,幸溫泉宮。丙戌,至自溫泉
 宮。



 十六年春正月庚子,始聽政於興慶宮。春、瀧等州獠首領瀧州刺史陳行範、廣州首領馮仁智、何游魯叛,遣驃騎大將軍楊思勖討之。壬寅,安西副大都護趙頤貞敗吐蕃於曲子城。甲子,黑水靺鞨遣使來朝獻。秋七月,吐蕃寇瓜州,刺史張守珪擊破之。乙巳,檢校兵部尚書蕭嵩、鄯州都督張志亮攻拔吐蕃門城,斬獲數千級,收其資畜而還。丙辰,新羅王金興光遣使貢方物。八月
 己巳,特進張說進《開元大衍歷》,詔命有司頒行之。辛卯,蕭嵩又遣杜賓客擊吐蕃於祁連城,大破之,獲其大將一人,斬首五千級。九月丙午,以久雨,降死罪從流,徒以下原之。



 冬十月己卯,幸溫泉宮。己丑,至自溫泉宮。十一月癸巳朔,檢校兵部尚書、河西節度判涼州事蕭嵩為兵部尚書、同中書門下平章事,餘如故。十二月丁卯,幸溫泉宮。丁丑,至自溫泉宮。



 十七年二月丁卯,巂州都督張審素攻破蠻,拔昆明城
 及鹽城,殺獲萬人。庚子,特進張說復為尚書左丞相,同州刺史陸象先為太子少保。甲寅,禮部尚書、信安王禕帥眾攻拔吐蕃石堡城。夏四月癸亥,令中書門下分就大理、京兆、萬年、長安等獄疏決囚徒。制天下系囚死罪減一等,餘並宥之。丁亥,大風震電,藍田山崩。



 五月癸巳,復置十道按察使。右散騎常侍徐堅卒。六月甲戌,尚書左丞相源乾曜停兼侍中,黃門侍郎杜暹為荊州大都督府長史,中書侍郎李元紘為曹州刺史。兵部尚書蕭
 嵩兼中書令。戶部侍郎兼鴻臚卿宇文融為黃門侍郎,兵部侍郎裴光庭為中書侍郎,並同中書門下平章事。秋七月辛丑,工部尚書張嘉貞卒。八月癸亥,上以降誕日,宴百僚於花萼樓下。百僚表請以每年八月五日為千秋節,王公已下獻鏡及承露囊,天下諸州咸令宴樂,休暇三日,仍編為令,從之。丙寅,越州大水,漂壞廨宇及居人廬舍。己卯,中書侍郎裴光庭兼御史大夫,依舊知政事。乙酉,尚書右丞相、開府儀同三司兼吏部尚書宋
 璟為尚書左丞相,尚書左丞相源乾曜為太子少傅。九月壬子,宇文融左遷汝州刺史,俄又貶昭州平樂尉。壬寅,裴光庭為黃門侍郎,依舊知政事。



 冬十月戊午朔,日有蝕之,不盡如鉤。癸未,睦州獻竹實。庚申,前太子賓客元行沖卒。十一月庚申,親饗九廟。辛卯,發京師。丙申,謁橋陵。上望陵涕泣、左右並哀感。制奉先縣同赤縣,以所管萬三百戶供陵寢,三府兵馬供宿衛,曲赦縣內大闢罪已下。戊戌,謁定陵。己亥,謁獻陵。壬寅,謁昭陵。乙巳,謁
 乾陵。戊申,車駕還宮。大赦天下,流移人並放還,左降官移近處。百姓無出今年地稅之半。每陵取側近六鄉供陵寢。內外官三品巳上加爵一等,四品已下賜一階,五品已上清官父母亡者,依級賜官及邑號。十二月辛酉,幸溫泉宮。乙丑,校獵渭濱。壬申,至自溫泉宮。是冬無雪。



 十八年春正月辛卯,黃門侍郎裴光庭為侍中,依舊兼御史大夫。左丞相張說加開府儀同三司。丙午,幸薛王業宅,即日還宮。二月丙寅,大雨雪,俄而雷震,左飛龍廄
 災。三月辛卯,改定州縣上中下戶口之數,依舊給京官職田。夏四月乙卯,築京城外郭城,凡十月而功畢。壬戌,幸寧親公主第,即日還宮。乙丑,裴光庭兼吏部尚書。是春,命侍臣及百僚每旬暇日尋勝地宴樂,仍賜錢令所司供帳造食。丁卯,侍臣已下宴於春明門外寧王憲之園池,上御花萼樓邀其回騎,便令坐飲,遞起為舞,頒賜有差。五月,契丹衙官可突干殺其主李召固,率部落降於突厥,奚部落亦隨西叛。奚王李魯蘇來奔,召固妻東
 華公主陳氏及魯蘇妻東光公主韋氏並奔投平盧軍。制幽州長史趙含章率兵討之。



 六月庚申,命左右丞相、尚書及中書門下五品已上官,舉才堪邊任及刺史者。甲子,彗星見於五車。癸酉,有星孛于畢、昴。丙子,命單于大都護、忠王浚為河北道行軍元帥,御史大夫李朝隱、京兆尹裴伷先為副,率十八總管以討契丹及奚等。事竟不行。壬午,東都瀍、洛泛漲,壞天津、永濟二橋及提象門外仗舍,損居人廬舍千餘家。閏月甲申,分幽州置薊
 州。已丑,令範安及、韓朝宗就瀍、洛水源疏決,置門以節水勢。辛卯,禮部奏請千秋節休假三日,及村閭社會,並就千秋節先賽白帝,報田祖,然後坐飲散之。



 秋七月庚辰,幸寧王憲第,即日還宮。八月丁亥,上御花萼樓,以千秋節百官獻賀,賜四品已上金鏡、珠囊、縑彩,賜五品已下束帛有差。上賦八韻詩,又制《秋景詩》。辛亥,幸永穆公主宅,即日還宮。九月,先是高戶捉官本錢;乙卯,御史大夫李朝隱奏請薄稅百姓一年租錢充,依舊高戶及典
 正等捉,隨月收利,供官人稅錢。冬十月,吐蕃遣其大臣名悉獵獻方物,請降,許之。庚寅,幸岐州之鳳泉湯。癸卯,至自鳳泉湯。十一月丁卯,幸新豐溫泉宮。十二月戊子,豐州刺史袁振坐妖言下獄死。戊申,尚書左丞相、燕國公張說薨。是歲,百僚及華州父老累表請上尊號內請加「聖文」兩字,並封西嶽,不允。



 十九年春正月壬戌,開府儀同三司、霍國公王毛仲貶為襄州別駕,中路賜死,黨與貶黜者十數人。辛卯,遣鴻
 臚卿崔琳入吐蕃報聘。丙子,親耕於興慶宮龍池。己卯,禁採捕鯉魚。天下州府春秋二時社及釋奠,停牲牢,唯用酒酺,永為常式。



 二月甲午,以崔琳為御史大夫。三月乙酉朔,崔琳使於吐蕃。夏四月壬午,於京城置禮院。丙申,令兩京及天下諸州各置太公尚父廟,以張良配饗,春秋二時仲月上戊日祭之。五月壬戌,五嶽各置老君廟。六月乙酉,大風拔木。秋八月辛巳,降天下死罪從流,徒已下悉原之。九月辛未,吐蕃遣其國相論尚他硉來朝。冬
 十月丙申,幸東都。



 十一月丙辰,至自東都。甲子,太子少傅源乾曜薨。十二月,巂州都督張審素以劫制使監察御史楊汪伏誅。是冬,浚苑內洛水,六十餘日而罷。戊戌,裴光庭上《瑤山往則》、《維城前軌》各一卷,上令賜太子、諸王各一本。



 二十年春正月乙卯,以禮部尚書、信安王禕率兵討契丹。丁巳,幸長芬公主宅;乙丑,幸薛王業宅:並即日還宮。二月己未,敕文武選人,承前例三月三十日為例,然開
 選門,比團甲進官至夏來。自今已後,選門並正月內開,團甲二月內訖。分命宰相錄京城諸獄系囚。三月,信安王禕與幽州長史趙含章大破奚、契丹於幽州之北山。



 夏四月乙亥,宴百僚於上陽東州,醉者賜以床褥,肩輿而歸,相屬於路。癸巳,改造天津橋,毀皇津橋,合為一橋。五月癸卯,寒食上墓,宜編入五禮,永為恆式。辛亥,金仙長公主薨。戊辰,信安王獻奚、契丹之俘,上御應天門受之。



 六月丁丑,單于大都護、河北東道行軍元帥、忠王浚
 加司徒,都護如故;副大使信安王禕加開府儀同三司。庚寅,幽州長史趙含章坐盜用庫物,左監門員外將軍楊元方受含章饋餉,並於朝堂決杖,流瀼州,皆賜死於路。其月,遣範安及於長安廣花萼樓,築夾城至芙蓉園。



 秋七月戊辰,幸寧王憲宅,即日還宮。八月辛未朔,日有蝕之。己卯,戶部尚書王晙卒。九月乙巳,中書令蕭嵩等奏上《開元新禮》一百五十卷,制所司行用之。渤海靺鞨寇登州,殺刺史韋俊,命左領軍將軍蓋福順發兵討之。



 冬
 十月丙戌,命巡幸所至,有賢才未聞達者舉之。仍令中書門下疏決囚徒。辛卯,至潞州之飛龍宮,給復三年,兵募丁防先差未發者,令改出餘州。辛丑,至北都。癸丑,曲赦太原,給復三年。十一月庚午,祀后土於脽上,大赦天下,左降官量移近處。內外文武官加一階,開元勛臣盡假紫及緋。大酺三日。十二月壬申,至京師。



 其年戶部計戶七百八十六萬一千二百三十六,口四千五百四十三萬一千二百六十五。



 二十一年春正月庚子朔,制令士庶家藏《老子》一本,每年貢舉人量減《尚書》、《論語》兩條策,加《老子》策,乙巳,遷祔肅明皇后神主於廟,毀儀坤廟。丁巳,幸溫泉宮。己未,命工部尚書李嵩使於吐蕃。癸亥,至自溫泉宮。三月乙巳,侍中裴光庭薨。甲寅,尚書右丞韓休為黃門侍郎、同中書門下平章事。閏月,幽州道副總管郭英傑等討契丹,為所敗於都山之下,英傑死之。夏四月丁巳,以久旱,命太子少保陸象先、戶部尚書杜暹等七人往諸道宣慰
 賑給,及令黜陟官吏,疏決囚徒。丁酉,寧王憲為太尉,薛王業為司徒,慶王潭為太子太師,忠王浚為開府儀同三司,棣王洽為太子少傅,鄂王涓為太子太保。五月甲申,皇太子納妃薛氏。制天下死罪降從流,流已下釋放。京文武官賜勛一轉。秋七月乙丑朔,日有蝕之。九月壬午,封皇子溢為濟王,沔為信王,泚為義王,漼為陳王,澄為豐王,潓為恆王,漩為涼王,滔為深王。



 冬十月庚戌,幸溫泉宮。十一月戊子,尚書右丞相宋璟以年老請致仕,
 許之。十二月丁未,兵部尚書、徐國公蕭嵩為尚書右丞相,黃門侍郎韓休為兵部尚書,並罷知政事。京兆尹裴耀卿為黃門侍郎,前中書侍郎張九齡起復舊官,並同中書門下平章事。是歲,關中久雨害稼,京師饑,詔出太倉米二百萬石給之。



 二十二年春正月癸亥朔,制古聖帝明皇、岳、瀆、海鎮用牲牢,餘並以酒酺充奠。己巳,幸東都。辛未,太府卿嚴挺之、戶部侍郎裴寬於河南存問賑給。乙酉,懷、衛、邢、相等
 五州乏糧,遣中書舍人裴敦復巡問,量給種子。己丑,至東都。二月壬寅,秦州地震,廨宇及居人廬舍崩壞殆盡,壓死官吏以下四十餘人,殷殷有聲,仍連震不止。命尚書右丞相蕭嵩往祭山川,並遣使存問賑恤之,壓死之家給復一年,一家三人已上死者給復二年。辛亥,初置十道採訪處置使。徵恆州張果先生,授銀青光祿大夫,號曰通玄先生三月,沒京兆商人任令方資財六十餘萬貫。壬午,欲令不禁私鑄錢,遣公卿百僚詳議可否。眾
 以為不可,遂止。四月乙未,伊西、北庭且依舊為節度。廢太廟署,以太常寺奉宗廟。庚子,唐州界準勝州例立表,測候日晷影長短。乙巳,詔京都見禁囚徒,令中書門下及留守檢覆降罪,天下諸州委刺史。丁未,眉州鼎皇山下江水中得寶鼎。甲寅,北庭都護劉渙謀反,伏誅。



 五月戊子,黃門侍郎裴耀卿為侍中,中書侍郎張九齡為中書令,黃門侍郎李林甫為禮部尚書、同中書門下平章事。關中大風拔木,同州尤甚。是夏,上自於苑中種麥,
 率皇太子已下躬自收獲,謂太子等曰:「此將薦宗廟,是以躬親,亦欲令汝等知稼穡之難也。」因分賜侍臣,謂曰:「比歲令人巡檢苗稼,所對多不實,故自種植以觀其成;且《春秋》書麥禾,豈非古人所重也!」六月乙未,遣左金吾將軍李佺於赤嶺與吐蕃分界立碑。



 七月己巳,司徒、薛王業薨,追謚為惠宣太子。甲申,遣中書令張九齡充河南開稻田使。八月,先是駕至東都,遣侍中裴耀卿充江淮、河南轉運使,河口置輸場。壬寅,於輸場東置河陰縣。
 又遣張九齡於許、豫、陳、亳等州置水屯。九月壬申,改饒樂都督府為奉誠都督府。辛巳,移登州平海軍於海口安置。冬十月甲辰,試司農卿陳思問以贓私配流瀼州。十二月戊子朔,日有蝕之。乙巳,幽州長史張守珪發兵討契丹,斬其王屈烈及其大臣可突幹於陣,傳首東都,餘叛奚皆散走山谷。立其酋長李過折為契丹王。是歲,突厥毗伽可汗死。斷京城乞兒。



 二十三年春正月己亥,親耕籍田,上加至九推而止,卿
 已下終其畝。大赦天下。京文武官及朝集採訪使三品已下加一爵,四品已下加一階,外官賜勛一轉。其才有霸王之略、學究天人之際、及堪將帥牧宰者,令五品已上清官及刺史各舉一人。致仕官量與改職,依前致仕。賜酺三日。



 三月丁卯,殿中侍御史楊萬頃為仇人所殺。夏五月戊寅,宗子請率月俸於興慶宮建龍池,上《聖德頌》。秋七月丙子,皇太子鴻改名瑛,慶王直已下十四王並改名。又封皇子玭為義王,珪為陳王,珙為豐王,琪為
 恆王,璿為涼王,敬為汴王。其榮王琬已下並開府置官屬,各食實封二千戶。八月戊子,制鰥寡惸獨免今年地稅之半,江淮已南有遭水處,本道使賑給之。九月戊申,移泗州就臨淮縣置。冬十月辛亥,移隸伊西、北庭都護屬四鎮節度。突騎施寇北庭及安西撥換城。十一月壬申朔,日有蝕之。十二月,新羅遣使朝獻。



 二十四年春正月,吐蕃遣使獻方物。北庭都護蓋嘉運率兵擊突騎施,破之。



 三月乙未,始移考功貢舉遣禮部侍
 郎掌之。夏六月丙午,京兆醴泉妖人劉志誠率眾為亂,將趨京城,咸陽官吏燒便橋以斷其路,俄而散走,京兆府盡擒斬之。是夏大熱,道路有暍死者。秋七月庚子,太子太保陸象先卒。辛丑,李林甫為兵部尚書,依舊知政事。己巳,初置壽星壇,祭老人星及角、亢等七宿。



 八月戊申朔,加親舅小功服,舅母緦麻服,堂舅袒免。己亥,深王滔薨九月壬午,改尚書主爵曰司封。冬十月戊申,車駕發東都,還西京。甲子,至華州,曲赦行在系囚。丁丑,至自
 東都。十一月壬寅,侍中裴耀卿為尚書左丞相,中書令張九齡為尚書右丞相,並罷知政事。兵部尚書李林甫兼中書令,殿中監牛仙客兵部尚書、同中書門下三品。尚書右丞相蕭嵩為太子太師,工部尚書韓休為太子少保。十二月戊申,太子太師、慶王琮為司徒。丙寅,牛仙
 客知門
 下省事。



\end{pinyinscope}