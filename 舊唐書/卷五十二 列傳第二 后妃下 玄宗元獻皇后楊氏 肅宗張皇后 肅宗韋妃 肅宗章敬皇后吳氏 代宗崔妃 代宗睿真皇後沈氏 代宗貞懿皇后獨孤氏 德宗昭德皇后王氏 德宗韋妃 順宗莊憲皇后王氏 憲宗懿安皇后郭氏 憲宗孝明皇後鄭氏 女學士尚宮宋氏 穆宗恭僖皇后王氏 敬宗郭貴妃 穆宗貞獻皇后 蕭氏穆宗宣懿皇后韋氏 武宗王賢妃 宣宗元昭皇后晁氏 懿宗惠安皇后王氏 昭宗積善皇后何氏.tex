\article{卷五十二 列傳第二 后妃下 玄宗元獻皇后楊氏 肅宗張皇后 肅宗韋妃 肅宗章敬皇后吳氏 代宗崔妃 代宗睿真皇後沈氏 代宗貞懿皇后獨孤氏 德宗昭德皇后王氏 德宗韋妃 順宗莊憲皇后王氏 憲宗懿安皇后郭氏 憲宗孝明皇後鄭氏 女學士尚宮宋氏 穆宗恭僖皇后王氏 敬宗郭貴妃 穆宗貞獻皇后 蕭氏穆宗宣懿皇后韋氏 武宗王賢妃 宣宗元昭皇后晁氏 懿宗惠安皇后王氏 昭宗積善皇后何氏}

\begin{pinyinscope}

 ○玄宗元獻皇后楊氏肅宗張皇后肅宗韋妃肅宗章敬皇后吳氏代宗崔妃代宗睿真皇後沈氏代宗貞懿皇后獨孤氏德宗昭德皇后王氏德宗韋妃順宗莊憲皇后王氏憲宗懿安皇后郭氏憲宗孝明皇後鄭氏
 女
 學士尚宮宋氏穆宗恭僖皇后王氏敬宗郭貴妃穆宗貞獻皇后蕭氏穆宗宣懿皇后韋氏武宗王賢妃宣宗元昭皇后晁氏懿宗惠安皇后王氏昭宗積善皇后何氏



 玄宗元獻皇后楊氏,弘農華陰人。曾祖士達,隋納言,天授中,以則天母族,追封士達為鄭王,贈太尉。父知慶,左千牛將軍,贈太尉、鄭國公。後景雲元年八月選入太子宮。時太平公主用事,尤忌東宮。宮中左右持兩端,而潛附太平者,必陰伺察,事雖纖芥,皆聞於上,太子心不自安。後時方娠,太子密謂張說曰:「用事者不欲吾多息胤,恐禍及此婦人,其如之何?」密令說懷去胎藥而入。太子於曲室躬自煮藥,醺然似寐,夢神人覆鼎。既寤如夢,如
 是者三。太子異之,告說。說曰:「天命也,無宜他慮。」既而太平誅,後果生肅宗。太子妃王氏無子,後班在下,後不敢母肅宗。王妃撫鞠,慈甚所生。開元中,肅宗為忠王,後為妃,又生寧親公主。張說以舊恩特承寵異,說亦奇忠王儀表,心知運歷所鐘,故寧親公主降說子垍。



 開元十七年,後薨,葬細柳原,玄宗命說為志文,其銘云:「石獸澀兮綠苔黏,宿草殘兮白露霑。園寢閉兮脂粉膩,不知何年開鏡奩。」二十四年,忠王立為皇太子。至德元年,肅宗即
 位於靈武。二載五月,玄宗在蜀,誥曰:「聖人垂範,是推顧復之恩;王者建極,抑有追尊之禮。蓋母以子貴,德以謚尊。故妃弘農楊氏,特稟坤靈,久厘陰教。往以續塗山之慶,降華渚之祥。誕發異圖,載光帝業。而冊命猶闕,幽靈尚牴。夏王繼統,方軫陽城之恩;漢後褒榮,庶協昭靈之稱。宜於彼追冊為元獻太后。」寶應二年正月,祔葬泰陵。



 肅宗張皇后,本南陽西鄂人,後徙家昭應。祖母竇氏,玄宗母昭成皇太后之妹也。昭成為天后所殺,玄宗幼失
 所恃,為竇姨鞠養。景雲中,封鄧國夫人,恩渥甚隆。其子去惑、去疑、去奢、去逸,皇姨弟也,皆至大官。去盈尚玄宗女常芬公主。去逸生後,天寶中,選入太子宮為良娣。後弟清,又尚大寧郡主。



 後辯惠豐碩,巧中上旨。祿山之亂,玄宗幸蜀,太子與良娣俱從,車駕渡渭,百姓遮道請留太子收復長安。肅宗性仁孝,以上皇播越,不欲違離左右。宦者李靖忠啟太子請留,良娣贊成之,白於玄宗。太子如靈武,時賊已陷京師,從官單寮,道路多虞。每太子
 次舍宿止,良娣必居其前。太子曰:「捍禦非婦人之事,何以居前?」良娣曰:「今大家跋履險難,兵衛非多,恐有倉卒,妾自當之,大家可由後而出,庶幾無患。」及至靈武,產子,三日起,縫戰士衣。太子勞之曰:「產忌作勞,安可容易?」後曰:「此非妾自養之時,須辦大家事。」肅宗即位,冊為淑妃。贈父太僕卿去逸左僕射,母竇氏封義章縣主,姊李曇妻封清河郡夫人,妹師師封郕國夫人。乾元元年四月,冊為皇后。弟駙馬都尉清加特進、太常卿,同正,封範陽
 郡公。皇后寵遇專房,與中官李輔國持權禁中,干預政事,請謁過當,帝頗不悅,無如之何。後於光順門受外命婦朝,親蠶苑中,內外命婦相見,儀注甚盛。先在靈武時,太子弟建寧王倓為後誣譖而死。自是太子憂懼,常恐後之構禍,乃以恭遜取容,後以建寧之隙,常欲危之。張後生二子:興王佋、定王侗。興王早薨,侗又孩幼,故儲位獲安。



 寶應元年四月,肅宗大漸,後與內官硃輝光、馬英俊、啖廷瑤、陳仙甫等謀立越王系,矯詔召太子入侍疾。中
 官程元振、李輔國知其謀,及太子入,二人以難告,請太子在飛龍廄。元振率禁軍收越王,捕硃輝光等。俄而肅宗崩,太子監國,遂移後於別殿,幽崩。誅馬英俊,女道士許靈素配流,山人申大芝賜死,駙馬都尉清貶硤州司馬,弟延和郡主婿鴻臚卿潛貶郴州司馬,舅鴻臚卿竇履信貶道州刺史。



 肅宗韋妃。父元珪,兗州都督。肅宗為忠王時,納為孺人,及升儲位,為太子妃,生兗王僴、絳王佺、永和公主、永穆
 公主。天寶中,宰相李林甫不利於太子,妃兄堅為刑部尚書,林甫羅織,起柳勣之獄,堅連坐得罪,兄弟並賜死。太子懼,上表自理,言與妃情義不睦,請離婚,玄宗慰撫之,聽離。妃遂削發被尼服,居禁中佛舍。西京失守,妃亦陷賊。至德二年,薨於京城。



 肅宗章敬皇后吳氏,坐父事沒入掖庭。開元二十三年,玄宗幸忠王邸,見王服御蕭然,傍無媵侍,命將軍高力士選掖庭宮人以賜之,而吳後在籍中。容止端麗,性多
 謙抑,寵遇益隆。明年,生代宗皇帝。二十八年薨,葬於春明門外。



 代宗即位之年十二月,群臣以肅宗山陵有期,準禮以先太后祔陵廟。宰臣郭子儀等上表曰:



 儷宸極者,允歸於淑德;謚徽號者,必副於鴻名。當履運而承天,則因心而追往,此先王之明訓,聖人之茂典也。伏惟先太后圓精挺質,方祗稟秀。禎符協於四星,典禮敦於萬國,得元和之正氣,韞霄漢之清英。顧史求箴,道先於壺則;捴謙率禮,教備於中闈。太陰無昃朓之徵,丙殿有祝
 延之慶。尊敬師傅,佩服禮經,勤於蘋藻之薦,罔貴珩璜之飾。徽音允穆,嘉慶聿彰,憲度輔佐之勞,緝熙玄默之化,足以光昭宗祀,作配紫微。豈《騶虞》之風,行於江、漢之域;《葛覃》之詠,起自岐陽之下。爰膺歷數,作啟聖明,大拯艱難,永清夷夏。雖復文母成周王之業,慶都誕帝堯之聖,異代同符,彼多慚德。昊蒼不吊,聖善長違。當圓魄之成,玉英早落;有坤儀之美,象服未加。悲懷於先遠之辰,感慟於易名之日。伏以山陵貞兆,良吉有期,虞祔之儀,
 式資配享。率由故實,敬奉嘉名。謹按謚法:「敬慎高明曰章,法度明大曰章,夙興夜寐曰敬,齊莊中正曰敬。」敢遵先典,仰圖懿德,謹上尊謚曰章敬皇后。



 二年三月,祔葬建陵。啟春明門外舊壟,後容狀如生,粉黛如故,而衣皆赭黃色,見者駭異,以為聖子符兆之先。



 後父令珪,寶應初贈太尉;母李氏,贈秦國夫人。叔令瑤,拜太子家令,封馮翊郡公;令瑜,太子右諭德,封濟陰郡公。後兄漵,鴻臚少卿,封鄄城縣公;澄,太子賓客,濮陽縣公;湊,太子詹事,
 臨濮縣公;並加開府儀同三司。漵位終金吾大將軍,湊位終京兆尹,見《外戚傳》。



 代宗睿真皇後沈氏,吳興人,世為冠族。父易直,秘書監。開元末,以良家子選入東宮,賜太子男廣平王。天寶元年,生德宗皇帝。祿山之亂,玄宗幸蜀,諸王、妃、主從幸不及者,多陷於賊,後被拘於東都掖庭。及代宗破賊,收東都,見之,留於宮中,方經略北征,未暇迎歸長安。俄而史思明再陷河洛。及朝義敗,復收東都,失後所在,莫測存
 亡。代宗遣使求訪,十餘年寂無所聞。德宗即位,下詔曰:「王者事父孝,故事天明;事母孝,故事地察。則事天莫先於嚴父,事地莫盛於尊親。朕恭承天命,以主社稷,執珪璧以事上帝,祖宗克配,園寢永終。而內朝虛位,闕問安之禮,銜悲內惻,憂戀終歲。思欲歷舟車之路,以聽求音問,而主茲重器,莫匪深哀。是用仰稽舊儀,敬崇大號,舉茲禮命,式遵前典。宜令公卿大夫稽度前訓,上皇太后尊號。」



 建中元年十一月,遙尊聖母沈氏為皇太后,陳禮
 於含元殿庭,如正至之儀。上袞冕出自東序門,立於東方,朝臣班於位,冊曰:「嗣皇帝臣名言:恩莫重於顧復,禮莫貴於徽號,上以展愛敬之道,下以正《春秋》之義,則祖宗之所稟命,臣子之所盡心,尊尊親親,此焉而在。兩漢而下,帝王嗣位,崇奉尊稱,厥有舊章。永惟丕烈,敢墜前典,臣名謹上尊號曰皇太后。」帝再拜,歔唏不自勝,左右皆泣下。仍以睦王述為奉迎皇太后使,工部尚書喬琳副之,候太后問至,升平公主宜備起居。於是分命使臣,
 周行天下。明年二月,吉問至,群臣稱賀,既而詐妄。自是詐稱太后者數四,皆不之罪,終貞元之世無聞焉。



 德宗敦崇外族,贈太后父易直太師,易直子庫部員外郎介福贈太傅,介福子德州刺史士衡贈太保,易直第二子秘書少監震贈太尉;時沈氏封贈拜爵者百餘人。貞元七年,詔外曾祖隋陜令沈琳贈司徒,追封徐國公,與外祖贈太師易直等立五廟,以琳為始,緣祠廟所須,官給。後無近屬,惟族子房為近,德宗用為金吾將軍,主沈氏
 之祀。



 憲宗即位之年九月,禮儀使奏:「太后沈氏厭代登真,於今二十七載,大行皇帝至孝惟深,哀思罔極。建中之初,已發明詔,舟車所至,靡不周遍,歲月滋深,迎訪理絕。按晉庾蔚之議,尋求三年之後,又俟中壽而服之。今參詳禮例,伏請以大行皇帝啟攢宮日,百官舉哀於肅章門內之正殿,先令有司造禕衣一副,發哀日令內官以禕衣置於幄。自後宮人朝夕上食,先啟告元陵,次告天地宗廟、昭德皇后廟。太皇太后謚冊,造神主,擇日祔
 於代宗廟。其禕衣備法駕奉迎於元陵祠,復置於代宗皇帝袞衣之右。便以發哀日為國忌。」詔如奏。其年十一月,冊謚曰睿真皇后,奉神主祔於代宗之室。



 代宗崔妃,博陵安平人。父峋,秘書少監。母楊氏,韓國夫人。天寶中,楊貴妃寵幸,即妃之姨母也。時韓國、虢國之寵,冠於戚里。時代宗為廣平王,故玄宗選韓國之女,嬪於廣平邸,禮儀甚盛。生召王偲。初,妃挾母氏之勢,性頗妒悍,及西京陷賊,母黨皆誅,妃從王至靈武,恩顧漸薄,
 達京而薨。



 代宗貞懿皇后獨孤氏,父穎,左威衛錄事參軍,以後貴,贈工部尚書。後以美麗入宮,嬖幸專房,故長秋虛位,諸姬罕所進御。後始冊為貴妃,生韓王迥、華陽公主。華陽聰悟過人,能候上顏色,發言必隨喜慍。上之所賞,則因而美之;上之所惡,則曲以全之,由是鐘愛特異。大歷九年,公主薨,上嗟悼過深,數日不視朝。宰臣等因中使吳承倩附奏,言修短常理,以社稷之重,宜節哀視事。初,公
 主疾,上令宗師道教,名曰瓊華真人。及疾亟,上親自臨視,屬纊之際,嚙傷上指,其愛念如此。上既未聽朝,宰臣等諫曰:「公主夙成神悟,仁眷特鐘,嘗禱必親,已承減膳,幽明遽間,倍軫慈衷。臣等微誠,無由感達。伏惟陛下守累聖之公器,御群生之重畜,夷百戰之艱患,撫四海之傷殘。虜候為虞,戎師近警,一言萬務,裁成聖心,得失謬於毫厘,安危存於晷刻。伏慮顧懷猶切,神志未和,眾情以之不寧,臣子以之兢悸。伏願抑周喪之私痛,均品物
 於至公,下慰黔黎,上安宗社。」上始聽朝。



 大歷十年五月,貴妃薨,追謚曰貞懿皇后,殯於內殿,累年不忍出宮。十三年十月方葬,命宰臣常袞為哀冊曰:



 維大歷十年,歲在辛卯,十月辛酉朔。六日丙寅,貴妃獨孤氏薨。粵明日,追謚曰貞懿皇后,殯於內殿之西階。十三年十月癸酉,乃命門下侍郎、同平章事常袞持節冊命。以其月二十五日丁酉,遷座於莊陵,禮也。素紗列位,黼奕周庭,輅升玉綴,軒珠欞。皇帝悼鸞掖以追懷,感麟跡而增慟,備
 百禮以殷遣,命六宮而哀送。宗祝薦告,司儀降收,爰詔侍臣,紀垂鴻休。其辭曰:



 祚祉悠久,寵靈誕受,元魏戚籓,周、隋帝後。五侯迭興,七貴居右,肇啟皇運,光膺文母。纘女是因,以綱大倫,生知陰教,育我蒸人。瑞雲呈彩,瑤星降神,聰明睿智,婉麗貞仁。惟昔天監,搜求才淑,龍德在田,葛覃於穀。周姜胥宇,漢後推轂,王業惟艱,嬪風已穆。繼文傳聖,嗣徽克令,不曜其光,乃終有慶。祗奉園寢,肅恭靈命,越在哀煢,聿追孝敬。文織絲組,硃綠玄黃,上供
 祭服,以祀明堂。法度有節,不待珩璜,篇訓之制,自盈縑緗。敘我邦族,風於天下,始於憂勤,協成王化。慈厚諸女,寵臨下嫁,登進賢才,勞謙日夜。服繒示儉,脫簪申誡,訪問後言,宴游夙退。內加群娣,動有矜誨;外睦諸親,泣辭封拜。闕翟有日,親蠶俟時,忽歸清漢,言復方祗。萬乘悼懷,群臣慕思,玉衣追慶,金鈿同儀。嗚呼哀哉!去昭陽兮窅然,乘雲駕兮何在?人代宛兮如舊,炎涼倏兮已改。翠葆森以成列,素旗儼而相待。言從玉兆之貞,永牴瑤華
 之彩。別長秋之西苑,過望春兮南登,招帝子於北渚,從母後於東陵。下土清兮動金翠,外無像兮中有馮,合簫挽以攢咽,結雲雨之淒凝。吾君感於幽期,俯層亭而望思,慘嬪媛以延踔,極容衛以盡時。搖巾袂兮遠訣,隔軒檻兮群悲,不復見兮回禦輦,傷如何兮軫睿慈。下蘭皋兮背芷陽,旌悠悠兮野蒼蒼,帶白花兮掩淚,衣玄帉兮斷腸。當盛明兮共樂,忽幽處兮獨傷,去故廷兮日遠,即新宮兮夜長。襚無文繡之飾,器無珠貝之藏,蓋自我之
 立制,刑有國之大方。嗚呼哀哉!見送往之空歸,嘆終焉之如此,方士神兮是與非,甘泉畫兮疑復似。遺音在於玉瑱,陳跡留於金所,獻萬壽兮無期,存《二南》之餘美。



 帝追思不已,每事欲極哀情。常袞當代才臣,詔為哀詞,文旨淒悼,覽之者惻然。華陽公主先葬於城東,地卑濕,至是徙葬,祔於莊陵之園,故哀詞云:「招帝子於北渚,從母後於東陵。」乃詔常參官為挽歌,上自選其傷切者,令挽士歌之。大歷初,後寵遇無雙,以恩澤官其宗屬,叔太常
 少卿卓為少府監,後兄良佐太子中允。



 德宗昭德皇后王氏,父遇,官至秘書監。德宗為魯王時,納後為嬪。上元二年,生順宗皇帝,特承寵異。德宗即位,冊為淑妃。貞元二年,妃病。十一月甲午,冊為皇后,是日崩於兩儀殿。臨畢,素服視事。既大殮成服,百僚服三日而釋,用晉文明后崩天下發哀三日止之義,上服凡七日而釋。謚曰昭德。初,令兵部侍郎李紓撰謚冊,文既進,帝以紓文謂皇后曰「大行皇后」非禮,留中不出。詔翰林
 學士吳通玄為之,通玄又云「咨後王氏」,議者亦以為非。知禮者以貞觀中岑文本撰文德皇后謚冊曰「皇后長孫氏」,斯得之矣。五月,葬於靖陵。後母郕國夫人鄭氏請設祭,詔曰:「祭筵不可用假花果,欲祭者從之。」自是宗室諸親,及李晟、渾瑊、神策六軍大將皆設祭。自啟攢後,日數祭,至發引方止。宰臣韓滉為哀冊。又命宰相張延賞、柳渾撰《昭德皇后廟樂章》,既進,上以詞句非工,留中不下,令學士吳通玄別撰進。初,後為淑妃,德宗贈後父遇
 揚州大都督,遇子果眉州司馬,甥侄拜官者二十餘人。永貞元年十一月,徙靖陵,祔葬於崇陵。



 德宗韋賢妃,不知氏族所出,初為良娣,貞元二年,冊為賢妃。性敏惠,言無茍容,動必由禮,德宗深重之,六宮師其德行。及德宗崩,請於崇陵終喪紀,因侍於寢園。元和四年薨。



 順宗莊憲皇后王氏,瑯邪人。曾祖思敬,試太子賓客;祖難得,贈潞州都督,封瑯邪郡公;父顏,金紫光祿大夫、衛
 尉卿。後幼以良家子選入宮為才人,順宗在籓時,代宗以才人賜之,時年十三。大歷十三年,生憲宗皇帝,立為宣王孺人。順宗升儲,冊為良娣。後言容恭謹,宮中稱其德行。順宗即位,疾恙未平,後供侍醫藥,不離左右。屬帝不能言,冊禮將行復止。及永貞內禪,冊為太上皇后。元和元年正月,順宗晏駕,五月,尊太上皇后為皇太后,冊禮畢,憲宗御紫宸殿宣赦。太后居興慶宮。後性仁和恭遜,深抑外戚,無絲毫假貸,訓厲內職,有母儀之風焉。元
 和十一年三月,崩於南內之咸寧殿,謚曰莊憲皇后。



 初,太常少卿韋纁進謚議,公卿署定,欲告天地宗廟。禮院奏議曰:「謹按《曾子問》:『賤不誄貴,幼不誄長,禮也。』古者天子稱天以誄之,皇后之謚,則讀於廟。《江都集禮》引《白虎通》曰:『皇后何所謚之,以為於廟。』又曰:『皇后無外事,無為於郊。』《傳》曰:『故雖天子,必有尊也。』準禮,賤不得誄貴,子不得爵母。所以必謚於廟者,謚宜受成於祖宗;故天子謚成於郊,後妃謚成於廟。今請準禮,集百官連署謚狀訖,
 讀於太廟,然後上謚於兩儀殿。既符故事,允合禮經。」從之。初稱謚並云莊憲皇太后,禮儀使鄭絪奏議:「奏、漢已來,天子之後稱皇后,母稱皇太后,祖母稱太皇太后,崩亦如之。加『太』字者,所以別尊稱也。國朝典禮,皆依舊制。開元六年正月,太常奏昭成皇太后謚號,以牒禮部,禮部非之。太常報曰:『入廟稱後,義系於夫;在朝稱太后,義系於子。』此載於史冊,垂之不刊。今百司移牒及奏狀,參詳典故,恐不合除『太』字;如謚冊入陵,神主入廟,即當去
 之。」其年八月,祔葬於豐陵。後生福王綰,漢陽、雲安、遂安三公主。後之祖、父、母、弟見《外戚傳》。



 憲宗懿安皇后郭氏,尚父子儀之孫,贈左僕射、駙馬都尉曖之女。母代宗長女升平公主。憲宗為廣陵王時,納後為妃。以母貴,父、祖有大勛於王室,順宗深寵異之。貞元十一年,生穆宗皇帝。元和元年八月,冊為貴妃。八年十二月,百僚拜表請立貴妃為皇后,凡三上章。上以歲暮,來年有子午之忌,且止。帝後庭多私愛,以後門族華
 盛,慮正位之後,不容嬖幸,以是冊拜後時。元和十五年正月,穆宗嗣位,閏正月,冊為皇太后,陳儀宣政殿庭,冊曰:



 嗣皇帝臣名再拜言:伏以正坤元,母天下,符至德以升大號,因晉運而飾鴻徽,煥乎前聞,焯彼古訓,以極尊尊親親之義,明因天事地之經,有自來矣。伏惟大行皇帝貴妃,大虹毓慶,霽月披祥,導靈派於昭回,揖殊仁於氣母,範圍百行,表飭六宮,粵在中闈,流宣陰教,輔佐先聖,勤勞庶工。顧以沖眇,遭罹閔兇,荷成命於守器之時,
 奉寶圖於鑄鼎之日,哀纏易月,痛鉅終天。而四海無虞,萬邦有截,仰惟顧復之德,敢揚聖善之風,謹上尊號曰皇太后。



 是日,百僚稱慶,外命婦奉賀光順門。詔皇太后曾祖贈太保,追封岐國公敬之,贈太傅,太后父駙馬都尉曖贈太尉,母虢國大長公主贈齊國大長公主,後兄司農卿釗為刑部尚書、鏦為金吾大將軍。



 太后居興慶宮,帝每月朔望參拜,三朝慶賀,帝自率百官詣門上壽。或遇良辰美景,六宮命婦,戚里親屬,車騎駢噎於南內,
 鑾佩之音,鏘如九奏。穆宗意頗奢縱,朝夕供御,尤為華侈。太后嘗幸驪山,登石甕寺,上命景王率禁軍侍從,帝自於昭應奉迎,游豫行樂,數日方還。敬宗即位,尊為太皇太后



 及寶歷季年,兇徒竊發,昭愍暴殞,內外震駭。宦官迎絳王監國,尋又加害。太皇太后下令曰:「大行皇帝睿哲多能,對越天命,宜荷九廟之重,永享億年之祚。豈謂奸妖竊發,矯專神器,蠱惑中外,扇誘群情,駭動神人,釁深梟鏡。咨爾江王,聰哲精粹,清明在躬,智算機閑,玄
 謀雷發,躬率義勇,大清醜類,允膺當璧之符,爰攄枕戈之憤,既殲巨逆,當享豐福。是命爾陟於元後,宜令司空、平章事、晉國公度奉冊即皇帝位。」文宗孝而謙謹,奉祖母有禮。膳羞珍果,蠻夷奇貢,獻郊廟之後,及三宮而後進御。武宗即位,以後祖母之尊,門地素貴,奉之益隆。既而宣宗繼統,即後之諸子也,恩禮愈異於前朝。大中年崩於興慶宮,謚曰懿安皇太后,祔葬於景陵。後歷位七朝,五居太母之尊,人君行子孫之禮,福壽隆貴,四十餘
 年,雖漢之馬、鄧,無以加焉。識者以為汾陽社稷之功未泯,復鐘慶於懿安焉。



 憲宗孝明皇後鄭氏,宣宗之母也。蓋內職御女之列,舊史殘缺,未見族姓所出、入宮之由。宣宗為光王時,後為王太妃。既即位,尊為皇太后。會昌六年,後弟光夢車中載日月,光芒燭六合,占者曰:「必暴貴。」月餘,武宗崩,宣宗即位,光以元舅之尊,檢校戶部尚書、諸衛將軍,出為平盧節度使。後大中末崩,謚曰孝明。



 女學士、尚宮宋氏者,名若昭,貝州清陽人。父庭芬,世為儒學,至庭芬有詞藻。生五女,皆聰惠,庭芬始教以經藝,既而課為詩賦,年未及笄,皆能屬文。長曰若莘,次曰若昭、若倫、若憲、若荀。若莘、若昭文尤淡麗,性復貞素閑雅,不尚紛華之飾。嘗白父母,誓不從人,願以藝學揚名顯親。若莘教誨四妹,有如嚴師。著《女論語》十篇,其言模仿《論語》,以韋逞母宣文君宋氏代仲尼,以曹大家等代顏、閔,其間問答,悉以婦道所尚。若昭注解,皆有理致。貞元
 四年,昭義節度使李抱真表薦以聞。德宗俱召入宮,試以詩賦,兼問經史中大義,深加賞嘆。德宗能詩,與侍臣唱和相屬,亦令若莘姊妹應制。每進御,無不稱善。嘉其節概不群,不以宮妾遇之,呼為學士先生。庭芬起家受饒州司馬,習藝館內,敕賜第一區,給俸料。



 元和末,若莘卒,贈河內郡君。自貞元七年已後,宮中記注簿籍,若莘掌其事。穆宗復令若昭代司其職,拜尚宮。姊妹中,若昭尤通曉人事,自憲、穆、敬三帝,皆呼為先生,六宮嬪媛、諸
 王、公主、駙馬皆師之,為之致敬。進封梁國夫人。寶歷初卒,將葬,詔所司供鹵簿。敬宗復令若憲代司宮籍。文宗好文,以若憲善屬文,能論議奏對,尤重之。



 大和中,神策中尉王守澄用事,委信翼城醫人鄭注、賊臣李訓,干竊時權。訓、注惡宰相李宗閔、李德裕,構宗閔憸邪,為吏部侍郎時,令駙馬都尉沈通賂於若憲,求為宰相。文宗怒,貶宗閔為潮州司戶,柳州司馬,幽若憲於外第,賜死。若憲弟侄女婿等連坐者十三人,皆流嶺表。李訓敗,
 文宗悟其誣構,深惜其才。若倫、若荀早卒。



 穆宗恭僖皇后王氏,越人。父紹卿,婺州金華令。後少入太子宮,元和四年生敬宗。穆宗皇帝立為妃。長慶四年二月,尊為皇太后。昭愍崇重母族,贈紹卿司空,後母張氏贈趙國夫人。文宗即位之初,號寶歷太后。大和八年詔:「伏以皇太后與寶歷太后,每有司行遣,稱號未分,禮式非便,稽諸前代,詔令所施,不斥言太后,以宮名為稱。今寶歷太后居義安殿,宜準故事稱義安太后。」



 敬宗郭貴妃,父義,右威衛將軍。長慶末,以姿貌選入太子宮。敬宗即位,為才人,生晉王普。帝以少年有子,復以才人容德冠絕,特寵異之。贈其父禮部尚書,又以兄環為少府少監,賜第一區。俄冊為貴妃。及昭愍遇盜,宮闈變起,文宗即位,尤憐晉王,有若己子,故貴妃禮遇不衰。大和二年晉王薨,帝深嗟惜,贈曰悼懷太子。



 穆宗貞獻皇后蕭氏,福建人。初,入十六宅為建安王侍者,元和四年十月,生文宗皇帝。寶歷三年正月,敬宗遇
 弒,中尉王守澄率兵討賊,迎江王即位。文宗踐祚之日,奉冊曰:「嗣皇帝臣名言:古先哲王之有天下也,必以孝敬奉於上,慈惠浹於下,極誠意以厚人倫,思由近以及遠,故自家而刑國。以臣奉嚴慈之訓,承教撫之仁,而長樂尚鬱其鴻名,內朝未崇於正位,則率土臣子,勤勤懇懇,延頸企踵,曷以塞其心乎!是用特舉彞章,式遵舊典,稽首再拜,謹上穆宗睿文惠孝皇帝妃尊號曰皇太后。伏惟與天合德,義申錫慶,允厘陰教,祗修內則。廣六宮
 之教,參十亂之功,頤神保和,弘覆萬有。」



 後因亂去鄉里,自入王邸,不通家問,別時父母已喪,有母弟一人。文宗以母族鮮親,惟舅獨存,詔閩、越連率於故里求訪。有戶部茶綱役人蕭洪,自言有姊流落。估人趙縝引洪見后姊徐國夫人女婿呂璋,夫人亦不能省認,俱見太后,嗚咽不自勝。上以為復得元舅,遂拜金吾將軍、檢校戶部尚書、河陽懷節度使,遷檢校左僕射、鄜坊節度使。先是,有自神策兩軍出為方鎮者,軍中多資其行裝,至鎮三
 倍償之。時有自左軍出為鄜坊者,資錢未償而卒於鎮,乃徵錢於洪。宰相李訓雅知洪詐稱國舅,洪懼,請訓兄仲京為鄜坊從事以彌縫之。洪恃與訓交,不與所償;又徵於卒者之子,洪俾其子接訴於宰相,李訓判絕之。左軍中尉仇士良深銜之。時有閩人蕭本者,復稱太后弟,士良以本上聞,發洪詐假,自鄜坊追洪下獄,御史臺按鞠,具服其偽,詔長流驩州,賜死於路,趙縝、呂璋亦從坐。洪以偽敗,謂本為真,乃拜贊善大夫,賜緋龜,仍追封其
 曾祖倰為太保,祖聰為太傅,父俊為太師,賜與鉅萬計。本,福建人,太后有真母弟,孱弱不能自達,本就之,得其家代及內外族屬名諱,復士良保任之,上亦不疑詐妄。本歷衛尉少卿、左金吾將軍。開成二年,福建觀察使唐扶奏,得泉州晉江縣令蕭弘狀,自稱是皇太后親弟,送赴闕庭,詔送御史臺按問,事皆偽妄,詔逐還本貫。開成四年,昭義節度使劉從諫上章,論蕭本偽稱太后弟,云:「今自上及下,異口同音,皆言蕭弘是真,蕭本是偽。請追
 蕭弘赴闕,與本證明。若含垢於一時,終取笑於千古。」遂詔御史中丞高元裕、刑部侍郎孫簡、大理卿崔郇三司按弘、本之獄,具,並偽。詔曰:



 恭以皇太后族望,承齊、梁之後,僑寓流滯,久在閩中。慶靈鐘集,早歸椒掖,終鮮兄弟,常所咨嗟。朕自臨御已來,便遣尋訪,冀得諸舅,以慰慈顏。而奸濫之徒,探我情抱,因緣州里之近,附會祖先之名,覬幸我國恩,假托我外族。蕭洪之惡跡未遠,蕭本之覆轍相尋,弘之本末,尤更乖戾。三司推鞫,曾無似是之
 蹤;宰臣參驗,見其難容之狀。文款繼入,留中久之。朕於視膳之時,頻有咨稟,恭聞處分,惟在真實。丐沐墮桑,既無可驗;鑿空作偽,豈得更容?據其罪狀,合當極法,尚為含忍,投之荒裔。蕭本除名,長流愛州;蕭弘配流儋州。



 初,蕭洪詐稱國舅十數年,兩授旄鉞,寵貴崇於天下。蕭本因士良鄉導,發洪之詐,聯歷顯榮。及從諫奏論,偽跡難掩,而太后終不獲真弟。



 文宗孝義天然,大和中,太皇太后居興慶宮,寶歷太后居義安殿,皇太后居大內,時號「
 三宮太后」。上五日參拜,四節獻賀,皆由復道幸南內,朝臣命婦詣宮門起居,上尤執禮,造次不失。有司嘗獻新瓜、櫻桃,命獻陵寢宗廟之後,中使分送三宮、十宅。初,有司送三宮物,一例稱賜。帝曰:「物上三宮,安得名賜?」遽取筆塗籍,改「賜」為「奉」。開成中正月望夜,帝於咸泰殿陳燈燭,奏《仙韶樂》,三宮太后俱集,奉觴獻壽,如家人禮,諸親王、公主、駙馬、戚屬皆侍宴。上性恭儉,延安公主衣裾寬大,即時遣還,罰駙馬竇浣兩月賜錢。武宗即位,供養彌
 謹。蕭太后徙居積慶殿,號積慶太后。會昌中崩,謚曰貞獻。



 穆宗宣懿皇后韋氏,武宗昭肅皇帝之母也。事闕



 武宗王賢妃。事闕



 宣宗元昭皇后晁氏,懿宗皇帝之母也。事闕



 懿宗惠安皇后王氏,僖宗皇帝之母也。事闕



 昭宗積善皇后何氏,東蜀人。入侍壽王邸,婉麗多智,特承恩顧,生德王、輝王。昭宗即位,立為淑妃。乾寧中,車駕
 在華州,冊為皇后。國家自乾符已後,盜滿天下,妖生九重,宮廟榛蕪,奔播不暇。景福之際,奸臣內侮,後於蒙塵薄狩之中,嘗膳禦侮,不離左右。左關、右輔之幸,時事危迫,後消息撫御,終獲保全。自岐下還京,崔胤盡誅黃門宦官,每宣諭宰臣,但令宮嬪來往。是時國命奪於硃氏,左右前後,皆是汴人,宮中動息,雖纖芥必聞於硃全忠。宮人常懷惴怵,帝後垂泣相視。天祐初,全忠逼遷輿駕,東幸洛陽。其年八月,昭宗遇弒。翌日,宰相柳璨、獨孤損
 等詐宣皇后令云:「帝為宮人害,輝王祚宜升帝位。」仍尊後為皇太后。遭罹變故,迫以兇威,宮中哭泣,不敢聲聞於外。明年十二月,全忠將僭位,先行九錫,然後受禪。全忠牙將蔣玄暉在洛陽宮知樞密,與太常卿張廷範私議云:「山西、河北未平,禪代無利,請俟蕩定。」欲有咨諫。宣徽副使趙殷衡素與張、蔣不協,且欲代知樞密事,因使於梁,誣告云:「玄暉私於何太后,相與盟詛,誓復唐室,不欲王受九錫。」全忠大怒,即日遣使至洛陽,誅玄暉、廷範、
 柳璨等,太后亦被害於積善宮,又殺宮人阿秋、阿虔,仍廢太后為庶人。



 贊曰:坤德既軌,彤管有煒。韋、武喪邦,毒侔蛇虺。陰教斯僻,嬪風浸毀。賢哉長孫,母儀何偉。



\end{pinyinscope}