\article{卷一百九十七}

\begin{pinyinscope}

 ○邢文偉高子貢郎餘令路敬淳王元感王紹宗韋叔夏祝欽明郭山惲
 柳沖盧粲尹知章孫季良附徐岱蘇弁兄袞冕陸質馮伉韋表微許康佐



 邢文偉,滁州全椒人也。少與和州高子貢、壽州裴懷貴俱以博學知名於江、淮間。咸亨中,累遷太子典膳丞。時孝敬在東宮,罕與宮臣接見,文偉輒減膳,上書曰:



 臣竊見《禮·戴記》曰:「太子既冠成人,免於保傅之嚴,則有司過之史,徹膳之宰。史之義,不得不司過;宰之義,不得不徹
 膳,不徹膳則死。」今皇帝式稽前典,妙簡英俊,自庶子已下,至諮議、舍人及學士、侍讀等,使翼佐殿下,以成聖德。近日已來,未甚延納,談議不狎,謁見尚稀,三朝之後,但與內人獨居,何由發揮聖智,使睿哲文明者乎?今史雖闕官,宰當奉職,忝備所司,未敢逃死,謹守禮經,輒申減膳。



 太子答書曰:



 顧以庸虛,早尚墳典,每欲研精政術,極意書林。但往在幼年,未閑將衛,竭誠耽誦,因即損心。比日以來,風虛更積,中奉恩旨,不許重勞。加以趨侍含元,溫
 清朝夕,承親以無專之道,遵禮以色養為先。所以屢闕坐朝,時乖學緒。公潛申勖戒,聿薦忠規,敬尋來請,良符宿志。自非情思審諭,義均弼諧,豈能進此藥言,形於簡墨!撫躬三省,感愧兼深!



 文偉自是益知名。



 其後右史缺官,高宗謂侍臣曰:「邢文偉事我兒,能減膳切諫,此正直人也。」遂擢拜右史。則天臨朝,累遷鳳閣侍郎,兼弘文館學士。載初元年,遷內史。



 天授初,內史宗秦客以奸贓獲罪,文偉坐附會秦客,貶授珍州刺史。後有制使至其州
 境,文偉以為殺己,遽自縊而死。



 高子貢者,和州歷陽人也。弱冠游太學,遍涉《六經》,尤精《史記》。與文偉及亳州硃敬則為莫逆之交。明經舉,歷秘書正字、弘文館直學士。鬱鬱不得志,棄官而歸。



 屬徐敬業作亂於揚州,遣弟敬猷統兵五千人,緣江西上,將逼和州。子貢率鄉曲數百人拒之,自是賊不敢犯。以功擢授朝散大夫,拜成均助教。



 虢王鳳之子東莞公融,曾為和州刺史,從子貢受業,情義特深。及融為申州,陰懷異
 志。令黃公撰結交於子貢,推為謀主。潛謀密議,書信往復,諸王內外相應,皆出自其策。尋而事發,被誅。



 郎餘令,定州新樂人也。祖楚之,少與兄蔚之,俱有重名。隋大業中,蔚之為左丞,楚之為尚書民曹郎。煬帝重其兄弟,稱為二郎。楚之,武德初為大理卿,與太子少保李綱、侍中陳叔達撰定律令。後受詔招諭山東,為竇建德所獲,脅以兵刃,又誘以厚利,楚之竟不為屈。及還,以年老致仕。貞觀初,卒,時年八十。



 餘令父知運,貝州刺史;兄
 餘慶,高宗時萬年令,理有威名,京城路不拾遺,後卒於交州都督。



 餘令少以博學知名,舉進士。初授霍王元軌府參軍,數上詞賦,元軌深禮之。先是,餘令從父知年為霍王友,亦見推仰。元軌謂人曰:「郎氏兩賢,人之望也。相次入府,不意培塿而松柏成林。」轉幽州錄事參軍。時有客僧聚眾欲自焚,長史裴照率官屬欲往觀之。餘令曰:「好生惡死,人之性也。違越教義,不近人情。明公佐守重籓,須察其奸詐,豈得輕舉,觀此妖妄!」照從其言,因收僧
 按問,果得詐狀。



 孝敬在東宮,餘令續梁元帝《孝德傳》,撰《孝子後傳》三十卷,以獻,甚見嗟重。累轉著作佐郎。撰《隋書》未成,會病卒,時人甚痛惜之。



 路敬淳,貝州臨清人也。父文逸。隋大業末,闔門遇盜,文逸潛匿草澤,晝伏於死人中,夜行避難。自傷窮梗,閉口不食。同侶閔其謹願,勸以不當滅性,捃拾以食之,遞負之而行,遂免於難。貞觀末,官至申州司馬。



 敬淳與季弟敬潛俱早知名。敬淳尤勤學,不窺門庭,遍覽墳籍,而孝
 友篤敬。遭喪,三年不出廬寢。服免,方號慟入見其妻,形容羸毀,妻不之識也。



 後舉進士。天授中,歷司禮博士、太子司議郎,兼修國史,仍授崇賢館學士。數受詔修緝吉兇雜儀,則天深重之。萬歲通天二年,坐與綦連耀結交,下獄死。



 敬淳尤明譜學,盡能究其根源枝派,近代已來,無及之者。撰《著姓略記》十卷,行於時。又撰《衣冠本系》,未成而死。神龍初,追贈秘書少監。



 敬潛仕至中書舍人。



 王元感,濮州鄄城人也。少舉明經,累補博城縣丞。兗州
 都督、紀王慎深禮之,命其子東平王續從元感受學。天授中,稍遷左衛率府錄事,兼直弘文館。是後則天親祠南郊及享明堂,封嵩岳,元感皆受詔共諸儒撰定儀注,凡所立議,眾咸推服之。轉四門博士,仍直弘文館。元感時雖年老,猶能燭下看書,通宵不寐。



 長安三年,表上其所撰《尚書糾謬》十卷、《春秋振滯》二十卷、《禮記繩愆》三十卷,並所注《孝經》、《史記》稿草,請官給紙筆,寫上秘書閣。詔令弘文、崇賢兩館學士及成均博士詳其可否。



 學士祝
 欽明、郭山惲、李憲等皆專守先儒章句,深譏元感掎摭舊義,元感隨方應答,竟不之屈。鳳閣舍人魏知古、司封郎中徐堅、左史劉知幾、右史張思敬,雅好異聞,每為元感申理其義,連表薦之。尋下詔曰:「王元感質性溫敏,博聞強記,手不釋卷,老而彌篤。掎前達之失,究先聖之旨,是謂儒宗,不可多得。可太子司議郎,兼崇賢館學士。」魏知古嘗稱其所撰書曰:「信可謂《五經》之指南也。」中宗即位,以春宮舊僚,進加朝散大夫,拜崇賢館學士。尋卒。



 王紹宗,揚州江都人也,梁左民尚書銓曾孫也,其先自瑯邪徙焉。紹宗少勤學,遍覽經史,尤工草隸。家貧,常傭力寫佛經以自給,每月自支錢足即止,雖高價盈倍,亦即拒之。寓居寺中,以清凈自守,垂三十年。文明中,徐敬業於揚州作亂,聞其高行,遣使徵之,紹宗稱疾固辭。又令唐之奇親詣所居逼之,竟不起。敬業大怒,將殺之。之奇曰:「紹宗人望,殺之恐傷士眾之心。」由是獲免。及賊平,行軍大總管李孝逸以其狀聞,則天驛召赴東都,引入
 禁中,親加慰撫,擢拜太子文學,累轉秘書少監,仍侍皇太子讀書。



 紹宗性淡雅,以儒素見稱,當時朝廷之士,咸敬慕之。張易之兄弟,亦加厚禮。易之伏誅,紹宗坐以交往見廢,卒於鄉里。



 韋叔夏,尚書左僕射安石兄也。少而精通《三禮》。其叔父太子詹事琨嘗謂曰:「汝能如是,可以繼丞相業矣!」舉明經。調露年,累除太常博士。後屬高宗崩,山陵舊儀多廢缺,叔夏與中書舍人賈太隱、太常博士裴守貞等,草創
 撰定,由是授春官員外郎。則天將拜洛及享明堂,皆別受制,共當時大儒祝欽明、郭山惲撰定儀注。凡所立議,眾咸推服之。累遷成均司業。久視元年,特下制曰:「吉兇禮儀,國家所重,司禮博士,未甚詳明。成均司業韋叔夏、太子率更令祝欽明等,博涉禮經,多所該練,委以參掌,冀弘典式。自今司禮所修儀注,並委叔夏等刊定訖,然後進奏。」



 長安四年,擢春官侍郎。神龍初,轉太常少卿,充建立廟社使。以功進銀青光祿大夫。三年,拜國子祭酒。
 累封沛國郡公。卒時年七十餘。撰《五禮要記》三十卷,行於代。贈兗州都督、修文館學士,謚曰文。



 子縚,太常卿。



 祝欽明,雍州始平人也。少通《五經》,兼涉眾史百家之說。舉明經。長安元年,累遷太子率更令,兼崇文館學士。中宗在春宮,欽明兼充侍讀。



 二年,遷太子少保。中宗即位,以侍讀之故,擢拜國子祭酒、同中書門下三品,加位銀青光祿大夫,歷刑部、禮部二尚書,兼修國史,仍舊知政事,累封魯國公,食實封三百戶。尋以匿忌日,為御史中
 丞蕭至忠所劾,貶授申州刺史。久之,入為國子祭酒。



 景龍三年,中宗將親祀南郊,欽明與國子司業郭山惲二人奏言皇后亦合助祭,遂建議曰:



 謹按《周禮》,天神曰祀,地祇曰祭,宗廟曰享。大宗伯職曰:「祀大神,祭大祇,享大鬼,理其大禮。若王有故不預,則攝位。凡大祭祀,王後不預,則攝而薦豆籩,徹。」又追師職:「掌王後之首服,以待祭祀。」又內司服職:「掌王後之六服。凡祭祀,供後之衣服。」又九嬪職:「大祭祀,後稞獻則贊,瑤爵亦如之。」據此諸文,即
 皇後合助皇帝祀天神、祭地祇,明矣。故鄭玄注《內司服》云:「闕狄,皇后助王祭群小祀之服。」然則小祀尚助王祭,中、大推理可知。闕狄之上,猶有兩服:第一禕衣,第二搖狄,第三闕狄。此三狄,皆助祭之服。闕狄即助祭小祀,即知搖狄助祭中祀,禕衣助祭大祀。鄭舉一隅,故不委說。唯祭宗廟,《周禮》王有兩服,先王袞冕,先公柷冕。鄭玄因此以後助祭宗廟,亦分兩服,云:「禕衣助祭先王,搖狄助祭先公。」不言助祭天地社稷,自宜三隅而反。



 且《周禮》正
 文:「凡祭,王後不預。」既不專言宗廟,即知兼祀天地,故云「凡」也。又《春秋外傳》云:「禘郊之事,天子親射其牲,王後親舂其粢。」故代婦職但云:「詔王後之禮事」,不主言宗廟也。若專主宗廟者,則內宗、外宗職皆言「掌宗廟之祭祀」。此皆禮文分明,不合疑惑。



 舊說以天子父天、母地、兄日、姊月,所以祀天於南郊,祭地於北郊,朝日於東門之外,以昭事神,訓人事,君必躬親以禮之,有故然後使攝,此其義也。《禮記·祭統》曰:「夫祭也者,必夫婦親之,所以備內外
 之官也。官備則具備。」又,「哀公問於孔子曰:『冕而親迎,不已重乎?』孔子愀然作色而對曰:『合二姓之好,以繼先聖之後,以為天地宗廟社稷之主,君何謂已重焉!』」又《漢書·郊祀志》云:「天地合祭,先祖配天,先妣配地。天地合精,夫婦判合。祭天南郊,則以地配,一體之義也。」據此諸文,即知皇後合助祭,望請別修助祭儀注同進。



 帝頗以為疑,召禮官親問之。太常博士唐紹、蔣欽緒對曰:「皇后南郊助祭,於禮不合。但欽明所執,是祭宗廟禮,非祭天地禮。
 謹按魏、晉、宋及齊、梁、周、隋等歷代史籍,至於郊天祀地,並無皇后助祭之事。」帝令宰相取兩家狀對定。欽緒與唐紹及太常博士彭景直又奏議曰:



 《周禮》凡言祭、祀、享三者,皆祭之互名,本無定義。何以明之?按《周禮》典瑞職云:「兩珪有邸,以祀地。」則祭地亦稱祀也。又司筵云:「設祀先王之胙席。」則祭宗廟亦稱祀也。又內宗職云:「掌宗廟之祭祀。」此又非獨天稱祀,地稱祭也。又按《禮記》云:「惟聖為能享帝。」此即祀天帝亦言享也。又按《孝經》云:「春秋祭
 祀,以時思之。」此即宗廟亦言祭祀也。經典此文,不可備數。據此則欽明所執天曰祀,地曰祭,廟曰享,未得為定,明矣!又《周禮》凡言大祭祀者,祭天地宗廟之總名,不獨天地為大祭也。何以明之?按《爵人職》云:「大祭祀,與量人授舉斝之卒爵。」尸與斝,皆宗廟之事,則宗廟亦稱大祭祀。又欽明狀引九嬪職:「大祭祀,後稞獻則贊瑤爵。」據祭天無稞,亦無瑤爵,此乃宗廟稱大祭祀之明文。欽明所執大祭祀即為祭天地,未得為定,明矣!



 又《周禮》大宗伯
 職云:「凡大祭祀,王後有故不預,則攝而薦豆籩,徹。」欽明唯執此文,以為王後有祭天地之禮。欽緒等據此,乃是王後薦宗廟之禮,非祭天地之事。何以明之?按此文:「凡祀大神,祭大祇,享大鬼,帥執事而卜日宿,視滌濯,蒞玉鬯,省牲鑊,奉玉齏,詔大號,理其大禮,詔相王之大禮。若王不與祭祀,則攝位。」此已上一「凡」,直是王兼祭天地宗廟之事,故通言大神、大祇、大鬼之祭也。已下文云:「凡大祭祀,王後不與,則攝而薦豆籩,徹。」此一「凡」,直是王後祭
 廟之事,故唯言大祭祀也。若云王後助祭天地,不應重起「凡大祭祀」之文也。為嫌王後有祭天地之疑,故重起後「凡」以別之耳。王後祭廟,自是大祭祀,何故取上「凡」相王之禮,以混下「凡」王後祭宗廟之文?此是本經科段明白。



 又按《周禮》:「外宗掌宗廟之祭祀,佐王後薦玉豆。凡後之獻,亦如之。王後有故不預,則宗伯攝而薦豆籩。」外宗無佐祭天地之禮。但天地尚質,宗廟尚文。玉豆,宗廟之器,初非祭天所設。請問欽明,若王後助祭天地,在《周禮》
 使何人贊佐?若宗伯攝後薦豆祭天,又合何人贊佐?並請明徵禮文,即知攝薦是宗廟之禮明矣。



 按《周禮·司服》云:「王祀昊天上帝,則服大裘而冕。享先王,則袞冕。」內司服,「掌王後祭服」,無王後祭天之服。按《三禮義宗》明王後六服,謂禕衣、搖翟、闕翟、鞠衣、展衣、褖衣。「禕衣從王祭先王則服之,搖翟祭先公及饗諸侯則服之,鞠衣以採桑則服之,展衣以禮見王及見賓客則服之,褖衣燕居服之。」王後無助祭於天地之服,但自先王已下。又《三禮義
 宗》明後夫人之服云:「後不助祭天地五岳,故無助天地四望之服。」按此,則王後無祭天之服,明矣。《三禮義宗》明王後五輅,謂重翟、厭翟、安車、翟車、輦車也。「重翟者,後從王祭先王、先公所乘也;厭翟者,後從王饗諸侯所乘也;安車者,後宮中朝夕見於王所乘也;翟車者,後求桑所乘也;輦車者,後游宴所乘也。」按此,則王後無祭天之車明矣。



 又《禮記·郊特牲·義贊》云:「祭天無稞。鄭玄注云:『唯人道宗廟有稞。天地大神,至尊不稞。』圓丘之祭,與宗廟不
 同。朝踐,王酌泛齊以獻,是一獻。後無祭天之事,大宗伯次酌醴齊以獻,是為二獻。」按此,則祭圓丘,大宗伯次王為獻,非攝王後之事。欽明等所執王後有故不預,則宗伯攝薦豆籩,更明攝王後宗廟之薦,非攝天地之祀明矣。



 欽明建議引《禮記·祭統》曰:「夫祭也者,必夫婦親之」。按此,是王與後祭宗廟之禮,非關祀天地之義。按漢、魏、晉、宋、後魏、齊、梁、周、陳、隋等歷代史籍,興王令主,郊天祀地,代有其禮,史不闕書,並不見往代皇后助祭之事。又高
 祖神堯皇帝、太宗文武聖皇帝南郊祀天,無皇后助祭處。高宗天皇大帝永徽二年十一月辛西親有事於南郊,又總章元年十二月丁卯親拜南郊,亦並無皇后助祭處。又按《大唐禮》,亦無皇后南郊助祭之禮。



 欽緒等幸忝禮官,親承聖問,竭盡聞見,不敢依隨。伏以主上稽古,志遵舊典,所議助祭,實無明文。



 時尚書左僕射韋巨源又希旨,協同欽明之議。上納其言,竟以後為亞獻,仍補大臣李嶠等女為齊娘,以執籩豆。及禮畢,特詔齊娘有
 夫婿者,咸為改官。



 景雲初,侍御史倪若水劾奏欽明及郭山惲曰:「欽明等本自腐儒,素無操行,崇班列爵,實為叨忝。而涓塵莫效,諂佞為能。遂使曲臺之禮,圜丘之制,百王故事,一朝墜失。所謂亂常改作,希旨病君,人之不才,遂至於此。今聖明馭歷,賢良入用,惟茲小人,猶在朝列。臣請並從黜放,以肅周行。」於是左授欽明饒州刺史。後入為崇文館學士。尋卒。



 郭山惲,蒲州河東人。少通《三禮》。景龍中,累遷國子司業。
 時中宗數引近臣及修文學士,與之宴集,嘗令各效伎藝,以為笑樂。工部尚書張錫為《談容娘舞》,將作大匠宗晉卿舞《渾脫》,左衛將軍張洽舞《黃麞》,左金吾衛將軍杜元琰誦《婆羅門咒〗》,給事中李行言唱《駕車西河》,中書舍人盧藏用效道士上章。山惲獨奏曰:「臣無所解,請誦古詩兩篇。」帝從之,於是誦《鹿鳴》、《蟋蟀》之詩。奏未畢,中書令李嶠以其詞有「好樂無荒」之語,頗涉規諷,怒為忤旨,遽止之。



 翌日,帝嘉山惲之意,詔曰:「郭山惲業優經史,識貯
 古今,《八索》、《九丘》,由來遍覽;前言往行,實所該詳。昨者因其豫游,式宴朝彥,既乘歡洽,咸使詠歌。遂能志在匡時,潛申規諷,謇謇之誠彌切,諤諤之操逾明。宜示褒揚,美茲鯁直。」賜時服一幅。尋與祝欽明同獻皇后助祭郊祀之議。景雲中,左授括州長史。開元初,復入為國子司業。卒於官。



 柳沖,蒲州虞鄉人也,隋饒州刺史莊曾孫也。其先仕江左,世居襄陽。陳亡,還鄉里。父楚賢,大業末,為河北縣長。
 時堯君素固守郡城,以拒義師。楚賢進說曰:「隋之將亡,天下皆知。唐公名應圖籙,動以信義,豪傑響應,天所贊也!君子見機而作,不俟終日,轉禍為福,今其時也!」君素不從,楚賢潛行歸國。高祖甚悅,拜侍御史。貞觀中,累轉光祿少卿,使突厥存撫李思摩,突厥贈馬百匹及方物,悉拒而不受。累轉交、桂二州都督,皆有能名。卒於杭州刺史。



 沖博學,尤明世族,名亞路敬淳。天授初,為司府主簿,受詔往淮南安撫。使還,賜爵河東縣男。景龍中,累遷
 為左散騎常侍,修國史。



 初,貞觀中太宗命學者撰《氏族志》百卷,以甄別士庶;至是向百年,而諸姓至有興替,沖乃上表請改修氏族。中宗命沖與左僕射魏元忠及史官張錫、徐堅、劉憲等八人,依據《氏族志》,重加修撰。元忠等施功未半,相繼而卒,乃遷為外職。至先天初,沖始與侍中魏知古、中書侍郎陸象先及徐堅、劉子玄、吳兢等撰成《姓族系錄》二百卷,奏上。



 沖後歷太子詹事、太子賓客、宋王傅、昭文館學士,以老疾致仕。開元二年,又敕沖
 及著作郎薛南金刊定《系錄》,奏上,賜絹百匹。五年卒。



 盧粲,幽州範陽人,後魏侍中陽烏五代孫。祖彥卿,撰《後魏紀》二十卷,行於時,官至合肥令。叔父行嘉,亦有學涉,高宗時為雍王記室。粲博覽經史,弱冠舉進士。景龍二年,累遷給事中。時節愍太子初立,韋庶人以非己所生,深加忌嫉,勸中宗下敕令太子卻取衛府封物,每年以供服用。粲駁奏曰:「皇太子處繼明之重,當主鬯之尊,歲時服用,自可百司供擬。又據《周官》,諸應用財器,歲終則
 會,唯王及太子應用物,並不會。此則儲君之費,咸與王同。今與列國諸侯齊衡入封,豈所謂憲章在昔,垂法將來者也!必謂青宮初啟,服用所資,自當廣支庫物,不可長存籓封。」詔從之。



 後安樂公主婿武崇訓為節愍太子所殺,特追封為魯王,令司農少卿趙履溫監護葬事。履溫諷公主奏請依永泰公主故事,為崇訓造陵。詔從其請。粲駁奏曰:



 伏尋陵之稱謂,本屬皇王及儲君等。自皇家已來,諸王及公主墓,無稱陵者。唯永泰公主承恩特
 葬,事越常塗,不合引以為名。《春秋左氏傳》云:「衛孫桓子與齊戰。衛新築大夫仲叔于奚救孫桓子,桓子以免。衛人賞之以邑,於奚辭,請曲懸、繁纓以朝,許之。仲尼聞之,曰:『惜也,不如多與之邑。唯名與器,不可以假人。若以假人,與之政也,政亡則國從之。』」聖人知微知章,不可不慎。魯王哀榮之典,誠別承恩;然國之名器,豈可妄假!又塋兆之稱,不應假永泰公主為名,請比貞觀已來諸王舊例,足得豐厚。



 手敕答曰:「安樂公主與永泰公主無異。同
 穴之義,古今不殊。魯王緣自特為陵制,不煩固執。」粲又奏曰:



 臣聞陵之稱謂,施於尊極,不屬王公已下。且魯王若欲論親等第,則不親於雍王。雍王之墓,尚不稱陵,魯王自不可因尚公主而加號。且君之舉事,則載於方冊,或稽之往典,或考自前朝。臣歷檢貞觀已來,駙馬墓無得稱陵者。且君人之禮,服絕於傍期,蓋為不獨親其親,不獨子其子。陛下以膝下之恩愛,施及其夫,贈賵之儀,哀榮足備,豈得使上下無辨,君臣一貫者哉!又安樂公
 主承兩儀之澤,履福祿之基,指南山以錫年,仰北辰而永庇。魯王之葬,車服有章,加等之儀,備有常數,塋兆之稱,不應假永泰公主為名,非所謂垂法將來,作則群闢者也!



 帝竟依粲所奏。公主大怒。粲以忤旨出為陳州刺史。累轉秘書少監。開元初卒。



 尹知章,絳州翼城人。少勤學,嘗夢神人以大鑿開其心,以藥內之,自是日益開朗,盡通諸經精義。未幾,而諸師友北面受業焉。長安中,駙馬都尉武攸暨重其經學,奏
 授其府定王文學。神龍初,轉太常博士。中宗初即位,建立宗廟,議者欲以涼武昭王為始祖,以備七代之數。知章以為武昭遠世,非王業所因,特奏議以為不可。當時竟從知章之議。俄拜陸渾令,以公玷棄官。時散騎常侍解琬亦罷職歸田園,與知章共居汝、洛間,以修學為事。



 睿宗初即位,中書令張說薦知章有古人之風,足以鎮雅俗,拜禮部員外郎。俄轉國子博士。後秘書監馬懷素奏引知章就秘書省與學者刊定經史。知章雖居吏
 職,歸家則講授不輟,尤明《易》及莊、老玄言之學,遠近咸來受業。其有貧匱者,知章盡其家財以衣食之。



 性和厚,喜慍不形於色,未嘗言及家人產業。其子嘗請並市樵米,以備歲時之費,知章曰:「如汝所言,則下人何以取資?吾幸食祿,不宜奪其利也!」竟不從。



 開元六年卒,時年五十有餘。所注《孝經》、《老子》、《莊子》、《韓子》、《管子》、《鬼谷子》,頗行於時。門人孫季良等立碑於東都國子監之門外,以頌其德。



 孫季良者,河南偃師人也,一名翌。開元中,為左拾遺、集賢院直學士。撰《正聲詩集》三卷,行於代。



 徐岱,字處仁,蘇州嘉興人也。家世以農為業。岱好學,六籍諸子,悉所探究,問無不通,難莫能詘。大歷中,轉運使劉晏表薦之,授校書郎。浙西觀察使李棲筠厚遇之,敕故所居為復禮鄉。尋為朝廷推援,改河南府偃師縣尉。建中年,禮儀使蔣鎮特薦為太常博士,掌禮儀。從幸奉天、興元,改膳部員外郎兼博士。貞元初,遷水部郎中,充
 皇太子及舒王已下侍讀。尋改司封郎中,擢拜給事中,加兼史館修撰,並依舊侍讀。承兩宮恩顧,時無與比。而謹慎過甚,未嘗洩禁中語,亦不談人之短。婚嫁甥侄之孤遺者,時人以此稱之。然吝嗇頗甚,倉庫管鑰,皆自執掌,獲譏於時。卒,時年五十。上嘆惜之,賻以帛絹。皇太子又遺絹一百疋,贈禮部尚書。



 蘇弁,字元容,京兆武功人。曾叔祖良嗣,天后朝宰相,國史有傳。弁少有文學,舉進士,授秘書省正字,轉奉天主
 簿。



 硃泚之亂,德宗倉卒出幸,縣令杜正元上府計事;聞大駕至,官吏惶恐,皆欲奔竄山谷。弁諭之曰:「君上避狄,臣下當伏難死節。昔肅宗幸靈武,至新平、安定,二太守皆潛遁,帝命斬之以徇,諸君知其事乎!」眾心乃安。及車駕至,迎扈儲備無闕。德宗嘉之,就加試大理司直。賊平,拜監察御史,歷三院,累轉倉部郎中。仍判度支案。



 裴延齡卒,德宗聞其才,特開延英,面賜金紫。授度支郎中,副知度支事,仍命立於正郎之首。副知之號,自弁始也。承
 延齡之後,以寬簡代煩虐,人甚稱之。遷戶部侍郎,依前判度支,改太子詹事。弁初入朝,班位失序,殿中侍御史鄒儒立對仗彈之。弁於金吾待罪數刻,特釋放。舊制,太子詹事班次太常、宗正卿已下。貞元三年,御史中丞竇參敘定班,移詹事在河南、太原尹之下。弁乃引舊班制立。臺官詰之,仍紿云:「自己白宰相,請依舊。」故為儒立彈之。旋坐給長武城軍糧朽敗,貶河州司戶參軍。當德宗時,朝臣受譴,少蒙再錄,至晚年尤甚。唯弁與韓皋得起
 為刺史,授滁州,轉杭州。



 弁與兄冕、袞,皆以友弟、儒學稱。



 冕,纘國朝政事,撰《會要》四十卷,行於時。弁聚書至二萬卷,皆手自刊校,至今言蘇氏書,次於集賢秘閣焉。貞元二十一年,卒於家。



 袞自贊善大夫貶永州司戶參軍,敕:「蘇袞貶官,本緣弟連坐。矜其年暮,加以疾患,宜令所在勒回,任歸私第。」袞年且七十,兩目無見已逾年。以弁之故,竟未停官。及貶,上聞之哀憫,故許還家。尋卒。



 初,冕既坐弁貶官,或有人言冕才學,上悔不早知。業已貶出,又
 復還袞,難於再追冕,乃止。



 陸質,吳郡人,本名淳,避憲宗名改之。質有經學,尤深於《春秋》,少師事趙匡,匡師啖助。助、匡皆為異儒,頗傳其學,由是知名。陳少游鎮揚州,愛其才,闢為從事。後薦於朝,拜左拾遺。轉太常博士,累遷左司郎中,坐細故,改國子博士,歷信、臺二州刺史。順宗即位,質素與韋執誼善,由是徵為給事中、皇太子侍讀,仍改賜名質。



 時執誼得幸,順帝寢疾,與王叔文等竊弄權柄。上在春宮,執誼懼,質
 已用事,故令質入侍,而潛伺上意,因用解。及質發言,上果怒曰:「陛下令先生與寡人講義,何得言他?」質惶懼而出。未幾病卒。質著《集注春秋》二十卷,《類禮》二十卷,《君臣圖翼》二十五卷,並行於代。貞元二十一年卒。



 馮伉,本魏州元城人。父玠,後家於京兆。少有經學。大歷初,登《五經》秀才科,授秘書郎。建中四年,又登博學《三史》科。三遷尚書膳部員外郎,充睦王已下侍讀。澤潞節度使李抱真卒,為吊贈使,抱真男遺伉帛數百匹,不納。又
 專送至京,伉因表奏,固請不受。屬醴泉闕縣令,宰臣進人名,帝意不可,謂宰臣曰:「前使澤潞不受財帛者,此人必有清政,可以授之。」遂改醴泉令。縣中百姓多猾,為著《諭蒙》十四篇,大略指明忠孝仁義,勸學務農,每鄉給一卷,俾其傳習。在縣七年,韋渠牟薦為給事中,充皇太子及諸王侍讀。召見於別殿,賜金紫。著《三傳異同》三卷。順宗即位,拜尚書兵部侍郎。改國子祭酒,為同州刺史。入拜左散騎常侍,復領太學。元和四年卒,年六十六,贈禮
 部尚書。



 子藥,進士擢第,又登制科,仕至尚書郎。



 韋表微,始舉進士登第,累佐籓府。元和十五年,拜監察御史。逾年,以本官充翰林學士。遷左補闕、庫部員外郎、知制誥。滿歲,擢遷中書舍人。俄拜戶部侍郎,職並如故。時自長慶、寶歷,國家比有變故,凡在翰林,遷擢例無滿歲,由是表微自監察,六七年間,秩正貳卿,命服金紫,承遇恩渥,盛於一時。卒,年六十。



 表微少時,克苦自立。著《九經師授譜》一卷,《春秋三傳總例》二十卷。



 子蟾,進士登第,
 咸通末,為尚書左丞。



 許康佐,父審。康佐登進士第,又登宏詞科。以家貧母老,求為知院官,人或怪之,笑而不答。及母亡,服除,不就侯府之闢,君子始知其不擇祿養親之志也,故名益重。遷侍御史,轉職方員外郎,累遷至駕部郎中,充翰林侍講學士,仍賜金紫。歷諫議大夫、中書舍人,皆在內庭。為戶部侍郎,以疾解職。除兵部侍郎,轉禮部尚。卒,年七十二,贈吏部尚書。撰《九鼎記》四卷。



 弟堯佐、元佐,堯佐子道
 敏,並登進士第,歷官清顯。



 贊曰:積學成功,開談辨治。儒道玄機,聖人雅旨。出必由戶,行跡其軌。邈有其人,光乎信史。



\end{pinyinscope}