\article{卷十九上 本紀第十九上 懿宗}

\begin{pinyinscope}

 懿宗昭聖恭惠孝皇帝漼,宣宗長子,母曰元昭皇太后晁氏。大和七年十一月十四日,生於籓邸。會昌六年十月,封鄆王,本名溫。大中十三年八月七日,宣遺詔立
 為皇太子監國,改今名。十三日,柩前即帝位,年二十七。帝姿貌雄傑,有異稠人。籓邸時常經重疾,郭淑妃侍醫藥,見黃龍出入於臥內。既間,妃以異告,帝曰:「慎勿復言。又嘗大雪數尺,而帝寢室之上獨無,人皆異之。宣宗制《泰邊陲樂曲詞》有「海岳晏咸通」之句。又大中末,京城小兒疊布漬水,紐之向日,謂之拔暈。帝果以鄆王即大位,以咸通為年號。



 九月,釋服,追尊母後晁氏為太后,謚曰元昭。十月癸未,制以門下侍郎、守左僕射、同平章事令狐
 綯守司空,門下侍郎、兵部尚書、同平章事蕭鄴兼尚書右僕射,中書侍郎、禮部尚書、平章事夏侯孜兼兵部尚書,中書侍郎、平章事蔣伸兼工部尚書,並依前知政事。又以兵部侍郎鄭顥為河南尹。以昭義軍節度、潞邢磁洺觀察等使、光祿大夫、檢校吏部尚書、兼潞州大都督府長史、上柱國、河東縣開國子、食邑五百戶裴休為太原尹、北都留守、河東節度管內觀察處置等使;以河中節度使、檢校尚書左僕射畢諴為汴州刺史,充宣武軍
 節度、宋亳觀察等使。以中書舍人裴坦權知禮部貢舉。十二月,以戶部侍郎、翰林學士杜審權為檢校禮部尚書、河中晉絳節度等使。咸通元年春正月,上御紫宸殿受朝,對室韋使。



 二月,葬宣宗皇帝於貞陵。以右拾遺劉鄴充翰林學士。以河中節度使杜審權為兵部侍郎、判度支,尋以本官同平章事;以門下侍郎、守司徒、同平章事令狐綯檢校司徒、同平章事,出鎮河中;尚書左僕射、諸道鹽鐵轉運使杜忭
 同平章事。浙東觀察使王式斬草賊仇甫,浙東郡邑皆平。



 八月,以河東節度使裴休為鳳翔尹、鳳翔隴右節度使,以鳳翔隴右節度使、銀青光祿大夫、檢校刑部尚書盧簡求為太原尹、北都留守、河東節度使。



 十一月丙午朔。丁未,上有事於郊廟,禮畢,御丹鳳門,大赦,改元。以中書舍人薛耽權知貢舉。



 二年春二月,吏部尚書蕭鄴檢校尚書右僕射、太原尹、北都留守、河東節度觀察等使。鄭滑節度使、檢校工部
 尚書李福奏:「屬郡潁州去年夏大雨,沈丘、汝陰、潁上等縣平地水深一丈,田稼、屋宇淹沒皆盡,乞蠲租賦。」從之。以中書侍郎兼工部尚書蔣伸兼刑部尚書,右僕射、門下侍郎杜忭為左僕射,依前知政事。四月,以前婺州刺史裴閔為潁州刺史,充本州團練鎮遏等使。以駕部郎中王鐸本官知制誥。八月,以中書舍人衛洙為工部侍郎。尋改銀青光祿大夫、檢校禮部尚書,兼滑州刺史、御史大夫、駙馬都尉,充義成軍節度、鄭滑潁觀察處置等
 使。洙奏狀稱:「蒙恩除授滑州刺史,官號內一字與臣家諱音同,雖文字有殊,而聲韻難別,請改授閑官者。」敕曰:「嫌名不諱,著在禮文,成命已行,固難依允。」以兵部侍郎曹確判度支,以兵部員外郎楊知遠、司勛員外郎穆仁裕試吏部宏詞選人。



 九月,以前兵部侍郎、判度支畢諴為工部尚書、同平章事。蔣伸罷知政事。林邑蠻寇安南府,遣神策將軍康承訓率禁軍及江西、湖南之兵赴援。



 三年春正月,左僕射、門下侍郎、平章事杜忭率百僚上
 徽號曰睿文明聖孝德皇帝。



 五月,敕:「嶺南分為五管,誠已多年。居常之時,同資御捍,有事之際,要別改張。邕州西接南蠻,深據黃洞,控兩江之獷俗,居數道之游民。比以委人太輕,軍威不振,境連內地,不並海南。宜分嶺南為東、西道節度觀察處置等使,以廣州為嶺南東道,邕州為嶺南西道,別擇良吏,付以節旄。其所管八州,俗無耕桑,地極邊遠,近罹盜擾,尤甚凋殘。將盛籓垣,宜添州縣。宜割桂州管內龔州、象州,容州管內藤州、巖州,並隸
 嶺南西道收管。」宰臣杜忭兼司空,畢諴兼兵部尚書。駕部郎中、知制誥王鐸為中書舍人。以邕管經略使鄭愚為廣州刺史,充嶺南東道節度、觀察處置等使;將軍宋戎為嶺南西道節度使。夏,淮南、河南蝗旱,民饑。南蠻陷交址,徵諸道兵赴嶺南。詔湖南水運,自湘江入澪渠,江西造切麥粥以饋行營。湘、漓溯運,功役艱難,軍屯廣州乏食。潤州人陳磻石詣闕上書,言:「江西、湖南,溯流運糧,不濟軍師,士卒食盡則散,此宜深慮。臣有奇計,以饋南軍。
 天子召見,磻石因奏:「臣弟聽思曾任雷州刺史,家人隨海船至福建,往來大船一隻,可致千石,自福建裝船,不一月至廣州。得船數十艘,便可致三萬石至廣府矣。」又引劉裕海路進軍破盧循故事。執政是之,以磻石為鹽鐵巡官,往楊子院專督海運。於是康承訓之軍皆不闕供。七月,徐州軍亂,以浙東觀察使王式檢校工部尚書、徐州刺史、御史大夫、武寧軍節度、徐泗濠觀察等使。初,王智興得徐州,召募兇豪之卒二千人,號曰銀刀、雕旗、
 門槍、挾馬等軍,番宿衙城。自後浸驕,節度使姑息不暇。田牟鎮徐日,每與驕卒雜坐,酒酣撫背,時把板為之唱歌。其徒日費萬計。每有賓宴,必先厭食飫酒,祁寒暑雨,卮酒盈前,然猶喧噪邀求,動謀逐帥。前年壽州刺史溫璋為節度使,驕卒素知璋嚴酷,深負憂疑。璋開懷撫諭,終為猜貳,給與酒食,未嘗瀝口,不期月而逐璋。上是以式代璋。時式以忠武、義成之師三千平定仇甫,便詔式率二鎮之師渡淮。徐卒聞之,懼其勢,無如之何。至大彭
 館,方來迎謁。居三日,犒勞兩鎮兵令還,既擐甲執兵,即命環驕卒殺之。徐卒三千餘人,是日盡誅,由是兇徒悉殄。



 九月,以戶部侍郎李晦檢校工部尚書,兼興元尹、山南西道節度使。



 十一月,遣將軍蔡襲率禁軍三千,會諸道之師赴援安南。以吏部侍郎鄭處誨蕭仿、吏部員外郎楊儼、戶部員外郎崔彥昭等試宏詞選人。十二月,以吏部侍郎蕭仿權知禮部貢舉。



 四年春正月甲子朔。庚午,上有事於圓丘,禮畢,御丹鳳
 樓,大赦。中外官宜準建中元年敕,授官後三日舉一人自代。州牧令錄上佐官,在任須終三考。河東節度使、檢校刑部尚書盧簡求以病求罷,詔以太子少師致仕歸東都。以昭義節度使、檢校禮部尚書、上柱國、賜紫金魚袋劉潼為太原尹、北都留守、御史大夫,充河東節度觀察處置等使。



 二月,以左散騎常侍李荀檢校工部尚書、滑州刺史、義成軍節度、鄭滑觀察等使。



 三月,以兵中侍郎、判度支楊收本官同平章事;以刑部侍郎曹汾為河
 南尹;以戶部侍郎李蠙檢校禮部尚書、潞州大都督府長史,充昭義節度、觀察處置等使。四月,敕徐州罷防禦使,為支郡,隸兗州。七月朔,制:「安南寇陷之初,流人多寄溪洞。其安南將吏官健走至海門者人數不少,宜令宋戎、李良瑍察訪人數,量事救恤。安南管內被蠻賊驅劫處,本戶兩稅、丁錢等量放二年,候收復後別有指揮。其安南溪洞首領,素推誠節,雖蠻寇竊據城壁,而酋豪各守土疆。如聞溪洞之間,悉藉嶺北茶藥,宜令諸道一任
 商人興販,不得禁止往來。廉州珠池,與人共利。近聞本道禁斷,遂絕通商,宜令本州任百姓採取,不得止約。其徐州銀刀官健,其中先有逃竄者,累降敕旨,不令捕逐。其今年四月十八日,草賊頭首已抵極法,其餘徒黨各自奔逃,所在更勿捕逐。」是月,東都、許、汝、徐、泗等州大水,傷稼。初,大中末,安南都護李琢貪暴,侵刻獠民,群獠引林邑蠻攻安南府。三年,大徵兵赴援,天下騷動。其年冬,蠻竟陷交州,赴安南諸軍並令抽退,分保嶺南東、西道。



 十一月,長安縣尉、集賢校理令狐滈為左拾遺。制出,左拾遺劉蛻、起居郎張云上疏,論滈父綯秉權之日,廣納賂遺,受李琢賄,除安南,致生蠻寇,滈不宜居諫諍之列。時綯在淮南,上表論訴,乃貶雲興元少尹,蛻華陰令,水高改詹事司直。以中書舍人王鐸權知禮部貢舉,以兵部侍郎、判度支曹確同平章事,以中書侍郎、平章事畢諴檢校吏部尚書、河中尹、晉絳慈隰節度使。就加幽州張允伸檢校司徒。以兵部侍郎高璩本官同平章事,以戶
 部侍郎裴寅判本司事。



 五年春正月戊午朔,以用兵罷元會。諫議大夫裴坦上疏,論天下徵兵,財賦方匱,不宜過興佛寺,以困國力。優詔答之。



 二月,以兵部尚書牛叢檢校兵部尚書,兼成都尹、劍南西川節度副大使、知節度事。徐州處置觀察防禦使。以門下侍郎、兵部尚書、平章事杜審權為潤州刺史、浙江西道節度使。三月,以兵部郎中高湜、員外於懷試吏部,平判選人。四月,右僕射、平章事夏侯孜增爵五
 百戶。以中書舍人王鐸為禮部侍郎,以晉州刺史孟球檢校工部尚書,兼徐州刺史。南蠻寇邕管,以秦州經略使高駢率禁軍五千赴邕管,會諸道之師御之。



 五月丁酉,制:



 朕以寡昧,獲承高祖、太宗之丕構,六載於茲矣。罔畋游是娛,罔聲色是縱,罔刑戮是濫,罔邪佞是惑。夙夜悚惕,以憂以勤,庶幾乎八表用康,兆人以泰。而西戎款附,北狄懷柔,獨惟南蠻,奸宄不率。侵陷交趾,突犯朗寧,爰及雋州,亦用攘寇。勞我士卒,興吾甲兵,騷動黎元,役
 力飛輓,每一軫念,閔然疚懷。顧惟生人,罹此愁苦,宜布自天之澤,俾垂及物之仁。如聞湖南、桂州,是嶺路系口,諸道兵馬綱運,無不經過,頓遞供承,動多差配,凋傷轉甚,宜有特恩。潭、桂兩道各賜錢三萬貫文,以助軍錢,亦以充館驛息利本錢。其江陵、江西、鄂州三道,比於潭、桂,徭配稍簡,宜令本道觀察使詳其閑劇,準此例與置本錢。邕州已西黎、雋界內,昨因蠻寇,互有殺傷,宜令本道收拾埋瘞,量設祭酹。



 徐州土風雄勁,甲士精強,比以制
 馭乖方,頻致騷擾。近者再置使額,卻領四州,勞逸既均,人心甚泰。但聞比因罷節之日,或有被罪奔逃,雖朝廷頻下詔書,並令一切不問,猶恐尚懷疑懼,未委招攜,結聚山林,終成詿誤。況邊方未靜,深藉人才,宜令徐泗團練使選揀召募官健三千人,赴邕管防戍。待嶺外事寧之後,即與替代歸還。仍令每召滿五百人,即差軍將押送,其糧料賞給,所司準例處分。



 淮南、兩浙海運,虜隔舟船,訪聞商徒,失業頗甚,所由縱舍,為弊實深。亦有搬貨財
 委於水次,無人看守,多至散亡,嗟怨之聲,盈於道路。宜令三道據所搬米石數,牒報所在鹽鐵巡院,令和雇入海同船,分付所司。通計載米數足外,輒不更有隔奪,妄稱貯備。其小舸短船到江口,使司自有船,不在更取商人舟船之限。如官吏妄行威福,必議痛刑。於戲!萬方靡安,寧忘於罪己;百姓不足,敢怠於責躬。用伸欽恤之懷,式表憂勤之旨。



 壬寅,制以中書侍郎、平章事楊收為門下侍郎、兼刑部尚書,以中書侍郎、平章事曹確兼工部
 尚書,兵部侍郎、平章事高璩為中書侍郎、知政事,餘並如故。



 秋七月壬子,延資庫使夏侯孜奏:



 鹽鐵戶部先積欠當使咸通四年已前延資庫錢絹三百六十九萬餘貫匹。內戶部每年合送錢二十六萬四千一百八十貫匹,從大中十二年至咸通四年九月已前,除納外,欠一百五十萬五千七百一十四萬貫匹。當使緣戶部積欠數多,先具申奏,請於諸道州府場鹽院合納戶部所收八十文除陌錢內,割一十五文,屬當使自收管。敕命雖
 行,送納稽緩。今得戶部牒稱,所收管除陌錢絹外,更有諸雜物貨,延資庫徵收不便,請起今年合納延資庫錢絹一時便足。其已前積欠,候物力稍充,積漸填納。其所割一十五文錢,即當司仍舊收管。又緣累歲以來,嶺南用兵,多支戶部錢物。當使不欲堅論舊欠,請依戶部商量,合納今年一年額色錢絹須足,明年即依舊制,三月、九月兩限送納畢。其以前積欠,仍令戶部自立填納期限者。



 敕旨依之。



 十月丙辰,以中書舍人李蔚權知禮部
 貢舉。



 十一月乙酉,以大同軍防禦使盧簡方檢校工部尚書、滄州刺史、御史大夫,充義昌軍節度、滄濟德觀察等使。乙未,以兵部侍郎蕭寘本官同中書門下平章事。



 六年正月癸未朔。丁亥,制以河東節度使、檢校刑部尚書孔溫裕為鄆州刺史、天平軍節度、鄆曹棣觀察處置等使。



 二月,制以御史中丞徐商為兵部侍郎、同平章事。高璩罷知政事。以吏部尚書崔慎由、吏部侍郎鄭從讜、吏部侍郎王鐸、兵部員外郎崔謹張彥遠等考宏詞選
 人;金部員外郎張乂思、大理少卿董賡試拔萃選人。以給事中楊嚴為工部侍郎,尋召為翰林學士。四月,西川節度使牛叢奏於蠻界築新城、安城、遏戎州功畢。時南詔蠻入寇姚、雋,陳許大將顏復戍雋州新築二城。其年秋,六姓蠻攻遏戎州,為復所敗,退去。兵部侍郎、平章事徐商、蕭寘轉中書侍郎、知政事。



 五月,以左丞楊知溫為河南尹,以神策大將軍馬舉為秦州經略招討使,以右金吾大將軍李宴元為夏州刺史、朔方節度等使。安
 南都護高駢奏於邕管大敗林邑蠻。七月,以右衛大將軍薛綰檢校工部尚書、徐州刺史,充徐泗團練觀察防禦等使。



 九月,以中書舍人趙騭權知禮部貢舉;以吏部侍郎蕭仿檢校禮部尚書、滑州刺史、御史大夫,充義成軍節度、判滑潁觀察等使。十二月,太皇太后鄭氏崩,謚曰孝明。是歲秋,高駢自海門進軍破蠻軍,收復安南府。自李琢失政,交趾湮沒十年,蠻軍北寇邕容界,人不聊生,至是方復故地。



 七年春正月戊寅朔,以太皇太后喪罷元會。



 三月,成德軍節度、鎮冀深趙等州觀察處置等使、金紫光祿大夫、檢校司空、鎮州大都督府長史、御史大夫、太原縣開國伯、食邑七百戶、襲食實封一百戶王紹懿卒,贈司徒。紹鼎之弟,俱壽安公主之子也。三軍推紹鼎子景崇知兵馬留後事。就加幽州張允伸兼太保、平章事,進封燕國公。以吏部侍郎鄭從讜檢校禮部尚書、兼太原尹、北都留守、御史大夫、上柱國、滎陽縣開國男、食邑三百戶,充
 河東節度管內觀察處置等使。四月,壽安公主上表請入朝,詔曰:「志興奏汝以景崇未降恩命,欲來朝覲事,具悉。景崇素聞孝悌,頗有義方,洽三軍愛戴之情,荷千里折沖之寄。纘乃舊服,綽有令猷,朝廷獎能,續有處分。緣孝明太后園寢有日,庶事且停,候祔廟禮成,當允誠請。」七月,沙州節度使張義潮進甘峻山青鷹四聯、延慶節馬二匹、吐蕃女子二人。僧曇延進《大乘百法門明論》等。



 八月,鎮州王景崇起復忠武將軍、左金吾衛將軍同
 正、檢校右散騎常侍,兼鎮州大都督府左司馬、知府事、御史中丞,充成德軍節度觀察留後。上柱國、賜紫金魚袋、中書侍郎、平章事徐商兼工部尚書。十月,沙州張義潮奏:差回鶻首領僕固俊與吐蕃大將尚恐熱交戰,大敗蕃寇,斬尚恐熱,傳首京師。右僕射、門下侍郎、平章事夏侯孜檢校司空、平章事,兼成都尹、劍南西川節度等副大使、知節度使。安南高駢奏蠻寇悉平。



 十一月十日,御宣政殿,大赦,以復安南故也。以翰林學士承旨、戶部
 侍郎路巖為兵部侍郎、同平章事。義成軍節度蕭仿就加檢校兵部尚書,褒能政也。以禮部郎中李景溫、吏部員外郎高湘試拔萃選人。



 八年春正月壬寅朔。丁未,河中、晉、絳地大震,盧舍壓僕傷人,有死者。



 三月,安南高駢奏:「南至邕管,水路湍險,巨石梗途,令工人開鑿訖,漕船無滯者。」降詔褒之。制以門下侍郎、兼戶部尚書平章事、上柱國、晉陽縣開國男、食邑三百戶、賜紫金魚袋楊收檢校兵部尚書,充浙江西
 道觀察使;以浙西觀察使杜審權守尚書左僕射;以兵部侍郎於忭本官同平章事。



 九月丁酉,延資庫使曹確奏:



 戶部每年合送當使三月、九月兩限絹二十一萬四千一百匹,錢萬貫,自大中八年已後,至咸通四年,積欠一百五十萬五千七百餘貫匹。前使杜忭申奏,請起咸通五年正月以後,於諸道州府場監院合送戶部八十文除陌錢內,割十五文當使收管,以填積欠。續據戶部牒稱,州府除陌錢有折色零碎,請起咸通五年所合送
 延資庫錢絹,逐年兩限須足,其除陌十五文,當司仍舊收管。前使夏侯孜具事由申奏,且請依戶部論請期限。其咸通五年錢絹,戶部已送納。自六年至八年,其錢絹依前不旋送納,又積欠三十六萬五千五百七貫匹者。伏以所置延資庫,初以備邊為名,至大中三年始改今號。若財貨不充,則名額虛設。當制置之時,所令三司逐年分減送當使收管。元敕只有錢數,但令本司減割送庫,不定色目。以此因循,漸隳舊制,年月既久,積欠漸多。既無
 計以徵收,乃指色以取濟,稍稱備邊名號,得遵元敕指揮。乃割戶部除陌八十文內十五文收管,及戶部請逐年送庫,須且稟從。今既積欠又多,終慮不及期限。臣今酌量諸道州府場監院合送戶部錢絹內分配,令勒留下合送延資庫數目,令本處別為綱運,與戶部綱同送上都,直納延資庫,則戶部免有逋懸,不至累年積欠。從之。



 十月丙寅,戶部侍郎、判度支崔彥昭奏:當司應收管江、淮諸道州府咸通八年已前兩稅榷酒及支米價,並
 二十文除陌諸色屬省錢,準舊例逐年商人投狀便換。自南蠻用兵已來,置供軍使,當司在諸州府場監錢,猶有商人便換,齎省司便換文牒至本州府請領,皆被諸州府稱準供軍使指揮占留。以此商人疑惑,乃致當司支用不充。乞下諸道州府場監院依限送納及給還商人,不得托稱占留者。」敕旨從之。宰相、門下侍郎、戶部尚書曹確兼吏部尚書,門下侍郎、禮部尚書路巖兼戶部尚書,中書侍郎、工部尚書徐商兼刑部尚書,兵部侍郎、
 平章事於忭為中書侍郎。以中書舍人劉允章權知禮部貢舉,以吏部侍郎盧匡、吏部侍郎李蔚、兵部員外郎薛崇、司勛員外郎崔殷夢考吏部宏詞選人。



 九年春正月丙申,以吏部侍郎李蔚檢校刑部尚書、汴州刺史、御史大夫,充宣武節度、汴宋亳觀察處置等使。幽州節度使張允伸就加檢校太傅。以兵部員外郎焦瀆、司勛員外郎李嶽考宏詞選人。七月戊戌,白虹橫亙西方。其月,徐州赴桂林戍卒五百人,官健許佶、趙可立
 殺其將王仲甫,以糧料判官龐勛為都頭,剽掠湘潭、衡山兩縣,有眾千人,擅還本鎮。



 九月辛卯朔。甲午,龐勛陷宿州,知州判官焦潞奔歸於徐。乙未,龐勛陷徐州,殺節度使崔彥曾、判官焦潞、李稅溫延皓、崔蘊、韋廷乂,惟免監軍張道謹。遂出徐、宿官庫錢帛,召募兇徒,不旬日其徒五萬。勛抗表請罪,仍命群兇邀求節鉞。上遣中使因而撫之。賊令別將梁伾守宿州,以姚周為柳子寨主,又遣劉行及、丁景琮、吳玫迥攻圍泗州。十月,詔征河南、河東、
 山南諸道之師。貶浙西觀察使楊收為端州司馬同正,收弟前浙東觀察使、越州刺史、御史中丞嚴為韶州刺史,檢校工部尚書、洪州刺史、鎮南節度、江南西道觀察處置等使嚴譔長流嶺南。賊攻泗州勢急,淮南節度使令狐綯慮失泗口,為賊奔沖,乃令大將李湘赴援,為賊所誘,示弱乞降,乘其無備,為賊所襲,舉軍皆沒。湘與都監郭厚本俱為賊所執,送徐州。



 十一月庚寅朔。丁酉戌時,妖星初出,如匹練亙空,化為雲,沒在楚分。吳迥既執
 李湘,乃令小將張行簡、吳約攻滁州。城內無兵,有淮南游奕兵三百人在州界,見賊至,徑來奔郡,賊乘之,遂陷滁州。張行簡執刺史高錫望,手刃之,屠其城而去。行簡又進攻和州,刺史崔雍登城樓謂吳迥曰:「城中玉帛、女子不敢惜,只勿取天子城池。」賊許之,遂剽城中居民,殺判官張琢,以琢浚城壕故也。龐勛又令將劉贄攻濠州,陷之,囚刺史盧望回於回車館,望回鬱憤而死,僕妾數人皆為賊蒸而食之。十二月庚辰朔,將軍戴可師率沙陀、
 吐渾部落二萬人,於淮南與賊轉戰,賊黨屢敗,盡棄淮南之守。是歲,江、淮蝗食稼,大旱。龐勛奏:「當道先發戍嶺南兵士三千人春冬衣,今欲差人送赴邕管。」鄂岳觀察使劉允章上書言;「龐勛聚徒十萬,今若遣人達嶺表,如戍卒與勛合勢,則禍難非細。」尋詔龐勛止絕,兼令江、淮諸道紀綱捕之。



 十年春正月己未朔,以徐州用兵罷元會。癸亥,以右拾遺韋保衡為銀青光祿大夫、守起居郎、駙馬都尉,尚皇
 女同昌公主,出降之日,禮儀甚盛。以神武大將軍王晏權檢校工部尚書、徐州刺史、御史大夫,充武寧軍節度、徐泗濠觀察,兼徐州北路行營招討等使,智興之從子也;以將軍硃克誠充北路招討都虞候;王宥北路招討前軍使。以翰林學士、戶部侍郎劉瞻守本官同平章事。中書侍郎、兼戶部尚書、平章事蔣伸為太子太保,罷知政事,病免也。以門下侍郎、兼刑部尚書、同平章事徐商檢校兵部尚書、江陵尹、荊南節度使。以右神策大將軍、
 知軍使、兼御史大夫、上柱國、龍陽縣開國伯、食邑一千戶康承訓可金紫光祿大夫、檢校刑部尚書、兼右神策大將軍、御史大夫、上柱國、扶風郡開國公、食邑一千五百戶,充徐泗行營都招討使;又以將軍李邵為徐州南路行營招討都虞候;以將軍史忠用為潁州行營都知兵馬使;將軍馬澹為徐州行營都知兵馬使;將軍董濤充盧州行營都知兵馬使;將軍戴可師充曹州行營招討使;將軍硃邪赤心充太原行營招討使、沙陀三部落
 等軍使;將軍王建充淮泗行營招討使;將軍曹翔充兗海節度行營招討使;將軍馬舉為揚州都督府司馬,充淮南行營招討使;將軍高羅銳為楚州刺史,本州行營招討使;將軍秦匡謨為濠州刺吏,本州行營如討使,李播為宿州刺史,赴盧州行營招討使;以將軍孟彪為太僕卿,充都糧料使。凡十八將,分董諸道之兵七萬三千一十五人,正月一日進軍攻徐州。魏博何弘敬奏當道點檢兵馬一萬三千赴行營。時賊將劉行及、丁景
 琮、吳迥攻圍泗州,可師乘勝救之,屯於石梁驛。賊自退去,可師追擊,生擒劉行及,賊保都梁城,乃斷行及之指,懸於城下以示賊。賊登城拜曰:「見與都頭謀歸朝。」可師既知其窘,乃退軍五里。其城西面有水,三面大軍,賊乃夜中涉水而遁。明早開城門,惟病嫗數人而已。王師入壘未整,翌日詰旦重霧,賊軍大至,可師方大醉,單馬奔出,為虹縣人郭真所殺,一軍盡沒,惟忠武、太原、沙陀之騎軍保全而退。副將王健為賊所擒,劉行及卻為賊將
 吳迥所得,吳迥乃進軍復圍泗州。自是梯沖雲合,內外不通。龐勛恃其驟勝,遣人上表,詞語不恭,又與康承訓書,指斥朝政。王晏權者,智興之猶子也,故授以武寧節制以招之,以冀招懷。徐人怨王式之誅,相扇構亂,數月招攜,啖之以利,民闕卒無革心者。康承訓大軍攻宿州,賊將梁伾出戰屢敗,乃授承訓檢校尚書右僕射,靈滑州刺史、義成軍節度使。責授端州司馬楊收長流驩州,與嚴譔並賜死於路;其黨楊公慶、嚴季實、楊全益、史明、
 廉遂、何師玄、李孟勛、馬全祐、李羽、王彥復等長流儋、崖、播等州;判官硃偘、常濆、閻均等配流嶺南。以河中節度使、開府儀同三司、檢校司徒、平章事、上柱國、譙郡開國公、食邑二千戶夏候孜為太子少保,分司東都。時南平蠻寇西川,責孜在蜀日失政也。



 二月己丑,龐勛急攻泗州,遣牙將李員入城見刺史杜慆曰:「留後知中丞名族,不敢令軍士失禮,但開城門,令百姓存活,無相疑也。」慆執而殺之。詔司農卿薛瓊使淮南盧、壽、楚等州,點集鄉
 兵以自固。四月,康承訓奏大敗柳子寨賊,詔監軍楊玄價與康承訓商量,拔汴河水以灌宿州。



 六月丁亥朔。戊戌,制曰:



 動天地者莫若精誠,致和平者莫若修政。朕顧惟庸昧,托於王公之上,於茲十一年矣。祗荷丕構,寅畏小心,慕唐堯之欽若昊天,遵周王之昭事上帝。念茲夙夜,靡替虔恭,同馭朽之憂勤,思納隍之軫慮。內戒奢靡,外罷畋游,匪敢期於雍熙,所自得於清凈,止望寰區無事,稼穡有年。然而燭理不明,涉道唯淺,氣多堙鬱,誠未
 感通。旱是虞,蟲螟為害,蠻蜒未賓於遐裔,寇盜復蠹於中原。尚駕戎車,益調兵食,俾黎元之重困,每宵旰而忘安。今盛夏驕陽,時雨久曠,憂勤兆庶,旦夕焦勞。內修香火以虔祈,外罄牲玉以精禱。仰俟玄貺,必致甘滋。而油雲未興,秋稼闕望,因茲愆亢,軫於誠懷。矧復暴政煩刑,強官酷吏,侵漁蠹耗,陷害孤煢,致有冤抑之人,構成災沴之氣。主守長吏,無忘奉公。伐叛興師,蓋非獲已,除奸討逆,必使當辜,茍或陷及平人,自然風雨愆候。凡行營
 將帥,切在審詳,昭示惻憫之心,敬聽勤恤之旨。應京城天下諸州府見禁囚徒,除十惡忤逆、官典犯贓、故意殺人、合造毒藥、放火持仗、開劫墳墓及關連徐州逆黨外,並宜量罪輕重,速令決遣,無久系留。雷雨不同,田疇方瘁,誠宜愍物,以示好生。其京城未降雨間,宜令坊市權斷屠宰。昨陜虢中使回,方知蝗旱有損處,諸道長史,分憂共理,宜各推公,共思濟物。內有饑歉,切在慰安,哀此蒸人,毋俾艱食。徐方寇孽未殄,師旅有征,凡合誅鋤,審
 分淑慝,無令脅從橫死,元惡偷生。宜申告伐之文,使知逆順之理。於戲!每思禹、湯之罪己,其庶成、康之措刑。孰謂德信未孚,教化猶梗。咨爾多士,俾予一人,既引過在躬,亦漸幾於理。布告中外,稱朕意焉。賊將鄭鎰急攻壽州,詔南面招討使馬舉救之,賊解圍而去。康承訓悉兵攻賊小睢寨,不利而退。七月,康承訓攻賊柳子寨,垂克而賊將王弘立救至,王師大敗,承訓退保宋州。龐勛乘勝自率徐州勁卒並攻泗州,留其都將許佶守徐州。詔
 南面招討使馬舉為行營都招討使,代承訓率諸軍以援泗州。



 八月,和州防虞行官石侔等一百三十人狀訴刺史崔雍,稱:「賊初劫烏江縣,雍令步奏官二人探知,雍猶不信,二人並被枷杻。續差人探見賊已去州十里。賊尋逼州城,崔雍與賊頭吳約於鼓角樓上飲酒,許與賊州。又認軍事判官李譙為親弟,表狀驅使官張立為男,只乞二人並身,其餘將士一任處置。便令押衙李詞等各脫下衣甲,防虞官健束手被斬者八百餘人。行官石
 瓊脫衣甲稍遲,便被崔雍遣賊處斬。其崔雍所有料錢並家口,累差人押送往採石,今在潤州。豈有將一千人兵士之命,贖拔己之一身,不惟辜其神明,實亦生負聖主。兼科配軍州官吏修葺城池,妄稱出料錢修城者。」敕曰:「臣子之節,無如盡忠;士人之風,宜當遠恥。崔雍任居牧守,賊犯州城,御捍曾不發言,從容乃與命酒。況石瓊未脫衣甲,志在當鋒,不能獎其赤誠,翻令擒送賊所。原其深意,與賊通和,臣節全虧,情狀可見,欲行朝典,宜更
 推窮。其崔雍家口並在宣州,宜令宣歙觀察使追崔雍收禁速勘,逐具事由申奏。」是月,馬舉率師解泗州之圍,賊黨遁去。敕曰:「當崔雍守郡之日,是龐勛肆逆之初。屬狂寇奔沖,望風和好,置酒以邀賊將,啟關而納兇徒。城內不許持兵,皆令解甲,致使三軍百姓,抆血相視,連頭受誅。初聞奏陳,深駭觀聽。錫望守城而死,已有追榮;杜慆孤壘獲全,尋加殊獎。既褒忠節,難赦罪人,玉石固分,懲勸斯在。將垂誡於四海,當何愛於一夫。其崔雍宜差
 內養孟公度專往宣州,賜自盡。」公度至,雍死於陵陽館,其男黨兒、歸僧配流康州,錮身遞送。司勛郎中崔原貶柳州司戶,比部員外郎崔福昭州司戶,長安縣令崔朗澧州司戶,左拾遺崔庚連州司戶,荊南觀察支使崔序衡州司戶,皆雍之親黨也。



 九月,賊宿州守將張玄稔以城降,有兵萬人,馬舉率師赴之。龐勛聞之,以其眾將攻玄稔。玄稔,賊之勁將也,遂與舉合勢,急圍徐州。許佶登城拒守者三日,佶敗走出。玄稔收復徐州,龐勛方來赴援,聞
 城已拔,欲南趨濠州,馬舉追及渙河,擊敗之,勛溺水而死。蕭縣主將又斬許佶首來降,徐寇悉平。初,龐勛據徐州,倉庫素無貯蓄,乃令群兇四出,於揚、楚、盧、壽、滁、和、兗、海、沂、密、曹、濮等州界剽牛馬輓運糧糗,以夜繼晝。招致亡命,有眾二十萬,男女十五已上,皆令執兵,其人皆舒鋤鉤為兵,號曰「霍錐」。首尾周歲,十餘郡生靈,受其酷毒,至是盡平。與玄稔詔曰:「去歲災興分野,毒起徐方,蕞爾庸夫,稱兵犯命,招諭不復,猖狂罔悛,脅從三州之人,污
 染萬姓之俗。逆順之理,邪正坐分,果有忠臣,悉殲逆黨,再清郡邑,不舉干戈。此皆眾人協心,合州受福。但以首尾周歲,取制兇威,裏閭不安,農桑失業,言念於此,倍積憂懷。已有詔指揮,今授玄稔銀青光祿大夫、檢校右散騎常侍、兼右驍衛大將軍、御史大夫,賜分帛五千匹、金榼一枚、蓋碗一具、金腰帶一條。軍將張皋已下二十人,等第優給。今差高品李志承押領宣賜。」制曰:



 朕以眇身,獲承丕業,虔恭惕厲,十一載於茲。況荷十七聖之鴻休,
 紹三百年之慶祚,將求理本,敢忘宵衣。雖誠信未孚,而寅畏不怠,既絕意於苑囿,固無心於畋游,業業兢兢,日慎一日。休徵罔應,沴氣潛生,南蠻將罷於戰爭,徐寇忽孤於惠養。招諭不至,虐暴滋深,竊弄干戈,擅攻州鎮。將邀符印,輒恣兇殲,不畏神祗,自貽覆滅。股肱之臣,以罪惡之難舍;腹心之眾,謂悖逆之可誅。爰征甲兵,用救塗炭,上將宣力,內臣協心。選用皆得於良材,掃蕩才及於周歲,誅干紀反常之噍類,懲亂臣賊子之奸謀。



 今則已
 及偃戈,重康黎庶。疇庸之典,在絲發以無私;懋賞之時,貴纖毫之必當。其四面行營節度使,既成茂勛,宜加酬獎,並取別敕處分。應諸道行營都將已下節級軍將,各委本道具功勞名銜,分析聞奏,當續有處分。被堅執銳,冒涉寒暄,解甲橐弓,還鄉復業,頒繒帛之賜,免差役之徵。應四面行營將士,今既平寧,宜令次第放歸本道。其賞賜匹段,已從別敕處分,到本道後,仍令節度使各犒宴放歸私第,便令歇息,未用差使。如行營人,並免差
 科色役;如本廂本將,今後有節級員闕,且以行營軍健量材差置,用酬征伐之勤。臨敵用命,力屈殞身,須慰傷魂,以彰忠節。超與職事,仍加任使。如無父兄子弟,即有妻女者,即委州使厚加贈恤,常令安撫。如是都將至都虞候陣亡者,與贈官。應陣亡將士有父兄子弟願入軍者,便令本道填替。如無父兄子弟,仍且與給衣糧三年。因戰陣傷損手足永廢者,終身不得停給。如將士被賊殺害者,委所在州縣量事救接,重與改瘞,勿令暴露,兼
 與設祭。



 王者以仁恕為本,拯濟是謀,元惡既已誅鋤,脅從宜從寬宥。除寵勛親屬及桂州回戈逆黨,為賊脅從及因戰陣拒敵官軍,招諭不悛,懼法逃走,皆非本惡,蓋鋒刃所驅,今並釋放,一切不問。應舊軍將軍吏節級所由,既已歸還,征賦先宜蠲免。其徐、宿、濠、泗等州應合征秋夏兩稅及諸色差科色役,一事已上,宜放十年,已後蠲放三年,待三年後續議條疏處分。編甿失業,丘井無人,桑柘枌榆,鞠為茂草,應行營處百姓田宅產業為賊
 殘毀燒焚者,今既平寧,並許識認,各還本主,諸色人不得妄有侵占。九原可作,千載不忘,尚禁樵蘇,寧傷丘壟。應有先賢墳墓碑記為人所知,被賊毀廢者,即與掩藏,仍量致祭。自用兵已來,郡邑皆罹攻劫,遠念驚撓,尤在慰安。今遣右散騎常侍劉異、兵部郎中薛崇等往彼宣撫。於戲!朕以四海為家,兆人為子。一物失所,每軫納隍之憂;一方未寧,常負阽危之戒。今元兇就戮,逆黨誅夷,載戢干戈,永銷氛昆,庶平妖氣,允洽嘉祥。遐邇臣僚,當
 體予意。



 制以徐州南面招討使、檢校尚書左僕射、右神武大將軍、權知淮南節度事、扶風縣開國伯、食邑一千戶馬舉可檢校司空,兼揚州大都督府長史、淮南節度副大使、知節度事;以右武衛大將軍、徐州東南面招討使曹翔檢校兵部尚書,兼徐州刺史、御史大夫、徐泗濠團練防禦等使;以前淮南節度使、檢校司空、平章事、上柱國、涼國公、食邑三千戶令狐綯為太子太保,分司東都。魏博節度使、檢校太傅、同平章事何弘敬卒,三軍立
 其子全皞為兵馬留後。



 十一月,南詔蠻驃信坦綽酋龍率眾二萬寇雋州。定邊軍節度都頭安再榮守清溪關,為賊所攻,再榮退保大渡河,北去清溪關二百里,隔水相射,凡九日八夜。定邊軍節度使竇滂勒兵拒之。十二月,驃信遣清平官十餘人來偽和,與竇滂語次,蠻軍船筏競渡,忠武、武寧軍兵士結陣抗之,接戰自午及申,蠻軍稍卻。竇滂自縊於帳中,徐州將苗全緒解之,謂滂曰:「都統何至於是,但安心,全緒與再榮、弘節等血戰取勝。」
 全緒三人率兵而出,滂乃單騎宵遁。其夜,蠻軍營於山下。全緒等謀曰:「彼眾我寡,若明日對陣,吾屬敗矣。可夜擊之,令其軍亂,自解去。」忠武、武寧之師乃夜入蠻軍,弓弩亂發,蠻眾大駭,全緒等三將保軍而去。蠻軍乘勝進攻西川平,朝廷以顏慶復為大渡河制置、劍南應接等使,宋威為行營都知兵馬使,將兵數萬,與忠武、武寧之師合,與蠻軍戰於漢州之毗橋,大捷,解西川之圍。明日,蠻軍遁走,西川平。以蜀王佶為開府儀同三司、成都尹、
 劍南西川節度副大使、知節度事,不出閣;以盧耽知節度事。詔河東節度使鄭從赴闕。以義成軍節度使、光祿大夫、檢校尚書左僕射、同平章事、滑州刺史、上柱國、會稽縣開國伯、食邑二千戶康承訓以本官兼太原尹、北都留守,充河東軍節度使。以吏部侍郎楊知溫、吏部侍郎於德孫李玄考官;司封員外郎盧蕘、刑部侍郎楊戴考試宏詞選人;以虞部郎中宋震、前昭應主簿胡德融考科目舉人。詔以兵戈才罷,且務撫寧,其禮部貢舉,
 宜權停一年,付中書行敕指揮,其兩省官等,不用論奏。敕荊南節度使杜忭:「據司天奏,有小孛星氣經歷分野,恐有外夷兵水之患。緣邊籓鎮,最要提防,宜訓習師徒,增築城堡。凡關制置,具事以聞。」制以魏博節度使何全皞起復檢校司空、同平章事。



 十一年春正月甲寅朔,制尚書右僕射杜審權為檢校司徒、河中尹、絳慈隰節度觀察處置等使。丙午,制宰相、門下侍郎、吏部尚書曹確可兼尚書左僕射,門下侍郎、
 戶部尚書路巖可兼右僕射,中書侍郎於忭可兼戶部尚書,平章事劉瞻可中書侍郎、知政事。餘並如故。己酉,制:「河東節度使康承訓,將門瑣質,戎壘微才,曾不知兵,謬膺重祿。憂韜鈐以效任,畜奸惡以事君,幾授鉞於戎籓,嘗執金以徼道,謂其盡節,委以專征。屬者徐部匪寧,敢干紀律,俾護諸將,坐覆危巢。罄國幣以佐軍,頒王爵而賞士,而玩寇莫戰,按甲不前,立法未學於穰苴,申令頓虧於孫子。況部伍不戰,逼撓無謀,人數空多,軍威何
 振。使農夫釋耒,工女下機,始凝望於天誅,翻有思於賊至。洎元兇自潰,玄稔效忠,彭門洞開,爾功何有!而負恩已甚,瀆貨是求,叨榮茍幸於一時,遺患遂逾於積歲。爰行國典,俾傅戎籓,可蜀王傅,分司東都。」再貶恩州司馬同正,馳驛發遣。以檢校左散騎常侍、泗州刺史杜慆檢校工部尚書、滑州刺史、義成軍節度、鄭滑觀察等使。以河東行營沙陀三部落羌渾諸部招討使、檢校太子賓客、監察御史硃邪赤心為檢校工部尚書、單于大都護、
 御史大夫、振武節度、麟勝等州觀察等使,仍賜姓名曰李國昌。以吏部尚書蕭鄴、吏部侍郎於德孫、吏部侍郎楊知溫考官;司勛員外郎李耀、禮部員外郎崔澹等考試應宏詞選人。以河陽三城節度、孟懷澤觀察使、中散大夫、檢校禮部尚書、孟州刺史、御史大夫崔彥昭為金紫光祿大夫、檢校刑部尚書、太原尹、北都留守、河東節度觀察等使。以兵部侍郎、翰林學士承旨、扶風縣開國子、食邑五百戶、駙馬都尉韋保衡本官同平章事。以兵
 部侍郎劉鄴判度支。左僕射、門下侍郎、同平章事曹確以病求免,授檢校司空、同平章事,兼潤州刺史,充浙江西道觀察等使。魏博節度使何全皞酷政,為衙軍所殺,推其大將韓君雄為留後。四月癸未朔。戊子,敕:「去年屬以用軍之際,權停貢舉一年,今既去戈,卻宜仍舊。來年宜別許三十人及第,進士十人,明經二十人,已後不得援例。」八月辛巳朔。己酉,同昌公主薨,追贈衛國公主,謚曰文懿。主,郭淑妃所生,主以大中三年七月三日生,咸
 通九年二月二日下降。上尤鐘念,悲惜異常。以待詔韓宗紹等醫藥不效,殺之,收捕其親族三百餘人,系京兆府。宰相劉瞻、京兆尹溫璋上疏論諫行法太過,上怒,叱出之。



 九月丙辰,制以正議大夫、守中書侍郎、兼刑部尚書、同平章事、充集賢殿大學士、上柱國、彭城縣開國侯、食邑一千戶、賜紫金魚袋劉瞻檢校刑部尚書、同平章事,兼江陵尹,充荊南節度等使。翰林學士、戶部侍郎、知制誥、上柱國、賜紫金魚袋鄭畋為梧州刺史;正議大夫、
 御史中丞、上柱國、賜紫金魚袋孫瑝為汀州刺史;將仕郎、右諫議大夫、柱國、賜紫金魚袋高湘為高州刺史;中散大夫、比部郎中、知制誥、柱國、賜紫金魚袋楊知至為瓊州司馬;將仕郎、守禮部郎中魏簹為春州司馬;朝議大夫、行兵部員外郎、判度支案、柱國張顏為播州司戶;朝議大夫、行刑部員外郎、柱國崔顏融為雷州司戶;並坐劉瞻親善,為韋保衡所逐也。京兆尹溫璋貶振州司馬,制出之夜,璋仰藥而死。劉瞻再貶康州刺史。十月,以
 給事中薛能為京兆尹,以中書舍人高湜權知禮部貢舉。



 十一月己酉朔。辛亥,制以禮部尚書王鐸本官同平章事。丁卯,敕:「徐州地當沛野,軍本驍雄,實為壯國之都,固協建侯之制。況山河素異,土俗甚殷,豈欲削卑,挫其繁盛。蓋緣比因稔禍,或至亂常,罪由己招,孽非天作。桂林叛卒,繼有逆謀,塗炭生靈,首尾周歲。殺傷黎庶,污染忠良,所不忍言,尋加翦滅,是以卑其鎮額,隸彼籓方。近屬大兵已來,饑年薦至,且聞軍人百姓,深恥前非,願行
 舊規,卻希建節。朕每深軫念,思致小康,特示渥恩,復其軍額。宜賜宣徽庫綾絹十萬匹,助其宴犒,必獲周豐。其徐州都團練使改為感化軍節度、徐宿濠泗等州觀察處置等使。」以吏部侍郎鄭從讜檢校戶部尚書,兼汴州刺史、御史大夫,充宣武軍節度使,代李蔚;以蔚檢校吏部尚書、揚州大都督府長史,兼淮南節度副大使、知節度事。



 十二年春正月戊申,宰相路巖率文武百僚上徽號曰
 睿文英武明德至仁大聖廣孝皇帝,御含元殿。冊禮畢,大赦。辛酉,葬衛國公主於少陵原。先是,詔百僚為挽歌詞,仍令韋保衡自撰神道碑,京兆尹薛能為外監護,供奉楊復璟為內監護,威儀甚盛,上與郭淑妃御延興門哭送。幽州節度使張允伸病,請以子簡會為節度副大使、權知兵馬事,詔從之。



 三月,以吏部尚書蕭鄴、吏部侍郎歸仁晦李當考官;司封郎中鄭紹業、兵部員外郎陸勛等考試宏詞選人。四月,以左僕射、門下侍郎、同平章
 事路巖檢校司徒,兼成都尹、劍南西川節度等使。



 五月庚申,敕:「慎恤刑獄,大《易》格言。《語》曰:如得其情,即哀矜而勿喜。而獄吏苛刻,務在舞文,守臣因循,罕聞視事。以此械系之輩,溢於狴牢;追捕之徒,系於簡牘。實傷和氣,因致沴氛。況時屬熇蒸,化先茂育,並赦罪戾,式順生成。應天下所禁系罪人,除十惡忤逆、故意殺人、合造毒藥、持仗行劫、開發墳墓外,餘並宜疏理釋放。或信任人吏,多有生情系留,續察訪得知,本道觀察使判官、州府本曹
 官必加懲譴,以誡慢易。到後十日內,速疏理分析聞奏。」上幸安國寺,賜講經僧沉香高座。七月辛丑,中書門下奏:



 準今年六月十二日敕,厘革諸道及在京諸司奏官並請章服事者。其諸道奏州縣官司錄、縣令、錄事、參軍,或見任公事,敗闕不理,切要替換,及前任實有勞效,並見有闕員,即任各舉所知。每道奏請,仍不得過兩人。其河東、潞府、邠寧、涇原、靈武、鹽夏、振武、天德、鄜坊、滄德、易定、三川等道觀察防禦等使及嶺南五管,每道每年除
 令、錄外,許量奏簿、尉及中下州判司及縣丞共三人。福州不在奏縣官限。其黔中所奏州縣官及大將管內官,即任準舊例處分。在京諸司及諸道帶職奏官,或非時僉替,考限未滿,並卻與本資官。諸道節度及都團練防禦使下將校奏轉試官及憲御等,令諸節度事每年量許五人,都團練防禦量許三人為定,不得更於其外奏請。其御史中丞已下,即準敕文條疏,須有軍功,方可授任。自今後如顯立戰伐功勞者,任具事績申奏,如檢
 勘不虛,當別與商量處分,以外輒不得更有奏請。其幽、鎮、魏三道望且準承前舊例處分。敕旨從之。十二月,以檢校戶部尚書、汴州刺史、御史大夫、宣武軍節度使鄭從讜為廣州刺史、嶺南東道節度觀察處置等使。



 十三年春正月壬寅朔。甲戌,制以兵部侍郎、判度支劉鄴本官同平章事。幽州盧龍等軍節度使、檢校司徒、同平章事、幽州大都督府長史、上柱國、燕國公、食邑三千戶張允伸卒,贈太尉,謚曰忠烈。允伸鎮幽州二十三年。



 二月,幽州牙將張公素奪留後張簡會軍政,自稱留後。丁巳,制以尚書右僕射、門下侍郎、同平章事於琮檢校司空、襄州刺史,充山南東道節度觀察處置等使;以御史中丞趙隱為戶部侍郎、本官同平章事。



 三月,以吏部尚書蕭鄴、吏部侍郎獨孤雲考官,職方郎中趙蒙、駕部員外郎李超考試宏詞選人。試日,蕭鄴替,差右丞孔溫裕權判。



 五月庚午朔。辛未,敕檢校尚書左僕射、守左羽林軍統軍、御史大夫張直方貶康州司馬同正,以其部
 下為盜故也。乙亥,國子司業韋殷裕於閣門進狀,論淑妃弟郭敬述陰事。上怒甚,即日下京兆府決殺殷裕,籍沒其家。殷裕妻崔氏,音聲人鄭羽客、王燕客,婢微娘、紅子等九人配入掖庭。閣門使田獻銛奪紫,配於橋陵,閣門司閻敬直決十五,配南衙,為受殷裕文狀故也。給事中杜裔休貶端州司馬。中書舍人崔沆循州司戶,殷裕妻兄也;太僕少卿崔元應州司戶,殷裕妻父也;前河陰院官韋君卿為愛州崇平尉,殷裕季父也。以前大理正
 萬俟鎔為國子司業,前興元少尹馮彭為普州刺史,前大理正陽琯為昌州刺史。丙子,制開府儀同三司、檢校尚書左僕射、兼襄州刺史、御史大夫、充山南西道節度觀察等使於琮可正議大夫、守普王傅,分司東都。辛巳,敕尚書左丞李當貶道州刺史,吏部侍郎王珮貶漳州刺史,左散騎常侍李鬱貶賀州刺史,前中書舍人封彥卿貶潮州司戶,翰林學士承旨、兵部侍郎、知制誥張裼貶封州司馬,右諫議大夫楊塾貶和州司戶,工部尚書
 嚴祁貶郴州刺史,給事中李貺蘄州刺史,給事中張鐸藤州刺史,左金吾衛大將軍、充左街使李敬伸儋州司戶,前青州刺史、平盧軍節度使于涓為涼王府長史,分司東都;前湖南觀察使於瑰為袁州刺史。涓、瑰,琮之兄也。於藹、于蔇亦配流。自李當已下,皆於琮之親黨也,為韋保衡所逐。以天德防禦使、檢校左散騎常侍段文楚為雲州刺史、大同軍防禦使。



 六月,義成軍節度使、檢校工部尚書杜慆奏:當管潁州僧道百姓舉留刺史宗回,
 敕曰:「回清幹臨人,自有月限,方藉綏輯,未議替移。」六月,中書門下奏:



 今月十七日,延英面奉聖旨,令誡約天下州府,應有逃亡戶口,其賦稅差科,不得攤配見在人戶上者。伏以諸道州府,或兵戈之後,災沴之餘,戶口逃亡,田疇荒廢,天不敷佑,人多艱危。鄉閭屢困於征徭,帑藏因茲而耗竭,遂使從來經費色額,太半空系簿書。緩徵斂則闕於供須,促期限則迫於貧苦。言念凋弊,勞乃憂勤,不降明文,孰知聖念。其亡戶口賦稅及雜差科等,
 須有承佃戶人,方可依前應役。如將闕稅課額,攤於見在人戶,則轉成逋債,重困黎元。或富者有連阡之田,貧者無立錐之地,欲令均一,固在公平。若令狡猾之徒,得以升降由己,望其完葺,不亦難乎!全由長吏竭誠,方使疲甿漸泰。臣等商量,令諸道州府準此條疏,應有逃亡戶口稅賦並雜色差科等,並不得輒更攤配於見存人戶之上。務設法招攜,多方撫御,乘茲豐稔,重獲昭蘇。茍致安寧,自當遷陟,不遵詔令,必舉典刑。



 從之。七月,以前
 義昌軍節度使盧簡方為太僕卿。十二月,以振武節度李國昌為檢校右僕射、雲州刺史、大同軍防禦等使。國昌恃功頗橫,專殺長吏,朝廷不能平,乃移鎮雲中。國昌稱病辭軍務,乃以太僕卿盧簡方檢校刑部尚書、雲州刺史,充大同軍防禦等使。上召簡方於思政殿,謂之曰:「卿以滄州節鎮,屈轉大同。然朕以沙陀、羌、渾撓亂邊鄙,以卿曾在雲中,惠及部落,且忍屈為朕此行,具達朕旨,安慰國昌,勿令有所猜嫌也。」是月,李國昌小男克用殺
 雲中防禦使段文楚,據雲州,自稱防禦留後。制追謚宣宗為元聖至明成武獻文睿智章仁神聰懿道大孝皇帝。



 十四年春正月丙寅朔。御史中丞韋蟾奏:應諸州刺史除授,正衙辭謝後托故陳牒請假,實為容易。自今後如實有故為眾所知者,三日外不在陳牒之限。應內外除官入京,合便朝謝,如遇假日,且合在都亭驛。近日多因請假,便歸私家,既犯條章,頗乖禮敬。自今已後,望準故
 事,如未朝謝,須於都亭驛。如違越,臺司勘當申奏。」從之。辛未,以雲、朔暴亂,代北騷動,賜盧簡方詔曰:「李國昌久懷忠赤,顯著功勞,朝廷亦三授土疆,兩移旄節,其為寵遇,實寡比倫。昨者徵發兵師,又令克讓將領,惟嘉節義,同絕嫌疑。近知大同軍不安,殺害段文楚,推國昌小男克用主領兵權。事雖出於一時,心豈忘於長久?段文楚若實刻剝,自結怨嫌,但可申論,必行朝典。遽至傷殘性命,刳剔肌膚,慘毒憑凌,殊可驚駭。況忠烈之後,節義之
 門,致茲橫亡,尤悚觀聽。若克用暫勿主兵務,束手待朝廷除人,則事出權宜,不足猜慮。若便圖軍柄,欲奄有大同,則患系久長,故難依允。料國昌輸忠效節,必當已有指揮。知卿兩任雲中,恩及國昌爺子,敬憚懷感,不同常人。宜悚與書題,深陳禍福,殷勤曉喻,劈析指宜。切令大節無虧,勿使前功並棄。」簡方準詔諭之,國昌不奉詔。乃詔太原節度使崔彥昭、幽州節度使張公素帥師討之。



 三月,以新除大同軍使盧簡方為單于大都護、振武節
 度、麟勝等州觀察等使。時李國昌據振武。簡方至嵐州而卒。自是沙陀侵掠代北諸軍鎮。庚午,詔兩街僧於鳳翔法門寺迎佛骨,是日天雨黃土遍地。四月八日,佛骨至京,自開遠門達安福門,彩棚夾道,念佛之音震地。上登安福門迎禮之,迎入內道場三日,出於京城諸寺。士女雲合,威儀盛飾,古無其比。制曰:「朕以寡德,纘承鴻業,十有四年。頃屬寇猖狂,王師未息。朕憂勤在位,愛育生靈,遂乃尊崇釋教,至重玄門,迎請真身,為萬姓祈福。今
 觀睹之眾,隘塞路歧。載念狴牢,寢興在慮,嗟我黎人,陷於刑闢。況漸當暑毒,系於縲紲,或積幽凝滯,有傷和氣,或關連追擾,有妨農務。京畿及天下州府見禁囚徒,除十惡忤逆、故意殺人、官典犯贓、合造毒藥、放火持仗、開發墳墓外,餘罪輕重節級遞減一等。其京城軍鎮,限兩日內疏理訖聞奏;天下州府,敕到三日內疏理聞奏。」以吏部侍郎蕭仿為兵部侍郎、同平章事。



 六月,帝不豫。七月癸亥朔。戊寅,疾大漸。庚午,制立普王儼為皇太
 子,權勾當軍國政事。辛巳,遺詔曰:



 朕祗事九廟,君臨四海,夕惕如厲,宵分靡寧,必求政化之源,思建大中之道。至於懷柔夷貊,偃戢干戈,皆以德綏,亦自馴致,冀清凈之為理,庶治平之可臻。自秋已來,忽爾嬰疹,坐朝既闕,逾旬未瘳。六疾斯侵,萬機多曠,醫和無驗,以至彌留。嗚呼!數哉有窮,聖賢之所必同,明於斯言,是為達節。載申顧命,式葉典謨。皇太子權勾當軍國事儼,性稟寬和,生知忠孝,德苞睿哲,聖表徇齊,必能揚祖宗之重光,荷邦家之
 丕構。宜令所司具禮,於柩前即皇帝位。以司空、門下侍郎、平章事韋保衡攝塚宰。軍國務殷,豈可久曠,況易月之制,行之自古,皇帝宜三日而聽政,二十七日釋服。諸道節度、觀察、團練、防禦等使,及監軍、諸州刺史,受寄至重,並不得離任赴哀。文武常參官朝晡之臨,十五舉音。宮中當臨者,非時無得擅哭。天下人吏百姓告哀後出臨三日,皆釋服,勿禁食肉、飲酒、婚姻、祭祀,釋服之後無禁當舉。薄葬之禮,宜遵漢魏之文。其山陵制度,切在儉
 約,並不得以金銀錦繡文飾喪具。五坊鷹犬等,除搜狩外,餘並解放。其醫官段璲、趙、苻虔休、馬及等並釋放。咨爾將相卿士、中外臣僚,竭力盡忠,匡予令嗣,送往事居,無違朕志。



 是日,崩於咸寧殿,聖壽四十一。百僚上謚曰睿文昭聖恭惠孝皇帝,廟號懿宗。十五年二月,葬於簡陵。



 史臣曰:臣常接咸通耆老,言恭惠皇帝故事。當大中時,四海承平,百職修舉,中外無粃政,府庫有餘貲,年穀屢
 登,封疆無擾。恭惠始承丕構,頗亦勵精,延納讜言,尊崇耆德,數稔之內,洋洋頌聲。然器本中庸,流於近習,所親者巷伯,所暱者桑門。以蠱惑之侈言,亂驕淫之方寸,欲無怠忽,其可得乎!及釁結蠻陬,奸生戍卒。發五嶺之轉輸,寰海動搖;徵二蜀之捍防,蒸人蕩覆。徐寇雖殄,河南幾空。然猶削軍賦而飾伽藍,困民財而修凈業,以諛佞為愛己,謂忠諫為妖言。爭趨險陂之途,罕勵貞方之節。見豕負塗之愛豎,非次寵升;燋頭爛額之輔臣,無辜竄
 逐。是以干戈布野,蟲旱彌年,佛骨才入於應門,龍輴已泣於蒼野,報應無必,斯其驗歟!土德凌夷,禍階於此。雖有文、景之英繼,難以興焉。自茲龜玉之不昌,固其宜矣。黃發遺叟,言之涕零。



 贊曰:邦家治亂,在君聽斷。恭惠驕奢,賢良貶竄。兇豎當國,憸人滿朝。奸雄乘釁,貽謀道消。



\end{pinyinscope}