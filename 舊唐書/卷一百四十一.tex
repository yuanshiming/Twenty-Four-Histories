\article{卷一百四十一}

\begin{pinyinscope}

 ○徐浩趙涓子博宣盧南史附劉太真李紓邵說於邵崔元翰於公異呂渭子溫恭儉讓鄭雲逵李益李賀



 徐浩,字季海,越州人。父嶠,官至洛州刺史。浩少舉明經,
 工草隸,以文學為張說所器重,調授魯山主簿。說薦為麗正殿校理,三遷右拾遺,仍為校理。幽州節度使張守珪奏在幕府,改監察御史。丁父憂,服除,授京兆司錄,以母憂去職。數年,調授河南司錄,歷河陽令,以善政稱。拜太子司議郎,遷金部員外郎,歷憲部郎中。安祿山反,出為襄陽太守、本郡防禦使,賜以金紫之服。肅宗即位,召拜中書舍人,時天下事殷,詔令多出於浩。浩屬詞贍給,又工楷隸,肅宗悅其能,加兼尚書左丞。玄宗傳位誥
 冊,皆浩為之,參兩宮文翰,寵遇罕與為比。除國子祭酒,坐事貶廬州長史。代宗徵拜中書舍人、集賢殿學士,尋遷工部侍郎、嶺南節度觀察使、兼御史大夫,又為吏部侍郎、集賢殿學士。坐以妾弟冒選,托侍郎薛邕注授京尉,為御史大夫李棲筠所彈,坐貶明州別駕。



 德宗即位,徵拜彭王傅。建中三年,以疾卒,年八十,贈太子少師。初,浩以文雅稱;及授廣州,典選部,多積貨財,又嬖其妾侯莫陳氏,頗干政事,為時論所貶。



 趙涓,冀州人也。幼有文學。天寶初,舉進士,補郾城尉,累授監察御史、右司員外郎。河南副元帥王縉奏充判官,授檢校兵部郎中、兼侍御史,遷給事中、太常少卿,出為衢州刺史。



 永泰初,涓為監察御史。時禁中失火,燒屋室數十間,火發處與東宮稍近,代宗深疑之,涓為巡使,俾令即訊。涓周歷需囿,按據跡狀,乃上直中官遺火所致也,推鞫明審,頗盡事情。既奏,代宗稱賞焉。德宗時在東宮,常感涓之究理詳細,及刺衢州,年考既深,又與觀察
 使韓滉不相得,滉奏免涓官,德宗見其名,謂宰臣曰:「豈非永泰初御史趙涓乎?」對曰:「然。」即拜尚書左丞。無何,知吏部選,扈從梁州。興元元年卒,贈戶部尚書。



 子博宣,登進士第,文章俊拔,性率多酒。陳許節度使曲環闢為從事,賓筵之間多所忽略,環不能容。朝廷方討淮、蔡,環誣奏博宣受吳少誠賂為反間,又妄說國家休咎,扇惑軍情。時博宣權知舞陽縣事,詔令環決杖四十,流於康州,人皆以為枉。



 先是,侍御史盧南史坐事貶信州員外司
 馬,至郡,準例得吏一人,每月請紙筆錢,前後五年,計錢一千貫。南史以官閑冗,放吏歸,納其紙筆錢六十餘千。刺史姚驥劾奏南史,以為贓,又劾南史買鉛燒黃丹。德宗遣監察御史鄭楚相、刑部員外郎裴澥、大理評事陳正儀充三司使,同往按鞫。將行,並召於延英,謂之曰:「卿等必須詳審,無令漏罪銜冤。」三人將退,裴澥獨留,奏曰:「臣按姚驥奏狀,稱南史取吏紙筆錢計贓六十餘貫,雖於公法有違,量事且非巨蠹。」上曰:「此事亦未為甚,
 未知燒鉛何如?」澥曰:「燒鉛為丹,格令不禁。準天寶十三載敕,鉛、銅、錫不許私家買賣貨易,蓋防私鑄錢,本亦不言燒鉛為丹。南史違敕買鉛,不得無罪。伏以陛下自登寶位,及天寶、大歷以來,未曾降三司使至江南;今忽錄此小事,令三司使往,非唯損耗州縣,亦恐遠處聞之,各懷憂懼。臣聞開元中張九齡為五嶺按察使,有錄事參軍告齡非法,朝廷止令大理評事往按。大歷中,鄂岳觀察使吳仲孺與轉運使判官劉長卿紛競,仲孺奏長卿
 犯贓二十萬貫,時止差監察御史苗伾就推。今姚驥所奏事狀無多,臣堪任此行,即請獨往,恐不須三司並行為使。」德宗忻然曰:「卿言是矣。」乃復召楚相、正儀與澥俱坐,謂之曰:「朕懵於理道,處事未精,適見裴澥所奏,深協事宜,亦不用三人總去,但行首一人行可也,卿等使宣付宰臣改敕。」德宗不務大體,以察為明,皆此類也。而博宣、南史坐誣枉擯逐,賴裴澥悟主,南史不至深罪,後得召還。



 劉太真,宣州人。涉學,善屬文,少師事詞人蕭穎士。天寶末,舉進士。大歷中,為淮南節度使陳少游掌書記,徵拜起居郎。累歷臺閣,自中書舍人轉工部、刑部二侍郎。性怯懦詭隨。及轉禮部侍郎,掌貢舉,宰執姻族,方鎮子弟,先收擢之。又常敘少游勛績,擬之桓、文,大招物論。貞元五年,貶信州刺史,到州尋卒。



 太真尤長於詩句,每出一篇,人皆諷誦。德宗文思俊拔,每有禦制,即命朝臣畢和。貞元四年九月,賜宴曲江亭,帝為詩,序曰:



 朕在位僅將
 十載,實賴忠賢左右,克致小康。是以擇三令節,錫茲宴賞,俾大夫、卿士得同歡洽也。夫共其戚者同其休,有其初者貴其終,咨爾群僚,頒朕不暇,樂而能節,職思其憂,咸若時則,庶乎理矣。因重陽之會,聊示所懷。早衣對庭燎,躬化勤意誠。時此萬樞暇,適與佳節並。曲池絜寒流,芳菊舒金英。乾坤爽氣澄,臺殿秋光清。朝野慶年豐,高會多歡聲。永懷無荒誡,良士同斯情。



 因詔曰:「卿等重陽會宴,朕想歡洽,欣慰良多,情發於中,因制詩序。今賜卿
 等一本,可中書門下簡定文詞士三五十人應制,同用『清』字,明日內於延英門進來。」宰臣李泌等雖奉詔簡擇,難於取舍,由是百僚皆和。上自考其詩,以太真及李紓等四人為上等,鮑防、於邵等四人為次等,張濛、殷亮等二十三人為下等;而李晟、馬燧、李泌三宰相之詩,不加考第。



 初,硃泚、懷光之亂,關輔薦饑,貞元三年以後,仍歲豐稔,人始復生人之樂。德宗詔曰:「比者卿士內外,朝夕公務,今方隅無事,蒸民小康,其正月晦日、三月三日、九
 月九日三節日,宜任文武百僚擇勝地追賞。每節宰相、常參官共賜錢五百貫文、翰林學士一百貫文,左右神威、神策等十軍各賜五百貫。金吾英武、威遠及諸衛將軍共賜二百貫,客省奏事共賜一百貫,委度支每節前五日支付,永為常制。」



 李紓,字仲舒,禮部侍郎希言之子。少有文學。天寶末,拜秘書省校書郎。大歷初,吏部侍郎李季卿薦為左補闕,累遷司封員外郎、知制誥,改中書舍人。尋自虢州刺史
 徵拜禮部侍郎。德宗居奉天,擇為同州刺史,尋棄州詣梁州行在,拜兵部侍郎。反正,兼知選事。李懷光誅,河東節度及諸軍會河中,詔往宣勞節度,使還,敷奏合旨,拜禮部侍郎。



 紓通達,善詼諧,好接後進,厚自奉養,鮮華輿馬,以放達蘊藉稱。雖為大官,而佚游佐宴,不嘗自忘。嘗議享武成王不當視文宣廟,奏云:「準開元十九年敕,置齊太公廟,以張良配,太常卿及少卿、丞充三獻官。又按《開元禮》祝文云『皇帝遣某官昭告於齊太公、漢留侯』。至
 上元年,敕追贈太公為武成王,享祭之典,一同文宣王,有司因差太尉充獻官,兼御署祝板。伏以太公即周之太師,張良即漢之少傅,聖朝列於祀典,已極褒崇;今屈禮於至尊,施敬於臣佐,理或過當,神何敢歆。伏以文宣垂教,百代宗師,五常三綱,非其訓不明,有國有家,非其制不立,故孟軻稱『生人已來,一人而已』。由是正素王之位,加先聖之名,樂用宮懸,獻差太尉,尊師崇道,雅合政經。且太公述作止於《六韜》,勛業形於一代,豈宜擬諸盛
 德,均其殊禮!其祝文請不進署,『敢昭告』請改為『敬祭於』,『其昭告』請改為『致祭於留侯』,其獻官請準舊式,差太常卿已下充。」詔百僚進議。文武官上言,互有異同。詔曰:「帝德廣運,乃武乃文,文化武功,皇王之二柄,祀禮教敬,國章孔明。自今宜上將軍以下充獻官,餘依紓所奏。」紓又奏詔為《興元紀功述》及郊廟樂章,諸所論著甚眾。卒於官,年六十二。貞元八年,贈禮部尚書。



 邵說,相州安陽人。舉進士,為史思明判官,歷事思明、朝
 義,常掌兵事。朝義之敗,說降於軍前,郭子儀愛其才,留於幕下。累授長安令、秘書少監,遷吏部侍郎、太子詹事,以才幹稱。談者或以宰相許之,金吾將軍裴儆謂諫議大夫柳載曰:「以鄙夫所度,說得禍不久矣。且說與史思明父子定君臣之分,居劇官,掌兵柄,亡軀犯順,前後百戰,於賊庭掠名家子女以為婢僕者數十人,剽盜寶貨,不知紀極。力屈然後降,朝廷宥以不死。獲齒班序,無厚顏,而又遑遑求財,崇飾第宅,附托貴幸,以求大用,不知
 愧懼,而有得色,其能久乎!」建中三年,嚴郢得罪,說與郢厚善,勸硃泚抗疏申其冤,說為草其奏,上知之,貶說歸州刺史,竟卒於貶所。



 於邵,字相門,其先家於代,今為京兆萬年人。曾祖筠,戶部尚書。邵天寶末進士登科,書判超絕,授崇文館校書郎。累歷使府,入為起居郎,再遷比部郎中,尚二十考第於吏部,以當稱。無何,出為道州刺史,未就道,轉巴州。時歲儉,夷獠數千相聚山澤,圍州掠眾,邵勵州兵以拒之。
 旬有二日,遣使說喻,盜邀邵面降,邵儒服出城,盜羅拜而降,圍解,節度使李抱玉以聞,超遷梓州,以疾不至,遷兵部郎中。西川節度使崔寧請留為支度副使。尋拜諫議大夫、知制誥,再遷禮部侍郎、史館修撰,為三司使。以撰上尊號冊,賜階三品,當時大詔令,皆出於邵。頃之,與御史中丞袁高、給事中蔣鎮雜理左丞薛邕詔獄。邵以為邕犯在赦前,奏出之,失旨,貶桂州長史。貞元初,除原王傅,後為太子賓客,與宰相陸贄不睦。八年,出為杭州
 刺史,以疾請告,坐貶衢州別駕,移江州別駕,卒年八十一。



 邵性孝悌,內行修潔,老而彌篤。初,樊澤常舉賢良方正,邵一見之於京師,曰:「將相之材也。」不十五年,澤為節將。崔元翰年近五十,始舉進士,邵異其文,擢第甲科,且曰:「不十五年,當掌詔令。」竟如其言。獨孤授舉博學宏詞,吏部考為乙第,在中書覆升甲科,人稱其當。有集四十卷。



 崔元翰者,博陵人。進士擢第,登博學宏詞制科,又應賢
 良方正、直言極諫科,三舉皆升甲第,年已五十餘。李汧公鎮滑臺,闢為從事。後北平王馬燧在太原,聞其名,致禮命之,又為燧府掌書記。入朝為太常博士、禮部員外郎。竇參輔政,用為知制誥,詔令溫雅,合於典謨。然性太剛褊簡傲,不能取容於時,每發言論,略無阿徇,忤執政旨,故掌誥二年,而官不遷。竟罷知制誥,守比部郎中。元翰苦心文章,時年七十餘,好學不倦。既介獨耿直,故少交游,唯秉一操,伏膺翰墨。其對策及奏記、碑志、師法班
 固、蔡伯喈,而致思精密。為時所擯,終於散位。



 於公異者,吳人。登進士第,文章精拔,為時所稱。建中末,為李晟招討府掌書記。興元元年,收京城,公異為露布上行在云:「臣已肅清宮禁,祗奏寢園,鐘虡不移,廟貌如故。」德宗覽之,泣下不自勝,左右為之嗚咽。既而曰:「不知誰為之?」或對曰:「於公異之詞也。」上稱善久之。



 公異初應進士時,與舉人陸贄不協;至是贄為翰林學士,聞上稱與,尤不悅。時議者言之,公異少時不為後母所容,自游
 宦成名,不歸鄉里;及貞元中陸贄為宰相,奏公異無素行,黜之。詔曰:「祠部員外郎於公異,頃以才名,升於省闥。其少也,為父母之所不容,宜其引慝在躬,孝行不匱,匿名跡於畎畝,候安否於門閭,俾其親之過不彰,庶其誠之至必感。安於棄斥,游學遠方,忘其溫凊之戀,竟至存亡之隔,為人子者,忍至是乎!宜放歸田里,俾自循省。其舉公異官尚書左丞盧邁,宜奪俸兩月。」時中書舍人高郢薦監察御史元敦義,及睹公異譴逐,懼為所累,乃上
 疏首陳敦義虧於禮教,詔嘉郢之知過,俾敦義罷歸。公異竟名位不振,感軻而卒,人士惜其才,惡贄之褊急焉。



 呂渭,字君載,河中人。父延之,越州刺史、浙江東道節度使。渭舉進士,累授婺州永康令、大理評事。浙西觀察使李涵闢為支使,再遷殿中侍御史。涵自御史大夫改太子少傅,渭上言:「涵父名少康,今涵為少傅,恐乖朝典。」由是特授渭司門員外郎。尋為御史臺劾奏:「涵再任少卿,此時都不言;今為少傅,疑以散慢,乃為不可。」由是貶渭
 歙州司馬,改涵檢校工部尚書、兼光祿卿。



 渭累授舒州刺史、吏部員外、駕部郎中、知制詔、中書舍人,母憂罷。服闋,授太子右庶子、禮部侍郎。中書省有柳樹,建中末枯死,興元元年車駕還京後,其樹再榮,人謂之瑞柳。渭試進士,取瑞柳為賦題,上聞而嘉之。渭又結附裴延齡之子操,舉進士,文詞非工,渭擢之登第,為正人嗤鄙。因入閣遺失請托文記,遂出為潭州刺史、兼御史中丞、湖南都團練觀察使,在任三歲,政甚煩碎。貞元十六年卒,年
 六十六,贈陜州大都督。子溫、恭、儉、讓。



 溫,字化光,貞元末登進士第,與翰林學士韋執誼善。順宗在東宮,侍書王叔文勸太子招納時之英俊以自輔,溫與執誼尤為叔文所睠,起家再命拜左拾遺。二十年冬,副工部侍郎張薦為入吐蕃使,行至鳳翔,轉侍御史,賜緋袍牙笏。明年,德宗晏駕,順宗即位,張薦卒於青海,吐蕃以中國喪禍,留溫經年。時王叔文用事,故與溫同游東宮者,皆不次任用,溫在蕃中,悲嘆久之。元和元年,使還,轉戶部員外
 郎。時柳宗元等九人坐叔文貶逐。唯溫以奉使免。



 溫天才俊拔,文彩贍逸,為時流柳宗元、劉禹錫所稱。然性多險詐,好奇近利,與竇群、羊士諤趣尚相狎。群為韋夏卿所薦,自處士不數年至御史中丞,李吉甫尤奇待之。三年,吉甫為中官所惡,將出鎮揚州,溫欲乘其有間傾之。溫自司封員外郎轉刑部郎中,竇群請為知雜。吉甫以疾在第,召醫人陳登診視,夜宿於安邑里第。溫伺知之,詰旦,令吏捕登鞫問之,又奏劾吉甫交通術士。憲宗異
 之,召登面訊,其事皆虛,乃貶群為湖南觀察使,羊士諤資州刺史,溫均州刺史。朝議以所責太輕,群再貶黔南,溫貶道州刺史。五年,轉衡州,秩滿歸京,不得意,發疾卒。溫文體富艷,有丘明、班固之風,所著《凌煙閣功臣銘》、《張始興畫贊》、《移博士書》,頗為文士所賞,有文集十卷。



 恭、儉皆至侍御史,讓至太子右庶子,皆有美才。自後吉甫再入中書,長慶以後,李德裕黨盛,呂氏諸子無至達官者。



 鄭雲逵,滎陽人。大歷初,舉進士。性果誕敢言。客游兩河,
 以畫干於硃泚,泚悅,乃表為節度掌書記、檢校祠部員外郎,仍以弟滔女妻之。泚將入覲,先令雲逵入奏;及泚至京,以事怒雲逵,奏貶莫州參軍。滔代泚後,請為判官。滔助田悅為逆,雲逵渝之不從,遂棄妻子馳歸長安,帝嘉其來,留於客省,超拜諫議大夫。奉天之難,雲逵奔赴行在,李晟以為行軍司馬,戎略多以咨之。歷秘書少監、給事中,尋拜大理卿,遷刑部、兵部二侍郎、遷御史中丞,充順宗山陵橋道置頓使。



 雲逵初為硃泚判官,常忤同
 幕蔡庭玉;庭玉白泚,黜為莫州錄事參軍。滔復奏為判官,因深構庭玉於滔;滔為泚留後事,有請於泚,庭玉又輒隳之。又有判官硃體微,亦蒙泚親信,與庭玉常從容言於泚曰:「滔非長者,不可付以兵權。」滔竊知之。後滔南討有功,雲逵數激怒之,滔乃抗表論庭玉等離間骨肉;及滔叛,帝乃召泚以表示之,故歸罪於庭玉等以悅滔,滔亦終叛。三年,雲逵奏:其弟前太僕丞方逵,「受性兇悖,不知君親,眾惡備身,訓教莫及,結聚兇黨,江中劫人。臣
 亡父先臣昈杖至一百,終不能斃。張延賞任揚州日,亦曾犯延賞法,決殺復蘇。至於常言,皆呼臣亡父先臣名,親戚所知,無可教語。昨聞於邠、寧、慶等州幹謁節度及州縣乞丐,今見在武功縣南,西戎俯近,恐有異謀;若不冒死奏聞,必恐覆臣家族。」詔令京兆府錮身遞送黔州,付李模於僻遠州驅使,勿許東西。



 雲逵元和元年拜右金吾衛大將軍,歲中改京兆尹。五年五月卒。



 李益,肅宗朝宰相揆之族子。登進士第,長為歌詩。貞元
 末,與宗人李賀齊名。每作一篇,為教坊樂人以賂求取。唱為供奉歌詞。其《征人歌》、《早行篇》,好事者畫為屏障;「回樂峰前沙似雪,受降城外月如霜」之句,天下以為歌詞。然少有癡病,而多猜忌,防閑妻妾,過為苛酷,而有散灰扃戶之譚聞於時,故時謂妒癡為「李益疾」;以是久之不調,而流輩皆居顯位。益不得意,北游河朔,幽州劉濟闢為從事,常與濟詩而有「不上望京樓」之句。



 憲宗雅聞其名,自河北召還,用為秘書少監、集賢殿學士。自負才地,
 多所凌忽,為眾不容,諫官舉其幽州詩句,降居散秩。俄復用為秘書監,遷太子賓客、集賢學士判院事,轉右散騎常侍。太和初,以禮部尚書致仕,卒。



 李賀,字長吉,宗室鄭王之後。父名晉肅,以是不應進士,韓愈為之作《諱辨》,賀竟不就試。手筆敏捷,尤長於歌篇。其文思體勢,如崇巖峭壁,萬仞崛起,當時文士從而效之,無能仿佛者。其樂府詞數十篇,至於雲韶樂工,無不諷誦。補太常寺協律郎,卒,時年二十四。



 史臣曰:文學之士,代不乏才。永泰、貞元之間,如徐浩、趙涓諸公,可謂一時之秀也。然太真以畏懦聞,邵說以僭侈失,於公異、呂渭、李益皆有微累,故知全其德者罕矣。



 贊曰:名以才顯,才兼德尊。徐、趙、劉、李,厥聲遠聞。邵、于、呂、鄭,其名久存。半乏全德,愧於後人。



\end{pinyinscope}