\article{卷一百一十九}

\begin{pinyinscope}

 ○崔器趙國珍崔瓘敬
 括韋元甫魏少游衛伯玉李承



 崔器,深州安平人也。曾祖恭禮,狀貌豐碩,飲酒過斗。貞觀中,拜駙
 馬都尉,尚神堯館陶
 公主。父肅然,平陰丞。



 器
 有吏才,性介而少通,舉明經,歷官清謹。天寶六載,為萬年尉,逾月拜監察御史。中丞宋渾為東畿採訪使,引器為判官;渾坐贓流貶嶺南,器亦隨貶。十三年,量移京兆府司錄,轉都官員外郎,出為奉先令。逆胡陷西京。器沒於賊,仍守奉先。居無何,屬賊黨同羅叛賊,長安守將安守忠、張通儒並亡匿。又渭上義兵起,一朝聚徒數萬,器懼,所受賊文牒符敕,一時焚之,榜召義師,欲應渭上軍。及渭上軍破,賊將崔乾祐先鎮蒲、同,使麾下騎三十人
 捉器,器遂北走靈武。



 器素與呂諲善,諲引為御史中丞、兼戶部侍郎。從肅宗至鳳翔,加禮儀使。克復二京,為三司使。器草儀注,駕入城,令陷賊官立於含元殿前,露頭跣足,撫膺頓首請罪,以刀杖環衛,令扈從群官宰臣已下視之。及收東京,令陳希烈已下數百人如西京之儀。器性陰刻樂禍,殘忍寡恩,希旨奏陷賊官準律並合處死。肅宗將從其議。三司使、梁國公李峴執奏,固言不可,乃六等定罪,多所原宥,唯陳希烈、達奚珣斬於獨柳樹
 下。後蕭華自相州賊中仕賊官歸闕,奏云:「賊中仕官等重為安慶緒所驅,脅至相州,初聞廣平王奉宣恩命,釋放陳希烈已下,皆相顧曰:『我等國家見待如此,悔恨何及。』及聞崔器議刑太重,眾心復搖。」肅宗曰:「朕幾為崔器所誤。」



 呂諲驟薦器為吏部侍郎、御史大夫。上元元年七月,器病腳腫,月餘疾亟,瞑目則見達奚珣,叩頭曰:「大尹不自由。」左右問之,器答曰:「達奚大尹嘗訴冤於我,我不之許。」如是三日而器卒。



 趙國珍,牂柯之苗裔也。天寶中,以軍功累遷黔府都督,兼本管經略等使。時南蠻閣羅鳳叛,宰臣楊國忠兼劍南節度,遙制其務,屢喪師徒。中書舍人張漸薦國珍有武略,習知南方地形,國忠遂奏用之。在五溪凡十餘年,中原興師,唯黔中封境無虞。代宗踐祚,特嘉之,召拜工部尚書。大歷三年九月,以疾終,贈太子太傅。



 崔瓘,博陵人也。以士行聞,蒞職清謹。累遷至澧州刺史,下車削去煩苛,以安人為務。居二年,風化大行,流亡襁
 負而至,增戶數萬。有司以聞,優詔特加五階,至銀青光祿大夫,以甄能政。遷潭州刺史、兼御史中丞,充湖南都團練觀察處置使。瓘到官,政在簡肅,恭守禮法。將吏自經時艱,久不奉法,多不便之。大歷五年四月,會月給糧儲,兵馬使臧玠與判官達奚覯忿爭,覯曰:「今幸無事。」玠曰:「有事何逃?」厲色而去。是夜,玠遂構亂,犯州城,以殺達奚覯為名。瓘惶遽走,逢玠兵至,遂遇害。代宗聞其事,悼惜久之。



 敬括,河東人也。少以文詞稱。鄉舉進士,又應制登科,再遷右拾遺、內供奉、殿中侍御史。天寶末,宰臣楊國忠出不附己者,括以例為果州刺史。累遷給事中、兵部侍郎、大理卿。性深厚。志尚簡淡,在職不務求名,因循而已。大歷初,叛臣周智光伏誅,詔選循良為近輔,以括為同州刺史。歲餘,入為御史大夫。遲重推誠於下,未嘗以私害公,士頗稱焉;而從容養望,不舉綱紀,士亦以此少之。大歷六年三月卒。



 韋元甫,少修謹,敏於學行。初任滑州白馬尉,以吏術知名。本道採訪使韋涉深器之,奏充支使,與同幕判官員錫齊名。元甫精於簡牘,錫詳於訊覆,涉推誠待之,時謂「員推韋狀。」元甫有器局,所蒞有聲,累遷蘇州刺史、浙江西道都團練觀察等使。大歷初,宰臣杜鴻漸首薦之,徵為尚書右丞。會淮南節度使缺,鴻漸又薦堪當重寄,遂授揚州長史、兼御史大夫、淮南節度觀察等使。在揚州三年,政尚不擾,事亦粗理。大歷六年八月,以疾卒於位。



 魏少游,鉅鹿人也。早以吏乾知名,歷職至朔方水陸轉運副使。肅宗幸靈武,杜鴻漸等奉迎,留少游知留後,備宮室掃除之事。少游以肅宗遠離宮闕,初至邊籓,故豐供具以悅之。將至靈武,少游整騎卒千餘,干戈耀日,於靈武南界鳴沙縣奉迎,備威儀振旅而入。肅宗至靈武,殿宇御幄,皆象宮闈,諸王、公主各設本院,飲食進御,窮其水陸。肅宗曰:「我至此本欲成大事,安用此為!」命有司稍去之。累遷衛尉卿。乾元二年十月,議率朝臣馬以助
 軍,少游與漢中郡王瑀沮其議,上知之,貶渠州長史。後為京兆尹,請中書門下及兩省五品已上、尚書省四品已上、諸司正員三品已上、諸王、駙馬中期周已上親及女婿外甥,不得任京兆府判官、畿令、赤縣丞簿尉,敕從之。遷刑部侍郎。



 大歷二年四月,出為洪州刺史、兼御史大夫,充江南西道都團練觀察等使。四年六月,封趙國公。賈明觀者,本萬年縣捕賊小胥,事劉希暹,恃魚朝恩之勢,恣行兇忍,毒甚豺虺。朝恩、希暹既誅,元載當權,納明
 觀奸謀,容之,特令江西效力。明觀未出城,百姓萬眾聚於城外,皆懷磚石候之,期投擊以快意。載聞之,特令所由吏擁百姓入城內,由是獲免。在洪州二年,少游為觀察使,承元載意茍容之。及路嗣恭代少游,到州,即日杖殺,識者以是減魏之名,多路之政。大歷六年三月己未卒於官,贈太師。



 少游居職,緣飾成務,有規檢,善任人,果於集事。前後四領京尹,雖無鶴赫之名,而齪齪廉謹,有足稱者。



 衛伯玉,有膂力,幼習藝。天寶中杖劍之安西,以邊功累遷至員外諸衛將軍。肅宗即位,興師靖難,伯玉激憤,思立功名,自安西歸長安。初為神策軍兵馬使出鎮。乾元二年十月,逆賊史思明遣偽將李歸仁鐵騎三千來犯,伯玉以數百騎於疆子阪擊破之,積尸滿野,虜馬六百匹,歸仁與其黨東走。以功遷右羽林軍大將軍,知軍事。轉四鎮、北庭行營節度使。獻俘百餘人至闕下,詔解縛而赦之,遷伯玉神策軍節度。上元二年二月,史思明領
 眾西下圖長安,史朝義率其黨夜襲陜州。伯玉以兵逆擊,大破賊於永寧。賊退,進位特進,封河東郡公。



 廣德元年冬,吐蕃寇京師,乘輿幸陜。以伯玉有幹略,可當重寄,乃拜江陵尹、兼御史大夫,充荊南節度觀察等使。尋加檢校工部尚書,封城陽郡王。大歷初,丁母憂,朝廷以王昂代其任,伯玉潛諷將吏不受詔,遂起復以本官為荊南節度等使,時議丑之。大歷十一年二月入覲,以疾卒於京師。



 李承,趙郡高邑人,吏部侍郎至遠之孫,國子司業畬之第二子也。承幼孤,曄鞠養之。既長,事兄以孝聞。舉明經高第,累至大理評事,充河南採訪使郭納判官。尹子奇圍汴州,陷賊,拘承送洛陽。承在賊庭,密疏奸謀,多獲聞達。兩京克復,例貶撫州臨川尉。數月除德清令,旬日拜監察御史。淮南節度使崔圓請留充判官,累遷檢校刑部員外郎、兼侍御史。圓卒,歷撫州、江州二刺史,課績連最。遷檢校考功郎中兼江州刺史,徵拜吏部郎中。尋
 為淮南西道黜陟使,奏於楚州置常豐堰以御海潮,屯田瘠鹵,歲收十倍,至今受其利。時梁崇義縱恣倨慢,朝廷將加討伐。李希烈揣知之,上表數崇義過惡,請率先誅討。上悅之,每對朝臣多稱希烈忠誠。承自黜陟回,因奏之曰:「希烈將兵討伐,必有微勛,但恐立功之後,縱恣跋扈,不稟朝憲,必勞王師問罪。」上初未之信。無幾,希烈既平崇義,果有不順之跡,上思承言,故驟加擢用。建中二年七月,拜同州刺史、河中尹、晉絳都防禦觀察使。九
 月,轉襄州刺史、山南東道節度觀察鹽鐵等使。希烈既破崇義,擁兵襄州,遂有其地。朝廷慮不受命,欲以禁兵送承,承請單騎徑行。既至,希烈處承於外館,迫脅萬態,承恬然自安,誓死王事。希烈不能屈,遂剽虜闔境所有而去,襄、漢為之空。承治之一年,頗得完復。



 初,希烈雖歸蔡州,留將校等於襄州守當時所掠得財帛什物等,後使襄、漢,往來不絕。承亦使腹心臧叔雅往來許、蔡,厚結希烈腹心周曾、王玢、姚憺等。及曾等謀殺希烈,以眾歸
 朝,多承首建謀也。累賜密詔褒美之。承尋改檢校工部尚書,兼潭州刺史、湖南都團練觀察使。建中四年七月,卒於位,年六十二,贈吏部尚書。承少有雅望,至其從官,頗以貞廉才術見稱於時。



 史臣曰:自古酷吏濫刑,幸免者多矣,茍無強魂為祟,沮議者惑焉。器深文樂禍,居官令終,非達奚訴冤,無以顯其陰責矣。國珍守黔溪,瓘修禮法,括推誠馭下,元甫為政寬簡,少游規檢集事,皆可稱者。伯玉破敵立功,足為
 猛士,丁憂冒寵,終是武夫。承忠愨謀議,勤勞盡瘁,方之者鮮矣。



 贊曰:崔器深文,達奚作祟。七子伊何?李承為最。



\end{pinyinscope}