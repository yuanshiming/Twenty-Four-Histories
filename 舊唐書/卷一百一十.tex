\article{卷一百一十}

\begin{pinyinscope}

 ○李林甫楊國忠張暐王
 琚王毛仲陳玄禮附



 李林甫,高祖從父弟長平王叔良之曾孫。叔良生孝斌,官至原州長史。孝斌生思誨,官至揚府參軍,思誨即林
 甫之父也。林甫善音律,初為千牛直長,其舅楚國公姜皎深愛之。開元初,遷太子中允。時源乾曜為侍中,乾曜侄孫光乘,姜皎妹婿,乾曜與之親。乾曜之男潔白其父曰:「李林甫求為司門郎中。」乾曜曰:「郎官須有素行才望高者,哥奴豈是郎官耶?」數日,除諭德。哥奴,林甫小字。累遷國子司業。



 十四年,宇文融為御史中丞,引之同列,因拜御史中丞,歷刑、吏二侍郎。時武惠妃愛傾後宮,二子壽王、盛王以母愛特見寵異,太子瑛益疏薄。林甫多與
 中貴人善,乃因中官白惠妃云:「願保護壽王。」惠妃德之。初,侍中裴光庭妻武三思女,詭譎有材略,與林甫私。中官高力士本出三思家,及光庭卒,武氏銜哀祈於力士,請林甫代其夫位,力士未敢言,玄宗使中書令蕭嵩擇相,嵩久之以右丞韓休對,玄宗然之,乃令草詔。力士遽漏於武氏,乃令林甫白休。休既入相,甚德林甫,與嵩不和,乃薦林甫堪為宰相,惠妃陰助之,因拜黃門侍郎,玄宗眷遇益深。



 二十三年,以黃門侍郎平章事裴耀卿為
 侍中,中書侍郎平章事張九齡為中書令,林甫為禮部尚書、同中書門下三品,並加銀青光祿大夫。林甫面柔而有狡計,能伺侯人主意,故驟歷清列,為時委任。而中官妃家,皆厚結托,伺上動靜,皆預知之,故出言進奏,動必稱旨。而猜忌陰中人,不見於詞色,朝廷受主恩顧,不由其門,則構成其罪;與之善者,雖廝養下士,盡至榮寵。尋歷戶、兵二尚書,知政事如故。



 尋又以太子瑛、鄂王瑤、光王琚皆以母失愛而有怨言,駙馬都尉楊洄白惠妃。
 玄宗怒,謀於宰臣,將罪之。九齡曰:「陛下三個成人兒不可得。太子國本,長在宮中,受陛下義方,人未見過,陛下奈何以喜怒間忍欲廢之?臣不敢奉詔。」玄宗不悅。林甫惘然而退,初無言,既而謂中貴人曰:「家事何須謀及於人。」時朔方節度使牛仙客在鎮,有政能,玄宗加實封,九齡又奏曰:「邊將馴兵秣馬,儲蓄軍實,常務耳,陛下賞之可也;欲賜實賦,恐未得宜。惟聖慮思之。」帝默然。林甫以其言告仙客,仙客翌日見上,
 泣讓官爵。玄宗欲行實封之命,兼為尚書,九齡執奏如初。帝變色曰:「事總由卿?」九齡頓首曰:「陛下使臣待罪宰相,事有未允,臣合盡言。違忤聖情,合當萬死。」玄宗曰:「卿以仙客無門籍耶?卿有何門閥?」九齡對曰:「臣荒徼微賤,仙客中華之士。然陛下擢臣踐臺閣,掌綸誥;仙客本河湟一使典,目不識文字,若大任之,臣恐非宜。」林甫退而言曰:「但有材識,何必辭學;天子用人,何有不可?」玄宗滋不悅。



 九齡與中書侍郎嚴挺之善。挺之初娶妻出之,妻乃嫁蔚州刺史王
 元琰。時元琰坐贓,詔三司使推之,挺之救免其罪。玄宗察之,謂九齡曰:「王元琰不無贓罪,嚴挺之囑托所由輩有顏面。」九齡曰:「此挺之前妻,今已婚崔氏,不合有情。」玄宗曰:「卿不知,雖離之。亦卻有私。」玄宗籍前事,以九齡有黨,與裴耀卿俱罷知政事,拜左、右丞相,出挺之為洺州刺史,元琰流於嶺外。即日林甫代九齡為中書、集賢殿大學士、修國史;拜牛仙客工部尚書、同中書門下平章事,知門下省事。監察御史周子諒言仙客非宰相器,玄宗怒而
 殺之。林甫言子諒本九齡引用,乃貶九齡為荊州長史。



 玄宗終用林甫之言,廢太子瑛、鄂王瑤、光王琚為庶人,太子妃兄駙馬都尉薛銹長流瀼州,死於故驛,人謂之「三庶」,聞者冤之。其月,佞媚者言有烏鵲巢於大理獄戶,天下幾致刑措。玄宗推功元輔,封林甫晉國公,仙客豳國公。其冬,惠妃病,三庶人為崇而薨。儲宮虛位,玄宗未定所立。林甫曰:「壽王年已成長,儲位攸宜。」玄宗曰:「忠王仁孝,年又居長,當守器東宮。」乃立為皇太子。自是林甫
 懼,巧求陰事以傾太子。



 林甫既秉樞衡,兼領隴右、河西節度,又加吏部尚書。天寶改易官名,為右相,停知節度事,加光祿大夫,遷尚書左僕射。六載,加開府儀同三司,賜實封三百戶,而恩渥彌深。凡御府膳羞,遠方珍味,中人宣賜,道路相望。與宰相李適之雖同宗屬,而適之輕率,嘗與林甫同論時政,多失大體,由是主恩益疏,以至罷免。黃門侍郎陳希烈性便佞,嘗曲事林甫,適之既罷,乃引希烈同知政事。林甫久典樞衡,天下威權,並歸於
 己,臺司機務,希烈不敢參議,但唯諾而已。每有奏請,必先賂遺左右,伺察上旨,以固恩寵。上在位多載,倦於萬機,恆以大臣接對拘檢,難徇私欲,自得林甫,一以委成。故杜絕逆耳之言,恣行宴樂,衣任席無別,不以為恥,由林甫之贊成也。



 林甫京城邸第,田園水磑,利盡上腴。城東有薛王別墅,林亭幽邃,甲於都邑,特以賜之,及女樂二部,天下珍玩,前後賜與,不可勝紀。宰相用事之盛,開元已來,未有其比。然每事過慎,條理眾務,增修綱紀,中外
 遷除,皆有恆度。而耽寵固權,己自封植,朝望稍著,必陰計中傷之。初,韋堅登朝,以堅皇太子妃兄,引居要職,示結恩信,實圖傾之,乃潛令御史中丞楊慎矜陰伺堅隙。會正月望夜,皇太子出游,與堅相見,慎矜知之,奏上。上大怒,以為不軌,黜堅,免太子妃韋氏。林甫因是奏李適之與堅暱狎,及裴寬、韓朝宗並曲附適之,上以為然,賜堅自盡,裴、韓皆坐之斥逐。後楊慎矜權位漸盛,林甫又忌之,乃引王鉷為御史中丞,托以心腹。鉷希林甫意,遂
 誣罔密奏慎矜左道不法,遂族其家。楊國忠以椒房之親,出入中禁,奏請多允,乃擢在臺省,令按刑獄。會皇太子良娣杜氏父有鄰與子婿柳勣不葉,勣飛書告有鄰不法,引李邕為證,詔王鉷與國忠按問。鉷與國忠附會林甫奏之,於是賜有鄰自盡,出良娣為庶人,李邕、裴敦復枝黨數人並坐極法。林甫之苞藏安忍,皆此類也。



 林甫自以始謀不佐皇太子,慮為後患,故屢起大獄以危之,賴太子重慎無過,流言不入。林甫嘗令濟陽別駕魏
 林告隴右、河西節度使王忠嗣,林往任朔州刺史,忠嗣時為山東節度,自云與忠王同養宮中,情意相得,欲擁兵以佐太子。玄宗聞之曰:「我兒在內,何路與外人交通?此妄也。」然忠嗣亦左授漢陽太守。八載,咸寧太府趙奉章告林甫罪狀二十餘條。告未上,林甫知之,諷御史臺逮捕,以為妖言,重杖決殺。



 十載,林甫兼領安西大都護、朔方節度,俄兼單于副大都護。十一載,以朔方副使李獻忠叛,讓節度,舉安思順自代。國家武德、貞觀已來,蕃
 將如阿史那杜爾、契苾何力,忠孝有才略,亦不專委大將之任,多以重臣領使以制之。開元中,張嘉貞、王晙、張說、蕭嵩、杜暹皆以節度使入知政事,林甫固位,志欲杜出將入相之源,嘗奏曰:「文士為將,怯當矢石,不如用寒族、蕃人,蕃人善戰有勇,寒族即無黨援。」帝以為然,乃用思順代林甫領使。自是高仙芝、哥舒翰皆專任大將,林甫利其不識文字,無入相由,然而祿山竟為亂階,由專得大將之任故也。



 林甫恃其早達,輿馬被服,頗極鮮華。
 自無學術,僅能秉筆,有才名於時者尤忌之。而郭慎微、苑咸文士之闒茸者,代為題尺。林甫典選部時,選人嚴迥判語有用「杕杜」二字者,林甫不識「杕」字,謂吏部侍郎韋陟曰:「此云『杖杜』,何也?」陟俯首不敢言。太常少卿姜度,林甫舅子,度妻誕子,林甫手書慶之曰:「聞有弄麞之慶。」客視之掩口。



 初,楊國忠登朝,林甫以微才不之忌;及位至中司,權傾朝列,林甫始惡之。時國忠兼領劍南節度,會南蠻寇邊,林甫請國忠赴鎮。帝雖依奏,然待國忠方
 渥,有詩送行,句末言入相之意。又曰:「卿止到蜀郡處置軍事,屈指待卿。」林甫心尤不悅。林甫時已寢疾。其年十月,扶疾從幸華清宮,數日增劇,巫言一見聖從差減,帝欲視之,左右諫止。乃敕林甫出於庭中,上登降聖閣遙視,舉紅巾招慰之,林甫不能興,使人代拜於席。翌日,國忠自蜀還,謁林甫,拜於床下,林甫垂涕托以後事。尋卒,贈太尉、揚州大都督,給班劍、西園秘器。諸子以吉儀護柩還京師,發喪於平康坊之第。



 林甫晚年溺於聲妓,姬
 侍盈房。自以結怨於人,常憂刺客竊發,重扃復壁,絡板甃石,一夕屢徙,雖家人不之知。有子二十五人、女二十五人:岫為將作監,崿為司儲郎中,嶼為太常少卿;子婿張博濟為鴻臚少卿,鄭平為戶部員外郎,杜位為右補闕,齊宣為諫議大夫,元捴為京兆府戶曹。



 初,林甫嘗夢一白晰多須長丈夫逼己,接之不能去。既寤,言曰:「此形狀類裴寬,寬謀代我故也。」時寬為戶部尚書、兼御史大夫,故因李適之黨斥逐之。是時楊國忠始為金吾胄曹參軍,
 至是不十年,林甫卒,國忠竟代其任,其形狀亦類寬焉。國忠素憾林甫,既得志,誣奏林甫與蕃將阿布思同構逆謀,誘林甫親族間素不悅者為之證。詔奪林甫官爵,廢為庶人,岫、崿諸子並謫於嶺表。林甫性沉密,城府深阻,未嘗以愛憎見於容色。自處臺衡,動循格令,衣寇士子,非常調無仕進之門。所以秉鈞二十年,朝野側目,憚其威權。及國忠誣構,天下以為冤。



 楊國忠,本名釗,蒲州永樂人也。父珣,以國忠貴,贈兵
 部尚書。則天朝幸臣張易之,即國忠之舅也。國忠無學術拘檢,能飲酒,蒱博無行,為宗黨所鄙。乃發憤從軍,事蜀帥,以屯優當遷,益州長史張寬惡其為人,因事笞之,竟以屯優授新都尉。稍遷金吾衛兵曹參軍。太真妃,即國忠從祖妹也。天寶初,太真有寵,劍南節度使章仇兼瓊引國忠為賓佐,既而擢授監察御史。去就輕率,驟履清貴,朝士指目嗤之。



 時李林甫將不利於皇太子,掎摭陰事以傾之。侍御史楊慎矜承望風旨,誣太子妃兄韋堅與
 皇甫惟明私謁太子,以國忠怙寵敢言,援之為黨,以按其事。京兆府法曹吉溫舞文巧詆,為國忠爪牙之用,因深竟堅獄,堅及太子良娣杜氏、親屬柳勣、杜昆吾等,痛繩其罪,以樹威權。於京城別置推院,自是連歲大獄,追捕擠陷,誅夷者數百家,皆國忠發之。林甫方深阻保位,國忠凡所奏劾,涉疑似於太子者,林甫雖不明言以指導之,皆林甫所使,國忠乘而為邪,得以肆意。上春秋高,意有所愛惡,國忠探知其情,動契所欲。驟遷檢校度支員外
 郎,兼侍御史,監水陸運及司農、出納錢物、內中市買、召募劍南健兒等使。以稱職遷度支郎中,不期年,兼領十五餘使,轉給事中、兼御史中丞,專判度支事。是歲,貴妃姊虢國、韓國、秦國三夫人同日拜命,兄銛拜鴻臚卿。八載,玄宗召公卿百僚觀左藏庫,喜其貨幣山積,面賜國忠金紫,兼權太府卿事。國忠既專錢穀之任,出入禁中,日加親幸。



 初,楊慎矜希林甫旨,引王鉷為御史中丞,同構大獄,以傾東宮。既帝意不回,慎矜稍避事防患,
 因與鉷有隙。鉷乃附國忠,奏誣慎矜,誅其昆仲,由是權傾內外,公卿惕息。吉溫為國忠陳移奪執政之策,國忠用其謀,尋兼兵部侍郎。京兆尹蕭炅、御史中丞宋渾皆林甫所親善,國忠皆誣奏譴逐,林甫不能救。王鉷為御史大夫,兼京兆尹,恩寵侔於國忠,而位望居其右。國忠忌其與己分權,會邢縡事洩,乃陷鉷兄弟誅之,因代鉷為御史大夫,權京兆尹,賜名國忠。乃窮竟邢縡獄,令引林甫交私鉷、銲與阿布思事狀,而陳希烈、哥舒翰附會
 國忠,證成其狀,上由是疏薄林甫。



 南蠻質子閤羅鳳亡歸不獲,帝怒甚,欲討之。國忠薦閬州人鮮於仲通為益州長史,令率精兵八萬討南蠻,與羅鳳戰於瀘南,全軍陷沒。國忠掩其敗狀,仍敘其戰功,仍令仲通上表請國忠兼領益部。十載,國忠權知蜀郡都督府長史,充劍南節度副大使,知節度事,仍薦仲通代己為京兆尹。國忠又使司馬李宓率師七萬再討南蠻。宓渡瀘水,為蠻所誘,至和城,不戰而敗,李宓死於陣。國忠又隱其敗,以捷
 書上聞。自仲通、李宓再舉討蠻之軍,其徵發皆中國利兵,然於土風不便,沮洳之所陷,瘴疫之所傷,饋餉之所乏,物故者十八九。凡舉二十萬眾,棄之死地,只輪不還,人銜冤毒,無敢言者。國忠尋兼山南西道採訪使。十一載,南蠻侵蜀,蜀人請國忠赴鎮,林甫亦奏遣之。將辭,雨泣懇陳必為林甫所排,帝憐之,不數月召還。會林甫卒,遂代為右相,兼吏部尚書、集賢殿大學士、太清太微宮使、判度支、劍南節度、山南西道採訪、兩京出納租庸
 鑄錢等使並如故。



 國忠本性疏躁,強力有口辯,既以便佞得宰相,剖決機務,居之不疑。立朝之際,或攘袂扼腕,自公卿已下,皆頤指氣使,無不讋憚。故事,宰相居臺輔之地,以元功盛德居之,不務威權,出入騎從簡易。自林甫承恩顧年深,每出車騎滿街,節將、侍郎有所關白,皆趨走闢易,有同案吏。舊例,宰相午後六刻始出歸第,林甫奏太平無事,以巳時還第,機務填委,皆決於私家。主書吳珣持籍就左相陳希烈之第,希烈引籍署名,都無可
 否。國忠代之,亦如前政。國忠自侍御史以至宰相,凡領四十餘使,又專判度支、吏部三銓,事務鞅掌,但署一字,猶不能盡,皆責成胥吏,賄賂公行。



 國忠既以宰臣典選,奏請銓日便定留放,不用長名。先天已前,諸司官知政事,午後歸本司決事,兵部尚書、侍郎亦分銓注擬。開元已後,宰臣數少,始崇其任,不歸本司。故事,吏部三銓,三注三唱,自春及夏,才終其事。國忠使胥吏於私第暗定官員,集百僚於尚書省對注唱,一日令畢,以誇神速,
 資格差謬,無復倫序。明年注擬,又於私第大集選人,令諸女弟垂簾觀之,笑語之聲,朗聞於外。故事,注官訖,過門下侍中、給事中。國忠注官時,呼左相陳希烈於座隅,給事中在列,曰:「既對注擬,過門下了矣。」吏部侍郎韋見素、張倚皆衣紫,是日與本曹郎官同咨事,趨走於屏樹之間。既退,國忠謂諸妹曰:「兩員紫袍主事何如人?」相對大噱。其所暱京兆尹鮮於仲通、中書舍人竇華、侍御史鄭昂諷選人於省門立碑,以頌國忠銓綜之能。



 貴妃姊虢
 國夫人,國忠與之私,於宣義里構連甲第,土木被綈繡,棟宇之盛,兩都莫比,晝會夜集,無復禮度。有時與虢國並轡入朝,揮鞭走馬,以為諧謔,衢路觀之,無不駭嘆。玄宗每年冬十月幸華清宮,常經冬還宮。國忠山第在宮東門之南,與虢國相對,韓國、秦國甍棟相接,天子幸其第,必過五家,賞賜宴樂。每扈從驪山,五家合隊,國忠以劍南幢節引於前,出有餞路,還有軟腳,遠近餉遺,珍玩狗馬,閹侍歌兒,相望於道。進封衛國公,食實封三百戶,
 俄拜司空。



 時安祿山恩寵特深,總握兵柄,國忠知其跋扈,終不出其下,將圖之,屢於上前言其悖逆之狀,上不之信。是時,祿山已專制河北,聚幽、並勁騎,陰圖逆節,動未有名,伺上千秋萬歲之後,方圖叛換。及見國忠用事,慮不利於己,祿山遙領內外閑廄使,遂以兵部侍郎吉溫知留後,兼御史中丞、京畿採訪使,內伺朝廷動靜。國忠使門客蹇昂、何盈求祿山陰事,圍捕其宅,得李超、安岱等,使侍御史鄭昂縊殺
 於御史臺。又奏貶吉溫於合浦,以激怒祿山,幸其搖動,內以取信於上,上竟不之悟。由是祿山惶懼,遂舉兵以誅國忠為名。玄宗聞河朔變起,欲以皇太子監國,自欲親征,謀於國忠。國忠大懼,歸謂姊妹曰:「我等死在旦夕。今東宮監國,當與娘子等並命矣。」姊妹哭訴於貴妃,貴妃銜土請命,其事乃止。及哥舒翰守潼關,諸將以函關距京師三百里,利在守險,不利出攻。國忠以翰持兵未決,慮反圖己,欲其速戰,自中督促之。翰不獲已出關,及接戰桃林,王師奔敗,哥舒受
 擒,敗國喪師,皆國忠之誤惑也。



 自祿山兵起,國忠以身領劍南節制,乃布置腹心於梁、益間,以圖自全之計。六月九日,潼關不守。十二日凌晨,上率龍武將軍陳玄禮、左相韋見素、京兆尹魏方進,國忠與貴妃及親屬,擁上出延秋門,諸王妃主從之不及,慮賊奄至,令內侍曹大仙擊鼓於春明門外,又焚芻槁之積,煙火燭天。既渡渭,即令斷便橋。辰時,至咸陽望賢驛,官吏駭竄,無復貴賤,坐宮門大樹下。亭午,上猶未食,有老父獻麥,帝令具飯,
 始得食。翌日,至馬嵬,軍士饑而憤怒,龍武將軍陳玄禮懼亂,先謂軍士曰:「今天下崩離,萬乘震蕩,豈不由楊國忠割剝氓庶,朝野怨咨,以至此耶?若不誅之以謝天下,何以塞四海之怨憤!」眾曰:「念之久矣。事行,身死固所願也。」會吐蕃和好使在驛門遮國忠訴事,軍士呼曰:「楊國忠與蕃人謀叛。」諸軍乃圍驛擒國忠,斬首以徇。是日,貴妃既縊,韓國、虢國二夫人亦為亂兵所殺。御史大夫魏方進死,左相韋見素傷。良久兵解,陳玄禮等見上謝罪
 曰:「國忠撓敗國經,構興禍亂,使黎元塗炭,乘輿播越,此而不誅,患難未已。臣等為社稷大計,請矯制之罪。」帝曰:「朕識之不明,任寄失所。近亦覺悟,審其詐佞,意欲到蜀,肆諸市朝。今神明啟卿,諧朕夙志,將疇爵賞,何至言焉。」



 是時,祿山雖據河洛,其兵鋒東止於梁、宋,南不過許、鄧。李光弼、郭子儀統河朔勁卒,連收恆、定,若崤、函固守,兵不妄動,則AT逆之勢,不討自弊。及哥舒翰出師,凡不數日,乘輿遷幸,朝廷陷沒,百僚系頸,妃主被戮,兵滿天下,
 毒流四海,皆國忠之召禍也。



 國忠子:暄、昢、曉、晞。暄為太常卿兼戶部侍郎,尚延和郡主;昢為鴻臚卿,尚萬春公主。兄弟各立第於親仁里,窮極奢侈。國忠娶蜀倡裴氏女曰裴柔,國忠既死,柔與虢國夫人皆自剄死。暄死於馬嵬;昢陷賊被殺;曉走漢中郡,漢中王瑀榜殺之;晞走至陳倉,為追兵所殺。



 國忠之黨翰林學士張漸竇華、中書舍人宋昱、吏部郎中鄭昂等,憑國忠之勢,招來賂遺,車馬盈門,財貨山積;及國忠敗,皆坐誅滅,其斫喪王室,
 俱一時之沴氣焉。



 張暐,汝州襄城人也。祖德政,武德中鄆州刺史。暐,景龍初為銅鞮令,家本豪富,好賓客,以弋獵自娛。會臨淄王為潞州別駕,暐潛識英姿,傾身事之,日奉游處。及樂人趙元禮自山東來,有女美麗,善歌舞,王幸之,止於暐第,生廢太子瑛。唐隆元年六月,王清內難,升為皇太子,召暐拜宮門大夫,每與諸王、姜皎、崔滌、李令問、王守一、薛伯陽在太子左右以接歡。其年,擢拜左臺侍御史,
 數月遷左御史臺中丞。



 先天元年,太子即位,帝居武德殿。太平公主有異謀,廣樹朋黨,暐與僕射劉幽求請先為備。太平聞之,白於睿宗,乃流暐於嶺南峰州,幽求謫於嶺外。及太平之敗,幽求追拜尚書左僕射、兼侍中;暐為大理卿,封鄧國公,實封三百戶,逾月又加權兼雍州長史。其年十二月,改元開元,以雍州為京兆府,長史為尹。暐首遷京兆尹,入侍宴私,出主都政,以為榮寵之極。暐亦有應務才幹,遷太子詹事,判尚書左右丞,再除
 左羽林大將軍,三為左金吾大將軍,又為殿中監、太僕卿。



 二十年,以暐年高,加特進。子履冰、季良、弟晤皆居清列。天寶初,暐還鄉拜掃,特賜錦袍繒彩,御賜詩以寵異之,乘傳來往,敕郡縣供擬。暐鬢發華皓,在輿中,子弟車馬連接數里,衣冠榮之。中使中路追賜藥物。至襄城月餘,詔還京。五載薨,年九十餘,贈開府儀同三司。其後,履冰為金吾將軍,季良殿中監,俱列啟戟,時人美之。暐壽考。善保終始。



 王琚,懷州河內人也。叔父隱客,則天朝為鳳閣侍郎。琚少孤而聰敏,有才略,好玄象合煉之學。神龍初,年二十餘,嘗謁駙馬王同皎,同皎甚器之,益歡洽。言及刺武三思事,琚義而許之,與周璟、張仲之為忘年之友。及同皎敗,琚恐為吏所捕,變姓名詣於江都,傭書於富商家,主人後悟其非傭者,以女嫁之,資給其財。經四五年,睿宗登極,琚具白主人,厚資其行裝,乃至長安。遇玄宗為太子監國,為太平公主所忌,思立孱弱,以竊威權,太子憂
 危。沙門普潤先與玄宗筮,克清內難,加三品,食實封,常入太子宮。琚見之,說以天時人事,歷然可觀。普潤白玄宗,玄宗異之。及琚於吏部選補諸暨主簿,於東宮過謝,及殿,而行徐視高,中官曰:「殿下在簾下。」琚曰:「在外只聞有太平公主,不聞有太子。太子有大功於社稷,大孝於君親,何得有此聲?」玄宗遽召見之,琚曰:「頃韋庶人智識淺短,親行弒逆,人心盡搖,思立李氏,殿下誅之為易。今社稷已安,太平則天之女,兇狡無比,專思立功,朝之大
 臣,多為其用。主上以元妹之愛,能忍其過。賤臣淺識,為殿下深憂。」玄宗命之同榻而坐。玄宗泣曰:「四哥仁孝,同氣唯有太平,言之恐有違犯,不言憂患轉深,為臣為子,計無所出。」琚曰:「天子之孝,貴於安宗廟。定萬人。征之於昔,蓋主,漢帝之長姊,帝幼,蓋主共養帝於宮中,後與上官桀、燕王謀害大司馬霍光,不議及君上,漢主恐危劉氏,以大義去之。況殿下功格天地,位尊儲貳。太平雖姑,臣妾也,何敢議之!今劉幽求、張說、郭元振一二大臣,心
 輔殿下。太平之黨,必有移奪安危之計,不可立談。」玄宗又曰:「公有何小藝,可隱跡與寡人游處?」琚曰:「飛丹煉藥,談諧嘲詠,堪與優人比肩。」玄宗益喜,與之為友,恨相知晚,呼為王十一。翌日,奏授詹事府司直、內供奉兼崇文學士,日與諸王及姜皎等侍奉焉,獨琚常預秘計。逾月,又拜太子舍人,尋又兼諫議大夫、內供奉,又贈其父故下邽丞仲友楚州刺史。



 先天元年七月,玄宗居尊位,在武德殿。八月,擢拜中書侍郎。時劉幽求、張暐並流於嶺
 外,琚見事迫,請早為之計。二年七月三日,琚與岐王範、薛王業、姜皎、李令問、王毛仲、王守一並預誅逆,以鐵騎至承天門。時睿宗聞鼓噪聲,召郭元振升承天樓,宣詔下關,侍御史任知古召募數百人於朝堂,不得入。頃間,琚等從玄宗至樓上,誅蕭至忠、岑義、竇懷貞、常元楷、李慈、李猷等。睿宗遜居百福殿。十日,拜琚銀青光祿大夫、戶部尚書,封趙國公,食實封五百戶;皎銀青光祿大夫、工部尚書,封楚國公,實封五百戶;令問銀青光祿大
 夫、殿中監、宋國公,實封三百戶;毛仲輔國大將軍、左武衛大將軍、檢校閑廄兼知監牧使、霍國公,實封五百戶;守一銀青光祿大夫、太常卿員外置同正員,進封晉國公,實封五百戶。琚、皎、令問並固讓尚書、殿中監,不上。十八日,琚、皎依舊官各加實封二百戶,通前七百戶。累日,玄宗宴於內殿,賜功臣金銀器皿各一床、雜彩各一千匹、絹一千匹,列於庭,宴慰終夕,載之而歸。



 琚轉見恩顧,每延入閣中,迄夜方出。歸休之日,中官至第召之。中官亦
 使尚宮就琚宅問訊琚母,時果珍味齎之,助其甘旨。琚在帷幄之側,常參聞大政,時人謂之「內宰相」,無有比者。又贈其父魏州刺史。或有上說於玄宗曰:「彼王琚、麻嗣宗譎詭縱橫之士,可與履危,不可得志。天下已定,宜益求純樸經術之士。」玄宗乃疏之。



 十一月,令御史大夫持節巡天兵以北諸軍。十二月,改年號為開元,又改官名,與蘇頲同為紫微侍郎。二年二月回,未及京,便除澤州刺史,削封。歷衡、郴、滑、虢、沔、夔、許、潤九州刺史,又復其封。
 二十年,丁母憂。二十二年,起復右庶子,兼巂州刺史,又改同、蒲、通、鄧、蔡五州刺史。天寶後,又為廣平、鄴郡二太守。性豪侈,著勛中朝,又食實封,典十五州,常受饋遺,下簷帳設,皆數千貫。玄宗念舊,常優容之。侍兒二十人,皆居寶帳。家累三百餘口,作造不遵於法式。雖居州伯,與佐官、胥吏、酋豪連榻飲謔,或樗蒱、藏金句以為樂。每移一州,車馬填路,數里不絕。攜妓從禽,恣為歡賞,垂四十年矣。



 時李邕、王弼與琚皆年齒尊高,久在外郡,書疏尺題
 來往,有「譴謫留落」之句。右相林甫以琚等負材使氣,陰議除之。五載正月,琚果為林甫構成其罪,貶琚江華郡員外司馬,削階封。至任未幾,林甫使羅希奭重按之。希奭排馬牒至,琚懼,仰藥,竟不能死;及希奭至,遂自縊而卒。死非其罪,人用憐之。寶應元年,贈太子少保。



 王毛仲,本高麗人也。父游擊將軍職事求婁,犯事沒官,生毛仲,因隸於玄宗。性識明悟,玄宗為臨淄王,常伏事左右。及出兼潞州別駕,又見李宜德趫捷善騎射,為人
 蒼頭,以錢五萬買之。景龍三年冬,玄宗還長安,以二人挾弓矢為翼。



 初,太宗貞觀中,擇官戶蕃口中少年驍勇者百人,每出游獵,令持弓矢於御馬前射生,令騎豹文韉,著畫獸文衫,謂之「百騎」。至則天時,漸加其人,謂之「千騎」,分隸左右羽林營。孝和謂之「萬騎」,亦置使以領之。玄宗在籓邸時,常接其豪俊者,或賜飲食財帛,以此盡歸心焉。毛仲亦悟玄宗旨,待之甚謹,玄宗益憐其敏惠。



 及四年六月,中宗遇弒,韋後稱制,令韋播、高嵩為羽林將
 軍,令押千騎營,榜棰以取威。其營長葛福順、陳玄禮等相與見玄宗訴冤,會玄宗已與劉幽求、麻嗣宗、薛崇簡等謀舉大計,相顧益歡,令幽求諷之,皆願決死從命。及二十日夜,玄宗入宛中,宜德從焉,毛仲避之不入。乙夜,福順等至,玄宗曰:「與公等除大逆,安社稷,各取富貴,在於俄頃,何以取信?」福順等請號而行,斯須斬韋播、韋璿、高嵩等頭來,玄宗舉火視之。又召鐘紹京領總監丁匠刀鋸百人至,因斬關而入,後及安樂公主等皆為亂
 兵所殺。其夜,少帝以玄宗著大勛,進封平王。以紹京、幽求知政事,署詔敕。崇簡、嗣宗及福順、宜德,功大者為將軍,次者為中郎將。其時,梓宮在殯,舉城縞素。及明,玄宗引新立功者皆衣紫衣緋,持滿鐵騎而出,傾城聚觀歡慰。其犯逆者,盡曝尸於城外。毛仲數日而歸,玄宗不責,又超授將軍。



 及玄宗為皇太子監國,因奏改左右萬騎左右營為龍武軍,與左右羽林為北門四軍,以福順等為將軍以押之。龍武官盡功臣,受錫齎,號為「唐元功臣」。長
 安良家子避征徭,納資以求隸於其中,遂每軍至數千人。毛仲專知東宮駝馬鷹狗等坊,未逾年,已至大將軍,階三品矣。及先天二年七月,毛仲預誅蕭、岑等功,授輔國大將軍、左武衛大將軍、檢校內外閑廄兼知監牧使,進封霍國公,實封五百戶。毛仲奉公正直,不避權貴,兩營萬騎功臣、閑廄官吏皆懼其威,人不敢犯。苑中營田草萊常收,率皆豐溢,玄宗以為能。開元十四年,贈其父秦州刺史。



 毛仲雖有賜莊宅,奴婢、駝馬、錢帛不可勝紀,
 常於閑廄側內宅住。每入侍宴賞,與諸王、姜皎等御幄前連榻而坐。玄宗或時不見,則悄然如有所失;見之則歡洽連宵,有至日晏。其妻已邑虢國夫人;賜妻李氏又為國夫人。每入內朝謁,二夫人同承賜齎,生男,孩稚已授五品,與皇太子同游,故中官楊思勖、高力士等常避畏之。七年,進位特進,行太僕卿,餘並如故。九年,持節充朔方道防禦討擊大使,仍以左領軍大總管王晙與天兵軍節度張說,東與幽州節度裴伷先等計會。



 毛仲部
 統嚴整,群牧孳息,遂數倍其初。芻粟之類,不敢盜竊,每歲回殘,常致數萬斛。不三年,扈從東封,以諸牧馬數萬匹從,每色為一隊,望如雲錦,玄宗益喜。於嶽下以宰相源乾曜、張說加左右丞相,毛仲加開府儀同三司。自玄宗先天正位後,以後父王同皎及姚崇、宋璟及毛仲十五年間四人至開府,又敕張說為《監牧頌》以美之。十七年,從朝五陵,又贈毛仲父益州大都督。毛仲益驕,嘗求為兵部尚書,玄宗不悅,毛仲怏怏,見於詞色。又福順子
 娶毛仲女,宜德、唐地文等數十人皆與毛仲善,倚之多為不法。中官等妒其全盛逾己,專發其罪,尤倨慢之。中官高品者,毛仲視之蔑如也;如卑品者,小忤意則挫辱如己之僮僕。力士輩恨入骨髓。毛仲承恩遇,妻產,嘗借苑中亭子納涼,玄宗借之。中官構之彌甚,曰:「北門奴官太盛,豪者皆一心,不除之,必起大患。」



 後毛仲索甲仗於太原軍器監,時嚴挺之為少尹,奏之。玄宗恐其黨震懼為亂,乃隱其實狀,詔曰:「開府儀同三司、兼殿中監、霍國
 公、內外閑廄監牧都使王毛仲,是惟微細,非有功績,擢自家臣,升於朝位。恩寵莫二,委任斯崇。無涓塵之益,肆驕盈之志。往屬艱難,遽茲逃匿,念深惟舊,義在優容,仍荷殊榮,蔑聞悛悔。在公無竭盡之效,居常多怨望之詞。跡其深愆,合從誅殛;恕其庸昧,宜從遠貶。可瀼州別駕員外置長任,差使馳驛領送至任,忽許東西及判事。」左領軍大將軍耿國公葛福順,貶壁州員外別駕;左監門將軍盧龍子唐地文,貶振州員外別駕;右武衛將軍成
 紀侯李守德,貶嚴州員外別駕,守德,本宜德也,立功後改名;右威衛將軍王景耀,貶黨州員外別駕;右威衛將軍高廣濟,貶道州員外別駕。毛仲男太子僕守貞,貶施州司戶;太子家令守廉,貶溪州司戶;率更令守慶,貶鶴州司倉;左監門長史守道,貶涪州參軍。連累者數十人。又詔殺毛仲,及永州而縊之。



 其後,中官益盛,而陳玄禮以淳樸自檢,宿衛宮禁,志節不衰。天寶中,玄宗在華清宮,乘馬出宮門,欲幸虢國夫人宅,玄禮曰:「未宣敕報臣,
 天子不可輕去就。」玄宗為之回轡。他年在華清宮,逼正月半,欲夜游,玄禮奏曰:「宮外即是曠野,須有備預,若欲夜游,願歸城闕。」玄宗又不能違。及安祿山反,玄禮欲於城中誅楊國忠,事不果,竟於馬嵬斬之。從玄宗入巴蜀回,封蔡國公,實封三百戶。上元元年八月致仕。



 史臣曰:李林甫以諂佞進身,位極臺輔,不懼盈滿,蔽主聰明,生既唯務陷人,死亦為人所陷,得非彼蒼假手,以示禍淫者乎!楊國忠稟性奸回,才薄行穢,領四十餘使,
 恣弄威權,天子莫見其非,群臣由之杜口,致祿山叛逆,鑾輅播遷,梟首覆宗,莫救艱步。以玄宗之睿哲,而惑於二人者,蓋巧言令色,先意承旨,財利誘之,迷而不悟也。開元任姚崇、宋璟而治,幸林甫、國忠而亂,與夫齊桓任管仲、隰朋,幸豎刁、易牙,亦何異哉!《書》曰:「臣有作福作威,害於而家,兇於而國。」孔子曰:「佞人殆。」誠哉是言也。張暐、王琚、王毛仲,皆鄧通、閎孺之流也。琚有締構之功,過多僭侈,死於非罪,亦何惜之!



 贊
 曰:天啟亂階,甫、忠當國。蔽主聰明,秉心讒慝。暐同二王,亦承恩德。籲哉僭逾,不知紀極。



\end{pinyinscope}