\article{卷一百九十三}

\begin{pinyinscope}

 ○夏侯端
 劉感常達羅士信呂子臧張道源族子楚金附李公逸張善相李玄通敬君弘馮立
 謝叔方王義方成三郎尹元貞高睿子仲舒崔琳附王同皎周憬附蘇安恆俞文俊王求禮燕欽融郎岌附安金藏



 《語》曰:「無求生以害仁,有殺身以成仁。」孟軻曰:「生亦我所欲,義亦我所欲,舍生而取義可也。」古之德行君子,動必由禮,守之以仁,造次顛沛,不愆於素。有若仲由之結纓,鉏麑之觸樹,紀信之蹈火,豫讓之斬衣,此所謂殺身成
 仁,臨難不茍者也!然受刑一代,顧瞻七族。不犯難者,有終身之利;隨市道者,獲當世之榮。茍非氣義不群,貞剛絕俗,安能碎所重之支體,徇他人之義哉!則由、麑、信、讓之徒,君人者常宜血祀,況自有其臣乎!即如安金藏剖腹以明皇嗣,段秀實挺笏而擊元兇,張巡、姚摐之守城,杲卿、真卿之罵賊,又愈於金藏。秀實等各見本傳。今採夏侯端、李心妻已下,附於此篇。



 夏侯端,壽州壽春人,梁尚書左僕射詳之孫也。仕隋為
 大理司直,高祖龍潛時,與其結交。大業中,高祖帥師於河東討捕,乃請端為副。時煬帝幸江都,盜賊日滋。端頗知玄象,善相人,說高祖曰:「金玉床搖動,此帝座不安。參墟得歲,必有真人起於實沉之次。天下方亂,能安之者,其在明公。但主上曉察,情多猜忍,切忌諸李,強者先誅,全才既死,明公豈非其次?若早為計,則應天福;不然者,則誅矣!」高祖深然其言。及義師起,端在河東,為吏所捕,送於長安,囚之。高祖入京城,釋之。引入臥內,與語極歡,
 授秘書監。



 屬李密為王世充所破,以眾來降,關東之地,未有所屬。端固請往招諭之,乃加大將軍,持節為河南道招慰使。至黎陽,李勣發兵送之,自澶水濟河,傳檄郡縣,東至於海,南至於淮,二十餘州,並遣使送款。行次譙州,會亳州刺史丁叔則及汴州刺史王要漢並以所部降於世充,路遂隔絕。



 端素得眾心,所從二千人,雖糧盡,不忍委去。端知事必不濟,乃坐澤中,盡殺私馬,以會軍士。因歔欷曰:「今王師已敗,諸處並沒,卿等土壤,悉皆從
 偽,特以共事之情,未能見委。然我奉王命,不可從。卿有妻子,無宜效我。可斬吾首,持歸於賊,必獲富貴。」眾皆流涕。端又曰:「卿不忍見殺,吾當自刎。」眾士抱持之,皆曰:「公於唐家,非有親屬,但以忠義之故,不辭於死。諸人與公共事,經涉艱危,豈有害公而取富貴!」復與同進。潛行五日,餒死者十三四;又為賊所擊,奔潰相失者大半。端唯與三十餘人東走,採生瑩豆而食之。猶持節與之俱臥起,謂眾人曰:「平生不知死地乃在此中。我受國恩,所以
 然耳,今卿等何乃相伴死乎!可散投賊,猶全性命。吾當抱此一節,與之俱殞。」眾又不去。



 屬李公逸為唐守杞州,聞而勒兵迎館之。於時河南之地,皆入世充,唯公逸感端之義,獨堅守不下。世充遣使召端,解衣遺之。禮甚厚,仍送除書,以端為淮南郡公、吏部尚書。端對其使者曰:「夏侯端天子大使,豈受王世充之官!自非斬我頭將往見汝,何容身茍活而屈於賊乎!」遂焚其書,拔刀斬其所遺衣服。因發路西歸,解節旄懷之,取竿加刃,從間道得
 至宜陽。



 初,山中險峻,先無蹊徑,但冒履榛梗,晝夜兼行,從者三十二人,或墜崖溺水、遇猛獸而死又半,其餘至者,皆鬢發禿落,形貌枯瘠。端馳驛奉見,但謝無功,殊不自言艱苦。高祖憫之,復以為秘書監。俄出為梓州刺史。所得料錢,皆散施孤寡。貞觀元年病卒。



 劉感,岐州鳳泉人,後魏司徒高昌王豐生之孫也。武德初,以驃騎將軍鎮涇州。薛仁杲率眾圍之。感嬰城拒守,城中糧盡,遂殺所乘馬以分將士,感一無所啖,唯煮馬
 骨取汁,和木屑食之。城垂陷者數矣。長平王叔良援兵至,仁杲解圍而去。感與叔良出戰,為賊所擒。仁杲復圍涇州,令感語城中云:「援軍已敗,徒守孤城,何益也!宜早出降,以全家室。」感許之。及至城下,大呼曰:「逆賊饑餓,亡在朝夕!秦王率數十萬眾,四面俱集,城中勿憂,各宜自勉,以全忠節!」仁杲大怒,執感於城邊,埋腳至膝,馳騎射殺之,至死聲色逾厲。



 賊平,高祖購得其尸,祭以少牢,贈瀛州刺史,封平原郡公,謚曰忠壯。令其子襲官爵,並賜田
 宅。



 常達,陜人也。初仕隋為鷹揚郎將,數從高祖征伐,甚蒙親待。及義兵起,達在霍邑,從宋老生來拒戰。老生敗,達懼,自匿不出。高祖謂達已死,令人閱尸求之。及達奉見,高祖大悅,以為統軍。武德初,拜隴州刺史。時薛舉屢攻之,不能克,乃遣其將仵士政以數百人偽降達。達不之測,厚加撫接。士政伺隙以其徒劫達,擁城中二千人而叛,牽達以見於舉。達詞色抗厲,不為之屈。舉指其妻謂
 達曰:「識皇後否?」達曰:「正是癭老嫗,何足可識!」竟釋之。有賊帥張貴謂達曰:「汝識我否?」答曰:「汝逃死奴。」真目視之,貴怒,拔刀將斫達。人救之,獲免。



 及仁杲平,高祖見達,謂曰:「卿之忠節,便可求之古人。」命起居舍人令狐德棻曰:「劉感、常達,須載之史策也。」執仵士政,撲殺之。賜達布帛三百段,復拜隴州刺史,卒。



 羅士信,齊州歷城人也。大業中,長白山賊王簿、左才相、孟讓來寇齊郡,通守張須陀率兵討擊。士信年始十四,
 固請自效。須陀謂曰:「汝形容未勝衣甲,何可入陣!」士信怒,重著二甲,左右雙鞬而上馬,須陀壯而從之。擊賊濰水之上。陣才列,士信馳至賊所,刺倒數人,斬一人首,擲於空中,用槍承之,戴以略陣。賊眾愕然,無敢逼者;須陀因而奮擊,賊眾大潰。士信逐北,每殺一人,輒劓其鼻而懷之;及還,則驗鼻以表殺賊之多少也。須陀甚加嘆賞,以所乘馬遺之,引置左右。每戰,須陀先登,士信為副。煬帝遣使慰喻之,又令畫工寫須陀、士信戰陣之圖,上於
 內史。



 及須陀為李密所殺,士信隨裴仁基率眾歸於密,署為總管。使統所部,隨密擊王世充。敗,士信躍馬突進,身中數矢,乃陷於世充軍。世充知其驍勇,厚禮之,與同寢食。後世充破李密,得密將邴元真等,盡拜為將軍,不復專重之。士信恥與為伍,率所部千餘人奔於穀州。高祖以為陜州道行軍總管,使圖世充。及大軍至洛陽,士信以兵圍世充千金堡。中有大罵之者,士信怒,夜遣百餘人將嬰兒數十至於堡下,詐言「從東都來投羅總管」。
 因令嬰兒啼噪,既而佯驚曰:「此千金堡,吾輩錯矣!」忽然而去。堡中謂是東都逃人,遽出兵追之。士信伏兵於路,俟其開門,奮擊大破之,殺無遺類。世充平,擢授絳州總管,封剡國公。



 尋從太宗擊劉黑闥於河北,有洺水人以城來降,遣士信入城據守。賊悉眾攻之甚急,遇雨雪,大軍不得救,經數日,城陷,為賊所擒。黑闥聞其勇,意欲活之;士信詞色不屈,遂遇害,年二十。太宗聞而傷惜,購得其尸,葬之,謚曰勇。士信初為裴仁基所禮,嘗感其知己
 之恩,及東都平,遂以家財收斂,葬於北邙。又云:「我死後,當葬此墓側。」及卒,果就仁基左而托葬焉。



 呂子臧,蒲州河東人也。大業末,為南陽郡丞。高祖克京師,遣馬元規撫慰山南,子臧堅守不下,元規遣使諷諭之,前後數輩,皆為子臧所殺。及煬帝被殺,高祖又遣其婿薛君倩齎手詔諭旨,子臧乃為煬帝發喪成禮。而後歸國,拜鄧州刺史,封南陽郡公。



 時硃粲新敗,子臧率所部數千人,與元規並力將擊之。謂元規曰:「硃粲新破之
 後,上下危懼,一戰可擒。若更遷延,部眾稍集,力強食盡,必死戰於我,為患不細也。」元規不納,子臧請以本兵獨戰,又不許。俄而粲眾大至,元規懼,退保南陽。子臧謂元規曰:「言不見納,以至於此,老夫今坐公死矣!」粲果率兵圍之,遇霖雨,城壁皆壞,所親者知城必陷,固勸其降。子臧曰:「安有天子方伯降賊者乎!」於是率其麾下,赴敵而死。俄而城陷,元規亦遇害。



 張道源,並州祁人也。年十五,父死,居喪以孝行稱,縣令
 郭湛改其所居為復禮鄉至孝里。道源嘗與友人客游,友人病,中宵而卒,道源恐驚擾主人,遂共尸臥,達曙方哭,親步營送,至其本鄉里。高祖舉義,召授大將軍府戶曹參軍。及平京城,遣道源撫慰山東,燕、趙之地爭來款附。高祖下書褒美,累封範陽郡公,後拜大理卿。時何稠、士澄有罪,家口籍沒,仍以賜之。道源嘆曰:「人有否泰,蓋亦是常。安可因己之泰,利人之否,取其子女以為僕妾,豈近仁者之心乎#」皆舍之,一無所取。尋轉太僕卿,後歷
 相州都督。武德七年卒官,贈工部尚書,謚曰節。道源雖歷職九卿,身死日,唯有粟石兩,高祖深異之,賜其家帛三百段。



 族子楚金。



 楚金,少有志行,事親以孝聞。初,與兄越石同預鄉貢進士,州司將罷越石而薦楚金,辭曰:「以順則越石長,以才則楚金不如。」固請俱退。時李勣為都督,嘆曰:「貢士本求才行,相推如此,何嫌雙居也。」乃俱薦擢第。楚金,高宗時累遷刑部侍郎。儀鳳年,有妖星見,楚金上疏,極言得失。高宗優納,賜帛二百段。則天臨朝,歷
 位吏部侍郎、秋官尚書,賜爵南陽侯。為酷吏周興所陷,配流嶺表,竟卒於徙所。著《翰苑》三十卷、《紳誡》三卷,並傳於時。



 李公逸,汴梁雍丘人也。隋末,與族弟善行以義勇為人所附。初歸王世充,知其必敗,遣間使請降。高祖因以雍丘置杞州,拜為總管,封陽夏郡公。又以善行為杞州刺史。世充遣其從弟辨率眾攻之,公逸遣使請援。高祖以其懸隔賊境,未即出兵。公逸乃留善行居守,自入朝請援,
 行至襄城,為世充伊州刺史張殷所獲,送於洛陽。世充謂曰:「卿越鄭臣唐,其說安在?」公逸答曰:「我於天下,唯聞有唐。」世充怒,斬之。善行竟沒於賊。高祖聞而悼惜,封其子為襄邑縣公。



 張善相,許州襄城人也。大業末,為里長,每督縣兵,逐小盜,為眾所附,遂據本郡,歸於李密。密敗,以城歸國,高祖授伊州總管。王世充數攻之,善相頻遣使請救。兵既不赴,城中糧盡,自知必敗,謂僚屬曰:「死當斬吾頭以歸世
 充。」眾皆泣曰:「寧與公同死,終不獨生!」後城陷被擒,送於世充,辭色不撓,罵世充極口,尋被害。高祖嘆曰:「吾負善相,善相不負吾。」封其子為襄城郡公。



 李玄通,雍州藍田人。仕隋鷹揚郎將。義兵入關,率所部歸國,累除定州總管。劉黑闥反叛,攻之,城陷被擒。黑闥重其才,欲以為大將,玄通嘆息曰:「吾荷朝恩,作籓東夏,孤城無援,遂陷虜庭。當守臣節,以忠報國,豈能降志,輒受賊官。」拒而不受。故吏有以酒食饋之者,玄通曰:「諸君
 哀吾困辱,故以酒食來相寬慰,吾當為諸君一醉。」遂與樂飲。謂守者曰:「吾能舞劍,可借吾刀。」守者與之。及曲終,太息而言:「大丈夫受國厚恩,鎮撫方面,不能保全所守,亦何面目視息世間哉!」因潰腹而死。高祖聞而為之流涕,拜其子伏護為大將。



 敬君弘,絳州太平人,齊右僕射顯雋曾孫也。武德中,為驃騎將軍,封黔昌縣侯,掌屯營兵於玄武門,加授雲麾將軍。隱太子建成之誅也,其餘黨馮立、謝叔方率兵犯
 玄武門,君弘挺身出戰。其所親止之曰:「事未可知,當且觀變,待兵集,成列而戰,未晚也。」君弘不從,乃與中郎將呂世衡大呼而進,並遇害。太宗甚嗟賞之,贈君弘左屯衛大將軍,世衡右驍衛將軍。



 馮立,同州馮翊人也。有武藝,略涉書記,隱太子建成引為翊衛車騎將軍,托以心膂。建成被誅,其左右多逃散,立嘆曰:「豈有生受其恩而死逃其難!」於是率兵犯玄武門,苦戰久之,殺屯營將軍敬君弘。謂其徒曰:「微以報太
 子矣!」遂解兵遁於野。俄而來請罪。太宗數之曰:「汝在東宮,潛為間構,阻我骨肉,汝罪一也。昨日復出兵來戰,殺傷我將士,汝罪二也。何以逃死!」對曰:「出身事主,期之效命,當職之日,無所顧憚。」因伏地歔欷,悲不自勝。太宗慰勉之。立歸,謂所親曰:「逢莫大之恩,幸而獲濟,終當以死奉答。」



 未幾,突厥至便橋。立率數百騎與虜戰於咸陽,殺獲甚眾。太宗聞而嘉嘆,拜廣州都督。前後作牧者,多以黷貨為蠻夷所患,由是數怨叛。立到,不營產業,衣食取
 給而已。嘗至貪泉,嘆曰:「此吳隱之所酌泉也。飲一杯水,何足道哉!吾當汲而為食,豈止一杯耶,安能易吾性乎!」遂畢飲而去。在職數年,甚有惠政,卒於官。



 謝叔方,雍州萬年人也。初從巢剌王元吉征討,數有戰功,元吉奏授屈咥直府左軍騎。太宗誅隱太子及元吉於玄武門,叔方率府兵與馮立合軍,拒戰於北闕下,殺敬君弘、呂世衡。太宗兵不振,秦府護軍尉遲敬德傳元吉首以示之,叔方下馬號哭而遁。明日出首,太宗曰:「義
 士也!」命釋之。歷遷西、伊二州刺史,善綏邊鎮,胡戎愛而敬之,如事嚴父。貞觀末,累加銀青光祿大夫,歷洪、廣二州都督。永徽中卒。



 王義方,泗州漣水人也。少孤貧,事母甚謹,博通《五經》,而謇傲獨行。初舉明經,因詣京師,中路逢徒步者,自云父為潁上令,聞病篤,倍道將往焉,徒步不前,計無所出。義方解所乘馬與之,不告姓名而去。俄授晉王府參軍,直弘文館。特進魏徵甚禮之,將以侄女妻之。義方竟娶征
 之侄女,告人曰:「昔不附宰相之勢,今感知己之言故也。」轉太子校書。



 無何,坐與刑部尚書張亮交通,貶為儋州吉安丞。行至海南,舟人將以酒脯致祭。義方曰:「黍稷非馨,義在明德。」乃酌水而祭,為文曰:「思帝鄉而北顧,望海浦而南浮。必也行愆諸己,義負前修。長鯨擊水,天吳覆舟。因忠獲戾,以孝見尤。四維霧廓,千里安流。靈應如響,無作神羞。」時當盛夏,風濤蒸毒,既而開霽,南渡吉安。蠻俗荒梗,義方召諸首領,集生徒,親為講經,行釋奠之禮;
 清歌吹籥,登降有序,蠻酋大喜。



 貞觀二十三年,改授洹水丞。時張亮兄子皎,配流在崖州,來依義方而卒。臨終托以妻子及致尸還鄉。義方與皎妻自誓於海神,使奴負柩,令皎妻抱其赤子,乘義方之馬,身獨步從而還。先之原武葬皎,告祭張亮,送皎妻子歸其家而往洹水。轉雲陽丞,擢為著作佐郎。



 顯慶元年,遷侍御史。時中書侍郎李義府執權用事,婦人淳于氏有美色,坐事系大理,義府悅之,托大理丞畢正義枉法出之。高宗又敕給事
 中劉仁軌、侍御史張倫重按其事。正義自縊。高宗特原義府之罪。義方以義府奸蠹害政,將加彈奏,以問其母。母曰:「昔王陵母伏劍成子之義,汝能盡忠立名,吾之願也,雖死不恨!」義方乃先奏曰:



 臣聞春鶯鳴於獻歲,蟋蟀吟於始秋,物有微而應時,人有賤而言忠。臣去歲冬初,雲陽下縣丞耳。今春及夏,陛下擢臣著作佐郎,極文學之清選。未幾,又拜臣侍御史,濫朝廷之雄職。顧視生涯,隕首非報,唯欲有犯無隱,以廣天聽。



 伏以李義府枉殺
 寺丞,陛下已赦之,臣不應更有鞫問。然天子置三公、九卿、二十七大夫、八十一元士,本欲水火相濟,鹽梅相成,然後庶績咸熙,風雨交泰。亦不可獨是獨非,皆由聖旨。昔唐堯失之於四兇,漢祖失之於陳豨,光武失之於逢萌,魏武失之於張邈。此四帝者,英傑之主,然失之於前,得之於後。今陛下繼聖,撫育萬邦,蠻陬夷落,猶懼疏網。況輦轂咫尺,奸臣肆虐,足使忠臣抗憤,義士扼腕。縱令正義自縊,彌不可容,便是畏義府之權勢,能殺身以滅
 口。此則生殺之威,上非王出;賞罰之柄,下移佞寵。臣恐履霜堅冰,積小成大,請重鞫正義死由,雪冤氣於幽泉,誅奸臣於白日。



 及廷劾義府,曰:



 臣聞附下罔上,聖主之所宜誅;心狠貌恭,明時之所必罰。是以隱賊掩義,不容唐帝之朝;竊幸乘權,終齒漢皇之劍。中書侍郎李義府,因緣際會,遂階通顯。不能盡忠竭節,對揚王休,策蹇勵駑,祗奉皇眷,而反憑附城社,蔽虧日月,請托公行,交游群小。貪冶容之美,原有罪之淳于;恐漏洩其謀,殞無辜
 之正義。雖挾山超海之力,望此猶輕;回天轉日之威,方斯更劣。此而可恕,孰不可容!金風屆節,玉露啟塗,霜簡與秋典共清,忠臣將鷹鸇並擊。請除君側,少答鴻私,碎首玉階,庶明臣節。



 高宗以義方毀辱大臣,言詞不遜,左遷萊州司戶參軍。秩滿,家於昌樂,聚徒教授。母卒,遂不復仕進。總章二年卒,年五十五。撰《筆海》十卷、文集十卷。門人何彥光、員半千為義方制師服,三年喪畢而去。



 半千者,齊州全節人也。事義方經十餘年,博涉經史,知名
 河朔。則天時官至天官侍郎。撰《三國春秋》二十卷,行於代。自有傳。



 成三郎,幽州漁陽人也。光宅年,為左豹韜衛長上果毅。李孝逸之討徐敬業,以為前鋒,與賊戰於高郵。軍國敗績,被擒,送於江都。賊黨唐之奇紿其眾曰:「此李孝逸也!」將斬之。三郎大呼曰:「我,是果毅成三郎,不是將軍李孝逸。官軍已圍爾數重,破爾在於朝夕。我死,妻子受榮;爾死,家口配沒,終不及我!」之奇怒,斬之。敬業平,贈左監門
 將軍,謚曰勇。時曲阿令尹元貞,亦死敬業之難。



 尹元貞者,瀛州河間人也。在曲阿,聞敬業攻陷潤州,率兵赴援。及戰敗,被擒。敬業臨以白刃,脅令附己,將加任用。元貞詞色慷慨,竟不之屈,尋遇害。敬業平,贈潤州刺史,謚曰壯。



 高睿,雍州萬年人,隋尚書左僕射崿孫也。父表仁,穀州刺史。睿少以明經累除桂州都督,尋加銀青光祿大夫,轉趙州刺史,封平昌縣子。聖歷初,突厥默啜來寇,睿又嬰
 城固守。長史唐波若見城圍甚急,遂潛謀應賊。睿覺之,將自殺,不死,俄而城陷被擒,更令招喻諸縣未降者。睿竟不從,遂為所殺。



 初,賊將至州境。或謂睿曰:「突厥所向無前,百姓喪膽;明公力不能御,不若降之。」睿曰:「吾為天子刺史,不戰而降,其罪大矣。」則天聞而深嘆息之,贈冬官尚書,謚曰節。及賊退,唐波若伏誅,家口籍沒。因下制曰:「故趙州刺史高睿,狂賊既至,死節不降;長史唐波若,不能固城,相率歸賊。高睿已加褒柱,波若等身死破家。
 賞罰既行,須敦懲勸,宜頒示天下,咸使知聞。」



 子仲舒,博通經史,尤明《三禮》及詁訓之書。神龍中,為相王府文學,王甚敬重之。開元中,累授中書舍人,侍中宋璟、中書侍郎蘇頲每詢訪故事焉。



 時又有中書舍人崔琳,深達政理,璟等亦禮焉。嘗謂人曰:「古事問高仲舒,今事問崔琳,則又何所疑矣!」仲舒累遷太子右庶子卒。



 王同皎,相州安陽人,陳侍中、駙馬都尉寬之曾孫。其先自瑯邪仕江左,陳亡,徙家河北。同皎,長安中尚皇太子
 女定安郡主。授朝散大夫,行太子典膳郎。敬暉等討張易之兄弟也。遣同皎與右羽林將軍李多祚迎太子於東宮,請太子至玄武門指麾將士。太子初拒而不許,同皎諷諭切至,太子乃就駕。以功授右千牛將軍,封瑯邪郡公,賜實封五百戶。及郡主進封為公主,拜同皎為駙馬都尉。尋加銀青光祿大夫,遷光祿卿。



 神龍二年,同皎以武三思專權任勢,謀為逆亂,乃招集壯士,期以則天靈駕發引,劫殺三思。同謀人撫州司倉冉祖雍,具以其
 計密告三思。三思乃遣校書郎李悛上言:「同皎潛謀殺三思後,將擁兵詣闕,廢黜皇后。」帝然之,遂斬同皎於都亭驛前,籍沒其家。臨刑神色不變,天下莫不冤之。睿宗即位,令復其官爵。執冉祖雍、李悛,並誅之。



 初與同皎葉謀,有武當丞周憬者,壽州壽春人也。事既洩,遁於比干廟中,自刎而死。臨終,謂左右曰:「比干,古之忠臣也。倘神道聰明,應知周憬忠而死也。韋後亂朝,寵樹邪佞,武三思乾上犯順,虐害忠良,吾知其滅亡不久也!可懸吾頭
 於國門,觀其身首異門而出。」其後皆如其言。



 蘇安恆,冀州武邑人也。博學,尤明《周禮》及《春秋左氏傳》。大足元年,投匭上疏曰:



 陛下欽聖皇之顧托,受嗣子之推讓,應天順人,二十年矣。豈不思虞舜褰裳,周公復闢,良以大禹至聖,成王既長,推位讓國,其道備焉!故舜之於禹,是其族親;旦舉成王,不離叔父。且族親何如子之愛?叔父何如母之恩?今太子孝敬是崇,春秋既壯,若使統臨宸極,何異陛下之隧!陛下年德既尊,寶位將倦,機
 務殷重,浩蕩心神,何不禪位東宮,自怡聖體!



 臣聞自昔明王之孝理天下者,不見二姓而俱王也。當今梁、定、河內、建昌諸王等,承陛下之廕覆,並得封王,臣恐千秋萬歲之後,於事非便,臣請黜為公侯,任以閑簡。



 臣又聞陛下有二十餘孫,今無尺土之封,此非長久之計也。臣請四面都督府及要沖州郡,分土而王之。縱今年尚幼小,未嫻養人之術,請擇立師傅,成其孝敬之道,將以夾輔周室,籓屏皇家,使累葉重光,饗祀不輟,斯為美矣,
 豈不大哉!



 疏奏,則天召見,賜食慰諭而遣之。長安二年,又上疏曰:



 忠臣不順時而取寵,烈士不惜死而偷生。故君道不明者,忠臣之過歟!臣道不軌者,烈士之過歟!昔者先皇晏駕,留其顧托,將以萬機殷廣,令陛下兼知其事。雖唐堯、虞舜居其位,而共工、驩兜在其間,陛下骨肉之恩阻,陛下子母之愛忘。臣謂聖情以運祚將喪,極斯大節;天下謂陛下微弱李氏,貪天之功。何以年在耄倦,而不能復子明闢,使忠言莫進,奸佞成朋,夷狄紛擾,屠害黎
 庶!陛下雖納隍軫念,亦罔能救此生靈。



 臣聞天下者,神堯、文武之天下也。昔有隋失馭,小人道長,群雄駭鹿,四海瞻烏。皇唐親事戎旃,鳳翔參野,削平宇縣,龍踐宸極。歃血為盟,指河為誓,非李氏不王,非功臣不封。陛下雖居正統,實唐氏舊基。故《詩》曰:「惟鵲有巢,唯鳩居之。」此言雖小,可以喻大。陛下自坤生德,乘乾作主,豈不以上符天意,下順人心!東宮昔在諒陰,相王又非長子,陛下恐宗祀中絕,所以應其謳歌。當今太子追回,年德俱盛,陛
 下貪其寶位而忘母子深恩。臣聞京邑翼翼,四方取則。陛下蔽太子之元良,枉太子之神器,何以教天下母慈子孝!焉能使天下移風易俗焉?惟陛下思之,將何聖顏以見唐家宗廟?將何誥命以謁大帝墳陵?陛下何故日夜積憂,不知鐘鳴漏盡?臣愚以天意人事,還歸李家。陛下雖安天位,殊不知物極則反,器滿則傾。故語曰:「當斷不斷,反受其亂。」此之謂也。陛下不如高揖機務,自恬聖躬,命史臣以書之,令樂府以歌之,斯亦太平之盛事也!



 臣聞見過不諫,非忠臣也;畏死不言,非勇士也。臣何惜一朝之命,而不安萬乘之國哉!故曰:茍利國家,雖死可矣!願陛下稍輟萬機,詳臣愚見。陛下若以臣為忠,則從諫如流,擇是而用;若以臣為不忠,則斬取臣頭,以令天下。



 疏奏不納。明年,御史大夫魏元忠為張易之兄弟所構,安恆又抗疏申理之曰:



 臣聞明王有含天下之量,有濟天下之心,能進天下之善,除天下之惡。若為君王而不行此四者,則當神冤鬼怒,陰錯陽亂,欲使國家榮泰,
 其可得乎!陛下革命之初,勤於庶政,親總萬機,博採謀猷,傍求俊乂,故海內以陛下為納諫之主矣!暮年已來,怠於政教,讒邪結黨,水火成災,百姓不親,五品不遜,故四海之內,以陛下為受佞之主矣!當今邪正莫辯,訴訟含冤,豈陛下昔是而今非,蓋居安忘危之失也!



 臣竊見御史大夫、檢校太子右庶子、同鳳閣鸞臺平章事魏元忠,廉直有聞,位居宰輔。履忠正之基者,用元忠為龜鏡;踐邪佞之路者,嫉元忠若仇讎。麟臺監張易之兄弟,在身無德,於
 國無功,不逾數年,遂極隆貴。自當飲冰懷懼,酌水思清,夙夜兢兢,以答恩造。不謂溪壑其志,豺狼其心,欲指鹿而獻馬,先害忠而損善;將斯亂代之法,污我明君之朝。自元忠下獄,臣見長安城內,街談巷議,皆以陛下委任奸宄,斥逐賢良,以元忠必無不順之言,以易之必有交亂之意,相逢偶語,人心不安。雖有忠臣烈士,空撫髀於私室。而鉗口不敢言者,皆懼易之等威權,恐無辜而受戮,亦徒虛死耳!



 今賊虜強盛,徵斂煩重,以臣言之,萬姓
 不勝其弊。況又聞陛下縱逸讒慝,禁錮良善,賞刑失中,則遐邇生變。臣恐四夷因之,則窺覘得失,以為邊郡之患;百姓因之,即結聚義兵,以除君側之惡。復恐逐鹿之黨,叩關而至;亂階之徒,從中相應;爭鋒於硃雀門內,問鼎於大明殿前,陛下將何事以謝之?復何方以御之?臣今為陛下計,安百姓之心者,莫若收雷電之威,解元忠之網,復其爵位,君臣如初,則天下幸甚!陛下好生惡殺,縱不能斬佞臣頭以塞人望,臣請奪其榮寵,翦其羽翼,
 無使權柄在手,驕橫日滋。專國倍於穰侯,回天過於左悺,則社稷危矣,惟陛下圖之!



 臣本微賤,不識元忠、易之,豈此可親而彼可疏?但恐讒邪長而忠臣絕!伏願陛下暫垂天鑒,察臣此心,即微臣朝志得行,夕死無恨!



 疏奏,易之等大怒,欲遣刺客殺之。賴正諫大夫硃敬則、鳳閣舍人桓彥範、著作郎魏知古等保護以免。



 安恆,神龍初為集藝館內教。節愍太子之殺武三思也,或言安恆預其謀,遂下獄死。睿宗即位,知其冤,下制曰:「故蘇安恆,文
 學基身,鯁直成操,往年抗疏,忠讜可嘉。屬回邪擅構,奄從非命,興言軫悼,用惻於懷。宜贈寵章,式旌徽烈,可贈諫議大夫。」時又有俞文俊、王求禮,亦以直言見稱。



 俞文俊者,荊州江陵人。則天載初年,新豐因風雷山移,乃改縣名為慶山,四方畢賀。文俊詣闕上書曰:「臣聞天氣不和而寒暑並,人氣不知而疣贅生,地氣不和而塠阜出。今陛下以女主處陽位,反易剛柔,故地氣隔塞而山變為災。陛下謂之慶山,臣以為非慶也。臣愚以為宜
 側身修德,以答天譴。不然,恐殃禍至矣!」則天大怒,流於嶺外。後為六道使所殺。



 王求禮者,許州長社人。則天時,為左拾遺。時武懿宗統兵討契丹,畏心耎不敢進。及賊平,懿宗奏滄、瀛等數百家從賊,請誅之。求禮廷折之曰:「此等素無武備,城池不完,遇賊畏懼,茍從之以求生,豈素有背叛之心也!懿宗擁強兵數十萬,聞賊輒退,使其滋蔓。又欲移罪於草澤詿誤之人,豈為臣之道!臣請先斬懿宗,以謝河北。」懿宗不
 能答。則天遂寬脅從者之罪。後都城三月雨雪,鳳閣侍郎蘇味道以為瑞雪,率群官表賀。求禮曰:「公為宰相,不能燮理陰陽,非時降雪,又將災而為瑞,誣罔視聽。若以三月雪為瑞雪,即臘月雷亦為瑞雷耶?」味道不從。求禮累遷左臺殿中侍御史。神龍初,為衛王掾,病卒。



 燕欽融,洛州偃師人也。景龍末,為許州司戶參軍。時韋庶人干預國政,盛封拜群從子弟。又與悖逆庶人及駙馬都尉武延秀、中書令宗楚客等將圖危宗社。欽融連
 上奏其事,庶人大怒,勸中宗召欽融廷見,撲殺之。宗楚客又私令執法者加刃,欽融因而致死。睿宗即位,下制曰:「故許州司戶參軍燕欽融,先陳忠讜,頗列章奏,雖干非其位,而進不顧身。永言奄亡,誠所傷悼,方開諫路,宜慰窀穸。可贈諫議大夫,仍令備禮改葬,特授一子官。」



 先是,定州人郎岌,亦備陳韋庶人及宗楚客將為逆亂之狀,中宗不納,而韋庶人勸杖殺之。睿宗即位,追贈諫議大夫。



 安金藏,京兆長安人,初為太常工人。載初年,則天稱制,睿宗號為皇嗣。少府監裴匪躬、內侍範雲仙並以私謁皇嗣腰斬。自此公卿已下,並不得見之,唯金藏等工人得在左右。或有誣告皇嗣潛有異謀者,則天令來俊臣窮鞫其狀。左右不勝楚毒,皆欲自誣,唯金藏確然無辭,大呼謂俊臣曰:「公不信金藏之言,請剖心以明皇嗣不反。」即引佩刀自剖其胸,五藏並出,流血被地,因氣絕而僕。則天聞之,令輿入宮中,遣醫人卻內五藏,以桑白皮
 為線縫合,傅之藥。經宿,金藏始甦。則天親臨視之,嘆曰:「吾子不能自明,不如爾之忠也!」即令俊臣停推,睿宗由是免難。



 金藏,神龍初喪母,寓葬於都南闕口之北,廬於墓側,躬造石墳石塔,晝夜不息。原上舊無水,忽有湧泉自出。又有李樹盛冬開花,犬鹿相狎。本道使盧懷慎上聞,敕旌表其門。景雲中,累遷右武衛中郎將。玄宗即位,追思金藏忠節,下制褒美,擢拜右驍衛將軍,乃令史官編次其事。開元二十年,又特封代國公,仍於東岳等諸
 碑鐫勒其名。竟以壽終,贈兵部尚書。



\end{pinyinscope}