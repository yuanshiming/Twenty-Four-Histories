\article{卷一百七十}

\begin{pinyinscope}

 ○元稹龐嚴附白居易弟
 行簡敏中附



 元稹,字微之,河南人。後魏昭成皇帝,稹十代祖也。兵部尚書、昌平公巖,六代祖也。曾祖延景,岐州參軍。祖悱,南頓丞。父寬,比部郎中、舒王府長史,以稹貴,贈左僕射。



 稹
 八歲喪父。其母鄭夫人,賢明婦人也;家貧,為稹自授書,教之書學。稹九歲能屬文。十五兩經擢第。二十四調判入第四等,授秘書省校書郎。二十八應制舉才識兼茂、明於體用科,登第者十八人,稹為第一,元和元年四月也。制下,除右拾遺。



 稹性鋒銳,見事風生。既居諫垣,不欲碌碌自滯,事無不言,即日上疏論諫職。又以前時王叔文、王伾以猥褻待詔,蒙幸太子,永貞之際,大撓朝政。是以訓導太子宮官,宜選正人。乃獻《教本書》曰:



 臣伏見陛
 下降明詔,修廢學,增胄子,選司成。大哉,堯之為君,伯夷典禮,夔教胄子之深旨也!然而事有萬萬於此者,臣敢冒昧殊死而言之。臣聞諸賈生曰:「三代之君,仁且久者,教之然也。」誠哉是言!且夫周成王,人之中才也,近管、蔡則讒入,有周、召則義聞,豈可謂天聰明哉?然而克終於道者,得不謂教之然耶?俾伯禽、唐叔與之游,《禮》、《樂》、《詩》、《書》為之習,目不得閱淫艷妖誘之色,耳不得聞優笑凌亂之音,口不得習操斷擊博之書,居不得近容順陰邪之
 黨,游不得縱追禽逐獸之樂,玩不得有遐異僻絕之珍。凡此數者,非謂備之於前而不為也,亦將不得見之矣。及其長而為君也,血氣既定,游習既成,雖有放心快己之事日陳於前,固不能奪已成之習、已定之心矣。則彼忠直道德之言,固吾之所習聞也,陳之者有以諭焉;彼庸佞違道之說,固吾之所積懼也,諂之者有以辨焉。人之情,莫不欲耀其所能而黨其所近;茍將得志,則必快其所蘊矣。物之性亦然。是以魚得水而游,馬逸駕而走,
 鳥得風而翔,火得薪而熾。此皆物之快其所蘊也。今夫成王所蘊道德也,所近聖賢也。是以舉其近,則周公左而召公右,伯禽魯而太公齊。快其蘊,則興禮樂而朝諸侯,措刑罰而美教化。教之至也,可不謂信然哉!



 及夫秦則不然。滅先王之學,曰將以愚天下;黜師保之位,曰將以明君臣。胡亥之生也,《詩》、《書》不得聞,聖賢不得近。彼趙高者,詐宦之戮人也;而傅之以殘忍戕賊之術,且曰恣睢天下以為貴,莫見其面以為尊。是以天下之人人未盡
 愚,而胡亥固已不能分獸畜矣。趙高之威懾天下,而胡亥固已自幽於深宮矣。彼李斯,秦之寵丞相也。因讒冤死,無所自明,而況於疏遠之臣庶乎!若然,則秦之亡有以致之也。



 漢高承之以兵革,漢文守之以廉謹,卒不能蘇復大訓。是以景、武、昭、宣,天資甚美,才可以免禍亂;哀、平之間,則不能虞篡弒矣。然而惠帝廢易之際,猶賴羽翼以勝邪心。是後有國之君,議教化者,莫不以興廉舉孝、設學崇儒為意,曾不知教化之不行,自貴始。略其貴
 者,教其賤者,無乃鄰於倒置乎?



 洎我太宗文皇帝之在籓邸,以至於為太子也,選知道德者十八人與之游習。即位之後,雖游宴飲食之間,若十八人者,實在其中。上失無不言,下情無不達。不四三年而名高盛古,豈一日二日而致是乎?游習之漸也!貞觀已還,師傅皆宰相兼領,其餘宮僚,亦甚重焉。馬周以位高恨不得為司議郎,此其驗也。文皇之後,漸疏賤之。用至母后臨朝,翦棄王室。當中、睿二聖勤勞之際,雖有骨鯁敢言之士,既不得
 在調護保安之職,終不能吐扶衛之一辭。而令醫匠安金藏剖腹以明之,豈不大哀也耶?



 兵興已來,茲弊尤甚。師資保傅之官,非疾廢眊聵不任事者為之,即休戎罷帥不知書者處之。至於友諭贊議之徒,疏冗散賤之甚者,縉紳恥由之。夫以匹士之愛其子者,猶求明哲慈惠之師以教之,直諒多聞之友以成之。豈天下之元良,而可以疾廢眊聵不知書者為之師乎?疏冗散賤不適用者為之友乎?此何不及上古之甚也!近制,宮僚之外,往
 往以沉滯僻老之儒,充侍直、侍讀之選,而又疏棄斥逐之,越月逾時,不得召見,彼又安能傅成道德而保養其身躬哉?臣以為積此弊者,豈不以皇天眷佑,祚我唐德,以舜繼堯,傳陛下十一聖矣,莫不生而神明,長而仁聖,以是為屑屑習儀者故不之省耳。臣獨以為於列聖之謀則可也,計傳後嗣則不可。脫或萬代之後,若有周成之中才,而又生於深宮優笑之間,無周、召保助之教,則將不能知喜怒哀樂之所自矣,況稼穡艱難乎?



 今陛下
 以上聖之資,肇臨海內,是天下之人傾耳注心之日。特願陛下思成王訓導之功,念文皇游習之漸,選重師保,慎擇宮僚,皆用博厚弘深之儒,而又明達機務者為之。更相進見,日就月將。因令皇太子聚諸生,定齒胄講業之儀,行嚴師問道之禮。至德要道以成之,徹膳記過以警之。血氣未定,則去禽色之娛以就學;聖質已備,則資游習之善以弘德。此所謂「一人元良,萬方以貞」之化也。豈直修廢學,選司成,而足倫匹其盛哉?而又俾則百王,
 莫不幼同師,長同術,識君道之素定,知天倫之自然,然後選用賢良,樹為籓屏。出則有晉、鄭、魯、衛之盛,入則有東牟、硃虛之強,蓋所謂宗子維城、犬牙盤石之勢也,又豈與夫魏、晉以降,囚賤其兄弟而自翦其本枝者,同年而語哉?



 憲宗覽之甚悅。



 又論西北邊事,皆朝政之大者。憲宗召對,問方略。為執政所忌,出為河南縣尉。丁母憂,服除,拜監察御史。



 四年,奉使東蜀,劾奏故劍南東川節度使嚴礪違制擅賦,又籍沒塗山甫等吏民八十八戶
 田宅一百一十一、奴婢二十七人、草千五百束、錢七千貫。時礪已死,七州刺史皆責罰。稹雖舉職,而執政有與礪厚者惡之。使還,令分務東臺。浙西觀察使韓皋封杖決湖州安吉令孫澥,四日內死。徐州監軍使孟升卒,節度使王紹傳送升喪柩還京,給券乘驛,仍於郵舍安喪柩。稹並劾奏以法。河南尹房式為不法事,稹欲追攝,擅令停務。既飛表聞奏,罰式一月俸,仍召稹還京。宿敷水驛,內官劉士元後至,爭。士元怒,排其戶,稹襪而走
 後。士元追之,後以棰擊稹傷面。執政以稹少年後輩,務作威福,貶為江陵府士曹參軍。



 稹聰警絕人,年少有才名,與太原白居易友善。工為詩,善狀詠風態物色,當時言詩者,稱元、白焉。自衣冠士子,至閭閻下俚,悉傳諷之,號為「元和體」。既以俊爽不容於朝,流放荊蠻者僅十年。俄而白居易亦貶江州司馬,稹量移通州司馬。雖通、江懸邈,而二人來往贈答。凡所為詩,有自三十、五十韻乃至百韻者。江南人士,傳道諷誦,流聞闕下,里巷相傳,為
 之紙貴。觀其流離放逐之意,靡不淒惋。



 十四年,自虢州長史徵還,為膳部員外郎。宰相令狐楚一代文宗,雅知稹之辭學,謂稹曰:「嘗覽足下制作,所恨不多,遲之久矣。請出其所有,以豁予情。」稹因獻其文,自敘曰:



 稹初不好文,徒以仕無他歧,強由科試。及有罪譴棄之後,自以為廢滯潦倒,不復為文字有聞於人矣。曾不知好事者抉擿芻蕪,塵瀆尊重。竊承相公特於廊廟間道稹詩句,昨又面奉教約,令獻舊文。戰汗悚踴,慚靦無地。



 稹自御史
 府謫官,於今十餘年矣。閑誕無事,遂專力於詩章。日益月滋,有詩句千餘首。其間感物寓意,可備矇瞽之風者有之。辭直氣粗,罪尤是懼,固不敢陳露於人。唯杯酒光景間,屢為小碎篇章,以自吟暢。然以為律體卑痺,格力不揚,茍無姿態,則陷流俗。常欲得思深語近,韻律調新,屬對無差,而風情宛然,而病未能也。江湖間多新進小生,不知天下文有宗主,妄相放效,而又從而失之,遂至於支離褊淺之辭,皆目為元和詩體。



 稹與同門生白居
 易友善。居易雅能詩,就中愛驅駕文字,窮極聲韻,或為千言,或五百言律詩,以相投寄。小生自審不能過之,往往戲排舊韻,別創新辭,名為次韻相酬,蓋欲以難相排。自爾江湖間為詩者,復相放效,力或不足,則至於顛倒語言,重復首尾,韻同意等,不異前篇,亦目為元和詩體。而司文者考變雅之由,往往歸咎於稹。嘗以為雕蟲小事,不足以自明。始聞相公記憶,累旬已來,實慮糞土之墻,庇之以大廈,使不復破壞,永為板築者之誤。輒寫古
 體歌詩一百首,百韻至兩韻律詩一百首,為五卷,奉啟跪陳。或希構廈之餘,一賜觀覽,知小生於章句中欒櫨榱桷之材,盡曾量度,則十餘年之邅回,不為無用矣。



 楚深稱賞,以為今代之鮑、謝也。



 穆宗皇帝在東宮,有妃嬪左右嘗誦稹歌詩以為樂曲者,知稹所為,嘗稱其善,宮中呼為元才子。荊南監軍崔潭峻甚禮接稹,不以掾吏遇之,常征其詩什諷誦之。長慶初,潭峻歸朝,出稹《連昌宮辭》等百餘篇奏御。穆宗大悅,問稹安在。對曰:「今為南
 宮散郎。」即日轉祠部郎中、知制誥。朝廷以書命不由相府,甚鄙之。然辭誥所出,夐然與古為侔,遂盛傳於代,由是極承恩顧。嘗為《長慶宮辭》數十百篇,京師競相傳唱。居無何,召入翰林,為中書舍人、承旨學士。中人以潭峻之故,爭與稹交,而知樞密魏弘簡尤與稹相善,穆宗愈深知重。河東節度使裴度三上疏,言稹與弘簡為刎頸之交,謀亂朝政,言甚激訐。穆宗顧中外人情,乃罷稹內職,授工部侍郎。上恩顧未衰。長慶二年,拜平章事。詔下
 之日,朝野無不輕笑之。



 時王廷湊、硃克融連兵圍牛元翼於深州,朝廷俱赦其罪,賜節鉞,令罷兵,俱不奉詔。稹以天子非次拔擢,欲有所立以報上。有和王傅於方者,故司空頔之子,干進於稹。言有奇士王昭、王友明二人,嘗客於燕、趙間,頗與賊黨通熟,可以反間而出元翼。仍自以家財資其行,仍賂兵吏部令史為出告身二十通,以便宜給賜,稹皆然之。有李賞者,知於方之謀,以稹與裴度有隙,乃告度云:「於方為稹所使,欲結客王昭等刺
 度。」度隱而不發。及神策軍中尉奏於方之事,乃詔三司使韓皋等訊鞫,而害裴事無驗,而前事盡露。遂俱罷稹、度平章事,乃出稹為同州刺史,度守僕射。諫官上疏,言責度太重,稹太輕。上心憐稹,止削長春宮使。



 稹初罷相,三司獄未奏,京兆尹劉遵古遣坊所由潛邏稹居第,稹奏訴之。上怒,罰遵古,遣中人撫諭稹。稹至同州,因表謝上,自敘曰:



 臣稹辜負聖明,辱累恩獎,便合自求死所,豈謂尚忝官榮?臣稹死罪。



 臣八歲喪父,家貧無業。母兄乞
 丐以供資養。衣不布體,食不充腸。幼學之年,不蒙師訓。因感鄰里兒稚有父兄為開學校,涕咽發憤,願知《詩》、《書》。慈母哀臣,親為教授。年十有五,得明經出身,由是苦心為文,夙夜強學。年二十四,登吏部乙科,授校書郎。年二十八,蒙制舉首選,授左拾遺。始自為學,至於升朝,無朋友為臣吹噓,無親戚為臣援庇。莫非苦己,實不因人,獨立性成,遂無交結。任拾遺日,屢陳時政,蒙先皇帝召問於延英。旋為宰相所憎,出臣河南縣尉。及為監察御史,
 又不規避,專心糾繩,復為宰相怒臣下庇親黨,因以他事貶臣江陵判司。廢棄十年,分死溝瀆。



 元和十四年,憲宗皇帝開釋有罪,始授臣膳部員外郎。與臣同省署者,多是臣登朝時舉人;任卿相者,半是臣同諫院時拾遺、補闕。愚臣既不料陛下天聽過卑,知臣薄藝,硃書授臣制誥,延英召臣賜緋。宰相惡臣不出其門,由是百萬侵毀。陛下察臣無罪,寵獎逾深,召臣固授舍人,遣充承旨翰林學士,金章紫服,光飾陋軀,人生之榮,臣亦至矣。然
 臣益遭誹謗,日夜憂危。唯陛下聖鑒昭臨,彌加保任,竟排群議,擢授臺司。臣忝有肺肝,豈並尋常宰相?況當行營退散之後,牛元翼未出之間,每聞陛下軫念之言,愚臣恨不身先士卒。所問於方計策,遣王友明等救解深州,蓋欲上副聖情,豈是別懷他意?不料奸人疑臣殺害裴度,妄有告論,塵瀆聖聰,愧羞天地。臣本待辨明一了,便擬殺身謝責,豈料聖慈尚加,薄貶同州。雖違咫尺之間,不遠郊圻之境,伏料必是宸衷獨斷,乞臣此官。若遣
 他人商量,乍可與臣遠處方鎮,豈肯遣臣俯近闕廷?



 所恨今月三日,尚蒙召對延英。此時不解泣血,仰辭天顏,乃至今日竄逐。臣自離京國,目斷魂銷。每至五更朝謁之時,實制淚不已。臣若餘生未死,他時萬一歸還,不敢更望得見天顏,但得再聞京城鐘鼓之音,臣雖黃土覆面,無恨九泉。臣無任自恨自慚,攀戀聖慈之至。



 在郡二年,改授越州刺史、兼御史大夫、漸東觀察使。會稽山水奇秀,稹所闢幕職,皆當時文士,而鏡湖、秦望之游,月三
 四焉。而諷詠詩什,動盈卷帙。副使竇鞏,海內詩名,與稹酬唱最多,至今稱蘭亭絕唱。稹既放意娛游,稍不修邊幅,以瀆貨聞於時。凡在越八年。



 太和初,就加檢校禮部尚書。三年九月,入為尚書左丞。振舉紀綱,出郎官頗乖公議者七人。然以稹素無檢操,人情不厭服。會宰相王播倉卒而卒,稹大為路歧,經營相位。四年正月,檢校戶部尚書,兼鄂州刺史、御史大夫、武昌軍節度使。五年七月二十二日暴疾,一日而卒於鎮,時年五十三,贈尚書
 右僕射。有子曰道護,時年三歲。稹仲兄司農少卿積,營護喪事。所著詩賦、詔冊、銘誄、論議等雜文一百卷,號曰《元氏長慶集》。又著古今刑政書三百卷,號《類集》,並行於代。



 稹長慶末因編刪其文稿,《自敘》曰:



 劉歆云:制不可削。予以為有可得而削之者,貢謀猷,持嗜欲,君有之則譽歸於上,臣專之則譽歸於下。茍而存之,其攘也,非道也。經制度,明利害,區邪正,辨嫌惑,存之則事分著,去之則是非泯。茍而削之,其過也,非道也。



 元和初,章武皇帝新
 即位,臣下未有以言刮視聽者。予時始以對詔在拾遺中供奉,由是獻《教本書》、《諫職》、《論事》等表十數通,仍為裴度、李正辭、韋熏訟所言當,而宰相曲道上語。上頗悟,召見問狀。宰相大惡之,不一月,出為河南尉。後累歲,補御史,使東川。謹以元和赦書,劾節度使嚴礪籍塗山甫等八十八家,過賦梓、遂之民數百萬。朝廷異之,奪七刺史料,悉以所籍歸於人。會潘孟陽代礪為節度使,貪過礪,且有所承迎,雖不敢盡廢詔,因命當得所籍者皆入資。
 資過其稱,榷薪盜賦無不為,仍為礪密狀不當得醜謚。予自東川還,朋礪者潛切齒矣。



 無何,分蒞東都臺。天子久不在都,都下多不法者。百司皆牢獄,有裁接吏械人逾歲而臺府不得而知之者,予因飛奏絕百司專禁錮。河南尉判官,予劾之,忤宰相旨。監徐使死於軍,徐帥郵傳其柩,柩至洛,其下歐詬主郵吏,予命吏徙柩於外,不得復乘傳。浙西觀察使封杖決安吉令至死;河南尹誣奏書生尹太階請死之;飛龍使誘趙寔家逃奴為養子;
 田季安盜娶洛陽衣冠女;汴州沒入死商錢且千萬;滑州賦於民以千,授於人以八百;朝廷饋東師,主計者誤命牛車四千三百乘飛芻越太行。類是數十事,或移或奏,皆主之。貞元已來,不慣用文法,內外寵臣皆喑嗚。會河南尹房式詐諼事發,奏攝之。前所喑嗚者叫噪。宰相素以劾叛官事相銜,乘是黜予江陵掾。後十年,始為膳部員外郎。



 穆宗初,宰相更相用事,丞相段公一日獨得對,因請亟用兵部郎中薛存慶、考功員外郎牛僧孺,予
 亦在請中,上然之。不十數日次用為給、舍,他忿恨者日夜構飛語,予懼罪,比上書自明。上憐之,三召與語。語及兵賦洎西北邊事,因命經紀之。是後書奏及進見,皆言天下事,外間不知,多臆度。陛下益憐其不漏禁中語,召入禁林,且欲亟用為宰相。是時裴度在太原,亦有宰相望,巧者謀欲俱廢之,乃以予所無構於裴。裴奏至,驗之皆失實。上以裴方握兵,不欲校曲直,出予為工部侍郎,而相裴之期亦衰矣。不累月,上盡得所構者,雖不能暴
 揚之,遂果初意,卒用予與裴俱為宰相。復有購狂民告予借客刺裴者,鞫之復無狀,而裴與予以故俱罷免。



 始元和十五年八月得見上,至是未二歲,僭忝恩寵,無是之速者;遭罹謗咎,亦無是之甚者。是以心腹腎腸,糜費於扶衛危亡之不暇,又惡暇經紀陛下之所付哉!然而造次顛沛之中,前後列上兵賦邊防之狀,可得而存者一百一十五。茍而削之,是傷先帝之器使也。至於陳暢辨謗之章,去之則無以自明於朋友矣。其餘郡縣之奏
 請,賀慶之禮,因亦附於件目。始《教本書》,至於為人雜奏,二十有七軸,凡二百二十有七奏。終歿吾世,貽之子孫式,所以明經制之難行,而銷毀之易至也。



 其自敘如此,欲知其作者之意,備於此篇。



 稹文友與白居易最善。後進之士,最重龐嚴,言其文體類己,保薦之。



 龐嚴者,壽春人。父景昭。嚴元和中登進士第,長慶元年應制舉賢良方正、能直言極諫科,策入三等,冠制科之首。是月,拜左拾遺。聰敏絕人,文章峭麗。翰林學士元稹、
 李紳頗知之。明年二月,召入翰林為學士。轉左補闕,再遷駕部郎中、知制誥。嚴與右拾遺蔣防俱為稹、紳保薦,至諫官內職。



 四年,昭愍即位,李紳為宰相李逢吉所排,貶端州司馬。嚴坐累,出為江州刺史。給事中於敖素與嚴善,制既下,敖封還,時人凜然相顧曰:「於給事犯宰相怒而為知己,不亦危乎!」及覆制出,乃知敖駁制書貶嚴太輕,中外無不嗤誚,以為口實。初李紳謫官,朝官皆賀逢吉,唯右拾遺吳思不賀。逢吉怒,改為殿中侍御史,充
 入蕃告哀使。嚴復入為庫部郎中。



 太和二年二月,上試制舉人,命嚴與左散騎常侍馮宿、太常少卿賈餗為試官,以裴休為甲等制科之首。有應直言極諫舉人劉蕡,條對激切,凡數千言。不中選,人咸以為屈。其所對策,大行於時,登科者有請以身名授蕡者。嚴再遷太常少卿。



 五年,權知京兆尹,以強幹不避權豪稱,然無士君子之檢操,貪勢嗜利。因醉而卒。



 白居易,字樂天,太原人。北齊五兵尚書建之仍孫。建生
 士通,皇朝利州都督。士通生志善,尚衣奉御。志善生溫,檢校都官郎中。溫生鍠,歷酸棗、鞏二縣令。鍠生季庚,建中初為彭城令。時李正己據河南十餘州叛。正己宗人洧為徐州刺史,季庚說洧以彭門歸國,因授朝散大夫、大理少卿、徐州別駕,賜緋魚袋,兼徐泗觀察判官。歷衢州、襄州別駕。自鍠至季庚,世敦儒業,皆以明經出身。季庚生居易。初,建立功於高齊,賜田於韓城,子孫家焉,遂移籍同州。至溫徙於下邽,今為下邽人焉。



 居易幼聰慧
 絕人,襟懷宏放。年十五六時,袖文一編,投著作郎吳人顧況。況能文,而性浮薄,後進文章無可意者。覽居易文,不覺迎門禮遇,曰:「吾謂斯文遂絕,復得吾子矣。」



 貞元十四年,始以進士就試,禮部侍郎高郢擢升甲科,吏部判入等,授秘書省校書郎。元和元年四月,憲宗策試制舉人,應才識兼茂、明於體用科,策入第四等,授盩厔縣慰、集賢校理。



 居易文辭富艷,尤精於詩筆。自讎校至結綬畿甸,所著歌詩數十百篇,皆意存諷賦,箴時之病,補政
 之缺。而士君子多之,而往往流聞禁中。章武皇帝納諫思理,渴聞讜言,二年十一月,召入翰林為學士。三年五月,拜左拾遺。居易自以逢好文之主,非次拔擢,欲以生平所貯,仰酬恩造。拜命之日,獻疏言事曰:



 蒙恩授臣左拾遺,依前翰林學士,已與崔群同狀陳謝。但言忝冒,未吐衷誠。今再瀆宸嚴,伏惟重賜詳覽。臣謹按《六典》,左右拾遺,掌供奉諷諫,凡發令舉事,有不便於時、不合於道者,小則上封,大則廷諍。其選甚重,其秩甚卑,所以然者,
 抑有由也。大凡人之情,位高則惜其位,身貴則愛其身;惜位則偷合而不言,愛身則茍容而不諫,此必然之理也。故拾遺之置,所以卑其秩者,使位未足惜,身未足愛也。所以重其選者,使下不忍負心,上不忍負恩也。夫位不足惜,恩不忍負,然後能有闕必規,有違必諫。朝廷得失無不察,天下利病無不言。此國朝置拾遺之本意也。由是而言,豈小臣愚劣暗懦所宜居之哉?



 況臣本鄉校豎儒,府縣走吏,委心泥滓,絕望煙霄。豈意聖慈,擢居近
 職,每宴飲無不先預,每慶賜無不先沾,中廄之馬代其勞,內廚之膳給其食。朝慚夕惕,已逾半年,塵曠漸深,憂愧彌劇。未申微效,又擢清班。臣所以授官已來僅經十日,食不知味,寢不遑安。唯思粉身以答殊寵,但未獲粉身之所耳。



 今陛下肇臨皇極,初受鴻名,夙夜憂勤,以求致理。每施一政、舉一事,無不合於道、便於時者。萬一事有不便於時者,陛下豈不欲聞之乎?萬一政有不合於道者,陛下豈不欲知之乎?倘陛下言動之際,詔令之間,
 小有闕遺,稍關損益,臣必密陳所見,潛獻所聞,但在聖心裁斷而已。臣又職在禁中,不同外司,欲竭愚誠,合先陳露。伏希天鑒,深察赤誠。



 居易與河南元稹相善,同年登制舉,交情隆厚。稹自監察御史謫為江陵府士曹掾,翰林學士李絳、崔群上前面論稹無罪,居易累疏切諫曰:



 臣昨緣元稹左降,頻已奏聞。臣內察事情,外聽眾議,元稹左降有不可者三。何者?元稹守官正直,人所共知。自授御史已來,舉奏不避權勢,只如奏李佐公等事,多
 是朝廷親情。人誰無私,因以挾恨,或假公議,將報私嫌,遂使誣謗之聲,上聞天聽。臣恐元稹左降已後,凡在位者,每欲舉職,必先以稹為誡,無人肯為陛下當官守法,無人肯為陛下嫉惡繩愆。內外權貴親黨,縱有大過大罪者,必相容隱而已,陛下從此無由得知。此其不可者一也。



 昨元稹所追勘房式之事,心雖徇公,事稍過當。既從重罰,足以懲違,況經謝恩,旋又左降。雖引前事以為責辭,然外議喧喧,皆以為稹與中使劉士元爭,因此
 獲罪。至於爭事理,已具前狀奏陳。況聞士元蹋破驛門,奪將鞍馬,仍索弓箭,嚇辱朝官,承前已來,未有此事。今中官有罪,未聞處置;御史無過,卻先貶官。遠近聞知,實損聖德。臣恐從今已後,中官出使,縱暴益甚;朝官受辱,必不敢言。縱有被凌辱毆打者,亦以元稹為戒,但吞聲而已。陛下從此無由得聞。此其不可二也。



 臣又訪聞元稹自去年已來,舉奏嚴礪在東川日枉法,沒入平人資產八十餘家;又奏王沼違法給券,令監軍押柩及家
 口入驛;又奏裴玢違敕征百姓草;又奏韓皋使軍將封杖打殺縣令。如此之事,前後甚多,屬朝廷法行,悉有懲罰。計天下方鎮,皆怒元稹守官。今貶為江陵判司,即是送與方鎮,從此方便報怨,朝廷何由得知?臣伏聞德宗時有崔善貞者,告李錡必反,德宗不信,送與李錡,錡掘坑熾火,燒殺善貞。曾未數年,李錡果反,至今天下為之痛心。臣恐元稹貶官,方鎮有過,無人敢言,陛下無由得知不法之事。此其不可者三也。



 若無此三不可,假如朝
 廷誤左降一御史,蓋是小事,臣安敢煩瀆聖聽,至於再三!誠以所損者深,所關者大,以此思慮,敢不極言!



 疏入不報。



 又淄青節度使李師道進絹,為魏徵子孫贖宅。居易諫曰:「徵是陛下先朝宰相,太宗嘗賜殿材成其正室,尤與諸家第宅不同。子孫典貼,其錢不多,自可官中為之收贖,而令師道掠美,事實非宜。」憲宗深然之。



 上又欲加河東王鍔平章事,居易諫曰:「宰相是陛下輔臣,非賢良不可當此位。鍔誅剝民財,以市恩澤,不可使四方之
 人謂陛下得王鍔進奉,而與之宰相,深無益於聖朝。」乃止。



 王承宗拒命,上令神策中尉吐突承璀為招討使,諫官上章者十七八。居易面論,辭情切至。既而又請罷河北用兵,凡數千百言,皆人之難言者,上多聽納。唯諫承璀事切,上頗不悅,謂李絳曰:「白居易小子,是朕拔擢致名位,而無禮於朕,朕實難奈。」絳對曰:「居易所以不避死亡之誅,事無巨細必言者,蓋酬陛下特力拔擢耳,非輕言也。陛下欲開諫諍之路,不宜阻居易言。」上曰:「卿言是
 也。」由是多見聽納。



 五年,當改官,上謂崔群曰:「居易官卑俸薄,拘於資地,不能超等,其官可聽自便奏來。」居易奏曰:「臣聞姜公輔為內職,求為京府判司,為奉親也。臣有老母,家貧養薄,乞如公輔例。」於是,除京兆府戶曹參軍。六年四月,丁母陳夫人之喪,退居下邽。九年冬,入朝,授太子左贊善大夫。



 十年七月,盜殺宰相武元衡,居易首上疏論其冤,急請捕賊以雪國恥。宰相以宮官非諫職,不當先諫官言事。會有素惡居易者,掎摭居易,言浮華
 無行,其母因看花墮井而死,而居易作《賞花》及《新井》詩,甚傷名教,不宜置彼周行。執政方惡其言事,奏貶為江表刺史。詔出,中書舍人王涯上疏論之,言居易所犯狀跡,不宜治郡,追詔授江州司馬。



 居易儒學之外,尤通釋典,常以忘懷處順為事,都不以遷謫介意。在湓城,立隱舍於廬山遺愛寺,嘗與人書言之曰:「予去年秋始游廬山,到東西二林間香爐峰下,見雲木泉石,勝絕第一。愛不能舍,因立草堂。前有喬松十數株,修竹千餘竿,青羅
 為墻援,白石為橋道,流水周於舍下,飛泉落於簷間,紅榴白蓮,羅生池砌。」居易與湊、滿、朗、晦四禪師,追永、遠、宗、雷之跡,為人外之交。每相摧游詠,躋危登險,極林泉之幽邃。至於翛然順適之際,幾欲忘其形骸。或經時不歸,或逾月而返,郡守以朝貴遇之,不之責。



 時元稹在通州,篇詠贈答往來,不以數千里為遠。嘗與稹書,因論作文之大旨曰:



 夫文,尚矣,三才各有文。天之文三光首之;地之文五材首之;人之文《六經》道之。就《六經》言,《詩》又首之。
 何者?聖人感人心而天下和平。感人心者,莫先乎情,莫始乎言,莫切乎聲,莫深乎義。詩者,根情,苗言,華聲,實義。上自賢聖,下至愚騃,微及豚魚,幽及鬼神。群分而氣同,形異而情一。未有聲入而不應、情交而不感者。聖人知其然,因其言,經之以六義;緣其聲,緯之以五音。音有韻,義有類。韻協則言順,言順則聲易入;類舉則情見,情見則感易交。於是乎孕大含深,貫微洞密,上下通而二氣泰,憂樂合而百志熙。二帝三王所以直道而行、垂拱而
 理者,揭此以為大柄,決此以為大竇也。故聞「元首明,股肱良」之歌,則知虞道昌矣。聞五子洛汭之歌,則知夏政荒矣。言者無罪,聞者作誡,言者聞者莫不兩盡其心焉。



 洎周衰秦興,採詩官廢,上不以詩補察時政,下不以歌洩導人情。用至於諂成之風動,救失之道缺。於時六義始剚矣。《國風》變為《騷辭》,五言始於蘇、李。《詩》、《騷》皆不遇者,各系其志,發而為文。故河梁之句,止於傷別;澤畔之吟,歸於怨思。徬徨抑鬱,不暇及他耳。然去《詩》未遠,梗概尚
 存。故興離別則引雙鳧一雁為喻,諷君子小人則引香草惡鳥為比。雖義類不具,猶得風人之什二三焉。於時六義始缺矣。晉、宋已還,得者蓋寡。以康樂之奧博,多溺於山水;以淵明之高古,偏放於田園。江、鮑之流,又狹於此。如梁鴻《五噫》之例者,百無一二。於時六義浸微矣!陵夷至於梁、陳間,率不過嘲風雪、弄花草而已。噫!風雪花草之物,三百篇中豈舍之乎?顧所用何如耳。設如「北風其涼」,假風以刺威虐;「雨雪霏霏」,因雪以愍征役;「棠棣之
 華」,感華以諷兄弟;「採採芣苡」,美草以樂有子也。皆興發於此而義歸於彼。反是者,可乎哉!然則「餘霞散成綺,澄江凈如練」,「歸花先委露,別葉乍辭風」之什,麗則麗矣,吾不知其所諷焉。故僕所謂嘲風雪、弄花草而已。於時六義盡去矣。



 唐興二百年,其間詩人不可勝數。所可舉者,陳子昂有《感遇詩》二十首,鮑防《感興詩》十五篇。又詩之豪者,世稱李、杜。李之作,才矣!奇矣!人不迨矣!索其風雅比興,十無一焉。杜詩最多,可傳者千餘首。至於貫穿古
 今,覙縷格律,盡工盡善,又過於李焉。然撮其《新安》、《石壕》、《潼關吏》、《蘆子關》、《花門》之章,「硃門酒肉臭,路有凍死骨」之句,亦不過十三四。杜尚如此,況不迨杜者乎?僕常痛詩道崩壞,忽忽憤發,或廢食輟寢,不量才力,欲扶起之。嗟乎!事有大謬者,又不可一二而言,然亦不能不粗陳於左右。



 僕始生六七月時,乳母抱弄於書屏下,有指「之」字、「無」字示僕者,僕口未能言,心已默識。後有問此二字者,雖百十其試,而指之不差。則知僕宿習之緣,已在文字
 中矣。及五六歲,便學為詩。九歲諳識聲韻。十五六,始知有進士,苦節讀書。二十已來,書課賦,夜課書,間又課詩,不遑寢息矣。以至於口舌成瘡,手肘成胝。既壯而膚革不豐盈,未老而齒發早衰白;瞀然如飛蠅垂珠在眸子中者,動以萬數,蓋以苦學力文之所致!



 又自悲家貧多故,年二十七,方從鄉賦。既第之後,雖專於科試,亦不廢詩。及授校書郎時,已盈三四百首。或出示交友如足下輩,見皆謂之工,其實未窺作者之域耳。自登朝來,年齒
 漸長,閱事漸多。每與人言,多詢時務;每讀書史,多求理道。始知文章合為時而著,歌詩合為事而作。是時皇帝初即位,宰府有正人,屢降璽書,訪人急病。



 僕當此日,擢在翰林,身是諫官,月請諫紙。啟奏之間,有可以救濟人病,裨補時闕,而難於指言者,輒詠歌之,欲稍稍進聞於上。上以廣宸聽,副憂勤;次以酬恩獎,塞言責;下以復吾平生之志。豈圖志未就而悔已生,言未聞而謗已成矣!



 又請為左右終言之。凡聞僕《賀雨詩》,眾口籍籍,以為非
 宜矣;聞僕《哭孔戡詩》,眾面脈脈,盡不悅矣;聞《秦中吟》,則權豪貴近者,相目而變色矣;聞《登樂游園》寄足下詩,則執政柄者扼腕矣;聞《宿紫閣村》詩,則握軍要者切齒矣!大率如此,不可遍舉。不相與者,號為沽譽,號為詆訐,號為訕謗。茍相與者,則如牛僧孺之誡焉。乃至骨肉妻孥,皆以我為非也。其不我非者,舉世不過三兩人。有鄧魴者,見僕詩而喜,無何魴死。有唐衢者,見僕詩而泣,未幾而衢死。其餘即足下。足下又十年來困躓若此。嗚呼!豈
 六義四始之風,天將破壞,不可支持耶?抑又不知天意不欲使下人病苦聞於上耶?不然,何有志於詩者,不利若此之甚也!



 然僕又自思關東一男子耳,除讀書屬文外,其他懵然無知,乃至書畫棋博,可以接群居之歡者,一無通曉,即其愚拙可知矣!初應進士時,中朝無緦麻之親,達官無半面之舊;策蹇步於利足之途,張空拳於戰文之場。十年之間,三登科第,名落眾耳,跡升清貫,出交賢俊,入侍冕旒。始得名於文章,終得罪於文章,亦其
 宜也。



 日者聞親友間說,禮、吏部舉選人,多以僕私試賦判為準的。其餘詩句,亦往往在人口中。僕恧然自愧,不之信也。及再來長安,又聞有軍使高霞寓者,欲聘倡妓,妓大誇曰:「我誦得白學士《長恨歌》,豈同他哉?」由是增價。又足下書云:到通州日,見江館柱間有題僕詩者。何人哉?又昨過漢南日,適遇主人集眾娛樂,他賓諸妓見僕來,指而相顧曰:此是《秦中吟》、《長恨歌》主耳。自長安抵江西三四千里,凡鄉校、佛寺、逆旅、行舟之中,往往有題僕
 詩者;士庶、僧徒、孀婦、處女之口,每有詠僕詩者。此誠雕篆之戲,不足為多,然今時俗所重,正在此耳。雖前賢如淵、雲者,前輩如李、杜者,亦未能忘情於其間。



 古人云:「名者公器,不可多取。」僕是何者,竊時之名已多。既竊時名,又欲竊時之富貴,使己為造物者,肯兼與之乎?今之屯窮,理固然也。況詩人多蹇,如陳子昂、杜甫,各授一拾遺,而屯剝至死。孟浩然輩不及一命,窮悴終身。近日孟郊六十,終試協律;張籍五十,未離一太祝。彼何人哉!況僕
 之才又不迨彼。今雖謫佐遠郡,而官品至第五,月俸四五萬,寒有衣,饑有食,給身之外,施及家人。亦可謂不負白氏子矣。微之,微之!勿念我哉!



 僕數月來,檢討囊帙中,得新舊詩,各以類分,分為卷目。自拾遺來,凡所遇所感,關於美刺興比者;又自武德至元和,因事立題,題為《新樂府》者,共一百五十首,謂之諷諭詩。又或退公,或臥病閑居,知足保和,吟玩性情者一百首,謂之閑適詩。又有事物牽於外,情理動於內,隨感遇而形於嘆詠者一百
 首,謂之感傷詩。又有五言、七言、長句、絕句,自百韻至兩韻者,四百餘首,謂之雜律詩。凡為十五卷,約八百首。異時相見,當盡致於執事。



 微之,古人云:「窮則獨善其身,達則兼濟天下。」僕雖不肖,常師此語。大丈夫所守者道,所待者時。時之來也,為雲龍,為風鵬,勃然突然,陳力以出;時之不來也,為霧豹,為冥鴻,寂兮寥兮,奉身而退。進退出處,何往而不自得哉!故僕志在兼濟,行在獨善,奉而始終之則為道,言而發明之則為詩。謂之諷諭詩,兼濟
 之志也;謂之閑適詩,獨善之義也。故覽僕詩者,知僕之道焉。其餘雜律詩,或誘於一時一物,發於一笑一吟,率然成章,非平生所尚者,但以親朋合散之際,取其釋恨佐歡,今銓次之間,未能刪去。他時有為我編集斯文者,略之可也。



 微之,夫貴耳賤目,榮古陋今,人之大情也。僕不能遠征古舊,如近歲韋蘇州歌行,才麗之外,頗近興諷;其五言詩,又高雅閑淡,自成一家之體,今之秉筆者誰能及之?然當蘇州在時,人亦未甚愛重,必待身後,人
 始貴之。今僕之詩,人所愛者,悉不過雜律詩與《長恨歌》已下耳。時之所重,僕之所輕。至於諷諭者,意激而言質;閑適者,思澹而辭迂。以質合迂,宜人之不愛也。今所愛者,並世而生,獨足下耳。然百千年後,安知復無如足下者出,而知愛我詩哉?故自八九年來,與足下小通則以詩相戒,小窮則以詩相勉,索居則以詩相慰,同處則以詩相娛。知吾罪吾,率以詩也。



 如今年春游城南時,與足下馬上相戲,因各誦新艷小律,不雜他篇,自皇子陂歸
 昭國里,迭吟遞唱,不絕聲者二十里餘。攀、李在傍,無所措口。知我者以為詩仙,不知我者以為詩魔。何則?勞心靈,役聲氣,連朝接夕,不自知其苦,非魔而何?偶同人當美景,或花時宴罷,或月夜酒酣,一詠一吟,不覺老之將至。雖驂鸞鶴、游蓬瀛者之適,無以加於此焉,又非仙而何?微之,微之!此吾所以與足下外形骸、脫蹤跡、傲軒鼎、輕人寰者,又以此也。



 當此之時,足下興有餘力,且欲與僕悉索還往中詩,取其尤長者,如張十八古樂府,李二
 十新歌行,盧、楊二秘書律詩,竇七、元八絕句,博搜精掇,編而次之,號為《元白往還集》。眾君子得擬議於此者,莫不踴躍欣喜,以為盛事。嗟乎!言未終而足下左轉,不數月而僕又繼行,心期索然,何日成就?又可為之太息矣!



 僕常語足下,凡人為文,私於自是,不忍於割截,或失於繁多。其間妍媸,益又自惑。必待交友有公鑒無姑息者,討論而削奪之,然後繁簡當否,得其中矣。況僕與足下,為文尤患其多。己尚病,況他人乎?今且各纂詩筆,粗為
 卷第,待與足下相見日,各出所有,終前志焉。又不知相遇是何年,相見是何地,溘然而至,則如之何?微之知我心哉!



 潯陽臘月,江風苦寒,歲暮鮮歡,夜長少睡。引筆鋪紙,悄然燈前,有念則書,言無銓次。勿以繁雜為倦,且以代一夕之話言也。



 居易自敘如此,文士以為信然。



 十三年冬,量移忠州刺史。自潯陽浮江上峽。十四年三月,元稹會居易於峽口,停舟夷陵三日。時季弟行簡從行,三人於峽州西二十里黃牛峽口石洞中,置酒賦詩,戀戀
 不能訣。南賓郡當峽路之深險處也,花木多奇。居易在郡,為《木蓮荔枝圖》,寄朝中親友,各記其狀曰:「荔枝生巴、峽間,形圓如帷蓋。葉如桂,冬青;華如橘,春榮;實如丹,夏熟。朵如蒲萄,核如枇杷,殼如紅繒,膜如紫綃,瓤肉瑩白如雪,漿液甘酸如醴酪。大略如此,其實過之。若離本枝,一日而色變,二日而香變,三日而味變,四五日外,色香味盡去矣。」「木蓮大者高四五丈,巴民呼為黃心樹,經冬不凋。身如青楊,有白文。葉如桂,厚大無脊。花如蓮,香色
 艷膩皆同,房獨蕊有異。四月初始開,自開迨謝,僅二十日。元和十四年夏,命道士毋丘元志寫之。惜其遐僻,因以三絕賦之。」有「天教拋擲在深山」之句,咸傳於都下,好事者喧然模寫。



 其年冬,召還京師,拜司門員外郎。明年,轉主客郎中、知制誥,加朝散大夫,始著緋。時元稹亦徵還為尚書郎、知制誥,同在綸閣。長慶元年三月,受詔與中書舍人王起覆,試禮部侍郎錢徽下及第人鄭朗等一十四人。十月,轉中書舍人。十一月,穆宗親試制舉人,
 又與賈餗、陳岵為考策官。凡朝廷文字之職,無不首居其選,然多為排擯,不得用其才。



 時天子荒縱不法,執政非其人,制御乖方,河朔復亂。居易累上疏論其事,天子不能用,乃求外任。七月,除杭州刺史。俄而元稹罷相,自馮翊轉浙東觀察使。交契素深,杭、越鄰境,篇詠往來,不間旬浹。嘗會於境上,數日而別。秩滿,除太子左庶子,分司東都。寶歷中,復出為蘇州刺史。文宗即位,徵拜秘書監,賜金紫。九月上誕節,召居易與僧惟澄、道土趙常盈
 對御講論於麟德殿。居易論難鋒起,辭辨泉注,上疑宿構,深嗟挹之。太和二年正月,轉刑部侍郎,封晉陽縣男,食邑三百戶。三年,稱病東歸,求為分司官,尋除太子賓客。



 居易初對策高第,擢入翰林,蒙英主特達顧遇,頗欲奮厲效報,茍致身於訏謨之地,則兼濟生靈,蓄意未果,望風為當路者所擠,流徙江湖。四五年間,幾淪蠻瘴。自是宦情衰落,無意於出處,唯以逍遙自得,吟詠情性為事。太和已後,李宗閔、李德裕朋黨事起,是非排陷,朝升
 暮黜,天子亦無如之何。楊穎士、楊虞卿與宗閔善,居易妻,穎士從父妹也。居易愈不自安,懼以黨人見斥,乃求致身散地,冀於遠害。凡所居官,未嘗終秩,率以病免,固求分務,識者多之。五年,除河南尹。七年,復授太子賓客分司。



 初,居易罷杭州,歸洛陽。於履道里得故散騎常侍楊憑宅,竹木池館,有林泉之致。家妓樊素、蠻子者,能歌善舞。居易既以尹正罷歸,每獨酌賦詠於舟中,因為《池上篇》曰:



 東都風土水木之勝在東南偏,東南之勝在履
 道里,里之勝在西北隅,西閈北垣第一第,即白氏叟樂天退老之地。地方十七畝,屋室三之一,水五之一,竹九之一,而島樹橋道間之。初樂天既為主,喜且曰:「雖有池臺,無粟不能守也」,乃作池東粟廩。又曰:「雖有子弟,無書不能訓也。」乃作池北書庫。又曰:「雖有賓朋,無琴酒不能娛也」,乃作池西琴亭,加石樽焉。



 樂天罷杭州刺史,得天竺石一、華亭鶴二以歸。始作西平橋,開環池路。罷蘇州刺史時,得太湖石五、白蓮、折腰菱、青板舫以歸,又作中
 高橋,通三島逕。罷刑部侍郎時,有粟千斛,書一車,洎臧獲之習管磬弦歌者指百以歸。先是潁川陳孝仙與釀酒法,味甚佳;博陵崔晦叔與琴,韻甚清;蜀客姜發授《秋思》,聲甚淡;弘農楊貞一與青石三,方長平滑,可以坐臥。



 太和三年夏,樂天始得請為太子賓客,分秩於洛下,息躬於池上。凡三任所得,四人所與,洎吾不才身,今率為池中物。每至池風春,池月秋,水香蓮開之旦,露清鶴唳之夕,拂楊石,舉陳酒,援崔琴,彈《秋思》,頹然自適,不知其
 他。酒酣琴罷,又命樂童登中島亭,含奏《霓裳散序》,聲隨風飄,或凝或散,悠揚於竹煙波月之際者久之。曲未竟,而樂天陶然石上矣。睡起偶詠,非詩非賦,阿龜握筆,因題石間。視其粗成韻章,命為《池上篇》云:



 十畝之宅,五畝之園,有水一池,有竹千竿。勿謂土狹,勿謂地偏,足以容膝,足以息肩。有堂有亭,有橋有船,有書有酒,有歌有弦。有叟在中,白須颯然,識分知足,外無求焉。如鳥擇木,姑務巢安;如蛙作坎,不知海寬。靈鵲怪石,紫菱白蓮,皆吾
 所好,盡在我前。時引一杯,或吟一篇。妻孥熙熙,雞犬閑閑。優哉游哉,吾將老乎其間。



 又效陶潛《五柳先生傳》,作《醉吟先生傳》以自況。文章曠達,皆此類也。



 太和末,李訓構禍,衣冠塗地,士林傷感,居易愈無宦情。開成元年,除同州刺史,辭疾不拜。尋授太子少傅,進封馮翊縣開國侯。四年冬,得風病,伏枕者累月,乃放諸妓女樊、蠻等,仍自為墓志,病中吟詠不輟。自言曰:「予年六十有八,始患風痺之疾,體郤首胘,左足不支。蓋老病相乘,有時而至
 耳。予棲心釋梵,浪跡老、莊,因疾觀身,果有所得。何則?外形骸而內忘憂患,先禪觀而後順醫治。旬月以還,厥疾少間,杜門高枕,淡然安閑。吟詠興來,亦不能遏,遂為《病中詩》十五篇以自諭。」



 會昌中,請罷太子少傅,以刑部尚書致仕。與香山僧如滿結香火社,每肩輿往來,白衣鳩杖,自稱香山居士。



 大中元年卒,時年七十六,贈尚書右僕射。有文集七十五卷,《經史事類》三十卷,並行於世。長慶末,浙東觀察使元稹,為居易集序曰:



 樂天始未言,試
 指「之」、「無」字,能不誤。始既言,讀書勤敏,與他兒異。五六歲識聲韻,十五志辭賦,二十七舉進士。貞元末,進士尚馳競,不尚文,就中六籍尤擯落。禮部侍郎高郢始用經藝為進退,樂天一舉擢上第。明年,中拔萃甲科,由是《性習相近遠》、《玄珠》、《斬白蛇劍》等賦洎百節判,新進士競相傳於京師。會憲宗皇帝策召天下士,對詔稱旨,又登甲科。未幾,選入翰林,掌制誥。比比上書言得失,因為《賀雨詩》、《秦中吟》等數十章,指言天下事,時人比之《風》、《騷》焉。



 予始
 與樂天同秘書,前後多以詩章相贈答。予譴掾江陵,樂天猶在翰林,寄予百韻律體及雜體,前後數十詩。是後各佐江、通,復相酬寄。巴、蜀、江、楚間洎長安中少年,遞相仿效,競作新辭,自謂為元和詩。而樂天《秦中吟》、《賀雨》諷諭閑適等篇,時人罕能知者。然而二十年間,禁省觀寺、郵候墻壁之上無不書;王公妾婦、牛童馬走之口無不道。其繕寫模勒,炫賣於市井,或因之以交酒茗者,處處皆是。其甚有至盜竊名姓,茍求自售,雜亂間廁,無可奈
 何。予嘗於平水市中,見村校諸童,競習歌詠,召而問之,皆對曰:「先生教我樂天、微之詩。」固亦不知予為微之也。又雞林賈人求市頗切,自云:「本國宰相,每以一金換一篇,甚偽者,宰相輒能辨別之。」自篇章已來,未有如是流傳之廣者。



 長慶四年,樂天自杭州刺史以右庶子召還,予時刺會稽,因得盡徵其文,手自排纘,成五十卷,凡二千二百五十一首。前輩多以前集、中集為名,予以為陛下明年當改元,長慶訖於是矣,因號《白氏長慶集》。



 大凡
 人之文各有所長,樂天長可以為多矣。夫諷諭之詩長於激,閑適之時長於遣,感傷之詩長於切,五字律詩百言而上長於贍,五字、七字百言而下長於情,賦贊箴誡之類長於當,碑記敘事制誥長於實,啟奏表狀長於直,書檄辭冊剖判長於盡。總而言之,不亦多乎哉!



 人以為稹序盡其能事。



 居易嘗寫其文集,送江州東西二林寺、洛城香山聖善等寺,如佛書雜傳例流行之。無子,以其侄孫嗣。遺命不歸下邽,可葬於香山如滿師塔之側,家
 人從命而葬焉。



 行簡,字知退。貞元末,登進士第,授秘書省校書郎。元和中,盧坦鎮東蜀,闢為掌書記。府罷,歸潯陽。居易授江州司馬,從兄之郡。十五年,居易入朝為尚書郎,行簡亦授左拾遺。累遷司門員外郎、主客郎中。長慶末,振武奏水運營田使賀拔志言營田數過實,詔令行簡按覆之。不實,志弘,自刺死。行簡寶歷二年冬病卒,有文集一十卷。行簡文筆有兄風,辭賦尤稱精密,文士皆師法之。居易友愛過人,兄弟相待如賓客。行簡子龜
 兒,多自教習,以至成名。當時友悌,無以比焉。



 敏中,字用晦,居易從父弟也。祖鏻,位終揚府錄事參軍。父季康,溧陽令。敏中少孤,為諸兄之所訓歷。長慶初,登進士第,佐李聽,歷河東、鄭滑、邠寧三府節度掌書記,試大理評事。大和七年,丁母憂,退居下邽。會昌初,為殿中侍御史,分司東都。尋除戶部員外郎,還京。



 武宗皇帝素聞居易之名,及即位,欲徵用之。宰相李德裕言居易衰病,不任朝謁,因言從弟敏中辭藝類居易,即日知制誥,召入翰林
 充學士,遷中書舍人。累至兵部侍郎、學士承旨。會昌末,同平章事,兼刑部尚書、集賢史館大學士。宣宗即位,加右僕射、金紫光祿大夫、太清宮使、太原郡開國公、食邑二千戶。及李德裕再貶嶺南,敏中居四輔之首,雷同毀譽,無一言伸理,特論罪之。五年,罷相,檢校司空,出為邠州刺史、邠寧節度、招撫黨項都制置等使。七年,進位特進、成都尹、劍南西川節度副大使、知節度等事。十一年二月,檢校司徒、平章事、江陵尹、荊南節度使。懿宗即位,
 徵拜司徒、門下侍郎、平章事,復輔政。尋加侍中。三年罷相,為河中尹、河中晉絳節度使。累遷中書令。太子太師致仕,卒。



 史臣曰:舉才選士之法,尚矣!自漢策賢良,隋加詩賦,罷中正之法,委銓舉之司。由是爭務雕蟲,罕趨函丈,矯首皆希於屈、宋,駕肩並擬於《風》、《騷》。或侔箴闕之篇,或敩補亡之句。咸欲錙銖《採葛》,糠秕《懷沙》,較麗藻於碧雞,鬥新奇於白鳳。暨編之簡牘,播在管弦,未逃季緒之詆訶,孰
 望《子虛》之稱賞?迨今千載,不乏辭人,統論六義之源,較其三變之體,如二班者蓋寡,類七子者幾何?至潘、陸情致之文,鮑、謝清便之作,迨於徐、庾,踵麗增華,纂組成而耀以珠璣,瑤臺構而間之金碧。國初開文館,高宗禮茂才,虞、許擅價於前,蘇、李馳聲於後。或位升臺鼎,學際天人,潤色之文,咸布編集。然而向古者傷於太僻,徇華者或至不經,齷齪者局於宮商,放縱者流於鄭、衛。若品調律度,揚搉古今,賢不肖皆賞其文,未如元、白之盛也。昔
 建安才子,始定霸於曹、劉;永明辭宗,先讓功於沈、謝。元和主盟,微之、樂天而已。臣觀元之制策,白之奏議,極文章之壺奧,盡治亂之根荄。非徒謠頌之片言,盤盂之小說。就文觀行,居易為優,放心於自得之場,置器於必安之地,優游卒歲,不亦賢乎。



 贊曰:文章新體,建安、永明。沈、謝既往,元、白挺生。但留金石,長有《莖英》。不習孫、吳,焉知用兵?



\end{pinyinscope}