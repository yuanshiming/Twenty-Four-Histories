\article{卷二百一十四}

\begin{pinyinscope}

 ○硃泚黃巢秦宗權



 硃泚,幽州昌平人。曾祖利,贊善大夫,贈禮部尚書。祖思明,太子洗馬,贈太子太師。父懷珪,天寶初,事範陽節度使裴寬為衙前,授折沖將軍。及安祿山、史思明叛,累
 為管兵將。寶應中,李懷仙歸順,奏為薊州刺史、平盧軍留後、柳城軍使。大歷元年卒,累贈左僕射。祖、父之贈,皆以泚故也。



 泚以父資從軍,幼壯偉,腰帶十圍,騎射武藝亦不出人。外若寬和,中頗殘忍。然輕財好施,每征戰所得賞物,輒分與麾下將士,以是為眾所推,故得濟其兇謀。初隸李懷仙為部將,改經略副使。硃希彩既殺李懷仙,自為節度,以泚宗姓,甚委信之。希彩為政苛酷,人不堪命。



 大歷七年秋,希彩為其下所殺,倉卒之際,未有所
 從。泚營在城北,弟滔,主衙內兵,亦得眾心。滔變詐多端,潛使百餘人於眾中大言曰:「節度使非城北硃副使莫可。」眾既無從,因共推泚。泚遂權知留後,遣使奉表京師。十月,拜檢校左散騎常侍、兼御史中丞、幽州盧龍節度等使、幽州長史、兼御史大夫。其年,泚上表令弟滔率兵二千五百人赴京西防秋。代宗嘉之,手詔褒美。



 九年,就加檢校戶部尚書,賜實封百戶。幽州及河北諸鎮,自天寶末便為逆亂之地,李懷仙、硃希彩與連境三節度,名
 雖向順,未嘗朝謁。至是泚率先上表,請自領步騎三千人入覲,詔修甲第以待之。九月,泚至京師,代宗御內殿引見,賜御馬兩匹、戰馬十匹、金銀錦彩甚厚。又以器物十床、馬四十匹、絹二萬匹、衣一千七百襲賜其將士,宴犒之盛,近時未有。泚又上表,請留京師,從之。因授其弟滔兼御史大夫、幽州節度留後。仍以河陽永平軍防秋兵,郭子儀統之;決勝軍楊猷兵,李抱玉統之;淮西鳳翔兵,馬璘統之;汴宋、淄青兵,俾泚統焉。



 十一年八月,加拜
 同平章事。尋令出鎮奉天行營,復賜金銀繒彩並內庫弓箭以寵之。十二年,加檢校司空,代李抱玉為隴右節度使,權知河西、澤潞行營兵馬事。



 德宗嗣位,加太子太師、鳳翔尹,實封至三百戶。建中元年,涇州將劉文喜阻兵為亂,加泚四鎮北庭行軍、涇原節度使,與諸軍討之。涇州平,加泚中書令,還鎮鳳翔,而以舒王讓遙領涇原節度。二年,加泚太尉。硃滔將反叛,陰使人與泚計議,以帛書納蠟丸中,置發髻間。河東節度馬燧搜獲之,以聞,並
 送帛書及所遣使。泚惶懼,頓首乞歸罪有司。上勉之曰:「千里不同謀,非卿之過。」三年四月,以張鎰代泚為鳳翔隴右節度留後,留泚京師,加實封至一千戶,與一子正員官,其幽州盧龍節度、太尉、中書令並如故。



 四年十月,涇原兵叛,鑾駕幸奉天。叛卒等以泚嘗統涇州,知其失權廢居,怏怏思亂。群寇無帥,幸泚政寬,乃相與謀曰:「硃太尉久囚空宅,若迎而為主,事必濟矣!」姚令言乃率百餘騎迎



 泚於晉昌里第。泚乘馬擁從北向,燭炬星羅,觀
 者萬計,入居含元殿。明日,移處白華殿,但稱太尉。朝官有謁泚者,悉勸奉迎鑾駕,既不合泚意,皆逡巡而退。源休至,遂屏人移時,言多悖逆。又盛陳成敗,稱述符命,勸其僭偽,泚甚悅之。又李忠臣、張光晟繼至,咸以官閑積憤,樂於禍亂。鳳翔涇原大將張廷芝、段誠諫以潰卒三千餘自襄城而至。賊泚自謂眾望所集,僭竊之心,自此而定。乃以源休為京兆尹、判度支,李忠臣為皇城使。須秀實久失兵柄,故推心委之。遂發銳師三千,言奉迎
 乘輿,實陰有逆謀。秀實與劉海賓謀誅泚,且虞叛卒之震驚法駕,乃潛為賊符,追所發兵。至六日,兵及駱驛而回。因與海賓同入見泚,為陳逆順之理,而海賓於靴中取匕首,為其所覺,遂不得前。秀實知不可以義動,遽奪源休象笏,挺而擊泚,仍大呼曰:「反虜萬段!」泚舉臂衛首,秀實格拉之,忷々然。李忠臣馳肋泚,泚素多力,才破其面,逆徒噪集,秀實、海賓遂並見害。



 明日,聲言以關王權主社稷,士庶競往觀之。八日,源休、姚令言、李忠臣、張光晟
 等八人導泚自白華入宣政殿,僭即偽位,自稱大秦皇帝,號應天元年,愚智莫不憤心。侍衛皆卒伍,行列不過十餘人。下偽詔曰:「幽囚之中。神器自至,豈朕薄德所能經營。」彭偃之詞也。偽署姚令言為侍中,李忠臣為司空、兼侍中,源休為中書侍郎、平章事、判度支,蔣鎮為吏部侍郎,樊系為禮部侍郎、禮儀使,許季常為京兆尹,洪經綸為太常少卿,彭偃為中書舍人,裴揆、崔幼貞為給事中,崔莫為御史中丞,張光晟、仇敬忠、敬



 釭、張寶、何望之、
 段誠諫、張庭芝、杜如江為節度使,仍以其兄子遂為太子,遙封弟滔為冀王。太尉、尚書令,尋又號皇太弟。



 十日,泚自領兵侵逼奉天,竊威儀輦輅,闐溢道途,蟻聚之眾軍勢頗盛;以姚令言為元帥,張光晟為副。以李忠臣為京兆尹、皇城留守,居中書省。尋以蔣鎮為門下侍郎,李子平為諫議大夫兼平章事。泚軍合於城下,渾瑊、韓游瑰御之,泚眾大敗,死者萬計。泚收軍於奉天東三里下營,大修攻具。明日,泚又分兵營於乾陵下瞰,城內大震。



 十一月三日,杜希全與泚眾戰於漠谷,官軍不利,自是泚益驕大。王師乘城而戰,人百其勇,賊多敗恤。或出野戰,官軍又獲利焉。泚乃大驅百姓填塹,夜攻城,城中設奇以應之,賊乃退縮。西明寺僧法堅有巧思,為泚造雲梯。十五日辰時,梯臨城東北隅,城內震駭。渾瑊使侯仲莊設大坑,為地道陷之。又縱火焚其梯,東風起,吹我軍,眾頗危。俄而風回,吹賊軍,瑊益薪潑油,萬鼓齊震,風吹俱熾,須臾雲梯與兇黨同為灰燼。城中三門悉出兵,王
 師又捷,其夜兵復出攻,泚眾敗績。李懷光以五萬人來援,自河北至,泚眾惶駭,因而大潰,長圍遂解焉。眾庶以懷光三日不至,城則危矣。



 三十日夜,泚走至京城。時姚令言於城中造戰格拋樓,每坊團結,人心大異。泚自奉天回,乃悉令去之,曰:「攻戰吾自有計。」前此每三五日,即使人偽自城外來,周走號令曰:「奉天已破!」百姓聞之,莫不飲泣,道路闃寂。時有入臺省吏人,不過十數輩,郎官六七人,而亦令依常年舉選,初有數十人陳狀,旬日亦
 皆屏退。泚自號其宅曰潛龍宮,悉移內庫珍貨瑰寶以實之。識者曰:「《易》稱『潛龍勿用』,此敗徵也。」無幾,百姓剽奪其珍寶,泚不能禁止。



 明年正月一日,泚改偽國號曰漢,稱天皇元年。二月,李懷光既圖叛逆,遣使與泚通和。鑾駕幸梁、洋,自此衣冠之潛匿者,出受偽官十七八焉。懷光初與泚往復通好甚密,以錢穀金帛互相饋遺。泚與書,事之如兄,約云:「削平關中,當割據山河,永為鄰國。」及懷光決計背叛,逼乘輿遷幸,泚乃下偽詔書,待懷光以
 臣禮,仍徵兵馬。懷光既為所賣,慚怒憤恥,遂領眾遁歸河中。



 三月,李晟、駱元光、尚可孤之眾,悉於城東累敗泚眾。四月,泚使韓旻、宋歸朝、張庭芝等寇武功,渾瑊以眾及吐蕃論莽羅大敗歸朝,殺逆黨萬餘人於武亭川。



 五月,泚又使仇敬忠寇藍田,尚可孤擊之,大破泚眾,擒敬忠斬之。李晟、駱元光、尚可孤遂悉師齊進,晟屯光泰門,逆徒拒官軍,王師累捷。二十八日,官軍入苑,收復京師,逆黨大潰。



 泚與姚令言、張庭芝、源休、李子平、硃遂以數
 千人西走,其餘黨或奔竄,或來降。泚眾緣路潰散,乃奔涇州,才百餘騎。田希鑒閉門登陴,泚令謂鑒曰:「我與爾節度,何故背恩?」希鑒乃使人自城上擲泚所送旌節於外,續又投火焚之。泚遂過數里,息於逆旅。泚將梁庭芬入涇州說田希鑒曰:「公比日殺馮河清背叛,今雖歸順,國家必不能久容,公他日不免受禍。何如開門納硃公,與共成大事!」希鑒以為然。庭芬乃追及泚言之,泚大悅,使庭芬卻往涇州。庭芬請授己尚書、平章事,泚不從。梁
 庭芬既求宰相不得,不復往涇州,從泚至寧州彭原縣西城屯,復與泚心腹硃惟孝共射泚。泚走,墜故窖中。泚左右韓旻、薛綸、高幽嵓、武震、硃進卿、董希芝共斬泚,使宋膺傳首以獻。泚死時年四十三。姚令言投涇州,源休、李子平走鳳翔,尋並斬獲。宋歸朝之敗武功,降於李懷光,送興元斬之。唯不獲硃遂,傳為野人所殺,或云與泚婿偽金吾將軍馬悅潛走黨項部落,數月得達幽州。



 泚之僭逆,宦豎硃重曜頗親密用事,泚每呼之為兄。時賊
 中以臘月大雨,偽星官謂泚曰:「當以宗中年長者禳其災變。」泚乃毒殺重曜,而以王禮葬焉。及京師平,亦出其尸而斬之。姚令言自有傳。



 黃巢,曹州冤句人,本以販鹽為事。乾符中,仍歲兇荒,人饑為盜,河南尤甚。初,里人王仙芝、尚君長聚盜,起於濮陽,攻剽城邑,陷曹、濮及鄆州。先有謠言云:「金色蛤蟆爭努眼,翻卻曹州天下反。」及仙芝盜起,時議畏之。左金吾衛上將軍齊克讓為兗州節度使,以本軍討仙芝。
 仙芝懼,引眾歷陳、許、襄、鄧,無少長皆虜之,眾號三十萬。三年七月,陷江陵。十月,又遣將徐君莒陷洪州。時仙芝表請符節,不允。以神策統軍使宋威為荊南節度招討使,中使楊復光為監軍。復光遣判官吳彥宏諭以朝廷釋罪,別加官爵,仙芝乃令尚君長、蔡溫球、楚彥威相次詣闕請罪,且求恩命。時宋威害復光之功,並擒送闕,敕於狗脊嶺斬之。賊怒,悉精銳擊官軍,威軍大敗,復光收其餘眾以統之。朝廷以王鐸代為招討。五年八月,收復亳州,
 斬仙芝首獻於闕下。



 先是,君長弟讓以兄奉使見誅,率部眾入嵖岈山。黃巢、黃揆昆仲八人,率盜數千依讓。月餘,眾至數萬。陷汝州,虜刺史王鐐,又掠關東。官軍加討,屢為所敗,其眾十餘萬。尚讓乃與群盜推巢為王,號沖天大將軍,仍署官屬,籓鎮不能制。時天下承平日久,人不知兵。僖宗以幼主臨朝,號令出於臣下。南衙北司,迭相矛盾,以至九流濁亂,時多朋黨,小人才勝,君子道消,賢豪忌憤,退之草澤。既一朝有變,天下離心。巢之起也,
 人士從而附之。或巢馳檄四方,章奏論列,皆指目朝政之弊,蓋士不逞者之辭也。巢徒黨既盛,與仙芝為形援。及仙芝敗,東攻亳州不下,乃襲破沂州據之。仙芝餘黨悉附焉。



 時王鐸雖銜招討之權,緩於攻取。時高駢鎮淮南,表請招討賊,許之,議加都統。巢乃渡淮,偽降於駢。駢遣將張璘率兵受降於天長鎮。巢擒璘殺之,因虜其眾。尋南陷湖、湘,遂據交、廣。托越州觀察使崔璆奏乞天平軍節度,朝議不允。又乞除官,時宰臣鄭畋與樞密使楊
 復恭奏,欲請授同正員將軍。盧攜駁其議,請授率府率,如其不受,請以高駢討之。及巢見詔,大詬執政,又自表乞安南都護、廣州節度,亦不允。然巢以士眾烏合,欲據南海之地,永為窠穴,坐邀朝命。



 是歲自春及夏,其眾大疫,死者十三四。眾勸請北歸,以圖大利。巢不得已,廣明元年,北逾五嶺,犯湖、湘、江、浙,進逼廣陵,高駢閉門自固,所過鎮戍,望風降賊。九月,渡淮。十一月十七日,陷洛陽,留守劉允章率分司官迎之。繼攻陜、虢,逼潼關,陷華州,
 留將奮鈐守之。河中節度使李都詐進表於賊。朝廷以田令孜率神策、博野等軍十萬守潼關。時禁軍皆長安富族,世籍兩軍,豐給厚賜,高車大馬,以事權豪,自少迄長,不知戰陣。初聞科集,父子聚哭,憚於出征。各於兩市出值萬計,傭雇負販屠沽及病坊窮人,以為戰士,操刀載戟,不知金敫銳。復任宦官為將帥,驅以守關。關之左有谷,可通行人,平時捉稅,禁人出入,謂之禁穀。及賊至,官軍但守潼關,不防禁谷,以為穀既官禁,賊無得而逾也。
 尚讓、林言率前鋒由禁穀而入,夾攻潼關。官軍大潰,博野都徑還京師,燔掠西市。



 十二月三日,僖宗夜自開遠門出,趨駱谷,諸王官屬相次奔命。觀軍容使田令孜、王若儔收合禁軍扈從。四日,賊至昭應,金吾大將軍張直方率在京兩班迎賊灞上。五日,賊陷京師。



 時巢眾累年為盜,行伍不勝其富,遇窮民於路,爭行施遺。既入春明門,坊市聚觀,尚讓慰曉市人曰:「黃王為生靈,不似李家不恤汝輩,但各安家。」巢賊眾競投物遺人。十三日,賊巢
 僭位,國號大齊,年稱金統,仍御樓宣赦,且陳符命曰:「唐帝知朕起義,改元廣明,以文字言之,唐已無天分矣。『唐」去『醜』『口』而安『黃』,天意令黃在唐下,乃黃家日月也。土德生金,予以金王,宜改年為金統。」賊搜訪舊宰相不獲,以前浙東觀察使崔璆、楊希古、尚讓、趙章為四相,孟楷、蓋洪為左右軍中尉,費傳古為樞密使,王璠為京兆尹,許建、硃實、劉塘為軍庫使,硃溫、張言、彭攢、季逵為諸衛大將軍、四面游奕使。又選驍勇形體魁梧者五百人,曰功
 臣。令其甥林言為軍使,比之控鶴。



 中和元年二月,尚讓寇鳳翔,鄭畋出師御之,大敗賊於龍尾坡,畋乃馳檄告喻天下籓鎮。四月,涇原行軍唐弘夫之師屯渭北,河中王重榮之師屯沙苑,易定王處存之師屯渭橋,鄜延拓拔思恭之師屯武功,鳳翔鄭畋之師屯盩至。六月,邠寧硃玫之師屯興平,忠武之師三千屯武功。是歲諸侯勤王之師,四面俱會。十二月,宰相王鐸率荊、襄之師自行在至,鄭畋帳下小校竇玫者,驍勇無敵,每夜率敢死之
 士百人,直入京師,放火燔諸門,斬級而還,賊人悚駭。



 時京畿百姓皆砦於山谷,累年廢耕耘,賊坐空城,賦輸無入,穀食騰踴,米斗三十千。官軍皆執山砦百姓,鬻於賊為食,人獲數十萬。朝士皆往來同、華,或以賣餅為業,因奔於河中。宰相崔沆、豆盧瓚扈從不及,匿之別墅,所由搜索嚴急,乃微行入永寧里張直方之家。朝貴怙直方之豪,多依之。既而或告賊云:「直方謀反,納亡命。」賊攻其第,直方族誅,沆、瓚數百人皆遇害。自是賊始酷虐,族滅
 居人。遣使傳命召故相駙馬都尉於琮於其第。琮曰:「吾唐室大臣,不可佐黃家草昧,加之老疾。」賊怒,令誅之。廣德公主並賊號咷而謂曰:「予即天子女,不宜復存,可與相公俱死。」是日並遇害。



 二年,王處存合忠武之師,敗賊將尚讓,乘勝入京師,賊遁去。處存不為備,是夜復為賊寇襲,官軍不利。賊怒坊市百姓迎王師,乃下令洗城,丈夫丁壯,殺戮殆盡,流血成渠。九月,賊將同州刺史硃溫降重榮。十一月,李克用率代北之師,自夏陽渡河,屯沙
 苑。



 三年正月,敗黃揆於沙苑,進營乾坑。二月,賊將林言、趙章、尚讓率眾十萬援華州。克用合河中、易定、忠武之師,戰於梁田坡,大敗賊軍,俘斬數萬,乘勝攻華州,塹柵以環之。克用騎軍在渭北,令薛志勤、康君立每夜突入京師,燔積聚,俘級而旋。黃揆棄華州,官軍收城。四月八日,克用合忠武騎將龐從遇賊於渭南,決戰三捷,大敗賊軍。十日夜,賊巢散走。詰旦,克用由光泰門入,收京師。巢賊出藍田、七盤路,東走關東。天下兵馬都監押楊復
 光露布獻捷於行在,陳破賊事狀曰:



 頃者妖興霧市,盜嘯叢祠,而岳牧籓侯,備盜不謹。謂大同之運,常可容奸;謂無事之秋,縱其長惡。賊首黃巢,因得充盈窟穴,蔓延萑蒲,驅我蒸黎,徇其兇逆。展鉏鶴以成鋒刃,殺耕牛以恣燔砲,魑魅晝行,虺蜴夜噬。自南海失守,湖外喪師,養虎災深,馴梟逆大,物無不害,惡靡不為,豺狼貽朝市之憂,瘡磐及腹心之痛。遂至毒流萬姓,盜污兩京。衣冠銜塗炭之悲,郡邑起丘墟之嘆。萬方共怒,十道齊攻,伏九
 廟之威靈,殄積年之兇醜。



 河中節度使王重榮神資壯烈,天付機謀,誓立功名,志安家國。至於屯田待敵,率士當沖,收百姓十萬餘家,降賊黨三萬餘眾。法當持重,功遂晚成,久稽原野之刑,未快雷霆之怒。自收同、華,逼近京師,夕烽高照於國門,游騎俯臨於灞岸。既知四隅斷絕,百計奔沖,如窮鳥觸籠,似飛蛾赴燭。



 雁門節度使李克用神傳將略,天付忠貞,機謀與武藝皆優,臣節共本心相稱。殺賊無非手刃,入陣率以身先,可謂雄才,得名
 飛將。自統本軍南下,與臣同力前驅,雖在寢餐,不忘寇孽。



 今月八日,遣衙隊前鋒楊守宗、河中騎將白志遷、橫野軍使滿存、躡雲都將丁行存、朝邑鎮將康師貞、忠武黃頭軍使龐從等三十都,隨李克用自光泰門先入京師,力摧兇寇。又遣河中將劉讓、王環、冀君武、孫珙,忠武將喬從遇,鄭滑將韓從威,荊南將申屠悰,滄州將賈滔,易定將張仲慶,壽州將張行方,天德將顧彥朗,左神策弩手甄君楚、公孫佐,橫沖軍使楊守亮,躡雲都將高周
 彞,忠順都將胡真,絳州監軍毛宣伯、聶弘裕等七十都繼進。賊尚為堅陣,來抗官軍。雁門李克用率勵驍雄,整齊金革,叫噪而聲將動瓦,喑嗚而氣欲吞沙,寬列戈矛,密張羅網。於是麾軍背擊,分騎橫沖,日明而劍躍飛輪,風急而旗開走電。使賊如浪,便可塞流;使賊如山,亦須折角。蹂踐則橫尸入地,騰凌則積血成塵,不煩即墨之牛,若駕昆陽之象。楊守宗等齊驅直入,合勢夾攻,從卯至申,群兇大潰。自望春宮前蹙殺,至昇陽殿下攻圍,戈
 不濫揮,矢無虛發。其賊一時奔走,南入商山,徒延漏刃之生,佇作飲頭之器。



 自收平京闕,二面皆立大功,若破敵摧兇,李克用實居其首。其餘將佐,同效驅馳。兼臣所部領萬餘人,數歲櫛風沐雨。既茲平蕩,並錄以聞。



 五月,巢賊先鋒將孟楷攻蔡州,節度使秦宗權以兵逆戰,為賊所敗。攻城急,宗權乃稱臣於賊。遂攻陳、許,營於溵水。陳州刺史趙犨迎戰,敗賊前鋒,生擒孟楷,斬之。黃巢素寵楷,悲惜之。乃悉眾攻陳州,營於城北五里,為宮闕之
 制,曰八仙營。於是自唐、鄧、許、汝、孟、洛、鄭、汴、曹、濮、徐、兗數十州,畢罹其毒。賊圍陳郡百日,關東仍歲無耕稼,人餓倚墻壁間,賊俘人而食,日殺數千。賊有舂磨砦,為巨碓數百,生納人於臼碎之,合骨而食,其流毒若是。



 趙犨求援於太原。四年二月,李克用率山西諸軍,由蒲、陜濟河,會關東諸侯,赴援陳州。三月,諸侯之師復集。四月,官軍敗賊於太康,俘斬萬計,拔其四壁。又敗賊將黃鄴於西華,拔其壁。巢賊大恐,收軍營於故陽里,官軍進攻之。五
 月,大雨震雷,平地水深三尺,壞賊壘,賊自離散,復聚於尉氏,逼中牟。翌日,營汴水北。是日,復大雨震電,溝塍漲流。賊分寇汴州,李克用自鄭州引軍襲擊,大敗之,獲賊將李用、楊景。殘眾保胙縣、冤句,官軍追討,賊無所保。其將李讜、楊能、霍存、葛從周、張歸厚、張歸霸各率部下降於大梁,尚讓率部下萬人歸時薄。賊自相猜間,相殺於營中,所殘者千人,中夜遁去。克用追擊至濟陰而還。賊散於兗、鄆界。黃巢入泰山,徐帥時薄遣將張友與尚讓之眾
 掩捕之。至狼虎谷,巢將林言斬巢及二弟鄴、揆等七人首,並妻子皆送徐州。是月賊平。



 秦宗權者,許州人,為郡牙將。廣明元年十月,巢賊渡淮而北。十一月,忠武軍亂,逐其帥薛能。是月,朝廷授別校周岌為許帥。初軍城未變,宗權因調發至蔡州,聞府軍亂,乃閱集蔡州之兵,欲赴難。俄聞府主殂,周岌未至,巢賊充斥,日寇郡城,宗權乃督勵士眾,登城拒守。洎岌至,即令典郡事。天子幸蜀,姑務翦寇,上蔡有勁兵萬人,宗
 權即與監軍楊復光同議勤王,出師破賊,以蔡牧授之,仍置節度之號。



 中和三年,巢賊走關東,宗權逆戰不利,因與合從為盜。巢賊既誅,宗權復熾,僭稱帝號,補署官吏。遣其將秦彥亂江淮,秦賢亂江南,秦誥陷襄陽,孫儒陷孟、洛、陜、虢至於長安,張眰陷汝、鄭,盧塘攻汴州。賊首皆慓銳慘毒,所至屠殘人物,燔燒郡邑。西至關內,東極青、齊,南出江淮,北至衛滑,魚爛鳥散,人煙斷絕,荊榛蔽野。賊既乏食,啖人為儲,軍士四出,則鹽尸而從。關東郡
 邑,多被攻陷。唯趙犨兄弟守陳州,硃溫保汴州,城門之外,為賊疆場。汴帥與兗、鄆合勢,屢敗賊軍,兇勢日削。



 龍紀元年二月,其愛將申叢執宗權,撾折其足,送於汴。硃溫出師迎勞,接之以禮。謂之曰:「下官屢以天子命達於公,如前年中翻然改圖,與下官同力勤王,則豈有今日之事乎?」宗權曰:「僕若不死,公何以興?天以僕霸公也。」略無懼色,乃檻送京師。昭宗御延喜樓受俘,京兆尹孫揆以組練礫之,徇於兩市。宗權檻中引頸謂揆曰:「尚書明
 鑒,宗權豈反者耶!但輸忠不效耳。」眾大笑。與妻趙氏俱斬於獨柳之下。



 史臣曰:我唐之受命也,置器於安,千年惟永,百蠻響化,萬國來王。但否泰之無恆,故夷險之不一。三百算祀,二十帝王。雖時有竊邑叛君之臣。乘危徼幸之輩,莫不才興兵革,即就誅夷。其間沸騰,大盜三發,安祿山、硃泚、黃巢是也。



 夫謀危社稷,將害君親,轘裂瀦宮,未塞其罪,故不俟於多談也。然盜之所起,必有其來,且無問於天時,
 宜決之於人事。



 祿山母為巫者,身是牙郎,偶緣微立邊功,遂至大加寵用,總知馬牧,特委兵權。愛天子之獨尊,與國忠之相忌,故不能以義制事,以禮制心,遂稱向闕之兵,以期非望之福,此所以為亂也!



 硃泚家本漁陽,性惟兇狡,耳習聞於篡奪,心本之於忠貞。暨弟為亂階,身留京邑,小不如意,別懷異圖。但樂荒雞之鳴,唯幸和鑾之動,緣幽帥之嘗因亂得,謂神器之可以徼求。



 黃巢亹茸微人,萑蒲賤類,因饑饉之歲,躡王、尚之蹤,志在奪攘,
 謀非遠大。一旦長驅江表,徑入關中,見五輅之蒙塵,謂寶命之在我。



 必若玄宗採九齡之語,行三令之威,不然使祿山名位不高,委任得所,則群黎未必陷於塗炭,萬乘未必越於岷,峨。



 德宗能含垢匿瑕,不佳兵尚勇,不然則取李承之言,不委希烈伐叛,不然則取公輔之諫,早令硃泚就行,如此則未必有涇原之亂兵,未必有奉天之危急!



 僖宗能知人疾苦,惠彼困窮,不然則從鄭畋之謀,赦群偷之罪,如此則黃巢不必能犯順,鑾御未必須
 省方。



 蓋差之毫厘,失之千里。蛇螫不能斷腕,蟻穴所以壞堤。後之帝王,足為殷鑒!



 史朝義、秦宗權乘彼亂離,肆行暴虐,虔劉我郡邑,僭竊我衣裳,終雖滅亡,為害斯甚,茲亦沴氣之餘也。



 贊曰:天地否閉,反逆亂常。祿山犯闕,硃泚稱皇。賊巢陵突,群豎披攘。徵其所以,存乎慢藏!



\end{pinyinscope}