\article{卷一百七十三}

\begin{pinyinscope}

 ○李訓鄭注王涯王璠賈餗舒元輿郭行餘羅立言李孝本



 李訓,肅宗時宰相揆之族孫也。始名仲言。進士擢第。形
 貌魁梧,神情灑落;辭敏智捷,善揣人意。寶歷中,從父逢吉為宰相,以訓陰險善計事,愈親厚之。初與茅匯等欲中傷李程,及武昭事發,訓坐長流嶺表,會赦得還。丁母憂,居洛中。



 時逢吉為留守,思復為宰相,且深怨裴度,居常憤鬱不樂。訓揣知其意,即以奇計動之。自言與鄭注善,逢吉以為然,遺訓金帛珍寶數百萬,令持入長安,以賂注。注得賂甚悅,乘間薦於中尉王守澄,乃以注之藥術,訓之《易》道,合薦於文宗。守澄以訓縗粗,難入禁中。帝
 令訓戎服,號王山人,與注入內。帝見其指趣,甚奇之。及訓釋服,在京師。太和八年,自流人補四門助教,召入內殿,面賜緋魚。其年十月,遷國子《周易》博士,充翰林侍講學士。入院日,賜宴,宣法曲弟子二十人就院奏法曲以寵之。兩省諫官伏閣切諫,言訓奸邪,海內聞知,不宜令侍宸扆,終不聽。



 文宗性守正嫉惡,以宦者權寵太過,繼為禍胎,元和末弒逆之徒尚在左右,雖外示優假,心不堪之。思欲芟落本根,以雪讎恥,九重深處,難與將相明
 言。前與侍講宋申錫謀。謀之不臧,幾成反噬,自是巷伯尤橫。因鄭注得幸守澄,俾之援訓,冀黃門之不疑也。訓既在翰林,解《易》之際,或語及巷伯事,則再三憤激,以動上心。以其言論縱橫,謂其必能成事,遂以真誠謀於訓、注。自是二人寵幸,言無不從;而深秘之謀,往往流聞於外。上慮中人猜慮,乃疏《易》義六條,示於百闢,有能出訓之意者賞之,蓋欲知上以師友寵之。九年七月,改兵部郎中、知制誥,充翰林學士。九月,遷禮部侍郎、同平章事,
 仍賜金紫之服。詔以平章之暇,三五日一入翰林。



 訓既秉權衡,即謀誅內豎。中官陳弘慶者,自元和末負弒逆之名,忠義之士無不扼腕。時為襄陽監軍,乃召自漢南,至青泥驛,遣人封杖決殺。王守澄自長慶已來知樞密,典禁軍,作威作福。訓既作相,以守澄為六軍十二衛觀軍容使,罷其禁旅之權,尋賜鴆殺之。訓愈承恩顧,每別殿奏對,他宰相莫不順成其言,黃門禁軍迎拜戢斂。訓本以纖達,門庭趨附之士,率皆狂怪險異之流。時亦能
 取正人偉望,以鎮人心。天下之人,有冀訓以致太平者,不獨人主惑其言。



 訓雖為鄭注引用,及祿位俱大,勢不兩立;托以中外應赴之謀,出注為鳳翔節度使。俟誅內豎,即兼圖注。約以其年十一月誅中官,須假兵力,乃以大理卿郭行餘為邠寧節度使,戶部尚書王璠為太原節度使,京兆少尹羅立言權知大尹事,太府卿韓約為金吾街使,刑部郎中知雜李孝本權知中丞事,皆訓之親厚者。冀王璠、郭行餘未赴鎮間,廣令召募豪俠及金
 吾臺府之從者,俾集其事。



 是月二十一日,帝御紫宸。班定,韓約不報平安,奏曰:「金吾左仗院石榴樹,夜來有甘露,臣已進狀訖。」乃蹈舞再拜。宰相百官相次稱賀。李訓奏曰:「甘露降祥,俯在宮禁。陛下宜親幸左仗觀之。」班退,上乘軟舁出紫宸門,由含元殿東階升殿,宰相侍臣分立於副階,文武兩班,列於殿前。上令宰相兩省官先往視之。既還,曰:「臣等恐非真甘露,不敢輕言。言出,四方必稱賀也。」上曰:「韓約妄耶?」乃令左右軍中尉、樞密內臣往
 視之。



 既去,訓召王璠、郭行餘曰:「來受敕旨!」璠恐悚不能前,行餘獨拜殿下。時兩鎮官健,皆執兵在丹鳳門外,訓已令召之,唯璠從兵入,邠寧兵竟不至。中尉、樞密至左仗,聞幕下有兵聲,驚恐走出。閽者欲扃鎖之,為中人所叱,執關而不能下。內官回奏,韓約氣懾汗流,不能舉首。中官謂之曰:「將軍何及此耶?」又奏曰:「事急矣,請陛下入內。」即舉軟輿迎帝。訓殿上呼曰:「金吾衛士上殿來,護乘輿者,人賞百千。」內官決殿後罘罳,舉輿疾趨。訓攀呼曰:「
 陛下不得入內。」金吾衛士數十人,隨訓而入。羅立言率府中從人自東來,李孝本率臺中從人自西來,共四百餘人,上殿縱擊內官,死傷者數十人。訓時愈急,邐迤入宣政門。帝瞋目叱訓,內官卻志榮奮拳擊其胸,訓即殭僕於地。帝入東上閣門,門即闔,內官呼萬歲者數四。須臾,內官率禁兵五百人,露刃出閣門,遇人即殺。宰相王涯、賈餗、舒元輿、方中書會食,聞難出走,諸司從吏死者六七百人。



 是日,訓中拳而僕,知事不濟,乃單騎走入終
 南山,投寺僧宗密。訓與宗密素善,欲剃其發匿之。從者止之,乃趨鳳翔,欲依鄭注。出山,為盩厔鎮將宗楚所得,械送京師。至昆明池,訓恐入軍別受搒掠,乃謂兵士曰:「所在有兵,得我者即富貴,不如持我首行,免被奪取。」乃斬訓,持首而行。



 訓弟仲景、再從弟戶部員外郎元皋,皆伏法。



 仇士良以宗密容李訓,遺人縛入左軍,責以不告之罪。將殺之,宗密怡然曰:「貧僧識訓年深,亦知反叛。然本師教法,遇苦即救,不愛身命,死固甘心。」中尉魚弘志
 嘉之,奏釋其罪。



 鄭注,絳州翼城人,始以藥術游長安權豪之門。本姓魚,冒姓鄭氏,故時號魚鄭。注用事時,人目之為「水族」。



 元和十三年,李愬為襄陽節度使,注往依之。愬得其藥力,因厚遇之,署為節度衙推。從愬移鎮徐州,又為職事,軍政可否,醖與之參決。注詭辯陰狡,善探人意旨,與愬籌謀,未嘗不中其意。然挾邪任數,專作威福,軍府患之。時王守澄監徐軍,深怒注。一日,以軍情患注白於愬。愬曰:「彼
 雖如此,實奇才也。將軍試與之語;茍不如旨,去未為晚」愬即令謁監軍。守澄初有難色,及延坐與語,機辯縱衡,盡中其意,遂延於內室,促膝投分,恨相見之晚。翌日,守澄謂愬曰:「誠如公言,實奇士也。」自是出入守澄之門,都無限隔。愬署為巡官,齒於賓席。



 及守澄入知樞密,當長慶、寶歷之際,國政多專於守澄。注晝伏夜動,交通賂遺。初則讒邪奸巧之徒附之以圖進取;數年之後,達僚權臣,爭湊其門。累從山東、京西諸軍,歷衛佐、評事、御史,又
 檢校庫部郎中,為昭義節度副使。既以陰事誣陷宋申錫,守道正人,始側目焉。



 太和七年,罷邠寧行軍司馬,入京師。御史李款閣內彈之曰:「鄭注內通敕使,外結朝官,兩地往來,卜射財貨,晝伏夜動,干竊化權。人不敢言,道路以目。請付法司。」旬日內,諫章十數,文宗不納。尋授注通王府司馬,充右神策判官,中外駭嘆。八年九月,注進藥方一卷,令守澄召注對浴堂門,賜錦彩。召對之夕,彗出東方,長三尺,光耀甚緊。其年十二月,拜太僕卿、兼御
 史大夫。



 注起第善和里,通於永巷,長廊復壁。日聚京師輕薄子弟、方鎮將吏,以招權利。間日入禁軍,與守澄款密,語必移時,或通夕不寐。李訓既附注以進,承間入謁;而輕浮躁進者,盈於注門。九年八月,遷工部尚書,充翰林侍講學士。召自九仙門,帝面賜告身。時李訓已在禁庭,二人相洽,日侍君側,講貫太平之術,以為朝夕可致升平。兩奸合從,天子益惑其說。是時,訓、注之權,赫於天下。既得行其志,生平恩仇,絲毫必報。因楊虞卿之獄,
 挾忌李宗閔、李德裕,心所惡者,目為二人之黨。朝士相繼斥逐,班列為之一空,人人惴慄,若崩厥角。帝微知之,下詔慰諭,人情稍安。



 訓、注天資狂妄,偷合茍容,至於經略謀猷,無可稱者。初浴堂召對,上訪以富人之術,乃以榷茶為對。其法,欲以江湖百姓茶園,官自造作,量給直分,命使者主之。帝惑其言,乃命王涯兼榷茶使。又言秦中有災,宜興工役以禳之。文宗能詩,嘗吟杜甫《江頭篇》云:「江頭宮殿鎖千門,細柳新蒲為誰綠?」始知天寶已前,環
 曲江四岸,有樓臺行宮廨署,心切慕之。既得注言,即命左右神策軍差人淘曲江、昆明二池,仍許公卿士大夫之家於江頭立亭館,以時追賞。時兩軍造紫雲樓、彩霞亭,內出樓額以賜之。注言無不從,皆此類也。



 九月,檢校尚書左僕射、鳳翔尹、鳳翔節度使。蓋與李訓謀事有期,欲中外協勢。十一月,注聞訓事發,自鳳翔率親兵五百餘人赴闕。至扶風,聞訓敗,乃還。監軍使張仲清已得密詔,迎而勞之,召至監軍府議事。注倚兵衛即赴之,仲清已
 伏兵幕下。注方坐,伏兵發,斬注,傳首京師,部下潰散。注家屬屠滅,靡有孑遺。初未獲注,京師憂恐。至是,人人相慶。



 注兩目不能遠視,自言有金丹之術,可去痿弱重膇之疾。始李愬自云得效,乃移之守澄,亦神其事。由是中官視注皆憐之,卒以是售其狂謀。而守澄自貽其患,復致衣冠塗地,豈一時之沴氣歟?既籍沒其家財,得絹一百萬匹,他貨稱是。



 王涯,字廣津,太原人。父晃。涯,貞元八年進士擢第,登宏
 辭科。釋褐藍田尉。貞元二年十一月,召充翰林學士,拜右拾遺、左補闕、起居舍人,皆充內職。元和三年,為宰相李吉甫所怒,罷學士,守都官員外郎,再貶虢州司馬。五年,入為吏部員外。七年,改兵部員外郎、知制誥。九年八月,正拜舍人。十年,轉工部侍郎、知制誥,加通議大夫、清源縣開國男,學士如故。十一年十二月,加中書侍郎、同平章事。十三年八月,罷相,守兵部侍郎,尋遷吏部。



 穆宗即位,以檢校禮部尚書、梓州刺名、劍南東川節度使。其年
 十一月,吐蕃南北倚角入寇,西北邊騷動,詔兩川兵拒之。時蕃軍逼雅州,涯上疏曰:「臣當道出軍,徑入賊腹有兩路:一路從龍州清川鎮入蕃界,徑抵故松州城,是吐蕃舊置節度之所;一路從綿州威蕃柵入蕃界,徑抵棲雞城,皆吐蕃險要之地。」又曰:「臣伏見方今天下無犬吠之警,海內同覆盂之安。每蕃戎一警,則中外咸震,致陛下有旰食軫懷之憂,斯乃臣等居大官、受重寄者之深責也。雖承詔發卒,心馳寇廷,期於為國討除,使戎人芟
 剪。晝夜思忖,何補涓毫?所以淒淒愚心,願陳萬一。臣觀自古長策,昭然可徵。在於實邊兵,選良將,明斥候,廣資儲,杜其奸謀,險其走集,此立朝士大夫皆知,不獨微臣知之也,只在舉行之耳。然臣愚見所及,猶欲布露者,誠願陛下不愛金帛之費,以釣北虜之心。臨遣信臣,與之定約曰:犬戎悖亂負恩,為邊鄙患者數矣,能制而服之者,唯在北蕃。如能發兵深入,殺若干人,取若干地,則受若干之賞。開懷以示之,厚利以啗之,所以勸聳要約者
 異於他日,則匈奴之銳,可得出矣。一戰之後,西戎之力衰矣。」穆宗不能用其謀。



 長慶元年,幽、鎮復亂,王師征之,未聞克捷。涯在鎮上書論用兵曰:



 伏以幽、鎮兩州,悖亂天紀,迷亭育之厚德,肆豺虎之非心。囚系鼎臣,戕賊戎帥,毒流列郡,釁及賓僚。凡在有情,孰不扼腕?咸欲橫戈荷戟,問罪賊廷。伏以國家文德誕敷,武功繼立,遠無不服,邇無不安。矧茲二方,敢逆天理?臣竊料詔書朝下,諸鎮夕驅,以貔貅問罪之師,當猖狂失節之寇,傾山壓卵,
 決海灌熒,勢之相懸,不是過也。



 但以常山、燕郡,虞、虢相依,一時興師,恐費財力。且夫罪有輕重,事有後先,攻堅宜從易者。如聞範陽肇亂,出自一時,事非宿謀,情亦可驗。鎮州構禍,殊匪偶然,扇動屬城,以兵拒境。如此則幽、薊之眾,可示寬刑;鎮、冀之戎,必資先討。況廷湊亹茸,不席父祖之恩;成德分離,人多迫脅之勢。今以魏博思復讎之眾,昭義願盡敵之師,參之晉陽,輔以滄、易,掎角而進,易若建瓴,盡屠其城,然後北首燕路。在朝廷不為失
 信,於軍勢實得機宜。臣之愚忠,輒在於此。



 臣又聞用兵若鬥,先扼其喉。今瀛、莫、易、定,兩賊之咽喉也,誠宜假之威柄,戍以重兵。俾其死生不相知,間諜無所入,而以大軍先迫冀、趙,次下井陘,此百舉百全之勢也。臣受恩深至,無以上酬,輕冒陳聞,不勝戰越。



 洎涯疏至,盧士玫已為賊劫,陷瀛、莫州,兇勢不可遏。俄而二兇俱宥之。



 三年,入為御史大夫。敬宗即位,改戶部侍郎、兼御史大夫,充鹽鐵轉運使,俄遷禮部尚書,充職。寶歷二年,檢校尚書
 左僕射、興元尹、山南西道節度使,就加檢校司空。



 太和三年正月,入為太常卿。文宗以樂府之音,鄭衛太甚,欲聞古樂,命涯詢於舊工,取開元時雅樂,選樂童按之,名曰《雲韶樂》。樂曲成,涯與太常丞李廓、少府監庾承憲、押樂工獻於黎園亭,帝按之於會昌殿。上悅,賜涯等錦彩。



 四年正月,守吏部尚書、檢校司空,復領鹽鐵轉運使。其年九月,守左僕射,領使。奏李師道前據河南十二州,其兗、鄆、淄、青、濮州界,舊有銅鐵冶,每年額利百餘萬,自收
 復,未定稅額,請復系鹽鐵司,依建中元年九月敕例制置,從之。



 七年七月,以本官同平章事,進封代國公,食邑二千戶。八年正月,加檢校司空、門下侍郎、弘文館大學士、太清宮使。九年五月,正拜司空,仍令所司冊命,加開府儀同三司,仍兼領江南榷茶使。



 十一月二十一日,李訓事敗,文宗入內。涯與同列歸中書會食,未下箸,吏報有兵自閣門出,逢人即殺。涯等蒼惶步出,至永昌裏茶肆,為禁兵所擒,並其家屬奴婢,皆系於獄。仇士良鞫涯
 反狀,涯實不知其故。械縛既急,搒笞不勝其酷,乃令手書反狀,自誣與訓同謀。獄具,左軍兵馬三百人領涯與王璠、羅立言,右軍兵馬三百人領賈餗、舒元輿、李孝本,先赴郊廟,徇兩市,乃腰斬於子城西南隅獨柳樹下。涯以榷茶事,百姓怨恨詬罵之,投瓦礫以擊之。中書房吏焦寓、焦璇、臺吏李楚等十餘人,吏卒爭取殺之,籍沒其家。涯子工部郎中、集賢殿學士孟賢,太堂博士仲翔,其餘稚小妻女,連襟系頸,送入兩軍,無少長盡誅之。自涯
 已下十一家,資貨悉為軍卒所分。涯積家財鉅萬計,兩軍士卒及市人亂取之,竟日不盡。



 涯博學好古,能為文,以辭藝登科。踐揚清峻,而貪權固寵,不遠邪佞之流,以至赤族。涯家書數萬卷,侔於秘府。前代法書名畫,人所保惜者,以厚貨致之;不受貨者,即以官爵致之。厚為垣竅,而藏之復壁。至是,人破其垣取之,或剔取函奩金寶之飾與其玉軸而棄之。



 涯之死也,人以為冤。昭義節度使劉從諫三上章,求示涯等三相罪名,仇士良頗懷憂
 恐。初宦官縱毒,凌藉南司。及從諫奏論,兇焰稍息,人士賴之。



 王璠,字魯玉。父礎,進士,文辭知名。元和五年,擢進士第,登宏辭科。風儀修飾,操履甚堅,累闢諸侯府。元和中,入朝為監察御史,再遷起居舍人,副鄭覃宣慰於鎮州。長慶中,累歷員外郎。十四年,以職方郎中知制誥。寶歷元年二月,轉御史中丞。



 時李逢吉為宰相,與璠親厚,故自郎官掌誥,便拜中丞。恃逢吉之勢,稍橫。嘗與左僕射李
 絳相遇於街,交車而不避。絳上疏論之曰:「左、右僕射,師長庶僚,開元中名之丞相。其後雖去三事機務,猶總百司之權。表狀之中,不署其姓。尚書已下,每月合衙。上日百僚列班,宰相居上,中丞御史列位於廷。禮儀之崇,中外特異。所以自武德、貞觀已來,聖君賢臣,布政除弊,不革此禮,謂為合宜。茍有不安,尋亦合廢。近年緣有才不當位,恩加特拜者,遂從權便,不用舊儀。酌於群情,事實未當。今或有僕射初除,就中丞院門相看,即與欲參何
 殊。或中丞新授,亦無見僕射處。及參賀處,或僕射先至,中丞後來,憲度乖宜,尊卑倒置。倘人才忝位,自合別授賢良;若朝命守官,豈得有虧法制?伏望下百僚詳定事體,使永可遵行。」敕旨令兩省詳議。兩省奏曰:「元和中,伊慎忝居師長之位,太常博士韋謙削去舊儀。今李絳所論,於禮甚當。」逢吉素惡絳之直,天子雖許行舊儀,中書竟無處分,乃罷璠中丞,遷工部侍郎。尋罷絳僕射,以太子少師分司東都。其弄權怙寵如此。



 璠二年七月出為
 河南尹。太和二年,以本官權知東都選。十月,轉尚書右丞,敕選畢入朝。三年,改吏部侍郎。四年七月,拜京兆尹、兼御史大夫。十二月,遷左丞,判太常卿事。六年八月,檢校禮部尚書、潤州刺史、浙西觀察使。



 八年,李訓得幸,累薦於上。召還,復拜右丞。璠以逢吉故吏,自是傾心於訓,權幸傾朝。九年五月,遷戶部尚書、判度支。謝日,召對浴堂,錫之錦彩。其年十一月,李訓將誅內官,令璠召募豪俠,乃授太原節度使,托以募爪牙為名。訓敗之日,璠歸
 長興里第。是夜為禁軍所捕,舉家下獄;斬璠於獨柳樹,家無少長皆死。



 璠子遐休,直弘文館。李訓舉事之日,遐休於館中禮上,同職駕部郎中令狐定等五六人送之,是日悉為亂兵所執。定以兄楚為僕射,軍士釋之,獨執遐休誅之。



 初璠在浙西,繕城壕。役人掘得方石,上有十二字,云:「山有石,石有玉,玉有瑕,瑕即休。」璠視莫知其旨,京口老人講之曰:「此石非尚書之吉兆也。尚書祖名崟,崟生礎,是山有石也。礎生尚書,是石有玉也。尚書之子名
 遐休,休,絕也。此非吉徵。」果赤族。



 賈餗,字子美,河南人。祖渭。大父寧。餗進士擢第,又登制策甲科,文史兼美,四遷至考功員外郎。長慶初,策召賢良,選當時名士考策,餗與白居易俱為考策官,選文人以為公。尋以本官知制誥,遷庫部郎中,充職。四年,為張又新所構,出為常州刺史。太和初,入為太常少卿。二年,以本官知制誥。三年七月,拜中書舍人。四年九月,權知禮部貢舉。五年,榜出後,正拜禮部侍郎。凡典禮闈三歲,所
 選士七十五人,得其名人多至公卿者。七年五月,轉兵部侍郎。八年十一月,遷京兆尹、兼御史大夫。九年四月,檢校禮部尚書、潤州刺史、浙西觀察使。制出未行,拜中書侍郎、同平章事,進金紫階,封姑臧男,食邑三百戶。未幾,加集賢殿學士,監修國史。



 其年十一月,李訓事發,兵交殿廷,禁軍肆掠。餗易服步行出內,潛身人間。翌日,自投神策軍,與王涯等皆族誅。餗雖中立自持,然不能以身犯難,排斥奸纖,脂韋其間,遂至覆族。逢時多僻,死非
 其罪,世多冤之。



 舒元輿者,江州人。元和八年登進士第,釋褐諸府從事。太和初,入朝為監察,轉侍御史。



 初,天寶中,玄宗祀九宮壇,次郊壇行事,御署祝板。元輿為監察,監祭事,以為太重,奏曰:「臣伏見祀九宮貴神祝板九片,陛下親署御名,及稱臣於九宮之神。臣伏以天子之尊,除祭天地宗廟之外,無合稱臣者。王者父天母地,兄日姊月。而貴神以九宮為目,是宜分方而守其位。臣數其名號,太一、天一、
 招搖、軒轅、咸池、青龍、太陰、天符、攝提也。此九神,於天地猶子男也,於日月猶侯伯也。陛下為天子,豈可反臣於天之子男耶?臣竊以為過。縱陰陽者流言其合祀,則陛下當合稱『皇帝遣某官致祭於九宮之神』,不宜稱臣與名。臣雖愚瞽;未知其可。乞下禮官詳議。」從之。尋轉刑部員外郎。



 元輿自負奇才,銳於進取,乃進所業文章,乞試效用,宰執謂其躁競。五年八月,改授著作郎,分司東都。時李訓丁母憂在洛,與元輿性俱詭激,乘險蹈利,相得
 甚歡。及訓為文宗寵遇,復召為尚書郎。九年,以右司郎中知臺雜。七月,權知中丞事。九年,拜御史中丞,兼判刑部侍郎。是月,以本官同平章事,與訓同知政事。而深謀詭算,熒惑主聽,皆生於二兇也。訓竊發之日,兵自內出。元輿易服單馬出安化門,為追騎所擒,送左軍族誅之。



 郭行餘者,亦登進士第。太和初,累官至楚州刺史。五年,移刺汝州,兼御史中丞。九月,入為大理卿。李訓在東都時,與行餘親善,行餘數相餉遺,至是用為九列,十一月,
 訓欲竊發,令其募兵,乃授邠寧節度使。訓敗,族誅。



 羅立言者,父名歡。貞元末,登進士第。寶歷初,檢校主客員外郎,為鹽鐵河陰院官。二年,坐糴米不實,計贓一萬九千貫,鹽鐵使惜其吏能,定罪止削所兼侍御史。太和中,為司農少卿,主太倉出納物,以貨厚賂鄭注,李訓亦重之。訓將竊發,須兵集事,以京兆府多吏卒,用立言為京兆少尹,知府事。訓敗日,族誅。



 長安縣令孟琯貶硤州長史,萬年縣令姚中立朗州長史。以兩縣捕賊官受立
 言指使故也。初立言集兩縣吏卒,萬年捕賊官鄭洪懼禍托疾,既而詐死,令家人喪服聚哭。姚中立陰知其故,恐以詐聞,不免其累,乃以狀告洪之詐。仇士良拘洪入軍,洪銜中立之告,謂士良曰:「追集所由,皆因縣令處分,予何罪也。」故中立坐貶,洪免死。



 李孝本者,宗室之子也。累官至刑部郎中,而依於訓、注以求進。舒元輿作相,訓用孝本知臺雜,權知中丞事,最預訓謀。竊發之日,孝本從人殺內官十餘人於殿廷。知
 事不濟,單騎走投鄭注。至咸陽西原,為追騎所捕,族誅之。坐訓、注而族者,凡十一家,人以為冤。



 史臣曰:王者之政以德,霸者之政以權。古先後王,率由茲道,而遂能息人靖亂,垂統作則者。如梓人共柯而殊工,良奕同枰而獨勝,蓋在得其術,則事無後艱。昭獻皇帝端冕深帷,憤其廝養,欲鏟宮居之弊,載澄刑政之源。當宜禮一代正人,訪先朝耆德,修文教而厚風俗,設武備以服要荒。俾西被東漸,皆陶於景化;柔祗蒼昊,必降
 於闕祥,自然懷德以寧,無思不服。況區區宦者,獨能悖化哉?故豎刁、易牙,不廢齊桓之霸;韓嫣、籍孺,何妨漢帝之明。蓋有管仲、亞夫之賢,屬之以大政故也。此二君者,制御閽寺,得其道也。而昭獻忽君人之大體,惑纖狡之庸儒。雖終日橫經,連篇屬思,但得好文之譽,庸非致治之先。且李訓者,狙詐百端,陰險萬狀,背守澄而勸鴆,出鄭注以擅權。只如盡隕四星,兼權八校,小人方寸,即又難知。但慮為蚤虱而採溪蓀,翻獲螾蜓之患也。嗚呼明
 主!夫何不思,遽致血濺黃門,兵交青瑣。茍無籓後之勢,黃屋危哉!涯、餗綽有士風,晚為利喪,致身鬼蜮之伍,何逃瞰室之災。非天不仁,子失道也!



 贊曰:奭、旦興周,斯、高亡秦。禍福非天,治亂由人。訓、注奸偽,血頹象魏。非時乏賢,君迷倒置。



\end{pinyinscope}