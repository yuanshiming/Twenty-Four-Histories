\article{卷四十七 志第二十七 經籍下}

\begin{pinyinscope}

 丙部子錄,十七家,七百五十三部,書一萬五千六百三十七卷。



 儒家類一道家類二法家類三名家類四



 墨家類五縱橫家類六雜家類七農家類八



 小說類九天文類十歷算類十一兵書類十二



 五行類十三雜藝術類十四事類十五經脈類十六



 醫術類十七



 《曾子》二卷曾參撰。



 《晏子春秋》七卷晏嬰撰。



 《子思子》八卷孔伋撰。



 《公孫尼子》一卷公孫尼撰。



 《孟子》十四卷孟軻撰,趙岐注。



 又七卷
 劉熙注。



 又七卷鄭玄注。



 又七卷綦毋邃注。



 《孫卿子》十二卷荀況撰。



 《董子》二卷董無心撰。



 《魯連子》五卷魯仲連撰。



 《新語》二卷陸賈撰。



 《賈子》九卷賈誼撰。



 《鹽鐵論》十卷桓寬撰。



 《新序》三十卷劉向撰。



 《說苑》三十卷劉向撰。



 《楊子法言》六卷楊雄撰。



 又十卷宋衷注。



 又十三卷李軌注。



 《楊子太玄經》十二卷楊雄撰,陸績注。



 又十四卷
 虞翻注。



 又十二卷範望注。



 又一十卷蔡文邵注。



 《桓子新論》十七卷桓譚撰。



 《潛夫論》十卷王符撰。



 《申鑒》五卷荀悅撰。



 《魏子》三卷魏朗注。



 《典論》五卷魏文帝撰。



 《徐氏中論》六卷徐幹撰。



 《去伐論集》三卷王粲撰。



 《杜氏體論》四卷杜恕撰。



 《顧子新語》五卷顧譚撰。



 《通語》十卷文禮撰,殷興續。



 《集誡》二卷諸葛亮撰。



 《典訓》十卷陸景撰。



 《譙子法訓》八卷譙周撰。



 《古今通論》三卷王嬰撰。



 《周生烈子》五卷周生烈志。



 《譙子五教》五卷譙周撰。



 《袁子正論》二十卷袁準撰。



 《袁子正書》二十五卷袁準撰。



 《孫氏成敗志》三卷孫毓撰。



 《新論》十卷
 夏侯湛撰。



 《物理論》十六卷楊泉撰。



 《太元經》十四卷楊泉撰,劉緝注。



 《新論》十卷華譚撰。



 《志林新書》二十卷虞喜撰。



 《後林新書》十卷虞喜撰。



 《顧子義訓》十卷顧夷撰。



 《清化經》十卷蔡洪撰。



 《正言》十卷干寶撰。



 《要覽》五卷呂竦撰。



 《立言》十卷干寶撰。



 《正覽》六卷周舍撰。



 《缺文》
 十卷陸澄撰。



 《魯史欹器圖》一卷劉徽撰。



 《誡林》三卷綦毋氏撰。



 《家訓》七卷顏之推撰。



 《典言》四卷李若等撰。



 《墳典》三十卷盧辯撰。



 《中說》五卷王通撰。



 《讀書記》三十二卷王邵撰。



 《正訓》二十卷辛德源志。



 《太宗序志》一卷太宗撰。



 《帝範》四卷太宗撰,賈行注。



 《天訓》四卷高宗天皇大帝撰。



 《紫樞要錄》十卷大聖天后撰。



 《青宮
 記要》三十卷天后撰。



 《少陽正範》三十卷天后撰。



 《臣軌》二卷天后撰。



 《百僚新誡》四卷天后撰。



 《春宮要錄》十卷章懷太子撰。



 《君臣相發起事》三卷章懷太子撰。



 《修身要錄》十卷章懷太子撰。



 《百里昌言》二卷王滂撰。



 《崔子至言》六卷崔靈童撰。



 《平臺百一寓言》三卷張大素撰。



 《女誡》一卷曹大家撰。



 《內訓》二十卷辛德源、王邵等撰。



 《女則要錄》十卷文德皇后撰。



 《鳳樓新誡》二
 十卷張後撰。



 右儒家二十八部,凡七百七十六卷。



 《老子》二卷老子撰。



 《老子》二卷河上公注。



 《老子章句》二卷安丘望之撰。



 《老子道德經指趣》四卷安丘望之撰。



 《老子》二卷湘注。



 《玄言新記道德》二卷王弼注。



 《老
 子》二卷鐘會注。



 《老子》二卷羊祜注。



 《老子》二卷程韶集注。



 《老子》二卷王尚注。



 《老子》二卷蜀才注。



 《老子》二卷孫登注。



 《老子》二卷袁真注。



 《老子》二卷張憑注。



 《老子》二卷鳩摩羅什注。



 《老子》二卷釋惠嚴注。



 《老子》四卷陶弘景注。



 《老子道德經品》四卷梁曠注。



 《老子》二卷樹鐘山注。



 《老子》二卷傅奕注。



 《老子》二卷楊上善注。



 《老子集注》四卷張道相集注。



 《老子》二卷闢閭仁住注。



 《老子》二卷成玄英注。



 《老子》二卷李允願注。



 《老子》二卷陳嗣古注。



 《老子》二卷釋義盈注。



 《老子道德經集解》四卷任真子注。



 《
 老子節解》二卷"



 《老子指歸》十四卷嚴遵志。



 《老子指歸》十三卷馮廓撰。



 《老子道德經序訣》二卷葛洪撰。



 《老子道德簡要義》五卷玄景先生注。



 《太上玄元皇帝道德經》二卷楊上器撰。



 《太上老君玄元皇帝聖紀》十卷尹父操撰。



 《老子章門》一卷



 《老子玄旨》八卷韓莊撰。



 《老子玄譜》一卷
 劉道人撰。



 《老子道德論》二卷何晏撰



 《老子指例略》二卷



 《老子道德經義疏》四卷顧歡撰。



 《老子解釋》四卷羊祜撰。



 《老子義疏理綱》一卷



 《老子講疏》六卷梁武帝撰。



 《老子私記》十卷梁簡文帝撰。



 《老子講疏》四卷



 《老子義疏》四卷孟智周撰。



 《老子述義》十卷賈大
 隱撰。



 《老子道德指略論》二卷楊上善撰。



 《道德經》三卷



 《略論》三卷楊上善撰。



 《老子西升經》一卷



 《老子黃庭經》一卷



 《老子探真經》一卷



 《老君科律》一卷



 《老子宣時誡》一卷



 《
 老子入室經》一卷



 《老子華蓋觀天訣》一卷



 《老子消水經》一卷



 《老子神策百二十條經》一卷



 《莊子》十卷崔譔注。



 又十卷郭象注。



 又二十卷向秀注。



 又二十一卷司馬彪注。



 《莊子集解》二十卷李頤集解。



 又二十卷王玄古撰。



 《莊子》十卷楊上善撰。



 《莊子講疏》三十卷梁簡文撰。



 《莊子疏》七卷



 《
 南華仙人莊子論》三十卷梁曠撰。



 《釋莊子論》二卷李充撰。



 《南華真人道德論》三卷



 《莊子疏》十卷王穆撰。



 《莊子音》一卷王穆撰。



 《莊子文句義》二十卷陸德明撰。



 《莊子古今正義》十卷馮廓撰。



 《莊子疏》十二卷成玄英撰。



 《文子》十二卷



 《鶡冠子》三卷鶡冠子撰。



 《列子》八卷列禦寇撰,張湛注。



 《廣成子》十二卷商洛公撰。



 《任子道論》十卷任嘏撰。



 《渾輿經》一卷姖威撰。



 《唐子》十卷唐滂撰。



 《
 蘇子》七卷蘇彥撰。



 《宣子》二卷宣聘撰。



 《陸子》十卷陸云撰。



 《抱樸子內篇》二十卷葛洪撰。



 《孫子》十二卷孫綽撰。



 《顧道士論》二卷顧穀撰。



 《幽求子》三十卷杜夷撰。



 《符子》三十卷符朗撰。



 《賀子》十卷賀道養撰。



 《真誥》十卷陶弘景撰。



 《無名子》一卷張太衡撰。



 《養生要集》十卷張湛撰。



 《無上秘要》七十二卷



 《玄書通義》十卷張機撰。



 《道要》三十卷



 《登真隱訣》二十五卷陶弘景撰。



 《
 同光子》八卷劉無待撰,侯儼注。



 《牟子》二卷牟融撰。



 《凈住子》二十卷蕭子良撰,王融頌。



 《統略凈住子》二卷釋道宣撰。



 《法苑》十五卷釋僧祐撰。



 《內典博要》三十卷虞孝景撰。



 《真言要集》十卷釋賢明撰。



 《歷代三寶記》三卷



 《修名羅法門》二十卷郭瑜撰。



 《集古今佛道論衡》四卷釋道宣撰。



 《六趣論》六卷楊上善撰。



 《十門辯惑論》二卷釋復禮志。



 《經論纂要》十卷駱子義撰。



 《通惑決疑錄》二卷釋道宜撰。



 《夷夏論》二卷顧歡撰。



 《
 笑道論》三卷甄鸞撰。



 《齊三教論》七卷衛元嵩撰。



 《辯正論》八卷《釋法琳撰》。



 《破邪論》三卷釋法琳撰。



 《三教詮衡》十卷楊上善撰。



 《甄正論》三卷杜乂撰。



 《心鏡論》十卷李思慎撰。



 《崇正論》六卷釋彥琮撰。



 右道家一百二十五部,老子六十一家,莊子十七家,道釋諸說四十七家,凡九百六十卷。



 《管子》十八卷管夷吾撰。



 《商子》五卷商鞅撰。



 《慎子》十卷慎到撰,滕輔注。



 《申子》三卷申不害撰。



 《
 韓子》二十卷韓非撰。



 《晁氏新書》三卷晁錯撰。



 《崔氏政論》五卷崔實撰。



 《劉氏法言》十卷劉邵撰。



 《劉氏正論》五卷劉撰。



 《阮子正論》五卷阮武撰。



 《桓氏代要論》十卷桓範撰。



 《陳子要言》十四卷陳融撰。



 《治道集》十卷李文博撰。



 《春秋決獄》十卷董仲舒撰。



 《五經析疑》三十卷邯鄲綽撰。



 右法家十五部,凡一百五十八卷。



 《鄧析子》一卷鄧析撰。



 《尹文子》二卷尹文子撰。



 《
 公孫龍子》三卷公孫龍撰。



 又一卷賈大隱注。



 又一卷陳嗣古注。



 《人物志》三卷劉邵撰。



 又三卷劉邵撰,劉炳注。



 《士緯》十卷姚信撰。



 《士操》一卷魏文帝撰。



 《九州人士論》一卷盧毓撰。



 《兼名苑》十卷釋遠年撰。



 《辯名苑》十卷範謐撰。



 右名家十二部,凡五十六卷。



 《墨子》十五卷墨翟撰。



 《胡非子》一卷胡非子撰。



 右墨家二部,凡一十六卷。



 《
 鬼谷子》二卷蘇秦撰。



 又三卷樂臺撰。



 又三卷尹知章注。



 《補闕子》十卷梁元帝撰。



 右縱橫家四部,凡十八卷。



 《尸子》二十卷尸佼撰。



 《尉繚子》六卷尉繚子撰。



 《呂氏春秋》二十六卷呂不韋撰。



 《淮南商詁》二十一卷劉安撰。



 《淮南子注解》二十一卷高誘撰。



 《淮南鴻烈音》二卷高誘撰。



 《三將軍論》一卷嚴尤撰。



 《論衡》三十卷王充撰。



 《風俗通義》三十卷應劭撰。



 《仲長子昌言》十卷仲長統
 撰。



 《萬機論》八卷蔣濟撰。



 《篤論》四卷杜恕撰。



 《芻蕘論》五卷鐘會撰。



 《傅子》一百二十卷傅玄撰。



 《默記》三卷張儼撰。



 《新言》五卷裴玄撰。



 《新義》十八卷劉欽撰。



 《秦子》三卷秦菁撰。



 《誓論》三十卷張儼撰。



 《說林》五卷孔衍撰。



 又二十卷張大素撰。



 《抱樸子外篇》五十卷葛洪撰。



 《時務論》十二卷楊偉撰。



 《古今善言》三十卷範泰撰。



 《記聞》三卷徐益壽撰。



 《何子》五卷何楷撰。



 《劉
 子》十卷劉勰撰。



 《金樓子》十卷梁元帝撰。



 《語麗》十卷硃儋遠撰。



 《袖中記》一卷



 《要覽》三卷陸士衡撰。



 《古今注》五卷崔豹撰。



 《採璧記》三卷庾肩吾撰。



 《新略》十卷韋道孫撰。



 《名數》十卷徐陵撰。



 《典墳數》十卷範謐撰。



 《荊楚歲時記》十卷宗懍撰。



 又二卷杜公瞻撰。



 《玉燭寶典》十二卷。杜臺卿撰。



 《四時錄》十二卷王氏撰。



 《物始》十卷謝昊撰。



 《事始》三卷劉孝孫撰。



 《
 古今辯作錄》三卷



 《文章始》一卷任昉撰,張績補。



 《續文章始》一卷姚察撰。



 《戚苑纂要》十卷劉揚名撰。



 《張掖郡玄石圖》一卷孟眾撰。



 《瑞應圖記》二卷孫柔之撰。



 《張掖郡玄石圖》一卷高堂隆撰。



 《瑞應圖贊》三卷熊理撰。



 《祥瑞圖》十卷



 《符瑞圖》十卷顧野王撰。



 《皇隋靈感志》十卷王邵撰。



 《皇隋瑞文》十四卷許善心撰。



 《諫林》十卷何望之撰。



 《善諫》二卷虞通之撰。



 《諫事》五卷魏徵撰。



 《諫苑》三十卷於志寧撰。



 《
 子林》二十卷孟儀撰。



 《子鈔》三十卷沈約撰。



 又三十卷庾仲容撰。



 《子林》三十卷薛克構撰。



 《述正論》十三卷陸澄撰。



 《博覽》十五卷



 《文府》七卷徐陵撰,宗道寧注。



 《翰墨林》十卷



 《群書理要》五十卷魏徵撰。



 《四部言心》十卷劉守敬撰。



 《麟閣詞英》六十卷高宗敕撰。



 右雜家七十一部,凡九百八十二卷。



 《氾勝之書》二卷氾勝之撰。



 《四人月令》一卷崔實撰。



 《
 齊人要術》十卷賈思勰撰。



 《竹譜》一卷戴凱之撰。



 《錢譜》一卷顧烜撰。



 《禁苑實錄》一卷



 《種植法》七十七卷諸葛穎撰。



 《兆人本業》三卷天后撰。



 《相鶴經》一卷浮丘公撰。



 《鷙擊錄》二十卷堯須跂撰。



 《鷹經》一卷



 《蠶經》一卷。



 《相馬經》一卷伯樂撰。



 又二卷



 又二卷徐成等撰。



 《相馬經》六十卷諸葛穎等撰。



 《相牛經》一卷寧戚撰。



 《相貝經》一卷



 《
 養魚經》一卷範蠡撰。



 右農家二十部,凡一百九十二卷。



 《鬻子》一卷鬻熊撰。



 《燕丹子》三卷燕太子撰。



 《笑林》三卷邯鄲淳撰。



 《博物志》十卷張華撰。



 《郭子》三卷郭澄之撰。賈泉注。



 《世說》八卷劉義慶撰。



 《續世說》十卷劉孝標撰。



 《小說》十卷劉義慶撰。



 《小說》十卷殷蕓撰。



 《釋俗語》八卷劉霽撰。



 《辨林》二十卷蕭賁撰。



 《酒孝經》一卷劉炫定撰。



 《
 座右方》三卷庾元威撰。



 《啟顏錄》十卷侯白撰。



 右小說家十三部,凡九十卷。



 《周髀》一卷趙嬰注。



 又一卷甄鸞注。



 又二卷李淳風撰。



 《靈憲圖》一卷張衡撰。



 《渾天儀》一卷張衡撰。



 《渾天象注》一卷王蕃撰。



 《昕天論》一卷姚信撰。



 《石氏星經簿贊》一卷石申甫撰。



 《安天論》一卷虞喜撰。



 《甘氏四七法》一卷甘德撰。



 《論二十八宿度數》一卷



 《荊州星占》二卷劉表撰。



 又二十卷劉睿撰。



 《天文集占》七卷陳卓撰。



 《四方星占》一卷陳卓撰。



 《五星占》三卷陳卓撰。



 《天文集占》三卷



 《天文錄》三十卷祖恆之撰。



 《天文橫圖》一卷高文洪撰。



 《天文雜占》一卷吳雲撰。



 《星占》三十三卷孫僧化撰。



 《十二次二十八宿星占》十二卷史崇撰。



 《乙巳占》十卷李淳風撰。



 《靈臺秘苑》一百二十卷庾季才撰。



 《玄機內事》七卷逢行珪撰。



 右天文二十六家,凡二百六十卷。



 《三統歷》一卷劉歆撰。



 《乾象歷》三卷闞澤注,闞洋撰。



 《魏景初歷》三卷楊褘撰。



 《四分歷》一卷



 《乾象歷術》三卷劉洪撰



 《乾象歷》三卷



 《宋元嘉歷》二卷何承天撰。



 《梁大同歷》一卷虞廣刂撰。



 《後魏永安歷》一卷孫僧化撰。



 《後魏武定歷》一卷



 《北齊天保歷》一卷宋景業撰。



 《周天象歷》二卷王琛撰。



 《隋開皇歷》一卷劉孝孫撰。



 又一卷李德林撰。



 《
 隋大業歷》一卷張胄玄撰。



 《皇極歷》劉焯撰。



 又一卷李淳風撰。



 《河西壬辰元歷》一卷趙匪又撰。



 《河西甲寅元歷》一卷李淳風撰。



 《大唐麟德歷》一卷



 《大唐光宅歷草》十卷



 《周甲子元歷》一卷



 《齊甲子歷》一卷



 《大唐甲子元辰歷》一卷瞿曇撰。



 《大唐戊寅歷》一卷



 《陳七曜歷》五卷吳伯善撰。



 《七曜本起歷》二卷



 《七曜歷算》二卷甄鸞撰。



 《七曜雜術》二卷劉孝孫撰。



 《
 七曜歷疏》二卷張胄玄撰。



 《歷疏》一卷崔浩撰。



 《歷術》一卷甄鸞撰。



 《玄歷術》一卷張胄玄撰。



 《刻漏經》一卷何承天撰。



 又一卷硃史撰。



 又一卷宋景撰。



 《大唐刻漏經》一卷。



 《九章算經》一卷徐岳撰。



 《九章重差》一卷劉向撰。



 《九章重差圖》一卷劉徽撰。



 《九章算經》九卷甄鸞撰。



 《九章雜算文》二卷劉祐撰。



 《九章術疏》九卷宋泉之撰。



 《五曹算經》五卷甄鸞撰。



 《孫子算經》三卷甄鸞撰。



 《
 海島算經》一卷甄鸞撰注。



 《張丘建算經》一卷(甄鸞撰。



 《夏侯陽算經》三卷甄鸞注。



 《數術記遺》一卷徐岳撰,甄鸞注。



 《三等數》一卷董泉撰,甄鸞注。



 《算經要用百法》一卷徐岳撰。



 《綴術》五卷祖沖之撰,李淳風注。



 《五曹算經》三卷甄鸞撰。



 《七經算術通義》七卷陰景愉撰。



 《緝古算術》四卷王孝通撰,李淳風注。



 《算經表序》一卷



 右歷算五十八部,凡一百六十七卷。



 《黃帝問玄女法》三卷玄女撰。



 《太公陰謀》三卷



 《
 太公金匱》二卷



 《太公六韜》六卷



 《司馬法》三卷田穰苴撰。



 《孫子兵法》十三卷孫武撰,魏武帝注。



 又二卷孟氏解。



 又二卷沈友注。



 《黃石公三略》三卷。



 《三略訓》三卷



 《張良經》一卷張良撰。



 《雜兵法》二十四卷



 《兵法捷要》七卷魏武帝撰。



 《兵法要略》十卷魏文帝撰。



 《兵記》十二卷司馬彪撰。



 《兵林》六卷孔衍撰。



 《玉韜》十卷梁元帝撰。



 《
 真人水鏡》十卷陶弘景撰。



 《握鏡》一卷陶弘景撰。



 《兵書要略》十卷宇文憲撰。



 《太一兵法》一卷



 《太公陰謀三十六用》一卷



 《伍子公兵法》一卷



 《吳孫子三十二壘經》一卷



 《玉帳經》一卷



 《黃石公陰謀乘斗魁剛行軍秘》一卷



 《武德圖五兵八陣法要》一卷



 《三陰圖》一卷



 《黃帝太公三宮法要訣》一卷



 《
 張氏七篇》七卷張良撰。



 《承神兵書》八卷



 《兵機》十五卷



 《兵書要略》一卷



 《新授兵書》三十卷隋高祖撰。



 《六軍鏡》三卷李靖撰。



 《用兵撮要》二卷



 《兵春秋》一卷



 《許子新書軍勝》十卷



 《金海》四十七卷蕭吉撰。



 《王佐秘珠》五卷樂產撰。



 《金韜》十卷劉祐撰。



 《懸鏡》十卷李淳風撰。



 《龍武玄兵圖》二卷解忠鯁饌。



 《臨戎孝經》二卷員半千撰。



 右兵書四十五部,凡二百八十九卷。



 《焦氏周易林》十六卷焦贛撰。



 《京氏周易四時候》二卷



 《京氏周易飛候》六卷



 《京氏周易混沌》四卷



 《京氏周易錯卦》八卷京房撰。



 《費氏周易林》二卷費直撰。



 《崔氏周易林》十六卷



 《許氏周易雜占》七卷許峻撰。



 《周易參同契》二卷魏伯陽撰。



 《周易五相類》一卷魏伯陽撰。



 《周易林》四卷管輅撰。



 《周易雜占》八卷尚廣撰。



 《徐氏周易筮占》二十四卷徐苗
 撰。



 《周易立成占》六卷



 《武氏周易雜占》八卷武氏撰。



 《周易集林》十二卷伏曼容撰。



 又一卷伏氏撰。



 《連山》三十卷梁元帝撰。



 《易林》十四卷



 《新易林占》三卷杜氏撰。



 《周易雜占筮決文》二卷梁運撰。



 《周易新林》一卷



 《周易林》七卷張滿撰。



 《易律歷》一卷



 《周易服藥法》一卷



 《周易洞林解》三卷郭璞撰。



 《洞林》三卷梁元帝撰。



 《易三備》三卷



 又一卷



 《易髓》一卷



 《易腦》一卷郭氏撰。



 《孝經元辰》二卷



 《推元辰厄命》一卷



 《元辰章》三卷



 《六甲周天歷》一卷孫僧化作。



 《風角要候》一卷翼奉撰。



 《風角六情訣》一卷王琛撰。



 《風角》十卷



 《風角鳥情》二卷劉孝恭撰



 《鳥情占》一卷



 《鳥情逆占》一卷管輅撰。



 《九宮經解》二卷



 《九宮行棋經》三卷鄭玄撰。



 《九宮行棋立成》一卷王琛撰。



 逆刺三卷京房撰。



 《婚嫁書》二卷



 《推產婦何時產法》一卷王琛撰。



 《產圖》一卷崔知悌撰。



 《登壇經》一卷



 《太一大游歷》二卷



 《大游太一歷》一卷



 《曜靈經》一卷



 《七政歷》一卷



 《六壬歷》一卷



 《靈寶登圖》一卷



 《推二十四氣歷》一卷



 《太一歷》一卷



 《式經》一卷宋琨撰。



 《九旗飛變》一卷鄭玄撰,李淳風注。



 《太史公萬歲歷》一卷司馬談撰。



 《
 萬歲歷祠》二卷



 《千歲歷祠》二卷任氏撰。



 《黃帝飛鳥歷》一卷張衡撰。



 《太乙飛鳥歷》一卷



 《堪輿歷注》二卷



 《黃帝四序堪輿》二卷殷紹撰



 《遁甲經》一卷



 《遁甲文》一卷伍子胥撰。



 《遁甲囊中經》一卷



 《三元遁甲圖》三卷葛洪撰。



 《遁甲萬一訣》三卷



 《遁甲立成圖》二卷



 《遁甲立成法》三卷



 《遁甲九宮八門圖》一卷



 《遁甲開山圖》一卷王琛撰。



 又二卷榮氏撰。



 《
 白澤圖》一卷



 《武王須臾》二卷



 《師曠占書》一卷



 《東方朔占書》一卷



 《範子問計然》十五卷範蠡問,計然答。



 《淮南王萬畢術》一卷劉安撰。



 《神樞靈轄》十卷樂產撰。



 《祿命書》二十卷劉孝恭撰。



 又二卷王琛撰。



 《五行記》五卷蕭吉撰。



 《五姓宅經》二卷



 《陰陽書》五十卷呂才撰。



 《青烏子》三卷



 《葬經》八卷



 又十卷



 又二卷蕭吉撰。



 《
 葬書地脈經》一卷



 《墓書五陰》一卷



 《雜墓圖》一卷



 《墓圖立成》一卷



 《六甲塚名雜忌要訣》二卷



 《五姓墓圖要訣》五卷孫氏撰。



 《壇中伏尸》一卷



 《玄女彈五音法相塚經》一卷胡君撰



 《新撰陰陽書》三十卷王粲撰。



 《龜經》三卷柳彥詢撰。



 又一卷劉寶真撰。



 又一卷王弘禮撰



 又一卷莊道名撰。



 又一卷孫思邈撰。



 《
 百怪書》一卷



 《祠灶經》一卷



 《解文》一卷



 《占夢書》二卷



 又三卷周宣撰。



 《玄悟經》三卷李淳風撰。



 右五行一百一十三部,凡四百八十五卷。



 《投壺經》一卷郝沖、虞譚法撰。



 《大小博法》二卷



 《皇博經》一卷魏文帝撰



 《大博經行棋戲法》二卷



 《小博經》一卷鮑宏撰。



 《博塞經》一卷鮑宏撰。



 《二儀簿經》一卷隋煬帝撰。



 《大博經》二卷呂才撰。



 《
 棋勢》六卷



 《棋品》五卷範汪等注。



 《圍棋後九品序錄》一卷



 《竹苑仙棋圖》一卷



 《棋評》一卷梁武帝撰。



 《象經》一卷周武帝撰。



 又一卷何妥撰。



 又一卷王裕撰。



 《今古術藝》十五卷



 右雜藝術一十八部,凡四十四卷。



 《皇覽》一百二十二卷何承天撰。



 又八十四卷徐爰並合。



 《類苑》一百二十卷劉孝標撰。



 《壽光書苑》二百卷劉香撰。



 《
 華林編略》六百卷徐勉撰。



 《修文殿御覽》三百六十卷



 《長洲玉鏡》一百三十八卷虞綽等撰。



 《藝文類聚》一百卷歐陽詢等撰。



 《北堂書抄》一百七十三卷虞世南撰。



 《要錄》六十卷



 《書圖泉海》七十卷張氏撰。



 《檢事書》一百六十卷



 《帝王要覽》二十卷



 《玉藻瓊林》一百卷孟利貞撰。



 《玄覽》一百卷天后撰。



 《累璧》四百卷許敬宗撰。



 《
 碧玉芳林》四百五十卷孟利貞撰。



 《策府》五百八十二卷張大素撰。



 《玄門寶海》一百二十卷諸葛穎撰。



 《文思博要》並目一千二百一十二卷張大素撰。



 《三教珠英》並目一千三百一十三卷張昌宗等撰。



 右類事二十二部,凡七千八十四卷。



 《黃帝三部針經》十三卷皇甫謐撰。



 《黃帝八十一難經》一卷秦越人撰。



 《赤烏神針經》一卷張子存撰。



 《黃帝明堂經》三卷



 《黃帝針灸經》十二卷



 《
 明堂圖》三卷秦承祖撰。



 《龍銜素針經並孔穴蝦蟆圖》三卷



 《黃帝素問》八卷



 《黃帝內經明堂》十三卷



 《黃帝雜注針經》一卷



 《黃帝十二經脈明堂五藏圖》一卷



 《黃帝十二經明堂偃側人圖》十二卷



 《黃帝針經》十卷



 《黃帝明堂》三卷



 《黃帝九靈經》十二卷靈寶注。



 《玉匱針經》十二卷



 《
 黃帝內經太素》三十卷楊上善注。



 《三部四時五臟辨候診色脈經》一卷



 《黃帝內經明堂類成》十三卷楊上善撰。



 《黃帝明堂經》三卷楊玄孫撰注。



 《灸經》一卷



 《鈴和子》十卷賈和光撰。



 《脈經訣》三卷徐氏撰。



 《脈經》二卷



 《五藏訣》一卷



 《五藏論》一卷



 右明堂經脈二十六家,凡一百七十三卷。



 《
 神農本草》三卷



 《桐君藥錄》三卷桐君撰。



 《雷公藥對》二卷



 《藥類》二卷



 《本草用藥要妙》二卷



 《本草病源合藥節度》五卷



 《本草要術》三卷



 《本草藥性》三卷甄立言撰



 《療癰疽耳眼本草要妙》五卷



 《種芝經》九卷



 《芝草圖》一卷



 《吳氏本草因》六卷吳普撰。



 《李氏本草》三卷



 《名醫別錄》三卷



 《藥目要用》二卷



 《
 本草集經》七卷陶弘景撰。



 《靈秀本草圖》六卷原平仲撰。



 《諸藥異名》十卷釋行智撰。



 《四時採取諸藥及合和》四卷



 《本草圖經》七卷蘇敬撰。



 《新修本草》二十一卷蘇敬撰。



 《新修本草圖》一十六卷蘇敬等撰。



 《本草音》三卷蘇敬等撰。



 《本草音義》二卷殷子嚴撰。



 《太清神丹中經》三卷



 《太清神仙服食經》五卷



 又一卷抱樸子撰。



 《太清璇璣文》七卷沖和子撰。



 《金匱仙藥錄》三卷京裏先生撰。



 《
 神仙服食經》十二卷京裏先生撰。



 《太清諸丹要錄集》四卷



 《神仙藥食經》一卷



 《神仙服食方》十卷



 《神仙服食藥方》十卷



 《服玉法並禁忌》一卷



 《太清諸草木方集要》三卷



 《太清玉石丹藥要集》三卷陶弘景撰



 《太一鐵胤神丹方》三卷蘇游撰。



 《養生要集》十卷張湛撰。



 《補養方》三卷孟詵撰。



 《諸病源候論》五十卷吳景撰。



 《四海類聚單方》十六卷隋煬帝撰。



 《太官食法》一
 卷



 《太官食方》十九卷



 《食經》九卷崔浩撰。



 又十卷



 又四卷竺暄撰。



 《四時食法一卷》趙氏撰。



 《淮南王食經》一百二十卷諸葛穎撰。



 《淮南王食目》十卷



 《淮南王食經音》十三卷諸葛穎撰。



 《食經》三卷盧仁宗撰。



 《張仲景藥方》十五卷王叔和撰。



 《華氏藥方》十卷華佗方,吳普集。



 《肘後救卒方》四卷葛洪
 撰。



 《補肘後救卒備急方》六卷陶弘景撰。



 《阮河南方》十六卷阮炳撰。



 《雜藥方》一百七十卷範汪方,尹穆撰。



 《胡居士方》三卷胡洽撰。



 《劉涓子男方》十卷龔慶宜撰。



 《療癰疽金瘡要方》十四卷甘浚之撰。



 《雜療方》二十卷徐叔和撰。



 《體療雜病方》六卷徐叔和撰。



 《腳弱方》八卷徐叔向撰。



 《藥方》十七卷秦承祖撰。



 《療癰疽金瘡要方》十二卷甘伯齊撰。



 《
 雜藥方》十二卷褚澄撰。



 《效驗方》十卷陶弘景撰。



 《百病膏方》十卷



 《雜湯方》八卷



 《療目方》五卷



 《雜藥方》十卷陳山提撰。



 又六卷



 《雜丸方》一卷



 《調氣方》一卷釋鸞撰。



 《黃素方》十五卷



 《雜湯丸散方》五十七卷孝思撰。



 《僧深集方》三十卷釋僧深撰。



 《刪繁方》十二卷謝士太撰。



 《徐王八代效驗方》十卷徐之才撰。



 《徐氏落年方》三卷徐嗣伯撰。



 《
 雜病論》一卷徐嗣伯撰。



 《徐氏家秘方》二卷徐之才撰。



 《集驗方》十卷姚僧垣撰。



 《小品方》十二卷陳延之撰。



 《經心方》八卷宋俠撰。



 《名醫集驗方》三卷。



 《古今錄驗方》五十卷甄權撰。



 《崔氏纂要方》十卷崔知悌撰。



 《孟氏必效方》十卷孟詵撰。



 《延年秘錄》十二卷



 《玄感傳尸方》一卷蘇游撰。



 《骨蒸病灸方》一卷崔知悌撰。



 《寒食散方並消息節度》二卷



 《解寒食散方》十三卷徐叔和撰。



 《婦人方》十卷



 又二十卷



 《少小方》十卷



 《少小雜方》二十卷



 《少小節療方》一卷俞寶撰。



 《狐子雜訣》三卷



 《狐子方金訣》二卷葛仙公撰。



 《陵陽子秘訣》一卷明月公撰。



 《神臨藥秘經》一卷黃公撰。



 《黃白秘法》一卷



 又二十卷



 《玉房秘術》一卷葛氏撰。



 《玉房秘錄訣》八卷沖和子撰。



 《類聚方》二千六百卷



 右醫術本草二十五家,養生十六家,病源單方
 二家,食經十家,雜經方五十八家,類聚方一家,共一百一十家,凡三千七百八十九卷。



 丁部集錄,三類,共八百九十部,書一萬二千二十八卷。



 《楚詞》類一別集類二總集類三



 《楚詞》十六卷王逸注



 《楚詞》十卷郭璞注。



 《楚詞九悼》一卷楊穆撰。



 《離騷草木蟲魚疏》一卷劉沓撰。



 《楚詞音》一卷孟奧撰。



 又一卷徐邈撰。



 又一卷釋道騫撰。



 《
 漢武帝集》二卷



 《魏武帝集》三十卷



 《魏文帝集》十卷



 《魏明帝集》十卷



 《魏高貴鄉公集》二卷



 《晉文帝集》一卷



 《晉明帝集》五卷



 《晉宣帝集》十卷



 《晉簡文帝集》五卷



 《宋武帝集》二十卷



 《宋文帝集》十卷



 《梁文帝集》十八卷



 《梁武帝集》十卷



 《梁簡文帝集》八十卷



 《梁元帝集》五十卷



 《梁元帝集》十卷



 《
 後魏明帝集》一卷



 《後魏文帝集》四十卷



 《後周明帝集》十卷



 《陳後主集》五十卷



 《隋煬帝集》三十卷



 《太宗文皇帝集》三十卷



 《高宗大帝集》八十六卷



 《中宗皇帝集》四十卷



 《睿宗皇帝集》十卷



 《垂拱集》一百卷



 《金輪集》十卷天后撰。



 《梁昭明太子集》二十卷



 《漢淮南王集》二卷



 《漢東平王集》二卷



 《魏陳思王集》二十卷



 又三十卷



 晉《齊王集》二卷



 《晉會稽王集》八卷



 晉《彭城王集》八卷



 晉《譙王集》三卷



 宋《長沙王集》十卷



 宋《臨川王集》八卷



 宋《衡陽王集》十卷



 宋《江夏王集》十三卷



 宋《南平王集》五卷



 宋《建平王集》十卷



 《宋建平王小集》十五卷



 《齊竟陵王集》三十卷



 《梁邵陵王集》四卷



 《梁武陵王集》八卷



 後周《趙王集》十卷



 後周《滕王集》十二卷



 趙《荀況集》二卷



 楚《宋玉集》二卷



 前漢《賈誼集》二卷



 《枚乘集》二卷



 《司馬遷集》二卷



 《東方朔集》二卷



 《董仲舒集》二卷



 《李陵集》二卷



 《司馬相如集》二卷



 《孔臧集》二卷



 《魏相集》二卷



 《張敞集》二卷



 《韋玄成集》二卷



 《劉向集》五卷



 《王褒集》五卷



 《谷永集》五卷



 《
 杜鄴集》五卷



 《師丹集》五卷



 《息夫躬集》五卷



 《劉歆集》五卷



 《楊雄集》五卷



 《崔篆集》一卷



 後漢《桓譚集》二卷



 《史岑集》二卷



 《王文山集》二卷



 《硃勃集》二卷



 《梁鴻集》二卷



 《黃香集》二卷



 《馮衍集》五卷



 《班彪集》二卷



 《杜篤集》五卷



 《傅毅集》五卷



 《
 班固集》十卷



 《崔駰集》十卷



 《賈逵集》二卷



 《劉瑀駼集》二卷



 《崔瑗集》五卷



 《蘇順集》二卷



 《竇章集》二卷



 《胡廣集》二卷



 《高彪集》二卷



 《王逸集》二卷



 《桓驎集》二卷



 《邊韶集》二卷



 《皇甫規集》五卷



 《張奐集》二卷



 《硃穆集》二卷



 《趙壹集》二卷



 《
 張升集》二卷



 《侯瑾集》二卷



 《酈炎集》二卷



 《盧植集》二卷



 《劉珍集》二卷



 《張衡集》十卷



 《葛龔集》五卷



 《李固集》十卷



 《馬融集》五卷



 《崔琦集》二卷



 《延篤集》二卷



 《劉陶集》二卷



 《荀爽集》二卷



 《劉梁集》二卷



 《鄭玄集》二卷



 《蔡邕集》二十卷



 《
 應劭集》四卷



 《士孫瑞集》二卷



 《張劭集》五卷



 《禰衡集》二卷



 《孔融集》十卷



 《虞翻集》三卷



 《潘勖集》二卷



 《阮瑀集》五卷



 《陳琳集》十卷



 《張紘集》一卷



 《繁欽集》十卷



 《楊修集》二卷



 《王粲集》十卷



 魏《華歆集》二十卷



 《王朗集》三十卷



 《邯鄲淳集》二卷



 《
 袁渙集》五卷



 《應瑒集》二卷



 《徐幹集》五卷



 《劉楨集》二卷



 《路粹集》二卷



 《丁儀集》二卷



 《丁暠集》二卷



 《吳質集》五卷



 《劉暠集》二卷



 《孟達集》三卷



 《陳群集》三卷



 《王修集》三卷



 《管寧集》二卷



 《劉邵集》二卷



 《麋元集》五卷



 《李康集》二
 卷



 《孫該集》二卷



 《卞蘭集》二卷



 《傅巽集》二卷



 《高堂隆集》十卷



 《繆襲集》五卷



 《殷褒集》二卷



 《韋誕集》三卷



 《曹羲集》五卷



 《傅嘏集》二卷



 《桓範集》二卷



 《夏侯霸集》二卷



 《鐘毓集》五卷



 《江奉集》二卷



 《夏侯惠集》二卷



 《毋丘儉集》二卷



 《王弼集》五卷



 《
 呂安集》二卷



 《王昶集》五卷



 《王肅集》五卷



 《何晏集》十卷



 《應瑗集》十卷



 《杜摯集》一卷



 《夏侯玄集》二卷



 《程曉集》二卷



 《阮籍集》五卷



 《嵇康集》十五卷



 《鐘會集》十卷



 蜀《許靖集》二卷



 《諸葛亮集》二十四卷



 吳《張溫集》五卷



 《士燮集》五卷



 《駱統集》十卷



 《
 暨艷集》二卷



 《謝承集》四卷



 《姚信集》十卷



 《楊厚集》二卷



 《華核集》三卷



 《胡綜集》二卷



 《薛綜集》二卷



 《張儼集》二卷



 《韋昭集》二卷



 《紀騭集》三卷



 晉《王沉集》五卷



 《鄭矰集》二卷



 《應貞集》五卷



 《嵇喜集》二卷



 《傅玄集》五十卷



 《成公綏集》十卷



 《
 裴秀集》三卷



 《何禎集》五卷



 《袁準集》二卷



 《山濤集》五卷



 《向秀集》二卷



 《阮沖集》二卷



 《阮侃集》五卷



 《羊祜集》二卷



 《賈充集》二卷



 《荀勖集》二十卷



 《杜預集》二十卷



 《王浚集》二卷



 《皇甫謐集》二卷



 《程咸集》二卷



 《劉毅集》二卷



 《庾峻集》三卷



 《
 卻正集》一卷



 《薛瑩集》二卷



 《楊泉集》二卷



 《陶浚集》三卷



 《宣聘集》三卷



 《曹志集》二卷



 《鄒湛集》四卷



 《孫毓集》二卷



 《王渾集》五卷



 《王深集》四卷



 《江偉集》五卷



 《閔鴻集》二卷



 《裴楷集》二卷



 《何劭集》二卷



 《劉頌集》三卷



 《劉實集》二卷



 《
 裴頠集》十卷



 《許孟集》二卷



 《王祜集》二卷



 《王濟集》二卷



 《華嶠集》一卷



 《庾壝集》三卷



 《謝衡集》二卷



 《傅咸集》三十卷



 《棗據集》二卷



 《劉寶集》三卷



 《孫楚集》十卷



 《王贊集》三卷



 《夏侯湛集》十卷



 《夏侯淳集》十卷



 《張敏集》二卷



 《劉訏集》二卷



 《
 李重集》二卷



 《樂廣集》二卷



 《阮渾集》二卷



 《楊乂集》三卷



 《張華集》十卷



 《李虔集》十卷



 《石崇集》五卷



 《潘岳集》十卷



 《潘尼集》十卷



 《歐陽建集》二卷



 《嵇紹集》二卷



 《衛展集》四十卷



 《盧播集》二卷



 《欒肇集》二卷



 《應亨集》二卷



 《司馬彪集》三卷



 《
 杜育集》二卷



 《摯虞集》二卷



 《繆徵集》二卷



 《左思集》五卷



 《夏侯靖集》二卷



 《鄭豐集》二卷



 《陳略集》二卷



 《張翰集》二卷



 《陸機集》十五卷



 《陸雲集》十卷



 《陸沖集》二卷



 《孫極集》二卷



 《張載集》三卷



 《張協集》二卷



 《束皙集》五卷



 《華譚集》二卷



 《
 曹攄集》二卷



 《江統集》十卷



 《胡濟集》五卷



 《卞粹集》二卷



 《閭丘沖集》二卷



 《庾敳集》二卷



 《阮瞻集》二卷



 《阮循集》二卷



 《裴邈集》二卷



 《郭象集》五卷



 《嵇含集》十卷



 《孫惠集》十卷



 《蔡洪集》三卷



 《牽秀集》五卷



 《蔡克集》二卷



 《索靖集》二
 卷



 《閻纂集》二卷



 《張輔集》二卷



 《殷巨集》二卷



 《陶佐集》五卷



 《仲長敖集》二卷



 《虞溥集》二卷



 《吳商集》五卷



 《劉弘集》三卷



 《山簡集》二卷



 《宗岱集》三卷



 《王曠集》五卷



 《王峻集》二卷



 《棗腆集》二卷



 《棗嵩集》二卷



 《劉琨集》十卷



 《盧諶集》十卷



 《
 傅暢集》五卷



 東晉《顧榮集》二卷



 《荀組集》二卷



 《周顗集》二卷



 《周嵩集》三卷



 《王導集》十卷



 《荀邃集》二卷



 《王敦集》五卷



 《謝鯤集》二卷



 《張抗集》二卷



 《賈霖集》三卷



 《劉隗集》三卷



 《應詹集》三卷



 《陶侃集》二卷



 《王洽集》三卷



 《傅毅集》五卷



 《
 張闓集》三卷



 《卞壼集》二卷



 《劉超集》二卷



 《楊方集》二卷



 《傅純集》二卷



 《卻鑒集》十卷



 《溫嶠集》十卷



 《孔坦集》五卷



 《王濤集》五卷



 《王篾集》五卷



 《甄述集》五卷



 《戴邈集》五卷



 《賀循集》二十卷



 《張俊集》二卷



 《曾瑰集》五卷



 《熊遠集》五卷



 《
 郭璞集》十卷



 《王鑒集》五卷



 《庾亮集》二十卷



 《虞預集》十卷



 《顧和集》五卷



 《範宣集》十卷



 《張虞集》五卷



 《庾冰集》二十卷



 《庾翼集》二十卷



 《何充集》五卷



 《諸葛恢集》五卷



 《祖臺之集》十五卷



 《李充集》十四卷



 《蔡謨集》十卷



 《謝艾集》八卷



 《範汪集》八卷



 《
 範寧集》十五卷



 《阮放集》五卷



 《王暠集》十卷



 《王彪之集》二十卷



 《謝安集》五卷



 《謝萬集》十卷



 《王羲之集》五卷



 《干寶集》四卷



 《殷融集》十卷



 《劉遐集》五卷



 《殷浩集》五卷



 《劉惔集》二卷



 《王濛集》五卷



 《謝尚集》五卷



 《張憑集》五卷



 《張望集》三卷



 《
 韓康伯集》五卷



 《王胡之集》五卷



 《江AN集》五卷



 《範宣集》五卷



 《江淳集》五卷



 《王述集》五卷



 《郝默集》五卷



 《黃整集》十卷



 《王浹集》二卷



 《王度集》五卷



 《劉系之集》五卷



 《劉恢集》五卷



 《範起集》五卷



 《殷康集》五卷



 《孫嗣集》三卷



 《王坦之集》五卷



 《
 桓溫集》二十卷



 《卻超集》十五卷



 《謝朗集》五卷



 《謝玄集》十卷



 《王珣集》十卷



 《許詢集》三卷



 《孫統集》五卷



 《孫綽集》十五卷



 《孔嚴集》五卷



 《江逌集》五卷



 《車灌集》五卷



 《丁纂集》二卷



 《曹毗集》十五卷



 《蔡系集》二卷



 《李顒集》十卷



 《顧夷集》五卷



 《
 袁喬集》五卷



 《謝沉集》五卷



 《庾闡集》十卷



 《王隱集》十卷



 《殷允集》十卷



 《徐邈集》八卷



 《殷仲堪集》十卷



 《殷叔獻集》三卷



 《伏滔集》五卷



 《桓嗣集》五卷



 《習鑿齒集》五卷



 《鈕滔集》五卷



 《邵毅集》五卷



 《孫盛集》十卷



 《袁質集》二卷



 《袁宏集》二十卷



 《
 袁邵集》三卷



 《羅含集》三卷



 《孫放集》十五卷



 《辛昞集》四卷



 《庾統集》二卷



 《郭愔集》五卷



 《滕輔集》五卷



 《庾和集》二卷



 《庾軌集》二卷



 《庾茜集》二卷



 《庾肅之集》十卷



 《王修集》二卷



 《戴逵集》十卷



 《桓玄集》二十卷



 《殷仲文集》七卷



 《卞湛集》五卷



 《
 蘇彥集》十卷



 《袁豹集》十卷



 《王謐集》十卷



 《周祗集》十卷



 《梅陶集》十卷



 《湛方生集》十卷



 《劉瑾集》八卷



 《羊徽集》一卷



 《卞裕集》十四卷



 《王愆期集》十卷



 《孔璠之集》二卷



 《王茂略集》四卷



 《薄肅之集》十卷



 《滕演集》一卷



 《宋劉義宗集》十五卷



 《謝瞻集》二卷



 《
 孔琳之集》十卷



 《王叔之集》十卷



 《徐廣集》十五卷



 《孔寧子集》十五卷



 《蔡廓集》十卷



 《傅亮集》十卷



 《孫康集》十卷



 《鄭鮮之集》二十卷



 《陶淵明集》五卷



 《範泰集》二十卷



 《王弘集》二十卷



 《謝靈運集》十五卷



 《荀昶集》十四卷



 《孔欣集》八卷



 《卞伯玉集》五卷



 《王曇首集》二卷



 《
 謝弘微集》二卷



 《王韶之集》二十四卷



 《沈林子集》七卷



 《姚濤之集》二十卷



 《賀道養集》十卷



 《衛令元集》八卷



 《褚詮之集》八卷



 《荀欽明集》六卷



 《殷淳集》三卷



 《劉瑀集》七卷



 《劉緄集》五卷



 《雷次宗集》三十卷



 《宗炳集》十五卷



 《伍緝之集》十一卷



 《荀雍集》十卷



 《袁淑集》十卷



 《
 顏延之集》三十卷



 《王微集》十卷



 《王僧達集》十卷



 《張暢集》十四卷



 《何偃集》八卷



 《沈懷文集》十三卷



 《江智泉集》十卷



 《謝莊集》十五卷



 《殷琰集》八卷



 《顏竣集》十三卷



 《何承天集》三十卷



 《裴松之集》三十卷



 《卞瑾集》十卷



 《丘泉之集》六卷



 《顏測集》十一卷



 《湯惠休集》三卷



 《
 沈勃集》十五卷



 《徐爰集》十卷



 《鮑照集》十卷



 《庾蔚之集》十一卷



 《虞通之集》五卷



 《劉愔集》十卷



 《孫緬集》十卷



 《袁伯文集》十卷



 《袁粲集》十卷



 《齊褚彥回集》五十卷



 《王儉集》六十卷



 《周顒集》二十卷



 《徐孝嗣集》十二卷



 《王融集》十卷



 《謝朓朓集》十卷



 《孔稚珪集》十卷



 《
 陸厥集》十卷



 《虞羲集》十一卷



 《宗躬集》十二卷



 《江奐集》十二卷



 張融《玉海集》六十卷



 梁《範雲集》十二卷



 《江淹前集》十卷



 《江淹後集》十卷



 《任昉集》三十四卷



 《宗史集》十卷



 《王瑓集》二十卷



 《魏道微集》三卷



 《司馬蒨集》九卷



 《沈約集》一百卷



 《沈約集略》三十卷



 《傅昭集》十卷



 《
 袁昂集》二十卷



 《徐勉前集》二十五卷



 《徐勉後集》十六卷



 《陶弘景集》三十卷



 《周舍集》二十卷



 《何遜集》八卷



 《謝琛集》五卷



 《謝鬱集》五卷



 《王僧孺集》三十卷



 《張率集》三十卷



 《楊眺集》十卷



 《鮑畿集》八卷



 《周興嗣集》十卷



 《蕭洽集》二卷



 《裴子野集》十四卷



 《庾景興集》十卷



 《
 陸倕集》二十卷



 《劉之遴前集》十卷



 《劉之遴後集》三十卷



 《虞爵集》六卷



 《王冏集》三卷



 《劉孝綽集》十一卷



 《劉孝儀集》二十卷



 《劉孝威前集》十卷



 《劉孝威後集》十卷



 《丘遲集》十卷



 《王錫集》七卷



 《蕭子範集》三卷



 《蕭子雲集》二十卷



 《蕭子暉集》十一卷



 《江革集》十卷



 《吳均集》二十卷



 《
 庾肩吾集》十卷



 王筠《洗馬集》十卷



 王筠《中庶子集》十卷



 王筠《左右集》十卷



 王筠《臨海集》十卷



 王筠《中書集》十卷



 王筠《尚書集》十一卷



 《鮑泉集》一卷



 《謝瑱集》十卷



 《任孝恭集》十卷



 《張纘集》十卷



 《陸雲公集》四卷



 《張綰集》十卷



 《甄玄成集》十卷



 《蕭欣集》十卷



 《沈君攸集》十二卷



 後魏《高允集》二十卷



 《宗欽集》二卷



 《李諧集》十卷



 《韓宗集》五卷



 《袁躍集》九卷



 《薛孝通集》六卷



 《溫子升集》二十五卷



 《盧元明集》六卷



 《陽固集》三卷



 《魏孝景集》一卷



 北齊《楊休之集》二十卷



 《邢子才集》三十卷



 《魏收集》七十卷



 《劉逖集》四十卷



 後周《宗懍集》三十卷



 《王褒集》三十卷



 《
 蕭捴集》十卷



 《庾信集》二十卷



 《王衡集》三卷



 陳《沈炯前集》六卷



 《沈炯後集》十三卷



 《周弘正集》二十卷



 《徐陵集》三十卷



 《張正見集》四卷



 《陸珍集》五卷



 《陸瑜集》十卷



 《沈不害集》十卷



 《張式集》十三卷



 《褚介集》十卷



 《顧越集》二卷



 《顧覽集》五卷



 《姚察集》二十卷



 隋《盧思道集》二十卷



 《李元操集》二十二卷



 《辛德源集》三十卷



 《李德林集》十卷



 《牛弘集》十二卷



 《薛道衡集》三十卷



 《何妥集》十卷



 《柳顧言集》十卷



 《江總集》二十卷



 《殷英童集》三十卷



 《蕭愨集》九卷



 《魏澹集》四卷



 《尹式集》五卷



 《諸葛穎集》十四卷



 《王胄集》十卷



 《虞茂代集》五卷



 《
 劉興宗集》三卷



 《李播集》三卷



 唐《陳叔達集》五卷



 《褚亮集》二十卷



 《虞世南集》三十卷



 《蕭瑀集》一卷



 《沈齊家集》十卷



 《薛收集》十卷



 《楊師道集》十卷



 《庾抱集》六卷



 《孔穎達集》五卷



 《王績集》五卷



 《郎楚之集》十卷



 《魏徵集》二十卷



 《許敬宗集》六十卷



 《於志寧集》四十卷



 《
 上官儀集》三十卷



 《李義府集》三十九卷



 《顏師古集》四十卷



 《岑文本集》六十卷



 《劉子翼集》十卷



 《殷聞禮集》十卷



 《陸士季集》十卷



 《劉孝孫集》三十卷



 《鄭代翼集》八卷



 《崔君實集》三十卷



 《李百藥集》三卷



 《孔紹安集》三卷



 《高季輔集》三卷



 《溫彥博集》二十卷



 《李玄道集》十卷



 《謝偃集》十卷



 《
 沈叔安集》二十卷



 《陸楷集》十卷



 《曹憲集》三十卷



 《蕭德言集》三十卷



 《潘求仁集》三卷



 《殷芊集》三卷



 《蕭鈞集》三十卷



 《袁朗集》四卷



 《楊續集》十卷



 《王約集》一卷



 《任希古集》五卷



 《凌敬集》十四卷



 《王德儉集》十卷



 《徐孝德集》十卷



 《杜之松集》十卷



 《宋令文集》十卷



 《
 陳子良集》十卷



 《顏顗集》十卷



 《劉潁集》十卷



 《司馬僉集》十卷



 《鄭秀集》十二卷



 《耿義褒集》七卷



 《楊元亨集》五卷



 《劉綱集》三卷



 《王歸一集》十卷



 《馬周集》十卷



 《薛元超集》三十卷



 《高智周集》五卷



 《褚遂良集》二十卷



 《劉禕之集》五十卷



 《郝處俊集》十卷



 《崔知悌集》五卷



 《
 李安期集》二十卷



 《唐覲集》五卷



 《張大素集》十卷



 《鄧玄挺集》十卷



 《劉允濟集》二十卷



 《駱賓王集》十卷



 《盧照鄰集》二十卷



 《楊炯集》三十卷



 《王勃集》三十卷



 《狄仁傑集》十卷



 《李懷遠集》八卷



 《盧受採集》十卷



 《王適集》二十卷



 《喬知之集》二十卷



 《蘇味道集》十五卷



 《薛曜集》二十卷



 《
 郎餘慶集》十卷



 《盧光容集》五卷



 《崔融集》四十卷



 《閻鏡機集》十卷



 《李嶠集》三十卷



 《喬備集》六卷



 《陳子昂集》十卷



 《元希聲集》十卷



 《李適集》二十卷



 《沈牷期集》十卷



 《徐彥伯前集》十卷



 《後集》十卷



 《宋之問集》十卷



 《杜審言集》十卷



 《穀倚集》十卷



 《富嘉謨集》十卷



 《
 吳少微集》十卷



 《劉希夷集》三卷



 《張柬之集》十卷



 《桓彥範集》三卷



 《韋承慶集》六十卷



 《閭丘均集》三十卷



 《郭元振集》二十卷



 《魏知古集》二十卷



 《閻朝隱集》五卷



 《蘇瑰集》十卷



 《員半千集》十卷



 《李乂集》五卷



 《姚崇集》十卷



 《丘悅集》十卷



 《劉子玄集》十卷



 《盧藏用集》二十卷



 道士《江旻集》三十卷



 沙門《曇諦集》六卷



 沙門《惠遠集》十五卷



 沙門《惠琳集》五卷



 沙門《曇瑗集》六卷



 沙門《亡名集》十卷



 沙門《靈裕集》二卷



 沙門《支遁集》十卷



 《曹大家集》二卷



 《鐘夫人集》二卷



 劉臻妻《陳氏集》五卷



 《左九嬪集》一卷



 《臨安公主集》三卷



 範靖妻《沈滿願集》五卷



 徐悱妻《劉氏集》六卷



 《文章流別集》三十卷摯虞撰。



 《
 善文》四十九卷杜預撰。



 《名文集》四十卷謝沈撰。



 《文苑》一百卷孔逭撰。



 《文選》三十卷梁昭明太子撰。



 《文選》六十卷李善注。



 又六十卷《公孫羅注。



 《文選音》十卷蕭該撰。



 又十卷公孫羅撰。



 《文選音義》十卷釋道淹撰。



 《小詞林》五十三卷



 《集古今帝王正位文章》九十卷



 《文海集》三十六卷蕭圓撰。



 《詞苑麗則》二十卷康明貞撰。



 《芳林要覽》三百卷許敬宗撰。



 《
 類文》三百七十七卷庾自直撰。



 《文館詞林》一千卷許敬宗撰。



 《賦集》四十卷宋明帝撰。



 《皇帝瑞應頌集》十卷



 《五都賦》五卷



 《獻賦集》十卷卞鑠撰。



 《上林賦》一卷司馬相如撰。



 《幽通賦》一卷班固撰,曹大家注。



 又一卷項岱撰。



 《二京賦》二卷張衡撰。



 《二京賦音》二卷薛綜撰。



 《三都賦》三卷



 《齊都賦》一卷左太沖撰。



 《齊都賦音》一卷李軌撰。



 《百賦音》一卷褚令之撰。



 《賦音》二卷郭微之撰。



 《
 三京賦音》一卷綦毋邃撰。



 《木連理頌》二卷



 《靖恭堂頌》一卷李暠撰。



 《諸郡碑》一百六十六卷



 《雜碑文集》二十卷



 《翰林論》二卷李充撰。



 《雜論》九十五卷殷仲堪撰。



 《設論集》三卷劉楷撰。



 又五卷謝靈運撰。



 《連珠集》五卷謝靈運撰。



 《制旨連珠》四卷梁武帝撰。



 又十一卷陸緬撰。



 《贊集》五卷謝莊撰。



 《七國敘贊》十卷



 《吳國先賢贊論》三卷



 《會稽先賢贊》四卷賀氏撰。



 《
 會稽太守像贊》二卷賀氏撰。



 《列女傳敘贊》一卷孫夫人撰。



 《古今箴銘集》十三卷張湛撰。



 《眾賢誡集》十五卷



 《雜誡箴》上十四卷



 《詔集區別》二十七卷宋乾撰。



 《霸朝雜集》五卷李德林撰。



 《古今詔集》三十卷溫彥博撰。



 又一百卷李義府撰。



 《聖朝詔集》三十卷薛堯撰。



 《書集》八十卷王履撰。



 《書林》六卷夏赤松撰。



 《山濤啟事》三卷



 《範寧啟事》十卷



 《梁中書表集》二百五十卷



 《薦文集》七卷



 《
 宋元嘉策》五卷



 《策集》六卷謝靈運撰。



 《七林集》十二卷卞氏撰。



 《七悟集》一卷顏延之撰。



 《俳諧文》十五卷袁淑撰。



 《弘明集》十四卷釋僧祐撰。



 《廣弘明集》三十卷釋道宣撰。



 《陶神論》五卷釋靈祐撰。



 《婦人訓誡集》十卷徐湛之撰。



 《婦人詩集》二卷顏竣撰。



 《女訓集》六卷



 《文釋》十卷江邃撰。



 《文心雕龍》十卷劉勰撰。



 《百志詩集》五卷干寶撰。



 《百國詩集》二十九卷崔光撰。



 《百一詩》八卷應璩撰。



 《
 百一詩集》二卷李夔撰。



 《清溪集》三十卷齊武帝命撰。



 《晉元氏宴會游集》四卷伏滔、袁豹、謝靈運等撰。



 《元嘉宴會游山詩集》五卷



 《元嘉西池宴會詩集》三卷顏延之撰。



 《齊釋奠會詩集》二十卷



 《文會詩集》四卷徐伯陽撰。



 《文林詩府》六卷北齊後主作。



 《西府新文》十卷蕭淑撰。



 《詩集新撰》三十卷宋明帝撰。



 《詩集》二十卷宋明帝撰。



 《詩集抄》十卷謝靈運撰。



 《詩集》五十卷謝靈運撰。



 《
 詩集》二十卷劉和撰。



 又一百卷顏竣撰。



 《詩例錄》二卷顏竣撰。



 《詩英》十卷謝靈運撰。



 《古今詩苑英華集》二十卷梁昭明太子撰。



 《續古今詩苑英華》二十卷釋惠靜撰。



 《詩林英選》十一卷



 《類集》一百一十三卷虞綽等撰。



 《詩纘》十二卷



 又《詞英》八卷



 《六代詩集鈔》四卷徐陵撰。



 《古今類序詩苑》三十卷劉孝孫撰。



 《麗正文苑》二十卷許敬宗撰。



 《
 古今詩類聚》七十九卷郭瑜撰。



 《歌錄集》八卷



 《漢魏吳晉鼓吹曲》四卷



 《樂府歌詩》十卷



 《太樂雜歌詞》三卷荀勖撰。



 《太樂歌詞》二卷



 《樂府歌詞》十卷



 《樂府歌詩》十卷



 《三調相和歌詞》三卷



 《新撰錄樂府集》十一卷謝靈運撰。



 《玉臺新詠》十卷徐陵撰。



 《回文詩集》一卷謝靈運撰。



 《金門待詔集》十卷劉允濟撰。



 《集苑》六十卷謝琨撰。



 《集林》二百卷劉義慶撰。



 《
 集鈔》四十卷



 右集錄楚詞七家,帝王二十七家,太子諸王二十一家,七國趙、楚各一家,前漢二十家,後漢五十家,魏四十六家,蜀二家,吳十四家,西晉一百一十九家,東晉一百四十四家,宋六十家,南齊十二家,梁五十九家,陳十四家,後魏十家,北齊四家,周五家,隋十八家,唐一百一十二家,沙門七家,婦人七家;總集一百二十四家。凡八百
 九十二部,一萬二千二十八卷。



 三代之書,經秦燔煬殆盡。漢武帝、河間王始重儒術,於灰燼之餘,拓纂亡散,篇卷僅而復存。劉更生石渠典校之書,卷軸無幾。逮歆之《七略》,在《漢藝文志》者,裁三萬三千九百卷。後漢蘭臺、石室、東觀、南宮諸儒撰集,部帙漸增。董卓遷都,載舟西上,因罹寇盜,沉之於河,存者數船而已。及魏武父子,採掇遺亡,至晉總括群書,裁二萬七千九百四十五卷。及永嘉之亂,洛都覆沒,靡有孑遺。江
 表所存官書,凡三千一十四卷。至宋謝靈運造《四部書目錄》,凡四千五百八十二卷。其後王儉復造書目,凡五千七十四卷。南齊王亮、謝朏《四部書目》,凡一萬八千一十卷。齊末兵火延燒秘閣,書籍煨燼。梁元帝克平侯景,收公私經籍歸於江陵,凡七萬餘卷。蓋佛老之書,計於其間。及周師入郢,咸自焚煬。周武保定之中,官書裁盈萬卷。平齊所得,數止五千。及隋氏平陳,南北一統,秘書監牛弘奏請搜訪遺逸,著定書目,凡三萬餘卷。煬帝寫
 五十副本,分為三品。國家平王世充,收其圖籍,溯河西上,多有沉沒,存者重復八萬卷。自武德已後,文士既有修纂,篇卷滋多。開元時,甲乙丙丁四部書各為一庫,置知書官八人分掌之。凡四部庫書,兩京各一本,共一十二萬五千九百六十卷。皆以益州麻紙寫。其集賢院御書,經庫皆鈿白牙軸,黃縹帶,紅牙簽,史書庫鈿青牙軸,縹帶,綠牙簽,子庫皆雕紫檀軸,紫帶,碧牙簽,集庫皆綠牙軸,硃帶,白牙簽,以分別之。



\end{pinyinscope}