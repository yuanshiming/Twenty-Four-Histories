\article{卷一百三十一}

\begin{pinyinscope}

 ○姚令言張光晟源休喬琳張涉蔣鎮洪經綸彭偃



 姚令言,河中人也。少應募,起於卒伍,隸涇原節度馬璘。以戰功累授金吾大將軍同正,為衙前兵馬使,改試太
 常卿、兼御史中丞。建中元年,孟暤為涇原節度留後,自以文吏進身,不樂軍旅,頻表薦令言謹肅,堪任將帥。暤尋歸朝廷,遂拜令言為四鎮北庭行營涇原節度使、涇州刺史、兼御史大夫。



 建中四年,李希烈叛,寇陷汝州,詔哥舒曜率師攻之,營於襄城。希烈兵數萬圍襄城,勢甚危急。十月,詔令言率本鎮兵五萬赴援。涇師離鎮,多攜子弟而來,望至京師以獲厚賞,及師上路,一無所賜。時詔京兆尹王翃犒軍士,唯糲食菜啖而已,軍士覆而不
 顧,皆憤怒,揚言曰:「吾輩棄父母妻子,將死於難,而食不得飽,安能以草命捍白刃耶!國家瓊林、大盈,寶貨堆積,不取此以自活,何往耶?」行次滻水,乃返戈,大呼鼓噪而還。令言曰:「比約東都有厚賞,兒郎勿草草,此非求活之良圖也。」眾不聽,以戈環令言請退,令言急奏之。上恐,令內庫出繒彩二十車馳賜之,軍聲浩浩,令言不能戢。街市居人狼狽走竄,亂兵呼曰:「勿走,不稅汝間架矣!」德宗令普王與學士姜公輔往撫勞之,才出內門,賊已斬關,
 陣於丹鳳樓下。是日,德宗倉卒出幸,賊縱入府庫輦運,極力而止。



 時太尉硃泚罷鎮居晉昌里第,是夜,叛卒謀曰:「硃太尉久囚於宅,若迎為主,大事濟矣。」泚嘗節制涇州,眾知其失權,廢居怏怏,又幸泚寬和,乃請令言率騎迎泚於晉昌里。泚初遲疑,以食飼之,徐觀眾意,既而諸校齊至,乃自第張炬火入居含元殿。既僭號,乃以令言為侍中,與源休同知賊政事。既以身先逆亂,頗盡心於賊,害宗室,圍奉天,皆令言為首帥也。群兇宴樂,既醉,令
 言與源休論功,令言自此蕭何,源休曰:「帷幄之謀,成秦之業,無出予之右者。吾比蕭何無讓,子當曹參可矣。」時朝士在賊廷者,聞之皆笑,謂源休為火迫酂侯。硃泚敗,令言與張廷芝尚有眾萬人,從泚將入吐蕃。至涇州,欲投田希鑒,希鑒偽致禮誘之,與泚俱斬首來獻。



 張光晟,京兆盩厔人,起於行間。天寶末,哥舒翰兵敗潼關,大將王思禮所乘馬中流矢而斃,光晟時在騎卒之中,因下,以馬授思禮。思禮問其姓名,不告而退,思禮陰
 記其形貌,常使人密求之。無何,思禮為河東節度使,其偏將辛云京為代州刺史,屢為將校譖毀,思禮怒焉。云京惶懼,不知所出。光晟時隸云京麾下,因間進曰:「光晟素有德於王司空,比不言諸,恥以舊恩受賞。今使君憂迫,光晟請奉命一見司空,則使君之難可解。」云京然其計,即令之太原。乃謁思禮,未及言舊,思禮識之,遽曰:「爾豈非吾故人乎?何相見之晚也!」光晟遂陳潼關之事,思禮大喜,因執其手感泣曰:「吾有今日,子之力也。求子頗
 久,竟此相遇,何慰如之?」命同榻而坐,結為兄弟。光晟遂述云京之屈,思禮曰:「云京比涉謗言,過亦不細,今為故人,特舍之矣。」即日擢光晟為兵馬使,賚田宅、縑帛甚厚,累奏特進,試太常少卿,委以心腹。及云京為河東節度使,又奏光晟為代州刺史。



 大歷末,遷單于都護、兼御史中丞、振武軍使。代宗密謂之曰:「北蕃縱橫日久,當思所御之計。」光晟既受命,至鎮,威令甚行。建中元年,回紇突董梅錄領眾並雜種胡等自京師還國,輿載金帛,相屬
 於道。光晟訝其裝橐頗多,潛令驛吏以長錐刺之,則皆輦歸所誘致京師婦人也。遂給突董及所領徒悉令赴宴,酒酣,光晟伏甲盡拘而殺之,死者千餘人,唯留二胡歸國復命。遂部其婦人,給糧還京,收其金帛,賞賚軍士。後回紇遣使來訴,上不欲甚阻蕃情,徵拜右金吾將軍。回紇猶怨懟不已,又降為睦王傅,尋改太僕卿,負才怏怏不得志。



 賊泚僭逆,署光晟偽節度使兼宰相。及泚眾頻敗,遂擇精兵五千配光晟,營於九曲,去東渭橋凡十
 餘里。光晟潛使於李晟,有歸順之意。晟進兵入苑,光晟勸賊泚宜速西奔,光晟以數千人送泚出城,因率眾回降於晟。晟以其誠款,又愛其材,欲奏用之,俾令歸私第,表請特減其罪。每大宴會,皆令就坐,華州節度使駱元光詬之曰:「吾不能與反虜同席!」拂衣還營。晟不得已,拘之私第,後有詔言其狀跡不可原,乃斬之。



 源休,相州臨漳人,京兆尹光輿之子也。休以幹局,累授監察御史、殿中侍御史、青苗使判官,遷虞部員外郎,。出
 潭州刺史,入為主客郎中,遷給事中、御史中丞、左庶子。其妻即吏部侍郎王翊女也。因小忿而離,妻族上訴,下御史臺驗理,休遲留不答款狀,除名,配流溱州。久之,移嶽州。



 建中初,楊炎執政,以京兆尹嚴郢威名稍著,心欲傾之。郢,即王翊甥婿也。休與王氏離絕之時,炎風聞休、郢有隙,遂擢休自流人為京兆少尹,俾令伺郢過失。休既職久,與郢親善,炎怒之,奏令以本官兼御史中丞,奉使回紇。休至振武,軍使張光晟已殺回紇突董等,上初
 欲遂絕其使,令休還,待命於太原。久之方遣,仍令休歸其突董、翳密施大小梅錄等四尸。突董者,即武義可汗之叔父也。尸既至,可汗令宰臣已下具彩服車馬來迎。其宰相頡於思迦坐大帳,立休等於帳外雪中,詰殺突董等故。休曰:「突董等自與張光晟忿鬥而死,非天子命也。」又問:「使者背唐國,負罪當死,不能自戮耶?不然,何假手於我殺之也?」凡將殺者數矣,言頗悖慢,乃引去,供餼甚薄,留之五十餘日,乃得還。可汗使謂休曰:「我國人皆欲殺
 汝,唯我不然。汝國已殺突董等,吾又殺汝,猶以血洗血,汙益甚爾。吾今以水洗血,不亦善乎!所欠吾馬直絹一百八十萬疋,當速歸之。」遣散支將軍康赤心等隨休來朝,休竟不得見其可汗。尋遣赤心等歸,與之帛十萬疋、金銀十萬兩,償其馬直。休履危而還,宰相盧杞又恐復命之日以口辯結恩,將至太原,遽奏為光祿卿。休以其還使賞薄,居常怨望。



 會涇原兵叛,立硃泚為主。初但稱太尉,朝官謁泚者,悉勸奏迎鑾駕,既不合泚意而退。及
 休至,遂屏人移時,言多悖逆,盛陳成敗,稱述符命,勸令僭號。泚悅其言,以休為宰相,判度支。休遂為謀主,至於兵食軍資,遷除補擬,內外咨謀,一稟休畫。故時人云:「源休之逆,甚於硃泚。」朝廷大臣之奔竄不獲者,多為休所誘致,以至戮辱,職休而為,蓋非一焉。又勸泚鋤翦宗室,以絕人望,命萬年縣賊曹尉楊偡專其斷決,諸王子孫遇害不可勝數。泚敗走,休隨至寧州。泚死,休走鳳翔,為其部曲所殺,傳首來獻。休三子並斬於東市,籍沒其家。



 喬琳,太原人。少孤貧志學,以文詞稱。天寶初,舉進士,補成武尉,累授興平尉。朔方節度郭子儀闢為掌書記,尋拜監察御史。琳倜儻疏誕,好談諧,侮謔僚列,頗無禮檢。同院御史畢耀初與琳嘲誚往復,因成釁隙,遂以公事互相告訴,坐貶巴州員外司戶。遂起為南郭令,改殿中侍御史,充山南節度張獻誠行軍司馬。使罷,為劍南東川節度鮮於叔明判官。改檢校駕部郎中、果綿遂三州刺史、兼御史中丞。入為大理少卿、國子祭酒。出為懷州
 刺史。琳素與張涉友善,上在春宮,涉嘗為侍讀。及嗣位,多以政事詢訪於涉,盛稱琳識度材略,堪備大用,因拜御史大夫、平章事。琳本粗材,又年高有耳疾,上每顧問,對答失次,論奏不合時。幸居相位凡八十餘日,除工部尚書,罷知政事,尋加迎皇太后副使。



 硃泚之亂,扈從至奉天,轉吏部尚書,遷太子少師。再幸梁、洋,從至盩厔,托以馬乏遲留,上以琳舊老,心敬重之,慰諭頗至,以御馬一匹給焉。又懇辭以老疾不堪山阻登頓,上悵然,賜
 之所執策曰:「勉為良圖,與卿決矣。」後數日,乃削發為僧,止仙游寺。賊泚聞之,遂令數十騎追至京城,俾為偽吏部尚書。令源休被公服,饋肉食,琳雖辭讓,而僧言求施。琳掌賊中吏部,選人前請曰:「所注某官不穩便。」琳謂之曰:「足下謂此選竟穩便乎?」及官軍收京師,當處極刑,時琳已七十餘,李晟憫其衰老,表請減死。上以其累經重任,頓虧臣節,自受逆命,頗聞譏諧悖慢之言,背義負恩,固不可舍,命斬之。臨刑嘆曰:「喬琳以七月七日生,亦以
 此日死,豈非命歟!」



 張涉者,蒲州人,家世儒者。涉依國學為諸生講說,稍遷國子博士,亦能為文,嘗請有司日試萬言,時呼張萬言。德宗在春宮,受經於涉。及即位之夕,召涉入宮,訪以庶政,大小之事皆咨之。翌日,詔居翰林,恩禮甚厚,親重莫比。自博士遷散騎常侍。上方屬意宰輔,唯賢是擇,故求人於不次之地。涉舉懷州刺史喬琳為相,上授之不疑,天下聞之者皆愕然。數月,琳以不稱職罷,上由是疏涉。
 俄受前湖南都團練使辛京杲贓事發,詔曰:「尊師之道,禮有所加;議故之法,恩有所掩。張涉賄賂交通,頗駭時聽,常所親重,良深嘆惜。宜放歸田里。」



 蔣鎮,常州義興人,尚書左丞洌之子也。與兄練並以文學進。天寶末舉賢良,累授左拾遺、司封員外郎,轉諫議大夫。時戶部侍郎、判度支韓滉上言:「河中鹽池生瑞鹽,實土德之上瑞。」上以秋霖稍多,水潦為患,不宜生瑞,命鎮馳驛檢行之。鎮奏與滉同,仍上表賀,請宣付史館,並
 請置神祠,錫其嘉號寶應靈慶池。地霖潦彌月,壞居人廬舍非一,鹽池為潦水所入,其味多苦。韓滉慮鹽戶減稅,詐奏雨不壞池,池生瑞鹽,鎮庇之飾詐,識者醜之。轉給事中、工部侍郎,以簡儉稱於時。



 其妹婿源溥,即休之弟也,以姻媾之故,與休交好。涇師之叛,鎮潛竄,夜至鄠縣西,馬躓墮溝澗中,傷足不能進。時史練已與源休相率受賊偽官。鎮僕人有逃歸投練,雲鎮病足在鄠。練與源休聞之大喜,遂言於賊泚此。泚素慕鎮清名,即令騎二
 百求之鄠縣西。明日,擁鎮而至,署為偽宰相。既知不免,每憂沮,常懷刃將自裁,多為兄練所救而罷。數日後,復謀竄匿,竟以性懦畏怯,計終不果。然源休與泚頻議,欲逼脅潛藏衣冠,大加殺戮,鎮輒力爭救,獲全者甚眾。至是,與兄練等並授偽職,斬於東市西北街。



 初鎮父洌,叔渙,當祿山、思明之亂,並授偽職,然以家風修整,為士大夫所稱。鎮兄弟亦以教義禮法為己任,而貪祿愛死,節隳身戮,為天下笑。



 洪經綸,建中初為黜陟使。至東都,訪聞魏州田悅食糧兵凡七萬人,經綸素昧時機,先以符停其兵四萬人,令歸農畝。田悅偽順命,即依符罷之;而大集所罷兵士,激怒之曰:「爾等在軍旅,各有父母妻子,既為黜陟使所罷,如何得衣食?」遂大哭。悅乃盡出家財衣服厚給之,各令還其部伍,自此人堅叛心,由是罷職。及硃泚反,偽授太常少卿。



 彭偃,少負俊才,銳於進取,為當塗者所抑,形於言色。大
 歷末,為都官員外郎。時劍南東川觀察使李叔明上言,以「佛、道二教,無益於時,請粗加澄汰。其東川寺觀,請定為二等:上寺留僧二十一人;上觀留道士十四人,降殺以七,皆精選有道行者,餘悉令返初。蘭若、道場無名者皆廢。」德宗曰:「叔明此奏,可為天下通制,不唯劍南一道。」下尚書集議。偃獻議曰:



 王者之政,變人心為上,因人心次之,不變不因,循常守固者為下。故非有獨見之明,不能行非常之事。今陛下以惟新之政,為萬代法,若不革
 舊風,令歸正道者,非也。當今道士,有名無實,時俗鮮重,亂政猶輕。唯有僧尼,頗為穢雜。自西方之教,被於中國,去聖日遠,空門不行五濁,比丘但行粗法。爰自後漢,至於陳、隋,僧之廢滅,其亦數乎!或至坑殺,殆無遺餘。前代帝王,豈惡僧道之善如此之深耶?蓋其亂人亦已甚矣。且佛之立教,清凈無為,若以色見,即是邪法,開示悟入,唯有一門,所以三乘之人,比之外道。況今出家者皆是無識下劣之流,縱其戒行高潔,在於王者,已無用矣,況
 是茍避征徭,於殺盜淫,無所不犯者乎!今叔明之心甚善,然臣恐其奸吏詆欺,而去者未必非,留者不必是,無益於國,不能息奸。既不變人心,亦不因人心,強制力持,難致遠耳。



 臣聞天生烝人,必將有職,游行浮食,王制所禁。故有才者受爵祿,不肖者出租征,此古之常道也。今天下僧道,不耕而食,不織而衣,廣作危言險語,以惑愚者。一僧衣食,歲計約三萬有餘,五丁所出,不能致此。舉一僧以計天下,其費可知。陛下日旰憂勤,將去人害,
 此而不救,奚其為政?臣伏請僧道未滿五十者,每年輸絹四疋;尼及女道士未滿五十者,每年輸絹二疋;其雜色役與百姓同。有才智者令入仕,請還俗為平人者聽。但令就役輸課,為僧何傷。臣竊料其所出,不下今之租賦三分之一,然則陛下之國富矣,蒼生之害除矣。其年過五十者,請皆免之。夫子曰:「五十而知天命。」列子曰:「不班白,不知道。」人年五十,嗜欲巳衰,縱不出家,心已近道,況戒律檢其情性哉!臣以為此令既行,僧道規避還俗
 者固已太半。其年老精修者,必盡為人師,則道、釋二教益重明矣。



 議者是之,上頗善其言。大臣以二教行之已久,列聖奉之,不宜頓擾,宜去其太甚,其議不行。



 偃以才地當掌文誥,以躁求為時論所抑,鬱鬱不得志。涇師之亂,從駕不及,匿於田家,為賊所得。硃泚素知之,得偃甚喜,偽署中書舍人,僭號辭令,皆偃為之。賊敗,與偽中丞崔宣、賊將杜如江、吳希光等十三人,李晟收之,俱斬於安國寺前。



 史臣曰:肇分陰陽,爰有生死,修短二事,賢愚一途。故君子遇夷險之機,不易其節;小人昧逆順之道,而陷於刑。鴻毛泰山,斯為至論。令言遠總師徒,首為叛逆;光晟初當委任,危輸款誠;源休雖曰士流,甚於元惡;喬琳巧辭真主,俯就偽官;蔣鎮貪祿隳節,皆曰小人。經綸之徒,不足言爾。



 贊曰:時爭逆順,命擊死生。君子守節,小人正刑。



\end{pinyinscope}