\article{卷一百一十八}

\begin{pinyinscope}

 ○魯炅裴來瑱周智光



 魯炅,範陽人也。身長七尺餘,涉獵書史。天寶六年,隴右節度使哥舒翰引為別奏。顏真卿為監察御史,使至隴右,翰嘗設宴,真卿謂翰曰:「中丞自郎將授將軍,便登節
 制,後生可畏,得無人乎?」炅時立在階下,翰指炅曰:「此人後當為節度使矣。」後以隴右破吐蕃跳蕩功,累授右領軍大將軍同正員,賜紫金魚袋。



 祿山之亂,選任將帥。十五載正月,拜炅上洛太守,未行,遷南陽太守、本郡守捉,仍充防禦使。尋兼御史大夫,充南陽節度使,以嶺南、黔中、山南東道子弟五萬人屯葉縣北,滍水之南,築柵,四面掘壕以自固。至五月,賊將武令珣、畢思琛等來擊之,眾欲出戰,炅不許。賊於營西順風燒煙,營內坐立不得,
 橫門扇及木爭出,賊矢集如雨,炅與中使薛道等挺身遁走,餘眾盡沒。嶺南節度使何履光、黔中節度使趙國珍、襄陽太守徐浩未至,裨將嶺南、黔中、荊襄子弟半在軍,多懷金銀為資糧,軍資器械盡棄於路如山積。至是賊徒不勝其富。炅收合殘卒,保南陽郡,為賊所圍。尋而潼關失守,賊使哥舒翰招之,不從。又使偽將豫州刺史武令珣等攻之,累月不能克。武令珣死,又令田承嗣攻之。潁川太守來瑱、襄陽太守魏仲犀合勢救之。犀使弟孟
 馴為將,領兵至明府橋,望賊而走,眾遂大敗。炅城中食盡,煮牛皮筋角而食之,米斗至四五十千,有價無米,鼠一頭至四百文,餓死者相枕藉。肅宗使中官將軍曹日昇來宣慰,路絕不得入。日昇請單騎入致命,仲犀曰:「不可,賊若擒吾敕使,我亦何安!」顏真卿適自河北次於襄陽,謂仲犀曰:「曹使既果決,不顧萬死之地,何得沮之!縱為賊所獲,是亡一使者;敬得入城,則萬人之心固矣。公何愛焉?」中官馮廷環曰:「將軍必能入,我請以兩騎助之。」
 日昇又自有傔騎數人,仲犀又以數騎共十人同行。賊徒望見,知其驍銳,不敢逼。日昇既入城,炅眾初以為望絕,忽有使來宣命,皆踴躍一心。日昇以其十人至襄陽取糧,賊雖追之,不敢擊,遂以一千人取音聲路運糧而入,賊亦不能遏,又得相持數月。



 炅在圍中一年,救兵不至,晝夜苦戰,人相食。至德二年五月十五日,率眾持滿傳矢突圍而出南陽,投襄陽。田承嗣來追,苦戰二日,殺賊甚眾。賊又知其決死,遂不敢逼,朝廷因除御史大夫、
 襄陽節度使。時賊志欲南侵江、漢,賴炅奮命扼其沖要,南夏所以保全。十月,王師收兩京,承嗣、令珣等奔於河北。南陽遭大亂之後,距鄧州二百里,人煙斷絕,遺骸委積於墻塹間。



 十二月,策勛行賞。詔曰:「特進、太僕卿、南陽郡守、兼御史大夫、權知襄陽節度事、上柱國、金鄉縣公魯炅,蘊是韜略,副茲節制,竭節保邦,悉心陷敵。表之旗幟,分以土田。可開府儀同三司、兼御史大夫,封岐國公,食實封二百戶,兼京兆尹。」



 乾元元年,兼鄭州刺史,充鄭、
 陳、潁、亳等州節度使。上元二年,為淮西襄陽節度使、鄧州刺史。十月,與朔方節度使司徒郭子儀、河東節度使太尉李光弼等九節度同圍安慶緒於相州。炅領淮西、襄陽節度行營步卒萬人、馬軍三百,以李抱玉為兵馬使,炅分界知東面之北。二年六月六日,賊將史思明自範陽來救,戰於安陽河北,王師不利,炅中流矢奔退。時諸節度以回紇戰敗,因而退散,盡棄軍糧器械,所過虜掠,炅兵士剽奪尤甚,人因驚怨。五日,至新鄭縣,聞郭子
 儀已整眾屯谷水,李光弼還太原,炅憂懼,仰藥而卒。



 裴,以門廕入仕,累遷京兆府司錄參軍。來瑱鎮陜州,引為判官;瑱移襄州,又為瑱行軍司馬,瑱遇之甚厚。及瑱淮西之敗,逗留不行,密表聞奏。朝廷以瑱掌重兵,惡之,密詔以代瑱為襄州刺史,充防禦使。本鎮穀城,及受密命,乃率麾下二千人赴襄陽。時瑱亦奉詔依舊任,瑱遂設具於江津以俟之。初聲言假道入朝,及見瑱,即云奉代,且欲視事。瑱報曰:「瑱已奉恩命復任此。」
 惶惑,喻其麾下曰:「此言必妄。」遂引射瑱軍,因與瑱兵交戰,大敗,士卒死傷殆盡。走還穀城舊營,瑱追擒之。朝旨務安漢南,乃歸咎於。寶應元年七月,敕曰:「前襄州刺史裴,性本頑疏,行惟狂悖。頃因試用,爰委軍戎,守在要沖,無聞方略。所以申命來瑱,重撫漢南,即宜奔赴闕廷,謝其曠職。而乃顧惜名位,輕圖異端,誣構忠良,妄興兵甲。遽令追召,敢欲逗留,是有無君之心,不唯罔上之罪。又轉輸之物,軍國所資,擅為費用,其數甚廣。
 據其抵犯,合置嚴誅。但自朕登極已來,屢施恩宥,肆諸朝市,所未忍為。宜寬殊死之刑,俾就投荒之謫,宜除名,長流費州。」



 器局輕褊,初興師徒,給用無節。及敗撓,遲回赴召,將至京師,會有是命。既行,至藍田驛,賜自盡。



 來瑱,邠州永壽人也。父曜,起於卒伍。開元十八年,為鴻臚卿同正員、安西副都護、持節磧西副大使、四鎮節度使,後為右領軍大將軍、仗內五坊等使,名著西陲。寶應元年,以子貴,贈太子太保。



 瑱少尚名節,慷慨有大志,頗
 涉書傳。天寶初,四鎮從職。十一載,為左贊善大夫、殿中侍御史,充伊西、北庭行軍司馬。玄宗詔朝臣舉智謀果決、才堪統眾者各一人。拾遺張鎬薦瑱有縱橫之略,臨事能斷,堪當御悔之任。丁母憂,以孝聞。



 安祿山反,張垍復薦之,起復兼汝南郡太守,未行,改潁川太守。賊攻之。城中積粟素多,瑱繕修有備。賊繼至城下,瑱親射之,無不應弦而斃。賊使降將畢思琛招瑱,琛即瑱父曜故將,城下拜泣吊瑱,瑱不應。前後殺賊頗眾,咸呼瑱為「來嚼
 鐵」。以功加銀青光祿大夫,攝御史中丞、本郡防禦使及河南淮南游奕逐要招討等使。魯炅敗於葉縣,退守南陽,乃以瑱為南陽太守、兼御史中丞,充山南東道節度防禦處置等使以代炅。尋以嗣虢王巨為御史大夫、河南節度使,因奏炅守南陽,詔各復本位。賊攻圍南陽累月,瑱分兵與襄陽節度使魏仲犀救之。犀遣弟孟馴將兵至明府橋,望風敗走,賊追蹙,大敗而還。兵素少,遇敗,人情恟懼,瑱綏撫訓練,賊不能侵。詔為淮南西道節
 度使。收復兩京,與魯炅同制加開府儀同三司、兼御史大夫,封潁國公,食實封二百戶,餘如故。



 乾元元年,召為殿中監。二年,初除涼州刺史、河南節度經略副大使。未行,屬相州官軍為史思明所敗,東京震駭。元帥司徒郭子儀鎮谷水,乃以瑱為陜州刺史,充陜、虢等州節度,並潼關防禦、團練、鎮守使。乾元三年四月十三日,襄州軍將張維瑾、曹玠率眾謀亂,殺刺史史翽。以瑱為襄州刺史、兼御史大夫,充山南東道襄、鄧、均、房、金、商、隨、郢、復十州
 節度觀察處置使。



 上元三年,肅宗召瑱入京。瑱樂襄州,將士亦慕瑱之政,因諷將吏、州牧、縣宰上表請留之,身赴詔命,行及鄧州,復詔歸鎮。肅宗聞其計而惡之。後呂諲、王仲昇及中官皆言瑱布恩惠,懼其得士心,以瑱為鄧州刺史,充山南東道襄、鄧、唐、復、郢、隨等六州節度,餘並如故。俄而淮西節度王仲昇與賊將謝欽讓戰於申州城下,為賊所虜。初,仲昇被圍累月,呂諲病於江陵,瑱在襄州,又恐仲昇構己,遂顧望不救。及師出,仲昇已沒。
 裴頻表陳瑱之狀,謀奪其位,稱「瑱善謀而勇,崛強難制,宜早除之,可一戰而擒也。」肅宗然之,遂以瑱檢校戶部尚書、兼御史大夫、安州刺史,充淮西申、安、蘄、黃、光、沔節度觀察,兼河南陳、豫、許、鄭、汴、曹、宋、潁、泗十五州節度觀察使,外示尊崇,實奪其權也。加裴兼御史中丞、襄鄧等七州防禦使以代之。瑱懼不自安,上表稱「淮西無糧饋軍,臣去秋種得麥,請待收麥畢赴上」,復諷屬吏請留之。裴於商州召募,以窺去就。



 寶應元年五月,代宗
 即位,因復授瑱襄州節度、奉義軍渭北兵馬等使,官如故,潛令裴圖之。其月十九日,裴率眾浮漢江而下。日暮,候者白瑱,謀於帳下,副使薛南陽曰:「尚書奉詔留鎮,裴以兵代,是無名也。且之智勇,非尚書敵也,眾心歸尚書,不歸於。彼若乘我之不虞,今夕而至,直燒城市,我眾必懼而亂,彼乘亂而擊,則可憂也。若及明而至,尚書破之必矣。」翌日平明,督軍士五千列於谷水北,瑱以兵逆之,登高而陣,呼將士告之曰:「爾何事來?」
 曰:「尚書不受命,謹奉中丞伐罪人。若尚書受替,謹當釋兵。」瑱曰:「恩制復除瑱此州。」及取告身敕書以示,軍皆曰:「偽也。承命討君,豈千里空歸,富貴在於今日。」遂爭射之。瑱奔歸旗下,薛南陽曰:「事急矣,請以三百騎為奇兵,尚書勿與之戰。」兩軍相見,遂以麾下旁萬山而出其背,表裏夾擊,軍大敗,投水而死,殺獲殆盡。及弟薦脫身北走,妻子並為瑱所擒,瑱甚厚撫之。因抗表謝罪。擒於申口,送至京師,長流費州,賜死於藍田故驛。



 八月,
 瑱入朝謝罪,代宗特寵異之,遷兵部尚書、同中書門下平章事,依前山南東道節度、觀察等使,代左僕射裴冕充山陵使。時中官驃騎大將軍程元振居中用事,發瑱言涉不順,王仲昇賊平來歸,證瑱與賊合,故令仲昇陷賊三年。代宗含怒久之,因是下詔曰:



 《春秋》之義,貴在於必書;君臣之間,法存於無赦。沮勸式遵於前典,進退莫匪於至公,惡稔既彰,明罰難貸。開府儀同三司、行兵部尚書、中書門下平章事、充山南東道節度觀察處置等
 使、上柱國、潁國公來瑱,謬當任用,素乏器能,亟歷班榮,累經節制。蒞職蔑聞於成績,登朝虛美於崇名。頃者分閫頒條,久淹江、漢。或頻征不至,或移鎮遲留,實乖堂陛之儀,爰及干戈之忿。朕以舊臣宿將,道在含弘,會其來庭,用甄後效。超登宰輔,光拜夏卿,列在三臺,掩其一眚。山陵先遠,事委近臣,謀謨素闕於大猷,卜祝頗聞於私議。實虧周慎,且間樞言,何以輔弼鼎司,儀刑簪紱?據其所犯,合置殊科。以嘗侍軒闥,用存寬免
 之辜;緬範舊章,兼膺黜削之譴。其身官爵,一切削除。



 寶應二年正月,貶播州縣尉員外置。翌日,賜死於鄠縣,籍沒其家。瑱之被刑也,門客四散,掩於坎中。校書郎殷亮後至,獨哭於尸側,貨所乘驢以備棺衾,夜詣縣令長孫演以情告之,演義而從之。亮夜葬而祭,走歸京師。代宗既悟元振之誣構,積其過而配流溱州。



 先是,瑱行軍司馬龐充統兵二千人赴河南,至汝州,聞瑱死,將士魚目等回兵襲襄州,左兵馬使李昭御之,奔房州。昭及薛南陽與右兵馬使
 梁崇義不葉相圖,為崇義所殺。朝廷授崇義節度使、兼御史中丞以代瑱。崇義瑱立祠,四時拜饗,不居瑱及正堂視事,於東廂下構一小室而寢止,抗疏哀請收葬,優詔許之。廣德元年,追復官爵。



 周智光,本以騎射從軍,常有戎捷,自行間登偏裨。宦官魚朝恩為觀軍容使,鎮陜州,與之暱狎。朝恩以扈從功,恩渥崇厚,奏請多允,屢於上前賞拔智光,累遷華州刺史、同華二州節度使及潼關防禦使,加檢校工部尚書、
 兼御史大夫。



 永泰元年,吐蕃、回紇、黨項薔、渾、奴束刂十餘萬眾寇奉天、醴泉等縣,智光邀戰,破於澄城,收駝馬軍資萬計,因逐賊至鄜州。智光與杜冕不協,遂殺鄜州刺史張麟,坑杜冕家屬八十一人,焚坊州廬舍三千餘家。懼罪,召不赴命。朝廷外示優容,俾杜冕使梁州,實避仇也。



 永泰二年十二月,智光專殺前虢州刺史、兼御史中孫龐充。充方居縗絰,潛行,智光追而斬之。又劫諸節度使進奉貨物及轉運米二萬石,據州反。智光自鄜坊專
 殺,朝廷患之,遂聚亡命不逞之徒,眾至數萬,縱其剽掠,以結其心。初,與陜州節度使皇甫溫不協,監軍張志斌自陜入奏,智光館給禮慢,志斌責其不肅。智光大怒曰:「僕固懷恩豈有反狀!皆由爾鼠輩作福作威,懼死不敢入朝。我本不反,今為爾作之。」因叱下斬之,臠其肉以飼從者。時淮南節度使、檢校右僕射崔圓入覲,方物百萬,智光強留其半。舉選之士竦駭,或竊同州路以過,智光使部將邀斬於乾坑店,橫死者眾。優詔以智光為尚書
 左僕射,遣中使余元仙持告身以授之。智光受詔慢罵曰:「智光有數子,皆彎弓二百斤,有萬人敵,堪出將入相。只如挾天子令諸侯。天下只有周智光合作。」因歷數大臣之過。元仙股怵,智光增絹百匹遣之。於州郭置生祠,俾將吏百姓祈禱。



 大歷二年正月,密詔關內河東副元帥、中書令郭子儀率兵討智光,許以便宜從事。時同、華路絕,上召子儀女婿工部侍郎趙縱受口詔付子儀,縱裂帛寫詔置蠟丸中,遣家童間道達焉。子儀奉詔將出
 師,華州將士相顧攜貳。智光大將李漢惠自同州以其所管降子儀。貶智光為澧州刺史,散官勛封如故。乃聽將一百人隨身,便路赴任,其所部將士官吏,一無所問。乃以兵部侍郎張仲光為華州刺史、兼御史大夫、潼關防禦使;又以大理卿敬括為同州刺史、兼御史大夫、長春宮等使。是日,智光為帳下將斬首,並子元耀、元幹等二人來獻。丁卯,梟智光首於皇城之南街,二子腰斬以示眾。判官監察御史邵賁、都虞候蔣羅漢並伏誅,餘黨
 各以親疏準法定罪。命有司具儀奏告太清宮、太廟、七陵。時淮西節度使李忠臣入觀,次潼關,聞智光阻兵,駐所部將往御之。及智光死,忠臣進兵入華州大掠,自赤水至潼關二百里間,畜產財物殆盡,官吏至有著紙衣或數日不食者。



 史臣曰:嘗讀《李陵傳》,戰敗不能死,屈節降虜庭,君不得為忠臣,母不得為孝子,每長嘆久之。炅收滍水敗眾,守南陽孤城,每蹈危機,竟效死節,料敵雖非其良將,事君
 不失為忠臣。浮躁無行,狂悖用兵,宜其死矣。瑱善軍政,得士心,庶幾干城御侮者哉!始固名位,為裴巧言;終歸朝廷,遭元振誣構。賜死之辜匪辨,用刑之道不明。致舊將立祠,門吏偷葬,出將入相,一至於斯,惜哉!智光狂悖,不足與論。



 贊曰:魯炅竭節,來瑱枉死。裴兇人,智光逆子。



\end{pinyinscope}