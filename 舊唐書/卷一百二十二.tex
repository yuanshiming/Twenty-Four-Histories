\article{卷一百二十二}

\begin{pinyinscope}

 ○元載
 王昂李少良郇謨附王縉楊炎黎幹劉忠翼附庾準



 元載,鳳翔岐山人也,家本寒微。父景昇,任員外官,不理產業,常居岐州。載母攜載適景昇,冒姓元氏。載自幼嗜
 學,好屬文,性敏惠,博覽子史,尤學道書。家貧,徒步隨鄉賦,累上不升第。天寶初,玄宗崇奉道教,下詔求明莊、老、文、列四子之學者。載策入高科,授邠州新平尉。監察御史韋鎰充使監選黔中,引載為判官,載名稍著,遷大理評事。東都留守苗晉卿又引為判官,遷大理司直。



 肅宗即位,急於軍務,諸道廉使隨才擢用。時載避地江左,蘇州刺史、江東採訪使李希言表載為副,拜祠部員外郎,遷洪州刺史。兩京平,入為度支郎中。載智性敏悟,善奏
 對,肅宗嘉之,委以國計,俾充使江、淮,都領漕輓之任,尋加御史中丞。數月徵入,遷戶部侍郎、度支使並諸道轉運使。既至朝廷,會肅宗寢疾。載與幸臣李輔國善。輔國妻元氏,載之諸宗,因是相暱狎。時輔國權傾海內,舉無違者,會選京尹,輔國乃以載兼京兆尹。載意屬國柄,詣輔國懇辭京尹,輔國識其意,然之。翌日拜載同中書門下平章事,度支轉運使如故。旬日,肅宗晏駕,代宗即位,輔國勢愈重,稱載於上前。載能伺上意,頗承恩遇,遷中
 書侍郎、同中書門下平章事,加集賢殿大學士,修國史。又加銀青光祿大夫,封許昌縣子。載以度支轉運使職務繁碎,負荷且重,慮傷名,阻大位,素與劉晏相友善,乃悉以錢穀之務委之,薦晏自代,載自加營田使。李輔國罷職,又加判天下元帥行軍司馬。廣德元年,與宰臣劉晏、裴遵慶同扈從至陜。及輿駕還宮,遵慶皆罷所任,載恩寵彌盛。輔國死,載復結內侍董秀,多與之金帛,委主書卓英倩潛通密旨。以是上有所屬,載必先知之,承意
 探微,言必玄合,上益信任之。妻王氏狠戾自專,載出朝謁,縱子伯和等游於外,上封人顧繇奏之,上方任載以政,反罪繇而已。



 內侍魚朝恩負恃權寵,不與載協,載常憚之。大歷四年冬,乘間密奏朝恩專權不軌,請除之。朝恩驕橫,天下咸怒,上亦知之,及聞載奏,適會於心。載遂結北軍大將同謀,以防萬慮。五年三月,朝恩伏法,度支使第五琦以朝恩黨坐累,載兼判度支,志氣自若,謂己有除惡之功,是非前賢,以為文武才略,莫己之若。外委
 胥吏,內聽婦言。城中開南北二甲第,室宇宏麗,冠絕當時。又於近郊起亭榭,所至之處,帷帳什器,皆於宿設,儲不改供。城南膏腴別墅,連疆接畛,凡數十所,婢僕曳羅綺一百餘人,恣為不法,侈僭無度。江、淮方面,京輦要司,皆排去忠良,引用貪猥。士有求進者,不結子弟,則謁主書,貨賄公行,近年以來,未有其比。與王縉同列,縉方務聚財,遂睦於載,二人相得甚歡,日益縱橫。代宗盡察其跡,以載任寄多年,欲全君臣之分,載嘗獨見,上誡之,不
 悛。



 初,扈駕自陜還,與縉上表,請以河中府為中都,秋杪行幸,春首還京,以避蕃戎侵軼之患。帝初納之,遣條奏以聞。自魚朝恩就誅,志頗盈滿,遂抗表請建中都,文多不載。大略以關輔、河東等十州戶稅入奉京師,創置精兵五萬,管在中都,以威四方,辭多開合。自以為表入事行,潛遣所由吏於河中經營。



 節度寄理於涇州。大歷八年,蕃戎入邠寧之後,朝議以為三輔已西,無襟帶之固,而涇州散地,不足為守。載嘗為西州刺史,知河西、隴右
 之要害,指畫於上前曰:「今國家西境極於潘源,吐蕃防戍在摧沙堡,而原州界其間。原州當西塞之口,接隴山之固,草肥水甘,舊壘存焉。吐蕃比毀其垣墉,棄之不居。其西則監牧故地,皆有長濠巨塹,重復深固。原州雖早霜,黍稷不藝,而有平涼附其東,獨耕一縣,可以足食。請移京西軍戍原州,乘間築之,貯粟一年。戎人夏牧多在青海,羽書覆至,已逾月矣。今運築並作,不二旬可畢。移子儀大軍居涇,以為根本。分兵守石門、木峽、隴山之關,
 北抵於河,皆連山峻嶺,寇不可越。稍置鳴沙縣、豐安軍為之羽翼,北帶靈武五城為之形勢。然後舉隴右之地以至安西,是謂斷西戎之脛,朝廷可高枕矣。」兼圖其地形以獻。載密使人逾隴山,入原州,量井泉,計徒庸,車乘畚鍤之器皆具。檢校左僕射田神功沮之曰:「夫興師料敵,老將所難。陛下信一書生言,舉國從之,聽誤矣。」上遲疑不決,會載得罪乃止。



 初,六年,載條奏應緣別敕授文武六品以下,敕出後望令吏部、兵部便附甲團奏,不得
 檢勘,從之。時功狀奏擬,結銜多謬,載欲權歸於己,慮有司駁正。會有上封人李少良密以載醜跡聞,載知之,奏於上前,少良等數人悉斃於公府。由是道路以目,不敢議載之短。門庭之內,非其黨與不接,平素交友,涉於道義者悉疏棄之。



 代宗寬仁明恕,審其所由,凡累年,載長惡不悛,眾怒上聞。大歷十二年三月庚辰,仗下後,上御延英殿,命左金吾大將軍吳湊收載、縉於政事堂,各留系本所,並中書主事卓英倩、李待榮及載男仲武、季能
 並收禁,命吏部尚書劉晏訊鞫。晏以載受任樹黨,布於天下,不敢專斷,請他官共事。敕御史大夫李涵、右散騎常侍蕭昕、兵部侍郎袁傪、禮部侍郎常袞、諫議大夫杜亞同推究其狀。辯罪問端,皆出自禁中,仍遣中使詰以陰事,載、縉皆伏罪。是日,宦官左衛將軍、知內侍省事董秀與載同惡,先載於禁中杖殺之。敕曰:「任直去邪,懸於帝典;獎善懲惡,急於時政。和鼎之寄,匪易其人。中書侍郎、同中書門下平章事元載,性頗奸回,跡非正直。寵待
 逾分,早踐鈞衡。亮弼之功,未能經邦成務;挾邪之志,常以罔上面欺。陰托妖巫,夜行解禱,用圖非望,庶逭典章。納受贓私,貿鬻官秩。兇妻忍害,暴子侵牟,曾不提防,恣其凌虐。行僻辭矯,心狠貌恭,使沉抑之流,無因自達,賞罰差謬,罔不由茲。頃以君臣之間,重於去就,冀其遷善,掩而不言。曾無悔非,彌益兇戾,年序滋遠,釁惡貫盈。將肅政於朝班,俾申明於憲綱,宜賜自盡。朕涉道猶淺,知人不明,理績未彰,遺闕斯眾,致茲刑闢,憫愧良深。僶俯
 行之,務申沮勸,凡在中外,悉朕懷焉。」又制曰:「門下侍郎、同中書門下平章事王縉,附會奸邪,阿諛讒佞。據茲犯狀,罪至難容,矜以耋及,未忍加刑。俾申屈法之恩,貸以岳牧之秩。可使持節括州諸軍事,守括州刺史,宜即赴任。於戲!朕恭己南面,推誠股肱,敷求哲人,將弼予理。昧於任使,過在朕躬,無曠厥官,各慎厥職。」初,晏等承旨,縉亦處極法,晏謂涵曰:「重刑再覆,國之常典,況誅大臣,豈得不覆奏!又法有首從,二人同刑,亦宜重取進止。」涵等
 咸聽命。及晏等覆奏,上乃減縉罪從輕。



 載長子伯和,先是貶在揚州兵曹參軍,載得罪,命中使馳傳於揚州賜死。次子仲武,祠部員外郎,次子季能,秘書省校書郎,並載妻王氏並賜死。女資敬寺尼真一,收入掖庭。王氏,開元中河西節度使忠嗣之女也,素以兇戾聞,恣其子伯和等為虐。伯和恃父威勢,唯以聚斂財貨,徵求音樂為事。



 載在相位多年,權傾四海,外方珍異,皆集其門,資貨不可勝計,故伯和、仲武等得肆其
 志。輕浮之士,奔其門者,如恐不及。名姝、異樂,禁中無者有之。兄弟各貯妓妾於室,倡優偎褻之戲,天倫同觀,略無愧恥。及得罪,行路無嗟惜者。中使董秀、主書卓英倩、李待榮及陰陽人李季連,以載之故,皆處極法。遣中官於萬年縣界黃臺鄉毀載祖及父母墳墓,斫棺棄柩,及私廟木主;並載大寧里、安仁里二宅,充修百司廨宇。以載籍沒鐘乳五百兩分賜中書門下御史臺五品已上、尚書省四品已上。



 王昂者,出自戎旅,以軍功累遷河中
 尹,充河中節度使。貪縱不法,務於聚斂,以貨籓身。永泰元年正月,檢校刑部尚書知省事,改殿中少監。元載秉政,與載深相結托。大歷五年六月,為江陵尹、兼御史大夫,充荊南節度觀察使,代衛伯玉。昂既行,伯玉諷大將楊金採等拒昂,乞留伯玉,詔許之。昂復檢校刑部尚書,知省事。專事奢靡,廣修第宅,多畜妓妾,以逞其志。在刑部,雖公務有程,昂耽徇私宴,連日不視曹事。性貪吝,無愧茍得,乃鬻公廨園菜,收其錢以潤屋,甚為時論所醜。元
 載誅,貶連州刺史,遣中使監至萬州,過硤江,墜江而卒。



 李少良者,以吏用,早從使幕,因職遷殿中侍御史。罷,游京師,幹謁權貴。時元載專政,所居第宅崇侈,子弟縱橫,貨賄公行,士庶咸嫉之。少良怨不見用,乘眾怒以抗疏上聞。留少良於禁內客省,少良友人韋頌因至禁門訪少良,少良漏其言;頌不慎密,遂為載備知之,乃奏少良狂妄,詔下御史臺訊鞫。是時御史大夫缺,載以張延賞為之,屬意焉。少良以洩禁中奏議,制使陸珽同伏罪。初,
 韋頌及珽俱與少良友善,與載子弟親黨款狎。頌得少良微旨,漏於載所親,遂達於載。載密召珽問之,珽具白其狀及禁中語。載得之,奏於上前,上大怒,並付京兆府決殺。珽,國子司業善經之子也,少傳父業,頗通經史,性浮躁而疏,故及於累。



 大歷中,元載弄權自恣,人皆惡之。八年七月,晉州男子郇謨以麻辮發,持竹筐及葦席哭於東市。人問其故,對曰:「有三十字請獻於上。若無堪,便以竹筐貯尸,棄之於野。」京兆府以聞。上既召見,賜衣,
 館於禁內客省。其獻三十字,各論一事。其要者:「團」字、「監」字。團者,請罷諸州團練使;監者,請罷諸道監軍使。殿中御史楊護職居左巡,郇謨哭市,護不聞奏,上以為蔽匿,貶連州桂陽縣丞員外置。元載當承寵得志,每改張朝政,出於載手,中外共怒,當時歸咎於載,故少良封事於前,郇謨哭市於後。凡百有位,宜為明誡。



 王縉,字夏卿,河中人也。少好學,與兄維早以文翰著名。縉連應草澤及文辭清麗舉,累授侍御史、武部員外。祿
 山之亂,選為太原少尹,與李光弼同守太原,功效謀略,眾所推先,加憲部侍郎,兼本官。時兄維陷賊,受偽署,賊平,維付吏議,縉請以己官贖維之罪,特為減等。



 縉尋入拜國子祭酒,改鳳翔尹、秦隴州防禦使,歷工部侍郎、左散騎常侍。撰《玄宗哀冊文》,時稱為工。改兵部侍郎。屬平殄史朝義,河朔未安,詔縉以本官河北宣慰,奉使稱旨。廣德二年,拜黃門侍郎、同平章事、太微宮使、弘文崇賢館大學士。其年,河南副元帥李光弼薨於徐州,以縉為
 侍中、持節都統河南、淮西、山南東道諸節度行營事。縉懇讓侍中,從之,加上柱國,兼東都留守。歲餘,遷河南副元帥,請減軍資錢四十萬貫修東都殿宇。大歷三年,幽州節度使李懷仙死,以縉領幽州、盧龍節度。縉赴鎮而還,委政於燕將硃希彩。又屬河東節度辛云京卒,遂兼太原尹、北都留守、河東節度營田觀察等使。縉又讓河南副元帥、東都留守,從之。太原舊將王無縱、張奉璋等恃功,且以縉儒者易之,每事多違約束。縉一朝悉召斬
 之,將校股慄。



 二歲,罷河東歸朝,授門下侍郎、中書門下平章事。時元載用事,縉卑附之,不敢與忤,然恃才與老,多所傲忽。載所不悅,心雖希載旨,然以言辭凌詬,無所忌憚。時京兆尹黎幹者,戎州人也,數論事,載甚病之,而力不能去也。乾嘗白事於縉,縉曰:「尹,南方君子也,安知朝禮!」其慢而侮人,率如此類。



 縉弟兄奉佛,不茹葷血,縉晚年尤甚。與杜鴻漸舍財造寺無限極。妻李氏卒,舍道政里第為寺,為之追福,奏其額曰寶應,度僧三十人住
 持。每節度觀察使入朝,必延至寶應寺,諷令施財,助己修繕。初,代宗喜祠祀,未甚重佛,而元載、杜鴻漸與縉喜飯僧徒。代宗嘗問以福業報應事,載等因而啟奏,代宗由是奉之過當,嘗令僧百餘人於宮中陳設佛像,經行念誦,謂之內道場。其飲膳之厚,窮極珍異,出入乘廄焉,度支具廩給。每西蕃入寇,必令群僧講誦《仁王經》,以攘虜寇。茍幸其退,則橫加錫賜。胡僧不空,官至卿監,封國公,通籍禁中,勢移公卿,爭權擅威,日相凌奪。凡京畿之
 豐田美利,多歸於寺觀,吏不能制。僧之徒侶,雖有贓奸畜亂,敗戮相繼,而代宗信心不易,乃詔天下官吏不得箠曳僧尼。又見縉等施財立寺,窮極瑰麗,每對揚啟沃,必以業果為證。以為國家慶祚靈長,皆福報所資,業力已定,雖小有患難,不足道也。故祿山、思明毒亂方熾,而皆有子禍。僕固懷恩將亂而死;西戎犯闕,未擊而退。此皆非人事之明徵也。帝信之愈甚。公卿大臣既掛以業報,則人事棄而不修,故大歷刑政,日以陵遲,有由然也。



 五臺山有金閣寺,鑄銅為瓦,塗金於上,照耀山谷,計錢巨億萬。縉為宰相,給中書符牒,令臺山僧數十人分行郡縣,聚徒講說,以求貨利。代宗七月望日於內道場造盂蘭盆,飾以金翠,所費百萬。又設高祖已下七聖神座,備幡節、龍傘、衣裳之制,各書尊號於幡上以識之,舁出內,陳於寺觀。是日,排儀仗,百僚序立於光順門以俟之,幡花鼓舞,迎呼道路。歲以為常,而識者嗤其不典,其傷教之源始於縉也。



 李氏,初為左丞韋濟妻,濟卒,奔縉。
 縉嬖之,冒稱為妻,實妾也。又縱弟妹女尼等廣納財賄,貪猥之跡如市賈焉。元載得罪,縉連坐貶括州刺史,移處州刺史。大歷十四年,除太子賓客,留司東都。建中二年十二月卒,年八十二。



 楊炎,字公南,鳳翔人。曾祖大寶,武德初為龍門令,劉武周陷晉、絳,攻之不降,城破被害,褒贈全節侯。祖哲,以孝行有異,旌其門閭。父播,登進士第,隱居不仕,玄宗徵為諫議大夫,棄官就養,亦以孝行禎祥,表其門閭。肅宗就
 加散騎常侍,賜號玄靖先生,名在《逸人傳》。



 炎美須眉,風骨峻峙,文藻雄麗,汧、隴之間,號為小楊山人。釋褐闢河西節度掌書記。神烏令李大簡嘗因醉辱炎,至是與炎同幕,率左右反接之,鐵棒撾之二百,流血被地,幾死。節度使呂崇賁愛其才,不之責。後副元帥李光弼奏為判官,不應,徵拜起居舍人,辭祿就養岐下。丁憂,廬於墓前,號泣不絕聲,有紫芝白雀之祥,又表其門閭。孝著三代,門樹六闕,古未有也。服闋久之,起為司勛員外郎,改兵
 部,轉禮部郎中、知制誥。遷中書舍人,與常袞並掌綸誥,袞長於除書,炎善為德音,自開元已來,言詔制之美者,時稱常、楊焉。



 炎樂賢下士,以汲引為己任,人士歸之。嘗為《李楷洛碑》,辭甚工,文士莫不成誦之。遷吏部侍郎,修國史。元載自作相,常選擢朝士有文學才望者一人厚遇之,將以代己。初,引禮部郎中劉單;單卒,引吏部侍郎薛邕,邕貶,又引炎。載親重炎,無與為比。載敗,坐貶道州司馬。德宗即位,議用宰相,崔祐甫薦炎有文學器用,上
 亦自聞其名,拜銀青光祿大夫、門下侍郎、同平章事。炎有風儀,博以文學,早負時稱,天下翕然,望為賢相。



 初,國家舊制,天下財賦皆納於左藏庫,而太府四時以數聞,尚書比部覆其出入,上下相轄,無失遺。及第五琦為度支、鹽鐵使,京師多豪將,求取無節,琦不能禁,乃悉以租賦進入大盈內庫,以中人主之意,天子以取給為便,故不復出。是以天下公賦,為人君私藏,有司不得窺其多少,國用不能計其贏縮,殆二十年矣。中官以冗名持簿
 書,領其事者三百人,皆奉給其間,連結根固不可動。及炎作相,頓首於上前,論之曰:「夫財賦,邦國之大本,生人之喉命,天下理亂輕重皆由焉。是以前代歷選重臣主之,猶懼不集,往往覆敗,大計一失,則天下動搖。先朝權制,中人領其職,以五尺宦豎操邦之本,豐儉盈虛,雖大臣不得知,則無以計天下利害。臣愚待罪宰輔,陛下至德,惟人是恤,參校蠹弊,無斯之甚。請出之以歸有司,度宮中經費一歲幾何,量數奉入,不敢虧用。如此,然後可
 以議政。惟陛下察焉。」詔曰:「凡財賦皆歸左藏庫,一用舊式,每歲於數中量進三五十萬入大盈,而度支先以其全數聞。」炎以片言移人主意,議者以為難,中外稱之。



 初定令式,國家有租賦庸調之法。開元中,玄宗修道德,以寬仁為理本,故不為版籍之書,人戶浸溢,堤防不禁。丁口轉死,非舊名矣;田畝移換,非舊額矣;貧富升降,非舊第矣。戶部徒以空文總其故書,蓋得非當時之實。舊制,人丁戍邊者,蠲其租庸,六歲免歸。玄宗方事夷狄,戍者
 多死不返,邊將怙寵而諱,不以死申,故其貫籍之名不除。至天寶中,王鉷為戶口使,方務聚斂,以丁籍且存,則丁身焉往,是隱課而不出耳。遂案舊籍,計除六年之外,積徵其家三十年租庸。天下之人苦而無告,則租庸之法弊久矣。迨至德之後,天下兵起,始以兵役,因之饑癘,徵求運輸,百役並作,人戶凋耗,版圖空虛。軍國之用,仰給於度支、轉運二使;四方征鎮,又自給於節度、都團練使。賦斂之司數四,而莫相統攝,於是綱目大壞,朝廷不
 能覆諸使,諸使不能覆諸州,四方貢獻,悉入內庫。權臣猾吏,因緣為奸,或公托進獻,私為贓盜者動萬萬計。河南、山東、荊襄、劍南有重兵處,皆厚自奉養,王賦所入無幾。吏職之名,隨人署置;俸給厚薄,由其增損。故科斂之名凡數百,廢者不削,重者不去,新舊仍積,不知其涯。百姓受命而供之,瀝膏血,鬻親愛,旬輸月送無休息。吏因其苛,蠶食千人。凡富人多丁者,率為官為僧,以色役免;貧人無所入則丁存。故課免於上,而賦增於下。是以天
 下殘瘁,蕩為浮人,鄉居地著者百不四五,如是者殆三十年。



 炎因奏對,懇言其弊,乃請作兩稅法,以一其名,曰:「凡百役之費,一錢之斂,先度其數而賦於人,量出以制入。戶無主客,以見居為簿;人無丁中,以貧富為差。不居處而行商者,在所郡縣稅三十之一,度所與居者均,使無僥利。居人之稅,秋夏兩征之,俗有不便者正之。其租庸雜徭悉省,而丁額不廢,申報出入如舊式。其田畝之稅,率以大歷十四年墾田之數為準而均徵之。夏稅
 無過六月,秋稅無過十一月。逾歲之後,有戶增而稅減輕,及人散而失均者,進退長吏,而以尚書度支總統焉。」德宗善而行之,詔諭中外。而掌賦者沮其非利,言租庸之令四百餘年,舊制不可輕改。上行之不疑,天下便之。人不土斷而地著,賦不加斂而增入,版籍不造而得其虛實,貪吏不誡而奸無所取。自是輕重之權,始歸於朝廷。



 炎救時之弊,頗有嘉聲。蒞事數月,屬崔祐甫疾病,多不視事,喬琳罷免,炎遂獨當國政。祐甫之所制作,炎隳之。
 初減薄護作元陵功優,人心始不悅。又專意報恩復仇。道州錄事參軍王沼有微恩於炎,舉沼為監察御史。感元載恩,專務行載舊事以報之。初,載得罪,左僕射劉晏訊劾之,元載誅,炎亦坐貶,故深怨晏。晏領東都、河南、江淮、山南東道轉運、租庸、青苗、鹽鐵使,炎作相數月,欲貶晏,先罷其使,天下錢穀皆歸金部、倉部。又獻議開豐州陵陽渠,發京畿人夫於西城就役,閭里騷擾,事竟無成。



 初,大歷末,元載議請城原州,以遏西番入寇之沖要,事
 未行而載誅。及炎得政,建中二年二月,奏請城原州,先牒涇原節度使段秀實,令為之具。秀實報曰:「凡安邊卻敵之長策,宜緩以計圖之,無宜草草興功也。又春事方作,請待農隙而緝其事。」炎怒,徵秀實為司農卿。以邠寧別駕李懷光居前督作,以檢校司空平章事硃泚、御史大夫平章事崔寧各統兵萬人以翼後。三月,詔下涇州為具。涇軍怒而言曰:「吾曹為國西門之屏,十餘年矣!始治於邠,才置農桑,地著之安;而徙於此,置榛莽之中,手
 披足踐,才立城壘;又投之塞外,吾何罪而置此乎!」李懷光監朔方軍,法令嚴峻,頻殺大將。涇州裨將劉文喜因人怨怒,拒不受詔,上疏復求段秀實為帥,否則硃泚。於是以硃泚代懷光,文喜又不奉詔。涇有勁兵二萬,閉城拒守,令其子入質吐蕃以求援。時方炎旱,人情騷動,群臣皆請赦文喜,上皆不省。德宗減服御以給軍人,城中軍士當受春服,賜與如故。命硃泚、李懷光等軍攻之,乃築壘環之。涇州別將劉海賓斬文喜首,傳之闕下。茍非
 海賓效順,必生邊患,皆因炎以喜怒易帥,涇帥結怨故也。原州竟不能城。



 炎既構劉晏之罪貶官,司農卿庾淮與晏有隙,乃用準為荊南節度使,諷令誣晏以忠州叛,殺之,妻子徙嶺表,朝野為之側目。李正己上表請殺晏之罪,指斥朝廷。炎懼,乃遣腹心分往諸道:裴冀,東都、河陽、魏博;孫成,澤潞、磁邢、幽州;盧東美,河南、淄青;李舟,山南、湖南;王定,淮西。聲言宣慰,而意實說謗。且言「晏之得罪,以昔年附會奸邪,謀立獨孤妃為皇后,上自惡之,非
 他過也。」或有密奏「炎遣五使往諸鎮者,恐天下以殺劉晏之罪歸己,推過於上耳。」乃使中人復炎辭於正己,還報信然。自此德宗有意誅炎矣,待事而發。乃擢用盧杞為門下侍郎、平章事,炎轉中書侍郎,仍平章事。二人同事秉政,杞無文學,儀貌寢陋,炎惡而忽之,每托疾息於他閣,多不會食,杞亦銜恨之。舊制,中書舍人分押尚書六曹,以平奏報,開元初廢其職,杞請復之,炎固以為不可。杞益怒,又密啟中書主書過,逐之。炎怒曰:「主書,吾局
 吏也,有過吾自治之,奈何而相侵?」



 屬梁崇義叛換,德宗欲以淮西節度使李希烈統諸軍討之。炎諫曰:「希烈始與李忠臣為子,親任無雙,竟逐忠臣而取其位,背本若此,豈可信也!居常無尺寸功,猶強不奉法,異日平賊後,恃功邀上,陛下何以馭之?」初,炎之南來,途經襄、漢,固勸崇義入朝,崇義不能從,已懷反側。尋又使其黨李舟使馳說,崇義固而拒命,遂圖叛逆,皆炎迫而成之。至是,德宗欲假希烈兵勢以討崇義,然後別圖希烈。炎又固言
 不可,上不能平,乃曰:「朕業許之矣,不能食言。」遂以希烈統諸軍。



 會德宗嘗訪宰相群臣中可以大任者,盧杞薦張鎰、嚴郢,而炎舉崔昭、趙惠伯。上以炎論議疏闊,遂罷炎相,為左僕射。後數日中謝,對於延英,及出,馳歸,不至中書,盧杞自是益怒焉。杞尋引嚴郢為御史大夫。初,郢為京兆尹,不附炎,炎怒之,諷御史張著彈郢,郢罷兼御史中丞。炎又夙聞源休與郢有隙,乃拔休自流人為京兆尹,令伺郢過。休蒞官後,與郢友善,炎大怒。張光晟方
 謀議殺回紇酋帥,炎乃以休為入回紇使,休幾為虜所殺。郢尋坐以度田不實,改為大理卿,時人惜之。至是,杞因群情所欲,又知郢與炎有隙,故引薦之。



 炎子弘業不肖,多犯禁,受賂請托,郢按之,兼得其他過。初,炎將立家廟,先有私第在東都,令河南尹趙惠伯貨之,惠伯為炎市為官廨。時惠伯自河中尹、都團練觀察等使初受代,郢奏追捕惠伯詰案。御史以炎為宰相,抑吏貨市私第,貴估其宅,賤入其幣,計以為贓。杞召大理正田晉評罪,
 晉曰:「宰臣於庶官,比之監臨,官市賈有羨利,計其利以乞取論罪,當奪官。」杞怒,謫晉衡州司馬。更召他吏繩之,曰:「監主自盜,罪絞。」開元中,蕭嵩將於曲江南立私廟,尋以玄宗臨幸之所,恐置廟非便,乃罷之。至是,炎以其地為廟,有飛語者云:「此地有王氣,炎故取之,必有異圖。」語聞,上愈怒。及臺司上具獄,詔三司使同覆之。建中二年十月,詔曰:「尚書左僕射楊炎,托以文藝,累登清貫。雖謫居荒服,而虛稱猶存。朕初臨萬邦,思弘大化,務擢非次,
 招納時髦。拔自郡佐,登於鼎司,獨委心膂,信任無疑。而乃不思竭誠,敢為奸蠹,進邪醜正,既偽且堅,黨援因依,動涉情故。隳法敗度,罔上行私,茍利其身,不顧於國。加以內無訓誡,外有交通,縱恣詐欺,以成贓賄。詢其事跡,本末乖謬,蔑恩棄德,負我何深!考狀議刑,罪在難宥。但以朕於將相,義切始終,顧全大體,特有弘貸,俾從遠謫,以肅具僚。可崖州司馬同正,仍馳驛發遣。」去崖州百里賜死,年五十五。



 炎早有文章,亦勵志節,及為中書舍人,
 附會元載,時議已薄之。後坐載貶官,憤恚益甚,歸而得政,睚眥必仇,險害之性附於心,唯其愛憎,不顧公道,以至於敗。惠伯亦坐炎貶費州多田尉,尋亦殺之。



 黎幹者,戎州人。始以善星緯數術進,待詔翰林,累官至諫議大夫。尋遷京兆尹,以嚴肅為理,人頗便之,而因緣附會,與時上下。大歷二年,改刑部侍郎。魚朝恩伏誅,坐交通出為桂州刺史、本管觀察使。至江陵,丁母憂。久之,會京兆尹缺,人頗思幹。八年,復拜京兆尹、兼御史大夫。
 干自以得志,無心為理,貪暴益甚,徇於財色。十三年,除兵部侍郎。性險,挾左道,結中貴,以希主恩,代宗甚惑之。時中官劉忠翼寵任方盛,乾結之素厚,嘗通其奸謀。及德宗初即位,干猶以詭道求進,密居輿中詣忠翼第。事發,詔曰:「兵部侍郎黎幹,害若豺狼,特進劉忠翼,掩義隱賊,並除名長流。」即行,市里兒童數千人噪聚,懷瓦礫投擊之,捕賊尉不能止,遂皆賜死於藍田驛。



 忠翼,宦官也,本名清潭,與董秀皆有寵於代宗。天憲在口,勢回日月,
 貪饕納賄,貨產巨萬。大歷中,德宗居東宮,干及清潭嘗有奸謀動搖。及是,積前罪以誅之。



 庾準,常州人。父光先,天寶中,文部侍郎。準以門入仕,暱於宰相王縉,縉驟引至職方郎中、知制誥,遷中書舍人。準素寡文學,以柔媚自進,既非儒流,甚為時論所薄。尋改御史中丞,遷尚書左丞。縉得罪,出為汝州刺史。復入為司農卿,與楊炎厚善。炎欲殺劉晏,知準與晏有隙,乃用為荊南節度。準乃上言得晏與硃泚書,且有怨望,
 又召補州兵以拒命。於是先殺晏,然後下詔賜自盡,海內冤之。炎以殺晏征準為尚書左丞。建中三年六月丁巳卒,時年五十一。贈工部尚書。



 史臣曰:仲尼云:富與貴是人之欲,不以道得之不處。反乎是道者小人。載諂輔國以進身,弄時權而固位,眾怒難犯,長惡不悛,家亡而誅及妻兒,身死而殃及祖禰。縉附會奸邪,以至顛覆。炎隳崔祐甫之規,怒段秀實之直,酬恩報怨,以私害公。三子者咸著文章,殊乖德行。「不常
 其德,或承之羞」,大《易》之義也。富貴不以其道,小人之事哉!觀庾準之憸,遭王縉之復,徇楊炎之意,曲致劉晏之冤。積惡而獲令終者,其在餘殃乎!



 贊曰:載、縉、炎、準,交相附會。《左傳》有言,貪人敗類。



\end{pinyinscope}