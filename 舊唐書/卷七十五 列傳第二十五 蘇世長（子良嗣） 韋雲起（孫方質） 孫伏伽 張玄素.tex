\article{卷七十五 列傳第二十五 蘇世長(子良嗣) 韋雲起(孫方質) 孫伏伽 張玄素}

\begin{pinyinscope}

 ○蘇世長子
 良嗣
 韋雲起孫方質孫伏伽張玄素



 蘇世長,雍州武功人也。祖彤,後魏直散騎常侍。父振,周宕州刺史、建威縣侯。周武帝時,世長年十餘歲,上書言
 事。武帝以其年小,召問:「讀何書?」對曰:「讀《孝經》、《論語》。」武帝曰:「《孝經》、《論語》何所言?」對曰:「《孝經》云:『為國者不敢侮於鰥寡。』《論語》云:『為政以德。』」武帝善其對,令於獸門館讀書。以其父歿王事,因令襲爵,世長於武帝前擗踴號泣,武帝為之改容。隋文帝受禪,世長又屢上便宜,頗有補益,超遷長安令。大業中,為都水少監,使於上江督運。會江都難作,世長為煬帝發喪慟哭,哀感路人。王世充僭號,署為太子太保、行臺右僕射。與世充兄子弘烈及將豆盧
 褒俱鎮襄陽。時弘烈娶褒女為妻,深相結托。高祖與褒有舊,璽書諭之,不從,頻斬使者。武德四年,洛陽平,世長首勸弘烈歸降。既至京師,高祖誅褒而責世長來晚之故,世長頓顙曰:「自古帝王受命,為逐鹿之喻,一人得之,萬夫斂手。豈有獲鹿之後,忿同獵之徒,問爭肉之罪也?陛下應天順人,布德施惠,又安得忘管仲、雍齒之事乎!且臣武功之士,經涉亂離,死亡略盡,惟臣殘命,得見聖朝,陛下若復殺之,是絕其類也。實望天恩,使有遺種。」高
 祖與之有故,笑而釋之。尋授玉山屯監。後於玄武門引見,語及平生,恩意甚厚。高祖曰:「卿自謂諂佞耶,正直耶?」對曰:「臣實愚直。」高祖曰:「卿若直,何為背世充而歸我?」對曰:「洛陽既平,天下為一,臣智窮力屈,始歸陛下。向使世充尚在,臣據漢南,天意雖有所歸,人事足為勍敵。」高祖大笑。嘗嘲之曰:「名長意短,口正心邪,棄忠貞於鄭國,忘信義於吾家。」世長對曰:「名長意短,實如聖旨;口正心邪,未敢奉詔。昔竇融以河西降漢,十世封侯;臣以山南歸
 國,惟蒙屯監。」即日擢拜諫議大夫。從幸涇陽校獵,大獲禽獸於旌門。高祖入御營,顧謂朝臣曰:「今日畋樂乎?」世長進曰:「陛下游獵,薄廢萬機,不滿十旬,未為大樂。」高祖色變,既而笑曰:「狂態發耶?」世長曰:「為臣私計則狂,為陛下國計則忠矣。」及突厥入寇,武功郡縣,多失戶口,是後下詔將幸武功校獵。世長又諫曰:「突厥初入,大為民害,陛下救恤之道猶未發言,乃於其地又縱畋獵,非但仁育之心有所不足,百姓供頓,將何以堪?」高祖不納。又嘗
 引之於披香殿,世長酒酣,奏曰:「此殿隋煬帝所作耶?是何雕麗之若此也?」高祖曰:「卿好諫似真,其心實詐。豈不知此殿是吾所造,何須設詭疑而言煬帝乎?」對曰:「臣實不知。但見傾宮鹿臺琉璃之瓦,並非受命帝王愛民節用之所為也。若是陛下作此,誠非所宜。臣昔在武功,幸常陪侍,見陛下宅宇,才蔽風霜,當此之時,亦以為足。今因隋之侈,民不堪命,數歸有道,而陛下得之,實謂懲其奢淫,不忘儉約。今初有天下,而於隋宮之內,又加雕飾,
 欲撥其亂,寧可得乎?」高祖深然之。後歷陜州長史、天策府軍諮祭酒。秦府初開文學館,引為學士。與房玄齡等一十八人皆蒙圖畫,令文學褚亮為之贊,曰:「軍諮諧噱,超然辯悟。正色於庭,匪躬之故。」貞觀初,聘於突厥,與頡利爭禮,不受賂遺,朝廷稱之。出為巴州刺史,覆舟溺水而卒。世長機辯有學,博涉而簡率,嗜酒,無威儀。初在陜州,部內多犯法,世長莫能禁,乃責躬引咎,自撻於都街。伍伯嫉其詭,鞭之見血,世長不勝痛,大呼而走,觀者咸以
 為笑,議者方稱其詐。



 子良嗣,高宗時遷周王府司馬。王時年少,舉事不法,良嗣正色匡諫,甚見敬憚。王府官屬多非其人,良嗣守文檢括,莫敢有犯,深為高宗所稱。遷荊州大都督府長史。高宗使宦者緣江採異竹,將於苑中植之。宦者科舟載竹,所在縱暴。還過荊州,良嗣囚之,因上疏切諫,稱:「遠方求珍異以疲道路,非聖人抑己愛人之道。又小人竊弄威福,以虧皇明。」言甚切直。疏奏,高宗下制慰勉,遽令棄竹於江中。永淳中,為雍州長史。時
 關中大饑,人相食,盜賊縱橫。良嗣為政嚴明,盜發三日內無不擒手適。則天臨朝,遷工部尚書。尋代王德真為納言,累封溫國公。為西京留守,則天賦詩餞送,賞遇甚渥。時尚方監裴匪躬檢校西苑,將鬻苑中果菜以收其利。良嗣駁之曰:「昔公儀相魯,猶能拔葵去織,未聞萬乘之主,鬻其果菜以與下人爭利也。」匪躬遂止。無幾,追入都,遷文昌左相、同鳳閣鸞臺三品。載初元年春,罷文昌左相,加位特進,仍依舊知政事。與地官尚書韋方質不協,
 及方質坐事當誅,辭引良嗣,則天特保明之。良嗣謝恩拜伏,便不能復起,輿歸其家,詔御醫張文仲、韋慈藏往視疾。其日薨,年八十五。則天輟朝三日,舉哀於觀風門,敕百官就宅赴吊。贈開府儀同三司,益州都督,賜絹布八百段、米粟八百碩,兼降璽書吊祭。其子踐言,太常丞,尋為酷吏所陷,配流嶺南而死。追削良嗣官爵,籍沒其家。景龍元年,追贈良嗣司空。



 踐言子務玄,襲爵溫國公,開元中,為邠王府長史。



 韋雲起,雍州萬年人。伯父澄,武德初國子祭酒、綿州刺史。雲起,隋開皇中明經舉,授符璽直長。嘗因奏事,文帝問曰:「外間有不便事,汝可言之。」時兵部侍郎柳述在帝側,雲起應聲奏曰:「柳述驕豪,未嘗經事,兵機要重,非其所堪,徒以公主之婿,遂居要職。臣恐物議以陛下官不擇賢,濫以天秩加於私愛,斯亦不便之大者。」帝甚然其言,顧謂述曰:「雲起之言,汝藥石也,可師友之。」仁壽初,詔在朝文武舉人,述乃舉雲起,進授通事舍人。大業初,改
 為通事謁者。又上疏奏曰:「今朝廷之內多山東人,而自作門戶,更相剡薦,附下罔上,共為朋黨。不抑其端,必傾朝政,臣所以痛心扼腕,不能默已。謹件朋黨人姓名及奸狀如左。」煬帝令大理推究,於是左丞郎蔚之、司隸別駕郎楚之並坐朋黨,配流漫頭赤水,餘免官者九人。會契丹入抄營州,詔雲起護突厥兵往討契丹部落。啟民可汗發騎二萬,受其處分。雲起分為二十營,四道俱引,營相去各一里,不得交雜。聞鼓聲而行,聞角聲而止,自
 非公使,勿得走馬。三令五申之後,擊鼓而發,軍中有犯約者,斬紇干一人,持首以徇。於是突厥將帥來入謁之,皆膝行股戰,莫敢仰視。契丹本事突厥,情無猜忌,雲起既入其界,使突厥詐云,向柳城郡欲共高麗交易,勿言營中有隋使,敢漏洩者斬之。契丹不備。去賊營百里,詐引南度,夜復退還,去營五十里,結陣而宿,契丹弗之知也。既明,俱發,馳騎襲之,盡獲其男女四萬口,女子及畜產以半賜突厥,餘將入朝,男子皆殺之。煬帝大喜,集百
 官曰:「雲起用突厥而平契丹,行師奇譎,才兼文武,又立朝謇諤,朕今親自舉之。」擢為治書御史。雲起乃奏劾曰:「內史侍郎虞世基,職典樞要,寄任隆重;御史大夫裴蘊,特蒙殊寵,維持內外。今四方告變,不為奏聞,賊數實多,或減言少。陛下既聞賊少,發兵不多,眾寡懸殊,往皆莫克,故使官軍失利,賊黨日滋。此而不繩,為害將大,請付有司,詰正其罪。」大理卿鄭善果奏曰:「雲起詆訾名臣,所言不實,非毀朝政,妄作威權。」由是左遷大理司直。煬帝
 幸揚州,雲起告歸長安,屬義旗入關,於長樂宮謁見。義寧元年,授司農卿,封陽城縣公。武德元年,加授上開府儀同三司,判農圃監事。是歲,欲大發兵討王世充,雲起上表諫曰:「國家承喪亂之後,百姓流離,未蒙安養,頻年不熟,關內阻饑。京邑初平,物情未附,鼠竊狗盜,猶為國憂。盩厔司竹,餘氛未殄;藍田、穀口,群盜實多。朝夕伺間,極為國害。雖京城之內,每夜賊發。北有師都,連結胡寇,斯乃國家腹心之疾也。舍此不圖,而窺兵函、洛,若師出
 之後,內盜乘虛,一旦有變,禍將不小。臣謂王世充遠隔千里,山川懸絕,無能為害,待有餘力,方可討之。今內難未弭,且宜弘於度外。如臣愚見,請暫戢兵,務穡勸農,安人和眾。關中小盜,自然寧息。秦川將卒,賈勇有餘,三年之後,一舉便定。今雖欲速,臣恐未可。」乃從之。會突厥入寇,詔雲起總領豳、寧已北九州兵馬,便宜從事。四年,授西麟州刺史,司農卿如故。尋代趙郡王孝恭為夔州刺史,轉遂州都督,懷柔夷獠,咸得眾心。遷益州行臺民部
 尚書,尋轉行臺兵部尚書。行臺僕射竇軌多行殺戮,又妄奏獠反,冀得集兵。因此作威,肆其兇暴,雲起多執不從。雲起又營私產,交通生獠,以規其利,軌亦對眾言之,由是構隙,情相猜貳。隱太子之死也,敕遣軌息馳驛詣益州報軌,軌乃疑雲起弟慶儉、堂弟慶嗣及親族並事東宮,慮其聞狀或將為變,先設備而後告之。雲起果不信,問曰:「詔書何在?」軌曰:「公,建成黨也,今不奉詔,同反明矣。」遂執殺之。初,雲起年少時,師事太學博士王頗,頗每
 與之言及時事,甚嘉嘆之,乃謂之曰:「韋生識悟如是,必能自取富貴;然剛腸嫉惡,終當以此害身。」竟如頗言。子師實,垂拱初,官至華州刺史、太子少詹事,封扶陽郡公。



 師實子方質,則天初鸞臺侍郎、地官尚書、同鳳閣鸞臺平章事。時改修《垂拱格式》,方質多所損益,甚為時人所稱。俄而武承嗣、三思當朝用事,諸宰相咸傾附之。方質疾假,承嗣等詣宅問疾,方質據床不為之禮。左右云:「踞見權貴,恐招危禍。」方質曰:「吉兇命也。大丈夫豈能折節
 曲事近戚,以求茍免也。」尋為酷吏周興、來子珣所構,配流儋州,仍籍沒其家。尋卒。神龍初雪免。



 孫伏伽,貝州武城人。大業末,自大理寺史累補萬年縣法曹。武德元年,初以三事上諫。其一曰:



 臣聞天子有諍臣,雖無道不失其天下;父有諍子,雖無道不陷於不義。故云子不可不諍於父,臣不可不諍於君。以此言之,臣之事君,猶子之事父故也。隋後主所以失天下者,何也?止為不聞其過。當時非無直言之士,由君不受諫,自謂
 德盛唐堯,功過夏禹,窮侈極欲,以恣其心。天下之士,肝腦塗地,戶口減耗,盜賊日滋,而不覺知者,皆由朝臣不敢告之也。向使修嚴父之法,開直言之路,選賢任能,賞罰得中,人人樂業,誰能搖動者乎?所以前朝好為變更,不師古訓者,止為天誘其咎,將以開今聖唐也。陛下龍舉晉陽,天下響應,計不旋踵,大位遂隆。陛下勿以唐得天下之易,不知隋失之不難也。陛下貴為天子,富有天下,動則左史書之,言則右史書之。既為竹帛所拘,何可
 恣情不慎?凡有搜狩,須順四時,既代天理,安得非時妄動?陛下二十日龍飛,二十一日有獻鷂雛者,此乃前朝之弊風,少年之事務,何忽今日行之!又聞相國參軍事盧牟子獻琵琶,長安縣丞張安道獻弓箭,頻蒙賞勞。但「普天之下,莫非王土;率土之濱,莫非王臣」,陛下必有所欲,何求而不得?陛下所少者,豈此物哉!願陛下察臣愚忠,則天下幸甚。



 其二曰:



 百戲散樂,本非正聲,有隋之末,大見崇用,此謂淫風,不可不改。近者,太常官司於人間
 借婦女裙襦五百餘具,以充散妓之服,雲擬五月五日於玄武門游戲。臣竊思審,實損皇猷,亦非貽厥子孫謀,為後代法也。故《書》云:「無以小怨為無傷而弗去。」恐從小至於大故也。《論語》云:「放鄭聲,遠佞人。」又云:「樂則《韶》舞。」以此言之,散妓定非功成之樂也。如臣愚見,請並廢之。則天下不勝幸甚。



 其三曰:



 臣聞性相近而習相遠,以其所好相染也。故《書》云:「與治同道罔弗興,與亂同事罔弗亡。」以此言之,興亂其在斯與!皇太子及諸王等左右群僚,
 不可不擇而任之也。如臣愚見,但是無義之人,及先來無賴,家門不能邕睦;及好奢華馳獵馭射,專作慢游狗馬、聲色歌舞之人,不得使親而近之也。此等止可悅耳目,備驅馳,至於拾遺補闕,決不能為也。臣歷窺往古,下觀近代,至於子孫不孝,兄弟離間,莫不為左右亂之也。願陛下妙選賢才,以為皇太子僚友,如此即克隆盤石,永固維城矣。



 高祖覽之大悅,下詔曰「秦以不聞其過而亡,典籍豈無先誡?臣僕諂諛,故弗之覺也。漢高祖反正,
 從諫如流。洎乎文、景繼業,宣、元承緒,不由斯道,孰隆景祚?周、隋之季,忠臣結舌,一言喪邦,諒足深誡。永言於此,常深嘆息。朕每惟寡薄,恭膺寶命,雖不能性與天道,庶思勉力,常冀弼諧,以匡不逮。而群公卿士,罕進直言,將申虛受之懷,物所未諭。萬年縣法曹孫伏伽,至誠慷慨,詞義懇切,指陳得失,無所回避。非有不次之舉,曷貽利行之益!伏伽既懷諒直,宜處憲司,可治書侍御史。仍頒示遠近,知朕意焉。」兼賜帛三百匹。時軍國多事,賦斂
 繁重,伏伽屢奏請改革,高祖並納焉。二年,高祖謂裴寂曰:「隋末無道,上下相蒙,主則驕矜,臣惟諂佞。上不聞過,下不盡忠,至使社稷傾危,身死匹夫之手。朕撥亂反正,志在安人,平亂任武臣,守成委文吏,庶得各展器能,以匡不逮。比每虛心接待,冀聞讜言。然惟李綱善盡忠款,孫伏伽可謂誠直,餘人猶踵弊風,俯首而已,豈朕所望哉!」及平王世充、竇建德,大赦天下,既而責其黨與,並令配遷。伏伽上表諫曰:



 臣聞王言無戲,自古格言;去食存信,
 聞諸舊典。故《書》云:「爾無不信,朕不食言。」又《論語》云,一言出口,駟不及舌。以此而論,言之出口,不可不慎。伏惟陛下光臨區宇,覆育群生,率土之濱,誰非臣妾。絲綸一發,取信萬方,使聞之者不疑,見之者不惑。陛下今月二日發雲雨之制,光被黔黎,無所間然,公私蒙賴。既云常赦不免,皆赦除之,此非直赦其有罪,亦是與天下斷當,許其更新。以此言之,但是赦後,即便無事。因何王世充及建德部下,赦後乃欲遷之?此是陛下自違本心,欲遣下
 人若為取則?若欲子細推尋,逆城之內,人誰無罪?故《書》云:「殲厥渠魁,脅從罔治。」若論渠魁,世充等為首,渠魁尚免,脅從何辜?且古人云:「蹠狗吠堯,蓋非其主。」在東都城內及建德部下,乃有與陛下積小故舊,編發友朋,猶尚有人敗後始至者。此等豈忘陛下,皆云被壅故也。以此言之,自外疏者,竊謂無罪。又《書》云:「非知之艱,行之惟艱。」上古以來,何代無君,所以只稱堯、舜之善者,何也?直由為天子者實難,善名難得故也。往者天下未平,威權須
 應機而作;今四方既定,設法須與人共之。但法者,陛下自作之,還須守之,使天下百姓信而畏之。今自為無信,欲遣兆人若為信畏?故《書》云:「無偏無黨,王道蕩蕩;無黨無偏,王道平平。」賞罰之行,達乎貴賤,聖人制法,無限親疏。如臣愚見,世充、建德下偽官,經赦合免責情,欲遷配者,請並放之,則天下幸甚。



 又上表請置諫官,高祖皆納焉。



 太宗即位,賜爵樂安縣男。貞觀元年,轉大理少卿。太宗嘗馬射,伏伽上書諫曰:「臣聞千金之子,坐不垂堂;百
 金之子,立不倚衡。以此言之,天下之主,不可履險乘危明矣。臣又聞天子之居也,則禁衛九重;其動也,則出警入蹕。此非極尊其居處,乃為社稷生靈之大計耳。故古人云:『一人有慶,兆人賴之。』臣竊聞陛下猶自走馬射帖,娛悅近臣,此乃無禁乘危,竊為陛下有所不取也。何者?一則非光史冊,二則未足顯揚,又非所以導養聖躬,亦不可以垂範後代。此只是少年諸王之所務,豈得既為天子,今日猶行之乎?陛下雖欲自輕,其奈社稷天下何!
 如臣愚見,竊謂不可。」太宗覽之大悅。五年,坐奏囚誤失免官。尋起為刑部郎中,累遷大理少卿,轉民部侍郎。十四年,拜大理卿,後出為陜州刺史。永徽五年,以年老致仕。顯慶三年卒。



 張玄素,蒲州虞鄉人。隋末,為景城縣戶曹。竇建德攻陷景城,玄素被執,將就戮,縣民千餘人號泣請代其命,曰:「此人清慎若是,今倘殺之,乃無天也。大王將定天下,當深加禮接,以招四方,如何殺之,使善人解體?」建德遽命
 釋之,署為治書侍御史,固辭不受。及江都不守,又召拜黃門侍郎,始應命。建德平,授景城都督府錄事參軍。太宗聞其名,及即位,召見,訪以政道。對曰:「臣觀自古以來,未有如隋室喪亂之甚,豈非其君自專,其法日亂。向使君虛受於上,臣弼違於下,豈至於此?且萬乘之重,又欲自專庶務,日斷十事而五條不中,中者信善,其如不中者何?況一日萬機,己多虧失,以日繼月,乃至累年,乖謬既多,不亡何待!如其廣任賢良,高居深視,百司奉職,誰
 敢犯之?臣又觀隋末沸騰,被於宇縣,所爭天下者不過十數人,餘皆保邑全身,思歸有道。是知人欲背主為亂者鮮矣,但人君不能安之,遂致於亂。陛下若近覽危亡,日慎一日,堯、舜之道,何以能加!」太宗善其對,擢拜侍御史,尋遷給事中。貞觀四年,詔發卒修洛陽宮乾陽殿,以備巡幸。玄素上書諫曰:



 微臣竊思秦始皇之為君也,藉周室之餘、六國之盛,將貽之萬葉,及其子而亡,良由逞嗜奔欲,逆天害人者也。是知天下不可以力勝,神祗不
 可以親恃,惟當弘儉約,薄賦斂,慎終如始,可以永固。方今承百王之末,屬凋弊之餘,必欲節之以禮制,陛下宜以身為先。東都未有幸期,即何須補葺?諸王今並出籓,又須營構,興發漸多,豈疲人之所望?其不可一也。陛下初平東都之始,層樓廣殿,皆令撤毀,天下翕然,同心欣仰。豈有初則惡其侈靡,今乃襲其雕麗?其不可二也。每承音旨,未即巡幸,此則事不急之務,成虛費之勞。國無兼年之積,何用兩都之好,勞役過度,怨讟將起。其不可三
 也。百姓承亂離之後,財力凋盡,天恩含育,粗見存立,饑寒猶切,生計未安,三五年間,恐未平復。奈何營未幸之都,奪疲人之力?其不可四也。昔漢高祖將都洛陽,婁敬一言,即日西駕,豈不知地惟土中,貢賦所均,但以形勝不如關內也。伏惟陛下化凋弊之人,革澆漓之俗,為日尚淺,未甚淳和。斟酌事宜,詎可東幸?其不可五也。臣又嘗見隋室造殿,楹棟宏壯,大木非隨近所有,多從豫章採來。二千人曳一柱,其下施轂,皆以生鐵為之,若用
 木輪,便即火出。鐵轂既生,行一二里即有破壞,仍數百人別齎鐵轂以隨之,終日不過進三二十里。略計一柱,已用數十萬功,則餘費又過於此。臣聞阿房成,秦人散;章華就,楚眾離;及乾陽畢功,隋人解體。且以陛下今時功力,何如隋日?役瘡痍之人,襲亡隋之弊,以此言之,恐甚於煬帝。深願陛下思之,無為由余所笑,則天下幸甚。



 太宗曰:「卿謂我不如煬帝,何如桀、紂?」對曰:「若此殿卒興,所謂同歸於亂。且陛下初平東都,太上皇敕大殿高門
 並宜焚毀,陛下以瓦木可用,不宜焚灼,請賜與貧人。事雖不行,然天下翕然謳歌至德。今若遵舊制,即是隋役復興。五六年間,趨舍頓異,何以昭示子孫,光敷四海?」太宗嘆曰:「我不思量,遂至於此。」顧謂房玄齡曰:「洛陽土中,朝貢道均,朕故修營,意在便於百姓。今玄素上表,實亦可依,後必事理須行,露坐亦復何苦,所有作役,宜即停之。然以卑乾尊,古來不易,非其忠直,安能若此?可賜彩二百匹。」侍中魏徵嘆曰:「張公論事,遂有回天之力,可謂
 仁人之言,其利博哉!」累遷太子少詹事,轉右庶子。



 時承乾居春宮,頗以游畋廢學,玄素上書諫曰:「臣聞皇天無親,惟德是輔,茍違天道,人神同棄。然古三驅之禮,非欲教殺,將為百姓除害,故湯羅一面,天下歸仁。今苑中娛獵,雖名異游畋,若行之無常,終虧雅度。且傅說曰:『學不師古,匪說攸聞。』然則弘道在於學古,學古必資師訓。既奉恩詔,令孔穎達侍講,望數存問,以補萬一。仍博遣有名行學士,兼朝夕侍奉。覽聖人之遺教,察既行之往事,
 日知其所不足,月無忘其所能。此則盡善盡美,夏啟、周誦,焉足言哉!夫為人上者,未有不求其善,但以性不勝情,耽惑成亂。耽惑既甚,忠言遂塞,所以臣下茍順,君道漸虧。古人有言:『勿以小惡而不去,小善而不為。』故知禍福之來,皆起於漸。殿下地居儲兩,當須廣樹嘉猷。既有好畋之淫,何以主斯匕鬯?慎終如始,猶懼漸衰,始尚不慎,終將安保!」尋又兼太子少詹事。十三年,又上書諫曰:「臣聞周公以大聖之材,猶握發吐飧,引納白屋,而況後
 之聖賢,敢輕斯道?是以禮制皇太子入學而行齒胄,欲使太子知君臣、父子、長幼之道。然君臣之義、父子之親、尊卑之序、長幼之節,用之方寸之內,弘之四海之外,皆因行以遠聞,假言以光被。伏惟殿下睿質已隆,尚須學文以飾其表。至如孔穎達、趙弘智等,非惟宿德鴻儒,亦兼達政要,望令數得侍講,開釋物理,覽古諭今,增暉睿德。而雕蟲小伎之流,只可時命追隨,以代博弈耳。若其騎射畋游,酣歌戲玩,以悅耳目,終穢心神,漸染既久,必
 移情性。古人有言:『心為萬事主,動而無節即亂。』臣恐殿下敗德之源,在於此矣。」承乾並不能納。太宗知玄素在東宮頻有進諫,十四年,擢授銀青光祿大夫,行太子左庶子。時承乾久不坐朝,玄素諫曰:「宮內止有婦人耳,不知如樊姬之徒,可與弘益聖德者有幾?若遂無賢哲,便是親嬖幸,遠忠良。人不見德,何以光敷三善?且宮儲之寄,於國為重,所以廣置群僚,以輔睿德。今乃動經時月,不見宮臣,納誨既疏,將何補闕?」承乾嫉其數諫,遣戶奴
 夜以馬撾擊之,殆至於死。承乾又嘗於宮中擊鼓,聲聞於外,玄素叩閣請見,極言切諫,承乾乃出宮內鼓,對玄素毀之。是歲,太宗嘗對朝問玄素歷官所由,玄素既出自刑部令史,甚以慚恥。諫議大夫褚遂良上疏曰:「臣聞君子不失言於人,聖主不戲言於臣。言則史書之,禮成之,樂歌之。居上能禮其臣,臣始能盡力以奉其上。近代宋孝武輕言肆口,侮弄朝臣,攻其門戶,乃至狼狽。良史書之,以為非是。陛下昨見問張玄素云:『隋任何官?』奏云:『
 縣尉。』又問:『未為縣尉已前?』奏云:『流外。』又問:『在何曹司?』玄素將出閣門,殆不能移步,精爽頓盡,色類死灰。朝臣見之,多所驚怪。大唐創歷,任官以才;卜祝庸保,量能使用。陛下禮重玄素,頻年任使,擢授三品,翼贊皇儲,自不可更對群臣,窮其門戶,棄昔日之殊恩,成一朝之愧恥。人君之御臣下也,禮義以導之,惠澤以驅之,使其負戴玄天,罄輸臣節,猶恐德禮不加,人不自勵。若無故忽略,使其羞慚,鬱結於懷,衷心靡樂,責其伏節死義,其可得乎?」
 書奏,太宗謂遂良曰:「朕亦悔此問,今得卿疏,深會我心。」承乾既敗德日增,玄素又上書諫曰:



 臣聞孔子云:「能近取譬,可謂仁之方也已。」然《書》、《傳》所載,言之或遠,尋覽近事,得失斯存。至如周武帝平定山東,卑宮菲食,以安海內。太子贇舉措無端,穢德日著。烏九軌知其不可,具言於武帝;武帝慈仁,望其漸改。及至踐祚,狂暴肆情,區宇崩離,宗祀覆滅,即隋文帝所代是也。文帝因周衰弱,憑藉女資,雖無大功於天下,然布德行仁,足為萬姓所賴。
 勇為太子,不能近遵君父之節儉,而務驕侈,今之山池遺跡,即殿下所親睹是也。此時亦恃君親之恩,自謂太山之固,詎知邪臣敢進其說?向使動靜有常,進退合度,親君子,疏小人,舍浮華,尚恭儉,雖有邪臣間之,何能致慈父之隙?豈不由積德未弘,令聞不著,讒言一至,遂成其禍?竊惟皇儲之寄,荷戴殊重,如其積德不弘,何以嗣守成業?聖上以殿下親則父子,事兼家國,所應用物,不為節限。恩旨未逾六旬,用物已過七萬,驕奢之極,孰雲
 過此。龍樓之下,惟聚工匠;望苑之內,不睹賢良。今言孝敬則闕視膳問安之禮,語恭順則違君父慈訓之方,求風聲則無愛學好道之實,觀舉措則有因緣誅戮之罪。宮臣正士,未嘗在側;群邪淫巧,暱近深宮。愛好者皆游手雜色,施與者並圖畫雕鏤。在外瞻仰,已有此失;居中隱密,寧可勝計哉!宣猷禁門,不異闤闠,朝入暮出,穢聲已遠。臣以德音日損,頻上諫書,自爾已來,縱逸尤甚。右庶子趙弘智經明行修,當今善士,臣每奏請,望數召進,
 與之談論,庶廣徽猷。令旨反有猜嫌,謂臣妄相推引。從善如流,尚恐不逮;飾非拒諫,必招敗損。方崇閉塞之源,不慕欽明之術,雖抱睿哲之資,終罹罔念之咎。古人云:「苦藥利病,苦言利行。」伏惟居安思危,日慎一日。



 書入,承乾不納,乃遣刺客將加屠害。俄屬宮廢,玄素隨例除史。十八年,起授潮州刺史,轉鄧州刺史。永徽中,以年老致仕。龍朔三年,加授銀青光祿大夫。麟德元年卒。



 史臣曰:伏伽上疏於高祖,玄素進言於太宗,從疏賤以
 干至尊,懷切直以明正理,可謂至難矣。既而並見抽獎,咸蒙顧遇。自非下情忠到,效匪躬之節,上聽聰明,致如流之美,孰能至於此乎?《書》曰:「木從繩則正,後從諫則聖。」斯之謂矣。世長幼而聰悟,長能規諫;雲起屏絕朋黨,罔避驕豪。歷覽言行,咸有可觀。而雲起吐茹無方,世長終成詭詐,其不令也宜哉!方諸孫、張二子,知不迨矣。



 贊曰:言為身文,感義忘身。不有忠膽,安輕逆鱗?蘇、韋果俊,伽、素忠純。悟主匡失,猗歟諍臣。



\end{pinyinscope}