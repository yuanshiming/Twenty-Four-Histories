\article{卷一百二十五}

\begin{pinyinscope}

 ○僕固懷恩梁崇義李懷光



 僕固懷恩,鐵勒部落僕骨歌濫拔延之曾孫,語訛謂之僕固。貞觀二十年,鐵勒九姓大首領率其部落來降,分置瀚海、燕然、金微、幽陵等九都督府於夏州,別為蕃
 州以御邊,授歌濫拔延為右武衛大將軍、金微都督。拔延生乙李啜拔,乙李啜拔生懷恩,世襲都督。天寶中,加左領軍大將軍同正員、特進。歷事節度王忠嗣、安思順,皆以善格鬥,達諸蕃情,有統御材,委之心腹。及安祿山反,從郭子儀討高秀巖於雲中,破之,又敗薛忠義於背度山下,抗賊七千騎,生擒忠義男,襲下馬邑郡。十五載,進軍與李光弼合勢,及史思明戰於常山、趙郡、沙河、嘉山,皆大破之,懷恩功居多。



 肅宗即位於靈武,懷恩從郭子
 儀赴行在所。時同羅部落自西京叛賊,北寇朔方,子儀與懷恩擊之。懷恩子玢領徒擊賊,兵敗而降,尋又自拔而歸,懷恩叱而斬之。將士懾駭,無不一當百,遂破同羅千餘騎於河上,盡收其器械、駝馬。肅宗雖仗朔方之眾,將假蕃兵以張形勢,乃遣懷恩與燉煌王承寀使於回紇,請兵結好。回紇可汗遂以女妻承寀,兼請公主,遣首領隨懷恩入朝。二年正月,又從子儀下馮翊、河東二郡。走偽將崔乾祐,又襲破潼關。賊將安守忠、李歸仁自京
 率眾來援,苦戰二日,官軍敗績。懷恩退至渭水,無舟楫,抱馬以渡,存者僅半,乃奔歸子儀於河東,整其餘眾。四月,子儀赴鳳翔,李歸仁以勁卒五千邀之於三原北。子儀窘急,使懷恩及王升、陳回光、渾釋之、李國貞等五將伏兵於白渠留運橋以待之,賊至伏發,歸仁大敗而走。又從子儀戰於清渠,不利,歸於鳳翔。及回紇使葉護、帝得數千騎來赴國難,南蠻、大食之卒相繼而至。肅宗乃遣廣平王為元帥,以子儀為副,而懷恩領回紇兵從之
 澧水。賊伏兵於營東,懷恩引回紇馳殺之,匹馬不歸,賊乃大潰。日暮,懷恩謂王曰:「賊必棄城走矣,請以二百騎馬追之,縛取李歸仁、田乾真、安守忠、張通儒。」王曰:「將軍戰亦疲矣,且休息,迨明而後圖之。」懷恩曰:「歸仁、守忠,天下驍賊也,聚勝而敗,此天與我也,奈何縱之不取?若使得眾,復為我患,雖悔無及。夫戰尚速,何明日為?」王固止之,令還營。懷恩又固請,往而復反,一夕四五起。遲明諜至,守忠等果逃。又從王大破賊於陜西之新店,收兩京,
 皆立殊功。以前後功加開府儀同三司、鴻臚卿同正員、同節度副使。十二月,封豐國公,食實封二百戶。



 乾元元年九月,遣九節度擊安慶緒於相州。從郭子儀領朔方行營,破安太清,下懷、衛二州,圍相州,戰愁思崗。凡經五月,常為先鋒,堅敵大陣,必經其戰,勇冠三軍。尋充都知兵馬使。及李光弼代子儀,懷恩又副之。乾元二年,進封大寧郡王,遷御史大夫、朔方行營節度。又從李光弼守河陽,破周出,擒徐璜玉、安太清,拔懷州,皆摧鋒陷敵,功
 冠諸將。其男瑒又以開府儀同三司從將兵於其軍,每深入虜陣,以勇敢聞,軍中號為「斗將」。



 懷恩為人雄毅寡言,應對舒緩,而剛決犯上,始居偏裨之中,意有不合,雖主將必詬怒之。郭子儀為帥,以寬厚容眾,素重懷恩,其麾下皆朔方蕃漢勁卒,恃功怙將,多為不法,子儀每事優容之,行師用兵,倚以輯事。而光弼持法嚴肅,法不貸下,懷恩心憚而頗不葉。上元二年,從李光弼與史思明戰於邙山,不利。肅宗以懷恩功高,恩顧特異諸將,至冬,加
 工部尚書,敕李輔國及常參官送上,太官造食以寵之。



 代宗即位,拜隴右節度,未行,改朔方行營節度,以副郭子儀。其秋,上使中官劉清潭請兵於回紇登里可汗,登裡已為史朝義誘之傾國入塞,眾號十萬,關中騷擾,上使殿中監樂子昂馳於塞上勞之,遇於忻州。先是,肅宗以寧國公主下嫁於毗伽闕可汗,毗伽闕可汗又以少子請婚,肅宗以懷恩女妻之。毗伽可汗死,少子代立,即登里可汗。登里立,以懷恩女為可敦。至是,可汗請與懷
 恩及懷恩之母相見,詔從之。懷恩嫌疑不敢,上因賜鐵券,手詔以遣之,即令其母便發。懷恩與回紇可汗相見於太原,可汗大悅,遂許助討朝義,於是進兵,歷太原、汾、晉,營於陜州以俟期。十月,詔天下兵馬元帥雍王為中軍先鋒,以懷恩為副,加同中書門下平章事,領河東、朔方節度行營及鎮西、回紇兵馬赴陜州,並令諸道節度一時齊進。懷恩與回紇左殺為先鋒,觀軍容使魚朝恩、陜州節度郭英乂為後殿,自澠池入;陳鄭節度李抱玉
 自河陽入;河南副元帥、雍王留陜州。懷恩等師至黃水,賊徒數萬,堅柵自固。懷恩陣於西原上,廣張旗幟以當之,命驍騎及回紇之眾傍南山出於東北,兩軍舉旗內應,表裏擊之,一鼓而拔,賊死者數萬。朝義領鐵騎十萬來救,陣於昭覺寺,賊皆殊死決戰,短兵既接,相殺甚眾。官軍驟擊之,賊陣而不動。魚朝恩令射生五百人下馬,弓弩亂發,多中賊而死,陣亦如初。鎮西節度使馬璘曰:「事急矣!」遂援旗而進,單騎奔擊,奪賊兩牌,突入萬眾之
 中,左右披靡,大軍乘之而入,朝義大敗,斬首一萬六千級,生擒四千六百人,降者三萬二千人。轉戰於石榴園、老君廟,賊黨又敗,人馬蹂踐,填於尚書谷,朝義輕騎而走。懷恩乃進收東京及河陽城,封其府庫,偽中書令許叔冀、王伷等,承制釋之,悉皆安堵。



 懷恩留回紇可汗營於河陽,乃使其子右廂兵馬使瑒、北庭朔方兵馬使高輔成以步軍萬餘眾乘勝逐北。懷恩常壓賊而行,至於鄭州,再戰皆捷;進至汴州,偽節度張獻誠開門降;又
 拔滑州,追破朝義於衛州。偽睢陽節度田承嗣、李進超、李達盧等兵馬四萬餘眾,又與朝義合,據河來拒。瑒連盤濟師,登岸薄之,賊黨悉奔,長驅至昌縣東。朝義率魏州兵馬來戰,又敗走,達盧來降,賊徒震駭。於是相州偽節度薛嵩以相、衛州、洺、邢、趙降於李抱玉、高輔成、尚文悊;偽恆陽節度李寶臣以深、恆、定、易四州降於河東節度辛云京。朝義至貝州,又與偽大將薛忠義兩節度合。瑒至臨清,懼賊氣盛,駐軍以俟變。朝義領眾三萬
 並攻具來攻,瑒令高彥崇、渾日進、李光逸等設三伏以待之,賊半渡,伏發,合擊而走之。其時回紇又至,官軍益振,瑒卷甲馳之,大戰於下博縣東南。賊背水而陣,大軍沖擊而崩之,積尸擁流而下。朝義又走莫州。於是河南副元帥都知兵馬使薛兼訓、兵馬使郝廷玉、兗鄆節度使辛云京會師於下博,進軍莫州城下。朝義與田承嗣頻出挑戰,大敗而旋,臨陣殺其偽尚書敬榮。朝義懼,自分萬餘眾投歸義縣,留承嗣守城。於是淄青節度侯
 希逸繼諸將同為攻守,凡月餘日。瑒與高彥崇、侯殺逸、薛兼訓等以眾三萬追及朝義於歸義縣,交鋒而賊潰。屬幽州節度使李懷仙送降款,瑒頓兵於其境,遣懷仙分兵追躡。二年三月,朝義至平州石城縣溫泉柵,窮蹙,走入長林自縊,懷仙使妻弟徐有濟傳其首以獻。又降田承嗣之軍,河北悉平,懷恩乃與諸將班師。



 先是,去冬郭子儀以懷恩有平定河朔之功,讓位於懷恩,遂授河北副元帥、尚書左僕射、兼中書令、靈州大都督府長史、單
 于鎮北大都護、朔方節度使,仍加實封四百戶,通前一千戶。春,又加太子少師,充朔方都知兵馬使、同節度副大使,食實封五百戶,莊宅各一所,仍與一子五品官。高輔成太子少傅、兼御史中丞,充河北副元帥都知兵馬使,加實封三百戶,仍與一子五品官。高彥崇太子賓客,仍舊朔方右廂兵馬使,實封二百戶,莊宅各賜一所,與一子五品官。



 遂詔懷恩統可汗還蕃,遂自相州西郭口趣潞州,與回紇可汗會,出太原之北。懷恩初至太原,辛云
 京以可汗是其子婿,疑其召戎,閉關不報,且懼可汗相襲,不敢犒軍;及還,亦如之。懷恩父子宣力王室,攻城野戰,無役不從,一舉滅史朝義,復燕、趙、韓、魏之地,自以為功無以讓。至是,又為云京所拒,懷恩怒,上表列其狀,頓軍汾州。會中官駱奉先使於云京,云京言懷恩與可汗為約,逆狀已露,乃與奉先厚結歡。奉先回至懷恩所,其母數讓奉先曰:「爾等與我兒約為兄弟,今又親云京,何兩面乎?雖然,前事勿諭,自今母子兄弟如初。」酒酣,懷恩
 起舞,奉先贈纏頭彩。懷恩將酬其貺,奉先遽告發,懷恩曰:「明日端午,請宿為令節。」奉先固辭,懷恩苦邀之,命藏其馬。中夕謂其從者曰:「向者責吾,又收吾馬,是將害我也。」奉先懼,遂逾垣而走。懷恩驚,遽令追還其馬。奉先使回,奏其反狀。懷恩累奏請誅云京、奉先,上以云京有功,手詔和解之,懷恩遂有貳於我。至七月,改元廣德,冊勛拜太保,仍與一子三品、一子四品官並階,仍加實封五百戶。僕固瑒一子五品官,加實封一百戶。仍賜鐵券,以名
 藏太廟,畫像於凌煙閣。尋以瑒為御史大夫、朔方行營節度。



 懷恩以寇難已來,一門之內死王事者四十六人,女嫁絕域,再收兩京,皆導引回紇,摧滅強敵,而為人媒孽,蕃性獷戾,怏怏不已。乃上書自敘功伐,曰:



 廣德元年八月二十三日,開府儀同三司、尚書左僕射、兼中書令、朔方節度副大使、河北副元帥、上柱國、大寧郡王臣懷恩,刺肝瀝血,謹頓首頓首上書寶應聖文神武皇帝陛下:臣家本蕃夷,代居邊塞,爰自祖父,早沐國恩。臣年未
 弱冠,即蒙上皇驅策,出入死生,竭力疆場,叨承先帝報功,時年已授特進。洎乎祿山作亂,大振王師,臣累任偏裨,決死靖難,上以安社稷,下以拯生靈,仗皇天之威神,滅狂胡之醜類。無何,思明繼逆,又據東周,宸極不安,海內騰沸。臣謬承大行皇帝委任,授以兵權,誓雪國仇,以匡時難。闔門忠烈,咸願殺身,野戰攻城,皆先士卒。兄弟死於陣敵,子侄沒於軍前,九族之親,十不存一,縱有在者,瘡痍遍身。況陛下潛龍之時,親統師旅,臣忝事麾下,
 陛下悉臣愚誠。大行皇帝未捐宮館之時,臣頻立微效,累沾官賞,遂被輔國等讒害,幾至破家,便奪兵權,逾年宿衛。臣雖內省無疚,終懼讒佞傾危,以日繼時,命懸秋葉,至將歸骨泉壤,永謝明時。幸遇陛下龍躍天衢,繼纘鴻業,知臣負謗,察臣丹心,遂開獨見之明,杜絕眾多之口,特拔臣於汧、隴,再任臣於朔方。誠謂游魂返骸,枯骨再肉,使臣得竭駑蹇之力,效錐刀之功,上答陛下再造之恩,下展微臣犬馬之志。



 去年秋末,回紇伏義而來,士
 庶不知,悉皆驚駭。陛下以臣與其姻婭,令至太原祗迎,一切事宜,許臣逐便處置。遂與可汗計議,分道用兵,克復洛陽,平蕩幽、薊,惟有神策兵馬,頓軍獨住陳留。可汗時在洛陽,即被朝恩猜阻,要為流議,已失蕃情。臣自平賊卻回,天恩又令餞送,臣遂罄竭家產,為國周旋,發遣外蕃,貴圖上道。行至山北,被奉先、云京共生異見,妄作加諸,閉城不出祗迎,仍令潛行竊盜。蕃夷怨怒,早欲相仇,臣遂彌縫,方得出界。及其祖餞事了,回至太原,臣忝
 跡鼎司,又承重寄,奉先、云京曾無禮數,閉關不出相看。臣遂過汾州,休息士馬,凡經數日,不遣一介知聞。自以行事乘疏,恐臣先有論奏,遂乃構其謗黷,妄起異端,扇動軍城,以為設備。又臣從潞府過日,見抱玉只迎回紇,庶事用心,懇稱家資罄於公用,又與臣馬兼銀器四事,臣於回紇處得絹,便與抱玉二千匹以充答贈。今被抱玉共相組織,將此往來之貺,便為結托之私,貴在厚誣,務相傾奪。陛下不垂明察,採聽流言,欲令忠直之臣,枉
 陷讒邪之黨。臣實不欺天地,不負神明,夙夜三思,臣罪有六:



 往年同羅背叛,河曲騷然,經略數軍,兵圍不解。臣不顧老母,走投靈州,先帝嘉臣忠誠,遂遣徵兵討叛,使得河曲清泰,賊徒奔亡。是臣不忠於國,其罪一也。臣男玢嘗被同羅虜將,蓋亦制不由己,旋即棄逆歸順,卻來投臣,臣斬之以令士眾。且臣不愛骨肉之重,而徇忠義之誠,是臣不忠於國,其罪二也。臣有二女,俱聘遠蕃,為國和親,合從討難,致使賊徒殄滅。寰宇清平。是臣不忠
 於國,其罪三也。臣及男瑒,不顧危亡,身先行陣,父子效命,志寧邦家。是臣不忠於國,其罪四也。陛下委臣副元帥之權,令臣指麾河北。其新附節度使,皆握強兵,臣之撫綏,悉安反側,州縣既定,賦稅以時。是臣不忠於國,其罪五也。臣葉和回紇,戡定兇徒,天下削平,蕃夷歸國,使其永為鄰好。義著急難,萬姓安寧,干戈止息,二聖山陵事畢,陛下忠孝兩全。是臣不忠於國,其罪六也。臣既負六罪,誠合萬誅,延頸轅門,以待斧金質。過此以往,更無他
 違。陛下若以此誅臣,何異伍子胥存吳,卒浮尸於江上,大夫種霸越,終賜劍於稽山。唯當吞恨九泉,銜冤千古,復何訴哉!復何訴哉!



 且葵藿尚解仰陽,犬馬猶能戀主,臣忝恩至重,委任非輕,夙夜思奉天顏,豈暫心離魏闕,誠恐以忠獲罪,龜鏡不遙。頃者來瑱受誅,朝廷不示其罪,天下忠義,從此生疑。況來瑱功業素高,人多所忌,不審聖衷獨斷,復為奸臣弄權?臣欲入朝,恐罹斯禍,諸道節度使皆懼,非臣獨敢如此。近聞追詔數人,並皆不至,
 實畏中官讒口,又懼陛下損傷,豈唯是臣不忠,只為回邪在側。且臣前後所奏駱奉先詞情,非不摭實,陛下竟無處置,寵用彌深。皆由同類相從,致蒙蔽聖聰,人皆懼死,誰復敢言!臣義切君臣,志憂社稷,若無極諫,有負聖朝,敢肆愚忠,以幹鼎鑊。況今西有犬戎背亂,東有吳、越不庭,均、房群盜縱橫,鄜、坊稽胡草擾。陛下不思外御,而乃內忌忠良,何以混一車書,而使梯航納贐?天下至大,豈可暫輕。



 伏承四方敷奏之人,引對之時,陛下皆云與
 驃騎商量,曾不委宰臣可否。或有稽留數月,不放歸還,遠近之心,轉加疑阻。且臣朔方將士,功效最高,為先帝中興主人,是陛下蒙塵故吏,曾不別加優獎,卻信嫉妒謗詞,子儀先已被猜,臣今又遭毀黷。弓藏鳥盡,兔死犬烹,臣昔謂非,今方知實。且臣息軍汾上,關鍵大開,收馬放羊,曾無守備,分兵數郡,貴免般糧,勸課農桑,務安黎庶,有何狀跡,而涉異端。陛下必信矯詞,何殊指鹿為馬?陛下倘斥逐邪佞,親附忠良,蠲削狐疑,敷陳政化,使君
 臣無二,天下歸心,則窺邊之戎,不足為患,梗命之寇,將復何憂,偃武修文,其則不遠。陛下若不納愚懇,且貴因循,臣實不敢保家,陛下豈能安國!忠言利行,良藥愈病,伏惟陛下圖之。



 臣今戎事已安,糧儲且繼,深願一至闕下,披露心肝,再睹聖顏,萬死無恨。臣欲公然進發,慮恐將士留連。臣今便托巡晉、絳等州,於彼遷延且住,謹遣押衙開府儀同三司、試太常卿張休臧先進書兼口奏事。伏惟陛下覽臣此書,知臣誠懇,特垂聖斷,勿議近臣,
 待臣如初,浮謗不入,臣當死節王命,誓酬國恩。仍請遣一介專使至絳州問臣,臣即便與同行,冀獲蹈舞軒陛。鄙臣愚慮,不顧死亡,輕觸天威,戰汗無地。



 九月,上以回紇近塞,懷恩又與辛云京有隙,上欲其悔過,推心以待之。恐其不信,詔召黃門侍郎裴遵慶使汾州喻旨,且察其去就。遵慶即至,懷恩抱其足號泣而訴,遵慶因宣聖恩優厚,諷令入朝,懷恩許諾。副將範志誠說之曰:「公以讒言交構,有功高不賞之懼,嫌隙已成,奈何入不測之朝?
 公不見來瑱、李光弼之事乎!功成而不見容,二臣以走誅。」懷恩然之。明日,又以懼死為辭,許令一子入朝,志誠又不可。遵慶復命。御史大夫王翊自回紇使還,懷恩與可汗往來,恐洩其事,乃止之。遂令子瑒率眾攻雲京,云京出戰,瑒大敗而旋,進圍榆次,朝廷患之。先是,尚書右丞顏真卿請奉詔懷恩,上因以真卿為刑部尚書、兼御史大夫往宣慰之。真卿曰:「臣往請行者,時也;今方受命,事無益矣。」上問其故,對曰:「懷恩阻兵,是其反側明矣。頃
 陛下避狄於陜郊,臣方責以《春秋》之義,雲寡君蒙塵於郊,敢不恭問官守。當是時也,懷恩來朝,以助討賊,則其辭順。今陛下攘去犬戎,即宮京邑,懷恩進不勤王,退不釋眾,其辭曲,必不來矣。且明懷恩反者,獨辛云京、李抱玉、駱奉先、魚朝恩四人耳,自外朝臣,咸言其枉。然懷恩將士,皆子儀部曲,恩信結其心,陛下何不以子儀代之,喻以逆順禍福,必相率而歸耳。」上從之。子儀至河中,僕固瑒已為朔方兵馬使張惟岳等四人斬其首,獻於闕
 下。懷恩聞之,率麾下數百騎,棄其母,渡河北走靈武。餘眾聞子儀到,束甲來奔,歸者數萬。懷恩至靈武,嘯聚亡命,其眾復振。上念其勛舊,不欲罪功臣,厚撫其家,懷恩終不從。其母月餘日竟以壽終。又遙授太師、兼中書令、大寧王,餘並停。



 是秋為鄉導,誘吐蕃十萬入寇涇、邠州,祭來瑱之墓,自序云「俱遭放逐」。寇奉天、醴泉,郭子儀拒之而退。永泰元年,上徵天下兵以防之。懷恩又糾合諸蕃,眾號二十萬,南犯京師:遣吐蕃之眾自北道先寇醴
 泉、奉天,任敷、鄭庭、郝德自東道寇奉先、同州,羌、渾、奴剌之眾自西道寇盩厔、鳳翔。朝廷大駭,詔遣郭子儀屯涇陽,渾日進、白元光屯奉天,李光時進屯雲陽,馬璘、郝廷玉屯中渭橋,董秦屯東渭橋,駱奉先、李日越屯盩厔,李抱玉屯鳳翔,周智光、杜冕屯同州。上親率六軍,令魚朝恩屯苑中,下詔親征。懷恩領回紇及朔方之眾繼進,行至鳴沙縣,遇疾舁歸。九月九日,死於靈武,部曲以鄉法焚而葬之。張韶代領其眾,為徐璜玉所殺;璜玉領其眾,又
 為範志誠所殺,志誠領其眾。回紇進寇涇陽,諸軍堅壁不戰。吐蕃相持二十餘日,又聞懷恩死,與回紇爭長,自相疑貳,莫敢先進,遂大掠居人,焚燒舍宇,驅男女數萬而去,所過踐禾穀殆盡。回紇乃詣子儀降,請擊吐蕃以自效。子儀分兵隨之,大破吐蕃於涇州界。任敷又敗走,羌、渾又多降於李抱玉。



 懷恩逆命三年,再犯順,連諸蕃之眾,為國大患,士不解甲,糧盡饋軍,適幸天亡,而上為之隱惡,前後下制,未嘗言其反。及懷恩死,群臣以聞,上
 為之憫默曰:「懷恩不反,為左右所誤。」其寬仁如此。閏十月,懷恩侄名臣領千餘騎來降。



 梁崇義,長安人。以升斗給役於市,有膂力,能卷金舒鉤。後為羽林射生,從來瑱於襄陽。沉默寡言,眾悅之,累遷為偏裨。瑱朝京師,分使諸將戍福昌、南陽。來瑱被誅,戍者皆潰歸。崇義時在南陽,統歸師徑入襄州,與同列李昭、薛南陽相讓為長,不決。諸將請曰:「兵非梁卿主之不可。」遂推崇義為帥。寶應二年三月,崇義殺昭與南陽,以
 脅眾心,朝廷因授其節度焉。以襄州薦履兵禍,屈法含容,姑務息人也。歷御史中丞、大夫、尚書。遂與田承嗣、李正己、薛嵩、李寶臣為輔車之勢,奄有襄、漢七州之地,帶甲二萬,連結根固,未嘗朝覲,然於群兇,地最褊,兵最少,法令最理,禮貌最恭。其地跨東南之沖,數有王命之所宣洽,故其人知化。所親嘗勸其來朝,崇義曰:「吾本帥來公有大勛庸,當上元中以閹豎讒讟,逡巡稽召,及代宗嗣位,不俟駕行,旋見誅族。今吾釁盈而事久,若之何見
 上。」



 建中元年,淮西節度使李希烈數請興師討崇義,崇義懼,軍旅之事加嚴焉。流人郭昔告其為變,崇義聞之,請罪昔,坐決杖配流,命金部員外郎李舟諭旨以安之。初,劉文喜作難,舟嘗入其城說利害,文喜拘之,會帳下殺文喜而降。四方反側者聞之,謂舟必能覆軍殺將,是以皆惡。及舟至,又勸其入覲,言頗切直,崇義益不悅。二年春,發五使宣諭諸道,而舟復如荊、襄,崇義慮有變,拒境不納,上言「軍中疑懼,請換他使」。由是益不安,兇謀日
 深,賓僚或有忠言沮勸,多遭傷害。



 時群兇方自疑阻,朝廷將仗大信,欲來而安之,以示天下。仍加崇義同平章事,其妻子悉加封賞,且賜鐵券誓之,兼授其裨將藺杲為鄧州刺史,遣御史張著賚手詔征之。崇義益恐怖,使持滿而受命。藺杲奉詔書,又不敢發,馳詣崇義請命,崇義益疑懼,對著號哭,不受詔。由是征四方兵,使希烈統擊之。崇義乃發兵攻江陵,以通黔、嶺,及四望,大敗而歸,遂屯襄、鄧。希烈先發千餘人守臨漢,崇義屠之,無遺
 噍。既而希烈統大軍緣漢而上,崇義使將翟暉、杜少誠迎戰於蠻水,希烈大破之;復合於涑口,又破之。二將求降,希烈受之,使統本兵入襄陽號令,以安百姓。崇義領親兵老小閉壁,將守者斬關爭出,不可止。其年八月,崇義與其妻投井而死,傳首闕下。其親戚希烈皆戮之,選其嘗從臨漢之役者三千人,悉斬之。



 李懷光,渤海靺鞨人也。本姓茹,其先徙於幽州,父常為朔方列將,以戰功賜姓氏,更名嘉慶。懷光少從軍,以武藝壯
 勇稱,朔方節度使郭子儀禮之益厚。上元中,累遷試太僕、太常卿,主右衙兵將,積功勞至開府儀同三司,為朔方軍都虞候。永泰初,實封三百戶。大歷六年,兼御史中丞,間一年,兼御史大夫,加為軍都虞候。性清勤嚴猛,而敢誅殺,雖親戚犯法,皆不撓避。子儀性寬厚,不親軍事,紀綱任懷光,軍中尤畏之,亦稱為理。十二年,以母憂罷職。明年,起復本官,仍兼邠、寧、慶三州都將。德宗即位,罷子儀節度副元帥,以其所部分隸諸將,遂以懷光起復
 檢校刑部尚書,兼河中尹、邠州刺史、邠寧慶晉降慈隰節度支度營田觀察押諸蕃部落等使。先是,懷光頻歲率師城長武以處軍士,城據原首,臨涇水,俯瞰通道,吐蕃自是不敢南侵,為西邊要防矣。建中初,涇原四鎮節度使段秀實為宰相楊炎所惡,徵為司農卿。上將復城原州,乃以懷光兼涇州刺史、涇原四鎮北庭節度使。時懷光挾私怨,親誅殺朔方舊將溫儒雅等數人,涇州軍士咸畏之。劉文喜因眾不欲,遂以城叛。詔硃泚與懷光
 將兵討平之,加檢校太子少師。二年,遷檢校左僕射,兼靈州大都督、單于鎮北大都護、朔方節度支度營田觀察鹽池押諸蕃部落六城水運使,實封四百戶。邠寧節度等使如故。



 時馬燧、李抱真諸軍同討魏城未拔,硃滔、王武俊皆反,連兵救悅。三年,詔遣懷光統朔方兵步騎一萬五千同討田悅。懷光勇而無謀,至魏城之日,營壘未設,因與滔等大戰於愜山,為滔等所敗。復為悅決水以灌之,諸軍不利,因與燧等退軍於魏縣。尋加同平章
 事,益實封二百戶。自是與滔等相持不戰。明年十月,涇原之卒叛,上居奉天。硃泚既僭大號,遣中使馳告河北諸帥,懷光率軍奔命。時屬泥淖,懷光奮厲軍士,道自蒲津渡河,敗泚騎兵於醴泉,直赴奉天。前數日,先遣裨將張韶持表封蠟丸隨賊攻城,乘間逾塹,呼城上人曰:「朔方軍使也。」乃以繩引上城而入,比登堞,身中數十矢。時上在重圍中,守拒益急,既知懷光軍至,令張韶號令於城上,人心乃安。懷光又敗泚兵於魯店,泚乃解兵還走
 入城。



 懷光性粗厲疏愎,緣道數言盧杞、趙贊、白志貞等奸佞,且曰:「天下之亂,皆此輩也,吾見上,當請誅之。」杞等微知之,懼甚,因說上令懷光乘勝逐泚,收復京師,不可許至奉天,德宗從之。懷光屯軍咸陽,數上表暴揚杞等罪惡,上不得已為貶杞、趙贊、白志貞以慰安之。又疏中使翟文秀,上之信任也,又殺之。懷光既不敢進軍,遷延自疑,因謀為亂。初,詔遣崔漢衡使於吐蕃,出兵佐收京城,蕃相尚結贊曰:「蕃法,進軍以統兵大臣為信。今奉制
 書,無懷光名署,故不敢前。」上聞之,遣翰林學士陸贄詣懷光議用蕃軍,懷光堅執言不可者三,不肯署制,詞慢,且謂贄曰:「爾何所能?」興元元年二月,詔加太尉,兼賜鐵券,遣李升及中使鄧鳴鶴賚券喻旨。懷光怒甚,投券於地曰:「凡人臣反,則賜鐵券,今授懷光,是使反也。」詞氣益悖,眾為之懼。時懷光部將韓游瑰掌兵在奉天,懷光乃與游瑰書,約令為變,游瑰密奏之。翌日,懷光又使趣之,游瑰復奏聞。數日,懷光又使趣游瑰,為門者所捕。懷光
 且宣言曰:「吾今與硃泚連和,車駕當須引避。」由是上遽幸梁州。時李晟已移軍東渭橋,懷光復劫李建徽、楊惠元等軍,移於好畤,其下頗多攜貳。先是硃泚甚畏之,至是因欲臣之。懷光虜劫無所得,益疑懼不自安,居二旬,乃驅兵分為部隊,掠涇陽、三原、富平,自同州往河中。神策將孟涉、段威勇自三原擁兵三千餘人奔歸李晟,懷光不能遏。韓游瑰殺懷光留後張昕,以邠州從順。戴休顏自奉天令於軍曰:「懷光已反。」乃令城守馳表以聞。上
 於是授游瑰、休顏節度使。乃除懷光太子太保,罷其餘官,其所管委本軍擇一人功高望崇者統之,皆不奉詔。四月,懷光至河中,遂偷有同、絳等州,按兵觀望。



 李晟既收復京師,上遣給事中孔巢父、中使啖守盈持詔征之,懷光素服受命。巢父乃宣言於眾:「太尉軍中誰可領軍事者?」懷光左右皆胡虜,因發怒,亂持兵殺巢父及守盈,自是繕兵益修守拒。上還京師,以侍中渾瑊為河中節度副元帥,將兵討懷光。瑊復破同州,屯軍不進,數為懷光
 所敗。時仍歲旱蝗,京師初復,經費不給,言事者多請赦懷光。時河東、節度使馬燧威名素著,乃加燧副元帥,與瑊及鎮國軍節度駱元光、邠寧節度韓游瑰、鄜坊節度唐朝臣會兵同討懷光。燧率軍拔絳州,至寶鼎,慮懷光西走,唐突京邑,乃舍軍朝京師。既還,與瑊先自河東而降其驍將尉珪、徐庭光,統諸軍以圍河中。貞元元年秋,朔方部將牛名俊斬懷光首以降燧瑊隹刃其弟數人,乃自殺。懷光死時年五十七。尋詔以男一人為嗣,賜
 莊宅各一所,仍還懷光尸首,任其收葬,妻子並徙澧州。五年,又詔曰:



 懷舊念功,仁之大也;興滅繼絕,義之弘也。昔蔡叔圮族,周公封其子於東土;韓信乾紀,漢後爵其孥以弓高。侯君集之不率景化,我太宗存其胤以主祀。詳考先王之道,洎乎烈祖之訓,皆以刑佐德,俾人向方,則斧鉞之誅,甲兵之伐,蓋不得已而用也。曩歲盜臣竊發,國步多虞,朕狩於近郊,指期薄伐,將振昆陽之旅,以興涿鹿之功,徵師未達於諸侯,衛士且疲於七萃。而李
 懷光三軍夙駕,千里勤王,上假雷霆之威,下逐虎狼之眾。議功方始,守節靡終,潛構禍胎,拒違朝命,棄同即異,舍順效逆。為臣至此,在法必誅,猶示綏懷,庶其牽復。而梟音益厲,犬希突莫遷,大戮所加,曾無噍類。雖自貽伊戚,與眾棄之,而言念爾勞,何嗟及矣!以其前效猶在,孤魂無歸,懷之怳然,是用心妻軫。予欲布陳大惠,冀以化成,保合太和,期於刑措。宜以懷光外孫燕八八賜姓李氏,名承緒,授左衛率府胄曹參軍,承懷光之後。仍賜錢一
 千貫,任於懷光墓側置立莊園,侍養懷光妻王氏,並備四時享奠之禮。嗚呼!朕實不德,臨於兆人,泣辜宥罪,素誠所志。爾其保姓受氏,宣力承家,勉紹乃考之建國庸,無若爾父之違王命。



 初,懷光授首,其子琟、瑗等皆死,唯妻王氏在,故上特舍其死。及是又思懷光舊勛,哀其絕後,乃命承緒繼之。



 史臣曰:僕固懷恩、李懷光,咸以勇力,有勞王家,為臣不終,遂行反噬,其罪大矣。然辛云京、駱奉先、盧杞、白志
 貞輩,致彼二逆,貽憂時君,亦可謂國之讒賊矣。梁崇義既無令始,又無善終,與妻投泉,何塞其咎。



 贊曰:臣之事君,有死無二。懷恩、懷光,兇終一致。崇義多奸,國家所棄。迷而亡歸,自速其斃。



\end{pinyinscope}