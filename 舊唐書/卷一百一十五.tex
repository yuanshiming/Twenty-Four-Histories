\article{卷一百一十五}

\begin{pinyinscope}

 ○崔光遠房琯子孺復從子式張鎬高適暢璀



 崔光遠,滑州靈昌人也。本博陵舊族。祖敬嗣,好樗蒱飲酒。則天初,為房州刺史。中宗為廬陵王,安置在州,官吏
 多無禮度,敬嗣獨以親賢待之,供給豐贍,中宗深德之。及登位,有益州長史崔敬嗣,既同姓名,每進擬官,皆御筆超拜之者數四。後引與語,始知誤寵。訪敬嗣已卒,乃遣中書令韋安石授其子汪官。汪嗜酒不堪職任,且授洛州司功,又改五品。



 光遠即汪之子,雖無學術,頗有祖風,勇決任氣,身長六尺餘,目睛白黑分明。少歷仕州縣。開元末為蜀州唐安令,與楊國忠以博徒相得,累遷至左贊善大夫。天寶十一載,京兆尹鮮於仲通舉光遠為長
 安令。十四載,遷京兆少尹。其載,使吐蕃吊祭。十五載五月,使回。十餘日,潼關失守,玄宗幸蜀,詔留光遠為京兆尹、兼御史中丞,充西京留守採訪使。駕發,百姓亂入宮禁,取左藏大盈庫物,既而焚之,自旦及午,火勢漸盛,亦有乘驢上紫宸、興慶殿者。光遠與中官將軍邊令誠號令百姓救火,又募人攝府縣官分守之,殺十數人方定。使其息東見祿山,祿山大悅,偽敕復本官。先是祿山已令張休攝京兆尹十餘日,既得光遠歸款,召休歸洛。
 八月,同羅背祿山,以廄馬二千出至滻水。孫孝哲、安神威從而召之,不得,神威懼而憂死,府縣官吏驚走,獄囚皆空。光遠以為賊且逃矣,命所由守神威孝哲宅。孝哲以光遠之狀報祿山。光遠閉府門,斬為盜曳落河二人,遂與長安令蘇震等同出。至開遠門,使人前謂門官曰:「尹巡諸門。」門官具器仗以迎,至則皆斬之。領府縣官十餘人,於京西號令百姓,赴召者百餘人,夜過咸陽,遂達靈武。上喜之,擢拜御史大夫,兼京兆尹,仍使光遠於渭北召集
 人吏之歸順者。嘗有賊剽掠涇陽縣界,於僧寺中椎牛釃酒,連夜酣飲,去光遠營四十里。光遠偵知之,率馬步二千乙夜趨其所。賊徒多醉,光遠領百餘騎持滿扼其要,分命驍勇持陌刀呼而斬之,殺賊徒二千餘人,虜馬千疋,俘其渠酋一人。賊中以光遠勇勁,常避其鋒。及扈從還京,論功行賞,制曰:「持節京畿採訪、計會、招召、宣慰、處置等使崔光遠,毀家成國,致命前茅。可特進,行禮部尚書,封鄴國公,食實封三百戶。」



 乾元元年,兼御史大夫。
 五月,為河南節度使。八月,代張鎬為汴州刺史,兼本州防禦使。十二月,代蕭華為魏州刺史,充魏州節度使。初,司徒郭子儀與賊戰於汲郡,光遠率汴師千人渡河援之。及代蕭華入魏州,使將軍李處崟拒賊,賊大至,連戰不利,子儀怒不救,處崟遂敗,奔還。賊逐處崟至城下,反問之曰:「處崟召我來,何為不出?」光遠乃腰斬處崟。處崟善戰有勇,眾皆倚之,及死,人用危懼。魏州城自祿山反,袁知泰、能元皓等皆繕完之,甚為堅峻。光遠不能守,遂
 夜潰圍而出,度河而還。肅宗不之罪,除太子少保。



 襄州將士康楚元、張嘉延率眾為亂,陷荊、襄、澧、朗等州,以光遠兼御史大夫,持節荊襄招討,仍充山南東道處置兵馬都使。三年,除鳳翔尹,充本府及秦隴觀察使。先是,岐、隴吏人郭愔等為土賊,掠州縣,為五堡,光遠使判官、監察御史嚴侁召而降之。光遠在官好蒱酒,晚年不親戎事。上元元年冬,愔等潛連黨項及奴束刂、突厥敗韋倫於秦、隴,殺監軍使,擊黃戍。肅宗追還,以李鼎代之。二年,兼
 成都尹,充劍南節度營田觀察處置使,仍兼御史大夫。及段子璋反,東川節度使李奐敗走,投光遠,率將花驚定等討平之。將士肆其剽劫,婦女有金銀臂釧,兵士皆斷其腕以取之,亂殺數千人,光遠不能禁。肅宗遣監軍官使按其罪,光遠憂恚成疾,上元二年十月卒。



 房琯,河南人,天后朝正議大夫、平章事融之子也。琯少好學,風儀沉整,以門廕補弘文生。性好隱遁,與東平呂向於陸渾伊陽山中讀書為事,凡十餘歲。開元十二年,
 玄宗將封岱嶽,琯撰《封禪書》一篇及箋啟以獻。中書令張說奇其才,奏授秘書省校書郎,調補同州馮翊尉。無幾去官,應堪任縣令舉,授虢州盧氏令,政多惠愛,人稱美之。二十二年,拜監察御史。其年坐鞫獄不當,貶睦州司戶。歷慈溪、宋城、濟源縣令,所在為政,多興利除害,繕理廨宇,頗著能名。天寶元年,拜主客員外郎。三年,遷試主客郎中。五年正月,擢試給事中,賜爵漳南縣男。時玄宗企慕古道,數游幸近甸,乃分新豐縣置會昌縣於驪
 山下,尋改會昌為昭應縣,又改溫泉宮為華清宮,於宮所立百司廨舍。以琯雅有巧思,令充使繕理。事未畢,坐與李適之、韋堅等善,貶宜春太守。歷瑯邪、鄴郡、扶風三太守,所至多有遺愛。十四年,徵拜左庶子,遷憲部侍郎。



 十五年六月,玄宗蒼黃幸蜀,大臣陳希烈、張倚等銜於失恩,不時赴難。琯結張均、張垍兄弟與韋述等行至城南十數里山寺,均、垍同行,皆以家在城中,逗留不進,琯獨馳蜀路。七月,至普安郡謁見,玄宗大悅,即日拜文部
 尚書、同中書門下平章事,賜紫金魚袋。從幸成都,加銀青光祿大夫,仍與一子官。其年八月,與左相韋見素、門下侍郎崔渙等奉使靈武,冊立肅宗。至順化郡謁見,陳上皇傳付之旨,因言時事,詞情慷慨,肅宗為之改容。時潼關敗將王思禮、呂崇賁、李承光等引於纛下,將斬之,琯從容救諫,獨斬承光而已。肅宗以琯素有重名,傾意待之,琯亦自負其才,以天下為己任。時行在機務,多決之於琯,凡有大事,諸將無敢預言。尋抗疏自請將兵以
 誅寇孽,收復京都,肅宗望其成功,許之。詔加持節、招討西京兼防禦蒲潼兩關兵馬節度等使,乃與子儀、光弼等計會進兵。琯請自選參佐,乃以御中史中丞鄧景山為副,戶部侍郎李揖為行軍司馬,中丞宋若思、起居郎知制誥賈至、右司郎中魏少游為判官,給事中劉秩為參謀。既行,又令兵部尚書王思禮副之。琯分為三軍:遣楊希文將南軍,自宜壽入;劉悊將中軍,自武功入;李光進將北軍,自奉天入。琯自將中軍,為前鋒。十月庚子,師
 次便橋。辛丑,二軍先遇賊於咸陽縣之陳濤斜,接戰,官軍敗績。時琯用春秋車戰之法,以車二千乘,馬步夾之。既戰,賊順風揚塵鼓噪,牛皆震駭,因縛芻縱火焚之,人畜撓敗,為所傷殺者四萬餘人,存者數千而已。癸卯,琯又率南軍即戰,復敗,希文、劉悊並降於賊。琯等奔赴行在,肉袒請罪,上並宥之。



 琯好賓客,喜談論,用兵素非所長,而天子採其虛聲,冀成實效。琯既自無廟勝,又以虛名擇將吏,以至於敗。琯之出師,戎務一委於李揖、劉秩,
 秩等亦儒家子,未嘗習軍旅之事。琯臨戎謂人曰:「逆黨曳落河雖多,豈能當我劉秩等?」及與賊對壘,琯欲持重以伺之,為中使邢延恩等督戰,蒼黃失據,遂及於敗。上猶待之如初,仍令收合散卒,更圖進取。



 會北海太守賀蘭進明自河南至,詔授南海太守,攝御史大夫,充嶺南節度使。中謝,肅宗謂之曰:「朕處分房琯與卿正大夫,何為攝也?」進明對曰:「琯與臣有隙。」上以為然。進明因奏曰:「陛下知晉朝何以至亂?」上曰:「卿有說乎?」進明曰:「晉朝以
 好尚虛名,任王夷甫為宰相,祖習浮華,故至於敗。今陛下方興復社稷,當委用實才,而琯性疏闊,徒大言耳,非宰相器也。陛下待琯至厚,以臣觀之,琯終不為陛下用。」上問其故,進明曰:「琯昨於南朝為聖皇制置天下,乃以永王為江南節度,潁王為劍南節度,盛王為淮南節度,制云『命元子北略朔方,命諸王分守重鎮』。且太子出為撫軍,入曰監國,琯乃以枝庶悉領大籓,皇儲反居邊鄙,此雖於聖皇似忠,於陛下非忠也。琯立此意,以為聖皇
 諸子,但一人得天下,即不失恩寵。又各樹其私黨劉秩、李揖、劉匯、鄧景山、竇紹之徒,以副戎權。推此而言,琯豈肯盡誠於陛下乎?臣欲正衙彈劾,不敢不先聞奏。」上由是惡琯,詔以進明為河南節度、兼御史大夫。



 崔圓本蜀中拜相,肅宗幸扶風,始來朝謁。琯意以為圓才到,當即免相,故待圓禮薄。圓厚結李輔國,到後數日,頗承恩渥,亦憾於琯。琯又多稱病,不時朝謁,於政事簡惰。時議以兩京陷賊,車駕出次外郊,天下人心惴恐。當主憂臣辱
 之際,此時琯為宰相,略無匪懈之意,但與庶子劉秩、諫議李揖、何忌等高談虛論,說釋氏因果、老子虛無而已。此外,則聽董庭蘭彈琴,大招集琴客筵宴。朝官往往因庭蘭以見琯,自是亦大招納貨賄,奸贓頗甚。顏真卿時為大夫,彈何忌不孝,琯既黨何忌,遽托以酒醉入朝,貶為西平郡司馬。憲司又奏彈董庭蘭招納貨賄,琯入朝自訴,上叱出之,因歸私第,不敢預人事。諫議大夫張鎬上疏,言琯大臣,門客受贓,不宜見累。二年五月,貶為
 太子少師,仍以鎬代琯為宰相。其年十一月,從肅宗還京師。十二月,大赦,策勛行賞,加琯金紫光祿大夫,進封清河郡公。琯既在散位,朝臣多以為言,琯亦常自言有文武之用,合當國家驅策,冀蒙任遇。又招納賓客,朝夕盈門,游其門者,又將琯言議暴揚於朝。琯又多稱疾,上頗不悅。乾元元年六月,詔曰:



 崇黨近名,實為害政之本;黜華去薄,方啟至公之路。房琯素表文學,夙推名器,由是累階清貴,致位臺衡。而率情自任,怙氣恃權。虛浮簡
 傲者進為同人,溫讓謹令者捐於異路。所以輔佐之際,謀猷匪弘。頃者時屬艱難,擢居將相,朕永懷反席,冀有成功。而喪我師徒,既虧制勝之任;升其親友,悉彰浮誕之跡。曾未逾時,遽從敗績。自合首明軍令,以謝師旅,猶尚矜其萬死,擢以三孤。



 或云緣其切直,遂見斥退。朕示以堂案,令觀所以,咸知乖舛,曠於政事。誠宜效茲忠懇,以奉國家,而乃多稱疾疹,莫申朝謁。郤犨為政,曾不疾其迂回;亞夫事君,翻有懷於鬱怏。又與前國子祭酒劉
 秩、前京兆少尹嚴武等潛為交結,輕肆言談,有朋黨不公之名,違臣子奉上之體。何以儀刑王國,訓導儲闈?但以嘗踐臺司,未忍致之於理。況秩、武遽更相尚,同務虛求,不議典章,何成沮勸?宜從貶秩,俾守外籓。琯可邠州刺史,秩可閬州刺史,武可巴州刺史,散官、封如故;並即馳驛赴任,庶各增修。朕自臨御寰區,薦延多士,常思聿求賢哲,共致雍熙。深嫉比周之徒,虛偽成俗。今茲所譴,實屬其辜。猶以琯等妄自標持,假延浮稱,雖周行具悉,恐
 流俗多疑,所以事必縷言,蓋欲人知不濫。凡百卿士,宜悉朕懷。



 時邠州久屯軍旅,多以武將兼領刺史,法度隳廢,州縣廨宇,並為軍營,官吏侵奪百姓室屋以居,人甚弊之。琯到任,舉陳令式,令州縣恭守,又緝理公館,僚吏各歸官曹,頗著政聲。二年六月,詔褒美之,徵拜太子賓客。上元元年四月,改禮部尚書,尋出為晉州刺史。八月,改漢州刺史。琯長子乘,自少兩目盲。琯到漢州,乃厚結司馬李銳以財貨,乘聘銳外甥女盧氏,時議薄其無
 士行。寶應二年四月,拜特進、刑部尚書。在路遇疾,廣德元年八月四日,卒於閬州僧舍,時年六十七。贈太尉。



 孺復,琯之孽子也。少黠慧,年七八歲,即粗解綴文,親黨奇之。稍長,狂疏傲慢,任情縱欲。年二十,淮南節度陳少游闢為從事,多招陰陽巫覡,令揚言已過三十必為宰相。德宗幸奉天,包佶掌賦於揚州,少游將抑奪之。佶聞而奔出,少游方遣人劫佶令回,孺復請行,會佶已過江南,乃還。及少游卒,浙西節度韓滉又闢入幕。其長兄宗偃
 先貶官嶺下而卒,及喪柩到揚州,孺復未嘗吊。初娶鄭氏,惡賤其妻,多畜婢僕,妻之保母累言之,孺復乃先具棺櫬而集家人,生斂保母,遠近驚異。及妻在產蓐三四日,遽令上船即路,數日,妻遇風而卒。孺復以宰相子,年少有浮名,而奸惡未甚露,累拜杭州刺史。又娶臺州刺史崔昭女,崔妒悍甚,一夕杖殺孺復待兒二人,埋之雪中。觀察使聞之,詔發使鞫案有實,孺復坐貶連州司馬,仍令與崔氏離異。孺復久之遷辰州刺史,改容州刺史、
 本管經略使。乃潛與妻往來,久而上疏請合,詔從之。二歲餘,又奏與崔氏離異,其為取舍恣逸,不顧禮法也如此。貞元十三年九月卒,時年四十二。



 式,琯之侄,舉進士。李泌觀察陜州,闢為從事。泌入為相,累遷起居郎,出入泌門,為其耳目。及泌卒,再除忠州刺史,韋皋表為雲南安撫使,兼御史中丞。皋卒,詔除兵部郎中。屬劉闢反,式留不得行。性便佞,又懼闢,每於座中數贊闢之德美,比之劉備,同陷於賊者皆惡之。高崇文既至成都,式與王
 良士、崔從、盧士玖等白衣麻蹻銜土請罪,崇文寬禮之,乃表其狀,尋除吏部郎中。時河朔節度劉濟、王士真、張茂昭皆以兵壯氣豪,相持短長,屢以表聞,迭請加罪。上欲止其兵,李吉甫薦式為給事中,將命於河朔。式歷使諸鎮諷諭之,還奏愜旨,除陜虢觀察使、兼御史中丞,轉河南尹。時討王丞宗於鎮州,配河南府饋運車四千兩,式表以兇旱,人貧力微,難以徵發,憲宗可其奏,既免力役,人懷而安之。明年,移授宣歙池觀察使。元和七年七
 月卒,贈左散騎常侍。



 張鎬,博州人也。風儀魁岸,廓落有大志,涉獵經史,好談王霸大略。少時師事吳兢,兢甚重之。後游京師,端居一室,不交世務。性嗜酒,好琴,常置座右。公卿或有邀之者,鎬仗策徑往,求醉而已。



 天寶末,楊國忠以聲名自高,搜天下奇傑。聞鎬名,召見薦之,自褐衣拜左拾遺。及祿山阻兵,國忠屢以軍國事咨於鎬,鎬舉贊善大夫來瑱可當方面之寄。數月,玄宗幸蜀,鎬自山谷徒步扈從。肅宗
 即位,玄宗遣鎬赴行在所。鎬至鳳翔,奏識多有弘益,拜諫議大夫,尋遷中書侍郎、同中書門下平章事。時供奉僧在內道場晨夜念佛,動數百人,聲聞禁外。鎬奏曰:「臣聞天子修福,要在安養含生,靖一風化,未聞區區僧教,以致太平。伏願陛下以無為為心,不以小乘而撓聖慮。」肅宗甚然之。時方興軍戎,帝注意將帥,以鎬有文武才,尋命兼河南節度使,持節都統淮南等道諸軍事。鎬既發,會張巡宋州圍急,倍道兼進,傳檄濠州刺史閭丘曉
 引兵出救。曉素愎戾,馭下少恩,好獨任己。及鎬信至,略無稟命,又慮兵敗,禍及於己,遂逗留不進。鎬至淮口,宋州已陷,鎬怒曉,即杖殺之。及收復兩京,加鎬銀青光祿大夫,封南陽郡公,詔以本軍鎮汴州,招討殘孽。時賊帥史思明表請以範陽歸順,鎬揣知其偽,恐朝廷許之,手書密表奏曰:「思明兇豎,因逆竊位,兵強則眾附,勢奪則人離。包藏不測,禽獸無異,可以計取,難以義招。伏望不以威權假之。」又曰:「滑州防禦使許叔冀,性狡多謀,臨難
 必變,望追入宿衛。」肅宗計意已定,表入不省。鎬為人簡澹,不事中要。會有宦官自範陽及滑州使還者,皆言思明、叔冀之誠愨。肅宗以鎬不切事機,遂罷相位,授荊州大都督府長史。後思明、叔冀之偽皆符鎬言。尋徵為太子賓客,改左散騎常侍。屬嗣岐王珍被誣告構逆伏法,鎬買珍宅坐累,貶辰州司戶。



 代宗即位,推恩海內,拜撫州刺史。遷洪州刺史、饒吉等七州都團練觀察等使,尋正授江南西道都團練觀察等使。廣德二年九月卒。



 鎬
 自入仕凡三年,致位宰相。居身清廉,不營資產,謙恭下士,善談論,多識大體,故天下具瞻,雖考秩至淺,推為舊德云。



 高適者,渤海蓚人也。父從文,位終韶州長史。適少濩落,不事生業,家貧,客於梁、宋,以求丐取給。天寶中,海內事干進者注意文詞。適年過五十,始留意詩什,數年之間,體格漸變,以氣質自高,每吟一篇,已為好事者稱誦。宋州刺史張九皋深奇之,薦舉有道科。時右相李林甫擅
 權,薄於文雅,唯以舉子待之。解褐汴州封丘尉,非其好也,乃去位,客游河右。河西節度哥舒翰見而異之。表為左驍衛兵曹,充翰府掌書記,從翰入朝,盛稱之於上前。



 祿山之亂,徵翰討賊,拜適左拾遺,轉監察御史,仍佐翰守潼關。及翰兵敗,適自駱谷西馳,奔赴行在,及河池郡,謁見玄宗,因陳潼關敗亡之勢曰:「僕射哥舒翰忠義感激,臣頗知之,然疾病沉頓,智力將竭。監軍李大宜與將士約為香火,使倡婦彈箜篌琵琶以相娛樂,樗蒱飲酒,
 不恤軍務。蕃渾及秦、隴武士,盛夏五六月於赤日之中,食倉米飯且猶不足,欲其勇戰,安可得乎?故有望敵散亡,臨陣翻動,萬全之地,一朝而失。南陽之軍,魯炅、何履光、趙國珍各皆持節,監軍等數人更相用事,寧有是,戰而能必勝哉?臣與楊國忠爭,終不見納。陛下因此履巴山、劍閣之險,西幸蜀中,避其蠆毒,未足為恥也。」玄宗嘉之,尋遷侍御史。至成都,八月,制曰:「侍御史高適,立節貞峻,植躬高朗,感激懷經濟之略,紛綸贍文雅之才。長策
 遠圖,可云大體;讜言義色,實謂忠臣。宜回糾逖之任,俾超諷諭之職,可諫議大夫,賜緋魚袋。」適負氣敢言,權幸憚之。



 二年,永王璘起兵於江東,欲據揚州。初,上皇以諸王分鎮,適切諫不可。及是永王叛,肅宗聞其論諫有素,召而謀之。適因陳江東利害,永王必敗。上奇其對,以適兼御史大夫、揚州大都督府長史、淮南節度使。詔與江東節度來瑱率本部兵平江淮之亂,會於安州。師將渡而永王敗,乃招季廣琛於歷陽。兵罷,李輔國惡適敢言,
 短於上前,乃左授太子少詹事。未幾,蜀中亂,出為蜀州刺史,遷彭州。劍南自玄宗還京後,於梓、益二州各置一節度,百姓勞敝,適因出西山三城置戍,論之曰:



 劍南雖名東西兩川,其實一道。自邛關、黎、雅,界於南蠻也;茂州而西,經羌中至平戎數城,界於吐蕃也。臨邊小郡,各舉軍戎,並取給於劍南。其運糧戍,以全蜀之力,兼山南佐之,而猶不舉。今梓、遂、果閬等八州分為東川節度,歲月之計,西川不可得而參也。而嘉、陵比為夷獠所陷,今雖
 小定,瘡痍未平。又一年已來,耕織都廢,而衣食之業,皆貿易於成都,則其人不可得而役明矣。今可稅賦者,成都、彭、蜀、漢州。又以四州殘敝,當他十州之重役,其於終久,不亦至艱?又言利者穿鑿萬端,皆取之百姓;應差科者,自朝至暮,案牘千重。官吏相承,懼於罪譴,或責之於鄰保,或威之以杖罰。督促不已,逋逃益滋,欲無流亡,理不可得。比日關中米貴,而衣冠士庶,頗亦出城,山南、劍南,道路相望,村坊市肆,與蜀人雜居,其升合鬥儲,皆求
 於蜀人矣。且田士疆界,蓋亦有涯;賦稅差科,乃無涯矣。為蜀人之計,不亦難哉!



 今所界吐蕃城堡而疲於蜀人,不過平戎以西數城矣。邈在窮山之巔,垂於險絕之末,運糧於束馬之路,坐甲於無人之鄉。以戎狄言之,不足以利戎狄;以國家言之,不足以廣土宇。奈何以險阻彈丸之地,而困於全蜀太平之人哉?恐非今日之急務也。國家若將已戍之地不可廢,已鎮之兵不可收,當宜卻停東川,並力從事,猶恐狼狽,安可仰於成都、彭、漢、蜀四
 州哉!慮乖聖朝洗蕩關東掃清逆亂之意也。倘蜀人復擾,豈不貽陛下之憂?昔公孫弘願罷西南夷、臨海,專事朔方,賈捐之請棄珠崖以寧中土,讜言政本,匪一朝一夕。臣愚望罷東川節度,以一劍南,西山不急之城,稍以減削,則事無窮頓,庶免倒懸。陛下若以微臣所陳有裨萬一,下宰相廷議,降公忠大臣定其損益,與劍南節度終始處置。



 疏奏不納。



 後梓州副使段子璋反,以兵攻東川節度使李奐,適率州兵從西川節度使崔光遠攻於
 璋,斬之。西川牙將花驚定者,恃勇,既誅子璋,大掠東蜀。天子怒光遠不能戢軍,乃罷之,以適代光遠為成都尹、劍南西川節度使。代宗即位,吐蕃陷隴右,漸逼京畿。適練兵於蜀,臨吐蕃南境以牽制之,師出無功,而松、維等州尋為蕃兵所陷。代宗以黃門侍郎嚴武代還,用為刑部侍郎,轉散騎常侍,加銀青光祿大夫,進封渤海縣侯,食邑七百戶。永泰元年正月卒,贈禮部尚書,謚曰忠。



 適喜言王霸大略,務功名,尚節義。逢時多難,以安危為己
 任,然言過其術,為大臣所輕。累為籓牧,政存寬簡,吏民便之。有文集二十卷。其《與賀蘭進明書》,令疾救梁、宋,以親諸軍;《與許叔冀書》,綢繆繼好,使釋他憾,同援梁、宋;《未過淮先與將校書》,使絕永王,各求自白,君子以為義而知變。而有唐已來,詩人之達者,唯適而已。



 暢璀,河東人也。鄉舉進士。天寶末,安祿山奏為河北海運判官。三遷大理評事,副元帥郭子儀闢為從事。至德初,肅宗即位,大收俊傑,或薦璀,召見悅之,拜諫議大夫。
 累轉吏部侍郎。廣德二年十二月,為散騎常侍、河中尹,兼御史大夫。永泰元年,復為左常侍,與裴冕並集賢院待制。大歷五年,兼判太常卿,遷戶部尚書。十年七月卒,贈太子太師。



 璀廓落有口才,好談王霸之略,居職責成屬吏。齪齪無過而已。



 史臣曰:祿山寇陷兩京,儒生士子,被脅從、懷茍且者多矣;去逆效順,毀家為國者少焉。如光遠勇決任氣,會權變以立功;房琯文學致身,全節義以避寇。阽危之時,顛
 沛之際,有足稱者。然光遠居重籓,掌軍政,琯登相位,奪將權,聚浮薄之徒,敗軍旅之事,不知機而固位,竟無德以自危。孺復兇狂,式之便佞,獲令終者幸焉。鎬直躬居位,重德鎮時,其為人也鮮矣。適以詩人為戎帥,險難之際,名節不虧,君子哉!璀擢第居官,守分無過,又何咎焉。



 贊曰:光遠、房琯,有始有終。張鎬國器,適、璀儒風。



\end{pinyinscope}