\article{卷一百五十六}

\begin{pinyinscope}

 ○馬璘郝廷玉王棲曜子茂元劉昌子士涇李景略張萬福高固郝玼段佐史敬奉野詩良輔附



 馬璘,扶風人也。祖正會,右威衛將軍。父晟,右司禦率府兵曹參軍。璘少孤,落拓不事生業。年二十餘,讀《馬援傳》至「大丈夫當死於邊野,以馬革裹尸而歸」,慨然嘆曰:「豈使吾祖勛業墜於地乎!」開元末,杖劍從戎,自效於安西。以前後奇功,累遷至左金吾衛將軍同正。



 至德初,王室多難,璘統甲士三千,自二庭赴於鳳翔。肅宗奇之,委以東討。殄寇陜郊,破賊河陽,皆立殊效。嘗從李光弼攻賊洛陽,史朝義自領精卒,拒王師於北邙,營壘如山,旌甲
 耀日,諸將愕眙不敢動。璘獨率所部橫戈而出,入賊陣者數四,賊因披靡潰去。副元帥李光弼壯之,曰:「吾用兵三十年,未見以少擊眾,有雄捷如馬將軍者。」遷試太常卿。



 明年,蕃賊寇邊,詔璘赴援河西。廣德初,僕固懷恩不順,誘吐蕃入寇,代宗避狄陜州。璘即日自河右轉鬥戎虜間,至於鳳翔。時蕃軍雲合,鳳翔節度使孫志直方閉城自守;璘乃持滿外向,突入懸門,不解甲,背城出戰,吐蕃奔潰。璘以勁騎追擊,俘斬數千計,血流於野,由是雄
 名益振。代宗還宮,召見慰勞之,授兼御史中丞。



 永泰初,拜四鎮行營節度,兼南道和蕃使,委之禁旅,俾清殘寇。俄遷四鎮、北庭行營節度及邠寧節度使、兼御史大夫,旋加檢校工部尚書。以犬戎浸驕,歲犯郊境,涇州最鄰戎虜,乃詔璘移鎮涇州,兼權知鳳翔隴右節度副使、涇原節度、涇州刺史,四鎮、北庭行營節度使如故。復以鄭、滑二州隸之。璘詞氣慷慨,以破虜為己任。既至涇州,分建營堡,繕完戰守之具,頻破吐蕃,以其生口俘馘來獻,
 前後破吐蕃約三萬餘眾。在涇州令寬而肅,人皆樂為之用。鎮守凡八年,雖無拓境之功,而城堡獲全,虜不敢犯,加檢校右僕射。上甚重之,遷檢校左僕射知省事,詔宰臣百僚於尚書省送上,進封扶風郡王。



 璘雖生於士族,少無學術,忠而能勇,武幹絕倫,艱難之中,頗立忠節,中興之猛將也。年五十六,大歷十二年卒,德宗悼之,廢朝,贈司徒。



 璘久將邊軍,屬西蕃寇擾,國家倚為屏翰。前後賜與無算,積聚家財,不知紀極。在京師治第舍,尤為
 宏侈。天寶中,貴戚勛家,已務奢靡,而垣屋猶存制度。然衛公李靖家廟,已為嬖臣楊氏馬廄矣。及安、史大亂之後,法度隳弛,內臣戎帥,競務奢豪,亭館第舍,力窮乃止,時謂「木妖」。璘之第,經始中堂,費錢二十萬貫,他室降等無幾。及璘卒於軍,子弟護喪歸京師,士庶觀其中堂,或假稱故吏,爭往赴吊者數十百人。德宗在東宮,宿聞其事;及踐祚,條舉格令,第舍不得逾制,仍詔毀璘中堂及內官劉忠翼之第;璘之家園,進屬官司。自後公卿賜宴,
 多於璘之山池。子弟無行,家財尋盡。



 郝廷玉者,驍勇善格鬥,事太尉李光弼,為帳中愛將。乾元中,史思明再陷洛陽,光弼拔東都之師保河陽。時三城壁壘不完,芻糧不支旬日;賊將安太清等率兵數萬,四面急攻。光弼懼賊勢西犯河、潼,極力保孟津以掎其後,晝夜嬰城,血戰不解,將士夷傷。光弼召諸將訊之曰:「賊黨何面難抗?」或對曰:「西北隅最為勍敵。」乃亟召廷玉謂之曰:「兇渠攻西北者難奈,爾為我決勝而還。」辭曰:「廷
 玉所領,步卒也,願得騎軍五百。」光弼以精騎三百授之。光弼法令嚴峻,是日戰不利而還者,不解甲斬之。廷玉奮命先登,流矢雨集,馬傷不能軍而退。光弼登堞見之,駭然曰:「廷玉奔還,吾事敗矣!」促令左右取廷玉首來。廷玉見使者曰:「馬中毒箭,非敗也。」光弼命易馬而復,徑騎沖賊陣,馳突數四。俄而賊黨大敗於河壖,廷玉擒賊將徐璜而還。由是賊解中水單之圍,信宿退去。前後以戰功累授開府儀同三司,試太常卿,封安邊郡王。從光弼鎮
 徐州。光弼薨,代宗用為神策將軍。



 永泰初,僕固懷恩誘吐蕃、回紇入犯京畿,分命諸將屯於要害,廷玉與馬璘率五千人屯於渭橋西窯底。觀軍容使魚朝恩以廷玉善陣,欲觀其教閱。廷玉乃於營內列部伍,鳴鼓角而出,分而為陣,箕張翼舒,乍離乍合,坐作進退,其眾如一。朝恩嘆曰:「吾在兵間十餘年,始見郝將軍之訓練耳。治戎若此,豈有前敵耶?」廷玉淒然謝曰:「此非末校所長,臨淮王之遺法也。太尉善御軍,賞罰當功過。每校旗之日,軍
 士小不如令,必斬之以徇,由是人皆自效,而赴蹈馳突,有心破膽裂者。太尉薨變已來,無復校旗之事,此不足軍容見賞。」



 王縉為河南副元帥,詔以廷玉為其都知兵馬使,累授秦州刺史。大歷八年卒,追錄舊勛,贈工部尚書。



 王棲曜,濮州濮陽人也。初游鄉學。天寶末,安祿山叛,尚衡起義兵討之,以棲曜為牙將。下兗、鄆諸縣,軍威稍振。進為衙前總管。初,逆將邢超然據曹州,棲曜攻之。超然
 乘城號令,棲曜曰:「彼可取也!」一箭殞之,城中氣懾,遂拔曹州。及衡居節制,授右威衛將軍、先鋒游奕使。隨衡入朝,授試金吾衛將軍。



 上元元年,王璵為浙東節度使,奏為馬軍兵馬使。廣德中,草賊袁晁起亂臺州,連結郡縣,積眾二十萬,盡有浙江之地。御史中丞袁傪東討,奏棲曜與李長為偏將,聯日十餘戰,生擒袁晁,收復郡邑十六,授常州別駕、浙西都知兵馬使。



 時江左兵荒,詔內常侍馬日新領汴滑軍五千鎮之。日新貪暴,賊蕭庭蘭乘
 人怨訴,逐之而劫其眾。時棲曜游奕近郊,為賊所脅,進圍蘇州。棲曜因其懈怠,挺身登城,率城中兵復出擊賊,其眾大潰。遷試金吾大將軍。



 李靈曜叛於汴州,浙西觀察使李涵俾棲曜將兵四千為河南掎角。以功加銀青光祿大夫,累加至御史中丞。李希烈既陷汴州,乘勝東侵,連陷陳留、雍丘,頓軍寧陵,期襲宋州。浙西節度使韓滉命棲曜將強弩數千,夜入寧陵。希烈不之知,晨朝,弩矢及希烈坐幄,希烈驚曰:「此江、淮弩士入矣!」遂不敢東
 去。



 貞元初,拜左龍武大將軍,旋授鄜坊、丹延節度觀察使、檢校禮部尚書、兼御史大夫。貞元十九年卒於位。子茂元。



 茂元,幼有勇略,從父征伐知名。元和中,為右神策將軍。太和中,檢校工部尚書、廣州刺史、嶺南節度使。在安南招懷蠻落,頗立政能。南中多異貨,茂元積聚家財鉅萬計。李訓之敗,中官利其財,掎摭其事,言茂元因王涯、鄭注見用。茂元懼,罄家財以賂兩軍,以是授忠武軍節度、陳許觀察使。會昌中,為河陽節度使。是時河北諸
 軍討劉稹,茂元亦以本軍屯天井,賊未平而卒。



 劉昌,字公明,汴州開封人也。出自行間,少學騎射。及安祿山反,昌始從河南節度張介然,授易州遂城府左果毅。及史朝義遣將圍宋州;昌在圍中,連月不解,城中食盡,賊垂將陷之。刺史李岑計蹙,昌為之謀曰:「今河陽有李光弼制勝,且江、淮足兵,此廩中有數千斤曲,可以屑食。計援兵不二十日當至。東南隅之敵,眾以為危,昌請守之。」昌遂被鎧持盾登城,陳逆順以告諭賊,賊眾畏服。
 後十五日,副元帥李光弼救軍至,賊乃宵潰。光弼聞其謀,召置軍中,超授試左金吾衛郎將。光弼卒,宰臣王縉令歸宋州,為牙門將。轉太僕卿,兼許州別駕。



 李靈曜據汴州叛,刺史李僧惠將受靈曜牽制;昌密遣曾神表潛說僧惠。僧惠召昌問計,昌泣陳其逆順;僧惠感之,乃使神表齎表詣闕,請討靈曜,遂翦靈曜左翼。汴州平,李忠臣嫉僧惠功,遂欲殺昌,昌潛遁。及劉玄佐為刺史,乃復其職。又轉太常卿,兼華州別駕。玄佐尋為宋亳潁宣武
 軍節度使;昌自下軍為左廂兵馬使。



 李納反,以師收考城,充行營諸軍馬步都虞候,加檢校太子詹事、兼御史中丞。明年,玄佐圍濮州,昌攝濮州刺史。李希烈既陷汴州,玄佐遣將高翼以精兵五千保援襄邑;城陷,翼赴水死。自宋及江、淮,人心震恐。時昌以三千人守寧陵,希烈率五萬眾陣於城下;昌深塹以遏地道,凡四十五日,不解甲胄,躬勵士卒,大破希烈。希烈解圍攻陳州,刺史李公廉計窮,昌從劉玄佐以浙西兵合三萬人救之。至陳
 州西五十里與賊遇,昌晨壓其陣,及未成列,大破之,生擒其將翟曜。希烈退保蔡州,自此不復侵軼。詔加檢校左散騎常侍。隨玄佐收汴州,加檢校工部尚書,增實封通前二百戶。丁母憂,起復加金吾衛大將軍,贈其母梁國夫人。



 貞元三年,玄佐朝京師,上因以宣武士眾八千委昌北出五原。軍中有前卻沮事,昌繼斬三百人,遂行。尋以本官授京西北行營節度使。歲餘,授涇州刺史,充四鎮、北庭行營,兼涇原節度支度營田等使。昌躬率士
 眾,力耕三年,軍食豐羨,名聞闕下。復築連雲堡,受詔城平涼,以扼彈箏峽口。昌命徒庀事,旬餘而畢。又於平涼西別築胡谷堡,名曰彰信。平涼當四會之沖,居北地之要,分兵援戍,遏其要沖,遂以保寧邊鄙,加檢校右僕射。



 昌初至平涼劫盟之所,收聚亡歿將士骸骨坎瘞之,因感夢於昌,有愧謝之意。昌上聞,德宗下詔深自克責,遣秘書少監孔述睿及中使以御饌、內造衣服數百襲,令昌收其骸骨,分為大將三十人,將士百人,各具棺槥衣
 服,葬於淺水原。建二塚,大將曰「旌義塚」,將士曰「懷忠塚」。詔翰林學士撰銘志祭文。昌盛陳兵設,幕次具牢饌祭之。昌及大將皆素服臨之,焚其衣服紙錢,別立二石堆。題以塚名。諸道師徒,莫不感泣。



 昌在西邊僅十五年,強本節用,軍儲豐羨。及嬰疾,約以是日赴京求醫,未發而卒,年六十四,廢朝一日,贈司空。子士涇。



 士涇,德宗朝尚主,官至少列十餘年,家富於財。結托中貴,交通權幸。憲宗朝,遷太府卿。制下,給事中韋弘景等封還制書,言士
 涇不合居九卿,辭語激切。憲宗謂弘景曰:「士涇父有功於國,又是戚屬,制書宜下。」弘景奉詔。士涇善胡琴,多游權幸之門,以此為之助,時論鄙之。



 李景略,幽州良鄉人也。大父楷固。父承悅,檀州刺史、密雲軍使。景略以門廕補幽州功曹。大歷末,寓居河中,闔門讀書。李懷光為朔方節度,招在幕府。五原有偏將張光者,挾私殺妻,前後不能斷。光富於財貨,獄吏不能劾。景略訊其實,光竟伏法。既而亭午有女厲被發血身,
 膝行前謝而去。左右有識光妻者,曰:「光之妻也。」因授大理司直,遷監察御史。及懷光屯軍咸陽,反狀始萌。景略時說懷光請復宮闕,迎大駕,懷光不從。景略出軍門慟哭曰;「誰知此軍一日陷於不義。」軍士相顧甚義之,因退歸私家。



 尋為靈武節度杜希全闢在幕府,轉殿中侍御史,兼豐州刺史、西受降城使。豐州北扼回紇,回紇使來中國,豐乃其通道。前為刺史者多懦弱,虜使至則敵禮抗坐。時回紇遣梅錄將軍隨中官薛盈珍入朝,景略欲以
 氣制之。郊迎,傳言欲先見中使,梅錄初未喻。景略既見盈珍,乃使謂梅錄曰:「知可汗初沒,欲申吊禮。」乃登高壟位以待之。梅錄俯僂前哭,景略因撫之曰:「可汗棄代,助爾號慕。」虜之驕容威氣,索然盡矣,遂以父行呼景略。自此回紇使至景略,皆拜之於庭,由是有威名。杜希全忌之,上表誣奏,貶袁州司馬。希全死,徵為左羽林將軍,對於延英殿,奏對衎潔,有大臣風彩。



 時河東李說有疾,詔以景略為太原少尹、節度行軍司馬。時方鎮節度使少
 徵入換代者,皆死亡乃命焉,行軍司馬盡簡自上意。受命之日,人心以屬。景略居疑帥之地,勢已難處。回紇使梅錄將軍入朝,說置宴會,梅錄爭上下坐,說不能遏,景略叱之。梅錄,前過豐州者也,識景略語音,疾趨前拜曰:「非豐州李端公耶?不拜麾下久矣,何其瘠也。」又拜,遂命之居次坐。將吏賓客顧景略,悉加嚴憚。說心不平,厚賂中尉竇文場,將去景略,使為內應。



 歲餘,風言回紇將南下陰山,豐州宜得其人。上素知景略在邊時事。上方軫
 慮,文場在旁,言景略堪為邊任,乃以景略為豐州刺史、兼御史大夫、天德軍西受降城都防禦使。迫塞苦寒,土地鹵瘠,俗貧難處。景略節用約己,與士同甘苦,將卒安之。鑿咸應、永清二渠,溉田數百頃,公私利焉。廩儲備,器械具,政令肅,智略明。二歲後,軍聲雄冠北邊,回紇畏之,天下皆惜其理未盡景略之能。貞元二十年,卒於鎮,年五十五,贈工部尚書。



 張萬福,魏州元城人。自曾祖至其父,皆明經,止縣令州
 佐。萬福以父祖業儒皆不達,不喜為書生,學騎射。年十七八,從軍遼東有功,為將而還。累攝舒廬壽三州刺史、舒廬壽三州都團練使。州送租賦詣京師,至潁州界為盜所奪,萬福領輕兵馳入潁州界討之。賊不意萬福至,忙迫不得戰,萬福悉聚而誅之,盡得其所亡物,並得前後所掠人妻子、財物、牛馬等萬計,悉還其家;不能自致者,萬福給船乘以遣之。



 尋真拜壽州刺史、淮南節度副使。為節度使崔圓所忌,失刺史,改鴻臚卿;以節度副使
 將千人鎮壽州,萬福不以為恨。



 許杲以平盧行軍司馬將卒三千人駐濠州不去,有窺淮南意。圓令萬福攝濠州刺史。杲聞,即提卒去,止當塗陳莊。賊陷舒州,圓又以萬福為舒州刺史,督淮南岸盜賊,連破其黨。



 大歷三年,召赴京師,代宗謂曰:「聞卿名久,欲一識卿面,且將累卿以許杲。」萬福拜謝,因前奏曰:「陛下以一許杲召臣,如河北諸將叛,欲以屬何人?」代宗笑謂曰:「且與吾了許杲事,方當大用卿。」以為和州刺史、行營防禦使,督淮南岸盜
 賊。至州,杲懼,移軍上元。杲至楚州大掠,節度使韋元甫命萬福追討之。未至淮陰,杲為其將康自勸所逐。自勸擁兵繼掠,循淮而東,萬福倍道追而殺之,免者十二三,盡得其虜掠金帛婦人等,皆送致其家。元甫將厚賞將士,萬福曰:「官健常虛費衣糧,無所事,今乃一小賴之,不足過賞,請用三之一。」代宗發詔以勞之,賜衣一襲、宮錦十雙。



 久之,詔以本鎮之兵千五百人防秋西京。萬福詣揚州交所領兵,會元甫死,諸將皆願得萬福為帥,監軍
 使米重耀亦請萬福知節度事。萬福曰:「某非幸人,勿以此相待。」遂去之。帶利州刺史鎮咸陽,因留宿衛。



 李正己反,將斷江、淮路,令兵守埇橋、渦口。江、淮進奏舡千餘隻,泊渦下不敢過。德宗以萬福為濠州刺史,召見謂曰:「先帝改卿名『正』者,所以褒卿也。朕以為江、淮草木亦知卿威名,若從先帝所改,恐賊不知是卿也。」復賜名萬福。馳至渦口,立馬岸上,發進奉舡,淄青兵馬倚岸睥睨不敢動,諸道舡繼進。改泗州刺史。魏州饑,父子相賣,餓死者
 接道。萬福曰:「魏州吾鄉里,安可不救?」令其兄子將米百車往饟之。又使人於汴口,魏人自賣者,給車牛贖而遣之。



 為杜亞所忌,徵拜右金吾將軍。召見,德宗驚曰:「杜亞言卿昏耄,卿乃如是健耶!」詔圖形於凌煙閣,數賜酒饌衣服,並敕度支籍口畜給其費。及陽城等於延英門外請對論事,伏閣不去。德宗大怒,不可測。萬福揚言曰:「國有直臣,天下太平矣!萬福年已八十,見此盛事。」閣前遍揖城等,天下益重其名。



 貞元二十一年,以左散騎常侍
 致仕。其年五月卒,年九十。萬福自始從軍至卒,祿食七十餘年,未嘗病一日,典九郡皆有惠愛。在泗州時,遇德宗幸奉天,李希烈反,陳少游悉令管內刺史送妻子在揚州以為質。萬福獨不送,謂使者曰:「為某白相公,萬福妻老且醜,不足煩相公寄意。」終不之遣,由是為人所稱。



 高固,高祖侃,永徽中,為北庭安撫使,有生擒車鼻可汗之功,官至安東都護,事具前錄。固生微賤,為叔父所賣,展轉為渾瑊家奴,號曰黃芩。性敏惠,有膂力,善騎射,好
 讀《左氏春秋》。瑊大愛之,養如己子,以乳母之女妻之,遂以固名,取《左氏傳》高固之名也。



 少隨瑊從戎於朔方,德宗幸奉天,固猶在瑊麾下。是時,賊兵已突入東壅門,固引甲士亂揮長刀,連斫數賊,拽車塞闔,一以當百,賊乃退去。眾咸壯之。以功封渤海郡王。李懷光既反,德宗再幸梁漢。懷光發跡邠寧,至是,使留後張昕取將士萬餘人以資援河中。固時在軍中,乃伺便突入張昕帳中,斬首以徇。拜檢校右散騎常侍、前軍兵馬使。貞元十七年,
 節度使楊朝晟卒,軍中請固為帥,德宗念固功,因授檢校工部尚書。順宗即位,就加檢校禮部尚書。憲宗朝,進檢校右僕射。數年受代,入為統軍,轉檢校左僕射,兼右羽林統軍。元和四年七月卒,贈陜州大都督。



 郝玼者,涇原之戍將也。貞元中,為臨涇鎮將,勇敢無敵,聲振虜庭。玼以臨涇地居險要,當虜要沖,白其帥曰:「臨涇草木豐茂,宜畜牧,西蕃入寇,每屯其地,請完壘益軍,以折虜之入寇。」前帥不從。及段佐節制涇原,深然其策。
 元和三年,佐請築臨涇城,朝廷從之。仍以為行涼州,詔玼為刺史以戍之。自此西蕃入寇,不過臨涇。



 玼出自行間,前無堅敵。在邊三十年,每戰得蕃俘,必刳剔而歸其尸,蕃人畏之如神。贊普下令國人曰:「有生得郝玼者,賞之以等身金。」蕃中兒啼者,呼玼名以怖之。十三年,檢校左散騎常侍、渭州刺史、御史大夫,充涇原行營節度、平涼鎮遏都知兵馬使,封保定郡王。吐蕃畏其威,綱紀欲圖之,朝廷慮失驍將,移授慶州刺史,竟終牖下。



 段佐者,亦以勇敢知名。少事汾陽王子儀為牙將,從征邊朔,績效居多。貞元末,為涇原節度使,練卒保邊,亦為西蕃畏憚。累至檢校工部尚書、右神策大將軍。元和五年卒。



 史敬奉,靈武人,少事本軍為牙將。元和十四年,敬奉大破吐蕃於鹽州城下,賜實封五十戶。先是,西戎頻歲犯邊,敬奉白節度杜叔良請兵三千,備一月糧,深入蕃界;叔良以二千五百人授之。敬奉既行十餘日,人莫知其
 所向,皆謂吐蕃盡殺之矣。乃由他道深入,突出蕃眾之後。戎人驚潰,敬奉率眾大破之,殺戮不可勝紀,驅其餘眾於蘆河,獲羊馬駝牛萬數。



 敬奉形甚短小,若不能勝衣。至於野外馳逐,能擒奔馬,自執鞍勒,隨鞍躍上,然後羈帶,矛矢在手,前無強敵。甥侄及僮使僅二百人,每以自隨;臨入敵,輒分其隊為四五,隨逐水草,每數日各不相知;及相遇,已皆有獲虜矣。



 與鳳翔將野詩良輔、涇原將郝玼各以名雄邊上。吐蕃嘗謂漢使曰:「唐國既與吐
 蕃和好,何妄語也!」問曰:「何謂?」曰:「若不妄語,何因遣野詩良輔作隴州刺史?」其畏憚如此。



 史臣曰:自盜起中原,河、隴陷虜,犬戎作梗,屢犯郊畿。謀臣運策以竭精,武士荷戈而不暇。如璘、昌之材力,扼腕奮命,欲吞虜於胸中;郝、史驍雄,斬將搴旗,將申威於塞外。而竟不能北逾白道,西出蕭關,俾十九郡生民,竟淪左衣任,僅能自保,功何取焉!雖運使時然,亦將略有所未至。棲曜、萬福之節概,景略之負氣,壯哉!



 贊曰:馬、劉、史、郝,氣雄邊朔。力捍獯虜,終慚衛、霍。萬福義勇,景略氣豪。為人所忌,慷慨徒勞。



\end{pinyinscope}