\article{卷二百一十三}

\begin{pinyinscope}

 ○安祿山子慶緒高尚孫孝哲史思明子朝義



 安祿山,營州柳城雜種胡人也,本無姓氏,名軋犖山。母阿史德氏,亦突厥巫師,以卜為業。突厥呼鬥戰為軋犖
 山,遂以名之。少孤,隨母在突厥中,將軍安波至兄延偃妻其母。開元初,與將軍安道買男俱逃出突厥中。道買次男貞節為嵐州別駕,收獲之。年十餘歲,以與其兄及延偃相攜而出,感愧之,約與思順等並為兄弟,冒姓為安。及長,解六蕃語,為互市牙郎。



 二十年,張守珪為幽州節度,祿山盜羊事覺,守珪剝坐,欲棒殺之,大呼曰:「大夫不欲滅兩蕃耶?何為打殺祿山!」守珪見其肥白,壯其言而釋之。令與鄉人史思明同捉生,行必克獲,拔為偏將。
 常嫌其肥,以守珪威風素高,畏懼不敢飽食。以驍勇聞,遂養為子。



 二十八年,為平盧兵馬使。性巧黠,人多譽之。授營州都督、平盧軍使。厚賂往來者,乞為好言,玄宗益信響之。天寶元年,以平盧為節度,以祿山攝中丞為使。入朝奏事,玄宗益寵之。



 三載,代裴寬為範陽節度,河北採訪、平盧軍等使如故。採訪使張利貞常受其賂;數載之後,黜陟使席建侯又言其公直無私;裴寬受代,及李林甫順旨,並言其美。數公皆信臣,玄宗意益堅不搖矣。
 後請為貴妃養兒,入對皆先拜太真。玄宗怪而問之,對曰:「臣是蕃人,蕃人先母而後父。」玄宗大悅,遂命楊銛已下並約為兄弟姊妹。



 六載,加大夫。常令劉駱谷奏事。與王鉷俱為大夫。李林甫為相,朝臣莫敢抗禮,祿山承恩深。入謁不甚罄折。林甫命王鉷,鉷趨拜謹甚,祿山悚息,腰漸曲。每與語,皆揣知其情而先言之。祿山以為神明,每見林甫,雖盛冬亦汗洽。林甫接以溫言,中書引坐,以己披袍覆之,祿山欣荷,無所隱,呼為十郎。駱谷奏事,
 先問:「十郎何言?」有好言則喜躍,若但言「大夫須好檢校」,則反手據床曰:「阿與,我死也!」李龜年嘗敩其說,玄宗以為笑樂。



 晚年益肥壯,腹垂過膝,重三百三十斤,每行以肩膊左右抬挽其身,方能移步。至玄宗前,作胡旋舞,疾如風焉。為置第宇,窮極壯麗,以金銀為篣筐笊籬等。上御勤政樓,於御坐東為設一大金雞障,前置一榻坐之,卷去其簾。十載入朝,又求為河東節度,因拜之。



 男十一人:長子慶宗,太僕卿,少子慶緒,鴻臚卿。慶宗又尚郡主。



 祿山陰有逆謀,於範陽北築雄武城,外示御寇,內貯兵器,積穀為保守之計,戰馬萬五千匹,牛羊稱是。兼三道節度,進奏無不允。引張通儒、李庭堅、平冽、李史魚、獨孤問俗在幕下,高尚掌書記,劉駱谷留居西京為耳目,安守忠、李歸仁、蔡希德、牛庭玠、向潤客、崔乾祐、尹子奇、何千年、武令珣、能元皓、田承嗣、田乾真,皆拔於行間。每月進奉生口駝馬鷹犬不絕,人無聊矣。既肥大不任戰,前後十餘度欺誘契丹,宴設酒中著莨菪子,預掘一坑,待
 其昏醉,斬首埋之,皆不覺死,每度數十人。十一載八月,祿山並率河東等軍五六萬,號十五萬,以討契丹。去平盧千餘里,至土護真河,即北黃河也。又倍程三百里,奄至契丹牙帳。屬久雨,弓箭皆漲濕,將士困極,奚又夾攻之,殺傷略盡。祿山被射,折其玉簪,以麾下奚小兒二十餘人走上山,墜坑中,其男慶緒等扶持之。會夜,解走,投平盧城。



 楊國忠屢奏祿山必反。十二載,玄宗使中官輔璆琳覘之,得其賄賂,盛言其忠。國忠又云「召必不至」,洎
 召之而至。十三載正月,謁於華清宮,因涕泣言:「臣蕃人,不識字,陛下擢臣不次,被楊國忠欲得殺臣。」玄宗益親厚之,遂以為左僕射,卻回。其月,又請為閑廄、隴右群牧等都使,奏吉溫為武部侍郎、兼中丞,為其副,又請知總監事。既為閑廄、群牧等使,上筋腳馬,皆陰選擇之,奪得樓煩監牧及奪張文儼馬牧。三月一日,歸範陽,疾行出關,日行三四百里,至範陽,人言反者,玄宗必大怒,縛送與之。十四載,玄宗又召之,托疾不至。賜其子婚,令就觀
 禮,又辭。



 十一月,反於範陽,矯稱奉恩命以兵討逆賊楊國忠。以諸蕃馬步十五萬,夜半行,平明食,日六十里。以高尚、嚴莊為謀主,孫孝哲、高邈、何千年為腹心。天下承平日久,人不知戰,聞其兵起,朝廷震驚。禁衛皆市井商販之人,乃開左藏庫出錦帛召募。因以高仙芝、封常清等相次為大將以擊之。祿山令嚴肅,得士死力,無不一當百,遇之必敗。



 十二月,度河至陳留郡,河南節度張介然城陷死之,傳首河北。陳留郭門祿山男慶緒見誅慶
 宗榜,泣告祿山,祿山在輿中驚哭曰:「吾子何罪而殺之!」狂而怒,官軍之降者夾道,命交相斫焉,死者六七千人,遂入陳留郡。太守郭納初拒戰,至是出降。至滎陽,太守崔無詖拒戰,城陷死之。次於泥水罌子穀,將軍荔非守瑜蹲而射之,殺數百人,矢及祿山輿。祿山不敢過,乃取谷南而過。守瑜箭盡,投河而死。東京留守李心妻、中丞盧奕、採訪使判官蔣清燒絕河陽橋。祿山怒,率軍大至。封常清自苑西隤墻,使伐樹塞路而奔。祿山入東京,殺李
 心妻、盧奕、蔣清,召河南尹達奚珣,使之蒞事。初,常清欲殺珣,恐應賊,心妻、奕諫止之。常清既敗,唯與數騎走至陜郡,高仙芝率兵守陜城,皆棄甲西走潼關,懼賊追躡,相蹂藉而死者塞路。陜郡太守竇庭芝走投河東。賊使崔乾祐守陜郡。臨汝太守韋斌降於賊。



 十五年正月,賊竊號燕國,立年聖武,達奚珣已下署為丞相,五月,南陽節度魯炅率荊、襄、黔中、嶺南子弟十萬餘,與賊將武令珣戰於葉縣城北枌河,王師盡沒。六月,李光弼、郭子儀出土
 門路,大破賊眾於常山郡東嘉山,河北諸郡歸降者十餘。祿山窘急,圖欲卻投範陽。會哥舒翰自潼關領馬步八萬,與賊將崔乾祐戰於靈寶西,為賊覆敗,翰西奔潼關,為其帳下執送於賊。關門不守,玄宗幸蜀,太子收兵靈武。賊乃遣張通儒為西京留守,田乾真為京兆尹,安守忠屯兵苑中。十一月,遣阿史那承慶攻陷潁川,屠之。



 祿山以體肥,長帶瘡。及造逆後而眼漸昏,至是不見物。又著疽疾。俄及至德二年正月朔受朝,瘡甚而中罷。以
 疾加躁急,動用斧鉞。嚴莊亦被捶撻,莊乃日夜謀之。立慶緒於戶外,莊持刀領豎李豬兒同入祿山帳內,豬兒以大刀斫其腹。祿山眼無所見,床頭常有一刀,及覺難作,捫床頭不得,但撼幄帳大呼曰:「是我家賊!」腹腸已數斗流在床上,言訖氣絕。因掘床下深數尺為坑,以氈罽包其尸埋之。又無哭泣之儀。莊即宣言於外,言祿山傳位於晉王慶緒,尊祿山為太上皇。慶緒縱樂飲酒無度,呼莊為兄,事之大小必咨之。



 初,豬兒出契丹部落,十數
 歲事祿山,甚黠慧。祿山持刃盡去其勢,血流數升,欲死,祿山以灰火傅之,盡日而蘇,因為閹人。祿山頗寵之,最見信用。祿山肚大,每著衣帶,三四人助之,兩人抬起肚,豬兒以頭戴之,始取裙褲帶及系腰帶。玄宗寵祿山,賜華清宮湯浴,皆許豬兒等入助解著衣服,然終見刳者,豬兒也。



 慶緒,祿山第二子也。母康氏,祿山糟糠之妻。慶緒善騎射,祿山偏愛之。未二十,拜鴻臚卿,兼廣陽太守。初名仁執,玄宗賜名慶緒,為祿山都知兵馬使。嚴莊、高
 尚立為偽主。慶緒素懦弱,言詞無序,莊恐眾不伏,不令見人。莊為偽御史大夫、馮翊郡王,以專其政。厚其軍將官秩,以固其心。



 二月,肅宗南幸鳳翔郡,始知祿山死,使僕固懷恩使於回紇,結婚請兵討逆。其月,郭子儀拔河東郡,崔乾祐南遁。八月,回紇三千騎至。九月,廣平王領蕃漢之眾收西京,走安守忠,賊之死者積如山阜。



 十月,賊將尹子奇攻陷睢陽郡,殺張巡、姚摐等。王師乘勝至陜郡,賊懼,令嚴莊傾其驍勇而來拒。廣平王遣副元帥
 郭子儀等與賊戰於陜西曲沃,大破之於新店,逐北二十里,斬首十餘萬,伏尸三十里。嚴莊奔至東京,告慶緒,慶緒率其餘眾奔河北,保鄴郡。嚴莊至河內,南來歸順。賊將阿史那承慶等麾下三萬餘人,悉奔恆、趙、範陽。從慶緒者,唯疲卒一千三百而已。偽中書令張通儒秉政,改相州為成安府,署置百官。旬日之內,賊將各以眾至者六萬餘,兇威復振。偽青、齊節度能元皓獨率眾歸順,明年,改乾元元年,偽德州刺史王暕、貝州刺史宇文寬
 等皆歸順,河北諸軍各以城守累月,賊使蔡希德、安太清急擊,復陷於賊,虜之以歸,臠食其肉。其下潛謀歸順者眾矣,賊皆易置之,以縱屠戮,人心始離。又不親政事,繕治亭沼樓船,為長夜之飲。高尚等各不相葉。蔡希德兵最銳,性剛直,張通儒譖而縊殺之,三軍冤痛不為用。以崔乾祐為天下兵馬使,權領中外兵。乾祐性愎戾,士卒不附。



 九月,肅宗遣郭子儀等九節度率步騎二十萬攻之,以魚朝恩為軍容使。初,子儀之列陳也,使善射者
 三千人伏於壘垣內。明日接戰,子儀麾其屬偽奔,慶緒逐之,伏者齊發,賊黨大潰。使薛嵩求救於史思明,言禪讓之禮。思明先遣李歸仁以步卒一萬、馬軍三千,先往滏陽以應。及至滏陽,子儀之圍已固,築城穿壕各三重,樓櫓之盛,古所未有。又引水以灌城下,城中水泉大上,井皆滿溢。以安太清代乾祐為都知兵馬使。思明南攻魏州,節度使崔光遠南走,思明據其城數日,即乾元二年正月一日也。思明偽稱燕王,立年號。



 慶緒自十月被
 圍至二月,城中人相食,米斗錢七萬餘,鼠一頭直數千,馬食隤墻麥鞬及馬糞濯而飼之。思明引眾來救,三月六日,子儀等戰敗,遂解圍而南,斷河陽橋以守谷水。思明領其眾營於鄴縣南。慶緒使收子儀等營中糧,尚六七萬石,復與孫孝哲、乾祐謀閉門自守,議更拒思明。諸將曰「今日安可更背史王乎!」張通儒、高尚、平冽謂慶緒曰:「史王遠來,臣等皆合迎謝。」對曰:「任公暫往見思明。」思明與之涕泗,厚其禮,復命歸城。經三日,慶緒不至。思明
 密召安太清令誘之。慶緒不獲已,以三百騎詣思明。思明引入,令三軍擐甲執兵待之。及諸弟領至於庭,再拜稽首曰:「臣不克負荷,棄失兩都,久陷重圍,不意大王以太上皇故,將兵遠救。」思明曰:「棄失兩都,用兵不利,亦何事也!爾為人子,殺汝父以求位,庸非大逆乎?吾為太上皇討賊。」即牽出,並其四弟及高尚、孫孝哲、崔乾祐,皆縊殺之。



 祿山父子僭逆三年而滅。初王師之圍相州也,意朝夕屠陷,唯術士桑道茂曰:「三月六日,西師必散,此城
 無憂。」卒如其言。



 高尚,幽州雍奴人也,本名不危。母老,乞食於人,尚周游不歸侍養。寓居河朔縣界,與令狐潮鄰里,通其婢,生一女,遂收之。尚頗篤學,贍文詞。嘗嘆息謂汝南周銑曰:「高不危寧當舉事而死,終不能咬草根以求活耳!」縣尉有姓高者,以其宗盟,引置門下,遂以尚入籍為兄弟。李齊物為懷州刺史,舉高尚不仕,送京師,並助錢三萬。齊物寓書於中官將軍吳懷實以托之。懷實引見高力士,置
 賓館中,令與男丞相錫為學,無問家事,一以委之。無何,令妻父呂令皓特表薦之。



 天寶元年,拜左領軍倉曹參軍同正員。六載,安祿山奏為平盧掌書記,出入祿山臥內。祿山肥多睡,尚執筆在旁或通宵焉,由是浸親厚之。遂與祿山解圖讖,勸其反。



 天寶十一年,祿山表為屯田員外郎。及隨祿山寇陷東京,偽授中書侍郎。偽赦書制敕多出其手。始,尚與嚴莊、孫孝哲計畫,白祿山以為事必成。及顏杲卿殺李欽湊於土門,揚聲言榮王琬、哥舒
 翰二十萬眾徇河北,十七郡皆歸順。顏真卿破袁知奉三萬眾於堂邑,賀蘭進明再拔信都,李光弼、郭子儀繼收常山、趙郡,河北路絕者再。河南諸郡皆有防禦,潼關有哥舒翰之師。祿山大懼,怒尚等曰:「汝元向我道萬全,必無所畏。今四邊若此,賴鄭、汴數州尚存,向西至關,一步不通,河北並已無矣,萬全何在?更不須見我。」尚等遂數日不得見祿山,憂悶不知所為。



 會田乾真自潼關至,曉諭祿山曰:「自古帝王,皆有勝敗,然後成大事,豈有一
 舉而得之者乎!今四邊兵馬雖多,皆非精銳,豈我之比。縱事不成,收取數萬眾,橫行天下,為一盜跖,亦十年五歲矣,豈有人能制我耶!尚、莊等皆佐命元勛,何得隔絕不與相見,令其憂懼?只此數人,豈不能為患乎?外間聞之,必心搖動。」祿山喜曰:「阿浩,非汝誰能開豁我心裡事,今無憂矣!為之奈何?」乾真曰:「不如喚取慰勞之。」遂召尚等飲宴作樂,祿山自唱歌以送酒,待之如初。阿浩,乾真小字也。及慶緒至相州,偽授侍中。



 孫孝哲,契丹人也。母為祿山所通,因得狎近。及祿山僭逆,偽授殿中監、閑廄使,封王。孝哲尤用事,亞於嚴莊。裘馬華侈,頗事豪貴,每食皆備珍饌。性殘忍,果於殺戮,聞者畏之。祿山使孝哲與張通儒同守西京,妃王宗枝皆罹其酷。與嚴莊爭權不睦。及祿山死,奪其使,以鄧季陽代之。慶緒之奔,莊懼為所圖,因而來奔。



 史思明,本名窣幹。營州寧夷州突厥雜種胡人也。姿瘦,少須發,鳶肩傴背,欽目側鼻。性急躁。與安祿山同鄉里,
 先祿山一日生,思明除日生,祿山歲日生。及長,相善,俱以驍勇聞。初事特進烏知義,每令騎覘賊,必生擒以歸。又解六蕃語,與祿山同為互市郎。張守珪為幽州節度,奏為折沖。天寶初,頻立戰功,至將軍,知平盧軍事。嘗入奏,玄宗賜坐,與語,甚奇之。問其年,曰「四十矣」。玄宗撫其背曰:「卿貴在後,勉之。」遷大將軍、北平太守。十一載,祿山奏授平盧節度都知兵馬使。



 十四載,安祿山反,命思明討饒陽等諸郡,陷之。十五載正月六日,思明與蔡希德
 圍顏杲卿於常山,九日拔之。又圍饒陽,二十九日不能拔。李光弼出土門,拔常山郡,思明解圍而拒光弼。光弼列兵於城南,相持累月。光弼草盡,使精卒以車數乘於旁縣取草,輒被擊之,其後率十匹唯共得兩束草,至剉蒿薦以飼之。初,祿山以賈循為範陽留後,謀歸順,為副留守向潤客所殺,以思明代之。又以征戰在外,令向潤客代其任。四月,朔方節度郭子儀以朔方蕃、漢二萬人自土門而至常山,軍威遂振,南拔趙郡,思明退保博陵。
 五月十日,子儀、光弼擊之,敗思明於沙河上。又攻之,思明以騎卒奔嘉山,光弼擊之,思明大敗,走入博陵郡。光弼圍之,城幾拔。屬潼關失守,肅宗理兵於朔方,使中官邢廷恩追朔方、河東兵馬。光弼入土門,思明隨後徼擊之。已而回軍並行擊劉正臣,正臣易之。初不設備,遂棄軍保北平,正臣妻子及軍資二千乘盡沒。



 思明將卒頗精銳,皆平盧戰士,南拔常山、趙郡,又攻河間。為尹子奇所圍,已四十餘日。顏真卿使和琳以一萬二千人、馬百
 匹以救之。至河間二十餘里,北風勁烈,鼓聲不相聞,賊縱擊之,擒和琳以至城下。思明既全,合勢,賊軍益盛。李奐為賊所擒,送東京。又攻景城,擒李暐,暐投河而死。遂使康沒野波攻平原。真卿覺之,兵馬既盡,渡河而南。攻清河,糧盡城陷,擒太守王懷忠以獻祿山。將軍莊嗣賢圍烏承恩於信都。承恩母、妻先為安祿山所獲,思明獲其男從則,使諭承恩,承恩遂降,思明與之把臂飲酒。饒陽陷,李系投火死。河北悉陷。尹子奇以五萬眾渡河至
 青州,欲便向江、淮。會回紇二千騎奄至範陽,範陽閉門二日,然後向太原,子奇行千里以救之。二年正月,思明以蔡希德合範陽、上黨兵馬十萬,圍李光弼於太原。光弼使為地道,至賊陣前。驍賊方戲弄城中人,地道中人出擒之。敵以為神,呼為「地藏菩薩」。思明留十月,會安祿山死,慶緒令歸範陽,希德留百餘日,皆不能拔而歸。自祿山陷兩京,常以駱駝運兩京御府珍寶於範陽,不知紀極。由是恣其逆謀。思明轉驕,不用慶緒之命。



 安慶緒
 為王師所敗,投鄴郡,其下蕃、漢兵三萬人,初不知所從,思明擊殺三千人,然後降之。



 慶緒使阿史那承慶、安守忠徵兵於思明,且欲圖之。判官耿仁智,忠謀之士,謂思明曰:「大夫崇重,人不敢言,仁智請一言而死。」思明曰:「試言之。」對曰:「大夫久事祿山,祿山兵權若此,誰敢不服!如大夫比者,逼於兇威耳,固亦無罪。今聞孝感皇帝聰明勇智,有少康、周宣之略,大夫發使輸誠,必開懷見納,此轉禍為福之上策也。」思明曰:「善。」承慶等以五千騎至範
 陽,思明悉眾介胄以逆之。眾且數萬,去之一里,使謂之曰:「相公及王遠至,將士等不勝喜躍。此皆邊兵怯懦,頗懼相公之來,莫敢進也。請弛弓以安之。」從之。思明遂以承慶、守忠入內,飲樂之。別令諸將於其所分收其甲仗。其諸郡兵皆給糧,恣歸之,欲留者分隸諸營。遂拘承慶,斬守忠、李立節之首。李光弼使衙官敬俛招之。遂令衙官竇子昂奉表,以所管兵眾八萬人,及以偽河東節度高秀巖來降。肅宗大悅,封歸義王、範陽長史、御史大
 夫、河北節度使,朝義已下並為列卿,秀巖雲中太守,以其男如岳等七人為大官。使內侍李思敬、將軍烏承恩宣慰使,令討殘賊。



 明年,改乾元元年,四月,肅宗使烏承恩為副使,候伺其過而殺之。初,承恩父知義為節度,思明常事知義,亦有開獎之恩,以此李光弼冀其無疑,因謀殺之。承恩至範陽,數漏其情,夜取婦人衣衣之,詣諸將家,以翻動之意諭之。諸將以白思明,甚懼,無以為驗。有頃,承恩與思敬從上京來,宣恩命畢,將歸私第。思明留承
 恩且於館中,明當有所議。已令幃其所寢之床,伏二人於其下。承恩有小男,先留範陽,思明令省其父。夜後,私於其子曰:「吾受命除此逆,明便授吾節度矣!」床下二人叫呼而出,以告思明。思明令執之,搜其衣曩,得朝廷所與阿史那承慶鐵券及光弼與承恩之牒,云:「承慶事了,即付鐵券;不了,不可付之。」又得簿書數百紙,皆載先所從反軍將名。思明語之曰:「我何負於汝而至是耶?」承恩稱:「死罪,此太尉光弼之謀也!」思明集軍將官吏百姓,西
 向大哭曰:「臣以十三州之地、十萬眾之兵降國家,赤心不負陛下,何至殺臣!」因搒殺承恩父子,囚李思敬,遣使表其事。朝廷又令中使慰諭云:「國家與光弼無此事,乃承恩所為,殺之善也。」



 又有使從京至,執三司議罪人狀。思明曰:「陳希烈已下,皆重臣,上皇棄之幸蜀,既收復天下,此輩當慰勞之。今尚見殺,況我本從祿山反乎!」諸將皆云:「烏承恩之前事,情狀可知,光弼尚在,憂不細也。大夫何不取諸將狀以誅光弼,以謝河北百姓!主上若不
 惜光弼,為大夫誅之,大夫乃安;不然,為患未已。」思明曰:「公等言是。」乃令耿仁智、張不矜修表:「請誅光弼以謝河北。若不從臣請,臣則自領兵往太原誅光弼。」不矜初以表示思明,及封入函,耿仁智盡削去之。寫表者密告思明,思明大怒,執二人於庭曰:「汝等何得負我?」命斬之。仁智事思明頗久,意欲活之,卻令召入,謂之曰:「我任使汝向三十年,今日之事,我不負汝。」仁智大呼曰:「人生固有一死,須存忠節。今大夫納邪說,為反逆之計,縱延旬月,
 不如早死,請速加斧鉞!」思明大怒,亂捶殺之,腦流於地。



 十月,郭子儀領九節度圍相州,安慶緒偷道求救於思明,思明懼軍威之盛,不敢進。十二月,蕭華以魏州歸順,詔遣崔光遠替之。思明擊而拔其城,光遠脫身南渡。思明於魏州殺三萬人,平地流血數日,既乾元二年正月一日也。思明於魏州北設壇,僭稱為大聖燕王,以周贄為行軍司馬。三月,引眾救相州,官軍敗而引退。思明召慶緒等殺之,並有其眾。四月,僭稱大號,以周贄為相,以
 範陽為燕京。九月,寇汴州,節度使許叔冀合於思明,思明益振。又陷洛陽,與太尉光弼相拒。思明恣行兇暴,下無聊矣!



 上元二年,潛遣人反說官軍曰:「洛中將士,皆幽、朔人,咸思歸。」魚朝恩以為然,告光弼及諸節度僕固懷恩、衛伯玉等:「可速出兵以討殘賊。」光弼等然之,乃出師兩道齊進。次榆林,賊委物偽遁,將士等不復設備,皆入城虜掠。賊伏兵在北邙山下,因大下,士卒咸棄甲奔散。魚朝恩、衛伯玉退保陜州,光弼、懷恩棄河陽城,退居聞
 喜。步兵散死者數千人,軍資器械盡為賊所有,河陽、懷州盡陷於賊。



 思明至陜州,為官軍所拒於姜子阪,戰不利,退歸永寧。築三角城,約一月內畢,以貯軍糧。朝義築城畢,未泥,思明至,詬之。對曰:「緣兵士疲乏,暫歇耳!」又怒曰:「汝惜部下兵,違我處分。」令隨身數十人立馬看泥,斯須而畢。又曰:「待收陜州,斬卻此賊。」朝義大懼。思明居驛,朝義在店中。思明令腹心曹將軍總中軍兵嚴衛,朝義將駱悅並許叔冀男季常等言:「主上欲害王,悅與王死
 無日矣!」因言:「廢興之事,古來有之,欲喚取曹將軍舉大事,可乎?」朝義回面不應。悅曰:「若不應,悅等即歸李家,王亦不全矣!」朝義然之,令許季常命曹將軍至。悅等告之,不敢拒。其夜,思明夢而驚悟,據床惆悵。每好伶人,寢食置左右,以其殘忍,皆恨之。及此,問其故,曰:「吾向夢見水中沙上群鹿渡水而至,鹿死水乾。」言畢如廁。伶人相謂曰:「鹿者,祿也;水者,命也。胡祿命俱盡矣!」駱悅入,問思明所在,未及對,殺數人,因指在廁。思明覺變,逾墻出,至馬
 槽,備馬騎之。悅等至,令傔人周子俊射,中其臂,落馬。曰:「是何事?」悅等告以懷王。思明曰:「我朝來語錯,今有此事。然汝殺我太疾,何不待我收長安?終事不成矣!」因急呼懷王者三,曰:「莫殺我!」卻罵曹將軍曰:「這胡誤我,這胡誤我!」悅遂令心腹擒思明赴柳泉驛,曰:「事已成矣!」朝義曰:「莫驚聖人否?莫損聖人否?」悅曰:「無有。」時周贄、許叔冀統後軍在福昌,朝義令許季常往告之。贄聞,驚欲仰倒。朝義卻領兵回,贄等來迎,因殺贄。思明至柳泉驛,縊殺之。朝
 義便僭偽位。



 朝義,思明孽子也。寬厚,人附之,使人往範陽,殺偽太子朝英等。偽留守張通儒覺之,戰於城中。數日,死者數千人,始斬之。時洛陽四面數百里,人相食,州縣為墟。諸節度使皆祿山舊將,與思明等夷,朝義徵召不至。



 寶應元年十月,遣元帥雍王領河東朔方諸節度、回紇兵馬赴陜。僕固懷恩與回紇左殺為先鋒,魚朝恩、郭英乂為後殿,自澠池入;李抱玉自河陽入;副元帥李光弼自陳留入;雍王留陜州。二十九日,與朝義戰於邙
 山之下。逆賊敗績,走渡河,斬首萬六千,生擒四千六百,降三萬二千人,器械不可勝數。朝義走投汴州,汴州偽將張獻誠拒之,乃渡河北投幽州。二年正月,賊偽範陽節度李懷仙於莫州生擒之,送款來降,梟首至闕下。又以偽官以城降者恆州刺史、成德軍節度張忠志為禮部尚書,餘如故。趙州刺史盧淑、定州程元勝、徐州劉如伶、相州節度薛嵩、幽州李懷仙、鄭州田承嗣並加封爵,領舊職。



 思明乾元二年僭號,至朝義寶應元年滅,凡四
 年。



\end{pinyinscope}