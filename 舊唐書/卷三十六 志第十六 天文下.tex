\article{卷三十六 志第十六 天文下}

\begin{pinyinscope}

 天文之為十
 二次,所以辨析天體,紀綱辰象,上以考七曜之宿度,下以配萬方之分野,仰觀變謫,而驗之於郡國也。《傳》曰:「歲在星紀,而淫於玄枵。」「姜氏、任氏,實守其地。」及七國交爭,善星者有甘德、石申,更配十二分野,故有周、秦、齊、楚、韓、趙、燕、魏、宋、衛、魯、鄭、吳、越等國。張衡、蔡邕,又以漢郡配焉。自此因循,但守其舊文,無所變革。且懸象在上,終天不易,而郡國沿革,名稱屢遷,遂令後學難為憑準。貞觀中,李淳風撰《法象志》,始以唐之州縣配焉。至開元初,沙門一行又增損其書,更為
 詳密。既事包今古,與舊有異同,頗裨後學,故錄其文著於篇。並配武德以來交蝕淺深及注蝕不虧,以紀日月之變云爾。



 須女、虛、危,玄枵之次。子初起女五度,二千三百七十四分,秒四少。中虛九度,終危十二度。其分野:自濟北郡東逾濟水,涉平陰至於山茌,漢太山郡山茌縣,屬齊州西南之界。東南及高密,漢高密國,今在密州北界。自此以上,玄枵之分。東盡東萊之地,漢之東萊郡及膠東國,今為萊州、登州也。又得漢之北海、
 千乘、淄川、濟南、齊郡,今為淄、青、齊等州,及濟州東界。及平原、渤海,盡九河故道之南,濱於碣石。今為德州、棣州,滄州其北界。自九河故道之北,屬析木分也。



 營室、東壁,陬訾之次。亥初起危十三度,二千九百二十六分太。中室十二度,五百五十分,秒二十一半。終奎一度。其分野:自王屋、太行而東,盡漢河內之地,今為懷州、洺、衛州之西境。北負漳、鄴,東及館陶、聊城,漢地自黎陽、內黃及鄴、魏、武安,東至館陶、元城,皆屬魏郡;自頓邱、三城、武陽,東至聊城,皆屬東郡。今為相、魏、衛州。東盡漢東郡之地,漢東郡、清河,西南至白馬、濮陽,東至東河、須昌,濱濟,至於鄆城。今為
 滑州、濮州、鄆州。其須昌、濟東之地,屬降婁,非豕韋也。



 奎、婁及胃,降婁之次。戌初起奎二度,一千二百一十七分,秒十七少。中婁一度,一千八百八十
 三。終胃
 三度。其分野:南屆鉅野,東達梁父,以負東海。又東至於呂梁,乃東南抵淮水,而東盡於徐夷之地。東為降婁之次。得漢東平、魯國。漢東平國在任城、平陸,今在
 兗州。奎為大澤,在陬訾之下流,濱於淮、泗,東北負山,為婁、胃之墟。蓋中國膏腴之地,百穀之所阜也。胃星得馬牧之氣,與冀之北土同占。



 昴、畢,大梁之
 次。畢酉初起胃四度,二千五百四十九分,秒八太。中昴六度,一百七十四分半。終畢九度。其分野:自魏郡濁漳之北,得漢之趙國、廣平、鉅鹿、常山,東及清河、信都,北據中山、真定。今為洺、趙、邢、恆、定、冀、貝、深八州。又分相、魏、博之北界,與瀛州之西,全趙之分。又北盡漢代郡、雁門、雲中、定襄之地,與北方群狄之國,皆大梁分也。



 觜觿、參伐,實沈之次也。申初起畢十度,八百四十一分,十五太。中
 參七度,
 一千五百
 二十六,終井十一度。其分野:得漢河東郡,今為蒲、絳、晉州,又得澤州及慈州界也。及上黨、今為澤、潞、儀、沁也。太原,今為並、汾州。盡西河之地。今為隰州、石州、嵐州,西涉河,得銀州以北也。又西河戎狄之國,皆實沈分也。今河東郡永樂、芮城、河北縣及河曲豐、勝、夏州,皆為實沈之次,東井之分也。參伐為戎索,為武政,故殷河東,盡大夏之墟。上黨次居下流,與趙、魏相接,為觜觿之分。



 東井、輿鬼,鶉首之次也。未初起井十二度,二千一百七十二秒,十五太。中井二十七度,二千八百二十八分,秒一半。終柳六度。其分野:自漢之三輔及北地、上郡、安定,西自隴坻至河西,西南盡巴、蜀、漢中之地,及西南夷犍為、越巂、益州郡,極南河之表,東至䍧柯,皆鶉首分也。鶉首之分,得《禹貢》雍、梁二州,其郡縣易知,故不詳載。狼星分野在江、河上源之西,孤矢、犬、雞,皆徼外之象。今之西羌、吐蕃、蕃渾,及西南徼外夷,皆狼星之象。



 柳、星、張,鶉火之次。午初起柳七度,四百六十四,秒七少。中七星七
 度,一千一百三。終張十四度。其分野:北自滎澤、滎陽,並京、索,暨山南,得新鄭、密縣,至於方陽。方陽之南得漢之潁川郡陽翟、崇高、郟城、襄城,南盡鄴縣。今為鄧、汝、唐、仙四州界。又漢南陽郡,北自宛、葉,南盡漢東申、隨之地,大抵以淮源桐柏、東陽為限。今之唐州、隨州屬鶉火,申州屬壽星。又自洛邑負河之南,西及函谷南紀,達武當漢水之陰,盡弘農郡。漢弘農盧氏、陜縣,今為虢、陜二州。上洛、商洛為商州。丹水為均州。宜陽、沔池、新安、陸渾,今屬洛州。古成周、虢、鄭、管、鄶、東虢、密、滑、焦、唐、申、鄧,皆鶉火分也,及祝融氏之都。新鄭為祝融氏之墟,屬鶉火。其東鄙則入壽星。舊說皆在函谷,非也。柳、星、輿鬼之東,又接漢源,故殷商、洛之陽,接
 南河之上流。七星上系軒轅,得土行之正位,中嶽象也,故為河南之分。張星直河南漢東,與鶉尾同占。



 翼、軫,鶉尾之次。巳初起張十五度,一千七百九十五,秒二十二少。中翼十二度,二千四百六十一,秒八半。終軫九度。其分野:自房陵、白帝而東,盡漢之南郡、南郡:巫縣,今在夔州。秭歸在西,夷陵在峽州。襄、夔、郢、申在襄、郢界,餘為荊州。江夏,江夏:竟陵今為復州,安、鄂、蘄、沔、黃五州,皆漢江夏界。東達廬江南郡。漢廬江之尋陽,今在江州,於山河之像,宜屬鶉尾也。濱彭蠡之西,得漢長沙、武陵、桂陽、零陵郡。零陵今為道州、永州。桂陽今為郴州。大抵自沅、湘上流,西通黔安之左,皆楚之分也。又逾南
 紀,盡鬱林、合浦之地。鬱林縣今在貴州。定林縣今在廉州。合浦縣今為桂州。今自富、昭、蒙、龔、繡、容、白、罕八州以西,皆屬鶉尾之墟也。荊、楚、鄖、鄀、羅、權、巴、夔與南方蠻貊,殷河南之南。其中一星主長沙國,逾嶺徼而南,皆甌東、青丘之分。今安南諸州,在雲漢上源之東,宜屬鶉火。



 角、亢,壽星之次。辰初起軫十度,八十七,秒十四半。中角八度,七百五十,秒三十。終氐一度。其分野:自原武、管城,濱河、濟之南,東至封邱、陳留,盡陳、蔡、汝南之地,逾淮源至於弋陽。漢陳留郡,自封邱、陳留已東,皆入大火之分。漢汝南,今為豫州。西華、南頓、項城縣今為陳州。汝陰縣今在潁州。弋陽縣在光
 州。西涉南陽郡,至於桐柏,又東北抵嵩之東陽。漢南陽郡舂陵、湖陽,蔡陽後分為舂陵郡,後魏以為南荊州,今有舊義陽郡,在申國之東界,今為申州。按中國地絡,在南北河之間,故申、隨、光三州,皆屬《禹貢》豫州之分,宜屬鶉火、壽星。非南方負海之地。古陳、蔡、隨、許,皆屬壽星分也。氐星涉壽星之次,故其分野殷雒邑眾山之東,與亳土相接。



 氐、房、心,大火之次也。卯初起氐二度,一千四百一十九分,秒五太。中房二度,二千八百五分,秒一半。終尾六度。其分野:得漢之陳留縣,自雍丘、襄邑、小黃而東,循濟陰,界於齊、魯,右泗水,達於
 呂梁,乃東南抵淮,西南接太昊之墟,盡濟陰、山陽、楚國、豐、沛之地。濟陰郡之定陶、冤句、乘氏,今在東郡。大抵曹、宋、徐、亳及鄆州西界,皆屬大火分。自商、亳以負北河,陽氣之所升也,為心分。自豐、沛以負南河,陽氣之所布也,為房分。故其下流皆與尾星同占,西接陳、鄭,為氐星之分。



 尾、箕,析木之次也。寅初起尾七度,二千七百五十分,秒二十一少。中箕星五度,三百七十分,秒六十七。終斗八度。其分野:自渤海九河之北,盡河間、涿郡、廣陽國,漢渤海郡浮陽,今為清池縣,屬滄州。涿郡之饒陽,今屬瀛州。涿縣、良
 鄉與廣陽國薊縣,今在幽州。及上谷、漁陽、右北平、遼東、樂浪、玄菟,漁陽在幽州。右北平在白狼無終縣,隋代為漁陽郡,古孤竹國,後置北平郡,今為平州。遼東在無慮縣,即《周禮》醫無閭山。樂浪在朝鮮縣,玄菟在高句驪縣,今皆在東夷也。古之北燕、孤竹、無終及東方九夷之國,皆析木之分也,尾得雲漢之末流,北紀之所窮也。箕與南斗相近,故其分野在吳、越之東。



 南斗、牽牛,星紀之次也。丑初起斗九度,一千四十二十分,秒二太。中斗二十四度,一千一百分,秒八半。終女四度。其分野:自廬江、九江,負淮水之南,盡臨淮、廣陵,至於東海,廬、壽、和、濠、揚,皆屬星紀也。又逾
 南河,得漢丹陽、會稽、豫章郡,西濱彭蠡,南涉越州,盡蒼梧、南海。又逾嶺表,自韶、廣、封、梧、藤、羅、雷州,南及珠崖自北以東為星紀,其西皆屬鶉尾之次。古吳、越及東南百越之國,皆星紀分也。南斗在雲漢之下流,當淮、海之間,為吳分。牽牛去南河浸遠,故其分野自豫章東達會稽,南逾嶺徼,為越分。島夷蠻貊之人,聲教之所不洎,皆系於狗國。李淳風刊定《隋志》,郡國頗為詳悉,所注郡邑多依用。其後州縣又隸管屬不同,但據山河以分耳。



 災異



 武德元年十月壬申朔,四年八月丙戌朔,六年十二月壬寅朔,九年十月丙辰朔。貞觀元年閏三月癸丑朔,九月庚戌朔,二年三月戊申朔,三年八月己巳朔,四年閏正月丁卯朔,六年正月乙卯朔,九年閏四月丙寅朔,十一年三月丙戌朔,十二年閏二月庚辰朔,十三年八月辛未朔,十七年六月己卯朔,十八年十月辛丑朔,二十年閏三月癸巳朔,二十二年八月己酉朔。高宗顯慶五年六月庚午朔。乾封二年八月己酉朔。總章二年六月
 戊申朔。咸亨元年六月壬寅朔,二年十一月甲午朔,三年十一月戊子朔。上元元年三月辛亥朔,二年九月壬寅朔。調露二年四月乙巳朔,十一月壬寅朔。開耀元年十月丙寅朔。永淳元年四月甲子朔,十一月庚申朔。則天垂拱二年二月辛未朔,四年六月丁亥朔。天授二年四月壬寅朔。如意元年四月丙申朔。長壽二年九月丁亥朔,三年九月壬午朔。延載元年九月壬午朔。證聖元年二月己酉朔。聖歷三年五月乙酉朔。久視元年五月
 己酉朔。長安二年九月乙丑朔,三年三月壬戌朔,九月庚寅朔。中宗神龍三年六月丁卯朔。景龍元年十二月乙丑朔。睿宗太極元年二月丁卯朔。玄宗先天元年九月丁卯朔。開元三年七月庚辰朔,六年五月乙丑朔,九年五月乙巳朔,十二年閏十二月壬辰朔,十七年十月丙午朔,二十年二月癸酉朔,八月辛未朔,二十一年七月乙丑朔,二十二年十二月戊子朔,二十三年閏十一月壬午朔,二十六年九月丙申朔,二十八年三月丁亥
 朔。天寶元年七月癸卯朔,五載五月壬子朔,十三載六月乙丑朔。



 肅宗至德元載十月辛巳朔。上元二年七月癸未朔,蝕既,大星皆見。代宗大歷三年三月乙巳朔,四年正月十五日甲午蝕。十三年甲戌,有司奏合蝕不蝕。十四年二月丙寅朔。德宗貞元三年八月辛巳朔,日蝕。有司奏,準禮請伐鼓於社,不許。太常卿董晉諫曰:「伐鼓所以責群陰,助陽德,宜從經義。」竟不報。六年正月戊戌朔,有司奏合蝕不蝕,百僚稱賀。七年六月庚寅朔,有司
 奏蝕,是夜陰雲不見,百官表賀。八年十一月壬子朔,先是司天監徐承嗣奏:「據歷,合蝕八分,今退蝕三分。準占,君盛明則陰匿而潛退。請書於史。」從之。十年四月癸卯朔,有司奏太陽合虧,巳正後刻蝕之既,未正後五刻復滿。太常奏,準禮上不視朝。其日陰雲不見,百官表賀。十七年五月壬戌蝕。



 元和三年七月癸巳蝕。憲宗謂宰臣曰:「昨司天奏太陽虧蝕,皆如其言,何也?又素服救日,其儀安在?」李吉甫對曰:「日月運行,遲速不齊。日凡周天三
 百六十五度有餘,日行一度,月行十三度有餘,率二十九日半而與日會。又月行有南北九道之異,或進或退,若晦朔之交,又南北同道,即日為月之所掩,故名薄蝕。雖自然常數可以推步,然日為陽精,人君之象,若君行有緩有急,即日為之遲速。稍逾常度,為月所掩,即陰浸於陽。亦猶人君行或失中,應感所致。故《禮》云:「男教不修,陽事不得,謫見於天,日為之蝕。』古者日蝕,則天子素服而修六官之職,月蝕,則後素服而修六宮之職,皆所以
 懼天戒而自省惕也。人君在民物之上,易為驕盈,故聖人制禮,務乾恭兢惕,以奉若天道。茍德大備,天人合應,百福斯臻。陛下恭己向明,日慎一日,又顧憂天譴,則聖德益固,升平何遠。伏望長保睿志,以永無疆之休。」上曰:「天人交感,妖祥應德,蓋如卿言。素服救日,自貶之旨也,朕雖不德,敢忘兢惕。卿等當匡吾不迨也。」十年八月己亥朔,十三年六月癸丑朔。



 長慶二年四月辛酉朔,三年九月壬子朔。大和八年二月壬午朔。開成二年十二月庚寅朔,
 當蝕,陰雲不見。會昌三年二月庚申朔,四年二月甲寅朔,五年七月丙午朔,六年十二月戊辰朔,皆蝕。武德九年二月二十三日夜,星孛於胃、昴間,凡二十八日,又孛於卷舌。貞觀八年八月二十三日,星孛於虛、危,歷於玄枵,凡十一日而滅。太宗謂侍臣曰:「是何妖也?」虞世南對曰:「齊景公時,有彗星。晏子對曰:『公穿池畏不深,築臺恐不高,行刑恐不重,是以彗為誡耳。』景公懼而修德,十六日而星滅。臣聞若德政不修,麟鳳數見,無所補
 也;茍政教無闕,雖有災愆,何損於時。伏願陛下勿以功高古人而矜大,勿以太平日久而驕逸,慎終如始,彗何足憂。」帝深嘉之。十三年三月二十二日夜,星孛于畢、昴。十五年六月十九日,星孛於太微,犯郎位。七月甲戌滅。總章元年四月,彗見五車,上避正殿,減膳,令內外五品已上上封事,極言得失。許敬宗曰:「星雖孛而光芒小,此非國眚,不足上勞聖慮,請御正殿,復常膳。」不從。敬宗又進曰:「星孛於東北,王師問罪,高麗將滅之徵。」帝曰:「我為萬
 國主,豈移過於小蕃哉!」二十二日星滅。上元二年十月,彗見於角、亢南,長五尺。三年七月二十一日,彗見東井,指南河、積薪,長三尺餘,漸向東北,光芒益穀,長三丈,掃中臺,指文昌,經五十八日而滅。永隆二年九月一日,萬年縣女子劉凝靜乘白馬,著白衣,男子從者八九十人,入太史局,升令床坐,勘問比有何災異。太史令姚玄辯執之以聞。是夜彗見西方天市中,長五尺,漸小,向東行,出天市,至河鼓右旗,十七日滅。永淳二年三月十八
 日,彗見五車之北,凡二十五日而滅。



 文明元年七月二十二日,西方有彗,長丈餘,凡四十九日滅。光宅元年九月二十九日,有星如半月,見西方。景龍元年十月十八日,彗見西方,凡四十三日而滅。二年二月,天狗墜於西南,有聲如雷,野雉皆雊。七月七日,星孛胃、昴之間。三年八月八日,星孛於紫宮。太極元年七月四日,彗入太微。開元十八年六月十一日,彗見五車;三十日,星孛于畢、昴。二十六年三月八日,星孛於紫微垣,歷斗魁,十餘日,
 陰雲不見。武德元年六月三日,熒惑犯左執法。八年九月二十二日,熒惑入太微。九年五月,傅奕奏:太白晝見於秦,秦國當有天下。高祖以狀授太宗。及太宗即位,召奕謂曰:「汝前奏事幾累我,然而今後但須悉心盡言,無以前事為慮。」貞觀十三年五月,熒惑犯右執法。十五年二月十五日,熒惑逆犯太微東籓上相。十七年三月七日,熒惑守心前星,十九日退。其月二十二日,熒惑犯句陳。九月二
 十九日,熒惑犯太微西籓上將。十九年九月二十四日,太白在太微,犯左執法,光芒相及。永徽三年六月二日,熒惑犯右執法;三日,太白入太微,犯右執法。顯慶五年二月三日,熒惑入南斗。龍朔元年九月十四日,太白犯太微左執法。乾封二年五月,熒惑入軒轅。咸亨元年十二月,熒惑入太微。上元二年正月九日,熒惑犯房星。儀鳳四年四月九日,熒惑犯羽林。調露二年五月二十四日,太白經天。



 長安四年,熒惑入月及鎮星,犯天關。太史
 令嚴善思奏:法有亂臣伏罪,臣下謀上之變。歲餘,誅二張,五王立中宗。景龍三年六月八日,太白晝見於東井。景雲二年三月二十七日,太白入羽林。太極元年三月三日,熒惑入東井;四月十二日,熒惑與太白守東井。先天元年八月十四日夜,月蝕盡,有星入月魄中。十六日,太白襲月。開元十年七月二十九日,熒惑入南斗。天寶十三載五月,熒惑守心五十餘日。至德元載十一月二十六日,熒惑、太白同犯昴。
 武德二年三月二十七日,太白、辰、鎮聚於東井。九年六月十八日,辰、歲會於東井。二十三日,辰、歲、太白又會於東井。貞觀十八年五月,太白、辰合於東井。景雲二年七月,太白、鎮同在張宿。武德三年十月三十日,有流星墜於東都城內,殷殷有聲。高祖謂侍臣曰:「此何祥也?」起居舍人令狐德棻曰:「昔司馬懿伐遼,有流星墜於遼東梁水上,尋而公孫淵敗走,晉軍追之,至其星墜處斬之。此王世充滅亡之兆也。」
 貞觀十八年五月,有流星大如斗,五日出東壁,光照地,聲如雷。咸亨三年二月三日,有流星如雷。景龍二年二月十九日,大星墜於西南,聲如雷,野雉皆雊。景雲二年八月十七日,東方有流星出五車,至於上臺。天寶三載閏二月十七日,星墜於東南,有聲。京師訛言官遣棖棖捕人肝以祭天狗,人相恐,畿內尤甚。景龍元年九月十八日,有赤氣竟天,其光燭地,經三日乃止。九月四日,黃霧昏。唐隆元年六月八日,虹霓竟天。



 災異編年至德後



 至德元年三月乙酉,歲、太白、熒惑合於東井。十月辛丑朔,日有食之。十一月壬戌五更,有流星大如斗,流於東北,長數丈,蛇行屈曲,有碎光迸空。乾元元年四月,熒惑、鎮、太白合於營室。太史南宮沛奏:所合之處戰不勝,大人惡之,恐有喪禍。明年冬,郭子儀等九節度之師自潰於相州。五月癸未夜一更三籌,月掩心前星,二更四籌方出。六月癸丑,月入南斗魁。二年二月丙辰,月犯心前大星,相去三寸。三年四月丁巳夜五更,彗出東方,色白,
 長四尺,在婁、胃間,疾行向東北角,歷昴、畢、觜、參、井、鬼、柳、軒轅,至太微右執法七寸所,凡五十餘日方滅。閏四月辛酉朔,妖星見於南方,長數丈。是時自四月初大霧大雨,至閏四月末方止。是月,逆賊史思明再陷東都,米價踴貴,鬥至八百文,人相食,殍尸蔽地。上元元年十二月癸未夜,歲掩房星。二年七月癸未朔,日有蝕之,大星皆見。司天秋官正瞿曇譔奏曰:「癸未太陽虧,辰正後六刻起虧,巳正後一刻既,午前一刻復滿。虧於張四度,周之
 分野。甘德云,『日從巳至午蝕為周』,周為河南,今逆賊史思明據。《乙巳占》曰,『日蝕之下有破國』。」其年九月,制去上元之號,單稱元年,月首去正、二、三之次,以「建」冠之。其年建子月癸巳亥時一鼓二籌後,月掩昴,出其北,兼白暈;畢星有白氣從北來貫昴。司天監韓潁奏曰:「按石申占,『月掩昴,胡王死』。又『月行昴北,天下福』。臣伏以三光垂象,月為刑殺之徵。二石殲夷,史官常占。畢、昴為天綱,白氣兵喪,掩其星則大破胡王,行其北則天下有福。已為周
 分,癸主幽、燕,當羯胡竊據之郊,是殘寇滅亡之地。」明年,史思明為其子朝義所殺。十月,雍王收復東都。上元三年正月時去上元之號,今存之以正年。建辰月,肅宗病。是月丙戌,月上有黃白冠連成暈,東井、五諸侯、南北河、輿鬼皆在中。建巳月,以楚州獻定國寶,乃改元寶應,月復以正、二、三為次。其月,肅宗崩。



 代宗即位。其月壬子夜,西北方有赤光見,炎赫亙天,貫紫微,漸流於東,彌漫北方,照耀數十里,久之乃散。辛未夜,江陵見赤光貫北斗,俄僕固懷恩叛。明
 年十月,吐蕃陷長安,代宗避狄幸陜州。廣德二年五月丁酉朔,日當蝕不蝕,群臣賀。十二月三日夜,星流如雨,自亥及曉。永泰元年九月辛卯,太白經天,是月吐蕃逼京畿。二年六月丁未,日重輪,其夜月重輪,是年大水。大歷元年十二月己亥,彗星出匏瓜,長尺餘,犯宦者星。二年七月癸亥,熒惑色赤黃,順行入氐。乙丑夜,鎮星色黃,近辰星,在東井初度。丙寅申時,有青赤氣長四十餘尺,見日旁,久之乃散。己巳夜,歲星順行去司怪七寸。庚午夜,月
 逼天關。壬申十二月,赤氣長二丈亙日上。甲戌酉時,白氣亙天。八月壬午,月入氐。戊子,月犯牽牛,相去九寸。己丑夜,月犯畢,相去四寸。九月戊申朔,歲星守東井,凡七日。乙卯,吐籓入寇,至邠寧。戊午夜,白霧起尾西北,彌漫亙天。乙丑晝,有流星大如一升器,其色黃明,尾跡長六七十尺,出於午,流於丑。戊辰夜,熒惑去南斗五寸。乙亥,青赤氣亙於日旁。十一月辛酉夜,月去東井一尺。甲子夜,月去軒轅一尺。壬戌,京師地震,有聲如雷,自東北來。
 十二月丁酉夜,熒惑入壁壘。戊戌,有黑氣如霧,亙北方,久之方散。三年正月壬子夜,月掩畢。丁巳巳時,日有黃冠,青赤珥。三月乙巳朔,日有蝕之,自午虧,至後一刻,凡蝕十分之六分半。癸丑夜,太白去天衢八寸。癸酉夜,太白順行,去歲星二尺。七月壬申夜,五星並列東井。占云:「中國之利。」八月己酉,月入畢。辛酉,月入東井。壬戌,火星去太白四寸。庚午夜,太白犯左執法,相去一尺。九月壬申夜,歲星入輿鬼。乙亥夜,大星如斗,自南流北,其光燭
 地。丁丑夜,熒惑入太微垣。己卯夜,太白犯左執法,相去六寸。戊子夜,歲星去輿鬼一尺。己丑夜,月犯東井,去五寸。庚戌,熒惑去太微五寸,太白去進賢四寸。癸巳,月去靈臺一尺。四年正月十五日,日有蝕之。二月丙午夜,熒惑有芒角,去房星二尺所。丙辰夜,地震,有聲如雷者三。三月壬午,熒惑有芒角,入氐。癸未,月去氐一尺。戊子夜,鎮星近輿鬼。五月丙戌,京師地震。七月,熒惑犯次相星。九月丁卯,熒惑犯郎位。是歲自四月霖雨,至秋末方息,
 京師米斗八百文。五年四月乙巳夜,歲星入軒轅。己未夜,彗出五車,蓬孛,光芒長三丈。五月己卯夜,彗出北方,其色白。癸未夜,彗隨天東行,近八穀。甲申,西北方白氣竟天。六月丙申,月去太微左執法一寸。丁酉,月去哭星二寸。庚子,月去氐七寸。癸卯,彗去三公二尺。庚戌,太白入東井。甲寅,白氣出西北方,竟天。己未,彗星滅。七月,京師米價騰踴,鬥千錢。六年七月乙巳夜,月掩畢,入昴畢中。壬子,月去太微二寸。八月庚辰,月入太微。九月壬辰,熒惑
 犯哭星,去二寸。庚子夜,火去泣星四寸,月掩畢。甲辰夜,西南流星大如一升器,有尾跡,光明照地,珠子散落,長五丈餘,出須女,入天市南垣滅。丁未,月入太微。辛亥,熒惑入壁壘。十月丁卯,月掩畢。甲戌,月入軒轅。十一月壬寅,月入太微。丙午夜,月掩氐。十二月己巳,月入太微。七年正月乙未夜,月近軒轅。二月戊午,月掩天關。辛酉,月逼輿鬼。己巳,熒惑逼天衢。三月辛卯,月逼靈臺。四月丁巳,熒惑入東井。辛未,歲星入東角。壬申,月入羽林。丙子,鎮星
 臨太微。五月丙戌,月入太微。六月乙亥,月臨東井。十二月甲子,太白入羽林。丙寅,雨土,是夜,長星出於參。八年五月庚辰,熒惑入羽林。六月戊辰,流星大如一升器,有尾跡,長三丈,流入太微。七月己卯,太白入東井,留七日而出。庚寅酉時,有氣三道竟天。辛卯,熒惑臨月。乙未,月掩畢中。八月戊午夜,熒惑臨月。其月,硃滔自幽州入朝。九月癸未,月入羽林。己丑,月入太微。十月癸卯,太白臨鎮星。丙午夜,太白臨進賢。丁巳夜,月掩畢。壬戌夜,月入
 鬼中。庚午,月近太白,並入氐中。十一月己卯,月入羽林。壬午,鎮星逼進賢。癸未,太白掩房。癸巳,月入太微垣。閏十一月壬寅夜,太白、辰星會於危。癸丑,月掩天關。甲寅,月入東井。乙丑,月掩天關。丙寅,月入氐。十二月癸酉,月入羽林。九年正月癸丑,熒惑逼諸王星。三月丁未,熒惑入東井。四月乙亥,月臨軒轅。丁丑,月入太微。五月己酉,太白逼熒惑。乙未夜,太白入軒轅。辛酉,辰星逼軒轅。六月戊寅,月逼天綱。己卯,月掩南斗。庚辰,月入太微。戊子,
 太白臨左執法。七月甲辰,月掩房。辛亥,月入羽林。壬戌,月入輿鬼。八月辛卯,月掩軒轅。九月庚子,硃泚自幽州入朝,是夜,太白入南斗。甲子,熒惑入氐。十月戊子,木入南斗。十二月戊辰,月入羽林。十年正月,昭義軍亂,逐薛崿;田承嗣據河北叛。戊申,月逼軒轅。甲寅夜,熒惑、歲星合於南斗,並順行。二月,河陽軍亂,逐常休明。三月,陜州軍亂,逐李國青。庚戌,熒惑入壁壘。四月甲子,熒惑順行入羽林。庚午,月臨軒轅。六月癸亥,太白臨東井。乙丑夜,
 熒惑入壁,臨天囷。戊辰,月入太微。乙亥,月臨南斗。七月庚子,辰星、太白順行,同在柳。八月乙酉,熒惑順行,臨天高。戊子,月入太微。九月甲午,月臨房。十月辛酉朔,日有蝕之。十二月丙子夜,東方月上有白氣十餘道,如匹帛,貫五車、東井、輿鬼、觜、參、畢、柳、軒轅,三更後方散。十一年閏八月丁酉,太白晝見。其年七月,李靈耀以汴州叛,十月,方誅之。十二年正月乙丑夜,月掩軒轅。癸酉夜,月掩心前星。丙子,月入南斗魁中。二月乙未,鎮星入氐。辛亥夜,流
 星大如桃,尾長十丈,出匏瓜,入太微。三月壬戌,月入太微。戊辰,月逼心星。是月,幸臣元載誅,王縉黜。四月庚寅,月臨左執法。乙未夜,月掩心前星。五月丙辰,月入太微。六月戊戌,月入羽林。七月庚戌,月入南斗。癸丑,熒惑逼司怪。己巳,宰相楊綰卒。乙亥,熒惑順行,入東井。是歲,春夏旱,八月大雨,河南大水,平地深五尺。吐蕃入寇,至坊州。十月己丑,月臨歲星。壬辰,月掩昴。乙未,月臨五諸侯。庚子,月臨左執法,遂入太微垣。十一月癸丑,太白臨哭
 星。乙卯夜,月入羽林。戊辰,月臨左執法。十二月辛巳,鎮星臨關鍵。壬午,月入羽林。十四年五月十一日,代宗崩。



 德宗即位。明年改元建中。至四年十月,硃泚亂,車駕幸奉天。貞元四年五月丁卯,月犯歲星。乙亥,熒惑、鎮、歲聚於營室三十餘日。八月辛卯朔,日有蝕之。十年三月乙亥,黃霧四塞,日無光。四月,太白晝見。元和七年正月辛未,月掩熒惑。六月乙亥,月去南斗魁第四星西北五寸所。八年七月四日夜,月去太微東垣之南首星南一尺
 所。癸酉夜,月去五諸侯之西第四星南七寸所。十月己丑,熒惑順行,去太微西垣之南首星西北四寸所。九年二月丁酉,月去心大星東北七寸所。四月辛巳,北方有大流星,跡尾長五丈,光芒照地,至右攝提南三尺所。九月己丑,月掩軒轅。十二年正月戊子,彗出畢南,長二尺餘,指西南,凡三日,近參旗沒。十三年正月乙未,歲星退行,近太微西垣之南第一星。八月己未,月近南斗魁。壬戌,太白順行,近太微。十四年正月己丑,月近東井北轅
 星。癸卯夜,月近南斗魁星。五月庚寅,月犯心前大星西南一尺所。十五年正月二十七日,憲宗崩。



 穆宗即位。七月庚申,熒惑退行,入羽林。癸亥夜,大星出勾陳,南流至婁北滅。八月己卯,月掩牽牛。長慶元年正月丙午,月掩鉞星;二更後,月去東井南轅第一星南七寸。丙辰,南方大流星色赤,尾有跡,長三丈,光明燭地,出狼星北二尺所,東北流至七星三尺所滅。己未夜,星孛于翼。丁卯夜,星孛在辰上,去太微西垣南第一星七寸所。二月八日
 夜,太白犯昴東南五寸所。丁亥夜,月犯歲星南六寸所,在尾十三度。三月庚戌,太白犯五車東南七寸所。七月壬寅,月掩房次相星。乙丑夜,東方大流星,色黃,有尾跡,長六七丈,光明燭地,出參西北,向西流,至羽林東北滅。其月幽州軍亂,囚其帥張弘靖,立硃克融。其月二十八日,鎮州軍亂,殺其帥田弘正、王廷湊。元和末,河北三鎮皆以疆土歸朝廷;至是,幽、鎮俱失。俄而史憲誠以魏州叛,三鎮復為盜據,連兵不息。八月辛巳夜,東北有大星
 自雲中出流,白光照地,前後長丈二尺五寸,西北入蜀滅;太白在軒轅左角西北一尺所。是月壬辰夜,太白去太微垣南第一星一尺所。九月戊戌夜,太白順行,入太微,去左執法星西北一尺所。乙巳夜,去左執法二寸所。辛亥,月去天關西北八寸。二年正月戊申,魏帥田布伏劍死,史憲誠據郡叛。二月甲戌夜,熒惑在歲星南七寸所。四月辛酉朔,日有蝕之,在胃十二度,不盡者四之一,燕、趙見之既。七月丙子夜,東方大星西流,至昴滅,其聲
 如雷。十月甲子夜,月掩牽牛中星。乙丑夜,太白去南斗魁第四星西一寸所。十一月丁丑,月掩左角。庚辰,月去房一尺所。十二月丁亥,月掩左角。庚戌夜,月近房星。壬子五更後,月近太白,相去一尺所。四年正月二十二日,穆宗崩。



 敬宗即位。二月癸卯,太白犯東井,近北轅。三月甲子,熒惑犯鎮星。壬申,太白犯東井,近北轅。四月十七日,染院作人張韶於柴草車中載兵器,犯銀臺門,共三十七人,入大內,對食於清思殿;其日禁兵誅之。七月乙
 卯夜,有大星出於天船,流犯斗魁第一星西南滅。八月丁亥,熒惑犯鎮星。癸未,熒惑入東井。己丑,太白犯軒轅右角。十二月戊子夜,月掩東井。甲午夜,西北有流星出閣道,至北極滅。寶歷元年七月乙酉,月犯西咸,去八寸所。甲子夜,月掩畢。閏七月癸巳夜,月去心,距九寸。庚子,流星去北極,至南斗柄滅。八月乙卯,太白犯房,相去九寸。九月癸未,太白犯南斗。丙戌,月犯畢。甲午,月犯太微左執法。十月辛卯,月犯天囷,相去七寸。癸亥,太白臨哭
 星,相去九寸。十一月庚辰,鎮星犯東井,相去七寸。癸未夜,月去東井六寸。戊戌,西南大流星出羽林,入濁。十二月戊申夜,月犯畢。乙酉夜,西北方有霧起,須臾遍天。霧上有赤氣,其色或深或淺,久而方散。二年正月甲戌夜,北方大流星長五丈餘,出紫微,過軫滅。甲申,月犯右執法,相去五寸。二月丙午夜,月犯畢。三月己巳,流星出河鼓,東過天市,入濁滅。四月甲子夜,西方大流星長三丈,穿天市垣,至房星滅。其月十七日,白虹貫日連環,至午
 方滅。五月甲戌,月去太微八寸所。癸巳,西北方大流星長三丈,光明照地,入天市垣中滅。甲午五更。熒惑犯昴。六月庚申,太白犯昴。七月壬申,流星長二丈,出鬥北,入濁滅。其夜,月初入,巳上有流星向南滅。其夜,辰犯畢。八月丙申夜,北方大流星長四丈餘,出王良,流至北斗柄滅。甲辰夜,太白去太微八寸所。丁未夜,熒惑近鎮星西北。丁丑,熒惑去輿鬼七寸。十二月八日夜,敬宗為內官劉克明所弒,立絳王。樞密使王守澄等殺絳王,立文宗。



 大和元年九月戊寅,月掩東井南轅星。四年四月辛酉夜四更五籌後,月掩南斗第二星。十一月辛未朔,熒惑犯右執法西北五寸,五年二月,宰相宋申錫、漳王被誣得罪。八年二月朔,日有蝕之。六月辛巳五更,有六流星,赤色,有尾跡,光明照地,珠子散落,出河鼓北流,近天棓滅,有聲如雷。七月己巳夜,流星出紫微西北,長二丈,至北斗第一星滅。是夜五更,月犯昴。九月辛亥夜五更,太微宮近郎位有彗星,長丈餘,西指,西北行,凡九夜,越郎
 位星西北五尺滅。癸丑,月入南斗。庚申,右軍中尉王守澄,宣召鄭注對於浴殿門。是夜,彗星出東方,長三尺,芒耀甚猛。十二月丙戌夜,月掩昴。九年三月乙卯,京師地震。四月辛丑,大風震雷,拔殿前古樹。六月庚寅夜,月掩歲星。丁酉夜一更至四更,流星縱橫旁午,約二十餘處,多近天漢。其年十一月,李訓謀殺內官,事敗,中尉仇士良殺王涯、鄭注、李訓等十七家,朝臣多有貶逐。開成元年正月甲辰,太白掩西建第一星。其月十五日,日有蝕
 之。二月乙亥夜四更,京師地震。二年二月丙午夜,彗出東方,長七尺餘,在危初度,西指。戊申夜,危之西南,彗長七尺,芒耀愈猛,亦西指。癸丑夜,彗在危八度。庚申夜,在虛三度半。辛酉夜,彗長丈餘,直西行,稍南指,在虛一度半。壬戌夜,彗長二丈,其廣三尺,在女九度。癸亥夜,彗愈長廣,在女四度。三月甲子朔,其夜,彗長五丈,岐分兩尾,其一指氐,其一掩房,在斗十度。丙寅夜,彗長六丈,尾無岐,北指,在亢七度。文宗召司天監硃子容問星變之由,子容曰:「
 彗主兵旱,或破四夷,古之占書也。然天道懸遠,唯陛下修政以抗之。」乃敕尚食,今後每日御食料分為十日。其夜彗長五丈,闊五尺,卻西北行,東指。戊辰夜,彗長八丈有餘,西北行,東指,在張十四度。詔天下放系囚,撤樂減膳,避正殿;先是,群臣拜章上徽號,宜並停。癸未夜,彗長三尺,出軒轅之右,東指,在張七度。六月,河陽軍亂,逐李詠。是歲,夏蝗大旱。八月丁酉,彗出虛、危之間。十月,地南北震。三年十月十九日,彗見,長二丈餘;二十日夜,長二
 丈五尺;二十一日夜,長三丈;二十二日夜,長三丈五尺:並在辰上,西指軫、魁。十一月乙卯朔,是夜彗出東方,東西竟天。五月五日,太白犯輿鬼。六月一日,太白犯熒惑。二十八日,太白犯右執法。十月七日,太白犯南斗。四年正月丁巳,熒惑、太白、辰聚於南斗。癸酉,彗出於西方,在室十四度。閏月二十三日,又見於卷舌北,凡三十三日,至二十六日夜滅。二月二十六日,自夜四更至五更,四方中央流星大小二百餘,並西流,有尾跡,長二丈。三月
 乙酉夜,月掩東井第三星。是歲,夏大旱,禱祈無應,文宗憂形於色。宰臣進曰:「星官言天時當爾,乞不過勞聖慮。」帝改容言曰:「朕為人主,無德庇人,比年災旱,星文謫見。若三日內不雨,朕當退歸南內,卿等自選賢明之君以安天下。」宰相楊嗣復等嗚咽流涕不已。七月辛丑,月犯熒惑,河南大水。八月辛未,流星出羽林,有尾跡,長十丈,有聲如雷。十月辛酉,辰入南斗魁。五年正月,文宗崩。



 武宗即位。會昌元年六月二十九日,從一鼓至五鼓,小流
 星五十餘,交橫流散。七月二日,北方流星光明照地,東北流星有聲如雷。九月癸巳,熒惑犯輿鬼。閏九月丁酉,熒惑貫鬼宿;戊戌,在鬼中。十一月六日,彗見西南,在室初度,凡五十六日而滅。其夜上方大流星光明燭地,東北流星有聲。二年六月乙丑,熒惑犯歲星。丙寅,太白犯東井。其夜,熒惑蒼赤色,動搖於井中,至八月十六日,犯輿鬼。五年二月五日,太白掩昴北側,在昴宿一度。五月辛酉,太白入畢口,距星東南一尺。八月七日,太白犯軒
 轅大星。



 舊儀:太史局隸秘書省,掌視天文歷象。則天朝,術士尚獻輔精於歷算,召拜太史令。獻輔辭曰:「臣山野之人,性靈散率,不能屈事官長。」天后惜其才,久視元年五月十九日,敕太史局不隸秘書省,自為職局,仍改為渾天監。至七月六日,又改為渾儀監。長安二年八月,獻輔卒,復為太史局,隸秘書省,緣進所置官員並廢。景龍二年六月,改為太史監,不隸秘書省。景雲元年七月,復為太史
 局,隸秘書省。八月,又改為太史監。十一月,又改為太史局。二年閏九月,改為渾儀監。開元二年二月,改為太史監。十五年正月,改為太史局,隸秘書省。天寶元年,又改為太史監。



 乾元元年三月,改太史監為司天臺,於永寧坊張守珪故宅置。敕曰:「建邦設都,必稽玄象;分列曹局,皆應物宜。靈臺三星,主觀察雲物;天文正位,在太微西南。今興慶宮,上帝廷也,考符之所,合置靈臺。宜令所司量事修理。」舊臺在秘書省之南。仍置五官正五人。司天臺內別置
 一院,曰通玄院。應有術藝之士,徵闢至京,於崇玄院安置。其官員:大監一員,正三品。少監二人,正四品。丞三人,正六品。主簿三人,主事二人,五官正五人,五官副正五人,靈臺郎一人,五官保章正五人,五官挈壺正五人,五官司歷五人,五官司辰十五人,觀生、歷生七百二十六人。凡官員六十六人。寶應元年,司天少監瞿曇譔奏曰:「司天丞請減兩員,主簿減兩員,主事減一員,保章正減三員,挈壺正減三員,監候減兩員,司辰減七員,五陵司
 辰減五員。」從之。



 天寶十三載三月十四日,敕太史監官除朔望朝外,非別有公事,一切不須入朝,及充保識,仍不在點檢之限。



 開成五年十二月,敕:「司天臺占候災祥,理宜秘密。如聞近日監司官吏及所由等,多與朝官並雜色人交游,既乖慎守,須明制約。自今已後,監司官吏不得更與朝官及諸色人等交通往來,委御史臺察訪。」



\end{pinyinscope}