\article{卷一百六十二}

\begin{pinyinscope}

 舊唐書卷一百六十二



 列傳第一百八



 ○武元
 衡從父弟儒衡鄭餘慶子瀚瀚子允謨茂休處誨從讜韋貫之兄綬弟纁子澳



 武元衡,字伯蒼,河南緱氏人。曾祖載德,天后從父弟,官至湖州刺史。祖平一,善屬文,終考功員外郎、修文館學
 士,事在《逸人傳》。父就,殿中侍御史,以元衡貴,追贈吏部侍郎。元衡進士登第,累闢使府,至監察御史。後為華原縣令。時畿輔有鎮軍督將恃恩矜功者,多撓吏民。元衡苦之,乃稱病去官。放情事外,沉浮宴詠者久之。德宗知其才,召授比部員外郎。一歲,遷左司郎中。時以詳整稱重。



 貞元二十年,遷御史中丞。嘗因延英對罷,德宗目送之,指示左右曰:「元衡真宰相器也。」



 順宗即位,以病不親政事。王叔文等使其黨以權利誘元衡,元衡拒之。時奉
 德宗山陵,元衡為儀仗使。監察御史劉禹錫,叔文之黨也,求充儀仗判官。元衡不與,其黨滋不悅。數日,罷元衡為右庶子。憲宗即位,始冊為皇太子,元衡贊引,因識之。及登極,復拜御史中丞。持平無私,綱條悉舉,人甚稱重。尋遷戶部侍郎。元和二年正月,拜門下侍郎、平章事,賜金紫,兼判戶部事。上為太子時,知其進退守正,及是用為宰相,甚禮信之。



 初,浙西節度李錡請入覲,乃拜為右僕射,令入朝。既而又稱疾,請至歲暮。上問宰臣,鄭絪請
 如錡奏。元衡曰:「不可。且錡自請入朝,詔既許之,即又稱疾,是可否在錡。今陛下新臨大寶,天下屬耳目,若使奸臣得遂其私,則威令從茲去矣。」上以為然,遽追之。錡果計窮而反。



 先是,高崇文平蜀,因授以節度使。崇文理軍有法,而不知州縣之政。上難其代者,乃以元衡代崇文,拜檢校吏部尚書,兼門下侍郎、平章事,充劍南西川節度使。將行,上御安福門以臨慰之。高崇文既發成都,盡載其軍資、金帛、帟幕、伎樂、工巧以行。元衡至,則庶事節
 約,務以便人。比三年,公私稍濟。撫蠻夷,約束明具,不輒生事。重慎端謹,雖淡於接物,而開府極一時之選。八年,徵還。至駱谷,重拜門下侍郎、平章事。



 時李吉甫、李絳情不相葉,各以事理曲直於上前。元衡居中,無所違附,上稱為長者。及吉甫卒,上方討淮、蔡,悉以機務委之。時王承宗遣使奏事,請赦吳元濟。請事於宰相,辭禮悖慢,元衡叱之。承宗因飛章詆元衡,咎怨頗結。元衡宅在靜安里,十年六月三日,將朝,出里東門,有暗中叱使滅燭者,
 導騎訶之,賊射之,中肩。又有匿樹陰突出者,以棓擊元衡左股。其徒馭已為賊所格奔逸,賊乃持元衡馬,東南行十餘步害之,批其顱骨懷去。及眾呼偕至,持火照之,見元衡已踣於血中,即元衡宅東北隅墻之外。時夜漏未盡,陌上多朝騎及行人,鋪卒連呼十餘里,皆云賊殺宰相,聲達朝堂,百官恟々,未知死者誰也。須臾,元衡馬走至,遇人始辨之。既明,仗至紫宸門,有司以元衡遇害聞。上震驚,卻朝而坐延英,召見宰相。惋慟者久之,為之
 再不食。冊贈司徒,贈賻布帛五百匹、粟四百碩,輟朝五日,謚曰忠愍。



 元衡工五言詩,好事者傳之,往往被於管弦。



 初,八年,元衡自蜀再輔政,時太白犯上相,歷執法。占者言:「今之三相皆不利,始輕末重。」月餘,李絳以足疾免。明年十月,李吉甫以暴疾卒。至是,元衡為盜所害,年五十八。始元衡與吉甫齊年,又同日為宰相。及出鎮,分領揚、益。及吉甫再入,元衡亦還。吉甫先一年以元衡生月卒,元衡後一年以吉甫生月卒。吉兇之數,若符會焉。先
 是,長安謠曰「打麥麥打三三三」,既而旋其袖曰「舞了也」。解者謂:「打麥」者,打麥時也;「麥打」者,蓋謂暗中突擊也;「三三三」,謂六月三日也;「舞了也」,謂元衡之卒也。自是京師大恐,城門加衛兵,察其出入,物色伺之。其偉狀異制、燕趙之音者,多執訊之。元衡從父弟儒衡。



 儒衡,字庭碩。才度俊偉,氣直貌莊,言不妄發,與人交友,終始不渝。相國鄭餘慶不事華潔,後進趨其門者多垢衣敗服,以望其知。而儒衡謁見,未嘗輒易所好,但與之正言直論,餘慶
 因亦重之。憲宗以元衡橫死王事,嘗嗟惜之,故待儒衡甚厚。累遷戶部郎中。十二年,權知諫議大夫事,尋兼知制誥。皇甫鎛以宰相領度支,剝下以媚上,無敢言其罪者。儒衡上疏論列,鎛密訴其事,帝曰:「勿以儒衡上疏,卿將報怨耶!」鎛不復敢言。



 儒衡氣岸高雅,論事有風彩,群邪惡之。尤為宰相令狐楚所忌。元和末年,垂將大用,楚畏其明俊,欲以計沮之,以離其寵。有狄兼謨者,梁公仁傑之後,時為襄陽從事。楚乃自草制詞,召狄兼謨為拾
 遺,曰:「朕聽政餘暇,躬覽國書,知奸臣擅權之由,見母後竊位之事。我國家神器大寶,將遂傳於他人。洪惟昊穹,降鑒儲祉,誕生仁傑,保佑中宗,使絕維更張,明闢乃復。宜福胄胤,與國無窮。」及兼謨制出,儒衡泣訴於御前,言其祖平一在天后朝辭榮終老,當時不以為累。憲宗再三撫慰之。自是薄楚之為人。然儒衡守道不回,嫉惡太甚,終不至大任。尋正拜中書舍人。時元稹依倚內官,得知制誥,儒衡深鄙之。會食瓜閣下,蠅集於上,儒衡以扇
 揮之曰:「適從何處來,而遽集於此?」同僚失色,儒衡意氣自若。遷禮部侍郎。長慶四年卒,年五十六。



 鄭餘慶,字居業,滎陽人。祖長裕,官至國子司業,終潁川太守。長裕弟少微,為中書舍人、刑部侍郎。兄弟有名於當時。父慈,與元德秀友善,官至太子舍人。



 餘慶少勤學,善屬文。大歷中舉進士。建中末,山南節度使嚴震闢為從事,累官殿中侍御史,丁父憂罷。貞元初入朝,歷左司、兵部員外郎,庫部郎中。八年,選為翰林學士。



 十三年六
 月,遷工部侍郎,知吏部選事。時有玄法寺僧法湊為寺眾所,萬年縣尉盧伯達斷還俗,後又復為僧,伯達上表論之。詔中丞宇文邈、刑部侍郎張彧、大理卿鄭雲逵等三司,與功德使判官諸葛述同按鞫。時議述胥吏,不合與憲臣等同入省按事。餘慶上疏論列,當時翕然稱重。



 十四年,拜中書侍郎、平章事。餘慶通究《六經》深旨,奏對之際,多以古義傅之。與度支使於䪹素善,每奏事餘慶皆議可之。未幾,䪹以罪貶。時又歲旱人饑,德宗與宰
 臣議,將賑給禁衛六軍。事未行,為中書吏所洩,餘慶貶郴州司馬,凡六載。順宗登極,徵拜尚書左丞。



 憲宗嗣位之月,又擢守本官、平章事。未幾,屬夏州將楊惠琳阻命,宰臣等論奏,多議兵事。餘慶復以古義上言,夏州軍士皆仰給縣官,又有「介馬萬蹄」之語。時議以餘慶雖好古博雅而未適時。有主書滑渙,久司中書簿籍,與內官典樞密劉光琦情通。宰相議事,與光琦異同者,令渙達意,未嘗不遂所欲。宰相杜佑、鄭絪皆姑息之。議者云佑私
 呼為滑八,四方書幣貲貨,充集其門,弟泳官至刺史。及餘慶再入中書,與同僚集議。渙指陳是非,餘慶怒其僭,叱之。尋而餘慶罷相,為太子賓客。其年八月,渙贓污發,賜死。上浸聞餘慶叱渙事,甚重之,乃改為國子祭酒,尋拜河南尹。三年,檢校兵部尚書,兼東都留守。六年四月,正拜兵部尚書。



 餘慶再為相,罷免皆非大過,尤以清儉為時所稱。洎中外踐更,鬱為耆德,朝廷得失,言成準的。時京兆尹元義方、戶部侍郎判度支盧坦,皆以勛官前
 任至三品,據令合立門戟,各請戟立於其第。時義方以加上柱國、坦以前任宣州觀察使請戟。近代立戟者,率有銀青階,而義方只據勛官,有司不詳覆而給之,議者非之,臺司將劾而未果。會餘慶自東都來,發論大以為不可。由是,臺司移牒詰禮部,左司郎中陸則、禮部員外崔備皆罰俸,奪元、盧之門戟。



 餘慶受詔撰《惠昭太子哀冊》,其辭甚工。有醫工崔環,自淮南小將為黃州司馬。敕至南省,餘慶執之封還,以為諸道散將無故授正員五
 品官,是開僥幸之路,且無闕可供。言或過理,由是稍忤時權,改太子少傅,兼判太常卿事。初德宗自山南還宮,關輔有懷光、吐蕃之虞,都下驚憂,遂詔太常集樂去大鼓。至是,餘慶始奏復用大鼓。



 九年,拜檢校右僕射,兼興元尹,充山南西道節度觀察使,三歲受代。



 十二年,除太子少師。尋以年及懸車,請致仕,詔不許。時累有恩赦敘階,及天子親謁郊廟,行事官等皆得以恩授三品五品,不復計考,其使府賓吏,又以軍功借賜命服而後入拜
 者十八九。由是,在朝衣綠者甚少,郎官諫官有被紫垂金者。又丞郎中謝洎郎官出使,多賜章服,以示加恩。於是寵章尤濫,當時不以服章為貴,遂詔餘慶詳格令,立制條,奏以聞。



 十三年,拜尚書左僕射。自兵興以來,處左右端揆之位者多非其人,及餘慶以名臣居之,人情美洽。憲宗以餘慶諳練典章,朝廷禮樂制度有乖故事,專委餘慶參酌施行,遂用為詳定使。餘慶復奏刑部侍郎韓愈、禮部侍郎李程為副使,左司郎中崔郾、吏部郎中
 陳珮、刑部員外郎楊嗣復、禮部員外郎庾敬休,並充詳定判官。朝廷儀制、吉兇五禮,咸有損益焉。改鳳翔尹、鳳翔隴節度使。



 十四年,兼太子少師、檢校司空,封滎陽郡公,兼判國子祭酒事。以太學荒毀日久,生徒不振,奏率文官俸給修兩京國子監。



 及穆宗登極,以師傅之舊,進位檢校司徒,優禮甚至。元和十五年十一月卒,詔曰:「故金紫光祿大夫、檢校司徒、兼太子少師、上柱國、滎陽郡開國公、食邑二千戶鄭餘慶,始以衣冠禮樂,行於山東,
 餘力文章,遂成志學。出入清近,盈五十年。再秉臺衡,屢分戎律。凡所要職,無不踐更。貴而能貧,卑以自牧。謇諤聞於臺閣,柔睦化於閨門。受命有考父之恭,待士比公孫之廣。焚書逸禮,盡可口傳;古史舊章,如因心匠。朕方咨稟,庶罔昏逾。神將祝予,痛悼何及!乞言既阻,賵禮宜優,可贈太保。」時年七十五,謚曰貞。



 餘慶砥名礪行,不失儒者之道;清儉率素,終始不渝。四朝居將相之任,出入垂五十年,祿賜所得,分給親黨,其家頗類寒素。自至德
 已來,方鎮除授,必遣中使領旌節就第宣賜,皆厚以金帛遣之。求媚者唯恐其數不廣,故王人一來,有獲錢數百萬者。餘慶每受方任,天子必誡其使曰:「餘慶家貧,不得妄有求取。」專欲振起儒教,後生謁見者,率以經學諷之。而周其所急,理家理身,極其儉薄。及修官政,則喜開廣。鎮岐下一歲,戎事可觀。又創立儒宮以來,學者雖行己可學,而往往近於沽激,故當時議者不全德許之。上以家素清貧,不辦喪事,宜令所司特給一月俸料,以充
 賻贈,用示哀榮。有文集、表疏、碑志、詩賦共五十卷行於世。



 兄承慶,官不顯。弟膺甫,官至主客員外郎中、楚懷鄭三州刺史。次弟具瞻、羽客、時然,皆官至縣令賓佐。餘慶子瀚。



 翰本名涵,以文宗籓邸時名同,改名瀚。貞元十年舉進士。以父謫官,累年不任。自秘書省校書郎遷洛陽尉,充集賢院修撰。改長安尉、集賢校理。轉太常寺主簿,職仍故。遷太常博士,改右補闕。獻疏切直,人為危之。及餘慶入朝,憲宗謂餘慶曰:「卿之令子,朕之直臣,可更相
 賀。」遂遷起居舍人,改考功員外郎。刺史有驅迫人吏上言政績,請刊石紀政者,瀚探得其情,條責廉使,巧跡遂露,人服其敏識。時餘慶為僕射,請改省郎。乃換國子博士、史館修撰。丁母憂,除喪,拜考功郎中。復丁內艱,終制,退居汜上。長慶中,徵為司封郎中、史館修撰,累遷中書舍人。



 文宗登極,擢為翰林侍講學士。上命撰《經史要錄》二十卷。書成,上喜其精博,因摘所上書語類。上親自發問,瀚應對無滯,錫以金紫。太和二年,遷禮部侍郎。典貢
 舉二年,選拔造秀,時號得人。轉兵部侍郎,改吏部,出為河南尹,皆著能名。入為左丞,旋拜刑部尚書,兼判左丞事。出為山南西道節度觀察使,檢校戶部尚書、興元尹、兼御史大夫。餘慶之鎮興元,創立儒宮,開設學館,至瀚之來,復繼前美。開成四年閏正月,以戶部尚書徵。詔下之日,卒於興元,年六十四,贈右僕射,謚曰宣。有文集、制誥共三十卷,行於世。浣四子:允謨、茂諶、處誨、從讜。



 允謨,以廕累官臺省,歷蜀、彭、濠、晉四州刺史,位終太子右庶
 子。



 茂諶,避國諱改茂休,開成二年登進士第,四遷太常博士、兵部員外郎、吏部郎中、絳州刺史,位終秘書監。



 處誨,字延美,於昆仲間文章拔秀,早為士友所推。太和八年登進士第。釋褐秘府,轉監察、拾遺、尚書郎、給事中。累遷工部、刑部侍郎,出為越州刺史、浙東觀察使、檢校刑部尚書、汴州刺史、宣武軍節度觀察等使,卒於汴。處誨族父朗。初朗為定州節度使時,處誨為工部侍郎,因早朝假寐於待漏院,忽夢己為浙東觀察使,經過汴州,而
 朗為汴帥,留連飲餞,仰視屋棟,飾以黃土,賓從皆所識。明年,朗果自定州鎮宣武,闢韋重掌書記。重將行,處誨告以所夢。明年,處誨轉刑部侍郎。其年秋,授浙東觀察使。行及潼關,朗遣從事迎勞,仍致手書,令先疏所夢。比至汴,宴於清暑亭,賓佐悉符夢中。朗仰視屋棟曰:「此亦黃土也。」四座感嘆移時。後五年,朗卒,處誨繼為汴州節度使,乃賦詩一章,刻於事,以盡思朗之悲。處誨方雅好古,且勤於著述,撰集至多。為校書郎時,撰次《明皇雜
 錄》三篇,行於世。



 從讜,字正求,會昌二年登進士第,釋褐秘書省校書郎,歷拾遺、補闕、尚書郎、知制誥。故相令狐、魏扶,皆父貢舉門生,為之延譽,尋遷中書舍人。咸通三年,知貢舉,拜禮部侍郎,轉刑部,改吏部侍郎。典選平允,時無屈人。垂將作輔,以權臣請托不行,改檢校刑部尚書、太原尹、北都留守、河東節度觀察等使。逾年,乞還,不允,改檢校兵部尚書、汴州刺史、宣武軍節度觀察等使。期年報政,美聲流聞。當途者懼其大用,改廣州刺史、
 嶺南節度使。



 五管為南詔蠻所擾,天下徵兵,時有龐勛之亂,不暇邊事。從讜在鎮,北兵寡弱,夷獠棼然,乃擇其土豪,授之右職,禦侮捍城,皆得其效。雖郡邑屢陷,而交、廣晏然。俄而懿宗厭代,從讜以久在番禺,不樂風土,思歸戀闕,形於賦詠,累上章求為分司散秩。僖宗征還,用為刑部尚書。尋以本官同平章事。



 乾符中,盜起河南,天下騷動。陰山府沙陀都督李國昌部族方強,虎視北邊。屬靈州防禦使段文楚軍儲不繼,郡兵乏食,乃密引沙
 陀部攻城,殺文楚,遂據振武軍雲、朔等州。又令其子克章、克用大合諸部,南侵忻、代。前帥竇瀚、李侃、李蔚相繼以重臣鎮並部,皆不能遏。俄而康傳圭為三軍所殺,軍士益驕,矜功責賞,勸為噪聚。加以河南、河北七道兵帥,雲合都下,人不聊生,沙陀連陷城邑,朝廷難於擇帥。僖宗欲以宰臣臨制之,詔曰:「開府儀同三司、門下侍郎、兼兵部尚書、充太清宮使、弘文館大學士、延資庫使、上柱國、滎陽郡開國公、食邑二千戶鄭從讜:自處鈞衡,屢來
 麟鳳,才高應變,動必研機。朕以北門興王故地,以爾嘗施惠化,尚有去思。方當用武之時,暫輟調元之職,佇殲兇醜,副我憂勤。可檢校司空、司平章事、太原尹、北都留守、河東節度,兼行營招討等使。」制下,許自擇參佐。乃奏長安令王調為副使,兵部員外郎、史館修撰劉崇龜為節度判官,前司勛員外郎、史館修撰趙崇為觀察判官,前進士劉崇魯充推官,前左拾遺李渥充掌書記,前長安尉崔澤充支使。開幕之盛,冠於一時。時中朝瞻望者,
 目太原為「小朝廷」,言名人之多也。時新承軍亂之後,殺掠攻剽,無日無之。



 從讜貌溫而氣勁,沉機善斷,奸無遁情。凡兇謀盜發,無不落其彀中,以是群豪惕息。舊府城都虞候張彥球者,前帥令率兵三千逐沙陀於百井,中路而還,縱兵破鑰,殺故帥康傳圭。及從讜至,搜索其魁誅之。知彥球意善,有方略,召之開喻,坦然無疑,悉以兵柄委之。



 廣明初,李鈞、李涿繼率本道之師出雁門,為沙陀所敗。十二月,黃巢犯長安,僖宗出幸。傳詔謂從讜曰:「
 卿志安封域,權總戎麾,夷夏具瞻,社稷全賴。今月五日,草賊黃巢奔沖;十六日,駐蹕梁、漢。上慚九廟,下愧萬方。籓閫乍聞,痛憤應切。專差供奉官劉全及往彼慰喻。卿宜差點本道兵士,酌量多少,付北面副招討使諸葛爽,俾令入援。」從讜承詔雪涕,團結戎伍,遣牙將論安、後院軍使硃玫率步騎五千,從諸葛爽入關赴難。時中和元年五月也。



 論安軍次離石。是月,沙陀李克用軍奄至,營於汾東,稱奉詔赴難入關。從讜具廩餼犒勞,信宿不發。
 克用傅城而呼曰:「本軍將南下,欲與相公面言。」從讜登城謂之曰:「僕射父子,咸通以來,舊激忠義,血戰為國,天下之人受賜。老夫歷事累朝,位忝將相,今日群盜擾攘,輿駕奔播,蕩覆神州,不能荷戈討賊,以酬聖獎,老夫之罪也。然多難圖勛,是僕射立功立事之時也。所恨受命守籓,不敢辱命,無以仰陪戎棨。若僕射終以君親為念,破賊之後,車駕還宮,卻得待罪闕庭,是所願也。唯僕射自愛。」克用拜謝而去。然雜虜不戢,肆掠近甸。從讜遣大
 將王蟾、薛威出師追擊之。翌日,契苾部救兵至,沙陀大敗而還。



 初,論安率師入關,至陰地,以數百卒擅歸,從讜集諸部校斬之於鞠場,並以兵眾付硃玫赴難。時鄭畋亦以宰相鎮鳳翔,與從讜宗人,同年登進士。畋亦舉兵岐下,以遏賊巢。廣明首唱仗義,斷賊首尾,逆徒名為「二鄭」。國威復振,二儒帥之功也。



 二年十一月,代北監軍使陳景思奉詔赦沙陀部,許討賊自贖。由是沙陀五部數萬人南下,不敢蹈境。乃自嵐、石沿河而南,唯李克用以
 數百騎臨城敘別。從讜遺之名馬、器幣而訣。三年,克用破賊立功,授河東節度代從讜。還至榆次,遣使致禮,謂從讜曰:「予家尊在雁門,且還覲省。相公徐治行裝,勿遽首途。」從讜承詔,即日牒監軍使周從寓請知兵馬留後事。書記劉崇魯知觀察留後事,戒之曰:「俟面李公,按籍而還。」



 五月十五日,從讜離太原。時京城雖復,車駕未還,道途多寇。行次絳州,唐彥謙為刺史,留駐數月。冬,詔使追赴行在,復輔政,歷司空、司徒,正拜侍中。光啟末,固辭
 機務,以疾還第。卒。有司謚曰文忠。



 從讜知人善任,性不驕矜,故所至有聲績。在太原時,大將張彥球強傑難制,前後帥守以疑間貽釁,故軍旅不寧。及從讜撫封四年,知其才用可委,開懷任遇,得其死力。故抗虜全城,多彥球之效也。累奏為行軍司馬。及再秉政,用為金吾將軍,累郡刺史。在絳州時,彥謙判官陸扆嗜學有才思,寓於郡齋,日與之談宴,無間先後。乃稱之於朝,位至清顯。在汴時,以兄處誨嘗為鎮帥,歿於是郡,訖一政受代,不於
 公署舉樂,其友悌知禮,操履如此。國之名臣,文忠有焉。



 韋貫之,本名純,以憲宗廟諱,遂以字稱。八代祖夐,仕周,號逍遙公。父肇,官至吏部侍郎,有重名於時。貫之即其第二子。少舉進士。貞元初,登賢良科,授校書郎。秩滿,從調判入等,再轉長安縣丞。



 德宗末年,京兆尹李實權移宰相,言其可否,必數日而詔行。人有以貫之名薦於實者,答曰:「是其人居與吾同里,亟聞其賢,但吾得識其面而進於上。」舉笏示說者曰:「實已記其名氏矣。」說者喜,驟
 以其語告於貫之,且曰:「子今日詣實而明日受賀矣。」貫之唯唯,數歲終不往,然是後竟不遷。



 永貞中,始除監察御史。上書舉季弟纁自代,時議不以為私。轉右補闕,而纁代為監察。元和元年,杜從鬱為左補闕,貫之與崔群奏論,尋降為左拾遺。又論遺、補雖品不同,皆是諫官。父為宰相,子為諫官,若政有得失,不可使子論父。改為秘書丞。



 後與中書舍人張弘靖考制策,第其名者十八人,其後多以文稱。轉禮部員外郎。新羅人金忠義以機巧
 進,至少府監,廕其子為兩館生。貫之持其籍不與,曰:「工商之子,不當仕。」忠義以藝通權幸,為請者非一,貫之持之愈堅。既而疏陳忠義不宜污朝籍,詞理懇切,竟罷去之。改吏部員外郎。三年,復策賢良之士,又命貫之與戶部侍郎楊於陵、左司郎中鄭敬、都官郎中李益同為考策官。貫之奏居上第者三人,言實指切時病,不顧忌諱,雖同考策者皆難其詞直,貫之獨署其奏。遂出為果州刺史,道中黜巴州刺史。俄徵為都官郎中、知制誥。逾年,
 拜中書舍人,改禮部侍郎。凡二年,所選士大抵抑浮華,先行實,由是趨競者稍息。轉尚書右丞,中謝日,面賜金紫。



 明年,以本官同中書門下平章事。淮西之役,鎮州盜竊發輦下,殺宰相武元衡,傷御史中丞裴度。及度為相,二寇並徵,議者以物力不可。貫之請釋鎮以養威,攻蔡以專力。上方急於太平,未可其奏。貫之進言:「陛下豈不知建中之事乎?天下之兵,始於蔡急魏應,齊趙同惡。德宗率天下兵,命李抱真、馬燧急攻之,物力用屈,於是硃
 泚乘之為亂,硃滔隨而向闕,致使梁、漢為府,奉天有行,皆陛下所聞見。非他,不能忍待次第,速於撲滅故也。陛下獨不能寬歲月,俟拔蔡而圖鎮邪?」上深然之,而業已下伐鎮詔。後滅蔡而鎮自服,如其策焉。



 初,王師征蔡,以汴帥韓弘為都統,又命汝帥烏重胤、許帥李光顏合兵而進。貫之以為諸將四面討賊,各稅進取,今若置統督,復令二帥連營,則持重養威,未可以歲月下也。貫之議不從,四年而始克蔡。尋遷中書侍郎。同列以張仲素、段
 文昌進名為學士,貫之阻之,以行止未正,不宜在內庭。



 貫之為相,嚴身律下,以清流品為先,故門無雜賓。有張宿者,有口辯,得幸於憲宗,擢為左補闕。將使淄青,宰臣裴度欲為請章服。貫之曰:「此人得幸,何要假其恩寵耶?」其事遂寢。宿深銜之,卒為所構,誣以朋黨,罷為吏部侍郎。不涉旬,出為湖南觀察使。弟虢州刺史纁,亦貶遠郡。時兩河留兵,國用不足,命鹽鐵副使程異使諸道督課財賦。異所至方鎮,皆諷令捃拾進獻。貫之謂兩稅外,不
 忍橫賦加人,所獻未滿異意,遂率屬內六州留錢以繼獻。由是罷為太子詹事,分司東都。



 上即位,擢為河南尹,徵拜工部尚書。未行,長慶元年卒於東都,年六十二,詔贈尚書右僕射。



 貫之自布衣至貴位,居室無改易。歷重位二十年,苞苴寶玉,不敢到門。性沉厚寡言,與人交,終歲無款曲,未曾偽詞以悅人。身歿之後,家無羨財。有文集三十卷。



 伯兄綬,德宗朝為翰林學士。貞元之政,多參決於內署。綬所議論,常合中道,然畏慎致傷,晚得心疾,
 故不極其用。



 纁有精識奧學,為士林所器。閨門之內,名教相樂。故韋氏兄弟令稱,推於一時。纁累官至太常少卿。



 貫之子澳、潾。



 澳,字子斐,太和六年擢進士第,又以弘詞登科。性貞退寡欲,登第後十年不仕。伯兄溫,與御史中丞高元裕友善。溫請用澳為御史,謂澳曰:「高二十九持憲綱,欲與汝相面,汝必得御史。」澳不答。溫曰:「高君端士,汝不可輕。」澳曰:「然恐無呈身御史。」竟不詣元裕之門。



 周墀鎮鄭滑,闢為從事。墀輔政,以澳為考功員外郎、史
 館修撰。墀初作相,私謂澳曰:「才小任重,何以相救?」澳曰:「荷公重知,願公無權足矣。」墀愕然,不喻其旨。澳曰:「爵賞刑罰,非公共欲行者,願不以喜怒憎愛行之。但令百司群官各舉其職,則公斂衽於廟堂之上,天下自理,何要權耶?」墀深然之。不周歲,以本官知制誥。尋召充翰林學士,累遷戶部、兵部侍郎、學士承旨。與同僚蕭寘深為宣宗所遇,每二人同直,無不召見,詢訪時事。每有邦國刑政大事,中使傳宣草詞,澳心欲論諫,即曰:「此一事,須降
 御札,方敢施行。」遲留至旦,必論其可否。上旨多從之。出為京兆尹,不避權豪,亦師璟憚。



 會判戶部宰相蕭鄴改判度支,澳於延英對。上曰:「戶部闕判使。」澳對以府事。上言「戶部闕判使」者三,又曰:「卿意何如?」澳對曰:「臣近年心力減耗,不奈繁劇,累曾陳乞一小鎮,聖慈未垂矜允。」上默然不樂其奏。澳甥柳玭知其對,謂澳曰:「舅之獎遇,特承聖知,延英奏對,恐未得中。」澳曰:「吾不為時相所信,忽自宸旨委以使務,必以吾他歧得之,何以自明?我意不
 錯。爾須知時事漸不堪,是吾徒貪爵位所致,爾宜志之!」



 大中十二年,檢校工部尚書,兼孟州刺史,充河陽三城懷孟澤節度等使,辭於內殿。上曰:「卿自求便,我不去卿。」在河陽累年,中使王居方使魏州,令傳詔旨謂澳曰:「久別無恙,知卿奉道,得何藥術,可具居方口奏。」澳因中使上章陳謝,又曰:「方士殊不可聽,金石有毒,切不宜服食。」帝嘉其忠,將召之,而帝厭代。



 懿宗即位,遷檢校戶部尚書,兼青州刺史、平戶節度觀察處置等使。入為戶部侍
 郎,轉吏部,絟綜平允,不受請托。為執政所惡,出為邠州刺史、邠寧節度使。宰相杜審權素不悅於澳,會吏部發澳時簿籍,吏緣為奸,坐罷鎮,以秘書監分司東都。嘗戲吟云:「若將韋鑒同殷鑒,錯認容身作保身。」此句聞於京師,權幸尤怒之。上表求致仕,宰相疑其怨望,拜河南尹。制出,累上章辭疾,以松檟在秦川,求歸樊川別業,許之。逾年,復授戶部侍郎。以疾不拜而卒。贈戶部尚書,謚曰貞。



 潾亦登進士第,無位而卒。潾子庾、庠、序、雍、郊。



 庾登進
 士第,累佐使府,入朝為御史,累遷兵部郎中、諫議大夫。從僖宗幸蜀,改中書舍人,累拜刑部侍郎,判戶部事。車駕還京,充頓遞使,至鳳翔病卒。



 序、雍、郊皆登進士第。序、雍官至尚書郎。郊文學尤高,累歷清顯。自禮部員外郎知制誥,正拜中書舍人。昭宗末,召充翰林學士,累官戶部侍郎、學士承旨,卒。



 史臣曰:二武朗拔精裁,為時羽儀,嫉惡太甚,遭罹不幸,



 倳刃喋血,誠可哀哉!令狐中傷,為惡滋甚,君子之行,其
 若是乎?鄭貞公博雅好古,一代儒宗。文忠致君,無忝乃祖,衣冠之盛,近代罕儔。韋氏三宗,世多才俊。純、纁忠懿,為時元龜,作輔論兵,言皆體國。澳之貞亮,不替祖風。三代謚貞,考行無愧。



 贊曰:後族崢嶸,平一辭榮。高風襲慶,鐘在二衡。猗與貞公,繼以文忠。純、纁文雅,綽有父風。



\end{pinyinscope}