\article{卷一百六十八}

\begin{pinyinscope}

 ○王
 播子式弟炎起起子龜龜子蕘炎子鐸李絳楊於陵



 王播,字明揚。曾祖璡,嘉州司馬。祖升,咸陽令。父恕,揚府參軍。播擢進士第,登賢良方正制科,授集賢校理,再遷
 監察御史,轉殿中,歷侍御史。貞元末,幸臣李實為京兆尹,恃恩頗橫,嘗遇播於途,不避。故事,尹避臺官。播移文詆之;實怒,後奏播為三原令,欲挫之。播受命,趨府謁謝,盡府縣之儀。及臨所部,政理修明,恃勢豪門,未嘗貸法。歲終考課,為畿邑之最。實以其人有政術,甚禮重之,頻薦之於上。德宗奇之,將不次拔用,會母喪。



 順宗即位,除駕部郎中,改長安令。歲中,遷工部郎中,知臺雜,刺舉綱憲,為人所稱。轉考功郎中,出為虢州刺史。李巽領鹽鐵,
 奏為副使、兵部郎中。



 元和五年,代李夷簡為御史中丞。振舉朝章,百職修舉。十月,代許孟容為京兆尹。時禁軍諸鎮布列畿內,軍人出入,屬鞬佩劍,往往盜發,難以擒奸。布播奏請畿內軍鎮將卒,出入不得持戎具,諸王駙馬權豪之家,不得於畿內按試鷹犬畋獵之具。詔從之,自是奸盜弭息。六年三月,轉刑部侍郎,充諸道鹽鐵轉運使。



 播長於吏術,雖案牘鞅掌,剖析如流,黠吏詆欺,無不彰敗。時天下多故,法寺議讞,科條繁雜。播備舉前後
 格條,置之座右。凡有詳決,疾速如神。當時屬僚,嘆服不暇。



 十年四月,改禮部尚書,領使如故。先是,李巽以程異為江淮院官,異又通泉貨,及播領使,奏之為副。當王師討吳元濟,令異乘傳往江淮,賦輿大集,以至賊平,深有力焉。及皇甫鎛用事,恐播大用,乃請以使務命程異領之,播守本官而已。十三年,檢校戶部尚書、成都尹、劍南西川節度使。



 穆宗即位,皇甫鎛貶,播累表求還京師。長慶元年七月,徵還,拜刑部尚書,復領鹽鐵轉運等使。十
 月,兼中書侍郎、平章事,領使如故。長慶中,內外權臣,率多假借。播因銅鹽擢居輔弼,專以承迎為事,而安危啟沃,不措一言。時河北復叛,朝廷用兵。會裴度自太原入覲,朝野物論,言度不宜居外。明年三月,留度復知政事,以播代度為淮南節度使、檢校右僕射,領使如故。仍請攜鹽鐵印赴鎮,上都院印,請別給賜,從之。播至淮南,屬歲旱儉,人相啖食,課最不充,設法掊斂,比屋嗟怨。



 敬宗即位,就加銀青光祿大夫、檢校司空,罷鹽鐵轉運使。時
 中尉王守澄用事,播自落利權,廣求珍異,令腹心吏內結守澄,以為之助。守澄乘閑啟奏,言播有才,上於延英言之。諫議大夫獨孤朗、張仲方,起居郎孔敏行、柳公權、宋申錫,補闕韋仁實、劉敦儒,拾遺李景讓、薛廷老等,請開延英面奏播之奸邪,交結寵幸,復求大用。天子沖幼,不能用其言。自是,物議紛然不息。明年正月,播復領鹽鐵轉運使。播既得舊職,乃於銅鹽之內,巧為賦斂,以事月進。名為羨餘,其實正額,務希獎擢,不恤人言。



 時揚州
 城內官河水淺,遇旱即滯漕船。乃奏自城南閶門西七里港開河向東,屈曲取禪智寺橋通舊官河,開鑿稍深,舟航易濟;所開長一十九里,其工役料度,不破省錢,當使方圓自備,而漕運不阻。後政賴之。



 文宗即位,就加檢校司徒。太和元年五月,自淮南入覲,進大小銀碗三千四百枚、綾絹二十萬匹。六月,拜尚書左僕射、同平章事,領使如故。二年,進封太原公、太清宮使。四年正月,患喉腫暴卒,時年七十二。廢朝三日,贈太尉。



 播出自單門,以
 文辭自立;踐升華顯,鬱有能名。而隨勢沉浮,不存士行;奸邪進取,君子恥之。然天性勤於吏事,使務填委,胥吏盈廷取決,簿書堆案盈幾,他人若不堪勝,而播用此為適。播子式,弟炎、起。



 炎,貞元十五年登進士第,累官至太常博士,早世。子鐸、鐐。



 起,字舉之,貞元十四年擢進士第,釋褐集賢校理,登制策直言極諫科,授藍田尉。宰相李吉甫鎮淮南,以監察充掌書記。入朝為殿中,遷起居郎、司勛員外郎、直史館。元和十四年,以比部郎中知制誥。
 穆宗即位,拜中書舍人。



 長慶元年,遷禮部侍郎。其年,錢徽掌貢士,為朝臣請托,人以為濫。詔起與同職白居易覆試,覆落者多。徽貶官,起遂代徽為禮部侍郎。掌貢二年,得士尤精。先是,貢舉猥濫,勢門子弟,交相酬酢;寒門俊造,十棄六七。及元稹、李紳在翰林,深怒其事,故有覆試之科。及起考貢士,奏當司所選進士,據所考雜文,先送中書,令宰臣閱視可否,然後下當司放榜。從之。議者以為起雖避是非,失貢職也,故出為河南尹。入為吏部
 侍郎。



 文宗即位,加集賢學士、判院事。以兄播為僕射輔政,不欲典選部,改兵部侍郎。太和二年,出為陜虢觀察使、兼御史大夫。四年,入拜尚書左丞。居播之喪,號毀過禮,友悌尤至。遷戶部尚書、判度支。以西北邊備,歲有和市以給軍,勞人饋挽,奏於靈武,邠寧起營田。六年,檢校吏部尚書、河中尹、河中晉絳節度使。時屬蝗旱,粟價暴踴,豪門閉糴,以邀善價。起嚴誡儲蓄之家,出粟於市,隱者致之於法,由是民獲濟焉。七年,入為兵部尚書。八年,
 檢校右僕射、襄州刺史,充山南東道節度。江、漢水田,前政撓法,塘堰缺壞。起下車,命從事李業行屬郡,檢視而補繕,特為水法,民無兇年。九年,就加銀青光祿大夫。時李訓用事,訓即起貢舉門生也,欲援起為相。八月,詔拜兵部侍郎,判戶部事。其冬,訓敗,起以儒素長者,人不以為累,但罷判戶部事。



 文宗好文,尤尚古學。鄭覃長於經義,起長於博洽,俱引翰林,講論經史。起僻於嗜學,雖官位崇重,耽玩無篸;夙夜孜孜,殆忘寢食,書無不覽,經目
 靡遺。轉兵部尚書。以莊恪太子登儲,欲令儒者授經,乃兼太子侍讀,判太常卿,充禮儀詳定使,創造禮神九玉,奏議曰:



 邦國之禮,祀為大事;珪璧之議,經有前規。謹按《周禮》:「天地四方,以蒼璧禮天,黃琮禮地,青珪禮東方,赤璋禮南方,白琥禮西方,黑璜禮北方。」又云:「四圭有邸以祀天」,「兩圭有邸以祀地」,「圭璧以祀日月星辰」。凡此九器,皆祀神之玉也。又云:「以禋祀祀昊天上帝。」鄭玄云:「禋,煙也,為玉幣,祭訖燔之,而升煙以報陽也。」今與《開元禮》義
 同,此則焚玉之驗也。又《周禮》:「掌國之玉鎮大寶器,若大祭,既事而藏之。」此則收玉之證也。梁代崔靈恩撰《三禮義宗》云:「凡祭天神,各有二玉:一以禮神,一則燔之。禮神者,訖事卻收;祀神者,與牲俱燎。」則靈恩之義,合於《禮經》。今國家郊天祀地,祀神之玉常用;守經據古,禮神之玉則無。臣等請下有司,精求良玉,創造蒼璧、黃琮等九器,祭訖則藏之。其燎玉即依常制。



 從之。為太子廣《五運圖》及《文場秀句》等獻之。三年,以本官充翰林侍講學士。莊
 恪太子薨,詔起為哀冊文,辭情婉麗。



 四年,遷太子少師,判兵部事,侍講如故。以其家貧,特詔每月割仙韶院月料錢三百千添給。起富於文學,而理家無法,俸料入門,即為僕妾所有。帝以師友之恩,特加周給。議者以與伶官分給,可為恥之。



 武宗即位,八月,充山陵鹵簿使。樞密使劉弘逸、薛季稜懼誅,欲因山陵兵士謀廢立。起與山陵使知其謀,密奏,皆伏誅。尋檢校左僕射、東都留守,判東都尚書省事。



 會昌元年,徵拜吏部尚書,判太常卿事。
 三年,權知禮部貢舉。明年,正拜左僕射,復知貢舉。



 起前後四典貢部,所選皆當代辭藝之士,有名於時,人皆賞其精鑒徇公也。其年秋,出為興元尹,兼同平章事,充山南西道節度使。赴鎮日,延英辭。帝謂之曰:「卿國之耆老,宰相無內外,朕有闕政,飛表以聞。」宴賜頗厚。在鎮二年,以老疾求代,不許。大中元年,卒於鎮,時年八十八。廢朝三日,贈太尉,謚曰文懿。文集一百二十卷,《五緯圖》十卷,《寫宣》十卷。起侍講時,或僻字疑事,令中使口宣,即以榜
 子對,故名曰《寫宣》。子龜嗣。



 龜,字大年。性簡淡蕭灑,不樂仕進。少以詩酒琴書自適,不從科試。京城光福里第,起兄弟同居,斯為宏敞。龜意在人外,倦接朋游,乃於永達裏園林深僻處創書齋,吟嘯其間,目為「半隱亭」。及從父起在河中,於中條山谷中起草堂,與山人道士游,朔望一還府第,後人目為「郎君穀」。及起保厘東周,龜於龍門西谷構松齋,棲息往來,放懷事外。起鎮興元,又於漢陽之龍山立隱舍,每浮舟而往,其閑逸如此。武宗知之,以
 左拾遺徵。久之,方至殿廷一謝,陳情曰:「臣才疏散,無用於時,加以疾病所嬰,不任祿仕。臣父年將九十,作鎮遠籓,喜懼之年,闕於供侍。乞罷今職,以奉晨昏。」上優詔許之。明年,丁父憂。服闋,以右補闕徵,遷侍御史、尚書郎。



 大中末,出為宣歙團練觀察副使,賜緋。入為祠部郎中、史館修撰。前從崔璵貳宣歙,及璵鎮河中,又奏為副使。入為兵部郎中,賜金紫,尋知制誥。



 咸通末,以弟鐸在中書,不欲在禁掖,改太常少卿,尋檢校右散騎常侍、同州刺
 史。牙將白約者,甚狡蠹,前後防禦使不能制。龜因事發,笞死以徇,人皆畏威自效。十四年,轉越州刺史、御史大夫、浙東團練觀察使。先是,龜兄式撫臨此郡,有惠政;聞龜復至,舞抃迎之。屬徐、泗之亂,江淮盜起,山越亂,攻郡,為賊所害。贈工部尚書。子蕘。



 蕘苦學,善屬文。以季父作相,避嫌不就科試。乾符初,崔瑾廉察湖南,崔涓鎮江陵,皆闢為從事。蕭遘作相,奏授藍田尉,直史館,遷左拾遺、右補闕,中丞盧涯奏為侍御史。從僖宗幸山南,拜右司
 員外郎,卒。子權,中興仕至兵部尚書。



 式以門廕,累遷監察御史,轉殿中,亦巧宦。太和中,依倚鄭注,謁王守澄,為中丞歸融所劾,出為江陵少尹。大中後,踐更省署。咸通初,為浙東觀察使。草賊仇甫據明州叛,來攻會稽,式討平之。式有威略。三年,徐州銀刀軍叛,以式為徐州節度使。式至鎮,盡誅銀刀等七軍,徐方平定。天子嘉之。後累歷方任,卒。



 鐸,字歸範。會昌初進士第,兩闢使府。大中初,入為監察御史。咸通初,由駕部郎中知制誥,拜中書舍
 人。五年,轉禮部侍郎,典貢士兩歲,時稱得人。七年,以戶部侍郎、判度支,遷禮部尚書。十二年,以本官同平章事。時宰相韋保衡以拔擢之恩,事鐸尤謹,累兼刑部、吏部尚書。僖宗即位,加右僕射。保衡得罪,以鐸檢校右僕射,出為汴州刺史、宣武軍節度使。



 鐸有經世大志,以安邦為己任,士友推之。乾符二年,河南、江左相繼寇盜結集,內官田令孜素聞鐸名,乃復召鐸,拜右僕射、門下侍郎、同平章事。四年,賊陷江陵,楊知溫失守,宋威破賊失策。
 朝議統率,宰相盧攜稱高駢累立戰功,宜付軍柄,物議未允。鐸廷奏曰:「臣忝宰執之長,在朝不足分陛下之憂。臣願自率諸軍,蕩滌群盜。」朝議然之。五年,以鐸守司徒、門下侍郎、同平章事,兼江陵尹、荊南節度使,充諸道行營兵馬都統。鐸至鎮,綏懷流散,完葺軍戎,期年之間,武備嚴整。



 時兗州節度使李系者,西平王晟之孫,以其家世將才,奏用為都統都押衙,兼湘南團練使。時黃巢在嶺南,鐸悉以精甲付系,令分兵扼嶺路。系無將略,微有
 口才,軍政不理。廣明初,賊自嶺南寇湖南諸郡,系守城自固,不敢出戰。賊編木為伐,沿湘而下,急攻潭州,陷之。系甲兵五萬,皆為賊所殺,投尸於江。鐸聞系敗,令部將董漢宏守江陵,自率兵萬餘會襄陽之師。江陵竟陷於賊。天子不之責。罷相,守太子太師。宰相盧攜用事,竟以淮南高駢代鐸為都統。



 其年秋,賊焚剽淮南,高駢挫敗。及賊陷兩京,盧攜得罪,天子用鄭畋為兵馬都統。明年,畋病歸行在,朝議復以鐸為侍中、滑州刺史、義成軍節
 度使,充諸道行營都統。率禁軍、山南、東蜀之師三萬,營於盩厔東,進屯靈感寺。



 明年春,兗、鄆、徐、許、鄭、滑、邠、寧、鳳翔十鎮之師大集關內。時賊已僭名號;以前漸東觀察使崔璆、尚讓為宰相,傳偽命。天下籓帥,多持兩端。既聞鐸傳檄四方,諸侯翻然景附。賊之號令,東西不過岐、華,南北止及山、河。而勁卒驍將,日馳突於國門,群賊由是離心。其年秋,賊將硃溫降,收同州。十一月,賊華州戍卒七千來奔。三年二月,沙陀軍至,收華州。四月,敗賊於良
 田坡,遂收京城。封鐸晉國公。鐸加中書令,以收城諸將,量其功伐高下,承制爵賞以聞。是時國命危若綴旒,天子播越蠻陬,大事去矣。若非鄭畋之奮發,鐸之忠義,則土運之隆替,未可知也。



 自巢、讓之亂,關東方鎮牙將,皆逐主帥,自號籓臣。時溥據徐州,硃瑄據鄆州,硃瑾據兗州,王敬武據青州,周岌據許州,王重榮據河中,諸葛爽據河陽,皆自擅一籓,職貢不入,賞罰由己。既逐賊出關,尤恃功伐,朝廷姑息不暇。巢賊出關東,與蔡帥秦宗權
 合縱。時溥舉兵徐方,請身先討賊,乃授溥都統之命。十軍軍容使田令孜,以內官楊復光有監護用師之功,尤忌儒臣立事,故有時溥之授。



 初,鐸出軍,兼鄭滑節度使,以便供饋。至是,罷鐸都統之權,令仗節歸籓。鐸以硃全忠於己有恩,倚為籓蔽。初,全忠辭禮恭順,既而全忠軍旅稍集,其意漸倨。鐸知不可依,表求還朝。



 其年冬,僖宗自蜀將還,乃以鐸為滄景節度使。時楊全玫在滄州,聞鐸之來,訴於魏州樂彥貞。鐸受命赴鎮,至魏州旬日,彥
 貞迎謁,宴勞甚至。鐸以上臺元老,功蓋群後,行則肩輿,妓女夾侍,賓僚服御,盡美一時。彥貞子從訓,兇戾無行,竊所慕之;令甘陵州卒數百人,伏於漳南之高雞泊。及鐸行李至,皆為所掠,鐸與賓客十餘人,皆遇害。時光啟四年十二月也。



 鐸弟鐐,累官至汝州刺史。王仙芝陷郡城,被害。



 李絳,字深之,趙郡贊皇人也。曾祖貞簡。祖剛,官終宰邑。父元善,襄州錄事參軍。絳舉進士,登宏辭科,授秘書省
 校書郎。秩滿,補渭南尉。貞元末,拜監察御史。元和二年,以本官充翰林學士。未幾,改尚書主客員外郎。逾年,轉司勛員外郎。五年,遷本司郎中、知制誥。皆不離內職,孜孜以匡諫為己任。



 憲宗即位,叛臣李錡阻兵於浙右。錡既誅,朝廷將輦其所沒家財。絳上言曰:「李錡兇狡叛戾,僭侈誅求,刻剝六州之人,積成一道之苦。聖恩本以叛亂致討,蘇息一方。今輦運錢帛,播聞四海,非所謂式遏亂略,惠綏困窮。伏望天慈,並賜本道,代貧下戶今年租
 稅,則萬姓欣戴,四海歌詠矣。」憲宗嘉之。



 時中官吐突承璀自籓邸承恩寵,為神策護軍中尉,乃於安國佛寺建立《聖政碑》,大興功作,仍請翰林為其文。絳上言曰:



 陛下布惟新之政,刬積習之弊,四海延頸,日望德音。今忽立《聖政碑》,示天下以不廣。《易》稱:大人者與天地合德,與日月合明。執契垂拱,勵精求理,豈可以文字而盡聖德,碑表而贊皇猷?若可敘述,是有分限,虧損盛德,豈謂敷揚至道哉?故自堯、舜、禹、湯、文、武,並無建碑之事。至秦始皇
 荒逸之君,煩酷之政,然後有罘、嶧之碑,揚誅伐之功,紀巡幸之跡,適足為百王所笑,萬代所譏。至今稱為失道亡國之主,豈可擬議於此!陛下嗣高祖、太宗之業,舉貞觀、開元之政,思理不遑食,從諫如順流;固可與堯、舜、禹、湯、文、武方駕而行,又安得追秦皇暴虐不經之事,而自損聖政?近者,閻巨源請立紀聖功碑,陛下詳盡事宜,皆不允許。今忽令立此,與前事頗乖。況此碑既在安國寺,不得不敘載游觀宗飾之事。述游觀且乖理要,敘崇飾
 又匪政經,固非哲王所宜行也。其碑,伏乞聖恩特令寢罷。



 憲宗深然之,其碑遂止。



 絳後因浴堂北廊奏對,極論中官縱恣、方鎮進獻之事。憲宗怒,厲聲曰:「卿所論奏,何太過耶?」絳前論不已,曰:「臣所諫論,於臣無利,是國家之利。陛下不以臣愚,使處腹心之地,豈可見事虧聖德,致損清時,而惜身不言?仰屋竊嘆,是臣負陛下也。若不顧患禍,盡誠奏論,旁忤幸臣,上犯聖旨,以此獲罪,是陛下負臣也。且臣與中官,素不相識,又無嫌隙,只是威福太
 盛,上損聖朝,臣所以不敢不論耳。使臣緘默,非社稷之福也。」憲宗見其誠切,改容慰喻之曰:「卿盡節於朕,人所難言者,卿悉言之,使朕聞所不聞,真忠正誠節之臣也。他日南面,亦須如此。」絳拜恩而退。遽宣宰臣,令與改官,乃授中書舍人,依前翰林學士。翌日,面賜金紫,帝親為絳擇良笏賜之。



 前後朝臣裴武、柳公綽、白居易等,或為奸人所排陷,特加貶黜;絳每以密疏申論,皆獲寬宥。及鎮州節度使王士真死,朝廷將用兵討除,絳深陳以為
 未可。絳既盡心匡益,帝每有詢訪,多協事機。六年,猶以中人之故,罷學士,守戶產侍郎,判本司事。嘗因次對,憲宗曰:「戶部比有進獻,至卿獨無,何也?」絳曰:「將戶部錢獻入內藏,是用物以結私恩。」上聳然,益嘉其直。吐突承璀恩寵莫二,是歲,將用絳為宰相;前一日,出承璀為淮南監軍。翌日,降制,以絳為中書侍郎、同中書門下平章事。同列李吉甫便僻,善逢迎上意;絳梗直,多所規諫,故與吉甫不協。時議者以吉甫通於承璀,故絳尤惡之。絳性
 剛訐,每與吉甫爭論,人多直絳。憲宗察絳忠正自立,故絳論奏,多所允從。



 上嘗謂絳曰:「卜筮之事,習者罕精,或中或否。近日風俗。,尤更崇尚,何也?」對曰:「臣聞古先哲王畏天命,示不敢專,邦有大事可疑者,故先謀於卿士庶人,次決於卜筮,俱協則行之。末俗浮偽,幸以徼福。正行慮危,邪謀覬安,遲疑昏惑,謂小數能決之。而愚夫愚婦,假時日鬼神者,欲利欺詐,參之見聞,用以刺射小近之事,神而異之。近者,風俗近巫,此誠弊俗。聖旨所及,實辨
 邪源。但存而不論,弊斯息矣。」



 他日延英,上曰:「朕讀《玄宗實錄》,見開元致理,天寶兆亂。事出一朝,治亂相反,何也?」絳對曰:



 臣聞理生於危心,亂生於肆志。玄宗自天后朝出居籓邸,嘗蒞官守,接時賢於外,知人事之艱難。臨御之初,任姚崇、宋璟,二人皆忠鯁上才,動以致主為心。明皇乘思理之初,亦勵精聽納,故當時名賢在位,左右前後,皆尚忠正。是以君臣交泰,內外寧謐。開元二十年以後,李林甫、楊國忠相繼用事,專引柔佞之人,分居要劇,茍
 媚於上,不聞直言。嗜欲轉熾,國用不足,奸臣說以興利,武夫說以開邊。天下騷動,奸盜乘隙,遂至兩都覆敗,四海沸騰,乘輿播遷,幾至難復。蓋小人啟導,縱逸生驕之致也。至今兵宿兩河,西疆削盡,甿戶凋耗,府藏空虛,皆因天寶喪亂,以至於此。安危理亂,實系時主所行。陛下思廣天聰,親覽國史,垂意精賾,鑒於化源,實天下幸甚。



 上又曰:「凡人行事,常患不通於理,已然之失,追悔誠難。古人處此,復有道否?」絳對曰:「行事過差,聖哲皆所不
 免,故天子致諍臣以匡其失。故主心理於中,臣論正於外,制理於未亂,銷患於未萌。主或過舉,則諫以正之,故上下同體,猶手足之於心膂,交相為用,以致康寧。此亦常理,非難遵之事。但矜得護失,常情所蔽。古人貴改過不吝,從善如流,良為此也。臣等備位,無所發明,但陛下不廢芻言,則端士賢臣,必當自效。」帝曰:「朕擢用卿等,所冀直言。各宜盡心無隱,以匡不逮。無以護失為慮也!」



 其秋,魏博節度使田季安死,其子懷諫幼弱,軍中立其大
 將田興,使主軍事,興卒以六州之地歸命。其經始營創,皆絳之謀也。



 時教坊忽稱密旨,取良家士女及衣冠別第妓人,京師囂然。絳謂同列曰:「此事大虧損聖德,須有論諫。」或曰:「此嗜欲間事,自有諫官論列。」絳曰:「相公居常病諫官論事,此難事即推與諫官,可乎?」乃極言論奏。翌日延英,憲宗舉手謂絳曰:「昨見卿狀所論採擇事,非卿盡忠於朕,何以及此?朕都不知向外事,此是教坊罪過,不諭朕意,以至於此。朕緣丹王已下四人,院中都無侍
 者,朕令於樂工中及閭里有情願者,厚其錢帛,只取四人,四王各與一人。伊不會朕意,便如此生事。朕已令科罰,其所取人,並已放歸。若非卿言,朕寧知此過?」



 八年,封高邑縣男。絳以足疾,拜章求免。九年,罷知政事,授禮部尚書。十年,檢校戶部尚書,出為華州刺史。未幾,入為兵部尚書。丁母憂。十四年,檢校吏部尚書,出為河中觀察使。河中舊為節制,皇甫鎛惡絳,只以觀察命之。十五年,鎛得罪,絳復為兵部尚書。



 穆宗即位,改御史大夫。穆宗
 亟於畋游行幸,絳於延英切諫,帝不能用。絳以疾辭,復為兵部尚書。長慶元年,轉吏部尚書。是歲,加檢校尚書右僕射,判東都尚書省事,充東都留守。二年正月,檢校本官、兗州刺史、兗海節度觀察待使。三年,復為東都留守。四年,就加檢校司空。



 寶歷初,入為尚書左僕射。二年九月,昭議節度使劉悟卒,遺表請以子從諫嗣襲,將吏詣闕論請。絳密奏請速除近澤潞四面將帥一人,以充節度;令倍程赴鎮,使從諫未及拒命,新使已到,所謂「疾
 雷不及掩耳」。潞州軍心,自有所系。從諫無位,何名主張。時宰相李逢吉、王守澄已受從諫賂,俱請以從諫留後,不能用絳言。



 絳以直道進退,聞望傾於一時。然剛腸嫉惡,賢不肖太分,以此為非正之徒所忌。又嘗與御史中丞王播相遇於道,播不為之避;絳奏論事體,敕命兩省詳議,咸以絳論奏是。李逢吉佑播惡絳,乃罷絳僕射,改授太子少師,分司東都。



 文宗即位,徵為太常卿。二年,檢校司空,出為興元尹、山南西道節度使。三年冬,南蠻寇
 西蜀,詔征赴援。絳於本道募兵千人赴蜀;及中路,蠻軍已退,所募皆還。興元兵額素定,募卒悉令罷歸。四年二月十日,絳晨興視事,召募卒,以詔旨喻而遣之,仍給以廩麥,皆怏怏而退。監軍使楊叔元貪財怙寵,怨絳不奉己,乃因募卒賞薄,眾辭之際,以言激之,欲其為亂,以逞私憾。募卒因監軍之言,怒氣益甚,乃噪聚趨府,劫庫兵以入使衙。絳方與賓僚會宴,不及設備。聞亂北走登陴,衙將王景延力戰以御之。兵折矢窮,景延死。絳乃為亂
 兵所害,時年六十七。



 絳初登陴,左右請絳縋城,可以避免,絳不從。乃並從事趙存約、薛齊俱死焉。



 文宗聞奏震悼,下制曰:「朝有正人,時稱令德,入參廟算,出總師干。方當寵任之臣,橫罹不幸之酷。殄瘁興嘆,搢紳所同。故山南西道節度、管內觀察處置等使、銀青光祿大夫、檢校司空,兼興元尹、御史大夫、上柱國、趙郡開國公、食邑二千戶李絳,神授聰明,天賦清直。抱仁義以希前哲,立標準以程後來。抑揚時情,坐致臺輔。佐我烈祖,格於皇天。
 仗鉞宣風,聯居樂土。乘軒鳴玉,嘗極清班。先聲而物議皆歸,不約而群情自許。漢中名部,俾遂便安。而變起不圖,禍生無兆。殲良之慟,聞訃增傷。是極哀榮,用優典禮。三公正秩,品數甚崇,式表異恩,以攄沉痛。可贈司徒。仍令所司擇日備禮冊命。」賻布帛三千段、米粟二百碩。子璋、頊。



 璋,登進士第。盧鈞鎮太原,闢為從事。大中末,入朝為監察,轉侍御史。出刺兩郡,終宣歙觀察使。子德林。



 楊於陵,字達夫,弘農人。漢太尉震之第五子奉之後。曾
 祖珪,為辰州掾曹。祖冠俗,奉先尉。父太清,宋州單父尉。於陵,天寶末家寄河朔。祿山亂,其父歿於賊,於陵始六歲。及長,客於江南。好學,有奇志。弱冠舉進士,釋褐為潤州句容主簿。時韓滉節制金陵,滉性剛嚴,少所接與。及於陵以屬吏謁謝,滉甚奇之,謂其妻柳氏曰:「夫人常擇佳婿,吾閱人多矣,無如楊主簿者。」後竟以女妻之。秩滿,為鄂岳、江南二府從事,累官至侍御史。



 韓滉自江南入朝,總將相財賦之任,頗承顧遇,權傾中外。於陵自江西
 府罷,以婦翁權幸方熾,不欲進取。乃卜築於建昌,以讀書山水為樂。滉歿,貞元八年始入朝,為膳部員外郎,歷考功、吏部三員外,判南曹。時宰相有密親調集,文書不如式,於陵駁之,大協物論。遷右司郎中,復轉吏部郎中,改京兆少尹。出為絳州刺史。德宗雅聞其名,將辭赴郡,詔留之,拜中書舍人。時李實為京兆尹,恃承恩寵,於陵與給事中許孟容俱不附協,為實媒孽,孟容改太常少卿,於陵為秘書少監。貞元末,實輩敗,遷於陵為華州刺
 史,充潼關防禦、鎮國軍等使。未幾,遷浙江東道都團練觀察等使。政聲流聞,入拜戶部侍郎,復改京兆尹。先是,禁軍影占編戶,無以區別。自於陵請致挾名,每五丁者,得兩丁入軍,四丁、三丁者,各以條限。由是京師豪強,復知所畏。再遷戶部侍郎。



 元和初,以考策,升直言極諫牛僧孺等,為執政所怒,出為嶺南節度使。會監軍使許遂振悍戾貪恣,干撓軍政。於陵奉公潔己,遂振無能奈何,乃以飛語上聞。憲宗驚惑,賴宰相裴垍為於陵申理,憲
 宗感悟。



 五年,入為吏部侍郎。遂振終自得罪。



 於陵為吏部,凡四周歲,監察奸吏,調補平允,當時稱之。初,吏部試判,別差考判官三人校能否,元和初罷之。



 七年,吏部尚書鄭餘慶以疾請告,乃復置考判官,以兵部員外郎韋顗、屯田員外張仲素、太學博士陸亙等為之。於陵自東都來,言曰:「本司考判,自當公心。非次置官,不知曹內公事。考官只論判之能否,不計闕員;本司只計員闕幾何,定其留放。置官不便。」宰執以已置顗等,只令考科目選
 人,其餘常調,委本司自考。於陵又以甲歷年深朽斷,吏緣為奸,奏換大歷七年至貞元二十年甲庫歷,令本司郎官監換。



 九年,妖人楊叔高自廣州來干於陵,請為己輔,於陵執奏殺之。改兵部侍郎、判度支。時淮西用兵,於陵用所親為唐鄧供軍使,節度使高霞寓以供軍有闕,移牒度支,於陵不為之易,其闕如舊。霞寓軍屢有摧敗,詔書督責之;乃奏以度支饋運不繼。憲宗怒,



 十一年,貶於陵為桂陽郡守,量移原王傅。復遷戶部侍郎,知吏部
 選事。會誅李師道,分其地為三鎮,朝廷思有所制置,以於陵兼御史大夫,充淄、青十二州宣慰使,還奏合旨。



 穆宗即位,遷戶部尚書。長慶初,拜太常卿,充東都留守,年高,拜章辭位。寶歷二年,授檢校右僕射、兼太子太傅。旋以左僕射致仕,詔給全俸,懇讓不受。



 於陵器度弘雅,進止有常。居朝三十餘年,踐更中外,始終不失其正。居官奉職,亦善操守,時人皆仰其風德。太和四年十月卒,年七十八,冊贈司空,謚貞孝。



 子四人:景復、嗣復、紹復、師復。



 嗣復自有傳。景復位終同州刺史。紹復進士擢第,弘辭登科,位終中書舍人。師復位終大理卿。



 大中後,楊氏諸子登進士第者十人:嗣復子授、技、拭、捴;紹復子擢、拯、據、揆;師復子拙、振等。擢終給事中。拯司封員外郎。據右補闕。揆左諫議大夫。拙左庶子。振左拾遺。



 史臣曰:王氏二英,播、起位崇將相,善始令終。而炎薄祐短齡,美鐘於鐸,而能驤首矯翼,凌厲亨衢,仗鉞秉衡,扶持衰運。天胡罰善,遇盜而殂,悲哉!李趙公頡頏禁林,訏
 謨相府,嘉言啟沃,不以身為。糜軀將壇,沒有餘裕。楊僕射避婦翁之當軸,疏驕尹之怙權,守道居貞,壽考終吉,行己始卒,人以為難。美哉!



 贊曰:王氏儒宗,一門三相。趙公排擯,言猶鯁亮。干將雖折,不改其剛。楊君之德,《韶》、《夏》洋洋。



\end{pinyinscope}