\article{卷九十五}

\begin{pinyinscope}

 ○桓彥範敬暉崔玄暐張柬之袁恕己



 桓彥範,潤州曲阿人也。祖法嗣,雍王府諮議參軍、弘文館學士。彥範慷慨俊爽,少以門廕調補右翊衛。聖歷初,
 累除司衛寺主簿。納言狄仁傑特相禮異,嘗謂曰:「足下才識如是,必能自致遠大。」尋擢授監察御史。



 長安三年,歷遷御史中丞。四年,轉司刑少卿。時司僕卿張昌宗坐遣術人李弘泰占己有天分,御史中丞宋璟請收付制獄,窮理其罪,則天不許。彥範上疏曰:



 昌宗無德無才,謬承恩寵,自宜粉骨碎肌,以答殊造,豈得苞藏禍心,有此占相?陛下以簪履恩久,不忍加刑;昌宗以逆亂罪多,自招其咎。此是皇天降怒,非唯陛下故誅。違天不祥,乞陛
 下裁擇。原其本奏,以防事敗。事敗即言奏訖,不敗則候時為逆。此乃奸臣詭計,疑惑聖心,今果遂其所謀,陛下何忍不察?若昌宗無此占相,奏後不合更與弘泰往還,尚令修福,復擬禳厄,此則期於必遂,元無悔心。縱雖奏聞,情實難恕,此而可舍,誰其可刑?況經兩度事彰,天恩並垂舍宥,昌宗自為得計,人亦以為應運,即不勞兵甲,天下皆從,萬方譏之,以為陛下縱成其亂也。君在,臣圖天分,是為逆臣,不誅,社稷亡矣。伏請付鸞臺鳳閣三司
 考竟其罪。



 疏奏不報。時又內史李嶠等奏稱:「往屬革命之時,人多逆節,鞫訊決斷,刑獄至嚴,刻薄之吏,恣行酷法。其周興、丘勣、來俊臣所劾破家者,並請雪免。」彥範又奏請自文明元年以後得罪人,除揚、豫、博三州及諸謀逆魁首,一切赦之。表疏前後十奏,辭旨激切,至是方見允納。彥範凡所奏議,若逢人主詰責,則辭色無懼,爭之愈厲。又嘗謂所親曰:「今既躬為大理,人命所懸,必不能順旨詭辭,以求茍免。」



 是歲冬,則天不豫。張易之與弟昌
 宗入閣侍疾,潛圖逆亂。鳳閣侍郎張柬之與桓彥範及中臺右丞敬暉等建策將誅之。柬之遽引彥範及暉並為左右羽林將軍,委以禁兵,共圖其事。時皇太子每於北門起居,彥範與暉因得謁見,密陳其計,太子從之。神龍元年正月,彥範與敬暉及左羽林將軍李湛、李多祚、右羽林將軍楊元琰、左威衛將軍薛思行等,率左右羽林兵及千騎五百餘人討易之、昌宗於宮中,令李湛、李多祚就東宮迎皇太子。兵至玄武門,彥範等奉太子
 斬關而入,兵士大噪。時則天在迎仙宮之集仙殿。斬易之、昌宗於廓下,並就第斬其兄汴州刺史昌期、司禮少卿同休,並梟首於天津橋南。士庶見者,莫不歡叫相賀,或臠割其肉,一夕都盡。明日,太子即位,彥範以功加銀青光祿大夫,拜納言,賜勛上柱國,封譙郡公,賜實封五百戶。又改為侍中,從新令也。



 彥範嘗表論時政數條,其大略曰:「昔孔子論《詩》以《關雎》為始,言后妃者人倫之本,理亂之端也。故皇、英降而虞道興,任、姒歸而姬宗盛。桀奔
 南巢,禍階妹喜,魯桓滅國,惑以齊媛。伏見陛下每臨朝聽政,皇后必施帷幔坐於殿上,預聞政事。臣愚歷選列闢,詳求往代,帝王有與婦人謀及政者,莫不破國亡身,傾輈繼路。且以陰乘陽,違天也,以婦凌夫,違人也。違天不祥,違人不義。由是古人譬以『牝雞之晨,惟家之索。』《易》曰『無攸遂,在中饋』,言婦人不得預於國政也。伏願陛下覽古人之言,察古人之意,上以社稷為重,下以蒼生在念。宜令皇后無往正殿干預外朝,專在中宮,聿修陰教,
 則坤儀式固,鼎命惟永。」



 又曰:「臣聞京師喧喧,道路籍籍,皆云胡僧慧範矯托佛教,詭惑後妃,故得出入禁闈,撓亂時政。陛下又輕騎微行,數幸其室,上下媟黷,有虧尊嚴。臣抑嘗聞興化致理,必由進善;康國寧人,莫大棄惡。故孔子曰:『執左道以亂政者殺,假鬼神以危人者殺。』今慧範之罪,不殊於此也。若不急誅,必生變亂。除惡務本,去邪勿疑,實願天聰,早加裁貶。」疏奏不納。時有墨敕授方術人鄭普思秘書監,葉凈能國子祭酒,彥範苦言其
 不可。帝曰:「既要用之,無容便止。」彥範又對曰:「陛下自龍飛寶位,遽下制云:『軍國政化,皆依貞觀故事。』昔貞觀中嘗以魏徵、虞世南、顏師古為秘書監,孔穎達為國子祭酒。至如普思等是方伎庸流,豈足以比蹤前烈?臣恐物議謂陛下官不擇才,濫以天秩加於私愛。惟陛下少加慎擇。」帝竟不納。



 時韋皇后既乾朝政,德靜郡王武三思又居中用事,以則天為彥範等所廢,常深憤怨,又慮彥範等漸除武氏,乃先事圖之。皇后韋氏既雅為帝所信
 寵,言無不從,三思又私通於韋氏,乃日夕讒毀彥範等。帝竟用三思計,進封彥範為扶陽郡王、敬暉為平陽郡五、張柬之為漢陽郡五、崔玄暐為博陵郡王、袁恕己為南陽郡王,並加特進,令罷知政事。彥範仍賜姓韋氏,令與皇后同屬籍,仍賜雜彩、錦繡、金銀、鞍馬等。雖外示優崇,而實奪其權也。易州刺史趙履溫者,即彥範之妻兄也。彥範誅易之後,奏言先與履溫共謀其事,於是召拜司農少卿。履溫德之,乃以二婢遺彥範。及彥範罷知政
 事,履溫又協奪其婢,大為時論所譏。尋出為洺州刺史,轉濠州刺史。



 二年,光祿卿、駙馬都尉王同皎以武三思與韋氏奸通,潛謀誅之。事洩,為三思誣構,言同皎將廢皇后韋氏,彥範等通知其情。乃貶彥範為瀧州司馬、敬暉崖州司馬、袁恕己竇州司馬、崔玄暐白州司馬、張柬之新州司馬,並仍令長任,勛封並削。彥範仍復其本姓桓氏。



 是歲秋,武三思又陰令人疏皇后穢行,榜於天津橋,請加廢黜。中宗聞之怒,命御史大夫李承嘉推求其
 人。承嘉希三思旨,奏言:「彥範與敬暉、張柬之、袁恕己、崔玄暐等教人密為此榜。雖托廢後為名,實有危君之計,請加族滅。」制依承嘉所奏。大理丞李朝隱執奏云:「敬暉等既未經鞫問,不可即肆誅夷。請差御史按罪,待至,準法處分。」大理卿裴談奏云:「敬暉等只合據敕斷罪,不可別俟推鞫,請並處斬籍沒。」中宗納其議,仍以彥範等五人嘗賜鐵券,許以不死,乃長流彥範於瀼州,敬暉於崖州,張柬之於瀧州,袁恕己於環州,崔玄暐於古州,並終
 身禁錮,子弟年十六已上者亦配流嶺外。擢授承嘉金紫光祿大夫,進封襄武郡公。韋氏又特賜承嘉彩物五百段、端錦被一張。擢拜裴談為刑部尚書,左貶李朝隱為聞喜令。三思俄又諷節愍太子抗表請夷彥範等三族。中宗以既有前命,不依其請。三思猶慮彥範等重被進用,又納中書舍人崔湜之計,特令湜姨兄嘉州司馬周利貞攝右臺侍御史,就嶺外並矯制殺之。彥範赴流所,行至貴州,利貞遇之於途,乃令左右執縛,曳於竹槎
 之上,肉盡至骨,然後杖殺,時年五十四。



 睿宗即位,延和元年,並追復其官爵,仍特還其子孫實封二百戶。玄宗即位,開元六年,詔曰:「皇輿肇建必有輔佐之臣;天步多艱,爰仗經綸之業。故侍中、譙國公桓彥範,侍中、平陽郡公敬暉,中書令兼吏部尚書、漢陽郡公張柬之,特進、博陵郡公崔玄暐,中書令、南陽郡公袁恕己等,並德惟神降,材與運生,道協臺岳,名書言千緯。寅亮帝載,勤勞王家,參復禹之元謀,奉升唐之景命。雖殂謝既久,而勛烈益
 彰,撫彞鼎以念功,想旂常而增感。緬遵故實,用表徽懿,俾列在清廟,登於明堂,克申從祀之儀,式茂疇庸之典。並可配享中宗孝和皇帝廟庭,其子北咸加收擢。」建中元年,重贈司徒。



 敬暉,絳州太平人也。弱冠舉明經。聖歷初,累除衛州刺史。時河北新有突厥之寇,方秋而而修城不輟,暉下車謂曰:「金湯非粟而不守,豈有棄收獲而繕城郭哉?」悉令罷散,由是人吏咸歌詠之。再遷夏官侍郎,出為泰州刺史。
 大足元年,遷洛州長史。天后幸長安,令暉知副留守事。在職以清乾著聞;璽書勞勉,賜物百段。長安三年,拜中臺右丞,加銀青光祿大夫。



 神龍元年,轉右羽林將軍。以誅張易之、昌宗功,加金紫光祿大夫,擢拜侍中,賜爵平陽郡公,食實封五百戶。尋進封齊國公。天后崩,遺制加實封通前滿七百戶。暉等以唐室中興,武氏諸王咸宜降爵,上章論奏,於是諸武降為公。武三思益怒,乃諷帝陽尊暉等為郡王,罷知政事。仍賜鐵券,恕十死,朔望趨
 朝。



 初,暉與彥範等誅張易之兄弟也,洛州長史薛季昶謂暉曰:「二兇雖除,產、祿猶在。請因兵勢誅武三思之屬,匡正王室,以安天下。」暉與張柬之屢陳不可,乃止。季昶嘆曰:「吾不知死所矣。」翌日,三思因韋後之助,潛入宮中,內行相事,反易國政,為天下所患,時議以此歸咎於暉。暉等既失政柄,受制於三思,暉每推床嗟惋,或彈指出血。柬之嘆曰:「主上疇昔為英王時,素稱勇烈,吾留諸武,冀自誅鋤耳。今事勢已去,知復何道。」



 三思既深憤惋,以
 許州司功參軍鄭愔素被暉等廢黜,因令上表陳其罪狀。中宗詔曰:「則天大聖皇后,往以憂勞不豫,兇豎弄權。暉等因興甲兵,鏟除妖孽,朕錄其勞效,備極寵勞。自謂勛高一時,遂欲權傾四海,擅作威福,輕侮國章,悖道棄義,莫斯之甚。然收其薄效,猶為隱忍,錫其郡王之重,優以特進之榮。不謂溪壑之志,殊難盈滿,既失大權,多懷怨望。乃與王同皎窺覘內禁,潛相謀結,更欲權兵絳闕,圖廢椒宮,險跡醜辭,驚視駭聽。屬以帝圖伊始,務靜狴
 牢,所以久為含容,未能暴諸遐邇。自同皎伏法,釁跡彌彰,倘若無其發明,何以懲茲悖亂?跡其巨逆,合置嚴誅。緣其昔立微功,所以特從寬宥,咸宜貶降,出佐遐籓。暉可崖州司馬,柬之可新州司馬,恕己可竇州司馬,玄暐可白州司馬,並員外置。」暉到崖州,竟為周利貞所殺。睿宗即位,追復五王官爵,贈暉秦州都督,謚曰肅愍。建中初,重贈太尉。曾孫元膺,開成三年,自試太子通事舍人為河南縣丞。



 崔玄暐,博陵安平人也。父行謹,為胡蘇令。本名曄,以字下體有則天祖諱,乃改為玄暐。少有學行,深為叔父秘書監行功所器重。龍朔中,舉明經,累補庫部員外郎。其母盧氏嘗誡之曰:「吾見姨兄屯田郎中辛玄馭云:『兒子從宦者,有人來云貧乏不能存,此是好消息。若聞貲貨充足,衣馬輕肥,此惡消息。』吾常重此言,以為確論。比見親表中仕宦者,多將錢物上其父母,父母但知喜悅,竟不問此物從何而來。必是祿俸餘資,誠亦善事。如其非
 理所得,此與盜賊何別?縱無大咎,獨不內愧於心?孟母不受魚鮓之饋,蓋為此也。汝今坐食祿俸,榮幸已多,若其不能忠清,何以戴天履地?孔子云:『雖日殺三性之養,猶為不孝。』又曰:『父母惟其疾之憂。』持宜修身潔已,勿累吾此意也。」玄暐遵奉母氏教誡,以清謹見稱。尋授天宮郎中,遷鳳閣舍人。



 長安元年,超拜天官侍郎。每介然自守,都絕請謁,頗為執政者所忌。轉文昌左丞。經月餘,則天謂曰:「自卿改職以來,選司大有罪過。或聞令史乃設
 齋自慶,此欲盛為貪惡耳。今要卿復舊任。」又除天官侍郎,賜雜彩七十段。三年,拜鸞臺侍郎、同鳳閣鸞臺平章事,兼太子左庶子。四年,遷鳳閣侍郎,加銀青光祿大夫,仍依舊知政事。先是,來俊臣、周興等誣陷良善,冀圖爵賞,因緣籍沒者數百家。玄暐固陳其枉狀,則天乃感悟,咸從雪免。



 則天季年,宋璟劾奏張昌宗謀為不軌,玄暐亦屢有讜言,則天乃令法司正斷其罪。玄暐弟升時為司刑少卿,又請置以大闢。其兄弟守正如此。是時,則天
 不豫,宰相不得召見者累月。及疾少間,玄暐奏言:「皇太子、相王仁QD明孝友,足可親侍湯藥。宮禁事重,伏願不令異姓出入。」則天曰:「深領卿厚意。」尋以預誅張易之功,擢拜中書令,封博陵郡公。中宗將授方術人鄭普思為秘書監,玄暐切諫,竟不納。尋進爵為王,賜實封四百戶,檢校益州大都督府長史,兼知都督事。其後被累貶,授白州司馬,在道病卒。建中初,贈太子太師。



 玄暐與弟升甚相友愛。諸子弟孤貧者,多躬自撫養教授,頗為當時所
 稱。升,官至尚書左丞。玄暐少時頗屬詩賦,晚年以為非己所長,乃不復構思,唯篤志經籍,述作為事。所撰《行己要範》十卷、《友義傳》十卷、《義士傳》十五卷、訓注《文館辭林策》二十卷,並行於代。子璩,頗以文學知名,官歷中書舍人、禮部侍郎。璩子渙,自有傳。曾孫郢,開成三年,自商州防禦判官兼殿中侍御史,入為監察御史。



 張柬之,字孟將,襄州襄陽人也。少補太學生,涉獵經史,尤好《三禮》,國子祭酒令狐德棻甚重之。進士擢第,累補
 青城丞。永昌元年,以賢良徵試,同時策者千餘人,柬之獨為當時第一,擢拜監察御史。



 聖歷初,累遷鳳閣舍人。時弘文館直學士王元感著論云:「三年之喪,合三十六月。」柬之著論駁之曰:



 三年之喪,二十五月,不刊之典也。謹案《春秋》:「魯僖公三十三年十二月乙巳,公薨。」「文公二年冬,公子遂如齊納幣。」《左傳》曰「禮也。」杜預注云:「僖公喪終此年十一月,納幣在十二月。士婚禮,納採納徵,皆有玄纁束帛,諸侯則謂之納幣。蓋公為太子,已行婚禮。」故《
 傳》稱禮也。《公羊傳》曰:「納幣不書,此何以書?譏喪娶。在三年之外何以譏?三年之內不圖婚。」何休注云:「僖公以十二月薨,至此冬未滿二十五月,納採、問名、納吉,皆在三年之內,故譏。」何休以公十二月薨,至此冬十二月才二十四月,非二十五月,是未三年而圖婚也。按《經》書「十二月乙巳公薨」,杜預以《長歷》推乙巳是十一月十二日,非十二月,書十二月,是《經》誤。「文公元年四月,葬我君僖公」,《傳》曰,緩也。諸侯五月而葬,若是十二月薨,即是五月,不
 得言緩。明知是十一月薨,故注僖公喪終此年,至十二月而滿二十五月,故丘明《傳》曰,禮也。據此推步,杜之考校,豈公羊之所能逮,況丘明親受《經》於仲尼乎?且二《傳》何、杜所爭,唯爭一月,不爭一年。其二十五月除喪,由來無別。此則《春秋》三年之喪,二十五月之明驗也。



 《尚書伊訓》云:「成湯既沒,太甲元年,惟元祀十有二月,伊尹祀於先王,奉嗣王祗見厥祖。」孔安國注云:「湯以元年十一月崩。」據此,則二年十一月小祥,三年十一月大祥。故《太甲》
 中篇云:「惟三祀十有二月朔,伊尹以冕服奉嗣王歸於亳。」是十一月大祥,訖十二月朔日,加王冕服吉而歸亳也。是孔言「湯元年十一月」之明驗。《顧命》云:「四月哉生魄,王不懌」,是四月十六日也。「翌日乙丑,王崩」,是十七日也。「丁卯,命作冊度」,是十九日也。「越七日癸酉,伯相命士須材」,是四月二十五日也。則成王崩至康王麻冕黼裳,中間有十月,康王方始見廟。則知湯崩在十一月,淹停至殮訖,方始十二月,祗見其祖。《顧命》見廟訖,諸侯出廟門
 俟,《伊訓》言「祗見厥祖,侯甸群後咸在』,則崩及見廟,殷、周之禮並同。此周因於殷禮,損益可知也。不得元年以前,別有一年。此《尚書》三年之喪,二十五月之明驗也。



 《禮記三年問》云:「三年之喪,二十五月而畢,哀痛未盡,思慕未忘,然而服以是斷之者,豈不送死有已,復生有節?」又《喪服四制》云:「變而從宜,故大祥鼓素琴,告人以終。」又《間傳》云:「期而小祥,食菜果。又期而大祥,有醯醬。中月而禫,食酒肉。」又《喪服小記》云:「再期之喪,三年也。期之喪,二年也。
 九月七月之喪,三時也。五月之喪,二時也。三月之喪,一時也。」此《禮記》三年之喪,二十五月之明驗也。



 《儀禮士虞禮》云:「期而小祥。又期而大祥。中月而禫,是月也吉祭。」此禮周公所制,則《儀禮》三年之喪,二十五月之明驗也。



 此四驗者,並禮經正文,或周公所制,或仲尼所述,吾子豈得以《禮記》戴聖所修,輒欲排毀?漢初高堂生傳《禮》,既未周備,宣帝時少傳後蒼因淹中孔壁所得五十六篇著《曲臺記以授弟子戴德、戴聖、慶溥三人,合以正經及孫
 卿所述,並相符會。列於學官,年代已久。今無端構造異論,既無依據,深可嘆息。其二十五月,先儒考校,唯鄭康成注《儀禮》「中月而禫」,以「中月間一月,自死至禫凡二十七月」。又解禫云:「言澹澹然平安之意也。今皆二十七月復常,從鄭議也。逾月入禫,禫既復常,則二十五月為免喪矣。二十五月、二十七月,其議本同。



 竊以子之於父母喪也,有終身之痛,創巨者日久,痛深者愈遲,豈徒歲月而已乎?故練而慨然者,蓋悲慕之懷未盡,而踴擗之情
 已歇;祥而廓然者,蓋哀傷之痛已除,而孤邈之念更起。此皆情之所致,豈外飾哉。故《記》曰:三年之喪,義同過隙,先王立其中制,以成文理。是以祥則縞帶素紕,禫則無所不佩。今吾子將徇情棄禮,實為乖僻。夫棄縗麻之服,襲錦縠之衣,行道之人,皆不忍也,直為節之以禮,無可奈何。故由也不能過制為姊服,鯉也不能過期哭其母。夫豈不懷,懼名教逼己也。若孔、鄭、何、杜之徒,並命代挺生,範模來裔,宮墻積仞,未易可窺。但鉆仰不休,當漸入
 勝境,詎勞終年矻矻,虛肆莠言?請所有掎手適先儒,願且以時消息。



 時人以柬之所駁,頗合於禮典。



 是歲,突厥默啜表言有女請和親,則天盛意許之,欲令淮陽郡王延秀娶之。柬之奏曰:「自古無天子求娶夷狄女以配中國王者。」表入,頗忤其旨。神功初,出為合州刺史,尋轉蜀州刺史。舊例,每歲差兵募五百人往姚州鎮守,路越山險,死者甚多。柬之表論其弊曰:



 臣竊按姚州者,古哀牢之舊國。絕域荒外,山高水深,自生人以來,洎於後漢,不與
 中國交通。前漢唐蒙開夜郎滇筰,而哀牢不附。至光武季年,始請內屬,漢置永昌郡以統理之,乃收其鹽布毯罽之稅,以利中土。其國西通大秦,南通交趾,奇珍異寶,進貢歲時不闕。劉備據有巴蜀,常以甲兵不充。及備死,諸葛亮五月渡瀘,收其金銀鹽布以益軍儲,使張伯岐選其勁卒搜兵以增武備。故《蜀志》稱自亮南征之後,國以富饒,甲兵充足。由此言之,則前代置郡,其利頗深。今鹽布之稅不供,珍奇之貢不入,戈戟之用不實於戎行,
 寶貨之資不輸於大國,而空竭府庫,驅率平人,受役蠻夷,肝腦塗地,臣竊為國家惜之。



 昔漢以得利既多,歷博南山,涉蘭倉水,更置博南、哀牢二縣。蜀人愁怨,行者作歌曰:「歷博南,越蘭津,渡蘭蒼,為他人。」蓋譏漢貪珍奇鹽布之利,而為蠻夷之所驅役也。漢獲其利,人且怨歌。今減耗國儲,費用日廣,而使陛下之赤子身膏野草,骸骨不歸,老母幼子,哀號望祭於千里之外。於國家無絲發之利,在百姓受終身之酷。臣竊為國家痛之。



 往者,諸葛
 亮破南中,使其渠率自相統領,不置漢官,亦不留兵鎮守。人問其故,亮言置官留兵有三不易。大意以置官夷漢雜居,猜嫌必起;留兵運糧,為患更重;忽若反叛,勞費更多。但粗設紀綱,自然安定。臣竊以亮之此策,妙得羈縻蠻夷之術。



 今姚府所置之官,既無安邊靜寇之心,又無葛亮且縱且擒之伎。唯知詭謀狡算,恣情割剝,貪叨劫掠,積以為常。扇動酋渠,遺成朋黨,折支諂笑,取媚蠻夷,拜跪趨伏,無復慚恥。提挈子弟,嘯引兇愚,聚會蒲博,
 一擲累萬。劍南逋逃,中原亡命,有二千餘戶,見散在彼州,專以掠奪為業。姚州本龍朔中武陵縣主簿石子仁奏置之,後長史李孝讓、辛文協並為群蠻所殺。前朝遣郎將趙武貴討擊,貴及蜀兵應時破敗,噍類無遺。又使將軍李義總等往征,郎將劉惠基在陣戰死,其州乃廢。臣竊以諸葛亮稱置官留兵有三不易,其言乃驗。至垂拱四年,蠻郎將王善寶、昆州刺史爨乾福又請置州,奏言所有課稅,自出姚府管內,更不勞擾蜀中。及置州後,
 錄事參軍李棱為蠻所殺。延載中,司馬成琛奏請於瀘南置鎮七所,遣蜀兵防守,自此蜀中騷擾,於今不息。



 且姚府總管五十七州,巨猾游客,不可勝數。國家設官分職,本以化俗妨奸,無恥無厭,狼籍至此。今不問夷夏,負罪並深,見道路劫殺,不能禁止,恐一旦驚擾,為禍轉大。伏乞省罷姚州,使隸巂府,歲時朝覲,同之蕃國。瀘南諸鎮,亦皆悉廢,於瀘北置關,百姓自非奉使入蕃,不許交通往來。增巂府兵選,擇清良宰牧以統理之。臣愚將為
 穩便。



 疏奏,則天不納。



 後累拜荊州大都督府長史。長安中,召為司刑少卿,遷秋官侍郎。時夏官尚書姚崇為靈武軍使,將行,則天令舉外司堪為宰相者。崇對曰:「張柬之沉厚有謀,能斷大事,且其人年老,惟陛下急用之。」則天登時召見,尋同鳳閣鸞臺平章事。未幾,遷鳳閣侍郎,仍知政事。及誅張易之兄弟,柬之首謀其事。中宗即位,以功擢拜天官尚書、鳳閣鸞臺三品,封漢陽郡公,食實封五百戶,未幾,遷中書令,監修國史。月餘,進封漢陽郡
 王,加授特進,令罷知政事。



 其年秋,柬之表請歸襄州養疾。許之,仍特授襄州刺史,又拜其子漪為著作郎,令隨父之任。上親賦詩祖道,又令群公餞送於定鼎門外。柬之至襄州,有鄉親舊交抵罪者,必深文致法,無所縱舍。其子漪恃以立功,每見諸少長,不以禮接,時議以為不能易荊楚之剽性焉。尋為武三思所構,貶授新州司馬。柬之至新州,憤恚而卒,年八十餘,景雲元年,制曰:「褒德紀功,事華典冊;飾終追遠,理光名教。故吏部尚書張
 柬之翼戴興運,謨明帝道,經綸謇諤,風範猶存。往屬回邪,構成釁咎,無辜放逐,淪沒荒遐。言念勛賢,良深軫悼,宜加寵贈,式賁幽泉。可贈中書令,封漢陽郡公。」建中初,又贈司徒。玄孫璟,開成二年,自宜城尉遷壽安尉。



 袁恕己,滄州東光人也。長安中,歷遷司刑少卿,兼知相王府司馬事。敬暉等將誅張易之兄弟,恕己預其謀議,又從相王統率南衙兵仗,以備非常。及事定,加銀青光祿大夫,行中書侍郎、同中書門下三品,封南陽郡公,食
 實封五百戶。將作少匠楊務廉素以工巧見用,中興初,恕己恐其更啟游娛侈靡之端,言於中宗曰:「務廉致位九卿,積有歲年,苦言嘉謀,無足可紀。每宮室營構,必務其侈,若不斥之,何以廣昭聖德?」由是左授務廉陵州刺史。恕己俄擢拜中書令,仍加特進,封南陽郡王,罷知政事。則天崩,遺制加實封滿七百戶。後與敬暉等累遭貶黜,流於環州。尋為周利貞所逼,飲野葛汁數升,恕己常服黃金,餌毒發,憤悶,以手掘地,取土而食,爪甲殆盡,竟
 不死,乃擊殺之。建中初,贈太子太傅。曾孫德文,舉進士,開成三年,授秘書省校書郎。



 史臣曰:昔夫差入越,勾踐保於會稽,不聽子胥之言,而有甬東之嘆。此五王除兇返正,得計成功。當是時,彥範、敬暉握兵全勢,三思、攸暨其黨半殲,若從季昶之言,寧有利貞之禍?蓋以心懷不忍,遽失後圖,黜削流移,理固然也。且芟蔓而不能拔本,建謀而尚欠防微,死即無辜,禍由自掇。失斷召亂也,不亦宜哉!



 贊曰:嗟彼五王,忠於有唐。知火在木,謂其無傷。禍發既克,勢摧靡當。何事不敏,周身之防。



\end{pinyinscope}