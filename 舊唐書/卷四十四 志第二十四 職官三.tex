\article{卷四十四 志第二十四 職官三}

\begin{pinyinscope}

 御史臺秦、漢曰御史府,後漢改為憲臺,魏、晉、宋改為蘭臺,梁、陳、北朝咸曰御史臺。武德因之。龍朔二年
 改名憲臺。咸亨復。光宅元年分臺為左右,號曰左右肅政臺。左臺專知京百司,右臺按察諸州。神龍復為左右御史臺。延和年廢右臺,先天二年復置,十月
 又廢也。



 大夫一員,正三品。秦、漢之制,御史大夫、副丞相為三公之官。魏、晉之後,多不置大夫,以中丞為臺主。隋諱中,復大夫,降為正四品。《武德令》改為從三品。龍朔改為大司憲,咸亨復為大夫。光宅分臺為左、右,置左、右臺大夫。及廢右臺,去「左」「右」字。本從三品,會昌二年十二月敕:「大夫,秦為正卿,漢為副相,漢末改為大司空,與丞相俱為三公。掌邦國刑憲,肅正朝廷。其任既重,品秩宜峻。準六尚書例,升為正三品,著之於令。」



 中丞二員。正四品下。漢御史臺有二丞,掌殿內秘書,謂之中丞。漢末改為御史長史,後漢復為中丞。後魏改為中尉正,北齊復曰中丞。後周曰司憲中大夫。隋諱中,改為持書御史,為從五品。武德因之。貞觀末,避高宗名,改持書御史為中丞,置二員。龍朔改為司憲大夫,咸亨復為中丞。本正五品上,會昌二年十
 二月敕:「中丞為大夫之貳,緣大夫秩崇,官不常置,中丞為憲臺長。今九寺少卿及諸少監、國子司業、京兆少尹,並府寺省監之貳,皆為四品,唯中丞官重,品秩未崇,可升為正四品下,與丞郎出入迭用,著之於令。」



 大夫、中丞之職,掌持邦國刑憲典章,以肅正朝廷。中丞為之貳。凡天下之人,有稱冤而無告者,與三司訊之。凡中外百僚之事,應彈劾者,御史言於大夫。大事則方幅奏彈之,小事則署名而已。若有制使覆囚徒,則與刑部尚書參擇之。凡國有大禮,則乘輅車以為之導。



 侍御史四員。從六品下。御史之名,《周官》有之,亦名柱下史。秦改為侍御史。後周曰司憲中士,隋為侍御史,品第七。武德品第六也。



 掌糾舉百僚,推鞫獄訟。侍御史年深者一人判臺事,知公廨雜事,次一人知西推,
 一人知東推也。



 凡有別付推者,則按其實狀以奏。若尋常之獄,推訖斷於大理。凡事非大夫、中丞所劾,而合彈奏者,則具其事為狀,大夫、中丞押奏。大事則冠法冠,衣硃衣纁裳,白紗中單以彈之。小事常服而已。凡三司理事,則與給事中、中書舍人更直,直於朝堂受表。若三司所按而非其長官,則與刑部郎中員外、大理司直評事往訊之。



 主簿一人,
 從七品下。錄事二人,從九品下。



 主簿掌印及受事發辰,勾檢稽失。兼知官廚及黃卷。主事二人,令史十七人,書令史二十三人。



 殿中侍御史六人,從七品下。令史八人,書令史十八人。殿中侍御史掌殿廷供奉之儀式。凡冬至、元正大朝會,則具服升殿。若郊祀、巡幸,則於鹵簿
 中糾察非違,具服從於旌門,視文物有所虧闕,則糾之。凡兩京城內,則分知左右巡,各察其所巡之內有不法之事。



 監察御史十員,正八品上。貞觀初,馬周以布衣進用,大宗令於監察御史裏行。自此因置里行之名。龍朔元年,以王本立為監察裏行也。



 監察掌分察巡按郡縣、屯田、鑄錢、嶺南選補、知太府、司農出納,監決囚徒。監祭祀則閱
 牲牢,省器服,不敬則劾祭官。尚書省有會議,亦監其過謬。凡百官宴會、習射,亦如之。



 殿中省魏初置殿中監,隋初改為殿中局,煬帝改為殿內省,武德改為殿中省。龍朔改為中御府,咸亨復為殿中省。



 監一員,從三品。魏初置,品第二。梁品第三。隋品第四。武德品第三也。少監二員,從四品上。丞二人,從五品上。



 主事二人,從九品上。令史四人,書令史十二人,亭長、掌固各八人。殿中監掌天子服御,總領尚食、尚藥、尚衣、尚舍、尚舍、尚輦六局之官屬,備其禮物,供其職
 事。少監為之貳。凡聽朝,則率其屬執傘扇以列於左右。凡大祭祀,則進大珪、鎮珪於壝門之外。既事,受而藏之。凡行幸,則侍奉於仗內,驂乘以從。若元正、冬至大朝會,則有進爵之禮。丞掌副監事,兼勾檢稽失,省署抄目。主事掌印及知受事發辰。



 尚食局:奉御二人,正五品下。隋初為典御,又改為奉御。直長五人,正七品上。



 食醫八人。正九品下。奉御掌謹其儲供,辨名數。直長為之貳。若進御,必辨其時禁。春肝,夏心,秋肺,冬腎,四季之月脾
 王,皆不可食。當進,必先嘗。正、至大朝會饗宴,與光祿大夫視其品秩之差。其賜王公賓客,亦如之。諸陵月享,則視膳而獻之。食醫掌率主食王膳,以供其職。



 尚藥局:奉御二人,正五品下。直長四人,正七品上。



 書吏四人。侍御醫四人,從六品上。主藥十二人,藥童三十人。司醫四人,正八品下。



 醫佐八人,正八品下。按摩師四人,咒禁師四人,合口脂匠四人,掌固四人。奉御掌合和御藥及診候方脈之事。直長為之貳。凡藥有上、中、下三品,上藥為君,中藥為臣,下
 藥為佐。合造之法,一君三臣九佐,別入五藏,分其五味。有湯丸膏散之用。診脈有寸、關、尺之三部,醫之大經。凡合和與監視其分劑,藥成嘗而進焉。侍御醫,掌診候調和。主藥、藥童,主刮削搗簁。



 尚衣局:奉御二人,從五品上。直長四人,正七品下。



 書令史三人,書吏四人,主衣十六人,掌固四人。奉御掌衣服,詳其制度,辨其名數。直長為之貳。凡天子之服冕十有三;一,大裘冕,二袞冕,一鷩冕,四毳冕,五黻冕,六玄冕,七通天冠,
 八武弁,九弁服,十介幘,十一白紗帽,十二平巾幘,十三翼善冠。事具《輿服志》。



 凡天子之大珪,曰珽,長三尺。鎮珪,長尺有二寸。若有事於郊丘社稷,則出之於內。將享,至於中壝門,則奉鎮珪於監而進之。既事,受而藏之。凡大朝會,則設案,服畢而徹之。



 尚舍局:奉御二人,從五品上。直工六人,正七品下。



 書令史三人,書吏七人,掌固十人,幕士八十人。奉御掌殿廷張設、湯沐、燈燭、灑掃之事。直長為之貳。凡行幸,預設三部帳幕,
 有古帳、大帳、次帳、小次帳、小帳,凡五等之帳為三部。



 其外置排城以為蔽捍。排城,連板為之,板上畫闢邪獸,表裏皆漆之。凡大祭祀,有事於郊壇,則先設行宮於壇之東南向,隨地之宜,將祀三日,則設大次於外壝東門之外道北,南向而設坐。若有事於明堂太廟,則設大次於東門,如郊壇之制。凡致齋則設幄於正殿西序及室內,俱東向,張於楹下。凡元正、冬至大朝會,則設斧扆於正殿。



 施蹋席薰爐。朔望受朝,則施幄於正殿,帳裙頂帶,方闊一丈四尺也。



 尚乘局:奉御二人,從五品上。直長一人,正七品下。



 奉乘十八人,正九
 品下。習馭五十人,掌閑五十人,獸醫七十人。進馬六人,七品下。



 司庫一人,正九品下。司廩二人,正九品下。



 書令史一人,書吏十四人。奉御掌內外閑廄之馬,辨其粗良,而率其習馭。直長為之貳。一曰左右飛黃閑,二曰左右吉良閑,三曰左右龍媒閑,四曰左右抃駼閑,五曰左右駃騠閑,六曰左右天苑閑。開元時仗內六閑,曰飛龍、祥麟、鳳苑、鵷鶵、吉良、六群等,號六廄馬。



 凡秣馬給料,以時為差。凡外牧進良馬,印以三花飛風之字而為志。奉乘掌率習馭、掌閑、駕士及秣飼之法。司庫掌鞍轡乘具。司廩掌槁秸出納。獸醫掌療馬病。



 初尚乘局掌六閑馬,後置
 內外閑廄使,專掌御馬。開元初,以尚乘局隸閑廄使,乃省尚乘,其左右六閑及局官,並隸閑廄使領之也。進馬舊儀,每日尚乘以廄馬八匹,分為左右廂,立於正殿側宮門外,候仗下即散。若大陳設,即馬在樂懸之北,與大象相次。進馬二人,戎服執鞭,侍立於馬之左,隨馬進退。雖名管殿中,其實武職,用資廕簡擇,一如千牛備身。天寶八載,李林甫用事,罷立仗馬,亦省進馬官。十二載,楊國忠當政,復立仗馬及進馬官,乾元復省,上元復置也。



 尚輦局:奉御二人,從五品上。直長四人,正七品下。



 尚輦二人,正九品下。書令史二人,書吏四人,掌扇六人,掌翰二十四人,主輦三十二人,奉輿十二人,掌固四人。奉御掌輿輦,分其次序而辨其名數。直長為之貳。凡大朝會則陳於廷,大
 祭祀則陳於廟。凡大朝會,則傘二翰一,陳之於廷。



 孔雀扇一百五十有六,分居左右。舊翟尾扇,開元年初改為繡孔雀。若常聽朝,皆去扇,左右各留其三,以備常儀。



 內官



 妃三人。正一品。《周官》三夫人之位也。隋依周制,立三夫人。武德立四妃:一貴妃,二淑妃,三德妃,四賢妃,位次後之下。玄宗以為后妃四星,其一正後,不宜更有四妃,乃改定三妃之位:惠妃一,麗妃二,華妃三,下有六儀、美人、才人四等,共二十人以備內官之位也。



 三妃佐後,坐而論婦禮者也。其於內,則無所不統,故不以一務名焉。六儀六人,正二品,《周
 官》九嬪之位也。



 掌教九御四德,率其屬以贊導後之禮儀。美人四人,正三品,《周官》二十七世婦之位也。



 掌率女官,修祭祀賓客之事。才人七人,正四品,《周官》八十一御女之位。掌敘宴寢,理絲枲,以獻歲功。



 宮官六尚,如六尚書之職掌。



 尚宮二人,正五品。司記二人,正六品。



 典記二人,正七品。掌記二人,正八品。



 女史六人。司言二人,正七品。典言二人,正八品。



 掌言二人,正八品。女史四人。司簿二人,正六品。



 典簿二人,正七品。掌簿二人,正八品。



 女史六人。司闈六人,正六品。典闈六人,正七品。



 掌闈六人,正八品。女史四人。尚宮職,掌導引中宮,總司記、司言、司簿、司闈四司之官屬。凡六尚書物出納文簿,皆印署之。司記掌印,凡宮內諸司簿書出入目錄,審而付行焉。典記佐之,女史掌執文書。司言掌宣傳啟奏。司簿掌宮人名簿廩賜。司闈掌宮闈管籥。



 尚儀二人,正五品。司籍二人,正六品。



 典籍二人,正七品。掌籍二人,正八品。



 女史十人,司樂四人,正六品。典樂四人,正七品。



 掌樂二人,正八品。女史二人。司賓二人,正六品。



 典賓二人,正七
 品,掌賓二人。正八品。



 司贊二人,正六品。典贊二人,正六品。



 掌贊二人,正六品。女史二人。尚儀之職,掌禮儀起居,總司籍、司樂、司賓、司贊四司之官屬。司籍掌四部經籍、筆札幾案。司東掌率樂人習樂,陳懸、拊擊、進退。司賓掌賓客朝見、宴會賞賜。司贊掌朝見宴會贊相。



 尚服二人,正五品。司寶二人,正六品。



 典寶二人,正七品。掌寶二人,正八品。



 女史四人。司衣二人,正六品。典衣二人,正七品。



 掌衣二人,正八品。女史四人。司飾二人,正六品。



 典飾二人,正七品。掌
 飾二人,正八品。



 女史四人。司仗二人,正六品。典仗二人,正七品。



 掌仗二人,正八品。女史二人。尚服之職,掌供內服用採章之數,總司寶、司衣、司飾、司仗四司之官屬。司寶掌瑞寶、符契、圖籍。司衣掌衣服首飾。司飾掌膏沐巾櫛。司仗掌羽儀仗衛。



 尚食二人,正五品。司膳四人,正六品。



 典膳四人,正七品。掌膳四人,正八品。



 掌醖二人,正八品。女史四人,司醖二人,正七品。



 典醖二人,正七品。女史二人。司藥二人,正六品。



 典藥二人,正七品。掌
 藥二人,正八品。



 女史四人。司饎二人,正六品。典饎二人,正七品。



 掌饎二人,正八品。女史四人。尚食之職,掌供膳羞品齊之數,總司膳、司醖、司藥、司饎四司之官屬。凡進食,先嘗之。司膳掌制烹煎和。司醖掌酒醴枌飲。司藥掌方藥。司饎掌給宮人廩餼飯食、薪炭。



 尚寢二人,正五品。司設二人,正六品。



 典設二人,正七品。掌設二人,正八品。



 女史四人。司輿二人,正六品。典輿二人,正七品掌輿二人,正八品。女史一人。司苑二人,正六品。



 典苑二人,正七品。掌
 苑二人,正八品。



 女史二人。司燈二人,正六品。典燈二人,正七品。



 掌燈二人,正八品。女史二人。尚寢之職,掌燕寢進御之次序,總司設、司輿、司苑、司燈四司之官屬。司設掌幃帳茵席、掃灑張設。司輿掌輿輦傘扇羽儀。司苑掌園苑種植蔬果。司燈掌燈燭。



 尚功二人,正五品。司制二人,正六品。



 典制二人,正七品。掌制二人,正八品。



 女史二人。司珍二人,正六品。典珍二人,正七品。



 掌珍二人,正八品。女史六人。司彩二人,正六品。



 典彩二人,正七品。掌
 彩二人,正八品。



 女史二人。司計二人,正六品。典計二人,正七品。



 掌計二人,正八品。女史二人。尚功之職,掌女功之程課,總司制、司珍、司彩、司計四司之官屬。司制掌衣服裁縫。司珍掌寶貨。司彩掌繒錦絲枲之事。司計掌支度衣服、飲食、薪炭。



 宮正一人,正五品。司正二人,正六品。



 典正二人,正七品。女史四人。宮正之職,掌戒令、糾禁、謫罰之事。司正、典正佐之。



 右唐制定宮官六尚書、二十四司職事官,以備內職之數。



 內侍省《星經》有宦者四星,在天市垣,帝坐之西。《周官》有巷伯、寺人之職,皆內官也。前漢宮官,多用士人,後漢始用宦者為宮官。晉置大長秋卿為後宮官,以宦者為之。隋為內侍省,煬帝改為長秋監。武德復為內侍。龍朔改為內侍監,光宅改為司宮臺,神龍復為內侍省也。



 內侍四員。從四品上。漢、魏曰長秋卿,梁曰大長秋,北齊曰中侍中,後周曰司內上士,隋曰內侍,置二人。煬帝曰長秋令,正四品。武德復為中侍。中官之貴,極於此矣。若有殊勛懋績,則有拜大將軍者,仍兼內侍之官。德宗置左、右神策、威遠等禁兵,命中官掌之。每軍置中尉一人,宦者為之。自李輔國、魚朝恩之後,京師兵柄,歸於內官,號左、右軍中尉。將兵於外者,謂之觀軍容使。而天下軍鎮節度使,皆內官一人監之,事具《宦者傳》也。



 內常侍六人。正五品下。漢代謂之中常侍。內侍之職,掌在內侍奉
 出入宮掖宣傳之事,總掖廷、宮闈、奚官、內僕、內府五局之官屬。內常侍為之貳。凡皇后祭先蠶,則相儀。後出,則為之夾引。



 內給事八人,從五品下。主事二人,從九品下。



 令史八人,書令史十六人。內給事掌判省事。凡元正、冬至群臣朝賀中宮,則出入宣傳。凡宮人衣服費用,則具其品秩,計其多少,春秋二時,宣送中書。



 內謁者監六人,正六品下。內謁者十二人,從八品下。



 內寺伯二人。
 正七品下。內謁者監掌內宣傳。凡諸親命婦朝會,所司籍其人數,送內侍省。內謁者掌諸親命婦朝集班位。內寺伯掌糾察諸不法之事。歲大儺,則監其出入。



 掖廷局:令二人,從七品下。丞三人,從八品下。



 宮教博士二人,從九品下。監作四人,從九品下。



 令史四人,計史二人,書令史八人。掖廷令掌宮禁女工之事。凡宮人名籍,司其除附,公桑養蠶,會其課業。丞掌判局事。博士掌教習宮人書算眾藝。監作掌監當雜作。



 宮闈局:令二人,從七品下。丞二人,從八品下。



 令史三人,書吏六人,內閽人二十人,內掌扇十六人,內給使無常員。宮闈局令掌侍奉宮闈,出入管鑰。凡大享太廟,帥其屬詣於室,出皇后神主置於輿而登座焉。既事,納之。凡宮人無官品者,稱內給使。若有官及經解免應敘選者,得令長上,其小給使學生五十人,皆總其名籍,以給其糧廩。丞掌判局事。內給使掌諸門進物出納之歷。



 奚官局:令二人,正八品下。丞二人,正九品下。



 書令史三人,書吏六
 人,藥童四人。奚官令掌奚隸工役、宮官品命。丞為之貳。凡宮人有疾病,則供其醫藥,死亡則供其衣服,各視其品命。仍於隨近寺觀,為之修福。雖無品,亦如之。凡內命婦五品已上亡,無親戚於墓側,三年內取同姓中男一人,以時主祭。無同姓,則所司春秋以少牢祭之。



 內僕局:令二人,正八品下。丞二人,正九品下。



 書令史二人,書吏四人,駕士二百人。內僕令掌中宮車乘出入導引。丞為之貳。凡中宮有出入則令居左,丞居右,而夾引之。凡皇
 后之車有六,事在《輿服》也。內府局:令二人,正八品下。丞二人,正九品下。



 書令史二人,書吏四人。內府令掌中藏寶貨,給納名數。丞為之貳。凡朝會五品已上,賜絹帛金銀器於殿廷者,並供之。諸將有功,並蕃酋辭還,亦如之。



 太常寺古曰秩宗,秦曰奉常,漢高改為太常,梁加「寺」字,後代因之。



 卿一員,正三品。梁置十二卿,太常卿為一。周、隋品第三。龍朔二年改為奉常,光宅改為司禮卿,神龍復為太常卿也。



 少卿二人。正四品。隋置少卿二人,從四品。武德置一人,貞觀加置一員。太常
 卿之職,掌邦國禮樂、郊廟、社稷之事,以八署分而理之:一曰郊社,二曰太廟,三曰諸陵,四曰太樂,五曰鼓吹,六曰太醫,七曰太卜,八曰廩犧。總其官屬,行其政令。少卿為之貳。凡國有大禮,則贊相禮儀。有司攝事,則為之亞獻。率太樂官屬,宿設樂懸,以供其事。宴會,亦如之。若三公行園陵,則為之副,公服乘輅備鹵簿而奉其禮。若大祭祀,則先省牲器。凡太卜占國之大事及祭祀卜日,皆往蒞之於太廟南門之外。凡仲春薦冰及四時品物甘
 滋新成者,皆薦焉。凡有事於宗廟,少卿帥太祝、齋郎入薦香燈,整拂神幄,出入神主。將享,則興良醖令實樽罍。凡備大享之器服,有四院。



 一曰天府院,二曰御衣院,三曰樂懸院,四曰神廚院。



 丞二人,從五品上。主簿二人,從七品上。



 錄事二人,從九品下。府十二人,史二十三人。博士四人,從七品上。謁者十人,贊引二十人。太祝六人,正九品上。祝史六人。奉禮二人,從九品上。贊者十六人。協律郎二人,正八品上。



 亭長八人,掌固十二人,太廟齋郎,京、都各一百三十人。太廟門僕,京、都各三十人。丞掌判寺
 事。凡大饗太廟,則修七祀於太廟西門之內。若祫享,則兼修配享功臣之禮。主簿掌印,勾檢稽失,省署抄目。錄事掌受事發辰。博士掌五禮之儀式,本先王之法制,適變隨時而損益焉。凡大祭祀及有大禮,則與卿導贊其儀。凡公已下擬謚,皆跡其功行,為之褒貶。無爵稱子,養德邱園,聲實明著,則謚曰先生。大行大名,小行小名之。



 古有《周書謚法》,《大戴禮謚法》,漢劉熙《謚法》一卷,晉張靖《謚法》兩卷,又有《廣謚法》一卷,梁沈約總聚古今謚法,凡有一百六十五稱也。



 若大祭禮,卿省牲器,謁者為之導。若小祀及
 公卿大夫有嘉禮,亦命謁者以贊之。太祝掌出納神主於太廟之九室,而奉享禘佩給之儀。凡國有大祭祀,凡郊廟之祝版,先進取署,乃送祠所。將事,則跪讀祝文,以信於神;禮成而焚之。凡大祭祀,卿省牲而告充。凡祭天及日月星辰之玉帛,則焚之;祭地及社稷山嶽,則瘞之;瀆污,則沉之。奉禮郎掌朝會祭祀君臣之版位。凡樽卣之制,十有四,祭則陳之。祭器之位,簠簋為前,鈃次之,籩豆為後。大凡祭祀朝會,在位者拜跪之節,皆贊導之,
 贊者承傳焉。又設牲榜之位,以成省牲之儀。凡春秋二仲,公卿巡陵,則主其威儀鼓吹之節而相禮焉。協律郎掌和六呂六律,辨四時之氣,八風五音之節。凡太樂,則監試之,為之課限。若大祭祀饗宴奏於廷,則升堂執麾以為之節制,舉麾工鼓柷而後樂作,偃麾戛敔而後止。



 兩京郊社署:令各一人,從七品下。丞一人,從八品上。府二人,史四人,典事三人,掌固五人,門僕八人,齋郎一百一十人。郊社令掌五郊社稷明堂之位,祠祀祈禱之禮。丞為之
 貳。凡大祭祀,則設神坐於壇上而別其位,立燎壇而先積柴。凡有合朔之變,則置五兵於太社,以硃絲縈之以俟變,過時而罷之。



 諸陵署:令一人,從五品上。錄事一人,府二人,史四人,主衣四人,主輦四人,主藥四人,典事三人,掌固二人。陵戶,乾,橋、昭四百人,獻、定、恭三百人。陵令掌先帝山陵,率戶守衛之。丞為之貳。凡朔望、元正、冬至,皆修享於諸陵。凡功臣密戚陪葬者聽之,以文武分為左右列。
 諸太子陵令各一人,從八品下。丞一人。從九品下。



 太樂署:令一人,從七品下。丞一人,從八品下。



 府三人,史六人。樂正八人,從九品下。典事八人,掌固八人,文武二舞郎一百四十人。太樂令調合鐘律,以供邦國之祭祀享宴。丞為之貳。凡天子宮懸鐘磬,凡三十六虡,鎛鐘十二,編鐘二二,編磬十二,共為三十六架。東方西方,磬虡起北,鐘虡次之。南方北方,磬虡起西,鐘虡次之。鎛鐘在編鐘之間,各依辰位。四隅建鼓,左柷右敔。又設巢、竽、笛、管、篪、塤、系於編鐘之下。偶歌琴、瑟、箏、築,系於編磬之下。其在殿廷前,則加鼓吹十二案,於建鼓之外,羽葆之鼓、大鼓、金鐓、歌簫、笳置於其上。又設登歌鐘、節鼓、瑟、琴、箏、笳於堂上,笙、和、簫、篪於堂下。太子之
 廷,陳軒懸,去其南面鎛鐘、編鐘編磬各三,凡九,設於辰、丑、申之位。三建鼓亦如之。凡宮懸之作,則奏文武舞,事在《音樂志》也。



 凡大宴會,則設十部伎。凡大祭祀、朝會用樂,辨其曲度章服,而分始終之次。在事於太廟,每室酌獻各用舞。事具《音樂志》。



 凡祀昊天上帝及五方《大明》、《夜明》之樂,皆六成,祭皇地祇神州社稷之樂,皆八成,享宗廟之樂,皆九成。其餘祭祀,三成而已。



 五音有成數,觀其數而用之也。凡習樂,立師以教。每歲考其師之課業,為上中下三等,申禮部。十年大校之,量優劣而黜陟焉。凡樂人及音聲人應教習,皆著
 簿籍,核其名數,分番上下。



 鼓吹署:令一人,從七品下。丞三人,從八品下。



 府三人,史六人。樂正四人,從九品下。典事四人,掌固四人。鼓吹令掌鼓吹施用調習之節,以備鹵簿之儀。丞為之貳。凡大駕行幸,鹵簿則分前後二部以統之。法駕則三分減一,小駕則減大駕之半。皇太后、皇后出,則如小駕之例。皇太子之鼓吹,亦有前後二部。親王已下各有差。凡大駕行幸,有夜警晨嚴之制。



 大駕夜警十二部,晨嚴三通。太子諸王公卿已下,警嚴有差。凡合朔之變,則率
 工人設五鼓於太社。太儺,則帥鼓角以助侲子唱之。



 大醫署:令二人,從七品下。丞二人,從八品下。



 府二人,史四人,主藥八人,藥童二十四人。醫監四人,從八品下。醫正八人,從九品下。



 藥園師二人,藥園生八人,掌固四人。太醫令掌醫療之法。丞為之貳。其屬有四,曰:醫師、針師、按摩師、禁咒師。皆有博士以教之。其考試登用,如國子之法。凡醫師、醫工、醫正療人疾病,以其全多少而書之以為考課。藥園師,以時種蒔收採。



 諸藥醫博士一人,正八品上。助教一人,從九品下。醫師二十人,醫工一百人,醫生四十人,典藥二人。博士掌以醫術教授諸生。醫術,謂習《本草》,《甲乙脈經》。分而為業,一曰體療,二曰瘡腫,三曰少小,四曰耳目口齒,五曰角法也。



 針博士一人,從八品下。針助教一人,從九品下。針師十人,針工二十人,針生二十人。針博士掌教針生以經脈孔穴,使識浮沉澀滑之候,又以九針為補瀉之法。其針名有九,應病用之也。



 按摩博士一人,從九品下。按摩師四人,按摩工十六人,按摩
 生十五人。按摩博士掌教按摩生消息導引之法。



 咒禁博士一人,從九品下。咒禁師二人,咒禁工八人,咒禁生十人。咒禁博士掌教咒禁生以咒禁,除邪魅之為厲者。



 太卜署:令一人,從八品下。丞一人,正九品。卜正二人,從九品下。卜博士二人。從九品下。太卜令掌卜筮之法。丞為之貳。



 其法有四:一龜,二五兆,三易,四式。皆辨其象數,通其消息,所以定吉兇焉。凡國有祭祀,則率卜正、占者,卜日於太廟南門之外。歲季冬
 之晦,帥侲子入宮中堂贈大儺。



 贈,送也,堂中舞侲子,以送不祥也。



 廩犧署:令一人,正八品下。丞一人。正九品。廩犧令掌薦犧牲及粢盛之事。丞為之貳。凡三祀之牲牢,各有名數。大祭祀,則與太祝以牲就榜位,太常卿省牲,則北面告腯,乃牽牲以授太官。



 汾祠署:令一人,從七品下。丞一人。從八品上。汾祠令、丞,掌神祀、享祭、灑掃之制。



 兩京齊太公廟署:令各一人,從七品下。丞各一人,從八品上。令、
 丞掌開合、灑掃及春秋仲釋尊之禮。



 光祿寺秦曰郎中令,漢曰光祿勛,掌宮殿門戶。梁置十二卿,加「寺」字,除「勛」字,曰光祿卿,掌膳食,後因之。品第三。龍朔改為司膳寺正卿,光宅改為司膳寺卿,神龍復為光祿寺也。



 卿一員,從三品。少卿二人。從四品上。卿之職,掌邦國酒醴、膳羞之事,總太官、珍羞、良醖、掌醢之屬,修其儲備,謹其出納。少卿為之貳。國有大祭祀,則省牲獲,視濯滌。若三公攝祭,則為之終獻。朝會宴享,則節其等差,量其豐約以供焉。



 丞二人,從六品上。主簿二人,從七品上。錄事二人,從九品上。府十二人,
 史二十一人,亭長六人,掌固六人。丞掌判寺事。主簿掌印,勾檢稽失。錄事掌受事發辰。



 太官署:令二人,從七品下。丞四人,從八品下。府四人,史人。監膳十人,從九品下。主膳十五人,供膳二千四百人,掌固四人。太官令掌供膳食之事。丞為之貳。凡祭之日,與卿詣廚省牲鑊,取明水於陰鑒,取明火於陽燧,帥宰人以鑾刀割牲,取其毛血,實之於豆,遂烹牲焉。又師進饌者實簠簋,設於饌幕之內。凡朝會宴享,九品已上並供其膳食。
 凡供奉祭祀致齋之官,則視其品秩為之差降。國子監釋奠,百官觀禮,亦如之。凡突衛當上,及命婦朝參宴會者,亦如之。



 珍羞署:令一人,正八品下。丞二人,正九品下。府三人,史六人,典書八人,餳匠五人,掌固四人。令掌庶羞之事,丞為之貳,以實籩豆。陸產之品,曰榛慄脯修,水物之類,曰魚鹽菱芡。辨其名數,會其出入,以供祭祀朝會賓客之禮也。



 良醖署:令二人,正八品下。丞二人,正九品下。府三人,史六人。監事
 二人,從九品下。掌醖三十人,酒匠十三人,奉觶一百二十人,掌固四人。令掌供奉邦國祭祀五齊三酒之事。丞為之貳。



 五齊三酒,義見《周官》。郊祀之日,帥其屬以實樽罍。若享太廟,供其鬱鬯之酒,以實六彞。若應進者,則供春暴、秋清、酴累、桑落等酒。



 掌醢署:令一人,正八品下。丞二人,正九品下。府二人,史四人,主醢十人。令掌供醯醢之屬,而辨其名物。丞為之貳。凡鹿、兔、羊、魚等四醢。



 凡祭神祇,享宗廟,用菹醢以實豆。宴賓客,會百官,
 醢醬以和羹。



 衛尉寺秦置衛尉,掌宮門衛屯兵,屬官有公車司馬、衛士、旅賁三令。梁置十二卿,衛尉加「寺」字,官加「卿」字。龍朔改為司衛寺,咸亨復也。



 卿一員,從三品。少卿二人。從四品上。卿之職,掌邦國器械文物之事,總武庫、武器、守宮三署之官屬。少卿為之貳。凡天下兵器入京師者,皆籍其名數而藏之。凡大祭祀大朝會,則供其羽儀、節鉞、金鼓、帷帟、茵席之屬。



 丞二人,從六品上。主簿二人,從七品上。錄事一人,從九品上。府六人,史
 十一人,亭長四人,掌固六人。丞掌判寺事,辨器械出納之數。主簿掌印,勾檢稽失。錄事掌受事發辰。



 武庫:令、兩京各一人,從六品下。丞二人,從八品下。府二人,史六人,監事一人,正九品上。典事二人,掌固五人。令掌藏邦國之兵仗、器械,辨其名數,以備國用。丞為之貳。凡親征及大田巡狩,以羝羊、猳豬、雄雞釁鼓。若太子親征及大將出師,則用猳。



 凡有赦,則先建金雞,兼置鼓於宮城門之右。視大理及府縣囚徒至,則撾其鼓。



 武器署:令一人,正八品下。丞二人,從九品下。



 府二人,史六人,監事一人,從九品下。典事二人,掌固四人。令掌在外戎器,辨其名物,會其出入。丞為之貳。凡大祭祀大朝會及巡幸,則納於武庫,供其鹵簿。若王公百官婚葬之禮,應給鹵簿,亦供之。



 守宮署:令一人,正八品下。丞二人,正九品下。



 府二人,史四人,監事二人,掌設六人,幕士一千六百人。令掌邦國供帳之屬,辨其名物,會其出入。丞為之貳。凡大祭祀大朝會及巡
 幸,則設王公百官位於正殿南門外。



 宗正寺《星經》有宗正星,在帝座之東南。秦置宗正,掌宗屬。梁置十二卿,宗正為一,署加「寺」字。隋品第二。光宅改為司屬,神龍復之也。



 卿一員,從三品上。少卿二員。從四品上。丞二人,從六品上。主簿一人,從七品上。



 錄事一人,從九品上。府五人,史九人,亭長四人,掌固四人。卿之職,掌九族六親之屬籍,以別昭穆之序,並領崇玄署。少卿為之貳。九廟之子孫,繼統為宗,餘曰族。凡大祭祀及冊命朝會之禮,皇親諸親應陪位預會者,則為
 之簿書,以申司封。若皇親為三公子孫應襲封者,亦如之。丞掌判寺事。主簿掌印及勾檢稽失。錄事掌受事發辰。



 崇玄署:令一人,正八品下。丞一人,正九品下。府二人,史三人,典事六人,掌固二人。令掌京都諸觀之名數、道士之帳籍,與其齋醮之事。丞為之貳。



 太僕寺太僕,古官。梁置十二卿,署加「寺」字,後因之。龍朔改為司馭寺,光宅為司僕寺,神龍復也。



 卿一員。從三品。古有太僕正,即其名也。後無正字,唯名太僕。梁置為列卿,隋品第三。龍朔為司馭正卿,光宅曰司僕卿,神龍復也。



 少卿二人。從四品上。卿之職,掌邦國廄牧、車
 輿之政令,總乘黃、典車之屬。凡監牧羊馬所通籍帳,每歲則受而會之,以上尚書駕部,以議其官吏之考課。凡四仲之月,祭馬祖、馬步、先牧、馬社。



 丞四人,從六品上。主簿二人,從七品上。錄事二人,從九品上。府十七人。史三十四人,獸醫六百人,獸醫博士四人,學生一百人,亭長四人,掌固六人。丞掌判寺事。主簿掌印,勾檢稽
 失,省署抄目。錄事掌受事發辰。



 乘黃署:令一人,從七品下。丞一人,從八品下。府一人,史二人,典事八人,駕士一百四十人,羊車小吏十四人,掌固六人。令掌天子車輅,辨其名數與馴馭之法。丞為之貳。凡乘輿五輅,事具《輿服志》也。皆有副車,又有十二車,曰指南車、曰記里鼓車、白鷺車、鑾旗車、闢惡車、皮軒車、耕根車、安車、四望車、羊車、黃鉞車、豹尾車,其車飾見《輿服志》也。



 屬車一十有二。古者屬車八十一乘,皇朝置十二乘也。乘輿有大駕、法駕、小駕,車服各有名數之差。若有大禮,則以所御之輅進內。既事,則
 受而藏之。凡將有事,先期四十日,尚乘供馬如輅色,率駕士預調習指南等十二車。



 典廄署:令二人,從七品下。丞四人,從八品下。府二人,史六人。主乘六人,正九品下。典事八人,執馭一百人,駕士八百人,掌固六人。令掌系飼馬牛,給養雜畜之事。丞為之貳。



 典牧署:令二人,正八品下。丞四人,正九品下。府四人,史八人,監事八人,典事十六人,從九品下。主酪五十人。令掌牧雜畜,造酥酪脯臘給納之事。丞為之貳。凡群牧所送羊犢,皆受
 之,而供廩犧、尚食之用。諸司合供者,亦如之。



 車府署:令一人,正八品下。丞一人,正九品下。府一人,史二人,典事四人,掌固六人。令掌王公已下車輅,辨其名數及馴馭之法。丞為之貳。凡公已下,四軺車。



 一象輅,二革軺,三木輅,四軺輅。視其品秩而給之。兼給馭士也。



 上牧監一人,從五品下。牧監,皆皇朝置也。副監二人,正六品下。丞二人,正八品上。主簿一人,正八品下。錄事一人,府三人,史六人,典事八人,掌固四人。



 中牧監一人,正六品下。副監一人,從六品下。丞一人,從八品下。主簿一人,從九品下。錄事一人,府二人,史四人,典事四人,掌固四人。



 下牧監一人,從六品下。副監一人,正七品下。丞一人,正九品上。主簿一人,從九品下。諸牧監掌群牧孳課之事。凡馬五千匹為上監,三千匹已上為中監,一千匹已上為下監,凡馬之群,有牧長尉。凡馬,有左、右監,以別其粗良,以數紀名,著之簿籍。細馬稱左,粗馬稱右。凡諸群牧,立南北東西四使以分統之,其馬皆印。每年終,監牧使巡按孳數,以功過
 相除,為之考課。



 沙苑監一人,從六品下。副監一人,正七品下。丞一人,正九品下。主簿二人,從九品下。錄事一人,府三人,史六人,典事四人,掌固二人。沙苑監,掌牧養隴右諸牧牛羊,以供其宴會祭祀及尚食所用。每歲與典牧分月以供之。丞為之貳。若百司應供者,則四時皆供。凡羊毛及雜畜毛皮角,皆具數申有司。



 大理寺古謂掌刑為士,又曰理。漢景帝加「大」字,取天官貴人之牢曰大理之義。後漢後,改為廷尉,魏復
 為大理。南朝又名廷尉,梁改名秋卿,北齊、隋為大理,加「寺」字。龍朔改為詳刑寺,光宅為司刑,神龍復改。



 卿一員,從三品。古或名廷尉,北齊加「寺」字。隋品第三。龍朔為詳刑正卿,光宅為司刑卿,神龍復為大理卿。少卿二員。從四品上。



 卿之職,掌邦國折獄詳刑之事。少卿為之貳。凡犯至流死,皆詳而質之,以申刑部。仍於中書、門下詳覆。凡吏曹補署法官,則與刑部尚書、侍郎議其人可否,然後注擬。



 正二人,從五品下。丞六人,從六品上。主簿二人,從七品上。錄事二人,從九品上。府二十八人,史五十六人。正掌參議刑闢,詳正科條之事。凡六丞
 斷罪不當,則以法正之。丞掌分判寺事。主簿掌印,省署抄目,勾檢稽失。錄事掌受事發辰。獄丞四人,掌率獄吏,檢校囚徒,及枷杖之事。獄史六人,亭長四人,掌固八人。問事一百四十八人,掌決罪人。司直六人,從六品上。評事十二人,從八品下。掌出使推核。評事史十四人。其刑法科目,已載於刑部。



 鴻臚寺周曰大行人,秦曰典客,漢景帝曰大行,武帝曰大鴻臚。梁置十二卿,鴻臚為冬卿,去「大」字,署為寺。後周曰賓部,隋曰鴻臚寺。龍朔改為同文寺,光宅曰司賓寺,神龍復也。



 卿一員,從三品。少卿二人。從四品上。卿之職,掌賓客及兇儀
 之事,領典客、司儀二署,以率其官屬,供其職務。少卿為之貳。凡四方夷狄君長朝見者,辨其等位,以賓待之。凡二王後及夷狄君長之子襲官爵者,皆辨其嫡庶,詳其可否。若諸蕃人酋渠有封禮命,則受冊而往其國。凡天下寺觀三綱,及京都大德,皆取其道德高妙、為眾所推者補充,申尚書祠部。皇帝太子為五服之親及大臣發哀臨沛,則贊相焉。凡詔葬大臣,一品則卿護其喪事,二品則少卿,三品丞一人往。皆命司儀,以示禮制。



 丞二
 人,從六品上。主簿一人,從七品上。錄事二人,從九品上。府五人,史十一人,亭長四人,掌固六人。丞掌判寺事。主簿掌印,勾檢稽失。錄事掌受事發辰。



 典客署:令一人,從七品下。丞二人,從八品下。掌客十五人,正九品上。典客十三人,府四人,史八人,賓僕十八人,掌固二人。典客令掌二王後之版籍及四夷歸化在蕃者之名數。丞為之貳。凡朝貢、宴享、送迎,皆預焉。辨其等位,供其職事。凡酋渠首領朝見者,皆館供之。如疾病死喪,量事給之。
 還蕃,則佐其辭謝之節。



 司儀署:令一人,正八品下。丞一人,正九品下。司儀六人,府二人,史四人,掌設十八人,齋郎三十三人,掌因四人,幕士六十人。司儀令掌兇禮之儀式及喪葬之具。丞為之貳。凡京官職事三品已上、散官二品已上、京官四品已上,如遭喪薨卒,量品贈祭葬,皆供給之。



 司農寺漢初治粟內史,景帝改為大農,武帝加「司」字。梁置十二卿,以署為寺,以官為卿,隋為司農卿,龍朔二年改為司稼卿,咸亨復也。



 卿一員,從三品上。少卿二員。從四品上。卿之職,掌邦國倉儲委積之事,總上林、太倉、鉤盾、導官四署
 與諸監之官屬,謹其出納。少卿為之貳。凡京百司官吏祿給及常料,皆仰給之。孟春藉田祭先農,則進耒耜,季冬藏冰,仲春頒冰,皆祭司寒。



 丞六人,從六品上。主簿二人,從七品上。錄事二人,從九品上。府二十分人,史七十六人,計史三人,亭長九人,掌固七人。丞掌判寺事。凡天下租及折造轉運於京都,皆閱而納之,以供國用,以祿百官。主簿掌印,署抄目,勾檢稽失。凡置木契二十隻,應須出納,與署合之。
 錄事掌受事發辰。



 上林署:令二人,從七品下。丞四人,從八品下。府七人,史十四人,監事十九人,典事二十四人,掌固五人。令掌苑囿園池之事。丞為之貳。凡植果樹蔬,以供朝會祭祀。其尚食所進,及諸司常料,季冬藏冰,皆主之。



 太倉署:令三人,從七品下。丞二人,從八品下。府十人,史二十人,監事十人。從九品下。



 令掌九穀廩藏。丞為之貳。凡鑿窖置屋,皆銘磚為庾斛之數,與其年月日,受領粟官吏姓名。又立牌如其銘。



 鉤盾署:令二人,正八品上。丞四人,正九品上。



 府七人,史十四人,監事十人,從九品下。典事十九人,掌固五人。令掌供邦國薪芻之事。丞為之貳。凡祭祀、朝會、賓客享宴,隨差降給之。



 導官署:令二人,正八品上。丞四人,正九品上。



 府八人,史十六人,監事十人。從九品上。令掌導擇米麥之事。丞為之貳。凡九穀之用,隨其精粗、差其耗損而供之。



 太原、永豐、龍門諸倉:每倉監一人,正七品下。丞二人,從八品上。



 錄事一人,典事六人,府二人,史四人,掌固四人。倉監掌倉窖
 儲積之事。丞為之貳。凡出納帳紙,歲終上於寺司。



 司竹監:監一人,正七品下。副監一人,正八品下。



 丞二人,從八品上。錄事一人,府二人,史四人,典事三十人,掌固四人。司竹監掌植養園竹。副監為之貳。歲終,以竹功之多少為考課。



 溫泉監:泉在京兆府昭應縣之西。監一人,正七品下。



 丞二人,從八品上,錄事一人,府二人,史二人,掌固四人。溫泉監掌湯池宮禁之事。丞為之貳。凡王公已下至於庶人,湯泉館有差,別其貴賤,而禁其逾越。凡近湯之地,潤澤所及,瓜果之屬先
 時而毓者,必苞匭而進之,以薦陵廟。



 京、都苑總監:監各一人,從五品下。副監一人,從六品下,丞二人,從七品下。主簿一人,從九品上。錄事各三人,府八人,史十六人,亭長四人,掌固六人。苑總監掌宮苑內館園池之事。副監為之貳。凡禽魚果木,皆總而司之。凡給總監及苑內官屬,人畜出入,皆為差降之數。



 京、都苑四面監;監各一人,從六品下。副監一人,從七品下。



 丞二人,正八品下。錄事一人,府三人,史三人,典事六人,掌固四人。
 四面監掌所管面苑內宮館園池,與其種植修葺之事。副監為之貳。丞掌判監事。



 諸屯:監一人,從七品下。丞二人。從八品下。



 諸屯監各掌其屯稼穡。丞為之貳。凡每年定課有差。



 九成宮總監:監一人,從五品下。副監一人,從六品下,丞一人,從七品下。主簿一人,從九品下。錄事一人,府三人,史五人。宮監掌檢校宮樹,供進煉餌之事。副監為之貳。



 太府寺《周官》有太府下士,掌財賦。秦、漢已後,財賦屬司農少府。梁始置太府卿,掌帑藏。龍朔改為外府,
 光宅改為司府,神龍復為太府寺也。



 卿一員,從三品。即後周太府中大夫。少卿二員。從四品上。



 卿掌邦國財貨,總京師四市、平準、左右藏、常平八署之官屬,舉其綱目,修其職務。少卿為之貳。以二法平物。一曰度量,二曰權衡。凡四方之貢賦,百官之俸秩,謹其出納,而為之節制焉。凡祭祀,則供其幣。



 丞四人,從六品上。主簿二人,從七品上。



 錄事二人,從九品上。府十五人,史五十人,計史四人,亭長七人,掌固七人。丞掌判寺
 事。凡正、至大朝所貢方物,應陳於殿廷者,受而進之。



 兩京都市署:京師有東西兩市,東都有南北兩市。令一人,從六品上。



 丞各二人,正八品上。錄事一人,府三人,史七人,典事三人,掌固一人。京、都市令掌百族交易之事。丞為之貳。凡建標立候,陳肆辨物,以二物平市,謂秤以格,斗以概。以三賈均市。賈有上、中、下之差。



 平準署:令二人,從七品下。丞四人,從八品下。錄事一人,府六人,史十三人,監事二人,從九品下。典事二人,價人十人,掌固十人。平準令掌供官市易之事。丞為之貳。凡百司不任用
 之物,則以時出貨。其沒官物,亦如之。



 左藏署:左右藏令,晉始有之,後代因之。皇家左藏,有東庫、西庫、朝堂庫。又有東都庫。各木契一,與太府主簿合也。令三人,從七品下。



 丞五人,從八品下。府九人,史十八人,監事九人,從九品下。典事一人,掌固八人。左藏令掌邦國庫藏。丞為之貳。凡天下賦調,先於輸場簡其合尺度斤兩者,卿及御史監閱,然後納於庫藏,皆題以州縣年月,所以別粗良,辨新舊。凡出給,先勘木契,然後錄其名數,請人姓名,署印送監門,乃聽出。若外給者,以墨印印之。凡藏
 院之內,禁人燃火,及無故入院者。晝則外四面常持仗為之防守,夜則擊柝,而分更以巡警之。



 右藏署:令二人,正八品上。丞三人,正九品上。府五人,史十人,監事四人,從九品下。典事七人,掌固十人。右藏令掌國寶貨。丞為之貳。凡四方所獻金玉、珠貝、玩好之物,皆藏之。出納禁令,如左藏。



 常平署:漢宣帝時,始置常平倉,以平歲之兇穰。後漢改為常滿倉,晉曰常平,後魏曰邸閣倉。隋於衛州置黎陽倉,洛州置河陽倉,陜州置常平倉,華州置廣運倉,轉相委輸,漕關東之粟,以給京師。國家垂拱初,兩京置
 常平署,天下州府亦置之。



 令一人,從七品下。丞二人,從八品下。府四人,史八人,監事五人,從九品下。典事五人,掌固六人。常平令掌倉儲之事。丞為之貳也。



 國子監國子之義,見《周官》。晉武始立國子學。北齊曰國子寺,隋初曰學,後改為寺,大業三年改為監。龍朔曰大司成,光宅曰成均,神龍復為國子監也。



 祭酒一員,從三品。《周官》曰師氏、保氏。漢始置祭酒博士,歷代因之。隋祭酒,品第三。龍朔、光宅,隨曹改易。司業二員。從四品下。隋大業三年,始置司業一人,從四品。官名隨曹改易。



 祭酒、司業之職,掌邦國儒學訓導之政令,有六學。一國子學、二太學、三
 四門、四律學、五書學、六算學也。凡春秋二分之月,上丁釋奠於孔宣父,祭以太牢,樂用登歌軒懸。祭酒為初獻,司業為亞獻。凡教授之經,以《周易》、《尚書》、《周禮》、《儀禮》、《禮記》、《毛詩》、《春秋左氏傳》、《公羊傳》、《穀梁傳》各為一經,《孝經》、《論語》兼習之。每歲終,考其學官訓導功業之多少,為之殿最。



 丞一人,從六品下。主簿一人,從七品下。錄事一人,從九品下。府七人,史十三人,亭長六人,掌固八人。丞掌判監事。凡六學生每歲有業成上於監者,以其業與祭酒、司業試所習業,
 上尚書禮部。



 國子博士二人,正五品上。助教二人,從六品上。學生三百人,典學四人,廟幹二人,掌固四人。博士掌教文武官三品已上、國公子孫,二品已上曾孫為生者。生初入,置束帛一篚,酒一壺,修一案。每歲生有能通兩經已上求出仕者,則上於監。堪秀才進士者,亦如之。典學掌抄錄課業。廟干掌灑掃學廟。



 太學博士三人,正六品上。助教三人,從七品上。學生五百人。太
 學博士掌教文武五品已上及郡縣公子孫,從三品曾孫之為生者。教法並如國子。



 四門博士三人,正七品上。助教三人,從八品上。四門博士掌教文武七品已上及侯伯子男子之為生者,若庶人子為俊士生者,教法如太學。學生五百人。直講四人,掌佐博士助教之職。大成二十人。



 通四經業成,上於尚書吏部,試登第者,加階放選也。



 律學博士一人,從八品下。太宗置。助教一人,從九品上。學生五十人。博士掌教文武官八品已下及庶人子為生者。以律令
 為專業,格式法例亦兼習之。



 書學博士二人,從九品下。學生三十人。博士掌教文武八品已下及庶人之子為生者。以《石經》、《說文》、《字林》為專業,餘字書兼習之。



 算學博士二人,從九品下。學生三十人。博士掌教文武官八品已下及庶人之子為生者。二分其經,以為之業。習《九章》、《海島》、《孫子》、《五曹》、《張邱建》、《夏侯陽》、《周髀》十五人,習《綴術》、《緝古》十五人。其《紀遺》、《三等數》亦兼習之。



 《
 五經》博士各一人。五品下。舊無《五經》學科,自貞元五年一月敕特置《三禮》《開元禮》科,長慶二年二月,始置《三傳》《三史》科,後又置《五經》博士。檢年月,未獲也。



 廣文館博士二人。正六品上。天寶九載置,試附監修進士業者。置助教一人,至德後廢也。



 少府監秦置少府,掌山澤之稅。漢掌內府珍貨。梁始為卿。歷代或置或省。隋大業五年,始分太府置少府監。龍朔改為內府,光宅改為尚方,神龍復為少府監。



 監一員,從三品。秦、漢有少府,梁始為卿,隋改為監,從三品,少監,從四品。煬帝改為令,武德復為監,龍朔、光宅,隨曹改易之。少監二員。



 從四品下。監之職,掌供百工伎巧之事,總中尚、左尚、右尚、織染、掌冶五署之官屬,庀其工徒,
 謹其繕作。少監為之貳。凡天子之服御,百官之儀制,展採備物,皆率其屬以供之。



 丞四人,從六品下。主簿二人,從七品下。錄事二人,從九品上。府二十七人,史十七人,計史三人,亭長八人,掌固四人。丞掌判監事。凡五署所修之物,則申尚書省,下所司,以供給焉。



 中尚署:令一人,從六品下。丞四人,從八品下。府九人,史十八人,監作四人,典事四人,掌固四人。中尚令,掌供郊祀之圭璧、器玩之物。中宮服飾,雕文錯彩之制,皆供之。丞為之
 貳。其所用金玉齒革毛羽之屬,任土以時,而供送之。



 左尚署:令一人,正七品下。丞五人,從七品下。監作六人,從九品下。典事十八人,掌固四人。左尚令掌供天子之五輅、五副、七輦、三輿、十有二車、大小方圓華蓋一百五十有六,諸翟尾扇及小傘翰,辨其名數,而頒其制度。丞為之貳。



 右尚署:令一人,正七品下。丞四人,從八品下。監作六人,從九品下。典事十三人,掌固十人。右尚署令供天子十有二閑馬之鞍轡及五品三部之帳,備其材革,而修其制度。丞為之
 貳。凡刀劍、斧鉞、甲胄、紙筆、茵席、履舄之物,靡不畢供。具用綾絹、金玉、毛革等,所出方土,以時支送。



 織染署:令一人,正八品上。丞二人,正九品上。監作六人,從九品下。典事十一人,掌固五人。織染令掌供天子太子群臣之冠冕,辨其制度,而供其職。丞為之貳。



 掌冶署:令一人,正八品上。丞一人,正九品上。監作四人,從九品下。掌冶令掌熔鑄銅鐵器物。丞為之貳。凡天下出銅鐵州府,聽人私採,官收其稅。若白鑞,則官市之。其西北諸州,禁
 人無置鐵冶及採鐵。若器用所須,具名移於所由官供之。



 諸冶:監一人,正七品下。丞二人,從八品下。錄事一人,府一人,史二人,監作四人,從九品下。典事二人,掌固四人。諸冶監掌鑄銅鐵之事。



 北都軍器監一人,正四品上。少監一人,正五品上。丞二人,正七品上。主簿一人,正八品上。錄事一人,從九品上。



 府十人,史十八人,典事四人,亭長二人,掌固四人。軍器監掌繕造甲弩,以時納
 於武庫。



 甲坊署:令一人,正八品下。丞一人,正九品下。府二人,史五人,監作二人,從九品下。典事二人。



 弩坊署:令一人,正八品下。丞一人,正九品下。府二人,史五人,監作二人,從九品下。典事二人。



 諸鑄錢監:絳州三十爐,揚、宣、鄂、蔚四州各十爐,益鄧、郴三州各五爐,洋州三爐,定州一爐也。諸鑄錢監以所在州府都督刺史判之。副監一人,上佐判之。丞一人,判司判之。監事一人,或參軍或縣尉知之。錄
 事、府、史,士人為之。



 諸互市:監各一人,從六品下。丞一人。正八品下。諸互市監掌諸蕃交易馬駝驢牛之事。



 將作監秦置將作,掌營繕宮室,歷代不改。隋為將作寺,龍朔改為繕工監,光宅改為營繕監,神龍復為將作監也。



 大匠一員,從三品。大匠之名,漢景帝置。梁置十二卿,將作為一卿。後周曰匠師中大夫。隋初為將作寺,置大匠一人,又改為監,以大匠為監。煬帝改為令,武德改為大匠。龍朔、光宅,隨曹改易也。



 少匠二員。從四品下。大匠掌供邦國修建土木工匠之政令,總四
 署、三監、百工之官屬,以供其職事。凡兩京宮殿、宗廟、城郭、諸臺省監寺廨宇樓臺橋道,謂之內外作,皆委焉。



 丞四人,從六品下。主簿二人,從七品下。錄事二人,從九品下。府十四人,史二十八人,計史三人,亭長四人,掌固六人。



 左校署:令二人,從八品下。丞四人,正九品下。府六人,史十二人,監作十人。從九品下。左校令掌供營構梓匠。凡宮室樂懸簨弶,兵仗器械,喪葬所須,皆供之。



 右校署:令二人,從八品下。丞三人,正九品下。府五人,史十人,監作
 十人,從九品下。典事十四人。右校令掌供版築、塗泥、丹頠之事。



 中校署:令一人,從八品下。丞三人,正九品下。府三人,史六人,監事四人,從九品下。典事八人,掌固二人。中校令掌供舟車兵仗、廄牧雜作器用之事。凡行幸陳設供三梁竿柱,閑廄供銼碓行槽,祭祀供葛竹塹等。



 甄官署:令一人,從八品下。丞二人,正九品下。府五人,史十人,監作四人,從九品下。典事十八人。甄官令掌供琢石陶土之事。
 凡石磬碑碣、石人獸馬、碾磑專瓦、瓶缶之器、喪葬明器,皆供之。



 百工、就穀、庫谷、斜谷、太陰、伊陽等監:百工監在陳倉,就穀監在王屋,庫谷監在鄠縣,太陰監在陸渾,伊陽監在伊陽,皆在出材之所。



 監各一人,從七品下。丞一人,正八品下。府各一人,史三人,典事各二十一人,錄事各一人,監事四人。從九品下。百工等監掌採伐材木。



 都水監



 都水監:使者二人,正五品上。漢官有都水長,屬主爵,掌諸池沼,後改為使者,後漢改為河堤
 謁者。晉復置都水臺,立使者一人,掌舟楫之事。梁改為太舟卿,北齊亦曰都水臺。隋改為都水監,大業復為使者,尋又為監,復改監為令,品第三。武德復為監,貞觀改為使者,從六品。龍朔改為司津監,光宅為水衛都尉,神龍復為使者,正五品上,仍隸將作監。



 丞二人,從七品上。主簿二人,從八品下。錄事一人,府五人,史十人,掌固三人。使者掌川澤津梁之政令,總舟楫、河渠二署之官屬,凡虞衡之採捕,渠堰陂池之壞決,水田斗門灌溉,皆行其政令。



 舟楫署:令一人,正八品下。丞二人。正九品下。舟楫署令掌公私舟船運漕之事。



 河渠署:令一人,正八品下。丞一人,正九品上。府三人,史六人。河堤謁者六人,掌修補堤堰漁釣之事。典事三人,掌固四人,長上漁師十人,短番漁師一百二十人,明資漁師一百二十人。河渠令掌供川澤魚醢之事。祭祀則供魚醢。諸司供給魚及冬藏者,每歲支錢二十萬,送都水,命河渠以時價市供之。



 諸津:令一人,正九品上。丞一人。從九品下。津令各掌其津濟渡舟梁之事。



 武官



 左右衛周制:軍萬二千五百人。天子六軍,大國三軍,次國二軍,小國一軍。軍將皆命卿。至秦、漢,始置衛將軍,後漢、魏因之。晉武帝始置左、右、中三衛將軍。至隋始置左右衛、左右武衛、左右候、左右領軍、左右率府。各有大將軍一人,謂十二衛大將軍也。國家因之。



 大將軍各一員,正三品。將軍各二員。從三品。左右衛將軍之職,掌統領宮廷警衛之法,以督其屬之隊仗,而總諸曹之職務。凡親勛翊五中郎將府及折沖府所隸,皆總制之。凡宿衛,內廊閣門外,分為五仗,一供奉仗、二親仗、三勛仗、四翊仗、五
 散手仗也。皆坐於東西廊下。若御坐正殿,則為黃旗仗,分立於兩階之次,在正門之內,以挾門隊坐於東西廂。皆大將軍守之。



 長史各一人,從六品上。錄事參軍事各一人,正八品上。倉曹、兵曹參軍各二人,正八品下。騎曹、胄曹參軍各一人,正六品下。



 司階二人,正六品上。中候三人,正七品下。司戈五人,正八品下。執戟五人,正九品下。奉車都尉五人。從五品下。長史掌判諸曹、親勛翊五府及武安、武成等五十府之事。諸曹參軍皆掌本曹勾檢之
 事。



 隨曹各有府史。



 親府、勛一府、勛二府、翊一府、翊二府等五府每府中郎一人、中郎將一人,皆四品下。左右郎各一人,正五品上。錄事一人,兵曹參軍事一人,正九品上。校尉五人,旅帥十人,隊正二十人,副隊正二十人。中郎將領本府之屬以宿衛。左右郎將貳之。若大朝會、巡幸,以鹵簿之法以領其儀仗。



 左、右驍衛古曰驍騎,隋改左、右備身為左右驍衛,所領名豹騎,國家去「騎」字曰驍衛府,龍朔去「府」字,
 改為左、右武威,神龍復為驍衛。



 大將軍各一員,正三品。將軍各二員。從三品。驍衛將軍之職,掌如左、右衛。大朝會在正殿之前,則以黃旗隊及胡祿隊坐於東西廊下。若御坐正殿,則以其隊仗次立左、右衛下。



 長史、錄事參軍、倉兵騎胄四曹參軍、員數、品秩如左右衛。司階、中候、司戈、執戟等,四色人數、品秩如左、右衛也。校尉、旅帥、隊正、副隊、(人數如左右衛。
 翊府中郎、中郎將、左右中郎將、左、右郎將。職掌如左右衛。



 左右武衛魏武為丞相,有武衛營。隋採其名,置左右武衛府,有大將軍。光宅改為左右鷹揚衛,神龍復也。



 大將軍各一員,正三品。將軍各二員。從三品。其職掌如左、右衛。大朝會,被白鎧甲,執器盾及旗等,蹕稱長唱。警持鈒隊,應蹕為左、右廂儀仗。在正殿前,則以諸隊次立於驍衛之下。



 長史、錄事參軍、倉兵騎胄四曹參軍、司階、中候、司戈、執
 戟、人數、品秩皆如左、右衛也。翊府中郎將、左右郎將、錄事、兵曹。



 人數、品秩如左、右衛。



 左右威衛隋為左、右屯衛,龍朔改為威衛,光宅改為左、右豹韜衛,神龍復為威衛也。



 大將軍各一員,正三品。將軍各二員。從三品。其職掌,大朝會則被黑甲鎧,弓箭刀盾旗等,分為左、右廂隊,次武衛之下。



 長史、錄事參軍、倉兵騎胄四曹參軍、職掌、人數、品秩皆如左、右衛也。司階、中候、司戈、執戟、人數、品秩如左右衛。翊府中郎將、左右郎將、錄
 事、兵曹、校尉、旅帥、隊正、副隊正。



 人數、品秩皆如左右衛之親府。



 左右領軍衛漢建安中,魏武為丞相,始置中領軍,後因之。北齊置領軍府,後因之。煬帝改為屯衛,國家改為領軍衛。龍朔改為戎衛,光宅改為玉鈐衛,神龍後為領軍衛。



 大將軍各一員。正三品。將軍各二員。從三品。其職掌,大朝會則被青甲鎧,弓箭刀盾旗等,分為左右廂儀仗,次立威衛之下。



 長史、錄事參軍、倉兵騎胄四曹參軍、司階、中候、司戈、執戟、人數、品秩如左右衛。翊府中郎將、左右郎將、錄事、兵曹、校尉、旅
 帥、隊正、副隊正。



 人數、品秩、掌如左右衛也。



 左右金吾衛秦曰中尉,掌徼巡,武帝改名執金吾,魏復為中尉。,南朝不置。隋曰候衛。龍朔二年改為左、右金吾衛,採古名也。



 大將軍各一員,正三品。將軍各二員。從三品。左右金吾衛之職,掌宮中及京城晝夜巡警之法,以執御非違。凡翊府及同軌等五十府皆屬之。凡車駕出入,則率其屬以清游隊,建白澤硃雀等旗隊先驅,如鹵簿之法。從巡狩畋獵,則執其左、右營衛之禁。凡翊衛、翊府、同軌、寶圖等
 五十府彍騎衛士應番上者,各領所職焉。



 長史、錄事參軍、倉兵騎胄四曹參軍、司階、中候、司戈、執戟、人數、品秩、職掌如左右衛也。翊府中郎將、左右郎將、兵曹、校尉、旅帥、隊正、副隊正。



 品秩、人數、職掌如左右衛也。



 左右監門衛漢、魏曰城門校尉,始置左右監門府,省將軍、郎將等官,國家因之。龍朔二年,去府字為衛。



 大將軍各一員,正三品。將軍各二員,從三品。中郎將四人。正四品下。監門將軍之職,掌宮禁門籍之法。凡京司應入宮
 殿門者,皆有籍。左將軍判入,右將軍判出。若大駕行幸,即依鹵簿法,率其屬於牙門之下,以為監守。中郎將,掌監諸門,檢校出入。



 長史、錄事參軍、兵曹胄曹二曹參軍。品秩如諸衛。



 監門校尉,各三百二十人,立長各六百八十人,長人長上二十人,立長長上各二十人。



 左右千牛衛宋謝綽《拾遣》有千牛刀,即人主防身刀也。後魏有千牛備身,取《莊子》庖刀解牛之義,後代因之。隋置左右千牛備身二十人,掌供御弓箭,備身六十人,掌宿衛侍從。煬帝置備身府,皇家改為千牛
 府。龍朔為左右奉宸衛,神龍復為千牛衛。



 大將軍各一員,正三品。將軍各二員,從三品。中郎將各二人。正四品下。千牛將軍之職,掌宮殿侍衛及供御之儀仗,而統其曹務。凡千牛備身左右,執弓箭以宿衛,主仗守戎服器物。凡受朝之日,則領備身左右升殿,而侍列於御坐之左右。凡親射於射宮,則將軍率其屬以從。凡千牛備身之考課、賜會及祿秩之升降,同京職事官之制。中郎將升殿侍奉。凡侍奉,禁橫過座前者,禁對語及傾身
 與階下人語者,禁搖頭舉手以相招召者。若有口敕,通事舍人承受傳聲階下而不聞者,中郎將宣之。



 長史、錄事參軍、兵胄二曹參軍、人數、品秩同諸衛。司階各二人,正六品上。中候各三人,司戈各五人,執戟各五人,品秩同諸衛。千牛備身十二人,備身左右各二人。



 左右羽林軍漢置南北軍,掌衛京師。南軍,若今諸衛也;北軍,若今羽林軍也。漢武置羽林,名曰建章營騎,屬光祿勛,後更名羽林騎,取六郡良家子,及死事之孤為之。後漢置左右羽林監,南朝因之,後魏、周曰羽林率,隨左右屯衛,所領兵名曰羽林。龍朔二年,置左右羽林軍。



 大將軍各一員,正三品下。將軍各二員。從三品下。羽林將軍統領北衛禁兵之法令,而督攝左右廂飛騎之儀仗,以統諸曹之職。若大朝會,率其儀仗以周衛階陛。大駕行幸,則夾道馳而為內仗。凡飛騎每月番上者,皆據其名歷而配於所職。其飛騎仗或有敕上南衙者,則大將軍承墨敕白移於金吾引駕仗,引駕仗官與監門覆奏,又降墨敕,然後得入。



 長史、錄事參軍、倉兵胄三曹參軍、品秩如諸衛。司階、中候、司
 戈、執戟,如千衛品秩、人數。翊府中郎將、左右郎將、錄事、兵曹、校尉、旅帥、隊正、副隊正。



 人數、品秩如諸衛。



 左右龍武軍初,太宗選飛騎之尤驍健者,別署百騎,以為翊衛之備。天后初,加置千騎,中宗加置萬騎,分為左右營,置使以領之。自開元以來,與左右羽林軍名曰北門四軍。開元二十七年,改為左右龍武軍,官員同羽林軍也。



 大將軍一員,正三品。將軍二員。從三品。



 長史一人,錄事參軍事一人,錄事一人,史二人,倉兵胄三曹參軍事各一人。隨曹有府、史、掌固人數。司階二人,中候三人,
 司戈、執戟各五人,長上各十人。



 右件官員階品、人數、職掌,如羽林軍也。



 左右神武軍至德二年,肅宗在鳳翔置。初,貞觀中置北衛七營,後改為左右羽林軍。皆選才力驍勇者充,每月一營十人為番當上。又置左右龍武軍,皆唐元功臣子弟並外州人。如宿衛兵,分日上下。肅宗在鳳翔,方收京城,以羽林軍減耗,寇難未息,乃別置神武軍,同羽林制度官吏,謂之北衙六軍。又置衙前射生手千餘人,謂之左右英武軍。非六軍之例也。乾元二年十月敕,左右羽林、左右龍武、左右神武官員並升同金吾四衛,置大將軍二人,將軍二人也。



 左右神策軍上元中,以北衙軍使衛伯玉為神策軍節度使,鎮陜州,以拒東寇,以中使魚朝恩為觀軍容使,監伯玉軍。及伯玉入為羽林帥,出為荊南節度使,朝恩專統神策軍,鎮陜。廣德元年,吐蕃犯京師,代
 宗避狄幸陜,朝恩以神策軍迎扈。及永泰元年,吐蕃犯京畿,朝恩以神策兵屯於苑中。自是,神策軍恆以中官為帥。建中末,盜發京師,竇文場以神策軍扈蹕山南。及還京師,賞勞無比。貞元中,特置神策軍護軍中尉,以中官為之,時號兩軍中尉。貞元已後,中尉之權傾於天下,人主廢立,皆出其可否,事見《宦者傳》也。



 大將軍各二員,正三品。貞元二年九月敕,改神策左右廂為左右軍,置大將軍各二人,正三品。將軍各二員。從三品。至貞元三年五月,敕左、右神策將軍各加二員,左、右神武將軍各加一員也。



 神威軍本號殿前射生左右廂,貞元二年九月改殿前左右射生軍,三年四月改為左右神威軍,非六軍之例也。



 大將軍二員,正三品。將軍二員。從三品。職田、俸錢、手力、糧
 料等,同六軍諸衛。



 六軍統軍興元元年正月二十九日敕,左右羽林、左右龍武、左右神武各置統軍一人,秩從二品。



 十六衛上將軍舊無此官。貞元二年九月一日敕:「六軍先有敕,各置統軍一人,十六衛宜各置上將軍一員,秩從二品。其左右衛及左右金吾衛上將軍俸料、隨軍人馬等,並同六軍統軍。其諸衛上將軍,次統軍例支給。」貞元二年九月十三日,六軍十二衛上將軍,並放入宿,已後為例也。



 諸府隋置驃騎、鷹揚等府,凡天下守戍兵,不成軍曰牙,府有上中下也。



 折沖都尉各一人,上府,都尉正四品上。中府,從四品下。下府,正五品下。武德中,採隋折沖、果毅郎將之名,改統軍為折沖都尉,別將為果毅都尉。



 左右果毅都尉各一人,上府,果毅
 從五品下。中府,正六品上。下府從六品下。隋煬帝置果毅郎將,國家置折沖都尉。別將各一人,上府,別將正七品下。中府,從七品上。下府,從七品下。



 長史一人,上府,正七品下。中府,從七品上。下府,從七品下。兵曹參軍一人,上府,從八品下。中府,正九品上。下府,從九品下也。



 錄事一人,校尉五人。每校尉,旅帥二人,每旅帥,隊正、副隊正各二人。諸府折沖都尉掌領五校之屬,以備宿衛,以從師役,總其戎具、資糧、差點、教習之法令。凡衛士,三百人為一團,以校尉領之,以便習騎射者為越騎,餘為步兵。其團,十人為火,火備六馱之馬。每歲十一月,以衛士帳
 上尚書省天下兵馬之數以聞。凡兵馬在府,每歲季冬,折沖都尉率五校之屬以教其軍陣、戰鬥之法也。



 具有教習簿籍。



 東宮官屬



 太子太師、太傅、太保各一員。並從一品。師傅,宮官,南朝不置。後魏、北齊,師傅品第二,號東宮三太。隋品亦第二。武德定令,加從一品也。



 太子少師、少傅、少保各一員。並正二品。三少,亦古官,歷代或置或省。南朝並不置。後魏、北齊置之,品第三,號東宮三少。皇家定令,正二品。



 三師三少之職,掌教
 諭太子。無其人,則闕之。



 太子賓客四員。正三品。古無此官,皇家顯慶元年春始置四員也。掌侍從規諫,贊相禮儀。



 太子詹事一員,正三品。少詹事一員。正四品上。詹事,秦官,掌皇太子宮。龍朔二年改為端尹,天授為宮尹,神龍復也。詹事統東宮三寺十率府之政令。少詹為之貳。凡天子六官之典制,皆視其事而承受之。



 丞二人,正六品上。主簿一人,從七品上。錄事二人,正九品下。令史九人,書令史十八人。丞掌判府事。主簿掌印,檢勾稽。錄事
 掌受事發辰。



 司直一人,正七品上,令史一人,書令史二人,亭長四人,掌固六人。司直掌彈劾宮僚,糾舉職事。太子朝,宮臣則分知東西班。凡諸司文武應參官,每月皆具在否以刺之。



 太子左春坊:左庶子二人,正四品上。中允二人。正五品下。左庶子掌侍從贊相,駁正啟奏。中允為之貳。



 司議郎四人,正六品上。錄事二人,從八品下。主事二人,從九品下。令史七人,書令史十四人。司議郎掌啟奏記注宮內祥瑞,
 宮長除拜薨卒,每年終送史館。



 左諭德一人,正四品下。左贊善大夫五人,正五品上。傳令四人,掌儀二人,贊者四人。左諭德掌諷諭規諫。



 崇文館:貞觀中置,太子學館也。學士,直學士,員數不定。學生二十人,校書二人,從九品下,令史二人,典書二人,搨書手二人,書手十人,熟紙匠三人,裝潢匠五人,筆匠三人。學士掌東宮經籍圖書,以教授諸生。凡課試舉送,如弘文館。校書掌校理四庫書籍。



 司經局:洗馬二人,從五品下。洗馬,漢官,為太子前馬。太子文學三人,正六品。校書四人,正九品。正字二人,從九品上。



 書令史二人,楷書手二十五人,典書四人。洗馬掌四庫圖籍繕寫、刊緝之事。文學掌侍奉文章。校書、正字掌典校四庫書籍。



 典膳局:典膳郎三人,正六品上。丞二人,正八品上。書令史二人,主食六人,典食二百人,掌固四人。典膳郎掌進膳嘗食,每夕局官於廚更直。



 藥藏局:藥藏郎二人,正六品上。丞二人,正八品上。侍醫典藥九人,
 藥童十八人,掌固六人。藥藏郎掌和劑醫藥。



 內直局:內直郎二人,從六品下。丞二人,正八品下。典服三十人,典扇十五人,典翰十五人,裳固六人。內直郎掌符璽、傘扇、幾案、衣服之事。



 典設局:典設郎四人,從六品下,丞二人,正八品下。幕士六百人。典設郎掌湯沐、灑掃、鋪陳之事。凡大祭祀,太子助祭,則於正殿東設幄坐。



 宮門局:宮門郎二人,從六品下。丞二人,正八品下。門僕一百三十
 人。宮門郎掌內外宮門管鑰之事。其鐘鼓刻漏,一如皇居之制也。



 太子右春坊:右庶子二人,正四品下。中舍人二人,正五品上。舍人四人,正六品上。錄事一人,從八品下。主事二人,從九品下。舍人掌行令書令旨及表啟之事。太子通表,如諸臣之禮。諸臣及宮臣上皇太子,大事以箋,小事以啟,其封題皆曰上,右春坊通事舍人開封以進。其事可施行者皆下於坊,舍人開,庶子參詳之,然後進。不可者則否。



 右諭德一人,正四品下。右贊善大夫五人,正五品上。傳令四人,諭德、
 贊善,掌事如左。通事舍人八人,正七品下。典謁二十人。舍人掌導引宮臣辭見及承令勞問之事。



 太子內坊:皆宦者為司局。典內二人,從五品下。錄事一人,典直四人,正九品下。導客舍人六人,閣帥六人,內閽八人,內給使,無員數。內廄二十人,典事二人,駕士三十人。典內掌東宮閣門之禁令,及宮人衣廩賜與之出入。丞為之貳。典直主儀式。導客主儐序。閣帥主門戶。內閽主出入。給使主傘扇。內廄主車輿。典事主牛馬。典內統而監之。



 太子內官:司閨二人,從六品。掌導引妃及宮人名簿,總掌正、掌書、掌筵三司。掌正三人,從八品。掌文書出入,目錄為記。並閣門管鑰,糾察推罰。女史,流外三品。掌典文簿而執行焉。,掌書三人,從八品。掌寶、符契、經簿、宣傳、啟奏、教學、廩賜、紙筆、監印。



 掌筵三人,從八品。掌帷幄、床褥、幾案、傘扇、灑掃、鋪設之事。司禮二人,從六品。掌禮儀參見,以總掌嚴掌縫,掌藏,而領其事。掌嚴三人,從八品。掌首飾、衣服、巾櫛、膏沐、仗衛。掌縫三人,從八品。掌裁縫、織繢。
 掌藏三人,從八品。掌貨貝、珠玉、錦彩。



 司饌二人,從六品。掌膳羞。進食先嘗,總掌食、掌醫、掌園三司,而領其事。掌食三人,從八品。掌膳羞、酒醴、燈燭。掌醫三人,從八品。掌醫主醫藥。掌園三人,從八品。掌園苑樹藝、蔬果。



 太子家令寺:令一人,從四品上。丞二人,從七品上。主簿一人,正九品下。錄事一人。家令掌太子飲膳、倉儲、庫藏之政令,總食官、典倉、司藏三署之官屬。



 食官署:令一人,從八品下。丞二人,從九品下。掌膳十二人,奉觶三十人。
 食官令掌飲膳之事。



 典倉署:令一人,從八品下。丞二人,從九品下。園丞二人,典事六人。典倉令掌九穀入藏,及醯醢、庶羞、器皿、燈燭之事。



 司藏署:令一人,從八品下。丞二人。從九品下。司藏令掌庫藏財貨、出納、營繕之事。



 太子率更寺:令一人,從四品上。丞二人,從七品上。主簿一人,正九品下。錄事一人,伶官師二人,漏刻博士二人,掌漏六人,漏童六十人,典鼓二十四人。率更令掌宗族次序、禮樂、刑
 罰及漏刻之政令。



 太子僕寺:僕一人,從四品下。丞一人,從七品上。主簿一人,正九品下。錄事一人,太子僕掌車輿、乘騎、儀仗之政令及喪葬之禮物,辨其次序。



 廄牧署:令一人,從八品下。丞二人,從九品下。典乘四人,牧長四人,翼馭十五人,駕士三十人,獸醫二十人。廄牧令掌車馬、閑廄、牧畜之事。



 東宮武官



 太子左、右衛率府:秦、漢有太子衛率,主門衛。晉分左、右、中、前四衛率,後代因置左、右率。北齊為衛率坊。隋初始分置左右衛率府、左右宗衛率、左右虞候、左右內率、左右監門率十府,以備儲闈武衛之職。煬帝改為左、右侍率,國家復為衛率。龍朔改為左、右典戎衛,咸亨復。



 率各一員,正四品上。副率各一人,從四品上。左右衛率掌東宮兵仗羽衛之政令,總諸曹之事。凡親勛翊府及廣濟等五府屬焉。凡正、至太子朝,宮臣率其屬儀仗,為左右廂之周衛,出入如鹵簿之法。



 長史各一人,正七品上。錄事參軍事各一人。從八品上。倉曹參軍
 一人,從八品下,兵曹參軍一人,從八品下。胄曹參軍一人,從八品下。司階一人,從六品上。中候二人,從七品下。司戈二人,從八品下。執戟三人。從九品下。長史掌判諸曹及三府五府之貳。錄事掌監印勾稽。官掌本曹簿籍。



 職事皆視上臺。親府勛翊府中郎將各一人,從四品上。左、右郎將各一人,正五品下。錄事一人,兵曹參軍一人,校尉五人,旅帥十人,隊正二十人,副隊正二十人。郎將掌其府之屬以宿衛,而總其事。



 職掌一視上臺親府。



 太子左、右司禦率府:本號左、右宗衛府,龍朔改為司禦率府。率各一人,正四
 品上。副率各二人,從四品上。衛禦率掌同左右率。



 長史、錄事參軍事、倉兵胄三曹參軍、司階、中候、司戈、執戟。人數、品秩、職掌如左右衛府也。



 太子左、右清道率府:隋文置左右虞候府,各開府一人,掌斥候。國初亦為左、右虞候,龍朔改為清道率府,神龍又為虞候,開元復為清道也。



 率各一人,正四品上。副率各二人。從四品上。清道率掌東宮內外晝夜巡警之法。



 長史、錄事參軍事、倉兵胄三曹參軍、司階、中候、司戈、執戟。人數、品秩如左右衛率府。



 太子左右監門率府:隋置此官,國家因之。率各一人,正四品上。副率各一人。從四品上。左右監門率掌東宮禁衛之法,應以籍入宮殿門者,二率司其出入,如上臺之法。



 長史、錄事參軍事、兵胄二曹參軍。監門直長七十八人。人數、品秩同諸率府。



 太子左、右內率府:隋初置內率府,擬上臺千牛衛。龍朔初,為奉裕率,咸亨復。率各一人,正四品上。副率各一人。從四品上。



 左、右內率之職,掌東宮千牛備身侍奉之事,而立其兵仗,總其府事。



 長史、錄事參軍事、兵胄二曹參軍,人數、品秩如諸率。千牛十六人,備身二十
 八人,主仗六十人。



 王府官屬公主邑司。



 親王府:傅一人,從三品。漢官有王傅、太傅,魏、晉後唯置師,國家因之,開元改為傅。諮議參軍一人,正五品上。友一人,從五品下。文學二人,從六品上。東閣、西閣祭酒各一人。從七品上。傅掌傅相贊導,而匡其過失。諮議訏謀左右。友陪侍規諷。文學讎校典籍,侍從文章。祭酒接對賓客。



 長史一人,從四品上。司馬一人,從四品下。掾一人,正六品上。屬一
 人,正六品上。主簿一人,從六品上。史二人,記室參軍事二人,從六品上。錄事參軍事一人,從六品上。錄事一人,從九品上。功倉戶兵騎法士等七曹參軍事各一人,正七品上。參軍事二人,正八品下。



 行參軍四人,從八品。典簽二人。從八品下。長史、司馬統領府僚,紀綱職務。掾統判七曹參軍事。主簿掌覆省王教。記室掌表啟書疏。錄事參軍事勾稽省署鈔目。錄事掌受事發辰。七曹參軍各督本曹事,出使檢校。典簽宣傳教命。



 親王親事府:典軍二人,正五品上。副典軍二人,從五品上。執仗親
 事十六人,執乘親事十六人,親事三百三十三人,校尉、旅帥、隊正、隊副。



 準部內人數多少置。親王帳內府典軍二人,副典軍二人,品秩如親事府。帳內六百六十七人,校尉、旅帥、隊正、隊副。看人數置。典軍、副典軍之職,掌率校尉已下守衛陪從之事。



 親王國:令一人,從七品下。大農二人,從八品下。尉二人,正九品下。丞一人,從九品下。錄事一人,典衛八人,舍人四人,學官長一人,食官長一人,丞一人,廄牧長二人,丞二人,典府長二人,
 丞二人。國令、大農掌通判國事。國尉、國丞掌判國司,勾稽監印事。典衛守居宅。舍人引納。學官教授內人。



 公主邑司:令一人,從七品下。丞一人,從八品下。錄事一人,從九品下。主簿二人,謁者二人,舍人二人,家吏二人。公主邑司官各掌主家財貨出入、田園征封之事。其制度,皆隸宗正寺。



 州縣官員



 京兆河南太原等府:自秦、漢已來為雍、洛、並州。周、隋或置總管都督,通名為府。開元初,乃
 為京兆府、河南府、太原府。三府牧各一員,從二品。牧,古官,舜置十二牧是也。秦以京城守為內史,漢武改為尹。後魏、北齊、周、隋又以京守為牧。武德初,因隋置牧,以親王為之。或不出閣,長史知府事。



 尹各一員,從三品。京城守,秦曰內史,漢曰尹,後代因之。隋為內史。武德初置牧,以長史總府事。開元初,雍、洛、並改為府,乃升長史為尹,從三品,專總府事也。



 少尹各二員,從四品下。魏、晉已下,州府有治中,隋文改為司馬,煬帝改為贊理,又為丞,武德改為治中,永徽避高宗名,改為司馬,開元初,改為少尹。



 司錄參軍二人,正七品。錄事四人,從九品上。功倉戶兵法士等六曹參軍事各二人,正七品下。府史、《隋書》有之。參軍事六人,正八品下。執刀十五人,典獄十一人,問事十二人,白直二十四人。經
 學博士一人,從八品上。助教二人,學生八十人。醫藥博士一人,助教一人,學生二十人。



 大都督府:魏黃初二年,始置都督諸州軍事之名,後代因之。至隋改為總管府。武德四年又改為都督,貞觀中分為上、中、下都督府也。



 都督一員,從二品。長史一人,從三品。司馬二人,從四品下。錄事參軍事二人,正七品上。錄事二人,從九品上。功倉戶兵法士六曹參軍事,功士二曹各一員,餘曹各二員,並正七品下也。典獄十六人,問事十人,白直二十四人,市令一人,從九品上。丞一人,佐一人,史二人,倉督
 二人。經學博士一人,從八品上。助教二人,學生六十人。醫學博士一人,從八品下。助教一人,學生十五人。



 中都督府:都督一員,正三品上。別駕一人,正四品下。長史一人,正五品上。司馬一人,正五品下。錄事參軍事一人,從七品下。錄事二人,從九品上。功倉戶兵法士六曹參軍事各一人,並從七品上。參軍事四人,從八品上。典獄十四人,白直二十人,市令一人,從九品上。丞一人,佐一人,史二人,帥三人,倉督二人。經學博士一人,從八品下。助教二
 人,學生六十人。醫藥博士一人,學生十五人。



 下都督府:都督一員,從三品。別駕一人,從四品下。長史一人,從五品上。司馬一人,從五品上。錄事參軍事一人,從七品上。



 錄事二人,從九品下。功倉戶兵法士六曹參軍事各一人,從七品下。參軍事三人,從八品下。典獄十二人,問事六人,白直十六人,市令一人,從九品上。丞二人,佐一人,史二人,帥二人,倉督二人。經學博士一人,從八品下。助教一人,學生五十人。醫學博士一人,助教一人,學生
 十二人。



 上州:州之名,古也。舜置十二州,《禹貢》九州,漢置十三州。秦並六國,置三十六郡。漢則以州統郡。其後武德改郡為州,改州為郡,事見諸卷。國家制,戶滿四萬以上為上州。



 刺史一員,從三品。秦分天下為三十六郡,郡置守、都尉各一人,仍以御史一人監郡。漢廢監郡御史,丞相遣掾吏分察諸郡。漢武元光五年,分天下置十三州,分統諸郡。每州遣使者一人,督察官吏清濁,謂之十三州刺史。後漢遂以名臣為刺史,專州郡之政,仍置別駕、治中、諸曹掾屬,號曰外置刺史。天寶改州為郡,置太守。乾元元年,改郡為州,州置刺史。初,漢代奉使者皆持節,故刺史臨部,皆持節。至魏、晉,刺史任重者,為使持節
 都督,輕者為持節。後魏、北齊,總官、刺史,則加使持節諸軍事,以此為常。隋開皇三年罷郡,以州統縣,刺史之名存而職廢。而於刺史太守官位中,不落持節之名,至今不改,有名無實也。至德之後,中原用兵,大將為刺史者,兼治軍旅,遂依天寶邊將故事,加節度使之號,連制數郡。奉辭之日,賜雙旌雙節,如後魏、北齊故事。名目雖殊,得古刺史督郡之制也。



 別駕一人,從四品下。長史一人,從五品上。司馬一人,從五品下。錄事參軍事一人,從七品上。錄事三人,從九品上。司功、司倉、司戶、司兵、司法、司士六曹參軍事各一人,並從七品下。參軍事四人,典獄十四人,問事八人,白直二十四人,市令一人,從九品上。丞一人,佐一人,史二人,帥三人,倉督二人。經學
 博士一人,從八品下。助教二人,學生六十人,醫學博士一人,正九品下。助教一人,學生十五人。



 中州:戶滿二萬戶已上,為中州。刺史一員,正四品上。別駕一人,正五品下。長史一人,正六品上。司馬一人,六品上。



 錄事參軍事一人,正八品上。錄事一人,從九品上。司功、司倉、司戶、司法、司士六曹參軍事各一人,並正八品下。隨曹有佐史人數。



 參軍事三人,正九品上。執刀十人,典獄十二人,問事六人,白直十六人,市令一人,丞、佐各一人,史、帥、倉督各二人。經
 學博士一人,正九品上。助教一人,學生五十人。醫藥博士一人,從九品下。助教一人,學生十二人。



 下州:戶不滿二萬,為下州也。刺史一員,正四品下。別駕一人,從五品上。司馬一人,從六品下。錄事參軍事一人,從八品上。



 錄事一人,從九品下。司倉、司戶、司法三曹參軍事各一人,從八品下。隨曹有佐史人數。參軍事一人,從九品下。典獄八人,問事四人,白直十六人,市令一人,佐、史各一人,帥二人,倉督一人。經學博士一人,正九品下。助教一人,學生四十
 人。醫學博士一人,從九品下。學生十人。



 京兆、河南、太原牧及都督、刺史掌清肅邦畿,考核官吏,宣布德化,撫和齊人,勸課農桑,敦敷五教。每歲一巡屬縣,觀風俗,問百年,錄囚徒,恤鰥寡,閱丁口,務知百姓之疾苦。部內有篤學異能聞於鄉閭者,舉而進之。有不孝悌,悖禮亂常,不率法令者,糾而繩之。其吏在官公廉下己,清直守節者,必謹而察之。其貪穢諂諛,求名狥私者,亦謹而察之。皆附於考課,以為褒貶。若善惡殊尤者,隨即奏聞。若獄
 訟疑議,兵甲興造便宜,符瑞尤異,亦以上聞。其常則申於尚書省而已。若孝子順孫,義夫節婦,精誠感通,志行聞於鄉閭者,亦具以申奏,表其門閭。其孝悌力田,頗有詞學者,率與計偕。其所部有須改更,得以便宜從事。若親王典州,及邊州都督刺史不可離州局者,應巡屬縣,皆委上佐行焉。尹、少尹、別駕、長史、司馬掌貳府州之事,以綱紀眾務,通判列曹。歲終則更入奏計。司錄、錄事參軍掌勾稽,省署鈔目,監符印。功曹、司功掌官吏考課、祭
 祀、禎祥、道佛、學校、表疏、醫藥、陳設之事。倉曹、司倉掌公廨、度量、庖廚、倉庫、租賦、徵收、田園、市肆之事。戶曹、司戶掌戶籍、計帳、道路、逆旅、婚田之事。兵曹、司兵掌武官選舉、兵甲器仗、門戶管鑰、烽候傳驛之事。法曹、司法掌刑法。士曹、司士掌津梁、舟車、舍宅、百工眾藝之事。市令掌市厘交易、禁斥非違之事。經學博士掌《五經》,教授諸生。醫藥博士以百藥救民疾病。下至執刀、白直、典獄、佐史,各有其職。州府之任備焉。



 縣令三代之制,五等諸侯,自理其人。周衰,諸侯相侵,大國分置郡邑縣鄙,以聚其人。齊、晉謂之大夫,魯、衛謂之宰,楚謂之公、尹,秦謂之令、長。秦制:萬戶已上為令,秩千石至六百石,減萬戶為長,秩五百石至三百石,皆有丞、尉,秩四百石至二百石也。



 長安、萬年、河南、洛陽、太原、晉陽六縣,謂之京縣。令各一人,正五品上。丞二人,從七品。主簿二人,從八品上。錄事二人,從九品下。佐二人,史四人,尉六人,從八品下。司功、佐三人,史六人。司倉、佐四人,史八人。司戶、佐五人,史十人。司兵、佐三人,史六人。司法、佐五人,史十人。司士,佐四人,史八人,典獄十四人,問事八人,白直十八人。博士一人,助教一人,學
 生五十人。



 京兆、河南、太原所管諸縣,謂之畿縣。令各一人,正六品下。丞一人,正八品下。主簿一人,正九品上。尉二人,正九品下。



 錄事二人,史三人。司功、佐三人,史五人。司倉、佐四人,史七人。司戶、佐四人,史七人,帳史一人。司法,佐四人,史八人。



 典獄十四人,問事四人,白直十人,市令一人。佐一人,史一人,帥二人。經學博士一人,助教一人,學生四十人。



 諸州上縣:令一人,從六品上。丞一人,從八品下。主簿一人,正九品下,尉二人,從九品上。錄事二人,史三人。司戶、佐四人,史七人,帳史一人。司法,佐四人,史
 八人。倉督二人,典獄十人,問事四人,白直十人,市令一人,佐、史各一人,帥一人。



 博士一人,助教一人,學生四十人。



 諸州中縣:令一人,正七品上。丞一人,從八品下。主簿一人,從九品上。尉一人,從九品下。錄事一人,史四人,司戶、佐三人,史五人,帳史一人。司法,佐二人,史六人。倉督一人,典獄八人,問事四人,白直八人。博士一人。助教一人,學生二十五人。



 諸州中下縣:令一人,從七品下。丞一人,正九品下。主簿一人,從九品上。尉一人,從九品下。錄事一人,司戶、佐二人,史三人,帳史一人。



 司法,佐二人,史四人。
 典獄六人,問事四人,白直八人,市令一人。佐、史各一人,帥二人。博士一人,助教一人,學生二十五人。



 諸州下縣:令一人,從七品下。丞一人,正九品下。主簿一人,從九品上。尉一人,從九品下。錄事一人,司戶、佐二人,史四人,帳史一人。



 司法,佐一人,史四人。典獄六人,問事四人,白直八人,市令一人。佐一人,史二人,帥二人也。博士一人,助教一人,學生二十人。



 京畿及天下諸縣令之職,皆掌導揚風化,撫字黎氓,敦四人之業,崇五土之利,養鰥寡,恤孤窮。審察冤屈,躬親獄訟,務知百姓之疾
 苦。



 大都護府:大都護一員,從二品。副都護四人,正四品上。長史一人,正五品上。司馬一人,正五品上。錄事參軍事一人,正七品上。



 錄事二人,從九品上。功曹、倉曹、戶曹、兵曹、法曹五參軍事各一人,並正七品下。參軍事三人。正八品下。



 上都護府:都護一員,正三品。副都護二人,從四品上。長史一人,正五品上。司馬
 一人,正五品上。錄事參軍事一人,正七品下。錄事二人,功曹、倉曹、戶曹、兵曹四參軍事各一人,從七品上。參軍事三人。從八品上。



 都護之職,掌撫尉諸蕃,輯寧外寇,覘候奸譎,征討攜貳。長史、司馬貳焉。諸曹,如州府之職。



 節度使:天寶中,緣邊御戎之地,置八節度使。受命之日,賜之旌節,謂之節度使,得以專制軍事。行則建節符,樹六纛。外任之重,無比焉。至德已後,天下用兵,中原刺史亦循其例,受節度使之號。



 節度使一人,副使一人,行軍司馬一人,判官二人,掌書記一人,參謀,無員數也。隨軍四人。皆天寶後置。檢討未見品秩。



 元帥、都統、招討等使



 元帥。舊無其名。安、史之亂,肅宗討賊,以廣平王為天下兵馬元帥,又以大臣郭子儀、李光弼隨其方面副之,號為副元帥。及代宗即位,又以雍王為之。自後不置。昭宗又以輝王為之也。



 都統。乾元中置,或總三道,或總五道,至上元末省。大中後,討徐州以康承訓,討黃巢以荊南王鐸,皆為都統。



 招討使。貞元末置。自後隨用兵權置,兵罷則停。



 防禦團練使。至德後,中原置節度使。又大郡要害之地,置防禦使,以治軍事,刺史兼之,不賜旌節。上元後,改防禦使為團練守捉使,又與團練兼置防禦使,名前使,各有副使、判官,皆天寶後置,未見品秩。



 諸鎮魏有鎮東、鎮西、鎮南、鎮北四將軍,後代因之。隋因始置鎮將、鎮副之名也



 上鎮:將一人,正六品下。鎮副一人,正七品下。錄事一人,倉曹、兵曹二參軍。從八品下。各有佐史。



 中鎮:將一人,正七品上。鎮副一人,從七品上。錄事一人,兵曹參軍一人。正九品下。



 下鎮:將一人,正七品下。鎮副一人,從七品下。錄事一人,兵曹參軍一人。從九品下。



 諸戍春秋有戍,葵丘之義。東晉、後魏以屯兵守境處為戍,隋因之。



 上戍:主一人,正八品下。戍副一人。從八品下。佐一人,史二人。



 中戍:主一人。從八品下。



 下戍:主一人。正九品下。



 五岳四瀆廟:令各一人,正九品上。齋郎三十人,祝史三人。



 上關:令一人,從八品下。丞二人。正九品下。錄事一人,有府、史、典事。津吏八人。



 中關:令一人,正九品下。丞一人。從九品下。錄事一人,津吏六人。



 下關:令一人,從九品下。津吏四人。關令各有府、史。



 關令掌禁末游,
 伺奸慝。凡行人車馬出入往來,必據過所以勘之。



\end{pinyinscope}