\article{卷五十一 列傳第一 后妃上 高祖太穆皇后竇氏 太宗文德皇后長孫氏 賢妃徐氏 高宗廢后王氏 良娣蕭氏 中宗和思皇后趙氏 中宗韋庶人上官昭容 睿宗肅明皇后劉氏 睿宗昭成皇后竇氏 玄宗廢后王氏 玄宗貞順皇后武氏 玄宗楊貴妃}

\begin{pinyinscope}

 高
 祖太穆皇后竇氏太宗文德皇后長孫氏賢妃徐氏高宗廢後王氏良娣蕭氏中宗和思皇后趙氏中宗韋庶人
 上官昭容睿宗肅明皇後劉氏睿宗昭成皇后竇氏玄宗廢後王氏玄宗貞順皇后武氏玄宗楊貴妃



 三代宮禁之職,《周官》最詳。自周已降,彤史沿革,各載本書,此不備述。唐因隋制,皇后之下,有貴妃、淑妃、德妃、賢妃各一人,為夫人,正一品;昭儀、昭容、昭媛、修儀、修容、修媛、充儀、充容、充媛各一人,為九嬪,正二品;婕妤九人,正三品;美人九人,正四品;才人九人,正五品;寶林二十七
 人,正六品;御女二十七人,正七品;採女二十七人,正八品;其餘六尚諸司,分典乘輿服御。龍朔二年,官名改易,內職皆更舊號。咸亨二年復舊。開元中,玄宗以皇后之下立四妃,法帝嚳也。而後妃四星,一為正後;今既立正後,復有四妃,非典法也。乃於皇后之下立惠妃、麗妃、華妃等三位,以代三夫人,為正一品;又置芳儀六人,為正二品;美人四人,為正三品;才人七人,為正四品;尚宮、尚儀、尚服各二人,為正五品;自六品至九品,即諸司諸典
 職員品第而序之,後亦參用前號。



 然而三代之政,莫不以賢妃開國,嬖寵傾邦。秦、漢已還,其流浸盛。大至移國,小則臨朝,煥車服以王宗枝,裂土壤而侯肺腑,洎末塗淪敗,赤族夷宗。高祖龍飛,宮無正寢,而婦言是用,釁起維城。大帝孝和,仁而不武,但恣池臺之賞,寧顧衽席之嫌?武室、韋宗,幾危運祚。東京帝後,歿從夫謚,光烈、和熹之類是也。高宗自號天皇,武氏自稱天後,而韋庶人生有翌聖之名,肅宗欲後張氏,此不經之甚,皆以兇終。玄
 宗以惠妃之愛,擯斥椒宮,繼以太真,幾喪天下。歷觀前古邦家喪敗之由,多基於子弟召禍。子弟之亂,必始於宮闈不正。故息隱鬩墻,秦王謀歸東洛;馬嵬塗地,太子不敢西行。若中有聖善之慈,胡能若是?《易》曰「家道正而天下定」,不其然歟!自後累朝,長秋虛位,或以旁宗入繼,母屬皆微,徒有冊拜之文,諒乏「關雎」之德。今錄其存於史冊者,為《後妃傳》云。



 高祖太穆皇后竇氏,京兆始平人,隋定州總管、神武公
 毅之女也。後母,周武帝姊襄陽長公主。後生而發垂過頸,三歲與身齊。周武帝特愛重之,養於宮中。時武帝納突厥女為後,無寵,後尚幼,竊言於帝曰:「四邊未靜,突厥尚強,願舅抑情撫慰,以蒼生為念。但須突厥之助,則江南、關東不能為患矣。」武帝深納之。毅聞之,謂長公主曰:「此女才貌如此,不可妄以許人,當為求賢夫。」乃於門屏畫二孔雀,諸公子有求婚者,輒與兩箭射之,潛約中目者許之。前後數十輩莫能中,高祖後至,兩發各中一目。
 毅大悅,遂歸於我帝。及周武帝崩,後追思如喪所生。隋文帝受禪,後聞而流涕,自投於床曰:「恨我不為男,以救舅氏之難。」毅與長公主遽掩口曰:「汝勿妄言,滅吾族矣!」



 後事元貞太后,以孝聞。太后素有羸疾,時或危篤。諸姒以太后性嚴懼譴,皆稱疾而退,惟后晝夜扶侍,不脫衣履者,動淹旬月焉。善書學,類高祖之書,人不能辨。工篇章,而好存規戒。大業中,高祖為扶風太守,有駿馬數匹。常言於高祖曰:「上好鷹愛馬,公之所知,此堪進御,不可
 久留,人或言者,必為身累,願熟思之。」高祖未決,竟以此獲譴。未幾,後崩於涿郡,時年四十五。高祖追思後言,方為自安之計,數求鷹犬以進之,俄而擢拜將軍,因流涕謂諸子曰:「我早從汝母之言,居此官久矣。」初葬壽安陵,後祔葬獻陵。上元元年八月,改上尊號曰太穆順聖皇后。



 太宗文德順聖皇后長孫氏,長安人,隋右驍衛將軍晟之女也。晟妻,隋揚州刺史高敬德女,生後。少好讀書,造
 次必循禮則。年十三,嬪於太宗。隋大業中,常歸寧於永興里,後舅高士廉媵張氏,於後所宿舍外見大馬,高二丈,鞍勒皆具,以告士廉。命筮之,遇《坤》之《泰》,筮者曰:「至哉坤元,萬物資生,乃順承天。坤厚載物,德合無疆。牝馬地類,行地無疆。變而之《泰》,內陽而外陰,內健而外順,是天地交而萬物通也。《象》曰:後以輔相天地之宜而左右人也。龍,《乾》之象也。馬,《坤》之象也。變而為《泰》,天地交也。繇協於《歸妹》,婦人之兆也。女處尊位,履中居順也。此女貴不
 可言。」武德元年,冊為秦王妃。時太宗功業既高,隱太子猜忌滋甚。後孝事高祖,恭順妃嬪,盡力彌縫,以存內助。及難作,太宗在玄武門,方引將士入宮授甲,後親慰勉之,左右莫不感激。九年,冊拜皇太子妃。



 太宗即位,立為皇后。贈後父晟司空、齊獻公。後性尤儉約,凡所服御,取給而已。太宗彌加禮待,常與後論及賞罰之事,對曰:「牝雞之晨,惟家之索。妾以婦人,豈敢豫聞政事?」太宗固與之言,竟不之答。時后兄無忌,夙與太宗為布衣之交,又
 以佐命元勛,委以腹心,出入臥內,將任之朝政。後固言不可,每乘間奏曰:「妾既托身紫宮,尊貴已極,實不願兄弟子侄布列朝廷。漢之呂、霍可為切骨之誡,特願聖朝勿以妾兄為宰執。」太宗不聽,竟用無忌為左武候大將軍、吏部尚書、右僕射。後又密遣無忌苦求遜職,太宗不獲已而許焉,改授開府儀同三司,後意乃懌。有異母兄安業,好酒無賴。獻公之薨也,後及無忌並幼,安業斥還舅氏,後殊不以介意,每請太宗厚加恩禮,位至監門將
 軍。及預劉德裕逆謀,太宗將殺之,後叩頭流涕為請命曰:「安業之罪,萬死無赦。然不慈於妾,天下知之,今置以極刑,人必謂妾恃寵以復其兄,無乃為聖朝累乎!」遂得減死。



 後所生長樂公主,太宗特所鐘愛,及將出降,敕所司資送倍於長公主。魏徵諫曰:「昔漢明帝時,將封皇子,帝曰:『朕子安得同於先帝子乎!』然謂長主者,良以尊於公主也,情雖有差,義無等別。若令公主之禮有過長主,理恐不可,願陛下思之。」太宗以其言退而告後,後嘆曰:「
 嘗聞陛下重魏徵,殊未知其故。今聞其諫,實乃能以義制主之情,可謂正直社稷之臣矣。妾與陛下結發為夫婦,曲蒙禮待,情義深重,每言必候顏色,尚不敢輕犯威嚴,況在臣下,情疏禮隔,故韓非為之說難,東方稱其不易,良有以也。忠言逆於耳而利於行,有國有家者急務,納之則俗寧,杜之則政亂,誠願陛下詳之,則天下幸甚。」後因請遣中使齎帛五百匹,詣徵宅以賜之。太子承乾乳母遂安夫人常白後曰:「東宮器用闕少,欲有奏請。」後
 不聽,曰:「為太子,所患德不立而名不揚,何憂少於器物也!」



 八年,從幸九成宮,染疾危惙,太子承乾入侍,密啟後曰:「醫藥備盡,尊體不瘳,請奏赦囚徒,並度人入道,冀蒙福助。」後曰:「死生有命,非人力所加。若修福可延,吾素非為惡。若行善無效,何福可求?赦者,國之大事;佛道者,示存異方之教耳,非惟政體靡弊,又是上所不為,豈以吾一婦人而亂天下法?」承乾不敢奏,以告左僕射房玄齡,玄齡以聞,太宗及侍臣莫不噓唏。朝臣咸請肆赦,太宗
 從之;後聞之,固爭,乃止。將大漸,與太宗辭訣,時玄齡以譴歸第,後固言:「玄齡事陛下最久,小心謹慎,奇謀秘計,皆所預聞,竟無一言漏洩,非有大故,願勿棄之。又妾之本宗,幸緣姻戚,既非德舉,易履危機,其保全永久,慎勿處之權要,但以外戚奉朝請,則為幸矣。妾生既無益於時,今死不可厚費。且葬者,藏也,欲人之不見。自古聖賢,皆崇儉薄,惟無道之世,大起山陵,勞費天下,為有識者笑。但請因山而葬,不須起墳,無用棺槨,所須器服,皆以
 木瓦,儉薄送終,則是不忘妾也。」十年六月己卯,崩於立政殿,時年三十六。其年十一月庚寅,葬於昭陵。



 後嘗撰古婦人善事,勒成十卷,名曰《女則》,自為之序。又著論駁漢明德馬皇后,以為不能抑退外戚,令其當朝貴盛,乃戒其龍馬水車,此乃開其禍源而防其末事耳。且戒主守者曰:「此吾以自防閑耳。婦人著述無條貫,不欲至尊見之,慎勿言。」崩後,宮司以聞,太宗覽而增慟,以示近臣曰:「皇后此書,足可垂於後代。我豈不達天命而不能割
 情乎!以其每能規諫,補朕之闕,今不復聞善言,是內失一良佐,以此令人哀耳!」上元元年八月,改上尊號曰文德順聖皇后。



 太宗賢妃徐氏,名惠,右散騎常侍堅之姑也。生五月而能言,四歲誦《論語》、《毛詩》,八歲好屬文。其父孝德試擬《楚辭》,云「山中不可以久留」,詞甚典美。自此遍涉經史,手不釋卷。太宗聞之,納為才人。其所屬文,揮翰立成,詞華綺贍。俄拜婕妤,再遷充容。時軍旅亟動,宮室互興,百姓頗
 倦勞役,上疏諫曰:



 自貞觀已來,二十有二載,風調雨順,年登歲稔,人無水旱之弊,國無饑饉之災。昔漢武守文之常主,猶登刻玉之符;齊桓小國之庸君,尚圖泥金之事。望陛下推功損己,讓德不居。億兆傾心,猶闕告成之禮;雲亭佇謁,未展升中之儀。此之功德,足以咀嚼百王,網羅千代者矣。古人有云:「雖休勿休」,良有以也。守初保末,聖哲罕兼。是知業大者易驕,願陛下難之;善始者難終,願陛下易之。



 竊見頃年已來,力役兼總,東有遼海之
 軍,西有昆丘之役,士馬疲於甲胄,舟車倦於轉輸。且召募役戍,去留懷死生之痛;因風阻浪,人米有漂溺之危。一夫力耕,卒無數十之獲;一船致損,則傾數百之糧。是猶運有盡之農功,填無窮之巨浪,圖未獲之他眾,喪已成之我軍。雖除兇伐暴,有國常規;然黷武玩兵,先哲所戒。昔秦皇並吞六國,反速危亡之基;晉武奄有三方,翻成覆敗之業。豈非矜功恃大,棄德而輕邦;圖利忘害,肆情而縱欲?遂使悠悠六合,雖廣不救其亡;嗷嗷黎庶,因
 弊以成其禍。是知地廣非常安之術,人勞乃易亂之源。願陛下布澤流人,矜弊恤乏,減行役之煩,增湛露之惠。妾又聞為政之本,貴在無為。竊見土木之功,不可兼遂。此闕初建,南營翠微,曾未逾時,玉華創制。雖復因山藉水,非無架築之勞;損之又損,頗有工力之費。終以茅茨示約,猶興木石之疲;假使和雇取人,不無煩擾之弊。是以卑宮菲食,聖主之所安;金屋瑤臺,驕主之為麗。故有道之君,以逸逸人;無道之君,以樂樂身。願陛下使之以
 時,則力無竭矣;用而息之,則人斯悅矣。



 夫珍玩伎巧,乃喪國之斧斤;珠玉錦繡,實迷心之鴆毒。竊見服玩纖靡,如變化於自然;織貢珍奇,若神仙之所制。雖馳華於季俗,實敗素於淳風。是知漆器非延叛之方,桀造之而人叛;玉杯豈招亡之術,紂用之而國亡。方驗侈麗之源,不可不遏。作法於儉,猶恐其奢;作法於奢,何以制後?伏惟陛下明鑒未形,智周無際,窮奧秘於麟閣,盡探賾於儒林。千王治亂之蹤,百代安危之跡,興衰禍福之數,得失
 成敗之機,故亦苞吞心府之中,循環目圍之內,乃宸衷之久察,無假一二言焉。惟恐知之非難,行之不易,志驕於業泰,體逸於時安。伏願抑志裁心,慎終如始,削輕過以添重德,循今是以替前非,則令名與日月無窮,盛業與乾坤永大。



 太宗善其言,優賜甚厚。及太宗崩,追思顧遇之恩,哀慕愈甚,發疾不自醫。病甚,謂所親曰:「吾荷顧實深,志在早歿,魂其有靈,得侍園寢,吾之志也。」因為七言詩及連珠以見其志。永徽元年卒,時年二十四,詔贈
 賢妃,陪葬於昭陵之石室。



 高宗廢後王氏,並州祁人也。父仁祐,貞觀中羅山令。同安長公主,即後之從祖母也。公主以後有美色,言於太宗,遂納為晉王妃。高宗登儲,冊為皇太子妃,以父仁祐為陳州刺史。永徽初,立為皇后,以仁祐為特進、魏國公,母柳氏為魏國夫人。仁祐尋卒,贈司空。



 初,武皇后貞觀末隨太宗嬪御居於感業寺,後及左右數為之言,高宗由是復召入宮,立為昭儀。俄而漸承恩寵,遂與後及良
 娣蕭氏遞相譖毀。帝終不納後言,而昭儀寵遇日厚。後懼不自安,密與母柳氏求巫祝厭勝。事發,帝大怒,斷柳氏不許入宮中,後舅中書令柳奭罷知政事,並將廢後,長孫無忌、褚遂良等固諫,乃止。俄又納李義府之策,永徽六年十月,廢後及蕭良娣皆為庶人,囚之別院。武昭儀令人皆縊殺之。後母柳氏、兄尚衣奉御全信及蕭氏兄弟,並配流嶺外。遂立昭儀為皇后。尋又追改後姓為蟒氏,蕭良娣為梟氏。



 庶人良娣初囚,大罵曰:「願阿武為
 老鼠,吾作貓兒,生生扼其喉!」武后怒,自是宮中不畜貓。初囚,高宗念之,閑行至其所,見其室封閉極密,惟開一竅通食器出入。高宗惻然,呼曰:「皇后、淑妃安在?」庶人泣而對曰:「妾等得罪,廢棄為宮婢,何得更有尊稱,名為皇后?」言訖悲咽,又曰:「今至尊思及疇昔,使妾等再見日月,出入院中,望改此院名為『回心院』,妾等再生之幸。」高宗曰:「朕即有處置。」武後知之,令人杖庶人及蕭氏各一百,截去手足,投於酒甕中,曰:「令此二嫗骨醉!」數日而卒。後
 則天頻見王、蕭二庶人披發瀝血,如死時狀。武后惡之,禱以巫祝,又移居蓬萊宮,復見,故多在東都。中宗即位,復後姓為王氏,梟氏還為蕭氏。



 中宗和思皇后趙氏,京兆長安人。祖綽,武德中以戰功至右領軍衛將軍。父瑰,尚高祖女常樂公主,歷遷左千牛將軍。中宗為英王時,納後為妃。既而妃母公主得罪,妃亦坐廢,幽死於內侍省。則天臨朝,瑰為壽州刺史,坐與越王貞連謀被誅,公主亦坐死。神龍元年,贈後謚為
 恭皇后,贈瑰左衛大將軍。及中宗崩,將葬於定陵,議者以韋後得罪,不宜祔葬,於是追謚后為和思,莫知瘞所,行招魂祔葬之禮。太常博士彭景直上言:「古無招魂葬之禮,不可備棺槨,置轀鬻京。宜據《漢書郊祀志》葬黃帝衣冠於橋山故事,以皇后禕衣於陵所寢宮招魂,置衣於魂輿,以太牢告祭,遷衣於寢宮,舒於御榻之右,覆以夷衾而祔葬焉。」從之。



 中宗韋庶人,京兆萬年人也。祖弘表,貞觀中為曹王府
 典軍。中宗為太子時,納後為妃,仍擢後父普州參軍玄貞為豫州刺史。嗣聖元年,立為皇后。其年,中宗見廢,後隨從房州。時中宗懼不自安,每聞制使至,惶恐欲自殺。後勸王曰:「禍福倚伏,何常之有?豈失一死,何遽如是也!」累年同艱危,情義甚篤。所生懿德太子、永泰、永壽、長寧、安樂四公主,安樂最幼,生於房州,帝自脫衣裹之,遂名曰裹兒,特寵異焉。及中宗復立為太子,又立後為妃。時昭容上官氏常勸後行則天故事,乃上表請天下士庶
 為出母服喪三年;又請百姓以年二十三為丁,五十九免役,改易制度,以收時望。制皆許之。



 帝在房州時,常謂後曰:「一朝見天日,誓不相禁忌。」及得志,受上官昭容邪說,引武三思入宮中,升御床,與後雙陸,帝為點籌,以為歡笑,醜聲日聞於外。乃大出宮女,雖左右內職,亦許時出禁中。上官氏及宮人貴幸者,皆立外宅,出入不節,朝官邪佞者候之,恣為狎游,祈其賞秩,以至要官。時侍中敬暉謀去諸武,武三思患之,乃結上官氏以為援,因得
 幸於後,潛入宮中謀議,乃諷百官上帝尊號為應天皇帝,後為順天皇后。帝與後親謁太廟,告謝受尊號之意。於是三思驕橫用事,敬暉、王同皎相次夷滅,天下咸歸咎於後。後方優寵親屬,內外封拜,遍列清要。又欲寵樹安樂公主,乃制公主開府,置官屬。太平公主儀比親王。長寧、安樂二府不置長史而已。宜城公主等以非後所生,各減太平之半。安樂恃寵驕恣,賣官鬻獄,勢傾朝廷,常自草制敕,掩其文而請帝書焉,帝笑而從之,竟不省
 視。又請自立為皇太女,帝雖不從,亦不加譴。所署府僚,皆猥濫非才。又廣營第宅,侈靡過甚。長寧及諸公主迭相仿效,天下咸嗟怨之。



 神龍三年,節愍太子死後,宗楚客率百僚上表,加後號為順天翊聖皇后。景龍二年春,宮中希旨,妄稱後衣箱中有五色雲出,帝使畫工圖之,出示於朝,乃大赦天下,百僚母妻各加邑號。右驍衛將軍、知太史事迦葉志忠上表曰:「昔高祖未受命時,天下歌《桃李子》;太宗未受命時,天下歌《秦王破陣樂》;高宗未
 受命時,天下歌《側堂堂》;天后未受命時,天下歌《武媚娘》。伏惟應天皇帝未受命時,天下歌《英王石州》;順天皇后未受命時,天下歌《桑條韋也》。女行六合之內,齊首蹀足,應四時八節之會,歌舞同歡。豈與夫《簫韶》九成、百獸率舞同年而語哉!伏惟皇后降帝女之精,合為國母,主蠶桑以安天下,後妃之德,於斯為盛。謹進《桑條歌》十二篇,伏請宣布中外,進入樂府,皇后先蠶之時,以享宗廟。」帝悅而許之,特賜志忠莊一區、雜彩七百段。太常少卿鄭
 愔又引而申之,播於舞詠,亦受厚賞。兵部尚書宗楚客又諷補闕趙延禧表陳符命,解《桑條》以為十八代之符,請頒示天下,編諸史冊。帝大悅,擢延禧為諫議大夫。時上官昭容與其母鄭氏及尚宮柴氏、賀婁氏,樹用親黨,廣納貨賂,別降墨敕,斜封授官,或出臧獲屠販之類,累居榮秩。又引女巫趙氏出入禁中,封為隴西夫人,勢與上官氏為比。



 三年冬,帝將親祠南郊,國子祭酒祝欽明、司業郭山惲建議云:「皇后亦合助祭。」太常博士唐紹、蔣
 欽緒上疏爭之。尚書右僕射韋巨源詳定儀注,遂希旨協同欽明之議。帝納其言,以後為亞獻,仍以宰相女為齊娘,以執籩豆。欽明又欲請安樂公主為終獻,迫於時議而止。四年正月望夜,帝與後微行市里,以觀燒燈。又放宮女數千,夜游縱觀,因與外人陰通,逃逸不還。時國子祭酒葉靜能善符禁小術,散騎常侍馬秦客頗閑醫藥,光祿少卿楊均以調膳侍奉,皆出入宮掖。均與秦客皆得幸於後,相次丁母憂,旬日悉起復舊職。時安樂公
 主與駙馬武延秀、侍中紀處訥、中書令宗楚客、司農卿趙履溫互相猜貳,迭為朋黨。



 六月,帝遇毒暴崩。時馬秦客侍疾,議者歸罪於秦客及安樂公主。後懼,秘不發喪,引所親入禁中,謀自安之策。以刑部尚書裴談、工部尚書張錫知政事,留守東都;又命左金吾大將軍趙承恩及宦者左監門衛大將軍薛崇簡帥兵五百人往筠州,以備譙王重福。後與兄太子少保溫定策,立溫王重茂為皇太子,召諸府兵五萬人屯京城,分為左右營,然後
 發喪。少帝即位,尊後為皇太后,臨朝攝政。韋溫總知內外兵馬,守援宮掖;駙馬韋捷、韋濯分掌左右屯營;武延秀及溫從子播、族弟璿、外甥高崇,共典左右羽林軍及飛騎、萬騎。播、璿欲先樹威嚴,拜官日先鞭萬騎數人,眾皆怨,不為之用。時京城恐懼,相傳將有革命之事,往往偶語,人情不安。臨淄王率薛崇簡、鐘紹京、劉幽求領萬騎及總監,丁未,入自玄武門,至左羽林軍,斬將軍韋璿、韋播及中郎將高崇於寢帳。遂斬關而入,至太極殿。後
 惶駭遁入殿前飛騎營,及武延秀、安樂公主皆為亂兵所殺。分遣萬騎誅其黨與韋溫、溫從子捷,及族弟嬰;宗楚客、弟晉卿,紀處訥,馬秦客,葉靜能,楊均,趙履溫,衛尉卿王哲,太常卿李曳,將作少匠李守質及韋氏武氏宗族,無少長皆斬之。梟後及安樂公主首於東市。翌日,敕收後尸,葬以一品之禮,追貶為庶人;安樂公主葬以三品之禮,追貶為悖逆庶人。



 中宗上官昭容,名婉兒,西臺侍郎儀之孫也。父庭芝,與
 儀同被誅,婉兒時在襁褓,隨母配入掖庭。及長,有文詞,明習吏事。則天時,婉兒忤旨當誅,則天惜其才不殺,但黥其面而已。自聖歷已後,百司表奏,多令參決。中宗即位,又令專掌制命,深被信任。尋拜為昭容,封其母鄭氏為沛國夫人。婉兒既與武三思淫亂,每下制敕,多因事推尊武後而排抑皇家。節愍太子深惡之,及舉兵,至肅章門,扣閣索婉兒。婉兒大言曰:「觀其此意,即當次索皇后以及大家。」帝與後遂激怒,並將婉兒登玄武門樓以
 避兵鋒,俄而事定。婉兒常勸廣置昭文學士,盛引當朝詞學之臣,數賜游宴,賦詩唱和。婉兒每代帝及後、長寧安樂二公主,數首並作,辭甚綺麗,時人咸諷誦之。婉兒又通於吏部侍郎崔湜,引知政事。湜嘗充使開商山新路,功未半而中宗崩,婉兒草遺制,曲敘其功而加褒賞。及韋庶人敗,婉兒亦斬於旗下。玄宗令收其詩筆,撰成文集二十卷,令張說為之序。初,婉兒在孕時,其母夢人遺己大秤,占者曰:「當生貴子,而秉國權衡。」既生女,聞者
 嗤其無效,及婉兒專秉內政,果如占者之言。



 睿宗肅明順聖皇後劉氏,刑部尚書德威之孫也。父延景,陜州刺史,景雲元年,追贈尚書右僕射、沛國公。儀鳳中,睿宗居籓,納後為孺人,尋立為妃,生寧王憲、壽昌代國二公主。文明元年睿宗即位,冊為皇后;及降為皇嗣,後從降為妃。長壽中,與昭成皇后同被譴,為則天所殺。景雲元年,追謚肅明皇后,招魂葬於東都城南,陵曰惠陵。睿宗崩,遷祔橋陵。以昭成太后故,不得入太廟配饗,
 常別祀於儀坤廟。開元二十年,始祔太廟。



 睿宗昭成順聖皇后竇氏,將作大匠抗曾孫也。祖誕,大理卿、莘國公。父孝諶,潤州刺史,景雲元年,追贈太尉、邠國公。後姿容婉順,動循禮則,睿宗為相王時為孺人,甚見禮異。光宅元年,立為德妃。生玄宗及金仙、玉真二公主。長壽二年,為戶婢團兒誣譖與肅明皇后厭蠱咒詛。正月二日,朝則天皇后於嘉豫殿,既退而同時遇害。梓宮秘密,莫知所在。睿宗即位,謚曰昭成皇后,招魂葬於
 都城之南,陵曰靖陵。又立廟於京師,號為儀坤廟。睿宗崩,後以帝母之重,追尊為皇太后,謚仍舊,祔葬橋陵,遷神主於太廟。



 玄宗廢後王氏,同州下邽人,梁冀州刺史神念之後。上為臨淄王時,納後為妃。上將起事,頗預密謀,贊成大業。先天元年,為皇后,以父仁皎為太僕卿,累加開府儀同三司、邠國公。後兄守一以後無子,常懼有廢立,導以符厭之事。有左道僧明悟為祭南北斗,刻霹靂木,書天地
 字及上諱,合而佩之,且祝曰:「佩此有子,當與則天皇后為比。」事發,上親究之,皆驗。開元十二年秋七月己卯,下制曰:「皇后王氏,天命不祐,華而不實。造起獄訟,朋扇朝廷,見無將之心,有可諱之惡。焉得敬承宗廟,母儀天下?可廢為庶人,別院安置。刑於家室,有愧昔王,為國大計,蓋非獲已。」守一賜死。其年十月,庶人卒,以一品禮葬於無相寺。寶應元年,雪免,復尊為皇后。



 玄宗貞順皇后武氏,則天從父兄子恆安王攸止女也。
 攸止卒後,後尚幼,隨例入宮。上即位,漸承恩寵。及王庶人廢後,特賜號為惠妃,宮中禮秩,一同皇后。所生母楊氏,封為鄭國夫人。同母弟忠,累遷國子祭酒;信,秘書監。惠妃開元初產夏悼王及懷哀王、上仙公主,並襁褓不育,上特垂傷悼。及生壽王瑁,不敢養於宮中,命寧王憲於外養之。又生盛王琦,咸宜、太華二公主。惠妃以開元二十五年十二月薨,年四十餘。下制曰:「存有懿範,沒有寵章,豈獨被於朝班,故乃施於亞政,可以垂裕,斯為通
 典。故惠妃武氏,少而婉順,長而賢明,行合禮經,言應圖史。承戚里之華胄,升後庭之峻秩,貴而不恃,謙而益光。以道飭躬,以和逮下,四德粲其兼備,六宮咨而是則。法度在己,靡資珩佩;躬儉化人,率先絺紘。夙有奇表,將加正位,前後固讓,辭而不受,奄至淪歿,載深感悼,遂使玉衣之慶,不及於生前;象服之榮,徒增於身後。可贈貞順皇后,宜令所司擇日冊命。」葬於敬陵。時慶王琮等請制齊衰之服,有司請以忌日廢務,上皆不
 許之。立廟於京中昊天觀南,乾元之後,祠享亦絕。



 玄宗楊貴妃,高祖令本,金州刺史。父玄琰,蜀州司戶。妃早孤,養於叔父河南府士曹玄璬。開元初,武惠妃特承寵遇,故王皇后廢黜。二十四年惠妃薨,帝悼惜久之,後庭數千,無可意者。或奏玄琰女姿色冠代,宜蒙召見。時妃衣道士服,號曰太真。既進見,玄宗大悅。不期歲,禮遇如惠妃。太真姿質豐艷,善歌舞,通音律,智算過人。每倩盼承迎,動移上意。宮中呼為「娘子」,禮數實同皇后。有姊
 三人,皆有才貌,玄宗並封國夫人之號:長曰大姨,封韓國;三姨,封虢國;八姨,封秦國。並承恩澤,出入宮掖,勢傾天下。妃父玄琰,累贈太尉、齊國公;母封涼國夫人;叔玄珪,光祿卿。再從兄銛,鴻臚卿。錡,侍御史,尚武惠妃女太華公主,以母愛,禮遇過於諸公主,賜甲第,連於宮禁。韓、虢、秦三夫人與銛、錡等五家,每有請托,府縣承迎,峻如詔敕,四方賂遺,其門如市。



 五載七月,貴妃以微譴送歸楊銛宅。比至亭午,上思之,不食。高力士探知上旨,請送
 貴妃院供帳、器玩、廩餼等辦具百餘車,上又分御饌以送之。帝動不稱旨,暴怒笞撻左右。力士伏奏請迎貴妃歸院。是夜,開安興里門入內,妃伏地謝罪,上歡然慰撫。翌日,韓、虢進食,上作樂終日,左右暴有賜與。自是寵遇愈隆。韓、虢、秦三夫人歲給錢千貫,為脂粉之資。銛授三品、上柱國,私第立戟。姊妹昆仲五家,甲第洞開,僭擬宮掖,車馬僕御,照耀京邑,遞相誇尚。每構一堂,費逾千萬計,見制度宏壯於己者,即撤而復造,土木之工,不舍晝
 夜。玄宗頒賜及四方獻遺,五家如一,中使不絕。開元已來,豪貴雄盛,無如楊氏之比也。玄宗凡有游幸,貴妃無不隨侍,乘馬則高力士執轡授鞭。宮中供貴妃院織錦刺繡之工,凡七百人,其雕刻熔造,又數百人。揚、益、嶺表刺史,必求良工造作奇器異服,以奉貴妃獻賀,因致擢居顯位。玄宗每年十月幸華清宮,國忠姊妹五家扈從,每家為一隊,著一色衣,五家合隊,照映如百花之煥發,而遺鈿墜舄,瑟瑟珠翠,燦爛芳馥於路。而國忠私於虢
 國而不避雄狐之刺,每入朝或聯鑣方駕,不施帷幔。每三朝慶賀,五鼓待漏,艷妝盈巷,蠟炬如晝。而十宅諸王百孫院婚嫁,皆因韓、虢為紹介,仍先納賂千貫而奏請,罔不稱旨。天寶九載,貴妃復忤旨,送歸外第。時吉溫與中貴人善,溫入奏曰:「婦人智識不遠,有忤聖情,然貴妃久承恩顧,何惜宮中一席之地,使其就戮,安忍取辱於外哉!」上即令中使張韜光賜御饌,妃附韜光泣奏曰:「妾忤聖顏,罪當萬死。衣服之外,皆聖恩所賜,無可遺留,然
 發膚是父母所有。」乃引刀翦發一繚附獻。玄宗見之驚惋,即使力士召還。



 國忠既居宰執,兼領劍南節度,勢漸恣橫。十載正月望夜,楊家五宅夜游,與廣平公主騎從爭西市門。楊氏奴揮鞭及公主衣,公主墮馬,駙馬程昌裔扶主,因及數撾。公主泣奏之,上令殺楊氏奴,昌裔亦停官。國忠二男昢、暄,妃弟鑒,皆尚公主,楊氏一門尚二公主、二郡主。貴妃父祖立私廟,玄宗御制家廟碑文並書。玄珪累遷至兵部尚書。天寶中,範陽節度使安祿
 山大立邊功,上深寵之。祿山來朝,帝令貴妃姊妹與祿山結為兄弟。祿山母事貴妃,每宴賜,錫賚稠沓。及祿山叛,露檄數國忠之罪。河北盜起,玄宗以皇太子為天下兵馬元帥,監撫軍國事。國忠大懼,諸楊聚哭,貴妃銜土陳請,帝遂不行內禪。及潼關失守,從幸至馬嵬,禁軍大將陳玄禮密啟太子,誅國忠父子。既而四軍不散,玄宗遣力士宣問,對曰「賊本尚在」,蓋指貴妃也。力士復奏,帝不獲已,與妃詔,遂縊死於佛室。時年三十八,瘞於驛西
 道側。



 上皇自蜀還,令中使祭奠,詔令改葬。禮部侍郎李揆曰:「龍武將士誅國忠,以其負國兆亂。今改葬故妃,恐將士疑懼,葬禮未可行。」乃止。上皇密令中使改葬於他所。初瘞時以紫褥裹之,肌膚已壞,而香囊仍在。內官以獻,上皇視之淒惋,乃令圖其形於別殿,朝夕視之。



 馬嵬之誅國忠也,虢國夫人聞難作,奔馬至陳倉。縣令薛景仙率人吏追之,走入竹林。先殺其男裴徽及一女。國忠妻裴柔曰:「娘子為我盡命。」即刺殺之。已而自刎,不死,縣
 吏載之,閉於獄中。猶謂吏曰:「國家乎?賊乎?」吏曰:「互有之。」血凝至喉而卒,遂瘞於郭外。韓國夫人婿秘書少監崔峋,女為代宗妃。虢國男裴徽尚代宗女延安公主,女嫁讓帝男。秦國夫人婿柳澄先死,男鈞尚長清縣主,澄弟潭尚肅宗女和政公主。



\end{pinyinscope}