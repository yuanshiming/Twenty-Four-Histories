\article{卷一百五十八}

\begin{pinyinscope}

 ○孔巢父從子戡戣戢許孟容中元膺劉棲楚張宿熊望柏耆



 孔巢父,冀州人,字弱翁。父如珪,海州司戶參軍,以巢父贈工部郎中。巢父早勤文史,少時與韓準、裴政、李白、張
 叔明、陶沔隱於徂來山,時號「竹溪六逸」。永王璘起兵江淮,聞其賢,以從事闢之。巢父知其必敗,側身潛遁,由是知名。



 廣德中,李季卿為江淮宣撫使,薦巢父,授左衛兵曹參軍。大歷初,澤潞節度使李抱玉奏為賓幕,累授監察御史,轉殿中、檢校庫部員外郎,出授歸州刺史。建中初,涇原節度留後孟皞表巢父試秘書少監,兼御史中丞、行軍司馬。尋拜汾州刺史,入為諫議大夫,出為潭州刺史、湖南觀察使。未行,會普王為荊襄副元帥,以巢父
 為元帥府行軍司馬,兼御史大夫。



 尋屬涇師之難,從德宗幸奉天,遷給事中、河中陜華等州招討使。累獻破賊之謀,德宗甚賞之。尋兼御史大夫,充魏博宣慰使。巢父博辯多智,對田悅之眾,陳逆順利害君臣之道,士眾欣悚喜抃,曰:「不圖今日復睹王化!」及就宴,悅酒酣,自矜其騎射之藝、拳勇之略,因曰:「若蒙見用,無堅不摧。」巢父謂之曰:「若如公言而不早歸國者,但為一好賊耳。」悅曰:「為賊既曰好賊,為臣當作功臣。」巢父曰:「國方有虞,待子而
 息。」悅起謝焉。悅背叛日久,其下厭亂,且喜巢父之至。數日,田承嗣之子緒以失職怨望,因人心之搖動,遂構謀殺悅,而與大將邢曹俊等稟命於巢父。巢父因其眾意,令田緒權知軍務,以紓其難。



 興元元年,李懷光擁兵河中。七月,復以巢父兼御史大夫,充宣慰使。既傳詔旨,懷光以巢父嘗使魏博,田悅死於帳下,恐禍及。又朔方蕃渾之眾數千,皆在行列,頗驕悖不肅。聞罷懷光兵權,時懷光素服待命,巢父不止之。眾咸忿恚,咄嗟曰:「太尉盡
 無官矣!」方宣詔,喧噪,懷光亦不禁止,巢父、守盈並遇害。上聞之震悼,贈尚書左僕射,仍詔收河中日備禮葬祭。賜其家布帛米粟甚厚,仍授子正員官。從子戡、戣、戢。



 戡,巢父兄岑父之子。方嚴有家法,重然諾,尚忠義。盧從史鎮澤潞,闢為書記。從史浸驕,與王承宗、田緒陰相連結,欲效河朔事以固其位。戡每秉筆至不軌之言,極諫以為不可,從史怒戡,歲餘,謝病歸洛陽。李吉甫鎮揚州,召為賓佐。從史知之,上疏論列,請行貶逐。憲宗不得已,授
 衛尉丞,分司洛陽。初,貞元中籓帥誣奏從事者,皆不驗理,便行降黜。及戡詔下,給事中呂元膺執之,上令中使慰喻元膺,制書方下。戡不調而卒,贈駕部員外郎。



 戣,字君嚴。登進士第,鄭滑節度使盧群闢為從事。群卒,命戣權掌留務,監軍使以氣凌之,戣無所屈降。入為侍御史,累轉尚書郎。元和初,改諫議大夫,侃然忠讜,有諫臣體。上疏論時政四條,帝意嘉納。



 六年十月,內官劉希光受將軍孫璹賂二十萬貫,以求方鎮。事敗,賜希光死。時吐
 突承璀以出軍無功,諫官論列,坐希光事出為淮南監軍使。太子通事舍人李涉知上待承璀意未衰,欲投匭上疏,論承璀有功,希光無事,久委心腹,不宜遽棄。戣為匭使,得涉副章,不受,面詰責之。涉乃進疏於光順門。戣極論其與中官交結,言甚激切。詔貶涉為陜州司倉。幸臣聞之側目,人為危之。



 戣高步公卿間,以方嚴見憚。俄兼太子侍讀,遷吏部侍郎,轉左丞。



 九年,信州刺史李位為州將韋岳讒譖於本使監軍高重謙,言位結聚術士,
 以圖不軌。追位至京師,鞫於禁中。戣奏曰:「刺史得罪,合歸法司按問,不合劾於內仗。」乃出付御史臺。戣與三司訊鞫,得其狀。位好黃老道,時修齋籙,與山人王恭合煉藥物,別無逆狀。以岳誣告,決殺。貶位建州司馬。時非戣論諫,罪在不測,人士稱之。愈為中官所惡,尋出為華州刺史、潼關防禦等使。入為大理卿,改國子祭酒。



 十二年,嶺南節度使崔詠卒,三軍請帥,宰相奏擬皆不稱旨。因入對,上謂裴度曰:「嘗有上疏論南海進蚶菜者,詞甚忠
 正,此人何在,卿第求之。」度退訪之。或曰祭酒孔戣嘗論此事,度徵疏進之。即日授廣州刺史,兼御史大夫、嶺南節度使。



 戣剛正清儉,在南海,請刺史俸料之外,絕其取索。先是帥南海者,京師權要多托買南人為奴婢,戣不受托。至郡,禁絕賣女口。先是準詔禱南海神,多令從事代祠。戣每受詔,自犯風波而往。韓愈在潮州,作詩以美之。時桂管經略使楊旻、桂仲武、裴行立等騷動生蠻,以求功伐,遂至嶺表累歲用兵。唯戣以清儉為理,不務邀
 功,交、廣大理。



 穆宗即位,召為吏部侍郎。長慶中,或告戣在南海時家人受賂,上不之責,改右散騎常侍。二年,轉尚書左丞。累請老,詔以禮部尚書致仕,優詔褒美。仍令所司歲致羊酒,如漢禮徵士故事。長慶四年正月卒,時年七十三。



 子遵孺、溫裕,皆登進士第。大中已後,迭居顯職。溫裕位京兆尹、天平軍節度使。遵孺子緯,自有傳。



 戢,字方舉,戣母弟也。以季父巢父死難,德宗嘉其忠,詔與一子正員官,因授戢修武尉。以長兄戡未仕,固乞回授。
 舉明經登第,判入高等,授秘書省校書郎、陽翟尉,入拜監察御史,轉殿中,分司東都。時昭義節度判官徐玟,以狡慝助成從史之惡。從史既得罪,孟元陽為昭義節度,復欲用玟為賓佐,戢遂牒澤潞收玟以俟命,然後列狀上聞,竟流玟播州。轉侍御史、庫部員外郎。



 初,涇師之亂,硃泚署彭偃為舍人。至是偃子充符為鄜坊從事,或薦其才,執事者召至京師。戢謂京兆尹裴武曰:「硃泚為偽詔,指斥乘輿,皆彭偃之詞也。悖逆之子,不能鳥竄獸伏,
 乃違道以干譽,子盍效季孫行父之逐莒僕,以勉事君者。」武即日逐充符。



 遷京兆尹,出為汝州刺史、大理卿。出為潭州刺史、湖南觀察使。時兄戣為嶺南,兄弟皆居節鎮,朝野榮之。入為右散騎常侍,拜京兆尹。時累月亢旱,深軫聖情。戢自禱雨於曲池,是夕大雨。文宗甚悅,詔兼御史大夫。大和三年正月卒,贈工部尚書。



 子溫業,登進士第。大中後,歷位通顯。溫業子晦。



 許孟容,字公範,京兆長安人也。父鳴謙,究通《易象》,官至
 撫州刺史,贈禮部尚書。孟容少以文詞知名,舉進士甲科,後究《王氏易》登科,授秘書省校書郎。趙贊為荊、襄等道黜陟使,表為判官。貞元初,徐州節度使張建封闢為從事,四遷侍御史。李納屯兵境上,揚言入寇。建封遣將吏數輩告諭,不聽。於是遣孟容單車詣納,為陳逆順禍福之計。納即日發使追兵,因請修好。遂表孟容為濠州刺史。無幾,德宗知其才,徵為禮部員外郎。



 有公主之子,請補弘文、崇文館諸生,孟容舉令式不許。主訴於上,命
 中使問狀。孟容執奏,竟得遷本曹郎中。德宗降誕日,御麟德殿,命孟容等登座,與釋、老之徒講論。十四年,轉兵部郎中。未滿歲,遷給事中。



 十七年夏,好畤縣風雹傷麥,上命品官覆視,不實,詔罰京兆尹顧少連已下。敕出,孟容執奏曰:「府縣上事不實,罪止奪俸停官,其於弘宥,已是殊澤。但陛下使品官覆視後,更擇憲官一人,再令驗察,覆視轉審,隱欺益明。事宜觀聽,法歸綱紀。臣受官中謝日,伏請詔敕有須詳議者,則乞停留晷刻,得以奏陳。
 此敕既非急,宣可以少駐。」詔雖不許,公議是之。



 十八年,浙江東道觀察使裴肅卒,以攝副使齊總為衢州刺史。時總為肅剝下進奉以希恩,遽授大郡,物議喧然。詔出,孟容執奏曰:「陛下比者以兵戎之地,或有不獲已超授者。今衢州無他虞,齊總無殊績,忽此超授,群情驚駭。總是浙東判官,今詔敕稱權知留後,攝都團練副使,向來無此敕命。便用此詔,尤恐不可。若總必有可錄,陛下須要酬勞,即明書課最,超一兩資與改。今舉朝之人,不知
 總之功能,衢州浙東大郡,總自大理評事兼監察御史授之,使遐邇不甘,兇惡騰口。如臣言不切,乞陛下暫停此詔,密使人聽察,必賀聖朝無私。今齊總詔謹隨狀封進。」尋有諫官論列,乃留中不下。德宗召孟容對於延英,諭之曰:「使百執事皆如卿,朕何憂也。」自給事中袁高論盧杞後,未嘗有可否,及聞孟容之奏,四方皆感上之聽納,嘉孟容之當官。



 十九年夏旱,孟容上疏曰:



 臣伏聞陛下數月已來,齋居損膳,為兆庶心疲,又敕有司,走於群
 望,牲於百神,而密雲不雨,首種未入。豈觴醪有闕,祈祝非誠,為陰陽適然,豐歉前定,何聖意精至,甘澤未答也?臣歷觀自古天人交感事,未有不由百姓利病之急者、切者,邦家教令之大者、遠者。京師是萬國所會,強幹弱枝,自古通規。其一年稅錢及地租,出入一百萬貫。臣伏冀陛下即日下令,全放免之;其次,三分放二。且使旱涸之際,免更流亡。若播種無望,徵斂如舊,則必愁怨遷徙,不顧墳墓矣。臣愚以為德音一發,膏澤立應,變災為福,
 期在斯須。戶部所收掌錢,非度支歲計,本防緩急別用。今此炎旱,直支一百餘萬貫,代京兆百姓一年差科,實陛下巍巍睿謀,天下鼓舞歌揚者也。復更省察庶政之中,有流移征防,當還而未還者,徒役禁錮,當釋而未釋者,逋懸饋送,當免而未免者,沉滯鬱抑,當伸而未伸者,有一於此,則特降明命,令有司條列,三日內聞奏。其當還、當釋、當免、當伸者,下詔之日,所在即時施行。臣愚以為如此而神不監,歲不稔,古未之有。



 事雖不行,物議嘉
 之。貞元末,坐裴延齡、李齊運等讒謗流貶者,動十數年不量移,故因旱歉,孟容奏此以諷。然終貞元世,罕有遷移者。



 孟容以諷諭太切,改太常少卿。元和初,遷刑部侍郎、尚書右丞。四年,拜京兆尹,賜紫。神策吏李昱假貸長安富人錢八千貫,滿三歲不償。孟容遣吏收捕械系,克日命還之,曰:「不及期當死。」自興元已後,禁軍有功,又中貴之尤有渥恩者,方得護軍。故軍士日益縱橫,府縣不能制。孟容剛正不懼,以法繩之,一軍盡驚,冤訴於上。立
 命中使宣旨,令送本軍,孟容系之不遣。中使再至,乃執奏曰:「臣誠知不奉詔當誅,然臣職司輦轂,合為陛下彈抑豪強。錢未盡輸,昱不可得。」上以其守正,許之。自此豪右斂跡,威望大震。改兵部侍郎。俄以本官權知禮部貢舉,頗抑浮華,選擇才藝。出為河南尹,亦有威名。俄知禮部選事,徵拜吏部侍郎。



 會十年六月,盜殺宰相武元衡,並傷議臣裴度。時淮夷逆命,兇威方熾,王師問罪,未有成功。言事者繼上章疏請罷兵。是時盜賊竊發,人情甚
 惑,獨孟容詣中書雪涕而言曰:「昔漢廷有一汲黯,奸臣尚為寢謀。今主上英明,朝廷未有過失,而狂賊敢爾無狀,寧謂國無人乎?然轉禍為福,此其時也。莫若上聞,起裴中丞為相,令主兵柄,大索賊黨,窮其奸源。」後數日,度果為相,而下詔行誅。時孟容議論人物,有大臣風彩。由太常卿為尚書左丞,奉詔宣慰汴宋陳許河陽行營諸軍,俄拜東都留守。元和十三年四月卒,年七十六,贈太子少保,謚曰憲。



 孟容方勁,富有文學。其折衷禮法,考詳
 訓典,甚堅正,論者稱焉。而又好推轂,樂善拔士,士多歸之。



 呂元膺,字景夫,鄆州東平人。曾祖紹宗,右拾遺。祖霈,殿中侍御史。父長卿,右衛倉曹參軍,以元膺贈秘書監。



 元膺質度瑰偉,有公侯之器。建中初,策賢良對問第,授同州安邑尉。同州刺史侯鐈聞其名,闢為長春宮判官。屬浦賊侵軼,鐈失所,元膺遂潛跡不務進取。



 貞元初,論惟明節制渭北,延在賓席,自是名達於朝廷。惟明卒,王棲
 曜代領其鎮。德宗俾棲曜留署使職,咨以軍政。累轉殿中侍御史,徵入,真拜本官,轉侍御史。丁繼母憂,服闋,除右司員外郎。出為蘄州刺史,頗著恩信。嘗歲終閱郡獄囚,囚有自告者曰:「某有父母在,明日元正不得相見。」因泣下。元膺憫焉,盡脫其械縱之,與為期。守吏曰:「賊不可縱。」元膺曰:「吾以忠信待之。」及期,無後到者。由是群盜感義,相引而去。



 元和初,徵拜右司郎中、兼侍御史,知雜事,遷諫議大夫、給事中。規諫駁議,大舉其職。及鎮州王承
 宗之叛,憲宗將以吐突丞璀為招討處置使。元膺與給事中穆質、孟簡,兵部侍郎許孟容等八人抗論不可,且曰:「承璀雖貴寵,然內臣也。若為帥總兵,恐不為諸將所伏。」指諭明切,憲宗納之,為改使號,然猶專戎柄,無功而還。出為同州刺史,及中謝,上問時政得失,元膺論奏,辭氣激切,上嘉之。翌日謂宰相曰:「元膺有讜言直氣,宜留在左右,使言得失,卿等以為何如?」李籓、裴垍賀曰:「陛下納諫,超冠百王,乃宗社無疆之休。臣等不能廣求端士,
 又不能數進忠言,孤負聖心,合當罪戾。請留元膺給事左右。」尋兼皇太子侍讀,賜以金紫。



 尋拜御史中丞。未幾,除鄂岳觀察使,入為尚書左丞。度支使潘孟陽與太府卿王遂迭相奏論,孟陽除散騎常侍,遂為鄧州刺史,皆假以美詞。元膺封還詔書,請明示枉直。江西觀察使裴堪奏虔州刺史李將順贓狀,朝廷不覆按,遽貶將順道州司戶。元膺曰:「廉使奏刺史贓罪,不覆檢即謫去,縱堪之詞足信,亦不可為天下法。」又封詔書,請發御史按問,
 宰臣不能奪。代權德輿為東都留守、檢校工部尚書、兼御史大夫、都畿防禦使。舊例,留守賜旗甲,與方鎮同。及元膺受任不賜,朝論以淮西用兵,特用元膺守洛,不宜削其儀制,以沮威望,諫官論列,援華、汝、壽三州例。上曰:「此數處並宜不賜。」留守不賜旗甲,自元膺始。



 十年七月,鄆州李師道留邸伏甲謀亂。初,師道於東都置邸院,兵諜雜以往來,吏不敢辨。因吳元濟北犯,郊畿多警,防禦兵盡戍伊闕。師道伏甲百餘於邸院,將焚宮室而肆殺
 掠。已烹牛饗眾,明日將出。會小將李再興告變,元膺追兵伊闕,圍之,半月無敢進攻者。防禦判官王茂元殺一人而後進。或有毀其墉而入者,賊眾突出,圍兵奔駭。賊乃團結,以其孥偕行。出長夏門,轉掠郊墅,奪牛馬,東濟伊水,望山而去。元膺誡境上兵重購以捕之。數月,有山棚賣鹿於市。賊過,山棚乃召集其黨,引官兵圍於谷中,盡獲之。窮理其魁,乃中嶽寺僧圓凈,年八十餘,嘗為史思明將,偉悍過人。初執之,使折其脛,錘之不折。圓凈罵
 曰:「腳猶不解折,乃稱健兒乎!」自置其足教折之。臨刑嘆曰:「誤我事,不得使洛城流血!」死者凡數十人。留守防御將二人,都亭驛卒五人,甘水驛卒三人,皆潛受其職署而為之耳目,自始謀及將敗無知者。初,師道多買田於伊闕、陸渾之間,凡十餘處,故以舍山棚而衣食之。有訾嘉珍、門察者,潛部分之,以屬圓凈。以師道錢千萬偽理佛寺,期以嘉珍竊發時舉火於山中,集二縣山棚人作亂。及窮按之,嘉珍、門察皆稱害武元衡者。元膺以聞,送
 之上都,賞告變人楊進、李再興錦彩三百匹、宅一區,授之郎將。無膺因請募山河子弟以衛宮城,從之。盜發之日,都城震恐,留守兵寡弱,不可倚,而元膺坐皇城門,指使部分,氣意自若,以故居人帖然。



 數年,改河中尹,充河中節度等使。時方鎮多事姑息,元膺獨以堅正自處,監軍使洎往來中貴,無不敬憚。入拜吏部侍郎,因疾固讓,改太子賓客。元和十五年二月卒,年七十二,贈吏部尚書。



 元膺學識深遠,處事得體,正色立朝,有臺輔之望。初
 游京師時,故相齊映謂人曰:「吾不及識婁、郝,殆斯人之類乎!」其業官行己,始終無缺云。



 劉棲楚,出於寒微,為吏鎮州,王承宗甚奇之。後有薦於李逢吉,自鄧掾擢為拾遺。性果敢。逢吉以為鷹犬之用,欲中傷裴度及殺李紳。



 敬宗即位,畋游稍多,坐朝常晚。棲楚出班,以額叩龍墀出血,苦諫曰:「臣歷觀前王,嗣位之初,莫不躬勤庶政,坐以待旦。陛下即位已來,放情嗜寢,樂色忘憂,安臥宮闈,日晏方起。西宮密邇,未過山陵,
 鼓吹之聲,日喧於外。伏以憲宗皇帝、大行皇帝,皆是長君,恪勤庶政,四方猶有叛亂。陛下運當少主,即位未幾,惡德布聞,臣慮福祚之不長也。臣忝諫官,致陛下有此,請碎首以謝!」遂以額叩龍墀,久之不已。宰臣李逢吉出位宣曰:「劉棲楚休叩頭,候詔旨。」棲楚捧首而起,因更陳論,磕頭見血。上為之動容,以袖連揮令出。棲楚又云:「不可臣奏,臣即碎首死。」中書侍郎牛僧孺復宣示而出,敬宗為之動容。



 無何,遷起居郎,至諫議。俄又宣授刑部侍
 郎。丞郎宣授,未之有也。改京兆尹,摧抑豪右,甚有鉤距,人多比之於西漢趙廣漢者。後恃權寵,常以詞氣凌宰相韋處厚,遂出為桂州觀察使。逾年,卒於任,時大和元年九月。



 張宿者,布衣諸生也。憲宗為廣陵王時,因軍使張茂宗薦達,出入邸第。及上在東宮,宿時入謁,辯譎敢言。洎監撫之際,驟承顧擢,授左拾遺。以舊恩數召對禁中,機事不密,貶郴州郴縣丞。十餘年征入,歷贊善大夫、左補闕、
 比部員外郎。宰相李逢吉惡之,數於上前言其狡譎,不可保信,乃用為濠州刺史。制下,宿自理乞留,乃追制。上欲以為諫議大夫,逢吉奏曰:「諫議職重,當以能可否朝政者為之。宿細人,不足以污賢者位。陛下必須用宿,請先去臣即可。」上不悅。又逢吉與裴度是非不同,上方委度討伐,乃出逢吉為劍南東川節度。乃用宿權知諫議大夫,俄而內使宣授。



 初,宰相崔群、王涯奏曰:「諫議大夫前時亦有拔自山林、起於卒伍者,其例則少,用皆有由。
 或道義彰明,不求聞達;或山林卓異,出於群萃。以此選求,是愜公議。或事跡未著,恩由一時,雖有例超升,即時議未允。宿本非文辭入用,望實稍輕。驟加不次之榮,翻恐以身為累。臣等所以累有論諫,依資且與郎中,事冀適中,非於此人情有厚薄,請授職方郎中。」上命如初,群等乃請權知,尋又宣援。宿怨執政擯己,頗加讒毀。依附皇甫鎛等,傷害清正之士,陰事中要,以圖進取。



 十三年正月,充淄青宣慰使,至東都,暴病卒,於是正人相賀。詔
 贈秘書監。



 熊望者,登進士第。粗有文詞,而性憸險。有口辯,往往得游公卿間,率以大言詭意,指抉時政。既由此而得進士第,務進不已。而京兆尹劉棲楚以不次驟居清貫,廣樹朋黨,門庭無晝夜填委不息。望出入棲楚之門,為伺密機,陰佐計畫,人無知者。昭愍嬉游之隙,學為歌詩。以翰林學士崇重,不可褻狎,乃議別置東頭學士,以備曲宴賦詩,令採卑官才堪任學士者為之。棲楚以望名薦送,
 事未行而昭愍崩。



 文宗即位,韋處厚輔政,大去奸黨。既逐棲楚,又詔曰:「孔門高懸百行,由至順者,其身必榮;朝廷廣設眾官,踐正途者,其道必達。前鄉貢進士熊望,因緣薄伎,偷冀褻幸。營居中之密職,擾惑朝經;鼓逼下之囂聲,因依邪隙。及眾議波湧,累月不寧;司門驗繻,累月至四。考覆謬妄,乃非坦途。朕大啟康莊,以端群望,俾示投荒之典,用正向方之流。可漳州司戶。」



 柏耆者,將軍良器之子。素負志略,學縱橫家流。會王承宗
 以常山叛,朝廷厭兵,欲以恩澤撫之。耆於蔡州行營以畫乾裴度,請以朝旨奉使鎮州,乃自處士授左拾遺。既見承宗,以大義陳說。承宗泣下,請質二男,獻兩郡,由是知名。



 元和十年,王承宗歸國,移鎮滑州,朝廷賜成德軍賞錢一百萬貫,令諫議大夫鄭覃宣慰軍人,賞錢未至,浩浩然騰口。穆宗詔耆往諭旨。耆至,令承宗集三軍,宣導上旨,眾心乃安。轉兵部郎中。



 太和初,遷諫議大夫。俄而,李同捷叛,兩河籓帥加兵滄、德,宿師於野連年。同
 捷窮蹙求降。耆既宣諭訖,與節度使李祐謀。耆乃帥數百騎入滄州,取同捷赴京。滄、德平。諸將害耆邀功,爭上表論列。文宗不獲已,貶循州司戶判官,沈亞之貶虔州南康尉。內官馬國亮又奏耆於同捷處取婢九人,再命長流愛州,尋賜死。



 史臣曰:人臣事君,犯顏匡政,不避死亡之誅。議者以為徇名,臣惡其訐也。如許京兆之劾軍吏,呂尚書之封詔書,詞義可觀,聳動人聽,以為沽激,傷善何多!而棲楚、張
 宿之徒,鷹犬下材,為人鳴吠,誠可醜也。柏耆恃縱橫之算,欲俯拾卿相,忘身蹈利,旋踵而誅,宜哉!巢父使不辱命,志在致君,遭罹喪亂,竟陷虎吻。而戣、戢諸子,世載忠貞,大中之後,鬱為昌族。為善之利,豈虛言哉!



 贊曰:君子重義,小人殉利。巢殞耆誅,其道即異。許、呂封駁,照耀黃扉。死而可作,吾誰與歸?



\end{pinyinscope}