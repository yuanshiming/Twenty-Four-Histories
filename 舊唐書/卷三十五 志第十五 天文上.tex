\article{卷三十五 志第十五 天文上}

\begin{pinyinscope}

 《易》曰:「觀乎天文以察時變。」是故古之
 哲王,法垂象以施
 化,考庶徵以致理,以授人時,以考物紀,修其德以順其度,改其過以慎其災,去危而就安,轉禍而為福者也。夫其五緯七紀之名數,中官外官之位次,凌歷犯守之所主,飛流彗孛之所應,前史載之備矣。



 武德年中,薛頤、庾儉等相次為太史令,雖各善於占候,而無所發明。貞觀初,將仕郎直太史李淳風始上言靈臺候儀是後魏遺範,法制疏略,難為占步。太宗因令淳風改造渾儀,鑄銅為之,至七年造成。淳風因撰《法象志》七卷,以論前代渾儀得失之差,語在《淳風傳》。其所造渾儀,太宗令置於凝暉閣以用測候,既在宮中,尋而失其所在。玄宗開元九年,太史頻奏日蝕
 不效,詔沙門一行改造新歷。一行奏云,今欲創歷立元,須知黃道進退,請太史令測候星度。有司云:「承前唯依赤道推步,官無黃道游儀,無由測候。」時率府兵曹梁令瓚待制於麗正書院,因造游儀木羕,甚為精密。一行乃上言曰:「黃道游儀,古有其術而無其器。以黃道隨天運動,難用常儀格之,故昔人潛思皆不能得。今梁令瓚創造此圖,日道月交,莫
 不自然契合,既於推步尤要,望就書院更以銅鐵為之,庶得考驗星度,無有差舛。」從之,至十三年造成。又上疏曰:



 按《舜典》云:「在璇樞玉衡,以齊七政。」說者以為取其轉運者為樞,持正者為衡,皆以玉為之,用齊七政之變,知其盈縮進退,得失政之所在,即古太史渾天儀也。



 自周室衰微,疇
 人喪職,其制度遺象,莫有傳者。漢興,丞相張蒼首創律歷之學。至武帝詔司馬遷等更造漢歷,乃定東西、立晷儀、下漏刻,以追二十八宿相距星度,與古不同。故唐都分天部,洛下閎運算轉歷,今赤道歷星度,則其遺法也。後漢永元中,左中郎將賈逵奏言:「臣前上傅安等用黃道度日月,弦望多近。史官壹以赤道度之,不與天合,至差一日以上。願請太史官日月宿簿及星度課,與待詔星官考
 校。」奏可。問典星待詔姚崇等十二人,皆曰:『星圖有規法,日月實從黃道,官無其器,不知
 施行。』甘露二年,大司農丞耿壽昌奏,以圓儀度日月行,考驗天運。日月行赤道,至牽牛、東井,日行一度,月行十五度;至婁、角,日行一度,月行十三度,此前代所共知也。」是
 歲永元四載也。明年,始詔太史造黃道銅儀。冬至,日在斗十九度四分度之一,與赤道定差二度。史官以校日月弦望,雖密近,而不為望日。儀,黃道與度運轉,難候,是以少終其事。其後劉洪因黃道渾儀,以考月行出入遲速。而後代理歷者不遵其法,更從赤道命文,以驗賈逵所言,差謬益甚,此理歷者之大惑也。



 今靈臺鐵儀,後魏明元時都匠解蘭所造,規制樸略,度刻
 不均,赤道不動,乃如膠柱,不置黃道,進退無準。此據赤道月行以驗入歷遲速,多者或至十七度,少者僅出十度,不足以上稽天象,敬授人時。近秘閣郎中李淳風著《法象志》,備載黃道渾儀法,以玉衡旋規,別帶日道,傍列二百四十九交,以攜月游,用法頗雜,其術竟寢。



 臣伏承恩旨,更造游儀,使黃道運行,以追列舍之變,因二分之中以立黃道,交於軫、奎之間,二
 至陟降各二十四度。黃道之內,又施白道月環,用究陰陽朓朒之數,動合天運,簡而易從,足以制器垂象,永傳不朽。



 於是玄宗親為制銘,置之於靈臺以考星度。其二十八宿及中外官與古經不同者,凡數十條。又詔一行與梁令瓚及諸術士更造渾天儀,鑄銅為圓天之象,上具列宿赤道及周天度數。注水激輪,令其自轉,一
 日一夜,天轉一周。又別置二輪絡在天外,綴以日月,令得運行。每天西轉一幣,日東行一度,月行十三度
 十九分度之七,凡二十九轉有餘而日月會,三百六十五轉而日行匝。仍置木櫃以為地平,令儀半在地下,晦明朔望,遲速有準。又立二木人於地平之上,前置鐘鼓以候辰刻,每一刻自然擊鼓,每辰則自然撞鐘。皆於櫃中各施輪軸,鉤鍵交錯,關鎖相持。既與天道合同,當時共稱其妙。鑄成,命之曰水運渾天俯視圖,置於武成殿前以示百僚。無幾而銅鐵漸澀,不能自轉,遂收置於集賢院,不復行用。



 今錄游儀制度及所測星度異同,開元十二
 年分遣使諸州所測日晷長短,李淳風、僧一行所定十二次分野,武德已來交蝕及五星祥變,著於篇。



 黃道游儀規尺寸:



 旋樞雙環:外一丈四尺六寸一分,豎八分,厚三分,直徑四尺五寸九分,即古所謂旋儀也。南北斜兩極,上下循規各三十四度,兩面各畫周天度
 數。一
 面加釘,並用銀飾,使東西運轉如渾天游儀。中旋樞軸至兩極首內,孔徑大兩度半,長與旋環徑齊,並用古尺四分為度。



 玉衡望筒:長四尺五寸八分,廣一寸二分,厚一寸,孔徑六分,古用玉飾之。玉衡,衡施於軸中,旋運持正,用窺七曜及列星之闊狹,外方內圓,孔徑一度半,周日輪也。



 陽經雙環:外一丈七尺三寸,內一丈四尺六
 寸四分,廣四寸,厚四分,直徑五尺四寸四分,置於子午。左右用八柱相固,兩面畫周天度數,一面加釘,並銀飾之。半出地上,半入地下,雙間挾樞軸及玉衡望筒,旋環於中也。



 陰緯單環:外內廣厚周徑,皆準陽經,與陽經相銜各半,內外俱齊。面平上為天,以下為地,橫周陽環,謂之陰渾也。面上為兩界,內外為周天百刻。平上禦制銘序及書,並金為字。



 天頂單環:外一丈七尺三寸,豎廣八分,厚三分,直徑五尺四寸四分。當中國人頂之上,東西當
 卯酉之中,稍南,使見日出入,令與陽經、陰緯相固,如殼之裹黃。南去赤道三十六度,去黃道十二度,去北極五十五度,去南北平各九十一度強。



 赤道單環:外一丈四尺五寸九分,橫八分,厚三分,直徑四尺九寸。赤道者,當天之中,二十八宿之列位也。其本,後魏解蘭所造也。因著雙規,不能運動。臣今所造者,上列周天星度,使轉運隨天,仍度穿一穴,隨穴退交,不有差謬。即知古者秋分,日在角五度,今在軫十三度;冬至,日在牽牛初,今在
 斗十度。擬隨差卻退,故置穴也。傍在卯酉之南,上去天頂三十六度而橫置之。



 黃道單環:外一丈五尺四寸一分,橫八分,厚四分,直徑四尺八寸四分。日之所行,故名黃道。古人知有其事,竟無其器,遂使太陽陟降,積歲有差。月及五星,亦隨日度出入,規制不知準的,斟量為率,疏闊尤多。臣今創置此環,置於赤道環內,仍開合使隨轉運,出入四十八度,而極畫兩方,東西列周天度數,南北列百刻,使見日知時,不有差謬。上列三百六十策,與
 用卦相準,度穿一穴,與赤道相交。



 白道月環:外一丈五尺一寸五分,橫度八分,厚三分,直徑四尺七寸六分。月行有迂曲遲疾,與日行緩急相反。古無其器,今創置於黃道環內,使就黃道為交合,出入六十度,以測每夜行度。上畫周天度數,穿一穴,擬移交會,並用銅鐵為之。



 李淳風《法象志》說有此日月兩環,在旋儀環上。既用玉衡,不得遂於玉衡內別安一尺望筒。運用既難,其器已澀。



 游儀四柱,龍各高四尺七寸。水槽、山各高一尺七寸五
 分。槽長六尺九寸,高廣各四寸。水池深一寸,廣一寸五分。龍者能興雲雨,故以飾柱。柱在四維,龍下有山雲,俱在水平槽上,並銅為之。



 游儀初成,太史所測二十八宿等與《經》同異狀:



 角二星,十二度;赤道黃道度與古同。舊《經》去極九十一度,今則九十三度半。《星經》云:「角去極九十一度,距星正當赤道,其黃道在赤道南,不經角中。」今測角在赤道南二度半,黃道復經角中,即與天象符合。



 亢四星,九度。舊去極八十九度,今九十一度半。氐四星,
 十六度。舊去極九十四度,今九十八度。房四星,五度。舊去極一百八度,今一百一十度半。心三星,五度。舊去極一百八度,今一百一十一度。尾九星,十八度。舊去極一百二十度,一云一百四十一度,今一百二十四度。箕四星,十一度。舊去極一百一十八度,今一百二十度。南斗六星,二十六度。舊去極一百一十六度,今一百一十九度。牽牛六星,八度。舊去極一百六度,今一百四度。須女四星,十二度。舊去極一百度,今一百一度。虛二星,十度。
 舊去極一百四度,今一百一度。北星舊圖入虛宿,今測在須女九度。危三星,十七度。舊去極九十七度,今九十七度。北星舊圖入危宿,今測在虛六度半。室二星,十六度。舊去極八十五度,今八十三度。東壁二星,九度。舊去極八十六度,今八十四度。



 奎十六星,十六度。舊去極七十六度,一云七十度,今七十三度。東壁九度,奎十六度,此錯以奎西大星為距,即損壁二度,加奎二度,今取西南大星為距,即奎、壁各不失本度。婁三星,十三度。舊去
 極八十度,今七十七度。胃三星,十四度。昴七星,十一度。舊去極七十四度,今七十二度。畢八星,十七度。舊去極七十八度,今七十六度。觜觿三度,舊去極八十四度,今八十二度。畢赤道與黃道度同。觜赤道二度,黃道三度。其二宿俱當黃道斜虛。畢有十六度,尚與赤道度同。觜總二度,黃道損加一度,此即承前有誤。今測畢有十七度半,觜觿半度,並依天正。參十星,舊去極九十四度,今九十二度。東井八星,三十三度。舊去極七十度,今六十
 八度。輿鬼五星,舊去極六十八度,今古同也。柳八星,十五度。舊去極七十七度,一云七十九度,今八十度半。柳,合用西頭第三星為距,比來錯取第四星,今依第三星為正。七星十度,舊去極九十一度,一云九十三度,今九十三度半。張六星,十八度。舊去極九十七度,今一百度。張六星,中央四星為硃鳥膆,外二星為翼。比來不取膺前為距,錯取翼星,即張加二度半,七星欠二度半。今依本《經》為定。



 翼二十二星,十八度。舊去極九十七度,今一
 百三度。軫四星,十七度。舊去極九十八度,今一百度。文昌,舊二星在鬼,四星在井;今四星在柳,一星在鬼,一星在井。北斗,魁第一星舊在七星一度,今在張十三度。第二星舊在張二度,今在張十二度半。第三星舊在翼二度,今在翼十三度。第四星舊在翼八度,今在翼十七度太。第五星舊在軫八度,今在軫十度半。第六星舊在角七度,今在角四度少。第七星舊在亢四度,今在角十二度少。天關,舊在黃道南四度,今當黃道。天江,舊在黃道
 外,今當黃道。天囷,舊在赤道外,今當赤道。三臺:上臺舊在井,今測在柳;中臺舊在七星,今在張。建星,舊去黃道北半度,今四度半。天苑,舊在昴、畢,今在胃、昴。王良,舊五星在壁,今四星在奎,一星在壁外。屏,舊在觜,今在畢宿。雲雨,舊在黃道外,今在黃道內七度。雷電,舊在赤道外五度,今在赤道內二度。霹靂,舊五星並在赤道外四度,今四星在赤道內,一星在外。土公吏,舊在赤道外,今在赤道內六度。虛梁,舊在黃道內四度。外屏,舊在黃道外三
 度,今當黃道。八魁,舊九星並在室,今五星在壁,四星在室。長垣,舊當黃道,今在黃道北五度。軍井,準《經》,在玉井東南二度半。天槨,舊在黃道北,今當黃道。天高,舊在黃道外,今當黃道。狗國,舊在黃道外,今當黃道。羅堰,舊當黃道,今在黃道北。



 黃道,春分之日與赤道交於奎五度太;秋分之日交於軫十四度少;冬至之日於斗十度,去赤道南二十四度;夏至之日於井十三度少,去赤道北二十四度。其赤道帶天之中,用分列宿之度;黃道斜運,
 以明日月之行。其冬至,洛下閎起於牛初,張衡等遷於斗度,由每歲差分不及舊次也。



 日晷:《周禮》大司徒,常「以土圭之法測土深,正日景,以求地中。日東則景夕多風,日西則景朝多陰。日至之景尺五寸,謂之地中,天地之所合也,四時之所交也,風雨之所會也,陰陽之所合也。然則百物阜安,乃建王國焉。」鄭氏以為「凡日景於地,千里而差一寸。」「景尺有五寸者,南戴日下萬五千里,地與星辰四游升降於三萬里之中,
 是以半之,得地之中焉。」鄭司農云:「土圭之長尺有五寸,以夏至之日立八尺之表,其景適與土圭等,謂之地中。今潁川陽城為然。



 謹按《南越志》:「宋元嘉中,南征林邑,以五月立表望之,日在表北,影居表南。交州日影覺北三寸,林邑覺九寸一分,所謂開北戶以向日也。」交州,大略去洛九千餘里,蓋水陸曲折,非論圭表所度,惟直考實,其五千乎!開元十二年,詔太史交州測景,夏至影表南長三寸三分,與元嘉中所測大同。然則距陽城而南,使直路應弦,至於日下,蓋不盈五千里也。測影使者大相元
 太云:「交州望極,才出地二十餘度。以八月自海中南望老人星殊高。老人星下,環星燦然,其明大者甚眾,圖所不載,莫辨其名。大率去南極二十度以上,其星皆見。乃古渾天家以為常沒地中,伏而不見之所也。」又按貞觀中,史官所載鐵勒、回紇部在薛延陀之北,去京師六千九百里。又有骨利乾居回紇北方瀚海之北,草多百藥,地出名馬,駿者行數百里。北又距大海,晝長而夕短,既日沒後,天色正曛,煮一羊胛才熟,而東方已曙。蓋近日出
 入之所云。凡此二事,皆書契所未載也。開元十二年,太史監南宮說擇河南平地,以水準繩,樹八尺之表而以引度之。始自滑州白馬縣,北至之晷,尺有五寸七分。自滑州臺表南行一百九十八里百七十九步,得汴州浚儀古臺表,夏至影長一尺五寸微強。又自浚儀而南百六十七里二百八十一步,得許州扶溝縣表,夏至影長一尺四寸四分。又自扶溝而南一百六十里百一十步,至豫州上蔡武津表,夏至影長一尺三寸六分半。大率
 五百二十六里二百七十步,影差二寸有餘。而先儒以為王畿千里,影移一寸,又乖舛而不同矣。



 今以句股圖校之,陽城北至之晷,一尺四寸八分弱;冬至之晷,一丈二尺七寸一分半;春秋分,其長五尺四寸三分。以覆矩斜視,北極出地三十四度四分。凡度分皆以十分為法。自滑臺表視之,高三十五度三分。差陽城九分。自浚儀表視之,高三十四度八分。差陽城四分。自武津表視之,高三十三度八分。差陽城九分。雖秒分稍有盈縮,雖以目校,然大率五百二十六里
 二百七十步而北極差一度半,三百五十一里八十步而差一度。樞極之遠近不同,則黃道之軌景固隨而遷變矣。



 自此為率,推之比歲朗州測影,夏至長七寸七分,冬至長一丈五寸三分,春秋分四尺三寸七分半。以圖測之,定氣長四尺四寸七分。按圖斜視,北極出地二十九度半。差陽城五度二分。蔚州橫野軍測影,夏至長二尺二寸九分,冬至長一丈五尺八寸九分,春秋分長六尺四寸四分半。以圖測之,定氣六尺六寸三分半。按圖斜視,北極出地四十度。差陽城五度二分。凡南北
 之差十度半,其徑三千六百八十里九十步。自陽城至朗州,一千八百二十六里百九十六步,自陽城至蔚州橫野軍,一千八百六十一里二百一十四步。北至之晷,差一尺五寸三分,自陽城至朗州,差七寸二分,自陽城至橫野軍,差八寸。南至之晷,差五尺三寸六分。自陽城至朗州,差二尺一寸八分,自陽城至橫野軍,差三尺一寸八分。率夏至與南方差步,冬至與北方差多。又以圖校安南,日在天頂北二度四分,北極高二十度四分,冬至影長七尺九寸四分,定春秋分影長二尺九寸三分。差陽城十四度三分,其徑五千二十三里。至林邑國,日在天頂北六度六分強,
 北極之高十七度四分,周圓三十五度,常見不隱。冬至影長六尺九寸,其徑六千一百一十二里。假令距陽城而北,至鐵勒之地亦十七度四分,合與林邑與等,則五月日在天頂南二十七度四分,北極之高五十二度,周圓一百四度,常見不隱。北至之齕四尺一寸三分,南至之齕二丈九就十寸六分。定春秋分影長九尺八寸七分。北方其沒地才十五度餘,昏伏於亥之正西,晨見於丑之正東,以里數推之,已在回紇之北,又南距洛陽九千八百一十里,則五月
 極長之日,其夕常明,然則骨利幹猶在其南矣。又先儒以南戴日下萬五千里為句股,斜射陽城為弦,考周徑之率以揆天度,當一千四百六里二十四步有餘。今測日影,距陽城五千餘里,已居戴日之南,則一度之廣,皆宜三分去二,計南北極相去才八萬餘里,其徑五萬餘里,宇宙之廣,豈若是乎?然則王蕃所傳,蓋以管窺天,以蠡測海之義也。古人所以恃句股之術,謂其有徵於近事。顧未知目視不能遠,浸成微分之差,其差不已,遂與
 術錯。如人游於大湖,廣不盈百里,而睹日月朝夕出入湖中。及其浮於巨海,不知幾千萬里,猶睹日月朝出其中,夕入其中。若於朝夕之際,俱設重差而望之,必將小大同術而不可分矣。



 夫橫既有之,縱亦宜然。假令設兩表,南北相距十里,其崇皆數十里,若置火炬於南表之端,而植八尺之木於其下,則當無影。試從南表之下,仰望北表之端,必將積微分之差,漸與南表參合。表首參合,則置炬於其上,亦當無影矣。又置火炬於北表之
 端,而植八尺之木於其下,則當無影。試從北表之下,仰望南表之端,又將積微分之差,漸與北表參合。表首參合,則置炬於其上,亦當無影矣。復於二表之間,相距各五里,更植八尺之木,仰而望之,則表首環屈而相會。若置火炬於兩表之端,皆當無影。夫數十里之高與十里之廣,然則斜射之影與仰望不殊。今欲求其影差以推遠近高下,猶尚不可知也;而況稽周天積里之數於不測之中,又可必乎!假令學者因二十里之高以立句股
 之術,尚不知其所以然,況八尺之木乎!原人所以步圭景之意,將欲節宣和氣,輔相物宜,而不在於辰次之周徑;其所以重歷數之意,將欲敬授人時,欽若乾象,而不在於渾、蓋之是非。若乃述無稽之談於視聽之所不及,則君子闕疑而不質,仲尼慎言而不論也。而或者各守所傳之器以述天體,謂渾元可任數而測,大象可運算而窺,終以六家之說,迭為矛盾。今誠以為蓋天,則南方之度漸狹;以為渾天,則北方之極浸高。此二者,又渾、蓋
 之家未能有以通其說也。由是而觀,則王仲任、葛稚川之徒,區區於異同之辨,何益人倫之化哉!



 又凡日晷差,冬夏至不同,南北亦異,而先儒一以里數齊之,喪其事實。沙門一行因修《大衍圖》,更為《覆矩圖》,自丹穴以暨幽都之地,凡為圖二十四,以考日蝕之分數,知夜漏之短長。今載諸州測景尺寸如左:



 林邑國,北極高十七度四分。冬至影在表北六尺九寸。定春秋分影在表北二尺八寸五分,夏至影在表南五寸七分。安南都護府,北極高二十六度六分。
 冬至影在表北七尺九寸四分。定春秋分影在表北二尺九寸三分,夏至影在表南三寸三分。朗州武陵縣,北極高二十九度五分。冬至影在表北一丈五寸三分。定春秋分影在表北四尺三寸七分半,夏至影在表北七寸七分。襄州。恆春分影在表北四尺八寸。蔡州上蔡縣武津館,北極高三十三度八分。冬至影在表北一丈二尺三寸八分。定春秋分影在表北五尺二寸八分,夏至影在表北一尺三寸六分半。許州扶溝,北極高三十四度三分。冬至影在表北一丈二尺五寸三分。定春秋分影在表北五尺三寸七分,夏至影在表北一尺四寸四分。汴州浚儀太嶽臺,北極高三十四度八分。冬至影在表北一丈二尺八寸五分。定春秋分影在表北五尺五寸,夏至影在表北一尺五寸三分。滑州白馬,
 北極高三十五度三分。冬至影在表北一丈三尺。定春秋分影在表北五尺三寸六分,夏至影在表北一尺五寸七分。太原府。恆春分影在表北六尺。蔚州橫野軍,北極高四十度。冬至影在表北一丈五尺八寸九分。定春秋分影在表北六尺六寸三分,夏至影在表北二尺二寸九分。



\end{pinyinscope}