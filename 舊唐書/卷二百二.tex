\article{卷二百二}

\begin{pinyinscope}

 ○王績田游巖史德義王友貞盧鴻一王希夷衛大經李元愷王守慎徐仁紀孫處玄白履忠
 王遠知潘師正劉道合司馬承禎吳筠孔述睿子敏行陽城崔覲



 前代賁丘園,招隱逸,所以重貞退之節,息貪競之風。故蒙叟矯《讓王》之篇,玄晏立高人之傳,箕、潁之跡,粲然可觀。而漢二龔之流,乃心王室,不事莽朝,忍渴盜泉,本非絕俗,甚可嘉也。皇甫謐、陶淵明慢世逃名,放情肆志,逍遙泉石,無意於出處之間,又其善也。即有身在江湖之上,心游魏闕之下,托薛蘿以射利,假巖壑以釣名,退無
 肥遁之貞,進乏濟時之具,《山移》見誚,海鳥興譏,無足多也。阮嗣宗傲世佯狂,王無功嗜酒放蕩,才不足而智有餘,傷其時而晦其用,深識之士也。高宗天後,訪道山林,飛書巖穴,屢造幽人之宅,堅回隱士之車。而游巖、德義之徒,所高者獨行;盧鴻一、承禎之比,所重者逃名。至於出處語默之大方,未足與議也。今存其舊說,以備雜篇。



 王績,字無功,絳州龍門人。少與李播、呂才為莫逆之交。隋大業中,應孝悌廉潔舉,授揚州六合縣丞。非其所好,
 棄官還鄉里。績河渚中先有田數頃,鄰渚有隱士仲長子先,服食養性,績重其真素,願與相近,乃結廬河渚,以琴酒自樂。嘗游北山,因為《北山賦》以見志,詞多不載。



 績嘗躬耕於東皋,故時人號東皋子。或經過酒肆,動經數日,往往題壁作詩,多為好事者諷詠。貞觀十八年卒。臨終自克死日,遺命薄葬,兼預自為墓志。有文集五卷。又撰《隋書》,未就而卒。



 兄通,字仲淹,隋大業中名儒,號文中子,自有傳。



 田游巖,京兆三原人也。初,補太學生,後罷歸,游於太白山。每遇林泉會意,輒留連不能去。其母及妻子並有方外之志,與游巖同游山水二十餘年。後入箕山,就許由廟東築室而居,自稱「許由東鄰」。調露中,高宗幸嵩山,遣中書侍郎薛元超就問其母。游巖山衣田冠出拜,帝令左右扶止之。謂曰:「先生養道山中,比得佳否?」游巖曰:「臣泉石膏肓,煙霞痼疾,既逢聖代,幸得逍遙。」帝曰:「朕今得卿,何異漢獲四皓乎?」薛元超曰:「漢高祖欲廢嫡立庶,黃、
 綺方來,豈如陛下崇重隱淪,親問巖穴!」帝甚歡,因將游巖就行宮,並家口給傳乘赴都,授崇文館學士,令與太子少傅劉仁軌談論。帝後將營奉天宮於嵩山,游巖舊宅,先居宮側。特令不毀,仍親書題額懸其門,曰「隱士田游巖宅」。文明中,進授朝散大夫,拜太子洗馬,垂拱初,坐與裴炎交結,特放還山。



 史德義,蘇州昆山人也。咸亨初,隱居武丘山,以琴書自適。或騎牛帶瓢,出入郊郭廛市,號為逸人。高宗聞
 其名,征赴洛陽。尋稱疾東歸。公卿已下,皆賦詩餞別,德義亦以詩留贈,其文甚美。天授初,江南道宣勞使、文昌左丞周興表薦之,則天征赴都,詔曰:「蘇州隱士史德義,志尚虛玄,業履貞確,謙沖彰於里騕,孝友表於閨庭。固辭徵闢,長往嚴陵之瀨;多謝簪裾,高蹈愚公之穀。博聞強識,說《禮》敦《詩》,繕性丘園,甘心畎畝。朕承天革命,建極開階,寤寐星云,物色林壑。順禎期而捐薛帶,應休運而解荷裳;粵自海隅,來游魏闕,行藏之理斯得,去就之節無違。
 風操可嘉,啟沃攸佇,特宜優獎,委以諫曹。可朝散大夫。」後周興伏誅,德義坐為所薦免官。以朝散大夫放歸丘壑,自此聲譽稍減於隱居之前。



 王友貞,懷州河內人也。父知敬,則天時麟臺少監,以工書知名。友貞弱冠時,母病篤,醫言唯啖人肉乃差。友貞獨念無可求治,乃割股肉以飴親,母病尋差。則天聞之,令就其家驗問,特加旌表。友貞素好學,讀《九經》皆百遍,訓誨子弟,如嚴君焉。口不言人過,尤好釋典;屏絕亶味,
 出言未曾負諾,時論以為真君子也。



 長安年,歷任長水令。後罷歸田里。中宗在春宮,召為司議郎,不就。神龍初,又拜太子中舍,仍令所司以禮徵赴。及至,固以疾辭。詔曰:



 敦夷齊之行,可以激貪;尚顏閔之道,用能勸俗。新除太子中舍人王友貞,德義泉藪,人倫茂異,孝始於事親,信表於行己。富有文史,廉於財貨,久歷官政,累聞課績。有古人之風,保君子之德。乃抗志塵外,棲情物表,深歸解脫之門,誓守薰修之誡。頃加徵命,作護儲闈,固在辭
 榮,累陳情懇。堅持凈義,不登於車服;惟悅禪綱,味靡求於珍饌。朕方崇獎廉退,懲抑澆浮,雖思廊廟之賢,豈違山林之願,宜加優秩,仍遂雅懷。可太子中舍人員外置,給全祿以畢其身,任其在家修道。仍令所在州縣存問,四時送祿至其住所。



 玄宗在東宮,又表請禮徵之,以年老,竟辭疾不赴。年九十餘,開元四年卒。時下制曰:「貴德尊賢,飾終念遠,此聖人所以治天下、厚風俗也。王友貞稟氣元精,游心大樸。孝惟不匱,獨貫於神明;道則難
 名,高謝於人代。言念錫類,方期鎮俗,遽爾凋殂,良深愍悼。生無大位,雖隔外臣之儀,歿有餘榮,宜贈上卿之服。可贈銀青光祿大夫,仍委本縣令長特加吊祭。」



 盧鴻一,字浩然,本範陽人,徙家洛陽。少有學業,頗善籀篆楷隸,隱於嵩山。開元初,遣幣禮再徵不至。五年,下詔曰:



 朕以寡薄,忝膺大位。嘗恨玄風久替,淳化未升,每用翹想遺賢,冀聞上皇之訓。以卿黃中通理,鉤深詣微,窮太一之道,踐中庸之德,確乎高尚,足侔古人。故比下征
 書,佇諧善績,而每輒托辭,拒違不至。使朕虛心引領,於今數年,雖得素履幽人之貞,而失考父滋恭之命。豈朝廷之故與生殊趣耶?將縱欲山林不能反乎?禮有大倫,君臣之義,不可廢也!今城闕密邇,不足為難,便敕齎束帛之貺,重宣斯旨,想有以翻然易節,副朕意焉!



 鴻一赴征。六年,至東都,謁見不拜。宰相遣通事舍人問其故,奏曰:「臣聞老君言,禮者,忠信之所薄,不足可依。山臣鴻一敢以忠信奉見。」上別召升內殿,賜之酒食。詔曰:「盧鴻一
 應闢而至,訪之至道,有會淳風,舉逸人,用勸天下。特宜授諫議大夫。」鴻一固辭,又制曰:



 昔在帝堯,全許由之節;糸面惟大禹,聽伯成之高。則知天子有所不臣,諸侯有所不友,《遁》之時義大矣哉!嵩山隱士盧鴻一,抗跡幽遠,凝情篆素;隱居以求其志,行義以達其道;雲臥林壑,多歷年載。傳不云乎:「舉逸人,天下之人歸心焉。」是乃飛書巖穴,備禮徵聘,方佇獻替,式弘政理。而矯然不群,確乎難拔,靜已以鎮其操,洗心以激其流,固辭榮寵,將厚風俗,
 不降其志,用保厥躬。會稽嚴陵,未可名屈;太原王霸,終以病歸。宜以諫議大夫放還山。歲給米百碩、絹五十匹,充其藥物,仍令府縣送隱居之所。若知朝廷得失,具以狀聞。



 將還山,又賜隱居之服,並其草堂一所,恩禮甚厚。



 王希夷,徐州滕縣人也。孤貧好道。父母終,為人牧羊,收傭以供葬。葬畢,隱於嵩山,師道士黃頤,向四十年,盡能傳其閉氣導養之術。頤卒,更居兗州徂來山中,與道士劉玄博為棲遁之友。好《易》及《老子》,嘗餌松柏葉及雜花
 散。



 景龍中,年七十餘,氣力益壯。刺史盧齊卿就謁致禮,因訪以字人之術,希夷曰:「孔子稱『己所不欲,勿施於人』,可以終身行之矣。」及玄宗東巡,敕州縣以禮徵,召至駕前,年已九十六。上令中書令張說訪以道義,宦官扶入宮中,與語甚悅。



 開元十四年,下制曰:「徐州處士王希夷,絕學棄智,抱一居貞,久謝囂塵,獨往林壑。朕為封巒展禮,側席旌賢,賁然來思,克應嘉召。雖紆綺季之跡,已過伏生之年,宜命秩以尊儒,俾全高於尚齒。可朝散大夫,
 守國子博士,聽致仕還山。州縣春秋致束帛酒肉,仍賜衣一副、絹一百匹。」尋壽終。



 自則天、中宗已後,有蒲州人衛大經、邢州人李元愷,皆潔志不仕;蒲州人王守慎、常州人徐仁紀、潤州人孫處玄,皆退身辭職,為時所稱。



 衛大經者,篤學善《易》,口無二言。則天降詔征之,辭疾不赴。與魏州人夏侯乾童有舊,聞乾童母卒,徒步往吊之。鄉人止之曰:「當夏溽暑,豈可步涉千里,致書可也。」大經曰:「尺書無能盡意。」遂行。至魏州,會乾童出行,大經造門
 設席,行吊禮,不訊其家人而還。開元初,畢構為刺史,謂解令孔慎言曰:「衛生德厚,宜有旌異。古人式乾木之閭,禮賢故也。」慎言造門就謁,時大經已年老,辭疾不見。嘗預筮死日,鑿墓自為志文,果如筮而終。



 李元愷者,博學善天文律歷,然性恭慎,口未嘗言人之過。鄉人宋璟,年少時師事之。及璟作相,使人遺元愷束帛,將薦舉之,皆拒而不答。景龍中,元行沖為洺州刺史,邀元愷至州,問以經義,因遺衣服。元愷辭曰:「微軀不宜
 服新麗,但恐不能勝其美以速咎也。」行沖乃以泥塗污而與之,不獲已而受。及還,乃以己之所蠶素絲五兩以酬行沖,曰:「義不受無妄之財。」先是,定州人崔元鑒明《三禮》,鄉人張易之寵幸用事,薦之。起家拜朝散大夫,致仕於家,在鄉請半祿。元愷誚之曰:「無功受祿,災也。」元愷年八十餘,壽終。



 王守慎者,有美名。垂拱中為監察御史。時羅織事起,守慎舅秋官侍郎張知默推詔獄,奏守慎同知其事,守慎
 以疾辭,因請為僧。則天初甚怪之;守慎陳情,詞理甚高,則天欣然從之,賜號法成。識鑒高雅,為時賢所重。以壽終。



 徐仁紀者,聖歷中徵拜左拾遺。三上書論得失,不納。謂人曰:「三諫不聽,可去矣!」遂移病歸鄉里。神龍初,宣慰使舉仁紀之行可以激俗,又徵拜左補闕。三上書,又不省,乃詣執政求出。俄授靈昌令。妻子不之官,廨舍唯衣履及書疏而已,餘無所蓄。



 孫處玄,長安中徵為左拾遺。頗善屬文,嘗恨天下無書以廣新聞。神龍初,功臣桓彥範等用事,處玄遺彥範書,論時事得失,彥範竟不用其言,乃去官還鄉里。以病卒。



 白履忠,陳留浚儀人也。博涉文史。嘗隱居於古大梁城,時人號為梁丘子。景雲中,徵拜校書郎。尋棄官而歸。



 開元十年,刑部尚書王志愔表薦履忠隱居讀書,貞苦守操,有古人之風,堪代褚無量、馬懷素入閣侍讀。十七年,國子祭酒楊瑒籙又表薦履忠堪為學官,乃徵赴京師。及
 至,履忠辭以老病,不任職事。詔曰:「處士前秘書省校書郎白履忠,學優緗簡,道賁丘園,探賾以見其微,隱居能達其志。故以汲引洙、泗,物色夷門,素風自高,玄冕非貴。幾杖雲暮,章秩宜加,俾承禮命之優,式副寵賢之美。可朝散大夫。」



 履忠尋表請還鄉,手詔曰:「孝悌立身,靜退放俗,年過從耄,不雜風塵。盛德予聞,通班是錫,豈惟旌賁山藪,實欲獎勸人倫。且游上京,徐還故里。」乃停留數月而歸。履忠鄉人左庶子吳兢謂履忠曰:「吾子家室屢空,
 竟不沾斗米匹帛,雖得五品,何益於實也?」履忠欣然曰:「往歲契丹入寇,家家盡著括排門夫,履忠特以少讀書籍,縣司放免,至今惶愧。今雖不得,且是吾家終身高臥,免徭役,豈易得也!」尋壽終。著《三玄精辯論》一卷,注《老子》及《黃庭內景經》,有文集十卷。



 道士王遠知,瑯邪人也。祖景賢,梁江州刺史。父曇選,陳揚州刺史。遠知母,梁駕部郎中丁超女也。嘗晝寢,夢靈鳳集其身,因而有娠,又聞腹中啼聲,沙門寶志謂曇選
 曰:「生子當為神仙之宗伯也。」



 遠知少聰敏,博綜群書。初入茅山,師事陶弘景,傳其道法。後又師事宗道先生臧兢。陳主聞其名,召入重陽殿,令講論,甚見嗟賞。及隋煬帝為晉王,鎮揚州,使王子相、柳顧言相次召之。遠知乃來謁見,斯須而須發變白,晉王懼而遣之,少頃又復其舊。煬帝幸涿郡,遣員外郎崔鳳舉就邀之,遠知見於臨朔宮,煬帝親執弟子之禮,敕都城起玉清玄壇以處之。及幸揚州,遠知諫不宜遠去京國,煬帝不從。



 高祖之龍
 潛也,遠知嘗密傳符命。武德中,太宗平王世充,與房玄齡微服以謁之。遠知迎謂曰:「此中有聖人,得非秦王乎?」太宗因以實告。遠知曰:「方作太平天子,願自惜也。」太宗登極,將加重位,固請歸山。至貞觀九年,敕潤州於茅山置太受觀,並度道士二十七人。降璽書曰:「先生操履夷簡,德業沖粹,屏棄塵雜,棲志虛玄,吐故納新,食芝餌術,念眾妙於三清之表,返華發於百齡之外,道邁前烈,聲高自古。非夫得秘訣於金壇,受幽文於玉笈者,其孰能
 與此乎!朕昔在籓朝,早獲問道,眷言風範,無忘寤寐。近覽來奏,請歸舊山,已有別敕,不違高志,並許置觀,用表宿心。未知先生早晚已屆江外,所營棟宇,何當就功?佇聞委曲,副茲引領。近已令太史薛頤等往詣,令宣朕意。」



 其年,遠知謂弟子籓師正曰:「吾見仙格,以吾小時誤損一童子吻,不得白日升天。見署少室伯,將行在即。」翌日,沐浴,加冠衣,焚香而寢。卒,年一百二十六歲。調露二年,追贈遠知太中大夫,謚曰升真先生。則天臨朝,追贈金
 紫光祿大夫。天授二年,改謚曰升玄先生。



 潘師正,趙州贊皇人也。少喪母,廬於墓側,以至孝聞。大業中,度為道士,師事王遠知,盡以道門隱訣及符籙授之。師正清凈寡欲,居於嵩山之逍遙谷,積二十餘年,但服松葉飲水而已。高宗幸東都,因召見與語,問師正:「山中有何所須?」師正對曰:「所須松樹清泉,山中不乏。」高宗與天后甚尊敬之,留連信宿而還。尋敕所司於師正所居造崇唐觀,嶺上別起精思觀以處之。初置奉天宮,
 帝令所司於逍遙谷口特開一門,號曰仙游門;又於苑北面置尋真門,皆為師正立名焉。時太常奏新造樂曲,帝又令以《祈仙》、《望仙》、《翹仙》為名。前後贈詩,凡數十首。



 師正以永淳元年卒,時年九十八。高宗及天後追思不已,贈太中大夫,賜謚曰體玄先生。



 道士劉道合者,陳州宛丘人。初與潘師正同隱於嵩山。高宗聞其名,令於隱所置太一觀以居之。召入宮中,深尊禮之。及將封太山,屬久雨,帝令道合於儀鸞殿作止
 雨之術,俄而霽朗,帝大悅。又令道合馳傳先上太山,以祈福祐。前後賞賜,皆散施貧乏,未嘗有所蓄積。



 高宗又令道合合還丹,丹成而上之。咸亨中,卒。及帝營奉天宮,遷道合之殯室,弟子開棺將改葬,其尸惟有空皮,而背上開拆,有似蟬蛻,盡失其齒骨,眾謂尸解。高宗聞之,不悅,曰:「劉師為我合丹,自服仙去。其所進者,亦無異焉!」



 道士司馬承禎,字子微。河內溫人,周晉州刺史、瑯邪公裔玄孫。少好學,薄於為吏,遂為道士。事籓師正,傳其符
 籙及闢穀導引服餌之術。師正特賞異之,謂曰:「我自陶隱居傳正一之法,至汝四葉矣。」承禎嘗遍游名山,乃止於天臺山。則天聞其名,召至都,降手敕以贊美之。及將還,敕麟臺監李嶠餞之於洛橋之東。



 景雲二年,睿宗令其兄承禕就天臺山追之至京,引入宮中,問以陰陽術數之事。承禎對曰:「道經之旨:『為道日損,損之又損,以至於無為。』且心目所知見者,每損之尚未能已,豈復攻乎異端,而增其智慮哉!」帝曰:「理身無為,則清高矣!理國無
 為,如何?」對曰:「國猶身也。《老子》曰:『游心於淡,合氣於漠,順物自然而無私焉,而天下理。』《易》曰:『聖人者,與天地合其德。』是知天不言而信,不為而成。無為之旨,理國之道也。」睿宗嘆息曰:「廣成之言,即斯是也!」承禎固辭還山,仍賜寶琴一張,及霞紋帔而遣之,朝中詞人贈詩者百餘人。



 開元九年,玄宗又遣使迎入京,親受法籙,前後賞賜甚厚。十年,駕還西都,承禎又請還天臺山,玄宗賦詩以遣之。十五年,又召至都。玄宗令承禎於王屋山自選形勝,
 置壇室以居焉。承禎因上言:「今五嶽神祠,皆是山林之神,非正真之神也。五岳皆有洞府,各有上清真人降任其職,山川風雨,陰陽氣序,是所理焉。冠冕章服,佐從神仙,皆有名數。請別立齋祠之所。」玄宗從其言,因敕五嶽各置真君祠一所,其形象制度,皆令承禎推按道經,創意為之。



 承禎頗善篆隸書,玄宗令以三體寫《老子經》,因刊正文句,定著五千三百八十言為真本以奏上之。以承禎王屋所居為陽臺觀,上自題額,遣使送之。賜絹三
 百匹,以充藥餌之用。俄又令玉真公主及光祿卿韋縚至其所居,修金籙齋,復加以錫齎。



 是歲,卒於王屋山,時年八十九。其弟子表稱;「死之日,有雙鶴饒壇,及白雲從壇中湧出,上連於天,而師容色如生。」玄宗深嘆之,乃下制曰:「混成不測,入寥自化。雖獨立有象,而至極則冥。故王屋山道士司馬子微,心依道勝,理會玄遠,遍游名山,密契仙洞。存觀其妙,逍遙得意之場;亡復其根,宴息無何之境。固以名登真格,位在靈官。林壑未改,遐霄已曠;
 言念高烈,有愴於懷。宜贈徽章,用光丹籙。可銀青光祿大夫,號真一先生。」仍為親制碑文。



 吳筠,魯中之儒士也。少通經,善屬文,舉進士不第。性高潔,不奈流俗。乃入嵩山,依潘師正為道士,傳正一之法,苦心鉆仰,乃盡通其術。開元中,南游金陵,訪道茅山。久之,東游天臺。



 筠尤善著述,在剡與越中文士為詩酒之會,所著歌篇,傳於京師。玄宗聞其名,遣使徵之。既至,與語甚悅,令待詔翰林。帝問以道法,對曰:「道法之精,無如
 五千言,其諸枝詞蔓說,徒費紙札耳!」又問神仙修煉之事,對曰:「此野人之事,當以歲月功行求之,非人主之所宜適意。」每與緇黃列坐,朝臣啟奏,筠之所陳,但名教世務而已,間之以諷詠,以達其誠。玄宗深重之。



 天寶中,李林甫、楊國忠用事,綱紀日紊。筠知天下將亂,堅求還嵩山。累表不許,乃詔於岳觀別立道院。祿山將亂,求還茅山,許之。既而中原大亂,江淮多盜,乃東游會稽。嘗於天臺剡中往來,與詩人李白、孔巢父詩篇酬和,逍遙泉石,
 人多從之。竟終於越中。文集二十卷,其《玄綱》三篇、《神仙可學論》等,為達識之士所稱。



 筠在翰林時,特承恩顧,由是為群僧之所嫉。驃騎高力士素奉佛,嘗短筠於上前,筠不悅,乃求還山。故所著文賦,深詆釋氏,亦為通人所譏。然詞理宏通,文彩煥發,每制一篇,人皆傳寫。雖李白之放蕩,杜甫之壯麗,能兼之者,其唯筠乎!



 孔述睿,趙州人也。曾祖昌宇,膳部郎中。祖舜,監察御史。父齊參,寶鼎令。述睿少與兄克符、弟克讓,皆事親以孝
 聞。既孤,俱隱於嵩山。述睿好學不倦,大歷中,轉運使劉晏累表薦述睿有顏、閔之行,游、夏之學。代宗以太常寺協律郎征之。轉國子博士,歷遷尚書司勛員外郎、史館修撰。述睿每加恩命,暫至朝廷謝恩,旬日即辭疾,卻歸舊隱。



 德宗踐祚,以諫議大夫銀章硃綬,命河南尹趙惠伯齎詔書、玄纁束帛,就嵩山以禮徵聘。述睿既至,召對於別殿,特賜第宅,給以廄馬,兼為皇太子侍讀。旬日後累表固辭,依前乞還舊山。詔報之曰:「卿懷伊摯匡時之
 道,有廣成嘉遁之風。養素丘園,屢辭命秩。朕以峒山問道,渭水求師,亦何必務執勞謙,固求退讓!無違朕旨,且啟乃心。」述睿既懇辭不獲,方就職。久之,改秘書少監,兼右庶子,再加史館修撰。述睿精於地理,在館乃重修《地理志》,時稱詳究。



 而又性謙和退讓,與物無競,每親朋集會,嘗恂恂然似不能言者,人皆敬之。時令狐峘亦充修撰,與述睿同職,多以細碎之事侵述睿,述睿皆讓之,竟不與爭,時人稱為長者。



 貞元四年,命齎詔並御饌、衣服
 數百襲,往平涼盟會處祭陷歿將士骸骨,以述睿性精愨故也。九年,以疾上表,請罷官。詔不許,報之曰:「朕以卿德重朝端,行敦風俗,不言之教,所賴攸深,未依來請,想宜悉也。」



 述睿再三上表,方獲允許,乃以太子賓客賜紫金魚袋致仕,放還鄉里。仍賜帛五十匹,衣一襲。故事,致仕還鄉者皆不給公乘,德宗優寵儒者,特命給而遣之。貞元十六年九月卒,年七十一。贈工部尚書。子敏行。



 敏行,字至之,舉進士,元和五年禮部侍郎崔樞下擢第。呂
 元膺廉問岳鄂,闢為賓佐。丁母憂而罷。後元膺為東都留守,移鎮河中。敏行皆從之。十四年,入為右拾遺,遷左補闕。長慶中,為起居郎,改左司員外郎,歷司勛郎中,充集賢殿學士,遷吏部郎中,俄拜諫議大夫。上疏論興元監軍楊叔元陰激募卒為亂,殺節度使李絳。人不敢發其事,敏行上表極諍之,故叔元得罪,時論稱美。



 敏行名臣之子,少而修潔,為人所稱;及游宦,與當時豪俊為友。雖名華為一時冠,而貞規雅操,與父遠矣。大和九年正
 月卒,年四十九,贈尚書工部侍郎。



 陽城,字亢宗,北平人也。代為宦族。家貧不能得書,乃求為集賢寫書吏,竊官書讀之,晝夜不出房;經六年,乃無所不通。既而隱於中條山。遠近慕其德行,多從之學。閭里相訟者,不詣官府,詣城請決。陜虢觀察使李泌聞其名,親詣其里訪之,與語甚悅。泌為宰相,薦為著作郎。德宗令長安縣尉楊寧齎束帛詣夏縣所居而召之,城乃衣褐赴京,上章辭讓。德宗遣中官持章服衣之,而後詔,
 賜帛五十匹。尋遷諫議大夫。



 初未至京,人皆想望風彩,曰:「陽城山人能自刻苦,不樂名利,今為諫官,必能以死奉職。」人咸畏憚之。及至,諸諫官紛紜言事,細碎無不聞達,天子益厭苦之。而城方與二弟及客日夜痛飲,人莫能窺其際,皆以虛名譏之。有造城所居,將問其所以者。城望風知其意,引之與坐,輒強以酒。客辭,城輒引自飲;客不能已,乃與城酬酢。客或時先醉,臥席上,城或時先醉,臥客懷中,不能聽客語。約其二弟云:「吾所得月俸,汝
 可度吾家有幾口,月食米當幾何,買薪、菜、鹽凡用幾錢,先具之,其餘悉以送酒媼,無留也。」未嘗有所蓄積。雖所服用有切急不可闕者,客稱某物佳可愛,城輒喜,舉而授之。有陳某者,候其始請月俸,常往稱其錢帛之美,月有獲焉。



 時德宗在位,多不假宰相權,而左右得以因緣用事。於是裴延齡、李齊運、韋渠牟尋以奸佞相次進用,誣譖時宰,毀詆大臣,陸贄等咸遭枉黜,無敢救者。城乃伏閣上疏,與拾遺王仲舒共論延齡奸佞,贄等無罪。德
 宗大怒,召宰相入議,將加城罪。時順宗在東宮,為城獨開解之,城賴之獲免。於是金吾將軍張萬福聞諫官伏閣諫,趨往,至延英門,大言賀曰:「朝廷有直臣,天下必太平矣!」乃造城及王仲舒等曰:「諸諫議能如此言事,天下安得不太平?」已而連呼「太平,太平」。



 萬福武人,年八十餘,自此名重天下。時朝夕欲相延齡,城曰:「脫以延齡為相,城當取白麻壞之。」竟坐延齡事改國子司業。



 城既至國學,乃召諸生,告之曰:「凡學者所以學,為忠與孝也。諸生
 寧有久不省其親者乎?」明日,告城歸養者二十餘人。



 有薛約者,嘗學於城,性狂躁,以言事得罪,徙連州,客寄無根蒂。臺吏以蹤跡求得之於城家。城坐臺吏於門,與約飲酒訣別,涕泣送之郊外。德宗聞之,以城黨罪人,出為道州刺史。太學生王魯卿、季償等二百七十人詣闕乞留,經數日,吏遮止之,疏不得上。



 在道州,以家人法待吏人,宜罰者罰之,宜賞者賞之,不以簿書介意。道州土地產民多矮,每年常配鄉戶,竟以其男號為「矮奴」。城下車,禁
 以良為賤,又憫其編甿歲有離異之苦,乃抗疏論而免之,自是乃停其貢。民皆賴之,無不泣荷。前刺史有贓罪。觀察使方推鞫之,吏有幸於前刺史者,拾其不法事以告,自為功,城立杖殺之。賦稅不登,觀察使數加誚讓。州上考功第,城自署其第曰:「撫字心勞,徵科政拙,考下下。」觀察使遣判官督其賦,至州,怪城不出迎,以問州吏。吏曰:「刺史聞判官來,以為有罪,自囚於獄,不敢出。」判官大驚,馳入謁城於獄,曰:「使君何罪!某奉命來候安否耳。」留
 一二日未去,城因不復歸館;門外有故門扇橫地,城晝夜坐臥其上,判官不自安,辭去。其後又遣他判官往按之,他判官義不欲按,乃載妻子行,中道而自逸。



 順宗即位,詔征之,而城已卒。士君子惜之,是歲四月,賜其家錢二百貫文,仍令所在州縣給遞,以喪歸葬焉。



 崔覲,梁州城固人。為儒不樂仕進,以耕稼為業。老而無子,乃以田宅家財分給奴婢,令各為生業。覲夫妻遂隱於城固南山,家事不問。約奴婢遞過其舍,至則供給酒
 食而已。夫婦林泉相對,以嘯詠自娛。山南西道節度使鄭餘慶高其行,闢為節度參謀,累邀方至府第。為吏無方略,苦不達人事,餘慶以長者優容之。太和八年,左補闕王直方上疏論事,得召見,文宗便殿訪以時事。直方亦興元人,與覲城固山為鄰,是日因薦覲有高行,詔以起居郎征之。覲辭疾不起。卒於山。



 贊曰:高士忘懷,不隱不顯。依隱釣名,真風漸鮮。結廬泉石,投紱市朝。心無出處,是曰逍遙。



\end{pinyinscope}