\article{卷一百二十六}

\begin{pinyinscope}

 ○張獻
 誠弟獻恭獻甫獻恭子煦路嗣恭子恕曲環崔漢衡楊朝晟樊澤李叔明裴胄



 張獻誠,陜州平陸人,幽州節度使、幽州大都督府長史守珪之子也。天寶末,陷逆賊安祿山,受偽官;連陷史思
 明,為思明守汴州,統逆兵數萬。寶應元年冬,東都平,史朝義逃歸汴州,獻誠不納,舉州及所統兵歸國,詔拜汴州刺史,充汴州節度使。逾年來朝,代宗寵賜甚厚。三遷檢校工部尚書,兼梁州刺史,充山南西道觀察使。廣德二年十月,擒南山賊帥高玉以獻。永泰二年正月,獻名馬二、絲絹雜貨共十萬匹。是月,兼充劍南東川節度觀察使,封鄧國公。西川崔旰殺郭英乂,獻誠率眾戰於梓州,為旰所敗,獻誠僅以身免。大歷二年四月,獻誠以
 疾上表乞歸私第,仍薦堂弟試太常卿兼右羽林將軍獻恭以自代。詔許之,以獻誠檢校戶部尚書,知省事。八月,獻誠以疾抗疏辭官,無幾,卒於私第。



 獻恭,守珪之弟守瑜子。累以軍功官至試太常卿,兼右羽林將軍,代獻誠為梁州刺史、兼御史中丞,充山南西道節度觀察使。大歷十二年七月,獻恭破吐蕃萬餘眾於岷州。建中二年正月,加檢校兵部尚書,為東都留守。三年正月,為太府卿、容州刺史、本管經略招討使。四年七月,與渾瑊、盧杞、
 司農卿段秀實與吐蕃尚結贊築壇於京城之西會盟,如清水之儀。興元元年六月,轉檢校吏部尚書,仍與一子正員官。盧杞移饒州刺史,給事中袁高論其不可。獻恭因入對紫宸殿,上言:「高所奏至當,臣恐煩聖聽,不敢縷陳其事。」德宗不悟,獻恭復奏曰:「袁高是陛下一良臣,望特優異。」德宗顧謂宰臣李勉等曰:「朕欲授杞一小州刺史可乎?」對曰:「陛下授大州亦可,其奈士庶失望何!」獻恭守正不撓也如此。



 獻甫,守珪弟左武衛將軍、贈戶部
 尚書守琦之子。獻甫少隨諸兄從軍,初為偏裨,以軍功累授試光祿卿、殿中監、河中節度副元帥都知兵馬使,檢校兵部尚書、兼御史大夫。建中初,從節度使賈耽征梁崇義於襄、漢,以功加太子詹事。及幸奉天、興元,獻甫首至,從渾瑊征討有功,及復京邑,入為金吾將軍。時李懷光未平,吐蕃侵擾西邊,獻甫領禁軍出鎮咸陽,凡累年,軍民悅之。貞元四年,遷檢校刑部尚書,兼邠州刺史、邠寧慶節度觀察使。乃於彭原置義倉,方渠、馬嶺等縣
 選險要之地以為烽堡。又上疏請復鹽州及洪門、洛原等鎮,各置兵防以備蕃寇,朝廷從之。貞元四年九月,吐蕃將尚志董星、論莽羅等寇寧州,獻甫率眾御之,斬首百餘級,吐蕃遁邊城。貞元十二年,加檢校左僕射。五月丙申卒,年六十一,廢朝三日,贈司空,賻物有差。



 獻恭子煦,嘗隨獻甫征討,積戰功累遷至夏州節度使。元和八年十二月,振武軍逐出節度使李進賢而屠其家,殺判官嚴澈。憲宗怒,遣煦以夏州兵二千人赴振武,仍許以
 便宜擊斷。九年正月,賜絹三萬匹以助軍資。河東節度使王鍔遣兵五千會煦於善羊柵,詔煦入振武,誅作亂蘇國珍等二百五十三人乃定。是歲十二月卒,贈太子太保。



 路嗣恭,京兆三原人。始名劍客,歷仕郡縣,有能名,累至神烏令,考績上上,為天下最,以其能,賜名嗣恭。歷工部尚書、兼御史大夫、靈州大都督府長史,充關內副元帥郭子儀副使,知朔方節度營田押諸蕃部落等使,嗣恭
 披荊棘以守之。大將御史中丞孫守亮握重兵,倔強不受制,嗣恭稱疾召至,因殺之,威信大行。永泰三年,檢校刑部尚書,知省事。大歷六年七月,為江南西道都團練觀察使,在官恭恪,善理財賦。賈明觀者,事北軍都虞候劉希暹,魚朝恩誅,希暹從坐,明觀積惡犯眾怒。時宰相元載受賂,遣江南效力,魏少游承載意茍容之。及嗣恭代少游,即日杖殺,識者稱之。大歷八年,嶺南將哥舒晃殺節度使呂崇賁反,五嶺騷擾,詔加嗣恭兼嶺南節
 度觀察使。嗣恭擢流人孟瑤、敬冕,使分其務:瑤主大軍,當其沖;冕自間道輕入,招集義勇,得八千人,以撓其心腹。二人皆有全策詭計,出其不意,遂斬晃及誅其同惡萬餘人,築為京觀。俚洞之宿惡者皆族誅之,五嶺削平。拜檢校兵部尚書,知省事。



 嗣恭起於郡縣吏,以至大官,皆以恭恪為理著稱。及平廣州,商舶之徒,多因晃事誅之,嗣恭前後沒其家財寶數百萬貫,盡入私室,不以貢獻。代宗心甚銜之,故嗣恭雖有平方面功,止轉檢校兵部
 尚書,無所酬勞。及德宗即位,楊炎受其貨,始敘前功,除兵部尚書、東都留守。尋加懷鄭汝陜四州、河陽三城節度及東都畿觀察使。徵至京師卒,時年七十一,廢朝一日,贈左僕射。



 子恕,字體仁。初,嶺南衙將哥舒晃反,詔嗣恭自江西致討,授檢校工部員外郎,得以軍前便宜從事。俄而降者繼路,於是擢降將伊慎,推心用之。賊平,恕功居多,年才三十,為懷州刺史。久之,轉京兆少尹、監門衛大將軍、兼御史中丞、教練招討等使。其後為鄜
 坊觀察使、太子詹事。坐事貶吉州刺史,遷太子賓客。以右散騎常侍致仕卒,年七十三,贈洪州都督。恕私第有佳林園,自貞元初李紓、包佶輩迄於元和末,僅四十年,朝之名卿,咸從之游,高歌縱酒,不屑外慮,未嘗問家事,人亦以和易稱之。



 曲環,陜州安邑人也。父彬,為南使正監,因家於隴右,以環故累贈兵部尚書。環少讀兵書,尤以勇敢騎射聞。天寶中,從哥舒翰攻拔石堡城,收黃河九曲、洪濟等城,累
 授果毅別將。安祿山反,從襄陽節度魯炅守鄧州,拒賊將武令珣,戰數十合,環功居多,超授左清道率。又從李抱玉守河陽南城,尋將兵守澤州,破賊驍將安曉,敕特拜羽林將軍。又將別部兵合諸軍同討史朝義,平河北,累轉金吾大將軍,並同正員,隨李抱玉移軍京西。大歷中領兵隴州,頻破吐蕃,加特進、太常卿。上初嗣位,吐蕃大寇劍南,詔環以邠、隴兵五千馳往,大破戎虜,收七盤城、威武軍及維、茂二州,西戎奔遁。環大振功名而還,加
 太子賓客,賜以名馬。與諸將討涇州叛將劉文喜,平之,加開府儀同三司、兼御史中丞,充邠、隴兩軍都知兵馬使。時李納擁兵侵逼徐州,令環與劉玄佐同救援,累破李納叛黨,環以功最,加御史大夫。建中三年十月,加檢校左常侍,充邠、隴行營節度使。



 李希烈侵陷汴州,環與諸軍守固寧陵、陳州,大破希烈軍於陳州城下,殺逆黨三萬五千人,擒其驍將翟暉以獻,希烈因遁歸蔡州。環以功加檢校工部尚書,兼陳州刺史。希烈平,加環兼許
 州刺史、陳許等州節度觀察,加實封三百戶。陳、蔡二州以希烈擾亂,遭剽劫頗甚,人多逃竄他邑以避禍。環勤身恭儉,賦稅均平,政令寬簡,不三二歲,襁負而歸者相屬,訓農理戎,兵食皆豐羨。十二年,加檢校左僕射。卒時年七十四,廢朝一日,贈司空,賻布帛米粟有差。



 崔漢衡,博陵人也。性沉厚寬博,善與人交。釋褐,授沂州費令。滑州節度使令狐彰奏署掌記,累遷殿中侍御史。大歷六年,拜檢校禮部員外郎,為和吐蕃副使;還,遷右
 司郎中,改萬年令。建中三年,為殿中少監、兼御史大夫,充和蕃使,與吐蕃使區頰贊至自蕃中。時吐蕃大相尚結息忍而好殺,以常覆敗於劍南,思刷其恥,不肯約和。其次相尚結贊有材略,因言於贊普,請定界明約,以息邊人,贊普然之,竟以結贊代結息為大相,約和好,期以十月十五日會盟於境上。戊申,以漢衡為鴻臚卿。四年,吐蕃朝貢,加檢校工部尚書,復使吐蕃。興元初,上居奉天,吐蕃遣帥佐渾瑊敗硃泚兵於武功,以功轉檢校兵
 部尚書、兼秘書監、西京留守。無幾,真拜兵部尚書,為東都、淄青、魏博賑給宣慰使。明年,為幽州宣慰使,所至皆稱職。貞元三年,副侍中渾瑊與吐蕃會盟於平涼,吐蕃背約,瑊僅免,時無備預,在會免者什無一二,士卒,死者以千數。漢衡與同陷者並至河州,結贊令召之,以頻使於蕃,結贊素信重,與孟日華、中官劉延邕俱至石門,而遣五騎送至境上。四年七月,加檢校吏部尚書、晉慈隰觀察使,尋加都防禦使。十一年四月卒。



 楊朝晟,字叔明,夏州朔方人也。初在朔方為步軍先鋒,嘗有功,授甘泉府果毅。建中初,從李懷光討劉文喜於涇州,斬獲生擒居多,授驃騎大將軍,稍為右先鋒兵馬使。後李納寇徐州,從唐朝臣征討,嘗冠軍鋒,以功授開府儀同三司、檢校太子賓客。



 上在奉天,李懷光自山東赴難,以朝晟為左廂兵馬使,將千餘人下咸陽以挫硃泚,加御史中丞,實封一百五十戶。及懷光反於河中,朝晟被脅在軍。上幸梁、洋,韓游瑰退於邠、寧。懷光以嘗
 在邠、寧,迫制如屬城,以賊黨張昕在邠州總後務。昕懼難作,乃大索軍資,征卒乘,約明潛發,歸於懷光。朝晟父懷賓為游瑰將,因夜以數十騎斬昕及同謀,游瑰即日使懷賓奉表聞奏,上召勞問,授兼御史中丞,正除游瑰邠寧節度使。間諜至河中,朝晟聞其事。泣告懷光曰:父立功於國,子合誅戮,不可主兵矣。」懷光遂縶之。及諸軍進圍河中,韓游瑰營於長春宮,懷賓身當戰伐。及懷光平,上念其忠,俾副元帥渾瑊特原朝晟,遂為游瑰都虞侯。
 時父子同軍,皆為開府賓客、御史中丞,榮於軍中。



 後詔徵游瑰宿衛,以左金吾將軍張獻甫為檢校刑部尚書、兼御史大夫、邠寧慶節度觀察使,代韓游瑰。初,游瑰以吐蕃犯塞,自將兵戍寧州,及受代,以是月壬子夜輕騎潛遁歸闕。其將卒素驕怠,畏張獻甫之嚴,因游瑰夜出,衙內千餘人遂叛掠,且因監軍楊明義邀奏出奔將範希朝為節度。朝晟時為都虞候,初逃於郊,翌日乃來,紿其眾曰:「所請甚愜,我來賀也。」由是稍安。朝晟及諸將謀
 誅首惡者。乙卯,朝晟率諸將經數日以告曰:「前請者不獲,張尚書昨日已入邠州,汝等皆當死,吾不能盡殺,各言戎首以歸罪焉,餘無所問。」於是眾中唱二百餘人,斬之乃定。上擢希朝為寧州刺史,以副獻甫。獻甫入奏朝晟功,加御史大夫。



 九年,城鹽州,徵兵以護外境,朝晟分統士馬鎮木波。獻甫卒,詔以朝晟代之。其年,丁母憂,起復左金吾大將軍同正、邠州刺史,大夫如故。十年春,朝晟奏:「方渠、合道、木波,皆賊路也,請城其地以備之。」詔問:「
 所須幾何?」朝晟奏曰:「臣部下兵自可集事,不煩外助。」復問:「前築鹽州,凡興師七萬,今何其易也?」朝晟曰:「鹽州之役,諸軍蕃戎盡知之。今臣境迫虜,若大興兵,即蕃戎來寇,寇則戰,戰則無暇城矣。今請密發軍士,不十日至塞下,未三旬而功畢。」蕃人始乘障,數日而退。初,軍次方渠,無水,師徒囂然,遽有青蛇乘高而下,視其跡,水隨而流。朝晟令築防環之,遂為停泉,軍人仰飲以足,圖其事上聞,詔置祠焉。十五年二月,免喪,加檢校工部尚書。是
 夏,以防秋移軍寧州,遘疾,來年正月卒。



 樊澤,字安時,河中人也。父詠,開元中舉草澤,授試大理評事,累贈兵部尚書。澤長於河朔,相衛節度薛嵩奏為磁州司倉、堯山縣令。建中元年,舉賢良對策,禮部侍郎於邵厚遇之。與楊炎善,薦為補闕,歷都官員外郎。澤好讀兵書,朝廷以其有將帥材,尋兼御史中丞,充通和蕃使,蕃中用事宰相尚結贊深禮之。尋從鳳翔節度張鎰與吐蕃會盟於清水,遷金部郎中、御史中丞、山南節度
 行軍司馬。時李希烈背叛,詔以普王為行軍元帥,徵澤為諫議大夫、元帥行軍右司馬。屬駕幸奉天,普王不行,澤改右庶子、兼中丞,復為山南東道行軍司馬。尋代賈耽為襄州刺史、兼御史大夫、山南東道節度觀察等使。



 澤有武藝,每與諸將射獵,常出其右,人心服之,賊眾畏焉。頻與李希烈兇黨接戰,前後擒降其驍將張嘉瑜、杜文朝、梁俊之、李克誠、薛翼等,收唐、隨二州。希烈既平,澤丁母憂,起復右衛大將軍同正,餘如故。三年,代張伯儀為
 荊南節度觀察等使、江陵尹、兼御史大夫。三歲,加檢校禮部尚書,會襄州節度曹王皋卒於鎮,軍中剽劫擾亂,以澤威惠素著於襄、漢,復代曹王皋為襄州刺史、山南東道節度使。十二年,加檢校右僕射。卒年五十,贈司空,賻布帛米粟有差。其日將宴百官,廢朝,改取他日。



 李叔明,字晉卿,閬州新政人。本姓鮮于氏,代為豪族。兄仲通,天寶末為京兆尹、劍南節度使。兄弟並涉學,輕財好施。叔明初為劍南節度使楊國忠判官。乾元後為司
 勛員外郎,副漢中王瑀使回紇,回紇接禮稍倨,叔明離位責之曰:「大國通好,賢王奉使,可汗於大唐子婿,豈可恃微功而傲乎!唐法不然。」可汗改容加敬。復命,遷司門郎中。後為京兆少尹,無幾,以疾辭,除右庶子,出為邛州刺史。尋拜東川節度遂州刺史,後移鎮梓州,檢校戶部尚書。時東川兵荒之後,凋殘頗甚,叔明理之近二十年,招撫甿庶,夷落獲安。大歷末,有閬州嚴氏子上疏稱:「叔明少孤,養子於外族,遂冒姓焉,請復之」。詔從焉。叔明初
 不知其從外氏姓,意醜其事,遂抗表乞賜宗姓。代宗以戎鎮寄重,許之,仍置嚴氏子於法。及駕幸奉天,其子升翊從。叔明每私疏誡勵,見危臨難,當誓以死。升奉父嚴訓,果著勛效,識者嘉之。叔明既朝京師,以本官兼右僕射,乞骸骨,改太子太傅致仕,卒,謚曰襄。叔明總戎年深,積聚財貨,子孫驕淫,歿才數年,遺業蕩盡。



 裴胄,字胤叔,其先河東聞喜人,今代葬河南。伯父寬,戶部尚書,有名於開元、天寶間。胄明經及第,解褐補太僕
 寺主簿。屬二京陷覆,淪避他州。賊平,授秘書省正字,累轉秘書郎。陳少游陳鄭節度留後,奏胄試大理司直。少游罷,隴右節度李抱玉奏授監察御史,不得意,歸免。陳少游為宣歙觀察,復闢在幕府,抱玉怒,奏貶桐廬尉。浙西觀察使李棲筠有重望,虛心下士,幕府盛選才彥。觀察判官許鴻謙有學識,棲筠常異席,事多咨之;崔造輩皆所薦引,一見胄,深重之,薦於棲筠,奏授大理評事、觀察支度使。代宗以元載隳紊朝綱,徵筠入朝,內制授御
 史大夫,方將大用,載怙權,棲筠居顧問刺舉之職,與不平。及棲筠卒,胄護棲筠喪歸洛陽,眾論危之,胄坦然行心,無所顧望。淮南節度陳少游奏檢校主客員外、兼侍御史、觀察判官。尋為行軍司馬,遷宣州刺史。



 楊炎初作相,銳意為元載報仇,凡其枝黨無漏。適會胄部人積胄官時服雜俸錢為贓者,炎命酷吏員深按其事,貶汀州司馬。尋徵為少府少監,除京兆少尹,以父名不拜,換國子司業。遷湖南觀察都團練使,移江南西道。前江西
 觀察使李兼罷省南昌軍千餘人,收其資糧,分為月進,胄至,奏其本末,罷之。會荊南節度樊澤移鎮襄陽,宰相方議其人,上首命胄代澤,仍兼御史大夫。



 胄簡儉恆一,時諸道節度觀察使競剝下厚斂,制奇錦異綾,以進奉為名。又貴人宣命,必竭公藏以買其歡。胄待之有節,皆不盈數金,常賦之外無橫斂,宴勞禮止三爵,未嘗酣樂。時武臣多廝養畜賓介,微失則奏流死,胄以書生始,奏貶書記梁易從,君子薄其進退賓客不以禮,物議薄之。
 貞元十九年十月卒,時年七十五,贈右僕射,謚曰成。



 史臣曰:三獻軍謀臣節,克紹家風。路嗣恭從微至著,執法簡廉。環理兵勸農,獨彰善政。漢衡誠愨奉職。朝晟忠孝權謀。澤威惠荊、襄。叔明見危誓死,立政惠民。胄抱義危行,守政奉公。皆賢帥矣。然嗣恭聚財,為功名之瑕玷;叔明聚財,致子孫之驕淫。財之污人,誠可誡也。



 贊曰:張、路、曲、崔、樊、楊、李、裴,守忠臣之道,皆賢帥
 之才。



\end{pinyinscope}