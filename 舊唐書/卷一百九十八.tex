\article{卷一百九十八}

\begin{pinyinscope}

 ○孔紹
 安子禎孫若思袁朗弟承序利貞孫誼賀德仁庾抱蔡允恭鄭世翼謝偃崔信明張蘊古
 劉胤之弟子延祐兄子藏器張昌齡崔行功孟利貞董思恭元思敬徐齊聃杜易簡從祖弟審言盧照鄰楊炯王勃兄勮勔駱賓王鄧玄挺



 臣觀前代秉筆論文者多矣。莫不憲章《謨》、《誥》,祖述《詩》、《騷》;遠宗毛、鄭之訓論,近鄙班、揚之述作。謂「採採苤諲」,獨高比興之源;「湛湛江楓」,長擅詠歌之體。殊不知世代有文
 質,風俗有淳醨;學識有淺深,才性有工拙。昔仲尼演三代之《易》,刪諸國之《詩》,非求勝於昔賢,要取名於今代。實以淳樸之時傷質,民俗之語不經,故飾以文言,考之弦誦。然後致遠不泥,永代作程,即知是古非今,未為通論。夫執鑒寫形,持衡品物,非伯樂不能分駑驥之狀,非延陵不能別《雅》、《鄭》之音。若空混吹竽之人,即異聞《韶》之嘆。近代唯沈隱俟斟酌《二南》,剖陳三變;攄雲、淵之抑鬱,振潘、陸之風徽。俾律呂和諧,宮商輯洽,不獨子建總建安
 之霸,客兒擅江左之雄。爰及我朝,挺生賢俊,文皇帝解戎衣而開學校,飾賁帛而禮儒生;門羅吐鳳之才,人擅握蛇之價。靡不發言為論,下筆成文,足以緯俗經邦,豈止雕章縟句。韻諧金奏,詞炳丹青,故貞觀之風,同乎三代。高宗、天後,尤重詳延;天子賦橫汾之詩,臣下繼柏梁之奏;巍巍濟濟,輝爍古今。如燕、許之潤色王言,吳、陸之鋪揚鴻業,元稹、劉賁之對策,王維、杜甫之雕蟲,並非肄業使然,自是天機秀絕。若隋珠色澤,無假淬磨,孔璣翠
 羽,自成華彩,置之文苑,實煥緗圖。其間爵位崇高,別為之傳。今採孔紹安已下,為《文苑》三篇,凱懷才憔悴之徒,千古見知於作者。



 孔紹安,越州山陰人,陳吏部尚書奐之子。少與兄紹新,俱以文詞知名。十三,陳亡入隋,徙居京兆鄠縣。閉門讀書,誦古文集數十萬言,外兄虞世南嘆異之。紹新嘗謂世南曰:「本朝淪陷,分從湮滅,但見此弟,竊謂家族不亡矣!」時有詞人孫萬壽,與紹安篤忘年之好,時人稱為孫、
 孔。紹安大業末為監察御史。時高祖為隋討賊於河東,詔紹安監高祖之軍,深見接遇。及高祖受禪,紹安自洛陽間行來奔。高祖見之甚悅,拜內史舍人,賜宅一區、良馬兩匹、錢米絹布等。時夏侯端亦嘗為御史,監高祖軍,先紹安歸朝,授秘書監。紹安因侍宴,應詔詠《石榴詩》曰:「只為時來晚,開花不及春。」時人稱之。尋詔撰《梁史》,未成而卒。有文集五卷。



 子禎,高宗時為蘇州長史。曹王明為刺史,不循法度,禎每進諫。明曰:「寡人天子之弟,豈失於
 為王哉!」禎曰:「恩寵不可恃,大王不奉行國命,恐今之榮位,非大王所保,獨不見淮南之事乎?」明不悅。明左右有侵暴下人者,禎捕而杖殺之。明後果坐法,遷於黔中,謂人曰:「吾愧不用孔長史言,以及於此!」禎累遷絳州刺史,封武昌縣子。卒,謚曰溫。



 子季詡,早知名,官至左補闕。



 紹安孫若思。



 若思孤,母褚氏親自教訓,遂以學行知名。年少時,有人齎褚遂良書跡數卷以遺若思,唯受其一卷。其人曰:「此書當今所重,價比黃金,何不總取?」若思曰:「若價
 比金寶,此為多矣!」更截去半以還之。明經舉,累遷庫部郎中。若思常謂人曰:「仕至郎中足矣!」至是持一石止水,置於座右,以示有止足之意。尋遷給事中。



 中宗即位,敬暉、桓彥範等知國政,以若思多識故事,所有改革大事及疑議,多訪於若思。再轉禮部侍郎,出衛州刺史。先是,諸州別駕,皆以宗室為之,不為刺史致敬,由是多行不法。若思至州,舉奏別駕李道欽犯狀,請加鞫訊。乃詔別駕於刺史致禮,自若思始也。俄以清白稱,加銀青光祿
 大夫,賜絹百匹。歷汝州刺史、太子右諭德,封梁郡公。開元十七年卒,謚曰惠。



 袁朗,雍州長安人,陳尚書左僕射樞之子。其先自陳郡仕江左,世為冠族,陳亡徙關中。



 朗勤學,好屬文。在陳,釋褐秘書郎,甚為尚書令江總所重。嘗制千字詩,當時以為盛作。陳後主聞而召入禁中,使為《月賦》,朗染翰立成。後主曰:「觀此賦,謝希逸不能獨美於前矣!」又使為《芝草》、《嘉蓮》二頌,深見優賞。歷太子洗馬、德教殿學士,遷秘書
 丞。陳亡,仕隋為尚書儀曹郎。武德初,授齊王文學、祠部郎中,封汝南縣男,再轉給事中。貞觀初卒官。太宗為之廢朝一日,謂高士廉曰:「袁朗在任雖近,然其性謹厚,特使人傷惜。」因敕給其喪事,並存問妻子。有文集十四卷。



 從父弟承序,陳尚書僕射憲之子。武德中,齊王元吉聞其名,召為學士。府廢,累轉建昌令。在任清靜,士吏懷之。高宗在籓,太宗選學行之士為其僚屬,謂中書侍郎岑文本曰:「梁、陳名臣,有誰可稱?復有子弟堪招引否?」文本
 因言:「隋師入陳,百司奔散,莫有留者,唯袁憲獨在其主之傍。王世充將受隋禪,群僚表請勸進,憲子給事中承家托疾,獨不署名。此父子足稱忠烈。承家弟承序,清貞雅操,實繼先風。」由是召守晉王友,仍令侍讀,加授弘文館學士。未幾,卒。



 朗從祖弟利貞,陳中書令敬之孫也。高宗時為太常博士、周王侍讀。永隆二年,王立為皇太子,百官上禮。高宗將會百官及命婦於宣政殿,並設九部伎及散樂。利貞上疏諫曰:「臣以前殿正寢,非命婦宴會
 之地;象闕路門,非倡優進御之所。望詔命婦會於別殿,九部伎從東西門入,散樂一色,伏望停省。若於三殿別所,自可備極恩私。微臣庸蔽,不閑典則,忝預禮司,輕陳狂瞽。」帝納其言,即令移於麟德殿。至會日,酒酣,帝使中書侍郎薛元超謂利貞曰:「卿門承忠鯁,能抗疏直言,不加厚賜,何以獎勸!」賜物百段。俄遷祠部員外郎,卒。中宗即位,以侍讀恩,追贈秘書少監。



 朗十三代祖漢司徒滂,滂生魏國郎中、御史大夫渙,渙生晉尚書準,準生東晉
 右將軍、豫章太守沖,沖生司徒從事中郎耽,耽生瑯邪內史質,質生丹陽尹、宋公長史豹,豹生宋吳郡太守洵,累代有高名重位,前史有傳。五代叔祖宋太尉淑,高祖父左僕射、雍州刺史顗,高祖司空察,皆死國難。曾祖梁中書監、司空、穆公昂,仕齊為吳興太守,及梁高祖禪齊,久辭朝命。父樞,叔父憲,仕陳,皆為陳僕射。叔祖敬,中書令。及陳亡,憲冒難扶護後主。朗自以中外人物,為海內冠族,雖瑯邪王氏繼有臺鼎,而歷朝自為佐命,鄙之不
 以為伍。



 朗孫誼,又虞世南外孫。神功中,為蘇州刺史。嘗因視事,司馬、清河張沛通謁,沛即侍中文瓘之子。誼揖之曰:「司馬何事?」沛曰:「此州得一長史,是隴西李亶,天下甲門。」誼曰:「司馬何言之失!門戶須歷代人賢,名節風教,為衣冠顧矚,始可稱舉,老夫是也!夫山東人尚於婚媾,求於祿利;作時柱石,見危授命,則曠代無人。何可說之,以為門戶!」沛懷慚而退。時人以為口實。



 賀德仁,越州山陰人也。父朗,陳散騎常侍。德仁少與從
 兄基俱事國子祭酒周弘正,咸以詞學見稱。時人語曰:「學行可師賀德基,文質彬彬賀德仁。」德仁兄弟八人,時人方之荀氏。陳鄱陽王伯山為會稽太守,改其所居甘滂里為高陽里。德仁事陳,至吳興王友。



 入隋,僕射楊素薦之,授豫章王府記室參軍。王以師資禮之,恩遇甚厚。及煬帝即位,豫章王改封齊王,又授齊王府屬。及齊王獲譴,府僚皆被誅責,唯德仁以忠謹免罪,出補河東郡司法。素與隱太子善,及高祖平京師,隱太子封隴西
 公,用德仁為隴西公友。尋遷太子中舍人,以衰老不習吏事,轉太子洗馬。時蕭德言亦為洗馬,陳子良為右衛率府長史,皆為東宮學士。貞觀初,德仁轉趙王友。無幾,卒,年七十餘。有文集二十卷。



 德仁弟子紀、敳,亦以博學知名。高宗時,紀官至太子洗馬,修《五禮》。敳至率更令,兼太子侍讀。兄弟並為崇賢館學士,學者榮之。



 庾抱,潤州江寧人也,其先自潁川徙家焉。祖眾,陳御史中丞。父超,南平王記室。抱開皇中為延州參軍事。後累
 歲,調吏部。尚書牛弘知其有學術,給筆札令自序。援翰便就,弘甚奇之。後補元德太子學士,禮賜甚優。會皇孫載誕,太子宴賓客,抱於坐中獻《嫡皇孫頌》,深被嗟賞。後為越巂主簿,稱病不行。義寧中,隱太子弘引為隴西公府記室。時軍國多務,公府文檄皆出於抱。尋轉太子舍人,未幾,卒。有集十卷。



 蔡允恭,荊州江陵人也。祖點,梁尚書儀曹郎。父大業,後梁左民尚書。允恭有風彩,善綴文。仕隋歷著作佐郎、起
 居舍人。雅善吟詠。煬帝屬詞賦,多令諷誦之。嘗遣教宮女,允恭深以為恥,因稱氣疾,不時應召。煬帝又許授以內史舍人,更令入內教宮人,允恭固辭不就,以是稍被疏絕。江都之難,允恭從宇文化及西上,沒於竇建德。及平東夏,太宗引為秦府參軍,兼文學館學士。貞觀初,除太子洗馬。尋致仕,卒於家。有集十卷,又撰《後梁春秋》十卷。



 鄭世翼,鄭州滎陽人也,世為著姓。祖敬德,周儀同大將
 軍。父機,司武中士。世翼弱冠有盛名。武德中,歷萬年丞、揚州錄事參軍。數以言辭忤物,稱為輕薄。時崔信明自謂文章獨步,多所凌轢;世翼遇諸江中,謂之曰:「嘗聞『楓落吳江冷。』」信明欣然示百餘篇。世翼覽之未終,曰:「所見不如所聞。」投之於江,信明不能對,擁楫而去。世翼貞觀中坐怨謗,配流巂州,卒。文集多遺失,撰《交游傳》,頗行於時。



 謝偃,衛縣人也,本姓直勒氏。祖孝政,北齊散騎常侍,改
 姓謝氏。偃仕隋為散從正員郎。貞觀初,應詔對策及第,歷高陵主簿。十一年,駕幸東都,穀、洛泛溢洛陽宮,詔求直諫之士。偃上封事,極言得失。太宗稱善,引為弘文館直學士,拜魏王府功曹。偃嘗為《塵》、《影》二賦,甚工。太宗聞而詔見,自制賦序,言「區宇乂安,功德茂盛」。令其為賦,偃奉詔撰成,名曰《述聖賦》,賜採數十匹。偃又獻《惟皇誡德賦》以申諷,曰:



 臣聞理忘亂,安忘危,逸忘勞,得忘失。此四者,人君莫不皆然。是以夏桀以瑤臺璇室為麗,而不悟
 鳴條南巢之禍;殷辛以象箸玉杯為華,而不知牧野白旗之敗。故當其盛也,謂四海為己力;及其衰焉,乃匹夫之不制。當其信也,謂天下為無危;及其疑也,則顧盼皆仇敵。是知必有其德,則誠結戎夷,化行荒裔。茍失其度,則變生骨肉,釁起腹心矣!是以為人主者,不可忘初。處殿堂,則思前主之所以亡;朝萬國,則思今己之所以貴;巡府庫,則思今己之所以得;視功臣,則思其為己之始;見名將,則思其用力之初。茍弗忘舊,則人無易心,何患
 乎天下之不化!故旦行之則為堯、舜,暮失之則為桀、紂,豈異人哉!其詞曰:



 周墳籍以遷觀,總宇宙而一窺;結繩往而莫紀,書契崇而可知。惟皇王之迭代,信步驟之恆規,莫不慮失者常得,懷安者必危。是以戰戰怵怵,日慎一日,守約守儉,去奢去逸。外無荒禽,內無荒色,唯賢是授,唯人斯恤。則三皇不足六,五帝不足十。若夫恃聖驕力,狠戾倔強,忠良是棄,諂佞斯獎。構崇臺以造天,穿深池以絕壤。厚賦重斂,積寶藏鏹;無罪加刑,有功不賞。則
 夏桀可二,殷辛易兩。在危所恃,居安勿忘。功臣無逐,故人無放。放故者亡,逐功者喪。四海岌岌,九土漫漫,覆之甚易,存之實難。是以一人有悅,萬國同歡;一人失所,兆庶俱殘。喜則隆冬可熱,怒則盛夏成寒;一動而八表亂,一言而天下安。舉君過者曰忠,述主美者為佞,茍承顏以順旨,必蔽視而稱聖。故使曲者亂直,邪者疑正;改華服以就胡,變雅音而入鄭;雖往古之軌躅,亦當今之龜鏡。崔嵬龍殿,赫奕鳳門,苞四海以稱主,冠天下而獨尊。
 既兄日而姊月,亦父乾而母坤。視則金翠溢目,聽則絲竹盈耳。信賞罰之在躬,實榮辱之由己;語義皇而易匹,言堯、舜之可擬。驕志自此而生,侈心因茲而起。常懼覆而懼亡,必思足而思止;勿忘潛龍之初,當懷布衣之始。在位稱寶,居器曰神,鐘鼓庭設,玉帛階陳。得必有兆,失必有因;一替一立,或周或秦。既承前代,當思後人。唯德可以久,天道無常親。



 時李百藥工為五言詩,而偃善作賦,時人稱為李詩謝賦焉。十七年,府廢,出為湘潭令,卒。
 文集十卷。



 崔信明,青州益都人也,後魏七兵尚書光伯曾孫也。祖縚,北海郡守。信明以五月五日日正中時生,有異雀數頭,身形甚小,五色畢備,集於庭樹;鼓翼齊鳴,聲清宛亮。隋太史令史良使至青州,遇而占之曰:「五月為火,火為《離》,《離》為文彩。日正中,文之盛也。又有雀五色,奮翼而鳴。此兒必文藻煥爛,聲名播於天下。雀形既小,祿位殆不高。」及長,博聞強記,下筆成章。鄉人高孝基有知人之鑒,
 每謂人曰:「崔信明才學富贍,雖名冠一時,但恨其位不達耳!」



 大業中,為堯城令。竇建德僭號,欲引用之。信明族弟敬素為建德鴻臚卿,說信明曰:「隋主無道,天下鼎沸,衣冠禮樂,掃地無餘。兄遁跡下僚,不被收用,豫讓所以不報範中行,只以眾人遇我者也。夏王英武,有並吞天下之心,士女襁負而至者,不可稱數。此時不立功立事,豈是見幾而作者乎?」信明曰:「昔申胥海畔漁者,尚能固其節;吾終不能屈身偽主,求斗筲之職。」遂逾城而遁,隱
 於太行山。貞觀六年,應詔舉,授興世丞。遷秦川令,卒。



 信明頗蹇傲自伐,常賦詩吟嘯,自謂過於李百藥,時人多不許之。又矜其門族,輕侮四海士望,由是為世所譏。



 子冬日,則天時為黃門侍郎,被酷吏所殺。



 張蘊古,相州洹水人也。性聰敏,博涉書傳,善綴文,能背碑覆局。尤曉時務,為州閭所稱。自幽州總管府記室直中書省。太宗初即位,上《大寶箴》以諷,其詞曰:



 今來古往,俯察仰觀,惟闢作福,為君實難。主普天之下,處王公之
 上;任土貢其所求,具僚和其所唱。是故競懼之心日弛,邪僻之情轉放,豈知事起乎所忽,禍生乎無妄。固以聖人受命,拯溺亨屯,歸過於己,推恩於民。大明無偏照,至公無私親。故以一人治天下,不以天下奉一人。禮以禁其奢,樂以防其佚。左言而右事,出警而入蹕。四時同其慘舒,三光同其得失。故身為之度,而聲為之律。勿謂無知,居高聽卑;勿謂何害,積小成大。樂不可極,極樂生哀;欲不可縱,縱欲成災。壯九重於內,所居不過容膝;彼昏
 不知,瑤其臺而瓊其室。羅八品於前,所食不過適口;唯狂罔念,丘其糟而池其酒。勿內荒於色,勿外荒於禽,勿貴難得之貨,勿聽亡國之音。內荒伐人性,外荒蕩人心,難得之貨侈,亡國之聲淫。勿謂我尊而傲賢侮士,勿謂我智而拒諫矜己。聞之夏王,據饋頻起;亦有魏帝,牽裾不止。安彼反側,如春陽秋露,巍巍蕩蕩,恢漢高大度;撫茲庶事,如履薄臨深,戰戰慄慄,用周文小心。



 《詩》云:「不識不知」,《書》曰:「無偏無黨」。一彼此於胸臆,捐好惡於心想。眾
 棄而後加刑,眾悅而後命賞。弱其強而治其亂,申其屈而直其枉。故曰:「如衡如石,不定物以數,物之懸者,輕重自具;如水如鏡,不示物以情,物之鑒者,妍媸自生。」勿渾渾而濁,勿皎皎而清,勿沒沒而暗,勿察察而明。雖冕旒蔽目而視於未形,雖黈纊塞耳而聽於無聲。縱心乎湛然之域,游神於至道之精。扣之者應洪纖而效響,酌之者隨深淺而皆盈。故曰:天之清,地之寧,王之貞。四時不言而代序,萬物無為而受成。豈知帝有其力,而天下和
 平。



 吾王撥亂,戡以智力,民懼其威,未懷其德。我皇撫運,扇以淳風,民懷其始,未保其終。爰述金鏡,窮神盡聖;使人以心,應言以行。包括治體,抑揚詞令,天下為公,一人有慶。開羅起祝,援琴命詩,一日二日,念茲在茲。唯人所召,自天祐之。爭臣司直,敢告前疑!



 太宗嘉之,賜以束帛,除大理丞。



 初,河內人李孝德,素有風疾,而語涉妄妖。蘊古究其獄,稱好德癲病有征,法不當坐。治書侍御史權萬紀劾蘊古家住相州,好德之兄厚德為其刺史,情在
 阿縱,奏事不實。太宗大怒,曰:「小子乃敢亂吾法耶?」令斬於東市。太宗尋悔,因發制,凡決死者,命所司五覆奏,自蘊古始也。



 劉胤之,徐州彭城人也。祖禕之,後魏臨淮鎮將。胤之少有學業,與隋信都丞孫萬壽、宗正卿李百藥為忘年之友。武德中,御史大夫杜淹表薦之,再遷信都令,甚存惠政。永徽初,累轉著作郎、弘文館學士,與國子祭酒令狐德棻、著作郎楊仁卿等,撰成國史及實錄,奏上之,封陽
 城縣男。尋以老,不堪著述,出為楚州刺史,卒。



 弟子延祐,弱冠本州舉進士,累補渭南尉。刀筆吏能,為畿邑當時之冠。司空李勣嘗謂曰:「足下春秋甫爾,便擅大名,宜稍自貶抑,無為獨出人右也。」後歷右司郎中,檢校司賓少卿,封薛縣男。



 徐敬業之亂,揚州初平,所有刑名,莫能決定,延祐奉使至軍所決之。時議者斷受賊五品官者斬,六品者流。延祐以為諸非元謀,迫脅從盜,則置極刑,事涉枉濫,乃斷受賊五品者流,六品已下俱除名而已。其
 得全濟者甚眾。



 出為箕州刺史,轉安南都護。嶺南俚戶,舊輸半課,及延祐到,遂勒全輸。由是其下皆怨,謀欲將叛,延祐乃誅其首惡李嗣仙。垂拱三年,嗣仙黨與丁建、李思慎等,遂率眾圍安南府。時城中勝兵不過數百,乃禁門堅守,以候鄰境之援。廣州大族馮子猷幸災樂禍,欲因危立功,遂按兵縱敵,使其為害滋甚。延祐遂為思慎所害。其後桂州司馬曹玄靜率兵討思慎等,擒之。盡斬於安南城下。



 胤之從父兄子藏器,亦有詞學,官至宋
 州司馬。藏器子知柔,開元初,為工部尚書。知柔弟知幾,避玄宗名改子玄。自有傳。



 張昌齡,冀州南宮人。弱冠以文詞知名。本州欲以秀才舉之,昌齡以時廢此科已久,固辭。乃充進士貢舉及第。貞觀二十一年,翠微宮成,詣闕獻頌。太宗召見,試作《息兵詔》草,俄頃而就。太宗甚悅,因謂之曰:「昔禰衡、潘岳,皆恃才傲物,以至非命。汝才不減二賢,宜追鑒前軌,以副吾所取也。」乃敕於通事舍人裡供奉。尋為昆山道行軍
 記室,破盧明月,平龜茲,軍書露布,皆昌齡之文也。再轉長安尉,出為襄州司戶,丁憂去官。後賀蘭敏之奏引於北門修撰,尋又罷去。乾封元年卒。文集二十卷。



 兄昌宗,亦有學業,官至太子舍人、修文館學士。撰《古文紀年新傳》三十卷。



 崔行功,恆州井陘人,北齊鉅鹿太守伯讓曾孫也,自博陵徙家焉。行功少好學,中書侍郎唐儉愛其才,以女妻之。儉前後征討,所有文表,皆行功之文。高宗時,累轉吏
 部郎中。以善敷奏,嘗兼通事舍人、內供奉。坐事貶為游安令,尋徵為司文郎中。當時朝廷大手筆,多是行功及蘭臺侍郎李懷儼之詞。



 先是,太宗命秘書監魏徵寫四部群書,將進內貯庫,別置讎校二十人、書手一百人。征改職之後,令虞世南、顏師古等續其事。至高宗初,其功未畢。顯慶中,罷讎校及御書手,令工書人繕寫,計直酬傭,擇散官隨番讎校。其後又詔東臺侍郎趙仁本、東臺舍人張文瓘及行功、懷儼等相次充使檢校。又置詳正
 學士以校理之,行功仍專知御集。遷蘭臺侍郎。



 咸亨中,官名復舊,改為秘書少監。上元元年,卒官。有集六十卷。兄子玄暐,別有傳。



 行功前後預撰《晉書》及《文思博要》等。同時又有孟利貞、董思恭、元思敬等,並以文藻知名。



 孟利貞者,華州華陰人也。父神慶,高宗初為沁州刺史,以清介著名。利貞初為太子司議郎,中宗在東宮,深懼之。受詔與少師許敬宗、崇賢館學士郭瑜、顧胤、董思恭等撰《瑤山玉彩》五百卷。龍朔二年奏上之,高宗稱善,加
 級賜物有差。利貞累轉著作郎,加弘文館學士。垂拱初卒。又撰《續文選》十三卷。



 兄允忠,垂拱中為天官侍郎。



 董思恭者,蘇州吳人。所著篇詠,甚為時人所重。初為右史,知考功舉事,坐預洩問目,配流嶺表而死。



 元思敬者,總章中為協律郎。預修《芳林要覽》,又撰《詩人秀句》兩卷,傳於世。



 徐齊聃,湖州長城人也。父孝德,以女為才人,官至果州刺史。齊聃少善屬文,高宗時累遷蘭臺舍人。時敕令有
 突厥酋長子弟事東宮,齊聃上疏曰:



 昔姬誦與伯禽同業,晉儲以師曠為友,匪唯專賴師資,固亦詳觀近習。皇太子自可招集園、綺,寤寐應、劉。階闥小臣,必採於端士;驅馳所任,並歸於正人。方流好善之風,永播崇賢之美。今乃使氈裘之子,解辮而侍春闈;冒頓之苗,削衣任而陪望苑。在於道義,臣竊有疑。詩云:「敬慎威儀,以近有德。」《書》曰:「任官惟賢才,左右惟其人。」蓋殷勤於此,防微之至也。



 齊聃又嘗上奏曰:「齊獻公即陛下外氏,雖子孫有犯,不
 合上延於祖。今周忠孝公廟甚修崇,而齊獻公廟遽毀壞,不審陛下將何以重示海內,以彰孝理之風?」帝皆納其言。



 齊聃善於文誥,甚為當時所稱。高宗愛其文,令侍周王等屬文,以職在樞劇,仍敕間日來往焉。以漏洩機密,左授蘄州司馬。俄又坐事配流欽州。咸亨中卒,年四十餘。睿宗即位,追錄舊恩,累贈禮部尚書。



 子堅,別有傳。



 杜易簡,襄州襄陽人,周硤州刺史叔毗曾孫也。九歲能屬文,及長,博學有高名。姨兄中書令岑文本甚推重之。
 登進士第,累轉殿中侍御史。咸亨中,為考功員外郎。時吏部侍郎裴行儉、李敬玄相與不葉,易簡與吏部員外郎賈言忠希行儉之旨,上封陳敬玄罪狀。高宗惡其朋黨,左轉易簡為開州司馬,尋卒。



 易簡頗善著述,撰《御史臺雜注》五卷,文集二十卷,行於代。



 易簡從祖弟審言。



 審言,進士舉,初為隰城尉。雅善五言詩,工書翰,有能名。然恃才謇傲,甚為時輩所嫉。乾封中,蘇味道為天官侍郎,審言預選。試判訖,謂人曰:「蘇味道必死。」人問其故,審言
 曰:「見吾判,即自當羞死矣!」又嘗謂人曰:「吾之文章,合得屈、宋作衙官;吾之書跡,合得王羲之北面。」其矜誕如此。



 累轉洛陽丞。坐事貶授吉州司戶參軍。又與州僚不葉,司馬周季重與員外司戶郭若訥共構審言罪狀,系獄,將因事殺之。既而季重等府中酣宴,審言子並年十三,懷刃以擊之。季重中傷死,而並亦為左右所殺。季重臨死曰:「吾不知審言有孝子,郭若訥誤我至此!」審言因此免官,還東都,自為文祭並。士友咸哀並孝烈,蘇頲為墓
 志,劉允濟為祭文。後則天召見審言,將加擢用。問曰:「卿歡喜否?」審言蹈舞謝恩。因令作《歡喜詩》,甚見嘉賞,拜著作佐郎。俄遷膳部員外郎。神龍初,坐與張易之兄弟交往,配流嶺外。尋召授國子監主簿,加修文館直學士。年六十餘卒。有文集十卷。



 次子閑。閑子甫,別有傳。



 盧照鄰,字升之,幽州範陽人也。年十餘歲,就曹憲、王義方授《蒼》、《雅》及經史,博學善屬文。初授鄧王府典簽,王甚愛重之,曾謂群官曰:「此即寡人相如也。」後拜新都尉。因
 染風疾去官,處太白山中,以服餌為事。後疾轉篤,徙居陽翟之具茨山,著《釋疾文》、《五悲》等誦。頗有騷人之風,甚為文士所重。



 照鄰既沉痼攣廢,不堪其苦,嘗與親屬執別,遂自投潁水而死,時年四十。文集二十卷。



 兄光乘,亦知名,長壽中為隴州刺史。



 楊炯,華陰人。伯祖虔威,武德中官至右衛將軍。炯幼聰敏博學,善屬文。神童舉,拜校書郎,為崇文館學士。儀鳳中,太常博士蘇知幾上表,以公卿已下冕服,請別立節
 文。敕下有司詳議,炯獻議曰:



 古者太昊庖羲氏,仰以觀象,俯以察法,造書契而文籍生。次有黃帝軒轅氏,長而敦敏,成而聰明,垂衣裳而天下理。其後數遷五德,君非一姓,體國經野,建邦設都,文質所以再而復,正朔所以三而改。夫改正朔者,謂夏后氏之建寅,殷人建丑,周人建子。至於以日系月,以月系時,以時系年,此三王相襲之道也!夫易服色者,謂夏后氏尚黑,殷人尚白,周人尚赤。至於山、龍、華蟲、宗彞、藻、火、粉米、黼、黻,此又百代可知
 之道。



 謹按《虞書》曰:「予欲觀古人之象,日、月、星辰、山、龍、華蟲作會,宗彞、藻、火、粉米、黼、黻、絺繡。」由此言之,則其所從來者尚矣。日月星辰者,明光照下土也。山者,布散雲雨,象聖王大澤沾下也。龍者,變化無方,象聖王應時布教也。華蟲者,雉也,身被五彩,象聖王體兼文明也。宗彞者,武蜼也,以剛猛制物,象聖王神武定亂也。藻者,逐水上下,象聖王隨代而應也。火者,陶冶烹飪,象聖王至德日新也。粉米者,人恃以生,象聖王為物之所賴也。黼能斷
 割,象聖王臨事能決也。黻者,兩己相背,象君臣可否相濟也。



 迨有周氏,乃以日月星辰為旌旗之節,又登龍於山,登火於宗彞,於是乎制袞冕以祀先王也。九章者,法陽數也,以龍為首章。袞者,卷也,龍德神異,應變潛見,表聖王深識遠智,卷舒神化也。又制柷冕以祭先公也。柷者,雉也,有耿介之志,表公有賢才,能守耿介之節也。又制毳冕以祭四望也。四望者,岳瀆之神也。武蜼者,山林所生,明其象也。制絺冕以祭社稷也。社稷者,土穀之神
 也。粉米由之而成,象其功也。又制玄冕以祭群小祀也。百神異形,難可遍擬,但取黻之相背,昭異名也。夫以周公之多才也,故治定制禮,功成作樂。夫以孔宣之將聖也,故行夏之時,服周之冕。先王之法服,乃此之自出矣;天下之能事,又於是乎畢矣。



 今知幾表狀請制大明冕十三章,乘輿服之者。謹按,日月星辰者,已施於旌旗矣。龍武山火者,又不逾於古矣。而云麟鳳有四靈之名,玄龜有負圖之應,云有紀官之號,水有盛德之祥,此蓋別
 表休徵,終是無逾比象。然則皇王受命,天地興符,仰觀則璧合珠連,俯察則銀黃玉紫。殫南宮之粉壁,不足寫其形狀;罄東觀之鉛黃,未可紀其名實。固不可畢陳於法服也。云者,龍之氣也;水者,藻之自生也。又不假別為章目,此蓋不經之甚也!



 又鸞冕八章,三公服之者。鸞者,太平之瑞也,非三公之德也。鷹鸇者,鷙鳥也,適可以辨祥刑之職也。熊羆者,猛獸也,適可以旌武臣之力也。又稱藻為水草,無所法象,引張衡賦「蒂倒茄於藻井,披紅
 葩之狎獵」,請為蓮華,取其文彩者。夫茄者,蓮也。若以蓮代藻,變古從今,既不知草木之名,亦未達文章之意,此又不經之甚也!



 又毳冕六章,三品服之者。按此王者祀四望服之名也。今三品乃得同王之毳冕,而三公不得同王之袞名,豈唯顛倒衣裳,抑亦自相矛盾,此又不經之甚也!



 又黻冕四章,五品服之者。考之於古,則無其名;驗之於今,則非章首,此又不經之甚也!



 若夫禮唯從俗,則命為制,令為詔,乃秦皇之故事,猶可以適於今矣!若
 夫義取隨時,則出稱警,入稱蹕,乃漢國之舊儀,猶可以行於代矣。亦何取變周公之軌物,改宣尼之法度者哉!



 由是竟寢知幾所請。



 俄遷詹事司直。則天初,坐從祖弟神讓犯逆,左轉梓州司法參軍。秩滿,選授盈川令。如意元年七月望日,宮中出盂蘭盆,分送佛寺,則天禦洛南門,與百僚觀之。炯獻《盂蘭盆賦》,詞甚雅麗。炯至官,為政殘酷,人吏動不如意,輒搒殺之。又所居府舍,多進士亭臺,皆書榜額,為之美名,大為遠近所笑。無何卒官。中
 宗即位,以舊僚追贈著作郎。文集三十卷。



 炯與王勃、盧照鄰、駱賓王以文詞齊名,海內稱為王楊盧駱,亦號為「四傑」。炯聞之,謂人曰:「吾愧在盧前,恥居王後。」當時議者,亦以為然。



 其後崔融、李嶠、張說俱重四傑之文。崔融曰:「王勃文章宏逸,有絕塵之跡,固非常流所及。炯與照鄰可以企之,盈川之言信矣!」說曰:「楊盈川文思如懸河注水,酌之不竭,既優於盧,亦不減王。『恥居王後』,信然;『愧在盧前』,謙也。」



 開元中,說為集賢大學士十餘年。常與學士
 徐堅論近代文士,悲其凋喪。堅曰:「李趙公、崔文公之筆術,擅價一時,其間孰優?」說曰:「李嶠、崔融、薛稷、宋之問之文,如良金美玉,無施不可。富嘉謨之文,如孤峰絕岸,壁立萬仞,濃雲鬱興,震雷俱發,誠可畏也,若施於廊廟,則駭矣!閻朝隱之文,如麗服靚莊,燕歌趙舞,觀者忘疲,若類之風、雅,則罪人矣!」問後進詞人之優劣,說曰:「韓休之文,如大羹旨酒,雅有典則,而薄於滋味。許景先之文,如豐肌膩理,雖穠華可愛,而微少風骨。張九齡之文,如輕
 縑素練,實濟時用,而微窘邊幅。王翰之文,如瓊懷玉斝,雖爛然可珍,而多有玷缺。」堅以為然。



 虔威子德乾,高宗末,歷澤、齊、汴、相四州刺史,治有威名,郡人為之語曰:「寧食三鬥蒜,不逢楊德乾。」



 子神讓,天授初與徐敬業於揚州謀叛,父子伏誅。



 王勃。字子安,絳州龍門人。祖通,隋蜀郡司戶書佐。大業末,棄官歸,以著書講學為業。依《春秋》體例,自獲麟後,歷秦、漢至於後魏,著紀年之書,謂之《元經》。又依《孔子家語》、
 揚雄《法言》例,為客主對答之說,號曰《中說》。皆為儒士所稱。義寧元年卒,門人薛收等相與議謚曰文中子。二子:福畤、福郊。



 勃六歲解屬文,構思無滯,詞情英邁,與兄勔、勮,才藻相類。父友杜易簡常稱之曰:「此王氏三珠樹也。」勃年未及冠,應幽素舉及第。乾封初,詣闕上《宸游東嶽頌》。時東都造乾元殿,又上《乾元殿頌》。沛王賢聞其名,召為沛府修撰,甚愛重之。諸王鬥雞,互有勝負,勃戲為《檄英王雞文》。高宗覽之,怒曰:「據此是交構之漸。」即日斥勃,
 不令入府。久之,補虢州參軍。



 勃恃才傲物,為同僚所嫉。有官奴曹達犯罪,勃匿之,又懼事洩,乃殺達以塞口。事發,當誅,會赦除名。時勃父福畤為雍州司戶參軍,坐勃左遷交趾令。上元二年,勃往交趾省父,道出江中,為《採蓮賦》以見意,其辭甚美。渡南海,墮水而卒,時年二十八。



 苾,弱冠進士登第,累除太子典膳丞。長壽中,擢為鳳閣舍人。時壽春王成器、衡陽王成義等五王初出閣,同日授冊。有司撰儀注,忘載冊文。及百僚在列,方知闕禮,宰
 相相顧失色。苾立召書吏五人,各令執筆,口占分寫,一時俱畢。詞理典贍,人皆嘆服。尋加弘文館學士,兼知天官侍郎。苾頗任權勢,交結非類。萬歲通天二年,綦連耀謀逆事洩,閟坐與耀善,並弟閟並伏誅。



 閟累官至涇州刺史。神龍初,有詔追復苾、閟官位。



 福畤,天后朝以子貴,累轉澤州長史,卒。



 初,吏部侍郎裴行儉典選,有知人之鑒,見苾與蘇味道,謂人曰:「二子亦當掌銓衡之任。」李敬玄尤重楊炯、盧照鄰、駱賓王與勃等四人,必當顯貴。行
 儉曰:「士之致遠,先器識而後文藝。勃等雖有文才,而浮躁淺露,豈享爵祿之器耶!楊子沉靜,應至令長,餘得令終為幸。」果如其言。



 勃文章邁捷,下筆則成,尤好著書。撰《周易發揮》五卷,及《次論》等書數部。勃亡後,並多遺失。有文集三十卷。勃聰警絕眾,於推步歷算尤精,嘗作《大唐千歲歷》,言唐德靈長千年,不合承周、隋短祚。其論大旨云:「以土王者,五十代而一千年;金王者,四十九代而九百年;水王者,二十代而六百年;木王者,三十代而八百
 年;火王者,二十代而七百年。此天地之常期,符歷之數也。自黃帝至漢,並是五運真主。五行已遍,土運復歸,唐德承之,宜矣!魏、晉至於周、隋,咸非正統,五行之沴氣也,故不可承之。」大率如此。



 駱賓王,婺州義烏人。少善屬文,尤妙於五言詩,嘗作《帝京篇》,當時以為絕唱。然落魄無行,好與博徒游。高宗末,為長安主簿。坐贓,左遷臨海丞,怏怏失志,棄官而去。文明中,與徐敬業於揚州作亂。敬業軍中書檄,皆賓王之
 詞也。敬業敗,伏誅,文多散失。則天素重其文,遣使求之。有兗州人卻雲卿集成十卷,盛傳於世。



 鄧玄挺,雍州藍田人。少善屬文,累遷左史。坐與上官儀善,出為頓丘令。有善政,璽書勞問。累授中書舍人。性俊辨,機捷過人,每有嘲謔,朝廷稱為口實。則天臨朝,遷吏部侍郎,既不稱職,甚為時談所鄙。又患消渴之疾,選人目為「鄧渴」,為榜於衢路。自有唐已來,掌選之失,未有如玄挺者。坐此左遷澧州刺史。在州復以善政聞,遷晉州
 刺史,召拜麟臺少監,重為天官侍郎,其失又甚於前。玄挺女為道王子諲妻,又與蔣王子煒相善。諲謀迎中宗於房陵,以問玄挺。煒又嘗謂玄挺曰:「欲作急計如何?」玄挺雖皆不答,而不以告。永昌元年得罪,下獄死。



\end{pinyinscope}