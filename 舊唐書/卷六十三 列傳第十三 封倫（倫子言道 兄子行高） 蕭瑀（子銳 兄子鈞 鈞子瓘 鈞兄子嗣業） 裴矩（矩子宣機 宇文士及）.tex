\article{卷六十三 列傳第十三 封倫(倫子言道 兄子行高) 蕭瑀(子銳 兄子鈞 鈞子瓘 鈞兄子嗣業) 裴矩(矩子宣機 宇文士及)}

\begin{pinyinscope}

 ○封倫倫
 子言道兄子行高蕭瑀子銳
 兄子鈞鈞子瓘鈞兄子嗣業裴矩矩子宣機宇文士及



 封倫,字德彞,觀州蓚人。北齊太子太保隆之孫。父子繡,隋通州刺史。倫少時,其舅盧思道每言曰:「此子智識過
 人,必能致位卿相。」開皇末,江南作亂,內史令楊素往征之,署為行軍記室。船至海曲,素召之,倫墜於水中,人救免溺,乃易衣以見,竟寢不言。素後知,問其故,曰:「私事也,所以不白。」素甚嗟異之。素將營仁壽宮,引為土木監。隋文帝至宮所,見制度奢侈,大怒曰:「楊素為不誠矣!殫百姓之力,雕飾離宮,為吾結怨於天下。」素惶恐,慮將獲譴。倫曰:「公當弗憂,待皇后至,必有恩詔。」明日,果召素入對,獨狐後勞之曰:「公知吾夫妻年老,無以娛心,盛飾此宮,
 豈非孝順。」素退問倫曰:「卿何以知之?」對曰:「至尊性儉,故初見而怒,然雅聽後言。後,婦人也,惟麗是好,後心既悅,帝慮必移,所以知耳。」素嘆伏曰:「揣摩之才,非吾所及。」素負貴恃才,多所凌侮,唯擊賞倫。每引與論宰相之務,終日忘倦,因撫其床曰:「封郎必當據吾此座。」驟稱薦於文帝,由是擢授內史舍人。大業中,倫見虞世基幸於煬帝而不閑吏務,每有承受,多失事機。倫又托附之,密為指畫,宣行詔命,諂順主心。外有表疏如忤意者,皆寢而不
 奏。決斷刑法,多峻文深誣;策勛行賞,必抑削之。故世基之寵日隆,而隋政日壞,皆倫所為也。宇文化及之亂,逼帝出宮,使倫數帝之罪。帝謂曰:「卿是士人,何至於此?」倫赧然而退。化及尋署內史令,從至聊城。倫見化及勢蹙,乃潛結化及弟士及,請於濟北運糧,以觀其變。遇化及敗,與士及來降。高祖以其前代舊臣,遣使迎勞,拜內史舍人。尋遷內史侍郎。



 高祖嘗幸溫湯,經秦始皇墓,謂倫曰:「古者帝王,竭生靈之力,殫府庫之財,營起山陵,此復
 何益?」倫曰:「上之化下,猶風之靡草。自秦、漢帝王盛為厚葬,故百官眾庶競相遵仿。凡是古塚丘封,悉多藏珍寶,咸見開發。若死而無知,厚葬深為虛費;若魂而有識,被發豈不痛哉!」高祖稱善,謂倫曰:「從今之後,宜自上導下,悉為薄葬。」太宗之討王世充,詔倫參謀軍事。高祖以兵久在外,意欲旋師,太宗遣倫入朝親論事勢。倫言於高祖曰:「世充得地雖多,而羈縻相屬,其所用命者,唯洛陽一城而已,計盡力窮,破在朝夕。今若還兵,賊勢必振,更
 相連結,後必難圖。未若乘其已衰,破之必矣。」高祖納之。及太宗凱旋,高祖謂侍臣曰:「朕初發兵東討,眾議多有不同,唯秦王請行,封倫贊成此計。昔張華協同晉武,亦復何以加也!」封平原縣公,兼天冊府司馬。會突厥寇太原,復遣使來請和親,高祖問群臣:「和之與戰,策將安出?」多言戰則怨深,不如先和。倫曰:「突厥憑凌,有輕中國之意,必謂兵弱而不能戰。如臣計者,莫若悉眾以擊之,其勢必捷,勝而後和,恩威兼著。若今歲不戰,明年必當復
 來,臣以擊之為便。」高祖從之。六年,以本官檢校吏部尚書,曉習吏職,甚獲當時之譽。八年,進封道國公,尋徙封於密。蕭瑀嘗薦倫於高祖,高祖任倫為中書令。太宗嗣位,瑀遷尚書左僕射,倫為右僕射。倫素險詖,與瑀商量可奏者,至太宗前,盡變易之,由是與瑀有隙。貞觀元年,遘疾於尚書省,太宗親自臨視,即命尚輦送還第,尋薨,年六十。太宗深悼之,廢朝三日,冊贈司空,謚曰明。初,倫數從太宗征討,特蒙顧遇。以建成、元吉之故,數進
 忠款,太宗以為至誠,前後賞賜以萬計。而倫潛持兩端,陰附建成。時高祖將行廢立,猶豫未決,謀之於倫,倫固諫而止。然所為秘隱,時人莫知,事具《建成傳》。卒後數年,太宗方知其事。十七年,治書侍御史唐臨追劾倫曰:「臣聞事君之義,盡命不渝;為臣之節,歲寒無貳。茍虧其道,罪不容誅。倫位望鼎司,恩隆胙土,無心報效,乃肆奸謀,熒惑儲籓,獎成元惡,置於常典,理合誅夷。但苞藏之狀,死而後發,猥加褒贈,未正嚴科。罪惡既彰,宜加貶黜,豈可仍
 疇爵邑,尚列臺槐!此而不懲,將何沮勸?」太宗令百官詳議,民部尚書唐儉等議:「倫罪暴身後,恩結生前,所歷眾官,不可追奪,請降贈改謚。」詔從之,於是改謚繆,黜其贈官,削所食實封。



 子言道,尚高祖女淮南長公主,官至宋州刺史。倫兄子行高,以文學知名。貞觀中,官至禮部郎中。



 蕭瑀,字時文。高祖梁武帝,曾祖昭明太子,祖察,後梁宣帝。父巋,明帝。瑀年九歲,封新安郡王,幼以孝行聞。姊為
 隋晉王妃,從入長安。聚學屬文,端正鯁亮。好釋氏,常修梵行,每與沙門難及苦空,必詣微旨。常觀劉孝標《辯命論》,惡其傷先王之教,迷性命之理,乃作《非辯命論》以釋之。大旨以為:「人稟天地以生,孰云非命,然吉兇禍福,亦因人而有,若一之於命,其蔽已甚。」時晉府學士柳顧言、諸葛穎見而稱之曰:「自孝標後數十年間,言性命之理者,莫能詆詰。今蕭君此論,足療劉子膏肓。」煬帝為太子也,授太子右千牛。及踐祚,遷尚衣奉御,檢校左翊衛鷹
 揚郎將。忽遇風疾,命家人不即醫療,仍云:「若天假餘年,因此望為棲遁之資耳。」蕭後聞而誨之:「以爾才智,足堪揚名顯親,豈得輕毀形骸而求隱逸?若以此致譴,則罪在不測。」病且愈,其姊勸勉之,故復有仕進志。累加銀青光祿大夫、內史侍郎。既以後弟之親,委之機務,後數以言忤旨,漸見疏斥。煬帝至雁門,為突厥所圍,瑀進謀曰:「如聞始畢托校獵至此,義成公主初不知其有違背之心。且北蕃夷俗,可賀敦知兵馬事。昔漢高祖解平城之
 圍,乃是閼氏之力。況義成以帝女為妻,必恃大國之援。若發一單使以告義成,假使無益,事亦無損。臣又竊聽輿人之誦,乃慮陛下平突厥後更事遼東,所以人心不一,或致挫敗。請下明詔告軍中,赦高麗而專攻突厥,則百姓心安,人自為戰。」煬帝從之,於是發使詣可賀敦諭旨。俄而突厥解圍去,於後獲其諜人,云:義成公主遣使告急於始畢,稱北方有警,由是突厥解圍,蓋公主之助也。煬帝又將伐遼東,謂群臣曰:「突厥狂悖為寇,勢何能為?
 以其少時未散,蕭瑀遂相恐動,情不可恕。」因出為河池郡守,即日遣之。既至郡,有山賊萬餘人寇暴縱橫,瑀潛募勇敢之士,設奇而擊之,當陣而降其眾。所獲財畜,咸賞有功,由是人竭其力。薛舉遣眾數萬侵掠郡境,瑀要擊之,自後諸賊莫敢進,郡中復安。



 高祖定京城,遣書招之。瑀以郡歸國,授光祿大夫,封宋國公,拜民部尚書。太宗為右元帥,攻洛陽,以瑀為府司馬。武德五年,遷內史令。時軍國草創,方隅未寧,高祖乃委以心腹,凡諸政務,
 莫不關掌。高祖每臨軒聽政,必賜升御榻,瑀既獨孤氏之婿,與語呼之為蕭郎。國典朝儀,亦責成於瑀,瑀孜孜自勉,繩違舉過,人皆憚之。常奏便宜數十條,多見納用,手敕曰:「得公之言,社稷所賴。運智者之策,以能成人之美;納諫者之言,以金寶酬其德。今賜金一函,以報智者,勿為推退。」瑀固辭,優詔不許。其年,州置七職,務取才望兼美者為之。及太宗臨雍州牧,以瑀為州都督。高祖常有敕而中書不時宣行,高祖責其遲,瑀曰:「臣大業之日,
 見內史宣敕,或前後相乖者,百司行之,不知何所承用。所謂易必在前,難必在後,臣在中書日久,備見其事。今皇基初構,事涉安危,遠方有疑,恐失機會。比每受一敕,臣必勘審,使與前敕不相乖背者,始敢宣行。遲晚之愆,實由於此。」高祖曰:「卿能用心若此,我有何憂?」初,瑀之朝也,關內產業並先給勛人。至是特還其田宅,瑀皆分給諸宗子弟,唯留廟堂一所,以奉烝嘗。及平王世充,瑀以預軍謀之功,加邑二千戶,拜尚書右僕射。內外考績皆
 委之司會,為群僚指南,庶務繁總。瑀見事有時偏駁,而持法稍深,頗為時議所少。瑀嘗薦封倫於高祖,高祖以倫為中書令。太宗即位,遷尚書左僕射,封倫為右僕射。倫素懷險詖,與商量將為可奏者,至太宗前盡變易之。於時房玄齡、杜如晦既新用事,疏瑀親倫,瑀心不能平,遂上封事論之,而辭旨寥落。太宗以玄齡等功高,由是忤旨,廢於家。俄拜特進、太子少師。未幾,復為尚書左僕射,賜實封六百戶。太宗常謂瑀曰:「朕欲使子孫長久,社
 稷永安,其理如何?」瑀對曰:「臣觀前代國祚所以長久者,莫若封諸侯以為盤石之固。秦並六國,罷侯置守,二代而亡;漢有天下,郡國參建,亦得年餘四百。魏、晉廢之,不能永久。封建之法,實可遵行。」太宗然之,始議封建。尋坐與侍中陳叔達於上前忿諍,聲色甚厲,以不敬免。歲餘,授晉州都督。明年,徵授左光祿大夫,兼領御史大夫。與宰臣參議朝政,瑀多辭辯,每有評議,玄齡等不能抗。然心知其是,不用其言,瑀彌怏怏。玄齡、魏徵、溫彥博嘗有
 微過,瑀劾之,而罪竟不問,因此自失。由是罷御史大夫,以為太子少傅,不復預聞朝政。六年,授特進,行太常卿。八年,為河南道巡省大使,人有坐當推劾苦未得其情者,遂置格置繩,以至於死,太宗特免責之。九年,拜特進,復令參預政事。太宗嘗從容謂房玄齡曰:「蕭瑀大業之日,進諫隋主,出為河池郡守。應遭割心之禍,翻見太平之日,北叟失馬,事亦難常。」瑀頓首拜謝。太宗又曰:「武德六年以後,太上皇有廢立之心而不之定也,我當此日,
 不為兄弟所容,實有功高不賞之懼。此人不可以厚利誘之,不可以刑戮懼之,真社稷臣也。」因賜瑀詩曰:「疾風知勁草,版蕩識誠臣。」又謂瑀曰:「卿之守道耿介,古人無以過也。然而善惡太明,亦有時而失。」瑀再拜謝曰:「臣特蒙誡訓,又許臣以忠諒,雖死之日,猶生之年也。」魏徵進而言曰:「臣有逆眾以執法,明主恕之以忠;臣有孤特以執節,明主恕之以勁。昔聞其言,今睹其實,蕭瑀不遇明聖,必及於難!」太宗悅其言。



 十七年,與長孫無忌等二
 十四人並圖形於凌煙閣。是歲,立晉王為皇太子,拜瑀太子太保,仍知政事。太宗之伐遼東也,以洛邑沖要,襟帶關、河,以瑀為洛陽宮守。車駕自遼還,請解太保,仍同中書門下。太宗以瑀好佛道,嘗賚繡佛像一軀,並繡瑀形狀於佛像側,以為供養之容。又賜王褒所書《大品般若經》一部,並賜袈裟,以充講誦之服焉。瑀嘗稱:「玄齡以下同中書門下內臣,悉皆朋黨比周,無至心奉上。」累獨奏云:「此等相與執權,有同膠漆,陛下不細諳知,但未反耳。」
 太宗謂瑀曰:「為人君者,驅駕英材,推心待士,公言不亦甚乎,何至如此!」太宗數日謂瑀曰:「知臣莫若君,夫人不可求備,自當舍其短而用其長。朕雖才謝聰明,不應頓迷臧否。」因數為瑀信誓。瑀既不自得,而太宗積久銜之,終以瑀忠貞居多而未廢也。會瑀請出家,太宗謂曰:「甚知公素愛桑門,今者不能違意。」瑀旋踵奏曰:「臣頃思量,不能出家。」太宗以對群臣吐言,而取舍相違,心不能平。瑀尋稱足疾,時詣朝堂,又不入見,太宗謂侍臣曰:「瑀豈
 不得其所乎,而自慊如此?」遂手詔曰:



 朕聞物之順也,雖異質而成功;事之違也,亦同形而罕用。是以舟浮楫舉,可濟千里之川;轅引輪停,不越一毫之地。故知動靜相循易為務,曲直相反難為功,況乎上下之宜、君臣之際者矣。朕以無明於元首,期托德於股肱,思欲去偽歸真,除澆反樸。至於佛教,非意所遵,雖有國之常經,固弊俗之虛術。何則?求其道者,未驗福於將來;修其教者,翻受辜於既往。至若梁武窮心於釋氏,簡文銳意於法門,傾
 帑藏以給僧祗,殫人力以供塔廟。及乎三淮沸浪,五嶺騰煙,假餘息於熊蹯,引殘魂於雀穀。子孫覆亡而不暇,社稷俄頃而為墟,報施之徵,何其繆也!而太子太保、宋國公瑀踐覆車之餘軌,襲亡國之遺風。棄公就私,未明隱顯之際;身俗口道,莫辯邪正之心。修累葉之殃源,祈一躬之福本,上以違忤君主,下則扇習浮華。往前朕謂張亮云:「卿既事佛,何不出家?」瑀乃端然自應,請先入道,朕即許之,尋復不用。一回一惑,在於瞬息之間;自可自
 否,變於帷扆之所。乖棟梁之大體,豈具瞻之量乎?朕猶隱忍至今,瑀尚全無悛改。宜即去茲朝闕,出牧小籓,可商州刺史,仍除其封。



 二十一年,徵授金紫光祿大夫,復封宋國公。從幸玉華宮,遘疾薨於宮所,年七十四。太宗聞而輟膳,高宗為之舉哀,遣使吊祭。太常謚曰「肅」。太宗曰:「易名之典,必考其行。瑀性多猜貳,此謚失於不直,更宜摭實。」改謚曰貞褊公。冊贈司空、荊州都督,賜東園秘器,陪葬昭陵。臨終遺書曰:「生而必死,理之常分。氣絕後
 可著單服一通,以充小斂。棺內施單席而已,冀其速朽,不得別加一物。無假卜日,惟在速辦。自古賢哲,非無等例,爾宜勉之。」諸子遵其遺志,斂葬儉薄。



 子銳嗣,尚太宗女襄城公主,歷太常卿、汾州刺史。公主雅有禮度,太宗每令諸公主,凡厥所為,皆視其楷則。又令所司別為營第,公主辭曰:「婦人事舅姑如事父母,若居處不同,則定省多闕。」再三固讓,乃止,令於舊宅而改創焉。永徽初,公主薨,詔葬昭陵。



 瑀兄璟,亦有學行。武德中為黃門侍郎,
 累轉秘書監,封蘭陵縣公。貞觀中卒,贈禮部尚書。



 瑀兄子鈞,隋遷州刺史、梁國公珣之子也。博學有才望。貞觀中,累除中書舍人,甚為房玄齡、魏徵所重。永徽二年,歷遷諫議大夫,兼弘文館學士。時有左武候別駕盧文操,逾垣盜左藏庫物,高宗以別駕職在糾繩,身行盜竊,命有司殺之。鈞進諫曰:「文操所犯,情實難原。然恐天下聞之,必謂陛下輕法律,賤人命,任喜怒,貴財物。臣之所職,以諫為名,愚衷所懷,不敢不奏。」帝謂曰:「卿職在司諫,能
 盡忠規。」遂特免其死罪,顧謂侍臣曰:「此乃真諫議也。」尋而太常樂工宋四通等,為宮人通傳信物,高宗特令處死,乃遣附律,鈞上疏言:「四通等犯在未附律前,不合至死。」手詔曰:「朕聞防禍未萌,先賢所重,宮闕之禁,其可漸歟?昔如姬竊符,朕用為永鑒,不欲今茲自彰其過,所搦憲章,想非濫也。但朕翹心紫禁,思覿引裾,側席硃楹,冀旌折檻。今乃喜得其言,特免四通等死,遠處配流。」鈞尋為太子率更令,兼崇賢館學士。顯慶中卒。所撰《韻旨》二
 十卷,有集三十卷行於代。



 子瓘,官至渝州長史。母終,以毀卒。瓘子嵩,別有傳。



 鈞兄子嗣業,少隨祖姑隋煬帝後入於突厥。貞觀九年歸朝,以深識蕃情充使,統領突厥之眾。累轉鴻臚卿,兼單于都護府長史。調露中,單於突厥反叛,嗣業率兵戰,敗,配流嶺南而死。



 裴矩,字弘大,河東聞喜人。祖佗,後魏東荊州刺史。父訥之,北齊太子舍人。矩襁褓而孤,為伯父讓之所鞠。及長,博學,早知名,仕齊為高平王文學。齊亡,隋文帝為定州
 總管,召補記室,甚親敬之。文帝即位,遷給事郎,直內史省,奏舍人事。伐陳之役,領元帥記室。及陳平,晉王廣令矩與高熲收陳圖籍,歸之秘府。累遷吏部侍郎,以事免。大業初,西域諸蕃款張掖塞與中國互市,煬帝遣矩監其事。矩知帝方勤遠略,欲吞並夷狄,乃訪西域風俗及山川險易、君長姓族、物產服章,撰《西域圖記》三卷,入朝奏之。帝大悅,賜物五百段。每日引至御座,顧問西方之事。矩盛言西域多珍寶及吐谷渾可並之狀,帝信之。仍委
 以經略,拜民部侍郎。俄遷黃門侍郎,參預朝政。令往張掖引致西蕃,至者十餘國。三年,帝有事於恆岳,咸來助祭。帝將巡河右,復令矩往燉煌,矩遣使說高昌王鞠伯雅及伊吾吐屯設等,啖以厚利,導使入朝。及帝西巡,次燕支山,高昌王、伊吾設等及西蕃胡二十七國,盛服珠玉錦罽,焚香奏樂,歌舞相趨,謁於道左。復令武威、張掖士女盛飾縱觀,填咽周亙數十里,帝見之大悅。及滅吐谷渾,蠻夷納貢,諸蕃懾服,相繼來庭。雖拓地數千里,
 而役戍委輸之費,歲巨萬計,中國騷動焉。帝以矩有綏懷之略,加位銀青光祿大夫。其年,帝至東都,矩以蠻夷朝貢者多,諷帝大征四方奇技,作魚龍曼延角牴於洛邑,以誇諸戎狄,終月而罷。又令三市店肆皆設帷帳,盛酒食,遣掌蕃率蠻夷與人貿易,所至處悉令邀延就座,醉飽而散。夷人有識者,咸私哂其矯飾焉。帝稱矩至誠,謂宇文述、牛弘曰:「裴矩大識朕意,凡所陳奏,皆朕之成算,朕未發頃,矩輒以聞。自非奉國用心,孰能若是?」尋令
 與將軍薛世雄城伊吾而還,賜錢四十萬。矩因進計縱反間於射匱,使潛攻處羅。後處羅為射匱所迫,竟隨使者入朝,帝甚悅,賜矩貂裘及西域珍器。從帝巡於塞北,幸啟民可汗帳。時高麗遣使先通於突厥,啟民不敢隱,引之見帝。矩因奏曰:「高麗之地,本孤竹國也,周代以之封箕子,漢時分為三郡,晉氏亦統遼東。今乃不臣,列為外域,故先帝欲征之久矣,但以楊諒不肖,師出無功。當陛下時,安得不有事於此,使冠帶之境,仍為蠻貊之鄉
 乎?今其使者朝於突厥,親見啟民從化,必懼皇靈之遠暢,慮後服之先亡,脅令入朝,當可致也。請面詔其使還本國,遣詔其王令速朝覲。不然者,當率突厥即日誅之。」帝納焉。高麗不用命,始建征遼之策。王師臨遼,以本官領虎賁郎將。明年,復從至遼東。兵部侍郎斛斯政亡入高麗,帝令矩兼掌兵部事。以前後渡遼功,進位右光祿大夫。矩後從幸江都。及義兵入關,屈突通敗問至,帝問矩方略,矩曰:「太原有變,京畿不靜,遙為處分,恐失事機。
 唯鑾輿早還,方可平定。」矩見天下將亂,恐為身禍,每遇人盡禮,雖至胥吏,皆得其歡心。時從駕驍果多逃散,矩言於帝曰:「車駕留此,已經二歲,人無匹合,則不能久安。請聽兵士於此納室,私相奔誘者,因而配之。」帝從其計,軍中漸安,咸曰:「裴公之惠也。」是時,帝既昏侈逾甚,矩無所諫諍,但悅媚取容而已。宇文化及弒逆,署為尚書右僕射。化及敗,竇建德復以為尚書右僕射,令專掌選事。時建德起自群盜,事無節文,矩為之創定朝儀,權設法
 律,憲章頗備,建德大悅,每諮訪焉。



 及建德敗,矩與偽將曹旦及建德之妻齎傳國八璽,舉山東之地來降,封安邑縣公。武德五年,拜太子左庶子。俄遷太子詹事。令與虞世南撰《吉兇書儀》,參按故實,甚合禮度,為學者所稱,至今行之。八年,兼檢校侍中。及太子建成被誅,其餘黨尚保宮城,欲與秦王決戰,王遣矩曉諭之,宮兵乃散。尋遷民部尚書。矩年且八十,而精爽不衰,以曉習故事,甚見推重。太宗初即位,務止奸吏,或聞諸曹案典,多有受
 賂者,乃遣人以財物試之。有司門令史受饋絹一匹,太宗怒,將殺之,矩進諫曰:「此人受賂,誠合重誅。但陛下以物試之,即行極法,所謂陷人以罪,恐非導德齊禮之義。」太宗納其言,因召百僚謂曰:「裴矩遂能廷折,不肯面從,每事如此,天下何憂不治!」貞觀元年卒,贈絳州刺史,謚曰敬。撰《開業平陳記》十一卷,行於代。



 子宣機,高宗時官至銀青光祿大夫、太子左中護。



 宇文士及,雍州長安人。隋右衛大將軍述子,化及弟也。
 開皇末,以父勛封新城縣公。隋文帝嘗引入臥內,與語,奇之,令尚煬帝女南陽公主。大業中,歷尚輦奉御,從幸江都。以父憂去職,尋起為鴻臚少卿。化及之潛謀逆亂也,以其主婿,深忌之而不告,既弒煬帝,署為內史令。初,高祖為殿內少監,時士及為奉御,深自結托。及隨化及至黎陽,高祖手詔召之。士及亦潛遣家僮間道詣長安申赤心,又因使密貢金環。高祖大悅,謂侍臣曰:「我與士及素經共事,今貢金環,是其來意也。」及至魏縣,兵威日
 蹙,士及勸之西歸長安,化及不從,士及乃與封倫求於濟北征督軍糧。俄而化及為竇建德所擒,濟北豪右多勸士及發青、齊之眾,北擊建德,收河北之地,以觀形勢。士及不納,遂與封倫等來降。高祖數之曰:「汝兄弟率思歸之卒,為入關之計,當此之時,若得我父子,豈肯相存,今欲何地自處?」士及謝曰:「臣之罪誠不容誅,但臣早奉龍顏,久存心腹,往在涿郡,嘗夜中密論時事,後於汾陰宮,復盡丹赤。自陛下龍飛九五,臣實傾心西歸,所以密
 申貢獻,冀此贖罪耳。」高祖笑謂裴寂曰:「此人與我言天下事,至今已六七年矣,公輩皆在其後。」時士及妹為昭儀,有寵,由是漸見親待,授上儀同。從太宗平宋金剛,以功復封新城縣公,妻以壽光縣主,仍遷秦王府驃騎將軍。又從平王世充、竇建德,以功進爵郢國公,遷中書侍郎,再轉太子詹事。太宗即位,代封倫為中書令,真食益州七百戶。尋以本官檢校涼州都督。時突厥屢為邊寇,士及欲立威以鎮邊服,每出入陳兵,盛為容衛;又折節
 禮士,涼士服其威惠。徵為殿中監,以疾出為蒲州刺史,為政寬簡,吏人安之。數歲,入為右衛大將軍,甚見親顧,每延入閣中,乙夜方出,遇其歸沐,仍遣馳召,同列莫與為比。然尤謹密,其妻每問向中使召有何樂事,士及終無所言。尋錄其功,別封一子為新城縣公。在職七年,復為殿中監,加金紫光祿大夫。及疾篤,太宗親問,撫之流涕。貞觀十六年卒,贈左衛大將軍、涼州都督,陪葬昭陵。士及撫幼弟及孤兄子,以友愛見稱,親戚故人貧乏者,
 輒遺之。然厚自封植,衣食服玩必極奢侈。謚曰「恭」,黃門侍郎劉洎駁之曰:「士及居家侈縱,不宜為恭。」竟謚曰縱。



 史臣曰:封倫多揣摩之才,有附托之巧。黨化及而數煬帝,或有赧顏;托士及以歸唐朝,殊無愧色。當建成之際,事持兩端;背蕭瑀之恩,奏多異議。太宗,明主也,不見其心;玄齡,賢相焉,尚容其諂。狡算醜行,死而後彰,茍非唐臨之劾,唐儉等議,則奸人得計矣。蕭瑀骨鯁亮直,儒術清明。執政隋朝,忠而獲罪;委質高祖,知無不為。及太宗
 臨朝,房、杜用事,不容小過,欲居成功,既形猜貳之言,寧固或躍之位?易名而祗加「褊」字,所幸者猶多;奉佛而不失道情,非善也而何謂。裴矩方略寬簡,士及通變謹密,皆一時之稱也。



 贊曰:封倫揣摩諂詐,蕭瑀骨鯁儒術。裴矩方略寬簡,士及通變謹密。



\end{pinyinscope}