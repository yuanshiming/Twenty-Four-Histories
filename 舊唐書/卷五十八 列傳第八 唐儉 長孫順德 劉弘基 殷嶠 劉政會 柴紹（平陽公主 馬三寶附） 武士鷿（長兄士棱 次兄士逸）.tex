\article{卷五十八 列傳第八 唐儉 長孫順德 劉弘基 殷嶠 劉政會 柴紹(平陽公主 馬三寶附) 武士鷿(長兄士棱 次兄士逸)}

\begin{pinyinscope}

 ○唐
 儉長孫順德劉弘基殷嶠劉政會柴紹平陽公主馬三寶附武士鷿長兄士棱次兄士逸



 唐儉,字茂約,並州晉陽人,北齊尚書左僕射邕之孫也。
 父鑒,隋戎州刺史。儉落拓不拘規檢,然事親頗以孝聞。初,鑒與高祖有舊,同領禁衛。高祖在太原留守,儉與太宗周密,儉從容說太宗以隋室昏亂,天下可圖。太宗白高祖,乃召入,密訪時事。儉曰:「明公日角龍庭,李氏又在圖牒,天下屬望,非在今朝。若開府庫,南嘯豪傑,北招戎狄,東收燕、趙,長驅濟河,據有秦、雍,海內之權,指麾可取。願弘達節,以順群望,則湯、武之業不遠。」高祖曰:「湯、武之事,非所庶幾。今天下已亂,言私則圖存,語公則拯溺。卿
 宜自愛,吾將思之。」及開大將軍府,授儉記室參軍。太宗為渭北道行軍元帥,以儉為司馬。平京城,加光祿大夫、相國府記室,封晉昌郡公。武德元年,除內史舍人,尋遷中書侍郎,特加授散騎常侍。



 王行本守蒲州城不降,敕工部尚書獨孤懷恩率兵屯於其東,以經略之。尋又夏縣人呂崇茂以城叛,降於劉武周,高祖遣永安王孝基、工部尚書獨孤懷恩、陜州總管於筠等率兵討之。時儉使至軍所,屬武周遣兵援崇茂,儉與孝基、筠等並為所
 獲。初,懷恩屯兵蒲州,與其屬元君實謀反,時君實亦陷於賊中,與儉同被拘執,乃謂儉曰:「古人有言:『當斷不斷,反受其亂。』獨孤尚書近者欲舉兵圖事,遲疑之間,遂至今日,豈不由不斷耶?」俄而懷恩脫身得還,仍令依前屯守,君實又謂儉曰:「獨孤尚書今遂拔難得還,復在蒲州屯守,可謂王者不死。」儉聞之,懼懷恩為逆,乃密令親信劉世讓以懷恩之謀奏聞。適遇王行本以蒲州歸降,高祖將入其城,浮舟至中流,世讓謁見,高祖讀奏,大驚曰:「
 豈非天命也!」回舟而歸,分捕反者按驗之,懷恩自縊,餘黨伏誅。俄而太宗擊破武周部將宋金剛,追至太原,武周懼而北走,儉乃封其府庫,收兵甲,以待太宗。高祖嘉儉身沒虜庭,心存朝闕,復舊官,仍為並州道安撫大使,以便宜從事,並賜獨狐懷恩田宅貲財等。使還,拜禮部尚書,授天策府長史,兼檢校黃門侍郎,封莒國公,與功臣等元勛恕一死,仍除遂州都督,食綿州實封六百戶,圖形凌煙閣。



 貞觀初,使於突厥,說誘之,因以隋蕭後及
 楊正道以歸。太宗謂儉曰:「卿觀頡利可圖否?」對曰:「銜國威恩,亦可望獲。」遂令儉馳傳至虜庭,示之威信。頡利部落歡然定歸款之計,因而兵眾弛懈。李靖率輕騎掩襲破之,頡利北走,儉脫身而還。歲餘,授民部尚書。後從幸洛陽苑射猛獸,群豕突出林中,太宗引弓四發,殪四豕,有雄彘突及馬鐙,儉投馬搏之,太宗拔劍斷豕,顧笑曰:「天策長史,不見上將擊賊耶!何懼之甚?」對曰:「漢祖以馬上得之,不以馬上治之;陛下以神武定四方,豈復逞雄
 心於一獸。」太宗納之,因為罷獵。尋加光祿大夫,又特令其子善識尚豫章公主。儉在官每盛修肴饌,與親賓縱酒為樂,未嘗以職務留意。又嘗托鹽州刺史張臣合收其私羊,為御史所劾,以舊恩免罪,貶授光祿大夫。永徽初,致仕於家,加特進。顯慶元年卒,年七十八。高宗為之舉哀,罷朝三日,贈開府儀同三司、並州都督,賻布帛一千段、粟一千石,賜東園秘器,陪葬昭陵,謚曰襄,官為立碑。



 儉少子觀,最知名,官至河西令,有文集三卷。儉孫從
 心,神龍中,以子晙娶太平公主女,官至殿中監。晙,先天中為太常少卿,坐與太平連謀,伏誅。



 長孫順德,文德順聖皇后之族叔也。祖澄,周秦州刺史。父愷,隋開府。順德仕隋右勛衛,避遼東之役,逃匿於太原,深為高祖、太宗所親委。時群盜並起,郡縣各募兵為備。太宗外以討賊為名,因令順德與劉弘基等召募,旬月之間,眾至萬餘人,結營於郭下,遂誅王威、高君雅等。義兵起,拜統軍。從平霍邑,破臨汾,下絳郡,俱有戰功。尋
 與劉文靜擊屈突通於潼關,每戰摧鋒。及通將奔洛陽,順德追及於桃林,執通歸京師,仍略定陜縣。高祖即位,拜左驍衛大將軍,封薛國公。武德九年,與秦叔寶等討建成餘黨於玄武門。太宗踐祚,真食千二百戶,特賜以宮女,每宿內省。



 後,順德監奴,受人饋絹事發,太宗謂近臣曰:「順德地居外戚,功即元勛,位高爵厚,足稱富貴。若能勤覽古今,以自鑒誡,弘益我國家者,朕當與之同有府庫耳。何乃不遵名節,而貪冒發聞乎!」然惜其功,不忍
 加罪,遂於殿庭賜絹數十匹,以愧其心。大理少卿胡演進曰:「順德枉法受財,罪不可恕,奈何又賜之絹?」太宗曰:「人生性靈,得絹甚於刑戮;如不知愧,一禽獸耳,殺之何益!」尋坐與李孝常交通除名。歲餘,太宗閱功臣圖,見順德之像,閔然憐之,遣宇文士及視其所為,見順德頹然而醉,論者以為達命。召拜澤州刺史,復其爵邑。順德素多放縱,不遵法度,及此折節為政,號為明肅。先是,長吏多受百姓饋餉,順德糾擿,一無所容,稱為良牧。前刺史
 張長貴、趙士達並占境內膏腴之田數十頃,順德並劾而追奪,分給貧戶。尋又坐事免。發疾,太宗聞而鄙之,謂房玄齡曰:「順德無慷慨之節,多兒女之情,今有此疾,何足問也!」未幾而卒,太宗為之罷朝,遣使吊祭,贈荊州都督,謚曰襄。貞觀十三年,追改封為邳國公。永徽五年,重贈開府儀同三司。



 劉弘基,雍州池陽人也。父升,隋河州刺史。弘基少落拓,交通輕俠,不事家產,以父廕為右勛侍。大業末,嘗從煬
 帝征遼東,家貧不能自致,行至汾陰,度已後期當斬,計無所出,遂與同旅屠牛,潛諷吏捕之,系於縣獄,歲餘,竟以贖論。事解亡命,盜馬以供衣食,因至太原。會高祖鎮太原,遂自結托,又察太宗有非常之度,尤委心焉。由是大蒙親禮,出則連騎,入同臥起。義兵將舉,弘基召募得二千人。王威、高君雅欲為變,高祖伏弘基及長孫順德於事之後,弘基因麾左右執威等。又從太宗攻下西河。義軍次賈胡堡,與隋將宋老生戰,破之,進攻霍邑。老
 生率眾陣於城外,弘基從太宗擊之,老生敗走,棄馬投塹,弘基下斬其首,拜右光祿大夫。師至河東,弘基以兵千人先濟河,進下馮翊,為渭北道大使,得便宜從事,以殷開山為副。西略地扶風,有眾六萬。南渡渭水,屯於長安故城,威聲大振,耀軍金光門。衛文升遣兵來戰,弘基逆擊走之,擒甲士千餘人、馬數百匹。時諸軍未至,弘基先至,一戰而捷。高祖大悅,賜馬二十匹。及破京城,功為第一。從太宗擊薛舉於扶風,破之,追奔至隴山而返。累
 拜右領都督,封河間郡公。又從太宗經略東都,戰於瓔珞門外,破之。師旋,弘基為殿。隋將段達、張志陳於三王陵,弘基擊敗之。武德元年,拜右驍衛大將軍,以元謀之勛,恕其一死,領行軍左一總管。又從太宗討薛舉。時太宗以疾頓於高墌城,弘基、劉文靜等與舉接戰於淺水原,王師不利,八總管咸敗;唯弘基一軍盡力苦鬥,矢盡,為舉所獲。高祖嘉其臨難不屈,賜其家粟帛甚厚。仁杲平,得歸,復其官爵。會宋金剛陷太原,遣弘基屯晉州。裴
 寂為宋金剛所敗,人情崩駭,莫有固志。金剛以兵造城下,弘基不能守,復陷於賊。俄得逃歸,高祖慰諭之,授左一總管。從太宗屯於柏壁,率兵二千自隰州趨西河,斷賊歸路。時賊鋒甚勁,弘基堅壁,不能進。及金剛遁,弘基率騎邀之,至於介休,與太宗會,追擊大破之。累封任國公。尋從擊劉黑闥於洺州,師旋,授秉鉞將軍。會突厥入寇,弘基率步騎一萬,自豳州北界東拒子午嶺,西接臨涇,修營障塞,副淮安王神通,備胡寇於北鄙。九年,以佐命功,真食九百戶。



 太宗即
 位,顧待益隆。李孝常、長孫安業之謀逆也,坐與交游除名。歲餘,起為易州刺史,復其封爵,徵拜衛尉卿。九年,改封夔國公,世襲朗州刺史,例停不行。後以年老乞骸骨,授輔國大將軍,朝朔望,祿賜同於職事。太宗征遼東,以弘基為前軍大總管。從擊高延壽於駐蹕山,力戰有功,太宗屢加勞勉。永徽元年加實封通前一千一百戶。其年卒,年六十九。高宗為之舉哀,廢朝三日,贈開府儀同三司、並州都督,陪葬昭陵,仍為立碑,謚曰襄。弘基遺令給諸子奴婢各十五人、良田五頃,謂所親曰:「若賢,固不藉多財;不賢,守此可以免饑凍。」餘財悉以散施。



 子仁實襲,官至左典戎衛郎將。從子仁景,神龍初,官至司農卿。



 殷嶠,字開山,雍州鄠縣人,陳司農卿不害孫也。其先本居陳郡,陳亡,徙關中。父僧首,隋秘書丞,有名於世。嶠少以學行見稱,尤工尺牘。仕隋太谷長,有治名。義兵起,召補大
 將軍府掾,參預謀略,授心腹之寄,累以軍功拜光祿大夫。從隱太子攻克西河。太宗為渭北道元帥,引為長史。時關中群盜往往聚結,眾無適從,令嶠招慰之,所至皆下。又與統軍劉弘基率兵六萬屯長安故城,隋將衛孝節自金光門出戰,嶠與弘基擊破之。京城
 平,賜爵陳郡公,遷丞相府掾。尋授吏部侍郎。從擊薛舉,為元帥府司馬。時太宗有疾,委軍於劉文靜,誡之曰:「賊眾遠來,利在急戰,難與爭鋒。且宜持久,待糧盡,然後可圖。」嶠退謂文靜曰:「王體不安,慮公不濟,故發此言。宜因機破賊,何乃以勍敵遺王也!」久之,言於文靜曰:「王不豫,恐賊輕我,請耀武以威之。」遂陳兵於折墌,為舉所乘,軍乃大敗。嶠坐減死除名。後從平薛仁杲,復其爵位。武德二年,兼陜東道大行臺兵部尚書,遷吏部尚書。從太宗討平王世充,以功進爵鄖國公。復從征劉黑闥,道病卒。太宗親臨喪,哭之甚慟,贈陜東道大行臺右僕射,謚曰節。貞觀十四年,詔與贈司空、淮安王神通,贈司空、河間王孝恭,贈民部尚書劉政會,俱以佐命功配饗高祖廟庭。十七年,又與長孫無忌、唐儉、長孫順德、劉弘基、劉政會、柴紹等十七人,俱圖其形於凌煙閣。永徽五年,追贈司空。



 嶠從祖弟聞禮,有文學,武德中,為太子中舍人,修梁
 史,未就而卒。聞禮子仲容,亦知名,則天深愛其才。官至申州刺史。



 劉政會,滑州胙城人也。祖環雋,北齊中書侍郎。政會,隋大業中為太原鷹揚府司馬。高祖為太原留守,政會率兵隸於麾下。太宗與劉文靜謀起義兵,副留守王威、高君雅獨懷猜貳。後數日,將大會於晉祠,威與君雅謀危高祖。有人以白,太宗既知迫急,欲先事誅之,因遣政會為急變之書,詣留守告威等二人謀反。是日,高祖與威、君雅同坐視事,文靜引政會入,至庭中,云有密狀,知人欲反。高祖指威等令視之,政會不肯,曰:「所告是副留守事,唯唐公得省之耳。」君雅攘袂大呼曰:「此是反人,欲殺我也!」時太宗已列兵馬布於街巷,文靜因令左右引威等囚於別室。既拘威等,竟得舉兵,政會之功也。大將軍府建,引為戶曹參軍。從平長安,除丞相府掾。武德初,授衛尉少卿,留守太原。政會內輯軍士,外和戎狄,遠近莫不悅服。尋而劉武周進逼並州,晉陽豪右薛深等以城應賊,政會為賊所擒,於賊中密表論武周形勢。賊平,復其官爵。歷刑部尚書、光祿卿,封邢國公。貞觀初,累轉洪州都督,賜實封三百戶。九年卒,太宗手敕曰:「舉義之日,實有殊功,所葬並宜優厚。」贈民部尚書,謚曰襄。後與殷開山同配饗高祖廟庭。



 子玄意襲爵,改封渝國公,尚南平公主,授駙馬都尉。高宗時為汝州刺史。次子奇,長壽中為天官侍郎,為酷吏所陷也。



 柴紹,字嗣昌,晉州臨汾人也。祖烈,周驃騎大將軍,歷遂、梁二州刺史,封冠軍縣公。父慎,隋太子右內率,封鉅鹿郡
 公。紹幼趫捷有勇力,任俠聞於關中。少補隋元德太子千牛備身。高祖微時,妻之以女,即平陽公主也。



 義旗建,紹自京間路趣太原。時建成、元吉自河東往,會於道,建成謀於紹曰:「追書甚急,恐已起事。隋郡縣連城千有餘里,中間偷路,勢必不全,今欲且投小賊,權以自濟。」紹曰:「不可。追既急,宜速去,雖稍辛苦,終當獲全。若投小賊,知君唐公之子,執以為功,徒然死耳。」建成從之,遂共走太原。入雀鼠谷,知已起義,於是相賀,以紹之計為得。授右領軍大都督府長史。大軍發晉陽,兼領馬軍總管。將至霍邑,紹先至城下,察宋老生形勢,白曰:「老生有匹夫之勇,我師若到,必來出戰,戰則成擒矣。」及義師至,老生果出,紹力戰有功。下臨汾,平絳郡,並先登陷陣,授右光
 祿大夫。隋將桑顯和來擊,孫華率精銳渡河以援之,紹引軍直掩其背,與史大奈合勢擊之,顯和大敗,因與諸將進下京城。武德元年,累遷左翊衛大將軍。尋從太宗平薛舉,破宋金剛,攻平王世充於洛陽,擒竇建德於武牢,封霍國公,賜實封千二百戶,轉右驍衛大將軍。吐谷渾與黨項俱來寇邊,命紹討之。虜據高臨下,射紹軍中,矢下如雨。紹乃遣人彈胡琵琶,二女子對舞,虜異之,駐弓矢而相與聚觀。紹見虜陣不整,密使精騎自後擊之,
 虜大潰,斬首五百餘級。貞觀元年,拜右衛大將軍。二年,擊梁師都於夏州,平之。轉左衛大將軍,出為華州刺史。七年,加鎮軍大將軍,行右驍衛大將軍,改封譙國公。十二年,寢疾,太宗親自臨問。尋卒,贈荊州都督,謚曰襄。



 平陽公主,高祖第三女也,太穆皇后所生。義兵將起,公主與紹並在長安,遣使密召之。紹謂公主曰:「尊公將掃清多難,紹欲迎接義旗;同去則不可,獨行恐罹後患,為計若何?」公主曰:「君宜速去。我一婦人,臨時易可藏隱,當別
 自為計矣。」紹即間行赴太原。公主乃歸鄠縣莊所,遂散家資,招引山中亡命,得數百人,起兵以應高祖。時有胡賊何潘仁聚眾於司竹園,自稱總管,未有所屬。公主遣家僮馬三寶說以利害,潘仁攻鄠縣,陷之。三寶又說群盜李仲文、向善志、丘師利等,各率眾數千人來會。時京師留守頻遣軍討公主,三寶、潘仁屢挫其鋒。公主掠地至盩厔、武功、始平,皆下之。每申明法令,禁兵士,無得侵掠,故遠近奔赴者甚眾,得兵七萬人。公主令間使以聞,
 高祖大悅。及義軍渡河,遣紹將數百騎趨華陰,傍南山以迎公主。時公主引精兵萬餘與太宗軍會於渭北,與紹各置幕府,俱圍京城,營中號曰「娘子軍」。京城平,封為平陽公主,以獨有軍功,每賞賜異於他主。六年,薨。及將葬,詔加前後部羽葆鼓吹、大輅、麾幢、班劍四十人、虎賁甲卒。太常奏議,以禮,婦人無鼓吹。高祖曰:「鼓吹,軍樂也。往者公主於司竹舉兵以應義旗,親執金鼓,有克定之勛。周之文母,列於十亂;公主功參佐命,非常婦人之所
 匹也。何得無鼓吹!」遂特加之,以旌殊績;仍令所司按謚法「明德有功曰昭」,謚公主為昭。



 子哲威,歷右屯營將軍,襲爵譙國公。坐弟令武謀反,徙嶺南。起為交州都督,卒官。令武尚巴陵公主,累除太僕少卿、衛州刺史,封襄陽郡公。永徽中,坐與公主及房遺愛謀反,遣使收之。行至華陰,自殺,仍戮其尸。公主賜死。



 馬三寶,初以平京城功,拜太子監門率。別擊叛胡劉拔真於北山,破之。又從平薛仁杲,遷左驍衛將軍。復從柴紹擊吐谷渾於岷州,先
 鋒陷陣,斬其名王,前後虜男女數千口,累封新興縣公。嘗從幸司竹,高祖顧謂三寶曰:「是汝建英雄之處,衛青大不惡!」累除左驍衛大將軍。貞觀三年卒。太宗為之廢朝,謚曰忠。



 武士鷿,並州文水人也。家富於財,頗好交結。高祖初行軍於汾、晉,休止其家;因蒙顧接,及為太原留守,引為行軍司鎧。時盜賊蜂起,士鷿嘗陰勸高祖舉兵,自進兵書及符瑞,高祖謂曰:「幸勿多言。兵書禁物,尚能將來,深識
 雅意,當同富貴耳。」及義兵將起,高祖募人,遣劉弘基、長孫順德等分統之。王威、高君雅陰謂士鷿曰:「弘基等皆背征三衛,所犯當死,安得領兵?吾欲禁身推覈。」士鷿曰:「此並唐公之客也,若爾,便大紛紜。」威等由是疑而不發。留守司兵田德平又欲勸威等鞫問募人之狀,士鷿謂德平曰:「討捕之兵,總隸唐公。王威、高君雅等,並寄坐耳,彼何能為!」德平遂止。義旗起,以士鷿為大將軍府鎧曹。從平京城功,拜光祿大夫,封太原郡公。初為義師將起,士
 鷿不預知,及平京師,乃自說云:「嘗夢高祖入西京,升為天子。」高祖哂之曰:「汝王威之黨也。以汝能諫止弘基等,微心可錄,故加酬效;今見事成,乃說迂誕而取媚也?」武德中,累遷工部尚書,進封應國公,又歷利州、荊州都督。貞觀九年卒官,贈禮部尚書,謚曰定。顯慶元年,以後父累贈司徒,改封周國公。咸亨中,又贈太尉、太原王,特詔配饗高祖廟庭,列在功臣之上。孫承嗣,事在《外戚傳》。



 鷿長兄士棱,性恭順,勤於稼穡。從起義,官至司農少卿,
 封宣城縣公。常居苑中,委以農囿之事。貞觀中卒,贈潭州都督。



 次兄士逸,亦有戰功,武德初,為齊王府戶曹,賜爵安陸縣公。從齊王鎮並州,為劉武周所獲,於賊中密令人詣京師,陳武周可圖之計。及武周平,甚見慰勉,累授益州行臺左丞。數陳時政得失,高祖每嘉納之。貞觀初,為韶州刺史,卒。



 史臣曰:唐儉委質義旗之下,立功草昧之初,被拘虜庭,脫高祖蒲州之急;侍獵苑囿,諫太宗馬上之言,可謂純
 臣矣。順德佐命立功,理郡著明肅之政;弘基臨難不屈,陷陣多克捷之勛。殷嶠、劉政會、柴嗣昌並在太原,首預舉義,從微至著,善始令終。馬三寶出廝養之徒,處將軍之位,亦馬之善走者也。武士鷿首參起義,例封功臣,無戡難之勞,有因人之跡,載窺他傳,過為褒詞。慮當武后之朝,佞出敬宗之筆,凡涉虛美,削而不書。



 贊曰:茂約忠純,順德功勛。弘基六士,義合風雲。



\end{pinyinscope}