\article{卷六十 列傳第十 宗室(太祖諸子 代祖諸子) 永安王孝基 淮安王神通(子道彥 孝察 孝同 孝慈 孝友 孝節 孝義 孝逸) 襄邑王神符(子德懋 文暕) 長平王叔良(子孝協 孝斌 孝斌子思訓 思誨 叔良弟德良 幼良) 襄武王琛 河間王孝恭(子晦 孝恭弟 瑊 瑰) 廬江王瑗(王君廓附) 淮陽王道玄 江夏王道宗 隴西王博乂}

\begin{pinyinscope}

 ○永安王孝基淮安王神通子道彥孝察孝同孝慈孝友孝節孝義孝逸襄邑王神符子德懋文暕
 長平王叔良子孝協孝斌孝斌子思訓思誨叔良弟德良幼良襄武王琛河間王孝恭子晦孝恭弟瑊瑰廬江王瑗王君廓附淮陽王道玄江夏王道宗隴西王博乂



 永安王孝基,高祖從父弟也。父璋,周梁州刺史,與趙王祐謀殺隋文帝,事洩被誅,高祖即位,追封畢王。孝基,武德元年封永安王,歷陜州總管、鴻臚卿,以罪免。二年,劉武周將宋金剛來寇汾、澮。夏縣人呂崇茂殺縣令,舉兵
 反,自稱魏王,請援於武周。復以孝基為行軍總管討之,工部尚書獨孤懷恩、內史侍郎唐儉、陜州總管於筠悉隸焉。武周遣其將尉遲敬德潛援崇茂,大戰於夏縣,王師敗績,孝基與唐儉等皆沒於賊。後謀歸國,為武周所害,高祖為之發哀,廢朝三日,賜其家帛千匹。賊平,購其尸不得,招魂而葬之,贈左衛大將軍,謚曰壯。無子,以從兄韶子道立為嗣,封高平郡王。九年,降為縣公。永徽初,卒於陳州刺史。



 淮安王神通,高祖從父弟也。父亮,隋海州刺史,武德初追封鄭王。神通,隋末在京師。義師起,隋人捕之,神通潛入鄠縣山南,與京師大俠史萬寶、河東裴勣、柳崇禮等舉兵以應義師。遣使與司竹賊帥何潘仁連結。潘仁奉平陽公主而至,神通與之合勢,進下鄠縣,眾逾一萬。自稱關中道行軍總管,以史萬寶為副,裴勣為長史,柳崇禮為司馬,令狐德棻為記室。高祖聞之大悅,授光祿大夫。從平京師,拜宗正卿。武德元年,拜右翊衛大將軍,封
 永康王,尋改封淮安王,為山東道安撫大使。擊宇文化及於魏縣,化及不能抗,東走聊城。神通進兵躡之,至聊城。會化及糧盡請降,神通不受。其副使黃門侍郎崔幹勸納之,神通曰:「兵士暴露已久,賊計窮糧盡,克在旦暮,正當攻取,以示國威,散其玉帛,以為軍賞。若受降者,吾何以藉手乎?」乾曰:「今建德方至,化及未平,兩賊之間,事必危迫。不攻而下之,此勛甚大。今貪其玉帛,敗無日矣!」神通怒,囚干於軍中。既而士及自濟北饋之,化及軍稍
 振,遂拒戰。神通督兵薄而擊之,貝州刺史趙君德攀堞而上,神通心害其功,因止軍不戰,君德大詬而下,城又堅守。神通乃分兵數千人往魏州取攻具,中路復為莘人所敗。竇建德軍且至,遂引軍而退。後二日,化及為建德所虜,賊勢益張,山東城邑多歸建德。神通兵漸散,退保黎陽,依徐勣,俄為建德所陷。及建德敗,復授河北道行臺尚書左僕射。從太宗平劉黑闥,遷左武衛大將軍。貞觀元年,拜開府儀同三司,賜實封五百戶。時太宗謂
 諸功臣曰:「朕敘公等勛效,量定封邑,恐不能盡當,各自言。」神通曰:「義旗初起,臣率兵先至,今房玄齡、杜如晦等刀筆之人,功居第一,臣且不服。」上曰:「義旗初起,人皆有心。叔父雖率兵先至,未嘗身履行陣。山東未定,受委專征,建德南侵,全軍陷沒;及劉黑闥翻動,叔父望風而破。今計勛行賞,玄齡等有籌謀帷幄定社稷功,所以漢之蕭何,雖無汗馬,指縱推轂,故功居第一。叔父於國至親,誠無所愛,必不可緣私濫與勛臣同賞耳。」四年,薨。太宗
 為之廢朝,贈司空,謚曰靖。十四年,詔與河間王孝恭、贈陜州大行臺右僕射鄖節公殷開山、贈民部尚書渝襄公劉政會配饗高祖廟庭。有子十一人:長子道彥,武德五年,封膠東王;次孝察,高密王;孝同,淄川王;孝慈,廣平王;孝友,河間王;孝節,清河王;孝義,膠西王。



 初,高祖受禪,以天下未定,廣封宗室以威天下,皇從弟及侄年始孩童者數十人,皆封為郡王。太宗即位,因舉宗正屬藉問侍臣曰:「遍封宗子,於天下便乎?」尚書右僕射封德彞對
 曰:「歷觀往古,封王者,今最為多。兩漢已降,唯封帝子及親兄弟,若宗室疏遠者,非有大功如周之郇、滕,漢之賈、澤,並不得濫封,所以別親疏也。先朝敦睦九族,一切封王,爵命既隆,多給力役,蓋以天下為私,殊非至公馭物之道。」太宗曰:「朕理天下,本為百姓,非欲勞百姓以養己之親也。」於是宗室率以屬疏降爵為郡公,唯有功者數十人封王。是時道彥等並隨例降爵。道彥與季弟孝逸最知名。



 道彥幼而事親甚謹。初,義師起,神通逃難,被疾於山谷,綿歷數旬,山中食盡。道彥著故弊衣,出人間乞丐,及採野實,以供其父,身無所啖。其父分以食之,輒詐言已啖,而覆藏留之,以備闕乏。及神通應義舉,授朝請大夫。高祖受禪,封義興郡公,進封膠東王,授隴州刺史。貞觀初,轉相州都督,例降爵為公,拜岷州都督。丁父憂,廬於墓側,負土成墳,躬植松柏,容貌哀毀,親友皆不復識之。太宗聞而嘉嘆,令侍中王珪就加開喻。復授岷州都督。道
 彥遣使告喻黨項諸部,申國威靈,多有降附。李靖之擊吐谷渾也,詔道彥為赤水道行軍總管。時朝廷復厚幣遺黨項,令為鄉導,黨項首領拓拔赤辭來詣靖軍,請諸將曰:「往者隋人來擊吐谷渾,我黨項每資軍用,而隋人無信,必見侵掠。今將軍若無他心者,我當資給糧運;如或我欺,當即固險以塞軍路。」諸將與之歃血而盟,赤辭信之。道彥既至闊水,見赤辭無備,遂襲之,虜牛羊數千頭。於是諸羌怨怒,屯兵野狐硤,道彥不能進,為赤辭所
 乘,軍大敗,死者數萬人。道彥退保松州,竟坐減死徙邊。後起為涼州都督,尋卒,贈禮部尚書。



 孝逸少好學,解屬文。初封梁郡公。高宗末,歷給事中,四遷益州大都督府長史。則天臨朝,入為左衛將軍,甚見親遇。光宅元年,徐敬業據揚州作亂,以孝逸為左玉鈐衛大將軍、揚州行軍大總管,督軍以討之。孝逸引軍至淮,而敬業方南攻潤州,遣其弟敬猷屯兵淮陰;偽將韋超據都梁山,以拒孝逸。裨將馬敬臣擊斬賊之別帥尉遲昭、夏侯瓚等,超
 乃擁眾憑山以自固。或謂孝逸曰:「超眾守險,且憑山為阻,攻之則士無所施其力,騎無所騁其足,窮寇殊死,殺傷必眾。不若分兵守之,大軍直趣揚州,未數日,其勢必降也。」支度使、廣府司馬薛克構曰:「超雖據險,其卒非多,今逢小寇不擊,何以示武?若加兵以守,則有闕前機;舍之而前,則終為後患,不如擊之。克超則淮陰自懾,淮陰破,則楚州諸縣必開門而候官軍。然後進兵高郵,直趣江都,逆豎之首,可指掌而懸也。」孝逸從其言,進兵擊超
 賊,眾壓伏,官軍登山急擊之,殺數百人,日暮圍解,超銜枚夜遁。孝逸引兵擊淮陰,大破敬猷之眾。時敬業回軍屯於下阿溪以拒官軍,有流星墜其營。孝逸引兵渡溪以擊之。敬業初勝後敗,孝逸乘勝追奔數十里,敬業窘迫,與其黨攜妻子逃入海曲。孝逸進據揚州,盡捕斬敬業等,振旅而還,以功進授鎮軍大將軍,轉左豹韜衛大將軍,改封吳國公。孝逸素有名望,自是時譽益重,武承嗣等深所忌嫉,數讒毀之。垂拱二年,左遷施州刺史。其
 冬,承嗣等又使人誣告孝逸往任益州,嘗自解「逸」字云:「走繞兔者,常在月中。月既近天,合有天分。」則天以孝逸常有功,減死配徙儋州,尋卒。景雲初,贈益州大都督。孝銳孫齊物,孝同曾孫國貞,別有傳。



 襄邑王神符,神通弟也。幼孤,事兄以友悌聞。義寧初,授光祿大夫,封安吉郡公。武德元年,進封襄邑郡王。四年,累遷並州總管。突厥頡利可汗率眾來寇,神符出兵與戰於汾水東,敗之,斬首五百級,虜其馬二千匹。又戰於
 沙河之北,獲其乙利達官並可汗所乘馬及甲獻之,由是召拜太府卿。九年,遷揚州大都督,移州府及居人自丹陽渡江,州人賴焉。貞觀初,再遷宗正卿。後以疾辭職,太宗幸其第問疾,賜以縑帛,每給羊酒。又令乘小輿,引入紫微殿,以神符腳疾,乃遣三衛輿之而升。尋授開府儀同三司。永徽二年薨,年七十三,贈司空、荊州都督,陪葬獻陵,謚曰恭。有子七人,武德初,並封郡王,後例降封縣公。次子德懋、少子文暕最知名。德懋官至少府監、臨川
 郡公。文暕歷幽州都督、魏郡公。垂拱中,坐事貶為藤州別駕,尋被誅。文暕子佺,開元中為宗正卿。



 長平王叔良,高祖從父弟也。父禕,隋上儀同三司,武德初,追封郇王。叔良,義寧中授左光祿大夫,封長平郡公。武德元年,拜刑部侍郎,進爵為王。師鎮涇州,以御薛舉。舉乃陽言食盡,引兵南去,遣高墌人偽以降。叔良遣驃騎劉感率眾赴之,至百里細川,伏兵發,官軍敗績,劉感沒於陣。叔良大懼,出金以賜士卒。嚴為守備,涇州僅全。
 四年,突厥入寇,命叔良率五軍擊之。叔良中流矢而薨,贈左翊衛大將軍、靈州總管,謚曰肅。



 子孝協嗣,武德五年,封範陽郡王。貞觀初,以屬疏例降封郇國公,累遷魏州刺史。麟德中,坐受贓賜死。



 孝協弟孝斌,官至原州都督府長史。



 孝斌子思訓,高宗時累轉江都令。屬則天革命,宗室多見構陷,思訓遂棄官潛匿。神龍初,中宗初復宗社,以思訓舊齒,驟遷宗正卿,封隴西郡公,實封二百戶。歷益州長史。開元初,左羽林大將軍,進封彭國公,更
 加實封二百戶,尋轉右武衛大將軍。開元六年卒。贈秦州都督,陪葬橋陵。思訓尤善丹青,迄今繪事者推李將軍山水。



 思訓弟思誨,垂拱中揚州參軍。思誨子林甫別有傳。



 叔良弟德良,少有疾,不仕。武德初,封新興王。貞觀十一年薨,贈涼州都督。



 德良孫晉,先天中,為殿中監,兼雍州長史,甚有威名,紹封新興王。尋坐附會太平公主伏誅,改姓厲氏。初,晉之就誅,僚吏皆奔散,唯司功李捴步從,不失在官之禮,仍哭其尸。姚崇聞之曰:「欒、向之儔
 也。」擢為尚書郎。後官至澤州刺史。



 德良弟幼良,武德初,封長樂王。時有人盜其馬者,幼良獲盜而閃殺之,高祖怒曰:「昔人賜盜馬者酒,終獲其報,爾輒行戮,何無古風!盜者信有罪矣,專殺豈非枉邪?」遣禮部尚書李綱於朝堂集宗室王公而撻之。自後累遷涼州都督,嘗引不逞百餘人為左右,多侵暴市里,行旅苦之。太宗即位,有告幼良陰養死士,交通境外,恐謀為反叛,詔遣中書令宇文士及代為都督,並按其事。士及慮其為變,遂縊殺之。



 襄武王琛,高祖從父兄子也。祖蔚,周朔州總管。父安,隋領軍大將軍。武德初,追封蔚為蔡王,安為西平王。琛,義寧中封襄武郡公,與太常卿鄭元璹齎女妓遺突厥始畢可汗,以結和親。始畢甚重之,贈名馬數百匹,遣骨咄祿特勒隨琛貢方物。高祖大悅,拜刑部侍郎,進爵為王。歷蒲、絳二州總管。及宋金剛陷澮州,時稽胡多叛,轉琛為隰州總管以鎮之。馭眾寬簡,夷夏安之。三年,薨。子儉嗣,後隨例降爵為公。



 河間王孝恭,琛之弟也。高祖克京師,拜左光祿大夫,尋為山南道招慰大使。自金州出於巴蜀,招攜以禮,降附者三十餘州。孝恭進擊硃粲,破之,諸將曰:「此食人賊也,為害實深,請坑之。」孝恭曰:「不可!自此已東,皆為寇境,若聞此事,豈有來降者乎?」盡赦而不殺,由是書檄所至,相繼降款。武德二年,授信州總管,承制拜假。蕭銑據江陵,孝恭獻平銑之策,高祖嘉納之。三年,進爵為王。改信州為夔州,使拜孝恭為總管,令大造舟楫,教習水戰,以圖
 蕭銑。孝恭召巴蜀首領子弟,量才授用,致之左右,外示引擢,而實以為質也。尋授荊湘道行軍總管,統水陸十二總管,發自硤州,進軍江陵。攻其水城,克之,所得船散於江中。諸將皆曰:「虜得賊船,當藉其用,何為棄之,無乃資賊耶?」孝恭曰:「不然,蕭銑偽境,南極嶺外,東至洞庭。若攻城未拔,援兵復到,我則內外受敵,進退不可,雖有舟楫,何所用之?今銑緣江州鎮忽見船舸亂下,必知銑敗,未敢進兵,來去覘伺,動淹旬月,用緩其救,克之必矣。」
 銑救兵至巴陵,見船被江而下,果狐疑不敢輕進。既內外阻絕,銑於是出降。高祖大悅,拜孝恭荊州大總管,使畫工貌而視之。於是開置屯田,創立銅冶,百姓利焉。六年,遷襄州道行臺尚書左僕射。時荊襄雖定,嶺表尚未悉平。孝恭分遣使人撫慰,嶺南四十九州皆來款附。及輔公祏據江東反,發兵寇壽陽,命孝恭為行軍元帥以擊之。七年,孝恭自荊州趣九江,時李靖、李勣、黃君漢、張鎮州、盧祖尚並受孝恭節度。將發,與諸將宴集,命取水,
 忽變為血,在座者皆失色。孝恭舉止自若,徐諭之曰:「禍福無門,唯人所召。自顧無負於物,諸公何見憂之深!公祏惡積禍盈,今承廟算以致討,碗中之血,乃公祏授首之後徵。」遂盡飲而罷。時人服其識度而能安眾。公祏遣其偽將馮惠亮、陳當時領水軍屯於博望山,陳正通、徐紹宗率步騎軍於青林山。孝恭至,堅壁不與鬥,使奇兵斷其糧道。賊漸食委,夜薄我營,孝恭安臥不動。明日,縱羸兵以攻賊壘,使盧祖尚率精騎列陣以待之。俄而攻壘
 者敗走,賊出追奔數里,遇祖尚軍,與戰,大敗之。正通棄營而走,復與馮惠亮保梁山。孝恭乘勝攻之,破其梁山別鎮,赴水死者數千人,正通率陸軍夜遁。總管李靖又下廣陵城,拔楊子鎮。公祏窮蹙,棄丹陽東走。孝恭命騎將追之,至武康,擒公祏及其偽僕射西門君儀等數十人,致於麾下,江南悉平。璽書褒賞,賜甲第一區、女樂二部、奴婢七百人、金寶珍玩甚眾,授東南道行臺尚書左僕射。後廢行臺,拜揚州大都督。孝恭既破公祏,江淮及
 嶺南皆統攝之。自大業末,群雄競起,皆為太宗所平,謀臣猛將並在麾下,罕有別立勛庸者,唯孝恭著方面之功,聲名甚盛。厚自崇重,欲以威名鎮遠,築宅於石頭,陳廬徼以自衛。尋徵拜宗正卿。九年,賜實封一千二百戶。貞觀初,遷禮部尚書,以功臣封河間郡王,除觀州刺史,與長孫無忌等代襲刺史。孝恭性奢豪,重游宴,歌姬舞女百有餘人,然而寬恕退讓,無驕矜自伐之色。太宗甚加親待,諸宗室中莫與為比。孝恭嘗悵然謂所親曰:「吾
 所居宅微為宏壯,非吾心也,當賣之,別營一所,粗令充事而已。身歿之後。諸子若才,守此足矣;如其不才,冀免他人所利也。」十四年,暴薨,年五十。太宗素服舉哀,哭之甚慟,贈司空、揚州都督,陪葬獻陵,謚曰元,配享高祖廟庭。



 子崇義嗣,降爵為譙國公,歷蒲、同二州刺史,益州大都督長史,甚有威名。後卒於宗正卿。



 孝恭次子晦,乾封中,累除營州都督,以善政聞;璽書勞問,賜物三百段。轉右金吾將軍,兼檢校雍州長史,糾發奸豪,無所容貸,為
 人吏畏服。晦私第有樓,下臨酒肆,其人嘗候晦言曰:「微賤之人,雖則禮所不及,然家有長幼,不欲外人窺之。家迫明公之樓,出入非便,請從此辭。」晦即日毀其樓。高宗將幸洛陽,令在京居守,顧謂之曰:「關中之事,一以付卿。但令式跼人,不可以成官政,令式之外,有利於人者,隨事即行,不須聞奏。」晦累有異績。則天臨朝,遷戶部尚書。垂拱初,拜右金吾衛大將軍,轉秋官尚書。永昌元年卒,贈幽州都督。子榮,為酷吏所殺。



 孝恭弟瑊,武德中,為尚
 書右丞,封濟北郡王,卒於始州刺史。



 瑊弟瑰,義師克京城,授瑰左光祿大夫。武德元年,封漢陽郡公。五年,進爵為王。時突厥屢為侵寇,高祖使瑰齎布帛數萬段與結和親。頡利可汗初見瑰,箕踞;瑰餌以厚利,頡利大悅,改容加敬,遣使隨瑰獻名馬。後復將命,頡利謂左右曰:「李瑰前來,恨不屈之,今者必令下拜。」瑰微知之,及見頡利,長揖不屈節。頡利大怒,乃留瑰不遣。瑰神意自若,竟不為之屈。頡利知不可以威脅,終禮遣之。拜左武候將軍,
 轉衛尉卿,代兄孝恭為荊州都督。政存清靜,深為士庶所懷。嶺外豪帥屢相攻擊,遣使喻以威德,皆相次歸附,嶺表遂定。太宗即位,例降爵為公。時長史馮長命曾為御史大夫,素矜衒,事多專決,瑰怒杖之,坐是免。貞觀四年,拜宜川刺史,加散騎常侍,卒。



 子沖玄,垂拱中官至冬官尚書;沖虛,卒於尚方監。



 廬江王瑗,高祖從父兄子也。父哲,隋柱國、備身將軍,追封濟南王。瑗,武德元年歷信州總管,封廬江王。九年,累
 遷幽州大都督。朝廷以瑗懦曌,非邊將才,遣右領軍將軍王君廓助典兵事。君廓故嘗為盜,勇力絕人,瑗倚仗之,許結婚姻,以布心腹。時隱太子建成將有異圖,外結於瑗。及建成誅死,遣通事舍人崔敦禮召瑗入朝,瑗有懼色。君廓素險薄,欲因事陷之以為己功,遂紿瑗曰:「京都有變,事未可知。大王國之懿親,受委作鎮,寧得擁兵數萬而從一使召耶!且聞趙郡王先以被拘,太子、齊王又言若此,大王今去,能自保乎?」相與共泣。瑗乃囚敦禮,
 舉兵反。召北齊州刺史王詵,將與計事,兵曹參軍王利涉說瑗曰:「王不奉詔而擅發兵,此為反矣。須改易法度,以權宜應變,先定眾心。今諸州刺史或有逆命,王徵兵不集,何以保全?」瑗曰:「若之何?」利涉曰:「山東之地,先從竇建德,酋豪首領,皆是偽官,今並黜之,退居匹庶,此人思亂,若旱苗之望雨。王宜發使復其舊職,各於所在遣募本兵,諸州倘有不從,即委隨便誅戮。此計若行,河北之地可呼吸而定也。然後分遣王詵北連突厥,道自太原,
 南臨蒲、絳;大王整駕親詣洛陽,西入潼關。兩軍合勢,不盈旬月,天下定矣。」瑗從之。瑗以內外機悉付君廓。利涉以君廓多翻覆,又說瑗委兵於王詵而除君廓,瑗不能決。君廓知之,馳斬詵,持首告其眾曰:「李瑗與王詵共反,禁錮敕使,擅追兵集。今王詵已斬,獨李瑗在,無能為也。汝若從之,終亦族滅;從我取之,立得富貴。禍福如是,意欲何從?」眾曰:「皆願討賊。」君廓領其麾下登城西面,瑗未之覺。君廓自領千餘人先往獄中出敦禮,瑗始知之,遽
 率數百人披甲,才出至門外,與君廓相遇。君廓謂其眾曰:「李瑗作逆誤人,何忽從之,自取塗炭?」眾皆倒戈,一時潰走。瑗塊然獨存,謂君廓曰:「小人賣我以自媚,汝行當自及矣。」君廓擒瑗,縊殺之,年四十一,傳首京師,絕其屬籍。



 君廓,並州石艾人也。少亡命為群盜,聚徒千餘人,轉掠長平,進逼夏縣。李密遣使召之,遂投於密。尋又率眾歸國,歷遷右武衛將軍,累封彭國公。從平劉黑闥,令鎮幽州。會突厥入寇,君廓邀擊破之,俘斬二千餘人,獲馬
 五千匹。高祖大悅,徵入朝,賜以御馬,令於殿庭乘之而出,因謂侍臣曰:「吾聞藺相如叱秦皇,目皆出血。君廓往擊竇建德,將出戰,李靖遏之,君廓發憤大呼,目及鼻耳一時流血。此之壯氣,何謝古人,不可以常例賞之。」復賜錦袍金帶,還鎮幽州。尋以誅瑗功,拜左領軍大將軍,兼幽州都督,以瑗家口賜之,加左光祿大夫,賜物千段,食實封千三百戶。在職多縱逸,長史李玄道數以朝憲脅之,懼為所奏,殊不自安。後追入朝,行至渭南,殺驛史而
 遁。將奔突厥,為野人所殺,追削其封邑。



 淮陽王道玄,高祖從父兄子也。祖繪,隋夏州總管,武德初,追封雍王。父贄,追封河南王。道玄,武德元年封淮陽王,授右千牛。從太宗擊宋金剛於介州,先登陷陣,時年十五,太宗壯之,賞物千段。後從討王世充,頻戰皆捷。竇建德至武牢,太宗以輕騎誘賊,領道玄率伏兵於道左,會賊至,追擊破之。又從太宗轉戰於汜水,麾戈陷陣,直出賊後,眾披靡,復沖突而歸。太宗大悅,命副乘以給道
 玄。又從太宗赴賊,再入再出,飛矢亂下,箭如蝟毛,猛氣益厲,射人無不應弦而倒。東都平,拜洛州總管。及府廢,改授洛州刺史。五年,劉黑闥引突厥寇河北,復授山東道行軍總管。師次下博,與賊軍遇,道玄帥騎先登,命副將史萬寶督軍繼進。萬寶與之不協,及道玄深入,而擁兵不進,謂所親曰:「吾奉手詔,言淮陽小兒雖名為將,而軍之進止皆委於吾。今其輕脫,越濘交戰,大軍若動,必陷泥溺,莫如結陣以待之,雖不利於王,而利於國。」道玄
 遂為賊所擒,全軍盡沒,惟萬寶逃歸。道玄遇害,年十九。太宗追悼久之,嘗從容謂侍臣曰:「道玄終始從朕,見朕深入賊陣,所向必克,意嘗企慕,所以每陣先登,蓋學朕也。惜其年少,不遂遠圖。」因為之流涕。贈左驍衛大將軍,謚曰壯。無子,詔封其弟武都郡公道明為淮陽王,令主道玄之祀。累遷左驍衛將軍。送弘化公主還蕃,坐洩主非太宗女,奪爵國除,後卒於鄆州刺史。



 江夏王道宗,道玄從父弟也。父韶,追封東平王,贈戶部
 尚書。道宗,武德元年封略陽郡公,起家左千牛備身。討劉武周,戰於度索原,軍敗,賊徒進逼河東。道宗時年十七,從太宗率眾拒之。太宗登玉壁城望賊,顧謂道宗曰:「賊恃其眾來邀我戰,汝謂如何?」對曰:「群賊乘勝,其鋒不可當,易以計屈,難與力競。今深壁高壘,以挫其鋒;烏合之徒,莫能持久,糧運致竭,自當離散,可不戰而擒。」太宗曰:「汝意暗與我合。」後賊果食盡夜遁,追及介州,一戰滅之。又從平竇建德,破王世充,屢有殊效。五年,授靈州總管。
 梁師都據夏州,遣弟洛仁引突厥兵數萬至於城下。道宗閉門拒守,伺隙而戰,賊徒大敗。高祖聞而嘉之,謂左僕射裴寂、中書令蕭瑀曰:「道宗今能守邊,以寡制眾。昔魏任城王彰臨戎卻敵,道宗勇敢,有同於彼。」遂封為任城王。初,突厥連於梁師都,其鬱射設入居五原舊地,道宗逐出之。振耀威武,開拓疆界,斥地千餘里,邊人悅服。



 貞觀元年,徵拜鴻臚卿,歷左領軍、大理卿。時太宗將經略突厥,又拜靈州都督。三年,為大同道行軍總管。遇
 李靖襲破頡利可汗,頡利以十餘騎來奔其部。道宗引兵逼之,徵其執送頡利。頡利以數騎夜走,匿於荒谷,沙鈐羅懼,馳追獲之,遣使送於京師。以功賜實封六百戶,召拜刑部尚書。吐谷渾寇邊,詔右僕射李靖為昆丘道行軍大總管,道宗與吏部尚書侯君集為之副。賊聞兵至,走入嶂山,已行數千里。諸將議欲息兵,道宗固請追討,李靖然之,而君集不從。道宗遂率偏師並行倍道,去大軍十日,追及之。賊據險苦戰,道宗潛遣千餘騎逾山
 襲其後,賊表裏受敵,一時奔潰。十二年,遷禮部尚書,改封江夏王。尋坐贓下獄。太宗謂侍臣曰:「朕富有四海,士馬如林,欲使轍跡周宇內,游觀無休息,絕域採奇玩,海外訪珍羞,豈不得耶?勞萬姓而樂一人,朕所不取也。人心無厭,唯當以理制之。道宗俸料甚高,宴賜不少,足有餘財,而貪婪如此,使人嗟惋,豈不鄙乎!」遂免官,削封邑。十三年,起為茂州都督,未行,轉晉州刺史。十四年,復拜禮部尚書。時侯君集立功於高昌,自負其才,潛有異志。
 道宗嘗因侍宴,從容言曰:「君集智小言大,舉止不倫,以臣觀之,必為戎首。」太宗曰:「何以知之?」對曰:「見其恃有微功,深懷矜伐,恥在房玄齡、李靖之下。雖為吏部尚書,未滿其志,非毀時賢,常有不平之語。」太宗曰:「不可億度,浪生猜貳。其功勛才用,無所不堪,朕豈惜重位?第未到耳。」俄而君集謀反誅,太宗笑謂道宗曰:「君集之事,果如公所揣。」及大軍討高麗,令道宗與李靖為前鋒,濟遼水,克蓋牟城。逢賊兵大至,軍中僉欲深溝保險,待太宗至徐
 進,道宗曰:「不可。賊赴急遠來,兵實疲頓,恃眾輕我,一戰必摧。昔耿弇不以賊遺君父,我既職在前軍,當須清道以待輿駕。」李靖然之。乃與壯士數十騎直沖賊陣,左右出入,靖因合擊,大破之。太宗至,深加賞勞,賜奴婢四十人。又築土山攻安市城,土山崩,道宗失於部署,為賊所據。歸罪於果毅傅伏愛,斬之。道宗跣行詣旗下請罪,太宗曰:「漢武殺王恢,不如秦穆赦孟明,土山之失,且非其罪。」舍而不問。道宗在陣損足,太宗親為其針,賜以御膳。
 二十一年,以疾請居閑職,轉太常卿。永徽元年,加授特進,增實封並前六百戶。四年,房遺愛伏誅,長孫無忌、褚遂良素與道宗不協,上言道宗與遺愛交結,配流象州。道病卒,年五十四。及無忌、遂良得罪,詔復其官爵。道宗晚年頗好學,敬慕賢士,不以地勢凌人,宗室中唯道宗及河間王孝恭昆季最為當代所重。



 道宗子景恆,降封盧國公,官至相州刺史。



 隴西王博乂,高祖兄子也。高祖長兄曰澄,次曰湛,次曰
 洪,並早卒。武德初,追封澄為梁王,湛為蜀王,洪為鄭王。澄、洪並無後,博乂即湛第二子也。武德元年受封。高祖時,歷宗正卿、禮部尚書,加特進。博乂有妓妾數百人,皆衣羅綺,食必粱肉,朝夕糸玄歌自娛,驕侈無比。與其弟渤海王奉慈俱為高祖所鄙,帝謂曰:「我怨仇有善,猶擢以不次,況於親戚而不委任?聞汝等唯暱近小人,好為不軌,先王墳典,不聞習學。今賜絹二百匹,可各買經史習讀,務為善事。」咸亨二年薨,贈開府儀同三司、荊州都督,
 謚曰恭。奉慈,武德初,封渤海王。顯慶中,累遷原州都督,薨,謚曰敬。



 史臣曰:無私於物,物亦公焉。高祖才定中原,先封疏屬,致廬江為叛,神通爭功,封德彞論之於前,房玄齡譏之於後。若河間機謀深沉,識度弘遠,縱虛舟而降蕭銑,飲妖血而平公祏,入朝定君臣之分,賣第為子孫之謀,善始令終,論功行賞,即無私矣。或問曰:「水變為血,信妖矣;竟成功而無咎者,何也?」答曰:河間節貫神明,志匡宗社,
 故妖不勝德明矣。道宗軍謀武勇,好學下賢,於群從之中,稱一時之傑。無忌、遂良銜不協之素,致千載之冤。永徽中,無忌、遂良忠而獲罪,人皆哀之。殊不知誣陷劉洎、吳王恪於前,枉害道宗於後,天網不漏,不得其死也宜哉!



 贊曰:疏屬盡封,啟亂害公。河間孝恭,獨稱軍功。



\end{pinyinscope}