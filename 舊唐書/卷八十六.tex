\article{卷八十六}

\begin{pinyinscope}

 ○許敬宗李義府
 少子湛



 許敬宗,杭州新城人,隋禮部侍郎善心子也。其先自高陽南渡,世仕江左。敬宗幼善屬文,舉秀才,授淮陽郡司法書佐,俄直謁者臺,奏通事舍人事。江都之難,善心為
 宇文化及所害。敬宗流轉,投於李密,密以為元帥府記室,與魏徵同為管記。武德初,赤牒擬漣州別駕。太宗聞其名,召補秦府學士。貞觀八年,累除著作郎,兼修國史,遷中書舍人。十年,文德皇后崩,百官縗絰。率更令歐陽詢狀貌醜異,眾或指之,敬宗見而大笑,為御史所劾,左授洪州都督府司馬。累遷給事中,兼修國史。十七年,以修《武德》、《貞觀實錄》成,封高陽縣男,賜物八百段,權檢校黃門侍郎。高宗在春宮,遷太子右庶子。十九年,太宗親
 伐高麗,皇太子定州監國,敬宗與高士廉等共知機要。中書令岑文本卒於行所,令敬宗以本官檢校中書侍郎。太宗大破遼賊於駐蹕山,敬宗立於馬前受旨草詔書,詞彩甚麗,深見嗟賞。先是,庶人承乾廢黜,宮僚多被除削,久未收敘。敬宗上表曰:「臣聞先王慎罰,務在於恤刑,往哲寬仁,義在於宥過。聖人之道,莫尚於茲。竊見廢官,五品以上,除名棄斥,頗歷歲時。但庶人疇昔之年,身處不疑之地,苞藏悖逆,陰結宰臣,所預奸謀,多連宗戚。
 禍生慮表,非可防萌,宮內官僚,迥無關預。今乃投鼠及器,孰謂無冤?焚山毀玉,稍同遷怒。伏尋先典,例有可原。昔吳國陪臣,則爰絲不坐於劉濞;昌邑中尉,則王吉免緣於海昏。譬諸欒布,乃策名於彭越;比乎田叔,亦委質於張敖。主以兇逆,陷其誅夷;臣以賢良,荷彼收擢。歷觀往代,此類尤多。近者有隋,又遵斯義。楊勇之廢,罪止加於佞人,李綱之徒,皆不預於刑網。古今裁其折衷,史籍稱為美談。而今張玄素、令狐德棻、趙弘智、裴宣機、蕭鈞等,並
 砥節勵操,有雅望於當朝;經明行修,播令名於天下。或以直言而遭箠撲,或以忤意而見猜嫌,一概雷同,並罹天憲,恐於王道,傷在未弘。」由是玄素等稍得敘用。二十一年,加銀青光祿大夫。



 高宗嗣位,代於志寧為禮部尚書。敬宗嫁女與蠻酋馮盎之子,多納金寶,為有司所劾,左授鄭州刺史。永徽三年,入為衛尉卿,加弘文館學士,兼修國史。六年,復拜禮部尚書,高宗將廢皇后王氏而立武昭儀,敬宗特贊成其計。長孫無忌、褚遂良、
 韓瑗等並直言忤旨,敬宗與李義府潛加誣構,並流死於嶺外。顯慶元年,加太子賓客,尋冊拜侍中,監修國史。三年,進封郡公,尋贈其父善心為冀州刺史。高宗因於古長安城游覽,問侍臣曰:「朕觀故城舊基,宮室似與百姓雜居,自秦、漢已來,幾代都此?」敬宗對曰:「秦都咸陽,郭邑連跨渭水,故云『渭水貫都,以象天河。』至漢惠帝始築此城,其後苻堅、姚萇、後周並都之。」帝又問:「昆明池是漢武帝何年中開鑿?」敬宗對曰:「武帝遣使通西南夷,而為昆
 明滇池所閉,欲伐昆明國,故因鎬之舊澤,以穿此池,用習水戰,元狩三年事也。」帝因令敬宗與弘文館學士具檢秦、漢已來歷代宮室處所以奏。其年,代李義府為中書令,任遇之重,當朝莫比。龍朔二年,從新令改為右相,加光祿大夫。三年,冊拜太子少師、同東西臺三品,並依舊監修國史。乾封初,以敬宗年老,不能行步,特令與司空李勣,每朝日各乘小馬入禁門至內省。



 敬宗自掌知國史,記事阿曲。初,虞世基與敬宗父善心同為宇文化
 及所害,封德彞時為內史舍人,備見其事,因謂人曰:「世基被誅,世南匍匐而請代;善心之死,敬宗舞蹈以求生。」人以為口實,敬宗深銜之,及為德彞立傳,盛加其罪惡。敬宗嫁女與左監門大將軍錢九隴,本皇家隸人,敬宗貪財與婚,乃為九隴曲敘門閥,妄加功績,並升與劉文靜、長孫順德同卷。敬宗為子娶尉遲寶琳孫女為妻,多得賂遺,及作寶琳父敬德傳,悉為隱諸過咎。太宗作《威鳳賦》以賜長孫無忌,敬宗改雲賜敬德。白州人龐孝泰,
 蠻酋凡品,率兵從征高麗,賊知其懦,襲破之。敬宗又納其寶貨,稱孝泰頻破賊徒,斬獲數萬。漢將驍健者,唯蘇定方與龐孝泰耳,曹繼叔、劉伯英皆出其下。虛美隱惡如此!初,高祖、太宗兩朝實錄,其敬播所修者,頗多詳直,敬宗又輒以己愛憎曲事刪改,論者尤之。然自貞觀已來,朝廷所修《五代史》及《晉書》、《東殿新書》、《西域圖志》、《文思博要》、《文館詞林》、《累璧》、《瑤山玉彩》《姓氏錄》、《新禮》,皆總知其事,前後賞賚,不可勝紀。敬宗好色無度。其長子昂頗有
 才藻,歷位太子舍人。母裴氏早卒。裴侍婢有姿色,敬宗嬖之,以為繼室,假姓虞氏。昂素與通,烝之不絕,敬宗怒黜虞氏,加昂以不孝,奏請流於嶺外。顯慶中,表乞昂還,除虔化令,尋卒。咸亨元年,抗表乞骸骨,詔聽致仕,仍加特進,俸祿如舊。三年薨,年八十一。高宗為之舉哀,廢朝三日,詔文武百官就第赴哭,冊贈開府儀同三司、揚州大都督,陪葬昭陵。文集八十卷。太常將定謚,博士袁思古議曰:「敬宗位以才升,歷居清級,然棄長子於荒徼,嫁
 少女於夷落。聞《詩》學《禮》,事絕於趨庭;納採問名,唯聞於黷貨。白圭斯玷,有累清塵,易名之典,須憑實行。按謚法『名與實爽曰繆』,請謚為『繆』。」敬宗孫、太子舍人彥伯不勝其恥,與思古大相忿競,又稱思古與許氏先有嫌隙,請改謚官。太常博士王福畤議曰:「謚者,飾終之稱也,得失一朝,榮辱千載。若使嫌隙是實,即合據法推繩;如其不虧直道,義不可奪,官不可侵。二三其德,何以言禮?福畤忝當官守,匪躬之故。若順風阿意,背直從曲,更是甲令
 虛設,將謂禮院無人,何以激揚雅道,顧視同列!請依思古謚議為定。」戶部尚書戴至德謂福畤曰:「高陽公任遇如此,何以定謚為『繆』?」答曰:「昔晉司空何曾薨,太常博士秦秀謚為繆醜公。何曾既忠且孝,徒以日食萬錢,所以貶為繆醜。況敬宗忠孝不逮於曾,飲食男女之累,有逾於何氏,而謚之為『繆』,無負於許氏矣。」時有詔令尚書省五品已上重議,禮部尚書袁思敬議稱:「按謚法『既過能改曰恭』,請謚曰『恭』。」詔從其議。彥伯,昂之子,起家著作郎。
 敬宗末年文筆,多令彥伯代作。又納婢妾讒言,奏流於嶺表,後遇赦得還,除太子舍人。早卒,有集十卷。



 李義府,瀛州饒陽人也。其祖為梓州射洪縣丞,因家於永泰。貞觀八年,劍南道巡察大使李大亮以義府善屬文,表薦之。對策擢第,補門下省典儀。黃門侍郎劉洎、侍書御史馬周皆稱薦之,尋除監察御史。又敕義府以本官兼侍晉王。及升春宮,除太子舍人,加崇賢館直學士,與太子司議郎來濟俱以文翰見知,時稱來、李。義府嘗
 獻《承華箴》,其辭曰:



 邃初冥昧,元氣氤氳。二儀始闡,三才既分。司乾立宰,出《震》為君。化昭淳樸,道映典墳。功成揖讓,事極華、勛。肇興夏啟,降及姬文。咸資繼德,永樹高芬。百代沿襲,千齡奉聖。粵若我後,丕承寶命。允穆三階,爰齊七政。時雍化洽,風移俗盛。載崇國本,式延家慶。《震》維標德,《離》警體正。寄切宗祧,事隆監撫。思皇茂則,敬詢端輔。業光啟、誦,藝優干羽。九載崇儒,三朝問豎,歷選儲儀,遺文在斯。望試登俎,高諭喬枝。俯容思順,非禮無施。前
 修盛業,來哲通規。飭躬是蹈,則叡問風馳;立志或爽,則玄猷日虧。無恃尊極,修途難測;無恃親賢,失德靡全。勿輕小善,積小而名自聞;勿輕微行,累微而身自正。佞諛有類,邪巧多方。其萌不絕,其害必彰。監言斯屏,儲業攸昌。竊惟令嗣,有殊前事。雖以貴以賢,而非長非次。皇明眷德,超倫作貳。匪懋聲華,莫酬恩異。匪崇徽烈,莫符天志。勉之又勉,光茲守器。下臣司箴,敢告近侍。



 太子表上其文,優詔賜帛四十匹,又令預撰《晉書》。高宗嗣位,遷中
 書舍人。永徽二年,兼修國史,加弘文館學士。高宗將立武昭儀為皇后,義府嘗密申協贊,尋擢拜中書侍郎、同中書門下三品,監修國史,賜爵廣平縣男。



 義府貌狀溫恭,與人語必嬉怡微笑,而褊忌陰賊。既處權要,欲人附己,微忤意者,輒加傾陷。故時人言義府笑中有刀,又以其柔而害物,亦謂之「李貓。」顯慶元年,以本官兼太子右庶子,進爵為侯。有洛州婦人淳于氏,坐奸系於大理,義府聞其姿色,囑大理丞畢正義求為別宅婦,特為雪其
 罪。卿段寶玄疑其故,遽以狀聞,詔令按其事,正義惶懼自縊而死。侍御史王義方廷奏義府犯狀,因言其初容貌為劉洎、馬周所幸,由此得進,言詞猥褻。帝怒,出義方為萊州司戶,而不問義府奸濫之罪。義府云:「王御史妄相彈奏,得無愧乎?」義方對云:「仲尼為魯司寇七日,誅少正卯於兩觀之下;義方任御史旬有六日,不能去奸邪於雙闕之前,實以為愧。」尋兼太子左庶子。二年,代崔敦禮為中書令,兼檢校御史大夫,監修國史、學士並如故。
 尋加太子賓客,進封河間郡公。三年,又追贈其父德晟為魏州刺史,諸子孩抱者並列清官,詔為造甲第,榮寵莫之能比。而義府貪冒無厭,與母、妻及諸子、女婿賣官鬻獄,其門如市。多引腹心,廣樹朋黨,傾動朝野。初,杜正倫為中書侍郎,義府時任典儀,至是乃與正倫同為中書令。正倫每以先進自處,不下義府,而中書侍郎李友益密與正倫共圖議義府,更相伺察。義府知而密令人封奏其事。正倫與義府訟於上前,各有曲直。上以大臣
 不和,兩責之,左貶義府為普州刺史,正倫為橫州刺史,友益配流峰州。四年,復召義府兼吏部尚書、同中書門下三品,自餘官封如故。龍朔元年,丁母憂去職。二年,起復為司列太常伯、同東西臺三品。義府尋請改葬其祖父,營墓於永康陵側。三原令李孝節私課丁夫車牛,為其載土築墳,晝夜不息。於是高陵、櫟陽、富平、雲陽、華原、同官、涇陽等七縣,以孝節之故,懼不得已,悉課丁車赴役。高陵令張敬業恭勤怯懦,不堪其勞,死於作所。王公
 已下,爭致贈遺,其羽儀、導從、轜輶、器服,並窮極奢侈。又會葬車馬祖奠供帳,自灞橋屬於三原,七十里間,相繼不絕。武德已來,王公葬送之盛,未始有也。義府本無藻鑒才,怙武后之勢,專以賣官為事。銓序失次,人多怨讟。時殷王初出閣,又以義府兼王府長史。三年,遷右相,殷王府長史仍知選事並如故。義府入則諂言自媚,出則肆其奸宄,百僚畏之,無敢言其過者。帝頗知其罪失,從容誡義府云:「聞卿兒子、女婿皆不謹慎,多作罪過,我亦
 為卿掩覆,未即公言,卿可誡勖,勿令如此。」義府勃然變色,腮頸俱起,徐曰:「誰向陛下道此?」上曰:「但我言如是,何須問我所從得耶!」義府睆然,殊不引咎,緩步而去。上亦優容之。初,五禮儀注,自前代相沿,吉兇畢舉。太常博士蕭楚材、孔志約以皇室兇禮為預備兇事,非臣子所宜言之。義府深然之。於是悉刪而焚焉。義府既貴之後,又自言本出趙郡,始與諸李敘昭穆,而無賴之徒茍合,藉其權勢,拜伏為兄叔者甚眾。給事中李崇德初亦與同
 譜敘昭穆,及義府出為普州刺史,遂即除削。義府聞而銜之,及重為宰相,乃令人誣構其罪,竟下獄自殺。初,貞觀中,太宗命吏部尚書高士廉、御史大夫韋挺、中書侍郎岑文本、禮部侍郎令狐德棻等及四方士大夫諳練門閥者修《氏族志》,勒成百卷,升降去取,時稱允當,頒下諸州,藏為永式。義府恥其家代無名,乃奏改此書,專委禮部郎中孔志約、著作郎楊仁卿、太子洗馬史玄道、太常丞呂才重修。志約等遂立格云:「皇朝得五品官者,皆
 升士流。」於是兵卒以軍功致五品者,盡入書限,更名為《姓氏錄》。由是搢紳士大夫多恥被甄敘,皆號此書為「勛格」。義府仍奏收天下《氏族志》本焚之。關東魏、齊舊姓,雖皆淪替,猶相矜尚,自為婚姻。義府為子求婚不得,乃奏隴西李等七家,不得相與為婚。



 陰陽占候人杜元紀為義府望氣,云「所居宅有獄氣,發積錢二千萬乃可厭勝。」義府信之,聚斂更急切。義府居母服,有制朔望給哭假,義府輒微服與元紀凌晨共出城東,登古塚候望,哀禮
 都廢。由是人皆言其窺覘災眚,陰懷異圖。義府又遣其子右司議郎津,召長孫無忌之孫延,謂曰:「相為得一官,數日詔書當出。」居五日,果授延司津監,乃取延錢七百貫。於是右金吾倉曹參軍楊行穎表言義府罪狀,制下司刑太常伯劉祥道與侍御詳刑對推其事,仍令司空李勣監焉。按皆有實,乃下制曰:「右相、行殷王府長史、河間郡公李義府,洩禁中之語,鬻寵授之朝恩;交占候之人,輕朔望之哀禮。蓄邪黷貨,實玷衣冠;稔惡嫉賢,載虧
 政道。特以任使多年,未忍便加重罰,宜從遐棄,以肅朝倫。可除名長流巂州。其子太子右司議郎津,專恃權門,罕懷忌憚,奸淫是務,賄賂無厭,交游非所,潛報機密,亦宜明罰,屏跡荒裔。可除名長流振州。」義府次子率府長史洽、千牛備身洋、子婿少府主簿柳元貞等,皆憑恃受贓,並除名長流延州。朝野莫不稱慶,時人為之語曰:「今日巨唐年,還誅四兇族。」四兇者,謂洽及柳元貞等四人也。或作《河間道行軍元帥劉祥道破銅山大賊李義府
 露布》,榜之通衢。義府先多取人奴婢,及敗,一時奔散,各歸其家。《露布》稱「混奴婢而亂放,各識家而競入」者,謂此也。乾封元年,大赦,長流人不許還,義府憂憤發疾卒,年五十餘。文集三十卷,傳於代;又著《宦游記》二十卷,尋亡失。自義府流放後,朝士常憂懼,恐其復來,及聞其死,於是始安。



 上元元年,大赦,義府妻子得還洛陽。如意元年,則天以義府與許敬宗、御史大夫崔義玄、中書舍人王德儉、大理正侯善業、大理丞袁公瑜等六人,在永徽中
 有翊贊之功,追贈義府揚州大都督,義玄益州大都督,德儉魏州刺史,公瑜江州刺史。長安元年,又賜義府子左千牛衛將軍湛及敬宗諸子實封各三百戶,義玄子司賓卿基、德儉子殿中監璇實封各二百五十戶,善業子太子右庶子知一、公瑜子殿中丞忠臣實封各二百戶。睿宗即位,景雲元年,並停義府等六家實封。



 義府少子湛,年六歲時,以父貴授周王文學。神龍初,累遷右散騎常侍,襲封河間郡公。時鳳閣侍郎張柬之將誅張易
 之兄弟,遂引湛為左羽林將軍,令與敬暉等啟請皇太子,備陳將誅易之兄弟意,太子許之。及兵發,湛與右羽林大將軍李多祚等詣東宮迎皇太子,拒而不時出,湛進啟曰:「逆豎反道亂常,將圖不軌,宗社危敗,實在須臾。湛等諸將與南衙執事克期誅翦,伏願殿下暫至玄武門,以副眾望。」太子曰:「兇豎悖亂,誠合誅夷,然聖躬不豫,慮有驚動。公等且止,以俟後圖。」湛曰:「諸將棄家族,共宰相同心戮力,匡輔社稷,殿下奈何不哀其懇誠而欲陷
 之鼎鑊?湛等微命,雖不足惜,殿下速出自止遏。」太子乃馳馬就路。湛從至玄武門,斬關而入,率所部兵直至則天所寢長生殿,環繞侍衛。因奏:「臣等奉令誅逆賊易之、昌宗,恐有漏洩,遂不獲預奏,輒陳兵禁掖,是臣等死罪。」則天謂湛曰:「卿亦是誅易之軍將耶?我於汝父子恩不少,何至是也!」則天移就上陽宮,因留湛宿衛。中宗即位,拜右羽林大將軍,進封趙國公,加實封通前滿五百戶。頃之,復授左散騎常侍,累轉左領軍衛大將軍。開元初
 卒。崔義玄別有傳。



 史臣曰:許高陽武德之際,已為文皇入館之賓,垂三十年,位不過列曹尹;而馬周、劉洎起羈旅徒步,六七年間,皆登宰執。考其行實,則高陽之文學宏奧,周、洎無以過之,然而太宗任遇相殊者,良以高陽才優而行薄故也。及屬嗣君沖暗,嬖妾奸邪,阿附豺狼,窺圖權軸,人之兇險,一至於斯。仲尼所謂「雖有周公之才,不足觀也。」義府才思精密,所謂「猩猩能言」,鄙哉!



 贊曰:貞觀文士,高陽、河間。圖形學館,染翰書山。進身以筆,得位由奸。為虎傅翼,即又胡顏。



\end{pinyinscope}