\article{卷三 本紀第三 太宗下}

\begin{pinyinscope}

 四年春正月乙亥,定襄道行軍總管李靖大破突厥,獲隋皇后蕭氏及煬帝之孫正道,送至京師。癸巳,武德殿北院火。二月己亥,幸溫湯。甲辰,李靖又破突厥於陰山,頡利可汗輕騎遠遁。丙午,至自溫湯。甲寅,大赦,賜酺五日。民部尚書戴胄
 以本官檢校吏部尚書,參預朝政。太常卿蕭瑀為御史大夫,與宰臣參議朝政。御史大夫、西河郡公溫彥博為中書令。三月庚辰,大同道行軍副總管張寶相生擒頡利可汗,獻於京師。甲申,尚書右僕射、蔡國公杜如晦薨。甲午,以俘頡利告於太廟。



 夏四月丁酉,御順天門,軍吏執頡利以獻捷。自是西北諸蕃咸請上尊號為「天可汗」,於是降璽書冊命其君長,則兼稱之。秋七月甲子朔,日有蝕之。上謂房玄齡、蕭瑀曰:「隋文何等主?」對曰:「克己復禮,勤勞思政,每一坐朝,或至日昃。五品已上,引之
 論事。宿衛之人,傳餐而食。雖非性體仁明,亦勵精之主也。」上曰:「公得其一,未知其二。此人性至察而心不明。夫心暗則照有不通,至察則多疑於物。自以欺孤寡得之,謂群下不可信任,事皆自決,雖勞神苦形,未能盡合於理。朝臣既知上意,亦復不敢直言,宰相已下,承受而已。朕意不然。以天下之廣,豈可獨斷一人之慮?朕方選天下之才,為天下之務,委任責成,各盡其用,庶幾於理也。」因令有司:「詔敕不便於時,即宜執奏,不得順旨施
 行。」八月丙午,詔三品已上服紫,五品已上服緋,六
 品七品以綠,八品九品以青;婦人從夫色。甲寅,兵部尚書、代國公
 李靖為尚書左僕射。九月庚午,令收瘞長城之南骸骨,仍令致祭。壬午,令自古明王聖帝、賢臣烈士
 墳墓無得芻牧,春秋致祭。



 冬十月壬辰,幸隴州,曲赦隴、岐二州,給復一年。辛丑,校獵於貴泉穀。甲辰,校獵於魚龍川,自射鹿,獻於大安宮。甲子,至
 自隴州。戊寅,制決罪人不得鞭背,以明堂孔穴針灸之所。兵部尚書侯君集參議朝政。十二月辛亥,開府儀同三司、淮安王神通薨。甲寅,高昌王麴文泰來朝。是歲,斷死刑二十九人,幾致刑措。東至於海,南至於嶺,皆外戶不閉,行旅不賚糧焉。



 五年正月癸酉,大蒐於昆明池,蕃夷君長咸從。丙
 子,親獻禽於大安宮。己卯,幸左藏庫,賜三品已上帛,任其輕重。癸未,朝集使請封禪。己酉,封皇弟元裕為鄶王,元名為譙王,靈夔為魏王,元祥為許王,元曉為密王。庚戌,封皇子愔為梁王,貞為漢王,惲為郯王,治為晉王,慎為申王,囂為江王,簡為代王。



 夏四月壬辰,代王簡薨。以金帛購中國人因
 隋亂沒突厥者男女八萬人,盡還其家屬。六月甲寅,太子少師、新昌縣公李綱薨。七月甲辰,遣使毀高麗所立京觀,收隋人骸骨,祭而葬之。戊申,初令天下決死刑必三覆奏,在京諸司五覆奏,其日尚食進蔬
 食,內教坊及太常不舉樂。九月乙丑,賜群官大射於武德殿。



 冬十月,右衛大將軍、順州都督、北平郡王阿史那什缽苾卒。十二月壬寅,幸溫湯。癸卯,獵於驪山。丙午,賜新豐高年帛有差。戊申,至自溫湯。



 六年春正月乙卯朔,日有蝕之。二月丙戌,置三師官員。戊子,初置律學。



 三月戊辰,幸九成宮。六月己亥,酆王元亨薨。辛亥,江王囂薨。



 冬十月乙卯,至自九成宮。十二月辛未,親錄囚徒,歸死罪者二百九十人於家,令明年秋
 末就刑。其後應期畢至,詔悉原之。是歲,黨項羌前後內屬者三十萬口。



 七年春正月戊子,詔曰:「宇文化及弟智及、司馬德戡、裴虔通、孟景、元禮、楊覽、唐奉義、牛方裕、元敏、薛良、馬舉、元武達、李孝本、李孝質、張愷、許弘仁、令狐行達、席德方、李覆等,大業季年,咸居列職,或恩結一代,任重一時;乃包藏兇慝,罔思忠義,爰在江都,遂行弒逆,罪百閻、趙,釁深梟獍。雖事是前代,歲月已久,而天下之惡,古今同棄,宜
 置重典,以勵臣節。其子孫並宜禁錮,勿令齒敘。」是日,上制《破陣樂舞圖》。辛丑,賜京城酺三日。丁卯,雨土。乙酉,薛延陀遣使來朝。庚寅,秘書監、檢校侍中魏徵為侍中。癸巳,直太史、將仕郎李淳風鑄渾天黃道儀,奏之,置於凝暉閣。夏五月癸未,幸九成宮。八月,山東、河南三十州大水,遣使賑恤。



 冬十月庚申,至自九成宮。十一月丁丑,頒新定《五經》。壬辰,開府儀同三司、齊國公長孫無忌為司空。十二月丙辰,狩於少陵原,詔以少牢祭杜如晦、杜淹、
 李綱之墓。



 八年正月癸未,右衛大將軍阿史那吐苾卒。辛丑,右屯衛大將軍張士貴討東、西五洞反獠,平之。壬寅,命尚書右僕射李靖、特進蕭瑀楊恭仁、禮部尚書王珪、御史大夫韋挺、鄜州大都督府長史皇甫無逸、揚州大都督府長史李襲譽、幽州大都督府長史張亮、涼州大都督李大亮、右領軍大將軍竇誕、太子左庶子杜正倫、綿州刺史劉德威、黃門侍郎趙弘智使於四方,觀省風俗。



 二月
 乙巳,皇太子加元服。丙午,賜天下酺三日。三月庚辰,幸九成宮。五月辛未朔,日有蝕之。丁丑,上初服翼善冠,貴臣服進德冠。七月,始以雲麾將軍階為從三品。隴右山崩,大蛇屢見。山東、河南、淮南大水,遣使賑恤。八月甲子,有星孛於虛、危,歷於氐,十一月上旬乃滅。九月丁丑,皇太子來朝。



 冬十月,右驍衛大將軍、褒國公段志玄擊吐谷渾,破之,追奔八百餘里。甲子,至自九成宮。十一月辛未,右僕射、代國公李靖以疾辭官,授特進。丁亥,吐谷渾
 寇涼州。己丑,吐谷渾拘我行入趙道德。十二月辛丑,命特進李靖、兵部尚書侯君集、刑部尚書任城王道宗、涼州都督李大亮等為大總管,各帥師分道以討吐谷渾。壬子,越王泰為雍州牧。乙卯,帝從太上皇閱武於城西。是歲,龜茲、吐蕃、高昌、女國、石國遣使朝貢。



 九年春三月,洮州羌叛,殺刺史孔長秀。壬午,大赦。每鄉置長一人,佐二人。乙酉,鹽澤道總管高甑生大破叛羌之眾。庚寅,敕天下戶立三等,未盡升降,置為九等。



 夏四
 月壬寅,康國獻獅子。閏月丁卯,日有蝕之。癸巳,大總管李靖、侯君集、李大亮、任城王道宗破吐谷渾於牛心堆。五月乙未,又破之於烏海,追奔至柏海。副總管薛萬均、薛萬徹又破之於赤水源,獲其名王二十人。庚子,太上皇崩於大安宮。壬子,李靖平吐谷渾於西海之上,獲其王慕容伏允。以其子慕容順光降,封為西平郡王,復其本國。秋七月甲寅,增修太廟為六室。



 冬十月庚寅,葬高祖太武皇帝於獻陵。戊申,祔於太廟。辛丑,左僕射、魏國
 公房玄齡加開府儀同三司,餘如故。十二月甲戌,吐谷渾西平郡王慕容順光為其下所弒,遣兵部尚書侯君集率師安撫之,仍封順光子諾曷缽為河源郡王,使統其眾。右光祿大夫、宋國公蕭瑀依舊特進,復令參預朝政。



 十年春正月壬子,尚書左僕射房玄齡、侍中魏徵上梁、陳、齊、周、隋五代史,詔藏於秘閣。癸丑,徙封趙王元景為荊王,魯王元昌為漢王,鄭王元禮為徐王,徐王元嘉為
 韓王,荊王元則為彭王,滕王元懿為鄭王,吳王元軌為霍王,豳王元鳳為虢王,陳王元慶為道王,魏王靈夔為燕王,蜀王恪為吳王,越王泰為魏王,燕王祐為齊王,梁王愔為蜀王,郯王惲為蔣王,漢王貞為越王,申王慎為紀王。夏六月,以侍中魏徵為特進,仍知門下省事。壬申,中書令溫彥博為尚書右僕射。甲戌,太常卿、安德郡公楊師道為侍中。己卯,皇后長孫氏崩於立政殿。冬十一月庚寅,葬文德皇后於昭陵。十二月壬申,吐谷渾河源
 郡王慕容諾曷缽來朝。乙亥,親錄京師囚徒。是歲,關內、河東疾病,命醫賚藥療之。



 十一年春正月丁亥朔,徙鄶王元裕為鄧王,譙王元名為舒王。癸巳,加魏王泰為雍州牧、左武候大將軍。庚子,頒新律令於天下。作飛山宮。甲寅,房玄齡等進所修《五禮》。詔所司行用之。



 二月丁巳,詔曰:



 夫生者天地之大德,壽者修短之一期。生有七尺之形,壽以百齡為限,含靈稟氣,莫不同焉,皆得之於自然,不可以分外企也。是以《
 禮記》云:「君即位而為椑」。莊周云:「勞我以形,息我以死。」豈非聖人遠鑒,通賢深識?末代已來,明闢蓋寡,靡不矜黃屋之尊,慮白駒之過,並多拘忌,有慕遐年。謂雲車易乘,羲輪可駐,異軌同趣,其蔽甚矣。有隋之季,海內橫流,豺狼肆暴,吞噬黔首。朕投袂發憤,情深拯溺,扶翼義師,濟斯塗炭。賴蒼昊降鑒,股肱宣力,提劍指麾,天下大定。此朕之宿志,於斯已畢。猶恐身後之日,子子孫孫,習於流俗,猶循常禮,加四重之櫬,伐百祀之木,勞擾百姓,崇厚
 園陵。今預為此制,務從儉約,於九嵕之山,足容棺而已。積以歲月,漸而備之。木馬塗車,土桴葦龠,事合古典,不為時用。



 又佐命功臣,或義深舟楫,或謀定帷幄,或身摧行陣,同濟艱危,克成鴻業,追念在昔,何日忘之!使逝者無知,咸歸寂寞;若營魂有識,還如疇曩,居止相望,不亦善乎!漢氏使將相陪陵,又給以東園秘器,篤終之義,恩意深厚,古人豈異我哉!自今已後,功臣密戚及德業佐時者,如有薨亡,宜賜塋地一所,及以秘器,使窀穸之時,
 喪事無闕。所司依此營備,稱朕意焉。



 甲子,幸洛陽宮,命祭漢文帝。三月丙戌朔,日有蝕之。丁亥,車駕至洛陽。丙申,改洛州為洛陽宮。辛亥,大蒐於廣城澤。癸丑,還宮。



 夏四月甲子,震乾元殿前槐樹。丙寅,詔河北、淮南舉孝悌淳篤,兼閑時務;儒術該通,可為師範;文辭秀美,才堪著述;明識政體,可委字人:並志行修立,為鄉閭所推者,給傳詣洛陽宮。六月甲寅,尚書右僕射、虞國公溫彥博薨。丁巳,幸明德宮。己未,定制諸王為世封刺史。戊辰,定制
 勛臣為世封刺史。改封任城王道宗為江夏郡王,趙郡王孝恭為河間郡王。己巳,改封許王元祥為江王。秋七月癸未,大霪雨。谷水溢入洛陽宮,深四尺,壞左掖門,毀宮寺十九所;洛水溢,漂六百家。庚寅,詔以災命百官上封事,極言得失。丁酉,車駕還宮。壬寅,廢明德宮及飛山宮之玄圃院,分給遭水之家,仍賜帛有差。丙午,修老君廟於亳州,宣尼廟於兗州,各給二十戶享祀焉。涼武昭王復近墓二十戶充守衛,仍禁芻牧樵採。九月丁亥;河
 溢,壞陜州河北縣,毀河陽中潭。幸白司馬阪以觀之,賜遭水之家粟帛有差。冬十一月辛卯,幸懷州。乙未,狩於濟源。丙午,車駕還宮。十二月辛酉,百濟王遣其太子隆來朝。



 十二年春正月乙未,吏部尚書高士廉等上《氏族志》一百三十卷。壬寅,松、叢二州地震,壞人廬舍,有壓死者。二月乙卯,車駕還京。癸亥,觀砥柱,勒銘以紀功德。甲子,夜郎獠反,夔州都督齊善行討平之。乙丑,次陜州,自新橋
 幸河北縣,祀夏禹廟。丁卯,次柳谷頓,觀鹽池。戊寅,以隋鷹揚郎將堯君素忠於本朝,贈蒲州刺史,仍錄其子孫。閏二月庚辰朔,日有蝕之。丙戌,至自洛陽宮。夏五月壬申,銀青光祿大夫、永興縣公虞世南卒。六月庚子,初置玄武門左右飛騎。秋七月癸酉,吏部尚書、申國公高士廉為尚書右僕射。



 冬十月己卯,狩於始平,賜高年粟帛有差。乙未,至自始平。己亥,百濟遣使貢金甲雕斧。十二月辛巳,右武候將軍上官懷仁大破山獠於壁州。



 十三年春正月乙巳朔,謁獻陵。曲赦三原縣及行從大闢罪。丁未,至自獻陵。戊午,加房玄齡為太子少師。二月丙子,停世襲刺史。三月乙丑,有星孛于畢、昴。



 夏四月戊寅,幸九成宮。甲申,阿史那結社爾犯御營,伏誅。壬寅,雲陽石燃者方丈,晝如灰,夜則有光,投草木於上則焚,歷年而止。自去冬不雨至於五月。甲寅,避正殿,令五品以上上封事,減膳罷役,分使賑恤,申理冤屈,乃雨。



 六月丙申,封皇弟元嬰為滕王。秋八月辛未朔,日有蝕之。庚辰,
 立右武候大將軍、化州都督、懷化郡王李思摩為突厥可汗,率所部建牙於河北。



 冬十月甲申,至自九成宮。十一月辛亥,侍中、安德郡公楊師道為中書令。十二月丁丑,吏部尚書、陳國公侯君集為交河道行軍大總管,帥師伐高昌。乙亥,封皇子福為趙王。壬午,巂州都督王志遠有罪,伏誅。詔於洛、相、幽、徐、齊、並、秦、蒲等州並置常平倉。己丑,吐谷渾河源郡王慕容諾曷缽來逆女。壬辰,狩於咸陽。是歲,滁州言:「野蠶食槲葉,成繭大如柰,其色綠,
 凡六千五百七十石。」高麗、新羅、西突厥、吐火羅、康國、安國、波斯、疏勒、于闐、焉耆、高昌、林邑、昆明及荒服蠻酋,相次遣使朝貢。



 十四年春正月庚子,初命有司讀時令。甲寅,幸魏王泰宅。赦雍州及長安獄大闢罪已下。二月丁丑,幸國子學,親釋奠,赦大理、萬年系囚,國子祭酒以下及學生高第精勤者加一級,賜帛有差。庚辰,左驍衛將軍、淮陽王道明送弘化公主歸於吐谷渾。壬午,幸溫湯。辛卯,至自溫
 湯。乙未,詔以梁皇侃、褚仲都,周熊安生、沈重,陳沈文阿、周弘正、張機,隋何妥、劉焯、劉炫等前代名儒,學徒多行其義,命求其後。



 三月戊午,置寧朔大使,以護突厥。夏五月壬戌,徙封燕王靈夔為魯王。六月乙酉,大風拔木。己丑,薛延陀遣使求婚。乙未,滁州野蠶成繭,凡收八千三百石。八月庚午,新作襄城宮。癸巳,交河道行軍大總管侯君集平高昌,以其地置西州。九月癸卯,曲赦西州大闢罪。乙卯,於西州置安西都護府。冬十月己卯,詔以贈
 司空、河間元王孝恭,贈陜東道大行臺尚書右僕射、鄖節公殷開山,贈民部尚書、渝襄公劉政會等配饗高祖廟庭。閏月乙未,幸同州。甲辰,狩於堯山。庚戌,至自同州。丙辰,吐蕃遣使獻黃金器千斤以求婚。



 十一月甲子朔,日南至。有事於圓丘。十二月丁酉,交河道旋師。吏部尚書、陳國公侯君集執高昌王麴智盛,獻捷於觀德殿,行飲至之禮,賜酺三日。乙卯,高麗世子相權來朝。



 十五年春正月丁卯,吐蕃遣其國相祿東贊來逆女。丁
 丑,禮部尚書、江夏王道宗送文成公主歸吐蕃。辛巳,幸洛陽宮。三月戊申,幸襄城宮。庚午,發襄城宮。



 夏四月辛卯,詔以來年二月有事泰山,所司詳定儀制。五月壬申,並州僧道及老人等抗表,以太原王業所因,明年登封已後,願時臨幸。上於武成殿賜宴,因從容謂侍臣曰:「朕少在太原,喜群聚博戲,暑往寒逝,將三十年矣。」時會中有舊識上者,相與道舊以為笑樂。因謂之曰:』他人之言,或有面諛。公等朕之故人,實以告朕,即日政教,於百姓
 何如?人間得無疾苦耶?」皆奏:「即日四海太平,百姓歡樂,陛下力也。臣等餘年,日惜一日,但眷戀聖化,不知疾苦。」因固請過並州。上謂曰:「飛鳥過故鄉,猶躑躅徘徊;況朕於太原起義,遂定天下,復少小游觀,誠所不忘。岱禮若畢,或冀與公等相見。」於是賜物各有差。丙子,百濟王扶餘璋卒。詔立其世子扶餘義慈嗣其父位,仍封為帶方郡王。



 六月戊申,詔天下諸州,舉學綜古今及孝悌淳篤、文章秀異者,並以來年二月總集泰山。己酉,有星孛於
 太微,犯郎位。丙辰,停封泰山,避正殿以思咎,命食減膳。



 秋七月甲戌,孛星滅。



 冬十月辛卯,大閱於伊闕。壬辰,幸嵩陽。辛丑,還宮。十一月壬戌,廢鄉長。壬申,還京師。癸酉,薛延陀以同羅、僕骨、回紇、靺鞨、霫之眾度漠,屯於白道川。命營州都督張儉統所部兵壓其東境;兵部尚書李勣為朔方行軍總管,右衛大將軍李大亮為靈州道行軍總管,涼州都督李襲譽為涼州道行軍總管,分道以御之。十二月戊子朔,至自洛陽宮。甲辰,李勣及薛延陀戰
 於諾真水,大破之,斬首三千餘級,獲馬萬五千匹,薛延陀跳身而遁。勣旋破突厥思結於五臺縣,虜其男女千餘口,獲羊馬稱是。



 十六年春正月辛未,詔在京及諸州死罪囚徒,配西州為戶;流人未達前所者,徙防西州。兼中書侍郎、江陵子岑文本為中書侍郎,專知機密。夏六月辛卯,詔復隱王建成曰隱太子,改封海陵剌王元吉曰巢剌主。秋七月戊午,司空、趙國公無忌為司徒,尚書左僕射、梁國公玄
 齡為司空。



 九月丁巳,特進、鄭國公魏徵為太子太師,知門下省事如故。冬十一月丙辰,狩於岐山。辛酉,使祭隋文帝陵。丁卯,宴武功士女於慶善宮南門。酒酣,上與父老等涕泣論舊事,老人等遞起為舞,爭上萬歲壽,上各盡一杯。庚午,至自岐州。十二月癸卯,幸溫湯。甲辰,狩於驪山,時陰寒晦冥,圍兵斷絕。上乘高望見之,欲舍其罰,恐虧軍令,乃回轡入谷以避之。是歲,高麗大臣蓋蘇文弒其君高武,而立武兄子藏為王。



 十七年春正月戊辰,右衛將軍、代州都督劉蘭謀反,腰斬。太子太師、鄭國公魏徵薨。戊申,詔圖畫司徒、趙國公無忌等勛臣二十四人於凌煙閣。三月丙辰,齊州都督齊王祐殺長史權萬紀、典軍韋文振,據齊州自守,詔兵部尚書李勣、刑部尚書劉德威發兵討之。兵未至,兵曹杜行敏執之而降,遂賜死於內侍省。丁巳,熒惑守心前星,十九日而退。



 夏四月庚辰朔,皇太子有罪,廢為庶人。漢王元昌、吏部尚書侯君集並坐與連謀,伏誅。丙戌,立晉
 王治為皇太子,大赦,賜酺三日。丁亥,中書令楊師道為吏部尚書。己丑,加司徒、趙國公長孫無忌太子太師,司空、梁國公房玄齡太子太傅;特進、宋國公蕭瑀太子太保,兵部尚書、英國公李勣為太子詹事,仍同中書門下三品。庚寅,上親謁太廟,以謝承乾之過。癸巳,魏王泰以罪降爵為東萊郡王。五月乙丑,手詔舉孝廉茂才異能之士。



 六月己卯朔,日有蝕之。壬午,改葬隋恭帝。丁酉,尚書右僕射高士廉請致仕,詔以為開府儀同三司、同中
 書門下三品。閏月戊午,薛延陀遣其兄子突利設獻馬五萬匹、牛駝一萬、羊十萬以請婚,許之。丙子,徙封東萊郡王泰為順陽王。秋七月庚辰,京城訛言云:「上遣棖棖取人心肝,以祠天狗。」遞相驚悚。上遣使遍加宣諭,月餘乃止。丁酉,司空、太子太傅、梁國公房玄齡以母憂罷職。八月,工部尚書、鄖國公張亮為刑部尚書,參預朝政。九月癸未,徙庶人承乾於黔州。



 冬十月丁巳,房玄齡起復本職。十一月己卯,有事於南郊。壬午,賜天下酺三日。以
 涼州獲瑞石,曲赦涼州,並錄京城及諸州系囚,多所原宥。



 十八年春正月壬寅,幸溫湯。



 夏四月辛亥,幸九成宮。秋八月甲子,至自九成宮。丁卯,散騎常侍清苑男劉洎為侍中,中書侍郎江陵子岑文本、中書侍郎馬周並為中書令。九月,黃門侍郎褚遂良參預朝政。冬十月辛丑朔,日有蝕之。甲辰,初置太子司議郎官員。甲寅,幸洛陽宮。安西都護郭孝恪帥師滅焉耆,執其王突騎支送行在
 所。十一月壬寅,車駕至洛陽宮。庚子,命太子詹事、英國公李勣為遼東道行軍總管,出柳城,禮部尚書、江夏郡王道宗副之;刑部尚書、鄖國公張亮為平壤道行軍總管,以舟師出萊州,左領軍常何、瀘州都督左難當副之。發天下甲士,召募十萬,並趣平壤,以伐高麗。十二月辛丑,庶人承乾死。



 十九年春二月庚戌,上親統六軍發洛陽。乙卯,詔皇太子留定州監國;開府儀同三司、申國公高士廉攝太子
 太傅,與侍中劉洎、中書令馬周、太子少詹事張行成、太子右庶子高季輔五人同掌機務;以吏部尚書、安德郡公楊師道為中書令。贈殷比干為太師,謚曰忠烈,命所司封墓,葺祠堂,春秋祠以少牢,上自為文以祭之。三月壬辰,上發定州,以司徒、太子太師兼檢校侍中、趙國公長孫無忌,中書令岑文本、楊師道從。



 夏四月癸卯,誓師於幽州城南,因大饗六軍以遣之。丁未,中書令岑文本卒於師。癸亥,遼東道行軍大總管、英國公李勣攻蓋
 牟城,破之。五月丁丑,車駕渡遼。甲申,上親率鐵騎與李勣會圍遼東城,因烈風發火弩,斯須城上屋及樓皆盡,麾戰士令登,乃拔之。



 六月丙辰,師至安市城。丁巳,高麗別將高延壽、高惠真帥兵十五萬來援安市,以拒王師。李勣率兵奮擊,上自高峰引軍臨之,高麗大潰,殺獲不可勝紀。延壽等以其眾降,因名所幸山為駐蹕山,刻石紀功焉。賜天下大酺二日。秋七月,李勣進軍攻安市城,至九月不克,乃班師。



 冬十月丙辰,入臨渝關,皇
 太子自定州迎謁。戊午,次漢武臺,刻石以紀功德。十一月辛未,幸幽州。癸酉,大饗,還師。十二月戊申,幸並州。侍中、清苑男劉洎以罪賜死。是歲,薛延陀真珠毗伽可汗死。



 二十年春正月,上在並州。丁丑,遣大理卿孫伏伽、黃門侍郎褚遂良等二十二人,以六條巡察四方,黜陟官吏。庚辰,曲赦並州,宴從官及起義元從,賜粟帛、給復有差。三月己巳,車駕至京師。己丑,刑部尚書、鄭國公張亮謀反,
 誅。閏月癸巳朔,日有蝕之。



 夏四月甲子,太子太師、趙國公長孫無忌,太子太傅、梁國公房玄齡,太子太保、宋國公蕭瑀各辭調護之職,詔許之。六月,遣兵部尚書、固安公崔敦禮,特進、英國公李勣擊破薛延陀於鬱督軍山北,前後斬首五千餘級,虜男女三萬餘人。秋八月甲子,封皇孫為陳王。己巳,幸靈州。庚午,次涇陽頓。鐵勒回紇、拔野古、同羅、僕骨、多濫葛、思結、阿跌、契苾、跌結、渾、斛薛等十一姓各遣使朝貢,奏稱:「延陀可汗不事大國,部
 落烏散,不知所之。奴等各有分地,不能逐延陀去,歸命天子,乞置漢官。」詔遣會靈州。九月甲辰,鐵勒諸部落俟斤、頡利發等遣使相繼而至靈州者數千人,來貢方物,因請置吏,咸請至尊為可汗。於是北荒悉平,為五言詩勒石以序其事。辛亥,靈州地震有聲。



 冬十月,前太子太保、宋國公蕭瑀貶商州刺史。丙戌,至自靈州。



 二十一年春正月壬辰,開府儀同三司、申國公高士廉薨。丁酉,詔以來年二月有事泰山。甲寅,賜京師酺三日。
 二月壬申,詔以左丘明、卜子夏、公羊高、穀梁赤、伏勝、高堂生、戴聖、毛萇、孔安國、劉向、鄭眾、杜子春、馬融、盧植、鄭康成、服子慎、何休、王肅、王輔嗣、杜元凱、範寧等二十一人,代用其書,垂於國胄,自今有事於太學,並命配享宣尼廟堂。丁丑,皇太子於國學釋菜。



 夏四月乙丑,營太和宮於終南之上,改為翠微宮。五月戊子,幸翠微宮。六月癸亥,司徒、趙國公無忌加授揚州都督。秋七月庚子,建玉華宮於宜君縣之鳳凰穀。庚戌,至自翠微宮。八月壬
 戌,詔以河北大水,停封禪。辛未,骨利幹國遣使貢名馬。丁酉,封皇子明為曹王。冬十一月癸卯,徙封順陽王泰為濮王。十二月戊寅,左驍衛大將軍阿史那社爾、右驍衛大將軍契苾何力、安西都護郭孝恪、司農卿楊弘禮為琯山道行軍大總管,以伐龜茲。是歲,墮婆登、乙利、鼻林送、都播、羊同、石、波斯、康國、吐火羅、阿悉吉等遠夷十九國,並遣使朝貢。又於突厥之北至於回紇部落,置驛六十六所,以通北荒焉。



 二十二年春正月庚寅,中書令馬周卒。司徒、趙國公無忌兼檢校中書令,知尚書門下二省事。已亥,刑部侍郎崔仁師為中書侍郎,參知機務。戊戌,幸溫湯。戊申,還宮。二月,前黃門侍郎褚遂良起復黃門侍郎。中書侍郎崔仁師除名,配流連州。癸丑,西番沙缽羅葉護率眾歸附,以其俟斤屈裴祿為忠武將軍,兼大俟斤。戊午,以結骨部置堅昆都督。乙亥,幸玉華宮,乙卯,賜所經高年篤疾粟帛有差。己卯,蒐於華原。



 四月甲寅,磧外蕃人爭牧馬
 出界,上親臨斷決,然後咸服。丁巳,右武候將軍梁建方擊松外蠻,下其部落七十二所。五月庚子,右衛率長史王玄策擊帝那伏帝國,大破之,獲其王阿羅那順及王妃、子等,虜男女萬二千人、牛馬二萬餘以詣闕。使方土那羅邇娑婆於金飆門造延年之藥。吐蕃贊普擊破中天竺國,遣使獻捷。六月癸酉,特進、宋國公蕭瑀薨。秋七月癸卯,司空、梁國公房玄齡薨。八月己酉朔,日有蝕之。九月己亥,黃門侍郎褚遂良為中書令。



 十月癸亥,至自玉
 華宮。十一月戊戌,眉、邛、雅三州獠反,右衛將軍梁建方討平之。庚子,契丹帥窟哥、奚帥可度者並率其部內屬。以契丹部為松漠都督,以奚部置饒樂都督。十二月乙卯,增置殿中侍御史、監察御史各二員,大理寺置平事十員。閏月丁丑朔,昆山道總管阿史那社爾降處密、處月,破龜茲大撥等五十城,虜數萬口,執龜茲王訶黎布失畢以歸,龜茲平,西域震駭。副將薛萬徹脅於闐王伏闍信入朝。癸未,新羅王遣其相伊贊千金春秋及其子
 文王來朝。是歲,新羅女王金善德死,遣冊立其妹真德為新羅王。



 二十三年春正月辛亥,俘龜茲王訶黎布失畢及其相那利等,獻於社廟。二月丙戌,置瑤池都督府,隸安西都護府。丁亥,西突厥肆葉護可汗遣使來朝。三月丙辰,置豐州都督府。自去冬不雨,至於此月己未乃雨。辛酉,大赦。丁卯,敕皇太子於金液門聽政。是月,日赤無光。



 四月己亥,幸翠微宮。五月戊午,太子詹事、英國公李勣為
 疊州都督。辛酉,開府儀同三司、衛國公李靖薨。己巳,上崩於含風殿,年五十二。遺詔皇太子即位於柩前,喪紀宜用漢制。秘不發喪。庚午,遣舊將統飛騎勁兵從皇太子先還京,發六府甲士四千人,分列於道及安化門,翼從乃入;大行御馬輿,從官侍御如常。壬申,發喪。六月甲戌朔,殯於太極殿。八月丙子,百僚上謚曰文皇帝,廟號太宗。庚寅,葬昭陵。上元元年八月,改上尊號曰文武聖皇帝。天寶十三載二月,改上尊號為文武大聖大廣孝
 皇帝。



 史臣曰:臣觀文皇帝發跡多奇,聰明神武。拔人物則不私於黨,負志業則咸盡其才。所以屈突、尉遲,由仇敵而願傾心膂;馬周、劉洎,自疏遠而卒委鈞衡。終平泰階,諒由斯道。嘗試論之:礎潤雲興,蟲鳴螽躍。雖堯、舜之聖,不能用檮杌、窮奇而治平;伊、呂之賢,不能為夏桀、殷辛而昌盛。君臣之際,遭遇斯難,以至抉目剖心,蟲流筋擢,良由遭值之異也。以房、魏之智,不逾於丘、軻,遂能尊主庇
 民者,遭時也。或曰:以太宗之賢,失愛於昆弟,失教於諸子,何也?曰:然,舜不能仁四罪,堯不能訓丹硃,斯前志也。當神堯任讒之年,建成忌功之日,茍除畏逼,孰顧分崩,變故之興,間不容發,方懼「毀巢」之禍,寧虞「尺布」之謠?承乾之愚,聖父不能移也。若文皇自定儲於哲嗣,不騁志於高麗;用人如貞觀之初,納諫比魏徵之日。況周發、周成之世襲,我有遺妍;較漢文、漢武之恢弘,彼多慚德。跡其聽斷不惑,從善如流,千載可稱,一人而已!



 贊曰:昌、發啟國,一門三聖。文定高位,友於不令。管、蔡既誅,成、康道正。貞觀之風,到今歌詠。



\end{pinyinscope}