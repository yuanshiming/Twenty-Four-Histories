\article{卷一 本紀第一 高祖}

\begin{pinyinscope}

 高祖神堯大聖大光孝皇帝姓李氏,諱淵。其先隴西狄道人,涼武昭王暠七代孫也。暠生歆。歆生重耳,仕魏為弘農太守。重耳生熙,為金門鎮將,領豪傑鎮武川,因家焉。儀鳳中,追尊宣皇帝。熙生天錫,仕魏為幢主。大統中,贈司空。儀鳳中,追尊光皇帝。皇祖諱虎,後魏左僕射,封隴西郡公,與周文帝及太保李弼、大司馬獨孤信等以功參佐命,當時稱為「八柱國家」,仍賜姓大野氏。周受禪,追封唐國公,謚曰襄。至隋文帝作相,還復本姓。武德初,追尊景皇帝,廟號太祖,陵曰永康。皇考諱昞,周安州總管、柱國大將軍,襲唐國公,謚曰仁。武德初,追尊元皇帝,廟號世祖,陵曰興寧。



 高祖以周天和元年生於長安,七
 歲襲唐國公。及長,倜儻豁達,任性真率,寬仁容眾,無貴賤咸得其歡心。隋受禪,補千牛備身。文帝獨孤皇后,即高祖從母也,由是特見親愛,累轉譙、隴、岐三州刺史。有史世良者,善相人,謂高祖曰:「公骨法非常,必為人主,願自愛,勿忘鄙言。」高祖頗以自負。大業初,為滎陽、樓煩二郡太守,徵為殿內少監。九年,遷衛尉少卿。遼東之役,督運於懷遠鎮。及楊玄感反,詔高祖馳驛鎮弘化郡,兼知關右諸軍事。高祖歷試中外,素樹恩德,及是結納豪傑,
 眾多款附。時煬帝多所猜忌,人懷疑懼。會有詔征高祖詣行在所,遇疾未謁。時甥王氏在後宮,帝問曰:「汝舅何遲?」王氏以疾對,帝曰:「可得死否?」高祖聞之益懼,因縱酒沉湎,納賄以混其跡焉。十一年,煬帝幸汾陽宮,命高祖往山西、河東黜陟討捕。師次龍門,賊帥母端兒帥眾數千薄於城下。高祖從十餘騎擊之,所射七十發,皆應弦而倒,賊乃大潰。十二年,遷右驍衛將軍。



 十三年,為太原留守,郡丞王威、武牙郎將高君雅為副。群賊蜂起,江
 都阻絕,太宗與晉陽令劉文靜首謀,勸舉義兵。俄而馬邑校尉劉武周據汾陽宮舉兵反,太宗與王威、高君雅將集兵討之。高祖乃命太宗與劉文靜及門下客長孫順德、劉弘基各募兵,旬日間眾且一萬,密遣使召世子建成及元吉於河東。威、君雅見兵大集,恐高祖為變,相與疑懼,請高祖祈雨於晉祠,將為不利。晉陽鄉長劉世龍知之,以告高祖,高祖陰為之備。



 五月甲子,高祖與威、君雅視事,太宗密嚴兵於外,以備非常。遣開陽府司馬
 劉政會告威等謀反,即斬之以徇,遂起義兵。甲戌,遣劉文靜使於突厥始畢可汗,令率兵相應。六月甲申,命太宗將兵徇西河,下之。癸巳,建大將軍府,並置三軍,分為左右:以世子建成為隴西公、左領大都督,左統軍隸焉;太宗為燉煌公、右領大都督,右統軍隸焉。裴寂為大將軍府長史,劉文靜為司馬,石艾縣長殷開山為掾,劉政會為屬,長孫順德、劉弘基、竇琮等分為左右統軍。開倉庫以賑窮乏,遠近響應。秋七月壬子,高祖率兵西圖關
 中,以元吉為鎮北將軍、太原留守。癸丑,發自太原,有兵三萬。丙辰,師次靈石縣,營於賈胡堡。隋武牙郎將宋老生屯霍邑以拒義師。會霖雨積旬,饋運不給,高祖命旋師,太宗切諫乃止。有白衣老父詣軍門曰:「餘為霍山神使謁唐皇帝曰:『八月雨止,路出霍邑東南,吾當濟師。』高祖曰:「此神不欺趙無恤,豈負我哉!」八月辛巳,高祖引師趨霍邑,斬宋老生,平霍邑。丙戌,進下臨汾郡及絳郡。癸巳,至龍門,突厥始畢可汗遣康稍利率兵五百人、馬二
 千匹,與劉文靜會於麾下。隋驍衛大將軍屈突通鎮河東,津梁斷絕,關中向義者頗以為阻。河東水濱居人,競進舟楫,不謀而至,前後數百人。



 九月壬寅,馮翊賊帥孫華、士門賊帥白玄度各率其眾送款,並具舟楫以待義師。高祖令華與統軍王長諧、劉弘基引兵渡河。屈突通遣其武牙郎將桑顯和率眾數千,夜襲長諧,義師不利。太宗以游騎數百掩其後,顯和潰散,義軍復振。丙辰,馮翊太守蕭造以郡來降。戊午,高祖親率眾圍河東,屈突
 通自守不出,乃命攻城,不利而還。文武將吏請高祖領太尉,加置僚佐,從之。華陰令李孝常以永豐倉來降。庚申,高祖率軍濟河,舍於長春宮。三秦士庶至者日以千數,高祖禮之,咸過所望,人皆喜悅。丙寅,遣隴西公建成、司馬劉文靜屯兵永豐倉,兼守潼關,以備他盜。太宗率劉弘基、長孫順德等前後數萬人,自渭北徇三輔,所至皆下。高祖從父弟神通起兵鄠縣,柴氏婦舉兵於司竹,至是並與太宗會。郿縣賊帥丘師利、李仲文,盩厔賊帥
 何潘仁等,合眾數萬來降。乙亥,命太宗自渭汭屯兵阿城,隴西公建成自新豐趣霸上。高祖率大軍自下邽西上,經煬帝行宮園苑,悉罷之,宮女放還親屬。



 冬十月辛巳,至長樂宮,有眾二十萬。京師留守刑部尚書衛文升、右翊衛將軍陰世師、京兆郡丞滑儀挾代王侑以拒義師。高祖遣使至城下,諭以匡復之意,再三皆不報。諸將固請圍城。十一月丙辰,攻拔京城。衛文升先已病死,以陰世師、滑儀等拒義兵,並斬之。癸亥,率百僚,備法駕,立
 代王侑為天子,遙尊煬帝為太上皇,大赦,改元為義寧。甲子,隋帝詔加高祖假黃鉞、使持節、大都督內外諸軍事、大丞相,進封唐王,總錄萬機。以武德殿為丞相府,改教為令。以隴西公建成為唐國世子;太宗為京兆尹,改封秦公;姑臧公元吉為齊公。十二月癸未,丞相府置長史、司錄已下官僚。金城賊帥薛舉寇扶風,命太宗為元帥擊之。遣趙郡公孝恭招慰山南,所至皆下。癸巳,太宗大破薛舉之眾於扶風。屈突通自潼關奔東都,劉文靜
 等追擒於閿鄉,虜其眾數萬。河池太守蕭瑀以郡降。丙午,遣雲陽令詹俊、武功縣正李仲袞徇巴蜀,下之。



 二年春正月戊辰,世子建成為撫寧大將軍、東討元帥,太宗為副,總兵七萬,徇地東都。二月,清河賊帥竇建德僭稱長樂王。吳興人沈法興據丹陽起兵。三月丙辰,右屯衛將軍宇文化及弒隋太上皇於江都宮,立秦王浩為帝,自稱大丞相。徙封太宗為趙國公。戊辰,隋帝進高祖相國,總百揆,備九錫之禮。唐國置丞相以下,立皇高祖
 已下四廟於長安通義里第。



 夏四月辛卯,停竹使符,頒銀菟符於諸郡。戊戌,世子建成及太宗自東都班師。五月乙巳,天子詔高祖冕十有二旒,建天子旌旗,出警入蹕。王后、王女爵命之號,一遵舊典。戊午,隋帝詔曰:



 天禍隋國,大行太上皇遇盜江都,酷甚望夷,釁深驪北。憫予小子,奄造丕愆,哀號永感,心情糜潰。仰惟荼毒,仇復靡申,形影相吊,罔知啟處。相國唐王,膺期命世,扶危拯溺,自北徂南,東征西怨。致九合於諸侯,決百勝於千里。糾率
 夷夏,大庇氓黎,保乂朕躬,系王是賴。德侔造化,功格蒼旻,兆庶歸心,歷數斯在,屈為人臣,載違天命。在昔虞、夏,揖讓相推,茍非重華,誰堪命禹。當今九服崩離,三靈改卜,大運去矣,請避賢路。兆謀布德,顧己莫能,私僮命駕,須歸籓國。予本代王,及予而代,天之所廢,豈其如是!庶憑稽古之聖,以誅四兇;幸值惟新之恩,預充三恪。雪冤恥於皇祖,守禋祀為孝孫,朝聞夕殞,及泉無恨。今遵故事,遜於舊邸,庶官群闢,改事唐朝。宜依前典,趨上尊號,
 若釋重負,感泰兼懷。假手真人,俾除醜逆,濟濟多士,明知朕意。仍敕有司,凡有表奏,皆不得以聞。



 遣使持節、兼太保、邢部尚書、光祿大夫、梁郡公蕭造,兼太尉、司農少卿裴之隱奉皇帝璽綬於高祖。高祖辭讓,百僚上表勸進,至於再三,乃從之。隋帝遜於舊邸。改大興殿為太極殿。



 甲子,高祖即皇帝位於太極殿,命刑部尚書蕭造兼太尉,告於南郊,大赦天下,改隋義寧二年為唐武德元年。官人百姓,賜爵一級。義師所行之處,給復三年。罷郡
 置州,改太守為刺史。丁卯,宴百官於太極殿,賜帛有差。東都留守官共立隋越王侗為帝。壬申,命相國長史裴寂等修律令。



 六月甲戌,太宗為尚書令,相國府長史裴寂為尚書右僕射,相國府司馬劉文靜為納言,隋民部尚書蕭瑀、相國府司錄竇威並為內史令。廢隋《大業律令》,頒新格。己卯,備法駕,迎皇高祖宣簡公已下神主,祔於太廟。追謚妃竇氏為太穆皇后,陵曰壽安。庚辰,立世子建成為後太子。封太宗為秦王,齊國公元吉為齊王。
 封宗室蜀國公孝基為永安王,柱國公道玄為淮陽王,長平公叔良為長平王,鄭國公神通為永康王,安吉公神符為襄邑王,柱國德良為長樂王,上開府道素為竟陵王,上柱國博乂為隴西王,奉慈為渤海王。諸州總管加號使持節。癸未,封隋帝為酅國公。薛舉寇涇州,命秦王為西討元帥征之。改封永康王神通為淮安王。壬辰,加秦王雍州牧,餘官如故。辛丑,內史令竇威卒。秋七月丙午,刑部尚書蕭造為太子太保。追封皇子玄霸為衛
 王。西突厥遣使內附。秦王與薛舉大戰於涇州,我師敗績。



 八月壬午,薛舉死,其子仁杲復僭稱帝,命秦王為元帥以討之。丁亥,詔曰:「隋太常卿高熲、上柱國賀若弼,並抗節不阿,矯枉無撓;司隸大夫薛道衡、刑部尚書宇文弼、左翊衛將軍董純,並懷忠抱義,以陷極刑:宜從褒飾,以慰泉壤。熲可贈上柱國、郯國公,弼贈上柱國、杞國公,各令有司加謚;道衡贈上開府、臨河縣公,贈上開府、平昌縣公,純贈柱國、狄道縣公。」又詔曰:「隋右驍衛大將
 軍李金才、左光祿大夫李敏,並鼎族高門,元功世胄,橫受屠殺,朝野稱冤。然李氏將興,天祚有應,冥契深隱,妄肆誅夷。朕受命君臨,志存刷蕩,申冤旌善,無忘寤寐。金才可贈上柱國、申國公,敏可贈柱國、觀國公。又前代酷濫,子孫被流者,並放還鄉里。」涼州賊帥李軌以其地來降,拜涼州總管,封涼王。



 九月乙巳,親錄囚徒,改銀菟符為銅魚符。辛未,追謚隋太上皇為煬帝。宇文化及至魏州,鴆殺秦王浩,僭稱天子,國號許。



 冬十月壬申朔,日有
 蝕之。李密率眾來降。封皇從父弟襄武公琛為襄武王,黃臺公瑗為廬江王。癸巳,詔行傅仁均所造《戊寅歷》。十一月己酉,以京師穀貴,令四面入關者,車馬牛驢各給課米,充其自食。秦王大破薛仁杲於淺水原,降之,隴右平。乙巳,涼王李軌僭稱天子於涼州。詔頒五十三條格,以約法緩刑。十二月壬申,加秦王太尉、陜東道大行臺。丁丑,封上柱國李孝常為義安王。庚子,李密反於桃林,行軍總管盛彥師追討斬之。



 二年春正月乙卯,初令文官遭父母喪者聽去職。黃門侍郎陳叔達兼納言。二月丙戌,詔天下諸宗人無職任者,不在徭役之限,每州置宗師一人,以相統攝。丁酉,竇建德攻宇文化及於聊城,斬之,傳首突厥。閏月辛丑,劉武周侵我並州。己酉,李密舊將徐世勣以黎陽之眾及河南十郡降,授黎州總管,封曹國公,賜姓李氏。庚戌,上微行都邑,以察氓俗,即日還宮。甲寅,賊帥硃粲殺我使散騎常侍段確,奔洛陽。



 夏四月乙巳,王世充篡越王侗
 位,僭稱天子,國號鄭。辛亥,李軌為其偽尚書安興貴所執以降,河右平。突厥始畢可汗死。五月己卯,酅國公薨,追崇為隋帝,謚曰恭。六月戊戌,令國子學立周公、孔子廟,四時致祭,仍博求其後。癸亥,尚書右僕射裴寂為晉州道行軍總管,以討劉武周。秋七月壬申,置十二軍,以關內諸府分隸焉。王世充遣其將羅士信侵我穀州,士信率其眾來降。西突厥葉護可汗及高昌並遣使朝貢。



 九月辛未,賊帥李子通據江都,僭稱天子,國號吳。沈法
 興據毗陵,僭稱梁王。丁丑,和州賊帥杜伏威遣使來降,授和州總管、東南道行臺尚書令,封楚王。裴寂與劉武周將宋金剛戰於介州,我師敗績,右武衛大將軍姜寶誼死之。並州總管、齊王元吉懼武周所逼,奔於京師,並州陷。乙未,京師地震。



 冬十月己亥。封幽州總管羅藝為燕郡王,賜姓李氏。黃門侍郎楊恭仁為納言。殺民部尚書、魯國公劉文靜。乙卯,討劉武周,軍於蒲州,為諸軍聲援。壬子,劉武周進圍晉州。甲子,上親祠華嶽。十一月丙
 子,竇建德陷黎陽,盡有山東之地。淮安王神通、左武候大將軍李世勣皆沒於賊。十二月丙申,永安王孝基、工部尚書獨孤懷恩、總管於筠為劉武周將宋金剛掩襲,並沒焉。甲辰,狩於華山。壬子,大風拔木。



 三年春正月辛巳,幸蒲州,命祀舜廟。癸巳,至自蒲州。甲午,李世勣於竇建德所自拔歸國。建德僭稱夏王。二月丁酉,京師西南地有聲如山崩。庚子,幸華陰。工部尚書獨孤懷恩謀反,伏誅。三月癸酉,西突厥葉護可汗、高昌
 王曲伯雅遣使朝貢。突厥貢條支巨鳥。己卯,改納言為侍中,內史令為中書令,給事郎為給事中。甲戌,內史侍郎封德彞兼中書令。封賊帥劉孝真為彭城王,賜姓李氏。



 夏四月壬寅,至自華陰。於益州置行臺尚書省。甲寅,加秦王益州道行臺尚書令。秦王大破宋金剛於介州,金剛與劉武周俱奔突厥,遂平並州。偽總管尉遲敬德、尋相以介州降。



 六月壬辰,徙封楚王杜伏威為吳王,賜姓李氏,加授東南道行臺尚書令。丙午,親錄囚徒。封皇
 子元景為趙王,元昌為魯王,元亨為酆王;皇孫承宗為太原王,承道為安陸王,承乾為恆山王,恪為長沙王,泰為宜都王。



 秋七月壬戌,命秦王率諸軍討王世充。遣皇太子鎮蒲州,以備突厥。丙申,突厥殺劉武周於白道。冬十月庚子,懷戍賊帥高開道遣使降,授蔚州總管,封北平郡王,賜姓李氏。



 四年春正月丁卯,竇建德行臺尚書令胡大恩以大安鎮來降,封定襄郡王,賜姓李氏。辛巳,命皇太子總統諸
 軍討稽胡。三月,徙封宜都王泰為衛王。竇建德來援王世充,攻陷我管州。



 夏四月甲寅,封皇子元方為周王,元禮為鄭王,元嘉為宋王,元則為荊王,元茂為越王。初置都護府官員。五月己未,秦王大破竇建德之眾於武牢,擒建德,河北悉平。丙寅,王世充舉東都降,河南平。秋七月甲子,秦王凱旋,獻俘於太廟。丁卯,大赦天下。廢五銖錢,行開元通寶錢。斬竇建德於市;流王世充於蜀,未發,為仇人所害。甲戌,建德餘黨劉黑闥據漳南反。置山東道
 行臺尚書省於洺州。八月,兗州總管徐圓朗舉兵反,以應劉黑闥,僭稱魯王。



 冬十月己丑,加秦王天策上將,位在王公上,領司徒、陜東道大行臺尚書令;齊王元吉為司空。乙巳,趙郡王孝恭平荊州,獲蕭銑。十一月甲申,於洺州置大行臺,廢洺州都督府。庚寅,焚東都紫微宮乾陽殿。會稽賊帥李子通以其地來降。十二月丁卯,命秦王及齊王元吉討劉黑闥。壬申,徙封宋王元嘉為徐王。



 五年春正月丙申,劉黑闥據洺州,僭稱漢東王。三月丁
 未,秦王破劉黑闥於洺水上,盡復所陷州縣,黑闥亡奔突厥。蔚州總管、北平王高開道叛,寇易州。



 夏四月庚戌,秦王還京師,高祖迎勞於長樂宮。壬申,代州總管、定襄郡王大恩為虜所敗,戰死。六月,劉黑闥引突厥寇山東。置諫議大夫官員。秋七月丁亥,吳王伏威來朝。隋漢陽太守馮盎以南越之地來降,嶺表悉定。八月辛亥,以洺、荊、並、幽、交五州為大總管府。改封恆山王承乾為中山王。葬隋煬帝於揚州。丙辰,突厥頡利寇雁門。己未,進寇
 朔州。遣皇太子及秦王討擊,大敗之。



 冬十月癸酉,遣齊王元吉擊劉黑闥於洺州。時山東州縣多為黑闥所守,所在殺長吏以應之。行軍總管、淮陽王道玄與黑闥戰於下博,道玄敗沒。十一月甲申,命皇太子率兵討劉黑闥。丙申,幸宜州,簡閱將士。十二月丙辰,校獵於華池。庚申,至自宜州。皇太子破劉黑闥於魏州,斬之,山東平。



 六年春正月,吳王杜伏威為太子太保。二月辛亥,校獵於驪山。三月乙未,幸昆明池,宴百官。



 夏四月己未,舊宅
 改為通義宮,曲赦京城系囚,於是置酒高會,賜從官帛各有差。癸酉,以尚書右僕射、魏國公裴寂為左僕射,中書令、宋國公蕭瑀為右僕射,侍中、觀國公楊恭仁為吏部尚書。秋七月,突厥頡利寇朔州,遣皇太子及秦王屯並州以備之。



 八月壬子,東南道行臺僕射輔公祏據丹陽反,僭稱宋王,遣趙郡王孝恭及嶺南道大使、永康縣公李靖討之。丙寅,吐谷渾內附。九月丙子,突厥退,皇太子班師。改東都為洛州。高開道引突厥寇幽州。冬十月,
 幸華陰。



 十一月,校獵於沙苑。十二月乙巳,以奉義監為龍躍宮,武功宅為慶善宮。甲寅,至自華陰。



 七年春正月己酉,封高麗王高武為遼東郡王,百濟王扶餘璋為帶方郡王,新羅王金真平為樂浪郡王。二月,高開道為部將張金樹所殺,以其地降。丁巳,幸國子學,親臨釋奠。改大總管府為大都督府。吳王伏威薨。三月戊寅,廢尚書省六司侍郎,增吏部郎中秩正四品,掌選事。戊戌,趙郡王孝恭大破輔公祏,擒之,丹陽平。



 夏四月
 庚子,大赦天下,頒行新律令。以天下大定,詔遭父母喪者聽終制。五月,造仁智宮於宜州之宜君縣。李世勣討徐圓朗,平之。六月辛丑,幸仁智宮。



 秋七月甲午,至自仁智宮。巂州地震山崩,江水咽流。八月戊辰,突厥寇並州,京師戒嚴。壬午,突厥退。乙未,京師解嚴。冬十月丁卯,幸慶善宮。癸酉,幸終南山,謁老子廟。十一月戊辰,校獵於高陵。庚午,至自慶善宮。



 八年春二月己巳,親錄囚徒,多所原宥。



 夏四月,造太和
 宮於終南山。六月甲子,幸太和宮。突厥寇定州,命皇太子往幽州,秦王往並州,以備突厥。八月,並州道總管張公謹與突厥戰於太谷,王師敗績,中書令溫彥博沒於賊。九月,突厥退。冬十月辛巳,幸周氏陂校獵,因幸龍躍宮。十一月辛卯,幸宜州。庚子,講武於同官縣。改封蜀王元軌為吳王,漢王元慶為陳王。加授秦王中書令,齊王元吉侍中。天策上將府司馬宇文士及權檢校侍中。十二月辛酉,至自宜州。



 九年春正月丙寅,命州縣修城隍,備突厥。尚書左僕射、魏國公裴寂為司空。



 二月庚申,加齊王元吉為司徒。戊寅,親祠社稷。三月辛卯,幸昆明池。夏五月辛巳,以京師寺觀不甚清凈,詔曰:



 釋迦闡教,清凈為先,遠離塵垢,斷除貪欲。所以弘宣勝業,修植善根,開導愚迷,津梁品庶。是以敷演經教,檢約學徒,調懺身心,舍諸染著,衣服飲食,咸資四輩。



 自覺王遷謝,像法流行,末代陵遲,漸以虧濫。乃有猥賤之侶,規自尊高;浮惰之人,茍避徭役。妄為
 剃度,托號出家,嗜欲無厭,營求不息。出入閭里,周旋闤闠,驅策田產,聚積貨物。耕織為生,估販成業,事同編戶,跡等齊人。進違戒律之文,退無禮典之訓。至乃親行劫掠,躬自穿窬,造作妖訛,交通豪猾。每罹憲網,自陷重刑,黷亂真如,傾毀妙法。譬茲稂莠,有穢嘉苗;類彼淤泥,混夫清水。又伽藍之地,本曰凈居,棲心之所,理尚幽寂。近代以來,多立寺舍,不求閑曠之境,唯趨喧雜之方。繕採崎嶇,棟宇殊拓,錯舛隱匿,誘納奸邪。或有接延鄽邸,鄰
 近屠酤,埃塵滿室,膻腥盈道。徒長輕慢之心,有虧崇敬之義。且老氏垂化,實貴沖虛,養志無為,遺情物外。全真守一,是謂玄門,驅馳世務,尤乖宗旨。



 朕膺期馭宇,興隆教法,志思利益,情在護持。欲使玉石區分,薰蕕有辨,長存妙道,永固福田,正本澄源,宜從沙汰。諸僧、尼、道士、女寇等,有精勤練行、守戒律者,並令大寺觀居住,給衣食,勿令乏短。其不能精進、戒行有闕、不堪供養者,並令罷遣,各還桑梓。所司明為條式,務依法教,違制之事,悉
 宜停斷。京城留寺三所,觀二所。其餘天下諸州,各留一所。餘悉罷之。事竟不行。



 六月庚申,秦王以皇太子建成與齊王元吉同謀害己,率兵誅之。詔立秦王為皇太子,繼統萬機,大赦天下。八月癸亥,詔傳位於皇太子。尊帝為太上皇,徙居弘義宮,改名太安宮。



 貞觀八年三月甲戌,高祖宴西突厥使者於兩儀殿,顧謂長孫無忌曰:「當今蠻夷率服,古未嘗有。」無忌上千萬歲壽。高祖大悅,以酒賜太宗。太宗又奉觴上壽,流涕而言曰:「百姓獲安,四
 夷咸附,皆奉遵聖旨,豈臣之力!」於是太宗與文德皇后互進御膳,並上服御衣物,一同家人常禮。是歲,閱武於城西,高祖親自臨視,勞將士而還。置酒於未央宮,三品已上咸侍。高祖命突厥頡利可汗起舞,又遣南越酋長馮智戴詠詩,既而笑曰:「胡、越一家,自古未之有也。」太宗奉觴上壽曰:「臣早蒙慈訓,教以文道;爰從義旗,平定京邑。重以薛舉、武周、世充、建德,皆上稟睿算,幸而克定。三數年間,混一區宇。天慈崇寵,遂蒙重任。今上天垂祐,時
 和歲阜,被發左衽,並為臣妾。此豈臣智力,皆由上稟聖算。」高祖大悅,群臣皆呼萬歲,極夜方罷。



 九年五月庚子,高祖大漸,下詔:「既殯之後,皇帝宜於別所視軍國大事。其服輕重,悉從漢制,以日易月。園陵制度,務從儉約。」是日,崩於太安宮之垂拱前殿,年七十。群臣上謚曰大武皇帝,廟號高祖。十月庚寅,葬於獻陵。高宗上元元年八月,改上尊號曰神堯皇帝。天寶十三載二月,上尊號神堯大聖大光孝皇帝。



 史臣曰:有隋季年,皇圖板蕩,荒主燀燎原之焰,群盜發逐鹿之機,殄暴無厭,橫流靡救。高祖審獨夫之運去,知新主之勃興,密運雄圖,未伸龍躍。而屈己求可汗之援,卑辭答李密之書,決神機而速若疾雷,驅豪傑而從如偃草。洎謳謠允屬,揖讓受終,刑名大刬於煩苛,爵位不逾於珝軸。由是攫金有恥,伏莽知非,人懷漢道之寬平,不責高皇之慢罵。然而優柔失斷,浸潤得行,誅文靜則議法不從,酬裴寂則曲恩太過。奸佞由之貝錦,嬖幸得
 以掇蜂。獻公遂間於申生,小白寧懷於召忽。一旦兵交愛子,矢集申孫。匈奴尋犯於便橋,京邑咸憂於左衽。不有聖子,王業殆哉!



 贊曰:高皇創圖,勢若摧枯。國運神武,家難聖謨。言生床笫,禍切肌膚。《鴟鴞》之詠,無損於吾。



\end{pinyinscope}