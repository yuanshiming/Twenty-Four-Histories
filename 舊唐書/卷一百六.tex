\article{卷一百六}

\begin{pinyinscope}

 ○馬懷素褚無量劉子玄兄知柔子貺餗匯秩迅迥徐堅元行沖吳兢韋述弟逌迪蕭直蕭穎士母煚殷踐猷附



 馬懷素,潤州丹徒人也。寓居江都,少師事李善。家貧無燈燭,晝採薪蘇,夜燃讀書,遂博覽經史,善屬文。舉進士,
 又應制舉,登文學優贍科,拜郿尉,四遷左臺監察御史。



 長安中,御史大夫魏元忠為張易之所構,配徙嶺表,太子僕崔貞慎、東宮率獨孤禕之餞於郊外。易之怒,使人誣告貞慎等與元忠同謀,則天令懷素按鞫,遣中使促迫,諷令構成其事,懷素執正不受命。則天怒,召懷素親加詰問,懷素奏曰:「元忠犯罪配流,貞慎等以親故相送,誠為可責,若以為謀反,臣豈誣罔神明?昔彭越以反伏誅,欒布奏事於其尸下,漢朝不坐,況元忠罪非彭越,陛
 下豈加追送之罪。陛下當生殺之柄,欲加之罪,取決聖衷可矣。若付臣推鞫,臣敢不守陛下之法?」則天意解,貞慎等由是獲免。時夏官侍郎李迥秀恃張易之之勢,受納貨賄,懷素奏劾之,迥秀遂罷知政事。懷素累轉禮部員外郎,與源乾曜、盧懷慎、李傑等充十道黜陟使。懷素處事平恕,當時稱之。使還,遷考功員外郎。時貴戚縱恣,請托公行,懷素無所阿順,典舉平允,擢拜中書舍人。開元初,為戶部侍郎,加銀青光祿大夫,累封常山縣公,三
 遷秘書監,兼昭文館學士。



 懷素雖居吏職,而篤學,手不釋卷,謙恭謹慎,深為玄宗所禮,令與左散騎常侍褚無量同為侍讀。每次閣門,則令乘肩輿以進。上居別館,以路遠,則命宮中乘馬,或親自送迎,以申師資之禮。是時秘書省典籍散落,條疏無敘,懷素上疏曰:「南齊已前墳籍,舊編王儉《七志》。已後著述,其數盈多,《隋志》所書,亦未詳悉。或古書近出,前志闕而未編;或近人相傳,浮詞鄙而猶記。若無編錄,難辯淄、澠。望括檢近書篇目,並前志
 所遺者,續王儉《七志》,藏之秘府。」上於是召學涉之士國子博士尹知章等,分部撰錄,並刊正經史,粗創首尾。會懷素病卒,年六十,上特為之舉哀,廢朝一日,贈潤州刺史,謚曰文。



 褚無量,字弘度,杭州鹽官人也。幼孤貧,勵志好學。家近臨平湖,時湖中有龍鬥,傾裏閈就觀之,無量時年十二,讀書晏然不動。及長,尤精《三禮》及《史記》,舉明經,累除國子博士。景龍三年,遷國子司業,兼修文館學士。是歲,中
 宗將親祀南郊,詔禮官學士修定儀注。國子祭酒祝欽明、司業郭山惲皆希旨,請以皇后為亞獻,無量獨與太常博士唐紹、蔣欽緒固爭,以為不可。無量建議曰:



 夫郊祀者,明王之盛事,國家之大禮。行其禮者,不可以臆斷,不可以情求,皆上順天心,下符人事,欽若稽古,率由舊章,然後可以交神明,可以膺福祐。然禮文雖眾,莫如《周禮》。《周禮》者,周公致太平之書,先聖極由衷之典,法天地而行教化,辯方位而敘人倫。其義可以幽贊神明,其文
 可以經緯邦國,備物致用,其可忽乎!至如冬至圓丘,祭中最大,皇后內主,禮位甚尊。若合郊天助祭,則當具著禮典。今遍檢《周官》,無此儀制。蓋由祭天南郊,不以地配,唯將始祖為主,不以祖妣配天,故唯皇帝親行其禮,皇后不合預也。



 謹按《大宗伯》職云:「若王不祭祀,則攝位。」《注》云:「王有故,代行其祭事。」下文云:「凡大祭祀,王後不與,則攝而薦豆籩,徹。」若皇後合助祭,承此下文,即當云「若不祭祀,則攝而薦豆籩。」今於文上更起凡,則是別生餘事。
 夫事與上異,則別起凡。凡者,生上起下之名,不專系於本職。《周禮》一部之內,此例極多,備在文中,不可具錄。又王後助祭,親薦豆籩而不徹。案《九嬪》職云:「凡祭,贊後薦,徹豆籩。」《注》云:「後進之而不徹。」則知中徹者,為宗伯生文。若宗伯攝祭,則宗伯親徹,不別使人。又案「外宗掌宗廟之祀,王後不與,則贊宗伯」。此之一文,與上相證。何以明之?案外宗唯掌宗廟祭祀,不掌郊天,足明此文是宗廟祭也。案王後行事,總在《內宰》職中。檢其職文,唯云:「大祭
 祀,後稞獻則贊,瑤爵亦如之。」《鄭注》云:「謂祭宗廟也。」《注》所以知者,以文云「稞獻」,祭天無稞,以此得知。又祭天之器,則用陶匏,亦無瑤爵,《注》以此得知是宗廟也。又內司服掌王後六服,無祭天之服;而巾車職掌王後之五輅,亦無後祭天之輅;祭天七獻,無後亞獻。以此諸文參之,故知後不合助祭天也。



 唯《漢書》《郊祀志》則有天地合祭,皇后預享之事,此則西漢末代,強臣擅朝,悖亂彞倫,黷神諂祭,不經之典,事涉誣神。故《易傳》曰:「誣神者,殃及三代。」《
 太誓》曰:「正稽古立功立事,可以永年,承天之大律。」斯史策之良誡,豈可不知。今南郊禮儀,事不稽古,忝守經術,不敢默然。請旁詢碩儒,俯摭舊典,採曲臺之故事,行圓丘之正儀,使聖朝葉昭曠之塗,天下知文物之盛,豈不幸甚。



 時左僕射韋巨源等阿旨,葉同欽明之議,竟不從無量所奏。



 尋以母老請停官歸侍。景雲初,玄宗在春宮,召拜國子司業,兼皇太子侍讀,嘗撰《翼善記》以進之,皇太子降書嘉勞,齎絹四十匹。太極元年,皇太子國學親
 釋奠,令無量講《老經》、《禮記》,各隨端立義,博而且辯,觀者嘆服焉。既畢,進授銀青光祿大夫,兼賜以章服,並彩絹百段。玄宗即位,遷郯王傅,兼國子祭酒。尋以師傅恩遷左散騎常侍,仍兼國子祭酒,封舒國公,實封二百戶。未幾,丁憂解職,廬於墓側。其所植松柏,時有鹿犯之,無量泣而言曰:「山中眾草不少,何忍犯吾先塋樹哉!」因通夕守護。俄有群鹿馴狎,不復侵害,無量因此終身不食鹿肉。服闋,召拜左散騎常侍,復為侍讀。以其年老,每隨仗
 出入,特許緩行,又為造腰輿,令內給使輿於內殿。無量頻上書陳時政得失,多見納用。又嘗手敕褒美,賜物二百段。



 無量以內庫舊書,自高宗代即藏在宮中,漸致遺逸,奏請繕寫刊校,以弘經籍之道。玄宗令於東都乾元殿前施架排次,大加搜寫,廣採天下異本。數年間,四部充備,仍引公卿已下入殿前,令縱觀焉。開元六年駕還,又敕無量於麗正殿以續前功。皇太子及郯王嗣直等五人,年近十歲,尚未就學,無量繕寫《論語》、《孝經》各五本
 以獻。上覽之曰:「吾知無量意無量。」遽令選經明篤行之士國子博士卻恆通郭謙光、左拾遺潘元祚等,為太子及郯王已下侍讀。七年,詔太子就國子監行齒胄之禮,無量登座說經,百僚集觀,禮畢,賞賜甚厚。明年,無量病卒,年七十五。臨終遺言以麗正寫書未畢為恨。上為舉哀,廢朝兩日,贈禮部尚書,謚曰文。



 初,無量與馬懷素俱為侍讀,顧待甚厚;及無量等卒後,秘書少監康子原、國子博士侯行果等又入侍講,雖屢加賞賜,而禮遇不逮
 褚焉。



 劉子玄,本名知幾,楚州刺史胤之族孫也。少與兄知柔俱以詞學知名,弱冠舉進士,授獲嘉主簿。證聖年,有制文武九品已上各言時政得失,知幾上表陳四事,詞甚切直。是時官爵僭濫而法網嚴密,士類競為趨進而多陷刑戮,知幾乃著《思慎賦》以刺時,且以見意。鳳閣侍郎蘇味道、李嶠見而嘆曰:「陸機《豪士》所不及也。」



 知幾長安中累遷左史,兼修國史。擢拜鳳閣舍人,修史如故。景龍
 初,再轉太子中允,依舊修國史。時侍中韋巨源紀處訥、中書令楊再思、兵部尚書宗楚客、中書侍郎蕭至忠並監修國史,知幾以監修者多,甚為國史之弊。蕭至忠又嘗責知幾著述無課,知幾於是求罷史任,奏記於至忠曰:



 僕自策名士伍,待罪朝列,三為史臣,再入東觀,竟不能勒成國典,貽彼後來者,何哉?靜言思之,其不可者五也。何者?古之國史,皆出自一家,如魯、漢之丘明、子長,晉、齊之董狐、南史,咸能立言不朽,藏諸名山,未聞藉以眾
 功,方雲絕筆。唯後漢東觀,大集群儒,而著述無主,條章靡立。由是伯度譏其不實,公理以為可焚,張、蔡二子紀之於當代,傅、範兩家嗤之於後葉。今史司取士,有倍東京,人自以為荀、袁,家自稱為政、駿。每欲記一事,載一言,皆閣筆相視,含毫不斷。故首白可期,而汗青無日。其不可一也。



 前漢郡國計書,先上太史,副上丞相;後漢公卿所撰,始集公府,乃上蘭臺。由是史官所修,載事為博。原自近古,此道不行,史臣編錄,唯自詢採。而左右二史,闕
 注起居;衣冠百家,罕通行狀。求風俗於州郡,視聽不該;討沿革於臺閣,簿籍難見。雖使尼父再出,猶且成其管窺,況限以中才,安能遂其博物。其不可二也。



 昔董狐之書法也,以示於朝;南史之書弒也,執簡以往。而近代史局,皆通籍禁門,幽居九重,欲人不見。尋其義者,由杜彼顏面,防諸請謁故也。然今館中作者,多士如林,皆願長喙,無聞䶦舌。倘有五始初成,一字加貶,言未絕口而朝野具知,筆未棲毫而搢紳咸誦。夫孫盛實錄,取嫉權門;
 王韶直書,見讎貴族。人之情也,能無畏乎!其不可三也。



 古者刊定一史,纂成一家,體統各殊,指歸咸別。夫《尚書》之教也,以疏通知遠為主;《春秋》之義也,以懲惡勸善為先。《史記》則退處士而進奸雄,《漢書》則抑忠臣而飾主闕。斯並曩賢得失之例,良史是非之準,作者言之詳矣。頃史官注記,多取稟監修,楊令公則云「必須直詞」,宗尚書則云「宜多隱惡」。十羊九牧,其事難行;一國三公,適從焉在?其不可四也。



 竊以史置監修,雖無古式,尋其名號,可
 得而言。夫言監者,蓋總領之義耳。如創紀編年,則年有斷限;草傳敘事,則事有豐約。或可略而不略,或應書而不書,此失刊削之例也。屬詞比事,勞逸宜均;揮鉛奮墨,勤惰須等。某帙某篇,付之此職;某紀某傳,歸之此官。此銓配之理也。斯並宜明立科條,審定區域,倘人思自勉,則書可立成。今監之者既不指授,修之者又無遵奉。用使爭學茍且,務相推避,坐變炎涼,徒延歲月。其不可五也。



 凡此不可,其流實多,一言以蔽,三隅自反。而時談物
 議,焉得笑僕編次無聞者哉!比者伏見明公每汲汲於勸誘,勤勤於課績。或云墳籍事重,努力用心;或云歲序已淹,何時輟手?竊以綱維不舉,而督課徒勤,雖威以次骨之刑,勖以懸金之賞,終不可得也。語曰:「陳力就列,不能則止。」僕所以比者布懷知己,歷詆群公,屢辭載筆之官,願罷記言之職者,正為此耳。當今朝號得人,國稱多士。蓬山之下,良直差肩;蕓閣之中,英奇接武。僕既功虧刻鵠,筆未獲麟,徒殫太官之膳,虛索長安之米,乞以本
 職,還其舊居,多謝簡書,請避賢路。惟明公足下哀而許之。



 至忠惜其才,不許解史任。宗楚客嫉其正直,謂諸史官曰:「此人作書如是,欲置我何地!」



 時知幾又著《史通子》二十卷,備論史策之體。太子右庶子徐堅深重其書,嘗云:「居史職者,宜置此書於座右。」知幾自負史才,常慨時無知己,乃委國史於著作郎吳兢,別撰《劉氏家史》十五卷、《譜考》三卷。推漢氏為陸終苗裔,非堯之後。彭城叢亭里諸劉,出自宣帝子楚孝王囂曾孫司徒居巢侯劉愷
 之後,不承楚元王交。皆按據明白,正前代所誤,雖為流俗所譏,學者服其該博。初,知幾每云若得受封,必以居巢為名,以紹司徒舊邑;後以修《則天實錄》功,果封居巢縣子。又鄉人以知幾兄弟六人進士及第,文學知名,改其鄉里為高陽鄉居巢裏。



 景雲中,累遷太子左庶子,兼崇文館學士,仍依舊修國史,加銀青光祿大夫。時玄宗在東宮,知幾以名音類上名,乃改子玄。二年,皇太子將親釋奠於國學,有司草儀注,令從塵皆乘馬著衣冠。子
 玄進議曰:



 古者自大夫已上,皆乘車而以馬為騑服。魏、晉已降,迄乎隋代,朝士又駕牛車,歷代經史,具有其事,不可一二言也。至如李廣北征,解鞍憩息;馬援南伐,據鞍顧盼。斯則鞍馬之設,行於軍旅;戎服所乘,貴於便習者也。按江左官至尚書郎而輒輕乘馬,則為御史所彈。又顏延之罷官後,好騎馬出入閭里,當代稱其放誕。此則專車憑軾,可擐朝衣;單馬御鞍,宜從褻服。求之近古,灼然之明驗也。



 自皇家撫運,沿革隨時。至如陵廟巡謁,
 王公冊命,則盛服冠履,乘彼輅車。其士庶有衣冠親迎者,亦時以服箱充馭。在於他事,無復乘車,貴賤所行,通用鞍馬而已。臣伏見比者鑾輿出幸,法駕首途,左右侍臣,皆以朝服乘馬。夫冠履而出,只可配車而行,今乘車既停,而冠履不易,可謂唯知其一而未知其二也。何者?褒衣博帶,革履高冠,本非馬上所施,自是車中之服。必也韈而升鐙,跣以乘鞍,非唯不師古道,亦自取驚今俗。求諸折中,進退無可。且長裾廣袖,示詹如翼如,鳴珮行組,
 鏘鏘奕奕,馳驟於風塵之內,出入於旌棨之間,倘馬有驚逸,人從顛墜,遂使屬車之右,遣履不收,清道之傍,絓驂相續,固以受嗤行路,有損威儀。



 今議者皆雲秘閣有《梁武帝南郊圖》,多有危冠乘馬者,此則近代故事,不得謂無其文。臣案此圖是後人所為,非當時所撰。且觀代間有古今圖畫者多矣,如張僧繇畫《群公祖二疏》,而兵士有著芒屩者;閻立本畫《明君入匈奴》,而歸人有著帷帽者。夫芒屩出於水鄉,非京華所有;帷帽創於隋代,非
 漢官所作。議者豈可徵此二畫,以為故實者乎?由斯而言,則《梁氏南郊之圖》,義同於此。又傅稱因俗,禮貴緣情。殷輅周冕,規模不一;秦冠漢佩,用舍無常。況我國家道軼百王,功高萬古,事有不便,理資變通,其乘馬衣冠,竊謂宜從省廢。臣懷此異議,其來自久,日不暇給,未及搉楊。今屬殿下親從齒胄,將臨國學,凡有衣冠乘馬,皆憚此行,所以輒進狂言,用申鄙見。



 皇太子手令付外宣行,仍編入令,以為常式。



 開元初,遷左散騎常侍,修史如故。
 九年,長子貺為太樂令,犯事配流。子玄詣執政訴理,上聞而怒之,由是貶授安州都督府別駕。子玄掌知國史,首尾二十餘年,多所撰述,甚為當時所稱。禮部尚書鄭惟忠嘗問子玄曰:「自古已來,文士多而史才少,何也?」對曰:「史才須有三長,世無其人,故史才少也。三長:謂才也,學也,識也。夫有學而無才,亦猶有良田百頃,黃金滿籝,而使愚者營生,終不能致於貨殖者矣。如有才而無學,亦猶思兼匠石,巧若公輸,而家無楩楠斧斤,終不果成
 其宮室者矣。猶須好是正直,善惡必書,使驕主賊臣,所以知懼,此則為虎傅翼,善無可知,所向無敵者矣。脫茍非其才,不可叨居史任。自夐古已來,能應斯目者,罕見其人。」時人以為知言。子玄至安州,無幾而卒,年六十一。自幼及長,述作不倦,朝有論著,必居其職。預修《三教珠英》、《文館詞林》、《姓族系錄》,論《孝經》非鄭玄注、《老子》河上公注,修《唐書實錄》,皆行於代,有集三十卷。後數年,玄宗敕河南府就家寫《史通》以進,讀而善之,追贈汲郡太守;尋
 又贈工部尚書,謚曰文。



 兄知柔,少以文學政事,歷荊揚曹益宋海唐等州長史刺史、戶部侍郎、國子司業、鴻臚卿、尚書右丞、工部尚書、東都留守。卒,贈太子少保,謚曰文。代傳儒學之業,時人以述作名其家。



 子玄子貺、餗、匯、秩、迅、迥,皆知名於時。



 貺,博通經史,明天文、律歷、音樂、醫算之術,終於起居郎、修國史。撰《六經外傳》三十七卷、《續說苑》十卷、《太樂令壁記》三卷、《真人肘後方》三卷、《天宮舊事》一卷。



 餗,右補闕、集賢殿學士、修國史。著《史例》三卷、《傳
 記》三卷、《樂府古題解》一卷。



 匯,給事中、尚書右丞、左散騎常侍、荊南長沙節度,有集三卷。



 秩,給事中、尚書右丞、國子祭酒。撰《政典》三十五卷、《止戈記》七卷、《至德新議》十二卷、《指要》三卷。論喪紀制度加籩豆,許私鑄錢,改制國學,事各在本志。



 迅,右補闕,撰《六說》五卷。



 迥,諫議大夫、給事中,有集五卷。



 貺子浹、滋,匯子贊。滋,貞元中位至宰輔。贊,觀察使,自有傳。



 徐堅,西臺舍人齊聃子也。少好學,遍覽經史,性寬厚長
 者。進士舉,累授太學。聖歷中,車駕在三陽宮,御史大夫楊再思、太子左庶子王方慶為東都留守,引堅為判官,表奏專以委之。方慶善《三禮》之學,每有疑滯,常就堅質問,堅必能征舊說,訓釋詳明,方慶深善之。又賞其文章典實,常稱曰:「掌綸誥之選也。」再思亦曰:「此鳳閣舍人樣,如此才識,走避不得。」堅又與給事中徐彥伯、定王府倉曹劉知幾、右補闕張說同修《三教珠英》。時麟臺監張昌宗及成均祭酒李嶠總領其事,廣引文詞之士,日夕談
 論,賦詩聚會,歷年未能下筆。堅獨與說構意撰錄,以《文思博要》為本,更加《姓氏》、《親族》二部,漸有條匯。諸人依堅等規制,俄而書成,遷司封員外郎。則天又令堅刪改《唐史》,會則天遜位而止。



 神龍初,再遷給事中。時雍州人韋月將上書告武三思不臣之跡,反為三思所陷,中宗即令殺之。時方盛夏,堅上表曰:「月將誣構良善,故違制命,準其情狀,誠合嚴誅。但今硃夏在辰,天道生長,即從明戮,有乖時令。謹按《月令》:『夏行秋令,則丘隰水潦,禾稼不
 熟。』陛下誕膺靈命,中興聖圖,將弘義、軒之風,以光史策之美,豈可非時行戮,致傷和氣哉!君舉必書,將何以訓?伏願詳依國典,許至秋分,則知恤刑之規,冠於千載;哀矜之惠,洽乎四海。」中宗納堅所奏,遂令決杖,配流嶺表。



 睿宗即位,堅自刑部侍郎加銀青光祿大夫,拜左散騎常侍,俄轉黃門侍郎。時監察御史李知古請兵以擊姚州西貳河蠻,既降附,又請築城,重征稅之。堅以蠻夷生梗,可以羈縻屬之,未得同華夏之制,勞師涉遠,所損不
 補所獲,獨建議以為不便。睿宗不從,令知古發劍南兵往築城,將以列置州縣。知古因是欲誅其豪傑,沒子女以為奴婢。蠻眾恐懼,乃殺知古,相率反叛,役徒奔潰,姚、巂路由是歷年不通。



 堅妻即侍中岑羲之妹,堅以與羲近親,固辭機密,乃轉太子詹事,謂人曰:「非敢求高,蓋避難也。」及羲誅,堅竟免坐累。出為絳州刺史,五轉復入為秘書監。開元十三年,再遷左散騎常侍。其年,玄宗改麗正書院為集賢院,以堅為學士,副張說知院事,累封東
 海郡公。以修東封儀注及從升太山之功,特加光祿大夫。堅多識典故,前後修撰格式、氏族及國史等,凡七入書府,時論美之。十七年卒,年七十餘。上深悼惜之,遣中使就家吊,內出絹布以賻,贈太子少保,謚曰文。堅長姑為太宗充容,次姑為高宗婕妤,並有文藻。堅父子以詞學著聞,議者方之漢世班氏。



 元行沖,河南人,後魏常山王素連之後也。少孤,為外祖司農卿韋機所養。博學多通,尤善音律及詁訓之書。舉
 進士,累轉通事舍人,納言狄仁傑甚重之。行沖性不阿順,多進規誡,嘗謂仁傑曰:「下之事上,亦猶蓄聚以自資也。譬貴家儲積,則脯臘膎胰以供滋膳,參術芝桂以防痾疾。伏想門下賓客,堪充旨味者多,願以小人備一藥物。」仁傑笑而謂人曰:「此吾藥籠中物,何可一日無也!」九遷至陜州刺史,兼隴右、關內兩道按察使,未行,拜太常少卿。



 行沖以本族出於後魏,而未有編年之史,乃撰《魏典》三十卷,事詳文簡,為學者所稱。初魏明帝時,河西柳
 谷瑞石有牛繼馬後之象,魏收舊史以為晉元帝是牛氏之子,冒姓司馬,以應石文。行沖推尋事跡,以後魏昭成帝名犍,繼晉受命,考校謠讖,著論以明之。



 開元初,自太子詹事出為岐州刺史,又充關內道按察使。行沖自以書生不堪博擊之任,固辭按察,乃以寧州刺史崔琬代焉。俄復入為右散騎常侍、東都副留守。時嗣彭王志柬庶兄志謙被人誣告謀反,考訊自誣,系獄待報,連坐十數人,行沖察其冤濫,並奏原之。四遷大理卿。時揚州
 長史李傑為侍御史王旭所陷,詔下大理結罪,行沖以傑歷政清貞,不宜枉為讒邪所構,又奏請從輕條出之。當時雖不見從,深為時論所美。俄又固辭刑獄之官,求為散職。七年,復轉左散騎常侍。九遷國子祭酒,月餘,拜太子賓客、弘文館學士。累封常山郡公。



 先是,秘書監馬懷素集學者續王儉《今書七志》,左散騎常侍褚無量於麗正殿校寫四部書,事未就而懷素、無量卒,詔行沖總代其職。於是行沖表請通撰古今書目,名為《群書四錄》,
 命學士鄠縣尉毋煚、櫟陽尉韋述、曹州司法參軍殷踐猷、太學助教余欽等分部修檢,歲餘書成,奏上,上嘉之。又特令行沖撰御所注《孝經》疏義,列於學官。尋以衰老罷知麗正殿校寫書事。



 初,有左衛率府長史魏光乘奏請行用魏徵所注《類禮》,上遽令行沖集學者撰《義疏》,將立學官。行沖於是引國子博士範行恭、四門助教施敬本檢討刊削,勒成五十卷,十四年八月奏上之。尚書左丞相張說駁奏曰:「今之《禮記》,是前漢戴德、戴聖所編錄,
 歷代傳習,已向千年,著為經教,不可刊削。至魏孫炎始改舊本,以類相比,有同抄書,先儒所非,竟不行用。貞觀中,魏徵因孫炎所修,更加整比,兼為之注,先朝雖厚加賞錫,其書竟亦不行。今行沖等解征所注,勒成一家,然與先儒第乖,章句隔絕,若欲行用,竊恐未可。」上然其奏,於是賜行沖等絹二百匹,留其書貯於內府,竟不得立於學官。行沖恚諸儒排己,退而著論以自釋,名曰《釋疑》。其詞曰:



 客問主人曰:「小戴之學,行之已久;康成銓注,見
 列學官。傳聞魏公,乃有刊易;又承制旨,造疏將頒。未悉二經,孰為優劣?」主人答曰:「小戴之禮,行於漢末,馬融注之,時所未睹。盧植分合二十九篇而為說解,代不傳習。鄭絪子干,師於季長。屬黨錮獄起,師門道喪,康成於竄伏之中,理紛拿之典,志存探究,靡所咨謀。而猶緝述忘疲,聞義能徙,具於《鄭志》,向有百科。章句之徒,曾不窺覽,猶遵覆轍,頗類刻舟。王肅因之,重茲開釋,或多改駁,仍按本篇。又鄭學之徒,有孫炎者,雖扶玄義,乃易前編。自
 後條例支分,箴石間起。馬伷增革,向逾百篇;葉遵刪修,僅全十二。魏公病群言之錯雜,紬眾說之精深。經文不同,未敢刊正;注理睽誤,寧不芟礱。成畢上聞,太宗嘉賞,齎縑千匹,錄賜儲籓。將期頒宣,未有疏義。聖皇纂業,耽古崇儒,高曾規矩,宜所修襲,乃制昏愚,甄分舊義。其有注遺往說,理變新文,務加搜窮,積稔方畢。具錄呈進,敕付群儒,庶能斟詳,以課疏密。豈悟章句之士,堅持昔言,特嫌知新,欲仍舊貫,沉疑多月,擯壓不申,優劣短長,定
 於通識,手成口答,安敢銓量。」



 客曰:「當局稱迷,傍觀見審,累朝銓定,故是周詳,何所為疑,不為申列?」答曰:「是何言歟?談豈容易!昔孔安國注壁中書,會巫蠱事,經籍道息。族兄臧與之書曰:『相如常忿俗儒淫詞冒義,欲撥亂反正而未能果。然雅達通博,不代而生;浮學宋株,比肩皆是。眾非難正,自古而然。誠恐此道未申,而以獨智為議也。』則知變易章句,其難一矣。



 「漢有孔季產者,專於古學;有孔扶者,隨俗浮沉。扶謂產云:『今朝廷皆為章句內學,
 而君獨修古義,修古義則非章句內學,非章句內學則危身之道也。獨善不容於代,必將貽患禍乎!」則知變易章句,其難二矣。



 「劉歆以通書屬文,待詔官署,見《左氏傳》而大好之,後蒙親近,欲建斯業。哀帝欣納,令其討論,各遷延推辭,不肯置對。劉歆移書責讓,其言甚切,諸博士等皆忿恨之。名儒龔勝,時為光祿,見歆此議,乃乞骸骨;司空師丹,因大發怒,奏歆改亂前志,非毀先朝所立。帝曰:「此廣道術,何為毀耶?」由是犯忤大臣,懼誅,求出為河
 南太守,宗室不典三河,又徙五原太守。以君實之著名好學,公仲之深博守道,猶迫同門朋黨之議,卒令子駿負謗於時。則知變易章句,其難三矣。



 「子雍規玄數十百件,守鄭學者,時有中郎馬昭,上書以為肅繆。詔王學之輩,占答以聞。又遣博士張融案經論詰,融登召集,分別推處,理之是非,具《呈證論》。王肅酬對,疲於歲時。則知變易章句,其難四矣。



 「卜商疑聖,納誚於曾輿;木賜近賢,貽嗤於武叔。自此之後,唯推鄭公。王粲稱伊、洛已東,淮、漢
 之北,一人而已,莫不宗焉。咸云先儒多闕,鄭氏道備,粲竊嗟怪,因求其學。得《尚書注》,退而思之,以盡其意,意皆盡矣。所疑之者,猶未喻焉。凡有兩卷,列於其集。又王肅改鄭六十八條,張融核之,將定臧否。融稱玄注泉深廣博,兩漢四百餘年,未有偉於玄者。然二郊之祭,殊天之祀,此玄誤也。其如皇天祖所自出之帝,亦玄慮之失也。及服虔釋《傳》,未免差違,後代言之,思弘聖意,非謂揚己之善,掩人之名也。何者?君子用心,願聞其過,故仲尼曰:『
 過也人皆見之,更也人皆仰之』是也。而專門之徒,恕己及物,或攻先師之誤,如聞父母之名,將謂亡者之德言而見壓於重壤也。故王劭《史論》曰:『魏、晉浮華,古道夷替,洎王肅、杜預,更開門戶。歷載三百,士大夫恥為章句。唯草野生以專經自許,不能究覽異義,擇從其善。徒欲父康成,兄子慎,寧道孔聖誤,諱聞鄭、服非。然於鄭、服甚憒憒,鄭、服之外皆仇也。』則知變易章句,其難五也。



 「伏以安國《尚書》、劉歆《左傳》,悉遭擯於曩葉,見重於來今。故知二
 人之鑒,高於漢廷遠矣。孔秀產云:『物極則變。比及百年外,當有明直君子,恨不與吾同代者。』於戲!道之行廢,必有其時者歟!僕非專經,罕習章句,高名不著,易受經誣。頃者修撰,殆淹年月,賴諸賢輩能左右之,免致愆尤,仍叨賞齎,內省昏朽,其榮已多。何遽持一己之區區,抗群情之噂沓褷,舍勿矜之美,成自我之私,觸近名之誡,興犯眾之禍?一舉四失,中材不為,是用韜聲,甘此沉默也。」



 行沖俄又累表請致仕,制許之。十七年卒,年七十七,贈禮
 部尚書,謚曰獻。



 吳兢,汴州浚儀人也。勵志勤學,博通經史。宋州人魏元忠、亳州人硃敬則深器重之,及居相輔,薦兢有史才,堪居近侍,因令直史館,修國史。累月,拜右拾遺內供奉。神龍中,遷右補闕,與韋承慶、崔融、劉子玄撰《則天實錄》成,轉起居郎。俄遷水部郎中,丁憂還鄉里。開元三年服闋,抗疏言曰:「臣修史已成數十卷,自停職還家,匪忘紙札,乞終餘功。」乃拜諫議大夫,依前修史。俄兼修文館學士,
 歷衛少卿、右庶子。居職殆三十年,敘事簡要,人用稱之。末年傷於太簡。《國史》未成,十七年,出為荊州司馬,制許以史稿自隨。中書令蕭嵩監修國史,奏取兢所撰《國史》,得六十五卷。累遷臺、洪、饒、蘄四州刺史,加銀青光祿大夫,遷相州長垣縣子。天寶初改官名,為鄴郡太守,入為恆王傅。



 兢嘗以梁、陳、齊、周、隋五代史繁雜,乃別撰《梁》、《齊》、《周史》各十卷、《陳史》五卷、《隋史》二十卷,又傷疏略。兢雖衰耗,猶希史職,而行步傴僂,李林甫以其年老不用。天
 寶八年,卒於家,時年八十餘。兢卒後,其子進兢所撰《唐史》八十餘卷,事多紕繆,不逮於壯年。兢家聚書頗多,嘗目錄其卷第,號《吳氏西齋書目》。



 韋述,司農卿弘機曾孫也。父景駿,房州刺史。述少聰敏,篤志文學。家有書二千卷,述為兒童時,記覽皆遍。人駭異之。景龍中,景駿為肥鄉令,述從父至任。洺州刺史元行沖,景駿之姑子,為時大儒,常載書數車自隨。述入其書齋,忘寢與食。行沖異之,引與之談,貫穿經史,事如指
 掌,探賾奧旨,如遇師資。又試以綴文,操牘便就。行沖大悅,引之同榻曰:「此吾外家之寶也。」舉進士,西入關,時述甚少,儀形眇小。考功員外郎宋之問曰:「韋學士童年有何事業?」述對曰:「性好著書。述有所撰《唐春秋》三十卷,恨未終篇。至如詞策,仰待明試。」之問曰:「本求異才,果得遷、固。」是歲登科。



 開元五年,為櫟陽尉。秘書監馬懷素受詔編次圖書,乃奏用左散騎常侍元行沖、左庶子齊澣、秘書少監王珣、衛尉少卿吳兢並述等二十六人,同於秘閣
 詳錄四部書。懷素尋卒,行沖代掌其事,五年而成,其總目二百卷。述好譜學,秘閣中見常侍柳沖先撰《姓族系錄》二百卷,述於分課之外手自抄錄,暮則懷歸。如是周歲,寫錄皆畢,百氏源流,轉益詳悉。乃於《柳錄》之中,別撰成《開元譜》二十卷。其篤志忘倦,皆此類也。



 轉右補闕,中書令張說專集賢院事,引述為直學士,遷起居舍人。說重詞學之士,述與張九齡、許景先、袁暉、趙冬曦、孫逖、王幹常游其門。趙冬曦兄冬日,弟知壁、居貞、安貞、頤貞等
 六人,述弟迪、逌、迥、起、巡亦六人,並詞學登科。說曰:「趙、韋昆季,令之杞梓也。」十八年,兼知史官事,轉屯田員外郎、職方吏部二郎中,學士、知史官事如故。及張九齡為中書令,即集賢之同職,裴耀卿為侍中,即述之舅,皆相推重,語必移晷。二十七年,轉國子司業,停知史事。俄而復兼史職,充集賢學士。天寶初,歷左右庶子,加銀青光祿大夫。九載,兼充禮儀使。其載遷尚書工部侍郎,封方城縣侯。



 述在書府四十年,居史職二十年,嗜學著書,手不釋卷。
 國史自令狐德棻至於吳兢,雖累修撰,竟未成一家之言。至述始定類例,補遺續闕,勒成《國史》一百一十二卷,並《史例》一卷,事簡而記詳,雅有良史之才,蘭陵蕭穎士以為譙周、陳壽之流。述早以儒術進,當代宗仰,而純厚長者,澹於勢利,道之同者,無間貴賤,皆禮接之。家聚書二萬卷,皆自校定鉛槧,雖御府不逮也。兼古今朝臣圖,歷代知名人畫,魏、晉已來草隸真跡數百卷,古碑、古器、藥方、格式、錢譜、璽譜之類,當代名公尺題,無不畢備。及
 祿山之亂,兩京陷賊,玄宗幸蜀,述抱《國史》藏於南山,經籍資產,焚剽殆盡。述亦陷於賊庭,授偽官。至德二年,收兩京,三司議罪,流于渝州,為刺史薛舒困辱,不食而卒。其甥蕭直為太尉李光弼判官,廣德二年,直因入奏言事稱旨,乃上疏理述於蒼黃之際,能存《國史》,致聖朝大典,得無遺逸,以功補過,合霑恩宥。乃贈右散騎常侍。



 議者云自唐已來,氏族之盛,無逾於韋氏。其孝友詞學,承慶、嗣立為最;明於音律,則萬石為最;達於禮義,則叔夏
 為最;史才博識,以述為最。所撰《唐職儀》三十卷、《高宗實錄》三十卷、《御史臺記》十卷、《兩京新記》五卷,凡著書二百餘卷;皆行於代。



 逌,學業亦亞於述,尤精《三禮》,與述對為學士,迪,同為禮官,時人榮之。累遷考功員外郎、國子司業,以風疾卒。



 蕭穎士者,聰人雋過人,富詞學,有名於時,賈曾、席豫、張垍及述皆引為談客。開元二十三年登進士第,考功員外郎孫逖稱之於朝。褊躁無威儀,與時不偶,前後五授官,旋即駁落。乾元初,終於揚府功曹。



 述在秘
 閣時,與鄠縣尉母煚、曹州司法殷踐猷並友善,二人相次卒。踐猷,申州刺史仲容從子,明《班史》,通於族姓。子寅,有至性,早孤,事母以孝聞。應宏詞舉,為永寧尉。



 史臣曰:前代文學之士,氣壹矣,然以道義偶乖,遭遇斯難。馬懷素、褚無量好古嗜學,博識多聞,遇好文之君,隆師資之禮,儒者之榮,可謂際會矣。劉、徐等五公,學際天人,才兼文史,俾西垣、東觀,一代粲然,蓋諸公之用心也。然而子玄鬱結於當年,行沖徬徨於極筆,官不過俗吏,
 寵不逮常才,非過使然,蓋此道非趨時之具也,其窮也宜哉!



 贊曰:學者如市,博通甚難;文士措翰,典麗惟艱。馬、褚、兢、術,徐、元、子玄,文學之書,胡寧比焉!



\end{pinyinscope}