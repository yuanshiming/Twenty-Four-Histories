\article{卷三十四 志第十四 歷三}

\begin{pinyinscope}

 開元《大衍歷經》



 演紀上元閼逢困敦之歲,距今開元十二年甲子歲,歲積九千六百六十六萬一千七百四
 十算。



 大衍步中朔第一



 大衍通法:三千四十。



 策實:一百一十一萬三百四十三。



 揲法:八萬九千七百七十三。



 滅法:九萬一千三百。



 策餘:一萬五千九百四十三。



 用差:一萬七千一百二十四。



 掛限:八萬七千一十八。



 三元之策:一十五;餘,六百六十四;秒,七。



 四象之策:二十九;餘,一千六百一十三。



 中盈分:一千三百二十八;秒,十四。



 爻數:六十。



 象統:二十四。



 推天正中氣以策實乘入元距所求積算,命曰中積分。盈大衍通法得一,為積日。不盈者,為小餘。爻數去積日,不盡日為大餘。數從甲子起算外,即所求年天正中氣冬至日
 及小餘也。



 求
 次氣因天正中氣大小餘,以三元之策及餘秒加之。其秒盈象統,從小餘。小餘滿大衍通法,從大餘。大餘滿爻數,去之。命如前,即次氣恆日及餘秒。凡率相因加者,下有餘秒,皆以類相從。而滿其法,則迭進之,用加上位。日盈爻數,去之也。



 推天正合朔以揲法去中積分。其所不盡,曰歸餘之卦。以減積
 積分,餘為朔積分。乃如大衍通法而一,為日。不盡,為小餘。日盈爻數,去之。不盈者,為大餘。命以甲子算外,即所求年天正合朔經日及小餘也。



 求次朔及弦望因天正經朔大小餘,以四象之策及餘加之。數除如法,即次朔經日及餘也。又自經朔加一象之日七及餘一千一百六十三少,得上弦。倍之,得望。參之,得下弦。四之,是謂一揲,復得後月之朔。凡四分一為少,二為半,三為太,四為全。加滿其前數,去之,從上位。綜中朔盈虛分,累益歸餘之卦,每其月閏衰。凡歸餘之卦五萬六千七百六
 十以上,其歲有閏。因考其閏衰,滿卦限以上,其月及合置閏。或有進退,皆以定朔無中氣裁焉。



 推沒日置有沒之氣恆小餘,以象統乘之,內秒分,參而伍之,以減策實。餘滿策餘,為日。不滿,為沒餘。命起也。凡恆氣小餘,不滿大衍通法,如中盈分半法已下,為有沒之氣。



 推滅日以有滅之朔經小餘,減大衍通法。餘,倍參伍乘之,用減滅法。餘,滿朔虛分,為日。不滿,為滅餘。命起經朔初日算外,即合朔後滅日也。凡經朔小餘不滿朔虛分者,為有滅之朔。



 大衍步發斂術第二



 天中之策:五;餘,二百二十二;秒,三十一。秒法:七十二。



 地中之策:十八;餘,一百六十五;秒,八十六。秒法:一百二十。



 貞晦之策:三;餘,一百三十二;秒,一百三。秒法:如前。



 辰法:七百六十。



 刻法:三百四。



 推七十二候各因中節大小餘命之,即初候日也。以天中之策及餘秒加之,數除如法,即次候日。又加,得末候日。凡發斂,皆以恆氣。



 推六十卦各因中氣大小餘命之,公卦用事日也。以地
 之策及餘秒累加之,數除如法,各次卦用事日。若以貞晦之策加諸候卦,得十二節之初外卦用事日。



 推五行用事各因四立大小餘命之,即春木、夏火、秋金、冬水首用事日也。以貞晦之策及餘秒,減四季中氣大小餘,即其月土始用事日。凡抽加減而有秒者,母若不齊,當令母互乘子。乃加減之。母相乘為法。



 推發斂去朔各置其月閏衰,以大衍通法約之,為日。不盡為餘,即其月中氣去經朔日算及餘秒也。求卦候者,各以天地之策及餘秒
 累
 加減之,中氣之前以減,中氣之後以加。得去經朔日算及餘秒。



 推發斂加時各置其小餘,以六爻乘之,如辰法而一,為半辰之數。不盡者,五之,三刻法除之,為刻。又不盡者,三約為分。此分滿刻法為刻,若令滿象積為刻者,即置不盡之數,十之,十九而一,為分。命起子半算外,各其加時所在辰刻及分也。



 大衍步日躔術第三



 乾實:一百一十一萬三百七十九太。周天度:三百六十五。虛分七百七十九太。



 歲差:三十六太。



 求每日先後定數以所入氣並後氣盈縮分,倍六爻乘之,綜兩氣辰數除,入之,為末率。又列二氣盈縮分,皆倍六爻乘之,各如辰數而一,以少減多,餘為氣差。加減末率,至後以差加,分後以差減。為初率。倍氣差,亦六爻乘之,復綜兩氣辰數以除之,為日差。半之,以加減初末,各為定率。以日差累加減氣初定率,至後以差減,
 分後以差加。為每日盈縮分。乃馴積之,隨所入氣日加減氣下先後數,各其日定。冬至後為陽復,在盈加之,在
 縮減之。夏至後為陰復,在縮加之,在盈減之。距四正前一氣,
 在陰陽變革之際,不可相並,皆因前末為初率。以氣差至前加之,分前減之,為末率。餘依前率,各得所求。其朓朒亦放此求之,各得每日定數。其分不滿全數,母又每氣不同,當退法除之,用百為母,半已上從一,已下棄之。下求軌漏,餘分不滿準此。



 推二十四氣定日冬夏至皆在天地之中,無有盈縮。餘各以氣下先後數,先減後加恆氣小餘。滿若不足,進退
 其日。命從甲子算外,各其定日及餘秒也。凡推日月行度及軌漏交蝕,並依定氣。若注歷即依恆氣也。



 推平朔四象以定氣相距置朔弦望經日大小餘,以所入定氣大小餘及秒分減之,各其所入定氣日算及餘秒也。若大餘少不足減者,加爻數,然後減之。其弦望小餘有少半太,當以爻乘之,乃以氣秒分減,退一加象統。小餘不足減,退日算一,加大衍通法也。



 求朔弦望經日入朓朒各置其所入定氣日算及餘秒。減日算一,各以日差乘而半之,以加減其氣初定率,前少,加之;前多,減之。以乘
 其所入定氣日算及餘秒。凡除者,先以母通全,內子,乃相乘,母相乘除之也。若忽微之數煩多而不甚相校者,過半收為全,不盈半法,棄之。所得以損益朓朒積,各為其日所入朓朒定數。若非朔望有交者,以十二乘所入日算。三其小餘,辰法除而從之。以乘損益率,如定氣辰數而一。所得以損益朓朒積,各為定數也。



 赤道宿度



 右北方七宿九十八度
 虛分七百七十九太



 右西方七宿八十一度



 右東方七宿七十五度



 前皆赤道度。其畢、觜、參及輿鬼四宿度數,與古不同,今並依天以儀測定,用為常數。紘帶天中,儀極攸憑,以格黃道也。推黃道,準冬至歲差所在,每距冬至前後各五
 度為限。初數十二,每限減一,盡九限,數終於四。殷二立之際,一度少強,依平。乃距春分前、秋分後,初限起四,每限增一,盡九限,終於十二,而黃道交復。計春分後、秋分前,亦五度為限,初數十二,盡九限,數終於四。殷二立之際,一度少強,依平。乃距夏至前後,初限起四,盡九限,終於十二。皆累裁之,以數乘限度,百二十而一,得度。不滿者,十二除為分。若以十除,則大分。十二為母,命以太半少及強弱。命曰黃赤道差數。二至前後,各九限,以差減赤道度,為黃道度。二分
 前後,各九限,以差加赤道度,為黃道度。若從黃道度反推赤道,二至前後各加之,二分前後須減之。



 黃道宿度



 右北方九十七度六虛之差十九太



 右西方八十二度半



 右南方一百一十度半



 右東方七十五度少



 前皆黃道度。其步日行月與五星出入,循此。求此宿度,皆有餘分。前後輩之成少、半、太,準為全度。若上考古下驗將來,當據歲差。每移一度,各依術算,使得當時宿度及分,然可步日月五星,知其犯守也。



 推日度以乾實去中積分。不盡者,盈大衍通法為度。不滿,為度餘。命起赤道虛九,去分。不滿宿算外,即所求年
 天正冬至加時日所在度及餘也。以三元之策累加之,命宿次如前,各得氣初日加時赤道宿度。



 求黃道日度以度餘減大衍通法。餘以冬至日躔之宿距度所入限乘之,為距前分。置距度下黃赤道差,以大衍通法乘之,減去距前分。餘,滿百二十除,為定差。不滿者,以象統乘之。復除,為秒分。乃以定差及秒減赤道宿度。餘,依前命之,即天正冬至加時所在黃道宿度及餘也。



 求次定氣置歲差,以限數乘之,滿百二十除,為秒分。
 不盡為小分。以加於三元之策秒分,因累而裁之,命以黃道宿次去之,各得定氣加時日躔所在宿及餘也。



 求定氣初日夜半日所在度各置其氣定小餘,副之,以乘其日盈縮分,滿大衍通法而一,盈加縮減其副,用減其日時度餘,命如前,各其日夜半日躔行在。求次日,各因定氣初日夜半度,累加一策,乃以其日盈縮分,盈加縮減度餘,命以宿次,即半日所在度及餘也。



 大衍步月離術第四



 轉終分:六百七十萬一千二百七十九。



 轉終日:二十七;餘,一千六百八十五;秒,七十九。



 轉法:七十六。



 轉秒法:八十。



 推天正經朔入轉以轉終分去朔積分,不盡,以秒法乘,盈轉終分又去之,餘如秒法一而入轉分。不盡為秒。入轉分滿大衍通法,為日。不滿為餘。命日算外,即所求年天正經朔加時入轉日及餘秒。



 求次朔入轉因天正所入轉差日一、轉餘二千九百六十七、秒分一,盈轉終日
 餘秒者去之。數除如前,即次日經朔加時所入。考上下弦望,如求經朔四象術,循變相加,若以經朔望小餘減之,各其日夜半所入轉日及餘秒。



 求朔弦望入朓朒定數各朔其所入日損益而半之,為通率。又二率相減為率差。前多者,以入餘減大衍通法,餘乘率差,盈大衍通法得一,並率差而半之。前少者,半入餘,乘率差,亦以大衍通法除之,為加時轉率。乃半之,以損益加時所入,餘為轉餘。其轉餘,應益者,減法;應損者,因餘。皆以乘率差,盈大衍通法得一,加於通率。轉率乘之,大衍通法約之,以朓減朒加轉率為定率。乃以定率損益朓朒積為定數。其後無同率者,亦因前率,益者以通率為初數,半率差而減之。應通率,其損益入餘,進退
 日者,分為二日,隨餘初末如法求之,所得並以損益轉率。此術本出《皇極歷》,以究算術之微變。若非朔望有交者,直以入餘乘損益,如大衍通法而一,以損益朓朒為定數,各得所求。



 七日初:二千七百一,約為大分八。末:三百三十九,約為大分一。



 十四日初:二千三百六十三,約為大分七。末:六百七
 十七,約為大分二。



 二十一日初:二千二十四,約為大分六。末:一千一十六,約為大分三。



 二十八
 日初:一千六百八十六,約為大分五。末:一千三百五十四,
 約為
 大分四。



 右以四象約轉終日及餘,均得六日二千七百一分。就全數約為大分,是為之八分。以減法,餘為末數。乃四象馴變相加,各其所當之日初末數也。視入轉餘,如初數以下者,加減損益,因循前率;如初數以上,則反其衰,歸於後率雲。



 求朔弦望定日及餘以入氣、入轉朓朒定數,同名相從,異名相消。乃以朓減朒加四象經小餘。滿若不足,進大餘。命以甲子算外,各其定日及小餘。幹名與後朔葉同者,月大。不同者,小;無中氣者,為閏月。凡言夜半者,皆起晨前子正之中。若注歷觀弦望定小餘,不盈晨初餘數者,退一日。其望,小餘雖滿此數,若有交蝕,虧初起在晨初已前者,亦如之。又月行九道遲疾,則三大二小。以日行盈縮,累增損之,則容有四大三小,理數然也。若俯循常儀,當察加時早晚,隨其所近而進退之,使不過三小。其正月朔,若有交加時正見者,消息前後一兩月,以定大小,令虧在晦
 二。



 推定朔弦望夜半日所在度各隨定氣次日以所直日度及餘分命焉。若以五星相加減者,以四約度餘。乃列朔弦望小餘,副之,以乘其日盈縮分,如大衍通法而一,盈加縮減其副,以加其日夜半度餘,命如前,各其日加時日躔所次。



 推月九道度凡合朔所交,冬在陰歷,夏在陽歷,月行青道。冬、夏至後,青道半交在春分之宿,殷黃道東。立冬、夏後,青道半交在立春之宿,殷黃道東南。至所沖之宿亦如之也。冬在陽歷,夏在陰歷,月行白道。冬至夏至後,白道半交在秋分之宿,殷黃道西。立北。至所沖之宿亦如之也。春在陽歷,秋
 在陰歷,月行硃道。春、秋分後,硃道半交在夏至之宿,殷黃道南。立春立秋後,硃道半交在立夏之宿,殷黃道西南。至所沖之宿亦如之也。春在陰歷,秋在陽歷,月行黑道。春、秋分後,黑道半交在冬至之宿,殷黃道北。立春立秋後,黑道半交在立冬之宿,殷黃道東北。至所沖之宿亦如之也。四序離為八節,至陰陽之始交,皆以黃道相會,故月有九行。各視月交所入七十二候,距交初黃道日每五度為限。交初交中同。亦初數十二,每限減一,數終於四,乃一度強,依平。更從四起,每限增一,終於十二,而至半交,其去黃道六度。又自十二,每限減一,數終於四,亦一度強,依平。
 更從四起,每限增一,終於十二,復與日軌相會。各累計其數,以乘限度,二百四十而一,得度。不滿者,二十四除,為分。若以二十除之,則大分。十二為母,命以半太及強弱也。為月行與黃道差數。距半交前後各九限,以差數為減;距正交前後各九限,以差數為加。此加減是出入六度,單與黃道相交之數也。若交赤道,則隨氣遷變不恆。計去冬至夏至以來候數,乘黃道所差,十八而一,為月行與赤道差數。凡日以赤道內為陰,赤道外為陽;月以黃道內為陰,黃道外為陽。故月行宿度入春
 分交後行陰歷,秋分交後行陽歷,皆為同名;若入春分交後行陽歷,秋分交後行陰歷,皆為異名。其在同名,以差數為加者加之,減者減之;若在異名,以差數為加者減之,減者加之。皆以增損黃道度為九道定數。



 推月九道平交入氣各以其月恆中氣,去經朔日算及餘秒,加其月經朔加時入交泛日及餘秒,乃以減交終日及餘秒,其餘即各平交入其月恆中氣日算及餘秒也。滿三元之策及餘秒則去之,其餘即平交入後月恆
 節氣日算及餘秒。因求次交者,以交終日及餘秒加之。滿三元之策及餘秒,去之。不滿者,為平交入其氣日算及餘秒。各以其氣初先後數先加、後減其入餘。滿若不足,進退日算,即平交入定氣日算及餘秒也。



 求平交入氣朓朒定數置所入定氣日算,倍六爻乘之,三其小餘,辰法除而從之,以乘其氣損益率,如定氣辰數而一,所得以損益其氣朓朒積為定數也。



 求平交入轉朓朒定數置所入定氣餘,加其日夜半入轉餘,以乘其日損益率,滿大衍通法而一,所得以損益
 其日朓朒積,乃以交率乘之,交數而一,為定數。



 求正交入氣置平交入氣及入轉朓朒定數,同名相從,異名相消。乃以朓減、朒加平交入氣餘,滿若不足,進退日算,即為正交入定氣日算及餘也。



 求正交加時黃道宿度置正交入定氣餘,副之,乘其日盈縮分,滿大衍通法而一,所得以盈加縮減其副,以加其日夜半日度,即正交加時所在黃度及餘也。



 求正交加時月離九道宿度以正交加時度餘,減大衍
 通法。餘以正交之宿距度所入限數乘之,為距前分。置距度下月道與黃道差,以大衍通法乘之,減去距前分,餘滿二百四十除,為定差。不滿者,一退為秒。以定差及秒加黃道度,餘,仍計去冬至夏至以來候數,乘定差,十八而一,所得依名同異而加減之,滿若不足,進退其度,命如前,即正交加時月離所在九道宿度及餘也。



 推定朔弦望加時月所在度各置其日加時日躔所在,變從九道,循次相加。凡合朔加時月行潛在日下,與太
 陽同度,是為離象。凡置朔弦望加時黃道日度,以正交加時所在黃道宿度減之,餘以加其正交九道宿度,命起正交宿度算外,即朔弦望加時所當九道宿度也。其合朔加時若非正交,則日在黃道,月在九道,各入宿度,雖多少不同,考其去極,若應準繩,故云月行潛在日下,與太陽同度。



 以一象之度九十一、餘九百五十四、秒二十二半為上弦,兌象。倍之而與日沖,得望,坎象。參之,得下弦,震象。各以加其所當九道宿度,秒盈象統從餘,餘滿大衍通法從度。命如前,各其日加時月所在度及餘秒也。綜五位成數四十,以約度餘,為分。不盡者,因為小分也。



 推定朔夜半入轉恆視經朔夜半所入,若定朔大餘有進退者,亦加減轉日,否則因經朔為定。徑求次定朔夜半入轉,因前定朔夜半所入,大月加轉差日二,小月加日一,轉餘皆一千三百五十四秒分一。數除如前,即次月定朔夜半所入。



 求次日累加一日,去命如,各其夜半所入轉日及餘秒。



 求每日月轉定度各以夜半入轉餘,乘列衰,如大衍通法而一,所得以進加退減其日轉分,為月每所轉定分,
 滿轉法為度也。



 求朔弦望定日前夜半月所在度各半列衰,減轉分。退者,定餘乘衰,以大衍通法除,並衰而半之;進者,半定餘乘衰,定以大衍通法除,皆加所減。乃以定餘乘之,盈大衍通法得一,以減加時月度及分。因夜半準此求轉分以加之,亦得加時月度。若非朔望有交,直以定小餘乘所入日轉交分,如大衍通法而一,以減其日時月度,亦得所求。



 求次日夜半月度各以其日轉定分加之,分滿轉法從度,命如前,即次日夜半月所在度及分。



 推月晨昏度各以所入轉定分乘其日夜漏,倍百刻除,為晨分。以減轉定分,餘為昏分。分滿轉法,從度。以加夜半度,望前以昏加,望後以晨加。各得其日晨昏月所在度及分。



 大衍步軌漏第五



 爻統:一千五百二十。



 象積:四百八十。



 辰刻:八;刻分,一百六十。



 昏明刻:各二;刻分,二百四十。



 求每日消息定衰各置其氣消息衰,依定氣日數,每日以陟降率陟減降加其分,滿百從衰,不滿為分。各得每日消息定衰及分。其
 距二分前後各一氣之外,陟降不等,各每以三日為一限,損益如後。



 雨水初日:降七十八。初限每日損十二,次限每日損八,次限每日損三,次限每日損二,末限每日損一。



 清明初日:陟一。初限每日益一,次限每日益二,次限每日益三,次限每日益八,末限每日益十九。



 處暑初日:降九十九。
 初限每日損十九,次限每日損八,次限每日損三,次限每日損二,末限每日損一。



 寒露初日:陟一。初限每日益一,次限每日益二,次限每日益三,次限每日益八,末限每日益十二。



 求前件四氣置初日陟降率,每日依限次損益之,各為每日率。乃遞以陟減降加其氣初日消息衰分,亦得每日定衰及分也。



 推戴日之北每度晷數南方戴日之下,正中無晷。自戴日之北一度,乃初數一千三
 百七十九。從此起差,每度增一,終於二十五度。又每度增二,終於四十度。又每度增六,終於四十
 四度,增六十八。每度增二,終於五十五度。又每度增十九,
 終於六十度,度增一百六十。又每度增三十三,終於六十五度。又每度增三十六,終於七十度。又每度增三十九,終於七十二度,增二百六十。又度增四百四十,又度增一千六十,又度增
 一千八百六十,又度增二千八百四十,又度增四千,又度增五千三百四十,而各為每度差。因累其差以遞加初數,滿百為分,分滿十為寸,
 各為每度晷差。又每度晷差數。



 求陽城日晷每日中常數各置其氣去極度,以極去戴日下度五十六,盈分八十二減半之,各得戴日之北度數及分。各以其消息定衰戴日北所直度分之晷差,滿百為分,分滿十為寸,各為每日晷差。乃遞以息減消加其氣初晷數,得每日中晷常數也。



 求每日中晷定數各置其日所在氣定小餘,以爻統減之,餘為中後分。置前後分,以其日晷差乘之,如大衍通法而一,為變差。乃以變差加減其日中晷常數,冬至後,中前以差減,中後以差加。夏至後,中前以差加,中後以差減。冬至一日有減無加,夏至一日有加無減。各得每日中晷定數。



 求
 每日夜半漏定數置消息定衰,滿象積為刻,不滿為分。各遞以息減消加其氣初夜半漏,各得每日夜半漏定數。



 求晨初餘數置夜半定漏全刻,以九千
 一百二十乘之,十九乘刻分從之,如三百而一,所得為晨初餘數,不盡為小分。



 求每日
 晝夜漏及日出入所在辰刻各倍夜半之漏,為
 夜刻。以減百刻,餘為晝刻。減晝五刻以加夜,即晝為見刻,夜為沒刻。半沒刻以半辰刻加之,命起子初刻算外,即日出辰刻。以見刻加之,命如前,即日入辰刻。置夜刻以五除之,得每更差刻,又五除之,得每籌差刻。以昏刻加日入辰刻,得甲夜初刻。又以更籌差加之,得次更一籌之數。以次累加,滿辰刻去之,命如前,即得五夜更籌所當辰及分也。其夜半定漏,亦名晨初夜刻。



 求每日黃道去極定數置消息定衰,滿百為度,不滿為分,各遞以息減消加其氣初去極度,各得每日去極定
 數。



 求每日距中度定數置消息定衰,以一萬二千三百八十六乘之,如一萬六千二百七十七而一,為每日度差。差滿百為度,不滿為分。各遞以息加消減其氣初距中度,各得每日距中度定數。倍距中度以減周天度,五而一,所得為每更度差。



 求每日昏明及每更中宿度所臨置其日所在赤道宿度,以距中度加之,命宿次如前,即得其日昏中所臨宿度。以每更差度加之,命如前,即乙夜初中所臨宿度及分也。



 求九服所在每氣初日中晷常數置氣去極度數相減,各為生氣消息定數,因測所在冬夏至日晷長短,但測至即得,不必要須冬至。於其戴日之北度及分晷數中,校取長短,同者便為所在戴日北度數及分。氣各以消定數加減之,因冬至後者每氣以減,因夏至後者每氣以加。各得每氣戴日北度數及分。各因其氣所直度分之晷數長短,即各為所在每定氣初日中晷常數。其測晷有在表南者,亦據其晷尺寸長短,與戴日北每度晷數同者,因取其所直之度,去戴日北度數,反之,為去戴日南度,然後以消息定數加減。



 求九服所在晝夜漏刻冬夏至各於所在下水漏,以定當處晝夜刻數。乃相減,為冬夏至差刻。半之,以加減二至晝夜刻數,加夏至、減冬至。為春秋分定日晝夜刻數。乃置每氣消息定數,以當處二至差刻數乘之,如二至去極差度四十七分,八十而一,所得依分前後加減二分初日晝夜漏刻,春分前秋分後,加夜減晝;春分後秋分前,加晝減夜。各得所在定氣初日晝夜漏刻數。求次日者,置每日消息定衰,亦以差刻乘之,差度而一,所得以息減消加其氣初漏刻,各得所求。其求
 距中度及昏明中宿日出入所在,皆依陽城法求,仍以差度而今有之,即得也。



 又術置所在春秋分定日中晷常數,與陽城每日晷數校取同者,因其日夜半漏,即為所在定春秋分初日夜半漏。求餘氣定日,每以消息定數,依分前後加減刻分。春分前以加,分後以減;秋分前以減,分後以加。滿象積為刻,不滿為分,各為所在定氣初日夜半定漏。



 求次日以消息定衰依陽城法求之,即得。此術究理,大體合通。但高山平川,視日不等。校其日晷,長短乃同。考其日漏,多少懸別。以茲參課,前術為審也。



 大衍步交會術第六



 交終:八億二千七百二十五萬一千三百二十二。



 交中:四萬一千三百六十二;秒,五千六百六十一。



 終日:二十七;餘,六百四十五;秒,一千三百二十二。



 中日:十三;餘,一千八百四十二;秒,五千六百六十一。



 朔差日:二;餘,九百六十七;秒,八千六百七十八。



 望差日:一;餘,四百八十三;秒,九千三百三十九。



 望數日:十四;餘,二千三百二十六;秒,五十。



 交限日:十二;餘,一千三百五十八;秒,六千三百二十二。



 交率:三百四十三。



 交數:四千三百六十九。



 辰法:七百六十。



 秒分法:一萬。



 推天正經朔入交以交終去朔積分,不盡,以秒分法乘。盈交終,又去之。餘如秒法而一,為入交分。不盡,為秒。入交分滿大衍通法,為日;不滿,為餘。命日算外,即所求年天正經朔加時入交泛日及餘秒。



 求次朔入交因天正所入,加朔差日及餘秒,盈終日及餘秒者,去之。數除如前,
 即次月經朔加時所入。



 求望以望數日及餘秒加之,去命如前,即得所求。若以經朔望小餘減之,各其日夜半所入交泛日及餘秒。



 求定朔夜半入交恆視經朔望夜半所入,定朔望大餘。有進退者,亦加減交日。否則,因經為定,各得所求。求次定朔夜半入交:因前定朔夜半所入,大月加交差日二,月小加日一,餘皆二千三百九十四、秒八千六百七十八。求次日:累加一百,數除如前,各其夜半所入交泛日
 及餘秒。



 求朔望入交常日各以其日入氣朓朒定數,朓減朒加其入交泛,餘滿大衍通法從日,即為入交常及餘秒。



 求朔望入交定日各置其日入轉朓朒定數,以交率乘之,如交數而一。所得以朓減朒加入交常,餘數如前,即為入交定日及餘秒。



 求月交入陰陽歷恆視其朔望入交定日及餘秒,如中日及餘秒已下者,為月入陽歷,已上者,以中日及餘秒
 去之,餘為月入陰歷。



 求四象六
 爻每度加減分及月去黃道定數以其爻加減率與後爻加減率相減,為前差。又以後爻率與次後
 爻率相減,為後差。二差相減,為中差。置所在爻並後爻加減率,半中差以加而半之,十五而一,為爻末率,國為後爻初率。每以本爻初末率相減,為爻差。十五而一,為度差。半之,以加減初率,少象減之,老象加之。為定初率。每次度差累加減之,少象以差減,老象以差加。各得每度加減定分。乃修積其分,滿百二十為度,各為每度月去黃道度數及分。其四象,初爻無初率,上爻無末率,皆倍本爻加減率,十五而一。所得各以初末率減之,皆互得其率。餘依術算,各得所求。



 求朔望夜半月行入陰陽度數各置其日夜半入轉日
 及餘秒,餘以其日夜半入交定日及餘秒減之也,其秒母不等,當循率相通,然後減之,如不足減,即轉終日及一餘秒,然後減之。餘為定交初日夜半入轉日及餘秒。乃以定交初日夜半入餘與其日夜半入餘,各乘其日轉定分,如大衍通法而一。所得滿轉法為度,不滿為分。各以加其日轉積度及分,乃相減,其餘即為其夜半月行入陰陽度數及分也。轉求次日,但以其日轉定分加之,滿轉法為度,即得。



 求朔望夜半月行入四象度數置其日夜半入陰陽度
 數及分,以一象之度九十除之。若以小象除之,則兼除差度一、度分一百六、大分十三、小分十四,訖,然以次象除之。所得以少陽、老陽、少陰、老陰為次,命起少陽算外,即其日夜半所入象度數及分也。先以三十乘陰陽度分,十九而一,為度分。乘又除,為小分。然以象度及分除之。



 求朔望夜半月行入六爻度數置其日夜半所入象度數及分,以一爻之度一十五除之。所得命起其象初爻算外,即以其日夜半所入爻度數及分也。其月行入少象初爻之內,皆為沾近黃道度。當朔望則有虧蝕。求入蝕限:其入交定日及餘秒,如望
 差已下交限已上者,為入蝕限。望入蝕限,則月蝕;朔入蝕限,月在陰歷則日蝕。入限,如望差已下,為交後。交限已上者,以減中日及餘,為交前。置交前後定日及餘秒通之,為去交前後定分。置去交定分,以十一乘之,如二千六百四十三除之,為去交度數。不盡,以大衍通法乘之,復除為餘。大抵去交十三度以上,雖入蝕限,為涉交數微,光影相接,或不見蝕。



 求月蝕分其去交定分七百七十九已下者,皆蝕既。已上者,以交定分減望差,餘以一百八十三約之。盡半已
 下,為半弱;已上,為半強。命以十五為限,得月蝕之大分。



 求月蝕所起月在陰歷,初起東南,甚於正南,復於西南。月在陽歷,初起東北,甚於正北,復於西北。其蝕十二分已上者,皆起於正東,復於正西。此皆據南方正午而論之,若蝕於餘方者,各隨方面所在,準此取正,而定其蝕起復也。



 求月蝕用刻置月蝕之大分。五已下,因增三。十已下,因增四。十已上,因增五。其去交定分五百二十已下,又增半。二百六十已下,又增半。各為泛用刻率。



 求每日差積定數以所入氣並後氣增損差,倍六爻乘之,綜兩氣辰數除之,為氣末率。又列二氣增損差,皆倍六爻乘之,各如辰數而一。少減多,餘為氣差。加減末
 率,冬至後以差減,夏至後以差加。為初率。倍氣差,亦倍六爻乘之,復綜兩氣辰數以除之,為日差。半之,以加減初末,各為定率。以日差累加減氣初定率,冬至後以差加,夏至後以差減。為每日增損差。乃循積之,隨所入氣日加減氣下差積,各其日定數。其二至之前一氣,皆後無同差,不可相並,各因前末為初率。以氣差冬至前減,夏至前加,為末率。餘依算術,各得所求也。



 陰歷:



 蝕差:一千二百七十五。



 蝕限:二千五百二十四。



 或限:三千六百五十九。



 陽歷:



 蝕限:一百三十五。



 或限:九百七十四。



 求蝕差及諸限定數各置其差、限,以蝕朔所入氣日下差積,陰歷減之,陽歷加之,各為蝕定差及定限。



 求陰歷陽歷的蝕或蝕其陰歷去交定分滿蝕定差已上,為陰歷蝕。不滿者,雖在陰歷,皆類同陽歷蝕也。其去交定分滿蝕定限已下者,其蝕的見。或限以下者,其蝕
 或見或不見。



 求日蝕分陰歷蝕者,置去交定分,以蝕定差減之,餘一百四已下者,皆蝕既。已上者,以一百四減之,其餘以一百四十三約之,其入或限者,以一百五十二約之。半已下為半弱,半已上為半強,以減十五,餘為日蝕之大分。其同陽歷蝕者,但去交定分,少於蝕定差六十已下者,皆蝕既。六十已上者,置去交定分,以陽歷蝕定限加之,以九十約之。其陽歷蝕者,直置去交定分,亦以九十約之。其入或限者,以一百四十三約
 之。半已下為半弱,半已上為半強,命以十五為限,亦得日蝕之大分。



 求日蝕所起月在陰歷,初起西北,甚於正北,復於東北。月在陽歷,初起西南,甚於正南,復於東南。其蝕十二分已上,皆起正西,復於正東。此亦據南方正午而論之。



 求日蝕用刻置所蝕之大分,皆因增二。其陰歷去交定分多於蝕定差七十已上者,又增三十五;已下者,又增半。其同陽歷去交定分少於蝕定差二十已下者,又增
 半;四十已下者,又增半少。各為泛月刻半率。



 求日月蝕甚所在辰置去交定分,以交率乘之,二十乘交數除之,所得為差。其月道與黃道同名者,以差加朔望定小餘;異名,以差減朔望定小餘,置餘定餘。如求發斂加時術入之,即蝕甚所在辰刻及分也。其望甚辰月當沖蝕。



 求虧初復末置日月蝕泛用刻率,副之,以乘其日入轉損益率,如大衍通法而一。所得應朒者,依其損益;應朓者,損加益減其副,為定用刻數。半之,以減蝕甚辰刻,為虧初;以加蝕
 甚辰刻,為復末。其月蝕求入更籌者,置月蝕定用刻數,以其日每更差刻除,為更數;不盡,以每籌差刻除,為籌數。綜之為定用更籌。乃累計日入至蝕甚辰刻置之,以昏刻加日入辰刻減之,餘以更籌差刻除之。所得命以初更籌外,即蝕甚籌。半定用更籌減之,為虧初;以加之,為復末。按天竺僧俱摩羅所傳斷日蝕法,其蝕朔日度躔於鬱車宮者,的蝕。諸斷不得其蝕,據日所在之宮,有火星在前三後一之宮並伏在日下,並不蝕。若五星總出,並水見,又水在陰歷,及三星已上同聚一宿,亦不蝕。凡星與日別宮或別宿則易斷,若同宿則難斷。更有諸斷,理多煩碎,略陳梗概,不復具詳者。其天竺所云十二宮,則中國之十二次也。曰鬱車宮者,即中國降婁之次也。十二次宿度,首尾具載「歷儀分野」卷中也。



 求九服所在蝕差先測所在冬、夏至及春分定日中晷
 長短、陽城每日中晷常數,校取同者,各因其日蝕差,即為所在冬、夏至及春秋分定日蝕差。



 求九服所在每氣蝕差以夏至差減春分差,以春分差減冬至差,各為率。並二率半之,六而一,為夏率。二率相減,六一為差。置總差,六而一,為氣。半氣差,以加夏率,又以總差減之,為冬率。冬率即是冬至之率也。每以氣差加之各氣,為每氣定率。乃循其率,以減冬至蝕差,各得每氣初日蝕差。求每日,如陽城求之,若戴日之北,當計其所在,皆反之,即得。



 大衍步五星術第七



 歲星



 終率:一百二十一萬二千三百七十九;秒,十八。



 終日:三百九十八;餘,二千六百五十九;秒,六。



 變差算:空;餘,三十四;秒,十四。



 象算:九十一;餘,二百三十八;秒,五十七十二。



 爻算:十五;餘,一百六十六;秒,四十六十二。



 鎮星



 終率:一百一十四萬九千三百九十九;秒,九十八。



 終日:三百七十八;餘,二百七十九;秒,九十八。



 變差算:空;餘,二十二;秒,九十二。



 象算:九十二;餘,二百三十七;秒,八十七。



 爻算:十五;餘,一百六十六;秒,三十一。



 太白



 終率:一百七十七萬五千三十;秒,十二。



 終日:五百八十三;餘,二千七百一十一;秒,十二。



 中合日:二百九十一;餘,二千八百七十五;秒,六。



 變差算:空;餘,三十;秒,五十三。



 象算:九十二;餘,二百三十八;秒,三十四五十四。



 爻算:十五;餘,一百六十
 六;秒,三十九九。



 辰星



 終率:三十五萬二千二百七十九;秒,七十二。



 終日:一百一十五;餘,二千六百七十九;秒,七十二。



 中合日:五十七;餘,二千八百五十九;秒,八十六。



 變差算:空;餘,一百三十六;秒,七十八六十。



 象算:九十一;餘,二百四十四;秒,九十八六十。



 爻算:十五;餘,一百六十七;秒,三十九七十四。



 辰法:七百六十。



 秒法:一百。



 微分法:九十六。



 推五星平合置中積分,以天正冬至小餘減之,各以其星終率去之,不盡者,返以減終率,滿大衍通法為日,不滿為餘,即所求年天正冬至夜半後星平合日算及餘秒也。



 求平合入爻象歷置積年,各以其星變以差乘之,滿乾實去之,不滿者,以大衍通法約之,為日。不盡為餘秒。以減其星冬至夜半後平合日算及餘秒,即平合入歷算數及餘秒也。各四約其餘,同其辰法也。



 求平合入四象置歷算數及秒,以一象之算及餘秒除之,所得,依入爻象次命起少陽算外,即平合所入象算數及餘秒也。



 求平合入六爻置所入象算數及餘秒,以一爻之算及餘秒除之,所得,命起其象初爻算外,即平合所入爻算數及餘秒也。



 求四象六爻每算損益及進退定數以所入爻與後爻損益率相減為前差,又以後爻與次後爻損益率相減為後差,前後差相減為中差。置所入爻並後爻損益率,半中差以加之,九之,二百七十四而一,為爻末率,因為後爻初率。皆因前爻末率,以為後爻初率。初末之率相減,為爻差。倍爻差,九之,二百七十四而一為算差。半之,加減初末,各為定
 率。以算差累加減爻初定率,少象以差減,老象以差加。為每損益率。循累其率,隨所入爻,損益其下進退,即各得其算定。其四象初爻無初率,上爻無末率,皆置本爻損益,四而九之,二百七十四而一,各以初末率減之,皆互得其率。餘依術算,各得所求。



 求平合入進退定數各置其星平合所入爻之算差,半之,以減其所入算損益率。損者,以所入餘乘限差,辰法除,並差而半之;益者,半入餘乘差,亦辰法除。加所減之率,乃以入餘乘之,辰法而一,所得以損益其算下進退,
 各為平合所入進退定數。此法微密,用算稍繁。若從省求之,亦可置其所入算餘,以乘其下損益率,如辰法而一,所得以損益其算下進退,各為定數。



 求常合置平合所入進退定數,金星則倍置之。各以合下乘數乘之,除數除之,所得滿辰法為日,不滿為餘,以進加退減平合日算及餘秒,先以四約平合餘,然以進加退減也。即為冬至夜半後常合日算及餘也。



 求定合置常合日先後定數,四而一,所得滿辰法為日,不滿為餘。乃以先減後加常合算及餘,即為冬至夜半後定合日算及餘也。



 求定合度置其日盈縮分,四而一以
 定合餘乘之,滿辰法而一,所得以盈加縮減其定餘,以加其日夜半日度餘,先四約夜半日度餘以加之。滿辰法從度。依前命之算外,即為定合加時度及餘也。



 求定合月日置冬至夜半後定合日算及餘秒,以天正冬至大小餘加之,天正經朔大小餘減之。其至、朔小餘,皆以四約之,然用加減。若至大餘少於經朔大餘者,又以爻數加之,然以經朔大小餘減之。其餘滿四象之策及餘,除之,為月數,不盡者,為入朔日算及餘。命月數起天正日算起經朔算外,即定所在日月也。其定朔大餘有進退,
 進減退加一日,為在其日月定及餘也。



 求定合入爻置常合及定合應加減定數,同名相從,異名相消。乃以加減其平合入爻算餘,滿若不足,進退其算,即為定合入爻算數及餘也。



 求變行初日入爻置定合入爻算數及餘,以合後伏下變行度常率加之,滿爻率去之,命爻次如前,即次變初日入爻算數及餘也。更求次變入爻變入,但以其下行度常加之,去命如上節。



 求變行初日入進退定數各置其變行初日入爻算數
 及餘,如平合求進退術入之,即得變行初日所入進退定數也。置進退定數,各以其下乘數乘之,除數除之,所得各為進退變率。



 求變行日度率置其本進退變率與後變率,同名者,相消為差。在進前少,在退前多,各以差為加;在進前多,在退前少,各以差為減。異名者,相從謂並。前退後進,各以並為加;前進後退,各以並為減。逆行度率則反之。皆以差及並,加減日度中率,各為日度變率。其水星疾行,直以差以並加減度之中率,為變率。其日直因中率為變率,不煩加減也。



 求變行日度定率以定合日與後變初日先後定數,同名相消為差,異名者相從為並。四而一,所得滿辰法為度。乃以盈加縮減其合後伏度之變率及合前伏日之變率。金水夕合日度,加減反之。其二留日之變率,若差於中率者,即以所差之數為度,各加減本遲度之變率。謂以多於中率之數加之,少於中率之數減之。以下加減準此。退行度變率,若差於中率者,即倍所差之數,各加減本疾度之變率。其木土二星,既無遲疾,即加減前後順行度之變率。其水星疾行度之變率,若差於中率者,即以所差之數為
 日,各加減留日變率。其留日變率若少不足減者,即侵減遲日變率也。各加減變率訖,皆為日度定率。其日定率有分者,前後輩之。輩,配也。以少分配多分,滿全為日,有餘轉配。其諸變率不加減者,皆依變率為定率。



 求定合後夜半星所在度置其星定合餘,以減辰法,餘以其星初日行分乘之,辰法而一,以加定合加時度餘,滿辰法為度。依前命之算外,即定合後夜半星所在宿及餘。自此以後,各依其星,計日行度所至,皆從夜半為始也。轉求次日夜半星行至:各以其星一日所行度分,順加退減之。其行有小分
 者,各滿其法從行分一。行分滿辰法,從度一。合之前後,伏不注度,留者因前,退則依減。順行出虛,去六虛之差;退行入虛,先加此差。先置六虛之差,四而一,然用加減。訖,皆以轉法約行分為度分,各得每日所至。其三星之行日度定率,或加或減,益疾益遲,每日漸差,難為預定,今且略據日度中率商量置之。其定率既有盈縮,即差數合隨而增損,當先檢括諸變定率與中率相近者,因用其差,求其初末之日行分為主。自餘變因此消息,加減其差,各求初末行分。循環比校,使際會參合,衰殺相循。其金水皆以平行為主,前後諸變,亦準此求之。其合前伏雖有日度定率,如至合而與後算計卻不葉者,皆從後算為定。其五星初見伏之度,去日不等,各以日度與星度相校。木去日十四度,金十一度,火土水各十七
 度,皆見;各減一度皆伏。其木火土三星前順之初,後順之末,又金水疾行、留、退初末,皆是見伏之初日,注歷消息定之。其金水及日月等度,並棄其分也。



 求每日差置所差分為實,以所差日為法。實如法而一,所得為行分,不盡者為小分。即是也每日差所行分及小分也。其差若全,不用此術。



 求平行度及分置度定率,以辰法乘之,有分者從之,如日定率而一,為平行分。不盡,為小分。其行分滿辰法為度,即是一日所行度及分。



 求差行初末日行度及分置日
 定率減一,以差分乘之。二而一,為差率,以加減平行分。益疾者,以差率減平為初日,加平為末日。益遲者,以差率加平為初日,減平為末日也。加減訖,即是初末日所行度及分。其差不全而與日相合者,先置日定率減一,以所差分乘之,為實。倍所差日為法。實如法而一,為行分。不盡者,因為小分,然為差率。



 求差行次日行度及分置初日行分,益遲者,以每日差減之;益疾者,以每日差加之,即為次日行度及分也。其每日差、初日行皆有小分,母既不同,當令同之。然用加減,轉求次日,準此各得所求也。



 徑求差行餘日行度及分置所求日減一,以每日差乘
 之,以加減初日行分,益遲減之,益疾加之。滿辰法為度,不滿為行分,即是所求日行度及分也。



 求差行,先定日數,徑求積度及分置所求日減一,次每日差乘之,二而一,所得,以加減初日行分。益遲減之,益疾加之。以所求日乘之,如辰法而一,為積度。不盡者,為行分。即是從初日至所求日積度及分也。



 求差行,先定度數,徑求日數置所求行度,以辰法乘之,有分者從之。八之,如每日差而一,為積。倍初日行分,以
 每日差加減之。益遲者加之,益疾者減之。如每日差而一,為率。今自乘,以積加減之,益遲者以積減之,益疾者以積加之。開方除之。所得,以率加減之。益遲者以率加之,益疾者以率減之。乃半之,即所求日數也。其開方除者,置所開之數為實,借一算於實之下,名曰下法。步之,超一位,置商於上方,副商於下法之上,名曰方法。命上商以除實,畢,倍方法一折,下法再折,乃置後商於下法之上,名曰隅法。副隅並方,命後商以除實,畢,隅從方法折下就除,如前開之。訖除,依上術求之即得也。



 求星行黃道南北各視其星變行入陰陽爻而定之。其前變入陽爻為黃道北,入陰爻為黃道南;後變入陽爻為黃道南,入陰爻為黃道北。其金
 水二星,以爻變為前變,各計其變行,起初日入爻之算,盡老象上爻末算之數,不滿變行度常率者,因置其數,以變行日定率乘之,如變行度常率而一,為日。其入變日數,與此日數以下者,星在黃道南北,依本所入陰陽爻為定。過此日數之外者,黃道南北則返之。



\end{pinyinscope}