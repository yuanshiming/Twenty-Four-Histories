\article{卷一百六十四}

\begin{pinyinscope}

 ○韓愈張籍孟郊唐衢李翱宇文籍劉禹錫柳宗
 元韓辭



 韓愈,字退之,昌黎人。父仲卿,無名位。愈生三歲而孤,養於從父兄。愈自以孤子,幼刻苦學儒,不俟獎勵。大歷、貞
 元
 之間,文字多尚古學,效楊雄、董仲舒之述作,而獨孤及、梁肅最稱淵奧,儒林推重。愈從其徒游,銳意鉆仰,欲自振於一代。洎舉進士,投文於公卿間,故相鄭餘慶頗為之延譽,由是知名於時。尋登進士第。



 宰相董晉出鎮大梁,闢為巡官。府除,徐州張建封又請為其賓佐。愈發言真率,無所畏避,操行堅正,拙於世務。調授四門博士,轉監察御史。德宗晚年,政出多門,宰相不專機務。宮市之弊,諫官論之不聽。愈嘗上章數千言極論之,不聽,怒
 貶為連州山陽令,量移江陵府掾曹。



 元和初,召為國子博士,遷都官員外郎。時華州刺史閻濟美以公事停華陰令柳澗縣務,俾攝掾曹。居數月,濟美罷郡,出居公館,澗遂諷百姓遮道索前年軍頓役直。後刺史趙昌按得澗罪以聞,貶房州司馬。愈因使過華,知其事,以為刺史相黨,上疏理澗,留中不下。詔監察御史李宗奭按驗,得澗贓狀,再貶澗封溪尉。以愈妄論,復為國子博士。愈自以才高,累被擯黜,作《進學解》以自喻曰:



 國子先生晨入
 太學,召諸生立館下,誨之曰:「業精於勤,荒於嬉;行成於思,毀於隨。方今聖賢相逢,治具華張。拔去兇邪,登崇俊良。占小善者率以錄,名一藝者無不庸。爬羅剔抉,刮垢磨光。蓋有幸而獲選,孰雲多而不揚?諸生業患不能精,無患有司之不明;行患不能成,無患有司之不公!」



 言未既,有笑於列者曰:「先生欺予哉!弟子事先生,於茲有年矣。先生口不絕吟於六藝之文,手不停披於百家之編。記事者必提其要,纂言者必鉤其玄。貪多務得,細大不
 捐。燒膏油以繼晷,常矻矻以窮年。先生之業,可謂勤矣。牴排異端,攘斥佛、老;補苴罅漏,張皇幽眇;尋墜緒之茫茫,獨旁搜而遠紹;障百川而東之,回狂瀾於既倒。先生之於儒,可謂有勞矣。沉浸醲鬱,含英咀華,作為文章,其書滿家。上規姚、姒,渾渾無涯;《周誥》、《殷盤》,佶屈聱牙;《春秋》謹嚴,《左氏》浮誇;《易》奇而法,《詩》正而葩;下迨《莊》、《騷》,太史所錄,子雲、相如,同工異曲。先生之於文,可謂閎其中而肆其外矣。少始知學,勇於敢為;長通於方,左右具宜。先生
 之於為人,可謂成矣。然而公不見信於人,私不見助於友;跋前躓後,動輒得咎。暫為御史,遂竄南夷;三為博士,冗不見治。命與仇謀,取敗幾時。冬暖而兒號寒,年豐而妻啼饑。頭童齒豁,竟死何裨?不知慮此,而反教人為!」



 先生曰:「籲,子來前!夫大木為杗,細木為桷,MM櫨侏儒,椳闑扂楔,各得其宜,施以成室者,匠氏之工也。玉札丹砂,赤箭青芝,硃溲馬勃,敗鼓之皮,俱收並蓄,待用無遺者,醫師之良也。登明選公,雜進巧拙,紆餘為妍,卓犖為傑,校
 短量長,唯器是適者,宰相之方也。昔者,孟軻好辯,孔道以明,轍環天下,卒老於行。茍卿守正,大論是弘,逃讒於楚,廢死蘭陵。是二儒者,吐辭為經,舉足為法,絕類離倫,優入聖域,其遇於世何如也?今先生學雖勤,不由其統;言雖多,不要其中;文雖奇,不濟於用;行雖修,不顯於眾。猶且月費俸錢,歲靡廩粟,子不知耕,婦不知織,乘馬從徒,安坐而食,踵常塗之促促,窺陳編以盜竊。然而聖主不加誅,宰臣不見斥,此非其幸哉!動而得謗,名亦隨之。
 投閑置散,乃分之宜。若夫商財賄之有無,計班資之崇庳,忘己量之所稱,指前人之瑕疵,是所謂詰匠氏之不以杙為楹,而訾醫師以昌陽引年,欲進其豨苓也。」



 執政覽其文而憐之,以其有史才,改比部郎中、史館修撰。逾歲,轉考功郎中、知制誥,拜中書舍人。



 俄有不悅愈者,摭其舊事,言愈前左降為江陵掾曹,荊南節度使裴均館之頗厚,均子鍔凡鄙,近者鍔還省父,愈為序餞鍔,仍呼其字。此論喧於朝列,坐是改太子右庶子。



 元和十二年
 八月,宰臣裴度為淮西宣慰處置使,兼彰義軍節度使,請愈為行軍司馬,仍賜金紫。淮、蔡平,十二月隨度還朝,以功授刑部侍郎,仍詔愈撰《平淮西碑》,其辭多敘裴度事。時先入蔡州擒吳元濟,李愬功第一,愬不平之。愬妻出入禁中,因訴碑辭不實,詔令磨愈文。憲宗命翰林學士段文昌重撰文勒石。



 鳳翔法門寺有護國真身塔,塔內有釋迦文佛指骨一節,其書本傳法,三十年一開,開則歲豐人泰。十四年正月,上令中使杜英奇押宮人三
 十人,持香花赴臨皋驛迎佛骨。自光順門入大內,留禁中三日,乃送諸寺。王公士庶,奔走舍施,唯恐在後。百姓有廢業破產、燒頂灼臂而求供養者。愈素不喜佛,上疏諫曰:



 伏以佛者,夷狄之一法耳。自後漢時始流入中國,上古未嘗有也。昔黃帝在位百年,年百一十歲;少昊在位八十年,年百歲;顓頊在位七十九年,年九十八歲;帝嚳在位七十年,年百五歲;帝堯在位九十八年,年百一十八歲;帝舜及禹年皆百歲。此時天下太平,百姓安樂
 壽考,然而中國未有佛也。其後殷湯亦年百歲,湯孫太戊在位七十五年,武丁在位五十年,書史不言其壽,推其年數,蓋亦俱不減百歲。周文王年九十七歲,武王年九十三歲,穆王在位百年。此時佛法亦未至中國,非因事佛而致此也。



 漢明帝時始有佛法,明帝在位,才十八年耳。其後亂亡相繼,運祚不長。宋、齊、梁、陳、元魏已下,事佛漸謹,年代尤促。唯梁武帝在位四十八年,前後三度舍身施佛,宗廟之祭,不用牲牢,晝日一食,止於菜果。其
 後竟為侯景所逼,餓死臺城,國亦尋滅。事佛求福,乃更得禍。由此觀之,佛不足信,亦可知矣。



 高祖始受隋禪,則議除之。當時群臣識見不遠,不能深究先王之道、古今之宜,推闡聖明,以救斯弊,其事遂止。臣嘗恨焉!伏惟皇帝陛下,神聖英武,數千百年以來未有倫比。即位之初,即不許度人為僧尼、道士,又不許別立寺觀。臣當時以為高祖之志,必行於陛下之手。今縱未能即行,豈可恣之轉令盛也!



 今聞陛下令群僧迎佛骨於鳳翔,御樓以
 觀,舁入大內,令諸寺遞迎供養。臣雖至愚,必知陛下不惑於佛,作此崇奉以祈福祥也。直以年豐人樂,徇人之心,為京都士庶設詭異之觀、戲玩之具耳。安有聖明若此而肯信此等事哉!然百姓愚冥,易惑難曉,茍見陛下如此,將謂真心信佛。皆云天子大聖,猶一心敬信;百姓微賤,於佛豈合惜身命。所以灼頂燔指,百十為群,解衣散錢,自朝至暮。轉相仿效,唯恐後時,老幼奔波,棄其生業。若不即加禁遏,更歷諸寺,必有斷臂臠身以為供養
 者。傷風敗俗,傳笑四方,非細事也。



 佛本夷狄之人,與中國言語不通,衣服殊制。口不道先王之法言,身不服先王之法行,不知君臣之義、父子之情。假如其身尚在,奉其國命,來朝京師,陛下容而接之,不過宣政一見,禮賓一設,賜衣一襲,衛而出之於境,不令惑於眾也。況其身死已久,枯朽之骨,兇穢之餘,豈宜以入宮禁!孔子曰:「敬鬼神而遠之。」古之諸侯,行吊於國,尚令巫祝先以桃,祓除不祥,然後進吊。今無故取朽穢之物,親臨觀之,巫
 祝不先,桃不用,群臣不言其非,御史不舉其失,臣實恥之。乞以此骨付之水火,永絕根本,斷天下之疑,絕後代之惑。使天下之人,知大聖人之所作為,出於尋常萬萬也,豈不盛哉!豈不快哉!佛如有靈,能作禍祟,凡有殃咎,宜加臣身。上天鑒臨,臣不怨悔。



 疏奏,憲宗怒甚。間一日,出疏以示宰臣,將加極法。裴度、崔群奏曰:「韓愈上忤尊聽,誠宜得罪,然而非內懷忠懇,不避黜責,豈能至此?伏乞稍賜寬容,以來諫者。」上曰:「愈言我奉佛太過,我猶
 為容之。至謂東漢奉佛之後,帝王咸致夭促,何言之乖刺也?愈為人臣,敢爾狂妄,固不可赦!」於是人情驚惋,乃至國戚諸貴,亦以罪愈太重,因事言之,乃貶為潮州刺史。



 愈至潮陽,上表曰:



 臣今年正月十四日,蒙恩授潮州刺史,即日馳驛就路。經涉嶺海,水陸萬里。臣所領州,在廣府極東。去廣府雖雲二千里,然來往動皆逾月。過海口,下惡水,濤瀧壯猛,難計期程,颶風鱷魚,患禍不測。州南近界,漲海連天,毒霧瘴氛,日夕發作。臣少多病,年才
 五十,發白齒落,理不久長。加以罪犯至重,所處又極遠惡,憂惶慚悸,死亡無日。單立一身,朝無親黨,居蠻夷之地,與魍魅同群。茍非陛下哀而念之,誰肯為臣言者。



 臣受性愚陋,人事多所不通,唯酷好學問文章,未嘗一日暫廢,實為時輩推許。臣於當時之文,亦未有過人者。至於論述陛下功德,與《詩》、《書》相表裏。作為歌詩,薦之郊廟,紀太山之封,鏤白玉之牒;鋪張對天之宏休,揚厲無前之偉跡;編於《詩》、《書》之策而無愧,措於天地之間而無虧。
 雖使古人復生,臣未肯多讓。伏以大唐受命有天下,四海之內,莫不臣妾南北東西,地各萬里。自天寶之後,政治少懈,文致未優,武克不綱。孽臣奸隸,外順內悖;父死子代,以祖以孫。如古諸侯,自擅其地,不朝不貢,六七十年。四聖傳序,以至陛下,躬親聽斷,干戈所麾,無不從順。宜定樂章,以告神明;東巡泰山,奏功皇天,使永永萬年,服我成烈。當此之際,所謂千載一時,不可逢之嘉會。而臣負罪嬰釁,自拘海島,戚戚嗟嗟,日與死迫;曾不得奏
 薄伎於從官之內、隸御之間,窮思畢精,以贖前過。懷痛窮天,死不閉目!瞻望宸極,魂神飛去。伏惟陛下,天地父母,哀而憐之。



 憲宗謂宰臣曰:「昨得韓愈到潮州表,因思其所諫佛骨事,大是愛我,我豈不知!然愈為人臣,不當言人主事佛乃年促也。我以是惡其容易。」上欲復用愈,故先語及,觀宰臣之奏對。而皇甫鎛惡愈狷直,恐其復用,率先對曰:「愈終大狂疏,且可量移一郡。」乃授袁州刺史。



 初,愈至潮陽,既視事,詢吏民疾苦,皆曰:「郡西湫水有
 鱷魚,卵而化,長數丈,食民畜產將盡,以是民貧。」居數日,愈往視之,令判官秦濟砲一豚一羊,投之湫水,祝之曰:



 前代德薄之君,棄楚、越之地,則鱷魚涵泳於此可也。今天子神聖,四海之外,撫而有之。況揚州之境,刺史縣令之所治,出貢賦以共天地宗廟之祀,鱷魚豈可與刺史雜處此土哉?刺史受天子命,令守此土,而鱷魚睅然不安溪潭,食民畜熊鹿麞豕,以肥其身,以繁其卵,與刺史爭為長。刺史雖駑弱,安肯為鱷魚低首而下哉!今潮州
 大海在其南,鯨鵬之大,蝦蟹之細,無不容,鱷魚朝發而夕至。今與鱷魚約,三日乃至七日,如頑而不徙,須為物害,則刺史選材伎壯夫,操勁弓毒矢,與鱷魚從事矣!



 祝之夕,有暴風雷起於湫中。數日,湫水盡涸,徙於舊湫西六十里。自是潮人無鱷患。



 袁州之俗,男女隸於人者,逾約則沒入出錢之家。愈至,設法贖其所沒男女,歸其父母。仍削其俗法,不許隸人。



 十五年,徵為國子祭酒,轉兵部侍郎。會鎮州殺田弘正,立王廷湊,令愈往鎮州宣諭。
 愈既至,集軍民,諭以逆順。辭情切至,廷湊畏重之。改吏部侍郎。轉京兆尹,兼御史大夫。以不臺參,為御史中丞李紳所劾。愈不伏,言準敕仍不臺參。紳、愈性皆褊僻,移刺往來,紛然不止,乃出紳為浙西觀察使,愈亦罷尹為兵部侍郎。及紳面辭赴鎮,泣涕陳敘。穆宗憐之,乃追制以紳為兵部侍郎,愈復為吏部侍郎。長慶四年十二月卒,時年五十七,贈禮部尚書,謚曰文。



 愈性弘通,與人交,榮悴不易。少時與洛陽人孟郊、東郡人張籍友善。二人
 名位未振,愈不避寒暑,稱薦於公卿間,而籍終成科第,榮於祿仕。後雖通貴,每退公之隙,則相與談宴,論文賦詩,如平昔焉。而觀諸權門豪士,如僕隸焉,瞪然不顧。而頗能誘厲後進,館之者十六七,雖晨炊不給,怡然不介意。大抵以興起名教,弘獎仁義為事。凡嫁內外及友朋孤女僅十人。



 常以為自魏、晉已還,為文者多拘偶對,而經誥之指歸,遷、雄之氣格,不復振起矣。故愈所為,文,務反近體;抒意立言,自成一家新語。後學之士,取為師法。
 當時作者甚眾,無以過之,故世稱「韓文」焉。然時有恃才肆意,亦有盩孔、孟之旨。若南人妄以柳宗元為羅池神,而愈撰碑以實之;李賀父名晉,不應進士,而愈為賀作《諱辨》,令舉進士;又為《毛穎傳》,譏戲不近人情:此文章之甚紕繆者。時謂愈有史筆,及撰《順宗實錄》,繁簡不當,敘事拙於取舍,頗為當代所非。穆宗、文宗嘗詔史臣添改,時愈婿李漢、蔣系在顯位,諸公難之。而韋處厚竟別撰《順宗實錄》三卷。有文集四十卷,李漢為之序。



 子昶,亦登
 進士第。



 張籍者,貞元中登進士第。性詭激,能為古體詩,有警策之句傳於時。調補太常寺太祝,轉國子助教、秘書郎。以詩名當代,公卿裴度、令狐楚,才名如白居易、元稹,皆與之游,而韓愈尤重之。累授國子博士、水部員外郎,轉水部郎中,卒。世謂之張水部云。



 孟郊者,少隱於嵩山,稱處士。李翱分司洛中,與之游。薦於留守鄭餘慶,闢為賓佐。性孤僻寡合,韓愈一見以為
 忘形之契,常稱其字曰東野,與之唱和於文酒之間。鄭餘慶鎮興元,又奏為從事,闢書下而卒。餘慶給錢數萬葬送,贍給其妻子者累年。



 唐衢者,應進士,久而不第。能為歌詩,意多感發。見人文章有所傷嘆者,讀訖必哭,涕泗不能已。每與人言論,既相別,發聲一號,音辭哀切,聞之者莫不淒然泣下。嘗客游太原,屬戎帥軍宴,衢得預會。酒酣言事,抗音而哭,一席不樂,為之罷會,故世稱唐衢善哭。左拾遺白居易遺
 之詩曰:「賈誼哭時事,阮籍哭路歧。唐生今亦哭,異代同其悲。唐生者何人?五十寒且饑。不悲口無食,不悲身無衣。所悲忠與義,悲甚則哭之。太尉擊賊日,尚書叱盜時。大夫死兇寇,諫議謫蠻夷。每見如此事,聲發涕輒隨。我亦君之徒,鬱鬱何所為?不能發聲哭,轉作樂府辭。」其為名流稱重若此。竟不登一命而卒。



 李翱,字習之,涼武昭王之後。父楚金,貝州司法參軍。翱幼勤於儒學,博雅好古,為文尚氣質。貞元十四年登進
 士第,授校書郎。三遷至京兆府司錄參軍。元和初,轉國子博士、史館修撰。



 十四年,太常丞王涇上疏請去太廟朔望上食,詔百官議。議者以《開元禮》,太廟每歲礿、祠、蒸、嘗、臘,凡五享。天寶末,玄宗令尚食每月朔望具常饌,令宮闈令上食於太廟,後遂為常。由是朔望不視朝,比之大祠。翱奏議曰:



 《國語》曰:王者日祭。《禮記》曰:王立七廟,皆月祭之。《周禮》時祭,礿祠蒸嘗。漢氏皆雜而用之。蓋遭秦火,《詩》、《書》、《禮經》燼滅;編殘簡缺,漢乃求之。先儒穿鑿,各伸
 己見,皆托古聖賢之名,以信其語,故所記各不同也。古者廟有寢而不墓祭;秦、漢始建寢廟於園陵,而上食焉。國家因之而不改。《貞觀》、《開元禮》並無宗廟日祭、月祭之禮,蓋以日祭、月祭,既已行於陵寢矣。故太廟之中,每歲五饗六告而已。不然者,房玄齡、魏徵輩皆一代名臣,窮極經史,豈不見《國語》、《禮記》有日祭、月祭之詞乎?斯足以明矣。



 伏以太廟之饗,籩豆牲牢,三代之通禮,是貴誠之義也。園陵之奠,改用常饌;秦、漢之權制,乃食味之道也。
 今朔望上食於太廟,豈非用常褻味而貴多品乎?且非《禮》所謂「至敬不饗味而貴氣臭」之義也。《傳》稱:屈到嗜芰,有疾,召其宗老而屬之曰:「祭我必以芰。」及祭,薦芰,其子違命去芰而用羊,饋籩豆脯醢,君子是之。言事祖考之義,當以禮為重,不以其生存所嗜為獻,蓋明非食味也。然則薦常饌於太廟,無乃與芰為比乎?且非三代聖王之所行也。況祭器不陳俎豆,祭官不命三公,執事者唯宮闈令與宗正卿而已。謂之上食也,安得以為祭乎?且
 時享於太廟,有司攝事,祝文曰:「孝曾孫皇帝臣某,謹遣太尉臣名,敢昭告於高祖神堯皇帝、祖妣太穆皇后竇氏。時惟孟春,永懷罔極。謹以一元大武、柔毛剛鬣、明粢薌萁、嘉蔬嘉薦醴齊,敬脩時享,以申追慕。」此祝辭也。前享七日質明,太尉誓百官於尚書省曰:「某月某日時享於太廟,各揚其職。不供其事,國有常刑。」凡陪享之官,散齋四日,致齋三日,然後可以為祭也。宗廟之禮,非敢擅議,雖有知者,其誰敢言?故六十餘年行之不廢。今聖朝
 以弓矢既橐,禮樂為大,故下百僚,可得詳議。臣等以為《貞觀》、《開元禮》並無太廟上食之文,以禮斷情,罷之可也。至若陵寢上食,採《國語》、《禮記》日祭、月祭之詞,因秦、漢之制,修而存之,以廣孝道可也。如此,則經義可據,故事不遺。大禮既明,永息異論,可以繼二帝三王,而為萬代法。與其瀆禮越古,貴因循而憚改作,猶天地之相遠也。



 知禮者是之,事竟不行。



 翱性剛急,論議無所避。執政雖重其學,而惡其激訐,故久次不遷。翱以史官記事不實,奏
 狀曰:「臣謬得秉筆史館,以記注為職。夫勸善懲惡,正言直筆,紀聖朝功德,述忠賢事業,載奸臣醜行,以傳無窮者,史官之任也。凡人事跡,非大善大惡,則眾人無由得知,舊例皆訪於人,又取行狀謚議,以為依據。今之作行狀者,多是其門生故吏,莫不虛加仁義禮智,妄言忠肅惠和。此不唯其處心不實,茍欲虛美於受恩之地耳。蓋為文者,又非游、夏、遷、雄之列,務於華而忘其實,溺於文而棄其理。故為文則失《六經》之古風,紀事則非史遷之
 實錄。臣今請作行狀者,但指事實,直載事功。假如作《魏徵傳》,但記其諫諍之辭,足以為正直;段秀實但記其倒用司農印以追逆兵,以象笏擊硃泚,足以為忠烈。若考功視行狀,不依此者不得受。依此,則考功下太常,牒史館,然後定謚。伏乞以臣此奏下考功。」從之。尋權知職方員外郎。十五年六月,授考功員外郎,並兼史職。



 翱與李景儉友善。初,景儉拜諫議大夫,舉翱自代。至是,景儉貶黜,七月,出翱為朗州刺史。俄而景儉復為諫議大夫,翱
 亦入為禮部郎中。翱自負辭藝,以為合知制誥,以久未如志,鬱鬱不樂。因入中書謁宰相,面數李逢吉之過失。逢吉不之校。翱心不自安,乃請告。滿百日,有司準例停官,逢吉奏授廬州刺史。太和初,入朝為諫議大夫,尋以本官知制誥。三年二月,拜中書舍人。



 初,諫議大夫柏耆將使滄州軍前宣諭,翱嘗贊成此行。柏耆尋以擅入滄州得罪,翱坐謬舉,左授少府少監。俄出為鄭州刺史。五年,出為桂州刺史、御史中丞,充桂管都防禦使。七年,改
 授潭州刺史、湖南觀察使。八年,徵為刑部侍郎。九年,轉戶部侍郎。七月,檢校戶部尚書、襄州刺史,充山南東道節度使。會昌中,卒於鎮,謚曰文。



 宇文籍,字夏龜。父滔,官卑。少好學,尤通《春秋》。竇群自處士徵為右拾遺,表籍自代,由是知名。登進士第。宰相武元衡出鎮西蜀,奏為從事。以咸陽尉直史館,與韓愈同修《順宗實錄》,遷監察御史。王承宗叛,詔捕其弟駙馬都尉承系,其賓客中有為誤識者。又蘇表以破淮西策幹
 宰相武元衡,元衡不用。以籍舊從事,令召表訊之,籍因與表狎。元衡怒,坐貶江陵府戶曹參軍。至任,節度使孫簡知重之,欲令兼幕府職事。籍辭曰:「籍以君命譴黜,亦當以君命升。假榮偷獎,非所願也。」後考滿,連闢籓府,入為侍御史,轉著作郎,遷駕部員外郎、史館修撰。與韋處厚、韋表微、路隨、沈傳師同修《憲宗實錄》。俄以本官知制誥,轉庫部郎中。太和中,遷諫議大夫,專掌史筆,罷知制誥。



 籍性簡淡寡合,耽玩經史,精於著述,而風望峻整,為
 時輩推重。太和二年正月卒,時年五十九,贈工部侍郎。子監,大中初登進士第。



 劉禹錫,字夢得,彭城人。祖云。父漵,仕歷州縣令佐,世以儒學稱。禹錫貞元九年擢進士第,又登宏辭科。禹錫精於古文,善五言詩,今體文章復多才麗。從事淮南節度使杜佑幕,典記室,尤加禮異。從佑入朝,為監察御史。與吏部郎中韋執誼相善。



 貞元末,王叔文於東宮用事,後輩務進,多附麗之。禹錫尤為叔文知獎,以宰相器待之。
 順宗即位,久疾不任政事,禁中文誥,皆出於叔文。引禹錫及柳宗元入禁中,與之圖議,言無不從。轉屯田員外郎、判度支鹽鐵案,兼崇陵使判官。頗怙威權,中傷端士。宗元素不悅武元衡,時武元衡為御史中丞,乃左授右庶子。侍御史竇群奏禹錫挾邪亂政,不宜在朝。群即日罷官。韓皋憑藉貴門,不附叔文黨,出為湖南觀察使。既任喜怒凌人,京師人士不敢指名,道路以目,時號「二王、劉、柳。」



 叔文敗,坐貶連州刺史。在道,貶朗州司馬。地居西
 南夷,士風僻陋,舉目殊俗,無可與言者。禹錫在朗州十年,唯以文章吟詠,陶冶情性。蠻俗好巫,每淫祠鼓舞,必歌俚辭。禹錫或從事於其間,乃依騷人之作,為新辭以教巫祝。故武陵溪洞間夷歌,率多禹錫之辭也。



 初,禹錫、宗元等八人犯眾怒,憲宗亦怒,故再貶。制有「逢恩不原」之令。然執政惜其才,欲洗滌痕累,漸序用之。會程異復掌轉運,有詔以韓皋及禹錫等為遠郡刺史。屬武元衡在中書,諫官十餘人論列,言不可復用而止。



 禹錫積歲
 在湘、澧間,鬱悒不怡,因讀《張九齡文集》,乃敘其意曰:「世稱曲江為相,建言放臣不宜於善地,多徙五溪不毛之鄉。今讀其文章,自內職牧始,安有瘴癘之嘆,自退相守荊州,有拘囚之思。托諷禽鳥,寄辭草樹,鬱然與騷人同風。嗟夫,身出於遐陬,一失意而不能堪,矧華人士族,而必致丑地,然後快意哉!議者以曲江為良臣,識胡雛有反相,羞與凡器同列,密啟廷諍,雖古哲人不及。而燕翼無似,終為餒魂。豈忮心失恕,陰謫最大,雖二美莫贖耶?
 不然,何袁公一言明楚獄而鐘祉四葉。以是相較,神可誣乎?」



 元和十年,自武陵召還,宰相復欲置之郎署。時禹錫作《游玄都觀詠看花君子詩》,語涉譏刺,執政不悅,復出為播州刺史。詔下,御史中丞裴度奏曰:「劉禹錫有母,年八十餘。今播州西南極遠,猿狖所居,人跡罕至。禹錫誠合得罪,然其老母必去不得,則與此子為死別,臣恐傷陛下孝理之風。伏請屈法,稍移近處。」憲宗曰:「夫為人子,每事尤須謹慎,常恐貽親之憂。今禹錫所坐,更合重
 於他人,卿豈可以此論之?」度無以對。良久,帝改容而言曰:「朕所言,是責人子之事,然終不欲傷其所親之心。」乃改授連州刺史。去京師又十餘年。連刺數郡。



 太和二年,自和州刺史徵還,拜主客郎中。禹錫銜前事未已,復作《游玄都觀詩序》曰:「予貞元二十一年為尚書屯田員外郎,時此觀中未有花木。是歲出牧連州,尋貶朗州司馬。居十年,召還京師,人人皆言有道士手植紅桃滿觀,如爍晨霞,遂有詩以志一時之事。旋又出牧,於今十有四
 年,得為主客郎中。重游茲觀,蕩然無復一樹,唯兔葵燕麥動搖於春風,因再題二十八字,以俟後游。」其前篇有「玄都觀裏桃千樹,總是劉郎去後栽」之句,後篇有「種桃道士今何在,前度劉郎又到來」之句,人嘉其才而薄其行。禹錫甚怒武元衡、李逢吉,而裴度稍知之。太和中,度在中書,欲令知制誥。執政又聞《詩序》,滋不悅。累轉禮部郎中、集賢院學士。度罷知政事,禹錫求分司東都。終以恃才褊心,不得久處朝列。六月,授蘇州刺史,就賜金紫。
 秩滿入朝,授汝州刺史,遷太子賓客,分司東都。



 禹錫晚年與少傅白居易友善,詩筆文章,時無在其右者。常與禹錫唱和往來,因集其詩而序之曰:「彭城劉夢得,詩豪者也。其鋒森然,少敢當者。予不量力,往往犯之。夫合應者聲同,交爭者力敵。一往一復,欲罷不能。由是每制一篇,先於視草,視竟則興作,興作則文成。一二年來,日尋筆硯,同和贈答,不覺滋多。太和三年春以前,紙墨所存者,凡一百三十八首。其餘乘興仗醉,率然口號者,不在
 此數。因命小侄龜兒編勒成兩軸。仍寫二本,一付龜兒,一授夢得小男侖郎,各令收藏,附兩家文集。予頃與元微之唱和頗多,或在人口。嘗戲微之云:『僕與足下二十年來為文友詩敵,幸也!亦不幸也。吟詠情性,播揚名聲,其適遺形,其樂忘老,幸也!然江南士女語才子者,多雲元、白,以子之故,使僕不得獨步於吳、越間,此亦不幸也!今垂老復遇夢得,非重不幸耶?』夢得夢得,文之神妙,莫先於詩。若妙與神,則吾豈敢?如夢得『雪裏高山頭白早,
 海中仙果子生遲』,『沉舟側畔千帆過,病樹前頭萬木春』之句之類,真謂神妙矣!在在處處,應有靈物護持,豈止兩家子弟秘藏而已!」其為名流許與如此。夢得嘗為《西塞懷古》、《金陵五題》等詩,江南文士稱為佳作,雖名位不達,公卿大僚多與之交。



 開成初,復為太子賓客分司,俄授同州刺史。秩滿,檢校禮部尚書、太子賓客分司。會昌二年七月卒,時年七十一,贈戶部尚書。



 子承雍,登進士第,亦有才藻。



 柳宗元,字子厚,河東人。後魏侍中濟陰公之系孫。曾伯祖奭,高祖朝宰相。父鎮,太常博士,終侍御史。宗元少聰警絕眾,尤精《西漢詩騷》。下筆構思,與古為侔。精裁密致,璨若珠貝。當時流輩咸推之。登進士第,應舉宏辭,授校書郎、藍田尉。貞元十九年,為監察御史。



 順宗即位,王叔文、韋執誼用事,尤奇待宗元。與監察呂溫密引禁中,與之圖事。轉尚書禮部員外郎。叔文欲大用之,會居位不久,叔文敗,與同輩七人俱貶。宗元為邵州刺史。在道,再
 貶永州司馬。既罹竄逐,涉履蠻瘴,崎嶇堙厄,蘊騷人之鬱悼。寫情敘事,動必以文。為騷文十數篇,覽之者為之淒惻。



 元和十年,例移為柳州刺史。昌朗州司馬劉禹錫得播州刺史,制書下,宗元謂所親曰:「禹錫有母年高,今為郡蠻方,西南絕域,往復萬里,如何與母偕行?如母子異方,便為永訣。吾於禹錫為執友,胡忍見其若是?」即草章奏,請以柳州授禹錫,自往播州。會裴度亦奏其事,禹錫終易連州。



 柳州土俗,以男女質錢,過期則沒入錢主,
 宗元革其鄉法。其已沒者,仍出私錢贖之,歸其父母。江嶺間為進士者,不遠數千里皆隨宗元師法;凡經其門,必為名士。著述之盛,名動於時,時號柳州云。有文集四十卷。



 元和十四年十月五日卒,時年四十七。子周六、周七,才三四歲。觀察使裴行立為營護其喪及妻子還於京師,時人義之。



 韋辭,字踐之。祖召卿,洛陽丞。父翃,官至侍御史。辭少以兩經擢第,判入等,為秘書省校書郎。貞元末,東都留守
 韋夏卿闢為從事。後累佐使府,皆以參畫稱職。元和九年,自藍田令入拜侍御史,以事累出為朗州刺史,再貶江州司馬。



 長慶初,韋處厚、路隨以公望居顯要,素知辭有文學理行,亟稱薦之。擢為戶部員外,轉刑部郎中,充京西北和糴使。尋為戶部郎中、兼御史中丞,充鹽鐵副使,轉吏部郎中。文宗即位,韋處厚執政,且以澄汰浮華、登用藝實為事,乃以辭與李翱同拜中書舍人。



 辭素無清藻,文筆不過中才,然處事端實,游官無黨。與李翱特
 相善,俱擅文學高名。疏達自用,不事檢操。處厚以激時用,頗不厭公論;辭亦倦於潤色,苦求外任。乃出為潭州刺史、御史中丞、湖南觀察使。在鎮二年,吏民稱治。大和四年卒,時年五十八,贈右散騎常侍。



 史臣曰:貞元、太和之間,以文學聳動搢紳之伍者,宗元、禹錫而已。其巧麗淵博,屬辭比事,誠一代之宏才。如俾之詠歌帝載,黼藻王言,足以平揖古賢,氣吞時輩。而蹈道不謹,暱比小人,自致流離,前隳素業。故君子群而不
 黨,戒懼慎獨,正為此也。韓、李二文公,於陵遲之末,遑遑仁義;有志於持世範,欲以人文化成,而道未果也。至若抑楊、墨,排釋、老,雖於道未弘,亦端士之用心也。



 贊曰:天地經綸,無出斯文。愈、翱揮翰,語切典墳。犧雞斷尾,害馬敗群。僻塗自噬,劉、柳諸君。



\end{pinyinscope}