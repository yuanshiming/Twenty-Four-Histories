\article{卷二十二 志第二 禮儀二}

\begin{pinyinscope}

 隋文帝開皇中,將作大匠宇文愷依《月令》造明堂木樣以獻。帝令有司於京城安業里內規兆其地,方欲崇建,而諸儒爭論不定,竟議罷之。煬帝時,愷復獻明堂木樣並議狀,屬遷都興役,事又不就。終於隋代,季秋大享,恆在雩壇設祀。



 高祖受禪,不遑創儀。太宗平定天下,命儒官議其制。貞觀五年,太子中允孔穎達以諸儒立議違古,上言曰:「臣伏尋前敕,依禮部尚書盧寬、國
 子助教劉伯莊等議,以為『從昆侖道上層祭天』。又尋後敕云:『為左右閣道,登樓設祭。』臣檢六藝群書百家諸史,皆名基上曰堂,樓上曰觀,未聞重樓之上而有堂名。《孝經》云:『宗祀文王於明堂』。不云明樓、明觀,其義一也。又明堂法天,聖王示儉,或有翦蒿為柱,葺茅作蓋。雖復古今異制,不可恆然,猶依大典,惟在樸素。是以席惟颭秸,器尚陶匏,用繭慄以貴誠,服大裘以訓儉,今若飛樓架道,綺閣凌雲,考古之文,實堪疑慮。按《郊祀志》:漢武明堂之制,四面無壁,上覆以茅。祭五帝於上座,祀后土於下防。臣以上座正為基上,下防惟是基下。既
 云無四壁,未審伯莊以何知上層祭神,下有五室?且漢武所為,多用方士之說,違經背正,不可師祖。又盧寬等議云:上層祭天,下堂布政,欲使人神位別,事不相干。臣以古者敬重大事,與接神相似,是以朝覲祭祀,皆在廟堂,豈有樓上祭祖,樓下視朝?閣道升樓,路便窄隘,乘輦則接神不敬,步往則勞勩聖躬。侍衛在旁,百司供奉。求之典誥,
 全無此理。臣非敢
 固執愚見,以求己長。伏以國之大典,
 不可不慎。乞以臣言下群臣詳議。」侍中魏徵議曰:「稽諸古訓,參以舊圖,其上圓下方,復廟重屋,百慮一致,異軫同歸。洎當塗膺籙,未遑斯禮;典午聿興,無所取則。裴頠以諸儒持論,異端蜂起,是非舛互,靡所適從,遂乃以人廢言,止為一殿。宋、齊即仍其舊,梁、陳遵而不改。雖嚴配有所,祭享不匱,求之典則,道實未弘。夫孝因心生,禮緣情立。心不可極,故備物以表其誠;情無以盡,故飾宮以廣其敬。宣尼美意,其在茲乎!臣等親奉德音,令參大議,思竭塵露,微增山海。凡聖人有
 作,
 義重隨時,萬物斯睹,事資通變。若據蔡邕之說,則至理失於文繁;若依裴頠所為,則又傷於質略。求之情理,未允厥中。今之所議,非無用舍。請為五室重屋,上圓下方,既
 體有則象,又事多故實。下室備布政之居,上堂為祭天之所,人神不雜,禮亦宜之。其高下廣袤之規,幾筵尺丈之制,則並隨時立法,因事制宜。自我而作,何必師古。廓千載之疑議,為百王之懿範。不使泰山之下,惟聞黃帝之法;汶水之上,獨稱漢武之圖。則通乎神明,庶幾可俟,子來經始,成之不日。」議猶未決。



 十七年五月,秘書監顏師古議曰:



 明堂之制,爰自古昔,求之簡牘,全文莫睹。始之黃帝,降及有虞,彌歷夏、殷,迄於周代,各立名號,別創
 規模。眾說舛駁,互執所見,巨儒碩學,莫有詳通。斐然成章,不知裁斷。究其指要,實布政之宮也。徒以戰國縱橫,典籍廢棄;暴秦酷烈,經禮湮亡。今之所存,傳記雜說,用為準的,理實蕪昧。然《周書》之敘明堂,紀其四面,則有應門、雉門,據此一堂,固是王者之常居耳。其青陽、總章、玄堂、太廟及左個、右個,與四時之次相同,則路寢之義,足為明證。又《文王居明堂》之篇:帶以弓蜀,祠於高禖。下九門磔禳以御疾疫,置梁除道以利農夫,令國有酒以合三
 族。」凡此等事,皆合《月令》之文。觀其所為,皆在路寢者也。《戴禮》:「昔周公朝諸侯於明堂之位,天子負斧扆南向而立。明堂也者,明諸侯之尊卑也。」《周官》又云「周人明堂,度九尺之筵,東西九筵,堂一筵。」據其制度,即大寢也。《尸子》亦曰:「黃帝曰合宮,有虞氏曰總章,殷曰陽館,周曰明堂。」斯皆路寢之徵,知非別處。大戴所說,初有近郊之言,復稱文王之廟,進退無據,自為矛盾。原夫負扆受朝,常居出入,既在皋庫之內,亦何云於郊野哉?《孝經傳》云「在國之陽」,又無
 里數。



 漢武有懷創造,詢於搢紳,言論紛然,終無定據,乃立於汶水之上而宗祀焉,明其不拘遠近,無擇方面。孝成之代,表行城南,雖有其文,厥功靡立。平帝元始四年,大議營創。孔牢等乃以為明堂、闢雍、太學,其實一也,而有三名。金褒等又稱經傳無文,不能分別同異。中興之後,蔡邕作論,復云明堂太廟,一物二名。鄭玄則曰:「在國之陽,三里之外。」淳于登又云:「三里之外,七里之內,丙巳之地。」潁容《釋例》亦云:「明堂太廟,凡有八名,其體一也。」茍立同異,競為巧說,並出
 自胸懷,曾無師祖。審夫功成作樂,理定制禮,草創從宜,質文遞變。旌旗冠冕,古今不同,律度權衡,前後不一,隨時之義,斷可知矣。假如周公舊章,猶當擇其可否;宣尼彞則,尚或補其闕漏。況鄭氏臆說,淳于謏聞,匪異守株,何殊膠柱?愚謂不出墉雉,邇接宮闈,實允事宜,諒無所惑。但當上遵天旨,祗奉德音,作皇代之明堂,永貽範於來葉。區區碎議,皆略而不論。



 又上表曰:「明堂之制,陛下已發德音,久令詳議。但以學者專固,人人異言,損益不
 同,是非莫定。臣愚以為五帝之後,兩漢已前,高下方圓,皆不相襲。惟在陛下聖情創造,即為大唐明堂,足以傳於萬代,何必論戶牖之多少,疑階庭之廣狹?若恣儒者互說一端,久無斷決,徒稽盛禮,昔漢武欲草封禪儀,博望諸生,所說不同,莫知孰是。唯御史大夫倪寬勸上自定制度,遂成登封之禮。臣之愚誠,亦望陛下斟酌繁省,為其節文,不可謙拒,以淹大典。」尋以有事遼海,未暇營創。



 永徽二年七月二日,敕曰:「上玄幽贊,崇高而不言;
 皇王提象,代神功而理物。是知五精降德,爰應帝者之尊;九室垂文,用紀處天之業。且合宮、靈府,創鴻規於上代;太室、總章,標茂範於中葉。雖質文殊制,奢儉異時,然其立天中,作人極,布政施教,其歸一揆。朕嗣膺下武,丕承上烈,思所以答眷上靈,聿遵孝享,而法宮曠禮,明堂寢構。今國家四表無虞,人和歲稔,作範垂訓,今也其時。宜令所司與禮官學士等考核故事,詳議得失,務依典禮,造立明堂。庶曠代闕文,獲申於茲日;因心展敬,永垂
 於後昆。其明堂制度,令諸曹尚書及左右丞侍郎、太常、國子秘書官、弘文館學士同共詳議。」



 於是太常博士柳宣仍鄭玄義,以為明堂之制,當為五室。內直丞孔志約據《大戴禮》及盧植、蔡邕等義,以為九室。曹王友趙慈皓、秘書郎薛文思等各造明堂圖。諸儒紛爭,互有不同。上初以九室之議為是,乃令所司詳定形制及闢雍門闕等。



 明年六月,內出九室樣,仍更令有司損益之。有司奏言:



 內樣:堂基三重,每基階各十二。上基方九雉,八角,高
 一尺。中基方三百尺,高一筵。下基方三百六十尺,高一丈二尺。上基象黃琮,為八角,四面安十二階。請從內樣為定。基高下仍請準周制高九尺,其方共作司約準一百四十八尺。中基下基,望並不用。又內樣:室各方三筵,開四闥、八窗。屋圓楣徑二百九十一尺。按季秋大饗五帝,各在一室,商量不便,請依兩漢季秋合饗,總於太室。若四時迎氣之祀,則各於其方之室。其安置九室之制,增損明堂故事,三三相重。太室在中央,方六丈。其四隅之
 室,謂之左右房,各方二丈四尺。當太室四面,青陽、明堂、總章、玄堂等室,各長六丈,以應太室;闊二丈四尺,以應左右房。室間並通巷,各廣一丈八尺。其九室並巷在堂上,總方一百四十四尺,法坤之策。屋圓楣、楯、簷,或為未允。請據鄭玄、盧植等說,以前梁為楣,其徑二百一十六尺,法乾之策。圓柱旁出九室四隅,各七尺,法天以七紀。柱外餘基,節作司約準面別各餘一丈一尺。內樣:室別四闥、八窗,檢與古同,請依為定。其戶依古外設而不開。內
 樣:外有柱三十六,每柱十梁。內有七間,柱根以上至梁高三丈,梁以上至屋峻起,計高八十一尺。上圓下方,飛簷應規,請依內樣為定。其屋蓋形制,仍望據《考工記》改為四阿,並依禮加重簷,準太廟安鴟尾。堂四向五色,請依《周禮》白盛為便。其四向各隨方色。請施四垣及四門。



 闢雍,按《大戴禮》及前代說,闢雍多無水廣、內徑之數。蔡邕云:「水廣二十四丈,四周於外。」《三輔黃圖》云「水廣四周」,與蔡邕不異,仍云「水外周堤」。又張衡《東京賦》稱「造舟為梁」。《
 禮記·明堂位》、《陰陽錄》云:「水左旋以象天。」商量水廣二十四丈,恐傷於闊,今請減為二十四步,垣外量取周足。仍依故事造舟為梁,其外周以圓堤,並取《陰陽》「水行左旋」之制。



 殿垣,按《三輔黃圖》,殿垣四周方在水內,高不蔽日,殿門去殿七十二步。準今行事陳設,猶恐窄小。其方垣四門去堂步數,請準太廟南門去廟基遠近為制。仍立四門八觀,依太廟門別各安三門,施玄閫,四角造三重魏闕。



 此後群儒紛競,各執異議。尚書左僕射於志寧等
 請為九室,太常博士唐等請為五室。高宗令於觀德殿依兩議張設,親與公卿觀之。帝曰:「明堂之禮,自古有之。議者不同,未果營建。今設兩議,公等以何者為宜?」工部尚書閻立德對曰:「兩議不同,俱有典故。九室似暗,五室似明。取舍之宜,斷在聖慮。」上以五室為便,議又不定,由是且止。



 至乾封二年二月,詳宜略定,乃下詔曰:「朕以寡薄,忝承丕緒。奉二聖之遺訓,撫億兆以初臨,馭朽兢懷,推溝在念。而上玄垂祐,宗社降休,歲稔時和,人殷俗
 阜。車書混一,文軌大同。檢玉泥金,升中告禪,百蠻執贄,萬國來庭,朝野俱娛,華夷胥悅。但為郊禋嚴配,未安太室,布政施行,猶闕合宮。朕所以日昃忘疲,中宵輟寢,討論墳籍,錯綜群言,採三代之精微,探九皇之至賾,斟酌前載,制造明堂。棟宇方圓之規,雖兼故實;度筵陳俎之法,獨運財成。宣諸內外,博考詳議,求其長短,冀廣異聞。而鴻生碩儒,俱稱盡善,搢紳士子,並奏該通。創此宏模,自我作古。因心既展,情禮獲伸,永言宗祀,良深感慰。宜
 命有司,及時起作,務從折中,稱朕意焉。」於是大赦天下,改元為總章,分萬年置明堂縣。明年三月,又具規制廣狹,下詔曰:



 合宮聽朔,闡皇軒之茂範;靈府通和,敷帝勛之景化。殷人陽館,青珪備禮;姬氏玄堂,彤璋合獻。雖運殊驪翰,時變質文,至於立天中,建皇極,軌物施教,其歸一揆。考圖汶上,僅存公玉之儀;度室圭躔,才紀中元之制。屬炎精墜駕,睿宮毀籥,四海淪於沸鼎,九土陷於塗原。高祖太武皇帝杖鉞唐郊,收鈐雍野,納祥符於蒼水,
 受靈命於丕山。飛沈泳沫,動植游源。太宗文皇帝盟津光誓,協降火而登壇;豐穀斷蛇,應屯雲而鞠旅。封金貸嶺,昭累聖之鴻勛;勒石丸都,成文考之先志。固可以作化明堂,顯庸太室。傍羅八柱,周建四門,木工不琢,土事無文,豐約折衷,經始勿亟,闕文斯備,大禮聿修。



 其明堂院每面三百六十步,當中置堂。按《周易》乾之策二百一十有六,坤之策一百四十有四,總成三百六十,故方三百六十步。當中置堂,處二儀之中,定三才之本,構茲一
 宇,臨此萬方。自降院每面三門,同為一宇,徘徊五間。按《尚書》,一期有四時,故四面各一所開門;每時有三月,故每一所開三門;一期十有二月,故周回總十二門。所以面別一門,應茲四序,既一時而統三月,故於一舍而置三門。又《周易》三為陽數,二為陰數,合而為五,所以每門舍五間。院四隅各置重樓,其四墉各依本方色。按《淮南子》,地有四維,故四樓。又按《月令》,水、火、金、木、土五方各異色,故其墻各依本方之色。



 基八面,象八方。按《周禮》「黃琮
 禮地」。鄭玄注:琮者,八方之玉,以象地形,故以祀地。則知地形八方。又按《漢書》,武帝立八觚壇以祀地。登地之壇,形象地,故令為八方之基,以象地形。基高一丈二尺,徑二百八十尺。按《漢書》,陽為六律,陰為六呂。陽與陰合,故高一丈二尺。又按《周易》,三為陽數,八為陰數。三八相乘,得二百四十尺。按《漢書》,九會之數有四十,合為二百八十,所以基徑二百八十尺。故以交通天地之和,錯綜陰陽之數。以明陽不獨運,資陰和以助成;陰不孤行,待陽
 唱而方應。陰陽兩順,天地咸亨,則百寶斯興,九疇攸序。基每面三階,周回十二階,每階為二十五級。按《漢書》,天有三階,故每面三階;地有十二辰,故周回十二階。又按《文子》,從凡至聖,有二十五等,故每階二十五級。所以應符星而設階,法臺耀以疏陛,上擬霄漢之儀,下則地辰之數。又列茲重級,用準聖凡。象皇極之高居,俯庶類而臨耀。



 基之上為一堂,其宇上圓。按《道德經》:天得一以清,地得一以寧,侯王得一以為天下貞。又曰:道生一,一生
 二,二生三,三生萬物。又按《漢書》:太極元氣,函三為一。又曰:天子以四海為家。故置一堂以象元氣,並取四海為家之義。又按《周禮》,「蒼璧禮天」。鄭玄注:璧圓以象天。故為宇上圓。堂每面九間,各廣一丈九尺。按《尚書》,地有九州,故立九間。又按《周易》,陰數十,故間別一丈九尺,所以規模厚地,準則陰陽,法二氣以通基,置九州於一宇。堂周回十二門,每門高一丈七尺,闊一丈三尺。按《禮記》,一歲有十二月,所以置十二門。又按《周易》,陰數十,陽數七,故
 高一丈七尺;又曰陽數五,陰數八,故闊一丈三尺。所以調茲玉燭,應彼金輝,葉二氣以循環,逐四序而迎節。堂周回二十四窗,高一丈三尺,闊一丈一尺,二十三櫺,二十四明。按《史記》,天有二十四氣,故置二十四窗。又按《書》,一年十二月,並象閏,故高一丈三尺。又按《周易》,天數一,地數十,故闊一丈一尺;又天數九,地數十,並四時成二十三,故二十三櫺。又按《周易》,八純卦之本體,合二十四爻,故有二十四明。列牖疏窗,象風候氣,遠周天地之數,曲
 準陰陽之和。



 堂心八柱,各長五十五尺。按《河圖》,八柱承天,故置八柱。又按《周易》,大衍之數五十有五,故長五十五尺。聳茲八柱,承彼九間,數該大衍之規,形符立極之制。且柱為陰數,天實陽元,柱以陰氣上升,天以陽和下降,固陰陽之交泰,乃天地之相承。堂心之外,置四柱為四輔。按《漢書》,天有四輔星,故置四柱以象四星。內以八柱承天,外象四輔明化,上交下泰,表裏相成,葉臺耀以分輝,契編珠而拱極。八柱四輔之外,第一重二十柱。按《周
 易》,天數五,地數十,並五行之數合而為二十,故置二十柱。體二儀而立數,葉五位以裁規,式符立極之功,允應剛柔之道。八柱四輔之外,第二重二十八柱。按《史記》,天有二十八宿,故有二十八柱。所以仰則乾圖,上符景宿,考編珠而紀度,觀列宿以迎時。八柱四輔之外,第三重三十二柱。按《漢書》,有八節、八政、八風、八音,四八三十二柱。調風御節,萬物資以化成;布政流音,九區仰而貽則。外面周回三十六柱。按《漢書》,一期三十六旬,故法之以置三
 十六柱。所以象歲時而致用,順寒暑以通微,璇璣之度無愆,玉歷之期永契。八柱之外,修短總有三等。按《周易》,天、地、人為三才,故置柱長短三等。所以擬三才以定位,高下相形;體萬物以資生,長短兼運。八柱之外,都合一百二十柱。按《禮記》,天子置三公、九卿、二十七大夫、八十一元士,合為一百二十,是以置一百二十柱。分職設官,翊化資於多士;開物成務,構春廈藉於群材。其上檻周回二百四柱。按《周易》,坤之策一百四十有四,又《漢書》,九會
 之數有六十,故置二百四柱。所以採坤策之玄妙,法甲乙之精微,環回契辰象之規,結構準陰陽之數。又基以象地,故葉策於坤元;柱各依方,復規模於甲子。



 重楣,二百一十六條。按《周易》,乾之策二百一十有六,故置二百一十六條。所以規模《易》象,擬法乾元,應大衍之深玄,葉神策之至數。大小節級拱,總六千三百四十五。按《漢書》,會月之數,六千三百四十五,故置六千三百四十五枚。所以遠採三統之文,傍符會月之數,契金儀而調節,偶
 璇歷以和時。重幹,四百八十九枚。按《漢書》,章月二百三十五,閏月周回二百五十四,總成四百八十九,故置四百八十九枚。所以法履端之奧義,象舉正之芳猷,規模歷象,發明章、閏。下璟,七十二枚。按《易緯》,有七十二候,故置七十二枚。所以式模芳節,取規貞候,契至和於昌歷,偶神數於休期。上璟,八十四枚。按《漢書》,九會之數有七十八。又按《莊子》:六合之外,聖人存而不論。司馬彪注:天地四方為六合。總成八十四,故置八十四枚。所以模範二
 儀,包羅六合,準會陰陽之數,周通氣候之源。枅,六十枚。按《漢書》,推太歲之法有六十,故置六十枚。所以兼該歷數,包括陰陽,採甲乙之深微,窮辰子之玄奧。連栱,三百六十枚。按《周易》,當期之日,三百有六十,故置三百六十枚。所以葉周天之度,準當期之日,順平分而成歲,應晷運以循環。小梁,六十枚。按《漢書》,有六十甲子,故置六十枚。構此虹梁,遐規鳳歷,傍竦四宇之制,遙符六甲之源。牽,二百二十八枚。按《漢書》,章中二百二十八,故置二百
 二十八枚。所以應長歷之規,象中月之度,廣綜陰陽之數,傍通寒暑之和。方衡,一十五重。按《尚書》,五行生數一十有五,故置十五重。結棟分間,法五行而演秘;疏楹疊構,葉生數以成規。南北大梁,二根。按《周易》太極生兩儀,故置二大梁。軌範乾坤,模擬天地,象玄黃之合德,表覆載以生成。陽馬,三十六道。按《易緯》,有三十六節,故置三十六道。所以顯茲嘉節,契此貞辰,分六氣以燮陰陽,環四象而調風雨。椽,二千九百九十根。按《漢書》,月法二千三
 百九十二,通法五百九十八,共成二千九百九十。所以偶推步之規,合通法之數。是知疏椽構宇,則大壯之架斯隆,積月成年,則會歷之規無爽。大梠,兩重,重別三十六條,總七十二。按《淮南子》,太平之時,五日一風,一年有七十二風,故置七十二條。所以通規瑞歷,葉數祥風,遙符淳俗之年,遠則休徵之契。飛簷椽,七百二十九枚。按《漢書》,從子至午,其數七百二十九,故置七百二十九枚。所以採辰象之宏模,法周天之至數。且午為陰本,子實陽源,子午分時,
 則生成之道自著;陰陽合德,則覆載之義茲隆。



 堂簷,徑二百八十八尺。按《周易》,乾之策二百一十六,《易緯》云,年有七十二候,合為二百八十八,故徑二百八十八尺。所以仰葉乾策,遠承貞候,順和氣而調序,擬圓蓋以照臨。堂上棟,去基上面九十尺。按《周易》,天數九,地數十,以九乘十,數當九十,故去基上面九十尺。所以上法圓清,下儀方載,契陰陽之至數,葉交泰之貞符。又以茲天九,乘於地十,象陽唱而陰和,法乾施而坤成。簷,去地五十五
 尺。按《周易》,大衍之數五十有五,故去地五十五尺。所以擬大《易》之嘉數,通惟神之至賾,道合萬象,理貫三才。上以清陽玉葉覆之。按《淮南子》,清陽為天,合以清陽之色。



 詔下之後,猶群議未決。終高宗之世,未能創立。



 則天臨朝,儒者屢上言請創明堂。則天以高宗遺意,乃與北門學士議其制,不聽群言。垂拱三年春,毀東都之乾元殿,就其地創之。四年正月五日,明堂成。凡高二百九十四尺,東西南北各三百尺。有三層:下層象四時,各隨方
 色;中層法十二辰,圓蓋,蓋上盤九龍捧之;上層法二十四氣,亦圓蓋。亭中有巨木十圍,上下通貫,栭、櫨LR、勣,藉以為本,亙之以鐵索。蓋為鸞鷟,黃金飾之,勢若飛翥。刻木為瓦,夾紵漆之。明堂之下施鐵渠,以為闢雍之象。號萬象神宮。因改河南縣為合宮縣。詔曰:



 黃軒御歷,朝萬方於合宮;丹陵握符,咨四岳於衢室。有虞輯瑞,總章之號既存;大禹錫珪,重屋之名攸建。殷人受命,置陽館以辨方;周室凝圖,立明堂以經野。用能範圍三極,幽贊五神,
 展尊祖之懷,申宗祀之典。爰從漢、魏,迨及周、隋、經始之制雖興,修廣之規未備。朕以庸昧,虔膺厚托,受寄於綴衣之夕,荷顧于仍幾之前。伏以高宗往年,已屬意於陽館,故京輔之縣,預紀明堂之名;改元之期,先著總章之號。朕於乾封之際,已奉表上塵,雖簡宸心,未遑榮構。今以鼎郊勝壤,圭邑奧區,處天地之中,順陰陽之序,舟車是湊,貢賦攸均,爰藉子來之功,式遵奉先之旨。



 夫明堂者,天子宗祀之堂,朝諸侯之位也。開乾坤之奧策,法氣
 象之運行,故能使災害不生,禍亂不作。眷言盛烈,豈不美歟!比者鴻儒禮官,所執各異,咸以為明堂者,置之三里之外,七里之內,在國陽明之地。今既俯邇宮掖,恐黷靈祇,誠乃布政之居,未為宗祀之所。朕乃為丙巳之地,去宮室遙遠,每月所居,因時饗祭,常備文物,動有煩勞,在於朕懷,殊非所謂。今故裁基紫掖,闢宇彤闈,經始肇興,成之匪日。但敬事天地,神明之德乃彰;尊祀祖宗,嚴恭之志方展。若使惟雲布政,負扆臨人,則茅宇土階,取
 適而已,豈必勞百姓之力,制九筵而御哉!誠以獲執蘋蘩,虔奉宗廟故也。時既沿革,莫或相遵,自我作古,用適於事。今以上堂為嚴配之所,下堂為布政之居,光敷禮訓,式展誠敬。來年正月一日,可於明堂宗祀三聖,以配上帝。宜令禮官、博士、學士、內外明禮者,詳定儀禮,務從典要,速以奏聞。



 永昌元年正月元日,始親享明堂,大赦改元。其月四日,御明堂布政,頒九條以訓於百官。文多不載。翌日,又御明堂,饗群臣,賜縑纁有差。自明堂成後,
 縱東都婦人及諸州父老入觀,兼賜酒食,久之乃止。吐蕃及諸夷以明堂成,亦各遣使來賀。載初元年冬正月庚辰朔,日南至,復親饗明堂,大赦改元,用周正。翼日,布政於群後。其年二月,則天又御明堂,大開三教。內史邢文偉講《孝經》,命侍臣及僧、道士等以次論議,日昃乃罷。



 天授二年正月乙酉,日南至,親祀明堂,合祭天地,以周文王及武氏先考、先妣配,百神從祀,並於壇位次第布席以祀之。於是春官郎中韋叔夏奏曰:「謹按明堂大享,
 唯祀五帝。故《月令》云:『是月也,大享帝。』則《曲禮》所云『大享不問卜』,鄭玄注云『謂遍祭五帝於明堂,莫適卜』是也。又按《祭法》云:『祖文王而宗武王。』鄭玄注云:『祭五帝、五神於明堂曰祖、宗。』故《孝經》云:「宗祀文王於明堂,以配上帝。』據此諸文,明堂正禮,唯祀五帝,配以祖宗及五帝、五官神等,自外餘神,並不合預。伏惟陛下追遠情深,崇禋志切,於明堂祀,加昊天上帝、皇地祇,重之以先帝、先後配享,此乃補前王之闕典,弘嚴配之虔誠。往以神都郊壇未建,乃於明
 堂之下,廣祭眾神,蓋義出權時,非不刊之禮也。謹按禮經:其內官、中官、五岳、四瀆諸神,並合從祀於二至。明堂總奠,事乃不經。然則宗祀配天之親,雜與小神同薦,於嚴敬之道,理有不安。望請每歲元日,惟祀天地大神,配以帝後。其五岳以下,請依禮於冬、夏二至,從祀方丘、圓丘,庶不煩黷。」從之。時則天又於明堂後造天堂,以安佛像,高百餘尺。始起建構,為大風振倒。俄又重營,其功未畢。證聖元年正月丙申夜,佛堂災,延燒明堂,至曙,二堂並
 盡。尋時又無雲而雷,起自西北。則天欲責躬避正殿。宰相姚璹曰:「此實人火,非是天災。至如成周宣榭,卜代逾長;漢武建章,盛德彌永。今明堂是布政之所,非宗祀也。」則天乃御端門觀酺宴,下詔令文武九品已上各上封事,極言無有所隱。左拾遺劉承慶上疏曰:



 臣聞自古帝王,皆有美惡,休祥所以昭其德,災變所以知其咎,天道之常理,王者之常事。然則休祥屢臻,不可矜功而自滿;災變奄降,不可輕忽而靡驚。故殷宗以桑穀生朝,懷懼
 而自省,妖不勝德,遂立中興之功;辛紂以雀生大鳥,恃福而自盈,祥不勝驕,終至傾亡之禍。故知災變之生,將以覺悟明主,扶持大業,使盛而不衰。理須祗畏神心,驚懼天誡,飭身正事,業業兢兢,則兇往而吉來,轉禍而為福。昔殷湯禱身而降雨,成王省事以反風,宋公憂熒惑之災,而應三舍之壽,高宗懲雊鼎之異,而享百年之福,此其類也。



 自陛下承天理物,至道事神,美瑞嘉祥,洊臻狎委,非臣所能盡述。日者變生人火,損及神宮,驚惕聖
 心,震動黎庶。臣謹按《左傳》曰:「人火曰火,天火日災。」人火因人而興,故指火體而為稱;天火不知何起,直以所災言之。其名雖殊,為害不別。又《漢書·五行志》曰:「火失性則自上而降,及濫焰妄起,災宗廟,燒宮館。」自上而降,所謂天火;濫焰妄起,所謂人火。其來雖異,為患實同。王者舉措營為,必關幽顯。幽為天道,顯為人事,幽顯跡通,天人理合。今工匠宿藏其火,本無放燎之心:明堂教化之宮,復非延火之所。孽煨潛扇,倏忽成災,雖則因人,亦關神理。臣愚以為火發既先從麻主,後及總章,意將
 所營佛舍,恐勞而無益。但崇其教,即是津梁,何假紺宮,言存汲引?既僻在明堂之後,又前逼牲牢之筵,兼以厥構崇大,功多難畢。立像弘法,本擬利益黎元,傷財役人,卻且煩勞家國。承前大風摧木,天誡已顯;今者毒焰冥熾,人孽復彰。聖人動作,必假天人之助,一興功役,二者俱違,厥應昭然,殆將緣此。



 臣以為明堂是正陽之位,至尊所居,展禮班常,崇化立政,玉帛朝會,神靈依憑。營之可曰大功,損之實非輕事,既失嚴禋之所,復傷孝理之
 情。陛下昨降明制,猶申寅畏之旨,群僚理合兢畏震悚,勉力司存,豈合承恩耽樂,安然酺宴?又下人感荷聖德,睹變憎惶,神體克寧,豈非深悅。但以火氣初止,尚多驚懼,餘憂未息,遽以歡事遏之。臣恐憂喜相爭,傷於情理。故傳曰:「可憂而為樂,取憂之道。」又古者有火,祭四墉。四墉,積陰之氣,祈之以禳火災。火,陽之氣,歡樂陽事,火氣方勝,不可復興陽事。臣聞災變之興,至聖不免,聿修其德,來患可禳。陛下垂制博訪,許陳至理。而左史張鼎以
 為「今既火流王屋,彌顯大周之祥」,通事舍人逢敏奏稱,「當彌勒初成佛道時,有天魔燒宮,七寶臺須臾散壞。」斯實諂妄之邪言,實非君臣之正論。晻昧王化,無益萬機。夫天道雖高,其察彌近;神心雖寂,其聽彌聰。交際皇王,事均影響。今大風烈火,譴告相仍,實天人丁寧,匡諭聖主,便鴻基盆固,天祿永終之意也。伏願陛下乾乾在慮,翼翼為懷,若涉巨川,如承大祭,審其致災之理,詳其降眚之由,無瞢天人之心,而興不急之役。則兆人蒙賴,福
 祿靡窮,幸甚,幸甚。



 則天尋令依舊規制重造明堂,凡高二百九十四尺,東西南北廣三百尺。上施寶鳳,俄以火珠代之。明堂之下,圜饒施鐵渠,以為闢雍之象。天冊萬歲二年三月,重造明堂成,號為通天宮。四月朔日,又行親享之禮,大赦,改元為萬歲通天。翼日,則天禦通天宮之端扆殿,命有司讀時令,布政於群後。其年,鑄銅為九州鼎,既成,置於明堂之庭,各依方位列焉。神都鼎高一丈八尺,受一千八百石。冀州鼎名武興,雍州鼎名長安,
 兗州名日觀,青州名少陽,徐州名東原,揚州名江都,荊州名江陵,梁州名成都。其八州鼎高一丈四尺,各受一千二百石。司農卿宗晉卿為九鼎使,都用銅五十六萬七百一十二斤。鼎上圖寫本州山川物產之像,仍令工書人著作郎賈膺福、殿中丞薛昌容、鳳閣主事李元振、司農錄事鐘紹京等分題之,左尚方署令曹元廓圖畫之。鼎成,自玄武門外曳入,令宰相、諸王率南北衙宿衛兵十餘萬人,並仗內大牛、白象共曳之。則天自為《曳鼎歌》,
 令相唱和。其時又造大儀鐘,斂天下三品金,竟不成。九鼎初成,欲以黃金千兩塗之。納言姚璹曰:「鼎者神器,貴於質樸,無假別為浮飾。臣觀其狀,光有五彩輝煥錯雜其間,豈待金色為之炫耀?」乃止。其年九月,又大享於通天宮。以契丹破滅,九鼎初成,大赦。改元為神功。



 聖歷元年正月,又親享及受朝賀。尋制:每月一日於明堂行告朔之禮。司禮博士闢閭仁住奏議曰:



 謹按經史正文,無天子每月告朔之事。惟《禮記·玉藻》云:「天子聽朔於南門
 之外。」《周禮·天官·太宰》:「正月之吉,布政於邦國都鄙。」干寶注云:「周正建子之月,告朔日也。」此即《玉藻》之聽朔矣。今每歲首元日,於通天宮受朝,讀時令,布政事,京官九品以上、諸州朝集使等咸列於庭,此則聽朔之禮畢,而合於《周禮》、《玉藻》之文矣。而鄭玄注《玉藻》「聽朔」,以秦制月令有五帝五官之事,遂云:「凡聽朔,必特牲告其時帝及其神,配以文王、武王。」此鄭注之誤也。故漢魏至今莫之用。按《月令》云「其帝太昊,其神勾芒」者,謂宣布時令,告示下人,其
 令詞云其帝其神耳。所以為敬授之文,欲使人奉其時而務其業。每月有令,故謂之《月令》,非謂天子月朔日以祖配帝而察告之。其每月告朔者,諸候之禮也。故《春秋左氏傳》曰:「公既視朔,遂登觀臺。」又鄭注《論語》云:「禮,人君每月告朔於廟,有祭謂之朝享。魯自文公始不視朔。」是諸候之禮明矣。今王者行之,非所聞也。按鄭所謂告其帝者即太昊等五人帝,其神者即重黎等五行官。雖並功施於人,列在祀典,無天子每月拜祭告朔之文。臣等謹
 檢《禮論》及《三禮義宗》、《江都集禮》、《貞觀禮》、《顯慶禮》及祠令,並無天子每月告朔之事。若以為代無明堂,故無告朔之禮,則《江都集禮》、《貞觀禮》、《顯慶禮》及祠令,著祀五方上帝於明堂,即《孝經》「宗祀文王於明堂」也。此則無明堂而著其享祭,何為告朔獨闕其文?若以君有明堂即合告朔,則周、秦有明堂,而經典正文,無天子每月告朔之事。臣等歷觀今古,博考載籍,既無其禮,不可習非。望請停每月一日告朔之祭,以正國經。竊以天子之尊,而用
 諸侯之禮,非所謂頒告朔、令諸侯、使奉而行之之義也。



 鳳閣侍郎王方慶又奏議曰:



 謹按明堂,天子布政之宮也。蓋所以順天氣,統萬物,動法於兩儀,德被於四海者也。夏曰世室,殷曰重屋,姬曰明堂,此三代之名也。明堂,天子太廟,所以宗祀其祖,以配上帝。東曰青陽,南曰明堂,西曰總章,北曰玄堂,中曰太室。雖有五名,而以明堂為主。漢代達學通儒,咸以明堂、太廟為一。漢左中郎將蔡邕立議,亦以為然。取其宗祀,則謂之清廟;取其正室,則謂之太室;取其向陽,則
 謂之明堂;取其建學,則謂之太學;取其圜水,則謂之闢雍。異名而同事,古之制也。天子以孟春正月上辛日,於南郊總受十二月之政,還藏於祖廟,月取一政班於明堂。諸侯孟春之月,朝於天子,受十二月之政藏於祖廟,月取一政而行之。蓋所以和陰陽、順天道也。如此則禍亂不作,災害不生矣。故仲尼美而稱之曰:「明王之以孝理天下也。」人君以其禮告廟,則謂之告朔;聽視此月之政,則謂之視朔,亦曰聽朔。雖有三名,其實一也。



 今禮官
 議稱「經史正文無天子每月告朔之事」者。臣謹按《春秋》:「文公六年閏十月,不告朔。」《穀梁傳》曰:「閏,附月之餘日,天子不以告朔。」《左氏傳》云:「閏月不告朔,非禮也。閏以正時,時以作事,事以厚生,生人之道,於是乎在矣。不告閏朔,棄時政也。」臣據此文,則天子閏月亦告朔矣。寧有他月而廢其禮者乎?博考經籍,其文甚著。何以明之?《周禮·太史》職云:「頒告朔於邦國。閏月,告王居門終月。」又《禮記·玉藻》云:「閏月則合門左扉,立於其中。」並是天子閏月而行告
 朔之事也。



 禮官又稱:「《玉藻》,『天子聽朔於南門之外。』《周禮·天官·太宰》,『正月之吉,布政於邦國都鄙。』干寶注云,『周正建子之月,告朔日也。』此即《玉藻》之聽朔矣。今每歲首元日,通天宮受朝,讀時令,布政事,京官九品以上、諸州朝集使等咸列於庭,此聽朔之禮畢,而合於《周禮》、《玉藻》之文矣。《禮論》及《三禮義宗》、《江都集禮》、《貞觀禮》、《顯慶禮》及祠令,無王者告朔之事者。臣謹按《玉藻云》:「玄冕而朝日於東門之外,聽朔於南門之外。」鄭注云:「朝日,春分之時也。
 東門、南門,皆謂國門也。明堂在國之陽,每月就其時之堂而聽朔焉,卒事,反宿於路寢。凡聽朔,必以特牲告其時帝及其神,配以文王、武王。」臣謂今歲首元日,通天宮受朝,讀時令及布政,自是古禮孟春上辛,受十二月之政藏於祖廟之禮耳,而月取一政,班於明堂,其義昭然,猶未行也。即如禮官所言,遂闕其事。



 臣又按《禮記·月令》,天子每月居青陽、明堂、總章、玄堂,即是每月告朔之事。先儒舊說,天子行事,一年十八度入明堂:大享不問卜,一入也;每月告朔,十二入也;四時迎
 氣,四入也;巡狩之年,一入也。今禮官立義,王惟歲首一入耳,與先儒既異,臣不敢同。鄭玄云:「凡聽朔告其帝。」臣愚以為告朔之日,則五方上帝之一帝也。春則靈威仰,夏則赤熛怒,秋則白招拒,冬則葉光紀,季月則含樞紐也,並以始祖而配之焉。人帝及神,列在祀典,亦於其月而享祭之。魯自文公始不視朔,子貢見其禮廢,欲去其羊,孔子以羊存猶可識其禮,羊亡其禮遂廢,故云:「爾愛其羊,我愛其禮。」



 漢承秦滅學,庶事草創,明堂、闢雍,其制
 遂闕。漢武帝封禪,始造明堂於太山,既不立於京師,所以無告朔之事。至漢平帝元始中,王莽輔政,庶幾復古,乃建明堂、闢雍焉。帝祫祭於明堂,諸侯王、列侯、宗室子弟九百餘人助祭畢,皆益戶、賜爵及金帛、增秩、補吏各有差。漢末喪亂,尚傳其禮。爰至後漢,祀典仍存。明帝永平二年,郊祀五帝於明堂,以光武配,祭牲各一犢,奏樂如南郊。董卓西移,載籍湮滅,告朔之禮,於此而墜。暨於晉末,其馬生郊,禮樂衣冠,掃地總盡。元帝過江,是稱狼
 狽,禮樂制度,南遷蓋寡,彞曲殘缺,無復舊章,軍國所資,臨事議之。既闕明堂,寧論告朔。宋朝何承天纂集其文,以為《禮論》,雖加編次,事則闕如。梁代崔靈恩撰《三禮義宗》,更無異文。《貞觀》、《顯慶禮》及祠令不言告朔者,蓋為歷代不傳,其文遂闕,各有由緒,不足依據。今禮官引為明證,在臣誠實有疑。



 陛下肇建明堂,聿遵古典,告朔之禮,猶闕舊間,欽若稽古,應須補葺。
 若每月聽政於明堂,事亦煩數,孟月視朔,恐不可廢。



 上又命奉常廣集眾儒,取方慶、仁住所奏,議定得失。當時大儒成均博士吳揚吾、太學博士郭山惲曰:「臣等謹按《周禮》、《禮記》及《三傳》,皆有天子告朔之禮。夫天子頒告朔於諸侯,秦政焚滅《詩》、《書》,由是告朔禮廢。今明堂肇建,總章新立,紹百王之絕軌,樹萬代之鴻規,上以嚴配祖宗,下以敬授人時,使人知禮樂,道適中和,災害不生,禍亂不作。今若因循頒朔,每月依行,禮貴隨時,事須沿革。望
 依王方慶議,用四時孟月日及季夏於明堂修復告朔之禮,以頒天下。其帝及神,亦請依方慶用鄭玄義,告五時帝於明堂上。則嚴配之道,通於神明;至孝之德,光於四海。」制從之。



 長安四年,始制:「元日明堂受悲,停讀時令。」中宗即位,神龍元年九月,親享明堂,合祭天地,以高宗配。禮畢,曲赦京師。明年駕入京,於季秋大享,復就圓丘行事,迄於睿宗之世。



 開元二年八月,太子賓客薛謙光獻《九鼎銘》。其《蔡州鼎銘》,天后御撰,曰:「羲、農首出,軒、昊膺
 期。唐、虞繼踵,湯、禹乘時。天地光宅,域中雍熙。上天降鑒,方建隆基。」紫微令姚崇奏曰:「聖人啟運,休兆必彰。請宣付史館。」從之。五年正月,幸東都,將行大享之禮。太常少卿王仁忠、博士馮宗陳貞節等議,以武氏所造明堂,有乖典制,奏議曰:



 明堂之建,其所從來遠矣!自天垂象,聖人則之。蒿柱茅簷之規,上圓下方之制,考之大數,不逾三七之間,定之方中,必居丙巳之地者,豈非得房心布政之所,當太微上帝之宮乎?故仰葉俯從,正名定位,人
 神不雜,各司其序,則嘉應響至,保合太和。



 昔漢氏承秦,經籍道息,旁求湮墜,詳究難明。孝武初,議立明堂於長安城南,遭竇太后不好儒術,事乃中廢。孝成之代,又欲立於城南,議其制度,莫之能決。至孝平元始四年,始創造於南郊,以申嚴配。光武中元元年,立於國城之南。自魏、晉迄於梁朝,雖規制或殊,而所居之地,常取丙巳者,斯蓋百王不易之道也。



 高宗天皇大帝纂承平之運,崇樸素之風,四夷來賓,九有咸乂。永徽三年,詔禮官學士
 議明堂制度,群儒紛競,各執異端,久之不決,因而遂止者,何也?非謂財不足、力不堪也。將以周、孔既遙,禮經且紊,事不師古,或爽天心,難用作程,神不孚祐者也。則天太后總禁闈之政,藉軒臺之威,屬皇室中圮之期,躡和熹從權之制。以為乾元大殿,承慶小寢,當正陽亭午之地,實先聖聽斷之宮。表順端闈,儲精營室,爰從朝享,未始臨御。乃起工徒,挽令摧覆。既毀之後,雷聲隱然,眾庶聞之,或以為神靈感動之象也。於是增土木之麗,因府
 庫之饒,南街北闕,建天樞大儀之制;乾元遺趾,興重閣層樓之業。煙焰蔽日,梁柱排雲,人斯告勞,天實貽誡。煨燼甫爾,遽加修復。況乎地殊丙巳,未答靈心,跡匪膺期,乃申嚴配。事昧彞典,神不昭格。此其不可者一也。又明堂之制,木不鏤,土不文。今體式乖宜,違經紊禮,雕鐫所及,窮侈極麗。此其不可者二也。高明爽塏,事資虔敬,密邇宮掖,何以祈天?人神雜擾,不可放物。此其不可者三也。況兩京上都,萬方取則,而天子闕當陽之位,聽政居
 便殿之中,職司其憂,豈容沉默。當須審考歷之計,擇煩省之宜,不便者量事改修,可因者隨宜適用,削彼明堂之號,克復乾元之名,則當寧無偏,人識其舊矣。



 詔令所司詳議奏聞。



 刑部尚書王志愔等奏議,咸以此堂所置,實乖典制,多請改削,依舊造乾元殿。乃下詔曰:「古之操插皇綱、執大象者,何嘗不上稽天道,下順人極,或變通以隨時,爰損益以成務。且衢室創制,度堂以筵,用之以禮神,是光孝享,用之以布政,蓋稱視朔,先王所以厚人倫、
 感天地者也。少陽有位,上帝斯歆,此則神貴於不黷,禮殷於至敬。今之明堂,俯鄰宮掖,此之嚴祀,有異肅恭,茍非憲章,將何軌物?由是禮官博士、公卿大夫,廣參群議,欽若前古,宜存露寢之式,用罷闢雍之號。可改為乾元殿,每臨御宜依正殿禮。」自是駕在東都,常以元日冬至於乾元殿受朝賀。季秋大享祀,依舊於圓丘行事。十年,復題乾元殿為明堂,而不行享祀之禮。二十五年,駕在西京,詔將作大匠康紵素往東都毀之。紵素以毀拆勞人,
 乃奏請且拆上層,卑於舊制九十五尺。又去柱心木,平座上置八角樓,樓上有八龍,騰身捧火珠。又小於舊制,周圍五尺,覆以真瓦,取其永逸。依舊為乾元殿。



\end{pinyinscope}