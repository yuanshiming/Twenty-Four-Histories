\article{卷一百三十二}

\begin{pinyinscope}

 ○段秀實子伯倫顏真卿子頵曾孫弘式



 段秀實,字成公,隴州陽人也。祖達,左衛中郎。父行琛,洮州司馬,以秀實贈揚州大都督。秀實性至孝,六歲,母疾,水漿不入口七日,疾有間,然後飲食。及長,沉厚有斷。



 天寶四載,安西節度馬靈察署為別將,從討護蜜有功,授安西府別將。七載,高仙芝代靈察,舉兵圍怛邏斯,黑衣救至,仙芝大衄,軍士相失。夜中聞都將李嗣業之聲,因大呼責之曰:「軍敗而求免,非丈夫也。」嗣業甚慚,遂與秀實收合散卒,復得成軍。師還,嗣業請於仙芝,以秀實為判官,授斥候府果毅。十二載,封常清代仙芝,討大勃律,師次賀薩勞城,戰而勝。常清逐之,秀實進曰:「賊兵羸,餌我也,請備左右,搜其山林。」遂殲其伏,改綏德府折
 沖。肅宗即位於靈武,徵安西兵節度使梁宰,宰潛懷異圖。秀實謂嗣業曰:「豈有天子告急,臣下晏然,信浮妄之說,豈明公之意耶?」嗣業遂見宰,請發兵,從之。乃出步騎五千,令嗣業統赴朔方,以秀實為援,累有戰功。而秀實父歿,哀毀過禮。嗣業既授節制,思秀實如失左右手,表請起復,為義王友,充節度判官。



 安慶緒奔鄴,嗣業與諸軍圍之,安西輜重委於河內。乃奏秀實為懷州長史,知軍州,加節度留後。諸軍進戰於愁思岡,嗣業為流矢所
 中,卒於軍,眾推安西兵馬使荔非元禮代之。秀實聞嗣業之喪,乃遺先鋒將白孝德書,令發卒護嗣業喪送河內。秀實率將吏哭待於境,傾私財以奉葬事。元禮多其義,奏試光祿少卿,依前節度判官。



 邙山之敗,軍徙翼城,元禮為麾下所殺,將佐亦多遇害,而秀實獨以智全。眾推白孝德為節度使,人心稍定。又遷試光祿卿,為孝德判官。孝德改鎮邠寧,奏秀實試太常卿、支度營田二副使。大軍西遷,所過掠奪。又以邠寧乏食,難於饋運,乃請
 軍於奉天。是時公廩亦竭,縣吏憂恐多逃匿,群行剽盜,孝德不能禁。秀實私曰:「使我為軍候,當不如此。」軍司馬言之,遂以秀實為都虞候,權知奉天行營事,號令嚴一,軍府安泰,代宗聞而嗟賞久之。兵還於邠寧,復為都虞候,尋拜涇州刺史。



 大歷元年,馬璘奏加開府儀同三司。軍中有能引二十四弓而犯盜者,璘欲免之,秀實曰:「將有私愛,則法令不一,雖韓、白復生,亦不能為理。」璘善其議,竟使殺之。璘決事有不合理者,必固爭之,得璘引過
 乃已。璘城涇州,秀實掌留後,歸還,加御史中丞。璘既奉詔徙鎮涇州,其士眾嘗自四鎮、北庭赴難中原,僑居驟移,頗積勞怨。刀斧將王童之因人心動搖,導以為亂。或告其事,且曰:「候嚴,警鼓為約矣。」秀實乃召鼓人,陽怒失節,且戒之曰:「每更籌盡,必來報。」每白之,輒延數刻,四更畢而曙。既差互,童之亂不能作。明日,告者復曰:「今夜將焚草場,期救火者同作亂。」秀實使嚴加警備。夜半火發,乃使令於軍中曰:「救火者斬。」童之居外營,請入救火,不
 許。明日斬之,捕殺其黨凡十餘人以徇,曰:「敢後徙者族!」於是遷涇州。既至其理所,人煙夐絕,且無廩食。朝廷憂之,遂詔璘遙管鄭、潁二州,以贍涇原軍,俾秀實為留後,二州甚理。璘思其績用,又奏行軍司馬,兼都知兵馬使。



 八年,吐蕃來寇,戰於鹽倉,我軍不利。璘為寇戎所隔,逮暮未還,敗將潰兵爭道而入。時都將焦令諶與諸將四五輩狼狽而至,秀實召讓之曰:「兵法:失將,麾下當斬。公等忘其死而欲安其家耶!」令諶等恐懼,下拜數十。秀實
 乃悉驅城中士卒未出戰者,使驍將統之,東依古原,列奇兵示賊將戰,且以收合敗亡。蕃眾望之,不敢逼。及夜,璘方獲歸。十一年,璘疾甚,不能視事,請秀實攝節度副使兼左廂兵馬使。秀實乃以十將張羽飛為招召將,分兵按甲,以備非常。璘卒,而軍中行哭赴喪事於內,李漢惠接賓客於外,非其親不得居喪側,族談離立者捕而囚之。都虞候史廷幹、裨將崔珍張景華謀作亂,秀實乃送廷幹於京師,徙珍及景華外鎮,軍中遂定,不戮一人。
 尋拜秀實涇州刺史、兼御史大夫,四鎮北庭行軍涇原鄭潁節度使。三四年間,吐蕃不敢犯塞,清約率易,遠近稱之。非公會,不聽樂飲酒,私室無妓媵,無贏財,退公之後,端居靜慮而已。德宗嗣位,就加檢校禮部尚書、張掖郡王。



 建中元年,宰相楊炎欲行元載舊志,築原州城,開陵陽渠,詔中使上聞,仍問秀實可否之狀。秀實以為方春不可興土功,請俟農隙。炎以其沮己之謀,遂除司農卿,以邠寧節度李懷光兼涇原節度使,以事西拓。無何,
 劉文喜叛,亦不果城。



 四年,硃泚盜據宮闕,源休教泚偽迎鑾駕,陰濟逆志。泚乃遣其將韓旻領馬步三千疾趨奉天。時蒼黃之中,未有武備。泚以秀實嘗為涇原節度,頗得士心,後罷兵權,以為蓄憤且久,必肯同惡,乃召與謀議。秀實初詐從之,陰說大將劉海賓、何明禮、姚令言判官岐靈岳同謀殺泚,以兵迎乘輿。三人者,皆秀實夙所獎遇,遂皆許諾。及韓旻追駕,秀實以為宗社之危,期於頃刻,乃使人走諭靈岳,竊令言印。不遂,乃倒用司農
 印印符以追兵。旻至駱驛得符,軍人亦莫辯其印文,惶遽而回。秀實謂海賓等曰:「旻之來,吾黨無遺類矣!我當直搏殺泚,不得則死,終不能向此賊稱臣。」乃與海賓約,事急為繼,而令明禮應於外。明日,泚召秀實議事,源休、姚令言、李忠臣、李子平皆在坐。秀實戎服,與泚並膝,語至僭位,秀實勃然而起,執休腕奪其象笏,奮躍而前,唾泚面大罵曰:「狂賊,吾恨不斬汝萬段,我豈逐汝反耶!」遂擊之。泚舉臂自捍,才中其顙,流血匍匐而走。兇徒愕然,初
 不敢動;而海賓等不至,秀實乃曰:「我不同汝反,何不殺我!」兇黨群至,遂遇害焉。海賓、明禮、靈岳相次被殺。德宗在奉天聞其事,惜其委用不至,垂涕久之。



 初,秀實見禁兵寡少,不足以備非常,乃上疏曰:「臣聞天子曰萬乘,諸侯日千乘,大夫曰百乘,此蓋以大制小,以十制一也。尊君卑臣,強幹弱枝之義,在於此矣。今外有不庭之虜,內有梗命之臣,竊觀禁兵不精,其數全少,卒有患難,將何待之!且猛虎所以百獸畏者,為爪牙也。若去其爪牙,則
 犬彘馬牛悉能為敵。伏願少留聖慮,冀裨萬一。」及涇原兵作亂,召神策六軍,遂無一人至者。秀實守節不二,竟歿於賊,其明略義烈如此。



 興元元年二月,詔曰:「見危致命之謂忠,臨義有勇之謂烈。惟爾勵臣節,不憚殺身;惟予式嘉乃勛,懋昭大典。曰臺不德,罔克若天,遘茲殷憂,變起都邑。惟爾卿士,嗷然靡依,逼畏所加,淄澠共混。故開府儀同三司、檢校禮部尚書、兼司農卿、上柱國、張掖郡王段秀實,操行嶽立,忠厚精至,義形於色,勇必有
 仁。頃者嘗鎮涇原,克著威惠,叛卒知訓,咨爾以誠。賊泚藏奸,欺爾以詐。守人臣之大節,見元惡之深情,端委國門,挺身白刃。誓碎兇渠之首,以敵君父之仇,視死如歸,履虎致咥。噫,天未悔禍,事乖垂成,雄風壯圖,振駭群盜。昔王蠋守死以全節,周顗正色而抗詞,惟我信臣,無愧前哲。聲震寰宇,義冠古今,足以激勵人倫,光昭史冊。不有殊等之賞,孰表非常之功。爰議疇庸,特超檢限,著之甲令,樹此風聲。可贈太尉,謚曰忠烈,宣付史官,仍賜實
 封五百戶、莊宅各一區。長子與三品正員官,諸子並與五品正員官。仍廢朝三日,收京城之後,以禮葬祭,旌表門閭。朕承天子人,臨馭億兆,一夫不獲,時予之辜,況誠信不達,屢致寇戎,使抱義之臣陷於兇逆。有臨危致命,歿而逾彰;有因事成功,權以合道。茍利社稷,存亡一致,酬報之典,豈限常倫。並委所司訪其事跡,續具條奏,當加褒異,錫其井賦。圖形雲閣,書功鼎彞,以彰我有服節死義之臣,傳於不朽。」德宗還京,又詔曰:「贈太尉秀實,授
 乎貞烈,激其頹風,蒼黃之中,密蘊雄斷。將紓國難,詭收寇兵,撓其兇謀,果集吾事。挺身徑進,奮擊渠魁,英名凜然,振邁千古。宜差官致祭,並旌表門閭,緣葬所須,一切官給。仍於墓所官為立碑,以揚徽烈。」自貞元後累朝凡赦書節文褒獎忠烈,必以秀實為首。



 其子伯倫,累官至太子詹事。大和二年正月奏:「亡父贈太尉秀實,準前後制敕令所司置廟立碑,今營造已畢,取今月二十五日行升祔禮。」詔曰:「秀實忠衛宗社,功配廟食,義風所激,千
 載凜然。間代勛力,須異等夷,宜賜綾絹五百疋,以度支物充。仍令所司供少牢,並給鹵簿人夫,兼太常博士一人檢校。」尋加伯倫檢校左散騎常侍,兼殿中監。大和四年十一月,遷右金吾衛大將軍、兼御史大夫,充街使。八年七月,檢校工部尚書,充福建等州都團練觀察使,入為太僕卿,卒。宰臣李石奏曰:「伯倫,秀實之子。自古歿身以衛社稷者,無如秀實之賢。」文宗憫然曰:「伯倫宜加賻贈。」仍輟朝一日,以禮忠臣之嗣。



 顏真卿,字清臣,瑯邪臨沂人也。五代祖之推,北齊黃門侍郎。真卿少勤學業,有詞藻,尤工書。開元中,舉進士,登甲科。事親以孝聞。四命為監察御史,充河西隴右軍試覆屯交兵使。五原有冤獄,久不決,真卿至,立辯之。天方旱,獄決乃雨,郡人呼之為「御史雨」。又充河東朔方試覆屯交兵使。有鄭延祚者,母卒二十九年,殯僧舍垣地,真卿劾奏之,兄弟三十年不齒,天下聳動。遷殿中侍御史、東都畿採訪判官,轉侍御史、武部員外郎。楊國忠怒其
 不附己,出為平原太守。



 安祿山逆節頗著,真卿以霖雨為托,修城浚池,陰料丁壯,儲廩實,乃陽會文士,泛舟外池,飲酒賦詩。或讒於祿山,祿山亦密偵之,以為書生不足虞也。無幾,祿山果反,河朔盡陷,獨平原城守具備,乃使司兵參軍李平馳奏之。玄宗初聞祿山之變,嘆曰:「河北二十四郡,豈無一忠臣乎!」得平來,大喜,顧左右曰:「朕不識顏真卿形狀何如,所為得如此!」祿山初尚移牒真卿,令以平原、博平軍屯七千人防河津,以博平太守張
 獻直為副。真卿乃募勇士,旬日得萬人,遣錄事參軍李擇交統之簡閱,以刁萬歲、和琳、徐浩、馬相如、高抗朗等為將。祿山既陷洛陽,殺留守李心妻、御史中丞盧奕、判官蔣清,以三首遣段子光來徇河北。真卿恐搖人心,乃許謂諸將曰:「我識此三人,首皆非也。」遂腰斬子光,密藏三首。異日,乃取三首冠飾,草續支體,棺斂祭殯,為位慟哭,人心益附。祿山遣其將李飲湊、高邈、何千年等守土門。真卿從父兄常山太守杲卿與長史袁履謙謀殺湊、邈,
 擒千年送京師。土門既開,十七郡同日歸順,共推真卿為帥,得兵二十餘萬,橫絕燕、趙。詔加真卿戶部侍郎,依前平原太守。



 清河客李萼,年二十餘,與郡人來乞師,謂真卿曰:「聞公義烈,首唱大順,河朔諸郡恃公為長城。今清河,實公之西鄰也,僕幸寓家,得其虛實,知可為長者用。今計其蓄積,足以三平原之富,士卒可以二平原之強。公因而撫之,腹心輔車之郡,其他小城,運之如臂使指耳。唯公所意,誰敢不從。」真卿借兵千人。萼將去,真卿
 謂之曰:「兵出也,吾子何以教我?」萼曰:「今聞朝廷使程千里統眾十萬自太行東下,將出郭口,為賊所扼,兵不得前。今若先伐魏郡,斬袁知泰,太守司馬垂使為西南主;分兵開郭口之路,出千里之兵使討鄴、幽陵;平原、清河合同志十萬之眾徇洛陽,分兵而制其沖。計王師亦不下十萬,公當堅壁,無與挑戰,不數十日,賊必潰而相圖矣。」真卿然之,乃移牒清河等郡,遣其大將李擇交、副將平原縣令範東馥、裨將和琳、徐浩等進兵,與清河四千
 人合勢,而博平以千人來,三郡之師屯於博平,去堂邑縣西南十里。袁知泰遣其將白嗣深、乙舒蒙等以二萬人來拒戰,賊大敗,斬首萬餘級。肅宗幸靈武,授工部尚書、兼御史大夫、河北採訪招討使。祿山乘虛遣史思明、尹子奇急攻河北諸郡,饒陽、河間、景城、東安相次陷沒,獨平原、博平、清河三郡城守,然人心危蕩,不可復振。



 至德元年十月,棄郡渡河,歷江淮、荊襄。二年四月,朝於鳳翔,授憲部尚書,尋加御史大夫。中書舍人兼吏部侍郎
 崔漪帶酒容入朝,諫議大夫李何忌在班不肅,真卿劾之;貶漪為右庶子,何忌西平郡司馬。元帥廣平王領朔方蕃漢兵號二十萬來收長安,出辭之日,百僚致謁於朝堂。百僚拜,答拜,辭亦如之。王當闕不乘馬,步出木馬門而後乘。管崇嗣為王都虞候,先王上馬,真卿進狀彈之。肅宗曰:「朕兒子每出,諄諄教誡之,故不敢失禮。崇嗣老將,有足疾,姑欲優容之,卿勿復言。」乃以奏狀還真卿。雖天子蒙塵,典法不廢。洎鑾輿將復宮闕,遣左司郎中
 李巽先行,陳告宗廟之禮,有司署祝文,稱「嗣皇帝」。真卿謂禮儀使崔器曰:「上皇在蜀,可乎?」器遽奏改之。中旨宣勞,以為名儒深達禮體。時太廟為賊所毀,真卿奏曰:「春秋時,新宮災,魯成公三日哭。今太廟既為盜毀,請築壇於野,皇帝東向哭,然後遣使。」竟不能從。軍國之事,知無不言。為宰相所忌,出為同州刺史,轉蒲州刺史。為御史唐旻所構,貶饒州刺史。旋拜升州刺史、浙江西道節度使,徵為刑部尚書。李輔國矯詔遷玄宗居西宮,真卿乃
 首率百僚上表請問起居,輔國惡之,奏貶蓬州長史。



 代宗嗣位,拜利州刺史,遷戶部侍郎,除荊南節度使,未行而罷,除尚書左丞。車駕自陜將還,真卿請皇帝先謁五陵、九廟而後還宮。宰相元載謂真卿曰:「公所見雖美,其如不合事宜何?」真卿怒,前曰:「用舍在相公耳,言者何罪?然朝廷之事,豈堪相公再破除耶!」載深銜之。旋改檢校刑部尚書知省事,累進封魯郡公。時元載引用私黨,懼朝臣論奏其短,乃請:百官凡欲論事,皆先白長官,長官
 白宰相,然後上聞。真卿上疏曰:



 御史中丞李進等傳宰相語,稱奉進止:「緣諸司官奏事頗多,朕不憚省覽,但所奏多挾讒毀;自今論事者,諸司官皆須先白長官,長官白宰相,宰相定可否,然後奏聞者。」臣自聞此語已來,朝野囂然,人心亦多衰退。何則?諸司長官皆達官也,言皆專達於天子也。郎官、御史者,陛下腹心耳目之臣也。故其出使天下,事無巨細得失,皆令訪察,回日奏聞,所以明四目、達四聰也。今陛下欲自屏耳目,使不聰明,則天
 下何述焉。《詩》云:「營營青蠅,止於棘。讒言罔極,交亂四國。」以其能變白為黑,變黑為白也。詩人深惡之,故曰:「取彼讒人,投畀豺虎。豺虎不食,投畀有北。」則夏之伯明、楚之無極、漢之江充,皆讒人也,孰不惡之?陛下惡之,深得君人之體矣。陛下何不深回聽察,其言虛誣者,則讒人也,因誅殛之;其言不虛者,則正人也,因獎勵之。陛下舍此不為,使眾人皆謂陛下不能明察,倦於聽覽,以此為辭,拒其諫諍,臣竊為陛下痛惜之。



 臣聞太宗勤於聽覽,庶
 政以理,故著《司門式》云:「其有無門籍人,有急奏者,皆令監門司與仗家引奏,不許關礙。」所以防壅蔽也。並置立仗馬二匹,須有乘騎便往,所以平治天下,正用此道也。天寶已後,李林甫威權日盛,群臣不先諮宰相輒奏事者,仍托以他故中傷,猶不敢明約百司,令先白宰相。又閹官袁思藝日宣詔至中書,玄宗動靜,必告林甫,先意奏請,玄宗驚喜若神。以此權柄恩寵日甚,道路以目。上意不下宣,下情不上達,所以漸致潼關之禍,皆權臣誤
 主,不遵太宗之法故也。陵夷至於今日,天下之蔽,盡萃於聖躬,豈陛下招致之乎?蓋其所從來者漸矣。自艱難之初,百姓尚未凋紘,太平之理,立可便致。屬李輔國用權,宰相專政,遞相姑息,莫肯直言。大開三司,不安反側,逆賊散落,將士北走黨項,合集士賊,至今為患。偽將更相驚恐,因思明危懼,扇動卻反。又今相州敗散,東都陷沒,先帝由此憂勤,至於損壽,臣每思之,痛切心骨。



 今天下兵戈未戢,瘡磐未平,陛下豈得不日聞讜言以廣視
 聽,而欲頓隔忠讜之路乎!臣竊聞陛下在陜州時,奏事者不限貴賤,務廣聞見,乃堯、舜之事也。凡百臣庶以為太宗之理,可翹足而待也。臣又聞君子難進易退,由此言之,朝廷開不諱之路,猶恐不言,況懷厭怠,令宰相宣進止,使御史臺作條目,不令直進。從此人人不敢奏事,則陛下聞見,只在三數人耳。天下之士,方鉗口結舌,陛下後見無人奏事,必謂朝廷無事可論,豈知懼不敢進,即林甫、國忠復起矣。凡百臣庶,以為危殆之期,又翹足
 而至也。如今日之事,曠古未有,雖李林甫、楊國忠猶不敢公然如此。今陛下不早覺悟,漸成孤立,後縱悔之無及矣!臣實知忤大臣者,罪在不測,不忍孤負陛下,無任懇迫之至。



 其激切如此。於是中人爭寫內本布於外。



 後攝祭太廟,以祭器不修言於朝,載坐以誹謗,貶硤州別駕、撫州湖州刺史。元載伏誅,拜刑部尚書。代宗崩,為禮儀使。又以高祖已下七聖謚號繁多,乃上議請取初謚為定。袁傪以諂言排之,遂罷。楊炎為相,惡之,改太子少
 傅,禮儀使如舊,外示崇寵,實去其權也。



 盧杞專權,忌之,改太子太師,罷禮儀使,諭於真卿曰:「方面之任,何處為便?」真卿候杞於中書曰:「真卿以褊性為小人所憎,竄逐非一。今已羸老,幸相公庇之。相公先中丞傳首至平原,面上血真卿不敢衣拭,以舌舐之,相公忍不相容乎?」杞矍然下拜,而含怒心。會李希烈陷汝州,杞乃奏曰:「顏真卿四方所信,使諭之,可不勞師旅。」上從之,朝廷失色,李勉聞之,以為失一元老,貽朝廷羞,乃密表請留。又遣逆
 於路,不及。



 初見希烈,欲宣詔旨,希烈養子千餘人露刃爭前迫真卿,將食其肉。諸將叢繞慢罵,舉刃以擬之,真卿不動。希烈遽以身蔽之,而麾其眾,眾退,乃揖真卿就館舍。因逼為章表,令雪己,願罷兵馬。累遣真卿兄子峴與從吏凡數輩繼來京師。上皆不報。每於諸子書,令嚴奉家廟,恤諸孤而已。希烈大宴逆黨,召真卿坐,使觀倡優斥黷朝政為戲,真卿怒曰:「相公,人臣也,奈何使此曹如是乎?」拂衣而起,希烈慚,亦呵止。時硃滔、王武俊、田悅、
 李納使在坐,目真卿謂希烈曰:「聞太師名德久矣,相公欲建大號,而太師至,非天命正位?欲求宰相,孰先太師乎?」真卿正色叱之曰:「是何宰相耶!君等聞顏杲卿無?是吾兄也。祿山反,首舉義兵,及被害,詬罵不絕於口。吾今生向八十,官至太師,守吾兄之節,死而後已,豈受汝輩誘脅耶!」諸賊不敢復出口。希烈乃拘真卿,令甲士十人守,掘方丈坎於庭,曰「坑顏」,真卿怡然不介意。後張伯儀敗績於安州,希烈令賚伯儀旌節首級言誇示真卿,真卿
 慟哭投地。後其大將周曾等謀襲汝州,因回兵殺希烈,奉真卿為節度。事洩,希烈殺曾等,遂送真卿於龍興寺。真卿度必死,乃作遺表,自為墓志、祭文,常指寢室西壁下云:「吾殯所也。」希烈既陷汴州,僭偽號,使人問儀於真卿,真卿曰:「老夫耄矣,曾掌國禮,所記者諸侯朝覲禮耳。」



 興元元年,王師復振,逆賊慮變起蔡州,乃遣其將辛景臻、安華至真卿所,積柴庭中,沃之以油,且傳逆詞曰:「不能屈節,當自燒。」真卿乃投身赴火,景臻等遽止之,復告
 希烈。德宗復宮闕,希烈弟希倩在硃泚黨中,例伏誅。希烈聞之怒。興元元年八月三日,乃使閹奴與景臻等殺真卿。先曰:「有敕」。真卿拜,奴曰:「宜賜卿死。」真卿曰:「老臣無狀,罪當死,然不知使人何日從長安來?」奴曰:「從大梁來。」真卿罵曰:「乃逆賊耳,何敕耶!」遂縊殺之,年七十七。



 及淮、泗平,貞元元年,陳仙奇使護送真卿喪歸京師。德宗痛悼異常。廢朝五日,謚曰文忠。復下詔曰:「君臣之義,生錄其功,歿厚其禮,況才優匡國,忠至滅身。朕自興嘆,勞於
 寤寐。故光祿大夫、守太子太師、上柱國、魯郡公顏真卿,器質天資,公忠傑出,出入四朝,堅貞一志。屬賊臣擾亂,委以存諭,拘肋累歲,死而不撓,稽其盛節,實謂猶生。朕致貽斯禍,慚悼靡及,式崇嘉命,兼延爾嗣。可贈司徒,仍賜布帛五百端。男頵、碩等喪制終,所司奏超授官秩。」貞元六年十一月南郊,赦書節文授真卿一子五品正員官,故頵得錄用。文宗詔曰:「朕每覽國史,見忠烈之臣,未嘗不嗟嘆久之,思有以報。如聞從覽、弘式,實杲卿、真卿之孫。
 永惟九原,既不可作,旌其嗣續,諒協典彞。考績已深於宦途者,命列於中臺;官次未齒於搢紳者,俾佐於左輔。庶使天下再新義風。」以真卿曾孫弘式為同州參軍。



 國,是武之英也;茍無楊炎弄權,若任之為將,遂展其才,豈有硃泚之禍焉!如清臣富於學,守其正,全其節,昌文之傑也;茍無盧杞惡直,若任之為相,遂行其道,豈有希烈之叛焉!夫國得賢則安,
 失賢則危。德宗內信奸邪,外斥良善,幾致危亡,宜哉。噫,「仁以為己任,不亦重乎;死而後已,不亦遠乎!」二君守道歿身,為時垂訓,希代之士也,光文武之道焉。



 贊曰:自古皆死,得正為順。二公云亡,萬代垂訓。



\end{pinyinscope}