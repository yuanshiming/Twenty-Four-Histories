\article{卷六十六 列傳第十六 房玄齡(子遺直 遺愛) 杜如晦(弟楚客 叔淹)}

\begin{pinyinscope}

 ○房玄齡子遺直遺愛杜如晦弟楚客叔淹



 房喬,字玄齡,齊州臨淄人。曾祖翼,後魏鎮遠將軍、宋安郡守,襲壯武伯。祖熊,字子繹,褐州主簿。父彥謙,好學,通涉《五經》,隋涇陽令,《隋書》有傳。玄齡幼聰敏,博覽經史,工
 草隸,善屬文。嘗從其父至京師,時天下寧晏,論者咸以國祚方永,玄齡乃避左右告父曰:「隋帝本無功德,但誑惑黔黎,不為後嗣長計,混諸嫡庶,使相傾奪,諸後籓枝,競崇淫侈,終當內相誅夷,不足保全家國。今雖清平,其亡可翹足而待。」彥謙驚而異之。年十八,本州舉進士,授羽騎尉。吏部侍郎高孝基素稱知人,見之深相嗟挹,謂裴矩曰:「僕閱人多矣,未見如此郎者。必成偉器,但恨不睹其聳壑凌霄耳。」父病綿歷十旬,玄齡盡心藥膳,未嘗
 解衣交睫。父終,酌飲不入口者五日。後補隰城尉。會義旗入關,太宗徇地渭北,玄齡杖策謁於軍門,溫彥博又薦焉。太宗一見,便如舊識,署渭北道行軍記室參軍。玄齡既遇知己,罄竭心力,知無不為。賊寇每平,眾人競求珍玩,玄齡獨先收人物,致之幕府。及有謀臣猛將,皆與之潛相申結,各盡其死力。



 既而隱太子見太宗勛德尤盛,轉生猜間。太宗嘗至隱太子所,食,中毒而歸,府中震駭,計無所出。玄齡因謂長孫無忌曰:「今嫌隙已成,禍機
 將發,天下心匈恟,人懷異志。變端一作,大亂必興,非直禍及府朝,正恐傾危社稷。此之際會,安可不深思也!僕有愚計,莫若遵周公之事,外寧區夏,內安宗社,申孝養之禮。古人有云,『為國者不顧小節』,此之謂歟!孰若家國淪亡,身名俱滅乎?」無忌曰:「久懷此謀,未敢披露,公今所說,深會宿心。」無忌乃入白之。太宗召玄齡謂曰:「阽危之兆,其跡已見,將若之何?」對曰:「國家患難,今古何殊。自非睿聖欽明,不能安輯。大王功蓋天地,事鐘壓紐,神贊所在,
 匪藉人謀。」因與府屬杜如晦同心戮力。仍隨府遷授秦王府記室,封臨淄侯;又以本職兼陜東道大行臺考功郎中,加文學館學士。玄齡在秦府十餘年,常典管記,每軍書表奏,駐馬立成,文約理贍,初無稿草。高祖嘗謂侍臣曰:「此人深識機宜,足堪委任。每為我兒陳事,必會人心,千里之外,猶對面語耳。」隱太子以玄齡、如晦為太宗所親禮,甚惡之,譖之於高祖,由是與如晦並被驅斥。隱太子將有變也,太宗令長孫無忌召玄齡及如晦,令衣
 道士服,潛引入閣計事。及太宗入春宮,擢拜太子右庶子,賜絹五千匹。貞觀元年,代蕭瑀為中書令。論功行賞,以玄齡及長孫無忌、杜如晦、尉遲敬德、侯君集五人為第一,進爵邢國公,賜實封千三百戶。太宗因謂諸功臣曰:「朕敘公等勛效,量定封邑,恐不能盡當,各許自言。」皇從父淮安王神通進曰:「義旗初起,臣率兵先至。今房玄齡、杜如晦等刀筆之吏,功居第一,臣竊不服。」上曰:「義旗初起,人皆有心。叔父雖率得兵來,未嘗身履行陣。山東
 未定,受委專征,建德南侵,全軍陷沒。及劉黑闥翻動,叔父望風而破。今計勛行賞,玄齡等有籌謀帷幄、定社稷之功,所以漢之蕭何,雖無汗馬,指蹤推轂,故得功居第一。叔父於國至親,誠無所愛,必不可緣私,濫與功臣同賞耳。」初,將軍丘師利等咸自矜其功,或攘袂指天,以手畫地,及見神道理屈,自相謂曰:「陛下以至公行賞,不私其親,吾屬何可妄訴?」三年,拜太子少師,固讓不受,攝太子詹事,兼禮部尚書。明年,代長孫無忌為尚書左僕射,
 改封魏國公,監修國史。既任總百司,虔恭夙夜,盡心竭節,不欲一物失所。聞人有善,若己有之。明達吏事,飾以文學,審定法令,意在寬平。不以求備取人,不以己長格物,隨能收敘,無隔卑賤。論者稱為良相焉。或時以事被譴,則累日朝堂,稽顙請罪,悚懼踧,若無所容。九年,護高祖山陵制度,以功加開府儀同三司。十一年,與司空長孫無忌等十四人並代襲刺史,以本官為宋州刺史,改封梁國公,事竟不行。十三年,加太子少師,玄齡頻表請
 解僕射,詔報曰:「夫選賢之義,無私為本;奉上之道,當仁是貴。列代所以弘風,通賢所以協德。公忠肅恭懿,明允篤誠。草昧霸圖,綢繆帝道。儀刑黃閣,庶政惟和;輔翼春宮,望實斯著。而忘彼大體,徇茲小節,雖恭教諭之職,乃辭機衡之務,豈所謂弼予一人,共安四海者也?」玄齡遂以本官就職。時皇太子將行拜禮,備儀以待之,玄齡深自卑損,不敢修謁,遂歸於家。有識者莫不重其崇讓。玄齡自以居端揆十五年,女為韓王妃,男遺愛尚高陽公
 主,實顯貴之極,頻表辭位,優詔不許。十六年,又與士廉等同撰《文思博要》成,錫賚甚優。進拜司空,仍綜朝政,依舊監修國史。玄齡抗表陳讓,太宗遣使謂之曰:「昔留侯讓位,竇融辭榮,自懼盈滿,知進能退,善鑒止足,前代美之。公亦欲齊蹤往哲,實可嘉尚。然國家久相任使,一朝忽無良相,如失兩手。公若筋力不衰,無煩此讓。」玄齡遂止。十八年,與司徒長孫無忌等圖形於凌煙閣,贊曰:「才兼藻翰,思入機神。當官勵節,奉上忘身。」高宗居春宮,加
 玄齡太子太傅,仍知門下省事,監修國史如故。尋以撰《高祖、太宗實錄》成,降璽書褒美,賜物一千五百段。其年,玄齡丁繼母憂去職,特敕賜以昭陵葬地。未幾,起復本官。太宗親征遼東,命玄齡京城留守,手詔曰:「公當蕭何之任,朕無西顧之憂矣。」軍戎器械,戰士糧廩,並委令處分發遣。玄齡屢上言敵不可輕,尤宜誡慎。尋與中書侍郎褚遂良受詔重撰《晉書》,於是奏取太子左庶子許敬宗、中書舍人來濟、著作郎陸元仕、劉子翼、前雍州刺史
 令狐德棻、太子舍人李義府、薛元超、起居郎上官儀等八人,分功撰錄,以臧榮緒《晉書》為主,參考諸家,甚為詳洽。然史官多是文詠之士,好採詭謬碎事,以廣異聞;又所評論,競為綺艷,不求篤實,由是頗為學者所譏。唯李淳風深明星歷,善於著述,所修《天文》、《律歷》、《五行》三志,最可觀採。太宗自著宣、武二帝及陸機、王羲之四論,於是總題云御撰。至二十年,書成,凡一百三十卷,詔藏於秘府,頒賜加級各有差。



 玄齡嘗因微譴歸第,黃門侍郎褚
 遂良上疏曰:「君為元首,臣號股肱,龍躍雲興,不嘯而集,茍有時來,千年朝暮。陛下昔在布衣,心懷拯溺,手提輕劍,仗義而起。平諸寇亂,皆自神功,文經之助,頗由輔翼。為臣之懃,玄齡為最。昔呂望之扶周武,伊尹之佐成湯,蕭何關中,王導江外,方之於斯,可以為匹。且武德初策名伏事,忠勤恭孝,眾所同歸。而前宮、海陵,憑兇恃亂,乾時事主,人不自安。居累卵之危,有倒懸之急,命視一刻,身縻寸景,玄齡之心,終始無變。及九年之際,機臨事迫,
 身被斥逐,闕於謨謀,猶服道士之衣,與文德皇后同心影助,其於臣節,自無所負。及貞觀之始,萬物惟新,甄吏事君,物論推與,而勛庸無比,委質惟舊。自非罪狀無赦,搢紳同尤,不可以一犯一愆,輕示遐棄。陛下必矜玄齡齒發,薄其所為,古者有諷諭大臣遣其致仕,自可在後,式遵前事,退之以禮,不失善聲。今數十年勛舊,以一事而斥逐,在外雲云,以為非是。夫天子重大臣,則人盡其力;輕去就,則物不自安。臣以庸薄,忝預左右,敢冒天威,
 以申管見。」二十一年,太宗幸翠微宮,授司農卿李緯為民部尚書。玄齡時在京城留守,會有自京師來者,太宗問曰:「玄齡聞李緯拜尚書如何?」對曰:「玄齡但云李緯好髭須,更無他語。」太宗遽改授緯洛州刺史,其為當時準的如此。



 二十三年,駕幸玉華宮,時玄齡舊疾發,詔令臥總留臺。及漸篤,追赴宮所,乘擔輿入殿,將至御座乃下。太宗對之流涕,玄齡亦感咽不能自勝。敕遣名醫救療,尚食每日供御膳。若微得減損,太宗即喜見顏色;如聞
 增劇,便為改容淒愴。玄齡因謂諸子曰:「吾自度危篤,而恩澤轉深,若孤負聖君,則死有餘責。當今天下清謐,咸得其宜,唯東討高麗不止,方為國患。主上含怒意決,臣下莫敢犯顏;吾知而不言,則銜恨入地。」遂抗表諫曰:



 臣聞兵惡不戢,武貴止戈。當今聖化所覃,無遠不屆,洎上古所不臣者,陛下皆能臣之,所不制者,皆能制之。詳觀今古,為中國患害者,無如突厥。遂能坐運神策,不下殿堂,大小可汗,相次束手,分典禁衛,執戟行間。其後延陀
 鴟張,尋就夷滅;鐵勒慕義,請置州縣,沙漠以北,萬里無塵。至如高昌叛渙於流沙,吐渾首鼠於積石,偏師薄伐,俱從平蕩。高麗歷代逋誅,莫能討擊。陛下責其逆亂,弒主虐人,親總六軍,問罪遼、碣。未經旬月,即拔遼東,前後虜獲,數十萬計,分配諸州,無處不滿。雪往代之宿恥,掩崤陵之枯骨,比功較德,萬倍前王。此聖心之所自知,微臣安敢備說。且陛下仁風被於率土,孝德彰於配天。睹夷狄之將亡,則指期數歲;授將帥之節度,則決機萬里。
 屈指而候驛,視景而望書,符應若神,算無遺策。擢將於行伍之中,取士於凡庸之末。遠夷單使,一見不忘;小臣之名,未嘗再問。箭穿七札,弓貫六鈞。加以留情墳典,屬意篇什,筆邁鐘、張,辭窮班、馬。文鋒既振,則管磬自諧;輕翰暫飛,則花FM競發。撫萬姓以慈,遇群臣以禮。褒秋毫之善,解吞舟之網。逆耳之諫必聽,膚受之訴斯絕。好生之德,焚障塞於江湖;惡殺之仁,息鼓刀於屠肆。鳧鶴荷稻粱之惠,犬馬蒙帷蓋之恩。降乘吮思摩之瘡,登堂臨
 魏徵之柩。哭戰亡之卒,則哀動六軍;負填道之薪,則精感天地。重黔黎之大命,特盡心於庶獄。臣心識昏憒,豈足論聖功之深遠,談天德之高大哉!陛下兼眾美而有之,靡不備具,微臣深為陛下惜之重之,愛之寶之。《周易》曰:「知進而不知退,知存而不知亡,知得而不知喪。」又曰:「知進退存亡,不失其正者,惟聖人乎!」由此言之,進有退之義,存有亡之機,得有喪之理,老臣所以為陛下惜之者,蓋此謂也。老子曰:「知足不辱,知止不殆。」謂陛下威名
 功德,亦可足矣;拓地開疆,亦可止矣。彼高麗者,邊夷賤類,不足待以仁義,不可責以常禮。古來以魚鱉畜之,宜從闊略。若必欲絕其種類,恐獸窮則搏。且陛下每決一死囚,必令三覆五奏,進素食、停音樂者,蓋以人命所重,感動聖慈也。況今兵士之徒,無一罪戾,無故驅之於行陣之間,委之於鋒刃之下,使肝腦塗地,魂魄無歸,令其老父孤兒、寡妻慈母,望轊車而掩泣,抱枯骨以摧心,足以變動陰陽,感傷和氣,實天下冤痛也。且兵者兇器,戰
 者危事,不得已而用之。向使高麗違失臣節,陛下誅之可也;侵擾百姓,而陛下滅之可也;久長能為中國患,而陛下除之可也。有一於此,雖日殺萬夫,不足為愧。今無此三條,坐煩中國,內為舊王雪恥,外為新羅報仇,豈非所存者小,所損者大?願陛下遵皇祖老子止足之誡,以保萬代巍巍之名。發霈然之恩,降寬大之詔,順陽春以布澤,許高麗以自新。焚凌波之船,罷應募之眾,自然華夷慶賴,遠肅邇安。臣老病三公,旦夕入地,所恨竟無塵
 露,微增海岳。謹罄殘魂餘息,預代結草之誠。倘蒙錄此哀鳴,即臣死且不朽。



 太宗見表,謂玄齡子婦高陽公主曰:「此人危惙如此,尚能憂我國家。」後疾增劇,遂鑿苑墻開門,累遣中使候問。上又親臨,握手敘別,悲不自勝。皇太子亦就之與之訣。即日授其子遺愛右衛中郎將,遺則中散大夫,使及目前,見其通顯。尋薨,年七十。廢朝三日,冊贈太尉、並州都督,謚曰文昭,給東園秘器,陪葬昭陵。玄齡嘗誡諸子以驕奢沉溺,必不可以地望凌人,故
 集古今聖賢家誡,書於屏風,令各取一具,謂曰:「若能留意,足以保身成名。」又云:「袁家累葉忠節,是吾所尚,汝宜師之。」高宗嗣位,詔配享太宗廟庭。



 子遺直嗣,永徽初為禮部尚書、汴州刺史。次子遺愛,尚太宗女高陽公主,拜駙馬都尉,官至太府卿、散騎常侍。初,主有寵於太宗,故遺愛特承恩遇,與諸主婿禮秩絕異。主既驕恣,謀黜遺直而奪其封爵,永徽中誣告遺直無禮於己。高宗令長孫無忌鞫其事,因得公主與遺愛謀反之狀。遺愛伏誅,
 公主賜自盡,諸子配流嶺表。遺直以父功特宥之,除名為庶人。停玄齡配享。



 杜如晦,字克明,京兆杜陵人也。曾祖皎,周贈開府儀同、大將軍、遂州刺史。高祖徽,周河內太守。祖果,周溫州刺史,入隋,工部尚書、義興公,《周書》有傳。父吒,隋昌州長史。如晦少聰悟,好談文史。隋大業中以常調預選,吏部侍郎高孝基深所器重,顧謂之曰:「公有應變之才,當為棟梁之用,願保崇令德。今欲俯就卑職,為須少祿俸耳。」遂
 補滏陽尉,尋棄官而歸。太宗平京城,引為秦王府兵曹參軍,俄遷陜州總管府長史。時府中多英俊,被外遷者眾,太宗患之。記室房玄齡曰:「府僚去者雖多,蓋不足惜。杜如晦聰明識達,王佐才也。若大王守籓端拱,無所用之;必欲經營四方,非此人莫可。」太宗大驚曰:「爾不言,幾失此人矣!」遂奏為府屬。後從征薛仁杲、劉武周、王世充、竇建德,嘗參謀帷幄。時軍國多事,剖斷如流,深為時輩所服。累遷陜東道大行臺司勛郎中,封建平縣男,食邑
 三百戶。尋以本官兼文學館學士。天策府建,以為從事中郎,畫象於丹青者十有八人,而如晦為冠首,令文學褚亮為之贊曰:「建平文雅,休有烈光。懷忠履義,身立名揚。」其見重如此。隱太子深忌之,謂齊王元吉曰:「秦王府中所可憚者,唯杜如晦與房玄齡耳。」因譖之於高祖,乃與玄齡同被斥逐。後又潛入畫策,及事捷,與房玄齡功等,擢拜太子左庶子,俄遷兵部尚書,進封蔡國公,賜實封千三百戶。貞觀二年,以本官檢校侍中,攝吏部尚書,
 仍總監東宮兵馬事,號為稱職。三年,代長孫無忌為尚書右僕射,仍知選事,與房玄齡共掌朝政。至於臺閣規模及典章人物,皆二人所定,甚獲當代之譽,談良相者,至今稱房、杜焉。如晦以高孝基有知人之鑒,為其樹神道碑以紀其德。其年冬,遇疾,表請解職,許之,祿賜特依舊。太宗深憂其疾,頻遣使存問,名醫上藥,相望於道。四年,疾篤,令皇太子就第臨問,上親幸其宅,撫之流涕,賜物千段;及其未終,見子拜官,遂超遷其子左千牛構為
 尚舍奉御。尋薨,年四十六。太宗哭之甚慟,廢朝三日,贈司空,徙封萊國公,謚曰成。太宗手詔著作郎虞世南曰:「朕與如晦,君臣義重。不幸奄從物化,追念勛舊,痛悼于懷。卿體吾此意,為制碑文也。」太宗後因食瓜而美,愴然悼之,遂輟食之半,遣使奠於靈座。又嘗賜房玄齡黃銀帶,顧謂玄齡曰:「昔如晦與公同心輔朕,今日所賜,唯獨見公。」因泫然流涕。又曰:「朕聞黃銀多為鬼神所畏。」命取黃金帶遣玄齡親送於靈所。其後太宗忽夢見如晦若
 平生,及曉,以告玄齡,言之歔欷,令送御饌以祭焉。明年如晦亡日,太宗復遣尚宮至第慰問其妻子,其國官府佐並不之罷。終始恩遇,未之有焉。子構襲爵,官至慈州刺史,坐弟荷謀逆,徙於嶺表而卒。初,荷以功臣子尚城陽公主,賜爵襄陽郡公,授尚乘奉御。貞觀中,與太子承乾謀反,坐斬。



 如晦弟楚客,少隨叔父淹沒於王世充。淹素與如晦兄弟不睦,譖如晦兄於王行滿,王世充殺之,並囚楚客,幾至餓死,楚客竟無怨色。洛陽平,淹當死,楚
 客泣涕請如晦救之。如晦初不從,楚客曰:「叔已殺大兄,今兄又結恨棄叔,一門之內,相殺而盡,豈不痛哉!」因欲自刎。如晦感其言,請於太宗,淹遂蒙恩宥。楚客因隱於嵩山。貞觀四年,召拜給事中,上謂曰:「聞卿山居日久,志意甚高,自非宰相之任,則不能出,何有是理耶?夫涉遠者必自邇,升高者必自下,但在官為眾所許,無慮官之不大。爾兄雖與我體異,其心猶一,於我國家非無大功。為憶爾兄,意欲見爾。宜識朕意,繼爾兄之忠義也。」拜楚
 客蒲州刺史,甚有能名。後歷魏王府長史,拜工部尚書,攝魏王泰府事。楚客知太宗不悅承乾,魏王泰又潛令楚客友朝臣用事者,至有懷金以賂之,因說泰聰明,可為嫡嗣。人或以聞,太宗隱而不言。及釁發,太宗始揚其事,以其兄有佐命功,免死,廢於家。尋授處化令,卒。



 如晦叔父淹。淹,字執禮。祖業,周豫州刺史。父征,河內太守。淹聰辯多才藝,弱冠有美名,與同郡韋福嗣為莫逆之交,相與謀曰:「上好用嘉遁,蘇威以幽人見征,擢居美職。」遂
 共入太白山,揚言隱逸,實欲邀求時譽。隋文帝聞而惡之,謫戍江表。後還鄉里,雍州司馬高孝基上表薦之,授承奉郎。大業末,官至御史中丞。王世充僭號,署為吏部,大見親用。及洛陽平,初不得調,淹將委質於隱太子。時封德彞典選,以告房玄齡,恐隱太子得之,長其奸計,於是遽啟太宗,引為天策府兵曹參軍、文學館學士。武德八年,慶州總管楊文乾作亂,辭連東宮,歸罪於淹及王珪、韋挺等,並流於越巂。太宗知淹非罪,贈以黃金三百
 兩。及即位,徵拜御史大夫,封安吉郡公,賜實封四百戶。以淹多識典故,特詔東宮儀式簿領,並取淹節度。尋判吏部尚書,參議朝政。前後表薦四十餘人,後多知名者。淹嘗薦刑部員外郎郅懷道,太宗因問淹:「懷道才行何如?」淹對曰:「懷道在隋日作吏部主事,甚有清慎之名。又煬帝向江都之日,召百官問去住之計。時行計已決,公卿皆阿旨請去,懷道官位極卑,獨言不可。臣目見此事。」太宗曰:「卿爾可從何計?」對曰:「臣從行計。」太宗曰:「事君之
 義,有犯無隱。卿稱懷道為是,何因自不正諫?」對曰:「臣爾日不居重任,又知諫必不從,徒死無益。」太宗曰:「孔子稱從父之命,未為孝子。故父有爭子,國有爭臣。若以主之無道,何為仍仕其世?既食其祿,豈得不匡其非?」因謂群臣曰:「公等各言諫事如何?」王珪曰:「昔比干諫紂而死,孔子稱其仁;洩冶諫而被戮,孔子曰:『民之多闢,無自立闢。』是則祿重責深,理須極諫;官卑望下,許其從容。」太宗又召淹笑謂曰:「卿在隋日,可以位下不言;近仕世充,何不
 極諫?」對曰:「亦有諫,但不見從。」太宗曰:「世充若修德從善,當不滅亡;既無道拒諫,卿何免禍?」淹無以對。太宗又曰:「卿在今日,可為備任,復欲極諫否?」對曰:「臣在今日,必盡死無隱。且百里奚在虞虞亡,在秦秦霸,臣竊比之。」太宗笑。時淹兼二職,而無清潔之譽,又素與無忌不協,為時論所譏。及有疾,太宗親自臨問,賜帛三百匹。貞觀二年卒,贈尚書右僕射,謚曰襄。子敬同襲爵,官至鴻臚少卿。敬同子從則,中宗時為蒲州刺史。



 史臣曰:房、杜二公,皆以命世之才,遭逢明主,謀猷允協,以致升平。議者以比漢之蕭、曹,信矣!然萊成之見用,文昭之所舉也。世傳太宗嘗與文昭圖事,則曰「非如晦莫能籌之」。及如晦至焉,竟從玄齡之策也。蓋房知杜之能斷大事,杜知房之善建嘉謀,裨諶草創,東裏潤色,相須而成,俾無悔事,賢達用心,良有以也。若以往哲方之,房則管仲、子產,杜則鮑叔、罕虎矣。



 贊曰:肇啟聖君,必生賢輔。猗歟二公,實開運祚。文含經
 緯,謀深夾輔。笙磬同音,唯房與杜。



\end{pinyinscope}