\article{卷七十四 列傳第二十四 劉洎 馬周 崔仁師(孫湜 湜弟液 液子論 液弟滌)}

\begin{pinyinscope}

 ○劉洎馬周崔仁師孫湜湜弟液液子論液弟滌



 劉洎,字思道,荊州江陵人也。隋末,仕蕭銑為黃門侍郎。銑令略地嶺表,得五十餘城,未還而銑敗,遂以所得城
 歸國,授南康州都督府長史。貞觀七年,累拜給事中,封清苑縣男。十五年,轉治書侍御史。上疏曰:



 尚書萬機,實為政本,伏尋此選,受授誠難。是以八座比於文昌,二丞方於管轄,爰至曹郎,上應列宿,茍非稱職,竊位興譏。伏見比來尚書省詔敕稽停,文案壅滯,臣誠雖庸劣,請述其源。貞觀之初,未有令僕,於時省務繁雜,倍多於今。左丞戴胄、右丞魏徵,並曉達吏方,質性平直,事應彈舉,無所回避。陛下又假以恩慈,自然肅物,百司匪懈,抑此之
 由。及杜正倫續任右丞,頗亦厲下。比者綱維不舉,並為勛親在位,品非其任,功勢相傾。凡在官僚,未循公道,雖欲自強,先懼囂謗。所以郎中抑奪,唯事諮稟;尚書依違,不得斷決。或憚聞奏,故事稽延。案雖理窮,仍更盤下。去無程限,來不責遲,一經出手,便涉年載。或希旨失情,或避嫌抑理。勾司以案成為事了,不究是非;尚書用便僻為奉公,莫論當否。遞相姑息,唯務彌縫。且選賢授能,非材莫舉,天工人代,焉可妄加?至於懿戚元勛,但優其禮
 秩,或年高耄及,或積病智昏,既無益於時宜,當致之以閑逸。久妨賢路,殊為不可。將救茲弊,且宜精簡四員。左右丞、左右司郎中如並得人,自然綱維略舉,亦當矯正趨競,豈唯息其稽滯哉!



 書奏未幾,拜尚書右丞。十三年,遷黃門侍郎。十七年,加授銀青光祿大夫,尋除散騎常侍。洎性疏峻敢言。太宗工王羲之書,尤善飛白,嘗宴三品已上於玄武門,帝操筆作飛白字賜群臣,或乘酒爭取於帝手,洎登御座引手得之。皆奏曰:「洎登御床,罪當
 死,請付法。」帝笑而言曰:「昔聞婕妤辭輦,今見常侍登床。」尋攝黃門侍郎,加上護軍。



 太宗善持論,每與公卿言及古道,必詰難往復。洎上書諫曰:「帝王之與凡庶,聖哲之與庸愚,上下相懸,擬倫斯絕。是知以至愚而對至聖,以極卑而對至尊,徒思自強,不可得也。陛下降恩旨,假慈顏,凝旒以聽其言,虛襟以納其說,猶恐群下未敢對揚,況動神機,縱天辯,飾辭以折其理,援古以排其議,欲令凡庶何階應答?臣聞皇天以無言為貴,聖人以不言為
 德,老君稱大辯若訥,莊生稱至道無文,此皆不欲煩也。齊侯讀書,輪扁竊笑;漢皇慕古,長孺陳譏,此亦不欲勞也。且多記則損心,多語則損氣,心氣內損,形神外勞,初雖不覺,後必為累。須為社稷自愛,豈為性好自傷乎?竊以今日升平,皆陛下力行所至,欲其長久,匪由辯博。但當忘彼愛憎,慎茲取舍,每事敦樸,無非至公,若貞觀之初則可矣。至如秦政強辯,失人心於自矜;魏文宏才,虧眾望於虛說。此才辯之累,較然可知矣。伏願略茲雄辯,
 浩然養氣;簡彼緗圖,淡焉自怡。固萬壽於南嶽,齊百姓於東戶,則天下幸甚,皇恩斯畢。」手詔答曰:「非慮無以臨下,非言無以述慮。比有談論,遂致煩多。輕物驕人,恐由茲道。形神心氣,非此為勞。今聞讜言,虛懷以改。」時皇太子初立,洎以為宜尊賢重道,上書曰:



 臣聞郊迎四方,孟侯所以成德;齒學三讓,元良由是作貞。斯皆屈主祀之尊,申下交之義。故得芻言咸薦,睿問旁通,不出軒庭,坐知天壤。率由茲道,永固鴻基者焉。原夫太子,宗祧是系,
 善惡之際,興亡斯在。不勤於始,將悔於終。是以晁錯上書,令先通政術;賈誼獻策,務前知禮教。竊惟皇太子孝友仁義,明允篤誠,皆挺自天姿,非勞審諭,固以華夷仰德,翔泳希風矣。然則寢門視膳,已表於三朝;藝宮論道,宜弘於四術。雖春秋鼎盛,飭躬有漸,實恐歲月易往,墮業興譏,取適宴安,方從此始。臣以愚短,幸參侍從,思廣離明,願聞徑術。不敢曲陳故事,請以聖德言之。



 伏惟陛下誕睿膺圖,登庸歷試。多才多藝,道著於匡時;允武允
 文,功成於纂祀。萬方即序,九圍清宴。尚且雖休勿休,日慎一日,求異聞於振古,勞睿思於當年。乙夜觀書,事高漢帝;馬上披卷,勤過魏後。陛下自勵如此,而令太子優游棄日,不習圖書,臣所未諭一也。加以暫屏機務,即寓雕蟲。綜寶思於天文,則長河韜映;摛玉字於仙札,則流霞成彩。固以錙銖萬代,冠冕百王,屈、宋不足以升堂,鐘、張何階於入室。陛下自好如此,而太子悠然靜處,不尋篇翰,臣所未諭二也。陛下歷該眾妙,獨秀寰中,猶晦天
 聽,俯詢凡識,聽朝之隙,引見群官,降以溫顏,訪以今古。故得朝廷是非,裏閭好惡,凡有巨細,必關聽覽。陛下自好如此,而令太子久入趨侍,不接正人,臣所未諭三也。陛下若謂無益,則何事勞神;若謂有成,則宜申貽厥。蔑而不急,未見其可。伏願俯推睿範,訓及儲君,授以良書,娛之嘉客。晨披經史,觀成敗於前蹤;晚接賓游,訪得失於當代。間以書札,繼以篇章,則日聞所未聞,日見所未見。副德逾光,群生之福也。古之太子,問安而退,所以廣
 敬於君父;異宮而處,所以分別於嫌疑。今太子一侍天闈,動移旬朔,師傅以下,無由接見。假令供奉有隙,暫還東宮,拜謁既疏,且事欣仰,規諫之道,固所未暇。陛下不可以親教,宮寀無由以進言,雖有具僚,竟將何補?伏願俯循前躅,稍抑下流,弘遠大之規,展師友之義。則儲徽克茂,帝圖斯廣,凡在黎元,孰不慶賴!



 自此敕洎令與岑文本同馬周遞日往東宮,與皇太子談論。太宗嘗怒苑西守監穆裕,命於朝堂斬之,皇太子遽進諫。太宗謂司
 徒長孫無忌曰:「夫人久相與處,自然染習。自朕臨御天下,虛心正直,即有魏徵朝夕進諫。自徵云亡,劉洎、岑文本、馬周、褚遂良等繼之。皇太子幼在朕膝前,每見朕心悅諫,昔者因染以成性,固有今日之諫耳。」十八年,遷侍中。太宗嘗謂侍臣曰:「夫人臣之對帝王,多順旨而不逆,甘言以取容。朕今發問,欲聞己過,卿等須言朕愆失。」長孫無忌、李勣、楊師道等咸云:「陛下聖化致太平,臣等不見其失。」洎對曰:「陛下化高萬古,誠如無忌等言。然頃上
 書人不稱旨者,或面加窮詰,無不慚退,恐非獎進言者之路。」太宗曰:「卿言是也,當為卿改之。」



 太宗征遼,令洎與高士廉、馬周留輔皇太子定州監國,仍兼左庶子、檢校民部尚書。太宗謂洎曰:「我今遠征,使卿輔翼太子,社稷安危之機,所寄尤重,卿宜深識我意。」洎進曰:「願陛下無憂,大臣有愆失者,臣謹即行誅。」太宗以其妄發,頗怪之,謂曰:「君不密則失臣,臣不密則失身。卿性疏而太健,恐以此取敗,深宜誡慎,以保終吉。」十九年,太宗遼東還,發
 定州,在道不康,洎與中書令馬周入謁。洎、周出,遂良傳問起居,洎泣曰:「聖體患癰,極可憂懼。」遂良誣奏之曰:「洎云:『國家之事不足慮,正當傅少主行伊、霍故事,大臣有異志者誅之,自然定矣。』」太宗疾愈,詔問其故,洎以實對,又引馬周以自明。太宗問周,周對與洎所陳不異。遂良又執證不已,乃賜洎自盡。洎臨引決,請紙筆欲有所奏,憲司不與。洎死,太宗知憲司不與紙筆,怒之,並令屬吏。洎文集十卷,行於時。則天臨朝,其子弘業上言洎被遂良
 譖而死,詔令復其官爵。



 馬周,字賓王,清河茌平人也。少孤貧,好學,尤精《詩》、《傳》,落拓不為州里所敬。武德中,補博州助教,日飲醇酎,不以講授為事。刺史達奚恕屢加咎責,周乃拂衣游於曹、汴,又為浚儀令崔賢所辱,遂感激西游長安。宿於新豐逆旅,主人唯供諸商販而不顧待周,遂命酒一斗八升,悠然獨酌,主人深異之。至京師,舍於中郎將常何之家。貞觀五年,太宗令百僚上書言得失,何以武吏不涉經學,
 周乃為何陳便宜二十餘事,令奏之,事皆合旨。太宗怪其能,問何,何答曰:「此非臣所能,家客馬周具草也。每與臣言,未嘗不以忠孝為意。」太宗即日召之,未至間,遣使催促者數四。及謁見,與語甚悅,令直門下省。六年,授監察御史,奉使稱旨。帝以常何舉得其人,賜帛三百匹。是歲,周上疏曰:



 微臣每讀經史,見前賢忠孝之事,臣雖小人,竊希大道,未嘗不廢卷長想,思履其跡。臣以不幸,早失父母,犬馬之養,已無所施,顧來事可為者,唯忠義而
 已。是以徒步二千里而自歸於陛下,陛下不以臣愚瞽,過垂齒錄。竊自顧瞻,無階答謝,輒以微軀丹款,惟陛下所擇。



 臣伏見大安宮在宮城之西,其墻宇宮闕之制,方之紫極,尚為卑小。臣伏以東宮皇太子之宅,猶處城中,大安乃至尊所居,更在城外。雖太上皇游心道素,志存清儉,陛下重違慈旨,愛惜人力;而蕃夷朝見及四方觀聽,有不足焉。臣願營築雉堞,修起門樓,務從高顯,以稱萬方之望,則大孝昭乎天下矣。臣又伏見明敕以二月
 二日幸九成宮。臣竊惟太上皇春秋已高,陛下宜朝夕視膳而晨昏起居。今所幸宮去京三百餘里,鑾輿動軔,嚴蹕經旬,非可以旦暮至也。太上皇情或思感,而欲即見陛下者,將何以赴之?且車駕今行,本為避署。然則太上皇尚留熱所,而陛下自逐涼處,溫凊之道,臣竊未安。然敕書既出,業已成就,願示速返之期,以開眾惑。臣又見詔書,令宗室勛賢作鎮籓部,貽厥子孫,嗣守其政,非有大故,無或黜免。臣竊惟陛下封植之者,誠愛之重之,
 欲其胤裔承守而與國無疆也。臣以為如詔旨者,陛下宜思所以安存之,富貴之,然則何用代官也。何則?以堯、舜之父,猶有硃、均之子。倘有孩童嗣職,萬一驕愚,兆庶被其殃而國家受其敗。正欲絕之也,則子文之治猶在;正欲留之也,而欒黶之惡已彰。與其毒害於見存之百姓,則寧使割恩於已亡之臣,明矣。然則向所謂愛之者,乃適所以傷之也。臣謂宜賦以茅土,疇其戶邑,必有材行,隨器方授,則雖其翰翮非強,亦可以獲免尤累。昔漢
 光武不任功臣以吏事,所以終全其代者,良得其術也。願陛下深思其事,使夫得奉大恩而子孫終其福祿也。



 臣又聞聖人之化天下,莫不以孝為基。故曰:「孝莫大於嚴父,嚴父莫大於配天。」又曰:「國之大事,在祀與戎。」孔子亦云:「吾不預祭如不祭。」是聖人之重祭祀也如此。伏惟陛下踐祚以來,宗廟之享,未曾親事。伏緣聖情,獨以鑾輿一出,勞費稍多,所以忍其孝思,以便百姓。遂使一代之史,不書皇帝入廟之事,將何以貽厥孫謀,垂則來葉?
 臣知大孝誠不在俎豆之間,然聖人之訓人,固有屈己以從時,願聖慈顧省愚款。臣又聞致化之道,在於求賢審官;為政之基,在於揚清激濁。孔子曰:「唯名與器,不以假人。」是言慎舉之為重也。臣伏見王長通、白明達本自樂工輿皁雜類,韋槃提、斛斯正則更無他材,獨解調馬。縱使術逾儕輩,伎能有取,乍可厚賜錢帛,以富其家;豈得列預士流,超授高爵?遂使朝會之位,萬國來庭,騶子倡人,鳴玉曳履,與夫朝賢君子,比肩而立,同坐而食,臣
 竊恥之。然朝命既往,縱不可追,謂宜不使在朝班,預於士伍。



 太宗深納之。尋除侍御史,加朝散大夫。十一年,周又上疏曰:



 臣歷觀前代,自夏、殷及漢氏之有天下,傳祚相繼,多者八百餘年,少者猶四五百年,皆為積德累業,恩結於人心。豈無僻王?賴前哲以免。自魏、晉以還,降及周、隋,多者不過六十年,少者才二三十年而亡。良由創業之君,不務廣恩化,當時僅能自守,後無遺德可思。故傳嗣之主,政教少衰,一夫大呼而天下土崩矣。今陛下
 雖以大功定天下,而積德日淺,固當思隆禹、湯、文、武之道,廣施德化,使恩有餘地,為子孫立萬代之基,豈欲但令政教無失,以持當年而已!然自古明王聖主,雖因人設教,寬猛隨時,而大要唯以節儉於身、恩加於人二者是務。故其下愛之如日月,畏之如雷霆,此其所以卜祚遐長而禍亂不作也。今百姓承喪亂之後,比於隋時才十分之一。而供官徭役,道路相繼,兄去弟還,首尾不絕。遠者往來五六千里,春秋冬夏,略無休時。陛下雖每有
 恩詔令其減省,而有司作既不廢,自然須人,徒行文書,役之如故。臣每訪問,四五年來,百姓頗有嗟怨之言,以為陛下不存養之。昔唐堯茅茨土階,夏禹惡衣菲食,如此之事,臣知不可復行於今。漢文帝惜百金之費,輟露臺之役,集上書囊以為殿帷,所幸慎夫人衣不曳地。至景帝以錦繡纂組妨害女功,特詔除之,所以百姓安樂。至孝武帝雖窮奢極侈,而承文、景遺德,故人心不動。向使高祖之後,即有武帝,天下必不能全。此於時代差近,
 事跡可見。今京師及益州諸處,營造供奉器物,並諸王妃主服飾,議者皆不以為儉。臣聞昧旦丕顯,後世猶怠;作法於理,其弊猶亂。陛下少處人間,知百姓辛苦,前代成敗,目所親見,尚猶如此。而皇太子生長深宮,不更外事,即萬歲之後,固聖慮所當憂也。臣尋往代以來之事,但有黎庶怨叛,聚為盜賊,其國無不即滅,人主雖改悔,未有重能安全者。凡修政教,當修於可修之時,若事變一起而後悔之,則無益者也。故人主每見前代之亡,則
 知其政教之所由喪,而皆不知其身之失。是以殷紂笑夏桀之亡,而幽、厲亦笑殷紂之滅;隋煬帝大業之初又笑齊、魏之失國。今之視煬帝,亦猶煬帝之視齊、魏也。故京房謂漢元帝云,「臣恐後之視今,亦猶今之視古」。此言不可不誡也。往者貞觀之初,率土荒儉,一匹絹才得一斗米,而天下帖然。百姓知陛下甚愛憐之,故人人自安,曾無謗讟。自五六年來,頻歲豐稔,一匹絹得粟十餘石,而百姓皆以為陛下不憂憐之,咸有怨言。又今所營為
 者,頗多不急之務故也。自古以來,國之興亡,不由積畜多少,唯在百姓苦樂。且以近事驗之,隋家貯洛口倉,而李密因之;東都積布帛,而世充據之;西京府庫,亦為國家之用,至今未盡。向使洛口、東都無粟帛,則世充、李密未能必聚大眾。但貯積者固是有國之常事,要當人有餘力而後收之,豈人勞而強斂之?更以資寇,積之無益也。然儉以息人,貞觀之初,陛下已躬為之,故今行之不難也。為之一日,則天下知之,式歌且舞矣。若人既勞矣
 而用之不息,倘中國被水旱之災,邊方有風塵之患,狂狡因之以竊發,則有不可測之事,非徒聖躬旰食晏寢而已。古語云:「動人以行不以言,應天以實不以文。」以陛下之明,誠欲勵精為政,不煩遠採上古之術,但及貞觀之初,則天下幸甚。昔賈誼為漢文帝云,可慟哭及長嘆息者,言當韓信王楚、彭越王梁、英布王淮南之時,使文帝即天子位,必不能安。又言賴諸王年少,傅相制之;長大之後,必生禍亂。歷代以來,皆以誼言為是。臣竊觀今
 諸將功臣,陛下所與定天下者,皆仰稟成規,備鷹犬之用,無威略振主,如韓、彭之難駕馭者。而諸王年並幼少,縱其長大,當陛下之日,必無他心。然即萬代之後,不可不慮。自漢、晉以來,亂天下者,何嘗不是諸王?皆為樹置失宜,不預為節制,以至於滅亡。人主熟知其然,但溺於私愛,故使前車既覆而後車不改轍也。今天下百姓極少,諸王甚多,寵遇之恩,有過厚者。臣之愚慮,不唯慮其恃恩驕矜也。昔魏武帝寵陳思,及文帝即位,防守禁閉,
 有同獄囚。以先帝加恩太多,故嗣王疑而畏之也。此則武帝寵舐思,適所以苦之也。且帝子何患不富貴?身食大國,封戶不少,好衣美食之外,更何所須?而每年加別優賜,曾無紀極。俚語曰:「貧不學儉,富不學奢」,言自然也。今大聖創業,豈唯處置見在子弟而已?當制長久之法,使萬代遵行。



 又言:



 臨天下者,以人為本。欲令百姓安樂,唯在刺史、縣令。縣令既眾,不能皆賢,若每州得良刺史,則合境蘇息;天下刺史悉稱聖意,則陛下端拱巖廊之
 上,百姓不慮不安。自古郡守、縣令,皆妙選賢德,欲有擢升宰相,必先試以臨人,或從二千石入為丞相。今朝廷獨重內官,縣令、刺史,頗輕其選。刺史多是武夫勛人,或京官不稱職,方始外出。而折沖果毅之內,身材強者,先入為中郎將,其次始補州任。邊遠之處,用人更輕,其材堪宰位,以德行見稱擢者,十不能一。所以百姓未安,殆由於此。



 疏奏,太宗稱善久之。



 先是,京城諸街,每至晨暮,遣人傳呼以警眾。周遂奏諸街置鼓,每擊以警眾,令罷
 傳呼,時人便之,太宗益加賞勞。俄拜給事中。十二年,轉中書舍人。周有機辨,能敷奏,深識事端,動無不中。太宗嘗曰:「我於馬周,暫不見則便思之。」中書侍郎岑文本謂所親曰:「吾見馬君論事多矣,援引事類,揚榷古今,舉要刪蕪,會文切理,一字不可加,一言不可減,聽之靡靡,令人亡倦。昔蘇、張、終、賈,正應此耳。然鳶肩火色,騰上必速,恐不能久耳。」十五年,遷治書侍御史,兼知諫議大夫,又兼檢校晉王府長史。王為皇太子,拜中書侍郎,兼太子
 右庶子。十八年,遷中書令,依舊兼太子右庶子。周既職兼兩宮,處事精密,甚獲當時之譽。太宗伐遼東,皇太子定州監守,令周與高士廉、劉洎留輔皇太子。太宗還,以本官攝吏部尚書。二十一年,加銀青光祿大夫。太宗嘗以神筆賜周飛白書曰:「鸞鳳凌雲,必資羽翼。股肱之寄,誠在忠良。」周病消渴,彌年不瘳。時駕幸翠微宮,敕求勝地,為周起宅。名醫中使,相望不絕,每令尚食以膳供之,太宗躬為調藥,皇太子親臨問疾。周臨終,索所陳事表
 草一帙,手自焚之,慨然曰:「管、晏彰君之過,求身後名,吾弗為也。」二十二年卒,年四十八。太宗為之舉哀,贈幽州都督,陪葬昭陵。高宗即位,追贈尚書右僕射、高唐縣公。垂拱中,配享高宗廟庭。子載,咸亨年累遷吏部侍郎,善選補,於今稱之。卒於雍州長史。



 崔仁師,定州安喜人。武德初,應制舉,授管州錄事參軍。五年,侍中陳叔達薦仁師才堪史職,進拜右武衛錄事參軍,預修梁、魏等史。貞觀初,再遷殿中侍御史。時青州
 有逆謀事發,州縣追捕反黨,俘囚滿獄,詔仁師按覆其事。仁師至州,悉去杻械,仍與飲食湯沐以寬慰之,唯坐其魁首十餘人,餘皆原免。及奏報,詔使將往決之,大理少卿孫伏伽謂仁師曰:「此獄徒侶極眾,而足下雪免者多,人皆好生,誰肯讓死?今既臨命,恐未甘心,深為足下憂也。」仁師曰:「嘗聞理獄之體,必務仁恕,故稱殺人刖足,亦皆有禮。豈有求身之安,知枉不為申理?若以一介暗短,但易得十囚之命,亦所願也。」伏伽慚而退。及敕使至
 青州更訊,諸囚咸曰:「崔公仁恕,事無枉濫,請伏罪。」皆無異辭。仁師後為度支郎中,嘗奏支庶財物數千言,手不執本,太宗怪之,令黃門侍郎杜正倫齎本,仁師對唱,一無差殊,太宗大奇之。時校書郎王玄度注《尚書》、《毛詩》,毀孔、鄭舊義,上表請廢舊注,行己所注者,詔禮部集諸儒詳議。玄度口辯,諸博士皆不能詰之。郎中許敬宗請付秘閣藏其書,河間王孝恭特請與孔、鄭並行。仁師以玄度穿鑿不經,乃條其不合大義,駁奏請罷之。詔竟依仁
 師議,玄度遂廢。十六年,遷給事中。時刑部以《賊盜律》反逆緣坐兄弟沒官為輕,請改從死,奏請八座詳議。右僕射高士廉、吏部尚書侯君集、兵部尚書李勣等議請從重;民部尚書唐儉、禮部尚書江夏王道宗、工部尚書杜楚客等議請依舊不改。時議者以漢及魏、晉謀反皆夷三族,咸欲依士廉等議。仁師獨駁曰:「自羲、農以降,爰及唐,虞,或設言而人不犯,或畫象而下知禁。三代之盛,泣辜解網,父子兄弟,罪不相及,咸臻至理,俱為稱首。及其
 世亂,獄訟滋煩,周之季年,不勝其弊,烈火原於子產,峭澗起於安於,韓、季、申、商,爭持急刻,參夷相坐,始於此也。秦用其法,遂至土崩。漢高之務寬大,未為盡善;文帝之存仁厚,仍多涼德。遂使新垣族滅,信、越菹醢,見譏良史,謂之過刑。魏、晉至隋,有損有益,凝脂猶密,秋荼尚煩。皇上爰發至仁,念茲刑憲,酌前王之令典,探往代之嘉猷,革弊蠲苛,可大可久,仍降綸綍,頒之九區。故得斷獄數簡,手足有措,刑清化洽,未有不安。忽以暴秦酷法,為隆
 周中典,乖惻隱之情,反惟行之令。進退參詳,未見其可。且父子天屬,昆季同氣,誅其父子,足累其心,此而不顧,何愛兄弟。既欲改法,請更審量。」竟從仁師駁議。後仁師密奏請立魏王為太子,忤旨,轉為鴻臚少卿,遷民部侍郎。征遼之役,詔太常卿韋挺知海運,仁師為副,仁師又別知河南水運。仁師以水路險遠,恐遠州所輸不時至海,遂便宜從事,遞發近海租賦以充轉輸。及韋挺以壅滯失期,除名為民,仁師以運夫逃走不奏,坐免官。既不
 得志,遂作《體命賦》以暢其情,辭多不載。太宗還至中山,起為中書舍人,尋兼檢校刑部侍郎。太宗幸翠微宮,仁師上《清暑賦》以諷,太宗稱善,賜帛五十段。二十二年,遷中書侍郎,參知機務。時仁師甚承恩遇,中書令褚遂良頗忌嫉之。會有伏閣上訴者,仁師不奏,太宗以仁師罔上,遂配龔州。會赦還。永徽初,起授簡州刺史,尋卒,年六十餘。神龍初,以子挹為國子祭酒,恩例贈同州刺史。挹子湜。



 湜少以文辭知名,舉進士,累轉左補闕,預修《三教
 珠英》,遷殿中侍御史。神龍初,轉考功員外郎。時桓彥範、敬暉等既知國政,懼武三思讒間,引湜為耳目,使伺其動靜。俄而中宗疏忌功臣,於三思恩寵漸厚,湜乃反以桓、敬等計議潛告三思。尋遷中書舍人。及桓、敬等徙於嶺外,湜又說三思盡宜殺之,以絕其歸望。三思問誰可使者,湜表兄周利貞先為桓、敬等所惡,自侍御史出嘉州司馬,湜乃舉充此行。桓、敬等聞利貞至,多自殺,三思引利貞為御史中丞。湜,景龍二年遷兵部侍郎,挹為禮
 部,父子同為南省副貳,有唐已來未有也。時昭容上官氏屢出外宅,湜托附之。由是中宗遇湜甚厚,俄拜吏部侍郎,尋轉中書侍郎、同中書門下平章事。與鄭愔同知選事,銓綜失序,為御史李尚隱所劾,愔坐配流嶺表,湜左轉為江州司馬。上官昭容密與安樂公主曲為申理,中宗乃以愔為江州司馬,授湜襄州刺史。未幾,入為尚書左丞。韋庶人臨朝,復為中書侍郎、同中書門下三品。睿宗即位,出為華州刺史,俄又拜太子詹事。初,湜景龍
 中獻策開南山新路,以通商州水陸之運,役徒數萬,死者十三四。仍嚴錮舊道,禁行旅,所開新路以通,竟為夏潦沖突,崩壓不通。至是追論湜開山路功,加銀青光祿大夫。俄為太平公主所引,復遷中書門下三品。先天元年,拜中書令,與劉幽求爭權不協,陷幽求徙於嶺表。仍促廣州都督周利貞以逗留殺之,不果而止。時挹以年老,累除戶部尚書致仕。挹性貪冒,受人請托,數以公事干湜,湜多違拒不從,大為時論所嗤。玄宗在東宮,數幸
 其第,恩意甚密。湜既私附太平公主,時人咸為之懼,門客陳振鷺獻《海鷗賦》以諷之,湜雖稱善而心實不悅。及帝將誅蕭至忠等,召將托為腹心,湜弟滌謂湜曰:「主上若有所問,不得有所隱也。」湜不從,及見帝,對問失旨。至忠等既誅,湜坐徙嶺外。時新興王晉亦連坐伏誅,臨刑嘆曰:「本謀此事,出自崔湜,今我就死而湜得生,何冤濫也!」俄而所司奏宮人元氏款稱與湜曾密謀進鴆,乃追湜賜死。初,湜與張說有隙,說時為中書令,議者以為說
 構陷之。時湜與尚書右丞盧藏用同配流俱行,湜謂藏用曰:「家弟承恩,或冀寬宥。」因遲留不速進。行至荊州,夢於講堂照鏡,曰:「鏡者明象,吾當為人主所明也。」以告占夢人張由,對曰:「講堂者,受法之所;鏡者,於文為『立見金』,此非吉徵。」其日追使至,縊於驛中,時年四十三。湜美姿儀,早有才名。弟液、滌及從兄蒞,並有文翰。居清要,每宴私之際,自比東晉王導、謝安之家。謂人曰:「吾之一門及出身歷官,未嘗不為第一。丈夫當先據要路以制人,豈
 能默默受制於人也!」是故進趣不已,而不以令終。



 液尤工五言之作,湜常嘆伏之曰:「海子,我家之龜也。」海子即液小名,官至殿中侍御史,坐兄配流,逃匿於郢州人胡履虛之家。作《幽征賦》以見意,辭甚典麗。遇赦還,道病卒。友人裴耀卿纂其遺文為集十卷。



 液子論,以吏乾稱。天寶中自櫟陽令遷司勛員外郎、濛陽太守。乾元後,歷典名郡,皆以理行稱。大歷末,元載以罪誅,朝廷方振起淹滯,遷同州刺史。未幾,為黜陟使庾何所按,廢免。議者以
 何舉奏涉於深刻,復用論為衢州刺史。秩滿,寓於揚、楚間。德宗以舊族耆年,授大理卿致仕卒。



 液弟滌,多辯智,善諧謔,素與玄宗款密。兄湜坐太平黨誅,玄宗常思之,故待滌逾厚,用為秘書監。出入禁中,與諸王侍宴不讓席,而坐或在寧王之上。後賜名澄。從東封還,加金紫光祿大夫,封安喜縣子。開元十四年卒,贈兗州刺史。



 史臣曰:劉洎始以章疏切直,以至位望隆顯。至於提綱整帶,咨聖嘉猷,籍國士之談,體廊廟之器。噫,樞機之發,
 榮辱之主,一言不慎,竟陷誣奏。雖君親甚悔,而駟不及舌,良足悲矣!馬周道承際會,天性深沉,悟主談微,作忠本孝,沖識廣度,宛涉穹崇。《詩》曰:「嘉樂君子,顯顯令德。」惜其中壽,不憖遺乎!崔仁師以史材獲進,其刊正褒貶,雅得詳明。至於本仁恕,申枉濫,其事可觀。沮穿鑿之注,止從重之刑,其言甚直。《書》曰「疑謀勿成」,而以魏王為請,不亦惑乎!及參機務,竟致忌嫉,罔上之名,抑有由也。崔湜之德,去祖逾遠,謂勢可恃,謂進無傷,及位極人臣,而心
 無止足。覽《海鷗賦》,知而不誡,及荊州之夢,人知不免。《易》曰:「不節之嗟,又誰咎也!」



 贊曰:驥逢造父,一日千里。英主取賢,不拘階陛。賓王徒步,洎為賊吏。一見文皇,皆登相位。



\end{pinyinscope}