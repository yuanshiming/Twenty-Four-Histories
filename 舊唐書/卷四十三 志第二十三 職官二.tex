\article{卷四十三 志第二十三 職官二}

\begin{pinyinscope}

 太師、太傅、太保各一員。謂之三師,並正一品。後漢初,太傅置府僚。至周、隋,三師不置府僚,初拜於尚書省上。隋煬帝廢三
 師之官。武德復置,一如隋制。



 三師,訓導之官,天子所師法,大抵無所統職,然非道德崇重,則不居其位。無其人,則闕之。



 太尉、司徒、司空各一員。謂之三公,並正一品。魏、晉至北齊,三公置府僚。隋初亦置府僚,尋省府僚,初拜於尚書省上,唐因之。武德初,太宗為之,其後親王拜三公,皆不視事,祭祀則攝者行也。



 三公,論道之官也。蓋以佐天子理陰陽,平邦國,無所不統,故不以一職名其官。大祭祀,則太尉亞獻,司徒奉俎,司空掃除。



 尚書都省龍朔二年,改為中臺,光宅元年,改為文昌臺,神龍初復。



 尚書省領二十四司。六尚書,各分領四司。



 尚書令一員。正二品。武德中,太宗為之,自是闕而不置。令總領百官,儀刑端揆。其屬有六尚書:一曰吏部,二曰戶部,三曰禮部,四曰兵部,五曰刑部,六曰工部。凡庶務,皆會而決之。



 左右僕射各一員,從二品。龍朔二年,改為左右匡,光宅元年,改為文昌左右相,開元元年,改為左右丞相,天寶元年,復為左右僕射。



 掌統理六官,綱紀庶務,以貳令之職。自不置令,僕射總判省事。御史糾劾不當,兼得彈之。



 左右丞各一員。左丞,正四品上。右丞,
 正四品下。龍朔改為左右肅機,咸亨復,永昌元年,升為從三品也,如意元年,復四品也。



 左丞掌管轄諸司,糾正省內,勾吏部、戶部、禮部十二司,通判都省事。若右丞闕,則並行之。右丞管兵部、刑部、工部十二司。若
 左闕,
 右丞兼知其事。御史有糾劾不當,兼得彈之。



 左右
 司郎中各一員。並
 從五品
 上。隋置,武德初省。貞觀初,復置。龍朔二年,改為左右丞務,咸亨復也。



 左司郎中,副左丞所管諸司事,省署鈔目,勘稽失,知省內宿直之事。若右司郎中闕,則並行之。左右司員外郎各一員。天后永昌元年,置左右司員外郎各一人。神龍初省,後復置。



 左右司郎中、員外郎各掌副十有二司之事,以舉正稽違,省署符目焉。



 凡都省掌舉諸司之綱紀與百僚之程式,以正邦理,以宣邦教。凡上之
 所以迨下,其制有六,曰制、敕、冊、令、教、符。



 天子曰制,曰敕,曰冊。皇太子曰令。親王、公主曰教。尚書省下於州,州下縣,縣下鄉,皆曰符也。凡下之所以達上,其制亦有六,曰表、狀、箋、啟、辭、牒。



 表上於天子。其近臣亦為狀。箋、啟上皇太子,然於其長亦為之。非公文所施,有品已上公文,皆曰牒。庶人言曰辭也。



 諸司自相質問,其義有三:關、刺、移。關,謂關通其事;刺,謂刺舉之;移,謂移其事於他司。移則通判之官皆連署也。



 凡內外百司所受之事,皆印其發日,為之程限。凡尚書省施行制敕,案成則給程以鈔之。若急速者,不出其日。若諸州計奏達於京師,量事之大小與多少,以為之節。凡京師
 諸司,有符、移、關、牒下諸州者,必由於都省以遣之。凡文案既成,勾司行硃施訖,皆書其上端,記年月日,納諸庫。凡施行公文應印者,監印之官考其事目無差,然後印之,必書於歷。每月終納諸庫。凡尚書省官,每日一人宿直。都司報廢直簿,轉以為次。凡內外百僚,日出而視事,既午而退,有事則直官省之。其務繁,不在此例。凡天下制敕計奏之數。省符宣告之節,率以歲終為斷。京師諸司,皆以四月一日納於都省。其天下諸州,則本司推校,以授
 勾官。勾官審之。連署封印,附計帳,使納於都省。常以六月一日,都事集諸司令史對覆。若有隱漏不同,皆附於考課焉。



 主事六人,從九品上。令史十八人,書令史三十六人,亭長六人,掌固十四人。



 凡令史掌案文簿,亭長、掌固檢校省門戶倉庫事陳設之事也。



 吏部尚書一員,正三品。龍朔二年,改為司列太常伯,光宅元年,改為天官尚書,神龍復為吏部尚書也。



 侍郎二員。正四品上。隋煬帝大業三年,尚書六曹,各置侍郎一人,以貳尚書之職,並正四品。國家定令,諸曹侍郎降為正四品下,唯吏部侍郎為正四品上。龍朔改為司列少常伯,咸亨復。總章元年,吏
 部、兵部各增置侍郎一員也。



 尚書、侍郎之職,掌天下官吏選授、勛封、考課之政令。其屬有四:一曰吏部,二曰司封,三曰司勛,四曰考功。總其職務,而行其制命。凡中外百司之事,由於所屬,皆質正焉。凡選授之制,每歲集於孟冬。去王城五百里之內以上旬,千里之內以中旬,千里之外以下旬。尚書、侍郎,分為三銓。



 尚書為尚書銓,侍郎二人分為中銓、東銓也。凡擇人以四才,校功以三實。四才,謂身、言、書、判。其優長者,有可取焉。三實,謂德行、才用、勞效,德均以才,才均以勞,勞必考其實而進退之。



 較之優劣,而定其留放,所以正
 權衡,明與奪,抑貪冒,進賢能,然後據其官資,量其注擬。五品已上,以名上中書門下,聽制授其官。六品已下,量資任定。其才職頗高,可擢為拾遣、補闕、監察御史者,亦以名送中書門下,聽敕授。



 其有歷職清要,考第頗深者,得隔品授之,不然即否。凡出身非清流者,不注清資官。凡注官,若官資未相當,及以為非便者,聽至三注。凡伎術之官,皆本司定,送吏部附甲。凡同司聯事勾檢之官,皆不得注大功已上親。凡皇親諸親及軍功,兼注員外郎。凡注擬,必先具官階團甲,送門下以聞。



 注官,階高擬卑曰「行」,階卑擬高曰「守」。三銓注擬訖,皆
 當銓團甲,過左右僕射。若中銓、東銓,則過尚書訖,乃上門下省。給事中讀,黃門侍郎省,侍中審,然後進甲以聞,聽旨授而施行焉。若左右僕射門下批官不當者,別改注,亦有重執而上者也。



 凡大選,終於季春之月,若選人有身在軍旅,則軍中試書判,封送吏部。亦有春中下解而後集,謂之春選。若優勞人,有敕則有處分及即與官者,並聽非時選,一百日內注擬之。



 所以定九流之品格,補萬方之闕政,官人之道備焉。



 郎中二員,並從五品上。龍朔為司列大夫,咸亨、光宅並隋曹改也。員外郎二員。



 並從六品上。令史三十人,書令史六十人,亭長八人,掌固十二人。郎中一人,掌考天下文吏之班秩階品。凡敘階二十有九,品在都序,自一品至
 九品,品有上下,凡散官四品已下,九品已上,並於吏部當番上下。



 其應當番四十五日。若都省須人送符,諸司須人者,並兵部、吏部散官上。經兩番已上,聽簡入選。不第者,依番名不過五六也。



 凡敘階之法,有以封爵,有以親戚,有以勛庸,有以資廕,有以秀孝,有以勞考,有除免而復敘者,皆循法以申之,無或枉冒。凡應入三品五品者,皆待別制而進之,不然則否。凡文武百僚之班序,官同者先爵,爵同者先齒。凡京司有常參官、謂五品以上職事官、八品已上供奉官、員外郎、監察御史、太常博士。供奉官、兩省自侍中、中書令已下,盡名供奉官。



 諸司長官、清
 望官、四品已下八品已上清官。每日以六品已上清官兩人,待制於衙。供奉官、宿衛官不在此例。



 凡授四品已下清望官,才職相當,不應進讓。凡職事官應覲省及移疾,不得過程。年七十已上應致仕,若齒力未衰,亦聽厘務。凡官人身及同居大功已上親,自執工商,家專其業,及風疾、使酒,皆不得入仕。凡內外官有清白著聞,應以名薦,則中書門下改授,五品已上,量加升進,六品已下,有付吏部即量等第遷轉。若第二第三等人,五品已上,改日稍優之。
 六品已下,秩滿聽選,不在放限。其嶺南、黔中,三年一置選補使,號為南選。凡天下官吏,各有常員。凡諸司置直,皆有定數。



 諸司諸色有品直官。內外官吏,則有假寧之節,行李之命。簿書景跡,功賞殿最,具員皆與員外郎分掌之。郎中一人掌小銓,亦分為九品,通謂之行署。以其在九流之外,故謂之流外銓,亦謂之小選。其校試銓注,與流內銓略同。其吏部、兵部、禮部、考功、都省、御史臺、中書、門下,謂之前八司,其餘則曰後行。凡擇流外,取工書、計,兼頗曉時
 務。



 三事中,有一優長,則在敘限。每經三考轉選,量其才能而進之,不則從舊任。小銓,舊委郎中專知。開元二十五年,又敕銓試訖留放,皆尚書侍郎定之也。



 員外郎一人掌判南曹。曹在選曹之南,故謂之南曹。每歲選人,有解狀、簿書、資歷、考課,必由之以核其實,乃上三銓。其三銓進甲則署焉。員外郎一人裳判曹務。凡預太廟齋郎帖試,如貢舉之制。



 司封郎中一員,從五品上。隋曰主爵郎,武德因之。龍朔二年改為司封大夫,光宅改司封郎中也。



 司封員外郎一員,從六品上。主事二人,從九品上。



 令史四人,書
 令史九人,掌固四人。司封郎中、員外郎之職,掌國之封爵,凡有九等。一曰王,正一品,食邑一萬戶。二曰郡王,從一品,食邑五千戶。三曰國公,從一品,食邑三千戶。四曰郡公,正二品,食邑二千戶。五曰縣公,從二品,食邑一千五百戶。六曰縣侯,從三品,食邑一千戶。七曰縣伯,正四品,食邑七進戶。八曰縣子,正五品,食邑五百戶。九曰縣男,從五品,食邑三百戶。



 凡名山大川,及畿內諸縣,皆不以封。至郡公有餘爵,聽回授子孫。其國公皆特封。凡天下觀有定數。每觀立三綱,以道德高者充。凡三元諸齋日,修金錄、明真等齋。凡道士、女道士簿籍,三年一造。凡外命婦之制,皇之姑,封大長
 公主,皇姊妹,封長公主,皇女,封公主,皆視正一品。皇太子之女,封郡主,視從一品。王之女,封縣主,視正二品。王母妻,為妃。一品及國公母妻,為國夫人。三品已上母妻,為郡夫人。四品母妻,為郡君。五品若勛官,三品有封,母妻為縣君。散官並同職事。勛官四品有封,母妻為鄉君。其母邑號,皆加「太」字,各視其夫、子之品。若兩有官爵者,從其高。若內命婦,一品之母,為正四品郡君;二品之母,為從四品郡君;三品四品之母,並為正五品縣君。凡婦人,不因夫及子而
 別加邑號,夫人云某品夫人,郡君為某品郡君,縣君、鄉君亦然。凡庶子,有五品已上官,皆封嫡母。無嫡母,封所生母。凡二王後夫人,職事五品已上,散官三品已上,王及國公母妻,朝參各視其夫及子之禮。凡親王,孺人二人,視正五品,媵十人,視正六品。嗣王、郡王及一品,媵十人,視從六品。二品,媵八人,視正七品。三品及國公,媵六人,視從七品。四品,媵四人,視正八品。五品,媵三人,視從八品。降此外皆為妾。凡皇家五等親,及諸親三等,存亡
 升降,皆立簿書籍,每三年一造。除附之制,並載於宗正寺。



 司勛郎中一員,從五品上。隋曰司勛郎,武德初乃加「中」字。龍朔改為司勛大夫,咸亨復也。



 司勛員外郎二員,從六品上。主事四人,從九品上。



 令史三十三人,書令史六十人,掌固四人。郎中、員外郎之職,掌邦國官人之勛級。凡勛,十有二轉為上柱國,比正二品。十一轉為柱國,比從二品。十轉為上護軍,比正三品。九轉為護軍,比從三品。八轉為上輕車都尉,比正四品。七轉為輕車都尉,比從四品。六轉為上騎都尉,比正五品。五轉
 為騎都尉,比從五品。四轉為驍騎尉,比正六品。三轉為飛騎尉,比從六品。二轉為雲騎尉,比正七品。一轉為武騎尉,比從七品。凡有功效之人,合授勛官者,皆委之覆定,然後奏擬。



 考功郎中一員,從五品上。龍朔二年改為司績大夫,咸亨初乃復。考功員外郎一員,從六品上。龍朔改為司績員外郎,咸亨復。主事三人,從八品上。令史十三人,書令史二十五人,掌固四人。郎中、員外郎之職,掌內外文武官吏之考課。凡應考之官家,具錄當年功過行
 能,本司及本州長官對眾讀,議其優劣,定為九等考第,各於所由司準額校定,然後送省。內外文武官,量遠近以程之有差,附朝集使送簿至省。每年別敕定京官位望高者二人,其一人校京官考,一人校外官考。又定給事中、中書舍人各一人,其一人監京官考,一人監外官考。郎中判京官考,員外判外官考。其檢覆同者,皆以功過上使。京官則集應考之人對讀注定,外官對朝集使注定。凡考課之法,有四善:一曰德義有聞,二曰清慎明著,
 三曰公平可稱,四曰恪勤匪懈。善狀之外,有二十七最:其一曰獻可替否,拾遺補闕,為近侍之最。其二曰銓衡人物,擢盡才良,為選司之最。其三曰揚清激濁,褒貶必當,為考校之最。其四曰禮制儀式,動合經典,為禮官之最。其五曰音律克諧,不失節奏,為樂官之最。其六曰決斷不滯,與奪合理,為判事之最。其七曰都統有方,警守無失,為宿衛之最。其八曰兵士調習,戎裝充備,為督領之最。其九曰推鞫得情,處斷平允,為法官之最。其十曰
 讎校精審,明為刊定,為校正之最。其十一曰承旨敷奏,吐納明敏,為宣納之最。其十二曰訓導有方,生徒充業,為學官之最。其十三曰賞罰嚴明,攻戰必勝,為將帥之最。其十四曰禮義興行,肅清所部,為政教之最。其十五曰詳錄典正,辭理兼舉,為文史之最。其十六曰訪察精審,彈舉必當,為糾正之最。其十七曰明於勘覆,稽失無隱,為勾檢之最。其十八曰職事修理,供承強濟,為監掌之最。其十九曰功課皆充,丁匠無怨,為役使之最。其二
 十曰耕耨以時,收獲成課,為屯官之最。其二十一曰謹於蓋藏,明於出納,為倉庫之最。其二十二曰推步盈虛,究理精密,為歷官之最。其二十三曰占候醫卜,效驗居多,為方術之最。其二十四曰譏察有方,行旅無壅,為關津之最。其二十五曰市廛不擾,奸濫不作,為市司之最。其二十六曰牧養肥碩,蕃息孳多,為牧官之最。其二十七曰邊境肅清,城隍修理,為鎮防之最。一最以上,有四善,為上上。一最以上,有三善,或無最而有四善,為上中。
 一最以上,有二善,或無最而有三善,為上下。一最以上,而有一善,或無最而有二善,為中上。一最以上,或無最而有一善,為中中。職事粗理,善最不聞,為中下。愛憎任情,處斷乖理,為下上。背公向私,職務廢闕,為下中。居官諂詐,貪濁有狀,為下下。若於善最之外,別可加尚,及罪雖成殿,情狀可矜,雖不成殿,而情狀可責者,省校之日,皆聽考官臨時量定。內外官從見任改為別官者,其年考從日申校,百司量其閑劇,諸州據其上下。進考之人,
 皆有定限,茍無其功,不要充數。功過於限,亦聽量進。其流外官,本司量其行能功過,立四等考第而勉進之。凡親勛翊衛,皆有考第。考第之中,略有三等。衛主帥,如三衛之考。其監門、校尉、直長,如主帥之考。凡謚議之法,古之通典,皆審其事,以為旌別。



 戶部尚書一員,正三品。隋為民部尚書,貞觀二十三年改為戶部。明慶元年改為度支,龍朔二年改為司元太常伯,光宅元年改為地官尚書,神龍復為戶部。



 侍郎二員。正四品下。因隋已來改易名位,皆隋尚書也。尚書、侍郎之職,掌天下田戶、均輸、錢穀之
 政令,其屬有四:一曰戶部,二曰度支,三曰金部,四曰倉部。總其職務,而行其制命。凡中外百司之事,由於所屬,皆質正焉。



 郎中二員,從五品上。員外郎二員,從六品上。郎中、員外,自隋已來,隨曹改易。



 主事四人,從九品上。令史十五人,書令史三十四人,亭長六人,掌固十人。郎中、員外郎之職,掌分理戶口、井田之事。凡天下十道,任土所出,為貢賦之差。凡天下之州府,三百一十有五,而羈縻之州,迨八百焉。四萬戶已上為上州,二萬戶以上為中州,不滿為下州。凡三都之
 縣,在內曰京縣,城外曰畿,又望縣有八十五焉。其餘則六千戶以上為上縣,二千戶已上為中縣,一千戶已上為中下縣,不滿一千戶皆為下縣。凡天下之戶,八百一萬八千七百一十,口四千六百二十八萬五千一百六十一。百戶為里,五里為鄉。兩京及州縣之郭內,分為坊,郊外為村。里及坊村皆有正,以司督察。四家為鄰,五鄰為保。保有長,以相禁約。凡男女,始生為黃,四歲為小,十六為中,二十有一為丁,六十為老。每一歲一造計帳。三
 年一造戶籍。縣以籍成於州,州成於省,戶部總而領焉。凡天下之戶,量其資定為九等,每定戶以仲年,造籍以季年。州縣之籍,恆留五比,省籍留九比。凡戶之兩貫者,先從邊州為定,次從關內,次從軍府州。若俱者,各從其先貫焉。樂住之制:居狹鄉者,聽其從寬;居遠者,聽其從近;居輕役之地者,聽其從重。辨天下之四人,使各專其業。凡習學文武者為士,肆力耕桑者為農,巧作器用者為工,屠沽興販者為商。工商之家,不得預於士。食祿之
 人,不得奪下人之利。凡天下之田,五尺為步,步二百有四十為畝,畝百為頃。度其肥瘠寬狹,以居其人。凡給田之制有差,園宅之地亦如之。凡給口分田,皆從便近。居城之人,本縣無田者,則隔縣給授。凡應收授之田,皆起十月,畢十二月。凡授田,先課後不課,先貧後富,先多後少。凡州縣界內所部,受田悉足者,為寬鄉,不足者為狹鄉。凡官人及勛,授永業田。凡天下諸州有公廨田,凡諸州及都護府官人有職分田。凡賦役之制有四:一曰租,
 二曰調,三曰役,四曰雜徭。課戶每丁租粟二石。其調,隨鄉土所產綾絹絁各二丈,布加五分之一。輸綾絹絁者,綿三兩。輸布者,麻三斤。皆書印焉。凡丁,歲役二旬,無事則收其庸,每日三尺。有事而加役者,旬有五日免調,三旬則租調俱免。凡庸調之物,仲秋斂之,季秋發於州。租則準州土收獲早晚,量事而斂之。仲冬起輸,孟春而納畢。本州納者,季冬而畢。凡諸國蕃胡內附者,亦定為九等。凡嶺南諸州稅米,及天下諸州稅錢,各有準常。凡丁戶皆有
 優復蠲免之制。若孝子順孫、義夫節婦志行聞於鄉閭者,州縣申省奏聞,而表其門閭,同籍悉免課役。有精誠致應者,則加優賞焉。凡京司文武職事官,皆有防閣。凡州縣官僚,皆有白直。凡州縣官及在外監官,皆有執衣。凡諸親王府屬,並給士力,具品數如白直。凡有功之臣,賜實封者,皆以課戶充。凡食封,皆傳於子孫。凡庶人年八十及篤疾,給侍丁一人,九十,給二人,百歲,三人。凡天下朝集使,皆以十月二十五日至京師,十一月一日戶
 部引見訖,於尚書省與群官禮見,然後集於考堂,應考績之事。元日,陳其貢篚於殿廷。凡京都諸縣令,每季一朝。



 度支郎中一員,從五品上。龍朔改為司度大夫,咸亨復。員外郎一員,從六品上。



 主事二人,從九品上。令史十六人,書令史三十三人,計史一人,掌固四人。郎中、員外郎之職,掌判天下租賦多少之數,物產豐約之宜,水陸道途之利。每歲計其所出而度其所用,轉運徵斂送納,皆準程而節其遲速。凡和糴和
 市,皆量其貴賤,均天下之貨,以利於人。凡金銀寶貨綾羅之屬,皆折庸調以造。凡天下舟車水陸載運,皆具為腳直,輕重貴賤、平易險澀而為之制。凡天下邊軍,有支度使,以計軍資糧仗之用。每歲所費,皆申度支會計,以長行旨為準。



 金部郎中一員,從五品上。龍朔為司珍大夫,咸亨復。員外郎一員,從六品上。



 主事三人,從九品上。令史八人,書令史二十一人,計史一人,掌固四人。郎中、員外郎之職,掌判天下庫藏錢帛出納
 之事,頒其節制,而司其簿領。凡度,以北方秬黍中者一黍之廣為分,十分為寸,十寸為尺,一尺二寸為大尺,十尺為丈。凡量,以秬黍中者容一千二百為龠,二龠為合,十合為升,十升為斗,三斗為大斗,十斗為斛。凡權衡,以秬黍中者百黍之重為銖,二十四銖為兩,三兩為大兩,十六兩為斤。凡積秬黍為度量權衡,調鐘律,測晷景,合湯藥,及冠冕之制用之。內外官私,悉用大者。凡庫藏出納,皆行文榜,季終會之。若承命出納,則於中書、門下省
 覆而行之。百司應請月俸,符牒到,所由皆遞覆而行之,乃置木契,與應出物之司相合。凡官私互市,物數有制。凡縑帛之類,有長短、廣狹、端疋、屯綟之差。凡賜十段,其率絹三疋,布三端,綿三屯。若雜彩十段,則絲布二疋,紬二疋,綾二疋,縵四疋。若賜蕃客錦彩,率十段則錦一張,綾二疋,縵三疋,綿四屯。凡遣使覆囚,則給時服。若諸使經二年不還,亦如之。凡時服稱一具者,全給之,一副者,減給之。正冬之會,稱束帛有差者,皆賜絹,五品已上五
 疋,六品已下三疋,命婦視其夫、子。



 倉部郎中一員,從五品上。龍朔為司度大夫,咸亨復也。員外郎一員,從六品上。



 主事三人,從九品上。令史九人,書令史二十人,計史一人,掌固四人。郎中、員外郎之職,掌判天下倉儲,受納租稅,出給祿廩之事。凡中外文武官,品秩有差,歲再給之。乃置木契一百枚,以與出給之司合。諸司官人及諸色人應給食者,皆給米。凡致仕之官,五品已上及解官充侍者,各給半祿。即遷官者,通計前祿,以充後數。凡都已東租
 納含嘉倉,自含嘉轉運以實京太倉。自洛至陜為陸運,自陜至京為水運,置使,以監充之。凡王公已下,每歲田苗,皆有簿書。凡義倉所以備歲不足,常平倉所以均貴賤也。



 禮部尚書一員,正三品。隋舊。龍朔改為司禮太常伯,光宅改為春官尚書,神龍復也。



 侍郎一員。正四品下。名因隨曹改易也。尚書、侍郎之職,掌天下禮儀、祭享、貢舉之政令。其屬有四:一曰禮部,二曰祠部,三曰膳部,四曰主客。總其職務,而行其制命。凡中外百司之事,
 由於所屬,皆質正焉。凡舉試之制,每歲仲冬,率與計偕。其科有六:一曰秀才,試方略策五條。此科取人稍峻,貞觀已後遂絕。二曰明經,三曰進士,四曰明法,五曰書,六曰算。凡此六科,求人之本,必取精究理實,而升為第。其有博綜兼學,須加甄獎,不得限以常科。其弘文、崇文館學生,雖同明經、進士,以其資廕全高,試取粗通文義。其郊社齋郎簡試,如太廟齋郎。其國子監大成十二員,取明經及第人聰明灼然者,試日誦千言,並口試,仍策所習業,十條通七,然後補
 充。各授散官,依舊令於學內習業,以通四經為限。



 郎中一員,從五品上。員外郎一員,從六品上。隋曰儀曹郎,武德改禮部郎中員外,龍朔為司禮大夫司禮員外,咸亨復。



 主事二人,從八品上。令史五人,書令史十一人,亭長六人,掌固八人。郎中、員外郎之職,掌貳尚書、侍郎。舉其儀制,而辨其名數。凡五禮之儀,一百五十有二。一曰吉禮,其儀五十有五;二曰賓禮,其儀有六;三曰軍禮,其儀二十有三:四曰嘉禮,其儀五十;五曰兇禮,其儀一十有八。凡元日,大陳設於含元殿,服袞冕臨軒,展宮
 縣之樂,陳歷代寶玉輿輅,備黃麾仗,二王後及百官、朝集使、皇親,並朝服陪位。大會之日,陳設如初。凡冬至,大陳設如元正之儀。其異者,無諸州表奏祥瑞貢獻。凡元正、冬至大會之明日,百官、朝集使等皆詣東宮慶賀。凡千秋節,御樓設九部之樂,百官褲褶陪位。凡京司文武職事,九品已上,每朔、望朝參。五品已上及供奉官、員外郎、監察御史、太常博士,每日參。凡諸蕃國來朝,皆設宮縣之樂及黃麾仗。若蕃國使,則減黃麾之半。凡冊皇后、
 太子、太子妃、諸王、諸王妃、公主,並臨軒冊命,陳設如冬、正之儀。訖,皆拜太廟。凡祥瑞,皆辨其名物。有大瑞、上瑞、中瑞,皆有等差。凡太陽虧,所司預奏,其日置五鼓五兵於太社,而不視事。百官各素服守本司,不聽事。過時乃罷。月蝕,則擊鼓於所司。若五岳、四鎮四瀆崩竭,皆不視事三日。凡二分之月,三公巡行山陵,則太常卿為之副。凡百官拜禮,各有差。致敬之士,若非連屬,應敬之官相見,或自親戚者,各從其私禮。凡樂,有五聲、八音、六律、六
 呂,陳四懸之度,分二舞之節,以和人倫,以調節氣,以享鬼神,以序賓客。凡私家不得設鐘聲。三品已上,得備女樂。五品女樂不得過三人。居大功已上喪,受冊及之官,雖有鼓皆給銅印,有魚符之制。



 並出於門下省。凡服飾尚黃,旗幟尚赤。天子、皇后、太子已下之服,事在《輿服志》也。凡百僚冠笏、傘幌、珂珮,各有差。常服亦如之。凡兇服,不入公門。凡授都督、刺史階未入五品
 者,並聽著緋珮魚,離任則停。凡文武官赴朝詣府,導從各有差。凡職事官薨卒,有賻贈、柳翣、碑碣,各有制度。



 祠部郎中一員,從五品上。龍朔為司禋大夫,咸亨復。員外郎一員,從六品上。



 主事二人,從九品上。令史五人,書令史十一人,亭長六人,掌固八人。郎中、員外郎之職,掌祠祀、享祭、天文、漏刻、國忌、廟諱、卜筮、醫藥、僧尼之事。凡祭祀之名有四:一曰祀天神,二曰祭地祇,三曰享人鬼,四曰釋奠於先聖先師。其差有三:若昊天上帝、皇地祇、州、宗廟為大祀。



 祀天地皆以祖
 宗配享。日月星辰、社稷、先代帝王、岳鎮海瀆、帝社、先蠶、孔宣父、齊太公、諸太子廟為中祀。司中、司命、風師、雨師、眾星、山林、川澤、五龍祠等,及州縣社稷、釋奠為小祀。大祀,皇帝親祭,則太尉為亞獻,光祿卿為終獻。若有司攝事,則太尉為初獻,太常卿為亞獻。凡大祀,散齋四日,致齋二日。小祀,散齋二日,致齋一日。皆祀前習禮、沐浴,並給明衣。凡官爵二品已上,祠四廟。五品
 已上,祠三廟。六品已下達於庶人,祭祖禰而已。凡國有封禪之禮,則依圓丘方澤之神位。凡天下寺有定數,每寺立三綱,以行業高者充。



 諸州寺總五千三百五十八所,三千二百三十五所僧,二千一百二十二所尼。每寺上座一人,寺主一人,都維那一人。



 凡僧簿籍,三年一造。凡別敕設齋,應行道並官給料。凡國忌日,兩京大寺各二,以散齋僧尼。文武五品已上,清官七品已上皆集,行香而退。天下州府亦然。凡遠忌日,雖不廢務,然非軍務急切,亦不舉事。餘如常式。



 膳部郎中一員,從五品上。龍朔為司膳大夫,咸亨復也。員外郎一員,從六品上。主事二人,從九品上。令史四人,書令史九人,掌固四人。郎中、員外郎之職,掌邦之祭器、牲豆、酒膳,辨其品數,及藏冰食料之事。



 主客郎中一員,從五品上。隋曰司蕃郎,武德改主客郎中,龍朔為司蕃大夫,咸亨復。員外郎一員,從六品上。主事二人,從九品上。



 令史四人,書令史九人,掌固四人。郎中、員外郎之職,掌二王後及諸蕃朝聘之事。二王之後,酅公、介公。凡四蕃之國。經朝貢之後,自相
 誅絕,及有罪滅者,蓋三百餘國。今所存者,七十餘蕃。其朝貢之儀,享宴之數,高下之等,往來之命,皆載於鴻臚之職焉。



 兵部尚書一員,正三品。南朝謂之五兵尚書,隋曰兵部尚書。龍朔改為司戎太常伯,咸亨復。



 侍郎二員。正四品下。龍朔為司戎少常伯,咸亨復。尚書、侍郎之職,掌天下武官選授及地圖與甲仗之政令。其屬有四:一曰兵部,二曰職方,三曰駕部,四曰庫部。總其職務,而行其制命。凡中外百官之事,由於所屬,咸質正焉。凡選授之制,
 每歲集於孟冬。去王城五百里以上旬,千里之內以中旬,千里之外以下旬。尚書、侍郎分為三銓。



 尚書為中銓,侍郎分東西。凡試能有五,五謂長垛、馬步射、馬槍、步射、應對。互有優長,即可取之。



 較異有三。三謂驍勇、材藝及可為統領之用也。審其功能,而定其留放,所以錄才藝、備軍國、辨虛冒、敘勛勞也。然後據其資勞,量為注擬。



 五品已上送中書門下,六品已下量資注定。其在軍鎮要籍,不得赴選,委節度使銓試其等第申省。



 凡官階注擬團甲進甲,皆如吏部之制。凡大選,終於季春之月,所以約資敘之淺深,審才略之優劣,軍國之用在焉。



 郎
 中二員,從五品上。龍朔為司戎大夫,咸亨復也。員外郎二人,從六品上。主事四人,從八品下。令史三十人,書令史六十人,亭長八人,掌固十二人。郎中一員掌判帳及天下武官之階品,衛府之名數。凡敘階有二十九。將軍之階。



 具於敘目。凡敘階之法,一如文散官之制。凡天下之府,五百九十有四,有上中下,並載於諸衛之職。凡應宿衛官,各從番第。凡千牛備身左右及太子千牛備身,皆取三品已上職事官子孫,四品清官子,儀容端正,武藝可稱者充。五考,本司隨文武
 簡試聽選。



 四品,謂諸司侍郎、左右庶子也。凡殿中省進馬,取左右衛三衛及高廕,簡儀容可觀者補充,簡試同千牛例。僕寺進馬,亦如之。五品已下、七品已上,五年,多至八年,年滿簡送吏部。不第者,如初。無文,聽以武選。凡左右衛親衛、勛衛、翊衛,及左右率府親勛翊衛,及諸衛之翊衛,通謂之三衛。擇其資廕高者,為親衛,其次者,為勛衛及率府之親衛,又次者,為翊衛及率府之勛衛,又次者,為諸衛及率府之翊衛,又次者,為親王府之執仗乘。量遠邇以
 定其番第。應補之人,周親已上有犯刑戮者,配令兵部上下。凡諸衛及率府三衛,貫京兆、河南、蒲、同、華、岐、陜、懷、汝、鄭等州,皆令番上。餘州皆納資。凡左右衛之三衛,分為五仗。凡王公已下,皆有親事帳內,限年十八已下,舉諸州率萬人以充之。皆限十周年,則聽其簡試。文理高者送吏部,其餘留本司,全下者退還本色。凡兵士隸衛,各有其名。左、右衛曰驍騎,左、右驍衛曰豹騎,左、右武衛曰熊渠,左、右威衛曰羽林,左、右領軍衛曰射聲,左、右金
 吾衛曰佽飛。東宮左、右衛率府曰超乘,左、右司禦率府曰旅賁,左、右清道率府曰直蕩。總名曰衛士。皆取六品已下子孫,及白丁無職役者點充。凡三年一簡點,成丁而入,六十而免。量其遠邇,以定番第。凡衛士,各立名簿。其三年已來征防差遣,仍定優劣為三第。每年正月十日送本府印記,仍錄一道送本衛府。若有差行上番,折沖府據簿而發之。凡差衛士征戍鎮防,亦有團伍。其善弓馬者,為越騎團,餘為步兵團,主帥已下統領之。火十
 人,有六馱馬。若父兄子弟,不並遣之。若祖父母老疾,家無兼丁,免征行及番上。其居常則皆習射,唱大角歌。番集之日,府官率而課試。凡左、右金吾衛,有角手,諸衛有駑手,左、右羽林軍有飛騎及左右萬騎、彍騎。天下諸軍,有健兒。皆定其名籍,每季上中書、門下。凡關內,有團結兵,秦、成、岷、渭、河、蘭六州,有高麗羌兵。黎、雅、邛、翼、茂五州,有鎮防團結兵。天下諸州差失,募取戶殷丁多,人材驍勇,選前資官勛官部分強明堪統攝者,節級擢補主帥
 以領之。其義征者,別為行伍,不入募人之營。凡軍行器物,皆於當州分給之。如不足,以自備,貧富必以均焉。凡諸州軍府應行兵之名簿,器物之多少,皆申兵部。軍散之日,亦錄其存亡多少,以申而勘會之。凡諸道回兵糧Я之物,衣資之費,皆令所在州縣分而給之。郎中一人掌判簿,以總軍戎差遣之名數。凡天下節度使有八,若諸州在節度內者,皆受節度焉。其福州經略使,登平州海軍,則不在節度之內。



 節度名與所管軍鎮名,並見《地理志》也。凡親王
 總戎,曰元帥,文武官總統者,則曰總管。以奉使言之,則曰節度使,有大使、副使、判官。若大使加旌節以統軍,置木契以行。凡將帥出行,兵滿一萬人已上,置長史、司馬、倉曹兵曹胄曹等參軍各一人。五千人已上,減司馬。諸軍各置使一人,五千人已上置副使一人,一萬人已上置營田副使一人。每軍各有倉、兵、胄三參軍。其橫海、高陽、唐興、恆陽、北平等五軍,皆本州刺史為使。凡鎮,皆有使一人,副使一人。萬人已上,置司馬、倉兵二曹參軍。五
 千人已下,減司馬。凡諸軍鎮,每五百人置押官一人,千人置子總官一人,五千人置總管一人。凡諸軍鎮使、副使已上,皆四年一替;總管已下,二年一替;押官隨兵交替。凡諸軍鎮大使、副使已下,皆有傔人,別奏以從之。凡幸三京,即東都南、北衛,皆置左、右屯營,別立使以統之。若在都,則京城亦如之。凡大將出征,皆告廟授鉞,辭齊太公廟訖,不宿於家。臨軍對寇,士卒不用命,並得專行其罰。既捷,及軍未散,皆會眾而書勞與其費用,乃告太
 廟。元帥凱旋之日,皆使郊勞。有司先獻捷於太廟,又告齊太公廟。員外郎一人掌貢舉及雜請之事。凡貢舉,每歲孟春,亦與計偕。有二科:一曰平射,二曰武舉。凡科之優劣,勛獲之等級,皆審其實而受敘焉。員外郎一人掌判南曹。每歲選人,有解狀、簿書、資歷、考課。必由之以核其實,乃上三銓。進甲則署焉。



 職方郎中一員,從五品上。龍朔為司域大夫也。員外郎一員,正六品上。



 主事二人,從九品上。令史四人,書令史九人,掌固四人。郎中、員
 外郎之職,掌天下地圖及城隍、鎮戍、烽堠之數,辨其邦國都鄙之遠近,及四夷之歸化。凡五方之區域,都邑之廢置,疆埸之爭訟者,舉而正之。凡天下上鎮二十,中鎮九十,下鎮一百三十五。上戍十有一,中戍八十六,下戍二百四十五。凡烽堠所置,大率相去三十里。其逼邊境者,築城置之。每烽置帥一人,副一人。凡州縣城門及倉庫門,須有備守。



 駕部郎中一員,從五品上。龍朔為司輿大夫也。員外郎一人,從六品上。



 主事
 三人,從九品上。令史十人,書令史二十人,掌固四人。郎中、員外郎之職,掌邦國輿輦、車乘、傳驛、廄牧、官私馬牛雜畜簿籍,辨其出入,司其名數。凡三十里一驛,天下驛凡一千六百三十九,而監牧六十有五,皆分使統之。若畜養之宜,孳生之數,皆載於太僕之職。凡諸衛有承直之馬,凡諸司有備運之牛,皆審其制,以定數焉。



 庫部郎中一員,從五品上。龍朔為司庫大夫也。員外郎一員,從六品上。



 主事二人,從九品上。令史七人,書令史十五人,掌固四人。郎中、
 員外郎之職,掌邦國軍州戎器、儀仗。凡元正、冬至陳設,並祠祭喪葬所貢之物,皆辨其出入之數,量其繕造之功,以分給焉。



 刑部尚書一員,正三品。隋初改都官尚書,又改為刑部。龍朔改為司刑太常伯,光宅改為秋官尚書,神龍復也。侍郎一員。正四品下。龍朔為司刑少常伯。



 尚書、侍郎之職,掌天下刑法及徒隸、勾覆、關禁之政令。其屬有四:一曰刑部,二曰都官,三曰比部,四曰司門。總其職務,而行其制命。凡中外百司之事,由於所屬,咸質正焉。



 郎中二員,
 從五品上。隋曰憲部郎,武德為刑部郎中,龍朔改為司刑大夫。員外郎二員,從六品上。主事四人,從九品上。



 令史十九人,書令史三十八人,亭長六人,掌固十人。郎中、員外郎之職,掌貳尚書、侍郎,舉其典憲,而辨其輕重。凡文法之名有四:一曰律,二曰令,三曰格,四曰式。凡律,十有二章:一名例,二禁衛,三職制,四戶婚,五廄庫,六擅興,七賊盜,八斗訟,九詐偽,十雜律,十一捕亡,十二斷獄,而大凡五百條。令,二十有七篇,分為三十卷。第一至第七曰官品職員,八祠,九戶,十選舉,十一考
 課,十二宮衛,十三軍防,十四衣服,十五儀制,十六鹵簿,十七公式,十八田,十九賦役,二十倉庫,二十一廄牧,二十二關市,二十三醫疾,二十四獄官,二十五營繕,二十六喪葬,二十七雜令,而大凡一千五百四十六條。凡格,二十四篇。式,三十三篇。以尚書、御史臺、九寺、三監、諸軍為目。凡律,以正刑定罪。令,以設範立制。格,以禁違正邪。式,以軌物程事。乃立刑名之制五焉:一笞,二杖,三徒,四流,五死。笞刑五,杖刑五,徒刑五,流刑三,死刑二。而斷獄
 之大典,有十惡、八議、五聽、六贓。贖配之典,具在《刑法志》。凡決死刑,皆於中書門下詳覆。凡死罪,枷而杻。婦人及流徒,枷而不杻。官品及勛散之階第七已上,鎖而不枷。在京諸司,則徒已上送大理,杖已下當司斷之。若金吾糾獲,亦送大理。凡決大闢罪,在京者,行決之司,皆五覆奏;在外者,刑部三覆奏。若犯惡逆已上,及部曲奴婢殺主者,一覆奏。凡京城決囚之日,減膳徹樂。每歲立春後至秋分,不得決死刑。大祭祀及致齋、朔望、上下弦、二十
 四氣、雨未晴、夜未明、斷屠月日及休假,亦如之。凡犯流罪已下,應除免官。當未奏,身死者,免其追奪。流移之人,皆不得棄放妻妾,及私遁還鄉。至六載,然後聽仕。即本犯不應流而特配流者,三載已後聽仕。其應徒則皆配居作。凡禁囚,五日一慮。凡鞫獄官與被鞫人有親屬仇嫌者,皆聽更之。凡在京諸司見禁囚,每月二十五已前,本司錄其所犯及禁時月日,以報刑部,凡國有赦宥之事,先集囚徒於闕下,命衛尉樹金雞,待宣制訖,乃釋之。



 都官郎中一員,從五品上。龍朔改司僕大夫,咸亨復。員外郎一員,從六品上。



 主事二人,從九品上。令史發人,書令史十二人,掌固四人。郎中、員外郎之職,掌配役隸,簿隸俘囚以給衣糧藥療,以理訴競雪冤。凡公私良賤,必周知之。凡反逆相坐,沒其家為官奴婢。一免為蕃戶,再免為雜戶,三免為良民,皆因赦宥所及則免之。年六十及廢疾,雖赦令不該,亦並免為蕃戶,七十則免為良人,任所樂處而編附之。凡初被沒有伎藝者,各從其能,而配諸司。婦人工巧者,入於
 掖庭。其餘無能,咸隸司農。



 比部郎中一員,從五品上。龍朔為司計大夫。員外郎一員,從六品上。



 主事二人,從九品上。令史十四人。書令史二十七人,計史一人,掌固四人。郎中、員外郎之職,掌勾諸司百僚俸料、公廨、贓贖、調斂、徒役、課程、逋懸數物,周知內外之經費,而總勾之。凡內外官料俸,以品第高下為差。外官以州縣府之上中下為差。凡稅天下戶錢,以充州縣官月料,皆分公廨本錢之利。羈縻州所補漢官,給以當土之物。關監
 之官,以品第為差。其給以年支輕貨。鎮軍司馬,判官俸祿,同京官。鎮戍之官,以鎮戍上中下為差。凡京師有別借食本,每季一申省,諸州歲終而申省,比部總勾覆之。凡倉庫、出內、營造、傭市、丁匠、功程、贓贖、賦斂、勛賞、賜與、軍資、器仗、和糴、屯牧,亦勾覆之。



 司門郎中一員,從五品上。龍朔曰司門大夫。員外郎一員,從六品上。主事二人,從九品上。令史六人,書令史十三人,掌固四人。郎中、員外郎之職,掌天下諸門及關出入往來之籍賦,而審
 其政。凡關二十有六,為上中下之差。京城四面關有驛道者,為上關。餘關有驛道及四面無驛道者,為中關。他皆為下關。關所以限中外,隔華夷,設險作固,閑邪正禁者也。凡關呵而不征,司貨賄之出入,其犯禁者,舉其貨,罰其人。凡度關者,先經本部本司請過所,在京則省給之,在外則州給之。而雖非所部,有來文者,所在亦給。



 工部尚書一員,正三品。南朝謂之起部。有所營造,則置起部尚書,畢則省之。隋初改置工部尚書。龍朔為司平太常伯,光宅改為冬官尚書,神龍復舊也。



 侍郎一員。正四品下。龍朔為司平少常伯。
 尚書、侍郎之職,掌天下百工、屯田、山澤之政令。其屬有四:一曰工部,二曰屯田,三曰虞部,四曰水部。總其職務,而行其制命。凡中外百司之事,由於所屬,咸質正焉。



 郎中一員,從五品上。龍朔為司平大夫也。員外郎一員,從六品上。



 主事二人,從九品上。令史十二人,書令史二十一人,亭長六人,掌固八人。郎中、員外郎之職,掌經營興造之眾務。凡城池之修浚,土木之繕葺,工匠之程式,咸經度之。凡京師、東都有營繕,則下少府、將作,以供其事。



 屯田郎中一員,從五品上。龍朔為司田大夫也。員外郎一員,從六品上。



 主事二人,從九品上。令史七人,書令史十二人,計史一人,掌固四人。郎中、員外郎之職,掌天下屯田之政令。凡邊防鎮守,轉運不給,則設屯田,以益軍儲。其水陸腴瘠,播種地宜,功庸煩省,收率等級,咸取決焉。諸屯田役力,各有程數。凡天下諸軍州管屯,總九百九十有二。大者五十頃,小者二十頃。凡當屯之中,地有良薄,歲有豐儉,各定為三等。凡屯皆有屯官、屯副。凡京文武職事官,有職分田。
 京兆,河南府及京縣官,亦準此。凡在京諸司,有公廨田,皆視其品命而審其分給。



 虞部郎中一員,從五品上。龍朔為司虞大夫。員外郎一員,從六品上。主事二人,從九品上。令史四人,書令史九人,掌固四人。郎中、員外郎之職,掌京城街巷種植,山澤苑囿,草木薪炭,供頓田獵之事。凡採捕漁獵,必以其時。凡京兆、河南二都,其近為四郊,三百里皆不得弋獵採捕。殿中、太僕所管閑廄馬,兩都皆五百里內供其芻槁。其關內、隴右、西使、南
 使諸牧監馬牛駝羊,皆貯槁及茭草。其柴炭木橦進內及供百官蕃客,並於農隙納之。



 水部郎中一員,從五品上。龍朔為司川大夫。員外郎一員,從六品上。



 主事二人,從九品上。令史四人,書令史九人,掌固四人。郎中、員外郎之職,掌天下川瀆陂池之政令,以導達溝洫,堰決河渠。凡舟楫溉灌之利,咸總而舉之。凡天下水泉,三億二萬三千五百五十九。其在遐荒絕域,迨不可得而知矣。其江、河,自西極達於東溟,中國之大川者也。其餘百
 三十五水,是為中川。其又千二百五十二水,斯為小川也。若渭、洛、汾、濟、漳、淇、淮、漢,皆互達方域,通濟舳艫,從有之無,利於生人者也。凡天下造舟之梁四,河則蒲津、大陽、河陽,洛則孝義也。石柱之梁四,洛則天津、永濟、中橋,灞則灞橋。



 木柱之梁三,皆渭川,便橋、中渭橋、東渭橋也。巨梁十有一,皆國工修之。其餘皆所管州縣隨時營葺。其大津無梁,皆給船人,量其大小難易,以定其差。



 門下省



 秦、漢初,置侍中,曾無臺省之名。自晉始置門下省,南、北朝皆因之。龍朔改為東臺,光宅改為鸞
 臺,神龍復。



 侍中二員。隋曰納言,又名侍內。武德為納言,又改為侍中。龍朔改東臺左相,光宅元年改為納言,神龍復為侍中。開元元年改為黃門監,五年復為侍中。天寶二年改為左相。至德二年復改為侍中。武德定令,侍中正三品,大歷二年十一月九日,升為正二品。舊制,宰相常於門下省議事,謂之政事堂。永淳二年七月,中書令裴炎以中書執政事筆,遂移政事堂於中書省。開元十一年,中書令張說改政事堂為中書門下,其政事印,改為中書門下之印也。



 侍中之職,掌出納帝命,緝熙皇極,總典吏職,贊相禮儀,以和萬邦,以弼庶務,所謂佐天子而統大政者也。凡軍國之務,與中書令參而總焉,坐而論之,
 舉而行之,此其大較也。凡下之通上,其制有六:一曰奏抄,二曰奏彈,三曰露布,四曰議,五曰表,六曰狀;皆審署申覆而施行焉。凡法駕行幸,則負寶而從。大朝會、大祭祀,則板奏中嚴外辦,以為出入之節。輿駕還宮,則請解嚴,所以告禮成也。凡大祭祀,皇帝致齋,既朝,則請就齋室。將奠,則奉玉及幣以進。盥手,則取匜以沃。洗爵,則酌罍水以奉。及贊酌泛齊,進福酒以成其禮焉。若享宗廟,則進瓚而贊酌鬱酒以稞。既稞,則贊酌醴齊。其餘如饗
 神祇之禮。藉田,則奉耒以贊事。凡諸侯王及四夷之君長朝見,則承詔而勞問之。臨軒命使,冊后及太子,則承詔以命之。凡制敕慰問外方之臣及徵召者,則監其封題。若發驛遣使,則給其傳符,以通天下之信。凡官爵廢置,刑政損益,皆授之於記事之官。既書於策,則監其記注焉。凡文武職事六品已下,所司進擬,則量其階資,校其才用,以審定之。若擬職不當,隨其優屈,退而量焉。



 門下侍郎二員。隋曰黃門侍郎。龍朔為東臺侍郎,咸亨改為黃門侍郎,垂拱改為鸞臺侍郎,天
 寶二年改為門下侍郎,乾元元年改為黃門侍郎,大歷二年四月復為門下侍郎。武德定令,中書門下侍郎,同尚書侍郎,正四品上。大歷二年九月敕升為正三品也。



 門下侍郎掌貳侍中之職。凡政之弛張,事之與奪,皆參議焉。若大祭祀,則從升壇以陪禮。皇帝盥手,則奉巾以進。既帨,則奠巾於篚,奉瓠爵以贊獻。凡元正、冬至天子視朝,則以天下祥瑞奏聞。



 給事中四員。正五品上。隋曰給事郎,置四員,位次門下侍郎。武德定令,曰給事中。龍朔改為東臺舍人,咸亨復。



 給事中掌陪侍左右,分判省事。凡百司奏抄,
 侍中審定,則先讀而署之,以駁正違失。凡制敕宣行,大事則稱揚德澤,褒美功業,覆奏而請施行;小事則署而頒之。凡國之大獄,三司詳決,若刑名不當,輕重或失,則援法例退而裁之。凡發驛遣使,則審其事宜,與黃門侍郎給之;其緩者給傳,即不應給,罷之。凡文武六品已下授職官,所司奏擬,則校其仕歷淺深,功狀殿最,訪其德行,量其才藝;若官非其人,理失其事,則白侍中而退量焉。若弘文館圖書之繕寫、讎校,亦課而察之。凡天下冤
 滯未申及官吏刻害者,必聽其訟,與御史、中書舍人同計其事宜,而申理之。



 錄事四人,從七品上。主事四人,從八品下。



 令史十一人,書令史二十二人,甲庫令史七人,傳制八人,亭長六人,掌固十人,修補制敕匠五人。



 左散騎常侍二人。從三品。魏、晉置散騎常侍、侍郎,與侍中、黃門侍郎共平尚書奏事。其後用人或雜,江左不重此官,或省或置。隋初省散騎侍郎,置常侍四人,從三品,掌陪從朝直。煬帝又省之。武德初,以為加官。貞觀初,置常侍二人,隸門下省。明慶二年,又置二員,隸中書省,始有左右之號,並金蟬珥貂。左常侍與侍中左貂,右常侍與中書令右貂,謂之八貂。龍朔為左侍極,咸亨復。廣德二年五月,升為正三品,加置四員。興
 元元年正月,左右各加一員。貞元四年正月敕,依舊四員也。



 常侍掌侍奉規諷,備顧問應對。寶應二年敕,左右散騎常侍各置參官兩人,令自揀擇聞奏,參典亦置兩人,後省。



 諫議大夫四員。秦、漢曰諫大夫,光武加議字。隋於門下省置諫議大夫七員,從四品下。武德四年敕置四員,正五品上。龍朔改為正諫大夫,神龍復。大歷四年敕只四員,正五品上。龍朔七年三月敕,其諫議四員,內供奉不得為正員。至貞元四年五月十五日敕,諫議分為左右,加置八員,四員隸門下為左,會昌二年十一月中書奏:隋於門下省置諫議大夫七員,從四品下。今正五品上。自大歷二年門下中書侍郎升為正三品,兩省遂闕四品官。其諫議大夫望升為正四品下,分為左右,以備兩省四品之闕。向後與丞郎出入迭用,以重其選。敕可之。



 諫議大夫掌侍從贊相,規諫諷諭。凡諫有五:
 一曰諷諫,二曰順諫,三曰規諫,四曰致諫,五曰直諫。



 起居郎二員,從六品上。古無其名,隋始置起居舍人二員。貞觀二年省起居舍人,移其職於門下,置起居郎二員。明慶中又置起居舍人,始與起居郎分在左右。龍朔二年改為左史,咸亨復。天授元年又改為左史,神龍復也。



 楷書手三人。起居郎掌起居注,錄天子之言動法度,以修記事之史。凡記事之制,以事系日,以日系月,以月系時,以時系年。必書其朔日甲乙,以紀歷數,典禮文物,以考制度,遷拜旌賞以勸善,誅伐黜免以懲惡。季終則授之國史焉。



 自漢獻帝後,歷代帝王有起居注,著作編之,每季為卷,送史館也。



 左補闕二員,從七品上。左拾遣二員。從八品上。古無此官名。天後垂拱元年二月二十九日敕:「記言書事,每切於旁求;補闕拾遣,未弘於注選。瞻言共理,必藉眾才,寄以登賢,期之進善。宜置左右補闕各二員,從七品上,左右拾遣各二員,從八品上,掌供奉諷諫,行立次左右史之下。仍附於令。」天授二年二月,加置三員,通前五員。大歷四年,補闕、拾遣,各置內供奉兩員。七年五月十一日敕,補闕、拾遣,宜各置兩員也。



 補闕、拾遣之職,掌供奉訥諫,扈從乘輿。凡發令舉事,有不便於時,不合於道,大則廷議,小則上封。若賢良之遺滯於下,忠孝之不聞於上,則條其事狀而薦言之。



 典儀二員。從九品。南齊有典儀錄事一員,梁有典儀之官,後省。皇朝又置典儀二人,隸門下省。初用
 人皆輕,貞觀末,李義府為之,自是用士人為之。



 贊者十二人。隋太常、鴻臚二寺,皆有贊者,皇朝因置之,隸門下省,掌贊唱,為行事之節。分番上下,謂之番官。



 典儀掌殿上贊唱之節,及殿廷版位之次。凡國有大禮,侍中行事,及進中嚴外辦之版,皆贊相焉。



 城門郎四員。從六品上。漢有城門校尉,掌京城諸門啟閉之節。隋改校尉為城門郎,置四員,從六品,皇朝因之也。



 令史一人,書令史二人,門僕八百人。門僕,晉代有之。皇朝隸城門局,分番上下,掌送管鑰。



 城門郎掌京城皇城宮殿諸門啟閉之節,奉出納管鑰。開則先外而後內,合則先內而後外,
 所以重中禁,尊皇居也。候其晨昏擊鼓之節而啟閉之。凡皇城宮城合門之鑰,先酉而出,後戌而入;開門之鑰,後醜而出,夜盡而入。京城合門之鑰,後申而出,先子而入;開門之鑰,後子而出,先卯而入。若非其時而有命啟閉,則詣閣覆奏。



 符寶郎四員。從六品上。周有典瑞之職,秦有符璽令,漢曰符璽郎。兩漢得秦六璽及傳國璽,後代傳之。隋置符璽郎二員,從六品。天后惡璽字,改為寶。其受命傳國等八璽文。並改雕寶字。神龍初,復為符璽郎。開元初,又改為符寶,從璽文也。



 令史二人,書令史三人,主寶六人,主符
 三十人,主節十八人。符寶郎掌天子八寶及國之符節,辨其所用。有事則請於內,既事則奉而藏之。八寶:一曰神寶,所以承百王,鎮萬國。二曰受命寶,所以修封禪,禮神祇。三曰皇帝行寶,答疏於王公則用之;四曰皇帝之寶,勞來勛賢則用之。五曰皇帝信寶,徵召臣下則用之。六曰天子行寶,答四夷書則用之。七曰天子之寶,慰撫蠻夷則用之;八曰天子信寶,發番國兵則用之。凡大朝會,則捧寶以進於御座。車駕行幸,則奉寶以從於黃
 鉞之內。凡國有大事,則出納符節,辨其左右之異,藏其左而班其右,以合中外之契焉。一曰銅魚符,所以起軍旅,易守長。二曰傳符,所以給郵驛,通制命。三曰隨身魚符,所以明貴賤,應徵召。四曰木契,所以重鎮守,慎出納。五曰旌節,所以委良能,假賞罰。魚符之制,王畿之內,左三右一;王畿之外,左五右一。



 左者在內,右者在外。行用之日,從第一為首,後事須用,以次發之,周而復始。大事兼敕書,小事但降符,函封遣使,合而行之。傳符之制,太子監國曰雙龍之符,左右各十。京都
 留守曰麟符,左二十,其右一十有九。東方曰青龍之符,西方曰騶虞之符,南方曰硃雀之符,北方曰玄武之符,左四右三。



 左者進內,右者付外。隨身魚符之制,左二右一,太子以玉,親王以金,庶官以銅,佩以為飾。刻姓名者,去官而納焉;不刻者,傳而佩之。木契之制,太子監國,則王畿之內,左右各三;王畿之外,左右各五;庶官鎮守,則左右各十。旌節之制,命大將帥及遣使於四方,則請而佩之。旌以專賞,節以專殺。



 《周禮》之制,山國用虎節,土國用人節,澤國用龍節,皆金也。又云,道路用旌節,即漢使
 所持者是也。



 弘文館:後漢有東觀,魏有崇文館,宋有玄、史二館,南齊有總明館,梁有士林館,北齊有文林館,後周有崇文館,皆著撰文史,鳩聚學徒之所也。武德初置修文館,後改為弘文館。後避太子諱,改曰昭文館。開元七年,復為弘文館,隸門下省。



 學士。學士無員數,自武德已來,皆妙簡賢良為學士。故事,五品已上稱學士,六品已下為直學士,又有文學直館學士,不定員數。館中有四部書及圖籍,自垂拱已後,皆宰相兼領,號為館主,常令給事中一人判館事。學生三十人,校書郎二人,從九品上。令史二人,楷書手三十人,典書二人,拓書手三人,筆匠三人,熟紙裝潢匠九人,
 亭長二人,掌固四人。弘文館學士掌詳正圖籍,教授生徒。凡朝廷有制度沿革,禮儀輕重,得參議焉。校書郎掌校理典籍,刊正錯謬。其學生教授考試,如國子學之制焉。



 中書省秦始置中書謁者,漢元帝去「謁者」二字。歷代但雲中書。後周謂之內史省,隋因為內史省,置內史監、令各一員。煬帝改為內書省。武德復為內史省,三年改為中書省。龍朔改為西臺,光宅改為鳳閣,神龍復為中書省。開元元年改為紫微省,五年復舊。



 中書令二員。漢、魏品卑而付重。魏置監、令各一員,歷南朝不改。隋省監,置令二人,正三品。隋文帝
 廢三公府僚,令中書令與侍中知政事,遂為宰相之職。隋曰內書令。武德日內史令,尋改為中書令。龍朔為西臺右相,咸亨復為中書令。光宅為鳳閣令。開元元年改為紫微令,五年復為中書令。天寶改為右相,至德二年復為中書令。本正三品,大歷二年十一月九日,與侍中同升正二品,自後不改也。



 中書令之職,掌軍國之政令,緝熙帝載,統和天人。入則告之,出則奉之,以厘萬邦,以度百揆,蓋佐天子而執大政也。凡王言之制有七:一曰冊書,二曰制書,三曰慰勞制書,四曰發敕,五曰敕旨,六曰論事敕書,七曰敕牒,皆宣署申覆而施行之。凡大祭祀群神,則從升壇以相禮。享宗廟,則
 從升阼階。親征篡嚴,戒敕百僚,冊命親賢,臨軒則使讀冊。若命之於朝,則宣而授之。凡冊太子,則授璽。凡制詔宣傳,文章獻納,皆授之於記事之官。



 武德、貞觀故事,以尚書省左右僕射各一人及侍中、中書令各二人,為知政事官。其時以他官預議國政者,雲與宰相參議朝政,或云平章國計,或云專典機密,或參議政事。貞觀十七年,李勣為太子詹事,特詔同知政事,始謂同中書門下三品。自是,僕射常帶此稱。自餘非兩省長官預知政事者,亦皆以此為名。永淳中,始詔郭正一、郭待舉、魏玄同等,與中書門下同承受進旨平章事。自天後已後,兩省長官及同中書門下三品並平章事,為宰相。其僕射不帶同中書門下三品者,但厘尚書省而已。總章二年,東臺侍郎張文瓘,西臺侍郎戴至德等,始以同中書門下三品著之入銜。自
 是相承至今。永淳二年,黃門侍郎劉齊賢知政事,稱同中書門下平章事,自後兩省長官,及他官執政未至侍中書令者,皆稱同中書門下平章事也。



 中書侍郎二員。漢置中書,掌密詔,有令、僕、丞、郎四官。魏曰中書郎,晉加「侍」字。隋置內書省,改為內書侍郎,正四品。武德初為內史侍郎,三年改為中書侍郎。龍朔、光宅、開元,隨曹易號。至德復為中書侍郎。武德定令,與尚書侍郎俱第四品。大歷二年九月,與門下侍郎共升為正三品也。



 中書侍郎掌貳令之職。凡邦國之庶務,朝廷之大政,皆參議焉。凡臨軒冊命大臣,令為之使,則持冊書以授之。凡四夷來朝,監軒則受其表疏,升於西階而奏。若獻贄幣,則受之以
 授於所司。



 中書舍人六員。正五品上。曹魏於中書置通事一人,掌呈奏按章。高貴鄉公於通事下加「舍人」二字。晉於中書置舍人、通事各一人。自魏、晉、齊、梁,詔誥皆出於中書令、中書侍郎,中書通事舍人但掌呈奏而已。或通事有文字者,別敕知詔誥。至梁武,制誥專令舍人掌之,兼去「通事」二字,但雲中書舍人。隋曰內史舍人,置八員,掌制誥,品第六。尋升五品上。煬帝改內書舍人,置四員。武德初為內史舍人,三年,改為中書舍人。龍朔、光宅、開元,隨曹改易。



 舍人掌侍奉進奏,參議表章。凡詔旨敕制,及璽書冊命,皆按典故起草進畫;既下,則署而行之。其禁有四:一曰漏洩,二曰稽緩,三曰違失,四曰忘誤;所以
 重王命也。制敕既行,有誤則奏而正之。凡大朝會,諸方起居,則受其表狀而奏之。國有大事,若大克捷及大祥瑞,百僚表賀,亦如之。凡冊命大臣於朝,則使持節讀冊命之。凡將帥有功及有大賓客,皆使勞問之。凡察天下冤滯,與給事中及御史三司鞫其事。凡百司奏議,文武考課,皆預裁焉。



 主書四人,從七品上。主事四人,從八品下。



 令史二十五人,書令史五十人,傳制十人,亭長十八人,修補敕匠五十人。



 右散騎常侍二員,從三品。右補闕二員,從七品上。



 右拾遣二員,從八品上。起居舍人二員。從六品上。



 右常侍、補闕、拾遣。掌事同左省。起居舍人,掌修記言之史,錄天子之制誥德音,如記事之制,以記時政損益。季終,則授之於國史。



 通事舍人十六人。從六品上。通事舍人,奏謁者之官也。掌賓贊、贊受事,隸光祿勛。晉置舍人、通事各一人,隸中書。東晉曰通事舍人。隋因晉制,置十六人,從六品上,又為通事謁者。武德初,廢謁者臺,改通事謁者為通事舍人,隸四方館,屬中書省也。



 通事舍人掌朝見引納及辭謝者,於殿廷通奏。凡近臣入侍,文武就列,引以進退,而
 告其拜起出入之節。凡四方通表,華夷納貢,皆受而進之。凡軍旅之出,則命受慰勞而遣之。既行,則每月存問將士之家,以視其疾苦。凱旋,則郊迓之,皆復命。凡致仕之臣,與邦之耋老,時巡問亦如之。



 令史十人,亭長十八人,掌固二十四人。



 集賢殿書院:開元十三年置。漢、魏已來,職在秘書。梁於文德殿內藏聚群書。北齊有文林館學士,後周有麟趾殿學士,皆掌著述。隋平陳之後,寫群書正副二本,藏於宮中,其餘以實秘書外閣。煬帝於東都觀文殿東西廂貯書。自漢延熹至隋,皆秘書掌國籍,而禁中之書,時或有焉。及太宗在籓府時,有秦府學士十八
 人。其後弘文、崇文二館皆有。玄宗即位,大校群書。開元五年,於乾元殿東廊下寫四部書,以充內庫,置校定官四人。七年,駕在東都,於麗正殿置修書使。十二年,駕在東都,十三年與學士張說等宴於集仙殿,因改名集賢,改修書使為集賢書院學士。其大明宮所置書院,本命婦院,屋宇宏敞。永泰元年三月,詔僕射裴冕等十三人,每日於集賢書院待詔。



 集賢學士。初定制以五品已上官為學士,六品已下為直學士。每宰相為學士者,為知院事。常侍一人,為副知院事。



 學士知院事一人,開元初,以褚無量、馬懷素、元行沖相次知乾元殿寫書,及在麗正,乃有使名。張說代元行沖,改院為集賢,以說為大學士,知院事,說懇讓大字,詔許之。自是,每以宰相一人知院事。



 副知院事一人,初,宰相張說知院事,以左常侍徐堅為副知院事,因為故事。判
 院一人,初在乾元殿,刊正官一人判事,其後因之。押院中使一人。自乾元殿寫書,則置掌出入,宣進奏,兼頌中官,監守院門,掌同宮禁。



 侍講學士,開元初,褚無量、馬懷素侍講禁中,名為侍讀。其後康子元為侍講學士。修撰官,校理官,並無常員,以官人兼之。待制官,古之待詔金馬門是。



 留院官,檢討官。皆以學士別敕留之。孔目官一人,專知御書典四人,並開元五年置。



 知書官八人,開元五年置,掌分四庫書。書直、寫御書一百人,拓書六人,書直八人,裝書直十四人,造筆直四人。



 並開元六年置。集賢學士之職,掌刊緝古今之經籍,以辨明邦國之大典。凡天下圖書之遺逸,賢才之隱滯,則
 承旨而徵求焉。其有籌策之可施於時,著述之可行於代者,較其才藝而考其學術,而申表之。凡承旨撰集文章,校理經籍,月終則進課於內,歲終則考最於外。



 史館:歷代史官,隸秘書省著作局,皆著作郎掌修國史。武德因隋舊制。貞觀三年閏十二月,始移史館於禁中,在門下省北,宰相監修國史,自是著作郎始罷史職。及大明宮初成,置史館於門下省之南。館門下東西有棗樹七十四株,無雜樹。開元二十五年三月,右相李林甫以中書地切樞密,記事者官宜附近,史官尹愔奏移史館於中書省北,以舊尚藥院充館也。



 史官。古者天子諸侯,皆有史官,以紀言動、歷數之事。到後漢明帝,如當時名士入東觀,撰《光武紀》,而史官
 因以他官兼之。魏明帝始置著作郎,專掌國史,隸中書。晉改隸秘書省,因而不改。貞觀年修《五代史》,移史館於禁中。史官無常員,如有修撰大事,則用他官兼之,事畢日停。



 監修國史。貞觀已後,多以宰相監修國史,遂成故事也。修撰直館。



 天寶已後,他官兼領史職者,謂之史館修撰,初入為直館也。元和六年,宰相裴垍奏:「登朝官領史職者,並為修撰,未登朝官入館者,並為直館。修撰中以一人官高者判館事,其餘名目,並請不置。」從之。



 楷書手二十五人,典書四人,亭長二人,掌固六人,裝滿直一人,熟紙匠六人。史官掌修國史,不虛美,不隱惡,直書其事。凡天地日月之祥,山川封域之分,昭穆繼代之序,禮樂師旅之事,誅賞廢興之政,皆本於
 起居注、時政記,以為實錄,然後立編年之體,為褒貶焉。既終藏之於府。



 知匭使。天後垂拱二年,置匭以達冤滯。其制,一房四面,各以方色,東曰延恩,西曰申冤,南曰招諫,北曰通玄。所以申天下之冤滯,達萬人之情狀。蓋古善旌、誹謗木之意也。天寶九年,改匭為獻納。乾元元年,復名曰匭。垂拱已來,常以諫議大夫及補闕、拾遣一人充使,受納訴狀。每日暮進內,而晨出之也。



 翰林院。天子在大明宮,其院在右銀臺門內。在興慶宮,院在金明門內。若在西內,院在顯福門。若在東都、華清宮,皆有待詔之所。其待詔者,有詞學、經術、合煉、僧道、卜祝、術藝、書奕,各別院以稟之,日晚而退。其所重者詞學。武德、貞觀時,有溫大雅、魏徵、李百藥、岑文本、許敬宗、褚遂良。永徽後,有許敬宗、上官儀,皆召入禁中驅
 使,未有名目。乾封中,劉懿之劉禕之兄弟、周思茂、元萬頃、範履冰,皆以文詞召入待詔,常於北門候進止,時號北門學士。天后時,蘇味道、韋承慶,皆待詔禁中。中宗時,上官昭容獨當書詔之任。睿宗時、薛稷、賈膺福、崔湜,又代其任。玄宗即位,張說,陸堅、張九齡、徐安貞、張垍等,召入禁中,謂之翰林待詔。王者尊極,一日萬機,四方進奏、中外表疏批答,或詔從中出。宸翰所揮,亦資其檢討,謂之視草,故嘗簡當代士人,以備顧問。至德已後,天下用兵,軍國多務,深謀密詔,皆從中出。尤擇名士,翰林學士得充選者,文士為榮。亦如中書舍人例置學士六人,內擇年深德重者一人為承旨,所以獨承密命故也。德宗好文,尤難其選。貞元已後,為學士承旨者,多至宰相焉。



 內教坊。武德已來,置於禁中,以按習雅樂,以中官人充使。則天改為雲韶府,神龍復為教坊。



 習藝館。本名內文學館,選宮人有儒學者一人為學士,教習宮人。則天改為習藝館,又改為翰林內教
 坊,以事在禁中故也。



 秘書省。隸中書之下。漢代藏書之所,有延閣、廣內、石渠之藏。又御史中丞,在殿內,掌蘭臺秘書圖籍。後漢桓帝延熹二年,始置秘書監,屬太常寺,掌禁中圖書秘文,後並入中書。至晉惠帝,別置秘書寺,掌中外二閣圖書。梁武改寺為省。龍朔改為蘭臺,光宅改為麟臺,神龍復為秘書省。



 秘書監一員,從三品。監之名,後漢桓帝置,魏、晉不改。後周謂之外史下大夫。隋復為秘書監,從第三品。煬帝改為秘書令,武德復為監。龍朔改為蘭臺太史,天授改為麟臺監,神龍復為秘書監也。



 少監二員,從四品上。少監,隋煬帝置。龍朔改為蘭臺侍郎,天授為麟臺少監,神龍復為秘書少監。比置一員,太極初增置一員也。



 丞一員。從五品上。魏武帝置,丞二人。隋置一人,正第五品也。秘書監
 之職,掌邦國經籍圖書之事。有二局:一曰著作,二曰太史,皆率其屬而修其職。少監為之貳,丞掌判省事。



 秘書郎四員。從六品上。校書郎八人,正九品上。



 正字四人,正九品下。主事一人,從九品上。



 令史四人,書令史九人,典書八人,楷書手八十人,亭長六人。掌固八人。秘書郎掌甲乙丙丁四部之圖籍,謂之四庫。經庫類十,史庫類十三,子庫類十四,集庫類三。



 事在《經籍志》。



 著作局:龍朔為司文局。
 著作郎二人,從五品上。龍朔為司文郎中,咸亨復也。



 佐郎四人,從六品上。校書郎二人,正九品上。



 正字二人,正九品下。楷書手五人,掌固四人。著作郎、佐郎掌修撰碑志、祝文、祭文,與佐郎分判局事也。



 司天臺:舊太史局,隸秘書監。龍朔二年改為秘閣局,久視元年改為渾儀監。景雲元年改為太史監,復為太史局,隸秘書。乾元元年三月十九日敕,改太史監為司天臺,改置官屬,舊置於子城內秘書省西,今在永寧坊東南角也。



 監一人,從三品。本太史局令,從五品下。乾元元年改為監,升從三品,一如殿中秘書品秩也。



 少
 監二人。本曰太史丞,從七品下。乾元升為少監,與諸司少監卿同品也。太史令掌觀察天文,稽定歷數。凡日月星辰之變,風雲氣色之異,率其屬而占候之。



 其屬有司歷二人,掌造歷。保章正一人,掌教。歷生四十一人。監候五人,掌候天文。觀生九十人,掌晝夜司候天文氣色。



 靈臺郎二人,掌教習天文氣色。天文生六十人。挈壺正二人。掌知漏刻。司辰七十人,漏刻典事二十二人,漏刻博士九人,漏刻生三百六十人,典鐘一百一十二人,典鼓八十八人,楷書手二人,亭長、掌固各四人。自乾元元年別置司天臺。改置官吏,不同太史局舊數,今據司天職掌書之也。



 凡玄象器物、天文圖書,茍非其任,不得預焉。每季錄所見災
 祥,送門下中書省,入起居注。歲終總錄,封送史館。每年預造來年歷,頒於天下。五官正五員,正五品。乾元元年置五官,有春、夏、秋、冬、中五官之名。丞二員,正七品。



 主簿二員,正七品。定額直五人,五官靈臺郎五員,正七品。舊靈臺郎,正八品下,掌觀天文之變而占候之。凡二十八宿,分為十二次,事具《天文志》也。



 五官保章正五員,正七品。五官司歷五員,正八品。舊司歷二人,從九品上,掌國之歷法,造歷以頒四方。其歷有《戊寅歷》、《麟德歷》、《神龍歷》、《大衍歷》。天下之測量之處,分至表準,其詳可載,故參考星度,稽驗晷影,各有典章。



 五官監候五員,正八品。五官挈壺正五員,正九品。五官司
 辰十五員。正九品。舊挈壺正二員,從八品下。司辰十七人,正九品下。皆掌知漏刻。孔壺為漏,浮箭為刻,以告中星昏明之候也。



 五官禮生十五人,五官楷書手五人,令史五人,漏刻博士二十人,漏刻之法,孔壺為漏,浮箭為刻。其箭四十有八,晝夜共百刻。冬夏之間,有長短。冬至之日,晝漏四十刻,夜漏六十刻。夏至,晝漏六十刻夜漏四十刻。春分秋分之時,晝夜各五十刻。秋分之後,減晝益夜,凡九日加一刻。春分已後,減夜益晝,九日減一刻。二至前後,加減遲,用日多。二分之間,加減速,用日少。候夜以為更點之節。每夜分為五更,每更分為五點。更以擊鼓為節,點以擊鐘為節也。



 典鐘、典鼓三百五十人,天文觀生九十人,天文生五十人,歷生五十五人,漏生四十人,視品十人。已上官吏,皆乾元元年隨監司
 新置也。



\end{pinyinscope}