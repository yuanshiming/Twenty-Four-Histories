\article{卷一百九十二}

\begin{pinyinscope}

 ○姚紹之周利貞王旭吉溫王鈞嚴安之盧鉉附羅希奭毛若虛敬羽裴升畢曜附



 姚紹之,湖州武康人也。解褐典儀,累拜監察御史。中宗
 朝,武三思恃庶人勢,駙馬都尉王同皎謀誅之。事洩,令紹之按問而誅同皎。紹之初按問同皎,張仲之、祖延慶謀衣袖中發調弩射三思,伺其便,未果。宋之遜以其外妹妻延慶,曰:「今日將行何事,而以妻為?」之遜固抑與延慶,且洽其心矣。之遜子曇密發之,乃敕右臺大夫李承嘉與紹之按於新開門內。



 初,紹之將直盡其事。詔宰相李嶠等對問。諸相懼三思威權,但僶俛佯不問。仲之、延慶言曰:「宰相中有附會三思者。」嶠與承嘉耳言,復說誘
 紹之,其事乃變。遂密置人力十餘,命引仲之對問。至,即為紹之所擒,塞口反接,送獄中。紹之還,謂仲之曰:「張三,事不諧矣!」仲之固言三思反狀,紹之命棒之而臂折,大呼天者六七。謂紹之曰:「反賊,臂且折矣,命已輸汝,當訴爾於天帝!」因裂衫以束之,乃自誣反而遇誅。紹之自此神氣自若,朝廷側目。累遷左臺侍御史。奉使江左,經汴州,辱錄事參軍魏傳弓。尋拜監察御史。紹之後坐贓污,詔傳弓按之,獲贓五千餘貫以聞,當坐死。韋庶人妹保
 持之,遂黜放為嶺南瓊山尉。傳弓初按紹之,紹之在揚州,色動,謂長吏盧萬石曰:「頃辱傳弓,今為所按,紹之死矣!」逃入西京,為萬年尉擒之,擊折其足,因授南陵令員外置。開元十三年,累轉括州長史同正員,不預知州事,死。



 周利貞,神龍初為侍御史。附托權要,為桓彥範、敬暉等五王嫉之,出為嘉州司馬。時中書舍人崔湜與桓、敬善。武三思用事禁中,彥範憂之,托心腹於湜。湜反露其計
 於三思,為三思所中,盡流嶺南。湜勸盡殺之。以絕其歸望。三思問:「誰可使者?」利貞即湜之表兄,因舉為此行。利貞至,皆鴆殺之,因擢為左臺御史中丞。先天元年,為廣州都督。時湜為中書令,與僕射劉幽求不葉,陷幽求徙於嶺表,諷利貞殺之,為桂州都督王晙護之,逗留獲免。無何,玄宗正位,利貞與薛季昶、宋之問同賜死於桂州驛。



 王旭,太原祁人也。曾祖珪,貞觀初為侍中,尚永寧公
 主。旭解褐鴻州參軍,轉兗州兵曹。神龍元年正月,張柬之、桓彥範等誅張易之、昌宗兄弟,尊立孝和皇帝。其兄昌儀先貶乾封尉,旭斬之,齎其首,赴於東都。遷並州錄事參軍。唐隆元年,玄宗誅韋庶人等。並州長史周仁軌,韋氏之黨,有詔誅之。旭不覆敕,又斬其首,馳赴西京。



 開元二年,累遷左臺侍御史。時光祿少卿盧崇道以崔湜妻父,貶於嶺外。逃歸,匿於東都,為仇家所發,詔旭究其獄。旭欲擅其威權,因捕崇道親黨數十人,皆極其楚毒,然
 後結成其罪。崇道及三子並杖死於都亭驛,門生親友皆決杖流貶。時得罪多是知名之士,四海冤之。旭又與御史大夫李傑不葉,遞相糾訐,傑竟左遷衢州刺史。旭既得志,擅行威福,由是朝廷畏而鄙之。



 五年,遷左司郎中,常帶侍御史。旭為吏嚴苛,左右無敢支梧,每銜命推劾,一見無不輸款者。時宋王憲府掾紀希虯兄任劍南縣令,被告有贓私,旭使至蜀鞫之。其妻美,旭威逼之,因奏決殺縣令,納贓數千萬。至六年,希虯遣奴詐為祗
 承人,受顧在臺,事旭累月。旭賞之,召入宅中,委以腹心。其奴密記旭受饋遺囑托事,乃成數千貫,歸謁希虯。希虯銜泣見憲,敘以家冤。憲憫之,執其狀以奏,詔付臺司劾之。贓私累巨萬,貶龍平尉,憤恚而死,甚為時人之所慶快。



 吉溫,天官侍郎頊弟琚之孽子也。譎詭能諂事人,游於中貴門,愛若親戚。性禁害,果於推劾。天寶初,為新豐丞。時太子文學薛嶷承恩幸,引溫入對。玄宗目之而謂嶷
 曰:「是一不良漢,朕不要也。」時蕭炅為河南尹,河南府有事,京臺差溫推詰,事連炅,堅執不舍,賴炅與右相李林甫善,抑而免之。及溫選,炅已為京兆尹,一唱萬年尉,即就其官,人為危之。時驃騎高力士常止宿宮禁,或時出外第,炅必謁焉。溫先馳與力士言謔甚洽,握手呼行第,炅覷之嘆伏。及他日,溫謁炅於府庭,遽布心腹曰:「他日不敢隳國家法,今日已後,洗心事公。」炅復與盡歡。



 會林甫與左相李適之、駙馬張垍不葉,適之兼兵部尚書,垍
 兄均為兵部侍郎,林甫遣人訐出兵部銓曹主簿事令史六十餘人偽濫事,圖覆其官長,詔出付京兆府與憲司對問。數日,竟不究其由。炅使溫劾之。溫於院中分囚於兩處,溫於後佯取兩重囚訊之,或杖或壓,痛苦之聲,所不忍聞。即云:「若存性命,乞紙盡答。」



 令史輩素諳溫,各自誣伏罪,及溫引問,無敢違者。晷刻間事輯,驗囚無栲訊決罰處。常云:「若遇知己,南山白額獸不足縛也。」會李林甫將起刑獄,除不附己者,乃引之於門,與羅希奭
 同鍛煉詔獄。



 五載,因中官納其外甥武敬一女為盛王琦妃,擢京兆府士曹。時林甫專謀不利於東儲,以左驍衛兵曹柳湜杜良娣妹婿,令溫推之。溫追著作郎王曾、前右司禦率府倉曹王修己、左武衛司戈盧寧、左威衛騎曹徐徵同就臺鞫,數日而獄成。勣等杖死,積尸於大理寺。



 六載,林甫又以戶部侍郎、兼御史中丞楊慎矜違忤其旨,御史中丞王鉷與慎矜親而嫉之,同構其事,云:「蓄圖讖,以己是隋煬帝子孫,規於興復」,林甫又奏付溫
 鞫焉,慎矜下獄系之。使溫於東京收捕其兄少府少監慎餘、弟洛陽令慎名,於汝州捕其門客史敬忠。敬忠頗有學,嘗與朝貴游。蹉跎不進。與溫父琚情契甚密,溫孩孺時,敬忠嘗抱撫之。溫令河南丞姚開就擒之,鎖其頸,布袂蒙面以見溫。溫驅之於前,不交一言。欲及京,使典誘之云:「楊慎矜今款招己成,須子一辨。若解人意,必活;忤之,必死。」敬忠回首曰:「七郎,乞一紙。」溫佯不與,見詞懇,乃於桑下令答,三紙辯皆符溫旨。喜曰:「丈人莫相怪!」遂
 徐下拜。及至溫湯,始鞫慎矜,以敬忠詞為證。及再搜其家,不得圖讖。林甫恐事洩,危之,乃使御史盧鉉入搜。鉉乃袖讖書而入,於隱僻中詬而出曰:「逆賊牢藏秘記,今得之矣!」指於慎矜小妻韓珠團婢,見舉家惶懼,且行捶擊,誰敢忤焉!獄乃成,慎矜兄弟賜死。溫自是威振,衣冠不敢偶言。



 溫早以嚴毒聞,頻知詔獄,忍行枉濫,推事未訊問,已作奏狀,計贓數。及被引問,便懾懼,即隨意而書,無敢惜其生者。因不加栲擊,獄成矣。林甫深以溫為能,
 擢戶部郎中,常帶御史。林甫雖倚以爪牙,溫又見安祿山受主恩,驃騎高力士居中用事,皆附會其間,結為兄弟。常謂祿山曰:「李右相雖觀察人事,親於三兄,必不以兄為宰相。溫雖被驅使,必不超擢。若三兄奏溫為相,即奏兄堪大任,擠出林甫,是兩人必為相矣。」祿山悅之。



 時祿山承恩無敵,驟言溫能,玄宗亦忘曩歲之語。十載,祿山加河東節度,因奏溫為河東節度副使,並知節度營田及管內採訪監察留後事。其載,又加兼雁門太守,仍
 知安邊郡鑄錢事,賜紫金魚袋。及丁所生憂,祿山又奏起復為本官。尋復奏為魏郡太守、兼侍御史。



 楊國忠入相,素與溫交通,追入為御史中丞,仍充京畿、關內採訪處置使。溫於範陽辭,祿山令累路館驛作白紬帳以候之,又令男慶緒出界送,攏馬出驛數十步。及至西京,朝廷動靜,輒報祿山,信宿而達。



 十三載正月,祿山入朝,拜左僕射,充閑廄使。因奏加溫武部侍郎、兼御史中丞,充閑廄、苑內、營田、五坊等副使。時楊國忠與祿山嫌隙已
 成,溫轉厚於祿山,國忠又忌之。其冬,河東太守韋陟入奏於華清宮,陟自謂失職,托於溫結歡於祿山,廣載河東土物饋於溫,又及權貴。國忠諷評事吳豸之使鄉人告之,召付中書門下,對法官鞫之,陟伏其狀,貶桂嶺尉,溫澧陽長史。溫判官員錫新興尉。



 明年,溫又坐贓七千匹及奪人口馬奸穢事發,貶端州高要尉。溫至嶺外,遷延不進,依於張博濟,止於始安郡。八月,遣大理司直蔣沇鞫之。溫死於獄中,博濟及始安太守羅希奭死於州
 門。



 初,溫之貶斥,玄宗在華清宮,謂朝臣曰:「吉溫是酷吏子侄,朕被人誑惑,用之至此。屢勸朕起刑獄以作威福,朕不受其言。今去矣,卿等皆可安枕也!」初,開元九年,有王鈞為洛陽尉。十八年,有嚴安之為河南丞。皆性毒虐,笞罰人畏其不死,皆杖訖不放起,須其腫憤,徐乃重杖之,懊血流地,苦楚欲死,鈞與安之始眉目喜暢,故人吏懾懼。溫則售身權貴,噬螫衣冠,來頗異耳。溫九月死始興。十一月,祿山起兵作亂,人謂與溫報仇耳。祿山入洛
 陽城,即偽位。玄宗幸蜀後,祿山求得溫一子,才六七歲,授河南府參軍,給與財帛。



 初,溫之按楊慎矜,侍御史盧鉉同其事。鉉初為御史,作韋堅判官。及堅為李林甫所嫉,鉉以堅款曲發於林甫,冀售其身。及按慎矜,鉉先與張瑄同臺,情旨素厚,貴取媚於權臣,誣瑄與楊慎矜共解圖讖。持之,為驢駒板橛以成其獄。又為王鉷閑廄判官,鉷緣邢縡事朝堂被推,鉉證云:「大夫將白帖索廄馬五百匹以助逆,我不與之。」鉷死在晷刻,鉉忍誣之,眾咸
 怒恨焉。及被貶為廬江長史,在郡忽見瑄為祟,乃云:「端公何得來乞命?不自由。」鉉須臾而卒。



 羅希奭,本杭州人也,近家洛陽,鴻臚少卿張博濟堂外甥。為吏持法深刻。天寶初,右相李林甫引與吉溫持獄,又與希奭姻婭,自御史臺主簿再遷殿中侍御史。自韋堅、皇甫惟明、李適之、柳勣、裴敦復、李邕、鄔元昌、楊慎矜、趙奉璋下獄事,皆與溫鍛煉,故時稱「羅鉗吉網」,惡其深刻也。八載,除刑部員外,轉郎中。十一載,李林甫卒,出
 為中部、始安二太守,仍充當管經略使。



 十四載,以張博濟、吉溫,韋陟、韋誡奢、李從一、員錫等流貶,皆於始安,希奭或令假攝。右相楊國忠奏遣司直蔣沇往按之,復令張光奇替為始安太守。仍降敕曰:



 前始安郡太守、充當管經略使羅希奭,幸此資序,叨居牧守。地列要荒,人多竄殛,尤加委任,冀絕奸訛。翻乃嘯結逋逃,群聚不逞,應是流貶,公然安置。或差攝郡縣,割剝黎氓;或輟借館宇,侵擾人吏。不唯輕侮典憲,實亦隳壞紀綱。擢發數愆,豈多
 其罪,可貶海東郡海康尉、員外置。張博濟往托回邪,跡惟憑恃,嘗自抵犯,又坐親姻,前後貶官,歲月頗久,逗留不赴,情狀難容。及命按舉,仍更潛匿,亡命逭刑,莫斯為甚。並當切害,合峻常刑,宜於所在各決重杖六十。使夫為政之士,克守章程;負罪之人,期於悛革。凡厥在位,宜各悉心。



 時員錫、李從一、韋誡奢、吉承恩並決杖,遣司直宇文審往監之。



 毛若虛,絳州太平人也。眉毛覆於眼,其性殘忍。初為蜀
 川縣尉,使司以推勾見任。天寶末,為武功丞,年已六十餘矣。肅宗收兩京,除監察御史。審國用不足,上策征剝財貨。有潤於公者,日有進奉,漸見任用稱旨。每推一人,未鞫,即先收其家資,以定贓數。不滿望,即攤征鄉里近親。峻其威權,人皆懼死,輸納不差晷刻。



 乾元二年,鳳翔府七坊押官先行剽劫,州縣不能制,因有劫殺事。縣尉謝夷甫因眾怒,遂搒殺之。其妻訴於李輔國,輔國奏請御史孫瑩鞫之。瑩不能正其事。又令中丞崔伯陽三司
 使雜訊之,又不證成其罪。因令若虛推之,遂歸罪於夷甫。伯陽與之言,若虛頗不遜。伯陽數讓之,若虛馳謁告急。肅宗曰:「卿且出。」對曰:「臣出即死矣。」肅宗潛留若虛簾內,召伯陽至,伯陽頗短若虛。上怒,叱出之。因流貶伯陽同推官十餘人,皆於嶺外遠惡處。宰相李峴以左右於瑩等,亦被貶斥。於是若虛威震朝列,公卿懾懼矣!尋擢為御史中丞。上元元年,貶賓化尉而死。



 敬羽,寶鼎人也。父昭道,開元初為監察御史。羽貌寢而
 性便僻,善候人意旨。天寶九載,為康成縣尉。安思順為朔方節度使,引在幕下。及肅宗於靈武即大位,羽尋擢為監察御史。以苛刻征剝求進。及收兩京後,轉見委任。作大枷,有鸘尾榆,著即悶絕。又臥囚於地,以門關輾其腹,號為「肉飀飥」。掘地為坑,實以棘刺,以敗席覆上,領囚臨坑訊之,必墜其中,萬刺攢之。又捕逐錢貨,不減毛若虛。



 上元中,擢為御史中丞。太子少傅、宗正卿、鄭國公李遵,為宗子通事舍人李若冰告其贓私,詔羽按之。羽延
 遵,各危坐於小床。羽小瘦,遵豐碩,頃間問即倒,請垂足。羽曰:「尚書下獄是囚,羽禮延坐,何得慢耶!」遵絕倒者數四。請問,羽徐應之,授紙筆,書贓數千貫,奏之。肅宗以勛舊舍之,但停宗正卿。



 及嗣薛王珍潛謀不軌,詔羽鞫之。羽召支黨羅於廷,索鸘尾榆枷之,布栲訊之具以繞之,信宿成獄。珍坐死,右衛將軍竇如玢、試都水使者崔昌等九人並斬,太子洗馬趙非熊、陳王府長史陳閎、楚州司馬張昴、左武衛兵曹參軍焦自榮,前鳳翔府郿縣主
 簿李、廣文館進士張夐等六人決殺,駙馬都尉薛履謙賜自盡,左散騎常侍張鎬貶辰州司戶。



 胡人康謙善賈,資產億萬計。楊國忠為相,授安南都護。至德中,為試鴻臚卿,專知山南東路。驛人嫉之,告其陰通史朝義。謙髭須長三尺過帶,按之兩宿,鬢發皆禿,膝踝亦栲碎,視之者以為鬼物,非人類也。乞舍其生,以後送狀奏殺之,沒其資產。



 羽與毛若虛在臺五六年間,臺中囚系不絕。又有裴升、畢曜同為御史,皆酷毒。人之陷刑,當時有毛、
 敬、裴、畢之稱。



 裴、畢尋又流黔中。羽,寶應元年貶為道州刺史。尋有詔殺之,羽聞之,衣兇服南奔溪洞,為吏所擒。臨刑,袖中執州縣官吏犯贓私狀數紙,曰:「有人通此狀,恨不得推究其事。主州政者,無宜寢也。」



 贊曰:王德將衰,政在奸臣。鷹犬搏擊,縱之者人。遭其毒螫,可為悲辛。作法為害,延濫不仁。



\end{pinyinscope}