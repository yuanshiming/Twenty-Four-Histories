\article{卷二百六}

\begin{pinyinscope}

 回紇,其先匈奴之裔也。在後魏時,號鐵勒部落。其象微小,其俗驍強,依托高車,臣屬突厥,近謂之特勒。無君長,居無恆所,隨水草流移。人性兇忍,善騎射,貪婪尤甚,以
 寇抄為生。自突厥有國,東西征討,皆資其用,以制北荒。隋開皇末,晉王廣北征突厥,大破步迦可汗,特勒於是分散。大業元年,突厥處羅可汗擊特勒諸部,厚斂其物。又猜忌薛延陀,恐為變,遂集其渠帥數百人盡誅之,特勒由是叛。特勒始有僕骨、同羅、回紇、拔野古、覆羅步,號俟斤,後稱回紇焉。在薛延陀北境,居娑陵水側,去長安六千九百里。隨逐水草,勝兵五萬,人口十萬人。



 初,有特健俟斤死,有子曰菩薩,部落以為賢而立之。貞觀初,菩
 薩與薛延陀侵突厥北邊,突厥頡利可汗遣子欲谷設率十萬騎討之。菩薩領騎五千與戰,破之於馬鬣山。因逐北至於天山,又進擊,大破之,俘其部眾,回紇由是大振。因率其眾附於薛延陀,號菩薩為「活頡利發」,仍遣使朝貢。



 菩薩勁勇,有膽氣,善籌策,每對敵臨陣,必身先士卒,以少制眾,常以戰陣射獵為務。其母烏羅渾,主知爭訟之事,平反嚴明,部內齊肅。回紇之盛,由菩薩之興焉。



 貞觀中,擒降突厥頡利等可汗之後,北虜唯菩薩、薛延
 陀為盛。太宗冊北突厥莫賀咄為可汗,遣統回紇、僕骨、同羅、思結、阿跌等部,回紇酋帥吐迷度與諸部大破薛延陀多彌可汗,遂並其部曲,奄有其地。



 貞觀二十年,南過賀蘭山,臨黃河,遣使入貢,以破薛延陀功,賜宴內殿。太宗幸靈武,受其降款,因請回鶻已南置郵遞,通管北方。太宗為置六府七州,府置都督,州置刺史,府州皆置長史、司馬已下官主之。以回紇部為瀚海府,拜其俟利發吐迷度為懷化大將軍,兼瀚海都督。時吐迷度已自
 稱可汗,署官號皆如突厥故事。以多覽為燕然府,僕骨為金徽府,拔野古為幽陵府,同羅為龜林府,思結為盧山府,渾都為皋蘭州,斛薩為高闕州,阿跌為雞田州,契苾為榆溪州,跌結為雞鹿州,阿布思為歸林州,白霫為寘顏州。又以回紇西北結骨為堅昆府,其北骨利乾為玄闕州,東北俱羅勃為燭龍州。於故單于臺置燕然都護府統之,以導賓貢。



 貞觀二十二年,吐迷度為其侄烏紇所殺。初,烏紇蒸其叔母,遂與俱陸莫賀達干俱羅
 勃潛謀殺吐迷度以歸車鼻。烏紇、俱羅勃,並車鼻之婿也,烏紇遂夜領騎十餘劫吐迷度,殺之。燕然副都護元禮臣遣人紿烏紇云:「將奏而為都督,替吐迷度也。」烏紇輕騎至禮臣所,跪拜致謝;禮臣擒而斬之以聞。



 太宗恐回紇部落攜離,十月,遣兵部尚書崔敦禮往安撫之,仍以敦禮為金山道副將軍。贈吐迷度左衛大將軍,賻物及衣服設祭甚厚。以吐迷度子前左屯衛大將軍翊、左郎將婆閏為左驍衛大將軍、大俟利發、使持節回紇部
 落諸軍事、瀚海都督。後俱羅勃來朝,太宗留之不遣。詔西突厥可汗阿史那賀魯統五啜、五俟斤二十餘部,居多羅斯水南,去西州馬行十五日程。回紇不肯西屬突厥。



 永徽二年,賀魯破北庭,詔將軍梁建方、契苾何力領兵二萬,取回紇五萬騎,大破賀魯,收復北庭。顯慶元年,賀魯又犯邊。詔程知節、蘇定方、任雅相、蕭嗣業領兵並回紇大破賀魯於陰山,再破於金牙山,盡收所據之地,西遂至耶羅川。賀魯西奔石國,婆閏隨蘇定方逐賀魯
 至石國西北蘇咄城,城主伊涅達干執賀魯送洛陽。以其地置濛池、昆陵府,以阿史那彌射、阿史那步真為二府都督,統十姓左廂五弩失畢、右廂五咄陸。以賀魯種落分置州縣,西盡波斯。加婆閏右衛大將軍、兼瀚海都督。



 永徽六年,回鶻遣兵隨蕭嗣業討高麗。龍朔中,婆閏死,妹比粟毒主領回鶻,與同羅、僕固犯邊。高宗命鄭仁泰討平僕固等,比粟毒敗走,因以鐵勒本部為天山縣。永隆中,獨解支,嗣聖中,伏帝匐。開元中,承宗、伏帝難,並
 繼為酋長,皆受都督號,以統蕃州,左殺右殺分管諸部。



 開元中,回鶻漸盛,殺涼州都督王君掞,斷安西諸國入長安路。玄宗命郭知運等討逐,退保烏德健山,南去西城一千七百里。西城即漢之高闕塞也。西城北去磧石口三百里。



 有十一都督,本九姓部落:一曰藥羅葛,即可汗之姓;二曰胡咄葛;三曰咄羅勿;四曰貊歌息訖;五曰阿勿嘀;六曰葛薩;七曰斛嗢素;八曰藥勿葛;九曰奚耶勿。每一部落一都督。破拔悉密,收一部落,破葛邏祿,收
 一部落,各置都督五人,統號十一部落。每行止鬥戰,常以二客部落為軍鋒。



 天寶初,其酋長葉護頡利吐發遣使入朝,封奉義王。三載,擊破拔悉密,自稱骨咄祿毗伽闕可汗。又遣使入朝,因冊為懷仁可汗。及至德元載七月,肅宗於靈武即位。遣故邠王男承採,封為燉煌王,將軍石定番,使於回紇,以修好征兵。及至其牙,可汗以女嫁於承採,遣首領來朝,請和親,封回紇公主為毗伽公主。肅宗在彭原,遇之甚厚。二載二月,回紇又使首領大
 將軍多攬等十五人入朝。九月戊寅,加承採開府儀同三司,拜宗正卿,納回紇公主為妃。回紇遣其太子葉護領其將帝德等兵馬四千餘眾,助國討逆。肅宗宴賜甚厚。又命元帥廣平王見葉護,約為兄弟,接之頗有恩義。葉護大喜,謂王為兄。



 戊子,回紇大首領達干等一十三人先至扶風,與朔方將士見僕射郭子儀。留之,宴設三日。葉護太子曰:「國家有難,遠來相助,何暇食為!」子儀固留之,宴畢便發。其軍每日給羊二百口、牛二十頭、米四
 十碩。及元帥廣平王率郭子儀等至香積寺東二十里,西臨澧水。賊埋精騎於大營東,將襲我軍之背。朔方左廂兵馬使僕固懷恩指回紇馳救之,匹馬不歸,因收西京。十月,廣平王、副元帥郭子儀領回紇兵馬,與賊戰於陜西。



 初,次於曲沃,葉護使其將軍車鼻施吐撥裴羅等旁南山而東,遇賊伏兵於谷中,盡殪之。子儀至新店,遇賊戰,軍卻數里。回紇望見,逾山西嶺上曳白旗而趨擊之,直出其後,賊眾大敗,軍而北坑,逐北二十餘里,人馬
 相枕藉,蹂踐而死者不可勝數,斬首十餘萬,伏尸三十里。賊黨嚴莊馳告安慶緒,率其黨背東京北走渡河,而葉護從廣平王、僕射郭子儀入東京。



 初,收西京,回紇欲入城劫掠,廣平王固止之。及收東京,回紇遂入府庫收財帛,於市井村坊剽掠三日而止。財物不可勝計,廣平王又齎之以錦罽寶貝,葉護大喜。及肅宗還西京,十一月癸酉,葉護自東京至。敕百官於長樂驛迎,上御宣政殿宴勞之。葉護升殿,其餘酋長列於階下,賜錦繡繒彩
 金銀器皿。及辭歸蕃,上謂曰:「能為國家就大事成義勇者,卿等力也。」葉護奏曰:「回紇戰兵,留在沙苑,今且須歸靈夏取馬,更收範陽,討除殘賊。」己丑,詔曰:「功濟艱難,義存邦國,萬里絕域,一德同心,求之古今,所未聞也。回紇葉護,特稟英姿,挺生奇略,言必忠信,行表溫良,才為萬人之敵,位列諸蕃之長。屬匈醜亂常,中原未靖,以可汗有兄弟之約,與國家興父子之軍,奮其智謀,討彼兇逆,一鼓作氣,萬里摧鋒,二旬之間,兩京克定。力拔山嶽,精貫
 風雲,蒙犯不以辭其勞,急難無以逾其分。固可懸之日月,傳之子孫,豈惟裂土之封,誓河之賞而已矣!夫位之崇者,司空第一;名之大者,封王最高。可司空、仍封忠義王,每載送絹二萬匹至朔方軍,宜差使受領。」



 乾元元年五月壬申朔,回紇使多亥阿波八十人,黑衣大食酋長閣之等六人並朝見,至閣門爭長,通事舍人乃分為左右,從東西門並入。六月戊戌,宴回紇使於紫宸殿前。



 秋七月丁亥,詔以幼女封為寧國公主出降。其降蕃日,
 仍以堂弟漢中郡王瑀為特進、試太常卿、攝御史大夫,充冊命英武威遠毗伽可汗使;以堂侄左司郎中巽為兵部郎中、攝御史中丞、鴻臚卿,副之,兼充寧國公主禮會使。特差重臣開府儀同三司、行尚書右僕射、冀國公裴冕送至界首。癸巳,以冊立回紇英武威遠毗伽可汗,上御宣政殿,漢中王瑀受冊命。甲午,肅宗送寧國公主至咸陽磁門驛,公主泣而言曰:「國家事重,死且無恨!」上流涕而還。及瑀至其牙帳,毗伽闕可汗衣赭黃袍,胡帽,
 坐於帳中榻上,儀衛甚盛,引瑀立於帳外,謂瑀曰:「王是天可汗何親?」瑀曰:「是唐天子堂弟。」又問:「於王上立者為誰?」瑀曰:「中使雷盧俊。」可汗又報曰:「中使是奴,何得向郎君上立?」雷盧俊竦懼,跳身向下立定。瑀不拜而立。可汗報曰:「兩國主君臣有禮,何得不拜?」瑀曰:「唐天子以可汗有功,故將女嫁與可汗結姻好,比者中國與外蕃親,皆宗室子女,名為公主。今寧國公主,天子真女,又有才貌,萬里嫁與可汗。可汗是唐家天子女婿,合有禮數。豈得
 坐於榻上受詔命耶!」可汗乃起奉詔,便受冊命。翼日,冊公主為可敦,蕃酋歡欣曰:「唐國天子貴重,將真女來。」瑀所送國信繒彩衣服金銀器皿,可汗盡分與衙官、酋長等。及瑀回,可汗獻馬五百匹、貂裘、白赩。八月,回紇使王子骨啜特勒及宰相帝德等驍將三千人助國討逆。肅宗嘉其遠至,賜宴,命隨朔方行營使僕固懷恩押之。九月甲申,回紇使大首領蓋將等謝公主下降,兼奏破堅昆五萬人,宴於紫宸殿,賜物有差。十二月甲午,回紇使
 三婦人,謝寧國公主之聘也,賜宴紫宸殿。



 乾元二年,回紇骨啜特勒等率眾從郭子儀與九節度於相州城下戰,不利。三月壬子,回紇王子骨啜特勒及宰相帝德等十五人自相州奔於西京,肅宗宴之於紫宸殿,賞物有差。其月庚寅,回紇特勒辭還行營,上宴之於紫宸殿,賜物有差。乙未,以回紇王子新除左羽林軍大將軍、員外置,骨啜特勒為銀青光祿大夫、鴻臚卿、員外置。



 夏四月,回紇毗伽闕可汗死。長子葉護先被殺,乃立其少子登
 里可汗,其妻為可敦。六月丙午,以左金吾衛將軍李通為試鴻臚卿、攝御史中丞,充吊祭回紇使。毗伽闕可汗初死,其牙官、都督等欲以寧國公主殉葬。公主曰:「我中國法,婿死,即持喪,朝夕哭臨,三年行服。今回紇娶婦,須慕中國禮。若今依本國法,何須萬里結婚。」然公主亦依回紇法,剺面大哭,竟以無子得歸。秋八月,寧國公主自回紇還,詔百官於明鳳門外迎之。



 上元元年九月己丑,回紇九姓可汗使大臣俱陸莫達干等入朝奉表起居。
 乙卯,回紇使二十人於延英殿通謁,賜物有差。十一月戊辰,回紇使延支伽羅等十人於延英殿謁見,賜物有差。



 寶應元年,代宗初即位,以史朝義尚在河洛,遣中使劉清潭徵兵於回紇,又修舊好。其秋,清潭入回紇庭,回紇已為史朝義所誘,云唐家天子頻有大喪,國亂無主,請發兵來收府庫。可汗乃領眾而南,已八月矣。清潭齎敕書國信至,可汗曰:「我聞唐家已無主,何為更有敕書?」中使對曰:「我唐家天子雖棄萬國,嗣天子廣平王天生
 英武,往年與回紇葉護兵馬同收兩京,破安慶緒,與可汗有故。又每年與可汗繒絹數萬匹,可汗豈忘之耶?」然回紇業已發至三城北,見荒城無戍卒,州縣盡為空壘,有輕唐色,乃遣使北收單于兵馬倉糧,又大辱清潭。清潭發使來奏云:「回紇登里可汗傾國自來,有眾十萬,羊馬不知其數。」京師大駭。上使殿中監藥子昂馳勞之。及於太原北忻州南,子昂密數其丁壯,得四千人,老小婦人相兼萬餘人,戰馬四萬匹,牛羊不紀。



 先是,毗伽闕可
 汗請以子婚,肅宗以僕固懷恩女嫁之。及是為可敦,與可汗同來,請懷恩及懷恩母相見。上敕懷恩自汾州見之於太原。懷恩又諫國家恩信不可違背。初欲自蒲關入,取沙苑路,由潼關東向破賊,子昂說之云:「國家頻遭寇逆,州縣虛乏,難為供擬,恐可汗失望,不如取土門路入,直取邢、洺、衛、懷。賊中兵馬盡在東京,可汗收其財帛,束裝南向,最為上策。」可汗不從。又說:「取懷州太行路,南據河陰之險,直扼賊之喉,亦上策也。」可汗又不從。又說:「
 取陜州太陽津路,食太原倉粟而東,與澤潞、河南、懷鄭節度同入,亦上策也。」可汗從之。子昂因入奏,上以雍王適為兵馬元帥,加懷恩同中書門下平章事。又以子昂兼御史中丞,與前潞府兼御史中丞魏琚為左右廂兵馬使,以中書舍人韋少華充元帥判官、兼掌書記,給事中李進兼御史中丞,充元帥行軍司馬,東會回紇。登里可汗營於陜州黃河北。



 元帥雍王領子昂等從而見之。可汗責雍王不於帳前舞蹈,禮倨。子昂辭以元帥是嫡
 孫,兩宮在殯,不合有舞蹈,回紇宰相及車鼻將軍庭詰曰:「唐天子與登里可汗約為兄弟,今可汗即雍王叔,叔侄有禮數,何得不舞蹈?」子昂苦辭以身有喪禮,不合。又報云:「元帥即唐太子也,太子即儲君也,豈有中國儲君向外國可汗前舞蹈。」相拒久之,車鼻遂引子昂、李進、少華、魏琚各搒捶一百,少華、琚因搒捶,一宿而死。以王少年未諳事,放歸本營。而懷恩與回紇右殺為先鋒,及諸節度同攻賊,破之,史朝義率殘寇而走。元帥雍王退歸
 靈寶。回紇可汗繼進於河陽,列營而止數月。去營百餘里,人被剽劫逼辱,不勝其弊。懷恩常為軍殿。及諸節度收河北州縣,僕固瑒與回紇之眾追躡二千餘里,至平州石城縣,梟朝義首而歸,河北悉平。懷恩自相州西出崞口路而西,可汗自河陽北出澤、潞與懷恩會,歷太原。遣使拔賀那上表賀收東京,並進逆賊史朝義旌旗等物。辭還蕃,代宗引見於內殿,賜彩二百段。



 初,回紇至東京,以賊平,恣行殘忍,士女懼之,皆登聖善寺及白馬寺
 二閣以避之。回紇縱火焚二閣,傷死者萬計,累旬火焰不止。及是朝賀,又縱橫大辱官吏。以陜州節度使郭英乂權知東都留守。時東都再經賊亂,朔方軍及郭英乂、魚朝恩等軍不能禁暴,與回紇縱掠坊市及汝、鄭等州,比屋蕩盡,人悉以紙為衣,或有衣經者。



 代宗御宣政殿,出冊文,加冊可汗為登里頡咄登密施含俱錄英義建功毗伽可汗,可敦加冊為婆墨光親麗華毗伽可敦。「頡咄」,華言「社稷法用」;「登密施」,華言「封竟」;「含俱錄」,華言「婁羅」;「
 毗伽」,華言「足意智」。「婆墨」,華言「得憐」。以散騎常侍兼御史大夫王翊充使,就可汗行營行冊命焉。可汗、可敦及左右殺、諸都督、內外宰相已下,共加實封二千戶,令王翊就牙帳前禮冊。左殺封為雄朔王,右殺封為寧朔王,胡祿都督封金河王,拔覽將軍封為靜漠王,諸都督一十一人並封國公。



 尋而懷恩叛,投靈武,有朔方舊將任敷、張韶等,收合餘燼,眾至數萬。廣德二年秋,乃引吐蕃之眾數萬人至奉天縣,朔方節度郭子儀率眾拒之而退。
 永泰元年秋,懷恩遣兵馬使範至誠、任敷將兵,又誘回紇、吐蕃、吐谷渾、黨項、奴刺之眾二十餘萬,以犯奉天、醴泉、鳳翔、同州等處,被其逆命。先以郭子儀屯涇陽,渾日進屯奉天,數摧其鋒。又聞懷恩死,吐蕃將馬重英等十月初引退,取邠州舊路而歸。回紇首領羅達干等率其眾二千餘騎,詣涇陽請降。子儀許之,率眾被甲持滿數千人。回紇譯曰:「此來非惡心,要見令公。」子儀曰:「我令公也。」回紇曰:「請去甲。」子儀便脫兜鍪槍甲,策馬挺身而前。
 回紇酋長相顧曰:「是也。」時太子太保李光進、兼御史大夫路嗣恭戎裝介馬在子儀之側。子儀指視回紇曰:「此是渭北節度李太保。」又曰:「此是朔方軍糧使路大夫。」回紇便下馬羅拜,子儀亦下馬。回紇之眾為左右翼,各數百人,漸進;子儀麾下亦馳而至,子儀麾退之。子儀命酒與之飲,贈之纏頭彩三千匹。子儀執回紇大將可汗弟合胡祿都督藥羅葛等手,責讓之曰:「我國家知汝回紇有功,報汝大厚,汝何背約負信,犯我王畿?我須與汝戰,
 何乃降為!我一身挺入汝營,任汝拘縶,我麾下將士,須與汝戰。」回紇又譯曰:「懷恩負心,來報可汗,云唐國天子今已向江淮,令公亦不主兵,我是以敢來。今知天可汗見在上郭,令公為將,懷恩天又殺之。今請追殺吐蕃,收其羊馬,以報國恩。然懷恩子,可敦兄弟,請勿殺之。」合胡祿都督等與宰相磨咄莫賀達干、宰相暾莫賀達干、宰相護都毗伽將軍、宰相揭拉裴羅達干、宰相梅錄大將軍羅達干、平章事海盈闕達干等,子儀先執杯,合胡祿
 都督請咒,子儀咒曰:「大唐天子萬萬歲!回紇可汗亦萬歲!兩國將相亦萬歲!若起負心違背盟約者,身死陣前,家口屠戮。」合胡祿都督等失色,及杯至,即譯曰:「如令公盟約。」皆喜曰:「初發本部來日,將巫師兩人來,云:『此行大安穩,然不與唐家兵馬鬥,見一大人即歸。』今日領兵見令公,令公不為疑,脫去衣甲,單騎相見,誰有此心膽!是不戰鬥見一大人,巫師有徵矣!」歡躍久之。子儀撫其背,首領等分纏頭彩以賞巫師,請諸將同擊吐蕃,子儀如
 其約。翌日,使領回紇首領開府古野那等六人入京朝見。



 又五日,朔方先鋒兵馬使、開府、南陽郡王白元光與回紇兵馬合於涇州靈臺縣西五十里赤山嶺,共破吐蕃等十餘萬眾,斬首五萬餘級,生擒一萬餘人,駝馬牛羊凡百里相繼,不可勝紀,收得蕃落五千餘人。初白元光等到靈臺縣西,探知賊勢,為月明,思少陰晦,回紇使巫師便致風雪。及遲明戰,吐蕃盡寒凍,弓矢皆廢,披氈徐進,元光與回紇隨而殺之蔽野。僕固名臣,懷恩之侄,
 尤為驍將,亦領千餘騎來降。尋而子儀又使回紇宰相護地毗伽將軍,宰相梅錄大將軍、開府儀同三司、試太常卿羅達干等一百九十六人來見。上賜宴於延英殿,錫齎甚厚。閏月,子儀自涇陽領僕固名臣入奏,回紇進馬,及宴別,前後齎繒彩十萬匹而還。時帑藏空虛,朝官無祿俸,隨月給手力,謂之資課錢。稅朝官閏十月、十一月、十二月課以供之。



 大歷六年正月,回紇於鴻臚寺擅出坊市,掠人子女,所在官奪返,毆怒,以三百騎犯金光
 門、硃雀門。是日,皇城諸門盡閉,上使中使劉清潭宣慰,乃止。



 七年七月,回紇出鴻臚寺,入坊市強暴,逐長安令邵說於含光門之街,奪說所乘馬將去。說脫身避走,有司不能禁。



 八年十一月,回紇一百四十人還蕃,以信物一千餘乘。回紇恃功,自乾元之後,屢遣使以馬和市繒帛,仍歲來市,以馬一匹易絹四十匹,動至數萬馬。其使候遣繼留於鴻臚寺者非一,蕃得帛無厭,我得馬無用,朝廷甚苦之。是時特詔厚賜遣之,示以廣恩,且俾知愧
 也。是月,回紇使使赤心領馬一萬匹來求市,代宗以馬價出於租賦,不欲重困於民,命有司量入計許市六千匹。



 十年九月,回紇白晝刺人於東市,市人執之,拘於萬年縣。其首領赤心聞之,自鴻臚寺馳入縣獄,劫囚而出,斫傷獄吏。



 十三年正月,回紇寇太原,過榆次、太谷,河東節度留後、太原尹、兼御史大夫鮑防與回紇戰於陽曲,我師敗績,死者千餘人。代州都督張光晟與回紇戰於羊武谷,破之,回紇引退。先是,辛云京守太原,回紇懼云
 京,不敢窺並、代,知鮑防無武略,乃敢凌逼,賴光晟邀戰勝之,北人乃安。



 德宗初即位,使中官梁文秀告哀於回紇,且修舊好,可汗移地健不為禮。而九姓胡素屬於回紇者,又陳中國便利以誘其心,可汗乃舉國南下,將乘我喪。其宰相頓莫賀達干諫曰:「唐,大國也,且無負於我。前年入太原,獲羊馬數萬計,可謂大捷矣。以道途艱阻,比及國,傷耗殆盡。今若舉而不捷,將安歸乎?」可汗不聽。頓莫賀乘人之心,因擊殺之,並殺其親信及九姓胡所
 誘來者凡二千人。



 頓莫賀自立號為合骨咄祿毗伽可汗,使其酋長建達干隨文秀來朝。命京兆尹源休持節冊為武義成功可汗。貞元三年八月,回紇可汗遣首領墨啜達干、多覽將軍合闕達干等來貢方物,且請和親。四年十月,回紇公主及使至自蕃,德宗御延喜門見之。時回紇可汗喜於和親,其禮甚恭,上言:「昔為兄弟,今為子婿,半子也。」又詈辱吐蕃使者,及使大首領等妻妾凡五十六婦人來迎可敦,凡遣人千餘,納聘馬二千。德宗
 令朔州、太原分留七百人,其宰相首領皆至,分館鴻臚,將作。癸巳,見於宣政殿。乙未,德宗召回紇公主出,使者對於麟德殿,各有頒賜。庚子,詔咸安公主降回紇可汗,仍置府官屬視親王例。以殿中監、嗣滕王湛然為咸安公主婚禮使,關播檢校右僕射、送咸安公主及冊回紇可汗使。貞元五年十二月,回紇汨咄祿長壽天親毗伽可汗薨,廢朝三日,文武三品已上就鴻臚寺吊其來使。



 貞元六年六月,回紇使移職伽達干歸蕃,賜馬價絹三
 十萬匹。以鴻臚卿郭鋒兼御史大夫,充冊回紇忠貞可汗使。是歲四月,忠貞可汗為其弟所殺而篡立。時回紇大將頡乾迦斯西擊吐蕃未回,其次相率國人縱殺纂者而立忠貞之子為可汗,年方十六七。及六月,頡乾迦斯西討回,將至牙帳,次相等懼其後有廢立,不欲漢使知之,留鋒數月而回。頡乾迦斯之至也,可汗等出迎郊野,陳郭鋒所送國信器幣,可汗與次將相等皆俯伏自說廢立之由,且請命曰:「惟大相生死之。」悉以所陳器幣
 贈頡乾迦斯以悅之。可汗又拜泣曰:「兒愚幼無知,今幸得立,惟仰食於阿爹。」可汗以子事之,頡乾迦斯以卑遜興感,乃相持號哭,遂執臣子之禮焉。盡以所陳器幣頒賜左右諸從行將士,己無所取,自是其國稍安,乃遣達比特勒梅錄將軍告忠貞可汗之哀於我,且請冊新君。使至,廢朝三日,仍令三品已上官就鴻臚寺吊其使。是歲,吐蕃陷北庭都護府。



 初,北庭、安西既假道於回紇以朝奏,因附庸焉。回紇徵求無厭,北庭差近,凡生事之資,
 必強取之。又有沙陀部落六千餘帳,與北庭相依,亦屬於回紇,肆行抄奪,尤所厭苦。其先葛祿部落及白服突厥素與回紇通和,亦憾其侵掠。因吐蕃厚賂見誘,遂附之。於是吐蕃率葛祿、白服之眾去冬寇北庭,回紇大相頡乾迦斯率眾援之,頻敗。吐蕃急攻之,北庭之人既苦回紇,乃舉城降焉,沙陀部落亦降。節度使、檢校工部尚書楊襲古將麾下二千餘眾出奔西州,頡干利亦還。十年秋,悉其國丁壯五萬人,召襲古,將復焉。俄為所敗,死
 者大半。頡干利收合餘燼,晨夜奔還。襲古餘眾僅百六十,將復入西州,頡乾迦斯紿之曰:「第與我同至牙帳,當送君歸本朝。」既及牙帳,留而不遣,竟殺之。自是安西阻絕,莫知存亡,唯西州之人,猶固守焉。頡士迦斯敗,葛祿乘勝取回紇之浮圖川,回紇震恐,悉遷西北部落羊馬於牙帳之南以避之。



 貞元七年五月庚申朔,以鴻臚少卿庾鋌兼御史大夫,冊回紇可汗及吊祭使。是月,回紇遣使律支達干等來朝,告小寧國公主薨。廢朝三日,故,
 肅宗以寧國公主降回紇,又以榮王女媵之;及寧國來歸,榮王女為可敦,回紇號為小寧國公主,歷配英武、英義二可汗。及天親可汗立,出居於外,生英武二子,為天親可汗所殺。無幾薨。七年八月,回紇遣使獻敗吐蕃、葛祿於北庭所捷及其俘畜。先是,吐蕃入靈州,為回紇所敗,夜以火攻,駭而退。十二月,回紇遣殺支將軍獻吐蕃俘大首領結心,德宗御延喜門觀之。八年七月,以回紇藥羅葛靈檢校右僕射。靈本唐人,姓呂氏,因入回紇,為
 可汗養子,遂以可汗姓為藥羅葛靈,在國用事。因來朝,寵賚甚厚,仍給市馬絹七萬匹。九年九月,遣使來朝貢。



 貞元十一年六月庚寅,冊拜回紇騰里邏羽錄沒密施合祿胡毗迦懷信可汗。元和四年,藹德曷里祿沒弭施合密毗迦可汗遣使改為回鶻,義取回旋輕捷如鶻也。八年四月,回鶻請和親,使伊難珠還蕃,宴於三殿,賜以銀器繒帛。是歲,回鶻數千騎至鷿鵜泉,邊軍戒嚴。十二月二日,宴歸國回鶻摩尼八人,令至中書見宰臣。先是,回
 鶻請和親,憲宗使有司計之。禮費約五百萬貫,方內有誅討,未任其親,以摩尼為回鶻信奉,故使宰臣言其不可。乃詔宗正少卿李孝誠使於回鶻,太常博士殷侑副之,諭其來請之意。



 長慶元年,毗迦保義可汗薨,輟朝三日,仍令諸司三品已上官就鴻臚寺吊其使者。四月,正衙冊回鶻君長為登羅羽錄密施句主錄毗伽可汗,以少府監裴通為檢校左散騎常侍、兼御史大夫,持節冊立、兼吊祭使。五月,回鶻宰相、都督、公主、摩尼等五百
 七十三人入朝迎公主,於鴻臚寺安置。敕:「太和公主出降回鶻為可敦,宜令中書舍人王起赴鴻臚寺宣示;以左金吾衛大將軍胡證檢校戶部尚書,持節充送公主入回鶻及冊可汗使;光祿卿李憲加兼御史中丞,充副使;太常博士殷侑改殿中侍御史,充判官。」吐蕃犯青塞堡,以回紇和親故也。鹽州刺史李文悅發兵擊退之。回鶻奏:「以一萬騎出北庭,一萬騎出安西,拓吐蕃以迎太和公主歸國。」其月敕:「太和公主出降回紇,宜持置府,其
 官屬宜視親王例。」



 回紇自咸安公主歿後,屢歸款請繼前好,久未之許。至元和末,其請彌切,憲宗以北虜有勛勞於王室,又西戎比歲為邊患,遂許以妻之。既許而憲宗崩。穆宗即位,逾年乃封第十妹為太和公主,將出降,回紇登邏骨沒密施合毗伽可汗遣使伊難珠、句錄都督思結並外宰相、駙馬、梅錄司馬,兼公主一人、葉護公主一人,及達干並駝馬千餘來迎。太和公主發赴回紇國,穆宗御通化門左個臨送,使百僚章敬寺前立班,儀
 衛甚盛,士女傾城觀焉。十一月,振武節度張惟清奏:「準詔發兵三千赴蔚州,數內已發一千人訖,餘二千人,待太和公主出界即發遣。」又奏:「天德轉牒云:回鶻七百六十人將駝馬及車,相次至黃蘆泉迎候公主。」豐州刺史李祐奏:「迎太和公主回鶻三千於卿泉下營拓吐蕃。」



 二年二月,賜回紇馬價絹五萬匹。三月,又賜馬價絹七萬匹。是月,裴度招討幽、鎮之亂,回鶻請以兵從度討伐。朝議以寶應初回紇收復兩京,恃功驕恣難制,咸以為不
 可,遂命中使止回紇令歸。會其已上豐州北界,不從。上詔發繒帛七萬匹賜之,方還。五月,命使冊立登囉骨沒密施合毗伽禮可汗,遣品官田務豐領國信十二車使回鶻,賜可汗及太和公主。



 長慶二年閏十月,金吾大將軍胡證、副使光祿卿李憲、婚禮使衛尉卿李銳、副使宗正少卿李子鴻、判官虞部郎中張敏、太常博士殷侑送太和公主至自回紇,皆云:初,公主去回紇牙帳尚可信宿,可汗遣數百騎來請與公主先從他道去。胡證曰:「不
 可。」虜使曰:「前咸安公主來時,去花門數百里即先去,今何獨拒我?」證曰:「我天子詔送公主以投可汗,今未見可汗,豈宜先往!」虜使乃止。既至虜庭,乃擇吉日,冊公主為回鶻可敦。可汗先升樓東向坐,設氈幄於樓下以居公主,使群胡主教公主以胡法。公主始解唐服而衣胡服,以一嫗侍,出樓前西向拜。可汗坐而視,公主再俯拜訖,復入氈幄中,解前所服而披可敦服,通裾大襦,皆茜色,金飾冠如角,前指後出樓,俯拜可汗如初禮。虜先設大
 輿曲扆,前設小座,相者引公主升輿,回紇九姓相分負其輿,隨日右轉於庭者九,公主乃降輿升樓,與可汗俱東向坐。自此臣下朝謁,並拜可敦。可敦自有牙帳,命二相出入帳中。證等將歸,可敦宴之帳中,留連號啼者竟日,可汗因贈漢使以厚貺。



 太和元年,命中使以絹二十萬匹付鴻臚寺宣賜回鶻充馬價。三年正月,中使以絹二十三萬匹賜回紇充馬價。七年三月,回紇李義節等將駝馬到,且報可汗三月二十七日薨,已冊親弟薩特
 勒。廢朝三日,仍令諸司文武三品、尚書省四品以上官就鴻臚寺吊其使者。以左驍衛將軍、皇城留守唐弘實為金吾將軍兼御史大夫,持節充入回鶻吊祭冊立使。九年六月,入朝回鶻進太和公主所獻馬射女子七人,沙陀小兒二人。開成初,其相有安允合者,與特勒柴草欲篡薩特勒可汗,薩特勒可汗覺,殺柴草及安允合。又有回鶻相掘羅勿者,擁兵在外,怨誅柴草、安允合,又殺薩特勒可汗,以馺特勒為可汗。有將軍句錄末賀恨
 掘羅勿,走引黠戛斯領十萬騎破回鶻城,殺馺,斬掘羅勿,燒蕩殆盡,回鶻散奔諸蕃。有回鶻相馺職者,擁外甥龐特勒及男鹿並遏粉等兄弟五人、一十五部西奔葛邏祿,一支投吐蕃,一支投安西,又有近可汗牙十三部,以特勒烏介為可汗,南來附漢。



 初,黠戛斯破回鶻,得太和公主。黠戛斯自稱李陵之後,與國同姓,遂令達干十人送公主至塞上。烏介途遇黠戛斯使,達干等並被殺。太和公主卻歸烏介可汗,乃質公主同行,南渡大磧。
 至天德界,奏請天德城與太和公主居。有回鶻相赤心者,與連位相姓僕固者,與特勤那頡啜擁部眾,不賓烏介。赤心欲犯塞,烏介遣其屬霡沒斯先布誠於天德軍使田牟,然後誘赤心宰相同謁烏介可汗,戮赤心於可汗帳下並僕固二人。那頡戰勝,全占赤心下七千帳,東瞰振武、大同,據室韋、黑沙、榆林,東南入幽州雄武軍西北界。幽州節度使張仲武遣弟仲至率兵大破那頡之眾,全收七千帳,殺戮收擒老小近九萬人。那頡中箭,透
 駝群潛脫,烏介獲而殺之。



 烏介諸部猶稱十萬眾,駐牙大同軍北閭門山,時會昌二年秋,頻劫東陜已北,天德、振武、雲朔,比罹俘戮。詔諸道兵悉至防捍,以河東節度使劉沔充南面招控回鶻使;以幽州節度使張仲武充東面招控回鶻使。



 二年冬,三年春,回鶻特勒龐俱遮、阿敦寧二部,回鶻公主密羯可敦一部,外相諸洛固阿跌一部,及牙帳大將曹磨你等七部,共三萬眾,相次降於幽州,詔配諸道。有特勒霡沒斯、阿歷支、習勿啜三部,回
 鶻相愛耶勿弘順、回鶻尚書呂衡等諸部降振武,三部首領皆賜姓李氏,及名思忠、思貞、思惠、思恩,充歸義使。有特勒葉被沽兄李二部南奔吐蕃,有特勒可質力二部東北奔大室韋,有特勒荷勿啜東討契丹,戰死。



 會昌三年,回鶻尚書僕固繹到幽州,約以太和公主歸幽州,烏介去幽州界八十里下營。其親信骨肉及摩尼志凈等四人已先入振武軍。是夜,河東劉沔率兵奄至烏介營,烏介驚走東北約四百里外,依和解室韋下營,不及
 將太和公主同走。豐州刺史石雄兵遇太和公主帳,因迎歸國。烏介部眾至大中元年詣幽州降,留者漂流餓凍,眾十萬,所存止三千已下。烏介嫁妹與室韋,托附之。為回鶻相美權者逸隱啜逼諸回鶻殺烏介於金山,以其弟特勒遏捻為可汗,復有眾五千以上,其食用糧羊皆取給於奚王碩舍朗。



 大中元年春,張仲武大破奚眾,其回鶻無所取給,日有耗散。至二年春,唯存名王貴臣五百人已下,依室韋。張鐘武因賀正室韋經過幽州,仲
 武卻令還蕃,遣送遏捻等來向幽州,遏捻等懼,是夜與妻葛祿、子特勒毒斯等九騎西走,餘眾奔之不及,回鶻諸相達官老幼大哭。室韋分回鶻餘眾為七分,七姓室韋各占一分。經三宿,黠戛斯相阿播領諸蕃兵稱七萬,從西南天德北界來取遏捻及諸回鶻,大敗室韋。回鶻在室韋者,阿播皆收歸磧北。在外猶數帳,散藏諸山深林,盜劫諸蕃,皆西向傾心望安西龐勒之到。龐勒已自稱可汗,有磧西諸城。其後嗣君弱臣強,居甘州,無復昔
 時之盛。到今時遣使入朝,進玉馬二物及本土所產,交易而返。



 史臣曰:自三代以前,兩漢之後,西羌、北狄,互興部族,其名不同,為患一也。蔡邕云:「邊陲之患,為手足之疥;中國之困,為胸背之疽。突厥為煬帝之患深矣,隋竟滅;中國之困,其理昭然。」自太宗平突厥,破延陀,而回紇興焉。太宗幸靈武以降之,置州府以安之,以名爵玉帛以恩之。其義何哉?蓋以狄不可盡,而以威惠羈縻之。開元中,三
 綱正,百姓足,四夷八蠻,翕然向化,要荒之外,畏威懷惠,不其盛矣!天寶末,奸臣弄權於內,逆臣跋扈於外,內外結釁而車駕遽遷,華夷生心而神器將墜。肅宗誘回紇以復京畿。代宗誘回紇以平河朔。戡難中興之功,大即大矣!然生靈之膏血已乾,不能供其求取;朝廷之法令並弛,無以抑其憑陵。忍恥和親,姑息不暇。僕固懷恩為叛,尤甚阽危;郭子儀之能軍,終免侵軼。比昔諸戎,於國之功最大,為民之害亦深。及勢利日隆,盛衰時變,冰消
 瓦解,如存若亡,竟為手足之疥焉。僖、昭之世,黃、硃迭興,竟為胸背之疽焉。手疥背疽,誠為確論。



 贊曰:土德初隆,比屋可封。朝綱中否,邊鄙興戎。安、史亂國,回紇恃功。恃功伊何?咸議姑息。民不聊生,國殫其力。華夷有截,盛衰如織。彼既長惡,我乃修德,疽疥之義,百代可則。



\end{pinyinscope}