\article{卷一百四}

\begin{pinyinscope}

 ○尹思貞李傑解琬畢構蘇珦子晉鄭惟忠王志愔盧從願李朝隱裴漼從祖弟寬王丘



 尹思貞,京兆長安人也。弱冠明經舉,補隆州參軍。時晉
 安縣有豪族蒲氏,縱橫不法,前後官吏莫能制。州司令思貞推按,發其奸贓萬計,竟論殺之,遠近稱慶,刻石以紀其事,由是知名。累轉明堂令,以善政聞。三遷殿中少監,檢校洺州刺史。會契丹孫萬榮作亂,河朔不安,思貞善於綏撫,境內獨無驚擾,則天降璽書褒美之。



 長安中,七遷秋官侍郎,以忤張昌宗被構,出為定州刺史,轉晉州刺史。尋復入為司府少卿。時卿侯知一亦厲威嚴,吏人為之語曰:「不畏侯卿杖,惟畏尹卿筆。」其為人所伏若
 此。尋加銀青光祿大夫。於宅中掘得古戟十二,俄而門加棨戟,時人異焉。



 神龍初,為大理卿,時武三思擅權,御史大夫李承嘉附會之。壅州人韋月將上變,告三思謀逆,中宗大怒,命斬之。思貞以發生之月,固執奏以為不可行刑,竟有敕決杖配流嶺南。三思令所司因此非法害之,思貞又固爭之。承嘉希三思旨,托以他事,不許思貞入朝廷。謂承嘉曰:「公擅作威福,不顧憲章,附托奸臣,以圖不軌,將先除忠良以自恣耶?」承嘉大怒,遂劾奏思
 貞,出為青州刺史。境內有蠶一年四熟者,黜陟使、衛州司馬路敬潛八月至州,見繭嘆曰:「非善政所致,孰能至於此乎!」特表薦之。思貞前後為十三州刺史,皆以清簡為政,奏課連最。



 睿宗即位,徵為將作大匠,累封天水郡公。時左僕射竇懷貞興造金仙、玉真兩觀,調發夫匠,思貞常節減之。懷貞怒,頻詰責思貞,思貞曰:「公職居端揆,任重弼諧,不能翼贊聖明,光宣大化,而乃盛興土木,害及黎元,豈不愧也!又受小人之譖,輕辱朝臣,今日之事,
 不能茍免,請從此辭。」拂衣而去,闔門累日,上聞而特令視事。其年,懷貞伏誅,乃下制曰:「國之副相,位亞中臺,自匪邦直,孰司天憲?將作大匠尹思貞,賢良方正,碩儒耆德,剛不護缺,清而畏知,簡言易從,莊色難犯。征先王之體要,敷衽必陳;折佞臣之怙權,拂衣而謝。故以事聞海內,名動京師,鷹隼是擊,豺狼自遠。必能條理前弊,發揮舊章,宜承弄印之榮,式允登車之志。可御史大夫。」俄兼申王府長史,遷戶部尚書,轉工部尚書。以老疾累表請
 致仕,許之。開元四年卒,年七十七,贈黃門監,謚曰簡。



 李傑,本名務光,相州滏陽人。後魏並州刺史寶之後也,其先自隴西徙焉。傑少以孝友著稱,舉明經,累遷天官員外郎,明敏有吏才,甚得當時之譽。神龍初,累遷衛尉少卿,為河東道巡察黜陟使,奏課為諸使之最。開元初,為河南尹。傑既勤於聽理,每有訴列,雖衢路當食,無廢處斷。由是官無留事,人吏愛之。先是,河、汴之間有梁公堰,年久堰破,江、淮漕運不通。傑奏調發汴、鄭丁夫以浚
 之,省功速就,公私深以為利,刊石水濱,以紀其績。



 尋代宋璟為御史大夫。時皇后妹婿尚衣奉御長孫昕與其妹婿楊仙玉因於里巷遇傑,遂毆擊之,上大怒,令斬昕等。散騎常侍馬情素以為陽和之月,不可行刑,累表陳請。乃下敕曰:「夫為令者自近而及遠,行罰者先親而後疏。長孫昕、楊仙玉等憑恃姻戚,恣行兇險,輕侮常憲,損辱大臣,情特難容,故令斬決。今群官等累陳表疏,固有誠請,以陽和之節,非肅殺之時,援引古今,詞義懇切。朕
 志從深諫,情亦惜法,宜寬異門之罰,聽從枯木之斃。即宜決殺,以謝百僚。」



 傑明年以護橋陵作,賜爵武威子。初,傑護作時,引侍御史王旭為判官。旭貪冒受贓,傑將繩之而不得其實,反為旭所構,出為衢州刺史。俄轉揚州大都督府長史,又為御史所劾,免官歸第。尋卒,贈戶部尚書。



 解琬,魏州元城人也。少應幽素舉,拜新政尉,累轉成都丞。因奏事稱旨,超遷監察御史,丁憂離職。則天以琬識
 練邊事,起復舊官,令往西域安撫夷虜,抗疏固辭。則天嘉之,下敕曰:「解琬孝性淳至,哀情懇切,固辭權奪之榮,乞就終憂之典。足可以激揚風俗,敦獎名教,宜遂雅懷,允其所請。仍令服闋後赴上。」



 聖歷初,遷侍御史,充使安撫烏質勒及十姓部落,咸得其便宜,蕃人大悅,以功擢拜御史中丞,兼北庭都護、持節西域安撫使。琬素與郭元振同官相善,遂為宗楚客所毀,由是左遷滄州刺史。為政務存大體,甚得人和。景龍中,遷右
 臺御史大夫,兼持節朔方行軍大總管。琬前後在軍二十餘載,務農習戰,多所利益,邊境安之。



 景雲二年,復為朔方軍大總管。琬分遣隨軍要籍官河陽丞張冠宗、肥鄉令韋景駿、普安令於處忠等校料三城兵募,於是減十萬人,奏罷之。尋授右武衛大將軍,兼檢校晉州刺史,賜爵濟南縣男。以年老乞骸骨,拜表訖,不待報而去。優詔加金紫光祿大夫,聽致仕,其祿準品全給。尋降璽書勞之曰:「卿器局堅正,才識高遠,公忠彰其立身,貞固足以幹事。類張騫
 之出使,同魏絳之和戎。職綰文武,功申方面,勤於王家,是為國老。頃者,顧斯側景,願言勇退,深惜馬援之能,未遂祁奚之請。然章疏頻上,雅懷難奪。今知脫屣歸閑,拂衣高謝,固可以激勵頹俗,儀刑庶僚。永言終始,良可嘉尚。宜善攝養,以介期頤。」



 未幾,吐蕃寇邊,復召拜左散騎常侍,令與吐蕃分定地界,兼處置十姓降戶。琬言吐蕃必潛懷叛計,請預支兵十萬於秦、渭等州嚴加防遏。其年冬,吐蕃果入寇,竟為支兵所擊走之。俄又表請致仕,
 不許,遷太子賓客。開元五年,出為同州刺史。明年卒,年八十餘。



 畢構,河南偃師人也。父憬,則天時為司衛少卿。構少舉進士。神龍初,累遷中書舍人。時敬暉等奏請降削武氏諸王,構次當讀表,既聲韻朗暢,兼分析其文句,左右聽者皆歷然可曉。由是武三思惡之,出為潤州刺史。累除益州大都督府長史。景雲初,召拜左御史大夫,轉陜州刺史,加銀青光祿大夫,封魏縣男。頃之,復授益州大都
 督府長史,兼充劍南道按察使。所歷州府,咸著聲績,在蜀中尤革舊弊,政號清嚴。睿宗聞而善之,璽書勞曰:



 我國家創開天地,再造黎元,四夷來王,萬邦會至,置州立郡,分職設官。貞觀、永徽之前,皇猷惟穆;咸亨、垂拱之後,淳風漸替。征賦將急,調役頗繁,選吏舉人,涉於浮濫。省閣臺寺,罕有公直,茍貪祿秩,以度歲時。中外因循,紀綱弛紊,且無懲革,弊乃滋深。為官既不擇人,非親即賄;為法又不按罪,作孽寧逃?貪殘放手者相仍,清白潔己者
 斯絕。蓋由賞罰不舉,生殺莫行。更以水旱時乖,邊隅未謐,日損一日,徵斂不休,大東小東,杼軸為怨,就更割剝,何以克堪!



 昔聞當官,以留犢還珠為上。今之從職,以充車聯駟為能。或交結富家,抑棄貧弱;或矜假典正,樹立腹心。邑屋之間,囊篋俱委,或地有椿乾梓漆,或家有畜產資財,即被暗通,並從取奪。若有固吝,即因事以繩,粗杖大枷,動傾性命,懷冤抱痛,無所告陳。比差御史委令巡察,或有貴要所囑,未能不避權豪;或有親故在官,又
 罕絕於顏面。載馳原隰,徒煩出使之名;安問狐貍,未見埋車之節。揚清激濁,涇、渭不分;嫉惡好善,蕭、蘭莫別。官守既其若此,下人豈以聊生。數年已來,凋殘更甚。



 卿孤潔獨行,有古人之風,自臨蜀川,弊化頓易。覽卿前後執奏,何異破柱求奸?諸使之中,在卿為最。並能盡節似卿如此,百郡何憂乎不理,萬人何慮乎不安?卿當益堅,勿為後顧。朕嘉卿直道,今賜袍帶並衣一副。



 尋拜戶部尚書,轉吏部尚書,並遙領益州大都督府長史。玄宗即位,
 累拜河南尹,遷戶部尚書。開元四年,遇疾,上手疏醫方以賜之。時議戶部尚書為兇官,遽改授太子詹事,冀其有瘳。尋卒,贈黃門監,謚曰景。



 構初喪繼母時,有二妹在襁褓,親加鞠養,咸得成立。及構卒,二妹號絕久之,以撫育恩,遂制三年之服。其弟栩亦甚哀毀,並為當時所稱。栩官至荊州司馬。



 蘇珦,雍州藍田人。明經舉,累授鄠縣尉。雍州長史李義琰召而謂曰:「鄠縣本多訴訟,近日遂絕,訪問果由明公
 為其疏理。」因顧指事曰:「此座即明公座也,但恨非遲暮所見耳。」



 垂拱初,拜右臺監察御史。時則天將誅韓、魯等諸王,使珦按其密狀,珦訊問皆無徵驗。或誣告珦與韓、魯等同情,則天召見詰問,珦抗議不回。則天不悅,曰:「卿大雅之士,朕當別有驅使,此獄不假卿也。」遂令珦於河西監軍。五遷右司郎中。時御史王弘義托附來俊臣,構陷無罪,朝廷疾之。嘗受詔於虢州採木,役使不節,丁夫多死,珦按奏其事,弘義竟以坐黜。珦尋遷給事中,累
 授左肅政臺御史大夫。時有詔白司馬阪營大像,糜費巨億,珦以妨農,上疏切諫,則天納焉。



 神龍初,武三思擅權,韋月將告三思將有逆謀,反為三思所構,中宗令斬之。珦奏非時不可行刑,由是忤三思旨,轉為右御史大夫。尋出為岐州刺史,復為右臺大夫。會節愍太子敗,詔珦窮其黨與。時睿宗在籓,為得罪者所引,珦因辯析事狀,密奏以保持之。中宗意解,因是多所原免,擢珦為戶部尚書,賜爵河內郡公。尋授太子賓客、檢校詹事,以年
 老致仕。開元三年卒,年八十一,贈兗州都督,謚曰文。子晉,亦知名。



 晉,數歲能屬文,作《八卦論》,吏部侍郎房穎敘、秘書少監王紹宗見而賞嘆曰:「此後來王粲也。」弱冠舉進士,又應大禮舉,皆居上第。先天中,累遷中書舍人,兼崇文館學士。玄宗監國,每有制命,皆令晉及賈曾為之。晉亦數進讜言,深見嘉納。俄出為泗州刺史,以父老乞辭職歸侍,許之。父卒後,歷戶部侍郎,襲爵河內郡公。開元十四年,遷吏部侍郎。時開府宋璟兼尚書事,晉及齊
 澣遞於京都知選事,既糊名考判,晉獨多賞拔,甚得當時之譽。俄而侍中裴光庭知尚書事,每遇官應批退者,但對眾披簿,以硃筆點頭而已。晉遂榜選院云:「門下點頭者,更引注擬。」光庭以為侮己,甚不悅,遂出為汝州刺史。三遷魏州刺史,加銀青光祿大夫,入為太子左庶子。二十二年卒,年五十九。



 初,晉與洛陽人張循之、仲之兄弟友善,循之等並以學業著名。循之,則天時上書忤旨被誅。仲之,神龍中謀殺武三思,為友人宋之愻所發,下
 獄死。晉厚撫仲之子漸,有如己子,教之書記,為營婚宦。及晉卒,漸制猶子之服,時人甚以此稱之。



 鄭惟忠,宋州宋城人也。儀鳳中,進士舉,授井陘尉,轉湯陰尉。天授中,應舉召見,則天臨軒問諸舉人:「何者為忠?」諸人對不稱旨。惟忠對曰:「臣聞忠者,外揚君之美,內匡君之惡。」則天曰:「善。」授左司禦率府胄曹參軍,累遷水部員外郎。則天幸長安,惟忠待制引見,則天謂曰:「朕識卿,前於東都言『忠臣外揚君之美,內匡君之惡』,至今不忘。」
 尋加朝散大夫,再遷鳳閣舍人。



 中宗即位,甚敬重之,擢拜黃門侍郎。時議請禁嶺南首領家畜兵器,惟忠曰:「夫為政不可革以習俗,且《吳都賦》云:『家有鶴膝,戶有犀渠。』如或禁之,豈無驚擾耶?」遂寢。無何,守大理卿。節愍太子與將軍李多祚等舉兵誅武三思,事變伏誅。其詿誤守門者並配流,將行,有韋氏黨與密奏請盡誅之。中宗令推斷,惟忠奏曰:「今大獄始決,人心未寧,若更改推,必遞相驚恐,則反側之子,無由自安。」敕令百司議,遂依舊斷,
 所全者甚多。俄拜御史大夫,持節賑給河北道,仍黜陟牧宰。還,敷奏稱旨,加銀青光祿大夫,封滎陽縣男。開元初,為禮部尚書,轉太子賓客。十年卒,贈太子少保。



 王志愔,博州聊城人也。少以進士擢第。神龍年,累除左臺御史,加朝散大夫。執法剛正,百僚畏憚,時人呼為「皁雕」,言其顧瞻人吏,如雕鶚之視燕雀也。尋遷大理正,嘗奏言:「法令者,人之堤防,堤防不立,則人無所禁。竊見大理官僚,多不奉法,以縱罪為寬恕,以守文為苛刻。臣濫
 執刑典,實恐為眾所謗。」遂表上所著《應正論》以見志,其詞曰:



 嘗讀《易》至「萃,利見大人,亨,聚以正也。六二,引吉無咎。」注曰:「居萃之時,體柔當位。處《坤》之中,己獨處正。異操而聚,獨正者危,未能變體,以遠於害。故必見引,然後乃吉而無咎。」王肅曰:「六二與九五相應,俱履貞正。引由迎也,為吉所迎,何咎之有?」未嘗不輟書而嘆曰:「居中履正,事之常體,見引無咎,道亦宜然。



 有客聞而惑之,因謂僕曰:今主上文明,域中理定,君累司典憲,不務和同。處正
 之志雖存,見引之吉誰應?行之不已,余竊懼焉。



 僕斂襟降階揖而謝曰:補遺闕於袞職,用忠讜為己任,以蒙養正,見引獲吉,應此道也,仁何遠哉!昔咎繇謨虞,登朝作士,設教理物,開訓成務。是以五流有宅,五宅三居,怙終賊刑,刑故無小。於是舜美其事曰:「汝明於五刑,以弼五教,期於予理,刑期於無刑,人協於中,時乃功,懋哉!」故孔子嘆其政曰:「舜舉咎繇,不仁者遠。」此非明闢執法,大人見引之應乎?季孫行父之事君也,舉竊寶
 之愆,黜授邑之賞,明善惡而糾慝,議僭賞以塞違。在虞舜之功,居二十之一,主司得行其道,時君不以為嫌,此非己獨處正,應正而無咎。觀魚於棠,臧伯正色;賂鼎在廟,哀伯抗詞。言者得盡其忠,聞之不加其罪。故《春秋》稱臧氏之正,曰:「積善之家,必有餘慶。」此非異操而聚,引吉之所致乎?魏絳理直,晉侯乃復其位;邾人辭順,趙盾不伐其國。此非正體未變,為吉所迎者乎?



 夫在上垂拱,臣下守制,若正應乎上,乃引吉於下。而中士聞道,若存若亡,交戰于譎
 正之門,懷疑乎語默之境,懼獨正之莫引,忘此正之必亨。籲嗟乎!行己立身,居正踐義,其動也直,其正也方。維正直而是與,何往而非攸利。何以明之?《坤》六二:「直方大,不習無不利。」《文言》曰:「直其正也,方其義也,君子敬以直內,義以方外。敬義立而德不孤,直方大則不疑其所行也。」嵇康撰《釋私論》,曹羲著《至公篇》,皆以崇公激俗,抑私事主,一言可以蔽之,歸於體正而已矣。《禮記》曰:「刑者侀也,侀者成也,一成而不可變,故君子盡心焉。」若以喜怒
 制刑,輕重設比,是則橋前驚馬,用希旨論人,苑中獵兔,以從欲廢法。理有違而合道,物貴和而不同,不同之和,正在其中矣。



 昔任延為武威太守,漢帝誡之曰:「善事上官,無失名譽。」延對曰:「臣聞忠臣不私,私臣不忠,上下雷同,非國家之福。善事上官,臣不敢奉詔。」任延雅奏,漢主是其言。此則歸正不回,乖旨順義,不以忤懷見忌,斯亦違而合道。《晏子春秋》:景公見梁丘據曰:「據與我和。」晏子曰:「此同也。和者,君甘則臣酸,君淡則臣咸。今據也,君甘
 亦甘,所謂同也,安得為和?」是以濟鹽梅以調羹,乃適平心之味;獻可否而論道,方恢政體之節。俟引正而遵度,故曰物貴和而不同。劉曼山辯和同之義,有旨哉!若以不同見譏,未敢聞誨。



 客曰:和同乖訓,則已聞之。援法成而不變者,豈恤獄之寬憲耶?《書》曰:「御眾以寬。」《傳》曰:「寬則得眾。」若以嚴統物,異乎寬政矣。



 對曰:刑賞二柄,唯人主操之,崇厚任寬,是謂帝王之德。慎子曰:「以力役法者,百姓也;以死守法者,有司也;以道變法者,君上也。」然則匪
 人臣所操。後魏游肇之為廷尉也,魏帝嘗私敕肇有所降恕,肇執而不從曰:「陛下自能恕之,豈足令臣曲筆也?」是知寬恕是君道,曲從非臣節。人或未達斯旨,不料其務,以平刑為峻,將曲法為寬,謹守憲章,號為深密。《內律》:「釋種虧戒,一誅五百人,如來不救其罪。」豈謂佛法為殘刻耶?老子《道德經》云:「天網恢恢,疏而不漏。」豈謂道教為凝峻耶?《家語》曰:「王者之誅有五,而寢盜不預焉。」即心辯言偽之流。《禮記》亦陳四殺,破律亂名之謂。豈是儒家執
 禁,孔子之深文哉?此三教之用法者,所以明真諦,重玄猷,存天綱,立人極也。



 然則乾象震曜,天道明威。齊眾惟刑,百王所以垂範;析人以法,三后於是成功。所務掌憲決平,斯廷尉之職耳。《易》曰:「家人嗃嗃,無咎;婦子嘻嘻,終吝。」嚴於其家,可移於國。昔崔實達於理而作《政論》,仲長統曰:「凡為人主,宜寫《政論》一通,置諸坐側。」其大抵雲為國者以嚴致平,非以寬致平者也。然則稱嚴者不必逾條越制,凝網重罰,在於施隱括以矯枉,用平典以禁非。
 刑故有常,罰輕無舍,人不易犯,防之難越故也。但人慢吏濁,偽積贓深,而曰以寬理之,可以無過。何異乎命王良御駻,舍銜策於奔踶;請俞跗攻疾,停藥石於膚腠!適見秋駕轉逸,膏肓更深,醫人僕夫,何功之有?



 又謂僕曰:成法而變,唯帝王之命歟?對曰:何為其然也?昔漢武帝甥昭平君殺人,以公主子,廷尉上請論。左右為言,武帝垂涕嘆曰:「法令者,先帝之所造也,用親故誣先帝子法,吾何面目入高廟乎?又下負萬人!」乃可其奏。近代隋文
 帝子秦王俊為並州總管,以奢縱免官。僕射楊素奏言:「王,陛下愛子,請舍其過。」文帝曰:「法不可違。若如公意,我是五兒之父,非兆人之父,何不別制天子兒律乎?我安能虧法!」卒不許。此是帝王操法,協於禮經不變之義。況於秋官典職,司寇肅事,而可變動者乎!我皇睿哲登圖,高視巖廊之上;宰衡明允就列,輯穆廟堂之下。乾坤交泰,日月光華,庶績其凝,眾工咸理。聚以正也,僕幸利見大人;引其吉焉,期養正於下位。中正是托,予何懼乎?



 夫
 君子百行之基,出處二途而已。出則策名委質,行直道以事人,進善納忠,仰太階而緝政。諤諤其節,思為社稷之臣;謇謇匪躬,願參柱石之任。處則高謝公卿,孝友揚名,是亦為政。煙霞尚志,其用永貞,行藏事業,心跡斯在。至如水中泛泛,天下悠悠,執馭為榮,掃門自媚,拜塵邀勢,括囊守祿,從來長息,以為深恥。客乃逡巡不對,遂無以間僕也。



 中宗覽而嘉之。稍遷駕部郎中。



 景雲元年,累轉左御史中丞,尋遷大理少卿。二年,制依漢置刺史監
 郡,於天下沖要大州置都督二十人,妙選有威重者為之,遂拜志愔齊州都督,事竟不行。又授齊州刺史,充河南道按察使。未幾,遷汴州刺史,仍舊充河南道按察使。太極元年,又令以本官兼御史中丞、內供奉,特賜實封一百戶。尋加銀青光祿大夫,拜戶部侍郎。出為魏州刺史,轉揚州大都督府長史,俱充本道按察使。所在令行禁止,奸猾屏跡,境內肅然。久之,召拜刑部尚書。



 開元九年,上幸東都,令充京師留守。十年,有京兆人權梁山偽
 稱襄王男,自號光帝,與其黨及左右屯營押官謀反。夜半時擁左屯營兵百餘人自景風、長樂等門斬關入宮城,將殺志愔,志愔逾墻避賊。俄而屯營兵潰散,翻殺梁山等五人,傳首東都,志愔遂以駭卒。



 盧從願,相州臨漳人,後魏度支尚書昶六代孫也。自範陽徙家焉,世為山東著姓。冠明經舉,授絳州夏縣尉,又應制舉,拜右拾遺。俄遷右肅政監察御史,充山南道黜陟巡撫使,奉使稱旨,拜殿中侍御史。累遷中書舍
 人。



 睿宗踐祚,拜吏部侍郎。中宗之後,選司頗失綱紀,從願精心條理,大稱平允。其有冒名偽選及虛增功狀之類,皆能擿發其事。典選六年,前後無及之者。上嘉之,特與一子太子通事舍人。從願上疏乞回恩贈父,乃贈其父吉陽丞敬一為鄭州長史。初,高宗時裴行儉、馬載為吏部,最為稱職。及是,從願與李朝隱同時典選,亦有美譽。時人稱曰:吏部前有馬、裴,後有盧、李。



 開元四年,上盡召新授縣令,一時於殿庭策試,考入下第者,一切放歸學
 問。從願以注擬非才,左遷豫州刺史。為政嚴簡,按察使奏課為天下第一等,璽書勞問,賜絹百匹。無幾,入為工部侍郎,轉尚書左丞。又與楊滔及吏部侍郎裴漼、禮部侍郎王丘、中書舍人劉令植刪定《開元後格》,遷中書侍郎。十一年,拜工部尚書,加銀青光祿大夫,仍令東都留守。十三年,從升泰山,又加金紫光祿大夫,代韋抗為刑部尚書。頻年充校京外官考使,前後咸稱允當。



 御史中丞宇文融承恩用事,以括獲田戶之功,本司校考為上
 下,從願抑不與之。融頗以為恨,密奏從願廣占良田,至有百餘頃。其後,上嘗擇堪為宰相者,或薦從願,上曰:「從願廣占田園,是不廉也。」遂止不用。從願又因早朝,途中為人所射,中其從者,捕賊竟不獲。時議從願久在選司,為被抑者所讎。



 十六年,東都留守。時坐子起居郎論糶米入官有剩利,為憲司所糾,出為絳州刺史,再遷太子賓客。二十年,河北穀貴,敕從願為宣撫處置使,開倉以救饑餒。使回,以年老抗表乞骸骨,乃拜吏部尚書,聽致
 仕,給全祿。二十五年卒,年七十餘,贈益州大都督,謚曰文。



 李朝隱,京兆三原人也。少以明法舉,拜臨汾尉,累授大理丞。神龍年,功臣敬暉、桓彥範為武三思所構,諷侍御史鄭愔奏請誅之,敕大理結其罪。朝隱以暉等所犯,不經推窮,未可即正刑名。時裴談為大理卿,異筆斷斬,仍籍沒其家,朝隱由是忤旨。中宗令貶嶺南惡處,侍中韋巨源、中書令李嶠奏曰:「朝隱素稱清正,斷獄亦甚當事,
 一朝遠徙嶺表,恐天下疑其罪。」中宗意解,出為聞喜令。



 尋遷侍御史,三遷長安令,有宦官閭興貴詣縣請托,朝隱命拽出之。睿宗聞而嘉嘆,廷召朝隱,勞曰:「卿為京縣令能如此,朕復何憂。」乃下制曰:「夫不吐剛而謅上、不茹柔而黷下者,君子之事也。踐霤必繩、登車無屈者,正人之務也。長安縣令李朝隱,德義不回,清強自遂,亟聞嘉政,累著能名。近者品官入縣,有乖儀式,遂能責之以禮,繩之以愆。但閹豎之流,多有憑恃,柔寬之代,必弄威權。
 歷觀載籍,常所嘆息。朕規誡前古,勤求典憲,能副朕意,實賴斯人。昔虞延持皇后之客,梅陶鞭太子之傅,古稱遺直,復見於今。思欲旌其美行,遷以重職,為時屬閱戶,政在養人,宜加一階,用表剛烈。可太中大夫。特賜中上考,兼絹百匹。」七遷絳州刺史,兼知吏部選事。



 開元二年,遷吏部侍郎,銓敘平允,甚為當時所稱,降璽書褒美,授一子太子通事舍人。四年春,以授縣令非其人,出為滑州刺史,轉同州刺史。駕幸東都,路由同州,朝隱蒙旨召
 見賞慰,賜衣一副、絹百匹。尋遷河南尹,政甚清嚴,豪右屏跡。時太子舅趙常奴恃勢侵害平人,朝隱曰:「此而不繩,何以為政?」執而杖之。上聞,又降敕書慰勉之。



 十年,遷大理卿。時武強令裴景仙犯乞取贓積五千匹,事發逃走。上大怒,令集眾殺之。朝隱執奏曰:「裴景仙緣是乞贓,犯不至死。又景仙曾祖故司空寂,往屬締構,首預元勛。載初年中,家陷非罪,凡有兄弟皆被誅夷,唯景仙獨存,今見承嫡。據贓未當死坐,準犯猶入請條。十代宥賢,功
 實宜錄;一門絕祀,情或可哀。願寬暴市之刑,俾就投荒之役,則舊勛斯允。」手詔不許。朝隱又奏曰:



 有斷自天,處之極法。生殺之柄,人主合專;輕生有條,臣下當守。枉法者,枉理而取,十五匹便抵死刑;乞取者,因乞為贓,數千匹止當流坐。今若乞取得罪,便處斬刑,後有枉法當科,欲加何闢?所以為國惜法,期守律文,非敢以法隨人,曲矜仙命。射兔魏苑,驚馬漢橋,初震皇赫,竟從廷議,豈威不能制,而法貴有常。又景仙曾祖寂,草昧忠節,定為
 元勛,位至臺司,恩倍常數。載初之際,被枉破家,諸子各犯非辜,唯仙今見承嫡。若寂勛都棄,仙罪特加,則叔向之賢何足稱者,若敖之鬼不其餧而?舍罪念功,乞垂天聽。應敕決杖及有犯配流,近發德音,普標殊澤,杖者既聽減數,流者仍許給程。天下顒顒,孰不幸甚!瞻彼四海,已被深恩,豈於一人,獨峻常典?伏乞採臣之議,致仙於法。



 乃下制曰:「罪不在大,本乎情;罰在必行,不在重。朕垂範作訓,庶動植咸若,豈嚴刑逞戮,使手足無措者哉?裴景
 仙幸藉緒餘,超升令宰,輕我憲法,蠹我風猷,不慎畏知之金,詎識無貪之寶,家盈黷貨,身乃逃亡。殊不知天孽可違,自愆難逭,所以不從本法,加以殊刑,冀懲貪暴之流,以塞侵漁之路。然以其祖父昔預經綸,佐命有功,締構斯重,緬懷賞延之義,俾協政寬之典,宜舍其極法,以竄遐荒。仍決杖一百,流嶺南惡處。」



 朝隱俄轉岐州刺史,母憂去官。起為揚州大都督府長史,抗疏固辭,制許之。朝隱性孝友,時年已衰暮,在喪尤加毀瘠。明年,制又起
 為揚州長史,不獲已而就職,復入為大理卿,累封金城伯,代崔隱甫為御史大夫。朝隱素有公直之譽,每御史大夫缺,時議咸許之。及居其職,竟無所糾劾,唯煩於細務,時望由是稍減。俄轉太常卿。二十一年,兼判廣州事,仍攝御史大夫,充嶺南採訪處置使。明年,卒於嶺外,年七十,贈吏部尚書,官給靈輿,兼家口給遞還鄉,謚曰貞。



 裴漼,絳州聞喜人也。世為著姓。父琰之,永徽中,為同州司戶參軍,時年少,美容儀,刺史李崇義初甚輕之。先是,
 州中有積年舊案數百道,崇義促琰之使斷之,琰之命書吏數人,連紙進筆,斯須剖斷並畢,文翰俱美,且盡與奪之理。崇義大驚,謝曰:「公何忍藏鋒以成鄙夫之過!」由是大知名,號為「霹靂手」。後為永年令,有惠政,人吏刊石頌之。歷任倉部郎中,以老疾廢於家。



 漼色養劬勞,十數年不求仕進。父卒後,應大禮舉,拜陳留主簿,累遷監察御史。時吏部侍郎崔湜、鄭愔坐贓為御史李尚隱所劾,漼同鞫其獄。安樂公主及上官昭容阿黨湜等,漼竟
 執正奏其罪,甚為當時所稱。三遷中書舍人。



 太極元年,睿宗為金仙、玉真公主造觀及寺等,時屬春旱,興役不止。漼上疏諫曰:



 臣謹案《禮記》春、秋令曰:無聚大眾,無起大役,不可興土功,恐妨農事。若號令乖度,役使不時,則加疾疫之危,國有水旱之災,此五行之必應也。今自春至夏,時雨愆期,下人憂心,莫知所出。陛下雖降哀矜之旨,兩都仍有寺觀之作,時旱之應,實此之由。且春令告期,東作方始,正是丁壯就功之日,而土木方興,臣恐所妨
 尤多,所益尤少,耕夫蠶妾,饑寒之源。故《春秋》「莊公三十一年冬,不雨」,《五行傳》以為「歲三築臺」;「僖公二十一年夏,大旱」,《五行傳》以「時作南門,勞人興役」。陛下每以萬方為念,睿旨殷勤,安國濟人,防微慮遠。伏願下明制,發德音,順天時,副人望,兩京公私營造及諸和市木石等並請且停,則蒼生幸甚。農桑失時,戶口流散,縱寺觀營構,豈救黎元饑寒之弊哉!



 疏奏不報。尋轉兵部侍郎,以銓敘平允,持授一子為太子通事舍人。



 開元五年,遷吏部侍
 郎,典選數年,多所持拔。再轉黃門侍郎,代韋抗為御史大夫。漼早與張說特相友善,時說在相位,數稱薦之。漼又善於敷奏,上亦嘉重焉。由是擢拜吏部尚書,尋轉太子賓客。漼家世儉約,既久居清要,頗飾妓妾,後庭有綺羅之賞,由是為時論所譏。二十四年卒,年七十餘,贈禮部尚書,謚曰懿。



 漼從祖弟寬。寬父無晦,袁州刺史。寬通略,以文詞進,騎射、彈棋、投壺特妙。景雲中,為潤州參軍,刺史韋銑為按察使,引為判官,清幹善於剖斷,銑重其
 才,以女妻之。後應拔萃,舉河南丞。再轉為長安尉。時宇文融為侍御史,括天下田戶,使奏差為江南東道勾當租庸地稅兼覆田判官。轉太常博士。禮部擬國忌之辰享廟用樂,下太常,寬深達禮節,特建新意,以為廟尊忌卑則登歌,廟卑忌尊則去龠。中書令張說謂寬明識,舉而行之。再遷為刑部員外郎。有萬騎將軍馬崇正晝殺人,時開府、霍國公王毛仲恩幸用事,將鬻其獄,寬執之不回。兵部尚書蕭嵩為河西節度使,奏寬及郭虛己為
 判官,累年專見委任,嵩加中書令,寬歷中書舍人、御史中丞、兵部侍郎。開元二十一年冬,裴耀卿以黃門侍郎知政事,扈從出關,知江、淮轉運,於河陰置倉,奏寬為戶部侍郎,為其副。



 寬性友愛,弟兄多宦達,子侄亦有名稱,於東京立第同居,八院相對,甥侄皆有休憩所,擊鼓而食,當世榮之。選吏部侍郎,及玄宗還京,又改蒲州刺史。州境久旱,入境,雨乃大浹。遷河南尹,不附權貴,務於恤隱,政乃大理。改左金吾衛大將軍,一年,除太原尹,賜紫
 金魚袋。玄宗賦詩而餞之,曰:「德比岱雲布,心如晉水清。」



 天寶初,除陳留太守,兼採訪使。尋而範陽節度李適之入為御史大夫,除寬範陽節度兼採訪使河北替之。其年,又加御史大夫,時北平軍使烏承恩恃以蕃酋與中貴通,恣求貨賄,寬以法按之。檀州刺史何僧獻生口數十人,寬悉命歸之,故夷夏感悅。



 三載,以安祿山為範陽節度,寬為戶部尚書、兼御史大夫。玄宗素重寬,日加恩顧。刑部尚書裴敦復討海賊回,頗張賊勢,又廣敘功以
 開請托之路,寬嘗幾微奏之。居數日,有河北將士入奏,盛言寬在範陽能政,塞上思之,玄宗嗟賞久之。李林甫懼其入相,又惡寬與李適之善,乃呼裴敦復,且以寬之語告之。敦復使氣性疏,與寬素不相下,以為林甫推誠於己,因願結之,且訴其冤。先是,寬以親故名囑敦復,求請軍功。至是敦復氣憤發其事,林甫曰:「公宜速奏,無後於人。」尋而敦復扈從幸溫泉宮,寬在京城未發。遇有敦復下軍將程藏曜、郎將曹鑒。鑒,郴州富人;藏曜,嶺南首
 領之子。皆有他事,與人詣臺告訴,寬受其狀,捕鑒等鞫之。敦復判官太常博士王悅聞之,謂寬求其過,連夜詣湯所以告。敦復大懼,促裝待罪,因令子婿以五百金賂於貴妃姊楊三娘。楊氏遽為言之,明日貶寬為睢陽太守。



 寬以清簡為政,故所蒞人皆愛之。當時望為宰輔。及韋堅構禍,寬又以親累貶為安陸別駕員外置。林甫使羅希奭南殺李適之,紆路至安陸過,擬怖死之。寬叩頭祈請,希奭不宿而過。寬又懼死,上表請為僧,詔不許。然
 崇信釋典,常與僧徒往來,焚香禮懺,老而彌篤。累遷東海太守、襄州採訪使、銀青光祿大夫,轉馮翊太守,入拜禮部尚書。十四載卒,年七十五。詔贈太子少傅,賻帛一百五十段、粟一百五十石。兄弟八人,皆明經及第,入臺省、典郡者五人。



 寬歿之後,弟珣為河內郡太守。安祿山反,以執父喪,將投闕庭,恐累其母,乃詣河東節度訴誠而退。後在母憂,又陷史思明,授其偽官委任,使弟朗密奉表疏至上京。代宗時,為左司郎中、兼侍御史、河東道
 租庸判官。



 王丘,光祿卿同皎從兄子也。父同晊,左庶子。丘年十一,童子舉擢第,時類皆以誦經為課,丘獨以屬文見擢,由是知名。弱冠,又應制舉,拜奉禮郎。丘神氣清古,而志行修潔,尤善詞賦,族人左庶子方慶及御史大夫魏元忠皆稱薦之。長安中,自偃師主簿擢第,拜監察御史。



 開元初,累遷考功員外郎。先是,考功舉人,請托大行,取士頗濫,每年至數百人,丘一切核其實材,登科者僅滿百人。
 議者以為自則天已後凡數十年,無如丘者,其後席豫、嚴挺之為其次焉。三遷紫微舍人,以知制誥之勤,加朝散大夫,再轉吏部侍郎。典選累年,甚稱平允。擢用山陰尉孫逖、桃林尉張鏡微、湖城尉張晉明、進士王泠然,皆稱一時之秀。俄換尚書左丞。



 十一年,拜黃門侍郎。其年,山東旱儉,朝議選朝臣為刺史以撫貧民,制曰:「昔咎繇與禹言曰:『在知人,在安人。』此皆念存邦本,光於帝載,乾乾夕惕,無忘一日。而長吏或不稱,蒼生或未寧,深思循
 良,以矯過弊,仍重諸侯之選,故自朝廷始之。」於是以丘為懷州刺史,又以中書侍郎崔沔等數人皆為山東諸州刺史。至任,皆無可稱,唯丘在職清嚴,人吏甚畏慕之。俄又分知吏部選事,入為尚書左丞,丁父憂去職,服闋,拜右散騎常侍,仍知制誥。



 二十一年,侍中裴光庭病卒,中書令蕭嵩與丘有舊,將薦丘知政事,丘知而固辭,且盛推尚書右丞韓休,嵩因而奏之。及休作相,遂薦丘代崔琳為御史大夫。丘既訥於言詞,敷奏多不稱旨。俄轉
 太子賓客,襲父爵宿預男,尋以疾拜禮部尚書,仍聽致仕。



 丘雖歷要職,固守清儉,未嘗受人饋遺,第宅輿馬,稱為敝陋。致仕之後,藥餌殆將不給。上聞而嘉嘆,下制曰:「王丘夙負良材,累升茂秩,比緣疾疹,假以優閑。聞其家道屢空,醫藥靡給,久此從宦,遂無餘資。持操若斯,古人何遠!且優賢之義,方冊所先,周急之宜,沮勸攸在。其俸祿一事已上,並宜全給,式表殊常之澤,用旌貞白之吏。」天寶二年卒,贈荊州大都督。



 史臣曰:有唐之興,綿歷年所,骨鯁清廉之士,懷忠抱義之臣,臺省之間,駕肩接武。但時有夷險,道有污隆,用與不用而已。睿、玄之世,若李傑、畢構、蘇珦、鄭惟忠、王志愔、盧從願、裴漼、王丘並位歷亞臺,名德兼著。如尹思貞、李朝隱折李承嘉、竇懷貞,辱閭興貴、趙常奴,詩人所謂不畏強御者也。解琬總兵朔野,料敵如神,功遂身退,深知止足,茲亦有足多也。



 贊曰:尚書亞臺,京尹方伯。我朝重官,云誰稱職?傑、構、
 珦、忠,能竭其力。愔、願、漼、丘,聿修厥德。貞蔑大僚,隱繩貴戚。琬馳令名,燕、蜀之北。



\end{pinyinscope}