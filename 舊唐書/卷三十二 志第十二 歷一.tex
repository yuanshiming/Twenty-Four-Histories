\article{卷三十二 志第十二 歷一}

\begin{pinyinscope}

 太
 古聖人,體二氣之權輿,賾三才之物象,乃創紀以窮其數,畫卦以通其變。而紀有大衍之法,卦有推策之文,由是歷法生焉。殷人用九疇、五紀之書,《周禮》載馮相、保
 章之職,所以辨三辰之躔次,察九野之吉兇。歷代疇人,迭相傳授,蓋推步之成法,協用之舊章。暨秦氏焚書,遺文殘缺,漢興作者,師法多門。雖同征鐘律之文,共演蓍龜之說,而建元或異,積蔀相懸,旁取證於《春秋》,強乩疑於《系》、《象》,靡不揚眉抵掌,謂甘、石未稱日官;運策播精,言裨、梓不知天道。及至清臺眎祲,黃道考祥,言縮則盈,少中多否,否則矯雲差算,中則自負知時。章、亥不生,憑何質證?



 高齊天保中,六月日當蝕朔,文宣先期問候官蝕
 何時,張孟賓言蝕申,鄭元偉、董峻言蝕辰,宋景業言蝕巳。是日蝕於申酉之間,言皆不中時。景業造《天保歷》則疏密可知矣。昔鄧平、洛下閎造漢《太初歷》,非之者十七家。後劉洪、蔡伯喈、何承天、祖沖之,皆數術之精粹者,至於宣考歷書之際,猶為橫議所排。斯道寂寥,知音蓋寡。所以張胄玄佩印而沸騰,劉孝孫輿棺而慟哭,俾諸後學,益用為疑。以臣折衷,無如舊法。



 高祖受隋禪,傅仁均首陳七事,言戊寅歲時正得上元之首,宜定新歷,以符
 禪代,由是造《戊寅歷》。祖孝孫、李淳風立理駁之,仁均條答甚詳,故法行於貞觀之世。高宗時,太史奏舊歷加時浸差,宜有改定,乃詔李淳風造《麟德歷》。初,隋末劉焯造《皇極歷》,其道不行,淳風約之為法,時稱精密。天后時,瞿曇羅造《光宅歷》。中宗時,南宮說造《景龍歷》,皆舊法之所棄者,復取用之。徒雲革易,寧造深微,尋亦不行。開元中,僧一行精諸家歷法,言《麟德歷》行用既久,晷緯漸差。宰相張說言之,玄宗召見,令造新歷。遂與星官梁令瓚先
 造《黃道游儀圖》,考校七曜行度,準《周易》大衍之數,別成一法,行用垂五十年。肅宗時,韓潁造《至德歷》。代宗時,郭獻之造《五紀歷》。德宗時,徐承嗣造《正元歷》。憲宗時,徐昂造《觀象歷》。其法今存,而元紀蔀章之數,或異前經;而察斂啟閉之期,何殊舊法。至論徵驗,罕及研精。綿代流行,示存經法耳。



 前史取傅仁均、李淳風、南宮說、一行四家歷經,為《歷志》四卷。近代精數者,皆以淳風、一行之法,歷千古而無差,後人更之,要立異耳,無逾其精密也。《景龍
 歷》不經行用,世以為非,今略而不載。但取《戊寅》、《麟德》、《大衍》三歷法,以備此志,示於疇官爾。



 戊寅歷經



 已上闕文日。自入立秋,初日加四千八十分,後日減七十六分,置初日所加之分,計後日減之數以減之。訖,餘以行分法約之,為日數。及加平見日及分,滿行分法,又去之,從日一,為定見日及分。後皆放此。畢於秋分。自入寒露,日減一百二十七分,減若不足,即一日加行分法,反減之,為定見日及分。後皆放此。畢於立冬。自入小雪,畢於大雪,均減八日。初見去日十四度。



 熒惑



 平見:入冬至,初日減一萬六千三百五十四分,後日減五百四十五分,畢於小寒。自入大寒,日加四百二十六分,畢於啟蟄。自入雨水,畢於穀雨,均加二十九日。入立夏,初日加一萬九千三百九十二分,後日減二百一十三分,畢於大暑。自入立秋,依平。自入處暑,日減一百八十四分,畢於立冬。自入小雪,畢於大雪。均減二十五日。初見去日十七度。



 鎮星



 平見:入冬至,初日減四千八百一十四分,後日加
 七十九分,畢於氣盡。自入小寒,畢於大寒。均減九日。入立春,均減八日。入啟蟄,均減七日。入雨水,均減六日。入春分,均減五日。入清明,均減四日。入穀雨,畢芒種,均減三日。入夏至,畢十日內,均減二日。十日外,入小暑,畢五日內,均減一日。五日外,畢於氣盡,依平。自入大暑,日加一百八十一分,畢於立秋。自入處暑,均加九日。自入白露,初日加六千二分,後日減一百三十三分,畢於寒露。自入霜降,日減七十九分,畢於大雪。初見去日十七度。



 太白



 晨平見:入冬至,依平。自入小寒,日加六十六分,畢於大寒。自入立春,畢於立夏,均加三日。自入小滿,初日加一千九百六十四分,後日減六十六分,畢於芒種。自入夏至,依平。自入小暑,減六十分,畢於大暑。自入立秋,畢於立冬,均減三日。自入小雪,初日減一千九百六十四分,後日減六十六分,畢大寒。



 夕平見:入冬至,日減一百分,畢於立春。自入啟蟄,畢於春分,均減九日。自入清明,初日減五千九百八十六分,後日減一百分,畢於小滿。
 自入芒種,依平。自入夏至,日加一百分,畢於立秋。自入處暑,畢於秋分,均加九日。自入寒露,初日加五千九百八十六分,後日減一百分,畢於小雪。自入大雪,依平。初見去日十一度。



 辰星



 晨平見:入冬至,均減四日。自入小寒,畢於大寒,依平。自入立春,畢啟蟄,減三日。其在啟蟄氣內,去日一十八度外、四十度內,晨無木、土、金一星已上者,不見也。自入雨水,畢於立夏,應見不見。其在立夏氣內,去日度如前,晨有木、火、土、金一星已上者,亦見之。自入小滿,畢於寒露,依平。自入
 霜降,畢於立冬,加一日。自入小雪,畢於大雪十二日,依平。若在大雪十三日,即減一日。在十四日,減二日。在十五日,減三日。在十六日,減四日。



 夕平見:入冬至,畢於清明,依平。自入穀雨,畢於芒種,減二日。自入夏至,畢於大暑,依平。自入立秋,畢於霜降,應見不見。其在立秋及霜降二氣之內,夕有星去日如前晨者,亦見。自入立冬,畢於大雪,依平。初見去日十七度。



 行五星法



 各置星定見之前夜半日所在宿度算及分,各以定見去朔日算及一分加之。小分滿法十四分,從行分一。行分滿法六百七十六分,從度一。又以星初見去日度數,晨減夕加之。命度以次,即星初見所在度及分。自此已後,皆棄此小分也。



 求次日術



 各加一日所行度及分。其火、金之行而有小分者,各以日率為母。小分滿其母,去從行分一。行分滿法,去從度
 一。其行有益疾遲者,副置一日行分。各以其分疾益遲損,乃加之。留者因前,退則減之,伏不注度。順行出斗去其分,行入斗先加分。訖,皆以二十六副行分為度分。



 歲星



 初見:順,日行一百七十六分五十秒,日益遲一分。一百一十四日行十九度二百九分。而留,二十八日。乃退,日九十七分。八十四日退十二度五十分。又留,二十六日五百九十六,小分七四分。即以初定見日分而加之,若滿行分法,即去之,從月去之,從一日。乃順,初日行六十分,日益疾一分。一百十四
 日行十九度四百三十七分而伏。



 熒惑



 初見:入冬至,初率二百四十一日行一百六十三度。已後二日損日及度各一。盡一百二十八日,率一百七十七日行九十九度。畢一百六十一日皆同。已後三日損日及度各一。盡一百八十二日,率一百七十日行九十二度。畢一百八十八日皆同。已後三日益日及度各一。盡二百二十七日,率一百八十三日行一百五度。已後二日益日及度各一。盡二百四十九日,率一百九十四日行一百一十六度。已後一日益日及度各一。盡三百一十日,率二百五十五日
 行一百七十七度。畢三百三十七日皆同。已後二日損。盡三百六十五日,復二百四十一日行一百六十三度。



 初見:入小寒已後,三日去日率一,畢於啟蟄。自入雨水,畢於立夏,均去日率二十。自入小滿,初去日率二十。以次三日去十九,日日去十八。以次三日去一日,畢於小暑,即依平,為定日之率。若入處暑,畢於秋分,皆去度率六,各依冬至後日數而損益之,又依所入之氣以減之,名為前疾。日數及度數之率,若初行。入大寒,畢於大暑,皆差行,日益遲一
 分。其餘皆平行。若入白露,畢於秋分,初日行半度,四十日行二十度。即去日率四十,度率二十,別為半度之行,訖,然後求平行之分以續之。平行分者,置定行度率,以分法乘之,以定日率除之,所得即平行一日之分,不盡為小分。求差行者,置日率之數,減一。訖,又半之,加平行一日之分,為初日行分。各盡其日度而遲。初日行三百二十六分,日益遲一分半,六十日行二十五度五分。其前疾去度六者,此遲初日加六十七分、小分三十六。小分滿六十,去之,從行分一,即六十日行三十一度,分同。而留,十二日。前去日分日於二留,奇後從後留。乃退,日一百九十二分,六十日退十七度二十八分。又留,十二日六百二十六分、小分
 三十分。亦如初定見之分,滿去如前。又順,後遲。初日行二百三十八分,日益疾一分半,六十日行二十五度三十五分。此遲在立秋至秋分者,加一日,行六十七、小分三十六。滿去如前,即六十日行三十一度。分同也。而後疾。入冬至,初率二百一十四日行一百三十六度。已後一日損日及度各一。盡三十七日,率一百七十七日行九十九度。已後二日損日及度各一。盡五十七日,率一百六十七日行八十九度。畢七十九日皆同。已後三日益日及度各一。盡一百三十日,率一百八十四日行一百六十度。已後二日益日及度各一。盡一百四十四日,率一百九
 十一日行一百一十三度。已後一日益日及度各一。盡一百九十日,率二百三十七日行一百五十九度。已後一日益日及度各一。盡二百一十日,率二百六十七日行一百八十九度。畢二百五十九日皆同。已後二日損日及度各一。盡三百六十五日,復率二百一十四行一百三十六度。後遲加六度者,此後疾去度率六,為定度。各依冬至後日數而損益之,為後疾日及度之率。若入立夏,於夏至,日行半度,盡六十日,行三十度。若入小暑,於大暑,盡四十日,行二十度。皆去日及度之率,別為半度之行,訖,然後
 求平行之分以續之。各盡其日度而伏。



 鎮星



 初見:順,日行六十分,八十三日行七度二百四十八分。而留,三十八日。乃退,日四十一分,一百日退六度四十四分。又留,三十七日六十一分小分四。亦以初定見日分加之。滿去如前。乃順,日行六十分,八十三日行七度二百四十八分而伏。



 太白



 晨初見:乃退,日一度半,十日退十五度。而留,九日。乃順遲,差行。先遲,日益疾八分,四十日行三十度。若此
 遲入大雪已後,畢於小滿,即依此為定而求行分。自入芒種,十日減一度為定度,畢於夏至。自入小暑,畢於霜降,均減三度。自入立冬,初日減三度,後十日減一度,畢於霜降,小雪,皆為定度。求一日行分者,以行分法乘定度,以四十餘之,為平分,不盡為小分。又以四乘三十九,以減平分,為初日行分。平行,日一度,十五日行十五度。若此平行入小寒後,十日益日及度各一,畢於啟蟄。自入雨水之氣,皆二十一日行二十一度。自入春分後,十日減一,畢於立夏,即十五日。自入處暑,畢於寒露,即無此平行。自入霜降,即四日益一,畢於大雪,後十五日行十五度。疾,百七十日行二百四度。前順遲減度者,計所減之數,以益此度為定度。求一日行度及分者,以百七十日減度數,餘行以分法乘,以百七十餘之,所得為之日平行度分。晨伏東方。



 夕初見:順疾,百七十日行二百。畢於立夏,依此
 順疾。入冬至已後,畢於立夏,依此率為定。自入小滿,六日加一度。自入大暑初,畢於芒種,自入夏至,畢於小暑,均五度。自入大暑,初加五度,後三日減一度,畢於氣盡。自入立秋,畢於大雪,還依本率。從白露畢春分,皆差行。先疾,日益遲一分半。自入清明,畢於處暑,並平行,同晨疾。求差行者,半一百六十九,乃以一分半乘之,以加平行分,為初日行度分也。平行,日一度,十五日行十五度。此平行入冬至後,十日減日及度各一,畢於立春。自入啟蟄,畢於芒種,皆均九日行九度。自入夏至後,五日益一,畢於小暑。自入大暑,畢於氣盡,皆十五日行十五度。自入立秋後,六日一,畢於小雪。自入大雪,畢於氣盡,皆十五日行十五度者也。順遲,差行。先疾,日益遲八分,四十日行三十度。前加度者,此依數減之,求一日行分,如晨遲準減者為加之。又留,九日。乃退,日半度,十日
 退五度,而夕伏西方。



 辰星:晨初見,留,六日。順遲,日行一百六十九分,四日行一度。若初見入大寒,畢於啟蟄之內,即不須此遲行。平行,日一度,十日行十度。此平行若入大寒已後,二日去日及度各一,畢於二十日,日及度俱盡,即無此平行。疾,日行一度六百九十分,十日行十九度六分。前無遲行者,此疾日減二百三分,十日行十七度四分。晨伏東方。



 夕初見:順疾,日行一度六百九分,十日行十九度六分。此疾者,入小暑畢於處暑之內,日減二百三分,十日行十六度四分。平行,日一度,十日行十度。此平行若入大暑已後,於二日去日及度各一。畢於二十日,
 日及度俱盡,即無此平行。遲,日行一百六十九分,四日行一度。若疾減二百三分者,即不須此遲行。又留,六日九分。夕伏西方。



 推交會



 交會法:一千二百七十四萬一千二百五分。



 交分法:六百三十七萬六百二十九分。



 朔差:一百八萬五千四百九十二分。



 望分:六百九十一萬三千三百五十分。



 交限:五萬八十二萬七千八百五十八分。



 望差:五十四萬二千七百四十七一分。



 外限:六百七十六萬七
 百八十二九分。



 中限:一千二百三十五萬一千二十五八分。



 內限:一千二百一十九萬八千四百五十八七分。



 交時法:二萬九千一十八。



 推交分術



 置入上元已來積月,以交會法去之。餘,以朔差乘之。滿交會法,又去之。仁均本術,武德年加交差七百七十五萬五千一百六十四分。餘為所求年天正朔入平交分。求望平交分術,以望分加之,滿去如前,為平分。次月平分術,其朔望,入冬至氣內,依平
 為定。若入小寒已後,日加氣差一千六百五十分,畢於立春。自入啟蟄,畢於清明,均加七萬六千一百分。後日減一千六百五十分,畢於小滿。置初日所加之分,計後日減之數以減之,餘以加平交分。自入芒種,畢於夏至,依平為定。加之,滿交會法,即去。餘為定交分。其朔入災交,若入小寒,畢於雨水,及立夏,畢於小滿,值盈二時已下,皆半氣差而加之。二時已上,皆不加。其朔入時交分,如望差分已下,外限已上,有星伏,木土去見十日外,火去見四十日外,金星伏去見二十二日外。有一星者,不加氣差。其朔望,入小暑已後,日減氣差一千二百分,畢於處暑。自入白露,畢於霜降,均減九
 萬五千八百二十分。自入立冬,初日減六萬三千三百分,後日減二千一百一十分,畢於小雪。置初日所減之分,計後日減之數以減之,餘以減平交分也。自入大雪,亦依平為定。減若不足者,加交會法,乃減之。餘為定交分。其朔入交分,如交限內限已上,交分中限已下,有星伏如前者,不減氣差。



 推道在內外及先後去交術,其定交分不滿交分法者,為在外道。滿去之,餘為在內道。其餘如望差已下,即是去先交分。以時法約之得一,為去先交時數。交限已上,即以減交分法。餘為去後交分,亦以時法約之,為時數。望則月蝕也。其朔在內道者,朔則日蝕。
 或雖在內道去交而遠,在外道去交而近,亦為蝕也。



 推月蝕加時術



 置有蝕之望定小餘。若入歷一日,即減二百八十。入十五日,即加之。若入十四日,即加五百五十。入二十八日,即減之。自入諸日,值盈皆加二百八十,值縮皆減之,為定餘。乃以十二乘之,以時法六千五百三除之,所得為半辰之數。命以子半起算外,即所在辰。初命子半以一算,自後皆以二算為一辰。不盡為時餘。若時餘在辰半之前者,乃倍之。如法
 無所得,為辰初。又以三因之,如法得一,名為強。若得強,若得二強,即名少弱。若倍之,如法得一,為少。凡四分一為少,二為半,三為太。不盡者,又三之,如法得一,名為強。若得二強者,即名為半弱。若時餘在辰半之後者亦倍之。如法無所得,為正在辰半。以三因之,如法得二,名為強,即名半強。若得二強,即名太弱。若倍之,如法得一,為態。不盡者,又三之,如法得一,為強,即名太強;若得者,又二強者,為辰末。亦可前辰名之。月在沖上蝕,日出後入前各一時半外,不注蝕。



 推日蝕加時術



 置有蝕之朔定小餘。若入歷一日,即減三百。入十五日,即加之。若入十四日,即加五百五十。入二十八日,即減之以為定。自後不入四時加減之限。春三月,內道,去交四時已上,入歷,值盈加二百八十,值縮反減之。夏三月,內道,值盈加二百八十,值縮反減之。秋三月,內道,去交十一時已下,值盈加二百八十,值縮不加;十一時已上,值盈加五百五十,值縮不加一百八十。冬三月,內道,去交五時已下,值盈加二百八十,縮不加。皆為定餘。乃以十二
 乘之,以時法除之,所得半辰之數,命以子半起算外,即所在辰。命辰如前法。不盡為時餘,別置為副。若入仲辰半前,即以副減法,餘為差率。若在半後,即退其半辰,還以法加餘,即以副為差率。若入季辰半前,即以法加副,而為差率。若在半後,即其半辰,還以法加餘,乃倍法以加副,而為差率。若入孟辰半前,即三因其法,而以副減之,餘為差率。若半後,即退其半辰,還以法加餘,又以法加副,乃三因其法而以副減之,為差率。又置去交時數,三已下加
 三,六已下加二,九已下加一,九已上依數,十二以上從十二,以乘差率。若在季辰半後,孟辰半前,去交六時以上者,皆從其六,以乘差率。六時已下,自依數,不須加。如十四得一,為時差。子至卯半,午至酉半,以時餘加之;卯至午半,酉至子半,以減時餘。加之若滿時法者,乃去之,加於辰,即進之於前也。減之若不足者,減半辰,加時法,乃減之,即退之於後也。餘為定時餘。乃如月蝕法,子午卯酉為仲,辰戌丑未為季,寅申已亥為孟。日出前後各一時半外,不注日蝕。



 推內道日不蝕術



 夏五月朔,加時在南方三辰,先交十三時外,六月朔,後交十三時外者,不蝕。啟蟄畢清明,先交十三時外,值縮,加時在未巳西者,亦不蝕。入處暑,畢寒露,後交十三時,值盈,加時在己巳東者,亦不蝕。



 推外道日蝕術



 不問交之先後,但去交一時內者,皆蝕也。若先交二時內者,值盈二時外者,亦蝕。若後交二時內,值縮二時外者,亦
 蝕。其夏去交二時在南方三辰者,亦蝕。若去分至十二時內,去交六時內者,亦蝕。若去交春分三日內,後交二時內者,亦蝕。秋分三日內,先交二時內者,亦蝕。諸去交三時內,星伏如前者,亦蝕。



 推月蝕分術



 置去交分。其在冬,先後交皆去不蝕分二時之數。若在於春,先交去半時,後交去二時。夏即依定。若在於秋,先交去二時,後交去半時。若不足去者,蝕既,乃以三萬六
 千一百八十三為法除之,所得為不蝕分。不盡者,半法已上為半強,已下為半弱,而以減十五,餘為蝕之大分。



 推月蝕所起術



 若在外道,初起東北,蝕甚西北。若在內道,初起東南,蝕甚西南。十三分已上,正東起。推皆據正南而言。



 推日蝕分術



 置去交分。若入冬至已後,畢於立春,皆均減十二萬八百,餘為不蝕分。不足減者,反以交分減之,餘為不蝕分。
 亦減望差為定法。其後交值縮者,直以望差為定法,不須減之。自入啟蟄,初日減二十二萬八百分,後日減一千八百一十分,置初日所減之分,計後日減之數以減之,餘以減交分。畢於芒種。自入夏至,日減二千四百分,畢於白露。自入秋分,畢於大雪,皆均減二十二萬八百分。但不足減者,皆如前,反以交分減之,訖,皆為不蝕。若入冬至,畢於小寒,不蝕分依定。若入大寒,畢於立夏,後去交五時外,皆去不蝕分一時。時差值減者,先交減之,後交加之。不足減者,蝕既。
 時差值加者,先交加之,後交減之。不足減者,蝕既。乃為定分,以十五乘之,以定法除之,所得為不蝕分。不盡者,半法已上為半強,已下為半弱,而以減十五,餘為蝕之大分也。



 推日蝕所起術



 若在外道,初起西南,蝕甚東南。若在內道,初起西北,蝕甚東北。十三度已上,正西起。亦據正南而言之。



 求日出入所在術



 以所入氣辰刻及分,與後氣辰刻及分相減。餘乘入氣日算,以十五除之。所得以加減所入氣為定日出人。從
 冬至至夏至,日出減之,日入加之。從夏至至冬至,日出加之,日入減之。入餘為定刻及分。



 武德九年五月二日校歷人前歷博士臣南宮子明



 校歷人前歷博士臣薛弘疑



 校歷人算歷博士臣王孝通



 監校歷大理卿清河縣公崔善為



 夜漏半



 右依武德元年經,加於漏刻日出沒二十四氣下。



 推月蝕加時術



 右加有蝕之望,以百刻乘定小餘,日法而一,以課所近氣不滿夜半者,命日以甲子算上注歷。



 推月蝕虧初復滿先造每箭更籌用刻



 倍月蝕日所入氣夜漏半,二十五而一,為籌刻分,亦注于歷下。



 月蝕分用刻率置月蝕分



 推日月蝕加時定刻術



 置日月蝕加時定餘。在辰半後者,加時法於時餘,以二十五乘之,三萬九千一十八而一刻,命刻算外,即所入辰刻。



 求虧初復滿術



 置蝕分,用刻率副之,以乘所入歷損益率,四千五十七
 而一。值盈反其損益,值縮依其損謚,副為蝕定用刻數,乃六乘之,十而一,以減蝕加時辰刻,為虧初。丈四乘餘之用刻數,十而一,以加蝕加時辰刻,為復滿。



 求所蝕夜初甚末更籌刻術



 因其日日所入辰殘刻及分,依次加辰刻及分,至蝕初辰刻及分,減二刻十二分,從其更用刻及分除之,不滿更,即初蝕更籌。依所求得至甚刻加之,命即甚。依求得甚後刻數加之,命即末更籌刻及分。日出前復滿,日入後初虧,皆不注蝕。



\end{pinyinscope}