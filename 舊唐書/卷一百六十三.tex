\article{卷一百六十三}

\begin{pinyinscope}

 ○衛次公子洙鄭絪子祗德祗德子顥韋處厚崔群路隨父泌



 衛次公,字從周,河東人。器韻和雅,弱冠舉進士。禮部侍郎潘炎目為國器,擢居上第。參選調禮部侍郎盧翰嘉
 其才,補崇文館校書郎,改渭南尉。次公善鼓琴,京兆尹李齊運使其子交歡,意欲次公授之琴。次公拒之,由是終身未嘗操弦。



 嚴震之鎮興元,闢為從事,授監察,轉殿中侍御史。貞元八年,徵為左補闕,尋兼翰林學士。二十一年正月,德宗升遐,時東宮疾恙方甚,倉卒召學士鄭絪等至金鑾殿。中人或云:「內中商量,所立未定。」眾人未對。次公遽言曰:「皇太子雖有疾,地居塚嫡,內外系心。必不得已,當立廣陵王。若有異圖,禍難未已。」絪等隨而唱
 之,眾議方定。



 及順宗在諒闇,外有王叔文輩操權樹黨,無復經制。次公與鄭絪同處內廷,多所匡正。



 轉司勛員外郎。久之,以本官知制誥,賜紫金魚袋,仍為學士,權知中書舍人。尋知禮部貢舉,斥浮華,進貞實,不為時力所搖。真拜中書舍人,仍充史館修撰,遷兵部侍郎、知制誥,復兼翰林學士。與鄭絪善,會鄭絪罷相,次公左授太子賓客,改尚書右丞,兼判戶部事,拜陜、虢等州都防禦觀察處置等使。請蠲錢三百萬,人得蘇息,
 政聞於朝。徵為兵部侍郎。選人李勣、徐有功之孫,名在黜中,次公召而謂之曰:「子之祖先,勛在王府,豈限常格。」並優秩而遣之。改尚書左丞,恩顧頗厚。上方命為相,已命翰林學士王涯草詔。時淮夷宿兵歲久,次公累疏請罷。會有捷書至,相詔方出,憲宗令追之。遂出為淮南節度使、檢校工部尚書,兼揚州大都督府長史、御史大夫。



 元和十三年十月,受代歸朝,道次病卒。贈太子少保,年六十六,謚曰敬。次公自少入仕,歷大寮,節操趨尚,始終如一,為眾推重。



 子洙,登進士第,尚憲宗女臨真公主。累官至給事中、駙馬都尉、工部侍郎。



 鄭絪,字文明。父羨,池州刺史。絪少有奇志,好學,善屬文。大歷中,有儒學高名如張參、蔣乂、楊綰、常袞,皆相知重。絪擢進士第,登宏詞科,授秘書省校書郎、鄠縣尉。張延賞鎮西川,闢為書記,入除補闕、起居郎,兼史職。無幾,擢為翰林,轉司勛員外郎、知制誥。德宗朝,在內職十三年,小心兢謙,上遇之頗厚。



 貞元末,德宗晏駕,順宗初即
 位,遺詔不時宣下。絪與同列衛次公密申正論,中人不敢違。及王伾、王叔文朋黨擅權之際,絪又能守道中立。憲宗監國,遷中書舍人,依前學士。俄拜中書侍郎、平章事,加集賢殿大學士,轉門下侍郎、弘文館大學士。



 憲宗初,勵精求理,絪與杜黃裳同當國柄。黃裳多所關決,首建議誅惠琳、斬劉闢及他制置。絪謙默多無所事,由是貶秩為太子賓客。出為嶺南節度觀察等使、廣州刺史、檢校禮部尚書。以廉政稱。為工部尚書,轉太常卿,又為同州
 刺史、長春宮使,改東都留守。入歷兵部尚書,旋為河中節度使。太和二年,入為御史大夫、檢校左僕射、兼太子少保。



 絪以文學進,恬淡,踐歷華顯,出入中外者逾四十年。所居雖無赫奕之稱,而守道敦篤,耽悅墳典,與當時博聞好古之士,為講論名理之游,時人皆仰其耆德焉。及文宗即位,以年力衰耄,累表陳乞,遂以太子太傅致仕。三年十月卒,年七十八,贈司空,謚曰宣。子祗德。



 祗德子顥,登進士第,始綬弘文館校書。遷右拾遺、內供奉,詔
 授銀青光祿大夫,遷起居郎。尚宣宗女萬壽公主,拜駙馬都尉。歷尚書郎、給事中、禮部侍郎。典貢士二年,振拔滯才,至今稱之。遷刑部、吏部侍郎。大中十三年,檢校禮部尚書、河南尹。



 顥居戚里,有器度。大中時,恩澤無對。及宣宗棄代,追感恩遇,嘗為詩序曰:「去年壽昌節,赴麟德殿上壽,回憩於長興里第。昏然晝寢,夢與十數人納涼於別館。館宇蕭灑,相與聯句。予為數聯,同游甚稱賞。既寤,不全記諸聯,唯省十字云『石門霧露白,玉殿莓苔青』,
 乃書之於楹。私怪語不祥,不敢言於人。不數日,宣宗不豫,廢朝會,及宮車上仙,方悟其事。追惟顧遇,續石門之句為十韻云:『間歲流虹節,歸軒出禁扃。奔波陶畏景,蕭灑夢殊庭。境象非曾到,崇嚴昔未經。日車烏斂翼,風動鶴飄翎。異苑人爭集,涼臺筆不停。石門霧露白,玉殿莓苔青。若匪災先兆,何當思入冥。御鑢虛仗馬,華蓋負雲亭。白日成千古,金滕閟九齡。小臣哀絕筆,湖上泣青萍。』」未幾,顥亦卒。



 韋處厚,字德載,京兆人。父萬,監察御史,為荊南節度參謀。處厚本名淳,避憲宗諱,改名處厚。幼有至性,事繼母以孝聞。居父母憂,廬於墓次。既免喪,游長安。通《五經》,博覽史籍,而文思贍逸。



 元和初,登進士第,應賢良方正,擢居異等,授秘書省校書郎。裴垍以宰相監修國史,奏以本官充直館,改咸陽縣尉,遷右拾遺,並兼史職。修《德宗實錄》五十卷上之,時稱信史。轉左補闕、禮部考功二員外。早為宰相韋貫之所重,時貫之以議兵不合旨出官,
 處厚坐友善,出為開州刺史。入拜戶部郎中,俄以本官知制誥。穆宗以其學有師法,召入翰林,為侍講學士,換諫議大夫,改中書舍人,侍講如故。



 時張平叔以便佞詼諧,他門捷進,自京兆少尹為鴻臚卿、判度支,不數月,宣授戶部侍郎。平叔以征利中穆宗意,欲希大任。以榷鹽舊法,為弊年深,欲官自糶鹽,可富國強兵,勸農積貨,疏利害十八條。詔下其奏,令公卿議。處厚抗論不可,以平叔條奏不周,經慮未盡,以為利者返害,為簡者至煩,乃
 取其條目尤不可者,發十難以詰之。時平叔傾巧有恩,自謂言無不允。及處厚條件駁奏,穆宗稱善,令示平叔。平叔詞屈無以答,其事遂寢。



 處厚以幼主荒怠,不親政務,既居納誨之地,宜有以啟導性靈,乃銓擇經義雅言,以類相從,為二十卷,謂之《六經法言》,獻之。錫以繒帛銀器,仍賜金紫。以《憲宗實錄》未成,詔處厚與路隨兼充史館修撰。實錄未成,許二人分日入內,仍放常參。處厚俄又權兵部侍郎。



 敬宗嗣位,李逢吉用事,素惡李紳,乃構
 成其罪,禍將不測。處厚與紳皆以孤進,同年進士,心頗傷之,乃上疏曰:



 臣竊聞朋黨議論,以李紳貶黜尚輕。臣受恩至深,職備顧問,事關聖德,不合不言。紳先朝獎用,擢在翰林,無過可書,無罪可戮。今群黨得志,讒嫉大興。詢於人情,皆甚嘆駭。《詩》云:「萋兮菲兮,成是貝錦。彼譖人者,亦已太甚。」又曰:「讒言罔極,交亂四國。」自古帝王,未有遠君子近小人而致太平者。古人云:「三年無改於父之道,可謂孝矣。」李紳是前朝任使,縱有罪愆,猶宜洗釁滌
 瑕,念舊忘過,以成無改之美。今逢吉門下故吏,遍滿朝行,侵毀加誣,何詞不有?所貶如此,猶為太輕。蓋曾參有投杼之疑,先師有拾塵之戒。伏望陛下斷自聖慮,不惑奸邪,則天下幸甚!建中之初,山東向化,只緣宰相朋黨,上負朝廷。楊炎為元載復讎,盧杞為劉晏報怨,兵連禍結,天下不平。伏乞聖明,察臣愚懇。



 帝悟其事,紳得減死,貶端州司馬。



 處厚正拜兵部侍郎,謝恩於思政殿。時昭愍狂恣,屢出畋游。每月坐朝不三四日。處厚因謝,從容
 奏曰:「臣有大罪,伏乞面首。」帝曰:「何也?」處厚對曰:「臣前為諫官,不能先朝死諫,縱先聖好畋及色,以至不壽,臣合當誅。然所以不死諫者,亦為陛下此時在春宮,年已十五。今則陛下皇子始一歲矣,臣安得更避死亡之誅?」上深感悟其意,賜錦彩一百匹、銀器四事。



 寶歷元年四月,群臣上尊號,御殿受冊肆赦。李逢吉以李紳之故,所撰赦文但云左降官已經量移者與量移,不言未量移者,蓋欲紳不受恩例。處厚上疏曰:「伏見赦文節目中,左降
 官有不該恩澤者。在宥之體,有所未弘。臣聞物議皆言逢吉恐李紳量移,故有此節。若如此,則應是近年流貶官,因李紳一人皆不得量移。事體至大,豈敢不言?李紳先朝獎任,曾在內廷,自經貶官,未蒙恩宥。古人云:『人君當記人之功,忘人之過。』管仲拘囚,齊桓舉為國相;冶長縲紲,仲尼選為密親。有罪猶宜滌蕩,無辜豈可終累?況鴻名大號,冊禮重儀,天地百靈之所鑒臨,億兆八紘之所瞻戴。恩澤不廣,實非所宜。臣與逢吉素無讎嫌,與李
 紳本非親黨,所論者全大體,所陳者在至公,伏乞聖慈察臣肝膽。倘蒙允許,仍望宣付宰臣,應近年左降官,並編入赦條,令準舊例,得量移近處。」帝覽奏其事,乃追改赦文,紳方沾恩例。處厚為翰林承旨學士,每立視草,愜會聖旨。常奉急命於宣州征鷹鷙及楊、益、兩浙索奇文綾錦,皆抗疏不奉命,且引前時赦書為證,帝皆可其奏。



 寶歷季年,急變中起。文宗底綏內難,詔命將降,未有所定。處厚聞難奔赴,昌言曰:「《春秋》之法,大義滅親,內惡必
 書,以明逆順。正名討罪,於義何嫌?安可依違,有所避諱!」遂奉籓教行焉。是夕,詔命制置及踐祚禮儀,不暇責所司,皆出於處厚之議。及禮行之後,皆葉舊章。以佐命功,旋拜中書侍郎、同中書門下平章事、監修國史,加銀青光祿大夫,進爵靈昌郡公。處厚在相位,務在濟時,不為身計。中外補授,咸得其宜。



 初,貞元中,宰相齊抗奏減冗員,罷諸州別駕,其在京百司,當入別駕者,多處之朝列。元和以來,兩河用兵,偏裨立功者,往往擢在周行。率以
 儲採王官雜補之,皆盛服趨朝,硃紫填擁。久次當進,及受代閑居者,常數十人,趨中書及宰相私第,摩肩候謁,繁於辭語。及處厚秉政,復奏置六雄、十望、十緊、三十四州別駕以處之。而清流不雜,朝政清肅。



 文宗勤於聽政,然浮於決斷,宰相奏事得請,往往中變。處厚常獨論奏曰:「陛下不以臣等不肖,用為宰相,參議大政。凡有奏請,初蒙聽納,尋易聖懷。若出自宸衷,即示臣等不信;若出於橫議,臣等何名鼎司?且裴度元勛宿德,歷輔四朝,孜
 孜竭誠,人望所屬,陛下固宜親重。竇易直良厚,忠事先朝,陛下固當委信。微臣才薄,首蒙陛下擢用,非出他門,言既不從,臣宜先退。」即趨下再拜陳乞。上矍然曰:「何至此耶!卿之志業,朕素自知,登庸作輔,百職斯舉。縱朕有所失,安可遽辭,以彰吾薄德?」處厚謝之而去,出延英門,復令召還。謂曰:「凡卿所欲言,並宜啟論。」處厚因對彰善癉惡,歸之法制,凡數百言。又裴度勛高望重,為人盡心切直,宜久任,可壯國威。帝皆聽納。自是宰臣敷奏,人不
 敢橫議。



 俄而滄州李同捷叛,朝廷加兵。魏博史憲誠,中懷向背,裴度以宿舊自任,待憲誠於不疑。嘗遣親吏請事至中書。處厚謂曰:「晉公以百口於上前保爾使主,處厚則不然,但仰俟所為,自有朝典耳。」憲誠聞之大懼,自此輸竭,竟有功於滄州。又嘗以理財制用為國之本,撰《太和國計》二十卷以獻。李載義累破滄、鎮兩軍,兵士每有俘執,多遣刳剔。處厚以書喻之,載義深然其旨。自此滄、鎮所獲生口,配隸遠地,前後全活數百千人。



 處厚居
 家循易,如不克任。至於廷諍敷啟,及馭轄待胥吏,勁確嶷然不可奪。質狀非魁偉,如甚懦者;而庶僚請事,畏惕相顧,雖與語移晷,不敢私謁。急於用才,酷嗜文學。嘗病前古有以浮議坐廢者,故推擇群材,往往棄瑕錄用,亦為時所譏。雅信釋氏因果,晚年尤甚。聚書逾萬卷,多手自刊校。奉詔修《元和實錄》,未絕筆,其統例取舍,皆處厚創起焉。太和二年十二月,因延英奏對,造膝之際,忽奏「臣病作」,遽退。文宗命中官扶出,歸第一夕而卒,年五十
 六,贈司空。



 處厚當國柄二周歲,啟沃之謀,頗協時譽,咸共惜之。



 崔群,字敦詩,清河武城人,山東著姓。十九登進士第,又制策登科,授秘書省校書郎,累遷右補闕。元和初,召為翰林學士,歷中書舍人。群在內職,常以讜言正論聞於時。憲宗嘉賞,降宣旨云:「自今後學士進狀,並取崔群連署,然後進來。」群以禁密之司,動為故事,自爾學士或惡直醜正,則其下學士無由上言。群堅不奉詔,三疏論奏
 方允。



 元和七年,惠昭太子薨,穆宗時為遂王,憲宗以澧王居長,又多內助,將建儲貳,命群與澧王作讓表。群上言曰:「大凡己合當之,則有陳讓之儀;己不合當,因何遽有讓表?今遂王嫡長,所宜正位青宮。」竟從其奏。時魏博節度使田季安進絹五千匹,充助修開業寺。群以為事實無名,體尤不可,請止其所進。群前後所論多愜旨,無不聽納。遷禮部侍郎,選拔才行,咸為公當。轉戶部侍郎。



 二年七月,拜中書侍郎、同中書門下平章事。十四年,誅
 李師道,上顧謂宰臣曰:「李師古雖自襲祖父,然朝廷待之始終。其妻於師道即嫂叔也,雖云逆族,若量罪輕重,亦宜降等。又李宗奭雖抵嚴憲,其情比之大逆,亦有不同。其妻士族也,今其子女俱在掖廷,於法皆似稍深。卿等留意否?」群對曰:「聖情仁惻,罪止元兇。其妻近屬,倘獲寬宥,實合弘煦之道。」於是師古妻裴氏、女宜娘,詔出於鄧州安置。宗奭妻韋氏及男女,先沒掖廷,並釋放;其奴婢、資貨皆復賜之。又鹽鐵福建院官權長孺坐贓,詔付
 京兆府決殺。長孺母劉氏求哀於宰相,群因入對言之。憲宗愍其母耄年,乃曰:「朕將屈法赦長孺何如?」群曰:「陛下仁惻即赦之,當速令中使宣諭。如待正敕,即無及也。」長孺竟得免死長流。群之啟奏平恕,多此類也。



 時憲宗急於蕩寇,頗獎聚斂之臣。故籓府由是希旨,往往捃拾,目為進奉。處州刺史苗稷進羨餘錢七千貫,群議以為違詔,受之則失信於天下,請卻賜本州,代貧下租稅。時論美之。



 度支使皇甫鎛陰結權幸,以求宰相,群累疏其奸
 邪。嘗因對面論,語及天寶、開元中事,群曰:「安危在出令,存亡系所任。玄宗用姚崇、宋璟、張九齡、韓休、李元紘、杜暹則理;用林甫、楊國忠則亂。人皆以天寶十五年祿山自範陽起兵,是理亂分時,臣以為開元二十年罷賢相張九齡,專任奸臣李林甫,理亂自此已分矣。用人得失,所系非小。」詞意激切,左右為之感動。鎛深恨之。而憲宗終用鎛為宰相。無何,群臣議上尊號,皇甫鎛欲加「孝德」兩字,群曰:「有睿聖,則孝德在其中矣。」竟為鎛所構。憲
 宗不樂,出為湖南觀察都團練使。



 穆宗即位,徵拜吏部侍郎,召見別殿,謂群曰:「我升儲位,知卿為羽翼。」群曰:「先帝之意,元在陛下。頃者授陛下淮西節度使,臣奉命草制,且曰:『能辨南陽之牘,允符東海之貴。』若不知先帝深旨,臣豈敢輕言?」數日,拜御史中丞。浹旬,授檢校兵部尚書,兼徐州刺史、武寧軍節度、徐泗濠觀察等使。



 初,幽、鎮逆命,詔授沂州刺史王智興為武寧軍節度副使,領徐州兵討伐。群以智興早得士心,表請因授智興旄鉞,竟
 寢不報。智興自河北回戈,城內皆是父兄,開關延入,群為智興所逐。朝廷坐其失守,授秘書監,分司東都。未幾,改華州刺史、兼御史大夫。復改宣州刺史、歙池等州都團練觀察等使,徵拜兵部尚書。久之,改檢校吏部尚書、江陵尹、荊南節度觀察使。逾歲,改檢校右僕射,兼太常卿。太和五年,拜檢校左僕射,兼吏部尚書。六年八月卒,年六十一,冊贈司空。



 群有沖識精裁,為時賢相。清議以儉素之節,其終不及厥初。群年未冠舉
 進士,陸贄知舉,訪於梁肅,議其登第有才行者,肅曰:「崔群雖少年,他日必至公輔。」果如其言。



 群弟於,登進士,官至郎署,有令名。



 子充,亦以文學進,歷三署,終東都留守。



 路隨,字南式,其先陽平人。高祖節,高宗朝為越王府東閣祭酒。曾祖惟恕,官至睦州刺史。祖俊之,仕終太子通事舍人。



 父泌,字安期,少好學,通《五經》,尤嗜《詩》、《易》、《左氏春秋》,能諷其章句,皆究深旨。博涉史傳,工五言詩。性端亮寡言,以孝悌聞於宗族。建中末,以長安尉從調,舉李益、
 韋綬等書判同居高第,泌授城門郎。屬德宗違難奉天,泌時在京師,棄妻子潛詣行在所。又從幸梁州,排潰軍而出,再為流矢所中,裂裳濡血。以策說渾瑊,瑊深重之,闢為從事。瑊討懷光,累奏為副元帥判官、檢校戶部郎中、兼御史中丞。河中平,隨瑊與吐蕃會盟於平涼,因劫盟陷蕃。在絕域累年,棲心於釋氏之教,為贊普所重,待以賓禮,卒於戎鹿。



 貞元十九年,吐蕃遺邊將書求和。隨哀泣上疏,願允其請。表三上,德宗命中使諭旨。朝廷懲
 其宿詐,俟更要於後信,訖數歲不報。元和中,蕃使復款塞,隨復五獻封章,請修和好。又上書於宰執哀訴。裴垍、李籓皆協力敷奏,憲宗可之。命祠部郎中徐復報聘,乃特於詔中疏平涼陷蕃者名氏,令歸中國。吐蕃因復等還,遣使來朝。遂以泌及鄭叔矩之喪與銘及遺錄至,朝野傷嘆。憲宗憫之,贈絳州刺史,賜絹二百匹。至葬日,委所在官給喪事。泌累贈太子少保。



 泌陷蕃之歲,隨方在孩提;後稍長成,知父在蕃,乃日夜啼號,坐必西向,饌不
 食肉,母氏言其形貌肖先君,遂終身不照鏡。後以通經調授潤州參軍,為李錡所困。使知市事,隨翛然坐市中,一不介意。韋夏卿為東都留守,聞而闢之,由是聲名日振。元和五年,邊吏以訃至。隨居喪,益以孝聞。服闋,擢拜左補闕。



 會李絳諷上納諫,憲宗皇帝曰:「諫官路隨、韋處厚章疏相繼,朕常深用其言。」自是識者敬伏焉。俄遷起居郎,轉司勛員外郎。自補闕至司勛員外,皆充史館修撰。穆宗即位,遷司勛郎中,賜緋魚袋。與韋處厚同入翰
 林為侍講學士。採三代皇王興衰,著《六經法言》二十卷奏之。拜諫議大夫,依前侍講學士。將修《憲宗實錄》,復命兼充史職。敬宗登極,拜中書舍人、翰林學士,仍賜紫。有以金帛謝除制者,必叱而卻之曰:「吾以公事接私財耶?」終無所納。文宗即位,韋處厚入相,隨代為承旨,轉兵部侍郎、知制誥。太和二年,處厚薨,隨代為相,拜中書侍郎,加監修國史。初,韓愈撰《順宗實錄》,說禁中事頗切直內官惡之,往往於上前言其不實,累朝有詔改修。及隨進《
 憲宗實錄》後,文宗復令改正永貞時事,隨奏曰:



 臣昨面奉聖旨,以《順宗實錄》頗非詳實,委臣等重加刊正,畢日聞奏。臣自奉宣命,取史本欲加筆削。近見衛尉卿周居巢、諫議大夫王彥威、給事中李固言、史官蘇景胤等各上章疏,具陳刊改非甚便宜。又聞班行如此議論頗眾。臣伏以史冊之作,勸誡所存,事有當書,理宜歸實。匹夫美惡尚不可誣,人君得失無容虛載。聖旨以前件《實錄》記貞元末數事,稍非摭實,蓋出傳聞,審知差舛,便令刊
 正。頃因坐日,屢形聖言,通計前後,至於數四。臣及宗閔、僧孺亦以永貞已來,歲月至近,禁中行事,在外固難詳知。陛下所言,皆是接於耳目。既聞乖謬,因述古今,引前史直不疑盜嫂之言,及第五倫撾公之說,皆多此比類,難盡信書。所冀睿鑒詳於聽言,深宮慎於行事。持此比類,上開聰明,特蒙降察,稍恕前謬。由是近垂宣命,令有改修。



 臣等伏以貞觀已來,累朝實錄有經重撰,不敢固辭。但欲粗刪深誤,亦固盡存諸說。宗閔、僧孺相與商量,
 緣此書成於韓愈,今史官李漢、蔣系皆愈之子婿,若遣參撰,或致私嫌。以臣既職監修,盍令詳正,及經奏請,事遂施行。今者庶僚競言,不知本起,表章交奏,似有他疑。臣雖至昧,容非自請。既迫群議,輒冒上聞。縱臣果獲修成,必懼終為時累。且韓愈所書,亦非己出,元和之後,已是相循。縱其密親,豈害公理?使歸本職,實謂正名。其《實錄》伏望條示舊記最錯誤者,宣付史官,委之修定。則冀聖祖垂休,永無慚於傳信。下臣非據,獲減戾於侵官。彰
 清朝立政之方,表公器不私之義。流言自弭,時論攸宜。



 詔曰:「其《實錄》中所書德宗、順宗朝禁中事,尋訪根柢,蓋起謬傳,諒非信史。宜令史官詳正刊去,其他不要更修。餘依所奏。」



 四年,轉門下侍郎,加崇文館大學士。七年,兼太子太師,備禮冊拜。表上史官所修憲宗穆宗《實錄》。八年,辭疾,不得謝。會李德裕連貶至袁州長史,隨不署奏狀,始為鄭注所忌。九年四月,拜檢校尚書右僕射、同中書門下平章事,兼潤州刺史、鎮海軍節度、浙江西道觀
 察等使。



 太和九年七月,遘疾於路,薨於揚子江之中流,年六十。冊贈太保,謚曰貞。



 隨有學行大度,為諫官能直言,在內廷匡益。自寶歷初為承旨學士,即參大政矣。後十五年在相位。宗閔、德裕朋黨交興,攘臂於其間;李訓、鄭注始終奸詐,接武於其後。而隨藏器韜光,隆污一致,可謂得君子中庸而常居之也。



 史臣曰:衛次公、鄭絪、韋處厚、崔群、路隨等,皆以文學飾身,致位崇極。兼之忠讜,垂名簡書,茲實有足多也。絪有
 其位,有其時,懷獨善之謀,晦眾濟之道,左遷非不幸也。次公因獻捷之書,輟已成之詔,命也夫。處厚危言切議,振士友之急,稱同列之善,君子哉!



 贊曰:衛、鄭、韋、路,兼之博陵。文學政事,為時所稱。



\end{pinyinscope}