\article{卷一百三十}

\begin{pinyinscope}

 ○李揆李涵陳少游盧裴住



 李揆字端卿,隴西成紀人,而家於鄭州,
 代為冠族。秦府學士、給事中玄道玄孫,秘書監、贈吏部
 尚書成裕之子。少聰敏好學,善屬文。開元末,舉進士,補陳留尉,獻書闕
 下,詔中書試文章,擢拜右拾遺。改右補闕、起居郎,知宗子表疏。遷司勛員外郎、考功郎中,並知制誥。扈從劍南,拜中書舍人。



 乾元初,兼禮部侍郎。揆嘗以主司取士,多不考實,徒峻其堤防,索其書策,殊未知藝不至者,文史之囿亦不能摛詞,深昧求賢之意也。其試進士文章,請於庭中設《五經》、諸史及《切韻》本於床,而引貢士謂之曰:「大國選士,但務得者,經籍在此,請恣尋檢。」由是數月之間,美聲上聞,未及畢事,遷中書侍郎、平章事、集賢殿崇
 文館大學士、修國史。



 揆美風儀,善奏對,每有敷陳,皆符獻替。肅宗賞嘆之,嘗謂揆曰:「卿門地、人物、文章,皆當代所推。」故進人稱為三絕。其為舍人也,宗室請加張皇后「翊聖」之號,肅宗召揆問之,對曰;「臣觀往古后妃,終則有謚。生加尊號,未之前聞。景龍失政,韋氏專恣,加號翊聖,今若加皇后之號,與韋氏同。陛下明聖,動遵典禮,豈可蹤景龍故事哉!」肅宗驚曰:「凡才幾誤我家事。」遂止。時代宗自廣平王改封成王,張皇后有子數歲,陰有奪宗之
 議。揆因對見,肅宗從容曰:「成王嫡長有功,今當命嗣,卿意何如?」揆拜賀曰:「陛下言及於此,社稷之福,天下幸甚,臣不勝大慶。」肅宗喜曰:「朕計決矣。」自此頗承恩遇,遂蒙大用。



 時京師多盜賊,有通衢殺人置溝中者,李輔國方恣橫,上請選羽林騎士五百人以備巡檢。揆上疏曰:「昔西漢以南北軍相攝,故周勃因南軍入北軍,遂安劉氏。皇朝置南北衙,文武區分,以相伺察。今以羽林代金吾警夜,忽有非常之變,將何以制之?」遂制罷羽林之請。



 揆在相位,決事獻替,雖甚博辨,性銳於名利,深為物議所非。又其兄皆自有時名,滯於冗官,竟不引進。同列呂諲,地望雖懸,政事在揆之右,罷相,自賓客為荊南節度,聲問甚美。懼其重入,遂密令直省至諲管內拘求諲過失。諲密疏自陳,乃貶揆萊州長史同正員,其制旨曰:「扇湖南之八州,沮江陵之節制。」揆既黜官,數日,其兄皆改授為司門員外郎。後累年,揆量移歙州刺史。初,揆秉政,侍中苗晉卿累薦元載為重官。揆自恃門望,以載地寒,意
 甚輕易,不納,而謂晉卿曰:「龍章鳳姿之士不見用,麞頭鼠目之子乃求官。」載銜恨頗深。及載登相位,因揆當徙職,遂奏為試秘書監,江淮養疾。既無祿俸,家復貧乏,孀孤百口,丐食取給。萍寄諸州,凡十五六年,其牧守稍薄,則又移居,故其遷徙者,蓋十餘州焉。元載以罪誅,除揆睦州刺史,入拜國子祭酒、禮部尚書,為盧杞所惡。德宗在山南,令充入蕃會盟使,加左僕射。行至鳳州,以疾卒,興元元年四月也,年七十四。贈司空,喪事官給。



 李涵,高平王道立曾孫。父少康,宋州刺史。涵簡素恭慎,有名宗室,累授贊善大夫、兼侍御史。朔方節度郭子儀奏為關內鹽池判官。肅宗北幸平涼,未有所適。涵與朔方留後杜鴻漸,草箋具朔方兵馬招集之勢,軍資倉儲庫物之數,咸推涵宗枝之英,純厚忠信,乃令涵奉箋至平涼謁見。涵敷奏明辯,動合事機,肅宗大悅,除右司員外郎,累至司封郎中、宗正少卿。



 寶應元年,初平河朔,代宗以涵忠謹洽聞,遷左庶子、兼御史中丞、河北宣慰使。
 會丁母憂,起復本官而行,每州縣郵驛,公事之外,未嘗啟口,疏飯飲水,席地而息。使還,請罷官終喪制,代宗以其毀瘠,許之。服闋,除給事中,遷尚書左丞。以幽州之亂,充河朔宣慰使。大歷六年正月,為蘇州刺史、兼御史大夫,充浙江西道都團練觀察等使。十一年,來朝,拜御史大夫。京畿觀察使李棲筠歿,代之。德宗即位,以涵和易,無剸割之才,除太子少傅,充山陵副使。涵判官殿中侍御史呂渭上言:「涵父名少康,今官名犯諱,恐乖禮典。」宰
 相崔祐甫奏曰:「若朝廷事有乖舛,群臣悉能如此,實太平之道。」除渭司門員外郎。尋有人言:「涵昔為宗正少卿,此時無言,今為少傅,妄有奏議。」詔曰:「呂渭僭陳章奏,為其本使薄訴官名。朕以宋有司城之嫌,晉有詞曹之諱,嘆其忠於所事,亦謂確以上聞。乃加殊恩,俾膺厚賞。近聞所陳「少」字,往歲已任少卿,昔是今非,罔我何甚!豈得謬當朝典,更廁周行,宜佐遐籓,用誡薄俗。可歙州司馬同正。」由是改涵為檢校工部尚書、兼光祿卿,仍充山陵
 副使。無幾,以右僕射致仕。興元元年九月卒,追贈太子太保。



 陳少游,博州人也。祖儼,安西副都護。父慶,右武衛兵曹參軍,以少游累贈工部尚書。少游幼聰辯,初習《莊》、《列》、《老子》,為崇玄館學生,眾推引講經。時同列有私習經義者,期升坐日相問難。及會,少游攝齊升坐,音韻清辯,觀者屬目。所引文句,悉兼他義,諸生不能對,甚為大學士陳希烈所嘆賞,又以同宗,遇之甚厚。既擢第,補渝州南
 平令,理甚有聲。至德中,河東節度王思禮奏為參謀,累授大理司直、監察殿中侍御史、節度判官。寶應元年,入為金部員外郎。尋授侍御史、迥紇糧料使,改檢校職方員外郎。充使檢校郎官,自少游始也。明年,僕固懷恩奏為河北副元帥判官、兵部郎中、兼侍御史。遷晉州刺史,改同州刺史,未視事,又歷晉、鄭二州刺史。少游為理,長於權變,時推幹濟,然厚斂財貨,交結權幸,以是頻獲遷擢。無幾,澤潞節度使李抱玉表為副使、御史中丞、陳鄭二
 州留後。



 永泰二年,抱玉又奏為隴右行軍司馬,拜檢校左庶子,依前兼中丞。其年,除桂州刺史、桂管觀察使。少游以嶺徼遐遠,欲規求近郡。時中官董秀掌樞密用事,少游乃宿於其里,候其下直,際晚謁之,從容曰:「七郎家中人數幾何?每月所費復幾何?」秀曰:「久忝近職,家累甚重,又屬時物騰貴,一月過千餘貫。」少游曰:「據此之費,俸錢不足支數日,其餘常須數求外人,方可取濟。倘有輸誠供億者,但留心庇覆之,固易為力耳。少游雖不才,請
 以一身獨供七郎之費,每歲請獻錢五萬貫。今見有大半,請即受納,餘到官續送。免貴人勞慮,不亦可乎?」秀既逾於始望,欣愜頗甚,因與之厚相結。少游言訖,泣曰:「南方炎瘴,深愴違辭,但恐不生還再睹顏色矣。」秀遽曰:「中丞美才,不當遠官,請從容旬日,冀竭蹇分。」時少游又已納賄於元載子仲武矣。秀、載內外引薦,數日,拜宣州刺史、宣歙池都團練觀察使。



 大歷五年,改越州刺史、兼御史大夫、浙東觀察使。八年遷揚州大都督府長史、淮南
 節度觀察使。仍加銀青光祿大夫,封潁川縣開國子。所在悉心綏輯,而多以任數為政,好行小惠,胥吏得職,人亦獲安。及朝廷多事。奏請本道兩稅錢千增二百。因詔諸道悉如淮南,鹽每一斗更加一百文。少游十餘年間,三總大籓,皆天下殷厚處也。以故徵求貿易,且無虛日,斂積財寶,累巨億萬,多賂遺權貴,視文雅清流之士,蔑如也。初結元載,每年饋金帛約十萬貫,又多納賂於用事中官駱奉先、劉清潭、吳承倩等,由是美聲達於中禁。
 後見元載在相位年深,以過犯漸見疑忌,少游亦稍疏之。無何,載子伯和貶官揚州,少游外與之交結,而陰使人伺其過失,密以上聞。代宗以為忠,待之益厚。



 上即位,累加檢校禮部、兵部尚書。建中三年,李納反叛,少游以師收徐、海等州,尋棄之,退軍盱眙。又加檢校左僕射,賜實封三百戶。其年,就加同平章事。關播嘗為少游賓僚,盧杞早年與之同在僕固懷恩使府,故驟加其官秩。



 四年十月,駕幸奉天,度支汴東兩稅使包佶在揚州,尚未
 知也。佶判官崔沅遽報少游,佶時所總賦稅錢帛約八百萬貫在焉,少游意以為賊據京師,未即收復,遂脅取其財物。先使判官崔䪻就佶強索其納給文歷,並請供二百萬貫錢物以助軍費,佶答曰:「所用財帛,須承敕命。」未與之。䪻勃然曰:「中丞若得,為劉長卿;不爾,為崔眾矣。」長卿嘗任租庸使,為吳仲孺所困,崔眾供軍吝財,為光弼所殺,故䪻言及之,佶大懼,不敢固護,財帛將轉輸入京師者,悉為少游奪之。佶自謁,少游止焉,長揖而遣,既
 懼禍,奔往白沙。少游又遣判官房孺復召之,佶愈懼,托以巡檢,因急棹過江,妻子伏案牘中。至上元,復為韓滉所拘留。佶先有兵三千,守禦財貨,令高越、元甫將焉,少游盡奪之。隨佶渡江者,又為韓滉所留,佶但領胥吏往江、鄂等州。佶於彈丸中置表,以少游脅取財帛事。會少游使繼至,上問曰:「少游取包佶財帛,有之乎?」對曰:「臣發揚州後,非所知也。」上曰:「少游國之守臣,或防他盜,供費軍旅,收亦何傷。」時方隅阻絕,國命未振,遠近聞之大驚,
 咸以聖情達於變通,明見萬里。少游後聞之,乃安。



 及李希烈陷汴州,聲言欲襲江淮。少游懼,乃使參謀溫述由壽州送款於希烈曰:「濠、壽、舒、廬,尋令罷壘,韜戈卷甲,佇候指揮。」少游又遣巡官趙詵於鄆州結李納。其年,希烈僭號,遣其將楊豐齎偽赦書赴揚州,至壽州,為刺史張建封候騎所得,建封對中使二人及少游判官許子瑞廷責豐而斬之。希烈聞之大怒,即署其大將杜少誠為偽僕射、淮南節度,令先平壽州,後取廣陵。建封於霍丘
 堅柵,嚴加守禁,少誠竟不能進。後包佶入朝,具奏少游奪財賦事狀,少游大懼,乃上表,以所取包佶財貨,皆是供軍急用,今請據數卻納。既而州府殘破,無以上填,乃與腹心孔目官等設法重稅管內百姓以供之。無何,劉洽收汴州,得希烈偽起居注「某月日陳少游上表歸順。」少游聞之,慚惶發疾,數日而卒,年六十一,贈太尉,賻布帛,葬祭如常儀。



 盧鸑,幽州範陽人也,貞觀中工部侍郎義恭玄孫也。父
 子騫,潁王府諮議參軍,以鸑贈秘書少監。鸑少以門廕入仕,在職以幹局稱。累授閬州錄事參軍、監察殿中御史、侍御史、金州刺史。宰相楊炎遇之頗厚,召入左司郎中、京兆少尹,遷大尹。鸑無術學,善事權要,為政苛躁。盧杞甚惡之,諷有司彈奏,坐貶撫州司馬同正,改饒州刺史,遷福州刺史、福建觀察使。貞元二年七月,以疾終。



 裴住,字士明,河南洛陽人。父寬,禮部尚書,有重名於開元、天寶間。住少舉明經,補河南府參軍,通達簡率,不好
 苛細。積官至京兆倉曹,丁父喪,居東都。是時,安祿山盜陷二京,東都收復,遷太子司議郎。無幾,虢王巨奏署侍御史、襄鄧營田判官,丁母憂。東都復為史思明所陷,住藏匿山谷。思明嘗為住父將校,懷舊恩,又素慕住名,欲必得之,因令捕騎數十跡逐得住。思明見之,甚喜,呼為郎君,不名,偽授御史中丞,主擊斷。時思明殘殺宗室,住陰緩之,全活者數百人。又嘗疏賊短長以聞,事洩,思明大怒詬罵,僅而免死。賊平,除太子中允,遷考功郎中,數
 召見言事。



 代宗居陜,住步懷考功及南曹二印赴行在,上見而謂之曰:「疾風知勁草,果信矣。」將以為御史中丞,為無載所排,為河東道租庸鹽鐵等使。時關輔大旱,住入計,代宗召見便殿,問住:「榷酤之利,一歲出入幾何?」住久之不對。上復問之,對曰:「臣有所思。」上曰:「何思?」對曰:「臣自河東來,其間所歷三百里,見農人愁嘆,穀菽未種。誠謂陛下軫念,先問人之疾苦,而乃責臣以利。孟子曰:理國者,仁義而已,何以利為?由是未敢即對也。」上前坐曰:「
 微公言,吾不聞此。」拜左司郎中。上時訪以事,執政者忌之,出為虔州刺史,歷饒、廬、亳三州刺史。入為右金吾將軍。



 建中初,上以刑名理天下,百吏震悚。時十月禁屠殺,以甫近山陵,禁益嚴。尚父、汾陽王郭子儀隸人殺羊以入,門者覺之,住列奏狀,上以為不畏強御,累遣宣諭。或謂住曰:「郭公有社稷功,豈不為蓋之?」住笑曰:「非爾所解。且郭公威權太盛,上新即位,必謂黨附者眾。今發其細過,以明不弄權耳。吾上以盡事君之道,下以安大臣,不
 亦可乎?」時於朝堂別置三司以決庶獄,辯爭者輒擊登聞鼓,住上疏曰:「夫諫豉謗木之設,所以達幽枉,延直言。今輕猾之人,援桴鳴鼓,始動天聽,竟因纖微。若然者,安用吏理乎!」上然之,悉歸有司。住以法吏舞文,多挾宿怨,因獻《獄官箴》以諷。無何,坐所善僧抵法,貶閬州司馬。徵為右庶子,改千牛上將軍。會吐蕃入寇,尋拜吏部侍郎、兼御史大夫,為吐蕃使,不行。無幾,轉太子賓客、兵部侍郎、河南尹、東都副留守。



 住自河南凡五代為官,入視事,
 未嘗當正處,不鞫認於贓罪,以寬厚和易為理。貞元九年十一月,以疾終,年七十五,贈禮部尚書。



 史臣曰:李揆發言沃心,幸遇明主;蔽賢固位,終非令人。少游逐勢利隨時,盧惎事權要巧宦,察言觀行,皆無可稱。涵節行著聞,住和易為理,庶幾近仁也。



 贊曰李、陳、盧鸑,言行非真。涵、住和易,庶乎近仁。



\end{pinyinscope}