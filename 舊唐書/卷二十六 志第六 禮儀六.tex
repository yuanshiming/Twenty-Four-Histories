\article{卷二十六 志第六 禮儀六}

\begin{pinyinscope}

 建中元年三月,
 禮儀使上言:「東都太廟闕木主,請造以祔。」初,武后於東都立高祖、太宗、高宗三
 廟。至中宗已後,兩京太廟,四時並饗。至德亂後,木主多亡缺未祔。於是
 議者紛然,而大旨有三:「其一曰,必存其廟,遍立群主,時饗之。其二曰,建廟立主,存而不祭,若皇輿時巡,則就饗焉。其三曰,存其廟,瘞其主,駕或東幸,則飾齋車奉京師群廟之主以往。議者皆不決而罷。



 貞元十五年四月,膳部郎中歸崇敬上疏:「東都太廟,不合置木主。謹按典禮。虞主用桑,練主用慄,重作慄主,則埋桑主。所以神無二主,猶天無二日,土無二王也。今東都太廟,是則天皇后所建,以置武氏木主。中宗去其主而存其廟,蓋將以備
 行幸遷都之所也。且殷人屢遷,前八後五,前後遷都一十三度,不可每都而別立神主也。議者或云:『東都神主,已曾虔奉而禮之,豈可以一朝廢之乎?』且虞祭則立桑主而虔祀,練祭則立慄主而埋桑主,豈桑主不曾虔祀,而乃埋之?又所闕之主,不可更作,作之不時,非禮也。」



 長慶元年二月,分司官庫部員外郎李渤奏:「太微宮神主,請歸祔太廟。」敕付東都留守鄭絪商量聞奏。『絪奏云:「臣謹詳三代典禮,上稽高祖、太宗之制度,未嘗有並建兩
 朝、並饗二主之禮。天授之際,祀典變革。中宗初復舊物,未暇詳考典章,遂於洛陽創宗廟。是行遷都之制,實非建國之儀。及西歸上都,因循未廢。德宗嗣統,墜典克修,東都九廟,不復告饗。謹按《禮記》,仲尼答曾子問曰:『天無二日,土無二王,嘗、禘、郊、社,尊無二上。』所以明二主之非禮也。陛下接千載之大統,揚累聖之耿光,憲章先王,垂法後嗣。況宗廟之禮,至尊至重,違經黷祀,時謂不欽。特望擇三代令典,守高祖、太宗之憲度,鑒神龍權宜之制,
 遵建中矯正之禮,依經復古,允屬聖明。伏以太微宮光皇帝三代、睿宗聖文孝武皇帝神主,參考經義,不合祔饗。至於遷置神主之禮,三代以降,經無明文。伏望委中書門下與公卿禮官質正詳定。」敕付所司。



 太常博士王彥威等奏議曰:



 謹按國初故事,無兩都並建宗廟、並行饗祭之禮。伏尋《周書》、《召誥》、《洛誥》之說,實有祭告豐廟、洛廟之文,是則周人兩都並建宗祧,至則告饗。然則兩都皆祭祖考,禮祀並興。自神龍復闢,中宗嗣位,廟既偕作,饗
 亦並行。天寶末,兩都傾陷,神主亡失。肅宗既復舊物,但建廟作主於上都。其東都神主,大歷中始於人間得之,遂寓於太微宮,不復祔饗。



 臣等謹按經傳,王者之制,凡建居室,宗廟為先,廟必有主,主必在廟。是則立廟兩都,蓋行古之道,主必在廟,實依禮經。今謹參詳,理合升祔。謹按光皇帝是追王,高宗、中宗、睿宗是祧廟之主,其神主合藏於太廟從西第一夾室。景皇帝是始封不遷之祖,其神主合藏於太廟從西第一室。高祖、太宗、玄宗、肅
 宗、代宗是創業有功親廟之祖。伏準《江都集禮》:『正廟之主,藏於太室之中。』《禮記》:『君廟之主,有故則聚而藏諸祖廟。』伏以德宗之下,神主未作,代宗之上,後主先亡,若歸本室,有虛神主。事雖可據,理或未安。今高祖已下神主,並合藏於太祖之廟,依舊準故事不饗。如陛下肆覲東後,移幸洛陽,自非祧主,合歸本室。其餘闕主,又當特作,而祔饗時祭、禘、祫如儀。臣又按國家追王故事,太祖之上,又有德明、興聖、懿祖別廟。今光皇帝神主,即懿祖也。
 伏緣東都先無前件廟宇,光皇帝神主今請權祔於太廟夾室,居元皇帝之上。如駕在東都,即請準上都式營建別廟,作德明、興聖、獻祖神主,備禮升祔。又於太廟夾室奉迎光皇帝神主歸別廟第四室,禘、祫如儀。



 或問曰:「禮,作慄主,瘞桑主。漢、魏並有瘞桑之議,大歷中亦瘞孝敬皇帝神主,今祔而不瘞,如之何?」答曰:「作主依神,理無可埋,漢魏瘞藏,事非允愜。孝敬尊非正統,廟廢而主獨存,從而瘞藏,為葉情理。」



 又問:「古者巡狩,必載遷主,今東都主又祔於廟。」答曰:「古者師行以遷主,無則主命,自非遷祖之主,別無出廟之文。凡邑有宗廟先君之主曰都,則兩都宗廟,各宜有主。」



 又問曰:「古者作
 主,必因虞、練,若主必歸祔,則室不可虛,則當補已亡之主,創當祔之主。禮經無說,如之何?」答曰:「虞、練作主,禮之正也。非時作主,事之權也。王者遭時為法,因事制宜,茍無其常,則思其變。如駕或東幸,廟仍虛主,即準肅宗廣德二年上都作主故事,特作闕主而祔。蓋主不可闕,故禮貴從宜,《春秋》之義,變而正之者。」臣伏思祖宗之主,神靈所憑,寓於太微,不入宗廟,據經復本,允屬聖明。



 至是下尚書省集議,而郎吏所議,與彥威多同。丞郎則各執
 所見,或曰「神主合藏於太微宮;」或云「並合埋瘞」;或云「闕主當作」;或云「輿駕東幸,即載上都神主而東」。咸以意言,不本經據。竟以紛議不定,遂不舉行。



 會昌五年八月,中書門下奏:「東都太廟九室神主,共二十六座,自祿山叛後,取太廟為軍營,神主棄於街巷,所司潛收聚,見在太微宮內新造小屋之內。其太廟屋室並在,可以修崇。大和中,太常博士議,以為東都不合置神主,車駕東幸,即載主而行。至今因循,尚未修建。望令尚書省集公卿及
 禮官、學官詳議。如不要更置,須有收藏去處。如合置,望以所拆大寺材木修建。既是宗室官居守,便望令充修東都太廟使,勾當修繕。」奉敕宜依。



 六年三月,太常博士鄭路等奏:「東都太微宮神主二十座,去年二月二十九日禮院分析聞奏訖。伏奉今月七日敕,『此禮至重,須遵典故,宜令禮官、學官同議聞奏』者。臣今與學官等詳議訖,謹具分析如後:獻祖宣皇帝、宣莊皇后、懿祖光皇帝、光懿皇后、文德皇后、高宗天皇大帝、則天皇后、中宗大
 聖大昭孝皇帝、和思皇后、昭成皇后、孝敬皇帝、地敬哀皇后已前十二座,親盡迭毀,宜遷諸太廟,祔於興聖廟。禘祫之歲,乃一祭之。東都無興聖廟可祔,伏請且權藏於太廟夾室。未題神主十四座,前件神主既無題號之文,難伸祝告之禮。今與禮官等商量,伏請告遷之日,但瘞於舊太微宮內空閑之地。恭酌事理,庶協從宜。」制可。



 太常博士段瑰等三十九人奏議曰:



 禮之所立,本於誠敬;廟之所設,實在尊嚴。既曰薦誠,則宜統一。昔周之東
 西有廟,亦可徵其所由。但緣卜洛之初,既須營建,又以遷都未決,因議兩留。酌其事情,匪務於廣,祭法明矣。



 伏以東都太廟,廢已多時,若議增修,稍乖前訓。何者?東都始制寢廟於天后、中宗之朝,事出一時,非貞觀、開元之法,前後因循不廢者,亦踵鎬京之文也。《記》曰:「祭不欲數,數則煩。」天寶之中,兩京悉為寇陷,西都廟貌如故,東都因此散亡。是知九廟之靈,不欲歆其煩祀也。自建中不葺之後,彌歷歲年。今若廟貌惟新,即須室別有主。舊主
 雖在,大半合祧,必幾筵而存之,所謂宜祧不祧也。孔子曰,「當七廟五廟,無虛主也」,謂廟不得無主者也。舊主如有留去,新廟便合創添。謹按《左傳》云:「祔練作主。」又戴聖云:「虞而立幾筵。」如或過時成之,便是以兇乾吉。創添既不典,虛廟又非儀。考諸禮文,進退無守。



 或曰「漢於郡國置宗廟凡百餘所,今止東西立廟,有何不安」者。當漢氏承秦焚燒之餘,不識典故,至於廟制,率意而行。比及元、成二帝之間,貢禹、韋玄成等繼出,果有正論,竟從毀除。足
 知漢初不本於禮經,又安可程法也?或曰「幾筵不得復設,廟寢何妨修營,侯車駕時巡,便合於所載之主」者。究其終始,又得以論之。昨者降敕參詳,本為欲收舊主,主既不立,廟何可施?假令行幸九州;一一皆立廟乎」愚以為廟不可修,主宜藏瘞,或就瘞於坎室,或瘞於兩階間,此乃百代常行不易之道也。



 其年九月敕:「段瑰等詳議,東都不可立廟。李福等別狀,又有異同。國家制度,須合典禮,證據未一,則難建立。宜並令赴都省對議,須歸至
 當。」



 工部尚書薛元賞等議:



 伏以建中時,公卿奏請修建東都慶廟,當時之議,大旨有三:其一曰,必存其廟,備立其主,時饗之日,以他官攝行。二曰,建廟立主,存而不祭,皇輿時巡,則就饗焉。三曰,存其廟,一瘞其主。臣等立其三議,參酌禮經,理宜存廟,不合置主。



 謹按《禮祭義》曰:「建國之神位,右社稷而左宗廟。」《禮記》云:「君子將營宮室,宗廟為先。」是知王者建邦設都,必先宗廟、社稷。況周武受命,始都於豐,成王相宅,又卜於洛,烝祭歲於新邑,冊周公
 於太室。故《書》曰:「戊辰,王在新邑,烝祭歲。王入太室祼。」成王厥後復立於豐,雖成洛邑,未嘗久處。逮於平王,始定東遷。則周之豐、鎬,皆有宗廟明矣。又按,曾子問「廟有二主」,夫子對以「天無二日,土無二王,嘗、禘、郊、社,尊無二上,未知其為禮」者。昔齊桓公作二主,夫子譏之,以為偽主。是知二主不可並設,亦明矣。夫聖王建社以厚本,立廟以尊祖,所以京邑必有宗社。今國家定周、秦之兩地,為東西之兩宅,闢九衢而立宮闕,設百司而嚴拱衛,取法
 玄象,號為京師。既嚴帝宅,難虛神位,若無宗廟,何謂皇都?然依人者神,在誠者祀,誠非外至,必由中出,理合親敬,用交神明。位宜存於兩都,廟可偕立;誠難專於二祭,主不並設。



 或以《禮》云「七廟五廟無虛主」,是謂不可無主。所以天子巡狩,亦有所尊,尚飾齋車,載遷主以行。今若修廟瘞主,同東都太廟,九室皆虛,既違於經,須徵其說。臣復探賾禮意,因得盡而論之。所云「七廟五廟無虛主」,是謂見饗之廟不可虛也。今之兩都,雖各有廟,禘祫饗
 獻,斯皆親奉於上京,神主幾筵,不可虛陳於東廟。且《禮》云:「唯聖人為能饗帝,孝子為能饗親。」昔漢韋玄成議廢郡國祀,亦曰:「立廟京師,躬親承事,四海之內,各以其職來祭。」人情禮意,如此較然。二室既不並居,二廟豈可偕祔?但所都之國,見饗之廟,既無虛室,則葉通經議者,又欲置主不饗,以俟巡幸。昔魯作僖公之主,不於虞、練之時,《春秋》書而譏之。合祔之主,作非其時,尚為所譏。今若置不合祔之主,不因時而作,違經越禮,莫甚於此。豈有
 九室合饗之主,而有置而不饗之文?兩廟始創於周公,二主獲譏於夫子。自古制作,皆範周孔,舊典猶在,足可明徵。臣所以言東都廟則合存,主不合置。今將修建廟宇,誠不虧於典禮。其見在太微宮中六主,請待東都建修太廟畢,具禮迎置於西夾室,閟而不饗,式彰陛下嚴祀之敬,以明聖朝尊祖之義。



 吏部郎中鄭亞等五人議:「據禮院奏,以為東都太廟既廢,不可復修,見在太微宮神主,請瘞於所寓之地。有乖經訓,不敢雷同。臣所以別
 進議狀,請修祔主,並依典禮,兼與建中元年禮儀使顏真卿所奏事同。臣與公卿等重議,皆以為廟固合修,主不可瘞,即與臣等別狀意同。但眾議猶疑東西二廟,各設神主,恐涉廟有二主之義,請修廟虛室,以太微宮所寓神主藏於夾室之中。伏以六主神位,內有不祧之宗,今用遷廟之儀,猶未合禮。臣等猶未敢署眾狀,蓋為闕疑。」



 太學博士直弘文館鄭遂等七人議曰:「夫論國之大事,必本乎正而根乎經,以臻於中道。聖朝以廣孝為先,
 以得禮為貴,而臣下敢不以經對。三論六故,已詳於前議矣。再捧天問,而陳乎諸家之說,求於典訓,考乎大中,廟有必修之文,主無可置之理。何則?正經正史,兩都之廟可徵。《禮》稱『天子不卜處太廟』,『擇日卜建國之地,則宗廟可知』。則廢廟之說,恐非所宜廢。謹按《詩》、《書》、《禮》三經及漢朝兩史,兩都並設廟,而載主之制,久已行之。敢不明征而去文飾,援據經文,不易前見,東都太廟,合務修崇,而舊主當瘞,請於太微宮所藏之所。皇帝有事於洛,則
 奉齋車載主以行。」



 太常博士顧德章議曰:



 夫禮雖緣情,將明厥要,實在得中,必過禮而求多,則反虧於誠敬。伏以神龍之際,天命有歸,移武氏廟於長安,即其地而置太廟,以至天寶初復,不為建都。而設議曰:「中宗立廟於東都,無乖舊典。」徵其意,不亦謬乎?



 又曰「東都太廟,至於睿宗、玄宗,猶奉而不易」者。蓋緣嘗所尊奉,不敢輒廢也。今則廢已多時,猶循莫舉之典也。又曰「雖貞觀之始,草創未暇,豈可謂此事非開元之法」者。謹按定《開元六典
 敕》曰:「聽政之暇,錯綜古今,法以《周官》,作為《唐典》。覽其本末,千載一朝。《春秋》謂考古之法也。行之可久,不曰然歟?」此時東都太廟見在,《六典》序兩都宮闕,西都具太廟之位,東都則存而不論,足明事出一時,又安得曰「開元之法」也?又三代禮樂,莫盛於周。昨者論議之時,便宜細大,取法於周,遷而立廟。今立廟不因遷,何美之而不能師之也?又曰「建國神位,右社稷而左宗廟,君子將營宮室,宗廟為先」者。謹按《六典》,永昌中則天以東都為神都。爾
 後漸加營構,營室百司,於是備矣。今之宮室百司,乃武氏改命所備也。上都已建國立宗廟,不合引言。又曰:「東都洛陽祭孝宣等五帝,長安祭孝成等三帝」。以此為置廟之例,則大非也。當漢兩處有廟,所祭之帝各別。今東都建廟作主,與上都盡同,概而論之,失之甚者。又曰「今或東洛復太廟,有司同日侍祭,以此為數,實所未解」者。謹按天寶三載詔曰:「頃四時有事於太廟,兩京同日。自今已後,兩京各宜別擇日。」載在祀典,可得而詳。且立廟造主,
 所以祭神,而曰存而勿祀,出自何經?「當七廟五廟無虛主」,而欲立虛廟,法於何典?前稱廟貌如故者,即指建中之中,就有而言,以為國之先也。前以非時不造主者,謂見有神主,不得以非時而造也。若江左至德之際,主並散亡,不可拘以例也。或曰「廢主之瘞,請在太微宮」者。謹按天寶二年敕曰:「古之制禮,祭用質明,義兼取於尚幽,情實緣於既沒。我聖祖澹然當在,為道之法,既殊有盡之期,宜展事生之禮。自今已後,每至聖祖宮有昭告,宜
 改用卯時」者。今欲以主瘞於宮所,即與此敕全乖。又曰:主不合瘞,請藏夾室」者。謹按前代藏主,頗有異同。至如夾室,宜用以序昭穆也。今廟主俱不中禮,則無禘祫之文。又曰君子將營宮室,以宗廟為先,則建國營宮室而宗廟必設。東都既有宮室,而太廟不合不營。凡以論之,其義斯勝。而西周、東漢,並曰兩都,其各有宗廟之證,經史昭然,又得以極思於揚榷。《詩》曰:「其繩則直,縮板以載,作廟翼翼。」《大雅》「瓜瓞」,言豐廟之作也。又曰:「於穆清廟,肅
 雍顯相。」洛邑既成,以率文王之祀。此《詩》言洛之廟也。《書》曰:「成王既至洛,烝祭歲,文王騂牛一。」又曰「裸於太室」,康王又居豐,「命畢公保厘東郊。」豈有無廟而可烝祭,非都而設保厘?則《書》東西之廟也。逮於後漢卜洛,西京之廟亦存。建武二年,於洛陽立廟,而成、哀、平三帝祭於西京。一十八年,親幸長安,行禘禮,當時五室列於洛都,三帝留於京廟,行幸之歲,與合食之期相會,不奉齋車,又安可以成此禮?則知兩廟周人成法,載主以
 行,漢家通制。或以當虛一都之廟為不可,而引「七廟無虛主」之文。《禮》言一都之廟,室不虛主,非為兩都各廟而不可虛也。既聯出征之辭,更明載主之意,因事而言,理實相統,非如詩人更可斷章以取義也。古人求神之所非一,奉神之意無二,故廢桑主,重作慄主,既事理之,以明其一也。



 或又引《左氏傳》築郿凡例,謂「有宗廟先君之主曰都」,而立建主之論。按魯莊公二十八年冬,築郿,《左傳》為築發凡例,《穀梁》譏因藪澤之利,《公羊》稱避兇年造邑
 之嫌。三傳異同,左氏為短。何則?當春秋二百年間,魯凡城二十四邑,唯郿一邑稱築,其二十三邑,豈皆有宗廟先君之主乎;執此為建主之端,又非通論。或又曰:「廢主之瘞,何以在於太微宮所藏之所;宜舍故依新,前已列矣。」按瘞主之位有三:或於北牖之下,或在西階之間,廟之事也。其不當立之主,但隨其所以瘞之。夫主瘞乎當立之廟,斯不然矣。以在所而言,則太微宮所藏之所,與漢之寢園無異。歷代以降,建一都者多,兩都者少。今國
 家崇東西之宅,極嚴奉之典,而以各廟為疑,合以建都故事,以相質正,即周、漢是也。今詳議所征,究其年代,率皆一都之時,豈可以擬議,亦孰敢獻酬於其間?詳考經旨,古人謀寢必及於廟,未有設寢而不立廟者。國家承隋氏之弊,草創未暇,後雖建於垂拱,而事有所合。其後當干戈寧戢之歲,文物大備之朝,歷於十一聖,不議廢之。豈不以事雖出於一時,廟有合立之理,而不可一一革也。今洛都之制,上自宮殿樓觀,下及百闢之司,與西
 京無異。鑾輿之至也,雖廝役之賤,必歸其所理也。豈先帝之主,獨無其所安乎?時也,虞主尚瘞,廢主宜然。或以馬融、李舟二人稱「寢無傷於偕立,廟不妨於暫虛」,是則馬融、李舟,可法於宣尼矣。以此擬議,乖當則深。



 或稱「凡邑有宗廟先君之主曰都,無曰邑,邑曰築,都曰城」者。謹按春秋二百四十年間,惟郿一邑稱築。如城郎、費之類,各有所因,或以他防,或以自固,謂之盡有宗廟,理則極非。或稱「聖主有復古之功,簡冊有考文之美,五帝不同
 樂,三王不同禮,遭時為法,因事制宜」。此則必作有為,非有司之事也。如有司之職,但合一一據經;變禮從時,則須俟明詔也。



 凡不修之證,略有七條:廟立因遷,一也;已廢不舉,二也;廟不可虛,三也;非時不造主,四也;合載遷主行,五也;尊無二上,六也;《六典》不書,七也。謹按文王遷豐立廟,武王遷鎬立廟,成王遷洛立廟,今東都不因遷而欲立廟,是違因遷立廟也。謹按《禮記》曰:「凡祭,有其廢之,莫敢舉也。有其舉之,莫敢廢也。」今東都太廟,廢已八
 朝,若果立之,是違已廢不舉也。謹按《禮記》曰:「當七廟五廟無虛主。」今欲立虛廟,是違廟不可虛也。謹按《左傳》:「丁丑,作僖公主。書不時也。」《記》又曰:「過時不祭,禮也。」合禮之祭,過時猶廢,非禮之主,可以作乎?今欲非時作主,是違非時不作主也。謹按《曾子問》:「古者師行以遷廟主行乎?孔子曰:天子巡狩,必以遷廟主行,載於齋車,言必有尊也。今也取七廟之主以行,則失之矣。」皇氏云:「遷廟主者,載遷一室之主也。」今欲載群廟之主以行,是違載遷之
 主也。謹按《禮記》曰:「天無二日,土無二王。嘗、禘、郊、社,尊無二上也。」今欲兩都建廟作主,是違尊無二上也。謹按《六典》序兩都宮闕及廟宇,此時東都有廟不載,是違《六典》不書也。遍考書傳,並不合修。浸以武德、貞觀之中,作法垂範之日,文物大備,儒彥畢臻,若可修營,不應議不及矣。《記》曰:樂由天作,禮以地制。天之體,動也。地之體,止也。」此明樂可作,禮難變也。伏惟陛下誠明載物,莊敬御天,孝方切於祖宗,事乃求於根本。再令集議,俾定所長。臣
 實職司,敢不條白以對。



 德章又有上中書門下及禮院詳議兩狀,並同載於後。其一曰:



 伏見八月六日敕,欲修東都太廟,令會議事。此時已有議狀,準禮不合更修。尚書丞郎已下三十八人,皆同署狀。德章官在禮寺,實忝司存,當聖上嚴禋敬事之時,會相公尚古黜華之日,脫國之祀典,有乖禮文,豈唯受責於曠官,竊懼貽恥於明代。所以勤勤懇懇,將不言而又言也。



 昨者異同之意,盡可指陳。一則以有都之名,便合立廟;次同欲崇修廟宇,
 以候時巡。殊不知廟不合虛,主惟載一也。謹按貞觀九年詔曰:「太原之地,肇基王業,事均豐、沛,義等宛、譙,約禮而言,須議立廟。」時秘書監顏師古議曰:「臣傍觀祭典,遍考禮經,宗廟皆在京師,不於下土別置。昔周之豐、鎬,實為遷都,乃是因事便營,非雲一時別立。」太宗許其奏,即日而停。由是而言,太原豈無都號,太原爾時猶廢,東都不立可知。且廟室惟新,即須有主,主既藏瘞,非虛而何?是有都立廟之言,不攻而自破矣。又按《曾子問》曰:「古者
 師行,必以遷廟主行乎?孔子曰:天子巡狩,必以遷廟主行,載於齋車,言必有尊也。今也取七廟之主以行,則失矣。」皇氏云:「遷廟主者,惟載新遷一室之主也。」未祧之主,無載行之文。假使候時巡,自可修營一室,議構九室,有何依憑?



 夫宗廟,尊事也,重事也,至尊至重,安得以疑文定論。言茍不經,則為擅議。近者敕旨,凡以議事,皆須一一據經。若無經文,任以史證。如或經史皆不據者,不得率意而言。則立廟東都,正經史無據,果從臆說,無乃前
 後相違也。《書》曰:「三人占,則從二人之言。」會議者四十八人,所同者六七人耳,比夫二三之喻,又何其多也!夫堯、舜之為帝,迄今稱詠之者,非有他術異智者也,以其有賢臣輔翼,能順考古道也。故堯之書曰「若稽古帝堯。」《孔氏傳》曰:「能順考古道。」傳說佐殷之君,亦曰「事不師古,匪說攸聞。」考之古道既如前,驗以國章又如此,將求典實,無以易諸。伏希必本正經,稍抑浮議,踵皋、夔之古道,法周、孔之遺文,則天下守貞之儒,實所幸甚。其餘已具前議。



 其二曰:



 夫
 宗廟之設,主於誠敬,旋觀典禮,貳則非誠。是以匪因遷都,則不別立廟宇。《記》曰:「天無二日,土無二王,嘗、禘、郊、社,尊無二上。」又曰:「凡祭,有其廢之,莫敢舉也。有其舉之,莫敢廢也。」則東都太廟,廢已多時,若議增修,稍違前志。何者?聖歷、神龍之際,武后始復明闢,中宗取其廟易置太廟焉,本欲權固人心,非經久之制也。伏以所存神主,既請祧藏,今廟室惟新,即須有主。神主非時不造,廟寢又無虛議,如修復以俟時巡,惟載一主,備在方冊,可得而
 詳。又引經中義有數等,或是弟子之語,或是他人之言。今廟不可虛,尊無二上,非時不造主,合載一主行,皆大聖祖及宣尼親所發明者,比之常據,不可同塗。又丘明修《春秋》,悉以君子定褒貶,至陳洩以忠獲罪,晉文以臣召君,於此數條,不復稱君子,將評得失,特以宣尼斷之。《傳》曰:「危疑之理,須聖言以明也。」或以東都不同他都,地有壇社宮闕,欲議權葺,似是無妨。此則酌於意懷,非曰經據也。但以遍討今古,無有壇社立廟之證,用以為說,實
 所未安。謹按上自殷、周,傍稽故實,除因遷都之外,無別立廟之文。



 制曰:「自古議禮,皆酌人情。必稷嗣知幾,賈生達識,方可發揮大政,潤色皇猷,其他管窺,蓋不足數。公卿之議,實可施行,德章所陳,最為淺近,豈得茍申獨見,妄有異同?事貴酌中,理宜從眾。宜令有司擇日修崇太廟,以留守李石充使勾當。」六年三月,擇日既定,禮官既行,旋以武宗登遐,其事遂寢。宣宗即位,竟迎太微宮神主祔東都太廟,禘祫之禮,盡出神主合食於太祖之前。



 《貞
 觀禮》,祫享,功臣配享於廟庭,禘享則不配。當時令文,祫禘之日,功臣並得配享。貞觀十六年,將行禘祭,有司請集禮官學士等議,太常卿韋挺等一十八人議曰:「古之王者,富有四海,而不朝夕上膳於宗廟者,患其禮過也。故曰:『春秋祭祀,以時思之。』至於臣有大功享祿,其後孝子率禮,潔粢豐盛,禮、祀、烝、嘗,四時不輟,國家大祫,又得配焉。所以昭明其勛,尊顯其德,以勸嗣臣也。其禘及時享,功臣皆不應預。故周禮六功之官,皆配大烝而已。先
 儒皆取大烝為祫祭。高堂隆、庾蔚之等多遵鄭學,未有將為時享。又漢、魏祫祀,皆在十月,晉朝禮官,欲用孟秋殷祭,左僕射孔安國啟彈,坐免者不一。梁初誤禘功臣,左丞何佟之駁議,武帝允而依行。降洎周、齊,俱遵此禮。竊以五年再殷,合諸天道,一大一小,通人雅論,小則人臣不預,大則兼及功臣。今禮禘無功臣,誠謂禮不可易。」乃詔改令從禮。至開元中改修禮,復令禘祫俱以功臣配饗焉。



 高宗上元三年十月,將祫享於太廟。時議者以《
 禮緯》「三年一祫,五年一禘」《公羊傳》云「五年而再殷祭」,議交互莫能斷決。太學博士史璨等議曰:「按《禮記正義》引鄭玄《禘祫志》云:『《春秋》:僖公三十三年十二月薨。文公二年八月丁卯,大享於太廟。《公羊傳》云:大享者何?祫也。』是三年喪畢,新君二年當祫,明年當禘於群廟。僖公、宣公八年皆有禘,則後禘去前禘五年。以此定之,則新君二年祫,三年禘。自爾已後,五年而再殷祭,則六年當祫,八年當禘。又昭公十年,齊歸薨,至十三
 年喪畢當祫,為平丘之會,冬,公如晉。至十四年祫,十五年禘《傳》云『有事於武宮』是也。至十八年祫,二十年禘。二十三年祫,二十五年禘。昭公二十五年『有事於襄宮』是也。如上所云,則禘已後隔三年祫,已後隔二年禘。此則有合禮經,不違《傳》義。」自此依璨等議為定。



 開元六年秋,睿宗喪畢,祫享於太廟。自後又相承三年一祫,五年一禘,各自計年,不相通數。至二十七年,凡經五禘、七祫。其年夏禘訖,冬又當祫。太常議曰:



 禘祫二禮,俱為殷祭,祫為合食祖廟,禘謂諦序尊卑。申先君逮下之慈,
 成群嗣奉親之孝,事異常享,有時行之。然而祭不欲數,數則黷;亦不欲疏,疏則怠。故王者法諸天道,制祀典焉。烝嘗象時,禘祫如閏。五歲再閏,天道大成,宗廟法之,再為殷祭者也。謹按《禮記·王制》、《周官·宗伯》,鄭玄注解,高堂所議,並云「國君嗣位,三年喪畢,祫於太祖。明年禘於群廟。自爾已後,五年再殷,一祫一禘。」漢、魏故事,貞觀實錄,並用此禮。又按《禮緯》及《魯禮禘祫注》云,三年一祫,五年一禘,所謂五年而再殷祭也。又按《白虎通》及《五經通義》、
 許慎《異義》、何休《春秋》、賀循《祭議》,並云三年一禘。何也?以為三年一閏,天道小備,五年再閏,天道大備故也。此則五年再殷,通計其數,一祫一禘,迭相乘矣。今太廟禘祫,各自數年,兩岐俱下,不相通計。或比年頻合,或同歲再序,或一禘之後,並為再祫,或五年之內,驟有三殷。法天象閏之期,既違其度;五歲再殷之制,數又不同。求之禮文,頗為乖失。



 說者或云:「禘祫二禮,大小不侔,祭名有殊,年數相舛。祫以三紀,抵小而合;禘以五斷,至十而周。有茲
 參差,難以通計。」竊以三祫五禘之說,本出《禮緯》,五歲再殷之數,同在其篇,會通二文,非相詭也。蓋以禘後置祫,二周有半,舉以全數,謂之三年,譬如三年一閏,只用三十二月也。其禘祫異稱,各隨四時,秋冬為祫,春夏為禘。祭名雖異,為殷則同,譬如礿、祠、烝、嘗,其體一也。鄭玄謂祫大禘小,傳或謂祫小禘大,肆陳之間,或有增減,通計之義,初無異同。蓋象閏之法,相傳久矣。惟晉代陳舒有三年一殷之議,自五年、八年又十一、十四,尋其議文所引,亦以象
 閏為言。且六歲再殷,何名象閏?五年一禘,又奚所施?矛盾之說,固難憑也。



 夫以法天之度,既有指歸,稽古之理,若茲昭著。禘祫二祭,通計明矣。今請以開元二十七年己卯四月禘,至辛巳年十月祫,至甲申年四月又禘,至丙戌年十月又祫,至己丑年四月又禘,至辛卯年十月又祫。自此五年再殷,周而復始。又禘祫之說,非唯一家,五歲再殷之文,既相師矣,法天象閏之理,大抵亦同。而禘後置祫,或近或遠,盈縮之度,有二法焉:鄭玄宗高堂,則先三而後二;徐邈之議,則
 先二而後三。謹按鄭氏所注,先三之法,約三祫五禘之文,存三歲五年之位。以為甲年既禘,丁年當祫,己年又禘,壬年又祫,甲年又禘,丁年又祫,周而復始,以此相承。祫後去禘,十有八月而近,禘後去祫,三十二月而遙,分析不均,粗於算矣。假如攻乎異端,置祫於秋,則三十九月為前,二十一月為後,雖小有愈,其間尚偏。竊據本文,皆云象閏,二閏相去,則平分矣。兩殷之序,何不等耶?且又三年之言,本舉全數,二周有半,實準三年,於此置祫,
 不違文矣,何必拘滯隔三正乎?蓋千慮一失,通儒之蔽也。徐氏之議,有異於是,研核周審,最為可憑。以為二禘相去,為月六十,中分三十,置一祫焉。若甲年夏禘,丙年冬祫,有象閏法,毫厘不偏。三年一祫之文,既無乖越;五歲再殷之制,疏數有均。校之諸儒,義實長久。今請依據以定二殷,預推祭月,周而復始。



 禮部員外郎崔宗之駁下太常,令更詳議,令集賢學士陸善經等更加詳核,善經亦以其議為允。於是太常卿韋縚奏曰:「禮有禘祫,俱
 稱殷祭,二法更用,鱗次相承。或云五歲再殷,一禘一祫。或云三年一祫,五年一禘。法天象閏,大趣皆同。皆以太廟禘祫,計年有差,考於經傳,微有所乖。頃在四月,已行禘享,今指孟冬,又申祫儀,合食禮頻,恐違先典。伏以陛下能事畢舉,舊物咸甄,宗祏祗慎之時,經訓申明之日。臣等忝在持禮,職司討論,輒據舊文,定其倫序。請以今年夏禘,便為殷祭之源,自此之後,禘、祫相代,五年再殷,周而復始。其今年冬祫,準禮合停,望令所司但行時享,
 即嚴禋不黷,庶合舊儀。」制從之。



 舊儀,天寶八年閏六月六日敕文:「禘祫之禮,以存序位,質文之變,蓋取隨時。國家系本仙宗,業承聖祖,重熙累盛,既錫無疆之休,合享登神,思弘不易之典。自今已後,每禘祫並於太清宮聖祖前設位序正,上以明陟配之禮,欽若玄象,下以盡虔祭之誠,無違至道。比來每緣禘祫,時享則停,事雖適於從宜,禮或虧於必備。已後每緣禘祫,其常享以素饌,三焚香以代三獻。」



 建中二年九月四日,太常博士陳京上
 疏言:「今年十月,祫享太廟,並合饗遷廟獻祖、懿祖二神主。《春秋》之義,毀廟之主,陳於太祖,未毀廟之主,皆升合食於太祖。太祖之位,在西而東向,其下子孫,昭穆相對,南北為別,初無毀廟遷主不享之文。徵是禮也,自於周室,而國朝祀典,當與周異。且周以後稷配天,為始封之祖,而下乃立廟。廟毀主遷,皆在太祖之後。禘祫之時,無先於太廟太祖者。正太祖東向之位,全其尊而不疑。然今年十月祫饗太廟,伏請據魏、晉舊制為比,則構築別
 廟。東晉以征西等四府君為別廟,至禘祫之時,則於太廟正太祖之位以申其尊,別廟祭高皇、太皇、征西等四府君以敘其親。伏以國家若用此義,則宜別為獻祖、懿祖立廟,禘祫祭之,以重其親;則太祖於太廟遂居東向,以全其尊。伏以德明、興聖二皇帝,曩立廟,至禘祫之時,常用饗禮,今則別廟之制,使就興聖廟藏祔為宜。」敕下尚書省百僚集議。禮儀使太子少師顏真卿議曰:「議者或云獻祖、懿祖親遠廟遷,不當祫享,宜永閟西夾室。
 又議者云二祖宜同祫享,於太祖並昭穆,而空太祖東向之位。又議者云,二祖若同袷享,即太祖之位永不得正,宜奉遷二祖神主祔藏於德明皇帝廟。臣伏以三議俱未為允。且禮經殘缺,既無明據,儒者能方義類,斟酌其中,則可舉而行之,蓋協於正也。伏惟太祖景皇帝以受命始封之功,處百代不遷之廟,配天崇享,是極尊嚴。且至禘祫之時,暫居昭穆之位,屈己申孝,敬奉祖宗,緣齒族之禮,廣尊先之道,此實太祖明神烝烝之本意,亦
 所以化被天下,率循孝悌也。請依晉蔡謨等議,至十月祫享之日奉獻祖神主居東向之位,自懿祖、太祖洎諸祖宗,遵左昭右穆之列。此有彰國家重本尚順之明義,足為萬代不易之令典也。又議者請奉二祖神主於德明皇帝廟,行祫祭之禮。夫祫,合也。故《公羊傳》云:『大事者何?祫也。』若祫祭不陳於太廟而享於德明廟,是乃分食也,豈謂合食乎?名實相乖,深失禮意,固不可行也。」



 貞元七年十一月二十八日,太常卿裴鬱奏曰:「禘、祫之禮,殷、
 周以遷廟皆出太祖之後,故得合食有序,尊卑不差。及漢高受命,無始封祖,以高皇帝為太祖。太上皇,高帝之父,立廟享祀,不在昭穆合食之列,為尊於太祖故也。魏武創業,文帝受命,亦即以武帝為太祖。其高皇、太皇、處士君等,並為屬尊,不在昭穆合食之列。晉宣創業,武帝受命,亦即以宣帝為太祖。其征西、潁川等四府君,亦為屬尊,不在昭穆合食之列。國家誕受天命,累聖重光。景皇帝始封唐公,實為太祖。中間世數既近,於三昭三穆之
 內,故皇家太廟,惟有六室。其弘農府君、宣、光二祖,尊於太祖,親盡則遷,不在昭穆之數。著在禮志,可舉而行。開元中,加置九廟,獻、懿二祖皆在昭穆,是以太祖景皇帝未得居東向之尊。今二祖已祧,九室惟序,則太祖之位又安可不正?伏以太祖上配天地,百代不遷,而居昭穆,獻、懿二祖,親盡廟遷,而居東向,徵諸故實,實所未安。請下百僚僉議。」敕旨依。



 八年正月二十三日,太子左庶子李嶸等七人議曰:



 《王制》:「天子七廟,三昭三穆,與太祖而
 七。」周制也。七者,太祖及文王、武王之祧,與親廟四也。太祖,後稷也。殷則六廟,契及湯與二昭二穆。夏則五廟,無太祖,禹與二昭二穆而已。晉朝博士孫欽議云:「王者受命太祖及諸侯始封之君,其已前神主,據已上數過五代即毀其廟,禘祫不復及也。禘祫所及者,謂受命太祖之後,迭毀主升藏於二祧者也。雖百代,禘祫及之。」伏以獻、懿二祖,太祖以前親盡之主也。擬三代以降之制,則禘祫不及矣。代祖神主,則太祖已下毀廟之主,則《公羊傳》所
 謂「已毀廟之主,陳於太祖」者是也。謹按漢永光四年詔,議罷郡國廟及親盡之祖,丞相韋玄成議太上、孝惠廟,皆親盡宜毀,太上廟主宜瘞於園,孝惠主遷於太祖廟。奏可。太上,同太祖已前之主,瘞於園,禘祫不及故也,則今獻、懿二祖之比也。孝惠遷於太祖廟,明太祖已下子孫,同禘祫所及,則今代祖元皇帝神主之比也。自魏、晉及宋、齊、陳、隋相承,始受命之君皆立廟,虛太祖之位。自太祖之後至七代君,則太祖東向位,乃成七廟。太祖以前
 之主,魏明帝則遷處士主置於園邑,歲時使令丞奉薦,世數猶近故也。至東晉明帝崩,以征西等三祖遷入西除,名之曰祧,以準遠廟。至康帝崩,穆帝立,於是京兆遷入西除,同謂之祧,如前之禮,並禘祫所不及。



 國朝始饗四廟,宣、光並太祖、世祖神主祔於廟。貞觀九年,將祔高祖於太廟,硃子奢請準禮立七廟,其三昭三穆,各置神主。太祖,依晉宋以來故事,虛其位,待遞遷方處之東向位。於是始祔弘農府君及高祖為六室,虛太祖之位而行禘
 祫。至二十三年,太宗祔廟,弘農府君乃藏於西夾室。文明元年,高宗祔廟,始遷宣皇帝於西夾室。開元十年,玄宗特立九廟,於是追尊宣皇帝為獻祖,復列於正室,光皇帝為懿祖,以備九室。禘祫猶虛太祖之位。祝文於三祖不稱臣,明全廟數而已。至德二載克復後,新作九廟神主,遂不造弘農府君神主,明禘祫不及故也。至寶應二年,祔玄宗、肅宗於廟,遷獻、懿二祖於西夾室,始以太祖當東向位,以獻、懿二祖為是太祖以前親盡神主,準禮
 禘祫不及,凡十八年。至建中二年十月,將祫饗,禮儀使顏真卿狀奏:合出獻、懿二祖神主行事,其布位次第及東面尊位,請準東晉蔡謨等議為定。遂以獻祖當東向,以懿祖於昭位南向,以太祖於穆位北向,以次左昭右穆,陳列行事。且蔡謨當時雖有其議,事竟不行,而我唐廟祧,豈可為準?嶸伏以嘗、禘、郊、社,尊無二上,瘞毀遷藏,禮有義斷。以獻、懿為親盡之主,太祖已當東向之尊,一朝改移,實非典故。謂宜復先朝故事,獻、懿神主藏於西
 夾室,以類《祭法》所謂「遠廟為祧,去祧為壇,去壇為墠,壇、墠有禱則祭,無禱乃止。」太祖既昭配天地,位當東向之尊。庶上守貞觀之首制,中奉開元之成規,下遵寶應之嚴式,符合經義,不失舊章。



 吏部郎中柳冕等十二人議曰:



 天子受命之君,諸侯始封之祖,皆為太祖。故雖天子,必有尊也,是以尊太祖焉;故雖諸侯,必有先也,亦以尊太祖焉。故太祖已下,親盡而毀。洎秦滅學,漢不及禮,不列昭穆,不建迭毀。晉失之,宋因之。於是有違五廟之制,於
 是有虛太祖之位。夫不列昭穆,非所以示人有序也;不建迭毀,非所以示人有殺也;違五廟之制,非所以示人有別也;虛太祖之位,非所以示人有尊也。此禮之所由廢。按《禮》:「父為士,子為天子,祭以天子,葬以士。」今獻祖祧也,懿祖亦祧也,唐未受命,猶士禮也。是故高祖、太宗以天子之禮祭之,不敢以太祖之位易之。今而易之,無乃亂先王之序乎?昔周有天下,追王太王、王季以天子之禮,及其祭也,親盡而毀之。漢有天下,尊太上皇以天子之禮,及其
 祭也,親盡而毀之。唐有天下,追王獻、懿二祖以天子之禮,及其祭也,親盡而毀之。則不可代太祖之位明矣。



 又按《周禮》有先公之祧,有先王之祧。先公之遷主,藏乎后稷之廟,其周未受命之祧乎?先王之遷主,藏乎文王之廟,其周已受命之祧乎?故有二祧,所以異廟也。今獻祖已下之祧,猶先公也;太祖已下之祧,猶先王也。請築別廟以居二祖,則行周之禮,復古之道。故漢之禮,因於周也;魏之禮,因於漢也;隋之禮,因於魏也。皆立三廟,有二
 祧。又立私廟四於南陽,亦後漢制也。以為人之子,事大宗降其私親,故私廟所以奉本宗也。太廟所以尊正統也。雖古今異時,文質異禮,而右禮之情,與問禮之本者,莫不通其變,酌而行之。故上致其崇,則太祖屬尊乎上矣;下盡其殺,則祧主親盡於下矣;中處其中,則王者主祧於中矣。



 工部郎中張薦等議曰:「昔殷、周以稷、珣始封,為不遷之祖,其毀廟之主,皆稷、珣之後,所以昭、穆合祭,尊卑不差。如夏后氏以禹始封,遂為不遷之祖。故夏五廟,禹與二昭二穆而已。據此則鯀之親
 盡,其主已遷。左氏既稱『禹不先鯀』,足明遷廟之主,中屬尊於始封祖者,亦在合食之位矣。又據晉、宋、齊、梁、北齊、周、隋史,其太祖已下,並同禘祫,未嘗限斷遷毀之主。伏以南北八代,非無碩學巨儒,宗廟大事,議必精博,驗於史冊,其禮僉同。又詳魏、晉、宋、齊、梁、北齊、周、隋故事,及《貞觀》、《顯慶》、《開元禮》所述,禘袷並虛東向。既行之已久,實群情所安。且太祖處清廟第一之室,其神主雖百代不遷,永歆烝嘗,上配天地,於郊廟無不正矣。若至禘、祫之時,暫居
 昭穆之列,屈己申孝,以奉祖禰,豈非伯禹烝烝敬鯀之道歟?亦是魏、晉及周、隋之太祖,不敢以卑厭尊之義也。議者或欲遷二祖於興聖廟,及請別置築室,至禘祫年饗之。夫祫,合也。此乃分食,殊乖禮意。又欲藏於西夾室,永不及祀,無異漢代瘞園,尤為不可。輒敢徵據正經,考論舊史,請奉獻、懿二祖與太祖並從昭穆之位,而虛東向。」



 司勛員外郎裴樞議曰:「禮之必立宗子者,蓋為收其族人,東向之主,亦猶是也。若祔於遠廟,無乃中有一間,等上不倫。西位常虛,則太祖永厭於
 昭穆;異廟別祭,則祫饗何主乎合食?永閟比於姜嫄,則推祥禖而無事。《禮》云:『親親故尊祖,尊祖故敬宗,敬宗故收族,所以宗廟嚴,社稷重。』由是言也,太祖之上復有追尊之祖,則親親尊祖之義,無乃乖乎?太廟之外,輕置別祭之廟,則宗廟無乃不嚴,社稷無乃不重乎?且漢丞相韋玄成請瘞於園,晉徵士虞喜請瘞於廟兩階之間。喜又引左氏說,古者先王日祭於祖考,月祀於曾高,時享及二祧,歲祫及壇墠,終禘及郊宗石室。是謂郊宗之上,復有石室之祖,
 斯最近矣。但當時議所居石室,未有準的。喜請於夾室中,愚以為石室可據,所以處之之道未安。何者?夾室謂居太祖之下毀主,非是安太祖之上藏主也。未有卑處正位,尊在傍居。考理即心,恐非允協。今若建石室於園寢,遷神主以永安,採漢、晉之舊章,仍禘袷之一祭,修古禮之殘缺,為國朝之典故,庶乎《春秋》變禮之正,動也中者焉。」



 考功員外郎陳京議曰:「京前為太常博士,已於建中二年九月四日,奏議祫饗獻、懿二祖所安之位,請下百僚
 博採所疑。其時禮儀使顏真卿因是上狀,與京議異,京議未行。伏見去年十一月二十八日詔下太常卿裴鬱所奏,大抵與京議相會。伏以興聖皇帝,同獻祖之曾祖,懿祖之高祖。夫以曾孫祔列於曾、高之廟,豈禮之不可哉?實人情之大順也。」



 京兆少尹韋武議曰:「凡三年一祫,五年一禘。祫則群廟大合,禘則各序其祧。謂主遷彌遠,祧室既修,當袷之歲,當以獻祖居於東向,而懿祖序其昭穆,以極所親。若行禘禮,則太祖復筵於西,以眾主列
 其左右。是則於太祖不為降屈,於獻祖無所厭卑。考禮酌情,謂當行此為勝。」



 同官縣尉仲子陵議曰:「今儒者乃援『子雖齊聖,不先父食』之語,欲令已祧獻祖,權居東向,配天太祖,屈居昭穆,此不通之甚也。凡左氏『不先食』之言,且以正文公之逆祀,儒者安知非夏后廟數未足之時,而言禹不先鯀乎!且漢之禘、祫,蓋不足徵。魏、晉已還,太祖皆近,是太祖之上,皆有遷主。歷代所疑,或引《閟宮》之詩而永閟,或因虞主之義而瘞園,或緣遠廟為祧以
 築宮,或言太祖實卑而虛位。惟東晉蔡謨憑左氏『不先食』以為說,欲令征西東向。均之數者,此最不安。且蔡謨此議,非晉所行。前有司不本謨改築之言,取征西東向之一句為萬代法,此共不可甚也。臣又思之,永閟瘞園,則臣子之心有所不安;權虛正位,則太祖之尊無時而定。則別築一室,義差可安。且興聖之於獻祖,乃曾祖也,昭穆有序,饗祀以時。伏請奉獻、懿二祖遷於德明、興聖廟,此其大順也。或以祫者合也,今二祖別廟,是分食也,
 何合之為?臣以為德明、興聖二廟,每禘祫之年,亦皆饗薦,是亦分食,奚疑於二祖乎?」



 其月二十七日,吏部郎中柳冕上《禘祫義證》,凡一十四道,以備顧問,並議奏聞。至三月十二日,祠部奏鬱等議狀。



 至十一年七月十二日,敕:「於頎等議狀,所請各殊,理在討論,用求精當。宜令尚書省會百僚與國子監儒官,切磋舊狀,定可否,仍委所司具事件聞奏。」其月二十六日,左司郎中陸淳奏曰:「臣尋七年百僚所議,雖有一十六狀,總其歸趣,三端而已。
 於頎等一十四狀,並云復太祖之位。張薦狀則云並列昭穆,而虛東饗之位。韋武狀同雲當祫之歲,獻祖居於東向,行禘之禮,太祖復筵於西。謹按禮經及先儒之說,復太祖之位,位既正也,義在不疑。太祖之位既正,懿、獻二主,當有所歸。詳考十四狀,其意有四:一曰藏諸夾室,二曰置之別廟,三曰遷於園寢,四曰祔於興聖。藏諸夾室,是無饗獻之期,異乎周人藏於二祧之義,禮不可行也。置之別廟,始於魏明之說,實非《禮經》之文。晉義熙九
 年,雖立此義,已後亦無行者。遷於園寢,是亂宗廟之儀,既無所憑,殊乖經意,不足徵也。惟有祔於興聖之廟,禘祫之歲乃一祭之,庶乎亡於禮者之禮,而得變之正也。」



 十九年三月,給事中陳京奏:「禘是大合祖宗之祭,必尊太祖之位,以正昭穆。今年遇禘,伏恐須定向來所議之禮。」敕曰:「禘祫之禮,祭之大者,先有眾議,猶未精詳,宜令百僚會議以聞。」時左僕射姚南仲等獻議狀五十七封,詔付都省再集百僚議定聞奏。戶部尚書王紹等五十
 五人奏議:「請奉遷獻祖、懿祖神主祔德明、興聖廟,請別增兩室奉安神主。緣二十四日禘祭,修廟未成,請於德明、興聖廟垣內權設幕屋為二室,暫安神主。候增修廟室成,準禮遷祔神主入新廟。每至禘祫年,各於本室行饗禮。」從之。是月十五日,遷獻祖、懿祖神主權祔德明、興聖廟之幕殿。二十四日,饗太廟。自此景皇帝始居東向之尊,元皇帝已下依左昭右穆之列矣。二祖新廟成,敕曰:「奉遷獻祖、懿祖神主,正太祖景皇帝之位,虔告之禮,
 當任重臣。宜令檢校司空平章事杜佑攝太尉,告太清宮;門下侍郎平章事崔損攝太尉,告太廟。」又詔曰:「國之大事,式在明禋。王者孝饗,莫重於禘祭,所以尊祖而正昭穆也。朕承列聖之休德,荷上天之睠命,虔奉牲幣,二十五年。永惟宗廟之位,禘嘗之序,夙夜祗慄,不敢自專。是用延訪公卿,稽參古禮,博考群議,至於再三。敬以令辰,奉遷獻祖宣皇帝神主、懿祖光皇帝神主,祔於德明、興聖皇帝廟。太祖景皇帝正東向之位。宜令所司循禮,
 務極精嚴,祗肅祀典,載深感惕。咨爾中外,宜悉朕懷。」



 會昌六年十月,太常禮院奏:「禘祫祝文稱號,穆宗皇帝、宣懿皇后韋氏、敬宗皇帝、文宗皇帝、武宗皇帝,緣從前序親親,以穆宗皇帝室稱為皇兄,未合禮文。得修撰官硃儔等狀稱:『禮敘尊尊,不敘親親。陛下於穆宗、敬宗、武宗三室祝文,恐須但稱嗣皇帝臣某昭告於某宗。』臣等同考禮經,於義為允。」從之。貞元十二年,祫祭太廟。近例,祫祭及親拜郊,皆令中使一人引伐國寶至壇所,所以昭
 示武功。至是上以伐國大事,中使引之非宜,乃令禮官一人,就內庫監領至太廟焉。



 舊儀,高祖之廟,則開府儀同三司淮安王神通、禮部尚書河間王孝恭、陜東道大行臺右僕射鄖國公殷開山、吏部尚書渝國公劉政會配饗。太宗之廟,則司空梁國公房玄齡、尚書右僕射萊國公杜如晦、尚書左僕射申國公高士廉配饗。高宗之廟,則司空英國公李勣、尚書左僕射北平縣公張行成、中書令高唐縣公馬周配饗。中宗之廟,則侍中平陽郡
 王敬暉、侍中扶陽郡王桓彥範、中書令南陽郡王袁恕己配享。睿宗之廟,則太子太傅許國公蘇瑰、尚書左丞相徐國公劉幽求配饗。



 天寶六載正月,詔:京城章懷、節愍、惠莊、惠文、惠宣太子,與隱太子、懿德太子同為一廟,呼為七太子廟,以便於祀享。太廟配饗功臣,高祖室加裴寂、劉文靜,太宗室加長孫無忌、李靖、杜如晦,高宗室加褚遂良、高季輔、劉仁軌,中宗室加狄仁傑、魏元忠、王同皎等十一人。大祭祀,騂犢減數。十載,太廟置內官。十一載閏三
 月,制:「自今已後,每月朔望日,宜令尚食造食,薦太廟,每室一牙盤,內官享薦。仍五日一開室門灑掃。」其後又有玄宗子靜德太子廟,肅宗子恭懿太子廟。孝敬廟在東京太廟院內,貞順皇后、讓皇帝廟在京中。餘皆四時致祭。



\end{pinyinscope}