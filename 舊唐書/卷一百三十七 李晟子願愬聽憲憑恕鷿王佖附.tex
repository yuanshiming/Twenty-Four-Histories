\article{卷一百三十七 李晟子願愬聽憲憑恕鷿王佖附}

\begin{pinyinscope}

 ○李晟子願愬
 聽憲憑恕鷿王佖附



 李晟,字良器,隴右臨洮人。祖思恭,父欽,代居隴右為裨將。晟生數歲而孤,事母孝謹,性雄烈,有才,善騎射。年十八從軍,身長六尺,勇敢絕倫。時河西節度使王忠嗣擊
 吐蕃,有驍將乘城拒鬥,頗傷士卒,忠嗣募軍中能射者射之。晟引弓一發而斃,三軍皆大呼,忠嗣厚賞之,因撫其背曰:「此萬人敵也。」鳳翔節度使高升雅聞其名,召補列將。嘗擊疊州叛羌於高當川,又擊宕州連狂羌於罕山,皆破之,累遷左羽林大將軍同正。廣德初,鳳翔節度使孫志直署晟總游兵,擊破黨項羌高玉等,以功授特進、試光祿卿,轉試太常卿。大歷初,李抱玉鎮鳳翔,署晟為右軍都將。四年,吐蕃圍靈州,抱玉遣晟將兵五千以
 擊吐蕃,晟辭曰:「以眾則不足,以謀則太多。」乃請將兵千人疾出大震關,至臨洮,屠定秦堡,焚其積聚,虜堡帥慕容穀鐘而還,吐蕃因解靈州之圍而去。拜開府儀同三司。無幾,兼左金吾衛大將軍、涇原四鎮北庭都知兵馬使,並總游兵。無何,節度使馬璘與吐蕃戰於鹽倉,兵敗,晟率所部橫擊之,拔璘出亂兵之中,以功封合川郡王。璘忌晟威名,又遇之不以禮,令朝京師,代宗留居宿衛,為右神策都將。德宗即位,吐蕃寇劍南,時節度使崔寧
 朝京師,三川震恐,乃詔晟將神策兵救之,授太子賓客。晟乃逾漏天,拔飛越,廓清肅寧三城,絕大渡河,獲首虜千餘級,虜乃引退,因留成都數月而還。



 建中二年,魏博田悅反,將兵圍臨洺、邢州,詔以晟為神策先鋒都知兵馬使,與河東節度使馬燧、昭義節度使李抱真合兵救臨洺。尋加兼御史中丞。河東、昭義軍攻楊朝光於臨洺南,晟與河東騎將李自良、李奉國擊悅於雙岡,悅兵卻,遂斬朝光。戰於臨洺,諸軍皆卻。晟引兵渡洺水,乘冰而
 濟,橫擊悅軍,王師復振,擊悅,大破之。三年正月,復以諸道軍擊敗悅軍於洹水,遂進攻魏州,以功加檢校左散騎常侍,實封百戶。無幾,兼魏府左司馬。時硃滔、王武俊聯兵在深、趙,怒朝廷賞功薄,田悅知其可間,遣使求援,滔與武俊應之,遂以兵圍康日知於趙州。李抱真分兵二千人守邢州,馬燧大怒,欲班師。晟謂燧曰:「初奉詔進討,三帥齊進。李尚書以邢州與趙州接圵,分兵守之,誠未為害,其精卒銳將皆在於此,令公遽自引去,奈王事
 何?」燧釋然謝晟,燧乃自造抱真壘,與之交歡如初。



 王武俊攻趙州,晟乃獻狀請解趙州之圍,欲引兵赴定州與張孝忠合勢,欲圖範陽,德宗壯之,加晟御史大夫,俾禁軍將軍莫仁擢、趙光銑、杜季泚皆隸焉。晟自魏州引軍而北,徑趨趙州,武俊聞之,解圍而去。晟留趙州三日,與孝忠兵合,北略恆州,圍硃滔將鄭景濟於清苑,決水以灌之。田悅、王武俊皆遣兵來救,戰於白樓。賊犯義武軍,稍卻,晟引步騎擊破之,晟所乘馬連中流矢。逾月,城中
 益急,滔、武俊大懼,乃悉收魏博之眾而來,復圍晟軍。晟內圍景濟,外與滔等拒戰,日數合,自正月至於五月。會晟病甚,不知人者數焉。軍吏合謀,乃以馬輿還定州,賊不敢逼。晟疾間,復將進師,會京城變起,德宗在奉天,詔晟赴難。晟承詔泣下,即日欲赴關輔。義武軍間於硃滔、王武俊,倚晟為輕重,不欲晟去,數謀沮止晟軍。晟謂將吏曰:「天子播越於外,人臣當百舍一息,死而後已。張義武欲沮吾行,吾當以愛子為質,選良馬以啖其意。」乃留
 子憑以為婚。義武軍有大將為孝忠委信者謁晟,晟乃解玉帶以遺之,因曰:「吾欲西行,願以為別。」陳赴難之意,受帶者果德晟,乃諫孝忠勿止晟。晟得引軍逾飛狐,師次代州,詔加晟檢校工部尚書、神策行營節度使,實封二百戶。晟軍令嚴肅,所過樵採無犯。自河中由蒲津而軍渭北,壁東渭橋以逼泚。時劉德信將子弟軍救襄城,敗於扈澗,聞難,率餘軍先次渭南,與晟合軍。軍無統一,晟不能制,因德信入晟軍,乃數其罪斬之。晟以數騎馳
 入德信軍,撫勞其眾,無敢動者。既人並德信軍,軍益振。



 時朔方節度使李懷光亦自河北赴難,軍於咸陽,不欲晟獨當一面以分己功,乃奏請與晟兵合,乃詔晟移軍合懷光軍。晟奉詔引軍至陳濤斜,軍壘未成,賊兵遽至,晟乃出陣,且言於懷光曰:「賊堅保宮苑,攻之未必克;今離其窟穴,敢出索戰,此殆天以賊賜明公也!」懷光恐晟立功,乃曰:「召軍適至,馬未秣,士未飯,詎可戰耶?不如蓄銳養威,俟時而舉。」晟知其意,遂收軍入壘,時興元元年正
 月也。每將合戰,必自異,衣錦裘、繡帽前行,親自指導。懷光望見惡之,乃謂晟曰:「將帥當持重,豈宜自表飾以啖賊也!」晟曰:「晟久在涇原,軍士頗相畏服,故欲令其先識以奪其心耳。」懷光益不悅,陰有異志,遷延不進。晟因人說懷光曰:「寇賊竊據京邑,天子出居近甸,兵柄廟略,屬在明公。公宜觀兵速進,晟願以所部得奉嚴令,為公前驅,雖死不悔。」懷光益拒之。晟兵軍於朔方軍北,每晟與懷光同至城下,懷光軍輒虜驅牛馬,吾姓苦之;晟軍無
 所犯。懷光軍惡其獨善,乃分所獲與之,晟軍不敢受。



 久之,懷光將謀沮晟軍,計未有所出。時神策軍以舊例給賜厚於諸軍,懷光奏曰:「賊寇未平,軍中給賜,咸宜均一。今神策獨厚,諸軍皆以為言,臣無以止之,惟陛下裁處。」懷光計欲因是令晟自署侵削己軍,以撓破之。德宗憂之,欲以諸軍同神策,則財賦不給,無可奈何,乃遣翰林學士陸贄往懷光軍宣諭,仍令懷光與晟參議所宜以聞。贄、晟俱會於懷光軍,懷光言曰:「軍士稟賜不均,何以
 令戰?」贄未有言,數顧晟。晟曰:「公為元帥,弛張號令,皆得專之。晟當將一軍,唯公所指,以效死命。至於增損衣食,公當裁之。」懷光默然,無以難晟,又不欲侵刻神策軍發於自己,乃止。



 懷光屯咸陽,堅壁八十餘日,不肯出軍,德宗憂之,屢降中使,促以收復之期。懷光托以卒疲,更請休息,以伺其便,然陰與硃泚交通,其跡漸露。晟懼為所人並,乃密疏請移軍東渭橋,以分賊勢。上初未之許。晟以懷光反狀已明,緩急宜有所備。蜀、漢之路,不可壅也,請
 以裨將趙光銑為洋州刺史,唐良臣為利州刺史,晟子婿張彧為劍州刺史,各將兵五百以防未然。上初納之,未果行。無何,吐蕃請以兵佐誅泚,上欲親總六師,移幸咸陽,以促諸軍進討。懷光聞之大駭,疑上奪其軍,謀亂益急。時鄜坊節度李建徽、神策將楊惠元及晟,並與懷光聯營,晟以事迫,會有中使過晟軍,晟乃宣令云:「奉詔徙屯渭橋。」乃結陣而行,至渭橋。不數日,懷光果劫建徽、惠元而人並其兵,建徽遁免,惠元為懷光所害。是日,車駕
 幸梁州。時變生倉卒,百官扈從者十二三,駱谷道路險阻,儲供無素,從官乏食,上嘆曰:「早從李晟之言,三蜀可坐致也。」晟大將張少弘自行在傳口詔授晟尚書左僕射、同中書門下平章事,以安眾心。晟拜哭受命,且曰:「長安宗廟所在,為天下本,若皆執羈靮,誰復京師?」乃浚城隍,繕兵甲,以圖收復。晟以孤軍獨當強寇,恐為二賊之所人並,乃卑詞厚幣,偽致誠於懷光,外示推崇,內為之備。時芻粟未集,乃令檢校戶部郎中張彧假京兆少尹,擇
 官吏以賦渭北畿縣。不旬日,芻糧皆足,晟乃大陳三軍,令之曰:「國家多難,亂逆繼興,屬車駕西幸,關中無主。予代受國恩,見危死節,臣子之分,況當此時,不能誅滅兇渠,以取富貴,非人豪也。渭橋橫跨大川,斷賊首尾,吾與公等戮力勤王,擇利而進,興復大業,建不世之功,能從我乎?」三軍無不泣下,曰:「唯公所使。」晟亦歔欷流涕。



 是時,硃泚盜據京城,懷光圖為反噬,河朔僭偽者三,李納虎視於河南,希烈鴟張於汴、鄭。晟內無貨財,外無轉輸,以
 孤軍而抗劇賊,而銳氣不衰,徒以忠義感於人心,故英豪歸向。戴休顏率奉天之眾,韓游瑰治邠寧之師,駱元光以華州之兵守潼關,尚可孤以神策之旅屯七盤,皆稟晟節度,晟軍大振。懷光以休顏、游瑰從晟,益懼。晟又致書於懷光,諭以禍福,令破賊迎鑾,以掩前過。懷光卒不悟,軍眾漸多離散,糗糧且竭,虜剽無所得,懼為晟所襲。三月,懷光自三原、富平東抵奉天,所至焚掠,乃自馮翊入據河中。懷光將孟涉、段威勇者,本神策將,惡懷光
 之不臣,既至富平,結陣於軍中,外向大呼而去,懷光不能制。涉、威勇以數千人歸晟,乃陳兵受涉等降卒,乃奏授涉檢校工部尚書,威勇兼御史大夫。



 德宗之幸山南,既入駱谷,謂渾瑊曰:「渭橋在賊腹內,兵勢懸隔,李晟可辦事乎?」瑊對曰:「李晟秉義執志,臨事不可奪,以臣計之,破賊必矣。」帝意始安。是月,渾瑊步將上官望自間道懷詔書加晟檢校右僕射,兼河中尹、河中晉絳慈隰節度使,益實封三百戶,又兼京畿、渭北、鄜坊丹延節度招討
 使。晟承詔流涕。時帝欲移幸西川,晟上表:「請駐蹕梁漢,系億兆之心,圖翦滅之勢。若規小舍大,作都岷峨,即人心失望,武士謀臣無所施矣。」四月,有詔加晟京畿、渭北、鄜坊、商華兵馬副元帥。時京兆府司錄李敬仲自京城來,諫議大夫鄭雲逵自奉天至,晟以京兆少尹張彧為副使,鄭雲逵為行軍司馬,李敬仲為節度判官,俾同主軍畫。又請以懷光舊將唐良臣保潼關,以河中節度授之;戴休顏守奉天,請以鄜坊節度授之;上皆從之。渭橋
 舊有粟十餘萬斛,度支先饋懷光軍欲盡,晟又奏曰:「近畿雖乘兵亂,猶可賦斂,儻寇賊未滅,宿兵曠時,人廢耕桑,又無儲蓄,非防微制勝之術也。」上納之。晟乃於畿甸率聚征賦,吏民樂輸,守禦益固,由是軍不乏食。



 神策軍家族多陷於泚,晟家亦百口在賊中,左右或有言及家者,晟因泣下曰:「乘輿何在,而敢恤家乎!」泚又使晟小吏王無忌之婿詣晟軍,且曰:「公家無恙,城中有書聞。」晟曰:「爾敢與賊為間!」遽命斬之。時轉輸不至,盛夏軍士或衣
 裘褐,晟亦同勞苦,每以大義奮激士心,卒無離叛者。會將吏數輩自賊中逃來,言泚眾攜離可滅之狀,士心益奮。先是,賊將姚令言及偽中丞崔宣咸使諜覘我軍,為邏騎所得,拘送於晟,晟解縛,食而遣之,誡之曰:「爾報崔宣,善為賊守,諸人勉力自固,勿不忠於賊也!」



 五月三日,晟引軍抵通化門,耀武而還,賊不敢出。晨集將佐,圖兵所向,諸將曰:「先拔外城,既有市里,然後北清宮闕。」晟曰:「若先收坊市,巷陌隘狹,間以居人,若賊設伏格鬥,百姓
 囂潰,非計也。且賊重兵堅甲,皆在苑中,若自苑擊其心腹,彼將圖走不暇,如此則宮闕保安,市不易肆,計之上也。」諸將曰:「善」。乃移書瑊、駱元光、尚可孤,克期進軍於城下。



 其月二十五日夜,晟自東渭橋移軍於光泰門外米倉村,以薄京城。晟臨高指麾,令設壕柵以候賊軍。俄而賊眾大至,賊驍將張庭芝、李希倩逼柵求戰,晟謂諸將曰:「吾恐賊不出,今冒死而來,天贊我也!」勒吳詵、康英俊、史萬頃、孟涉等縱兵擊之。時華州營在北,
 兵少,賊人並力攻之,晟遣李演、孟華以精卒救之。中軍鼓噪,演力戰,大破之,乘勝入光泰門;再戰,又敗之,殭尸蔽地,餘眾走入白華,夜聞慟哭之聲。



 翌日,將復出師,諸將請待西軍至,則左右夾攻。晟曰:「賊既傷敗,須乘勝撲滅,若俟其有備,豈王師之利耶!如待西軍,恐失機便。」二十八日,晟大集諸將駱元光、尚可孤,兵馬使吳詵、王佖,都虞候邢君牙、李演、史萬頃,神策將孟涉、康英俊,華州將郭審金、權文成,商州將彭元俊等,號令誓師畢,陳兵於光泰門外。
 乃使王佖、李演率騎軍,史萬頃領步卒,直抵苑墻神麚村。晟先是夜使人開苑墻二百餘步,至是賊已樹木柵之,賊倚柵拒戰。晟叱軍士曰:「安得縱賊如此,當先斬公等!」萬頃懼,先登,拔柵而入,王佖騎軍繼進,賊即奔潰,獲賊將段誠諫,大軍分道並入,鼓噪雷動。姚令言、張庭芝、李希倩猶力捍官軍,晟令決勝軍使唐良臣、兵馬使趙光銑、楊萬榮、孟日華等步騎齊進,賊軍陣成而屢北。戰十餘合,乘勝驅蹙,至於白華。忽有賊騎千餘出於官軍
 之背,晟以麾下百餘騎馳之,左右呼曰:「相公來!」賊聞之驚潰,官軍追斬,不可勝計。硃泚、姚令言、張庭芝尚有眾萬人,相率遁走,晟遣田子奇追之,其餘兇黨相率來降。是日,晟軍入京城,勒兵屯於含元殿前,晟舍於右金吾仗,仍號令諸軍曰「晟實不武,上憑睿算,下賴士心,幸得殲厥兇渠,肅清宮禁,皆三軍之力也。長安士庶,久陷賊庭,若小有震驚,則非伐罪吊人之義也。晟與公等各有家室,離別數年,今已成功,相見非晚,五日內不得輒通
 家信,違命者斬。」乃遣京兆尹李齊運、攝長安令陳元眾、攝萬年令韋上人及告喻百姓,居人安堵,秋毫無所犯。尚可孤軍人有擅取賊馬者,晟大將高明曜虜賊女妓一人,司馬伷取賊馬二匹,晟皆立斬之,莫敢忤視。士庶無不感悅,咸歔欷流涕,遠坊居人,亦有經宿方知者。二十九日,令孟涉屯於白華,尚可孤屯望仙門,駱元光屯章敬寺,晟自屯於安國寺。是日,斬賊將李希倩等八人,徇於市。



 六月四日,晟破賊露布至梁州,上覽之感泣,群臣
 無不隕涕,因上壽稱萬歲,奏曰:「李晟虔奉聖謨,蕩滌兇醜。然古之樹勛,力復都邑者,往往有之;至於不驚宗廟,不易市肆,長安人不識旗鼓,安堵如初,自三代以來,未之有也。」上曰:「天生李晟,為社稷萬人,不為朕也。」百官拜賀而退。是日,晟斬偽相李忠臣、張光晟、蔣鎮、喬琳、洪經綸、崔宣等,又表守臣節不屈於賊者程鎮之、劉乃、蔣沇、趙曄、薛岌等。



 晟初屯渭橋時,熒惑守歲,久之方退,賓介或勸曰:「今熒惑已退,皇家之利也,可速用兵。」晟曰:「天子
 外次,人臣但當死節,垂象玄遠,吾安知天道耶!」至是,謂參佐曰:「前者士大夫勸晟出兵,非敢拒也,且軍可用之,不可使知之。嘗聞五緯盈縮無準,晟懼復來守歲,則我軍不戰而自潰。」參佐嘆服,皆曰:「非所及也。」尋拜晟司徒,兼中書令,實封一千戶。



 晟綜理以備百司,令大將吳詵將兵三千至寶雞清道,晟又請至鳳翔迎扈,不許。七月十三日,德宗至自興元,渾瑊、韓游瑰、戴休顏以其兵扈從,晟與駱元光、尚可孤以其兵奉迎。時元從禁軍及山
 南、隴州、鳳翔之眾,步騎凡十餘萬,旌旗連亙數十里,傾城士庶,夾道歡呼。晟以戎服謁見於三橋,上駐馬勞之。晟再拜稽首,初賀元惡殄滅,宗廟再清,宮闈咸肅,抃舞感涕,跪而言曰:「臣忝備爪牙之任,不能早誅妖逆,至鑾輿再遷。及師於城隅,累月方殄賊寇,皆臣庸懦不任職之責,敢請死罪。」伏於路左。上為之掩涕,命給事中齊映宣旨,令左右起晟於馬前。是月,御殿大赦,贈晟父欽太子太保,母王氏贈代國夫人,賜永崇里第及涇陽上田、
 延平門之林園、女樂八人。入第之日,京兆府供帳酒饌,賜教坊樂具,鼓吹迎導,宰臣節將送之,京師以為榮觀。上思晟勛力,制紀功碑,俾皇太子書之,刊石立於東渭橋,與天地悠久,又令太子書碑詞以賜晟。



 晟以涇州倚邊,屢害戎帥,數為亂階,乃上書請理不用命者,兼備耕以積粟,攘卻西蕃,上皆從之。詔以晟兼鳳翔尹、鳳翔隴右節度使,仍充隴右涇原節度,兼管內諸軍及四鎮、北庭行營兵馬副元帥,改封西平郡王。初,帝在奉天,鳳翔
 軍亂,殺其帥張鎰,立小將李楚琳。至是楚琳在朝,晟請以楚琳俱往鳳翔,將誅之,上以初復京師,方安反側,不許。八月,晟至鳳翔,理殺張鎰之罪,斬王斌等十餘人。初,硃泚亂時,涇州亦殺其帥馮河清,立別將田希鑒,方屬播遷,不遑討伐,以涇帥授之。至是,晟奏曰:「近者中原兵禍,皆起涇州,且其地逼西戎,易為反覆。希鑒兇徒,將校驕逆,若不懲革,終為後患。」從之。晟至鳳翔,托以巡邊,至涇州,希鑒迎謁,於坐執而誅之,並誅害河清者石奇等
 三十餘人,具事以聞。上曰:「涇州亂逆泉藪,非晟莫能理之。」還鎮,表右龍武將軍李觀為涇原節度使,吐蕃深畏之。晟常曰;「河、隴之陷也,豈吐蕃力取之,皆因將帥貪暴,種落攜貳,人不得耕稼,展轉東徙,自棄之耳。且土無絲絮,人苦征役,思唐之心,豈有已乎!」乃傾家財以賞降者,以懷來之。降虜浪息曩,晟奏封王,每蕃使至,晟必置息曩於坐,衣以錦袍、金帶以寵異之。蕃人皆相指目,榮羨息曩。



 蕃相尚結贊頗多詐謀,尤惡晟,乃相與議云:「唐之
 名將,李晟與馬燧、渾瑊耳。不去三人,必為我憂。」乃行反間,遣使因馬燧以請和,既和,即請盟,復因盟以虜瑊,因以賣燧。貞元二年九月,吐蕃用尚結贊之計,乃大興兵入隴州,抵鳳翔,無所虜掠,且曰:「召我來,何不以牛酒犒勞?」徐乃引去,持是間晟也。是役也,晟先令衙將王佖選銳兵三千,設伏於汧陽,誡之曰:「蕃軍過城下,勿擊首尾,首尾縱敗,中軍力全,若合勢攻汝,必受其弊。但俟其前軍已過,見五方旗、武豹衣,則其中軍也,突其不意,可
 建奇功。」佖如晟節度,果遇結贊。及出奮擊,賊皆披靡,佖軍不識結贊,故結贊僅而獲免。十月,晟出師襲吐蕃摧沙堡,拔之,斬其堡使扈屈律悉蒙等,自是結贊數遣使乞和。十二月,晟朝京師,奏曰:「戎狄無信,不可許。」宰相韓滉又扶晟議,請調軍食以給晟,命將擊之。上方厭兵,疑將帥生事邀功。會滉卒,張延賞秉政,與晟有隙,屢於上前間晟,言不可久令典兵。延賞欲用劉玄佐、李抱真,委以西北邊事,俾立功以壓晟,德宗竟納延賞之言,罷晟兵
 柄。三年三月,冊拜晟為太尉、中書令,奉朝請而已。其年閏五月,渾瑊與尚結贊同盟於平涼,果為蕃兵所劫,瑊單馬僅免,將吏皆陷。六月,罷河東節度使馬燧為司徒,盡中尚結贊之謀。



 晟既罷兵權,朝謁之外,罕所過從。有通王府長史丁瓊者,亦為張延賞所排,心懷怨望,乃求見晟言事,且曰:「太尉功業至大,猶罷兵權,自古功高,無有保全者。國家倘有變故,瓊願備左右,狡兔三穴,盍早圖之。」晟怒曰:「爾安得不祥之言!」遽執瓊以聞。四年三月,
 詔為晟立五廟,以晟高祖芝贈隴州刺史,曾祖嵩贈澤州刺史,祖思恭贈幽州大都督。廟成,官給牲牢、祭器、床帳,禮官相儀以祔焉。



 五年九月,晟與侍中馬燧見於延英殿,上嘉其勛力,詔曰:「昔我列祖,乘乾坤之蕩滌,掃隋季之荒屯,體元禦極,作人父母;則亦有熊羆之士,不二心之臣,左右經綸,參翊締構,昭文德,恢武功,威不若,康不乂,用端命於上帝,付畀四方。宇宙既清,日月既貞,王業既成,太階既平;乃圖厥容,列於斯閣,懋昭績效,式表
 儀形,一以不忘於朝夕,一以永垂乎來裔,君臣之義,厚莫重焉。貞元己巳歲秋九月,我行西宮,瞻宏閣崇構,見老臣遺象,顒然肅然,和敬在色,想雲龍之葉應,感致來之艱難。睹往思今,取類非遠。且功與時並,才為代生,茍蘊其才,遇其時,尊主庇人,何代不有?在中宗,則桓彥範等著其輔戴之績;在玄宗,則劉幽求等申翼奉之勛;在肅宗,則郭子儀掃殄氛昆;今則李晟等保寧朕躬。咸宣力肆勤,光復宗社。訂之前烈,夫豈多謝,闕而未錄,孰謂
 旌賢。況念功紀德,文祖所為也,在予曷其敢怠!有司宜敘年代先後,各圖其像於舊臣之次,仍令皇太子書朕是命,紀於壁焉。庶播嘉庸,式昭於下,俾後來者尚揖清顏,知元勛之不朽。」復命皇太子書其文以賜晟,晟刻石於門左。



 初,晟在鳳翔,謂賓介曰:「魏徵能直言極諫,致太宗於堯、舜之上,真忠臣也,僕所慕之。」行軍司馬李叔度對曰:「此搢紳儒者之事,非勛德所宜。」晟斂容曰:「行軍失言。」傳稱『邦有道,危言危行」。今休明之期,晟幸得備位將
 相,心有不可,忍而不言,豈可謂有犯無隱,知無不為者耶!是非在人主所擇耳。」叔度慚而退。故晟為相,每當上所顧問,必極言匪躬。盡大臣之節。性沉默,未嘗洩於所親。臨下明察,每理軍,必曰某有勞,某能其事,雖廝養小善,必記姓名。尤惡下為朋黨相構,好善嫉惡,出於天性。嘗有恩者,厚報之。初,譚元澄為嵐州刺史,嘗有恩於晟,後坐貶於岳州;比晟貴,上疏理之,詔贈元澄寧州刺史。元澄三子,晟撫待勤至,皆為成就宦學,人皆義之。理家
 以嚴稱,諸子侄非晨昏不得謁見,言不及公事,視王氏甥如己子。嘗正歲,崔氏女歸省,未及階,晟卻之曰:「爾有家,況姑在堂,婦當奉酒醴從饋,以待賓客。」遂不視而遣還家,其達禮敦教如此。貞元九年八月薨,時年六十七。上震悼出涕,廢朝五日,令百官就第臨吊,命京兆尹李充監護喪事,官給葬具,賵賻加等。比大斂,上手書致意,送柩前,曰:



 皇帝遣宮闈令第五守進致旨於故太尉、中書令、西平郡王、贈太師之靈曰:「天祚我邦,是生才傑,稟
 陰陽之粹氣,實山嶽之降靈。弘濟患難,保佐王室:掃蕩氛昆,廓清上京。忠誠感於人神,功業施於社稷,匡時定亂,實賴元勛。洎領上臺,克諧中外,訏謨帝道,葉贊皇猷。常竭嘉言,以匡不迨,情所親重,義無間然。方期與國同休,永為邦翰。比嬰疾恙,雖歷旬時,日冀痊除,重期相見,弼予在位,終致和平。豈圖藥餌無征,奄至薨逝,喪我賢哲,虧我股肱,天不憖遺,痛惜何極,嗚呼!大廈方構,旋失棟梁;巨川未濟,遂亡舟楫。君臣之義,追慟益深,循省遺
 章,倍增感切。卿一門胤嗣,朕必終始保持。況願等弟兄,承卿教訓,朕之志義,豈忘平生?縱卿不言,朕亦存信。比者卿在之日,卻未見朕深心,今卿與朕長乖,方冀知朕誠志。無以為念,發言涕零,是用躬述數行,貴寫所懷得盡。臨紙遣使,不能飾詞,魂而有知,當體朕意。



 冊贈太師,謚曰忠武。晟薨後,城鹽州,復鹽池,上賜宰臣新鹽,惻然思晟,乃令致鹽於靈座。又時遣中使至晟第存撫諸子,教戒備至,聞願等有一善,上喜形於色。眷遇終始,無與
 晟比。



 元和四年,詔曰:「夫能定社稷,濟生人,存不朽之名,垂可久之業者,必報以殊常之寵,待以親比之恩,與國無窮,時惟茂典。故奉天定難功臣、太尉、兼中書令、上柱國、西平郡王、食實封一千五百戶、贈太師李晟,間代英賢,自天忠義,邁濟時之宏算,抱經武之長材,貫以至誠,協於一德,嘗遭屯難之際,實著戡定之功。鯨鯢既殲,宮廟斯復,眷茲勛伐,則既褒崇。永言天步之夷,載懷邦傑之功,思加崇於往烈,爰協比於後昆,睦以宗親,將予厚
 意。其家宜令編附屬籍。晟饗德宗廟庭。」



 晟十五子:侗、伷、偕,無祿早世;次願、聰、總、愻、憑、恕、憲、愬、懿、聽、惎、慇、聰、總官卑而卒,而願、愬、聽最知名。



 願,幼謙謹寡過,晟立大勛,諸子猶無官,宰相奏陳,德宗即日召願拜銀青光祿大夫、太子賓客、上柱國。舊制,勛至上柱國,賜門戟,即令賜願,乃與父並列棨戟於門。九年,丁父憂。十二年,服闋,德宗召見願等於延英,憫然久之曰:「朕在宮中,常念卿等,追懷勛德,何日忘之。又聞卿等居喪得禮,朕甚嘉之。」各
 賜衣一襲、絹三千匹。願依前授太子賓客,兄弟同日拜官者九人。尋轉左衛大將軍。元和元年八月,檢校禮部尚書,兼夏州刺史、夏綏銀宥等州節度使,威令簡肅,甚得綏懷之術。客有亡馬者,以狀告願,願以狀榜於路,懸金以購之。不三日,所亡馬系之榜下,仍置書一緘曰:「馬逸及群,不時告,罪當死,敢以良馬一匹贖罪,並亡馬謹納於路。」願付客亡馬而縱其良馬。境內嚴肅,多如此類。轉徐州刺史、武寧軍節度。到鎮,以青、鄆不恭,奉命討
 伐,屠城下邑,捷奏屢聞。無何,有疾,以其弟愬代為徐帥,入為刑部尚書。疾愈,檢校尚書左僕射,兼鳳翔尹、鳳翔隴右節度使。然自是頗怠於為理,無復素志,聲色之外,全不介懷。



 長慶二年二月,檢校司空,兼汴州刺史、宣武軍節度使。先是,張弘靖為汴帥,以厚賞安士心。及願至,帑藏已竭,而願恣其奢侈,門內數百口,仰給官司,不恤軍政,賞賚不及弘靖時,而以威刑馭下。又令妻弟竇緩將親兵,緩亦驕傲黷貨,以是群情聚怨。是歲七月四日
 夜,牙將李臣則、薛志忠、秦鄰等三人宿直,突入竇緩帳中,斬緩首以徇。願聞有變,與左右數人露發而走,登子城北樓,懸縋而下,由水竇而出。比曉,行十數里,遇野人驅驢,奪而乘之,得至鄭州。願妻竇氏死於亂兵之手,子三人匿而獲免,僕妾為軍士所俘。城中大掠三日,乃立其牙將李為留後,以邀旄鉞,月餘,方誅之。願坐貶隨州刺史。朝廷念晟之勛,終不加罪,入為左金吾衛大將軍。長慶四年六月,復檢校司空,兼河中尹,充河中、晉、絳、
 慈、隰節度使。河中之政,亦如岐、梁。加以願結托權幸,厚行賂遺,賦入隨盡,軍府蕭然,賴遽疾終,不爾,蒲人必有更變。寶應元年六月卒,贈司徒。



 愬以父廕起家,授太常寺協律郎,遷衛尉少卿。愬早喪所出,保養於晉國夫人王氏,及卒,晟以本非正室,令服緦,號哭不忍,晟感之,因許服縗。既練,丁父憂,愬與仲弟憲廬於墓側,德宗不許,詔令歸第。居一宿,徒跣復往,上知不可奪,遂許終制。服闋,授右庶子,轉少府監、左庶子。出為坊、晉二州刺史。以
 理行殊異,加金紫光祿大夫。復為庶子,累遷至太子詹事,宮苑閑廊使。



 愬有籌略,善騎射。元和十一年,用兵討蔡州吳元濟。七月,唐鄧節度使高霞寓戰敗,又命袁滋為帥,滋亦無功。愬抗表自陳,願於軍前自效。宰相李逢吉亦以愬才可用,遂檢校左散騎常侍,兼鄧州刺史、御史大夫,充隨、唐、鄧節度使。兵士摧敗之餘,氣勢傷沮,愬揣知其情,乃不肅軍陣,不齊部伍。或以不肅為言,愬曰:「賊方安袁尚書之寬易,吾不欲使其改備。」乃紿告三軍
 曰;「天子知愬柔而忍恥,故令撫養爾輩。戰者,非吾事也。」軍眾信而樂之。愬又散其優樂,未嘗宴樂,士卒傷痍者,親自撫之。賊以嘗敗高、袁二帥,又以愬名位非所畏憚者,不甚增其備。愬沉勇長算,推誠待士,故能用其卑弱之勢,出賊不意。居半歲,知人可用,乃謀襲蔡,表請濟師。詔河中、鄜坊騎兵二千人益之,由是完緝器械,陰計戎事。嘗獲賊將丁士良,召入與語,辭氣不撓,愬異之,因釋其縛,置為捉生將。士良感之,乃曰:「賊將吳秀琳總眾數
 千,不可遽破者,用陳光洽之謀也。士良能擒光洽以降秀琳。」愬從之,果擒光洽。十二月,吳秀琳以文成柵兵三千降。醖乃徑徙之新興柵,遂以秀琳之眾攻吳房縣,收其外城。初,將攻吳房,軍吏曰:「往亡日,請避之。」愬曰:「賊以往亡謂吾不來,正可擊也。」及戰,勝捷而歸。賊以驍騎五百追愬,愬下馬據胡床,令眾悉力赴戰,射殺賊將孫忠憲,乃退。或勸愬遂拔吳房,愬曰:「取之則合勢而固其穴,不如留之以分其力。」



 初,吳秀琳之降,愬單騎至柵下與
 之語,親釋其縛,署為衙將。秀琳感恩,期於效報,謂愬曰:「若欲破賊,須得李祐,某無能為也。」祐者,賊之騎將,有膽略,守興橋柵,常侮易官軍,去來不可備。愬召其將史用誠誡之曰:「今祐以眾獲麥於張柴,爾可以三百騎伏旁林中,又使搖旆於前,示將焚麥者。祐素易我軍,必輕而來逐,爾以輕騎搏之,必獲祐。」用誠等如其料,果擒祐而還。官軍常苦祐,皆請殺之,愬不聽,解縛而客禮之。愬乘間常召祐及李忠義,屏人而語,或至夜分。忠義,亦降將
 也,本名憲,愬致之。軍中多諫愬,愬益寵祐。始募敢死者三千人以為突將,醖自教習之。愬將襲元濟,會雨水,自五月至七月不止,溝塍潰溢,不可出師。軍吏咸以不殺祐為言,簡翰日至,且言得賊諜者具言其事。愬無以止之,乃持祐泣曰:「豈天意不欲平此賊,何爾一身見奪於眾口!」愬又慮諸軍先以謗聞,則不能全祐,乃械送京師,先表請釋,且言:「必殺祐,則無以成功者。」比祐至京,詔釋以還愬,乃署為散兵馬使,令佩刀巡警,出入帳中,略無
 猜閑。又改為六院兵馬使。舊軍令,有舍賊諜者屠其家,愬除其令,因使厚之,諜反以情告愬,愬益知賊中虛實。



 陳許節度使李光顏勇冠諸軍,賊悉以精卒抗光顏。由是愬乘其無備,十月,將襲蔡州。其月七日,使判官鄭澥告師期於裴度。十日夜,以李祐率突將三千為先鋒,李忠義副之,愬自帥中軍三千,田進誠以後軍三千殿而行。初出文成柵,眾請所向,愬曰:「東六十里止。」至賊境,曰張柴砦,盡殺其戍卒,令軍士少息,繕羈靮甲胄,發刃
 彀弓,復建旆而出。是日,陰晦雨雪,大風裂旗旆,馬慄而不能躍,士卒苦寒,抱戈殭僕者道路相望。其川澤梁逕險夷,張柴已東,師人未嘗蹈其境,皆謂投身不測。初至張柴,諸將請所止,愬曰:「入蔡州取吳元濟也。」諸將失色。監軍使哭而言曰:「果落李祐計中!」愬不聽,促令進軍,皆謂必不生還,然已從愬之令,無敢為身計者。醖道分五百人斷洄曲路橋,其夜凍死者十二三。又分五百人斷朗山路。自張柴行七十里,比至懸瓠城,夜半,雪愈甚。近城
 有鵝鴨池,愬令驚擊之,以雜其聲。賊恃吳房、朗山之固,晏然無一人知者。李祐、李忠義坎墉而先登,敢銳者從之,盡殺守門卒而登其門,留擊柝者。黎明,雪亦止,愬入,止元濟外宅。蔡吏告元濟曰:「城已陷矣。」元濟曰:「是洄曲子弟歸求寒衣耳。」俄聞愬軍號令將士云:「常侍傳語。」乃曰:「何常侍得至於此?」遂驅率左右乘子城拒捍。田進誠以兵環而攻之。愬計元濟猶望董重質來救,乃令訪重質家安恤之,使其家人持書召重質。重質單騎而歸愬,
 白衣泥首,愬以客禮待之。田進誠焚子城南門,元濟城上請罪,進誠梯而下之,乃檻送京師。其申、光二州及諸鎮兵尚二萬餘人,相次來降。



 自元濟就擒,愬不戮一人,其為元濟執事帳下廚廄之間者,皆復其職,使之不疑。乃屯兵鞠場以待裴度。翌日,度至,愬具櫜鞬候度馬首。度將避之,愬曰:「此方不識上下等威之分久矣,請公因以示之。」度以宰相禮受愬迎謁,眾皆聳觀。明日,愬軍還於文成柵。十一月,詔以愬檢校尚書左僕射,兼襄州刺
 史、山南東道節度、襄鄧隨唐復郢均房等州觀察等使、上柱國,封涼國公,食邑三千戶,食實封五百戶,一子五品正員。



 憲宗有意復隴右故地,元和十三年五月,授愬鳳翔隴右節度使,仍詔路由闕下。愬未發,屬李師道再叛,詔田弘正、義成、宣武等軍討之,乃移愬為徐州刺史、武寧軍節度使,代其兄願。兄弟交換岐、徐二鎮,旬日間再踐父兄之任。愬至徐方,理兵有方略。時蔡將董重質貶春州司戶,愬上表請恕重質賜之,堪於軍前驅使,即
 詔徵還送武寧軍,愬乃署為牙將。愬破賊金鄉,凡十一戰,擒賊將五十,俘斬萬計。淄青平,將有事燕、趙。元和十五年九月,以愬檢校左僕射、同中書門下平章事、潞州大都督府長史、昭義節度使,仍賜興寧里第。十月,王承宗卒,魏博田弘正移任鎮州。愬至潞州,四月,遷魏州大都督府長史、魏博節度使。長慶元年,幽、鎮復亂,愬聞之,素服以令三軍曰:「魏人所以富庶而能通知聖化者,由田公故也。天子以其仁而愛人,使理鎮、冀。且田公出於
 魏,撫師七年,一旦鎮人不道,敢茲殘害,以魏為無人也。若父兄子弟食田公恩者,其何以報?」眾皆慟哭。又以玉帶、寶劍與牛元翼,遣使謂之曰:「吾先人常以此劍立大勛,吾又以此劍平蔡寇,今鎮人叛逆,公以此翦之。」元翼承命感激,乃以劍及帶令於軍中,報之曰:「願以眾從,竭其死力」。方有制置,會疾作,不能治軍,人違紀律,功遂無成。朝廷以田布代之,除太子少保,歸東都。是年十月,卒於洛陽,時年四十九。穆宗聞之震悼,賵賻加等,贈太尉。



 始,晟克復京城,市不改肆;及愬平淮蔡,復踵其美。父子仍建大勛,雖昆仲皆領兵符,而功業不侔於愬,近代無以比倫。加以行己有常,儉不違禮,弟兄席父勛寵,率以僕馬第宅相矜,唯愬六遷大鎮,所處先人舊宅一院而已。晚歲忽於取士,闢請不得其人,至使吏緣為奸,軍政不肅,物論稍減,惜哉!



 聽七歲以廕授太常寺協律郎,常入公署,吏胥小之,不為致敬,聽令鞭之見血,父晟奇之。後隨吐突承璀討王承宗,為神策行營兵馬使。時昭義
 盧從史持兩端,無心討賊,承璀用聽計,擒從史以獻。轉左驍衛將軍、兼御史中丞。出為安州刺史,隨鄂岳觀察使柳公綽討吳元濟,軍中動靜,悉用聽謀,軍聲遂振。元和中,討李師道,聽為楚州刺史,統淮南之師。鄆人素易淮軍,聽潛訓練,出其不意,趨海州,據險要,破沐陽兵,降朐山戍,懷仁、東海兩城望風乞降,山東平。元和十四年五月,以功授檢校左散騎常侍、夏州刺史、夏綏銀宥節度使。十五年六月,改靈州大都督府長史、靈鹽節度使。
 境內有光祿渠,廢塞歲久,欲起屯田以代轉輸,聽復開決舊渠,溉田千餘頃,至今賴之。就加檢校工部尚書。



 初,聽為羽林將軍,有名馬,穆宗在東宮,令近侍諷聽獻之,聽以職總親軍,不敢從。及即位之始,幽、冀不廷,太原與二鎮接境,方議易帥,宰臣進擬,上皆不允,謂宰臣曰:「李聽為羽林將軍,不與朕馬,是必可任。」長慶二年二月,授檢校兵部尚書、太原尹、北京留守、河東節度使,代裴度。四年七月,轉滑州刺史、義成軍節度使。大和二年,討李
 同捷。時魏博行營將丌志沼潛結滄、鎮,擅回戈攻其帥史憲誠。詔聽帥師援之,大破其叛卒,志沼奔鎮州,為王庭湊所殺,聽遂凱旋,以功封涼國公,授一子五品官。王庭湊再違朝旨,詔聽以全師屯貝州。路由魏州,史憲誠懼聽見襲,衷甲郊迎,候吏密白呼,乃令兵士匣刃櫜弓,休於野外,魏人遂安。後憲誠欲入覲,竭其府庫,魏人怨之,殺憲誠,衙軍立其大將何進滔。詔聽兼領魏博節度使,將兵北渡,魏人不納聽,乘城拒守,乃屯兵館陶。魏兵
 遽襲,聽不為備,其軍大敗,無復部伍,晝夜奔走,僅而獲免,喪師過半,輜車兵仗並皆委棄。御史中丞溫造、殿中侍御史崔蠡彈之曰:



 臣聞賞罰不立,無以示天下;是非一貫,莫能建大中。竊見義成軍節度使李聽,昨者資其承藉,委以統戎,俾代憲誠,付之雄鎮。總二萬虎貔之旅,位極寵榮;兼兩籓節制之權,心無報效。況陛下授以神算,假以天威,入魏之期,克日先定。而聽擁旄觀望,按甲遷延,熒惑人心,逗撓軍政。遂使憲誠陷於屠戮,亂眾肆
 其奸兇,失六郡於垂成,固危巢於已覆。委貝州而不守,燒劫無遺;望淺口而疾驅,狼狽就道。自圖茍免,不吝苞羞,蔑棄朝章,有同兒戲。魏州之亂,職聽之由,論其負恩,萬死猶幸。伏以封常清河南失律,斬於關門;高霞寓唐鄧破傷,投諸遐裔;渾鎬節制易定,將戰而兵力不支;袁滋逗留西川,欲進而兇渠尚在。或親當矢石,或躬歷艱危,勢屈賊鋒,竟申朝典,未曾貸法,必震皇威。今李聽罪狀夙聞,中外憤惋,比之常清等輩,萬萬過之。若陛下猶
 示含弘,不置極法,臣等恐憲章墜地,天下寒心。伏請付法。



 上不之罪,罷兵柄,為太子少師。



 聽頗賂遺權幸以為援,居無何,復檢校司徒,起為邠寧節度使。邠州衙,相傳不利葺修,以至隳環,聽曰:「帥臣鑿兇門而出,豈有拘於巫祝而隳公署耶!」遂命葺之,卒無變異。大和六年,轉武寧軍節度使。時聽有蒼頭為徐州將,不欲聽至,聽先使親吏慰勞徐人,為蒼頭所殺。聽不敢進,固以疾辭,用為太子太保。七年,出守鳳翔,時人榮之。九年,改陳許節
 度,未至鎮,復除太子太保分司。開成元年,出為河中尹、河中晉慈隰節度使。四年,以疾求代,除太子太保。是歲十月卒,時年六十一,贈司徒。



 聽十領節旌,所不至者三鎮。蒞官苛細,好將迎遺賂,故急於聚斂,窮極侈欲。位至一品,竟終牖下,非西平之遺德,焉能及此乎!



 憲,晟第五子。晟十子,憲、愬最仁孝。及長,好儒術,以禮法修整,起家太原府參軍、醴泉縣尉。于頔鎮襄陽,闢為從事。時吳少誠據淮西,獨憚頔之威,當時咸以憲謀畫致之。元和八
 年,田弘正以魏博奉朝旨,闢憲為從事,授衛州刺史,遷絳州,所至以理行稱。入為宗正少卿,遷光祿卿。穆宗即位,以太和公主降回鶻,命金吾大將軍胡證充送公主使,命憲副之。使還,獻《入蕃道里記》,遷檢校左散騎常侍,兼太府卿。出為洪州刺史、江西觀察使。大行二年,轉嶺南節度使。憲雖勛伐之家,然累歷事任,皆以吏能擢用,所履官秩,政績流聞。性本明恕,尤精律學,屢詳決冤獄,活無罪者數百人。以能入官,官無敗事,士君子多之。大
 和三年八月卒,時年五十六。



 憑累歷諸衛大將軍,恕太子洗馬,並以廕授官,累遷至少卿監。惎累官至右龍武大將軍,沉湎酒色,恣為豪侈,積債至數千萬。其子貸回鶻錢一萬餘貫不償,為回鶻所訴,文宗怒,貶惎為定州司法參軍。



 王佖,晟之甥。雄武善騎射,自晟河西、河北出師,佖無役不從。硃泚之亂,晟攻賊於光泰門,賊鋒尚勁,佖與兵馬使李演逾苑墻血戰,敗賊前鋒,諸軍方振,論功為神策
 將。吐蕃之寇涇原,佖伏卒擊尚結贊,幾獲,由是深為吐蕃所畏。晟視佖恩寵與願、愬不殊,給與過之。晟既為張延賞媒孽罷兵權,亦不用佖為將帥,入為左衛上將軍。元和中,願、愬醖兄弟在方鎮,佖檢校工部尚書、靈州大都督府長史、朔方靈鹽節度使。先是,吐蕃欲成烏蘭橋於河需,先貯材木,朔方節度使每遣人潛載之,委於河流,終莫能成。至是,蕃人知佖貪而無謀,先厚遺之,然後人並
 忠於事君,長於應變,誠一代之賢將也。觀恆山之役,立談釋二帥之憾;涇師之亂,號哭赴奉天之危,可不為忠義乎!對白華之進軍,知平涼之必詐,沮星變之議,移渭橋之軍,可不為應變乎!解帶結孝忠之心,請婚釋延賞之怨,嫉惡有楚琳之請,懲亂行希鑒之誅,可不為明於決斷乎!而德宗皇帝聽斷
 不明,無人君之量,俾功臣困讒慝之口,奸人秉衡石之權,丁瓊之言,誠堪太息。雖齪齪刻渭橋之石,區區賜煙閣之銘,亦何心哉!作善遺慶,諸子俱才,元和平賊之功,聽、愬居其半。父子昆弟,皆以功名始終,道家所忌之談,李氏以善勝矣。



 贊曰:桓桓太師,義勇天資。運鐘禍亂,力拯顛危。愬事章武,誅蔡平齊。凌煙畫圖,父子為宜。



\end{pinyinscope}