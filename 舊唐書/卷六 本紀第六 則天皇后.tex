\article{卷六 本紀第六 則天皇后}

\begin{pinyinscope}

 則
 天皇后武氏,諱曌,並州文水人也。父士鷿,隋大業末為鷹揚府隊正。高祖行軍於汾、晉,每休止其家。義旗初起,從平京城。貞觀中,累遷工部尚書、荊州都督,封應國
 公。



 初,則天年十四時,太宗聞其美容止,召入宮,立為才人。及太宗崩,遂為尼,居感業寺。大帝於寺見之,復召入宮,拜昭儀。時皇后王氏、良娣蕭氏頻與武昭儀爭寵,互讒毀之,帝皆不納。進號宸妃。永徽六年,廢王皇后而立武宸妃為皇后。高宗稱天皇,武后亦稱天後。後素多智計,兼涉文史。帝自顯慶已後,多苦風疾,百司表奏,皆委天後詳決。自此內輔國政數十年,威勢與帝無異,當時稱為「二聖」。



 弘道元年十二月丁巳,大帝崩,皇太子顯即
 位,尊天後為皇太后。既將篡奪,是日自臨朝稱制。庚午,加授澤州刺史、韓王元嘉為太尉,豫州刺史、滕王元嬰為開府儀同三司,絳州刺史、魯王靈夔為太子太師,相州刺史、越王貞為太子太傅,安州都督、紀王慎為太子太保。元嘉等地尊望重,恐其生變,故進加虛位,以安其心。甲戌,劉仁軌為尚書左僕射,岑長倩為兵部尚書,魏玄同為黃門侍郎,並依舊知政事。劉齊賢為侍中,裴炎為中書令。
 嗣聖元年春正月甲申朔,改元。



 二月戊午,廢皇帝為廬陵王,幽於別所,仍改賜名哲。己未,立豫王輪為皇帝,令居於別殿。大赦天下,改元文明。皇太后仍臨朝稱制。庚午,廢皇太孫重照為庶人。太常卿兼豫王府長史王德真為侍中,中書侍郎、豫王府司馬劉禕之同中書門下三品。



 三月,庶人賢死於巴州。夏四月,滕王元嬰薨。改封畢王上金為澤王,葛王素節為許王。丁丑,遷廬陵王哲於均州。閏五月,禮部尚書武承嗣同中書門下三品。秋
 七月,突厥骨咄祿、元珍寇朔州,命左威衛大將軍程務挺拒之。彗星見西北方,長二丈餘,經三十三日乃滅。九月,大赦天下,改元為光宅。旗幟改從金色,飾以紫,畫以雜文。改東都為神都,又改尚書省及諸司官名。初置右肅政御史臺官員。故司空李勣孫柳州司馬徐敬業偽稱揚州司馬,殺長史陳敬之,據揚州起兵,自稱上將,以匡復為辭。冬十月,楚州司馬李崇福率所部三縣以應敬業。命左玉鈐衛大將軍李孝逸為大總管,率兵三十
 萬以討之。殺內史裴炎。丁酉,追削敬業父祖官爵,復其本姓徐氏。十二月,前中書令薛元超卒。殺左威衛大將軍程務挺。



 垂拱元年春正月,以敬業平,大赦天下,改元。劉仁軌薨。三月,遷廬陵王哲於房州。頒下親撰《垂拱格》於天下。夏四月,內史騫味道左授青州刺史。五月,秋官尚書裴居道為內史,納言王德真配流象州,冬官尚書蘇良嗣為納言。詔內外文武九品已上及百姓,咸令自舉。是夏大
 旱。二年春正月,皇太后下詔,復政於皇帝。以皇太后既非實意,乃固讓。皇太后仍依舊臨朝稱制,大赦天下。初令都督、刺史並準京官帶魚。



 三月,初置匭於朝堂,有進書言事者聽投之,由是人間善惡事多所知悉。夏四月,岑長倩為內史。六月,蘇良嗣為文昌左相,天官尚書韋待價為文昌右相,並同鳳閣鸞臺三品。右肅政御史大夫韋思謙為納言。
 三年春正月,封皇子成義為恆王,隆基為楚王,隆範為衛王,隆業為趙王。二月,韋思謙請致仕,許之。夏四月,裴居道為納言,夏官侍郎張光輔為鳳閣侍郎、同鳳閣鸞臺平章事。庚午,劉禕之賜死於家。秋八月,地官尚書魏玄同檢校納言。



 四年春二月,毀乾元殿,就其地造明堂。山東、河南甚饑乏,詔司屬卿王及善、司府卿歐陽通、冬官侍郎狄仁傑巡撫賑給。夏四月,魏王武承嗣偽造瑞石,文云:「聖母臨
 人,永昌帝業。」令雍州人唐同泰表稱獲之洛水。皇太后大悅,號其石為「寶圖」,擢授同泰游擊將軍。



 五月,皇太后加尊號曰聖母神皇。秋七月,大赦天下。改「寶圖」曰「天授聖圖」,封洛水神為顯聖,加位特進,並立廟。就水側置永昌縣。天下大酺五日。八月壬寅,博州刺史、瑯邪王沖據博州起兵,命左金吾大將軍丘神勣為行軍總管討之。庚戌,沖父豫州刺史、越王貞又舉兵於豫州,與沖相應。九月,命內史岑長倩、鳳閣侍郎張光輔、左監門大將軍
 鞠崇裕率兵討之。丙寅,斬貞及沖等,傳首神都,改姓為虺氏。曲赦博州。韓王元嘉、魯王靈夔、元嘉子黃國公譔、靈夔子左散騎常侍範陽王藹、霍王元軌及子江都王緒、故虢王元鳳子東莞公融坐與貞通謀,元嘉、靈夔自殺,元軌配流黔州,譔等伏誅,改姓虺氏。自是宗室諸王相繼誅死者,殆將盡矣。其子孫年幼者咸配流嶺外,誅其親黨數百餘家。十二月己酉,神皇拜洛水,受「天授聖圖」,是日還宮。明堂成。



 永昌元年春正月,神皇親享明堂,大赦天下,改元,大酺七日。三月,張光輔為內史,武承嗣為納言。夏四月,誅蔣王惲、道王元慶、徐王元禮、曹王明等諸子孫,徙其家屬於巂州。五月,命文昌右相韋待價為安息道大總管以討吐蕃。



 六月,令文武官五品已上各舉所知。秋七月,紀王慎被誣告謀反,載以檻車,流於巴州,改姓虺氏。韋待價坐遲留不進,士卒多饑饉而死,配流繡州。八月,左肅政御史大夫王本立同鳳閣鸞臺三品。辛巳,誅內史張
 光輔。九月,納言魏玄同賜死於家。冬十月,春官尚書範履冰、鳳閣侍郎邢文偉並同鳳閣鸞臺平章事。改羽林軍百騎為千騎。



 載初元年春正月,神皇親享明堂,大赦天下。依周制建子月為正月,改永昌元年十一月為載初元年正月,十二月為臘月,改舊正月為一月,大酺三日。神皇自以「曌」字為名,遂改詔書為制書。春一月,蘇良嗣為特進,武承嗣為文昌左相,岑長倩為文昌右相,裴居道為太子少傅,並
 依舊同鳳閣鸞臺三品。鳳閣侍郎武攸寧為納言,邢文偉為內史。秋七月,殺豫章王亶,遷其父舒王元名於和州。有沙門十人偽撰《大雲經》,表上之,盛言神皇受命之事。制頒於天下,令諸州各置大雲寺,總度僧千人。丁亥,殺隨州刺史澤王上金、舒州刺史許王素節並其子數十人。



 九月九日壬午,革唐命,改國號為周。改元為天授,大赦天下,賜酺七日。乙酉,加尊號曰聖神皇帝,降皇帝為皇嗣。丙戌,初立武氏七廟於神都。追尊神皇父贈太
 尉、太原王士鷿為孝明皇帝。兄子文昌左相承嗣為魏王,天官尚書三思為梁王,堂侄懿宗等十二人為郡王。司賓卿史務滋為納言,鳳閣侍郎宗秦客為內史。給事中傅游藝為鸞臺侍郎,仍依舊知鳳閣鸞臺平章事。令史務滋等十人分道存撫天下。改內外官所佩魚並作龜。冬十月,改並州文水縣為武興縣,依漢豐、沛例,百姓子孫相承給復。



 二年正月,親祀明堂。春三月,改唐太廟為享德廟。夏四
 月,令釋教在道法之上,僧尼處道士女寇之前。六月,命岑長倩率諸軍討吐蕃。左肅政御史大夫格輔元為地官尚書,鸞臺侍郎樂思晦並同鳳閣鸞臺平章事。秋七月,徙關內雍、同等七州戶數十萬以實洛陽。分京兆置鼎、稷、鴻、宜四州。夏官尚書歐陽通知納言事。九月,傅游藝下獄死。右羽林衛大將軍、建昌王攸寧為納言,洛州司馬狄仁傑為地官侍郎、同鳳閣鸞臺平章事。



 冬十月,制官人者咸令自舉。殺文昌左相岑長倩、納言歐陽通、
 地官尚書格輔元。



 三年正月,親祀明堂。春一月,冬官尚書楊執柔同鳳閣鸞臺平章事。三月,五天竺國並遣使朝貢。四月,大赦天下,改元為如意,禁斷天下屠殺。秋七月,大雨,洛水泛溢,漂流居人五千餘家,遣使巡問賑貸。八月,魏王承嗣為特進,建昌王攸寧為冬官尚書,楊執柔為地官尚書,並罷知政事。秋官侍郎崔元琮為鸞臺侍郎,夏官侍郎李昭德為鳳閣侍郎,檢校天官侍郎姚璹為文昌左丞,地
 官侍郎李元素為文昌右丞,並同鳳閣鸞臺平章事。九月,大赦天下,改元為長壽。改用九月為社,大酺七日。並州改置北都。冬十月,武威軍總管王孝傑大破吐蕃,復龜茲、于闐、疏勒、碎葉鎮。



 二年春一月,親享明堂。癸亥,殺皇嗣妃劉氏、竇氏。臘月,改封皇孫成器為壽春郡王,恆王成義為衡陽郡王,隆基為臨淄郡王,衛王隆範為巴陵郡王,隆業為彭城郡王。春二月,尚方監裴匪躬坐潛謁皇嗣,腰斬於都市。秋
 九月,上加金輪聖神皇帝號,大赦天下,大酺七日。辛丑,司賓卿豆盧欽望為內史,文昌右丞韋巨源同鳳閣鸞臺平章事,秋官侍郎陸元方為鸞臺侍郎、同鳳閣鸞臺平章事。



 三年春一月,親享明堂。三月,鳳閣侍郎李昭德檢校內史,鸞臺侍郎蘇味道同鳳閣鸞臺平章事。韋巨源為夏官侍郎,依舊知政事。四月,夏官尚書王孝傑同鳳閣鸞臺三品。五月,上加尊號為越古金輪聖神皇帝,大赦天
 下,改元為延載,大酺七日。秋八月,司賓少卿姚璹為納言。左肅政御史中丞楊再思為鸞臺侍郎,洛州司馬杜景儉為鳳閣侍郎,仍並同鳳閣鸞臺平章事。梁王武三思勸率諸蕃酋長奏請大征斂東都銅鐵,造天樞於端門之外,立頌以紀上之功業。九月,內史李昭德左授欽州南賓縣尉。冬十月,文昌右丞李元素為鳳閣鸞臺平章事。



 證聖元年春一月,上加尊號曰慈氏越古金輪聖神皇
 帝,大赦天下,改元,大酺七日。戊子,豆盧欽望、韋巨源、杜景儉、蘇味道、陸元方並左授趙、鄜、集、綏等州刺史。丙申夜,明堂災,至明而並從煨燼。庚子,以明堂災告廟,手詔責躬,令內外文武九品已上各上封事,極言正諫。春二月,上去慈氏越古尊號。秋九月,親祀南郊,加尊號天冊金輪聖神皇帝,大赦天下,改元為天冊萬歲,大闢罪已下及犯十惡常赦所不原者,咸赦除之,大酺九日。



 萬歲登封元年臘月甲申,上登封於嵩岳,大赦天下,改
 元,大酺九日。丁亥,禪於少室山。己丑,又制內外官三品已上通前賜爵二等,四品已下加兩階。洛州百姓給復二年,登封、告成縣三年。癸巳,至自嵩嶽。甲午,親謁太廟。春三月,重造明堂成。夏四月,親享明堂,大赦天下,改元為萬歲通天,大酺七日。以天下大旱,命文武官九品以上極言時政得失。五月,營州城傍契丹首領松漠都督李盡忠與其妻兄歸誠州刺史孫萬榮殺都督趙文翽,舉兵反,攻陷營州。盡忠自號可汗。乙丑,命鷹揚將軍曹
 仁師、右金吾大將軍張玄遇、右武威大將軍李多祚、司農少卿麻仁節等二十八將討之。秋七月,命春官尚書、梁王三思為安撫大使,納言姚璹為之副。制改李盡忠為盡滅,孫萬榮為萬斬。秋八月,張玄遇、曹仁師、麻仁節與李盡滅戰於西硤石黃麞谷,官軍敗績,玄遇、仁節並為賊所虜。九月,命右武衛大將軍、建安王攸宜為大總管以討契丹。並州長史王方慶為鸞臺侍郎,與殿中監李道廣並同鳳閣鸞臺平章事。吐蕃寇涼州,都督許欽明
 為賊所執。庚申,王方慶為鳳閣侍郎,仍依舊知政事。李盡滅死,其黨孫萬斬代領其眾。



 冬十月,孫萬斬攻陷冀州,刺史陸寶積死之。十一月,又陷瀛州屬縣。



 二年正月,親享明堂。鳳閣侍郎李元素、夏官侍郎孫元亨坐與綦連耀謀反,伏誅。原州都督府司馬婁師德為鳳閣侍郎、同鳳閣鸞臺平章事。春二月,王孝傑、蘇宏暉等率兵十八萬與孫萬斬戰於硤石谷,王師敗績,孝傑沒於陣,宏暉棄甲而遁。夏四月,鑄九鼎成,置於明堂之
 庭,前益州大都督府長史王及善為內史。五月,命右金吾大將軍、河內王懿宗為大總管,右肅政御史大夫婁師德為副大總管,右武威衛大將軍沙吒忠義為前軍總管,率兵二十萬以討孫萬斬。



 六月,內史李昭德、司僕少卿來俊臣以罪伏誅。孫萬斬為其家奴所殺,餘黨大潰。魏王承嗣、梁王三思並同鳳閣鸞臺三品。秋八月,納言姚璹為益州大都督府長史。九月,以契丹李盡滅等平,大赦天下,改元為神功,大酺七日。婁師德為納言。冬
 十月,前幽州都督狄仁傑為鸞臺侍郎,司刑卿杜景儉為鳳閣侍郎,並同鳳閣鸞臺平章事。聖歷元年正月,親享明堂,大赦天下,改元,大酺九日。春三月,召廬陵王哲於房州。夏五月,禁天下屠殺。突厥默啜上言,有女請和親。秋七月,令淮陽王武延秀往突厥,納默啜女為妃。遣右豹韜衛大將軍閻知微攝春官尚書,赴虜庭。



 八月,突厥默啜以延秀非唐室諸王,乃囚於別所,率眾與閻知微入寇媯、檀等州。命司屬卿高平王
 重規、右武威衛大將軍沙吒忠義、幽州都督張仁亶、右羽林衛大將軍李多祚等率兵二十萬逆擊,乃放延秀還。己丑,默啜攻陷定州,刺史孫彥高死之,焚燒百姓廬舍,遇害者數千人。魏王承嗣卒。庚子,梁王三思為內史,狄仁傑為納言。九月,建昌王攸寧同鳳閣鸞臺平章事。默啜攻陷趙州,刺史高睿遇害。丙子,廬陵王哲為皇太子,令依舊名顯,大赦天下,大酺五日。令納言狄仁傑為河北道行軍元帥。辛巳,皇太子謁太廟。天官侍郎蘇味
 道鳳閣侍郎、同鳳閣鸞臺平章事。癸未,默啜盡殺所掠趙、定州男女萬餘人,從五回道而去,所至殘害,不可勝紀。



 冬十月,夏官侍郎姚元崇、麟臺少監李嶠並同鳳閣鸞臺平章事。是月,閻知微自突厥叛歸,族誅之。



 二年春二月,封皇嗣旦為相王。初為寵臣張易之及其弟昌宗置控鶴府官員,尋改為奉宸府,班在御史大夫下。左肅政御史中丞魏元忠為鳳閣侍郎,吉頊為天官侍郎,並同鳳閣鸞臺平章事。戊子,幸嵩山,過王子晉廟。
 丙申,幸緱山。丁酉,至自嵩山。



 夏四月,吐蕃大論贊婆來奔。秋七月,上以春秋高,慮皇太子、相王與梁王武三思、定王武攸寧等不協,令立誓文於明堂。八月,王及善為文昌左相,豆盧欽望為文昌右相,仍並同鳳閣鸞臺三品。冬十月乙亥,幸福昌縣。王及善薨。



 三年正月戊寅,梁王三思為特進,天官侍郎吉頊配流嶺表。臘月辛巳,封皇太子男重潤為邵王。狄仁傑為內史。戊寅,幸汝州之溫湯。甲戌,至自溫湯、造三陽宮於嵩
 山。春三月,李嶠為鸞臺侍郎,知政事如故。



 夏四月戊申,幸三陽宮。五月癸丑,上以所疾康復,大赦天下,改元為久視,停金輪等尊號,大酺五日。六月,魏元忠為左肅政御史大夫,仍舊知政事。是夏大旱。秋七月,至自三陽宮。天官侍郎張錫為鳳閣侍郎、同鳳閣鸞臺平章事;其甥鳳閣鸞臺平章事李嶠為成均祭酒,罷知政事。壬寅,制曰:「隋尚書令楊素,昔在本朝,早荷殊遇。稟兇邪之德,有諂佞之才,惑亂君上,離間骨肉。搖動塚嫡,寧唯握蠱之
 禍;誘扇後主,卒成請蹯之釁。隋室喪亡,蓋惟多僻,究其萌兆,職此之由。生為不忠之人,死為不義之鬼,身雖幸免,子竟族誅。斯則奸逆之謀,是為庭訓;險薄之行,遂成門風。刑戮雖加,枝胤仍在,何得肩隨近侍,齒列朝行?朕接統百王,恭臨四海,上嘉賢佐,下惡賊臣。常欲從容於萬機之餘,褒貶於千載之外,況年代未遠,耳目所存者乎!其楊素及兄弟子孫已下,並不得令任京官及侍衛。」九月,內史狄仁傑卒。冬十月甲寅,復舊正朔,改一月為
 正月,仍以為歲首,正月依舊為十一月,大赦天下。韋巨源為地官尚書,文昌左丞韋安石為鸞臺侍郎、同鳳閣鸞臺平章事。丁卯,幸新安,曲赦其縣。壬申,至自新安。十二月,開屠禁,諸祠祭令依舊用牲牢。



 大足元年春正月,制改元。二月,鸞臺侍郎李懷遠同鳳閣鸞臺平章事。三月,姚元崇為鳳閣侍郎,依舊知政事。丙申,鳳閣侍郎張錫坐贓配循州。夏五月,幸三陽宮。命左肅政御史大夫魏元忠為總管以備突厥。天官侍郎
 顧琮同鳳閣鸞臺平章事。六月,夏官侍郎李迥秀同鳳閣鸞臺平章事。辛未,曲赦告成縣。秋七月甲戌,至自三陽宮。九月,邵王重潤為易之讒構,令自死。



 冬十月,幸京師,大赦天下,改元為長安。



 二年春正月,突厥寇鹽、夏等州,殺掠人吏。秋九月乙丑,日有蝕之,不盡如鉤,京師及四方見之。冬十月,日本國遣使貢方物。十一月,相王旦為司徒。戊子,親祀南郊,大赦天下。



 三年春三月壬戌,日有蝕之。夏四月庚子,相王旦表讓司徒,許之。改文昌臺為中臺。李嶠知納言事。六月,寧州雨,山水暴漲,漂流二千餘家,溺死者千餘人。秋七月,殺右金吾大將軍唐休璟。秋九月,正諫大夫硃敬則同鳳閣鸞臺平章事。戊申,相王旦為雍州牧。是月,御史大夫兼知政事、太子右庶子魏元忠為張昌宗所譖,左授端州高要尉。京師大雨雹,人畜有凍死者。冬十月丙寅,駕還神都。乙酉,至自京師。



 四年春正月,造興泰宮於壽安縣之萬安山。天官侍郎韋嗣立為鳳閣侍郎、同鳳閣鸞臺平章事。硃敬則請致仕,許之。三月,進封平恩郡王重福為譙王,夏官侍郎宗楚客同鳳閣鸞臺平章事。夏四月,韋安石知納言事,李嶠知內史事。丙子,幸興泰宮六月,天官侍郎崔玄暐同鳳閣鸞臺平章事;李嶠為國子祭酒,知政事如故。七月丙戌,楊再思為內史。甲午,至自興泰宮。宗楚客左授原州都督。



 八月,姚元崇為司僕卿,知政事;韋安石檢校揚
 州大都督府長史。冬十月,秋官侍郎張柬之同鳳閣鸞臺平章事。十一月,李嶠為地官尚書,張柬之為鳳閣鸞臺平章事。自九月至於是,日夜陰晦,大雨雪,都中人有饑凍死者,令官司開倉賑給。



 神龍元年春正月,大赦,改元。上不豫,制自文明元年已後得罪人,除揚、豫、博三州及諸逆魁首,咸赦除之。癸亥,麟臺監張易之與弟司僕卿昌宗反,皇太子率左右羽林軍桓彥範、敬暉等,以羽林兵入禁中誅之。甲辰,皇太
 子監國,總統萬機,大赦天下。是日,上傳皇帝位於皇太子,徙居上陽宮。戊申,皇帝上尊號曰則天大聖皇帝。冬十一月壬寅,則天將大漸,遺制祔廟、歸陵,令去帝號,稱則天大聖皇后;其王、蕭二家及褚遂良、韓瑗等子孫親屬當時緣累者,咸令復業。是日,崩於上陽宮之仙居殿,年八十三,謚曰則天大聖皇后。二年五月庚申,祔葬於乾陵。睿宗即位,詔依上元年故事,號為天後,未幾,追尊為大聖天後,改號為則天皇太后。太后嘗召文學之士周
 思茂、範履冰、衛敬業,令撰《玄覽》及《古今內範》各百卷,《青宮紀要》、《少陽政範》各三十卷,《維城典訓》、《鳳樓新誡》、《孝子列女傳》各二十卷,《內軌要略》、《樂書要錄》各十卷,《百僚新誡》、《兆人本業》各五卷,《臣範》兩卷,《垂拱格》四卷,並文集一百二十卷,藏於秘閣。



 史臣曰:治亂,時也,存亡,勢也。使桀、紂在上,雖十堯不能治;使堯、舜在上,雖十桀不能亂;使懦夫女子乘時得勢,亦足坐制群生之命,肆行不義之威。觀夫武氏稱制之
 年,英才接軫,靡不痛心於家索,扼腕於朝危,竟不能報先帝之恩,衛吾君之子。俄至無辜被陷,引頸就誅,天地為籠,去將安所?悲夫!昔掩鼻之讒,古稱其毒;人彘之酷,世以為冤。武后奪嫡之謀也,振喉絕襁褓之兒,菹醢碎椒塗之骨,其不道也甚矣,亦奸人妒婦之恆態也。然猶泛延讜議,時禮正人。初雖牝雞司晨,終能復子明闢,飛語辯元忠之罪,善言慰仁傑之心,尊時憲而抑幸臣,聽忠言而誅酷吏。有旨哉,有旨哉!



 贊曰:龍漦易貌,丙殿昌儲。胡為穹昊,生此夔魖?奪攘神器,穢褻皇居。窮妖白首,降鑒何如。



\end{pinyinscope}