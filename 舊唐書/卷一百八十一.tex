\article{卷一百八十一}

\begin{pinyinscope}

 ○崔慎由弟安潛伯父能能子彥曾慎由子胤崔珙兄琯弟瑨璵球璵子淡淡子遠盧鈞裴休楊收兄發弟嚴子鉅鏻嚴子涉注韋保衡路巖夏侯孜劉瞻劉彖曹確畢諴杜審權子讓能彥林弘徽
 劉鄴豆盧彖



 崔慎由,字敬止,清河武城人。高祖融,位終國子司業,謚曰文,自有傳。曾祖翹,位終禮部尚書、東都留守。祖異,位終渠州刺史。



 父從,少孤貧。寓居太原,與仲兄能同隱山林,苦心力學。屬歲兵荒,至於絕食。弟兄採梠拾橡實,飲水棲衡;而講誦不輟。怡然終日,不出山巖,如是者十年。貞元初,進士登第,釋褐山南西道推官,府公嚴震,待以殊禮。以父優免。弟兄廬於父墓,手植松柏。免喪,不應闢
 命。久之,西川節度使韋皋開西南夷,置兩路運糧使,奏從掌西山運務,後權知邛州事。及皋薨,副使劉闢阻命,欲並東川,以謀告從。從以書諭闢,闢怒,出兵攻之,從嬰城拒守,卒不從之。高崇文平蜀,從事坐累多伏法,惟從以拒闢免。盧坦在宣州,闢為團練觀察副使。



 元和初入朝,累遷吏部員外郎。九年,裴度為中丞,奏從為侍御史知雜,守右司郎中。度作相,用從自代為中丞。



 從氣貌孤峻,正色立朝,彈奏不避權幸。事關臺閣或付仗內者,必
 抗章論列,請歸有司。選闢御史,必先質重貞退者。改給事中,數月,出為陜州大都督府長史、陜虢團練觀察使、兼御史中丞,賜紫金魚袋。入為尚書右丞。



 淄青賊平,鎮州王承宗懼,上章請割德、棣二州自贖,又令二子入侍。憲宗選使臣宣諭,以從中選。議者以承宗罪惡貫盈,每多奸譎,入朝二子,必非血胤,人頗憂之。從次魏州,田弘正以路由寇境,欲以五百騎援之,從辭之。以童奴十數騎,徑至鎮州。於鞠場宣敕,三軍大集。從諭以逆順,辭情
 慷慨,軍士感動,承宗泣下,禮貌益恭,遂按德、棣戶口符印而還。



 其年八月,出為興元尹、御史大夫、山南西道節度觀察等使。監軍使知上意欲大用之,每為中貴傳達意旨,欲其賂遺,從終不答。



 穆宗即位,召拜尚書左丞。長慶二年,檢校禮部尚書、鄜州刺史、鄜坊丹延節度等使。鄜畤內接畿甸,神策軍鎮相望,逾禁犯法,累政不能制。而從撫遏舉奏,軍士惕然。黨項羌有以羊馬來市者,必先遺帥守,從皆不受,撫諭遣之。群羌不敢為盜。四年,入
 為吏部侍郎,尋改太常卿。寶歷二年,檢校吏部尚書,充東都留守。



 太和三年,入為戶部尚書。李宗閔秉政,以從與裴度、李德裕厚善,惡之。改檢校尚書右僕射、太子賓客,東都分司。從請告百日,罷官,物論咎執政。宗閔懼,四年三月,召拜檢校左僕射,兼揚州大都督府長史、御史大夫,充淮南節度副大使,知節度事。揚府舊有貨曲之利,資產奴婢交易者,皆有貫率,羊有口算,每歲收利以給用,從悉除之。舊制,官吏祿俸有布帛加估之給,節度
 使獨不在此例。從至,一例估折給之。六年十月,卒於鎮,贈司空,謚曰貞。



 從少以貞晦恭讓自處,不交權利,忠厚方嚴,正人多所推仰。階品合立門戟,終不之請。四為大鎮,家無妓樂,士友多之。



 慎由,太和初擢進士第,又登賢良方正制科。聰敏強記,宇量端厚,有父風。釋褐諸侯府。大中初入朝,為右拾遺、員外郎、知制誥,正拜舍人,召充翰林學士、戶部侍郎。再歷方鎮,入朝為工部尚書。十年,以本官同平章事,兼集賢殿大學士,轉監修國史、上柱
 國,加太中大夫、兼禮部尚書。



 初,慎由與蕭鄴同在翰林,情不相洽。及慎由作相,罷鄴學士。俄而鄴自判度支為平章事,恩顧甚隆。鄴引劉彖同知政事。十二年二月,詔曰:「太中大夫、中書侍郎、兼禮部尚書、同中書門下平章事、監修國史、上柱國、賜紫金魚袋崔慎由,繼美德門,承家貴位,搢紳偉望,禮樂上流。挺松筠之貞姿,服蘭蓀之懿行。自居名器,累歷清華。禁林才擅於多能,綸閣詞推於巨麗。物情愈茂,延譽甚高,再列二卿之崇,亟闡六條
 之化。爰加獎任,益委重難。屢啟嘉謨,俄參大柄,而周涉寒暑,備見器能。道已著於始終,思豈殊於中外!可檢校禮部尚書、梓州刺史、兼御史大夫、劍南東川節度使。」



 咸通初,改為華州刺史、潼關防禦、鎮國軍等使,加檢校司空、河中尹、河中晉絳節度使。入為吏部尚書。移疾請老,拜太子太保,分司東都,卒。



 子胤。弟安潛。安潛,字進之,大中三年,登進士第。咸通中,累歷清顯,出為許州刺史、忠武軍節度觀察等使。乾符中,遷成都尹、劍南西川節度
 等使。黃巢之亂,從僖宗幸蜀。王鐸為諸道行營都統,奏安潛為副。收復兩京,以功累加至檢校侍中。龍紀初,青州王敬武卒,以安潛代。敬武子師範拒命,安潛赴鎮。至棣州,刺史張蟾出州兵攻青州,為師範所敗,朝廷竟授之節鉞。安潛還京師,累加太子太傅。卒,贈太師,謚曰貞孝。



 子柅、艤。柅,景福中為起居郎。艤為右拾遺。柅累官至尚書。



 從兄能,少勵志苦學,累闢使府。元和初,為蜀州刺史。六年,轉黔中觀察使。坐為南蠻所攻,陷郡邑,貶永州
 刺史。穆宗即位,弟從居顯列,召拜將作監。長慶四年九月,出為廣州刺史、御史大夫、嶺南節度使,卒。



 子彥曾,有幹局。大中末,歷三郡刺史。咸通初,累遷太僕卿。七年,檢校左散騎常侍、徐州刺史、御史大夫,充武寧軍節度使。



 彥曾通於法律,性嚴急。以徐軍驕,命彥曾治之,長於撫養,而短於軍政。用親吏尹戡、徐行儉當要職。二人貪猥,不恤軍旅,士卒怨之。先是,六年,南蠻寇五管,陷交址,詔徐州節度使孟球召募二千人赴援,分八百人戍桂州。
 舊三年一代,至是戍卒求代。尹戡以軍帑匱乏,難以發兵,且留舊戍一年。其戍卒家人飛書桂林。戍卒怒,牙官許佶、趙可立、王幼誠、劉景、傅寂、張實、王弘立、孟敬文、姚周等九人,殺都頭王仲甫,立糧料判官龐勛為都將。群伍突入監軍院取兵甲。乃剽湘潭、衡山兩縣,虜其丁壯。乃擅回戈,沿江自浙西入淮南界,由濁河達泗口。其眾千餘人,每將過郡縣,先令倡卒弄傀儡以觀人情,慮其邀擊。既離泗口,彥曾令押牙田厚簡慰喻,又令都虞候
 元密伏兵任山館。龐勛遣吏送狀啟訴,以軍士思歸,勢不能遏,願至府外解甲歸兵。便還家,彥曾怒,誅之。勛等擁眾攻宿州,陷之。出官帑召募。翌日,得兵二千人,乃虜奪舟船五千餘艘。步卒在船,騎軍夾岸,鼓噪而進。元密發伏邀之,為賊所敗。時亡命者歸賊如市,彥曾驅城中丁男城守。九年九月十四日,賊逼徐州。十五日後,每旦大霧不開。十六日,彥曾並誅逆卒家口。十七日,昏霧尤甚,賊四面斬關而入。龐勛先謁漢高祖廟,便入牙城。監
 軍張道謹相見,不交一言,乃止大彭館。收尹戡、徐行儉及判官焦璐、李棁、崔蘊、溫廷皓、韋廷義,並殺之。翌日,賊將趙可立害彥曾,龐勛自稱武寧軍節度使。



 慎由子胤。胤,字昌遐,乾寧二年登進士第。王重榮鎮河中,闢為從事。入朝,累遷考功、吏部二員外郎,轉郎中、給事中、中書舍人。大順中,歷兵部、吏部二侍郎,尋以本官同平章事。時王室多故,南北司爭權,咸樹朋黨,外結籓帥。胤長於陰計,巧於附麗;外示凝重而心險躁。自李茂貞、王行瑜
 怙亂,兵勢不遜,杜讓能、韋昭度繼遭誅戮,而宰臣崔昭緯深結行瑜以自固;而待胤以宗人之分,屢加薦用。累遷中書侍郎、判戶部事。昭宗出幸石門,胤與同列徐彥若、王摶等從。車駕還宮,加禮部尚書,並賜號「扶危匡國致理功臣」。



 三年,李茂貞犯京師,扈昭宗幸華州。帝復雪杜讓能、韋昭度、李磎之枉;懲昭緯之前慝,罷胤政事,檢校兵部尚書、廣州刺史、嶺南東道節度等使。時硃全忠方霸於關東,胤密致書全忠求援。全忠上疏理胤之功,
 不可離輔弼之地。胤已至湖南,復召拜平章事。胤既獲汴州之援,頗弄威權。恨徐彥若、王摶發昭緯前事,深排抑之。俄出彥為南海節度。又摭王摶交結敕使,同危宗社,令全忠上疏論之。光化中,貶摶溪州司馬,賜死於藍田驛。誅中尉宋道弼、景務修。自是朝廷權政,皆歸於己,兼領三司使務。宦官側目,不勝其忿。



 及劉季述幽昭宗於東內,以德王監國。季述畏全忠之強,不敢殺胤,但罷知政事,落使務,守本官而已。胤復致書於全忠,請出
 師反正。故全忠令大將張存敬急攻晉絳河中。胤以天子幽囚,諸侯觀釁,有神策軍巡使孫德昭者,頗怒季述之廢立,胤伺知之,令判官石戩與德昭游,伺其深意。每酒酣,德昭泣下,戩知其誠,乃與之謀曰:「今中外大臣,自廢立已來,無不含怒。至於軍旅,亦懷憤惋。今謀反者,獨季述、仲先耳。足下誅此二豎,復帝寶位,垂名萬代,今正其時。持疑不斷,則功落他人之手也!」德昭謝曰:「予軍吏耳,社稷大計,不敢自專。如相公委使,不敢避也。」胤乃割
 衣帶,手書以通其意。十二月晦,德昭伏兵誅季述。昭宗反正,胤進位司空,復知政事,兼領度支、鹽鐵、三司等使。



 明年夏,硃全忠攻陷河中、晉絳,進兵至同華。中尉韓全誨以胤交結全忠,慮汴軍逼京師,請罷知政事,落使務。其年冬,全誨挾帝幸鳳翔。胤怨帝廢黜,不扈從,遣使告全忠,請於岐陽迎駕,令太子太師盧知猷率百官迎全忠入京師。初,全忠至華州,遣掌書記裴鑄人奏鳳翔,言欲以兵士迎駕。及入京師,又上表曰:



 臣獨兼四鎮,迨事
 兩朝,分數千里之封疆,受二十年之恩渥。微同物類,猶解感知,忝齒人倫,寧忘報效?臣昨將兵士,奔赴闕庭,尋過京畿,遠迎車駕。初因幕吏,面奉德音;尋有宰臣,頻飛密札。或以京都紛擾,委制置於中朝;或以鑾輅播遷,俾奉迎於近甸。臣是以遠離籓鎮,不憚疲勞。昨奉詔書,兼宣口敕,令臣速抽兵士,且歸本籓,仍遣百官,俾赴行在。睹綸言於鳳紙,若面丹墀;認御札於龍衣,如親翠蓋。然知從來書詔,出自宰臣,每降宣傳,皆非聖旨。致臣誤將
 師旅,遽入關畿,比令迎駕之行,翻掛脅君之過。臣今見與茂貞要約,釋兩地猜嫌,早致萬乘歸京,以副八紘懇望。其宰臣百官已下,非臣輒有阻留,伏乞詔赴行朝,以備還駕。



 昭宗得全忠表,怒胤尤甚。是月二十六日詔曰:



 食君之祿,合務於盡忠;秉國之鈞,宜思於致理。其有疊膺異渥,繼執重權,遽萌狂悖之心,忽構傾危之計,人知不可,天固難容。扶危定亂致理功臣、開府儀同三司、守司空,兼門下侍郎、同平章事,充太清宮使、弘文館大學
 士、延資庫使、諸道鹽鐵轉運等使、判度支、上柱國、魏國公、食邑五千戶崔胤,奕葉公臺,蟬聯珪組。冠歲名升於甲乙,壯年位列於公卿,趣向有聞,行藏可尚。朕採於群議,詢彼輿情,有冀小康,遂登大用。殊不知漏卮難滿,小器易盈,曾無報國之心,但作危邦之計,四居極位,一無可稱。豈有都城,合聚兵甲,暗養死士,將亂國經。聚貔武以保其一坊,致刁斗遠連於右輔。始則將京兆府官錢委元規召卒,後則用度支使榷利令陳班聚兵;事去公
 朝,權歸私室。百闢休戚,由其顧眄之間;四方是非,系彼指呼之際。令狐渙奸纖有素,操守無堪,用作腹心,共張聲勢。遂令濫居深密,日在禁闈,罔惑朕躬,偽行書詔。致茲播越,職爾之由。豈有權重位崇,恩深獎厚,曾無惕厲,轉恣睢盱,顯構外兵,將圖不軌!



 朕以庶士流散,兵革繁多,遂命宰臣,與之商議。五降內使,一貢表章,堅臥不來,拒召如此。況又拘留庶吏,廢闕晨趨。人既奔驚,朕須巡幸。果見兵纏輦轂,火照宮闈,煙塵漲天,干戈匝野。致朕
 奔迫,及於岐陽。翠輦未安,鐵騎旋至,圍逼行在,焚燒屋廬。睹此阽危,咎將誰執?近者全忠章表,兼遣幕吏敷陳,言宰臣繼飛密緘,促其兵士西上,靜詳構扇,孰測苞藏,無功及人,為國生事。於戲!君人之道,委之宰衡,庶務殷繁,豈能親理?盡將機事,付爾主張,負我何多,構亂至此!仍存大體,不謂無恩。可責授朝散大夫,守工部尚書。



 初,天復反正之後,宦官尤畏胤,事無大小咸稟之。每內殿奏對,夜則繼之以燭。常說昭宗請盡誅內官,但以宮人
 掌內司事。中尉韓全誨、張弘彥、袁易簡等伺知之,於帝前求哀請命,乃詔胤密事進囊封,勿更口奏。宦官無由知其謀,乃求知書美婦人進內,以偵陰事。由是胤謀頗洩。宦官每相聚流涕,愈不自安。故全誨等為劫幸之謀,由胤忌嫉之太過也。



 及全忠攻鳳翔,胤寓居華州,為全忠畫圖王之策。天復二年,全忠自岐下還河中,胤迎謁於渭橋,捧卮上壽,持板為全忠唱歌,仍自撰歌辭,贊其功業。三年,李茂貞殺韓全誨等,與全忠通和,昭宗急詔
 徵胤赴行在。凡四降詔,三賜硃書御札,稱病不赴。及帝出鳳翔,胤乃迎於中路,即日降制,復舊官,知政事,進位司徒,兼判六軍諸衛事。仍詔移家入左軍,賜帳幄器用十車。胤奏京兆尹鄭元規為六軍副使。胤與全忠奏罷左右神策、內諸司等使及諸道監軍、副監、小使。內官三百餘人,同日斬之於內侍省。諸道監軍,隨處斬首以聞。



 昭宗初幸鳳翔,命盧光啟、韋貽範、蘇檢等作相;及還京,胤皆貶斥之。又貶陸扆為沂王傅,王溥太子賓客,學士
 薛貽矩夔州司戶,韓人屋濮州司戶,姚洎景王府咨議。應從幸群官,貶逐者三十餘人。唯用裴贄為相,以其孤立易制也。內官既盡屠戮,諸使悉罷,天子宣傳詔命,惟令宮人寵顏等宣事。而欺君蠹國,所不忍聞。胤所悅者闒茸下輩,所惡者正人君子。人人悚懼,朝不保夕。



 其年十月,全忠子友倫宿衛京師,因擊鞠墜馬而卒。全忠愛之,殺會鞠者十餘人,而疑胤陰謀,由是怒胤。初,天子還宮,全忠東歸,胤以事權在己,慮全忠急於篡代,乃與鄭元
 規謀招致兵甲,以捍茂貞為辭。全忠知其意,從之。胤毀城外木浮圖,取銅鐵為兵仗。全忠令汴州軍人入關應募者數百人。及友倫死,全忠怒,遣其子宿衛軍使友諒誅胤,而應募者突然而出。四年正月初,貶太子賓客,尋為汴軍所殺。



 胤傾險樂禍,外示寬宏。初拜平章事,其季父安潛謂所親曰:「吾父兄刻苦樹立門戶,一旦終當為緇郎所壞。」果如其言。胤累加至侍中,封魏國公。初,硃全忠雖竊有河南方鎮,憚河朔、河東,未萌問鼎之志。及得
 胤為鄉導,乃電擊潼關,始謀移國。自古與盜合從,覆亡宗社,無如胤之甚也。子有鄰。



 崔珙,博陵安平人。祖懿。父頲,貞元初進士登第。元和初累官至少府監。四年,出為同州刺史,卒。頲有子八人,皆至達官,時人比漢之荀氏,號曰「八龍」。



 長曰琯,貞元十八年進士擢第。又制策登科,釋褐諸侯府,入朝為尚書郎。太和初,累遷給事中,宣慰幽州稱旨。俄而興元兵亂,殺李絳,命琯平亂褒中,三軍寂然從命。使還,改工部侍郎。
 四年冬,拜京兆尹。五年四月,改尚書右丞。六年十二月,出為江陵尹、御史大夫、荊南節度使。八年,入為兵部侍郎,轉吏部,權判左丞事。開成二年,真拜左丞。時弟珙為京兆尹,兄弟並居顯列。以本官權判兵部西銓、吏部東銓事。三年,檢校戶部尚書,判東都尚書省事、東都留守、東畿汝都防禦等使。會昌中,遷銀青光祿大夫、檢校吏部尚書、興元尹,充山南西道節度使。以弟珙罷相貶官,琯亦罷鎮歸東都。五年卒。詔曰:



 孔氏以顏、冉之行,首於
 四科;漢代以荀、陳之門,方之「八凱」。乃睠時哲,得茲令名,用舉飾終之恩,以抒殲良之嘆。故山南西道節度使崔琯,誠明履正,粹密鄰幾,有子政之精忠,得公綽之不欲。禮樂二事,以為身文;仁義五常,自成家範。往以茂器,列於大僚。屬賢相受誣,廟堂議法,由長孺之道,以估正人;微京兆之言,豈聞非罪?既是魏其之直,益彰王鳳之邪。莊色於朝,群公聳視;讜詞不撓,淑問攸歸。歷踐名籓,皆留遺愛。居常慎獨,清則畏知。爰自青衿,迄於白首,厲翼
 之志,始終不渝。未陟臺階,實辜公論;追榮左相,式示優崇。可贈尚書左僕射。



 珙,琯之母弟也。以書判拔萃高等,累佐使府。性威重,尤精吏術。太和初,累官泗州刺史,入為太府卿。七年正月,拜廣州刺史、嶺南節度使。延英中謝,帝問以撫理南海之宜。珙奏對明辯,帝深嘉之。時高瑀鎮徐州,承智興之後,軍驕難制,軍士數犯法,上欲擇威望之帥以臨之,久難其才。會珙言事慷慨,謂宰臣曰:「崔珙言事,神氣精爽,此可以臨徐人。」即以王茂元代珙
 鎮廣南,授珙兼檢校工部尚書、徐州刺史、兼御史大夫,充武寧軍節度、徐泗濠觀察使。



 開成初,就加檢校兵部尚書。二年,檢校吏部尚書、右金吾大將軍,充街使。六月,遷京兆尹。是歲,京畿旱,珙奏滻水入內者,十分量減九分,賜貧民溉田,從之。三年正月,盜發親仁里,欲殺宰相李石。其賊出於禁軍,珙坐捕盜不獲,罰俸料。會昌初,李德裕用事,與珙親厚,累遷戶部侍郎,充諸道鹽鐵轉運等使。尋以本官同中書門下平章事,累兼刑部尚書、門
 下侍郎,進階銀青光祿大夫,兼尚書左僕射。素與崔鉉不葉,及李讓夷引鉉輔政,代珙領使務,乃掎摭珙領使日妄破宋滑院鹽鐵錢九十萬貫文,又言珙嘗保護劉從諫,坐貶澧州刺,再貶恩州司馬。宣宗即位,以赦召還,為太子賓客,出為鳳翔節度使。



 三年,崔鉉復知政事,珙辭疾請罷。制曰:「將相大臣,與國同體,誠欲自便,豈宜不從?茍非其時,涉於避事。前鳳翔隴州節度觀察處置等使、光祿大夫、檢校尚書右僕射、兼鳳翔尹、御史大夫、
 上柱國、安平郡開國公、食邑二千戶崔珙,早以器能,周歷顯重。行己每稱其友悌,在公亦竭其精忠。自負譴前朝,遠移南徼,及我嗣守,頗聞嘉名。由是剖竹近關,揚旍右輔,為國垣翰,適資謀猷。近者犬戎輸誠,歸我故地,下議納款,且籌開疆。宜其率先啟行,副此寵待。忽覽退閑之請,頗乖毗倚之誠。陳力之方,豈無其道?匪躬之故,或異於是。以其故老,特為優容,俾居青宮之輔,仍從分洛之命。君臣禮分,予無愧焉。可太子少師,分司東都。」未幾,
 卒。



 子涓,大中四年進士擢第。



 珙弟瑨、璪、璵、球、珦。



 瑨以書判拔萃,開成中,累遷至刑部郎中。會昌中,歷三郡刺史,位終方鎮。



 璪,開成初,為吏部郎中,轉給事中。會昌初,出為陜虢觀察使,遷河南尹,入為御史中丞,轉吏部侍郎。大中初,改兵部侍郎,充諸道鹽鐵轉運使。崔鉉再輔政,罷璪使務,檢校兵部尚書,兼河中尹、御史大夫,充河中晉絳磁隰等州節度觀察使。七年,入為左丞,再遷刑部尚書。子滔,大中初登進士第。



 璵,字朗士,長慶初進士擢
 第,又制策登科。開成末,累遷至禮部員外郎。會昌初,以考功郎中知制誥,拜中書舍人。大中五年,遷禮部侍郎。六年,選士,時謂得才。七年,權知戶部侍郎,進封博陵子,食邑五百戶,轉兵部侍郎。子淡。



 淡,大中十三年,登進士第,累遷禮部員外郎,位終吏部侍郎。淡子遠。



 遠,龍紀元年,登進士第。大順初,以員外郎知制誥,召充翰林學士,正拜中書舍人。乾寧三年,轉戶部侍郎、博陵縣男、食邑三百戶,轉兵部侍郎承旨。尋以本官同平章事,遷中書
 侍郎,兼吏部尚書。天祐初,從昭宗東遷洛陽。罷相,守右僕射。二年,為柳璨希、硃全忠旨,累貶白州長史。行至滑州,被害於白馬驛。



 遠文才清麗,風神峻整,人皆慕其為人,當時目為「釘座梨」,言席上之珍也。



 球,字叔休,寶歷二年登進士第。會昌中,為鳳翔節度判官,入朝為尚書郎。子瀆。瀆,大中末亦進士登第。



 崔氏咸通、乾符間,昆仲子弟紆組拖紳,歷臺閣、錢籓岳者二十餘人。大中以來盛族,時推甲等。



 盧鈞,字子和,本範陽人。祖炅,父繼。鈞,元和四年進士擢第,又書判拔萃,調補校書郎,累佐諸侯府。太和五年,遷左補闕。與同職理宋申錫之枉,由是知名。歷尚書郎,出為常州刺史。九年,拜給事中。開成元年,出為華州刺史、潼關防禦、鎮國軍等使。



 其年冬,代李從易為廣州刺史、御史大夫、嶺南節度使。南海有蠻舶之利,珍貨輻湊。舊帥作法興利以致富,凡為南海者,靡不梱載而還。鈞性仁恕,為政廉潔,請監軍領市舶使,己一不干預。自貞元
 已來,衣冠得罪流放嶺表者,因而物故,子孫貧悴,雖遇赦不能自還。凡在封境者,鈞減俸錢為營槥櫝。其家疾病死喪,則為之醫藥殯殮,孤兒稚女,為之婚嫁,凡數百家。由是山越之俗,服其德義,令不嚴而人化。三年將代,華蠻數千人詣闕請立生祠,銘功頌德。先是土人與蠻獠雜居,婚娶相通,吏或撓之,相誘為亂。鈞至,立法,俾華蠻異處,婚娶不通,蠻人不得立田宅;由是徼外肅清,而不相犯。



 會昌初,遷襄州刺史、山南東道節度使。四年,誅
 劉稹,以鈞檢校兵部尚書,兼潞州大都督府長史、昭義節度、澤潞邢洺磁觀察等使。是冬,詔鈞出潞軍五千戍代北。鈞升城門餞送,其家設幄觀之。潞卒素驕,因與家人訣別,乘醉倒戈攻城門。監軍以州兵拒之,至晚撫勞方定。詔鈞入朝,拜戶部侍郎、判度支,遷戶部尚書。



 大中初,檢校尚書右僕射、汴州刺史、御史大夫、宣武軍節度、宋亳汴潁觀察等使,就加檢校司空。四年,入為太子少師,進位上柱國、範陽郡開國公、食邑二千戶。六年,復檢
 校司空、太原尹、北都留守、河東節度使。



 九年,詔曰:「河東軍節度使盧鈞,長才博達,敏識宏深。藹山河之靈,抱瑚璉之器。多能不耀,用晦而彰。由嶺表而至太原,五換節鉞,仁聲載路,公論彌高。籓垣之和氣不衰,臺閣之清風常在,宜升揆路,以表群僚。可尚書左僕射。」



 鈞踐歷中外,事功益茂,後輩子弟,多至臺司。至是急徵,謂當輔弼,雖居端揆,心殊失望。常移病不視事,與親舊游城南別墅,或累日一歸。宰臣令狐綯惡之,乃罷僕射,仍加檢校司
 空,守太子太師。物議以鈞長者,罪綯弄權。綯懼。



 十一年九月,以鈞檢校司徒、同中書門下平章事、興元尹,充山南西道節度使,入為太子太師,卒。



 裴休,字公美,河內濟源人也。祖宣,父肅。肅,貞元中自常州刺史兼御史中丞、越州刺史、浙東團練觀察等使。時山賊慄鍠誘山越為亂,陷浙東郡縣。肅召州兵討平之,因紀其事,號《平戎記》,上之。德宗嘉賞。



 肅生三子,儔、休、俅,皆登進士第。



 休志操堅正。童齔時,兄弟同學於濟源別
 墅。休經年不出墅門,晝講經籍,夜課詩賦。虞人有以鹿贄儔者,儔、俅炰之,召休食。休曰:「我等窮生,菜食不充,今日食肉,翌日何繼?無宜改饌。」獨不食。長慶中,從鄉賦登第,又應賢良方正,升甲科。太和初,歷諸籓闢召,入為監察御史、右補闕、史館修撰。會昌中,自尚書郎歷典數郡。



 大中初,累官戶部侍郎,充諸道鹽鐵轉運使,轉兵部侍郎,兼御史大夫,領使如故。六年八月,以本官同平章事,判使如故。自太和已來,重臣領使者,歲漕江、淮米不過
 四十萬石,能至渭河倉者十不三四。漕吏狡蠹,敗溺百端。官舟沉溺者,歲七十餘只。緣河奸史,大紊劉晏之法。洎休領使,分命僚佐深按其弊。因是所過地里,悉令縣令兼董漕事,能者獎之。自江津達渭口,以四十萬之傭,歲計緡錢二十八萬貫,悉使歸諸漕吏,巡院無得侵牟。舉新法凡十條,奏行之,又立稅茶法二十條,奏行之,物議是之。



 初,休典使三歲,漕米至渭、河倉者一百二十萬斛,更無沉舟之弊。累轉中書侍郎,兼禮部尚書。休在相
 位五年。



 十年,罷相,檢校戶部尚書、汴州刺史、御史大夫,充宣武軍節度使。其年冬,進階金紫光祿大夫、上柱國、河東縣子、食邑五百戶,守太子少保,分司東都。



 十一年冬,檢校戶部尚書、潞州大都督府長史、御史大夫,充昭義節度、潞磁邢洺觀察使。十三年十月,加檢校吏部尚書、太原尹、北都留守、河東節度觀察等使。十四年八月,以本官兼鳳翔尹,充鳳翔隴州節度使。



 咸通初,入為戶部尚書,累遷吏部尚書、太子少師,卒。



 休性寬惠,為官不
 尚曒察,而吏民畏服。善為文,長於書翰,自成筆法。家世奉佛,休尤深於釋典。太原、鳳翔近名山,多僧寺。視事之隙,游踐山林,與義學僧講求佛理。中年後,不食葷血,常齋戒,屏嗜欲。香爐貝典,不離齋中;詠歌贊唄,以為法樂。與尚書紇干皋皆以法號相字。時人重其高潔而鄙其太過,多以詞語嘲之,休不以為忤。



 俅,字冠識,亦登進士第。休子攴。



 楊收,字藏之,同州馮翊人。自言隋越公素之後。高祖悟
 虛,應賢良制科擢第,位終朔州司馬。曾祖幼烈,位終寧州司馬。祖藏器,邠州三水丞。父遺直,位終濠州錄事參軍。家世為儒,遺直客於蘇州,講學為事,因家於吳。遺直生四子:發、假、收、嚴。



 發,字至之,太和四年登進士第,又以書判拔萃,釋褐校書郎、湖南觀察推官,再闢西蜀從事。入朝為監察,轉侍御史,累遷至禮部郎中。大中三年,改左司郎中。



 宣宗追尊順宗、憲宗等尊號,禮院奏廟中神主已題舊號,請改造及重題,詔禮官議。發與都官郎中
 盧搏獻議曰:



 臣等伏尋舊典,慄主升祔之後,在禮無改造之文,亦無重加尊謚、改題神主之例。求之曠古,夐無其文。周加太王、王季、文王之謚,但以德合王周,遂加王號,未聞改謚易主。且文物大備,禮法可稱,最在兩漢,並無其事。光武中興,都洛陽,遣大司馬鄧禹入關,奉高祖已下十一帝後神主祔洛陽宗廟,蓋神主不合新造故也。自魏、晉迄於周、隋,雖代有放恣之君,亦有知禮講學之士,不聞加謚追尊、改主重題。書之史策,可以覆視。



 今
 議者惟引東晉重造鄭太后神主事為證。伏以鄭太后本瑯邪王妃,薨後已祔瑯邪邸廟。其後,母以子貴,將升祔太廟。賀循請重造新主,改題皇后之號,備禮告祔,當時用之。伏以諸侯廟主與天子廟主長短不同。若以王妃八寸之主上配至極,禮似不同。時諂神貪君之私,用此謬禮,改造神主。比量晉事,又絕非宜。且宣懿非穆宗之後,實武宗之母。母以子之貴,已祔別廟,正為得禮,饗薦無虧。今若從祀至尊,題主稱為太后,因臣因子,正得
 其宜。今乃別造新主,題去太字,即是穆宗上仙之後,臣下追致作殯之禮,瀆亂正經,實驚有識。



 臣當時並列朝行,實知謬戾。以漢律,擅論宗廟者以大不敬論,又其時無詔下議,遂默塞不敢出言。今又欲重用東晉謬禮,穢媟聖朝大典。猥蒙下問,敢不盡言。



 臣謹按國朝前例,甚有明文。武德元年五月,備法駕於長安通義里舊廟,奉迎宣簡公、懿王、景皇帝神主,升祔太廟。既言於舊廟奉迎,足明必奉舊主。



 其加謚追尊之禮,自古本無其事,自
 則天太后攝政之後累有之。自此之後,數用其禮。歷檢國史,並無改造重題之文。若故事有之,無不書於簡冊。臣等愚見,宜但告新謚於廟而止。其改造重題之文,開元初,太常卿韋縚以高宗廟題武后神主云「天后聖帝武氏」,縚奏請削去「天后聖帝」之號,別題云「則天順聖皇后武氏」,詔從之。即不知其時削舊題耶?重造主耶?亦不知用何代典禮?禮之疑者,決在宸衷。以臣所見,但以新謚寶冊告陵廟,正得其宜。改造重題,恐乖禮意。



 時宰相
 覆奏就神主改題,而知禮者非之,以發議為是。



 改授太常少卿,出為蘇州刺史。蘇,發之鄉里也。恭長慈幼,人士稱之。還,改福州刺史、福建觀察使。甌閩之人,美其能政,耆老以善績聞。朝廷以發長於邊事,移授廣州刺史、嶺南節度使。屬前政不率,蠻、夏咸怨;發以嚴為理,軍亂,為軍人所囚,致於郵舍。坐貶婺州刺史,卒於治所。



 子乘,亦登進士第,有俊才,尤能為歌詩,歷顯職。



 假,字仁之,進士擢第。故相鄭覃刺華州,署為從事。從覃鎮京口,得大理
 評事。入為監察,轉侍御史。由司封郎中知雜事,轉太常少卿。出為常州刺史,卒官。



 初,遺直娶元氏,生發、假。繼室長孫氏,生收、嚴。



 收長六尺二寸,廣顙深頤,疏眉秀目;寡言笑,方於事上,博聞強記。初,家寄涔陽,甚貧。收七歲喪父,居喪有如成人。而長孫夫人知書,親自教授。十三,略通諸經義,善於文詠,吳人呼為「神童」。兄發戲令詠蛙,即曰:「兔邊分玉樹,龍底耀銅儀。會當同鼓吹,不復問官私。」又令詠筆,仍賦鉆字,即曰:「雖匪囊中物,何堅不可鉆?一
 朝操政事,定使冠三端。」每良辰美景,吳人造門觀神童,請為詩什,觀者壓敗其籓。收嘲曰:「爾幸無羸角,何用觸吾籓。若是升堂者,還應自得門。」收為母奉佛,幼不食肉,母亦勖之曰:「俟爾登進士第,可肉食也。」



 收以仲兄假未登第,久之不從鄉賦。開成末,假擢第;是冬,收之長安,明年,一舉登第,年才二十六。



 時發為潤州從事,因家金陵。收得第東歸,路由淮右,故相司徒杜悰鎮揚州,延收署節度推官,奏授校書郎。悰領度支,以收為巡官。悰罷相
 鎮東蜀,奏授掌書記,得協律郎。悰移鎮西川,復管記室。宰相馬植奏授渭南尉,充集賢校理,改監察御史。收辭曰:「僕兄弟進退以義。頃仲兄假鄉賦未第,收不出衡門。今假從事侯府,僕不忍先為御史。相公必欲振恤孤生,俟僕稟兄旨命可也。」馬公嘉之。收即密達意於西蜀杜公,願復為參佐,悰即表為節度判官。馬公乃以收弟嚴為渭南尉、集賢校理,代收之任。



 周墀罷相,鎮東蜀,表嚴為掌書記。墀至鎮而卒,悰乃闢嚴為觀察判官。兄弟同
 幕,為兩使判官,時人榮之。俄而假自浙西觀察判官入為監察御史,收亦自西川入為監察。兄弟並居憲府,特為新例。



 裴休作相,以收深於禮學,用為太常博士。時收弟嚴亦自揚州從事入為監察。尋丁母喪,歸蘇州。既除,崔珙罷相,鎮淮南,以收為觀察支使。入為侍御史,改職方員外郎,分司東都。宰相夏侯孜領度支,用收為判官。罷職,改司勛員外郎、長安令。秩滿,改吏部員外郎。上言先人未葬,旅殯毗陵,擬遷卜於河南之偃師,請兄弟自
 往。從之。及葬,東周會葬者千人。時故府杜悰、夏侯孜皆在洛,二公聯薦收於執政。宰相令狐綯用收為翰林學士,以庫部郎中知制誥,正拜中書舍人,賜金紫,轉兵部侍郎、學士承旨。左軍中尉楊玄價以收宗姓,深左右之,乃加銀青光祿大夫、中書侍郎、同平章事,累遷門下侍郎、刑部尚書。



 收以交址未復,南蠻擾亂,請治軍江西,以壯出嶺之師。乃於洪州置鎮南軍,屯兵積粟,以餉南海。天子嘉之,進位尚書右僕射、太清太微宮使、弘文館大
 學士、晉陽縣男、食邑三百戶。



 收居位稍務華靡,頗為名輩所譏。而門吏僮奴,倚為奸利。時楊玄價弟兄掌機務,招來方鎮之賂,屢有請托,收不能盡從。玄價以為背己,由是傾之。



 八年十月,罷知政事,檢校工部尚書,出為宣歙觀察使。韋保衡作相,又發收陰事,言前用嚴譔為江西節度,納賂百萬。明年八月,貶為端州司馬,尋盡削官封,長流驩州。又令內養郭全穆齎詔賜死。九年三月十五日,全穆追及之,宣詔訖,收謂全穆曰:「收為宰相無狀,
 得死為幸。心所悲者,弟兄淪喪將盡,只有弟嚴一人,以奉先人之祀。予欲昧死上塵天聽,可容一刻之命,以俟秉筆乎?」全穆許之。收自書曰:



 臣畎畝下才,謬當委任。心乖報國,罪積彌天;特舉朝章,賜之顯戮。臣誠悲誠感,頓首死罪。臣出自寒門,旁無勢援,幸逢休運,累污清資。聖獎曲流,遂叨重任。上不能罄輸臣節,以答寵光;下不能回避禍胎,以延俊乂。茍利尸素,頻歷歲時,果至聖朝,難寬大典。誠知一死未塞深愆,固不合將泉壤之詞,上塵
 天聽。伏乞陛下哀臣愚蠢,稍緩雷霆。臣頃蒙擢在臺衡,不敢令弟嚴守官闕下,旋蒙聖造,令刺浙東。所有罪愆,是臣自負,伏乞聖慈,貸嚴微命。臣血屬皆幼,更無近親,只有弟嚴,才力尪悴。家族所恃,在嚴一人,俾存歿曲全,在陛下弘覆。臣無任魂魄望恩之至。



 全穆復奏,懿崇愍然宥嚴。判官硃侃、常潾、閻均,族人楊公慶、嚴季實、楊全益、何師玄、李孟勛、馬全祐、李羽、王彥復等,皆配流嶺表。



 收子鑒、鉅、鏻,皆登進士第。



 鉅,乾寧初以尚書郎知制誥,
 召充翰林學士,拜中書舍人、戶部侍郎,封晉陽男、食邑三百戶。從昭宗東遷,為左散騎常侍,卒。



 鏻,登第後補集賢校理,藍田尉。乾寧中,累遷尚書郎。



 嚴,字凜之,會昌四年進士擢第。是歲僕射王起典貢部,選士三十人,嚴與楊知至、竇緘、源重、鄭樸五人試文合格,物議以子弟非之,起覆奏。武宗敕曰:「楊嚴一人可及第,餘四人落下。」嚴釋褐諸侯府。咸通中,累遷吏部員外,轉郎中,拜給事中、工部侍郎,尋以本官充翰林學士。兄收作相,封章請外
 職,拜越州刺史、御史中丞、浙東團練觀察使。收罷相貶官,嚴坐貶邵州刺史。收得雪,嚴量移吉王傅。乾符四年,累遷兵部侍郎。五年,判度支。其年病卒。二子:涉、注。



 涉,乾符二年登進士第。昭宗朝,累遷吏部郎中、禮、刑二侍郎。乾符四年,改吏部侍郎。天祐初,轉左丞。從昭宗遷洛陽,改吏部尚書。輝王即位,本官平章事,加中書侍郎。涉性端厚秉禮。乾寧之後,賊臣竊發,王室浸微。及天祐東遷,大事去矣。涉為時所嬰,不能自退。及命相之日,與家人
 相向灑泣曰:「吾不能脫此網羅,禍將至矣。」謂其子凝式曰:「今日之命,吾家重不幸矣,必累爾等。」涉謙退善處,竟以令終。



 注,中和二年進士登第。昭宗朝,累官考功員外、刑部郎中。尋知制誥,正拜中書舍人,召充翰林學士,累遷戶部侍郎。輝王纘歷,兄涉為宰相,注避嫌辭內職,守戶部侍郎。



 韋保衡者,字蘊用,京兆人。祖元貞,父愨,皆進士登第。愨,字端士,太和初登第,後累佐使府,入朝亟歷臺閣。大中
 四年,拜禮部侍郎。五年選士,頗得名人,載領方鎮節度,卒。



 保衡,咸通五年登進士第,累拜起居郎。十年正月,尚懿宗女同昌公主。公主郭淑妃所生,妃有寵,出降之日,傾宮中珍玩以為贈送之資。尋以保衡為翰林學士,轉郎中,正拜中書舍人、兵部侍郎,承旨。不期年,以本官平章事。



 保衡恃恩權,素所不悅者,必加排斥。王鐸貢舉之師,蕭遘同門生,以素薄其為人,皆擯斥之。以楊收、路巖在中書不加禮接,媒孽逐之。自起居郎至宰相,二年之
 間,階至特進、扶風縣開國侯、食邑二千戶、集賢殿大學士。十一年八月,公主薨,自後恩禮漸薄。咸通末,淮、徐盜起,素所怨者發其陰事,保衡竟得罪賜死。



 弟保乂,進士登第,尚書郎、知制誥,召充翰林學士,歷禮、戶、兵三侍郎、學士承旨。坐保衡免官。



 路巖者,字魯瞻,陽平寇氏人也。祖季登,大歷六年登進士第,累闢諸侯府。升朝為尚書郎,遷左諫議大夫,卒。生三子,群、庠、單,皆登進士第。



 群,字正夫,既擢進士,又書判
 拔萃,累佐使府。入朝為監察御史。穆宗初即位,遣使西北邊犒宴軍士,稱旨,累加兵部郎中。太和二年,遷諫議大夫,以本官充侍講學士。四年,罷侍講為翰林學士。五年,正拜中書舍人,學士如故。



 群精經學,善屬文。性仁孝,志行貞潔。父母歿後,終身不茹葷血。歷踐臺閣,受時君異寵,未嘗以勢位自矜。與士友結交,榮達如一。八年正月病卒,君子惜之。二子:岳、巖,大中中相次進士登第。



 巖,幼聰敏過人,父友踐方鎮,書幣交闢,久之方就。數年之
 間,出入禁署。累遷中書舍人、戶部侍郎。咸通三年,以本官同平章事,年始三十六。在相位八年,累兼左僕射。懿宗時,王政多僻,宰臣用事。巖既承委遇,稍務奢靡,頗通賂遺。及韋保衡尚公主,素惡巖為人。保衡作相,罷巖知政事,以檢校左僕射出為成都尹、劍南西川節度使。未幾,改荊南節度。詔令六月下峽赴鎮,尋復罷之。



 岳,歷兩郡刺史,入為給事中。子德延。



 夏侯孜,字好學,本譙人。父審封。孜,寶歷二年登進士第,
 釋褐諸侯府,累遷婺、絳二郡刺史。入為諫議大夫,轉給事中。十年,改刑部侍郎。十一年,兼御史中丞,遷尚書右丞、上柱國,賜紫金魚袋。十一年二月,遷朝議大夫,守戶部侍郎,判戶部事。再加兵部侍郎,充諸道鹽鐵轉運等使。懿宗即位,以本官同平章事,領使如故。累加左僕射、門下侍郎,封譙郡侯,與路巖、楊收同輔政。咸通八年,罷相,檢校司空、同平章事,兼成都尹,充劍南西川節度使。屬南蠻入寇,蜀中饑饉,軍儲不備,蠻陷巂州,蜀川大擾。
 尋移孜為河中尹、檢校司徒、河中晉絳節度使。



 九年,龐勛據徐州,南蠻深入。天子懲孜治蜀無政,詔曰:



 河中晉絳礠隰節度使、開府儀同三司、檢校司徒、同中書門下平章事、河中尹、上柱國、譙郡開國公、食邑二千戶夏侯孜,早以文詞,遂登科第,累更清貫,亦有能名。東陽推撫俗之能,故絳著臨人之稱。其後用司風憲,寵領籓條,皆以公才,不辜時選。洎掌於經費,備歷重難,居然要會之權,頗得均平之道。錄其績效,擢處鈞衡。造膝之時,亦聞
 其算畫;沃心之際,備見其謀猷。於是念彼邊隅,探臨巴蜀,藉其才術,再靜蠻陬。翻致帑廩空虛,軍資窘竭,冤流闔境,寇逼連甍。雖易帥已來,頻移星琯,而無備之後,歲有干戈。昨者徼障初安,瘡痍復釁。敷尋事實,果驗根由。既乖經濟之源,益昧君臣之義。出於物論,非獨予懷,是議難處近籓,爰更散秩。可太子少保,分司東都。



 未幾卒。



 子潭、澤,皆登進士第。潭,累官至禮部侍郎。中和三年選士,多至卿相。子坦。



 劉瞻,字幾之,彭城人。祖升,父景。瞻,太和初進士擢第。四年,又登博學宏詞科,歷佐使府。咸通初升朝,累遷太常博士。劉彖作相,以宗人遇之,薦為翰林學士。轉員外郎中,正拜中書舍人、戶部侍郎,承旨。出為太原尹、河東節度使。入拜京兆尹,復為戶部侍郎、翰林學士。十年,以本官同平章事,加中書侍郎,兼刑部尚書、集賢殿大學士。



 十一年八月,同昌公主薨,懿宗尤嗟惜之。以翰林醫官韓宗召、康仲殷等用藥無效,收之下獄。兩家宗族,枝蔓
 盡捕三百餘人,狴牢皆滿。瞻召諫官令上疏,無敢極言。瞻自上疏曰:



 臣聞修短之期,人之定分。賢愚共一,今古攸同。喬松蕣花,稟氣各異。至如篯鏗壽考,不因有智而延齡;顏子早亡,不為不賢而促壽。此皆含靈稟氣,修短自然之理也。一昨同昌公主久嬰危疾,深軫聖慈。醫藥無征,幽明遽隔。陛下過鐘宸愛,痛切追思,爰責醫工,令從嚴憲。然韓宗召等因緣藝術,備荷寵榮,想於診候之時,無不盡其方術。亦欲病如沃雪,藥暫通神,其奈禍福
 難移,竟成差跌。原其情狀,亦可哀矜。而差誤之愆,死未塞責。



 自陛下雷霆一怒,朝野震驚,囚九族於狴牢,因兩人之藥誤。老幼械系三百餘人,咸云:「宗召荷恩之日,寸祿不沾,進藥之時,又不同議。此乃禍從天降,罪匪己為。」物議沸騰,道路嗟嘆。



 陛下以寬仁厚德,御宇十年,四海萬邦,咸歌聖政。何事遽移前志,頓易初心。以達理知命之君,涉肆暴不明之謗。且殉宮女而違道,囚平人而結冤,此皆陛下安不思危,忿不顧難者也。



 陛下信崇釋典,
 留意生天,大要不過喜舍慈悲,方便布施,不生惡念,所謂福田。則業累盡消,往生忉利,比居濁惡,未可同年。伏望陛下盡釋系囚,易怒為喜,虔奉空王之教,以資愛主之靈。中外臣僚,同深懇激。



 帝閱疏大怒,即日罷瞻相位,檢校刑部尚書、同平章事、江陵尹,充荊南節度等使。再貶康州刺史,量移虢州刺史。入朝為太子賓客分司。翰林學士戶部侍郎鄭畋、右諫議大夫高湘、比部郎中知制誥楊知至、禮部郎中魏紵、兵部員外張顏、刑部員外
 崔彥融、御史中丞孫瑝等,皆坐瞻親善貶逐。京兆尹溫璋仰藥而卒。



 劉彖者,彭城人。祖璠,父煟。彖,開成初進士擢第。會昌末,累遷尚書郎、知制誥,正拜中書舍人。大中初,轉刑部侍郎。彖精於法律,選大中以前二百四十四年制敕可行用者二千八百六十五條,分為六百四十六門,議其輕重,別成一家法書,號《大中統類》,奏行用之。出為河南尹,遷檢校工部尚書、汴州刺史、宣武軍節度使。十一年五
 月,加檢校禮部尚書、太原尹、北都留守、河東節度觀察等使。其年十二月入朝,拜戶部侍郎,判度支。尋以本官同平章事,領使如故。十二年,累加集賢殿大學士。罷相,又歷方鎮,卒。弟頊,亦登進士第。



 曹確,字剛中,河南人。父景伯,貞元十九年進士擢第,又登制科。確,開成二年登進士第,歷聘籓府。入朝為侍御史,以工部員外郎知制誥,轉郎中,入內署為學士,正拜中書舍人,賜金紫,權知河南尹事。入為兵部侍郎。咸通
 五年,以本官同平章事,加中書侍郎、監修國史。



 確精儒術,器識謹重,動循法度。懿宗以伶官李可及為威衛將軍,確執奏曰:「臣覽貞觀故事,太宗初定官品令,文武官共六百四十三員,顧謂房玄齡曰:『朕設此官員,以待賢士。工商雜色之流,假令術逾儕類,止可厚給財物,必不可超授官秩,與朝賢君子比肩而立,同坐而食。』太和中,文宗欲以樂官尉遲璋為王府率,拾遺竇洵直極諫,乃改授光州長史。伏乞以兩朝故事,別授可及之官。」帝不
 之聽。



 可及善音律,尤能轉喉為新聲,音辭曲折,聽者忘倦。京師屠沽效之,呼為「拍彈」。同昌公主除喪後,帝與淑妃思念不已。可及乃為《嘆百年舞曲》。舞人珠翠盛飾者數百人,畫魚龍地衣,用官絁五千匹。曲終樂闋,珠璣覆地,詞語淒惻,聞者涕流,帝故寵之。嘗於安國寺作《菩薩蠻舞》,如佛降生,帝益憐之。可及嘗為子娶婦,帝賜酒二銀樽,啟之非酒,乃金翠也。人無敢非之者,唯確與中尉西門季玄屢論之,帝猶顧待不衰。僖宗即位,崔彥昭奏
 逐之,死於嶺表。



 確累加右僕射,判度支事。在相位六年。九年罷相,檢校司徒、平章事、潤州刺史、鎮海軍節度觀察等使。以出師捍龐勛功,就加太子太師。弟汾,亦進士登第,累官尚書郎、知制誥,正拜中書舍人。出為河南尹,遷檢校工部尚書、許州刺史、忠武軍節度觀察等使。入為戶部侍郎,判度支。弟兄並列將相之任,人士榮之。



 確與畢諴俱以儒術進用,及居相位,廉儉貞苦,君子多之,稱為曹、畢。



 畢諴者,字存之,鄆州須昌人也。伯祖構,高宗時吏部尚書。構弟栩,酆王府司馬,生凌。凌為汾州長史,生勻,為協律郎。勻生諴,少孤貧,燃薪讀書,刻苦自勵。既長,博通經史,尤能歌詩。端愨好古,交游不雜。太和中,進士擢第,又以書判拔萃,尚書杜悰鎮許昌,闢為從事。悰領度支,諴為巡官。悰鎮揚州,又從之。悰入相,諴為監察,轉侍御史。



 武宗朝,宰相李德裕專政,出悰為東蜀節度。悰之故吏,莫敢餞送問訊,唯諴無所顧慮,問遺不絕。德裕怒,出諴
 為磁州刺史。宣宗即位,德裕得罪,凡被譴者皆徵還。諴入為戶部員外郎,分司東都,歷駕部員外郎、倉部郎中。故事,勢門子弟,鄙倉、駕二曹,居之者不悅。唯諴受命,恬然恭遜,口無異言,執政多之。改職方郎中,兼侍御史知雜。其年。召為翰林學士、中書舍人,遷刑部侍郎。



 自大中末,黨項羌叛,屢擾河西。宣宗召學士對邊事。諴即援引古今,論列破羌之狀。上悅,曰:「吾方擇能帥,安集河西,不期頗、牧在吾禁署,卿為朕行乎?」諴忻然從命,即用諴為
 邠寧節度、河西供軍安撫等使。諴至軍,遣使告喻叛徒,諸羌率化。又以邊境御戎,以兵多積穀為上策。乃召募軍士,開置屯田,歲收穀三十萬石,省度支錢數百萬。詔書嘉之,就加檢校工部尚書,移鎮澤潞,充昭義節度使。二年,改太原尹、北都留守、河東節度使。太原近胡,九姓為亂。諴明賞罰,謹斥候,期年,諸部革心。就加檢校尚書左僕射,移授汴州刺史,充宣武軍節度、宋亳汴觀察等使。其年,入為戶部尚書,領度支。月餘,改禮部尚書,同平
 章事,累遷中書侍郎、兵部尚書、集賢大學士。



 在相位三年,十月以疾固辭位,詔守兵部尚書,以其本官同平章事,出鎮河中。十二月二十三日,卒於鎮,時年六十二。



 諴謹重,長於文學,尤精吏術。在相位,以同官任情不法,固辭而免,君子美之。



 子紹顏、知顏,登進士第,累歷顯官。



 杜審權,字殷衡,京兆人也。國初萊成公如晦六代孫。祖佐,位終大理正。佐生二子:元潁、元絳。



 元潁,穆宗朝宰相。絳位終太子賓客。絳生二子:審權、蔚,並登進士第。



 審權,
 釋褐江西觀察判官,又以書判拔萃,拜右拾遺,轉左補闕。大中初,遷司勛員外郎,轉郎中知雜。又以本官知制誥,正拜中書舍人。十年,權知禮部貢舉。十一年,選士三十人,後多至達官。正拜禮部侍郎。其年冬,出為陜州大都督府長史、陜虢都團練觀察使,加檢校戶部尚書、河中尹、河中晉絳節度使。



 懿宗即位,召拜吏部尚書。三年,以本官同平章事,累加門下侍郎、右僕射。九年罷相,檢校司空,兼潤州刺史、鎮海軍節度使、蘇杭常等州觀察
 使。



 時徐州戍將龐勛自桂州擅還,據徐、泗,大擾淮南。審權與淮南節度使令狐綯、荊南節度使崔鉉,奉詔出師,掎角討賊;而浙西饋運不絕,繼破徐戎。賊平,召拜尚書左僕射。十一年,制曰:



 開府儀同三司、檢校司空、守尚書左僕射、上柱國、襄陽郡開國公、食邑二千戶杜審權,韻合黃鐘,行真白璧。沖粹孕靈岳之秀,精明涵列宿之光,塵外孤標,雲間獨步。踐歷華貫,餘二十年;鑒裁名流,凡幾百輩。清切之任無不試,重難之務無不經。靜而立名,
 嚴以肅物。絕分毫徇己之意,秉尺寸度量之懷。貞方飾躬,溫茂繕性。儉不逼下,畏以居高。語默適時,喜慍莫見。頃罷機務,鎮於金陵,值淮夷猖狂,干戈悖起。累發猛士,挫彼賊鋒;廣備糗糧,助茲軍食。深惟將相之大體,頗睹文武之全才。王導以蕭灑之名,不忘戎事;謝安以恬淡之德,亦在兵間。及駟馬來朝,擢居端揆,嚴重自處,恬曠不渝。虞芮之故都,前蹤尚爾;郇瑕之舊地,往事依然。兼以股肱之良,為吾腹心之寄。改佩相印,更握兵符。仍五
 教之崇名,極一時之盛禮。可檢校司徒、同平章事、河中尹,充河中晉絳節度觀察等使。



 數年以本官兼許州刺史、忠武軍節度觀察等使,入為太子太傅,分司東都。卒,贈太師,謚曰德。



 三子:讓能、彥林、弘徽。



 讓能,咸通十四年登進士第,釋褐咸陽尉。宰相王鐸鎮汴,奏為推官。入為長安尉、集賢校理。丁母憂,以孝聞。服闋,淮南節度使劉鄴闢掌記室,得殿中,賜緋。入為監察。牛蔚鎮興元,奏為節度判官。入為右補闕,歷侍御史、起居郎、禮部、兵部員
 外郎。蕭遘領度支,以本官判度支案。



 黃巢犯京師,奔赴行在,拜禮部郎中、史館修撰。尋以本官知制誥,正拜中書舍人。謝日,面賜金紫之服,尋召充翰林學士。六飛在蜀,關東用兵,徵發招懷,書詔雲委。



 讓能詞才敏速,筆無點竄,動中事機,僖宗嘉之,累遷戶部侍郎。從駕還京,加禮部尚書,進階銀青光祿大夫,封建平縣開國子,食邑五百戶。轉兵部尚書、學士承旨。



 沙阤逼京師,僖宗蒼黃出幸。是夜,讓能宿直禁中,聞難作,步出從駕。出城十餘
 里,得遺馬一匹,無羈勒,以紳束首而乘之。駕在鳳翔,硃玫兵遽至;僖宗急幸寶雞,近臣唯讓能獨從。翌日。孔緯等六七人至。邠師攻關,帝幸梁、漢,棧道為石協所毀,崎嶇險阻之間,不離左右。帝顧謂之曰:「朕之失道,再致播遷。險難之中,卿常在側,古所謂忠於所事,卿無負矣!」讓能謝曰:「臣家世歷重任,蒙國厚恩,陛下不以臣愚,擢居近侍。臨難茍免,臣之恥也;獲捍牧圉,臣之幸也。」至褒中,加金紫光祿大夫,改兵部侍郎,同平章事。



 時硃玫立襄
 王稱制,天下牧伯附之者十六七,貢賦殆絕。朝士才十數人,行帑無寸金,衛兵不宿飽。帝垂泣側席,無如之何。讓能首陳大計,請以重臣使河中,諭王重榮以大義,果承詔請雪,以圖討逆。京師平,拜特進、中書侍郎,兼兵部尚書、集賢殿大學士,進封襄陽郡開國公,食邑二千戶。駕在鳳翔,李昌符作亂,倏然變起,讓能單步入侍。時朝臣受偽署者眾,法司請行極法,以戒事君。讓能固爭之,獲全者十七八。昭宗纂嗣,賜「扶危啟運保乂功臣」,加開
 府儀同三司、尚書左僕射,封晉國公,增邑千戶,仍賜鐵券。誅秦宗權,許、蔡平定,加司空、門下侍郎、監修國史。昭宗郊禮畢,進位司徒、太清宮使、弘文館大學士、延資庫使、諸道鹽鐵轉運等使,加食邑一千戶。明年,冊拜太尉,加食邑一千戶。



 自大順已來,鳳翔李茂貞大聚兵甲,恃功驕恣。會楊復恭走山南,茂貞欲兼有梁、漢之地,亟請問罪,詔未允而出師。昭宗怒其專,不得已而從之。及山南平,詔授以茂貞鎮興元,徐彥若鎮鳳翔,仍割果、閬兩
 州隸武定軍。茂貞怒,上章論列,語辭不遜。又與讓能書曰:



 宰相之職,外撫四夷,內安百姓。陰陽不順,猶資燮理之功;宇宙將傾,須假扶持之力。即萬靈舒慘,四海安危,盡系朝綱,咸由廟算,既為重任,方屬元臣。況今國步猶艱,皇居未壯。曩日九衢三市,草擁荒墟;當時萬戶千門,霜凝白骨。大廈傾欹而未已,沉痾綿息以無餘。皆云非賢後無以拯社稷之危,非真宰無以革寰區之弊。



 今明公舍築入夢,投竿為師,踐履中臺,制臨外閫,不究興亡
 之理,罕聞沉斷之機。蓋意有所不平,心有所未悟,輒思上問,願審臧謀。



 竊見楊守亮擅舉干戈,阻艱西道,將圖割據,吞並東川。居巴、幹為一窟豺狼,在梁、漢致十年荊棘。果聞敗衄,尋挫兇狂。既前去而不諧,思卻歸而無地。當道與邠州見為隔絕綱運,方舉問罪兵師,忽聞朝廷授武定之雙旌,割果、閬之兩郡,未審是何名目?酬何功勞?紊大國之紀綱,蠹天子之州縣,非惟取笑於童稚,抑亦包羞於馬牛。自謂奇謀,信為獨見。伏慮是明公賞兇
 黨無君之輩,挫忠臣奉國之心。要助奸邪,須摧正直。又聞公切於保位,利在安家。商量不自於中書剸割全通於內地。雖知深奧,罕測津涯,亦聞駭異群情,頗,是喧騰眾口。



 其悖戾如此。



 京師百姓,聞茂貞聚兵甲,群情恟々,數千百人守闕門。候中尉西門重遂出,擁馬論列曰:「乞不分割山南,請姑息鳳翔,與百姓為主。」重遂曰:「此非吾事,出於宰相也。」昭宗怒,詔讓能只在中書調發畫計,不歸第。月餘,宰相崔昭緯陰結邠、岐為城社,凡讓能出一
 言,即日達於茂貞、行瑜。茂貞令健兒數百人,雜市人於街。崔昭緯、鄭延昌歸第,市人擁肩輿訴曰:「岐帥無罪,幸相公不加討伐,致都邑不寧。」二相輿中喻之曰:「大政聖上委杜太尉,吾等不預。」市豪褰簾熟視,又不之識,因投瓦石,擊二相之輿。崔、鄭下輿散走,匿身獲免。是日,喪堂印公服,天子怒,捕魁首誅之,由是用兵之意愈堅。京師之人,相與藏竄,嚴刑不能已。讓能奏曰:「陛下初臨大寶,國步未安。自艱難以來,且行貞元故事,姑息籓鎮。茂貞
 邇在國門,不宜起怨。臣料此時未可行也。」帝曰:「政刑削弱,詔令不出城門,此賈生慟哭之際也。又《書》不云乎?藥不瞑眩,厥疾弗瘳。朕不能孱孱度日,坐觀凌弱。卿為我主張調發,用兵吾委諸王。」讓能對曰:「陛下憤籓臣之倔強,必欲強幹弱枝以隆王室,此則中外大臣所宜戮力,以成陛下之志,不宜獨任微臣。」帝曰:「卿位居元輔,與朕同休共戚,無宜避事。」讓能泣辭曰:「臣待罪臺司,未乞骸骨者,思有以報國恩耳,安敢愛身避事?況陛下之心,憲
 祖之志也。但時有所不便,勢有所必然。他日臣雖受晁錯之誅,但不足以殄七國之患,敢不奉詔,繼之以死!」



 景福二年秋,上以嗣覃王為招討使,神策將李金歲副之,率禁軍三萬,送彥若赴鎮。崔昭緯密與邠、鳳結托,心害讓能;言討伐非上意,出於大尉也。九月,茂貞出軍逆戰,王師敗於盩啡。岐兵乘勝至三橋。讓能奏曰:「臣固預言之矣。請歸罪於臣,可以紓難。」上涕下不能已,曰:「與卿訣矣。」即日貶為雷州司戶。茂貞在臨皋驛,請誅讓能。尋賜死,
 時年五十三。駕自石門還京,念讓能之冤,追贈太師。



 子光乂、曉,以父枉橫,不求聞達。曉入梁,位亦至宰輔。



 彥林、弘徽,乾符中相次登進士第。彥林,光化中累官至尚書郎、知制誥,拜中書舍人。天祐初,為御史中丞。



 弘徽,累官至中書舍人,遷戶部侍郎,充弘文館學士判館事,與兄同日被害。



 劉鄴,字漢籓,潤州句容人也。父三復,聰敏絕人,幼善屬文。少孤貧,母有廢疾,三復丐食供養,不離左右,久之不
 遂鄉賦。



 長慶中,李德裕拜浙西觀察使,三復以德裕禁密大臣,以所業文詣郡幹謁。德裕閱其文,倒屣迎之,乃闢為從事,管記室。母亡,哀毀殆不勝喪。德裕三為浙西,凡十年,三復皆從之。太和中,德裕輔政,用為員外郎。居無何,罷相,復鎮浙西,三復從之。汝州刺史劉禹錫以宗人遇之。深重其才,嘗為詩贈三復,序曰:「從弟三復,三為浙右從事,凡十餘年。往年主公入相,薦用登朝,中復從公之京口,未幾而罷。昨以尚書員外郎奉使至潞,旋承
 新命,改轅而東。三從公皆在舊地,徵諸故事,夐無其比,因賦詩餞別以志之。」又從德裕歷滑臺、西蜀、揚州,累遷御史中丞。會昌中,德裕用事,自諫議、給事拜刑部侍郎、弘文館學士判館事。



 朝廷用兵誅劉稹,澤潞既平。朝議以劉從諫妻裴氏是裴問之妹,欲原之。法司定罪,以劉稹之叛,裴以酒食會潞州將校妻女,泣告以固逆謀。三復奏曰:



 劉從諫苞藏逆謀,比雖已露,今推窮僕妾,尤得事情。據其圖謀語言,制度服物,人臣僭亂,一至於斯。雖
 生前幸免於顯誅,而死後已從於追戮,凡在朝野,同深慶快。且自古人臣叛逆,合有三族之誅。《尚書》曰:「乃有顛越不恭,我則劓殄滅之,無遺育,無俾易種於茲新邑。」如此則阿裴已不得免於極法矣!又況從諫死後,主張狂謀,罪狀非一。劉稹年既幼小,逆節未深,裴為母氏,固宜誡誘。若廣說忠孝之道,深陳禍福之源,必冀虺毒不施,梟音全革。而乃激厲兇黨,膠固叛心,廣招將校之妻,適有酒食之宴;號哭激其眾意,贈遺結其群情。遂使叛黨
 稽不舍之誅,孽童延必死之命,以至周歲。方就誅夷,此阿裴之罪也。雖以裴問之功,或希減等,而國家有法,難議從輕。伏以管叔,周公之親弟也,有罪而且誅之。以周公之賢,尚不舍兄弟之罪,況裴問之功效,安能破朝廷法耶?據阿裴廢臣妾之道,懷逆亂之謀,裴問如周公之功,尚合行周公之戮。況於朝典,固在不疑。阿裴請準法。



 從之。三復未幾病卒。



 鄴六七歲能賦詩,李德裕尤憐之,與諸子同硯席師學。大中初,德裕貶逐,鄴無所依,以文
 章客游江、浙。每有制作,人皆稱誦。高元裕廉察陜虢,署為團練推官,得秘書省校書郎。咸通初,劉瞻、高璩居要職,以故人子薦為左拾遺,召充翰林學士,轉尚書郎中知制誥,正拜中書舍人、戶部侍郎、學士承旨。



 鄴以李德裕貶死珠崖,大中朝以令狐綯當權,累有赦宥,不蒙恩例。懿宗即位,綯在方鎮,屬郊天大赦,鄴奏論之曰:「故崖州司戶參軍李德裕,其父吉甫,元和中以直道明誠,高居相位,中外咸理,訏謨有功。德裕以偉望宏才,繼登臺
 袞;險夷不易,勁正無群。稟周勃厚重之姿,慕楊秉忠貞之節。頃以微累,竄於遐荒,既迫衰殘,竟歸冥寞。其子燁坐貶象州立山縣尉。去年遇陛下布惟新之命,覃作解之恩,移授郴州郴縣尉,今已歿於貶所。倘德裕猶有親援,可期振揚,微臣固不敢上論,以招浮議。今骨肉將盡,生涯已空,皆傷棨戟之門,遽作荊榛之地;孤骨未歸於塋兆,一男又沒於湘江。特乞聖明,俯垂哀愍,俾還遺骨,兼賜贈官。上弘錄舊之仁,下激徇公之節。」詔從之。



 鄴尋
 以本官領諸道鹽鐵轉運使。其年同平章事,判度支,轉中書侍郎,兼吏部尚書,累加太清宮使、弘文館大學士。僖宗即位,蕭人放、崔彥昭秉政,素惡鄴,乃罷鄴知政事,檢校尚書左僕射、同平章事、揚州大都督府長史、淮南節度使。是日鄴押班宣麻竟,通事引鄴內殿謝,不及笏記,鄴自敘十餘句語云:「霖雨無功,深愧代天之用;煙霄失路,未知歸骨之期。」帝為之惻然。



 黃巢渡淮而南,詔以浙西高駢代還,尋除風翔尹、鳳翔隴右節度使,以疾辭,拜
 左僕射。巢賊犯長安,鄴從駕不及,與崔沆、豆盧彖匿於金吾將軍張直方之家。旬日,賊嚴切追捕,三人夜竄;為賊所得,迫以偽命,稱病不應,俱為賊所害。



 豆盧彖者,河東人。祖願,父籍,皆以進士擢第。彖,大中十三年亦登進士科。咸通末,累遷兵部員外郎,轉戶部郎中知制誥,召充翰林學士,正拜中書舍人。乾符中,累遷戶部侍郎、學士承旨。六年,與吏部侍郎崔沆同日拜平章事。宣制曰,大風雷雨拔樹。左丞韋蟾與王彖善,往賀之。
 彖言及雷雨之異,蟾曰:「此應相公為霖作解之祥也。」彖笑答曰:「霖何甚耶?」及巢賊犯京師,從僖宗出開遠門,為盜所制,乃匿於張直方之家,遇害。識者以風雷,不令之兆也。



 弟瓚、璨,皆進士登第,累歷清要。瓚子革,中興位亦至宰輔。



 史臣曰:近代衣冠人物,門族昌盛,從、頲之後,實富名流。而彥曾屬徐亂之秋,胤接李亡之數,計則繆矣,天可逃乎?楊、劉、曹畢諸族,門非世胄,位以藝升,伏膺典墳,俯拾
 青紫。而收得位求侈,以至敗名。行己飭躬,此為深誡!杜氏三世輔相,太尉陷於橫流,臨難忘身,可為流涕。



 贊曰:漢代荀、陳,我朝崔、杜。有子有弟,多登宰輔。裴士改節,楊子敗名。膏粱移性,信而有征。



\end{pinyinscope}