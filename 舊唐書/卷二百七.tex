\article{卷二百七}

\begin{pinyinscope}

 吐蕃,
 在長安之西八千里,本漢西羌之地也。其種落莫知所出也,或云南涼禿發利鹿孤之後也。利鹿孤有子曰樊尼,及利鹿孤卒,樊尼尚幼,弟傉檀嗣位,以樊尼為
 安西將軍。後魏神瑞元年,傉檀為西秦乞佛熾盤所滅,樊尼招集餘眾,以投沮渠蒙遜,蒙遜以為臨松太守。及蒙遜滅,樊尼乃率眾西奔,濟黃河,逾積石,於羌中建國,開地千里。樊尼威惠夙著,為群羌所懷,皆撫以恩信,歸之如市。遂改姓為窣勃野,以禿發為國號,語訛謂之吐蕃,其後子孫繁昌,又侵伐不息,土宇漸廣。歷周及隋,猶隔諸羌,未通於中國。



 其國人號其王為贊普,相為大論、小論,以統理國事。無文字,刻木結繩為約。雖有官,不常
 厥職,臨時統領。徵兵用金箭,寇至舉烽燧,百里一亭。用刑嚴峻,小罪剜眼鼻,或皮鞭鞭之,但隨喜怒而無常科。囚人於地牢,深數丈,二三年方出之。宴異國賓客,必驅犛牛,令客自射牲以供饌。與其臣下一年一小盟,刑羊狗獼猴,先折其足而殺之,繼裂其腸而屠之。令巫者告於天地山川日月星辰之神云:「若心遷變,懷奸反覆,神明鑒之,同於羊狗。」三年一大盟,夜於壇墠之上與眾陳設肴饌,殺犬馬牛驢以為牲,咒曰:「爾等咸須同心戮力,
 共保我家,惟天神地祇,共知爾志。有負此盟,使爾身體屠裂,同於此牲。」



 其地氣候大寒,不生秔稻,有青稞麥、褭豆、小麥、喬麥。畜多犛牛豬犬羊馬。又有天鼠,狀如雀鼠,其大如貓,皮可為裘。又多金銀銅錫。其人或隨畜牧而不常厥居,然頗有城郭。其國都城號為邏些城。屋皆平頭,高者至數十尺。貴人處於大氈帳,名為拂廬。寢處污穢,絕不櫛沐。接手飲酒,以氈為盤,捻鋋為碗,實以羹酪,並而食之。多事羱羝之神,人信巫覡。不知節候,麥熟為
 歲首。圍棋陸博,吹蠡鳴鼓為戲,弓劍不離身。重壯賤老,母拜於子,子倨於父,出入皆少者在前,老者居其後。軍令嚴肅,每戰,前隊皆死,後隊方進。重兵死,惡病終。累代戰沒,以為甲門。臨陣敗北者,懸狐尾於其首,表其似狐之怯,稠人廣眾,必以徇焉,其俗恥之,以為次死。拜必兩手據地,作狗吠之聲,以身再揖而止。居父母喪,截發,青黛塗面,衣服皆黑,既葬即吉。其贊普死,以人殉葬,衣服珍玩及嘗所乘馬弓劍之類,皆悉埋之。仍於墓上起大
 室,立土堆,插雜木為祠祭之所。



 貞觀八年,其贊普棄宗弄贊始遣使朝貢。弄贊弱冠嗣位,性驍武,多英略,其鄰國羊同及諸羌並賓伏之。太宗遣行人馮德遐往撫慰之。見德遐,大悅。聞突厥及吐谷渾皆尚公主,乃遣使隨德遐入朝,多齎金寶,奉表求婚,太宗未之許。使者既返,言於弄贊曰:「初至大國,待我甚厚,許嫁公主。會吐谷渾王入朝,有相離間,由是禮薄,遂不許嫁。」弄贊遂與羊同連,發兵以擊吐谷渾。吐谷渾不能支,遁於青海之上,以
 避其鋒。其國人畜並為吐蕃所掠。於是進兵攻破黨項及白蘭諸羌,率其眾二十餘萬,頓於松州西境。遣使貢金帛,雲來迎公主。又謂其屬曰:「若大國不嫁公主與我,即當入寇。」遂進攻松州,都督韓威輕騎覘賊,反為所敗,邊人大擾。太宗遣吏部尚書侯君集為當彌道行營大總管,右領軍大將軍執失思力為白蘭道行軍總管,左武衛將軍牛進達為闊水道行軍總管,右領軍將軍劉蘭為洮河道行軍總管,率步騎五萬以擊之。進達先鋒
 自松州夜襲其營,斬千餘級。弄贊大懼,引兵而退,遣使謝罪。因復請婚,太宗許之。弄贊乃遣其相祿東贊致禮,獻金五千兩,自餘寶玩數百事。



 貞觀十五年,太宗以文成公主妻之,令禮部尚書、江夏郡王道宗主婚,持節送公主於吐蕃。弄贊率其部兵次柏海,親迎於河源。見道宗,執子婿之禮甚恭。既而嘆大國服飾禮儀之美,俯仰有愧沮之色。及與公主歸國,謂所親曰:「我父祖未有通婚上國者,今我得尚大唐公主,為幸實多。當為公主
 築一城,以誇示後代。」遂築城邑,立棟宇以居處焉。公主惡其人赭面,弄贊令國中權且罷之,自亦釋氈裘,襲紈綺,漸慕華風。仍遣酋豪子弟,請入國學以習《詩》、《書》。又請中國識文之人典其表疏。



 太宗伐遼東還,遣祿東贊來賀。奉表曰:「聖天子平定四方,日月所照之國,並為臣妾,而高麗恃遠,闕於臣禮。天子自領百萬,度遼致討,隳城陷陣,指日凱旋。夷狄才聞陛下發駕,少進之間,已聞歸國。雁飛迅越,不及陛下速疾。奴忝預子婿,喜百常夷。夫鵝,
 猶雁也,故作金鵝奉獻。」其鵝黃金鑄成,其高七尺,中可實酒三斛。



 二十二年,右衛率府長史王玄策使往西域,為中天竺所掠。吐蕃發精兵與玄策擊天竺,大破之,遣使來獻捷。



 高宗嗣位,授弄贊為駙馬都尉,封西海郡王,賜物二千段。弄贊因致書於司徒長孫無忌等云:「天子初即位,若臣下有不忠之心者,當勒兵以赴國除討。」並獻金銀珠寶十五種,請置太宗靈座之前。高宗嘉之,進封為賓王,賜雜彩三千段。因請蠶種及造酒、碾、磑、紙、墨
 之匠,並許焉。乃刊石像其形,列昭陵玄闕之下。



 永徽元年,弄贊卒。高宗為之舉哀,遣右武候將軍鮮於臣濟持節齎璽書吊祭。弄贊子早死,其孫繼立,復號贊普,時年幼,國事皆委祿東贊。祿東姓MS氏,雖不識文記,而性明毅嚴重,講兵訓師,雅有節制,吐蕃之並諸羌,雄霸本土,多其謀也。



 初,太宗既許降文成公主,贊普使祿東贊來迎,召見顧問,進對合旨,太宗禮之,有異諸蕃,乃拜祿東贊為右衛大將軍,又以瑯邪長公主外孫女段氏妻之。
 祿東贊辭曰:「臣本國有婦,父母所聘,情不忍乖。且贊普未謁公主,陪臣安敢輒娶。」太宗嘉之,欲撫以厚恩,雖奇其答而不遂其請。祿東贊有子五人:長曰贊悉若,早死;次欽陵,次贊婆,次悉多干,次勃論。及東贊死,欽陵兄弟復專其國。



 後與吐谷渾不和,龍朔、麟德中遞相表奏,各論曲直,國家依違,未為與奪。吐蕃怨怒,遂率兵以擊吐谷渾。吐谷渾大敗,河源王慕容諾曷缽及弘化公主脫身走投涼州,遣使告急。



 咸亨元年四月,詔以右威
 衛大將軍薛仁貴為邏婆道行軍大總管,左衛員外大將軍阿史那道真、右衛將軍郭待封為副,率眾十餘萬以討之。軍至大非川,為吐蕃大將論欽陵所敗,仁貴等並坐除名。吐谷渾全國盡沒,唯慕容諾曷缽及其親信數千帳來內屬,仍徙於靈州。自是吐蕃連歲寇邊,當、悉等州諸羌盡降之。



 上元三年,進寇鄯、廓等州,殺掠人吏,高宗命尚書左僕射劉仁軌往洮河軍鎮守以御之。儀鳳三年,又命中書令李敬玄兼鄯州都督,往代仁軌於洮河
 鎮守。仍召募關內、河東及諸州驍勇,以為猛士,不簡色役。亦有嘗任文武官者召入殿庭賜宴,遣往擊之。又令益州長史李孝逸、巂州都督拓王奉等發劍南、山南兵募以防禦之。其年秋,敬玄與工部尚書劉審禮,率兵與吐蕃戰於青海。官軍敗積,審禮沒於陣,敬玄按軍不敢救。俄而收軍卻出,頓於承風嶺,阻泥溝不能動,賊屯於高岡以壓之。偏將左領軍員外將軍黑齒常之率敢死之士五百人,夜斫賊營,賊遂潰亂,自相蹂踐,死者三百
 餘人。敬玄遂擁眾鄯州,坐改為衡州刺史。往劍南兵募,於茂州之西南築安戎城以壓其境。俄有生羌為吐蕃鄉導,攻陷其城,遂引兵守之。時吐蕃盡收羊同、黨項及諸羌之地,東與涼、松、茂、巂等州相接,南至婆羅門,西又攻陷龜茲、疏勒等四鎮,北抵突厥,地方萬餘里,自漢、魏已來,西戎之盛,未之有也。



 高宗聞審禮等敗沒,召侍臣問綏御之策,中書舍人郭正一曰:「吐蕃作梗,年歲已深,命將興師,相繼不絕。空勞士馬,虛費糧儲,近討則徒損
 兵威,深入則未窮巢穴,望少發兵募,且遣備邊,明烽堠,勿令侵抄。使國用豐足,人心葉同,寬之數年,可一舉而滅。」給事中劉齊賢、皇甫文亮等皆言嚴守之便。尋而黑齒常之破吐蕃大將贊婆及素和貴於良非川,殺獲二千餘級,吐蕃遂引退。詔以常之為河源軍使以鎮御之。



 儀鳳四年,贊普卒,其子器弩悉弄嗣位,復號贊普,時年八歲,國政復委於欽陵。遣其大臣論寒調傍來告喪。且請和。高宗遣郎將宋令文入蕃會葬。永隆元年,文成公
 主薨,高宗又遣使吊祭之。



 則天臨朝,命文昌右相韋待價為安息道大總管,安西大都護閻溫古為副。永昌元年,率兵往征吐蕃,遲留不進,待價坐流浦州,溫古處斬。待價素無統御之才,遂狼狽失據,士卒饑饉,皆轉死溝壑。明年,又命文昌右相岑長倩為武威道行軍大總管以討吐蕃,中路退還,軍竟不行。如意元年,吐蕃大首領曷蘇率其所屬並貴川部落請降,則天令右玉鈐衛大將軍張玄遇率精卒二萬充安撫使以納之。師次大渡
 水,曷蘇事洩,為本國所擒,又有大首領昝捶率羌蠻部落八千餘人詣玄遇內附。玄遇以其部落置葉川州,以昝捶為刺史。仍於大度西山勒石紀功而還。長壽元年,武威軍總管王孝傑大破吐蕃之眾,克復龜茲、于闐、疏勒、碎葉等四鎮,乃於龜茲置安西都護府,發兵以鎮守之。萬歲登封元年,孝傑復為肅邊道大總管,率副總管婁師德與吐蕃將論欽陵、贊婆戰於素羅汗山。官軍敗績,李傑坐免官。萬歲通天元年,吐蕃四萬眾奄至涼州
 城下,都督許欽明初不之覺,輕出按部,遂遇賊,拒戰久之,力屈為賊所殺。時吐蕃又遣使請和,則天將許之;論欽陵乃請去安西四鎮兵,仍索分十姓之地,則天竟不許之。



 吐蕃自論欽陵兄弟專統兵馬,欽陵每居中用事,諸弟分據方面,贊婆則專在東境,與中國為鄰,三十餘年,常為邊患。其兄弟皆有才略,諸蕃憚之。



 聖歷二年,其贊普器弩悉弄年漸長,乃與其大臣論巖等密圖之。時欽陵在外,贊普乃佯言將獵,召兵執欽陵親黨二千餘
 人,殺之。發使召欽陵、贊婆等,欽陵舉兵不受召,贊普自帥眾討之,欽陵未戰而潰,遂自殺,其親信左右同日自殺者百餘人。贊婆率所部千餘人及其兄子莽布支等來降,則天遣羽林飛騎郊外迎之,授贊婆輔國大將軍、行右衛大將軍,封歸德郡王,優賜甚厚,仍令領其部兵於洪源谷討擊。尋卒,贈特進、安西大都護。



 久視元年,吐蕃又遣其將趨莽布支寇涼州,圍逼昌松縣。隴右諸軍州大使唐休璟與莽布支戰於洪源谷,斬其副將二人,
 獲首二千五百級。長安二年,贊普率眾萬餘人寇悉州,都督陳大慈與賊凡四戰,皆破之,斬首千餘級。於是吐蕃遣使論彌薩等入朝請求和,則天宴之於麟德殿,奏百戲於殿庭。論彌薩曰:「臣生於邊荒,由來不識中國音樂,乞放臣親觀。」則天許之。於是論彌薩等相視笑忭拜謝曰:「臣自歸投聖朝,前後禮數優渥,又得親觀奇樂,一生所未見。自顧微瑣,何以仰答天恩,區區褊心,唯願大家萬歲。」明年,又遣使獻馬千匹、金二千兩以求婚,則天
 許之。



 時吐蕃南境屬國泥婆羅門等皆叛,贊普自往討之,卒於軍中。諸子爭立,久之,國人立器弩悉弄之子棄隸蹜贊為贊普,時年七歲。中宗神龍元年,吐蕃使來告喪,中宗為之舉哀,廢朝一日。俄而贊普之祖母遣其大臣悉薰然來獻方物,為其孫請婚,中宗以所養雍王守禮女為金城公主許嫁之。自是頻歲貢獻。景龍三年十一月,又遣其大臣尚贊吐等來迎女,中宗宴之於苑內球場,命駙馬都尉楊慎交與吐蕃使打球,中宗率侍臣
 觀之。四年正月,制曰:



 聖人布化,用百姓為心;王者垂仁,以八荒無外。故能光宅遐邇,裁成品物。由是隆周理歷,恢柔遠之圖;強漢乘時,建和親之議。斯蓋宇長策,經邦茂範。朕受命上靈,克纂洪業,庶幾前烈,永致和平。睠彼吐蕃,僻在西服,皇運之始,早申朝貢。太宗文武聖皇帝德侔覆載,情深億兆,思偃兵甲,遂通姻好,數十年間,一方清凈。自文成公主化往,其國因多變革。我之邊隅,亟興師旅,彼之蕃落,頗聞雕弊。頃者贊普及祖母可敦、酋
 長等,屢披誠款,積有歲時,思托舊親,請崇新好。金城公主,朕之少女,豈不鐘念,但為人父母,志息黎元,若允乃誠祈,更敦和好,則邊土寧晏,兵役服息。遂割深慈,為國大計,築茲外館,聿膺嘉禮,降彼吐蕃贊普,即以今月進發,朕想自送於郊外。



 中宗召侍中紀處訥謂曰:「昔文成公主出降,則江夏王送之。卿雅識蕃情,有安邊之略,可為朕充吐蕃使也。」處訥拜謝,既而以不練邊事固辭。上又令中書侍郎趙彥昭充使。彥昭以既充外使,恐失其
 權寵,殊不悅。司農卿趙履溫私謂之曰:「公國之宰輔,而為一介之使,不亦鄙乎?」彥昭曰:「然計將安出?」履溫因陰托安樂公主密奏留之。於是以左衛大將軍楊矩使焉。其月,帝幸始平縣以送公主,設帳殿於百頃泊側,引王公宰相及吐蕃使入宴,中坐酒闌,命吐蕃使進前,諭以公主孩幼,割慈遠嫁之旨,上悲泣歔欷久之。因命從臣賦詩餞別,曲赦始平縣大闢罪已下,百姓給復一年,改始平縣為金城縣,又改其地為鳳池鄉愴別里。公主既
 至吐蕃,別築一城以居之。



 睿宗即位,攝監察御史李知古上言:「姚州諸蠻,先屬吐蕃,請發兵擊之。」遂令知古征劍南兵募往經略之。蠻酋傍名乃引吐蕃攻知古,殺之,仍斷其尸以祭天。時張玄表為安西都護,又與吐蕃比境,互相攻掠,吐蕃內雖怨怒,外敦和好。時楊矩為鄯州都督,吐蕃遣使厚遺之,因請河西九曲之地以為金城公主湯沐之所,矩遂奏與之。吐蕃既得九曲,其地肥良,堪頓兵畜牧,又與唐境接近,自是復叛,始率兵入寇。



 開
 元二年秋,吐蕃大將闉達焉、乞力徐等率眾十餘萬寇臨洮軍,又進寇蘭、渭等州,掠監牧羊馬而去。楊矩悔懼,飲藥而死。玄宗令攝左羽林將軍薛訥及太僕少卿王晙率兵邀擊之。仍下詔將大舉親征,召募將士,克期進發。俄而晙等與賊相遇於渭源之武階驛,前軍王海賓力戰死之,晙等率兵而進,大破吐蕃之眾,殺數萬人,盡收復所掠羊馬。賊餘黨奔北,相枕藉而死,洮水為之不流。上遂罷親征,命紫微舍人倪若水往按軍實,仍吊祭
 王海賓而還。吐蕃遣其大臣宗俄因子至洮河祭其死亡之士,仍款塞請和,不上許之。自是連年犯邊,郭知運、王君相次為河西節度使以捍之。



 吐蕃既自恃兵強,每通表疏,求敵國之禮,言詞悖慢,上甚怒之。及封禪禮畢,中書令張說奏言:「吐蕃醜逆,誠負萬誅,然又事征討,實為勞弊。且十數年甘、涼、河、鄯徵發不息,縱令屬勝,亦不能補。聞其悔過請和,惟陛下遣使。許其稽顙內屬,以息邊境,則蒼生幸甚。」上曰:「待吾與王君籌之。」說出,謂
 源乾曜曰:「君勇而無謀,常思僥幸,兩國和好,何以為勞?若入陳謀,則吾計不遂矣。」尋而君入朝奏事,遂請率兵深入以討之。



 十五年正月,君率兵破吐蕃於青海之西,虜其輜重及羊馬而還。先是,吐蕃大將悉諾邏率眾入攻大斗谷,又移攻甘州,焚燒市里。君畏其鋒,不敢出戰。會大雪,賊凍死者甚眾,遂取積石軍西路而還。君先令人潛入賊境,於其歸路燒草。悉諾邏軍還至大非山,將士息甲牧馬,而野草皆盡,馬死過半。君
 與秦州都督張景順等率眾襲其後,入至青海之西,時海水冰合,將士並乘冰而渡。會悉諾邏已渡大非川,輜重及疲兵尚在青海之側,君縱兵俘之而還。其年九月,吐蕃大將悉諾邏恭祿及燭龍莽布支攻陷瓜州城,執刺史田元獻及王君之父壽,盡取城中軍資及倉糧,仍毀其城而去。又進攻玉門軍及常樂縣,縣令賈師順嬰城固守,凡八十日,賊遂引退。俄而王君為回紇餘黨所殺,乃命兵部尚書蕭嵩為河西節度使,以建康
 軍使、左金吾將軍張守珪為瓜州刺史,修築州城,招輯百姓,令其復業。時悉諾邏恭祿威名甚振,蕭嵩乃縱反間於吐蕃,云其與中國潛通,贊普遂召而誅之。



 明年秋,吐蕃大將悉末朗復率眾攻瓜州,守珪出兵擊走之。隴右節度使、鄯州都督張忠亮引兵至青海西南渴波谷,與吐蕃接戰,大破之。俄而積石、莫門兩軍兵馬總至,與忠亮合勢追討,破其大莫門城,生擒千餘人,獲馬一千匹、犛牛五百頭,器仗衣資甚眾,又焚其駱駝橋而還。八
 月,蕭嵩又遣副將杜賓客率弩手四千人與吐蕃戰於祁連城下,自辰至暮,散而復合,賊徒大潰,臨陣斬其副將一人。賊敗,散走投山,哭聲四合。初,上聞吐蕃重來入寇,謂侍臣曰:「吐蕃驕暴。恃力而來,朕今按地圖。審利害,親指授將帥,破之必矣!」數日而露布至。



 十七年,朔方大總管信安王禕又率兵赴隴右,拔其石堡城,斬首四百餘級,生擒二百餘口,遂於石堡城置振武軍,仍獻其俘囚於太廟。於是吐蕃頻遣使請和,忠王友皇甫惟明因
 奏事面陳通和之便。上曰:「吐蕃贊普往年嘗與朕書,悖慢無禮,朕意欲討之,何得和也!」惟明曰:「開元之初,贊普幼稚,豈能如此。必是在邊軍將務邀一時之功,偽作此書,激怒陛下。兩國既鬥,興師動眾,因利乘便,公行隱盜,偽作功狀,以希勛爵,所損鉅萬,何益國家!今河西、隴右,百姓疲竭,事皆由此。若陛下遣使往視金城公主,因與贊普面約通和,令其稽顙稱臣,永息邊境,此永代安人之道也。」上然其言,因令惟明及內侍張元方充使往問
 吐蕃。惟明、元方等至吐蕃,既見贊普及公主,具宣上意。贊普等欣然請和,盡出貞觀以來前後敕書以示惟明等,令其重臣名悉獵隨惟明等入朝,上表曰:



 外甥是先皇帝舅宿親,又蒙降金城公主,遂和同為一家,天下百姓,普皆安樂。中間為張玄表、李知古等東西兩處先動兵馬,侵抄吐蕃,邊將所以互相征討,迄至今日,遂成釁隙。外甥以先代文成公主、今金城公主之故,深識尊卑,豈敢失禮!又緣年小,枉被邊將讒拘鬥亂,令舅致怪。伏
 乞垂察追留,死將萬足。前數度使人入朝,皆被邊將不許,所以不敢自奏。去冬公主遣使人婁眾失若將狀專往,蒙降使看公主來,外甥不勝喜荷。謹遣諭名悉獵及副使押衙將軍浪些紇夜悉獵入朝,奏取進止。兩國事意,悉獵所知。外甥蕃中已處分邊將,不許抄掠,若有漢人來投,便令卻送。伏望皇帝舅遠察赤心,許依舊好,長令百姓快樂。如蒙聖恩,千年萬歲,外甥終不敢先違盟誓。謹奉金胡瓶一、金盤一、金碗一、馬腦杯一、零羊衫段
 一,謹充微國之禮。



 金城公主又別進金鴨盤盞雜器物等。十八年十月,名悉獵等至京師,上御宣政殿,列羽林仗以見之。悉獵頗曉書記,先曾迎金城公主至長安,當時朝廷皆稱其才辯。及是上引入內宴,與語,甚禮之。賜紫袍金帶及魚袋,並時服、繒彩、銀盤、胡瓶,仍於別館供擬甚厚。悉獵受袍帶器物而卻進魚袋,辭曰:「本國無此章服,不敢當殊異之賞。」上嘉而許之。詔御史大夫崔琳充使報聘。仍於赤嶺各豎分界之碑,約以更不相侵。



 時
 吐蕃使奏云:「公主請《毛詩》、《禮記》、《左傳》《文選》各一部。」制令秘書省寫與之。正字於休烈上疏請曰:



 臣聞戎狄,國之寇也;經籍,國之典也。戎之生心,不可以無備;典有恆制,不可以假人。《傳》曰:「裔不謀夏,夷不亂華。」所以格其非心,在乎有備無患。昔東平王入朝求《史記》、諸子,漢帝不與。蓋以《史記》多兵謀,諸子雜詭術。夫以東平,漢之懿戚,尚不欲示征戰之書,今西戎,國之寇讎,豈可貽經典之事!



 且臣聞吐蕃之性,剽悍果決,敏情持銳,善學不回。若達
 於書,必能知戰。深於《詩》,則知武夫有師干之試;深於《禮》,則知月令有興廢之兵;深於《傳》,則知用師多詭詐之計;深於《文》,則知往來有書檄之制。何異借寇兵而資盜糧也!



 臣聞魯秉周禮,齊不加兵;吳獲乘車,楚疲奔命。一以守典存國,一以喪法危邦,可取鑒也。且公主下嫁從人,遠適異國,合慕夷禮,返求良書,愚臣料之,恐非公主本意也。慮有奔北之類,勸教於中。若陛下慮失蕃情,以備國信,必不得已,請去《春秋》。當周德既衰,諸侯強盛,禮樂
 自出,戰伐交興,情偽於是乎生,變詐於是乎起,則有以臣召君之事,取威定霸之名。若與此書,國之患也。《傳》曰:「於奚請曲縣鞶纓,仲尼曰:『惜也,不如多與之邑。惟名與器,不可假人。』」狄固貪婪,貴貨易土,正可錫之錦綺,厚以玉帛,何必率從其求,以資其智!臣忝叨列位,職刊秘籍,實痛經典,棄在戎夷。昧死上聞,惟陛下深察。



 疏奏不省。二十一年,又制工部尚書李皓往聘吐蕃。每唐使入境,所在盛陳甲兵及騎馬,以矜其精銳。二十二年,遣將軍
 李佺於赤嶺與吐蕃分界立碑。二十四年正月,吐蕃遣使貢方物金銀器玩數百事,皆形制奇異。上令列於提象門外,以示百僚。



 其年,吐蕃西擊勃律,遣使來告急。上使報吐蕃,令其罷兵。吐蕃不受詔,遂攻破勃律國,上甚怒之。時散騎常侍崔希逸為河西節度使,於涼州鎮守。時吐蕃與漢樹柵為界,置守捉使。希逸謂吐蕃將乞力徐曰:「兩國和好,何須守捉,妨人耕種。請皆罷之,以成一家豈不善也?」乞力徐報曰:「常侍忠厚,必是誠言。但恐朝
 廷未必皆相信任。萬一有人交拘,掩吾不備,後悔無益也。」希逸固請之,遂發使與乞力徐殺白狗為盟,各去守備。於是吐蕃畜牧被野。俄而希逸傔人孫誨入朝奏事,誨欲自邀其功,因奏言「吐蕃無備,若發兵掩之,必克捷。」上使內給事趙惠琮與孫誨馳往觀察事宜。惠琮等至涼州,遂矯詔令希逸掩襲之,希逸不得已而從之,大破吐蕃於青海之上,殺獲甚眾,乞力徐輕身遁逸。惠琮、孫誨皆加厚賞,吐蕃自是復絕朝貢。希逸以失信怏怏,在
 軍不得志。俄遷為河南尹,行至京師,與趙惠琮俱見白狗為祟,相次而死。孫誨亦以罪被戮。詔以岐州刺史蕭炅為戶部侍郎判涼州事,代希逸為河西節度使;鄯州都督杜希望為隴右節度使;太僕卿王昊為益州長史、劍南節度使,分道經略,以討吐蕃。仍令毀其分界之碑。



 二十六年四月,杜希望率眾攻吐蕃新城,拔之。以其城為威武軍,發兵一千以鎮之。其年七月,希望又從鄯州發兵奪吐蕃河橋,於河左築鹽泉城。吐蕃將兵三萬人
 以拒官軍,希望引眾擊破之,因於鹽泉城置鎮西軍。時王昊又率劍南兵募攻其安戎城。先於安戎城左右築兩城,以為攻拒之所,頓兵於蓬婆嶺下,運劍南道資糧以守之。其年九月,吐蕃悉銳以救安戎城,官軍大敗,兩城並為賊所陷,昊脫身走免,將士已下數萬人及軍糧資仗等並沒於賊。昊坐左遷括州刺史。初,昊之在軍,謬賞其子錢帛萬計,並擅與紫袍等,所費鉅萬,坐是尋又重貶為端州高要尉而死。



 二十七年七月,吐蕃又寇白
 草、安人等軍,敕臨洮、朔方等軍分兵救援。時吐蕃於中路屯兵,斷臨洮軍之路。白水軍守捉使高柬于拒守連旬,俄而賊退,蕭炅遣偏將掩其後,擊破之。王昊既敗之後,詔以華州刺史張宥為益州長史、劍南防禦使,主客員外郎章仇兼瓊為益州司馬、防禦副使。宥既文吏,素無攻戰之策,兼瓊遂專其戎事。俄而兼瓊入奏,盛陳攻取安戎之策。上甚悅,徙張宥為光祿卿,拔兼瓊令知益州長史事,代張宥節度,仍為之親畫取城之計。



 二十八
 年春,兼瓊密與安戎城中吐蕃翟都局及維州別駕董承宴等通謀。都局等遂翻城歸款,因引官軍入城,盡殺吐蕃將士,使監察御史許遠率兵鎮守。上聞之,甚悅。中書令李林甫等上表曰:「伏以吐蕃此城,正當沖要,憑險自固,恃以窺邊。積年以來,蟻聚為患,縱有百萬之眾,難以施功。陛下親紆秘策,不興師旅,頃令中使李思敬曉喻羌族,莫不懷恩,翻然改圖,自相謀陷。神算運於不測,睿略通於未然,累載逋誅,一朝蕩滅。又臣等今日奏事,
 陛下從容問臣等曰:『卿等但看四夷不久當漸摧喪。』德音才降,遽聞戎捷,則知聖與天合,應如響至,前古以來,所未有也。請宣示百僚,編諸史策。」手制答曰:「此城儀鳳年中羌引吐蕃,遂被固守,歲月既久,攻伐亦多。其地險阻,非力所制。朝廷群議,不合取之。朕以小蕃無知,事須處置,授以奇計,所以行之,獲彼戎心,歸我城守,有足為慰也。」其年十月,吐蕃又引眾寇安戎城及維州,章仇兼瓊遣裨將率眾御之,仍發關中彍騎以救援焉。時屬凝
 寒,賊久之自引退。詔改安戎城為平戎城。



 二十九年春,金城公主薨,吐蕃遣使來告哀,仍請和,上不許之。使到數月後,始為公主舉哀於光順門外,輟朝三日。六月,吐蕃四十萬攻承風堡,至河源軍,西入長寧橋,至安仁軍,渾崖峰騎將盛希液以眾五千攻而破之。十二月,吐蕃又襲石堡城,節度使蓋嘉運不能守,玄宗憤之。天寶初,令皇甫惟明、王忠嗣為隴右節度,皆不能克。七載,以哥舒翰為隴右節度使,攻而拔之,改石堡城為神武軍。



 天
 寶十四載,贊普乞黎蘇籠獵贊死,大臣立其子婆悉籠獵贊為主,復為贊普。玄宗遣京兆少尹崔光遠兼御史中丞,持節齎國信冊命吊祭之。及還,而安祿山已竊據洛陽,以河、隴兵募令哥舒翰為將,屯潼關。



 昔秦以隴山已西為隴西郡。漢懷匈奴於河右,置姑臧、張掖、酒泉、伊吾等郡;又於磧外置西域都護,控引胡國;又分隴西為金城、西平等郡,雜以氐、羌居之。歷代喪亂,不為賢豪所據,則為遠夷侵廢;迨千年矣。武德初。薛仁杲奄有隴上
 之地,至於河虜。李敷盡有涼州之域,通於磧外。貞觀中,李靖破吐谷渾,侯君集平高昌,阿史那社爾開西域,置四鎮。前王之所未伏,盡為臣妾,秦、漢之封域,得議其土境耶!於是歲調山東丁男為戍卒,繒帛為軍資,有屯田以資糗糧,牧使以娩羊馬。大軍萬人,小軍千人,烽戍邏卒,萬里相繼,以卻於強敵。隴右鄯州為節度。河西涼州為節度。安西、北庭亦置節度,關內則於靈州置朔方節度,又有受降城、單于都護庭為之籓衛。及潼關失守,河
 洛阻兵,於是盡征河隴、朔方之將鎮兵入靖國難,謂之行營。曩時軍營邊州無備預矣。乾元之後,吐蕃乘我間隙,日蹙邊城,或為虜掠傷殺,或轉死溝壑。數年之後,鳳翔之西,邠州之北,盡蕃戎之境,淹沒者數十州。



 肅宗元年建寅月甲辰,吐蕃遣使來朝請和,敕宰相郭子儀、蕭華、裴遵慶等於中書設宴。將詣光宇寺為盟誓,使者云:蕃法盟誓,取三牲血歃之,無向佛寺之事,請明日須於鴻臚寺歃血,以申蕃戎之禮。從之。寶應元年六月,吐蕃使燭番、莽耳等二
 人貢方物入朝,乃於延英殿引見,勞賜各有差。而劍南西山又與吐蕃、氐、羌鄰接,武德以來,開置州縣,立軍防,即漢之笮路,乾元之後,亦陷於吐蕃。寶慶二年三月,遣左散騎常侍兼御史大夫李之芳、左庶子兼御史中丞崔倫使於吐蕃,至其境而留之。



 廣德元年九月,吐蕃寇陷涇州。十月,寇邠州,又陷奉天縣。遣中書令郭子儀西御。吐蕃以吐谷渾、黨項羌之眾二十餘萬,自龍光度而東。郭子儀退軍,車駕幸陜州,京師失守。降將高暉引吐
 蕃入上都城,與吐蕃大將馬重英等立故邠王男廣武王承宏為帝,立年號,大赦,署置官員,尋以司封崔瑰等為相。郭子儀退軍南保商州,吐蕃居城十五日退,官軍收上都,以郭子儀為留守。



 初,車駕東幸,衣冠戚里盡南投荊襄及隱竄山谷,於是六軍將士持兵剽劫,所在阻絕。郭子儀領部曲數百人及其妻子僕從南入牛心谷,駝馬車牛數百兩,子儀遲留,未知所適。行軍判官、中書舍人王延昌、監察御史李萼謂子儀曰:「令公身為元帥,
 主上蒙塵於外,家國之事,一至於此。今吐蕃之勢日逼,豈可懷安於谷中,何不南趨商州,漸赴行在。」子儀遽從之。延昌曰:「吐蕃知令公南行,必分兵來逼,若當大路,事即危矣。不如取玉山路而去,出其不意。」子儀又從之。延昌與李萼皆從子儀,子儀之隊千餘人,山路狹隘,連延百餘里,人不得馳。延昌與萼恐狹徑被追,前後不相救,至倒回口,遂與子儀別行,逾絕澗,登七盤,趨於商州。先是,六軍將張知節與麾下數百人自京城奔於商州,大掠
 避難朝官、士庶及居人資財鞍馬,已有日矣。延昌與萼既至,說知節曰:「將軍身掌禁兵,軍敗而不赴行在,又恣其下虜掠,何所歸乎?今郭令公元帥也,已欲至洛南,將軍若整頓士卒,喻以禍福,請令公來撫之,以圖收長安,此則將軍非常之功也。」知節大悅。其時諸軍將臧希讓、高升、彭體盈、李惟詵等數人,各有部曲,率其數十騎,相次而至,又從其計,皆相率為軍,約不侵暴。延昌留於軍中主約,萼以數騎往迎子儀,去洛南十餘里,及之,遂與
 子儀回至商州。諸將大喜,皆遵其約束。



 吐蕃將入京師也,前光祿卿殷仲卿逃難而出,鞍馬衣服盡為土賊所掠。仲卿至藍田,糾合散兵及諸驍勇願從者百餘人,南保藍田,以拒吐蕃,其眾漸振,至於千人。子儀既至商州,未知仲卿之舉,募人往探賊勢。羽林將軍長孫全緒請行,以二百騎隸之。又令太子賓客第五琦攝京兆尹,同收長安。全緒至韓公堆,晝則擊鼓,廣張旗幟,夜則多燃火,以疑吐蕃。仲卿探知官軍,其勢益壯。遂相為表裏,以
 狀聞於子儀。仲卿帥二百餘騎游奕,直渡滻水。吐蕃懼,問百姓,百姓皆紿之曰:「郭令公自商州領眾卻收長安,大軍不知其數。」賊以為然,遂抽軍而還,餘眾尚在城。軍將王撫及御史大夫王仲升頓兵自苑中入,椎鼓大呼,仲卿之師又入城,吐蕃皆奔走,乃收上都。郭子儀乘之,鼓行入長安,人心乃安。



 吐蕃退至鳳翔,節度孫志直閉門拒之,吐蕃圍守數日。會鎮西節度、兼御史中丞馬璘領精騎千餘自河西救楊志烈回,引兵入城。遲明,單騎持
 滿,直沖賊眾,左右願從者百餘騎。璘奮擊大呼,賊徒披靡,無敢當者,賊疲而歸。賊眾恃其驍勇,翌日又逼城請戰。璘披甲開懸門,賊乃抽退。皆曰:「此將不惜死不可當,且避之。」又復居原、會、成、渭之地。



 十二月,乘輿還上都。二年五月,放李之芳還。九月,叛將僕射、大寧郡王僕固懷恩自靈武遣其黨範志誠、任敷等引吐蕃、吐谷渾之眾來犯王畿。十月,懷恩之眾至邠州挑戰,節度白孝德及副元帥先鋒郭鋒嬰城拒之,以挫其鋒。賊眾遂逼奉天
 縣西二十里為營,郭子儀屯於奉天,又按軍不戰。郭鋒於邠州西三十里,令精騎二百五十人、步卒五十人斫懷恩營,破五千眾,斬首千餘級,生擒八十五人,降其大將四人,為五百匹。十一月,僕固懷恩引吐蕃之眾退。



 廣德二年,河西節度楊志烈被圍,守數年,以孤城無援,乃跳身西投甘州,涼州又陷於寇。



 永泰元年三月,吐蕃請和,遣宰相元載、杜鴻漸等於興唐寺與之盟而罷。秋九月,僕固懷恩誘吐蕃、回紇之眾,南犯王畿。吐蕃大將尚
 結息贊磨、尚息東贊、尚野息及馬重英率二十萬眾至奉天界,邠州節度使白孝德不能御,京城戒嚴。先是,朔方先鋒兵馬使渾日進、孫守亮屯軍於奉天以拒之,於是詔追副元帥郭子儀於河中府領眾赴援,屯於涇陽,諸將各屯守要害。初,吐蕃列營奉天,渾日進單騎沖之,驍騎二百人繼進,沖突其營,左右擊刺,賊徒驚駭,無不應弦而斃。日進挾一蕃將,躍馬而歸,蕃將奮身,失其撒飯一。日進之眾,無中鋒鏑者,軍中望而益振。明日,吐蕃
 悉眾圍之,日進命拋車夾石投之,雜以弓弩,賊死傷眾。數日,斂軍回營。尋又日進夜斫賊營於梁母神下,殺千餘人,生擒五百人,獲駝馬器械。



 上又下詔親征,括朝官馬,京城置團練。鎮西節度馬璘遇吐蕃游奕四百餘人於武功東原,使五十人擊而盡殺之,無噍類。自十七日至二十五日晚際始止,議者以為天助。吐蕃移營於醴泉縣九飀山北,因攻掠醴泉。京城大駭,人皆空室,大戶鑿竇以出。逆黨任敷以兵五千餘人犯白水縣。渾日進
 露布而至,屯於奉天馬嵬店。今月十九日已後至二十五日已前,交戰二百餘陣,破吐蕃一萬餘眾,斬首五千級,生擒一百六十人,馬一千二百四十二匹,駝一百一十五頭,器械、幡旗共三萬餘事。朝官震懼,家口回避者十室八九,禁之不止。自前年吐蕃犯王畿後,於中渭橋鄠豐城以營兵,至是功畢。



 吐蕃退至永壽北,遇回紇之眾,雖聞懷恩死,皆悖其眾,相誘而奔,復來寇。至奉天,兩蕃猜貳爭長,別為營壘。吐蕃游奕至窯底,吐蕃又至馬
 嵬店,因縱火焚居人廬舍而退。回紇三千騎詣涇陽降款,請擊吐蕃為效,子儀許之。於是朔方先鋒兵馬使開府南陽郡王白元光與回紇合於涇陽靈臺縣東五十里,攻破吐蕃。斬首及生擒獲駝馬牛羊甚眾。上停親征,京師解嚴,宰相上表稱賀。



\end{pinyinscope}