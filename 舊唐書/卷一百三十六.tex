\article{卷一百三十六}

\begin{pinyinscope}

 ○李抱玉李
 抱真王虔休盧從史李芃李澄族弟元素



 李抱玉,武德功臣安興貴之裔。代居河西,善養名馬,為時所稱。群從兄弟,或徙居京華,習文儒,與士人通婚者,
 稍染士風。抱玉少長西州,好騎射,常從軍幕,沉毅有謀,小心忠謹。



 乾元初,太尉李光弼引為偏裨,屢建勛績,由是知名。二年,自特進、右羽林軍大將軍、知軍事,遷鴻臚卿員外置同正員,持節鄭州諸軍事兼鄭州刺史、攝御史中丞、鄭陳潁亳四州節度。時史思明陷洛陽,光弼守河陽,賊兵鋒方盛,光弼謂抱玉曰:「將軍能為我守南城二日乎?」抱玉曰:「過期若何?」光弼曰:「過期而救不至,任棄城也。」賊帥周摯領安太清、徐黃玉等先次南城,將陷之,
 抱玉乃紿之曰:「吾糧盡,明日當降。」賊眾大喜,斂軍以俟之。抱玉因得繕完設備,明日,堅壁請戰。賊怒欺紿,急攻之。抱玉出奇兵,表裏夾攻,殺傷甚眾,摯軍退。光弼自將於中水單城,摯舍南城攻中水單,不勝,乃整軍將攻北城。光弼以兵出戰,大敗之。固河陽,復懷州,皆功居第一,遷澤州刺史、兼御史中丞。代宗即位,擢為澤潞節度使、潞州大都督府長史、兼御史大夫,加領陳、鄭二州,遷兵部尚書。抱玉上言:「臣貫屬涼州,本姓安氏,以祿山構禍,恥與
 同姓,去至德二年五月,蒙恩賜姓李氏,今請割貫屬京兆府長安縣。」許之,因是舉宗並賜國姓。



 廣德元年冬,吐蕃寇京師,乘輿幸陜,諸軍潰卒及村閭亡命相聚為盜,京城南面子午等五穀群盜頗害居人,朝廷遣薛景仙領兵為五穀使招討,連月不捷,乃詔抱玉兼鳳翔節度使討之。抱玉探知賊帥行止之處,先分屯諸谷,乃設奇潛使輕銳數百南自洋州入攻之。賊帥高玉方與諸偷會,遽為銳卒數十人掩擒之,因大搜獲偷黨,悉斬之,餘
 黨不討自潰,旬日內五穀平。以功遷司空,餘並如故。



 時吐蕃每歲犯境,上以岐陽國之西門,寄在抱玉,恩寵無比,遷同中書門下平章事,又兼山南西道節度使、河西隴右山南西道副元帥、判梁州事,連統三道節制,兼領鳳翔、潞、梁三大府,秩處三公。抱玉以任位崇重,抗疏懇讓司空及山南西道節度、判梁州事,乞退授兵部尚書。上嘉其謙讓,許之。抱雲凡鎮鳳翔十餘年,雖無破虜之功,而禁暴安人,頗為當時所稱。大歷十二年卒,上甚悼
 之,輟朝三日,贈太保。



 李抱真,抱玉從父弟也。抱玉為澤潞節度使,甚器抱真,任以軍事,累授汾州別駕。當是時,僕固懷恩反於汾州,抱真陷焉,乃脫身歸京師。代宗以懷恩倚回紇,所將朔方兵又勁,憂甚,召見抱真問狀,因奏曰:「郭子儀領朔方之眾,人多思之。懷恩欺其眾,曰『子儀為朝恩所殺』,詐而用之。今復子儀之位,可不戰而克。」其後懷恩子瑒為其下所殺,懷恩奔遁,多如抱真策,因是遷殿中少監。居頃
 之,為陳鄭、澤潞節度留後,抱真因中謝言曰:「臣雖無可取,當今百姓勞逸,系在牧守,願得一郡以自試。」上許之,改授澤州刺史,兼為澤潞節度副使。居二年,轉懷州刺史,復為懷澤潞觀察使留後,凡八年。抱玉卒,抱真仍領留後。抱真密揣山東當有變,上黨且當兵沖,是時乘戰餘之地,土瘠賦重,人益困,無以養軍士。籍戶丁男,三選其一,有材力者免其租徭,給弓矢,令之曰:「農之隙,則分曹角射;歲終,吾當會試。」及期,按簿而征之,都試以示賞
 罰,復命之如初。比三年,則皆善射,抱真曰:「軍可用矣。」於是舉部內鄉兵,得成卒二萬,前既不廩費,府庫益實,乃繕甲兵,為戰具,遂雄視山東。是時,天下稱昭義軍步兵冠諸軍。無幾,復代李承昭為昭義軍及磁邢節度觀察留後,加散騎常侍。



 德宗即位,拜檢校工部尚書,兼潞州長史、昭義軍節度支度營田、澤潞磁邢觀察使。建中二年,田悅以魏博反,乃悉兵圍邢州及臨洺益急,詔河東節度使馬燧及神策兵救之。抱真與燧敗悅兵於雙岡,斬
 悅將楊朝光,又擊破悅於臨洺,遂解臨洺及邢州之圍,以功加檢校兵部尚書。復與燧大破悅於洹水,悅以數百騎走歸魏州。復與燧圍魏州,又敗悅於城下,以功加檢校右僕射。時悅窘蹙,硃滔、王武俊皆反,聯兵救悅,抱真與燧等退次魏縣。上幸奉天,中使告問至,諸將皆仰天慟哭。李懷光席卷奔命,馬燧、李芃各引兵歸鎮。硃泚既汙宮闕,時李希烈陷大梁,李納亦反鄆州。無何,上幸梁州,李懷光又竊據河中。抱真獨於擾攘傾潰之中,以
 山東三州外抗群賊,內輯軍士,群賊深憚之。



 興元初,遷檢校左僕射、平章事。時硃滔悉幽薊軍,借兵回紇,擁眾五萬,南向以應泚,攻圍貝州。初,群賊附於希烈,希烈僭偽,有臣屬群賊意,群心稍離。上自奉天下罪己之詔,悉赦群賊,抱真乃遣門客賈林以大義說武俊,合從擊硃滔,武俊許之。時兩軍尚相疑,抱真乃以數騎徑入武俊營。其將去也,賓客皆止之,抱真遣軍司馬盧玄卿勒軍部分曰:「僕今日此舉,系天下安危。僕死不還,領軍事以
 聽朝命,亦唯子;奮勵士馬,東向雪僕之恥,亦唯子。」言訖而去。武俊設備甚嚴,抱真曰:「硃泚、希烈僭竊大位,硃滔攻圍貝州,此輩皆欲陵駕吾屬。足下既不能自振數賊之上,舍九葉天子而北面臣反虜乎?乃者聖上奉天下罪己之詔,可謂禹、湯之主也。」因言及播越,持武俊哭,涕泗交下,武俊亦哭,感動左右。因退臥武俊帳中,酣寢久之。武俊感其不疑,待之益恭,指心仰天曰:「此身已許公死敵矣。」遂與結為兄弟而別,約明日合戰,遂擊破硃滔
 於經城,以功加檢校司空,實封五百戶。貞元初,朝於京師,居頃之,還鎮。



 抱真沉斷多智計,嘗欲招致天下賢俊,聞人之善,必令持貨幣數千里邀致之;至與語無可採者,漸退之。時天下無事,乃大起臺榭,穿池沼以自娛。晚節又好方士,以冀長生。有孫季長者,為抱真練金丹,紿抱真曰:「服之當升仙」遂署為賓僚。數謂參佐曰:「此丹秦皇、漢武皆不能得,唯我遇之,他年朝上清,不復偶公輩矣。」復夢駕鶴沖天,寤而刻木鶴、衣道士衣以習乘之。凡
 服丹二萬丸,腹堅不食,將死,不知人者數日矣。道士牛洞玄以豬肪穀漆下之,殆盡。病少間,季長復曰:「垂上仙,何自棄也!」益服三千丸,頃之卒。初,抱真久疾,好示幾祥,或令厭勝,為巫祝所惑,請降官爵以禳除之。是年,凡七上章讓司空,復為檢校左僕射。貞元十年卒,時年六十二,廢朝三日,贈太保,賻以布帛米粟有差。



 抱真薨之日,其子殿中侍御史緘匿喪不發。營田副使盧會昌令抱真從甥元仲經潛與緘謀,其明日,將吏會集,仲經詐為抱
 真令曰:「吾疾甚,不能蒞職,今令緘掌軍事,諸軍善佐之。」節度副使李說及諸將吏俯首,皆曰:「諾」。須臾,緘盛服而出,眾皆拜之,緘乃悉府藏頒賞軍士。盧會昌仍詐為抱真表,請以職事付緘。翌日,又令諸將連奏請緘領軍。上已聞真疾病,請見明日。如此者凡三日,緘乃出造中使,左右皆陳兵,甚嚴備。中使謂緘曰:「朝廷已知相公薨歿,令以兵務屬延
 貴,侍御宜歸發喪行服也。」緘愕然,出謂諸將曰:「有詔不許緘掌事,諸公意若何?」將吏莫有對者。緘懼而退,遽以使印及管鑰歸監軍。是日,乃發喪,畢一哭。中使召延貴,以口詔令視事,趣遣緘赴東都。元仲經逃於外,延貴捕得殺之。既歸罪仲經,盧會昌得不坐。緘初謀亂,遣裨將陳榮詐以文書告成德節度使王武俊,求假財帛,武俊大怒曰:「吾與汝府公善者,冀恭王命,非同惡也。今聞已亡,孰詐令其子而不俟朝旨耶?何敢告我,況有求也!」乃
 囚陳榮而遣使讓緘焉。



 王虔休,字君佐,汝州梁人也。本名延貴。少涉獵書籍,鄉里間以信義畏慕之,尤好武藝。大歷中,汝州刺史李深用之為將。久之,澤潞節度李抱真聞名,厚以財帛招之,累授兵馬使押衙。建中初,抱真統兵馬與諸將征討河北,其雙岡、水寨營等陣,虔休攻戰居多,擢為步軍都虞候,累加兼御史中丞、大夫,賜實封百戶。洎抱真卒,裨將元仲經等議立抱真子緘,軍中擾亂,虔休正色言於眾
 曰:「軍州是天子軍州,將帥闕,合待朝命,何乃云云,妄生異意!」軍中服從其言,由是竟免潰亂。朝廷知而嘉之,以邕王為昭義節度觀察大使,授虔休潞州左司馬,依前兼御史大夫,掌留後,仍賜名虔休。號令安撫。軍州大理。二歲,遷潞州長史、昭義軍節度、澤潞磁邢洺觀察使,尋加檢校工部尚書。貞元十五年卒,年六十二。廢朝三日,贈左僕射,賻以布帛米粟。



 虔休性恭勤,儉省節用,管內州倉庾皆積糧儲,可支軍人數歲。又嘗撰《誕聖樂曲》以
 進,其表曰:



 臣聞於師,夫君子為能知樂,是故審音以知聲,審樂以知政,則理道備矣。清明廣大,終始周旋,與天地同其和,與四時合其序,豈止於鐘鼓管磬云乎哉!臣伏見開元中天長節著於甲令,每於是日海縣歡娛,稱萬壽之無疆,樂一人之有慶,故能追堯接舜,邁禹逾湯,自周已後,不能議矣。臣竊以陛下降誕之辰,未有惟新之曲。雖太和已布於六氣,而大樂未宣於八音,無乃臣子之分,或有所闕。愚臣不揆頑昧,敢思祖述,每思歌竊
 抃,忘寢與食久矣。適遇有知音者,與臣論及樂章,探微賾奧,窮理盡性,臣乃遣造《繼天誕聖樂》一曲。大抵以宮為調,表五音之奉君也;以土為德,知五運之居中也。凡二十五遍,法二十四氣而足成一歲也。每遍一十六拍,象八元、八凱登庸於朝也。所冀《雲門》、《咸池》,永傳於律呂,空桑、孤竹,合薦於宮懸,不聞沾滯之聲,長作中和之樂。可使九域之人,頓忘於肉味;四夷之俗,皆播於薰風。與唐惟休,終古盡善。臣不勝懇款屏營之至,謹昧死陳獻
 以聞。其所造譜,謹同封進。



 先時,有太常樂工劉玠流落至潞州,虔休因令造此曲以進,今《中和樂》起此也。



 盧從史,其先自元魏已來,冠冕頗盛。父虔,少孤,好學,舉進士,歷御史府三院、刑部郎中、江汝二州刺史、秘書監。從史少矜力,習騎射,游澤、潞間,節度使李長榮用為大將。德宗中歲,每命節制,必令採訪本軍為其所歸者。長榮卒,從史因軍情,且善迎奉中使,得授昭義軍節度使。漸狂恣不道,至奪部將妻妾,而辯給矯妄,從事孔戡等
 以言直不從引去。前年丁父憂,朝旨未議起復,屬王士真卒,從史竊獻誅承宗計,以希上意,用是起授,委其成功。及詔下討賊,兵出,逗留不進,陰與承宗通謀,令軍士潛懷賊號,又高其芻粟之價,售於度支,諷朝廷求宰相;且誣奏諸軍與賊通,兵不可進。上深患之。



 護軍中尉吐突承璀將神策兵與之對壘,從史往往過其營博戲。從史沓貪好得,承璀出寶帶、奇玩以炫燿之,時其愛悅而遺焉,從史喜甚,日益狎。上知其事,取裴垍之謀,因戒承
 璀伺其來博,揖語,幕下伏壯士,突起,持捽出帳後縛之,內車中,馳以赴闕。從者驚亂,斬十數人,餘號令乃定,且宣諭密詔,追赴闕庭。都將烏重胤素懷忠順,乃嚴戒其軍,眾不敢動。會夜,使疾驅,未明出境,道路人莫知。元和五年四月,制曰:



 邪以蓄眾,自致覆車;奸以事君,所宜用鉞。故楚人告變,韓信患釋於事先;蜀土征災,鐘會禍生於部下。況害深楚、蜀,功匪鐘、韓,構此厲階,布於公議。懷私負德,合置於嚴科;屈法申恩,尚從於寬典。前昭義軍
 節度副大使、知節度事盧從史,擢自裨將,居於大籓,不思報國之誠,每設徇身之計。比丁家禍,曾無戚容,行棄人倫,孝虧大性。屬常山稱亂,朝制未行,固願興師,茍求復位。刻期效用,請以身先;指日投誠,誓云獨致。示於懷撫,推以信誠。排眾論以釋其苴麻,決中心而授之鈇鉞,委以重任,命之專征。章奏所陳,事無違者;恩光是貸,予何愛焉。而乃冒利蓄奸,隳政敗度,成師既出,保敵而交通;邪計以行,臨戎而向背。諸侯盡力而不應,遺寇游魂
 而是托。臣節既喪,恩豈念於生成;臺位於求,禮頓虧於忠敬。肆其醜行,熾以兇威,至於逼脅軍中,潛施賊號;陵污麾下,實玷皇風。貨以籓身,虐而用眾,士庶怨而罔恤,將校勞而不圖。稟於陶鈞,行事至此,視於天地,負我何多,且辜覆載之仁,寧逭神鬼之責。況頃年上請,就食山東,及遣旋師,不時恭命,致動其眾,覬生其心,賴劉濟抗忠正之辭,使邪豎絕遲回之計。加以遍毀鄰境,密疏事情,反覆百端,高下萬變,心無恥愧,事至滿盈。朕念以始
 終,務於含貸,所期悔過,豈謂逾兇。而昭義軍忠節夙彰,義聲昭著,發其眾怒,葉以一心,顧大惡而不容,幸全軀而自免,宜從大戮,以正彞章。尚以曾列方隅,嘗經任使,惜君臣之體,抑中外之情,俾投魑魅之鄉,以解人神之憤。可貶驩州司馬。嗚呼!奸由事驗,自開棄絕之門;禍實己招,豈漏恢疏之網。凡百多士,宜諒朕懷。



 子繼宗等四人並貶嶺外。



 李芃,字茂初,趙郡人也。解褐上邽主簿,三遷試大理評
 事,攝監察御史、山南東道觀察支使。嚴武為京兆尹,舉為長安尉。李勉為江西觀察使,署奏秘書郎、兼監察御史,為判官。永泰初,轉兼殿中侍御史。



 時宣、饒二州人方清、陳莊聚眾據山洞,西絕江路,劫商旅以為亂。芃乃請於秋浦置州,守其要地,以破其謀。李勉然其計,以聞,代宗嘉之,以宣州之秋浦、青陽、饒州之至德置池州焉。芃攝行事,無幾,乃兼侍御史。居無何,魏少游代勉為使,復署奏檢校虞部員外郎,賜金紫,為都團練副使。頃之,
 攝江州刺史,州人便之。丁母憂,免喪,永平軍節度李勉署奏檢校工部郎中、兼侍御史,為判官,尋攝陳州刺史。歲中,即值李靈曜反於汴州,勉署芃兼亳州防禦使,練達軍事,兵備甚肅;又開陳、潁運路,以通漕輓。



 德宗嗣位,授檢校太常少卿、兼御史中丞、河陽三城鎮遏使。撫勞備至,資廩善者,必先軍士。間一年,為節度使路嗣恭之副,加檢校左庶子、河陽三城懷州節度觀察使,以東畿汜水等五縣隸焉。時河南北連大兵,詔益以神策、汝、陜
 之師。芃進收新鄉、共城,遂圍衛州。明年,詔與河東節度馬燧等諸軍破田悅於洹水,以功加檢校兵部尚書,累封開郡王,實封一百戶。進圍悅於魏州,將符璘以精騎五百夜降,芃耳開營以納之。明日,歸璘於招討使。上居奉天,斂軍還。



 興元初,檢校右僕射,無何,以疾固讓罷歸。芃將請告,謂所親曰:「今年夏被蝗旱,人主厭兵革,然則天下城壘堅厚矣,戈鋋銛利矣,以力勝之,則有得失,其可盡乎!除弊之急,莫先德化,循而理之,斯易致耳。方鎮之
 戴翼時主,宜先退讓,貪權持祿,吾所不取也。吾既疾病,豈能言而不踐乎!」乃手疏乞罷。貞元元年卒,年六十四,廢朝一日,贈太子太保。



 李澄,遼東襄平人,隋蒲山公寬之後也,居京兆。父鎬,清江太守,以澄贈工部尚書。澄以武藝為偏將,累除試將作監,隸於江淮都統李峘。建中初,以檢校太子賓客、兼御史中丞隸於永平軍節度使李勉。及勉移理汴州,乃奏澄為滑州刺史。四年冬,李希烈陷汴州,勉奔歸行在,
 澄遂以城降希烈,偽署尚書令,兼滑州永平軍節度使。



 興元元年春,澄密令親信人盧融間道賚表達於奉天,上嘉之,乃以帛詔藏於蠟丸中,加澄刑部尚書,兼汴州刺史、汴滑節度觀察使。澄秘而未宣,乃集州兵嚴加訓習。希烈頗疑之,乃令養子六百人戍之,以虞其變。希烈苦攻寧陵,邀澄率其眾至石柱。澄令縱火焚營,而偽遁,誘六百人因驚行剽而加其罪,果大俘掠,悉令斬之以告。希烈不能窮詰焉。無凡,希烈遣其將翟暉等寇陳州,
 久之未復。是歲十月,澄以汴州兵寡,度希烈不能制己,又會中官薛盈珍持節且至,加檢校兵部尚書,封武威郡王,賜實封五百戶。澄乃乘勢力焚賊旌節,誓眾歸國。及十一月,希烈既失澄,又聞翟暉大敗,由是奔歸蔡州。澄遽率眾將復汴州,屯於城北門,恇怯不敢進。及宣武軍節度使劉洽師至城東門,賊將田懷珍開關以納之。翌日,澄方自北入,洽已據子城。澄乃舍於浚儀縣,兩軍將士,日有忿競,不自安。會鄭州賊將孫液通款於澄,澄
 遣其子清赴之。先是,河陽軍節度使李芃遣其將雍顥攻鄭州,顥所過縱掠,液拒之尤固;及清至,遂納之。顥怒攻液,清以眾助之,殺登城者數十人,顥方引退,又焚陽武而歸。澄乃出赴鄭州,朝廷特授清檢校太子賓客、兼御史中丞,更名克寧。



 貞元元年三月,就加澄檢校左僕射、義成軍鄭滑許等州節度使。二年卒,年五十四,廢朝一日,贈司空,賵布帛粟有差,仍令左散騎常侍歸崇敬充吊祭使,所緣喪葬,並勒官給。澄實以八月癸未終,克
 寧秘之,以九月庚寅,欲自起視事。其行軍司馬馬鉉不許,克寧陰遣殺之,乃墨絰而出,加卒於城門,將為不順。劉洽出師屯於境上以制之,且使告諭切至,由是克寧不敢妄發,然道路絕商旅者凡十四五日。及賈躭代澄,克寧護喪將歸,乃悉索府中財貨,以夜出城,軍人從而剽奪,及明殆盡。澄柩至京師,又賜克寧莊一所、錢千貫、粟麥二千石。澄初封隴西郡公,進武威郡王,每上疏連稱二封,頗為時人所哂。



 李元素,字大樸,蒲山公密之孫。任侍御史,時杜亞為東都留守,惡大將令狐運,會盜發洛城之北,運適與其部下畋於北郊,亞意其為盜,遂執訊之,逮系者四十餘人。監察御史楊寧按其事,亞以為不直,密表陳之,寧遂得罪。亞將逞其宿怒,且以得賊為功,上表指明運為盜之狀,上信而不疑。宰臣以獄大宜審,奏請覆之,命元素就決,亞迎路以獄成告。元素驗之五日,盡釋其囚以還。亞大驚,且怒,親追送,馬上責之,元素不答。亞遂上疏,又誣
 元素。元素還奏,言未畢,上怒曰:「出俟命。」元素曰:「臣未盡詞。」上又曰:「且去」。元素復奏曰:「一出不得復見陛下,乞容盡詞。」上意稍緩,元素盡言運冤狀明白,上乃寤曰:「非卿,孰能辨之!」後數月,竟得其真賊,元素由是為時器重,遷給事中。時美官缺,必指元素。遷尚書右丞。數月,鄭滑節度盧群卒,遂命元素兼御史大夫,鎮鄭滑,就加檢校工部尚書,在鎮稱理。



 元和初,徵拜御史大夫。自貞元中位缺,久難其人,至是元素以名望召拜,中外聳聽。及居位,
 一無修舉,但規求作相。久之,浸不得志,見客必曰:「無以某官散相疏也。」見屬官必先拜,脂韋在列,大失人情。李錡為亂江南,遂授元素浙西道節度觀察處置等使。數月受代,入拜國子祭酒,尋遷太常卿,轉戶部尚書、判度支。



 元素少孤,奉長姊友敬加於人,及其姊歿,沉悲遘疾,上疏懇辭,從之。數月,以出妻免官。初,元素再娶妻王氏,石泉公方慶之孫,性柔弱,元素為郎官時娶之,甚禮重,及貴,溺情僕妾,遂薄之。且又無子,而前妻之子已長,
 無良,元素寢疾昏惑,聽譖遂出之,給與非厚。妻族上訴,乃詔曰:「李元素病中上表,懇切披陳,云『妻王氏,禮義殊乖,願與離絕』。初謂素有醜行,不能顯言,以其大官之家,所以令自處置。訪聞不曾告報妻族,亦無明過可書,蓋是中情不和,遂至於此。脅以王命,當日遣歸,給送之間,又至單薄。不唯王氏受辱,實亦朝情悉驚。如此理家,合當懲責。宜停官,仍令與王氏錢物,通所奏數滿五千貫。」元和五年卒,贈陜州大都督。



 史臣曰:李抱玉、李抱真,以武勇之材,兼忠義之行,有唐之良將也。且如農隙教潞人之射,數騎入武俊之營,非有奇謀,孰能如是。惜乎服食求仙,為藥所誤。王虔休不黨僭命,有足可嘉;盧從史動多懷奸,自貽伊戚。芃則老也知足,澄則過而改圖。元素為御史時,執德不回;居大夫日,其心甚短。因緣七出,益露醜聲,善少惡多,又何足算。



 贊曰:抱玉、抱真,我朝良將。虔休之心,亦多可尚。史懷奸
 謀,芃將祿讓。澄迷卻行,素貪一響。吾誰與欺,豈如忠諒。



\end{pinyinscope}