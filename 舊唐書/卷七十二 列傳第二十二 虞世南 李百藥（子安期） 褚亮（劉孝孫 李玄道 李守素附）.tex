\article{卷七十二 列傳第二十二 虞世南 李百藥(子安期) 褚亮(劉孝孫 李玄道 李守素附)}

\begin{pinyinscope}

 ○虞世南李百藥子安期褚亮劉孝孫李玄道李守素附



 虞世南,字伯施,越州餘姚人,隋內史侍郎世基弟也。祖檢,梁始興王諮議;父荔,陳太子中庶子,俱有重名。叔父寄,陳中書侍郎,無子,以世南繼後,故字曰伯施。世南性
 沈靜寡欲,篤志勤學,少與兄世基受學於吳郡顧野王,經十餘年,精思不倦,或累旬不盥櫛。善屬文,常祖述徐陵,陵亦言世南得己之意。又同郡沙門智永,善王羲之書,世南師焉,妙得其體,由是聲名籍甚。天嘉中,荔卒,世南尚幼,哀毀殆不勝喪。陳文帝知其二子博學,每遣中使至其家將護之。及服闋,召為建安王法曹參軍。寄陷於陳寶應,在閩、越中,世南雖除喪,猶布衣蔬食。至太建末,寶應破,寄還,方令世南釋布食肉。至德初,除西陽王
 友。陳滅,與世基同入長安,俱有重名,時人方之二陸。時煬帝在籓,聞其名,與秦王俊闢書交至,以母老固辭,晉王令使者追之。大業初,累授秘書郎,遷起居舍人。時世基當朝貴盛,妻子被服擬於王者。世南雖同居,而躬履勤儉,不失素業。及至隋滅,宇文化及弒逆之際,世基為內史侍郎,將被誅,世南抱持號泣,請以身代,化及不納,因哀毀骨立,時人稱焉。從化及至聊城,又陷於竇建德,偽授黃門侍郎。



 太宗滅建德,引為秦府參軍。尋轉記室,
 仍授弘文館學士,與房玄齡對掌文翰。太宗嘗命寫《列女傳》以裝屏風,於時無本,世南暗疏之,不失一字。太宗升春宮,遷太子中舍人。及即位,轉著作郎,兼弘文館學士。時世南年已衰老,抗表乞骸骨,詔不許。遷太子右庶子,固辭不拜,除秘書少監。上《聖德論》,辭多不載。七年,轉秘書監,賜爵永興縣子。太宗重其博識,每機務之隙,引之談論,共觀經史。世南雖容貌懦曌,若不勝衣,而志性抗烈,每論及古先帝王為政得失,必存規諷,多所補益。
 太宗嘗謂侍臣曰:「朕因暇日,與虞世南商略古今,有一言之失,未嘗不悵恨,其懇誠若此,朕用嘉焉。群臣皆若世南,天下何憂不理!」



 八年,隴右山崩,大蛇屢見,山東及江淮多大水。太宗以問世南,對曰:「春秋時山崩,晉侯召伯宗而問焉,對曰:『國主山川,故山川崩竭,君為之不舉,降服、乘縵、徹樂、出次、祝幣以禮焉。』梁山,晉所主也,晉侯從之,故得無害。漢文帝元年,齊、楚地二十九山同日崩,水大出,令郡國無來貢獻,施惠於天下,遠近歡洽,亦不
 為災。後漢靈帝時,青蛇見御座。晉惠帝時,大蛇長三百步,見齊地,經市入朝。案蛇宜在草野,而入市朝,所以可為怪耳。今蛇見山澤,蓋深山大澤必有龍蛇,亦不足怪也。又山東足雨,雖則其常,然陰淫過久,恐有冤獄,宜省系囚,庶幾或當天意。且妖不勝德,唯修德可以銷變。」太宗以為然,因遣使者賑恤饑餒,申理獄訟,多所原宥。後有星孛於虛、危,歷於氐,百餘日乃滅。太宗謂群臣曰:「天見彗星,是何妖也?」世南曰:「昔齊景公時有彗星見,公問
 晏嬰,對曰:『穿池沼畏不深,起臺榭畏不高,行刑罰畏不重,是以天見彗為公誡耳。』景公懼而修德,後十六日而星沒。臣聞『天時不如地利,地利不如人和』,若德義不修,雖獲麟鳳,終是無補;但政事無闕,雖有災星,何損於時?然願陛下勿以功高古人而自矜伐,勿以太平漸久而自驕怠,慎終如始,彗星雖見,未足為憂。」太宗斂容謂曰:「吾之撫國,良無景公之過。但吾才弱冠舉義兵,年二十四平天下,未三十而居大位,自謂三代以降,撥亂之主,
 莫臻於此。重以薛舉之驍雄,宋金剛之鷙猛,竇建德跨河北,王世充據洛陽,當此之時,足為勍敵,皆為我所擒。及逢家難,復決意安社稷,遂登九五,降服北夷,吾頗有自矜之意,以輕天下之士,此吾之罪也。上天見變,良為是乎?秦始皇平六國,隋煬帝富四海,既驕且逸,一朝而敗,吾亦何得自驕也。言念於此,不覺惕焉震懼。」四月,康國獻獅子,詔世南為之賦,命編之東觀,辭多不載。後高祖崩,有詔山陵制度,準漢長陵故事,務從隆厚。程限既
 促,功役勞弊。世南上封事諫曰:



 臣聞古之聖帝明王所以薄葬者,非不欲崇高光顯,珍寶具物,以厚其親。然審而言之,高墳厚壟,珍物畢備,此適所以為親之累,非曰孝也。是以深思遠慮,安於菲薄,以為長久萬代之計,割其常情以定耳。昔漢成帝造延、昌二陵,制度甚厚,功費甚多。諫議大夫劉向上書,其言深切,皆合事理。其略曰:「孝文居霸陵,淒愴悲懷,顧謂群臣曰:『嗟乎!以北山石為槨,用紵絮斮陳漆其間,豈可動哉?』張釋之進曰:『使其中
 有可欲,雖錮南山猶有隙;使其中無可欲,雖無石槨,又何戚焉!』夫死者無終極,而國家有廢興,釋之所言,為無窮計也。孝文寤焉,遂以薄葬。」又漢氏之法,人君在位,三分天下貢賦,以一分入山陵。武帝歷年長久,比葬,陵中不復容物,霍光暗於大體,奢侈過度。其後至更始之敗,赤眉賊入長安,破茂陵取物,猶不能盡。無故聚斂百姓,為盜之用,甚無謂也。魏文帝於首陽東為壽陵,作終制,其略曰:「昔堯葬壽陵,因山為體,無封樹,無立寢殿園邑,
 為棺槨足以藏骨,為衣衾足以朽肉。吾營此不食之地,欲使易代之後,不知其處,無藏金銀銅鐵,一以瓦器。自古及今,未有不亡之國,無有不發之墓,至乃燒取玉匣金縷,骸骨並盡,乃不重痛哉!若違詔妄有變改,吾為戮尸於地下,死而重死,不忠不孝,使魂而有知,將不福汝。以為永制,藏之宗廟。」魏文帝此制,可謂達於事矣。向使陛下德止如秦、漢之君,臣則緘口而已,不敢有言。伏見聖德高遠,堯、舜猶所不逮,而俯與秦、漢之君同為奢泰,
 舍堯、舜、殷、周之節儉,此臣所以尤戚也。今為丘壟如此,其內雖不藏珍寶,亦無益也。萬代之後,但見高墳大墓,豈謂無金玉耶?臣之愚計,以為漢文霸陵,既因山勢,雖不起墳,自然高顯。今之所卜,地勢即平,不可不起,宜依《白虎通》所陳周制,為三仞之墳,其方中制度,事事減少。事竟之日,刻石於陵側,明丘封大小高下之式。明器所須,皆以瓦木,合於禮文,一不得用金銀銅鐵。使萬代子孫,並皆遵奉,一通藏之宗廟,豈不美乎!且臣下除服用
 三十六日,已依霸陵,今為墳壟,又以長陵為法,恐非所宜。伏願深覽古今,為長久之慮,臣之赤心,唯願萬歲之後,神道常安,陛下孝名,揚於無窮耳。



 書奏不報。世南又上疏曰:「漢家即位之初,便營陵墓,近者十餘歲,遠者五十年方始成就。今以數月之間而造數十年之事,其於人力,亦已勞矣。又漢家大郡五十萬戶,即目人眾未及往時,而功役與之一等,此臣所以致疑也。」時公卿又上奏請遵遺詔,務從節儉,因下其事付所司詳議,於是制
 度頗有減省焉。



 太宗後頗好獵,世南上疏諫曰:「臣聞秋獮冬狩,蓋惟恆典;射隼從禽,備乎前誥。伏惟陛下因聽覽之餘辰,順天道以殺伐,將欲躬摧班掌,親御皮軒,窮猛獸之窟穴,盡逸材於林藪。夷兇剪暴,以衛黎元;收革擢羽,用充軍器;舉旗效獲,式遵前古。然黃屋之尊,金輿之貴,八方之所仰德,萬國之所系心,清道而行,猶戒銜橛,斯蓋重慎防微,為社稷也。是以馬卿直諫於前,張昭變色於後,臣誠微淺,敢忘斯義?且天弧星畢,所殪已多,
 頒禽賜獲,皇恩亦薄。伏願時息獵車,且韜長戟,不拒芻蕘之請,降納涓澮之流,袒裼徒摶,任之群下,則貽範百王,永光萬代。」其有犯無隱,多此類也。太宗以是益親禮之。嘗稱世南有五絕:一曰德行,二曰忠直,三曰博學,四曰文辭,五曰書翰。十二年,又表請致仕,優制許之,仍授銀青光祿大夫、弘文館學士,祿賜防閣,並同京官職事。尋卒,年八十一。太宗舉哀於別次,哭之甚慟。賜東園秘器,陪葬昭陵,贈禮部尚書,謚曰文懿。手敕魏王泰曰:「虞
 世南於我,猶一體也。拾遺補闕,無日暫忘,實當代名臣,人倫準的。吾有小失,必犯顏而諫之。今其云亡,石渠、東觀之中,無復人矣,痛惜豈可言耶!」未幾,太宗為詩一篇,追述往古興亡之道,既而嘆曰:「鐘子期死,伯牙不復鼓琴。朕之此詩,將何以示?」令起居郎褚遂良詣其靈帳讀訖焚之,冀世南神識感悟。後數歲,太宗夜夢見之,有若平生。翌日,下制曰:「禮部尚書、永興文懿公虞世南,德行淳備,文為辭宗,夙夜盡心,志在忠益。奄從物化,倏移歲
 序,昨因夜夢,忽睹其人,兼進讜言,有如平生之日。追懷遺美,良增悲嘆。宜資冥助,申朕思舊之情,可於其家為設五百僧齋,並為造天尊像一區。」又敕圖其形於凌煙閣。有集三十卷,令褚亮為之序。世南子昶,官至工部侍郎。



 李百藥,字重規,定州安平人,隋內史令、安平公德林子也。為童兒時多疾病,祖母趙氏故以百藥為名。七歲解屬文。父友齊中書舍人陸乂、馬元熙嘗造德林宴集,有
 讀徐陵文者,云「既取成周之禾,將刈瑯邪之稻」,並不知其事。百藥時侍立,進曰:「《傳》稱『鄅人藉稻』。杜預《注》云『鄅國在瑯邪開陽』。」乂等大驚異之。開皇初,授東宮通事舍人,遷太子舍人,兼東宮學士。或嫉其才而毀之者,乃謝病免去。十九年,追赴仁壽宮,令襲父爵。左僕射楊素、吏部尚書牛弘雅愛其才,奏授禮部員外郎,皇太子勇又召為東宮學士。詔令修五禮,定律令,撰陰陽書。臺內奏議文表,多百藥所撰。時煬帝出鎮揚州,嘗召之,百藥辭疾
 不赴,煬帝大怒,及即位,出為桂州司馬。為沈法興所得,署為掾。其後,罷州置郡,因解職還鄉里。大業五年,授魯郡臨泗府步兵校尉。九年,充戍會稽。尋授建安郡丞,行達烏程,屬江都難作,復會沈法興為李子通所破,子通又命為中書侍郎、國子祭酒。及杜伏威攻滅子通,又以百藥為行臺考功郎中。或有譖之者,伏威囚之,百藥著《省躬賦》以致其情,伏威亦知其無罪,乃令復職。伏威既據有江南,高祖遣使招撫,百藥勸伏威入朝,伏威從之,
 遣其行臺僕射輔公祏與百藥留守,遂詣京師。及渡江至歷陽,狐疑中悔,將害百藥,乃飲以石灰酒,因大洩痢,而宿病皆除。伏威知百藥不死,乃作書與公祏令殺百藥,賴伏威養子王雄誕保護獲免。公祏反,又授百藥吏部侍郎。有譖百藥於高祖,云百藥初說杜伏威入朝,又與輔公祏同反。高祖大怒。及公祏平,得伏威與公祏令殺百藥書,高祖意稍解,遂配流涇州。



 太宗重其才名,貞觀元年,召拜中書舍人,賜爵安平縣男。受詔修定《五禮》
 及律令,撰《齊書》。二年,除禮部侍郎。朝廷議將封建諸侯,百藥上《封建論》曰:



 臣聞經國庇民,王者之常制;尊主安上,人情之本方。思闡治定之規,以弘長世之業者,萬古不易,百慮同歸。然命歷有賒促之殊,邦家有理亂之異,遐觀載籍,論之詳矣。咸云周過其數,秦不及期,存亡之理,在於郡國。可以監夏殷之長久,遵黃唐之並建,維城盤石,深根固本,雖王綱弛廢,枝幹相持,故使逆節不生,宗祀不絕。秦氏背師古之訓,棄先王之道,踐華恃險,罷
 侯置守,子弟無尺土之邑,兆庶罕共治之憂,故一夫號澤,七廟隳祀。臣以為自古皇王,君臨宇內,莫不受命上玄,飛名帝錄,締構遇興王之運,殷憂屬啟聖之期。雖魏武攜養之資,漢高徒役之賤,非止意有覬覦,推之亦不能去也。若其獄訟不歸,菁華已竭,雖帝堯之光被四表,大舜之上齊七政,非止情存揖讓,守之亦不可固焉。以放勛、重華之德,尚不能克昌厥後,是知祚之長短,必在天時,政或盛衰,有關人事。隆周卜代三十,卜年七百,雖
 淪胥之道斯極,而文、武之器猶存,斯則龜鼎之祚,已懸定於杳冥也。至使南征不返,東遷避逼,禋祀如線,郊畿不守,此乃凌夷之漸,有累於封建焉。暴秦運短閏餘,數鐘百六。受命之主,德異禹、湯;繼世之君,才非啟、誦。借使李斯、王綰之輩,盛開四履,將閭、子嬰之徒,俱啟千乘,豈能逆帝子之勃興,抗龍顏之基命者也!然則得失成敗,各有由焉。而著述之家,多守常轍,莫不情亡今古,理蔽澆淳,欲以百王之季,行三代之法。天下五服之內,盡封
 諸侯;王畿千乘之間,俱為採地。是以結繩之化,行虞、夏之朝;用象刑之典,治劉、曹之末,紀綱既紊,斷可知焉。鍥船求劍,未見其可;膠柱成文,彌所多惑。徒知問鼎請隧,有懼霸王之師;白馬素車,無復籓籬之援。不悟望夷之釁,未甚羿、浞之災;高貴之殃,寧異申、繒之酷!乃欽明昏亂,自革安危,固非守宰公侯,以成興廢。且數世之後,王室浸微,始自籓屏,化為仇敵。家殊俗,國異政,強凌弱,眾暴寡,疆場彼此,干戈日尋。狐駘之役,女子盡髽;崤陵之
 師,只輪不返。斯蓋略舉一隅,其餘不可勝數。陸士衡方規規然云:「嗣王委其九鼎,兇族據其大邑,天下晏然,以治待亂。」何斯言之謬也!而設官分職,任賢使能,以循吏之才,膺共治之寄,刺郡分竹,何代無人?至使地或呈祥,天不愛寶,民稱父母,政比神明。曹元首方區區然稱:「與人共其樂者,人必憂其憂,與人同其安者,人必拯其危。」豈容委以侯伯,則同其安危;任之牧宰,則殊其憂樂?何斯言之妄也!封君列國,藉慶門資,忘其先業之艱難,輕
 其自然之崇貴,莫不世增淫虐,代益驕侈。自離宮別館,切漢凌雲,或刑人力而將盡,或召諸侯而共樂。陳靈則君臣悖禮,共侮徵舒;衛宣則父子聚麀,終誅壽、朔。乃云為己思治,豈若是乎?內外群官,選自朝廷,擢士庶以任之,澄水鏡以鑒之,年勞優其階品,考績明其黜陟。進取事切,砥礪情深,或俸祿不入私門,妻子不之官舍。頒條之貴,食不舉火;剖符之重,衣唯補葛。南郡太守,敝布裹身;萊蕪縣長,凝塵生甑。專云為利圖物,何其爽歟!總而
 言之,爵非世及,用賢之路斯廣;民無定主,附下之情不固。此乃愚智所辨,安可惑哉!至如滅國弒君,亂常幹紀,春秋二百年間,略無寧歲。次睢咸秩,遂用玉帛之名;魯道有蕩,每等衣裳之會。縱使西漢哀、平之際,東洛桓、靈之時,下吏淫暴,必不至此。為政之理,可一言以蔽之。



 伏惟陛下握紀御天,膺期啟聖,救億兆之焚溺,掃氛昆於寰區。創業垂統,配二儀以立德;發號施令,妙萬物而為言。獨照宸衷,永懷前古,將復五等而修舊制,建萬國以
 親諸侯。竊以漢、魏以還,餘風之弊未盡;勛、華既往,至公之道斯革。況晉氏失馭,宇縣崩離;後魏時乘,華夷雜處。重以關河分阻,吳、楚懸隔,習文者學長短縱橫之術,習武者盡干戈戰爭之心,畢為狙詐之階,彌長澆浮之俗。開皇在運,因藉外家。驅御群英,任雄猜之數;坐移時運,非克定之功。年逾二紀,民不見德。及大業嗣文,世道交喪,一時人物,掃地將盡。雖天縱神武,削平寇虐,兵威不息,勞止未康。自陛下仰順聖慈,嗣膺寶歷,情深致治,綜
 核前王。雖至道無名,言象所紀,略陳梗概,實所庶幾。愛敬蒸蒸,勞而不倦,大舜之孝也。訪安內豎,親嘗御膳,文王之德也。每憲司讞罪,尚書奏獄,大小必察,枉直咸申,舉斷趾之法,易大闢之刑,仁心隱惻,貫徹幽顯,大禹之泣辜也。正色直言,虛心受納,不簡鄙陋,無棄芻蕘,帝堯之求諫也。弘獎名教,勸勵學徒,既擢明經於青紫,將升碩儒於卿相,聖人之善誘也。群臣以宮中暑濕,寢膳或乖,請徙御高明,營一小閣。遂惜家人之產,竟抑子來之
 願,不吝陰陽所感,以安卑陋之居。去歲荒儉,普天饑饉,喪亂甫爾,倉廩空虛。聖情矜愍,勤加惠恤,竟無一人流離道路,猶且食啖藜藿,樂撤簨弶,言必淒動,貌成臒瘠。公旦喜於重譯,文命矜其即序。陛下每四夷款附,萬里歸仁,必退思進省。凝神動慮,恐妄勞中國,以事遠方,不藉萬古之英聲,以存一時之茂實。心切憂勞,跡絕游幸,每旦視朝,聽受無倦。智周於萬物,道濟於天下。罷朝之後,引進名臣,討論是非,備盡肝膈,唯及政事,更無異辭。
 才及日昃,命才學之士,賜以清閑,高談典籍,雜以文詠,間以玄言,乙夜忘疲,中宵不寐。此之四道,獨邁往初。斯實生民以來,一人而已。弘茲風化,昭示四方,信可以期月之間,彌綸天壤。而淳粹尚阻,浮詭未移,此由習之永久,難以卒變。請待斫雕成樸,以質代文,刑措之教一行,登封之禮云畢,然後定疆理之制,議山河之賞,未為晚焉。《易》稱:「天地盈虛,與時消息,況於人乎?」美哉斯言也。



 太宗竟從其議。四年,授太子右庶子。五年,與左庶子於志
 寧、中允孔穎達、舍人陸敦信侍講於弘教殿。時太子頗留意典墳,然閑燕之後,嬉戲過度,百藥作《贊道賦》以諷焉,辭多不載。太宗見而遣使謂百藥曰:「朕於皇太子處見卿所獻賦,悉述古來儲貳事以誡太子,甚是典要。朕選卿以輔弼太子,正為此事,大稱所委,但須善始令終耳。」因賜彩物五百段。然太子卒不悟而廢。十年,以撰《齊史》成,加散騎常侍,行太子左庶子,賜物四百段。俄除宗正卿。十一年,以撰《五禮》及律令成,進爵為子。後數歲,以
 年老固請致仕,許之。太宗嘗制《帝京篇》,命百藥並作,上嘆其工,手詔曰:「卿何身之老而才之壯,何齒之宿而意之新乎!」二十二年卒,年八十四,謚曰康。百藥以名臣之子,才行相繼,四海名流,莫不宗仰。藻思沈鬱,尤長於五言詩,雖樵童牧豎,並皆吟諷。性好引進後生,提獎不倦。所得俸祿,多散之親黨。又至性過人,初侍父母喪還鄉,徒跣單衣,行數千里,服闋數年,容貌毀悴,為當時所稱。及懸車告老,怡然自得,穿池築山,文酒談賞,以舒平生
 之志。有集三十卷,子安期。



 安期幼聰辯,七歲解屬文。初,百藥大業末出為桂州司馬,行至太湖,遇逆賊,將加白刃,安期跪泣請代父命,賊哀而釋之。貞觀初,累轉符璽郎。預修《晉書》成,除主客員外郎。永徽中,遷中書舍人。又與李義府等於武德殿內修書,再轉黃門侍郎。龍朔中,為司列少常伯,參知軍國。有事太山,詔安期為朝覲壇碑文。安期前後三為選部,頗為當時所稱。時高宗屢引侍臣,責以不進賢良。眾皆莫對,獨安期進曰:「臣聞聖帝
 明王,莫不勞於求賢,逸於任使。設使堯、舜苦己臒瘠,不能用賢,終亦王化不行。自夏、殷已來,歷國數十,皆委賢良,以共致理。且十室之邑,必有忠信,況今天下至廣,非無英彥。但比來公卿有所薦引,即遭囂謗,以為朋黨。沉屈者未申,而在位者已損,所以人思茍免,競為緘默。若陛下虛己招納,務於搜訪,不忌親讎,唯能是用,讒毀亦既不入,誰敢不竭忠誠?此皆事由陛下,非臣等所能致也。」高宗深然其言。俄檢校東臺侍郎、同東西臺三品,出
 為荊州大都督府長史。咸亨初卒。自德林至安期三世,皆掌制誥。安期孫羲仲,又為中書舍人。



 褚亮,字希明,杭州錢塘人。曾祖湮,梁御史中丞;祖蒙,太子中舍人;父玠,陳秘書監,並著名前史。其先自陽翟徙居焉。亮幼聰敏好學,善屬文。博覽無所不至,經目必記於心。喜游名賢,尤善談論。年十八,詣陳僕射徐陵,陵與商榷文章,深異之。陳後主聞而召見,使賦詩,江總及諸辭人在坐,莫不推善。禎明初,為尚書殿中侍郎。陳亡,入
 隋為東宮學士。大業中,授太常博士。時煬帝將改置宗廟,亮奏議曰:



 謹按《禮記》:「天子七廟,三昭三穆,與太祖之廟而七。」鄭玄《注》曰:「此周制也。七者,太祖及文王、武王之祧,與親廟四也。殷則六廟,契及湯與二昭二穆也。夏則五廟,無太祖,禹與二昭二穆而已。」玄又據《禮》:「王者禘其祖之所自出而立四廟。」案鄭玄義,天子唯立四親廟,並始祖而為五。周以文、武為受命之祖,特立二祧,是為七廟。王肅注《禮記》曰:「尊者尊統上,卑者尊統下。故天子七
 廟,諸侯五廟。其有殊功異德,非太祖而不毀,不在七廟之數。」案肅以為天子七廟,是百代之言。又據《王制》天子七廟,諸侯五廟,大夫三廟,降二為差。是則天子立四親廟,又立高祖之父、高祖之祖父、太祖而為七。周有文、武、姜嫄合為十廟。漢世諸帝之廟各立,無迭毀之義。至元帝時,貢禹、匡衡之徒始議其禮,以高帝為太祖,而立四親,是為五廟。唯劉歆以為天子七廟,諸侯五廟,降殺以兩之義,七者,其正法可常數也。宗不在此數內,有功德
 則宗之,不可豫設為數也。是以班固稱「考論諸儒之儀,劉歆博而舊矣。」光武即位,建高廟於洛陽。乃立南頓君以上四廟,就祖宗而為七。至魏初,高堂隆為鄭學,議立親廟四,太祖武帝猶在四親之內,乃虛置太祖及二祧以待後世。至景初間,乃依王肅更立六廟,二世祖就四親而為六廟。晉武受禪,博議宗祀,自文帝以上至六世親祖征西府君,而宣帝亦序於昭穆,未升太祖,故祭止六世。江左中興,賀循知禮,至於寢廟之議,皆依魏、晉舊
 事。宋武初受命為王,依諸侯立親廟四,即位之後,增祠五世祖相國掾府君、六世祖右北平府君,止於六廟,建身沒主升,亦從昭穆,猶太祖之位也。降及齊、梁,守而勿革,加宗迭毀,禮無違舊。臣又按姬周自太祖已下,皆別立廟,至於禘祫,俱合食於太祖。是以炎漢之初,諸廟各立,歲時常享,亦隨處而祭,所用廟樂,皆像功德而歌舞焉。至光武乃總立一堂,而群主異室,斯則新承寇亂,欲從約省,自此已來,因循不變。皇隋太祖武元皇帝仁風
 潛暢,至澤傍通,以昆、彭之勛,開稷、契之緒。高祖文皇帝睿哲玄覽,神武應期,撥亂返正,遠肅邇安,受命開基,垂統聖嗣,鴻名冠於三代,寶祚傳於七百。當文明之運,定祖宗之禮。且損益不同,沿襲異趣,時王所制,可以垂法。自歷代已來,親用王、鄭二義。若尋其旨歸,校以優劣,康成止論周代,非謂經通;子雍總貫皇王,事兼長遠。今請依據古典,崇建七廟,受命之廟,宜別立廟,祧百世之後,不毀之法。至於鑾駕親奉,申孝享於高廟;有司行事,竭
 誠敬於群主。俾夫規模可則,嚴祀易遵,表有功而彰明德,大復古而貴能變。臣又按周人立廟,亦無處置之文,據塚人職而言之,先王居中,以昭穆為左右。阮忱所撰《禮圖》,亦從此義。漢京諸廟既遠,又不序禘祫。今若依周制,理有未安,雜用漢儀,事難全採,謹詳立別圖附之。



 議未行,尋坐與楊玄感有舊,左遷西海郡司戶。時京兆郡博士潘徽亦以筆札為玄感所禮,降威定縣主簿。當時寇盜縱橫,六親不能相保。亮與同行,至隴山,徽遇病終,
 亮親加棺斂,瘞之路側,慨然傷懷,遂題詩於隴樹,好事者皆傳寫諷誦,信宿遍於京邑焉。薛舉僭號隴西,以亮為黃門侍郎,委之機務。及舉滅,太宗聞亮名,深加禮接,因從容自陳。太宗大悅,賜物二百段、馬四匹。從還京師,授秦王文學。



 時高祖以寇亂漸平,每冬畋狩。亮上疏諫曰:「臣聞堯鼓納諫,舜木求箴,茂克昌之風,致升平之道。伏惟陛下應千祀之期,拯百王之弊,平壹天下,劬勞帝業,旰食思政,廢寢憂人。用農隙之餘,遵冬狩之禮。獲車
 之所游踐,虞旗之所涉歷,網唯一面,禽止三驅,縱廣成之獵士,觀上林之手搏,斯固畋弋之常規,而皇王之壯觀。至於親逼猛獸,臣竊惑之。何者?筋力驍悍,爪牙輕捷。連弩一發,未必挫其兇心;長戟才捴,不能當其憤氣。雖孟賁抗左,夏育居前,卒然驚軼,事生慮表。如或近起林叢,未填坑谷,駭屬車之後乘,犯官騎之清塵。小臣怯懦,私懷戰慄。陛下以至聖之資,垂將來之教,降情納下,無隔直言。臣叨逢明時,游宦籓邸,身漸榮渥,日用不知,敢
 緣天造,冒陳丹懇。」高祖甚納之。太宗每有征伐,亮常侍從,軍中宴筵,必預歡賞,從容諷議,多所裨益。又與杜如晦等十八人為文學館學士,太宗入居春宮,除太子舍人,遷太子中允。貞觀元年,為弘文館學士。九年,進授員外散騎常侍、封陽翟縣男,拜通直散騎常侍、學士如故。十六年,進爵為侯,食邑七百戶。後致仕歸於家。太宗幸遼東,亮子遂良為黃門侍郎,詔遂良謂亮曰:「昔年師旅,卿常入幕;今茲遐伐,君已懸車。倏忽之間,移三十載,眷
 言疇昔,我勞如何!今將遂良東行,想公於朕,不惜一兒於膝下耳,故遣陳離意,善居加食。」亮奉表陳謝。及寢疾,詔遣醫藥救療,中使候問不絕。卒時年八十八。太宗甚悼惜之,不視朝一日,贈太常卿,陪葬昭陵,謚曰康。長子遂賢,守雍王友。次子遂良,自有傳。



 始太宗既平寇亂,留意儒學,乃於宮城西起文學館,以待四方文士。於是,以屬大行臺司勛郎中杜如晦,記室考功郎中房玄齡及於志寧,軍諮祭酒蘇世長,天策府記室薛收,文學褚亮、
 姚思廉,太學博士陸德明、孔穎達,主簿李玄道,天策倉曹李守素,記室參軍虞世南,參軍事蔡允恭、顏相時,著作佐郎攝記室許敬宗、薛元敬,太學助教蓋文達,軍諮典簽蘇勖,並以本官兼文學館學士。及薛收卒,復征東虞州錄事參軍劉孝孫入館。尋遣圖其狀貌,題其名字、爵里,乃命亮為之像贊,號《十八學士寫真圖》,藏之書府,以彰禮賢之重也。諸學士並給珍膳,分為三番,更直宿於閣下,每軍國務靜,參謁歸休,即便引見,討論墳籍,商
 略前載。預入館者,時所傾慕,謂之「登瀛洲」。顏相時兄師古、蘇勖兄子幹。



 劉孝孫者,荊州人也。祖貞,周石臺太守。孝孫弱冠知名,與當時辭人虞世南、蔡君和、孔德紹、庾抱、庾自直、劉斌等登臨山水,結為文會。大業末,沒於王世充,世充弟偽杞王辯引為行臺郎中。洛陽平,辯面縛歸國,眾皆離散,孝孫猶攀援號慟,追送遠郊,時人義之。武德初,歷虞州錄事參軍,太宗召為秦府學士。貞觀六年,遷著作佐郎、吳王友。嘗採歷代文集,為王撰《古今類
 序詩苑》四十卷。十五年,遷本府諮議參軍。尋遷太子洗馬,未拜卒。



 李玄道者,本隴西人也,世居鄭州,為山東冠族。祖瑾,魏著作佐郎。父行之,隋都水使者。玄道仕隋為齊王府屬。李密據洛口,引為記室。及密破,為王世充所執。是時,同遇兇俘者並懼死,達曙不寐,唯玄道顏色自若,曰:「死生有命,非憂能了。」同拘者雅推其識量。及見世充,舉措不改其常。世充素知其名,益重之,釋縛以為著作佐郎。東
 都平,太宗召為秦王府主簿、文學館學士。貞觀元年,累遷給事中,封姑臧縣男。時王君廓為幽州都督,朝廷以其武將不習時事,拜玄道為幽州長史,以維持府事。君廓在州屢為非法,玄道數正議裁之。嘗又遺玄道一婢,玄道問婢所由,云本良家子,為君廓所掠,玄道因放遣之,君廓甚不悅。後遇君廓入朝,房玄齡即玄道之從甥也,玄道附書,君廓私發,不識草字,疑其謀己,懼而奔叛,玄道坐流巂州。未幾征還,為常州刺史。在職清簡,百姓安
 之,太宗下詔褒美,賜以綾彩。三年,表請致仕,加銀青光祿大夫,以祿歸第,尋卒。子雲將,知名。官至尚書左丞。



 李守素者,趙州人,代為山東名族。太宗平王世充,徵為文學館學士,署天策府倉曹參軍。守素尤工譜學,自晉宋已降,四海士流及諸勛貴華戎閥閱,莫不詳究,當時號為「行譜」。嘗與虞世南共談人物,言江左、山東,世南猶相酬對;及言北地諸侯,次第如流,顯其世業,皆有援證,世南但撫掌而笑,不復能答,嘆曰:「行譜定可畏。」許敬宗
 因謂世南曰:「李倉曹以善談人物,乃得此名,雖為美事,然非雅目。公既言成準的,宜當有以改之。」世南曰:「昔任彥升美談經籍,梁代稱為『五經笥』;今目倉曹為『人物志』可矣。」貞觀初卒。



 史臣曰:劉並州有言:「和氏之璧,不獨耀於郢握;夜光之珠,何專玩於隋掌?天下之寶,固當與天下共之。」虞永興之從建德,李安平之佐公祏,褚陽翟之依薛舉,蓋大渴不能擇泉而飲,大暑不能擇廕而息耳,非不識其飲憩
 之所。及文皇帝揭三辰而燭天下,群賢霧集,人之所奉,方得躍鱗天池,擅價春山,為一代之至寶,則所托之勢異也。隋掌郢握,曷有常哉!二虞昆仲,文章炳蔚於隋、唐之際;褚河南父子,箴規獻替,洋溢於貞觀、永徽之間。所謂代有人焉,而三家尤盛。



 贊曰:猗與文皇,蕩滌蒼昊。十八文星,連輝炳耀。虞、褚之筆,動若有神。安平之什,老而彌新。



\end{pinyinscope}