\article{卷三十一 志第十一 音樂四}

\begin{pinyinscope}

 享太廟
 樂章十三首貞觀中魏徵褚亮等作



 迎
 神用《永和》黃鐘宮三成,大呂角二成,太簇徵二成,應鐘羽二成,總九變同用。:



 於穆烈祖,弘此丕基。永言配命,子孫保之。百神既洽,萬
 國在茲。是用孝享,神其格思。



 皇帝行用《太和》詞同冬至圓丘。



 登歌酌鬯用《肅和》夾鐘均之黃鐘羽:



 大哉至德,允茲明聖。格於上下,聿遵誠敬。喜樂斯登,鳴珠以詠。神其降止,式隆景命。



 迎俎用《雍和》:



 崇茲享祀,誠敬兼至。樂以感靈,禮以昭事。粢盛咸潔,牲牷孔備。永言孝思,庶幾不匱。



 皇祖宣簡公酌獻用《長發》無射宮:



 浚哲惟唐,長發其祥。帝命斯祐,王業克昌。配天載德,就日重光。本枝百代,申錫無疆。



 皇祖懿王酌獻用《長發》同前詞,黃鐘宮。:



 太祖景皇帝酌獻用《大基》太簇宮:



 猗歟祖業,皇矣帝先。翦商德厚,封唐慶延。在姬猶稷,方晉逾宣。基我鼎運,於萬斯年。



 世祖元皇帝酌獻用《大成》姑洗宮:



 周稱王季,晉美帝文。明明盛德,穆穆齊芬。藏用四履,屈道三分,鏗鏘鐘石,載紀鴻勛。



 高祖大武皇帝酌獻用《大明》蕤賓宮:



 五紀更運,三正遞升。勛華既沒,禹湯勃興。神武命代,靈眷是膺。望雲彰德,察緯告徵。上紐天維,下安地軸。徵師涿野,萬國咸服。偃伯靈臺,九官允穆。殊域委FB,懷生介福。大禮既飾,大樂已和。黑章擾囿,赤字浮河。功宣載籍,德被詠歌。克昌厥後,百祿是荷。



 皇帝飲福用《壽和》:



 八音斯奏,三獻畢陳。寶祚惟永,暉光日新。



 送文舞出迎武舞入用《舒和》:



 聖敬通神光七廟,靈心薦祚和萬方。嚴禋克配鴻基遠,明德惟馨鳳歷昌。



 武舞用《凱安》詞同冬至圓丘。



 徹俎用《雍和》:



 於穆清廟,聿修嚴祀。四縣載陳,三獻斯止。籩豆徹薦,人
 祇介祉。神惟格思,錫祚不已。



 送神用《永和》:



 肅肅清祀,蒸蒸孝思。薦享昭備,虔恭在茲。雍歌徹俎,祝嘏陳辭。用光武志,永固鴻基。



 又享太廟樂章五首永徽已後續撰,不詳撰者。



 太宗文皇帝酌獻用《崇德》夷則宮,永徽元年造。



 五運改卜,千齡啟聖。彤雲曉聚,黃星夜映。葉闡珠囊,基開玉鏡。後為圖開。下臨萬宇,上齊七政。霧開三象,塵清九服。
 海溓星暉,遠安邇肅。天地交泰,華夷輯睦。翔泳歸仁,中外禔福。績逾黜夏,勛高翦商。武陳《七德》,刑設三章。祥禽巢閣,仁獸游梁。卜年惟永,景福無疆。



 高宗天皇大帝酌獻用《鈞天》黃鐘宮,光宅元年造。



 承籙,纂聖登皇。遐清萬宇,仰協三光。功成日用,道濟時康。璇圖載永,寶歷斯昌。日月揚暉,煙雲爛色。河岳修貢,神祇效職。舜風攸偃,堯曦先就。睿感通寰,孝思浹宙。奉揚先德,虔遵曩狩。展義天扃,飛英雲岫。化逸王表,
 神凝帝先。乘雲厭俗,馭日登玄。



 中宗孝和皇帝酌獻用《太和》太簇宮,景雲元年造。



 廣樂既備,嘉薦既新。述先惟德,孝饗惟親。七獻具舉,五齊畢陳。錫茲祚福,於萬斯春。



 睿宗大聖真皇帝酌獻用《景雲》黃鐘宮,開元四年造。:



 惟睿作聖,惟聖登皇。精感耀魄,時膺會昌。舜慚大孝,堯推讓王。能事斯極,振古誰方。文明履運,車書同軌。巍巍赫赫,盡善盡美。衢室凝旒,大庭端扆。釋負之寄,事光復
 子。脫屣高天,登遐上玄。龍湖超忽,象野芊綿。游衣復道,薦果初年。新廟奕奕,明德配天。



 皇祖宣皇帝酌獻用《光大》無射宮,舊樂章宣、光二宮同用《長發》,其詞亦同。開元十年,始定宣皇帝用《光大》,詞更別造。:



 大業龍祉,徽音駿尊。潛居皇德,赫嗣天昆。展儀宗祖,重誠孝孫。春秋無極,享奏存存。



 又亨太廟樂章三首太樂舊有此詞,不詳所起。



 迎神黃鐘宮、太呂角、太簇徵、應鐘羽,並同此詞。:



 七廟觀德,百靈攸仰。俗荷財成,物資含養。道光執契,化籠提象。肅肅雍雍,神其來享。



 金奏無射宮,次迎神。:



 肅肅清廟,巍巍盛唐。配天立極,累聖重光。樂和管磬,禮備蒸嘗。永惟來格,降福無疆。



 送神:



 五聲備奏,三獻終祠。車移鳳輦,旆轉紅旗。禮周籩豆,誠效虔祗。皇靈徙蹕,簪紳拜辭。



 則天皇后享清廟樂章十首



 第一:



 建清廟,贊玄功。擇吉日,展禋宗。樂已變,禮方崇。望神駕,降仙宮。



 第二:



 隆周創業,寶命惟新。敬宗茂典,爰表虔禋。聲明已備,文物斯陳。肅容如在,懇志方申。



 第三登歌:



 肅敷大禮,上謁尊靈。敬陳筐幣,載表丹誠。



 第四迎神:



 敬奠蘋藻,式罄虔襟。潔誠斯展,佇降靈歆。



 第五飲福:



 爰陳玉醴,式奠瓊漿。靈心有穆,介福無疆。



 第六送文舞:



 帝圖草創,王業初開。功高佐命,業贊雲雷。



 第七迎武舞:



 赫赫玄功被穹壤,皇皇至德洽生靈。開基撥亂祅氛廓,佐命宣威海內清。



 第八武舞作:



 荷恩承顧托,執契恭臨撫。廟略靜邊荒,天兵耀神武。



 第九徹俎:



 登歌已闋,獻禮方周。欽承景福,肅奉鴻休。



 第十送神:



 大禮言畢,仙衛將歸。莫申丹懇,空瞻紫微。



 中宗孝和皇帝神龍元年享太廟樂章二十首不詳所撰



 迎神用《嚴和》黃鐘宮三成,大呂角三成,太簇徵三成,應鐘羽二成,同用此詞。:



 肅肅清廟,赫赫玄猷。功高萬古,化奄十洲。中興丕業,上荷天休。祇奉先構,禮被懷柔。



 皇帝行用《升和》黃鐘宮:



 顧惟菲薄,纂歷應期。中外同軌,夷狄來思。樂用崇德,禮以陳詞。夕惕若厲,欽奉宏基。



 登歌稞鬯用《虔和》大呂均之無射羽:



 禮標薦鬯,肅事祠庭。敬申如在,敢托非馨。



 送文舞出迎武舞入用《同和》太簇羽:



 惟聖配天敷盛禮,惟天為大闡洪名。恭禋展敬光先德,萍藻申虔表志誠。



 武舞用《寧和》林鐘徵:



 炎馭失天綱,土德承天命。英猷被寰宇,懿躅隆邦政。七德已綏邊,九夷咸底定。景化覃遐邇,深仁洽翔泳。



 徹俎用《恭和》大呂均之
 無射羽:



 禮周三獻,樂闋九成。肅承靈福,悚惕益盈。



 送神用《通和》黃鐘宮:



 祠容既畢,仙座爰興。停停鳳舉,靄靄雲升。長隆寶運,永錫休徵。福覃貽厥,恩被黎蒸。



 皇后助享、皇后行用《正和》黃鐘宮,詞同貞觀中宮朝會《正和》:



 登歌奠鬯用《昭和》大呂均之無射羽8



 道洽二儀交泰,時休四宇和平。環佩肅於庭實,鐘石揚乎頌聲。



 皇后酌獻飲福用《誠敬》黃鐘宮:



 顧惟菲質,忝位椒宮。虔奉蘋藻,肅事神宗。敢申誠潔,庶罄深衷。睟容有裕,靈享無窮。



 徹俎用《肅和》大呂均之無射羽:



 月禮已周,雲和將變。爰獻其醑,載遷其奠。明德逾隆,非馨是薦。澤沾動植,仁覃宇縣。



 送神用《昭感》黃鐘羽:



 鏗鏘《韶》《濩》,肅穆神容。洪規赫赫,祠典雍雍。已周三獻,將
 乘六龍。虔誠有托,懇志無從。



 玄宗開元七年享太廟樂章十六首特進、行尚書左丞相燕國公張說作



 迎神用《永和》三章:



 肅九室,諧八音。歌皇慕,動神心。禮宿設,樂妙尋。聲明備,稞奠臨。



 律迓氣,音入玄。依玉幾,禦黼筵。聆愾息,僾周旋。《九韶》遍,百福傳。



 信工祝,永頌聲。來祖考,聽和平。相百闢,貢九瀛。神休委,
 帝孝成。



 皇帝行用《太和》一章:



 時文聖後,清廟肅邕。致誠勤薦,在貌思恭。玉節《肆夏》,金鏘五鐘。繩繩雲步,穆穆天容。



 登歌酌瓚用《肅和》一章:



 天子孝享,工歌溥將。躬稞鬱鬯,乃焚膋薌。臭以達旨,聲以求陽。奉時烝嘗,永代不忘。



 迎俎用《雍和》二章:



 在滌嘉豢,麗碑敬牲。角握之牡,色純之騂。火傳陽燧,水溉陰精。太公胖俎,傳說和羹。



 俎豆有馥,齋盛絜豐。亦有和羹,既戒既平。鼓鐘管磬,肅唱和鳴。皇皇后祖,賚我思成。



 皇帝酌醴齊用文舞一章:



 聖暮九德,真言五千。慶集昌胄,符開帝先。高文杖鉞,克配彼天。三宗握鏡,六合煥然。帝其承祀,率禮罔愆。圖書霧出,日月清懸。舞形德類,詠諗功傳。黃龍蜿蟺,彩雲蹁
 躚。五行氣順,八佾風宣。介此百祿,於皇萬年。



 獻祖宣皇帝室奠獻用《光大》之舞一章:



 肅肅藝祖,滔滔浚源。有雄玉劍,作鎮金門。玄王貽緒,後稷謀孫。肇禋九廟,四海來尊。



 懿祖光皇帝室奠獻用《長發》之舞一章:



 具禮崇德,備樂承風。魏推幢主,周贈司空。不行而至,無成有終。神興王業,天歸帝功。



 太祖景皇帝室奠獻用《大政》之舞一
 章:



 於赫元命,權輿帝文。天齊八柱,地半三分。宗廟觀德,笙鏞樂勛。封唐之兆,成天下君。



 代祖元皇帝室奠獻用《大成》之舞一章:



 帝舞季歷,襲聖生昌。後歌有蟜,胎炎孕黃。天地合德,日月齊光。肅邕孝享,祚我萬方。



 高祖神堯皇帝室奠獻用《大明》之舞一章:



 赤精亂德,四海困窮。黃旗舉義,三靈會同。旱望春雨,雲披大風。普天來祭,高祖之功。



 太宗文武聖皇帝室奠獻用《崇德》之舞一章:



 皇合一德,朝宗百神。削平天下,大拯生人。上帝配食,單于入臣。戎歌陳舞,嘩嘩震震。



 高宗天皇大帝室奠獻用《鈞天》之舞一章:



 高皇邁道,端拱無為。化懷獯鬻,兵戢句驪。禮尊封禪,樂盛來儀。合位媧後,同稱伏羲。



 中宗孝和皇帝室奠獻用《太和》之舞一章:



 退居江水,鬱起丹陵。禮物還舊,朝章中興。龍圖友及,駿
 命恭膺。鳴球秉瓚,大糦是承。



 睿宗大聖真皇帝室奠獻用《景雲》之舞一章:



 景雲霏爛,告我帝符。噫帝沖德,與天為徒。笙鏞遙遠,俎豆虛無。春秋孝獻,回復此都。



 又享太廟樂章十四首



 玄宗至道大聖大明孝皇帝室奠獻用《廣運》之舞一章司徒兼中書令、汾陽郡王郭子儀撰。:



 於赫皇祖,昭明有融。惟文之德,惟武之功。河海靜謐,車
 書混同。虔恭孝饗,穆穆玄風。



 肅宗文明武德大聖大宣孝皇帝室奠獻用《惟新》之舞一章吏部尚書、平章事、彭城郡公劉晏撰。:



 漢祚惟永,神功中興。風驅氛昆,天覆黎蒸。三光再朗,庶績其凝。重熙累葉,景命是膺。



 皇帝飲福受脤用《福和》一章:



 備禮用樂,崇親政尊。誠通慈降,敬徹愛存。獻懷稱壽,啐感承恩。皇帝孝德,子孫千億。大包天域,長亙不極。



 送文舞出、迎武舞入用《舒和》一章:



 六鐘翕協六變成,八佾倘佯八風生。樂《九韶》兮人神感,美《七德》兮天地清。



 亞獻、終獻行事、武舞用《凱安》四章:



 瑟彼瑤爵,亞維上公。室如屏氣,門不容躬。禮殷其本,樂執其中。聖皇永慕,天地幽通。禮匝三獻,樂遍九成。降循軒陛,仰欷皇情。福與仁合,德因孝明。百年神畏,四海風行。
 總總干戚,填填鼓鐘。奮揚增氣,坐作為容。離若鷙鳥,合如戰龍。萬方觀德,肅肅邕邕。烈祖順三靈,文宗威四海。黃鉞誅群盜,硃旗掃多罪。戢兵天下安,約法人心改。大哉幹羽意,長見風雲在。



 徹豆登歌一章:



 止笙磬,徹豆籩。廓無響,窅入玄。主在室,神在天。情餘慕,禮罔愆。喜黍稷,屢豐年。



 送神用《永
 和》一章:



 眇嘉樂,授靈爽。感若來,思如往。休氣散,回風上。返寂寞,還惚恍。懷靈駕,結空想。



 代宗睿文孝武皇帝室奠獻用《保大》之舞一章尚父郭子儀撰:



 於穆文考,聖神昭彰。《簫》《勺》群慝,含光遠方。萬物茂遂,九夷賓王。愔愔《云》《韶》,德音不忘。



 德宗神武孝文皇帝室奠獻用《文明》之舞一章尚書左丞平章事鄭餘慶撰:



 開邸除暴,時邁勛尊。三元告命,四極駿奔。金枝翠葉,輝燭瑤琨。象德億載,貽慶湯孫。



 順宗至德大聖大安孝皇帝室奠獻用《大順》之舞一章中書侍郎、平章事鄭絪撰。:



 於穆時文,受天明命。允恭玄默,化成理定。出震嗣德,應乾傳聖。猗歟緝熙,千億流慶。



 憲宗聖神章武孝皇帝室奠獻用《象德》之舞一章中書侍郎、平章事段文昌撰。:



 肅肅清廟,登顯至德。澤周八荒,兵定四極。生物咸遂,群盜滅息。明聖欽承,子孫千億。



 儀坤廟樂章十二首



 迎神用《永和》林鐘宮,散騎常侍、昭文館學士徐彥伯作。:



 猗若清廟,肅肅熒熒。國薦嚴祀,坤輿淑靈。有幾在室,有樂在庭。臨茲孝享,百祿惟寧。



 金奏夷則宮,不詳作者。一本無此章。:



 陰靈效祉,軒曜降精。祥符淑氣,慶集柔明。瑤俎既列,雕
 桐發聲。徽猷永遠,比德皇英。



 皇帝行用《太和》黃鐘宮,左諭德、昭文館學士邱說撰。:



 孝哉我後,沖乎乃聖。道映重華,德輝文命。慕深視篋,情殷撫鏡。萬國移風,兆人承慶。



 酌獻登歌用《肅和》中呂均之太簇羽,一云蕤賓均之夾鐘羽,太子洗馬、昭文館學士張齊賢撰:



 稞圭既濯,鬱鬯既陳。畫冪雲舉,黃流玉醇。儀充獻酌,禮盛眾禋。地察惟孝,愉焉饗親。



 迎俎用《雍和》姑洗羽,太中大夫、昭文館學士鄭善玉作。:



 酌鬱既灌,取蕭方爇。籩豆靜嘉,簠簋芬飶。魚臘薦美,牲牷表潔。是戢是將,載迎載列。



 肅明皇后室酌獻用《昭升》林鐘宮,禮部尚書、昭文館學士薛稷作。:



 陽靈配德,陰魄昭升。堯壇鳳下,漢室龍興。伣天作對,前旒是凝。化行南國,道盛西陵。造舟集灌,無德而稱。我粢既潔,我醴既澄。陰陰靈廟,光靈若憑。德馨惟饗,孝思
 蒸蒸。



 昭成皇后室酌獻用《坤貞》不詳作者:



 乾道既亨,坤元以貞。肅雍攸在,輔佐斯成。外睦九族,內光一庭。克生睿哲,祚我休明。欽若徽節,悠哉淑靈。建茲清宮,於彼上京。縮茅以獻,潔秬惟馨。實受其福,期乎億齡。



 飲福用《壽和》黃鐘宮,太子詹事、崇文館學士徐堅作。:



 於穆清廟,肅雍嚴祀。合福受釐,介以繁祉。



 送文舞出迎武舞入用《舒和》南呂商,銀青光祿大夫、崇文館學士胡雄
 作。:



 送文迎武遞參差,一始一終光聖儀。四海生人歌有慶,千齡孝享肅無虧。



 武舞用《安和》太簇徵,秘書少監、崇文館學士劉子玄作。:



 妙算申帷幄,神謀出廟庭。兩階文物備,《七德》武功成。校獵長楊苑,屯軍細柳營。將軍獻凱入,歌舞溢重城。



 徹俎用《雍和》蕤賓均之夾鐘羽,銀青光祿大夫、崇文館學士員半千作。:



 孝享雲畢,維徹有章。雲感玄羽,風棲素商。瞻望神座,只
 戀匪遑。禮終樂闋,肅雍鏘鏘。



 送神用《永和》林鐘宮,金紫光祿大夫、崇文館學士祝欽明作。:



 閟宮實實,清廟微微。降格無象,馨香有依。式昭纂慶,方融嗣徽。明禋是享,神保聿歸。



 又儀坤廟樂章二首太樂又有一本,與前本略同,二章不同如左,不詳撰者。



 迎神一本有此章而無徐彥伯之詞。:



 月靈降德,坤元授光。娥英比秀,任姒均芳。瑤臺薦祉,金屋延祥。迎神有樂,歆此嘉薌。



 送神一本有此章而無祝欽明之詞:



 玉帛儀大,金絲奏廣。靈應有孚,冥征不爽。降彼休福,歆茲禋享。送樂有章,神麾其上。



 昭德皇后室酌獻用《坤元》樂章九首內出



 迎神用《永和》:



 穆清廟,薦嚴禋。昭禮備,和樂新。望靈光,集元辰。祚無極,享萬春。



 登歌酌鬯用《肅和》:



 誠心達,娛樂分。升蕭膋,鬱氛氳。茅既縮,鬯既薰。後來思,福如云。



 迎俎用《雍和》:



 我將我享,盡明而誠。載芬黍稷,載滌犧牲。懿矣元良,萬邦以貞。心乎愛敬,若睹容聲。



 酌獻用《坤元》:



 於穆先後,儷聖稱崇。母臨萬宇,道被六宮。昌時協慶,理內成功。殷薦明德,傳芳國風。



 送文舞出迎武舞入用《舒和》:



 金枝羽部輟清歌,瑤堂肅穆笙磬羅。諧音遍響合明意,萬類昭融靈應多。



 武舞用《凱安》:



 辰位列四星,帝功參十亂。進賢勤內輔,扈蹕清多難。承天厚載均,並曜宵光燦。留徽藹前躅,萬古披圖煥。



 徹俎用《雍和》:



 公尸既起,享禮載終。稱歌進徹,盡敬由衷。澤流惠下,大
 小咸同。



 送神用《永和》:



 昭事終,幽享餘。移月御,返仙居。璇庭寂,靈幄虛。顧徘徊,感皇儲。



 孝敬皇帝廟樂章九首



 迎神用《永和》詞同貞觀太廟《永和》。



 皇帝行用《太和》詞同貞觀太廟《太和》。



 登歌酌鬯用《肅和》詞同貞觀太廟《肅和》。



 迎俎用《雍和》詞同貞觀太廟《雍和》。



 酌獻用《承光》詞同中宗享孝敬《承光》。



 送文舞出迎武舞入
 用《舒和》詞同太廟。



 武舞用《凱安》詞同太廟。



 徹俎用《雍和》詞同迎俎。



 送神用《永和》詞同太廟。



 享隱太子廟樂章六首貞觀中撰。



 迎神用《誠和》:



 道閟鶴關,運纏鳩里。門集大命,俾歆嘉祀。禮亞六瑚,誠殫二簋。有誠顒若,神斯戾止。



 登歌奠玉帛用《肅和》:



 歲肇春宗,乾開震長。瑤山既寂,戾園斯享。玉肅其事,物
 昭其象。弦誦成風,笙歌合響。



 迎俎用《雍和》:



 明典肅陳,神居邃啟。春伯聯事,秋官相禮。有來雍雍,登歌濟濟。緬惟主鬯,庶歆芳醴。



 送文舞出、迎武舞入用《舒和》:



 三縣已判歌鐘列,六佾將開羽戚分。尚想燕飛來蔽日,終疑鶴影降凌雲。



 武舞用《凱安》:



 天步昔將開,商郊初欲踐。撫戎金陣廓,貳極瑤圖闡。雞戟遂崇儀,龍樓期好善。弄兵隳震業,啟聖隆祠典。



 送神用《誠和》詞同迎神。



 又隱太子廟樂章二首太樂舊有此詞,不詳所出。



 迎神:



 蒼震有位,黃離蔽明。江充禍結,戾據災成。銜冤昔痛,贈典今榮。享靈有秩,奉樂以迎。



 送神:



 皇情悼往,祀儀增設。鐘鼓鏗鍠,羽旄昭晣。掌禮云備,司筵告徹。樂以送神,靈其鑒闋。



 章懷太子廟樂章六首神龍初作



 迎神第一姑洗宮:



 副君昭象,道應典離。銅樓備德,玉裕成規。仙氣靄靄,靈從師師。前驅戾止,控鶴來儀。



 登歌酌鬯第二南呂均之蕤賓羽:



 忠孝本著,羽翼先成。寢門昭德,馳道為程。幣帛有典,容
 衛無聲。司存既肅,廟享惟清。



 迎俎及酌獻第三大呂羽



 通三錫胤,明兩承英。太山比赫,伊水聞笙。宗祧是寄,禮樂其亨。嘉辰薦俎,以發聲明。



 送文舞出、迎武舞入第四蕤賓商:



 羽龠崇文禮以畢,干戚奮武事將行。用舍由來其有致,壯志宣威樂太平。



 武舞作第五夷則角:



 綠林熾炎歷,黃虞格有苗。沙塵驚塞外,帷幄命嫖姚。《七德》干戈止,三邊雲霧消。寶祚長無極,歌舞盛今朝。



 送神第六詞同隱廟。



 懿德太子廟樂章六首神龍初作



 迎神第一姑洗宮:



 甲觀昭祥,畫堂升位。禮絕群後,望尊儲貳。啟、誦慚德,莊、丕掩粹。伊浦鳳翔,緱峰鶴至。



 登歌酌鬯第二南呂均之蕤賓羽:



 譽闡元儲,寄崇明兩。玉裕雖晦,銅樓可想。弦誦輟音,笙歌罷響。幣帛言設,禮容無爽。



 迎俎酌獻第三大呂羽:



 雍雍盛典,肅肅靈祠。賓天有聖,對日無期。飄颻羽服,掣曳雲旗。眷言主鬯,心乎愴茲。



 送文舞出迎武舞入第四蕤賓商:



 八音協奏陳金石,六佾分行整禮容。滄溟赴海還稱少,素月開輪即是重。



 武舞作第五夷則角:



 隋季昔雲終,唐年初啟聖。纂戎將禁暴,崇儒更敷政。威略靜三邊,仁恩覃萬姓。



 送神第六詞同隱廟。



 節愍太子廟樂章六首景雲中作



 迎神第一姑洗宮:



 儲後望崇,元良寄切。寢門是仰,馳道不絕。仙袂雲會,靈旗電晣。煌煌而來,禮物攸設。



 登歌酌鬯第二南呂均之蕤賓羽:



 灼灼重明,仰承元首。既賢且哲,惟孝與友。惟孝雖遙,靈規不朽。祀因誠致,備潔玄酒。



 迎俎及酌獻第三大呂羽:



 嘉薦有典,至誠莫愆。畫梁云互,雕俎星聯。樂器周列,禮容備宣。依稀如在,若未賓天。



 送文舞出、迎武舞入第四蕤賓商:



 邕邕闡化憑文德,赫赫宣威藉武功。既執羽旄先拂吹,
 還持玉戚更揮空。



 武舞作第五夷則角:



 武德諒雄雄,由來掃寇戎。劍光揮作電,旗影列成虹。霧廓三邊靜,波澄四海同。睿圖今已盛,相共舞皇風。



 送神第六詞同隱太子廟。



 則天大聖皇后崇先廟樂章一首御撰



 先德謙捴冠昔,嚴規節素超今。奉國忠誠每竭,承家至孝純深。追崇懼乖尊意,顯號恐玷徽音。既迫王公屢請,
 方乃俯遂群心。有限無由展敬,奠醑每闕親斟。大禮虔申典冊,蘋藻敬薦翹襟。



 褒德廟樂章五首神龍中為皇后韋氏祖考所立,詞並內出。



 迎神用《昭德》姑洗宮二成:



 道赫梧宮,悲盈蒿里。爰暢徽烈,載敷嘉祀。享洽四時,規陳二簋。靈應昭格,神其戾止。



 登歌用進德南呂均之蕤賓羽:



 塗山懿戚,媯汭崇姻。祠筵肇啟,祭典方申。禮以備物,樂
 以感神。用隆敦敘,載穆彞倫。



 俎入初獻用《褒德》大呂角:



 家著累仁,門昭積善。瑤篚既列,金縣式展。



 武舞作:



 昭昭竹殿開,奕奕蘭宮啟。懿範隆丹掖,殊榮闢硃邸。六佾薦微容,三簋陳芳醴。萬古覃貽厥,分珪崇祖禰。



 亞獻及送神用《彰德》:



 名隆五岳,秩映三臺。嚴祠已備,睟影方回。



\end{pinyinscope}