\article{卷一百四十三}

\begin{pinyinscope}

 ○陸贄



 陸贄,字敬輿,蘇州嘉興人。父侃,溧陽令,以贄貴,贈禮部尚書。贄少孤,特立不群,頗勤儒學。年十八登進士第,以博學宏詞登科,授華州鄭縣尉。罷秩,東歸省母,路由壽
 州,刺史張鎰有時名,贄往謁之。鎰初不甚知,留三日,再見與語,遂大稱賞,請結忘年之契。及辭,遺贄錢百萬,曰:「願備太夫人一日之膳。」贄不納,唯受新茶一串而已,曰:「敢不承君厚意。」又以書判拔萃,選授渭南縣主簿,遷監察御史。德宗在東宮時,素知贄名,乃召為翰林學士,轉祠部員外郎。贄性忠盡,既居近密,感人主重知,思有以效報,故政或有缺,巨細必陳,由是顧待益厚。



 建中四年,硃泚謀逆,從駕幸奉天。時天下叛亂,機務填委,徵發指
 蹤,千端萬緒,一日之內,詔書數百。贄揮翰起草,思如泉注,初若不經思慮,既成之後,莫不曲盡事情,中於機會;胥吏簡札不暇,同舍皆伏其能。轉考功郎中,依前充職。嘗啟德宗曰:「今盜遍天下,輿駕播遷,陛下宜痛自引過,以感動人心。昔成湯以罪己勃興,楚昭以善言復國。陛下誠能不吝改過,以言謝天下,使書詔無忌,臣雖愚陋,可以仰副聖情,庶令反側之徒,革心向化。」德宗然之。故奉天所下書詔,雖武夫悍卒,無不揮涕感激,多贄所為
 也。



 其年冬,議欲以新歲改元。而卜祝之流,皆以國家數鐘百六,凡事宜有變革,以應時數。上謂贄曰:「往年群臣請上尊號『聖神文武』四字,今緣寇難,諸事並宜改更,眾欲朕舊號之中更加一兩字,其事何如?」贄奏曰:「尊號之興,本非古制。行於安泰之日,已累謙沖;襲乎喪亂之時,尤傷事體。今者鑾輿播越,未復宮闈,宗社震驚,尚愆禋祀,中區多梗,大憝猶存。此乃人情向背之秋,天意去就之際,陛下宜深自懲勵,收攬群心,痛自貶損,以謝靈譴,
 不可近從末議,重益美名。」帝曰:「卿所奏陳,雖理體甚切,然時運必須小有改跡,亦不可執滯,卿更思量。」贄曰:「古之人君稱號,或稱皇稱帝,或稱王,但一字而已。至暴秦,乃兼皇帝二字,後代因之。及昏僻之君,乃有聖劉、天元之號。是知人主輕重,不在自稱,崇其號無補於徽猷;損其名不傷其德美。然而損之有謙光稽古之善,崇之獲矜能納諂之譏,得失不侔,居然可辨。況今時遭迍否,事屬傾危,尤宜懼思,以自貶抑。必也俯稽術數,須有變更。與
 其增美稱而失人心,不若黜舊號以祗天戒。天時人事,理必相符,人既好謙,天亦助順。陛下誠能斷自宸鑒,煥發德音,引咎降名,深示刻責,惟謙與順,一舉而二美從之。」德宗從之,但改興元年號而已。



 初,德宗倉皇出幸,府藏委棄,凝冽之際,士眾多寒,服御之外,無尺縑丈帛。及賊泚解圍,諸籓貢奉繼至,乃於奉天行在貯貢物於廊下,仍題曰「瓊林」、「大盈」二庫名。贄諫曰:



 「瓊林」、「大盈」,自古悉無其制,傳諸耆舊之說,皆云創自開元。貴臣貪權,飾巧
 求媚,乃言:「郡邑貢賦所用,盍各區分:賦稅當委於有司,以給經用;貢獻宜歸於天子,以奉私求。」玄宗悅之。新是二庫,蕩心侈欲,萌柢於茲,迨乎失邦,終以餌寇。《記》曰:「貨悖而入,必悖而出。」豈其效歟!



 陛下嗣位之初,務遵理道,敦行儉約,斥遠貪饕。雖內庫舊藏,未歸太府,而諸方曲獻,不入禁闈,清風肅然,海內丕變。近以寇逆亂常,鑾輿外幸,既屬憂危之運,宜增儆勵之誠。臣昨奉使軍營,出經行殿,忽睹右廓之下,榜列二庫之名,戄然若驚,不識
 所以。何者?天衢尚梗,師旅方殷,痛心呻吟之聲,噢咻未息;忠勤戰守之效,賞賚未行。諸道貢珍,遽私別庫,萬目所視,孰能忍情?竊揣軍情,或生觖望,或忿形謗讟,或醜肆謳謠,頗含思亂之情,亦有悔忠之意。是知氓俗昏鄙,識昧高卑,不可以尊極臨,而可以誠義感。



 頃者六師初降,百物無儲,外捍兇徒,內防危堞,晝夜不息,殆將五旬。凍餓交侵,死傷相枕,畢命同力,竟夷大艱。良以陛下不厚其身,不私其欲,絕甘以同卒伍,輟食以啖功勞。無猛
 制人而不攜,懷所感也;無厚賞士而不怨,悉所無也。今者攻圍已解,衣食已豐,而謗讟方興,軍情稍沮,豈不以勇夫常性,嗜貨矜功,其患難既與之同憂,而好樂不與之同利,茍異恬默,能無怨咨!此理之常,故不足怪。《記》曰:「財散則民聚。」豈其效歟!陛下天資英聖,見善必遷,是將化蓄怨為銜恩,反過差為至當,促殄遺寇,永垂鴻名,大聖應機,固當不俟終日。



 上嘉納之,令去其題署。



 興元元年,李懷光異志已萌,欲激怒諸軍,上表論諸軍衣糧薄,
 神策衣糧厚,厚薄不均,難以驅戰,意在撓沮進軍。李晟密奏,恐其有變,上憂之,遣贄使懷光軍宣諭。使還,贄奏事曰:



 賊泚稽誅,保聚宮苑,勢窮援絕,引日偷生。懷光總仗順之軍,乘制勝之氣,鼓行芟翦,易若摧枯。而乃寇奔不追,師老不用,諸帥每欲進取,懷光輒沮其謀。據茲事情,殊不可解。陛下意在全護,委曲聽從,觀其所為,亦未知感。若不別為規略,漸相制持,唯以姑息求安,終恐變故難測。此誠事機危迫之秋也,故不可以尋常容易處
 之。



 今李晟奏請移軍,適遇臣銜命宣慰,懷光偶論此事,臣遂泛問所宜,懷光乃云:「李晟既欲別行,某亦都不要藉。」臣猶慮有翻覆,因美其軍強盛。懷光大自矜誇,轉有輕晟之意。臣又從容問云:「昨發離行在之日,未知有此商量;今日從此卻回,或恐聖旨顧問,事之可否,決定何如?」懷光已肆輕言,不可中變,遂云:「恩命許去,事亦無妨。」要約再三,非不詳審,雖欲追悔,固難為詞。伏望即以李晟表出付中書,敕下依奏,別賜懷光手詔,示以移軍事
 由。其手詔大意云:「昨得李晟奏,請移軍城東以分賊勢。朕緣未知利害,本欲委卿商量,適會陸贄從彼宣慰回,雲見卿論敘軍情,語及於此,仍言許去,事亦無妨,遂敕本軍允其所請。卿宜授以謀略,分路夾攻,務使葉齊,克平寇孽。」如此詞婉而直,理當而明,雖蓄異端,何由起怨?



 臣初奉使諭旨,本緣糧料不均,偶屬移軍,事相諧會。又幸懷光詭對,且無阻絕之言,機宜合並。若有幽贊,一失其便,後何可追,幸垂裁察!



 德宗初望懷光回意破賊,故
 晟屢奏移軍不許;及贄縷陳懷光反狀,乃可晟之奏,遂移軍東渭橋。而鄜坊節度李建徽、神策行營陽惠元猶在咸陽,贄慮懷光並建徽等軍,又奏曰:



 懷光當管師徒,足以獨制兇寇,逗留未進,抑有他由。所患太強,不資傍助。比者又遣李晟、李建徽、陽惠元三節度之眾附麗其營,無益成功,只憂生事。何則?四軍懸壘,群帥異心,論勢力則懸絕高卑,據職名則不相統屬。懷光輕晟等兵微位下,而忿其制不從心。晟等疑懷光養寇蓄奸,而怨其
 事多陵己。端居則互防飛謗,欲戰則遞恐分功,齟齬不和,嫌釁遂構,俾之同處,必不兩全。強者惡積而後亡,弱者勢危而先覆,覆亡之禍,翹足可期。舊寇未平,新患方起,憂嘆所切,實堪疚心。太上消慝於未萌,其次救失於始兆,況乎事情已露,禍難垂成,委而不謀,何以制亂?李晟見機慮變,先請移軍就東,建徽、惠元,勢轉孤弱,為其吞噬,理在必然。他日雖有良圖,亦恐不能自拔,拯其危急,唯在此時。今因李晟願行,便遣合軍同往,托言晟兵
 素少,慮為賊泚所邀,藉此兩軍,迭為掎角。仍先諭旨,密使促裝,詔書至營,即日進路。懷光意雖不欲,然亦計無所施。是謂先人有奪人之心,疾雷不及掩耳者也。



 夫制軍馭將,所貴見情,離合疾徐,各有宜適。當離者合之則召亂,當合者離之則寡功;當疾而徐則失機,當徐而疾則漏策。得其要,契其時,然後舉無敗謀,措無危勢。而今者屯兵而不肯為用,聚將而罔能葉心,自為鯨鯢,變在朝夕。留之不足以相制,徒長歷階;析之各競於擅能,或
 成勛績。事有必應,斷無可疑。



 德宗曰:「卿之所料極善。然李晟移軍,懷光心已惆悵,若更遣建徽、惠元就東,則使得為詞。且俟旬時。」晟至東渭橋,不旬日,懷光果奪兩節度兵,建徽單騎遁而獲免,惠元中路被執,害之。報至行在,人情大恐。翌日,移幸山南。贄練達兵機,率如此類。



 二月,從幸梁州,轉諫議大夫,依前充學士。先是,鳳翔衙將李楚琳乘涇師之亂,殺節度使張鎰,歸款硃泚。及奉天解圍,楚琳遣使貢奉,時方艱阻,不獲已,命為鳳翔節度
 使。然德宗忿其弒逆,心不能容,才至漢中,欲令渾瑊代為節度。贄諫曰:「楚琳之罪,固不容誅,但以乘輿未復,大憝猶存,勤王之師,悉在畿內,急宣速告,晷刻是爭。商嶺則道迂且遙,駱谷復為賊所扼,僅通王命,唯在褒斜,此路若又阻艱,南北便成隔絕。以諸鎮危疑之勢,居二逆誘脅之中,恟々群情,各懷向背。賊勝則往,我勝則來,其間事機,不容差跌。儻楚琳發憾,公肆猖狂,南塞要沖,東延巨猾,則我咽喉梗而心膂分矣,其勢豈不病哉!」上釋
 然開悟,乃善待楚琳使,優詔安慰其心。



 德宗至梁,欲以穀口已北從臣賜號曰「奉天定難功臣」,穀口已南隨扈者曰「元從功臣」,不選朝官,一例俱賜。贄奏曰:「破賊捍難,武臣之效。至如宮闈近侍,班列員僚,但馳走從行而已,忽與介胄奮命之士,俱號功臣,伏恐武臣憤惋。」乃止。



 李晟既收京城,遣中使宣付翰林院具錄先散失宮人名字,令草詔賜渾瑊,遣於奉天尋訪,以得為限,仍量與資糧送赴行在。贄不時奉詔,進狀論之曰:



 頃以理道乖錯,
 禍亂薦鐘,陛下思咎懼災,裕人罪己,屢降大號,誓將更新。天下之人,垂涕相賀,懲忿釋怨,煦仁戴明,畢力同心,共平多難。止土崩於絕岸,收版蕩於橫流,殄寇清都,不失舊物。實由陛下至誠動於天地,深悔感於神人,故得百靈降康,兆庶歸德。茍不如此,自古何嘗有捐棄宮闕,失守宗祧,繼逆於赴難之師,再遷於蒙塵之日,不逾半歲,而復興大業者乎!



 今渠魁始平,法駕將返,近自郊甸,遠周寰瀛,百役疲瘵之氓,重戰傷殘之卒,皆忍死扶病,
 傾耳聳肩,想聞德聲,翹望聖澤。陛下固當感上天悔禍之眷,荷列祖垂裕之休,念將士鋒刃之殃,愍黎元塗炭之酷。以致寇為戒,以居上為危,以務理為憂,以復宮為急。損之又損,尚懼汰侈之易滋;艱之惟艱,猶患戒慎之難久。謀始盡善,克終已稀;始而不謀,終則何有!夫以內人為號,蓋是中壺末流。天子之尊,富有宮掖,如此等輩,固繁有徒,但恐傷多,豈憂乏使!翦除元惡,曾未浹辰,奔賀往來,道途如織。何必自虧君德,首訪婦人,又令資裝
 速赴行在!萬目閱視,眾口流傳,恐非所以答慶賴之心,副惟新之望也。



 夫事有先後,義有重輕,重者宜先,輕者宜後。武王克殷,有未及下車而為之者,有下車而為之者,蓋美其不失先後之宜也。自翠華播越,萬姓靡依,清廟震驚,三時乏祀,當今所務,莫大於斯。誠宜速遣大臣,馳傳先往,迎復神主,修整郊壇,展禋享之儀,申告謝之意。然後吊恤死義,慰犒有功,綏輯黎蒸,優問耆耋。安定反側,寬宥脅從;宣暢鬱堙,褒獎忠直;官失職之士,復廢
 業之人。是皆宜先,不可後也。至如崇飾服器,繕緝殿臺,備耳目之娛,選巾櫛之侍,是皆宜後,不可先也。



 散失內人,已經累月,既當離亂之際,必為將士所私。其人若稍有知,不求當自陳獻;其人若甚無識,求之適使憂虞。自因寇亂喪亡,頗有大於此者,一聞搜索,懷懼必多;餘孽尚繁,群情未一,因而善撫,猶恐危疑,若又懼之,於何不有!昔人所以掩絕纓而飲盜馬者,豈必忘其情愛,蓋知為君之體然也。以小妨大,明者不為。天下固多褻人,何
 必獨在於此。所令撰賜渾瑊詔書,未敢順旨。



 帝遂不降詔,但遣使而已。



 德宗還京,轉中書舍人,學士如故。初,贄受張鎰知,得居內職;及鎰為盧杞所排,贄常憂惴;及杞貶黜,始敢上書言事。德宗好文,益深顧遇。奉天解圍後,德宗言及違離宗廟,嗚咽流涕曰:「致寇之由,實朕之過。」贄亦流涕而對曰:「臣思致今日之患者,群臣之罪也。」贄意蓋為盧杞、趙贊等也。上欲掩杞之失,則曰:「雖朕德薄,致茲禍亂,亦運數前定,事不由人。」贄又極言杞等罪狀,
 上雖貌從,心頗不說。吳通微兄弟俱在翰林,亦承德宗寵遇,文章才器不迨贄;而能交結權幸,共短贄於上前。故劉從一、姜公輔自卑品蒼黃之中,皆登輔相;而贄為朋黨所擠,同職害其能,加以言事激切,動失上之歡心,故久之不為輔相。其於議論應對,明練理體,敷陳剖判,下筆如神,當時名流,無不推挹。



 貞元初,李抱真入朝,從容奏曰:「陛下幸奉天、山南時,赦書至山東,宣諭之時,士卒無不感泣。臣即時見人情如此,知賊不足平也。」



 時贄
 母韋氏在江東,上遣中使迎至京師,搢紳榮之。俄丁母憂,東歸洛陽,寓居嵩山豐樂寺。籓鎮賻贈及別陳餉遺,一無所取。與韋皋布衣時相善,唯西川致遺,奏而受之。贄父初葬蘇州,至是欲合葬。上遣中使護其柩車至洛,其禮遇如此。免喪,權知兵部侍郎,依前充學士。申謝日,贄伏地而泣,德宗為之改容敘慰。恩遇既隆,中外屬意為輔弼,而宰相竇參素忌贄,贄亦短參之所為,言參黷貨,由是與參不平。



 七年,罷學士,正拜兵部侍郎,知貢舉。
 時崔元翰、梁肅文藝冠時,贄輸心於肅。肅與元翰推薦藝實之士,升第之日,雖眾望不愜,然一歲選士,才十四五,數年之內,居臺省清近者十餘人。



 八年四月,竇參得罪,以贄為中書侍郎、門下同平章事。贄久為邪黨所擠,困而得位,意在不負恩獎,悉心報國,以天下事為己任。上即位之初,用楊炎、盧杞秉政,樹立朋黨,排擯良善,卒致天下沸騰,鑾輿奔播。懲是之失,貞元已後,雖立輔臣,至於小官除擬,上必再三詳問,久之方下。及贄知政事,
 請許臺省長官自薦屬官,仍保任之,事有曠敗,兼坐舉主。上許之,俄又宣旨曰:「外議云:『諸司所舉,多引用親黨,兼通賂遺,不得實才。』此法行之非便,今後卿等宜自選擇,勿用諸司延薦。」贄論奏曰:



 臣實頑鄙,一無所堪,猥蒙任使,待罪宰相。雖懷竊位之懼,且乏知人之明,自揣庸虛,終難上報。唯知廣求才之路,使賢者各以匯徵;啟至公之門,令職司皆得自達。既蒙允許,即宜宣行。南宮舉人,才至十數,或非臺省舊吏,則是使府佐僚,累經薦延,
 多歷事任。論其資望,既不愧於班行;考其行能,又未聞於闕敗。遽以騰口,上煩聖聰,道之難行,亦可知矣!



 陛下勤求理道,務徇物情,因謂舉薦非宜,復委宰臣揀擇。其為崇任輔弼,博採輿詞,可謂聖德之盛者。然於委任責成之道,聽言考實之方,閑邪存誠,猶恐有闕。陛下既納臣言而用之,旋聞橫議而止之,於臣謀不責成,於橫議不考實,此乃謀失者得以辭其罪,議曲者得以肆其誣。率是而行,觸類而長,固無必定之計,亦無必實之言。計
 不定則理道難成,言不實則小人得志。國家之病,常必由之。昔齊桓公問管仲害霸之事,對曰:「得賢不能任,害霸也;用而不能終,害霸也;與賢人謀事,而與小人議之,害霸也。」為小人者,不必悉懷險詖,故覆邦家。蓋以其意性回邪,趣向狹促,以沮議為出眾,以自異為不群,趨近利而昧遠圖,效小信而傷大道,況又言行難保,恣其非心者乎!



 伏以宰輔,常制不過數人,人之所知,固有限極,不有遍諳諸士,備閱群才。若令悉命群官,理須展轉詢
 訪,是則變公舉為私薦,易明易又為暗投。儻如議者之言,所舉多有情故,舉於君上,且未絕私;薦於宰臣,安肯無詐!失人之弊,必又甚焉。所以承前命官,罕有不涉私謗,雖則秉鈞不一,或自行情,亦由私訪所親,轉為所賣。其弊非遠,聖鑒明知。今又將徇浮言,專任宰臣除吏,宰臣不遍諳識,踵前須訪於人。若訪親朋,則是悔其覆車,不易故轍;若訪於朝列,則是求其私薦,不如公舉之愈也。二者利害,惟陛下更詳擇焉。恐不如委任長官,慎揀僚
 屬,所揀既少,所求亦精,得賢有鑒識之名,失實當暗謬之責。人之常性,莫不愛身,況於臺省長官,皆是當朝華選,孰肯徇私妄舉,以傷名取責者耶!所謂臺省長官,即僕射、尚書、左右丞、侍郎及御史大夫、中丞是也。陛下比擇輔相,多亦出於其中。今之宰臣,則往日臺省長官也;今之臺省長官,乃將來之宰臣也,但是職名暫異,固非行業頓殊。豈有為長官之時不能舉一二屬吏,居宰臣之位則可擇千百具僚,物議悠悠,其惑斯甚。



 夫求才貴
 廣,考課貴精。求廣在於各舉所知,長吏之薦擇是也;貴精在於按名責實,宰臣之序進是也。往者則天太后踐祚臨朝,欲收人心,尤務拔擢,弘委任之意,開汲引之門,進用不疑,求訪無倦,非但人得薦士,亦許自舉其才。所薦必行,所舉輒試,其於選士之道,豈不傷於容易哉!而課責既嚴,進退皆速,不肖者旋黜,才能者驟升,是以當代謂知人之明,累朝賴多士之用。此乃近於求才貴廣,考課貴精之效也。



 陛下誕膺寶歷,思致理平,雖好賢之
 心,有逾於前哲,而得人之盛,未迨於往時。蓋由賞鑒獨任於聖聰,搜擇頗難於公舉,仍啟登延之路,罕施練核之方。遂使先進者漸益凋訛,後來者不相接續,施一令則謗沮互起,用一人則瘡磐立成。此乃失於選才太精,制法不一之患也。則天舉用之法,傷易而得人;陛下慎揀之規,太精而失士。陛下選任宰相,必異於庶官;精擇長官,必愈於末品。及至宰相獻規,長吏薦士,陛下即但納橫議,不稽始謀。是乃任以重者輕其言,待以輕者重
 其事,且又不辨所毀之虛實,不校所試之短長。人之多言,何所不至,是將使人無所措其手足,豈獨選任之道失其端而已乎!



 上雖嘉其所陳,長官薦士之詔,竟追寢之。



 國朝舊制,吏部選人,每年調集。自乾元已後,屬宿兵於野。歲或兇荒,遂三年一置選。由是選人停擁,其數猥多,文書不接,真偽難辨,吏緣為奸,注授乖濫,而有十年不得調者。贄奏吏部分內外官員為三分,計闕集人,每年置選。故選司之弊,十去七八,天下稱之。



 贄與賈耽、盧
 邁、趙憬同知政事,百司有所申覆,皆更讓不言可否。舊例,宰臣當旬,秉筆決事,每十日一易,贄請準故事,令秉筆者以應之。又以河隴陷蕃已來,西北邊常以重兵守備,謂之防秋,皆河南、江淮諸鎮之軍也,更番往來,疲於戍役。贄以中原之兵,不習邊事,及捍虜戰賊,多有敗衄,又苦邊將名目太多,諸軍統制不一,緩急無以應敵,乃上疏論其事曰:



 臣歷觀前代書史,皆謂鎮撫四夷,宰相之任,不揆闇劣,屢敢上言。誠以備邊御戎,國家之重事;
 理兵足食,備御之大經。兵不治則無可用之師,食不足則無可固之地。理兵在制置得所,足食在斂導有方。陛下幸聽愚言,先務積穀,人無加賦,官不費財,坐致邊儲,數逾百萬。諸鎮收糴,今已向終,分貯軍城,用防艱急,縱有寇戎之患,必無乏絕之憂。守此成規,以為永制,常收冗費,益贍邊農,則更經二年,可積十萬人三歲之糧矣。足食之原粗立,理兵之術未精,敢議籌量,庶備採擇。



 伏以戎狄為患,自古有之,其於制御之方,得失之論,備存
 史籍,可得而言。大抵尊即序者,則曰「非德無以化要荒」,曾莫知威不立,則德不能馴也。樂武威者,則曰「非兵無以服兇獷」,曾莫知德不修,則兵不可恃也。務和親者,則曰「要結可以睦鄰好」,曾莫知我結之而彼復解也。美長城者,則曰「設險可以固邦國而捍寇仇」,曾莫知力不足,兵不堪,則險之不能有也。尚薄伐者,則曰「驅遏可以禁侵暴而省征徭,」曾莫知兵不銳,壘不完,則遏之不能勝,驅之不能去也。議邊之要,略盡於斯,雖互相譏評,然各
 有偏駁。聽一家之說,則例理可徵;考歷代所行,則成敗異效。是由執常理以御其不常之勢,徇所見而昧於所遇之時。



 夫中夏有盛衰,夷狄有強弱,事機有利害,措置有安危,故無必定之規,亦無長勝之法。夏后以序戎而聖化茂,古公以避狄而王業興;周城朔方而獫狁攘,秦築臨洮而宗社覆;漢武討匈奴而貽悔,太宗征突厥而致安;文、景約和親而不能弭患於當年,宣、元弘撫納而足以保寧於累葉。蓋以中夏之盛衰異勢,夷狄之強弱
 異時,事機之利害異情,措置之安危異便。知其事而不度其時則敗,附其時而不失其稱則成。形變不同,胡可專一!



 夫以中國強盛,夷狄衰微,而能屈膝稱臣,歸心受制,拒之則阻其向化,威之則類於殺降,安得不存而撫之,即而序之也?又如中國強盛,夷狄衰微,而尚棄信奸盟,蔑恩肆毒,諭之不變,責之不懲,安得不取亂推亡,息人固境也?其有遇中國喪亡之弊,當夷狄強盛之時,圖之則彼釁未萌,御之則我力不足,安得不卑詞降禮,約
 好通和,啖之以親,紓其交禍?縱不必信,且無大侵,雖非御戎之善經,蓋時事亦有不得已也。儻或夷夏之勢,強弱適同,撫之不寧,威之不靖;力足以自保,不足以出攻,得不設險以固軍,訓師以待寇,來則薄伐以遏其深入,去則攘斥而戒於遠追?雖為安邊之令圖,蓋勢力亦有不得不然也。故夏之即序,周之於攘,太宗之翦亂,皆乘其時而善用其勢也。古公之避狄,文、景之和親,神堯之降禮,皆順其時而不失其稱也。秦皇之長城,漢武之窮
 討,皆知其事而不度其時者也。向若遇孔熾之勢,行即序之方,則見侮而不從矣!乘可取之資,懷畏避之志,則失機而養寇矣!有攘卻之力,用和親之謀,則示弱而勞費矣!當降屈之時,務翦伐之略,則召禍而危殆矣!故曰:知其事而不度其時則敗,附其時而不失其稱則成。是無必定之規,亦無長勝之法,得失著效,不其然歟!至於察安危之大情,計成敗之大數,百代之不變易者,蓋有之矣。其要在於失人肆欲則必蹶,任人從眾則必全,此
 乃古今所同,而物理之所壹也。



 國家自祿山構亂、河隴用兵以來,肅宗中興,撤邊備以靖中邦,借外威以寧內難。於是吐蕃乘釁,吞噬無厭;回紇矜功,憑陵亦甚。中國不遑振旅,四十餘年。使傷耗遺氓,竭力蠶織,西輸賄幣,北償馬資,尚不足塞其煩言,滿其驕志。復乃遠征士馬,列戍疆陲,猶不能遏其奔沖,止其侵侮。小入則驅略黎庶,深入則震驚邦畿。時有議安邊策者,多務於所難而忽於所易,勉於所短而略於所長。遂使所易所長者,行
 之而其要不精;所難所短者,圖之而其功靡就。憂患未弭,職斯之由。



 夫制敵行師,必量事勢,勢有難易,事有先後。力大而敵脆,則先其所難,是謂奪人之心,暫勞而永逸者也;力寡而敵堅,則先其所易,是謂固國之本,觀釁而後動者也。頃屬多故,人勞未瘳,而欲廣發師徒,深踐寇境,復其侵地,攻其堅城,前有勝負未必之虞,後有饋運不繼之患。儻或撓敗,適所以啟戎心而挫國威,以此為安邊之謀,可謂不量事勢而務於所難矣!



 天之授者,
 有分事,無全功;地之產者,有物宜,無兼利。是以五方之俗,長短各殊。長者不可逾,短者不可企;勉所短而敵其所長必殆,用所長而乘其所短必安。強者,乃以水草為邑居,以射獵供飲茹,多馬而尤便馳突,輕生而不恥敗亡,此戎狄之所長也。戎狄之所長,乃中國之所短;而欲益兵蒐乘,角力爭驅,交鋒原野之間,決命尋常之內,以此為御寇之術,可謂勉所短而校其所長矣!務所難,勉所短,勞費百倍,終於無成。雖果成之,不挫則廢,豈不以
 越天授而違地產,虧時勢以反物宜者哉!



 將欲去危就安,息費從省,在慎守所易,精用所長而已。若乃擇將吏以撫寧眾庶,修紀律以訓齊師徒,耀德以佐威,能邇以柔遠;禁侵抄之暴以彰吾信,抑攻取之議以安戎心;彼求和則善待而勿與結盟,彼為寇則嚴備而不務報復,此當今之所易也。賤力而貴智,惡殺而好生,輕利而重人,忍小以全大,安其居而後動,俟其時而後行。是以修封疆,守要害,塹蹊隧,壘軍營,謹禁防,明斥候,務農以足
 食,練卒以蓄威,非萬全不謀,非百克不鬥。寇小至則張聲勢以遏其入,寇大至則謀其人以邀其歸;據險以乘之,多方以誤之。使其勇無所加,眾無所用;掠則靡獲,攻則不能;進有腹背受敵之虞,退有首尾難救之患,所謂乘其弊,不戰而屈人之兵,此中國之所長也。我之所長,乃戎狄之所短;我之所易,乃戎狄之所難。以長制短,則用力寡而見功多;以易敵難,則財不匱而事速就。舍此不務,而反為所乘,斯謂倒持戈矛,以金尊授寇者也!今則
 皆務之矣,猶且守封未固,寇戎未懲者,其病在於謀無定用,眾無適從。所任不必才,才者不必任;所聞不必實,實者不必聞;所信不必誠,誠者不必信;所行不必當,當者未必行。故令措置乖方,課責虧度;財匱於兵眾,力分於將多,怨生於不均,機失於遙制。臣請為陛下粗陳六者之失,惟明主慎聽而熟察之:



 臣聞工欲善其事,必先利其器;武欲勝其敵,必先練其兵。練兵之中,所用復異。用之於救急,則權以紓難;用之於暫敵,則緩以應機。故
 事有便宜,而不拘常制;謀有奇詭,而不徇眾情。進退死生,唯將所命,此所謂攻討之兵也!用之於屯戍,則事資可久;勢異從權,非物理所愜不寧,非人情所欲不固。夫人情者,利焉則勸,習焉則安,保親戚則樂生,顧家業則忘死,故可以理術馭,不可以法制驅,此所謂鎮守之兵也。夫欲備封疆,御戎狄,非一朝一夕之事,固當選鎮守之兵以置焉。古之善選置者,必量其性習,辨其土宜,察其伎能,知其欲惡。用其力而不違其性,齊其俗而不易
 其宜;引其善而不責其所不能,禁其非而不處其所不欲。而又類其紀伍,安其室家,然後能使之樂其居,定其志,奮其氣勢,結其恩情。撫之以惠,則感而不驕;臨之以威,則肅而不怨。靡督課而人自為用,弛禁防而眾自不攜。故出則足兵,居則足食,守則固,戰則強。其術無他,便於人情而已矣!今者散徵士卒,分戍邊陲,更代往來,以為守備。是則不量性習,不辨土宜,邀其所不能,強其所不欲。求廣其數而不考其用,將致其力而不察其情,斯
 可以為羽衛之儀,而無益於備御之實也。何者?窮邊之地,千里蕭條,寒風裂膚,驚沙慘目;與豺狼為鄰伍,以戰鬥為嬉游;晝則荷戈而耕,夜則倚烽而覘;日有剽害之慮,永無休暇之娛,地惡人勤,於斯為甚!自非生於其域,習於其風,幼而睹焉,長而安焉,不見樂土而遷焉,則罕能寧其居而狎其敵也。關東之地,百物阜殷,從軍之徒,尤被優養。慣於溫飽,狎於歡康,比諸邊隅,若異天地。聞絕塞荒陬之苦,則辛酸動容;聆強蕃勁虜之名,則懾駭
 奪氣。而乃使之去親族,舍園廬,甘其所辛酸,抗其所懾駭,將冀為用,不亦疏乎!矧又有休代之期,無統帥之馭,資奉若驕子,姑息如倩人,進不邀之以成功,退不處之以嚴憲。其來也咸負得色,其止也莫有固心,屈指計歸,張頤待飼。徼倖者猶患還期之賒緩,常念戎醜之充斥;王師挫傷,則將乘其亂離,布路東潰,情志且爾,得之奚為?平居則殫耗資儲以奉浮冗之眾,臨難則拔棄城鎮以搖遠近之心,其弊豈惟無益哉!固亦將有所撓也。復
 有抵犯刑禁,謫徙軍城,意欲增戶實邊,兼令展效自贖。既是無良之類,且加懷土之情,思亂幸災,又甚戍卒。適足煩於防衛,諒無望於功庸,雖前代時或行之,固非良算之可遵者也。復有擁旄之帥,身不臨邊,但分偏師,俾守疆場。大抵軍中壯銳,元戎例選自隨,委其疲羸,乃配諸鎮。節將既居內地,精兵祗備紀綱,遂令守要禦沖,常在寡弱之輩。寇戎每至,乃勢不支,入壘者才足閉關,在野者悉遭劫執,恣其芟蹂,盡其搜驅。比及都府聞知,虜
 已克獲旋返。且安邊之本,所切在兵,理兵若斯,可謂措置乖方矣!



 夫賞以存勸,罰以示懲,勸以懋有庸,懲以威不恪。故賞罰之於馭眾也,猶繩墨之於曲直,權衡之揣重輕,輗軏之所以行車,銜勒之所以服馬也。馭眾而不用賞罰,則善惡相混而能否莫殊;用之而不當功過,則奸妄寵榮而忠實擯抑。夫如是,若聰明可衒,律度無章,則用與不用,其弊一也。自頃權移於下,柄失於朝,將之號令,既鮮克行之於軍,國之典章,又不能施之於將,務
 相遵養,茍度歲時。欲賞一有功,翻慮無功者反側;欲罰一有罪,復慮同惡者憂虞。罪以隱忍而不彰,功以嫌疑而不賞,姑息之道,乃至於斯。故使忘身效節者,獲誚於等夷;率眾先登者,取怨於士卒;僨軍蹙國者,不懷於愧畏;緩救失期者,自以為智能。褒貶既闕而不行,稱毀復紛然相亂,人雖欲善,誰為言之?況又公忠者,直己而不求於人,反罹困厄;敗撓者,行私而茍媚於眾,例獲優崇。此義士所以痛心,勇夫所以解體也。又有遇敵而所守
 不固,陳謀而其效靡成;將帥則以資糧不足為詞,有司復以供給無闕為解。既相執證,理合辨明,朝廷每為含糊,未嘗窮究曲直。措理者吞聲而靡訴,誣善者罔上而不慚。馭眾若斯,可謂課責虧度矣!



 課責虧度,措置乖方,將不得竭其材,卒不得盡其力,屯集雖眾,戰陣莫前。虜每越境橫行,若涉無人之地;遞相推倚,無敢誰何,虛張賊勢上聞,則曰兵少不敵。朝廷莫之省察,惟務徵發益師,無裨備御之功,重增供億之弊。閭井日耗,徵求日繁,
 以編戶傾家破產之資,兼有司榷鹽稅酒之利,總其所入,半以事邊,制用若斯,可謂財匱於兵眾矣!



 今四夷之最強盛為中國甚患者,莫大於吐蕃,舉國勝兵之徒,才當中國十數大郡而已。其於內虞外備,亦與中國不殊,所能寇邊,數則蓋寡。且又器非犀利,甲不堅完,識迷韜鈐,藝乏趫敏。動則中國畏其眾而不敢抗,靜則中國憚其強而不敢侵,厥理何哉?良以中國之節制多門,蕃丑之統帥專一故也。夫統帥專則人心不分,人心不分則
 號令不貳,號令不貳則進退可齊,進退可齊則疾徐如意,疾徐如意則機會靡愆,機會靡愆則氣勢自壯!斯乃以少為眾,以弱為強,變化翕闢,在於反掌之內。是猶臂之使指,心之制形,若所任得人,則何敵之有!夫節制多門則人心不一,人心不一則號令不行,號令不行則進退難必,進退難必則疾徐失宜,疾徐失宜則機會不及,機會不及則氣勢自衰!斯乃勇廢為尪,眾散為弱,逗撓離析,兆乎戰陣之前。是猶一國三公,十羊九牧,欲令齊
 肅,其可得乎?開元、天寶之間,控御西北兩蕃,唯朔方、河西、隴右三節度而已,猶慮權分勢散,或使兼而領之。中興已來,未遑外討,僑隸四鎮於安定,權附隴右於扶風,所當西北兩蕃,亦朔方、涇原、隴右、河東節度而已,關東戍卒,至則屬焉。雖委任未盡得人,而措置尚存典制。自頃逆泚誘涇、隴之眾叛,懷光污朔方之軍,割裂誅鋤,所餘無幾。而又分朔方之地,建牙擁節者,凡三使焉。其餘鎮軍,數且四十,皆承特詔委寄,各降中貴監臨,人得抗
 衡,莫相稟屬。每俟邊書告急,方令計會用兵,既無軍法下臨,唯以客禮相待。是乃從容拯溺,揖讓救焚,冀無阽危,固亦難矣!夫兵,以氣勢為用者也,氣聚則盛,散則消;勢合則威,析則弱。今之邊備,勢弱氣消,建軍若斯,可謂力分於將多矣。



 理戎之要,最在均齊,故軍法無貴賤之差,軍實無多少之異,是將所以同其志而盡其力也。如或誘其志意,勉其藝能,則當閱其材,程其勇,校其勞逸,度其安危,明申練覆優劣之科,以為衣食等級之制。使
 能者企及,否者息心,雖有薄厚之殊,而無觖望之釁。蓋所謂日省月試,餼稟均事,如權量之無情於物,萬人莫不安其分而服其平也。今者窮邊之地,長鎮之兵,皆百戰傷夷之餘,終年勤苦之劇,角其所能則練習,度其所處則孤危,考其服役則勞,察其臨敵則勇。然衣糧所給,唯止當身,例為妻子所分,常有凍餒之色。而關東戍卒,歲月踐更,不安危城,不習戎備,怯於應敵,懈於服勞。然衣糧所頒,厚逾數等,繼以茶藥之饋,益以蔬醬之資。豐
 約相形,懸絕斯甚。又有素非禁旅,本是邊軍,將校詭為媚詞,因請遙隸神策,不離舊所,唯改虛名,其於稟賜之饒,遂有三倍之益。此儔類所以忿恨,忠良所以憂嗟,疲人所以流亡,經費所以褊匱。夫事業未異,而給養有殊,人情之所不能甘也,況乎矯佞行而稟賜厚,績藝劣而衣食優,茍未忘懷,能無慍怒!不為戎首,則已可嘉,而欲使其協力同心,以攘寇難,雖有韓、白、孫、吳之將,臣知其必不能焉。養士若斯,可謂怨生於不均矣!



 凡欲選任將
 帥,必先考察行能,然後指以所授之方,語以所委之事,令其自揣可否,自陳規模。須某色甲兵,藉某人參佐,要若干士馬,用若干資糧,某處置軍,某時成績,始終要領,悉俾經綸,於是觀其計謀,校其聲實。若謂材無足取,言不可行,則當退之於初,不宜貽慮於其後也。若謂志氣足任,方略可施,則當要之於終,不宜掣肘於其間也。夫如是,則疑者不使,使者不疑;勞神於選才,端拱於委任。既委其事,既足其求,然後可以核其否臧,行其賞罰。受
 賞者不以為濫,當罰者無得而辭,付授之柄既專,茍且之心自息。是以古之遣將帥者,君親推轂而命之曰:「自閫以外,將軍裁之。」又賜鈇鉞,示令專斷。故軍容不入國,國容不入軍,將在軍,君命有所不受。誠謂機宜不可以遠決,號令不可以兩從,未有委任不專,而望其克敵成功者也。自頃邊軍去就,裁斷多出宸衷,選置戎臣,先求易制,多其部以分其力,輕其任以弱其心,雖有所懲,亦有所失。遂令分閫責成之義廢,死綏任咎之志衰,一則
 聽命,二亦聽命,爽於軍情亦聽命,乖於事宜亦聽命。若所置將帥,必取於承順無違,則如斯可矣;若有意平兇靖難,則不可。夫兩境相接,兩軍相持,事機之來,間不容息,蓄謀而俟,猶恐失之,臨時始謀,固已疏矣。況乎千里之遠,九重之深,陳述之難明,聽覽之不一,欲其事無遺策,雖聖者亦有所不能焉。設使謀慮能周,其如權變無及!戎虜馳突,迅如風飆,驛書上聞,旬月方報。守土者以兵寡不敢抗敵,分鎮者以無詔不肯出師,逗留之間,寇
 已奔逼,托於救援未至,各且閉壘自全。牧馬屯牛,鞠為椎剽;穡夫樵婦,罄作俘囚。雖詔諸鎮發兵,唯以虛聲應援,互相瞻顧,莫敢遮邀,賊既縱掠退歸,此乃陳功告捷。其敗喪則減百而為一,其捃獲則張百而成千。將帥既幸於總制在朝,不憂於罪累;陛下又以為大權由己,不究事情。用師若斯,可謂機失於遙制矣!



 理兵而措置乖方,馭將而賞罰虧度,制用而財匱,建兵而力分,養士而怨生,用師而機失,此六者,疆場之蟊賊,軍旅之膏肓也。
 蟊賊不除,而但滋之以糞溉,膏肓不療,而唯啖之以滑甘,適足以養其害,速其災,欲求稼穡豐登,膚革充美,固不可得也。



 臣愚謂宜罷諸道將士番替防秋之制,率因舊數而三分之:其一分委本道節度使募少壯願住邊城者以徙焉;其一分則本道但供衣糧,委關內、河東諸軍州募蕃、漢子弟願傅邊軍者以給焉;又一分亦令本道但出衣糧,加給應募之人,以資新徙之業。又令度支散於諸道和市耕牛,兼雇召工人,就諸軍城繕造器具。募
 人至者,每家給耕牛一頭,又給田農水火之器,皆令充備。初到之歲,與家口二人糧,並賜種子,勸之播植,待經一稔,俾自給家。若有餘糧,官為收糴,各酬倍價,務獎營田。既息踐更徵發之煩,且無幸災茍免之弊。寇至則人自為戰,時至則家自力農。是乃兵不得不強,食不得不足,與夫倏來忽往,豈可同等而論哉!



 臣又謂宜擇文武能臣一人為隴右元帥,應涇、隴、鳳翔、長武城、山南西道等節度管內兵馬,悉以屬焉;又擇一人為朔方元帥,
 應鄜坊、邠寧、靈夏等節度管內兵馬,悉以屬焉;又擇一人為河東元帥,河東、振武等節度管內兵馬,悉以屬焉。三帥各選臨邊要會之州以為理所,見置節度,有非要者,隨所便近而並之。唯元帥得置統軍,餘並停罷。其三帥部內太原、鳳翔等府及諸郡戶口稍多者,慎揀良吏以為尹守,外奉師律,內課農桑,俾為軍糧,以壯戎府。理兵之宜既得,選帥之授既明,然後減奸濫虛浮之費以豐財,定衣糧等級之制以和眾,弘委任之道以宣其用,
 懸賞罰之典以考其成。而又慎守中國之所長,謹行當今之所易,則八利可致,六失可除。如是而戎狄不威懷,疆場不寧謐者,未之有也。諸侯軌道,庶類服從。如是而教令不行,天下不理者,亦未之有也。以陛下之英鑒,民心之思安,四方之小休,兩寇之方靜,加以頻年豐稔,所在積糧,此皆天贊國家,可以立制垂統之時也。時不久居,事不常兼,已過而追,雖悔無及。明主者,不以言為罪,不以人廢言,罄陳狂愚,惟所省擇。



 德宗極深嘉納,優詔
 褒獎之。



 贄在中書,政不便於時者,多所條奏。德宗雖不能皆可,而心頗重之。初,竇參既貶郴州,節度使劉士寧餉參絹數千匹。湖南觀察使李巽與參有隙,具事奏聞,德宗不悅。會右庶子姜公輔於上前聞奏,稱「竇參嘗語臣云:陛下怒臣未已」,德宗怒,再貶參,竟殺之。時議云公輔奏竇參語得之於贄,雲參之死,贄有力焉。又素惡於公異、於邵,既輔政而逐之,談者亦以為厄。



 戶部侍郎、判度支裴延齡,奸宄用事,天下嫉之如仇。以得幸於天子,
 無敢言者。贄獨以身當之,屢於延英面陳其不可,累上疏極言其弊。延齡日加譖毀。十年十二月,除太子賓客,罷知政事。贄性畏慎,及策免私居,朝謁之外,不通賓客,無所過從。十一年春,旱,邊軍芻粟不給,具事論訴;延齡言贄與張滂、李充等搖動軍情,語在《延齡傳》。德宗怒,將誅贄等四人,會諫議大夫陽城等極言論奏,乃貶贄為忠州別駕。



 贄初入翰林,特承德宗異顧,歌詩戲狎,朝夕陪游。及出居艱阻之中,雖有宰臣,而謀猷參決,多出於
 贄,故當時目為「內相」。從幸山南,道途艱險,扈從不及,與帝相失,一夕不至,上喻軍士曰:「得贄者賞千金。」翌日贄謁見,上喜形顏色,其寵待如此。既與二吳不協,漸加浸潤,恩禮稍薄;及通玄敗,上知誣枉,遂復見用。贄以受人主殊遇,不敢愛身,事有不可,極言無隱。朋友規之,以為太峻,贄曰:「吾上不負天子,下不負吾所學,不恤其他。」精於吏事,斟酌決斷,不失錙銖。嘗以「詞詔所出,中書舍人之職,軍興之際,促迫應務,權令學士代之;朝野乂寧,合
 歸職分,其命將相制詔,卻付中書行譴。」又言「學士私臣,玄宗初令待詔,止於唱和文章而已」。物議是之。德宗以贄指斥通微、通玄,故不可其奏。



 贄在忠州十年,常閉關靜處,人不識其面,復避謗,不著書。家居瘴鄉,人多癘疫,乃抄撮方書,為《陸氏集驗方》五十卷,行於代。初,贄秉政,貶駕部員外郎李吉甫為明州長史,量移忠州刺史。贄在忠州,與吉甫相遇,昆弟、門人咸為贄憂,而吉甫忻然厚禮,都不銜前事,以宰相禮事之,猶恐其未信不安,日
 與贄相狎,若平生交契者。贄初猶慚懼,後乃深交。時論以吉甫為長者。後有薛延者,代吉甫為刺史,延朝辭日,德宗令宣旨慰安。而韋皋累上表請以贄代己。順宗即位,與陽城、鄭餘慶同詔徵還。詔未至而贄卒,時年五十二,贈兵部尚書,謚曰宣。



 子簡禮,登進士第,累闢使府。



 史臣曰:近代論陸宣公,比漢之賈誼,而高邁之行,剛正之節,經國成務之要,激切仗義之心,初蒙天子重知,末塗淪躓,皆相類也。而誼止中大夫,贄及臺鉉,不為不遇
 矣。昔公孫鞅挾三策說秦王,淳于髡以隱語見齊君,從古以還,正言不易。昔周昭戒急論議,正為此也。贄居珥筆之列,調飪之地,欲以片心除眾弊,獨手遏群邪,君上不亮其誠,群小共攻其短,欲無放逐,其可得乎!《詩》稱「其維哲人,告之話言」,又有「誨爾」、「聽我」之恨,此皆賢人君子,嘆言不見用也。故堯咨禹拜,千載一時,攜手提耳,豈容易哉!



 贊曰:良臣悟主,我有嘉猷。多僻之君,為善不周。忠言救
 失,啟沃曰讎。勿貽天問,蒼昊悠悠。



\end{pinyinscope}