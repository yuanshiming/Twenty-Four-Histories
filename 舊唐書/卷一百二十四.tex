\article{卷一百二十四}

\begin{pinyinscope}

 ○郭子儀子曜晞曖曙
 晤映晞子鋼曖子釗鏦釗子仲文族
 弟幼明子昕



 郭子儀,華州鄭縣人。父敬之,歷綏、渭、桂、壽、泗五州刺史,以子儀貴,贈太保,追封祁國公。子儀長六尺餘,體貌秀傑,始以武舉高等補左衛長史,累歷諸軍使。天寶八載,
 於木剌山置橫塞軍及安北都護府,命子儀領其使,拜左衛大將軍。十三載,移橫塞軍及安北都護府於永清柵北築城,仍改橫塞為天德軍,子儀為之使,兼九原太守、朔方節度右兵馬使。



 十四載,安祿山反。十一月,以子儀為衛尉卿,兼靈武郡太守,充朔方節度使,詔子儀以本軍東討。遂舉兵出單于府,收靜邊軍,斬賊將周萬頃,傳首闕下。祿山遣大同軍使高秀巖寇河曲,子儀擊敗之,進收雲中馬邑,開東陘,以功加御史大夫。十五載
 正月,賊將蔡希德陷常山郡,執顏杲卿,河北郡縣皆為賊守。二月,子儀與河東節度使李光弼率師下井陘,拔常山郡,破賊於九門,南攻趙郡,生擒賊四千,皆舍之,斬偽太守郭獻璆,獲兵仗數萬。師還常山,賊將史思明以數萬人踵其後,我行亦行,我止亦止。子儀選驍騎五百更挑之,三日至行唐,賊疲乃退,我軍乘之,又敗於沙河。祿山聞思明敗,乃以精兵益之。我軍至恆陽,賊亦隨至。子儀堅壁自固,賊來則守,賊去則追,晝揚其兵,夕襲其
 幕,賊人不及息。數日,光弼議曰:「賊怠矣,可以戰。」六月,子儀、光弼率僕固懷恩、渾釋之、陳回光等陣於嘉山,賊將史思明、蔡希德、尹子奇等亦結陣而至,一戰敗之,斬馘四萬級,生擒五千人,獲馬五千匹,思明露發跣足奔於博陵。於是河北十餘郡皆斬賊守者以迎王師。子儀將北圖範陽,軍聲大振。



 是月,哥舒翰為賊所敗,潼關不守,玄宗幸蜀,肅宗幸靈武,子儀副使杜鴻漸為朔方留後,奏迎車駕。七月,肅宗即位,以賊據兩京,方謀收復,詔子
 儀班師。八月,子儀與李光弼率步騎五萬至自河北。時朝廷初立,兵眾寡弱,雖得牧馬,軍容缺然。及子儀、光弼全師赴行在,軍聲遂振,興復之勢,民有望焉。詔以子儀為兵部尚書、同中書門下平章事,依前靈州大都督府長史、朔方軍節度使。肅宗大閱六軍,南趨關輔,至彭原郡,宰相房琯請兵萬人,自為統帥以討賊,帝素重琯,許之。兵及陳濤,為賊所敗,喪師殆盡。方事討除,而軍半殪,唯倚朔方軍為根本。十一月,賊將阿史那從禮以同羅、
 僕骨五千騎出塞,誘河曲九府、六胡州部落數萬,欲迫行在。子儀與回紇首領葛邏支往擊敗之,斬獲數萬,河曲平定。



 賊將崔乾祐守潼關。二年三月,子儀大破賊於潼關,崔乾祐退保蒲津。時永樂尉趙復、河東司戶韓旻、司士徐炅、宗子李藏鋒等,陷賊在蒲州,四人密謀俟王師至,則為內應。及子儀攻蒲州,趙復等斬賊守陴者,開門納子儀。乾祐與麾下數千人北走安邑,安邑百姓偽降,乾祐兵入將半,下懸門擊之,乾祐未入,遂得脫身東走。
 子儀遂收陜郡永豐倉。自是潼、陜之間無復寇鈔。



 是月,安祿山死,朝廷欲圖大舉,詔子儀還鳳翔。四月,進位司空,充關內、河東副元帥。五月,詔子儀帥師趨京城。師于潏水之西,與賊將安太清、安守忠戰,王師不利,其眾大潰,盡委兵仗於清渠之上。子儀收合餘眾,保武功,詣闕請罪,乞降官資,乃降為左僕射,餘如故。九月,從元帥廣平王率蕃漢之師十五萬進收長安。回紇遣葉護太子領四千騎助國討賊,子儀與葉護宴狎修好,相與誓平
 國難,相得甚好。子儀奉元帥為中軍,與賊將安守忠、李歸仁戰於京西香積寺之北,王師結陣橫亙三十里,賊眾十萬陳於北。歸仁先薄我軍,我軍亂,李嗣業奮命馳突,擒賊十餘騎乃定。回紇以奇兵出賊陣之後夾攻之,賊軍大潰,自午至酉,斬首六萬級。賊將張通儒守長安,聞歸仁等敗,是夜奔陜郡。翌日,廣平王入京師,老幼百萬,夾道歡叫,涕泣而言曰:「不圖今日復見官軍。」廣平王休士三日,率師東趨。肅宗在鳳翔聞捷,群臣稱賀,帝以
 宗廟被焚,悲咽不自勝,臣僚無不感泣。



 十月,安慶緒遣嚴莊悉其眾十萬來赴陜州,與張通儒同抗官軍。賊聞官軍至,悉其眾屯於陜西,負山為陣。子儀以大軍擊其前,回紇登山乘其背,遇賊潛師於山中,與鬥過期,大軍稍卻。賊分兵三千人,絕我歸路,眾心大搖,子儀麾回紇令進,盡殺之。師馳至其後,於黃埃中發十餘箭,賊驚顧曰:「回紇來!」即時大敗,殭尸遍山澤。嚴莊、張通儒走歸洛陽,遂與安慶緒渡河保相州。子儀奉廣平王入東都,陳
 兵於天津橋南,士庶歡呼於路。偽侍中陳希烈、偽中書令張垍等三百餘人素服請罪,王慰撫遣之。是時,河東、河西、河南賊所盜郡邑皆平,以功加司徒,封代國公,食邑千戶。尋入朝,天子遣兵仗戎容迎於灞上,肅宗勞之曰:「雖吾之家國,實由卿再造。」子儀頓首感謝。十二月,還東都,命子儀經營北討。乾元元年七月,破賊河上,擒偽將安守忠以獻,遂朝京師,敕百僚班迎於長樂驛,帝禦望春樓待之,進位中書令。九月,奉詔大舉,子儀與河東節
 度使李光弼、關內節度使王思禮、北庭行營節度李嗣業、襄鄧節度使魯炅、荊南節度季廣琛、河南節度使崔光遠、滑濮節度許叔冀、平盧兵馬使董秦等九節度之師討安慶緒。帝以子儀、光弼俱是元勛,難相統屬,故不立元帥。唯以中官魚朝恩為觀軍容宣慰使。十月,子儀自杏園渡河,圍衛州。安慶緒與其驍將安雄俊、崔乾祐、薛嵩、田承嗣悉其眾來援,分為三軍。子儀陣以待之,預選射者三千人伏於壁內,誡之曰:「俟吾小卻,賊必爭進,
 則登城鼓噪,弓弩齊發以迫之。」既戰,子儀偽遁,賊果乘之,及壘門,遽聞鼓噪,俄而弓弩齊發,矢注如雨,賊徒震駭,子儀整眾追之,賊眾大敗。是役也,獲偽鄭王安慶和以獻,遂收衛州。進軍趨鄴,與賊再戰於愁思岡,賊軍又敗,乃連營圍之。慶緒遣薛嵩以所乘馬十匹求救於史思明,且言禪代。十二月,思明遣將李歸仁率眾赴之,營於滏陽。



 二年正月,史思明自率範陽精卒復陷魏州,乃偽稱燕王。王師雖眾,軍無統帥,進退無所承稟,自冬徂
 春,竟未破賊,但引漳水以灌其城,城中食盡,易子而食。二月,思明率眾自魏州來。李光弼、王思禮、許叔冀、魯炅前軍遇賊於鄴南,與之接戰,夷傷相半,魯炅中流矢。子儀為後陣,未及合戰,大風遽起,吹沙拔木,天地晦暝,跬步不辯物色。我師潰而南,賊軍潰而北,委棄兵仗輜重,累積於路。諸軍各還本鎮。子儀以朔方軍保河陽,斷浮橋,有詔令留守東都。三月,以子儀為東都畿、山南東道、河南諸道行營元帥。



 中官魚朝恩素害子儀之功,因
 其不振,媒孽之,尋召還京師。天子以趙王系為天下兵馬元帥,李光弼副之,委以陜東軍事,代子儀之任。子儀雖失兵柄,乃思王室,以禍難未平,不遑寢息。俄而史思明再陷河洛,朝廷旰食,復慮蕃寇逼迫京畿,三年正月,授子儀邠寧、鄜坊兩鎮節度使,仍留京師。言事者以子儀有社稷大功,今殘孽未除,不宜置之散地,肅宗深然之。上元元年九月,以子儀為諸道兵馬都統,管崇嗣副之,令率英武、威遠等禁軍及河西、河東諸鎮之師,取邠寧、
 朔方、大同、橫野,徑抵範陽。詔下旬日,復為朝恩所間,事竟不行。



 上元二年二月,李光弼兵敗於邙山,河陽失守,魚朝恩退保陜州。三年二月,河中軍亂,殺其帥李國貞。時太原節度鄧景山亦為部下所殺,恐其合從連賊,朝廷憂之。後輩帥臣未能彈壓,勢不獲已,遂用子儀為朔方、河中、北庭、潞、儀、澤、沁等州節度行營兼興平、定國副元帥,充本管觀察處置使,進封汾陽郡王,出鎮絳州。三月,子儀辭赴鎮,肅宗不豫,群臣莫有見者。子儀請曰:「老
 臣受命,將死於外,不見陛下,目不瞑矣。」帝乃引至臥內,謂子儀曰:「河東之事,一以委卿。」子儀嗚咽流涕。賜御馬、銀器、雜彩,別賜絹四萬疋、布五萬端以賞軍。子儀至絳,擒其殺國貞賊首王元振數十人誅之。太原辛云京聞子儀誅元振,亦誅害景山者,由是河東諸鎮率皆奉法。四月,代宗即位,內官程元振用事,自矜定策之功,忌嫉宿將,以子儀功高難制,巧行離間,請罷副元帥,加實封七百戶,充肅宗山陵使。子儀既謝恩,上表進肅宗所賜
 前後詔敕,因自陳訴曰:



 臣德薄蟬翼,命輕鴻毛,累蒙國恩,猥廁朝列。會天地震蕩,中原血戰,臣北自靈武,冊先皇帝,乃舉兵而南,大搜於岐陽。先帝憂勤宗社,托臣以家國,俾副陛下掃兩京之妖昆。陛下雄圖丕斷,再造區宇,自後不以臣寡劣,委文武之二柄,外敷邦教,內調鼎飪,是以常許國家之死,實荷日月之明。臣本愚淺,言多詆直,慮此招謗,上瀆冕旒。陛下居高聽卑,察臣不貳,皇天后土,察臣無私。伏以器忌滿盈,日增兢惕,焉敢偷全,
 久妨賢路?自受恩塞下,制敵行間,東西十年,前後百戰。天寒劍折,濺血沾衣;野宿魂驚,飲冰傷骨。跋涉難阻,出沒死生,所仗唯天,以至今日。陛下曲申惠獎,念及勤勞,貽臣詔書一千餘首,聖旨微婉,慰諭綢繆,彰微臣一時之功,成子孫萬代之寶。自靈武、河北、河南、彭原、鄜坊、河東、鳳翔、兩京、絳州,臣所經行,賜手詔敕書凡二十卷,昧死上進,庶煩聽覽。



 詔答曰:「朕不德不明,俾大臣憂疑,朕之過也。朕甚自愧,公勿以為慮。」代宗以子儀頃同患難,
 收復兩京,禮之逾厚。時史朝義尚據洛陽,元帥雍王率師進討,代宗欲以子儀副之,而魚朝恩、程元振亂政,殺裴茂、來瑱,子儀既為所間,其事遂寢,乃留京師。



 俄而梁崇義據襄陽叛,僕固懷恩阻兵於汾州,引回紇、吐蕃之眾入寇河西。明年十月,吐蕃陷涇州,虜刺史高暉,暉遂與蕃軍為鄉導,引賊深入京畿,掠奉天、武功,濟渭而南,緣山而東。渭北行營兵馬使呂日將逆戰於盩厔,自辰至酉,殺蕃軍數千,然其徒多殞。賊將逼京師,君上計無
 所出,遽詔子儀為關內副元帥,出鎮咸陽。子儀自相州不利,李光弼代掌兵柄,及徵還朝廷,部曲散去。及是承詔,部下唯二十騎,強取民家畜產以助軍。至咸陽,蕃軍已過渭水。其日,天子避狄幸陜州。子儀聞上避狄,雪涕還京,至則車駕已發。射生將王獻忠從駕,沿路遂以四百騎叛,仍逼豐王已下十王欲投於賊。子儀入開遠門,遇之,詰豐王等所向,遂護送行在。子儀以三千騎傍南山,至商州,得武關防兵及六軍散卒四千人,招輯亡逸,
 其軍漸振。蕃犯京城,得故邠王守禮子廣武王承宏,立帝號,假署百官。子儀遣六軍兵馬使張知節、烏崇福、羽林軍使長孫全緒等將兵萬人為前鋒,營於韓公堆,盛張旗幟,鼓鞞震山谷。全緒遣禁軍舊將王甫入長安,陰結少年豪俠以為內應,一日,齊擊鼓於硃雀街,蕃軍惶駭而去。大將李忠義先屯兵苑中,渭北節度使王仲升守朝堂。子儀以大軍續進,至滻西。射生將王撫自署為京兆尹,聚兵二千人,擾亂京城,子儀召撫殺之。詔子
 儀權京城留守。



 自西蕃入寇,車駕東幸,天下皆咎程元振,諫官屢論之。元振懼,又以子儀復立功,不欲天子還京,勸帝且都洛陽以避蕃冠,代宗然之,下詔有日。子儀聞之,因兵部侍郎張重光宣慰回,附章論奏曰:



 臣聞雍州之地,古稱天府,右探隴、蜀,左扼崤、函,前有終南、太華之險,後有清渭、濁河之固,神明之奧,王者所都。地方數千里,帶甲十餘萬,兵強士勇,雄視八方,有利則出攻,無利則入守。此用武之國,非諸夏所同,秦、漢因之,卒成帝
 業。其後或處之而泰,去之而亡,前史所書,不唯一姓。及隋氏季末,煬帝南遷,河、洛丘墟,兵戈亂起。高祖唱義,亦先入關,惟能翦滅奸雄,底定區宇。以至於太宗、高宗之盛,中宗、玄宗之明,多在秦川,鮮居東洛。間者羯胡構亂,九服分崩,河北、河南,盡從逆命。然而先帝仗朔方之眾,慶緒奔亡;陛下藉西土之師,朝義就戮。豈唯天道助順,抑亦地形使然,此陛下所知,非臣飾說。



 近因吐蕃凌逼,鑾駕東巡。蓋以六軍之兵,素非精練,皆市肆屠沽之人,
 務掛虛名,茍避征賦,及驅以就戰,百無一堪。亦有潛輸貨財,因以求免。又中官掩蔽,庶政多荒。遂令陛下振蕩不安,退居陜服。斯蓋關於委任失所,豈可謂秦地非良者哉!今道路雲云,不知信否,咸謂陛下已有成命,將幸洛都。臣熟思其端,未見其利。夫以東周之地,久陷賊中,宮室焚燒,十不存一。百曹荒廢,曾無尺椽,中間畿內,不滿千戶。井邑榛荊,豺狼站嗥,既乏軍儲,又鮮人力,東至鄭、汴,達於徐方,北自覃懷,經於相土,人煙斷絕,千里蕭
 條。將何以奉萬乘之牲餼,供百官之次舍?矧其土地狹厄,才數百里間,東有成皋,南有二室,險不足恃,適為戰場。陛下奈何棄久安之勢,從至危之策,忽社稷之計,生天下之心。臣雖至愚,竊為陛下不取。



 且聖旨所慮,豈不以京畿新遭剽掠,田野空虛,恐糧食不充,國用有闕,以臣所見,深謂不然。昔衛文小國之君,諸侯之主耳,遭懿公為狄所滅,始廬於曹,衣大布之衣,冠大帛之冠,元年革車三十乘,季年三百乘,卒能恢復舊業,享無疆之
 休。況明明天子,躬儉節用,茍能黜素餐之吏,去冗食之官,抑豎刁、易牙之權,任蘧瑗、史鰌之直,薄征馳力,恤隱迨鰥,委諸相以簡賢任能,付老臣以練兵禦侮,則黎元自理,寇盜自平,中興之功,旬月可冀,卜年之期,永永無極矣。願時邁順動,回鑾上都,再造邦家,唯新庶政,奉宗廟以修薦享,謁陵寢以崇孝思,臣雖隕越,死無所恨。



 代宗省表,垂泣謂左右曰:「子儀用心,真社稷臣也。可亟還京師。」十一月,車駕自陜還宮,子儀伏地請罪,帝駐車
 勞之曰:「朕用卿不早,故及於此。」乃賜鐵券,圖形凌煙閣。



 是時,河北副元帥僕固懷恩方頓軍汾州,掠並、汾諸縣以為己邑。乃以子儀兼關內河東副元帥、河中節度觀察使,出鎮河中。蕃戎既退,僕固懷恩部下離散。是月,懷恩子瑒主兵榆次,為帳下將張惟岳所殺,傳首京師。惟岳以瑒之眾歸於子儀,懷恩懼,棄其母而走靈州。明年九月,以子儀守太尉,充北道邠寧、涇原、河西已東通和蕃及朔方招撫觀察使,其關內河東副元帥、中書令如故。子
 儀以懷恩未誅,不宜讓使,堅辭太尉,曰:「太尉職雄任重,竊憂非據,輒敢上聞。伏奉詔書,未允誠懇。臣疇昔之分,早知止足,今茲累請,竊懼滿盈。義實由衷,事非矯飾,志之所至,敢不盡言。自兵亂已來,紀綱浸壞,時多躁競,俗少廉隅。德薄而位尊,功微而賞厚,實繁有眾,不可殫論。臣每見之,深以為念。昔範宣子讓,其下皆讓,欒騕為汰,不敢違也。臣誠薄劣,竊慕古人,務欲以身率先,大變浮俗,是用勤勤懇懇,願罷此官,庶禮讓興行,由臣而致
 也。臣位為上相,爵為真王,參啟沃之謀,受腹心之寄,恩榮已極,功業已成,尋合乞骸,保全餘齒。但以冠仇在近,家國未安,臣子之心,不敢寧處。茍西戎即敘,懷恩就擒,疇昔官爵,誓無所受,必當追蹤範蠡,繼跡留侯。臣之鄙懷,切在於此。」優詔不許。子儀見上,感泣懇讓,乃止。



 十月,僕固懷恩引吐蕃、回紇、黨項數十萬南下,京師大恐,子儀出鎮奉天。帝召子儀問御戎之計,子儀曰:「以臣所見,懷恩無能為也。」帝問其故,對曰:「懷恩雖稱驍勇,素失士心,
 今所以能為亂者,引思歸之人耳。懷恩本臣偏將,其下皆臣之部曲,臣恩信嘗及之,今臣為大將,必不忍以鋒刃相向,以此知其無能為也。」虜寇邠州,子儀在涇陽,子儀令長男朔方兵馬使曜率師援之,與邠寧節度使白孝德閉城拒守。懷恩前鋒至奉天,近城挑戰,諸將請擊之,子儀止之曰:「夫客兵深入,利在速戰,不可爭鋒。彼皆吾之部曲,緩之自當攜貳;若迫之,是速其戰,戰則勝負未可知。敢言戰者斬!」堅壁待之,果不戰而退。子儀自涇
 陽入朝,帝御安福門待之,命子儀樓上行朝見之禮,宴賜隆厚。



 十一月,以子儀為尚書令,上表懇辭曰:「臣以薄劣,素乏行能,逢時擾攘,猥蒙驅策,內參朝政,外總兵權。上不能翼戴三光,下不能糾逖群慝,功微賞厚,任重恩深,覆餗之憂,實盈寤寐。臣昨所以固辭太尉,乞保餘年,殊私曲臨,遂見矜許。竊謂陛下已知其願,深察其心,豈意未歷旬時,復延寵命。以臣褊淺,又寡智謀,安可謬職南宮,當茲大任?況太宗昔居籓邸,嘗踐此官,累聖相承,
 曠而不置。皇太子為雍王之日,陛下以其總兵薄伐,平定關東,飲至策勛,再有斯授。豈臣末職,敢亂大倫?德薄位尊,難逃天子之責;負乘致冠,復速神明之誅。伏乞天慈,俯停新命。」答詔不允。翌日,敕所司令子儀於尚書省視事。詔宰相百僚送上,遣射生五百騎執戟翼從,自朝堂至省,賜教坊樂。子儀不受,復上表曰:



 臣伏以尚書令,武德之際,太宗為之,昨瀝懇上陳,請罷斯職;而陛下未垂亮察,務欲褒崇,區區微誠,益用惶懼。何則?太宗立極
 之主,聖德在人,自後因廢此官,永代作則。陛下守文繼體,固當奉而行之,豈可猥私老臣,隳厥成式,上掩陛下之德,下貽萬方之非。臣雖至愚,安敢輕受?況久經兵亂,僭賞者多,一人之身,兼官數四,硃紫同色,清濁不分,「爛羊」之謠,復聞聖代。臣頃觀其弊,思革其源,以逆寇猶存,未敢輕議。今元兇沮敗,計日成擒,中外無虞,妖氛漸息。此陛下作法之際,審官之時,固合始於老臣,化及班列。豈可輕為此舉,以亂國章?國章亂於上,則庶政隳於下,
 海內之政皆亂,則國家又安得永代而無患哉!陛下茍能從臣之言,俯察誠請,彼貪榮冒進者,亦將各讓其所兼之官,自然天下文明,百工式敘,太平之業,可得而復也。臣誠蒙鄙,識昧古今,志之所切,實在於此。



 手詔答曰:「優崇之命,所以報功;總領之司,期於賦政。卿入居臺鉉,出統戎旃,爰自先朝,累匡多難,靖群氛於海表,凝庶績於天階。敏事而寡言,居敬而行簡,人難其易,爾易其難。所以命掌六聯,首茲百闢,顧循時議,僉謂允諧。而屢拜
 封章,懇懷讓揖,守淳素之道,語政理之源,無待禮成,曲從德讓。宜宣示於外,編之史冊。」遣內侍魚朝恩傳詔,賜美人盧氏等六人、從者八人,並車服、帷帳、床蓐、珍玩之具。



 時蕃虜屢寇京畿,倚蒲、陜為內地,常以重兵鎮之。永泰元年五月,以子儀都統河南道節度行營,出鎮河中。八月,僕固懷恩誘吐蕃、回紇、黨項、羌、渾、奴剌,山賊任敷、鄭庭、郝德、劉開元等三十餘萬南下,先發數萬人掠同州,期自華陰趨藍田,以扼南路,懷恩率重兵繼其後。
 回紇、吐蕃自涇、邠、鳳翔數道寇京畿,掠奉天、醴泉。京師震恐,天子下詔親征,命李忠臣屯東渭橋,李光進屯雲陽,馬璘、郝廷玉屯便橋,駱奉先、李日越屯盩啡,李抱玉屯鳳翔。周智光屯同州,杜冕屯坊州,天子以禁軍屯苑內。京城壯丁,並令團結。城二門塞其一。魚朝恩括士庶私馬,重兵捉城門,市民由竇穴而遁去,人情危迫。



 是時,急召子儀自河中至,屯於涇陽,而虜騎已合。子儀一軍萬餘人,而雜虜圍之數重。子儀使李國臣、高升拒其東,魏
 楚玉當其南,陳回光當其西,硃元琮當其北。子儀率甲騎二千出沒於左右前後,虜見而問:「此誰也?」報曰:「郭令公也。」回紇曰:「令公存乎?僕固懷恩言天可汗已棄四海,令公亦謝世,中國無主,故從其來。今令公存,天可汗存乎?」報之曰:「皇帝萬歲無疆。」回紇皆曰:「懷恩欺我。」子儀又使諭之曰:「公等頃年遠涉萬里,翦除兇逆,恢復二京。是時子儀與公等周旋艱難,何日忘之。今忽棄舊好,助一叛臣,何其愚也!且懷恩背主棄親,於公等何有?」回紇
 曰:「謂令公亡矣,不然,何以至此。令公誠存,安得而見之?」子儀將出,諸將諫曰:「戎狄之心,不可信也,請無往。」子儀曰:「虜有數十倍之眾,今力固不敵,且至誠感神,況虜輩乎!」諸將曰:「請選鐵騎五百衛從。」子儀曰:「適足以為害也。」乃傳呼曰:「令公來!」虜初疑,持滿注矢以待之。子儀以數十騎徐出,免胄而勞之曰:「安乎?久同忠義,何至於是?」回紇皆舍兵下馬齊拜曰:「果吾父也。」子儀召其首領,各飲之酒,與之羅錦,歡言如初。子儀說回紇曰:「吐蕃本吾舅
 甥之國,無負而至,是無親也。若倒戈乘之,如拾地芥耳。其羊馬滿野,長數百里,是謂天賜,不可失也。今能逐戎以利舉,與我繼好而凱旋,不亦善乎!」會懷恩暴死於鳴沙,群虜無所統攝,遂許諾,乃遣首領石野那等入朝。子儀遣朔方兵馬使白元光與回紇會軍。吐蕃知其謀,是夜奔退。回紇與元光追之,子儀大軍繼其後,大破吐蕃十餘萬於靈武臺西原,斬首五萬,生擒萬人,收其所掠士女四千人,獲牛羊駝馬,三百里內不絕。子儀自涇陽
 入朝,加實封二百戶,還鎮河中。



 大歷元年十二月,華州節度使周智光殺監軍張志斌謀叛,帝以同、華路阻,召子儀女婿工部侍郎趙縱受口詔往河中,令子儀起軍討之。縱請為蠟書,令家僮間道賜子儀。奉詔大閱軍戎,將發,同華將吏聞軍起,乃斬智光父子,傳首京師。二年二月,子儀入朝,宰相元載、王縉、僕射裴冕、京兆尹黎幹、內侍魚朝恩共出錢三十萬,置宴於子儀第,恩出羅錦二百匹,為子儀纏頭之費,極歡而罷。九月,吐蕃寇涇州,
 詔子儀以步騎三萬自河中移屯涇陽。十月,蕃軍退至靈州,邀擊敗之,斬馘二萬。十二月,盜發子儀父墓,捕盜未獲。人以魚朝恩素惡子儀,疑其使之。子儀心知其故,及自涇陽將入,議者慮其構變,公卿憂之。及子儀入見,帝言之,子儀號泣奏曰:「臣久主兵,不能禁暴,軍士殘人之墓,固亦多矣。此臣不忠不孝,上獲天譴,非人患也。」朝廷乃安。三年三月,還河中。八月,吐蕃寇靈武。九月,詔子儀率師五萬自河中移鎮奉天。是月,白元光大破吐蕃
 靈武。十月,子儀入朝,還鎮河中。時議以西蕃侵寇,京師不安,馬璘雖在邠州,力不能拒,乃以子儀兼邠寧慶節度,自河中移鎮邠州,徙馬璘為涇原節度使。八年十月,吐蕃寇涇州,子儀遣先鋒兵馬使渾瑊逆戰於宜祿,不利。會馬璘設伏於潘源,與瑊合擊,大破蕃軍,俘斬數萬計。回紇赤心賣馬一萬匹,有司以國計不充,請市千匹。子儀以回紇前後立功,不宜阻意,請自納一年奉物,充回紇馬價,雖詔旨不允,內外稱之。九年,入朝,代宗召
 對延英。語及西蕃棄斥,苦戰不暇,言發涕零。既退,復上封論備吐蕃利害,曰:



 朔方,國之北門,西御犬戎,北虞獫狁,五城相去三千餘里。開元、天寶中,戰士十萬,戰馬三萬,才敵一隅。自先皇帝龍飛靈武,戰士從陛下收復兩京,東西南北,曾無寧歲。中年以僕固之役,又經耗散,人亡三分之二,比於天寶中有十分之一。今吐蕃充斥,勢強十倍,兼河、隴之地,雜羌、渾之眾,每歲來窺近郊。以朔方減十倍之軍,當吐蕃加十倍之騎,欲求制勝,豈易為
 力!入近內地,稱四節度,每將盈萬,每賊兼乘數四。臣所統將士,不當賊四分之一,所有征馬,不當賊百分之二,誠合固守,不宜與戰。又得馬璘牒,賊擬涉渭而南。臣若堅壁,恐犯畿甸;若過畿內,則國人大恐,諸道易搖。外有吐蕃之強,中有易搖之眾,外畏內懼,將何以安?



 臣伏以陛下橫制勝之術,力非不足,但慮簡練未精,進退未一,時淹師老,地闊勢分。願陛下更詢讜議,慎擇名將,俾之統軍,於諸道各抽精卒,成四五萬,則制勝之道必矣,未
 可失時。臣又料河南、河北、山南、江淮小鎮數千,大鎮數萬,空耗月餼,曾不習戰。臣請抽赴關中,教之戰陣,則軍聲益振,攻守必全,亦長久之計也。臣猥蒙任遇,垂二十年,今齒發已衰,願避賢路,止足不誡,神明所鑒。



 詔曰:「卿憂深虛遠,殊沃朕心,始終倚賴,未可執辭也。」



 德宗即位,詔還朝,攝塚宰,充山陵使,賜號「尚父」,進位太尉、中書令,增實封通計二千戶,給一千五百人糧,二百匹馬草料,所領諸使副元帥並罷。諸子弟女婿拜官者十餘人。建
 中二年夏,子儀病甚,德宗令舒王誼傳詔省問。及門,郭氏子弟迎拜於外,王不答拜;子儀臥不能興,以手叩頭謝恩而已。六月十四日薨,時年八十五,德宗聞之震悼,廢朝五日,詔曰:



 天地以四時成物,元首以股肱作輔,公臺之任,鼎足相承,上以調三光,下以蒙五嶽。允釐庶績,鎮撫四夷,體元和之氣,根貞一之德,功至大而不伐,身處高而更安。尚父比呂望之名,為師增周公之位,盛業可久,歿而彌光。故太尉、兼中書令、上柱國、汾陽郡王、尚父
 子儀,天降人傑,生知王佐,訓師如子,料敵若神。昔天寶多難,羯胡作禍,咸秦失險,河洛為戎。公能扶翼肅宗,載造區夏。於國有患,勞其戡定;於邊有寇,藉其驅除。安社稷必在於絳侯,定羌戎無逾於充國。絳臺綏四散之眾,涇陽降十萬之虜。勛高今古,名璟夷狄,而勞乎征鎮,二紀於茲。



 頃以春秋既高,疆埸多事,罷彼旌鉞,寵在臺衡。以公柱石四朝,籓翰萬里,忠貞懸於日月,寵遇冠於人臣,尊其元老,加以崇號,期壽考之永,養勛賢之德。膏
 肓生疾,藥石靡攻,人之云亡,梁木斯壞。雖賻禮加等,輟朝增日,悼之流涕,曷可弭忘!更議追崇,名位斯極,而尊為尚父,官協太師,雖爵秩則同,而體望尤重。斂以袞冕,旌我元臣。聖祖園陵,所宜陪葬,式墓表文終之德,象山追去病之勛。千載如存,九原可作,冊命之禮,有司備焉。可贈太師,陪葬建陵。仍令所司備禮冊命,賻絹三千匹、布三千端、米麥三千石。



 舊令一品墳高丈八,而詔特加十尺。群臣以次赴宅吊哭。兇喪所須,並令官給。及葬,上御
 安福門臨哭送之,百僚陪位隕泣,賜謚曰忠武,配饗代宗廟庭。



 子曜、旰、晞、昢、晤、曖、曙、映等八人,婿七人,皆朝廷重官。諸孫數十人,每群孫問安,不盡辨,頷之而已。參佐官吏六十餘人,後位至將相,升朝秩貴位,勒其姓名於石,今在河中府。人士榮之。



 史臣裴垍曰:汾陽事上誠藎,臨下寬厚,每降城下邑,所至之處,必得士心。前後遭罹幸臣程元振、魚朝恩譖毀百端,時方握強兵,或方臨戎敵,詔命征之,未嘗不即日應召,故讒謗不能行。代宗幸
 陜時,令以數十騎覘賊,及在涇陽,又陷於胡虜重圍之中,皆以身許國,未嘗以危亡易慮,亦遇天幸,竟免患難。田承嗣方跋扈魏州,傲狠無禮,子儀嘗遣使至,承嗣西望拜之,指其膝謂使者曰:「茲膝不屈於人若干歲矣,今為公拜。」李靈曜據汴州,公私財賦一皆遏絕,獨子儀封幣經其境,莫敢留之,必持兵衛送。其為豺虎所服如此。麾下老將若李懷光輩數十人,皆王侯重貴,子儀頤指進退,如僕隸焉。幕府之盛,近代無比。始與李光弼齊名,
 雖威略不逮,而寬厚得人過之。歲入官俸二十四萬貫,私利不在焉。其宅在親仁里,居其里四分之一,中通永巷,家人三千,相出入者不知其居。前後賜良田美器,名園甲館,聲色珍玩,堆積羨溢,不可勝紀。代宗不名,呼為大臣。天下以其身為安危者殆二十年。校中書令考二十有四。權傾天下而朝不忌,功蓋一代而主不疑,侈窮人欲而君子不之罪。富貴壽考,繁衍安泰,哀榮終始,人道之盛,此無缺焉。唯以讒怒,誣奏判官戶部郎中張譚
 杖殺之,物議為薄。



 曜,子儀長子。性孝友廉謹。子儀薨,出征於外,留曜治家,少長千人,皆得其所。諸弟爭飾池館,盛其車服,曜以儉樸自處。累遷至太子賓客。建中初,子儀罷兵柄,乃遍加諸子官,以曜為太子少保。子儀曜遵遺命,四朝所賜名馬珍玩,悉皆上獻,德宗復賜之,曜乃散諸昆弟。子儀薨後,楊炎、盧杞相次秉政,奸諂用事,尤忌勛族。子儀之婿太僕卿趙縱、少府少監李洞清、光祿卿王宰,皆以有人告訐細過,相次貶黜。曜家大恐,賴宰
 相張鎰力為庇護。奸人幸其危懼,多論奪田宅奴婢,曜不敢訴。德宗微知之,詔曰:「尚父子儀,有大勛力,保乂皇家,嘗誓以山河,琢之金石,十世之宥,其可忘也!其家前時與人為市,以子儀身歿,名被誣構,欲論奪之,有司無得為理。」詔下方已。曜居喪得禮,若儒家子,服未闋寢疾,或勸其茹蔥薤,曜竟不屬口。建中四年三月卒,贈太子太傅。



 晞,子儀第三子。少善騎射,常從父征伐。初以戰功授左贊善大夫,從廣平王收復兩京,晞力戰於香積寺、
 陜西,皆出奇兵克捷,以功加銀青光祿大夫、鴻臚卿。後河中軍亂,殺節度使李國貞、荔非元禮於絳,詔以子儀為河東關內副元帥,鎮絳州,時四方擾叛,多逐戎帥,子儀至絳,誅其元惡,其黨頗不自安,欲謀翻變。晞知其謀,選親兵四千,伏甲以防之,常持弓警夜,不寐者凡七十日,叛將竟不敢發,以功拜殿中監。廣德二年,僕固懷恩誘吐蕃、回紇入寇。加晞御史中丞,領朔方軍以援邠州,與馬璘合勢,大破蕃軍。其年冬,懷恩誘虜再寇邠州,陣
 於涇北,子儀令晞率步卒五千、騎軍五百,出西南掩擊之。晞以兵寡不敵,持而不戰,及至晡晚,乘其半濟而擊之,大破獯虜,斬首五千級。是時連戰皆捷,詔加御史大夫,子儀固讓不受。永泰二年,檢校左散騎常侍。大歷七年,加開府儀同三司。十二年,丁母憂;服除,加檢校工部尚書,判秘書省事。建中二年,丁父喪,持服京城。硃泚構逆,遣人就第問訊,欲令掌兵,晞佯瘖,噤口不言,泚以兵脅之,晞終不語,賊知其不可用,乃止。晞潛奔奉天,僅而
 獲免。



 初,晞兄曜襲父代國公,實封二千戶,及曜卒,詔曰:「故尚父、太尉中書令、汾陽王,功格上玄,道光下土,積其善慶,垂裕無窮。雖嫡長雲殂,支宗斯盛,汾陽舊邑,盍有丕承。其男前左散騎常侍、駙馬都尉、食實封五百戶曖,夙稟義方,居忠履孝,儷崇銀榜,攄美金章,繼撫先封,允宜聽復。曖兄檢校工部尚書、守太子賓客、趙國公晞,並弟右金吾將軍、祁國公、食實封二百五十戶曙,太子左諭德映等,並休有令名,保其先業,宜允推恩之典,以明延
 嗣之誠。其實封二千戶,宜準式減半,餘可分襲。曖可襲代國公,仍通前襲三百戶;晞可二百五十戶;曙可五十戶,通前三百七十戶;映可二百三十五戶。」尋又詔尚父子儀男晞、曖、映、曙四人所襲實封,各減五十戶,以賜郭曜男鉾、郭晤男鐇,各襲一百戶。



 晞至行在,復檢校工部尚書、太子詹事;從駕還京,改太子賓客。晞子鋼為朔方節度使杜希全賓佐,希全以鋼攝豐州刺史。晞以鋼幼弱,恐不任邊職,貞元七年,晞上章請罷鋼官。德宗遣中
 使召之,鋼疑以他事見攝,乃單騎走入吐蕃。蕃將見鋼獨叛,不納,置之筏上,流入黃河令歸,杜希全得之,送赴京師,賜鋼自盡,晞亦坐子免官。明年,復授太子賓客。貞元十年卒,贈兵部尚書。晞次子鈞。鈞子承嘏別有傳。



 曖,子儀第六子。年十餘歲,尚代宗第四女升平公主,時升平年亦與暖相類。大歷中,恩寵冠於戚里,歲時錫賚珍玩,不可勝紀。大歷十三年,有詔毀除白渠水支流碾磑,以妨民溉田。升平有脂粉磑兩輪,郭子儀私磑兩輪,所
 司未敢毀徹。公主見代宗訴之,帝謂公主曰:「吾行此詔,蓋為蒼生,爾豈不識我意耶?可為眾率先。」公主即日命毀。由是勢門碾磑八十餘所,皆毀之。曖檢校左散騎常侍。建中末,公主坐事,留之禁中,曖亦不令出入。既而硃泚之亂,不知車駕幸奉天,為賊所逼,欲授偽官,曖辭以居喪被疾。既而與兄晞、弟曙及升平公主皆奔奉天,德宗喜,並釋前咎,待之如初,復銀青光祿大夫、檢校左散騎常侍。從駕至山南,改太常卿同正員。



 貞元中,帝為皇孫
 廣陵郡王納曖女為妃。曖,貞元十六年七月卒,贈尚書左僕射。升平公主,元和五年十月薨,贈虢國大長公主,謚曰懿。廣陵王即位,為憲宗皇帝,妃生穆宗皇帝。元和十五年,穆宗即位,尊郭妃為皇太后,詔曰:「追遠飾終,先王令典。況積仁累義,事已顯於身前;祥會慶傳,福遂流於天下。式光盛德,爰舉徽章,尊尊親親,於是乎在。皇太后父贈尚書左僕射曖,克荷崇構,有勞王家,孝友本於生知,英華發於事任,實修一德,歷仕三朝。建中末年,屬
 有大難,畢力扈駕,忘軀即戎,忠貞之節,國史明備。才高望洽,是膺沁水之祥;德厚流光,乃啟塗山之祚。肆予小子,獲纘大業,未展定申之命,敢緣褒紀之恩,俾繼維師,用不縟禮。可贈太傅。」曖子釗、鏦、銛。



 曙,代宗朝累歷司農卿,居父憂。建中三年冬,舒王誼為淮西、山南諸大元帥,以曙檢校左庶子,為元帥府都押牙。京城亂,從幸山南,轉太府卿。隨駕還京,拜左金吾衛大將軍。貞元末卒。



 釗,偉姿儀,身長七尺,方口豐下,沉默寡言。母升平長公
 主。代宗朝,釗為外孫,恩寵逾等,起家為太常寺奉禮郎。德宗朝,累官至太子右庶子。元和初,為左金吾衛大將軍,充左街使。九年十一月,檢校工部尚書,兼邠州刺史,充邠寧節度使。數歲,檢校戶部尚書,入為司農卿。釗,大勛之後,姻聯戚里,而謙和接物,恭慎自持,居家臨民,無驕怠之色,無奢侈之失,士君子重之。十五年正月,憲宗寢疾彌旬,諸中貴人秉權者欲議廢立,紛紛未定。穆宗在東宮,心甚憂之,遣人問計於釗,釗曰:「殿下身為皇太
 子,但旦夕視膳,謹守以俟,又何慮乎!」迄今稱釗得元舅之體。



 穆宗即位,冊皇太后南內,推崇外氏,以釗兼司農卿。未幾,檢校戶部尚書,充河陽三城懷節度使。歲中,換河中尹、河中晉絳慈隰節度使。釗歷踐籓鎮,以汾陽胄胤,材能選用,不獨憑椒房之勢,所蒞簡約不撓,其俗自理。敬宗即位,尊郭太后為太皇太后,徵釗為兵部尚書,兼檢校尚書左僕射。明年,出為梓州刺史、劍南東川節度使。文宗即位,加司空。大和三年冬,南蠻陷巂州,遂寇
 西川,杜元穎失於控御,蠻軍陷成都府外城。朝廷未暇除帥,乃以釗兼領西川節度。蠻軍已寇樟州,諸道援軍未至,川軍寡弱,不可令戰。釗致書於蠻首領泬巔,責以侵寇之意,泬巔曰:「杜元穎不守疆埸,屢侵吾圉,以是修報也。」與釗修好而退。朝廷嘉之,授成都尹、劍南西川節度使。與南詔立約,疆陲不擾。以疾求代。四年,入為太常卿、檢校司徒。十二月,在道卒,詔贈司徒。子仲文、仲辭。



 鏦,母升平長公主,大歷、貞元之間,恩禮冠諸主。順宗在東宮,
 以女德陽郡主尚鏦,時鏦與公主年未及冠,郡主尤為德宗之所鐘愛,故鏦之貴寵,焜燿一時。順宗即位,改封德陽為漢陽公主。鏦累官至衛尉卿、駙馬都尉,改殿中監。穆宗即位,鏦為叔舅,改右金吾衛大將軍、兼御史大夫,充左街使。城南有汾陽王別墅,林泉之致,莫之與比,穆宗常游幸之,置酒極歡而罷,賜金從甚厚。俄加檢校工部尚書,兼太子詹事,充閑廄宮苑使。從容貴位三十餘年,而椒房之寵,國舅之恩,近代已來,無有其比。而鏦恭
 遜虔恪,不以富貴驕人,士無賢不肖,接之以禮,由是中外稱之。長慶二年十月卒,贈尚書左僕射,仍以其弟銛代鏦為太子詹事,充閑廄宮苑使。



 仲文,大和末為殿中少監。開成初,詔仲文襲父太原郡公,制上,給事中封敕奏曰:「伏準制書,贈司徒郭釗嫡男仲文襲封太原郡公者,臣近訪知郭釗妻沈氏,公主之女,代宗皇帝外孫,有男仲辭,已選尚主。仲文不合假冒,自稱嫡子。若仲文承嫡,即沈氏須黜居別室,仲辭不合配尚貴主。伏以郭仲
 文,尚父子儀之孫,太皇太后之侄,戚里勛門,無與儔比,婚姻嫡庶,朝野具知,奪宗之配,實玷風教。且仲文、仲辭既非同出,襲封尚主,不可並行。伏請付臺勘當。」詔曰:「以萬年縣尉仲辭襲封。」仲文落下,以太皇太后侄,不之罪。尋以仲辭為銀青光祿大夫、檢校中少監、駙馬都尉,襲封太原郡公,尚饒陽公主。又仲辭兄詹事府丞仲恭,為銀青光祿大夫,尚金堂公主。



 幼明,尚父子儀之母弟也。性謹願無過,不工武藝,喜賓客飲宴,居家御眾,皆得
 其歡心。以子儀勛業,累歷大卿監,大歷八年卒,贈太子太傅。



 子昕,肅宗末為四鎮留後。自關、隴陷蕃,為虜所隔,其四鎮、北庭使額,李嗣業、荔非元禮皆遙領之。昕阻隔十五年,建中二年,與伊西北庭節度使李元忠俱遣使於朝,德宗嘉之。詔曰:「四鎮、二庭,統任西夏五十七蕃十姓部落,國朝以來,相次率職。自關、隴失守,東西阻絕,忠義之徒,泣血相守,慎固封略,奉尊朝法,皆候伯守將交修共理之所致也。伊西北庭節度使李元忠,可北庭大都
 護;四鎮節度留後郭昕,可安西大都護、四鎮節度使。其將吏已下敘官,可超七資。」



 李元忠,本姓曹,名令忠,以功賜姓名。時昕使自回紇歷諸蕃部,方達於朝。又有袁光庭者,為伊州刺史,隴右諸郡皆陷,光庭堅守伊州,吐蕃攻之累年,兵盡食竭,光庭先刃其妻子,自焚而死。因昕使知之,贈工部尚書。



 史臣曰:天寶之季,盜起幽陵,萬乘播遷,兩都覆沒。天祚土德,實生汾陽。自河朔班師,關西殄寇,身捍豺虎,手
 披荊榛。七八年間,其勤至矣,再造王室,勛高一代。及國威復振,群小肆讒,位重懇辭,失寵無怨。不幸危而邀君父,不挾憾以報仇讎,晏然效忠,有死無二,誠大雅君子,社稷純臣。自秦、漢已還,勛力之盛,無與倫比。而晞、曖於縗粗之中,拔身虎口,赴難奉天,可謂忠孝之門有嗣矣。



 贊曰:猗歟汾陽,功扶昊蒼。秉仁蹈義,鐵心石腸。四朝靜亂,五福其昌。為臣之節,敢告忠良。



\end{pinyinscope}