\article{卷五十七 列傳第七 裴寂 劉文靜(弟文起 文靜子樹義 樹藝 李孟嘗 劉世龍 趙文恪 張平高 李思行 李高遷 許世緒 劉師立 錢九隴 樊興 公孫武達 龐卿惲 張長遜 李安遠)}

\begin{pinyinscope}

 ○裴
 寂劉文靜弟文起文靜子樹義樹藝李孟嘗劉世龍趙文恪張平高李思行李高遷許世緒劉師立錢九隴樊興公孫武達龐卿惲張長遜李安遠



 裴寂,字玄真,蒲州桑泉人也。祖融,司本大夫。父瑜,絳州刺史。寂少孤,為諸兄之所鞠養。年十四,補州主簿。及長,疏眉目,偉姿容。隋開皇中,為左親衛。家貧無以自業,每徒步詣京師,經華嶽廟,祭而祝曰:「窮困至此,敢修誠謁,神之有靈,鑒其運命。若富貴可期,當降吉夢。」再拜而去。夜夢白頭翁謂寂曰:「卿年三十已後方可得志,終當位極人臣耳。」後為齊州司戶。大業中,歷侍御史、駕部承務郎、晉陽宮副監。高祖留守太原,與寂有舊,時加親禮,每
 延之宴語,間以博奕,至於通宵連日,情忘厭倦。時太宗將舉義師而不敢發言,見寂為高祖所厚,乃出私錢數百萬,陰結龍山令高斌廉與寂博戲,漸以輸之。寂得錢既多,大喜,每日從太宗游。見其歡甚,遂以情告之,寂即許諾。寂又以晉陽宮人私侍高祖,高祖從寂飲,酒酣,寂白狀曰:「二郎密纘兵馬,欲舉義旗,正為寂以宮人奉公,恐事發及誅,急為此耳。今天下大亂,城門之外,皆是盜賊。若守小節,旦夕死亡;若舉義兵,必得天位。眾情已協,
 公意如何?」高祖曰:「我兒誠有此計,既已定矣,可從之。」及義兵起,寂進宮女五百人,並上米九萬斛、雜彩五萬段、甲四十萬領,以供軍用。大將軍府建,以寂為長史,賜爵聞喜縣公。從至河東,屈突通拒守,攻之不下,三輔豪傑歸義者日有千數。高祖將先定京師,議者恐通為後患,猶豫未決。寂進說曰:「今通據蒲關,若不先平,前有京城之守,後有屈突之援,此乃腹背受敵,敗之道也。未若攻蒲州,下之而後入關。京師絕援,可不攻而定矣。」太宗曰:「
 不然。兵法尚權,權在於速。宜乘機早渡,以駭其心。我若遲留,彼則生計。且關中群盜,所在屯結,未有定主,易以招懷,賊附兵強,何城不克?屈突通自守賊耳,不足為虞。若失入關之機,則事未可知矣。」高祖兩從之,留兵圍河東,而引軍入關。及京師平,賜良田千頃、甲第一區、物四萬段,轉大丞相府長史,進封魏國公,食邑三千戶。



 及隋恭帝遜位,高祖固讓不受,寂勸進,又不答。寂請見曰:「桀、紂之亡,亦各有子,未聞湯、武臣輔之,可為龜鏡,無所疑
 也。寂之茅土、大位,皆受之於唐,陛下不為唐帝,臣當去官耳。」又陳符命十餘事,高祖乃從之。寂出,命太常具禮儀,擇吉日。高祖既受禪,謂寂曰:「使我至此,公之力也。」拜尚書右僕射,賜以服玩不可勝紀,仍詔尚食奉御,每日賜寂御膳。高祖視朝,必引與同坐,入閣則延之臥內,言無不從,呼為裴監而不名。當朝貴戚,親禮莫與為比。武德二年,劉武周將黃子英、宋金剛頻寇太原,行軍總管姜寶誼、李仲文相次陷沒,高祖患之。寂自請行,因為晉
 州道行軍總管,得以便宜從事。師次介休,而金剛據城以抗寂。寂保於度索原,營中乏水,賊斷其澗路,由是危迫。欲移營就水,賊因犯之,師遂大潰,死散略盡。寂一日一夜馳至晉州。以東城鎮俱沒,金剛進逼絳州,寂抗表陳謝,高祖慰諭之,復令鎮撫河東之地。寂性怯,無捍禦之才,唯發使絡繹,催督虞、秦二州居人,勒入城堡,焚其積聚。百姓惶駭,復思為亂。夏縣人呂崇茂遂殺縣令舉兵反,引金剛為援,寂擊之,復為崇茂所敗。被徵入朝,高
 祖數之曰:「義舉之始,公有翼佐之勛,官爵亦極矣。前拒武周,兵勢足以破敵,致此喪敗,不獨愧於朕乎?」以之屬吏,尋釋之,顧待彌重。



 高祖有所巡幸,必令居守。麟州刺史韋雲起告寂謀反,訊之無端。高祖謂寂曰:「朕之有天下者,本公所推,今豈有貳心?皁白須分,所以推究耳。」因令貴妃三人齎珍饌、寶器就寂第,宴樂極歡,經宿而去。又嘗從容謂寂曰:「我李氏昔在隴西,富有龜玉,降及祖禰,姻婭帝室。及舉義兵,四海雲集,才涉數日,升為天子。
 至如前代皇王,多起微賤,劬勞行陣,下不聊生。公復世胄名家,歷職清顯,豈若蕭何、曹參起自刀筆吏也!唯我與公,千載之後,無愧前修矣。」其年,改鑄錢,特賜寂令自鑄造。又為趙王元景聘寂女為妃。六年,遷尚書左僕射,賜宴於含章殿,高祖極歡,寂頓首而言曰:「臣初發太原,以有慈旨,清平之後,許以退耕。今四海乂安,伏願賜臣骸骨。」高祖泣下沾襟曰:「今猶未也,要相偕老耳。公為臺司,我為太上,逍遙一代,豈不快哉!」俄冊司空,賜實封五
 百戶,遣尚書員外郎一人每日更直寂第,其見崇貴如此。



 貞觀元年,加實封並前一千五百戶。二年,太宗祠南郊,命寂與長孫無忌同升金輅,寂辭讓,太宗曰:「以公有佐命之勛,無忌亦宣力於朕,同載參乘,非公而誰?」遂同乘而歸。



 三年,有沙門法雅,初以恩幸出入兩宮,至是禁絕之,法雅怨望,出妖言,伏法。兵部尚書杜如晦鞫其獄,法雅乃稱寂知其言,寂對曰:「法雅惟雲時候方行疾疫,初不聞妖言。」法雅證之,坐是免官,削食邑之半,放歸本
 邑。寂請住京師,太宗數之曰:「計公勛庸,不至於此,徒以恩澤,特居第一。武德之時,政刑紕繆,官方弛紊,職公之由。但以舊情,不能極法,歸掃墳墓,何得復辭?」寂遂歸蒲州。未幾,有狂人自稱信行,寓居汾陰,言多妖妄,常謂寂家僮曰:「裴公有天分。」於時信行已死,寂監奴恭命以其言白寂,寂惶懼不敢聞奏,陰呼恭命殺所言者。恭命縱令亡匿,寂不知之。寂遣恭命收納封邑,得錢百餘萬,因用而盡。寂怒,將遣人捕之,恭命懼而上變。太宗大怒,謂
 侍臣曰:「寂有死罪者四:位為三公而與妖人法雅親密,罪一也;事發之後,乃負氣憤怒,稱國家有天下,是我所謀,罪二也;妖人言其有天分,匿而不奏,罪三也;陰行殺戮以滅口,罪四也。我殺之非無辭矣。議者多言流配,朕其從眾乎。」於是徙交州,竟流靜州。俄逢山羌為亂,或言反獠劫寂為主,太宗聞之曰:「我國家於寂有性命之恩,必不然矣。」未幾,果稱寂率家僮破賊。太宗思寂佐命之功,徵入朝,會卒,時年六十。贈相州刺史、工部尚書、河東
 郡公。



 子律師嗣,尚太宗妹臨海長公主,官至汴州刺史。律師子承先,則天時為殿中監,為酷吏所殺。



 劉文靜,字肇仁,自云彭城人,代居京兆之武功。祖懿用,石州刺史。父韶,隋時戰沒,贈上儀同三司。少以其父身死王事,襲父儀同三司。偉姿儀,有器幹,倜儻多權略。隋末,為晉陽令,遇裴寂為晉陽宮監,因而結友。夜與同宿,寂見城上烽火,仰天嘆曰:「卑賤之極,家道屢空,又屬亂離,當何取濟?」文靜笑曰:「世途若此,時事可知。吾二人相
 得,何患於卑賤?」



 及高祖鎮太原,文靜察高祖有四方之志,深自結托。又竊觀太宗,謂寂曰:「非常人也。大度類於漢高,神武同於魏祖,其年雖少,乃天縱矣。」寂初未然之。後文靜坐與李密連婚,煬帝令系於郡獄。太宗以文靜可與謀議,入禁所視之。文靜大喜曰:「天下大亂,非有湯、武、高、光之才,不能定也。」太宗曰:「卿安知無?但恐常人不能別耳。今入禁所相看,非兒女之情相憂而已。時事如此,故來與君圖舉大計,請善籌其事。」文靜曰:「今李密長
 圍洛邑,主上流播淮南,大賊連州郡、小盜阻澤山者,萬數矣,但須真主驅駕取之。誠能應天順人,舉旗大呼,則四海不足定也。今太原百姓避盜賊者,皆入此城。文靜為令數年,知其豪傑,一朝嘯集,可得十萬人,尊公所領之兵,復且數萬,君言出口,誰敢不從?乘虛入關,號令天下,不盈半歲,帝業可成。」太宗笑曰:「君言正合人意。」於是部署賓客,潛圖起義。候機當發,恐高祖不從,沉吟者久之。文靜見高祖厚於裴寂,欲因寂開說,於是引寂交於
 太宗,得通謀議。



 及高君雅為突厥所敗,高祖被拘,太宗又遣文靜共寂進說曰:「《易》稱『知幾其神乎』,今大亂已作,公處嫌疑之地,當不賞之功,何以圖全?其裨將敗衄,以罪見歸。事誠迫矣,當須為計。晉陽之地,士馬精強,宮監之中,府庫盈積,以茲舉事,可立大功。關中天府,代王沖幼,權豪並起,未有適從。願公興兵西入,以圖大事。何乃受單使之囚乎?」高祖然之。時太宗潛結死士,與文靜等協議,克日舉兵,會高祖得釋而止。乃命文靜詐為煬帝
 敕,發太原、西河、雁門、馬邑,人年二十已上、五十已下悉為兵,期以歲暮集涿郡,將伐遼東。由是人情大擾,思亂者益眾。文靜因謂裴寂曰:「公豈不聞『先發制人,後發制於人』乎?唐公名應圖讖,聞於天下,何乃推延,自貽禍釁?宜早勸唐公,以時舉義。」又脅寂曰:「且公為宮監,而以宮人侍客,公死可爾,何誤唐公也?」寂甚懼,乃屢促高祖起兵。會馬邑人劉武周殺太守王仁恭,自稱天子,引突厥之眾,將侵太原。太宗遣文靜及長孫順德等分部募兵,
 以討武周為辭;又令文靜與裴寂偽作符敕,出宮監庫物以供留守資用,因募兵集眾。及義兵將起,副留守王威、高君雅獨懷猜貳。後數日,將大會於晉祠,威及君雅潛謀害高祖,晉陽鄉長劉世龍以白太宗。太宗既知迫急,欲先事誅之,遣文靜與鷹揚府司馬劉政會投急變之書,詣留守告威等二人謀反。是日,高祖與威、君雅同坐視事,文靜引政會至庭中,云有密狀,知人欲反。高祖指威等取狀看之,政會不肯與,曰:「所告是副留守事,唯
 唐公得看之耳。」高祖陽驚曰:「豈有是乎!」覽狀訖,謂威等曰:「此人告公事,如何?」君雅大詬曰:「此是反人,欲殺我也!」文靜叱左右執之,囚於別室。既拘威等,竟得舉兵。



 高祖開大將軍府,以文靜為軍司馬。文靜勸改旗幟以彰義舉,又請連突厥以益兵威,高祖並從之。因遣文靜使於始畢可汗,始畢曰:「唐公起事,今欲何為?」文靜曰:「皇帝廢塚嫡,傳位後主,致斯禍亂。唐公國之懿戚,不忍坐觀成敗,故起義軍,欲黜不當立者。願與可汗兵馬同入京師,
 人眾土地入唐公,財帛金寶入突厥。」始畢大喜,即遣將康鞘利領騎二千,隨文靜而至,又獻馬千匹。高祖大悅,謂文靜曰:「非公善辭,何以致此?」尋率兵禦隋將屈突通於潼關,通遣武牙郎將桑顯和率勁兵來擊,文靜苦戰者半日,死者數千人。文靜度顯和軍稍怠,潛遣奇兵掩其後,顯和大敗,悉虜其眾。通尚擁兵數萬,將遁歸東都,文靜遣諸將追而執之,略定新安以西之地。轉大丞相府司馬,進授光祿大夫,封魯國公。



 高祖踐祚,拜納言。時
 高祖每引重臣共食,文靜奏曰:「陛下君臨億兆,率土莫非臣,而當朝捴抑,言尚稱名;又宸極位尊,帝座嚴重,乃使太陽俯同萬物,臣下震恐,無以措身。」帝不納。時制度草創,命文靜與當朝通識之士更刊《隋開皇律令》而損益之,以為通法。高祖謂曰:「本設法令,使人共解,而往代相承,多為隱語,執法之官,緣此舞弄。宜更刊定,務使易知。」會薛舉寇涇州,命太宗討之,以文靜為元帥府長史。遇太宗不豫,委於文靜及司馬殷開山,誡之曰:「舉糧少
 兵疲,懸軍深入,意在決戰,不利持久,即欲挑戰,慎無與決。待吾差,當為君等取之。」文靜用開山計,出軍爭利,王師敗績。文靜奔還京師,坐除名。俄又從太宗討舉,平之,以功復其爵邑,拜民部尚書,領陜東道行臺左僕射。武德二年,從太宗鎮長春宮。



 文靜自以才能幹用在裴寂之右,又屢有軍功,而位居其下,意甚不平。每廷議多相違戾,寂有所是,文靜必非之,由是與寂有隙。文靜嘗與其弟通直散騎常侍文起酣宴,出言怨望,拔刀擊柱曰:「
 必當斬裴寂耳!」家中妖怪數見,文起憂之,遂召巫者於星下被發銜刀,為厭勝之法。時文靜有愛妾失寵,以狀告其兄,妾兄上變。高祖以之屬吏,遣裴寂、蕭瑀問狀。文靜曰:「起義之初,忝為司馬,計與長史位望略同;今寂為僕射,據甲第,臣官賞不異眾人,東西征討,家口無托,實有觖望之心。因醉或有怨言,不能自保。」高祖謂群臣曰:「文靜此言,反明白矣。」李綱、蕭瑀皆明其非反。太宗以文靜義旗初起,先定非常之策,始告寂知;及平京城,任遇
 懸隔,止以文靜為觖望,非敢謀反,極佑助之。而高祖素疏忌之,裴寂又言曰:「文靜才略,實冠時人,性復粗險,忿不思難,醜言悖逆,其狀已彰。當今天下未定,外有勍敵,今若赦之,必貽後患。」高祖竟聽其言,遂殺文靜、文起,仍籍沒其家。文靜臨刑,撫膺嘆曰:「高鳥逝,良弓藏,故不虛也。」時年五十二。



 貞觀三年,追復官爵,以子樹義襲封魯國公,許尚公主。後與其兄樹藝怨其父被戮,又謀反,伏誅。



 文靜初為納言時,有詔以太原元謀立功,尚書令、秦
 王某,尚書左僕射裴寂及文靜,特恕二死。左驍衛大將軍長孫順德、右驍衛大將軍劉弘基、右屯衛大將軍竇琮、左翊衛大將軍柴紹、內史侍郎唐儉、吏部侍郎殷開山、鴻臚卿劉世龍、衛尉少卿劉政會、都水監趙文恪、庫部郎中武士鷿、驃騎將軍張平高、李思行、李高遷,左屯衛府長史許世緒等十四人,約免一死。武德九年十月,太宗始定功臣實封差第,文靜已死,於是裴寂加食九百戶,通前為一千五百戶;長孫無忌、王君廓、尉遲敬德、
 房玄齡、杜如晦等五人,食邑一千三百戶;長孫順德、柴紹、羅藝、趙郡王孝恭等四人,食邑一千二百戶;侯君集、張公謹、劉師立等三人食邑一千戶;李勣、劉弘基二人食邑九百戶;高士廉、宇文士及、秦叔寶、程知節四人食七百戶;安興貴、安修仁、唐儉、竇軌、屈突通、蕭瑀、封德彞、劉義節八人,各食六百戶;錢九隴、樊興、公孫武達、李孟嘗、段志玄、龐卿惲、張亮、李藥師、杜淹、元仲文十人,各食四百戶;張長遜、張平高、李安遠、李子和、秦行師、馬三寶
 六人,各食三百戶。其王君廓事在《廬江王瑗傳》,安興貴、安修仁事在《李軌傳》,李子和事在《梁師都傳》,馬三寶事在《柴紹傳》。



 李孟嘗,趙州平棘人,官至右威衛大將軍、漢東郡公。元仲文,洛州人,至右監門將軍、河南縣公。秦行師,並州太原人,至左監門將軍、清水郡公。並事微不錄。自餘無傳者,盡附於此。



 劉世龍者,並州晉陽人。大業末,為晉陽鄉長。高祖鎮太
 原,裴寂數薦之,由是甚見接待,亦出入王威、高君雅家,然獨歸心於高祖。義兵將起,威與君雅內懷疑惑,世龍輒探得其情,以白高祖。及誅威等,授銀青光祿大夫。從平京城,累轉鴻臚卿,仍改名義節。



 時草創之始,傾竭府藏以賜勛人,而國用不足,義節進計曰:「今義師數萬,並在京師,樵薪貴而布帛賤。若採街衢及苑中樹為樵以易布帛,歲收數十萬匹立可致也。又藏內繒絹,匹匹軸之,使申截取剩物,以供雜費,動盈十餘萬段矣。」高祖並
 從之,大收其利。再遷太府卿,封葛國公。貞觀初,轉少府監,以罪配流嶺南,尋授欽州別駕,卒。



 義節從子思禮,萬歲通天二年,為箕州刺史。思禮少嘗學相術於許州張憬藏,相己必歷刺史,位至太師。及授箕州,益自喜,以為太師之職,位極人臣,非佐命無以致之。與洛州錄事參軍綦連耀結構謀反,謂耀曰:「公體有龍氣。」耀亦謂思禮曰:「公是金刀,合為我輔。」因相解釋圖讖,即定君臣之契。又令思禮自衒相術,每所見人,皆謂之「合得三品」,使務
 進之士,聞之滿望,然始謂云:「綦連耀有天分,公因之以得富貴。」事發系獄,乃多證引朝士,冀以自免。所誅陷者三十餘家,耀、思禮並伏誅。鳳閣侍郎李元素、夏官侍郎孫元亨、知天官侍郎事石抱忠、鳳閣舍人王劇、劇兄前涇州刺史勔、太子司議郎路敬淳等,坐與耀及思禮交結,皆死。初,則天命河內王武懿宗按思禮之獄。懿宗寬思禮於外,令廣引逆徒。而思禮以為得計,從容自若,嘗與相忤者,必引令枉誅。臨刑猶在外,尚不之覺,及眾人
 就戮,乃收誅之。



 趙文恪者,並州太原人也。隋末,為鷹揚府司馬。義師之舉,授右三統軍。武德二年,拜都水監,封新興郡公。時大亂之後,中州少馬,遇突厥蕃市牛馬以資國用。俄而劉武周將宋金剛來寇太原,屬城皆沒。真鄉公李仲文退守浩州,城孤兵弱,元吉遣文恪率步騎千餘助為聲援。及太原為賊所陷,文恪遂棄城遁去,坐是賜死獄中。



 張平高,綏州膚施人也。隋末,為鷹揚府校尉,戍太原,為
 高祖所識,因參謀議。義旗建,以為軍頭。從平京城,累授左領軍將軍,封蕭國公。貞觀初,出為丹州刺史,坐事免,令以右光祿大夫還第,卒。後改封羅國公。永徽中,追贈潭州都督。



 李思行,趙州人也。嘗避仇太原。高祖將舉義兵,令赴京城觀覘動靜,及還,具論機變,深稱旨,授左三統軍。從破宋老生,平京城,累授嘉州刺史,封樂安郡公。永徽初卒,贈洪州都督,謚曰襄。



 李高遷,岐州岐山人也。隋末,客游太原,高祖常引之左右。及擒高君雅、王威等,高遷有功焉,授右三統軍。從平霍邑,圍京城,力戰功最,累遷左武衛大將軍,封江夏郡公,檢校西麟州刺史。武德初,突厥寇馬邑,朔州總管高滿政請救,高祖令高遷督兵助鎮。俄而賊兵甚盛,高遷乃斬關宵遁,其將士皆沒,竟坐除名徙邊。後以佐命功,拜陵州刺史。永徽五年卒,贈梁州都督。



 許世緒者,並州人也。大業末,為鷹揚府司馬。見隋祚將
 亡,言於高祖曰:「天道輔德,人事與能,蹈機不發,必貽後悔。今隋政不綱,天下鼎沸,公姓當圖籙,名應歌謠,握五都之兵,當四戰之地。若遂無他計,當敗不旋踵。未若首建義旗,為天下唱,此帝王業也。」高祖甚奇之,親顧日厚。義兵起,授右一府司馬。武德中,累除蔡州刺史,封真定郡公,卒。



 弟洛仁,亦以元從功臣至冠軍大將軍、行左監門將軍。永徽初卒,贈代州都督,謚曰勇,陪葬昭陵。



 劉師立者,宋州虞城人也。初為王世充將軍,親遇甚密。
 洛陽平,當誅;太宗惜其才,特免之,為左親衛。太宗之謀建成、元吉也,嘗引師立密籌其事,或自宵達曙。其後師立與尉遲敬德、龐卿惲、李孟嘗等九人,同誅建成有功,超拜左衛率。尋遷左驍衛將軍,封襄武郡公,賜絹五千匹。後人告師立自云「眼有赤光,體有非常之相,姓氏又應符讖」。太宗謂之曰:「人言卿欲反,如何?」師立大懼,俯而對曰:「臣任隋朝,不過六品,身材駑下,不敢輒希富貴。過蒙非常之遇,常以性命許國。而陛下功成事立,臣復致
 位將軍,顧己循躬,實逾涯分,臣是何人,輒敢言反!」太宗笑曰:「知卿不然,此妄言耳。」賜帛六十匹,延入臥內慰諭之。羅藝之反也,長安人情騷動,以師立檢校右武候大將軍,以備非常。及藝平,憲司窮究黨與,師立坐與交通,遂除名。又以籓邸之舊,尋檢校岐州都督。師立上書請討吐谷渾,書奏未報,便遣使間其部落,諭以利害,多有降附,列其地為開、橋二州。又有黨項首領拓拔赤辭,先附吐谷渾,負險自固,師立亦遣人為陳利害,赤辭遂率
 其種落內屬。太宗甚嘉之,拜赤辭為西戎州都督。後師立以母憂當去職,父老上表請留,詔不許赴哀,復令居任。時河西黨項破刃氏常為邊患,又阻新附,師立總兵擊之。軍未至,破刃氏大懼,遁於山谷,師立追之,至恤於真山而還。吐谷渾於小莫門川擊破之,多所虜獲。尋轉始州刺史。十四年卒,謚曰肅。



 錢九隴,本晉陵人也,父在陳為境上所獲,沒為皇家隸人。九隴善騎射,高祖信愛之,常置左右。義兵起,以軍功
 授金紫光祿大夫。及克京城,拜左監門郎將。從平薛仁杲、劉武周,以前後戰功累授右武衛將軍。其後從太宗擒獲竇建德,平王世充;從隱太子討劉黑闥於魏州,力戰破賊,策勛為最。累封郇國公,仍以本官為苑游將軍。貞觀初,出為眉州刺史,再遷右監門大將軍。十二年,改封郇國公,加食廬州實封六百戶。尋卒,贈左武衛大將軍、潭州都督,謚曰勇,陪葬獻陵。



 樊興者,本安陸人也,父犯罪,配沒為皇家隸人。興從平
 京城,累除右監門將軍。又從太宗破薛舉,平王世充、竇建德,積戰功,累封營國公,賜物二千段、黃金三十鋌。尋坐事削爵。貞觀六年,陵州獠反,興率兵討之,拜左驍衛將軍。又從特進李靖擊吐谷渾,為赤水道行軍總管,坐遲留不赴軍期,又士卒多死,失亡甲仗,以勛減死。久之,累拜左監門大將軍,封襄城郡公。太宗之征遼東,以興忠謹,令副司空房玄齡,留守京師。俄又檢校右武候將軍。永徽初卒,贈左武候大將軍、洪州都督,陪葬獻陵。



 公孫武達者,雍州櫟陽人也。少有膂力,稱為豪俠。在隋為驍果。武德初,至長春宮請謁太宗,從討劉武周,力戰,功居最。又從平王世充、竇建德,累遷秦王府右三軍驃騎,封清水縣公。貞觀初,檢校右監門將軍,尋除肅州刺史。歲餘,突厥數千騎、輜重萬餘入侵肅州,欲南入吐谷渾。武達領二千人與其精銳相遇,力戰,虜稍卻,急攻之,遂大潰,擠之於張掖河。又命軍士於上流以伐渡兵,擊其餘眾,賊半濟,兩岸夾攻之,斬溺略盡。璽書慰勉之,拜
 左監門將軍。後又受詔擊鹽州叛突厥,武達引兵趨靈州,追及之。賊方渡河,見武達至,據河南岸。武達引兵擊之,斬其渠帥可邏拔扈,餘黨幾盡。進封東萊郡公。永徽中,累授右武衛大將軍。及卒,高宗廢朝舉哀,贈荊州都督,給東園秘器,陪葬昭陵,謚曰壯。



 龐卿惲者,並州太原人。從太宗討隱太子有功,累拜右驍衛將軍,封邾國公。尋卒,追封濮國公。



 子同善,官至右金吾大將軍。同善子承宗,開元初,為太子賓客。



 張長遜,雍州櫟陽人也。隋代為里長,平陳有功,累至五原郡通守。及天下亂,遂附於突厥,號長遜為割利特勒。及義旗建,長遜以郡降,授五原太守,尋除豐州總管。是時梁師都、薛舉請兵於突厥,欲令渡河。長遜知之,偽為詔書與莫賀咄設,示知其謀。突厥乃拒師都等使,高祖嘉之。武德元年,敕右武候驃騎將軍高靜致幣於始畢可汗,路經豐州,會可汗死,敕於所到處納庫。突厥聞而大怒,欲南渡。長遜乃遣高靜出塞,申國家賻贈之禮,突
 厥乃引還。及征薛舉,長遜不待命而至,以功授豐州總管,進封巴國公,賜以錦袍金甲。是時言事者以長遜久居豐州,與突厥連結;長遜懼,請入朝,拜右武候將軍,徙封息國公,賜以宮人、彩物千餘段。會有疾,車駕親幸其第。及竇軌率巴蜀兵擊王世充,以長遜檢校益州行臺左僕射,歷遂、夔二州總管,所在皆有惠政。貞觀十一年卒。



 李安遠者,夏州朔方人也。隋雲州刺史徹子也。家富於
 財,少從博徒不逞,晚始折節讀書,敬慕士友。襲父爵城陽公。與王珪友善。大業初,珪坐叔頗當配流,安遠為之營護,免。後為正平令。及義兵攻絳郡,安遠與通守陳叔達嬰城自守。城陷,高祖與安遠有舊,馳至其宅撫慰之,引與同食。拜右翊衛統軍,封正平縣公。武德元年,授右武衛大將軍。從太宗征伐,特蒙恩澤,累戰功,改封廣德郡公。又使於吐谷渾,與敦和好,於是吐谷渾主伏允請與中國互市,安遠之功也。後隱太子建成潛引以為黨
 援,安遠固拒之,由是太宗益加親信。貞觀初,歷潞州都督、懷州刺史。歷任頗有聲績,然傷於嚴急,時論少之。七年卒,追贈涼州都督,謚曰密。十三年,追封為遂安郡公。



 史臣曰:裴寂歷任仕隋,官至為宮監,總子女玉帛之務,據倉廩兵甲之饒,喜博戲之利茍多,啟舉義之謀為首。謁嶽神以徼福,始彰不逞之心;留貴妃以經宿,終昧為臣之道。居第一之位,乏在三之規。恃高祖之舊恩,致文靜之極法。終歸四罪,尚保再生,幸也。文靜奮縱橫之略,
 立締構之功,罔思寵辱之機,過為輕躁之行,未及封而禍也,惜哉!凡關佐命,爰第實封,小大不遺,賢愚自勸,太宗之行賞也,明矣!



 贊曰:風雲初合,共竭智力。勢利既分,遽變仇敵。



\end{pinyinscope}