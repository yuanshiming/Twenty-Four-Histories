\article{卷四 本紀第四 高宗上}

\begin{pinyinscope}

 高宗天皇大聖大弘孝皇帝,諱治,太宗第九子也,母曰文德順聖長孫皇后。以貞觀二年六月,生於東宮之麗正殿。五年,封晉王。七年,遙授並州都督。幼而岐嶷端審,
 寬仁孝友。初授《孝經》於著作郎蕭德言,太宗問曰:「此書中何言為要?」對曰:「夫孝,始於事親,中於事君,終於立身。君子之事上,進思盡忠,退思補過,將順其美,匡救其惡。」太宗大悅曰:「行此,足以事父兄,為臣子矣。」及文德皇后崩,晉王時年九歲,哀慕感動左右,太宗屢加慰撫,由是特深寵異。尋拜右武候大將軍。十七年,皇太子承乾廢,魏王泰亦以罪黜,太宗與長孫無忌、房玄齡、李勣等計議,立晉王為皇太子。太宗每視朝,常令在側,觀決庶政,
 或令參議,太宗數稱其善。十八年,太宗將伐高麗,命太子留鎮定州。及駕發有期,悲啼累日,因請飛驛遞表起居,並遞敕垂報,並許之。飛表奏事,自此始也。及軍旋,太子從至並州。時太宗患癰,太子親吮之,扶輦步從數日。二十三年五月己巳,太宗崩。庚午,以禮部尚書、兼太子少師、黎陽縣公於志寧為侍中,太子少詹事、兼尚書左丞張行成為兼侍中、檢校刑部尚書,太子右庶子、兼吏部侍郎、攝戶部尚書高季輔為兼中書令、檢校吏部尚
 書,太子左庶子、高陽縣男許敬宗兼禮部尚書。辛未,還京。



 六月甲戌朔,皇太子即皇帝位,時年二十二。詔曰:「大行皇帝奄棄普天,痛貫心靈,若置湯火。思遵大孝,不敢滅身,永慕長號,將何逮及。粵以孤眇,屬當元嗣,思勵空薄,康濟黎元。敬順惟新,仰昭先德,宜布凱澤,被乎億兆。可大赦天下。內外文武賜勛官一級。諸年八十以上賚以粟帛。雍州及諸州比年供軍勞役尤甚之處,並給復一年。」辛巳,改民部尚書為戶部尚書。疊州都督、英國公
 勣為特進、檢校洛州刺史,仍於洛陽宮留守。癸未,詔司徒、揚州都督、趙國公無忌為太尉兼檢校中書令,知尚書門下二省事,餘並如故,賜物三千段。癸巳,特進、英國公勣為開府儀同三司、同中書門下三品。秋七月丙午,有司請改治書侍御史為御史中丞,諸州治中為司馬,別駕為長史,治禮郎為奉禮郎,以避上名。以貞觀時不諱先帝二字,詔有司,奏曰:「先帝二名,禮不偏諱。上既單名,臣子不合指斥。」上乃從之。己酉,於闐王伏闍信來朝。八月癸
 酉朔,河東地震,晉州尤甚,壞廬舍,壓死者五千餘人。三日又震。詔遣使存問,給復二年,壓死者賜絹三匹。以開府儀同三司、英國公勣為尚書左僕射、同中書門下三品。僕射始帶同中書門下。庚寅,葬太宗於昭陵。



 九月甲寅,加授鄜州刺史、荊王元景為司徒,前安州都督、吳王恪為司空兼梁州刺史。丙寅,贈太尉、梁國公玄齡,贈司徒、申國公士廉,贈左僕射、蔣國公屈突通,並可配食太宗廟庭。冬十一月甲子,以瑤池都督阿史那賀魯為左
 驍衛大將軍。乙丑,晉州地又震。是冬無雪。



 永徽元年春正月辛丑朔,上不受朝,詔改元。壬寅,御太極殿,受朝而不會。丙午。立妃王氏為皇后。丁未,以陳王忠為雍州牧。二月辛卯,封皇子孝為許王,上金為杞王,素節為雍王。夏四月己巳朔,晉州地又震。五月丁未,上謂群臣曰:「朕謬膺大位,政教不明,遂使晉州之地屢有震動。良由賞罰失中,政道乖方。卿等宜各進封事,極言得失,以匡不逮。」吐火羅遣使獻大鳥如駝,食銅鐵,上遣
 獻於昭陵。吐蕃贊普死,遣右武衛將軍鮮於匡濟賚璽書往吊祭。



 六月庚辰,晉州地震。秋七月丙寅,以旱,親錄京城囚徒。九月癸卯,右驍衛郎將高侃執車鼻可汗詣闕,獻於社廟及昭陵。己未,尚書左僕射、英國公勣固請解職,許之,令以開府儀同三司同中書門下三品。十一月已未,中書令、河南郡公褚遂良左授同州刺史。十二月,瑤池都督、沙缽羅葉護阿史那賀魯以府叛,自稱可汗,總有西域之地。是歲,雍、絳、同等九州旱蝗,齊、定等十
 六州水。



 二年春正月戊戌,詔曰:「去歲關輔之地,頗弊蝗螟,天下諸州,或遭水旱,百姓之間,致有罄乏。此由朕之不德,兆庶何辜?矜物罪己,載深憂惕。今獻歲肇春,東作方始,糧廩或空,事資賑給。其遭蟲水處有貧乏者,得以正、義倉賑貸。雍、同二州,各遣郎中一人充使存問,務盡哀矜之旨,副朕乃眷之心。」乙巳,黃門侍郎、平昌縣公宇文節加銀青光祿大夫,依舊同中書門下三品。守中書侍郎柳
 奭為中書侍郎,依舊同中書門下三品。夏四月乙酉,秩太廟令及獻、昭二陵令從五品,丞從七品。



 五月壬辰,開府儀同三司及京官文武職事四品、五品,並給隨身魚。六月辛酉,開府儀同三司、襄邑王神符薨。秋七月丁未,賀魯寇陷金嶺城、蒲類縣,遣武候大將軍梁建方、右驍衛大將軍契苾何力為弓月道總管以討之。八月乙丑,大食國始遣使朝獻。己巳,侍中、燕國公於志寧為尚書左僕射,侍中兼刑部尚書、北平縣公張行成為尚書
 右僕射,並同中書門下三品,猶不入銜。中書令兼檢校吏部尚書、條縣公高季輔為侍郎。九月癸巳,改九成宮為萬年宮,廢玉華宮以為佛寺。閏月辛未,頒新定律、令、格、式於天下。冬十月辛卯,晉州地震。十一月辛酉,有事於南郊。戊辰,定襄地震。丁丑,以高昌故地置安西都護府。白水蠻冠麻州,命左領軍將軍趙孝祖討平之。



 三年春正月癸亥,以去秋至於是月不雨,上避正殿,降天下死罪及流罪遞減一等,徒以下咸宥之。弓月道總
 管梁建方、契苾何力等大破處月硃耶孤注于牢山,斬首九千級,虜渠帥六千,俘生口萬餘,獲牛馬雜畜七萬。丙寅,太尉、趙國公無忌以旱請遜位,不許。己巳,同州刺史、河南郡公褚遂良為吏部尚書、同中書門下三品。丙子,親祠太廟。丁亥,籍於千畝,賜群官帛有差。三月辛巳,黃門侍郎、平昌縣公宇文節為侍中,中書侍郎柳奭為中書令。庚申,幸觀德殿,賜文武群官大射。



 夏四月庚寅,左領軍將軍趙孝祖大破白水蠻大勃律。甲午,澧州刺史、彭王
 元則薨。五月庚辰,詔以周司沐大夫裴融,齊侍中崔季舒、給事黃門侍郎裴澤、尚書左丞封孝琰,隋儀同三司豆盧毓、御史中丞游楚客等,並門挺忠鯁,其子孫各宜甄擢。秋七月丁巳,立陳王忠為皇太子,大赦天下,五品己上子為父後者賜勛一轉,大酺三日。乙丑,左僕射於志寧兼太子少師,右僕射張行成兼太子少傅,侍中高季輔兼太子少保,侍中宇文節兼太子詹事。丁丑,上問戶部尚書高履行:「去年進戶多少?」履行奏稱:「進戶總一
 十五萬。」又問曰「隋日有幾戶?今見有幾戶?」履行奏:「隋開皇中有戶八百七十萬,即今見有戶三百八十萬。」九月丁巳,改太子中允為內允,中書舍人為內史舍人,諸率府中郎改為旅賁郎將,以避太子名。冬十月戊戌,幸同安大長公主第,又幸高陽長公主第,即日還宮。



 十一月乙亥,駁馬國遣使朝貢。庚寅,弘化長公主自吐谷渾來朝。十二月癸巳,濮王泰薨。



 四年春正月癸丑朔,上臨軒,不受朝,以濮王泰在殯故
 也。丙子,新除房州刺史、駙馬都尉房遺愛,司徒、秦州刺史、荊王元景,司空、安州刺史、吳王恪,寧州刺史、駙馬都尉薛萬徹,嵐州刺史、駙馬都尉柴令武謀反。



 二月乙酉,遺愛、萬徹、令武等並伏誅;元景、恪、巴陵高陽公主並賜死。左驍衛大將軍、安國公執失思力配流巂州,侍中兼太子詹事、平昌縣公宇文節配流桂州。戊子,特進、太常卿、江夏王道宗配流桂州,恪母弟蜀王愔廢為庶人。己亥,絳州刺史、徐王元禮加授司徒,開府儀同三司、英國
 公勣為司空。三月壬子朔,頒孔穎達《五經正義》於天下,每年明經令依此考試。丙辰,上御觀德殿,陳逆人房遺愛等口馬資財為五垛,引王公、諸親、蕃客及文武九品己上射。



 夏四月戊子,林邑國王遣使來朝,貢馴象。壬寅,以旱避正殿,減膳,親錄系囚,遣使分省天下冤獄,詔文武官極言得失。八月己亥,隕石十八於同州之馮翊,有聲如雷。九月壬寅,尚書右僕射、北平縣公張行成薨。甲戌,吏部尚書、河南郡公褚遂良為尚書右僕射,依舊知
 政事。



 冬十月庚子,幸新豐之溫湯。甲辰,曲赦新豐。乙巳,至自溫湯。戊申,睦州女子陳碩貞舉兵反,自稱文佳皇帝,攻陷睦州屬縣。婺州刺史崔義玄、揚州都督府長史房仁裕各率眾討平之。十一月癸丑,兵部尚書、固安縣公崔敦禮為侍中。頒新律疏於天下。十二月庚子,侍中兼太子少保、條縣公高季輔卒。



 五年春三月戊午,幸萬年宮。辛未,曲赦所經州縣系囚。以工部尚書閻立德領丁夫四萬築長安羅郭。



 夏四月,
 守黃門侍郎潁川縣公韓瑗、守尚書侍郎來濟,並加銀青光祿大夫,依舊同中書門下三品。閏五月丁丑夜,大雨,水漲暴溢,漂溺麟游縣居人及當番衛士,死者三千餘人。六月,恆州大雨,滹沱河泛溢,溺五千餘家。癸丑,蒲州汾陰縣暴雨,漂溺居人,浸壞廬舍。癸亥,中書令柳奭兼吏部尚書。丙寅,河北諸州大水。



 七月辛巳,有小鳥如雀,生大鳥如鳩於萬年宮皇帝舊宅。八月,大理奏決死囚,總管七十餘人。辛亥,詔自今已後,五品已上有薨亡
 者,隨身魚並不須追收。辛未,吐蕃使人獻馬百匹及大廬可高五丈,廣袤各二十七步。九月丁酉,至自萬年宮。冬十一月癸酉,築京師羅郭,和雇京兆百姓四萬一千人,板築三十日而罷,九門各施觀。十二月癸丑,倭國獻琥珀、碼瑙,琥珀大如斗,碼瑙大如五斗器。戊午,發京師謁昭陵,在路生皇子賢。已未,敕二年一定戶。



 六年春正月壬申朔,親謁昭陵,曲赦醴泉縣民,放今年租賦。陵所宿衛將軍、郎將進爵一等,陵令、丞加階賜物。
 甲戌,至自昭陵。於陵側建佛寺。庚寅,封皇子弘為代王,賢為潞王。二月乙巳,皇太子忠加元服,內外文武職事五品已上為父後者,賜勛一級。大酺三日。



 三月,營州都督程名振破高麗於貴端水。嘉州辛道讓妻一產四男。壬戌,昭儀武氏著《內訓》一篇。夏五月癸未,命左屯衛大將軍、盧國公程知節等五將軍帥師出蔥山道以討賀魯。黃門侍郎、潁川郡公韓瑗為侍中,中書侍郎、南陽男來濟為中書令。兼吏部尚書、河東縣男柳奭貶遂州刺
 史。六月,大食國遣使朝貢。秋七月乙亥,侍中、固安縣公崔敦禮為中書令。乙酉,均天下州縣公廨。八月,尚藥奉御蔣孝璋員外特置,仍同正。員外同正,自蔣孝璋始也。己酉,大理更置少卿一員。先是大雨,道路不通,京師米價暴貴,出倉粟糶之,京師東西二市置常平倉。



 九月庚午,尚書右僕射、河南郡公褚遂良以諫立武昭儀,貶授潭州都督。乙酉,洛州大水,毀天津橋。冬十月己酉,廢皇后王氏為庶人,立儀武昭氏為皇后,大赦天下。十一月
 丁卯朔,臨軒,命司空勣、左僕射志寧冊皇后,文武群官及番夷之長,奉朝皇后於肅義門。十一月己巳,皇后見於廟。癸酉,追贈後父故工部尚書、應國公、贈並州都督武士鷿為司空。丙子,淄州高苑縣吳文威妻魏氏一產四男,三見育。癸巳,應國夫人楊氏改封代國夫人。十二月,遣禮部尚書、高陽縣男許敬宗每日待詔於武德殿西門。



 七年春正月辛未,廢皇太子忠為梁王,立代王弘為皇
 太子。壬申,大赦,改元為顯慶。文武九品已上及五品己下子為父後者,賜勛官一轉。大酺三日。甲子,尚書左僕射兼太子少師、燕國公於志寧兼太子太傅,侍中韓瑗、中書令來濟、禮部尚書許敬宗,並為太子賓客。始有賓客也。御玄武門,餞蔥山道大總管程知節。



 二月庚寅,名《破陣樂》為《神功破陣樂》。辛亥,贈司空武士鷿為司徒、周國公。三月辛巳,皇后祀先蠶於北郊。丙戌,戶部侍郎杜正倫為守黃門侍郎、同中書門下三品。



 夏四月戊申,御
 安福門,觀僧玄奘迎禦制並書慈恩寺碑文,導從以天竺法儀,其徒甚盛。五月己卯,太尉長孫無忌進史官所撰梁、陳、周、齊、隋《五代史志》三十卷。弘文館學士許敬宗進所撰《東殿新書》二百卷,上自製序。六月,岐州刺史、潞王賢為雍州牧。秋七月癸未,中書令兼檢校太子詹事、固安縣公崔敦禮為太子少師、同中書門下三品。改戶部尚書為度支尚書,侍郎亦然。



 八月丙申,太子少師崔敦禮卒。左衛大將軍程知節與賀魯所部歌邏祿獲剌
 頡發及處月預支俟斤等戰於榆幕谷,大破之,斬首千餘級,獲駝馬牛羊萬計。九月癸酉,初詔戶滿三萬已上為上州,二萬已上為中州;先為上州、中州者各依舊。皇后制《外戚誡》。庚辰,括州海水泛溢,壞安固、永嘉二縣,損四千餘家。辛巳,初制都督及上州各置執刀十五人,中州、下州十人。癸未,初置驃騎大將軍,官為從一品。程知節與賀魯男咥運戰,斬首數千級,進至怛篤城,俘其部落戶口及貨物巨積。



 冬十一月乙丑,皇子顯生,詔京官、
 朝集使各加勛級。十二月乙酉,置算學。左屯衛大將軍程知節坐討賀魯逗留,追賊不及,減死免官。罷蘭州都督。鄯州置都督。



 二年春正月庚寅,幸洛陽。命右屯衛將軍蘇定方等四將軍為伊麗道將軍,帥師以討賀魯。二月辛酉,入洛陽宮,曲赦洛州。庚午,封皇第七子顯為周王,徙封許王素節為郇王。三月甲子,中書侍郎李義府為中書令兼檢校御史大夫,黃門侍郎杜正倫兼度支尚書,依舊同中
 書門下三品。夏五月丙申,幸明德宮。秋七月丁亥,還洛陽宮。八月丁卯,侍中、潁川縣公韓瑗左授振州刺史,中書令兼太子詹事、南陽侯來濟左授臺州刺史,皆坐諫立武昭儀為皇后,救褚遂良之貶也。禮部尚書、高陽郡公許敬宗為侍中,以立武后之功也。九月庚寅,度支尚書杜正倫為中書令。



 冬十月戊戌,親講武於許、鄭之郊,曲赦鄭州。遣使祭鄭大夫國僑、漢太丘長陳實墓。十二月乙卯,還洛陽宮。庚午,改「皞」「葉」字。丁卯,手詔改洛陽宮
 為東都,洛州官員階品並準雍州。廢穀州,以福昌等四縣,並懷州河陽、濟源、溫,鄭州汜水並隸洛州。已巳,中書省置起居舍人兩員,品同起居郎。庚午,以周王顯為洛州牧。壬午,分散騎常侍為左右各兩員,其右散騎常侍隸中書省。



 三年春正月戊子,太尉、趙國公無忌等修《新禮》成,凡一百三十卷,二百五十九篇,詔頒於天下。二月丁巳,車駕還京。壬午,親錄囚徒,多所原宥。蘇定方攻破西突厥沙
 缽羅可汗賀魯及咥運、闕啜。賀魯走石國,副將蕭嗣業追擒之,收其人畜前後四十餘萬。甲寅,西域平,以其地置濛池、昆陵二都護府。復於龜茲國置安西都護府,以高昌故地為西州。置懷化大將軍正三品,歸化將軍從三品,以授初附首領,仍分隸諸衛。六月,程名振攻高麗。九月,廢書、算、律學。有司奏請造排車七百乘,擬行幸載排城;上以為勞民,乃於舊頓置院墻焉。



 冬十一月乙酉,兼中書令、皇太子賓客兼檢校御史大夫、河間郡公李
 義府左授普州刺史,兼中書令、皇太子賓客、襄陽郡公杜正倫左授橫州刺史。中書侍郎李友益除名,配流巂州。戊戌,侍中許敬宗權檢校中書令。戊子,侍中、皇太子賓客、權檢校中書令、高陽郡公許敬宗為中書令,賓客已下如故;大理卿辛茂將為侍中。鴻臚卿蕭嗣業於石國取賀魯至,獻於昭陵。甲辰,開府儀同三司、鄂國公尉遲敬德薨。



 四年春二月乙亥,上親策試舉人,凡九百人,惟郭待封、
 張九齡五人居上第,令待詔弘文館,隨仗供奉。三月,以左驍衛大將軍、郕國公契苾何力往遼東經略。夏四月己未,太子太傅、尚書左僕射、燕國公於志寧為太子太師,仍同中書門下三品。乙丑,黃門侍郎許圉師同中書門下三品。丙戌,太子太師、同中書門下三品、燕國公於志寧免官,放還私第。戊戌,太尉、揚州都督、趙國公無忌帶揚州都督於黔州安置,依舊準一品供給。五月丙申,兵部尚書任雅相、度支尚書盧承慶並參知政事。秋七
 月壬子,普州刺史李義府為吏部尚書,同中書門下三品。冬十月乙巳,皇太子加元服,大赦天下,文武五品己上子孫為父祖後者加勛官一級,大酺三日。閏十月戊寅,幸東都,皇太子監國。戊戌,至東都。十一月,以中書侍郎許圉師為散騎常侍、檢校侍中。戊午,兼侍中辛茂將卒。癸亥,以邢國公蘇定方為神丘道總管,劉伯英為昆夷道總管。



 五年春正月甲子,幸並州。二月辛巳,至並州。丙戌,宴從
 官及諸親、並州官屬父老,賜帛有差。曲赦並州及管內諸州。義旗初職事五品已上身亡歿墳墓在並州者,令所司致祭。佐命功臣子孫及大將軍府僚佐已下今見存者,賜階級有差,量才處分。起義之徒職事一品己下,賜物有差。年八十已上,版授刺史、縣令。佐命功臣食別封身已歿者,為後子孫各加兩階。賜酺三日。甲午,祠舊宅,以武士鷿、殷開山、劉政會配食。



 三月丙午,皇后宴親族鄰里故舊於朝堂,命婦婦人入會於內殿,及皇室諸
 親賜帛各有差,及從行文武五品以上。制以皇后故鄉並州長史、司馬各加勛級。又皇后親預會,每賜物一千段,期親五百段,大功已下及無服親、鄰里故舊有差。城內及諸婦女年八十已上,各版授郡君,仍賜物等。己酉,講武於並州城西,上御飛閣,引群臣臨觀。辛亥,發神丘道軍伐百濟。丁巳,左右領始改左右千牛。



 夏四月戊寅,車駕還東都,造八關宮於東都苑內。癸亥,至自並州。五月壬戌,幸八關宮,改為合璧宮。六月庚午朔,日有蝕之。
 辛卯,詔文武五品己上四科舉人。甲午,駕還東都。秋七月乙巳,廢梁王忠為庶人,徙於黔州。戊辰,度支尚書、同中書門下三品盧承慶以罪免。八月庚辰,蘇定方等討平百濟,面縛其王扶餘義慈。國分為五部,郡三十七,城二百,戶七十六萬,以其地分置熊津等五都督府。曲赦神丘、昆夷道總管已下,賜天下大酺三日。九月戊午,賜英國公勣墓塋一所。



 冬十月丙子,代國夫人楊氏改榮國夫人,品第一,位在王公母妻之上。十一月戊戌朔,邢
 國公蘇定方獻百濟王扶餘義慈、太子隆等五十八人俘於則天門,責而宥之。乙卯,狩於許、鄭之郊。十二月己卯,至自許州。



 六年春正月乙卯,於河南、河北、淮南六十七州募得四萬四千六百四十六人,往平壤帶方道行營。二月乙未,以益、綿等州皆言龍見,改元。曲赦洛州。龍朔元年三月丙申朔,改元。壬戌,幸合璧宮。夏五月丙申,命左驍衛大將軍、涼國公契苾何力為遼東道大總管,左武衛大將
 軍、邢國公蘇定方為平壤道大總管,兵部尚書、同中書門下三品、樂安縣公任雅相為浿江道大總管,以伐高麗。是日,皇后請禁天下婦人為俳優之戲,詔從之。甲子晦,日有蝕之。



 六月庚寅,中書令許敬宗等進《累璧》六百三十卷,目錄四卷。秋七月癸卯,車駕還東都。八月丙戌,令諸州舉孝行尤著及累葉義居可以勵風俗者。九月甲辰,以河南縣大女張年百三歲,親幸其第。又幸李勣之第。天宮寺是高祖潛龍時舊宅,上周歷殿宇,感愴久
 之,度僧二十人。皇后至許圉師第。壬子,徙封潞王賢為沛王。是日,以雍州牧、幽州都督、沛王賢為揚州都督、左武候大將軍,牧如故。以洛州牧、周王顯為並州都督。是日,敕中書門下五品已上諸司長官、尚書省侍郎並諸親三等已上,並詣沛王宅設宴禮,奏《九部樂》。禮畢,賜帛雜彩等各有差。



 冬十月丁卯,狩於陸渾。癸酉,還宮。是歲,新羅王金春秋卒,其子法敏嗣立。



 二年春正月乙巳,太府寺更置少卿一員,分兩京檢校。
 丙午,東都初置國子監,並加學生等員,均分於兩都教授。二月甲子,改京諸司及百官名:尚書省為中臺,門下省為東臺,中書省為西臺,左右僕射為左右匡政,左右丞為肅機,侍中為左相,中書令為右相,自餘各以義訓改之。又改六宮內職名。甲戌,司戎太常伯、浿江道總管、樂安縣公任雅相卒於軍。三月甲申,自東都還京。癸丑,幸同州。蘇定方破高麗於葦島,又進攻平壤城,不克而還。



 夏四月庚申朔,至自東都。辛巳,造蓬萊宮成,徙居之。
 五月丙申,左侍極許圉師為左相。乙巳,復置律、書、算三學。



 六月己未朔,皇子旭輪生。乙丑,初令道士、女冠、僧、尼等,並盡禮致拜其父母。乙亥,制蓬萊宮諸門殿亭等名。秋七月丁亥朔,以東宮誕育滿月,大赦天下,賜酺三日。八月甲午,右相許敬宗乞骸骨。壬寅,許敬宗為太子少師,同東西臺三品,仍知西臺事。九月,司禮少常伯孫茂道奏稱:「八品、九品舊令著青,亂紫,非卑品所服,望令著碧。」詔從之。戊寅,前吏部尚書、河間郡公李義府起復為
 司列太常伯,同東西臺三品。



 冬十月丁酉,幸溫湯,皇太子弘監國。丁未,至自溫湯。庚戌,西臺侍郎上官儀同東西臺二品。十一月辛未左相許圉師下獄。癸酉,封皇第四子旭輪為殷王。十二月辛丑,改魏州為冀州大都督府,改冀州為魏州。又以並、揚、荊、益四都督府並為大都督府。沛王賢為揚州大都督,周王顯為並州大都督,殷王旭輪遙領冀州大都督。左相許圉師解見任。



 三年春正月,左武衛大將軍鄭仁泰等帥師討鐵勒餘
 種,盡平之。乙丑,司列太常伯李義府為右相。二月丙戌,隴、雍、同、岐等一十五州戶口,徵修蓬萊宮。癸巳,置太子左右諭德及桂坊大夫等官員,改司經局為桂坊館,崇賢館罷隸左春坊。丁酉,減京官一月俸,助修蓬萊宮。庚戌,詔曰:「天德施生,陽和在節,言念幽圄,載惻分宵。雖復每有哀矜,猶恐未免枉濫。在京系囚應流死者,每日將二十人過。」於是親自臨問,多所原宥,不盡者令皇太子錄之。詔以書學隸蘭臺,算學隸秘閣,律學隸詳刑寺。改
 燕然都護府為瀚海都護府,瀚海都護府為雲中都護府。二月,前左相許圉師左遷虔州刺史。太子弘撰《瑤山玉彩》成,書凡五百卷。



 夏四月乙丑,右相李義府下獄。戊子,李義府除名,配流巂州。丙午,幸蓬萊宮新起含元殿。秋八月癸卯,彗星見於左攝提。戊申,詔百僚極言正諫。命司元太常伯竇德玄、司刑太常伯劉詳道等九人為持節大使,分行天下。仍令內外官五品已上各舉所知。



 冬十月丙申,絳州麟見於介山。丙午,含元殿前麟趾見。
 十一月癸酉,雨冰。十二月庚子,詔改來年正月一日為麟德元年。



 麟德元年春正月甲子,改雲中都護府為單于大都護府,官品同大都督府。二月丁亥,加授殷王旭輪單于大都護。戊子,幸萬年宮。三月辛亥,展大射禮。丁卯,長女追封安定公主,謚曰思,其鹵簿鼓吹及供葬所須,並如親王之制,於德業寺遷於崇敬寺。



 夏四月,衛州刺史、道王元慶薨。五月,許王孝薨。乙卯,於昆明之弄棟川置姚州都督府。秋八月丙子朔,至自萬年宮,便幸舊宅。己卯,降
 萬年縣系囚,因幸大慈恩寺。壬午,還蓬萊宮。戊子,兼司列太常伯、檢校沛王府長史、城陽縣侯劉祥道兼右相,大司憲竇德玄兼司元太常伯、檢校左相。九月己卯,詔曰:「周京兆尹、左右宮伯大將軍、司衛上將軍、少塚宰、廣陵郡公宇文孝伯,忠亮存心,貞賢表志。淫刑既逞,方納諫而求仁;忍忌將加,甘捐軀而徇節。年載雖久,風烈猶生,宜峻徽章,式旌胤胄。其孫左威衛長史思純,可加授朝散大夫。」十二月丙戌,殺西臺侍郎上官儀。戊子,庶人
 忠坐與儀交通,賜死。右相、城陽縣侯劉祥道為司禮太常伯。太子右中護檢校西臺侍郎樂彥瑋、西臺侍郎孫處約同知政事。是冬無雪。



 二年春正月壬午,幸東都。丁酉,幸合璧宮。戊子,慮雍、洛二州及諸司囚。甲子,以發向泰山,停選。三月甲寅,兼司戎太常伯、永安郡公姜恪同東西臺三品。辛未,東都造乾元殿成。閏月癸酉,日有蝕之。四月丙午,曲赦桂、廣、黔三都督府管內大闢罪已上。丙寅,講武邙山之陽,御城
 北樓觀之。戊辰,左侍極、仍檢校大司成、嘉興縣子陸敦信為檢校右相,其大司成宜停。西臺侍郎孫處約、樂彥瑋並停知政事。



 五月辛卯,以秘閣郎中李淳風造歷成,名《麟德歷》,頒之。以司空、英國公李勣,少師、高陽郡公許敬宗,右相、嘉興縣子陸敦信,左相、鉅鹿男竇德玄為檢校封禪使。六月,鄜州大水,壞城邑。秋七月,鄧王元裕薨。



 冬十月戊午,皇后請封禪,司禮太常伯劉祥道上疏請封禪。癸亥,高麗王高藏遣其子福男來朝。丁卯,將封泰山,
 發自東都。是歲大稔,米斗五錢,麰麥不列市。十一月丙子,次於原武,以少牢祭漢將紀信墓,贈驃騎大將軍。庚寅,華州刺史、燕國公於志寧卒。十二月丙午,御齊州大。乙卯,命有司祭泰山。丙辰,發靈巖頓。



\end{pinyinscope}