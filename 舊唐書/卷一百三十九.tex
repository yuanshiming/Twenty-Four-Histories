\article{卷一百三十九}

\begin{pinyinscope}

 ○盧杞子
 元輔白志貞裴延齡韋渠牟李齊運李實韋執誼王叔文王伾附程異皇甫抃弟鏞



 盧杞,字子良,故相懷慎之孫。父奕,天寶末為東臺御史中丞;洛城為安祿山所陷,奕守司而遇害。杞以門廕,解褐清道率府兵曹。朔方節度使僕固懷恩闢為掌書記、試大理評事、監察御史,以病免。入補鴻臚丞,遷殿中侍御史、膳
 部員外郎,出為忠州刺史。至荊南,謁節度使衛伯玉,伯玉不悅。杞移病歸京師,歷刑部員外郎、金部吏部二郎中。



 杞貌陋而色如藍,人皆鬼視之。不恥惡衣糲食,人以為能嗣懷慎之清節,亦未識其心。頗有口辯。出為虢州刺史。建中初,徵為御史中丞。時尚父子儀病,百官造問,皆不屏姬侍。及聞杞至,子儀悉令屏去,獨隱幾以待之。杞去,家人問其故,子儀曰「杞形陋而心險,左右見之必笑。若此人得權,即吾族無類矣。」及居糾彈顧問之地,論奏稱旨,遷御史大夫。旬日,為門下侍郎、同中書門下平章事。既居相位,忌能妒賢,迎吠陰害,小不附者,必致之於死,將起勢立威,以久其權。楊炎以杞陋貌無識,同處臺司,心甚不悅,為杞所譖,逐於崖州。德宗幸奉天,崔寧流涕論時事,杞聞惡之,譖於德宗,言寧與硃泚盟誓,故至遲回,寧遂見殺。惡顏真卿之直言,令奉使李希烈,竟歿於賊。初,京兆尹嚴郢與楊炎有隙,杞乃擢郢為御史大夫以傾炎;炎既貶死,心又惡郢,圖欲去之。宰相張鎰忠正有才,上所
 委信,杞頗惡之。會硃滔、硃泚弟兄不睦,有泚判官蔡廷玉者離間滔,滔論奏,請殺之。廷玉既貶,殿中侍御史鄭詹遣吏監送,廷玉投水而卒。杞因奏曰:「恐硃泚疑為詔旨,請三司按鞠詹;又御史所為,稟大夫命,並令按郢。」詹與張鎰善,每伺杞晝眠,輒詣鎰,杞知之。他日,杞假寢佯熟,伺詹果來,方與鎰語,杞遽至鎰閣中,詹趨避杞,杞遽言密事,鎰曰:「殿中鄭侍御在此。」杞佯愕曰:「向者所言,非他人所宜聞。」時三司使方按詹、郢,獄未具而奏殺詹,貶郢為驩州刺史。鎰尋罷相,出鎮鳳翔。其陰禍賊物如此。李揆舊德,慮德宗復用,乃遣
 使西蕃,天下無不扼腕痛憤,然無敢言者。戶部侍郎、判度支杜佑,甚承恩顧,為杞媒孽,貶饒州刺史。



 初,上即位,擢崔祐甫為相,頗用道德寬大,以弘上意,故建中初政聲藹然,海內想望貞觀之理;及杞為相,諷上以刑名整齊天下。初,李希烈請討梁崇義,崇義誅而希烈叛,盡據淮右、
 襄、鄧之郡邑。恆州李寶臣死,其子惟岳邀節鉞,遂與田悅締結以抗王師,由是河北、河南連兵不息。度支使杜佑計諸道用軍月費一百餘萬貫,京師帑廩不支數月;且得五百萬貫,可支半歲,則用兵濟矣。杞乃以戶部侍郎趙贊判度支,贊亦無計可施,乃與其黨太常博士韋都賓等謀行括率,以為泉貨所聚,在於富商,錢出萬
 貫者,留萬貫為業,有餘,官借以給軍,冀得五百萬貫。上許之,約以罷兵後以公錢還。敕即下,京兆少尹韋禎督責頗峻,長安尉薛萃荷校乘車,搜人財貨,意其不實,即行搒箠,人不勝冤痛,或有自縊而死者,京師囂然如被賊盜。都計富戶田宅奴婢等估,才及八十八萬貫。又以僦櫃納質積錢貨貯粟麥等,一切借四分之一,封其櫃窖,長安為之罷市,百姓相率千萬眾邀宰相於道訴之。杞初雖慰諭,後無以遏,即疾驅而歸。計僦質與借商,才
 二百萬貫。德宗知下民流怨,詔皆罷之,然宿師在野,日須供饋。



 明年六月,趙贊又請稅間架、算除陌。凡屋兩架為一間,分為三等:上等每間二千,中等一千,下等五百。所由吏秉筆執籌,入人第舍而計之。凡沒一間,杖六十,告者賞錢五十貫文。除陌法,天下公私給與貿易,率一貫舊算二十,益加算為五十,給與物或兩換者,約錢為率算之。市主人牙子各給印紙,人有買賣,隨自署記,翌日合算之。有自貿易不用市牙子者,驗其私簿,投狀自
 其有私簿投狀。其有隱錢百,沒入;二千,杖六十;告者賞錢十千,出於其家。法既行,主人市牙得專其柄,率多隱盜,公家所入,百不得半,怨讟之聲,囂然滿於天下。及十月,涇師犯闕,亂兵呼於市曰:「不奪汝商戶僦質矣!不稅汝間架除陌矣!」是時人心悉怨,涇師乘間謀亂,奉天之奔播,職杞之由。故天下無賢不肖,視杞如仇。



 德宗在奉天,為硃泚攻圍,李懷光自魏縣赴難。或謂王翃、趙贊曰:「懷光累嘆憤,以為宰相謀議乖方,度支賦斂煩重,京尹
 刻薄軍糧,乘輿播遷,三臣之罪也。今懷光勛業崇重,聖上必開襟布誠,詢問得失,使其言入,豈不殆哉!」翃、贊白於杞,杞大駭懼,從容奏曰:「懷光勛業,宗社是賴。臣聞賊徒破膽,皆無守心。若因其兵威,可以一舉破賊;今若許其朝覲,則必賜宴,賜宴則留連,使賊得京城,則從容完備,恐難圖之。不如使懷光乘勝進收京城,破竹之勢,不可失也。」帝然之,乃詔懷光率眾屯便橋,克期齊進。懷光大怒,遂謀異志,德宗方悟為杞所構。物議喧騰,歸咎於
 杞,乃貶為新州司馬,白志貞恩州司馬,趙贊為播州司馬。



 遇赦,移吉州長史。在貶所謂人曰:「吾必再入用。」是日,上果用杞為饒州刺史。給事中袁高宿直,當草杞制,遂執以謁宰相盧翰、劉從一曰:「杞作相三年,矯誣陰賊,排斥忠良,朋附者亥唾立至青雲,睚眥者顧盼已擠溝壑。傲很背德,反亂天常,播越鑾輿,瘡痍天下,皆杞之為也。幸免誅戮,唯示貶黜,尋已稍遷近地,更授大郡,恐失天下望,惟相公執奏之,事尚可救。」翰、從一不悅,遂改命舍
 人草制。明日詔下,袁高執奏曰:「盧杞為政,極恣兇惡,三軍將校,願食其肉,百闢卿士,嫉之若仇。」諫官趙需、裴佶、宇文炫、盧景亮、張薦等上疏曰:「伏以吉州長史盧杞,外矯儉簡,內藏奸邪,三年擅權,百揆失序,惡直醜正,亂國殄人,天地神祗所知,蠻夷華夏同棄。伏惟故事,皆得上聞,自杞為相,要官大臣,動逾月不敢奏聞,百僚惴惴,常懼顛危。及京邑傾淪,皇輿播越,陛下炳然覺悟,出棄遐荒,制曰:『忠讜壅於上聞,朝野為之側目。』由是忠良激勸,
 內外歡欣;今復用為饒州刺史,眾情失望,皆謂非宜。臣聞君之所以臨萬姓者,政也;萬姓之所以載君者,心也。倘加巨奸之寵,必失萬姓之心,乞回聖慈,遽輟新命。」疏奏不答。諫官又論曰:「盧杞蒙蔽天聽,隳紊朝典,致亂危國,職杞之由,可謂公私巨蠹,中外棄物。自聞再加擢用,忠良痛骨,士庶寒心。臣昨者瀝肝上聞,冒死不恐,冀回宸睠,用快群情;至今拳拳,未奉聖旨,物議騰沸,行路驚嗟。人之無良,一至於此。伏乞俯從眾望,永棄奸臣。幸免
 誅夷,足明恩貸;特加榮寵,恐造禍階。臣等忝列諫司,今陳狂瞽。」給事中袁高堅執不下,乃改授澧州別駕。翌日延英,上謂臣曰:「朕欲授杞一小州刺史,可乎?」李勉對曰:「陛下授杞大郡亦可,其如兆庶失望何?」上曰:「眾人論杞奸邪,朕何不知?」勉曰:「盧杞奸邪,天下人皆知;唯陛下不知,此所以為奸邪也!」德宗默然良久。散騎常侍李泌復對,上曰:「盧杞之事,朕已可袁高所奏,如何?」泌拜而言曰:「累日外人竊議,以陛下同漢之桓、靈;臣今親承聖旨,
 乃知堯、舜之不迨也!」德宗大悅,慰勉之。杞尋卒於澧州。



 子元輔,字子望,少以清行聞於時。進士擢第,授崇文館校書郎。德宗思杞不已,乃求其後,特恩拜左拾遺,再遷左司員外郎,歷杭、常、絳三州刺史。以課最高,徵為吏部郎中,遷給事中,改刑部侍郎。自兵部侍郎出為華州刺史、潼關防禦、鎮國軍等使,復為兵部侍郎。元輔自祖至曾,以名節著於史冊。元輔簡絜貞方,綽繼門風,歷踐清貫,人亦不以父之醜行為累,人士歸美。大和三年八月
 卒,時年五十六。



 白志貞者,太原人,本名琇珪。出於胥吏,事節度使李光弼,小心勤恪,動多計數,光弼深委信之,帳中之事,與琇珪參決。代宗素知之,光弼薨後,用為司農少卿,遷太卿,在寺十餘年。德宗嘗召見與語,引為腹心,遂用為神策軍使、檢校左散騎常侍、兼御史大夫,賜名志貞。善伺候上意,言無不從。



 建中四年,李希烈陷汝州,命志貞為京城召募使。時尚父子儀端王傅吳仲孺家財巨萬,以
 國家召募有急,懼不自安,乃上表請以子弟率奴客從軍,德宗嘉之,超授五品官。由是志貞請令節度、觀察、團練等使並嘗為是官者,令家出子弟甲馬從軍,亦與其男官。是時豪家不肖子幸之,貧而有知者苦之。自是京師人心搖震,不保家室。時禁軍募致,悉委志貞,兩軍應赴京師,殺傷殆盡,都不奏聞,皆以京師沽販之徒以填其闕。其人皆在市廛,及涇師犯闕,詔志貞以神策軍拒賊,無人至者,上無以禦寇,乃圖出幸。時令狐建以龍武
 軍四百人從駕至奉天,仍以志貞為行在都知兵馬使。聞李懷光至,恐暴揚其罪,乃與盧杞同沮懷光入朝,眾議喧沸,言致播遷,盧杞、志貞之罪也。故與杞同貶,遇赦量移閬州別駕。貞元二年,遷果州刺史,宰臣李勉及諫官表疏論列,言志貞與盧杞罪均,未宜敘用,固執不許,凡旬日,方下其詔。貞元三年,遷潤州刺史、兼御史大夫、浙西觀察使。是年六月卒。



 裴延齡,河東人。父旭,和州刺史。延齡,乾元末為汜水縣
 尉,遇東都陷賊,因寓居鄂州,綴緝裴駰所注《史記》之闕遺,自號小裴。後華州刺史董晉闢為防禦判官;黜陟使薦其能,調授太常博士。盧杞為相,擢為膳部員外郎、集賢院直學士,改祠部郎中。崔造作相,改易度支之務,令延齡知東都度支院。及韓滉領度支,召赴京,守本官,延齡不待詔命,遽入集賢院視事。宰相延賞惡其輕率,出為昭應令,與京兆尹鄭叔則論辨是非,攻訐叔則之短。時李泌為相,厚於叔則;中丞竇參恃恩寵,惡泌而佑延
 齡。叔則坐貶為永州刺史,延齡改著作郎。竇參尋作相,用為太府少卿,轉司農少卿。貞元八年,班宏卒,以延齡守本官,權領度支。自揣不通殖貨之務,乃多設鉤距,召度支老吏與謀,以求恩顧,乃奏云:「天下每年出入錢物,新陳相因,常不減六七千萬貫,唯有一庫,差舛散失,莫可知之。請於左藏庫中分置別庫:欠、負、耗、剩等庫及季庫、月庫,納諸色錢物。」上皆從之。且欲多張名目以惑上聽,其實於錢物更無增加,唯虛費簿書、人吏耳。



 其年,遷
 戶部侍郎、判度支,奏請令京兆府以兩稅青苗錢市草百萬圍送苑中。宰相陸贄、趙憬議,以為:「若市送百萬圍草,即一府百姓,自冬歷夏,般載不了,百役供應,須悉停罷,又妨奪農務。請令府縣量市三二萬圍,各貯側近處,他時要即支用。」京西有汙池卑濕處,時有蘆葦生焉,亦不過數畝,延齡乃奏曰:「廊馬冬月合在槽櫪秣飼,夏中即須牧放。臣近尋訪知長安、咸陽兩縣界有陂池數百頃,請以為內廊牧馬之地;且去京城十數里,與苑廊中
 無別。」上初信之,言於宰相,對曰:「恐必無此。」上乃差官閱視,事皆虛妄,延齡既慚且怒。又誣奏李充為百姓妄請積年和市物價,特敕令折填,謂之「底折錢」。嘗因奏對請積年錢帛以實帑藏,上曰:「若為可得錢物?」延齡奏曰:「開元、天寶中,天下戶僅千萬,百司公務殷繁,官員尚或有闕;自兵興已來,戶口減耗大半,今一官可兼領數司。伏請自今已後,內外百司官闕,未須補置,收其闕官祿俸,以實帑藏。」



 後因對事,上謂延齡曰:「朕所居浴堂院殿一
 栿,以年多之故,似有損蠹,欲換之未能。」對曰:「宗廟事至重,殿栿事至輕。況陛下自有本分錢物,用之不竭。」上驚曰:「本分錢何也?」對曰:「此是經義證據,愚儒常材不能知,陛下正合問臣,唯臣知之。準《禮經》,天下賦稅當為三分:一分充乾豆,一分充賓客,一分充君之庖廚。乾豆者,供宗廟也。今陛下奉宗廟,雖至敬至嚴,至豐至厚,亦不能一分財物也。只如鴻臚禮賓、諸國蕃客,至於回紇馬價,用一分錢物,尚有贏羨甚多。況陛下御膳宮廚皆極簡
 儉,所用外分賜百官充俸料、飧錢等,猶未能盡。據此而言,庖廚者之餘,其數尚多,皆陛下本分也。用修數十殿亦不合疑慮,何況一栿。」上曰:「經義如此,人總不曾言之。」頷之而已。又因計料造神龍寺,須長五十尺松木,延齡奏曰:「臣近於同州檢得一穀木,可數千條,皆長八十尺。」上曰:「人言開元、天寶中側近求覓長五六十尺木,尚未易,須於嵐、勝州採市,如今何為近處便有此木?」延齡奏曰:「臣聞賢材、珍寶、異物,皆在處常有,但遇聖君即出見。
 今此木生關輔,蓋為聖君,豈開元、天寶合得有也!」



 時陸贄秉政,上素所禮重,每於延英極論其誕妄,不可令掌財賦。德宗以為排擯,待延齡益厚。贄上書疏其失曰:



 前歲秋首,班宏喪亡,特詔延齡繼司邦賦。數日之內,遽衒功能,奏稱,「勾獲隱欺,計錢二十萬貫,請貯別庫以為羨餘,供御所須,永無匱乏。」陛下欣然信納,因謂委任得人。既賴盈餘之財,稍弘心意之欲,興作浸廣,宣索漸多。延齡務實前言,且希睿旨,不敢告闕,不敢辭難。勾獲既是
 虛言,無以應命;供辦皆承嚴約,茍在及期。遂乃搜求市廛,豪奪入獻;追捕夫匠,迫脅就功。以敕索為名,而不酬其直;以和雇為稱,而不償其傭。都城之中,列肆為之晝閉;興役之所,百工比於幽囚。聚詛連郡,遮訴盈路,持綱者莫敢致詰,巡察者莫敢為言。時有訐而言之,翻謂黨邪醜直。天子轂下,囂聲沸騰,四方觀瞻,何所取則。傷心於止,斂怨於人,欺天陷君,遠近危懼,此其罪之大者也。



 總制邦用,度支是司;出納貨財,太府攸職。凡是太府出
 納,皆稟度支文符,太府依符以奉行,度支憑案以勘覆,互相關鍵,用絕奸欺。其出納之數,則每旬申聞;見在之數,則每月計奏。皆經度支勾覆,又有御史監臨,旬旬相承,月月相繼。明若指掌,端如貫珠,財貨多少,無容隱漏。延齡務行邪諂,公肆誣欺,遂奏云「左藏庫司多有失落,近因檢閱使置簿書,乃於糞土之中收得十三萬兩,其匹段雜貨又百萬有餘,皆是文帳脫遺,並同已棄之物。今所收獲,即是羨餘,悉合移入雜庫,以供別敕支用者。」
 其時特宣進止,並依所奏施行。太府卿韋少華抗疏上陳,殊不引伏,確稱「每月申奏,皆是見在數中,請令推尋,足驗奸詐。」兩司既有論執,理須詳辦是非,陛下縱其妄欺,不加按問。以在庫之物為收獲之功,以常賦之財為羨餘之費,罔上無畏,示人不慚,此又罪之大者也。



 國家府庫,出納有常,延齡險猾售奸,詭譎求媚,遂於左藏之內,分建六庫之名,意在別貯贏餘,以奉人主私欲。曾不知王者之體,天下為家,國不足則取之於人,人不足則
 資之於國,在國為官物,在人為私財,何謂贏餘,須別收貯?是必巧詐以變移官物,暴法以刻削私財,舍此二途,其將安取?陛下方務崇信,不加檢裁,姑務保持,曾無詰責。延齡謂能蔽惑,不復懼思,奸威既沮於四方,憸態復行於內府。由是蹂躪官屬,傾倒貨財,移東就西,便為課績,取此適彼,遂號羨餘,愚弄朝廷,有同兒戲。



 夫理天下者,以義為本,以利為末,以人為本,以財為末,本盛則其末自舉,末大則其本必傾。自古及今,德義立而利用不
 豐,人庶安而財貨不給,因以喪邦失位者,未之有也。故曰:「不患寡而患不均,不患貧而患不安。」「有德必有人,有土必有土,有人必有財。」「百姓足,君孰與不足?」蓋謂此也。自古及今,德義不立而利用克宣,人庶不安而財貨可保,因以興邦固位者,未之有也。故曰:「財散則人聚,財聚則人散。」「與其有聚斂之臣,寧有盜臣。」無令侵削兆人,為天子取怨於下也。且陛下初膺寶歷,志翦群兇,師旅繁興,徵求浸廣,榷算侵剝,下無聊生。是以涇原叛徒,乘人
 怨咨,白晝犯闕,都邑甿庶,恬然不驚,反與賊眾相從,比肩而入宮殿。雖蚩蚩之性,靡所不為,然亦由德澤未浹,而暴令驅之,以至於是也。於時內府之積,尚如丘山,竟資兇渠,以餌貪卒,此則陛下躬睹之矣。是乃失人而聚貨,夫何利之有焉!



 車駕既幸奏天,逆泚旋肆圍逼,一壘之內,萬乘所屯,窘如涸流,庶物空匱。嘗欲發一健步出覘賊軍,其人懇以苦寒為辭,跪奏乞一襦褲,陛下為之求覓不致,竟閔默而遣之。又嘗宮壺之中,服用有闕,聖
 旨方戎事為急,不忍重煩於人,乃剝親王飾帶之金,賣以給直。是時行從將吏,赴難師徒,蒼黃奔馳,咸未冬服,漸屬凝冱,且無薪蒸,饑凍內攻,矢石外迫。晝則荷戈奮迅,夜則映堞呻吟,凌風飆,冒霜雪,逾四旬而眾無攜貳,卒能走強賊、全危城者,陛下豈有嚴刑重賞使之然耶?唯以不厚其身,不藏其貨,與眾庶同其憂患,與士伍共其有無,乃能使人捐軀命而捍寇仇,餒之不離,凍之不憾,臨危而不易其守,見死而不去其君,所謂「聖人感
 人心而天下和平」,此其效也。



 及乎重圍既解,諸路稍通,賦稅漸臻,貢獻繼至,乃於行宮外廡之下,別置瓊林、大盈之司。未賞功勞,遽私賄玩,甚沮惟新之望,頗攜死義之心,於是輿誦興譏,而軍士始怨矣。財聚人散,不其然乎!旋屬蟊賊內興,翠華南狩,奉天所積財貨,悉復殲於亂軍。即遷岷、梁,日不暇給,獨憑大順,遂復皇都。是知天子者,以得人為資,以蓄義為富,人茍歸附,何患蔑資?義茍修崇,何憂不富?豈在貯之內府,方為己有哉!故藏於
 天下者,天子之富也;藏於境內者,諸侯之富也;藏於囷倉篋櫝者,農夫、商賈之富也。奈何以天子之貴,海內之富,面猥行諸侯之棄德,守農商之鄙業哉!陛下若謂厚取可以恢武功,則建中之取既無成矣;若謂多積可以為己有,則建中之積又不在矣;若謂徇欲不足傷理化,則建中之失傷已甚矣;若謂斂怨不足致危亡,則建中之亂危亦至矣!然而遽能靖滔天之禍,成中興之功者,良以陛下有側身修勵之志,有罪己悔懼之辭,罷息誅
 求,敦尚節儉,渙發大號,與人更新;故靈祗感陛下之誠,臣庶感陛下之意,釋憾回慮,化危為安。陛下亦當為宗廟社稷建不拔之永圖,為子孫黎元立可久之休業,懲前事徇欲之失,復日新盛德之言;豈宜更縱憸邪,復行克暴,事之追悔,其可再乎!



 臣又竊慮陛下納彼盜言,墮其奸計,以為搏噬拏攫,怨集有司,積聚豐盈,利歸君上,是又大謬,所宜慎思。夫人主昏明,系於所任,咎繇、夔、契之道長,而虞舜享浚哲之名;皇甫、棸、楀之嬖行,而周厲
 嬰顛覆之禍。自古何嘗有小人柄用,而災患不及邦國者乎!譬猶操兵以刃人,天下不委罪於兵而委罪於所操之主;畜蠱以殃物,天下不歸咎於蠱而歸咎於所畜之家;理有必然,不可不察。



 臣伏慮陛下以延齡之進,獨出宸衷,延齡之言,多順聖旨,今若以罪置闢,則似為眾所擠,故欲保持,用彰堅斷。若然,陛下與人終始之意則美矣。其於改過勿吝、去邪勿疑之道,或未盡善。今希旨自默,浸以成風,獎之使言,猶懼不既,若又阻抑,誰當貢
 誠?或恐未亮斯言,請以一事為證。只如延齡兇妄,流布寰區,上自公卿近臣,下迨輿臺賤品,喧喧談議,億萬為徒,能以上言,其人有幾?陛下誠令親信博採輿詞,參較比來所聞,足鑒人間情偽。



 臣以卑鄙,位當臺衡,既極崇高,又承渥澤。豈不知觀時附會,足保舊恩,隨眾沉浮,免貽厚責。謝病黜退,獲知幾之名;黨奸茍容,無見嫉之患。何急自苦,獨當豺狼,上違歡情,下餌讒口。良以內顧庸昧,一無所堪,夙蒙眷知,唯以誠直,綢繆帷扆,一紀於茲,
 聖慈既襎此見容,愚臣亦以此自負。從陛下歷播遷之危,睹陛下致興復之難,至今追思,猶為心悸;所以畏覆車而駭慮,懼毀室而悲鳴,蓋情激於衷,雖欲罷而不能自默也!因事陳請,雖已頻煩,天聽尚高,未垂諒察,輒申悃款,以極愚誠。憂深故語煩,意懇故詞切,以微臣自固之謀則過,於陛下慮患之計則忠。糜軀奉君,所不敢避;沽名衒直,亦不忍為。願回睿聰,為國熟慮,社稷是賴,豈唯微臣。



 書奏,德宗不悅,待延齡益厚。時鹽鐵轉運使張
 滂、京兆尹李充、司農卿李銛,以事相關,皆證延齡矯妄。德宗罷陸贄知政事,為太子賓客;滂、充、銛悉罷職左遷。



 十一年春暮,上數畋於苑中,時久旱,人情憂惴,延齡遽上疏曰:「陸贄、李充等失權,心懷怨望,今專大言於眾曰:『天下炎旱,人庶流亡,度支多欠闕諸軍糧草。』以激怒群情。」後數日,上又幸苑中,適會神策軍人訴度支欠廄馬芻草。上思延齡言,即時回駕,下詔斥逐贄、充、滂、銛等,朝廷中外惴恐。延齡謀害在朝正直之士,會諫議大
 夫陽城等伏閣切諫,事遂且止。贄、充等雖已貶黜,延齡憾之未已,乃掩捕李充腹心吏張忠,捶掠楚痛,令為之詞,云「前後隱沒官錢五十餘萬貫,米麥稱是,其錢物多結托權勢,充妻常於犢車中將金寶繒帛遺陸贄妻。」忠不勝楚毒,並依延齡教抑之辭,具于款占。忠妻、母於光順門投匭訴冤,詔御史臺推問,一宿得其實狀,事皆虛,乃釋忠。延齡又奏京兆府妄破用錢穀,請令比部勾覆,以比部郎中崔元嘗為陸贄所黜故也。及崔元勾覆錢穀,
 又無交涉。延齡既銳意以苛刻剝下附上為功,每奏對際,皆恣騁詭怪虛妄,他人莫敢言者,延齡言之不疑,亦人之所未嘗聞。德宗頗知其誕妄,但以其敢言無隱,且欲訪聞外事,故斷意用之。延齡恃之,謂必得宰相,尤好慢罵,毀詆朝臣,班行為之側目。及臥病,載度支官物置於私家,亦無敢言者。貞元十二年卒,時年六十九。延齡死,中外相賀,唯德宗悼惜不已,冊贈太子少保。



 韋渠牟,京兆萬年人。六代祖範,魏西陽太守,後周封郿
 城公。渠牟少慧悟,涉覽經史。初為道士,後為僧。興元中,韓滉鎮浙西,奏授試秘書郎,累轉四門博士。



 貞元十二年四月,德宗誕日,御麟德殿,召給事中徐岱、兵部郎中趙需、禮部郎中許孟容與渠牟及道士萬參成、沙門譚延等十二人,講論儒、道、釋三教。渠牟枝詞游說,捷口水注;上謂其講耨有素,聽之意動。數日,轉秘書郎,奏詩七十韻,旬日,遷右補闕、內供奉,僚列初不有之。在延英既對宰相,多使中貴人召渠牟於官次,同輩始注目矣。歲
 終,遷右諫議大夫。時延英對秉政賦之臣,晝漏率下二三刻為常,渠牟奏事,率漏下五六刻,上笑語款狎,往往外聞。渠牟形神佻躁,無士君子器,志向不根道德,眾雅知不能以正道開悟上意。



 陸贄免相後,上躬親庶政,不復委成宰相,廟堂備員,行文書而已。除守宰、御史,皆帝自選擇。然居深宮,所狎而取信者裴延齡、李齊運、王紹、李實、韋執誼洎渠牟,皆權傾相府。延齡、李實,奸欺多端,甚傷國體;紹無所發明;而渠牟名素輕,頗張恩勢以招
 趨向者,門庭填委。茅山處士崔芊徵至闕下,鄭隨自山人再至補闕,馮伉自醴泉令為給事中、皇太子侍讀,皆渠牟延薦之。上既偏有所聽,浮薄率背本衒進,不復藏器蘊德,皆奔馳請謁,剚蹄甘辭以附渠牟。居無何,遷太府卿,賜金紫,又轉太常卿。貞元十七年卒,時年五十三,贈刑部尚書,仍謚曰忠。



 李齊運者,蔣王惲之孫也。解褐寧王府東閣祭酒,七遷至監察御史。江淮都統李峘闢為幕府,累轉工部郎中,
 為長安縣令,職事修理。歷京兆少尹、陜府長史。建中末,改河中尹、晉絳慈隰觀察使。時李懷光自山東卷甲奔難,晝夜倍道,比至河中,力疲,休兵三日,齊運傾力犒設,軍人皆悅。懷光既反,驅兵還保河中,齊運不能敵,棄城而走,除為京兆尹,兼御史大夫。時賊據京城,李晟軍東渭橋,齊運擾攘之中,徵募工役,版築城壘,飛芻輓粟以應晟。收復之際,頗有力焉。



 貞元中,蝗旱方熾,齊運無政術,乃以韓洄代之,改宗正卿,兼御史大夫、閑廄宮苑使。
 改檢校禮部尚書,兼殿中監。尋正拜禮部尚書,兼殿中監使如故。其後十餘歲,宰臣內殿對後,齊運常次進,貢其計慮,以決群議。齊運無學術,不知大體,但甘言取信而已。薦李錡為浙西觀察使,受賂數十萬計。舉李詞為湖州刺史,既而邑人告其贓犯,上以齊運故,不問而遣之。齊運被疾,歲餘不能朝請,朝廷除授,往往降中人就宅咨決。末以妾衛氏為正室,身為禮部尚書,冕服以行其禮,人士嗤誚。貞元十二年卒,時年七十二,贈尚書左
 僕射。



 李實者,道王元慶玄孫。以廕入仕,六轉至潭州司馬。洪州節度使、嗣曹王皋闢為判官,遷蘄州刺史。皋為山南東道節度使,復用為節度判官、檢校太子賓客、員外郎。皋卒,新帥未至,實知留後,刻薄軍士衣食,軍士怨叛,謀殺之,實夜縋城而出,歸詣京師,用為司農少卿,加檢校工部尚書、司農卿。



 貞元十九年,為京兆尹,卿及兼官如故。尋封嗣道王。自為京尹,恃寵強愎,不顧文法,人皆側
 目。二十年春夏旱,關中大歉,實為政猛暴,方務聚斂進奉,以固恩顧,百姓所訴,一不介意。因入對,德宗問人疾苦,實奏曰:「今年雖旱,穀田甚好。」由是租稅皆不免,人窮無告,乃徹屋瓦木,賣麥苗以供賦斂。優人成輔端因戲作語,為秦民艱苦之狀云:「秦城城池二百年,何期如此賤田園,一頃麥苗五碩米,三間堂屋二千錢。」凡如此語有數十篇。實聞之怒,言輔端誹謗國政,德宗遽令決殺,當時言者曰:「瞽誦箴諫,取其詼諧以托諷諫,優伶舊事
 也。設謗木,採芻蕘,本欲達下情,存諷議,輔端不可加罪。」德宗亦深悔,京師無不切齒以怒實。



 故事,府官避臺官。實常遇侍御史王播於道,實不肯避,導從如常。播詰其從者,實怒,奏播為三原令,謝之日,庭詬之。陵轢公卿百執事,隨其喜怒,誣奏遷逐者相繼,朝士畏而惡之。又誣奏萬年令李眾,貶虔州司馬,奏虞部員外郎房啟代眾,升黜如其意,怙勢之色,謷然在眉睫間。故事,吏部將奏科目,奧密,朝官不通書問,而實身詣選曹迫趙宗儒,且
 以勢恐之。前歲,權德輿為禮部侍郎,實托私薦士,不能如意,後遂大錄二十人迫德輿曰:「可依此第之;不爾,必出外官,悔無及也。」德輿雖不從,然頗懼其誣奏。



 二十一年,有詔蠲畿內逋租,實違詔征之,百姓大困,官吏多遭笞罰,剝割掊斂,聚錢三十萬貫,胥吏或犯者,即按之。有乞丐絲發固死;無者,且曰「死亦不屈」,亦杖殺之。京帥貴賤同苦其暴虐。順宗在諒陰逾月,實斃人於府者十數,遂議逐之,乃貶通州長史。制出,市人皆袖瓦石投其首;
 實知之,由月營門自苑西出,人人相賀。後遇赦量移虢州,在道卒。



 韋執誼者,京兆人。父浼,官卑。執誼幼聰俊有才,進士擢第,應制策高等,拜右拾遺,召入翰林為學士,年才二十餘。德宗尤寵異,相與唱和歌詩,與裴延齡、韋渠牟等出入禁中,略備顧問。德宗載誕日,皇太子獻佛像,德宗命執誼為畫像贊,上令太子賜執誼縑帛以酬之。執誼至東宮謝太子,卒然無以藉言,太子因曰:「學士知王叔文
 乎?彼偉才也。」執誼因是與叔文交甚密。俄丁母憂,服闋,起為南宮郎。德宗時,召入禁中。



 初,貞元十九年,補闕張正一因上書言事得召見,王仲舒、韋成季、劉伯芻、裴茝、常仲孺、呂洞等以嘗同官相善,以正一得召見,偕往賀之。或告執誼曰:「正一等上疏論君與王叔文朋黨事。」執誼信然之,因召對,奏曰:「韋成季等朋聚覬望。」德宗令金吾伺之,得其相過從飲食數度,於是盡逐成季等六七人,當時莫測其由。



 及順宗即位,久疾不任朝政,王叔文
 用事,乃用執誼為宰相,乃自朝議郎、吏部郎中、騎都尉賜緋魚袋,授尚書左丞、同平章事,仍賜金紫。叔文欲專政,故令執誼為宰相於外,己自專於內。執誼既為叔文引用,不敢負情,然迫於公議,時時立異,密令人謝叔文曰:「不敢負約為異,欲共成國家之事故也。」叔文詬怒,遂成仇怨;執誼既因之得位,亦欲矛盾掩其跡。及憲宗受內禪,王伾、王叔文徒黨並逐,尚以執誼是宰相杜黃裳之婿,故數月後貶崖州司戶。初,執誼自卑官,常忌諱
 不欲人言嶺南州縣名。為郎官時,嘗與同舍詣職方觀圖,每至嶺南州,執誼遽命去之,閉目不視。及拜相,還所坐堂,見北壁有圖,不就省,七八日,試觀之,乃崖州圖也,以為不祥,甚惡之,不敢出口。及坐叔文之貶,果往崖州,卒於貶所。



 王叔文者,越州山陰人也。以棋待詔,粗知書,好言理道。德宗令直東宮。太子嘗與侍讀論政道,因言宮市之弊,太子曰:「寡人見上,當極言之。」諸生稱贊其美,叔文獨無
 言。罷坐,太子謂叔文曰:「向論宮市,君獨無言何也」?叔文曰:「皇太子之事上也,視膳問安之外,不合輒預外事。陛下在位歲久,如小人離間,謂殿下收取人情,則安能自解?」太子謝之曰:「茍無先生,安得聞此言?」由是重之,宮中之事,倚之裁決。每對太子言,則曰:「某可為相,某可為將,幸異日用之。」密結當代知名之士而欲僥幸速進者,與韋執誼、陸質、呂溫、李景儉、韓曄、韓泰、陳諫、柳宗元、劉禹錫等十數人,定為死交;而凌準,程異,又因其黨以進;籓
 鎮侯伯,亦有陰行賂遺請交者。



 德宗崩,已宣遺詔,時上寢疾久,不復關庶政,深居施簾帷,閹官李忠言、美人牛昭容侍左右,百官上議,自帷中可其奏。王伾常諭上屬意叔文,宮中諸黃門稍稍知之。其日,召自右銀臺門,居於翰林,為學士。叔文與吏部郎中韋執誼相善,請用為宰相。叔文因王伾,伾因李忠言,忠言因牛昭容,轉相結構。事下翰林,叔文定可否,宣於中書,俾執誼承奏於外。與韓泰、柳宗元、劉禹錫、陳諫、凌準、韓曄唱和,曰管,曰葛,
 曰伊,曰周,凡其黨僴然自得,謂天下無人。



 叔文賤時,每言錢穀為國大本,將可以盈縮兵賦,可操柄市士。叔文初入翰林,自蘇州司功為起居郎,俄兼充度支、鹽鐵副使,以杜佑領使,其實成於叔文。數月,轉尚書戶部侍郎,領使、學士如故。內官俱文珍惡其弄權,乃削去學士之職。制出,叔文大駭,謂人曰:「叔文須時至此商量公事,若不帶此職,無由入內。」王伾為之論請,乃許三、五日一入翰林,竟削內職。叔文始入內廷,陰構密命,機形不見,因
 騰口善惡進退之。人未窺其本,信為奇才。及司兩使利柄,齒於外朝,愚智同曰:「城狐山鬼,必夜號窟居以禍福人,亦神而畏之;一旦晝出路馳,無能必矣。」



 叔文在省署,不復舉其職事,引其黨與竊語,謀奪內官兵柄,乃以故將範希朝統京西北諸鎮行營兵馬使,韓泰副之。初,中人尚未悟,會邊上諸將各以狀辭中尉,且言方屬希朝,中人始悟兵柄為叔文所奪,中尉乃止諸鎮無以兵馬入。希朝、韓泰已至奉天,諸將不至,乃還。無幾,叔文母死。
 前一日,叔文置酒饌於翰林院,宴諸學士及內官李忠言、俱文珍、劉光奇等。中飲,叔文白諸人曰:「叔文母疾病,比來盡心戮力為國家事,不避好惡難易者,欲以報聖人之重知也。若一去此職,百謗斯至,誰肯助叔文一言者,望諸君開懷見察。」又曰:「羊士諤非毀叔文,欲杖殺之,而韋執誼懦不遂。叔文生平不識劉闢,乃以韋皋意求領三川,闢排門相干,欲執叔文手,豈非兇人耶!叔文已令掃木場,將斬之,韋執誼苦執不可。叔
 文無以對。



 叔文未欲立皇太子。順宗既久疾未平,群臣中外請立太子,既而詔下立廣陵王為太子,天下皆悅;叔文獨有憂色,而不敢言其事,但吟杜甫題諸葛亮祠堂詩末句云:「出師未捷身先死,長使英雄淚滿襟。」因歔欷泣下,人皆竊笑之。皇太子監國,貶為渝州司戶,明年誅之。



 王伾,杭州人。始為翰林侍書待詔,累遷至正議大夫、殿中丞、皇太子侍書。順宗即位,遷
 左散騎常侍,依前翰林待詔。



 伾闒茸,不如叔文,唯招賄賂,無大志,貌寢陋,吳語,素為太子之所褻狎;而叔文頗任氣自許,粗知書,好言事,順宗稍敬之,不得如伾出入無間。叔文入止翰林,而伾入至柿林院,見李忠言、牛昭容等。然各有所主:伾主往來傳授;王叔文主決斷;韋執誼為文誥;劉禹錫、陳諫、韓曄、韓泰、柳宗元、房啟、凌準等謀議唱和,採聽外事。而伾與叔文及諸朋黨之門,車馬填湊,而伾門尤盛,珍玩賂遺,歲時不絕。室中為無門大
 櫃,唯開一竅,足以受物,以藏金寶,其妻或寢臥於上。與叔文同貶開州司馬。



 王叔文最所重者,李景儉、呂溫。叔文用事時,景儉居喪於東都;呂溫使吐蕃,留半歲,叔文敗方歸。陸質為皇太子侍讀,尋卒。



 伾、叔文既逐,詔貶其黨韓曄饒州司馬,韓泰虔州司馬,陳諫臺州司馬,柳宗元永州司馬,劉禹錫朗州司馬,凌準連州司馬,程異郴州司馬,韋執誼崖州司馬。



 韓曄,宰相滉之族子,有俊才,依附韋執誼,累遷尚書司封郎中。叔文敗,貶池州刺史,
 尋改饒州司馬,量移汀州刺史,又轉永州卒。



 陳諫至叔文敗,已出為河中少尹,自臺州司馬量移封州刺史,轉通州卒。



 凌準,貞元二十年自浙東觀察判官、侍御史召入,王叔文與準有舊,引用為翰林學士,轉員外郎。坐叔文貶連州。準有史學,尚古文,撰《邠志》二卷。



 韓泰,貞元中累遷至戶部郎中,王叔文用為範希朝神策行營節度行軍司馬。泰最有籌畫,能決陰事,深為伾、叔文之所重,坐貶,自虔州司馬量移漳州刺史,遷郴州。



 柳宗元、劉禹
 錫自有傳。



 程異,京兆長安人。嘗侍父疾,鄉里以孝悌稱。明經及第,釋褐揚州海陵主簿。登《開元禮》科,授華州鄭縣尉。精於吏職,剖判無滯。杜確刺同州,帥河中,皆從為賓佐。



 貞元末,擢授監察御史,遷虞部員外郎,充鹽鐵轉運、揚子院留後。時王叔文用事,由逕放利者皆附之,異亦被引用。叔文敗,坐貶岳州刺史,改郴州司馬。元和初,鹽鐵使李巽薦異曉達錢穀,請棄瑕錄用,擢為侍御史,復為揚子
 留後,累檢校兵部郎中、淮南等五道兩稅使。異自悔前非,厲己竭節,江淮錢穀之弊,多所鏟革。入為太府少卿、太卿,轉衛尉卿,兼御史中丞,充鹽鐵轉運副使。



 時淮西用兵,國用不足,異使江表以調征賦,且諷有土者以饒羨入貢,至則不剝下,不浚財,經費以贏,人頗便之。由是專領鹽鐵轉運使、兼御史大夫。十三年九月,轉工部侍郎、同中書門下平章事,領使如故。議者以異起錢穀吏,一旦位冠百僚,人情大為不可。異自知叨據,以謙遜自牧,
 月餘日,不敢知印秉筆。異知西北邊軍政不理,建議置巡邊使,上問誰可使者,異請自行。議未決,無疾而卒,元和十四年四月也。贈左僕射,謚曰恭。異性廉約,歿官第,家無餘財,人士多之。



 皇甫鎛,安定朝那人。祖鄰幾,汝州刺史。父愉,常州刺史。鎛貞元初登進士第,登賢良文學制科,授監察御史。丁母憂,免喪,坐居喪時薄游,除詹事府司直。轉吏部員外郎、判南曹,凡三年,頗鈐制奸吏。改吏部郎中,三遷司農
 卿、兼御史中丞,賜金紫,判度支,俄拜戶部侍郎。時方討淮西,切於饋運,鎛勾剝嚴急,儲供辦集,益承寵遇,加兼御史大夫。



 十三年,與鹽鐵使程異同日以本官同平章事,領使如故。鎛雖有吏才,素無公望,特以聚斂媚上,刻削希恩。詔書既下,物情駭異,至於賈販無識,亦相嗤誚。宰相崔群、裴度以物議上聞,憲宗怒而不聽。度上疏乞罷知政事,因論之曰:



 臣日昨於延英陳乞,伏奉聖旨,未遂愚衷。竊以上古明王聖帝,致理興化,雖由元首,亦在
 股肱。所以述堯、舜之道,則言稷、契、皋、夔;紀太宗、玄宗之德,則言房、杜、姚、宋。自古至今,未有不任輔弼而能獨理天下者。況今天下,異於十年已前,方驅駕文武,廓清寇亂,建升平之業,十已得八九。然華夏安否,系於朝廷,朝廷輕重,在於宰相。如臣駑鈍,夙夜戰兢,常以為上有聖君,下無賢臣,不能增日月之明,廣天地之德。遂使每事皆勞聖心,所以平賊安人,費力如此,實由臣輩不稱所職。方期陛下博採物議,旁求人望,致之輔弼,責之化成;
 而乃忽取微人,列於重地,始則殿庭班列,相與驚駭,次則街衢市肆,相與笑呼。伏計遠近流聞,與京師無異。何者?天子如堂,宰臣如陛,陛高則堂高,陛卑則堂不得高矣,宰臣失人,則天子不得尊矣。



 伏以陛下睿哲文明,唯在所授,凡所閱視,洞達無遺。所以比來選任宰相,縱道不周物,才不濟時,公望所歸,皆有可取。況皇甫鎛自掌財賦,唯事割剝,以苛為察,以刻為明。自京北、京西城鎮及百司並遠近州府,應是仰給度支之處,無不苦口切
 齒,願食其肉;猶賴臣等每加勸誡,或為奏論,庶事之中,抑令通濟。比者淮西諸軍糧料,所破五成錢,其實只與一成、兩成,士卒怨怒,皆欲離叛。臣到行營,方且慰喻,直其遷延不進,供軍漸難,俱能前行,必有優賞,以此約定,然後切勒供軍官,且支九月一日兩成已上錢,俱容努力,方將小安,不然必有潰散。今舊兵悉向淄青討伐,忽聞此人入相,則必相與驚擾,以為更有前時之事,則無告訴之憂。雖侵刻不少,然漏落亦多,所以罷兵之後,經
 費錢數一千三十萬貫,此事猶可。直以性惟狡詐,言不誠實,朝三暮四,天下共知,惟能上惑聖聰,足見奸邪之極。程異雖人品凡俗,然心事和平,處之煩劇,或亦得力,但升之相位,便在公卿之上,實亦非宜。如皇甫鎛,天下之人,怨入骨髓,陛下今日收為股肱,列在臺鼎,切恐不可,伏惟圖之。倘陛下納臣懇款,速賜移易,以副天下之望,則天下幸甚。伏聞李修疾病,亦求入來,如浙西觀察使,且與亦得。



 臣知一言出口,必犯天威,但使言行,甘心
 獲戾。今者臣若不退,天下之人謂臣有負恩寵;今退毀未許,言又不聽,如火燒心,若箭攢體。臣自無足惜,惜陛下今日事勢。何者?淮西蕩定,河北咸寧,承宗斂手削地,程權束身赴闕,韓弘輿疾討賊,此豈京師氣力能制其命,祗是朝廷處置能服其心。今既開中興,再造區夏,陛下何忍卻自破除,使億萬之眾離心,四方諸侯解體?凡百君子,皆欲慟哭。況陛下任臣之意,豈比常人;臣事陛下之心,敢同眾士?所以昧死重封以聞,如不足觀,臣當
 引領受責。陛下引一市肆商徒,與臣同列,在臣亦有何損,陛下實有所傷,不勝憤懣惶恐之至。



 時憲宗以世道漸平,欲肆意娛樂,池臺館宇,稍增崇飾,而異、鎛探知上旨,數貢羨餘,以備經構,故帝獨排物議相之;見裴度疏,以為朋黨,竟不省覽。鎛知公議不可,益以巧媚自固,奏減內外官俸錢以贍國用;敕下,給事中崔祐封還詔書,其事方罷。時內出積年庫物付度支估價,例皆陳朽,鎛盡以善價買之,以給邊軍。羅縠繒彩,觸風斷裂,隨手散
 壞,軍士怨怒,皆聚而焚之。裴度奏事,因言邊軍焚賜之意,鎛因引其足奏曰:「此靴乃內庫出者,臣以俸二千買之,堅韌可以久服,所言不可用,皆詐也。」帝以為然,由是鎛益無忌憚。裴度有用兵伐叛之功,鎛心嫉之,與宰相李逢吉、令狐楚合勢擠度出鎮太原。崔群有公望,為搢紳所重,屢言時政之弊,鎛惡之,因議憲宗尊號,乃奏曰:「昨群臣議上徽號,崔群於陛下惜『孝德』兩字。」憲宗怒,黜群為湖南觀察使。又與金吾將軍李道古葉為奸謀,
 薦引方士柳泌、僧大通,言可致長生。中尉吐突承璀恩寵莫二,鎛厚賂結其歡心,故及相位。



 穆宗在東宮,備聞鎛之奸邪,及居諒陰,聽政之日,詔:「皇甫鎛器本凡近,性惟險狹,行靡所顧,文無可觀,雖早踐朝倫,而素乖公望。自掌邦計,屬當軍興,以剝下為徇公,既鼓眾怒;以矯跡為孤立,用塞人言。洎塵臺司,益蠹時政,不知經國之大體,不慮安邊之遠圖,三軍多凍餒之憂,百姓深凋瘵之弊。事皆罔蔽,言悉虛誣,遠近咸知,朝野同怨。而又恣求
 方士,上惑先朝,潛通奸人,罪在難舍。合加竄殛,以正刑章,俾黜遐荒,尚存寬典。」又詔曰:「山人柳泌輒懷左道,上惑先朝,固求牧人,貴欲疑眾,自知虛誕,仍便奔逃。僧大通醫方不精,藥術皆妄。既延禍釁,俱是奸邪,邦國固有常刑,人神所宜共棄,宜付京兆府決重杖一頓處死。」



 柳泌本曰楊仁力,少習醫術,言多誕妄。李道古奸回巧宦,與泌密謀求進,言之於皇甫鎛,因徵入禁中。自云能致靈藥,言:「天臺山多靈草,君仙所會,臣嘗知之,而力不能
 致。願為天臺長吏,因以求之。」起徒步為臺州刺史,仍賜金紫。諫官論奏曰:「列聖亦有好方士者,亦與官號,未嘗令賦政臨民。」憲宗曰:「煩一郡之力而致神仙長年,臣子於君父何愛焉!」由是莫取有言者。裴潾以極言被黜。泌到天臺,驅役吏民於山谷間,聲言採藥,鞭笞躁急。歲餘一無所得,懼詐發獲罪,舉家入山谷。浙東觀察使追捕,送於京師,鎛與李道古懇保證之,必能可致靈藥,乃待詔翰林院。憲宗服泌藥,日益煩躁,喜怒不常,內官懼
 非罪見戮,遂為弒逆。大通自云壽一百五十歲,久得藥力。又有田佐元者,鳳翔虢人,自言有奇術,能變瓦礫為金,白衣授虢縣令。初,柳泌系京兆府,獄吏叱之曰:「何苦作此虛矯?」泌曰:「吾本無此心,是李道古教我,且云壽四百歲。」府吏防虞周密,恐其隱化;及解衣就誅,一無變異,但灸灼之瘢痕浹身而已。鎛卒於貶所。



 鎛弟鏞,端士也。亦進士擢第,累歷宣歙、鳳翔使府從事,入為殿中侍御史,轉比部員外郎、河南縣令、都官郎中、河南少尹。時鎛為
 宰相,領度支,恩寵殊異。鏞惡其太盛,每弟兄宴語,即極言之,鎛頗不悅。乃求為分司,除右庶子。及鎛獲罪,朝廷素知鏞有先見之明,不之罪,徵為國子祭酒,改太子賓客、秘書監。開成初,除太子少保分司,卒年四十九。鏞能文,尤工詩什,樂道自怡,不屑世務,當時名士皆與之交。有集十八卷,著《性言》十四篇。



 史臣曰:奸邪害正,自古有之;而矯誕無忌,妒賢傷善,未有如延齡、皇甫之甚也。臣每讀陸丞相論延齡疏,未嘗
 不泣下沾衿,其守正效忠,為宗社大計,非端士益友,安能感激犯難如此?異哉德宗之為人主也,忠良不用,讒慝是崇,乃至身播國屯,幾將覆滅,尚獨保延齡之是,不悟盧杞之非,悲夫!執誼、叔文,乘時多僻,而欲斡運六合,斟酌萬幾;劉、柳諸生,逐臭市利,何狂妄之甚也!章武雄材睿斷,翦削厲階;洎逐群、度而相異、鎛,蓋季年之妖惑也,夫何言哉!



 贊曰:貞元之風,好佞惡忠。齡、鎛害善,為國蠹蟲。裴、陸獻
 替,嫉惡如風。天聽匪諶,吾道斯窮。



\end{pinyinscope}