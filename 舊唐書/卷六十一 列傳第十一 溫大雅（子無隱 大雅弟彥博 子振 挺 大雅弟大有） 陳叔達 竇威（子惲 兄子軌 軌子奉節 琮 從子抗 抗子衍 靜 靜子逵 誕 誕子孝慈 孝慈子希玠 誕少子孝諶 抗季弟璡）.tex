\article{卷六十一 列傳第十一 溫大雅(子無隱 大雅弟彥博 子振 挺 大雅弟大有) 陳叔達 竇威(子惲 兄子軌 軌子奉節 琮 從子抗 抗子衍 靜 靜子逵 誕 誕子孝慈 孝慈子希玠 誕少子孝諶 抗季弟璡)}

\begin{pinyinscope}

 ○溫大雅子無隱大雅弟彥博子振挺大雅弟大有陳叔達竇威子惲兄子軌軌子奉節琮從子抗抗子衍靜靜子逵誕誕子孝慈孝慈子希玠誕少子孝諶抗季弟璡



 溫大雅,字彥弘,太原祁人也。父君悠,北齊文林館學士,
 隋泗州司馬。大業末,為司隸從事,見隋政日亂,謝病而歸。大雅性至孝,少好學,以才辯知名。仕隋東宮學士、長安縣尉,以父憂去職。後以天下方亂,不求仕進。高祖鎮太原,甚禮之。義兵起,引為大將軍府記室參軍,專掌文翰。禪代之際,與司錄竇威、主簿陳叔達參定禮儀。武德元年,歷遷黃門侍郎。弟彥博,為中書侍郎;對居近密,議者榮之。高祖從容謂曰:「我起義晉陽,為卿一門耳。」尋轉工部,進拜陜東道大行臺工部尚書。太宗以隱太子、巢
 刺王之故,令大雅鎮洛陽以俟變。大雅數陳秘策,甚蒙嘉賞。太宗即位,累轉禮部尚書,封黎國公。大雅將改葬其祖父,筮者曰:「葬於此地,害兄而福弟。」大雅曰:「若得家弟永康,我將含笑入地。」葬訖,歲餘而卒,謚曰孝。撰《創業起居注》三卷。永徽五年,贈尚書右僕射。



 子無隱,官至工部侍郎。大雅弟彥博。



 彥博幼聰悟,有口辯,涉獵書記。初,其父友薛道衡、李綱常見彥博兄弟三人,咸嘆異曰:「皆卿相才也。」開皇末,為州牧秦孝王俊所薦,授文林郎,直
 內史省,轉通直謁者。及隋亂,幽州總管羅藝引為司馬。藝以幽州歸國,彥博贊成其事,授幽州總管府長史。未幾,徵為中書舍人,俄遷中書侍郎,封西河郡公。時高麗遣使貢方物,高祖謂群臣曰:「名實之間,理須相副。高麗稱臣於隋,終拒煬帝,此亦何臣之有?朕敬於萬物,不欲驕貴,但據土宇,務共安人,何必令其稱臣以自尊大?可即為詔,述朕此懷也。」彥博進曰:「遼東之地,周為箕子之國,漢家之玄菟郡耳。魏、晉已前,近在提封之內,不可許
 以不臣。若與高麗抗禮,則四夷何以瞻仰?且中國之於夷狄,猶太陽之比列星,理無降尊,俯同夷貊。」高祖乃止。其年,突厥入寇,命右衛大將軍張瑾為並州道行軍總管,出拒之,以彥博為行軍長史。與虜戰於太谷,軍敗,彥博沒於虜庭。突厥以其近臣,苦問以國家虛實及兵馬多少,彥博固不肯言。頡利怒,遷於陰山苦塞之地。太宗即位,突厥送款,始征彥博還朝,授雍州治中,尋檢校吏部侍郎。彥博意有沙汰,多所損抑,而退者不伏,囂訟盈
 庭。彥博惟騁辭辯,與之相詰,終日喧擾,頗為識者所嗤。復拜中書侍郎,兼太子右庶子。貞觀二年,遷御史大夫,仍檢校中書侍郎事。彥博善於宣吐,每奉使入朝,詔問四方風俗,承受綸言,有若成誦。聲韻高朗,響溢殿庭,進止雍容,觀者拭目。四年,遷中書令,進爵虞國公。高祖常宴朝臣,詔太宗諭旨,既而顧謂近臣曰:「何如溫彥博?」其見重如此。



 初,突厥之降也,詔議安邊之術。朝士多言:「突厥恃強,擾亂中國,為日久矣。今天實喪之,窮來歸我,本
 非慕義之心也。因其歸命,分其種落,俘之河南,散屬州縣,各使耕田,變其風俗。百萬胡虜,可得化而為漢,則中國有加戶之利,塞北常空矣。」惟彥博議曰:「漢建武時,置降匈奴於五原塞下,全其部落,得為捍蔽,又不離其土俗,因而撫之。一則實空虛之地,二則示無猜之心。若遣向西南,則乖物性,故非含育之道也。」太宗從之,遂處降人於朔方之地,其入居長安者近且萬家。議者尤為不便,欲建突厥國於河外。彥博又執奏曰:「既已納之,無故
 遣去,深為可惜。」與魏徵等爭論,數年不決。十年,遷尚書右僕射。明年薨,年六十四。彥博自掌知機務,即杜絕賓客,國之利害,知無不言,太宗以是嘉之。及薨,謂侍臣曰:「彥博以憂國之故,勞精竭神,我見其不逮,已二年矣。恨不縱其閑逸,致夭性靈。」彥博家無正寢,及卒之日,殯於別室,太宗命有司為造堂焉。贈特進,謚曰恭,陪葬昭陵。



 子振,少有雅望,官至太子舍人,居喪以毀卒。振弟挺,尚高祖女千金公主,官至延州刺史。



 大雅弟大有,字彥將,
 性端謹,少以學行稱。隋仁壽中,尚書右丞李綱表薦之,授羽騎尉。尋丁憂去,職歸鄉里。義旗初舉,高祖引為太原令。從太宗擊西河,高祖謂曰:「士馬尚少,要資經略,以卿參謀軍事,其善建功名也!事之成敗,當以此行卜之。若克西河,帝業成矣。」及破西河而還,復以本官攝大將軍府記室,與兄大雅共掌機密。大有以昆季同在機務,意不自安,固請他職。高祖曰:「我虛心相待,不以為疑,卿何自疑也?」大有雖應命,然每退讓,遠避機權,僚列以此
 多之。武德元年,累轉中書侍郎。會卒,高祖甚傷惜之,贈鴻臚卿。初,大雅在隋,與顏思魯俱在東宮,彥博與思魯弟愍楚同直內史省,彥將與愍楚弟游秦典校秘閣。二家兄弟,各為一時人物之選。少時學業,顏氏為優;其後職位,溫氏為盛。



 陳叔達,字子聰,陳宣帝第十六子也。善容止,頗有才學,在陳封義陽王。年十餘歲,嘗侍宴,賦詩十韻,援筆便就,僕射徐陵甚奇之。歷侍中、丹陽尹、都官尚書。入隋,久不
 得調。大業中,拜內史舍人,出為絳郡通守。義師至絳郡,叔達以郡歸款,授丞相府主簿,封漢東郡公。與記室溫大雅同掌機密,軍書、赦令及禪代文誥,多叔達所為。武德元年,授黃門侍郎。二年,兼納言。四年,拜侍中。叔達明辯,善容止,每有敷奏,搢紳莫不屬目。江南名士薄游長安者,多為薦拔。五年,進封江國公。嘗賜食於御前,得蒲萄,執而不食。高祖問其故,對曰:「臣母患口乾,求之不能致,欲歸以遺母。」高祖喟然流涕曰:「卿有母遺乎!」因賜物
 三百段。貞觀初,加授光祿大夫。尋坐與蕭瑀對御忿爭免官。未幾,丁母憂。叔達先有疾,太宗慮其危殆,遣使禁絕吊賓。服闋,授遂州都督,以疾不行。久之,拜禮部尚書。建成、元吉嫉害太宗,陰行譖毀,高祖惑其言,將有貶責,叔達固諫乃止。至是太宗勞之曰:「武德時,危難潛構,知公有讜言,今之此拜,有以相答。」叔達謝曰:「此不獨為陛下,社稷計耳。」後坐閨庭不理,為憲司所劾。朝廷惜其名臣,不欲彰其罪,聽以散秩歸第。九年卒,謚曰繆。後贈戶
 部尚書,改謚曰忠。有集十五卷。



 竇威,字文蔚,扶風平陸人,太穆皇后從父兄也。父熾,隋太傅。威家世勛貴,諸昆弟並尚武藝,而威耽玩文史,介然自守。諸兄哂之,謂為「書癡」。隋內史令李德林舉秀異,射策甲科,拜秘書郎。秩滿當遷,而固守不調,在秘書十餘歲,其學業益廣。時諸兄並以軍功致仕通顯,交結豪貴,賓客盈門,而威職掌閑散。諸兄更謂威曰:「昔孔子積學成聖,猶狼狽當時,棲遲若此,汝效此道,復欲何求?名
 位不達,固其宜矣。」威笑而不答。久之,蜀王秀闢為記室,以秀行事多不法,稱疾還田里。及秀廢黜,府僚多獲罪,唯威以先見保全。大業四年,累遷內史舍人,以數陳得失忤旨,轉考功郎中,後坐事免,歸京師。高祖入關,召補大丞相府司錄參軍。時軍旅草創,五禮曠墜。威既博物,多識舊儀,朝章國典,皆其所定,禪代文翰多參預焉。高祖常謂裴寂曰:「叔孫通不能加也。」武德元年,拜內史令。威奏議雍容,多引古為諭,高祖甚親重之,或引入臥內,
 常為膝席。又嘗謂曰:「昔周朝有八柱國之貴,吾與公家咸登此職。今我已為天子,公為內史令,本同末異,乃不平矣。」威謝曰:「臣家昔在漢朝,再為外戚,至於後魏,三處外家,陛下隆興,復出皇后。臣又階緣戚里,位忝鳳池,自惟叨濫,曉夕兢懼。」高祖笑曰:「比見關東人與崔、盧為婚,猶自矜伐,公代為帝戚,不亦貴乎!」及寢疾,高祖自往臨問。尋卒,家無餘財,遺令薄葬。謚曰靖,贈同州刺史,追封延安郡公。葬日,詔太子及百官並出臨送。有文集十卷。



 子惲嗣,官至岐州刺史。威兄子軌,從兄子抗,並知名。



 軌,字士則,周雍州牧、酂國公恭之子也。隋大業中,為資陽郡東曹掾,後去官歸於家。義兵起,軌聚眾千餘人,迎謁於長春宮。高祖見之,大悅,降席握手,語及平生,賜良馬十匹,使掠地渭南。軌先下永豐倉,收兵得五千人。從平京城,封贊皇縣公,拜大丞相諮議參軍。時稽胡賊五萬餘人掠宜春,軌討之。行次黃欽山,與賊相遇,賊乘高縱火,王師稍卻。軌斬其部將十四人,拔隊中小帥以代之。
 軌自率數百騎殿於軍後,令之曰:「聞鼓聲有不進者,自後斬之。」既聞鼓,士卒爭先赴敵,賊射之,不能止,因大破之,斬首千餘級,虜男女二萬口。武德元年,授太子詹事。會赤排羌作亂,與薛舉叛將鐘俱仇同寇漢中。拜軌秦州總管,與賊連戰皆捷,餘黨悉降。進封酂國公。三年,遷益州道行臺左僕射,許以便宜從事。屬黨項寇松州,詔軌援之,又令扶州刺史蔣善合與軌連勢。時黨項引吐谷渾之眾,其鋒甚銳。軌師未至,善合先期至鉗川,遇賊
 力戰,走之。軌復軍於臨洮,進擊左封,破其部眾。尋令率所部兵從太宗討王世充於洛陽。四年,還益州。時蜀土寇往往聚結,悉討平之。軌每臨戎對寇,或經旬月,身不解甲。其部眾無貴賤少長,不恭命即立斬之。每日吏士多被鞭撻,流血滿庭,見者莫不重足股慄。軌初入蜀,將其甥以為心腹,嘗夜出,呼之不以時至,怒而斬之。每誡家僮不得出外。嘗遣奴就官廚取漿而悔之,謂奴曰:「我誠使汝,要當斬汝頭以明法耳!」遣其部將收奴斬之。其
 奴稱冤,監刑者猶豫未決,軌怒,俱斬之。行臺郎中趙弘安,知名士也,軌動輒榜箠,歲至數百。後徵入朝,賜坐御榻,軌容儀不肅,又坐而對詔,高祖大怒,因謂曰:「公之入蜀,車騎、驃騎從者二十人,為公所斬略盡,我隴種車騎,未足給公。」詔下獄,俄而釋之,還鎮益州。軌與行臺尚書韋雲起、郭行方素不協。及隱太子誅,有詔下益州,軌藏諸懷中,雲起問曰:「詔書安在?」軌不之示,但曰:「卿欲反矣!」執而殺之。行方大懼,奔於京師,軌追斬不及。是歲,行臺
 廢,即授益州大都督,加食邑六百戶。貞觀元年,徵授右衛大將軍。二年,出為洛州都督。洛陽因隋末喪亂,人多浮偽。軌並遣務農,各令屬縣有游手怠惰者皆按之。由是人吏懾憚,風化整肅。四年,卒官,贈並州都督。



 子奉節嗣,尚高祖永嘉公主,歷左衛將軍、秦州都督。



 軌弟琮,亦有武干,隋左親衛。大業末,犯法,亡命奔太原,依於高祖。琮與太宗有宿憾,每自疑。太宗方搜羅英傑,降禮納之,出入臥內,其意乃解。及將義舉,琮協贊大謀。大將軍府
 建,為統軍。從平西河,破霍邑,拜金紫光祿大夫、扶風郡公。尋從劉文靜擊屈突通於潼關,通遣裨將桑顯和來逼文靜,義軍不利。琮與段志玄等力戰久之,隋軍大潰,通遁走。琮率輕騎追至稠桑,獲通而返。進兵東略,下陜縣,拔太原倉。拜右領軍大將軍,賜物五百段。時隋河陽都尉獨孤武潛謀歸國,乃令琮以步騎一萬自柏崖道應接之。遲留不進,武見殺,坐是除名。武德初,以元謀勛特恕一死,拜右屯衛大將軍,復轉右領軍大將軍。時將
 圖洛陽,遣琮留守陜城,以督糧運。王世充遣其驍將羅士信來斷糧道,琮潛使人說以利害,士信遂帥眾降。及從平東都,賞物一千四百段。後以本官檢校晉州總管。尋從隱太子討平劉黑闥,以功封譙國公,賞黃金五十斤。未幾而卒。高祖以佐命之舊,甚悼之,贈左衛大將軍,謚曰敬。永徽五年,重贈特進。



 抗,字道生,太穆皇后之從兄也,隋洛州總管、陳國公榮之子也。母,隋文帝萬安公主。抗在隋以帝甥甚見崇寵。少入太學,略涉書史,釋褐
 千牛備身、儀同三司。屬其父寢疾,抗躬親扶侍,衣不解帶者五十餘日。及居喪,哀毀過禮。後襲爵陳國公,累轉梁州刺史。將之官,隋文帝幸其第,命抗及公主酣宴,如家人之禮,賞賜極厚。母卒,號慟絕而復蘇者數焉,文帝令宮人至第,節其哭泣。歲餘,起為岐州刺史,轉幽州總管,政並以寬惠聞。及漢王諒作亂,煬帝恐其為變,遣李子雄馳往代之。子雄因言抗得諒書而不奏,按之無驗,以疑貳除名。抗與高祖少相親狎,及楊玄感作亂,高祖
 統兵隴右,抗言於高祖曰:「玄感抑為發蹤耳!李氏有名圖籙,可乘其便,天之所啟也。」高祖曰:「無為禍始,何言之妄也!」大業末,抗於靈武巡長城以伺盜賊,及聞高祖定京城,抗對眾而忭曰:「此吾家妹婿也,豁達有大度,真撥亂之主矣!」因歸長安。高祖見之大悅,握手引坐曰:「李氏竟能成事,何如?」因縱酒為樂。尋拜將作大匠。武德元年,以本官兼納言。高祖聽朝,或升御坐,退朝之後,延入臥內,命之舍敬,縱酒談謔,敦平生之款。常侍宴移時,或留
 宿禁內。高祖每呼為兄而不名也,宮內咸稱為舅。常陪侍游宴,不知朝務。轉左武候大將軍,領左右千牛備身大將軍。尋從太宗平薛舉,勛居第一。四年,又從征王世充。及東都平,冊勛太廟者九人,抗與從弟軌俱預焉。朝廷榮之,賜女樂一部、金寶萬計。武德四年,因侍宴暴卒,贈司空,謚曰密。



 子衍。衍嗣,官至左武衛將軍。時抗群從內三品七人,四品、五品十餘人,尚主三人,妃數人,冠冕之盛,當朝無比。



 靜,字元休,抗第二子也。武德初,累轉並
 州大總管府長史。時突厥數為邊患,師旅歲興,軍糧不屬,靜表請太原置屯田以省饋運。時議者以民物凋零,不宜動眾,書奏不省。靜頻上書,辭甚切至。於是征靜入朝,與裴寂、蕭瑀、封德彞等爭論於殿庭,寂等不能屈,竟從靜議。歲收數千斛,高祖善之,令檢校並州大總管。靜又以突厥頻來入寇,請斷石嶺以為障塞,復從之。太宗即位,徵拜司農卿,封信都男,尋轉夏州都督。值突厥攜貳,諸將出征,多詣其所。靜知虜中虛實,潛令人間其部
 落,鬱射設所部鬱孤尼等九俟斤並率眾歸款,太宗稱善,賜馬百匹、羊千口。及擒頡利,處其部眾於河南,以為不便,上封曰:「臣聞夷狄者,同夫禽獸,窮則搏噬,群則聚塵。不可以刑法威,不可以仁義教。衣食仰給,不務耕桑,徒損有為之民,以資無知之虜,得之則無益於治,失之則無損於化。然彼首丘之情,未易忘也,誠恐一旦變生,犯我王略,愚臣之所深慮。如臣計者,莫如因其破亡之後,加其無妄之福,假以賢王之號,妻以宗室之女,分其
 土地,析其部落,使其權弱勢分,易為羈制。自可永保邊塞,俾為籓臣,此實長轡遠馭之道。」於時務在懷輯,雖未從之,太宗深嘉其志。制曰:「北方之務,悉以相委,以卿為寧朔大使,撫鎮華戎,朕無北顧之憂矣。」再遷民部尚書。貞觀九年卒,謚曰肅。子逵。



 逵尚太宗女遂安公主,襲爵信都男。



 誕,抗第三子也。隋仁壽中,起家為朝請郎。義寧初,闢丞相府祭酒,轉殿中監,封安豐郡公,尚高祖女襄陽公主。從太宗征薛舉,為元帥府司馬。遷刑部尚書,轉
 太常卿。高祖諸少子荊王元景等未出宮者十餘王,所有國司家產之事,皆令誕主之。出為梁州都督。貞觀初,召拜右領軍大將軍,轉大理卿、莘國公。修營太廟,賜物五百段。復為殿中監,以疾解官,復拜宗正卿。太宗常與之言,昏忘不能對,乃手詔曰:「朕聞為官擇人者治,為人擇官者亂。竇誕比來精神衰耗,殊異常時。知不肖而任之,睹尸祿而不退,非唯傷風亂政,亦恐為君不明。考績黜陟,古今常典,誕可光祿大夫還第。」尋卒,贈工部尚書、
 荊州刺史,謚曰安。



 子孝慈。孝慈嗣,官至左衛將軍。孝慈子希玠。希玠少襲爵,中宗時為禮部尚書,以恩澤賜實封二百五十戶。開元初為太子少傅、開府儀同三司。誕少子孝諶,在《外戚傳》。竇氏自武德至今,再為外戚,一品三人,三品已上三十餘人,尚主者八人,女為王妃六人,唐世貴盛,莫與為比。



 璡,字之推,抗季弟也。大業末,為扶風太守。高祖定京師,以郡歸國,歷禮部、民部二尚書。從太宗平薛仁杲。尋鎮益州,時蜀中尚多寇賊,璡屢討平
 之。時皇甫無逸在蜀,與之不協,璡屢請入朝。高祖征之,中路詔令還鎮。璡不得志,遂於路左題山,以申鬱積。有使者至其所,璡宴之臥內,遺以綾綺。無逸奏其事,坐免官。未幾,拜秘書監,封鄧國公。貞觀初,授太子詹事。後為將作大匠,修葺洛陽宮。璡於宮中鑿池起山,崇飾雕麗,虛費功力,太宗怒,遽令毀之。坐事免。會納其女為酆王妃,俄而復位,加右光祿大夫。七年卒,贈禮部尚書,謚曰安。璡頗曉音律,武德中,與太常少卿祖孝孫受詔定正
 聲雅樂,璡討論故實,撰《正聲調》一卷,行於代。



 史臣曰:得人者昌,如諸溫儒雅清顯,為一時之稱;叔達才學明辯,中二國之選。皆抱廊廟之器,俱為社稷之臣。威守道,軌臨戎,抗居喪,靜經略,璡音律,仍以懿親,俱至顯位;才能門第,輝映數朝,豈非得人歟?唐之昌也,不亦宜乎!然彥博之褊,竇軌之酷,亦非全器焉。



 贊曰:溫、陳才位,文蔚典禮。諸竇戚里,榮盛無比。



\end{pinyinscope}