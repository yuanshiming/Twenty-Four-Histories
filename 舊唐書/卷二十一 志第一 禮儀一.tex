\article{卷二十一 志第一 禮儀一}

\begin{pinyinscope}

 《記》曰:「人生而靜,天之性也;感物而動,性之欲也。」欲無限極,禍亂生焉。聖人懼其邪放,於是作樂以和其性,制禮以檢其情,俾俯仰有容,周旋中矩。故肆覲之禮立,則朝
 廷尊;郊廟之禮立,則人情肅;冠婚之禮立,則長幼序;喪祭之禮立,則孝慈著;搜狩之禮立,則軍旅振;享宴之禮立,則君臣篤。是知禮者,品匯之璿衡,人倫之繩墨,失之者辱,得之者榮,造物已還,不可須臾離也。五帝之時,斯為治本。類帝禋宗,吉禮也;遏音陶瓦,兇禮也;班瑞肆覲,賓禮也;誅苗殛鯀,軍禮也;厘降嬪虞,嘉禮也。故曰,修五禮五玉,堯、舜之事也。時代猶淳,節文尚簡。及周公相成王,制五禮六樂,各有典司,其儀大備。暨幽、厲失道,平王
 東遷,周室浸微,諸侯侮法。男女失冠婚之節,《野麕》之刺興焉;君臣廢朝會之期,踐土之譏著矣。葬則奢儉無算,軍則狙詐不仁。數百年間,禮儀大壞。雖仲尼自衛返魯,而有定禮之言,蓋舉周公之舊章,無救魯邦之亂政。仲尼之世,體教已亡。遭秦燔煬,遺文殆盡。



 漢興,叔孫通草定,止習朝儀。至於郊天祀地之文,配祖禋宗之制,拊石鳴球之備物,介丘璧水之盛猷,語則有之,未遑措思。及世宗禮重儒術,屢訪賢良,河間博洽古文,大搜經籍,有
 周舊典,始得《周官》五篇,《士禮》十七篇。王又鳩集諸子之說,為禮書一百四十篇。後倉二戴,因而刪擇,得四十九篇,此《曲臺集禮》,今之《禮記》是也。然數百載不見舊儀,諸子所書,止論其意。百家縱胸臆之說,五禮無著定之文。故西漢一朝,曲臺無制。郊上帝於甘泉,祀后土於汾陰。宗廟無定主,樂懸缺金石。巡狩非勛、華之典,封禪異陶匏之音。光武受命,始詔儒官草定儀注,經邦大典,至是粗備。漢末喪亂,又淪沒焉。而衛宏、應仲遠、王仲宣等掇
 拾遺散,裁志條目而已。東京舊典,世莫得聞。自晉至梁,繼令條纘。鴻生鉅儒,銳思綿蕝,江左學者,仿佛可觀。隋氏平陳,寰區一統,文帝命太常卿牛弘集南北儀注,定《五禮》一百三十篇。煬帝在廣陵,亦聚學徒,修《江都集禮》。由是周、漢之制,僅有遺風。



 神堯受禪,未遑制作,郊廟宴享,悉用隋代舊儀。太宗皇帝踐祚之初,悉興文教,乃詔中書令房玄齡、秘書監魏徵等禮官學士,修改舊禮,定著《吉禮》六十一篇,《賓禮》四篇,《《軍禮》二十篇,《嘉禮》四十二
 篇,《兇禮》六篇,《國恤》五篇,總一百三十八篇,分為一百卷。玄齡等始與禮官述議,以為《月令》示昔祭,唯祭天宗,謂日月而下。近代示昔五天帝、五人帝、五地祇,皆非古典,今並除之。又依禮,有益於人則祀之。神州者國之所托,餘八州則義不相及。近代通祭九州,今除八州等八座,唯祭皇地祇及神州,以正祀典。又漢建武中封禪,用元封時故事,封泰山於圓臺上,四面皆立石闕,並高五丈。有方石再累,藏玉牒書。石檢十枚,於四邊檢之,東西各三,南北
 各二。外設石封,高九尺,上加石蓋。周設石距十八,如碑之狀,去壇二步,其下石跗入地數尺。今案封禪者,本以成功告於上帝。天道貴質,故藉用槁秸,樽以瓦甒。此法不在經誥,又乖醇素之道,定議除之。近又案梁甫是梁陰,代設壇於山上,乃乖處陰之義。今定禪禮改壇位於山北。又皇太子入學及太常行山陵、天子大射、合朔、陳五兵於太社、農隙講武、納皇后行六禮、四孟月讀時令、天子上陵、朝廟、養老於闢雍之禮,皆周、隋所闕,凡增多二
 十九條。餘並準依古禮,旁求異代,擇其善者而從之。太宗稱善,頒於內外行焉。



 高宗初,議者以《貞觀禮》節文未盡,又詔太尉長孫無忌、中書令杜正倫李義府、中書侍郎李友益、黃門侍郎劉祥道許圉師、太子賓客許敬宗、太常少卿韋琨、太學博士史道玄、符璽郎孔志約、太常博士蕭,楚才孫自覺賀紀等重加緝定,勒成一百三十卷。至顯慶三年奏上之,增損舊禮,並與令式參會改定,高宗自為之序。時許敬宗、李義府用事,其所損益,多涉
 希旨,行用已後,學者紛議,以為不及貞觀。上元三年三月,下詔令依貞觀年禮為定。儀鳳二年,又詔顯慶新修禮多有事不師古,其五禮並依周禮行事。自是禮司益無憑準,每有大事,皆參會古今禮文,臨時撰定。然貞觀、顯慶二《禮》,皆行用不廢。時有太常卿裴明禮、太常少卿韋萬石相次參掌其事,又前後博士賀敱、賀紀、韋叔夏、裴守真等多所議定。則天時,以禮官不甚詳明,特詔國子博士祝欽明及叔夏,每有儀注,皆令參定。叔夏卒後,
 博士唐紹專知禮儀,博學詳練舊事,議者以為稱職。先天二年,紹為給事中,以講武失儀,得罪被誅。其後禮官張星、王琇又以元日儀注乖失,詔免官歸家學問。



 開元十年,詔國子司業韋絳為禮儀使,專掌五禮。十四年,通事舍人王嵒上疏,請改撰《禮記》,削去舊文,而以今事編之。詔付集賢院學士詳議。右丞相張說奏曰:「《禮記》漢朝所編,遂為歷代不刊之典。今去聖久遠,恐難改易。今之五禮儀注,貞觀、顯慶兩度所修,前後頗有不同,其中或未
 折衷。望與學士等更討論古今,刪改行用。」制從之。初令學士右散騎常侍徐堅及左拾遺李銳、太常博士施敬本等檢撰,歷年不就。說卒後,蕭嵩代為集賢院學士,始奏起居舍人王仲丘撰成一百五十卷,名曰《大唐開元禮》。二十年九月,頒所司行用焉。



 昊天上帝、五方帝、皇地祇、神州及宗廟為大祀,社稷、日月星辰、先代帝王、岳鎮海瀆、帝社、先蠶、釋奠為中祀,司中、司命、風伯、雨師、諸星、山林川澤之屬為小祀。大祀,所司每年預定日奏下。小
 祀,但移牒所由。若天子不親祭享,則三公行事;若官缺,則職事三品已上攝三公行事。大祀散齋四日,致齋三日。中祀散齋三日,致齋二日。小祀散齋二日,致齋一日。散齋之日,晝理事如舊,夜宿於家正寢,不得吊喪問疾,不判署刑殺文書,不決罰罪人,不作樂,不預穢惡之事。致齋惟為祀事得行,其餘悉斷。若大祀,齋官皆於散齋之日,集於尚書省受誓戒,太尉讀誓文。致齋之日,三公於尚書省安置;餘官各於本司,若皇城內無本司,於太
 常郊社、太廟署安置。皆日未出前至齋所。至祀前一日,各從齋所晝漏上水五刻向祠所。接神之官,皆沐浴給明衣。若天子親祠,則於正殿行致齋之禮。文武官服褲褶,陪位於殿庭。車駕及齋官赴祠祭之所,州縣及金吾清所行之路,不得見諸兇穢及縗絰者,哭泣之聲聞於祭所者權斷,訖事依舊。齋官至祠所,太官惟設食。祭訖,依班序餕,訖,均胙,貴者不重,賤者不虛。中祀已下,惟不受誓戒,自餘皆同大祀之禮。



 武德初,定令:



 每歲冬至,祀
 昊天上帝於圓丘,以景帝配。其壇在京城明德門外道東二里。壇制四成,各高八尺一寸,下成廣二十丈,再成廣十五丈,三成廣十丈,四成廣五丈。每祀則天上帝及配帝設位於平座,藉用槁秸,器用陶匏。五方上帝、日月、內官、中官、外官及眾星,並皆從祀。其五方帝及日月七座,在壇之第二等;內五星已下官五十五座,在壇之第三等;二十八宿已下中官一百三十五座,在壇之第四等;外官百十二座,在壇下外壝之內;眾星三百六十
 座,在外壝之外。其牲,上帝及配帝用蒼犢二,五方帝及日月用方色犢各一,內官已下加羊豕各九。夏至,祭皇地祗於方丘,亦以景帝配。其壇在宮城之北十四里。壇制再成,下成方十丈,上成五丈。每祀則地祇及配帝設位於壇上,神州及五岳、四鎮、四瀆、四海、五方、山林、川澤、丘陵、墳衍、原隰,並皆從祀。神州在壇之第二等。五岳已下三十七座,在壇下外壝之內。丘陵等三十座,在壝外。其牲,地祗及配帝用犢二,神州用黝犢一,岳鎮已下加
 羊豕各五。



 孟春辛日,祈穀,祀感帝於南郊,元帝配,牲用蒼犢二。孟夏之月,雩祀昊天上帝於圓丘,景帝配,牲用蒼犢二。五方上帝、五人帝、五官並從祀,用方色犢十。季秋,祀五方上帝於明堂,元帝配,牲用蒼犢二。五人帝、五官並從祀,用方色犢十。孟冬,祭神州於北郊,景帝配,牲用黝犢二。



 貞觀初,詔奉高祖配圓丘及明堂北郊之祀,元帝專配感帝,自餘悉依武德。永徽二年,又奉太宗配祀於明堂,有司遂以高祖配五天帝,太宗配五人
 帝。



 顯慶元年,太尉長孫無忌與禮官等奏議曰:



 臣等謹尋方冊,歷考前規,宗祀明堂,必配天帝,而伏羲五代,本配五郊,預入有堂,自緣從祀。今以太宗作配,理有示安。伏見永徽二年七月,詔建明堂,伏惟陛下天縱聖德,追奉太宗,已遵嚴配。時高祖先在明堂,禮司致惑,竟未遷祀,率意定儀,遂便著令。乃以太宗皇帝降配五人帝,雖復亦在明堂,不得對越天帝,深乖明詔之意,又與先典不同。



 謹案《孝經》云:「孝莫大於嚴父,嚴父莫大於配天。昔者周公宗祀文王於明
 堂,以配上帝。伏惟詔意,義在於斯。今所司行令殊為失旨。又尋漢、魏、晉、宋歷代禮儀,並無父子同配明堂之義。唯《祭法》云:「周人禘嚳而郊稷,祖文王而宗武王。」鄭玄注云:「禘、郊、祖、宗,謂祭祀以配食也。禘謂祭昊天於圓丘,郊謂祭上帝於南郊,祖、宗謂祭五帝、五神於明堂也。」尋鄭此注,乃以祖、宗合為一祭,又以文、武共在明堂,連衽配祀,良為謬矣。故王肅駁曰:「古者祖有功而宗有德,祖、宗自是不毀之名,非謂配食於明堂者也。審如鄭義,則《孝
 經》當言祖祀文王於明堂,不得言宗祀也。凡宗者,尊也。周人既祖其廟,又尊其祀,孰謂祖於明堂者乎?」鄭引《孝經》以解《祭法》,而不曉周公本意,殊非仲尼之義旨也。又解「宗武王」云:「配勾芒之類,是謂五神,位在堂下。」武王降位,失君敘矣。



 又案《六韜》曰:「武王伐紂,雪深丈餘,五車二馬,行無轍跡,詣營求謁。武王怪而問焉,太公對曰:『此必五方之神,來受事耳。』遂以其名召入,各以其職命焉。既而克殷,風調雨順。」豈有生來受職,歿同配之,降尊敵卑,
 理不然矣。故《春秋外傳》曰:「禘、郊、祖、宗、報五者,國之典祀也。」《傳》言五者,故知各是一事,非謂祖、宗合祀於明堂也。



 臣謹上考殷、周,下洎貞觀,並無一代兩帝同配於明堂。南齊蕭氏以武、明昆季並於明堂配食,事乃不經,未足援據。又檢武德時令,以元皇帝配於明堂,兼配感帝。至貞觀初緣情革禮,奉祀高祖配於明堂,奉遷世祖專配感帝。此即聖朝故事已有遞遷之典,取法宗廟,古之制焉。伏惟太祖景皇帝構室有周,建絕代之丕業;啟祚汾、晉,創
 歷聖之洪基。德邁發生,道符立極。又世祖元皇帝潛鱗韞慶,屈道事周,導浚發之靈源,肇光宅之垂裕。稱祖清廟,萬代不遷。請停配祀,以符古義。伏惟高祖太武皇帝躬受天命,奄有神州,創制改物,體元居正,為國始祖,抑有舊章。昔者炎漢高帝,當塗太祖,皆以受命,例並配天。請遵故實,奉祀高祖於圓丘,以配昊天上帝。伏惟太宗文皇帝道格上元,功清下瀆,拯率土之塗炭,協大造於生靈,請準詔書,宗祀於明堂,以配上帝。又請依武德故
 事,兼配感帝作主。斯乃二祖德隆,永不遷廟;兩聖功大,各得配天。遠協《孝經》,近申詔意。



 二年七月,禮部尚書許敬宗與禮官等又奏議:



 據祠令及新禮,並用鄭玄六天之議,圓丘祀昊天上帝,南郊祭太微感帝,明堂祭太微五帝。謹按鄭玄此義,唯據緯書,所說六天,皆謂星象,而昊天上帝,不屬穹蒼。故注《月令》及《周官》,皆謂圓丘所祭昊天下帝為北辰星曜魄寶。又說《孝經》「郊祀后稷以配天」及明堂嚴父配天,皆為太微五帝。考其所說,舛謬特
 深。按《周易》云:「日月麗於天,百穀草木麗於地。」又云:「在天成象,在地成形。」足明辰象非天,草木非地。《毛詩傳》云:「元氣昊大,則稱昊天。遠視蒼蒼,則稱蒼天。」此則蒼昊為體,不入星辰之例。且天地各一,是曰兩儀。天尚無二,焉得有六?是以王肅群儒,咸駁此義。又檢太史《圓丘圖》,昊天上帝座外,別有北辰座,與鄭義不同。得太史令李淳風等狀,昊天上帝圖位自在壇上,北辰自在第二等,與北斗並列,為星官內座之首,不同鄭玄據緯書所說。此乃
 羲和所掌,觀象制圖,推步有征,相沿不謬。



 又按《史記天官書》等,太微宮有五帝者,自是五精之神,五星所奉。以其是人主之象,故況之曰帝。亦如房心為天王之象,豈是天乎!《周禮》云:「兆五帝於四郊。」又云:「祀五帝則掌百官之誓戒。」惟稱五帝,皆不言天。此自太微之神,本非穹昊之祭。又《孝經》惟云「郊祀后稷」,無別祀圓丘之文。王肅等以為郊即圓丘,圓丘即郊,猶王城、京師,異名同實。符合經典,其義甚明。而今從鄭說,分為兩祭,圓丘之外,別有南
 郊,違棄正經,理深未允。且檢吏部式,惟有南郊陪位,更不別載圓丘。式文既遵王肅,祠令仍行鄭義,令、式相乖,理宜改革。



 又《孝經》云「嚴父莫大於配天」,下文即云:「周公宗祀文王於明堂,以配上帝。」則是明堂所祀,正在配天,而以為但祭星官,反違明義。又按《月令》:「孟春之月,祈穀於上帝。」《左傳》亦云:「凡祀,啟蟄而郊,郊而後耕。故郊祀后稷,以祈農事。」然則啟蟄郊天,自以祈穀,謂為感帝之祭,事甚不經。今請憲章姬、孔,考取王、鄭,四郊迎
 氣,存太微五帝之祀;南郊明堂,廢緯書六天之義。其方丘祭地之外,別有神州,謂之北郊,



 分地為二,既無典據,理又不通,亦請合為一祀,以符古義。仍並條附式令,永垂後則。



 敬宗等又議籩、豆之數曰:「按今光祿式,祭天地、日月、岳鎮、海瀆、先蠶等,籩、豆各四。祭宗廟,籩、豆各十二。祭社稷、先農等,籩、豆各九。祭風師、雨師,籩、豆各二。尋此式文,事深乖謬。社稷多於天地,似不貴多。風雨少於日月,又不貴少。且先農、先蠶,俱為中祭,或六或四,理不可
 通。又先農之神,尊於釋奠,籩、豆之數,先農乃少,理既差舛,難以因循。謹按《禮記郊特牲》云:『籩、豆之薦,水土之品,不敢用褻味而貴多品,所以交於神明之義也。』此即祭祀籩、豆,以多為貴。宗廟之數,不可逾郊。今請大祀同為十二,中祀同為十,小祀同為八,釋奠準中祀。自餘從座,並請依舊式。」詔並可之,遂附於禮令。



 乾封初,高宗東封回,又詔依舊祀感帝及神州。司禮少常伯郝處俊等奏曰:



 顯慶新禮,廢感帝之祀,改為祈穀。昊天上帝,以高祖
 太武皇帝配。檢舊禮,感帝以世祖元皇帝配。今既奉敕仍舊復祈穀為感帝,以高祖太武皇帝配神州,又高祖依新禮見配圓丘昊天上帝及方丘皇地祇,若更配感帝神州,便恐有乖古禮。按《禮記·祭法》云:「有虞氏禘黃帝而郊嚳,夏后氏亦禘黃帝而郊鯀,殷人禘嚳而郊冥,周人禘嚳而郊稷。」鄭玄注云:「禘謂祭昊天於圓丘也。祭上帝於南郊曰郊」。又按《三禮義宗》云,「夏正郊天者,王者各祭所出帝於南郊」,即《大傳》所謂「王者禘其祖之所自出,以其祖配之」是也。此則
 禘須遠祖,郊須始祖。今若禘郊同用一祖,恐於典禮無所據。其神州十月祭者,十月以陰用事,故以此時祭之,依檢更無故實。按《春秋》「啟蟄而郊」,鄭玄注「禮云:「三王之郊,一用夏正。」又《三禮義宗》云:「祭神州法,正月祀於北郊。」請依典禮,以正月祭者。請集奉常博士及司成博士等總議定奏聞。其靈臺、明堂,檢書禮用鄭玄義,仍祭五方帝,新禮用王肅義。



 又下詔依鄭玄義祭五天帝,其雩及明堂,並準敕祭祀。於是奉常博士陸遵楷、張統師、權無
 二、許子儒等議稱:「北郊之月,古無明文。漢光武正月辛未,始建北郊。咸和中議,北郊同用正月,然皆無指據。武德來禮令即用十月,為是陰用事,故於時祭之。請依舊十月致祭。」



 乾封二年十二月,詔曰:



 夫受命承天,崇至敬於明祀;膺圖纂籙,昭大孝於嚴配。是以薦鰷鱨於清廟,集振鷺于西雍,宣《雅》、《頌》於太師,明肅恭於考室。用能紀配天之盛業,嗣積德之鴻休,永播英聲,長為稱首。周京道喪,秦室政乖,禮樂淪亡,典經殘滅。遂使漢朝博士,空
 說六宗之文;晉代鴻儒,爭陳七祀之議。或同昊天於五帝,分感帝於五行。自茲以降,遞相祖述,異論紛紜,是非莫定。



 朕以寡薄,嗣膺丕緒,肅承禋祀,明發載懷,虔奉宗祧,寤寐興感。每惟宗廟之重,尊配之儀,思革舊章,以申誠敬。高祖太武皇帝撫運膺期,創業垂統,拯庶類於塗炭,寘懷生於仁壽。太宗文皇帝德光齊聖,道極幾神,執銳被堅,櫛風沐雨,勞形以安百姓,屈己而濟四方,澤被區中,恩覃海外。乾坤所以交泰,品物於是咸亨。掩玄闕
 而開疆,指青丘而作鎮。巍巍蕩蕩,無得名焉。《禮》曰:「化人之道,莫急於禮。禮有五經,莫重於祭。祭者,非物自外至也,自內生於心也。是以惟賢者乃能盡祭之義。」況祖功宗德,道冠百王;盡聖窮神,業高千古。自今以後,祭圓丘、五方、明堂、感帝、神州等祠,高祖太武皇帝、太宗文皇帝崇配,仍總祭昊天上帝及五帝於明堂。庶因心致敬,獲展虔誠,宗祀配天,永光鴻烈。



 儀鳳二年七月,太常少卿韋萬石奏曰:「明堂大享,準古禮鄭玄義,祀五天帝,王肅
 義,祀五行帝。《貞觀禮》依鄭玄義祀五天帝,顯慶已來新修禮祀昊天上帝。奉乾封二年敕祀五帝,又奉制兼祀昊天上帝。伏奉上元三年三月敕,五禮並依貞觀年禮為定。又奉去年敕,並依周禮行事。今用樂須定所祀之神,未審依古禮及《貞觀禮》,為復依見行之禮?」時高宗及宰臣並不能斷,依違久而不決。尋又詔尚書省及學者詳議,事仍不定。自此明堂大享,兼用貞觀、顯慶二《禮》。



 則天臨朝,垂拱元年七月,有司議圓丘、方丘及南郊、明堂
 嚴配之禮。成均助教孔玄義奏議曰:



 謹按《孝經》云:「孝莫大於嚴父,嚴父莫大於配天。」明配尊大,昊天是也。物之大者,莫若於天,推父比天,與之相配,行孝之大,莫過於此,以明尊配之極也。又《易》云:「先王以作樂崇德,殷薦之上帝,以配祖考。」鄭玄注:』上帝,天帝也。」故知昊天之祭,合祖考並配。請奉太宗文武聖皇帝、高宗天皇大帝配昊天上帝於圓丘,義符《孝經》、《周易》之文也。神堯皇帝肇基王業,應天順人,請配感帝於南郊,義符《大傳》之文。又《祭
 法》云:「祖文王而宗武王。祖,始也;宗,尊也。所以名祭為尊始者,明一祭之中,有此二義。又《孝經》云:「宗祀文王於明堂。」文王言祖,而云宗者,亦是通武王之義。故明堂之祭,配以祖考。請奉太宗文武聖皇帝、高宗天皇大帝配祭於明堂,義符《周易》及《祭法》之文也。



 太子右諭德沈伯儀曰:



 謹按《禮》:「有虞氏禘黃帝而郊嚳,祖顓頊而宗堯。夏后氏禘黃帝而郊鯀,祖顓頊而宗禹。殷人禘嚳而郊冥,祖契而宗湯。周人禘嚳而郊稷,祖文王而宗武王。」鄭玄注
 云:「禘、郊、祖、宗,謂祭祀以配食也。禘謂祭昊天於圓丘,祭上帝於南郊曰郊,祭五帝、五神於明堂曰祖、宗。」伏尋嚴配之文,於此最為詳備。虞、夏則退顓頊而郊嚳,殷人則舍契而郊冥。去取既多。前後乖次。得禮之序,莫尚於周。禘嚳郊稷,不間於二王;明堂宗祀,始兼於兩配。咸以文王、武王父子殊別,文王為父,上主五帝;武王對父,下配五神。《孝經》曰:「嚴父莫大於配天,則周公其人也。昔者周公宗祀文王於明堂,以配上帝。」不言嚴武王以配天,則武
 王雖在明堂,理未齊於配祭;既稱宗祀,義獨主於尊嚴。雖同兩祭,終為一主。故《孝經緯》曰「後稷為天地主,文王為五帝宗」也。必若一神兩祭便,則五祭十祠,薦獻頻繁,禮虧於數。此則神無二主之道,禮崇一配之義。竊尋貞觀、永徽,共尊專配;顯慶之後,始創兼尊。必以順古而行,實謂從周為美。高祖神堯皇帝請配圓丘、方澤,太宗文武聖皇帝請配南郊、北郊。高宗天皇大帝德邁九皇,功開萬宇,制禮作樂,告禪升中,率土共休,普天同賴,竊惟
 莫大之孝,理當總配五天。



 鳳閣舍人元萬頃、範履冰等議曰:



 伏惟高祖神堯皇帝鑿乾構象,闢土開基。太宗文武聖皇帝紹統披元,循機闡極。高宗天皇大帝弘祖宗之大業,廓文武之宏規。三聖重光,千年接旦。神功睿德,罄圖牒而難稱;盛烈鴻猷,超古今而莫擬。豈徒錙銖堯、舜,糠粃殷、周而已哉!謹案見行禮,昊天上帝等祠五所,咸奉高祖神堯皇帝、太宗文武皇帝兼配。今議者引《祭法》、《周易》、《孝經》之文,雖近稽古之辭,殊失因心之旨。但子
 之事父,臣之事君,孝以成志,忠而順美。竊以兼配之禮,特稟先聖之懷,爰取訓於前規,遂申情於大孝。《詩》云:「昊天有成命,二後受之。」《易》曰:「殷薦之上帝,以配祖考。」敬尋厥旨,本合斯義。今若遠摭遺文,近乖成典,拘常不變,守滯莫通,便是臣黜於君,遽易郊丘之位,下非於上,靡遵弓劍之心。豈所以申太后哀感之誠,徇皇帝孝思之德!慎終追遠,良謂非宜。嚴父配天,寧當若是?伏據見行禮,高祖神堯皇帝、太宗文武聖皇帝,今既先配五祠,理當
 依舊無改。高宗天皇大帝齊尊曜魄,等邃含樞,闡三葉之宏基,開萬代之鴻業。重規疊矩,在功烈而無差;享帝郊天,豈祀配之有別。請奉高宗天皇大帝歷配五祠。



 制從萬頃議。自是郊丘諸祠皆以三祖配。



 及則天革命,天冊萬歲元年,加號為天冊金輪大聖皇帝,親享南郊,合祭天地。以武氏始祖周文王追尊為始祖文皇帝,後考應國公追尊為無上孝明高皇帝,亦以二祖同配,如乾封之禮。其後長安年又親享南郊,合祭天地及諸郊丘,
 並以配焉。



 中宗即位,神龍元年九月,親享昊天上帝於東都之明堂,以高宗天皇大崇配,其儀亦依乾封故事。至景龍三年十一月,親祀南郊,初將定儀注,國子祭酒祝欽明希旨上言後亦合助祭,遂奏議曰:「謹按《周禮》:『天神曰祀,地祇曰祭,宗廟曰享。』又《內司服》:『職掌王後之六服,凡祭祀,供後之衣服。』又《祭統》曰:『夫祭也者,必夫婦親之。』據此諸文,即知皇後合助皇帝祀天神祭地祇明矣。望請別修助祭儀注同進。」上令宰相與禮官議詳其
 事。太常博士唐紹、蔣欽緒建議云:「皇后南郊助祭,於禮不合。但欽明所執,是祭宗廟禮,非祭天地禮。按漢、魏、晉、及後魏、齊、梁、隋等歷代史籍,興王令主,郊天祀地,代有其禮,史不闕書,並不見皇后助祭之事。又高祖神堯皇帝、太宗文武聖皇帝、高宗天皇大帝南郊祀天,並無皇后助祭之禮。」尚書右僕射韋巨源又協同欽明之議,上遂以皇后為亞獻,仍補大臣李嶠等女為齋娘,執籩豆焉。



 時十一月十三日乙丑,冬至,陰陽人盧雅、侯藝等
 奏請促冬至就十二日甲子以為吉會。時右臺侍御史唐紹奏曰:「禮所以冬至祀圓丘於南郊,夏至祭方澤於北郊者,以其日行躔次,極於南北之際也。日北極當晷度循半,日南極當晷度環周。是日一陽爻生,為天地交際之始。故《易》曰:『《復》,其見天地之心乎!』即冬至卦象也。一歲之內,吉莫大焉。甲子但為六旬之首,一年之內,隔月常遇,既非大會,晷運未周,唯總六甲之辰,助四時而成歲。今欲避環周以取甲子,是背大吉而就小吉也。」太史
 令傅孝忠奏曰:「準《漏刻經》,南陸北陸並日校一分,若用十二日,即欠一分。未南極,即不得為至。」上曰:「俗諺云,『冬至長於歲』,亦不可改。」竟依紹議以十三日乙丑祀圓丘。



 睿宗太極元年正月,初將有事南郊,有司立議,惟祭昊天上帝而不設皇地祇位。諫議大夫賈曾上表曰:



 微臣詳據典禮,謂宜天地合祭。謹按《禮祭法》曰:「有虞氏禘黃帝而郊嚳,夏后氏禘黃帝而郊鯀。」傳曰:大祭曰禘。然則郊之與廟,俱有禘祭。禘廟,則祖宗之主俱合於太祖之廟;
 禘郊,則地祇群望俱合於圓丘,以始祖配享。皆有事而大祭,異於常祀之義。《禮大傳》曰:「不王不禘。」故知王者受命,必行禘禮。《虞書》曰:「月正元日,舜格於文祖,肆類於上帝,祇於六宗,望於山川,遍於群神。」此則受命而行禘禮者也。言「格於文祖」,則餘廟之享可知矣。言「類於上帝」,則地祇之合可知矣。且山川之祀,皆屬於地,群望尚遍,況地祇乎!《周官》「以六律、六呂、五聲、八音、六舞、大合樂,以致神祇,以和邦國,以諧萬人。」又「凡六樂者,六變而致象物
 及天神」,此則禘郊合天神、地祇、人鬼而祭之樂也。



 《三輔故事》漢祭圓丘儀:昊天上帝位正南面,後土位兆亦南面而少東。又《東觀漢記》云:「光武即位,為壇於鄗之陽,祭告天地,採用元始故事。二年正月,於洛陽城南依鄗為圓壇,天地位其上,皆南向西上。」按兩漢時自有後土及北郊祀,而此已於圓丘設地位,明是禘祭之儀。又《春秋說》云:「王者一歲七祭,天地合食於四孟,別於分、至。」此復天地自常有同祭之義。王肅云:「孔子言兆圓丘於南郊,
 南郊即圓丘,圓丘即南郊也。」又云:「祭天地配。」此亦郊祀合祭之明說。惟鄭康成不論禘當合祭,而分昊天上帝為二神,專憑緯文,事匪經見。又其注《大傳》「不環不禘」義,則云:「正歲之首,祭感帝之精,以其祖配。」注《周官·大司樂》圓丘,則引《大傳》之禘以為冬至之祭。遞相矛盾,未足可依。



 伏惟陛下膺籙居尊,繼文在歷,自臨宸極,未親郊祭。今之南郊,正當禘禮,固宜合祀天地,咸秩百神,答受命之符,彰致敬之道。豈可不崇盛禮,同彼常郊,使地祇
 無位,未從禘享!今請備設皇地祇並從祀等座,則禮得稽古,義合緣情。然郊丘之祀,國之大事,或失其情,精禋將闕。臣術不通經,識慚博古,徒以昔謬禮職,今忝諫曹,正議是司,敢陳忠讜。事有可採,惟斷之聖慮。



 制令宰臣召禮官詳議可否。禮官國子祭酒褚無量、國子司業郭山惲等咸請依曾所奏。時又將親享北郊,竟寢曾之表。



 玄宗即位,開元十一年十一月,親享圓丘。時中書令張說為禮儀使,衛尉少卿韋絳為副,說建議請以高祖神
 堯皇帝配祭,始罷三祖同配之禮。至二十年,蕭嵩為中書令,改撰新禮。祀天一歲有四,祀地有二。冬至,祀昊天上帝於圓丘,高祖神堯皇帝配,中官加為一百五十九座,外官減為一百四座。其昊天上帝及配帝二座,每座籩、豆各用十二,簋、簠、、俎各一。上帝則太樽、著樽、犧樽、象樽、壺樽各二,山罍六。配帝則不設太樽及壺樽,減山罍之四,餘同上帝。五方帝座則籩、豆各十,簋、簠、、俎各一,太樽二。大明、夜明,籩、豆各八,餘同五方帝。內官每座
 籩、豆二,簋、俎各一。內官已上設樽於十二階之間。內官每道間著樽二,中官犧樽二,外官著樽二,眾星壺樽二。正月上辛,祈穀,祀昊天上帝於圓丘,以高祖配,五方帝從祀。其上帝、配帝,籩、豆等同冬至之數。五方帝,太樽、著樽、犧樽、山罍各一,籩、豆等亦同冬至之數。孟夏,雩昊天上之帝於圓丘,以太宗配,五方帝及太昊等五帝、勾芒等五官從祀。其上帝配帝、五方帝,籩、豆各八,簋、簠、、俎各一。五官每座籩、豆各二,簋、簠及俎各一。季秋,大
 享於明堂,祀昊天上帝,以睿宗配,其五方帝、五人帝、五官從祀。籩、豆之數,同於雩祀。夏至,禮皇地祇於方丘,以高祖配,其從祀神州已下六十八座,同貞觀之禮。地祇、配帝,籩、豆如圓丘之數。神州,籩、豆各四,簋、簠、、俎各一。五岳、四鎮、四海、四瀆、五方、山林、川澤等三十七座,每座籩、豆各二,簋、簠各一。五方五帝、丘陵、墳衍、原隰等三十座,籩、豆、簋、簠、、俎各一。立冬,祭神州於北郊,以太宗配。二座籩、豆各十二,簋、簠、、俎各一。自冬至圓丘已下,餘同貞
 觀之禮。



 時起居舍人王仲丘既掌知修撰,仍建議曰:



 按《貞觀禮》,正月上辛,祀感帝於南郊,《顯慶禮》,祀昊天上帝於圓丘以祈穀。《左傳》曰:「郊而後耕。」《詩》曰:「《噫嘻》,春夏祈穀於上帝。」《禮記》亦曰:「上辛祈穀於上帝。」則祈穀之文,傳於歷代,上帝之號,允屬昊天。而鄭康成云:「天之五帝遞王,王者之興,必感其一,因其所感,別祭尊之。故夏正之月,祭其所生之帝於南郊,以其祖配之。故周祭靈威仰,以後稷配之,因以祈穀。」據所說祀感帝之意,本非祈穀。先
 儒所說,事恐難憑。今祈穀之禮,請準禮修之。且感帝之祀,行之自久。《記》曰:「有其舉之,莫可廢也。」請於祈穀之壇,遍祭五方帝。夫五帝者,五行之精。五行者,九穀之宗也。今請二禮並行,六神咸祀。



 又按《貞觀禮》,孟夏雩祀五方上帝、五人帝、五官於南郊,《顯慶禮》,則雩祀昊天上帝於圓丘。且雩祀上帝,蓋為百穀祈甘雨。故《月令》云:「命有司大雩帝,用盛樂,以祈穀實。」鄭玄云:「雩上帝者,天之別號,允屬昊天,祀於圓丘,尊天位也。」然雩祀五帝既久,亦請
 二禮並行,以成大雩帝之義。



 又《貞觀禮》,季秋祀五方帝、五官於明堂,《顯慶禮》,禮昊天上帝於明堂。準《孝經》曰:「郊祀后稷以配天,宗祀文王於明堂,以配上帝。」先儒以為天是感精之帝,即太微五帝,此即皆是星辰之例。且上帝之號,皆屬昊天,鄭玄所引,皆云五帝。《周禮》曰:「王將旅上帝,張氈案,設皇邸。祀五帝,張大次小次。」由此言之,上帝之與五帝,自有差等,豈可混而為一乎!《孝經》云:「嚴父莫大於配天。」其下文即云:「宗祀文王於明堂,以配上帝。」
 鄭玄注云:「上帝者,天之別名,神無二主,故異其處。」孔安國之:「帝亦天也。」



 然則禋享上帝,有合經義。而五方皆祀,行之已久,有其舉之,難於即廢。亦請二禮並行,以成《月令》大享帝之義。



 天寶十載五月已前,郊祭天地,以高祖神堯皇帝配座,故將祭郊廟,告高祖神堯皇帝室。寶應元年,杜鴻漸為太常卿禮儀使,員外郎薛頎、歸崇敬等議:「以神堯為受命之主,非始封之君不得為太祖以配天地。太祖景皇帝始受封於唐,即殷之契,周之後稷也。
 請以太祖景皇帝郊祀配天地,告請宗廟,亦太祖景皇帝酌獻。諫議大夫黎幹議,以太祖景皇帝非受命之君,不合配享天地。二年五月,干進議狀為十詰十難,曰:



 集賢校理潤州別駕歸崇敬議狀及禮儀使判官水部員外郎薛頎等稱:禘謂冬至祭天於圓丘,周人則以遠祖帝嚳配,今欲以景皇帝為始祖,配昊天於圓丘。



 臣干詰曰:「《國語》曰:「有虞氏、夏后氏俱禘黃帝,商人禘舜,周人禘嚳。」俱不言祭昊天於圓丘,一也。《詩·商頌》曰:「《長發》,大禘也。」
 又不言昊天於圓丘,二也。《詩·周頌》曰:「《雍》,禘太祖也。」又不言祭昊天於圓丘,三也。《禮記·祭法》曰:「有虞氏、夏后氏俱禘黃帝,殷人、周人俱禘嚳。」又不言祭昊天於圓丘,四也。《禮記·大傳》曰:「不王不禘。王者禘其祖之所自出,以其祖配之。」又不言祭昊天於圓丘,五也。《爾雅·釋天》曰:「禘,大祭也。」又不言祭昊天於圓丘,六也。《家語》云:「凡四代帝王之所郊,皆以配天也。其所謂禘者,皆五年大祭也。」又不言祭昊天於圓丘,七也。盧植云:「禘,祭名。禘者諦也,事尊
 明諦,故曰禘。」又不言祭昊天於圓丘,八也。王肅云:「禘謂於五年大祭之時。」又不言祭昊天於圓丘,九也。郭璞云:「禘,五年之大祭。」又不言祭昊天於圓丘,十也。



 臣乾謂禘是五年宗廟之大祭,《詩》、《禮》經傳,文義昭然。今略舉十詰以明之。臣惟見《禮記·祭法》及《禮記·大傳》、《商頌·長發》等三處鄭玄注,或稱祭昊天,或云祭靈威仰。臣精詳典籍,更無以禘為祭昊天於圓丘及郊祭天者。審如禘是祭之最大,則孔子說《孝經》為萬代百王法,稱周公大孝,何不
 言禘祀帝嚳於圓丘以配天,而反言「郊祀后稷以配天?」是以《五經》俱無其說,聖人所以不言。輕議大典,亦何容易。猶恐不悟,今更作十難。



 其一難曰:《周頌》:「《雍》,禘祭太祖也。」鄭玄箋云:「禘,大祭。太祖,文王也。」《商頌》云:「《長發》,大禘也。」玄又箋云:「大禘,祭天也。」夫商、周之《頌》,其文互說。或云禘太祖,或云大禘,俱是五年宗廟之大祭,詳覽典籍,更無異同。惟鄭玄箋《長發》,乃稱是郊祭天。詳玄之意,因此《商頌》禘如《大傳》云大祭,如《春秋》「大事於太廟」,《爾雅》「禘大祭」,
 雖云大祭,亦是宗廟之祭,可得便稱祭天乎?若如所說,大禘即云郊祭天,稱禘即是祭宗廟。又《祭法》說虞、夏、商、周禘黃帝與嚳,《大傳》「不王不禘」,禘上俱無大字,玄何因復稱祭天乎?又《長發》文亦不歌嚳與感生帝,故知《長發》之禘,而非禘嚳及郊祭天明矣。殷、周五帝之大祭,群經眾史及鴻儒碩學,自古立言著論,序之詳矣,俱無以禘為祭天。何棄周、孔之法言,獨取康成之小注,便欲違經非聖,誣亂祀典,謬哉!



 其二難曰:《大傳》稱「禮,不王不禘,王
 者禘其祖之所自出,以其祖配之,諸侯及其太祖」者,此說王者則當禘。其謂《祭法》,虞、夏、殷、周禘黃帝及嚳,「不王則不禘,所當禘其祖之所自出」,謂虞、夏出黃帝,殷、周出帝嚳,以近祖配而祭之。自出之祖,既無宗廟,即是自外至者,故同之天地神祇,以祖配而祀之。自出之說,非但於父,在母亦然。《左傳》子產云:「陳則我周之自出。」此可得稱出於太微五帝乎?故曰「不王不禘,王者禘其祖之所自出,以其祖配之」,此之謂也。及諸侯之禘,則降於王者,不得
 祭自出之祖,只及太祖而已。故曰「諸侯及其太祖」,此之謂也。鄭玄錯亂,分禘為三:注《祭法》云「禘謂祭昊天於圓丘」,一也。注《大傳》稱「郊祭天,以後稷配靈威仰」,箋《商頌》又稱「郊祭天」,二也。注《周頌》云「禘大祭,大於四時之祭,而小於祫,太祖謂文王」,三也。禘是一祭,玄析之為三,顛倒錯亂,皆率胸臆,曾無典據,何足可憑。



 其三難曰:虞、夏、殷、周已前,禘祖之所自出,其義昭然。自漢、魏、晉已還千餘歲,其禮遂闕。又鄭玄所說,其言不經,先儒棄之,未曾行用。
 愚以為錯亂之義,廢棄之注,不足以正大典。



 其四難曰:所稱今《三禮》行於代者,皆是鄭玄之學,請據鄭學以明之。曰雖云據鄭學,今欲以景皇帝為始祖之廟以配天,復與鄭義相乖。何者?《王制》云:「天子七廟。」玄云:「此周禮也。」七廟者,太祖及文、武之祧與親廟四也。殷則六廟,契及湯與二昭二穆也。據鄭學,夏不以鯀及顓頊、昌意為始祖,昭然可知也。而欲引稷、契為例,其義又異是。爰稽邃古洎今,無以人臣為始祖者,惟殷以契,周以稷。夫稷、契
 者,皆天子元妃之子,感神而生。昔帝嚳次妃簡狄,有戎氏之女,吞玄鳥之卵,因生契。契長而佐禹治水,有大功。舜乃命契作司徒,百姓既和,遂封於商。故《詩》曰:「天命玄鳥,降而生商,宅殷土芒芒。」此之謂也。後稷者,其母有邰氏之女曰姜嫄,為帝嚳妃,出野履巨跡,歆然有孕,生稷。稷長而勤於稼穡,堯聞,舉為農師,天下得其利,有大功,舜封於邰,號曰后稷。唐、虞、夏之際,皆有令德。故《詩》曰:「履帝武敏歆,居然生子,即有邰家室。」此之謂也。舜、禹有天
 下,稷、契在其間,量功比德,抑其次也。舜授職,則播百穀,敷五教。禹讓功,則平水土,宅百揆。故《國語》曰:「聖人之制祀也,功施於人則祀之,以死勤事則祀之。」契為司徒而人輯睦,稷勤百穀而死,皆居前代祀典,子孫有天下,得不尊而祖之乎?



 其五難曰:既遵鄭說,小德配寡,遂以後稷只配一帝,尚不得全配五帝。今以景皇帝特配昊天,於鄭義可乎?



 其六難曰:眾難臣云:「上帝與五帝,一也。所引《春官》:祀天旅上帝,祀地旅四望。旅訓眾,則上帝是五
 帝。臣曰,不然。旅雖訓眾,出於《爾雅》,及為祭名,《春官》訓陳,注有明文。若如所言,旅上帝便成五帝,則季氏旅於泰山,可得便是四鎮耶?



 其七難曰:所云據鄭學,則景皇帝親盡,廟主合祧,卻欲配祭天地,錯亂祖宗。夫始祖者,經綸草昧,體大則天,所以正元氣廣大,萬物之宗尊,以長至陽氣萌動之始日,俱祀於南郊也。夫萬物之始,天也。人之始,祖也。日之始,至也。掃地而祭,質也。器用陶匏,性也。牲用犢,誠也。兆於南郊,就陽位也。至尊至質,不敢同
 於先祖,禮也。故《白虎通》曰:「祭天歲一,何?天至尊至質,事之不敢褻黷,故因歲之陽氣始達而祭之。」今國家一歲四祭之,黷莫大焉。上帝、五帝,其祀遂闕,怠亦甚矣。黷與怠,皆禮之失,不可不知。夫親有限,祖有常,聖人制禮,君子不以情變易。國家重光累聖,歷祀百數,豈不知景皇帝始封於唐。當時通儒議功度德,尊神堯克配彼天,宗太宗以配上帝。神有定主,為日已久。今欲黜神堯配含樞紐,以太宗配上帝,則紫微五精,上帝佐也,以子先父,
 豈禮意乎!非止神祇錯位,亦以祖宗乖序,何以上稱皇天祖宗之意哉!若夫神堯之功,太宗之德,格於皇天上帝,臣以為郊祀宗祀,無以加焉。



 其八難曰:欲以景皇帝為始祖,既非造我區宇,經綸草昧之主,故非夏始祖禹、殷始祖契、周始祖稷、漢始祖高帝、魏始祖武皇帝、晉始祖宣帝、國家始祖神堯皇帝同功比德,而忽升於宗祀圓丘之上,為昊天匹,曾謂圓丘不如林放乎?



 其九難曰:昨所言魏文帝丕以武帝操為始祖,晉武帝炎以宣帝
 懿為始祖者。夫孟德、仲達者,皆人傑也。擁天下之強兵,挾漢、魏之微主,專制海內,令行草偃,服袞冕,陳軒懸,天子決事於私第,公卿列拜於道左,名雖為臣,勢實凌君。後主因之而業帝,前王由之而禪代,子孫尊而祖之,不亦可乎?



 其十難曰:所引商、周、魏、晉,既不當矣,則景皇帝不為始祖明矣。我神堯拔出群雄之中,廓清隋室,拯生人於塗炭,則夏禹之勛不足多;成帝業於數年之間,則漢祖之功不足比。夏以大禹為始祖,漢以高帝為始祖,
 則我唐以神堯為始祖,法夏則漢,於義何嫌?今欲革皇天之禮,易太祖之廟,事之大者,莫大於斯,曾無按據,一何寡陋,不愧於心,不畏於天乎!



 以前奉詔,令諸司各據禮經定議者。臣干忝竊朝列,官以諫為名,以直見知,以學見達,不敢不罄竭以裨萬一。昨十四日,具以議狀呈宰相,宰相令朝臣與臣論難。所難臣者,以臣所見獨異,莫不勝辭飛辯,競欲碎臣理,鉗臣口。剖析毫厘,分別異同,序墳典之凝滯,指子傳之乖謬,事皆歸根,觸物不礙。
 但臣言有宗爾,豈辯者之流也。又歸崇敬、薛頎等援引鄭學,欲蕪祀典,臣為明辯,迷而不復。臣輒作十詰十難,援據墳籍,昭然可知。庶郊禘事得其真,嚴配不失其序,皇靈降祉,天下蒙賴。臣亦何顧不蹈鼎鑊?謹敢聞達,伏增悚越。



 議奏,不報。



 至二年春夏旱。言事者云:太祖景皇帝追封於唐,高祖實受命之祖,百神受職,合依高祖。今不得配享天地,所以神不降福,以致愆陽。代宗疑之,詔百僚會議。太常博士獨孤及獻議曰:



 禮,王者禘其祖之
 所自出,以其祖配之。凡受命始封之君,皆為太祖。繼太祖已下六廟,則以親盡迭毀。而太祖之廟,雖百代不遷。此五帝、三王所以尊祖敬宗也。故受命於神宗,禹也,而夏后氏祖顓頊而郊鯀。纘禹黜夏,湯也,而殷人郊冥而祖契。革命作周,武王也,而周人郊稷而祖文王。則明自古必以首封之君,配昊天上帝。唯漢氏崛起豐沛,豐公太公,皆無位無功,不可以為祖宗,故漢以高皇帝為太祖,其先細微也。非足為後代法。



 伏惟太祖景皇帝以柱國
 之任,翼周弼魏,肇啟王業,建封於唐。高祖因之,以為有天下之號,天所命也。亦如契之封商,後稷之封邰。禘郊祖宗之位,宜在百代不遷之典。郊祀太祖,宗祀高祖,猶周之祖文王而宗武王也。今若以高祖創業,當躋其祀,是棄三代之令典,尊漢氏之末制,黜景皇帝之大業,同豐公太公之不祀,反古違道,失孰大焉?夫追尊景皇,廟號太祖,高祖、太宗所以崇尊之禮也。若配天之位既異,則太祖之號宜廢,祀之不修,廟亦當毀。尊祖報本之道,
 其墜於地乎!漢制,擅議宗廟,以大不敬論。今武德、貞觀憲章未改,國家方將敬祀事,和神人,禘郊之間,恐非所宜。臣謹稽禮文,參諸往制,請仍舊典。



 竟依歸崇敬等議,以太祖配享天地。



 廣德二年正月十六日,禮儀使杜鴻漸奏:「郊、太廟,大禮,其祝文自今已後,請依唐禮,板上墨書。其玉簡金字者,一切停廢。如允臣所奏,望編為常式。」敕曰:「宜行用竹簡。」



 貞元元年十一月十一日,德宗親祀南郊。有司進圖,敕付禮官詳酌。博士柳冕奏曰:「開元定
 禮,垂之不刊。天寶改作,起自權制,此皆方士謬妄之說,非禮典之文,請一準《開元禮》。」從之。其年十月二十七日,詔:「郊禮之議,本於至誠。制禮定名,合從事實,使名實相副,則尊卑有倫。五方配帝,上古哲王,道濟烝人,禮著明祀。論善計功,則朕德不類,統天御極,朕位攸同。而於祝文稱臣以祭,既無益於誠敬,徒有瀆於等威。前京兆府司錄參軍高佩上疏陳請,其理精詳。朕重變舊儀,訪於卿士,申明大義,是用釋然。宜從改正,以敦至禮。自今已
 後,祀五方配帝祝文,並不須稱臣。其餘禮數如舊。」



 六年十一月八日,有事於南郊。詔以皇太子為亞獻,親王為終獻。上問禮官:「亞獻、終獻合受誓誡否?」吏部郎中柳冕曰:「準《開元禮》,獻官前七日於內受誓誡。辭云:『各揚其職,不供其事,國有常刑。』今以皇太子為亞獻,請改舊辭,云『各揚其職,肅奉常儀』。」從之。



 十五年四月,術士匡彭祖上言:「大唐土德,千年合符,請每於四季月郊祀天地。」詔禮官儒者議。歸崇敬曰:「準禮,立春迎春於東郊,祭青帝。
 立夏日迎夏於南郊,祭赤帝。立秋後十八日,迎黃靈於中地,祭黃帝。秋、冬各於其方。黃帝於五行為土,王在四季,土生於火,用事於木,而祭於秋,三季則否。漢、魏、周、隋,共行此禮。國家土德乘時,亦以每歲六月土王之日,祀黃帝於南郊,以後土配,合於典禮。彭祖憑候緯之說,據陰陽之書,事涉不經,恐難行用。」乃寢。



 元和十五年十二月,將有事於南郊。穆宗問禮官:「南郊卜日否?」禮院奏:「伏準禮令,祠祭皆卜。自天寶已後,凡欲郊祀,必先朝太清
 宮,次日饗太廟,又次日祀南郊。相循至今,並不卜日。」從之。及明年正月,南郊禮畢,有司不設御榻,上立受群臣慶賀。及御樓仗退,百僚復不於樓前賀,乃受賀於興慶宮。二者闕禮,有司之過也。



\end{pinyinscope}