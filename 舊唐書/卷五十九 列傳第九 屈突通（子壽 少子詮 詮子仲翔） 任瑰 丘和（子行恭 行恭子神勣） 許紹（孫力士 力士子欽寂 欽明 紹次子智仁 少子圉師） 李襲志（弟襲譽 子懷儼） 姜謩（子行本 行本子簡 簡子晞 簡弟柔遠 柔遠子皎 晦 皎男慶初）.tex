\article{卷五十九 列傳第九 屈突通(子壽 少子詮 詮子仲翔) 任瑰 丘和(子行恭 行恭子神勣) 許紹(孫力士 力士子欽寂 欽明 紹次子智仁 少子圉師) 李襲志(弟襲譽 子懷儼) 姜謩(子行本 行本子簡 簡子晞 簡弟柔遠 柔遠子皎 晦 皎男慶初)}

\begin{pinyinscope}

 ○屈突通子壽少子
 詮詮子仲翔任瑰丘和子行恭行恭子神勣許紹孫力士力士子欽寂欽明紹次子智仁少子圉師李襲志弟襲譽子懷人嚴



 姜謩子行本行本子簡簡子晞簡弟柔遠柔
 遠子皎晦皎男慶初



 屈突通,雍州長安人。父長卿,周邛州刺史。通性剛毅,志尚忠愨,檢身清正,好武略,善騎射。開皇中,為親衛大都督,文帝遣通往隴西檢覆群牧,得隱藏馬二萬餘匹。文帝盛怒,將斬太僕卿慕容悉達及諸監官千五百人,通諫曰:「人命至重,死不再生,陛下至仁至聖,子育群下,豈容以畜產之故,而戮千有餘人?愚臣狂狷,輒以死請。」文帝嗔目叱之,通又頓首曰:「臣一身如死,望免千餘人命。」帝寤,曰:「朕之不明,以至於是。感卿此意,良用惻然。今從所請,以旌諫諍。」悉達等竟以減死論。由是漸見委信,擢
 為右武候車騎將軍。奉公正直,雖親戚犯法,無所縱舍。時通弟蓋為長安令,亦以嚴整知名。時人為之語曰:「寧食三鬥艾,不見屈突蓋,寧服三鬥蔥,不逢屈突通。」為人所忌憚如此。及文帝崩,煬帝遣通以詔征漢王諒。先是,文帝與諒有密約曰:「若璽書召汝,於敕字之傍別加一點,又與玉麟符合者,當就徵。」及發書無驗,諒覺變,詰通,通占對無所屈,竟得歸長安。大業中,累轉左驍衛大將軍。時秦、隴盜賊蜂起,以通為關內討捕大使。有安定人
 劉迦論舉兵反,據雕陰郡,僭號建元,署置百官,有眾十餘萬。稽胡首領劉鷂子聚眾與迦論相影響。通發關中兵擊之,師臨安定,初不與戰,軍中以通為怯,通乃揚聲旋師而潛入上郡。迦論不之覺,遂進兵南寇,去通七十里而舍,分兵掠諸城邑。通候其無備,簡精甲夜襲之,賊眾大潰,斬迦論並首級萬餘,於上郡南山築為京觀,虜男女數萬口而還。



 煬帝幸江都,令通鎮長安。義兵起,代王遣通進屯河東。既而義師濟河,大破通將桑顯和於
 飲馬泉,永豐倉又為義師所克。通大懼,留鷹揚郎將堯君素守河東,將自武關趨藍田以赴長安。軍至潼關,為劉文靜所遏,不得進,相持月餘。通又令顯和夜襲文靜,詰朝大戰,義軍不利。顯和縱兵破二柵,惟文靜一柵獨存,顯和兵復入柵而戰者往覆數焉。文靜為流矢所中,義軍氣奪,垂至於敗。顯和以兵疲,傳餐而食,文靜因得分兵以實二柵。又有游軍數百騎自南山來擊其背,三柵之兵復大呼而出,表裏齊奮,顯和軍潰,僅以身免。悉
 虜其眾,通勢彌蹙。或說通歸降,通泣曰:「吾蒙國重恩,歷事兩主,受人厚祿,安可逃難?有死而已!」每自摩其頸曰:「要當為國家受人一刀耳!」勞勉將士,未嘗不流涕,人亦以此懷之。高祖遣其家僮召之,通遽命斬之。通聞京師平,家屬盡沒,乃留顯和鎮潼關,率兵東下,將趨洛陽。通適進路,而顯和降於劉文靜。遣副將竇琮、段志玄等率精騎與顯和追之,及於稠桑。通結陣以自固,竇琮縱通子壽令往諭之。通大呼曰:「昔與汝為父子,今與汝為仇
 讎。」命左右射之。顯和呼其眾曰:「京師陷矣,汝並關西人,欲何所去?」眾皆釋仗。通知不免,乃下馬東南向再拜號哭,曰:「臣力屈兵敗,不負陛下,天地神祗,實所鑒察。」遂擒通送於長安。高祖謂曰:「何相見晚耶?」通泣對曰:「通不能盡人臣之節,力屈而至,為本朝之辱,以愧代王。」高祖曰:「隋室忠臣也。」命釋之,授兵部尚書,封蔣國公,仍為太宗行軍元帥長史。



 從平薛舉,時珍物山積,諸將皆爭取之,通獨無所犯。高祖聞而謂曰:「公清正奉國,著自終始,名
 下定不虛也。」特賜金銀六百兩、彩物一千段。尋以本官判陜東道行臺僕射,復從太宗討王世充。時通有二子並在洛陽,高祖謂通曰:「東征之事,今以相屬,其如兩子何?」通對曰:「臣以老朽,誠不足以當重任。但自惟疇昔,執就軍門,至尊釋其縲囚,加之恩禮,既不能死,實荷再生。當此之時,心口相誓,暗以身命奉許國家久矣。今此行臣願先驅,兩兒若死,自是其命,終不以私害義。」高祖嘆息曰:「徇義之夫,一至於此!」及大兵圍洛陽,竇建德且至,
 太宗中分麾下以屬通,令與齊王元吉圍守洛陽。世充平,通功為第一,尋拜陜東大行臺右僕射,鎮於洛陽。數歲,徵拜刑部尚書,通自以不習文法,固辭之,轉工部尚書。隱太子之誅也,通復檢校行臺僕射,馳鎮洛陽。貞觀元年,行臺廢,授洛州都督,賜實封六百戶,加左光祿大夫。明年,卒,年七十二。太宗痛惜久之,贈尚書右僕射,謚曰忠。子壽襲爵。太宗幸洛陽宮,思通忠節,拜其少子詮果毅都尉,賜束帛以恤其家焉。十七年,詔圖形於凌煙
 閣。二十三年,與房玄齡配饗太宗廟庭。永徽五年,重贈司空。詮官至瀛州刺史。詮子仲翔,神龍中亦為瀛州刺史。



 任瑰,字瑋,廬州合肥人,陳鎮東大將軍蠻奴弟之子也。父七寶,仕陳定遠太守。瑰早孤,蠻奴愛之,情逾己子,每稱曰:「吾子侄雖多,並傭保耳,門戶所寄,惟在於瑰。」年十九,試守靈溪令。俄遷衡州司馬,都督王勇甚敬異之,委以州府之務。屬隋師滅陳,瑰勸勇據嶺南,求陳氏子孫
 立以為帝;勇不能用,以嶺外降隋,瑰乃棄官而去。仁壽中,為韓城尉,俄又罷職。



 及高祖討捕於汾、晉,瑰謁高祖於轅門,承制為河東縣戶曹。高祖將之晉陽,留隱太子建成以托於瑰。義師起,瑰至龍門謁見。高祖謂之曰:「隋氏失馭,天下沸騰。吾忝以外戚,屬當重寄,不可坐觀時變。晉陽是用武之地,士馬精強,今率驍雄以匡國難。卿將家子,深有智謀,觀吾此舉,將為濟否?」瑰曰:「後主殘酷無道,征役不息,天下恟恟,思聞拯亂。公天縱神武,親舉
 義師,所下城邑,秋毫無犯,軍令嚴明,將士用命。關中所在蜂起,惟待義兵。仗大順,從眾欲,何憂不濟?瑰在馮翊積年,人情諳練,願為一介之使,銜命入關,同州已東,必當款伏。於梁山船濟,直指韓城,進逼郃陽,分取朝邑。且蕭造文吏,本無武略,仰懼威靈,理當自下;孫華諸賊,未有適從,必當相率而至。然後鼓行整眾,入據永豐,雖未得京城,關中固已定矣。」高祖曰:「是吾心也。」乃授銀青光祿大夫,遣陳演壽、史大奈領步騎六千,趨梁山渡河,使
 瑰及薛獻為招慰大使。高祖謂演壽曰:「閫外之事,宜與任瑰籌之。」孫華、白玄度等聞兵且至,果競來降,並具舟於河,師遂利涉。瑰說下韓城縣,與諸將進擊飲馬泉,破之,拜左光祿大夫,留守永豐倉。



 高祖即位,改授穀州刺史。王世充數率眾攻新安,瑰拒戰破之,以功累封管國公。太宗率師討世充,瑰從至邙山,使檢校水運以供餉饋。關東初定,持節為河南道安撫大使。世充弟辯為徐州行臺尚書令,率所部詣瑰降。瑰至宋州,屬徐圓朗據
 兗州反,曹、戴諸州咸應之。副使柳浚勸瑰退保汴州,瑰笑曰:「柳公何怯也!老將居邊甚久,自當有計,非公所知。」圓朗俄又攻陷楚丘,引兵將圍虞城,瑰遣崔樞、張公謹自鄢陵領諸州豪右質子百餘人守虞城以拒賊。浚又諫曰:「樞與公謹並世充之將,又諸州質子父兄皆反,此必為變。」瑰不答。樞至,則分配質子,並與土人合隊居守。賊既稍近,質子有叛者,樞因斬其隊帥。城中人懼曰:「質子父兄悉來為賊,賊之子弟安可守城?」樞因縱諸隊各
 殺質子,梟首於門外,遣使報瑰。瑰陽怒曰:「遣將去者,欲招慰耳,何罪而殺之?」退謂浚曰:「固知崔樞辦之。既遣縣人殺賊質子,冤隙已大,吾何患焉?」樞果拒卻圓朗。事平,遷徐州總管,仍為大使。



 瑰選補官吏,頗私親故,或依倚其勢,多所求納,瑰知而不禁;又,妻劉氏妒悍無禮,為世所譏。及輔公祏平,拜邢州都督。隱太子之誅也,瑰弟璨,時為典膳監,瑰坐左遷通州都督。貞觀三年卒。



 丘和,河南洛陽人也。父壽,魏鎮東將軍。和少便弓馬,重
 氣任俠。及長,始折節,與物無忤,無貴賤皆愛之。周為開府儀同三司。入隋,累遷右武衛將軍,封平城郡公。漢王諒之反也,以和為蒲州刺史。諒使兵士服婦人服,戴冪旂,奄至城中,和脫身而免,由是除名。時宇文述方被任遇,和傾心附之,又以發武陵公元胄罪,拜代州刺史。屬煬帝北巡過代州,和獻食甚精,及至朔州,刺史楊廓獨無所獻,帝不悅,而宇文述又盛稱之,乃以和為博陵太守,仍令楊廓至博陵觀和為式。及駕至博陵,和上食又
 豐,帝益稱之。由是所幸處獻食者競為華侈。和在郡善撫吏士,甚得歡心,尋遷天水郡守。大業末,以海南僻遠,吏多侵漁,百姓咸怨,數為亂逆,於是選淳良太守以撫之。黃門侍郎裴矩奏言:「丘和歷居二郡,皆以惠政著聞,寬而不擾。」煬帝從之,遣和為交趾太守。既至,撫諸豪傑,甚得蠻夷之心。



 會煬帝為化及所弒,鴻臚卿寧長真以鬱林、始安之地附於蕭銑;馮盎以蒼梧、高涼、珠崖、番禺之地附於林士弘。各遣人召之,和初未知隋亡,皆不就。
 林邑之西諸國,並遣遺和明珠、文犀、金寶之物,富埒王者。銑利之,遣長真率百越之眾渡海侵和,和遣高士廉率交、愛首領擊之,長真退走,境內獲全,郡中樹碑頌德。會舊驍果從江都還者,審知隋滅,遂以州從銑。及銑平,和以海南之地歸國。詔使李道裕即授上柱國、譚國公、交州總管。和遣司馬高士廉奉表請入朝,詔許之。高祖遣其子師利迎之。及謁見,高祖為之興,引入臥內,語及平生,甚歡,奏《九部樂》以饗之,拜左武候大將軍。和時年
 已衰老,乃拜稷州刺史,以是本鄉,令自怡養。九年,除特進。貞觀十一年卒,年八十六。贈荊州總管,謚曰襄,賜東園秘器,陪葬獻陵。有子十五人,多至大官,惟行恭知名。



 行恭善騎射,勇敢絕倫。大業末,與兄師利聚兵於岐、雍間。有眾一萬,保故郿城,百姓多附之,群盜不敢入境。初,原州奴賊數萬人圍扶風,郡太守竇璡堅守,經數月,賊中食盡,野無所掠,眾多離散,投行恭者千餘騎。行恭遣其酋渠說諸奴賊共迎義軍。行恭又率五百人,皆負米
 麥,持牛酒,自詣賊營。奴帥長揖,行恭手斬之,謂其眾曰:「汝等並是好人,何因事奴為主,使天下號為奴賊?」眾皆俯伏曰:「願改事公。」行恭率其眾與師利共謁太宗於渭北,拜光祿大夫。從平京城,討薛舉、劉武周、王世充、竇建德,皆立殊勛,授左一府驃騎,賞賜甚厚。隱太子之誅也,行恭以功遷左衛將軍。貞觀中,坐與嫡兄爭葬所生母,為法司所劾,除名。因從侯君集平高昌,封天水郡公,累除右武候將軍。高宗嗣位,歷遷右武侯大將軍、冀陜二
 州刺史。尋請致仕,拜光祿大夫。麟德二年卒,年八十。贈荊州都督,謚曰襄,賜溫明秘器,陪葬昭陵。



 行恭性嚴酷,所在僚列皆懾憚之,數坐事解免。太宗每思其功,不逾時月復其官。初,從討王世充,會戰於邙山之上。太宗欲知其虛實強弱,乃與數十騎沖之,直出其後,眾皆披靡,莫敢當其鋒,所殺傷甚眾。既而限以長堤,與諸騎相失,惟行恭獨從。尋有勁騎數人追及太宗,矢中御馬;行恭乃回騎射之,發無不中,餘賊不敢復前。然後下馬拔箭,
 以其所乘馬進太宗。行恭於御馬前步執長刀,巨躍大呼,斬數人,突陣而出,得入大軍。貞觀中,有詔刻石為人馬以象行恭拔箭之狀,立於昭陵闕前。



 子神勣,嗣聖元年,為左金吾將軍,則天使於巴州,害章懷太子,既而歸罪於神勣,左遷疊州刺史。尋復入為左金吾衛大將軍,深見親委。嘗受詔鞫獄,與周興、來俊臣等俱號為酷吏。尋以罪伏誅。神龍初,禁錮其子孫。



 和少子行掩,高宗時為少府監。



 許紹,字嗣宗,本高陽人也,梁末徙於周,因家於安陸。祖弘,父法光,俱為楚州刺史。元皇帝為安州總管,故紹兒童時得與高祖同學,特相友愛。大業末,為夷陵郡通守。是時盜賊競起,紹保全郡境,流戶自歸者數十萬口,開倉賑給,甚得人心。及江都弒逆,紹率郡人大臨三日,仍以郡遙屬越王侗。王世充篡位,乃率黔安、武陵、澧陽等諸郡遣使歸國,授硤州刺史,封安陸郡公。高祖降敕書曰:「昔在子衿,同游庠序,博士吳琰,其妻姓仇,追想此時,
 宛然心目,荏苒歲月,遂成累紀。且在安州之日,公家乃蒞岳州;渡遼之時,伯裔又同戎旅。安危契闊,累葉同之,其間游處,觸事可想。雖盧綰與劉邦同里,吳質共曹丕接席,以今方古,何足稱焉!而公追硯席之舊歡,存通家之曩好,明鑒去就之理,洞識成敗之機。爰自荊門,馳心絳闕,綏懷士庶,糾合賓僚,逾越江山,遠申誠款。覽此忠至,彌以慰懷。」及蕭銑將董景珍以長沙來降,命紹率兵應之。以破銑功,拜其子智仁為溫州刺史,委以招慰。時
 蕭銑遣其將楊道生圍硤州,紹縱兵擊破之。銑又遣其將陳普環乘大艦溯江入硤,與開州賊蕭闍提規取巴蜀。紹遣智仁及錄事參軍李弘節、子婿張玄靜追至西陵硤,大破之,生擒普環,收其船艦。江南岸有安蜀城,與硤州相對,次東有荊門城,皆險峻,銑並以兵鎮守。紹遣智仁及李弘節攻荊門鎮,破之。高祖大悅,下制褒美,許以便宜從事。紹與王世充、蕭銑疆界連接,紹之士卒為賊所虜者,輒見殺害。紹執敵人,皆資給而遣之,賊感其義,
 不復侵掠,闔境獲安。趙郡王孝恭之擊蕭銑也,復令紹督兵以圖荊州,會卒於軍,高祖聞而流涕。貞觀中,贈荊州都督。嫡孫力士襲爵,官至洛州長史,卒。



 子欽寂嗣,萬歲登封年為夔州都督府長史。時契丹入寇,以欽寂兼龍山軍討擊副使,軍次崇州,戰敗被擒。其後,賊將圍安東,令欽寂說屬城之未下者。安東都督裴玄珪時在城下,欽寂謂之曰:「狂賊天殃,滅在朝夕,公但謹守勵兵,以全忠節。」賊大怒,遂害之。則天下制褒美,贈蘄州刺史,
 謚曰忠。又授其子輔乾左監門衛中候,仍為海東慰勞使;令迎其喪柩,以禮改葬。輔乾,開元中官至光祿卿。



 欽寂弟欽明,少以軍功歷左玉鈐衛將軍、安西大都護,封鹽山郡公。萬歲通天元年,授金紫光祿大夫、涼州都督。欽明嘗出按部,突厥默啜率眾數萬奄至城下,欽明拒戰。久之,力屈被執。賊將欽明至靈州城下,令說城中早降,欽明大呼曰:「賊中都無飲食,城內有美醬,乞二升,粱米乞二斗,墨乞一梃。」是時,賊營處四面阻泥河,惟有一
 路得入,欽明乞此物以喻城中,冀其簡兵陳將,候夜掩襲,城中無悟其旨者,尋遇害。兄弟同年皆死王事,論者稱之。



 紹次子智仁,初,以父勛授溫州刺史,封孝昌縣公。尋繼其父為硤州刺史,後歷太僕少卿、涼州都督。貞觀中卒。



 紹少子圉師,有器幹,博涉藝文,舉進士。顯慶二年,累遷黃門侍郎、同中書門下三品,兼修國史。三年,以修實錄功封平恩縣男,賜物三百段。四遷,龍朔中為左相。俄以子自然因獵射殺人,隱而不奏,又為李義府所擠,
 左遷虔州刺史。尋轉相州刺史。政存寬惠,人吏刊石以頌之。嘗有官吏犯贓事露,圉師不令推究,但賜清白詩以激之,犯者愧懼,遂改節為廉士,其寬厚如此。上元中,再遷戶部尚書。儀鳳四年卒,贈幽州都督,陪葬恭陵,謚曰簡。



 李襲志,字重光,本隴西狄道人也。五葉祖景避地安康,復稱金州安康人也。周信州總管、安康郡公遷哲孫也。父敬猷,隋臺州刺史、安康郡公。襲志,初任隋歷始安郡
 丞。大業末,江外盜賊尤甚,襲志散家產,招募得三千人,以守郡城。時蕭銑、林士弘、曹武徹等爭來攻擊,襲志固守久之。後聞宇文化及弒逆,乃集士庶舉哀三日。有郡人勸襲志曰;「公累葉冠族,久臨鄙郡,蠻夷畏威,士女悅服,雖曰隋臣,實我之君長。今江都篡逆,四海鼎沸,王號者非止一人,公宜因此時據有嶺表,則百越之人皆拱手向化。追蹤尉佗,亦千載一遇也。」襲志厲聲曰:「吾世樹忠貞,見危授命,今雖江都陷沒,而宗社猶存,當與諸君戮
 力中原,共雪仇恥,豈可怙亂稱兵,以圖不義!吾寧蹈忠而死,不為逆節而求生。尉佗愚鄙無識,何足景慕?」於是欲斬勸者,從眾議而止。襲志固守,經二年而無援,卒為蕭銑所陷,銑署為工部尚書、檢校桂州總管。武德初,高祖遣其子玄嗣齎書召之,襲志乃密說嶺南首領隨永平郡守李光度與之歸國。高祖又令間使齎書諭襲志曰:「卿昔久在桂州,仍屬隋室運終,四方圮絕,率眾保境,未知所統。朕撫臨天下,志在綏育,眷彼幽遐,思沾聲
 教。況卿朕之宗姓,情異於常。家弟侄並立誠效公,又分遣首領,申諭諸州,情深奉國,甚副所望。卿之子弟,並據州縣,俱展誠績,每所嘉嘆,不能已已。令並入屬籍,著於宗正。」及蕭銑平,江南道大使、趙郡王孝恭授襲志桂州總管。武德五年入朝,授柱國,封始安郡公,拜江州都督。及輔公祏反,又以襲志為水軍總管討平之,轉桂州都督。襲志前後凡任桂州二十八載,政尚清簡,嶺外安之。後表請入朝,拜右光祿大夫、行汾州刺史致仕,卒於家。
 襲志弟襲譽。



 襲譽,字茂實,少通敏,有識度。隋末為冠軍府司兵。時陰世師輔代王為京師留守,所在盜賊蜂起,襲譽說世師遣兵據永豐倉,發粟以賑窮乏,出庫物賞戰士,移檄郡縣,同心討賊。世師不能用,乃求外出募山南士馬,世師許之。既至漢中,會高祖定長安,召授太府少卿,封安康郡公,仍令與兄襲志附籍於宗正。太宗討王世充,以襲譽為潞州總管。時突厥與國和親,又通使於世充,襲譽掩擊,悉斬之。因委令轉運以饋大軍。後歷
 光祿卿、浦州刺史,轉揚州大都督府長史,為江南道巡察大使,多所黜陟。江都俗好商賈,不事農桑。襲譽乃引雷陂水,又築勾城塘,溉田八百餘頃,百姓獲其利。召拜太府卿。襲譽性嚴整,所在以威肅聞。凡獲俸祿,必散之宗親,其餘資多寫書而已。及從揚州罷職,經史遂盈數車。嘗謂子孫曰:「吾近京城有賜田十頃,耕之可以充食;河內有賜桑千樹,蠶之可以充衣;江東所寫之書,讀之可以求官。吾沒之後,爾曹但能勤此三事,亦何羨於人!」
 尋轉涼州都督,加金紫光祿大夫,行同州刺史。坐在涼州陰憾番禾縣丞劉武,杖而殺之,至是有司議當死,制除名,流於泉州,無幾而卒。撰《五經妙言》四十卷、《江東記》三十卷、《忠孝圖》二十卷。



 兄子懷儼,頗以文才著名。歷蘭臺侍郎,受制檢校寫四部書進內,以書有汙,左授郢州刺史。後卒於禮部侍郎。



 姜抃,秦州上邽人。祖真,後魏南秦州刺史。父景,周梁州總管、建平郡公。抃,大業末為晉陽長,會高祖留守太原,
 見抃深器之。抃退謂所親曰:「隋祚將亡,必有命世大才,以應圖籙,唐公有霸王之度,以吾觀之,必為撥亂之主。」由是深自結納。及大將軍府建,引為司功參軍。從平霍邑,拔絳郡,監督大軍濟河。時兵士爭渡,抃部勒諸軍,自昏至曉,六軍畢濟。高祖稱嘆之。平京城,除相國兵曹參軍,封長道縣公。時薛舉寇秦、隴,以抃西州之望,詔於隴右安撫,承制以便宜從事。抃將行,奏曰:「天人之望,誠有所歸,願早膺圖籙,以寧兆庶。老夫犬馬暮齒,恐先朝露,
 得一睹升紫殿,死無所恨。」高祖大悅。抃與竇軌出散關,下河池、漢陽二郡。軍次長道,與薛舉相遇,軌輕敵,為舉所敗。征抃還京,拜員外散騎常侍。及平薛仁杲,拜抃秦州刺史,高祖謂曰:「衣錦還鄉,古人所尚;今以本州相授,用答元功。涼州之路,近為荒梗,宜弘方略,有以靜之。」抃至州,撫以恩信,州人相謂曰:「吾輩復見太平官府矣。」盜賊悉來歸首,士庶安之。尋轉隴州刺史。七年,以老疾去職。貞觀元年卒,贈岷州都督,謚曰安。



 子行本,貞觀中為
 將作大匠。太宗修九成、洛陽二宮,行本總領之,以勤濟稱旨,賞賜甚厚。有所游幸,未嘗不從。又轉左屯衛將軍。時太宗選趫捷之士,衣五色袍,乘六閑馬,直屯營以充仗內宿衛,名為「飛騎」,每游幸,即騎以從,分隸於行本。及高昌之役,以行本為行軍副總管,率眾先出伊州。未至柳穀百餘里,依山造攻具。其處有班超紀功碑,行本磨去其文,更刻頌,陳國威德而去。遂與侯君集進平高昌,璽書勞之曰:「攻戰之重,器械為先,將士屬心,待以制敵。
 卿星言就路,躬事修營,干戈才動,梯沖暫臨。三軍勇士,因斯樹績;萬里逋寇,用是克平。方之前古,豈足相況!」及還,進封金城郡公,賜物一百五十段、奴婢七十人。十七年,太宗將征高麗,行本諫以為師未可動,太宗不從。行本從至蓋牟城,中流矢卒。太宗賦詩以悼之,贈左衛大將軍、郕國公,謚曰襄,陪葬昭陵。



 子簡嗣,永徽中,官至安北都護,卒。子晞嗣,開元初左散騎常侍。



 簡弟柔遠,美姿容,善於敷奏。則天時,至左鷹揚衛將軍、通事舍人、內供
 奉。



 柔遠子皎,長安中,累遷尚衣奉御。時玄宗在籓,見而悅之。皎察玄宗有非常之度,尤委心焉。尋出為潤州長史。玄宗即位,召拜殿中少監。數召入臥內,命之舍敬,曲侍宴私,與后妃連榻,間以擊球鬥雞,常呼之為姜七而不名也。兼賜以宮女、名馬及諸珍物不可勝數。玄宗又嘗與皎在殿庭玩一嘉樹,皎稱其美,玄宗遽令徙植於其家,其寵遇如此。及竇懷貞等潛謀逆亂,玄宗將討之,皎協贊謀議,以功拜殿中監,封楚國公,實封四百戶。玄
 宗以皎在籓之舊,皎又有先見之明,欲宣布其事,乃下敕曰:



 朕聞士之生代,始於事親,中於事君,終於立身,此其本也。若乃移孝成忠,策名委質。命有太山之重,義徇則為輕;草有疾風之力,節全則知勁。況君臣之相遇,而故舊之不遺乎!銀青光祿大夫、殿中監、楚國公姜皎,簪紱聯華,珪璋特秀。寬厚為量,體靜而安仁;精微用心,理和而專直。往居籓邸,潛款風雲,亦由彭祖之同書,子陵之共學。朕常游幸於外,至長楊、鄠杜之間,皎於此時與
 之累宿,私謂朕曰:「太上皇即登九五,王必為儲副。」凡如此者數四,朕叱而後止。寧知非僕,雖玩於鄧晨;可收護軍,遂訶於硃祐。皎復言於朕兄弟及諸駙馬等,因聞徹太上皇,太上皇遽奏於中宗孝和皇帝。尋遣嗣虢王邕等鞫問,皎保護無怠,辭意轉堅。李通之讖記不言,田叔之髡鉗罔憚。仍為宗楚客、紀處訥等密奏,請投皎炎荒。中宗特降恩私,左遷潤州長史。讒邪每構,忠懇逾深,戴於朕躬,憂存王室。以為天且有命,預睹成龍之徵;人而
 無禮,常懷逐鳥之志。游辭枉陷,旋罹貶斥;嚴憲將及,殆見誅夷。履危本於初心,遭險期於不貳,雖禍福之際昭然可圖,而艱難之中是所繄賴。洎朕祗膺寶位,又共翦奸臣,拜以光寵,不忘捴挹,敬愛之極,神明所知。造膝則曾莫詭隨,匪躬則動多規諫,補朕之闕,斯人孔臧。而悠悠之談,嗷嗷妄作,醜正惡直,竊生於謗,考言詢事,益亮其誠。昔漢昭帝之保霍光,魏太祖之明程昱,朕之不德,庶幾於此。矧夫否當其悔,則滅宗毀族,朕負之必深;泰
 至其亨,則如山如河,朕酬之未補。豈流言之足聽,而厚德之遂忘?謀始有之,圖終可也。宜告示中外,咸令知悉。



 尋遷太常卿,監修國史。弟晦,又歷御史中丞、吏部侍郎,兄弟當朝用事。侍中宋璟以其權寵太盛,恐非久安之道,屢奏請稍抑損之。開元五年下敕曰:「西漢諸將,多以權貴不全;南陽故人,並以優閑自保。觀夫先後之跡,吉兇之數,較然可知,良有以也。太常卿、上柱國、楚國公、監修國史姜皎,衣纓奕代,忠讜立誠,精識比於橋玄,密私
 方於硃祐。朕昔在籓邸,早申款洽,當謂我以不遺,亦起予以自愛。及膺大位,屢錫崇班,茅土列爵,山河傳誓,備蒙光寵,時冠等夷。朕每欲戒盈,用克終吉。未若避榮公府,守靖私第,自弘高尚之風,不涉囂塵之境,沐我恩貸,庇爾子孫。宜放歸田園,以恣娛樂。」又遷晦為宗正卿,以去其權。久之,皎復起為秘書監。十年,坐漏洩禁中語,為嗣濮王嶠所奏,敕中書門下究其狀。嶠,即王守一之妹夫;中書令張嘉貞希守一意,構成其罪,仍奏請先決杖
 配流嶺外。下制曰:「秘書監姜皎,往屬艱難,頗效誠信,功則可錄,寵是以加。既忘滿盈之誡,又虧靜慎之道,假說休咎,妄談宮掖。據其作孽,合處極刑,念茲舊勛,免此殊死。宜決一頓,配流欽州。」皎既決杖,行至汝州而卒,年五十餘。皎之所親都水使者劉承祖,配流雷州,自餘流死者數人。時朝廷頗以皎為冤,而咎嘉貞焉。源乾曜時為侍中,不能有所持正,論者亦深譏之。玄宗復思皎舊勛,令遞其柩還,以禮葬之,仍遣中使存問其家。十五年,追
 贈澤州刺史。晦坐皎左遷春州司馬,俄遷海州刺史,卒。



 天寶六載,授皎男慶初等官。七載,贈皎吏部尚書,仍贈實封二百戶以充享祀。慶初襲封楚國公。慶初生未晬,玄宗許尚公主,後淪落二十餘年。李林甫為相,當軸用事,林甫即皎之甥,從容奏之,故驟加恩命。天寶十載,詔慶初尚新平公主,授駙馬都尉。永泰元年,拜太常卿。



 史臣曰:或問屈突通盡忠於隋而功立於唐,事兩國而名愈彰者,何也?答云,若立純誠,遇明主,一心可事百君,
 寧限於兩國爾!被稠桑之擒,臨難無茍免;破仁杲之眾,臨財無茍得,君子哉!任瑰、丘和、許紹、李襲志咸遇真主,得為故人,或敘舊立功,或率眾歸國。尋其履跡,皆有可稱。襲志為政,襲譽訓子,庶幾弘遠矣。姜抃恩信,有能官之譽;行本勤濟,多克敵之功。皎雖故舊,恩幸不倫,雖嘉貞致冤,亦冒寵自掇,豈非無德而祿,福過災生之驗歟!任瑰縱妒妻無禮,任親戚求財,丘和進食邀幸,皆無取焉。



 贊曰:屈突守節,求仁得仁。諸君遇主,不足擬倫。



\end{pinyinscope}