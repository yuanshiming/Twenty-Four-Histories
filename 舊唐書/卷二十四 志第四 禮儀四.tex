\article{卷二十四 志第四 禮儀四}

\begin{pinyinscope}

 武德、貞觀之制,神祇大享之外,每歲立春之日,祀青帝於東郊,帝宓羲配,勾芒、歲星、三辰、七宿從祀。立夏,祀赤帝於南郊,帝神農氏配,祝融、熒惑、三辰、七宿從祀。季夏土王日,祀黃
 帝於南郊,帝軒轅配,後土、鎮星從祀。立秋,祀白帝於西郊,帝少昊配,蓐收、太白、三辰、七宿從祀。立冬,祀黑帝於北郊,帝顓頊配,玄冥、辰星、三辰、七宿從祀。每郊帝及配座,用方色犢各一,籩、豆各四,簠、簋各二,、俎各一。勾芒已下五星及三辰、七宿,每宿牲用少牢,每座籩、豆、簠、簋、、俎各一。孟夏之月,龍星見,雩五方上帝於雩壇,五帝配於上,五官從祀於下。牲用方色犢十,籩豆已下,如郊祭之數。帝嚳,祭於頓丘。唐堯,契配,祭於平陽。虞舜,咎繇配,祭於河東。夏禹,伯益配,祭於安邑。殷湯,伊尹配,祭於偃師。周文王,太公配,祭於邦。周武王、周公、召公配,祭於鎬。漢高祖,蕭何配,祭於長陵。三年一祭,以仲春之月。牲皆用太牢。祀官以當界州長官,有故,遣上佐行事。



 五岳、四鎮、四海、四
 瀆,年別一祭,各以五郊迎氣日祭之。東嶽岱山,祭於祇州;東鎮沂山,祭於沂州;東海,於萊
 州;東瀆大淮,於唐州。南岳衡山,於衡州;南鎮會稽,於越州;南海,於廣州;南瀆大江,於益州。中嶽嵩山,於洛州。西嶽華山,於華州;西鎮吳山,於隴州;西海、西瀆大河,於同州。北岳恆
 山,於
 定州;北鎮醫無閭山,於營州;北海、北瀆大濟,於洛州。其牲皆用太牢,籩、豆各四。祀官以當界都督刺史充。



 仲春、仲秋
 二時戊日,祭太社、太稷,社以勾龍
 配,稷以後稷配。社、稷各用太牢一,牲色並黑,籩、豆、簠、簋各二,鉶、俎各三。春分,朝日於國城之東;秋分,夕月於國城之西。各用方色犢一,籩、豆各四,簠、簋、、俎各一。孟春吉亥,祭帝社於藉田,天子親耕;季春
 吉巳,祭
 先蠶於公桑,皇后親桑。並用太牢,籩、豆各九。將蠶日,內侍省預奉移所司所事。諸祭祀卜日,皆先卜上旬;不吉,次卜中旬、下旬。筮日亦如之。其先蠶一祭,節氣若晚,即於節氣後取日。立春後丑,祀風師於國城東北;立夏後申,祀雨師於國城西南;立秋後辰,祀靈星於國城東南;立冬後亥,祀司中、司命、司人、司祿於國城西北。各用羊一,籩、豆各二,簠、簋各一。季冬晦,堂贈儺,磔牲於宮門及城四門,各用雄雞一。仲春,祭馬祖;仲夏,祭先牧;仲秋,祭
 馬社;仲冬,祭馬步。並於大澤,用剛日。牲各用羊一,籩、豆各二,簠、簋各一。季冬藏冰,仲春開冰,並用黑牡、秬黍,祭司寒之神於冰室,籩、豆各二,簠、簋、俎各一。其開冰,加以桃弧棘矢,設於神座。



 季冬寅日,蠟祭百神於南郊。大明、夜明,用犢二,籩、豆各四,簠、簋、、俎各一。神農氏及伊耆氏,各用少牢一,籩、豆各四,簠、簋、、俎各一。後稷及五方、十二次、五官、五方田畯、五岳、四鎮、四海、四瀆以下,方別各用少牢一,當方不熟者則闕之。其日祭井泉於川澤
 之下,用羊一。卯日祭社稷於社宮,辰日臘享於太廟,用牲皆準時祭。井泉用羊二。二十八宿,五方之山林、川澤,五方之丘陵、墳衍、原隰,五方之鱗、羽、臝、毛、介,五方之水墉、坊、郵表畷,五方之貓、於菟及龍、麟、硃鳥、白虎、玄武,方別各用少牢一,各座籩、豆、簠、簋、俎各一。蠟祭凡一百八十七座。當方年穀不登,則闕其祀。蠟祭之日,祭五方井泉於山澤之下,用羊一,籩、豆各二,簠、簋、及俎各一。蠟之明日,又祭社稷於社宮,如春秋二仲之禮。



 顯慶中,更定
 籩、豆之數,始一例。大祀籩、豆各十二,中祀各十,小祀各八。



 京師孟夏以後旱,則祈雨,審理冤獄,賑恤窮乏,掩骼埋胔。先祈嶽鎮、海瀆及諸山川能出雲雨,皆於北郊望而告之。又祈社稷,又祈宗廟,每七日皆一祈。不雨,還從岳瀆。旱甚,則大雩,秋分後不雩。初祈後一旬不雨,即徙市,禁屠殺,斷傘扇,造土龍。雨足,則報祀。祈用酒醢,報準常祀,皆有司行事。已齊未祈而雨,及所經祈者,皆報祀。若霖雨不已,禜京城諸門,門別三日,每日一禜。不止,乃祈山川、岳鎮、海瀆;三日不止,祈
 社稷、宗廟。其州縣,禜城門;不止,祈界內山川及社稷。三禜、一祈,皆準京式,並用酒脯醢。國城門報用少牢,州縣城門用一特牲。



 太宗貞觀三年正月,親祭先農,躬御耒耜,藉於千畝之甸。初,晉時南遷,後魏來自雲、朔,中原分裂,又雜以犬熏戎,代歷周、隋,此禮久廢,而今始行之,觀者莫不駭躍。於是秘書郎岑文本獻《藉田頌》以美之。初,議藉田方面所在,給事中孔穎達曰:「禮,天子藉田於南郊,諸侯於東郊。晉武帝猶於東南。今於城東置壇,不合古禮。」太
 宗曰:「禮緣人情,亦何常之有。且《虞書》云『平秩東作』,則是堯、舜敬授人時,已在東矣。又乘青輅、推黛耜者,所以順於春氣,故知合在東方。且朕見居少陽之地,田於東郊,蓋其宜矣」於是遂定。自後每歲常令有司行事。則天時,改藉田壇為先農。神龍元年,禮部尚書祝欽明與禮官等奏曰:「謹按經典,無先農之文。《禮記·祭法》云:『王自為立社,曰王社。』先儒以為社在藉田,《詩》之《載芟篇序》云『春藉田而祈社稷』是也。永徽年中猶名藉田,垂拱已後刪定,
 改為先農。先農與社,本是一神,頻有改張,以惑人聽。其先農壇請改為帝社壇,以應禮經王社之義。其祭先農既改為帝社壇,仍準令用孟春吉亥祠后土,以勾龍氏配。」制從之。於是改先農為帝社壇,於壇西立帝稷壇,禮同太社、太稷,其壇不備方色,所以異於太社也。睿宗太極元年,親祀先農,躬耕帝藉。禮畢,大赦,改元。



 玄宗開元二十二年冬,禮部員外郎王仲丘又上疏請行藉田之禮。二十三年正月,親祀神農於東郊,以勾芒配。禮畢,躬
 御耒耜於千畝之甸。時有司進儀注:「天子三推,公卿九推,庶人終畝。」玄宗欲重勸耕藉,遂進耕五十餘步,盡壟乃止。禮畢,輦還齋宮,大赦。侍耕、執牛官皆等級賜帛。玄宗開元二十六年,又親往東郊迎氣,祀青帝,以勾芒配,歲星及三辰七宿從祀。其壇本在春明門外,玄宗以祀所隘狹,始移於滻水之東面,而值望春宮。其壇一成,壇上及四面皆青色。勾芒壇在東南。歲星已下各為一小壇,在青壇之北。親祀之時,有瑞雪,壇下侍臣及百僚
 拜賀稱慶。



 肅宗乾元二年春正月丁丑,將有事於九宮之神,兼行藉田禮。自明鳳門出,至通化門,釋軷而入壇,行宿齋於宮。戊寅,禮畢,將耕藉,先至於先農之壇。因閱耒耜,有雕刻文飾,謂左右曰:「田器,農人執之,在於樸素,豈文飾乎?」乃命徹之。下詔曰:「古之帝王,臨御天下,莫不務農敦本,保儉為先,蓋用勤身率下也。屬東耕啟候,爰事藉田,將欲勸彼蒸人,所以執茲耒耜。如聞有司所造農器,妄加雕飾,殊匪典章。況紺轅縹軏,固前王有制,崇奢尚靡,諒為政所疵。靖言思之,良用嘆息,豈朕法堯舜、重茅茨之意耶!其
 所造雕飾者宜停。仍令有司依農用常式,即別改造,庶萬方黎庶,知朕意焉。」翌日己卯,致祭神農氏,以後稷配享。肅宗冕而硃紘,躬秉耒耜而九推焉。禮官奏陛下合三推,今過禮。肅宗曰:「朕以身率下,自當過之,恨不能終於千畝耳。」既而佇立久之,觀公卿、諸侯、王公已下耕畢。



 太宗貞觀十四年春正月庚子,命有司讀春令,詔百官之長,升太極殿列坐面聽之。開元二十六年,玄宗命太常卿韋絳每月進《月令》一篇。是後每孟月視日,玄宗御
 宣政殿,側置一榻,東面置案,命韋絳坐而讀之。諸司官長,亦升殿列座而聽焉。歲餘,罷之。乾元元年十二月丙寅立春,肅宗御宣政殿,命太常卿於休烈讀春令。常參官五品已上正員,並升殿預坐而聽之。舊儀,岳瀆已下,祝版御署訖,北面再拜。證聖元年,有司上言曰:「伏以天子父天而母地,兄日而姊月,於禮應敬,故有再拜之儀。謹按五岳視三公,四瀆視諸侯,天子無拜公侯之禮,臣愚以為失尊卑之序。其日月已下,請依舊儀。五岳已下,
 署而不拜。」制可,從之。



 貞觀之禮,無祭先代帝王之文。顯慶二年六月,禮部尚書許敬宗等奏曰:「謹案《禮記·祭法》云:『聖王之制祀也,法施於人則祀之,以死勤事則祀之,以勞定國則祀之,能御大災則祀之,能捍大患則祀之。」又:『堯、舜、禹、湯、文、武,有功烈於人,及日月星辰,人所瞻仰;非此族也,不在祀典』。準此,帝王合與日月同例,常加祭享,義在報功。爰及隋代,並遵斯典。漢高祖祭法無文,但以前代迄今,多行秦、漢故事。始皇無道,所以棄之。漢祖
 典章,法垂於後。自隋已下,亦在祠例。伏惟大唐稽古垂化,網羅前典,唯此一禮,咸秩未申。今請聿遵故事,三年一祭。以仲春之月,祭唐堯於平陽,以契配;祭虞舜於河東,以咎繇配;祭夏禹於安邑,以伯益配;祭殷湯於偃師,以伊尹配;祭周文王於邦,以太公配;祭武王於鎬,以周公、召公配;祭漢高祖於長陵,以蕭何配。



 玄宗開元二十二年正月,詔曰:「古聖帝明王、岳瀆海鎮,用牲牢,餘並以酒脯充奠祀。」二十三年正月,詔:「自今已後,明衣絹布,並
 祀前五日預給。」丁酉,詔:「自今已後,有大祭,宜差丞相、特進、開府、少保、少傅、尚書、御史大夫攝行事。」天寶六載正月,詔:「三皇、五帝,於京城置令,丞。」七載五月,詔:「三皇已前帝王,宜於京城共置廟官。歷代帝王肇跡之處,德業可稱者,忠臣義士、孝婦烈女,所在亦置一祠宇。晉陽真人等並追贈,得道升仙處,度道士永修香火。」九載九月,處士崔昌上《大唐五行應運歷》,以王者五十代而一千年,請國家承周、漢,以周、隋為閏。十一月,敕:「唐承漢後,其周
 武王、漢高祖同置一廟並官吏。」十二載九月,以魏、周、隋依舊為三王後,封韓公、介、酅公等,仍舊五廟。



 天寶六載正月,詔大祭祀騂犢,量減其數。肅宗上元元年閏四月,改元,制以歲儉,停中小祠享祭。至其年仲秋,復祠文宣於太學。永泰二年,春夏累月亢旱,詔大臣裴冕等十餘人,分祭川瀆以祈雨。禮儀使右常侍於休烈請依舊祠風伯、雨師於國門舊壇,復為中祠,從之。



 高祖武德二年,國子立周公、孔子廟。七年二月己酉,詔「諸州有明一
 經已上未被升擢者,本屬舉送,具以名聞,有司試策,皆加敘用。其吏民子弟,有識性明敏,志希學藝,亦具名申送,量共差品,並即配學。州縣及鄉,並令置學。」丁酉,幸國子學,親臨釋奠。引道士、沙門有學業者,與博士雜相駁難,久之乃罷。



 貞觀十四年三月丁丑,太宗幸國子學,親觀釋奠。祭酒孔穎達講《孝經》,太宗問穎達曰:「夫子門人,曾、閔俱稱大孝,而今獨為曾說,不為閔說,何耶?」對曰:「曾孝而全,獨為曾能達也。」制旨駁之曰:「朕聞《家語》云:曾皙使
 曾參鋤瓜,而誤斷其本,皙怒,援大杖以擊其背,手僕地,絕而復蘇。孔子聞之,告門人曰:『參來勿內。』既而曾子請焉,孔子曰:『舜之事父母也,使之,常在側;欲殺之,乃不得。小棰則受,大杖則走。今參於父,委身以待暴怒,陷父於不義,不孝莫大焉。』由斯而言,孰愈於閔子騫也?」穎達不能對。太宗又謂侍臣:「諸儒各生異意,皆非聖人論孝之本旨也。孝者,善事父母,自家刑國,忠於其君,戰陳勇,朋友信,揚名顯親,此之謂孝。具在經典,而論者多離其文,
 迥出事外,以此為教,勞而非法,何謂孝之道耶!」二十一年,詔曰:「左丘明、卜子夏、公羊高、穀梁赤、伏勝、高堂生、戴聖、毛萇、孔安國、劉向、鄭眾、杜子春、馬融、盧植、鄭玄、服虔、何休、王肅、王弼、杜預、範甯、賈逵總二十二座,春秋二仲,行釋奠之禮。」初,以儒官自為祭主,直云博士姓名,昭告於先聖。又州縣釋奠,亦以博士為主。敬宗等又奏曰:



 按《禮記·文王世子》:凡學,春官釋奠於其先師。」鄭注云:「官,謂《詩》、《書》、《禮》、《樂》之官也。」彼謂四時之學,將習其道,故儒官釋
 奠,各於其師。既非國學行體,所以不及先聖。至於春、秋二時合樂之日,則天子視學,命有司典秩,即總祭先聖、先師焉。秦、漢釋奠,無文可檢。至於魏武,則使太常行事。自晉、宋已降,時有親行,而學官主祭,全無典實。且名稱國學,樂用軒懸,樽俎威儀,蓋皆官備,在於臣下,理不合專。況凡在小神,猶皆遣使行禮,釋奠既準中祀,據理必須稟命。今請國學釋奠,令國子祭酒為初獻,祝辭稱「皇帝謹遣」,仍令司業為亞獻,國子博士為終獻。其州學,刺
 史為初獻,上佐為亞獻,博士為終獻。縣學,令為初獻,丞為亞獻,博士既無品秩,請主薄及尉通為終獻。若有闕,並以次差攝。州縣釋奠,既請各刺史、縣令親獻主祭,望準祭社,同給明衣。修附禮令,以為永則。



 高宗顯慶二年七月,禮部尚書許敬宗等議:「依令,周公為先聖,孔子為先師。又《禮記》云:『始立學,釋奠於先聖。』鄭玄注云:『若周公、孔子也。」且周公踐極,功比帝王,請配武王。以孔子為先聖。」二年,廢書、算、律學。龍朔二年正月,東都置國子監丞、
 主簿、錄事各一員,四門助教博士、四門生三百員,四門俊士二百員。二月,復置律及書、算學。三年,以書隸蘭臺,算隸秘閣局,律隸詳刑寺。乾封元年正月,高宗東封還,次鄒縣頓,祭宣父,贈太師。總章元年二月,皇太子弘幸國學,釋奠,贈顏回太子少師,曾參太子少保。儀鳳三年五月,詔:「自今已後,《道德經》並為上經,貢舉人皆須兼通。其餘經及《論語》,任依常式。」則天天授三年,追封周公為褒德王,孔子為隆道公。則天長壽二年,自制《臣軌》兩卷,
 令貢舉人為業,停《老子》。神龍元年,停《臣軌》,復習《老子》。以鄒、魯百戶封隆道公,謚曰文宣。睿宗景雲二年八月丁巳,皇太子釋奠於太學。太極元年正月,詔:「孔宣父祠廟,令本州修飾,取側近三十戶以供灑掃。」



 開元七年十月戊寅,皇太子詣國學,行齒胄之禮。開元十一年,春秋二時釋奠,諸州宜依舊用牲牢,其屬縣用酒脯而已。十九年正月,春秋二時社及釋奠,天下州縣等停牲牢,唯用酒脯,永為常式,二十四年三月,始移貢舉,遣禮部侍郎
 姚奕請進士帖《左傳》、《禮記》,通五及第。二十五年三月,敕:「明經自今已後,貼十通五已上;口問大義十條,取通六已上;仍答時務策三道,取粗有文理者及第。進士停帖小經,宜準明經例試大經,帖十通四,然後試雜文及策,訖,封所試雜文及策,送中書、門下詳覆。」二十六年正月,敕:「諸州鄉貢見訖,令引就國子監謁先師,學官為之開講,質問疑義,有司設食。弘文、崇文兩館學生及監內得舉人,亦聽預焉。」其日,祀先聖已下,如釋奠之禮。青宮五
 品已下及朝集使,就監觀禮,遂為常式,每年行之至今。



 初,開元八年,國子司業李元瓘奏稱:「先聖孔宣父廟,先師顏子配座,今其像立侍,配享合坐。十哲弟子,雖復列像廟堂,不預享祀。謹檢祠令:何休、範甯等二十二賢,猶沾從祀,望請春秋釋奠,列享在二十二賢之上。七十子,請準舊都監堂圖形於壁,兼為立贊,庶敦勸儒風,光崇聖烈。曾參等道業可崇,獨受經於夫子,望準二十二賢預饗。」敕改顏生等十哲為坐像,悉預從祀。曾參大孝,德冠同
 列,特為塑像,坐於十哲之次。圖畫七十子及二十二賢於廟壁上。以顏子亞聖,上親為之贊,以書於石。閔損已下,令當朝文士分為之贊。二十七年八月,又下制曰:



 弘我王化,在乎儒術。孰能發揮此道,啟迪含靈,則生人已來,未有如夫子者也。所謂自天攸縱,將聖多能,德配乾坤,身揭日月。故能立天下之大本,成天下之大經,美政教,移風俗,君君臣臣,父父子子,人到於今受其賜。不其猗歟!於戲!楚王莫封,魯公不用,俾夫大聖,才列陪臣,棲
 遲旅人,固可知矣。年祀浸遠,光靈益彰,雖代有褒稱,而未為崇峻,不副於實,人其謂何?



 朕以薄德,祗膺寶命,思闡文明,廣被華夏。時則異於今古,情每重於師資。既行其教,合旌厥德。爰申盛禮,載表徽猷。夫子既稱先聖,可追謚為文宣王。宜令三公持節冊命,應緣冊及祭,所司速擇日,並撰儀注進。其文宣陵並舊宅立廟,量加人灑掃,用展誠敬。其後嗣可封文宣公。至如辨方正位,著自禮經,茍非得所,何以示則?昔緣周公南面,夫子西坐,今
 位既有殊,坐豈如舊,宜補其墜典,永作成式。自今已後,兩京國子監,夫子皆南面而坐,十哲等東西列侍。天下諸州亦準此。



 且門人三千,見稱十哲,包夫眾美,實越等夷。暢玄聖之風規,發人倫之耳目,並宜褒贈,以寵賢明。顏子淵既云亞聖,須優其秩,可贈兗公。閔子騫可贈費侯,冉伯牛可贈鄆侯,冉仲弓可贈薛侯,冉子有可贈徐侯,仲子路可贈衛侯,宰子我可贈齊侯,端木子貢可贈黎侯,言子游可贈吳侯,卜子夏可贈魏侯。又夫子格言,
 參也稱魯,雖居七十之數,不載四科之目。頃雖異於十哲,終或殊於等倫,允稽先旨,俾循舊位。庶乎禮得其序,人焉式瞻,宗洙泗之丕烈,重膠庠之雅範。



 又贈曾參、顓孫師等六十七人皆為伯。於是正宣父坐於南面,內出王者袞冕之服以衣之。遣尚書左丞相裴耀卿就國子廟冊贈文宣王。冊畢,所司奠祭,亦如釋奠之儀,公卿已下預觀禮。又遣太子少保崔琳就東都廟以行冊禮,自是始用宮懸之樂。春秋二仲上丁,令三公攝行事。



 天寶
 元年,明經、進士習《爾雅》。九載七月,國子監置廣文館,知進士業,博士、助教各一人,秩同太學博士。十二載七月,詔天下舉人不得充鄉貢,皆補學生。四門俊士停。



 寶應二年六月,敕令州縣每歲察秀才孝廉,取鄉閭有孝悌廉恥之行薦焉。委有司以禮待之,試其所通之學,《五經》之內,精通一經,兼能對策,達於理體者,並量行業授官。其明經、進士並停。國子學道舉,亦宜準此。因楊綰之請也。詔下朝臣集議,中書舍人賈至議,請依綰奏。有司奏曰:「竊
 以今年舉人等,或舊業既成,理難速改,或遠州所送,身已在途,事須收獎。其今秋舉人中有情願舊業舉試者,亦聽明年已後,一依新敕。」後綰議竟不行。自至德後,兵革未息,國學生不能廩食,生徒盡散,堂墉頹壞,常借兵健居止。至永泰二年正月,國子祭酒蕭昕上言:「崇儒尚學,以正風教,乃王化之本也。」其月二十九日,敕曰:



 理道同歸,師氏為上,化人成俗,必務於學。俊造之士,皆從此途,國之貴游,罔不受業。修文行忠信之教,崇祗庸孝友
 之德,盡其師道,乃謂成人。兼復揚於王廷,考以政事,征之以禮,任之以官。置於周行,莫匪邦彥,樂得賢也,其在茲乎!



 朕志求理體,尤重儒術,先王大教,敢不底行。頃以戎狄多難,急於經略,太學空設,諸生蓋寡。弦誦之地,寂寥無聲,函丈之間,殆將不掃。上庠及此,甚用憫焉。今宇縣攸寧,文武兼備,方投戈而講藝,俾釋菜而行禮。四科咸進,六藝復興,神人以和,風化浸美。日用此道,將無間然。



 其諸道節度、觀察、都防禦使等,朕之腹心,久鎮方面。
 眷其子弟,各奉義方,修德立身,事資括羽。恐干戈之後,學校尚微,僻居遠方,無所諮稟。山東寡學,質疑必就於馬融;關西盛名,尊儒乃稱於楊震。負經來學,當集京師。並宰相、朝官及神策六軍軍將子弟欲習業者,自今已後,並令補國子生。欲其業重籝金,器成琢玉,日新厥德,代不乏賢。其中身雖有官,欲附學讀書者,亦聽。其學官,委中書、門下即簡擇行業堪為師範者充。學生員數多少,所習經業,考試等第,並所供糧料,及學館破壞,要量
 事修理,各委本司作條件聞奏。務須詳悉,稱朕意焉。



 及二月朔上丁釋奠,蕭昕又奏:諸宰相元載、杜鴻漸、李抱玉及常參官、六軍軍將就國子學聽講論,賜錢五百貫。令京兆尹黎幹造食。集諸儒、道、僧,質問竟日。此禮久廢,一朝能舉。八月,國子學成祠堂、論堂、六館及官吏所居宇,用錢四萬貫,拆曲江亭子瓦木助之。四日,釋奠,宰相、常參官、軍將盡會於講堂,京兆府置食,講論。軍容使魚朝恩說《易》,又於論堂畫《周易》鏡圖。自至德二載收
 兩京,唯元正含元殿受朝賀,設宮懸之樂,雖郊廟大祭,只有登歌樂,亦無文、武二舞。其時軍容使魚朝恩知監事,廟庭乃具宮懸之樂於講堂前,又有教坊樂府雜會,竟日而罷。二十五日,詔曰:「古者設官分土,所以崇德報功。總內署之綱,事密於清禁;弘上庠之教,德潤於鴻業。賦開千乘,禮序九賓。必資兼濟之能,用協至公之選。開府儀同三司、兼右監門衛大將軍、仍知觀軍容宣慰處置使、知內侍省事、內飛龍閑廄使,內弓箭庫使、知神策
 軍兵馬使、上柱國、馮翊郡開國公魚朝恩,溫良恭儉,寬柔簡廉,長才博達,敏識高妙。學究儒玄之秘,謀窮遁甲之精。百行資身,一心奉上。自王室多故,雲雷經始,五原之北,以先啟行;三河之表,爰整其旅。成師必勝,每合於韜鈐;料敵無遺,可徵於蓍蔡。關洛既定,幽燕復開,海外有截,厥功惟茂。歷事三聖,始終竭力。頃東都扈蹕,釋位勤王,時當綴旒,節見披棘,下江助我,甲令先書,社稷之衛,邦家是賴。及邊陲罷警,戎務解嚴,方獎勵於《易》象。才
 兼文武,所謂勛賢,亦既任能,斯焉命賞,宜膺朝典,式副公議。可行內侍監,判國子監事,充鴻臚禮賓等使,封鄭國公,食邑三千戶。」二十四日,於國子監上。詔宰相及中書門下官、諸司常參官、六軍軍將送上。京兆府造食,內教坊音樂、竿木渾脫,羅列於論堂前。朝恩辭以中官不合知南衙曹務,宰相、僕射、大夫皆勸之,朝恩固辭,乃奏之。宰相引就食。奏樂,中使送酒及茶果,賜充宴樂,竟日而罷。元載奏狀。又使中使宣敕云:「朝恩既辭不止,但任
 知學生糧料。」是日,宰相軍將已下子弟三百餘人,皆衣紫衣,充學生房,設食於廊下。貸錢一萬貫,五分收錢,以供監官學生之費。俄又請青苗地頭取百文資課以供費同。舊例,兩京國子監生二千餘人,弘文館、崇文館、崇玄館學生,皆廩飼之。十五載,上都失守,此事廢絕。乾元元年,以兵革未息,又詔罷州縣學生,以俟豐歲。



 則天垂拱四年四月,雍州永安人唐同泰偽造瑞石於洛水,獻之。其文曰:「聖母臨人,永昌帝業。」於是號其石為「
 寶圖」,賜百官宴樂,賜物有差。授同泰為游擊將軍。其年五月下制,欲親拜洛受「寶圖。」先有事於南郊,告謝昊天上帝。令諸州都督、刺史並諸親,並以拜洛前十日集神都。於是則天加尊號為聖母神皇。大赦天下。改「寶圖」為「天授聖圖」,洛水為永昌。封其神為顯聖侯,加特進,禁漁釣,祭享齊於四瀆。所出處號曰聖圖泉,於泉側置永昌縣。又以嵩山與洛水接近,因改嵩山為神岳,授太師、使持節、神岳大都督、天中王,禁斷芻牧。其天中王及顯聖侯,並為
 置廟。又先於汜水得瑞石,因改汜水縣為廣武縣。至其年十二月,則天親拜洛受圖,為壇於洛水之北,中橋之左。皇太子皆從。內外文武在僚、蠻夷酋長,各依方位而立。珍禽奇獸,並列於壇前。文物鹵簿,自有唐已來,未有如此之盛者也。禮畢,即日還宮。神都父老勒碑於拜洛壇前,號曰:「天授聖圖之表。」開元五年,左補闕盧履冰上言曰:「則天皇后拜洛受圖壇及碑文,雲垂拱四年唐同泰得石,文云『聖母臨人,永昌帝業』之所建。因改元為永
 昌,仍置永昌縣。縣既尋廢,同泰亦已貶官,唯碑壇獨立。準天樞、頌臺之例,不可更留。」始令所司毀之,其顯聖侯廟亦尋毀拆。



 開元二十九年正月己丑,詔兩京及諸州各置玄元皇帝廟一所,並置崇玄學。其生徒令習《道德經》及《莊子》、《列子》、《文子》等,每年準明經例舉送。至閏四月,玄宗夢京師城南山趾有天尊之像,求得之於盩厔樓觀之側。至天寶元年正月癸丑,陳王府參軍田同秀稱於京永昌街空中見玄元皇帝,以「天下太平,聖壽無疆」之
 言傳於玄宗,仍云桃林縣故關令尹喜宅傍有靈寶符。發使求之,十七日,獻於含元殿。於是置玄元廟於太寧坊,東都於積善坊舊邸。二月丁亥,御含元殿,加尊號為開元天寶聖文神武皇帝。辛卯,親祔玄元廟。丙申,詔:《古今人表》,玄元皇帝升入上聖。莊子號南華真人,文子號通玄真人,列子號沖虛真人,庚桑子號洞虛真人。改《莊子》為《南華真經》,《文子》為《通玄真經》,《列子》為《沖虛真經》,《庚桑子》為《洞虛真經》。亳州真源縣先天太后及玄元
 廟各置令一人。兩京崇玄學各置博士、助教,又置學生一百員。桃林縣改為靈寶縣。田同秀與五品官。四月,詔崇文習《道德經》。七月,隴西李氏敦煌、姑臧、絳郡、武陽四房隸於宗正寺。九月,兩京玄元廟改為太上玄元廟,天下準此。十月,改新豐驪山為會昌山,仍於秦坑儒之所立祠宇。新作長生殿改為集靈臺。



 二年正月丙辰,加玄元皇帝尊號「大聖祖」三字,崇玄學改為崇玄館,博士為學士,助教為直學士,更置大學士員。三月壬子,親謁玄
 元宮,聖祖母益壽氏號先天太后,仍於譙郡置廟。尊皋繇為德明皇帝,涼武昭王為興聖皇帝。西京玄元廟為太清宮,東京為太微宮,天下諸州為紫極宮。九月,譙郡紫極宮宜準西京為太清宮,先天太皇及太后廟亦並改為宮。三載三月,兩京及天下諸郡於開元觀、開元寺,以金銅鑄玄元等身天尊及佛各一軀。七載二月,於大同殿修功德處,玉芝兩莖生於柱礎上。五月,玄宗御興慶殿,授冊尊號曰開元天寶聖文神武應道皇帝。十二
 月,以玄元皇帝見於朝元閣,改為降聖閣。改會昌縣為昭應縣,改會昌山為昭應山。封昭應山神為玄德公,立祠宇。



 初,太清宮成,命工人於太白山採白石,為玄元聖容,又採白石為玄宗聖容,侍立於玄元之右。皆依王者袞冕之服,繒彩珠玉為之。又於像設東刻白石為李林甫、陳希烈之形。及林甫犯事,又刻石為楊國忠之形,而瘞林甫之石。及希烈、國忠貶,盡毀瘞之。



 八載六月,玉芝產於大同殿。先是,太白山人李渾稱於金星洞仙人見,
 語老人云,有玉版石記符「聖上長生久視。」令御史中丞王鉷入山洞,求而得之。閏六月四日,玄宗朝太清宮,加聖祖玄元皇帝尊號曰聖祖大道玄元皇帝,高祖、太宗、高宗、中宗、睿宗尊號並加「大聖」字,皇后並加「順聖」字。五日,玄宗御含元殿,加尊號曰開元天寶聖文神武應道皇帝。大赦。自今已後,每至禘祫,並於太清宮聖祖前設位序昭穆。太白山封神應公,金星洞改嘉祥洞,所管華陽縣改為真符縣。兩京及十道一大郡,置真符玉芝觀。九
 載十月,先是,御史大夫王鉷奏稱太白山人王玄翼見玄元皇帝於寶山洞中。乃遣王鉷、張均、王倕、韋濟、王翼、王岳靈於洞中得玉石函《上清護國經》、寶券、紀籙等,獻之。



 十一月,制:「承前宗廟,皆稱告享。自今已後,每親告獻太清、太微宮,改為朝獻,有司行事為薦獻。親告享宗廟改為朝享,有司行事為薦享。親巡陵改為朝陵,有司行事為拜陵。應諸事告宗廟者,並改為表。其郊天、后土及享祠祝文云『敢昭告』者,並改為『敢昭薦』。」十載正月,有事於
 南郊,於壇所大赦。制:「自今已後,攝祭南郊,薦獻太清宮,薦享太廟,其太尉行事前一日,於致齋所具羽儀鹵簿,公服引入,親授祝版,乃赴清齋所。」



 汾陰后土之祀,自漢武帝後廢而不行。玄宗開元十年,將自東都北巡,幸太原,便還京,乃下制曰:「王者承事天地以為主,郊享泰尊以通神。蓋燔柴泰壇,定天位也;瘞埋泰折,就陰位也。將以昭報靈祇,克崇嚴配。爰逮秦、漢,稽諸祀典,立甘泉於雍畤,定後土於汾陰,遺廟嶷然,靈光可燭。朕觀風唐、晉,望秩
 山川,肅恭明神,因致禋敬,將欲為人求福,以輔升平。今此神符,應於嘉德。行幸至汾陰,宜以來年二月十六日祠后土,所司準式。」



 先是,脽上有後土祠,嘗為婦人塑像,則天時移河西梁山神塑像,就祠中配焉。至是,有司送梁山神像於祠外之別室,內出錦繡衣服,以上後土之神,乃更加裝飾焉。又於祠堂院外設壇,如皇地祇之制。及所司起作,獲寶鼎三枚以獻,十一年二月,上親祠於壇上,亦如方丘儀。禮畢,詔改汾陰為寶鼎。亞獻邠王守
 禮、終獻寧王憲已,頒賜各有差。二十年,車駕又從東都幸太原,還京。中書令蕭嵩上言:「去十一年親祠后土,為祈穀,自是神明昭格,累年豐登。有祈必報,禮之大者。且漢武親祠脽上,前後數四,伏請準舊祀后土,行賽之禮。」上從之。其年十一月至寶鼎,又親祠以申賽謝。禮畢,大赦。仍令所司刊石祠所,上自為其文。



 開元二十四年七月乙巳,初置壽星壇,祭老人星及角、亢等七宿。天寶三年,有術士蘇嘉慶上言:「請於京東朝日壇東,置九宮
 貴神壇,其壇三成,成三尺,四階。其上依位置九壇,壇尺五寸,東南曰招搖,正東曰軒轅,東北曰太陰,正南曰天一,中央曰天符,正北曰太一,西南曰攝提,正西曰咸池,西北曰青龍。五為中,戴九履一,左三右七,二四為上,六八為下,符於遁甲。四孟月祭,尊為九宮貴神,禮次昊天上帝,而在太清宮太廟上。用牲牢、璧幣,類於天地神祇。」玄宗親祀之。如有司行事,即宰相為之。肅宗乾元三年正月,又親祀之。初,九宮神位,四時改位,呼為飛位。乾元
 之後,不易位。



 大和二年八月,監察御史舒元輿奏:「七月十八日,祀九宮貴神,臣次合監祭,職當檢察禮物。伏見祝版九片,臣伏讀既竟,竊見陛下親署御名及稱臣於九宮之神。臣伏以天子之尊,除祭天地、宗廟之外,無合稱臣者。王者父天母地,兄日姊月,此以九宮為目,是宜分方而守其位。臣又觀其名號,乃太一、天一、招搖、軒轅、咸池、青龍、太陰、天符、攝提也。此九神,於天地猶子男也,於日月猶侯伯也。陛下尊為天子,豈可反臣於天之子男
 耶?臣竊以為過。縱陰陽者流言其合祀,則陛下當合稱皇帝遣某官致祭於九宮之神,不宜稱臣與名。臣實愚瞽,不知其可。伏緣行事在明日雞初鳴時,成命已行,臣不敢滯。伏乞聖慈異日降明詔禮官詳議,冀嘉萬乘之尊,無所虧降,悠久誤典,因此可正。」詔都省議,皆如元輿之議。乃降為中祠,祝版稱皇帝,不署。



 會昌元年十二月,中書門下奏:「準天寶三載十月六日敕,『九宮貴神,實司水旱,功佐上帝,德庇下人。冀喜穀歲登,災害不作。每至
 四時初節,令中書門下往攝祭』者。準禮,九宮次昊天上帝,壇在太清宮、太廟上,用牲牢、璧幣,類於天地。天寶三載十二月,玄宗親祠。乾元二年正月,肅宗親祀。伏自累年已來,水旱愆候,恐是有司禱請,誠敬稍虧。今屬孟春,合修祀典,望至明年正月祭日,差宰臣一人禱請。向後四時祭,並請差僕射、少師、少保、尚書、太常卿等官,所冀稍重其事,以申嚴敬。臣等十一月二十五日已於延英面奏,伏奉聖旨令檢儀注進來者。今欲祭時,伏望令有
 司崇飾舊壇,務於嚴潔。」敕旨依奏。



 二年正月四日,太常禮院奏:「準監察御史關牒:『今月十三日,祀九宮貴神,已敕宰相崔珙攝太尉行事,合受誓誡,及有司徒、司空否?』伏以前件祭本稱大祠,準大和三年七月二十四日敕,降為中祠。昨據敕文,只稱崇飾舊壇,務於嚴潔,不令別進儀注,更有改移。伏恐不合卻用大祠禮料,伏候裁旨。」中書門下奏曰:



 臣準天寶三載十月六日敕,「九宮貴神,實司水旱。」臣等伏睹,既經兩朝親祠,必是祈請有征,況
 自大和已來,水旱愆候,陛下常憂稼穡,每念烝黎。臣等合副聖心,以修墜典。伏見大和三年禮官狀云:「縱司水旱兵荒,品秩不過列宿。今者五星悉是從祀,日月猶在中祀。」竊詳其意,以星辰不合比於天官。曾不知統而言之,則為天地,在於辰象,自有尊卑。謹按後魏王鈞《志》:「北辰第二星,盛而常明者乃為元星露寢,天帝常居,始由道奧而為變通之跡。又天皇大帝,其精曜魄寶,蓋萬神之秘圖,河海之命紀皆稟焉。」據茲說即昊天上帝也。天
 一掌八氣、九精之政令,以佐天極。徵明而有常,則陰陽序,大運興。太一掌十有六神之法度,以輔人極。徵明面得中,,則神人和而王道升平。又北斗有權、衡二星,天一、太一參居其間,所以財成天地,輔相神道也。若一概以列宿論之,實為淺近。按《漢書》曰:「天神貴者太一,佐曰五帝。」古者天子以春秋祭太一,列於祀典,其來久矣。今五帝猶為大祀,則太一無宜降祀,稍重其祀,固為得所。劉向有言曰:「祖宗所立神祇舊典,誠未易動。」又曰:「古今異
 制,經無明文,至尊至重,難以疑說正也。」其意不欲非祖宗舊典。以劉向之博通,尚難於改作,況臣等學不究於天人,識尤懵於祀典,欲為參酌,恐未得中。伏望更令太常卿與學官同詳定,庶獲明據。



 從之。



 檢校左僕射太常卿王起、廣文博士盧就等獻議曰:



 伏以九宮貴神,位列星座;往因致福,詔立祠壇。降至尊以稱臣,就東郊以親拜。在祀典雖雲過禮,庇群生豈患無文,思福黔黎,特申嚴奉,誠聖人屈已以安天下之心也。厥後祝史不明,精
 誠亦怠,禮官建議,降處中祠。今聖德憂勤,期臻壽域,兵荒水旱,寤寐軫懷,爰命臺臣,緝興墜典。



 伏惟九宮所稱之神,即太一、攝提、軒轅、招搖、天符、青龍、咸池、太陰、天一者也。謹按《黃帝九宮經》及蕭吉《五行大義》:「一宮,其神太一,其星天蓬,其卦坎,其行水,其方白。二宮,其神攝提,其星天芮,其卦坤,其行土,其方黑。三宮,其神軒轅,其星天沖,其卦震,其行木,其方碧。四宮,其神招搖,其星天輔,其卦巽,其行木,其方綠。五宮,其神天符,其星天禽,其卦離,
 其行土,其方黃。六宮,其神青龍,其星天心,其卦乾,其行金,其方白。七宮,其神咸池,其星天柱,其卦兌,其行金,其方赤。八宮,其神太陰,其星天任,其卦艮,其行土,其方白。九宮,其神天一,其星天英,其卦離,其行火,其方紫。」觀其統八卦,運五行,土飛於中,數轉於極,雖敬事迎厘,不聞經見,而範圍亭育,有助昌時,以此兩朝親祀而臻百祥也。然以萬物之精,上為列星,星之運行,必系於物。貴而居者,則必統八氣,總萬神,斡權化於混茫,賦品匯於陰
 騭,與天地日月,誠相參也。豈得醫賴於敷祐,而屈降於等夷?



 又據太尉攝祀九宮貴神舊儀:前七日,受誓誡於尚書省,散齋四日,致齋三日。牲用犢。祝版御署,稱嗣天子臣。圭幣樂成。比類中祠,則無等級。今據《江都集禮》及《開元禮》:蠟祭之日,大明、夜明二座及朝日、夕月,皇帝致祝,皆率稱臣。若以為非泰壇配祀之時,得主日報天之義。卑緣厭屈,尊用德伸,不以著在中祠,取類常祀。此則中祠用大祠之義也。又據太社、太稷,開元之制,列在中
 祠。天寶三載二月十四日敕,改為大祠,自後因循,復用前禮。長慶三年正月,禮官獻議,始準前敕,稱為大祠。唯御署祝文,稱天子謹遣某官昭告。文義以為殖物粒人,則宜增秩,致祝稱禱,有異方丘,不以伸為大祠,遂屈尊稱。此又大祠用中祠之禮也。參之日月既如彼,考之社稷又如此,所謂功鉅者因之以殊禮,位稱者不敢易其文,是前聖後儒陟降之明徵也。今九宮貴神,既司水旱,降福禳災,人將賴之,追舉舊章,誠為得禮。然以立祠
 非古,宅位有方,分職既異其司存,致祝必參乎等列。求之折中,宜有變通,稍重之儀,有以為比。伏請自今已後,卻用大祠之禮,誓官備物,無有降差。唯御署祝文,以社稷為本,伏緣已稱臣於天帝,無二尊故也。



 敕旨依之,付所司。



 天寶十載四月二十九日,移黃帝壇於子城內坤地,將親祠祭,壇成而止。



 玄宗先天二年,封華嶽神為金天王。開元十三年,封泰山神為天齊王。天寶五載,封中嶽神為中天王,南嶽神為司天王,北嶽神為安天王。六
 載,河瀆封靈源公,濟瀆封清源公,江瀆封廣源公,淮瀆封長源公。十載正月,四海並封為王。遣國子祭酒嗣吳王祗祭東嶽天齊王,太子家令嗣魯王宇祭南嶽司天王,秘書監崔秀祭中嶽中天王,國子祭酒班景倩祭西岳金天王,宗正少卿李成裕祭北岳安天王;衛尉少卿李浣祭江瀆廣源公,京兆少尹章恆祭河瀆靈源公,太子左諭德柳偡祭淮瀆長源公,河南少尹豆盧回祭濟瀆清源公;太子率更令嗣道王煉祭沂山東安公,吳郡太
 守趙居貞祭會稽山永興公,大理少卿李稹祭吳嶽山成德公,潁王府長史甘守默祭霍山應聖公,範陽司馬畢炕祭醫無閭山廣寧公;太子中允李隨祭東海廣德王,義王府長史張九章祭南海廣利王,太子中允柳奕祭西海廣潤王,太子洗馬李齊榮祭北海廣澤王。取三月十七日一時禮冊。



 玄宗御極多年,尚長生輕舉之術。於大同殿立真仙之像,每中夜夙興,焚香頂禮。天下名山,令道士、中官合煉醮祭,相繼於路,投龍奠玉,造精舍,
 採藥餌,真訣仙蹤,滋於歲月。



 肅宗至德二載春,在鳳翔,改汧陽郡吳山為西嶽,增秩以祈靈助。及上元二年,聖躬不康,術士請改吳山為華山,華山為泰山,華州為泰州,華陽縣為太陰縣。寶應元年,復舊。



 則天長安三年,令天下諸州宜教人武藝,每年準明經進士例申奏。開元十九年,於兩京置太公尚父廟一所,以漢留侯張良配饗。天寶六載,詔諸州武舉人上省,先謁太公廟,拜將帥亦告太公廟。至肅宗上元元年閏四月,又尊為武成王,選
 歷代良將為十哲。



 高宗顯慶元年三月辛巳,皇后武氏有事於先蠶。玄宗先天二年三月辛卯,皇后王氏祀先蠶。肅宗乾元二年三月己巳,皇后張氏祠先蠶於苑內,內外命婦同採焉。



 舊儀,大祭祀,宮懸、軒縣奏於庭,登歌於堂上。自至德二載克復兩京後,樂工不備,時又艱食,諸壇廟祭享,空有登歌,無壇下、庭中樂及三舞。舊儀,凡祭享,有司行事,則太尉奠瓚幣,司徒拜俎,司空掃除,太尉初獻,太常卿亞獻,光祿卿終獻。自上元後,南郊、九宮
 神壇、太廟,備此五官,餘即太常卿攝司空,光祿卿攝司徒,貴省於事。舊儀,有協律郎立於阼階上,麾竿以節樂,今無協律之位。舊儀,光祿欲為祭饌,將陽燧望日取火,謂之明火。太牢皆棧飼於廩犧署,以至充腯。臨祭視其充瘦,謂之省牲,肅宗上元二年九月,改元為元年,詔:「圓丘方澤,依恆存一太牢。皇廟諸祠,臨時獻熟。」今昊天上帝、太廟,一牢,羊豕各三,餘祭盡隨事辦供以備禮。明火、棧飼之禮,亦不暇矣。



\end{pinyinscope}