\article{卷一百一十七}

\begin{pinyinscope}

 ○苗晉卿裴冕裴遵慶子向向子寅寅子樞



 苗晉卿,上黨壺關人。世以儒素稱。祖夔,高道不仕,追贈禮部尚書。父殆庶,官至絳州龍門縣丞,早卒,以晉卿贈太子少保。



 晉卿幼好學,善屬文,進士擢第。初授懷州修
 武縣尉,歷奉先縣尉,坐累貶徐州司戶參軍。秩滿隨調,判入高等,授萬年縣尉。遷侍御史,歷度支、兵、吏部三員外郎。開元二十三年,遷吏部郎中。二十四年,與吏部郎中孫逖並拜中書舍人。二十七年,以本官權知吏部選事。晉卿性謙柔,選人有訴訟索好官者,雖至數千言,或聲色甚厲者,晉卿必含容之,略無慍色。二十九年,拜吏部侍郎。前後典選五年,政既寬弛,胥吏多因緣為奸,賄賂大行。時天下承平,每年赴選常萬餘人。李林甫為尚
 書,專任廟堂,銓事唯委晉卿及同列侍郎宋遙主之。選人既多,每年兼命他官有識者同考定書判,務求其實。天寶二年春,御史中丞張倚男奭參選,晉卿與遙以倚初承恩,欲悅附之,考選人判等凡六十四人,分甲乙丙科,奭在其首。眾知奭不讀書,論議紛然。有蘇孝慍者,嘗為範陽薊令,事安祿山,具其事告之。祿山恩寵特異,謁見不常,因而奏之。玄宗大集登科人,御花萼樓親試,登第者十無一二;而奭手持試紙,竟日不下一字,時謂之「
 曳白」。上怒,晉卿貶為安康郡太守,遙為武當郡太守,張倚為淮陽太守。敕曰:「門庭之間,不能訓子;選調之際,仍以托人。」時士子皆以為戲笑。



 天寶三載閏二月,轉魏郡太守,充河北採訪處置使,居職三年,政化洽聞。會入計,因上表請歸鄉里。既至壺關,望縣門而步。小吏進曰:「太守位高德重,不宜自輕。」晉卿曰:「《禮》:『下公門,式路馬。』況父母之邦,所宜尊敬。汝何言哉!」大會鄉黨,歡飲累日而去。又俸錢三萬為鄉學本,以教授子弟。尋改
 河東太守、河東採訪使,入為尚書、東京留守,徵為憲部尚書。屬祿山叛逆,楊國忠以晉卿有時望,將抑之,乃奏云:「宜以大臣鎮東道。」遂出為陜州刺史、陜虢兩州防禦使。及入對,固辭老病,由是忤旨,改憲部尚書致仕。及朝廷失守,衣冠流離道路,多為逆黨所脅,自陳希烈、張均已下數十人盡赴洛陽,晉卿潛遁山谷,南投金州。會肅宗至鳳翔,手詔追晉卿赴行在,即日拜為左相,軍國大務悉以咨之。既收兩京,以功封韓國公,食實封五百戶,改為侍
 中。後以賊寇漸除,屢乞骸骨,優詔許之,罷知政事,為太子太傅。明年,帝思舊臣,復拜為侍中。



 晉卿寬厚廉謹,為政舉大綱,不問小過,所到有惠化。魏人思之,為立碑頌德。及秉鈞衡,小心畏慎,未嘗忤人意。性聰敏,達練事體,百司文簿,經目必曉,而修身守位,以智自全,議者比漢之胡廣。



 玄宗崩,肅宗詔晉卿攝塚宰。上表固辭曰:「臣聞古者殷高宗在諒闇之中,百官聽於塚宰,更無事跡,但存文字。且一時之事,禮不相沿。今殘寇猶虞,日殷萬務,
 皆緣兵馬屯守討襲,善算良謀,立勝擒敵。陛下若行古之道,居喪不言,蒼生何依,百事皆廢。伏讀國家起居注,亦於禮部檢見舊敕,恭惟太宗、高宗、、大行皇帝在位之日,皆有國哀,視事不輟,以為君臨天下,難徇常情。今遺詔有處分,皇帝宜三日而聽政。陛下遵太宗故事,則無塚宰;遵大行皇帝遺詔,便合聽朝。萬姓顒顒,不勝大願。伏惟陛下知理國之重,順人心之切,以義斷恩,從宜無改。今朝臣一命已上,皆言臣心昏貌朽,加以疾病,事有急
 速,斷在須臾,凡聖不同,豈合受詔。陛下發哀已五日矣,願準遺詔聽政,則四夷萬國,無任悲幸。」肅宗時疾彌留,覽表殞絕,乃許。



 數日,肅宗晏駕,代宗踐祚,又詔晉卿攝塚宰。晉卿上表懇辭曰:「臣以昔者天子居喪之時,百官聽於塚宰者,蓋君幼小,御極事殷,情理當然。沿革不一,今古異同,而周武、漢文,合於通變,垂範作則,可舉而行。又士或墨縗,時遇金革,豈非銜恤,謂義在斷恩。且百善之至,無加於孝也,其有容瘁心絕,指景悼生,此匹夫
 守節之常情,殊王者嗣續之大計。昨二十日,陛下於大行皇帝柩前即位,是承先帝遺顧之言,亦前代不易之典。則知所略不為害,所存是適權,防威滅端,所利者大。陛下因心純至,天地明察。伏以報劬勞之恩,申罔極之思,終身之痛,豈計朝夕!但以一日之內,萬務在中,須達宸聰,始成國政。百僚萬姓及僧道耆壽等,相顧聚言,以臣老且無能,愚豈測聖,況久無居攝,臣不敢奉詔。特乞陛下遵遺命,三日而政。臣博聽眾情,不勝懇願,伏望
 割痛抑哀,則天下悲幸。」上號泣從之。時晉卿年已衰暮,又患兩足,上特許肩輿至中書,入閣不趨,累日一視事。歷三朝,皆以謹密見稱。



 廣德初,吐蕃寇長安。晉卿時病臥於私第,蕃聞之,輿入逼脅,晉卿閉口不言,賊不敢害。及上自陜至,冊為太保,罷知政事,又詔以太保致仕。永泰元年四月薨。輟朝三日,令京兆少尹一員護喪事,緣葬諸物並官給,賻絹布五百段、米粟五百石。太常議謚曰「懿獻」。初,晉卿東都留守,引用大理評事元載為推官。
 至是載為中書侍郎、平章事,懷舊恩,諷有司改謚曰文貞。大歷七年,令配享肅宗廟庭。



 裴冕,河東人也,為河東冠族。天寶初,以門廕再遷渭南縣尉,以吏道聞。御史中丞王鉷充京畿採訪使,表為判官。遷監察御史,歷殿中侍御史。冕雖無學術,守職通明,果於臨事,鉷甚委之。及鉷得罪伏法,時宰臣李林甫方竊權柄,人咸懼之,鉷賓佐數百,不敢窺鉷門。冕獨收鉷尸,親自護喪,瘞於近郊,冕自是知名。河西節度使哥舒
 翰表為行軍司馬,累遷員外郎中。



 玄宗幸蜀,至益昌郡,遙詔太子充天下兵馬元帥,以冕為御史中丞兼左庶子,為之副。是時,冕為河西行軍司馬,授御史中丞,詔赴朝廷。遇太子於平涼,具陳事勢,勸之朔方,亟入靈武。冕與杜鴻漸、崔漪等勸進曰:「主上厭勤大位,南幸蜀川,宗社神器,須有所歸,天意人事,不可固違。若逡巡退讓,失億兆心,則大事去矣!臣等猶知之,況賢智乎!」太子曰:「南平寇逆,奉迎鑾輿,退居儲貳,侍膳左右,豈不樂哉!公等何言之過也?」冕與杜鴻漸又進曰:「殿下藉累聖之
 資,有天下之表。元貞萬國,二十餘年,殷憂啟聖,正在今日。所從殿下六軍將士,皆關輔百姓,日夜思歸。大軍一散,不可復集,不如因而撫之以從眾,臣等敢以死請。」凡勸進五上,乃依。肅宗即位,以定策功,遷中書侍郎、同中書門下平章事,倚以為政。



 冕性忠勤,悉心奉公,稍得人心。然不識大體,以聚人曰財,乃下令賣官鬻爵,度尼僧道士,以儲積為務。人不願者,科令就之,其價益賤,事轉為弊。肅宗移幸鳳翔,罷冕知政事,遷右僕射。兩京平,以
 功封冀國公,食實封五百戶。尋加御史大夫、成都尹,充劍南西川節度使。又入為右僕射。永泰元年,與裴遵慶等並集賢待制。代宗求舊,拜冕兼御史大夫,充護山陵使。冕以幸臣李輔國權盛,將附之,乃表輔國親暱術士中書舍人劉烜充山陵使判官。烜坐法,冕坐貶施州刺史。數月,移澧州刺史,復徵為左僕射。元載秉政。載為新平縣尉,王鉷闢在巡內,冕常引之,載頗德冕。會宰臣杜鴻漸卒,載遂舉冕代之。冕時已衰瘵,載以其順己,引為
 同列。受命之際,蹈舞絕倒,載趨而扶起,代為謝詞。冕兼掌兵權留守之任,俸錢每月二千餘貫。性本侈靡,好尚車服及營珍饌,名馬在櫪,直數百金者常十數。每會賓友,滋味品數,坐客有昧於名者。自創巾子,其狀新奇,市肆因而效之,呼為「僕射樣」。初代鴻漸,小吏以俸錢文簿白之,冕顧子弟,喜見於色,其嗜利若此。拜職未盈月,卒,大歷四年十二月也。上悼之,輟朝三日,贈太尉,賻制五百匹、粟五百石。



 裴遵慶,絳州聞喜人也。代襲冠冕,為河東著族。遵慶志氣深厚,機鑒敏達,自幼強學,博涉載籍,謹身晦跡,不乾當世之務。以門廕累授潞府司法參軍,時年已老,未為人所知。隨調吏部,授大理寺丞,剖斷刑獄,舉正綱條,理行始著。遷司門員外、吏部員外郎,專判南曹。天寶中,海內無事,九流輻輳會府,每歲吏部選人,動盈萬數。遵慶敏識強記,精核文簿,詳而不滯,時稱吏事第一,由是大知名。



 天寶末,楊國忠當國,出不附己者例為外官,遵慶
 亦出為郡守。肅宗即位,徵拜給事中、尚書右丞、吏部侍郎。恭儉克己,遲重謹密,頗有時望。上元中,蕭華輔政,素知遵慶,每奏見,累稱之,遷黃門侍郎、同中書門下平章事。廣德初,僕固懷恩阻兵汾上,指中官為詞,上以遵慶忠純,特遣往汾州宣慰懷恩。遵慶既見懷恩,具陳朝旨,懷恩引過聽命,將隨遵慶朝謁,為副將範志誠以邪說惑之,懷恩遂以懼死為詞。會蕃寇陷京師,乘輿幸陜,遵慶自汾州奔赴行在。及乘輿還京,以遵慶為太子少傅。
 永泰元年,與裴冕等並於集賢院待制,罷知政事。尋改吏部尚書、右僕射,復知選事。時選人天興縣尉陳琯於銓庭言詞不遜,凌突無禮,代宗詔付遵慶於省門鞭三十,貶為吉州員外司戶參軍。遵慶敦守儒行,老而彌謹。嘗為風狂族侄撾登聞鼓告以不順,上知遵度,不省,其見信如此。大歷十年十月薨於位,年九十餘。



 遵慶初登省郎,嘗著《王政記》,述今古禮體,識者覽之,知有公輔之量。



 子向,字傃仁,少以門廕歷官至太子司議郎。建中
 初,李紓為同州刺史,奏向為從事。硃泚反,李懷光又叛河中,使其將趙貴先築壘於同州,紓來奔奉天,向領州務。貴先因脅縣尉林寶役徒板築,不及期,將斬之,吏人百姓奔竄。向即詣貴先軍壘,以逆順之理責之,貴先感悟,遂來降,故同州不陷。向由是知名。累為京兆府戶曹,轉櫟陽、渭南縣令,奏課皆第一,朝廷亟聞其理行,擢為戶部員外郎。



 德宗季年,天下方鎮副人卒多自選於朝,防一日有變,遂就而授之節制。向已選為太原少尹,德宗
 召見喻旨,尋用為行軍司馬、兼御史中丞,改汾州刺史,轉鄭州。又復為太原少尹,兼河東節度副使。改晉州刺史,充本州防禦使,遷虢州刺史。入為京兆少尹,拜同州刺史,充本州防禦使。入為大理寺卿,出遷陜虢都防禦、觀察使。三歲,拜左散騎常侍,自常侍復為大理。



 向本以名相子,以學行自飭,謹守其門風。歷官仁智推愛,利及於人。至是,以年過致政,朝廷優異,乃以吏部尚書致仕於新昌里第。內外支屬百餘人,向所得俸祿,必同其費,
 及領外任,亦挈而隨之。有孤煢疾苦不能自恤,向尤周給,至今稱其孝睦焉。大和四年九月卒,年八十。贈太子少保。



 子寅,登進士第,累官至御史大夫卒。



 子樞,字紀聖,咸通十二年登進士第。宰相杜審權出鎮河中,闢為從事,得秘書省校書郎,再遷藍田尉。直弘文館。大學士王鐸深知之,鐸罷相失職,樞亦久之不調。從僖宗幸蜀,中丞李煥奏為殿中侍御史,遷起居郎。中和初,王鐸復見用,以舊恩徙為鄭滑掌書記、檢校司封郎中,賜金紫,入
 朝歷兵、吏二員外郎。龍紀初,擢拜給事中,改京兆尹。宰相孔緯尤深獎遇。大順中,緯以用兵無功貶官,樞坐累為右庶子,尋出為歙州刺史。乾寧初,入為右散騎常侍,從昭宗幸華州,為汴州宣諭使。



 初,樞自歙州罷郡歸朝,路經大梁,時硃全忠兵威已振,樞以兄事之,全忠由是重之。及樞傳詔,全忠皆稟朝旨,獻奉相繼,昭宗甚悅,遷兵部侍郎。時崔胤專政,亦倚全忠,二人因是相結,改樞吏部侍郎。未幾,換戶部侍郎、同平章事。其年冬,昭宗
 幸華州,崔胤貶官,樞亦為工部尚書。天子自岐下還宮,以樞檢校右僕射、同平章事,出為廣南節度使。制出,硃全忠保薦之,言樞有經世才,不可棄之嶺表,尋復拜門下侍郎,監修國史,累兼吏部尚書,判度支。崔胤誅,以全忠素厚,相位如故。從昭宗遷洛陽,駐蹕陜州,進右僕射、弘文館大學士、太清宮使,充諸道鹽鐵轉運使。



 哀帝初嗣位,柳璨用事,全忠嘗奏用牙將張廷範為太常卿,諸相議,樞曰:「廷範勛臣,幸有方鎮節鉞之命,何藉樂卿?恐
 非元帥梁王之旨。」乃持之不下。俄而全忠聞樞言,謂賓佐曰:「吾常以裴十四器識真純,不入浮薄之伍,觀此議論,本態露矣。」切齒含怒。柳璨聞全忠言,尋希旨罷樞相位,和陵祔享,拜尚書左僕射。五月,責授朝散大夫、登州刺史,尋再貶瀧州司戶。六月十一日,行及滑州,全忠遣人殺之於白馬驛,投尸於河,時年六十五。



 史臣曰:晉卿謹身蒞事,足為純臣,避寇全忠,固彰大節。然博達精審,豈不知寬猛之道哉!奉林甫之旨,順胥吏
 之意,悅附張倚,欺罔時君。生為重臣,諂林甫之勢也;歿改美謚,引元載之恩焉。或言晉卿不為巧宦者,誠不信也。冕力贊中興,名居大位,奉公抱義,可以致身;賣官度僧,是何為政?及其老也,貪冒尤深。遵慶學術貞明,為國忠所出;恭儉謹密,遇蕭華素知。位重行純,老而彌篤,彼二公固有慚德。向克荷堂構,不墜門風。樞因盜而振,盜憎而亡,宜哉!君子守道遠刑,蓋慮此也。



 贊曰:奧矣晉卿,貪哉裴冕。遵慶父子,及之者鮮。



\end{pinyinscope}