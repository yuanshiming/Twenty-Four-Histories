\article{卷一百四十八}

\begin{pinyinscope}

 ○尚可孤李觀戴休顏陽惠元李元諒韓游瑰賈隱林杜希全尉遲勝邢君牙楊朝晟張敬則



 尚可孤,東部鮮卑宇文之別種也,代居松、漠之間。天寶
 末歸國,隸範陽節度安祿山,後事史思明。上元中歸順,累授左、右威衛二大將軍同正,充神策大將,以前後功改試太常卿,仍賜實封一百五十戶。魚朝恩之統禁軍,愛其勇,甚委遇之,俾為養子,奏姓魚氏,名智德,以禁兵三千鎮於扶風縣,後移武功。可孤在扶風、武功凡十餘年,士伍整肅,軍邑安之。朝恩死,賜可孤姓李氏,名嘉勛。會李希烈反叛,建中四年七月,除兼御史中丞、荊襄應援淮西使,仍復本姓名尚可孤,以所統之眾赴山南,累
 有戰功。



 及涇原兵叛,詔徵可孤軍至藍田,賊眾方盛,遂營於七盤,修城柵而居之。賊將仇敬等來寇,可孤頻擊破之,因收藍田縣。興元元年三月,遷檢校工部尚書、兼御史大夫、神策京畿渭南商州節度使。四月,仇敬又來寇,可孤率兵急擊,擒仇敬斬之,遂進軍與副元帥李晟決策攻討。五月,晟率可孤及駱元光之軍收京城,可孤之師為先鋒。京師平,以功升檢校右僕射,封馮翊郡王,增邑通前八百戶,實封二百戶。



 可孤性謹願沉毅,既有
 勛勣,眾會之中,未嘗言功。賊平之後,營於白花亭,御眾公平,號令嚴整,時人稱焉。李晟甚親重之。及李懷光以河中叛,詔可孤帥師與諸軍進討,次於沙苑,遇疾,卒於軍。贈司徒,賻布帛米粟加等,喪葬所須,並令官給。



 李觀,洛陽人,其先自趙郡徙焉,秋官員外郎敬仁侄孫也。少習武藝,沉厚寡言,有將帥識度。乾元中,以策干朔方節度使郭子儀;子儀善之,令佐坊州刺史吳伷,充防遏使。尋以憂免,居盩厔別業。廣德初,吐蕃入寇,鑾駕之
 陜,覲於盩厔,率鄉里子弟千餘人守黑水之西,戎人不敢近。會嶺南節度楊慎微將之鎮,以觀權謀,奏充偏將,俾總軍政。及徐浩、李勉繼領廣州,尤加信任,麾下兵甲悉委之。平馮崇道、硃泚時有功,累遷大將。李勉移鎮滑州,累奏授試殿中監,加開府儀同三司。追赴闕,授右龍武將軍。



 建中末,涇師叛,觀時上直,領衛兵千餘人扈從奉天。詔都巡警訓練諸軍戍卒,三數日間,加召二千餘眾,列之通衢,整肅鼙鼓,城內因之增氣。德宗倚賴之,
 賜封二百戶;二子宏、寓,授八品京官。及駕出奉天,與令狐建、李升、韋清等咸執羈靮,周旋艱險,皆著功勞。駕還京師,詔總後軍禁衛。



 興元元年閏十月,拜四鎮北庭行軍涇原節度使、檢校兵部尚書。在鎮四年,雖無拓境之績,勵卒儲糧,訓整寧輯。及平涼之師會,渾瑊既無戎備,觀伺知狡謀,潛擇精兵五千要伏險道。及瑊遁歸,賴觀游軍及李元諒之師表裏以免。帝優賞,賜賚甚厚,特詔褒美。其年,朝京師,除少府監、檢校工部尚書,以疾終。貞
 元四年,贈太子少傅。



 戴休顏,夏州人。在軍伍以膽略稱。大歷中,為郭子儀部將,以戰功累遷至鹽州刺史。奉天之難,倍道以所部蕃漢三千人號泣赴難;德宗嘉之,賜實封二百戶。與渾瑊、杜希全、韓游瑰等捍禦有功。車駕再幸梁、洋,留守奉天。及李懷光叛據咸陽,使誘休顏;休顏集三軍斬其使,嬰城自守。懷光大駭,遂自涇陽夜遁。其月,拜檢校工部尚書、奉天行營節度使。李晟收京師,乃與渾瑊破泚偏師,
 斬首三千級,休顏追賊至中渭橋。李晟既清宮闕,休顏與瑊等率兵赴岐陽邀擊泚餘眾。及策勛,加檢校右僕射,封至六百戶。七月,扈駕至京,特賜女樂、甲第以褒功伐,尋拜左龍武將軍。貞元元年卒,年五十九,廢朝一日,贈賻有差。



 陽惠元,平州人。以材力從軍,隸平盧節度劉正臣。後與田神功、李忠臣等相繼泛海至青、齊間,忠勇多權略,稱為名將。又以兵隸神策,充神策京西兵馬使,鎮奉天。



 初,
 大歷中,兩河平定,事多姑息。李正己有淄、青、齊、海、登、萊、沂、密、德、棣、曹、濮、徐、兗、鄆十五州之地,養兵十萬;李寶臣有恆、易、深、趙、滄、冀、定七州之地,有兵五萬;田承嗣有魏、博、相、衛、洺、貝、澶七州之地,有兵五萬;梁崇義有襄、鄧、均、房、復、郢六州之地,其眾二萬。皆始因叛亂得侯,各擅土宇,雖泛稟朝旨,而威刑爵賞,生殺自專,盤根結固,相為表裏。朝廷常示大信,不為拘限,緩之則嫌釁自作,急之則合謀。或聞詔旨將增一城,浚一池,必皆怨怒有辭,則為
 之罷役;而自於境內治兵繕壘以自固。凡歷三朝,殆二十年,國家不敢興拳石撮土之役。



 代宗性寬柔無怒,一切從之。凡河朔諸道健步奏計者,必獲賜賚。及德宗即位,嚴察神斷,自誅劉文喜之後,知朝法不可犯,四盜俱不自安。奏計者空還,無所賞賜,歸者多怨。或傳說飛語,云帝欲東封,汴州奏以城隘狹,增築城郭。李正己聞之,移兵萬人屯於曹州,田悅亦加兵河上;河南大擾,羽書警急。乃詔移京西戎兵萬二千人以備關東。帝禦望春
 樓親誓師以遣之,曰:「嗚呼!東鄙之警,事非獲已。唯爾將校群士,各以忠節,勤於王家;南赴蜀門,西定涇壘,甲胄不解,瘡痍未平;今載用爾分鎮於周、鄭之郊,敬聽明命。夫王者之師,有征無戰,稽諸理道,用正邦國。宜勵乃戈甲,保固城池,以德和人,以義制事。將備其侵軼,不用越境攻取,戢而後動,可謂正矣!今外夷來庭,方春生植,品物資始,農桑是時。俾爾將士,暴露中野,我心痛悼,鬱如焚灼。嗟爾有眾,其悉予懷。」士卒多泣下。及賜宴,諸將列
 坐;酒至,神策將士皆不飲,帝使問之。惠元時為都將,對曰:「臣初發奉天,本軍帥張巨濟與臣等約曰:『斯役也,將策大勛,建大名。凱旋之日,當共為歡;茍未戎捷,無以飲酒。』故臣等不敢違約而飲。」既發,有司供餼於道路,他軍無孑遺,唯惠元一軍瓶罍不發。上稱嘆久之,降璽書慰勞。



 及田悅反,詔惠元領禁兵三千與諸將討伐,戰御河,奪三橋,皆惠元之功也。尋加檢校工部尚書,攝貝州刺史,令以兵屬李懷光。建中四年冬,自河朔與懷光同赴
 國難,解奉天之圍。明年二月,懷光背國叛逆,惠元義不受污,脫身奔竄奉天。會乘輿南幸,懷光怒惠元之逸,令其將冉宗以百餘騎追及於好畤縣。惠元計窮,父子三人並投人家井中,冉宗並出而害之。興元元年,贈右僕射,仍賻絹百匹。惠元男尚食奉御晟,贈殿中監,左衛兵曹參軍皓贈邠州刺史,褒死難也。



 李元諒,本駱元光,姓安氏,其先安息人也。少為宦官駱奉先所養,冒姓駱氏。元諒長大美須,勇敢多計。少從軍,
 備宿衛,積勞試太子詹事。鎮國軍節度使李懷讓署奏鎮國軍副使,俾領州事。元諒嘗在潼關領軍,積十數年,軍士皆畏服。



 德宗居奉天,賊泚遣偽將何望之輕騎襲華州,刺史董晉棄州走;望之遂據城,將聚兵以絕東道。元諒自潼關將所部,仍令義兵因其未設備,徑攻望之。遂拔華州,望之走歸。元諒乃修城隍器械召募,不數日,得兵萬餘人,軍益振。以功加御史中丞。賊泚數遣兵來寇,輒擊卻之。是時,尚可孤守藍田,與元諒掎角;賊東不
 能逾渭南,元諒功居多。無幾,遷華州刺史、兼御史大夫、潼關防禦、鎮國軍節度使,尋加檢校工部尚書。



 興元元年五月,詔元諒與副元帥李晟進收京邑。兵次於滻西,賊悉眾來攻,元諒先士卒奮擊,大敗之。進軍至苑東,與晟力戰,壞苑垣而入,賊聯戰皆敗,遂復京師。元諒讓功於晟,出屯於章敬佛寺。帝還宮,加檢校尚書右僕射,實封七百戶,賜甲第、女樂,仍與一子六品正員官。



 李懷光反於河中,絕河津。詔元諒與副元帥馬燧、渾瑊同討之。
 時賊將徐庭光以銳兵守長春宮,元諒遣使招之。庭光素輕易元諒,且慢罵之;又以優胡為戲於城上,辱元諒先祖。元諒深以為恥。及馬燧以河東兵至,庭光降於馬燧,詔以庭光為試殿中監、兼御史大夫。河中平,燧待庭光益厚。元諒因遇庭光於軍門,命左右劫而斬之,乃詣燧匍匐請罪。燧盛怒,將殺元諒;久之,以其功高,乃止。德宗以元諒專殺,慮有章疏,先令宰相諭諫官勿論。



 貞元三年,詔元諒將本軍從渾瑊與吐蕃會盟於平涼。元諒
 謂瑊曰:「本奉詔,令營於潘原堡,以應援侍中。竊思潘原去平涼六七十里,蕃情多詐,倘有急變,何由應赴?請次侍中為營。」瑊以違詔,固止之。元諒竟與瑊同進。瑊營距盟所二十里,元諒營次之,壕柵深固。及瑊赴會,乃戒嚴部伍,結陣營中。是日,虜果伏甲,乘瑊無備竊發。時士大夫皆朝服就執,軍士死者十七八。瑊單馬奔還,群虜追躡,瑊營將李朝彩不能整眾,多已奔散;瑊至,空營而已。賴元諒之軍嚴固;瑊既入營,虜皆散去。是日無元諒軍,
 瑊幾不免。元諒乃整軍,先遣輜重,次與瑊俱申號令,嚴其部伍而還,時謂元諒有將帥之風。德宗嘉之,賜良馬十匹,金銀器、錦彩等甚厚。丁母憂,加右金吾衛上將軍,起復本官。帝念其勛勞,又賜姓李氏,改名元諒。



 四年春,加隴右節度支度營田觀察、臨洮軍使,移鎮良原。良原古城多摧圮,隴東要地,虜入寇,常牧馬休兵於此。元諒遠烽堠,培城補堞,身率軍士,與同勞逸。芟林薙草,斬荊榛,俟乾,盡焚之,方數十里,皆為美田。勸軍士樹藝,歲收
 粟菽數十萬斛,生殖之業,陶冶必備。仍距城築臺,上彀車弩,為城守備益固。無幾,又進築新城,以據便地。虜每寇掠,輒擊卻之,涇、隴由是乂安,虜深憚之。以疾,貞元九年十一月,卒於良原,年六十二。帝甚悼惜,廢朝三日,贈司空,賻布帛米粟有差。



 韓游瑰,河西靈武人。仕本軍,累歷偏裨,積功至邠寧節度使。德宗出幸奉天,衛兵未集,游瑰與慶州刺史論惟明合兵三千人赴難,自乾陵北過赴醴泉以拒泚。會有
 人自京城來,言賊信宿當至,上遽令追游瑰等軍伍。才入壁,泚黨果至。乃出鬥城下,小不利,乃退入城。賊急奪門,游瑰與賊隔門血戰,會暝方解。自是賊日攻城,游瑰、惟明乘城拒守,躬當矢石,不暇寢息,赴難之功,游瑰首焉。



 李懷光反,從駕山南。德宗以禁軍無職局,六軍特置統軍一員,秩從二品,以游瑰、惟明、賈隱林等分典從駕禁兵。李晟移軍東渭橋,與駱元光、尚可孤分扼京東要路;渾瑊與游瑰、戴休顏分典京西要路,掎角進攻。興元
 元年,檢校刑部尚書、兼御史大夫,例授「奉天定難功臣」。李晟收京城,游瑰三將亦破賊於咸陽。德宗自興元還京,渾瑊與游瑰、休顏三將從;李晟、尚可孤、駱元光三將奉迎,論功行封,與瑊等相次,還鎮邠寧。



 三年,以子欽緒與妖賊李廣弘同謀不軌,時游瑰鎮長武城,事將發,欽緒奔於邠州;邠州將吏械送京師。游瑰以子大逆,請代歸,固欲詣闕,詔不許。游瑰鎖系欽緒二子送京師,請從坐,上亦宥之。十二月,游瑰入朝,素服待罪,入朝堂;遽命
 釋之,勞遇如故,復令還鎮。初,游瑰入覲,邠州將吏以其子謀叛,又御軍無政,謂必受代,餞送之禮甚薄。及游瑰見上,盛論邊事,請築豐義城以備蕃寇,上以特達,委用如初。及還鎮,軍中懼不自安。大將範希朝善將兵,名聞軍中;游瑰畏其逼己,將因事誅之。希朝懼,出奔鳳翔。上素知名,召入宿衛。及游瑰遣五百人築豐義城,兩板而潰。又寧州戍卒數百人,縱掠而叛。其無方略,失士心,皆此類也。自寧州卒叛,吐蕃入寇,游瑰自率眾戍寧州。



 四
 年七月,除將軍張獻甫代游瑰,不俟獻甫至,又不告眾知,乃輕騎夜出歸朝。將卒素驕,聞獻甫嚴急,因其無帥,縱兵大掠,且圍監軍楊明義第,請奏範希朝為帥。都虞侯楊朝晟初逃難郊外,翌日聞請希朝,乃復入城,與軍眾曰:「所請甚愜,我來賀也。」叛卒稍安。朝晟乃與諸將密謀,晨率甲兵而出,召叛卒告曰:「前請者不獲,張尚書來,昨日已入邠州。汝等謀叛,皆當死。吾不盡殺,誰為賊首,各言之,以罪歸之,餘悉不問。」於眾中唱二百餘人,立斬
 之,軍城方定。上聞軍情欲希朝,乃授寧州刺史,為獻甫邠寧之副。游瑰至京,授右龍武統軍。十四年卒。



 李廣弘者,或云宗室親王之胤。落發為僧,自雲見五岳、四瀆神,己當為人主。貞元三年,自邠州至京師,有市人董昌者,通導廣弘,舍於資敬寺尼智因之室。智因本宮人。董昌以酒食結殿前射生將韓欽緒、李政諫、南珍霞,神策將魏修、李傪,前越州參軍劉昉、陸緩、陸絳、陸充、徐綱等,同謀為逆。廣弘言嶽瀆神言,可以十月十日舉事,必捷。自
 欽緒已下,皆有署置為宰相,以智因尼為後。謀於舉事日,夜令欽緒擊鼓於凌霄門,焚飛龍廄舍草積;又令珍霞盜擊街鼓,集城中人;又令政諫、修、傪等領射生、神策兵內應;事克,縱剽五日,朝官悉殺之。事未發,魏修、李傪上變,令內官王希遷等捕其黨與斬之,德宗因禁止諸色人不得輒入寺觀。



 賈隱林者,滑州牙將也。建中初,為本軍兵馬使,令率兵宿衛。硃泚之亂,諸軍未集,隱林率眾扈從。性質樸,在奉
 天,賊急攻城,隱林與侯仲莊逐急救應,難險備至。既而懷光軍至,逆賊解圍,從臣稱慶。隱林抃舞畢,奏曰:「賊泚奔遁,臣下大慶,此皆宗社無疆之休。然陛下性靈太急,不能容忍,若舊性未改,賊雖奔亡,臣恐憂未艾也。」上不以為忤,甚稱之。累官至檢校右散騎常侍,封武威郡王。將幸山南而卒,贈左僕射,賜其家實封三百戶,賻絹百匹、米百碩,喪葬官給。



 杜希全,京兆醴泉人也。少從軍,嘗為郭尚父子儀裨將,
 積功至朔方軍節度使;軍令嚴肅,士卒皆悅服。初,德宗居奉天,希全首將所部與鹽州刺史戴休顏、夏州刺史時常春合兵赴難。軍已次漠谷,為賊泚邀擊,乘高縱礧,又以大弩射之,傷者眾。德宗令出兵援之,不得進;希全退次邠州。以赴難功,加檢校戶部尚書、行在都知兵馬使。從幸梁州。帝還京師,遷太子少師、檢校右僕射,兼靈州大都督、御史大夫、受降定遠城天德軍,靈鹽豐夏等州節度支度營田觀察押蕃落等使、餘姚郡王。



 希全將
 赴靈州,當獻《體要》八章,多所規諫。德宗深納之,乃著《君臣箴》以賜之,其辭曰:



 夫惟德惠人,惟闢奉天,從諫則聖,共理惟賢。皇立有極,駿命不易,總萬機以成務,齊六合之殊致。一心不能獨鑒,一目不能周視,敷求哲人,式序在位。於戲!君之任臣,必求一德;臣之事君,咸思正直。何啟沃之所宜,自古今而未得?且以讜言者逆耳,讒諛者伺側,故下情未通,而上聽已惑;俾夫忠賢,敗於兇慝。譬彼輕舟,烝徒楫之;亦有和羹,宰夫膳之。孰雲理國,不自
 得師,覆車之軌,予其懲而。高以下升,和由甘受,惟君無良,亦臣之咎。聞諸辛毗,牽裾魏後,則有禽息,竭忠碎首,勉思獻替,以平可否。勿謂無傷,自微而彰;勿謂何害,積小成大。事有隱而必見,令既出而焉悔!鼓鐘在宮,聲聞於外,浩然涉水,朕未有艾。將負扆以虛心,期盡忠而納誨。在昔稷、契,實匡舜、禹;近茲魏徵,佑我文祖,君臣協德,混一區宇。肆予寡昧,獲纘丕緒,臣哉鄰哉,爾翼爾輔。



 高秋始肅,我武惟揚,輟此禁衛,殿於大邦。戀闕方甚,嘉言
 乃昌,是規是諫,金玉其相。辭高理要,入德知方,總彼千慮,備於八章,宣父有言,啟予者商。殷有盤銘,周有欹器,或誡以辭,或警以事。披圖演義,發於爾志,與金鏡而高懸,將座右而同置。人皆有初,鮮慎厥終,汝其夙夜,期保朕躬。無曰爾身在外,而爾誠不通,一言之應,千里攸同。導彼遐徐,達餘四聰,華夷仰德,時乃之功。既往既來,懷賢忡忡,唱予和汝,式示深衷。



 尋兼本管及夏綏節度都統,加太子少師。希全以鹽州地當要害,自貞元三年西
 蕃劫盟之後,州城陷虜;自是塞外無保障,靈武勢隔,西通鄜坊,甚為邊患,朝議是之。九年,詔曰:



 設險守國,《易象》垂文,有備無患,先王令典。況修復舊制,安固疆里,偃甲息人,必在於此。



 鹽州地當沖要,遠介朔陲,東達銀夏,西援靈武,密邇延慶,保捍王畿。乃者城池失守,制備無據,千里庭障,烽燧不接,三隅要害,役戍其勤。若非興集師徒,繕修壁壘,設攻守之具,務耕戰之方,則封內多虞,諸華屢警,由中及外,皆靡寧居。深惟永圖,豈忘終食!顧以
 薄德,至化未孚,既不能復前古之治,致四夷之守,與其臨事而重擾,豈若先備而即安!是用弘久遠之謀,修五原之壘,使邊城有守,中夏克寧,不有暫勞,安能永逸?



 宜令左右神策及朔方河中絳邠寧慶兵馬副元帥渾瑊、朔方靈鹽豐夏綏銀節度都統杜希全、邠寧節度使張獻甫、神策行營節度使邢君牙、銀夏節度使韓潭、鄜坊節度使王棲曜、振武節度使範希朝,各於所部簡練將士,令三萬五千人同赴鹽州。神策將軍張昌宜權知鹽
 州事,應板築雜役,取六千人充。其鹽州防秋將士,率三年滿更代,仍委杜彥先具名奏聞,悉與改轉。



 朕情非己欲,志在靖人。咨爾將相之臣,忠良之士,輸誠奉命,陳力忘憂,勉茂功勛,永安疆場。必集兵事,實惟眾心,各相率勵,以副朕志。



 凡役六千人,二旬而畢。時將板築,仍詔涇原、劍南、山南諸軍深討吐蕃以牽制之,由是板築之時,虜不及犯塞。城畢,中外稱賀。由是靈武、銀夏、河西稍安,虜不敢深入。



 希全久鎮河西,晚節倚邊多恣橫,帝嘗寬
 之。豐州剌史李景略威名出其右,希全深忌之,疑畏代己,乃誣奏景略;德宗不得已為貶之。素病風眩,暴戾益甚。判官監察御史李起頗忤之,希全又誣奏殺之。將吏皆重足脅息。貞元十年正月卒,廢朝三日,贈司空。



 尉遲勝,本於闐王珪之長子,少嗣位。天寶中來朝,獻名馬、美玉,玄宗嘉之,妻以宗室女,授右威衛將軍、毗沙府都督,還國。與西安節度使高仙芝同擊破薩毗播仙,以功加銀青光祿大夫、鴻臚卿,改光祿卿,皆同正。



 至德初,
 聞安祿山反,勝乃命弟曜行國事,自率兵五千赴難。國人留勝,以少女為質而後行。肅宗待之甚厚,授特進,兼殿中監。廣德中,拜驃騎大將軍、毗沙府都督、于闐王,令還國。勝固請留宿衛,加開府儀同三司,封武都王,實封百戶。勝請以本國王授曜,詔從之。勝乃於京師修行裏盛飾林亭,以待賓客,好事者多訪之。



 建中末,從幸奉天,為兼御史中丞。駕在興元,勝為右領軍將軍,俄遷右威衛大將軍,歷睦王傅。



 貞元初,曜遣使上疏,稱:「有國以來,
 代嫡承嗣,兄勝既讓國,請傳勝子銳。」上乃以銳為檢校光祿卿、兼毗沙府長史還。固辭,且言曰:「曜久行國事,人皆悅服。銳生於京華,不習國俗,不可遣往。」因授韶王諮議。兄弟讓國,人多稱之。府除,以勝為原王傅。卒。時年六十四。貞元十年,贈涼州都督。子銳嗣。



 邢君牙,瀛州樂壽人也。少從軍於幽薊、平盧,以戰功歷果毅折沖郎將,充平盧兵馬使。安祿山反,隨平盧節度使侯希逸過海,至青、徐間。田神功之討劉展,君牙又從
 神功戰伐有功,歷將軍、試光祿卿。神功既為兗鄆節度使,令君牙領防秋兵入鎮好畤。屬吐蕃陵犯,代宗幸陜,君牙隸屬禁軍扈從。後又以戰功加鴻臚卿,累封河間郡公。



 建中初,河北諸節帥叛,李晟率禁軍助馬燧等征之。晟以君牙為都虞候,累於武安、襄國、洹水、魏縣、清豐討賊有功,君牙擒生斬級居多。屬德宗幸奉天,晟率君牙統所部兵,倍道兼程,來赴國難。及駐軍咸陽,移營渭橋,軍中之事,晟惟與君牙商之,他人莫可得而聞也。收
 復宮闕,驟加御史大夫、檢校常侍。既而晟為鳳翔、涇原元帥,數出軍巡邊,常令君牙掌知留後,軍府安悅。貞元三年,晟以太尉、中書令歸朝,君牙代為鳳翔尹、鳳翔隴州都防禦觀察使,尋遷右神策行營節度、鳳翔隴州觀察使,加檢校工部尚書。吐蕃連歲犯邊,君牙且耕且戰,以為守備,西戎竟不能為大患。尋加檢校右僕射。貞元十四年卒,時年七十一,廢朝一日,贈司空,賻布帛米粟有差。



 楊朝晟,字叔明,夏州朔方人。初,在朔方為部軍前鋒,常有功,授甘泉果毅。建中初,從李懷光討劉文喜於涇州,斬獲擒生居多,授驃騎大將軍,稍遷右先鋒兵馬使。後李納寇徐州,從唐朝臣征討,常冠軍鋒,以功授開府儀同三司、檢校太子賓客。



 上在奉天,李懷光自山東赴難,以朝晟為右廂兵馬使,將千餘人下咸陽,以挫硃泚。加御史中丞,實封一百五十戶。及懷光反於河中,朝晟被脅在軍。上幸梁、洋,韓游瑰退於邠寧,懷光以嘗在邠寧,
 迫制如屬城,以賊黨張昕在邠州總後務。昕懼難作,乃大索軍資,征卒乘,約明潛發,歸於懷光。時朝晟父懷賓為游瑰將,夜後以數十騎斬昕及同謀者。游瑰即日使懷賓奉表聞奏,上召勞問,授兼御史中丞,正授游瑰邠寧節度使。間諜至河中,朝晟聞其事,泣告懷光曰:「父立功於國,子合誅戮,不可主兵。」懷光遂系之。及諸軍進圍河中,韓游瑰營於長春宮,懷賓身當戰伐。及懷光平,上念其忠,俾副元帥渾瑊特原朝晟,用為游瑰都虞候。時
 父子同軍,皆為開府、賓客、御史中丞,異姓王,榮於軍中。



 後詔徵游瑰宿衛,以張獻甫代之。獻甫在道,軍中有裴滿者,扇亂劫朝晟,朝晟陽許之,密計斬三百餘人。獻甫入,改御史大夫。九年,城鹽州,徵兵以護外境,朝晟分統士馬鎮木波堡。獻甫卒,詔以朝晟代之。其年,丁母憂,起復左金吾大將軍同正、邠州刺史、兼御史大夫。十三年春,朝晟奏:「方渠、合道、木波,皆賊路也,請城其地以備之。」詔問:「須兵幾何?」朝晟奏曰:「臣部下兵自可集事,不煩外
 助。」復問:「前築鹽州,凡興師七萬,今何其易也?」朝晟曰:「鹽州之役,咸集諸軍,番戎盡知之。今臣境迫虜,若大興兵,即番戎來寇;來寇則戰,戰則無暇城矣!今請密發軍士,不十日至塞下,未旬而功畢,番人始知,已無奈何。」上從之。已事,軍還至馬嶺,吐蕃始來,數日而退。



 初,軍次方渠,無水,師旅囂然。遽有青蛇乘高而下,視其跡,水隨而流,朝晟命築防環之,遂為渟泉。軍人仰飲以足,圖其事上聞,詔置祠焉。免喪,加檢校工部尚書。是夏,以防秋移軍
 寧州,遘疾,旬餘而卒。



 張敬則者,不知何許人,本名昌,後賜名敬則。初助劉玄佐,累有軍功,官至鳳翔節度使。常有復河湟之志,遣大將野詩良輔發銳卒至隴西,番戎大駭。元和二年六月卒。



 史臣曰:有唐中否,逆寇勃興,天王窘以蒙塵,諸侯忠而赴難。可孤生居沙漠,挺然懷效命之風;功冠貔貅,屹爾有不矜之色。李觀文儒之胄,樂習兵戎,戴聖主著定難
 之勛,救渾瑊於會盟之變。休顏斬使嬰城,懷光股慄;惠元窮蹙自致,天子軫悼。元諒退兵章敬,力戰讓功,雅有器度。及不忍小忿,專殺庭光,請罪軍門,壯哉烈士!其下諸將,鬱有勞能。勝生異域,推位讓國,堅留宿衛,顧慕華風;居中土者,豈不思廉讓耶!斯乃高祖之基,太宗之業,貽厥孫謀,不徒虛語。



 贊曰:建中失國,嘯聚氛慝。景命載延,群雄畢力。歌鐘甲第,珪組繁錫。凡百人臣,忠為令德。



\end{pinyinscope}