\article{卷十二 本紀第十二 德宗上}

\begin{pinyinscope}

 德宗神武孝文皇帝諱適,代宗長子,母曰睿真皇後沈氏。天寶元年四月癸巳,生於長安大內之東宮。其年十二月,拜特進,封奉節郡王。代宗即位之年五月,以上為天下兵馬元帥,改封魯王。八月,改封雍王。時史朝義據東都,十月,遣上會諸軍於陜州,大舉討賊。十一月,破賊於洛陽,進收
 東都,河南平定。朝義走河北。分命諸將追之,俄而賊將懷仙斬朝義首以獻,河北平。以元帥功拜尚書令,食實封二千戶,與郭子儀等八人圖形凌煙閣。廣德二年二月,立為皇太子。



 大歷十四年五月辛酉,代宗崩。癸亥,即位於太極殿。閏月壬申,貶中書舍人崔祐甫為河南少尹。甲戌,貶門下侍郎、平章事常袞為潮州刺史。召崔祐甫為門下侍郎、同中書門下平章事。丙子,詔諸州府、新羅、渤海歲貢鷹鷂皆停。戊寅,詔山南枇杷、江南柑橘,歲一貢以供宗廟,餘貢
 皆停。庚寅,以兵部尚書路嗣恭為東都留守,以常州刺史蕭復為潭州刺史、湖南團練觀察使。辛巳,罷邕府歲貢奴婢。癸未,改括州為處州,括蒼縣為麗水縣。停梨園使及伶官之冗食者三百人,留者皆隸太常。劍南歲貢春酒十斛,罷之。甲申,以司徒、兼中書令、河中尹、靈州大都督、單于鎮北大都護充關內河東副元帥、朔方節度、關內支度鹽池六城水運大使、押諸蕃部落、管內及河陽等道觀察使、上柱國、汾陽郡王、山陵使、食實封一千
 九百戶郭子儀可加號尚父,守太尉,餘官如故,加實封通前二千戶,月給一千五百人糧、馬二百匹草料。以朔方都虞候李懷光為河中尹,邠、寧、慶、晉、絳、慈、顯等州節度觀察使;以朔方右留後常謙光靈州大都督,西受降城、定遠軍、天德、鹽、夏、豐節度等使;以朔方左留後、單于副都護渾璘為單于大都護,振武軍、東中二受降
 城、鎮北及綏、銀、麟、勝等軍州節度營田使。丙戌,詔禁天下不得貢珍禽異獸,銀器勿以金飾。丁亥,詔文單國所獻舞象三十二,令放荊山之陽,五坊鷹犬皆放之,出宮女百餘人。己丑,以右羽林大將軍吳希光檢校散騎常侍、兼御史中丞,充渭北鄜坊丹延都團練觀察使。辛卯,以河陽三城鎮遏使馬燧檢校工部尚書,兼太原尹、御史大夫、北都留守、河東節度使。壬辰,以河東節度留後鮑防為京畿觀察使;陳州刺史李芃檢校太常少卿,為河陽
 三城鎮遏使。癸巳,以壽州刺史杜亞為江西觀察使。甲午,冊太尉子儀。自開元以來,冊禮多廢,天寶中楊國忠冊司空,至是行子儀之冊。以江西觀察使杜亞為
 陜州長史,充轉運使。丙申,詔兵部侍郎黎幹害若豺狼,特進劉忠翼掩義隱賊,並除名長流。既行,俱賜死。丁酉,以京畿觀察使鮑防為福州刺史、福建都團練觀察使。以戶部侍郎、判度支韓滉為太常卿,吏部尚書劉晏判度支鹽鐵轉運等使。初,晏與滉分掌天下財賦,至是晏都領之。



 六月己亥朔,御丹鳳樓,大赦天下,罪無輕重,咸赦除之。內外文武三品已上賜爵一級,四品已下加一階,致仕官同見任,百姓為戶者賜古爵一級。加李正己司徒、太子太傅,崔寧、李勉本
 官同平章事。天下進獻,事緣郊祀陵廟所須,依前勿闕,餘並停。諸州刺史上佐今後準式入計。諸州刺史、常參官,父在未有官,量與五品致仕官;父亡歿,與追贈。自至德已來別敕,或因人奏,或臨事頒行,差互不同。使人疑惑,中書門下與詳定官決,取堪久長行用者
 編入格條。自今更不得奏置寺觀及度人。庚子,封元子育為宣王,次子謨為舒王,諶為通王,諒為虔王,詳為肅王,並加開府儀同三司。乙巳,封皇弟乃為益王,迅為隨王。丙午,舉先天故事,非供奉侍衛之官,自文武六品已上清望官,每日二人更直待制,以備顧問,乃以延英南藥院故地為廨。癸丑,詔皇族五服等已上居四方者,家一人赴山陵,縣次給食。己未,揚州每年貢端午日江心所鑄鏡,幽
 州貢麝香,皆罷之。辛酉,罷宣歙池、鄂岳沔二都團練觀察使。陜虢都防禦使,以其地分隸諸道。復置東都京畿觀察使,以
 御史中丞為之。壬戌,處州刺史王縉、湖州刺史第五琦皆為太子賓客,睦州刺史李揆為國子祭酒,並留司東
 都。
 中官邵光超送淮西旌節,李希烈遺縑七百匹,事發,杖六十,配流。由是中官不敢受賂。癸亥,詔中書門下、御史臺五品已上,諸司三品已上長官,各舉可任刺史縣令者一人,中書門下量才進擬,有犯坐
 舉主。秋



 七月戊辰朔,日有蝕之。禮儀使、吏部尚書顏真卿奏:「列聖謚號,文字繁多,請以初謚為定。」兵部侍郎袁傪議云:「陵廟玉冊已刻,不可輕改。」罷。傪妄奏,不知玉冊皆刻初謚而已。庚午,詔:邕州所奏金坑,誠為潤國,語人以利,非朕素懷。其坑任人開採,官不得禁。辛未,以吏部侍郎房宗偃為御史中丞、東都畿觀察使。罷右銀臺門客省歲給廩料萬二千斛。自永泰已來,或四方奏計未遣者,或上書言事忤旨者,及蕃客未報者,常數百人,於
 客省給食,橫費已甚,故罷之。壬申,毀元載、馬璘、劉忠翼之第,以其雄侈逾制也。癸酉,減宮中服御常貢者千數。丁丑,復置廄馬隨仗於月華門外。己卯,詔王公卿士不得與民爭利,諸節度觀察使於揚州置回易邸,並罷之。庚辰,詔鴻臚寺,蕃客入京,各服本國之服。罷商州歲貢離膠。辛卯,罷天下榷酒。丁酉,詔國用未給。其宣王已下開府俸料皆罷給。



 八月甲辰,以門下侍郎、平章事崔祐甫為中書侍郎、平章事,以道州司馬同正楊炎為門下
 侍郎、平章事,以懷州刺史喬琳為御史大夫、同平章事、京畿觀察使。乙巳,遣太常少卿韋倫使吐蕃,以蕃俘五百人還之,修好也。癸亥,詔人死亡於外以棺柩還城者勿禁。九月甲戌,以淮西節度為淮寧軍。辛巳,以檢校刑部尚書白孝德為太子少傅。丙戌,秘書少監邵說為吏部侍郎,給事中劉乃為兵部侍郎,中書舍人令狐峘為禮部侍郎。



 冬十月丁酉朔,吐蕃合南蠻之眾號二十萬,三道寇茂州、扶、文、黎、雅等州,連陷郡邑。發兵四千助蜀,
 大破之。己酉,葬代宗於元陵。戊午,九成宮貢立獸炭爐,襄州貢種蔗蒻之工,皆罷之。散官豢豬三千頭給貧民。十一月辛未,以鴻臚卿賈耽為梁州刺史、山南西道節度觀察使。丁丑,以陜州長史杜亞為河中尹、河中晉絳慈隰都防禦觀察使。壬午,御史大夫、平章事喬琳為工部尚書,罷知政事。加劍南西川節度觀察度支營田等使、檢校司空、平章事、成都尹崔寧兼御史大夫、京畿觀察使。癸巳,加崔寧兼靈州大都督、單于鎮北大都護、朔
 方節度等使、出鎮坊州。以荊南節度使、檢校禮部尚書、兼江陵尹、御史大夫張延賞檢校兵部尚書兼成都尹、御史大夫、劍南西川節度度支營田觀察等使。以朔方節度虞候杜希全為靈州留後;以鄜州刺史張光晟單于振武軍使、東中二受降城綏銀鄜勝等軍州留後;延州刺史李建為鄜坊慶延留後。楊炎素惡崔寧,雖授以三鎮,仍署此三人為留後,奪寧之權也,人皆憤之。十二月己亥,南選使可以專達,勿復以御史臨之。乙卯,制:宣
 王某可立為皇太子。丙寅晦,日有蝕之。詔元日朝會不得奏祥瑞事。



 建中元年春正月丁卯朔,御含元殿,政元建中,群臣上尊號曰聖神文武皇帝。己巳,上朝太清宮。庚午,謁太廟。辛未,有事於郊丘。是日還宮,御丹鳳門,大赦天下。自艱難以來,征賦名目頗多。今後除兩稅外,輒率一錢,以枉法論。常參官、諸道節度觀察防禦等使、都知兵馬使、刺史、少尹、畿赤令、大理司直評事等,授訖三日內,於四方
 館上表讓一人以自代。其外官委長吏附送其表,付中書門下。每官闕,以舉多者授之。王府六品以上官及諸州縣有司可並省及諸官減者,量事廢省。天下子為父後者賜勛兩轉。己巳,福建觀察使鮑防、湖南觀察使蕭復讓憲官,從之。自兵興已來,方鎮重任必兼臺省長官,以至外府僚佐,亦帶臺省銜。至是除韓滉蘇州刺史,杜亞河中少尹,而領都團練觀察使,不帶臺省兼官。自是諸道非節度而兼憲官者皆讓。甲午,詔:「東都河南江淮
 山南東道等轉運租庸青苗鹽鐵等使、尚書左僕射晏,頃以兵車未息,權立使名,久勤元老,集我庶務,悉心瘁力,垂二十年,朕以征稅多門,鄉邑凋耗,聽於群議,思有變更,將置時和之理,宜復有司之制。晏所領使宜停,天下錢穀委金部、倉部,中書門下揀兩司郎官,準格式調掌。」是月,浚豐州陵陽渠。



 二月丙申,遣黜陟使一十一人分行天下。癸卯,以戶部郎中韓洄為諫議大夫,以涇原節度使段秀實為司農卿。己酉,貶尚書左僕射劉晏
 為忠州刺史。癸丑,昭義軍節度留後李抱真為本道節度使。甲寅,貶史館修撰、禮部侍郎令狐峘郴州司馬,右補闕柳冕巴州司戶。日本國朝貢。癸亥,硃泚兼四鎮北庭行軍、涇原節度使。



 三月丙寅,禮儀使奏東都太廟闕木主,請造。詔下議之,不決。庚午,監察御史張著以法冠彈中丞嚴郢浚陵陽渠匿詔不行,消郢官,著賜緋魚。辛未,左散騎常侍、翰林學士張涉放歸田里。甲戌,以前司農卿庾準為江陵尹、兼御史中丞、荊南節度使。癸巳,以
 諫議大夫韓洄為戶部侍郎、判度支。時將貶劉晏,罷使名,歸尚書省本司。今又命洄判度支,令金部郎中杜佑權勾當江淮水陸運使,一如劉晏、韓滉之則,蓋楊炎之排晏也。



 夏四月乙未朔,涇原裨將劉文喜據城叛。己亥,地震。辛未,命江西觀察使崔昭冊命回紇可汗。戊申,以福建觀察使鮑防為洪州刺史、江西團練觀察使。癸丑,上誕日,不納中外之貢,唯李正己、田悅各獻縑三萬匹,詔付度支。妃父王景先、駙馬高怡獻金銅像,上曰:「有何
 功德?非吾所為。」退還之。壬戌,以衡州刺史、嗣曹王皋為潭州刺史、湖南團練觀察使,御史中丞元全柔為杭州刺史。五月甲子朔。戊辰,以太常少卿韋倫為太常卿,復使吐蕃。己卯,右金吾衛大將軍李通為黔州刺史、黔中經略招討觀察鹽鐵等使。潮州刺史常袞為福建觀察使。涇州將劉光國殺劉文喜降,涇州平。



 六月甲午朔,中書侍郎、同中書門下平章事崔祐甫卒。辛丑,築奉天城。加試殿中監劉海賓兼御史中丞封兵平郡王。海賓涇
 州將,賞殺劉文喜也。乙卯,京兆尹源休使回紇,冊武義成功可汗。秋七月丁丑,罷內出盂蘭盆,不命僧為內道場。壬申,以鴻臚寺左右威遠營隸金吾。己醜惡,忠州刺史劉晏賜自盡。



 八月甲午,振武軍使張光晟殺領蕃回紇首領突董統等千人,收駝馬千餘、繒錦十萬匹。乃徵光晟歸朝,以彭令芳代之。乙未,河中晉、絳觀察使杜亞為睦州刺史。丁未,加硃泚中書令,餘官使並如故。以舒王謨為涇原節度大使,尚書右丞孟皞為涇州刺史、知留
 後。東僰烏蠻守來朝貢。丁巳,遙尊上母沈氏曰皇太后。戊午,以吏部尚書顏真卿為太子少師,依前禮儀使。改封嗣舒王藻為嗣郢王。九月戊辰,判度支韓洄奏請於商州紅崖冶洛源監置十爐鑄錢。江淮七監每鑄一千費二千文,請皆罷,從之。己卯,雷。



 冬十月甲午,貶尚書左丞薛邕為連山尉,坐贓也。乙巳,太子少傅、昌化郡王白孝德卒。庚寅,以睦王述為奉迎皇太后使,工部尚書喬琳為副。十一月辛酉朔,朝集使及貢使見於宣政殿,兵
 興已來,四方州府不上計、內外不朝會者二十有五年,至此始復舊制。州府朝集者一百七十三人,詔每令分番二人待詔。乙丑,贈敬暉等五王官,又贈張九齡司徒,鐘紹京太子太傅。戊寅,諸王有官者初令出閣就班。又出嫁岳陽等一十縣主,皆在諸王院久而未適人者,上悉命以禮出降。十二月辛卯,韋倫使回,與吐蕃宰相論欽明思等五十五人同至,獻方物,修好也。丁酉,令詳定國初以來將相功臣房玄齡等一百八十七人,據功績分
 為三等。是歲,戶部計帳,戶總三百八萬五千七十有六,賦入一千三百五萬六千七十貫,鹽利不在此限。



 二年春正月庚申朔。戊辰,成德軍節度、恆定等州觀察使、司空、兼太子太傅、同中書門下平章事、恆州刺史、隴西郡王李寶臣卒。丙子,以汴宋滑亳陳潁泗節度觀察使、檢校帶領部尚書、同平章事李勉為永平軍節度、汴滑陳等州觀察等使;以兵部尚書、東都留守路嗣恭為鄭汝陜河陽三城節度、東畿觀察等使;以宋州刺史劉洽
 為宋亳潁節度使。以鄭州隸永平軍。自去年十月無雪,至甲申方雨雪。丁亥,檢校戶部尚書張獻恭為東都留守。以河南尹趙惠伯為河中尹、河中晉絳慈顯都防禦觀察使,以前鄭州刺史于頎為河南尹。



 二月乙未,以御史中丞盧杞為御史大夫、京畿觀察使,以桂管觀察使李昌巙為江陵尹、兼御史大夫、荊南節度等使。以前荊南節度使庾準為左丞。甲辰,以容州刺史盧嶽為桂州防禦觀察使。乙巳,以門下侍郎楊炎為中書侍郎、同中
 書門下平章事,以御史大夫盧杞為門下侍郎、同中書門下平章事。丙午,以宋亳節度為宣武軍。丁未,以御史中丞袁高為京畿觀察使。乙卯,振武軍亂,殺其帥彭令芳、監軍劉惠光。



 三月庚申朔,築汴州城。初,大歷中李正己有淄、青、齊、海、登、萊、沂、密、德、棣、曹、濮、徐、兗、鄆十五州之地,李寶臣有恆、定、易、趙、深、冀、滄七州之地,田承嗣有魏、博、相、衛、洺、貝、澶七州之地,梁崇義有襄、鄧、均、房、復、郢六州之地,各聚兵數萬。始因叛亂得位,雖朝廷寵待加恩,心
 猶疑貳,皆連衡盤結以自固。朝廷增一城,浚一池,便飛語有辭,而諸盜完城繕甲,略無寧日。至是田悅初稟命,劉文喜殄除,群兇震懼。又奏計者還,都無賜與,既歸,皆構怨言。先是汴州以城隘不容眾,請廣之。至是築城,正己、田悅移兵於境為備,故詔分汴、宋、滑為三節度,移京西防秋兵九萬二千人以鎮關東。又於郾城置溵州。辛巳,以汾州刺史王翃為振武軍使、東中二受降城鎮北綏銀麟勝等州留後。以萬年令崔漢衡為殿中少監,使
 吐蕃。



 夏四月己酉朔,省沔州。庚寅,襄州梁崇義兼同中書門下平章事。己亥,省燕州、順化州。乙卯,並平琴州為黨州。丁巳,貶禮部侍郎於召桂州刺史,御史中丞袁高韶州長史。五月丙寅,以軍興十一而稅。己巳,以淮寧軍節度使李希烈充漢南北諸道都知兵馬招撫處置等使,封南平王。庚寅,以浙江西道為鎮海軍。加蘇州刺史韓滉檢校禮部尚書、潤州刺史,充鎮海軍節度使、浙江東西道觀察等使。以御史中丞一員為理匭使,諫議大
 夫一員知匭使;給事中、中書舍人為監考使。辛丑,尚父、中書令、汾旭郡王郭子儀薨。丙午,以檢校秘書少監鄭叔則為御史中丞、東都畿觀察使。壬子,以懷鄭、河陽節度副使李芃為河陽三城、懷州節度使,仍割東畿五縣隸焉。



 秋七月戊子朔,詔曰:「二庭四鎮,統任西夏五十七蕃、十姓部落,國朝以來,相奉率職。自關、隴失守,東西阻絕,忠義之徒,泣血相守,慎固封略,奉遵禮教,皆侯伯守將交修共理之所致也。伊西、北庭節度觀察使李元忠
 可北庭大都護,四鎮節度留後郭昕可安西大都護、四鎮節度觀察使。」自河、隴陷虜,伊西、北庭為蕃戎所隔,間者李嗣業、荔非元禮、孫志直、馬璘輩皆遙領其節度使名。初,李元忠、郭昕為伊西北庭留後,隔絕之後,不知存亡,至是遣使歷回紇諸蕃入奏,方知音信,上嘉之。其伊西、北庭將士敘官,仍超七資。庚申,以中書侍郎、平章事楊炎為左僕射,以前永平軍節度使張鎰為中書待郎、同中書門下平章事。司空、淮陽郡王侯希逸卒,丁丑,以
 河中尹關播為給事中,同州刺史李承為河中尹、晉絳都防禦觀察使。辛巳,以邠寧節度使李懷光兼靈州大都督、單于鎮北大都護、朔方節度使。以鄜坊、丹延觀察留後李建徽為坊州刺史、鄜坊丹延都團練觀察使。壬午,以幽州隴右節度使、中書令硃泚為太尉。田悅攻寇臨洺,守將張伾城守。



 八月辛卯,平盧淄青節度觀察使、司徒、太子太保、同中書門下平章事李正己卒。庚戌,以中書舍人衛晏為御史中丞、京畿觀察使。壬子,淮寧軍
 節度使李希烈攻襄陽,誅梁崇義,斬其同惡三十餘人。



 九月辛酉,以易州刺史張孝忠為恆州刺史,充成德軍節度觀察使。壬戌,加李希烈同中書門下平章事。癸亥,兵部尚書、翼國公路嗣恭卒。甲子,以晉絳觀察使李承為襄州刺史、山南東道節度觀察等使。戊辰,以杭州刺史元全柔為黔中經略招討觀察等使。



 冬十月乙酉,尚書左僕射楊炎貶崖州司馬,尋賜死。戊申,加宣武軍節度使劉洽御史大夫。徐州刺史李洧棄其帥李納,以州來
 降。十一月辛未,宣武節度劉洽與神策將曲環大破李納之眾於徐州。己巳,詔:「成德軍節度都知兵馬使、恆州刺史、襲隴西郡王李惟岳,以其父寶臣有忠勞於王室,惟岳隳墜父業,蔑棄國恩,縗絰之中,擅掌戎務。外結兇黨,益固奸謀,不孝不忠,宜肆原野。削爾在身官爵。」乙亥,貶戶部侍郎、判度支韓洄蜀州刺史,以江淮轉運使、度支郎中杜佑代判度支、戶部事。丁丑,以陜州長史李齊為河中尹,充河中晉絳防禦觀察使;以商州刺史
 姚明揚為陜州長史、本州防禦、陸運使;以權鹽鐵使、戶部郎中包佶充江淮水陸運使。李納將海州刺史王涉以州降。十二月庚寅,河中節度使馬燧檢校左僕射,澤潞節度使李抱真檢校兵部尚書,賞破田悅之功也。丙申,太子賓客王縉卒。



 三年春正月乙卯朔。丙寅,幽州節度使硃滔、張孝忠破李惟岳之兵於束鹿。辛未,詔供御及太子諸王常膳有司宜減省之,於是宰臣上言,減堂廚百官月俸,請三分
 省一以助軍,從之。庚辰,追封皇叔僖為宋王,贈皇弟選荊王。閏月丙申,以文宣王三十七代孫齊賢為兗州司功,襲文宣公。辛丑,復置具員簿。甲辰,成德軍兵馬使王武俊殺李惟岳,傳首京師。庚戌,馬燧、李芃破田悅兵於洹水,進攻魏州。



 二月戊午,惟岳將定州刺史楊政義以州降。加硃滔檢校司徒,以張孝忠檢校兵部尚書、易定滄三州節度使,以檢校太子賓客王武俊檢校秘書監、恆州刺史、恆冀都團練觀察使,康日知為趙州刺史、
 深趙都團練觀察使。三月丁亥,贈故衛尉卿顏杲卿司徒,故常山太守袁履謙左散騎常侍,故許州長史龐堅右散騎常侍,故鞏縣主簿蔣清禮部侍郎。贈故驍衛將軍、代國公安金藏兵部尚書,授其子承恩廬州長史。乙未,以徐州刺史李洧為徐、沂、海團練觀察使。戊戌,田悅、洺州刺史田昂以城降。以嶺南節度使張伯儀檢校兵部尚書,兼江陵尹、御史大夫、荊南節度等使;以容管經略使元琇為廣州刺史、嶺南節度使。丙午,貶京兆尹盧
 惎撫州長史。



 夏四月,李納守德州將李士真、守棣州將李長卿皆以城降。庚申,先陷蕃僧尼將士八百人自吐蕃而還。壬戌,封硃滔為通義郡王。硃滔、王武俊與田悅合從而為叛。太常博士韋都賓、陳京以軍興庸調不給,請借京城富商錢,大率每商留萬貫,餘並入官,不一二十大商,則國用濟矣。判度支杜佑曰:「今諸道用兵,月費度支錢一百餘萬貫,若獲五百萬貫,才可支給數月。」甲子,詔京兆尹、長安萬年令大索京畿富商,刑法嚴峻,長
 安令薛蘋荷校乘車,於坊市搜索,人不勝鞭笞,乃至自縊。京師囂然,如被盜賊。搜括既畢,計其所得才八十萬貫,少尹韋禛又取僦櫃質庫法拷索之,才及二百萬。丁丑,彭王傅徐浩卒,贈太子少師。戊寅,以中書侍郎、平章事張鎰兼鳳翔尹、隴右節度使,以代硃泚。加泚實封五百戶,賜竇氏名園、涇水上腴田及錦彩金銀器,以安其意,時滔叛故也。壬午,貶御史大夫嚴郢為費州長史,杖殺左巡使、殿中侍御史鄭詹。郢歲餘卒。



 五月丙戌,增兩
 稅、鹽榷錢,兩稅每貫增二百,鹽每斗增一百。丁亥,貶太子詹事邵說歸州刺史,卒於貶所。辛卯,詔朔方節度使李懷光率神策及朔方軍東討。丙申,詔:「故伊西北庭節度使楊休明、故河西節度使周鼎、故西州刺史李琇璋、故瓜州刺史張銑等,寄崇方鎮,時屬殷憂,固守西陲,以抗期戎虜。歿身異域,多歷歲年,以迨於茲,旅櫬方旋,誠深追悼,宜加寵贈,以賁幽泉。休明可贈司徒,鼎贈太保,琇璋贈戶部尚書,銑贈兵部侍郎。」皆隴右牧守,至德已來
 陷吐蕃而歿故,至是西蕃通和,方得歸葬也。丁酉,加河東節度使、檢校左僕射馬燧同平章事,澤潞李抱真檢校右僕射,河陽李芃檢校兵部尚書,神策營招討使李晟右散騎常侍,賞破田悅功也。乙巳,貶戶部侍郎、判度支杜佑為蘇州刺史,以中書舍人趙贊為戶部侍郎、判度支。辛亥,易定節度賜名義武軍。



 六月丁巳,尚書左丞庾準卒。甲子,京師地震。以左散騎常侍李涵為入回紇吊祭使,京兆少尹源休為光祿卿。戊寅,以前衢州刺史
 趙涓為尚書左丞,右庶子柳載為右丞。辛未,硃滔、王武俊兵救田悅,至魏州北。是日李懷光兵亦至,馬燧、抱真、李芃等盛軍容迓懷光。硃滔等慮其掩襲,遽出兵,懷光與之接戰於連篋山之西,王師不利,各還營壘。賊乃壅河決水,絕我糧道。秋七月甲申,以前振武軍使王翃為京兆尹,以兵部郎中楊真為御史中丞、京畿觀察使。以括率商戶,人情不安,癸巳,詔除已收納入庫外,一切停,已貯納者仍明置簿歷,各給文牒,後準元數卻還。甲午,
 以前同州刺史蕭復為兵部侍郎。庚子,馬燧、李懷光、李抱真、李芃等四節度兵退保魏橋。硃滔、王武俊、田悅之眾亦屯於魏橋東南,與官軍隔河對壘。自五月不雨,甲辰始雨。宣武節度李勉為檢校司徒,懷寧李希烈檢校司空,邠寧李懷光同平章事,李芃封開陽郡王。



 八月丁未,初分置汴東西水陸運兩稅鹽鐵事,從戶部侍郎、判度支趙贊奏也。戊午,太子賓客第五琦卒於位,辛酉,以涇原節度留後姚令言為涇原節度使。戊辰,以江淮鹽
 鐵使、太常少卿包佶為汴東水陸運兩稅鹽鐵使。己巳,加劍南西川節度使張延賞檢校吏部尚書。甲戌,以大理少卿崔縱為汴西水陸運兩稅鹽鐵使。丁丑,以禮儀使、太子少師顏真卿為太子太師。庚辰,徐、海、沂都團練使李洧卒。江淮訛言有毛人捕人,食其心,人情大恐。



 九月丁亥,以李洧部將高承宗為徐州刺史、徐海沂都團練使。判度支趙贊上言,請為兩都、江陵、成都、鎩汴、蘇、洪等州署常平輕重本錢。上至百萬貫,下至十萬貫,收貯斛
 斗匹段絲麻,候貴則下價出賣,賤則加估收糴,權輕重以利民。從之。贊乃於諸道津要置吏稅商貨,每貫稅二十文,竹木茶漆皆什一稅一,以充常平之本。己亥夜,有猛獸入宜陽里,傷二人,詰朝獲之。



 冬十月辛亥,以湖南觀察使嗣曹王皋為洪州刺史、江西節度使。丙辰,以吏部侍郎關播為中書侍郎、同平章事。都官員外郎樊澤使吐蕃回,與蕃相尚結贊約來年正月望日會盟清水。丙子,肅王詳薨。十一月己卯,以山南西道節度使賈耽檢校
 工部尚書、兼襄州刺史、御史大夫、山南東道節度使,以興鳳團練使嚴震為梁州刺史、山南西道節度使。甲午,以前山南東道節度使李承為潭州刺史、湖南觀察使。是月,硃滔、田悅、王武俊於魏縣軍壘各相推獎,僭稱王號。滔稱大冀王,武俊稱趙王,悅稱魏王。又勸李納稱齊王。僭署官名如國初親王行臺之制。丁丑,李希烈自稱天下都元帥、太尉、建興王,與硃滔等四盜膠固為逆。



 四年春正月戊寅朔。丁亥,鳳翔節度使張鎰與吐蕃宰
 相尚結贊同盟於清水。庚寅,李希烈陷汝州,執州將李元平而去,東都震駭。甲午,遣顏真卿宣慰李希烈軍。戊戌,以龍武大將軍哥舒曜為東都畿汝節度使,率鳳翔、邠寧、涇原等軍,東討希烈。丙午,福建觀察使常袞卒。二月戊申,於河陽三城置河陽軍節度。乙卯,哥舒曜收汝州。丁丑,以工部侍郎蔣鎮充禮儀使。



 三月己卯,復置沔州。癸未,以左散騎常侍孟皞為福建都團練觀察使。辛卯,嗣曹王皋擊李希烈將陳質之眾,敗之,收復黃州。丁
 酉,荊南張伯儀與賊戰,敗績。嗣曹王收復蘄州。



 夏四月庚申,以永平宣武河陽等軍節度都統、檢校司待、平章事李勉為淮西招討使,襄陽帥賈耽、江西嗣曹王等為之副。甲子,京師地震,生黃白毛,長尺餘。丙子,哥舒曜進軍至潁橋,大震雷,人死者十之三四,乃退保襄城。五月辛巳夜,京師地震。乙酉,潁王璬薨。乙巳,滑、濮二州黃河清。滑州馬生角。



 六月庚戌,初稅屋間架、除陌錢。時馬燧、李懷光、李抱真、李芃屯魏縣,李晟屯易定,李勉、陳少游、
 哥舒曜屯懷汝間,神策諸軍皆臨賊境。凡諸道之軍出境,仰給於度支,謂之食出界糧,月費錢一百三十萬貫,判度支趙贊巧法聚斂,終不能給。至是又稅屋,所由吏秉筆持算,入人廬舍而抄計,峻法繩之,愁嘆之聲,遍於天下。秋七月甲申,以國子祭酒李揆為禮部侍郎,復其爵。甲午,以揆為左僕射、兼御史大夫,為入吐蕃會盟使。八月丁未,李希烈率眾三萬攻哥舒曜於襄城。湖南觀察使李承卒。九月戊寅,龍見於汝州之城濠。丙戌,李勉
 將唐漢臣、劉德信喪師于扈澗,汴軍自此不振,東都危急。



 冬十月丙午,詔涇原節度使姚令言率涇原之師救哥舒曜。丁未,涇原軍出京城,至滻水,倒戈謀叛,姚令言不能禁。上令載繒彩二車,遣晉王往慰諭之,亂兵已陳於丹鳳闕下,促神策軍拒之。無一人至者。與太子諸王妃主百餘人出苑北門,右龍武軍使令狐建方教射於軍中,聞難,聚射士得四百人扈從。其夕至咸陽,飯數匕而過。戊申,至奉天。己酉,元帥都虞候渾瑊以子弟家屬
 至,乃以瑊為行在都虞候,神策軍使白志貞為行在都知兵馬使,以令狐建為中軍鼓角使,金吾將軍侯仲莊為奉天防城使。亂兵既剽京城,屯於白華,乃於晉昌里迎硃泚為帥,稱太尉,居含元殿。上以奉天隘,欲幸鳳翔,壬子,鳳翔軍亂,殺節度使張鎰,乃止。癸丑,李希烈陷襄城,哥舒曜走洛陽。乙卯,賜檢校司空崔寧薨。丁巳,以吏部尚書蕭復,刑部侍郎劉從一、諫議大夫姜公輔並以本官同中書門下平章事。邠寧節度韓游瑰與論惟明
 率兵三千至,才入奉天,賊軍亦至,乃出拒之,王師不利。賊乘勝攻門,自卯至午,殺傷殆半,會有草車在門外,渾瑊令焚之,賊眾遂退。癸巳,泚賊三面攻城,渾瑊力戰御之,方退。大將呂希倩死之。賊自丁未攻城,至己巳二十餘日,矢石不絕。



 十一月乙亥,以隴右節度判官、隴州留後、殿中侍御史韋皋為隴州刺史、兼御史大夫、奉義軍節度使。靈武留後杜希全、鹽州刺史戴休顏、夏州刺史時常春合兵六千來援,至漠谷,為賊所敗而退。賊由是攻
 城愈急,矢石雨下,薨傷者眾,人心危蹙,上與渾瑊對泣。硃泚據乾陵作樂,下瞰城中,詞多侮慢。戊子,賊造雲橋,攻東北隅,兵仗不能及,城中憂恐,相顧失色。渾瑊預為地道,及雲橋傅城,腳陷不得進,瑊命焚之,風回焰轉,橋焚而賊退。朔方節度李懷光遣兵馬使張韶奉表,言大軍將至,乃令舁韶巡城,叫呼歡聲動地,賊不之測,疑懼緩攻。癸巳,懷光軍次醴泉,是夜賊解圍而去。神策將李晟自定州率師赴難,軍於渭橋。甲午,以商州都虞候王
 仙鶴權商州防禦使。十二月壬戌,貶門下侍郎、平章事盧杞為新州司馬,貶行在都知兵馬使白志貞為恩州司馬,戶部侍郎、判度支趙贊為播州司馬。癸亥,以京兆少尹裴腆判度支。甲子,以湖南觀察留後趙憬為湖南觀察使。乙丑,以祠部員外郎陸贄為考功郎中,金部員外郎吳通微為職方郎中,翰林學士並如故。以侍御史吳通玄為起居舍人,充翰林學士。己巳,以河中尹李齊運為宗正卿。庚午,李希烈陷汴州。以右庶子崔縱為京
 兆尹。癸酉,以中書侍郎、平章事關播為刑部尚書,司封郎中杜黃裳為給事中。命給事中孔巢父淄青宣慰,華州刺史董晉河北宣慰。



 興元元年春正月癸酉朔,上在奉天行宮受朝賀。詔曰:



 立政興化,必在推誠;忘己濟人,不吝改過。朕嗣服丕構,君臨萬邦,失守宗祧,越在草莽。不念率德,誠莫追於既往;永言思咎,期有復於將來。明徵其義,以示天下。小子懼德不嗣,罔敢怠荒。然以長於深宮之中,暗於經國之
 務,積習易溺,居安忘危。不知稼穡之艱難,不恤征戍之勞苦。致澤靡下究,情不上通,事既壅隔,人懷疑阻。猶昧省己,遂用興戎,徵師四方,轉餉千里。賦車籍馬,遠近騷然;行齎居送,眾庶勞止。力役不息,田萊多荒。暴令峻於誅求,疲民空於杼軸,轉死溝壑,離去鄉里,邑里丘墟,人煙斷絕。天譴於上而朕不寤,人怨於下而朕不知。馴致亂階,變起都邑,賊臣乘釁,肆逆滔天,曾莫愧畏,敢行凌逼。萬品失序,九廟震驚,上累於祖宗,下負於蒸庶。痛心
 靦面,罪實在予,永言愧悼,若墜泉穀。賴天地降祐,人祗協謀,將相竭誠,爪牙宣力,群盜斯屏,皇維載張。將弘遠圖,必布新令。朕晨興夕惕,惟省前非。乃者公卿百僚用加虛美,以「聖神文武」之號,被蒙暗寡昧之躬,固辭不獲,俯遂群議。昨因內省,良所瞿然。自今已後,中外書奏不得言「聖神文武」之號。



 今上元統歷,獻歲發祥,宜革紀年之號,式敷在宥之澤,可大赦天下,改建中五年為興元元年。李希烈、田悅、王武俊、李納,咸以勛舊,繼守籓維,朕
 扶馭乖方,致其疑懼,皆由上失其道而下罹其災。一切並與洗滌,復其爵位,待之如初,仍即遣使宣諭。硃滔以泚連坐,路遠必不同謀,永念舊勛,務存弘貸,如能交辦順,亦與維新。硃反易天常,盜竊名器,暴犯陵寢,所不忍言,獲罪祖宗,朕不敢赦。除泚外,並從原宥。應赴奉天並進收京城將士,並賜名「奉天定難功臣」,身有過犯,減罪三等,子孫過犯,減罪二等。先稅除陌、間架等錢,竹木茶漆等稅,並停。奉天升為赤縣。



 分命朝臣諸道宣諭。以奉
 天行營都團練使楊惠元檢校工部尚書。丙戌,以吏部侍郎蕭復為門下侍郎、同平章事,以吏部侍郎盧翰為兵部侍郎、同平章事。戊子,命宰臣蕭復往山南、荊南、湖南、江西、鄂岳、浙江東西、福建等道宣慰。己丑,以京兆尹裴腆為戶部侍郎、判度支。丙申,以山南東道行軍司馬樊澤為襄州刺史、山南東道節度使;以渾瑊為行在都知兵馬使;以前趙州觀察使康日知兼同州刺史,充奉誠軍節度使。辛丑,詔六軍各置統軍一員,秩從二品;左
 右常侍各加一員;太子賓客加四員。



 二月戊寅,詔故司農卿張掖王段秀實贈太尉,謚曰忠烈,賜實封五百戶。贈滑州兵馬使賈隱林左僕射,以滑州刺史李澄兼汴州刺史、汴滑節度使。是日,李晟自咸陽移兵東渭橋,避懷光也。晟以懷光反狀已明,請上幸蜀。王武俊效順,加中書門下平章事,兼幽州節度使,令討硃滔。吐蕃遣使來朝,請以兵助國討逆,乃令御史大夫於頎入蕃宣諭之。甲子,加李懷光太尉,仍賜鐵券,赦三死罪。懷光怒曰:「
 凡人臣反逆,乃賜鐵券,今賜懷光,是反必矣!」乃投之於地。上命翰林學士陸贄曉諭之。是日人心恐駭。懷光奪楊惠元、李建徽所將兵,惠元被害。丁卯,車駕幸梁州,留戴休顏守奉天,以御史中丞齊映為沿路置頓使。李晟大集兵賦,以收復為己任。李懷光患之,移軍涇陽,連硃泚,欲同滅晟。晟卑詞厚意,致書諭之,冀其感悟,懷光頗增愧懼。



 三月甲申,以秘書監崔漢衡為上都留守,右散騎常侍於頎為京兆尹。是日,懷光燒營,走歸河中。其將
 孟涉、段威勇等千人奔歸李晟。丙戌,以前饒州刺史杜佑為廣州刺史、嶺南節度使,加神策節度使李晟兼京畿渭北鄜坊丹延節度觀察使。庚寅,車駕次城固。唐安公主薨,上愛女,悼惜之甚。壬申,至梁州。丁丑,宣武節度使劉洽加冊平章事。己亥,以行在都知兵馬使渾瑊檢校左僕射、同平章事、靈州大都督,充朔方節度使、邠寧振武永平奉天行營副元帥。是日,詔授李懷光太子太保,其餘官職並罷。涇州亂,牙將田希鑒殺其帥馮河清,
 自稱留後。四月辛丑朔。時將士未給春衣,上猶夾服,漢中早熱,左右請禦暑服,上曰:「將士未易冬服,獨御春衫可乎!」俄而貢物繼至,先給諸軍而始御之。壬寅,詔奉天隨從將士並賜號「元從功臣。」以邠寧兵馬使韓游瑰為邠寧節度伎。尚書左丞趙涓卒。己巳,以陜虢防遏使唐朝臣為河中尹、河中同晉絳節度使,御史大夫李齊運兼京兆尹。魏博行軍司馬田緒殺其帥田悅,詔贈悅太尉,以緒為魏州長史、魏博節度觀察使。甲寅,以諫議大
 夫、平章事姜公輔為左庶子,加劍南節度使張延賞平章事,以前山南東道節度使貢耽為工部尚書。甲子,入蕃使、左僕射李揆卒於鳳州。乙丑,渾瑊與吐蕃將論莽羅之眾破賊將韓旻之眾於武功,斬首萬級。丙寅,加李納平章事。丁卯,義王玭薨。



 五月,淮南節度使陳少游加檢校司徒,東川節度使李叔明太子太傅,鎮海軍韓滉檢校右僕射。癸酉,涇王侹薨。徐沂海團練使高承宗卒,以其子明應知徐州事。丙子,李抱真、王武俊破硃滔
 於經城東南,斬首三萬級,擒偽相硃良祐、李俊以獻。硃滔遁歸幽州。癸未,岳州李兼、黔南元全柔、桂管盧岳加御史大夫,岳加中丞。庚寅,李納上章稟命,乃贈李正己太尉。壬辰,商州尚可孤破賊於藍田。乙未,安西四鎮節度使郭昕、北庭都護李元忠加左右僕射。是夜,李晟自渭北移軍於光泰門外。賊來薄,我軍爭奪擊,大敗之,蹙入光泰門,斬馘數千計,賊黨慟哭而入白華。戊辰,列陳於光泰門外。遣騎將史萬頃往神麚村,開苑墻二百餘
 步,賊樹柵當之。我軍爭柵,雲合電擊,與賊血戰,賊黨大敗,追擊至白華,硃泚、姚令言率眾萬餘遁去。晟收復京城。是日,渾瑊與戴休顏亦破賊三千於咸陽,韓游瑰追硃泚於經州。



 六月庚子朔,升恆州為大都督府。癸卯,贈神策兵馬使楊惠元右僕射。是日,李晟上《收京城露布》,上覽之,涕下沾襟。涇州田希鑒斬姚令言,幽州軍士韓旻於彭原斬硃泚,並傳首至行在。乙巳,遣吏部侍郎班宏入京宣諭。己酉,加李晟司徒、兼中書令,實封一千戶;
 駱元光、尚可孤加檢校左右僕射,皆實封五百戶。以涇州將田希鑒為涇州刺史、涇原節度使。癸丑,詔以梁州為興元府,南鄭縣為赤畿,官名品制視京兆、河南,百姓給復二年,見任官員加兩階,耆老與版授,南鄭縣令賜緋。加興元尹嚴震檢校右僕射,賜實封一百戶。加渾瑊待中,實封八百戶;韓游瑰檢校左僕射,實封四百戶;戴休顏檢校右僕射,實封二百戶。考功郎中、知制誥陸贄,司封郎中、知刺誥吉中孚,並為諫議大夫;水部員外郎顧少
 連為禮部郎中:並依前充翰林學士。行在左右廂兵馬使令狐建、時常春並加散騎常侍。丙辰,斬偽相李忠臣,籍沒其家。李晟奏受賊偽署同惡抵法之家,所沒財物、牛馬、奴婢,請以賞軍士。從之。戊午,車駕還京,發興元,是日大雨,及入斜谷,晴霽,從官將士歡然以為天助。秋七月丙子,車駕次鳳翔府,詔放管內今年秋稅;耆壽侍老八十已上,各與版授刺史,賜紫,其餘版授上佐,賜緋;府、縣置頓官,考滿日放選。受偽署官喬琳、蔣鎮、張光晟、李
 通、蔣金監伏誅。硃泚害郡王、王子、王孫七十七人於馬璘宅,丁丑,令所司具兇禮收殮於凈域寺。庚辰,詔:



 李懷光往因職任,頗著乾能,朕嗣之初,首加拔擢,托為心膂,授以節旄。頃歲河朔不寧,俾令征討,任兼將相,恩極丘山。及硃泚猖狂,擾亂京邑,懷光回軍赴難,宗社再寧,保佑朕躬,厥功甚茂。故元帥、河中之權,太尉、中書之秩,仍加實封,爰及宗親,人臣之榮,孰可為比?非朕於懷光不厚,豈朕報懷光不崇!賊寇未除,猜嫌已構,受硃泚奸兇
 之說,聽張佋罔惑之言,曾不沈思,遂生疑阻,交通逆孽,殘害忠良。朕志在推誠,事皆掩覆,禮遇轉厚,委任益隆。懷光都不改圖,愈深不軌。敕書慰問將士,懷光並不令宣;三軍咸欲收城,懷光並不令出。自云已共硃泚定約,不能更事國家。朕以眇身,獲承鴻業,務全大計,移幸山南,倉皇之間,備歷危險。據其罪狀,情實難容,然以解圍奉天,其功不細,昨又遣男璀謝罪,請束身歸朝,朕憫其知過之心,念其赴難之效,以功贖罪,務在優恩。今遣給
 事中孔巢父齎先授懷光太子太保敕牒,往河中宣諭,三日內便與懷光同赴上都,如欲家口同行,亦聽懷光自便。朕必能保全終始,寵待如初。



 朔方將士,嘗立大功,子儀再收京城,咸是此軍之效,昨遠從河朔,赴難奉天,逆賊畏威,望風奔遁,永言勞績,朕不暫忘。將士各竭忠謀,中遭迫協,朕每念及,痛心自咎。此者君臣阻隔,只為懷光一人,懷光既請入朝,尚手舍其罪,況諸將士並是功臣,各宜坦然,勿更憂慮。先賜官封,一切如舊。



 壬午,至自
 興元。時渾瑊、韓游瑰、戴休顏以其眾扈從,李晟、駱元光、尚可孤以其眾奉迎,步騎十餘萬,旌旗連亙數十里,都民僧道,歡呼感泣。李晟見於三橋,自陳收城遲晚之咎,伏地請罪,上慰勞遣之。丁亥,河中宣慰使孔巢父、中官啖守盈並為懷光所害。辛卯,御丹鳳樓,大赦天下。賜李晟永崇里第,女樂八人。甲午,命宰臣諸將送晟入新賜第。教坊樂,京兆府供帳食饌,鼓吹導從,京城以為榮觀。



 八月辛丑,詔所司為贈太尉段秀實樹碑立廟。淄青節
 度使承前帶陸海運、押新羅渤海兩蕃等使,宜令李納兼之。癸卯,加司徒、中書令、合川郡王李晟兼鳳翔尹,充鳳翔隴右節度等使、涇原四鎮北庭行營兵馬副元帥,改封西平郡王。河東保寧軍節度使、太原尹、北都留守、檢校司徒、平章事、北平郡王馬燧為奉誠軍晉絳慈隰節度行營兵馬副元帥;以靈鹽節度使、侍中、兼靈州大都督、樓煩郡王渾瑊為河中尹、晉絳節度使、河中同陜虢等州及管內行營兵馬副元帥,改封咸寧郡王。時方
 命瑊與馬燧各出師討懷光故也。甲辰,以金吾大將軍杜希全為靈州大都督、西受降城天德軍靈鹽相夏節度營田等使;以同絳節度使唐朝臣為鄜坊丹延等州節度使;以保義軍節度使、鳳翔尹李楚琳為金吾大將軍;以奉義軍節度使、隴州刺史韋皋為左金吾衛大將軍。戊申,以奉天行營節度戴休顏為左龍武統軍。己酉,以延王玢、隨王迅、西平長公主薨,廢朝。己未,前湖州刺史袁高為給事中。



 九月庚午,宗正卿李琬卒。賜渾瑊大
 寧里第,並女樂五人,詔宰臣諸將賜樂饋贈如送李晟入第故事。壬午,贈故右僕射致仕李涵太子太保。乙亥,王武俊加檢校司徒,李抱真檢校司空,並賜實封五百戶,賞破硃滔之功也。甲申,以前嶺南節度使元琇為戶部侍郎、判度支。丁亥,上顧謂宰臣曰:「今大盜雖除,時猶多難,宜廣延納,以達下情。近日諫官都無論奏,自今每正衙及延英坐日常令朝臣三兩人奏時政得失,庶有弘益也。」是秋,螟蝗蔽野,草木無遺。



 冬十月乙丑,馬燧
 收絳州。戊辰,令中官竇文場、王希遷監左右神策軍都知兵馬使。閏月庚午,詔:「朕臨御萬方,失於君道,兵革不息,於今五年。閔眾庶之勞,悔征伐之事。而李希烈蔑義棄德,反道虐人。朕哀彼生靈,陷於塗炭。敬存拯物,不憚屈身,故於歲首特布新令,赦其殊死,待以至誠。使臣才及於郊圻,巨猾已聞其僭竊。酷烈滋甚,吞噬無厭。將相大臣,咸懷憤激,繼陳章疏,固請討除。朕以所行天誅,本去人害,兵戈既接,玉石難分。言念勛臣,橫遭脅制,雖思
 改革,厥路無由。受污終身,銜冤沒代,淪胥以逞,誠可痛傷。豈孽自一夫,而毒流萬姓,為人父母,寧不愧懷!宜令諸道節度使明行曉諭,罪止元兇,脅制之徒,一切不問。」唐朝臣奏收永樂縣。癸酉,以右龍武大將軍李觀為涇州刺史、涇原節度使。乙亥,詔宋亳、淄青、澤潞、河東、恆冀、幽、易定、魏博等八節度,螟蝗為害,蒸民饑饉,每節度賜米五萬石,河陽、東畿各賜三萬石,所司般運,於楚州分付。丁丑,李晟至涇州,誅節度使田希鑒,罪其殺馮河清
 也。戊子,希烈將李澄以滑州歸國。甲午,以李澄為汴州刺史、汴滑節度使,封武威郡王。神策行營節度使、檢校尚書右僕射、馮翊郡王尚可孤卒。



 十一月癸卯,宋亳節度使劉洽與曲環破希烈之眾於陳州,俘斬三萬級,生擒賊將翟崇暉以獻。戊午,劉洽大破希烈之眾,擒其偽相鄭賁等五人以獻。希烈遁歸蔡州,汴州平。乙丑,宰相蕭復三上章乞罷免,許之。十二月乙亥,淮南節度使、檢校司空、平章事陳少游卒。贈蕭定太子太師。以壽州刺
 史張建封為濠壽都團練使。庚辰,以刑部侍郎杜亞為揚州長史、淮南節度使,戊子,以吏部郎中崔造為給事中。辛卯,以諫議大夫陸贄為中書舍人,依前翰林學士。詔翰林學士朝服班序,宜同諸司官知制誥例。



 貞元元年正月丁酉朔,御含元殿受朝賀,禮畢,宣制大赦天下,改元貞元。戊戌,大風雪,寒。去秋螟蝗,冬旱,至是雪,寒甚,民饑凍死者踣於路。丁未,以饒州刺史盧為福州刺史、福建允察使。癸丑,始聞太子太師、魯郡公顏
 真卿為希烈所害,追贈司徒,廢朝五日,謚曰文忠,乃特授男頵、碩等官。壬戌,以吉州長史盧杞為澧州別駕,尋卒。二月丙寅朔,遣工部尚書賈耽、侍郎劉太真分往東都、兩河宣慰。河南、河北饑,米斗千錢。癸未,李抱真、嚴震來朝。寒食節,上與諸將擊鞠於內殿。丙戌,以檢校秘書監金良相為檢校太尉、使持節、大都督、雞林州刺史、寧海軍使,襲封新羅王。辛卯,大雨。



 三月丙申朔,以蜀州刺史韓洄為兵部侍郎,以汴東水陸運等使、左庶子包佶
 為刑部侍郎。辛丑,戶部侍郎、判度支元琇兼諸道水陸運使。丁未,李希烈陷南陽,殺守將黃金嶽。甲寅,詔宰臣宣諭御史,今後上封彈奏,人自陳論,不得群署章疏。戊午,宣武帥劉洽檢校司空;以汴滑節度使李澄普滑州刺史,充鄭滑節度使。加李納司空。



 夏四月乙丑朔,普王誼改封舒王。癸酉,鄂岳觀察使李謙為洪州刺史、西都團練觀察使。江陵度
 支院失火,燒租賦錢穀百餘萬。時關東大饑,賦調不入,由是國用益窘。關中饑民蒸蝗蟲而食之。汴帥劉洽賜名玄佐。



 五月癸卯,分命朝臣禱群神以祈雨。蝗自海而至,飛蔽天,每下則草木及畜毛無復孑遺。穀價騰踴。辛酉,以河陽都知兵馬使雍希顏為河陽懷都團練使。



 六月丙子,以兵部侍郎韓洄為京兆尹。辛巳,劉玄佐兼汴州史。壬午,以工部尚書賈耽兼御史大夫、東都留守、都畿汝州防御刺使,以汴州刺史薛玨為河南尹。辛卯,以
 左金吾衛大將軍韋皋檢校戶部尚書,兼成都尹、御史大夫、劍南西川節度觀察使。以國子祭酒董晉為左金吾衛大將軍。幽州硃滔卒,贈司徒。



 秋七月甲午朔,河東節度使馬燧自河中行營來朝。庚子,大風拔樹。辛丑,以左散騎常侍李泌為陜州長史、陜虢都防禦觀察陸運使。丙午,以鎮海軍、浙江東西道節度使韓滉檢校尚書左僕射、同平章事、江淮轉運使,以河南尹薛玨為河南水陸運使。戊申,馬燧還行營。辛亥,加檢校工部尚書王
 士真為德棣都團練觀察使。壬子,以前涿州刺史、兼御史中丞劉怦為幽州長史、御史大夫、幽州盧龍節度副大使,兼知節度管理度支營田觀察、押奚契丹經略盧龍等軍使。丁巳,以左散騎常侍柳渾為兵部侍郎。庚申,以諫議大夫高參為中書舍人。關中蝗食草木都盡,旱甚,灞水將竭,井多無水。有司計度支錢穀,才可支七旬。甲子,詔:「夫人事失於下,則天變形於上,咎徵之作,必有由然。自頃已來,災沴仍集,雨澤不降,綿歷三時,蟲蝗繼
 臻,彌亙千里。菽粟翔貴,稼穡枯瘁,嗷嗷蒸人,聚泣田畝,與言及此,實切痛傷。遍祈百神,曾不獲應,方悟禱祠非救災之術,言詞非謝譴之誠。憂心如焚,深自刻責。得非刑法舛繆。忠良鬱湮,暴賦未蠲,勞師靡息。事或無益,而重為煩費;任或非當,而橫肆侵蟊。有一於茲,足傷和氣。本其所以,罪實在予,萬姓何辜,重罹饑殍。所宜出次貶食,節用緩刑,側身增修,以謹天戒。朕自今視朝不御正殿,有司供膳並宜減省,不急之務,一切停罷。除諸軍將
 士外,應食糧人諸色用度,本司本使長官商量減罷,以救兇荒。俟歲豐登,即令復舊。」甲子,李懷光大將尉圭以焦籬堡降。丁卯,懷光將徐庭光以長春宮兵六千人降。甲戌,朔方大將牛名俊斬李懷光,傅首闕下。馬燧收復河中。丁丑,始雨。己卯,詔:「朕誠信未著,撫御失宜,致使功臣陷於誅戮,謂之克敵,能不愧心!然以懷光一家,在法無舍;念其昔居將相,嘗寄腹心。罪雖掛於刑書,功已藏於王府。以干紀之跡,固合滅身;以赴難之勛,所宜有後。
 宜以懷光男一人為嗣,賜莊宅各一區。仍還懷光尸首,任其收葬。懷光妻、諸兒女遞送澧州,委李皋逐便安置,使得存立。其出嫁女、諸親並釋放。陷賊將士,一切並與洗雪。河中、絳百姓,給復一年。北平王馬燧、咸寧王渾瑊並與一子五品正員官。燧可侍中,瑊可檢校司空。駱元光、韓游瑰、唐朝臣各賜實封二百戶,與一子六品正員官。昨河中行營將士,共賜二十萬端匹以充宴賞,放歸本道。」新除中書侍郎、平章事張延賞為尚書左僕射。時
 宰相劉從一病,詔徵延賞。李晟與延賞有隙,自鳳翔上表論之。延賞罷鎮西川還,行至興元,改授左僕射。戊子,前河陽節度使、檢校尚書左僕射、開陽郡王李芃卒。



 九月己亥,幽州節度劉怦病,請以子濟權知軍州事,從之。癸卯,以牛名俊為丹州刺史。御史大夫崔縱奏:「準制勘會內外官員,商量並省停減,詳議聞奏者。伏以兵戎未息,仕進頗多,在官者既合序遷,有功者又頒褒賞。比來每至選集,不免據闕留人,嘗嘆遺才,仍招怨望。況有恩
 詔,甄錄功勞,諸道敘優,人數甚廣,見須處置,不可稽留。今若停減吏員,實恐未便於事,非但承優者無官可授,抑又敘進者無路可容,本冀便人,翻成斂怨。事仍舊貫,以適時宜,更待事平,然後經度。」制從之。乙巳,上御正殿,策賢良方正、能直言極諫等三科舉人。辛亥,宰相劉從一以疾辭任,授戶部尚書。庚申劉從一卒。幽州節度使劉怦卒。辛巳,以權知幽州盧龍軍府事劉濟為幽州長史、兼御史大夫、幽州盧龍節度觀察、押奚契丹兩蕃等
 使。丙戌,渾瑊自河中來朝。



 十一月癸巳朔,山南嚴震來朝。癸卯,上親祀昊天上帝於圓丘。時河中渾瑊、澤潞李抱真、山南嚴震、同華駱元光、邠寧韓游瑰、鄜坊唐朝臣、奉誠康日知等大將侍祠。郊壇畢,還宮,御丹鳳樓,大赦天下。丁丑,詔文武常參官共賜錢七百萬貫,以歲兇穀貴,衣冠窘乏故也。



 十二月戊辰,詔延英視事日,令常參官七人引對,陳時政得失。自是群官互進,有不達理道者,因多詆訐,不適事宜,上亦優容遣之。



 二年春正月壬辰朔,以歲饑罷元會,禮也。丙申,詔以民饑,御膳之費減半,宮人月共糧米都一千五百石,飛龍馬減半料;臺郎御史與兼官出為畿赤令。庚子,大雪,平地尺餘。壬寅,以散騎常侍劉滋、給事中崔造、中書舍人齊映並守本官,同中書門下平章事。門下侍郎、平章事盧翰為太子賓客。丁未,以禮部侍郎鮑防為京兆尹,京兆尹韓洄為刑部侍郎,國子祭酒包佶知禮部貢舉。以江陵少尹李復為容州刺史、本管經略使。癸丑,御史
 大夫崔縱為吏部侍郎。諫議大夫、知制誥、翰林學士吉中孚為戶部侍郎、判度支兩稅,元琇判諸道鹽鐵、榷酒。詔宰相齊映判兵部,李勉判刑部,劉滋判吏部、禮部,崔造判戶部、工部。甲寅,詔天下兩稅錢物,委本道觀察使、刺史差人送上都;其先置諸道水陸轉運使及度支巡院、江淮轉運等使並停。時崔造專政,改易錢穀,職事多隳敗;造尋以憂病歸第。二月癸亥,山南樊澤奏破希烈將杜文朝之眾五千,擒文朝以獻。乙丑,鹿入含元殿,衛
 士執之。甲戌,戶部侍郎元琇為尚書左丞,京兆少尹李竦為戶部侍郎、判鹽鐵榷酒。



 三月壬寅,滑州李澄奏破希烈之眾於鄭州。乙巳,以司農卿李模為黔中觀察使。四月丙寅,淮西李希烈為其牙將陳仙奇所CG,並誅其妻子,仙奇以淮西歸順。戊辰,以前黔中觀察使元全柔為湖南觀察使。辛巳,陜州觀察使李泌奏盧氏山冶出瑟瑟,請禁以充貢奉。上曰:「瑟瑟不產中土,有則與民共之,任人採取。」甲申,詔以淮西牙將陳仙奇為蔡州刺史、
 淮西節度使,都統劉玄佐、李澄、曲環、李皋、賈耽、張建封各與一子正員官,賞平淮、蔡功也。丁未,以劍南東川節度使李叔明為太子太傅,以東川兵馬使王叔邕梓州刺史、劍南東川節度使。五月丙申,自癸巳大雨至於茲日,饑民俟夏麥將登,又此霖澍,人心甚恐,米復千錢。丁酉,以伊西北庭節度留後楊襲古為北庭大都護、伊西北庭節度度支營田瀚海等使。己亥,百僚請上復常膳;是時民久饑困,食新麥過多,死者甚眾。伊西北庭
 節度使李元忠卒,贈司空。辛酉,大風雨,街陌水深數尺,人有溺死者。癸未,橫海軍使、滄州刺史程日華卒,以其子懷直權知軍州事。



 秋七月戊子,黔中觀察使理所復在黔州。辛卯,以開州別駕白志貞為果州刺史。乙未,福建觀察盧惎卒。己酉,以虔王諒為申光、隨、蔡節度大使,以淮西兵馬使吳少誠為蔡州刺史、知節度留後,加東都留守賈耽東都畿唐、汝、鄧都防禦觀察使,以隴右行營節度使曲環為陳許節度使。戊午,以鄜坊節度唐朝
 臣為單于大都護、振武綏銀節度使,右金吾大將軍論惟明為鄜州刺史、鄜坊都防禦觀察使。己巳,以金吾大將軍董晉為尚書右丞。庚辰,右散騎常侍蔣沇卒。丙戌,吐蕃寇涇、隴、邠、寧,諸鎮守閉壁自固,京師戒嚴。遣河中節度駱元光鎮咸陽。



 九月,詔:「左右金吾及十六衛將軍,故事皆擇勛臣,出鎮方隅,入居侍從。自天寶艱難之後,衛兵雖然廢闕,將軍品秩尤高。此誠文武勛臣出入轉遷之地,宜增祿秩,以示優崇。並宜加給料錢及隨身糧
 課,仍舉故事,置武班朝參,其廊下食亦宜加給。其十六衛各置上將軍一人,秩從二品;左右金吾上將軍,俸料次於六統軍支給。欲求致理,必藉兼才,文武遞遷,不全限隔。自今內外文武缺官,於文武班中量才望相參敘用。仍依故事,於本衛量置衛兵。所司條件以聞。」丁酉,義成軍節度、鄭滑觀察等使、檢校尚書左僕射、滑州刺史、武威郡王李澄卒。以東都畿、唐、鄧、汝等防禦觀察使賈耽檢校尚書右僕射,兼滑州刺史、義成軍節度、鄭滑等
 州觀察使。戊戌,以吏部侍郎崔縱檢校禮部尚書、東都留守、東都畿唐鄧汝防禦觀察使。己亥,敕左右衛上將軍、大將軍並於衛內宿。乙巳,吐蕃寇好畤,京師戒嚴。李晟部將王佖擊吐蕃於汧陽城,敗其中軍。辛亥,寇鳳翔,李晟出師御之,一夕而退。



 冬十月壬午,奏關內、河中,河南等道秋夏兩稅、青苗等錢,悉折納粟麥,兼加估收糴以便民,從之。是月,李晟破吐蕃摧沙堡。十一月甲午,冊淑妃王氏為皇后。乙未,兩浙節度使韓滉來朝。丁酉,冊
 皇后王氏。是日後崩,謚曰昭德。辛丑,吐蕃陷鹽州。壬寅,劉玄佐、曲環、鄂岳盧玄卿並來朝。十二月丁巳,以韓滉兼度支、諸道鹽鐵轉運使。吐蕃陷夏州,又陷銀州。庚申,以給事中、同平章事崔造為右庶子。貶尚書右丞、度支元琇為雷州司戶,為韓滉誣奏,人以為非罪,諫官屢論之。辛未,鳳翔李晟來朝。壬申,京城畿內榷酒,每斗榷錢一百五十文,蠲酒戶差役,從度支奏也。



 三年春正月丙戌朔。壬寅,以左僕射張延賞同中書門
 下平章事。乙巳,禮部侍郎薛播卒。辛亥,以戶部侍郎李竦為鄂岳觀察使。壬子,以兵部侍郎柳渾同中書門下平章事;劉滋守本官,罷知政事;中書舍人、平章事齊映貶夔州刺史。戊寅,度支鹽鐵轉運使、鎮海軍節度、浙江東西道觀察等使、檢校左僕射、同中書門下平章事、晉國公韓滉卒,贈太傅。以果州刺史白志貞為潤州刺史、兼御史大夫、浙西觀察使,宣州刺史皇甫政為越州刺史、浙東觀察使。



 三月庚寅,詔今年朝集使宜停。丙午,鳳
 翔隴右元帥副兵馬使吳詵為福建觀察使,鳳翔都虞候邢君牙為鳳翔尹、本府團練使。丁未,制鳳翔隴右涇原四鎮北庭管內兵馬副元帥、鳳翔隴右道節度使、奉天靖難功臣、司徒兼中書令、鳳翔尹、上柱國、西平郡王、食實封一千五百戶李晟可太尉兼中書令。庚戌,以晟甥元帥兵馬使王佖為右威衛上將軍。辛亥,河東馬燧來朝。時蕃相尚結贊使大將論頰熱卑辭厚意告馬燧,請兩國同盟和好,上疑其不誠,不允,故燧自將論頰熱
 入朝,盛言蕃相請盟,可以保信。上乃從之,許盟於平涼。



 夏四月庚申,詔:「蕃寇雖退,疆理猶虞,安邊之策,必有良算,宜令常參官各陳邊事,隨所見封進以聞。」入蕃使崔翰奏於蕃中誘問給役者,求蕃國人馬真數,云凡五萬九千餘人,馬八萬六千匹,可戰者僅三萬人,餘悉老幼。庚午,御麟德殿,試《定難樂曲》,馬燧所獻。



 五月丁亥,以侍中渾瑊為吐蕃清水會盟使,兵部尚書崔漢衡副之;瑊與駱元光率師二萬往會盟所。丁酉,以左丞暢悅為湖
 南觀察使。戊戌,左右神策、左右龍武各加將軍一員。丙午,以嶺南節度使杜佑為尚書右丞,以容管經略使李復為廣州刺史、嶺南節度使。蕃相尚結贊請改會盟之所於原州之土梨樹,神策將馬有麟奏:「土梨地多險厄,恐蕃軍隱伏;不如平涼川,其地坦平,又近涇州。」乃改盟於平涼川。



 十月,東都、河南、江陵、汴州、揚州大水,漂民廬舍。閏月乙卯,以國子司業裴胄為潭州刺史、湖南觀察使。戊午,陜虢李泌獻瑞麥,一莖五穗。庚申,詔省州縣官
 員,上州留上佐、錄事、參軍、司戶、司士各一員,中州上佐、錄事、參軍、司戶、司兵各一員,下州上佐、錄事、司戶各一員,京兆河南兩府司錄、判司及四赤丞、簿、尉量留一半,諸赤畿縣留令、丞、尉各一員。時宰相張延賞請減官收俸料以助軍討吐蕃故也。壬戌,日有黑暈,自辰及申方散。癸亥,以荊南節度使、檢校戶部尚書、嗣曹王皋為襄州刺史、山南東道節度、襄鄧郢安隨唐等州觀察使,以山南東道節度使樊澤為江陵尹、荊南節度使。辛未,侍
 中渾瑊與吐蕃宰相尚結贊同盟於平涼,為蕃兵所劫,瑊狼狽遁而獲免,崔漢衡已下將吏陷沒者六十餘人。癸酉,遣使齎書以讓結贊,蕃界不受。戊寅,枉矢墜於虛危。辛巳,以少府監盧嶽為陜虢觀察使。是月,太白晝見,凡四十餘日。



 六月丙戌,以檢校司徒、侍中馬燧為司徒兼侍中,以贊吐蕃之盟失策而罷兵柄也。以陜虢觀察使李泌為中書侍郎、平章事,以左龍武將軍李自良為檢校工部尚書、太原尹、河東節度使。乙巳,浙西觀察使
 白志貞卒。是月,吐蕃驅鹽、夏二州居民,焚其州城而去。



 七月甲寅,渾瑊自盟所來,素服待罪,釋之。乙卯,詔:「朕頃緣興師備邊,資用不給,遂權議減官,以務集事。近聞授官者皆已隨牒之任,扶老攜幼,盡室而行。俸祿未請,歸還無所,衣冠之弊,流寓何依?其先敕所減官員,並宜仍舊。」初既減員,內外咨怨張延賞。李泌初入相,乃諷官論之,乃下此詔。丙辰,平涼陷蕃官員崔漢衡已下各與一子正員官。以左羽林大將軍韓潭為夏州刺史、夏綏
 銀等州節度使。壬申,賜駱元光姓曰李元諒。尚書左僕射、同中書門下平章事張延賞薨,贈太保。癸酉,復置吏部小選。



 八月辛巳朔,日有蝕之。丁亥,陷蕃兵部尚書崔漢衡得還。己丑,以兵部侍郎、平章事柳渾為散騎常侍,罷知政事。壬申,以給事中王緯為潤州刺史、浙西觀察使,常州刺史劉贊為宣州刺史、宣歙池觀察使。戊戌,貶前門下侍郎、平章事蕭復為太子左庶子,饒州安置,坐宗人位、佩、儒、偲、鼎等連郜國長公主奸蠱事也。戊辰,
 吐蕃犯寒,諸軍戒嚴。九月丁巳,吐蕃大掠汧陽、吳山、華亭界民庶,徙於安化峽西。庚申,左庶子崔造卒。癸亥,回紇可汗遣使合闕將軍請婚於我,許以咸安公主降之。丙寅,吐蕃陷華亭,又陷涇州之連雲堡。甲戌,吐蕃退,俘掠邠、涇、隴等州民戶殆盡。自是蕃寇常至涇、隴。



 冬十月,吐蕃修原州城,屯據之。丁亥,太子太傅李叔明卒。丙戌,神策將魏循上言:「射生將韓欽緒等十餘人與資敬寺妖僧李廣弘同謀不軌,廣弘自言當為人主,約十月十
 日大舉,已署置將相名目。」詔捕劾之,連坐死者百餘人;欽緒游瑰之子,特赦之。是月,復降魚書停刺史務。十一月丁丑,以湖南觀察使趙憬為給事中。是夜,京師地震者三,鳥巢散落。壬申,禁商人不得以口馬兵械市於黨項。辛丑,鄜坊節度使論惟明卒。是歲,作玄英觀於大明宮北垣。



\end{pinyinscope}