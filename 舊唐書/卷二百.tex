\article{卷二百}

\begin{pinyinscope}

 ○李華蕭穎士李翰附陸據
 崔顥王昌齡孟浩然元德秀王維李白杜甫
 吳通玄兄通微王仲舒崔咸唐次子扶持持子彥謙劉鸘李商隱溫庭筠薛逢子廷珪李拯李巨川司空圖



 李華字遐叔,趙郡人。開元二十三年進士擢第。天寶中,登朝為監察御史。累轉侍御史,禮部、吏部二員外郎。華善屬文,與蘭陵蕭穎士友善。華進士時,著《含元殿賦》萬餘言,穎士見而賞之,曰:「《景福》之上,《靈光》之下。」華文體溫
 麗,少宏傑之氣;穎士詞鋒俊發。華自以所業過之,疑其誣詞。乃為《祭古戰場文》,熏污之,如故物,置於佛書之閣。華與穎士因閱佛書得之。華謂之曰:「此文何如?」穎士曰:「可矣。」華曰:「當代秉筆者,誰及於此?」穎士曰:「君稍精思,便可及此。」華愕然。華著論言龜卜可廢,通人當其言。



 祿山陷京師,玄宗出幸,華扈從不及,陷賊,偽署為鳳閣舍人。收城後,三司類例減等,從輕貶官,遂廢於家,卒。華嘗為《魯山令元德秀墓碑》,顏真卿書,李陽冰篆額,後人爭模
 寫之,號為「四絕碑」。有文集十卷,行於時。



 蕭穎士者,字茂挺。與華同年登進士第。當開元中,天下承平,人物駢集,如賈曾、席豫、張垍、韋述輩,皆有盛名,而穎士皆與之游,由是縉紳多譽之。李林甫採其名,欲拔用之,乃召見。時穎士寓居廣陵,母喪,即縗麻而詣京師,徑謁林甫於政事省。林甫素不識,遽見縗麻,大惡之,即令斥去。穎士大忿,乃為《伐櫻桃賦》以刺林甫云:「擢無庸之瑣質,因本枝而自庇。洎枝幹而非據,專廟廷之右地。
 雖先寢而或薦,豈和羹之正味。」其狂率不遜,皆此類也。然而聰警絕倫。嘗與李華、陸據同游洛南龍門,三人共讀路側古碑,穎士一閱,即能誦之,華再閱,據三閱,方能記之。議者以三人才格高下亦如此。是時外夷亦知穎士之名,新羅使入朝,言國人願得蕭夫子為師,其名動華夷若此。終以誕傲褊忿,困躓而卒。



 華宗人翰,亦以進士知名。天寶中,寓居陽翟。為文精密,用思苦澀。常從陽翟令皇甫曾求音樂,每思涸則奏樂,神逸則著文。祿山
 之亂,從友人張巡客宋州。巡率州人守城,賊攻圍經年,食盡矢窮方陷。當時薄巡者,言其降賊,翰乃序巡守城事跡,撰《張巡姚摐等傳》兩卷上之。肅宗方明巡之忠義,士友稱之。上元中,為衛縣尉,入朝為侍御史。



 陸據者,周上庸公騰六代孫。少孤。文章俊逸,言論縱橫。年三十餘,始游京師,舉進士。公卿覽其文,稱重之,闢為從事。累官至司勛員外郎。天寶十三載卒。



 開元、天寶間,文士知名者,汴州崔顥、京兆王昌齡、高適、襄陽孟浩然,
 皆名位不振,唯高適官達,自有傳。



 崔顥者,登進士第,有俊才,無士行,好蒱博飲酒。及游京師,娶妻擇有貌者,稍不愜意,即去之,前後數四。累官司勛員外郎。天寶十三年卒。



 王昌齡者,進士登第,補秘書省校書郎。又以博學宏詞登科,再遷汜水縣尉。不護細行,屢見貶斥,卒。昌齡為文,緒微而思清。有集五卷。



 孟浩然,隱鹿門山,以詩自適。年四十,來游京師,應進士,
 不第,還襄陽。張九齡鎮荊州,署為從事,與之唱和。不達而卒。



 元德秀者,河南人,字紫芝。開元二十一年登進士第。性純樸,無緣飾,動師古道。父為延州刺史。



 德秀少孤貧,事母以孝聞。開元中,從鄉賦,歲游京師,不忍離親,每行則自負板輿,與母詣長安。登第後,母亡,廬於墓所,食無鹽酪,藉無茵席,刺血畫像寫佛經。久之,以孤幼牽於祿仕,調授邢州南和尉。佐治有惠政,黜陟使上聞,召補龍武
 錄事參軍。



 德季早失恃怙,縗麻相繼,不及親在而娶。既孤之後,遂不娶婚。族人以絕嗣規之,德秀曰:「吾兄有子,繼先人之祀。」以兄子婚娶,家貧無以為禮,求為魯山令。先是,墮車傷足,不任趨拜,汝郡守以客禮待之。部人為盜,吏捕之,系獄。會縣界有猛獸為暴,盜自陳曰:「願格殺猛獸以自贖。」德秀許之。胥吏曰:「盜詭計茍免,擅放官囚,無乃累乎?」德秀曰:「吾不欲負約,累則吾坐,必請不及諸君。」即破械出之。翌日,格猛獸而還。誠信化人,大率此類。



 秩滿,南游陸渾,見佳山水,杳然有長往之志,乃結廬山阿。歲屬饑歉,庖廚不爨,而彈琴讀書,怡然自得。好事者載酒餚過之,不擇賢不肖,與之對酌,陶陶然遺身物外。琴觴之餘,間以文詠,率情而書,語無雕刻。所著《季子聽樂論》、《蹇士賦》,為高人所稱。



 天寶十三年卒,時年五十九,門人相與謚為文行先生。士大夫高其行,不名,謂之元魯山。



 王維,字摩詰,太原祁人。父處廉,終汾州司馬,徙家於蒲,
 遂為河東人。



 維開元九年進士擢第。事母崔氏以孝聞。與弟縉俱有俊才,博學多藝亦齊名,閨門友悌,多士推之。歷右拾遺、監察御史、左補闕、庫部郎中。居母喪,柴毀骨立,殆不勝喪。服闋,拜吏部郎中。天寶末,為給事中。



 祿山陷兩都,玄宗出幸,維扈從不及,為賊所得。維服藥取痢,偽稱喑病。祿山素憐之,遣人迎置洛陽,拘於普施寺,迫以偽署。祿山宴其徒於凝碧宮,其樂工皆梨園弟子、教坊工人。維聞之悲惻,潛為詩曰:「萬戶傷心生野煙,百官
 何日再朝天?秋槐花落空宮裏,凝碧池頭奏管弦。」賊平,陷賊官三等定罪。維以《凝碧詩》聞於行在,肅宗嘉之。會縉請削己刑部侍郎以贖兄罪,特宥之,責授太子中允。乾元中,遷太子中庶子、中書舍人,復拜給事中,轉尚書右丞。



 維以詩名盛於開元、天寶間,昆仲宦游兩都,凡諸王駙馬豪右貴勢之門,無不拂席迎之,寧王、薛王待之如師友。維尤長五言詩。書畫特臻其妙,筆蹤措思,參於造化;而創意經圖,即有所缺,如山水平遠,雲峰石色,絕
 跡天機,非繪者之所及也。人有得《奏樂圖》,不知其名,維視之曰:「《霓裳》第三疊第一拍也。」好事者集樂工按之,一無差,咸服其精思。



 維弟兄俱奉佛,居常蔬食,不茹葷血;晚年長齋,不衣文彩。得宋之問藍田別墅,在輞口;輞水周於舍下,別漲竹洲花塢,與道友裴迪浮舟往來,彈琴賦詩,嘯詠終日。嘗聚其田園所為詩,號《輞川集》。在京師日飯十數名僧,以玄談為樂。齋中無所有,唯茶鐺、藥臼、經案、繩床而已。退朝之後,焚香獨坐,以禪誦為事。妻亡
 不再娶,三十年孤居一室,屏絕塵累。乾元二年七月卒。臨終之際,以縉在鳳翔,忽索筆作別縉書,又與平生親故作別書數幅,多敦厲朋友奉佛修心之旨,舍筆而絕。



 代宗時,縉為宰相。代宗好文,常謂縉曰:「卿之伯氏,天寶中詩名冠代,朕嘗於諸王座聞其樂章。今有多少文集,卿可進來。」縉曰:「臣兄開元中詩百千餘篇,天寶事後,十不存一。比於中外親故間相與編綴,都得四百餘篇。」翌日上之,帝優詔褒賞。縉自有傳。



 李白,字太白,山東人。少有逸才,志氣宏放,飄然有超世之心。父為任城尉,因家焉。少與魯中諸生孔巢父、韓沔、裴政、張叔明、陶沔等隱於徂徠山,酣歌縱酒,時號「竹溪六逸」。



 天寶初,客游會稽,與道士吳筠隱於剡中。既而玄宗詔筠赴京師,筠薦之於朝,遣使召之,與筠俱待詔翰林。白既嗜酒,日與飲徒醉於酒肆。玄宗度曲,欲造樂府新詞,亟召白,白已臥於酒肆矣。召入,以水灑面,即令秉筆,頃之成十餘章,帝頗嘉之。嘗沉醉殿上,引足令高力士脫靴,由是斥去。乃浪跡江湖,終日沉飲。時侍御史崔宗之謫官金
 陵,與白詩酒唱和。嘗月夜乘舟,自採石達金陵,白衣宮錦袍,於舟中顧瞻笑傲,傍若無人。



 初,賀知章見白,賞之曰:「此天上謫仙人也。」祿山之亂,玄宗幸蜀,在途以永王璘為江淮兵馬都督、揚州節度大使,白在宣州謁見,遂闢為從事。永王謀亂,兵敗,白坐長流夜郎。後遇赦得還,竟以飲酒過度,醉死於宣城。有文集二十卷,行於時。



 杜甫,字子美,本襄陽人,後徙河南鞏縣。曾祖依藝,位終鞏令。祖審言,位終膳部員外郎,自有傳。父閑,終奉天令。



 甫天寶初應進士不第。天寶末,獻《三大禮賦》。玄宗奇之,召試文章,授京兆府兵曹參軍。十五載,祿山陷京師,肅宗徵兵靈武。甫自京師宵遁赴河西,謁肅宗於彭原郡,拜右拾遺。房琯布衣時與甫善,時琯為宰相,請自帥師討賊,帝許之。其年十月,琯兵敗於陳濤斜。明年春,琯罷相。甫上疏言琯有才,不宜罷免。肅宗怒,貶琯為刺史,出甫為華州司功參軍。時關畿亂離,穀食踴貴,甫寓居成州同谷縣,自負薪採梠,兒女餓殍者數人。久之,召補京
 兆府功曹。



 上元二年冬,黃門侍郎、鄭國公嚴武鎮成都,奏為節度參謀、檢校尚書工部員外郎,賜緋魚袋。武與甫世舊,待遇甚隆。甫性褊躁,無器度,恃恩放恣。嘗憑醉登武之床,瞪視武曰:「嚴挺之乃有此兒!」武雖急暴,不以為忤。



 甫於成都浣花裏種竹植樹,結廬枕江,縱酒嘯詠,與田夫野老相狎蕩,無拘檢。嚴武過之,有時不冠,其傲誕如此。永泰元年夏,武卒,甫無所依。及郭英乂代武鎮成都,英乂武人粗暴,無能刺謁,乃游東蜀依高適。既至
 而適卒。是歲,崔寧殺英乂,楊子琳攻西川,蜀中大亂。甫以其家避亂荊、楚,扁舟下峽,未維舟而江陵亂,乃溯沿湘流,游衡山,寓居耒陽。甫嘗游嶽廟,為暴水所阻,旬日不得食。耒陽聶令知之,自棹舟迎甫而還。



 永泰二年,啖牛肉白酒,一夕而卒於耒陽,時年五十九。



 子宗武,流落湖、湘而卒。元和中,宗武子嗣業,自耒陽遷甫之柩,歸葬於偃師縣西北首陽山之前。



 天寶末詩人,甫與李白齊名,而白自負文格放達,譏甫齷齪,而有飯顆山之嘲誚。
 元和中,詞人元稹論李、杜之優劣曰:



 予讀詩至杜子美,而知小大之有所總萃焉。始堯、舜之時,君臣以賡歌相和。是後詩人繼作,歷夏、殷、周千餘年,仲尼緝拾選揀,取其干預教化之尤者三百,餘無所聞。騷人作而怨憤之態繁,然猶去《風》、《雅》日近,尚相比擬。秦、漢已還,採詩之官既廢,天下妖謠民謳、歌頌諷賦、曲度嬉戲之辭,亦隨時間作。至漢武賦《柏梁》而七言之體具。蘇子卿、李少卿之徒,尤工為五言。雖句讀文律各異,雅鄭之音亦雜,而辭
 意簡遠,指事言情,自非有為而為,則文不妄作。建安之後,天下之士遭罹兵戰,曹氏父子鞍馬間為文,往往橫槊賦詩,故其遒壯抑揚、冤哀悲離之作,尤極於古。晉世風概稍存。宋、齊之間,教失根本,士以簡慢翕習舒徐相尚,文章以風容色澤、放曠精清為高,蓋吟寫性靈、留連光景之文也。意義格力無取焉。陵遲至於梁、陳,淫艷刻飾、佻巧小碎之詞劇,又宋、齊之所不取也。



 唐興,官學大振,歷世能者之文互出。而又沈、宋之流,研練精切,穩順
 聲勢,謂之為律詩。由是之後,文體之變極焉。然而莫不好古者遺近,務華者去實,效齊、梁則不迨於魏、晉,工樂府則力屈於五言,律切則骨格不存,閑暇則纖穠莫備。



 至於子美,蓋所謂上薄《風》、《騷》,下該沈、宋,言奪蘇、李,氣吞曹、劉,掩顏、謝之孤高,雜徐、庾之流麗,盡得古今之體勢,而兼人人之所獨專矣!使仲尼考鍛其旨要,尚不知貴其多乎哉!茍以為能所不能,無可無不可,則詩人已來未有如子美者。



 是時山東人李白,亦以文奇取稱,時人
 謂之李、杜。予觀其壯浪縱恣,擺去拘束,模寫物象,及樂府歌詩,誠亦差肩於子美矣。至若鋪陳終始,排比聲韻,大或千言,次猶數百,詞氣豪邁,而風調清深,屬對律切,而脫棄凡近,則李尚不能歷其籓翰,況堂奧乎!



 予嘗欲條析其文,體別相附,與來者為之準,特病懶未就爾。自後屬文者,以稹論為是。甫有文集六十。



 吳通玄,海州人。父道瓘為道士,善教誘童孺。大歷中,召入宮,為太子諸王授經。德宗在東宮,師道瓘,而通玄兄
 弟,出入宮掖,恆侍太子游,故遇之厚。



 通玄與兄通微,俱博學善屬文,文彩綺麗。通玄幼應神童舉,釋褐秘書正字、左驍衛兵曹、大理評事。建中初,策賢良方正等科,通玄應文詞清麗,登乙第,授同州司戶、京兆戶曹。



 貞元初,召充翰林學士。遷起居舍人、知制誥,與陸贄、吉中孚、韋執誼等同視草。陸贄富詞藻,特承德宗重顧,經歷艱難。通玄弟兄又以東宮侍上,由是爭寵,頗相嫌恨。贄性褊急,屢於上前短通玄,又言:「承平時工藝書畫之徒,待詔
 翰林,比無學士。只自至德後,天子召集賢學士於禁中草書詔,因在翰林院待進止,遂以為名。奔播之時,道途或豫除改,權令草制。今四方無事,百揆時序,制書職分,宜歸中書舍人。學士之名,理須停寢。」贄以通玄援引朋黨,於禁中葉力排己,故欲廢之,德宗缺文計。會贄權知兵部侍郎,知貢舉,乃正拜之,罷內職,皆通玄譖之也。



 七年,自起居郎拜諫議大夫、知制誥。通玄自以久次當拜中書舍人,而反除諫議,殊失望。



 陸贄與宰相竇參
 相惡。參從子給事中申,參尤寵之。每預中書擬議,所至人呼申為「喜鵲」。申,嗣虢王則之從父甥也。申與則之親善。則之為金吾將軍,好學有文,申與則之潛結吳通玄兄弟,為參共傾陸贄。則之令人造謗書,言贄考試舉人不實,招納賄賂。時通玄取宗室女為外婦,德宗知之。既聞申、則之譖陸贄,綱紀伺之,果與通玄結構其謀。帝大怒,罷竇參知政事,尋貶郴州司馬,竇申錦州司戶,李則之昭州司馬,通玄泉州司馬。帝召見之,親自臨問,責以
 污辱近屬。行至華州長城驛,賜死。尋以陸贄為中書侍郎、平章事,代竇參。



 通微,建中四年自壽安縣令入為金部員外,召充翰林學士。尋改職方郎中,知制誥。與弟通玄同職禁署,人士榮之。七年,改禮部郎中,尋轉中書舍人。通玄死,素服待罪於國門,帝特宥之。通微竟不敢為喪服。



 通玄詞藻婉麗,帝尤憐之。貞元初,昭德王皇后崩,詔李紓為謚冊文,宰相張延賞、柳渾為廟樂章。及進,皆不稱旨,並召通玄重撰。凡中旨撰述,非通玄之筆,無不
 慊然,重之如此。



 王仲舒,字弘中,太原人。少孤貧,事母以孝聞。嗜學工文,不就鄉舉。凡與結交,必知名之士,與楊頊、梁聿、裴樞為忘形之契。貞元十年,策試賢良方正、能直言極諫等科,仲舒登乙第,超拜右拾遺。裴延齡領度支,矯誕大言,中傷良善,仲舒上疏極論之。累轉尚書郎。元和五年,自職方郎中知制誥。



 仲舒文思溫雅,制誥所出,人皆傳寫。京兆尹楊憑為中丞李夷簡所劾,貶臨賀尉。仲舒與憑善,
 宣言於朝,言夷簡掎摭憑罪,仲舒坐貶硤州刺史。遷蘇州。穆宗即位,復召為中書舍人。其年出為洪州刺史、御史中丞、江南西道觀察使。江西前例榷酒私釀法深,仲舒至鎮,奏罷之。又出官錢二萬貫,代貧戶輸稅。長慶三年冬,卒於鎮。



 崔咸,字重易,博陵人。祖安石。父銳,位終給事中。咸元和二年進士擢第,又登博學宏詞科。鄭餘慶、李夷簡闢為賓佐,待如師友。及登朝,歷踐臺閣,獨行守正,時望甚重。
 敬宗欲幸東都,人心不安。裴度以勛舊自興元隨表入覲。既至,李逢吉不欲度復入中書。京兆尹劉棲楚,逢吉黨也。棲楚等十餘人駕肩排度,而朝士持兩端者日擁度門。一日,度留客命酒,棲楚矯求度之歡,曲躬附裴耳而語,咸嫉其矯,舉爵罰度曰:「丞相不當許所由官呫囁耳語。」度笑而飲之。棲楚不自安,趨出。坐客皆壯之。累遷陜州大都督府長史、陜虢觀察等使。自旦至暮,與賓僚痛飲,恆醉不醒。簿領堆積,夜分省覽,剖判決斷,無毫厘
 之差,胥吏以為神人。入為右散騎常侍、秘書監。太和八年十月卒。



 初,銳佐李抱真為澤潞從事,有道人自稱盧老,曾事隋朝雲際寺李先生,預知過往未來之事。屬河朔禁游客,銳館之於家。一旦辭去,且曰:「我死,當與君為子。」因指口下黑子,願以為志。咸之生也,果有黑子,其形神即盧老也,父即以盧老字之。



 既冠,棲心高尚,志於林壑,往往獨游南山,經時方還。尤長於歌詩,或風景晴明,花朝月夕,朗吟意愜,必淒愴沾襟,旨趣高奇,名流嗟挹。
 有文集二十卷。



 唐次,並州晉陽人也,國初功臣禮部尚書儉之後。建中初進士擢第,累闢使府。貞元初,歷侍御史。竇參深重之,轉禮部員外郎。八年,參貶官,次坐出為開州刺史。在巴峽間十餘年,不獲進用。西川節度使韋皋抗表請為副使,德宗密諭皋令罷之。次久滯蠻荒,孤心抑鬱,怨謗所積,孰與申明,乃採自古忠臣賢士,遭罹讒謗放逐,遂至殺身,而君猶不悟,其書三篇,謂之《辯謗略》,上之。德宗省
 之,猶怒,謂左右曰:「唐次乃方吾為古之昏主,何自諭如此!」改夔州刺史。憲宗即位,與李吉甫同自峽內召還,授次禮部郎中。尋以本官知制誥,正拜中書舍人,卒。



 章武皇帝明哲嫉惡,尤惡人朋比傾陷。嘗閱書禁中,得次所上書三篇,覽而善之。謂學士沈傳師曰:「唐次所集辯謗之書,實君人者時宜觀覽。朕思古書中多有此事,次編錄未盡。卿家傳史學,可與學士類例廣之。」傳師奉詔與令狐楚、杜元穎等分功修續,廣為十卷,號《元和辯謗略》,
 其序曰:



 臣聞乾坤定而上下分矣。至於播四時之候,遂萬物之宜,在驗乎妖、祥之二氣;祥氣降則為豐為茂,妖氣降則為沴為災。君臣立而卑高隔矣。至於處神明之奧,詢獻納之辭,在審乎邪、正之二說;正言勝則為忠為讜,邪言勝則為讒為諛。故《詩》云:「萋兮斐兮,成是貝錦。」刺其組織之甚巧也。語曰:「邪徑敗良田,讒口亂善人。」惡其莠言之蠹政也。蓋謂似信而詐,似忠而非,便便可以動心,捷捷可以亂德,豈止鶗鴂彫卉,薏苡惑珠者哉!況立
 國家,自中徂外,道偏則刑罰不中,讒勝則忠孝靡彰。逖覽前聞,緬想近古,招賢容鯁,遠佞嫉邪,慮之則深,防之未至。



 伏惟睿聖文武皇帝陛下,垂衣御宇,化洽文明,謨猷博訪於縉紳,旌賁屢臻於巖穴。尚復廣四目,周四聰,制理皆在於未萌,作範將垂於不朽。乃詔掌文之臣令狐楚等,上自周、漢,下洎隋朝,求史籍之忠賢,罹讒謗之事跡,敘瑕釁之本末,紀謠詠之淺深,編次指明,勒成十卷。昔虞舜有墾讒之命,我皇修辯謗之書,千古一心,同
 垂至理。將俟法宮退日昃之政,別殿備乙夜之觀,則聖慮先辯,謗何由興!上天不言,而民自信矣。



 憲宗優詔答之。



 次子扶、持。



 扶,字雲翔,元和五年進士登第,累佐使府。入朝為監察御史,出為刺史。太和初,入朝為屯田郎中。十五年,充山南道宣撫使,至鄧州。奏:「內鄉縣行市、黃澗兩場倉督鄧琬等,先主掌湖南、江西運到糙米,至浙川縣於荒野中囤貯,除支用外,六千九百四十五碩,裛爛成灰塵。度支牒徵元掌所由,自貞元二十年,鄧琬父子
 兄弟至玄孫,相承禁系二十八年,前後禁死九人。今琬孫及玄孫見在枷禁者。」敕曰:「如聞鹽鐵、度支兩使,此類極多。其鄧琬等四人,資產全已賣納,禁系三代,瘐死獄中,實傷和氣。鄧琬等並疏放。天下州府監院如有此類,不得禁經三年已上。速便疏理以聞。」物議嘉扶有宣撫之才。俄轉司勛郎中。



 八年,充弘文館學士,判院事。九年,轉職方郎中,權知中書舍人事。開成初,正拜舍人,逾月,授福州刺史、御史中丞、福建團練觀察使。四年十一月,
 卒於鎮。



 扶佐幕立事,登朝有名,及廉問甌、閩,政事不治。身歿之後,僕妾爭財,詣闕論訴,法司按劾,其家財十萬貫,歸於二妾。又嘗枉殺部人,為其家所訴。行己前後不類,時論非之。



 持,字德守,元和十五年擢進士第,累闢諸侯府。入朝為侍御史、尚書郎。大中末,自工部郎中出為容州刺史、御史中丞、容管經略招討使。入為給事中。大中末,檢校左散騎常侍、靈州大都督府長史、朔方節度、靈武六城轉運等使。進位檢校戶部尚書、潞州大都督
 府長史、昭義節度、澤潞邢洺磁觀察處置等使,卒。



 子彥謙,字茂業。咸通末應進士,才高負氣,無所屈降,十餘年不第。乾符末,河南盜起,兩都覆沒,以其家避地漢南。中和中,王重榮鎮河中,闢為從事。累奏至河中節度副使,歷晉、絳二州刺史。



 彥謙博學多藝,文詞壯麗,至於書畫音樂博飲之技,無不出於輩流。尤能七言詩,少時師溫庭筠,故文格類之。



 光啟末,王重榮為部下所害,朝議責參佐。彥謙與書記李巨川俱貶漢中掾曹。時楊守亮鎮
 興元,素聞其名。彥謙以本府參承,守亮見之,喜握手曰:「聞尚書名久矣,邂逅於茲。」翌日,署為判官。累官至副使,閬、壁二郡刺史。卒於漢中。有詩數百篇,禮部侍郎薛廷珪為之序,號《鹿門先生集》,行於時。子渙,位亦至郡守。



 次弟款、欣。款貞元六年登進士第,累闢使府,登朝為御史,出為郡守,卒。子技。技字己有,會昌末,累遷刑部員外,轉郎中,累歷刺史,卒。



 劉濩,字去華,昌平人。父勉。濩寶歷二年進士擢第。博學
 善屬文,尤精《左氏春秋》。與朋友交,好談王霸大略,耿介嫉惡。言及世務,慨然有澄清之志。自元和末,閽寺權盛,握兵宮闈,橫制天下。天子廢立,由其可否,干撓庶政。當時目為南北司,愛惡相攻,有同水火。濩草澤中居常憤惋。文宗即位,恭儉求理,太和二年策試賢良曰:



 朕聞古先哲王之理也,玄默無為,端拱思道;陶民心以居簡,凝日用而不宰;厚下以立本,推誠而建中。由是天人通,陰陽和,俗躋仁壽,物無疵癘。噫,盛德之所臻,夐乎莫可及
 也。三代令王,質文迭究,百偽滋熾,風流浸微,自漢而降,足征蓋寡。朕顧惟昧道,祗荷丕構,奉若謨訓,不敢怠荒。任賢惕厲,宵衣旰食,詎追三五之遐軌,庶紹祖宗之鴻緒。而心有所未達,行有所未孚,由中及外,闕政斯廣。是以人不率化,氣或堙厄,災旱竟歲,播植愆時。國廩罕蓄,乏九年之儲;吏道多端,微三載之績。京師,諸夏之本也,將以觀理,而豪猾時逾檢;太學,明教之源也,期於變風,而生徒多惰業。列郡在乎頒條,而干禁或未絕;百工在
 乎按度,而淫巧或未衰。俗墮風靡,積訛成蠹。其擇官濟理也,聽人以言,則枝葉難辨;御下以法,則恥格不形。其阜財發號也,生之寡而食之眾,煩於令而鮮於理。思所以究此繆盩,致之治平,茲心浩然,若涉泉水。故前詔有司,博延群彥,佇啟宿懵,冀臻時雍。子大夫識達古今,明於康濟,造廷待問,副朕虛懷。必當箴主之闕,辯政之疵,明綱條之致紊,稽富庶之所急。何施斯革於前弊!何澤斯惠乎下土!何脩而理古可近!何道而和氣克充!推之
 本源,著於條對。至於夷吾輕重之權,孰輔於理?嚴尤底定之策,孰葉於時?元凱之考課何先?叔子之克平何務?推此龜鏡,擇乎中庸,期在洽聞,朕將親覽。



 時對策者百餘人,所對止循常務,唯濩切論黃門太橫,將危宗社。對曰:



 臣誠不佞,有匡國致君之術,無位而不得行;有犯顏敢諫之心,無路而不得進。但懷憤鬱抑,思有時而一發耳。常欲與庶人議於道,商旅謗於市,得通上聽,一悟主心,雖被妖言之罪,無所悔焉!況逢陛下以至德嗣興,以
 大明垂照,詢求過闕,咨訪謨猷,制詔中外,舉直言極諫者。臣既辱斯舉,專承大問,敢不悉意以言!至於上之所忌,時之所禁,權幸之所諱惡,有司之所與奪,臣愚不識。伏惟陛下少加優容,不使聖朝有讜直而受戮者,乃天下之幸也!謹昧死以對。



 伏惟聖策,有思先古之理,念玄默之化。將欲通天人以齊俗,和陰陽以煦物,見陛下慕道之深也。臣以為哲王之理,其則不遠,惟陛下致之之道何如爾!



 伏惟聖策,有祗荷丕構而不敢荒寧,奉若謨
 訓而罔有怠忽,見陛下憂勞之志也。若夫任賢惕厲,宵衣旰食,宜黜左右之纖佞,進股肱之大臣;若夫追蹤三五,紹復祖宗,宜鑒前古之興亡,明當時之成敗。心有所未達,以下情塞而不得上通;行有所未孚,以上澤壅而不得下浹。欲人之化也,在脩己以先之;欲氣之和也,在遂性以導之。救災患在致乎精誠,廣播植在視乎食力。國廩罕蓄,本乎冗食尚繁;吏道多端,本乎選用失當。豪猾逾制,由中外之法殊;生徒惰業,由學校之官廢。列郡
 干禁,由授任非人;百工淫巧,由制度不立。



 伏以聖策,有擇官濟理之心,阜財發號之嘆,見陛下教化之本也。且進人以行,則枝葉安有難別乎?防下以禮,則恥格安有不形乎?念生寡而食眾,可罷斥惰游;念令煩而理鮮,要察其行否。博延群彥,願陛下必納其言;造廷待問,則小臣安敢愛死!



 伏以聖策,有求賢箴闕之言,審政辯疵之念,見陛下咨訪之勤也。遂小臣屏奸豪之志,則弊革於前;守陛下念康濟之心,則惠敷於下。邪正之道分,則理
 古可近;禮樂之方著,而和氣克充。至若夷吾之法,非皇王之權;嚴尤所陳,無最上之策。元凱之所先,不若唐、虞之考績;叔子之所務,不若重華之舞干。且俱非大德之中庸,未為上聖之龜鑒,何足以為陛下道之哉!或有以系安危之機,兆存亡之變者,臣請披瀝肝膽,為陛下別白而重言之。



 臣前所稱「哲王之理,其則不遠」者,在陛下慎思之,力行之,終始不懈而已。臣謹按《春秋》:「元者,氣之始也;春者,歲之始也。」《春秋》以元加於歲,以春加於王,明
 王者當奉若天道,以謹其始也。又舉時以終歲,舉月以終時,《春秋》雖無事,必書首月以存時,明王者當奉若天道,以謹其終也。



 王者動作終始必法於天者,以其運行不息也。陛下既能謹其始,又能謹其終,懋而修之,勤而行之,則可以執契而居簡,無為而不宰,廣立本之大業,崇建中之盛德矣!又安有三代循環之弊,而為百偽滋熾之漸乎?臣故曰:「惟陛下致之之道何如耳!」



 臣前所謂「若夫任賢惕厲,宵衣旰食,宜罷黜左右之纖佞,進股肱
 之大臣」者,實以陛下憂勞之至也。臣聞不宜憂而憂者,國必衰;宜憂而不憂者,國必危。今陛下不以國家存亡之事、社稷安危之策,而降於清問。臣未知陛下以布衣之臣不足以定大計耶?或萬機之勤,而聖慮有所未至耶?不然,何宜憂而不憂者乎?臣以為陛下宜先憂者,宮闈將變,社稷將危,天下將傾,海內將亂。此四者,國家已然之兆,故臣謂聖慮宜先及之。



 夫帝業既艱難而成之,故不可容易而守之。昔太祖肇其基,高祖勤其績,太宗
 定其業,玄宗繼其明,至於陛下,二百有餘載矣。其間明聖相因,憂亂繼作,未有不委用賢士,親近正人,而能紹興其徽烈者也!或一日不念,則顛覆大器,宗廟之恥,萬古為恨!



 臣謹按《春秋》,人君之道,在體元以居正,昔董仲舒為漢武帝言之略矣。其所未盡者,臣得為陛下備而論之。夫繼故必書即位,所以正其始也;終必書所終之地,所以正其終也。故為君者,所發必正言,所履必正道,所居必正位,所近必正人。



 臣又按《春秋》「閽弒吳子餘祭」,
 不書其君。《春秋》譏其疏遠賢士,暱近刑人,有不君之道矣。伏惟陛下思祖宗開國之勤,念《春秋》繼故之誡。將明法度之端,則發正言而履正道;將杜篡弒之漸,則居正位而近正人。遠刀鋸之賤,親骨鯁之直,輔相得以專其任,庶職得以守其官。奈何以褻近五六人,總天下大政,外專陛下之命,內竊陛下之權,威懾朝廷,勢傾海內,群臣莫敢指其狀,天子不得制其心!禍稔蕭墻,奸生帷幄,臣恐曹節、侯覽,復生於今日,此宮闈之所以將變也!



 臣
 謹按《春秋》,魯定公元年春王不言正月者。《春秋》以其先君不得正其終,則後君不得正其始,故曰定無正也。今忠賢無腹心之寄,閽寺持廢立之權,陷先君不得正其終,致陛下不得正其始。況皇儲未建,郊祀未脩,將相之職不歸,名分之宜不定,此社稷之所以將危也!



 臣謹按《春秋》「王札子殺召伯、毛伯」。《春秋》之義,兩下相殺不書。而此書者,重其專王命也。且天之所授者在君,君之所授者在命。操其命而失之者,是不君也;侵其命而專之者,
 是不臣也。君不君,臣不臣,此天下所以將傾也!



 臣謹按《春秋》,晉趙鞅以晉陽之兵叛入於晉。書其歸者,以其能逐君側惡人以安其君,故《春秋》善之。今威柄凌夷,籓臣跋扈。或有不達人臣之節,首亂者以安君為名;不究《春秋》之微,稱兵者以逐惡為義。則政刑不由乎天子,攻伐必自於諸侯,此海內之所以將亂也!



 又樊噲排闥而雪涕,爰盎當車以抗詞,京房發憤以殞身,竇武不顧而畢命,此皆陛下明知之矣。



 臣謹按《春秋》,晉狐射姑殺陽處
 父。書襄公殺之者,以其上漏言也。襄公不能固陰重之機,處父所以及戕賊之禍,故《春秋》非之。夫上漏其情,則下不敢盡意;上洩其事,則下不敢盡言。《傳》有「造膝」、「詭辭」之文,《易》有「殺身」、「害成」戒。今公卿大臣,非不能為陛下言之,慮陛下必不能用之。陛下既忽之而不用,必洩其言;臣下既言之而不行,必嬰其禍。適足以鉗直臣之口,重奸臣之威。是以欲盡其言,則起失身之懼;欲盡其意,則有害成之憂。故徘徊鬱塞,以俟陛下感悟,然後盡其
 啟沃耳。陛下何不以聽朝之餘,時御便殿,召當時賢相與舊德老臣,訪持變扶危之謀,求定傾救亂之術!塞陰邪之路,屏褻狎之臣;制侵凌迫脅之心,復門戶掃除之役;戒其所宜戒,憂其所宜憂。既不能治於前,當治於後;既不能正其始,當正其終。則可以虔奉典謨,克承丕構,終任賢之效,無旰食之憂矣!



 臣前所謂「若夫追蹤三五,紹復祖宗,宜鑒前古之興亡,明當時之成敗」者。臣聞堯、舜之為君而天下之人理者,以其能任九官四岳十二
 牧,不失其舉,不貳其業,不侵其職。居官惟其能,左右惟其賢。元凱在下,雖微必舉;四兇在朝,雖強必誅。考其安危,明其取舍。至秦之二代,漢之元、成,咸欲措國如唐、虞,致身如堯、舜,而終敗亡者,以其不見安危之機,不知取舍之道,不任大臣,不辯奸人,不親忠良,不遠讒佞。伏惟陛下察唐、虞之所以興,而景行於前;鑒秦、漢之所以亡,而戒懼於後。



 陛下無謂廟堂無賢相,庶官無賢士。今紀綱未絕,典刑猶在,人誰不欲致身為王臣,致時為太平,
 陛下何忽而不用之耶?又有居官非其能,左右非其賢,其惡如四兇,其詐如趙高,其奸如恭、顯,陛下又何憚而不去之耶?神器固有歸,天命固有分,祖廟固有靈,忠臣固有心,陛下其念之哉!昔秦之亡也,失於強暴;漢之亡也,失於微弱。強暴則賊臣畏死而害上,微弱則奸臣竊權而震主。伏見敬宗皇帝不虞亡秦之禍,不翦其萌。伏惟陛下深軫亡漢之憂,以杜其漸。則祖宗之鴻業可紹,三五之遐軌可追矣!



 臣前所謂「陛下心有所未達,以下
 情塞而不能上通;行有所未孚,以上澤壅而不得下浹」者。且百姓塗炭之苦,陛下無由而知;則陛下有子育之心,百姓無由而信。臣謹按《春秋》書「梁亡」,不書取者,梁自亡也。以其思慮昏而耳目塞,上出惡政,人為寇盜,皆不知其所以然,以自取其滅亡也。臣聞國君之所以尊者,重其社稷也;社稷之所以重者,存其百姓也。茍百姓之不存,則社稷不得固其重;茍社稷之不重,則國君不得保其尊。故治天下不可不知百姓之情。夫百姓者,陛下
 之赤子也。陛下宜令仁慈者親育之,如保傅焉,如乳哺焉,如師之教導焉。故人信於上也,敬之如神明,愛之如父母。今或不然。陛下親近貴幸,分曹補署,建除卒吏,召致賓客,因其貨賄,假其氣勢。大者統籓方,小者為牧守。居上無清惠之致,而有饕餮之害;居下無忠誠之節,而有奸欺之罪。故人之於上也,畏之如豺狼,惡之如仇敵。今海內困窮,處處流散,饑者不得食,寒者不得衣,鰥寡孤獨者不得存,老幼疾病者不得養。加以國之權柄,專
 在左右,貪臣聚斂以固寵,奸吏因緣而弄法。冤痛之聲,上達於九天,下流於九泉;鬼神怨怒,陰陽為之愆錯。君門萬里而不得告訴,士人無所歸化,百姓無所歸命。官亂人貧,盜賊並起,土崩之勢,憂在旦夕。即不幸因之以疾癘,繼之以兇荒,臣恐陳勝、吳廣不獨起於秦,赤眉、黃巾不獨起於漢。故臣所以為陛下發憤扼腕,痛心泣血爾。如此則百姓有塗炭之苦,陛下何由而知之;陛下有子育之心,百姓安得而信之乎?致使陛下「行有所未孚,
 心有所未達」者,固其然也!



 臣聞昔漢元帝即位之初,更制七十餘事,其心甚誠,其稱甚美。然而紀綱日紊,國祚日衰,奸宄日強,黎元日困者,以其不能擇賢明而任之,失其操柄也。自陛下御宇,憂勤兆庶,屢降德音,四海之內,莫不抗首而長息,自喜復生於死亡之中也。伏惟陛下慎終如始,以塞萬方之望。誠能揭國權以歸其相,持兵柄以歸其將,去貪臣聚斂之政,除奸吏因緣之害,惟忠賢是近,惟正直是用,內寵便僻,無所聽焉!選清慎之
 官,擇仁惠之長,敏之以利,煦之以仁,教之以孝慈,導之以德義,去耳目之塞,通上下之情,俾萬國歡康,兆民蘇息,則心無不達,行無不孚矣!



 臣前所謂「欲兆人之化也,在修己以先之」者。臣聞德以修己,教以導人。修之也,則人不勸而自至;導之也,則人敦行而率從。是以君子欲政之必行也,故以身先之;欲人之從化也,故以道御之。今陛下先之以身而政未必行,御之以道而人未從化,豈不以立教之旨未盡其方也?夫立教之方,在乎君以
 明制之,臣以忠行之。君以知人為明,臣以匡時為忠;知人則任賢而去邪,匡時則固本而守法。賢不任則重賞不足以勸善,邪不去則嚴刑不足以禁非。本不固則民流,法不守則政散。而欲教之使必至,化之使必行,不可得也!陛下能斥奸邪不私其左右,舉賢正不遺其疏遠,則化浹於朝廷矣。愛人以敦本,分職而奉法,修其身以及其人,始於中而成於外,則化行於天下矣!



 臣前所謂「欲氣之和也,在於遂性以導之」者,當納人於仁壽也。夫
 欲人之仁壽也,在乎立制度,修教化。夫制度立則財用省,財用省則賦斂輕,賦斂輕則人富矣。教化修則爭競息,爭競息則刑罰清,刑罰清則人安矣!既富矣,則仁義興焉;既安矣,則壽考至焉。仁壽之心感於下,和平之氣應於上,故災害不作,休祥薦臻,四方底寧,萬物咸遂矣!



 臣前所謂「救災旱在致乎精誠」者。臣謹按《春秋》,魯僖公七月之中,三書不雨者,以其君有恤人之志也;魯文公三年之中,一書不雨者,以其君無憫人之心也。故僖公
 致精誠而旱不害物,文公無恤憫而旱則成災。陛下誠能有恤人之心,則無成災之變矣!



 臣前所謂「廣播植在視乎食力」者。臣謹按《春秋》:「君人者,必時視人之所勤。人勤於力,則功築罕;人勤於財,則貢賦少;人勤於食,則百事廢。」今食與財力皆勤矣,願陛下廢百事之勞,廣三時之務,則播植不愆矣!



 臣前所謂「國廩罕蓄,本乎冗食尚繁」者。臣謹按《春秋》「臧孫辰告糴於齊」,《春秋》譏其國無九年之蓄,一年不登而百姓饑。臣願斥游惰之人以篤其
 耕植,省不急之費以贍其黎元,則廩蓄不乏矣!



 臣前所謂「吏道多端,本乎選用失當」者,由國家取人不盡其才,任人不明其要故也。今陛下之用人也,求其聲而不得其實;故人之趨進也,務其末而不務其本。臣願核考課之實,定遷序之制,則多端之吏息矣!



 臣前所謂「豪猾逾檢,由中外之法殊」者,以其官禁不一也。臣謹按《春秋》,齊桓公盟諸侯不以日,而葵丘之盟特以日者,美其能宣明天子之禁,率奉王官之法,故《春秋》備而書之。夫官者,
 五帝、三王之所建也;法者,高祖、太宗之所制也。法宜畫一,官宜正名。今又分外官、中官之員,立南司、北司之局,或犯禁於南,則亡命於北,或正刑於外,則破律於中,法出多門,人無所措,實由兵農勢異,而中外法殊也。臣聞古者因井田而制軍賦,間農事以修武備,提封約卒乘之數,命將在公卿之列,故兵農一致而文武同方,可以保乂邦家,式遏禍亂。暨太宗皇帝肇建邦典,亦置府兵,臺省軍衛,文武參掌;居閑歲則櫜弓力穡,將有事則釋
 耒荷戈,所以修復古制,不廢舊物。今則不然。夏官不知兵籍,止於奉朝請;六軍不主兵事,止於養勛階。軍容合中宮之政,戎律附內臣之職。首一戴武弁,嫉文吏如仇讎;足一蹈軍門,視農夫如草芥。謀不足以翦除兇逆,而詐足以抑揚威福;勇不足以鎮衛社稷,而暴足以侵軼裏閭。羈絏籓臣,乾凌宰輔,隳裂王度,汨亂朝經。張武夫之威,上以制君父;假天子之命,下以御英豪。有藏奸觀釁之心,無伏節死難之義。豈先王經文緯武之旨耶!臣
 願陛下貫文武之道,均兵農之功;正貴賤之名,一中外之法。選軍衛之職,修省署之官,近崇貞觀之規,遠復成周之制。自邦畿以刑於下國,始天子以達於諸侯,則可以制豪猾之強,無逾檢之患矣!



 臣前所謂「生徒墮業,由學校之官廢」者,蓋以國家貴其祿而賤其能,先其身而後其行。故庶官乏通經之學,諸生無修業之心矣。



 臣前所謂「列郡干禁,由授任非其人」者。臣以為刺史之任,理亂之根本系焉,朝廷之法制在焉。權可以抑豪猾,恩可
 以惠孤寡,強可以御奸寇,政可以移風俗。其將校有曾經戰陣,及功臣子弟,各請隨宜酬賞。如無治人之術者,不當授任此官,則絕干禁之患矣。



 臣前所謂「百工淫巧,由制度不立」者。臣請以官位祿秩,制其器用車服,禁人金銀珠玉錦繡雕鏤不蓄於私室,則無蕩心之巧矣。



 臣前所謂「辯枝葉」者,考其言以詢行也。



 臣前所謂「形於恥格」者,道德而齊禮也。



 臣前所謂「念生寡而食眾,可罷斥惰游」者,已備之於前矣。



 臣前所謂「令煩而理鮮,要察其行
 否」者,臣聞號令者,乃理國之具也,君審而出之,臣奉而行之,或虧上旨,罪在不赦。今陛下令煩而理鮮,得非持之者有所蔽欺乎?



 臣前所謂「博延群彥,願陛下必納其言;造廷待問,則小臣不敢愛死」者。臣聞晁錯為漢畫削諸侯之策,非不知禍之將至也。忠臣之心,壯夫之節,茍利社稷,死無悔焉!今臣非不知言發而禍應,計行而身戮,蓋所以痛社稷之危,哀生人之困,豈忍姑息時忌,竊陛下一命之寵哉!昔龍逢死而啟殷,比干死而啟周,韓
 非死而啟漢,陳蕃死而啟魏。今臣之來也,有司或不敢薦臣之言,陛下又無以察臣之心,退必受戮於權臣之手。臣幸得從四子於地下,固臣之願也。所不知殺臣者,臣死之後,將孰為啟之哉?至於人主之闕,政教之疵,前日之弊,臣既言之矣。若乃流下土之惠,條近古之理,而致其和平者,在陛下行之而已。然上之所陳者,實以臣親奉聖問,敢不條對!雖臣之愚,以為未極教化之大端,皇王之要道。伏惟陛下事天地以教人敬,奉宗廟以教
 人孝,養高年以教人悌長,字百姓以教人慈幼,調元氣以煦育,扇大和於仁壽,可以逍遙無為,垂拱成化。至若念陶鈞之道,在擇宰相而任之,使權造物之柄。念保定之功,在擇將帥而任之,使修分閫之寄。念百度之未貞,在擇庶官而任之,使專職業之守。念百姓之愁痛,在擇長吏而任之,使明惠育之術。自然言足以為天下教,行足以為天下法,仁足以勸善,義足以禁非,又何必宵衣旰食,勞神惕慮,然後以致其理哉!



 是歲,左散騎常侍馮
 宿、太常少卿賈餗、庫部郎中龐嚴為考策官,三人者,時之文士也,睹濩條對,嘆服嗟悒,以為漢之晁、董,無以過之。言論激切,士林感動。時登科者二十二人,而中官當途,考官不敢留濩在籍中,物論喧然不平之。守道正人,傳讀其文,至有相對垂泣者。諫官御史,扼腕憤發,而執政之臣,從而弭之,以避黃門之怨。唯登科人李郃謂人曰:「劉濩不第,我輩登科,實厚顏矣!」請以所授官讓濩。事雖不行,人士多之。令狐楚在興元,牛僧孺鎮襄陽,闢為
 從事,待如師友。位終使府御史。



 李商隱,字義山,懷州河內人。曾祖叔恆,年十九登進士第,位終安陽令。祖俌,位終邢州錄事參軍。父嗣。



 商隱幼能為文。令狐楚鎮河陽,以所業文乾之,年才及弱冠。楚以其少俊,深禮之,令與諸子游。楚鎮天平、汴州,從為巡官,歲給資裝,令隨計上都。開成二年,方登進士第,釋褐秘書省校書郎,調補弘農尉。會昌二年,又以書判拔萃。



 王茂元鎮河陽,闢為掌書記,得待御史。茂元愛其才,以
 子妻之。茂元雖讀書為儒,然本將家子,李德裕素遇之,時德裕秉政,用為河陽帥。德裕與李宗閔、楊嗣復、令狐楚大相仇怨。商隱既為茂元從事,宗閔黨大薄之。時令狐楚已卒,子綯為員外郎,以商隱背恩,尤惡其無行。俄而茂元卒,來游京師,久之不調。會給事中鄭亞廉察桂州,請為觀察判官、檢校水部員外郎。大中初,白敏中執政,令狐綯在內署,共排李德裕逐之。亞坐德裕黨,亦貶循州刺史。商隱隨亞在嶺表累載。



 三年入朝,京兆尹盧
 弘正奏署掾曹,令典箋奏。明年,令狐綯作相,商隱屢啟陳情,綯不之省。弘正鎮徐州,又從為掌書記。府罷入朝,復以文章幹襜,乃補太學博士。會河南尹柳仲郢鎮東蜀,闢為節度判官、檢校工部郎中。大中末,仲郢坐專殺左遷,商隱廢罷,還鄭州,未幾病卒。



 商隱能為古文,不喜偶對。從事令狐楚幕。楚能章奏,遂以其道授商隱,自是始為今體章奏。博學強記,下筆不能自休,尤善為誄奠之辭。與太原溫庭筠、南郡段成式齊名,時號「三十六」。文
 思清麗,庭筠過之。而俱無持操,恃才詭激,為當塗者所薄。名宦不進,坎壈終身。



 弟羲叟,亦以進士擢第,累為賓佐。商隱有表狀集四十卷。



 溫庭筠者,太原人,本名岐,字飛卿。大中初,應進士。苦心硯席,尤長於詩賦。初至京師,人士翕然推重。然士行塵雜,不修邊幅,能逐弦吹之音,為測艷之詞,公卿家無賴子弟裴誠、令狐縞之徒,相與蒱飲,酣醉終日,由是累年不第。徐商鎮襄陽,往依之,署為巡官。咸通中,失意歸江
 東,路由廣陵,心怨令狐綯在位時不為成名。既至,與新進少年狂游狹邪,久不刺謁。又乞索於楊子院,醉而犯夜,為虞候所擊,敗面折齒,方還揚州訴之。令狐綯捕虞候治之,極言庭筠狹邪醜跡,乃兩釋之。自是污行聞於京師。庭筠自至長安,致書公卿間雪冤。屬徐商知政事,頗為言之。無何,商罷相出鎮,楊收怒之,貶為方城尉。再遷隋縣尉,卒。



 子憲,以進士擢第。弟庭皓,咸通中為徐州從事,節度使崔彥魯為龐勛所殺,庭皓亦被害。



 庭筠著
 述頗多,而詩賦韻格清拔,文士稱之。



 薛逢,字陶臣,河東人。父倚。逢會昌初進士擢第,釋褐秘書省校書郎。崔鉉罷相鎮河中,闢為從事。鉉復輔政,奏授萬年尉,直弘文館,累遷侍御史、尚書郎。



 逢文詞俊拔,論議激切,自負經畫之略,久之不達。應進士時,與彭城劉彖尤相善,而彖詞藝不迨逢,逢每侮之。至大中末,彖揚歷禁署,逢愈不得意,自是相怨。俄而彖知政事,或薦逢知制誥,彖奏曰:「先朝立制,兩省官給事中、舍人除拜,
 須先歷州縣。逢未嘗治郡,宜先試之。」乃出為巴州刺史。既而沈詢、楊收、王鐸由學士相繼為將相,皆逢同年進士,而逢文藝最優。楊收作相後,逢有詩云:「須知金印朝天客,同是沙堤避路人。威鳳偶時皆瑞聖,潛龍無水謾通神。」收聞,大銜之,又出為蓬州刺史。收罷相,入為太常少卿。給事中王鐸作相,逢又有詩云:「昨日鴻毛萬鈞重,今朝山嶽一塵輕。」鐸又怨之。以恃才褊忿,人士鄙之。遷秘書監,卒。



 子廷珪。中和中登進士第。大順初,累遷司勛員
 外郎,知制誥,正拜中書舍人。乾寧三年,奉使太原復命,昭宗幸華州,改左散騎常侍。移疾免,客游成都。光化中,復為中書舍人。遷刑部、吏部二侍郎,權知禮部貢舉,拜尚書左丞。入梁,至禮部尚書。



 李拯,字昌時,隴西人。咸通十二年登進士第。乾符中,累佐府幕。黃巢之亂,避地平陽。僖宗還京,召拜尚書郎,轉考功郎中,知制誥。僖宗再幸寶雞,拯扈從不及,在鳳翔。襄王僭號,逼為翰林學士。拯既污偽署,心不自安。後硃
 玫秉政,百揆無敘,典章濁亂,拯嘗朝退,駐馬國門,望南山而吟曰:「紫宸朝罷綴鴛鸞,丹鳳樓前駐馬看。唯有終南山色在,晴明依舊滿長安。」吟已涕下。及王行瑜殺硃玫,襄王出奔,京城亂,拯為亂兵所殺。



 妻盧氏,知書能文,有姿色。拯既死,伏其尸慟哭。賊逼之,堅哭不動;又臨之以兵,至於斷一臂,終不顧,為賊所害,人皆傷之。



 李巨川,字下己,隴右人。國初十八學士道玄之後,故相逢吉之侄曾孫。父循,大中八年登進士第。



 巨川乾符中
 應進士,屬天下大亂,流離奔播,切於祿位,乃以刀筆從諸侯府。王重榮鎮河中,闢為掌書記。時車駕在蜀,賊據京師,重榮匡合諸籓,葉力誅寇,軍書奏請,堆案盈幾。巨川文思敏速,翰動如飛,傳之籓鄰,無不聳動,重榮收復功,巨川之助也。及重榮為部下所害,朝議罪參佐,貶為漢中掾。時楊守亮帥興元,素知之,聞巨川至,喜謂客曰:「天以李書記遺我也!」即命管記室,累遷幕職。



 景福中,守亮為李茂貞所攻,城陷,以部下數百人欲投太原。入秦,
 為華軍所擒。巨川時從守亮,亦被械系。在途,巨川題詩於樹葉以遺華帥韓建,詞情哀鳴,建欣然解縛。守亮誅,即命為掌書記。俄而李茂貞犯京師,天子駐蹕於華。韓建以一州之力,供億萬乘,慮其不濟,遣巨川傳檄天下,請助轉餉,同匡王室,完葺京城。四方書檄,酬報輻湊,巨川灑翰陳敘,文理俱愜,昭宗深重之,即時巨川之名聞於天下。昭宗還京,特授諫議大夫,仍留佐建。



 光化初,硃全忠陷河中,進兵入潼關。建懼,令巨川見全忠送款,至
 河中,從容言事。巨川指陳利害,全忠方圖問鼎,聞巨川所陳,心惡之。判官敬翔,亦以文筆見知於全忠,慮得巨川減落名價,謂全忠曰:「李諫議文章信美,但不利主人。」是日為全忠所害。



 司空圖,字表聖,本臨淄人。曾祖遂,密令。祖彖,水部郎中。父輿,精吏術。大中初,戶部侍郎盧弘正領鹽鐵,奏輿為安邑兩池榷鹽使、檢校司封郎中。先是,鹽法條例疏闊,吏多犯禁;輿乃特定新法十條奏之,至今以為便。入朝
 為司門員外郎,遷戶部郎中,卒。



 圖,咸通十年登進士第,主司王凝於進士中尤奇之。凝左授商州刺史,圖請從之。凝加器重,洎廉問宣歙,闢為上客。召拜殿中侍御史,以赴闕遲留,責授光祿寺主簿,分司東都。



 乾符六年,宰相盧攜罷免,以賓客分司,圖與之游,攜嘉其高節,厚禮之。嘗過圖舍,手題於壁曰:「姓氏司空貴,官班御史卑。老夫如且在,不用念屯奇。」明年,攜復入朝,路由陜虢,謂陜帥盧渥曰:「司空御史,高士也,公其厚之。」渥即日奏為賓
 佐。其年,攜復知政事,召圖為禮部員外郎,賜緋魚袋,遷本司郎中。



 其年冬,巢賊犯京師,天子出幸,圖從之不及,乃退還河中。時故相王徽亦在蒲,待圖頗厚。數年,徽受詔鎮潞,乃表圖為副使,徽不赴鎮而止。僖宗自蜀還,次鳳翔,召圖知制誥,尋正拜中書舍人。其年僖宗出幸寶雞,復從之不及,退還河中。



 龍紀初,復召拜舍人,未幾又以疾辭。河北亂,乃寓居華陰。景福中,又以諫議大夫征。時朝廷微弱,紀綱大壞,圖自深惟出不如處,移疾不起。
 乾寧中,又以戶部侍郎征,一至闕廷致謝,數日乞還山,許之。昭宗在華,徵拜兵部侍郎,稱足疾不任趨拜,致章謝之而已。昭宗遷洛,鼎欲歸梁,柳璨希賊旨,陷害舊族,詔圖入朝。圖懼見誅,力疾至洛陽,謁見之日,墮笏失儀,旨趣極野。璨知不可屈,詔曰:「司空圖俊造登科,硃紫升籍,既養高以傲代,類移山以釣名,心惟樂於漱流,任非專於祿食。匪夷匪惠,難居公正之朝;載省載思,當徇棲衡之志。可放還山。」



 圖有先人別墅在中條山之王官穀,
 泉石林亭,頗稱幽棲之趣。自考槃高臥,日與名僧高士游詠其中。晚年為文,尤事放達,嘗擬白居易《醉吟傳》為《休休亭記》曰:



 司空氏禎貽溪之休休亭,本名濯纓亭,為陜軍所焚。天復癸亥歲,復葺於壞垣之中,乃更名曰休休。休,休也,美也,既休而具美存焉。蓋量其才一宜休,揣其分二宜休,耄且聵三宜休。又少而惰,長而率,老而迂,是三者皆非濟時之用,又宜休也。尚慮多難不能自信,既而晝寢,遇二僧謂予曰:「吾嘗為汝師。汝昔矯於道,銳
 而不固,為利欲之所拘,幸悟而悔,將復從我於是溪耳!且汝雖退,亦嘗為匪人之所嫉,宜耐辱自警,庶保其終始,與靖節、醉吟第其品級於千載之下,復何求哉!」因為《耐辱居士歌》,題於東北楹曰:「咄咄,休休休,莫莫莫,伎倆雖多性靈惡,賴是長教閑處著。休休休,莫莫莫,一局棋,一爐藥,天意時情可料度。白日偏催快活人,黃金難買堪騎鶴。若曰:『爾何能?』答云:『耐辱莫。』」其詭激嘯傲,多此類也。



 圖既脫柳璨之禍還山,乃預為壽藏終制。故人來者,
 引之壙中,賦詩對酌。人或難色,圖規之曰:「達人大觀,幽顯一致,非止暫游此中。公何不廣哉!」圖布衣鳩杖,出則以女家人鸞臺自隨。歲時村社雩祭祠禱,鼓舞會集,圖必造之,與野老同席,曾無傲色。王重榮父子兄弟尤重之,伏臘饋遺,不絕於途。



 唐祚亡之明年,聞輝王遇弒於濟陰,不懌而疾,數日卒,時年七十二。有文集三十卷。



 圖無子,以其甥荷為嗣。荷官至永州刺史。以甥為嗣,嘗為御史所彈,昭宗不之責。



 贊曰:國之華彩,人文化成。間代傑出,奮藻摛英。騏驥逸步,《咸》、《韶》正聲。粲流緗素,下視姬、嬴。



\end{pinyinscope}