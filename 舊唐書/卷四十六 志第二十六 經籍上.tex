\article{卷四十六 志第二十六 經籍上}

\begin{pinyinscope}

 夫龜文成象,肇八卦於庖犧;鳥跡分形,創六書於蒼頡。聖作明述,同源異流。《墳》、《典》起之於前,《詩》、《書》繼之於後。先王陳跡,後王準繩。《易》曰:「觀乎人文以化成天下。」《禮》曰:「君子如
 欲化民成俗,其必由學乎!」學者非他,方策之謂也。琢玉成器,觀古知今,歷代哲王,莫不崇尚。自仲尼沒而微言絕,七十子喪而大義乖。嬴氏坑焚,以愚黔首;漢興學校,復創石渠。雄、向校讎於前,馬、鄭討論於後,兩京載籍,由是粲然。及漢末還都,焚溺過半。爰自魏、晉,迄於周、隋,而好事之君,慕古之士,亦未嘗不以圖籍為意也。然河北江南,未能混一;偏方購輯,卷帙未弘。而荀勖、李充、王儉、任昉、祖恆,皆達學多聞,歷世整比,群分類聚,遞相祖述。
 或為七錄,或為四部,言其部類,多有所遺。及隋氏建邦,寰區一統,煬皇好學,喜聚逸書,而隋世簡編,最為博洽。及大業之季,喪失者多。貞觀中,令狐德棻、魏徵相次為秘書監,上言以籍亡逸,請行購募,並奏引學士校定。群書大備。



 開元三年,左散騎常侍褚無量、馬懷素侍宴,言及經籍。玄宗曰:「內庫皆是太宗、高宗先代舊書,常令宮人主掌,所有殘缺,未遑補緝,篇卷錯亂,難於檢閱。卿試為朕整比之。」至七年,詔公卿士庶之家,所有異書,官借
 繕寫。及四部書成,上令百官入乾元殿東廊觀之,無不駭其廣。九年十一月,殷踐猷、王愜、韋述、余欽、毋煚、劉彥真、王灣、劉仲等重修成《群書四部錄》二百卷,右散騎常侍元行沖奏上之。自後毋煚又略為四十卷,名為《古今書錄》,大凡五萬一千八百五十二卷。祿山之亂,兩都覆沒,乾元舊籍,亡散殆盡。肅宗、代宗崇重儒術,屢詔購募。文宗時,鄭覃侍講禁中,以經籍道喪,屢以為言。詔令秘閣搜訪遺文,日令添寫。開成初,四部書至五萬六千四
 百七十六卷。及廣明初,黃巢乾紀,再陷兩京,宮廟寺署,焚蕩殆盡,曩時遺籍,尺簡無存。及行在朝諸儒購輯,所傳無幾。昭宗即位,志弘文雅。秘書省奏曰:「當省元掌四部御書十二庫,共七萬餘卷。廣明之亂,一時散失。後來省司購募,尚及二萬餘卷。及先朝再幸山南,尚存一萬八千卷。竊知京城制置使孫惟晟收在本軍,其御書秘閣見充教坊及諸軍人占住。伏以典籍國之大經,秘府校讎之地,其書籍並望付當省校其殘缺,漸令補輯。樂
 人乞移他所。」並從之。及遷都洛陽,又喪其半。平時載籍,世莫得聞。今錄開元盛時四部諸書,以表藝文之盛。



 四部者,甲、乙、丙、丁之次也。



 甲部為經,其類十二:一日《易》,以紀陰陽變化。二曰《書》,以紀帝王遺範。三曰《詩》,以紀興衰誦嘆。四曰《禮》,以紀文物體制。五曰《樂》,以紀聲容律度。六曰《春秋》,以紀行事褒貶。七曰《孝經》,以紀天經地義。八曰《論語》,以紀先聖微言。九曰圖緯,以紀六經讖候。十曰經解,以紀六經讖候。十一曰詁訓,以紀六經讖候。十二曰
 小學,以紀字體聲韻。



 乙部為史,其類十有三:一曰正史,以紀紀傳表志。二曰古史,以紀編年系事。三曰雜史,以紀異體雜紀。四曰霸史,以紀偽朝國史。五曰起居注,以紀人君言動。六曰舊事,以紀朝廷政令。七曰職官,以紀班序品秩。八曰儀注,以紀吉兇行事。九曰刑法,以紀律令格式。十曰雜傳,以紀先聖人物。十一曰地理,以紀山川郡國。十二曰譜系,以紀世族繼序。十三曰略錄,以紀史策條目。



 丙部為子,其類一十有四:一曰儒家,以紀仁義教化。二
 曰道家,以紀清凈無為。三曰法家,以紀刑法典制。四曰名家,以紀循名責實。五曰墨家,以紀強本節用。六曰縱橫家,以紀辯說詭詐。七曰雜家,以紀兼敘眾說。八曰農家,以紀播植種藝。九曰小說家,以紀芻辭輿誦。十曰兵法,以紀權謀制度。十一曰天文,以紀星辰象緯。十二曰歷數,以紀推步氣朔。十三曰五行,以紀卜筮占候。十四曰醫方,以紀藥餌針灸。



 丁部為集,其類有三:一曰楚詞,以紀騷人怨刺。二曰別集,以紀詞賦雜論。三
 曰總集,以紀文章事類。



 煚等撰集,依班固《藝文志》體例,諸書隨部皆有小序,發明其旨。近史官撰《隋書經籍志》,其例亦然。竊以紀錄簡編異題,卷部相沿,序述無出前修。今之殺青,亦所不取,但紀部帙而已。而煚等所序四部都錄以明新修之旨,今略載之:



 竊以經墳浩廣,史圖紛博,尋覽者莫之能遍,司總者常苦其多,何暇重屋復床,更繁其說?若先王有闕典,上聖有遺事,邦政所急,儒訓是先,宜垂教以作程,當闡規而開典,則不遑啟處,何
 獲宴寧。曩之所修,誠惟此義,然禮有未愜,追怨良深。於時秘書省經書,實多亡闕,諸司墳籍,不暇討論。此則事有未周,一也。其後周覽人間,頗睹闕文,新集記貞觀之前,永徽已來不取;近書採長安之上,神龍已來未錄。此則理有未弘,二也。書閱不遍,事復未周,或不詳名代,或未知部伍。此則體有未通,三也。書多闕目,空張第數,既無篇題,實乖標榜。此則例有所虧,四也。所用書序,咸取魏文貞;所分書類,皆據《隋經籍志》。理有未允,體有不通。
 此則事實未安,五也。昔馬談作《史記》,班彪作《漢書》,皆兩葉而僅成;劉歆作《七略》,王儉作《七志》,逾二紀而方就。孰有四萬卷目,二千部書,名目首尾,三年便令終竟,欲求精悉,不其難乎?所以常有遺恨,竊思追雪。乃與類同契,積思潛心,審正舊疑,詳開新制。永徽新集,神龍近書,則釋而附也。未詳名氏,不知部伍,則論而補也。空張之目,則檢獲便增。未允之序,則詳宜別作。紕繆咸正,混雜必刊。改舊傳之失者三百餘條,加新書之目者六千餘卷。
 凡經錄十二家,五百七十五部,六千二百四十一卷。史錄十三家,八百四十部,一萬七千九百四十六卷。子錄十七家,七百五十三部,一萬五千六百三十七卷。集錄三家,八百九十二部,一萬二千二十八卷。凡四部之錄四十五家,都管三千六十部,五萬一千八百五十二卷,成《書錄》四十卷。其外有釋氏經律論疏,道家經戒符籙,凡二千五百餘部,九千五百餘卷。亦具翻譯名氏,序述指歸,又勒成目錄十卷,名曰《開元內外經錄》。若夫先王
 秘傳,列代奧文,自古之粹籍錄符,絕域之神經怪牒,盡載於此二書矣。



 夫經籍者,開物成務,垂教作程,聖哲之能事,帝王之達典。而去聖已久,開鑿遂多,茍不剖判條源,甄明科部,則先賢遺事,有卒代而不聞,大國經書,遂終年而空泯。使學者孤舟泳海,弱羽憑天,銜石填溟,倚杖追日,莫聞名目,豈詳家代?不亦勞乎!不亦弊乎!將使書千帙於掌眸,披萬函於年祀,覽錄而知旨,觀目而悉詞,經墳之精術盡探,賢哲之睿思咸識,不見古人之面,
 而見古人之心,以傳後來,不其愈已!



 其序如此。



 煚等《四部目》及《釋道目》,並有小序及注撰人姓氏,卷軸繁多,今並略之,但紀篇部,以表我朝文物之大。其《釋道錄目》附本書,今亦不取,據開元以籍為之志。天寶已後,名公各著文章,儒者多有撰述,或記禮法之沿革,或裁國史之繁略,皆張部類,其徒實繁。臣以後出之書,在開元四部之外,不欲雜其本部,今據所聞,附撰人等傳。其諸公文集,亦見本傳,此並不錄。四部區分,詳之於下。



 甲部經錄,十二家,五百七十五部,六千二百四十一卷。



 《易》類一《書》類二《詩》類三《禮》類四《樂》類五《春秋》類六《孝經》類七《論語》類八讖緯類九經解類十詁訓類十一小學類十二



 《歸藏》十三卷殷易,司馬膺注。



 《周易》二卷卜商傳。



 又十卷孟喜章句。



 又十卷京房章句。



 又四卷費直章句。



 又十卷馬融章句。



 又九卷鄭玄注。



 又十卷荀爽章句。



 又五卷劉表注。



 又十卷王肅注。



 又十卷董遇注。



 又十卷宋衷注。



 又七卷王弼注。



 又九卷虞翻注。



 又十三卷陸績注。



 又十卷荀氏九家集解。



 又十卷馬、鄭、二王集解。



 又十卷姚信注。



 又十卷王弼、韓康伯注。



 又十卷二王集注。



 又十卷荀暉注。



 又十卷蜀才注。



 又十卷張璠集解。



 又十卷王暠注。



 又十卷干寶注。



 又十卷黃穎注。



 又十卷崔浩注。



 又十三卷崔覲注。



 又十卷何胤注。



 又十卷盧氏注。



 又十四卷傅氏注。



 又十卷王玄度注。



 又十卷王又玄注。



 又十卷任希古注。



 又十卷王凱沖注。



 《周易發揮》五卷王勃撰。



 《周易系辭》二卷謝萬注。



 又二卷桓玄注。



 又二卷荀諺注。



 又二卷宋褰注。



 《周易義疏》二十卷宋明帝注。



 《宋群臣講易疏》二十卷張該等注。



 《周易大義》二十卷梁武帝撰。



 《周易講疏》三十五卷梁武帝撰。



 《周易發題義》一卷



 《周易幾義》一卷蕭偉撰。



 《周易大義疑問》二十卷梁武帝撰。



 《周易義疏》十四卷蕭子政撰。



 《周易講疏》三十卷張譏注。



 又十三卷何妥撰。。



 又十六卷褚仲都撰。



 《周易正義》十四卷孔穎撻撰。



 《
 周易新論》十卷陰弘道撰。



 《周易文句義疏》二十四卷陸德明撰。



 《周易文外大義》二卷陸德明撰。



 《周易新注本義》十四卷薛仁貴撰。



 《周易開題論序疏》十卷。



 《周易文句義疏》二十卷已上並梁蕃撰。



 《周易大衍論》三卷玄宗撰。



 《周易論》四卷鐘會撰。



 《周易大衍論》一卷王弼撰。



 《周易論》一卷應吉甫撰。



 《周易統略論》三卷鄒湛撰。



 《周易略論》一卷張璠撰。



 《
 周易論》二卷暨長成難,暨仲容答。



 《易論》一卷宋處宗撰。



 《通易象論》一卷宣聘撰。



 又一卷欒永初撰。



 《周易系辭義疏》二卷劉瓛撰。



 《周易乾坤義疏》一卷劉瓛撰。



 《周易略譜》一卷沈熊撰。



 《周易爻義》一卷干寶撰。



 《周易卦序論》一卷楊乂撰。



 《周易譜》一卷袁宏撰。



 《周易論》四卷範氏撰。



 《周易雜音》三卷



 《周易釋序義》三卷梁蕃撰。



 右《易》七十八部,凡六百七十三卷



 《
 古文尚書》十三卷孔安國傳。



 又十卷孔安國傳,範寧注。



 又十卷李顒集注。



 又十卷姜道盛集注。



 又十卷馬融注。



 又九卷鄭玄注。



 又十卷王肅注。



 又十三卷謝沈注。



 《尚書暢訓》三卷伏勝注。



 《尚書洪範五行傳》十一卷劉向撰。



 《尚書答問》三卷王肅注。



 《尚書釋駁》五卷王肅撰。



 《尚書釋問》四卷鄭玄注。王粲問,田瓊、韓益正。



 《
 尚書義注》三卷呂文優撰。



 《尚書釋義》四卷伊說撰。



 《尚書要略》二卷李顒撰。



 《尚書新釋》二卷李顒撰。



 《尚書百問》一卷顧歡撰。



 《尚書義疏》十卷巢猗撰。



 《尚書百釋》三卷巢猗撰。



 《尚書義疏》十卷費甝撰。



 《古文尚書大義》二十卷任孝恭撰。



 《尚書義疏》三十卷蔡大寶撰。



 《尚書文外義》三十卷顧彪撰。



 《尚書義疏》二十卷劉焯撰。



 《尚書述義》二十卷劉炫撰。



 《尚書正義》二十卷孔穎達撰。



 《古文尚書音義》五卷顧彪撰。



 《尚書音義》四卷王儉撰。



 右《尚書》二十九部,凡二百七十二卷。



 《韓詩》二十卷卜商序,韓嬰撰。



 《韓詩外傳》十卷韓嬰撰。



 《毛詩》十卷毛萇撰。



 《毛詩詁訓》二十卷鄭玄箋。



 《毛詩》二十卷王肅注。



 《葉詩》二十卷葉遵注。



 《集注毛詩》二十四卷崔靈恩集注。



 《韓詩翼要》十卷卜商撰。



 《毛詩譜》二卷鄭玄撰。



 《毛詩集序》二卷卜商撰。



 《毛詩義注》五卷



 《毛詩雜義駁》八卷王肅撰。



 《毛詩問難》二卷王肅撰。



 《毛詩駁》五卷王伯輿撰。



 《
 毛詩義問》十卷劉楨撰。



 《毛詩雜答問》五卷



 《毛詩雜義難》十卷



 《毛詩異同評》十卷孫毓撰。



 《毛詩釋義》十卷謝沈撰。



 《毛詩辯》三卷楊乂撰。



 《毛詩序義》一卷劉氏撰。



 《毛詩表隱》二卷



 《毛詩義疏》五卷張氏撰。



 《毛詩誼府》三卷元延明撰。



 《毛詩草木鳥獸魚蟲疏》二卷陸璣撰。



 《毛詩述義》三十卷劉炫撰。



 《毛詩正義》四十卷孔穎達撰。



 《毛詩音義》二卷魯世達撰。



 《毛詩諸家音》十五卷鄭玄等
 注。



 《難孫氏詩評》四卷陳統撰。



 右《詩》三十部,凡三百十三卷。



 《周官》十二卷馬融傳。



 《周官禮》十三卷鄭玄注。



 又十卷伊說撰。



 又十二卷王肅注。



 又十二卷干寶注。



 《周官論評》十二卷陳邵駁,傅玄評。



 《周官寧朔新書》八卷司馬伷序,王懋約注。



 《周官駁難》五卷孫略問,干寶答。



 《周禮義疏》四十卷沈重撰。



 《周禮疏》五十卷賈公彥撰。



 《周禮義決》三卷王玄度撰。



 《
 周官音》三卷鄭玄撰。



 《儀禮》十七卷鄭玄注。



 又十七卷王肅注。



 《儀禮音》二卷



 《喪服紀》一卷馬融注。



 又一卷鄭玄注。



 又一卷袁準注。



 又一卷



 又一卷陳銓注。



 又二卷蔡超宗注。



 又二卷田僧紹注。



 《喪服變除》一卷戴德撰。



 《喪服要紀》一卷王肅注。



 《喪服要集議》三卷杜預撰。



 《喪服要紀》五卷賀循撰,謝微注。



 《儀禮疏》五十卷賈公彥撰。



 《
 喪服變除》一卷鄭玄撰。



 《喪服要紀》十卷賀循撰,庾蔚之注。



 《喪服古今集記》三卷王儉撰。



 《喪服五代行要記》十卷王逡之志。



 《喪服經傳義疏》四卷沈文阿撰。



 《喪服發題》二卷沈文阿撰。



 《喪服文句義》十卷皇侃撰。



 《喪服天子諸侯圖》二卷謝慈撰。



 《喪服圖》一卷崔游撰。



 《喪服譜》一卷蔡謨撰。



 《喪服譜》一卷賀循撰。



 《喪服要難》一卷趙成問,仇祈答。



 《大戴禮記》十三卷戴德撰。



 《小戴禮記》二十卷戴聖撰,鄭玄注。



 《禮記》二十卷盧植注。



 又三十卷王肅注。



 又三十卷孫炎注。



 又十二卷葉遵注。



 《禮記寧朔新書》二十卷司馬伷序,王懋約注。



 《次禮記》二十卷魏徵撰。



 《月令章句》十二卷戴顒撰。



 《禮記中庸傳》二卷戴顒撰。



 《禮記義記》四卷鄭小同撰。



 《禮記要鈔》六卷緱氏撰。



 《禮記音》二卷鄭玄注,曹耽解。



 又二卷謝慈撰。



 又二卷李軌撰。



 又二卷尹毅撰。



 又三卷徐邈撰。



 又二卷徐爰撰。



 《禮記隱》二十六卷



 《禮記略解》十卷庾蔚之撰。



 《禮記講疏》一百卷皇侃撰。



 《禮記義疏》五十卷皇侃撰。



 《禮記義疏》四十卷沈重撰。



 《禮記義疏》四十卷熊安生撰。



 《禮記義證》十卷劉芳撰。



 《禮記類聚》十卷



 《禮記正義》七十卷孔穎達撰。



 《禮記疏》八十卷賈公彥撰。



 《禮論》三百七卷何承天撰。



 《禮義》二十卷戴聖等撰。



 《三禮目錄》一卷鄭玄注。



 《問禮俗》十卷董勛撰。



 《禮記評》十卷劉雋撰。



 《
 禮儀問答》十卷王儉撰。



 《雜禮義》十一卷吳商等撰。



 《禮義雜記故事》十一卷



 《禮問》九卷範寧撰。



 《禮論答問》九卷範寧撰。



 《禮論問答》九卷徐廣撰。



 《雜禮儀問答》四卷戚壽撰。



 《禮論降議》三卷顏延之撰。



 《禮論條牒》十卷任預撰。



 《禮論帖》三卷任預撰。



 《禮論抄》六十六卷任預撰。



 《禮論抄》二十卷庾蔚之撰。



 《禮儀答問》十卷王儉撰。



 《禮雜抄略》二卷荀萬秋撰。



 《禮議》一卷傅伯祚撰。



 《禮統郊祀》六卷



 《
 禮論要抄》十三卷



 《禮記區分》十卷



 《禮論抄略》十三卷



 《禮大義》十卷梁武帝撰。



 《禮疑義》五十卷周舍撰。



 《禮記義》十卷何佟之撰。



 《禮答問》十卷何佟之撰。



 《三禮義宗》三十卷崔靈恩撰。



 《禮論要抄》一百卷賀瑒撰。



 《禮統》十三卷賀述撰。



 《三禮宗略》二十卷元延明撰。



 《三禮圖》十二卷夏侯伏朗撰。



 《江都集禮》一百二十卷潘徽等撰。



 《大唐新禮》一百卷房玄齡等撰。



 《紫宸禮要》十卷大聖天后撰。



 右《禮》一百四部,《周禮》十三家,《儀禮》、《喪服》二十八家,禮論答問三十五家,凡一千九百四十五卷。



 《樂書》九卷信都芳注。



 《管弦記》十二卷留進錄,凌秀注。



 《鐘磬志》二卷公孫崇撰。



 《樂社大義》十卷梁武帝撰。



 《樂論》三卷梁武帝撰。



 《鐘律》五卷沈重撰。



 《古今樂錄》十三卷釋智匠撰。



 《樂府聲調》六卷鄭譯撰。



 《樂譜集解》二十卷蕭吉撰。



 《樂志》十卷蘇夔撰。



 《樂經》三十卷季玄楚撰。



 《樂書要錄》十卷大聖天后撰。



 《
 樂略》四卷元殷撰。



 《聲律指歸》一卷元殷撰。



 《樂元起》二卷桓譚撰。



 《琴操》二卷桓譚撰。



 《琴操》三卷孔衍撰。



 《琴譜》四卷劉氏、周氏等撰。



 《琴譜》二十一卷陳懷撰。



 《琴敘譜》九卷趙耶律撰。



 《琴集歷頭拍簿》一卷



 《外國伎曲》三卷



 《論樂事》二卷



 《外國伎曲名》一卷



 《歷代曲名》一卷



 《推七音》一卷



 《十二律譜義》一卷



 《鼓吹樂章》一卷



 《
 古今樂記》八卷李守真撰。



 右《樂》二十九部,凡一百九十五卷。



 《春秋三家經詁訓》十二卷賈逵撰。



 《春秋經》十一卷士燮撰。



 《春秋傳》十卷王朗注。



 《春秋左氏長經章句》三十卷賈逵撰。



 《春秋左氏傳解詁》三十卷賈逵撰。



 《春秋左氏傳解誼》三十卷服虔注。



 《春秋左氏經傳章句》三十卷董遇注。



 《
 春秋左氏傳》三十卷王肅注。



 《春秋左氏傳》三十卷杜預注。



 《春秋左氏傳義注》三十卷孫毓注。



 《春秋左氏傳音》三卷高貴鄉公撰。



 《春秋左氏音》四卷曹耽、荀訥撰。



 《春秋左氏音隱》一卷服虔撰。



 《春秋左氏傳音》三卷杜預注。



 又三卷李弘範撰。



 又三卷孫邈撰。



 又三卷王元規撰。



 又十二卷



 《春秋左氏傳條例》二十卷劉歆撰。



 《春秋左氏傳條例章句》九卷鄭眾撰。



 《
 春秋左氏傳例》七卷



 又十五卷杜預撰。



 《春秋左氏條例》十卷劉實撰。



 《春秋左氏經例》十卷方範撰。



 《春秋左氏膏肓》十卷何休撰,鄭玄箴。



 《春秋成長說》七卷服虔撰。



 《春秋左氏膏肓釋痾》五卷服虔撰。



 《春秋達長義》一卷王玢撰。



 《春秋左氏傳說要》十卷糜信撰。



 《春秋塞難》三卷服虔撰。



 《春秋左氏傳賈服異同略》五卷孫毓撰。



 《春秋左氏傳例苑》十八卷梁簡文帝撰。



 《春秋義函傳》十六卷干寶撰。



 《春秋左氏釋滯》十卷殷興撰。



 《
 春秋序論》一卷干寶撰。



 《春秋左氏區分》十二卷何始貞撰。



 《春秋左氏義略》三十卷張沖撰。



 《春秋左氏抄》十卷



 《左氏杜預評》三卷



 《春秋圖》七卷嚴彭祖撰。



 《春秋辭苑》五卷



 《春秋經傳詭例疑隱》一卷吳略撰。



 《春秋雜義》五卷



 《春秋土地名》三卷



 《春秋旨通》十卷王延之撰。



 《春秋大夫譜》十一卷顧啟期撰。



 《
 春秋叢林》十二卷李謐撰。



 《春秋立義》十卷崔靈恩撰。



 《春秋申先儒傳例》十卷崔靈恩撰。



 《春秋經解》六卷沈宏撰。



 《春秋文苑》六卷沈宏撰。



 《春秋嘉語》六卷沈宏撰。



 《春秋義略》二十七卷沈文阿撰。



 《春秋攻昧》十二卷劉炫撰。



 《春秋規過》三卷劉炫撰。



 《春秋述議》三十七卷劉炫撰。



 《春秋正義》三十七卷孔穎達撰。



 《春秋公羊傳》五卷公羊高傳,嚴彭祖述。



 《春秋公羊經傳》十三卷何休注。



 《
 春秋公羊經傳集解》十四卷孔氏注。



 《春秋公羊》十二卷王愆期撰。



 《春秋公羊傳記》十二卷高襲注。



 《何氏春秋漢議》十一卷何休撰,鄭玄駁,糜信注。



 《何氏春秋漢記》十一卷服虔撰。



 《春秋公羊條傳》一卷何休注。



 《春秋公羊墨守》二卷何休撰,鄭玄發。



 《春秋公羊答問》五卷荀爽問,徐欽答。



 《春秋公羊音》二卷王儉撰。



 《
 春秋公羊違義》三卷劉實撰,劉晏注。



 《春秋公羊論》二卷庾翼難,王愆期答。



 《春秋穀梁傳》十三卷段氏注。



 《春秋穀梁章句》十五卷穀梁俶解,尹更始注。



 《春秋穀梁傳》十二卷唐固注。



 又十二卷糜信注。



 又十一卷張靖集解。



 《春秋公羊違義》三卷劉晏注。



 《春秋穀梁經傳》十六卷程闡集注。



 《春秋穀梁傳》十三卷孔衍訓注。



 又十二卷範寧集注。



 又十三卷徐乾注。



 《春秋穀梁》十二卷徐邈注。



 《
 春秋穀梁經集解》十卷沈仲義注。



 《春秋穀梁廢疾》三卷何休作,鄭玄釋,張靖箴。



 《穀梁傳義》三卷蕭邕注。



 《春秋穀梁傳義》十二卷徐邈注。



 《春秋穀梁音》一卷徐邈撰。



 《春秋穀梁傳疏》十三卷楊士勛撰。



 《春秋公羊穀梁左氏集解》十一卷劉兆撰。



 《春秋三傳論》十卷韓益撰。



 《春秋三傳經解》十一卷胡訥集撰。



 《春秋三傳評》十卷胡訥撰。



 《
 春秋公羊穀梁二傳評》三卷江熙撰。



 《春秋繁露》十七卷董仲舒撰。



 《春秋辯證明經論》六卷



 《春秋二傳異同》十一卷李鉉撰。



 《春秋合三傳通論》十卷潘叔度注。



 《春秋成集》十卷潘叔度注。



 《春秋外傳國語》二十卷左丘明撰。



 《春秋外傳國語章句》二十二卷王肅注。



 《春秋外傳國語》二十一卷虞翻撰。



 又二十一卷韋昭注。



 又二十一卷



 又二十一卷唐固注。



 右《春秋》一百二部,一千一百八十四卷。



 《古文孝經》一卷孔子說,曾參受,孔安國傳。



 《孝經》一卷王肅注。



 又一卷鄭玄注。



 《古文孝經》一卷劉邵注。



 《孝經》一卷韋昭注。



 又一卷孫熙注。



 又一卷蘇林注。



 《孝經默注》二卷徐整撰。



 又一卷謝萬注。



 又一卷虞盤佐注。



 又一卷孔光注。



 又一卷殷仲文注。



 又一卷殷叔道注。



 又一卷魏克己注。



 又一卷玄宗注。



 《講孝經義》四卷車胤等注。



 《講孝經集解》一卷荀勖撰。



 《孝經義疏》三卷皇侃撰。



 《大明中皇太子講孝經義疏》一卷何約之執經。



 《孝經疏》十八卷梁武帝撰。



 《孝經發題》四卷太史叔明撰。



 《孝經述義》五卷劉炫撰。



 《孝經疏》五卷賈公彥撰。



 《越王孝經新義》十卷任希古撰。



 《孝經應瑞圖》一卷



 《演孝經》十二卷張士儒撰。



 《孝經疏》三卷元行沖撰。



 《
 論語》十卷何晏集解。



 又十卷鄭玄注,虞喜贊。



 又十卷王肅注。



 又十卷鄭玄注。



 又十卷宋明帝補衛瓘注。



 又十卷李充注。



 又十卷孫綽集解。



 又十卷梁翽注。



 《論語集義》十卷盈氏撰。



 《論語》九卷孟厘注。



 《論語》十卷袁喬注。



 又十卷尹毅注。



 又十卷江熙集解。



 又十卷孫氏注。



 《次論語》五卷王勃撰。



 《論語音》二卷徐邈撰。



 《
 古論語義注譜》一卷徐氏撰。



 《論語釋義》十卷鄭玄注。



 《論語義注》十卷暢惠明撰。



 《論語義注隱》三卷



 《論語篇目弟子》一卷鄭玄注。



 《論語釋疑》二卷王弼撰。



 《論語釋》十卷欒肇撰。



 《論語駁》二卷欒肇撰。



 《論語大義解》十卷崔豹撰。



 《論語旨序》二卷繆播撰。



 《語體略》二卷郭象撰。



 《論語雜義》十三卷



 《論語剔義》十卷



 《論語疏》十卷皇侃撰。



 《論語述義》二十卷戴詵撰。



 《論語章句》二十卷劉炫
 撰。



 《論語疏》十五卷賈公彥撰。



 《論語講疏》十卷褚仲都撰。



 《孔子家語》十卷王肅注。



 《孔叢子》七卷孔鮒撰。



 右六十三部,《孝經》二十七家,《論語》三十六家,凡三百八十七卷。



 《易緯》九卷宋均注。



 《書緯》三卷鄭玄注。



 《詩緯》三卷鄭玄注。



 又十卷宋均注。



 《禮緯》三卷宋均注。



 《樂緯》三卷宋均注。



 《春秋緯》三十八卷宋均注。



 《論語緯》十卷宋均注。



 《
 孝經緯》五卷宋均注。



 《白虎通》六卷漢章帝撰。



 《五經雜義》七卷劉向撰。



 《五經通義》九卷劉向撰。



 《五經要義》五卷劉向撰。



 《五經異義》十卷許慎撰,鄭玄駁。



 《六藝論》一卷鄭玄注。



 《鄭志》九卷



 《鄭記》六卷



 《聖證論》十一卷



 《五經然否論》五卷譙周撰。



 《五經鉤沉》十卷楊方撰。



 《五經咨疑》八卷楊思撰。



 《孔子正言》二十卷梁武帝撰。



 《長春義記》一百卷梁簡文撰。



 《經典大義》十卷沈文阿撰。



 《
 五經宗略》四十卷元延明撰。



 《七經義綱略論》三十卷樊文深撰。



 《質疑》五卷樊文深撰。



 《游玄桂林》二十卷張譏撰。



 《五經正名》十五卷劉炫撰。



 《經典釋文》三十卷陸德明撰。



 《謚法》三卷荀翽演,劉熙注。



 又《謚例》十卷沈約撰。



 《謚法》三卷賀琛約撰。



 《匡謬正俗》八卷顏師古撰。



 《集天名稱》三卷



 右三十六部,經緯九家,七經雜解二十七家,凡四百七十四卷。



 爾雅》三卷李巡注。



 《爾雅》六卷樊光注。



 又六卷孫炎注。



 又三卷郭璞注。



 《集注爾雅》十卷沈璇注。



 《爾雅音義》一卷郭璞注。



 又二卷曹憲撰。



 《爾雅圖》一卷郭璞注。



 《爾雅圖贊》二卷江灌注。



 《爾雅音》六卷江灌注。



 《續爾雅》一卷劉伯莊撰。



 《別國方言》十三卷楊雄撰。



 《釋名》八卷劉熙撰。



 《廣雅》四卷張揖撰。



 《博雅》十卷曹憲撰。



 《小爾雅》一卷李軌撰。



 《
 纂文》三卷何承天撰。



 《纂要》六卷顏延之撰。



 《三蒼》三卷李斯等撰,郭璞解。



 《蒼頡訓詁》二卷杜林撰。



 《三蒼訓詁》二卷張揖撰。



 《埤蒼》三卷張揖撰。



 《廣蒼》一卷樊恭撰。



 《說文解字》十五卷許慎撰。



 《說文音隱》四卷



 《字林》十卷呂忱撰。



 《字統》二十卷楊承慶撰。



 《玉篇》三十卷顧野王撰。



 《字海》一百卷大聖天后撰。



 《文字釋訓》三十卷釋寶志撰。



 《括字苑》十三卷馮乾撰。



 《字屬篇》一卷賈魴撰。



 《
 古文奇字》二卷郭訓撰。



 《字旨篇》一卷郭訓撰。



 《古文字詁》二卷張揖撰。



 《詔定古文官書》一卷衛宏撰。



 《解字文》七卷周成撰。



 《雜文字音》七卷王延撰。



 《文字要說》一卷王氏注。



 《字書》十卷



 《古今八體六文書法》一卷



 《四體書勢》一卷衛恆撰。



 《要用字苑》一卷葛洪撰。



 《難要字》三卷



 《文字集略》一卷阮孝緒撰。



 《辯嫌音》二卷楊休之撰。



 《文字指歸》四卷曹憲撰。



 《證俗音略》二卷顏愍楚撰。



 《
 敘同音》三卷



 《覽字知源》三卷



 《文字辯嫌》一卷彭立撰。



 《聲類》十卷李登撰。



 《韻集》五卷呂靜撰。



 《韻略》一卷楊休之撰。



 《四聲韻略》十三卷夏侯詠撰。



 《四聲部》三十卷張諒撰。



 《韻篇》十二卷趙氏撰。



 《切韻》五卷陸慈撰。



 《桂苑珠叢》一百卷諸葛穎撰。



 《桂苑珠叢略要》二十卷



 《急就章》一卷史游撰,曹壽解。



 《急就章注》一卷顏之推撰。



 又一卷顏師古撰。



 《凡將篇》一卷司馬相如撰。



 《
 飛龍篇篆草勢》合三卷崔瑗撰。



 《在昔篇》一卷班固撰。



 《太甲篇》一卷班固撰。



 《聖草章》一卷蔡邕撰。



 《勸學篇》一卷蔡邕撰。



 《黃初章》一卷



 《吳章》一卷



 《初學篇》一卷硃嗣卿撰。



 《始學篇》十二卷項峻撰。



 《少學集》十卷楊方撰。



 《小學篇》一卷王羲之撰。



 《續通俗文》二卷李虔撰。



 《啟疑》三卷顧凱之撰。



 《詰幼文》三卷顏延之撰。



 《辯字》一卷戴規撰。



 《俗語難字》一卷李少通撰。



 《
 文字志》三卷王愔撰。



 《五十二體書》一卷蕭子云撰。



 《古來篆隸詁訓名錄》一卷



 《書品》一卷庾肩吾撰。



 《書後品》一卷李嗣貞撰。



 《筆墨法》一卷



 《鹿紙筆墨疏》一卷



 《千字文》一卷蕭子範撰。



 又一卷周興嗣撰。



 《篆書千字文》一卷



 《演千字文》五卷



 《今字石經易篆》三卷



 《今字石經尚書》五卷



 《今字石經鄭玄尚書》八卷



 《三字石經尚書古篆》三卷



 《今字石經毛詩》三卷



 《
 今字石經儀禮》四卷



 《三字石經左傳古篆書》十三卷



 《今字石經左傳經》十卷



 《今字石經公羊傳》九卷



 《今字石經論語》二卷蔡邕注。



 《雜字書》八卷釋正度作。



 右小學一百五部,《爾雅》、《廣雅》十八家,偏傍音韻雜字八十六家,凡七百九十七卷。



 乙部史錄,十三家,八百四十四部,一萬七千九百四十六卷。



 正史類一編年類二偽史類三雜史類四



 起居注類五故事類六職官類七雜傳類八



 儀注類九刑法類十目錄類十一譜牒類十二



 地理類十三



 《史記》一百三十卷司馬遷作。



 又八十卷裴駰集解。



 又一百三十卷許子儒注。



 《史記音義》十三卷徐廣撰。



 《史記音義》三卷鄒誕生撰。



 又三十卷劉伯莊撰。



 《
 漢書》一百十五卷班固作。



 又一百二十卷顏師古注。



 《御銓定漢書》八十一卷郝處俊等撰。



 《漢書音訓》一卷服虔撰。



 《漢書集解音義》二十四卷應劭撰。



 《漢書敘傳》五卷項岱撰。



 《漢書音義》九卷孟康撰。



 《漢書集注》十四卷晉灼注。



 《漢書音義》七卷章昭撰。



 《漢書駁義》二卷劉寶撰。



 《漢書新注》一卷陸澄撰。



 《孔氏漢書音義抄》二卷孔文詳撰。



 《漢書續訓》二卷章稜撰。



 《
 漢書訓纂》三十卷姚察撰。



 《漢書音義》二十六卷劉嗣等撰。



 《漢書音》二卷夏侯泳撰。



 又十二卷包愷撰。



 又十二卷蕭該撰。



 《漢書決疑》十二卷顏延年撰。



 《漢書古今集義》二十卷顧胤撰。



 《漢書正義》三十卷釋務靜撰。



 《漢書正名氏義》十三卷



 《漢書辯惑》三十卷李善撰。



 《漢書律歷志音義》一卷陰景倫作。



 《漢書英華》八卷



 《東觀漢記》一百二十七卷劉珍撰。



 《後漢書》一百三十三卷謝承撰。



 《後漢記》一百卷薛瑩作。



 《
 後漢書》八十三卷司馬彪撰。



 又五十八卷劉義慶撰。



 《後漢書》三十一卷華嶠作。



 又一百二卷謝沈撰。



 《後漢書外傳》十卷謝沈撰。



 《漢南紀》五十八卷張瑩撰。



 《後漢書》一百二卷袁山松作。



 又九十二卷範曄撰。



 《後漢書論贊》五卷範曄撰。



 《後漢書》五十八卷劉昭補注。



 又一百卷皇太子賢注。



 《後漢書音》三卷蕭該作。



 又三卷臧兢撰。



 《後漢書音義》二十七卷章機撰。



 《魏書》四十四卷王沈撰。



 《
 魏略》三十八卷魚豢撰。



 《魏國志》三十卷陳壽撰,裴松之注。



 《晉書》八十九卷王隱撰。



 又五十八卷虞預撰。



 又十四卷硃鳳撰。



 又三十五卷謝靈運撰。



 《晉中興書》八十卷何法盛撰。



 《晉書》一百一十卷臧榮緒撰。



 又九卷蕭子云撰。



 又一百三十卷許敬宗等撰。



 《宋書》四十二卷徐爰撰。



 又四十六卷孫嚴撰。



 又一百卷沈約撰。



 《後魏書》一百三十卷魏收撰。



 《後漢書》一百七卷魏澹撰。



 又一百卷張大素
 撰。



 《後周書》五十卷令狐德棻撰。



 《隋書》八十五卷魏徵等撰。



 又三十二卷張大素撰。



 《齊書》五十九卷蕭子顯撰。



 又八卷劉陟撰。



 《梁書》三十四卷謝昊、姚察等撰。



 又五十卷姚思廉撰。



 《陳書》三卷顧野王撰。



 又三卷傅縡撰。$又三十六卷姚思廉撰。



 《北齊未修書》二十四卷李德林撰。



 《北齊書》五十卷李百藥撰。



 又二十卷張大素撰。



 《通史》六百二卷梁武帝撰。



 《南史》八十卷李延壽撰。



 《北史》一百卷李延壽
 撰。



 右八十一部,《史記》六家,前漢二十五家,後漢十七家,魏三家,晉八家,宋三家,後魏三家,後周一家,隋二家,齊二家,梁二家,陳三家,北齊三家,都史三家,凡四千四百四十三卷。



 《紀年》十四卷汲塚書。



 《漢紀》三十卷荀悅撰。



 《漢紀音義》三卷崔浩撰。



 《漢皇德紀》三十卷侯瑾撰。



 《後漢紀》三十卷張璠撰。



 又三十卷袁宏撰。



 《漢晉春秋》五十四卷習鑿齒撰。



 《漢靈獻二帝紀》六卷劉艾撰。



 《
 漢獻帝春秋》十卷袁曄撰。



 《山陽義紀》樂資撰。



 《魏武本紀》三卷



 《魏武春秋》二十卷孫盛撰。



 《魏紀》十二卷魏澹撰。



 《國紀》十卷梁祚撰。



 《吳紀》十卷環濟撰。



 《晉帝紀》四卷陸機撰。



 《晉錄》五卷



 《晉紀》二十二卷干寶作。



 又六十卷干寶撰,劉協注。



 《晉陽秋》二十卷檀道鸞注。



 《晉紀》二十卷劉謙之撰。



 又十卷曹嘉之撰。



 又四十五卷徐廣撰。



 《晉陽春秋》二十二卷鄧粲撰。



 《
 晉史草》三十卷蕭景暢撰。



 《晉紀》十一卷鄧粲撰。



 《戰國春秋》二十卷李概撰。



 《崇安記》二卷周祗撰。



 又十卷王韶之撰。



 《晉續記》五卷郭季產撰。



 《三十國春秋》三十卷蕭方等撰。



 又一百卷武敏之撰。



 《晉春秋略》二十卷杜延業撰。



 《宋紀》三十卷王智深撰。



 《宋略》二十卷裴子野撰。



 《宋春秋》二十卷鮑衡卿撰。



 《齊紀》二十卷沈約撰。



 《齊春秋》三卷吳均撰。



 《乘輿龍飛記》二卷鮑衡卿撰。



 《梁典》三十卷劉璠撰。



 又三十卷何元之撰。



 《梁太清紀》十卷蕭韶撰。



 《皇帝紀》七卷



 《梁撮要》三十卷陰僧仁撰。



 《淮海亂離志》四卷蕭大圓撰。



 《棲鳳春秋》五卷臧嚴撰。



 《梁昭後略》十卷姚最撰。



 《天啟記》十卷守節先生撰。



 《梁末代記》一卷



 《後梁春秋》十卷蔡允恭撰。



 《北齊記》二十卷



 《北齊志》十七卷王劭撰。



 《鄴洛鼎峙記》十卷



 《隋大業略記》三卷趙毅撰。



 《隋後略》十卷張大素撰。



 《蜀國志》十五卷陳壽撰。



 《
 吳國志》二十一卷陳壽撰,裴松之注。



 《吳書》五十五卷章昭撰。



 《華陽國志》三卷常璩撰。



 《蜀李書》九卷常璩撰。



 《漢趙記》十卷和苞撰。



 《趙石記》二十卷田融撰。



 《二石記》二十卷田融撰。



 《二石偽事》六卷王度、隋翽等撰。



 《燕書》二十卷範亨撰。



 《秦記》十一卷裴景仁撰,杜惠明注。



 《涼記》十卷張諮撰。



 《西河記》二卷段龜龍撰。



 《南燕錄》六卷王景暄撰。



 《南燕書》五卷張銓撰。



 《拓跋涼錄》十卷



 《燕志》十卷



 《
 十六國春秋》一百二十卷崔鴻撰。



 右七十五部,編年五十五家,雜偽國史二十家,凡一千四百十卷。



 《周書》八卷孔晁注。



 《古文鎖語》四卷



 《春秋前傳》十卷何承天撰。



 《春秋前傳雜語》十卷何承天撰。



 《周載》三十卷孟儀注。



 《春秋國語》十卷孔衍撰。



 《越絕書》十六卷子貢撰。



 《吳越春秋》十二卷趙曄撰。



 《吳越春秋削煩》五卷楊方撰。



 《吳越春秋傳》十卷皇甫遵撰。



 《
 吳越記》六卷



 《春秋後傳》三十卷樂資撰。



 《戰國策》三十二卷劉向撰。



 《戰國策論》一卷延篤撰。



 《戰國策》三十二卷高誘注。



 《魯後春秋》二十卷劉允濟撰。



 《楚漢春秋》二十卷陸賈撰。



 《漢尚書》十卷孔衍撰。



 《漢春秋》十卷孔衍撰。



 《後漢尚書》六卷孔衍撰。



 《後漢春秋》六卷孔衍撰。



 《後漢尚書》十四卷孔衍撰。



 《後魏春秋》九卷孔衍撰。



 《典略》五十卷魚豢撰。



 《三史要略》三十卷張溫撰。



 《正史削繁》十四卷阮孝緒撰。



 《
 東殿新書》二百卷高宗大帝撰。



 《史記要傳》十卷衛颯撰。



 《古史考》二十五卷譙周撰。



 《史記正傳》九卷張瑩撰。



 《史要》三十八卷王延秀撰。



 《合史》二十卷



 《史漢要集》二卷王蔑撰。



 《後漢書抄》三十卷葛洪撰。



 《後漢書略》二十五卷張緬撰。



 《後漢書纘》十三卷範曄撰。



 《後漢文武釋論》二十卷王越客撰。



 《三國評》三卷徐眾撰。



 《晉書鈔》三十卷張緬撰。



 《代譜》四百八十卷周武帝敕撰。



 《漢末英雄記》十卷王粲等撰。



 《九州春秋》九卷司馬彪撰。



 《
 魏陽秋異同》八卷孫壽撰。



 《魏武本紀年歷》五卷



 《漢表》十卷袁希之撰。



 《刪補蜀記》七卷王隱撰。



 《吳錄》三十卷張勃撰。



 《魏記》三十三卷盧彥卿撰。



 《關東風俗傳》六十三卷宋孝王撰。



 《隋書》八十卷王劭撰。



 《王業歷》二卷趙弘禮撰。



 《隋開業平陳記》十二卷裴矩撰



 《古今注》八卷伏無忌撰。



 《帝王本紀》十卷來奧撰。



 《拾遺錄》三卷王嘉撰。



 《王子年拾遺記》十卷蕭綺錄。



 《帝王略要》十二卷環濟撰。



 《
 先聖本紀》十卷劉滔撰。



 《華夷帝王記》三十七卷楊曄撰。



 《後漢雜事》十卷



 《漢魏晉帝要記》三卷賈匪之撰。



 《魏晉代語》十卷郭頒撰。



 《吳朝人士品秩狀》八卷胡沖撰。



 《吳士人行狀名品》二卷虞尚撰。



 《江表傳》五卷虞溥撰。



 《晉諸公贊》二十二卷傅暢撰。



 《晉後略記》五卷荀綽撰。



 《宋拾遺錄》十卷謝綽撰。



 《宋齊語錄》十卷孔思尚撰。



 《帝王略論》五卷虞世南撰。



 《十世興王論》
 十卷硃敬則撰。



 《洞歷記》九卷周樹撰。



 《帝系譜》二卷張愔等撰。



 《洞記》九卷韋昭撰。



 《三五歷記》二卷徐整撰。



 《通歷》二卷徐整撰。



 《雜歷》五卷徐整撰。



 《國志歷》五卷孔衍撰。



 《帝王代記》十卷皇甫謐撰。



 《年歷》六卷皇甫謐撰。



 《續帝王代記》十卷何集撰。



 《十五代略》十卷吉文甫撰。



 《吳歷》六卷胡沖撰。



 《晉歷》二卷



 《帝王代紀》十六卷



 《年歷帝紀》二十六卷姚恭撰。



 《帝錄》十卷諸葛忱撰。



 《
 長歷》十四卷



 《歷代記》三十卷庾和之撰。



 《千年歷》二卷



 《千歲歷》三卷許氏作。



 《十代記》十卷熊襄撰。



 《帝王年歷》五卷陶弘景撰。



 《分王年表》八卷羊瑗撰。



 《歷紀》十卷



 《通歷》七卷李仁實撰。



 《帝王編年錄》五十一卷盧元福撰。



 《共和已來甲乙紀年》二卷盧元福撰。



 《帝王紀錄》三卷



 右雜史一百二部,凡二千五百五十九卷。



 《穆天子傳》六卷郭璞撰。



 《漢獻帝起居注》五卷



 《晉太始起居注》二十卷李軌撰。



 《晉愍帝起居注》三十卷李軌撰。



 《晉太康起居注》二十二卷李軌撰。



 《晉永平起居注》八卷李軌撰



 《晉建武大興永昌起居注》二十二卷



 《晉咸和起居注》十八卷李軌撰。



 《
 晉咸康起居注》二十二卷李軌撰。



 《晉建元起居注》四卷



 《晉永和起居注》二十四卷



 《晉升平起居注》十卷



 《晉崇和興寧起居注》五卷



 《晉太和起居注》六卷



 《晉咸安起居注》三卷



 《晉寧康起居注》六卷



 《晉太元起居注》五十二卷



 《晉崇安起居注》十卷



 《晉元興起居注》九卷



 《晉義熙起居注》三十四卷



 《晉元熙起居注》二卷



 《晉起居注》三百二十卷劉道會撰。



 《宋永初起居注》六卷



 《
 宋景平起居注》三卷



 《宋元嘉起居注》六十卷



 《宋大明起居注》八卷



 《梁皇帝實錄》三卷周興嗣撰。



 又五卷



 《梁太清實錄》八卷



 《後魏起居注》二百七十六卷



 《陳起居注》四十一卷



 《太唐創業起居注》三卷溫大雅撰。



 《高祖實錄》二十卷房玄齡撰。



 《太宗實錄》二十卷房玄齡撰。



 《太宗實錄》四十卷長孫無忌撰。



 《高宗實錄》三十卷許敬宗撰。



 《述聖記》一卷大聖天后撰。



 《
 高宗實錄》一百卷大聖天后撰。



 《聖母神皇實錄》十八卷宗秦客撰。



 《中宗皇帝實錄》二十卷吳兢撰。



 《漢武故事》二卷



 《西京雜記》一卷葛洪撰。



 《三輔舊事》一卷韋氏撰。



 《秦漢已來舊事》八卷



 《晉建武已來故事》三卷



 《漢魏吳蜀舊事》八卷



 《晉書雜詔書》一百卷



 又二十八卷



 《晉雜詔書》六十六卷



 《晉詔書黃素制》五卷



 《晉定品制》一卷



 《晉太元副詔》二十一卷



 《
 晉崇安元興大亨副詔》八卷



 《晉義熙詔》二十二卷



 《晉故事》四十三卷



 《晉諸雜故事》二十二卷



 《尚書大事》二十一卷



 《晉太始太康故事》五卷



 《晉建武咸和咸康故事》四卷孔愉撰。



 《晉建武以來故事》三卷



 《修復山林故事》五卷車灌撰。



 《先朝故事》二十卷劉道會撰。



 《東宮舊事》十一卷張敞撰。



 《交州雜故事》九卷



 《四王起事》四卷盧綝
 撰。



 《晉八王故事》十二卷盧綝撰。



 《晉故事》三卷



 《晉朝雜事》二卷



 《江南故事》三卷



 《大司馬陶公故事》三卷



 《郗太尉為尚書令故事》二卷



 《桓公偽事》二卷應德詹撰。



 《救襄陽上都督府事》一卷王愆期撰。



 《荊江揚州遷代記》四卷



 《宋永初詔》六卷



 《宋元嘉詔》二十一卷



 《晉宋舊事》一百三十卷



 《
 中興伐逆事》二卷



 《東宮儀記》二十二卷張鏡撰。



 《東宮典記》七十卷宇文愷等撰。



 《春坊要錄》四卷杜正倫撰。



 《春坊舊事》三卷



 《漢官儀》十卷應劭志。



 《公卿故事》二卷王方慶撰。



 《漢官解故》三卷



 《魏官儀》一卷荀攸撰。



 《晉公卿禮秩》九卷傅暢撰。



 《百官名》四十卷



 《晉惠帝百官名》三卷陸機撰。



 《晉官屬名》四卷



 《晉過江人士目》一卷



 《晉永嘉流士》十三卷衛禹撰。



 《登城三戰簿》三卷



 《
 百官階次》一卷範曄撰。



 《宋百官階次》三卷荀欽明撰。



 《百官春秋》十三卷王道秀撰。



 《齊職儀》五十卷範曄撰。



 《職官要錄》三十卷陶藻撰。



 《梁選簿》三卷徐勉撰。



 《陳將軍簿》一卷



 《職令百官古今注》十卷郭演之撰。



 《太建十一年百官簿狀》二卷



 《職員舊事》三十卷



 右一百四部,列代起居注四十一家,列代故事
 四十二家,列代職官二十一家,凡二千二百三十三卷。



 《三輔決錄》七卷趙岐撰,摯虞注。



 《海內先賢傳》四卷魏明帝撰。



 《海內先賢行狀》三卷李氏撰。



 《海內士品錄》二卷魏文帝撰。



 《四海耆舊傳》一卷李氏撰。



 《盧江七賢傳》一卷



 《陳留耆舊傳》三卷蘇林撰。



 《陳留先賢像贊》一卷陳英宗撰。



 《陳留志》十五卷江敞撰。



 《汝南先賢傳》三卷周裴撰。



 《廣州先賢傳》七卷陸胤撰。



 《諸國先賢傳》一卷



 《
 豫章舊志》八卷徐整撰。



 《濟北先賢傳》一卷



 《廣陵列士傳》一卷華隔撰。



 《桂陽先賢畫贊》五卷張勝撰。



 《會稽記》四卷硃育撰。



 《會稽典錄》二十四卷虞預撰。



 《會稽先賢傳》五卷謝承撰。



 《會稽後賢傳》三卷鐘離岫撰。



 《會稽先賢像贊》四卷賀氏撰。



 《會稽太守像贊》二卷賀氏撰。



 《吳國先賢贊》三卷



 《益部耆舊傳》十四卷陳壽撰。



 《魯國先賢志》十四卷白褒撰。



 《楚國先賢志》十二卷楊方撰。



 《荊州先賢傳》三卷高範撰。



 《
 兗州山陽先賢贊》一卷仲長統撰。



 《交州先賢傳》四卷範瑗撰。



 《襄陽耆舊傳》五卷習鑿齒撰。



 《零陵先賢傳》一卷



 《徐州先賢傳》一卷



 《長沙舊邦傳贊》三卷劉彧撰。



 《徐州先賢傳》九卷



 《燉煌實錄》二十卷劉延明撰。



 《武昌先賢傳》三卷郭緣生撰。



 《海岱志》十卷崔蔚祖撰。



 《吳郡錢塘先賢傳》五卷吳均撰。



 《幽州古今人物志》十三卷陽休之撰。



 《孝子傳》十五卷蕭廣濟撰。



 又八卷師覺授撰。



 《
 孝子傳贊》十五卷王韶之撰。



 《孝子傳》十卷宗躬撰。



 《雜孝子傳》二卷



 《孝子傳》一卷虞盤佐撰。



 又三卷徐廣撰。



 《孝子傳贊》十卷鄭緝之撰。



 《孝德傳》三十卷梁元帝撰。



 《孝友傳》八卷梁元帝撰。



 《忠臣傳》三十卷梁元帝撰。



 《顯忠錄》二十卷元懌撰。



 《忠孝圖傳贊》二十卷李襲譽撰。



 《英籓可錄事》二卷殷系撰。



 《自古諸侯王善惡錄》二卷魏徵撰。



 《列籓正論》三十卷章懷太子撰



 《良吏傳》十卷鐘岏
 撰。



 《丹陽尹傳》十卷梁元帝撰。



 《高士傳》三卷嵇康撰。



 《上古以來聖賢高士傳贊》三卷周續之撰。



 《高士傳》七卷皇甫謐撰。



 《續高士傳》八卷周弘讓撰。



 《逸人傳》三卷張顯撰。



 《逸人高士傳》八卷習鑿齒撰。



 《名士傳》三卷袁宏撰。



 《竹林七賢論》二卷戴逵撰。



 《真隱傳》二卷袁淑撰。



 《高士傳》二卷虞盤佐撰。



 《高隱傳》二卷阮孝緒撰。



 《七賢傳》七卷孟仲暉撰。



 《高才不遇傳》四卷劉晝撰。



 《列女傳》二卷劉向撰。



 《
 陰德傳》二卷範晏撰。



 《止足傳》十卷王子良撰。



 《同姓名錄》一卷梁元帝撰。



 《全德志》一卷梁元帝撰。



 《高僧傳》六卷虞孝敬撰。



 《悼善列傳》四卷



 《幼童傳》十卷劉昭撰。



 《知己傳》一卷盧思道撰。



 《交游傳》二卷鄭世翼撰。



 《秘錄》二百七十卷元暉等撰。



 《畫贊》五十卷漢明帝撰。



 《春秋列國名臣傳》九卷孫敏撰。



 《四科傳贊》四卷姚澹撰。



 《七國敘贊》十卷



 《益州文翁學堂圖》一
 卷



 《孔子弟子傳》五卷



 《先儒傳》五卷



 《雜傳》六十五卷



 又九卷



 又四十卷



 《集記》一百卷王孝恭撰。



 《東方朔傳》八卷



 《李固別傳》七卷



 《梁冀傳》二卷



 《何顒傳》一卷



 《曹瞞傳》一卷吳人作。



 《毋丘儉記》三卷



 《管輅傳》二卷管辰撰。



 《諸葛亮隱沒五事》一卷郭沖撰。



 《玄晏春秋》二卷皇甫謐撰。



 《
 薛常侍傳》二卷荀伯子撰。



 《桓玄傳》二卷



 《文林館記》十卷鄭忱撰。



 《文士傳》五十卷張騭撰。



 《文館詞林文人傳》一百卷許敬宗撰。



 《列仙傳贊》二卷劉向撰。



 《神仙傳》十卷葛洪撰。



 《洞仙傳》十卷見素子撰。



 《高士老君內傳》三卷尹喜、張林亭撰。



 《老子傳》一卷



 《關令尹喜傳》一卷鬼谷先生撰,四皓注。



 《王喬傳》一卷



 《茅君內傳》一卷



 《漢武帝傳》二卷



 《
 清虛真人王君內傳》一卷



 《蘇君記》一卷周季通撰。



 《靈人辛玄子自序》一卷辛玄子撰。



 《三天法師張君內傳》一卷王萇撰。



 《太極左仙公葛君內傳》一卷呂先生注。



 《紫陽真人周君傳》一卷華嶠撰。



 《仙人馬君陰君內傳》一卷趙升撰。



 《清虛真人裴君內傳》一卷鄭子云撰。



 《紫虛元君南嶽夫人內傳》一卷
 範邈撰。



 《九華真妃內記》一卷。



 《許先生傳》一卷王羲之撰。



 《養性傳》二卷



 《周氏冥通記》一卷陶弘景撰。



 《學道傳》二十卷馬樞撰。



 《嵩高少室寇天師傳》三卷宋都能撰。



 《華陽子自序》一卷茅處玄撰。



 《漢別國洞冥記》四卷郭憲撰。



 《名僧傳》三十卷釋寶唱撰。



 《比邱尼傳》四卷釋寶唱撰。



 《高僧傳》十四卷釋惠皎撰。



 《續高僧傳》二十卷釋道宣撰。



 《續高僧傳》三十卷釋道宜撰。



 《
 西域求法高僧傳》二卷釋義凈撰。



 《名僧錄》十五卷裴子野撰。



 《薩婆多部傳》四卷釋僧佑撰。



 《草堂法師傳》一卷陶弘景撰。



 又一卷蕭理撰。



 《稠禪師傳》一卷



 《列異傳》三卷張華撰。



 《甄異傳》三卷戴祚撰。



 《征應集》二卷



 《雜傳》十卷



 《搜神記》三十卷干寶撰。



 《志怪》四卷祖臺之撰。



 又四卷孔氏撰。



 《靈鬼志》三卷荀氏撰。



 《鬼神列傳》二卷謝氏撰。



 《幽明錄》三十卷劉義慶撰。



 《
 齊諧記》七卷東陽無疑撰。



 《續齊諧記》一卷吳均撰。



 《古異傳》三卷袁仁壽撰。



 《述異記》十卷祖沖之撰。



 《感應傳》八卷王延秀撰。



 《冥祥記》十卷王琰撰。



 《續冥祥記》十一卷王曼穎撰。



 《系應驗記》一卷陸果撰。



 《神錄》五卷劉之遴撰。



 《妍神記》十卷梁元帝撰。



 《因果記》十卷劉泳撰



 《近異錄》二卷劉質撰



 《冤魂志》三卷顏之推撰。



 《集靈記》十卷顏之推撰。



 《旌異記》十五卷侯君素撰。



 《冥報記》二卷唐臨撰。



 《
 列女傳》六卷皇甫謐撰。



 《列女後傳》十卷顏原撰。



 《列女傳》七卷綦毋邃撰。



 《女記》十卷杜預撰。



 《列女傳序贊》一卷孫夫人撰。



 《後妃記》四卷虞通之撰。



 《列女傳》一百卷大聖天后撰。



 《古今內範記》一百卷



 《內範要略》十卷



 《保傅乳母傳》一卷大聖天后撰。



 右雜傳一百九十四部,褒先賢耆舊三十九家,孝友十家,忠節三家,列籓三家,良史二家,高逸十八家,雜傳五家,科錄一家,雜傳十一家,文士
 三家,仙靈二十六家,高僧十家,鬼神二十六家,列女十六家,凡一千九百七十八卷。



 《漢舊儀》四卷衛宏撰。



 《輿服志》一卷董巴撰。



 《晉尚書儀曹新定儀注》四十一卷徐廣撰。



 《甲辰儀注》五卷



 《車服雜注》一卷徐廣撰。



 《司徒儀注》五卷干寶撰。



 《大駕鹵簿》一卷



 《冠婚儀》四卷



 《晉雜儀注》二十一卷



 《晉儀注》三十九卷



 《諸王國雜儀》十卷



 《
 宋儀注》三十六卷



 《雜儀注》一百八卷



 《雜府州郡儀》十卷範汪撰。



 《晉尚書儀曹吉禮儀注》三卷



 《古今輿服雜事》十卷周遷撰。



 《梁祭地祇陰陽儀注》二卷沈約撰。



 《宋儀注》二卷



 《梁吉禮》十八卷明山賓等撰。



 《梁吉禮儀注》十卷



 《北齊吉禮》七十二卷趙彥深撰。



 《陳吉禮儀注》五十卷雜
 撰。



 《梁皇帝崩兇儀》十一卷嚴植之撰。



 《隋吉禮》五十四卷高熲等撰。



 《梁兇禮天子喪禮》五卷嚴植之撰。



 《梁兇禮天子喪禮》七卷



 《梁王侯已下兇禮》九卷嚴植之撰。



 《梁太子妃薨兇儀注》九卷



 《北齊王太子喪禮》十卷趙彥深撰。



 《梁諸侯世子兇儀注》九卷



 《梁賓禮》一卷賀易等撰。



 《隋書禮》七卷高瑒等撰。



 《梁嘉禮》三十五卷司馬蒨撰。



 《陳賓禮儀注》六卷張彥志。



 《梁軍禮》四卷陸璉撰。



 《梁嘉禮儀注》二十一卷司馬蒨撰。



 《梁尚書儀注》十八卷雜
 撰。



 《梁儀注》十卷沈約撰。



 《梁陳大行皇帝崩儀注》八卷



 《陳尚書曹儀注》二十卷雜志。



 《陳諸帝后崩儀注》五卷



 《陳雜吉儀志》三十卷



 《梁大行皇后崩儀注》一卷



 《陳皇太子妃薨儀注》五卷儀曹志。



 《陳雜儀注兇儀》十三卷



 《陳皇太后崩儀注》四卷儀曹撰



 《陳雜儀注》六卷



 《後魏儀注》三十二卷常景撰。



 《理禮儀注》九卷何點撰。



 《
 晉謚議》八卷



 《魏明帝謚議》二卷何晏撰。



 《魏氏郊丘》三卷



 《晉簡文謚議》四卷



 《晉明堂郊社議》三卷孔朝等撰。



 《魏臺雜訪議》三卷高堂隆撰。



 《雜議》五卷干寶撰。



 《晉七廟議》三卷蔡謨撰



 《要典》三十九卷王景之撰。



 《晉雜議》十卷荀翽等撰。



 《皇典》五卷丘孝仲撰。



 《齊典》四卷王逸志。



 《吊答書儀》十卷王儉撰。



 《太宗文皇帝政典》三卷李延壽撰。



 《雜儀》三十卷鮑昶撰。



 《
 書筆儀》二十卷謝撰。



 《婦人書儀》八卷唐瑾撰。



 《皇室書儀》十三卷鮑行卿撰。



 《大唐書儀》十卷裴矩撰。



 《童悟》十三卷



 《封禪錄》十卷孟利貞撰。



 《皇帝封禪儀》六卷令狐德棻撰。



 《玉璽譜》一卷僧約貞撰。



 《神嶽封禪儀注》十卷裴守貞撰。



 《玉璽正錄》一卷徐令信撰。



 《傳國璽》十卷姚察撰。



 《大享明堂儀注》二卷郭山惲撰。



 《明堂義》一卷張大瓚撰。



 《明堂儀注》七卷姚璠等撰。



 《親享太廟儀》三卷郭山惲撰。



 《皇太子方岳亞獻儀》二卷



 右儀注八十四部,凡一千一百四十六卷。



 《漢建武律令故事》三卷



 《律略論》五卷劉邵撰。



 《漢朝駁義》三十卷應劭撰。



 《漢名臣奏》三十卷陳壽撰。



 又二十九卷



 《廷尉決事》二十卷



 《廷尉駁事》十一卷



 《廷尉雜詔書》二十六卷



 《晉令》四十卷賈充等撰。



 《刑法律本》二十一卷賈充等撰。



 《南臺奏事》二十二卷



 《晉駁事》四卷



 《晉彈事》九卷



 《齊永明律》八卷宋躬撰。



 《
 梁律》二十卷蔡法度撰。



 《梁令》三十卷蔡法度撰。



 《梁科》二卷蔡法度撰。



 《陳令》三十卷範泉等撰。



 《陳科》三十卷範泉志。



 《北齊律》二十卷趙郡王睿撰。



 《北齊令》八卷



 《周大律》二十五卷趙肅等撰。



 《隋律》十二卷高熲等撰。



 《隋大業律》十八卷



 《隋開皇令》三十卷裴正等撰。



 《法例》二卷崔知悌等撰。



 《令律》十二卷裴寂撰。



 《律疏》三十卷長孫無忌撰。



 《武德令》三十一卷裴寂等撰。



 《貞觀格》十八卷房玄齡
 撰。



 《永徽散行天下格中本》七卷



 永徽留本司行中本》十八卷源直心等撰。



 《永徽令》三十卷



 《永徽留本司格後本》十一卷劉仁軌撰。



 《永徽成式》十四卷



 《永徽散頒天下格》七卷



 《永徽留本司行格》十八卷長孫無忌撰。



 《永徽中式本》四卷



 《垂拱式》二十卷



 《垂拱格》二卷



 《垂拱留司格》六卷裴居道撰。



 《
 律解》二十一卷張斐撰。



 《開元前格》十卷劉崇等撰。



 《開元後格》九卷宋璟等撰。



 《令》三十卷



 《式》二十卷姚崇等撰。



 右刑法五十一部,凡八百一十四卷。



 《七略別錄》二十卷劉向撰。



 《七略》七卷劉歆撰。



 《今書七志》七十卷王儉撰,賀縱補。



 《七錄》十二卷阮孝緒撰。



 《中書簿》十四卷荀勖撰。



 《元徽元年書目》四卷王儉撰。



 《梁天監四年書目》四卷丘賓卿撰。



 《陳天嘉四部書目》四卷



 《
 隋開皇四年書目》四卷牛弘撰。



 《隋開皇二十年書目》四卷王邵撰



 《史目》三卷楊松珍撰。



 《文章志》四卷摯虞撰。



 《新撰文章家集》五卷荀勖撰。



 《續文章志》二卷傅亮撰。



 《義熙已來雜集目錄》三卷丘深之撰。



 《名手畫錄》一卷



 《法書目錄》六卷虞和撰。



 《群書四錄》二百卷元行沖撰。



 右雜四部書目十八部,凡二百一十七卷。



 《
 世本》四卷宋衷撰。



 《世本別錄》一卷



 《帝譜世本》七卷宋均撰。



 《世本譜》三卷



 《漢氏帝王譜》二卷



 《司馬氏世家》二卷



 《百家集譜》十卷王儉撰。



 《百家譜》三十卷王僧孺撰。



 《氏族要狀》十五卷賈希景撰。



 《永元中表簿》六卷



 《姓氏英賢譜》一百卷賈執撰。



 《百家譜》五卷賈執撰。



 《國親皇太子傳》四卷賈冠撰。



 《大同四年中表簿》三卷



 《齊梁宗簿》三卷



 《後魏辯宗錄》二卷元暉業撰。



 《
 姓苑》十卷何承天撰。



 《後魏譜》二卷



 《後魏方司格》一卷



 《十八州譜》七百一十二卷王僧孺撰。



 《冀州譜》七卷



 《洪州譜》九卷



 《袁州譜》七卷



 《大唐氏族志》一百卷高士廉撰。



 《姓氏譜》二百卷許敬宗撰。



 《著姓略記》十卷路敬淳撰。



 《衣冠譜》六十卷路敬淳撰。



 《大唐姓族系錄》二百卷柳沖撰。



 《
 褚氏家傳》一卷褚結撰,褚陶注。



 《殷氏家傳》三卷殷敬等撰。



 《桂氏世傳》七卷桂顏撰。



 《邵氏家傳》十卷



 《楊氏譜》一卷



 《蘇氏譜》一卷



 《韋氏家傳》三卷皇甫謐撰。



 《王氏家傳》二十一卷



 《江氏家傳》七卷江統撰。



 《暨氏家傳》一卷



 《虞氏家傳》五卷虞覽撰。



 《裴氏家記》三卷裴松之撰。



 《孫氏譜記》十五卷



 《諸葛傳》五卷



 《曹氏家傳》一卷曹毗撰。



 《荀氏家傳》十卷荀伯子撰。



 《
 諸王傳》一卷



 《陸史》十五卷陸煦撰。



 《明氏世錄》五卷明粲撰。



 《庾氏家傳》三卷庾守業撰。



 《韋氏譜》十卷韋鼎等撰。



 《爾硃氏家傳》二卷王邵撰。



 《何妥家傳》二卷



 《令狐家傳》一卷令狐德撰。



 《裴若弼家傳》一卷



 《燉煌張氏家傳》二十卷張太素撰。



 《裴氏家牒》二十卷裴守貞撰。



 右雜譜牒五十五部,凡一千六百九十一卷。



 《山海經》十八卷郭璞撰。



 《山海經圖贊》二卷郭璞
 撰。



 《山海經音》二卷



 《水經》二卷郭璞撰。



 又四十卷酈道元注。



 《三輔黃圖》一卷



 《漢宮閣簿》三卷



 《洛陽宮殿簿》三卷



 《關中記》一卷潘岳撰。



 《洛陽記》一卷陸機撰。



 《西京雜記》一卷葛洪撰。



 《洛陽圖》一卷楊佺期撰。



 《洛陽記》一卷戴延之撰。



 《廟記》一卷



 《洛陽伽藍記》五卷陽衒之撰。



 《西京記》三卷薛冥志。



 《東都記》三十卷鄧行儼撰。



 《分吳會丹陽三郡記》三卷



 《
 陳留風俗傳》三卷圈稱撰。



 《風土記》十卷周處撰。



 《吳地記》一卷張勃撰。



 《南雍州記》三卷郭仲彥撰。



 《南徐州記》二卷山謙之撰。



 《東陽記》一卷鄭緝之撰。



 《京口記》二卷劉損之撰。



 《湘州圖記》一卷



 《徐地錄》一卷劉芳撰。



 《齊州記》四卷李叔布撰。



 《中岳潁川志》五卷樊文深撰。



 《潤州圖經》二十卷孫處玄撰。



 《地記》五卷太康三年撰。



 《州郡縣名》五卷太康三年撰。



 《十三州志》十四卷闞駰撰。



 《魏諸州記》二十卷



 《
 地理書》一百五十卷陸澄撰。



 《地記》二百五十二卷任昉撰。



 《雜志記》十二卷



 《雜地記》五卷



 《國郡城記》九卷周明帝撰。



 《輿地志》三十卷顧野王撰。



 《周地圖》九十卷



 《隋圖經集記》一百卷郎蔚之撰。



 《區宇圖》一百二十八卷虞茂撰。



 《括地志序略》五卷魏王泰撰。



 《交州異物志》一卷楊孚撰。



 《暢異物志》一卷陳祈撰。



 《南州異物志》一卷萬震撰。



 《扶南異物志》一卷硃應撰。



 《臨海水土異物志》一卷沈瑩撰。



 《江記》五卷庾仲雍撰。



 《
 漢水記》五卷庾仲雍撰。



 《尋江源記》五卷庾仲雍撰。



 又一卷



 《四海百川水記》一卷釋道安撰。



 《西征記》一卷戴祚撰。



 《述征記》二卷郭緣生撰。



 《隋王入沔記》十卷沈懷文撰。



 《輿駕東幸記》一卷薛泰撰。



 《述行記》二卷姚最撰。



 《魏聘使行記》五卷



 《巡總揚州記》七卷諸葛穎撰。



 《諸郡土俗物產記》十九卷



 《京兆郡方物志》三十卷



 《十洲記》一卷東方朔撰。



 《神異經》二卷東方朔撰。



 《蜀王本紀》一卷楊雄撰。



 《
 三巴記》一卷譙周撰。



 《外國傳》一卷釋智猛撰。



 《歷國傳》二卷釋法盛撰。



 《南越志》五卷沈懷遠撰。



 《日南傳》一卷



 《職貢圖》一卷梁元帝撰。



 《林邑國記》一卷



 《真臘國事》一卷



 《魏國己西十一國事》一卷宋云撰。



 《交州已來外國傳》一卷



 《奉使高麗記》一卷



 《西域道里記》三卷



 《赤土國記》二卷常駿等撰。



 《高麗風俗》一卷裴矩撰。



 《中天竺國行記》十卷王玄策撰。



 《
 西南蠻入朝首領記》一卷



 《職方記》十六卷



 《長安四年十道圖》十三卷



 《開元三年十道圖》十卷



 《劍南地圖》二卷



 右地理九十三部,凡一千七百八十二卷。



\end{pinyinscope}