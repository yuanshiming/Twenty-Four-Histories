\article{卷一百二}

\begin{pinyinscope}

 ○魏知古盧懷慎子奐源乾曜從孫光裕光裕子洧李元紘杜暹韓休裴耀卿孫佶



 魏知古,深州陸澤人也。性方直,早有才名。弱冠舉進士,累授著作郎,兼修國史。長安中,歷遷鳳閣舍人、衛尉少
 卿。時睿宗居籓,兼檢校相王府司馬。神龍初,擢拜吏部侍郎,仍並依舊兼修國史,尋進位銀青光祿大夫。明年,丁母憂去職,服闋授晉州刺史。睿宗即位,以故吏召拜黃門侍郎,兼修國史。



 景雲二年,遷右散騎常侍。睿宗女金仙、玉真二公主入道,有制各造一觀,雖屬季夏盛暑,尚營作不止。知古上疏諫曰:



 臣聞《穀梁傳》曰:「古之君人者,必時視人之所勤:人勤於力則功築罕,人勤於財則貢賦少,人勤於食則百事廢。」《書》曰:「不作無益害有益。」又
 曰:「罔咈百姓以從己之欲。」《禮》曰:「季夏之月,樹木方盛,無有斬伐,不可興土功以妨農。」又曰:「季夏行冬令,則風寒不時。」《語》曰:「修己以安百姓。」此皆興化立理之教,為政養人之本。今陛下為公主造觀,將樹功德以祈福祐。但兩觀之地,皆百姓之宅,卒然迫逼,令其轉移,扶老攜幼,投竄無所,發剔椽瓦,呼嗟道路。乖人事,違天時,起無用之作,崇不急之務,群心搖搖,眾口籍籍。陛下為人父母,欲何以安之?且國有簡冊,君舉必記,動則左史書之,言則
 右史書之。是以非禮勿言,非禮勿動。夫如是,則君之所舉,可不慎歟!微臣備位諫諍,兼秉史筆,書而不法,後嗣何觀?臣愚必以為不可。伏願俯順人欲,仰稽天意,降德音,下明策,速罷功役,收之桑榆。



 疏奏不納。



 頃之,又進諫曰:「臣聞人以君為天,君以人為本。人安則政理,本固則邦寧。自陛下翦除兇逆,君臨寶位,蒼生顒顒,以為朝有新政。今風教頹替,日甚一日,府庫空虛,人力凋弊,造作不息,官員日增。今諸司試及員外、檢校等官,僅至二千
 餘人,太府之布帛以殫,太倉之米粟難給。又金仙、玉真等觀造作,咸非急務,臣先奏請停,竟仍未止。今歲前水後旱,五穀不熟,若至來春,必甚饑饉。陛下為人父母,欲何方以賑恤?療饑拯溺,須及其時。又突厥為患,其來自久,本無禮儀,焉有誠信。今雖遣使,來請結婚,豺狼之心,首鼠何定。弱則卑順,強則驕逆。屬草衰月滿,弓勁馬肥,乘中國饑虛,在和親際會,倘或窺犯亭障,國家何以防之?臣所論者,事甚急切,伏願特垂詳察。」睿宗嘉其切直,
 尋令同中書門下平章事。玄宗在春宮,又令兼左庶子。未幾,遷戶部尚書,餘如故。明年,擢拜侍中。



 先天元年冬,從上畋獵於渭川,因獻詩諷曰:「嘗聞夏太康,五弟訓禽荒。我後來冬狩,三驅盛禮張。順時鷹隼擊,講事武功揚。奔走未及去,翾飛豈暇翔。非熊從渭水,瑞雀想陳倉。此欲誠難縱,茲游不可常。子云陳《羽獵》,僖伯諫漁棠。得失鑒齊、楚,仁恩念禹、湯。邕熙諒在宥,亭毒匪多傷。《辛甲》今為史,《虞箴》遂孔彰。」手制褒之曰:「夫詩者,志之所以,寫其
 心懷,實可諷諭君主。是故揚雄陳《羽獵》,馬卿賦《上林》,爰自《風雅》,率由茲道。予頃向溫泉,觀省風俗,時因暇景,掩渭而畋,方開一面之羅,式展三驅之禮,躬親校獵,聊以從禽。豈意卿有箴規,輔予不逮,自非款誠夙著,其孰能繼於此耶?今賜卿物五十段,用申勸獎。」



 二年,累封梁國公。竇懷貞等將謀逆也,知古獨密奏其事。及懷貞誅,賜實封二百戶、物五百段。仍以前賞猶薄,又手敕曰:「魏知古去年十月已前,屢申啟沃,每竭忠誠,奸臣有謀,預奏
 其兆。事君之節,良有可嘉,可更賜實封一百戶。」其年冬,令往東都知吏部尚書事,深以為稱職,手制曰:「卿以宰臣,往知大選,官人之委,情寄尤切。遂能端本革弊,忘私徇公,正色而行,厝心不撓。鏡已澈則妍媸必鑒,衡已舉則輕重罔違。朕遠聞之,益用嘉嘆。今賜卿衣裳一副,以示所懷。」



 開元元年,官名改易,改為黃門監。二年,還京,上屢有顧問,恩意甚厚,尋改紫微令。姚崇深忌憚之,陰加讒毀,乃除工部尚書,罷知政事。三年卒,時年六十九。御
 史大夫宋璟聞而嘆曰:「叔向古之遺直,子產古之遺愛,能兼之者,其在魏公。」贈幽州都督,謚曰忠。



 知古初為黃門侍郎,表薦洹水令呂太一、蒲州司功參軍齊璟、前右內率府騎曹參軍柳澤。及知吏部尚書事,又擢用密縣尉宋遙、左補闕袁暉、右補闕封希顏、伊闕尉陳希烈,後咸累居清要,時論以為有知人之鑒。文集七卷。



 盧懷慎,滑州靈昌人。其先家於範陽,為山東著姓。祖悊,為靈昌令,因徙焉。懷慎少清謹,舉進士,歷監察御史、吏
 部員外郎。景龍中,遷右御史臺中丞,上疏以陳時政得失。今略載其三篇。



 其一曰:



 臣聞孔子曰:「為邦百年,可以勝殘去殺。」又曰:「茍有用我者,期月而已,三年有成。」故《書》云「三載考績」,校其功也。昔子產相鄭,更法令,布刑書,一年而人歌之曰:「取我田疇而伍之,取我衣冠而褚之,孰殺子產,吾其與之!」二年而人又歌之曰:「我有子弟,子產教之,我有田疇,子產殖之,子產而死,誰其嗣之?」終有遺愛,流芳史策。子產,賢者也,其為政尚累年而化成,況其
 常材乎。



 臣竊見比來州牧、上佐及兩畿縣令,下車布政,罕終四考。在任多者一二年,少者三五月,遽即遷除,不論課最。或有歷時未改,便傾耳而聽,企踵而望,爭求冒進,不顧廉恥。亦何暇為陛下宣風布化,求瘼恤人哉!禮義未能興行,風俗未能齊一,戶口所以流散,倉庫所以空虛,百姓凋弊,日更滋甚,職為此也。何則?人知吏之不久,則不從其教;吏知遷之不遙,又不盡其力,偷安爵祿,但養資望。陛下雖勤勞之懷,宵衣旰食,然僥幸路啟,上
 下相蒙,共為茍且而已,寧盡至公乎?此國之病也。昔賈誼所謂蹠盭之病,乃小小者耳。此弊久而不革,臣恐為膏肓,雖和、緩不能療,豈蹠盭而已哉!



 漢宣帝綜核名實,興理致化。黃霸,良二千石也,就增秩賜金,以旌其能,而不遷於潁川,前代之美政也。又古之為吏者長子孫,倉氏、瘐氏,即其後也。《書》云:「事不師古,以克永代,匪說攸聞。」臣望請諸州都督、刺史、上佐及兩畿縣令等,在任未經四考已上,不許遷除。察其課效尤異者,或錫以車裘,或
 就加祿秩,或降使臨問,並璽書慰勉。若公卿有闕,則擢以勸能。其政績無聞及犯貪暴者,免歸田里。以明聖朝賞罰之信,則萬方之人,一變於道矣。致此之美,革彼之弊,易於反掌,陛下何惜而不行哉!



 其二曰:



 臣聞《尚書》云:「唐、虞稽古,建官惟百;夏、商官倍,亦克用乂。」此省官之義也。又云:「官不必備,惟其才。」又云:「無曠庶官,天工人其代之。」此為官擇人之義也。臣竊見京諸司員外官,所在委積,多者數餘十倍,近古以來未之有也。官不必備,此則
 有餘,人代天工,多不厘務。廣有除拜,無所裨益,俸祿之費,歲巨億萬,空竭府藏而已,豈致理之基哉!方今倉庫空虛,百姓凋弊,河、渭漕輓,西給京師,公私損耗,不可勝紀。況邊隅未靜,兵革猶興,節用愛人,正在今日,增官廣費,豈曰其時?倘水旱成災,租稅減入,水衡無貫朽之蓄,京瘐闕流衍之儲。或疆場外守,兵車遠出;或收茂無歲,賑救在辰。此軍國之急務也,陛下將何以濟之乎?《書》云:「無輕人事,惟艱;無安厥位,惟危。」又云:「不見是圖。」此皆慎
 微之深旨也。



 臣竊見員外官中,或簪裾雅望,或臺閣舊人,或明習憲章,或諳閑政要,皆一時之良幹也。多不司案牘,空尸祿俸,滯其才而不申其用,尊其位而不盡其力。周稱多士,漢曰得人,豈其然歟?必有異於此矣。臣望請諸司員外官有才能器識、眾共聞知,堪為州牧縣宰及上佐者,並請遷擢,使宣力四方,申其智效。有老病及不堪理務者,咸從廢省,使賢不肖較然殊貫。此濟時之切務也,安可謂行之艱哉?



 其三曰:



 臣聞天吏逸德,烈於
 猛火;貪人敗類,取興大風。則知冒於寵賂,侮於鰥寡,為政之蠹,莫先於茲。臣竊見內外官人,有不率憲章,公犯贓污,侵牟萬姓,劓割蒸人,鞫按非虛,刑憲已及者,或俄復舊資,雖負殘削之名,還膺牧宰之任,或江、淮、嶺、磧,微示懲貶,而徇財黷貨,罕能悛革,委以共理,俟河之清。臣聞明主之於萬姓也,必暢以平分,而無偏施。若犯罪之吏,作牧遐方,便是屈法惠奸,恤近遺遠矣。凡左降之人,鮮能省過,必懷自棄,長惡滋深。則小州遠郡,蠻陬夷落,
 何負於聖化,獨受其弊政乎!昔孟嘗廉明,方臨合浦;隱之清絜,乃蒞番禺。郅都之鎮靜朔方,耿恭之輯寧疏勒。地則遐僻,必擇賢良,務以寧濟為懷,豈以遐荒見隔?況邊徼之地,夷夏雜處,負險恃遠,易擾難安,彌藉循良,以寄綏撫。若委失其任,官非其才,凌虐黎庶,侵剝蕃部,小則坐致流亡,大則起為盜賊。由此言之,不可用凡材,而況於猾吏乎!其內外官人有犯贓賄推勘得實者,臣望請削跡簪裾,十數年間不許齒錄。《書》云:「旌別淑匿,黜陟
 幽明。」即其義也。若不循此道,去邪有疑,善政能官,甄獎或未之偏,擔贓負賄,僥幸或即蒙升,則賞罰無章,沮勸安寄?浮競之風轉扇,廉恥之行漸隤,其源不塞,為蠹斯甚。



 疏奏不納。累遷黃門侍郎,賜爵漁陽伯。



 先天二年,與侍中魏知古於東都分掌選事,尋徵還同中書門下三品。開元三年,遷黃門監。懷慎與紫微令姚崇對掌樞密,懷慎自以為吏道不及崇,每事皆推讓之,時人謂之「伴食宰相。」四年,兼吏部尚書。其秋,以疾篤,累表乞骸骨,許
 之。旬日而卒,贈荊州大都督,謚曰文成。懷慎臨終遺表曰:



 臣素無才識,叨沐恩榮,待罪樞密,頗積年序。報國之心,空知自竭;推賢之志,終未克申。孤負明恩,夙夜惶懼。臣染疾已久,形神欲離,鳧雁之飛,未為之少,而犬馬之志,終祈上聞,其鳴也哀,乞求聖察。



 宋璟立性公直,執心貞固,文學足以經務,識略期於佐時,動惟直道,行不茍合,聞諸朝野之說,實為社稷之臣。李傑勤苦絕倫,貞介獨立,公家之事,知無不為,乾時之材,眾議推許。李朝隱
 操履堅貞,才識通贍,守文奉法,頗懷鐵石之心,事上竭誠,實盡人臣之節。盧從願清貞謹慎,理識周密,始終若一,朝野共知,簡要之才,不可多得。並明時重器,聖代良臣。比經任使,微有愆失,所坐者小,所棄者大,所累者輕,所貶者遠。日月雖近,譴責傷深,望垂矜錄,漸加進用。



 臣竊聞黃帝所以垂衣裳而天下理者,任風、力也;帝堯所以光宅天下者,任稷、祼也。且朝廷者天下之本,賢良者風化之源,得人則庶績其凝,失士則彞倫攸斁。臣每見
 陛下憂勞庶政,勤求理道,慎舉群司,必期稱職,使鵷鷺成列,草澤無遺。故得歲稔時和,政平訟理,比陛下用賢之明效也。臣非木石,早識天心,瞑目不遙,厚恩未報。黜殯之義,敢不庶幾,城郢之言,思布愚懇。



 上深嘉納之。懷慎清儉,不營產業,器用服飾,無金玉綺文之麗。所得祿俸,皆隨時分散,而家無餘蓄,妻子匱乏。及車駕將幸東都,四門博士張星上言:「懷慎忠清直道,終始不虧,不加寵贈,無以勸善。」乃下制賜其家物壹伯段、米粟貳伯石。
 明年,上還京師,因校獵於城南,經懷慎別業,見家人方設祥齋,憫其貧匱,賜絹百匹。仍遣中書侍郎蘇頲為其碑文,上自書焉。



 子奐,早修整,歷任皆以清白聞。開元中,為中書舍人、御史中丞、陜州刺史。二十四年,玄宗幸京師,次陜城頓,審其能政,於事題贊而去,曰:「專城之重,分陜之雄。人多惠愛,性實謙沖。亦既利物,在乎匪躬。斯為國寶,不墜家風。」尋除兵部侍郎。天寶初,為晉陵太守。時南海郡利兼水陸,環寶山積,劉巨鱗、彭杲相替為太
 守、五府節度,皆坐贓鉅萬而死。乃特授奐為南海太守。遐方之地,貪吏斂跡,人用安之。以為自開元已來四十年,廣府節度清白者有四:謂宋璟、裴伷先、李朝隱及奐。中使市舶,亦不干法。加銀青光祿大夫。經三年,入為尚書右丞,卒。弟弈,亦傳清白,歷御史中丞而死王事,見《忠義傳》。弈子杞,德宗朝位至宰輔,別有傳。



 源乾曜,相州臨漳人。隋比部侍郎師之孫也。父直心,高宗時為司刑太常伯,坐事配流嶺南而卒。乾曜舉進士,
 景雲中,累遷諫議大夫。時久廢公卿百官三九射禮,乾曜上疏曰:「夫聖王之教天下也,必制禮以正人情,人情正則孝於家,忠於國。此道不替,所以理也。所以君子三年不為禮,禮必壞;三年不為樂,樂必崩。竊以古之擇士,先觀射禮,以明和容之義,非取一時之樂。夫射者,別正邪,觀德行,中祭祀,闢寇戎。古先哲王,莫不遞襲。臣竊見數年已來,射禮便廢,或緣所司惜費,遂令大射有虧。臣愚以為所費者財,所全者禮。故孔子云:『爾愛其羊,我愛
 其禮。』今乾坤再闢,日月貞明,臣望大射之儀,春秋不廢,聖人之教,今古常行,則天下幸甚。」乾曜尋出為梁州都督。



 開元初,邠王府僚吏有犯法者,上令左右求堪為王府長史者,太常卿姜皎薦乾曜公清有吏乾,因召見與語。乾曜神氣清爽,對答皆有倫序,上甚悅之,乃拜少府少監,兼邠王府長史。尋遷戶部侍郎、兼御史中丞。無幾,轉尚書左丞。四年冬,擢拜黃門侍郎、同紫微黃門平章事。旬日,與姚元之俱罷知政事。



 時行幸東都,以乾曜為
 京兆尹,仍京師留守。乾曜政存寬簡,不嚴而理。嘗有仗內白鷹,因縱遂失所在,上令京兆切捕之。俄於野外獲之,其鷹掛於叢棘而死,官吏懼得罪,相顧失色。乾曜徐曰:「事有邂逅,死亦常理,主上仁明,當不以此置罪。必其獲戾,吾自當之,不須懼也。」遂入自請失旨之罪,上一切不問之,眾咸伏乾曜臨事不懾,而能引過在己也。在京兆三年,政令如一。



 八年春,復為黃門侍郎、同中書門下三品,尋加銀青光祿大夫,遷侍中。久之,上疏曰:「臣竊見
 形要之家並求京職,俊乂之士多任外官,王道平分,不合如此。臣三男俱是京任,望出二人與外官,以葉均平之道。」上從之,於是改其子河南府參軍弼為絳州司功,太祝絜為鄭尉。因下制曰:「源弼等父在樞近,深惟謙挹,恐代官之咸列,慮時才之未序,率先庶僚,崇是讓德,既請外其職,復降資以授。《傳》不云乎:『晉範宣子讓,其下皆讓。』『晉國之人,於是大和。』道之或行,仁豈雲遠!」因令文武百僚父子兄弟三人並任京司者,任自通容,依資次處
 分,由是公卿子弟京官出外者百餘人。俄又有上書者,以為「國之執政,同其休戚,若不稍加崇寵,何以責其盡心?」十年十一月,敕中書門下共食實封三百戶,自乾曜及張嘉貞始也。



 乾曜後扈從東封,拜尚書左丞相,仍兼侍中。乾曜在政事十年,時張嘉貞、張說相次為中書令,乾曜不敢與之爭權,每事皆推讓之。及李元紘、杜暹知政事,乾曜遂無所參議,但唯諾署名而已。初,乾曜因姜皎所薦,遂擢用;及皎得罪,為張嘉貞所擠,乾曜竟不救
 之,議者以此譏焉。十七年夏,停兼侍中事。其秋,遷太子少師,以祖名師,固辭,乃拜太子少傅,封安陽郡公。十九年,駕幸東都,乾曜以年老辭疾,不堪扈從,因留京養疾。是年冬卒,詔贈幽州大都督,上於洛城南門舉哀,輟朝二日。



 乾曜從孫光裕,亦有令譽。歷職清謹,撫諸弟以友義聞。初為中書舍人,與楊滔、劉令植等同刪定《開元新格》。歷刑部戶部二侍郎、尚書左丞,累遷鄭州刺史,稱為良吏。尋卒。光裕子洧,亦早有美稱。閨門雍睦,士友推之,
 歷踐清要。天寶中,為給事中、蔸鄭州刺史、襄州刺史、本道採訪使。及安祿山反,既犯東京,乃以洧為江陵郡大都督府長史、本道採訪防禦使、攝御史中丞,以兵部郎中徐浩為襄州刺史、本州防禦守捉使以御之。洧至鎮卒。



 李元紘,其先滑州人,世居京兆之萬年。本姓丙氏。曾祖粲,隋大業中屯衛大將軍。屬關中賊起,煬帝令粲往京城以西二十四郡逐捕盜賊,粲撫循士眾,甚得其心。及義旗入關,粲率其眾歸附,拜宗正卿,封應國公,賜姓李
 氏。高祖與之有舊,特蒙恩禮,遷為左監門大將軍,以年老特令乘馬於宮中檢校。年八十餘卒,謚曰明。祖寬,高宗時為太常卿,別封隴西郡公。父道廣,則天時為汴州刺史。時屬突厥及契丹寇陷河北,兼發河南諸州兵募,百姓騷擾。道廣寬猛折衷,稱為善政,存收慰撫,汴州獨不逃散。尋入為殿中監、同鳳閣鸞臺平章事,累封金城縣侯。卒,贈秦州都督,謚曰成。



 元紘少謹厚。初為涇州司兵,累遷雍州司戶。時太平公主與僧寺爭碾磑,公主方
 承恩用事,百司皆希其旨意,元紘遂斷還僧寺。竇懷貞為雍州長史,大懼太平勢,促令元紘改斷,元紘大署判後曰:「南山或可改移,此判終無搖動。」竟執正不撓,懷貞不能奪之。俄轉好畤令,遷潤州司馬,所歷咸有聲績。開元初,三遷萬年縣令,賦役平允,不嚴而理。俄擢為京兆尹,尋有詔令元紘疏決三輔。諸王公權要之家,皆緣渠立磑,以害水田,元紘令吏人一切毀之,百姓大獲其利。又歷工部、兵部、吏部三侍郎。十三年,戶部侍郎楊易、白
 知慎坐支度失所,皆出為刺史。上令宰臣及公卿已下精擇堪為戶部者,多有薦元紘者,將授以戶部尚書,時執政以其資淺,未宜超授,加中大夫,拜戶部侍郎。元紘因條奏人間利害及時政得失以奏之,上大悅,因賜衣一副、絹二百匹。明年,擢拜中書侍郎、同中書門下平章事。頃之,加銀青光祿大夫,賜爵清水男。



 元紘性清儉。既知政事,稍抑奔競之路,務進者頗憚之。時初廢京司職田,議者請於關輔置屯,以實倉稟。元紘建議曰:「軍國不
 同,中外異制。若人閑無役,地棄不墾,發閑人以耕棄地,省饋運以實軍糧,於是乎有屯田,其為益多矣。今百官所退職田,散在諸縣,不可聚也。百姓所有私田,皆力自耕墾,不可取也。若置屯田,即須公私相換,徵發丁夫,征役則業廢於家,免庸則賦闕於國。內地置屯,古所未有,得不補失,或恐未可。」其議遂止。



 先是,左庶子吳兢舊任史官,撰《唐書》一百卷、《唐春秋》三十卷,其書未成,以丁憂罷職。至是,上疏請終其功,有詔特令就集賢院修成其
 書。及張說致仕,又令在家修史。元紘奏曰:「國史者,記人君善惡,國政損益,一字褒貶,千載稱之,前賢所難,事匪容易。今張說在家修史,吳兢又在集賢撰錄,遂令國之大典,散在數處。且太宗別置史館,在於禁中,所以重其職而秘其事也。望勒說等就史館參詳撰錄,則典冊有憑,舊章不墜矣。」從之,乃詔說及吳兢並就史館修撰。



 元紘在政事累年,不改第宅,僕馬弊劣,未曾改飾,所得封物,皆散之親族。右丞相宋璟嘗嘉嘆之,每謂人曰:「李侍
 郎引宋遙之美才,黜劉晃之貪冒,貴為國相,家無儲積。雖季文子之德,何以加也!」後與杜暹多所異同,情遂不葉,至有相執奏者,上不悅,由是罷知政事,出為曹州刺史,以疾去官。久之,拜戶部尚書,仍聽致仕。二十一年疾瘳,起為太子詹事,旬日而卒。贈太子少傅,謚曰文忠。



 杜暹,濮州濮陽人也。父承志,則天初為監察御史。時懷州刺史李文暕以皇枝近屬,為讎人所告,承志推出之。俄而文暕得罪,承志坐貶,授方義令。累轉天官員外郎。
 既羅織事起,承志恐懼,遂稱疾去官而歸,卒於家。自暹高祖至暹,五代同居,暹尤恭謹,事繼母以孝聞。初舉明經,補婺州參軍,秩滿將歸,州吏以紙萬餘張以贈之,暹惟受一百,餘悉還之。時州僚別者,見而嘆曰:「昔清吏受一大錢,復何異也!」俄授鄭尉,復以清節見知。華州司馬楊孚,公直士也,深賞重之。尋而孚遷大理正,暹坐公事下法司結罪,孚謂人曰:「若此尉得罪,則公清之士何以勸矣?」特薦之於執政,由是擢拜大理評事。



 開元四年,遷
 監察御史,仍往磧西覆屯。會安西副都護郭虔瓘與西突厥可汗史獻、鎮守使劉遐慶等不葉,更相執奏,詔暹按其事實。時暹已回至涼州,承詔復往磧西,因入突厥騎施,以究虔齎等犯狀。蕃人齎金以遺,暹固辭不受。左右曰:「公遠使絕域,不可失蕃人情。」暹不得已受之,埋幕下,既去出境,乃移牒令收取之。蕃人大驚,度磧追之,不及而止。暹累遷給事中,丁繼母憂去職。十二年,安西都護張孝嵩遷為太原尹,或薦暹往使安西,蕃人伏其清
 慎,深思慕之,乃奪情擢拜黃門侍郎,兼安西副大都護。暹單騎赴職。明年,於闐王尉遲眺陰結突厥及諸蕃國圖為叛亂,暹密知其謀,發兵捕而斬之,並誅其黨與五十餘人,更立君長,於闐遂安。暹以功特加光祿大夫。暹在安西四年,綏撫將士,不憚勤苦,甚得夷夏之心。



 十四年,詔暹同中書門下平章事,仍遣中使往迎之。及謁見,又賜絹二百匹、馬一匹、宅一區。後與李元紘不葉,罷知政事,出為荊州大都督府長史。又歷魏州刺史、太原尹。
 二十年,上幸北都,拜暹為戶部尚書,便令扈從入京。行幸東都,詔暹為京留守。暹因抽當番衛士,繕修三宮,增峻城隍,躬自巡檢,未嘗休懈。上聞而嘉之,賜敕書曰:「卿素以清直,兼之勤幹。自委居守,每事多能,政肅官僚,惠及黎庶。城隍宮室,隨事修營,且有成功,不疲人力。甚善甚善,慰朕懷也。」俄代李林甫為禮部尚書,累封魏縣侯。二十八年,病卒,年六十餘,詔贈尚書右丞相。



 暹在家孝友,愛撫異母弟昱甚厚。然素無學術,每當朝談議,涉於
 淺近。常以公清勤儉為己任,時亦矯情為之。弱冠便自誓不受親友贈遺,以終其身。及卒,上甚悼惜之,遣中使就家視其喪事,內出絹三百匹以賜之。尚書省及故吏賻贈者,其子孝友遵其素約,皆拒而不受。太常謚曰「貞肅」。右司員外郎劉同升、都官員外郎韋廉以暹有忠孝之美,所謚不盡其行,建議駁之。太常博士裴總執曰:「杜尚書往以墨縗受職事,雖云奉國,不得為孝。請依舊為定。」孝友又詣闕陳訴上聞,而更令所司詳定,竟謚曰貞
 孝。



 韓休,京兆長安人。伯父大敏,則天初為鳳閣舍人。時梁州都督李行褒為部人誣告,云有逆謀,則天令大敏就州推究。或謂大敏曰:「行褒諸李近屬,太后意欲除之,忽若失旨,禍將不細,不可不為身謀也。」大敏曰:「豈有求身之安而陷人非罪!」竟奏雪之。則天俄又命御史重覆,遂構成其罪,大敏坐推反失情,與知反不告同罪,賜死於家。父大智,官至洛州司功。



 休早有詞學,初應制舉,累授
 桃林丞。又舉賢良。玄宗時在春宮,親問國政,休對策與校書郎趙冬曦並為乙第,擢授左補闕。尋判主爵員外郎,歷遷中書舍人、禮部侍郎,兼知制誥,出為虢州刺史。時虢州以地在兩京之間,駕在京及東都,並為近州,常被支稅草以納閑廄。休奏請均配餘州,中書令張說駁之曰:「若獨免虢州,即當移向他郡,牧守欲為私惠,國體固不可依。」又下符不許之。休復將執奏,僚吏曰:「更奏必忤執政之意。」休曰:「為刺史不能救百姓之弊,何以為政!
 必以忤上得罪,所甘心也。」竟執奏獲免。歲餘,以母艱去職,固陳誠乞終禮,制許之。服闋,除工部侍郎,仍知制誥,遷尚書右丞。



 開元二十一年,侍中裴光庭卒,上令蕭嵩舉朝賢以代光庭才,嵩盛稱休志行,遂拜黃門侍郎、同中書門下平章事。休性方直,不務進趨,及拜,甚允當時之望。俄有萬年尉李美玉得罪,上特令流之嶺外,休進曰:「美玉卑位,所犯又非巨害,今朝廷有大奸,尚不能去,豈得舍大而取小也!臣竊見金吾大將軍程伯獻,依恃
 恩寵,所在貪冒,第宅輿馬,僭擬過縱。臣請先出伯獻而後罪美玉。」上初不許之,休固爭曰:「美玉微細猶不容,伯獻巨猾豈得不問!陛下若不出伯獻,臣即不敢奉詔流美玉。」上以其切直,從之。初,蕭嵩以休柔和易制,故薦引之。休既知政事,多折正嵩,遂與休不葉。宋璟聞之曰:「不謂韓休乃能如是,仁者之勇也。」



 其年夏,加銀青光祿大夫。十二月,轉工部尚書,罷知政事。二十四年,遷太子少師,封宜陽子。二十七年病卒,年六十八,贈揚州大都督,
 謚曰文忠。寶應元年,重贈太子太師。



 子洽、洪、汯、滉,皆有學尚,風韻高雅。洽,天寶初為殿中侍御史卒。洪,為司庫員外郎。洽弟渾,除大理司直。御史大夫王鉷犯法,籍沒其家,洽兄浩為萬年主簿,捕其資財,有所容隱,為京兆尹鮮於仲通所發,配流循州。洪、汯並坐貶職。後遇赦,量移洪為華州長史。屬安祿山反,西京失守,洪陷於賊,賊授官,將見委任,洪與浩及汯、滉、渾同奔山谷,以投行在。至谷口,洪、浩、渾及洪子四人並為賊所擒,並命於通衢。
 洪重交友,籍甚於時,見者掩涕,肅宗聞其重臣子,能以忠而死,贈太常卿。浩贈吏部郎中,渾贈太常少卿。汯,上元中為諫議大夫。滉、洄,別有傳。



 裴耀卿,贈戶部尚書守真子也。少聰敏,數歲解屬文,童子舉。弱冠拜秘書正字,俄補相王府典簽。時睿宗在蕃,甚重之,令與掾丘悅、文學韋利器更直府中,以備顧問,府中稱為學直。及睿宗升極,拜國子主簿。開元初,累遷長安令。長安舊有配戶和市之法,百姓苦之。耀卿到官,
 一切令出儲蓄之家,預給其直,遂無奸僦之弊,公私甚以為便。在職二年,寬猛得中。及去官,縣人甚思詠之。十三年,為濟州刺史。其年,車駕東巡,州當大路,道里綿長,而戶口寡弱,耀卿躬自條理,科配得所。時大駕所歷凡十餘州,耀卿稱為知頓之最。又歷宣、冀二州刺史,皆有善政,入為戶部侍郎。



 二十年,禮部尚書、信安王禕受詔討契丹,詔以耀卿為副。俄又令耀卿齎絹二十萬匹分賜立功奚官,就部落以給之。耀卿謂人曰:「夷虜貪殘,見
 利忘義,今齎持財帛,深入寇境,不可不為備也。」乃令先期而往,分道互進,一朝而給付並畢。時突厥及室韋果勒兵邀險,謀劫襲之,比至而耀卿已還。



 其冬,遷京兆尹。明年秋,霖雨害稼,京城穀貴。上將幸東都,獨召耀卿問救人之術,耀卿對曰:



 臣聞前代聖王,亦時有憂害,更施惠澤,活國濟人,由是蒼生仰德,史冊書美。伏以陛下仁聖至深,憂勤庶政,小有饑乏,降情哀矜,躬親支計,救其危急。上玄降鑒,當更延福祚,是因有小災而增輝聖德
 也。今既大駕東巡,百司扈從,太倉及三輔先所積貯,且隨見在發重臣分道賑給,計可支一二年。從東都更廣漕運,以實關輔。待稍充實,車駕西還,即事無不濟。臣以國家帝業,本在京師,萬國朝宗,百代不易之所。但為秦中地狹,收粟不多,倘遇水旱,便即匱乏。往者貞觀、永徽之際,祿稟數少,每年轉運不過一二十萬石,所用便足,以此車駕久得安居。今國用漸廣,漕運數倍於前,支猶不給。陛下數幸東都,以就貯積,為國大計,不憚劬勞,只
 為憂人而行,豈是故欲不往。若能更廣陜運,支粟入京,倉稟常有三二年糧,即無憂水旱。今天下輸丁約有四百萬人,每丁支出錢百文,五十文充營窖等用,貯納司農及河南府、陜州以充其費。租米則各隨遠近,任自出腳送納東都。從都至陜,河路艱險,既用陸腳,無由廣致。若能開通河漕,變陸為水,則所支有餘,動盈萬計。且河南租船候水始進,吳人不便河漕,由是所在停留,日月既淹,遂生隱盜。臣望沿流相次置倉。



 上深然其言。尋拜
 黃門侍郎、同中書門下平章事,充轉運使,語在《食貨志》。凡三年,運七百萬石,省腳錢三十萬貫。或說耀卿請進所省腳錢,以明功利。耀卿曰:「此蓋公卿盈縮之利耳,不可以之求寵也。」乃奏充所司和市、和糴等錢。



 明年,遷侍中。二十四年,拜尚書左丞相,罷知政事,累封趙城侯。時夷州刺史楊浚犯贓處死,詔令杖六十,配流古州。耀卿上疏諫曰:



 伏以聖恩天覆,仁育庶類,凡死罪之屬,不欲尸諸市朝,全其性命,流竄而已。所以政致刑措,獄無冤
 人,曠古以來,未有斯美。臣愚以為全生免死,誠為至化,有恥且格,為訓將來。茍有未安,不敢緘默。



 臣以為刺史、縣令,與諸吏稍別,人之父母,風化所瞻,一為本部長官,即合終身致敬。決杖者,五刑之末,只施於抶撲徒隸之間,官廕稍高,即免鞭撻。令決杖贖死,誠則已優,解體受笞,事頗為辱。法至於死,天下共之,刑至於辱,或有所恥。況本州刺史,百姓所崇,一朝對其人吏,背脊加杖,屈挫拘執,人或哀憐,忘其免死之恩,且有傷心之痛,恐非敬
 官長勸風俗之意。



 又雜犯死罪,無杖刑,奏報三覆,然後行決。今非時不覆,決杖便發,倘獄或未盡,又暑熱不耐,因杖或死,即是促其處分,不得順時。將欲生之,卻夭其命,又恐非聖明寬宥之意。前後頻在州縣,或緣雜犯決人,每大暑盛夏之時,決杖多死,秋冬已後,至有全者。伏望凡刺史、縣令於本部決杖及夏暑生長之時,所定杖刑,並乞停減。即副陛下好生之德,於死者皆有再生之恩。



 俄而特進蓋嘉運破突騎施立功還,詔加河西、隴右兩
 節度使,仍令經略吐蕃。嘉運既承恩寵,日夕酣宴,不時赴軍。耀卿密上疏曰:「伏見蓋嘉運立功破賊,更委兩軍,以勇果之才,承戰勝之勢,吐蕃小醜,不足殲夷。然臣近日與其同班,觀其舉措,精勁勇烈,誠則有餘,言氣矜誇,恐難成事。莫敖敗於蒲騷之役,舉趾稍高,《春秋》書之為懲誡。恐其有驕敵之色,臣竊憂之。入秋防邊,日月稍逼,接對人吏,須識其宜。今將撫邊軍,未言發日,若臨事始去,人吏未識,雖決在一時,恐將非制勝萬全之道。況兵
 未訓練,不知禮法,人未懷惠,士未同心,求其忘性命於一時,憚嚴刑於少選,縱威逼而進,因而立功,恐非師出以律,久長之義。又萬人性命,決在將軍,不得已而行之,鑿兇門而即路。今酣宴朝夕,優渥有餘,亦恐非愛人憂國之意,不可不察。若不可回換,即望速遣進途,仍乞聖恩,勖以嚴命。」疏奏,上乃促嘉運赴軍,竟以無功而還。



 天寶元年,改為尚書右僕射,尋轉左僕射。一歲薨,年六十三,贈太子太傅,謚曰文獻。子綜,吏部郎中。綜子佶。佶,字
 弘正,幼能屬文。弱冠舉進士,補校書郎,判入高等,授藍田尉。時有詔命畿內諸縣城奉天,時嚴郢為京兆,政尚峻暴,加以朝旨甚迫,尹正之命,急如風霆。本曹尉韋重規其室方娠而疾,畏郢之暴,不敢以事故免。佶因請代,役無愆程,當時義之。德宗南狩,佶詣行在,拜拾遺,轉補闕。李懷光以河中叛,朝廷欲以含垢為意,佶抗議請討,上深器之,前席慰勉。三遷吏部員外,歷駕部兵部郎中,遷諫議大夫。會黔中觀察使韋士宗慘酷馭下,為夷獠
 所逐,俾佶代之,酋渠自化。其後為瘴毒所侵,堅請入覲,拜同州刺史。徵入為中書舍人,遷尚征入為中舍人,遷尚書右丞。時兵部尚書李巽兼鹽鐵使,將以使局置於本行,經構已半,會佶拜命,堅執以為不可,遂令徹之。巽恃恩而強,時重佶之有守,就拜吏部侍郎。以疾除國子祭酒,尋遷工部尚書致仕。元和八年卒,年六十二,贈吏部尚書。佶清勁溫敏,凡所定交,時稱為第一流。與鄭餘慶特相友善,佶歿後,餘慶行朋友之服,搢紳美之。



 史臣曰:魏知古、盧懷慎、源乾曜、李元紘、杜暹、韓休、裴耀卿,悉蘊器能,咸居宰輔。或心存啟沃,或志在薦賢,或出愛子為外官,或止屯田於關輔,或不受蕃人之賂,或堅劾伯獻之奸,或廣漕渠以充國用:此皆立事立功,有足嘉尚者也。盧、李、杜三君子,又以清白垂美簡書,公孫弘之流也。乾曜職當機密,無所是非,持祿保身,焉用彼相?



 贊曰:盧、魏、乾曜,弼違進賢。裴、韓、李、杜,遠財劾奸。汗簡書事,清風肅然。萬歲之後,其名不刊。



\end{pinyinscope}