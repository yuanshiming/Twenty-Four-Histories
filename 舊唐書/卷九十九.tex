\article{卷九十九}

\begin{pinyinscope}

 ○皇帝憲惠莊太子捴惠文太子範惠宣太子業隋王隆悌



 睿宗六子:昭成順聖皇后竇氏生玄宗,肅明順聖皇後
 劉氏生讓皇帝,宮人柳氏生惠莊太子,崔孺人生惠文太子,王德妃生惠宣太子,後宮生隋王隆悌。



 讓皇帝憲,本名成器,睿宗長子也。初封永平郡王。文明元年,立為皇太子,時年六歲。及睿宗降為皇嗣,則天冊授成器為皇孫,與諸弟同日出閣,開府置官屬。長壽二年,改封壽春郡王,仍卻入閣。長安中,累轉左贊善大夫。加銀青光祿大夫。中宗即位,改封蔡王,遷宗正員外卿,加賜實封四百戶,通舊為七百戶。成器固辭不敢當大
 國,依舊為壽春郡王。



 唐隆元年,進封宋王。其月,睿宗踐祚,拜左衛大將軍。時將建儲貳,以成器嫡長,而玄宗有討平韋氏之功,意久不定。成器辭曰:「儲副者,天下之公器,時平則先嫡長,國難則歸有功。若失其宜,海內失望,非社稷之福。臣今敢以死請。」累日涕泣固讓,言甚切至。時諸王、公卿亦言楚王有社稷大功,合居儲位。睿宗嘉成器之意,乃許之。玄宗又以成器嫡長,再抗表固讓,睿宗不許。乃下制曰:「左衛大將軍、宋王成器,朕之元子,當
 踐副君。以隆基有社稷大功,人神僉屬,由是朕前懇讓,言在必行。天下至公,誠不可奪。爰符立季之典,庶協從人之願。成器可雍州牧、揚州大都督、太子太師,別加實封二千戶。賜物五千段、細馬二十匹、奴婢十房、甲第一區、良田三十頃。」其年十一月拜尚書左僕射,尋遷司徒,其太師、都督並如故。明年,表讓司徒,拜太子賓客,兼揚州大都督如故。



 時太平公主陰有異圖,姚元之、宋璟等請出成器及申王成義為刺史,以絕謀者之心。由是成
 器以司徒兼蒲州刺史。玄宗嘗制一大被長枕,將與成器等共申友悌之好,睿宗知而大悅,累加賞嘆。



 先天元年八月,進封司空。及玄宗討平蕭至忠、岑羲等,成器又進位太尉,依舊兼揚州大都督,加實封一千戶。月餘,加授開府儀同三司,其太尉、揚州大都督並停。開元初,歷岐州刺史,開府如故。四年,避昭成皇后尊號,改名憲,封為寧王,實封累至五千五百戶。又歷澤、、涇等州刺史。



 初,玄宗兄弟聖歷初出閤,列第於東都積善坊,五人分院
 同居,號「五王宅」。大足元年,從幸西京,賜宅於興慶坊,亦號「五王宅」。及先天之後,興慶是龍潛舊邸,因以為宮。憲於勝業東南角賜宅,申王捴、岐王範於安興坊東南賜宅,薛王業於勝業西北角賜宅,邸第相望,環於宮側。玄宗於興慶宮西南置樓,西面題曰花萼相輝之樓,南面題曰勤政務本之樓。玄宗時登樓,聞諸王音樂之聲,咸召登樓同榻宴謔,或便幸其第,賜金分帛,厚其歡賞。諸王每日於側門朝見,歸宅之後,即奏樂。縱飲,擊球鬥雞,
 或近郊從禽,或別墅追賞,不絕於歲月矣。游踐之所,中使相望,以為天子友悌,近古無比,故人無間然。



 玄宗既篤於昆季,雖有讒言交構其間,而友愛如初。憲尤恭謹畏慎,未曾幹議時政及與人交結,玄宗尤加信重之。嘗與憲及岐王範等書曰:「昔魏文帝詩云:『西山一何高,高處殊無極。上有兩仙童,不飲亦不食。賜我一丸藥,光耀有五色。服藥四五日,身輕生羽翼。』朕每思服藥而求羽翼,何如骨肉兄弟天生之羽翼乎!陳思有超代之才,堪
 佐經綸之務,絕其朝謁,卒令憂死。魏祚未終,遭司馬宣王之奪,豈神丸之效也!虞舜至聖,拾象傲之愆以親九族,九族既睦,平章百姓,此為帝王之軌則,於今數千歲,天下歸善焉。朕未嘗不廢寢忘食欽嘆者也,頃因餘暇,妙選仙經,得此神方,古老云『服之必驗』。今分此藥,願與兄弟等同保長齡,永無限極。」



 憲,開元九年兼太常卿。十四年,停太常卿,依舊為開府儀同三司。二十一年,復拜太尉。二十八年冬,憲寢疾,上令中使送醫藥及珍膳,相
 望於路,僧崇一療憲稍瘳,上大悅,特賜緋袍魚袋,以賞異崇一。時申王等皆先薨,唯憲獨在,上尤加恩貸。每年至憲生日,必幸其宅,移時宴樂。居常無日不賜酒酪及異饌等,尚食總監及四方有所進獻,食之稍甘,即皆分以賜之。憲嘗奏請年終錄付史館每年至數百紙。



 二十九年冬,京城寒甚,凝霜封樹,時學者以為《春秋》「雨木冰」即此是,亦名樹介,言其象介胄也。憲見而嘆曰:「此俗謂樹稼者也。諺曰:『樹稼,達官怕。』必有大臣當之,吾其死矣。」
 十一月薨,時年六十三。上聞之,號叫失聲,左右皆掩涕。翌日,下制曰:



 能以位讓,為吳太伯,存則用成其節,歿則當表其賢,非常之稱,旌德斯在。故太尉、寧王憲,誕含粹靈,允膺大雅。孝悌之至,本乎中誠;仁和之深,非因外獎。率由禮度,雅尚文儒。謙以自牧,樂以為善。比兩獻而有光,與《二南》而合德。自出臨方鎮,入配臺階,逾勵忠勤,益聞周慎。實謂永為籓屏,以輔邦家。曾不籥遺,奄焉殂沒,友于之痛,震慟良深。惟王朕之元昆,合升王嗣,以朕奉
 先朝之睿略,定宗社之阽危,推而不居,請予主鬯,又承慈旨,焉敢固違。不然者,則宸極之尊,豈歸於薄德。茂行若此,易名是憑,自非大號,孰副休烈。按謚法推功尚善曰「讓」,德性寬柔曰「讓」,敬追謚曰讓皇帝,宜令所司擇曰備禮冊命。



 憲長子汝陽郡王璡又上表懇辭,盛陳先意,謙退不敢當帝號,手制不許。及冊斂之日,內出御衣一副,仍令右監門大將軍高力士齎手書置於靈座之前,其書曰:



 隆基白:一代兄弟,一朝存歿,家人之禮,是用申
 情,興言感思,悲涕交集。大哥孝友,近古莫儔,嘗號五王,同開邸第。遠自童幼,洎乎長成。出則同游,學則同業,事均形影,無不相隨。頃以國步艱危,義資克定,先帝御極,日月照臨。大哥嫡長,合當儲貳,以功見讓,爰在薄躬。既嗣守紫宸,萬機事總,聽朝之暇,得展於懷。十數年間,棣華凋落,謂之手足,唯有大哥。令復淪亡,眇然無對,以茲感慕,何恨如之。然以厥初生人,孰不殂謝?所貴光昭德行,以示崇高,立德立名,斯為不朽。大哥事跡。身歿讓存,
 故冊曰讓皇帝,神之昭格,當茲寵榮。況庭訓傳家,璡等申讓,善述先志,實有遺風,成其美也。恭惟緒言,恍焉如在,寄之翰墨,悲不自勝。



 又制追贈憲妃元氏為恭皇后,祔葬於橋陵之側。及將葬,上遣中使敕璡等務令儉約,送終之物,皆令眾見。所司請依諸陵舊例,壙內置千味食。監獲使、左僕射裴耀卿奏曰:「尚食所料水陸等味一千餘種,每色瓶盛,安於藏內,皆是非時瓜果及馬牛驢犢麞鹿等肉,並諸藥酒三十餘色。儀注禮料,皆無所憑。
 臣據禮司所料,奠祭相次,事無不備,典制分明。天恩每申讓帝之志,務令儉約,禮外加數,竊恐不安。又非時之物,馬犢驢等並野味魚雁鵝鷗之屬,所用銖兩,動皆宰殺,盛夏胎養,聖情所禁。又須造作什物,動逾千計,求徵市井,實謂煩勞。千味不供,禮無所闕。伏望依禮減省,以取折衷。」制從之。及發引,時屬大雨,上令慶王澤已下泥中步送十數里,制號其墓為惠陵。



 憲凡十子:璡、嗣莊、琳、璹、珣、瑀、玢、珽、琯、璀等十人,歷官封襲。璡,封汝陽郡王,歷
 太僕卿,與賀知章、褚庭誨為詩酒之交。天寶初,終父喪,加特進。九載卒,贈太子太師。嗣莊,封濟陰郡王,早卒。琳,封嗣寧王,歷秘書員外監。從玄宗幸蜀郡,至德二載卒。



 璹,封嗣申王。珣,封同安郡王。珣修身淳謹,不自矜貴,閨門之內,常默如也。開元二十五年薨,玄宗甚悼之,輟朝三日。制曰:「猶子之恩,特深於情禮;睦親之義,必備於哀榮。同安郡王珣,稟氣淳和,執心忠順,邦國垣翰,宗枝羽儀。磐石疏封,將期永固;逝川不舍,俄嘆促齡。悼往之懷,
 因心所切,宜增寵命,用飾幽泉。可贈太子少保。葬事官給,陪葬橋陵。」瑀,封漢中王,歷都水使者、恆王府司馬、衛尉員外卿。瑀早有才望,偉儀表。初為隴西郡公。天寶十五載,從玄宗幸蜀,至漢中,因封漢中王,仍加銀青光祿大夫、漢中郡太守。乾元二年,以特進試太常卿,送寧國公主至回紇,充冊立使。玢,蒼梧郡開國公,歷銀青光祿大夫、秘書監員外置同正員。卒,贈江陵大都督。珽,封晉昌郡開國公。琯,魏郡開國公。璀,文安郡開國公。天寶十
 一載,珽、琯、璀並食邑三千戶。



 惠莊太子捴,睿宗第二子也。本名成義。母柳氏,掖庭宮人。



 捴之初生,則天嘗以示僧萬回。萬回曰:「此兒是西域大樹之精,養之宜兄弟。」則天甚悅,始令列於兄弟之次。垂拱三年,封恆王。尋卻入閤,改封衡陽郡王,累授尚衣奉御。神龍元年,加賜實封二百戶,通前五百戶,遷司農少卿,加銀青光祿大夫。睿宗踐祚,進封申王,遷右衛大將軍。景雲元年七月,遷殿中監,兼檢校右衛大將軍。二
 年,轉光祿卿、右金吾衛大將軍。先天元年七月,加實封一千戶。八月,行司徒,兼益州大都督。開元二年,帶司徒兼幽州刺史。俄避昭成太后之稱,改名捴。歷鄧、虢、絳三州刺史。八年,因入朝,停刺史,依舊為司徒。性弘裕,儀形環偉,善於飲啖。十二年,病薨,冊贈惠莊太子,陪葬橋陵。無子。初養讓帝子珣,封同安郡王,先卒。天寶三載,又以讓帝子璹為嗣申王,授鴻臚員外卿。



 惠文太子範,睿宗第四子也。本名隆範,後避玄宗連名,
 改單稱範。初封鄭王,尋改封衛王。長壽二年,隨例卻入閤,徙封巴陵郡王,累授尚食奉御。神龍元年,遷太府員外少卿,加賜實封二百戶,通前五百戶。景龍年,兼隴州別駕,加銀青光祿大夫。睿宗踐祚,進封岐王,又加實封五百戶,拜太常卿,兼左羽林大將軍。先天二年,從上討竇懷貞、蕭至忠等,以功加賜實封滿五千戶,下制褒美。開元初,拜太子少師,帶本官,歷絳、鄭、岐三州刺史。八年,遷太子太傅。



 範好學工書,雅愛文章之士,士無貴賤,皆
 盡禮接待。與閻朝隱、劉庭琦、張諤、鄭繇篇題唱和,又多聚書畫古跡,為時所稱。時上禁約王公,不令與外人交結。駙馬都尉裴虛己坐與範游宴,兼私挾讖緯之書,配徙嶺外。萬年尉劉庭琦、太祝張諤皆坐與範飲酒賦詩,黜庭琦為雅州司戶,諤為山茌丞。然上未嘗間範,恩情如初,謂左右曰:「我兄弟友愛天至,必無異意,只是趨競之輩,強相托附耳。我終不以纖芥之故責及兄弟也。」時王毛仲等本起微賤,皆崇貴傾於朝廷,諸王每相見,假
 立引待,獨範見之色莊。十四年,病薨。上哭之甚慟,輟朝三日,為之追福,手寫《老子經》,徹膳累旬,百僚上表勸喻,然後復常。開元十四年,命工部尚書、攝太尉盧從願冊贈王為惠文太子,陪葬橋陵。



 一子瑾,封河東郡王,官至太僕卿。冒於酒色,竟暴卒,贈太子少師。



 天寶三載,又以惠宣太子男略陽公珍為嗣岐王、銀青光祿大夫、宗正員外卿。上元二年,珍與硃融善。珍儀表偉如,頗類玄宗,融乃誘崔昌、趙非熊等並中官六軍人同謀逆。融謂金
 吾將軍邢濟曰:今城中草草,關外近寇憑凌,若何?」濟曰:「我金吾,天子押衙,死生隨之,安能自脫?」融曰:「有一人,足下見之自當知,縱不出城亦無慮。」乃引以見珍。濟奏之,乃令御史中丞敬羽訊之。珍賜死。其同謀右武衛將軍竇如玢、試都水使者崔昌、右羽林軍大將軍劉從諫、蔚州長鎮將硃融、右衛將軍胡冽、直司天臺通玄院高抱素、右司禦率府率魏兆、內侍省內謁者監王道成等九人,特宜斬決。試太子洗馬兼知司天臺冬官正事趙非
 熊、陳王府長史陳閎、楚州司馬張昂、右武衛兵曹焦自榮、前鳳翔府郿縣主簿李屺、國子監廣文進士張奐等六人,特宜決殺。駙馬都尉薛履謙預逆謀,宜賜自盡。乃以濟兼桂州都督、侍御史,充桂管防禦都使。左散騎常侍張鎬坐與交通,貶辰州司戶。鄭繇者,鄭州滎陽人,北齊吏部尚書述五代孫也。工五言詩。開元初,範為岐州刺史,繇為長史,範失白鷹,繇為《失白鷹詩》,當時以為絕唱。後為湖州刺史。子審亦善詩詠,乾元中任袁州刺史。



 惠宣太子業,睿宗第五子也。本名隆業,後單名業。垂拱三年,封趙王,開府置官屬。長壽二年,隨例卻入閤,改封中山郡王,累授都水使者,尋又改封彭城郡王。神龍元年,加賜實封二百戶,通前五百戶。景龍二年,兼陳州別駕。銀青光祿大夫、太僕少卿,別駕如故。睿宗即位,進封薛王,加封滿一千戶,拜秘書監,兼右羽林大將軍。俄轉宗正卿。睿宗以業好學而授秘書監。及玄宗誅蕭至忠、岑羲等,業以翊從之功,加實封通舊為五千戶。開元初,
 歷太子少保、同涇豳衛虢等州刺史。八年,遷太子太保。



 初,業母早終,從母賢妃親鞠養之。至是,迎賢妃出就外宅,事之甚謹。業同母妹淮陽、涼國二公主亦早卒,業撫愛其子,逾於己子。上以業孝友,特加親愛。業嘗疾病,上親為祈禱,及愈,車駕幸其第,置酒宴樂,更為初生之歡。玄宗賦詩曰:「昔見漳濱臥,言將人事違。今逢誕慶日,猶謂學仙歸。棠棣花重滿,鴒原鳥再飛。」其恩意如此。



 十三年,上嘗不豫,業妃弟內直郎韋賓與殿中監皇甫恂私
 議休咎。事發,玄宗令杖殺韋賓,左遷皇甫恂為錦州刺史。妃惶懼,降服待罪,業亦不敢入謁。上遽令召之,業至階下,逡巡請罪。上降階就執其手曰:「吾若有心猜阻兄弟者,天地神明,所共咎罪。」乃歡宴久之。仍慰諭妃,令復其位。二十一年,業進拜司徒。二十二年正月,薨,冊贈惠宣太子,陪葬橋陵。有子十一人。



 瑗樂安郡王,易宗正卿、滎陽郡王,琄封嗣薛王,珍嗣岐王。琄為金紫光祿大夫、鴻臚卿同正員。天寶五載,坐舅刑部尚書韋堅為右相李
 林甫所構,貶夷陵郡別駕長任。母隨琄,竟以憂死。七載,琄於夜郎安置,後移南浦郡。十四載,安祿山反,赴於西京。



 隋王隆悌,睿宗第六子也。初封汝南郡王。長安初,拜尚乘直長。早薨。睿宗踐極,追封隋王,贈荊州大都督。無子。



 史臣曰:夫得天下而治者,其道舒而有變;讓天下而退者,其道卷而常存。何者?飛龍在天,舒也;亢龍有悔,變也。讓皇帝守無咎於或躍,利終吉於勞謙,其用有光,其聞
 莫朽。惠莊、惠文、惠宣、隋王等,或守常而獲免,終保皇枝;或過望而包羞,竟塵青史。略陽公信魁偉之狀,起圖謀之心,福善禍淫,宜哉不令。



 贊曰:謙而受益,讓以成賢。唐屬之美,憲得其先。長不居震,剛不乘乾。讓之大者,胡可比焉。捴、範已降,同氣連枝。性習何遠,非革即睽。有善有惡,禍福不欺。



\end{pinyinscope}