\article{卷一百一十六}

\begin{pinyinscope}

 ○李暠族弟齊物齊物子復暠族弟若水李麟李國貞子錡李峘弟嶧峴李巨子則之



 李暠,淮安王神通玄孫,清河王孝節孫也。暠少孤,事母甚謹。睿宗時,累轉衛尉少卿。丁憂去職,在喪柴毀,家人
 密親未嘗窺其言笑。開元初,授汝州刺史,為政嚴簡,州境肅然。與兄昇弟暈,尤相篤睦,昇等每月自東都省暠,往來微行,州人不之覺,其清慎如此。俄入授太常少卿,三遷黃門侍郎,兼太原尹,仍充太原已北諸軍節度使。太原舊俗,有僧徒以習禪為業,及死不殮,但以尸送近郊以飼鳥獸。如是積年,土人號其地為「黃坑」。側有餓狗千數,食死人肉,因侵害幼弱,遠近患之,前後官吏不能禁止。暠到官,申明禮憲,期不再犯,發兵捕殺群狗,其風
 遂革。久之,轉太常卿,旬日,拜工部尚書、東都留守。



 開元二十一年正月,制曰:「繼好之義,雖屬邊鄙;受命以出,必在親賢。事欲重於當時,禮故崇於殊俗,選眾之舉,無出宗英。工部尚書李暠,體含柔嘉,識致明允,為公族之領袖,是朝廷之羽儀。金城公主既在蕃中,漢庭公卿非無專對,有懷於遠,夫豈能忘,宜持節充入吐蕃使,準式發遣。」以國信物一萬匹、私覿物二千匹,皆雜以五彩遣之。及還,金城公主上言,請以今年九月一日樹碑於赤嶺,
 定蕃、漢界。樹碑之日,詔張守珪、李行褘與吐蕃使莽布支同往觀焉。既而吐蕃遣其臣隨漢使分往劍南及河西、磧西,歷告邊州曰:「兩國和好,無相侵掠。」漢使告亦如之。以暠奉使稱職,轉吏部尚書。時吏部告身印與曹印文同,行用參雜,難以區分,暠奏請準司勛兵部印文例,加「官告」兩字,至今行之。



 暠風儀秀整,所歷皆以威重見稱,朝廷稱其有宰相之望。累封武都縣伯,俄為太子少傅。病卒,年六十餘,贈益州大都督。



 齊物,淮安王神通子、
 鹽州刺史銳孫也。齊物無學術,在官嚴整。開元二十四年後,歷懷、陜二州刺史。齊物天寶初開砥柱之險,以通流運,於石中得古鐵犁鏵,有「平陸」字,因改河北縣為平陸縣,加齊物銀青光祿大夫,為鴻臚卿、河南尹。齊物與右相李適之善,適之為林甫所構貶官,齊物坐謫竟陵太守。入為司農、鴻臚卿。至德初,拜太子賓客,遷刑部尚書、鳳翔尹、太常卿、京兆尹。為政發官吏陰事,以察為能,於物少恩,而清廉自飭,人吏莫敢抵犯。晚年除太子太
 傅、兼宗正卿。上元二年五月卒,輟朝一日。詔曰:「故金紫光祿大夫、太子太傅、兼宗正卿齊物,宗室珪璋,士林楨幹,清廉獨斷,剛毅不群。歷踐周行,備經中外,威名益振,忠效彌彰。三尹神州,一登會府,擒奸掩鉤距之術,恤獄正喉舌之官。遂令調護儲闈,再登師傅,從容賓友,師長官僚。桑榆之時,壯志逾勵;松柏之性,晚歲常堅。天不憖遺,奄然殂謝,念親感舊,深軫於懷。宜錫寵章,載光營魄。可贈太子太師。」



 子復,字初陽,以父廕累官至江陵府司
 錄。精曉吏道,衛伯玉厚遇之,府中之事,多以咨委。性苛刻,為伯玉所信,奏為江陵縣令,遷少尹,歷饒州、蘇州刺史,皆著政聲。李希烈背叛,荊南節度張伯儀數出兵,為希烈所敗,朝廷憂之。以復久在江陵,得軍民心,復方在母喪,起為江陵少尹、兼御史中丞,充節度行軍司馬。伯儀既受代,以復為容州刺史、兼御史中丞,充本管招討使,加檢校常侍。先時西原叛亂,前後經略使征討反者,獲其人皆沒為官奴婢,配作坊重役,復乃令訪其親屬,
 悉歸還之。在容州三歲,南人安悅。遷廣州刺史、兼御史大夫、嶺南節度觀察使。會安南經略使高正平、張應相次卒官,其下參佐偏裨李元度、胡懷義等阻兵,黷亂州縣,奸贓狼藉。復誘懷義杖殺之,奏元度流於荒裔。又勸導百姓,令變茅屋為瓦舍。瓊州久陷於蠻獠中,復累遣使喻之,因奏置瓊州都督府以綏撫之。復曉於政道,所在稱理,徵拜宗正卿,加檢校工部尚書。未一歲,會華州節度李元諒卒,以復為華州刺史、潼關防禦鎮國軍使,仍
 檢校戶部尚書,兼御史大夫。



 貞元十年,鄭滑節度使李融卒,軍中潰亂,以復檢校兵部尚書,兼滑州刺史、義成軍節度、鄭滑觀察營田等使、兼御史大夫。復到任,置營田數百頃,以資軍食,不率於民,眾皆悅之。十二年,加檢校左僕射。十三年四月卒官,年五十九。廢朝三日,贈司空。賻布帛米粟有差。復久典方面,積財頗甚,為時所譏。



 若水,齊物族弟,累官至左金吾大將軍,兼通事舍人。容貌甚偉,在館三十年,多識舊儀,每宣勞贊導,周旋俯仰,
 有可觀者。建中元年八月卒。



 李麟,皇室之疏屬,太宗之從孫也。父浚,開元初置十道按察使,精選吏才,以浚為潤州刺史、江南東道按察使。轉虢潞二州刺史,益州大都督府長史、攝御史大夫、劍南節度按察使。所歷以誠信待物,稱為良吏。八年卒,贈戶部尚書,謚曰誠。



 麟以父任補職,累授京兆府戶曹。開元二十二年,舉宗室異能,轉殿中侍御史,歷戶部、考功、吏部三員外郎。天寶元年,遷郎中,尋改諫議大夫。五載,
 充河西、隴右、磧西等道黜陟使,稱旨,遷給事中。七載,遷兵部侍郎。同列楊國忠專權,不悅麟同職,宰臣奏麟以本官權知禮部貢舉。俄而國忠為御史大夫,麟復本官。十一載,遷銀青光祿大夫、國子祭酒。十四年七月,以本官出為河東太守、河東道採訪使,為政清簡,民吏稱之。其年冬,祿山構逆,朝廷以麟儒者,恐非禦侮之用,仍以將軍呂崇賁代還。復以祭酒歸朝,賜爵渭源縣男。六月,玄宗幸蜀,麟奔赴行在。既至成都,拜戶部侍郎,兼左丞。
 遷憲部尚書。至德二年正月,拜同中書門下平章事。時扈從宰相韋見素、房琯、崔渙已赴鳳翔,俄而崔圓繼去,玄宗以麟宗室子,獨留之,行在百司,麟總攝其事。其年十一月,從上皇還京,策勛行賞,加金紫光祿大夫、刑部尚書、同中書門下三品,進封褒國公。



 時張皇后干預朝政,殿中監李輔國以翊衛肅宗之勞,判天下兵馬事,充元帥府行軍司馬,勢傾同朝。宰相苗晉卿、崔圓已下懼其威權,傾心事之,唯麟正身謹事,無所依附,輔國不悅。
 乾元元年,罷麟知政事,守太子少傅。二年八月卒,時年六十六,贈太子太傅,賻絹二百匹。葬日,詔京兆府差官護送,官給所須。麟好學能文,嘗編聚皇朝已來制集五十卷,行於時。



 李國貞,淮安王神通子、淄川王孝同之曾孫。父廣業,劍州長史。國貞本名若幽,性剛正,有吏才,歷安定、扶風錄事參軍,皆稱職。乾元中累遷長安令,尋拜河南尹。會史思明逼城,元帥李光弼東保河陽,國貞領官吏寓於陜。
 數月,徵為京兆尹。上元初,改成都尹、兼御史大夫,充劍南節度使。入為殿中監。二年八月,遷戶部尚書、兼御史大夫,持節充朔方、鎮西、北庭、興平、陳鄭等節度行營兵馬及河中節度都統處置使,鎮於絳,賜名國貞。既至,又加充管內河中、晉、絳、慈、隰、沁等州觀察處置等使,餘並如故。



 國貞既至絳,屬軍中素無儲積,百姓饑饉,難為聚斂,將士等糧賜多闕。國貞頻以狀聞,未報。軍中喧喧怨讀,左右以告國貞,國貞喻之曰:「軍將何苦如是,已為奏
 聞,終有所給。」信宿軍亂,攻國貞,夜燒衙城門。國貞莫知所圖,左右勸國貞棄城遁去,國貞曰:「吾銜命為將,不能靖難,安可棄城乎!」左右固勸回避,乃隱於州獄,詐負縲紲。會國貞麾下為賊所擒,因指所在,遂於獄中執國貞,將害之,國貞曰:「軍中乏糧,已有陳請,人不堪賦,予無負於將士耳。」眾引退。突將王元振獨曰:「今日之事,豈須問焉!」抽刀害國貞及二男、三大將。



 國貞有風採,清白守法,為政急於操下,時論以辨吏稱之。追贈揚州大都督。



 子
 錡,以父廕貞元中累至湖、杭二州刺史。多以寶貨賂李齊運,由是遷潤州刺史兼鹽鐵使,持積財進奉,以結恩澤,德宗甚寵之。錡恃恩驕恣,有浙西人布衣崔善貞詣闕上封,論錡罪狀,而德宗械送賜錡,錡遂坑殺善貞,天下切齒。乃增置兵額,選善弓矢者聚之一營,名曰「挽硬隨身」;以胡、奚雜類虯須者為一將,名曰「蕃落健兒」。德宗復於潤州置鎮海軍,以錡為節度使,罷其鹽鐵使務。錡雖罷其利權,且得節度,反狀未發。



 憲宗即位已二年,諸
 道倔強者入朝,而錡不自安,亦請入朝,乃拜錡左僕射。錡乃署判官王澹為留後。既而遷延發期,澹與中使頻喻之,不悅,遂諷將士以給冬衣日殺澹而食之。監軍使聞亂,遣衙將趙錡慰喻,又臠食之。復以兵注中使之頸,錡佯驚救解之,囚於別館。遂稱兵,飾五劍,分授管內鎮將,令殺刺史。於是常州刺史顏防用客李雲謀,矯制傳檄於蘇、杭、湖、睦等州,遂殺其鎮將李深;湖州辛秘亦殺其鎮將趙惟忠;而蘇州刺史李素為鎮將姚志安所系,
 釘於船舷,生致於錡,未至而錡敗,得免。



 初,錡以宣州富饒,有並吞之意,遣兵馬使張子良、李奉仙、田少卿領兵三千分略宣、池等州。三將夙有向順志,而錡甥裴行立亦思向順,其密謀多決於行立,乃回戈趣城,執錡於幕,縋而出之,斬於闕下,年六十七。其「挽硬」、「蕃落」將士,或投井自縊,紛紛枕藉而死者,不可勝紀。



 宰相鄭絪等議錡所坐,親疏未定,乃召兵部郎中蔣武問曰:「詔罪李錡一房,當是大功內耶?」武曰:「大功是錡堂兄弟,即淮安王神
 通之下,淮安有大功於國,不可以孽孫而上累。」又問:「錡親兄弟從坐否?」武曰:「錡親兄弟是若幽之子,若幽有死王事之功,如令錡兄弟從坐,若幽即宜削籍,亦所未安。」宰相頗以為然,故誅錡詔下,唯止元惡一房而已。



 李峘,太宗第三子吳王恪之孫。恪第三子琨生信安王禕,禕生三子:峘、嶧、峴。峘志行修立,天寶中為南宮郎,歷典諸曹十餘年。居父喪,哀毀得禮,服闋,以郡王子例封趙國公。楊國忠秉政,郎官不附己者悉出於外,峘自考
 功郎中出為睢陽太守。尋而弟峴出為魏郡太守,兄弟夾河典郡,皆以理行稱。十四載,入計京師。屬祿山之亂,玄宗幸蜀,峘奔赴行在,除武部侍郎,兼御史大夫。俄拜蜀郡太守、劍南節度採訪使。上皇在成都,健兒郭千仞夜謀亂,上皇御玄英樓招諭,不從,峘與六軍兵馬使陳玄禮等平之,以功加金紫光祿大夫。時峴為鳳翔太守,匡翊肅宗,兄弟俱效勛力。從上皇還京,為戶部尚書,峴為御史大夫,兼京兆尹,封梁國公。兄弟同制封公。



 乾元
 初,兼御史大夫,持節都統淮南、江南、江西節度、宣慰、觀察處置等使。二年,以宋州刺史劉展握兵河南,有異志,乃陽拜展淮南節度使,而密詔揚州長史鄧景山與峘圖之。時展徒黨方強,既受詔,即以兵渡淮。景山、峘拒之壽春,為展所敗。峘走渡江,保丹陽,坐貶袁州司馬。寶應二年,病卒於貶所,追贈揚州大都督,官給遞乘,護柩還京。



 初,峘為戶部尚書,峴為吏部尚書、知政事,嶧為戶部侍郎、銀青光祿大夫,兄弟同居長興里第,門列三戟,兩
 國公門十六戟,一、三品門十二戟,榮耀冠時。嶧位終蜀州刺史。



 峴,樂善下士,少有吏乾。以門廕入仕,累遷高陵令,政術知名。特遷萬年令、河南少尹、魏郡太守;入為金吾將軍,遷將作監,改京兆府尹,所在皆著聲績。天寶十三載,連雨六十餘日,宰臣楊國忠惡其不附己,以雨災歸咎京兆尹,乃出為長沙郡太守。時京師米麥踴貴,百姓謠曰:「欲得米粟賤,無過追李峴。」其為政得人心如此。至德初,朝廷務收才傑,以清寇難,峴召至行在,拜扶風
 太守、兼御史大夫。至德二年十二月,制曰:「銀青光祿大夫、守禮部尚書李峴,饋軍周給,開物成務。可光祿大夫,行御史大夫,兼京兆尹,封梁國公。」乾元二年,制曰:「李峴朝廷碩德,宗室藎臣。可中書侍郎、同中書門下平章事。」與呂諲、李揆、第五錡同拜相。峴位望稍高,軍國大事,諸公莫敢言,皆獨決於峴,由是諲等銜之。



 初,李輔國判行軍司馬,潛令官軍於人間聽察是非,謂之察事。忠良被誣構者繼有之,須有追呼,諸司莫敢抗。御史臺、大理寺
 重囚在獄,推斷未了,牒追就銀臺,不問輕重,一時釋放,莫敢違者。每日於銀臺門決天下事,須處分,便稱制敕,禁中符印,悉佩之出入。縱有敕,輔國押署,然後施行。及峴為相,叩頭論輔國專權亂國,上悟,賞峴正直,事並變革。輔國以此讓行軍司馬,請歸本官,察事等並停,由是深怨峴。



 鳳翔七馬坊押官,先頗為盜,劫掠平人,州縣不能制,天興縣令知捕賊謝夷甫擒獲決殺之。其妻進狀訴夫冤。輔國先為飛龍使,黨其人,為之上訴,詔監察御
 史孫鎣推之。鎣初直其事。其妻又訴,詔令御史中丞崔伯陽、刑部侍郎李曄、大理卿權獻三司訊之,三司與鎣同。妻論訴不已,詔令侍御史毛若虛覆之,若虛歸罪於夷甫,又言伯陽等有情,不能質定刑獄。伯陽怒,使人召若虛,詞氣不順。伯陽欲上言之,若虛先馳謁,告急於肅宗,云:「已知,卿出去。」若虛奏曰:「臣出即死。」上因留在簾內。有頃,伯陽至,上問之,伯陽頗言若虛順旨,附會中人。上怒,叱出之。伯陽貶端州高要尉,權獻郴州桂陽尉,鳳翔尹嚴向及
 李曄皆貶嶺下一尉,鎣除名長流播州。峴以數人咸非其罪,所責太重,欲理之,遂奏:「若虛希旨用刑,不守國法,陛下若信之重輕,是無御史臺。」上怒峴言,出峴為蜀州刺史。時右散騎常侍韓擇木入對,上謂之曰:「峴欲專權耶?何乃云任毛若虛是無御史臺也?令貶蜀州刺史,朕自覺用法太寬。」擇木對曰:「峴言直,非專權。陛下寬之,祗益聖德爾。」



 代宗即位,徵峴為荊南節度、江陵尹,知江淮選補使。入為禮部尚書,兼宗正卿。屬鑾輿幸陜,峴由商
 山路赴行在。既還京師,拜峴為黃門侍郎、同中書門下平章事。故事,宰臣不於政事堂邀客,時海內多務,宰相元載等見中官傳詔命至中書者,引之升政事堂,仍置榻待之;峴為宰相,令去其榻。奏請常參官各舉堪任諫官、憲官者,不限人數。



 初收東京,受偽官陳希烈已下數百人,崔器希旨深刻,奏皆處死;上意亦欲懲勸天下,欲從器議。時峴為三司使,執之曰:「夫事有首從,情有輕重,若一概處死,恐非陛下含弘之義,又失國家惟新之典。
 且羯胡亂常,無不凌據,二京全陷,萬乘南巡,各顧其生,衣冠蕩覆。或陛下親戚,或勛舊子孫,皆置極法,恐乖仁恕之旨。昔者明王用刑,殲厥渠魁,脅從罔理。況河北殘寇未平,官吏多陷,茍容漏網,適開自新之路,若盡行誅,是堅叛逆之黨,誰人更圖效順?困獸猶鬥,況數萬人乎!」崔器、呂諲,皆守文之吏,不識大體,殊無變通。廷議數日,方從峴奏,全活甚眾。其料敵決事皆此類。竟為中官所擠,罷知政事,為太子詹事,尋遷吏部尚書,知江淮舉選,
 置銓洪州。明年,改檢校兵部尚書,兼衢州刺史。永泰二年七月以疾終,時年五十八。



 李巨,曾祖父虢王鳳,高祖之第十四子也。鳳孫邕,嗣虢王,巨即邕之第二子也。剛銳果決,頗涉獵書史,好屬文。開元中為嗣虢王。天寶五載,出為西河太守。皇太子杜良娣之妹婿柳勣陷詔獄,巨母扶餘氏,吉溫嫡母之妹也,溫為京兆士曹,推勣之黨,以徐徵等往來巨家,資給之,由是坐貶義陽郡司馬。六載,御史中丞楊慎矜為李
 林甫、王鉷構陷得罪,其黨史敬忠亦伏法。以巨與敬忠相識,坐解官,於南賓郡安置。又起為夷陵郡太守。及祿山陷東京,玄宗方擇將帥,張垍言巨善騎射,有謀略,玄宗追至京師。楊國忠素與巨相識,忌之,謂人曰:「如此小兒,豈得令見人主!」經月餘日不得見。玄宗使中官召入奏事,玄宗大悅,遂令中官劉奉庭宣敕令宰相與巨語,幾亭午,方出。國忠頗怠,對奉庭謂巨曰:「比來人多口打賊,公不爾乎?」巨曰:「不知若個軍將能與相公手打賊乎?」
 尋授陳留譙郡太守、攝御史大夫、河南節度使。翌日,巨稱官銜奉謝,玄宗驚曰:「何得令攝?」即日詔兼御史大夫。巨奏曰:「方今艱難,恐為賊所詐,如忽召臣,不知何以取信?」玄宗劈木契分授之,遂以臣兼統嶺南節度使何履光、黔中節度使趙國珍、南陽節度使魯炅,先領三節度事。有詔貶炅為果毅,以潁川太守來瑱兼御史中丞代之。巨奏曰:「若炅能存孤城,其功足以補過,則何以處之?」玄宗曰:「卿隨宜處置之。」巨至內鄉,趣南陽,賊將畢思琛
 聞之,解圍走。巨趣何履光、趙國珍同至南陽,宣敕貶炅,削其章服,令隨軍效力。至日晚,以恩命令炅復位。



 至德二年,為太子少傅。十月,收西京,為留守、兼御史大夫。三年夏四月,加太子少師、兼河南尹,充東京留守,判尚書省事,充東畿採訪等使。於城市橋梁稅出入車牛等錢以供國用,頗有乾沒,士庶怨讟。後與妃張氏不睦,張氏即皇后從父妹也。宗正卿李遵構之,發其所犯贓賄,貶為遂州刺史。屬劍南東川節度兵馬使、梓州刺史段子
 璋反,以眾襲節度使李奐於綿州,路經遂州,巨蒼黃修屬郡禮迎之,為子璋所殺。



 子則之,以宗室歷官,好學,年五十餘,每執經詣太學聽受。嗣曹王皋自荊南來朝,稱薦之,貞元二年,自睦王府長史遷左金吾衛大將軍,以從父甥竇申追游無閑親累,貶昭州司馬。



 史臣曰:暠孝友清慎,居官有稱;齊物貞廉整肅,復節制權謀;國貞清白守法,皆神通之曾玄,宗室之翹楚。錡之為逆,不累其親,前人之積德彰矣,當朝之用法明矣。然暠
 發人陰私,齊物積財興議,國貞急於操下,皆尺之短也。麟修整,峘循良,匪躬立事,始終無玷者,皆宗室之英也。峴之剛正才略,有足可稱。初為國忠所憎,終沮朝恩之勢。處群邪之內,堅獨正之心,是不吐也;活東都之命,是不茹也。庶幾乎仲山甫之道焉!巨以剛銳果決,亦可嘉焉,終以贓賄貪殘,良可痛也。



 贊曰:宗室賢良,枝葉茂盛。最尤者誰?峴獨守正。



\end{pinyinscope}