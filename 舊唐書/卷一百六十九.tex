\article{卷一百六十九}

\begin{pinyinscope}

 ○韋夏卿王正雅族孫凝柳公綽子仲郢孫璧玭弟公權伯父子華子華子公度崔玄亮溫造子璋郭承嘏殷侑孫盈孫徐晦



 韋夏卿,字雲客,杜陵人。父迢,檢校都官郎中、嶺南節度
 行軍司馬。夏卿苦學,大歷中與弟正卿俱應制舉,同時策入高等,授高陵主簿。累遷刑部員外郎。時久旱蝗,詔於郎官中選赤畿令,改奉天縣令。以課最第一,轉長安令。改吏部員外郎,轉本司郎中,拜給事中。出為常州刺史。夏卿深於儒術,所至招禮通經之士。時處士竇群寓於郡界,夏卿以其所著史論,薦之於朝,遂為門人。改蘇州刺史。貞元末,徐州張建封卒,初授夏卿徐州行軍司馬,尋授徐泗濠節度使。夏卿未至,建封子愔為軍人立
 為留後,因授旄鉞。征夏卿為吏部侍郎,轉京兆尹、太子賓客,檢校工部尚書、東都留守,遷太子少保。卒時年六十四,贈左僕射。



 夏卿有風韻,善談宴,與人同處,終年而喜慍不形於色。撫孤侄,恩逾己子,早有時稱。其所與游闢之賓佐,皆一時名士。為政務通適,不喜改作。始在東都,傾心闢士,頗得才彥,其後多至卿相,世謂之知人。



 王正雅,字光謙,其先太原尹東都留守翃之子。伯父翊,代宗朝御史大夫,以貞亮鯁直,名於當代,卒謚曰忠惠。
 正雅少時,以孝行修謹聞。元和初,舉進士,登甲科,禮部侍郎崔邠甚知之,累從職使府。元和十一年,拜監察御史,三遷為萬年縣令。



 當穆宗時,京邑號為難理,正雅抑強扶弱,政甚有聲。會柳公綽為京兆尹,上前褒稱,穆宗命以緋衣銀章,就縣宣賜。遷戶部郎中,尋加知臺雜事,再遷太常少卿,出為汝州刺史,充本州防禦使。有中人為監軍,怙權干政,正雅不能堪,乃謝病免。



 入為大理卿。會宋申錫事起,獄自內出,卒無證驗。是時王守澄之威
 權,鄭注之寵勢,雖宰相重臣,無敢顯言其事者。唯正雅與京兆尹崔綰上疏,請出造事者,付外考驗其事,別具狀聞。由是獄情稍緩,申錫止於貶官,中外翕然推重之。太和五年十一月卒,贈左散騎常侍。



 正雅從弟重,翊之子也,位止河東令。重子眾仲,登進士第,累官衡州刺史。眾仲子凝。



 凝,字致平,少孤,宰相鄭肅之甥,少依舅氏。年十五,兩經擢第。嘗著《京城六崗銘》,為文士所稱。再登進士甲科。崔璪領鹽鐵,闢為巡官。歷佐梓潼、宣歙使幕。宰
 相崔龜從奏為鄠縣尉、集賢校理,遷監察御史,轉殿中。宰相崔鉉出鎮揚州,奏為節度副使。入為起居郎,歷禮部、兵部、考功三員外。遷司封郎中、長安令。中丞鄭處誨奏知臺雜,換考功郎中,遷中書舍人。時政不協,出為同州刺史,賜金紫。暮年,移疾華州敷水別墅。逾年,以禮部侍郎征。



 凝性堅正,貢闈取士,拔其寒俊,而權豪請托不行,為其所怒,出為商州刺史。明年,檢校右散騎常侍、潭州刺史、湖南團練觀察使。入為兵部侍郎,領鹽鐵轉運
 使。又以不奉權幸,改秘書監。出為河南尹、檢校禮部尚書、宣州刺史、宣歙觀察使。凝咸通中兩佐宣城使幕,備究人之利病,滌除積弊,民俗阜康。



 逾歲,黃巢自嶺表北歸,大掠淮南,攻圍和州。凝令牙將樊儔率師據採石以援之。儔犯令,凝即斬之以徇,命別將烏穎代儔赴援,竟解歷陽之圍。賊怒,引眾攻宣城。大將王涓請出軍逆戰,凝曰:「賊忿恚而來,宜持重待之。彼眾我寡,萬一不捷,則州城危矣!」涓銳意請行,凝即閱集丁壯,分守要害,登陴
 設備。涓果戰死。賊乘勝而來,則守有備矣。賊為梯沖之具,急攻數月,御備力殫,吏民請曰:「賊之兇勢不可當,願尚書歸款退之,懼覆尚書家族。」凝曰:「人皆有族,予豈獨全?誓與此城同存亡也。」既而賊退去,時乾符五年也。其年夏,疾甚,有大星墜於正寢。八月卒於郡,時年五十八。無子,以弟子鑣為嗣。鑣兄鉅,位終兵部侍郎。



 柳公綽,字起之,京兆華原人也。祖正禮,邠州士曹參軍。父子溫,丹州刺史。公綽幼聰敏。年十八,應制舉,登賢良
 方正、直言極諫科,授秘書省校書郎,貞元元年也。貞元四年,復應制舉,再登賢良方正科,時年二十一。制出,授渭南尉。



 公綽性謹重,動循禮法。屬歲饑,其家雖給,而每飯不過一器。歲稔復初。家甚貧,有書千卷,不讀非聖之書。為文不尚浮靡。慈隰觀察使姚齊梧奏為判官,得殿中侍御史。冬,薦授開州刺史,入為侍御史,再遷吏部員外郎。武元衡罷相鎮西蜀,與裴度俱為元衡判官,尤相善。先度入為吏部郎中,度以詩餞別,有「兩人同日事征
 西,今日君先捧紫泥」之句。



 元和初,憲宗頗出游畋,銳意用兵;公綽欲因事諷諫。五年十一月,獻《太醫箴》一篇,其辭曰:



 天布寒暑,不私於人。品類既一,崇高以均。惟謹好愛,能保其身。清凈無瑕,輝光以新。寒暑滿天地之間,浹肌膚於外;好愛溢耳目之前,誘心知於內。清潔為隄,奔射猶敗,氣行無章,隙不在大。睿聖之姿,清明絕俗;心正無邪,志高寡欲。謂天高矣,氣蒙晦之;謂地厚矣,橫流潰之。聖德超邁,萬方賴之。飲食所以資身也,過則生患;衣
 服所以稱德也,侈則生慢。唯過與侈,心必隨之,氣與心流,疾亦伺之。聖心不惑,孰能移之?畋游恣樂,流情蕩志;馳騁勞形,吒叱傷氣。惟天之重,從禽為累。不養其外,前修所忌。聖心非之,孰敢違之。人乘氣生,嗜欲以萌,氣離有患,氣凝則成。巧必喪真,智必誘情,去彼煩慮,在此誠明。醫之上者,理於未然,患居慮後,防處事先。心靜樂行,體和道全,然後能德施萬物,以享億年。聖人在上,各有攸處。庶政有官,群藝有署。臣司太醫,敢告諸御。



 憲宗深
 嘉之。翌日,降中使獎勞之,曰:「卿所獻之文云:『氣行無間,隙不在大。』何憂朕之深也?」逾月,拜御史中丞。



 公綽素與裴垍厚,李吉甫出鎮淮南,深怨垍。六年,吉甫復輔政,以公綽為潭州刺史、兼御史中丞,充湖南觀察使。湖南地氣卑濕,公綽以母在京師,不可迎侍,致書宰相,乞分司洛陽,以便奉養,久不許。八年,移為鄂州刺史、鄂岳觀察使,乃迎母至江夏。



 九年,吳元濟據蔡州叛,王師討伐。詔公綽以鄂岳兵五千隸安州刺史李聽,率赴行營。公綽
 曰:「朝廷以吾儒生不知兵耶?」即日上奏,願自征行,許之。公綽自鄂濟湘江,直抵安州;李聽以廉使之禮事之。公綽謂之曰:「公所以屬鞬負弩者,豈非為兵事耶?若去戎容,被公服,兩郡守耳,何所統攝乎?以公名家曉兵,若吾不足以指麾,則當赴闕;不然,吾且署職名,以兵法從事矣。」聽曰:「唯公所命。」即署聽為鄂岳都知兵馬使、中軍先鋒、行營兵馬都虞候,三牒授之。乃選卒六千屬聽,戒其部校曰:「行營之事,一決都將。」聽感恩畏威,如出麾下。其
 知權制變,甚為當時所稱。鄂軍既在行營,公綽時令左右省問其家。如疾病、養生、送死,必厚廩給之。軍士之妻治容不謹者,沉之於江。行卒相感曰:「中丞為我輩知家事,何以報效?」故鄂人戰每克捷。



 十一年,入為給事中。李師道歸朝,遣公綽往鄆州宣諭。使還,拜京兆尹,以母憂免。



 十四年,起為刑部侍郎,領鹽鐵轉運使。轉兵部侍郎、兼御史大夫,領使如故。長慶元年,罷使,復為京兆尹、兼御史大夫。



 時河朔復叛,朝廷用兵,補授行營諸將,朝令夕
 改,驛騎相望。公綽奏曰:「自幽、鎮用兵,使命繁並,館遞匱乏,鞍馬多闕。又敕使行李人數,都無限約。其衣緋紫乘馬者,二十、三十匹,衣黃綠者,不下十匹、五匹。驛吏不得視券牒,隨口即供。驛馬既盡,遂奪路人鞍馬。衣冠士庶,驚擾怨嗟,遠近喧騰,行李將絕。伏望聖慈,聊為定限。」乃下中書條疏人數。自是吏不告勞。以言直為北司所惡,尋轉吏部侍郎。



 二年九月,遷御史大夫。韓弘病,自河中入朝。以弘守司徒、中書令,詔百僚問疾。弘遣其子達情,
 言不能接見。公綽謂其子曰:「聖上以公官重,令百司省問,異禮也。如拜君賜,宜力疾公見。安有臥令子弟傳言耶?」弘懼,挾扶而出,人皆聳然。



 三年,改尚書左丞,又拜檢校戶部尚書、襄州刺史、山南東道節度使。行部至鄧縣,縣二吏犯法,一贓賄,一舞文。縣令以公綽守法,必殺贓吏。獄具,判之曰:「贓吏犯法,法在;奸吏壞法,法亡。誅舞文者。」公綽馬害圉人,命斬之。賓客進言曰:「可惜良馬,圉人自防不至。」公綽曰:「安有良馬害人乎?」亟命殺之。牛僧孺
 罷相鎮江夏,公綽具戎容,於郵舍候之。軍吏自以漢上地高於鄂,禮太過。公綽曰:「奇章才離臺席,方鎮重宰相,是尊朝廷也。」竟以戎容見。有道士獻丹藥,試之有驗,問所從來,曰:「煉此丹於薊門。」時硃克融方叛,公綽遽謂之曰:「惜哉,至藥來於賊臣之境,雖驗何益!」乃沉之於江,而逐道士。鄧縣人鄭懷政病狂,妄稱天子,公綽捕而殺之。



 敬宗即位,加檢校左僕射。寶歷元年,入為刑部尚書。



 二年,授邠州刺史、邠寧慶節度使。所部有神策諸鎮,屯列
 要地,承前不受節度使制置,遂致北虜深入。公綽上疏論之,因詔諸鎮皆稟邠寧節度使制置。



 三年,入為刑部尚書,京兆人有姑鞭婦致死者,府斷以償死。公綽議曰:「尊毆卑非鬥,且其子在,以妻而戮其母,非教也。」竟減死。



 太和四年,復檢校左僕射、太原尹、北都留守、河東節度觀察等使。是歲,北虜遣梅祿將軍李暢以馬萬匹來市,托雲入貢。所經州府,守帥假之禮分,嚴其兵備。留館則戒卒於外,懼其襲奪。太原故事,出兵迎之。暢及界上,公
 綽使牙將祖考恭單馬勞問,待以修好之意。暢感義出涕,徐驅道中,不妄馳獵。及至,闢牙門,令譯引謁,宴以常禮。及市馬而還,不敢侵犯。陘北有沙陁部落,自九姓、六州皆畏避之。公綽至鎮,召其酋硃耶執宜,直抵雲、朔塞下,治廢柵十一所,募兵三千付之,留屯塞上,以御匈奴。其妻母來太原者,請梁國夫人對酒食問遺之。沙陁感之,深得其效。



 六年,以病求代。三月,授兵部尚書,徵還京師。四月卒,贈太子太保,謚曰成。



 公綽天資仁孝,初丁母
 崔夫人之喪,三年不沐浴。事繼親薛氏三十年,姻戚不知公綽非薛氏所生。外兄薛宮早卒,一女孤,配張毅夫,資遺甚於己子。性端介寡合,與錢微、蔣乂、杜元穎、薛存誠文雅相知,交情款密。凡六開府幕,得人尤盛。錢徽掌貢之年,鄭朗覆落,公綽將赴襄陽,首闢之,朗竟為名相。盧簡辭、崔璵、夏侯孜、韋長、李續、李拭,皆至公卿。為吏部侍郎,與舅左丞崔從同省,人士榮之。子仲郢,弟公權、公諒。



 仲郢,字諭蒙,元和十三年進士擢第,釋褐秘書省校
 書郎。牛僧孺鎮江夏,闢為從事。仲郢有父風,動修禮法,僧孺嘆曰:「非積習名教,安能及此!」入為監察御史。



 五年,遷侍御史。富平縣人李秀才,籍在禁軍,誣鄉人斫父墓柏,射殺之。法司以專殺論。文宗以中官所庇,決杖配流。右補闕蔣系上疏論之,不省。仲郢執奏曰:「聖王作憲,殺人有必死之令;聖明在上,當官無壞法之臣。今秀才犯殺人之科,愚臣備監決之任,此賊不死,是亂典章。臣雖至微,豈敢曠職?其秀才未敢行決,望別降敕處分。」乃詔
 御史蕭傑監之。傑又執奏。帝遂詔京兆府行決,不用監之。然朝廷嘉其守法。



 會昌中,三遷吏部郎中,李德裕頗知之。武宗有詔減冗官,吏部條疏,欲牒天下州府取額外官員。仲郢曰:「諸州每冬申闕,何煩牒耶?」幸門頓塞。仲郢條理旬日,減一千二百員,時議為愜。遷諫議大夫。



 五年,準南奏吳湘獄,御史崔元藻覆按得罪。仲郢上疏理之,人皆危懼。德裕知其無私,益重之。武宗築望仙臺,仲郢累疏切諫。帝召諭之曰:「聊因舊趾增葺,愧卿忠言。」德
 裕奏為京兆尹,謝日,言曰:「下官不期太尉恩獎及此,仰報厚德,敢不如奇章門館。」德裕不以為嫌。時廢浮圖法,以銅像鑄錢。仲郢為京畿鑄錢使,錢工欲於模加新字;仲郢止之,唯淮南加新字,後竟為僧人取之為像設鐘罄。紇干皋訴表甥劉詡毆母,詡為禁軍小校,仲郢不俟奏下,杖殺。為北司所譖,改右散騎常侍,權知吏部尚書銓事。



 宣宗即位,德裕罷相,出仲郢為鄭州刺史。周墀自江西移鎮滑臺。過鄭,觀其境內大理,甚獎之。俄而墀入
 輔政,遷為河南尹。蒞事逾月,召拜戶部侍郎。居無何,墀罷知政事。同列有疑仲郢與墀善,左授秘書監。數月,復出為河南尹。以寬惠為政,言事者以為不類京兆之政。仲郢曰:「輦轂之下,彈壓為先;郡邑之治,惠養為本。何取類耶?」



 大中年,轉梓州刺史、劍南東川節度使。孔目吏邊章簡者,以貨交近幸,前後廉使無如之何。仲郢因事決殺,部內肅然,不俟行法而自理。在鎮五年,美績流聞,徵為吏部侍郎。入朝未謝,改兵部侍郎,充諸道鹽鐵轉運
 使。



 大中十二年,罷使,守刑部尚書。咸通初,轉兵部,加金紫光祿大夫、河東男、食邑三百戶。俄出為興元尹、山南西道節度使。鳳州刺史盧方乂以輕罪決部叫,數日而斃。其妻列訴,又旁引他吏,械系滿獄。仲郢召其妻謂之曰:「刺史科小罪誡人,但本非死刑,雖未出辜,其實病死。」罰方乂百直,系者皆釋,郡人深感之。因決贓吏過當,以太子賓客分司東都。逾年,為虢州刺史。數月,檢校尚書左僕射、東都留守。盜發先人墓,棄官歸華原。除華州刺
 史,不拜。數月,以本官為鄆州刺史,天平軍節度觀察等使,授節鉞於華原別墅,卒於鎮。



 初,仲郢自拜諫議後,每遷官,群烏大集於升平里第,廷樹戟架皆滿,凡五日而散。詔下,不復集,家人以為候,唯除天平,烏不集。



 仲郢嚴禮法,重氣義,嘗感李德裕之知。大中朝,李氏無祿仕者。仲郢領鹽鐵時,取德裕兄子從質為推官,知蘇州院事,令以祿利贍南宅。令孤綯為宰相,頗不悅。仲郢與綯書自明,其要云:「任安不去,常自愧於昔人;吳詠自裁,亦何
 施於今日?李太尉受責既久,其家已空,遂絕蒸嘗,誠增痛惻。」綯深感嘆,尋與從質正員官。



 仲郢以禮法自持,私居未嘗不拱手,內齋未嘗不束帶。三為大鎮,廄無名馬,衣不薰香。退公布卷,不舍晝夜。《九經》、《三史》一鈔;魏、晉已來南北史再鈔;手鈔分門三十卷,號《柳氏自備》。又精釋典,《瑜伽》、《智度大論》皆再鈔;自餘佛書,多手記要義。小楷精謹,無一字肆筆。撰《尚書二十四司箴》,韓愈、柳宗元深賞之。有文集二十卷。子珪、璧、玭。



 珪,字鎮方,大中五年登
 進士第,累闢使府,早卒。



 璧,大中九年登進士第。文格高雅。嘗為《馬嵬詩》,詩人韓琮、李商隱嘉之。馬植鎮陳許,闢為掌書記,又從植汴州。李瓚鎮桂管,奏為觀察判官。軍政不愜,璧極言不納,拂衣而去。桂府尋亂,入為右補闕。僖宗幸蜀,召充翰林學士,累遷諫議大夫,充職。



 玭應兩經舉,釋褐秘書正字。又書判拔萃,高湜闢為度支推官。逾年,拜右補闕。湜出鎮澤潞,奏為節度副使。入為殿中侍御史。李蔚鎮襄陽,闢為掌書記。湜再鎮澤潞,復為副
 使。入為刑部員外。湜為亂將所逐,貶高要尉,玭三上疏申理。湜見疏本嘆曰:「我自辨析,亦不及此。」尋出廣州節度副使。明年,黃巢陷廣州,郡人鄧承勛以小舟載玭脫禍。召為起居郎。賊陷長安,為刃所傷,出奔行在,歷諫議給事中,位至御史大夫。



 玭嘗著書誡其子弟曰:



 夫門地高者,可畏不可恃。可畏者,立身行己,一事有墜先訓,則罪大於他人。雖生可以茍取名位,死何以見祖先於地下?不可恃者,門高則自驕,族盛則人之所嫉。實藝懿行,
 人未必信;纖瑕微累,十手爭指矣。所以承世胄者,修己不得不懇,為學不得不堅。夫人生世,以無能望他人用,以無善望他人愛,用愛無狀,則曰「我不遇時,時不急賢」。亦由農夫鹵莽而種,而怨天澤之不潤,雖欲弗餒,其可得乎!



 予幼聞先訓,講論家法。立身以孝悌為基,以恭默為本,以畏怯為務,以勤儉為法,以交結為末事,以氣義為兇人。肥家以忍順,保交以簡敬。百行備,疑身之未周;三緘密,慮言之或失。廣記如不及,求名如儻來。去吝與
 驕,庶幾減過。蒞官則潔己省事,而後可以言守法;守法而後可以言養人。直不近禍,廉不沽名。廩祿雖微,不可易黎氓之膏血;榎楚雖用,不可恣褊狹之胸襟。憂與福不偕,潔與富不並。比見門家子孫,其先正直當官,耿介特立,不畏強禦;及其衰也,唯好犯上,更無他能。如其先遜順處己,和柔保身,以遠悔尤;及其衰也,但有暗劣,莫知所宗。此際幾微,非賢不達。



 夫壞名災己,辱先喪家。其失尤大者五,宜深志之。其一,自求安逸,靡甘澹泊,茍利
 於己,不恤人言。其二,不知儒術,不悅古道:懵前經而不恥,論當世而解頤;身既寡知,惡人有學。其三,勝己者厭之,佞己者悅之,唯樂戲譚,莫思古道。聞人之善嫉之,聞人之惡揚之。浸漬頗僻,銷刻德義,簪裾徒在,廝養何殊。其四,崇好慢游,耽嗜曲糵,以銜杯為高致,以勤事為俗流,習之易荒,覺已難悔。其五,急於名宦,暱近權要,一資半級,雖或得之;眾怒群猜,鮮有存者。茲五不是,甚於痤疽。痤疽則砭石可瘳,五失則巫醫莫及。前賢炯戒,方冊
 具存,近代覆車,聞見相接。



 夫中人已下,修辭力學者,則躁進患失,思展其用;審命知退者,則業荒文蕪,一不足採。唯上智則研其慮,博其聞,堅其習,精其業,用之則行,舍之則藏。茍異於斯,豈為君子?



 初公綽理家甚嚴,子弟克稟誡訓,言家法者,世稱柳氏云。



 公權,字誠懇。幼嗜學,十二能為辭賦。元和初,進士擢第,釋褐秘書省校書郎。李聽鎮夏州,闢為掌書記。穆宗即位,入奏事,帝召見,謂公權曰:「我於佛寺見卿筆跡,思之久矣。」即日拜右拾遺,
 充翰林侍書學士。遷右補闕、司封員外郎。穆宗政僻,嘗問公權筆何盡善,對曰:「用筆在心,心正則筆正。」上改容,知其筆諫也。歷穆、敬、文三朝,侍書中禁。公綽在太原,致書於宰相李宗閔云:「家弟苦心辭藝,先朝以侍書見用,頗偕工祝,心實恥之,乞換一散秩。」乃遷右司郎中,累換司封、兵部二郎中、弘文館學士。



 文嘗思之,復召侍書,遷諫議大夫。俄改中書舍人,充翰林書詔學士。每浴堂召對,繼燭見跋,語猶未盡,不欲取燭,宮人以蠟淚揉紙繼
 之。從幸未央宮,苑中駐輦謂公權曰:「我有一喜事,邊上衣賜,久不及時,今年二月給春衣訖。」公權前奉賀,上曰:「單賀未了,卿可賀我以詩。」宮人迫其口進,公權應聲曰:「去歲雖無戰,今年未得歸。皇恩何以報,春日得春衣。」上悅,激賞久之。便殿對六學士,上語及漢文恭儉,帝舉袂曰:「此浣濯者三矣。」學士皆贊詠帝之儉德,唯公權無言。帝留而問之,對曰:「人主當進賢良,退不肖,納諫諍,明賞罰。服浣濯之衣,乃小節耳。」時周墀同對,為之股慄,公權
 辭氣不可奪。帝謂之曰:「極知舍人不合作諫議,以卿言事有諍臣風彩,卻授卿諫議大夫。」翌日降制,以諫議知制誥,學士如故。



 開成三年,轉工部侍郎,充職。嘗入對,上謂曰:「近日外議如何?」公權對曰:「自郭旼除授邠寧,物議頗有臧否。」帝曰:「旼是尚父之從子,太皇太后之季父,在官無過。自金吾大將授邠寧小鎮,何事議論耶?」公權曰:「以旼勛德,除鎮攸宜。人情論議者,言旼進二女入宮,致此除拜,此信乎?」帝曰:「二女入宮參太后,非獻也。」公權曰:「
 瓜李之嫌,何以戶曉?」因引王珪諫太宗出廬江王妃故事。帝即令南內使張日華送二女還旼。公權忠言匡益,皆此類也。累遷學士承旨。



 武宗即位,罷內職,授右散騎常侍。宰相崔珙用為集賢學士、判院事。李德裕素待公權厚,及為珙奏薦,頗不悅。左授太子詹事,改賓客。累遷金紫光祿大夫、上柱國、河東郡開國公、食邑二千戶。復為左常侍、國子祭酒。歷工部尚書。咸通初,改太子少傅,改少師,居三品、二品班三十年。六年卒,贈太子太師,時
 年八十八。



 公權初學王書,遍閱近代筆法,體勢勁媚,自成一家。當時公卿大臣家碑板,不得公權手筆者,人以為不孝。外夷入貢,皆別署貨貝,曰此購柳書。上都西明寺《金剛經碑》備有鐘、王、歐、虞、褚、陸之體,尤為得意。文宗夏日與學士聯句,帝曰:「人皆苦炎熱,我愛夏日長。」公權續曰:「薰風自南來,殿閣生微涼。」時丁、袁五學士皆屬繼,帝獨諷公權兩句,曰:「辭清意足,不可多得。」乃令公權題於殿壁,字方圓五寸,帝視之,嘆曰:「鐘、王復生,無以加焉!」



 大中初,轉少師,中謝,宣宗召升殿,御前書三紙,軍容使西門季玄捧硯,樞密使崔巨源過筆。一紙真書十字,曰「衛夫人傳筆法於王右軍」;一紙行書十一字,曰「永禪師真草《千字文》得家法」;一紙草書八字,曰「謂語助者焉哉乎也」。賜錦彩、瓶盤等銀器,仍令自書謝狀,勿拘真行,帝尤奇惜之。



 公權志耽書學,不能治生;為勛戚家碑板,問遺歲時鉅萬,多為主藏豎海鷗、龍安所竊。別貯酒器杯盂一笥,緘滕如故,其器皆亡。訊海鷗,乃曰:「不測其亡。」公
 權哂曰:「銀杯羽化耳。」不復更言。所寶唯筆硯圖畫,自扃鐍之。常評硯,以青州石末為第一,言墨易冷,絳州黑硯次之。尤精《左氏傳》、《國語》、《尚書》、《毛詩》、《莊子》。每說一義,必誦數紙。性曉音律,不好奏樂。常云:「聞樂令人驕怠故也。」



 公綽伯父子華,永泰初,為嚴武西蜀判官,奏為成都令。累遷池州刺史。入為昭應令,知府東十三縣捕賊,尋檢校金部郎中、修葺華清宮使。元載欲用為京兆尹,未拜而卒。自知死日,預為墓志。有知人之明。公綽生三日,視之,
 謂其弟子溫曰:「保惜此兒,福祚吾兄弟不能及。興吾門者,此兒也。」因以起之為公綽字。



 子華二子:公器、公度。



 公度善攝生,年八十餘,步履輕便。或祈其術,曰:「吾初無術,但未嘗以元氣佐喜怒,氣海常溫耳!」位止光祿少卿。



 公器子遵。遵子璨。璨仕至宰相,自有傳。



 崔玄亮,字晦叔,山東磁州人也。玄亮貞元十一年登進士第,從事諸侯府。性雅淡,好道術,不樂趨競,久游江湖。至元和初,因知己薦達入朝。再遷監察御史,轉侍御史。
 出為密、湖、曹三郡刺史。每一遷秩,謙讓輒形於色。



 太和初,入為太常少卿。四年,拜諫議大夫,中謝日,面賜金紫。朝廷推其名望,遷右散騎常侍。



 來年,宰相宋申錫為鄭注所構,獄自內起,京師震懼。玄亮首率諫官十四人,詣延英請對,與文宗往復數百言。文宗初不省其諫,欲置申錫於法。玄亮泣奏曰:「孟軻有言:眾人皆曰殺之,未可也;卿大夫皆曰殺之,未可也;天下皆曰殺之,然後察之,方置於法。今至聖之代,殺一凡庶,尚須合於典法,況無
 辜殺一宰相乎?臣為陛下惜天下法,實不為申錫也。」言訖,俯伏嗚咽,文宗為之感悟。玄亮由此名重於朝。



 七年,以疾求為外任;宰相以弘農便其所請。乃授檢校左散騎常侍、虢州刺史。是歲七月,卒於郡所,中外無不嘆惜。



 始玄亮登第,弟純亮、寅亮相次升進士科。蕃府闢召,而玄亮最達。玄亮孫貽孫,位至侍郎。



 溫造,字簡輿,河內人。祖景倩,南鄭令。父輔國,太常丞。造幼嗜學,不喜試吏,自負節概,少所降志,隱居王屋,以漁
 釣逍遙為事。壽州刺史張建封聞風致書幣招延,造欣然謂所親曰:「此可人也。」徙家從之。建封動靜咨詢,而不敢縻以職任。及建封授節彭門,造歸下邳,有高天下之心。建封恐一旦失造,乃以兄女妻之。



 時李希烈方悖,侵寇籓鄰,屢陷郡邑。天下城鎮恃兵者,從而動搖,多逐主帥,自立留後,邀求節鉞。德宗患之,以範陽劉濟方輸忠款,但未能盡達朝廷倚賴之意;與密詔建封選特達識略之士往喻之。建封乃強署造節度參謀,使於幽州。造
 與語未訖,濟俯伏流涕曰:「濟僻在遐裔,不知天子神聖,大臣忠藎。願得率先諸侯,效以死節。」造還,建封以其名上聞。德宗愛其才,召至京師,謂之曰:「卿誰家子?年復幾何?」造對曰:「臣五代祖大雅,外五代祖李勣。臣犬馬之年三十有二。」德宗奇之,欲用為諫官,以語洩事寢。



 長慶元年,授京兆府司錄參軍。奉使河朔稱旨,遷殿中侍御史。既而幽州劉總請以所部九州聽朝旨。穆宗選可使者,或薦造。帝召而謂之曰:「朕以劉總輸忠,雖書詔便蕃,未
 盡朕之深意。以卿素能辦事,為朕此行。」造對曰:「臣府縣走吏,初受憲職,望輕事重,恐辱國命,無能諭旨。」帝曰:「我在東宮時,聞劉總請覲;及我即位,比年上書不絕,及約以行期,即喑默不報。卿識機知變,往喻我懷,無多讓也。」乃拜起居舍人,賜緋魚袋,充太原、鎮州、幽州宣諭使。造初至範陽,劉總具櫜鞬郊迎;乃宣聖旨,示以禍福。總俯伏流汗,若兵加於頸矣。及造使還,總遂移家入覲,朝廷遂以張弘靖代之。及硃克融逐弘靖,鎮州殺田弘正,朝
 廷用兵,乃先令造銜命河東、魏博、澤潞、橫海、深冀、易定等道,喻以軍期,事皆稱旨。



 俄而坐與諫議大夫李景儉史館飲酒,景儉醉謁丞相,出造為朗州刺史。在任開後鄉渠九十七里,溉田二千頃,郡人獲利,乃名為右史渠。居四年,召拜侍御史,請復置彈事硃衣、豸冠於外廊,大臣阻而不行。李祐自夏州入拜金吾,違制進馬一百五十匹。造正衙彈奏,祐股戰汗流。祐私謂人曰:「吾夜逾蔡州城擒吳元濟,未嘗心動,今日膽落於溫御史。籲,可畏
 哉!」遷左司郎中,再知雜事。尋拜御史中丞。



 太和二年十一月,宮中昭德寺火。寺在宣政殿東隔垣,火勢將及,宰臣、兩省、京兆尹、中尉、樞密,皆環立於日華門外,令神策兵士救之,晡後稍息。是日,唯臺官不到。造奏曰:「昨宮中遺火,緣臺有系囚,恐緣為奸,追集人吏堤防,所以至朝堂在後,臣請自罰三十直。其兩巡使崔蠡、姚合火滅方到,請別議責罰。」敕曰:「事出非常,臺有囚系,官曹警備,亦為周慮,即合待罪朝堂,候取進止。量罰自許,事涉乖儀。
 溫造、姚合、崔蠡各罰一月俸料。」



 造性剛褊,人或激觸,不顧貴勢,以氣凌藉。嘗遇左補闕李虞於街,怒其不避,捕祗承人決脊十下。左拾遺舒元褒等上疏論之曰:「國朝故事,供奉官街中,除宰相外,無所回避。溫造蔑朝廷典禮,凌陛下侍臣,恣行胸臆,曾無畏忌。凡事有小而關分理者,不可失也。分理一失,亂由之生。遺、補官秩雖卑,陛下侍臣也;中丞雖高,法吏也。侍臣見凌,是不廣敬;法吏壞法,何以持繩?前時中書舍人李虞仲與造相逢,造乃
 曳去引馬。知制誥崔咸與造相逢,造又捉其從人。當時緣不上聞,所以暴犯益甚。臣聞元和、長慶中,中丞行李不過半坊,今乃遠至兩坊,謂之『籠街喝道』。但以崇高自大,不思僭擬之嫌。若不糾繩,實虧彞典。」敕曰:「憲官之職,在指佞觸邪,不在行李自大;侍臣之職,在獻可替否,不在道路相高。並列通班,合知名分,如聞喧競,亦已再三,既招人言,甚損朝體。其臺官與供奉官同道,聽先後而行,道途即祗揖而過,其參從人則各隨本官之後,少相
 闢避,勿言沖突。又聞近日已來,應合導從官,事力多者,街衢之中,行李太過。自今後,傳呼前後,不得過三百步。」然造之舉奏,無所吐茹。朝廷有喪不以禮、配不以類者,悉劾之。獲偽官王果等九十餘人,杖殺南曹吏李賨等六人,刑於都市。遷尚書右丞,加大中大夫,封祁縣開國子,賜金紫。



 四年,興元軍亂,殺節度使李絳。文宗以造氣豪嫉惡,乃授檢校右散騎常侍、興元尹、山南西道節度使。造辭赴鎮,以興元兆亂之狀奏之,文宗盡悟其根本,
 許以便宜從事。帝慮用兵勞費,造奏曰:「臣計諸道征蠻之兵已回,俟臣行程至褒縣,望賜臣密詔,使受約束。比臣及興元,諸軍相續而至,臣用此足矣。」乃授造手詔四通。神策行營將董重質、河中都將溫德彞、郃陽都將劉士和等,咸令稟造之命。造行至褒城,會興元都將衛志忠征蠻回,謁見。造即留以自衛,密與志忠謀。又召亞將張丕、李少直各諭其旨。暨發褒城,以八百人為衙隊,五百人為前軍,入府分守諸門。造下車置宴,所司供帳於
 事。造曰:「此隘狹,不足以饗士卒,移之牙門。」坐定,將卒羅拜,志忠兵周環之。造曰:「吾欲問新軍去住之意。可悉前,舊軍無得錯雜。」勞問既畢,傳言令坐,有未至者,因令舁酒巡行。及酒匝,未至者皆至,牙兵圍之亦合。坐卒未悟,席上有先覺者,揮令起,造傳言叱之,因帖息不敢動。即召坐卒,詰以殺絳之狀。志忠、張丕夾階立,拔劍呼曰「殺」。圍兵齊奮,其賊首教練使丘鑄等並官健千人,皆斬首於地,血流四注。監軍楊叔元在座,遽起求哀,擁造靴
 以請命;遣兵衛出之,以俟朝旨。敕旨配流康州。其親刃絳者斬一百斷,號令者斬三斷,餘並斬首。內一百首祭李絳,三十首祭王景延、趙存約等,並投尸於江。以功就加檢校禮部尚書。



 五年四月,入為兵部侍郎,以耳疾求退。七月,檢校戶部尚書、東都留守,判東都尚書省事、東畿汝防禦使。



 造至洛中。九月,制改授河陽懷節度觀察等使。造以河內膏腴,民戶凋瘵,奏開浚懷州古秦渠枋口堰;役工四萬,溉濟源、河內、溫、武陟四縣田五千餘頃。



 七年十一月,入為御史大夫。造初赴鎮漢中,遇大雨,平地水深尺餘,乃禱雞翁山祈晴,俄而疾風驅雲,即時開齊。文宗嘗聞其事,會造入對言之,乃詔封雞翁山為侯。



 九年五月,轉禮部尚書。其年六月病卒,時年七十,贈右僕射。有文集八十卷。造於晚年積聚財貨,一無散施,時頗譏之。子璋嗣。



 璋以廕入仕,累佐使府,歷三郡刺史。咸通末,為徐泗節度使,徐州牙卒曰銀刀軍,頗驕橫。璋至,誅其惡者五百餘人,自是軍中畏法。入為京兆尹,持法
 太深,豪右一皆屏跡。會同昌公主薨,懿宗怒,殺醫官,其家屬宗枝下獄者三百人。璋上疏切諫,以為刑法太深。帝怒,貶璋振州司馬。制出,璋嘆曰:「生不逢時,死何足惜?」是夜自縊而卒。



 郭承嘏,字復卿。曾祖尚父汾陽王。祖晞,諸衛將軍。父鈞。承嘏生而秀異,乳保之年,即好筆硯。比及成童,能通《五經》。元和四年,禮部侍郎張弘靖知其才,擢升進士第,累闢使幕。歷渭南尉。入朝為監察御史,遷起居舍人。丁內艱,
 以孝聞,終喪,為侍御史,職方、兵部二員外,兵部郎中。太和六年,拜諫議大夫。頻上疏,言時政得失。文宗以鄭注為太僕卿,承嘏論諫激切,注甚懼之。本官知匭院事。九年,轉給事中。



 開成元年,出為華州刺史、兼御史中丞。詔下,兩省迭詣中書,求承嘏出麾之由。給事中盧載封還詔書,奏曰:「承嘏自居此官,繼有封駁,能奉其職,宜在瑣闥。牧守之才,易為推擇。」文宗謂宰臣曰:「承嘏久在黃扉,欲優其祿俸,暫令廉問近關。而諫列拜章,惜其稱職,甚
 美事也。」乃復為給事中。



 文宗以淮南諸道累歲大旱,租賦不登,國用多闕。及是,以度支、戶部分命宰臣鎮之。承嘏論之曰:「宰相者,上調陰陽,下安黎庶,致君堯、舜,致時清平。俾之閱簿書,算緡帛,非所宜也。」帝深嘉之,遷刑部侍郎。時因朔望,以刑法官得對,文宗從容顧問,恩禮甚厚。未及大用,以二年二月卒。承嘏身歿之後,家無餘財,喪祭所費,皆親友共給而後具。搢紳之流,無不痛惜。贈吏部尚書。



 殷侑,陳郡人。父懌。侑為兒童時,勵志力學,不問家人資產。及長,通經,以講習自娛。貞元末,以《五經》登第,精於歷代沿革禮。元和中,累為太常博士。時回紇請和親,朝廷計費五百萬緡。朝廷方用兵伐叛,費用百端,欲緩其期。乃命宗正少卿李孝誠奉使宣諭,以侑為副。侑謹重有節概,臨事俊辯。既至虜庭,可汗初待漢使,盛陳兵甲,欲臣漢使而不答拜。侑堅立不動,宣諭畢,可汗責其倨,宣言欲留而不遣。行者皆懼,侑謂虜使曰:「可汗是漢家子
 婿,欲坐受使臣拜,是可汗失禮,非使臣之倨也。」可汗憚其言,卒不敢逼。使還,拜虞部員外郎。王承宗拒命,遣侑銜命招諭之。承宗尋稟朝旨,獻德、棣二州,遣二子入朝。遷侑諫議大夫。凡朝廷之得失,悉以陳論。前後上八十四章,以言激切,出為桂管觀察使。



 寶歷元年,檢校右散騎常侍、洪州刺史,轉江西觀察使。所至以潔廉著稱。入為衛尉卿。文宗初即位,滄州李同捷叛,而王廷湊助逆,欲加兵鎮州,詔五品已上都省集議。時上銳於破賊,宰
 臣莫敢異議。獨侑以廷湊再亂河朔,方徇招懷,雖附兇徒,未甚彰露,宜且含容,專討同捷。其疏末云:「伏願以宗社安危為大計,以善師攻心為神武,以含垢安人為遠圖,以網漏吞舟為至誡。」文宗雖不納,深所嘉之。



 滄景平,以侑嘗為滄州行軍司馬。太和四年,加檢校工部尚書、滄齊德觀察使。時大兵之後,滿目荊榛,遺骸蔽野,寂無人煙。侑不以妻子之官,始至,空城而已。侑攻苦食淡,與士卒同勞苦。周歲之後,流民襁負而歸。侑上表請借耕
 牛三萬,以給流民,乃詔度支賜綾絹五萬匹,買牛以給之。數年之後,戶口滋饒,倉稟盈積,人皆忘亡。初州兵三萬,悉取給於度支。侑一歲而賦入自贍其半,二歲而給用悉周,請罷度支給賜。而勸課多方,民吏胥悅,上表請立德政碑。以功加檢校吏部尚書。侑以郭下清池縣在子城北,非便,奏移於南郭之內。



 六年,入為刑部尚書,尋復檢校吏部尚書、鄆州刺史、兼御史大夫,充天平軍節度、鄆曹濮觀察等使。自元和末,收復師道十二州為三
 鎮。朝廷務安反側,征賦所入,盡留贍軍,貫緡尺帛,不入王府。侑以軍賦有餘,賦不上供,非法也,乃上表起太和七年,請歲供兩稅、榷酒等錢十五萬貫、粟五萬碩。詔曰:「鄆、曹、濮等州,元和已來,地本殷實,自分三道,十五餘年,雖頒詔書,竟未入賦。殷侑承兵戈之後,當歉旱之餘,勤力奉公,謹身守法。才及周歲,已致阜安。而又體國輸忠,率先入貢,成三軍奉上之志,陳一境樂輸之心。尋有表章,良用嘉嘆!」尋就加檢校右僕射。



 九年,御史大夫溫造
 劾侑不由制旨,增監軍俸入,賦斂於人。上不問,以庾承宣代還。



 其年,濮州錄事參軍崔元武,於五縣人吏率斂,及縣官料錢,以私馬抬估納官,計絹一百二十匹。大理寺斷三犯俱發,以重者論。只以中私馬為重,止令削三任官。而刑部覆奏,令決杖配流。獄未決。侑奏曰:「法官不習法律,三犯不同,即坐其所重。元武所犯,皆枉法取受,準律,枉法十五匹已上絞。《律疏》云:即以贓致罪,頻犯者並累科。據元武所犯,令當入處絞刑。」疏奏,元武依刑部
 奏,決六十,流賀州。乃授侑刑部尚書。八月,檢校右僕射,復為天平軍節度使。上以溫造所奏深文故也。



 開成元年,復召為刑部尚書。時初經李訓之亂,上問侑治安之術。侑極言委任責成,宜在朝之耆德,新進小生,無宜輕用。帝深嘉之,賜錦彩三百匹。及中謝,又令中使就第賜金十斤。其年七月,檢校左僕射,出為襄州刺史、山南東道節度使。



 二年三月,以病求代,以太子賓客分司東都。十一月,復檢校右僕射,出為忠武節度、陳許蔡觀察等
 使。三年七月,卒於鎮,時年七十二,贈司空。



 侑以通經入仕,觀風撫俗,所蒞有聲。而晚年急於大用,稍通權幸,物望減於往時。子羽。



 羽太和五年登進士第,籓府闢召,不至通顯。子盈孫。



 盈孫,乾符末為成都掾。駕在西川,用為太常博士,禮學有祖風。光啟二年冬,隨駕自成都還。三年二月,駐蹕鳳翔。時宗廟為賊所焚,車駕至京,告享無所。四月,盈孫謂宰執曰:「太廟十一室,並祧廟八室,及三太后三室,因光啟元年十二月二十五日車駕出宮,其
 緣室法物神主,本司載行,至鄠縣並被盜剽奪。皇帝還宮,合先制造。」宰相鄭延昌奏曰:「太廟大殿二十二間,功績至大,計料支費不少;兼宗廟制度,損益重難,今未審依元料修奉,為復別有商量。」敕付禮院詳議。



 時博士四人,杜用勵在利州,崔澄在河中,封舜卿在巴南。獨盈孫獻議曰:「太廟制度。歷代參詳,皆符典經,難議損益。謹按舊制,十一室,二十三間,十一架。垣墉廣袤之度,堂室淺深之規,階陛等級之差,棟宇崇低之則,前古所謂奢不
 能侈,儉不能逾者也。今以朝廷帑藏方虛,費用稍廣,須資變禮,將務從宜,固不可易前聖之規模,狹大朝之制度,當憑典實,別有參詳。謹按至德二年,以太廟方修,新作神主,於長安殿安置,便行饗告之禮,如同宗廟之儀,以俟廟成,方為遷祔。當時議論,無所是非。竊知今者京城除大內正衙外,別無殿宇。伏聞先有詔旨,且以少府監大權充太廟。伏緣十一室於五間之中,陳設隘狹,伏請接續之兩頭,成十一室,薦饗之。三太后廟,即於
 監內西南,別取屋宇三間,且充廟室。候太廟修奉畢日,別議遷祔。」敕旨依奏。其神主、法物、樂懸,皆盈孫奏重修制,知禮者稱為博洽。



 龍紀元年十一月,昭宗郊祀圓丘。兩中尉楊復恭及兩樞密,皆請朝服。盈孫上疏曰:「臣昨赴齋宮,見中尉、樞密內臣,皆具朝服。臣尋前代及國朝典令,無內官朝服制度。伏以皇帝陛下,承天御歷,聖祚中興,祗見宗祧,克陳大禮,皆稟高祖、太宗之成制,必循虞、夏、商、周之舊經。軒冕服章,式遵彞憲。若內官要衣朝
 服,令依所守官本品之服。事雖無據,粗可行之。臣忝禮司,合具陳奏。」時中貴皆如宰相大臣朝服,故盈孫論之。帝雖不從,嘉其所守。轉秘書少監,卒。



 徐晦,進士擢第,登直言極諫制科,授櫟陽尉,皆自楊憑所薦。及憑得罪,貶臨賀尉,交親無敢祖送者;獨晦送至藍田,與憑言別。時故相權德輿與憑交分最深,知晦之行,因謂晦曰:「今日送臨賀,誠為厚矣,無乃為累乎!」晦曰:「晦自布衣受楊公之眷,方茲流播,爭忍無言而別?如他
 日相公為奸邪所譖,失意於外,晦安得與相公輕別?」德輿嘉其真懇,大稱之於朝。不數日,御史中丞李夷簡請為監察,晦白夷簡曰:「生平不踐公門,公何取信而見獎拔?」夷簡曰:「聞君送楊臨賀,不顧犯難,肯負國乎?」由是知名。歷殿中侍御史、尚書郎,出為晉州刺史。入拜中書舍人。寶歷元年,出為福建觀察使。二年,入為工部侍郎,出為同州刺史、兼御史中丞。太和四年,徵拜兵部侍郎。五年,為太子賓客,分司東都。晦性強直,不隨世態,當官守
 正。唯嗜酒太過,晚年喪明,乃至沉廢。以禮部尚書致仕。開成三年三月卒,贈兵部尚書。



 史臣曰:溫、柳二公,以文行飾躬,砥礪名節,當官守法,侃侃有大臣之節,而竟不登三事,位止正卿。所以知公輔之量,以和為貴。漢武帝畏汲黯而相孫弘,太宗重魏徵而委玄齡,其旨遠也。韋、崔名士,薦賢致主,綽有古風。殷司空治民,斯為循吏,而忠規壯節,至晚不衰。徐、郭讜言,鬱為佳士。如數君者,實為令人。



 贊曰:柳氏禮法,公忠節概。搏擊為優,彌綸則隘。夏卿獎拔,晦叔匡將。徐、郭之議,金玉鏘鏘。



\end{pinyinscope}