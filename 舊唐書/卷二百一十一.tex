\article{卷二百一十一}

\begin{pinyinscope}

 ○高麗百濟新羅倭國日本



 高麗者,出自扶餘之別種也。其國都於平壤城,即漢樂浪郡之故地,在京師東五千一百里。東渡海至於新羅,
 西北渡遼水至於營州,南渡海至於百濟,北至靺鞨。東西三千一百里,南北二千里。其官大者號大對盧,比一品,總知國事,三年一代,若稱職者,不拘年限。交替之日,或不相祗服,皆勒兵相攻,勝者為之。其王但閉宮自守,不能制御。次曰太大兄,比正二品。對盧以下官,總十二級。外置州縣六十餘城。大城置傉薩一,比都督。諸城置道使,比刺史。其下各有僚佐,分掌曹事。衣裳服飾,唯王五彩,以白羅為冠,白皮小帶,其冠及帶,咸以金飾。官之
 貴者,則青羅為冠,次以緋羅,插二鳥羽,及金銀為飾,衫筒袖,褲大口,白韋帶,黃韋履。國人衣褐戴弁,婦人首加巾幗。好圍棋投壺之戲,人能蹴鞠。食用籩豆、簠簋、尊俎、罍洗,頗有箕子之遺風。



 其所居必依山谷,皆以茅草葺舍,唯佛寺、神廟及王宮、官府乃用瓦。其俗貧窶者多,冬月皆作長坑,下燃;煴火以取暖。種田養蠶,略同中國。其法:有謀反叛者,則集眾持火炬競燒灼之,燋爛備體,然後斬首,家悉籍沒;守城降敵,臨陣敗北,殺人行劫者,斬;
 盜物者,十二倍酬贓;殺牛馬者,沒身為奴婢。大體用法嚴峻,少有犯者,乃至路不拾遺。其俗多淫祀,事靈星神、日神、可汗神、箕子神。國城東有大穴,名神隧,皆以十月,王自祭之。



 俗愛書籍,至於衡門廝養之家,各於街衢造大屋,謂之扃堂,子弟未婚之前,晝夜於此讀書習射。其書有《五經》及《史記》、《漢書》、範曄《後漢書》、《三國志》、孫盛《晉春秋》、《玉篇》、《字統》、《字林》;又有《文選》,尤愛重之。



 其王高建武,即前王高元異母弟也。武德二年,遣使來朝。四年,又遣使
 朝貢。高祖感隋末戰士多陷其地,五年,賜建武書曰:



 朕恭膺寶命,君臨率士,祗順三靈,綏柔萬國。普天之下,情均撫字,日月所照,咸使乂安。王既統攝遼左,世居籓服,思稟正朔,遠循職貢。故遣使者,跋涉山川,申布誠懇,朕甚嘉焉。方今六合寧晏,四海清平,玉帛既通,道路無壅。方申輯睦,永敦聘好,各保疆蕣,豈非盛美。但隋氏季年,連兵構難,攻戰之所,各失其民。遂使骨肉乖離,室家分析,多歷年歲,怨曠不申。今二國通和,義無阻異,在此所
 有高麗人等,已令追括,尋即遣送;彼處有此國人者,王可放還,務盡撫育之方,共弘仁恕之道。



 於是建武悉搜括華人,以禮賓送,前後至者萬數,高祖大喜。



 七年,遣前刑部尚書沈叔安往冊建武為上柱國、遼東郡王、高麗王,仍將天尊像及道士往彼,為之講《老子》,其王及道俗等觀聽者數千人。高祖嘗謂侍臣曰:「名實之間,理須相副。高麗稱臣於隋,終拒煬帝,此亦何臣之有!朕敬於萬物,不欲驕貴,但據有土宇,務共安人,何必令其稱臣,以
 自尊大。即為詔述朕此懷也。」侍中裴矩、中書侍郎溫彥博曰:「遼東之地,周為箕子之國,漢家玄菟郡耳!魏、晉已前,近在提封之內,不可許以不臣。且中國之於夷狄,猶太陽之對列星,理無降尊,俯同籓服。」高祖乃止。



 九年,新羅、百濟遣使訟建武,雲閉其道路,不得入朝。又相與有隙,屢相侵掠。詔員外散騎侍郎硃子奢往和解之。建武奉表謝罪,請與新羅對使會盟。



 貞觀二年,破突厥頡利可汗,建武遣使奉賀,並上封域圖。五年,詔遣廣州都督
 府司馬長孫師往收瘞隋時戰亡骸骨,毀高麗所立京觀。建武懼伐其國,乃築長城,東北自扶餘城,西南至海,千有餘里。十四年,遣其太子桓權來朝,並貢方物,太宗優勞甚至。



 十六年,西部大人蓋蘇文攝職有犯,諸大臣與建武議欲誅之。事洩,蘇文乃悉召部兵,雲將校閱,並盛陳酒饌於城南,諸大臣皆來臨視。蘇文勒兵盡殺之,死者百餘人。焚倉庫,因馳入王宮,殺建武,立建武弟大陽子藏為王。自立為莫離支,猶中國兵部尚書兼中書
 令職也,自是專國政。蘇文姓錢氏,須貌甚偉,形體魁傑,身佩五刀,左右莫敢仰視。恆令其屬官俯伏於地,踐之上馬,及下馬,亦如之。出必先布隊仗,導者長呼以闢行人,百姓畏避,皆自投坑谷。



 太宗聞建武死,為之舉哀,使持節吊祭。十七年,封其嗣王藏為遼東郡王、高麗王。又遣司農丞相裏玄獎齎璽書往說諭高麗,令勿攻新羅。蓋蘇文謂玄獎曰:「高麗、新羅,怨隙已久。往者隋室相侵,新羅乘釁奪高麗五百里之地,城邑新羅皆據有之。自
 非反地還城,此兵恐未能已。」玄奘曰:「既往之事,焉可追論?」蘇文竟不從。太宗顧謂侍臣曰:「莫離支賊弒其主,盡殺大臣,用刑有同坑阱。百姓轉動輒死,怨痛在心,道路以目。夫出師吊伐,須有其名,因其弒君虐下,敗之甚易也。」



 十九年,命刑部尚書張亮為平壤道行軍大總管,領將軍常何等率江、淮、嶺、硤勁卒四萬,戰船五百艘,自萊州汎海趨平壤。又以特進英國公李勣為遼東道行軍大總管,禮部尚書江夏王道宗為副,領將軍張士貴等
 率步騎六萬趨遼東。兩軍合勢,太宗親御六軍以會之。



 夏四月,李勣軍渡遼,進攻蓋牟城,拔之。獲生口二萬,以其城置蓋州。五月,張亮副將程名振攻沙卑城,拔之,虜其男女八千口。是日,李勣進軍於遼東城。帝次遼澤,詔曰:「頃者隋師渡遼,時非天贊,從軍士卒,骸骨相望,遍於原野,良可哀嘆。掩骼之義,誠為先典,其令並收瘞之。」國內及新城步騎四萬來援遼東,江夏王道宗率騎四千逆擊,大破之,斬首千餘級。帝渡遼水,詔撤橋梁,以堅士
 卒志。帝至遼東城下。見士卒負擔以填塹者,帝分其尤重者,親於馬上持之。從官悚動,爭齎以送城下。時李勣已率兵攻遼東城。高麗聞我有拋車,飛三百斤石於一里之外者,甚懼之。乃於城上積木為戰樓以拒飛石。勣列車發石以擊其城,所遇盡潰。又推撞車撞其樓閣,無不傾倒。帝親率甲騎萬餘與李勣會。圍其城。俄而南風甚勁,命縱火焚其西南樓,延燒城中,屋宇皆盡。戰士登城,賊乃大潰,燒死者萬餘人,俘其勝兵萬餘口,以其城
 為遼州。初,帝自定州命每數十里置一烽,屬於遼城,與太子約,克遼東,當舉烽。是日,帝命舉烽,傳入塞。



 師次白崖城,命攻之,右衛大將軍李思摩中弩矢,帝親為吮血,將士聞之,莫不感勵。其城因山臨水,四面險絕。李勣以撞車撞之,飛石流矢,雨集城中。六月,帝臨其西北,城主孫伐音潛遣使請降,曰:「臣已願降,其中有貳者。」詔賜以旗幟,曰:「必降,建之城上。」伐音舉幟於城上,高麗以為唐兵登也,乃悉降。初,遼東之陷也,伐音乞降,既而中悔,帝
 怒其反覆,許以城中人物分賜戰士。及是,李勣言於帝曰:「戰士奮厲爭先,不顧矢石者,貪虜獲耳。今城垂拔,奈何更許其降,無乃辜將士之心乎?」帝曰:「將軍言是也。然縱兵殺戮,虜其妻孥,朕所不忍也。將軍麾下有功者,朕以庫物賞之,庶因將軍贖此一城。」遂受降,獲士女一萬,勝兵二千四百,以其城置巖州,授孫伐音為巖州刺史。我軍之渡遼也,莫離支遣加尸城七百人戍蓋牟城,李勣盡虜之,其人並請隨軍自效。太宗謂曰:「誰不欲爾之
 力,爾家悉在加尸,爾為吾戰,彼將為戮矣!破一家之妻子,求一人之力用,吾不忍也!」悉令放還。



 車駕進次安市城北,列營進兵以攻之。高麗北部傉薩高延壽、南部耨薩高惠貞率高麗、靺鞨之眾十五萬來援安市城。賊中有對盧,年老習事,謂延壽曰:「吾聞中國大亂,英雄並起。秦王神武,所向無敵,遂平天下,南面為帝,北夷請服,西戎獻款。今者傾國而至,猛將銳卒,悉萃於此,其鋒不可當也。今為計者,莫若頓兵不戰,曠日持久,分遣驍雄,斷
 其饋運,不過旬日,軍糧必盡,求戰不得,欲歸無路,此不戰而取勝也。」延壽不從,引軍直進。太宗夜召諸將,躬自指麾。遣李勣率步騎一萬五千於城西嶺為陣;長孫無忌率牛進達等精兵一萬一千以為奇兵,自山北於狹谷出,以沖其後;太宗自將步騎四千,潛鼓角,偃旌幟,趨賊營北高峰之上;令諸軍聞鼓角聲而齊縱。因令所司張受降幕於朝堂之側,曰:「明日午時,納降虜於此矣!」遂率軍而進。



 明日,延壽獨見李勣兵,欲與戰。太宗遙望無
 忌軍塵起,令鼓角並作,旗幟齊舉。賊眾大懼,將分兵御之,而其陣已亂。李勣以步卒長槍一萬擊之,延壽眾敗。無忌縱兵乘其後,太宗又自山而下,引軍臨之,賊因大潰,斬首萬餘級。延壽等率其餘寇,依山自保。於是命無忌、勣等引兵圍之,撤東川梁以斷歸路。太宗按轡徐行,觀賊營壘,謂侍臣曰:「高麗傾國而來,存亡所系,一麾而敗,天佑我也!」因下馬再拜以謝天。延壽等膝行而前,拜手請命。太宗簡傉薩以下酋長三千五百人,授以戎秩,
 遷之內地。收靺鞨三千三百,盡坑之,餘眾放還平壤。獲馬三萬疋、牛五萬頭、明光甲五千領,他器械稱是。高麗國振駭,後黃城及銀城並自拔,數百里無復人煙。因名所幸山為駐蹕山,令將作造《破陣圖》,命中書侍郎許敬宗為文勒石以紀其功。授高延壽鴻臚卿,高惠真司農卿。張亮又與高麗再戰於建安城下,皆破之,於是列長圍以攻焉。



 八月,移營安市城東,李勣遂攻安市,擁延壽等降眾營其城下以招之。城中人堅守不動,每見太宗
 旄麾,必乘城鼓噪以拒焉。帝甚怒。李勣曰:「請破之日,男子盡誅。」城中聞之,人皆死戰。乃令江夏王道宗築土山,攻其城東南隅;高麗亦埤城增雉以相抗。李勣攻其西面,令拋石撞車壞其樓雉;城中隨其崩壞,即立木為柵。道宗以樹條苞壤為土,屯積以為山,其中間五道加木,被土於其上,不舍晝夜,漸以逼城。道宗遣果毅都尉傅伏愛領隊兵於山頂以防敵,土山自高而陟,排其城,城崩。會伏愛私離所部,高麗百人自頹城而戰,遂據有土
 山而塹斷之,積火縈盾以自固。太宗大怒,斬伏愛以徇。命諸將擊之,三日不能克。



 太宗以遼東倉儲無幾,士卒寒凍,乃詔班師。歷其城,城中皆屏聲偃幟,城主登城拜手奉辭。太宗嘉其堅守,賜絹百匹,以勵事君之節。



 初。攻陷遼東城,其中抗拒王師,應沒為奴婢者一萬四千人,並遣先集幽州,將分賞將士。太宗愍其父母妻子一朝分散,令有司準其直,以布帛贖之,赦為百姓。其眾歡呼之聲,三日不息。高延壽自降後,常積嘆,尋以憂死。惠
 真竟至長安。



 二十年,高麗遣使來謝罪,並獻二美女。太宗謂其使曰:「歸謂爾主,美色者,人之所重。爾之所獻,信為美麗。憫其離父母兄弟於本國,留其身而忘其親,愛其色而傷其心,我不取也。」並還之。



 二十二年,又遣右武衛將軍薛萬徹等往青丘道伐之,萬徹渡海入鴨綠水,進破其泊灼城,俘獲甚眾。太宗又命江南造大船,遣陜州刺史孫伏伽召募勇敢之士,萊州刺史李道裕運糧及器械,貯於烏胡島,將欲大舉以伐高麗。未行而帝崩。高
 宗嗣位,又命兵部尚書任雅相、左武衛大將軍蘇定方、左驍衛大將軍契苾何力等前後討之,皆無大功而還。



 乾封元年,高藏遣其子入朝,陪位於太山之下。其年,蓋蘇文死,其子男生代為莫離支,與其弟男建、男產不睦,各樹朋黨,以相攻擊。男生為二弟所逐,走據國內城死守,其子獻誠詣闕求哀。詔令左驍衛大將軍契苾何力率兵應接之。男生脫身來奔,詔授特進、遼東大都督兼平壤道安撫大使,封玄菟郡公。十一月,命司空、英國公
 李勣為遼東道行軍大總管,率裨將郭待封等以征高麗。



 二年二月,勣度遼至新城,謂諸將曰:「新城是高麗西境鎮城,最為要害。若不先圖,餘城未易可下。」遂引兵於新城西南,據山築柵,且攻且守,城中窘迫,數有降者,自此所向克捷。高藏及男建遣太大兄男產將首領九十八人,持帛幡出降,且請入朝。勣以禮延接。男建猶閉門固守。



 總章元年九月,勣又移營於平壤城南,男建頻遣兵出戰,皆大敗。男建下捉兵總管僧信誠密遣人詣軍
 中,許開城門為內應。經五日,信誠果開門,勣從兵入,登城鼓噪,燒城門樓,四面火起。男建窘急自刺,不死。十一月,拔平壤城,虜高藏、男建等。十二月,至京師,獻俘於含元宮。詔以高藏政不由己,授司平太常伯;男產先降,授司宰少卿;男建配流黔州;男生以鄉導有功,授右衛大將軍,封汴國公,特進如故。



 高麗國舊分為五部,有城百七十六,戶六十九萬七千;乃分其地置都督府九、州四十一、縣一百,又置安東都護府以統之。擢其酋渠有功
 者授都督、刺史及縣令,與華人參理百姓。乃遣左武衛將軍薛仁貴總兵鎮之,其後頗有逃散。



 儀鳳中,高宗授高藏開府儀同三司、遼東都督,封朝鮮王,居安東,鎮本蕃為主。高藏至安東,潛與靺鞨相通謀叛。事覺,召還,配流邛州,並分徙其人,散向河南、隴右諸州,其貧弱者留在安東城傍。



 高藏以永淳初卒,贈衛尉卿,詔送至京師,於頡利墓左賜以葬地,兼為樹碑。垂拱二年,又封高藏孫寶元為朝鮮郡王。聖歷元年,進授左鷹揚衛大將軍,
 封為忠誠國王,委其統攝安東舊戶,事竟不行。二年,又授高藏男德武為安東都督,以領本蕃。自是高麗舊戶在安東者漸寡少。分投突厥及靺鞨等,高氏君長遂絕矣!



 男生以儀鳳初卒於長安,贈並州大都督。子獻誠,授右衛大將軍,兼令羽林衛上下。天授中,則天嘗內出金銀實物,令宰相及南北衙文武官內擇善射者五人共賭之。內史張光輔先讓獻誠為第一,獻誠復讓右玉鈐衛大將軍薛吐摩支,摩支又讓獻誠。既而獻誠奏曰:「陛
 下令簡能射者五人,所得者多非漢官。臣恐自此已後,無漢官工射之名,伏望停寢此射。」則天嘉而從之。



 時酷吏來俊臣嘗求貨於獻誠,獻誠拒而不答,遂為俊臣所構,誣其謀反,縊殺之。則天後知其冤,贈右羽林衛大將軍,以禮改葬。



 百濟國,本亦扶餘之別種,嘗為馬韓故地,在京師東六千二百里,處大海之北,小海之南。東北至新羅,西渡海至越州,南渡海至倭國,北渡海至高麗。其王所居有東
 西兩城。所置內官曰內臣佐平,掌宣納事;內頭佐平,掌庫藏事;內法佐平,掌禮儀事;衛士佐平,掌宿衛兵事;朝廷佐平,掌刑獄事;兵官佐平,掌在外兵馬事。又外置六帶方,管十郡,其用法:叛逆者死,籍沒其家;殺人者,以奴婢三贖罪;官人受財及盜者,三倍追贓,仍終身禁錮。凡諸賦稅及風土所產,多與高麗同。其王服大袖紫袍,青錦褲,烏羅冠,金花為飾,素皮帶,烏革履。官人盡緋為衣,銀花飾冠。庶人不得衣緋紫。歲時伏臘,同於中國。其書
 籍有《五經》、子、史,又表疏並依中華之法。



 武德四年,其王扶餘璋遣使來獻果下馬。七年,又遣大臣奉表朝貢。高祖嘉其誠款,遣使就冊為帶方郡王、百濟王。自是歲遣朝貢,高祖撫勞甚厚。因訟高麗閉其道路,不許來通中國,詔遣硃子奢往和之。又相與新羅世為仇敵,數相侵伐。



 貞觀元年,太宗賜其王璽書曰:「王世為君長,撫有東蕃。海隅遐曠,風濤艱阻,忠款之至,職貢相尋,尚想徽猷,甚以嘉慰!朕自祗承寵命,君臨區宇,思弘王道,愛育黎
 元。舟車所通,風雨所及,期之遂性,咸使乂安。新羅王金真平,朕之籓臣,王之鄰國。每聞遣師,征討不息,阻兵安忍,殊乖所望。朕已對王侄信福及高麗、新羅使人,具敕通和,咸許輯睦。王必須忘彼前怨,識朕本懷,共篤鄰情,即停兵革。」璋因遣使奉表陳謝,雖外稱順命,內實相仇如故。十一年,遣使來朝,獻鐵甲雕斧。太宗優勞之,賜彩帛三千段並錦袍等。



 十五年,璋卒,其子義慈遣使奉表告哀。太宗素服哭之,贈光祿大夫,賻物二百段,遣使冊
 命義慈為柱國,封帶方郡王、百濟王。



 十六年,義慈興兵伐新羅四十餘城,又發兵以守之,與高麗和親通好,謀欲取黨項城以絕新羅入朝之路。新羅遣使告急請救。太宗遣司農丞相裏玄獎齎書告諭兩蕃,示以禍福。及太宗親征高麗,百濟懷二,乘虛襲破新羅十城。二十二年,又破其十餘城。數年之中,朝貢遂絕。



 高宗嗣位,永徽二年,始又遣使朝貢。使還,降璽書與義慈曰:



 至如海東三國,開基自久,並列疆界,地實犬牙。近代已來,遂構嫌
 隙。戰爭交起,略無寧歲。遂令三韓之氓,命懸刀俎,尋戈肆憤,朝夕相仍。朕代天理物,載深矜愍。去歲王及高麗、新羅等使並來入朝,朕命釋茲讎怨,更敦款穆。新羅使金法敏奏書:「高麗、百濟,脣齒相依,競舉兵戈,侵逼交至。大城重鎮,並為百濟所並;疆宇日蹙,威力並謝。乞詔百濟,令歸所侵之城。若不奉詔,即自興兵打取。但得故地,即請交和。」朕以其言既順,不可不許。昔齊桓列土諸侯,尚存亡國;況朕萬國之主,豈可不恤危籓!王所兼新羅
 之城,並宜還其本國;新羅所獲百濟俘虜,亦遣還王。然後解患釋紛,韜戈偃革,百姓獲息肩之願,三蕃無戰爭之勞。比夫流血邊亭,積尸疆場,耕織並廢,士女無聊,豈可同年而語矣!王若不從進止,朕已依法敏所請,任其與王決戰;亦令約束高麗,不許遠相救恤。高麗若不承命,即令契丹諸蕃渡遼澤入抄掠。王可深思朕言,自求多福,審圖良策,無貽後悔。



 六年,新羅王金春秋又表稱百濟與高麗、靺鞨侵其北界,已沒三十餘城。顯慶五年,
 命左衛大將軍蘇定方統兵討之,大破其國。虜義慈及太子隆、小王孝演、偽將五十八人等送於京師,上責而宥之。其國舊分為五部,統郡三十七,城二百,戶七十六萬。至是乃以其地分置熊津、馬韓、東明等五都督府,各統州縣,立其酋渠為都督、刺史及縣令。命右衛郎將王文度為熊津都督,總兵以鎮之。義慈事親以孝行聞,友於兄弟,時人號「海東曾、閔」。及至京,數日而卒。贈金紫光祿大夫、衛尉卿,特許其舊臣赴哭。送就孫皓、陳叔寶墓
 側葬之,並為豎碑。



 文度濟海而卒。百濟僧道琛、舊將福信率眾據周留城以叛。遣使往倭國,迎故王子扶餘豐,立為王。其西部、北部並翻城應之。時郎將劉仁願留鎮於百濟府城,道琛等引兵圍之。帶方州刺史劉仁軌代文度統眾,便道發新羅兵合契以救仁願,轉鬥而前,所向皆下。道琛等於熊津江口立兩柵以拒官軍,仁軌與新羅兵四面夾擊之,賊眾退走入柵,阻水橋狹,墮水及戰死萬餘人。道琛等乃釋仁願之圍,退保任存城。新羅
 兵士以糧盡引還,時龍朔元年三月也。



 於是道琛自稱領軍將軍,福信自稱霜岑將軍,招誘叛亡,其勢益張。使告仁軌曰:「聞大唐與新羅約誓,百濟無問老少,一切殺之。然後以國府新羅。與其受死,豈若戰亡!所以聚結自固守耳!」仁軌作書,具陳禍福,遣使諭之。道琛等恃眾驕倨,置仁軌之使於外館。傳語謂曰:「使人官職小,我是一國大將,不合自參。」不答書遣之。尋而福信殺道琛,並其兵眾,扶餘豐但主祭而已。



 二年七月,仁願、仁軌等率留
 鎮之兵,大破福信餘眾於熊津之東,拔其支羅城及尹城、大山、沙井等柵,殺獲甚眾。仍令分兵以鎮守之。福信等以真峴城臨江高險,又當沖要,加兵守之。仁軌引新羅之兵乘夜薄城,四面攀堞而上,比明而入據其城,斬首八百級,遂通新羅運糧之路。仁願乃奏請益兵,詔發淄、青、萊、海之兵七千人,遣左威衛將軍孫仁師統眾浮海赴熊津,以益仁願之眾。時福信既專其兵權,與撫餘豐漸相猜貳。福信稱疾,臥於窟室,將候扶餘豐問疾,謀
 襲殺之。扶餘豐覺而率其親信掩殺福信,又遣使往高麗及倭國請兵以拒官軍。孫仁師中路迎擊,破之。遂與仁願之眾相合,兵勢大振。於是仁師、仁願及新羅王金法敏帥陸軍進,劉仁軌及別帥杜爽、扶餘隆率水軍及糧船,自熊津江往白江以會陸軍,同趨周留城。仁軌遇扶餘豐之眾於白江之口,四戰皆捷。焚其舟四百艘,賊眾大潰,扶餘豐脫身而走。偽王子扶餘忠勝、忠志等率士女及倭眾並降。百濟諸城皆復歸順。孫仁師與劉仁
 願等振旅而還。詔劉仁軌代仁願率兵鎮守。乃授扶餘隆熊津都督,遣還本國,共新羅和親,以招輯其餘眾。



 麟德二年八月,隆到熊津城,與新羅王法敏刑白馬而盟。先祀神祇及川谷之神,而後歃血。其盟文曰:



 往者百濟先王,迷於逆順,不敦鄰好,不睦親姻。結托高麗,交通倭國,共為殘暴,侵削新羅,破邑屠城,略無寧歲。天子憫一物之失所,憐百姓之無辜,頻命行人,遣其和好。負險恃遠,侮慢天經。皇赫斯怒,恭行吊伐,旌旗所指,一戎大定。
 固可水豬宮污宅,作誡來裔,塞源拔本,垂訓後昆。然懷柔伐叛,前王之令典;興亡繼絕,往哲之通規。事必師古,傳諸曩冊。故立前百濟太子司稼正卿扶餘隆為熊津都督,守其祭祀,保其桑梓。依倚新羅,長為與國,各除宿憾,結好和親。恭承詔命,永為籓服。仍遣使人右威衛將軍魯城縣公劉仁願親臨勸諭,具宣成旨,約之以婚姻,申之以盟誓。刑牲歃血,共敦終始;分災恤患,恩若弟兄。祗奉綸言,不敢失墜,既盟之後,共保歲寒。若有棄信不恆,
 二三其德,興兵動眾,侵犯邊陲,明神鑒之,百殃是降,子孫不昌,社稷無守,禋祀磨滅,罔有遺餘。故作金書鐵契,藏之宗廟,子孫萬代,無或敢犯。神之聽之,是饗是福。



 劉仁軌之辭也。歃訖,埋幣帛於壇下之吉地,藏其盟書於新羅之廟。仁願、仁軌等既還,隆懼新羅,尋歸京師。



 儀鳳二年,拜光祿大夫、太常員外卿、兼熊津都督、帶方郡王,令歸本蕃,安輯餘眾。時百濟本地荒毀,漸為新羅所據,隆竟不敢還舊國而卒。



 其孫敬,則天朝襲封帶方郡王、
 授衛尉卿。其地自此為新羅及渤海靺鞨所分,百濟之種遂絕。



 新羅國,本弁韓之苗裔也。其國在漢時樂浪之地,東及南方俱限大海,西接百濟,北鄰高麗。東西千里,南北二千里。有城邑村落。王之所居曰金城,周七八里。衛兵三千人,設獅子隊。文武官凡有十七等。其王金真平,隋文帝時授上開府、樂浪郡公、新羅王。武德四年,遣使朝貢。高祖親勞問之,遣通直散騎侍郎庾文素往使焉,賜以
 璽書及畫屏風、錦彩三百段,自此朝貢不絕。其風俗、刑法、衣服,與高麗、百濟略同,而朝服尚白。好祭山神。其食器作柳杯,亦以銅及瓦。國人多金、樸兩姓,異姓不為婚。重元日,相慶賀燕饗,每以其日拜日月神。又重八月十五日,設樂飲宴,賚群臣,射其庭。婦人發繞頭,以彩及珠為飾,發甚長美。



 高祖既聞海東三國舊結怨隙,遞相攻伐,以其俱為蕃附,務在和睦,乃問其使為怨所由。對曰:「先是百濟往伐高麗,詣新羅請救,新羅發兵大破百濟
 國,因此為怨,每相攻伐。新羅得百濟王,殺之,怨由此始。」七年,遣使冊拜金真平為柱國,封樂浪郡王、新羅王。



 貞觀五年,遣使獻女樂二人,皆鬒發美色。太宗謂侍臣曰:「朕聞聲色之娛,不如好德。且山川阻遠,懷土可知。近日林邑獻白鸚鵡,尚解思鄉,訴請還國。鳥猶如此,況人情乎!朕愍其遠來,必思親戚,宜付使者,聽遣還家。」



 是歲,真平卒,無子,立其女善德為王,宗室大臣乙祭總知國政。詔贈真平左光祿大夫,賻物二百段。九年,遣使持節冊
 命善德柱國,封樂浪郡王、新羅王。十七年,遣使上言:「高麗、百濟,累相攻襲,亡失數十城,兩國連兵,意在滅臣社稷。謹遣陪臣,歸命大國,乞偏師救助。」太宗遣相裏玄獎齎璽書賜高麗曰:「新羅委命國家,不闕朝獻。爾與百濟,宜即戢兵。若更攻之,明年當出師擊爾國矣!」太宗將親伐高麗,詔新羅纂集士馬,應接大軍。新羅遣大臣領兵五萬人,入高麗南界,攻水口城,降之。



 二十一年,善德卒,贈光祿大夫,餘官封並如故。因立其妹真德為王,加授
 柱國,封樂浪郡王。



 二十二年,真德遣其弟國相、伊贊子金春秋及其子文正來朝。詔授春秋為特進,文正為左武衛將軍。春秋請詣國學觀釋奠及講論,太宗因賜以所制《溫湯》及《晉祠碑》並新撰《晉書》。將歸國,令三品以上宴餞之,優禮甚稱。



 永徽元年,真德大破百濟之眾,遣其弟法敏以聞。真德乃織錦作五言《太平頌》以獻之,其詞曰:「大唐開洪業,巍巍皇猷昌。止戈戎衣定,修文繼百王。統天崇雨施,理物體含章。深仁偕日月,撫運邁陶唐。幡
 旗既赫赫,鉦鼓何鍠鍠。外夷違命者,翦覆被天殃。淳風凝幽顯,遐邇競呈祥。四時和玉燭,七曜巡萬方。維嶽降宰輔,維帝任忠良。五三成一德,昭我唐家光。」帝嘉之,拜法敏為太府卿。



 三年,真德卒,為舉哀。詔以春秋嗣,立為新羅王。加授開府儀同三司,封樂浪郡王。六年,百濟與高麗、靺鞨率兵侵其北界,攻陷三十餘城,春秋遣使上表求救。顯慶五年,命左武衛大將軍蘇定方為熊津道大總管,統水陸十萬。仍令春秋為嵎夷道行軍總管,與
 定方討平百濟,俘其王扶餘義慈,獻於闕下。自是新羅漸有高麗、百濟之地。其界益大,西至於海。



 龍朔元年,春秋卒,詔其子太府卿法敏嗣位,為開府儀同三司、上柱國、樂浪郡王、新羅王。三年,詔以其國為雞林州都督府,授法敏為雞林州都督。法敏以開耀元年卒,其子政明嗣位。垂拱二年,政明遣使來朝,因上表請唐禮一部並雜文章,則天令所司寫《吉兇要禮》,並於《文館詞林》採其詞涉規誡者,勒成五十卷以賜之。



 天授三年,政明卒,則
 天為之舉哀,遣使吊祭,冊立其子理洪為新羅王。仍令襲父輔國大將軍,行豹韜衛大將軍、雞林州都督。理洪以長安二年卒。則天為之舉哀,輟朝二日。遣立其弟興光為新羅王,仍襲兄將軍、都督之號。興光本名與太宗同,先天中則天改焉。



 開元十六年,遣使來獻方物,又上表請令人就中國學問經教,上許之。二十一年,渤海靺鞨越海入寇登州。時興光族人金思蘭先因入朝留京師,拜為太僕員外卿,至是遣歸國發兵以討靺鞨。仍加
 授興光為開府儀同三司、寧海軍使。



 二十五年,興光卒,詔贈太子太保。仍遣左贊善大夫邢璹攝鴻臚少卿,往新羅吊祭,並冊立其子承慶襲父開府儀同三司、新羅王。璹將進發,上制詩序,太子以下及百僚咸賦詩以送之。上謂璹曰:「新羅號為君子之國,頗知書記,有類中華。以卿學術,善與講論,故選使充此。到彼宜闡揚經典,使知大國儒教之盛」。又聞其人多善奕棋,因令善棋人率府兵曹楊季鷹為璹之副。璹等至彼,大為蕃人所敬。其
 國棋者皆在季鷹之下,於是厚賂璹等金寶及藥物等。



 天寶二年,承慶卒,詔遣贊善大夫魏曜往吊祭之。冊立其弟憲英為新羅王,並襲其兄官爵。



 大歷二年,憲英卒,國人立其子乾運為王,仍遣其大臣金隱居奉表入朝,貢方物,請加冊命。三年,上遣倉部郎中、兼御史中丞、賜紫金魚袋歸崇敬持節齎冊書往吊冊之。以乾運為開府儀同三司、新羅王、仍冊乾運母為太妃。七年,遣使金標石來賀正,授衛尉員外少卿,放還。八年,遣使來朝,並
 獻金、銀、牛黃、魚牙納朝霞紬等。九年至十二年,比歲遣使來朝,或一歲再至。



 建中四年,乾運卒,無子,國人立其上相金良相為王。貞元元年,授良相檢校太尉、都督雞林州刺史、寧海軍使、新羅王。仍令戶部郎中蓋塤持節冊命。其年,良相卒,立上相敬信為王,令襲其官爵。敬信即從兄弟也。



 十四年,敬信卒,其子先敬信亡,國人立敬信嫡孫俊邕為王。



 十六年,授俊邕開府儀同三司、檢校太尉、新羅王。令司封郎中、兼御史中丞韋丹持節冊命。
 丹至鄆州,聞俊邕卒,其子重興立,詔丹還。永貞元年,詔遣兵部郎中元季方持節冊重興為王。



 元和元年十一月,放宿衛王子金獻忠歸本國,仍加試秘書監。三年,遣使金力奇來朝。其年七月,力奇上言:「貞元十六年,奉詔冊臣故主金俊邕為新羅王,母申氏為太妃,妻叔氏為王妃。冊使韋丹至中路,知俊邕薨,其冊卻回,在中書省。今臣還國,伏請授臣以歸。」敕:「金俊邕等冊,宜令鴻臚寺於中書省受領,至寺宣授與金力奇,令奉歸國。仍賜其
 叔彥升門戟,令本國準例給。」四年,遣使金陸珍等來朝貢。五年,王子金憲章來朝貢。



 七年,重興卒,立其相金彥升為王,遣使金昌南等來告哀。其年七月,授彥升開府儀同三司、檢校太尉、持節大都督雞林州諸軍事,兼持節充寧海軍使、上柱國、新羅國王,彥升妻貞氏冊為妃,仍賜其宰相金崇斌等三人戟,亦令本國準例給。兼命職方員外郎、攝御史中丞崔廷侍節吊祭冊立,以其質子金士信副之。



 十一年十一月,其入朝王子金士信等
 遇惡風,飄至楚州鹽城縣界,淮南節度使李鄘以聞。是歲,新羅饑,其眾一百七十人求食於浙東。十五年十一月,遣使朝貢。



 長慶二年十二月,遣使金柱弼朝貢。寶歷元年,其王子金昕來朝。太和元年四月,皆遣使朝貢。五年,金彥升卒,以嗣子金景徽為開府儀同三司、檢校太尉、使侍節大都督雞林州諸軍事,兼持節充寧海軍使、新羅王;景徽母樸氏為太妃,妻樸氏為妃。命太子左諭德、兼御史中丞源寂持節吊祭冊立。開成元年,王子金
 義琮來謝恩,兼宿衛。二年四月,放還籓,賜物遣之。五年四月,鴻臚寺奏:新羅國告哀,質子及年滿合歸國學生等共一百五人,並放還。會昌元年七月,敕:「歸國新羅官、前入新羅宣慰副使、前充兗州都督府司馬、賜緋魚袋金雲卿,可淄州長史。」



 倭國者,古倭奴國也。去京師一萬四千里,在新羅東南大海中。依山島而居,東西五月行,南北三月行,世與中國通。其國,居無城郭,以木為柵,以草為屋。四面小島五
 十餘國,皆附屬焉。其王姓阿每氏,置一大率,檢察諸國,皆畏附之。設官有十二等。其訴訟者,匍匐而前。地多女少男。頗有文字,俗敬佛法。並皆跣足,以幅布蔽其前後。貴人戴錦帽,百姓皆椎髻,無冠帶。婦人衣純色裙,長腰襦,束發於後,佩銀花,長八寸,左右各數枝,以明貴賤等級。衣服之制,頗類新羅。



 貞觀五年,遣使獻方物。太宗矜其道遠,敕所司無令歲貢,又遣新州刺史高表仁持節往撫之。表仁無綏遠之才,與王子爭禮,不宣朝命而還。
 至二十二年,又附新羅奉表,以通起居。



 日本國者,倭國之別種也。以其國在日邊,故以日本為名。或曰:倭國自惡其名不雅,改為日本。或云:日本舊小國,並倭國之地。其人入朝者,多自矜大,不以實對,故中國疑焉。又云:其國界東西南北各數千里,西界、南界咸至大海,東界、北界有大山為限,山外即毛人之國。



 長安三年,其大臣朝臣真人來貢方物。朝臣真人者,猶中國戶部尚書,冠進德冠,其頂為花,分而四散,身服紫袍,以
 帛為腰帶。真人好讀經史,解屬文,容止溫雅。則天宴之於麟德殿,授司膳卿,放還本國。



 開元初,又遣使來朝,因請儒士授經。詔四門助教趙玄默就鴻臚寺教之。乃遺玄默闊幅布以為束修之禮。題云「白龜元年調布」。人亦疑其偽。所得錫賚,盡市文籍,泛海而還。其偏使朝臣仲滿,慕中國之風,因留不去,改姓名為朝衡,仕歷左補闕、儀王友。衡留京師五十年,好書籍,放歸鄉,逗留不去。天寶十二年,又遣使貢。上元中,擢衡為左散騎常侍、鎮南
 都護。貞元二十年,遣使來朝,留學生橘免勢、學問僧空海。元和元年,日本國使判官高階真人上言:「前件學生,藝業稍成,願歸本國,便請與臣同歸。」從之。開成四年,又遣使朝貢。



\end{pinyinscope}