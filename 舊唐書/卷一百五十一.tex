\article{卷一百五十一}

\begin{pinyinscope}

 ○杜
 黃裳高郢子定杜佑子式方從鬱式方子悰從鬱子牧



 杜黃裳,字遵素,京兆杜陵人也。登進士第、宏辭科,杜鴻漸深器重之。為郭子儀朔方從事,子儀入朝,令黃裳主
 留務於朔方。邠將李懷光與監軍陰謀代子儀,乃為偽詔書,欲誅大將溫儒雅等。黃裳立辨其偽,以告懷光,懷光流汗伏罪。諸將有難制者,黃裳矯子儀命盡出之,數月而亂不作。後入為臺省官,為裴延齡所惡,十年不遷。貞元末,為太常卿。王叔文之竊權,黃裳終不造其門。嘗語其子婿韋執誼,令率百官請皇太子監國,執誼遽曰:「丈人才得一官,可復開口議禁中事耶!」黃裳勃然曰:「黃裳受恩三朝,豈可以一官見買!」即拂衣而出。尋拜平章
 事。



 邠州節度使韓全義曾居討伐之任,無功,黃裳奏罷之。劉闢作亂,議者以劍南險固,不宜生事;唯黃裳堅請討除,憲宗從之。又奏請不以中官為監軍,只委高崇文為使。黃裳自經營伐蜀,以至成功,指授崇文,無不懸合。崇文素憚劉水雍,黃裳使人謂崇文曰:「若不奮命,當以劉澭代之。」由是得崇文之死力。既平闢,宰臣入賀,帝目黃裳曰:「此卿之功也。」後與憲宗語及方鎮除授,黃裳奏曰:「德宗自艱難之後,事多姑息。貞元中,每帥守物故,必先
 命中使偵伺其軍動息,其副貳大將中有物望者,必厚賂近臣以求見用,帝必隨其稱美而命之,以是因循,方鎮罕有特命帥守者。陛下宜熟思貞元故事,稍以法度整肅諸侯,則天下何憂不治!」憲宗然其言。由是用兵誅蜀、夏之後,不容籓臣蹇傲,克復兩河,威令復振,蓋黃裳啟其衷也。黃裳有經畫之才,達於權變,然檢身律物,寡廉潔之譽,以是居鼎職不久。二年正月,檢校司空,同平章事,兼河中尹、河中晉絳等州節度使。八月,封邠國公。
 三年九月,卒於河中,年七十一,贈司徒,謚曰宣。



 黃裳性雅淡寬恕,心雖從長,口不忤物。始為卿士,女嫁韋執誼,深不為執誼所稱;及執誼譴逐,黃裳終保全之,洎死嶺表,請歸其喪,以辦葬事。及是被疾,醫人誤進其藥,疾甚而不怒。然為宰相,除授不分流品,或官以賂遷,時論惜之。



 黃裳歿後,賄賂事發。八年四月,御史臺奏:「前永樂令吳憑為僧鑒虛受托,與故司空杜黃裳,於故州邠寧節度使高崇文處納賂四萬五千貫,並付黃裳男載,按問
 引伏。」敕曰:「吳憑曾佐使府,忝履宦途,自宜畏法惜身,豈得為人通貸!事關非道,理合懲愆,宜配流昭州。其付杜載錢物,宰輔之任,寵寄實深,致茲貨財,不能拒絕,已令按問,悉合徵收,貴全終始之恩,俾弘寬大之典。其所取錢物,並宜矜免,杜載等並釋放。」



 載為太子僕,長慶中,遷太僕少卿、兼御史中丞,充入吐蕃使。



 載弟勝,登進士第,大中朝位給事中。勝子廷堅,亦進士擢第。



 高郢,字公楚,其先渤海蓚人。九歲通《春秋》,能屬文。天寶
 末,盜據京邑,父伯祥先為好畤尉,抵賊禁,將加極刑。郢時年十五,被發解衣,請代其父,賊黨義之,乃俱釋。後舉進士擢第,應制舉,登茂才異行科,授華陰尉。嘗以魯不合用天子禮樂,乃引《公羊傳》,著《魯議》,見稱於時,由是授咸陽尉。



 郭子儀節制朔方,闢為掌書記。子儀嘗怒從事張曇,奏殺之;郢極言爭救,忤子儀旨,奏貶猗氏丞。李懷光節制邠寧,奏為從事,累轉副元帥判官、檢校禮部郎中。懷光背叛,將歸河中,郢言:「西迎大駕,豈非忠乎!」懷光
 忿而不聽。及歸鎮,又欲悉眾而西。時渾瑊軍孤,群帥未集,郢與李鄘誓死駐之。屬懷光長子琟候郢,郢乃諭以逆順曰:「人臣所宜效順。且自天寶以來阻兵者,今復誰在?況國家自有天命,非獨人力。今若恃眾西向,自絕於天,十室之邑,必有忠信,安知三軍不有奔潰者乎?」李琟震懼,流淚氣索。明年春,郢與都知兵馬使呂鳴岳、都虞候張延英同謀間道上表;及受密詔,事洩,二將立死。懷光乃大集將卒,白刃盈庭,引郢詰之。郢挺然抗辭,無所
 慚隱,憤氣感發,觀者淚下,懷光慚沮而止。德宗還京,命諫議大夫孔巢父、中人啖守盈赴河中宣慰懷光,授以太保;而懷光怒,激其親兵詬詈,殺守盈及巢父。巢父之被刃也,委於地,郢就而撫之。乃懷光被誅,馬燧闢郢為掌書記。



 未幾,徵拜主客員外,遷刑部郎中,改中書舍人。凡九歲,拜禮部侍郎。時應進士舉者,多務朋游,馳逐聲名;每歲冬,州府薦送後,唯追奉宴集,罕肄其業。郢性剛正,尤嫉其風,既領職,拒絕請托,雖同列通熟,無敢言者。
 志在經藝,專考程試。凡掌貢部三歲,進幽獨,抑浮華,朋濫之風,翕然一變。拜太常卿。貞元十九年冬,進位銀青光祿大夫,守中書侍郎、同中書門下平章事。順宗即位,轉刑部尚書,為韋執誼等所憚。尋罷知政事,以本官判吏部尚書事。明年,出鎮華州。



 元和元年冬,復拜太常卿,尋除御史大夫。數月,轉兵部尚書。逾月,再表乞骸,不許。又上言曰:「臣聞勞生佚老,天理自然,蠕動翾飛,日入皆息。自非貢禹之守經據古,趙喜之正身匪懈,韓暨之志
 節高潔,山濤之道德模表,縱過常期,詎為貪冒。其有當仁不讓,急病忘身,豈止君命,猶宜身舉。臣郢不才,久辱高位,無任由衷瀝懇之至。」乃授尚書右僕射致仕。六年七月卒,年七十二。贈太子太保,謚曰貞。



 郢性恭慎廉潔,罕與人交游,守官奉法勤恪,掌誥累年,家無制草。或謂之曰:「前輩皆留制集,公焚之何也?」曰:「王言不可存私家。」時人重其慎密。與鄭珣瑜並命拜相;未幾,德宗升遐。時同在相位,杜佑以宿舊居上,而韋執誼由朋黨專柄。順
 宗風恙方甚,樞機不宣,而王叔文以翰林學士兼戶部侍郎,充度支副使。是時政事,王叔文謀議,王伾通導,李忠言宣下,韋執誼奉行。珣瑜自受命,憂形顏色,至是以勢不可奪,因稱疾不起。郢則因循,竟無所發,以至于罷。物論定此為優劣焉。子定嗣。



 定,幼聰警絕倫,年七歲時,讀《尚書·湯誓》,問郢曰:「奈何以臣伐君?」郢曰:「應天順人,不為非道。」又問曰:「用命賞於祖,不用命戮於社,是順人乎?」父不能對。仕至京兆參軍。小字董二,人以幼慧,多以字
 稱之。尤精《王氏易》,嘗為《易圖》,合入出以畫八卦,上圓下方,合則重,轉則演,七轉而六十四卦六甲八節備焉。著《易外傳》二十二卷。



 杜佑,字君卿,京兆萬年人。曾祖行敏,荊、益二州都督府長史、南陽郡公。祖愨,右司員外郎、詳正學士。父希望,歷鴻臚卿、恆州刺史、西河太守,贈右僕射。佑以廕入仕,補濟南郡參軍、剡縣丞。時潤州刺史韋元甫嘗受恩於希望,佑謁見,元甫未之知,以故人子待之。他日,元甫視事,
 有疑獄不能決。佑時在旁,元甫試訊於佑;佑口對響應,皆得其要。元甫奇之,乃奏為司法參軍。元甫為浙西觀察、淮南節度,皆闢為從事,深所委信。累官至檢校主客員外郎,入為工部郎中,充江西青苗使,轉撫州刺史。改御史中丞,充容管經略使。楊炎入相,徵入朝,歷工部、金部二郎中,並充水陸轉運使,改度支郎中,兼和糴等使。時方軍興,饋運之務,悉委於佑;遷戶部侍郎、判度支。為盧杞所惡,出為蘇州刺史。佑母在,杞以蘇州憂闕授之。
 佑不行,俄換饒州刺史。未幾,兼御史大夫,充嶺南節度使。時德宗在興元。朝廷故事,執政往往遺脫;舊嶺南節度,常兼五管經略使,佑獨不兼。故五管不屬嶺南,自佑始也。



 貞元三年,徵為尚書左丞,又出為陜州觀察使,遷檢校禮部尚書、揚州大都督府長史,充淮南節度使。丁母憂,特詔起復,累轉刑部尚書、檢校右僕射。十六年,徐州節度使張建封卒,其子愔為三軍所立,詔佑以淮南節制檢校左僕射、同平章事,兼徐泗節度使,委以討伐。
 佑乃大具舟艦,遣將孟準先當之。準渡淮而敗,佑杖之,固境不敢進。及詔以徐州授愔,而加佑兼濠、泗等州觀察使。在揚州開設營壘三十餘所,士馬修葺。然於賓僚間依阿無制,判官南宮僔、李亞、鄭元均爭權,頗紊軍政,德宗知之,並竄於嶺外。



 十九年入朝,拜檢校司空、同平章事,充太清宮使。德宗崩,佑攝塚宰,尋進位檢校司徒,充度支鹽鐵等使,依前平章事。旋又加弘文館大學士。時王叔文為副使,佑雖總統,而權歸叔文。叔文敗,又奏
 李巽為副使,頗有所立。順宗崩,佑復攝塚宰,尋讓金穀之務,引李巽自代。先是,度支以制用惜費,漸權百司之職,廣署吏員,繁而難理;佑始奏營繕歸之將作,木炭歸之司農,染練歸之少府,綱條頗整,公議多之,朝廷允其議。



 元和元年,冊拜司徒、同平章事,封岐國公。時河西黨項潛導吐蕃入寇,邊將邀功,亟請擊之。佑上疏論之曰:



 臣伏見黨項與西戎潛通,屢有降人指陳事跡,而公卿廷議,以為誠當謹兵戎,備侵軼,益發甲卒,邀其寇暴。此
 蓋未達事機,匹夫之常論也。



 夫蠻夷猾夏,唐虞已然。周宣中興,獫狁為害,但命南仲往城朔方,追之太原,及境而止,誠不欲弊中國而怒遠夷也。秦平六國,恃其兵力,北築長城,以拒匈奴;西逐諸羌,出於塞外。勞力擾人,結怨階亂,中國未靜,白徒競起,海內雲擾,實生謫戍。漢武因文、景之富,命將興師,遂至戶口減半,竟下哀痛之詔罷田輪臺。前史書之,尚嘉其先迷而後復。蓋聖王之理天下也,唯務綏靜蒸人,西至流沙,東漸於海,在南與北,
 亦存聲教。不以遠物為珍,匪求遐方之貢,豈疲內而事外,終得少而失多。故前代納忠之臣,並有匡君之議。淮南王請息師於閩越,賈捐之願棄地於珠崖,安危利害,高懸前史。



 昔馮奉世矯漢帝之詔,擊莎車,傳其王首於京師,威震西域。宣帝大悅,議加爵土之賞。蕭望之獨以為矯制違命,雖有功效,不可為法;恐後之奉使者爭逐發兵,為國家生事,述理明白,其言遂行。國家自天後已來,突厥默啜兵強氣勇,屢寇邊城,為害頗甚。開元初,邊
 將郝靈佺親捕斬之,傳首闕下,自以為功,代莫與二,坐望榮寵。宋璟為相,慮武臣邀功,為國生事,止授以郎將。由是訖開元之盛,無人復議開邊,中國遂寧,外夷亦靜。此皆成敗可徵,鑒戒非遠。



 且黨項小蕃,雜處中國,本懷我德,當示撫綏。間者邊將非廉,亟有侵刻,或利其善馬,或取其子女,便賄方物,徵發役徒。勞苦既多,叛亡遂起,或與北狄通使,或與西戎寇邊,有為使然,固當懲革。《傳》曰:「遠人不服,則修文德以來之。」《管子》曰:「國家無使勇猛
 者為邊境。」此誠聖哲識微知著之遠略也。今戎醜方強,邊備未實,誠宜慎擇良將,誡之完葺,使保誠信,絕其求取,用示懷柔。來則懲御,去則謹備,自然懷柔,革其奸謀,何必遽圖興師,坐致勞費!



 陛下上聖君人,覆育群類,動必師古,謀無不臧。伏望堅保永圖,置兵衽席,天下幸甚!臣識昧經綸,學慚博究,竊鼎鉉之寵任,為朝廷之老臣,恩深莫倫,志懇思報,臧否備閱,芻蕘上陳,有瀆旒扆,伏深惶悚。



 上深嘉納。



 歲餘,請致仕,詔不許,但令三五日一
 入中書,平章政事。每入奏事,憲宗優禮之;不名,常呼司徒。佑城南樊川有佳林亭,卉木幽邃,佑每與公卿宴集其間,廣陳妓樂。諸子咸居朝列,當時貴盛,莫之與比。元和七年,被疾,六月,復乞骸骨。表四上,情理切至,憲宗不獲已,許之。詔曰:



 宣力濟時,為臣之懿躅;辭榮告老,行己之高風。況乎任重公臺,義深翼贊,秉沖讓之志,堅金石之誠。敦諭既勤,所執彌固,則當遂其衷懇,進以崇名;尚齒優賢,斯王化之本也。



 金紫光祿大夫、守司徒、同中書
 門下平章事、兼充弘文館大學士、太清宮使、上柱國、岐國公、食邑三千戶杜佑,巖廊上才,邦國茂器;蘊經通之識,履溫厚之姿,寬裕本乎性情,謀猷彰乎事業。博聞強學,知歷代沿革之宜;為政惠人,審群黎利病之要。由是再司邦用,累歷籓方,出總戎麾,入和鼎實,聿膺重寄,歷事先朝,左右朕躬,夙夜不懈。命以詔冊,登之上公,肅恭在廷,華發承弁。茲可謂國之元老,人之具瞻者也。



 朕纘承丕業,思弘景化,選勞求舊,期致時邕,方伸引翼之儀,
 遽抗懸車之請。而又固辭年疾,乞就休閑,已而復來,星琯屢變,有不可抑,良用耿然。永惟古先哲王,君臣之際,臣有耆艾以求其退,君有優賜以徇其情;乃輟鄧禹敷教之功,仍增王祥輔導之秩,俾養浩然之氣,安於敬止之鄉,庶乎怡神葆和,永綏福履。仍加階級,以厚寵章,可光祿大夫、守太保致仕,宜朝朔望。



 是日,上遣中使就佑第賜絹五百匹、錢五百千。其年十一月薨,壽七十八,廢朝三日,冊贈太傅,謚曰安簡。



 佑性敦厚強力,尤精吏職,
 雖外示寬和,而持身有術。為政弘易,不尚皦察,掌計治民,物便而濟,馭戎應變,即非所長。性嗜學,該涉古今,以富國安人之術為己任。初開元末,劉秩採經史百家之言,取《周禮》六官所職,撰分門書三十五卷,號曰《政典》,大為時賢稱賞;房琯以為才過劉更生。佑得其書,尋味厥旨,以為條目未盡,因而廣之,加以開元禮、樂,書成二百卷,號曰《通典》。貞元十七年,自淮南使人詣闕獻之,曰:



 臣聞太上立德,不可庶幾;其次立功,遂行當代;其次立言,
 見志後學。由是往哲遞相祖述,將施有政,用乂邦家。臣本以門資,幼登官序,仕非游藝,才不逮人,徒懷自強,頗玩墳籍。雖履歷叨幸,或職劇務殷,竊惜光陰,未嘗輕廢。夫《孝經》、《尚書》、《毛詩》、《周易》、《三傳》,皆父子君臣之要道;十倫五教之宏綱,如日月之下臨,天地之大德,百王是式,終古攸遵。然多記言,罕存法制;愚管窺測,莫達高深,輒肆荒虛,誠為億度。每念懵學,莫探政經,略觀歷代眾賢著論,多陳紊失之弊,或闕匡拯之方。臣既庸淺,寧詳損益,
 未原其始,莫暢其終。尚賴周氏典禮,秦皇蕩滅不盡,縱有繁雜,且用準繩。至於往昔是非,可為來今龜鏡,布在方冊,亦粗研尋。自頃纘修,年逾三紀,識寡思拙,心昧辭蕪。圖籍實多,事目非少,將事功畢,罔愧乖疏,固不足發揮大猷,但竭愚盡慮而已。書凡九門,計貳百卷,不敢不具上獻,庶明鄙志所之,塵瀆聖聰,兢惶無措。



 優詔嘉之,命藏書府。其書大傳於時,禮樂刑政之源,千載如指諸掌,大為士君子所稱。



 佑性勤而無倦,雖位極將相,手不
 釋卷。質明視事,接對賓客,夜則燈下讀書,孜孜不怠。與賓佐談論,人憚其辯而伏其博,設有疑誤,亦能質正。始終言行,無所玷缺,唯在淮南時,妻梁氏亡後,升嬖妾李氏為正室,封密國夫人,親族子弟言之不從,時論非之。



 三子,師損嗣,位終司農少卿。



 式方,字考元。以廕授揚府參軍,轉常州晉陵尉。浙西觀察使王緯闢為從事,入為太子通事舍人,改太常寺主簿。明練鐘律,有所考定,深為高郢所賞。時父作鎮揚州,家財鉅萬,甲第在安仁里,
 杜城有別墅,亭館林池,為城南之最。昆仲皆在朝廷,與時賢游從,樂而有節。既而佑入中書,出為昭應令。丁父憂,服闋,遷司農少卿,賜金紫,加正議大夫、太僕卿。時少子忭選尚公主,式方以右戚移病不視事。久之,穆宗即位,轉兼御史中丞,充桂管觀察都防禦使。長慶二年三月,卒於位,贈禮部尚書。



 式方性孝友,弟兄尤睦。季弟從鬱,少多疾病,式方每躬自煎調,藥膳水飲,非經式方之手,不入於口。及從鬱夭喪,終年號泣,殆不勝情,士友多
 之。



 子惲、憓、忭、恂。惲嗣,富平尉;憓,興平尉。



 忭,以廕三遷太子司議郎。元和九年,選尚公主,召見於麟德殿。尋尚岐陽公主,加銀青光祿大夫、殿中少監、駙馬都尉。岐陽,憲宗長女,郭妃之所生。



 自頃選尚,多於貴戚,或武臣節將之家。於時翰林學士獨孤鬱,權德輿之女婿,時德輿作相,鬱避嫌辭內職。上頗重學士,不獲已許之,且嘆德輿有佳婿,遂令宰臣於卿士家選尚文雅之士可居清列者。初於文學後進中選擇,皆辭疾不應,唯悰願焉。累遷
 至司農卿。太和六年,轉京兆尹。七年,檢校刑部尚書,出為鳳翔尹、鳳翔隴右節度。丁內艱,八年,起復授忠武軍節度使、陳許蔡觀察等使,就加兵部尚書。開成初,入為工部尚書、判度支。屬岐陽公主薨,久而未謝。文宗怪之,問左右。戶部侍郎李玨對曰:「近日駙馬為公主服斬衰三年,所以士族之家不願為國戚者,半為此也。杜忭未謝,拘此服紀也。」上愕然曰:「予初不知。」乃詔曰:「制服輕重,必由典禮。如聞往者駙馬為公主服三年,緣情之義,殊
 非故實,違經之制,今乃聞知。宜令行杖周,永為通制。」三年,改戶部尚書,兼判戶部度支事。會昌中,拜中書侍郎、同中書門下平章事,尋加左僕射。



 大中初,出鎮西川,降先沒吐蕃維州。州即古西戎地也,其地南界江陽,岷山連嶺而西,不知其極;北望隴山,積雪如玉:東望成都,若在井底。地接石紐山,夏禹生於石紐山是也。其州在岷山之孤峰,三面臨江。天寶後,河、隴繼陷,惟此州在焉。吐蕃利其險要,二十年間,設計得之,遂據其城,因號曰「無
 憂城」,吐蕃由是不虞邛、蜀之兵。先是,李德裕鎮西川,維州吐蕃首領悉怛謀以城來降,德裕奏之;執政者與德裕不協,遽勒還其城。至是復收之,亦不因兵刃,乃人情所歸也。俄復入相,加司空,繼加司徒,歷鎮重籓。至是加太傅、邠國公。忭無他才,常延接寒素,甘食竊位而已。



 從鬱,以廕貞元末再遷太子司議郎。元和初,轉左補闕。諫官崔群、韋貫之、獨孤鬱等以從鬱宰相子,不合為諫官,乃降授左拾遺。群等復執曰:「拾遺之與補闕,雖資品有
 殊,皆名諫列。父為宰相,子為諫官,若政有得失,不可使子論父。」乃改為秘書丞,終駕部員外郎。



 子牧、顗,俱登進士第。顗後病目而卒。



 牧,字牧之,既以進士擢第,又制舉登乙第,解褐弘文館校書郎,試左武衛兵曹參軍。沈傳師廉察江西宣州,闢牧為從事、試大理評事。又為淮南節度推官、監察御史裏行,轉掌書記。俄真拜監察御史,分司東都,以弟顗病目棄官。授宣州團練判官、殿中侍御史、內供奉。遷左補闕、史館修撰,轉膳部、比部員外郎,
 並兼史職。出牧黃、池、睦三郡,復遷司勛員外郎、史館修撰,轉吏部員外郎。又以弟病免歸。授湖州刺史,入拜考功郎中、知制誥,歲中遷中書舍人。牧好讀書,工詩為文,嘗自負經緯才略。武宗朝誅昆夷、鮮卑,牧上宰相書論兵事,言「胡戎入寇,在秋冬之間,盛夏無備,宜五六月中擊胡為便」。李德裕稱之。注曹公所定《孫武十三篇》行於代。



 牧從兄忭隆盛於時,牧居下位,心常不樂。將及知命,得病,自為墓志、祭文。又嘗夢人告曰:「爾改名畢。」逾月,奴
 自家來,告曰:「炊將熟而甑裂。」牧曰:「皆不祥也。」俄又夢書行紙曰:「皎皎白駒,在彼空谷。」寤寢而嘆曰:「此過隙也。吾生於角,徵還於角,為第八宮,吾之甚厄也。予自湖守遷舍人,木還角,足矣。」其年,以疾終於安仁里,年五十。有集二十卷,曰《杜氏樊川集》,行於代。子德祥,官至丞郎。



 史臣曰:黃裳以道致君,持誠奉主;辨懷光之詐,罷全義之徵。討賊闢之兇,舉無遺算;葬執誼之柩,豈曰不仁。郢天縱之性,總丱之年,代父命於臨刑,孝也;懷光之亂,王
 人被傷,撫巢父於賊庭,義也;抑浮濫之流,考藝文之士,盡搜幽滯,大變時風,正也;保止足之名,辭榮辱之路,高避世利,遐躅昔賢,智也。忠孝全矣,仁智備矣!此二子者,皆臨大節而不可奪也。佑承廕入仕,讞獄受知,博古該今,輸忠效用;位居極品,榮逮子孫,操修之報,不亦宜哉!及其賓僚紊法,嬖妾受封,事重因循,難乎語於正矣!牧之文章,忭之長厚,能否既異,才位不倫,命矣夫!



 贊曰:貞公壯節,臨難奮發。言行無玷,斯為明哲。戡亂阜
 俗,時泰位隆。國之名臣,鄭公、岐公。



\end{pinyinscope}