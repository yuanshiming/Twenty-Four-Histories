\article{卷一百八十六}

\begin{pinyinscope}

 ○王重榮子珂王處存弟處直諸葛爽高駢畢師鐸秦彥時溥硃瑄弟瑾



 王重榮,河中人。父縱,鹽州刺史,咸通中有邊功。重榮以父廕補軍校,與兄重盈俱號驍雄,名璟軍中。廣明初,重
 榮為河中馬步軍都虞候。巢賊據長安,蒲帥李都不能拒,稱臣於賊,賊偽授重榮節度副使。河中密邇京師,賊徵求無已,軍府疲於供億,賊使百輩,填委傳舍。重榮謂都曰:「吾以外援未至,詭謀附賊以紓難。今軍府積實,苦被徵求,復來收兵,是賊危我也。倘不改圖,危亡必矣!請絕橋道,嬰城自固。」都曰:「吾兵微力寡,絕之立見其患。唯公圖之,願以節鉞假公。」翌日,都歸行在,重榮知留後事,乃斬賊使,求援鄰籓。既而賊將硃溫舟師自同州至,黃
 鄴之兵自華陰至,數萬攻之。重榮戒勵士眾,大敗之。獲其兵仗,軍聲益振。朝廷遂授節鉞,檢校司空。時中和元年夏也。



 俄而忠武監軍楊復光率陳、蔡之師萬人,與重榮合。賊將李祥守華州,重榮合勢攻之,擒祥以徇。俄而硃溫以同州降。賊既失同、華,狂躁益熾。黃巢自率精兵數萬,至梁田坡。時重榮軍華陰南,楊復光在渭北,掎角破賊;出其不意,大敗賊軍,獲其將趙璋。巢中流矢而退。而重榮之師,亡耗殆半。懼賊復來,深憂之。謂復光曰:「軍
 雖小捷,銳旅亡失。萬一賊黨復來,其將何軍以應?吾之成敗,未可知也。」復光曰:「雁門李僕射,與僕家世事舊,其尊人與僕父兄同患難。僕射奮不顧身,死義知己。倘得李雁門為援,吾事濟矣。」因遣使傳詔徵兵。明年,李克用領兵至,大敗巢賊,收復京城。其倡義啟導之功,實重榮居首。京師平,以功檢校太尉、同平章事、瑯邪郡王。



 光啟元年,僖宗還京。喪亂之後,六軍初復,國藏虛竭。觀軍容使田令孜奏以安邑、解縣兩池榷課,直屬省司,以充贍
 給。舊事,河中節度兼榷使,每年額輸省課。重榮累表論列,既循往例,兼恃大功。令孜不許,奏請移重榮為定州節度。制下,不奉詔,令孜率禁軍攻之,屯於沙苑,為重榮擊敗之。十二月,令孜挾天子出幸寶雞,太原聞之,乃與重榮入援京師,遣使迎駕還宮。令孜尤懼,卻劫幸山南。及硃玫立襄王稱制,重榮不受命,會太原之師於河西,以圖興復。明年,王行瑜殺硃玫,僖宗反正,重榮之忠力居多。



 重榮用法稍嚴,季年尤甚。部下常行儒者,嘗有所
 譴罰,深銜之。光啟三年六月,行儒以兵攻府第,重榮夜出於城外別墅。詰旦,為行儒所害,行儒乃推重盈為帥。重盈既立,誅行儒與其黨,安集軍民。



 乾寧初,重盈卒,軍府推行軍司馬王珂為留後。重盈子珙,時為陜帥,瑤為絳州刺史。珂即重榮兄重簡子,出繼重榮。由是爭為蒲帥。瑤、珙上章論列,又與硃溫書云:「珂非吾兄弟,家之蒼頭也。小字蟲兒,安得繼嗣?」珂上章云:「亡父有興復之功。」遣使求援於太原,太原保薦於朝。珙厚結王行瑜、李茂
 貞、韓建為援,三鎮互相表薦。昭宗詔諭之曰:「吾以太原與重榮有再造之功,已俞其奏矣。」故明年五月,茂貞等三人率兵入覲,賊害時政,請以河中授珙。珙、瑤連兵攻河中。李克用怒,出師討三鎮。瑤、珙兵退,克用拔絳州,斬瑤,乃師於渭北。天子以珂為河中節度,授以旄鉞,仍充供軍糧料使。既誅王行瑜,克用以女妻之。珂親至太原、太原令李嗣昭將兵助珂攻珙,珙每戰頻敗。珙性慘刻,人有逾犯,必斬首置於座前,言笑自若,部下咸苦之。因
 其削弱,皆懷離叛。光化二年六月,部將李璠殺珙,自稱留後。



 光化末,硃溫初伏鎮、定,將圖關輔,屬劉季述廢立之際,京師俶擾。崔胤潛乞師於汴,以圖反正。溫謂其將張存敬、侯言曰:「王珂恃太原之勢,侮慢籓鄰,骨肉相殘,自大其事,爾為我持一繩以縛之。」存敬等率兵數萬渡河,由含山出其不意,天復元年正月,兵攻晉、絳。珂將絳州刺史陶建釗、晉州刺史張漢瑜既無備,即開門降。溫令別將何絪守晉州,扼其援路。二月,存敬大軍逼河中,
 珂遣告急於太原。晉、絳既當兵沖,援師不能進。珂妻書告太原曰:「賊勢攻逼,朝夕為俘囚,乞食大梁,大人安忍不救?」克用曰:「賊阻前途,眾寡不敵,救則與爾兩亡。可與王郎歸朝廷。」珂計無從出,即謀歸京師。又使人告李茂貞曰:「聖上初返正,詔籓鎮無相侵伐,同匡王室。硃公不顧國家約束,卒遣賊臣,急攻敝邑,則硃公之心可見矣!敝邑若亡,則同、華、邠、岐非諸君所能保也。天子神器,拱手而授人矣!此自然之勢也。公可與華州令公早出精
 銳固潼關,以應敝邑。僕自量不武,請於公之西偏求為鎮守,此地請公有之。關西安危,國祚延促,系公此舉也。」茂貞不答。



 珂勢蹙,將渡河歸京師,人情離合。時河橋毀圮,凌澌鯁塞,舟楫難濟。珂族艤舟有日。珂夜自慰諭守陴者,默然無應。牙將劉訓夜半至珂寢門,珂叱之曰:「兵欲反耶?」訓解衣袒臂曰:「公茍懷疑,訓請斷臂。」珂曰:「事勢如何,計將安出?」訓曰:「若夜出整棹待濟,人必爭舟。茍一夫鴟張,其禍莫測。不如俟明旦,以情諭三軍,願從者必
 半,然後登舟赴闕,可以前濟。不然,則召諸將校,且為款狀,以緩賊軍,徐圖向背,策之上也。」珂然之,即登城謂存敬曰:「吾於汴王有家世事分,公宜退舍。俟汴王至,吾自聽命。」存敬即日退舍。



 三月,硃溫自洛陽至,先哭於重榮之墓,悲不自勝,陳辭致祭,蒲人聞之感悅。珂欲面縛牽羊以見。溫報曰:「太師阿舅之恩,何時可忘耶?郎君若以亡國之禮相見,黃泉其謂我何?」及珂出,迎之於路,握手歔欷,聯轡而入。



 居半月,以存敬守河中,珂舉家徙於汴。
 後溫令珂入覲,遣人殺之於華州傳舍。自重榮初帥河中,傳至珂二十年。



 王處存,京兆萬年縣勝業里人。世隸神策軍,為京師富族,財產數百萬。父宗,自軍校累至檢校司空、金吾大將軍、左街使,遙領興元節度。宗善興利,乘時貿易,由是富擬王者,仕宦因貲而貴,侯服玉食,僮奴萬指。處存起家右軍鎮使,累至驍衛將軍、左軍巡使。乾符六年十月,檢校刑部尚書、義武軍節度使。



 明年,黃巢犯闕,僖宗出幸。
 處存號哭累日,不俟詔命,即率本軍入援。遣二千人間道往山南,衛從車駕。時李都守河中降賊,會王重榮斬偽使,通使於處存,乃同盟誓師,營於渭北。時巢賊僭號,天下籓鎮,多受其偽命,唯鄭畋守鳳翔,鄭從讜守太原。處存、王重榮首倡義舉,以招太原。俄而鄭畋破賊前鋒,王鐸自行在至,故諸鎮翻然改圖,以出勤王之師。



 中和元年四月,涇原行軍唐弘夫敗賊將林言、尚讓軍,乘勝進逼京師。處存自渭北親選驍卒五千,皆以白繻為號,
 夜入京城。賊已遁去。京師故人見處存,遮道慟哭,歡呼塞路。軍人皆釋兵,爭據第宅,坊市少年多帶白號雜軍。翌日,賊偵知,自灞上復襲京師,市人以為王師,歡呼迎之。處存為賊所迫,收軍還營。賊怒,召集兩市丁壯七八萬,並殺之,血流成渠。



 處存家在京師,世受國恩,以賊寇未平,鑾輿出狩,每言及時事,未嘗不喑嗚流涕,諸軍義之。前後遣使十輩迎李克用,既奕世姻好,特相款暱。洎收京師,王鐸第其功;勤王舉義,處存為之最;收城破賊,
 克用為之最。以功檢校司空。後又遣大將張公慶率勁兵三千,合諸軍滅賊巢於泰山,以功檢校司徒。



 田令孜討王重榮,詔處存為河中節度。處存上章申理,言:「重榮無罪,有大功於國,不宜輕有除改,以搖籓鎮之心。」初,幽、鎮兩籓,兵甲強盛,易定於其間,疲於侵寇。及匡威得志驕盈,恆欲兼並之。賴與太原姻好,每為之援。處存亦睦鄰以禮,優撫軍民,折節下士,人多歸之,以至抗衡列鎮。累加侍中、檢校太尉。乾寧二年九月卒,年六十五,贈太
 子太師,謚曰忠肅。



 三軍以河朔舊事,推其子副大使郜為留後。朝廷從而命之。授以旄鉞,尋加檢校司空、同平章事,累至太保。光化三年七月,汴將張存敬進寇幽州,旋入祁溝。郜遣馬步都將王處直將兵拒之,為存敬所敗,退營沙河。汴人進擊,營於懷德驛。處直之眾奔撓,城中大恐。十月,郜委城攜族奔於太原,太原累表授檢校太尉。天復初,卒於晉陽。



 其弟鄴,克用以女妻之,歷嵐、石、沔三州刺史、大同軍防禦使。天祐中卒。



 處直,字允明,處
 存母弟也。初為定州後院軍都知兵馬使。汴人入寇,處直拒戰不利而退;三軍大噪,推處直為帥。及郜出奔,乃權留後事。汴將張存敬攻城,梯沖雲合,處直登城呼曰:「敝邑於朝廷未嘗不忠,於籓鄰未嘗失禮,不虞君之涉吾地,何也?」硃溫遣人報之曰:「何以附太原而弱鄰道?」處直報曰:「吾兄與太原同時立勛王室,地又親鄰,修好往來,常道也。請從此改圖。」溫許之。仍歸罪於孔目吏梁問,出絹十萬匹,牛酒以犒汴軍。存敬修盟而退。溫因表授
 旄鉞,檢校左僕射。天祐元年,加太保,封太原王。後仕偽梁,授北平王,檢校太尉。不數歲,復於莊宗。後十餘年,為其子都廢歸私第,尋卒,年六十一。



 諸葛爽,青州博昌人。役屬縣為伍伯,為令所笞,乃棄役,以里謳自給。會龐勛之亂,乃委身為徐卒,累軍功至小校。官軍討徐,龐勛勢蹙,率百餘人與泗州守將陽群歸國。累授汝州防禦使。李琢為招討使,討沙阤於雲州,表爽為副。廣明元年,賊陷京師,詔爽率代北行營兵馬,赴
 難關中。爽軍屯櫟陽。潼關不守,車駕出幸,爽乃降賊。巢以爽為河陽節度使。巢賊敗,復表歸國,進位檢校司徒。



 時魏博韓簡軍勢方盛。中和元年四月,魏人攻河陽,大敗爽軍於修武,爽棄城遁走。簡令大將守河陽。乃出師討曹全晸於鄆州。十月,孟州人復誘爽,爽自金商率兵千人,復入河陽。乃犒勞魏人,令趙文弁率之而去。十一月,爽攻新鄉。簡自鄆來逆戰,軍於獲嘉西北。時簡將引魏人入趨關輔,誅除巢孽,自有圖王之志,三軍屢諫不
 從。偏將樂彥禎因眾心搖,說激之,牙軍奔歸魏州。爽軍乘之。簡鄉兵八萬大敗,奔騰亂死,清水為之不流。明年正月,簡為牙軍所殺,爽軍由是大振。



 及巢賊將敗,爽復歸國。爽雖起群盜,既貴之後,善於為理;所至法令澄清,人無怨嘆,人士以此多之。光啟二年,爽卒,帳中將劉經、張言以爽子仲方為孟帥。俄而蔡賊孫儒率眾攻之,城陷於賊,仲方歸於汴,儒遂據孟州。



 高駢,字千里,幽州人。祖崇文,元和初功臣,封南平王,自
 有傳。父承明,神策虞候。駢,家世仕禁軍,幼而朗拔,好為文,多與儒者游;喜言理道。兩軍中貴,翕然稱重,乃縻之勇爵,累歷神策都虞候。會黨項羌叛,令率禁兵萬人戍長武城。時諸將禦羌無功,唯駢伺隙用兵,出無不捷。懿宗深嘉之。西蕃寇邊,移鎮秦州,尋授秦州刺史、本州經略使。



 先是,李琢為安南都護,貪於貨賄,虐賦夷獠,人多怨叛;遂結蠻軍合勢攻安南,陷之。自是累年亟命將帥,未能收復。五年,移駢為安南都護。至則匡合五管之兵,
 期年之內,招懷溪洞,誅其首惡,一戰而蠻卒遁去,收復交州郡邑。又以廣州饋運艱澀,駢視其水路,自交至廣,多有巨石梗途,乃購募工徒,作法去之。由是舟楫無滯,安南儲備不乏,至今賴之。天子嘉其才,遷檢校工部尚書、鄆州刺史、天平軍節度觀察等使。治鄆之政,民吏歌之。



 南詔蠻寇巂州,渡滬肆掠。乃以駢為成都尹、劍南西川節度觀察等使。蜀土散惡,成都比無垣墉,駢乃計每歲完葺之費,甃之以磚甓。雉堞由是完堅。傳檄雲南,以
 兵壓境,講信修好,不敢入寇。進位檢校尚書右僕射、江陵尹、荊南節度觀察等使。乾符四年,進位檢校司空、潤州刺史、鎮海軍節度、浙江西道觀察等使,進封燕國公。



 時草賊王仙芝陷荊襄,宋威率諸道師討逐,其眾離散過江表。天子以駢前鎮鄆,軍民畏服,仙芝徒黨,鄆人也,故授駢京口節鉞,以招懷之。尋授諸道兵馬都統、江淮鹽鐵轉運等使。駢令其將張璘、梁纘分兵討賊,前後累捷,降其首領數十人。賊南趨嶺表,天子嘉之。六年冬,進
 位檢校司徒、楊州大都督府長史、淮南節度副大使知節度事,兵馬都統、鹽鐵轉運使如故。駢至淮南,繕完城壘,招募軍旅,土客之軍七萬。乃傳檄徵天下兵,威望大振。朝廷深倚賴之,進位檢校太尉、同平章事。



 既而黃巢賊合仙芝殘黨,復陷湖南、浙西州郡,眾號百萬。巢據廣州,求天平節鉞。朝廷議欲以南海節鉞授之。宰相盧攜與駢素善,以駢前在浙西已立討賊之效,今方集諸道之師於淮甸,不宜舍賊,以弱士心。鄭畋議且宜假賊方
 鎮以紓難。二人爭論於朝,以言詞不遜,由是兩罷之。駢方持兵柄,聞朝議異同,心頗不平之。



 廣明元年夏,黃巢之黨自嶺表北趨江淮,由採石渡江。張璘勒兵天長,欲擊之。駢怨朝議有不附己者,欲賊縱橫河洛,令朝廷聳振,則從而誅之。大將畢師鐸曰:「妖賊百萬,所經鎮戍若蹈無人之境。今朝廷所恃者都統,破賊要害之地,唯江淮為首。彼眾我寡,若不據津要以擊之,俾北渡長淮,何以扼束?中原陷覆必矣!」駢駭然曰:「君言是也。」即令出軍。
 有愛將呂用之者,以左道媚駢,駢頗用其言。用之懼師鐸等立功,即奪己權,從容謂駢曰:「相公勛業高矣,妖賊未殄,朝廷已有間言。賊若蕩平,則威望震主,功居不賞,公安稅駕耶?為公良畫,莫若觀釁,自求多福。」駢深然之,乃止諸將,但握兵保境而已。



 其年冬,賊陷河洛。中使促駢討賊,冠蓋相望。駢終逗撓不行。既而兩京覆沒,盧攜死。駢大閱軍師,欲兼並兩浙,為孫策三分之計。天子在蜀,亟命出師。中和二年五月,雉雊於揚州廨舍,占者云:「
 野鳥入室,軍府將空。」駢心惡之。其月,盡出兵於東塘,結壘而處,每日教閱,如赴難之勢。仍與浙西周寶書,請同入援京師。寶大喜,即點閱,將赴之,遣人偵之,知其非實。駢在東塘凡百日,復還廣陵,蓋禳雊雉之異也。



 僖宗知駢無赴難意,乃以宰臣王鐸為京城四面諸道行營兵馬都統,崔安潛副之,韋昭度領江淮鹽鐵轉運使。增駢階爵,使務並停。駢既失兵柄,又落利權,攘袂大詬,累上章論列,語詞不遜。其末章曰:



 臣伏奉詔命,令臣自省,更
 勿依違者。臣仰天訴地,自淚交流;如劍戟攢心,若湯火在己。只如黃巢大寇,圍逼天長小城,四旬有餘,竟至敗走。臣散徵諸道兵甲,盡出家財賞給,而諸道多不發兵,財物即為己有。縱然遣使徵得,敕旨不許過淮。其時黃巢殘兇,才及二萬,經過數千里,軍鎮盡若無人。只如潼關已東,止有一徑,其為險固,甚於井陘。豈有狂寇奔沖,略無阻礙,即百二之地,固是虛言。神策六軍,此時安在?陛下蒼黃西出,內官奔命東來,黎庶盡被殺傷,衣冠悉
 遭屠戮。今則園陵開毀,宗廟荊榛,遠近痛傷,遐邇嗟怨。



 雖然,奸臣未悟,陛下猶迷,不思宗廟之焚燒,不痛園陵之開毀。臣之痛也,實在於斯!此事見之多年,不獨知於今日。況自萑蒲盜起,朝廷徵用至多,上至帥臣,下及裨將,以臣所料,悉可坐擒,用此為謀,安能辦事?陛下今用王鐸,盡主兵權,誠知狂寇必殲,梟巢即覆。臣讀《禮》至宣尼射於矍相之圃,蓋觀者如堵墻,使子路出延射曰:「潰軍之將,亡國之大夫,與為人後者,不入於射也。」嚴誡如
 斯,圖功也,豈宜容易?陛下安忍委敗軍之將,陷一儒臣?崔安潛到處貪殘,只如西川,可為驗矣,委之副貳,詎可平戎?況天下兵驕,在處僭越,豈二儒士,能戢強兵,萬一乖張,將何救助?願陛下下念黎庶,上為宗祧,無使百代有抱恨之臣,千古留刮席之恥!臣但慮寇生東土,劉氏復興,即軹道之災,豈獨往日!乞陛下稍留神慮,以安宗社。



 今賢才在野,憸人滿朝,致陛下為亡國之君,此等計將安出?伏乞戮賣官鬻爵之輩,徵鯁直公正之臣,委之
 重難,置之左右,克復宮闕,莫尚於斯!若此時謗誹忠臣,沉埋烈士,匡復宗社,未見有期!臣受國恩深,不覺語切,無任憂懼之至。



 詔報駢曰:



 省表具悉。卿一門忠孝,三代勛庸,銘於景鐘,煥在青史。卿承祖父之訓,襲弓冶之基,起自禁軍,從微至著。始則囊錐露穎,稍有知音;尋則天驥呈才,急於試效。自秦州經略使,授交趾節旄,聯翩寵榮,汗漫富貴,未嘗斷絕,僅二十年。



 卿報國之功,亦可悉數。最顯赫者,安南拒蠻,至今海隅尚守。次則汶陽之日,
 政聲洽平。洎臨成都,脅歸驃信,三載之內,亦無侵凌。創築羅城,大新錦里,其為雄壯,實少比儔。渚宮不暇於施為,便當移鎮;建鄴才聞於安靜,旋即渡江。自到廣陵,並鐘多壘,即亦招降草寇,救援臨淮。大約昭灼功勛,不大於此數者。朝廷累加渥澤,靡吝徽章,位極三公,兵環大鎮。銅鹽重務,綰握約及七年;都統雄籓,幅圓幾於萬里。朕瞻如太華,倚若長城,凡有奏論,無不依允,其為托賴,豈愧神明?



 自黃巢肆毒咸京,卿並不離
 隋苑。豈金陵苑水,能遮鵝鸛之雄;風伯雨師,終阻帆檣之利?自聞歸止,寧免鬱陶。卿既安住蕪城,鄭畋以春初入覲,遂命上相,親領師徒,因落卿都統之名,固亦不乖事例。仍加封實,貴表優恩。何乃疑忿太深,指陳過當,移時省讀,深用震嗟。聊舉諸條,粗申報復。



 卿表云:「自是陛下不用微臣,固非微臣有負陛下」者。朕拔卿汶上,超領劍南,荊、潤、維、揚,聯居四鎮。綰利則牢盆在手,主兵則都統當權。直至京北、京南、神策諸鎮,悉在指揮之下,可知董制之雄。而乃
 貴作司徒,榮為太尉,以為不用,何名為用乎?



 卿又云:「若欲俯念舊勛,佇觀後效,何不以王鐸權位,與臣主持,必能糾率諸侯,誅鋤群盜」者。朕緣久付卿兵柄,不能翦滅元兇。自天長漏網過淮,不出一兵襲逐,奄殘京國,首尾三年;廣陵之師,未離封部,忠臣積望,勇士興譏。所以擢用元臣,誅夷巨寇,心期貔武,便掃欃槍。卿初委張璘,請放卻諸道兵士,辛勤召置,容易放還,璘果敗亡,巢益顛越。卿前年初夏,逞發神機,與京中朝貴書,題云:「得靈仙
 教導,芒種之後,賊必蕩平。」尋聞圍逼天長,必謂死在卿手,豈知魚跳鼎釜,狐脫網羅,遽過長淮,竟為大憝。都統既不能御遏,諸將更何以枝梧?果致連犯關河,繼傾都邑。從來倚仗之意,一旦控告無門,凝睇東南,惟增淒惻。及朕蒙塵入蜀,宗廟污於賊庭,天下人心,無不雪涕。既知歷數猶在,謳謠未移,則懷忠拗怒之臣,貯救難除奸之志,便須果決,安可因循?況恩厚者其報深,位重者其心急。此際天下義舉,皆望淮海率先。豈知近輔儒臣,先
 為首唱;而窮邊勇將,誓志平戎,關東寂寥,不見干羽。洎乎初秋覽表,方云仲夏發兵,便詔軍前,並移汶上。喜聞兵勢,渴見旌幢。尋稱宣潤阻艱,難從天討。謝玄破苻堅於淝水,裴度平元濟於淮西,未必儒臣不如武將!



 卿又云:「若不斥逐邪佞,親近忠良,臣既不能保家,陛下豈能安國?忽當今日,棄若寒灰」者。未委誰是忠良,誰為邪佞?終日寵榮富貴,何嘗不保其家;無人捍禦冠戎,所以不安其國。豈有位兼將相,使帶銅鹽,自謂寒灰,真同浪語。



 卿又云:「不通園陵之開毀,不念宗廟之焚燒,臣實痛之,實在茲也。」且龜玉毀於櫝中,誰之過也?鯨鯢漏於網外,抑有其由!卿手握強兵,身居大鎮,不能遮圍擒戮,致令脫漏猖狂,雖則上系天時,抑亦旁由人事。朕自到西蜀,不離一室之中,屏棄笙歌,杜絕游獵,蔬食適口,布服被身,焚香以望園陵,雪涕以思宗廟,省躬罪己,不敢遑安。「奸臣未悟」之言,誰人肯認?「陛下猶迷」之語,朕不敢當!



 卿又云:「自來所用將帥,上至帥臣,下及裨將,以臣所料,悉
 可坐擒,用此為謀,安能集事」者。且十室之邑,猶有忠信,天下至大,豈無英雄?況守固城池,悉嚴兵甲,縱非盡美,安得平欺?卿尚不能縛黃巢於天長,安能坐擒諸將?只如拓拔思恭、諸葛爽輩,安能坐擒耶?勿務大言,不堪垂訓。



 卿又云:「王鐸是敗軍之將,兼徵引矍相射義」者。昔曹沫三敗,終復魯讎;孟明再奔,竟雪秦恥。近代汾陽尚父,咸寧太師,亦曾不利鼓鼙,尋則功成鐘鼎。安知王鐸不立大勛?



 卿又云:「無使百代有抱恨之臣,千古留刮席之
 恥。但慮寇生東土,劉氏復興,即軹道之災,豈獨往日」者。我國家景祚方遠,天命未窮,海內人心,尚樂唐德。朕不荒酒色,不虧刑名,不結怨於生靈,不貪財於宇縣。自知運歷,必保延洪。況巡省已來,禎祥薦降;西蜀半年之內,聲名又以備全。塞北、日南,悉來朝貢;黠戛、善闡,並至梯航。但慮天寶、建中,未如今日;清宮復國,必有近期。卿云「劉氏復興」,不知誰為其首?遽言「刮席之恥」,比朕於劉盆子耶?仍憂「軹道之災」,方朕於秦子嬰也?雖稱直行,何太
 罔誣!三復斯言,尤深駭異。



 卿又云:「賢才在野,憸人滿朝,致陛下為亡國之君,此子等計將安出?伏乞戮賣官鬻爵之輩,徵鯁直公正之臣」者。且唐、虞之世,未必盡是忠良;今巖野之間,安得不遺賢彥?朕每令銓擇,亦遣訪求。其於選將料兵,安人救物,但屬收復之業,講求理化之基,自有長才,同匡大計。賣官鬻爵之士,中外必不有之,勿聽狂辭,以資游說。且朕遠違宮闕,寄寓巴邛,所失恩者甚多,尚不興怨,卿落一都統,何足介懷?況天步未傾,
 皇綱尚整,三靈不昧,百度猶存。但守君臣之軌儀,正上下之名分,宜遵教約,未可隳凌。朕雖沖人,安得輕侮!但以知卿歲久,許卿分深,貴存終始之恩,忽貯猜嫌之慮。所宜深省,無更過言!



 駢始以兵權,欲臨籓鎮,吞並江南;一朝失之,威望頓滅,陰謀自阻。故累表堅論,欲其復故。明年四月,王鐸與諸道之師敗賊關中,收復京城。駢聞之,悔恨萬狀。而部下多叛,計無所出,乃托求神仙,屏絕戎政,軍中可否,取決於呂用之。



 光啟初,僖宗再幸山南。
 李襜僭號,偽授駢中書令、諸道兵馬都統、江淮鹽鐵轉運等使。駢方怨望,而甘於偽署,稱籓納賄,不絕於途;晏安自得,日以神仙為事。呂用之又存暨工諸葛殷、張守一有長年之術,駢並署為牙將。於府第別建道院,院有迎仙樓、延和閣,高八十尺,飾以珠璣金鈿。侍女數百,皆羽衣霓服,和聲度曲,擬之鈞天。日與用之、殷、守一三人授道家法籙,談論於其間,賓佐罕見其面。



 府第有隋煬帝所造門屋數間,俗號中書門,最為宏壯,光啟元年,無
 故自壞。明年,淮南饑,蝗自西來,行而不飛,浮水緣城而入府第。道院竹木,一夕如翦,經像幢節,皆嚙去其首。撲之不能止。旬日之內,蝗自食啖而盡。



 其年九月,雨魚。是月十日夜,大星隕於延和閣前,其聲如雷,火光爍地。自二年十一月雨雪陰晦,至三年二月不解。比歲不稔,食物踴貴,道殣相望,饑骸蔽地。是月,浙西周寶為三軍所逐。駢喜,以為妖異當之。



 三月,蔡賊過淮口,駢令畢師鐸出軍御之。師鐸與高郵鎮將張神劍、鄭漢璋等,率行營
 兵反攻揚州。四月,城陷,師鐸囚駢於道院,召宣州觀察使秦彥為廣陵帥。既而蔡賊楊行密自壽州率兵三萬,乘虛攻城。城中米斗五十千,餓死大半。駢家屬並在道院,秦彥供給甚薄,薪蒸亦闕。奴僕徹延和閣闌檻煮革帶食之,互相篡啖。駢召從事盧涚謂之曰:「予三朝為國,粗立功名。比擺脫塵埃,自求清凈,非與人世爭利。一旦至此,神道其何望耶?」掩涕不能已。



 初,師鐸之入城也,愛將申及謂駢曰:「逆黨人數不多,即日弛於防禁,願奉令
 公潛出廣陵,依投支郡,以圖雪恥,賊不足平也。若持疑不決,及旦夕不得在公左右。」駢怯懼,不能行其謀。九月,師鐸出城戰敗,慮駢為賊內應,又有尼奉仙,自言通神,謂師鐸曰:「揚府災,當有大人死應之,自此善也。」秦彥曰:「大人非高令公耶?」即令師鐸以兵攻道院。侍者白駢曰:「有賊攻門。」曰:「此秦彥來。」整衣候之。俄而,亂卒升階,曳駢數之曰:「公上負天子恩,下陷揚州民,淮南塗炭,公之罪也。」駢未暇言,首已墮地矣。



 駢既死,左右奴客逾垣而遁,
 入行密軍。行密聞之,舉軍縞素,繞城大哭者竟日;仍焚紙奠酒,信宿不已。駢與兒侄死於道院,都一坎瘞之,裹之以氈。行密入城,以駢孫俞為判官,令主喪事。葬送未行而俞卒,後故吏鄺師虔收葬之。



 初,師鐸入城,呂用之、張守一出奔楊行密,詐言所居有金。行密入城,掘其家地下,得銅人,長三尺餘,身被桎梏,釘其心,刻「高駢」二字於胸,蓋以魅道厭勝蠱惑其心,以至族滅。



 畢師鐸者,曹州冤朐人。乾符初,與里人王仙芝嘯聚為
 盜,相與陷曹、鄆、荊、襄。師鐸善騎射,其徒自為「鷂子」。仙芝死,來降高駢。初敗黃巢於浙西,皆師鐸、梁纘之效也,頗寵待之。



 駢末年惑於呂用之,舊將俞公楚、姚歸禮皆為用之讒構見殺。師鐸意不自安,有愛妾復為用之所奪。



 光啟三年三月,蔡賊楊行密逼淮口,駢令師鐸率三百騎戍高郵。戍將張神劍亦怒用之,兩人謀自安之計。用之伺知,亟請召還。師鐸母在廣陵,遣信令師鐸遁去。或謂師鐸曰:「請殺神劍,並高郵之兵趨府,令公必殺用之
 為解。」又曰:「不如投徐州,則身存而家保。」師鐸曰:「非計也。呂用之誑惑主帥,塗炭生民,七八年來,鬼怨人怒。今日之事,安知天不假予誅妖亂而康淮甸耶?」又曰:「鄭漢璋是我歸順時副使,常切齒於用之,今率精兵在淮口。聞吾此舉,即樂從也。」乃趨淮口,與漢璋合,得兵千人。又相與至高郵,問計於張神劍。神劍曰:「公見事晚耶?用之一妖物耳,前受襄王偽命,作鎮廣州,遲留不行,志圖淮海節鎮。令公已奪其魄,彼一旦成事,焉能北面事妖物耶!」
 即割臂血為盟,推師鐸為盟主,稱大丞相。移檄郡縣,以誅用之、守一、殷為名,乃署其卒長唐宏、王朗、駱玄真、倪詳、逯本、趙簡等,分董其卒三千人。



 四月,趨廣陵,營於大明寺,揚州大駭。呂用之分兵城守,高駢登延和閣,聞鼓噪聲怪之。用之曰:「師鐸兵士回戈,止遏不得,適已隨宜處置,公幸勿憂。茍不聽,徒勞玄女一符耳。」師鐸陳兵數日,用之屢出戰,師鐸憂其不克,求救於宣州秦彥曰:「茍得廣陵,則迎公為帥。」彥令牙將秦稠,率兵三千助之。師
 鐸門客畢慕顏自城中出,曰:「人心已離,破之必矣!」秦稠軍至,兵威漸振。駢聞甚憂,謂用之曰:「吾以心腹仗爾,不能駕馭此輩,誤我何多?百姓遭罹饑饉,不可虐用。吾自枉手札喻師鐸,可令大將一人自行。」用之即以其黨許戡送駢書。師鐸怒曰:「梁纘、韓問何在?令爾來耶!」即斬之。用之選勁兵自衛。一日至道院,駢叱去之。乃令猶子傑握牙兵,令師鐸母作書,遣大將古鍔與師鐸子出城喻之。師鐸令子還,白曰:「不敢負令公恩德,正為淮南除弊。
 但斬用之、守一,即日退還高郵。」秦稠攻西南隅,城中應之,即日城陷。呂用之由參佐門遁走。駢聞師鐸至,改服俟之,與師鐸交拜,如賓主之儀。即日署為節度副使,漢璋、神劍皆署職事。



 秦稠點閱府庫,監守之,仍密召彥於宣州。或謂師鐸曰:「公昨舉兵誅二妖物,故人情樂從。今軍府已安,以事理論之,公宜還政高公,自典兵馬,戎權在手,取舍自由,籓鄰聞之,不失大義。議者皆言秦稠破城之日,已召秦彥。彥若為帥,兵權非足下有也。公感其
 援,但以金玉報之,阻其渡江,最為上策。若秦彥作帥,則楊行密朝聞夕至。如高令復帥,外寇必自卷懷。」師鐸猶豫未決,而秦彥軍至。



 五月,彥為節度使,署師鐸為行軍司馬,移居牙外,心頗不悅。是月,楊行密引軍攻揚州,彥兵拒戰繼敗。八月,師鐸與鄭漢璋出軍萬人擊行密,皆大敗而還,自是不復出。九月,師鐸殺高駢。十月,秦彥、師鐸突圍而遁。十一月,秦彥、師鐸引蔡賊孫儒之兵三萬圍揚州。行密求救於汴,硃全忠遣大將李璠率師淮口,
 以為聲援。孫儒以廣陵未下,而汴卒來,又慮秦彥、師鐸異志。四年正月,孫儒斬秦彥、師鐸於高郵之南,鄭漢璋亦死焉。



 秦彥者,徐州人,本名立。為卒,隸徐軍。乾符中,坐盜系獄,將死,夢人謂之曰:「爾可隨我。」及寤,械破,乃得逸去,因改名彥。乃聚徒百人,殺下邳令,取其資裝入黃巢軍。巢兵敗於淮南,乃與許勍俱降高駢,累奏授和州刺史。中和二年,宣歙觀察使竇潏病,彥以兵襲取之。遂代潏為觀
 察使,朝廷因而命之。



 光啟三年,揚州牙將畢師鐸囚其帥高駢,懼外寇來侵,乃迎彥為帥。彥召池州刺史趙鍠知宣州事,自率眾入揚州。師鐸推彥為帥。



 五月,壽州刺史楊行密率兵攻彥,遣其將張神劍令統兵屯灣頭山光寺。行密屯大雲寺,北跨長崗,前臨大道,自楊子江北至槐家橋,柵壘相聯。秦彥登城望之,懼形於色。令秦稠、師鐸率勁卒八千出鬥,為行密所掩,盡沒。稠死之。彥急求援於蘇州刺史張雄。雄率兵赴之,屯於東塘。重圍半
 年,城中芻糧並盡,草根木實、市肆藥物、皮囊革帶,食之亦盡。外軍掠人而賣,人五十千。死者十六七,縱存者鬼形鳥面,氣息奄然。張雄多軍糧,相約交市。城中以寶貝市米,金一斤,通犀帶一,得米五升。雄軍得貨,不戰而去。九月,畢師鐸出戰,又敗。自是日與秦彥相對嗟惋。問神尼奉仙何以獲濟,尼曰:「走為上計也。」十月,彥與師鐸突圍投孫儒,並為所殺。



 江淮之間,廣陵大鎮,富甲天下。自師鐸、秦彥之後,孫儒、行密繼踵相攻,四五年間,連兵不
 息,廬舍焚蕩,民戶喪亡,廣陵之雄富掃地矣!



 時溥,彭城人,徐之牙將。黃巢據長安,詔徵天下兵進討。中和二年,武寧軍節度使支詳遣溥與副將陳璠率師五千赴難。行至河陰,軍亂,剽河陰縣回。溥招合撫諭,其眾復集,懼罪,屯於境上。詳遣人迎犒,悉恕之,溥乃移軍向徐州。既入,軍人大呼,推溥為留後,送詳於大彭館。溥大出資裝,遣陳璠援詳歸京。詳宿七里亭,其夜為璠所殺,舉家屠害。溥以璠為宿州刺史,竟以違命殺詳。溥誅
 璠,又令別將帥軍三千赴難京師。天子還宮,授之節鉞。



 及黃巢攻陳州,秦宗權據蔡州,與賊連結。徐、蔡相近,溥出師討之。軍鋒益盛,每戰屢捷。黃巢之敗也,其將尚讓以數千人降溥,後林言又斬黃巢首歸徐州,時溥功居第一,詔授檢校太尉、中書令、鉅鹿郡王。宗權未平,仍授溥徐州行營兵馬都統。



 蔡賊平,硃全忠與之爭功,遂相嫌怨。淮南亂,朝廷以全忠遙領淮南節度,以平孫儒、行密之亂。汴人應援,路出徐方,溥阻之。全忠怒,出師攻徐。
 自光啟至大順六七年間,汴軍四集,徐、泗三郡,民無耕稼,頻歲水災,人喪十六七。溥窘蹙,求和於汴。全忠曰:「移鎮則可。」然之。朝廷以尚書劉崇望代溥,以溥為太子太師。溥懼出城見害,不受代。汴將龐師古陳兵於野,溥求援於兗州。硃瑾出兵救之,值大雪,糧盡而還。城中守陴者饑甚,加之病疫。汴將王重師、牛存節夜乘梯而入,溥與妻子登樓自焚而卒,景福二年四月也。地入於汴。



 硃瑄,宋州人。父慶,盜鹽抵法。瑄逃於青州,為王敬武牙
 卒。中和初,黃巢據長安,詔徵天下兵。敬武遣牙將曹全晸率兵三千赴難關西,以瑄為軍候。會青州警急,敬武召全晸還,路由鄆州。時鄆帥薛崇為草賊王仙芝所殺,鄆將崔君裕權知州事。全晸知其兵寡,襲殺君裕,據有鄆州,自稱留後。以瑄有功,署為濮州刺史,留將牙軍。



 光啟初,魏博韓簡欲兼並曹鄆,以兵濟河收鄆。全晸出兵逆戰,為魏軍所敗,全晸死之。瑄收合殘卒,保州城。韓簡攻圍半年,不能拔。會魏軍亂,退去。朝廷嘉之,授以節鉞。



 時瑄有眾三萬。其弟瑾,勇冠三軍,有爭天下之心。秦宗權之盛也,屢侵鄭、汴。硃全忠為賊所攻,甚窘,求救於瑄。瑄令硃瑾出師援之。擊敗秦宗權,全忠乃與瑄情極隆厚。



 全忠狡譎翻覆,虎視籓鄰。會宗權誅,乃急攻徐州。時溥求援於瑄,瑄與全忠書,請釋溥修好,偽許之。瑄以恩及全忠,遣使讓之,又令硃瑾出軍援溥。及徐、泗平,全忠乃移兵攻鄆。三四年間,每春秋入其境剽掠,人不得耕織,民為俘者十五六,瑄御備殫竭。景福末,與弟瑾合兩
 鎮之兵,與汴人大戰於魚山下,瑄、瑾俱敗,兵士陷沒。汴將硃友裕以長塹圍之。乾寧四年正月,城中食竭,瑄與妻榮氏出奔,至中都,為野人所害,傳首汴州。榮氏至汴州為尼。



 硃瑾,瑄之母弟,驍果善戰。初,乾符末,朝廷以將軍齊克讓為兗州節度。瑾將襲取之,乃求婚於克讓。及親迎,瑾選勇士衛從,禮會之夜竊發,逐克讓,遂據城稱留後。朝廷不獲已,以節鉞授之。及硃瑄平,汴人移兵攻兗,經年食盡,瑾出城求食。比還,為別將所拒,不得入。乃
 渡淮依楊行密。行密寵待之,用為壽州刺史,大敗汴軍於清口,自此全忠不敢以兵渡淮。瑾,楊溥時謀亂,為徐知訓所殺。



 史臣曰:疾風知勁草,世亂見忠臣,誠哉是言也!土運中微,賊巢僭越,籓伯勤王,赴難者,率有聲而無實。唯重榮斬賊使於近關,處存舉義師於安喜,橫身泣赴,不顧禍患,遂得義徒雲合,逆黨勢窮。宜乎服冕乘軒,傳家胙土。而重榮傷於峻法,嚴而少恩,禍發輿臺,誠悲枉橫。高駢
 起家禁旅,頗立功名,玩寇崇妖,致茲狼籍。後來勛德,可誡前軍。瑄、溥不以善取,固宜兇終。瑾持此狼心,安逃虎口?王綱之紊,群盜及茲,復何言哉!



 贊曰:王者撫運,居安慮危。不以德處,即為盜窺。乾坤蕩覆,生聚流離。讀駢章疏,可為涕洟!



\end{pinyinscope}