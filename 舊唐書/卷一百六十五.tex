\article{卷一百六十五}

\begin{pinyinscope}

 ○李光進弟光顏烏重胤王沛子逢李珙李祐董重質楊元卿子延宗劉悟子從諫孫稹劉沔石雄



 李光進,本河曲部落稽阿跌之族也。父良臣,襲雞田州
 刺史,隸朔方軍。光進姊適舍利葛旃,殺僕固瑒而事河東節度使辛云京。光進兄弟少依葛旃,因家於太原。



 光進勇毅果敢,其武藝兵略次於葛旃。肅宗自靈武觀兵,光進從郭子儀破賊,收兩京,累有戰功。至德中,授代州刺史,封範陽郡公,食邑二百戶。上元初,郭子儀為朔方節度,以軍討大同、橫野、清夷,範陽及河北殘寇,用光進為都知兵馬使。尋遷渭北節度使。永泰初,進封武威郡王。大歷四年,檢校戶部尚書,知省事。未幾,又轉檢校刑
 部尚書、兼太子太保。是歲冬十月,葬母於京城之南原,將相致祭者凡四十四幄,窮極奢靡,城內士庶,觀者如堵。



 元和四年,王承宗反。範希朝引師救易定,表光進為步都虞候,戰於木刀溝,光進有功。六年,拜銀青光祿大夫、檢校工部尚書,充單于大都護、振武節度使。詔以光進夙有誠節,克著茂勛,賜姓李氏。其弟光顏除洺州刺史,充本州團練使。兄弟恩澤同時,人皆嘆異。八年,遷靈武節度使。光進嘗從馬燧救臨洺,戰洹水,收河中,皆有功。
 前後軍中之職,無所不歷;中丞、大夫悉曾兼帶。先是救易定之師,光進、光顏皆在其行,故軍中呼光進為大大夫,光顏為小大夫。十年七月卒。



 光進兄弟少以孝睦推於軍中。及居母喪,三年不歸寢室。光顏先娶妻,其母委以家事。母卒,光進始娶。光顏使其妻奉管籥、家籍、財物,歸於其姒。光進命反之,且謂光顏曰:「新婦逮事母,嘗命以主家,不可改也。」因相持泣良久,乃如初。卒時年六十五,贈尚書左僕射。



 光顏與兄光進以葛旃善騎射,兄弟
 自幼皆師之,葛旃獨許光顏之勇健,己不能逮。及長,從河東軍為裨將,討李懷光、楊惠琳,皆有功。後隨高崇文平蜀,搴旗斬將,出入如神,由是稍稍知名。自憲宗元和已來,歷授代、洺二州刺史、兼御史大夫。



 九年,將討淮、蔡,九月,遷陳州刺史,充忠武軍都知兵馬使。逾月,遷忠武軍節度使、檢校工部尚書。會朝廷徵天下兵,環申、蔡而討吳元濟,詔光顏以本軍獨當一面。光顏於是引兵臨溵水,抗洄曲。明年五月,破元濟之師於時曲。初,賊眾晨
 壓光顏之壘而陣,光顏不得出,乃自毀其柵之左右,出騎以突之。光顏將數騎冒堅而沖之,出入者數四。賊眾盡識,矢集於身如蝟。其子攬光顏馬鞅,止其深入。光顏舉刃叱之,乃退。於是人爭奮躍。賊乃大潰,死者數千人。捷聲至京師,人人相賀。時伐蔡之師,大小凡十餘鎮,自裴度使還,唯奏光顏勇而知義,終不辱命。至是,果立功焉。是歲十一月,光顏又與懷汝節度烏重胤同破元濟之眾於小溵河,平其柵。



 初,都統韓弘令諸軍齊攻賊城,
 賊又徑攻烏重胤之壘。重胤御之,中數槍,馳請救於光顏。光顏以小溵橋賊之堡也,乘其無備,使田穎、宋朝隱襲而取之。乃平其城塹,由是克救重胤。韓弘以光顏違令,取穎及朝隱將戮之。穎及朝隱勇而材,軍中皆惋惜之。光顏畏弘不敢留。會中使景忠信至,知其情,乃矯詔令所在械系之。走馬入見,具以本末聞。憲宗赦忠信矯詔罪,令即往釋穎及朝隱。弘及光顏迭以表論。憲宗謂弘使曰:「穎等違都統令,固當處死。但光顏以其襲賊有
 功,亦可宥之。軍有三令五申,宜舍此以收來效。」及以詔諭弘,弘不悅。十一年,光顏連敗元濟之眾,拔賊凌雲柵,憲宗大悅,賜其告捷者奴婢銀錦。進位檢校尚書左僕射。



 十二年四月,光顏敗元濟之眾三萬於郾城。其將張伯良奔於蔡州,殺其賊什二三,獲馬千匹,器甲三萬聯,皆畫雷公符。仍書云:「速破城北軍。」尋而郾城守將鄧懷金請以城降。光顏許之,而收郾城。



 初,鄧懷金以官軍圍青陵城,絕其歸路,懷金懼,謀於郾城令董昌齡。昌齡母
 素誡其子令降,昌齡因此勸懷金歸款於光顏,且曰:「城中之人,父母妻子皆質於蔡州,如不屈而降,則家盡屠矣。請來攻城,我則舉烽求救。救兵將至,官軍逆擊之必敗,此時當以城降。」光顏從之,賊果敗走。於是昌齡執印,帥吏列於門外,懷金與諸將素服倒戈列於門內;光顏受降,乃入羅城,其城自壞五十餘步。



 時韓弘為汴帥,驕矜倔強。常倚賊勢索朝廷姑息,惡光顏力戰,陰圖撓屈,計無所施。遂舉大梁城求得一美婦人,教以歌舞弦管
 六博之藝,飾之以珠翠金玉衣服之具,計費數百萬,命使者送遺光顏,冀一見悅惑而怠於軍政也。使者即齎書先造光顏壘曰:「本使令公德公私愛,憂公暴露,欲進一妓,以慰公征役之思,謹以候命。」光顏曰:「今日已暮,明旦納焉。」詰朝,光顏乃大宴軍士;三軍咸集,命使者進妓。妓至,則容止端麗,殆非人間所有,一座皆驚。光顏乃於座上謂來使曰:「令公憐光顏離家室久,舍美妓見贈,誠有以荷德也。然光顏受國家恩深,誓不與逆賊同生日
 月下。今戰卒數萬,皆背妻子,蹈白刃,光顏奈何以女色為樂?」言訖,涕泣嗚咽。堂下兵士數萬,皆感激流涕。乃厚以縑帛酬其來使,俾領其妓自席上而回,謂使者曰:「為光顏多謝令公。光顏事君許國之心,死無貳矣!」自此兵眾之心,彌加激勵。



 及裴度至行營,率賓從於方城沱口觀板築、五溝。賊遽至,注弩挺刃,勢將及度。光顏決戰於前以卻之。時光顏預慮其來,先使田布以二百騎伏於溝中,出賊不意交擊之,度方獲免。布又先扼其溝中歸
 路,賊多棄騎越溝,相牽墜壓而死者千餘人。是日微光顏之救,度幾陷矣。是月,賊知光顏勇冠諸將,乃悉其眾出當光顏之師。時李愬乘其無備,急引兵襲蔡州,拔之,獲元濟。董重質棄洄曲軍,入城降愬。光顏知之,躍馬入賊營,大呼以降,賊眾萬餘人,皆解甲投戈請命。賊平,加檢校司空。



 十三年春,命中官宴光顏於居第,賜芻米二十餘車。憲宗又御麟德殿召對,賜金帶錦彩。朝廷東討李師道,授光顏義成軍節度使。至鎮,尋赴行營。數旬之
 內,再敗賊軍於濮陽,殺戮數千人,進軍深入。



 十四年,西蕃入寇,移授邠寧節度使。時鹽州為吐蕃所毀,命李文悅為刺史,令光顏充勾當修築鹽州城使。仍許以陳許六千人隨赴邠寧。是歲,吐蕃侵涇原。自田縉鎮夏州,以貪猥侵撓黨項羌,乃引吐蕃入寇。及蕃軍攻涇州,邊將郝玼血戰始退。初,光顏聞賊攻涇州,料兵赴救,邠師喧然曰:「人給五十千而不識戰陣,彼何人也!常額衣資不得而前蹈白刃,此何人也!」憤聲恟々不可遏。光顏素得
 士心,曲為陳說大義,言發涕流。三軍感之,亦泣下,乃欣然即路,擊賊退之。



 穆宗即位,就加特進,仍與一子四品正員官。尋詔赴闕,賜開化里第,進加同中書門下平章事。穆宗以光顏功冠諸將,故召赴闕,宴賜優給。已而帶平章復鎮,所以報勛臣也。



 長慶初,遷鳳翔節度使,依前檢校司空、同中書門下平章事。歲末,復授許州節度使。朝廷以光顏昔鎮陳許,頗得士心,將討鎮、冀,故有此拜。赴鎮日,宰相百僚以故事送別於章敬寺,穆宗御通化
 門臨送之,賜錦彩、銀器、良馬、玉帶等物。二年,討王廷湊,命光顏兼深州行營諸軍節度使。光顏既受命而行,懸軍討賊,艱於饋運。朝廷又以滄、景、德、棣等州俾之兼管,以其鄰賊之郡,可便飛挽。光顏以朝廷制置乖方,賊帥連結,未可朝夕平定,事若差跌,即前功悉棄,乃懇辭兼鎮。尋以疾作,表祈歸鎮。朝廷果討賊無功而赦廷湊。四年,敬宗即位,正拜司徒。



 汴州李絺逐其帥叛,詔光顏率陳許之師討之。營於尉氏,俄而誅絺。遷太原尹、北京留
 守、河東節度使,進階開府儀同三司,仍於正衙受冊司徒兼侍中。二年九月卒,年六十六,廢朝三日,贈太尉,謚曰忠。



 烏重胤,潞州牙將也。元和中,王承宗叛,王師加討。潞帥盧從史雖出軍,而密與賊通。時神策行營吐突承璀與從史軍相近,承璀與重胤謀,縛從史於帳下。是日,重胤戒嚴,潞軍無敢動者。憲宗賞其功,授潞府左司馬,遷懷州刺史,兼充河陽三城節度使。會討淮、蔡,用重胤壓境,
 仍割汝州隸河陽。自王師討淮西三年,重胤與李光顏掎角相應,大小百餘戰,以至元濟誅。就加檢校尚書右僕射,轉司空。蔡將有李端者,過溵河降重胤。其妻為賊束縛於樹,臠食至死,將絕,猶呼其夫曰:「善事烏僕射。」其得人心如此。



 元和十三年,代鄭權為橫海軍節度使。既至鎮,上言曰:「臣以河朔能拒朝命者,其大略可見。蓋刺史失其職,反使鎮將領兵事。若刺史各得職分,又有鎮兵,則節將雖有祿山、思明之奸,豈能據一州為叛哉?所
 以河朔六十年能拒朝命者,只以奪刺史、縣令之職,自作威福故也。臣所管德、棣、景三州,已舉公牒,各還刺史職事訖,應在州兵,並令刺史收管。又景州本是弓高縣,請卻廢為縣,歸化縣本是草市,請廢縣依舊屬德州。」詔並從之。由是法制修立,各歸名分。



 及屯軍深州,重胤以朝廷制置失宜,賊方憑凌,未可輕進,觀望累月。穆宗急於誅叛,遂以杜叔良代之,以重胤檢校司徒,兼興元尹,充山南西道節度使。召至京師,復以本官為天平軍節
 度、鄆曹濮等州觀察等使。李同捷據滄州,請襲父位,朝廷不從。議者慮狡童拒命,欲以重臣代。乃移鎮兗海,加太子太師、平章事,俾兼領滄景節度,仍舊割齊州隸之,蓋望不勞師而底定。制出旬日,重胤卒,贈太尉。



 重胤出自行間,及為長帥,赤心奉上。能與下同甘苦,所至立功,未嘗矜伐。而善待賓寮,禮分同至,當時名士,咸願依之。身歿之日,軍士二十餘人,皆割股肉以為祭酹,雖古之名將,無以加焉。



 子漢弘嗣,起復授左領軍衛將軍。漢弘
 上表乞終服紀,文宗嘉詔從之。服闋,方授官。



 王沛,許州人。年十八,有勇決。許州節度使上官涚奇其才,以女妻之,署為牙門將。及涚卒,子婿田偁迫脅涚子,欲邀襲位,懼監軍使不順其事,將結謀伏兵以圖之。沛竊知其謀,密告監軍,因盡擒其黨於伏匿之所。監軍範日用以其事聞,德宗乃以陳許行軍司馬劉昌裔總統其軍,賜沛手詔,令護涚之子赴上都。既至,召見,德宗謂之曰:「據卿忠義,寵宜加等。但昌裔所奏,只請加監察御
 史,朕意殊為不足。卿速歸,便宣付昌裔,更令奏來。」遂驛騎而還。未至許州,拜開府儀同三司、兼御史中丞,依前本職。



 吳元濟反,李光顏受命攻討,奇沛節概,署行營兵馬使,別統勁兵屯於近郊。及軍合,連破蔡寇。頻詔進軍,諸將觀望,無敢先渡溵河。沛率兵五千,夜渡溵河合流口,徑扼賊喉而成城。自是,河陽、宣武、太原、魏博等軍繼渡,掎角進攻郾城。沛先結壘與賊對,賊將鄧懷金率眾面縛而降。蔡賊平。沛隨李光顏入朝,光顏具陳沛功,加
 御史大夫。



 既還鎮,光顏受詔討鄆寇。及李師道誅,詔分許州兵戍於邠,以沛為都將,救鹽州,擊退吐蕃。以功加寧州刺史,遷陳州。李絺反,詔沛兼忠武節度副使,率師討絺。絺平,加檢校右散騎常侍,遷兗海沂密節度、觀察等使。此邦新造,人情獷驁,沛明申法令,選蒐軍政,期年大理。明年,改檢校工部尚書,充忠武軍節度、陳許蔡觀察等使。卒於鎮,贈右僕射。子逢。



 逢,少沉勇,從父征伐有功,為忠武都知兵馬使。太和中,入宿衛,歷諸衛將軍。從
 石雄、劉沔破回紇於天德。性果決,用法嚴。其時有二千人不上陣,官賜賞給,逢皆不與。或非之,逢曰:「健兒向前冒白刃,若無功而賞,其如冒刃者何?」王宰攻劉稹,逢領陳許七千人屯翼城,代田令昭。賊平,檢校左散騎常侍。累遷至忠武軍節度、陳許觀察等使。



 李珙,山東甲姓,代修婚姻。至珙,不好讀書,唯以弓馬為務。長六尺餘,氣貌魁岸。嘗詣澤潞謁李抱真,異之,將選為衙門將,旋以酒酣使氣,復欲棄之。都將王虔休謂抱
 真曰:「李珙,奇士也,若不能用,不如殺之,無為他人所得。」



 抱真死,虔休為帥,乃依虔休,累為昭義大將。吐突承璀之擒盧從史,烏重胤實預其謀,珙初不知,將救從史。聞重胤受朝旨,乃觀望不進,重胤以此德之。後領河陽,乃置於麾下。然朝廷以與從史厚善,竟出為北邊一校。



 元和十年,征淮西,重胤懇表為諸道行營都虞候,詔特從之,俄以母憂去職。服闋,除右武衛上將軍。長慶四年八月卒,年六十四,廢朝一日。



 李祐,本蔡州牙將。事吳元濟,驍勇善戰。自王師討淮西,祐為行營將,每抗官軍,皆憚之。元和十二年,為李愬所擒。愬知祐有膽略,釋其死,厚遇之。推誠定分,與同寢食,往往帳中密語,達曙不寐。人有耳屬於外者,但屢聞祐感泣聲。而軍中以前時為祐殺傷者多,營壘諸卒會議,皆恨不殺祐。愬以眾情歸怨,慮不能全,因送祐於京師,乃上表救之。憲宗特恕,遂遣祐賜愬。愬大喜,即以三千精兵付之。祐聽言,無有所疑,竟以祐破蔡,擒元濟。以功
 授神武將軍,遷金吾將軍、檢校左散騎常侍、夏州刺史、御史大夫、夏綏銀宥節度使。



 寶歷初,入為右金吾大將軍。尋以吐蕃入寇,出為涇州刺史、涇原節度使。太和初,討李同捷,遷檢校戶部尚書、滄州刺史、滄德景節度使。太和三年五月卒。



 董重質,本淮西牙將,吳少誠之子婿也。性勇悍,識軍機,善用兵。及元濟拒命,重質又為謀主,領大軍當王師,連歲不拔,皆重質之謀也。元和十二年,宰相裴度督兵淮
 西,至郾城,元濟乃悉發左右及守城之卒,委重質而拒度。時李愬乘虛入蔡。既擒元濟,重質之家在蔡,愬乃安恤之,仍使其子持書禮以召重質。重質見其子,知城已陷,及元濟囚窘之狀,乃慨然以單騎歸愬,白衣叩伏。愬揖登階,以賓禮與之食。憲宗欲殺之,愬奏許以不死而來降,請免之,且乞於本軍驅使。於是,貶春州司戶參軍。



 明年,轉太子少詹事,委武寧軍收管驅使,仍加金紫。十五年,徵入,授左神武軍將軍,知軍事,兼御史中丞。仍賜
 金帛,與有功者等。尋授鹽州刺史,又遷左右神策及諸道劍南西川行營節度使、檢校左散騎常侍。太和四年,又轉夏綏銀宥節度使。五年,就加檢校工部尚書。重質訓兵立法,羌戎畏服。八年八月卒,贈尚書右僕射。



 楊元卿,祖子華,德州安陵縣丞。父寓,申州鐘山縣令。元卿少孤,慷慨有才略。及冠,尚漂蕩江嶺之表,縱游放言,人謂之狂生。時吳少誠專蔡州,朝廷姑息之。元卿白衣謁見,署以劇縣,旋闢為從事,奏授試大理評事。亦事少
 陽,後奏轉監察裏行。因上奏,宰相李吉甫深加慰納,自是一歲或再隨奏至京師。元卿每與少陽言,諭以大義。乃為兇黨所構,賴節度判官蘇肇保持,故免。元卿潛奉朝廷,內耗少陽之事。



 及少陽死,其子元濟繼立。元卿說曰:「先尚書性吝,諸將皆饑寒。今須布惠以自固也。府中有無,元卿熟知之,曷若散聘諸道,卑辭厚禮,以丈人行呼群帥,庶幾一助,而諸將大獲矣。元卿願將留後表上聞,朝廷安得不從哉?」元濟許之。元卿即日離蔡,以賊勢
 盈虛條奏,潛請詔諸道拘留使者。及元濟覺,元卿妻陳氏並四男並為元濟所殺,同圬一射垛。蘇肇以保持元卿,亦同日被害。詔授元卿岳王府司馬,尋遷太子僕射。



 元和十三年,授蔡州刺史、兼御史中丞。未行,改授光祿少卿。初,朝廷比令元卿與李愬會議,於唐州東境選要便處,權置行蔡州。如百姓官健有歸順者,便準敕優恤,必令全活。既而召見,元卿遽奏請借度支錢,及言事頗多不合旨。宰相裴度亦以諸將討賊三年,功成在旦暮,
 如更分土地與元卿,即恐相侵生事,故罷前命而改授焉。是歲,既平淮西,元卿奏曰:「淮西甚有寶貨及犀帶,臣知之,往取必得。」上曰:「朕本討賊,為人除害,今賊平人安,則我求之得矣。寶貨犀帶,非所求也,勿復此言。」是月,詔授左金吾衛將軍。未幾,改汾州刺史,復徵為左金吾衛將軍。



 長慶初,易置鎮、魏守臣。元卿詣宰相深陳利害,並具表其事。後穆宗感悟,賜白玉帶,旋授檢校左散騎常侍、涇州刺史、涇原渭節度觀察等使,兼充四鎮北庭行
 軍。元卿乃奏置屯田五千頃,每屯築墻高數仞,鍵閉牢密,卒然寇至,盡可保守。加檢校工部尚書。營田成,復加使號。居六年,涇人論奏,為立德政碑,移授懷州刺史,充河陽三城節度觀察等使。太和五年,就加檢校司空,進階光祿大夫,以其營田納粟二十萬石,以裨經費故也。是歲,改授汴宋亳觀察等使。凡所廢置,皆有弘益,詔並從之。年七十,寢疾,歸洛陽,詔授太子太保。是歲八月卒,廢朝三日,贈司徒。元卿始以毀家效順,累授方鎮。然性
 險巧,所至好聚斂,善結交,涇人得情,亦由此也。



 子延宗,開成中為磁州刺史,坐謀逐河陽節度使以自立,為其黨所告,臺司推鞫得實,誅之。



 劉悟,正臣之孫也。正臣本名客奴。天寶末,祿山叛,平盧軍節度使柳知晦受賊偽署。客奴時職居牙門,襲殺知晦,馳章以聞。授平盧軍節度使,賜名正臣。



 悟少有勇力。叔逸準為汴帥,積緡錢數百萬於洛中;悟輒破扃鐍,悉盜用之。既而懼,亡歸李師古。始亦未甚知,後因擊球馳
 突,沖師古馬僕,師古怒,將斬之。悟猛以氣語押觸師古,師古奇而免之。因令管壯士,將後軍,累署衙門右職,奏授淄青節度都知兵馬使、兼監察御史。



 元和末,憲宗既平淮西,下詔誅師道。遣悟將兵拒魏博軍,而數促悟戰。悟未及進,馳使召之。悟度使來必殺己,乃偽疾不出,令都虞候往迎之。使者亦果以誠告其人,云「奉命殺悟以代悟」。都虞候即時先還,悟劾之得其實,乃召諸將與謀曰:「魏博田弘正兵強,出戰必敗,不出則死。今天子所誅
 者,司空一人而已,悟與公等皆為所驅迫,使就其死。何如殺其來使,整戈以取鄆,立大功,轉危亡為富貴耶!」眾咸曰:「善,唯都將所命!」悟於是立斬其使,以兵取鄆,圍其內城,兼以火攻其門。不數刻,擒師道並男二人,並斬其首以獻。擢拜悟檢校工部尚書、兼御史大夫、義成軍節度使,封彭城郡王,仍賜實封五百戶,錢二萬貫,莊、宅各一區。十五年正月入覲,又加檢校兵部尚書,餘如故。



 穆宗即位,以恩例遷檢校尚書右僕射。是歲十月,移鎮澤
 潞,旋以本官兼平章事。



 長慶元年,幽州大將硃克融叛,囚其帥張弘靖,朝廷求名將以鎮漁陽。乃加悟檢校司空、平章事,充盧龍軍節度使。悟以幽州方亂,未克進討,請授之節鉞,徐圖之。乃復以悟為澤潞節度,拜檢校司徒,兼太子太傅,依前平章事。時監軍劉承偕頗恃恩權,常對眾辱悟;又縱其下亂法,悟不能平。異日有中使至,承偕宴之,請悟,悟欲往。左右皆曰:「往則必為其困辱矣!」軍眾因亂,悟不止之。乃擒承偕至牙門,殺其二僕,欲並
 害承偕,悟救之獲免。朝廷不獲已,貶承偕。自是悟頗縱恣,欲效河朔三鎮。朝廷失意不逞之徒,多投寄潞州以求援。往往奏章論事,辭旨不遜。



 寶歷元年九月病卒,贈太尉。遺表請以其子從諫繼纘戎事。敬宗下大臣議。僕射李絳以澤潞內地,與三鎮事理不同,不可許。宰相李逢吉、中尉王守澄受其賂,曲為奏請。



 從諫自將作監主簿,起復雲麾將軍,守金吾衛大將軍同正、檢校左散騎常侍、兼御史大夫,充昭義節度副大使,知節度觀察等
 留後。二年,加金吾上將軍、檢校工部尚書,充昭義節度等使。文宗即位,進檢校司空。六年十二月入覲。七年春歸籓,加同中書門下平章事。九年,李訓事敗,宰相王涯等四人被禍。時涯兼掌邦計,雖不與李訓同謀,然不自異於其間,既死非其罪。從諫素德涯之私恩,心頗不平,四上章請涯等罪名,仇士良輩深憚之。是時中官頗橫,天子不能制。朝臣日憂陷族,賴從諫論列,而鄭覃、李石方能粗秉朝政。



 先是,有蕭洪者,詐稱太后弟,因仇士良保
 任,許之厚賂。及洪累授方鎮,納賂不滿士良之志,士良怒,遣人上書論洪非太后之親,又以蕭本者為太后弟。從諫深知內宮之故,乃自潞府飛章論之曰:「臣聞造偽以亂真者,匹夫知之尚不可,況天下皆知乎?執疏以為親者,在匹夫之家尚不可,況處大國之朝乎?臣受國恩深,奉公心切,知有此失,安敢不言!伏唯皇帝陛下仁及萬方,孝敦九族,而推心無黨,唯理是求。微臣所以不避直言,切論深事。伏見金吾將軍蕭本,稱是太后親弟,受
 此官榮。今喧然國都,迨聞籓府,自上及下,異口同音,皆言蕭弘是真,蕭本是偽。臣傍聽眾論,遍察群情,咸思發明,以正名分。今年二月,其蕭弘投臣當道,求臣上聞,自言:比者福建觀察使唐扶及監軍劉行立具審根源,已曾論奏。其時屬蕭本得為外戚,來自左軍,臺司既不敢研窮,聖意遂勒還鄉里。自茲議論,轉益沸騰。臣亦令潛問左軍,榷論大體,而士良推至公之道,發不黨之言。蓋蕭本自度孤危,妄有憑恃。伏以名居國舅,位列朝班,而
 真偽不分,中外所恥。切慮皇太后受此罔惑,已有恩情,若含垢於一時,終取笑於千古。伏乞追蕭弘赴闕,與蕭本對推,細詰根源,必辨真偽。」詔令三司使推按。帝以二蕭雖詐,托名太后之宗,不欲誅之,俱流嶺表。從諫進位檢校司徒。會昌三年卒。



 大將郭誼等匿喪,用其侄稹權領軍務。時宰相李德裕用事,素惡從諫之奸回,奏請劉稹護喪歸洛,以聽朝旨。稹竟叛。德裕用中丞李回奉使河朔,說令三鎮加兵討稹;乃削奪稹官,命徐許滑孟魏
 鎮幽並八鎮之師,四面進攻。四年,郭誼斬稹,傳首京師。



 從諫妻裴氏。初,稹拒命,裴氏召集大將妻同宴,以酒為壽,泣下不能已。諸婦請命,裴曰:「新婦各與汝夫文字,勿忘先相公之拔擢,莫效李丕背恩,走投國家。子母為托,故悲不能已也。」諸婦亦泣下,故潞將叛志益堅。稹死,裴亦以此極刑。稹族屬昆仲九人,皆誅。



 劉沔,許州牙將也。少事李光顏為帳中親將。元和末,光顏討吳元濟,常用沔為前鋒。蔡將有董重質者,守洄曲,
 其部下乘騾即戰,號「騾子軍」,最為勁悍,官軍常警備之。沔驍銳善騎射,每與騾軍接戰,必冒刃陷堅,俘馘而還,故忠武一軍,破賊第一。淮、蔡平,隨光顏入朝。憲宗留宿衛,歷三將軍。歷鹽州刺史、天德軍防禦使,在西北邊累立奇效。



 太和末,河西黨項羌叛。沔以天德之師屢誅其酋渠,移授振武節度使,檢校右散騎常侍、單于大都護。開成中,黨項雜虜大擾河西。沔率吐渾、契苾、沙陁三部落等諸族萬人、馬三千騎,徑至銀、夏討襲,大破之。俘獲
 萬計,告捷而還。以功加檢校戶部尚書。會昌初,回紇部饑,烏介可汗奉太和公主至漢南求食。過杷頭峰,犯雲、朔、北川。朝廷以太原重地,控扼諸戎,乃移沔河東節度使、檢校尚書左僕射、太原尹、北京留守。詔與幽州張仲武協力招撫回鶻,竟破虜寇,迎公主還宮。以功進位檢校司空,尋改滑州刺史、義成軍節度使。



 四年,潞帥劉從諫卒,子稹匿喪,擅主留務,要求旌鉞。武宗怒,命忠武節度使王宰、徐州節度李彥佐等,充潞府西南面招撫使。
 遂復授沔太原節度,充潞府北面招討使。沔與張仲武不協,方徵兵幽州,乃移沔為鄭滑節度使,進位檢校司徒。既而以疾求歸洛陽,授太子太保,卒。



 初,沔為忠武小校,從李光顏討淮西,為捉生將。前後遇賊血戰,鋒刃所傷,幾死者數四。嘗傷重臥草中,月黑不知歸路,昏然而睡,夢人授之雙燭,曰:「子方大貴,此行無患,可持此而還。」既行,炯然有雙光在前。自後破虜危難,每行常有此光。及罷鎮後,雙光息。五年,李德裕出鎮,罷沔為太子太保。
 明年,以太子太保致仕卒。



 石雄,徐州牙校也。王智興之討李同捷,以雄為石廂捉生兵馬使。勇敢善戰,氣凌三軍。自智興以兵臨賊境,率先收棣州,雄先驅渡河,前無堅陣。徐人伏雄之撫待,惡智興之虐,欲逐之而立雄。智興以軍在賊境,懼其變生,因其立功,請授一郡刺史。朝廷徵赴京師,授壁州刺史。智興尋殺雄之素相善諸將士百餘人,仍奏雄搖動軍情,請行誅戮。文宗雅知其能,惜之,乃長流白州。



 太和中,
 河西黨項擾亂,選求武士。乃召還,隸振武劉沔軍為裨將,累立破羌之功。文宗以智興故,未甚提擢,而李紳、李德裕以崔群舊將,素嘉之。



 會昌初,回鶻寇天德,詔命劉沔為招撫回鶻使。三年,回鶻大掠雲、朔北邊,牙於五原。沔以太原之師屯於雲州。沔謂雄曰:「黠虜離散,不足驅除。國家以公主之故,不欲急攻。今觀其所為,氣凌我輩。若稟朝旨,或恐依違。我輩捍邊,但能除患,專之可也。公可選驍健,乘其不意,徑趨虜帳,彼以疾雷之勢,不暇枝
 梧,必棄公主亡竄。事茍不捷,吾自繼進,亦無患也。」雄受教,自選勁騎,得沙陁李國昌三部落,兼契苾拓拔雜虜三千騎,月暗夜發馬邑,徑趨烏介之牙。時虜帳逼振武,雄既入城,登堞視其眾寡。見氈車數十,從者皆衣硃碧,類華人服飾。雄令諜者訊之:「此何大人?」虜曰:「此公主帳也。」雄喻其人曰:「國家兵馬欲取可汗。公主至此,家國也,須謀歸路。俟兵合時不得動帳幕。」雄乃大率城內牛馬雜畜及大鼓,夜穴城為十餘門。遲明,城上立旗幟炬火,
 乃於諸門縱其牛畜,鼓噪從之,直犯烏介牙帳。炬火燭天,鼓噪動地,可汗惶駭莫測,率騎而奔。雄率勁騎追至殺胡山,急擊之。斬首萬級,生擒五千,羊馬車帳皆委之而去。遂迎公主還太原。以功加檢校左散騎常侍、豐州刺名、兼御史大夫、天德防禦等使。



 雄沉勇徇義,臨財甚廉。每破賊立功,朝廷特有賜與,皆不入私室;置於軍門,首取一分,餘並分給,以此軍士感義,皆思奮發。累遷檢校左僕射、河中尹、河中晉絳節度使。



 俄而昭義劉從諫
 卒,其子稹擅主軍務,朝議問罪。令徐帥李彥佐為潞府西南面招撫使,以晉州刺史李丕為副。時王宰在萬善柵,劉沔在石會,相顧未進。雄受代之翌日,越烏嶺,破賊五砦,斬獲千計。武宗聞捷大悅,謂侍臣曰:「今之義而有勇,罕有雄之比者。」雄既率先破賊,不旬日,王宰收天井關,何弘敬、王元逵亦收磁洺等郡。先是潞州狂人折腰於市,謂人曰:「雄七千人至矣。」劉從諫捕而誅之。及稹危蹙,大將郭誼密款請斬稹歸朝,軍中疑其詐。雄倡言曰:「
 賊稹之叛,郭誼為謀主。今請斬稹,即誼自謀,又何疑焉?」武宗亦以狂人之言,詔雄以七千兵受降。雄即徑馳潞州降誼,盡擒其黨與。賊平,進加檢校司空。



 王宰,智興之子,於雄不足,雄以轅門子弟善禮之。然討潞之役,雄有始卒之功,宰心惡之。及李德裕罷相,宰黨排擯雄,罷鎮。既而聞德裕貶,發疾而卒。



 史臣曰:古所謂名將者,不必蒙輪拔距之材,拉虎批熊之力;要當以義終始,好謀而成。而阿跌昆仲,稟氣陰山,
 率多令範。讓家權於主婦,拒美妓於奸臣;章武恢復之功,義師之效也。重胤忠於事上,仁於撫下,淮、蔡之役,勛亞光顏;殿邦之臣也,不可多得。王沛之擒僚婿,李祐之執賊渠,皆因事立功,轉禍為福。智則智矣,仁者不為!而劉悟自恃太尤,世邀纘襲,至於赤族,報亦晚耶!雄、沔負羽邊城,聲馳沙漠,奉迎貴主,摧破昆戎,不亦壯乎!雄能感於知己,不為無義,美哉!



 贊曰:淮、鄆砥平,義將輸誠。二兇受縛,亦其同
 惡。毀義棄忠,必殄爾宗。孰稱善將?劉沔、
 石雄。



\end{pinyinscope}