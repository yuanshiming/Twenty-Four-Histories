\article{卷一百三十五}

\begin{pinyinscope}

 ○李
 勉李皋子象古道古



 李勉,字玄卿,鄭王元懿曾孫也。父擇言,為漢褒相岐四州刺史、安德郡公,所歷皆以嚴幹聞。在漢州,張嘉貞為益州長史、判都督事,性簡貴,待管內刺史禮隔,而引擇
 言同榻,坐談政理,時人榮之。勉幼勤經史,長而沉雅清峻,宗於虛玄,以近屬陪位,累授開封尉。時升平日久,且汴州水陸所湊,邑居龐雜,號為難理,勉與聯尉盧成軌等,並有擒奸擿伏之名。



 至德初,從至靈武,拜監察御史。屬朝廷右武,勛臣恃寵,多不知禮。大將管崇嗣於行在朝堂背闕而坐,言笑自若,勉劾之,拘於有司,肅宗特原之,嘆曰:「吾有李勉,始知朝廷尊也。」遷司膳員外郎。時關東獻俘百餘,詔並處斬,囚有仰天嘆者,勉過問之,對曰:「
 某被脅制守官,非逆者。」勉乃哀之,上言曰:「元惡未殄,遭點污者半天下,皆欲澡心歸化。若盡殺之,是驅天下以資兇逆也。」肅宗遽令奔騎宥釋,由是歸化日至。克復西京,累歷清要,四遷至河南少尹。累為河東節度王思禮、朔方河東都統李國貞行軍司馬,尋遷梁州都督、山南西道觀察使。勉以故吏前密縣尉王晬勤幹,俾攝南鄭令,俄有詔處死,勉問其故,乃為權幸所誣。勉詢將吏曰:「上方藉牧宰為人父母,豈以譖言而殺不辜乎!」即停詔
 拘晬,飛表上聞,晬遂獲宥,而勉竟為執政所非,追入為大理少卿。謁見,面陳王晬無罪,政事條舉,盡力吏也。肅宗嘉其守正,乃除太常少卿。王晬後以推擇拜大理評事、龍門令,終有能名,時稱知人。



 肅宗將大用勉,會李輔國寵任,意欲勉降禮於己。勉不為之屈,竟為所抑,出歷汾州、虢州刺史,改京兆尹、檢校右庶子、兼御史中丞、都畿觀察使。尋兼河南尹,明年罷尹,以中丞歸西臺,又除江西觀察使。賊帥陳莊連陷江西州縣,偏將呂太一、武
 日升相繼背叛,勉與諸道力戰,悉攻平之。部人有父病,以蠱道為木偶人,署勉名位,瘞於其隴,或以告,曰:「為父禳災,亦可矜也。」舍之。大歷二年,來朝,拜京兆尹、兼御史大夫,政尚簡肅。宦官魚朝恩為觀軍容使,仍知國子監事,恃寵含威,天憲在舌。前尹黎幹寫心候事,動必求媚,每朝恩入監,傾府人吏具數百人之餼以待之。及勉蒞職旬月,朝恩入監,府吏先期有請,勉曰:「軍容使判國子監事,勉候太學,軍容宜厚具主禮。勉忝京尹,軍容倘惠
 顧府廷,豈敢不具蔬饌。」朝恩聞而銜之,因不復至太學,勉亦尋受代。



 四年,除廣州刺史,兼嶺南節度觀察使。番禺賊帥馮崇道、桂州叛將硃濟時等阻洞為亂,前後累歲,陷沒十餘州。勉至,遣將李觀與容州刺史王翃人並力招討,悉斬之,五嶺平。前後西域舶泛海至者歲才四五,勉性廉潔,舶來都不檢閱,故末年至者四十餘。在官累年,器用車服無增飾。及代歸,至石門停舟,悉搜家人所貯南貨犀象諸物,投之江中,耆老以為可繼前朝宋璟、
 盧奐、李朝隱之徒。人吏詣闕請立碑,代宗許之。十年,拜工部尚書。及滑亳永平軍節度令狐彰卒,遺表舉勉自代,因除之。在鎮八年,以舊德清重,不嚴而理,東諸侯雖暴驁者,亦宗敬之。



 十一年,汴宋留後田神玉卒,詔加勉汴州刺史、汴宋節度使。未行,汴州將李靈曜阻兵,北結田承嗣,承嗣使侄悅將銳兵戍之。詔勉與李忠臣、馬燧等攻討,大破之,悅僅以身免。靈曜北走,勉騎將杜如江擒之以獻,代宗褒賞甚厚。既而李忠臣代鎮汴州,而勉
 仍舊鎮。忠臣遇下貪虐,明年為麾下所逐,詔復加勉汴宋節度使,移理汴州,餘並如故。德宗嗣位,加檢校吏部尚書,尋加平章事。建中元年,檢校左僕射,充河南汴宋滑亳河陽等道都統,餘如故。四年,李希烈反,以他盜為名,悉眾來寇汴州。勉城守累月,救援莫至,謂其將曰:「希烈兇逆殘酷,若與較力,必多殺無辜,吾不忍也。」遂潛師潰圍,南奔宋州。詔以司徒平章事征。既至朝廷,素服請罪,優詔復其位,勉引過備位而已。無何,盧杞自新州員
 外司馬除澧州刺史,給事中袁高以杞邪佞蠹政,貶未塞責,停詔執表,遂授澧州別駕。他日,上謂勉曰:「眾人皆言盧杞奸邪,朕何不知!卿知其狀乎?」對曰:「天下皆知其奸邪,獨陛下不知,所以為奸邪也。」時人多其正直,然自是見疏。累表辭位,遂罷知政事,加太了太保。貞元四年卒,年七十二,上頗愍悼之,冊贈太傅,賻物有差,喪葬官給。



 勉坦率素淡,好古尚奇,清廉簡易,為宗臣之表。善鼓琴,好屬詩,妙知音律,能自制琴,又有巧思。,及在相位向
 二十年,祿俸皆遺親黨,身沒而無私積。其在大官,禮賢下士,終始盡心。以名士李巡、張參為判官,卒於幕,三歲之內,每遇宴飲,必設虛位於筵次,陳膳執酹,辭色心妻惻,論者美之。或曰:「勉失守梁城,亦可貶也。」議者曰:「不然。當賊烈之始亂,其慓悍陰禍,兇焰不可當,天方厚其毒而降之罰。況勉應變非長,援軍莫至,又其時關輔已俶擾矣,人心已動搖矣。以文吏之才,當虎狼之隊,其全師奔宋,非量力之恥也。與其坐受喪敗,不猶愈乎!」



 李皋,字子蘭,曹王明玄孫,嗣王戢之子。少補左司禦率府兵曹參軍。天寶十一載嗣封授都水使者,三遷至秘書少監,皆同正。多智數,善因事以自便。奉太妃鄭氏以孝聞。



 上元初,京師旱,米斗直數千,死者甚多。皋度俸不足養,亟請外官,不允,乃故抵微法,貶溫州長史。無幾,攝行州事。歲儉,州有官粟數十萬斛,皋欲行賑救,掾吏叩頭乞候上旨,皋曰:「夫人日不再食,當死,安暇稟命!若殺我一身,活數千人命,利莫大焉。」於是開倉盡散之,以擅
 貸之罪,飛章自劾。天子聞而嘉之,答以優詔,就加少府監。皋行縣,見一媼垂白而泣,哀而問之,對曰:「李氏之婦,有二子:鈞、鍔,宦游二十年不歸,貧無以自給。」時鈞為殿中侍御史,鍔為京兆府法曹,俱以文藝登科,名重於時。皋曰:「『入則孝,出則悌,行有餘力,然後可以學文。』若二子者,豈可備於列位!」由是舉奏,並除名勿齒。改處州別駕,行州事,以良政聞。徵至京,未召見,因上書言理道,拜衡州刺史。坐小法,貶潮州刺史。時楊炎謫官道州,知皋事
 直,及為相,復拜衡州。初,皋為御史覆訊,懼貽太妃憂,竟出則素服,入則公服,言貌如平常,太妃竟不知。及為潮州,詭詞謂遷,至是復位,方泣以白,且言非疾不敢有聞。



 建中元年,遷湖南觀察使。前使辛京杲貪殘,有將王國良鎮邵州武岡縣,豪富,京杲以死罪加之。國良危懼,因人所苦,遂散財聚眾據縣以叛,諸道同討,聯歲不能下。皋授命日,乃曰:「驅疲甿,誅反側,非所以奉聖朝事。」遣使遺國良書曰;「觀將軍非敢大逆,蓋遭讒嫉,救誤死而已。
 將軍遇我,何不速降?我與將軍同為辛京杲所構,我已蒙聖朝昭雪,使我何心持刃殺將軍耶!將軍以為不然,我以陣術破將軍陣,以攻法屠將軍城,非將軍所度也。」國良捧書,且憂且喜,遣使請降,亦未必決。皋即日赴縣受降,中道有候騎馳告曰:「國良軍中有變,言降是詐也。」皋曰:「非爾輩所知。」遂留麾下兵,單騎假稱使者,徑入國良壘中。國良召使者入,皋遂大叫軍中曰:「有人識曹王否?只我是。國良何不速降?」一軍愕眙不敢動。適有識者
 走至,傳呼曰:「是」。國良匍匐叩頭請罪。皋執手約為兄弟,盡焚攻守之備,散倉庫,給兵士,令復農桑。有詔赦國良罪,賜名惟新。



 建中二年,丁母艱,奉喪至江陵。會梁崇義反,乃授起復左衛大將軍,復還湖南,尋加散騎常侍李希烈反,遷江西道節度使、洪州刺史、兼御史大夫。至州,集將吏而令曰:「嘗有功未申者,別為行;有策謀及器能堪佐軍者,別為行。」有裨將伊慎、李伯潛、劉旻皆自占,皋察其詞氣,驗其有功,悉補大將。擢王鍔委之中軍,以馬
 彞、許孟容為賓佐。繕甲兵,具戰艦,將軍二萬餘。初,伊慎將江西兵從李希烈平襄州,及反,懼皋任之,乃陰遣遺之鎖甲,又詐為慎書往復,置遺於境。上聞,即遣中使斬慎,皋表請舍令自效。會與賊夾江為陣,中使又至,皋乃勉令以功自贖,賜之以所乘馬及器甲,令將鋒而先,皋率軍繼之,責其有功,果大破賊,斬首數百級,慎方得免罪。賊樹堡柵於蔡山,皋度峻險不可攻,乃聲言西取蘄州,理戰艦,分兵傍南涯,與舟師溯江而上。賊以老弱守
 柵,引軍循江隨戰艦,南北與皋兵相直。去蔡山三百餘里,皋令步兵登舟,順流東下,不日拔蔡山。賊還救,間一日方至,大破之,因進拔蘄州,降其將李良,又取黃州,斬首千餘,兵益振,舒王為元帥,加皋前軍兵馬使。



 德宗居奉天,淮南節度陳少游強取鹽鐵錢,其使包佶以財幣溯江,次於蘄口。時希烈已屠汴州,又遣驍將杜少誠將步騎萬餘來寇蘄、黃,將絕江道。皋遣伊慎將七千眾御之,遇於永安戍。慎列三柵,相去才四里,列鼓角中柵。少
 誠至,分兵圍之,部隊未嚴,聲鼓而三柵齊出奮擊,不為行陣,賊亂,少誠敗走,斬首萬級,封尸為京觀。以功加銀青光祿大夫,進封五百戶。上至梁州,進獻繼至。皋以上蒙塵於外,不敢居城府,乃於西塞山上游大洲屯軍,從近縣為軍市,商貨畢至。加工部尚書。駕還京師,又遣伊慎、王鍔將兵圍安州,州城阻溳水為固,攻之累日不下。希烈遣甥劉戒虛將步八千來援。皋命李伯潛分師迎擊於應山,獲戒虛及大將二、裨將二十,斬首千餘。面
 縛戒虛等之城下,乃使人說之,賊曰:「得大將及賓佐一二人為信,當降。」皋乃使王鍔、馬彞繩城而入,城中大呼,乃出降。希烈又遣兵援隨州,皋令伊慎擊於厲鄉,大破之,復平靜、白雁等關。希烈懼,乃戢兵。貞元初,拜江陵尹、荊南節度等使,江漢倚皋為固。未幾,李思登以隨州降。凡下州四、縣十七,大小十餘陣,未嘗敗衄。淮西既平,請護喪祔東都,上遣中使吊,贈父右僕射,母曹國太妃。葬畢來朝,詔還鎮,出東都以拜墓,觀者榮之。



 先,江陵東北
 有廢田傍漢古堤二處,每夏則溢,皋始命塞之,廣田五千頃,畝得一鐘。規江南廢洲為廬舍,架江為二橋,流人自占二千餘戶。自荊至樂鄉凡二百里,旅舍鄉聚凡十數,大者皆數百家。楚俗佻薄,不穿井,飲陂澤,皋始命合錢開井以便人。



 初平希烈,吳少誠殺陳仙奇,上以襄、鄧要厄,三年,除襄州刺史、山南東道節度等使,割汝、隨隸焉。練兵積糧,市回鶻馬益騎兵,堂大畋以教士,少誠憚之。性勤儉,知人疾苦,設監司,能參聽下,持將吏短長,賞
 罰必信。所至常平物價,貴則出賣之,給將吏廩俸,豪家不得擅其利。常運心巧思為戰艦,挾二輪蹈之,翔風鼓浪,疾若掛帆席,所造省易而久固。又造欹器,進入內中。每遺人物,常自秤量。署之官匹帛皆印之,絕吏之私。



 初,扶風馬彞未知名,皋始闢之,卒以正直稱。漢陽王張柬之有林園在州西,公府多假之游宴,皋將買之,彞斂衽而言曰:「張漢陽有中興功,今遺業當百代保之,王縱欲之,奈何令其子孫自鬻焉!」皋謝曰:「主吏失詞,為足下羞;微足
 下,安得聞此言!」以改過遷善、知人任下為己任,故賓從將佐多至大官。貞元八年三月,暴卒於位,年六十,廢朝三日,贈右僕射,賻吊有差,謚曰成。子象古、道古、復古。



 象古自衡州刺史為安南都護。元和十四年,為楊清所殺,妻子支黨無噍類焉。楊清者,代為南方酋豪,屬象古貪縱,人心不附,又惡清之強,自驩州刺史召為牙門將,鬱鬱不快。無何,邕管黃家賊叛,詔象古發兵數道共討之,象古命清領兵三千赴焉。清與其子志烈及所親杜士
 交潛謀回戈,夜襲安南,數日城陷,象古故及於害。朝廷命唐州刺史桂仲武為都護,且招諭之。赦清,以為瓊州刺史。仲武至境,清不納,復約束部署,刑戮憯虐,人無聊生。仲武使人諭其酋豪,數月間,歸附繼至,約兵七千餘人,收其城,斬清及其子志貞,籍沒其家。志烈與士交敗,保於長州之鑿溪,尋以所部兵來降。



 道古登進士第,遷司門員外郎。便佞巧宦,早升朝籍,常以酒肴棋博游公卿門,角賭之際,每偽為不勝而厚償之,故當時有虛名,
 而嗜利者悉與之狎。歷處、隨、唐、睦四州刺史,由黔中觀察為鄂、岳、沔、蘄、安、黃團練觀察使,時元和十一年也。初,以柳公綽在鎮無功,議將代之,裴度言:「道古嗣曹王皋之子,皋嘗以江漢兵遏希烈之亂,威惠至今在人,復用其子必能繼美。」憲宗然之,故有此授。及赴鎮,倍道而行,以數騎徑入安州城。時公綽殊未意道古至,惶駭而出,家財多為所奪。十二年,道古攻申州,克其羅城,乃進圍逼其中城。城中守卒夜帥婦人登城而呼,懸門竊發,分
 出其眾,道古之眾驚亂,為虜所殺。初,李聽守安州,未嘗退衄。及道古至,誣奏聽,移去之,乃自帥兵出穆陵。士卒驕惰,賜給多闕,其度支供軍錢,道古半以奉權倖,半以沒己,人皆怨怒,不肯力戰。賊亦易道古,以羸兵抵之,故道古前後再攻破申州外城而不能拔。至李愬入蔡州,乃降。



 元和十三年,入為宗正卿。道古在鄂州日,以貪暴聞,懼終得罪,乃薦山人柳泌以媚於上。後又為左金吾衛將軍。憲宗季年頗信方士,銳於服食,詔天下搜訪奇
 士。宰相皇甫鎛方諛媚固寵,道古言柳泌有道術,鎛得進之,待詔翰林。憲宗服餌過當,暴成狂躁之疾,以至棄代。穆宗在東宮,扼腕於其事,及居喪,皆竄逐誅之。鎛既貶責,授道古循州司馬,終以服丹藥,歐血而卒。



 史臣曰:李勉、李皋,稟性端莊,處身廉潔,臨民蒞事,動有美聲,可謂宗臣之英也。若夫治軍旅,御寇戎,謀必臧,戰必勝,則又勉不及皋遠矣。道古便佞,奸以事君,何父子之不相類也。



 贊曰:我宗之英,曰皋與勉,才雖不同,道豈相遠。



\end{pinyinscope}