\article{卷一百二十一}

\begin{pinyinscope}

 ○嚴武郭英乂崔寧弟寬從孫蠡蠡子蕘從孫黯嚴震嚴礪



 嚴武,中書侍郎挺之子也。神氣雋爽,敏於聞見。幼有成人之風,讀書不究精義,涉獵而已。弱冠以門廕策名,隴
 右節度使哥舒翰奏充判官,遷侍御史。至德初,肅宗興師靖難,大收才傑,武杖節赴行在。宰相房琯以武名臣之子,素重之,及是,首薦才略可稱,累遷給事中。既收長安,以武為京兆少尹、兼御史中丞,時年三十二。以史思明阻兵不之官,優游京師,頗自矜大。出為綿州刺史,遷劍南東川節度使;入為太子賓客、兼御史中丞。



 上皇誥以劍兩川合為一道,拜武成都尹、兼御史大夫,充劍南節度使;入為太子賓客,遷京兆尹、兼御史大夫。二聖山
 陵,以武為橋道使。無何,罷兼御史大夫,改吏部侍郎,尋遷黃門侍郎。與宰臣元載深相結托,冀其引在同列。事未行,求為方面,復拜成都尹,充劍南節度等使。廣德二年,破吐蕃七萬餘眾,拔當狗城。十月,取鹽川城,加檢校吏部尚書,封鄭國公。



 前後在蜀累年,肆志逞欲,恣行猛政。梓州刺史章彞初為武判官,及是小不副意,赴成都杖殺之,由是威震一方。蜀土頗饒珍產,武窮極奢靡,賞賜無度,或由一言賞至百萬。蜀方閭里以徵斂殆至匱
 竭,然蕃虜亦不敢犯境。而性本狂蕩,視事多率胸臆,雖慈母言不之顧。初為劍南節度使,舊相房琯出為管內刺史,琯於武有薦導之恩,武驕倨,見琯略無朝禮,甚為時議所貶。永泰元年四月,以疾終,時年四十。



 郭英乂,先朝隴右節度使、左羽林軍將軍知運之季子也。少以父業,習知武藝,策名河、隴間,以軍功累遷諸衛員外將軍。至德初,肅宗興師朔野,英乂以將門子特見任用,遷隴右節度使、兼御史中丞。既收二京,徵還闕下,
 掌禁兵。遷羽林軍大將軍,加特進。以家艱去職。



 朝廷方討史思明,選任將帥,乃起英乂為陜州刺史,充陜西節度、潼關防禦等使,尋加御史大夫,兼神策軍節度。代宗即位,加檢校戶部尚書、兼御史大夫。元帥雍王自陜統諸軍討賊洛陽,留英乂在陜為後殿。東都平,以英乂權為東都留守。既至東都,不能禁暴,縱麾下兵與朔方、回紇之眾大掠都城,延及鄭、汝等州,比屋蕩盡。廣德元年,策勛加實封二百戶,徵拜尚書右僕射,封定襄郡王。恃
 富而驕,於京城創起甲第,窮極奢靡。與宰臣元載交結,以久其權。



 會劍南節度使嚴武卒,載以英乂代之,兼成都尹,充劍南節度使。既至成都,肆行不軌,無所忌憚。玄宗幸蜀時舊宮,置為道士觀,內有玄宗鑄金真容及乘輿侍衛圖畫。先是,節度使每至,皆先拜而後視事。英乂以觀地形勝,乃入居之,其真容圖畫,悉遭毀壞。見者無不憤怒,以軍政苛酷,無敢發言。又頗恣狂蕩,聚女人騎驢擊球,制鈿驢鞍及諸服用,皆侈靡裝飾,日費數萬,以
 為笑樂。未嘗問百姓間事,人頗怨之。又以西山兵馬使崔旰得眾心,屢抑之。旰因蜀人之怨,自西山率麾下五千餘眾襲成都,英乂出軍拒之,其眾皆叛,反攻英乂。英乂奔於簡州,普州刺史韓澄斬英乂首以送旰,並屠其妻子焉。



 崔寧,衛州人,本名旰。雖儒家子,喜縱橫之術。衛州刺史茹璋授旰符離令,既罷,久不調,遂客游劍南,從軍為步卒,事鮮於仲通。又隨李宓討雲南,宓戰敗,旰歸成都。行
 軍司馬崔論見旰,悅其狀貌,又以其宗姓厚遇,薦為衙將。歷事崔圓、裴冕。冕遭流謗,朝廷將遣使推按,旰部下截耳稱冤,中使奏之。旰亦赴京師,授司戈,歷司階、折沖郎將軍等官。



 寶應初,蜀中亂,山賊擁絕縣道,代宗憂之。嚴武薦旰為利州刺史,既至,山賊遁散,由是知名。嚴武為劍南節度,赴鎮過利州,心欲闢旰為部將,以利非屬部,旰難輒去,俾旰籌之。旰曰:「節度使張獻誠見忌,且又好利,誠能重賂之,旰可以從大夫矣。」武至劍南,遺獻誠
 奇錦珍貝,價兼百金,獻誠大悅。武乃遺獻誠書求旰,獻誠然之,令旰移疾去郡。旰乃之劍南,武奏為漢州刺史。久之,吐蕃與諸雜羌戎寇陷西山柘、靜等州,詔嚴武收復。武遣旰統兵西山,旰善撫士卒,皆願致死命。始次賊城,周圍皆石礫,攻具無所設。唯東南隅環丈之地,壤土可穴,諜知之以告。旰晝夜穿地道攻之,再宿而拔其城。因拓地數百里,下城寨數四。番眾相語曰:「崔旰,神兵也。」將更前進,以糧盡還師。武大悅,裝七寶輿迎旰入成都,
 以誇士眾,賞齎過厚。



 永泰元年五月,嚴武卒,杜濟為西川行軍司馬,權知軍府事。時郭英乾為都知兵馬使,郭嘉琳為都虞候,皆請英乾兄英乂為節度使。旰時為西山都知兵馬使,與軍眾共請大將王崇俊為節度使。二奏俱至京師,會朝廷已除英乂,旰使因見英乂陳其事。英乂至成都,數日,誣殺王崇俊,又召旰還成都。英乂減將健糧賜,人心怨怒。旰在西山聞之,大恐,乃托備吐蕃,未赴成都。英乂怒,出兵聲言助旰討吐蕃,其實襲之也。
 旰家在漢州,英乂遷之成都,通其妾媵。旰知之,轉入深山。英乂自率師攻旰,值天大寒,雪深數尺,英乂士馬凍死者數百人,眾心離叛。旰遂出兵拒敵,英乂與之接戰,英乂軍大敗而還,收餘兵才千人,歸成都,將卒因多逃散。



 初,天寶中,劍南節度使鮮於仲通嘗建一使院,院宇甚華麗。及玄宗幸蜀,嘗居之,因為道觀,兼寫玄宗真容,置之正室。英乂因入觀行香,悅其竹樹,遂奏請以仲通舊院為軍營,乃移去真容自居之。旰聞之,謂將士曰:「英
 乂反矣!不然,何得除毀玄宗真容而自居之?」乃率兵攻成都。英乂出兵於城西門,令柏茂琳為前軍,郭英乾為左軍,郭嘉琳為後軍,與旰戰。茂琳等軍累敗,軍人多投旰。旰令降將統兵與英乂轉戰,大敗之。兵至子城,英乂單騎奔簡州,為普州刺史韓澄所殺。時邛、劍所在起兵相攻,劍南大亂。



 永泰二年二月,乃以黃門侍郎平章事杜鴻漸兼成都尹、山南西道劍南東川西川邛南等道副元帥、劍南西川節度使。鴻漸出駱谷,有謀者曰:「相公
 駐車閬州,遙制劍南,數移牒述英乂過失,言旰有方略;旰腹心攝諸州刺史者皆奏正之,令旰及將校不疑怨。然後與東川節度使張獻誠及諸賊帥合議,數出兵攻旰。既數道連兵,未經一年,兵勢減耗,旰窮,必束身歸朝。此上策也。」鴻漸畏懦,計疑未決。會旰使至,卑辭厚禮,送繒錦數千匹。鴻漸貪其利,遂至成都,日與判官杜亞、楊炎將吏等高會縱觀,軍州政事悉委旰,乃連表聞薦。



 先時,張獻誠數與旰戰,獻誠屢敗,旌節皆為旰所奪。朝廷
 因鴻漸之請,加成都尹,兼西山防禦使、西川節度行軍司馬,仍賜名曰寧。大歷二年,鴻漸歸朝,遂授寧西川節度使。恃地險人富,乃厚斂財貨,結權貴,令弟寬留京師。元載及諸子有所欲,寬恣與之,故寬驟歷御史知雜事、御史中丞。寬兄審亦任郎中、諫議大夫、給事中。寧在蜀十餘年,地險兵強,肆侈窮欲,將吏妻妾,多為所淫污,朝廷患之而不能詰。累加尚書左僕射。



 大歷十四年入朝,遷司空、平章事,兼山陵使,尋代喬琳為御史大夫、平章
 事。寧以為選擇御史當出大夫,不謀及宰相,乃奏請以李衡、於結等數人為御史。楊炎大怒,其狀遂寢。炎又數讒毀劉晏,寧又求解之。寧既厚結元載已久,楊炎又出自載門,寧初附炎,炎因此大怒。



 其年十月,南蠻大下,與吐蕃三道合進。一出茂州,過文川及灌口。一出扶、文,過方維、白壩。一出黎壩、雅,過邛、郲。戎酋誡其眾曰:「吾要蜀川為東府,凡伎巧之工皆送邏娑,平歲賦一縑而已。」是蠻之入,連陷郡邑,士庶奔亡山谷。屬寧在朝,軍中無帥,
 德宗促寧還鎮。炎懼寧怨己,入蜀難制,謂德宗曰:「蜀川天下奧壤,自寧擅置其中,朝廷失其外府十四年矣。今寧來朝,尚有全師守蜀。貨利之厚,適中奉給,貢賦所入,與無地同。始寧與諸將等夷,獨因叛亂得位,不敢自有,以恩柔煦育,威令不行。今雖歸之,必無功,是徒遣也;若有功,義不可奪。則西川之奧,敗固失之,勝亦非國家所有。陛下熟察。」帝曰:「卿策何從?」炎曰:「請無歸寧。今硃泚所部範陽勁兵,戍在近甸,促令與禁兵雜往,舉無不捷。因
 是役得置親兵內其腹中,蜀將必不敢動。然後換授他帥,以收其權,得千里肥饒之地,是因小禍受大福也。」帝曰:「善」,即止寧不行。乃發禁兵四千、範陽兵五千,赴援東川。出軍自江油趣白壩,與山南兵合擊,蠻兵敗走。範陽軍又擊破於七盤,遂拔新城,戎、蠻大敗。凡斬馘六千,生擒六百,傷者殆半,饑寒隕於崖谷者八九萬。



 寧遂罷西川節度使,制授檢校司空、同中書門下平章事、御史大夫、京畿觀察使,兼靈州大都督、單于鎮北大都護、朔方
 節度等使,兼鄜坊丹延都團練觀察使。托以重臣綏靖北邊,但令居鄜州。雖以寧為節度,每道皆置留後,自得奏事,炎悉諷令伺寧過犯。杜希全為靈州,王翃為振武,李建徽為鄜州,及戴休顏、杜從政、呂希倩等,皆炎署置也。寧巡邊至夏州,刺史呂希倩與寧同力招撫黨項,歸降者甚多。炎惡之,因奏希倩撫綏之功,才堪委任。召歸朝,除右僕射知省事,以神武將軍時常春代之。



 硃泚之亂,上卒迫行幸,百僚諸王鮮有知者。寧後數日自賊中
 來,上初喜甚。寧私謂所親曰:「聖上聰明英邁,從善如轉規,但為盧杞所惑至此爾。」杞聞之,潛與王翃圖議陷之。初,涇原兵作亂之夕,寧與翃及御史大夫於頎俱出延平門而西,數下馬便液,每下輒良久。翃等促之,不敢前。又懼賊兵追及,翃乃大聲而言曰:「已至此,不必顧望。」至奉天,翃具以事聞。會硃泚行反間,偽除柳渾宰相,署寧中書令。寧朔方掌書記康湛時為盩厔尉,翃逼湛作寧遺硃泚書,使寧無以自辯,翃遂獻之。杞因誣奏曰:「崔寧
 初無葵藿向日之心,聞於城中與硃泚堅為盟約,所以後於百闢。今事果驗。使兇渠外逼,奸臣內謀,則大事去矣。」因俯伏歔欷曰:「臣備位宰相,危不能持,顛不能扶,宜當萬死,伏待斧鉞。。」上命左右扶起之。既還,俄有中人引寧於幕後,二力士自後縊殺之,時年六十一。初,將誅寧,召至朝堂,雲令江淮宣慰。尋命翰林學士陸贄草誅寧制;贄求寧與泚書,將以狀生之。復亂言云,其書已失。寧既得罪,籍沒其家,中外稱其冤,乃赦其家,歸其資產。貞
 元十二年六月,寧故將、夏、綏、銀節度使韓潭奏請以新加禮部尚書恩制以雪寧之罪。詔從之,任其家收葬。



 初,寧入朝,留弟寬守成都。瀘州楊子琳乘間以精騎數千突入成都,據城守之。寬屢戰力屈,子琳威聲頗盛。寧妾任氏魁偉果干,乃出其家財十萬募勇士,信宿間得千人,設隊伍將校,手自麾兵,以逼子琳。子琳懼,城內糧盡,乃拔城自潰。子琳素有妖術,其夕致大雨,引舟至庭除,登之而遁。



 寧季弟密,密子繪,父子皆以文雅稱,歷使府
 從事。繪生四子:蠡、黯、確、顏,皆以進士擢第。



 蠡,字越卿,元和五年擢第,累闢使府。寶歷中,入朝監察御史。大和初,為侍御史,三遷戶部郎中,出為汝州刺史。開成初,以司勛郎中徵,尋以本官知制誥。明年,正拜舍人。三年,權知禮部貢舉。四年,拜禮部侍郎,轉戶部。上疏論國忌日設僧齋,百官行香,事無經據。詔曰:「朕以郊廟之禮,嚴奉祖宗,備物盡誠,庶幾昭格。恭惟忌日之感,所謂終身之憂。而近代以來,歸依釋、老,征二教以設食,會百闢以行香。
 將以有助聖靈,冥資福祚。有異皇王之術,頗乖教義之宗。昨得崔蠡奏論,遂遣討尋本末,禮文令式,曾不該明,習俗因循,雅當整革。其兩京、天下州府,以國忌日為寺觀設齋焚香,從今已後,並宜停罷。」蠡尋為華州刺史、鎮國軍等使,再歷方鎮。子蕘。



 蕘,字野夫。大中二年,擢進士第,累官至尚書郎、知制誥。正拜中書舍人、戶部侍郎。乾符中,自尚書右丞遷吏部侍郎。蕘美文詞,善談論,而馭事簡率,銓管非所長。出為陜州觀察使,以器韻自高,不
 屑細故,權移僕下。時河南寇盜蜂起,王仙芝亂漢南,朝綱不振,而蕘自恃清貴,不恤人之疾苦。百姓訴旱,蕘指庭樹曰:「此尚有葉,何旱之有?」乃笞之,吏民結怨。既而為軍人所逐,饑渴甚,投民舍求水,民以溺飲之。初為軍人所俘,翦其髭發,拜而獲免。以失守貶端州司馬,復入為左散騎常侍,卒。



 子居敬、居儉。居敬終尚書郎,居儉中興終戶部尚書。



 黯,字直卿,大和二年,進士擢第。開成初,為青州從事。入為監察御史,奏郊廟祭器不虔,請敕有司。
 文宗謂宰臣曰:「宗廟之事,朕合親奉其禮,但以千乘萬騎,動費國用,每有司行事之日,被衣冠坐以俟旦。比聞主者不虔,祭器勞敝,非事神蠲潔之義。卿宜嚴敕有司,道吾此意。」黯具條奏以聞。尋遷員外郎。會昌中,為諫議大夫。



 確,字岳卿,顏,字希卿,位皆至尚書郎。



 嚴震,字遐聞,梓州鹽亭人。世為田家,以財雄於鄉里。至德、乾元已後,震屢出家財以助邊軍,授州長史、王府諮議參軍。東川節度判官韋收薦震才用於節度使嚴武,
 遂授合州長史。及嚴武移西川,署為押衙,改恆王府司馬。嚴武以宗姓之故,軍府之事多以委之,又歷試衛尉、太常少卿。嚴武卒,乃罷歸。東川節度使又奏為渝州刺史,以疾免。山南西道節度使又奏為鳳州刺史,加侍御史,丁母憂罷。起復本官,仍充興、鳳兩州團練使,累加開府儀同三司、兼御史中丞。為政清嚴,興利除害,遠近稱美。建中初,司勛郎中韋楨為山、劍黜陟使,薦震理行為山南第一,特賜上下考,封鄖國公。在鳳州十四年,能政
 不渝。



 建中三年,代賈耽為梁州刺史、兼御史大夫、山南西道節度觀察等使。及硃泚竊據京城,李懷光頓軍咸陽,又與之連結。泚令腹心穆庭光、宋瑗等齎白書誘震同叛,震集眾斬庭光等。時李懷光連賊,德宗欲移幸山南。震既聞順動,遣吏馳表往奉天迎駕,仍令大將張用誠領兵五千至盩厔已東迎護,上聞之喜。既而用誠為賊所誘,欲謀背逆,朝廷憂之。會震又遣牙將馬勛奉表迎候,上臨軒召勛與之語,勛對曰:「臣請計日至山南取
 節度使符召用誠,即不受召,臣當斬其首以復。」上喜曰:「卿何日當至?」勛克日時而奏,帝勉勞之。勛既得震符,乃請壯丁五人偕行。既出駱谷,用誠以勛未知其謀,乃以數百騎迎勛,勛與俱之傳舍,用誠左右森然。勛先聚草發火於驛外,軍士爭附火。勛乃從容出懷中符示之曰:「大夫召君。」用誠惶懼起走,壯士自背束手而擒之。不虞用誠子居後,引刀斫勛,勛左右遽承其臂,刀下不甚,微傷勛首。遂格殺其子,而僕用誠於地。壯士跨其腹,以刃
 擬其喉曰:「出聲即死!」勛即其營,軍士已被甲執兵矣。勛大言曰:「汝等父母妻子皆在梁州,一朝棄之,欲從用誠反逆,有何利也?但滅汝族耳!大夫使我取張用誠,不問汝輩,欲何為乎?」眾皆讋服。於是縛用誠送州,震杖殺之,拔其副將,使率其眾迎駕。勛以藥封首馳赴行在,愆約半日,上頗憂之,及勛至,上喜動顏色。翌日,車駕發奉天,及入駱谷,李懷光遣數百騎來襲,賴山南兵擊之而退,輿駕無警急之患。尋加震檢校戶部尚書,賜實封二百
 戶。



 三月,德宗至梁州。山南地貧,糧食難給,宰臣議請幸成都府。震奏曰:「山南地接京畿,李晟方圖收復,藉六軍聲援。如幸西川,則晟未見收復之期也。幸陛下徐思其宜。」議未決,李晟表至,請車駕駐蹕梁、洋,以圖收復,群議乃止。梁、漢之間,刀耕火耨,民以採穭為事,雖節察十五郡,而賦額不敵中原三數縣。自安、史之後,多為山賊剽掠,戶口流散大半。洎六師駐蹕,震設法勸課,鳩聚財賦,以給行在,民不至煩,供億無闕。其年六月,收復京城,車
 駕將還京師,進位檢校尚書左僕射。詔曰:「朕遭罹寇難,播越梁、岷,蒸庶煩於供億,武旅勤於捍衛。凡百執事,各奉厥司,眷於是邦,復我興運,宜加崇大,以示將來。宜改梁州為興元府,官名品制,同京兆、河南府;鄭縣升為赤,諸縣升為畿。見任州縣官,考滿日放選,百姓給復一年。洋州宜升為望,見任州縣官,考滿減兩選。山南西道將士,並與甄敘。」以震為興元尹,賜實封二百戶。



 貞元元年十一月,德宗親祀昊天上帝於南郊,震入朝陪祭。十一
 年二月,加同平章事。貞元十五年六月卒,時年七十六,廢朝三日,冊贈太保,賻布帛米粟有差。及喪將至,令百官以次赴宅吊哭。



 嚴礪,震之宗人也。性輕躁,多奸謀,以便佞在軍,歷職至山南東道節度都虞候、興州刺史、兼監察御史。貞元十五年,嚴震卒,以礪權留府事,兼遺表薦礪才堪委任。七月,超授興元尹,兼御史大夫,山南西道節度、支度營田、觀察使。詔下,諫官御史以為除拜不當。是日,諫議、給事、
 補闕、拾遺並歸門下省共議:礪資歷甚淺,人望素輕,遽領節旄,恐非允當。既兼雜話,發論喧然。拾遺李繁獨奏云:「昨除拜嚴礪,眾以為不當,諫議大夫苗拯云:『已三度表論,未見聽允。』給事中許孟容曰:『誠如此,不曠職矣。』」又云:「李元素、陳京、王舒並見拯及孟容言議。」上遣三司使詰之。拯狀云:「實於眾中言曾論奏,不言三度。」繁證之不已。孟容等又云:「拯實言兩度。」拯請依眾狀。翌日,貶拯萬州刺史,李繁播州參軍,並同正。礪在位貪殘,士民不堪
 其苦。素惡鳳州刺史馬勛,誣奏貶賀州司戶。縱情肆志,皆此類也。



 元和四年三月卒。卒後,御史元稹奉使兩川按察,糾劾礪在任日贓罪數十萬。詔徵其贓,以死,恕其罪。



 史臣曰:爵人於朝,與眾共之;刑人於市,與眾棄之。縊崔寧,除嚴礪,時君之政可知矣,輔相之才可見矣!武不稟父風,有違母誨,凡為人子者,得不戒哉!雖有周、孔之才,不足稱也,況狂夫乎!英乂失政,其死也宜哉。嚴震立功,
 其道也顯矣。



 贊曰:英乂失政,崔
 寧發身。武
 為士子,震作純臣。



\end{pinyinscope}