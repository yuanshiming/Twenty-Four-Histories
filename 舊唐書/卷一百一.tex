\article{卷一百一}

\begin{pinyinscope}

 ○劉幽求鐘紹京郭元振張說子均垍陳希烈附



 劉幽求,冀州武強人也。聖歷年,應制舉,拜閬中尉,刺史不禮焉,乃棄官而歸。久之,授朝邑尉。初,桓彥範、敬暉等
 雖誅張易之兄弟,竟不殺武三思。幽求謂桓、敬曰:「三思尚存,公輩終無葬地。若不早圖,恐噬臍無及。」桓、敬等不從其言,後果為三思誣構,死於嶺外。



 及韋庶人將行篡逆,幽求與玄宗潛謀誅之,乃與苑總監鐘紹京、長上果毅麻嗣宗及太平公主之子薛崇暕等夜從入禁中討平之。是夜所下制敕百餘道,皆出於幽求。以功擢拜中書舍人,令參知機務,賜爵中山縣男,食實封二百戶。翌日,又授其二子五品官,祖、父俱追贈刺史。



 睿宗即位,加
 銀青光祿大夫,行尚書右丞,仍舊知政事,進封徐國公,加實封通前五百戶,賜物千段、奴婢二十人、宅一區、地十頃、馬四匹,加以金銀雜器。景雲二年,遷戶部尚書,罷知政事。月餘,轉吏部尚書,擢拜侍中,降璽書曰:「頃者,王室不造,中宗厭代,外戚專政,奸臣擅國,將傾社稷,幾遷龜鼎,朕躬與王公,皆將及於禍難。卿見危思奮,在變能通,翊贊儲君,協和義士,殄殲元惡,放殛兇徒。我國家之復存,醫茲是賴,厥庸甚茂,朕用嘉焉。故委卿以衡軸,胙
 卿以茅土,然征賦未廣,寵錫猶輕。昔西漢行封,更擇多戶;東京定賞,復增大邑。故加賜卿實封二百戶,兼舊七百戶。使夫高岸為谷,長河如帶,子子孫孫,傳國無絕。又以卿忘軀徇難,宜有恩榮,故特免卿十死罪,並書諸金鐵,俾傳於後。卿其保茲功業,永作國楨,可不美歟!」



 先天元年,拜尚書右僕射、同中書門下三品,監修國史。幽求初自謂功在朝臣之右,而志求左僕射,兼領中書令。俄而竇懷貞為左僕射,崔湜為中書令,幽求心甚不平,形
 於言色。湜又托附太平公主,將謀逆亂。幽求乃與右羽林將軍張暐請以羽林兵誅之,乃令暐密奏玄宗曰:「宰相中有崔湜、崔羲,俱是太平公主進用,見作方計,其事不輕。殿下若不早謀,必成大患。一朝事出意外,太上皇何以得安?古人云:『當斷不斷,反受其亂。』唯請急殺此賊。劉幽求已共臣作定謀計訖,願以身正此事,赴死如歸。臣既職典禁兵,若奉殿下命,當即除翦。」上深以為然。暐又洩其謀於侍御史鄧光賓,玄宗大懼,遽列上其狀,睿
 宗下幽求等詔獄,令法官推鞫之。法官奏幽求等以疏間親,罪當死。玄宗屢救獲免,乃流幽求於封州,暐於峰州。



 歲餘,太平公主等伏誅,其日下詔曰:「劉幽求風雲玄感,川岳粹靈,學綜九流,文窮三變。義以臨事,精能貫日;忠以成謀,用若投水。茂勛立艱難之際,嘉話盈啟沃之初,存讜直以不顧,為奸邪之所忌。釁萌頗露,譖端潛發,元宰見逐,讒人孔多。既殄群兇,方宣大化,期問政於經始,載登賢於蘿卜。可依舊金紫光祿大夫,守尚書左僕
 射,知軍國事,監修國史,上柱國、徐國公,仍依舊還封七百戶,並賜錦衣一襲。」



 開元初,改尚書左右僕射為左右丞相,乃授幽求尚書左丞相,兼黃門監。未幾,除太子少保,罷知政事。姚崇素嫉忌之,乃奏言幽求鬱怏於散職,兼有怨言,貶授睦州刺史,削其實封六百戶。歲餘,稍遷杭州刺史。三年,轉桂陽郡刺史,在道憤恚而卒,年六十一,贈禮部尚書,謚曰文獻,配享睿宗廟庭。建中三年,重贈司徒。



 鐘紹京,虔州贛人也。初為司農錄事,以工書直鳳閣,則天時明堂門額、九鼎之銘,及諸宮殿門榜,皆紹京所題。景龍中,為苑總監。玄宗之誅韋氏,紹京夜中帥戶奴及丁夫以從。及事成,其夜拜紹京銀青光祿大夫、中書侍郎,參知機務。翌日,進拜中書令,加光祿大夫,封越國公,賜實封五百戶,賜物二千段、馬十匹。紹京既當朝用事,恣情賞罰,甚為時人所惡。俄又抗疏讓官,睿宗納薛稷之言,乃轉為戶部尚書,出為蜀州刺史。



 玄宗即位,復召
 拜戶部尚書,遷太子詹事。時姚崇素惡紹京之為人,因奏紹京發言怨望,左遷綿州刺史。及坐事,累貶琰川尉,盡削其階爵及實封。俄又歷遷溫州別駕。開元十五年,入朝,因垂泣奏曰:「陛下豈不記疇昔之事耶?何忍棄臣荒外,永不見闕庭。且當時立功之人,今並亡歿,唯臣衰老獨在,陛下豈不垂愍耶?」玄宗為之惘然,即日拜銀青光祿大夫、右諭德。久之,轉少詹事。年八十餘卒。紹京雅好書畫古跡,聚二王及褚遂良書至數十百卷。建中元
 年,重贈太子太傅。



 郭元振,魏州貴鄉人。舉進士,授通泉尉。任俠使氣,不以細務介意,前後掠賣所部千餘人,以遺賓客,百姓苦之。則天聞其名,召見與語,甚奇之。時吐蕃請和,乃授元振右武衛鎧曹,充使聘於吐蕃。吐蕃大將論欽陵請去四鎮兵,分十姓之地,朝廷使元振因察其事宜。元振還,上疏曰:



 臣聞利或生害,害亦生利。國家難消息者,唯吐蕃與默啜耳。今吐蕃請和,默啜受命,是將大利於中國也。
 若圖之不審,則害必隨之。今欽陵欲分裂十姓,去四鎮兵,此誠動靜之機,不可輕舉措也。今若直塞其善意,恐邊患之起,必甚於前,若以鎮不可拔,兵不可抽,則宜為計以緩之,藉事以誘之,使彼和望未絕,則其惡意亦不得頓生。



 且四鎮之患遠,甘、涼之患近,取舍之計,實宜深圖。今國之外患者,十姓、四鎮是也;內患者,甘、涼、瓜、肅是也。關、隴之人,久事屯戍,向三十年,力用竭矣。脫甘、涼有不虞,豈堪廣調發耶?夫善為國者,當先料內以敵外,不
 貪外以害內,然後夷夏晏安,昇平可保。如欽陵云「四鎮諸部接界,懼漢侵竊,故有是請」,此則吐蕃所要者。然青海、吐渾密邇蘭、鄯,比為漢患,實在茲輩,斯亦國家之要者。



 今宜報欽陵云:「國家非吝四鎮,本置此以扼蕃國之要,分蕃國之力,使不得並兵東侵。今委之於蕃,力強易為東擾。必實無東侵意,則還漢吐渾諸部及青海故地,即俟斤部落亦還吐蕃。」如此,則足塞欽陵之口,而事未全絕也。如欽陵小有乖,則曲在彼矣。又西邊諸國,款附
 歲久,論其情義,豈可與吐蕃同日而言。今未知其利害,未審其情實,遙有分裂,亦恐傷彼諸國之意,非制馭之長算也。



 則天從之。



 又上言曰:「臣揣吐蕃百姓倦徭戍久矣,咸願早和。其大將論欽陵欲分四鎮境,統兵專制,故不欲歸款。若國家每歲發和親使,而欽陵常不從命,則彼蕃之人怨欽陵日深,望國恩日甚,設欲廣舉醜徒,固亦難矣。斯亦離間之漸,必可使其上下俱懷情阻。」則天甚然之。自是數年間,吐蕃君臣果相猜貳,因誅大將論
 欽陵。其弟贊婆及兄子莽布支並來降,則天仍令元振與河源軍大使夫蒙令卿率騎以接之。後吐蕃將麴莽布支率兵入寇,涼州都督唐休璟勒兵破之。元振參預其謀,以功拜主客郎中。



 大足元年,遷涼州都督、隴右諸軍州大使。先是,涼州封界南北不過四百餘里,既逼突厥、吐蕃,二寇頻歲奄至城下,百姓苦之。元振始於南境破口置和戎城,北界磧中置白亭軍,控其要路,乃拓州境一千五百里,自是寇虜不復更至城下。元振又令甘州
 刺史李漢通開置屯田,盡其水陸之利。舊涼州粟斛售至數千,及漢通收率之後,數年豐稔,乃至一匹絹粟數十斛,積軍糧支數十年。元振風神偉壯,而善於撫御,在涼州五年,夷夏畏慕,令行禁止,牛羊被野,路不拾遺。



 神龍中,遷左驍衛將軍,兼檢校安西大都護。時西突厥首領烏質勒部落強盛,款塞通和,元振就其牙帳計會軍事。時天大雪,元振立於帳前,與烏質勒言議。須臾,雪深風凍,元振未嘗移足,烏質勒年老,不勝寒苦,會罷而死。
 其子娑葛以元振故殺其父,謀勒兵攻之。副使御史中丞解琬知其謀,勸元振夜遁,元振曰:「吾以誠信待人,何所疑懼,且深在寇庭,遁將安適?」乃安臥帳中。明日,親入虜帳,哭之甚哀,行吊贈之禮。娑葛乃感其義,復與元振通好,因遣使進馬五千匹及方物。制以元振為金山道行軍大總管。



 先是,娑葛與阿史那闕啜忠節不和,屢相侵掠。闕啜兵眾寡弱,漸不能支。元振奏請追闕啜入朝宿衛,移其部落入於瓜、沙等州安置,制從之。闕啜行至
 播仙城,與經略使、右威衛將軍周以悌相遇,以悌謂之曰:「國家有以高班厚秩待君者,以君統攝部落,下有兵眾故也。今輕身入朝,是一老胡耳,在朝之人,誰復喜見?非唯官資難得,亦恐性命在人。今宰相有宗楚客、紀處訥,並專權用事,何不厚貺二公,請留不行。仍發安西兵並引吐蕃以擊娑葛,求阿史那獻為可汗以招十姓,使郭虔瓘往拔汗那徵甲馬以助軍用。既得報讎,又得存其部落。如此,與入朝受制於人,豈復同也!」闕啜然其言,便
 勒兵攻陷於闐坎城,獲金寶及生口,遣人間道納賂於宗、紀。元振聞其謀,遽上疏曰:



 往者吐蕃所爭,唯論十姓、四鎮,國家不能舍與,所以不得通和。今吐蕃不相侵擾者,不是顧國家和信不來,直是其國中諸豪及泥婆羅門等屬國自有攜貳。故贊普躬往南征,身殞寇庭,國中大亂,嫡庶競立,將相爭權,自相屠滅。兼以人畜疲癘,財力困窮,人事天時,俱未稱愜。所以屈志,且共漢和,非是本心能忘情於十姓、四鎮也。如國力殷足之後,則必爭
 小事,方便絕和,縱其醜徒,來相吞擾,此必然之計也。



 今忠節乃不論國家大計,直欲為吐蕃作鄉導主人,四鎮危機,恐從此啟。頃緣默啜憑陵,所應處兼四鎮兵士,歲久貧羸,其勢未能得為忠節經略,非是憐突騎施也。忠節不體國家中外之意,而別求吐蕃,吐蕃得志,忠節則在其掌握,若為復得事漢?往年吐蕃於國非有恩有力,猶欲爭十姓、四鎮;今若效力樹恩之後,或請分於闐、疏勒,不知欲以何理抑之?又其國中諸蠻及婆羅門等國
 見今攜背,忽請漢兵助其除討,亦不知欲以何詞拒之?是以古之賢人,皆不願夷狄妄惠,非是不欲其力,懼後求請無厭,益生中國之事。故臣愚以為用吐蕃之力,實為非便。



 又請阿史那獻者,豈不以獻等並可汗子孫,來即可以招脅十姓?但獻父元慶、叔僕羅、兄俀子並斛瑟羅及懷道,豈不俱是可汗子孫?往四鎮以他匐十姓不安,請冊元慶為可汗,竟不能招脅得十姓,卻令元慶沒賊,四鎮盡淪。頃年,忠節請斛瑟羅及懷道俱為可汗,亦
 不能招脅得十姓,卻遣碎葉數年被圍,兵士饑餒。又,吐蕃頃年亦冊俀子及僕羅並拔布相次為可汗,亦不能招得十姓,皆自磨滅。何則?此等子孫非有惠下之才,恩義素絕,故人心不歸,來者既不能招攜,唯與四鎮卻生瘡磐,則知冊可汗子孫,亦未獲招脅十姓之算也。今料獻之恩義,又隔遠於其父兄,向來既未樹立威恩,亦何由即遣人心懸附。若自舉兵,力勢能取,則可招脅十姓,不必要須得可汗子孫也。



 又,欲令郭虔瓘入拔汗那稅
 甲稅馬以充軍用者,但往年虔瓘已曾與忠節擅入拔汗那稅甲稅馬,臣在疏勒其訪,不聞得一甲入軍,拔汗那胡不勝侵擾,南勾吐蕃,即將俀子重擾四鎮。又虔瓘往入之際,拔汗那四面無賊可勾,恣意侵吞,如獨行無人之境,猶引俀子為蔽。今此有娑葛強寇,知虔瓘等西行,必請相救。胡人則內堅城壘,突厥則外伺邀遮。必知虔瓘等不能更如往年得恣其吞噬,內外受敵,自陷危道,徒與賊結隙,令四鎮不安。臣愚揣之,亦為非計。



 疏奏
 不省。



 楚客等既受闕啜之賂,乃建議遣攝御史中丞馮嘉賓持節安撫闕啜,御史呂守素處置四鎮,持璽書便報元振。除牛師獎為安西副都護,便領甘、涼已西兵募,兼征吐蕃,以討娑葛。娑葛進馬使娑臘知楚客計,馳還報娑葛。娑葛是日發兵五千騎出安西,五千騎出撥換,五千騎出焉耆,五千騎出疏勒。時元振在疏勒,於河口柵不敢動。闕啜在計舒河口候見嘉賓,娑葛兵掩至,生擒闕啜,殺嘉賓等。呂守素至僻城,亦見害。又殺牛師獎
 於火燒城,乃陷安西,四鎮路絕。



 楚客又奏請周以悌代元振統眾,徵元振,將陷之。使阿史那獻為十姓可汗,置軍焉耆以取娑葛。娑葛遺元振書曰:「與漢本來無惡,只讎於闕啜。而宗尚書取闕啜金,枉擬破奴部落,馮中丞、牛都護相次而來,奴等豈坐受死!又聞史獻欲來,徒擾亂軍州,恐未有寧日,乞大使商量處置。」元振奏娑葛狀。楚客怒,奏言元振有異圖。元振使其子鴻間道奏其狀,以悌竟得罪,流於白州。復以元振代以悌,赦娑葛罪,冊
 為十四姓可汗。元振奏稱西土未寧,事資安撫,逗遛不敢歸京師。



 會楚客等被誅,睿宗即位,徵拜太僕卿,加銀青光祿大夫。景雲二年,同中書門下三品,代宋璟為吏部尚書。無幾,轉兵部尚書,封館陶縣男。時元振父愛年老在鄉,就拜濟州刺史,仍聽致仕。其冬,與韋安石、張說等俱罷知政事。先天元年,為朔方軍大總管,始築定遠城,以為行軍計集之所,至今賴之。明年,復同中書門下三品。及蕭至忠、竇懷貞等附太平公主潛謀不順,玄宗
 發羽林兵誅之,睿宗登承天門,元振躬率兵侍衛之。事定論功,進封代國公,食實封四百戶,賜物一千段。又令兼御史大夫,持節為朔方道大總管,以備突厥,未行。



 玄宗於驪山講武,坐軍容不整,坐於纛下,將斬以徇,劉幽求、張說於馬前諫曰:「元振有翊贊大功,雖有罪,當從原宥。」乃赦之,流於新州。尋又思其舊功,起為饒州司馬。元振自恃功勛,怏怏不得志,道病卒。開元十年,追贈太子少保。有文集二十卷。



 張說,字道濟,其先範陽人,代居河東,近又徙家河南之洛陽。弱冠應詔舉,對策乙第,授太子校書,累轉右補闕,預修《三教珠英》。久視年,則天幸三陽宮,自夏涉秋,不時還都,說上疏諫曰:



 陛下屯萬乘,幸離宮,暑退涼歸,未降還旨。愚臣固陋,恐非良策,請為陛下陳其不可。



 三陽宮去洛城一百六十里,有伊水之隔,崿阪之峻,過夏涉秋,水潦方積,道壞山險,不通轉運,河廣無梁,咫尺千里。扈從兵馬,日費資給,連雨彌旬,即難周濟。陛下太倉、武庫,
 並在都邑,紅粟利器,蘊若山丘。奈何去宗廟之上都,安山谷之僻處?是猶倒持劍戟,示人金尊柄,臣竊為陛下不取。夫禍變之生,在人所忽,故曰:「安樂必誡,無行所悔。」此不可止之理一也。



 宮成褊小,萬方輻湊,填城溢郭,並鍤無所。排斥居人,蓬宿草次,風雨暴至,不知庇托,孤煢老病,流轉衢巷。陛下作人父母,將若之何?此不可止之理二也。



 池亭奇巧,誘掖上心,削巒起觀,竭流漲海,俯貫地脈,仰出雲路,易山川之氣,奪農桑之土,延木石,運斧斤,
 山谷連聲,春夏不輟。勸陛下作此者,豈正人耶?《詩》云:「人亦勞止,汔可小康。」此不可止之理三也。



 御苑東西二十里,所出入來往,雜人甚多,外無墻垣局禁,內有榛溪谷,猛獸所伏,暴慝是憑。陛下往往輕行,警蹕不肅,歷蒙密,乘嶮戲,卒然有逸獸狂夫,驚犯左右,豈不殆哉!雖萬全無疑,然人主之動,不宜易也。《易》曰:「思患預防。」願陛下為萬姓持重。此不可止之理四也。



 今國家北有胡寇覷邊,南有夷獠騷徼。關西小旱,耕稼是憂;安東近平,輸漕
 方始。臣願陛下及時旋軫,深居上京,息人以展農,修德以來遠,罷不急之役,省無用之費。澄心澹懷,惟億萬年,蒼蒼群生,莫不幸甚。臣自度芻議,十不一從。何者?沮盤游之娛,間林沚之玩,規遠圖而替近適,要後利而棄前歡,未沃明主之心,已戾貴臣之意。然臣血誠密奏而不愛死者,不願負陛下言責之職耳。輕觸天威,伏地待罪。



 疏奏不省。



 長安初,修《三教珠英》畢,遷右史、內供奉,兼知考功貢舉事,擢拜鳳閣舍人。時臨臺監張易之與其弟
 昌宗構陷御史大夫魏元忠,稱其謀反,引說令證其事。說至御前,揚言元忠實不反,此是易之誣構耳。元忠由是免誅,說坐忤旨配流欽州。在嶺外歲餘。中宗即位,召拜兵部員外郎,累轉工部侍郎。景龍中,丁母憂去職,起復授黃門侍郎,累表固辭,言甚切至,優詔方許之。是時風教紊類,多以起復為榮,而說固節懇辭,竟終其喪制,大為識者所稱。服終,復為工部侍郎,俄拜兵部侍郎,加弘文館學士。



 睿宗即位,遷中書侍郎,兼雍州長史。景雲
 元年秋,譙王重福於東都構逆而死,留守捕系枝黨數百人,考訊結構之狀,經時不決。睿宗令說往按其獄,一宿捕獲重福謀主張靈均、鄭愔等,盡得其情狀,自餘枉被系禁者,一切釋放。睿宗勞之曰:「知卿按此獄,不枉良善,又不漏罪人。非卿忠正,豈能如此?」



 玄宗在東宮,說與國子司業褚無量俱為侍讀,深見親敬。明年,同中書門下平章事,監修國史。是歲二月,睿宗謂侍臣曰:「有術者上言,五日內有急兵入宮,卿等為朕備之。」左右相顧莫
 能對,說進曰:「此是讒人設計,擬搖動東宮耳。陛下若使太子監國,則君臣分定,自然窺覦路絕,災難不生。」睿宗大悅,即日下制皇太子監國。明年,又制皇太子即帝位。俄而太平公主引蕭至忠、崔湜等為宰相,以說為不附己,轉為尚書左丞,罷知政事,仍令往東都留司。說既知太平等陰懷異計,乃因使獻佩刀於玄宗,請先事討之,玄宗深嘉納焉。及至忠等伏誅,徵拜中書令,封燕國公,賜實封二百戶。其冬,改易官名,拜紫微令。



 自則天末年,
 季冬為潑寒胡戲,中宗嘗御樓以觀之。至是,因蕃夷入朝,又作此戲。說上疏諫曰:「臣聞韓宣適魯,見周禮而嘆;孔子會齊,數倡優之罪。列國如此,況天朝乎。今外蕃請和,選使朝謁,所望接以禮樂,示以兵威。雖曰戎夷,不可輕易,焉知無駒支之辯,由余之賢哉?且潑寒胡未聞典故,裸體跳足,盛德何觀;揮水投泥,失容斯甚。法殊魯禮,褻比齊優,恐非干羽柔遠之義,樽俎折沖之禮。」自是此戲乃絕。



 俄而為姚崇所構,出為相州刺史,仍充河北道
 按察使。俄又坐事左轉岳州刺史,仍停所食實封三百戶,遷右羽林將軍,兼檢校幽州都督。開元七年,檢校並州大都督府長史,兼天兵軍大使,攝御史大夫,兼修國史,仍齎史本隨軍修撰。八年秋,朔方大使王晙誅河曲降虜阿布思等千餘人。時並州大同、橫野等軍有九姓同羅、拔曳固等部落,皆懷震懼。說率輕騎二十人,持旌節直詣其部落,宿於帳下,召酋帥以慰撫之。副使李憲以為夷虜難信,不宜輕涉不測,馳狀以諫,說報書曰:「吾
 肉非黃羊,必不畏吃;血非野馬,必不畏刺。士見危致命,是吾效死之秋也。」於是九姓感義,其心乃安。



 九年四月,胡賊康待賓率眾反,據長泉縣,自稱葉護,攻陷蘭池等六州。詔王晙率兵討之,仍令說相知經略。時叛胡與黨項連結,攻銀城、連谷,以據倉糧,說統馬步萬人出合河關掩擊,大破之。追至駱駝堰,胡及黨項自相殺。阻夜,胡乃西遁入鐵建山,餘黨潰散。說招集黨項,復其居業。副使史獻請因此誅黨項,絕其翻動之計,說曰:「先王之道,
 推亡固存,如盡誅之,是逆天道也。」因奏置麟州,以安置黨項餘燼。其年,拜兵部尚書、同中書門下三品,仍依舊修國史。



 明年,又敕說為朔方軍節度大使,往巡五城,處置兵馬。時有康待賓餘黨慶州方渠降胡康願子自立為可汗,舉兵反,謀掠監牧馬,西涉河出塞。說進兵討擒之,並獲其家屬於木盤山,送都斬之,其黨悉平,獲男女三千餘人。於是移河曲六州殘胡五萬餘口配許、汝、唐、鄧、仙、豫等州,始空河南逆方千里之地。說以討賊功,復
 賜實封二百戶。先是,緣邊鎮兵常六十餘萬,說以時無強寇,不假師眾,奏罷二十餘萬,勒還營農。玄宗頗以為疑,說奏曰:「臣久在疆場,具悉邊事,軍將但欲自衛及雜使營私。若御敵制勝,不在多擁閑冗,以妨農務。陛下若以為疑,臣請以闔門百口為保。以陛下之明,四夷畏伏,必不慮減兵而招寇也。」上乃從之。



 時當番衛士,浸以貧弱,逃亡略盡。說又建策,請一切罷之,別召募強壯,令其宿衛,不簡色役,優為條例,逋逃者必爭來應募。上從之。旬日,得
 精兵一十三萬人,分系諸衛,更番上下,以實京師,其後彍騎是也。



 是歲,玄宗將還京,而便幸並州,說進言曰:「太原是國家王業所起,陛下行幸,振威耀武,並建碑紀德,以申永思之意。若便入京,路由河東,有漢武隹上後土之祀,此禮久闕,歷代莫能行之。願陛下紹斯墜典,以為三農祈穀,此誠萬姓之福也。」上從其言。及祀后土禮畢,說代張嘉貞為中書令。夏四月,玄宗親為詔曰:「動惟直道,累聞獻替之誠;言則不諛,自得謀猷之體。政令必俟
 其增損,圖書又藉其刊削,才望兼著,理合褒升。考中上。」



 說又首建封禪之議。十三年,受詔與右散騎常侍徐堅、太常少卿韋縚等撰東封儀注。舊儀不便者,說多所裁正,語在《禮志》。玄宗尋召說及禮官學士等賜宴於集仙殿,謂說曰:「今與卿等賢才同宴於此,宜改名為集賢殿。」因下制改麗正書院為集賢殿書院,授說集賢院學士,知院事。



 及將東封,授說為右丞相兼中書令,源乾曜為左丞相兼侍中,蓋勒成岱宗,以明宰相佐成王化也。說
 又撰《封禪壇頌》以紀聖德。初,源乾曜本意不欲封禪,而說因贊其事,由是頗不相平。及登山,說引所親攝供奉官及主事等從升,加階超入五品,其餘官多不得上。又行從兵士,惟加勛,不得賜物,由是頗為內外所怨。先是,御史中丞宇文融獻策,請括天下逃戶及籍外剩田,置十道勸農使,分往檢察。說嫌其擾人不便,數建議違之。及東封還,融又密奏分吏部置十銓,融與禮部尚書蘇頲等分掌選事。融等每有奏請,皆為說所抑,由是銓綜
 失敘。融乃與御史大夫崔隱甫、中丞李林甫奏彈說引術士夜解及受贓等狀,敕宰臣源乾曜、刑部尚書韋抗、大理少卿胡珪、御史大夫崔隱甫就尚書省鞫問。說兄左庶子光詣朝堂割耳稱冤。時中書主事張觀、左衛長史範堯臣並依倚說勢,詐假納賂,又私度僧王慶則往來與說占卜吉兇,為隱甫等所鞫伏罪。說經兩宿,玄宗使中官高力士視之,回奏:「說坐於草上,於瓦器中食,蓬首垢面,自罰憂懼之甚。」玄宗憫之。力士奏曰:「說曾為侍
 讀,又於國有功。」玄宗然其奏,由是停兼中書令,觀及慶則決杖而死,連坐遷貶者十餘人。隱甫及融等恐說復用為己患,又密奏毀之。明年,詔說致仕,仍令在家修史。



 初,說為相時,玄宗意欲討吐蕃,說密奏許其通和,以息邊境,玄宗不從。及瓜州失守,王掞死,說因獲巂州斗羊,上表獻之,以申諷諭。其表:「臣聞勇士冠雞,武夫戴鶡,推情舉類,獲此斗羊。遠生越巂,蓄性剛決,敵不避強,戰不顧死,雖為微物,志不可挫。伏惟陛下選良家於六郡,
 求猛士於四方,鳥不遁才,獸不藏伎。如蒙效奇靈圃,角力天場,卻鼓怒以作氣,前躑躅以奮擊。趹若奔雲之交觸,碎如轉石之相叩,裂骨賭勝,濺血爭雄,敢毅見而沖冠,鷙狠聞而擊節。冀將少助明主市駿骨、揖怒蛙之意也。若使羊能言,必將曰『若鬥不解,立有死者』。所賴至仁無殘,量力取勸焉。臣緣損足,未堪履地,謹遣男詣金明門奉進。」玄宗深悟其意,賜絹及雜彩一千匹。



 十七年,復拜尚書左丞相、集賢院學士,尋代源乾曜為尚書左丞
 相。視事之日,上敕所司供帳,設音樂,內出酒食,禦制詩一篇以敘其事。尋以修謁陵儀注功,加開府儀同三司。時長子均為中書舍人,次子垍尚寧親公主,拜駙馬都尉,又特授說兄慶王傅光為銀青光祿大夫。當時榮寵,莫與為比。



 十八年,遇疾,玄宗每日令中使問疾,並手寫藥方賜之。十二月薨,時年六十四。上慘惻久之,遽於光順門舉哀,因罷十九年元正朝會,詔曰:



 弘濟艱難,參其功者時傑;經緯禮樂,贊其道者人師。式瞻而百度允厘,
 既往而千載貽範。臺衡軒鼎,垂黼藻於當今;徽策寵章,播芳蕤於後葉。故開府儀同三司、尚書左丞相、集賢院學士知院事、上柱國、燕國公張說,辰象降靈,雲龍合契。元和體其沖粹,妙有釋其至賾。挹而莫測,仰之彌高。精義探系表之微,英辭鼓天下之動。昔侍春誦,綢繆歲華。含舂容之聲,叩而盡應;蘊泉源之智,啟而斯沃。授命興國,則天衢以通;濟用和民,則朝政惟允。司鈞總六官之紀,端揆為萬邦之式。方弘風緯俗,返本於上古之初;而
 邁德振仁,不臻於中壽之福。於嗟不憖,既喪斯文。宣室餘談,泠然在耳;王殿遺草,宛留其跡。言念忠賢,良深震悼。是使當寧撫幾,臨樂徹懸,罷稱觴之儀,遵往襚之禮。可贈太師,賜物五百段。



 始玄宗在東宮,說已蒙禮遇。及太平用事,儲位頗危,說獨排其黨,請太子監國,深謀密畫,竟清內難,遂為開元宗臣。前後三秉大政,掌文學之任凡三十年。為文俊麗,用思精密,朝廷大手筆,皆特承中旨撰述,天下詞人,咸諷誦之。尤長於碑文、墓志,當代
 無能及者。喜延納後進,善用己長,引文儒之士,佐佑王化,當承平歲久,志在粉飾盛時。其封泰山,祠脽上,謁五陵,開集賢,修太宗之政,皆說為倡首。而又敦氣義,重然諾,於君臣朋友之際,大義甚篤。時中書舍人徐堅自負文學,常以集賢院學士多非其人,所司供膳太厚,嘗謂朝列曰:「此輩於國家何益,如此虛費。」將建議罷之。說曰:「自古帝王功成,則有奢縱之失,或興池臺,或玩聲色。今聖上崇儒重道,親自講論,刊正圖書,詳延學者。今麗正
 書院,天子禮樂之司,永代規模,不易之道也。所費者細,所益者大。徐子之言,何其隘哉!」玄宗知之,由是薄堅。說既遭訕鑠,罷知政事,專集賢文史之任,每軍國大事,帝遣中使先訪其可否。說嘗自制其父《贈丹州刺史騭碑文》,玄宗聞之而御書其碑額賜之曰「嗚呼,積善之墓」。有文集三十卷。太常謚議曰「文貞」,左司郎中陽伯誠駁議,以為不稱,工部侍郎張九齡立議,請依太常為定,紛綸未決。玄宗為說自制神道碑文,御筆賜謚曰「文貞」,由是
 方定。



 均、垍皆能文。說在中書,兄弟已掌綸翰之任。居父憂服闋,均除戶部侍郎,轉兵部。二十六年,坐累貶饒州刺史,以太子左庶子征,復為戶部侍郎。九載,遷刑部尚書。自以才名當為宰輔,常為李林甫所抑。及林甫卒,依附權臣陳希烈,期於必取。既而楊國忠用事,心頗惡之,罷希烈知政事,引文部侍郎韋見素代之,仍以均為大理卿。均大失望,意常鬱鬱。祿山之亂,受偽命為中書令,掌賊樞衡。李峴、呂諲條疏陷賊官,均當大闢。肅宗於說
 有舊恩,特免死,長流合浦郡。



 垍,以主婿,玄宗特深恩寵,許於禁中置內宅,侍為文章,嘗賜珍玩,不可勝數。時兄均亦供奉翰林院,常以所賜示均,均戲謂垍曰:「此婦翁與女婿,非天子賜學士也。」天寶中,玄宗嘗幸垍內宅,謂垍曰:「希烈累辭機務,朕擇其代者,孰可?」垍錯愕未對,帝即曰:「無逾吾愛婿矣。」垍降階陳謝。楊國忠聞而惡之,及希烈罷相,舉韋見素代,垍深觖望。天寶十三年正月,範陽節度使安祿山入朝。時祿山立破奚、契丹功,尤加
 寵異。祿山求帶平章事,下中書擬議。國忠進言曰:「祿山誠立軍功,然眼不識字,制命若行,臣恐四夷輕國。」玄宗乃止,加左僕射而已。及祿山還鎮,命中官高力士餞於滻坡。既還,帝曰:「祿山慰意否?」力士曰:「觀其深心鬱鬱,必伺知宰相之命不行故也。」帝告國忠,國忠曰:「此議他人不知,必張垍所告。」帝怒,盡逐張垍兄弟。出均為建安太守,垍為盧溪郡司馬,埱為宜春郡司馬。歲中召還,再遷為太常卿。



 祿山之亂,玄宗幸蜀,宰相韋見素、楊國忠、御
 史大夫魏方進等從,朝臣多不至。次咸陽,帝謂高力士曰:「昨日蒼黃離京,朝官不知所詣,今日誰當至者?」力士曰:「張垍兄弟世受國恩,又連戚屬,必當先至。房琯素有宰相望,深為祿山所器,必不此來。」帝曰:「事未可料。」是日,琯至,帝大悅,因問均、垍,琯曰:「臣離京時,亦過其舍,比約同行,均報云『已於城南取馬』。觀其趣向,來意不切。」既而均弟兄果受祿山偽命,垍與陳希烈為賊宰相,垍死於賊中。



 陳希烈者,宋州人也。精玄學,書無不覽。開元中,玄宗留意經義,自褚無量、元行沖卒後,得希烈與鳳翔人馮朝隱,常於禁中講《老》、《易》。累遷至秘書少監,代張九齡專判集賢院事。玄宗凡有撰述,必經希烈之手。李林甫知上睠待深異,又以和裕易制,乃引為宰相,同知政事,相行甚歡。而林甫居位日久,雖陰謀奸畫足以自固,亦希烈佐佑唱和之力也。累遷兼兵部尚書、左相,封潁川郡開國公,寵遇侔於林甫。及林甫死,楊國忠用事,素忌嫉之。
 乃引韋見素同列,罷希烈知政事,守太子太師。希烈失恩,心頗怏怏。祿山之亂,與張垍、達奚珣同掌賊之機衡。六等定罪,希烈當斬,肅宗以上皇素遇,賜死於家。



 史臣曰:劉徐公負不羈之材,逢抵戲之運,遂能奮命決策,扶力中興,朝為徒步之人,夕據公侯之位,茍非輕死重利,不恥不義之富,安及此哉!郭代公、張燕公解逢掖而登將壇,驅貔虎之師,斷獯戎之臂,暨居衡軸,克致隆平,可謂武緯文經,惟申與甫而已。惜乎均、垍務速,失節
 賊廷。自武德已來,稱賢相者,房、杜、姚、宋四公,皆遭無賴子弟污圮先業,非獨燕國之不幸也。希烈柔而多智,長於名理,竟死於名。所謂離婁不見其眉睫,與夫平叔、太初,同膏肓耳。



 贊曰:箕、微去紂,閎、散扶昌。謀不近義,旋踵而亡。幽求不令,道濟允臧。偉哉郭侯,勛德煌煌。



\end{pinyinscope}