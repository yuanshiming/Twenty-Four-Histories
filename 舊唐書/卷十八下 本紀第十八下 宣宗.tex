\article{卷十八下 本紀第十八下 宣宗}

\begin{pinyinscope}

 宣
 宗聖武獻文孝皇帝諱忱,憲宗第十三子,母曰孝明皇後鄭氏。元和五年六月二十二日,生於大明宮。長慶元年三月,封光王,名怡。會昌六年三月一日,開宗疾篤,
 遺詔立為皇太叔,權勾當軍國政事。翌日,柩前即帝位,改今名,時年三十七。帝外晦而內朗,嚴重寡言,視瞻特異。幼時宮中以為不慧。十餘歲時,遇重疾沈綴,忽有光輝燭身,蹶然而興,正身拱揖,如對臣僚。乳媼以為心疾。穆宗視之,扶背曰:「此吾家英物,非心憊也。」賜以玉如意、御馬、金帶。常夢乘龍升天,言之於鄭后,乃曰L:「此不宜人知者,幸勿復言。」歷大和、會昌朝,愈事韜晦,群巨游處,未嘗有言。文宗、武宗幸十六宅宴集,強誘其言,以為戲
 劇,謂之「光叔」。武宗氣豪,尤不為禮。及監國之日,哀毀滿容,接待群僚,決斷庶務,人方見其隱德焉。四月辛未,釋服,尊母鄭氏曰皇太后。以兵部侍郎、翰林學士承旨白敏中守本官、同中書門下平章事;以特進、守太尉、門下侍郎、同平章事、上柱國、衛國公、食邑二千戶李德裕檢校太尉、同平章事、江陵尹、荊南節度使;以中散大夫、大理卿馬植為金紫光祿大夫、刑部侍郎,充諸道鹽鐵等使。以成德軍節度使王元逵檢校太保,山南西道節度
 使王起檢校司空,魏博節度使何弘敬、淮南節度使李紳並檢校司空,劍南西川節度使崔鄲檢校尚書右僕射,同中書門下平章事並如故。東都留守李石奏修奉太廟畢,所司迎奉太微宮神主祔廟訖。東都太廟者,本武后家廟,神龍中中宗反正,廢武氏廟主,立太祖已下神主付之。安祿山陷洛陽,以廟為馬廄,棄其神主,而協律郎嚴郢收而藏之。史思明再陷洛陽,尋又散失。賊平,東京留守盧正己又募得之。廟已焚毀,乃寄主於太微宮。
 大歷十四年,留守路嗣恭奏重修太廟,以迎神主。詔百官參議,紛然不定,禮儀使顏真卿堅請歸祔,不從。會昌五年,留守李石因太微宮正殿圮陊,以廢弘敬寺為太廟,迎神主祔之。又下百僚議,皆言準故事,無兩都俱置之禮,唯禮部侍郎陳商議云:「周之文、武,有鎬、洛二廟,今兩都異廟可也。然不宜置主於廟,主宜依禮瘞廟之北墉下。」事未行而武宗崩。宣宗即位,因詔有司迎太微宮寓主,祔廢寺之新廟,而知禮者非之。制皇長男溫可
 封鄆王,二男涇可封雅王,第三男滋可封蘄王,第四男沂可封慶王。



 五月,左右街功德使奏:「準今月五日赦書節文,上都兩街留四寺,外更添置八所。兩所依舊名興唐寺、保壽寺。六所請改舊名,寶應寺改為資聖寺,青龍寺改為護國寺,菩提寺改為保唐寺,清禪寺改為安國寺,法云尼寺改為唐安寺,崇敬尼寺改為唐昌寺。右街添置八所。西明寺改為福壽寺,莊嚴寺改為聖壽寺,舊留寺。二所舊名,千福寺改為興元寺,化度寺改為崇
 福寺,永泰寺改為萬壽寺,溫國寺改為崇聖寺,經行寺改為龍興寺,奉恩寺改為興福寺。」敕旨依奏。誅道士劉玄靖等十二人,以其說惑武宗,排毀釋氏故也。今月五日赦書節文,吏部三銓選士,祗憑資考,多匪實才,許觀察使、刺史有奇才異政之士,聞薦試用。又觀察使、刺史交代之時,冊書所交戶口如能增添至千戶,即與超遷;如逃亡至七百戶,罷後三年內不得任使。又徒流人在天德、振武者,管中量借糧種,俾令耕田以為業。以劍南
 東川節度使、檢校禮部尚書盧商為兵部侍郎、同平章事。



 六月,以戶部侍郎、充諸道鹽鐵轉運使馬植本官同平章事。七月,以兵部尚書李讓夷為劍南東川節度使。十月,敕:「太廟祫享,合以功臣配。其憲宗廟,以裴度、杜黃裳、李愬、高崇文等配享。」以荊南節度使李德裕為東都留守。



 十一月,有司享太廟,其穆宗室文曰「皇兄」。太常博士閔慶之奏:「夫禮有尊尊,而不敘親親。祝文稱弟未當,請改為『嗣皇帝』。」從之。京兆府奏:「京師百司職田斛斗,請
 準會昌三年例,許人永自送納京師,親冀州縣無得欺隱。」從之。以江西觀察使周墀為義成軍節度使、鄭滑觀察等使。十二月,刑部尚書、判度支崔元式奏:「準七月二日敕,綾紗絹等次弱疋段,並同禁斷,不得織造。臣欲與鹽鐵戶部三司同條疏,先勘左藏庫,令分析出次弱疋段州府,即牒本道官搜索狹小機杼,令焚毀。其已納到次弱疋段,具數以聞。」上從之。



 大中元年春正月戊戌朔,宮苑使奏:「皇帝致齋行事,內
 諸宮苑門共九十四所,並令鎖閉,鑰匙進內。候車駕還宮,則請領。」從之。戊申,皇帝有事於郊廟,禮畢,御丹鳳門,大赦,改元,制條曰:「古者郎官出宰,卿相治郡,所以重親人之官,急為政之本。自澆風久扇,此道稍消,頡頏清途,便臻顯貴。治人之術,未嘗經心,欲使究百姓艱危,通天下利病,不可得也。為政之始,思厚儒風,軒墀近臣,蓋備顧問,如其不知人疾苦,何以膺朕眷求?今後諫議大夫、給事中、中書舍人曾任刺史、縣令,或在任有贓累者,宰
 臣不得擬議。守宰親人,職當撫字,三載考績,著在格言。貞元年中,屢下明詔,縣令五考,方得改移。近者因循,都不遵守,諸州或得三考,畿府罕及二年,以此字人,若為成政?道塗郡吏有迎送之勞,鄉里庶民無蘇息之望。自今須滿三十六個月,永為常式。」二月丁卯,制憲宗第十七子惕封彭王,第十八子惴為棣王;皇第五子澤為濮王,第六子潤為鄂王。敕修百福殿。以檢校太尉、東都留守李德裕為太子少保,分司東都;以給事中鄭亞為桂州
 刺史、御史中丞、桂管防禦觀察等使。二月丁酉,禮部侍郎魏扶奏:「臣今年所放進士三十三人,其封彥卿、崔琢、鄭延休等三人,實有詞藝,為時所稱,皆以父兄見居重位,不令中選。」詔令翰林學士承旨、戶部侍郎韋琮重考覆,敕曰:「彥卿等所試文字,並合度程,可放及第。有司考試,祗在至公,如涉請托,自有朝典。今後但依常例放榜,不別有奏聞。」帝雅好儒士,留心貢舉。有時微行人間,採聽輿論,以觀選士之得失。每山池曲宴,學士詩什
 屬和,公卿出鎮,亦賦詩餞行。凡對臣僚,肅然拱揖,鮮有輕易之言。大臣或獻章疏,即燒香盥手而覽之。當時以大中之政有貞觀之風焉。又敕:「自今進士放榜後,杏園任依舊宴集,有司不得禁制。」武宗好巡游,故曲江亭禁人宴聚故也。閏三月,敕:「會昌季年,並省寺宇。雖云異方之教,無損致理之源。中國之人,久行其道,厘革過當,事體未弘。其靈山勝境、天下州府,應會昌五年四月所廢寺宇,有宿舊名僧,復能修創,一任住持,所司不得禁止。」
 四月,積慶太后蕭氏崩,謚曰貞獻,文宗母也。



 六月,以義成軍節度使周墀為兵部侍郎、判度支。冊黠戛斯王子為為英武誠明可汗,命鴻臚卿李業入蕃冊拜。以金紫光祿大夫、守太子少保分司東都、上柱國、奇章郡開國公、食邑二千戶牛僧孺守太子太師,銀青光祿大夫、行太子賓客、上柱國、隴西郡開國公、食邑二千戶李彥佐為太子太保。並依前分司。以左諫議大夫庾簡休為虢州刺史,以正議大夫、行尚書考功郎中、知制誥、上柱國崔
 璵為中書舍人,以中散大夫、前湖州刺史、彭陽縣開國男、食邑三百戶令狐綯行尚書考功郎中、知制誥。



 秋七月,制以正議大夫、尚書戶部侍郎、知制誥、翰林學士承旨、柱國、賜紫金魚袋韋琮以本官同中書門下平章事。以太子少保分司東都、衛國公李德裕為人所訟,貶潮州司馬員外置同正員。



 八月,工部尚書、中書侍郎、平章事盧商出為鄂岳觀察使,。神策軍奏修百福殿成,名其殿曰雍和殿,樓曰親親樓,凡廊舍屋宇七百間,以會諸
 王子孫。



 九月,前永寧縣尉吳汝納詣闕稱冤,言:「弟湘會昌四年任揚州江都縣尉,被節度使李紳誣奏湘贓罪,宰相李德裕曲情祔紳,斷臣弟湘致死。」詔下御史臺鞫按。



 二年春正月壬戌,宰臣率文武百僚上徽號曰聖敬文思和武光孝皇帝,御宣政殿受冊訖,宣德音。神策軍修左銀臺門樓、屋宇及南面城墻,至睿武樓。



 二月,制劍南西川節度、光祿大夫、檢校吏部尚書、同平章事、成都尹、
 上柱國、隴西郡開國公、食邑二千戶李回責授湖南觀察使,桂州刺史、御史中丞、桂管防禦觀察使鄭亞貶循州刺史,前淮南觀察判官魏鉶貶吉州司戶,陸渾縣令元壽貶韶州司戶,殿中侍御史蔡京貶澧州司馬。御史臺奏:



 據三司推勘吳湘獄,謹具逐人罪狀如後:揚州都虞候盧行立、劉群,於會昌二年一月十四日,於阿顏家吃酒,與阿顏母阿焦同坐,群自擬收阿顏為妻,妄稱監軍使處分,要阿顏進奉,不得嫁人,兼擅令人監守。其阿
 焦遂與江都縣尉吳湘密約,嫁阿顏與湘。劉群與押軍牙官李克勛即時遮欄不得,乃令江都百姓論湘取受,節度使李紳追湘下獄,計贓處死。具獄奏聞。朝廷疑其冤,差御史崔元藻往揚州按問,據湘雖有取受,罪不至死。李德裕黨附李紳,乃貶元藻嶺南,取淮南元申文案,斷湘處死。今據三司使追崔元藻及淮南元推判官魏鉶並關連人款狀,淮南都虞候劉群、元推判官魏鉶、典孫貞高利錢倚黃嵩、江都縣典沈頒陳宰、節度押牙白
 沙鎮遏使傅義、左都虞候盧行立、天長縣令張弘思、曲張洙清陳回、右廂子巡李行璠、典臣金弘舉、送吳湘妻女至澧州取受錢物人潘宰、前揚府錄事參軍李公佐、元推官元壽吳珙翁恭、太子少保分司李德裕、西川節度使李回、桂管觀察使鄭亞等,伏候敕旨。



 其月,敕:



 李回、鄭亞、元壽、魏鉶已從別敕處分。李紳起此冤訴,本由不真,今既身歿,無以加刑。粗塞眾情,量行削奪,宜追奪三任官告,送刑部注毀。其子孫稽於經義,罰不及嗣,並釋
 放。李德裕先朝委以重權,不務絕其黨庇,致使冤苦,直到於今,職爾之由,能無恨嘆!昨以李威所訴,已經遠貶。俯全事體,特為從寬,宜準去年敕令處分。張弘思、李公佐卑吏守官,制不由己,不能守正,曲附權臣,各削兩任官。崔元藻曾受無辜之貶,合從洗雪之條,委中書門下商量處分。李恪詳驗款狀,蠹害最深,以其多時,須議減等,委京兆府決脊杖十五,配流天德。李克勛欲收阿顏,決杖二十,配流硤州。劉群據其款狀,合議痛刑,曾效
 職官,不欲決脊,決臀杖五十,配流岳州。其盧行立及諸典吏,委三司使量罪科放訖聞奏。



 三月己酉,兵部侍郎、判度支周墀本官平章事。以禮部尚書、鹽鐵轉運使馬植本官同平章事。日本國王子入朝貢方物。王子善棋,帝令侍詔顧師言與之對手。



 五月己未,日有蝕之。



 六月己丑,太皇太后郭氏崩,謚懿安,憲宗妃,穆宗之母也。戶部侍郎,兼御史大夫、判度支崔龜從奏:「應諸司場院官請卻官本錢後,或有欺隱欠負,徵理須足,不得茍從
 恩蕩,以求放免。今後凡隱盜欠負,請如官典犯贓例處分。縱逢恩赦,不在免限。」從之。七月戊午,以前山南西道節度使高元裕為吏部尚書。



 八月戊子,朝散大夫、中書舍人、充翰林學士、上柱國、平陰縣開國男、食實封三百戶賜紫金魚袋畢諴為刑部侍郎。



 九月,敕:「比有無良之人,於街市投匿名文書,及於箭上或旗幡上縱為奸言,以亂國法。此後所由切加捉搦,如獲此色,便仰焚瘞,不得上聞。」十一月,兵部侍郎、判戶部事魏扶奏:「天下州府
 錢物、斛斗、文簿,並委錄事參軍專判,仍與長史通判,至交代時具數申奏。如無懸欠,量與減選注擬。」敕:「路隨等所修《憲宗實錄》舊本,卻仰施行。其會昌新修者,仰並進納。如有鈔錄得,敕到並納史館,不得輒留,委州府嚴加搜捕。」以戶部侍郎、判度支崔龜從本官同平章事。銀青光錄大夫、門下侍郎,兼禮部尚書、同平章事韋琮為太子詹事,分司東都。



 三年春正月丙寅,涇原節度使康季榮奏,吐蕃宰相論
 恐熱以秦、原、安樂三州及石門等七關之兵民歸國。詔太僕卿陸耽往喻旨,仍令靈武節度使硃叔明、邠寧節度使張君緒,各出本道兵馬應接其來。以太常卿封敖檢校兵部尚書,為興元尹、山南西道節度使。



 三月乙卯,敕待詔官宜令與刑法官、諫官次對。銀青光祿大夫、中書侍郎、同平章事、監修國史、上柱國、汝南縣開國子、食邑五百戶周墀檢校刑部尚書、梓州刺史,充劍南東川節度使。四月,以正議大夫、守中書侍郎、同平章事、集賢
 殿大學士、賜紫金魚袋馬植為太子賓客,分司東都;以正議大夫、守御史大夫、上柱國、博陵縣開國子、食邑五百戶、賜紫金袋崔鉉可中書侍郎、平章事;正議大夫、行兵部侍郎、判戶部事、上柱國、鉅鹿縣開國男、食邑五百戶、賜紫金魚袋魏扶可本官、平章事。



 五月,幽州節度使、檢校司徒、平章事張仲武卒,三軍以其子直方知留後事。



 六月癸未,五色雲見於京師。敕:先經流貶罪人,不幸歿於貶所,有情非惡逆,任經刑部陳牒,許令歸葬,絕
 遠之處,仍量事官給棺櫝。康季榮奏收原州、石門驛藏木峽制勝六盤石峽等六關訖。邠寧張君緒奏,今月十三日暇復蕭關。御史臺奏,義成軍節度使韋讓於懷真坊侵街造屋九間,已令毀拆訖。敕於蕭關置武州,改長樂為威州。七月,三州七關軍人百姓,皆河、隴遺黎,數千人見於闕下。上御延喜門撫慰,令其解辮,賜之冠帶,共賜絹十五萬疋。



 八月,鳳翔節度使李玭奏收復秦州,制曰:



 自昔皇王之有國也,曷嘗不文以守成,武以集事,
 參諸二柄,歸乎大寧。朕猥荷丕圖,思弘景運,憂勤庶政,四載於茲。每念河、湟土疆,綿亙遐闊。自天寶末,犬戎乘我多難,無力禦奸,遂縱腥膻,不遠京邑。事更十葉,時近百年。進士試能,靡不竭其長策;朝廷下議,皆亦聽其直詞。盡以不生邊事為永圖,且守舊地為明理,荏苒於是,收復無由。今者天地儲祥,祖宗垂佑,左衽輸款,邊壘連降,刷恥建功,所謀必克。實樞衡妙算,將帥雄稜,副玄元不爭之文,絕漢武遠征之悔。甌脫頓空於內地,斥堠全
 據於新封,莫大之休,指期而就。



 況將士等櫛沐風雨,暴露郊原,披荊棘而刁斗夜嚴,逐豺狼而穹廬曉破。動皆如意,古無與京,念此誠勤,宜加寵賞。涇原宜賜絹六萬疋,靈武五萬疋,鳳翔、邠寧各四萬疋,並以戶部產業物色充,仍侍季榮、叔明、李玼、君緒各回戈到鎮,度支差腳支送。四道立功將士,各具名銜聞奏,當議甄酬。其秦、威、原三州及七關側近,訪聞田土肥沃,水草豐美,如百姓能耕墾種蒔,五年內不加稅賦。五年已後重定戶籍,
 為永業。溫池鹽利,可贍邊陲,委度支制置聞奏。鳳翔、邠寧、靈武、涇原守鎮將士,如能於本戍處耕墾營田,即度支給賜牛糧子種,每年量得斛斗,便充軍糧,亦不限約定數。三州七關鎮守官健。每人給衣糧兩分,一分依常年例支給,一分度支加給,仍二年一替換。其家口委長吏切加安存。官健有莊田戶籍者,仰州縣放免差役。



 秦州至隴州已來道路,要置堡柵,與秦州應接,委李玭與劉皋即便計度聞奏。如商旅往來,官健父兄子弟通
 傳家信,關司並不得邀詰阻滯。三州七關刺史、關使,將來訓練捍防有效能者,並與超序官爵。劍南西川沿邊沒蕃州郡,如力能收復,本道亦宜接借。三州七關創置戍卒,且要務靜。如蕃人求市,切不得通;有來投降者,申取長吏處分。



 嗚呼!七關要害,三郡膏腴,候館之殘趾可尋,唐人之遺風尚在。追懷往事,良用興嗟。夫取不在廣,貴保其金湯;得必有時,詎計於遲速。今則便務修築,不進干戈,必使足食足兵,有備無患,載洽亭育之道,永致
 生靈之安。中外臣僚,宜體朕意。



 九月辛亥,西川節度使杜忭奏收復維州。制曰:



 朕祗荷丕業,思平泰階,將分邪正之源,冀使華夷胥悅。其有常登元輔,久奉武宗,深苞禍心,盜弄國柄,雖已行譴斥之典,而未塞億兆之言,是議再舉朝章,式遵彞憲。守潮州司馬員外置同正員李德裕,早藉門地,叨踐清華,累居將相之榮,唯以奸傾為業。當會昌之際,極公臺之榮,騁諛佞而得君,遂恣橫而持政,專權生事,妒賢害忠。動多詭異之謀,潛懷僭越之
 志。秉直者必棄,向善者盡排。誣貞良造朋黨之名,肆讒構生加諸之釁。計有逾於指鹿,罪實見其欺天。屬者方處鈞衡,曾無嫌避,委國史於愛婿之手,寵秘文於弱子之身,洎參信書,亦引親暱。恭惟《元和實錄》乃不刊之書,擅敢改張,罔有畏忌。奪他人之懿績,為私門之令猷。又附李紳之曲情,斷成吳湘之冤獄。凡彼簪纓之士,遏其取舍之途。驕居自誇,狡蠹無對,擢爾之發,數罪未窮。載窺立刻上之由,益驗無君之意。使天下之人,重足一跡,皆
 讋懼奉面,而慢易在心。為臣若斯,於法何逭。於戲!朕務全大體,久為含容,雖黜降其官榮,尚蓋藏其醜狀。而睥睨未已,兢惕無聞,積惡既彰,公議難抑。是宜移投荒服,以謝萬邦。中外臣僚,當知予意。可崖州司戶參軍,所在馳驛發遣,縱逢恩赦,不在量移之限。



 以起居郎庾道蔚、禮部員外郎李文儒並充翰林學士。



 十月辛巳,京師地震,河西、天德、靈夏尤甚,戍卒壓死者數千人。



 十一月,東川節度使鄭涯、鳳翔節度使李玭奏修文川谷路,自靈
 泉至白雲置十一驛,下詔褒美。經年為雨所壞,又令封敖修斜谷舊路。以刑部侍郎韋有翼為御史中丞,以職方員外郎鄭處誨兼御史知雜。幽州軍亂,逐其留後張直方,軍人推其衙將周綝為留後。十二月,追謚順宗曰至德大聖大安孝皇帝,憲宗曰昭文章武大聖孝皇帝。初以河、湟收復,百僚請加徽號,帝曰:「河、湟收復,繼成先志,朕欲追尊祖宗,以昭功烈。」白敏中等對曰:「非臣愚昧所能及。」至是,上御宣政殿行事,及冊出,俯樓目送,流涕
 嗚咽。崖州司戶參軍李德裕卒於貶所。



 四年春正月,以追尊二聖,御正殿,大赦天下。徒流比在天德者,以十年為限,既遇鴻恩,例減三載。但使循環添換,邊不闕人,次第放歸,人無怨苦。其秦、原、威、武諸州、諸關,先準格徒流人,亦量與立限,止於七年,如要住者,亦聽。諸州府縣官如請工假,一月巳下,權差諸判官;一月已上,即準勾當例,其課料等據數每貫刻二百文,與見判案官添給。有故意殺人者,雖已傷未死、已死更生,
 意欲殺傷,偶然得免,並同已殺人條處分。



 二月,皇女萬壽公主出降右拾遺鄭顥,以顥為銀青光祿大夫、行起居郎、駙馬都尉。



 三月己卯,刑部奏:「監臨主守,應將官物私自貸使並貸借人,及以己物中納官司者,並專知別當主掌所由有犯贓,並同犯入己贓,不在原赦之限。」從之。以幽州節度副大使檢校工部尚書張直方為左金吾衛將軍。四月,敕:「法司用刑,或持巧詐,分律兩端,遂成其罪。既奸吏得計,則黎庶何安?自今後應書罪定刑,宜
 直指其事,不得舞文,妄有援引。」又刑部奏:「淮今年正月一日敕節文,據會昌元年三月二十六日敕,竊盜贓至一貫文處死,宜委所司重詳定條目奏聞。臣等檢校,並請準建中三年三月二十四日敕,竊盜贓滿三疋已上決殺,如贓數不充,量請科放。」從之。七月丙子,大理卿劉蒙奏:「古者懸法示人,欲使人從善遠罪,至於不犯,以致刑措。準大和二年十月二十六日刑部侍郎高釴條疏,準勘節目一十一件,下諸州府粉壁書於錄事參軍食
 堂,每申奏罪人,須依前件節目。歲月滋久,文字湮淪,州縣推案,多違漏節目。今後請下諸道,令刻石置於會食之所,使官吏起坐觀省,記憶條目,庶令案牘周詳。」從之。



 八月,刑部侍郎、御史中丞魏掞奏:「諸道州府百姓詣臺訴事,多差御史推劾,臣恐煩勞州縣,先請差度支、戶部、鹽鐵院官帶憲銜者推劾。又各得三司使申稱,院官人數不多,例專掌院務,課績不辦。今諸道觀察使幕中判官,少不下五六人,請於其中帶憲銜者委令推劾。如累
 推有勞,能雪冤滯,御史臺闕官,便令奏用。」從之。



 九月,以朝請大夫、檢校禮部尚書、孟州刺史、河陽三城節度使李拭為太原尹、北都留守、河東節度等使。幽州節度周綝卒,軍人立其牙將張允伸為留後。十月,中書侍郎、平章事魏扶罷知政事。



 十一月己亥,敕:「收復成、維、扶等三州,建立已定,條令制置,一切合同。其已配到流人,宜淮秦、原、威、武等州流例,七年放還。」以戶部侍郎、判本司事令狐綯為兵部侍郎、同平章事。十二月,以華州刺史
 周敬復為光祿大夫、檢校左散騎常侍,兼洪州刺史、江南西道團練觀察使,賜金紫。



 五年春正月甲戌,制皇第七子洽封懷王,第八子汭為昭王,第九子汶為康王。敕兩京天下州府,起大中五年正月一日已後,三年內不得殺牛。如郊廟享祀合用者,即以諸畜代。



 二月,戶部侍郎裴休充諸道鹽鐵轉運等使。四月癸卯,刑部侍郎劉彖奏:據今年四月十三日已前,凡二百二十四年,雜制敕計六百四十六門,二千一
 百六十五條,議輕重,名曰《大中刑法統類》,欲行用之。



 五月,以太原尹、河東節度使李拭為鳳翔節度使;李業檢校戶部尚書、太原尹、北都留守,充河東節度使;守司空、門下侍郎、太原郡開國伯、食邑一千戶白敏中檢校司徒、同平章事邠州刺史,充邠寧節度觀察、東面招討黨項等使;以戶部侍郎、判戶部事魏謩本官同平章事。七月,宰相監修國史崔龜從續柳芳《唐歷》二十二卷上之。



 八月,敕:「公主邑司擅行文牒,恐多影庇,有率條章。今後
 公主除緣征封外,不得令邑司行文書牒府縣,如緣公事,令邑司申宗正寺,與酌事體施行。」沙州刺史張義潮遣兄義澤以瓜、沙、伊、肅等十一州戶口來獻,自河、隴陷蕃百餘年,至是悉復隴右故地。以義潮為瓜沙伊等州節度使。



 九月,敕:「條疏刺史交代,須一一交割公事與知州官,方得離任。準會昌元年敕,刺史只禁科率官吏抑配人戶,至於使州公廨及雜利潤,天下州府皆有規制,不敢違越。緣未有明敕處分,多被無良人吏致使恐嚇,或
 致言訟。起今後應刺史下擔什物,及除替後資送錢物,但不率斂官吏,科配百姓,一任各守州縣舊例色目支給。如無公廨,不在資送之限。若輒有率配,以入己贓論。」以正議大夫、兵部侍郎、諸道鹽鐵轉運使、上柱國、河東縣開國子裴休守禮部尚書,進階金紫;以前宣歙觀察使、太中大夫、檢校左散騎常侍裴諗權知兵部侍郎。十月己亥,京兆尹韋博奏:「京畿富戶為諸軍影占,茍免府縣色役,或有追訴,軍府紛然。請準會昌三年十二月
 敕,諸軍使不得強奪百姓入軍。」從之。十一月,中書侍郎,兼吏部尚書、平章崔龜從檢校尚書估僕射、汴州刺史,充宣武軍節度使。沙州置歸義軍,以張義潮為節度使。太子詹事姚康獻《帝王政纂》十卷;又撰《統史》三百卷,上自開闢,下盡隋朝,帝王美政、詔令、制置、銅鹽錢穀損益、用兵利害,下至僧道是非,無不備載,編年為之。國子祭酒馮審奏:「文宣王廟,始太宗立之,睿宗書額,武後竊政之日,改篆題『大周』二字,請削之。」從之。十二月,盜斫景陵
 神門戟,京兆尹韋博罰兩月俸,貶宗正卿李文舉睦州刺史,陵令吳閱岳州司馬,奉先令裴讓隋州司馬。是歲,湖南大饑。



 六年春正月戊辰,以隴州防禦使薛逵為秦州刺史、天雄軍使,兼秦、成兩州經略使。



 二月,右衛大將軍鄭光以賜田請免租稅。宰相魏謩奏曰:「鄭光以國舅之親,賜田可也,免稅無以勸蒸民。」敕曰:「一依人戶例供稅。」三月,隴州刺史薛逵奏修築定成關工畢。四月丁酉,敕:「常平
 義倉斛斗,每年檢勘,實水旱災處,錄事參軍先勘人戶多少,支給先貧下戶,富戶不在支給之限。」以禮部尚書、諸道鹽鐵轉運等使裴休可本官同平章事。



 五月,敕:「天下軍府有兵馬處,宜選會兵法能弓馬等人充教練使,每年合教習時,常令教習。仍於其時申兵部。」御史臺奏:「諸色刑獄有關連朝官者,尚書省四品已上、諸司三品已上官,宜先奏取進止。如取諸色官狀,即申中書取裁。」從之。



 秋七月丙辰,前淮南節度使、金紫光祿大夫、檢校
 尚書左僕射、兼揚州大都督府長史、御史大夫、上柱國、贊皇郡開國公、食邑一千五百戶李玨卒,贈司空。敕犯贓人平贓,據律以當時物價上旬估。請取所犯之處,其月內上旬時估平之。從之。檢校司空、太子少師、上柱國、範陽郡開國公、食邑二千戶盧鈞可太原尹、北都留守、河東節度使。



 九月,敕起居郎轉官月限,宜以二十個月。



 七年春正月壬辰,金紫光祿大夫、守太子少傅分司、上柱國、晉陵郡開國公、食邑二千戶歸融卒,贈右僕射。宗
 正卿李文會貶睦州刺史。四月,以御史大夫鄭朗為中書侍郎、同平章事。



 五月,左衛率府倉曹張戣集律令格式條件相類一千二百五十條,分一百二十一門,號曰《刑法統類》,上之。七月,以正議大夫、尚書左丞、上柱國、賜金魚袋崔璪為刑部尚書,以銀青光祿大夫、行兵部侍郎、知制誥、充翰林學士蘇滌為尚書左丞,權知戶部侍郎崔璵可權知兵部侍郎。十月,尚書左僕射、門下侍郎、平章事、太清宮使、弘文館大學士崔鉉進《續會要》四
 十卷,修撰官楊紹復、崔彖、薛逢、鄭言等,賜物有差。



 八年春正月,陜州黃河清。



 二月,南蠻進犀牛,詔還之。



 三月,敕以旱詔使疏決系囚。宰相監修國史魏謩修成《文宗實錄》四十卷,上之,修史官給事中盧耽、太常少卿蔣偕、司勛員外郎王渢、右補闕盧吉,頒賜銀器、錦彩有差。以山南東道節度使、檢校戶部尚書、襄州刺史、上柱國、酒泉縣開國子、食邑三百戶李景讓為吏部尚書。



 五月,以中書舍人、翰林學士韋澳為京兆尹;以戶部侍郎、翰
 林學士承旨、上柱國、武功縣開國子、食邑三百戶蘇滌檢校兵部尚書,兼江陵尹、御史大夫,充荊南節度管內觀察處置等使。七月,銀青光祿大夫、守門下侍郎、同平章事魏謩兼戶部尚書。



 八月,以司農卿鄭助為檢校左散騎常侍,兼夏州刺史、御史大夫、上柱國、滎陽縣開國男、食邑三百戶、夏綏銀宥等州節度營田觀察處置押蕃落安撫平夏黨項等使。九年春正月辛巳,銀青光祿大夫、秘書監、許昌縣開國
 男陳商卒,贈工部尚書。



 二月,中書侍郎,兼禮部尚書、同平章事裴休檢校吏部尚書,兼汴州刺史、御史大夫,充宣武軍節度使、汴宋亳潁觀察處置等使。



 三月,試宏詞舉人,漏匯題目,為御史臺所劾,侍郎裴諗改國子祭酒,郎中周敬復罰兩月俸料,考試官刑部郎中唐枝出為處州刺史,監察御史馮顓罰一月俸料。其登科十人並落下。其吏部東銓委右丞盧懿權判。以吏部侍郎鄭涯檢校禮部尚書,兼定州刺史、御史大夫,充義武軍節度、
 易定州觀察處置、北平軍等使。御史臺據正月八日禮部貢院捉到明經黃續之、趙弘成、全質等三人偽造堂印、堂帖,兼黃續之偽著緋衫,將偽帖入貢院,令與舉人虞蒸、胡簡、黨贊等三人及第,許得錢一千六百貫文。據勘黃續之等罪款,具招造偽,所許錢未曾入手,便事敗。奉敕並準法處死。主司以自獲奸人,並放。七月,以河東節度使、檢校司空、太原尹、北都留守、上柱國、範陽郡開國公、食邑三千戶盧鈞守尚書右僕射。



 八月,以門下侍
 郎、守尚書右僕射、監修國史、博陵縣開國伯、食邑一千戶崔鉉檢校司空、同平章事,兼揚州大都督府長史,充淮南節度副大使、知節度使事。宣宗宴餞,賦詩以賜之。九月,昭義節度使、檢校禮部尚書,兼潞州大都督府長史、御史大夫、上柱國、賜紫金魚袋鄭涓檢校刑部尚書、太原尹、北都留守、御史大夫,充河東節度、管內觀察處置等使。十一月,以河南尹劉彖檢校工部尚書、汴州刺史、兼御史大夫、充宣武軍節度、宋亳汴潁觀察處置等
 使。以中書舍人鄭顥為禮部侍郎。



 十年春正月乙巳,以正議大夫、華州刺史、潼關防禦、鎮國軍等使、上柱國、隴西縣開國男、食邑三百戶、賜紫金魚袋李訥檢校左散騎常侍,兼越州刺史、御史大夫、浙江東道都團練觀察等使。



 三月,中書門下奏:「據禮部貢院見置科目,《開元禮》、《三禮》、《三傳》、《三史》、學究、道舉、明算、童子等九科,近年取人頗濫,曾無實藝可採,徒添入仕之門。須議條疏,俾精事業。臣已於延英面論,伏奉聖旨,將
 文字來者。其前件九科,臣等商量,望起大中十年,權停三年,滿後,至時赴科試者,令有司據所舉人先進名,令中書舍人重覆問過。中有本業稍通,堪備朝廷顧問,即作等第進名,候敕處分。如有事業荒蕪,不合送名數者,考官即議朝責。其童子近日諸道所薦送者,多年齒已過,偽稱童子,考其所業,又是常流。起今日後,望今天下州府薦送童子,並須實年十一、十二已下,仍須精熟一經,問皆全通,兼自能書寫者。如違制條,本道長吏亦議懲
 法。」從之。四月癸丑,以刑部郎中盧搏為廬州刺史,以給事中、渤海郡開國公、食邑二千戶高少逸檢校禮部尚書、華州刺史、潼關防禦、鎮國軍等使。



 六月,以兵部郎中裴夷直為蘇州刺史。



 六月,以兵部郎中裴夷直為蘇州刺史。



 九月,以中書舍人杜審權知禮部貢舉。十月,邠寧慶節度使、檢校禮部尚書、邠州刺史、上柱國、賜紫金魚袋諴為檢校兵部尚書、潞州大都府長史、御史大夫,充昭義節度副大使、知節度使、潞邢洺等州觀察使。桂管觀察使令狐定卒,贈禮部尚書。



 十一年春正月,以銀青光祿大夫、守吏部尚書、上柱國、酒泉縣開國男、食邑三百戶李景讓為御史大夫;以朝請大夫、守御史中丞,兼尚書右丞、上柱國、賜紫金魚袋夏侯孜為戶部侍郎、判戶部事;以朝散大夫、守京兆尹、上柱國、扶風縣開國男、食邑三百戶、賜紫金魚袋韋澳檢校工部尚書、孟州刺史、御史大夫,充河陽三城節度、孟懷澤觀處置等使。先是,車駕將幸華清宮,兩省官進狀論奏,詔曰:「朕以驪山近宮,真聖廟貌,未嘗修謁,自
 謂闕然。今屬陽和氣清,中外事簡,聽政之暇,或議一行。蓋崇禮敬之心,非以逸游為事。雖申敕命,兼慮勞人。卿等職備禁闈,志勤奉上,援據前古,列狀上章,載陳懇到之詞,深睹盡忠之節。已允來請,所奏咸知。」以劍南西川節度副大使、知節度事、管內觀察處置統押近界諸蠻及西山八國雲南安撫等使、特進、檢校司徒、同中書門下平章事、兼成都尹、上柱國、太原郡開國公、食邑二千戶白敏中以本官兼江陵尹,充荊南節度、管內觀察處置等
 使。



 二月,以夏綏銀宥節度使、通議大夫、檢校左散騎常侍、夏州刺史、御史大夫、上柱國、滎陽縣開國男、食邑三百戶、賜紫金魚袋鄭助為檢校工部尚書、邠州刺史,充邠寧慶節度、管內營田觀察處置,兼充慶州南路救援、鹽州及當道沿路鎮寨糧料等使;以右金吾衛將軍田在賓校右散騎常侍,兼夏州刺史,代鄭助為夏、綏、銀、宥節度等使。以荊南節度使、銀青光祿大夫、檢校兵部尚書、兼江陵尹、御史大夫、上柱國、武功郡開國男、食邑
 三百戶蘇滌為太常卿。以銀青光祿大夫、守門下侍郎、兼戶部尚書、同平章事、監修國史、上柱國魏謩檢校戶部尚書、同平章事,兼成都尹,充劍南西川節度副大使、知節度事。以太中大夫、守工部尚書、上柱國、賜紫金魚袋崔慎由為中書侍郎、同平章事。以成德軍節度、鎮冀深趙觀察處置等使、起復雲麾將軍、守左金吾衛大將軍同正、檢校兵部尚書、鎮州大都督府長史王紹鼎為銀青光祿大夫、檢校尚書右僕射,餘官如故。以通議大
 夫、守中書門下侍郎、兼禮部尚書、同平章事、集賢殿大學士、上柱國、賜紫金魚袋鄭朗可監修國史。太中大夫、守工部尚書、同平章事、上柱國、賜紫金魚袋崔慎由可集賢院大學士。



 三月,起復朝請大夫、深州刺史、御史大夫,兼成德軍節度判官王紹懿可檢校左散騎常侍、鎮府左司馬、知府事,充成德軍節度副使,兼充都知兵馬使。以成德軍中軍兵馬使、銀青光祿大夫、檢校太子賓客、兼監察御史、上柱國王景胤可本官、深州刺史、本州
 團練守捉使。檢校左散騎常侍、右神武大交軍知軍事王紹孚可落起復,依前右神武大將軍。紹懿、紹孚,鎮州王紹鼎之弟也。景胤,紹鼎子也。以朝請大夫、檢校刑部尚書、華州刺史、上柱國、酂縣開國男、食邑三百戶、賜紫金魚袋蕭俶為太子賓客,分司東都。四月,以職方郎中、知制誥裴坦為中書舍人。以朝議大夫、權知京兆尹崔郢為濮王傅,分司東都,以決殺府吏也;以江西觀察使、洪州刺史、御史中丞、上柱國、賜紫金魚袋張毅夫為京
 兆尹。以鳳翔節度使、正議大夫、檢校戶部尚書,兼鳳翔尹、上柱國、襲晉國公、食邑三千戶、襲實封一百五十戶裴識可許州刺史,充忠武軍節度、陳許蔡觀察等使;以吏部侍郎盧懿檢校工部部尚書,兼鳳翔尹、御史大夫、鳳翔隴右節度使;以中書舍人鄭憲為洪州刺史、御史中丞、江南西道都團練觀察處置待使,仍賜紫金魚袋。以安南宣慰使、右千牛衛大將軍宋涯為安南都護、御史中丞、本管經略招討處置等使。以幽州節度使張允伸
 弟允中為荊州刺史,允千檀州刺史,允辛安塞軍使,允舉納降軍使,並兼御史中丞。以前邠寧節度使、朝議大夫、檢校工部尚書、邠州刺史、上柱國、賜紫金魚柳憙可檢校禮部尚書、河南尹。



 五月,以職方郎中李玄為壽州刺史。



 六月,以朔方靈武定遠等城節度使、朝散大夫、檢校左散騎常侍、靈州大都督府長史、上柱國、賜紫金魚袋劉潼為鄭州刺史,馳驛赴任,以給邊兵糧不及時也。以安南都護宋涯為容州刺史、容管經略招討處置
 等使。制皇第三男灌封衛王,第十一男澭封廣王。以朝散大夫、守尚書兵部侍郎、判度支、上柱國、彭城縣開國男、食邑三百戶、賜紫金魚袋蕭鄴本官同平章事、判度支。以右監門將軍、知內府省事、清河公崔巨淙為淮南監軍。以特進、檢校司空、兼太子太傅分司東都、上柱國、扶風郡開國公、食邑二千戶杜忭本官判東都尚書省、兼御史大夫,充東都留守、東畿汝都防禦使。七月,以飛龍使、宮闈局令王歸長守內侍省內常侍,知省事,充內樞
 密使。責授邠州員外司馬張直方為右驍衛大將軍。



 八月,成德軍節度使、檢校尚書右僕射王紹鼎卒,贈司空,賻布帛三百段。以皇子昭王汭為開府儀同三司、守鎮州大都督府長史、成德軍節度、鎮冀深趙觀察等大使;以成德軍節度副使、都知兵馬使、左司馬、知府事、御史中丞王紹懿為成德軍副使留後。以義武軍節度、易定觀察等使、檢校禮部尚書、定州刺史、上柱國、滎陽縣開國男、食邑三百戶鄭涯檢校戶部尚書、汴州刺史、上柱
 國,充宣武軍節度副大使、知節度事、宋亳觀察、亳州太清宮等使;以四鎮北庭行軍、涇原渭武節度使、銀青光祿大夫、檢校右散騎常侍、涇州刺史、御史大夫、上柱國、範陽縣開國男、食邑三百戶盧簡求可檢校工部尚書、定州刺史、義武節度使、易定觀察、北平軍等使;以鹽州防禦押蕃落諸軍防秋都知兵馬使、度支烏池榷稅使、檢校右散騎常侍、鹽州刺史、上柱國、賜紫金魚袋陸耽代簡求為涇原節度使。以翰林學士、朝散大夫、
 中書舍人、賜紫金魚袋曹確權知河南尹。汝州防禦使令狐緒有善政,郡人詣闕請立德政碑頌。緒以弟綯在中書,上表乞寢,從之。以太常卿蘇滌為兵部尚書、權知吏部銓事,以銀青光祿大夫、守散騎常侍、上柱國、渤海郡開國伯、食邑七百戶封敖為太常卿。是月,熒惑犯東井。



 九月,以秦州刺史李承勛為朝散大夫、檢校工部尚書、涇州刺史,充四鎮北庭涇原渭武節度等使;以禮部郎中楊知溫充翰林學士;以中散大夫、尚書禮部侍郎、
 上柱國、賜紫金魚袋杜審權為陜州大都督府長史、兼御史大夫、陜虢都防禦觀察處置等使;以銀青光祿大夫、檢校司空、兼太子太師、上柱國、範陽郡開國公、食邑二千戶盧鈞為檢校司空、同中書門下平章事、興元尹,充山南西道節度等使。右補闕陳嘏、左拾遺王譜、右拾遺薛傑上疏諫遣中使往羅浮山迎軒轅先生。詔曰:「朕以萬機事繁,躬親庶務,訪聞羅浮山處士軒轅集,善能攝生,年齡亦壽,乃遣使迎之,或冀有少保理也。朕每
 觀前史,見秦皇、漢武為方士所惑,常以之為誡。卿等位當論列,職在諫司,閱示來章,深納誠意。」仍謂崔慎由曰:「為吾言於諫官,雖少翁、欒大復生,不能相惑。如聞軒轅生高士,欲與之一言耳。」宰相鄭朗累月請告,三章求免。是月乙未,彗出於房初度,長三尺。十月,制通議大夫、守中書侍郎、禮部尚書、同平章事、監修國史、上柱國、賜紫金魚袋鄭朗可檢校尚書右僕射,兼太子少師。以山南西道節度使、中散大夫、檢禮校部尚書、興元尹、上柱國、
 賜紫金魚袋蔣系權知刑部尚書,宰相崔慎由兼修國史,蕭鄴兼集賢殿大學士。以華州刺史高少逸為左散騎常侍,以蘇州刺史裴夷直為華州刺史、潼關防禦、鎮國軍等使,以太常少卿崔鈞為蘇州刺史。入回鶻冊禮使、衛尉少卿王端章貶賀州司馬,副使國子《禮記》博士李尋為郴州司馬,判官河南府士曹李寂永州司馬。端章等出塞,黑車子阻路而回故也。以成德軍觀察留後、御史中丞、賜紫金魚袋王紹懿檢校工部尚書,兼鎮州
 大都督府長史、御史大夫,充成德軍節度、鎮冀深趙觀察等使。以中書舍人李籓權知禮部貢院。



 十一月,太子少師鄭朗卒,贈司空。銀青光祿大夫、檢校尚書左僕射、兼太子太保、充右羽林統軍、御史大夫、上柱國、滎陽縣開國男、食邑三百戶鄭光卒,輟朝三日,贈司徒,仍令百官奉慰。上之元舅也。宰相崔慎由為中書侍郎兼禮部尚書,尚書蕭鄴兼工部尚書,餘並如故。十二月,以昭義軍節度使、朝議大夫、檢校工部尚書、上柱國、平陰縣開
 國男、食邑三百戶畢諴為太原尹、北都留守、河東節度使;朝議大夫、檢校禮部尚書、兼太原尹、北都留守、上柱國、賜紫金魚袋劉彖為尚書戶部侍郎、判度支。以翰林學士承旨、通議大夫、守尚書戶部侍郎、知制誥、上護軍、賜紫金魚袋蔣伸為兵部侍郎,充職。以金紫光祿大夫、守太子少保分司東都、上柱國、河東縣開國男、食邑五百戶裴休檢校戶部尚書兼潞州大都督府長史、昭義軍節度副大使、知節度事、潞磁邢洺觀察等使。以正議
 大夫、行尚書兵部侍郎、上柱國、河東縣開國男、食邑三百戶、賜紫金魚袋柳仲郢本官兼御史大夫,充諸道鹽鐵轉運使。以正議大夫、檢校戶部尚書、兼太子賓客、上柱國、賜紫金魚袋孔溫業本官分司東都,以病請告故也。禮部郎中楊右溫本官知制誥,充翰林學士。以幽州中軍使、檢校國子祭酒、幽府左司馬、知府事、御史中丞張簡真檢校右散騎常侍,允伸之子也。以中散大夫、權知刑部尚書、上柱國、賜紫金魚袋蔣系檢校戶部尚書、
 鳳翔尹、御史大夫、鳳翔隴右節度觀察處置等使。是歲,舒州吳塘堰有眾禽成巢,闊七尺,高七丈,而水禽、山鳥、鷹隼、燕雀之類,無不馴狎。又有鳥人面緣毛,爪喙皆紺色,其聲曰「甘」,人呼為「甘蟲」。



 十二年春正月,以晉陽令鄭液為通州刺史。羅浮山人軒轅集至京師,上召入禁中,謂曰:「先生遐壽而長生可致乎?」曰:「徹聲色,去滋味,哀樂如一,德施周給,自然與天地合德,日月齊明,何必別求長生也。」留之月餘,堅求還
 山。以前鄉貢進士於琮為秘書省校書郎,尋尚皇女廣德公主,改銀青光祿大夫、守右拾遺、駙馬都尉。以安南本管經略招討處置使、朝散大夫、檢校左散騎常侍、安南都護、御史大夫、賜紫金魚袋李弘甫為宗正卿。以中大夫、守京兆尹、上柱國、賜紫金魚袋張毅夫為鄂州刺史、御史大夫、鄂岳蘄黃申等州都團練觀察使。以太中大夫、福州刺史、御史中丞、上柱國、賜紫金魚袋楊發檢校右散騎常侍、廣州刺史、御史大夫,充嶺南東道
 節度觀察處置等使。以朝散大夫、守康王傅分司東都、上柱國、襲魏郡開國公、食邑二千戶、賜紫金魚袋王式為安南都護、兼御史中丞,充安南本管經略招討處置等使。以朝請大夫、前守太了賓客分司東都、上柱國、酂縣開國男、食邑三百戶、賜紫金魚袋蕭俶守太子少保分司。以朝請大夫、檢校左散騎常侍、右金吾將軍、充右街使、上柱國、襲太原郡開國公、食邑二千戶、賜紫金魚袋王鎮為檢校左散騎常侍、使持節都督福州諸軍
 事,福州刺史、御史大夫,充福建等州都團練觀察處置等使。以翰林學士、朝議郎、守尚書司勛郎中、知制誥、賜緋魚袋孔溫裕為中書舍人,充職。以右驍衛上將軍李正源守大內皇城留守。以朝議大夫、守尚書戶部侍郎、判度支、上柱國、賜紫金魚袋劉彖本官同平章事,依前判度支。以太中大夫、守中書侍郎、兼禮部尚書、同平章事、監修國史、上柱國、賜紫金魚袋崔慎由檢校禮部尚書、梓州刺史、御史大夫、劍南東川節度副大使、知節
 度事,代韋有翼;以有翼為吏部侍郎。



 二月,以前邕管經略招討處置使、朝議郎、邕州刺史、御史中丞、賜紫金魚袋段文楚為昭武校尉、右金吾衛將軍;以朝議郎、守中書舍人、權知禮部貢舉、上柱國、賜緋魚袋李籓為尚書戶部侍郎。以朝散大夫、守工部尚書、同平章事、充集賢殿大學士、上柱國、彭城縣開國男、食邑三百戶、賜紫金魚袋蕭鄴為監修國史。以朝議大夫、守戶部侍郎、同平章事、判度支、主柱國、賜紫金魚袋劉彖可充集賢院學
 士。以渤海國王弟權知國務大虔晃為銀青光祿大夫、檢校秘書監、忽汗州都督,冊為渤海國王。以兵部侍郎柳仲郢為刑部尚書。以朝議大夫、守尚書戶部侍郎、判戶部事、上柱國、賜紫金魚袋夏侯孜為兵部侍郎,充諸道鹽鐵轉運使;以朝請大夫、權知刑部侍郎、賜紫金魚袋杜勝為戶部侍郎、判戶部事。光祿大夫、守左領軍衛大將軍分司東都、上柱國、會稽縣開國公、食邑一千五百戶康季榮可檢校尚書右僕射,兼右衛上將軍分
 司。貶前利州刺史杜倉為賀州司戶,蔡州刺史李叢邵州司馬。以工部郎中、知制誥於德孫,庫部郎中、知制誥苗恪,並可中書舍人,依前翰林學士。以前右金吾衛將軍鄭漢璋,前鴻臚少卿鄭漢卿,並起復授本官,國舅光之子也。以銀青光祿大夫、行給事中、駙馬都尉衛洙為工部侍郎,前濮王傅分司皇甫權為康王傅分司。以庫部員外郎、史館修撰李渙為長安令。閏二月,以司農少卿盧籍為代州刺史,前江陵少尹杜惲為司農少卿。以
 河東馬步都虞候段威為朔州刺史,充天寧軍使,兼興唐軍沙陀三部落防遏都知兵馬使。五月,以兵部侍郎、鹽鐵轉運使夏侯孜本官同平章事。



 六月,南蠻攻安南府。



 八月,洪州賊毛合、宣州賊康全大攻掠郡縣,詔兩浙兵討平之。十二月,太子少保魏掞卒,贈司徒。



 十三年春正月,以虢陜觀察使杜審權為戶部侍郎、判戶部事。



 三月,宰相蕭鄴罷知政事,守吏部尚書。四月,以翰林學士承旨、兵部侍郎、知制誥蔣伸本官同平章事。



 五月,上不豫,月餘不能視朝。



 八月七日,宣遺詔立鄆王為皇太子,勾當軍國事。是日,崩於大明宮,聖壽五十。詔門下侍郎、平章事令狐綯攝塚宰。群臣上謚曰聖武獻文孝皇帝,廟號宣宗。十四年二月,葬於貞陵。



 史臣曰:臣嘗聞黎老言大中故事,獻文皇帝器識深遠,久歷艱難,備知人間疾苦。自寶歷巳來,中人擅權,事多假借,京師豪右,大擾窮民。洎大中臨馭,一之日權豪斂跡,二之日奸臣畏法,三之日閽寺讋氣。由是刑政不濫,
 賢能效用,百揆四岳,穆若清風,十餘年間,頌聲載路。上宮中衣浣濯之衣,常膳不過數器,非母後侑膳,輒不舉樂,歲或小饑,憂形於色。雖左右近習,未嘗見怠惰之容。與群臣言,儼然煦接,如待賓僚,或有所陳聞,虛襟聽納。舊時人主所行,黃門先以龍腦、鬱金藉地,上悉命去之。宮人有疾,醫視之,既瘳,即袖金賜之,誡曰:「勿令敕使知,謂予私於侍者。」其恭儉好善如此。季年風毒,召羅浮山人軒轅集,訪以治國治身之要,其伎術詭異之道,未嘗
 措言。集亦有道之士也。十三年春,堅求還山。上曰:「先生少留一年,候於羅浮山別創一道館。」集無留意,上曰:「先生舍我亟去,國有災乎?朕有天下,竟得幾年?」集取筆寫「四十」字,而十字挑上,乃十四年也。興替有數,其若是乎!而帝道皇猷,始終無缺,雖漢文、景不足過也。惜乎簡藉遺落,舊事十無三四,吮墨揮翰,有所慊然。



 贊曰:李之英主,實惟獻文。粃粺盡去,淑慝斯分。河、隴歸地,朔漠消氛。到今遺老,歌詠明君。



\end{pinyinscope}