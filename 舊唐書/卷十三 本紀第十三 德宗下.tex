\article{卷十三 本紀第十三 德宗下}

\begin{pinyinscope}

 貞元
 四年春正月庚戌朔,上御丹鳳樓,制曰:「朕以菲薄,托於王公之上,恭承天地之序,虔奉祖宗之訓,遐想至理,思臻大和。而誠不感物,化不柔遠,聲教猶鬱,征賦仍繁。頃者務於安人,不憚屈己,與西蕃結好,申以齊盟。而戎心不厭,背義虧信,劫脅士庶,屢犯封疆,元元何辜,皆朕之失。乃者輦轂之下,兇狂結構,上帝垂祐,悉自伏誅,刑以止殺,諒非獲已。今三陽布和,萬物資始,思與群公兆庶,惟新政理,宜敷在宥之澤,以覃作解之恩。可大赦天下,大闢已下罪咸赦除之。」是日質明,含元殿前階基欄檻壞損三十餘間,壓死衛士十餘人。京師地震,辛亥又震,壬子又震。壬戌,以左龍武大將軍王棲曜為麟州刺史、鄜州刺史、鄜坊丹延節度使。丁卯,京師地震,戊辰又震,庚午又震。以
 宣武軍行營節度使劉昌為涇州刺史、四鎮北庭行軍涇原等州節度使。癸酉,京師地震。甲戌,以華州潼關節度使李元諒兼隴右節度使、臨洮軍使。乙亥,地
 震,
 金、房尤甚,江溢山裂,廬舍多壞,居人露處。陳留雨木如大指,長寸餘,有孔通中,下而植於地,凡十里許。辛巳,李泌以京官俸薄,請取中外給用除陌錢,及闕官俸外一分職田、額內官俸,及刺史執刀司馬軍事等錢,令戶
 部別庫貯之,以給京官月俸,令御史中丞竇參專掌之。歲得錢三百萬貫,謂之戶部別處錢,朝臣歲支不過五十萬,常有二百餘萬以資國用。壬午,地震,甲申又震,乙酉又震,丙申又震。甲辰,太僕郊牛生犢六足,又豕生兩首四足。築延喜門北復道屬永春門。涇原劉昌復築連雲堡。戊辰,鹿入京師市門。甲寅,地震。宴群臣於麟德殿,設《九部樂》,內出舞馬,上賦詩一章,群臣屬和。己未,地震。丁卯,有司條奏省官,其左右常侍、太子賓客請依前置四
 員,從之。庚午,地震。詔涇原劉昌於平涼會盟所收被害將士骸骨,葬於淺水原,為二塚,立石堠志之,題曰懷忠塚。辛未,地震。中書省梧樹有鵲以泥為巢。癸巳,以太子左庶子暢悅為桂管觀察使。改左右射生為左右神威軍。福建兵亂,逐觀察使吳詵。丁未,隴右李元諒築良原城。丁巳,右龍武統軍張伯儀卒。辛酉,以吉州刺史張庭為安南都護、本管經略使。升鄆州為大都督府。壬戌,加置諫議大夫八員,分中書四員為右,門下四員為左。檢
 校左庶子蕭復卒於饒州。丙寅,地震,丁卯,又震。月犯歲星。辛未,太子賓客吳湊為福建觀察使。乙亥,熒惑、歲、鎮三星聚營室,凡二十日。是月,吐蕃寇涇、邠、寧、慶、鄜等州,焚彭原縣,邊將閉城自固。賊驅人畜三萬計,凡二旬而退。吐蕃入寇以秋冬,今盛暑而來,華人陷蕃者道之也。



 六月丁丑,鄂岳觀察使李竦卒。乙酉,以尚書左丞杜祐為陜州長史、陜虢觀察使。征夏縣處士先除著作郎陽城為諫議大夫。城以褐衣詣闕,上賜之章服而後召。乙
 丑,桂管都防禦觀察使暢悅卒。乙未,以諫議大夫何士干為鄂、岳、沔、蘄、黃等州都團練觀察使。乙亥,封皇子、皇弟邕王謜等七人為王,兼卿、監、祭酒等官。癸卯,熒惑退行入羽林。



 秋七月庚戌,以左金吾將軍張獻甫為邠寧節度使;陳許防禦兵馬使韓全義檢校工部尚書,充長武城及諸軍行營節度使。癸丑,邠寧軍因韓游瑰受代,憚張獻甫之嚴,乘其無帥,縱兵大掠,仍脅監軍楊明義奏請範希朝為帥。都虞候楊朝晟斬其亂首二百餘人,
 方定。朝命仍以希朝副獻甫。己未,奚、室韋寇振武軍。壬戌,詔以太尉、中書令、西平郡王李晟長子願為銀青光錄大夫、太子賓客,賜勛上柱國,與晟門並列戟。乙丑,以前撫州刺史戴叔倫為容州刺史、兼御史中丞、本管經略使。丁丑,以兵部尚書崔漢衡為晉州刺史、晉慈隰觀察使。壬申,詔:「嗣王、郡王朝會,班位在本官班之上。左右庶子準令在左右丞侍郎之下、諸司四品之上,今在少卿之下,非也,宜改之。」乙亥,以蘇州刺史孫晟為桂州刺
 史、桂管觀察使。荊河自陜州至河陰,水色如墨,流入汴口,至汴州,一宿而復。又汴鄭管內烏皆入田緒、李納之境,銜柴為城,方十餘里,高二三尺,緒、納惡而去之,信宿復如之,烏口皆流血。



 八月,以權判吏部侍郎吉中孚為中書舍人。乙酉,檢校司徒、兼太子太師、汧國公李勉薨。甲午,京師地震,其聲如雷。九月丙午,詔:「比者卿士內外,左右朕躬,朝夕公門,勤勞庶務。今方隅無事,烝庶小康,其正月晦日、三月三日、九月九日三節日,宜任文武百
 僚選勝地追賞為樂。每節宰相及常參官共賜錢五百貫文,翰林學士一百貫文,左右神威、神策等軍每廂共賜錢五百貫文,金吾、英武、威遠諸衛將軍共賜錢二百貫文,客省奏事共賜錢一百貫文,委度支每節前五日支付,永為常式。」戊申,晉慈隰觀察使崔漢衡加都防禦使名。癸丑,賜百僚宴於曲江亭,仍作《重陽賜宴詩》六韻賜之。群臣畢和,上品其優劣,以劉太真、李紓為上等,鮑防、於邵為次等,張濛、殷亮等二十人又次之。唯李晟、馬
 燧、李泌三宰相之詩不加優劣。庚申,吐蕃寇邠、寧、坊等州。



 冬十月,詔中書門下選常參官曾為牧宰有理行者以名聞。宰臣奏於頎、董晉等十二人前任有治跡,詔頎等於左右丞各言政要,左右丞條奏,上乃御宣政殿親試其言而後用之。丙戌,以右神策將軍李長榮為河陽三城懷州團練使,仍賜名元。戊子,回紇公主將妾媵六十餘人、馬二千匹來迎咸安公主,命刑部尚書關播送公主歸蕃。十二月辛巳,少府監李觀卒。



 五年春正月壬辰朔。乙卯,詔:「四序嘉辰,歷代增置,漢崇上巳,晉紀重陽,或說禳除,雖因舊俗,與眾共樂,咸合當時。朕以春方發生,候及仲月,勾萌畢達,天地和同,俾其昭蘇,宜助暢茂。自今宜以二月一日為中和節,以代正月晦日,備三令節數,內外官司休假一日。」宰臣李泌請中和節日令百官進農書,司農獻穜AL之種,王公戚里上春服,士庶以刀尺相問遺,村社作中和酒,祭勾芒以祈年穀,從之。丁卯,右散騎常侍宜城縣子柳渾卒。



 二月
 己丑,貶京兆尹鄭叔則為永州長史。戊戌,以滄景留後程懷直為滄景觀察使。庚子,以大理卿董晉為門下侍郎、同中書門下平章事;以御史中丞竇參為中書侍郎、平章事兼轉運使;以戶部尚書,依前度支轉運副使。



 三月甲辰,中書侍郎、同平章事李泌卒。乙卯,以兵部郎中姚南仲為御史中丞,司農卿薛玨為京兆尹,以大理卿李速為黔州刺史、黔州觀察使。癸亥,以資州刺史龐復為安南都護、本管經略使。丙寅,貶禮
 部侍郎劉太真為信州刺史。以給事中杜黃裳為河南尹。戊辰,詔以李懷光外孫燕八八為左衛率府胄曹參軍,賜姓名曰李承緒,仍賜錢千貫,俾自營居業。



 夏四月乙未,以太子少師蕭昕為工部尚書,致仕,給半祿、料,永為常式。初致仕官只給半祿,無料,上加之以待老臣,半料自昕始也。五月戊辰,宋州麥一莖九岐者百餘本。六月乙未,以光祿卿裴腆為桂管觀察使。



 秋七月,以嗣滕王湛然為太子賓客、入回紇使。八月辛未,以同州刺史
 竇覦為戶部侍郎。九月壬戌,詔褚遂良已下至李晟等二十七,圖形於凌煙閣,以繼國初功臣之像。



 冬十月丙午,西川韋皋奏與東蠻合力大破吐蕃於故巂州,擒其將臧遮遮。自是吐蕃挫銳,竟復巂州。庚午,百僚請復徽號,不允。己丑,易定節度使、檢校司空、平章事張孝忠以擅出兵襲蔚州,降檢校司空為左僕射。桂管觀察、御史中丞孫晟卒。癸巳,以戶部侍郎竇覦為揚州長史、兼御史大夫、淮南節度使。十二月庚午,回紇汨咄錄長
 壽天親昆伽可汗卒。辛未,以淮南節度使杜亞為東都留守、畿汝州都防禦使,兵部侍郎裴住為河南尹,司農卿李翼為陜虢都防禦觀察使。壬申,以陜虢觀察使杜祐檢校禮部尚書,兼揚州長史、淮南節度使。六年春正月戊辰朔。戊申,大雪。



 二月戊辰朔,百僚會宴於曲江亭,上賦《中和節群臣賜宴》七韻。是日,百僚進《兆人本業》三卷,司農獻黍粟各一斗。岐州無憂王寺有佛指骨寸餘,先是取來禁中供養,乙亥,詔送還本寺。珍戌,
 以中書舍人陸贄權兵部侍郎。甲午,以吏部侍郎劉滋為吏部尚書。丁酉,王武俊守隸州將趙鎬以郡歸李納,武俊怒,以兵攻之。



 三月庚子,百僚宴於曲江亭,上賦《上巳詩》一篇賜之。壬寅,渾瑊自河中來朝。戊午,牂柯蠻來朝。甲子,以旱,日色如血,無光。



 夏四月甲辰,大風雷。閏月庚申,太白、辰星聚東井。戊午,始雨。五月丙寅朔,上御紫宸受朝。上以是月一陰生,臣子道長,父子必以是朔面焉,故取朔日受朝。壬午,以寧州刺史範希朝為單于大
 都護、麟勝節度使。是夏,淮南、浙東西、福建等道旱,井泉多涸,人渴乏,疫死者眾。



 秋七月丙寅,淮南節度使竇覦卒。癸酉,復呼親王母曰太妃,公主母曰太儀。八月丁未,工部尚書致仕鮑防卒。九月乙丑,收諸道進奏院官印,悉毀之。己卯,詔:「十一月八日,有事於南郊太廟,行從官吏將士等,一切並令自備食物。其諸司先無公廚者,以本司闕職物充。其王府官,度支量給廩物。其儀仗禮物,並仰御史撙節處分。」冬十月己亥,文武百僚京城道俗
 抗表請徽號,上曰:「朕以春夏亢旱,粟麥不登,朕精誠祈禱,獲降甘雨,既致豐穰,告謝郊廟。朕倘因禮祀而受徽號,是有為為之。勿煩固請也。」辛亥,回紇吊祭使、鴻臚卿郭鋒復命,回紇遣達北勒梅錄將軍來,告九姓回紇登里邏沒密施俱錄忠貞昆伽可汗之喪。



 十一月庚午,日南至,上親祀昊天上帝於郊丘。禮畢還宮,御丹鳳樓宣赦,見禁囚徒減罪一等,立仗將士及諸軍兵,賜十作萬段匹。今後刺史、縣令以四考為限。青州李納以棣州還
 王武俊,並其兵士三千。是歲,吐蕃陷北庭都護府,節度使楊襲古奔西州。回紇大相頡乾迦斯紿襲古,請合軍收復北庭,乃殺襲古,安西因是阻絕,唯西州猶固守之。回紇亦為吐蕃所逼,取浮圖川,乃遷部落羊馬於牙帳之南以避之。



 七年春正月壬戌朔。己巳,襄王僙薨。庚辰,以湖南觀察使裴胄為洪州刺史、江西觀察使,以常州刺史李衡為
 御史中丞。



 二月己巳,涇原帥劉昌復築平涼城。城去故原州一百五十里,本原之屬縣,地當御戎之沖要。昌復浹辰而功畢,分兵戍之,邊患稍弭。庚子,侍中渾瑊自河中來朝。



 三月辛酉,陳許節度使曲環奏請權停當道冗官,待一二後,民力稍給,則復之。壬戌,左龍武統軍戴休顏卒。甲子,涇原節度使劉昌築胡谷堡,改名彰義堡。堡在平涼西三十五里,亦御戎之要。壬申,詔:「頃來賜衣,文彩不常,非制也。朕今思
 之,宜有定制,節度使宜以鶻銜綬帶,觀察使宜以雁銜威儀。」威儀,瑞草也。關輔牛疫死,十亡五六。上遣中使以諸道兩稅錢賣牛,散給畿民無牛者。辛巳,詔神威、神策六軍將士自相訟,軍司推劾;與百姓相訟,委府縣推劾;小事移牒,大事奏取處分,軍司、府縣不得相侵。癸未,義武軍節度使、檢校司空、平章事張孝忠卒。



 夏四月庚子,太子少師致仕蕭昕卒。汴州獻白烏。戊午,詔:「仲夏之時,萬物敷暢,陽德方茂,陰事始承。昔者觀於法象,因天地
 交會之序,為父子相見之儀,沿習成風,古今不易。王者制事,感動因人,酌其情而用中,順其俗以為禮。咸覿之義,既行於父子之間;資事之情,豈隔君臣之際。申恩卿士,自我為初。起今年五月朔,御正殿,召見文武百官,外官因朝奏,咸聽就列。仍編禮式,以為常典。」己未,安南首領杜英翰叛,攻都護府,都護高正平憂死。五月庚申朔,上御宣政殿見百官,從新制也。辛未,置柔遠軍於安南都護府。甲申,端王遇薨。許州獻白烏。戊子,以衡州刺
 史齊映為桂管觀察使。六月庚子朔。乙巳,太常卿崔縱卒。



 秋七月庚午,以信州刺史鄭叔則為福建觀察使。癸酉,上幸章敬寺,賦詩九韻,皇太子與群臣畢和,題之寺壁。戊寅,以邕王謜為義武軍節度使、易定觀察等大使,以定州刺史張升雲為留後。庚辰,以虔州刺史趙昌為安南都護、經略招討使。八月己丑,以翰林學士歸從敬為工部尚書。甲午,給事中鄭瑜為中書舍人。丙申,貶宗正卿李翰為雅王傅;翰林學士陸贄為兵部侍郎。罷學
 士。庚戌,夏州奏開延化渠,引烏水入庫狄澤,溉田二百頃。九月庚申,兵部尚書致仕馬炫卒。



 冬十月癸丑,每御延英令諸司官長二人奏本司事。尋又敕常參官每一日二人引對,訪以政事,謂之巡對。



 十一月乙丑,令常參官趨朝入閣,不得奔走。周親已下喪者禁慘服,朝會須服本色綾袍金玉帶。丁酉,以前福建觀察使吳湊為陜州長史、陜虢觀察使。是冬無雪。



 八年春正月丙辰朔。癸酉,罷桂管經略招討使。



 二月丁
 亥,許州人李狗兒持杖入含元殿,擊欄檻,又格擒者。誅之。庚子,京師雨土。己酉,吏部尚書李紓卒。乙丑,山南東道節度使、檢校戶部尚書嗣曹王皋薨。庚午,宣武軍節度使、司徒、平章事劉玄佐卒。癸酉,劍南西川節度使韋皋奏請,有當道閑員官吏,增其俸祿,從之。己亥,以湖南觀察使李衡為洪州刺史、江西觀察使。襄州軍亂,掠府庫民財殆盡,都將徐誠斬其亂首楊清潭,方止。丙子,以荊南節度使樊澤為襄州刺史、山南東道節度使,以江
 西觀察使裴胄為江陵尹、荊南節度使。以戶部尚書班宏判度支,戶部侍郎張滂為諸道鹽鐵轉運使。己卯,以陜虢觀察使吳溱為汴州刺史、宣武軍節度、汴宋等州觀察使。辛巳,以同州刺史姚南仲為陜虢觀察使。壬午,以左庶子李充為京兆尹,以蘇州刺史齊抗為潭州刺史、湖南觀察使。



 夏四月丁丑,貶左金吾大將軍嗣虢王則之為昭州司馬,左諫議大夫、知制誥吳通玄為泉州司馬,給事中竇申道州司馬。戊子,以雅王傅李翰為金
 吾衛大將軍。翰前為竇參所惡貶官,至是參敗,上遽召翰,口授將軍,便令金吾仗上事,翌日除書方下。庚寅,以汴州長史劉士寧為汴州刺史、宣武軍節度使。時吳湊行次汜水,聞其有變而還。乙未,貶中書侍郎、平章事竇參為郴州別駕,竇申景州司戶。尋杖殺申。諸竇皆貶。以尚書左丞趙憬、兵部侍郎陸贄為中書侍郎、同中書門下平章事。丁酉,韋皋請十二而稅,以給官吏,從之。丙午,以東都、河南、淮南、江南、嶺南、山南東道兩稅等物,令戶
 部侍郎張滂主之;以河內、河東、劍南、山南西道等財,戶部尚書、判度支班宏主之。一遵大歷故事,如劉晏、韓滉分掌焉。給事中韋夏卿左遷常州刺史,坐交諸竇也。是月,吐蕃寇靈州。



 五月乙卯朔,上御宣政殿受朝。丙辰,初增稅京兆青苗畝三錢,以給掌閑廣騎。戊午,以光祿少卿崔穆為黔州觀察使。己未,大風,吹壞廬舍、門闕。丙寅,以大理卿王翃為福建觀察使。戊辰,初令授臺省官者各具舉主於授官詔。先是郎官缺,左右丞舉之,御史缺,
 大夫、中丞舉之,詔書不具所舉。及趙憬、陸贄為相,建議郎官不宜專於左右丞,宜令尚書、丞、郎各舉其可,詔書具所舉官名,御史亦如之,異日考殿最以舉主能否。從之。癸酉,平盧淄青節度使、檢校司徒、平章事李納卒。癸未,前太僕少卿劉士干有罪賜死,劉玄佐養子也。六月,吐蕃寇涇州。



 秋七月甲寅朔,戶部尚書、判度支蕭國公班宏卒。以桂管觀察使齊映為洪州刺史、江西觀察使;以翰林學士歸崇敬為兵部尚書,致仕。辛巳,大雨。八月
 乙丑,以天下水災,分命朝臣宣撫賑貸。河南、河北、山南、江淮凡四十餘州大水,漂溺死者二萬餘人。辛卯,以青州刺史李師古為鄆州大都督府長史、平盧淄青等州節度觀察海運陸運、押新羅渤海兩蕃等使。丁未,詔以歲兇罷九日賜宴。九月丁巳,韋皋攻吐蕃之維州,獲蕃將論莽熱以獻。貶太子賓客於邵江州別駕,尋卒。乙亥,以太子賓客薛玨為嶺南節度使。



 冬十月己亥,追封故皇弟遐為均王。庚戌,復命金吾置門籍。



 十一月壬子朔,
 日有蝕之。己巳,貶左庶子姜公輔泉州別駕。嚴震奏破吐蕃於芳州。壬申,詔自今死刑勿決,先杖。十二月庚寅,詔賜遭水縣乏絕戶米三十萬石。丁未,以給事中李巽為潭州刺史、湖南觀察使。閏月癸酉,門下省奏:「郵驛條式,應給紙券。除門下外,諸使諸州不得給往還券,至所詣州府納之,別給俾還朝。常參官在外除授及分司假寧往來,並給券。」從之。甲戌,牂柯、室韋、靺鞨皆使朝貢。



 九年春正月庚辰朔,朝賀畢,上賦《退朝觀仗歸營詩》。乙
 酉,劍南東川節度使王叔邕來朝。癸卯,初稅茶,歲得錢四十萬貫,從鹽鐵使張滂所奏。茶之有稅,自此始也。甲辰,禁賣劍銅器。天下有銅山,任人採取,其銅官買,除鑄鏡外,不得鑄造。



 二月庚戌朔。先是宰相以三節賜宴,府縣有供帳之弊,請以宴錢分給,各令諸司選勝宴會,從之。是日中和節宰相宴於曲江亭,諸司隨便,自是分宴焉。易定留守張升雲為義武軍節度使。辛酉,詔復築鹽州城。貞元三年,城為吐蕃所毀,自是塞外無堡障,犬戎
 入寇,既城之後,邊患息焉。



 三月己亥,以駕部郎中、知制誥張式為虢州刺史。



 夏四月辛酉,地震,有聲如雷,河中、關輔尤甚,壞城壁廬舍,地裂水湧。五月庚申,廢諸州府執刀。甲辰,以義成軍節度使、檢校右僕射賈耽為左僕射、同中書門下平章事,以尚書左丞盧邁本官同平章事。以鄭州刺史李融為滑州刺史、義成軍節度使。乙巳,韋皋奏,遣軍出西山,破吐蕃峨和城、定廉城、通鶴軍,凡平堡五十餘所。是日以蕃俘器仗來獻。丙戌,以門下侍
 郎、平章事董晉為禮部尚書,罷知政事。甲寅,加韋皋檢校右僕射,以司農少卿裴延齡為戶部侍郎、判度支。庚申,以給事中李衡為戶部侍郎、諸道鹽鐵轉運使。



 秋七月己未,敕縣令以四考為限,無替者宜至五考。庚子,以信州刺史孫公器為邕管經略使。故事,宰相秉筆決事,每人十日一易。至是賈耽、趙憬、陸贄、盧邁同平章政事,百僚有所關白,更相讓而不言。始詔令旬日秉筆,後詔每日更秉筆。劍南西山羌女國王湯立志、哥鄰王董臥
 庭、白狗王羅陀匆、弱水王董避和、逋租王弟鄧告知、南水王侄尚悉曩等六國君王,自來朝貢。六國初附吐蕃,韋皋擊西山討吐蕃,故六蠻內附,各授官秩遣之。八月庚戌,太尉、中書令、西平郡王李晟薨,贈太師,廢朝五日。己巳,皇太子長男廣陵王淳納妃郭氏。九月己卯,罷九日宴,以太師晟喪也。



 冬十月己酉,侍中馬燧對於延英。燧足疾,詔令不拜,行僕於地,命宦者扶持之。止謂之曰:「前日卿與太尉晟俱來,今公獨至。」因歔欷泣下。及燧退,
 上送及階。癸酉,環王國獻犀牛,上令見於太廟。



 十一月乙酉,日南至,上親郊圓丘。是日還宮,御丹鳳樓,制曰:「朕以寡德,祗膺大寶,勵精理道,十有五年。夙夜惟寅,罔敢自逸,小大之務,莫不祗勤。皇靈懷顧,宗社垂祐,年穀豐阜,荒服會同,遠至邇安,中外咸若。永惟多祐,實荷玄休。是用虔奉禮章,躬薦郊廟,克展因心之敬,獲申報三之誠。慶感滋深,悚惕惟勵,大福所賜,豈獨在予,思與萬方,均其惠澤,可大赦天下。」辛卯,華州潼關鎮國軍、隴右節
 度使李元諒卒於良原,以其部將阿史那敘統元諒之眾,戍良原。壬寅,河南尹、東都留守裴住卒。甲辰制以冬薦官,宜令尚書丞、郎於都堂訪以理術,試時務狀,考其通否及歷任考課事跡,定為三等,並舉主姓名。仍令御史一人為監試。如授官後政事能否,委御史臺、觀察使以聞,而殿最舉主。十二月丙午朔,制:「今後使府判官、副使、行軍已下,使罷後,如是檢校試五品以上官,不合集於吏部選,任準罷使郎官、御史例,冬季聞奏。」丙辰,宣武軍
 亂,逐節度使劉士寧。壬戌,以通王諶為宣武軍節度使,以宣武軍節度副使李萬榮為汴州刺史、宣武軍節度、汴宋等州觀察留後。朔方靈鹽節度副大使、太子少師、檢校左僕射、餘姚郡王杜希全卒。



 十年春正月乙亥朔。乙酉,以虔王諒為朔方靈鹽豐節度大使,以朔方等道行軍司馬李欒為留後。壬辰,南詔異牟尋大破吐蕃於神川,使來獻捷。己亥,昭義節度使、檢校司空平章事李抱真請降官,乃授檢校左僕射。時
 抱真病,巫祝言宜降爵,故有是請。



 二月丙午,以瀛州刺史劉澭為秦州刺史、隴右經略軍使、理普潤縣,仍以普潤軍為名。乙卯,以給事中齊抗為河南尹。乙丑,義成軍節度使、鄭滑觀察使李融卒。丁卯,詔:「君臣之際,義莫重焉,每聞薨殂,良深悼惻。應文武朝臣薨卒者,其月俸、料宜全給,仍更準本官一月俸料,以為賻贈。」三月乙亥,黃霧四塞,日無光。以華州刺史李復為滑州刺史、義成軍節度使。滄州程懷直來朝,賜安業坊宅,妓一人,復令還
 鎮。庚辰,南詔異牟尋攻收吐蕃鐵橋已東城壘一十六,擒其王五人,降其民眾十萬口。壬申,以同州刺史盧征為華州刺史、潼關防禦、鎮國軍等使。辛丑,以延州刺史李如暹所部蕃落賜名曰安塞軍,以如暹為軍使。



 夏四月戊辰,地震,癸丑復震。恆州奏見巨人跡。以雲南告捷使高細龍為左武衛將軍。是月,太白晝見。有大鳥飛集宮中,食雜骨。是春霖雨,罕有晴日。



 六月壬寅朔。昭義軍節度使、檢校左僕射、同中書門下平章事、義陽王李
 抱真卒,詔以其將王延貴權知昭義軍事。癸丑,以祠部郎中袁滋兼御史中丞,為冊南詔使。甲寅,以辰州刺史房孺復為容管經略使。丙寅,韋皋奏西山峨和城擊破吐蕃城柵,斬首二千八百級。庚午,度支使裴延齡兼靈、鹽等州鹽池井榷使。辛未晦,有水鳥集於左藏庫,是夜暴雨,大風折木。



 秋七月壬申朔,以邕朔,以邕王謜為昭義軍節度使,以昭義軍押衙王延貴為潞府左司馬,充昭義節度留後,賜名虔休。抱真別將權知洺州事元誼不悅虔休
 為留後,據洺州叛,陰結田緒。庚辰,賜南詔異牟尋金印銀窠,其文曰「貞元冊南詔印」。先是,吐蕃以金印授南詔,韋皋因其舊而請之。汴州軍亂,攻節度留後李萬榮,不勝而潰,萬榮悉捕斬其孥。己亥,前汴州節度使劉士寧宜於郴州安置,飲州守鎮黃少卿叛,攻邕管經略使孫公器,又陷欽、橫、潯、貴等州。吐蕃大將論乞髯、陽沒藏、悉諾硉以其家內附,授歸義將軍。因置四品已下武官,以授四夷歸附者,仍定懷化大將軍已下俸錢。九月辛未
 朔,以袁州刺史董鎮為邕管經略使。戊子,賜百僚九日宴,上賦詩賜之。辛卯。南詔獻鐸槊、浪人劍、吐蕃印八紐。戊戌,定州張升雲改名茂昭。



 冬十月癸卯,御宣政殿,試賢良方正、能直言極諫等舉人。壬戌,刑部尚書劉滋卒。



 十一月乙酉,諸道鹽鐵轉運使張滂為衛尉卿,以浙西觀察使王緯為諸道鹽鐵轉運使。庚寅,秘書監致仕穆寧卒。十二月庚子朔。壬戌,貶中書侍郎、平章事陸贄為太子賓客。



 十一年春正月庚午朔。乙亥,嶺南節度使薛玨卒。乙未,以秘書少監王礎為黔中經略觀察使,衛尉少卿武少儀為邕管經略使。丙申,以邕管經略使王鍔為廣州刺史、嶺南節度使。



 二月癸卯,以衢州刺史李若初為福建觀察使。乙巳,岫渤海大欽茂之子嵩為渤海郡王、忽汗州都督。乙卯,於涇州彰信堡置潘原縣。甲子,九姓回紇骨咄祿昆伽奉誠可汗卒。



 三月庚午,司徒兼侍中馬燧以疾請罷侍中,不許。辛未,賜宰臣兩省供奉官宴於曲
 江亭。乙丑,以吏部侍郎鄭瑜為河南、淮南水陸轉運使。丙申,諸州準例薦隱居丘園不求聞達蔡廣成等九人,各授試官,令給公乘,到京日量才敘用。



 夏四月,旱。壬戌,貶太子賓客陸贄為忠州別駕,京兆尹李充信州長史,衛尉卿張滂汀州長史。癸亥,以兵部侍郎韓皋為京兆尹。甲子,賜南詔敕書,始列中書三官奉宣行,復舊制也。丙寅,幽州劉濟奏大破奚王啜剌等六萬餘眾。



 五月丁卯朔。庚午,命有司慮囚,旱故也。丁丑,以宣武留後李萬
 榮為汴州刺史、宣武節度副使、知節度事。以昭義軍節度留後王虔休為潞州大都督府長史、昭義軍節度副大使、知節度事、管內度支營田、潞澤磁邢洺觀察使。又以朔方留後李欒為靈州大都督府長史、朔方靈鹽豐夏四州受降定遠城天德軍節度副大使、知節度事、管內度支營田觀察押蕃落等使。甲申,以河東節度使、檢校工部尚書、太原尹李自良卒。庚寅,遣使冊九姓回紇騰里羅羽錄沒密施合胡六骨咄祿毗伽懷信可汗。癸巳,
 以通王諶為河東節度使,以河東行軍司馬李悅為河東節度營田觀察留後、北都副留守。甲午,初鑄河東監軍印。監軍有印,自王定遠始也。六月,河陽獻白烏。甲辰,晉慈隰觀察使崔漢衡卒。癸丑,以絳州刺史姚齊梧為晉慈隰都防禦觀察使。



 秋七月丙寅朔,右諫議大夫陽城為國子司業。河東監軍王定遠配流崖州,坐專殺也。辛卯,江西觀察使、洪州刺史齊映卒。八月辛亥,司徒兼侍中、北平郡王馬燧薨,贈太傅。丙辰,以楚州刺史路寰
 為洪州刺史、江西觀察使。閏月己丑,國子司業裴澄表上《乘輿月令》十一卷,《禮典》十二卷。九月己卯,賜宰臣兩省供奉官宴於曲江,賦詩六韻賜之。丁巳,加韋皋統押近界諸蠻及西山八國、雲南安撫等使。滄州大將程懷信逐其帥程懷直。



 冬十月丁丑,以虔王諒為橫海軍節度大使,以兵馬使程懷信為留後。



 十一月丙申,日南至,不受朝賀,以司徒馬燧葬也。辛丑,太常定馬燧謚曰「景武」,上曰:「景,太祖謚,改莊武可也。」己酉,潭州獻赤烏。十二
 月戊辰,上獵苑中,戎多殺,止行三驅之禮,勞士而還。



 十二年春正月甲午朔。庚子,元誼、李文通率洺州兵五千、民五萬家東奔田緒。壬子,以前滄州節度使程懷直為左龍武統軍。乙丑,成德軍節度使、檢校司徒、兼侍中渾瑊兼中書令;興元節度使嚴震、魏博田緒、西川韋皋並加檢校左右僕射、中書門下平章事。於是方鎮皆敘進兼官。上制《貞元廣利藥方》五百八十六首,頒降天下。



 三月癸巳。甲午,韋皋奏收降蠻七千戶,得吐蕃所
 賜金字告身五十五片。乙巳,以戶部侍郎裴延齡為戶部尚書。戊申,以兵部尚書董晉充東都留守、判東都尚書省、東畿汝州都防禦使。四月壬戌朔。戊辰,左右十軍使奏:去年冬車駕幸諸營,欲於銀臺亭子門外立碑以紀聖跡。從之。庚午、魏情節度使、度支營田觀察使,檢校左僕射、平章事、魏州長史、駙馬都尉、雁門郡王田緒卒。庚辰,上降誕日,命沙門、道士加文儒官討論三教,上大悅。



 五月辛卯朔。丙申,邠寧節度使張獻甫卒。甲辰,以邠
 寧都虞候楊朝晟為邠州刺史、邠寧、慶節度使。銀夏節度使韓潭讓新授禮部尚書,乞雪崔寧,許其家收葬。丁已,駙馬郭暖、王士平、曖弟煦暄,坐代宗忌辰飲宴,貶官歸第。六月壬戌,故驩州司戶竇參,許其家收葬。乙丑,初置左右護軍中尉監、中護軍監,以授宦官。以左右神策軍使竇文場、霍仙鳴為左右神策護軍中尉監,以左右神威軍使張尚進、焦希望為左右神威中護軍監。辛巳,宣歙觀察使、宣州刺史劉贊卒。



 七月乙未,以東都留守、
 兵部尚書董晉檢校左僕射、同中書目門下平章事、汴州刺史、宣武軍節度使、宋亳潁觀察使。時李萬榮病,萬榮子乃自署為兵馬使,軍人又逐乃,汴州亂,故命董晉帥之。以太子賓客為東都留守、判東都尚書省事、東畿汝都防禦使。是日,汴州節度使李萬榮卒。八月辛未朔,日有蝕之。己巳,以前魏博節度副使田季安為魏州長史、魏博節度觀察等使。庚午,增修望仙門,廣夾城、十王宅、六王宅。癸酉,以虢州刺史崔衍為宣、歙、池觀察使,
 以乞髯子湯忠義為歸德將軍。丙子,以汝州刺史陸長源為宣武行軍司馬。丙戌,門下侍郎、平章事趙憬薨。九月甲午,以河東行軍司馬李景略為豐州刺史、天德軍豐州西受降城都防禦使。丙午,戶部尚書、判度支裴延齡卒。庚戌,幸魚藻宮,即日還內。壬子,吐蕃寇慶州。



 冬十月壬戌,詔以京畿旱,放租稅。甲戌,諫議大夫崔損、給事中趙宗儒並同中書門下平章事,俱賜金紫。以少府監崔穆為晉州刺史、晉慈隰觀察使。



 十一月辛卯,昭義王虔
 休造《誕聖樂曲》以獻。十二月己未,大雪平地二尺,竹柏多死。環王國所獻犀牛,甚珍愛之,是冬亦死。上著《刑政箴》一首。癸未,回紇、南詔、劍南西山國女國王並朝賀。



 十三年春正月戊子朔。庚寅,太子少師致仕關播卒。壬寅,吐蕃贊普遣使修好,塞上以聞,上以犬戎負約,不受其使。東都尚書省火。



 二月丁巳,賜宰臣、兩省供奉官宴於曲江亭。乙亥,度支郎中蘇弁為戶部侍郎、判度支,兵部郎中王紹判戶部。



 三月戊子,造會慶亭於麟德殿前。
 乙巳,以福建都團練使李若初為明州刺史、浙東觀察使,以婺州刺史柳冕為福建觀察使。



 夏四月壬戌,上幸興慶宮龍堂祈雨。乙丑,大雪。庚午,義成軍節度使、鄭滑觀察營田、檢校左僕射、滑州刺史李復卒。己卯,以大理卿于頔為陜州長史、陜虢觀察使。庚辰,以陜虢都防禦觀察轉運等使姚南仲為滑州刺史、義成軍節度、鄭滑觀察使。五月丙戌朔,韋皋收復巂州,畫圖來上。壬子,以庫部郎中、翰林學士鄭餘慶為工部侍郎、知吏部選事。
 六月己卯朔,以衡州刺史陳云為邕管經略使。辛巳,引龍首渠水自通化門入,至太清宮前。壬午,韋皋奏於巂州破吐蕃,生擒大籠官七人,馬畜器械不可勝紀。



 秋七月丙戌,宰相盧邁請告累月,四表避相位,是日,命宰臣問疾於盧邁私第。己丑,右神策中尉霍仙鳴病,賜馬十匹,令於諸寺齋僧。壬辰,浚湖渠、魚藻池,深五尺。乙未,地震。甲辰,以兵部郎中、判戶部王紹為戶部侍郎。乙丑,詔今後嗣王薨葬,所司並供鹵簿,永為常式。



 八月丁巳,詔
 京兆尹韓皋修昆明池石炭、賀蘭兩堰兼湖渠。壬午,容管經略使房孺復卒。九月己丑,盧邁懇讓相位,乃授太子賓客。辛卯九日,宴宰臣百官於曲江,上賦詩以賜之。己未,江西觀察使路寰卒。甲辰,升定州為大都督府。以湖南觀察使李巽為江州刺史、江西觀察使,以禮部侍郎呂渭為潭州刺史、湖南觀察使。



 冬十月癸丑朔,以前滁州刺史房濟為容管經略使。丙辰,黔中觀察使奏:「溪州人戶訴,被前刺史魏從琚於兩稅外,每年加進硃砂
 一千斤、水銀二百馱,戶民疾苦,請停。」從之。淮西吳少誠擅開淘刁河、汝河,詔使不能禁。癸酉,宰相賈耽以疾避相位,不允。丁丑,徐泗節度使張建封來朝,上嘉之,次日於延英召對。癸巳,贈太傅馬燧祔廟,命所司供少牢祭,仍給鹵簿,從宅至廟。十二月庚辰,右龍武統軍韓游瑰卒。



 十四年春正月壬午朔。庚寅,詔諸道州府應貞元八年至十一年兩稅及榷酒錢,在百姓腹內者,總五百六十
 萬七千貫,並除放。甲午,敕:「比來朝官或相過從,金吾皆上聞。其間如是親故,或嘗同僚,伏臘歲時,須有還往,亦人倫常禮,今後不須奏聞。」因張建封奏議也。二月壬子朔。戊午,上御麟德殿,宴文武百僚,初奏《破陳樂》,遍奏《九部樂》,及宮中歌舞妓十數人列於庭。先是上制《中和樂舞曲》,是日奏之,日晏方罷。比詔二月一日中和節宴,以雨雪,改用此日。上又賦《中春麟德殿宴群臣詩》八韻,群臣頒賜有差。乙亥,賜光蔡節度曰彰義軍。



 三月丙申,右
 神策行營節度、鳳翔隴右觀察使、檢校尚書右僕射、鳳翔尹邢君牙卒。以右神策將軍張昌為鳳翔尹、右神策行營節度、鳳翔隴右節度使,仍改名敬則。



 夏四月乙丑,以左諫議大夫、平章事崔損為修奉八陵使。先是昭陵寢殿為火所焚,至是獻、昭、乾、定、泰五陵各造屋三百八十間,橋、元、建三陵據闕補造。五月庚辰朔。甲午,前東都留守、東畿汝都防禦使、檢校吏部尚書杜亞卒。丙午,戶部侍郎、判度支蘇弁為太子詹事。上特召度支郎中
 於䪹於延英,兼御史中丞,賜金紫,令判度支。閏月庚申,以左神策行營節度韓全義為夏州刺史,兼鹽、夏、綏、銀節度使,以代韓潭。甲子,貶太子詹事蘇弁為汀州司戶,兄贊善大夫袞為永州司戶,兄贊善大夫袞為永州司戶,前京兆府士曹冕為信州司戶。



 六月癸卯,太子賓客盧邁卒。乙巳,以旱儉,出太倉粟賑貸。



 秋七月,以吉州刺史杜春為邕管經略使。乙卯,貶京兆尹韓皋為撫州司馬。召右金吾將軍吳湊於延英,面授京兆尹,即令入府府視事。是夏,熱甚。壬申,以給事中、
 中書門下平章事趙宗儒為太子左庶子,以左諫議大夫、平章事崔損為門下侍郎、平章事,以工部侍郎鄭餘慶為中書侍郎、同平章事。左神策護軍中尉霍仙鳴卒。丁丑,以宦者第五守亮代仙鳴為中尉。己卯,左右神策置統軍,品秩奉給視六軍統軍例。甲午,崔損修奉八陵寢宮畢,群臣於宣政殿行稱賀。浙西觀察使、潤州刺史王緯卒。



 九月丁未朔。己酉,山南東道節度使、檢校尚書右僕射、襄州刺史樊澤卒。乙卯,以同州刺史崔宗為
 陜州大都督府長史、陜虢觀察水陸轉運使,以浙東觀察李若初為潤州刺史、浙西觀察使及諸道鹽鐵轉運使,又以常州刺史裴肅為越州刺史、浙東觀察使。丙辰,以陜虢觀察使于頔為襄州刺史、山南東道節度使。丁卯,杞王倕薨。以太常卿杜確為同州刺史、本州防禦、長春宮使。癸酉,諫議大夫田登奏言:「兵部武舉人持弓挾矢,數千百人入皇城,恐非所宜。」上聞之瞿然,乃命停武舉。



 冬十月癸酉,以歲兇穀貴,擊太倉粟三十萬石,開場
 糶以惠民。庚子,夏州韓全義奏破吐蕃鹽州。



 十一月己未,韋皋進《開西南蠻事狀》十卷,敘開復南詔之由。十二月戊子,太子少師致仕郢國公韋倫卒。癸酉,出東都含嘉倉粟七萬石,開場糶以惠河南饑民。己亥,南詔異牟尋遣使賀正旦。明州鎮將慄鍠殺刺史盧云。



 十五年春正月丙午朔。甲寅,雅王逸薨。甲戌,浙西觀察使李若初卒。



 二月,罷中和節宴會,年兇故也。丁丑,宣武軍節度使、檢校左僕射、平章事、汴州刺史董晉卒。乙酉,
 以行軍司馬陸長源檢校禮部尚書、汴州刺史、御史大夫、宣武軍節度度支營田、汴宋亳潁觀察等使。以常州刺史李錡為潤州刺史、浙西觀察使及諸道鹽鐵轉運使。是日,汴州軍亂,殺陸長源及節度判官孟叔度、丘穎,軍人臠而食之。監軍俱文珍以宋州刺史劉逸準久為汴之大將,以書招之,俾靜亂。乙丑,以宋州刺史劉逸準檢校工部尚書、兼汴州刺史、宣武軍節度使,仍賜名全諒。乙未,裴肅奏於臺州擒慄鍠以獻,斬於獨柳樹。癸卯,
 罷三月群臣宴賞,歲饑也。出太倉粟十八萬石,糶於京畿諸縣。



 三月甲寅,吳少誠寇唐州,殺監軍邵國朝,掠居民千餘而去。丁巳,以度支郎中、兼中丞於䪹為戶部侍郎,依前判度支。戊午,昭義軍節度使、檢校工部尚書王虔休卒。戊辰,以河陽三城節度使李元為潞州長史、昭義軍節度、澤潞磁邢洺觀察使,以河陽節度押衙衡濟為懷州刺史、河陽三城懷州節度使。辛未,太子少師致仕於頎卒。壬申,於易州滿城縣置永清軍。癸酉,令江淮
 歲運米二百萬石。雖有是命,然歲運不過四十萬石。



 四月丁丑,以久旱,令陰陽人法術祈雨。壬午,內侍省加置內給事二員。癸未,以安州刺史伊慎為安黃節度營田觀察使。庚寅,應京城內外諸軍縣鎮職員官,見共五萬八千二百七十一人,宜令每人賜粟一石。乙未,特進、兵部尚書歸崇敬卒。五月甲辰朔。戊辰,宗正卿嗣吳王巘薨。



 六月己卯,黔中觀察使、御史中丞王礎卒。癸巳,山南西道節度使、檢校尚書左僕射、平章事嚴震卒。



 秋七月
 乙巳,以興州刺史、興元都虞候嚴礪為興元尹、兼御史大夫、山南西道節度度支營田觀察等使。丙午,故唐安公主賜謚曰莊穆。公主賜謚,自唐安始也。丁未,以王礎卒,廢朝一日。觀察使卒廢朝,自礎始也。戊午,貶諫議大夫苗拯萬州刺史,左拾遺李繁播州參軍,以私議除拜嚴礪不當而無章疏,而偽言累上疏故也。鄭、滑大水。八月壬申朔。丙申,陳許節度使、檢校尚書右僕射、許州刺史曲環卒。丁酉,以洋州刺史韋士宗為黔中觀察使。丙午,
 以陳許兵馬使、前陳州刺史上官涚為許州刺史、陳許節度使。吳少誠謀逆漸甚,陷臨潁,進圍許州。庚戌,宣武軍節度使、檢校工部尚書、汴州刺史劉全諒卒。丙辰,年:「吳少誠非次擢用,授以節旄,秩居端揆之榮,任總列城之重。期申報效,奉我典章,而秉心匪彞,自底不類。兇狡成性,扇構多端,擅動甲兵,暴越封圵。壽州茶園,輒縱凌奪;唐州詔使,潛構殺傷。干犯國章,罪在無赦。朕以王者之德,在乎好生;人君之體,務於含垢。寧屈已以宥罪,不
 殘人以興師。以上稽宗社之威,外抑忠賢之請,庶有悛革,尚議優容。幸鄰境之喪,逞貪亂之志,焚略縣邑,殘暴吾民。朕尤冀知非,為之忍恥,及頒恩命,未許出師。至乃攻逼許州,肆其蠆毒,恣行殺戮,流害黎蒸。惡稔禍盈,人神同棄。興言致討,實悼于懷。宜令諸道各出師徒,掎角齊進。吳少誠在身官爵,並宜削奪。」己巳,自今中和、重陽二節,每節只禁屠一日。辛酉,以大理評事宣武軍都知兵馬使韓弘檢校工部尚書,兼汴州刺史、御史大夫、宣
 武軍節度使。



 冬十月己丑,邕王謜薨。吏部侍郎奚陟卒。



 十一月乙巳,冬至,罷朝會,兵興也。壬子,襄州于頔奏,於朗山破淮西賊三千人。十二月庚午,朔方等道副元帥、河中絳州節度使、檢校司徒、兼奉朔中書令渾瑊薨。乙未,戰淮西賊於小溵河,王師不利,諸軍自潰。丁酉,以同州刺史杜確為河中尹、河中絳州觀察使。



 十六年春正月庚子朔。乙巳,恆冀、定州、許、河陽四鎮之師與賊戰,皆不利而退。南詔獻《奉聖樂舞曲》,上閱於麟
 德殿前。



 二月己酉,以左神策行營、銀夏節度等使韓全義為蔡州行營招討使,陳許節度使上官涚副之。己丑,左龍武統軍程懷直卒。己酉,華州刺史、潼關防禦、鎮國軍使盧征卒。壬子,以尚書右丞袁滋為華州刺史、潼關防禦、鎮國軍使。



 夏四月丁亥,黔中知宴設吏傅近逐觀察使韋士宗。己丑,以義成軍節度使姚南仲為右僕射。以權知新羅國事金俊邕襲祖開府檢校太尉、雞林州都督、新羅國王。辛卯,以義成軍行軍司馬盧群為滑州
 刺史、兼御史中丞、義成軍節度使。壬申,檢校兵部尚書、京兆尹吳湊卒。



 五月戊戌朔,以雨罷朝。庚戌,韓全義與蔡賊將吳少誠戰于溵水南,王師敗績。徐泗濠節度使、檢校尚書右僕射、徐州刺史張建封卒。壬子,徐州軍亂,不納行軍司馬韋夏卿,迫建封子愔為留後。丙寅,韋士宗卻入黔州。丁卯,以吏部侍郎顧少連為京兆尹。六月丙午,鄆州李師古、淮南杜祐並加同平章事,以祐兼領徐、泗、濠節度,以前虢州參軍張愔起復驍衛將軍,兼徐
 州刺史、御史中丞、本州團練使、知徐州留後。



 秋七月,湖南觀察使呂渭卒。八月癸酉,以河中尹王囗為潭州刺史、湖南觀察使。九月,宥吳少誠。駙馬都尉郭曖卒。義成軍節度使盧群卒。丙午,前太常卿裴鬱卒。戊辰,以左丞李元素為滑州刺史、兼御史大夫、義成軍節度使。庚戌,貶中書侍郎、同中書門下平章事鄭餘慶為郴州司馬,戶部侍郎、判度支於䪹為泉州司戶。以戶部侍郎王紹判度支,以戶閱郎中崔從質為戶部侍郎。癸酉,吳少誠
 賊迫官軍溵水砦下營,韓全義退保陳州,諸軍散還本道,官軍不振。以河南少尹張式為河南尹、水陸轉運使。庚申,以太常卿齊抗為中書侍郎、同平章事。癸亥,以虔王諒為徐州節度使,張愔為留後。



 冬十月辛未,興元嚴礪希監軍旨,誣奏流人通州別駕崔河圖,長流崖州,賜死,人士傷之。吳少誠引兵歸蔡州,上表待罪。戊子,詔雪吳少誠,復其官爵。乙丑,河東節度使、檢校禮部尚書、太原尹、兼御史大夫、北都留守李悅卒。甲午,以河東行軍
 司馬鄭儋檢校工部尚書、太原尹、河東節度使。



 十一月癸卯,泗州、濠州宜隸淮南觀察使。戊申,以太府卿韋渠牟為太常寺卿。十二月戊寅,罷吏部復考判官及禮部別頭貢舉。



 十七年春正月甲午朔。甲寅,韓全義自蔡州行營還,詔歸鎮華州。



 二月癸巳朔,賜群臣宴於曲江亭,上賦《中和節賜宴曲江詩》六韻賜之。丁酉,雨雹。己亥,雨霜。戊申夜,雷震,雨雹。庚戌,大雨兼雹。



 三月乙丑,賜群臣宴於曲
 江亭。己巳,黔中觀察使韋士宗復為三軍所逐。癸酉,衢州刺史鄭式瞻進絹五千匹,銀二千兩,上曰:「式瞻犯贓,已詔御史按問,所進宜付左藏庫。」丁丑,省天下州府別駕、司馬、田曹、參軍;京兆、河南、太原三府外,諸府判司雙曹者省一。



 夏四月丁未,始命駙馬及郡縣主婿無子者,養男不用母廕。辛亥,以諫議大夫裴佶為黔中觀察使。五月壬戌朔,日有蝕之。乙酉,邠寧節度使、檢校工部尚書、邠州刺史楊朝晟卒。丙戌,以工部侍郎趙植為廣州
 刺史、兼御史大夫、嶺南節度使。六月戊戌,以定平鎮兵馬使李朝寀校工部尚書,兼邠州刺史、朔方邠寧慶節度使;以中官楊志廉為右神策護軍中尉。浙西人崔善真詣闕上書,論浙西觀察使李錡罪狀。上覽奏不悅,令械善真送於李錡。為鑿坑待善真,既至,和械推而埋之。由是錡恣橫叛。己酉,以邠寧兵馬使高固為邠州刺史、兼御史大夫、邠寧慶節度使。丁巳,成德軍節度使、恆冀深趙德棣觀察等使、恆州大都督府長史、檢校太尉、
 中書令、瑯邪郡王王武俊薨,贈太師,謚曰忠烈。



 秋七月戊寅,吐蕃寇鹽州。辛巳,以前成德軍節度副使、檢校工部尚書、知恆府事、清河郡王王士真起復授恆州長史,充成德軍節度使。乙酉,太常卿韋渠牟卒。己丑,吐蕃陷麟州,殺刺史郭鋒,毀城壘而去。八月戊午,以河東行軍司馬嚴綬檢校工部尚書、兼太原尹、御史大夫、河東節度使。九月壬戌,韋皋奏大破吐蕃於雅州。戊辰,群臣宴曲江,上賦《九日賜宴曲江亭詩》六韻賜之。丁丑,禮部尚
 書李齊運卒。



 冬十月,加韋皋檢校司徒、中書令,封南康郡王,賞破吐蕃功也。戊午,鹽州刺史杜彥先委城奔慶州。辛未,宰相賈耽上《海內華夷圖》及《古今郡國縣道四夷述》四十卷。甲戌,翰林侍詔戴少平死十六日復生。庚戌,以京兆尹顧少連為吏部尚書,以吏部侍郎韋夏卿為京兆尹。淮南節度使杜祐進《通典》,凡九門,共二百卷。



 十八年春正月戊午朔,大雨雪,罷朝賀。乙丑,驃國王遣使悉利移來朝貢,並獻其國樂十二曲與樂工三十五
 人。乙亥,韋皋以所擒蕃相論莽熱來獻。庚辰,以常州刺史賈全為越州刺史、浙東觀察使。



 二月戊子朔,賜群臣宴於馬璘之山池。



 三月癸未,以劍南東川行軍司馬李康為梓州刺史、兼御史大夫、劍南東川節度使。乙丑,賜群臣宴於馬璘之山池。己巳,以蘄州刺史鄭紳為鄂州刺史、鄂岳蘄沔觀察使。癸酉,以浙東團練副使齊總為衢州刺史,總以橫賦進奉希恩,給事中許孟容封還制書。丙戌,以河中行軍司馬鄭元為河中尹、兼御史大夫、
 河中絳節度使。五月癸亥,以竇群為左拾遺。庚辰,以祠部員外郎裴泰為檢校兵部郎中,充安南都護、本管經略使。六月癸巳,以吏部尚書顧少連為兵部尚書、東都留守、東都畿汝防禦使。前東都留守、檢校禮部尚書王翃卒。



 秋七月庚辰,蔡、申、光三州春水夏旱,賜帛五萬段,米十萬石,鹽三千石。八月壬寅,以邕管經略使徐申為廣州刺史、嶺南節度使。甲辰,以嶺南節度掌書記、試大理評事張正元為邕州刺史、御史中丞、邕管經略使,給事
 中許孟容以非次遷授,封還詔書。丁未,以戶部侍郎、判度支王紹為戶部尚書、判度支。九月乙卯朔,以太常少卿楊憑為潭州刺史、湖南觀察使。賜群臣宴於馬璘山池,上賦《九日賜宴詩》六韻賜之。



 冬十月丁亥,以刑部尚書王鍔為淮南節度副使兼行軍司馬。己酉,鄜坊丹延節度使、檢校禮部尚書王棲耀卒。



 十一月丙辰,以同州刺史劉公濟為鄜州刺史、鄜坊丹延節度使。十二月乙巳,貶大理卿李正臣為衛尉少卿,正臣為御史彈
 劾下獄,不堪其辱而死。戊申,黎州蠻、牂柯使入朝。



 十九年春正月癸丑朔。



 二月壬午朔,賜宴馬璘山池。丁亥,修含元殿。賜安黃節度曰奉義軍。丙申,以桂管留後韋武為桂州刺史、桂管觀察使。己亥,安南經略使裴泰為州將王季元所逐。甲辰,淮南節度使杜祐來朝。



 三月壬子朔,以杜祐檢校司空、同中書門下平章事、太清宮使。以淮南行軍司馬王鍔檢校尚書右僕射,兼揚州大都督府長史、淮南節度使。丁卯,以今年孟夏禘饗,前議
 太祖、懿、獻之位未決,至此禘祭,方正太祖東向之位,已下列序昭穆。其獻祖、懿祖祔於德明、興聖之廟,每禘祫年就本室饗之。乙亥,以司農卿李實為京兆尹。



 夏四月乙未,涇原節度使劉昌奏請移行原州於平涼城,從之。戊戌,百官以祔廟畢,蹈舞稱賀。五月辛亥,荊南節度使、檢校工部尚書、江陵尹裴胄卒。乙未,以荊南行軍司馬裴筠為江陵尹、兼御史大夫、荊南節度使。甲子,四鎮北庭行軍涇原節度使、檢校右僕射、涇州刺史劉昌卒。甲
 戌,以涇原節度留後段祐為涇州刺史、兼御史大夫、四鎮北庭行軍涇原節度使。乙亥,吐蕃遣使論頻熱入朝。甲辰,以陳許行軍司馬劉昌裔檢校工部尚書,兼許州刺史、陳許節度使。自正月至是未雨,分命祈禱山川。



 秋七月戊午,以關輔,罷吏部選、禮部貢舉。己未,中書侍郎、平章事齊抗為太子賓客,病免也。甲戌,雨。乙亥,尚書右僕射姚南仲薨。貸京畿民麥種。八月乙未,大雨霖。



 冬十月乙未,以太子賓客韋夏卿為東都留守、東都畿
 汝都防禦使。閏月丁巳,門下侍郎、同平章事崔損卒。



 十一月戊寅朔,以鹽州兵馬使李興幹為鹽州刺史,許專達於上,不隸夏州。丙午,振、武、麟、勝節度使範希朝來朝。戊午,以振武行軍司馬閻巨源檢校工部尚書,兼單于大都護、振武麟勝節度使。庚申,以太常卿高郢為中書侍郎、同中書門下平章事。壬申,監察御史崔薳入臺近,不練故事,違式入右神策軍。上怒,笞四十,配流崖州。



 二十年春正月丁丑朔。丙申,天德軍防禦團練使、豐州
 刺史李景略卒,以其判官任迪簡代領其任。己亥,以鄜、坊、丹、延節度使劉公濟為工部尚書,以其行軍司馬裴玢代領其任。



 二月丙午朔,罷中和節宴,歲儉也。庚戌,大雷震,雨雹。



 三月甲申,以吐蕃贊普卒,廢朝。己亥,以國子祭酒趙昌為安南都護、御史大夫、本管經略使。



 夏四月辛酉,太子賓客齊抗卒。丙寅,吐蕃使臧河南觀察使論乞冉等五十四人來朝貢。陳許節度賜號忠武軍。五月甲戌朔,罷御宣政殿。乙亥,以史館修撰、秘書監張薦為
 工部侍郎、兼御史大夫,充入吐蕃吊祭使。七月癸酉朔,大雨雹。辛卯,福建觀察使柳冕奏置萬安監牧於泉州界,置群牧五,悉索部內馬牛羊近萬頭匹,監史主之。八月戊申,以房州刺史卻士美為黔中觀察使。己未,以昭義兵馬使盧從史為檢校工部尚書,兼潞州長史、昭義軍節度、澤潞磁邢洺觀察使。九月庚辰,賜群臣宴於馬璘山池。



 冬十月甲辰,於景州南皮縣置唐昌軍。辛亥,易定節度使張茂昭來朝。



 十一月丁酉,以監察御史李程、
 秘書正字張聿、藍田縣尉王涯並為翰林學士。十二月,吐蕃、南詔、日本國並遣使朝貢。庚午,以桂管防禦使顏證為桂州刺史、桂管觀察使。



 二十一年春正月辛未朔,御含元殿受朝賀。是日,上不康。丙子,以浙東觀察判官凌準為翰林學士,癸巳,會郡臣於宣政殿,宣遺詔:皇太子宜於柩前即位。是日,上崩於會寧殿,享壽六十四。甲午,遷神柩於太極殿。丙申,發喪,群臣縞素。皇太子即位。永貞元年九月丁卯,群臣上
 謚曰神武孝文,廟號德宗。十月己酉,葬於崇陵,昭德皇后王氏祔焉。



 史臣曰:德宗皇帝初總萬機,勵精治道。思政若渴,視民如傷。凝旒延納於讜言,側席思求於多士。其始也,去無名之費,罷不急之官;出永巷之嬪嬙,放文單之馴象;減太官之膳,誡服玩之奢;解鷹犬而放伶倫,止榷酤而絕貢奉。百神咸秩,五典克從,御正殿而策賢良,輟廷臣而治畿甸。此皆前王之能事,有國之大猷,率是而行,夫何
 敢議。加以天才秀茂,文思雕華。灑翰金鑾,無愧淮南之作;屬辭鉛槧,何慚隴坻之書。文雅中興,夐高前代,《二南》三祖,豈盛於茲。然而王霸跡殊,淳醨代變,揆時而理,斟酌斯難。茍於交喪之秋,輕取鄙夫之論,歷觀近世,靡不敗亡。德宗在籓齒胄之年,曾為統帥;及出震承乾之日,頗負經綸。故從初罷郭令戎權,非次聽楊炎謬計,遂欲混同華裔,束縛奸豪,南行襄漢之誅,北舉恆陽之代。出車雲擾,命將星繁,罄國用不足以餽軍,竭民力未聞於
 破賊。一旦德音掃地,愁嘆連甍,果致五盜僭擬於天王,二硃憑陵於宗社,奉天之窘,可為涕零,罪已之言,補之何益。所賴忠臣戮力,否運再昌。雖知非竟逐於楊炎,而受佞不忘於盧杞。用延賞之私怨,奪李晟之兵符;取延齡之奸謀。罷陸贄之相位,知人則哲,其若是乎!貞元之辰,吾道窮矣。



 贊曰:聰明文思,惟睿作聖。保奸傷善,聽斷不令。御歷三九,適逢天幸。賜宴之辰,徒矜篇詠。



\end{pinyinscope}