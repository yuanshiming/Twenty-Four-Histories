\article{卷三十三 志第十三 歷二}

\begin{pinyinscope}

 ○麟德甲子元歷



 上元甲子,距今大唐麟德元年甲子,歲積二十六萬九千八百八十算。推法:一千三百四十。期實:四十八萬九千四百二十八。旬周:六十。



 ○推氣序術



 置入甲子元積,算距今所求年,以期乘之,為期總。滿法得一為積日,不滿為小餘。旬去積日,不盡為大餘。命大餘起甲子算外,即所求年天正中氣冬至恆日及大小餘。天正建子,律氣所由,故陰陽發斂,皆從其時為自。



 ○求恆次氣術



 因冬至大小餘,加大餘十五、小餘二百九十二、小分六之五。小分滿,從小餘。小餘滿總法之,從大餘一。大餘滿旬周之。以次轉加,而命各得
 其所求。他皆放此。凡氣餘朔大餘為日,小餘為辰也。



 ○求土王



 置清明、小暑、寒露、小寒、大寒小餘,各加大餘十二、小餘二百四十四、小分八。互乘氣小分通之,加八。若滿三十,去,從小餘一。凡分餘相並不同者,互乘而並之。母相乘為法。其並滿法一為全,此即齊同之術。小餘滿總法,從命如前,即各其氣從土王日。



 沒日法:一千七百五十七。



 沒分:十二萬二千三百五十七。



 求沒日術



 以九十乘有沒氣小餘,十五乘小分,從之,以減沒分,餘,法
 得一,為日。不盡,餘,以日數加其氣大餘。去命如前,即其氣內沒日也。小氣餘一千四十已
 上,其氣有沒者,勿推也。沒餘皆盡者為減。求次沒:因前沒加日六十九,餘一千一百四,餘滿從沒日一,因而命之,以氣別日。



 盈朔實:三萬九千九百三十三。



 朒朔實:三萬九千二百二十。



 恆朔實:三萬九千五百七十一。



 推朔端



 列期總,以恆朔實除之為積月,不滿為閏餘。滿總法
 為閏日,不滿為閏辰。以閏日減冬至大餘,辰減小餘,即所求年天正月恆朔大小餘。命大餘以甲子算外,即其日也。天正者,日南至之月也。恆朔者,不朒不盈之常數也。凡減者,小餘不足減,退大餘一,如總法而減之。大餘不足減者,加旬周,乃減之。其須減分奇者,退分餘一,如其法而減,以其在宿度游實不足減者,加在宿過周連餘及奇,乃減之。以天正恆朔小餘加
 閏餘,以減期總,餘為總實。



 求恆弦望術



 因天正恆朔大小餘,加大餘十,小餘五百
 一十二太,凡
 四分一為少,二為半,三為太。滿法者,去命如前,即天正上弦恆日及大小餘。以次轉加,得望下弦及來月朔。以次轉加,去命如前,合得所求。他皆放此。因朔徑求望,加大餘十四,小餘一百二十五分半。因朔徑求下弦,加大餘二十二,小餘一百九十八少。因朔徑次朔,加大餘二十九,小餘七百
 十一。半總:六百七十。辰率:三百三十五。



 檢律候氣日術



 求恆氣初日影泛差術



 見所求氣陟降率,並後氣率,半之,十五而一,為泛末率。又二率相減,餘,十五而一,為總差。前少,以總差減泛末率;前多,以總差加泛末率。加減泛末率訖,即為泛初率。其後氣無同率,因前末率即為泛初率。以總差減初率,餘為泛末率。



 求恆氣初日影定差術



 十五除總差,為別差為限。前少者,以限差加泛初末率;
 前多者,以限差減泛初末率。加減泛初末率訖,即為定初末率,即恆氣初日影定差。



 求次日影差術



 以別定差,前少者加初日影定差,前多者減初日影定差。加減初日影定差訖,即為次日影定差。以次積累歲,即各得所求。每氣皆十五日為限。其有皆以十六除取泛末率及總差別差。



 求恆氣日中影定數術



 置其恆氣小餘,以半總減之,餘為中後分。不足減者反
 減半總,餘為中前分。置前後分,影定差乘之,總法而一,為變差。冬至後,午前以變差減氣影,午後以變差加氣影。夏至後,午前以變差加氣影,午後以變差減氣影。冬至一日,有減無加。夏至一日,有加無減。加減訖,各其恆氣日中定影。



 求次日中影術



 迭以定差陟減降加恆氣日中定影,各得次日中影。後漢及魏宋歷,冬至日中影一丈二尺,夏至一尺五寸,於今並短。各須隨時影校其陟降,及氣日中影應二至率。他
 皆仿此。前求每日中影術,古歷並無,臣等創立斯法也。



 求律呂應日及加時術



 十二律各以其月恆中氣日加時,應列其氣小餘,六乘之,辰率而一,為半總之數,不盡,為辰餘。命時起子算半,為加時所在辰。六乘辰餘,如法得一為初,二為少弱,三為少,四為少強,五為半弱。若在辰半後者,得一為半強,二為太弱,三為太,四為太強,五為辰末。



 求七十二候術



 恆氣日,即初候日也。加其大餘五,小餘九十七,小分十一。三乘氣小分加十一,滿十八從小餘一。滿法,去命如前,即次候日。以次轉加,得末候日。



 求次氣日檢盈虛術



 進綱一十六退紀一十七



 泛差一十一總辰一十二六十並平闕



 秋分後春分前日行速,春分後秋分前日行遲。速為進綱,遲為退紀。若取其數,綱為名;用其時,春分為至。進日
 分前,退日分後。凡用綱紀,皆準此例。



 見所在氣躔差率,並後氣率,半之,總辰乘之,綱紀而一,得氣末率。各以泛差通其綱紀,以同差辰也。又二率相減,餘以總辰乘而紀除之,為總差。辰之綱紀除之,為別差率。前少者,以總差減末率;前多者,以總差加末率。加減訖,皆為其氣初日損益率。前多者,以別差率減;前少者,以別差率加。加減氣初日損益率訖,即次日損益率。亦名每日躔差率。以次加減,得
 每日所求。各累所損益,隨歷定氣損益消息總,各為其日消息數。其後氣無同率,及有數同者,皆因前少,以前末率為初率,加總差為末率,別差漸加初率,為每日率。前多者,總差減初率為末率,別差漸減為日率。其有氣初末計會及綱紀所校多少不葉者,隨其增損調而御之,使際會相準。



 求氣盈朒所入日辰術



 冬夏二至,即以恆氣為定。自外,各以氣下消息數,息減消加其恆氣小餘,滿若不足,進退其日。即其氣朒日辰。亦因別其
 日,命以甲子,得所求。加之為盈氣,減之為朒氣,定其盈朒所在,故日定。凡推日月度及推發斂,皆依定氣推之。若注歷,依恆氣日。



 求定氣恆朔弦望夜半後辰數術



 各置其小餘,三乘,如辰率而一,為夜半後辰數。



 求每日盈朒積術



 各置其氣先後率與盈朒積,乃以先率後率加躔差率,盈朒積加消息總,亦如求消息法,即得每日所入盈朒及先後之數。



 求朔弦望恆日恆所入盈縮數術



 各以總辰乘其所入定氣日,算朒朔弦望夜半後辰數,乃以所入定氣夜半後辰數減之,餘為辰總。其恆朔弦望與定氣同日而辰多者,其朔弦望即在前氣氣末,而辰總時有多於進綱紀通數者,疑入後氣之初也。以乘其氣前多之末率、前少之初率,總辰而一,為總率。凡須相乘有分餘者,母必通全子乘訖報母,異者齊同也。其前多者,辰總減紀乘總差,綱紀而一,為差。並於總率差,辰總乘之,倍總辰除之,以加總率。前少者,辰總再乘別差,總辰自辰乘,倍而除之,以加總
 率,皆為總數。乃以先加後減其氣盈朒為定積,凡分餘不成全而更不復須者,過半更不後夜無氣也。以盈朒定積,盈加朒減其日小餘,滿若不足,進退之,各其入盈朒日及小餘。若非朔望有交從者速粗舉者,以所入定氣日算乘先後率,加十五而一,先加減盈朒為定積。入氣日十五算者,加十六而一。



 歷變周:四十四萬三千七十七。



 變奇率:十二。



 歷變日:二十七;變餘,七百四十三;變奇,一。



 月程法:六十三。



 推歷變術



 以歷變周去總實,餘,以變奇率乘之,滿變周又去
 之。不滿者,變奇率約之,為變分。不盡,為變奇。分滿總法為日,不滿為餘。命日算外,即所求年天正恆朔夜半入變日及餘,以天正恆朔小餘加之,即經辰所入。



 求朔弦望經辰所入



 因天正經辰所入日餘奇,加日七、餘五百一十二、奇九。奇滿率成餘。餘,如總法為日,得上弦經辰所入。以次轉加,得望、下弦及來月朔。所入滿變日及餘奇,則去之。凡相連去者,皆仿於此。徑求望者,加朔所入日十四、餘一千二十五、
 奇六。徑求次朔,加一日、餘一千三百七、奇十一。



 求朔望弦盈朒減辰所入術



 各以其日所入盈朒定積,盈加朒減其恆經辰所入,餘即各所求。



 求朔弦望盈朒日辰入變遲速定數術



 各列其所入日增減率,並後率而半之,為通率。又二率相減,餘為率差。增者,以入餘減
 總法,餘乘率差,總法而一,並率差而半之。減者,半入餘乘率差,亦總法而一,並以加於通率,入餘乘之,總法而一,所得為經辰變轉半經辰變。速減遲加盈朒經辰所入餘,為轉餘。應增者,減法。應減者,因餘。皆以乘率差,總法而一,加於通率。變率
 乘之,總法而一,以速減遲加變率為定率。乃以定率增減遲速積為定。此法微密至當,以示算理通途。若非朔望有交及欲考校速要者,但以入餘乘增減率,總法而一,增減速為要耳。其後無同率者,亦因前率,應增者以通率為初數,半率差而減之;應減入餘進退日者分為二日,隨餘初末,如法求之。所得並以加減變率為定。



 其入前件日餘,如初數已下者為初,已上者以初數減總法,餘為末之數。增減相反,約以九分為限。初雖少弱,而末微強,餘差不多,理況兼舉,皆今有雜差,各隨其數。若恆算所求,七日與二十一日得初率,而末之所減,隱而不顯。且數與平行正算,亦初末有數,而恆算所無。其十四日、二十八日既初末數存,而虛差亦減其數,數當去恆法不見。



 求朔弦望盈朒所入日名及小餘術



 各以其所入變歷速定數速減遲加其盈朒小餘。滿若不足,進退其日。命以甲子算外,各其盈朒日反餘。加其恆日,餘者為盈;減其恆日,餘者為朒。其日不動者,依恆朔日而定其小餘,推擬日月行度。其定小餘二十四已下,一千三百一十六已上者,其入氣盈朒、入歷遲速,皆須覆依本術推算,不得從粗舉速要之限。乃前朔後朔,迭相推校。盈朒之課,據實為準。損不侵朒,益不過盈。



 求定朔月大小術



 凡朔盈朒日名,即為定朔日名。其定朔日名,十干與來
 月同者大,不同者小。其月無中氣者為閏月。其正月朔有定加時正月者,消息前後各一兩月,以定月之大小。合虧在晦二者,弦望亦隨事消息。凡置月朔,盈朒之極,不過頻三。其或過者,觀定小餘近夜半者量之。



 檢宿度術



 前件周天二十八宿,相距三百六十五度,前漢唐都以渾儀赤道所量。其數常定,紘帶天中,儀圖所準。日月往來,隨交損益。所入宿度,進退不同。



 黃道宿度左中郎將賈達檢日月所去赤道不同,更鑄黃道渾儀所檢者。



 臣等今所修撰討論,更造木渾圖交絡調賦黃赤二道三百六十五度有奇,校量大率,與此符會。今歷以步日行月及五星出入循此。其月行交絡黃道,進退亦宜有別。每交輒差,不可詳盡。今亦依黃道推步。



 推日躔術



 置冬至初日躔差率,加總法,乘冬至小餘,如總法而一,以減天宿度分。其餘命起黃道鬥十二度,宿次去之,經斗去宿分度,不滿宿算外,即所求年冬至夜半所在宿度算及分。



 求每定氣初日夜半日所在定度術



 各以其定氣初日躔差率,乘氣定餘,總法而一,進加退減餘為分,以減定氣日度及分,命以宿次如前,即其夜
 半度及春秋二分定氣初日為進退之始,當平行一度。自餘依進加退減度之。



 求次日夜半日所在定度術



 各因定氣夜半所在為本,加度一。又以其日躔差率,進加退減度分。滿若不足,並依前例。去命如上,即得所求。其定朔弦望夜半日度,各隨定氣,以其日月名亦直而分別之。勘右依恆有餘,從定恆行度,不用躔差。



 求朔弦望定日夜辰所加日度術



 各以其定小餘為平分。又定小餘乘其日所躔差率,總
 法而一,乃進加退減其平分,以加其夜半日度,即各定辰所加。其與五星加減者,半其分,消息月朔者,應推月度所須,皆依本朔大小。若注歷,依甲子乙丑各擬入。



 推月離術



 求朔望定日辰月所在度術



 各置朔弦望定辰所加日度及分。



 凡朔定辰所加為合朔,日月同度。上弦加度九十一、分四百一十七。



 望加度一百八十三、分八百三十四。



 下弦加度二百七十三、分一千二百五十一。訖,各半而十退之,為程度分。



 求次月定朔夜半入變歷術



 置天正恆朔夜半所入變日及餘。定朔有進退一日者,進退一日,為定朔夜半所入。



 月大加二日,月小加一日。餘皆五百九十六、奇十六。



 求次日夜半所入變歷術



 因定朔夜半所入日算,加日一,滿皆如前。其弦皆依前定日所在求之。



 求變日定離程術



 各以其日夜半入變餘,乘離差,總法而一,為見差。以進加退減其日離程,為月每日所離定程。



 求朔弦望之定日夜半月所在度術



 各以其日定小餘,乘所入變日離定程,總法而一,為夜半後分。滿程法為度,餘為度分。以減其日加辰所在度及分,命以黃道宿度,即其所求。次日夜半,各以離定程加朔弦望夜半所在分,滿程法從度,去命以黃道宿度算外,則次日夜半月度。求晨昏度,以其日離定程乘其
 日夜刻,二百而一,為昏分,滿程法為度。望前以昏,後以晨,加夜半度,得所求。其弦望以五乘定小餘,程法一,為刻,即各其辰所入刻數。皆減其晨前刻,不盡為晨後刻。不滿晨前刻者,從前日注歷,伺候推。



 總刻:一百。辰刻:分十一。刻分法:七十二。



 求定氣日晝夜漏刻及日出沒術



 倍其氣晨前刻及分,滿法從刻,為日不見漏。以減百
 刻,
 餘為日見漏。五刻晝漏刻。以晝漏刻減百刻,餘夜漏刻。以四刻十二分加晨前漏刻,命起子初刻算外,即日出辰刻。以日見漏加日出刻辰,以次如前,即日沒所在辰刻。以二十五除從夜漏,得每更一籌之數。以二刻三十六分加日沒辰刻,即甲辰刻,又以更籌數加之,得甲夜一籌數。以次累加,滿辰去命之,即五更夜籌所以當辰刻及也,以配二十一箭漏之法也。



 求每日並屈申數術



 每氣準為一十五日,各置其氣屈申率。每以發斂差損益之,差滿十從分,分滿十從率一,即各每日屈申率。各
 累計屈申率為刻分,乃以一百八十乘刻分,泛差十一乘綱紀而除之,得為刻差,滿法為刻。隨氣所在,以申減屈加不見漏而半之,為晨前定刻。每求次日,各如前法。時加其如始,隨加辰日晚,以率課之。



 求黃道去極每日差術



 置刻差,三十而一為度。不滿三約為分。申減屈加其氣初黃道度,即每日所求。



 求昏旦去中星度術



 每日求其晝漏刻數,以乘期實,二百乘總法而除之,得昏去中星度。以減周天度,餘為晨去中星度。以昏旦去中星度,加其辰日所在,即各其日中宿度。其梗概粗舉者,加其夜半日度,各其日中星宿度。



 因求次日者,各置其四刻差,七十二乘之,二百八十八而一度。冬至後加,夏至後減。隨日加,各得每日去中度。晨昏所距日在黃道中星準度,以赤道計之。其赤道同太初星距。



 推游交術



 終率:一千九十三萬九千三百一十三。奇率:三百。



 約終:三萬六千四百六十四奇一百一十三。



 交中:一萬八千二百三十二奇五十六半。



 交中日:二十七餘二百八十四奇一百一十三。



 中日:十三餘八百一十二奇五十六半。



 虧朔:三千一百六奇一百八十七。



 實望:一萬九千七百八十五奇一百五十。



 後準:一百五十二奇九百三半。



 前準:一
 萬六千六百七十八奇二百六十三。



 求月行入交表裏術



 置總實,以終率去之。不足去者,奇率乘之。滿終率,又去之。不滿者,奇率約之,為天正恆朔夜半入交分。不盡,為奇。以總法約入交分,為日。不盡,為餘。命日算外,即天正恆朔夜半入交日算及餘、奇。天正定朔有進退日者,依所進退一日,為朔所入。日不滿中日及餘、奇者,為月在外;滿,去之,餘皆一為月在內。大月加二日,小月加一日,餘皆一千五十五、奇一
 百八十七。求次日,加一日,滿中日者,皆去之,餘為入次。一表一里,迭互入之。



 求
 月入交去日道遠近術



 置所入日差,並後差半之,為通率。進,以入日餘減總法,以乘差,總法而一,並差以半之。退者,半入餘,以乘差,總法而一。皆加通率,為交定率。乃以入餘乘定總法。乃進
 退差積,滿十為度,不滿為分,即各其日月去日道度數。每求日道宿度去極數,其入七日,餘一千七十六、奇二十八少已下者,進,已上,盡全;餘二百六十三、奇二百七十一大者,退入十四日,如交餘奇已下者,退;其入已上,盡全;餘五百二十七、奇二百四十二半者,進。而終其要為五分。初則七日四分,十四日三分;末則七日後一分,十四日後二分。雖初強末弱,差率有檢,月道一度半強已下者,為沾黃道。當朔望,則有虧。遇五星在黃道者,則
 相侵掩。



 求所在宿術



 求夜半入交日十三算者及餘,以減中日及餘,不盡者,以乘其日離定程,總法而一,為離分,滿程為度,以加其日夜半月所在宿度算及分,求次交準此,各得其定交所在度。置前後定交所宿度算及分,半之,即各表裏極所在宿度及分。



 求恆朔望泛交分野



 因天正恆朔夜半入交分,以天正恆朔泛交分求望泛交,以實望加之。又加,得次月恆朔泛交分。滿約終及奇,去之。次求次朔,以虧望加之。



 求朔望入常交分術



 以入氣盈朒定積,盈加朒減其恆泛交分,滿若不足,進退約終。即其常分交。



 求朔望定交分術



 以六十乘定遲速,以七百七十七降除之,所得為限數。
 速減遲加如常。其數朔入交月在日道里者,以所入限數減定遲速,餘以速減遲加其定交分。而出日道表者,為變交分。加減不出日道表,即依定交分求蝕分。其變交分出日道表三時半內者,檢其前後月望入交分數多少,依月虧初復末定蝕術,注消息,以定蝕不。



 求入蝕限術



 其入交定分,如交中已下者,為月在外道;交中已上者,以交中減之,餘為月在內。其分如後準已下、前準已上者,為入蝕限。望則月蝕,朔入限,月在里者,日蝕。入限如後準已下者,為交後分;前準已上者,反減交中,餘為交前
 分。以一百一十二約之,為交時。



 求月蝕所在辰術



 置望日不見刻,六十七乘之,十而一,所得,若蝕望定小餘與之等已下,又以此得減總法餘與之等已為蝕正見數定小餘。如求律氣應加時法,得加時所在辰月在沖辰蝕,若非正見者,於日出後日沒前十二刻半內,求其初末以候之。又以半總減蝕定小餘,不足減者半總加減訖,以六乘之,如辰率而一,命起子半算外,即月蝕所在辰。



 求日蝕所在辰術



 置有蝕朔定小餘副之,以辰率除之,所得以艮、坤、巽、乾為次,命退算外。不滿法者,半法減之。無可減者,為初;所減之餘,為末。初則減法,各為差率。月在內道者,乃以十加去交時數而三除之,以乘差率,十四而一,為差。其朔在二分前後一氣內,即以差為定。近冬至以去寒露雨水、近夏至以去清明白露氣數倍之,又三除去交時數增之。近冬至,艮巽以加,坤乾以減;近夏至,艮巽以減、坤乾以
 加其差,為定差。艮坤加副,巽乾減副。月在外道者,三除去交時數,以乘差率,十四而一,為之差。艮坤以減副,巽乾以加副,各加減副訖,為定副小餘。如求律氣應加時術,即日蝕所在辰及少太。其求入辰刻,以半辰刻乘朔,辰率而一,得刻及分。若蝕近朝夕者,以朔所入氣日出沒刻校蝕所在,知蝕見不之多少,所在辰為正見日月蝕既,在起復初末,亦或變常退於見前後十二刻半候之。



 求月起復依蝕分後術



 求月在日道表朔不應蝕準。朔在夏至初日,準去交前後二百四十八分為初準;已下,加時在午正前後七刻內者,食。朔去夏至前後,每一日損初準二分,畢於前後九十四日,各為每日變準。其朔去交如變準已下,加時如前者,蝕。



 又以末準六十減初準及變準,餘以十八約之,為刻準。以並午正前後七刻數為時準。加時準內去交分,如末準
 已下,並蝕。又置末準,每一刻加十八,為差準。每加時刻,去午前後如差準刻已下,去交分如差已下者,並蝕。自秋分至春分,去交如末準已下,加時南方三辰者,亦蝕。凡定交分在辰前後半時外者,雖入蝕準前為蝕。求月在日道里朔應蝕而不蝕準。朔在夏至日,去交一千三百七十三,為初準;已上,加時在午正前後十八刻內者,或不蝕。朔去夏至前後,每一日益初準一分半,畢於前後九十四日,各為每日變準。以初減變,餘十而一,為刻準。以
 刻減午正前後十八刻,餘,十而一為時準。其去交在變準已上,加時在準內者,或不蝕。



 求月蝕分術



 置去交前後定分,冬交前後,皆去二百二十四。春交後去一百,交前去二百。夏不問前後,去五十。秋交後去二百,交前去一百。不足去者,蝕既。有餘者,以減後準,一百四而一。餘半已下,為半弱;半已上,為半強。命以十五為限,得月蝕之大分。



 求月蝕所起術



 月在內道:蝕東方三辰,虧自月下斜南上,月從西而漸北,自東而漸南。蝕南方三辰,虧起左下,甚於正南,復於右下。蝕西方三辰。虧自南而漸東,月從北而漸西,起於月上,斜南而下。月在外道:蝕東方三辰,虧起自月下,斜北而上,虧起東而漸北,月從西漸南。蝕南方三辰,虧起左上,甚於正北,復於右上。蝕西方三辰。虧自北而漸東,月從南而漸西,起於月上,斜北而上。凡蝕十二分已上,皆隨黃道所在起復,於正傍逆順上下每過其分。又道有升降,每各不同,各隨時取正。



 求日蝕分術



 月在內道者,朔入冬至,畢朒雨水,及盈秋分,畢大雪,皆以五百五十八為蝕差。自入朒春分已後,日損六分,畢於白露。置蝕去交前後定分,皆以蝕差減之。但去交分不足減者,皆反以減蝕差為不蝕餘。自入朒小滿,畢盈小暑,加時在午正前後七刻外者,皆去不蝕餘一時;三刻內,加不蝕餘一時。朒大寒畢朒立春,交前五時外,大暑畢盈立冬,交後五時外,皆去不蝕餘一時,五
 時內加一時。諸加時蝕差應減者,交後減之,交前加之。應加者,交後加之,交前減之。但不足減去者,蝕既。加減入不蝕限者,或不蝕。其月在外道者,冬至初日無蝕差。自後日益六分,累計以為蝕差,畢於朒雨水。自入朒春分,畢於盈白露,皆以五百二十二為蝕差。自入盈秋分已後,日損六分,畢於大雪。所損之餘,為蝕差。以蝕差加去交定分,為蝕分。以減後準,餘為不蝕分。各置其朔蝕差,十五約之,以減一百四,餘為定法。不蝕分餘,各如定法得一分。餘半法已上,為半強;已
 下,為半弱。減十五,餘為蝕之大分。



 求日蝕所起術



 日在內道:日蝕東方三辰,虧自日上近北而斜下,月漸西北,日漸東南。日蝕南方三辰,虧起右下,甚正北,復左下。月在南而漸東,日在北而漸西。日蝕西方三辰。月漸東北,日漸西南,虧自日下近西而斜上。日在外道:日蝕東方三辰,虧自日上近南而斜下,月漸東南,日漸西北。日蝕南方三辰,虧起右下,甚正北,復左下。月在南而漸東,日在北而漸西。日蝕西方三辰。月漸西南,日漸東北,虧自日下近南而斜上。凡蝕十二分已上,起於正旁。各據黃道升降,以準其體。隨
 其所處,每各不同。蝕有初末,動涉其時,隨便益損,以定虧復所在之方也。



 求日月蝕虧初及復末時刻術



 置朔望所蝕大分數為率。四分已上,因增二。五分已上,因增三。九分已上,因增四。十三分已上,因增五。各為泛用刻率,副之。以乘所入率,副之。以乘所入變增減率,總法而一,應速增損、減加,應遲依其增減副,訖,為蝕定用刻數。乃四乘之,十而一,以減蝕甚辰刻,為虧初。又六乘之,十而一,加蝕甚辰刻,為復末。依其定加時所在辰刻加
 減命之,各其辰、其月蝕甚初末更籌。因其日月所入辰刻及分,依前定氣所遇夜刻更籌術,求其初末及甚時更籌。



 迦葉孝威等天竺法,先依日月行遲疾度,以推入交遠近日月蝕分加時,日月蝕亦為十五分。去交十五度、十四度、十三度,影虧不蝕法,自此已下,乃依驗蝕。十二度十五分,蝕二分少強,以漸差降,自五度半已上,蝕既,十四分強。若五度無餘分已下,皆蝕盡。又用前蝕多少,以定後蝕分餘。若既,其後蝕度及分,即加七度以為
 蝕度。若望月蝕既,來月朔日雖入而不注蝕。若蝕半已下,五分取一分;若半已上,三分取一分,以加來月朔蝕度及分。若今歲日餘度及分,然後可驗蝕度分數多少。又云:六月依節一蝕。是月十五日是月蝕節,黑月盡是月蝕節,亦以吉兇之象,警告王者奉順正法,蒼生福盛,雖時應蝕,由福故也,其蝕即退。更經六月,欲蝕之前,皆有先兆。月欲有蝕,先月形搖振,狀若驚懼,月兔及側月色黃如有憂狀。自常暈,月初生時,光不顯盛,或極細微。
 日欲有蝕,先日形搖振,極如驚懼狀。或光色微昧,不赫盛,或黎慘。日月蝕先同候,光隕墜,或旦暮際有赤色起,如火燒,金銀珠玉諸寶失光。或有闕盡如雲入日,或有黑盡入月,鳥聲細隱,烏不顯亮,雲交擾擾,光景渾亂,忽極令諸乳卒竭,月濕如汗狀,日形段裂無光,犬嗥貓叫,虹見有聲,三辰失闕,月時有缺,水赤色有膩。十四日、十五日,闢鳥圓集者,亦是蝕之先候。此等與中國法數稍殊,自外梗概相似也。



 步五星術



 見伏五十二日,
 晨見伏六十三日,餘、奇同終分奇。



 求五星平見術



 各以伏分減總實,餘以其星總率去之。不足去者,反減其餘總率。餘以總法約之,為日,不盡為餘奇,即所求年
 天正恆朔夜半後星晨夕平見日算及餘奇。天正定朔進退日者,進減退加一日為定朔夜半後星平見日及餘奇。其金水二星,先得夕平見,其滿見伏日及餘者去之,餘為晨平見日及餘奇。命見日天正歷月大小,以次去之,不滿月者為入其月,命日算外,即晨夕平見所在月日及餘奇。



 求後平見在月日術



 各以其星終日算及餘奇,如前平見所在月日算及餘奇。奇滿奇率,從餘。餘滿總法,為日。去命如前,即後平見所在月日及餘奇,其金水二星,加夕得晨,加晨得夕。各半見餘,以同半
 總。



 求五星常見術



 各依其星平見所入恆氣,計日損益。分滿半總為日,不滿為分,以損益所加減。訖,餘以加減訖平見日及分,即其常見日及分。星日初見去日度,平見入氣歷。加減日。損益率。



 歲星初見,去日十四度。見入冬至,畢小寒,均減六日。自入大寒已後,日損六十七分。見入春分初日,依平。自後日加八十九分。入立夏,畢小滿,均加六日。自入芒種已後,日損八十九分。入夏至,畢立秋,均加四日。
 自入處暑已後,日損一百七十八分。入白露,初日依平均,自後日減五十二分。入小雪,畢大雪,均減六日。



 熒惑初見,去日十七度。見入冬至,初日減二十七日。自後日損六百三分。入大寒,初日依平。自後日加四百二分。入雨水,畢穀雨,均加二十七日。入自立夏已後,日損一百九十八分。入立秋,依平。自入處暑已後,日減一百九十分。入小雪,畢大寒,均減二十七日。



 鎮星初見,去日十七度。見入冬至,初日減四日。自後日益八十九分。入大寒,畢春分,均減八日。自入清明已後,日損五十九分。入小暑,初日依平。自後日加八十九分。入白露,初日加八日。自後日損
 一百七十八分。入秋分,均加四日。自入寒露已後,日損五十九分。入小雪,初日依平。自平後日減八十九分。



 太白初見,去日十一度。夕見:入冬至,初日依平。自後日減一百分。入啟蟄,畢春分,均減九日。自入清明已後,日損一百分。入芒種,依平。自入夏至已後,日加一百分。入處暑,畢秋分,均加九日。自入寒露已後,日損一百分。入大雪,依平。晨見:入冬至,依平。自入小寒已後,日加六十七分。入立春,畢立夏,均加三日。自入小滿已後,日損六十七分。入夏至,依平。自入小暑已後,日減六十七分。入立秋,畢立冬,均減三日。自入小雪已後,日損六十七分。



 辰星初見,去日十七度。夕見:入冬至,
 畢清明,依平。入穀雨,畢芒種,均減二日。入夏至,畢大暑,依平。入立秋,畢霜降,應見不見。其在立秋及霜降二氣之內,夕去日十八度外,三十六度內,有木火土金一星已上者亦見。入立冬,畢大雪,依平。晨見:入冬至,均減四日。入小寒,畢大寒,依平。入立春,畢啟蟄,均減三日。其在啟蟄氣內,去日度如前,晨無木火土金,一星已上者不見。入雨水,畢立夏,應見不見。其在立夏氣內,去日度如前,晨有木火土金一星已上者,亦見。入小滿,畢寒露,依平。入露降,畢立冬,均加一日。入小雪,畢大雪,依平。



 求五星定見術



 各置其星常見日消息定數半之,息減消加常見日,即為定見日及分。五星休王光不同,喜怒盛衰大小尤異。茍變於常見或先後,今依日躔遲速考其行,度其格,以去日為之定準。



 求星見所在度術



 置星定見日夜半日所在宿度算及分,半其日躔差,乘定見餘,半總而一,進加退減定見餘,以加夜半度分,乃以其星初見去日度數,晨減夕加之,即星初見辰所在。



 宿度等及分行星術



 各置其星初見日消息定數,半之,息加消減,其星初見行留日率。其土木二星不須加減,即依本術。其加減不滿日者,與見通之。過半從一日,無半不從論。乃依行星日度之率,求日之行分。



 求初見日後夜半星所在術



 置其星定見餘,以減半總,以其星初見行分乘之,半總而一,以順加逆減星初見定辰所在度分。加之滿法,減之不足,進退一度。依前命之算外,即星見後夜半所在宿度及分。自此已後,每依其星計日行度,所至日度及益疾,皆從夜半為始。辰有少,隨所近也。



 轉求次日夜半星行所至術



 各以其星一日所行度及分,順逆加減之。其行有小分者,以日率為母。小分滿母,去之,從行分一。行分滿半總,去之,從度一。其行有益疾益遲者,副置一日行分。各以其差遲損疾加之,留者因前,逆則依減。順行出斗去其分,逆行入斗先加分。訖,皆以程法約行分為度分,各得每日所至。其五星後順留退所終日度,各依伏度,求其去日遠近,消息日度之所在,以定伏日所在。若注歷,其日度及金水等星,皆棄其分也。



 求平行度及分術



 置定度率,以半總乘之,以有分者從之,以日率除之,所得,為一日行分。不盡小分滿其行分。滿半總為度。即是一日所行度及行分、小分。置定日率,減一日,以所差分乘之,二而一,為差率。益疾者以差率減平行分,益遲者以差率加平行分,即是初日所行度及分。



 星名星行變日初行入氣歷行日率行度及度分率:損益率。



 歲星:初順,差行一百一十四日,行十八度五百九遲一分先疾,日益十四日。前留,
 二十六日。旋退西行,差行三十日,退六度十二分。先遲,日益疾二分。又退西行,差行四十二日,退六度十二分。先疾,日益遲二分。後留,二十五日。後順,差行一百一十四日,行十八度五百九。先進遲,日益疾分日盡而夕伏十四日。



 熒惑:初順,入冬至初日,率二百四十三日行一百六十五度。自後三日損日及度各三。小寒初日,二百三十五日行一百五十四度。自後二日損日及度各三。穀雨四日,平,畢小滿九日。一百七十八日行一百度。自入小滿九日已後,二日益日及度各一。夏至初日,平,畢六日。一百七十一日行九十三度。自入夏至六日已後,三日
 益日及度各一。立秋初日,一百八十四日行一百六度。自後一日益日及度各一。白露初日,二百一十四日行一百三十六度。自後五日益日及度各一。秋分初日,二百三十二日行一百五十四度。自後一日益日及度各一。寒露初日,二百四十七日行一百六十九度。自後五日益日及度各二。霜降五日,平,畢立冬十三日。二百五十九日行一百八十一度。自入立冬十三日已後,二日損日及度各一。復冬至初日,二百四十二日行一百六十五度。



 各依所入恆氣,平者依率,自餘計日損益,名為前疾日度定率。其前遲及留退入氣有損益日
 度者,計日損益,皆同此疾之法,以為遲留旋退定日度之率也。



 求變日率術:此疾,入大寒六日,損日率一,畢雨水。入春分,畢立夏,減日率十。入小滿初,減日率十。後三日損所減一。畢芒種,依平。若入立秋,三日益日率一,畢處暑。入白露,畢秋分,均加率十。入寒露初,加率十。後一日半損所加一。畢氣盡,依平。



 求變度率術:
 此疾,若入大寒,畢於啟蟄,立夏至大暑氣盡,霜降畢小雪,皆加度率四。清明畢穀雨,加率度十二。初行入處暑,減日率六十,度率三十。別為初遲半度之行,行盡此日度,及來所減之餘日度之率續為疾。入白露,畢秋分,四十四日行二十二度。皆為初遲半度之率。初行入大寒,畢大暑,差行,先疾,日益遲一分。各如上法,求其行分。其前遲後日率,既有增損,而益遲益疾若分,皆檢括前疾末日行分,為前遲初日行分。以前遲平行分減之,餘為前遲總差。後疾日分,為後遲末日行分。為後遲日行分減之,餘為後總差。減為後別日差分。其不滿者,皆調為小分。遲疾之際,行分衰殺不論。所差多者,依此推算。若所差不多者,各
 依本法。



 前遲:順,差行,入冬至,六十日行二十五度。先疾,日益。自入小寒已後,二遲二分,日損日及度各一。大寒初日,五十五日行二十度。自後三日益日及度各一。立春初日平。畢清明,六十日行二十五度。自穀雨氣別減一氣。立夏初日平。畢小滿,六十日行二十二度。自入芒種,別益一度。夏至初日平。畢處暑,六十日行二十五度。自入白露已後,三日損一度。秋分初日,六十日行二十王度。自後一日益一,日半益一度。寒露初日,六十日行二十五度。自後二日損一度。立冬一日平。畢氣,六十日行十七度。自大雪已後,五日益一度。大雪初日,六十日行
 二十度。自後三日益一度。



 前留:十三日。前疾減日率一度,以其數分益此留及後遲日率。前疾加日率者,以其數分遲日率。旋退,西行。入冬至安裝日,六十三日退二十一度。自自後四日益一度。小寒一日,六十三日退二十六度。自入小寒已後,三日半損一度。立春三日平。畢啟蟄,六十二日退十七度。自入雨水已後,二日益日及度各一。雨水八日平。畢氣盡,六十七日退二十一度。自入春分已後,一日損日及度各一。春分四日平。畢芒種,六十三日退七十度。自入夏至已後,六日損日及度各一。大暑初日平。畢氣盡,五十八日退十二度。立秋初日平。畢氣盡,五十七日退
 十一度。自入白露已後,二日益日及度各一。白露十二日平。畢秋分,六十三日退七十度。自入寒露已後,三日益日及度各一。寒露九日平。畢氣盡,六十六日退二十度。自入霜降已後,三日損日及度各一。霜降六日平。畢氣盡,六十三日退十七度。自立冬已後,三日益日及度各一。立冬十一日平。畢氣盡,六十七日退二十一度。自入小雪已後,二日損日及度各一。小雪八日平。畢氣盡,六十三日退十七度。自入大雪已後,三日益一度。



 後留:冬至留十三日。自後二日半益一日。大寒初平,畢氣盡,留二十五日。自入立春已後,二日半日損一。雨水初,留十三日。自後三日益一日。
 清明初,留二十三日。自後一日損一日。清明十日平,畢氣盡,留十五日。自入白露已後,二日損一日益一日。秋分十一日,無留。自入秋分十一日已後,一日益一日。霜降初日,留十九日。自後三日損一日。立冬三日平,畢大雪,留十三日。



 後遲:順,差行六十日行二十五度。先疾,日益疾二日。前後疾加度者,此遲依數減之為定度;前疾無加度者,此遲入秋分至立冬,減三度,入冬至減五度,後留定日朒十三日者,以所朒日數,加此遲日率也。



 後疾:冬至初日,率二百一一日行一百三十一度。自後一日損日及度各一。大寒八日,一百七十二日行九十四度。自入大寒八日已後,一日損日及度各一。啟蟄,平。畢氣盡,
 一百六十一日行八十三度。自入雨水已後,三日益日及度各一。穀雨三日,一百七十七日行九十九度。自入穀雨後,三日益日及度各一。芒種十四日平。畢夏至,二百三十三日行一百五十度。自入夏至已後,十日益日及度各一。小暑五日,二百五十三日行一百七十五度。自入小暑已後,五日益日及度各一。大暑初日平,畢處暑,二百六十三日行一百八十五度。自入白露已後,二日損日及度各一。秋分一日,二百五十五日行一百七十七度。自入秋分一日已後,一日半復日及度各一。大雪初日,二百五十日行一百二十度。自入秋分,三日益日及度各一。冬
 至初日,復二百一十日行一百二十七度。其入恆氣日度之率有損益者,,計日損益,並同前疾之法,以為後疾定度之率。



 求變日率術:其前遲定日朒六十,及退行定日朒六十三者,皆以所朒日數加此疾定日率,前遲定日盈六十三,後留定日盈十三者,皆以所盈日數減此疾定日率。加減訖,即變日率。



 求變度率術:其前遲定度朒二十五,退行定度盈十七,後遲入秋分至冬至減度者,皆以所盈朒度數,加此疾定度率。前遲定度盈二十五,及退行定度肉十七者,皆以所盈朒度數,減此疾定度率。加減訖,即變度率。



 初行,入春分,畢穀雨,差行。先遲,日益疾一分。初行,入立夏,畢夏至,日行半度。六十六日行二十二度。小暑,五十日行二十五度。立秋畢氣盡,二十日行十度,減率續行,並同前疾初遲法。損益依前,求其行分。各
 盡度而夕伏。



 鎮星:初順,差行,八十三日行七度二百九十分。先疾,日益遲半分。前留,三十七日。旋退,西行,差行,五十一日退三十分。先遲,日益疾少半。



 太白:夕見,順,入冬至畢立夏,入立秋畢大雪。一百七十二日行二百六度。自入小滿後,十日益一度,為定疾。初入白露,畢春分,差行。疾,日益遲二分。自餘平行。夏至畢小暑,一百七十二日行二百九度。自入大暑已後,五日損一度,畢氣盡。平行:入冬至初日及大暑,各畢氣盡。一十三日行一十三度。自入冬至後,十日損一,畢已後立春,入立秋,日益一,畢秋分。啟蟄畢芒種,七日行七度。自入夏至後,五日益一,畢於小雪。寒露初日,三
 十三日行二十二度。自後六日損一,畢於小雪。順遲:差行,三十二日行三十度。先疾,日益遲八分。前疾加度過二百六度者,準數損此度。夕留,七日。夕退,西行,一十日退五度。日盡而夕伏。晨初退,西行,十日退五度。日退半度。晨留,七日。順遲,差行,冬至畢立夏,大雪畢氣盡。三十二日,先遲,日益疾八分。自入小滿已後,率十日損一度,畢芒種。平行,冬至畢氣盡,立夏畢氣盡。一十三日行一十三度。日行一度。自入小寒已後,六日益日及度各一,畢於啟蟄。入小滿後,七日損日度各一,畢立秋。雨水初日,二十三日行二十三度。自後六日損日及度各一,畢於穀雨。處暑畢寒露,無此平行。自入霜降後,五日益日及度各一,畢大雪。前遲行損
 度不滿三十度者,此疾依數益之。疾行,一百七十二日行二百六度。處暑畢寒露,差行,先遲,日益疾一分。餘平行,行日盡而晨伏。



 辰星:夕見,順疾,一十二日行二十一度六分。日行一度五百三分。大暑畢處暑,一十二日行一十七度二分。日行一度二百八十分。平行,七日行七度。自入大暑後,二日損日及度各一。入立秋,無此平行。順遲行,六日行二度四分。日行二百二十四分,前疾行十一度者,無此遲行。日盡而夕伏。夕留,五日。晨見,留五日。順遲行,六日行二度四分。日行二百二十四分。自入大寒,畢於啟蟄,無此遲行。平行,七日行七度。日行一度。大寒已後,二日損日及度各一。入立春,無此平行。順疾行,一
 十二日行二十一度六分。日行一度五百三分。前無遲行者,一十三日行十七度十分。日行一度二百八十分。各日盡而晨伏。



 凡五星終日分奇,皆於伏分消遁,故於行星更不別見。



 武太后稱制,詔曰:「頃者所司造歷,以臘月為閏。稽考史籍,便紊舊章,遂令去歲之中,晦仍月見。重更尋討,果差一日。履端舉正,屬在於茲。宜改歷於惟新,革前非於既往。可以今月為閏十月,來月為正月。」是歲得甲子合朔冬至。於是改元聖歷,以建子月為正,建丑為臘,建
 寅為一月。命太史瞿曇羅造新歷。至三年,復用夏時,《光宅歷》亦不行用。中宗反正,太史丞南宮說奏:「《麟德歷》加時浸疏。又上元甲子之首,五星有入氣加時,非合璧連珠之正也。」乃詔說與司歷徐保乂、南宮季友,更治《乙巳元歷》。至景龍中,歷成,詔令施用。俄而睿宗即位,《景龍歷》寢廢不行。《麟德歷經》,今略載其法大端。



 母法一百。兩大衍之數為母法。



 旬周六十。六甲之終數為旬周。



 辰法八刻;分,三十三少半。以十二辰數除一百刻,得辰法。



 期周三百六十五日;餘,二十四;奇,四十八。一期之總日及餘奇數為期周。



 氣法十五日;餘,二十一;奇,八十五少半。以二十四氣分期周,得氣法。



 候法五日;餘,七;奇,二十八;小分,四。以七十二候分期周,得候法。



 月法二十九日;餘,十三;奇。為月法。



 日法日舒月遠乃舒一合朔之及餘奇為日法。



 望法十四日;餘,七十六;奇,五十三。因為陰後限。二分月法得望法。亦是月行陰歷,後與朔望會交限。



 弦法七日;餘,三十八;奇,二十六半。四分月法,得弦法。



 閏差十日;餘,八十七;奇,七十六。月法去期周,餘得閏差。



 沒數九十一;餘,三十一;奇,十二。四分期周,餘四分之得沒數。



 沒法一;餘,三十一;奇,十二。以旬周去期周,餘四分之,得沒法。



 月周法二十七日;餘,五十五;奇,四十五;小分,五十九。月行遲疾一周之數,為月周法。



 月差法一日;餘,九十七;奇,六十;小分,四十一。以月周減月法,餘得月差。



 周天法三百六十五度;餘,二十五;奇,七十一;小分,十三。二十八宿總度數、相距總數及餘奇,為周天法。



 交周法二十七日;餘,二十一;奇,二十二;小分,十六七分。日行陰陽一周交於是日之數,為交周法。



 交差法二日;餘,三十一;奇,八十三;小分,八十三分。以交周法減月法,得交差法。



 交中法十
 三日;餘,六十;奇,六十一;小分,三分半。二分交周,得交中法。



 陽前限十二日;餘,四十四;奇,六十九;小分,十六七分。月行陽歷,與朔望會之限。



 陽後限一日;餘,十五;奇,九十一;小分,九十一六分半。月行陽歷,後與朔望會之限。



 陰前限二十六日;餘,五;奇,三十;小分,二十五半分。月行陰歷,先與朔望會之限。



 木歲星合法三百九十八日;餘,八十六;奇,七十九;小分,八十。



 火熒惑合法七百七十九日;餘,九十;奇;五十五;小分,四十五。



 土鎮星合法三百七十八日;餘,八;奇,四;小分,八十。



 金太白合法五百八十三日;餘,九十一;奇,七十七;小分,七十。



 水辰星合法一百一十五日;餘,八十七;奇,九十五;小分,七十。



 太極上元,歲次乙巳,十一月甲子朔旦冬至之日,黃鐘之始,夜半之時,斗衡之末建於子中,日月如合璧,五星若連珠,俱起於星紀牽牛之初蹤。今大唐神龍元年,復歲次於乙巳,積四十一萬四千三百六十算外。上驗往古,年減一算。下求將來,年加一算。《乙巳元歷》法積數,大約如此。其算經不錄。



\end{pinyinscope}