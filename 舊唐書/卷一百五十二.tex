\article{卷一百五十二}

\begin{pinyinscope}

 ○裴垍李吉甫李籓權德輿子璩



 裴垍字,弘中,河東聞喜人。垂拱中宰相居道七代孫。垍弱冠舉進士。貞元中,制舉賢良極諫,對策第一,授美原縣尉。秩滿,籓府交闢,皆不就。拜監察御史,轉殿中侍御
 史、尚書禮部考功二員外郎。時吏部侍郎鄭珣瑜請垍考詞判,垍守正不受請托,考核皆務才實。



 元和初,召入翰林為學士,轉考功郎中、知制誥,尋遷中書舍人。李吉甫自翰林承旨拜平章事,詔將下之夕,感出涕。謂垍曰:「吉甫自尚書郎流落遠地,十餘年方歸,便入禁署,今才滿歲,後進人物,罕所接識。宰相之職,宜選擢賢俊,今則懵然莫知能否。卿多精鑒,今之才傑,為我言之。」垍取筆疏其名氏,得三十餘人。數月之內,選用略盡,當時翕然
 稱吉甫有得人之稱。三年,詔舉賢良,時有皇甫湜對策,其言激切;牛僧孺、李宗閔亦苦詆時政。考官楊於陵、韋貫之升三子之策皆上第,垍居中覆視,無所同異。及為貴幸泣訴,請罪於上,憲宗不得已,出於陵、貫之官,罷垍翰林學士,除戶部侍郎。然憲宗知垍好直,信任彌厚。



 其年秋,李吉甫出鎮淮南,遂以垍代為中書侍郎、同平章事。明年,加集賢院大學士、監修國史。垍奏:「集賢御書院,請準《六典》,登朝官五品已上為學士,六品已下為直學
 士;自非登朝官,不問品秩,並為校理;其餘名目一切勒停。史館請登朝官入館者,並為修撰;非登朝官,並為直史館。仍永為常式。」皆從之。



 元和五年,中風病。憲宗甚嗟惜,中使旁午致問,至於藥膳進退,皆令疏陳。疾益痼,罷為兵部尚書,仍進階銀青。明年,改太子賓客。卒,廢朝,賻禮有加,贈太子少傅。



 初,垍在翰林承旨,屬憲宗初平吳、蜀,勵精思理,機密之務,一以關垍。垍小心敬慎,甚稱中旨。及作相之後,懇請旌別淑慝,杜絕蹊徑,齊整法度,考
 課吏理,皆蒙垂意聽納。吐突承璀自春宮侍憲宗,恩顧莫二。承璀承間欲有所關說,憲宗憚垍,誡勿復言,在禁中常以官呼垍而不名。楊於陵為嶺南節度使,與監軍許遂振不和,遂振誣奏於陵,憲宗令追與慢官。垍曰:「以遂振故罪一籓臣,不可。」請授吏部侍郎。嚴綬在太原,其政事一出監軍李輔光,綬但拱手而已,垍具奏其事,請以李鄘代之。



 王士真死,其子承宗以河北故事請代父為帥。憲宗意速於太平,且頻蕩寇孽,謂其地可取。吐突
 承璀恃恩,謀撓垍權,遂伺君意,請自征討。盧從史陰苞逆節,內與承宗相結約,而外請興師,以圖厚利。垍一一陳其不可,且言:「武俊有大功於朝,前授李師道而後奪承宗,是賞罰不一,無以沮勸天下。」逗留半歲,憲宗不決,承璀之策竟行。及師臨賊境,從史果攜貳,承璀數督戰,從史益驕倨反覆,官軍病之。時王師久暴露無功,上意亦怠。



 後從史遣其衙門將王翊元入奏,垍延與語,微動其心,且喻以為臣之節,翊元因吐誠言從史惡稔可圖之
 狀。垍遣再往,比復還,遂得其大將烏重胤等要領。垍因從容啟言:「從史暴戾,有無君之心。今聞其視承璀如嬰孩,往來神策壁壘間,益自恃不嚴,是天亡之時也。若不因其機而致之,後雖興師,未可以歲月破也。」憲宗初愕然,熟思其計,方許之。垍因請密其謀,憲宗曰:「此唯李絳、梁守謙知之。」時絳承旨翰林,守謙掌密命。後承璀竟擒從史,平上黨,其年秋班師。垍以「承璀首唱用兵,今還無功,陛下縱念舊勞,不能加顯戮,亦請貶黜以謝天下」。遂
 罷承璀兵柄。



 先是,天下百姓輸賦於州府:一曰上供,二曰送使,三曰留州。建中初定兩稅,時貸重錢輕;是後貨輕錢重,齊人所出,固已倍其初征。而其留州送使,所在長吏又降省估使就實估,以自封殖而重賦於人。及垍為相,奏請:「天下留州、送使物,一切令依省估。其所在觀察使,仍以其所蒞之郡租賦自給;若不足,然後徵於支郡。」其諸州送使額,悉變為上供,故江淮稍息肩。



 垍雖年少,驟居相位,而器局峻整,有法度,雖大僚前輩,其造請
 不敢干以私。諫官言時政得失,舊事,操權者多不悅其舉職。垍在中書,有獨孤鬱、李正辭、嚴休復自拾遺轉補闕,及參謝之際,垍廷語之曰:「獨孤與李二補闕,孜孜獻納,今之遷轉,可謂酬勞愧矣。嚴補闕官業,或異於斯,昨者進擬,不無疑緩。」休復悚恧而退。垍在翰林,舉李絳、崔群同掌密命;及在相位,用韋貫之、裴度知制誥,擢李夷簡為御史中丞,其後繼踵入相,咸著名跡。其餘量材賦職,皆葉人望,選任之精,前後莫及。議者謂垍作相,才
 與時會,知無不為,於時朝無幸人,百度浸理;而再周遘疾,以至休謝,公論惜之。



 李吉甫,字弘憲,趙郡人。父棲筠,代宗朝為御史大夫,名重於時,國史有傳。吉甫少好學,能屬文。年二十七,為太常博士,該洽多聞,尤精國朝故實,沿革折衷,時多稱之。遷屯田員外郎,博士如故,改駕部員外。宰臣李泌、竇參推重其才,接遇頗厚。及陸贄為相,出為明州員外長史;久之遇赦,起為忠州刺史。時贄已謫在忠州,議者謂吉
 甫必逞憾於贄,重構其罪;及吉甫到部,與贄甚歡,未嘗以宿嫌介意。六年不徙官,以疾罷免。尋授柳州刺史,遷饒州。先是,州城以頻喪四牧,廢而不居,物怪變異,郡人信驗;吉甫至,發城門管鑰,剪荊榛而居之,後人乃安。



 憲宗嗣位,徵拜考功郎中、知制誥。既至闕下,旋召入翰林為學士,轉中書舍人,賜紫。憲宗初即位,中書小吏滑渙與知樞密中使劉光琦暱善,頗竊朝權,吉甫請去之。劉闢反,帝命誅討之;計未決,吉甫密贊其謀,兼請廣徵江
 淮之師,由三峽路入,以分蜀寇之力。事皆允從,由是甚見親信。二年春,杜黃裳出鎮,擢吉甫為中書侍郎、平章事。吉甫性聰敏,詳練物務,自員外郎出官,留滯江淮十五餘年,備詳閭里疾苦。及是為相,患方鎮貪恣,乃上言使屬郡刺史得自為政。敘進群材,甚有美稱。



 三年秋,裴均為僕射、判度支,交結權幸,欲求宰相。先是,制策試直言極諫科,其中有譏刺時政,忤犯權幸者,因此均黨揚言皆執政教指,冀以搖動吉甫,賴諫官李約、獨孤鬱、李
 正辭、蕭俛密疏陳奏,帝意乃解。吉甫早歲知獎羊士諤,擢為監察御史;又司封員外郎呂溫有詞藝,吉甫亦眷接之。竇群亦與羊、呂善。群初拜御史中丞,奏請士諤為侍御史,溫為郎中、知雜事。吉甫怒其不先關白,而所請又有超資者,持之數日不行,因而有隙。群遂伺得日者陳克明出入吉甫家,密捕以聞;憲宗詰之,無奸狀。吉甫以裴垍久在翰林,憲宗親信,必當大用,遂密薦垍代己,因自圖出鎮。其年九月,拜檢校兵部尚書,兼中書侍郎、
 平章事,充淮南節度使,上禦通化門樓餞之。在揚州,每有朝廷得失,軍國利害,皆密疏論列。又於高郵縣築堤為塘,溉田數千頃,人受其惠。



 五年冬,裴垍病免。明年正月,授吉甫金紫光祿大夫、中書侍郎、平章事、集賢殿大學士、監修國史、上柱國、趙國公。及再入相,請減省職員並諸色出身胥吏等,及量定中外官俸料,時以為當。京城諸僧有以莊磑免稅者,吉甫奏曰:「錢米所征,素有定額,寬緇徒有餘之力,配貧下無告之民,必不可許。」憲宗
 乃止。又請歸普潤軍於涇原。



 七年,京兆尹元義方奏:「永昌公主準禮令起祠堂,請其制度。」初,貞元中,義陽、義章二公主咸於墓所造祠堂一百二十間,費錢數萬;及永昌之制,上令義方減舊制之半。吉甫奏曰:「伏以永昌公主,稚年夭枉,舉代同悲,況於聖情,固所鐘念。然陛下猶減制造之半,示折衷之規,昭儉訓人,實越今古。臣以祠堂之設,禮典無文,德宗皇帝恩出一時,事因習俗,當時人間不無竊議。昔漢章帝時,欲為光武原陵、明帝顯節
 陵,各起邑屋,東平王蒼上疏言其不可。——東平王即光武之愛子,明帝之愛弟。賢王之心,豈惜費於父兄哉!誠以非禮之事,人君所當慎也。今者,依義陽公主起祠堂,臣恐不如量置墓戶,以充守奉。」翌日,上謂吉甫曰:「卿昨所奏罷祠堂事,深愜朕心。朕初疑其冗費,緣未知故實,是以量減。覽卿所陳,方知無據。然朕不欲破二十戶百姓,當揀官戶委之。」吉甫拜賀。上曰:「卿,此豈是難事!有關朕身,不便於時者,茍聞之則改,此豈足多耶!卿但勤匡正,
 無謂朕不能行也。」



 七年七月,上御延英,顧謂吉甫曰:「朕近日畋游悉廢,唯喜讀書。昨於《代宗實錄》中,見其時綱紀未振,朝廷多事,亦有所鑒誡。向後見卿先人事跡,深可嘉嘆。」吉甫降階跪奏曰:「臣先父伏事代宗,盡心盡節,迫於流運,不待聖時,臣之血誠,常所追恨。陛下耽悅文史,聽覽日新,見臣先父忠於前朝,著在實錄,今日特賜褒揚,先父雖在九泉,如睹白日。」因俯伏流涕,上慰諭之。



 八年十月,上御延英殿,問時政記記何事。時吉甫監修
 國史,先對曰:「是宰相記天子事以授史官之實錄也。古者,右史記言,今起居舍人是;左史記事,今起居郎是。永徽中,宰相姚璹監修國史,慮造膝之言,或不可聞,因請隨奏對而記於仗下,以授於史官,今時政記是也。」上曰:「間或不修,何也?」曰:「面奉德音,未及施行,總謂機密,故不可書以送史官;其間有謀議出於臣下者,又不可自書以付史官;及已行者,制令昭然,天下皆得聞知,即史官之記,不待書以授也。且臣觀時政記者,姚璹修之於長
 壽,及璹罷而事寢;賈耽、齊抗修之於貞元,及耽、抗罷而事廢。然則關時政化者,不虛美,不隱惡,謂之良史也。」



 是月,回紇部落南過磧,取西城柳谷路討吐蕃。西城防禦使周懷義表至,朝廷大恐,以為回紇聲言討吐蕃,意是入寇。吉甫奏曰:「回紇入寇,且當漸絕和事,不應便來犯邊,但須設備,不足為慮。」因請自夏州至天德,復置廢館一十一所,以通緩急。又請發夏州騎士五百人,營於經略故城,應援驛使,兼護黨項。九年,請於經略故城置宥
 州。六胡州以在靈鹽界,開元中廢六州。曰:「國家舊置宥州,以寬宥為名,領諸降戶。天寶末,宥州寄理於經略軍,蓋以地居其中,可以總統蕃部,北以應接天德,南援夏州。今經略遙隸靈武,又不置軍鎮,非舊制也。」憲宗從其奏,復置宥州,詔曰:「天寶中宥州寄理於經略軍,寶應已來,因循遂廢。由是昆夷屢擾,黨項靡依,蕃部之人,撫懷莫及。朕方弘遠略,思復舊規,宜於經略軍置宥州,仍為上州,於郭下置延恩縣,為上縣,屬夏綏銀觀察使。」



 淮西
 節度使吳少陽卒,其子元濟請襲父位。吉甫以為淮西內地,不同河朔,且四境無黨援,國家常宿數十萬兵以為守御,宜因時而取之。頗葉上旨,始為經度淮西之謀。



 元和九年冬,暴病卒,年五十七。憲宗傷悼久之,遣中使臨吊;常贈之外,內出絹五百匹以恤其家,再贈司空。吉甫初為相,頗洽時情,及淮南再徵,中外延望風採。秉政之後,視聽時有所蔽,人心疑憚之¨時負公望者慮為吉甫所忌,多避畏。憲宗潛知其事,未周歲,遂擢用李絳,大
 與絳不協;而絳性剛評,訐於上前,互有爭論,人多直絳。然性畏慎,雖其不悅者,亦無所傷。服物食味,必極珍美,而不殖財產,京師一宅之外,無他第墅,公論以此重之。有司謚曰敬憲;及會議,度支郎中張仲方駁之,以為太優。憲宗怒,貶仲方,賜吉甫謚曰忠懿。



 吉甫嘗討論《易象》異義,附於一行集注之下;及綴錄東漢、魏、晉、周、隋故事,訖其成敗損益大端,目為《六代略》,凡三十卷。分天下諸鎮,紀其山川險易故事,各寫其圖於篇首,為五十四卷,號
 為《元和郡國圖》。又與史官等錄當時戶賦兵籍,號為《國計簿》,凡十卷。纂《六典》諸職為《百司舉要》一卷。皆奏上之,行於代。子德修、德裕。



 李籓,字叔翰,趙郡人。曾祖至遠,天后時李昭德薦為天官侍郎,不詣昭德謝恩,時昭德怒,奏黜為壁州刺史。祖畬,開元時為考功郎中,事母孝謹,母卒,不勝喪死。至遠、畬皆以志行名重一時。父承,為湖南觀察使,亦有名。



 籓少恬淡修檢,雅容儀,好學。父卒,家富於財,親族吊者,有
 挈去不禁,愈務散施,不數年而貧。年四十餘未仕,讀書揚州,困於自給,妻子怨尤之,晏如也。杜亞居守東都,以故人子署為從事。洛中盜發,有誣牙將令狐運者,亞信之,拷掠竟罪。籓知其冤,爭之不從,遂辭出。後獲真盜宋瞿曇,籓益知名。



 張建封在徐州,闢為從事,居幕中,謙謙未嘗論細微。杜兼為濠州刺史,帶使職,建封病革,兼疾驅到府,陰有冀望。籓與同列省建封,出而泣語兼曰:「僕射公奄忽如此,公宜在州防遏,今棄州此來,欲何也?宜
 疾去!不若此,當奏聞。」兼錯愕不虞,遂徑歸。建封死,兼悔所志不就,怨籓甚。既歸揚州,兼因誣奏籓建封死時搖動軍中。德宗大怒,密詔杜佑殺之。佑素重籓,懷詔旬日不忍發,因引籓論釋氏,曰:「因報之事,信有之否?」籓曰:「信然。」曰:「審如此,君宜遇事無恐。」因出詔。籓覽之,無動色,曰:「某與兼信為報也。」佑曰:「慎勿出口,吾已密論,持百口保君矣。」德宗得佑解,怒不釋,亟追籓赴闕。及召見,望其儀形,曰:「此豈作惡事人耶!」乃釋然,除秘書郎。



 王紹持權,邀
 籓一相見即用,終不就。王仲舒、韋成季、呂洞輩為郎官,朋黨輝赫,日會聚歌酒,慕籓名,強致同會,籓不得已一至。仲舒輩好為訛語俳戲,後召籓,堅不去,曰:「吾與仲舒輩終日,不曉所與言何也。」後果敗。遷主客員外郎,尋換右司。時順宗冊廣陵王淳為皇太子,兵部尚書王純請改名紹,時議非之,皆云:「皇太子亦人臣也,東宮之臣改之宜也,非其屬而改之,諂也。如純輩豈為以禮事上耶!」籓謂人曰:「歷代故事,皆自不識大體之臣而失之,因不
 可復正,無足怪也。」及太子即位,憲宗是也。宰相改郡縣名以避上名,唯監察御史韋淳不改。既而有詔以陸淳為給事中,改名質;淳不得已改名貫之,議者嘉之。



 籓尋改吏部員外郎。元和初,遷吏部郎中,掌曹事,為使所蔽,濫用官闕,黜為著作郎。轉國子司業,遷給事中。制敕有不可,遂於黃敕後批之。吏曰:「宜別連白紙。」籓曰:「別以白紙,是文狀,豈曰批敕耶!」裴垍言於帝,以為有宰相器,屬鄭絪罷免,遂拜籓門下侍郎、同平章事。籓性忠藎,事無
 不言,上重之,以為無隱。



 四年冬,顧謂宰臣曰:「前代帝王理天下,或家給人足,或國貧下困,其故何也?」籓對曰:「古人云:『儉以足用。』蓋足用系於儉約。誠使人君不貴珠玉,唯務耕桑,則人無淫巧,俗自敦本,百姓既足,君孰與不足!自然帑藏充羨,稼穡豐登。若人君竭民力,貴異物,上行下效,風俗日奢,去本務末,衣食益乏,則百姓不足!君孰與足!自然國貧家困,盜賊乘隙而作矣!今陛下永鑒前古,思躋富庶,躬尚勤儉,自當理平。伏願以知之為非
 艱,保之為急務,宮室輿馬,衣服器玩,必務損之又損,示人變風,則天下幸甚。」帝曰:「儉約之事,是我誠心;貧富之由,如卿所說。唯當上下相勖,以保此道,似有逾濫,極言箴規,此固深期於卿等也。」籓等拜賀而退。



 帝又問曰:「禳災祈福之說,其事信否?」籓對曰:「臣竊觀自古聖達,皆不禱祠。故楚昭王有疾,卜者謂河為祟,昭王以河不在楚,非所獲罪,孔子以為知天道。仲尼病,子路請禱,仲尼以為神道助順,系於所行,己既全德,無愧屋漏。故答子路
 云:『丘之禱久矣。』《書》云:『惠迪吉,從逆兇。』言順道則吉,從逆則兇。《詩》云:『自求多福。』則禍福之來,咸應行事,若茍為非道,則何福可求?是以漢文帝每有祭祀,使有司敬而不祈,其見超然,可謂盛德。若使神明無知,則安能降福;必其有知,則私己求媚之事,君子尚不可悅也,況於明神乎!由此言之,則履信思順,自天祐之,茍異於此,實難致福。故堯、舜之德,唯在修己以安百姓。管仲云:『義於人者和於神。』蓋以人為神主,故但務安人而已。虢公求神,以
 致危亡,王莽妄祈,以速漢兵,古今明誡,書傳所紀。伏望陛下每以漢文、孔子之意為準,則百福具臻。」帝深嘉之。



 時河東節度使王鍔用錢數千萬賂遺權幸,求兼宰相。籓與權德輿在中書,有密旨曰:「王鍔可兼宰相,宜即擬來。」籓遂以筆塗「兼相」字,卻奏上云:「不可。」德輿失色曰:「縱不可,宜別作奏,豈可以筆塗詔耶!」曰:「勢迫矣!出今日,便不可止。日又暮,何暇別作奏!」事果寢。李吉甫自揚州再入相,數日,罷籓為詹事。後數月,上思籓,召對,復有所論
 列。元和六年,出為華州刺史、兼御史大夫。未行卒,年五十八,贈戶部尚書。籓為相材能不及裴垍,孤峻頗後韋貫之,然人物清規,亦其流也。



 權德輿,字載之,天水略陽人。父皋,字士繇,後秦尚書翼之後。少以進士補貝州臨清尉。安祿山以幽州長史充河北按察使,假其才名,表為薊縣尉,署從事。皋陰察祿山有異志,畏其猜虐,不可以潔退,欲潛去,又慮禍及老母。天寶十四年,祿山使皋獻戎俘,自京師回,過福昌。福
 昌尉仲謨,皋從父妹婿也,密以計約之。比至河陽,詐以疾亟召謨,謨至,皋示已喑,瞪謨而瞑。謨乃勉哀而哭,手自含襲,既逸皋而葬其棺,人無知者。從吏以詔書還,皋母初不知,聞皋之死,慟哭傷行路。祿山不疑其詐死,許其母歸。皋時微服匿跡,候母於淇門;既得侍其母,乃奉母晝夜南去,及渡江,祿山已反矣。由是名聞天下。淮南採訪使高適表皋試大理評事,充判官。屬永王璘亂,多劫士大夫以自從,皋懼見迫,又變名易服以免。玄宗在
 蜀,聞而嘉之,除監察御史。會丁母喪,因家洪州。時南北隔絕,或逾歲不聞詔命。有中使奉宣至洪州,經時未復,過有求取,州縣苦之。時有王遘為南昌令,將執按之,因見皋白其事;皋不言,久之,垂涕曰:「方今何由可致一敕使,而遽有此言。」因掩涕而起,遘遽拜謝之。浙西節度使顏真卿表皋為行軍司馬,詔徵為起居舍人,又以疾辭。嘗曰:「本自全吾志,豈受此之名耶!」李季卿為江淮黜陟使,奏皋節行,改著作郎,復不起。兩京蹂於胡騎,士君子
 多以家渡江東,知名之士如李華、柳識兄弟者,皆仰皋之德而友善之。大歷三年,卒於家,年四十六。元和中謚曰貞孝。



 初,皋卒,韓洄、王定為服朋友之喪,李華為其墓表,以為分天下善惡,一人而已。前贈秘書監,至是因子德輿為相,立家廟。至元和十二年,復贈太子太保。



 德輿生四歲,能屬詩;七歲居父喪,以孝聞;十五為文數百篇,編為《童蒙集》十卷,名聲日大。韓洄黜陟河南,闢為從事,試秘書省校書郎。貞元初,復為江西觀察使李兼判官,
 再遷監察御史。府罷,杜佑、裴胄皆奏請,二表同日至京。德宗雅聞其名,徵為太常博士,轉左補闕。八年,關東大水,上疏請降詔恤隱,遂命奚陟等四人使。



 裴延齡以巧幸判度支,九年,自司農少卿除戶部侍郎,仍判度支。德輿上疏曰:



 臣伏以爵人於朝,與眾共之,況經費之司,安危所系。延齡頃自權判,逮今間歲,不稱之聲,日甚於初。群情眾口,喧於朝市,不敢悉煩聖聽,今謹略舉所聞。多云以常賦正額支用未盡者,便為剩利,以為己功。又重
 破官錢買常平先所收市雜物,遂以再給估價,用充別貯利錢。又云邊上諸軍皆至懸闕,自今春已來,並不支糧。伏以疆場之事,所虞非細,誠聖謨前定,終事切有司。陛下必以延齡孤貞獨立,為時所抑,醜正有黨,結此流言,何不以新收剩利,徵其本末,為分析條奏?又擇朝賢信臣,與中使一人巡覆邊軍,察其資儲有無虛實。倘延齡受任已來,精心勤力,每事省約,別收羨餘,於正數各有區分,邊軍儲蓄,實猶可支,身自斂怨,為國惜費;自宜
 更示優獎,以洗群疑,明書厥勞,昭示天下。如或言者非謬,罔上實多,豈以邦國重務,委之非據!臣職在諫曹,合採群議,正拜已來,今已旬日,道路雲云,無不言此。豈京師士庶之眾,愚智之多,合而為黨,共有仇嫉。陛下亦宜稍回聖鑒,俯察群心。況臣之事君,如子事父;今當聖明不諱之代,若猶愛身隱情,是不忠不孝,莫大之罪。敢瀝肝血,伏待刑書。



 十年,遷起居舍人。歲中,兼知制誥。轉駕部員外郎、司勛郎中,職如舊。遷中書舍人。是時,德宗親
 覽庶政,重難除授,凡命於朝,多補自御札。始,德輿知制誥,給事有徐岱,舍人有高郢;居數歲,岱卒,郢知禮部貢舉,獨德輿直禁垣,數旬始歸。嘗上疏請除兩省宮,德宗曰:「非不知卿之勞苦,禁掖清切,須得如卿者,所以久難其人。」德輿居西掖八年,其間獨掌者數歲。貞元十七年冬,以本官知禮部貢舉。來年,真拜侍郎,凡三歲掌貢士,至今號為得人。轉戶部侍郎。元和初,歷兵部、吏部侍郎,坐郎吏誤用官闕,改太子賓客,復為兵部侍郎,遷太常
 卿。



 五年冬,宰相裴垍寢疾,德輿拜禮部尚書、平章事,與李籓同作相。河中節度王鍔來朝,貴幸多譽鍔者,上將加平章事,李籓堅執以為不可。德輿繼奏曰:「夫平章事,非序進而得,國朝方鎮帶宰相者,蓋有大忠大勛。大歷已來,又有跋扈難制者,不得已而與之。今王鍔無大忠勛,又非姑息之時,欲假此名,實恐不可!」上從之。



 運糧使董溪、於皋謨盜用官錢,詔流嶺南。行至湖外,密令中使皆殺之。他日,德輿上疏曰:



 竊以董溪等,當陛下憂山東
 用兵時,領糧料供軍重務,聖心委付,不比尋常;敢負恩私,恣其贓犯,使之萬死,不足塞責。弘寬大之典,流竄太輕,陛下合改正罪名,兼責臣等疏略。但詔令已下,四方聞知,不書明刑,有此處分,竊觀眾情,有所未喻。伏自陛下臨御已來,每事以誠,實與天地合德,與四時同符,萬方之人,沐浴皇澤。至如於、董所犯,合正典章,明下詔書,與眾同棄,即人各懼法,人各謹身。



 臣誠知其罪不容誅,又是已過之事,不合論辯,上煩聖聰。伏以陛下聖德聖
 姿,度越前古,頃所下一詔,舉一事,皆合理本,皆順人心。伏慮他時更有此比,但要有司窮鞫,審定罪名,或致之極法,或使自盡,罰一勸百,孰不甘心!巍巍聖朝,事體非細,臣每於延英奏對,退思陛下求理之言,生逢盛明,感涕自賀。況以愚滯樸訥,聖鑒所知,伏惟恕臣迂疏,察臣丹懇。



 及李吉甫自淮南詔征,未一年,上又繼用李絳。時上求理方切,軍國無大小,一付中書。吉甫、絳議政頗有異同,或於上前論事,形於言色;其有詣於理者,德輿亦
 不能為發明,時人以此譏之。竟以循默而罷,復守本官。尋以檢校吏部尚書為東都留守,後拜太常卿,改刑部尚書。先是,許孟容、蔣乂等奉詔刪定格敕。孟容等尋改他官,乂獨成三十卷,表獻之,留中不出。德輿請下刑部,與侍郎劉伯芻等考定,復為三十卷奏上。十一年,復以檢校吏部尚書出鎮興元。十三年八月,有疾,詔許歸闕,道卒,年六十。贈左僕射,謚曰文。



 德輿自貞元至元和三十年間,羽儀朝行,性直亮寬恕,動作語言,一無外飾,蘊
 藉風流,為時稱向。於述作特盛,《六經》百氏,游泳漸漬,其文雅正而弘博,王侯將相洎當時名人薨歿,以銘紀為請者什八九,時人以為宗匠焉。尤嗜讀書,無寸景暫倦,有文集五十卷,行於代。子璩,中書舍人。



 史臣曰:裴垍精鑒默識,舉賢任能,啟沃帝心,弼諧王道。如崔群、裴度、韋貫之輩,咸登將相,皆垍之薦達。立言立事,知無不為。吉甫該洽典經,詳練故實,仗裴垍之抽擢,致朝倫之式序。吉甫知垍之能別髦彥,垍知吉甫之善
 任賢良,相須而成,不忌不克。叔翰修身慎行,力學承家,批制敕有夕郎之風,塗御書見宰執之器;而乃輕財散施,天爵是期,偉哉,自待之意也!德輿孝悌力學,髫齔有聞,疏延齡恣行巧佞,論皋謨不書明刑,三十年羽儀朝行,實皋之餘慶所鐘。此四子者,所謂經緯之臣,又何慚於王佐矣!



 贊曰:二李秉鈞,信為名臣。甫柔而黨,籓俊而純。裴公鑒裁,朝無屈人。權之藻思,文質彬彬。



\end{pinyinscope}