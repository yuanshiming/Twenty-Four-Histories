\article{卷一百一十四}

\begin{pinyinscope}

 ○李光弼王思禮鄧景
 山辛云京



 李光弼,營州柳城人。其先,契丹之酋長。父楷洛,開元初,左羽林將軍同正、朔方節度副使,封薊國公,以驍果聞。光弼幼持節行,善騎射,能讀班氏《漢書》。少從戎,嚴毅有
 大略,起家左衛郎。丁父憂,終喪不入妻室。



 天寶初,累遷左清道率兼安北都護府、朔方都虞候。五載,河西節度王忠嗣補為兵馬使,充赤水軍使。忠嗣遇之甚厚,常云:「光弼必居我位。」邊上稱為名將。八載,充節度副使,封薊郡公。十一載,拜單于副使都護。十三載,朔方節度安思順奏為副使、知留後事。思順愛其材,欲妻之,光弼稱疾辭官。隴右節度哥舒翰聞而奏之,得還京師。祿山之亂,封常清、高仙芝戰敗,斬於潼關。又以哥舒翰率師拒賊。尋
 命郭子儀為朔方節度,收兵河西。玄宗眷求良將,委以河北、河東之事,以問子儀,子儀薦光弼堪當閫寄。十五載正月,以光弼為雲中太守,攝御史大夫,充河東節度副使、知節度事。二月,轉魏郡太守、河北道採訪使,以朔方兵五千會郭子儀軍,東下井陘,收常山郡。賊將史思明以卒數萬來援常山,追擊破之,進收槁城等十餘縣,南攻趙郡。三月八日,光弼兼範陽長史、河北節度使,拔趙郡。自祿山反,常山為戰場,死人蔽野,光弼酹其尸
 而哭之,為賊幽閉者出之,誓平寇難,以慰其心。六月,與賊將蔡希德、史思明、尹子奇戰於常山郡之嘉山,大破賊黨,斬首萬計,生擒四千。思明露發跣足,奔於博陵。河北歸順者十餘郡。



 光弼以範陽祿山之巢穴,將先斷之,使將絕根本。會哥舒翰潼關失守,玄宗幸蜀,人心驚駭。肅宗理兵於靈武,遣中使劉智達追光弼、子儀赴行在,授光弼戶部尚書,兼太原尹、北京留守、同中書門下平章事,以景城、河間之卒五千赴太原。時節度王承業軍政
 不修,詔御史崔眾交兵於河東。眾侮易承業,或裹甲持槍,突入承業事玩謔之。光弼聞之素不平。至是,交眾兵於光弼。眾以麾下來,光弼出迎,旌旗相接而不避。光弼怒其無禮,又不即次兵,令收系之。頃中使至,除眾御史中丞,懷其敕問眾所在。光弼曰:「眾有罪,系之矣!」中使以敕示光弼,光弼曰:「今只斬侍御史;若宣制命,即斬中丞;若拜宰相,亦斬宰相。」中使懼,遂寢之而還。翌日,以兵仗圍眾,至碑堂下斬之,威震三軍。命其親屬吊之。



 二
 年,賊將史思明、蔡希德、高秀巖、牛廷玠等四偽帥率眾十餘萬來攻太原。光弼經河北苦戰,精兵盡赴朔方,麾下皆烏合之眾,不滿萬人。思明謂諸將曰:「光弼之兵寡弱,可屈指而取太原,鼓行而西,圖河隴、朔方,無後顧矣!」光弼所部將士聞之皆懼,議欲修城以待之,光弼曰:「城周四十里,賊垂至,今興功役,是未見敵而自疲矣。」乃躬率士卒百姓外城掘壕以自固。作塹數十萬,眾莫知所用。及賊攻城於外,光弼即令增壘於內,環輒補之。賊
 城外詬詈戲侮者,光弼令穿地道,一夕而擒之,自此賊將行皆視地,不敢逼城。強弩發石以擊之,賊驍將勁卒死者十二三。城中長幼咸伏其勤智,懦兵增氣而皆欲出戰。史思明揣知之,先歸,留蔡希德等攻之。月餘,我怒而寇怠,光弼率敢死之士出擊,大破之,斬首七萬餘級,軍資器械一皆委棄。賊始至及遁,五十餘日,光弼設小幕,宿於城東南隅,有急即應,行過府門,未嘗回顧。賊退三日,決軍事畢,始歸府第。轉檢校司徒,收清夷、橫
 野等軍,擒賊將李弘義以歸。詔曰:「銀青光祿大夫、檢校司徒、兼戶部尚書、同中書門下平章事、兼御史大夫、鴻臚卿、太原尹、北京留守、河東節度副大使、薊國公光弼,全德挺生,英才間出,干城御侮,坐甲安邊。可守司空、兼兵部尚書、中書門下平章事,進封魏國公,食實封八百戶。」



 乾元元年,與關內節度使王思禮入朝,敕朝官四品已上出城迎謁。遷侍中,改封鄭國公。二年七月,制曰:「元帥之任,實屬於師貞;左軍之先,諒資於邦傑。自非道申啟沃,
 學富韜鈐,則何以翊分閫而專征,膺鑿門而受律。求諸將相,允得其人。司空、兼侍中、鄭國公光弼,器識弘遠,志懷沉毅,蘊孫、吳之略,有文武之材。往屬艱難,備彰忠勇,協風雲而經始,保宗社於阽危。由是出備長城,入扶大廈,茂功懸於日月,嘉績被於巖廊。屬殘寇猶虞,總戎有命,用擇惟賢之佐,式弘建親之典。必能緝寧邦國,協贊天人,誓於丹浦之師,剿彼綠林之盜。載明朝獎,爰籍舊勛。宜副出車之命,仍踐分麾之寵。為天下兵馬元帥趙
 王系之副,知節度行營事。」八月,兼幽州大都督府長史、河北節度支度營田經略等使,餘如故。與九節度兵圍安慶緒於相州,拔有日矣。史思明自範陽來救,屬絕糧道,光弼身先士卒,苦戰勝之。屬大風晦冥,諸將引眾而退,所在剽掠,唯光弼所部不散。東京留守崔圓、河南尹蘇震南奔襄陽,郭于儀率眾屯於谷水。史思明因殺安慶緒,即偽位,縱兵河南。加光弼太尉、兼中書令,代
 郭子儀為朔方節度、兵馬副元帥,以東師委之。左廂兵馬使張用濟承子儀之寬,懼光弼之令,與諸將頗有異議,欲逗留其眾。光弼以數千騎出次汜水縣,用濟單騎迎謁,即斬於轅門。諸將懾伏,都兵馬使僕固懷恩先期而至。



 初,光弼次汴州,聞思明悉眾且至,謂許叔冀曰:「大夫能守此城浹旬,我必將兵來救。」叔冀曰:「諾。」光弼還東京,思明至汴,叔冀與戰不利,遂與董秦、梁浦、劉從諫率眾降思明。賊勢甚熾,遣梁浦、劉從諫、田神功等將兵徇江淮,謂之曰:「收得其地,每人貢兩船玉帛。」思明乘勝而西。光
 弼整眾徐行,至洛,謂留守韋陟曰:「賊乘鄴下之勝,再犯王畿,按甲以挫其鋒,不利速戰。洛城非禦備之所,公計若何?」陟曰:「加兵陜州,退守潼關,據險以待之,足挫其銳矣!」光弼曰:「此蓋兵家常勢,非用奇之策也。夫兩軍相寇,貴進尺寸之間耳。今委五百里而不顧,是張賊勢也。若移軍河陽,北阻澤潞、三城以抗,勝則擒之,敗則自守,表裏相應,使賊不敢西侵,此則猿臂之勢也。夫辨朝廷之禮,光弼不如公;論軍旅之事,公不如光弼。」陟無以應。
 判官韋損曰:「東京帝宅,侍中何不守之?」光弼曰:「若守洛城,汜水、崿嶺皆須人守,子為兵馬判官,能守之乎?」遂移牒留守及河南尹並留司官、坊市居人,出城避寇,空其城,率軍士運油鐵諸物,以為戰守之備。時史思明已至偃師,光弼悉軍赴河陽。賊已至洛城,光弼軍方至石橋。日暮,令秉炬徐行,與賊相隨,而不敢來犯。乙夜,入河陽三城。排閱守備,號令嚴明,與士卒同甘苦,咸誓力戰。賊憚光弼威略,頓兵白馬寺,南不出百里,西不敢犯宮闕,
 於河陽南築月城,掘壕以拒光弼。十月,賊攻城。於中水單城西大破逆黨五千餘眾,斬首千餘級,生擒五百餘人,溺死者大半。



 初,光弼謂李抱玉曰:「將軍能為我守南城二日乎?」抱玉曰:「過期若何?」光弼曰:「過期而救不至,任棄也。」抱玉稟命,勒兵守南城,將陷,抱玉紿賊曰:「吾糧盡,明日當降。」賊眾大喜,斂軍以俟之。抱玉復得繕完設備,明日,堅壁請戰。賊怒見欺,急攻之。抱玉出奇兵,表裏夾擊,殺傷甚眾,賊帥周摯領軍而退。光弼自將於中水單城,城
 外置柵,柵外大掘塹,闊二丈,深亦如之。周摯舍南城,並力攻中水單。光弼命荔非元禮出勁卒於羊馬城以拒賊。光弼於城東北角樹小紅旗,下望賊軍。賊恃眾直逼其城,以車二乘載木鵝、蒙沖、斗樓、橦車隨其後,督兵填城下塹,三面各八道過其兵,又當塹開柵,各置一門。光弼遙望賊逼城,使人語荔非元禮曰:「中丞看賊填塹開柵過兵,居然不顧,何也?」元禮報曰:「太尉擬守乎,擬戰乎?」光弼曰:「戰。」元禮曰:「若戰,賊為我填塹,復何嫌也!」光弼曰:「吾
 智不及公,公其勉之!」元禮俟柵開,率其勇敢出戰,一逼賊軍,退走數百步。元禮料敵陣堅,雖出處馳突,不足破賊,收軍稍退,以怠其寇而攻之。光弼望見收軍,大怒,使人喚元禮,欲按軍令。元禮曰:「戰正忙,喚作何物?」良久,令軍中鼓噪出柵門,徒搏齊進,賊大潰。



 周摯復整軍押北城而下,將攻之。光弼遽率眾入北城,登城望曰:「彼雖眾,亂而囂,不足懼也。當為公等日午而破之。」命出將戰。及期,不決,謂諸將曰:「向來戰,何處最堅而難犯?」或曰:「西
 北角。」遽命郝玉曰:「爾往擊之。」玉曰:「玉,步卒也,請騎軍五百翼之。」光弼與之三百。又問:「何處最堅?」曰:「東南隅。」即命論惟貞以所部往擊之。對曰:「貞,蕃將也,不知步戰,請鐵騎三百。」與之百。光弼又出賜馬四十匹分給,且令之曰:「爾等望吾旗而戰,若麾旗緩,任爾觀望便宜;吾旗連麾三至地,則萬眾齊入,生死以之,少退者斬無舍。」玉策馬赴賊,有一人將援槍刺賊,洞馬腹,連刺數人;一人逢賊,不戰而退。光弼召不戰者斬,賞援槍者絹五百疋。須臾,郝
 玉奔歸。光弼望之,驚曰:「郝玉退,吾事危矣。」命左右取玉頭來。玉見使者曰:「馬中箭,非敢敗也。」使者馳報,光弼令換馬遣之。玉換馬復入,決死而前。光弼連麾,三軍望旗俱進,聲動天地,一鼓而賊大潰,斬萬餘級,生擒八千餘人,軍資器械糧儲數萬計,臨陣擒其大將徐璜玉、李秦授、周摯。其大將安太清走保懷州。思明不知摯等敗,尚攻南城。光弼悉驅俘囚臨河以示之,殺數十人以威之,餘眾懼,投河赴南岸,光弼皆斬之。初,光弼將戰,謂左右
 曰:「戰,危事,勝負系之。光弼位為三公,不可死於賊手,茍事之不捷,繼之以死。」及是擊賊,常納短刀於靴中,有決死之志,城上面西拜舞,三軍感動。賊既敗走,光弼收懷州,思明來救,迎擊於沁水之上,又敗之。城將安太清極力拒守,月餘不下。光弼令僕固懷恩、郝玉由地道而入,得其軍號,乃登陴大呼,我師同登,城遂拔。生擒安太清、周摯、楊希文等,送於闕下,即日懷州平。以功進爵臨淮郡王,累加實封至一千五百戶。



 觀軍容使魚朝恩屢言
 賊可滅之狀,朝旨令光弼速收東都。光弼屢表:「賊鋒尚銳,請候時而動,不可輕進。」僕固懷恩又害光弼之功,潛附朝恩,言賊可滅。由是中使督戰,光弼不獲已,進軍列陣於北邙山下。賊悉精銳來戰,光弼敗績,軍資器械並為賊所有。時李抱玉亦棄河陽,光弼渡河保聞喜。朝旨以懷恩異同致敗,優詔征之。光弼自河中入朝,抗表請罪,詔釋之。光弼懇讓太尉,遂加開府儀同三司、侍中、河南尹、行營節度使;俄復拜太尉,充河南、淮南、山
 南東道、荊南等副元帥,侍中如故,出鎮臨淮。史朝義乘邙山之勝,寇申、光等十三州,自領精騎圍李岑於宋州。將士皆懼,請南保揚州,光弼徑赴徐州以鎮之,遣田神功擊敗之。浙東賊首袁晁攻剽郡縣,浙東大亂。光弼分兵除討,克定江左,人心乃安。



 初,光弼將止臨淮,在道舁疾而行。監軍使以袁晁方擾江淮,光弼兵少,請保潤州以避其鋒。光弼曰:「朝廷寄安危於我,今賊雖強,未測吾眾寡,若出其不意,當自退矣。」遂徑往泗州。光弼未至河南也,田
 神功平劉展後,逗留於揚府,尚衡、殷仲卿相攻於兗、鄆、來瑱旅拒於襄陽,朝廷患之。及光弼輕騎至徐州,史朝義退走,田神功遽歸河南,尚衡、殷仲卿、來瑱皆懼其威名,相繼赴闕。寶應元年,進封臨淮王,賜鐵券,圖形凌煙閣。



 廣德初,吐蕃入寇京畿,代宗詔徵天下兵。光弼與程元振不協,遷延不至。十月,西戎犯京師,代宗幸陜。朝廷方倚光弼為援,恐成嫌疑,數詔問其母。吐蕃退,乃除光弼東都留守,以察其去就。光弼伺知之,辭以久待敕不
 至,且歸徐州,欲收江淮租賦以自給。代宗還京,二年正月,遣中使往宣慰。光弼母在河中,密詔子儀輿歸京師。其弟光進,與李輔國同掌禁兵,委以心膂。至是,以光進為太子太保、兼御史大夫、涼國公、渭北節度使,上遇之益厚。



 光弼御軍嚴肅,天下服其威名,每申號令,諸將不敢仰視。及懼朝恩之害,不敢入朝,田神功等皆不稟命,因愧恥成疾,遣衙將孫珍奉遺表自陳。廣德二年七月,薨於徐州,時年五十七。輟朝三日,贈太保,謚曰武穆。光
 弼既疾亟,將吏問以後事,曰:「吾久在軍中,不得就養,既為不孝子,夫復何言!」因取已封絹布各三千疋、錢三千貫文分給將士。部下護喪柩還京師。代宗遣中官開府魚朝恩吊問其母於私第,又命京兆尹第五琦監護喪事。十一月,葬於三原,詔宰臣百官祖送於延平門外。母李氏,有須數十莖,長五六寸,以子貴,封韓國太夫人,二子皆節制一品。光弼十年間三入朝,與弟光進在京師,雖與光弼異母,性亦孝悌,雙旌在門,鼎味就養,甲第並
 開,往來追歡,極一時之榮。



 王思禮,營州城傍高麗人也。父虔威,為朔方軍將,以習戰聞。思禮少習戎旅,隨節度使王忠嗣至河西,與哥舒翰對為押衙。及翰為隴右節度使,思禮與中郎周泌為翰押衙,以拔石堡城功,除右金吾衛將軍,充關西兵馬使,兼河源軍使。十一載,加雲麾將軍。十二載,翰徵九曲,思禮後期,欲引斬之,續使命釋之。思禮徐言曰:「斬則斬,卻喚何物?」諸將皆壯之。十三年,吐蕃蘇毗王款塞,
 詔翰至磨環川應接之。思禮墜馬損腳,翰謂中使李大宜曰:「思禮既損腳,更欲何之?」



 十四載六月,加金城太守。祿山反,哥舒翰為元帥,奏思禮加開府儀同三司,兼太常卿同正員,充元帥府馬軍都將,每事獨與思禮決之。十五載二月,思禮白翰謀殺安思順父元貞,於紙隔上密語翰,請抗表誅楊國忠,翰不應。復請以三十騎劫之,橫馱來潼關殺之,翰曰:「此乃翰反,何預祿山事。」六月,潼關失守,思禮西赴行在,至安化郡。思禮與呂崇賁、李承光
 並引於纛下,責以不能堅守,並從軍令。或救之可收後效,遂斬承光而釋思禮、崇賁,與房琯為副使。便橋之戰又不利,除為關內節度使。尋遣守武功。賊將安守忠及李歸仁、安泰清來戰,思禮以其眾退守扶風。賊兵分至大和關,去鳳翔五十里。王師大駭,鳳翔戒嚴,中官及朝官皆出其孥,上使左右巡御史虞候書其名,乃止。遂命司徒郭子儀以朔方之眾擊之而退。至德二年九月,思禮從元帥廣平王收西京,既破賊,思禮領兵先入景清
 宮。又從子儀戰陜城、曲沃、新店,賊軍繼敗,收東京。思禮又於絳郡破賊六千餘眾,器械山積,牛馬萬計。遷戶部尚書、霍國公,食實封三百戶。乾元二年,與子儀等九節度圍安慶緒於相州。思禮領關內及潞府行營步卒三萬、馬軍八千,大軍潰,唯思禮與李光弼兩軍獨全。及光弼鎮河陽,制以思禮為太原尹、北京留守、河東節度使、兼御史大夫,貯軍糧百萬,器械精銳。尋加守司空。自武德已來,三公不居宰輔,唯思禮而已。



 上元二年四月,以
 疾薨,輟朝一日,贈太尉,謚曰武烈,命鴻臚卿監護喪事。思禮長於支計,短於用兵,然立法嚴整,士卒不敢犯,時議稱之。



 鄧景山,曹州人也。文吏見稱。天寶中,自大理評事至監察御史。至德初,擢拜青齊節度使,遷揚州長史、淮南節度。為政簡肅,聞於朝廷。居職四年,會劉展作亂,引平盧副大使田神功兵馬討賊。神功至揚州,大掠居人資產,鞭笞發掘略盡,商胡大食、波斯等商旅死者數千
 人。



 上元二年十月,追入朝,拜尚書左丞。太原尹、北京留守王思禮軍儲豐實,其外又別積米萬石,奏請割其半送京師。屬思禮薨,以管崇嗣代之,委任左右,失於寬緩,數月之間,費散殆盡,唯存陳爛萬餘石。上聞之,即日召景山代崇嗣。及至太原,以鎮撫紀綱為己任,檢覆軍吏隱沒者,眾懼。有一偏將抵罪當死,諸將各請贖其罪,景山不許;其弟請以身代其兄,又不許;弟請納馬一匹以贖兄罪,景山許其減死。眾咸怒,謂景山曰:「我等人命輕如一
 馬乎?」軍眾憤怒,遂殺景山。上以景山統馭失所,不復驗其罪,遣使諭之。軍中因請以都知兵馬使、代州刺史辛云京為節度使,從之。



 辛云京者,河西之大族也。代掌戎旅,兄弟數人,並以將帥知名。云京有膽略,志氣剛決,不畏強御,每在戎行,以擒生斬馘為務。累建勛勞,官至北京都知兵馬使、代州刺史。鄧景山統馭失所,為軍士所殺,請云京為節度使,因授兼太原尹,以北門委之。云京質性沉毅,部下有犯
 令者,不貸絲毫,其賞功效亦如之,故三軍整肅。回紇恃舊勛,每入漢界,必肆狼貪。至太原,云京以戎狄之道待之,虜畏云京,不敢惕息。數年間,太原大理,無烽警之虞。累加檢校左僕射、同中書門下平章事。



 大歷三年八月庚午薨,上追悼發哀,為之流涕,冊贈太尉,輟朝三日,謚曰忠獻。後宰臣子儀、元載等見上,言及云京,泫然久之。十一月葬,命中使吊祭。時宰相及諸道節度使祭者凡七十餘幄。



 史臣曰:凡言將者,以孫、吳、韓、白為首。如光弼至性居喪,人子之情顯矣;雄才出將,軍旅之政肅然。以奇用兵,以少敗眾,將今比古,詢事考言,彼四子者,或有慚德。邙山之敗,閫外之權不專;徐州之留,郡側之人伺隙。失律之尤雖免,匪躬之義或虧,令名不全,良可惜也。然閫外之事,君側之人,得不慎諸?思禮法令嚴整,儲廩豐盈,節制之才,固不易得。景山始以文吏,或有虛名。仗鉞揚州,召匪人而劫掠士庶;分茅並部,持小法而全昧機權。貴馬
 賤人,眾怒身死,宜哉!云京賞善懲惡,靜亂安邊,功著軍中,寵加身後,不亦美歟!



 贊曰:光弼雄名,思禮刑清。始致亂者鄧景山,何以救之
 辛云京。



\end{pinyinscope}