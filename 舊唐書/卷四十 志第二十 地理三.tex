\article{卷四十 志第二十 地理三}

\begin{pinyinscope}

 ○淮南道六江南道七隴右道八



 淮南道



 揚
 州大都督府隋江都郡。武德三年,杜
 伏威歸國,於潤州江寧縣置揚州,以隋江都郡為兗州,置東南道行臺。七年,改兗州為邗州。九年,省江寧縣之揚州,改邗州為揚州。置大都督,督揚、和、滁、楚、舒、廬、壽七州。貞觀十年,改大都督為都督,督揚、滁、常、潤、和、宣、歙七州。龍朔二年,升為大都督府。天寶元年,改為廣陵郡,依舊大都督府。乾元元年,復為揚州。自後置淮南節度使,親王為都督,領使;長史為節度副大使,知節度事。恆以此為治所。舊領縣四:江都、六合、海陵、高郵、戶二萬三千一百九十九,口九萬四千三百四十七。天寶領縣七,戶七萬七千一百五,口四十六萬七千八百五十七。在京師東南二千七百五十三里,至東都一千七百四十九里。



 江都漢縣,屬廣陵國。隋為江都郡。武德三年,改為兗州,七年改為邗州,九年改為揚州都督府,皆以江都為治所



 江陽貞觀十八年,分江都縣置,
 在郭下,與江都分理



 六合漢堂邑縣,屬臨淮郡。晉置秦郡,北齊為秦州,後周為方州,隋改為兗州。武德七年,復為方州,置六合縣。又分六合置石梁縣。貞觀元年,省方州,並石梁入六合,屬揚州。



 海陵漢縣,屬臨淮郡。至隋,屬南袞州。武德二年,屬揚州,景龍二年,分置海安縣。開元十年省,並入海陵



 高郵漢縣,屬廣陵國,至隋不改。武德二年,屬兗州。州改,仍舊



 揚子永淳元年,分江都縣置



 天長天寶元年,割江都、六合、高郵三縣地置千秋縣,天寶七載,改為天長。



 楚州中隋江都郡之山陽縣。武德四年,臧君相歸附,立為東楚州,領山陽、安且、鹽城三縣。八年,廢西楚
 州,以盱眙來屬,仍去「東」字。天寶元年,改為淮陰郡。乾元元年,復為楚州。舊領縣四,戶三千三百五十七。口一萬六千二百六十二。天寶領縣五,戶二萬六千六十二,口十五萬三千。在京師西南二千五百一里,至東都一千六百六十里。



 山陽漢射陽縣地,屬臨淮郡。晉置山陽郡,改為山陽縣。武德四年,置東楚州。八年,去「東」字,治於此縣。縣東南有射陽湖



 鹽城漢鹽瀆縣地,屬臨淮郡。久無城邑,隋末,韋徹於此置射州,立射陽、安樂、新安三縣。武德四年
 歸國,因而不改。七年,廢射州及三縣,置鹽城縣於廢射州,仍屬楚州



 盱眙漢縣,屬臨淮郡。武德四年,置西楚州。置總管,管東楚、西楚。領盱眙一縣。八年,廢西楚州,以盱眙屬楚州。寶應漢平安縣,屬廣陵國。武德四年,置倉州,領安宜一縣。七年,州廢,縣屬楚州。肅宗上元三年建巳月,於此縣得定國寶十三枚,因改元寶應,仍改安宜為寶應



 淮陰乾封二年,分山陽縣置於隋舊廢縣。



 滁州下隋江都之清流縣。武德三年,杜伏威歸國,置滁州,又以揚州之全椒來屬。天寶元年,改為永陽郡。乾元元年,復為滁州。舊領縣二,戶四千六百八十九,
 口二萬
 一千五百三十五。天寶領縣三,戶二萬六千四百八十六,口十五萬二千三百七十四。在京師東南二千五百六十四里,至東都一千七百四十六里。



 清流漢全椒縣地,屬九江郡。梁置南譙州,居桑根山之朝陽,在今縣西南八十里南譙州故城是也。北齊自南譙故城經治於此新昌郡城,今州治是也。隋改南譙為滁州,後廢。武德三年復置,皆治於
 清流縣



 全椒漢舊縣名。梁北譙郡,又改為臨滁郡。隋改為滁縣,煬帝復為全椒。永陽景龍二年,分清流縣置。



 和州隋歷陽郡。武德三年,杜伏威歸國,改為和州。天寶元年,改為歷陽郡。乾元元年,復為和州。舊領縣二,戶五千七百三十,
 口三萬三千四百一。天寶領縣三,戶二萬四千七百九十四,口十二萬一千一十三。在京師東南二千六百八十三里,至東都一千八百一十一里。



 歷陽漢縣,屬九江郡。東晉置歷陽郡。宋為南豫州,北齊置和州。隋為歷陽郡。國初,復為和州。皆治此縣



 烏江漢東城縣之烏江亭,屬九江郡。北齊為密江郡,陳為臨江郡,後周為問江郡,隋為烏江郡,縣皆治此



 含山武德六年置,八年廢。長安四年復,為
 武壽縣。神龍元年,復為含山。



 濠州下隋為鐘離郡。武德三年,改為濠州。又改臨濠為定遠縣,化明為招義縣。領鐘離、塗山、定遠、招義四縣。武德四年,省塗
 山入鐘離。天寶元年,改為鐘離郡。乾元元年,復為濠州。舊領縣三,戶二千六百六十,口一萬三千八百五十五。天寶,戶二萬一千八百六十四,口十萬八千三百六十一。在京師東南二千一百五十里,至東都一千三百一十三里。



 鐘離漢縣,屬九江郡。晉、宋、齊、梁,置徐州。隋初為濠州,煬帝復為鐘離郡。武德三年,置濠州。皆治於此。武德
 七年,省塗山縣並入



 定遠漢曲陽縣地,屬九江郡。隋置定遠縣



 招義漢淮陵縣地,屬臨淮。宋置濟陰郡。武德七年,改為招義。



 廬州上隋廬江郡。武德三年,改為廬州,領合肥、廬江、慎三縣。七年,廢巢州
 為巢縣來屬。天寶元年,改為廬江郡。乾元元年,復為廬州,自中升為上。舊領縣四,戶五千三百五十八,口二萬七千五百一十三。天寶領縣五,戶四萬三千三百二十三,口二十萬五千三百九十六。在京師東南二千三百八十七里,至東都一千五百六十九里。



 合肥漢縣,屬九江郡。舊縣在北。夏水出城父東南,至此與肥水合,故曰合肥。梁置合州,隋初為廬江郡,皆治此縣



 慎漢逡遒縣,屬九江郡。古城在今縣南。隋為慎縣



 巢漢居巢縣,屬廬江郡。隋為襄安縣。武德三年,置巢州,分襄安立開城、扶陽二縣。七年,廢巢州及開城、扶陽
 二縣,改襄安為巢縣,屬廬州



 廬江漢郡名。漢龍舒縣地,屬廬江郡。梁置湖州,隋復舊也



 舒城開元二十三年,分合肥、廬江二縣置,取古龍舒縣
 為名。



 壽州中隋為淮南郡。武德三年,杜伏威歸國,改為壽州。七年,
 置
 都
 督府,督壽、蓼二州,領壽春、安豐、霍丘三縣。貞觀元年,廢都督府,又以廢霍州之霍山縣來屬。天寶元年,改為壽春郡,又置霍山縣。乾元元年,復為壽州。舊
 領縣四,戶二千九百九十六,口一萬四千七百一十八。天寶領縣五,戶三萬五千五百八十二,口十八萬七千五百八十七。在京師東南二千二百一十七里,至東都一千三百九里。



 壽春漢縣,屬九江郡。晉改為壽陽。晉於此置揚州,齊置豫州,後魏置揚州,梁復為豫州,後周置揚州。隋改壽州,煬帝為淮南郡,武德為壽州。皆以壽春為治所



 安豐漢六國,故城在縣南。梁置安豐郡。縣界有芍陂,灌
 萬頃,號安豐塘。隋因置縣



 霍山漢灊縣,屬廬江郡。隋置霍山應城三縣。貞觀元年,廢霍州,省應城、灊城二縣,以霍山屬壽州



 盛唐舊霍山縣。神功元年,改為武昌。神龍元年,復為霍山。開元二十七年,改為盛唐,仍移治於騶虞城



 霍丘漢松滋縣地,屬貫廬江郡。武德四年,置蓼州,領霍丘一縣。七年,蓼州廢,霍丘屬壽州。縣北有安豐津,斬毋丘儉處。



 光州緊中隋弋陽郡。武德三年,改為光州,置總管府,
 以定城縣為弦州,殷城縣為義州,以廢宋安郡為穀州,凡管光、弦、義、穀、廬五州。光州領光山、樂安、固始三縣。武德七年,改總管為都督府。貞觀元年,罷都督府,省弦州及義州,以定城、殷城二縣來屬。又省穀州,以宋安並入樂安。天寶元年,改為弋陽郡。乾元元年,復為光州。舊領縣五,戶五千六百四十九,口二萬八千二百九十一。天寶,戶三萬一千四百七十三,口十九萬八千五百八十。至京師一千八百五十五里,至東都九百二十五里。



 定城漢弋陽地,屬汝南郡。南齊為南弋陽縣,尋改為定城,武德三年,於縣置弦州,領定城一縣。貞觀元年,廢弦州,以定城屬光州,州所理也



 光山晉分弋陽置西陽縣,梁於縣置光州,隋為弋陽郡。武德三年,復為光州,治於光山縣。太極元年,移州理於定城



 仙居漢軑縣,屬江夏郡,古城在縣北十里。宋分軑縣置樂安縣。天寶元載,改為仙居



 殷城漢期思縣地,屬汝南郡。宋置苞信縣。隋改為殷城,取縣東古殷城為名。



 固始
 漢浸縣,屬汝南郡,後漢改為固始。



 蘄州中隋蘄春郡。武德四年,平硃粲,改為蘄州,領蘄春、蘄水、羅田、黃梅、浠水五縣。其年,省蘄水入蘄春,又分蘄春立永寧,省羅田入浠水。又改浠水為蘭溪,又於黃梅縣置南晉州。八年,州廢,以黃梅來屬。天寶元年,改為蘄春郡。乾元元年,復為蘄州。舊領縣四,戶一萬六百一十二,口三萬九千六百七十八。天寶,戶二萬六千八百九,口十八萬六千八百四十九。至京師二千五百六十
 里,至東都一千八百二十四里。



 蘄春漢縣,屬江夏郡。吳為蘄春郡。晉改為西陽,又改為蘄陽。周平淮南,改為蘄州。



 黃梅漢蘄春縣地。宋分置新蔡郡。隋改為黃梅。武德四年,置南晉州,領黃梅、義豐、長吉、塘陽、新蔡五縣。八年,廢州,仍省義豐等四縣,以黃梅來屬。



 廣濟漢蘄春縣地。武德四年,置永寧縣。天寶元年,改為廣濟縣。



 蘄水漢蘄春縣地。宋置浠水縣。武德四年,改為蘭溪。天寶元年,改為蘄水。



 申州中隋義陽郡。武德四年,置申州,領義陽、鐘山二縣。八年,省南羅州,又以羅山來屬。天寶元年,改為義陽郡。乾元元年,復為申州。舊領縣三,戶四千七百二十九,口二萬三千六十一。天寶,戶二萬五千八百六十四,口十四萬七千七百五十六。至京師一千七百九十六里,至東都九百四十三里。



 義陽漢平氏縣之義陽鄉,屬南陽郡。魏分南陽立義陽郡。晉自石城徙居於仁順,今州理也。宋置司州,後
 魏改為郢州,隋改為申州



 鐘山漢鄳縣地,屬江夏郡。隋改鐘山縣。羅山漢鄳縣地,隋為羅山縣。武德四年,置南羅州,領羅山一縣。八年廢,屬申州。



 黃州下隋永安郡。武德三年,改為黃州,置總管,管黃、蘄、亭、南司四州。黃州領黃岡、木蘭、麻城、黃陂四縣。其年,省木蘭縣,分黃岡置堡城縣,分麻城置陽城縣。仍於麻城縣置亭州,於黃陂縣置南司州。七年,廢南司州及亭州,縣並屬黃州。仍省堡城入黃岡。貞觀元年,罷都督府。天
 寶元年,改為齊安郡。乾元元年復為黃州。舊領縣三,戶四千八百九十六,口二萬二千六十。天寶,戶一萬五千五百一十二,口九萬六千三百六十八。在京師東南二千一百四十八里,至東都一千四百七十里。



 黃岡漢西陵縣地,江夏郡。北齊於舊城西南築小城,置衡州,領齊安一郡。隋改齊安為黃州,治黃岡



 黃陂漢西陵縣地。後周於古黃城西四十里獨家村置黃陂縣。武德三年,置南司州。七年,州廢,縣屬黃州。麻城
 漢西陵縣地。隋置麻城縣。武德三年,於縣置亭州,領麻城、陽城二縣。八年,州廢,仍省陽城入麻城,縣屬黃州。



 安州中都督府隋安陸郡。武德四年,平王世充,改為安州,領安陸、雲夢、應陽、孝昌、吉陽、應山、京山、富水八縣。其年,於應山縣置應州,領應山一縣。於孝昌縣置澴州,領孝昌一縣。以富水、京山二縣屬溫州。改應陽為應城縣。安州置總管,管澴、應二州。七年州廢,澴、應二州縣屬安州。改為大都督府,督安、申、陽、溫、復、沔、光、黃、蘄九州。六
 年,罷都督府。七年,又置,督安、隋、溫、沔、復五州。十二年,罷都督府。天寶元年,改為安陸郡,依舊為都督府,督安、隋、郢、沔四州。乾元元年,復為安州。舊領縣六,戶六千三百三十八,口二萬六千五百一十九。天寶,戶二萬二千二百二十一,口十七萬一千二百二。在京師東南二千五十一里,至東都一千一百九十里。



 安陸漢縣,屬江夏郡。宋分江夏立安陸郡。武德四年,改為安州,治於安陸。



 孝昌宋分安陸縣置。武德四
 年,置澴州,領孝昌、澴陽二縣。八年,州廢,以澴陽、孝昌屬安州。



 雲夢漢安陸縣地。後魏分安陸,於雲夢古城置雲夢縣。



 應城宋分安陸縣置應城縣,隋改為應陽。武德四年,復為應城。



 吉陽梁分安陸置平陽縣,後魏改為京池。隋改為吉陽,取山名。



 應山漢隋縣地,屬南陽郡。梁分隋縣置永陽縣。隋改為應山。以縣北山為名。



 舒州下隋同安郡。武德四年,改為舒州,領懷寧、宿松、
 太湖、望江、同安五縣。其年,割宿松置嚴州。五年,又割望江置高州,又改高州為智州。六年,舒州置總管府,管舒、嚴、智三州。七年,廢智州,望江屬嚴州,八年,又廢嚴州,以望江、宿松二縣來屬。貞觀元年,罷都督府。天寶元年,改為同安郡。至德二年二月,改盛唐郡。乾元元年,復為舒州。舊領縣五,戶九千三百六十一,口三萬七千五百三十八。天寶,戶三萬五千三百五十三,口十八萬六千三百九十八。在京師東南二千六百二十六里,至東京一
 千八百九十三里。



 懷寧漢皖縣地,晉於皖縣置懷寧縣。晉置晉熙郡。隋改為熙州,又為同安郡。武德四年,改為舒州,以懷寧為州治



 宿松漢皖縣地,梁置高塘郡。隋罷郡,置宿松縣。武德四年,置嚴州,領宿松一縣。七年,廢智州,以望江來屬。八年,廢嚴州,二縣來屬舒州



 望江漢皖縣地,晉置新治縣。陳於縣置大雷郡。隋改新治為義鄉,尋改為望江。武德四年,置高州,尋改為智州。七年,州廢,縣屬
 嚴州。八年,廢州,以縣屬舒州



 太湖漢皖縣地,宋置太湖縣



 同安漢樅陽縣,屬廬江郡。梁置樅陽郡。隋罷郡為同安縣,取界內古城名。



 江南道



 江南東道



 潤州上隋江都郡之延陵縣。武德三年,杜伏威歸國,置潤州於丹徒縣,改隋延陵縣為丹徒,移延陵還治故縣,屬茅州。六年,輔公祏反,復據其地。七年,平公祏,又置
 潤州,領丹徒縣。八年,廢簡州,以曲阿來屬。九年,揚州移理江都,以延陵、句容、白下三縣屬潤州。天寶元年,改為丹陽郡。乾元元年,復為潤州。永泰後,常為浙江西道觀察使理所。舊領縣五,戶二萬五千三百六十一,口十二萬七千一百四。天寶領縣六,戶十萬二千三十三,口六十六萬二千七百六。在京師東南二千八百二十一里,至東都一千七百九十七里。



 丹徒漢縣,屬會稽郡。春秋吳硃方之邑地,吳為京口
 戍。晉置南徐州。隋為延陵鎮,因改為延陵縣。尋以蔣州之延陵、永年,常州之曲阿三縣置潤州,東潤浦為名。皆治於丹徒縣。



 丹陽漢曲阿縣,屬會稽郡。又改名雲陽,後復為曲阿。武德五年,於縣置簡州。八年,州廢,縣屬潤州。天寶元年,改為丹陽縣,取漢郡名



 延陵漢曲阿縣地,晉分置延陵郡。隋移郡丹徒。武德三年,移於今所,屬茅州。七年,廢茅州,以縣屬蔣州。八年,改蔣州為揚州。九年,改屬潤州



 上元楚金陵邑,秦為秣陵。吳名
 建業,宋為建康。晉分秣陵置臨江縣,晉武改為江寧,武德三年,於縣置揚州,仍置東南道行臺,改江寧為歸化。六年,輔公祏反,據其地。七年,公祏平,置行臺尚書省,改揚州為蔣州。廢茅州,以句容二縣來屬蔣州。八年,罷行臺,改蔣州置揚州大都督府。改歸化縣為金陵。揚州領金陵、句容、丹陽、溧水六縣。九年,揚州移治江都,改金陵為白下縣。以延陵、句容、白下三縣屬潤州,丹陽、溧陽、溧水三縣屬宣州。移白下治故白下城。貞觀七年,復移今
 所。九年,改為江寧縣。至德二年二月,置江寧郡。乾元元年,於江寧置昇州,割潤州之句容江寧、宣州之當塗溧水四縣,置浙西節度使。上元二年,復為上元縣,還潤州。當塗等三縣各依舊屬



 句容漢縣,屬丹陽郡。武德四年,於縣置茅州,領句容。七年,州廢,以縣屬蔣州。九年,屬潤州。乾元元年,屬昇州。寶應元年州廢,屬潤州。金壇垂拱四年,分延陵縣置也。



 常州上隋毗陵郡。武德三年,杜伏威歸化,置常州,領
 晉陵、義興、無錫、武進四縣。六年,復陷輔公祏。七年,公祏平,復置常州,於義興置南興州。八年,州廢,義興來屬,省武進入晉陵。天寶元年,改為晉陵郡。乾元元年,復為常州。舊領縣四,戶二萬一千一百八十二,口十一萬一千六百六。天寶領縣五,戶十萬二千六百三十一,口六十九萬六百七十三。在京師東南二千八百四十三里,至東京一千九百八十三里。



 晉陵、漢毗陵縣,屬會稽郡,吳延陵邑也。晉改為晉陵
 郡。隋省郡,於常熟縣置常州。武德中,移於今治



 武進晉分曲阿縣置武進,梁改為蘭陵,隋廢。垂拱二年,又分晉陵置,治於州內



 江陰梁分蘭陵縣置。武德三年,於縣置暨州,領江陰、暨陽、利城三縣。九年,省暨陽、利城入江陰,屬常州



 義興漢陽羨縣,屬會稽郡。晉立義興郡及縣。武德七年,置南興州,領義興、陽羨、臨津三縣。八年,廢南興州及陽羨、臨津二縣,義興復隸常州



 無錫漢縣,屬會稽郡,隋屬常州。



 蘇州上隋吳郡,隋末陷賊。武德四年,平李子通,置蘇州。六年,又陷輔公祏。七年,平公祏,復置蘇州都督,督蘇、湖、杭、暨四州,治於故吳城,分置嘉興縣。八年,廢嘉興入吳縣。九年,罷都督。貞觀八年,復置嘉興縣。領吳城、昆山、嘉興、常熟四縣。天寶元年,改為吳郡。乾元元年,復為蘇州。舊領縣四,戶一萬一千八百五十九,口五萬四千四百七十一。天寶領縣六,戶七萬六千四百二十一,口六十三萬二千六百五十五。在京師東南三千一百九十
 九里,至東都二千五百里。



 吳春秋時吳都闔閭邑。漢為吳縣,屬會稽郡。隋平陳,置蘇州,取州西姑蘇山為名



 嘉興漢由拳縣,屬會稽郡。吳改嘉興,隋廢。武德七年,復置,屬蘇州。八年,廢入吳。貞觀八年,復置,屬蘇州



 昆山漢婁縣,屬會稽郡。梁分婁縣置信義縣。又分信義置昆山,取縣界山名。



 常熟晉分吳縣置海虞縣。梁改常熟縣。今昆山縣東一百三十里常熟故城是也。隋舊治南沙城,武德七年,
 移於今所治城。



 長洲萬歲通天元年,分吳縣置,在郭下,分治州界。



 海鹽漢縣,屬會稽郡。久廢。景雲二年,分嘉興縣復置。先天元年,復廢。開元五年,復置,治吳禦城。



 湖州上隋吳郡之烏程縣。武德四年,平李子通,置湖州,領烏程一縣。六年,復沒於輔公祏。七年平賊,復置,仍廢武州,以武康來屬。又省雉州,以長城縣來屬。天寶元年,改為吳興郡。乾元元年,復為湖州。舊領縣五,戶一萬
 四千一百三十五,口七萬六千四百三十。天寶領縣五,戶七萬三千三百六,口十七萬七千六百九十八。在京師東南三千四百四十一里,至東都二千六百四十四里。



 烏程漢縣,屬會稽部。梁置震州,取震澤名。隋改湖州,取州東太湖為名。皆治烏程。



 武康吳分烏程、餘杭二縣立永安縣,晉改為永康,又改為武康。武德四年,置武州,七年,州廢,縣屬湖州。



 長城晉分烏程置長
 城縣。武德四年,置雉州,領長城、原鄉二縣。七年,州廢及原鄉並入長城,屬湖州。



 安吉武德四年置,屬桃州。七年,廢入長城。麟德元年,復分長城縣置。



 德清天授二年,分武康置武原縣。景雲二年,改為臨溪。天寶元年,改為德清縣。



 杭州上隋餘杭郡。武德四年,平李子通,置杭州,領錢塘、富陽、餘杭三縣。六年,復沒於輔公祏。七年平賊,復置杭州。八年,廢潛州,以於潛縣來屬。貞觀四年,分錢塘置
 鹽官縣。天寶元年,改為餘杭郡。乾元元年,復為杭州。舊領縣五,戶三萬五百七十一,口十五萬三千七百二十。天寶領縣九,戶八萬六千二百五十八,口五十八萬五千九百六十三。在京師東南三千五百五十六里,至東都二千九百一十九里。



 錢塘漢縣,屬會稽郡。隋於餘杭縣置杭州,又自餘杭移州理錢塘。又移州於柳浦西,今州城是。貞觀六年,自州治南移於今所,去州十一里。又移治新城戍。開元二
 十一年,移治州郭下。二十五年,復還舊所。



 鹽官漢海鹽縣地,有鹽官,吳遂名縣。武德四年,屬東武州。七年,省入錢塘。貞觀四年,復分錢塘置。



 餘杭漢縣,屬會稽郡。隋置杭州,後徙治錢塘。



 富陽漢富春縣,屬會稽郡。晉改為富陽。隋舊縣。



 於潛漢縣,屬丹陽郡。武德七年,置潛州,領於潛、臨水二縣。八年,廢潛州及臨水縣,於潛還杭州。



 臨安垂拱四年,分餘杭、於潛,置於廢臨水縣。



 新城永淳元年,分富陽置。



 紫溪垂
 拱二年,分於潛置。萬歲通天元年,改為武隆。其年,依舊為紫溪。



 唐山萬歲通天元年,分紫溪,又別置武隆縣。神龍元年,改為唐山。



 越州中都督府隋會稽郡。武德四年,平李子通,置越州總管,管越、嵊、姚、鄞、浙、綱、衢、穀、麗、嚴、婺十一州。越州領會稽、諸暨、山陰三縣。七年,改總管為都督,督越、婺、鄞、嵊、麗五州。越州領會稽、諸暨、山陰、餘姚四縣。八年,廢鄞州為鄮縣,嵊州為剡縣,來屬。麗州為永康,屬婺州。省山陰
 縣。督越、婺二州。貞觀元年,更督越、婺、泉、建、臺、括六州。天寶元年,改越州為會稽郡。乾元元年,復為越州。舊領縣五,戶二萬五千八百九十,口十二萬四千一十。天寶領縣七,戶九萬二百七十九,口五十二萬九千五百八十九。在京師東南三千七百二十里,至東都二千八百七十里。



 會稽漢郡名。宋置東揚州,理於此,齊、梁不改。隋平陳,改東揚州為吳州。煬帝改為越州,尋改會稽郡,皆立於
 此縣



 山陰垂拱二年,分會稽縣置,在州治,與會稽分理



 諸暨漢縣,屬會稽郡。越王允常所都



 餘姚漢縣,屬會稽郡。隋廢。武德四年,復置,仍置姚州。七年,州廢,縣屬越州



 剡漢縣,屬會稽郡。武德四年,置嵊州及剡城縣。八年,廢嵊州及剡城,以剡縣來屬。



 蕭山儀鳳二年,分會稽、諸暨置永興縣。天寶元年,改為蕭山。



 上虞漢縣,屬會稽郡。



 明州上開元二十六年,於越州鄮縣置明州。天寶元
 年,改為餘姚郡。乾元元年,復為明州,取四明山為名。天寶領縣四,戶四萬二千二十七,口二十萬七千三十二。在京師東南四千一百里,至東都三千二百五十里。



 鄮漢縣,屬會稽郡。至隋廢。武德四年,置鄞州。八年,州廢為鄮縣,屬越州。開元二十六年,於縣置明州



 奉化,慈溪,翁山,已上三縣,皆鄮縣地。開元二十六年,析置。



 臺州上隋永嘉郡之臨海縣。武德四年,平李子通,置
 海州,領臨海、章安、始豐、樂安、寧海五縣。五年,改為臺州。六年,沒於輔公祏。七年平賊,仍置臺州,省寧海入章安。八年,廢始豐、樂安二縣入臨海。貞觀八年,復分置始豐。舊管二縣。永昌元年,置寧海縣。神龍二年,置象山縣。天寶元年,改為臨海郡。乾元元年,復為臺州。舊領縣二:臨海、始豐。戶六千五百八十三,口三萬五千三百八十三。天寶領縣六,戶八萬三千八百六十八,口四十八萬九千一十五。在京師東南四千一百七十七里,至東都三
 千三百三十里。



 臨海漢回浦縣,屬會稽郡。後漢改為章安。吳分章安置臨海縣。武德四年,於縣置臺州,取天臺山為名



 唐興吳始平縣,晉改始豐,隋末廢。武德四年,復置。八年,又廢。貞觀八年,復為始豐縣。上元二年,改為唐興



 黃巖上元二年,分臨海置



 樂安廢縣。上元二年,分臨海置,徙治孟溪



 寧海永昌元年,分臨海置



 象山神龍二年,分寧海及越州鄮縣置。



 婺州隋東陽郡。武德四年,平李子通,置婺州,領華川、長山二縣。七年,廢綱州,義烏來屬。八年,廢麗州為永康縣、衢州信安縣,並來屬。又廢穀州入信安,長山入金華縣。貞觀八年,復置龍丘縣。咸亨五年,置蘭溪、常山二縣。垂拱二年,分龍丘、信安、常山三縣置衢州,又置東陽縣。天授二年,又置武義縣。天寶元年,改婺州為東陽郡。乾元元年,復為婺州。舊領縣五,戶三萬七千八百一十九,口二十二萬八千九百九十。天寶領縣七,戶十四萬四
 千八十六,口七十萬七千一百五十二。在京師東南四千七十三里,至東都三千一百三十五里。



 金華漢烏傷縣,屬會稽郡。後漢分烏傷置長山縣。吳置東陽郡。隋改長山為金華,取州界山為名



 義烏晉分烏傷縣置。武德四年,置綱州,仍分置華川縣。七年,廢綱州及華川縣,改烏傷為義烏,以縣屬婺州



 永康吳分烏傷縣置。武德四年,置麗州,又分置縉雲縣。八年,廢麗州及縉雲縣,以永康來屬



 東陽垂拱二
 年,分烏傷縣,取舊郡名。蘭溪咸亨五年,析金華縣西界置,以溪水為名。武成天授二年,分永康置武義縣,又改為武成



 浦陽新置。



 衢州武德四年,平李子通,於信安縣置衢州。七年陷賊,乃廢。垂拱二年,分婺州之信安、龍丘置衢州,取武德廢州名。天寶元年,改為信安郡。乾元元年,復為衢州,又割常山入信州。天寶領縣五,戶六萬八千四百七十二,口四十四萬四百一十一。在京師東南四千七百十三
 里,至東都三千一百四十五里。



 信安後漢新安縣,晉改為信安。武德四年,置衢州,縣仍屬焉。又分置須江、定陽二縣。八年,廢衢州及須江、定陽二縣,以信安還屬婺州



 龍丘漢太末縣,屬會稽郡。晉置龍丘縣,以山為名。至隋廢。武德四年,置穀州及太末、白石二縣。八年,廢穀州及白石、太末二縣入信安縣。貞觀八年,分金華、信安二縣置龍丘縣,來屬婺州。垂拱二年,躭衢州



 須江武德四年,分信安置,以城南
 有須江。八年廢,永昌元年,分信安復置。



 盈川如意元年,分龍丘置,縣西有刑溪,陳時土人留異惡「刑」字,改名盈川,因以為縣名。



 常山咸亨五年,分信安置,屬婺州。垂拱二年,改屬衢州。乾元元年,屬信州,又還衢州。



 信州上乾元元年,割衢州之常山、饒州之弋陽、建州之三鄉、撫州之一鄉,置信州,又置上饒、永豐二縣。領縣四,戶四萬。在京師東南五千八百里,至東都二千九百五十里。



 上饒乾元元年置,州所理也。元和七年,省永豐縣入



 弋陽舊屬饒州,乾元元年,來屬。



 貴溪永泰元年十一月,分弋陽西界置。玉山證聖二年,分常山、須江置,屬衢州。乾元元年,割屬信州。



 睦州隋遂安郡。武德四年,平汪華,改為睦州,領雉山、遂安二縣。七年,廢嚴州之桐廬縣來屬,又改為東睦州。八年,去「東」字。舊管縣三,治雉山。萬歲登封二年,移治建德。天寶元年,改為新定郡。乾元元年,復為睦州。舊領縣
 三:雉山、遂安、桐廬。戶一萬二千六十四,口五萬九千六十八。天寶領縣六,戶五萬四千九百六十一,口三十八萬二千五百一大三。在京師東南三千六百五十九里,至東都二千八百三十一里。



 建德漢富春縣地,屬會稽郡。吳分置建德縣,隋廢。永淳二年,復分桐廬、雉山置。萬歲通天二年,移州治建德縣



 清溪漢歙縣地,屬丹陽郡。後分置新安縣,隋改為雉山。文明元年,復為新安。開元二十年,改為還淳。永
 貞元年十二月,避憲宗名,改為清溪。舊為睦州治所,移建德



 壽昌永昌元年七月,分雉山縣置。載初元年廢,神龍元年復。舊治白艾里,後移於今所。桐廬吳分富春縣置。武德四年,於縣置嚴州,領桐廬、分水、建德三縣。七年,廢州及分水、建德二縣。以桐廬屬睦州。舊治桐溪,開元二十六年,移治鐘山



 分水如意元年,分桐廬縣之四鄉,置武盛縣。神龍元年,改為分水



 遂安後漢分歙縣南鄉安定裏,置新定縣。晉改新定為遂
 安。



 歙州隋新安郡。武德四年,平汪華,置歙州總管,管歙、睦、衢三州。貞觀元年,罷都督府。天寶元年,改為新安郡。乾元元年,復為歙州。舊領縣三,戶六千二十一,口二萬六千六百一十七。天寶領縣五,戶三萬八千三百三十,口二十六萬九千一百九。在京師東南三千六百六十七里,至東都二千八百二十六里。



 歙漢縣,屬丹陽郡。縣南有歙浦,因為名。隋於縣置新
 安郡。武德改為歙州。



 休寧吳分歙縣置休陽縣,後改為海陽。晉武改為海寧,隋改為休寧。黟漢縣,屬丹陽郡。音同醫,縣南墨嶺山出石墨故也。縣置在黟川。



 績溪永徽五年,分置北野縣,後改為績溪。



 婺源開元二十八年正月九日置。



 處州隋永嘉郡。武德四年,平李子通,置括州,置總管府,管松、嘉、臺三州。括州領括蒼、麗水二縣。七年,改為都督府。八年,廢松州為松陽縣來屬。省麗水入括蒼。貞觀
 元年,廢都督府。省東嘉州,以永嘉、安固二縣來屬。天寶元年,改為縉雲郡。乾元元年,復為括州。大歷十四年夏五月,改為處州,避德宗諱。舊領縣四,戶一萬二千八百九十九,口十萬一千六百六。天寶領縣五,戶四萬二千九百三十六,口二十五萬八千二百四十八。今縣六。在京師東南四千二百七十八里,至東都三千一十五里。



 麗水漢回浦縣地,屬會稽郡。光武更為章安。隋平陳,改永嘉郡為處州,尋改為括州,又分松陽縣東界置括
 蒼縣。大歷十四年夏,改為麗水縣,州所治



 松陽後漢分章安之南鄉置松陽縣,縣東南大陽及松樹為名



 縉雲萬歲登封元年,分括蒼及婺州永康縣置



 青田景雲二年,分括蒼置



 遂昌舊縣。武德八年,並入松陽。景雲二年,分松陽縣復置



 龍泉乾元二年,越州刺史獨孤嶼奏請於括州龍泉鄉置縣,以龍泉為名,從之。



 溫州上隋永嘉郡之永嘉縣。武德五年,置東嘉州,領永
 嘉、永寧、安固、樂成、橫陽五縣。貞觀元年,廢東嘉州,以縣屬括州。上元二年,分括州之永嘉、安固二縣置溫州。天寶元年,改為永嘉郡。乾元元年,復為溫州。天寶領縣四,戶四萬二千八百一十四,口二十四萬一千六百九十四。在京師東南四千七百三十七里,至東都三千九百四十里。



 永嘉後漢分章安縣之東甌鄉置永寧縣,屬會稽郡。晉置永嘉郡。隋改為永嘉。上元二年,置溫州,治於北縣



 安固後漢章安縣,晉改為安固,隋廢。武德八年,分永嘉縣置,屬東嘉州。貞觀元年,廢東嘉州,安固屬括州。上元元年,屬溫州



 橫陽武德五年,分安固縣置。貞觀元年廢,大足元年,復分安固置



 樂城武德五年置,七年並入永嘉縣。載初元年,分永嘉復置也。



 福州中都督府隋建安郡之閩縣。貞觀初,置泉州。景雲二年,改為閩州,置都督府,督閩、泉、建、漳、湖五州。開元十三年,改為福州,依舊都督府,仍置經略使。二十二年,罷
 漳、湖二州,令督福、建、泉、汀四州。舊屬嶺南道,天寶初,改屬江南東道。尋改為長樂郡。乾元元年,復為福州都督府。天寶領縣八,戶三萬四千八十四,口七萬五千八百七十六。在京師東南五千三十三里,至東都四千二百三十三里。



 閩漢治縣,屬會稽郡。秦時為閩中郡。漢高立閩越王,都於此。武帝誅東越,徙其人於江淮,空其地。其逃亡者,自立為冶縣,後更名東冶縣。後漢改為侯官都尉,屬會
 稽郡。晉置晉安郡。宋、齊因之,陳置閩州,又改為豐州。隋平陳改為泉州,煬帝改為閩州,又為建安郡。開元十三年,改為福州。皆治閩縣



 侯官隋縣。後廢。長安二年,又分閩縣置



 長樂隋縣。後省。武德六年,分閩縣置新寧縣。其年,改為長樂



 福唐聖歷二年,分長樂置萬安縣。天寶元年,改為福唐



 連江武德六年,分閩縣置溫麻縣。其年,改為連江



 長溪武德六年置,其年並入連江。長安二年,分連江復置



 古田開元二
 十九年,開山洞置



 永泰永泰年分置



 梅青新置。



 泉州中隋建安郡,又為泉州。舊治閩縣,後移於南安縣。聖歷二年,分泉州之南安、莆田、龍溪三縣,置武榮州。三年,州廢,三縣還泉州。久視元年,又以三縣置武榮州。景雲二年,改為泉州。開元二十九年,割龍溪屬漳州。天寶元年,改泉州為清源郡。乾元元年,復為泉州。天寶領縣四,戶二萬三千八百六,口十六萬二
 百九十五。在京師東南六千二百一十六里,至東都五千四百一十三里。



 晉江開元八年,分南安置,今為州之治所



 南安隋縣。武德五年,置豐州,領南安、莆田二縣。貞觀元年,廢豐州,縣屬泉州。聖歷二年,屬武榮州。州廢來屬



 莆田武德五年,分南安縣置,屬豐州。州廢來屬。



 仙游聖歷二年,分莆田置清源縣。天寶元年,改為仙游。



 建州中隋建安郡之建安縣。武德四年,置建州,領綏
 城、唐興、建陽、沙、將樂、邵武等縣。天寶元年,改為建安郡。乾元元年,復為建州。舊領縣二,戶一萬五千三百三十六,口二萬二千八百二十。天寶領縣六,戶二萬七千二百七十,口一十四萬三千七百七十四。在京師東南四千九百三十五里,至東都三千八百八十八里。



 建安漢冶縣地。吳置建安縣,州所治,以建溪為名



 邵武隋縣



 浦城載初元年,分建安縣置唐興縣。天授二年,改為武寧。神龍元年,復為唐興。天寶元年,改
 為浦城



 建陽隋廢縣。垂拱四年,分建安置



 將樂隋廢縣。垂拱四年五月,分邵武復置



 沙隋廢縣。永徽六年,分建安置。



 汀州下開元二十四年,開福、撫二州山洞,置汀州。天寶元年,改為臨汀郡。乾元元年,復為汀州。天寶領縣三,戶四千六百八十,口一萬三千七百二。在京師東南六千一百七十三里,至東都五千三百七十里。



 長汀州治所。龍巖寧化已上三縣,並開元二
 十四年開山洞置。



 漳州垂拱二年十二月九日置。天寶元年,改為漳浦郡。舊屬嶺南道,天寶割屬江南東道。乾元元年,復為漳州。天寶領縣二,戶五千三百四十六,口一萬七千九百四十。在京師東南七千三百里,至東都六千五百里。



 漳浦垂拱二年十二月,與州同置。州所治



 龍溪舊屬泉州。聖歷二年,屬武榮州。景雲二年,還泉州。開元二十九年,屬漳州。



 江南西道



 宣州隋宣城郡。武德三年,杜伏威歸化。置宣州總管府。分宣城置懷安、寧國二縣。六年,陷輔公祏。七年賊平,改置宣州都督,督宣、潛、猷、池四州,廢桃州,以綏安來屬,省懷安、寧國二縣。宣州領宣城、綏安二縣。八年,廢南豫州,以當塗來屬,廢猷州,以涇縣來屬。九年,移揚州於江都,以溧陽、溧水、丹陽來屬。貞觀元年,罷都督府。廢池州,以秋浦、南陵二縣來屬。省丹陽入當塗縣。開元中,析置青陽、太
 平、寧國三縣,天寶元年,改為宣城郡。至德二年,又析置至德縣。乾元元年,復為宣州。永泰元年,割秋浦、青陽、至德三縣置池州。舊領縣八,戶二萬二千五百三十七,口九萬五千七百五十三。天寶領縣九,戶一十二萬一千二百四,口八十八萬四千九百八十五。今縣十。在京師東南三千五百五十一里,至東都二千五百一十里。



 宣城漢宛陵縣,屬丹陽郡。秦屬鄣郡。梁置南豫州,隋改為宣州,煬帝又為宣城郡,皆此治所



 當塗漢丹
 陽縣地,屬丹陽郡。晉分丹陽置於湖縣。成帝以江北當塗縣流人寓居於湖,乃改為當塗縣,屬宣州。牛渚山,一名採石,在縣北四十五里大江中。武德三年,置南豫州,以縣屬。八年,省南豫州,縣屬宣州



 涇漢涇縣,屬丹陽郡。武德三年,置猷州,領涇、南陽、安吳三縣。八年,廢猷州及南陽、安吳二縣。屬宣州。縣界有陵陽山



 廣德漢故鄣縣,屬丹陽郡。宋分宣城之廣德、吳興之故鄣,置綏安縣。至德二年九月,改為廣德,以縣界廣德故城為
 名



 溧陽漢縣,屬丹陽郡。上元元年十一月,割屬昇州。州廢來屬



 溧水漢溧陽地。隋為縣。武德三年,屬揚州。九年,屬宣州。乾元元年,屬昇州。州廢還屬



 南陵漢春穀縣地,屬丹陽郡。梁置南陵縣。武德七年,屬池州。州廢來屬。舊治赭圻城,長安四年,移理青陽城



 太平天寶十一載正月,析涇縣置



 寧國隋縣。武德六年廢,天寶三載復置



 旌德寶應二年二月,析太平縣置。



 池州下隋宣城郡之秋浦縣。武德四年,置池州,領秋浦、南陵二縣。貞觀元年,廢池州,以秋浦屬宣州。永泰元年,江西觀察使李勉以秋浦去洪州九百里,請復置池州,仍請割青陽、至德二縣隸之,又析置石埭縣,並從之。後隸宣州。領縣四,戶一萬九千,口八萬七千九百六十七。



 秋浦州所治。漢石城縣,屬丹陽郡。隋分南陵置秋浦縣,因水為名



 青陽天寶元年,分涇、南陵、秋浦三
 縣置,治古臨城



 至德至德二年析置



 石埭永泰二年,割秋浦、浮梁、黟三縣置,治古石埭城。



 饒州下隋鄱陽郡。武德四年,平江左,置饒州,領鄱陽、新平、廣晉、餘干、樂平、長城、玉亭、弋陽、上饒九縣。七年,省上饒入弋陽,省玉亭入長城、餘幹二縣。八年,又並長城入餘干,並新平、廣晉入鄱陽。舊領縣四,戶一萬一千四百,口五萬九千八百一十七。天寶,戶四萬八百九十九,口二十四萬四千三百五十。在京師東南三千二百六
 十三里,至東都二千四百一十三里。



 鄱陽漢縣,屬豫章郡。古城在今縣東界,有鄱江,今為州所理



 餘干漢餘干縣屬豫章郡。古所謂汗越也。汗音干,隋朝去「水」。樂平武德中置,九年省,後重置



 浮梁武德中,廢新平縣。開元四年,分鄱陽置,後改新昌。天寶元年復置。



 洪州上都督府隋豫章郡。武德五年,平林士弘,置洪州總管府,管洪、饒、撫、吉、虔、南平六州,分豫章置鐘陵縣。
 洪州領豫章、豐城、鐘陵三縣。八年,廢孫州、南昌州、米州,以南昌、建昌、高安三縣來屬。省鐘陵、南昌二縣入豫章。貞觀二年,加洪、饒、撫、吉、虔、袁、江、鄂等八州。顯慶四年,督饒、鄂等州。洪州舊領縣四,永淳二年,置新吳縣。長安四年,置武寧縣,又督洪、袁、吉、虔、撫五州。天寶元年,改為豫章郡。乾元元年,復為洪州。舊領縣四:豫章、豐城、高安、建昌。戶一萬五千四百五十六,口七萬四千四十四。天寶領縣六,戶五萬五千五百三十,口三十五萬三千二百三十
 一。在京師東南三千九十里,至東都二千二百一十一里。



 鐘陵漢南昌縣,豫章郡所治也。隋改為豫章縣,置洪州,煬帝復為豫章郡。寶應元年六月,以犯代宗諱,改為鐘陵,取地名



 豐城吳分南昌縣置富城縣,晉改為豐城



 高安漢建城縣,屬豫章郡。武德五年,改為高安,仍置靖州,領高安、望蔡、華陽三縣。七年,改靖州為米州。其年,又改為筠州。八年,廢筠州,省華陽、望蔡二縣,以高
 安屬洪州



 建昌漢海昏縣,屬豫章郡。後漢分立建昌。武德五年,分置南昌州總管府,管南昌、西吳、靖、米、孫五州。南昌州領建昌、龍安、永修三縣。七年,罷都督為南昌州。八年,廢南昌州及孫州,以南昌州新吳、永修、龍安入建昌縣,以孫州之建昌入豫章縣,而以建昌屬洪州



 新吳舊廢縣。永淳二年,分建昌置



 武寧長安四年,分建昌置武寧縣。景雲元年,改為豫寧。寶應元年,復為武寧



 分寧貞元十六年二月置。



 虔州中隋南康郡。武德五年,平江左,置虔州。天寶元年,改為南康郡。乾元元年,復為虔州。舊領縣四,戶八千九百九十四,口三萬九千九百一。天寶領縣六,戶三萬七千六百四十七,口二十七萬五千四百一十。今縣七。在京師東南四千一十七里,至東都三千四百里。



 贛古濫反



 州所理。漢縣,屬豫章郡。漢分豫章立廬陵郡,晉改為南康郡。隋初為虔州,煬帝為南康郡。皆治贛



 虔化吳分贛立陽都縣,晉改為寧都。隋平陳,改為虔化,屬
 虔州



 南康漢南野縣,屬豫章郡。吳分南野立南安縣,晉改為南康



 雩都漢縣,屬豫章郡



 信豐永淳元年,分南康置南安縣。天寶元年,改為信豐



 大庾神龍元年,分南康置



 安遠貞元四年八月四日置。



 撫州中隋臨川郡。武德五年,討平林士弘,置撫州,領臨川、南城、郡武、宜黃、崇仁、永城、東興、將樂八縣。七年,省東興、永城、將樂三縣,以邵武隸建州。八年,省宜黃縣。天
 寶元年,改為臨川郡。乾元元年,復為撫州。舊領縣三,戶七千三百五十四,口四萬六百八十五。天寶領縣四,戶三萬六百五,口十七萬六千三百九十四。在京師東南三千三百一十二里,至東都二千五百四十里。



 臨川州所理。漢南昌縣地。後漢分南昌置臨汝縣。吳置臨川郡,歷南朝不改。隋平陳,改臨川郡為撫州,仍改臨汝縣為臨川縣。州郡所理,皆此縣



 南城漢縣,屬豫章郡。開元八年,分南城置



 崇仁吳分臨汝置新
 建縣。梁改為巴山縣,仍僑置巴山郡。隋平陳,改巴山為崇仁縣



 南豐開元八年,分南城置。



 吉州上隋廬陵郡。武德五年,討平林士弘,置吉州,領廬陵、新淦二縣。七年,廢潁州,以安福縣來屬。八年,廢南平州,以太和縣來屬。天寶元年,改為廬陵郡。乾元元年,復為吉州。舊領縣四,戶一萬五千四十,口五萬三千二百八十五。天寶領縣五,戶三萬七千七百五十二,口二十三萬七千三十二。



 廬陵漢縣,屬豫章郡。後漢改為西昌。隋復為廬陵,州所治也。舊治子陽城,永淳元年,移於今所



 太和隋縣。武德五年,置南平州,領太和、永新、廣興、東昌四縣。八年,廢南平州,以永新等三縣並太和,屬吉州



 安福吳置安成郡於此。隋廢為安復,後改為安福



 新淦漢舊縣,屬豫章郡。淦,音紺,又音甘。



 永新廢縣。顯慶二年,分太和置。



 江州中隋九江郡。武德四年,平林士弘,置江州,領湓
 城、潯陽、彭澤三縣。五年,置總管,管江、鄂、智、浩四州,並管昌、洪四總管府。又分湓城置楚城縣,分彭澤置都昌縣。八年,廢浩州及樂城縣入彭澤縣,又廢湓城入潯陽。貞觀元年,罷都督府。八年,廢楚城縣入潯陽。天寶元年,改為潯陽郡。乾元元年,復為江州。舊領縣三,戶六千三百六十,口二萬五千五百九十九。天寶,戶二萬九千二十五,口十五五千七百四十四。在京師東南二千九百四十八里,至東都二千一百九十七里。



 潯陽州所理。漢縣,屬廬江郡。晉置江州。隋改為彭蠡縣,取州東南五十二里有彭蠡湖為名。煬帝改為湓城,取縣界湓水為名。武德四年,復為潯陽,潯水至此入江為名



 都昌武德五年,分彭澤置,屬浩州。八年,廢浩州,縣屬江州



 彭澤漢縣,屬豫章郡。隋為龍城縣。武德五年,置浩州,又分置都昌、樂城二縣。八年,罷浩州,以彭澤屬江州,仍省樂城入彭澤。至德二年九月,中丞宋若思奏置。



 袁州下隋宜春郡。武德四年,平蕭銑,置袁州。天寶元年,改為宜春郡。乾元元年,復為袁州。舊領縣三,戶四千六百三十六,口二萬五千七百一十六。天寶,戶二萬七千九十一,口一十四萬四千九十六。在京師東南三千五百八十里,至東都二千一百六十一里。



 宜春州所理。漢縣,屬豫章郡。吳為安成郡,南朝不改。晉改為宜陽。隋置袁州,煬帝為宜春郡。復改為宜春。宜春,泉水名,在州西。取此水為酒,作貢。



 萍鄉吳分宜
 春置萍鄉縣,屬安成郡。



 新喻吳分宜春置新喻,屬安成郡。



 鄂州上隋江夏郡。武德四年,平蕭銑,改為鄂州。天寶元年,改為江夏郡。乾元元年,復為鄂州。永泰後,置鄂岳觀察使,領鄂、岳、蘄、黃四州,恆以鄂州為使理所。舊領縣四,戶三千七百五十四,口一萬四千六百一十五。天寶領縣五,戶一萬九千一百九十,口八萬四千五百六十三。後並沔州入鄂州,以漢陽、水義川來屬。在京師東南二
 千三百四十六里,至東都一千五百三十里。



 江夏漢郡名。本漢沙羨縣地,屬江夏郡。晉改沙羨為沙陽。江、漢二水會於州西,春秋謂之夏汭,晉、宋謂之夏口。宋置江夏郡。治於此。隋不改。武德四年,改為鄂州,取漢縣名



 永興漢鄂縣地,屬江夏郡。吳分鄂置新陽縣,隋改為永興



 武昌漢鄂縣,屬江夏郡。吳、晉為重鎮,以名將為鎮守



 蒲圻吳分沙羨縣置



 唐年天寶二年,開山洞置



 漢陽漢安陸縣地,屬江夏郡。
 晉置沌陽縣。隋初為漢津縣,煬帝改為漢陽。武德四年,平硃粲,分沔陽郡置沔州,治漢陽縣。貞觀,戶一千五百一十七,口六千九百五十九。至太和七年,鄂岳節度使牛僧孺奏,沔州與鄂州隔江,都管一縣,請並入鄂州,從之。舊屬淮南道。



 水義川漢安陸縣地,後魏置水義川郡。武德四年,分漢陽縣置水義川縣,屬沔州。州廢,屬鄂州。



 岳州下隋巴陵郡。武德四年,平蕭銑,置巴州,領巴陵、華容、沅江、羅、湘陰五縣。六年,改為岳州。省羅縣。天寶元
 年,改為巴陵郡。乾元元年,復為岳州。舊領縣四,戶四千二,口一萬七千五百五十六。天寶領縣五,戶一萬一千七百四十,口五萬二百九十八。在京師東南二千二百三十七里,至東都一千八百一十六里。



 巴陵漢下雋縣,屬長沙郡。吳置巴陵縣。晉置建昌郡,隋改為巴州,煬帝改為巴陵郡。武德置岳州,皆置巴陵縣。縣界有古巴丘



 華容漢孱陵縣地,屬武陵郡,劉表改為安南。隋改為華容。垂拱二年,去「華」字,曰容城。神
 龍元年,復為華容



 沅江漢益陽縣,屬長沙國。隋改為安樂,又改為沅江,屬岳州



 湘陰漢羅縣,屬長沙國。宋置湘陰縣,縣界汨水,注入湘江



 昌江神龍三年,分湘陰縣置。



 潭州中都督府隋長沙郡。武德四年,平蕭銑,置潭州總管府,管潭、衡、永、郴、連、南梁、南雲、南營八州。潭州領長沙、衡山、醴陵、湘鄉、益陽、新康六縣。七年,廢雲州,改南梁為邵州,南營為道州。省新康縣。督潭、衡、郴、連、永、邵、道等
 七州。天寶元年,改為長沙郡。乾元元年,復為潭州。舊領縣五,戶九千三十一,口四萬四百四十九。天寶領縣六,戶三萬二千二百七十二,口十九萬二千六百五十七。在京師南二千四百四十五里,至東都二千一百八十五里。



 長沙秦置長沙郡。漢為長沙國,治臨湘縣。後漢為長沙郡。吳不改。晉懷帝置湘州,至梁初不改。隋平陳,為潭州,以昭潭為名。煬帝改為長沙郡,仍改臨湘為長沙縣。
 武德復為潭州。湘潭後漢湘南縣地,屬長沙郡。吳分汀南立衡陽縣,屬衡陽郡。隋廢郡,縣屬潭州。天寶八年,移治於洛口,因改為湘潭縣。



 湘鄉漢鐘武縣,屬零陵郡。後漢改為重安,永建三年,又名湘鄉,屬長沙郡。



 益陽漢縣,屬長沙國,故城在今縣東八十里。武德四年,分置新康縣。七年,省入。



 醴陵漢臨湘縣,界有醴陵,後漢立為縣,屬長沙郡。隋廢,武德四年,分長沙置。



 瀏陽吳分長沙置瀏陽縣,隋廢。景龍二年,於故城
 復置。



 衡州中隋衡山郡。武德四年,平蕭銑,置衡州,領臨蒸、湘潭、耒陽、新寧、重安、新城六縣。七年,省重安、新城二縣。貞觀元年,以廢南雲州之攸縣來屬。天寶元年,改為衡陽郡。乾元元年,復為衡州。舊領縣五,戶七千三百三十,口三萬四千四百八十一。天寶領縣六,戶三萬三千六百八十八,口十九萬九千二百二十八。在京師東南三千四百三里,至東都二千七百六十里。



 衡陽漢蒸陽縣,屬長沙國。吳分蒸陽立臨蒸縣,吳末分長沙東界郡立湘東郡。宋、齊、梁不改。隋罷湘東郡為衡州,改臨蒸為衡陽縣。武德四年,復為臨蒸。開元二十年,復為衡陽



 常寧吳分耒陽立新寧縣,屬湘東郡。舊治三洞,神龍二年,移治麻州。開元九年,治宜江。天寶元年,改為常寧



 攸漢縣,屬長沙國,縣北有攸溪故也



 茶陵漢縣,屬長沙國。隋廢。聖歷元年,分攸縣置



 耒陽漢縣,屬桂陽郡。隋改為耒陰。武德四年,復為
 耒陽。



 衡山吳分湘南縣置。舊屬潭州,後割屬衡州。



 澧州下隋澧陽郡。武德四年,平蕭銑,置澧州,領孱陵、安鄉、澧陽、石門、慈利、崇義六縣。貞觀元年,省孱陵縣。天寶元年,改為澧陽郡。乾元元年,復為澧州。天寶初,割屬山南東道。舊領縣五,戶三千四百七十四,口二萬五千八百二十六。天寶領縣四,戶一萬九千六百二十,口九萬三千三百四十九。在京師東南一千八百九十三里,至東都一千五百七十二里。



 澧陽漢零陽縣,屬武陵郡。吳分武陵西界置天門郡。晉末,以義陽流人集此,僑置南義陽郡。隋平陳,改南義陽為澧州。皆治此縣



 安鄉漢孱陵縣地,屬武陵郡。隋分立安鄉縣。貞觀元年,廢孱陵並入



 石門吳分零陽縣於此置天門郡。隋平陳,廢天門郡,以廢州為石門縣



 慈利本漢零陽縣,隋改零陽為慈利縣。麟德元年,省崇義並入。



 朗州下隋武陵郡。武德四年,平蕭銑,置朗州。天寶元
 年,改為武陵郡。乾元元年,復為朗州。天寶初,割屬山南東道,舊領縣二,戶二千一百四十九,口一萬九百一十三。天寶,戶九千三百六,口四萬三千七百十六。在京師東南二千一百五十九里,至東都一千八百五十八里。



 武陵漢臨沅縣地,屬武陵郡。秦屬黔中郡地。梁分武陵郡於縣置武州。陳改武州為沅陵郡。隋平陳,復為嵩州,尋又改為朗州。煬帝為武陵郡。武德復為朗州。皆治於武陵縣



 龍陽隋縣,取洲名。



 永州中隋零陵郡。武德四年,平蕭銑,置永州,領零陵、湘源、祁陽、灌陽四縣。七年,省灌陽。貞觀元年,省祁陽縣,四年,復置。天寶元年,改為零陵郡。乾元元年,復為永州。舊領縣三,戶六千三百四十八,口二萬七千五百八十三。天寶,戶二萬七千四百九十四,口十七萬六千一百六十入。在京師南三千二百七十四里,至東都三千六百六十五里。



 零陵漢泉陵縣地,屬零陵郡。漢郡治泉陵縣,故城在
 今州北二里。隋平陳,改泉陵為零陵縣,仍移於今理。梁、陳皆為零陵郡,隋置永州,煬帝復為零陵郡,皆治此縣



 祁陽吳分泉陵縣,於今縣東北九十里置祁陽縣,今有古城。隋平陳,並入零陵。武德四年,復分置,移於今治。貞觀元年省,四年又置。石燕岡在祁陽西北一百一十里,此岡穴出石燕,充貢。湘水南自零陵界來



 湘源漢零陵縣地,屬故城在今縣南七十八里。隋平陳,並零陵入湘源縣



 灌陽漢零陵縣地,大業末,蕭銑析
 湘源縣置。武德七年廢。上元二年,荊南節度使呂諲奏,復於故城置灌陽縣。灌水在城西,今名灌源。



 道州中隋零陵郡之永陽縣。武德四年,平蕭銑,置營州,領營道、江華、永陽、唐興四縣。五年,改為南營州。貞觀八年,改為道州,仍省永陽縣。十七年廢,並入永州。上元二年,復析永州置。天寶元年,改為江華郡。乾元元年,復為道州。舊領縣三,戶六千六百一十三,口三萬一千八百八十。天寶領縣四,戶二萬二千五百五十一,口
 十三萬九千六十三。今領縣五。



 弘道漢營浦縣,屬零陵郡。吳置營陽郡。晉改為永陽郡。隋平陳,改營浦為永陽縣。武德四年,於縣置營州,改為營道縣。五年,又加「南」字。貞觀八年,改為道州。天寶元年,改營道為弘道



 延唐漢泠道縣,屬零陵郡,古城在今縣東界南四十里。隋平陳,廢泠道入營道縣,仍於泠道廢城置營道縣。武德四年,移營道縣於州郭置,仍於此置唐興縣。長壽二年,改名武盛。神龍元
 年,復為唐興。天寶元年,改為延唐。泠水,在今縣南六十里



 江華漢馮乘縣,屬蒼梧郡。武德四年,析賀州馮乘縣置江華縣。貞觀十七年,改屬永州。上元二年,還道州。文明元年,改為雲溪縣。神龍元年二月,復為江華



 永明隋改漢營浦縣為永陽,置道州。後州郭內置營道縣,乃移永陽之名於州西南一百一十里置。貞觀八年省,地入營道。天授二年,復析營道置。天寶元年,改為永明縣



 大歷大歷二年,湖南觀察使韋貫之奏請析
 延唐縣,於道州東南二百二十里舂陵侯故城北十五里置縣,因以大歷為名。



 郴州中隋桂陽郡。武德四年,平蕭銑,置郴州,領郴、盧陽、義章、臨武、平陽、晉興六縣。七年,廢義章、平陽二縣。八年,省晉興。天寶元年,改為桂陽郡。乾元元年,復為郴州。舊領縣五,戶八千六百四十六,口四萬九千三百五十五。天寶領縣八,戶三萬一千三百三。在京師東南三千三百里,至東都三千五十七里。



 郴漢縣,屬桂陽郡,漢郡理所也。後漢郡理耒陽,尋還郴。宋、齊封子弟為桂陽王,皆治於此。隋平陳,改為郴州,煬帝為桂陽郡,武德四年,改郴州,皆以郴為理。



 義章大業末,蕭銑分郴置。武德七年省,八年復置。長壽元年,分義章南界置高平縣。開元二十三年,廢高平,仍移義章治高平廢縣。



 義昌晉分郴縣置汝城、晉寧二縣。陳廢二縣,立盧陽郡,領盧陽縣。開皇九年廢郡,以盧陽屬郴州。天寶元年,改為義昌。



 平陽晉分郴置平
 陽郡及縣。陳廢,後蕭銑復分郴置。武德七年省,八年復置。



 資興後漢分郴置漢寧縣。吳改為陽安,晉改為晉寧,隋改為晉興。貞觀八年省,咸亨三年復置,改名資興。



 高亭漢便縣地,屬桂陽郡。晉省,陳復置。隋廢。開元十三年,宇文融析郴縣北界四鄉置安陵縣。天寶元年,改為高亭,取縣東山名。



 臨武漢縣,屬桂陽郡,縣南臨武溪故也。



 藍山漢南平縣,屬桂陽郡。隋廢。咸亨二年,復置南平縣。天寶元年,改為藍山。九疑山,在縣西五
 十里。



 邵州隋長沙郡之邵陽縣。武德四年,平蕭銑,置南梁州,領邵陵、建興、武岡三縣。七年:省建興入武岡,省邵陵並邵陽。貞觀十年,改名邵州。天寶元年,改為邵陽郡。乾元元年,復為邵州。舊領縣二,戶二千八百五十六,口一萬三千五百八十三。天寶,戶一萬七千七十三,口七萬一千六百四十四。在京師東南三千四百里,至東都二千二百六十八里。



 邵陽漢昭陵縣,屬長沙國。後漢改為昭陽,晉改為邵陽。隋平陳,復於今理。吳分零陵北部置邵陵郡。隋平陳,廢郡,以邵陽屬潭州,尋又於邵陽置建州。武德四年,改置南梁州,貞觀十年,改為邵州,皆理邵陽縣。



 武岡漢都梁縣,屬零陵郡。晉分都梁置武岡縣。隋廢。武德四年,分邵陽復置。



 連州隋熙平郡。武德四年,平蕭銑,置連州。天寶元年,改為連山郡。乾元元年,復為連州。舊領縣三,戶五千五
 百六十三,口三萬一千九十四。天寶,戶三萬二千二百十,口一十四萬三千五百三十二。在京師南三千六百六十五里,至東都三千四百五里。



 桂陽漢縣,屬桂陽郡,今州理是也。隋開皇十年,於縣置連州,大業改為熙平郡,武德四年,復為連州,皆以桂陽為理所。



 陽山漢縣,屬桂陽郡。後漢省。晉平吳,分浛洭縣復置。梁於浛水匡縣西置陽山郡,以縣屬之。隋廢郡,縣屬連州。神龍元年,移於洭水之北,今縣理是也。一
 名湟水。



 連山晉武分桂陽立廣惠縣,隋改為廣澤。仁壽元年,改為連山。



 黔州下都督府隋黔安郡。武德元年,改為黔州,領彭水、都上、石城三縣。二年,又分置盈隆、洪杜、相永、萬資四縣。四年,置都督府,督務、施、業、辰、智、䍧、充、應、莊等州。其年,以相永、萬資二縣置費州,以都上分置夷州。十年,以思州高富來屬。十一年,又以高富屬夷州,以智州信寧來屬。今督思、辰、施、牢、費、夷、巫、應、播、充、莊、䍧、琰、池、矩十五州。
 其年,罷都督府。置莊州都督府。景龍四年廢,以播州為都督。先天二年廢,復以黔州為都督。天寶元年,改黔州為黔中郡,依舊都督施、夷、播、思、費、珍、溱、商九州。又領充、明、勞、羲、福、犍、邦、琰、清、莊、峨、蠻、䍧、鼓、儒、琳、鸞、令、那、暉、郝、總、敦、侯、晃、柯、樊、稜、添、普寧、功、亮、茂龍、延、訓、卿、雙、整、懸、撫水、矩、思源、逸、殷、南平、勛、姜、襲等五十州。皆羈縻,寄治山谷。乾元元年,復以黔中郡為黔州都督府。舊領縣五,戶五千九百一十三,口二萬七千四百三十三。天寶縣六,
 戶四千二百七十,口二萬四千二百四。在京師南三千一百九十三里,至東都三千二百七十一里。



 彭水漢酉陽縣,屬武陵郡。吳分西陽置黔陽郡。隋於郡置彭水縣。周置奉州,尋為黔州。貞觀四年,於州置都督府



 黔江隋分黔陽縣置石城縣。天寶元年,改為黔江



 洪杜武德二年,分置洪杜縣,治洪杜溪。麟德二年,移治龔湍



 洋水武德二年,分彭水於巴江西置盈隆縣。先天元年,改為盈川。天寶元年,改為洋水。



 信寧隋置信安縣,取界內山名。武德二年,改為信寧。武德五年,屬義州。州廢來屬。



 都濡貞觀二十年,分盈隆縣置。



 辰州下隋沅陵縣。武德四年,平蕭銑,置辰州,領沅陵等五縣。九年,分大鄉置大鄉五縣。五年,分辰溪置漵浦縣。貞觀九年,分大鄉置三亭縣。天授二年,分大鄉、三亭兩縣置溪州。景雲二年,置都督府,督巫、業、錦三州。開元二十七年,罷都督府。天寶元年,改為盧溪郡。乾元元年,
 復為辰州,取溪名。舊領縣七,戶九千二百八十三,口三萬九千二百二十五。天寶領縣五,戶四千二百四十一,口二萬八千五百五十四。在京師南微東三千四百五里,至東都三千二百六十里。



 沅陵漢辰陽縣,屬武陵郡,本秦黔中郡縣也。隋改辰陽為辰溪,仍分置沅陵縣,仍置沅陵郡。武德四年,改為辰州,以沅陵為理所



 盧溪武德三年,分沅陵縣置



 漵浦漢義陵縣地,屬武陵郡。武德五年,分辰溪置



 麻陽武德三年,分沅陵、辰溪二縣置。垂拱四年,分置龍門縣,尋廢



 辰溪漢辰陽縣地,隋分置辰溪縣。



 錦州下垂拱二年,分辰州麻陽縣地並開山洞置錦州及四縣。天寶元年,改錦州為盧陽郡。乾元元年,復為錦州。天寶領縣五,戶二千八百七十二,口一萬四千三百七十四。至京師三千五百里,至東都三千七百里。



 盧陽招諭渭陽常豐已上四縣,並垂拱三年與州同置。其常豐本名萬安,天寶元年,改為常豐



 洛
 浦天授二年,分辰州之大鄉置,屬溪州。長安四年,改屬錦州。



 施州下隋清江郡之清江縣。義寧二年,置施州,領清江、開夷二縣。貞觀八年,廢業州,以建始縣來屬。麟德元年,廢開夷縣入清江。天寶元年,改為清化郡。乾元元年,復為施州。舊領縣三,戶二千三百一十二,口一萬八百二十五。天寶領縣二,戶三千七百二,口一萬六千四百四十四。在京師南二千七百九里,至東都二千八百一
 十里。



 清江漢巫縣,南郡。吳分巫立沙渠縣。後周於縣立施州。隋為清江縣,州所理也



 建始後周分巫縣置建始縣。義寧二年,於縣置業州,領建始一縣。貞觀八年,廢業州,縣屬施州。



 巫州下貞觀八年,分辰州龍標縣置巫州。其年,置夜郎、朗溪、思征三縣。九年,廢思徵縣。天授二年,改為沅州,分夜郎渭溪縣。長安三年,割夜郎、渭溪二縣置舞州。先
 天二年,又置潭陽縣。開元十三年,改沅州為巫州。天寶元年,改為潭陽郡。乾元元年,復為巫州。舊領縣三,戶四千三十二,口一萬四千四百九十五。天寶,戶五千三百六十八,口一萬二千七百三十八。在京師南三千一百五十八里,至東都三千八百三十三里。



 龍標武德七年置,屬辰州。貞觀八年,置巫州,為理所也。



 朗溪貞觀八年置



 潭陽先天二年,分龍標置。



 業州下長安四年,分沅州二縣置舞州。開元十三年,改為鶴州。二十年,又改為業州。天寶元年,改龍標郡。乾元元年,復為業州。領縣三,戶一千六百七十二,口七千二百八十四。在京師南四千一百九十七里,至東都三千九百里。



 峨山貞觀八年,置夜郎縣,屬巫州。長安四年,置舞州。開元二十年,改夜郎為峨山縣。



 渭溪天授二年,分夜郎置,屬沅州。長安四年,改業州。



 梓姜舊於縣
 置充州,天寶三年,以充州荒廢,以梓姜屬業州,其充州為羈縻州。



 夷州下隋明陽郡之綏陽縣。武德四年,置夷州於思州寧夷縣,領夜郎、神泉、豐樂、綏養、雞翁、伏遠、明陽、高富、寧夷、思義、丹川、宣慈、慈岳等十三縣。六年,廢雞翁縣。貞觀元年,廢夷州,省夜郎、神泉、豐樂三縣,以伏遠、明陽、高富、寧夷、思義、丹川六縣隸務州,宣慈、慈岳二縣隸溪州,以綏養隸智州。四年,復置夷州於黔州都上縣。六年,又
 置雞翁縣。十一年,又以義州之綏陽、黔州之高富來屬。其年,又自都上移於今所。天寶元年,改為義泉郡。乾元元年,復為夷州。舊領縣四,戶二千二百四十一,口八千六百五十七。天寶縣五,戶一千二百八十四,口七千一十三。在京師南四千三百八十七里,至東都三千八百八十里。



 綏陽漢䍧柯郡地。隋朝招慰置綏陽縣,古徼外夷也。武德三年,屬義州。貞觀十一年,改屬夷州



 都上隋
 置。武德元年,屬黔州。貞觀四年,置夷州,為理所。十一年,州移治綏陽縣



 義泉隋舊。於縣置牢州。貞觀十七年,廢牢州,以義泉屬夷州



 洋川武德二年置,舊屬牢州。貞觀十七年,屬夷州



 寧夷舊屬思州。開元二十五年,屬夷州。



 播州下隋䍧柯郡之䍧柯縣。貞觀九年,分置郎州,領恭水、高山、貢山、柯盈、邪施、釋燕六縣。十一年,省郎州並六縣。十三年,又於其地置播州及恭水等六縣。十四年,
 改恭水等六縣名。二十年,以夷州之芙蓉、瑘川來屬。顯慶五年,廢舍月、胡江、羅為三縣。景龍四年,廢莊州都督府,以播州為都督府。先天二年,罷都督。開元二十六年,又廢胡刀、瑘川兩縣。天寶元年,改為播川郡。乾元元年,復為播州。領縣三,戶四百九十,口二千一百六十八。在京師南四千四百五十里,至東都四千九百六十里。



 遵義漢武開西南夷,置䍧柯郡,秦夜郎郡之西南境也。貞觀九年,置恭水縣,屬郎州。十一年省,十三年復置,
 屬播州。十四年,改為羅蒙。十六年,改為遵義。顯慶五年,廢舍月並入



 芙蓉舊屬牢州。貞觀十六年,改為夷州,二十年,又改屬播州。開元二十六年,廢胡刀、瑘川兩縣並入



 帶水貞觀九年,置柯盈縣。十四年,改為帶水。



 思州下隋巴東郡之務川縣。武德四年,置務州,領務川、涪川、扶陽三縣。貞觀元年,以廢夷州之伏遠、寧夷、思義、高富、明陽、丹川六縣,廢思州之丹陽、城樂、感化、思王、
 多田五縣來屬。其年,省思義、明陽、丹川三縣。二年,又省丹陽。四年,改務州為思州。其年,以涪川、扶陽二縣割入費州。八年,又以多田、城樂二縣割入費州,其年,又廢感化縣。十年,以高富隸黔州。十一年,又省伏遠縣。天寶元年,改為寧夷郡。乾元元年,復為思州。舊領縣三,戶二千六百三,口七千五百九十九。天寶,戶一千五百九十九,口一萬二千二十一。在京師南三千八百三十九里,至東都三千五百九十六里。



 務川州所治。漢酉陽縣,屬武陵郡。隋朝招慰置務川縣。武德四年,招慰使冉安昌以務川當䍧柯要路,請置務州。貞觀四年,改為思州,以思邛水為名



 思王武德三年置,屬思州。貞觀元年,改屬務州。四年,改屬思州



 寧夷隋置。武德四年,屬夷州。貞觀元年,屬思州



 思邛開元四年,開生獠置。



 費州下隋黔安郡之涪川縣。貞觀四年,分思州之涪川、扶陽二縣置費州。其年,割黔州之萬資、相永二縣來
 屬。八年,又割思州之多田、城樂來屬。十一年,廢相永、萬資二縣。天寶元年,復為涪川郡。乾元元年,復為費州。舊領縣四,戶二千七百九,口六千九百五十。天寶,戶四百二十九,口二千六百九。在京師南四千七百里,至東都四千九百里。



 涪川漢䍧柯郡之地,久不臣附。周宣政元年,信州總管、龍門公裕,招慰生獠王元殊、多質等歸國,乃置費州,以水為名。武德四年,置務州。貞觀四年,置費州治於此



 多田武德四年,務州刺史奏置。以土地稍平,懇田盈畛,故以多田為名。貞觀四年,屬思州。八年,改屬費州



 扶陽隋仁壽四年,庸州刺史奏置,以扶陽水為名



 城樂武德四年,山南道大使趙郡王孝恭招慰生獠,始築城,人歌舞之,故曰城樂。



 南州下武德二年置,領隆陽、扶化、隆巫、丹溪、靈水、南川六縣。三年,改為僰州。四年,復為南州。貞觀五年,置三溪縣。七年,又置當山、嵐山、歸德、汶溪四縣。八年,又廢當
 山、嵐山、歸德、汶溪四縣。十一年,又廢扶化、隆巫、靈水三縣。天寶元年,改為南川郡。乾元元年,復為南州。舊領縣三,戶三千五百八十三,口一萬三百六十六。天寶領縣二,戶四百四十三,口二千四十三。在京師南三千六百里,至東都三千七百里。



 南川武德二年,置隆陽縣。先天元年,改為南川,州所治



 三溪貞觀五年置。



 溪州下舊辰州之大鄉。天授二年,分置溪州。舊領縣
 二,又分置洛浦縣。長安四年,以洛浦屬錦州。天寶元年,改溪州為靈溪郡。乾元元年,復為溪州。領縣二,戶二千一百八十四,口一萬五千二百八十二。至京師二千八百九十三里,至東都二千六百九十六。



 大鄉漢沅陵、遷陵二縣地,屬武陵郡。梁分置大鄉縣。舊屬辰州,天授二年來屬,州所理也



 三亭貞觀九年分大鄉置,屬辰州。天授二年,改屬溪州。縣界有黔山,大西、小西二山。



 溱州下貞觀十六年,置溱州及榮懿、扶歡、樂來三縣。咸亨元年,廢樂來縣。天寶元年,改為溱溪郡。乾元元年,復為溱州。領縣二,戶八百七十九,口五千四十五。至京師三千四百八十里,至東都四千二百里。



 榮懿扶歡已上二縣,並貞觀十六年開山洞置。



 珍州下貞觀十六年置,天寶元年改為夜郎郡。乾元元年,復為珍州。領縣三,戶二百六十三,口一千三十四。至京師四千一百里,至東都三千七百里。



 夜郎漢夜郎郡之地。貞觀十七年,置於舊播州城,以縣界有隆珍山,因名珍州。麗皋樂源並貞觀十六年開山洞置。



 䍧州,領縣二



 充州,領縣八



 應州,領縣五



 琰州,領縣四



 牢州,領縣七。已上國初置,並屬黔中道羈縻州。永徽已後並省。



 隴右道



 秦州中都督府隋天水郡。武德二年,平薛舉。改置秦
 州,仍立總管府,管秦、渭、岷、洮、疊、文、武、成、康、蘭、宕、扶等十二州。秦州領上邽、成紀、秦嶺、清水四縣。四年,分清水置邽州。六年,廢邽州,以清水來屬。八年,廢文州,又以隴城來屬。其年,又廢伏州,以伏羌來屬。九年,於成羌廢城置鹽泉縣。貞觀元年,改鹽泉為夷賓。二年,省夷賓縣。六年,省長川縣。十四年,督秦、萬、渭、武四州,治上邽。十七年,廢秦嶺縣。開元二十二年,緣地震,移治所於成紀縣之敬親川。天寶元年,改為天水郡。依舊都督府,督天水、隴西、
 同谷三郡。其年,復還治上邽。乾元元年,復為秦州。舊領縣六,戶五千七百二十四,口二萬五千七十三。天寶領縣五,戶二萬四千八百二十七,口十萬九千七百。在京師西七百八十里,至東都一千六百五里。



 上邽漢縣,屬隴西郡。武帝分置天水郡。後漢分豲道立南安郡。後魏改上邽為上封。隋復於上邽置秦州。州前有湖水,四時增減,故名天水郡



 成紀漢縣,屬天水郡。舊治小坑川。開元二十二年,移治敬親川,成紀亦
 徙新城。天寶元年,州復移治上邽縣



 伏羌漢冀縣,屬天水郡。晉於此置秦州。後魏改為當亭縣,隋復為冀城縣。武德三年,改為伏羌縣,仍置伏州。八年,伏州廢,縣屬秦州。貞觀三年,廢夷賓縣,並入伏羌



 隴城漢隴縣,屬天水郡。隋加「城」字。武德二年,置文州,以隴城隸之。八年文州廢,來屬。貞觀三年,省長川縣並入。



 清水漢縣,屬天水郡。武德四年,置邽州於清水。六年,廢邽州,以清水來屬。



 成州下隋漢陽郡。武德元年,置成州,領上祿、長道、潭水三縣。貞觀元年,以潭水屬宕州,又割廢康州之同谷縣來屬。州理楊難當所築建安城。天寶元年,改為同谷郡。乾元元年,復為成州。舊領縣三,戶一千五百四十六,口七千二百五十九。天寶,戶四千七百二十七,口二萬一千五百八。在京師西南九百六十里,至東都一千八百里。



 上祿漢縣,屬武都郡,白馬氐之所處。州南八十里仇
 池山,其上有百頃地,可處萬家。晉時,氐酋楊難當據仇池,即此山上也。晉朝招慰,乃置仇池郡,以難當為守。梁置南秦州,又改為成州。隋以上祿為倉泉縣,又復為上祿



 長道元魏分上祿置長道縣,於縣置天水郡。隋改天水為漢陽郡,又改漢陽縣為長道



 同谷漢下辨步見反道,屬武都郡。後魏於此置廣業郡,領白石縣。又改白水為同谷。



 渭州下隋隴西郡。武德元年,置渭州。天寶元年,改為
 隴西郡。乾元元年,復為渭州。四月,鄯州都督郭英乂奏請以渭州、洮州為都督府,後廢。舊領縣四,戶一千九百八十九,口九千二十八。天寶,戶六千四百二十五,口二萬四千五百二十。在京師西一千一百五十三里,至東都二千里。



 襄武漢縣,屬隴西郡。後魏於縣置渭州,以水為名



 隴西漢豲音桓道地,屬天水郡。



 鄣後漢分武陽置鄣縣。天授二年,改為武陽縣。神龍元年,復為鄣縣



 渭源漢
 首陽縣地,屬隴西郡。後魏分隴西置渭源郡,又改首陽為渭源縣。上元二年,改首陽縣,仍於渭源故城分置渭源縣。儀鳳三年,廢首陽並入渭源。



 鄯州下都督府隋西平郡。武德二年,平薛舉,置鄯州,治故樂都城。貞觀中,置都督府。天寶元年,改為西平郡。乾元元年,復為鄯州。上元二年九月,州為吐蕃所陷,遂廢。所管鄯城三縣,今河州收管。舊領縣二,戶一千八百七十五,口九千五百八十二。天寶領縣三,戶五千三百
 八十九,口二萬七千一十九。在京師西一千九百一十三,至東都二千五百四十里。



 湟水漢破羌縣,屬金城郡。漢破匈奴,取西河地,開湟中處月氏,即此。湟水,俗呼湟河,又名樂都水,南涼禿發烏孤始都此。後魏置鄯州,改破羌為西都縣。隋改為湟水縣。縣界有浩亹水



 龍支漢允吾縣,屬金城郡。後漢改為龍耆縣。後魏改為金城縣,又改為龍支。積石山,在今縣南



 鄯城儀鳳三年置,漢西平郡故城在
 西。



 蘭州下隋金城郡。隋末,陷薛舉。武德二年,平賊,置蘭州。八年,置都督府,督蘭、河、廓四州。貞觀六年,又督西鹽州。十二年,又督涼州。今督蘭、鄯、儒、淳四州。領金城、狄道、廣武三縣。顯慶元年,罷都督府。天寶元年,改金城郡。二載,割狄道縣置狄道郡。乾元元年,復為蘭州。舊領縣三,戶一千六百七十五,口七千三百五。天寶領縣二,戶二千八百八十九,口一萬四千二百二十六。在京師西
 一千四百四十五里,至東都二千二百里。



 五泉漢金城縣,屬金城郡,西羌所處。後漢置西海郡,乞伏乾歸都此,稱涼。隋開皇初,置蘭州,以皋蘭山為名。煬帝改金城郡。隋置五泉縣。咸亨二年,復為金城。天寶元年,改為五泉。



 廣武漢枝楊縣,屬金城郡。張駿置廣武郡。隋廢為縣,屬蘭州。



 臨州下都督府天寶三載,分金城郡置狄道郡。乾元元年,改為臨州都督府,督保塞州,羈縻之名也。領縣二,
 戶二千八百九十九,口一萬四千二百二十六。在京師西一千四百四十五里,至東都二千二百里。



 狄道漢縣,屬隴西郡。晉改為武始縣。隋復為狄道,屬蘭州。天寶三載復置。



 長樂舊安樂縣。乾元後,改為長樂。



 河州下隋枹音桴罕郡。武德二年,平李軌,置河州,領枹罕、大夏二縣。貞觀元年,廢大夏縣。五年復置。十年,省米州,以米川縣來屬。十一年,廢烏州,以其城置安鄉縣,來
 屬。天寶元年,改為安鄉郡。乾元元年,復為河州。舊領縣三,戶三千三百九十一,口一萬二千六百五十五。天寶領縣三,戶五千七百八十二,口三萬六千八百八十六。在京師西一千四百一十五里,至東都二千二百七十里。



 枹罕漢縣,屬金城郡。張駿於縣置河州,至後魏不改,又名枹罕郡。隋初為河州,煬帝為枹罕郡。武德二年,改為河州。皆治於枹罕。



 大夏漢縣,屬隴西郡。張駿於
 縣置大夏郡及縣,取西大夏水為名。貞觀元年,廢入枹罕。五年又置。



 鳳林漢白石縣,屬金城郡。張駿改白石為永固。貞觀七年,廢縣,置烏州。十一年州廢,於城內置安鄉縣。天寶元年,改為鳳林,取關名也。



 武州下隋武都郡。武德元年,置武州,領將利、建威、覆津、盤堤四縣。貞觀元年,省建威入將利。天寶元年,改為武都郡。乾元元年,復為武州。舊領縣三,戶一千一百五十二,口五千三百八十一。天寶,戶二千九百二十三,口
 一萬五千三百一十三。在京師西一千二百九十里,至東都二千里。



 將利秦、漢白馬之地。漢置武都郡並縣。後魏改武都為石門縣,置武州。後周改為將利縣,仍置武都郡。隋初廢,煬帝復為郡,皆治將利縣。



 覆津後魏置武階郡,又於今縣東北三十里萬郡故城置覆津縣。隋廢武階郡,縣屬武都郡。



 盤堤漢河池縣地,屬武都郡。後魏於今縣東南百四十二里移盤堤縣於郡置武州。盤堤
 山為名。



 洮州下隋臨洮郡。武德二年,置洮州。貞觀五年,移州治於洪和城,後復移還洮陽城,今州治也。永徽元年,置都督府。開元十七年廢,並入岷州。臨潭縣置臨州。二十七年,又改為洮州。天寶元年,改為臨洮郡。管密恭縣,黨項部落也,寄治州界。乾元元年,復為洮州。舊領縣二,戶二千三百六十三,口八千二百六十。天寶,戶三千七百,口一萬五千六十。在京師西一千五百六里,至東都二
 千三百九十里。



 臨潭秦、漢時羌地,本吐谷渾之鎮,謂之洪和城。後周攻得之,改為美相縣,屬洮州。貞觀四年,洮州理於此。置臨潭縣,屬旭州。八年,廢旭州,來屬。其年,移理洮陽城,今州治也。仍於舊洪和城置美相縣,隸洮州。天寶中,廢美相並入。



 岷州下隋臨洮郡之臨洮縣。義寧二年,置岷州。武德四年,為總管府,管岷、宕、洮、疊、旭五州。七年,加督芳州。九
 年,又督文、武、扶三州。貞觀元年,督岷、宕、洮、旭四州。六年,督橋、意二州。十二年,廢都督府。神龍元年,廢當夷縣。天寶元年,改為和政郡。乾元元年,復為岷州。舊領縣四,戶四千五百八十三,口一萬九千二百三十九。天寶,縣三,戶四千三百二十五,口二萬三千四百四十一。在京師西一千三百七十八里,至東都二千一百里。



 溢樂秦臨洮縣,屬隴西郡。今州西二十里長城,蒙恬所築。岷山,在縣南一里。崆峒山,縣西二十里。後魏置岷
 州,仍改臨洮為溢樂。隋復改臨洮,義寧二年,改名溢樂。神龍元年,廢當夷縣並入



 祐川後周置基城縣。先天元年,改為祐川,避玄宗名



 和政後周置洮城郡。保定元年,置和政縣。



 廓州下隋澆河郡。武德二年,置廓州。天寶元年,改為寧塞郡。乾元元年,復為廓州。舊領縣二,戶二千二十,口九千七百三十二。天寶,縣三,戶四千二百六十一,口二萬四千四百。在京師二千三十里,至東都二千七百七
 十二里。



 廣威後漢燒當羌之地,段熲破羌斬澆河大帥即此也。漢末,置西平郡,此地即南界也。前涼置湟河郡。後魏置石城郡。廢帝因縣內化隆穀改為化隆縣。後周置廓州。先天元年,改為化成縣。天寶元年,改為廣威縣。縣界有拔延山。達化後周置達化郡並縣。吐渾澆河城,在縣西一百二十里。米川漢枹罕縣地,屬金城郡。貞觀五年,置米州及米川縣。十年,州廢,縣屬廓州。



 疊州下都督府隋臨洮郡之合川縣。武德二年,置疊州,領合川、樂川、疊川三縣。五年,又置安化、和同二縣,以處黨項,尋省。疊川、樂川縣。十三年,置都督,督疊、岷、洮、宕、津、序、壹、枯、嶂、王、蓋、立、橋等州。永徽元年,罷都督府。天寶元年,改為合川郡。乾元元年,復為疊州。舊領縣一,戶一千八十三,口四千六十九。天寶領縣二,戶一千二百七十五,口七千六百七十四。在京師西南一千一百一十里,至東都二千五百六十里。



 合川秦、漢已來,為諸羌保據。後周武帝逐諸羌,始有其地,置合川縣,仍於縣置疊州,取郡山重疊之義。舊治吐谷渾馬牧城,武德三年,移於交戍城



 常芬隋同昌郡之常芬縣。武德元年,置芳州,領常芬、恆香、丹嶺三縣。神龍元年,廢芳州為常芬縣,隸疊州。



 宕州下隋宕昌郡。武德元年,置宕州。領懷道、良恭、和戎三縣。貞觀三年,省和戎入懷道。天寶元年,改為懷道郡。乾元元年,復為宕州。舊領縣二,戶一百四十,口一千
 四百六十一。天寶,戶一千一百九十,口七千一百九十九。在京師西南一千六百五十六里,至東都二千二百八十五里。



 懷道歷代諸羌所據,後魏始附為蕃國。後周置宕昌郡,及懷道、良恭二縣。隋為宕昌郡。武德初,為宕州,理懷道



 良恭後周置陽宕縣,隋改為良恭。



 河西道



 此又從隴右道分出,不在十道之內。貞觀元年,分隴坻已西為隴右道。景雲二年,以江山闊
 遠,奉使者艱難,乃分山南為東西道,自黃河以西,分為河西道。



 涼州中都督府隋武威郡。武德二年,平李軌,置涼州總管府,管涼、甘、瓜、肅四州。涼州領姑臧、昌松、番禾三縣。三年,又置神烏縣。七年,改為都督府,督涼、肅、甘、沙、瓜、伊、芳、文八州。貞觀元年,廢神烏縣。總章元年,復置。咸亨元年,為大都督府。督涼、甘、肅、伊、瓜、沙、雄七州。上元二年,為中都督府。神龍二年,置嘉麟縣。天寶元年,改為武威郡,
 督涼、甘、肅三州。乾元元年,復為涼州。舊領縣三,戶八千二百三十一,口三萬三千三十。天寶領縣五,戶二萬二千四百六十二,口十二萬二百八十一。在京師西北二千一十里,至東都二千八百七十里。



 姑臧漢縣,屬武威郡。所理,秦月氐戎所處。匈奴本名蓋藏城,語訛為姑臧城。西魏復置涼州。晉末,張軌據姑臧,稱前涼。呂光又稱後涼。後入於元魏,為武威郡。武德初,平李軌,置涼州。州界有豬野澤



 神烏漢鸞鳥縣,
 屬武威郡。後魏廢。總章元年,復於漢武威城置武威縣。神龍元年,改為神烏。於漢鸞鳥古城置嘉麟縣



 昌松漢蒼松縣,屬武威郡。後涼呂光改為昌松



 天寶漢番音盤禾縣,屬張掖郡。縣南山曰天山,又名雪山。咸亨元年,於縣置雄州,調露元年,廢雄州,番禾還涼州。天寶三年,改為天寶縣



 嘉麟神龍二年,於漢鸞鳥古城置。景龍二年廢,先天二年復置



 吐渾部落興昔部落閣門府皋蘭府盧山府金水州蹛林州
 賀蘭州,已上八州府,並無縣,皆吐渾、契苾、思結等部,寄在涼州界內。共有戶五千四十八,口一萬七千二百一十二。



 甘州下隋張掖郡。武德二年,平李軌,置甘州。天寶元年,改為張掖郡。乾元元年,復為甘州。舊領縣二,戶二千九百二十六,口一萬一千六百八十。天寶,戶六千二百八十四,口二萬二千九十二。在京師西北二千五百里,至東都三千三百一十里。



 張掖故匈奴昆邪王地,屬漢武開置張掖郡及觻音祿得縣,郡所治也,匈奴王號也。後魏置張掖軍,孝文改為郡及縣,州置西涼州,尋改為甘州,取州東甘峻山為名。祁連山,在州西南二百里也



 刪丹漢縣,屬張掖郡。後漢分張掖置西海郡。晉分刪丹置蘭池、萬歲、仙提三縣。煬帝廢,並入刪丹。居延海、焉支山在縣界。刪丹山,即焉支山,語訛也。



 肅州下武德二年,分隋張掖郡置肅州。八年,置都督
 府,督肅、瓜、沙三州。貞觀元年,罷都督府。貞觀中,廢玉門縣。天寶元年,改為酒泉郡。乾元元年,復為肅州。舊領縣三,戶一千七百三十一,口七千一百一十八。天寶領縣二,戶二千三百三十,口八千四百七十六。在京師西北二千八百五十八里,至東都三千七百八里。



 酒泉漢福祿縣,屬酒泉郡。郡城下有金泉,泉味如酒,故為郡名。此月支地,為匈奴所滅,匈奴令休屠、昆邪王守之。漢武時,昆邪來降,乃置酒泉郡。張軌、李暠、沮渠蒙
 遜皆都於此。後魏置酒泉軍,復為郡,後周改為甘州,隋分甘州置肅州,皆治酒泉。義寧元年,置酒泉縣



 福祿漢舊縣,屬酒泉郡。今縣,漢樂涫縣地,屬燉煌郡。武德二年,於樂涫古城置福祿縣。



 瓜州下都督府隋燉煌郡之常樂縣。武德五年,置瓜州,仍立總管府,管西沙、肅三州。八年。罷都督。貞觀中,復為都督府。天寶元年,為晉昌郡。乾元元年,復為瓜州。舊領縣二,戶一千一百六十四,口四千三百二十二。天寶,
 戶四百七十七,口四千九百八十七。在京師西三千三百一十里,至東都四千三百六里。



 晉昌漢冥安縣,屬燉煌郡。冥,水名。置晉昌郡及冥安縣,周改晉昌為永興。隋改為瓜州,改冥安常樂。武德七年,復為晉昌



 常樂漢廣至縣,屬燉煌郡。魏分廣至置宜禾縣。李暠於此置涼興郡。隋廢,置常樂鎮。武德五年,改鎮為縣。



 伊州下隋伊吾郡。隋末,西域雜胡據之。貞觀四年,歸
 化,置西伊州。六年,去「西」字。天寶元年,為伊吾郡。乾元元年,復為伊州。舊領縣三,戶一千三百三十二,口六千七百七十八。天寶領縣二,戶二千四百六十七,口一萬一百五十七。在京師西北四千四百一十六里,至東都五千三百三十里。



 伊吾在燉煌之北,大磧之外。秦、漢之際,戎居之。南去玉門關八百里,東去陽關二千七百三十里。漢宣帝時,以鄭吉為都護,在玉門關。元帝時,置戊己校尉,皆治車
 師。後漢明帝時,取伊吾盧地,置宜禾都尉以屯田。竇憲、班超大破西域,始於此築城。班勇為西域長史,居此地也。後魏、後周,鄯善戎居之。隋始於漢伊吾屯城之東築城,為伊吾郡。隋末,為戎所據。貞觀四年,款附,置西伊州始於此。天山,在州北一百二十里,一名白山,胡人呼折羅漫山



 柔遠貞觀四年置,取縣東柔遠故城為名



 納職貞觀四年,於鄯善胡所築之城置納職縣。



 沙州下隋燉煌郡。武德二年,置瓜州。五年,改為西沙
 州。貞觀七年,去「西」字。天寶元年,改為燉煌郡。乾元元年,復為沙州。舊領縣二,戶四千二百六十五,口一萬六千二百五十。在京師西北三千六百五十里,至東都四千三百九里。



 燉煌漢郡縣名。月氐戎之地,秦、漢之際來屬。漢武開西域,分酒泉置燉煌郡及縣。周改燉煌為鳴沙縣,取縣界山名。隋復為燉煌。武德三年,置瓜州,取《春秋》「祖吾離於瓜州」之義。五年,改為西沙州。皆治於三危山,在縣東
 南二十里。鳴沙山,一名沙角山,又名神沙山,取州名焉,在縣七里



 壽昌漢龍勒縣地,屬燉煌郡。縣南有龍勒山。後魏改為壽昌縣。陽關,在縣西六里。玉門關,在縣西北一百一十八里。



 西州中都督府本高昌國。貞觀十三年,平高昌,置西州都督府,仍立五縣。顯慶三年,改為都督府。天寶元年,改為交河郡。乾元元年,復為西州。舊領縣五,戶六千四百六十六。天寶領縣五,戶九千一十六,口四萬九千四
 百七十六。在京師西北五千五百一十六里,至東都六千二百一十五里。



 高昌漢車師前王之庭。漢元帝置戊己校尉於此。以其地形高敞,故名高昌。其故壘有八城。張駿置高昌郡,後魏因之。魏末為蠕蠕所據,後麴嘉稱高昌王於此數代。貞觀十四年,討平之,以其地為西州。其高昌國境,東西八百里,南北五百里。尋置都督府,又改為金山都督府。



 柳中貞觀十四年置



 交河縣界有交河,水源出
 縣北天山,一名祁連山,縣取水名。地本漢車師前王庭



 蒲昌貞觀十四年,於始昌故城置,縣東南有蒲類海,胡人呼為婆悉海



 天山貞觀十四年置,取祁連山為名。



 北庭都護府貞觀十四年,侯君集討高昌,西突厥屯兵於浮圖城,與高昌相響應。及高昌平。二十年四月,西突厥泥伏沙缽羅葉護阿史那賀魯率眾內附,乃置庭州,處葉護部落。長安二年,改為北庭都護府。自永徽至
 天寶,北庭節度使管鎮兵二萬人,馬五千匹;所統攝突騎施、堅昆、斬啜;又管瀚海、天山、伊吾三軍鎮兵萬餘人,馬五千匹。至上元元年,陷吐蕃。舊領縣一,戶二千三百。天寶領縣三,戶二千二百二十六,口九千九百六十四。在京師西北五千七百二十里,東至伊州界六百八十里,南至西州界四百五十里,西至突騎施庭一千六百里,北至堅昆七千里,東至回鶻界一千七百里。



 金滿流沙州北,前漢烏孫部舊地,方五千里。後漢車
 師後王庭。胡故庭有五城,俗號「五城之地」。貞觀十四年平高昌後,置庭州以前,故及突厥常居之。



 輪臺取漢輪臺為名。



 蒲類海名



 已上三縣,貞觀十四年與庭州同置。



 瀚海軍開元中蓋嘉運置,在北庭都護府城內,管鎮兵萬二千人,馬四千二百匹。



 天山軍開元中,置西州城內,管鎮兵五千人,馬五百匹。在都護府南五百里。



 伊吾軍開元中置,在伊州西北五百里甘露川,管鎮兵三千人,馬三百匹,在北庭府東
 南七百里。



 鹽治州都督府鹽祿州都督府陰山州都督府



 大漠州都督府輪臺州都督府金滿州都督府



 玄池州哥系州咽面州



 金附州孤舒州西鹽州



 東鹽州叱勒州迦瑟州



 馮洛州已上十六番州,雜戎胡部落,寄於北庭府界內,無州縣戶口,隨地治畜牧。



 安西大都護府貞觀十四年,侯君集平高昌,置西州都護府,治在西州。顯慶二年十一月,蘇定方平賀魯,分其地置濛池、昆陵二都護府。分其種落,列置州縣。於是,西盡波斯國,皆隸安西都護府。仍移安西都護府理所於高昌故地。三年五月,移安西府於龜茲國。舊安西府復為西州。龍朔元年,西域吐火羅款塞,乃於於闐以西、波斯以東十六國,皆置都督,督州八十,縣一百一十,軍府一百二十六,仍立碑於吐火羅以志之。咸亨元年四
 月,吐蕃陷安西都護府。至長壽二年,收復安西四鎮,依前於龜茲國置安西都護府。至德後,河西、隴右戍兵皆徵集,收復兩京。上元元年,河西軍鎮多為吐蕃所陷。有舊將李元忠守北庭,郭昕守安西府,二鎮與沙陀、回鶻相依,吐蕃久攻之不下。建中元年,元忠、昕遣使間道奏事,德宗嘉之,以元忠為北庭都護,昕為安西都護。其後,吐蕃急攻沙陀、回鶻部落,北庭、安西無援,貞元三年,竟陷吐蕃。



 北庭都護府本龜茲國。顯慶中,自西州移府治於此。東至焉耆鎮守八百里,西至疏勒鎮守二千里,南至於闐二千里,東北至北庭府二千里,南至吐蕃界八百里,北至突騎施界雁沙川一千里。安西都護府,鎮兵二萬四千人,馬二千七百匹。都護兼鎮西節度使。



 安西都護所統四鎮


龜茲都護府本龜茲國。其王姓白,理白山之南。去瓜州三千里,勝兵數千。貞觀二十二年,阿史那社
 \gezhu{
  人小}
 破之,
 虜龜茲王而還,乃於其地置都督府,領蕃州之九。至顯慶三年,破賀魯,仍自西州移安西府置於龜茲國城



 毗沙都督府本於闐國。在蔥嶺北二百里,勝兵數千。俗多機巧。其王伏闍信,貞觀二十二年入朝。上元二年正月,置毗沙都督府,初管蕃州五。上元元年,分為十。在安西都護府西南二千里



 疏勒都督府本疏勒國。在白山之南,勝兵二千。去瓜州四千六百里。貞觀九年,遣使朝貢,自是不絕。上元中,
 置疏勒都督府,在安西都護府西南二千里



 焉耆都督府本焉耆國。其王姓龍,名突騎支,常役於西突厥。俗有魚鱉之利。貞觀十八年,郭孝恪平之,由是臣屬。上元中,置都督府處其部落,無蕃州。在安西都護府東八百里。



 西域十六都督州府



 龍朔元年,西域諸國,遣使來內屬,乃分置十六都督府,州八十,縣一百一十,軍府一百二十六,皆隸安西都護
 府,仍於吐火羅國立碑以紀之。



 月氏都督府於吐火羅國所治遏換城置,以其王葉護領之。於其部內分置二十四州,都督統之



 太汗都督府於嚈噠部落所治活路城置,以其王太汗領之。仍分其部置十五州,太汗領之



 條枝都督府於訶達羅支國所治伏寶瑟顛城置,以其王領之。仍於其部分置八州



 大馬都督府於解蘇國所治數瞞城置,以其王領之。
 仍分其部置三州



 高附都督府於骨咄施國所治妖沙城置,以其王領之。仍分其部置三州



 修鮮都督府於罽賓國所治遏紇城置,以其王領之。仍分其部置十一州



 寫鳳都督府於失苑延國所治伏戾城置,以其王領之。仍分其部置四州



 悅般都督府於石汗那國所治艷城置,以其王領之。
 仍分其部置雙縻州



 奇沙州於護特健國所治遏密城置,仍分其部置沛薄、大秦二州



 和默州於怛沒國所治怛城置,仍分置慄弋州



 挔手敖州於烏拉喝國所治摩竭城置



 昆墟州於護密多國所治抵寶那城置



 至秬州於俱密國所治措瑟城置



 鳥飛州於護密多國所治摸廷城置



 王庭州於久越得犍國所治步師城置



 波斯都督府於波斯國所治陵城置。



 右西域諸國,分置羈縻州軍府,皆屬安西都護統攝。自天寶十四載已前,朝貢不絕。今於安西府事末紀之,以表太平之盛業也。



\end{pinyinscope}