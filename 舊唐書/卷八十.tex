\article{卷八十}

\begin{pinyinscope}

 ○恆山王承乾楚王寬吳王恪子成王千里孫信安王禕濮王泰庶人祐蜀王愔蔣王惲越王貞子瑯邪王沖
 紀王慎江王囂代王簡趙王福曹王明



 太宗十四子:文德皇后生高宗大帝、恆山王承乾、濮王泰,楊妃生吳王恪、蜀王愔,陰妃生庶人祐,燕妃生越王貞、江王囂,韋妃生紀王慎,楊妃生趙王福,楊氏生曹王明,王氏生蔣王惲,後宮生楚王寬、代王簡。



 恆山王承乾,太宗長子也,生於承乾殿,因以名焉。武德三年,封恆山王。七年,徙封中山。太宗即位,為皇太子。時
 年八歲,性聰敏,太宗甚愛之。太宗居諒暗,庶政皆令聽斷,頗識大體。自此太宗每行幸,常令居守監國。及長,好聲色,慢游無度,然懼太宗知之,不敢見其跡。每臨朝視事,必言忠孝之道,退朝後,便與群小褻狎。宮臣或欲進諫者,承乾必先揣其情,便危坐斂容,引咎自責。樞機辨給,智足飾非,群臣拜答不暇,故在位者初皆以為明而莫之察也。承乾先患足,行甚艱難,而魏王泰有當時美譽,太宗漸愛重之。承乾恐有廢立,甚忌之。泰亦負其材
 能,潛懷奪嫡之計。於是各樹朋黨,遂成釁隙。有太常樂人年十餘歲,美姿容,善歌舞,承乾特加寵幸,號曰稱心。太宗知而大怒,收稱心殺之,坐稱心死者又數人。承乾意泰告訐其事,怨心逾甚。痛悼稱心不已,於宮中構室,立其形像,列偶人車馬於前,令宮人朝暮奠祭。承乾數至其處,徘徊流涕。仍於宮中起塚而葬之,並贈官樹碑,以申哀悼。承乾自此托疾不朝參者輒逾數月。常命戶奴數十百人專習伎樂,學胡人椎髻,翦彩為舞衣,尋橦
 跳劍,晝夜不絕,鼓角之聲,日聞於外。



 時左庶子於志寧、右庶子孔穎達受詔輔導,志寧撰《諫苑》二十卷諷之,穎達又多所規奏。太宗並嘉之,二人各賜帛百匹、黃金十斤。以勵承乾之意,仍遷志寧為詹事。未幾,志寧以母憂去職,承乾侈縱日甚。太宗復起志寧為詹事,志寧與左庶子張玄素數上書切諫,承乾並不納。又嘗召壯士左衛副率封師進及刺客張師政、紇干承基,深禮賜之,令殺魏王泰,不克而止。尋與漢王元昌、兵部尚書侯
 君集、左屯衛中郎將李安儼、洋州刺史趙節、駙馬都尉杜荷等謀反,將縱兵入西宮。貞觀十七年,齊王祐反於齊州。承乾謂紇干承基曰:「我西畔宮墻,去大內正可二十步來耳,此間大親近,豈可並齊王乎?」會承基亦外連齊王,系獄當死,遂告其事。太宗召承乾,幽之別室。命司徒長孫無忌、司空房玄齡、特進蕭瑀、兵部尚書李勣、大理卿孫伏伽、中書侍郎岑文本、御史大夫馬周、諫議大夫褚遂良等參鞫之,事皆明驗。廢承乾為庶人,徙黔州;元昌
 賜令自盡,侯君集等咸伏誅。其宮僚左庶子張玄素、右庶子趙弘智、令狐德棻、中書舍人蕭鈞,並以材選用,承乾既敗,太宗引大義以讓之,咸坐免。十九年,承乾卒於徙所,太宗為之廢朝,葬以國公之禮。二子象、厥。象官至懷州別駕,厥至鄂州別駕。象子適之,別有傳。



 楚王寬,太宗第二子也。出繼叔父楚哀王智云。早薨。貞觀初追封,無後,國除。



 吳王恪,太宗第三子也。武德三年,封蜀王,授益州大都
 督,以年幼不之官。十年,又徙封吳王。十二年,累授安州都督。及將赴職,太宗書誡之曰:「吾以君臨兆庶,表正萬邦。汝地居茂親,寄惟籓屏,勉思橋梓之道,善侔間、平之德。以義制事,以禮制心,三風十愆,不可不慎。如此則克固盤石,永保維城。外為君臣之忠,內有父子之孝,宜自勵志,以勖日新。汝方違膝下,淒戀何已,欲遺汝珍玩,恐益驕奢。故誡此一言,以為庭訓。」高宗即位,拜司空、梁州都督。恪母,隋煬帝女也。恪又有文武才,太宗常稱其類
 己。既名望素高,甚為物情所向。長孫無忌既輔立高宗,深所忌嫉。永徽中,會房遺愛謀反,遂因事誅恪,以絕眾望,海內冤之。有子四人:仁、瑋、琨、璄,並流於嶺表。



 尋追封恪為鬱林王,並為立廟。又封仁為鬱林縣侯。永昌元年,授襄州刺史。不知州事,後改名千里。天授後,歷唐、廬、許、衛、蒲五州刺史。時皇室諸王有德望者,必見誅戮,惟千里褊躁無才,復數進獻符瑞事,故則天朝竟免禍。長安三年,充嶺南安撫討擊使,歷遷右金吾將軍。中興初,進
 封成王,拜左金吾大將軍,兼領益州大都督,又追贈其父為司空。三年,又領廣州大都督、五府經略安撫大使。節愍太子誅武三思,千里與其子天水王禧率左右數十人斫右延明門,將殺三思黨與宗楚客、紀處訥等。及太子兵敗,千里與禧等坐誅,仍籍沒其家,改姓蝮氏。睿宗即位,詔曰:「故左金吾衛大將軍成王千里,保國安人,克成忠義,願除兇醜,翻陷誅夷。永言淪沒,良深痛悼。宜復舊班,用加新寵,可還舊官。」又令復姓。



 瑋早卒。中興初,
 追封朗陵王。子示玄,本名示俞,出繼蜀王愔。景龍四年,加銀青光祿大夫、秘書少監。開元十三年,改封廣漢郡王、太僕卿同正員,薨。



 琨,則天朝歷淄、衛、宋、鄭、梁、幽六州刺史,有能名。聖歷中,嶺南獠反,敕琨為招慰使,安輯荒徼,甚得其宜。長安二年卒官,贈司衛卿。神龍初,贈張掖郡王。開元十七年,以子禕貴,贈工部尚書,追封吳王。



 璄,中興初封歸政郡王,歷宗正卿,坐千里事貶南州司馬,卒。



 琨子禕。禕少有志尚,事母甚謹,撫弟祗等以友愛稱。景龍
 四年,為太子僕,兼徐州別駕,加銀青光祿大夫。少繼江王囂後,封為嗣江王。景雲元年,復為德、蔡、衢等州刺史。開元後,累轉蜀、濮等州刺史。政號清嚴,人吏畏而服之。漸見委任,入為光祿卿,遷將作大匠。丁母憂去官,起復授瀛州刺史。又上表固請終制,許之。十二年,改封信安郡王。十五年,服除,拜左金吾衛大將軍、朔方節度副大使、知節度事,兼攝御史大夫。尋遷禮部尚書,仍充朔方軍節度使。先是,石堡城為吐蕃所據,侵擾河右。敕禕與
 河西、隴右議取之。禕到軍,總率士伍,克期攻之。或曰:「此城據險,又為吐蕃所惜,今總軍深入,賊必並力拒守。事若不捷,退則狼狽,不如按軍持重,以觀形勢。」禕曰:「人臣之節,豈憚艱險?必期眾寡不敵,吾則以死繼之。茍利國家,此身何惜?」於是督率諸將,倍道兼進,並力攻之,遂拔石堡城,斬獲首級,並獲糧儲器械,其數甚眾。仍分兵據守,以遏賊路。上聞之大悅,始改石堡城為振武軍,自是河、隴諸軍游弈拓地千餘里。十九年,契丹衙官可突干
 殺其王邵固,率部落降於突厥。玄宗遣忠王為河北道行軍元帥以討奚及契丹兩蕃,以禕為副。王既不行,禕率戶部侍郎裴耀卿等諸副將分道統兵出於範陽之北,大破兩蕃之眾,擒其酋長,餘黨竄入山谷。軍還,禕以功加開府儀同三司,兼關內支度、營田等使,兼採訪處置使,仍與二子官。禕既有勛績,執政頗害其功,故其賞不厚,甚為當時所嘆。二十二年,遷兵部尚書,入為朔方節度大使。久之,坐事出為衢州刺史。俄歷滑、懷二州刺
 史。天寶初,拜太子少師,以年老仍聽致仕。二年,遷太子太師,制出,病薨,年八十餘。上聞而痛惜者久之。禕居家嚴毅,善訓諸子,皆有令命。三子:峘、嶧、峴,皆至達官,別有傳。



 祗,神龍中封為嗣吳王。景雲元年,加銀青光祿大夫。天寶十四載,為東平太守。安祿山反,率眾渡河,兇威甚盛,河南陳留、滎陽、靈昌等郡皆陷於賊。祗起兵勤王,玄宗壯之。十五載二月,授祗靈昌太守,又左金吾大將軍、河南都知兵馬使。其月,又加兼御史中丞、陳留太守,持
 節充河南道節度採訪使,本官如故。五月,詔以為太僕卿,遣御史大夫虢王巨代之。



 濮王泰,字惠褒,太宗第四子也。少善屬文。武德三年,封宜都王。四年,進封衛王,以繼衛懷王霸後。貞觀二年,改封越王,授揚州大都督。五年,兼領左武候、大都督,並不之官。八年,除雍州牧、左武候大將軍。七年,轉鄜州大都督。十年,徙封魏王,遙領相州都督,餘官如故。太宗以泰好士愛文學,特令就府別置文學館,任自引召學士。又
 以泰腰腹洪大,趨拜稍難,復令乘小輿至於朝所。其寵異如此。十二年,司馬蘇勖以自古名王多引賓客,以著述為美,勸泰奏請撰《括地志》。泰遂奏引著作郎蕭德言、秘書郎顧胤、記室參軍蔣亞卿、功曹參軍謝偃等就府修撰。十四年,太宗幸泰延康坊宅,因曲赦雍州及長安大闢罪已下,免延康坊百姓無出今年租賦,又賜泰府官僚帛有差。十五年,泰撰《括地志》功畢,表上之,詔令付秘閣,賜泰物萬段,蕭德言等咸加給賜物。俄又每月給泰
 料物,有逾於皇太子。諫議大夫褚遂良上疏諫曰:



 昔聖人制禮,尊嫡卑庶。謂之儲君,道亞睿極。其為崇重,用物不計,泉貨財帛,與王者共之。庶子體卑,不得為例。所以塞嫌疑之漸,除禍亂之源。而先王必本人情,然後制法,知有國家,必有嫡庶。然庶子雖愛,不得超越;嫡子正體,特須尊崇。如當親者疏,當尊者卑,則佞巧之奸,乘機而動,私恩害公,惑志亂國。伏惟陛下功超邃古,道冠百王,發號施令,為世作法。一日萬機,或未盡美,臣職在諫諍,
 無容靜默。伏見儲君料物,翻少魏王,朝野見聞,不以為是。《傳》曰:「臣聞愛子教之以義方。」忠孝恭儉,義方之謂。昔漢竇太后及景帝遂驕恣梁孝王,封四十餘城,苑方三百里,大營宮室,復道彌望,積財鉅萬計,出入警蹕,小不得意,發病而死。宣帝亦驕恣淮陽憲王,幾至於敗,輔以退讓之臣,僅乃獲免。且魏王既新出閣,伏願常存禮則,言提其耳,且示儉節,自可在後月加歲增。妙擇師傅,示其成敗,既敦之以謙儉,又勸之以文學。惟忠惟孝,因而
 獎之,道德齊禮,乃為良器。此所謂聖人之教,不肅而成者也。



 太宗又令泰入居武德殿,侍中魏徵上奏曰:「伏見敕旨,令魏王泰移居武德殿。此殿在內,處所寬閑,參奉往來,極為便近。但魏王既是愛子,陛下常欲其安全,每事抑其驕奢,不處嫌疑之地。今移此殿,便在東宮之西,海陵昔居,時人以為不可。雖時與事異,猶恐人之多言。又王之本心,亦不安息,既能以寵為懼,伏願成人之美。明早是朔日,或恐未得面陳,愚慮有疑,不敢寧寢,輕干
 聽覺,追深戰慄。」太宗並納其言。



 時皇太子承乾有足疾,泰潛有奪嫡之意,招駙馬都尉柴令武、房遺愛等二十餘人,厚加贈遺,寄以腹心。黃門侍郎韋挺、工部尚書杜楚客相繼攝泰府事,二人俱為泰要結朝臣,津通賂遺。文武群官,各有附托,自為朋黨。承乾懼其凌奪,陰遣人詐稱泰府典簽,詣玄武門為泰進封事。太宗省之,其書皆言泰之罪狀,太宗知其詐而捕之,不獲。十七年,承乾敗,太宗面加譴讓。承乾曰:「臣貴為太子,更何所求?但為
 泰所圖,特與朝臣謀自安之道。不逞之人,遂教臣為不軌之事。今若以泰為太子,所謂落其度內。」太宗因謂侍臣曰:「承乾言亦是。我若立泰,便是儲君之位可經求而得耳。泰立,承乾、晉王皆不存;晉王立,泰共承乾可無恙也。」乃幽泰於將作監,下詔曰:



 朕聞生育品物,莫大乎天地;愛敬罔極,莫重乎君親。是故為臣貴於盡忠,虧之者有罰;為子在於行孝,違之者必誅。大則肆諸市朝,小則終貽黜辱。雍州牧、相州都督、左武候大將軍魏王泰,朕
 之愛子,實所鐘心。幼而聰令,頗好文學,恩遇極於崇重,爵位逾於寵章。不思聖哲之誡,自構驕僭之咎,惑讒諛之言,信離間之說。以承乾雖居長嫡,久纏痾恙,潛有代宗之望,靡思孝義之則。承乾懼其凌奪,泰亦日增猜阻,爭結朝士,競引兇人。遂使文武之官,各有托附;親戚之內,分為朋黨。朕志存公道,義在無偏,彰厥巨釁,兩從廢黜。非惟作則四海,亦乃貽範百代。可解泰雍州牧、相州都督、左武候大將軍,降封東萊郡王。



 太宗因謂侍臣曰:「
 自今太子不道,籓王窺嗣者,兩棄之。傳之子孫,以為永制。」尋改封泰為順陽王,徙居均州之鄖鄉縣。太宗後嘗持泰所上表謂近臣曰:「泰文辭美麗,豈非才士。我中心念泰,卿等所知。但社稷之計,斷割恩寵,責其居外者,亦是兩全也。」二十一年,進封濮王。高宗即位,為泰開府置僚屬,車服羞膳,特加優異。永徽三年,薨於鄖鄉,年三十有五。贈太尉、雍州牧,謚曰恭。文集二十卷。二子欣、徽。欣封嗣濮王,徽封新安郡王。欣,則天初陷酷吏獄,貶昭州
 別駕,卒。子嶠,本名餘慶,中興初封嗣濮王。景雲元年,加銀青光祿大夫。開元十二年,為國子祭酒,同正員。以王守一妹婿貶邵州別駕,移鄧州別駕,後復其爵。



 庶人祐,太宗第五子也。武德八年,封宜陽王,其年改封楚王。貞觀二年,徙封燕王,累轉豳州都督。十年,改封齊王,授齊州都督。其舅尚乘直長陰弘智謂祐曰:「王兄弟既多,即上百年之後,須得武士自助。」乃引其妻兄燕弘信謁祐,祐接之甚厚,多賜金帛,令潛募劍士。初,太宗以
 子弟成長,慮乖法度,長史、司馬,必取正人。王有虧違,皆遣聞奏。而祐溺情群小,尤好弋獵,長史薛大鼎屢諫不聽,太宗以大鼎輔導無方,竟坐免。權萬紀前為吳王恪長史,有正直節,以萬紀為祐長史,以匡正之。萬紀見祐非法,常犯顏切諫。有昝君謨、梁猛彪者,並以善騎射得幸於祐。萬紀驟諫不納,遂斥逐之,而祐潛遣招延,狎暱逾甚。太宗慮其不能悔過,數以書責讓祐。萬紀恐並獲罪,謂祐曰:「王,帝之愛子,陛下欲王改悔,故加教訓。若能
 飭躬引過,萬紀請入言之。」祐因附表謝罪。萬紀既至,言祐必能改過。太宗意稍解,賜萬紀而諭之,仍以祐前過,敕書誥誡之。祐聞萬紀勞勉而獨被責,以為賣己,意甚不平。萬紀性又褊隘,專以嚴急維持之,城門外不許祐出,所有鷹犬並令解放,又斥出君謨、猛彪,不許與祐相見。祐及君謨以此銜怒,謀殺萬紀。會事洩,萬紀悉收系獄,而發驛奏聞。十七年,詔刑部尚書劉德威往按之,並追祐及萬紀入京。祐大懼,俄而萬紀奉詔先行,祐遣燕
 弘信兄弘亮追於路射殺之。既殺萬紀,君謨等勸祐起兵,乃召城中男子年十五以上,偽署上柱國、開府儀同三司,開官庫物以行賞。驅百姓入城,繕甲兵。署官司,其官有拓東王、拓西王之號。詔遣兵部尚書李勣與劉威便道發兵討之。祐每夜引弘亮等五人對妃宴樂,以為得志。戲笑之隙,語及官軍,弘亮曰:「不須憂也!右手持酒啖,左手刀拂之。」祐愛信弘亮,聞之甚樂。太宗手詔祐曰:「吾常誡汝勿近小人,正為此也。汝素乖誠德,重惑邪言,
 自延伊禍以取覆滅。痛哉,何愚之甚也!遂乃為梟為獍,忘孝忘忠,擾亂齊郊,誅夷無罪。去維城之固,就積薪之危;壞盤石之親,為尋戈之釁。且夫背禮違義,天地所不容;棄父逃君,人神所共怒。往是吾子,今為國讎。萬紀存為忠烈,死不妨義;汝生為賊臣,死為逆鬼。彼則嘉聲不隤,爾則惡跡無窮。吾聞鄭叔、漢戾,並為猖獗,豈期生子,乃自為之?吾所以上慚皇天,下愧後土,嘆惋之甚,知復何云。」太宗題書畢,為之灑泣。時李勣等兵未至齊境,而
 青、淄等數州兵並不從祐之命,祐又傳檄諸縣,亦不從。或勸祐虜城中子女走入豆子為盜,計未決而兵曹杜行敏謀將執祐,兵士多願從。是夜,乃鑿垣而入,祐與弘亮等五人披甲控弦,入室以自固。行敏列兵圍之,謂祐曰:「昔為帝子,今乃國賊。行敏為國討賊,更無所顧,王不速降,當為煨燼。」命薪草欲積而焚之,祐遂出就擒,餘黨悉伏誅。行敏送祐至京師,賜死於內省,貶為庶人。國除。尋以國公禮葬之。



 蜀王愔,太宗第六子也。貞觀五年,封梁王。七年,授襄州刺史。十年,改封蜀王,轉益州都督。十三年,賜實封八百戶,除岐州刺史。愔常非理毆擊所部縣令,又畋獵無度,數為非法。太宗怒曰:「禽獸調伏,可以馴擾於人;鐵石鐫煉,可為方圓之器。至如愔者,曾不如禽獸鐵石乎!乃削封邑及國官之半,貶為虢州刺史。二十三年,加實封滿千戶。愔在州數游獵,不避禾稼,深為百姓所怨。典軍楊道整叩馬諫,愔曳而捶之。永徽元年,為御史大夫李乾
 祐所劾。高宗謂荊王元景等曰:「先朝櫛風沐雨,平定四方,遠近肅清,車書混一。上天降禍,奄棄萬邦。朕纂承洪業,懼均馭朽,與王共戚同憂,為家為國。蜀王畋獵無度,侵擾黎庶,縣令、典軍,無罪被罰。阿諛即喜,忤意便嗔,如此居官,何以共理百姓?歷觀古來諸王,若能動遵禮度,則慶流子孫;違越條章,則誅不旋踵。愔為法司所劾,朕實恥之。」帝又引楊道整勞勉之,拜為匡道府折沖都尉,賜絹五十匹。貶愔為黃州刺史。四年,坐與恪謀逆,黜為
 庶人,徙居巴州。尋改為涪陵王。乾封二年薨。咸亨初,復其爵土,贈益州大都督,陪葬昭陵,謚曰悼。封子璠為嗣蜀王,永昌年配流歸誠州而死。神龍初,以吳王恪孫朗陵王瑋子榆為嗣蜀王。



 蔣王惲,太宗第七子也。貞觀五年,封郯王。八年,授洺州刺史。十年,改封蔣王、安州都督,賜實封八百戶。二十三年,加實封滿千戶。永徽三年,除梁州都督。惲在安州,多造器用服玩,及將行,有遞車四百兩。州縣不堪其勞,為
 有司所劾,帝特宥之。後歷遂、相二州刺史。上元年,有人詣闕誣告惲謀反,惶懼自殺,贈司空、荊州大都督,陪葬昭陵。子煒嗣,歷沂州刺史,垂拱中為則天所害。子銑,早卒。神龍初,封銑子紹宗為嗣蔣王。景龍二年,加銀青光祿大夫。開元初,為太子家令同正員卒。子欽福嗣,為率更令,同正員。天寶初削官,於錦州安置。十二載,為南郡長史同正。惲子煌,蔡國公。煌孫之芳,幼有令譽,頗善五言詩,宗室推之。開元末為駕部員外郎。天寶十三載,安
 祿山奏為範陽司馬。及祿山起逆,自拔歸西京,授右司郎中,歷工部侍郎、太子右庶子。廣德元年,兵革未清,吐蕃又犯邊,侵軼原、會。乃遣之芳兼御史大夫,使吐蕃,被留境上,二年而歸。除禮部尚書,尋改太子賓客。惲子休道。道子琚,本名思順。中興封嗣趙王,加銀青光祿大夫。開元十二年,改封中山郡王,右領軍將軍。



 越王貞,太宗第八子也。貞觀五年,封漢王。七年,授徐州都督。十年,改封原王,尋徙封越王,拜揚州都督,賜實
 封八百戶。十七年,轉相州刺史。二十三年,加實封滿千戶。永徽四年,授安州都督。咸亨中,復轉相州刺史。貞少善騎射,頗涉文史,兼有吏乾。所在或偏受讒言,官僚有正直者多被貶退,又縱諸僮豎侵暴部人,由是人伏其才而鄙其行。則天臨朝,加太子太傅,除蔡州刺史。自則天稱制,貞與韓王元嘉、魯王靈夔、霍王元軌及元嘉子黃國公譔、靈夔子範陽王藹、元軌子江都王緒並貞長子博州刺史瑯邪王沖等,密有匡復之志。垂拱三年七月,
 譔作謬書與貞云:「內人病漸重,恐須早療;若至今冬,恐成痼疾,宜早下手,仍速相報。」是歲,則天以明堂成,將行大享之禮,追皇宗赴集。元嘉因遞相語云:「大享之際,神皇必遣人告諸王密,因大行誅戮,皇家子弟無遺種矣。」譔遂詐為皇帝璽書與沖云:「朕被幽縶,王等宜各救拔我也。」沖在博州,又偽為皇帝璽書云:「神皇欲傾李家之社稷,移國祚於武氏。」遂命長史蕭德琮等召募士卒,分報韓、魯、霍、越、紀等五王,各令起兵應接,以赴神都。初,沖
 與諸王連謀,及沖先發而莫有應者,惟貞以父子之故,獨舉兵以應之。尋遣兵破上蔡縣,聞沖敗,恐懼,索鎖欲自拘馳驛詣闕謝罪。會其所署新蔡令傅延慶得勇士二千餘人,貞遂有拒敵之意。乃宣言於其眾曰:「瑯邪王已破魏、相數州,聚兵至二十萬,朝夕即到,爾宜勉之。」徵屬縣兵至七千人,分為五營。貞自為中營,署其所親汝陽縣丞裴守德為大將軍、內營總管;趙成美為左中郎將,押左營;閭弘道為右中郎將,押右營;安摩訶為郎將、
 後軍總管;王孝志為右將軍、前軍總管。又以蔡州長史韋慶禮為銀青光祿大夫,行其府司馬。凡署九品已上官五百餘人。令道士及僧轉讀諸經,以祈事集,家僮、戰士咸帶符以闢兵。其所署官皆迫脅見從,本無鬥志,惟裴守德實與之同。守德驍勇,善騎射,貞將起事,便以女良鄉縣主妻之,而委以爪牙心腹之任。



 則天命左豹韜衛大將軍麴崇裕為中軍大總管,夏官尚書岑長倩為後軍大總管,率兵十萬討之,仍令鳳閣侍郎張光輔為
 諸軍節度。於是制削貞及沖屬籍,改姓虺氏。崇裕等軍至蔡州城東四十里,貞命少子規及裴守德拒戰。規等兵潰而歸,貞大懼,閉門自守。裴守德排閣入,問王安在,意欲殺貞以自購也。官軍進逼州城,貞家僮悉力衛,貞曰:「事即如此,豈得受戮辱,當須自為計。」貞乃飲藥而死。家僮方始一時散,舍仗就擒。規亦縊其母自殺,守德攜良鄉縣主亦同縊於別所。麴崇裕斬貞父子及裴守德等,傳首東都,梟於闕下。貞起兵凡二十日而敗。貞之在
 蔡州,數奏免所部租賦以結人心,家僮千人,馬數千匹,外托以畋獵,內實習武備。嘗游於城西水門橋,臨水自鑒,不見其首,心甚惡之,未幾而及禍。神龍初,追復爵土,與子沖俱復舊姓。初,貞將起兵,作書與壽州刺史、駙馬都尉趙瑰曰:「佇總義兵,來入貴境。」瑰甚喜,復許率兵相應。瑰妻常樂長公主,高祖第七女,和思皇后之母也,謂其使曰:「為我報越王,與其進不與其退。爾諸王若是男兒,不應至許時尚未舉動。我常見耆老云,隋文帝將篡
 奪周室,尉遲迥是周家外甥,猶能起兵相州,連結突厥,天下聞風,莫不響應。況爾諸王,並國家懿親,宗社是托,豈不學尉遲迥感恩效節,舍生取義耶?夫為臣子,若救國家則為忠,不救則為逆。諸王必須以匡救為急,不可虛生浪死,取笑於後代。」及貞等敗,瑰與公主亦伏誅。



 沖,貞長子也。好文學,善騎射、歷密、濟、博三州刺史,皆有能名。初,沖自博州募得五千餘人,欲渡河攻濟州,先取武水縣。縣令郭務悌赴魏州請援,魏州莘縣令馬玄素領
 兵千七百人邀之於路,恐力不敵,先入武水城,閉門拒守。沖乃令積草車上,放火燒南門,擬乘火突入。火之未起,南風甚急,及火已燃,遽回為北風,未至城門,燒草已甚,沖軍由是沮氣。有堂邑丞董玄寂為沖統帥兵仗,及沖擊武水,玄寂曰:「瑯邪王與國家交戰,此乃反也。」沖聞之,斬玄寂以徇。兵眾懼而散入草澤,不可禁止,惟有家僮左右不過數十而已。乃卻走入博州城,為守門者所殺。則天命左金吾將軍丘神勣為清平道行軍大總管以
 討沖,兵未至,沖已死,傳首東都,梟於闕下。沖起兵凡七日而敗。沖三弟:倩,封常山公,歷常州別駕,坐與父兄連謀伏誅。溫,以告其朋黨得實,減死流嶺南,尋卒。神龍初,侍中敬暉等以沖父子翼戴皇家,義存社稷,請復其官爵,武三思令昭容上官氏代中宗手詔不許。開元四年,詔追復爵土,令備禮改葬。太常奏謚議曰:「故越王貞,往者願匡宗社,夙懷誅呂之謀;乃心王國,用擊非劉之議。以茲獲戾,上悼聖心。謹按謚法『死不忘君曰敬』,請謚曰
 敬。」從之。五年,下詔曰:「九族以親,克敦其教;百代必祀,允竟厥德。故蔡州刺史、越王貞,執心不回,臨事能斷。粵自籓國,勤於王家。弘道之後,寶圖將缺,懷劉章之輔漢,追鄭武之翊周。遂能奮不顧身,率先唱義。雖英謀未克,而忠節居多。嗣絕國除,年逾二紀,奠享淪廢,甚為憫焉。永言興繼,式備典冊。其封貞侄孫故許王男左監門衛將軍、夔國公琳為嗣越王,以奉其祀。仍官為立碑。」琳尋卒,國除。



 紀王慎,太宗第十子也。貞觀五年,封申王。七年,授秦州都督。十年,改封紀王,賜實封八百戶。十七年,遷襄州刺史,以善政聞,璽書勞勉,百姓為之立碑。二十三年,加實封滿千戶。永徽元年,拜左衛大將軍。二年,授荊州都督,累除刑州刺史。文明元年,加授太子太師,轉貝州刺史。慎少好學,長於文史,皇族中與越王貞齊名,時人號為紀、越。初,貞將起事,慎不肯同謀;及貞敗,慎亦下獄。臨刑放免,改姓虺氏,仍載以檻車,配流嶺表,道至蒲州而卒。
 慎長子和州刺史東平王續最知名,早卒。次子沂州刺史義陽王琮、楚國公睿、遂州別駕襄郡公秀、廣化郡公獻、建平郡公欽等五人,垂拱中並遇害,家屬徙嶺南。中興初,追復官爵,令以禮改葬。封慎少子鐵誠為嗣紀王,後改名澄。景雲元年,加銀青光祿大夫。開元初,歷德、瀛、冀三州刺史、左驍衛將軍,薨。子行同嗣,天寶中為右贊善大夫,同正員。



 江王囂,太宗第十一子也。貞觀五年受封,六年薨,謚曰
 殤。



 代王簡,太宗第十二子也。貞觀五年受封,其年薨,無後,國除。



 趙王福,太宗第十三子也。貞觀十三年受封,出後隱太子建成。十八年,授秦州都督,賜實封八百戶。二十三年,加右衛大將軍,累授梁州都督。咸亨元年薨,贈司空、並州都督,陪葬昭陵。中興初,封蔣王惲孫思順為嗣趙王。



 曹王明,太宗第十四子。貞觀二十一年受封。二十三年,
 賜實封八百戶,尋加滿千戶。顯慶中,授梁州都督,後歷虢、蔡、蘇三州刺史。詔令繼巢剌王元吉後。永崇中,坐與庶人賢通謀,降封零陵王,徙於黔州。都督謝祐希旨,逼脅令自殺,帝深悼之,黔府官僚咸坐免職。景雲元年,明喪柩歸於京師,陪葬昭陵。有二子,南州別駕零陵王俊、黎國公傑,垂拱中並遇害。中興初,封傑子胤為嗣曹王。胤叔父備自南州還,又封備為嗣曹王、衛尉少卿、同正員,胤遂停封。後備招慰忠州叛獠,沒於賊,又封胤為王、
 銀青光祿大夫、右武衛將軍。卒,子戢嗣,左衛率府中郎將。卒,子皋嗣。皋自有傳。



 史臣曰:太宗諸子,吳王恪、濮王泰最賢。皆以才高辯悟,為長孫無忌忌嫉,離間父子,遽為豺狼,而無忌破家,非陰禍之報歟?武後斫喪王室,潛移龜鼎,越王貞父子痛憤,義不圖全。毀室之悲,《鴟鴞》之詩,傷矣!比齊祐之妄作,豈同年而語哉!



 贊曰:子弟作籓,磐石維城。驕侈取敗,身無令名。沖、譔憤
 發,視死如生。承乾、齊祐,愚弟庸兄。



\end{pinyinscope}