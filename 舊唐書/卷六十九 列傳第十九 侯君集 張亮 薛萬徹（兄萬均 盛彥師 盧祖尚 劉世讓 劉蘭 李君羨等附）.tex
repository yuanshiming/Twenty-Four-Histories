\article{卷六十九 列傳第十九 侯君集 張亮 薛萬徹(兄萬均 盛彥師 盧祖尚 劉世讓 劉蘭 李君羨等附)}

\begin{pinyinscope}

 ○侯君集張亮薛萬徹兄萬均盛彥師盧祖尚劉世讓劉蘭李君羨等附



 侯君集,豳州三水人也。性矯飾,好矜誇,玩弓矢而不能成其藝,乃以武勇自稱。太宗在籓,引入幕府,數從征伐,
 累除左虞侯、車騎將軍,封全椒縣子。漸蒙恩遇,參預謀議。建成、元吉之誅也,君集之策居多。太宗即位,遷左衛將軍,以功進封潞國公,賜邑千戶,尋拜右衛大將軍。貞觀四年,遷兵部尚書,參議朝政。時將討吐谷渾伏允,命李靖為西海道行軍大總管,以君集及任城王道宗並為之副。九年三月,師次鄯州,君集言於靖曰:「大軍已至,賊虜尚未走險,宜簡精銳,長驅疾進,彼不我虞,必有大利。若此策不行,潛遁必遠,山障為阻,討之實難。」靖然其
 計,乃簡精銳,輕齎深入。道宗追及伏允之眾於庫山,破之。伏允輕兵入磧,以避官軍。靖乃中分士馬為兩道並入,靖與薛萬均、李大亮趣北路,使侯君集、道宗趣南路。歷破邏真穀,逾漢哭山,經途二千餘里,行空虛之地。盛夏降霜,山多積雪,轉戰過星宿川,至於柏海,頻與虜遇,皆大克獲。北望積玉山,觀河源之所出焉。乃旋師,與李靖會於大非川,平吐谷渾而還。十一年,與長孫無忌等俱受世封,授君集陳州刺史,改封陳國公。明年,拜吏部
 尚書,進位光祿大夫。君集出自行伍,素無學術,及被任遇,方始讀書。典選舉,定考課,出為將領,入參朝政,並有時譽。



 高昌王麴文泰時遏絕西域商賈,太宗徵文泰入朝,而稱疾不至,詔以君集為交河道行軍大總管討之。文泰聞王師將起,謂其國人曰:「唐國去此七千里,涉磧闊二千里,地無水草,冬風凍寒,夏風如焚。風之所吹,行人多死,當行百人不能得至,安能致大軍乎?若頓兵於吾城下,二十日食必盡,自然魚潰,乃接而虜之,何足憂
 也!」及軍至磧口,而文泰卒,其子智盛襲位。君集率兵至柳谷,候騎言文泰克日將葬,國人咸集。諸將請襲之,君集曰:「不可,天子以高昌驕慢無禮,使吾恭行天罰,今襲人於墟墓之間,非問罪之師也。」於是鼓行而前,攻其田地。賊嬰城自守,君集諭之,不行。先是,大軍之發也,上召山東善為攻城器械者,悉遣從軍。君集遂刊木填隍,推撞車撞其睥睨,數丈頹穴,拋車石擊其城中,其所當者無不糜碎,或張氈被,用障拋石,城上守陴者不復得立。
 遂拔之,虜其男女七千餘口,仍進兵圍其都城。智盛窮蹙,致書於君集曰:「有罪於天子者,先王也。天罰所加,身已喪背。智盛襲位未幾,不知所以愆闕,冀尚書哀憐。」君集報曰:「若能悔禍,宜束手軍門。」智盛猶不出,因命士卒填其隍塹,發拋車以攻之。又為十丈高樓,俯視城內,有行人及飛石所中處,皆唱言之,人多入室避石。初,文泰與西突厥欲谷設約,有兵至,共為表裏。及聞君集至,欲谷設懼而西走千餘里,智盛失援,計無所出,遂開門出
 降。君集分兵略地,遂平其國,俘智盛及其將吏,刻石紀功而還。君集初破高昌,曾未奏請,輒配沒無罪人,又私取寶物。將士知之,亦競來盜竊,君集恐發其事,不敢制。及京師,有司請推其罪,詔下獄。中書侍郎岑文本以為,功臣大將不可輕加屈辱,上疏曰:



 君集等或位居輔佐,或職惟爪牙,並蒙拔擢,受將帥之任,不能正身奉法,以報陛下之恩。舉措肆情,罪負盈積,實宜繩之刑典,以肅朝倫。但高昌昏迷,人神共棄,在朝議者,以其地在遐荒,
 咸欲置之度外。唯陛下運獨見之明,授決勝之略,君集等奉行聖算,遂得指期平殄。若論事實,並是陛下之功,君集等有道路之勞,未足稱其勛力。而陛下天德弗宰,乃推功於將帥。露布初至,便降大恩,從征之人,皆沾滌蕩。及其凱旋,特蒙曲宴,又對萬國,加之重賞。內外文武,咸欣陛下賞不逾時。而不經旬日,並付大理,雖乃君集等自掛網羅,而在朝之人未知所犯,恐海內又疑陛下唯錄其過,似遺其功。臣以下才,謬參近職,既有所見,不
 敢默然。臣聞古之人君,出師命將,克敵則獲重賞,不克則受嚴刑。是以賞其有功也,雖貪殘淫縱,必蒙青紫之寵;當其有罪也,雖勤躬潔己,不免鈇鉞之誅。故《周書》曰:「記人之功,忘人之過,宜為君者也。」昔漢貳師將軍李廣利損五萬之師,糜億萬之費,經四年之勞,唯獲駿馬三十匹。雖斬宛王之首,而貪不愛卒,罪惡甚多。武帝為萬里征伐,不錄其過,遂封廣利海西侯,食邑八千戶。又校尉陳湯矯詔興師,雖斬郅支單于,而湯素貪盜,所收康
 居財物,事多不法,為司隸所系。湯乃上疏曰:「與吏士共誅郅支,幸得擒滅。今司隸乃收系案驗,是為郅支報仇也。」元帝赦其罪,封湯關內侯,賜黃金百斤。又晉龍驤將軍王浚有平吳之功,而王渾等論浚違詔,不受節度,軍人得孫皓寶物,並燒皓宮及船。浚上表曰:「今年平吳,誠為大慶,於臣之身,更為咎累。」武帝赦而不推,拜輔國大將軍,封襄陽侯,賜絹萬匹。近隋新義郡公韓擒虎平陳之日,縱士卒暴亂叔寶宮內,文帝亦不問罪,雖不進爵,
 拜擒虎上柱國,賜物八千段。由斯觀之,將帥之臣,廉慎者寡,貪求者眾,是以黃石公《軍勢》曰:「使智,使勇,使貪,使愚。故智者樂立其功,勇者好行其志,貪者邀趨其利,愚者不計其死。」是知前聖莫不收人之長,棄人之短,良為此也。臣又聞,夫天地之道,以覆載為先;帝王之德,以含弘為美。夫以區區漢武及歷代諸帝,猶能宥廣利等,況陛下天縱神武,振宏圖以定六合,豈獨正茲刑網,不行古人之事哉!伏惟聖懷,當自已有斟酌。臣今所以陳聞,
 非敢私君集等,庶以螢爝末光,增暉日月。倘陛下降雨露之澤,收雷電之威,錄其微勞,忘其大過,使君集重升朝列,復預驅馳,雖非清貞之臣,猶是貪愚之將。斯則陛下聖德,雖屈法而德彌顯;君集等愆過,雖蒙宥而過更彰。足使立功之士,因茲而皆勸;負罪之將,由斯而改節矣。



 疏奏,乃釋。君集自以有功於西域,而以貪冒被囚,志殊怏怏。十七年,張亮以太子詹事出為洛州都督,君集激怒亮曰:「何為見排?」亮曰:「是公見排,更欲誰冤!」君集曰:「
 我平一國,還觸天子大嗔,何能抑排!」因攘袂曰:「鬱鬱不可活,公能反乎?當與公反耳。」亮密以聞,太宗謂亮曰:「卿與君集俱是功臣,君集獨以語卿,無人聞見,若以屬吏,君集必言無此。兩人相證,事未可知。」遂寢其事,待君集如初。尋與諸功臣同畫像於凌煙閣。時庶人承乾在東宮,恐有廢立,又知君集怨望,遂與通謀。君集子婿賀蘭楚石時為東宮千牛,承乾令數引君集入內,問以自安之術。君集以承乾劣弱,意欲乘釁以圖之,遂贊承乾陰
 圖不軌,嘗舉手謂承乾曰:「此好手,當為用之。」君集或慮謀洩,心不自安,每中夜蹶然而起,嘆吒久之。其妻怪而謂之曰:「公,國之大臣,何為乃爾?必當有故。若有不善之事,孤負國家,宜自歸罪,首領可全。」君集不能用。及承乾事發,君集被收,楚石又詣闕告其事。太宗親臨問曰:「我不欲令刀筆吏辱公,故自鞫驗耳。」君集辭窮。太宗謂百僚曰:「往者家國未安,君集實展其力,不忍置之於法。我將乞其性命,公卿其許我乎?」群臣爭進曰:「君集之罪,天
 地所不容,請誅之以明大法。」太宗謂君集曰:「與公長訣矣,而今而後,但見公遺像耳!」因歔欷下泣。遂斬於四達之衢,籍沒其家。君集臨刑,容色不改,謂監刑將軍曰:「君集豈反者乎,蹉跌至此!然嘗為將,破滅二國,頗有微功。為言於陛下,乞令一子以守祭祀。」由是特原其妻及一子,徙於嶺南。



 張亮,鄭州滎陽人也。素寒賤,以農為業。倜儻有大節,外敦厚而內懷詭詐,人莫之知。大業末,李密略地滎、汴,亮
 杖策從之,未被任用。屬軍中有謀反者,亮告之,密以為至誠,署驃騎將軍,隸於徐勣。及勣以黎陽歸國,亮頗贊成其事,乃授鄭州刺史。會王世充陷鄭州,亮不得之官,孤軍無援,遂亡命於共城山澤。後房玄齡、李勣以亮倜儻有智謀,薦之於太宗,引為秦府車騎將軍。漸蒙顧遇,委以心膂。會建成、元吉將起難,太宗以洛州形勝之地,一朝有變,將出保之。遣亮之洛陽,統左右王保等千餘人,陰引山東豪傑以俟變,多出金帛,恣其所用。元吉告
 亮欲圖不軌,坐是屬吏,亮卒無所言。事釋,遣還洛陽。及建成死,授懷州總管,封長平郡公。貞觀五年,歷遷御史大夫,轉光祿卿,進封鄅國公,賜實封五百戶。後歷豳、夏、鄜三州都督。七年,魏王泰為相州都督而不之部,進亮金紫光祿大夫,行相州大都督長史。十一年,改封鄖國公。亮所蒞之職,潛遣左右伺察善惡,發手適奸隱,動若有神,抑豪強而恤貧弱,故所在見稱。初,亮之在州也,棄其本妻,更娶李氏。李素有淫行,驕妒特甚,亮寵憚之。後至
 相州,有鄴縣小兒,以賣筆為業,善歌舞,李見而悅之,遂與私通。假言亮先與其母野合所生,收為亮子,名曰慎幾。亮前婦子慎微,每以養慎幾致諫,亮不從。李尤好左道,所至巫覡盈門,又干預政事,由是亮之聲稱漸損。十四年,又為工部尚書。明年,遷太子詹事,出為洛州都督。及侯君集誅,以亮先奏其將反,優詔褒美,遷刑部尚書,參預朝政。太宗將伐高麗,亮頻諫不納,因自請行。以亮為滄海道行軍大總管,管率舟師。自東萊渡海,襲沙卑
 城,破之,俘男女數千口。進兵頓於建安城下,營壘未固,士卒多樵牧。賊眾奄至,軍中惶駭。亮素怯懦,無計策,但踞胡床,直視而無所言,將士見之,翻以亮為有膽氣。其副總管張金樹等乃鳴鼓令士眾擊賊,破之。太宗知其無將帥材而不之責。有方術人程公穎者,亮親信之。初,在相州,陰召公穎謂曰:「相州形勝之地,人言不出數年有王者起,公以為何如?」公穎知其有異志,因言亮臥似龍形,必當大貴。又有公孫常者,頗擅文辭,自言有黃白
 之術,尤與亮善。亮謂曰:「吾嘗聞圖讖『有弓長之君當別都』,雖有此言,實不願聞之。」常又言亮名應圖錄,亮大悅。二十年,有陜人常德玄告其事,並言亮有義兒五百人。太宗遣法官按之,公穎及常證其罪,亮曰:「此二人畏死見誣耳。」又自陳佐命之舊,冀有寬貸。太宗謂侍臣曰:「亮有義兒五百,畜養此輩,將何為也?正欲反耳。」命百僚議其獄,多言亮當誅,唯將作少匠李道裕言亮反形未具,明其無罪。太宗既盛怒,竟斬於市,籍沒其家。歲餘,刑部
 侍郎有闕,令執政者妙擇其人,累奏皆不可。太宗曰:「朕得其人也。往者李道裕議張亮云『反形未具』,此言當矣。雖不即從,至今追悔。」遂授道裕刑部侍郎。



 薛萬徹,雍州咸陽人,自燉煌徙焉。隋左御衛大將軍世雄子也。世雄大業末卒於涿郡太守。萬徹少與兄萬均隨父在幽州,俱以武略為羅藝所親待。尋與藝歸附高祖,授萬均上柱國、永安郡公,萬徹車騎將軍、武安縣公。會竇建德率眾十萬來寇範陽,藝逆拒之。萬均謂藝曰:「
 眾寡不敵,今若出門,百戰百敗,當以計取之。可令羸兵弱馬阻水背城為陣以誘之,觀賊之勢,必渡水交兵。萬均請精騎百人伏於城側,待其半渡擊之,破賊必矣。」藝從其言。建德果引軍渡水,萬均邀擊,大破之。明年,建德率眾二十萬復攻幽州,賊已攀堞,萬均與萬徹率敢死士百人從地道而出,直掩賊背擊之,賊遂潰走。及太宗平劉黑闥,引萬均為右二護軍,恩顧甚至。隱太子建成又引萬徹置於左右。建成被誅,萬徹率宮兵戰於玄武
 門,鼓噪欲入秦府,將士大懼。及梟建成首示之,萬徹與數十騎亡於終南山。太宗累遣使諭意,萬徹釋仗而來。太宗以其忠於所事,不之罪也。



 萬均,貞觀初歷遷殿中少監。柴紹之擊梁師都,以萬徹為副。未至朔方數十里,突厥四面而至,官軍稍卻。萬均與萬徹橫出擊之,斬其驍將,虜陣亂,因而乘之,殺傷被野。鼓行而進,遂圍師都。俄而師都見殺,城降,突厥不敢來援。萬徹後從李靖擊突厥頡利可汗於塞北,以功授統軍,進爵郡公。初,靖將
 擊吐谷渾,請萬徹同行。及至賊境,與諸將各率百餘騎先行,卒與虜數千騎相遇。萬徹單騎馳擊之,虜無敢當者。還謂諸將曰:「賊易與耳!」躍馬復進,諸將隨之,斬數千級,人馬流血,勇冠三軍。又與萬均破吐谷渾天柱王於赤水源,獲其雜畜二十萬計,追至河源。萬均此後官至左屯衛大將軍,累封潞國公而卒。



 萬徹尋丁母憂解職,俄起為右衛將軍,出為蒲州刺史。會薛延陀率回紇、同羅之眾渡磧,南擊李思摩,萬徹副李勣援之。與虜相遇,
 率數百騎為先鋒,擊其陣後,騎皆散,賊顧見,遂大潰。追奔數十里,斬首三千餘級,獲馬萬五千匹。以功別封一子為縣侯。十八年,授左衛將軍,尚丹陽公主,拜附馬都尉。尋遷右衛大將軍,轉杭州刺史,遷代州都督,復召拜右武衛大將軍。太宗從容謂從臣曰:「當今名將,唯李勣、道宗、萬徹三人而已。李勣、道宗不能大勝,亦不大敗;萬徹非大勝,即大敗。」太宗嘗召司徒長孫無忌等十餘人宴於丹霄殿,各賜以貘皮,萬徹預焉。太宗意在賜萬徹,
 而誤呼萬均,因愴然曰:「萬均朕之勛舊,不幸早亡,不覺呼名,豈其魂靈欲朕之賜也。」因令取貘皮,呼萬均以同賜而焚之於前,侍坐者無不感嘆。二十二年,萬徹又為青丘道行軍大總管,率甲士三萬自萊州泛海伐高麗,入鴨綠水百餘里,至泊灼城,高麗震懼,多棄城而遁。泊灼城主所夫孫率步騎萬餘人拒戰,萬徹遣右衛將軍裴行方領步卒為支軍繼進,萬徹及諸軍乘之,賊大潰。追奔百餘里,於陣斬所夫孫,進兵圍泊灼城。其城因山
 設險,阻鴨綠水以為固,攻之未拔。高麗遣將高文率烏骨、安地諸城兵三萬餘人來援,分置兩陣。萬徹分軍以當之,鋒刃才接而賊大潰。萬徹在軍,仗氣凌物,人或奏之。及謁見,太宗謂曰:「上書者論卿與諸將不協,朕錄功棄過,不罪卿也。」因取書焚之。尋為副將、右衛將軍裴行方言其怨望,於是廷驗之,萬徹辭屈。英國公李勣進曰:「萬徹職乃將軍,親惟主婿,發言怨望,罪不容誅。」因除名徙邊,會赦得還。永徽二年,授寧州刺史。入朝與房遺愛
 款暱,因謂遺愛曰:「今雖患腳,坐置京師,諸輩猶不敢動。」遺愛謂萬徹曰:「公若國家有變,我當與公立荊王元景為主。」及謀洩,吏逮之,萬徹不之伏,遺愛證之,遂伏誅。臨刑大言曰:「薛萬徹大健兒,留為國家效死力固好,豈得坐房遺愛殺之乎!」遂解衣謂監刑者疾斫。執刃者斬之不殊,萬徹叱之曰:「何不加力!」三斫乃絕。



 萬徹長兄萬淑,亦有戰功。貞觀初,至營州都督,檢校東夷校尉,封梁郡公。季弟萬備,有孝行,母終,廬於墓側。太宗降璽書吊慰,
 仍旌表其門。後官至左衛將軍。並先萬徹卒。



 初,武德、貞觀之際,有盛彥師、盧祖尚、劉世讓、劉蘭、李君羨等,並有功名而不終其位。



 盛彥師者,宋州虞城人。大業中,為澄城長。義師至汾陰,率賓客千餘人濟河上謁,拜銀青光祿大夫、行軍總管,從平京城。俄與史萬寶鎮宜陽以拒東寇。及李密之叛,將出山南,史萬寶懼密威名,不敢拒,謂彥師曰:「李密,驍賊也,又輔以王伯當,決策而叛,其下兵士思欲東歸,若
 非計出萬全,則不為也。兵在死地,殆不可當。」彥師笑曰:「請以數千之眾邀之,必梟其首。」萬寶曰:「計將安出?」對曰:「軍法尚詐,不可為公說之。」便領眾逾熊耳山南,傍道而止,令弓弩者夾路乘高,刀楯者伏於溪谷。令曰:「待賊半渡,一時齊發,弓弩據高縱射,刀楯即亂出薄之。」或問之曰:「聞李密欲向洛州,而公入山,何也?」彥師曰:「密聲言往洛,實走襄城就張善相耳,必當出人不意。若賊入谷口,我自後追之,山路險隘,無所展力,一夫殿後,必不能制。
 今吾先得入谷,擒之必矣。」李密既度陜州,以為餘不足慮,遂擁眾徐行,果逾山南渡。彥師擊之,密眾首尾斷絕,不得相救,遂斬李密,追擒伯當。以功封葛國公,拜武衛將軍,仍鎮熊州。太宗討王世充,遣彥師與萬寶軍於伊闕,絕其山南之路。賊平,除宋州總管。初,彥師之入關也,王世充以其將陳寶遇為宋州刺史,處其家不以禮,及此,彥師因事殺之。平生所惡數十家亦皆殺之。州中震駭,重足而立。會徐圓朗反,彥師為安撫大使,因戰,遂沒
 於賊。圓朗禮厚之,令彥師作書報其弟,令舉城降己。彥師為書曰:「吾奉使無狀,被賊所擒,為臣不忠,誓之以死。汝宜善侍老母,勿以吾為念。」圓朗初色動,而彥師自若,圓朗乃笑曰:「盛將軍乃有壯節,不可殺也。」待之如舊。賊平,彥師竟以罪賜死。



 盧祖尚者,字季良,光州樂安人也。父禧,隋虎賁郎將。累葉豪富,傾財散施,甚得人心。大業末,召募壯士逐捕群盜。時年甚少,而武力過人,又御眾嚴整,所向有功。群盜
 畏憚,不敢入境。及宇文化及作亂,州人請祖尚為刺史。祖尚時年十九,升壇歃血,以誓其眾,泣涕歔欷,悲不自勝,眾皆感激。王世充立越王侗,祖尚遣使從之,侗授祖尚光州總管。及世充自立,遂舉州歸款,高祖嘉之,賜璽書勞勉,拜光州刺史,封弋陽郡公。武德六年,從趙郡王孝恭討輔公示石,為前軍總管,攻其宣、歙州,克之。進擊賊帥馮惠亮、陳正通,並破之。賊平,以功授蔣州刺史。又歷壽州都督、瀛州刺史,並有能名。貞觀初,交州都督、遂安
 公壽以貪冒得罪,太宗思求良牧,朝臣咸言祖尚才兼文武,廉平正直。徵至京師,臨朝謂之曰:「交州大籓,去京甚遠,須賢牧撫之。前後都督皆不稱職,卿有安邊之略,為我鎮邊,勿以道遠為辭也。」祖尚拜謝而出,既而悔之,以舊疾為辭。太宗遣杜如晦諭旨,祖尚固辭。又遣其妻兄周範往諭之曰:「匹夫相許,猶須存信。卿面許朕,豈得後方悔之?宜可早行,三年必自相召,卿勿推拒,朕不食言。」對曰:「嶺南瘴癘,皆日飲酒,臣不便酒,去無還理。」太宗
 大怒曰:「我使人不從,何以為天下命!」斬之於朝,時年三十餘。尋悔之,使復其官廕。



 劉世讓,字元欽,雍州醴泉人也。仕隋征仕郎。高祖入長安,世讓以湋川歸國,拜通議大夫。時唐弼餘黨寇扶風,世讓自請安輯,許之,俄得數千人。復為安定道行軍總管,率兵以拒薛舉,戰敗,世讓及弟寶俱為舉軍所獲。舉將至城下,令紿說城中曰:「大軍五道已趣長安,宜開門早降。」世讓偽許之,因告城中曰:「賊兵多少,極於此矣。宜善
 自固,以圖安全。」舉重其執節,竟不之害。太宗時屯兵高墌,世讓潛遣寶逃歸,言賊中虛實;高祖嘉之,賜其家帛千匹。及賊平,得歸,授彭州刺史。尋領陜東道行軍總管,與永安王孝基擊呂崇茂於夏縣,諸軍敗績,世讓與唐儉俱為賊所獲。獄中聞獨孤懷恩有逆謀,逃還以告高祖。時高祖方濟河,將幸懷恩之營,聞難驚曰:「劉世讓之至,豈非天命哉!」因勞之曰:「卿往陷薛舉,遣弟潛效款誠,今復冒危告難,是皆憂國忘身也。」尋封弘農郡公,賜
 莊一區、錢百萬。累轉並州總管,統兵屯於雁門。突厥處羅可汗與高開道、苑君璋合眾,攻之甚急。鴻臚卿鄭元璹先使在蕃,可汗令元璹來說之,世讓厲聲曰:「大丈夫奈何為夷狄作說客耶!」經日餘,虜乃退。及元璹還,述世讓忠貞勇干,高祖下制褒美之,錫以良馬。未幾,召拜廣州總管。將之任,高祖問以備邊之策,世讓答曰:「突厥南寇,徒以馬邑為其中路耳。如臣所計,請於崞城置一智勇之將,多儲金帛,有來降者厚賞賜之,數出奇兵略其
 城下,芟踐禾稼,敗其生業。不出歲餘,彼當無食,馬邑不足圖也。」高祖無可任者,乃使馳驛往經略之。突厥懼其威名,乃縱反間,言世讓與可汗通謀,將為亂。高祖不之察,遂誅世讓,籍沒其家。貞觀初,突厥來降者言世讓初無逆謀,始原其妻子。



 劉蘭,字文鬱,青州北海人也。仕隋鄱陽郡書佐。頗涉經史,善言成敗。然性多兇狡,見隋末將亂,交通不逞。於時北海完富,蘭利其子女玉帛,與群盜相應,破其本鄉城
 邑。武德中,淮安王神通為山東道安撫大使,蘭率宗黨往歸之。以功累遷尚書員外郎。貞觀初,梁師都尚據朔方,蘭上言攻取之計。太宗善之,命為夏州都督府司馬。時梁師都以突厥之師頓於城下,蘭偃旗臥鼓,不與之爭鋒,賊徒宵遁,蘭追擊破之,遂進軍夏州。及師都平,以功遷豐州刺史,徵為右領軍將軍。十一年,幸洛陽,以蜀王愔為夏州都督。愔不之籓,以蘭為長史,總其府事。時突厥攜離,有鬱射設阿史那摸末率其部落入居河南。
 蘭縱反間以離其部落,頡利果疑摸末,摸末懼,而頡利又遣兵追之,蘭率眾逆擊,敗之。太宗以為能,超拜豐州刺史,再轉夏州都督,封平原郡公。貞觀末,以謀反腰斬。右驍衛大將軍丘行恭探其心肝而食之,太宗聞而召行恭讓之曰:「典刑自有常科,何至於此!必若食逆者心肝而為忠孝,則劉蘭之心為太子諸王所食,豈至卿邪?」行恭無以答。



 李君羨者,洺州武安人也。初為王世充驃騎,惡世充之
 為人,乃與其黨叛而來歸,太宗引為左右。從討劉武周及王世充等,每戰必單騎先鋒陷陣,前後賜以宮女、馬牛、黃金、雜彩,不可勝數。太宗即位,累遷華州刺史,封武連郡公。貞觀初,太白頻晝見,太史占曰:「女三昌。」又有謠言:「當有女武王者。」太宗惡之。時君羨為左武衛將軍,在玄武門。太宗因武官內宴,作酒令,各言小名。君羨自稱小名「五娘子」,太宗愕然,因大笑曰:「何物女子,如此勇猛!」又以君羨封邑及屬縣皆有「武」字,深惡之。會御史奏君
 羨與妖人員道信潛相謀結,將為不軌,遂下詔誅之。天授二年,其家屬詣闕稱冤,則天乃追復其官爵,以禮改葬。



 史臣曰:侯君集摧兇克敵,效用居多;恃寵矜功,粗率無檢,棄前功而罹後患,貪愚之將明矣。張亮聽公穎之妖言,恃弓長之邪讖,義兒斯畜,惡跡遂彰,雖道裕云反狀未形,而詭詐之性,於斯驗矣。萬徹籌深行陣,勇冠戎夷,不能保其首領,以至誅戮。夫二三子,非慎始而保終也。



 贊曰:君子立功,守以謙沖。小人得位,足為身害。侯、張兇險,望窺聖代。雄若韓、彭,難逃菹醢。



\end{pinyinscope}