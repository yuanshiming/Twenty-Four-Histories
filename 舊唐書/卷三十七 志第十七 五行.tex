\article{卷三十七 志第十七 五行}

\begin{pinyinscope}

 昔禹得《河圖》、《洛書》六十五字,治水有功,因而寶之。殷太師箕子入周,武王訪其事,乃陳《洪範》九疇之法,其一曰五行。漢興,董仲舒、劉向治《春秋》,論災異,乃引九疇之說,附
 於二百四十二年行事,一推咎徵天人之變。班固敘漢史,採其說《五行志》。綿代史官,因而纘之。今略舉大端,以明變怪之本。



 《經》曰:「水曰潤下,火曰炎上,木曰曲直,金曰從革,土爰稼穡。」又曰:「建用皇極。」《傳》曰:「畋獵不時,飲食不享,出入不節,奪民農時,及有奸謀,則木不曲直。棄法律,逐功臣,殺太子,以妾為妻,則火不炎上。好治宮室,飾臺榭,內淫亂,犯親戚,侮父兄,則稼穡不成。好戰功,輕百姓,飾城郭,侵邊境,則金不從革。簡宗廟,不禱祠,廢祭祀,逆
 天時,則水不潤下。」《經》曰「敬用五事」,謂「貌曰恭,言曰從,視曰明,聽曰聰,思曰睿。恭作肅,從作乂,明作哲,聰作謀,睿作聖。」又曰「建用皇極」,「皇建其有極」。《傳》曰「貌之不恭,是謂不肅,厥咎狂,厥罰恆雨,厥極兇。時則有服妖,時則有龜孽,時則有雞禍,時則有下體生上之痾,時則有青眚青祥。凡草木之類謂之妖,蟲豸之類謂之孽,六畜謂之禍,及人謂之痾,甚則異物生謂之眚,身外而來謂之祥也。言之不從,是謂不乂,厥咎僭,厥罰恆昜,厥極憂。時則有詩妖,時則有介蟲之孽,時則有犬禍,時則有口舌之
 痾,時則有白眚白祥。視之不明,是謂不哲,厥咎豫,厥罰恆燠,厥極疾。時則有草妖,時則有臝蟲之孽,時則有羊禍,時則有目痾,時則有赤眚赤祥。聽之不聰,是謂不謀,厥咎急,厥罰恆寒,厥極貧。時則有鼓妖,時則有魚孽,時則有豕禍,時則有耳痾,時則有黑眚黑祥。思之不睿,是謂不聖,厥咎蒙,厥罰恆風,厥極兇短折。時則有脂夜之妖,時則有華孽,時則有牛禍,時則有心腹之痾,時則有黃眚黃祥。皇之不極,是謂不建,厥咎眊,厥罰恆陰,厥極
 弱。時則有射妖,時則有龍蛇之孽,時則有馬禍,時則有下體代上之痾,時則有日月亂行、星辰逆行。」九疇名數十五,其要五行、皇極之說,前賢所以窮治亂之變,談天人之際,蓋本於斯。故先錄其言,以傳於事。京房《易傳》曰:「臣事雖正,專必地震。其震,於水則波,於木則搖,於屋則瓦落,大經在闢而易臣,茲謂陰動。」又曰:「小人剝廬,厥妖山崩,茲謂陰乘陽,弱勝強。」劉向曰:「金木水沴土,地所以震。」《春秋》災異,先書地震、日蝕,惡陰盈也。



 貞觀十二年正月二十二日,松、叢二州地震,壞人廬舍。二十年九月十五日,靈州地震,有聲如雷。二十三年八月一日,晉州地震,壞人廬舍,壓死者五十餘人。三日,又震。十一月五日,又震。永徽元年四月一日,又震。六月十二日,又震。高宗顧謂侍臣曰:「朕政教不明,使晉州之地,屢有震動。」侍中張行成曰:「天,陽也;地,陰也。陽,君象;陰,臣象。君宜轉動,臣宜安靜。今晉州地震,彌旬不休,臣將恐女謁用事,大臣陰謀。且晉州,陛下本封,今地屢震,尤
 彰其應。伏願深思遠慮,以杜其萌。」帝深然之。開元二十二年二月十八日,秦州地震。先是,秦州百姓聞州西北地下殷殷有聲,俄而地震,壞廨宇及居人廬舍數千間,地拆而復合,震經時不定,壓死百餘人。玄宗令右丞相蕭嵩致祭山川,又遣倉部員外郎韋伯陽往宣慰,存恤所損之家。



 至德元年十一月辛亥朔,河西地震有聲,地裂陷,壞廬舍,張掖、酒泉尤甚。至二載六月始止。大歷二年十一月壬申,京師地震,有聲自東北來,如雷
 者三。四年二月丙辰夜,京師地震,有聲如雷者三。貞元三年十一月己卯夜,京師地震,是夕者三,巢鳥皆驚,人多去室。東都、蒲、陜亦然。四年正月朔日,德宗御含元殿受朝賀。是日質明,殿階及欄檻三十餘間,無故自壞,甲士死者十餘人。其夜,京師地震。二日又震,三日又震,十八日又震,十九日又震,二十日又震。帝謂宰臣曰:「蓋朕寡德,屢致後土震驚,但當修政,以答天譴耳。」二十三日又震,二十四日又震,二十五日又震,時金、房州尤甚,
 江溢山裂,屋宇多壞,人皆露處。至二月三日壬午,又震,甲申又震,乙酉又震,丙申又震。三月甲寅,已震,己未又震,庚午又震,辛未又震。京師地生毛,或白或黃,有長尺餘者。五月丁卯,又震。八月甲辰,又震,其聲如雷。九年四月辛酉,京師又震,有聲如雷。河中尤甚,壞城壘廬舍,地裂水湧。十年四月戊申,又震。十三年十月乙未日午時,震從東來,須臾而止。



 元和七年八月,京師地震。憲宗謂侍臣曰:「昨地震,草樹皆搖,何祥異也?」宰臣李絳曰:「昔
 周時地震,三川竭,太史伯陽甫謂周君曰:『天地之氣,不過其序。若過其序,人亂也。人政乖錯,則上感陰陽之氣,陽伏而不能出,陰迫而不能升,於是有地震。』又孔子修《春秋》,所紀災異,先地震、日蝕,蓋地載萬物,日君象,政有感傷,天地見眚,書之示戒,用儆後王。伏願陛下體勵虔恭之誠,動以利萬物、綏萬方為念,則變異自消,休徵可致。」九年三月丙辰,巂州地震,晝夜八十震方止,壓死者百餘人。大和九年三月乙卯,京師地震。開成元年
 二月乙亥夜四更,京師地震,屋瓦皆墜,戶牖之間有聲。二年十一月乙丑夜,地南北微震。大中三年十月,京師地震,振武、天德、靈武、鹽、夏等州皆震,壞軍鎮廬舍。



 武德六年七月二十日,巂州山崩,川水咽流。貞觀八年七月七日,隴右山崩,大蛇屢見。太宗問秘書監虞世南曰:「是何災異?」對曰:「春秋時梁山崩,晉侯召伯宗而問焉。對曰:『國主山川,故山崩川竭,君為之不舉,降服出次,祝幣以禮焉。』晉侯從之,卒亦無害。漢文帝九年,齊、楚地
 二十九山同日崩。文帝出令,郡國無來獻,施惠於天下,遠近歡洽,亦不為災。後漢靈帝時,青蛇見御座。晉惠帝時,大蛇長三百步,經市入廟。今蛇見山澤,蓋深山大澤,實生龍蛇,亦不足怪也。唯修德可以消變。」上然之。十七年八月四日,涼州昌松縣鴻池谷有石五,青質白文,成字曰「高皇海出多子李元王八十年太平天子李世民千年太子李治書燕山人士樂太國主尚汪譚獎文仁邁千古大王五王六王七王十NO毛才子七佛八菩薩
 及上果佛田天子文武貞觀昌大聖延四方上下治示孝仙戈入為善。」涼州奏。其年十一月三日,遣使祭之,曰:「嗣天子某,祚繼鴻業,君臨宇縣,夙興旰食,無忘於政,導德齊禮,愧於前修。天有成命,表瑞貞石,文字昭然,歷數唯永。既旌高廟之業,又錫眇身之祚。迨於皇太子治,亦降貞符,具紀姓氏,列於石言。仰瞻睿漢,空銘大造,甫惟寡薄,彌增寅懼。敢因大禮,重薦玉帛,上謝明靈之貺,以申祗心慄之誠。」



 永徽四年八月二十日,隕石十八於同
 州馮翊縣,光曜,有聲如雷。上問於志寧曰:「此何祥也?當由朕政之有闕。」對曰:「按《春秋》,隕石於宋五,內史過曰:『是陰陽之事,非吉兇所生。』自古災變,杳不可測,但恐物之自爾,未必關於人事。陛下發書誡懼,責躬自省,未必不為福矣。」永昌中,華州敷水店西南坡,白晝飛四五里,直抵赤水,其坡上樹木禾黍,宛然無損。則天時,新豐縣東南露臺鄉,因大風雨雹震,有山踴出,高二百尺,有池周三頃,池中有龍鳳之形、禾麥之異。則天以為休徵,
 名為慶山。荊州人俞文俊詣闕上書曰:「臣聞天氣不和而寒暑隔,人氣不和而疣贅生,地氣不和而堆阜出。今陛下以女主居陽位,反易剛柔,故地氣隔塞,山變為災。陛下以為慶山,臣以為非慶也。誠宜側身修德,以答天譴。不然,恐災禍至。」則天怒,流於嶺南。開元十七年四月五日,大風震電,藍田山開百餘步。乾元二年六月,虢州閺鄉縣界黃河內女媧墓,天寶十三載因大雨晦冥,失其所在,至今年六月一日夜,河濱人家忽聞風雨
 聲,曉見其墓踴出,上有雙柳樹,下有巨石二,柳各長丈餘。郡守圖畫以聞,今號風陵堆。大歷十三年,郴州黃芩山崩震,壓殺數百人。建中初,魏州魏縣西四十里,忽然土長四五尺數畝,里人駭異之。明年,魏博田悅反,德宗命河東馬燧、潞州李抱真討之,營於陘山。幽州硃滔、恆州王武俊帥兵救田悅,王師退保魏縣西。硃滔、武俊、田悅引軍與王師對壘。三年十一月,硃滔僭稱冀王,武俊稱趙王,田悅稱魏王。悅時壘正當土長之所,及僭
 署告天,乃因其長土為壇以祭。魏州功曹韋稔為《益土頌》以媚悅。馬燧聞之,笑曰:「田悅異常賊也。」



 貞觀十一年七月一日,黃氣竟天,大雨,穀水溢,入洛陽宮,深四尺,壞左掖門,毀宮寺一十九;洛水暴漲,漂六百餘家。帝引咎,令群臣直言政之得失。中書侍郎岑文本曰:「伏唯陛下覽古今之事,察安危之機,上以社稷為重,下以億兆為念。明選舉,慎賞罰,進賢才,退不肖。聞過即改,從諫如流。為善在於不疑,出令期於必信。頤神養性,省畋游之
 娛;去奢從儉,減工役之費。務靜方內,不求闢土;載橐弓矢,而無忘武備。凡此數者,願陛下行之不怠,必當轉禍為福,化咎為祥。況水之為患,陰陽常理,豈可謂之天譴而系聖心哉!」十三日,詔曰:「暴雨為災,大水泛溢,靜思厥咎,朕甚懼焉。文武百僚,各上封事,極言朕過,無有所諱。諸司供進,悉令減省。凡所力役,量事停廢。遭水之家,賜帛有差。」二十日,詔廢明德宮及飛山宮之玄圃院,分給河南、洛陽遭水戶。九月,黃河泛濫,壞陜州河北縣及太
 原倉,毀河陽中水單,太宗幸白馬阪以觀之。



 永徽五年六月,恆州大雨,自二日至七日。滹沱河水泛溢,損五千三百家。總章二年七月,冀州奏:六月十三日夜降雨,至二十日,水深五尺,其夜暴水深一丈已上,壞屋一萬四千三百九十區,害田四千四百九十六頃。九月十八日,括州暴風雨,海水翻上,壞永嘉、安固二縣城百姓廬舍六千八百四十三區,殺人九千七十、牛五百頭,損田苗四千一百五十頃。咸亨元年五月十四日,連日澍
 雨,山水溢,溺死五千餘人。永淳元年六月十二日,連日大雨,至二十三日,洛水大漲,漂損河南立德弘敬、洛陽景行等坊二百餘家,壞天津橋及中橋,斷人行累日。先是,頓降大雨,沃若懸流,至是而泛溢沖突焉。西京平地水深四尺已上,麥一束止得一二升,米一斗二百二十文,布一端止得一百文。國中大饑,蒲、同等州沒徙家口並逐糧,饑餒相仍,加以疾疫,自陜至洛,死者不可勝數。西京米斗三百已下。二年三月,洛州黃河水溺河陽
 縣城,水面高於城內五六尺。自鹽坎已下至縣十里石灰,並平流,津橋南北道無不碎破。文明元年七月,溫州大水,漂流四千餘家。長安三年,寧州大霖雨,山水暴漲,漂流二千餘家,溺死者千餘人,流尸東下。十七日,京師大雨雹,人有凍死者。四年,自九月至十月,晝夜陰晦,大雨雪。都中人畜,有餓凍死者。令開倉賑恤。



 神龍元年七月二十七日,洛水漲,壞百姓廬舍二千餘家。詔九品已上直言極諫,右衛騎曹宋務光上疏曰:



 臣聞自
 昔後王,樂聞過,罔不興;拒忠諫,罔不亂。何者,樂聞過則下情通,下情通則政無缺,此其所以興也;拒忠諫則群議壅,群議壅則主孤立,此其所以亂也。伏見明敕,令文武九品已上直言極諫,大哉德音,其堯、舜之用心,禹、湯之責己也!



 臣嘗讀書,觀天人相與之際,考休咎冥符之兆,有感必通,其間甚密。是以政失於此,變生於彼,亦猶影之像形,響之赴聲,動而輒隨,各以類應。故《易》曰:「天垂象,見吉兇,聖人象之。」竊見自夏已來,水氣悖戾,天下郡
 國,多罹其災。去月二十七日,洛水暴漲,漂損百姓。謹按《五行傳》曰:「簡宗廟,廢祭祀,則水不潤下。」夫王者即位,必郊祀天地,嚴配祖宗,是故鬼神歆饗,多獲福助。自陛下光臨寶極,綿歷炎涼,郊廟遲留,不得殷薦,山川寂寞,未議懷柔。暴水之災,殆因此發。臣又按,水者陰類,臣妾之道。陰氣盛滿,則水泉迸溢。加之虹蜺紛錯,暑雨滯淫,雖丁厥時,而汩恆度,亦陰勝之沴也。臣恐後庭近習,或有離中饋之職,幹外朝之政。伏願深思天變,杜絕其萌。又
 自春及夏,牛多病死,疫氣浸淫,於今未息。謹按《五行傳》曰:「思之不睿,時則有牛禍。」意者萬機之事,陛下或未躬親乎?昔太戊有異木生於朝,伊陟戒以修德,厥妖用殄;高宗有飛雉雊於鼎,祖己陳以政事,殷道再興。此皆視履考祥,轉禍為福之明鑒也。晁錯曰:「五帝其臣不及,則自親之。」今朝廷怪異,雖則多矣,然皆仰知陛下天光。伏願勤思德容,少凝大化,以萬方為念,不以聲色為娛,以百姓為憂,不以犬馬為樂。暫勞宵旰,用緝明良,豈不休
 哉!天下幸甚!



 臣聞三王之朝,不能免淫亢;太平之時,不能無小孽。備御之道,存乎其人。若細微之災,恬而不怪,及禍變成象,駭而圖之,猶水決而繕防,疾困而求藥,雖復黽勉,亦何救哉!夫災變應天,實系人事,故日蝕修德,月蝕修刑。若乃雨暘或愆,則貌言為咎,雩禜之法,在於禮典。今暫逢霖雨,即閉坊門,棄先聖之明訓,遵後來之淺術,時偶中之,安足神耶?蓋當屏翳收津,豐隆戢響之日也。豈有一坊一市,遂能感召皇靈;暫閉暫開,便欲發
 揮神道。必不然矣,何其謬哉!至今巷議街言,共呼坊門為宰相,謂能節宣風雨,變理陰陽。夫如是,則赫赫師尹,便為虛設;悠悠蒼生,復何所望?



 自數年已來,公私俱竭,戶口減耗。家無接新之儲,國無候荒之蓄。陛下不出都邑,近觀朝市,則以為率土之人,既康且富。及至踐閭陌,視鄉亭,百姓衣牛馬之衣,食犬彘之食,十室而九空,丁壯盡於邊塞,孤孀轉於溝壑,猛吏淫威奪其毒,暴徵急政破其資。馬困斯跌,人窮乃詐,或起為奸盜,或競為流
 亡,從而刑之,良可悲也!臣觀今之甿俗,率多輕佻,人貧而奢不息,法設而偽不止。長吏貪冒,選舉私謁。樂多繁淫,器尚浮巧。稼穡之人少,商旅之人多。誠願坦然更化,以身先之,端本澄源,滌瑕蕩穢。接凋殘之後,宜緩其力役;當久弊之極,宜法訓敦龐。良牧樹風,賢宰垂化,十年之外,生聚方足,三代之美,庶幾可及。



 臣聞太子者,君之貳,國之本,《易》有其卦,天有其星,今古相循,率由茲道。陛下自登皇極,未建元良,非所以守器承祧,養德贊業。離
 明不可輟曜,震位不可久虛,伏願早擇賢能,以光儲副,上安社稷,下慰黎元。且姻戚之間,謗議所集,假令漢帝無私於廣國,元規切讓於中書,天下之人,安可戶說。稽疑成患,馮寵生災,所謂愛之適足以害之。至如武三思等,誠能輟其機務,授以清閑,厚祿以富其身,蕃錫以獎其意,家國俱泰,豈不優乎?



 夫爵賞者,君之重柄。《傳》曰:「惟名與器,不可假人。」自頃官賞,頗亦乖謬,大勛未滿於人聽,高秩已越於朝倫,貪天之功,以為己力。秘書監鄭普
 思、國子祭酒葉靜能,或挾小道以登硃紫,或因淺術以取銀黃,既虧國經,實悖天道。《書》曰:「制理於未亂,保邦於未危。」此誠理亂安危之時也。伏願欽祖宗之丕烈,傷王業之艱難,遠佞人,親有德,乳保之愛,妃主之家,以時接見,無令媟瀆。



 凡此數者,當今急務,唯陛下留神採納,永保康寧。



 疏奏不省。



 右僕射唐休璟以霖雨為害,咎在主司,上表曰:「臣聞天運其工,人代之而為理;神行其化,為政資之以和。得其理則陰陽以調,失其和則災沴斯作。
 故舉才而授,帝唯其難,論道於邦,官不必備。頃自中夏,及乎首秋,郡國水災,屢為人害。夫水,陰氣也,臣實主之。臣忝職右樞,致此陰沴,不能調理其氣,而乃曠居其官。雖運屬堯年,則無治水之用;位侔殷相,且闕濟川之功。猶負明刑,坐逃皇譴。皇恩不棄,其若天何?昔漢家故事,丞相以天災免職。臣竊遇聖時,豈敢塤顏居位。乞解所任,待罪私門,冀移陰咎之徵,復免夜行之眚。



 神龍二年三月壬子,洛陽東十里有水影,月餘乃滅。四月,洛水
 泛濫,壞天津橋,漂流居人廬舍,溺死者數千人。三年夏,山東、河北二十餘州大旱,饑饉死者二千餘人。景龍二年正月,滄州雨雹,大如雞卵。開元五年六月十四日,鞏縣暴雨連日,山水泛漲,壞郭邑廬舍七百餘家,人死者七十二;汜水同日漂壞近河百姓二百餘戶。八年夏,契丹寇營州,發關中卒援之。軍次澠池縣之闕門,野營谷水上。夜半,山水暴至,二萬餘人皆溺死,唯行網役夫樗蒲,覺水至,獲免逆旅之家,溺死死人漂入苑中如積。其年六月二十一日夜,暴雨,東都谷、洛溢,入西上陽宮,宮人死者
 十七八。畿內諸縣,田稼廬舍蕩盡。掌關兵士,凡溺死者一千一百四十八人。京城興道坊一夜陷為池,一坊五百餘家俱失。其年,鄧州三鴉口大水塞谷,初見二小兒以水相潑,須臾,有大蛇十圍已上,張口向天,人或斫射之,俄而暴雷雨,漂溺數百家。十年二月四日,伊水泛漲,毀都城南龍門天竺、奉先寺,壞羅郭東南角,平地水深六尺已上,入漕河,水次屋舍,樹木蕩盡。河南汝、許、仙、豫、唐、鄧等州,各言大水害秋稼,漂沒居人廬舍。十四年六
 月戊午,大風拔木發屋,端門鴟吻盡落,都城內及寺觀落者約半。七月十四日,瀍水暴漲,流入洛漕,漂沒諸州租船數百艘,溺死者甚眾,漂失楊、壽、光、和、廬、杭、瀛、棣租米一十七萬二千八百九十六石,並錢絹雜物等。因開斗門決堰,引水南入洛,漕水燥竭,以搜漉官物,十收四五焉。七月甲子,懷、衛、鄭、滑、汴、濮、許等州澍雨,河及支川皆溢,人皆巢舟以居,死者千計,資產苗稼無孑遺。滄州大風,海運船沒者十一二,失平盧軍糧五千餘石,舟人
 皆死。潤州大風從東北,海濤奔上,沒瓜步洲,損居人。是秋,天下八十五州言旱及霜,五十州水,河南、河北尤甚。十五年七月甲寅,雷震興教門樓兩鴟吻,燒樓柱,良久乃滅。二十日,鄜州雨,洛水溢入州城,平地丈餘,損居人廬舍,溺死者不知其數。二十一日,同州損郭邑及市,毀馮翊縣。八月八日,澠池縣夜有暴雨,澗水、穀水漲合,毀郭邑百餘家及普門佛寺。是歲,天下六十三州大水損禾稼、居人廬舍,河北尤甚。十八年六月乙丑,東都瀍水暴漲,漂損揚、楚、淄、德等州租船。壬午,東都洛水泛漲,壞天津、永濟二
 橋及漕渠斗門,漂損提象門外助鋪及仗舍,又損居人廬舍千餘家。二十七年八月,東京改作明堂,訛言官取小兒埋於明堂下,以為厭勝。村邑童兒藏於山谷,都城騷然,或言兵至。玄宗惡之,遣主客郎中王佶往東都及諸州宣慰百姓,久之乃定。二十九年,暴水,伊、洛及支川皆溢,損居人廬舍,秋稼無遺,壞東都天津橋及東西漕;河南北諸州,皆多漂溺。



 天寶十載,廣陵郡大風架海潮,淪江口大小船數千艘。十三載秋,京城連月澍雨,損秋
 稼。



 九月,遣閉坊市北門,蓋井,禁婦人入街市,祭玄冥大社,禜門。京城坊市墻宇,崩壞向盡。東方瀍、洛水溢堤穴,沖壞一十九坊。上元二年,京師自七月霖雨,八月盡方止。京城宮寺廬舍多壞,街市溝渠中漉得小魚。永泰元年,先旱後水。九月,大雨,平地水數尺,溝河漲溢。時吐蕃寇京畿,以水,自潰而去。二年夏,洛陽大雨,水壞二十餘坊及寺觀廨舍。河南數十州大水。大歷四年秋,大雨。是歲,自四月霖澍,至九月。京師米斗八百文,官出
 太倉米賤糶以救饑人。京城閉坊市北門,門置土臺,臺上置壇及黃幡以祈晴。秋末方止。五年夏,復大雨,京城饑,出太倉米減價以救人。十二年秋,大雨。是歲,春夏旱,至秋八月雨,河南尤甚,平地深五尺,河決,漂溺田稼。



 貞元二年夏,京師通衢水深數尺。吏部侍郎崔縱,自崇義里西門為水漂浮行數十步,街鋪卒救之獲免;其日,溺死者甚眾。東都、河南、荊南、淮南江河泛溢,壞人廬舍。四年八月,連雨,灞水暴溢,溺殺渡者百餘人。八年秋,大
 雨,河南、河北、山南、江淮凡四十餘州大水,漂溺死者二萬餘人。時幽州七月大雨,平地水深二丈;鄚、涿、薊、檀、平五州,平地水深一丈五尺。又徐州奏:自五月二十五日雨,至七月八日方止,平地水深一丈二尺,郭邑廬里屋宇田稼皆盡,百姓皆登丘塚山原以避之。



 元和七年正月,振武界黃河溢,毀東受降城。五月,饒、撫、虔、吉、信五州山水暴漲,壞廬舍,虔州尤甚,水深處四丈餘。八年五月,許州奏:大雨摧大隗山,水流出,溺死者千餘人。六月
 庚寅,京師大風雨,毀屋揚瓦,人多壓死。水積城南,深處丈餘,入明德門,猶漸車輻。辛卯,渭水暴漲,毀三渭橋,南北絕濟者一月。時所在霖雨,百源皆發,川瀆不由故道。丙申,富平大風,折樹一千二百株。辛丑,出宮人二百車,人得娶納,以水害誡陰盈也。九年秋,淮南、宣州大水。十一年五月,京畿大雨,害田四萬頃,昭應尤甚,漂溺居人。衢州山水湧,深三丈,壞州城,民多溺死。浮梁、樂平溺死者一百七十人,為水漂流不知所在者四千七百戶。潤、
 常、湖、陳、許等州各損田萬頃。十二年秋,大雨,河南北水,害稼。其年六月,京師大雨,街市水深三尺,壞廬舍二千家,含元殿一柱陷。十五年九月十一日至十四日,大雨兼雪,街衢禁苑樹無風而摧折、連根而拔者不知其數。仍令閉坊市北門以禳之。滄州大水。



 長慶二年十月,好畤山水泛漲,漂損居人三百餘家,河南陳、許二州尤甚。詔賑貸粟五萬石,量人戶家口多少,等第分給。大和三年四月,同官暴水,漂沒三百餘家。六年,徐州自六
 月九日大雨至十一日,壞民舍九百家。四年夏,鄆、曹、濮雨,壞城郭田廬向盡。蘇、湖二州水,壞六堤,水入郡郭,溺廬井。許州自五月大雨,水深八尺,壞郡郭居民大半。會昌元年七月,襄州漢水暴溢,壞州郭。均州亦然。則天時,宗秦客以佞幸為內史,受命之日,無雲而雷聲震烈,未周歲而誅。延和元年六月,河南偃師縣之李材村,有霹靂閃入人家,地震裂,闊丈餘,長十五里,測之無底。所裂之處,井廁相通,所沖之塚,棺柩出植平地無損,
 竟不知其故。儀鳳三十年一月十四日,雨水冰。開元十五年七月四日,雷震興教門兩鴟吻,欄檻及柱災。二十九年十一月二十二日,雨木冰,凝寒凍冽,數日不解。寧王見而嘆曰:「諺云『樹稼達官怕』,必有大臣當之。」其月王薨。乾元三年閏四月,大霧,大雨月餘。是月,史思明再陷東都,京師米斗八百文,人相食,殍骸蔽地。永泰元年二月甲子夜,雷電震烈。三月,降霜為木冰。辛亥,大風拔木。



 大歷二年三月辛亥夜,京師大風發屋。十一
 月,紛霧如雪,草木冰。十年四月甲申夜,大雨雹,暴風拔樹,飄屋瓦,宮寺鴟吻飄失者十五六,人震死者十二,損京畿田稼七縣。七月己未夜,杭州大風,海水翻潮,飄蕩州郭五千餘家,船千餘隻,全家陷溺者百餘戶,死者四百餘人;蘇、湖、越等州亦然。貞元二年正月,大雨雪,平地深尺餘。雪上有黃色,狀如浮埃。四年正月,陳留十里許雨木,皆大如指,長寸餘,木有孔通中,所下立者如植。其年,宣州暴雨震電,有物墜地,豬首,手腳各有兩指,執
 一赤斑蛇食之。逡巡,黑雲合,不見。八年二月,京師雨土。五月己未,暴風破屋拔樹,太廟屋及諸門寺署壞者不可勝計。十年六月辛丑晦,有水鳥集於左藏庫。其夜暴雨,大風拔樹十七年二月五日,大雨雹。七日,大霜。十六夜,大雨,震雷且電。十九日,大雨雪而電。元和三年四月壬申,大風毀含元殿西闕欄檻二十七間。八年三月丙子,大風拔崇陵上宮衙殿西鴟尾,並上宮西神門六戟竿折,行墻四十間醿壞。



 長慶元年九月壬寅,京師
 震電,大風雨。四年五月庚辰,大風吹壞延喜、景風二門。



 大和八年六月癸未,暴風雷雨壞長安縣廨及經行寺塔。同、華大旱。七月辛酉,定陵臺大風雨,震,東廓之下地裂一百三十尺,其深五尺。詔宗正卿李仍叔啟告修之。九年四月二十六日夜,大風,含元殿四鴟吻皆落,拔殿前樹三,壞金吾仗舍,廢樓觀內外城門數處,光化門西城墻壞七十七步。是日,廢長生院,起內道場,取李訓言沙汰僧尼故也。開成元年夏六月,鳳翔、麟游縣暴
 風雨,飄害九成宮正殿及滋善寺佛舍,壞百姓屋三百間,死者百餘人,牛馬不知其數。長安四年九月後,霖雨並雪,凡陰一百五十餘日,至神龍元年正月五日,誅二張,孝和反正,方晴霽。先天二年四月,陰,至六月一百餘日,至七月三日,誅竇懷貞等一十七家,方晴。景龍中,東都霖雨百餘日,閉坊市北門,駕車者苦甚污,街中言曰:「宰相不能調陰陽,致茲恆雨,令我污行。」會中書令楊再思過,謂之曰:「於理則然,亦卿牛劣耳。」貞元二
 十一年,順宗風疾,叔文用事,連月霖雨不霽。乃以憲宗為皇太子,制出日即晴。《傳》所謂「皇之不極,厥罰恆陰」,皆此數也。



 貞觀二年六月,京畿旱,蝗食稼。太宗在苑中掇蝗,咒之曰:「人以穀為命,而汝害之,是害吾民也。百姓有過,在予一人,汝若通靈,但當食我,無害吾民。」將吞之,侍臣恐上致疾,遽諫止之。上曰:「所冀移災朕躬,何疾之避?」遂吞之。是歲蝗不為患。開元四年五月,山東螟蝗害稼,分遣御史捕而埋之。汴州刺史倪若水拒御史,執
 奏曰:「蝗是天災,自宜修德。劉聰時,除既不得,為害滋深。」宰相姚崇牒報之曰:「劉聰偽主,德不勝妖;今日聖朝,妖不勝德。古之良守,蝗蟲避境,若言修德可免,彼豈無德致然。今坐看食苗,忍而不救,因此饑饉,將何以安?」卒行埋瘞之法,獲蝗一十四萬,乃投之汴河,流者不可勝數。朝議喧然,上復以問崇,崇對曰:「凡事有違經而合道,反道而適權者,彼庸儒不足以知之。縱除之不盡,猶勝養之以成災。」帝曰:「殺蟲太多,有傷和氣,公其思之。」崇曰:「若
 救人殺蟲致禍,臣所甘心。」八月四日,敕河南、河北檢校捕蝗使狄光嗣、康瓘、敬昭道、高昌、賈彥璿等,宜令待蟲盡而刈禾將畢,即入京奏事。諫議大夫韓思復上言曰:「伏聞河北蝗蟲,頃日益熾,經歷之處,苗稼都盡。臣望陛下省咎責躬,發使宣慰,損不急之務,去至冗之人。上下同心,君臣一德,持此至誠,以答休咎。前後捕蝗使望並停之。」上出符疏付中書姚崇,乃令思復往山東檢視蟲災之所,及還,具以聞。二十五年,貝州蝗食苗,有白鳥數
 萬,群飛食蝗,一夕而盡。明年,榆林關有虸蚄食苗,群雀來食,數日而盡。



 天寶三載,貴州紫蟲食苗,時有赤鳥群飛,自東北來食之。廣德元年秋,虸蚄食苗,關西尤甚,米斗千錢。興元元年秋,關輔大蝗,田稼食盡,百姓饑,捕蝗為食,蒸曝,去颺足翅而食之。明年夏,蝗尤甚,自東海西盡河、隴,群飛蔽天,旬日不息。經行之處,草木牛畜毛,靡有孑遺。關輔已東,穀大貴,餓饉枕道。京師大亂之後,李懷光據河中,諸軍進討,國用罄竭。衣冠之家,多
 有殍殕者。旱甚,灞水將竭,井皆無水。有司奏國用裁可支七旬。德宗減膳,不御正殿。百司不急之費,皆減之。元和元年夏,鎮、冀蝗,害稼。長慶三年秋,洪州旱,螟蝗害稼八萬頃。大和元年秋,旱,罷選舉。開成二年,河南、河北旱,蝗害稼;京師旱尤甚,徙市,閉坊南門。四年六月,天下旱,蝗食田,禱祈無效,上憂形於色。宰臣曰:「星官奏天時當爾,乞不過勞聖慮。」文宗懍然改容曰:「朕為天下主,無德及人,致此災旱。今又彗星謫見於上,若三日
 內不雨,當退歸南內,卿等自選賢明之君以安天下。」宰臣嗚咽流涕不能已。是歲,河南府界黑蟲食苗。河南、河北蝗,害稼都盡。鎮、定等州,田稼既盡,至於野草樹葉細枝亦盡。會昌元年,山南鄧、唐等州蝗,害稼。



 貞觀十三年四月二十九日,雲陽石燃方丈,晝如炭,夜則光見,投草木於其上則焚,歷年方止。證聖元年正月十六日夜,明堂火,延及天堂,京城光照如晝,至曙並為灰燼。則天欲避殿徹樂,宰相姚璹以為火因麻主,人護不謹,
 非天災也,不宜貶損。乃勸則天御端門觀酺,引建章故事,令薛懷義重造明堂以厭勝之。則天時,建昌王武攸寧置內庫,長五百步,二百餘間,別貯財物以求媚。一夕為天災所燔,玩好並盡。景龍中,東都凌空觀災,火自東北來,其金銅諸像,銷鑠並盡。開元五年,洪、潭二州災,火延燒郡舍。郡人先見火精赤暾暾飛來,旋即火發。十五年,衡州災,火延燒三四百家。郡人見物大如甕,赤如燭籠,此物所至,即火發。十八年二月十八日,大雨
 雪,俄又雷震,飛龍廄災。天寶二年六月七日,東都應天門觀災,延燒左右延福門,經日不減。九載三月,華嶽廟災。十載正月,大風,陜州運船失火,燒二百一十五只,損米一百萬石,舟人死者六百人,又燒商人船一百隻。其年八月六日,武庫災,燒二十八間十九架,兵器四十七萬件。寶應元年十一月,回紇焚東都宜春院,延及明堂,甲子日而盡。廣德元年十二月二十五夜,鄂州失火,燒船三千艘,延及岸上居人二千餘家,死者四五
 千人。大歷十年二月,莊嚴寺佛圖災。初有疾風,震雷薄擊,俄而火從佛圖中出,寺僧數百人急救之,乃止,棟宇無損。



 貞元七年,蘇州火。十九年四月,家令寺火。二十年四月,開業寺火。元和四年,御史臺舍火。七年,鎮州甲仗庫一十三間災,節度使王承宗殺主守,坐死者百餘人。承宗方拒天軍,而兵仗為災所焚,天意嫉惡也。十年四月,河陰轉運院火。十一月,獻陵寢宮永巷火。十一年十二月,未央宮及飛龍草場火,皆王承宗、李師道謀
 撓用兵,陰遣盜縱火也。時李師道於鄆州起宮殿,欲謀僭亂。既成,是歲為災並盡,俄而族滅。大和元年十月甲辰,昭德宮火,延燒至宣政東垣及門下省,至晡方息。八年十二月,昭成宮火。九年六月乙亥朔,西市火。會昌三年六月,萬年縣東市火,燒屋宇貨財不知其數。又西內神龍宮火。大順二年七月,汴州相國寺佛閣災。是日晚,微雨,震電,寺僧見赤塊在三門樓藤網中,周繞一匝而火作。良久,赤塊北飛,越前殿飛入佛閣網中,
 如三門周繞轉而火作。如是三日不息,訖為灰燼。



 貞觀初,白鵲巢於殿庭之槐樹,其巢合歡如腰鼓,左右稱賀。太宗曰:「吾常笑隋文帝好言祥端。瑞在得賢,白鵲子何益於事?」命掇之,送於野。高宗文明後,天下頻奏雌雉化為雄,或半化未化,兼以獻之,則天臨朝之兆。調露元年,突厥溫傅等未叛時,有鳴鵽群飛入塞,相繼蔽野,邊人相驚曰:「突厥雀南飛,突厥犯塞之兆也。」至二年正月,還復北飛,至靈夏已北,悉墜地而死,視之,皆無頭。裴
 行儉問右史苗神客曰:「鳥獸之祥,乃應人事,何也?」對曰:「人雖最靈,而稟性含氣,同於萬類,故吉兇兆於彼,而禍福應於此。聖王受命,龍鳳為嘉瑞者,和氣同也。故漢祖斬蛇而驗秦之必亡,仲尼感麟而知己之將死。夷羊在牧,殷紂已滅。鸜鵒來巢,魯昭出奔。鼠舞端門,燕剌誅死。大鳥飛集,昌邑以敗。是故君子虔恭寅畏,動必思義,雖在幽獨,如承大事,知神明之照臨,懼患難之及己。雉升鼎耳,殷宗側身以修德,鵩止坐隅,賈生作賦以敘命。卒
 以無患者,德勝妖也。」



 大歷八年四月戊申,乾陵上仙觀天尊殿,有雙鵲銜泥及柴,補殿之隙壞,凡十五處。其年九月,大鳥見於武功縣,群鳥隨而噪之。神策將軍張日芬射得之,肉翅狐首,四足,足有爪,其廣四尺三寸,其毛色赤,形類蝙蝠。十一年,渭州獲赤烏。十三年五月,左羽林軍鸜鵒乳雀。貞元三年三月,中書省梧桐樹有鵲以泥為巢。四年夏,汴、鄭二州群鳥皆飛入田緒、李納境內,銜木為城,高二三尺,方十里。緒、納惡之,命焚之,信
 宿而復,鳥口皆流血。十年四月,有大鳥飛集宮中,食雜骨數日,獲之,不食而死。六月辛未晦,水鳥集左藏庫。十四年秋,有鳥色青,類鳩鵲,息於宋郊,所止之處,群鳥翼衛,朝夕嗛稻粱以哺之。睢陽之人適野聚觀者旬日,人不知其名,郡人李翱見之曰:「此鸞也,鳳之次。」長慶元年六月,濮州雷澤縣人張憲家榆樹鳥巢,因風墮二雛,別樹鵲引二鳥雛於巢哺之。開成二年六月,真興門外野鵲巢於古塚。



 永徽中,黑齒常之戍河源軍,有狼
 三頭,白晝入軍門,射之斃。常之懼,求代。將軍李謹代常之軍,月餘卒。先天初,洛陽市人牽一羊,左肋下有人手,長尺許,以之乞丐。開元二年,韶州鼠害稼,千萬為群。三年,有熊白晝入廣陵城,月餘,都督李處鑒卒。永泰二年十一月,乾陵赤兔見。



 大歷二年三月,河中獻玄狐。四年九月己卯,虎入京城長壽坊元載私廟,將軍周皓格殺之。六年八月丁丑,太極殿內廓下獲白兔。八年七月,白鼠出內侍。十二年六月,苑內獲白鼠。十三年
 六月戊戌,隴右汧源縣軍士趙貴家,貓鼠同乳,不相害,節度使硃泚籠之以獻。宰相常袞率百僚拜表賀,中書舍人崔祐甫曰:「此物之失性也。天生萬物,剛柔有性,聖人因之,垂訓作則。禮,迎貓,為食田鼠也。然貓之食鼠,載在祀典,以其能除害利人,雖微必錄。今此貓對鼠,何異法吏不勤觸邪,疆吏不勤捍敵?據禮部式錄三瑞,無貓不食鼠之目。以此稱慶,理所未詳。以劉向《五行傳》言之,恐須申命憲司,察聽貪吏,誡諸邊境,無失儆巡,則貓能
 致功,鼠不為害。」帝深然之。



 建中四年五月,滑洲馬生角。貞元四年二月,太僕寺郊牛生犢,六足,太僕卿周皓白宰相李泌,請上聞,泌笑而不答。又京師人家豕生子,兩首四足,有司以白御史中丞竇參,請上聞,參寢而不奏。三月癸丑,鹿入京師西市門,眾殺之。元和七年十一月,龍州武安川畬田中嘉禾生,有麟食之,復生。麟之來,一鹿引之,群鹿隨之,光華不可正視。使畫工圖麟及嘉禾來獻。八年四月,長安西市門家豕生子,三耳八
 足,自尾分為二。大和九年八月,易定監軍小將家馬,因飲水吐出寶珠一,獻之。



 貞觀中,汾州言青龍見,吐物在空中,有光明如火。墜地,地陷,掘之得玄金,廣尺,長七寸。大足元年,虔州別駕得六眼龜,一夕而失。神龍中,渭河有蛤蟆,大如一石鼎,里人聚觀,數日而失。是歲,大水漂溺京城數百家,商州水入城門,襄陽水至樹杪。先天二年六月,西京朝堂磚階,無故自壞。磚下有大蛇長丈餘,蛤蟆大如盤,面目赤如火,相向鬥。俄而蛇
 入大樹,蛤蟆入於草。其年七月三日,玄宗誅竇懷貞、岑羲等十七家。開元四年六月,郴州馬嶺山下,有白蛇長六七尺,黑蛇長丈餘。兩蛇鬥,白蛇吞黑蛇,至粗處,口眼流血,黑蛇頭穿白蛇腹出,俄而俱死。旬日內桂陽大雨,山水暴溢,漂五百家,殺三百餘人。



 天寶中,洛陽有巨蛇,高丈餘,長百尺,出於芒山下。胡僧無畏見之,嘆曰:「此欲決水注洛城。」即以天竺法咒之,數日蛇死。祿山陷洛之兆也。李揆作相前一月,有大蛤蟆如床,見室之中,
 俄失所在。占者以為蟆天使也,有福慶之事。乾元二年九月,通州三岡縣放生池中,日氣下照,水騰波湧上,有黃龍躍出,高丈餘,又於龍旁數處,浮出明珠。大歷八年,京師金天門外水渠獲毛龜。貞元三年,李納獻毛龜。元和七年四月,舒州桐城縣有黃、青、白三龍各一,翼風雷自梅天陂起,約高二百尺,凡六里,降於浮塘坡。九年四月,道州二青龍見於江中。大和二年六月七日,密州卑產山北面有龍見。初,赤龍從西來,續有青
 龍、黃龍從南來,後有白龍、黑龍從山北來,並形狀分明。自申至戌,方散去。



 天寶初,臨川郡人李嘉胤所居柱上生芝草,狀如天尊像,太守張景夫拔柱以獻。上元二年七月甲辰,延英殿御座生白芝,一莖三花。肅宗制《玉靈芝詩》三篇,群臣皆賀。占曰:「白芝主喪。」明年,上皇、肅宗俱崩。二年九月,含輝院生金芝。永泰二年二月,京城槐樹有蟲食葉,其形類蠶。其年六月,太廟第二室芝草生。大歷四年三月,潤州上元縣芝草生,一莖四葉,
 高七寸。八年,廬州廬江縣紫芝生,高一丈五尺。九年九月,晉州神山縣慶唐觀檜樹已枯重榮。十二年五月甲子,成都府人郭遠,因樵獲瑞木一莖,有文曰「天下太平」四字,其年十一月,蔡州汝陽縣芝草生,紫莖黃蓋。興元元年八月,亳州真源縣大空寺僧院李樹,種來十四年,才長一丈八尺,今春枝忽上聳,高六尺,周圍似蓋,九尺餘。又先天太后墓槐樹上有靈泉漏出,今年六月,其上有雲氣五色,又黃龍再見於泉上。元和十一年十
 二月雷,桃李俱花。長慶三年十二月,水不冰,草萌芽,如正二月之候。



 神龍二年三月,洛陽東七里有水影,側近樹木車馬之影,歷歷見水影中,月餘方滅。乾元二年七月,嵐州合河關黃河水,四十里間,清如井水,經四日而後復。寶應元年九月甲午,華州至陜州二百餘里,黃河清,澄澈見底。大歷二年,醴泉出櫟陽,愈疾。貞元四年七月,自陜州至河陰,河水色如墨,流入汴河,止於汴州城下,一宿而復。寶歷二年,亳州言出聖
 水愈病。江淮已南,遠來奔湊求水。浙西觀察使李德裕奏論其妖。宰相裴度判汴州所申狀曰:「妖由人興,水不自作。」牒汴州觀察使填塞訖申。



 玄宗初即位,東都白馬寺鐵像頭無故自落於殿門外。後姚崇秉政,以僧惠範附太平亂政,謀汰僧尼,令拜父母,午後不出院,其法頗峻。大歷十三年二月,太僕寺廨有佛堂,堂內小脫空金剛左臂上忽有黑汗滴下,以紙承色,色即血也。明年五月,代宗崩。



 上元三年,楚州刺史崔侁獻定國寶
 十三:一曰玄黃天符,形如笏,長八寸,有孔,闢人間兵疫;二曰玉雞毛,白玉也,以孝理天下則見;三曰穀璧,白玉也,粟粒,無雕鐫之跡,王者得之,五穀豐熟;四曰西王母白環二,所在處外國歸伏;五曰碧色寶,圓而有光;六曰如意寶珠,大如雞卵;七曰紅色靺鞨,大如巨慄;八曰瑯玕珠二;九曰玉玦,形如玉環,四分缺一;十曰玉印,大如半手,理如鹿形,陷入印中;十一曰皇后採桑鉤,如箸,屈其末;十二曰雷公石斧,無孔;十三缺。凡十三寶。置之日中,白氣連天。初,楚州有尼
 曰真如,忽有人接之升天,天帝謂之曰:「下方有災,令第二寶鎮之。」即以十三寶付真如。時肅宗方不豫,以為瑞,乃改元寶應,仍傳位皇太子,此近白祥也。寶歷二年五月,神策軍修苑內古漢宮,掘得白玉床,其長六尺,以獻。



 大歷十年二月,京兆神策昭應婦人張氏,產一男二女。貞元八年二月,許州人李狗兒持杖上含元殿,擊欄檻,又擊殺所擒卒,誅之。十年四月,巨人跡見常州。元和二年,開紅崖冶役夫將化為虎,眾以水沃之,化
 而不果。長慶四年四月十七日,染坊作人張韶與卜者蘇玄明,於柴草車內藏兵仗,入宮作亂,二人對食於清思殿。是日,禁軍誅張韶等三十七人。寶歷二年十二月,延州人賀文妻產三男。大和九年,京師訛言鄭注為主上合金丹,須小兒心肝,密旨捕小兒。或相告云,某處失幾兒。人家扃鎖小兒甚密。上恐,遣中使喻之,乃止。開成二年十二月二十八日,狂人劉德廣入含元殿,詔付京兆府杖殺之。



 隋末有謠云:「桃李子,洪水繞
 楊山。」煬帝疑李氏有受命之符,故誅李金才。後李密據洛口倉以應其讖。隋文時,自長安故城東南移於唐興村置新都,今西內承天門正當唐興村門。今有大槐樹,柯枝森鬱,即村門樹也。有司以行列不正,將去之,文帝曰:「高祖嘗坐此樹下,不可去也。」調露中,高宗欲封嵩山,累草儀注,有事不行。有謠曰:「不畏登不得,但恐不得登。三度徵兵馬,旁道打騰騰。」高宗至山下遘疾,還宮而崩。永徽末,里歌有《桑條韋也》、《女時韋也》樂。及神龍
 中,韋後用事,鄭愔作《桑條歌》十篇上之。龍朔中,俗中飲酒令,曰:「子母去離,連臺龍抝倒。」俗謂杯盤為子母,又名盤為臺,即中宗廢於房州之應也。時里歌有《突厥鹽》,及則天遣尚書閻知微送武延秀,立知微為可汗,挾之入寇。如意初,里歌云:「黃麞黃麞草裏藏,彎弓射爾傷。」後契丹李萬榮叛,陷營州,則天令總管曹仁師、王孝傑等將兵百萬討之,大敗於黃麞谷,契丹乘勝至趙郡。垂拱已後,東都有《契苾兒歌》,皆淫艷之詞。後張易之兄弟有
 內嬖,易之小字契苾。元和小兒謠云:「打麥打麥三三三」,乃轉身曰:「舞了也。」及武元衡為盜所害,是元和十年六月三日。



 《五行傳》所謂詩妖,皆此類也。



 上元中為服令,九品已上佩刀礪等袋,紛帨為魚形,結帛作之,為魚像鯉,強之意也。則天時此制遂絕,景雲後又佩之。



 張易之為母阿臧為七寶帳,有魚龍鸞鳳之形,仍為象床、犀簟。則天令鳳閣侍郎李迥秀妻之,迥秀不獲已,然心惡其老,薄之。阿臧怒,出迥秀為定州刺史。



 中宗女安樂公
 主,有尚方織成毛裙,合百鳥毛,正看為一色,旁看為一色,日中為一色,影中為一色,百鳥之狀,並見裙中。凡造兩腰,一獻韋氏,計價百萬。又令尚方取百獸毛為韉面,視之各見本獸形。韋後又集鳥毛為韉面。安樂初出降武延秀,蜀川獻單絲碧羅籠裙,縷金為花鳥,細如絲發,鳥子大如黍米,眼鼻嘴甲俱成,明目者方見之。自安樂公主作毛裙,百官之家多效之。江嶺奇禽異獸毛羽,採之殆盡。開元初,姚、宋執政,屢以奢靡為諫,玄宗悉命宮
 中出奇服,焚之於殿廷,不許士庶服錦繡珠翠之服。自是採捕漸息,風教日淳。



 韋庶人妹七姨,嫁將軍馮太和,權傾人主,嘗為豹頭枕以闢邪,白澤枕以闢魅,伏熊枕以宜男。太和死。再嫁嗣虢王。及玄宗誅韋後,虢王斬七姨首以獻。



\end{pinyinscope}