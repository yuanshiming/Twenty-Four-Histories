\article{卷一百九十一}

\begin{pinyinscope}

 ○來俊臣周興傅游藝丘神勣索元禮侯思止萬國俊來子珣王弘義
 郭霸吉頊



 古今御天下者,其政有四:五帝尚仁,體文德也;三王仗義,立武功也;五霸崇信,取威令也;七雄任力,重刑名也。蓋仁義既廢,然後齊之以威刑;威刑既衰,而酷吏為用,於是商鞅、李斯譎詐設矣。持法任術,尊君卑臣,奮其策而鞭撻宇宙,持危救弊,先王不得已而用之,天下之人謂之苛法。降及兩漢,承其餘烈。於是前有郅都、張湯之徒持其刻,後有董宣、陽球之屬肆其猛。雖然異代,亦克
 公方,天下之人謂之酷吏,此又鞅、斯之罪人也!然而網既密而奸不勝矣。夫子曰:「刑罰不中,則人無所措手足。」誠哉,是言也!



 唐初革前古之敝,務於勝殘,垂衣而理,且七十載,而人不敢欺。由是觀之,在彼不在此。逮則天以女主臨朝,大臣未附;委政獄吏,剪除宗枝。於是來俊臣、索元禮、萬國俊、周興、丘神勣、侯思止、郭霸、王弘義之屬,紛紛而出。然後起告密之刑,制羅織之獄,生人屏息,莫能自固。至於懷忠蹈義,連頸就戮者,不可勝言。武後因
 之坐移唐鼎,天網一舉,而卒籠八荒;酷之為用,斯害也已。遂使酷吏之黨,橫噬於朝,制公卿之死命,擅王者之威力。貴從其欲,毒侈其心,天誅發於脣吻,國柄秉於掌握。兇慝之士,榮而慕之,身赴鼎鑊,死而無悔。若是者,何哉?要時希旨,見利忘義也!



 嘗試而論之,今夫國家行斧鉞之誅,設狴牢之禁以防盜者,雖雲固矣,而猶逾垣掘塚,揭篋探囊,死者於前,盜者於後,何者?以其間有欲也!然所徇者不過數金之資耳!彼酷吏與時上下,取重人
 主,無怵惕之憂,坐致尊寵;杖起卒伍,富擬封君,豈唯數金之利耶?則盜官者為幸矣!故有國者則必窒凱覦之路,杜僥幸之門,可不務乎!況乎樂觀時變,恣懷陰賊,斯又郅都、董宣之罪人也。異哉,又有效於斯者!中興四十載而有吉溫、羅希奭之蠹政,又數載而有敬羽、毛若虛之危法。朝經四葉,獄訟再起,比周惡黨,剿絕善人。屢撓將措之刑,以傷太和之氣,幸災樂禍,茍售其身,此又來、索之罪人也!



 嗚呼!天道禍淫,人道惡殺,既為禍始,必以
 兇終。故自鞅、斯至於毛、敬,蹈其跡者,卒以誅夷,非不幸也。



 嗚呼!執愚賈害,任天下之怨;反道辱名,歸天下之惡。或肆諸原野,人得而誅之;或投之魑魅,鬼得而誅之。天人報應,豈虛也哉!俾千載之後,聞其名者,曾蛇豕之不若。



 悲夫!昔《春秋》之義,善惡不隱,今為《酷吏傳》,亦所以示懲勸也。語曰:「前事不忘,將來之師。」意在斯乎!意在斯乎!



 來俊臣,雍州萬年人也。父操,博徒。與鄉人蔡本結友,遂通其妻,因樗蒲贏本錢數十萬,本無以酬,操遂納本妻。
 入操門時,先已有娠,而生俊臣。兇險不事生產,反覆殘害,舉無與比。曾於和州犯奸盜被鞫,遂妄告密。召見奏,刺史東平王續杖之一百。後續天授中被誅,俊臣復告密,召見,奏言前所告密是豫、博州事,枉被續決杖,遂不得申。則天以為忠,累遷侍御史,加朝散大夫。按制獄,少不會意者,必引之,前後坐族千餘家。



 二年,擢拜左臺御史中丞。朝廷累息,無交言者,道路以目。與侍御史侯思止、王弘義、郭霸、李仁敬,司刑評事康暐、衛遂忠等,同惡
 相濟。招集無賴數百人,令其告事,共為羅織,千里響應。欲誣陷一人,即數處別告,皆是事狀不異,以惑上下。仍皆云:「請付來俊臣推勘,必獲實情。」則天於是於麗景門別置推事院,俊臣推勘必獲,專令俊臣等按鞫,亦號為新開門。但入新開門者,百不全一。弘義戲謂麗景門為「例竟門」,言入此門者,例皆竟也。



 俊臣與其黨硃南山輩造《告密羅織經》一卷,皆有條貫支節,布置事狀由緒。



 俊臣每鞫囚,無問輕重,多以醋灌鼻,禁地牢中,或盛之甕
 中,以火圜繞炙之,並絕其糧餉,至有抽衣絮以啖之者。又令寢處糞穢,備諸苦毒。自非身死,終不得出。每有赦令,俊臣必先遣獄卒盡殺重囚,然後宣示。



 又以索元禮等作大枷,凡有十號:一曰定百脈,二曰喘不得,三曰突地吼,四曰著即承,五曰失魂膽,六曰實同反,七曰反是實,八曰死豬愁,九曰求即死,十曰求破家。復有鐵籠頭連其枷者,輪轉於地,斯須悶絕矣。囚人無貴賤,必先布枷棒於地,召囚前曰:「此是作具。」見之魂膽飛越,無不自
 誣矣。則天重其賞以酬之,故吏競勸為酷矣。由是告密之徒,紛然道路;名流僶俛閱日而已。朝士多因入朝,默遭掩襲,以至於族,與其家無復音息。故每入朝者,必與其家訣曰:「不知重相見不?」



 如意元年,地官尚書狄仁傑、益州長史任令暉、冬官尚書李游道、秋官尚書袁智宏、司賓卿崔神基、文昌左丞盧獻等六人,並為其羅告。俊臣既以族人家為功,茍引之承反,乃奏請降敕,一問即承,同首例得減死。及脅仁傑等反,仁傑嘆曰:「大周革命,
 萬物惟新,唐朝舊臣,甘從誅戮。反是實。」俊臣乃少寬之。其判官王德壽謂仁傑曰:「尚書事已爾,得減死。德壽今業已受驅策,欲求少階級,憑尚書牽楊執柔,可乎?」仁傑曰:「若之何?」德壽曰:「尚書昔在春官時,執柔任某司員外,引之可也。」仁傑曰:「皇天后土,遣狄仁傑行此事!」以頭觸柱,血流被面,德壽懼而止焉。



 仁傑既承反,有司但待報行刑,不復嚴備。仁傑得憑守者求筆硯,拆被頭帛書之,敘冤苦,置於綿衣,遣謂德壽曰:「時方熱,請付家人去其
 綿。」德壽不復疑矣,家人得衣中書,仁傑子光遠持之稱變,得召見。則天覽之愕然,召問俊臣曰:「卿言仁傑等承反,今子弟訟冤,何故也?」俊臣曰:「此等何能自伏其罪!臣寢處甚安,亦不去其巾帶。」則天令通事舍人周綝視之。俊臣遽令獄卒令假仁傑等巾帶,行立於西,命綝視之。綝懼俊臣,莫敢西顧,但視東唯諾而已。俊臣令綝少留,附進狀,乃令判官妄為仁傑等作謝死表,代署而進之。鳳閣侍郎樂思晦男年八九歲,其家已族,宜隸於司農,
 上變,得召見,言「俊臣苛毒,願陛下假條反狀以付之,無大小皆如狀矣。」則天意少解,乃召見仁傑曰:「卿承反何也?」仁傑等曰:「不承反,臣已死於枷棒矣。」則天曰:「何謂作謝死表?」仁傑曰:「無。」因以表示之,乃知其代署,遂出此六家。



 俊臣復按大將軍張虔勖、大將軍內侍範雲仙於洛陽牧院。虔勖等不堪其苦,自訟於徐有功,言辭頗厲。俊臣命衛士以亂刀斬殺之。雲仙亦言歷事先朝,稱所司冤苦,俊臣命截去其舌。士庶破膽,無敢言者。



 俊臣累坐
 贓,為衛吏紀履忠所告下獄。長壽二年,除殿中丞。又坐贓,出為同州參軍。逼奪同列參軍妻,仍辱其母。



 萬歲通天元年,召為合宮尉,擢拜洛陽令、司農少卿。則天賜其奴婢十人,當受於司農。時西蕃酋長阿史那斛瑟羅家有細婢,善歌舞,俊臣因令其黨羅告斛瑟羅反,將圖其婢。諸蕃長詣闕割耳剺面訟冤者數十人,乃得不族。時綦連耀、劉思禮等有異謀,明堂尉吉頊知之,不自安,以白俊臣發之,連坐族者數十輩。俊臣將擅其功,復羅告
 頊,得召見,僅而免。



 俊臣先逼妻太原王慶詵女。俊臣與河東衛遂忠有舊。遂忠行雖不著,然好學,有詞辯。嘗攜酒謁俊臣,俊臣方與妻族宴集,應門者紿云:「已出矣。」遂忠知妄,入其宅,慢罵毀辱之。俊臣恥其妻族,命毆擊反接,既而免之,自此構隙。



 俊臣將羅告武氏諸王及太平公主、張易之等,遂相掎摭,則天屢保持之。而諸武及太平公主恐懼,共發其罪。乃棄市。國人無少長皆怨之,競剮其肉,斯須盡矣。



 中宗神龍元年三月八日,詔曰:



 國之
 大綱,惟刑與政。刑之不中,其政乃虧。劉光業、王德壽、王處貞、屈貞筠、鮑思恭、劉景陽等,庸流賤職,奸吏險夫,以粗暴為能官,以兇殘為奉法。往從按察,害虐在心,倏忽加刑,呼吸就戮,曝骨流血,其數甚多,冤濫之聲,盈於海內。朕唯布新澤,恩被人祇,撫事長懷,尤深惻隱。光業等五人積惡成釁,並謝生涯,雖其人已殂,而其跡可貶,所有官爵,並宜追奪。其枉被殺人,各令州縣以禮埋葬,還其官廕。劉景陽身今見在,情不可矜,特以會恩,免其嚴
 罰,宜從貶降,以雪冤情,可棣州樂單縣員外尉。



 自今內外法官,咸宜敬慎。其文深刺骨,跡徇凝脂,高下任情,輕重隨意,如酷吏丘神勣、來子珣、萬國俊、周興、來俊臣、魚承曄、王景昭、索元禮、傅游藝、王弘義、張知默、裴籍、焦仁亶、侯思止、郭霸、李仁敬、皇甫文備、陳嘉言等,其身已死,自垂拱已來,枉濫殺人,有官者並令削奪。唐奉一依前配流,李秦授、曹仁哲,並與嶺南惡處。



 開元十三年三月十二日,御史大夫程行諶奏:



 周朝酷吏來子珣、萬國俊、
 王弘義、侯思止、郭霸、焦仁亶、張知默、李敬仁、唐奉一、來俊臣、周興、丘神勣、索元禮、曹仁哲、王景昭、裴籍、李秦授、劉光業、王德壽、屈貞筠、鮑思恭、劉景陽、王處貞二十三人,殘害宗枝,毒陷良善,情狀尤重,子孫不許與官。陳嘉言、魚承曄、皇甫文備、傅游藝四人,情狀稍輕,子孫不許近任。」



 周興者,雍州長安人也。少以明習法律,為尚書省都事。累遷司刑少卿、秋官侍郎。自垂拱已來,屢受制獄,被其
 陷害者數千人。天授元年九月革命,除尚書左丞,上疏除李家宗正屬籍。二年十一月,與丘神勣同下獄。當誅,則天特免之,徙於嶺表。在道為仇人所殺。



 傅游藝,衛州汲人也。載初元年,為合宮主簿、左肅政臺御史,除左補闕。上書稱武氏符瑞,合革姓受命。則天甚悅,擢為給事中。數月,加同鳳閣鸞臺平章事。同月,又加朝散大夫,守鸞臺侍郎,依舊同平章事。其年九月革命,改天授元年,賜姓武氏。二年五月,加銀青光祿大夫。



 兄
 神童,為冬官尚書,兄弟並承榮寵。逾月,除司禮少卿,停知政事。夢登湛露殿,旦而陳於所親,為其所發,伏誅。時人號為四時仕宦,言一年自青而綠,及於硃紫也。希則天旨,誣族皇枝。神龍初,禁錮其子孫。



 初,游藝請則天發六道使,雖身死之後,竟從其謀,於是萬國俊輩恣斬戮矣。



 丘神勣,左衛大將軍行恭子也。永淳元年,為左金吾衛將軍。弘道元年,高宗崩,則天使於巴州害章懷太子,既
 而歸罪於神勣,左遷疊州刺史。尋復入為左金吾衛將軍,深見親委。受詔與周興、來俊臣鞫制獄,俱號為酷吏。垂拱四年,博州刺史、瑯邪王沖起兵,以神勣為清平道大總管。尋而沖為百姓孟青棒、吳希智所殺。神勣至州,官吏素服來迎,神勣揮刃盡殺之,破千餘家,因加左金吾衛大將軍。天授二年十月,下詔獄伏誅。



 索元禮,胡人也。光宅初,徐敬業起兵揚州,以匡復為名。則天震怒,又恐人心動搖,欲以威制天下。元禮探其旨,
 告事。召見,擢為游擊將軍,令於洛州牧院推案制獄。元禮性殘忍,推一人,廣令引數十百人,衣冠震懼,甚於狼虎。則天數召見賞賜,張其權勢,凡為殺戮者數千人。於是周興、來俊臣之徒,效之而起矣。時有諸州告密人,皆給公乘,州縣護送至闕下,於賓館以廩之。稍稱旨,必授以爵賞以誘之,貴以威於遠近。元禮尋以酷毒轉甚,則天收人望而殺之。天下之人謂之來、索,言酷毒之極,又首按制獄也。



 載初元年十月,左臺御史周矩上疏諫曰:



 頃者小人告訐,習以為常,內外諸司,人懷茍免。姑息臺吏,承接強梁,非故欲,規避誣構耳。又推劾之吏,皆以深刻為功,鑿空爭能,相矜以虐。泥耳籠頭,枷研楔轂,折脅簽爪,懸發熏耳,臥鄰穢溺,曾不聊生,號為「獄持」。或累日節食,連宵緩問,晝夜搖撼,使不得眠,號曰「宿囚」。此等既非木石,且救目前,茍求賒死。臣竊聽輿議,皆稱天下太平,何苦須反。豈被告者盡是英雄,以求帝王耶?只是不勝楚毒自誣耳。何以核之?陛下試取所告狀酌其虛實
 者,付令推,微訊動以探其情,所推者必上下其手,希聖旨也。願陛下察之。今滿朝側息不安,皆以為陛下朝與之密,夕與之仇,不可保也。聞有追攝,與妻子即為死訣。故為國者以仁為宗,以刑為助。周用仁而昌,秦用刑而亡,此之謂也。願陛下緩刑用仁,天下幸甚!



 則天從之,由是制獄稍息。



 侯思止,雍州醴泉人也。貧窮不能理生業,乃樂事渤海高元禮家。性無賴詭譎。時恆州刺史裴貞杖一判司。則
 天將不利王室,羅反之徒已興矣。判司教思止說游擊將軍高元禮,因請狀乃告舒王元名及裴貞反。周興按之,並族滅。授思止游擊將軍。元禮懼而曲媚,引與同坐,呼為侯大,曰:「國家用人以不次,若言侯大不識字,即奏云:『獬豸獸亦不識字,而能觸邪。』」則天果如其言,思止以獬豸對之,則天大悅。天授三年,乃拜朝散大夫、左臺侍御史。元禮復教曰:「在上知侯大無宅,倘以諸役官宅見借,可辭謝而不受。在上必問所由,即奏云:『諸反逆人,臣
 惡其名,不願坐其宅。』」則天復大悅,恩澤甚優。



 思止既按制獄,苛酷日甚。嘗按中丞魏元忠,曰:「急認白司馬,不然,即吃孟青。」白司馬者,洛陽有阪號白司馬阪。孟青者,將軍姓孟名青棒,即殺瑯邪王沖者也。思止閭巷庸奴,常以此謂諸囚也。



 元忠辭氣不屈,思止怒而倒曳元忠。元忠徐起曰:「我薄命,如乘惡驢墜,腳為鐙所掛,被拖曳。」思止大怒,又曳之曰:「汝拒捍制使,奏斬之。」元忠曰:「侯思止,汝今為國家御史,須識禮數輕重。如必須魏元忠頭,何
 不以鋸截將,無為抑我承反。奈何爾佩服硃紫,親銜天命,不行正直之事,乃言白司馬、孟青,是何言也!非魏元忠,無人抑教。」思止驚起悚怍,曰:「思止死罪,幸蒙中丞教。」引上床坐而問之。元忠徐就坐自若,思止言竟不正。時人效之,以為談謔之資。侍御史霍獻可笑之,思止以聞。則天怒,謂獻可曰:「我已用之,卿笑何也?」獻可具以其言奏,則天亦大笑。



 時來俊臣棄故妻,逼娶太原王慶詵女,思止亦奏請娶趙郡李自挹女,敕政事商量。鳳閣侍郎
 李昭德撫掌謂諸宰相曰:「大可笑。」諸宰相問故,昭德曰:「往年來俊臣賊劫王慶詵女,已大辱國。今日此奴又請索李自挹女,無乃復辱國乎!」竟為李昭德搒殺之。



 萬國俊,洛陽人。少譎異險詐。垂拱後,與來俊臣同為《羅織經》,屠覆宗枝朝貴,以作威勢。自司刑評事,俊臣同引為判官。



 天授二年,攝右臺監察御史,常與俊臣同按制獄。長壽二年,有上封事言嶺南流人有陰謀逆者,乃遣國俊就按之,若得反狀,便斬決。國俊至廣州,遍召流人,
 置於別所,矯制賜自盡,並號哭稱冤不服。國俊乃引出,擁之水曲,以次加戮,三百餘人,一時並命。然後鍛煉,曲成反狀,仍誣奏云:「諸流人咸有怨望,若不推究,為變不遙。」則天深然其奏,乃命右衛翊二府兵曹參軍劉光業、司刑評事王德壽、苑南面監丞鮑思恭、尚輦直長王大貞、右武衛兵曹參軍屈貞筠等,並攝監察御史,分往劍南、黔中、安南等六道鞫流人。尋擢授國俊朝散大夫、肅政臺侍御史。光業等見國俊盛行殘殺,得加榮貴,乃共
 肆其兇忍,唯恐後之。光業殺九百人,德壽殺七百人,其餘少者咸五百人。亦有遠年流人,非革命時犯罪,亦同殺之。則天後知其冤濫,下制:「被六道使所殺之家口未歸者,並遞還本管。」國俊等俄亦相次而死,皆見鬼物為祟,或有流竄而終。



 來子珣,雍州長安人。永昌元年四月,以上書陳事,除左臺監察御史。時朝士有不帶靴而朝者,子珣彈之曰:「臣聞束帶立於朝。」舉朝大噱。則天委之按制獄,多希旨,賜
 姓姓武氏,字家臣。天授中,丁父憂,起復朝散大夫、侍御史。時雅州剌史劉行實及弟渠州刺史行瑜、尚衣奉御行威並兄子鷹揚郎將軍虔通等,為子珣誣告謀反誅,又於盱眙毀其父左監門大將軍伯英棺柩。俄又轉為游擊將軍、右羽林中郎將。常衣錦半臂,言笑自若,朝士誚之。長壽元年,配流愛州卒。



 王弘義,冀州衡水人也。告變,授游擊將軍。天授中,拜右臺殿中侍御史。長壽中,拜左臺侍御史,與來俊臣羅告
 衣冠。延載元年,俊臣貶,弘義亦流放瓊州,妄稱敕追。時胡元禮為侍御史,使嶺南道,次於襄、鄧,會而按之。弘義詞窮,乃謂曰:「與公氣類。」元禮曰:「足下任御史,元禮任洛陽尉。元禮今為御史,公乃流囚,復何氣類?」乃搒殺之。



 弘義每暑月系囚,必於小房中積蒿而施氈褥,遭之者斯須氣絕矣。茍自誣引,則易於他房。與俊臣常行移牒,州縣懾懼,自矜曰:「我之文牒,有如狼毒野葛也。」弘義常於鄉里傍舍求瓜,主吝之,弘義乃狀言瓜園中有白兔,縣
 官命人捕逐,斯須園苗盡矣。內史李昭德曰:「昔聞蒼鷹獄吏,今見白免御史。」



 郭霸,廬江人也。天授二年,自宋州寧陵丞應革命舉,拜左臺監察御史。如意元年,除左臺殿中侍御史。長壽二年,右臺侍御史。初舉集,召見,於則天前自陳忠鯁云:「往年征徐敬業,臣願抽其筋,食其肉,飲其血,絕其髓。」則天悅,故拜焉,時人號為「四其御史」。



 時大夫魏元忠臥疾,諸御史盡往省之,霸獨居後。比見元忠,憂懼,請示元忠便
 液,以驗疾之輕重。元忠驚悚,霸悅曰:「大夫糞味甘,或不瘳。今味苦,當即愈矣。」元忠剛直,殊惡之,以其事露朝士。嘗推芳州刺史李思征,搒捶考禁,不勝而死。聖歷中,屢見思征,甚惡之。嘗因退朝遽歸,命家人曰:「速請僧轉經設齋。」須臾見思徵從數十騎上其廷,曰:「汝枉陷我,我今取汝。」霸周章惶怖,援刀自刳其腹,斯須蛆爛矣。是日,閭里亦見兵馬數十騎駐於門,少頃不復見矣。時洛陽橋壞,行李弊之,至是功畢。則天嘗問群臣:「比在外有何好
 事?」舍人張元一素滑稽,對曰:「百姓喜洛橋成,幸郭霸死,此即好事。」



 吉頊,洛州河南人也。身長七尺,陰毒敢言事。進士舉,累轉明堂尉。萬歲通天二年,有箕州刺史劉思禮,自云學於張憬藏,善相,云洛州錄事參軍綦連耀應圖讖,有「兩角騏麟兒」之符命。頊告之,則天付武懿宗與頊對訊。懿宗與頊誘思禮,令廣引朝士,必全其命。思禮乃引鳳閣侍郎李元素、夏官侍郎孫元通、天官侍郎劉奇、石抱忠、
 鳳閣舍人王處、來庭、主簿柳璆、給事中周潘、涇州刺史王勔、監察御史王助、司議郎路敬淳、司門員外郎劉慎之、右司員外郎宇文全志等三十六家,微有忤意者,必構之,楚毒百端,以成其獄。皆海內賢士名家,天下冤之,親故連累竄逐者千餘人。頊由是擢拜右肅政臺中丞,日見恩遇。



 明年,突厥寇陷趙、定等州。則天召頊檢校相州刺史,以斷賊南侵之路。頊以素不習武為辭,則天曰:「賊勢將退,藉卿威名鎮遏耳。」



 初,太原有術士溫彬茂,高
 宗時老,臨死,封一狀謂其妻曰:「吾死後,年名垂拱,即詣闕獻之,慎勿開也。」垂拱初,其妻獻之。狀中預陳則天革命及突厥至趙、定之事,故則天知賊至趙州而退。頊初至州募人,略無應者。俄而詔以皇太子為元帥,應募者不可勝數。及賊退,頊入朝奏之,則天甚悅。



 聖歷二年臘月,遷天官侍郎、同鳳閣鸞臺平章事。時易之、昌宗諷則天置控鶴監官員,則天以易之為控鶴監。頊素與易之兄弟親善,遂引頊,以殿中少監田歸道、鳳閣舍人薛稷、
 正諫大夫員半千、夏官侍郎李迥秀,俱為控鶴內供奉,時議甚不悅。



 初,則天以頊乾辯有口才,偉儀質,堪委以心腹,故擢任之。及與武懿宗爭趙州功於殿中,懿宗短小俯僂,頊聲氣凌厲,下視懿宗,嘗不相假。則天以為:「卑我諸武於我前,其可倚與!」其年十月,以弟作偽官,貶琰川尉,後改安固尉。尋卒。



 初,中宗未立為皇太子時,易之、昌宗嘗密問頊自安之策。頊云:「公兄弟承恩既深,非有大功於天下,則不全矣。今天下士庶,咸思李家,廬陵既
 在房州,相王又在幽閉,主上春秋既高,須有付托。武氏諸王,殊非屬意。明公若能從容請建立廬陵及相王,以副生人之望,豈止轉禍為福,必長享茅土之重矣!」易之然其言,遂承間奏請。則天知頊首謀,召而問之。頊曰:「廬陵王及相王,皆陛下之子,先帝顧托於陛下,當有主意,唯陛下裁之。」則天意乃定。頊既得罪,時無知者。睿宗即位,左右發明其事,乃下制曰:「故吏部侍郎、同中書門下平章事吉頊,體識宏遠,風規久大。嘗以經緯之才,允膺
 匡佐之委。時王命中否,人謀未輯,首陳返政之議,克副祈天之基。永懷遺烈,寧忘厥效。可贈左御史臺大夫。」



\end{pinyinscope}