\article{卷八十四}

\begin{pinyinscope}

 ○褚
 遂良韓瑗來濟上官儀



 褚遂良,散騎常侍亮之子也。太業末,隨父在隴右,薛舉僭號,署為通事舍人。舉敗歸國,授秦州都督府鎧曹參軍。貞觀十年,自秘書郎遷起居郎。遂良博涉文史,尤工
 隸書,父友歐陽詢甚重之。太宗嘗謂侍中魏徵曰:「虞世南死後,無人可以論書。」徵曰:「褚遂良下筆遒勁,甚得王逸少體。」太宗即日召令侍書。太宗嘗出御府金帛購求王羲之書跡,天下爭齎古書詣闕以獻,當時莫能辯其真偽,遂良備論所出,一無舛誤。十五年,詔有事太山,先幸洛陽,有星孛於太微,犯郎位。遂良言於太宗曰:「陛下撥亂反正,功超前烈,將告成東嶽,天下幸甚。而行至洛陽,彗星輒見,此或有所未允合者也。且漢武優柔數年,
 始行岱禮,臣愚伏願詳擇。」太宗深然之,下詔罷封禪之事。其年,遷諫議大夫,兼知起居事。太宗嘗問:「卿知起居,記錄何事,大抵人君得觀之否?」遂良對曰:「今之起居,古左右史,書人君言事,且記善惡,以為鑒誡,庶幾人主不為非法。不聞帝王躬自觀史。」太宗曰:「朕有不善,卿必記之耶?」遂良曰:「守道不如守官,臣職當載筆,君舉必記。」黃門侍郎劉洎曰:「設令遂良不記,天下亦記之矣。」太宗以為然。時魏王為太宗所愛,禮秩如嫡。其年,太宗問侍臣
 曰:「當今國家何事最急?」中書侍郎岑文本曰:「《傳》稱『導之以德,齊之以禮』,由斯而言。禮義為急。」遂良進曰:「當今四方仰德,誰敢為非?但太子、諸王,須有定分,陛下宜為萬代法以遺子孫。」太宗曰:「此言是也。朕年將五十,已覺衰怠。既以長子守器東宮,弟及庶子數將五十,心常憂慮,頗在此耳。但自古嫡庶無良佐,何嘗不傾敗國家?公等為朕搜訪賢德,以傅儲宮,爰及諸王,咸求正士。且事人歲久,即分義情深,非意窺窬,多由此作。」於是限王府官僚不
 得過四考。十七年,太宗問遂良曰:「舜造漆器,禹雕其俎,當時諫舜、禹者十餘人。食器之間,苦諫何也?」遂良對曰:「雕琢害農事,纂組傷女工。首創奢淫,危亡之漸。漆器不已,必金為之;金器不已,必玉為之。所以諍臣必諫其漸,及其滿盈,無所復諫。」太宗以為然,因曰:「夫為人君,不憂萬姓而事奢淫,危亡之機可反掌而待也。」時皇子年幼者多任都督、刺史,遂良上疏曰:「昔兩漢以郡國理人,除郡以外,分立諸子。割土分疆,雜用周制。皇唐州縣,祖依秦
 法。皇子幼年,或授刺史,陛下豈不以王之骨肉,鎮捍四方?此之造制,道高前烈。如臣愚見,有小未盡。何者?刺史郡帥,民仰以安。得一善人,部內蘇息;遇一不善,合州勞弊。是以人君愛恤百姓,常為擇賢。或稱河潤九里,京師蒙福;或人興歌詠,生為立祠。漢宣帝云:『與我共理者,惟良二千石。』如臣愚見,陛下兒子內年齒尚幼、未堪臨人者,且留京師,教以經學。一則畏天之威,不敢犯禁;二則觀見朝儀,自然成立。因此積習,自知為人。審堪臨州,然
 後遣出。臣謹按漢明、章、和三帝,能友愛於弟,自茲已降,取為準的。封立諸王,雖各有國土,年尚幼小者,召留京師,訓以禮法,垂以恩惠。訖三帝世,諸王數十百人,唯二王稍惡,自餘餐和染教,皆為善人。則前事已驗,惟陛下詳察。」太宗深納之。其年,太子承乾以罪廢,魏王泰入侍,太宗面許立為太子。因謂侍臣曰:「昨青雀自投我懷云:『臣今日始得與陛下為子,更生之日也。臣唯有一子,臣百年之後,當為陛下殺之,傳國晉王。』父子之道,故當天
 性,我見其如此,甚憐之。」遂良進曰:「陛下失言。伏願審思,無令錯誤也。安有陛下百年之後,魏王執權為天下之主,而能殺其愛子,傳國於晉王者乎?陛下昔立承乾為太子,而復寵愛魏王,禮數或有逾於承乾者,良由嫡庶不分,所以至此。殷鑒不遠,足為龜鏡。陛下今日既立魏王,伏願陛下別安置晉王,始得安全耳。」太宗涕泗交下曰:「我不能。」即日召長孫無忌、房玄齡、李勣與遂良等定策,立晉王為皇太子。時頻有飛雉集於宮殿之內,太宗
 問群臣曰:「是何祥也?」對曰:「昔秦文公時,有童子化為雉,雌者鳴於陳倉,雄者鳴於南陽。童子曰:得雄者王,得雌者霸。文公遂以為寶雞。後漢光武得雄,遂起南陽而有四海。陛下舊封秦王,故雄雉見於秦地,此所以彰表明德也。」太宗悅曰:「立身之道,不可無學,遂良博識,深可重也。」尋授太子賓客。



 時薛延陀遣使請婚,太宗許以女妻之,納其財聘,既而不與。遂良上疏曰:



 臣聞信為國本,百姓所歸,是以文王許枯骨而不違,仲尼寧去食而存信。
 延陀曩歲乃一俟斤耳,值神兵北指,蕩平沙塞,狼山、瀚海,萬里蕭條,陛下兵加諸外而恩起於內,以為餘寇奔波,須立酋長,璽書鼓纛,立為可汗。其懷恩光,仰天無極,而餘方戎狄,莫不聞知,以共沐和風,同餐恩信。頃者頻年遣使,請婚大國,陛下復降鴻私,許其姻媾。於是報吐蕃,告思摩,示中國,五尺童子人皆知之。於是御幸北門,受其獻食,於時百僚端笏,戎夷左衽,虔奉歡宴,皆承德音,口歌手舞,樂以終日。百官會畢,亦各有言,咸以為
 陛下欲得百姓安寧,不欲邊境交戰,遂不惜一女而妻可汗,預在含生,所以感德。今一朝生進退之意,有改悔之心,臣為國家惜茲聲聽。君子不失色於物,不失口於人。晉文公圍原,命三日糧,原不降,命去之。諜出曰:「原將降矣。」軍吏請待之,公曰:「信,國之寶也,民之庇也。得原失信,何以庇之?」陛下慮生意表,信在言前,今者臨事,忽然乖殊,所惜尤少,所失滋多。情既不通,方生嫌隙,一方所以相畏忌,邊境不得無風塵。西州、朔方,能無勞擾?彼胡以
 主被欺而心怨,此士以此無信而懷慚,不可以訓戎兵,不可以勵軍事。伏惟陛下以聖德神功,廓清四表。自君臨天下,十有七載,以仁恩而結庶類,以信義而撫戎夷,莫不欣然,負之無力。其見在之人,皆思報厚德;其所生胤嗣,亦望報陛下子孫。今者得一公主配之,以成陛下之信,有始有卒,其唯聖人乎!且又龍沙以北,部落無算,中國擊之,終不能盡。亦由可北敗,芮芮興,突厥亡,延陀盛。時以古人虛外實內,懷之以德,為惡在夷不在華,失
 信在彼不在此。伏惟陛下聖德無涯,威靈遠震,遂平高昌,破吐渾,立延陀,滅頡利。輕刑薄賦,庶事無壅,菽粟豐賤,祥符累臻。此則堯、舜、禹、湯不及陛下遠矣。伏願旁垂愷悌,廣茲含育,而常嗔絕域,有意遠籓,非偃伯興文之道,非止戈為武之義。臣以庸暗,忝居左右,敢獻瞽言,不勝戰懼。



 時太宗欲親征高麗,顧謂侍臣曰:「高麗莫離支賊殺其王,虐用其人。夫出師吊伐,當乘機便,今因其弒虐,誅之甚易。」遂良對曰:「陛下兵機神算,人莫能知。昔隋
 末亂離,手平寇亂。及北狄侵邊,西蕃失禮,陛下欲命將擊之,群臣莫不苦諫,陛下獨斷進討,卒並誅夷。海內之人,徼外之國,畏威懾伏,為此舉也。今陛下將興師遼東,臣意熒惑。何者?陛下神武,不比前代人君。兵既渡遼,指期克捷,萬一差跌,無以威示遠方,若再發忿兵,則安危難測。」太宗深然之。兵部尚書李勣曰:「近者延陀犯邊,陛下必欲追擊,此時陛下取魏徵之言,遂失機會。若如聖策,延陀無一人生還,可五十年間疆場無事。」帝曰:「誠如
 卿言,由魏徵誤計耳。朕不欲以一計不當而尤之,後有良算,安肯矢謀。」由是從勣之言,經畫渡遼之師。遂良以太宗銳意三韓,懼其遺悔,翌日上疏諫曰:



 臣聞有國家者譬諸身,兩京等於心腹,四境方乎手足,他方絕域,若在身外。臣近於坐下,伏奉口敕,布語臣下,雲自欲伐遼。臣數夜思量,不達其理。高麗王為陛下之所立,莫離支輒殺其主,陛下討逆收地,斯實乘機。關東賴陛下德澤,久無征戰,但命二、三勇將,發兵四、五萬,飛石輕梯,取如
 回掌。夫聖人有作,必履常規,貴能克平兇亂,駕馭才傑。惟陛下弘兩儀之道,扇三五之風,提厲人物,皆思效命。昔侯君集、李靖,所謂庸夫,猶能掃萬里之高昌,平千載之突厥,皆是陛下發蹤指示,聲歸聖明。臣旁求史籍,訖乎近代,為人之主,無自伐遼,人臣往征,則有之矣。漢朝則荀彘、楊僕,魏代則毋丘儉、王頎;司馬懿猶為人臣,慕容真僭號之子,皆為其主長驅高麗,虜其人民,削平城壘。陛下立功同於天地,美化包於古昔,自當超邁於百
 王,豈止俯同於六子?陛下昔翦平寇逆,大有爪牙,年齒未衰,猶堪任用,匪唯陛下之所使,亦何行而不克。方今太子新立,年實幼少,自餘籓屏,陛下所知。今一旦棄金湯之全,渡遼海之外,臣忽三思,煩愁並集。大魚依於巨海,神龍據於川泉,此謂人君不可輕而遠也。且以長遼之左,或遇霖淫,水潦騰波,平地數尺。夫帶方、玄菟,海途深渺,非萬乘所宜行踐。東京太原,謂之中地,東捴可以為聲勢,西指足以摧延陀,其於西京,逕路非遠,為其節
 度,以設軍謀,系莫離支頸,獻皇家之廟。此實處安全之上計,社稷之根本,特乞天慈,一垂省察。



 太宗不納。十八年,拜黃門侍郎,參綜朝政。高麗莫離支遣使貢白金,遂良言於太宗曰:「莫離支虐弒其主,九夷所不容,陛下以之興兵,將事吊伐,為遼山之人報主辱之恥。古者,討弒君之賊,不受其賂。昔宋督遺魯君以郜鼎,桓公受之於太廟,臧哀伯諫曰:『君人者昭德塞違,今滅德立違,而置其賂器於太廟,百官象之,其又何誅焉?武王克商,遷九
 鼎於洛邑,義士猶或非之,而況將昭違亂之賂器,置諸太廟,其若之何?』夫《春秋》之書,百王取法,若受不臣之筐篚,納弒逆之朝貢,不以為愆,何所致伐?臣謂莫離支所獻,自不得受。」太宗納焉,以其使屬吏。



 太宗既滅高昌,每歲調發千餘人防遏其地,遂良上疏曰:



 臣聞古者哲後,必先事華夏而後夷狄,務廣德化,不事遐荒。是以周宣薄伐,至境而止;始皇遠塞,中國分離。漢武負文、景之聚財,玩士馬之餘力,始通西域,初置校尉。軍旅連出,將三
 十年。復得天馬於宛城,採蒲萄於安息。而海內虛竭,生人失所,租及六畜,算至舟車,因之兇年,盜賊並起,搜粟都尉桑弘羊復希主意,遣士卒遠田輪臺,築城以威西域。帝翻然追悔,情發於中,棄輪臺之野,下哀痛之詔,人神感悅,海內乃康。向使武帝復用弘羊之言,天下生靈皆盡之矣。是以光武中興,不逾蔥嶺,孝章即位,都護來歸。



 陛下誅滅高昌,威加西域,收其鯨鯢,以為州縣。然則王師初發之歲,河西供役之年,飛芻挽粟,十室九空,
 數郡蕭然,五年不復。陛下歲遣千餘人遠事屯戍,終年離別,萬里思歸。去者資裝,自須營辦,既賣菽粟,傾其機杼。經途死亡,復在其外,兼遣罪人,增其防遏。彼罪人者,生於販肆,終朝惰業,犯禁違公。止能擾於邊城,實無益於行陣。所遣之內,復有逃亡,官司捕捉,為國生事。高昌途路,沙磧千里,冬風冰冽,夏風如焚。行人去來,遇之多死。《易》云:「安不忘危,理不忘亂。」設令張掖塵飛,酒泉烽舉,陛下豈能得高昌一人菽粟而及事乎?終須發隴右諸州,
 星馳電擊。由斯而言,此河西者方於心腹,彼高昌者他人手足,豈得糜費中華,以事無用?《書》曰:「不作無益害有益。」其此之謂乎!



 陛下道映先天,威行無外,平頡利於沙塞,滅吐渾於西海。突厥餘落,為立可汗;吐渾遺氓,更樹君長。復立高昌,非無前例,此所謂有罪而誅之,既伏而立之。四海百蠻,誰不聞見,蠕動懷生,畏威慕德。宜擇高昌可立者立之,徵給首領,遣還本國,負戴洪恩,長為籓翰。中國不擾,既富且寧,傳之子孫,以貽永世。



 二十年,太
 宗於寢殿側別置一院,令太子居,絕不令往東宮。遂良復上疏諫曰:



 臣聞周世問安,三至必退,漢儲視膳,五日乃來。前賢作法,規模弘遠。禮曰:「男子十年出就外傅,出宿於外,學書計也。然則古之達者,豈無慈心?減茲私愛,欲使成立。凡人尚猶如此,況君之世子乎?自當春誦夏弦,親近師傅,體人間之庶事,適君臣之大道,使翹足延首,皆聆善聲。若獻歲之有陽春,玄天之有日月,弘此懿德,乃作元良。伏惟陛下道育三才,功包九有,親樹太子,
 莫不欣欣。既云廢昏立明,須稱天下瞻望,而教成之道,實深乖闕。不離膝下,常居宮內,保傅之說無暢,經籍之談蔑如。且朋友不可以深交,深交必有怨;父子不可以滯愛,滯愛或生愆。伏願遠覽殷、周,近遵漢、魏,不可頓革,事須階漸。嘗計旬日,半遣還宮,專學藝以潤身,布芳聲於天下,則微臣雖死,猶曰生年。



 太宗從之。



 遂良前後諫奏及陳便宜書數十上,多見採納,其年,加銀青光祿大夫。二十一年,以本官檢校大理卿,尋丁父憂解。明年,起
 復舊職,俄拜中書令。



 二十三年,太宗寢疾,召遂良及長孫無忌入臥內,謂之曰:「卿等忠烈,簡在朕心。昔漢武寄霍光,劉備托葛亮,朕之後事,一以委卿。太子仁孝,卿之所悉,必須盡誠輔佐,永保宗社。」又顧謂太子曰:「無忌、遂良在,國家之事,汝無憂矣。」仍命遂良草詔。高宗即位,賜爵河南縣公。永徽元年,進封郡公。尋坐事出為同州刺史。三年,徵拜吏部尚書、同中書門下三品,監修國史,加光祿大夫。其月,又兼太子賓客。四年,代張行成為尚書
 右僕射,依舊知政事。



 六年,高宗將廢皇后王氏,立昭儀武氏為皇后,召太尉長孫無忌、司空李勣、尚書左僕射於志寧及遂良以籌其事。將入,遂良謂無忌等曰:「上意欲廢中宮,必議其事,遂良今欲陳諫,眾意如何?」無忌曰:「明公必須極言,無忌請繼焉。」及入,高宗難於發言,再三顧謂無忌曰:「莫大之罪,絕嗣為甚。皇后無胤息,昭儀有子,今欲立為皇后,公等以為何如?」遂良曰:「皇后出自名家,先朝所娶,伏事先帝,無愆婦德。先帝不豫,執陛下手
 以語臣曰:『我好兒好婦,今將付卿。』陛下親承德音,言猶在耳。皇后自此未聞有愆,恐不可廢。臣今不敢曲從,上違先帝之命,特願再三思審。愚臣上忤聖顏,罪合萬死,但願不負先朝厚恩,何顧性命?」遂良致笏於殿陛,曰:「還陛下此笏。」仍解巾叩頭流血。帝大怒,令引出。長孫無忌曰:「遂良受先朝顧命,有罪不加刑。」翌日,帝謂李勣曰:「冊立武昭儀之事,遂良固執不從。遂良既是受顧命大臣,事若不可,當且止也。」勣對曰:「此乃陛下家事,不合問外
 人。」帝乃立昭儀為皇后,左遷遂良潭州都督。顯慶二年,轉桂州都督。未幾,又貶為愛州刺史。明年,卒官,年六十三。



 遂良卒後二歲餘,許敬宗、李義府奏言長孫無忌所構逆謀,並遂良扇動,乃追削官爵,子孫配流愛州。弘道元年二月,高宗遺詔放還本郡。神龍元年,則天遺制復遂良及韓瑗爵位。



 韓瑗,雍州三原人也。祖紹,隋太僕少卿。父仲良,武德初為大理少卿,受詔與郎楚之等掌定律令。仲良言於高
 祖曰:「周代之律,其屬三千,秦法已來,約為五百。若遠依周制,繁紊更多。且官吏至公,自當奉法,茍若徇己,豈顧刑名?請崇寬簡,以允惟新之望。」高祖然之。於是採定《開皇律》行之,時以為便。貞觀中,位至刑部尚書、秦州都督府長史、潁川縣公。瑗少有節操,博學有吏才。貞觀中,累至兵部侍郎,襲父潁川公。永徽三年,拜黃門侍郎。四年,與中書侍郎來濟皆同中書門下三品,監修國史。五年,加銀青光祿大夫。六年,遷侍中,其年兼太子賓客。時高
 宗欲廢王皇后,瑗涕泣諫曰:「皇后是陛下在籓府時先帝所娶,今無愆過,欲行廢黜,四海之士,誰不惕然?且國家屢有廢立,非長久之術。願陛下為社稷大計,無以臣愚,不垂採察。」帝不納。明日,瑗又諫,悲泣不能自勝。帝大怒,促令引出。尋而尚書左僕射褚遂良以忤旨左授潭州都督,瑗復上疏理之曰:



 古之聖王,立諫鼓,設謗木,冀欲聞逆耳之言,甘苦口之議,發揚大化,裨益洪猷,垂令譽於將來,播休聲於不朽者也。伏見詔書以褚遂良為
 潭州都督,臣夙夜思之,用增感激。臣識慚知遠,業謝通經,載撫愚情,誠為未可。遂良運偶升平,道昭前烈,束發從宦,方淹累稔。趨侍陛下,俄歷歲年,不聞涓滴之愆,常睹勤勞之效。竭忠誠於早歲,罄直道於茲年。體國忘家,捐身徇物,風霜其操,鐵石其心。誠可重於皇明,詎專方於曩昔?且先帝納之於帷幄,寄之以心膂,德逾水石,義冠舟車,公家之利,言無不可。及纏悲四海,遏密八音,竭忠國家,親承顧托,一德無二,千古懍然。此不待臣言,陛
 下備知之矣。臣嘗有此心,未敢聞奏。且萬姓失業,旰食忘勞;一物不安,納隍軫慮,在於微細,寧得過差。況社稷之舊臣,陛下之賢佐,無聞罪狀,斥去朝廷,內外氓黎,咸嗟舉措。觀其近日言事,披誠懇切,詎肯後陛下之德,異於堯、舜;懼陛下之過,塵於史冊。而乃深遭厚謗,重負醜言,可以痛志士之心,損陛下之明也。臣聞晉武弘裕,不貽劉毅之誅;漢祖深仁,無恚周昌之直。而遂良被遷,已經寒暑,違忤陛下,其罰塞焉。伏願糸面鑒無辜,稍寬非罪,
 俯矜微款,以順人情。



 疏奏,帝謂瑗曰:「遂良之情,朕亦知之矣。然其悖戾犯上,以此責之,朕豈有過,卿言何若是之深也!」瑗對曰:「遂良可謂社稷忠臣,臣恐以諛佞之輩,蒼蠅點白,損陷忠貞。昔微子去之而殷國以亡,張華不死而綱紀不亂,國之欲謝,善人其衰。今陛下富有四海,八紘清泰,忽驅逐舊臣,而不垂省察乎!伏願違彼覆車,以收往過,垂勸誡於事君,則群生幸甚。」帝竟不納。瑗以言不見用,憂憤上表,請歸田里,詔不許。顯慶二年,許敬
 宗、李義府希皇后之旨,誣奏瑗與褚遂良潛謀不軌,以桂州用武之地,故授遂良桂州刺史,實以為外援。於是更貶遂良為愛州刺史,左授瑗振州刺史。四年,卒官,年五十四。明年,長孫無忌死,敬宗等又奏瑗與無忌通謀,遣使殺之。及使至,瑗已死,更發棺驗尸而還,籍沒其家,孫配徙嶺表。神龍元年,則天遺制令復其官爵。



 來濟,揚州江都人,隋左翊衛大將軍榮國公護子也。宇文化及之難,闔門遇害。濟幼逢家難,流離艱險,而篤志
 好學,有文詞,善談論,尤曉時務。舉進士,貞觀中累轉通事舍人。太子承乾之敗,太宗謂侍臣曰:「欲何以處承乾?」群臣莫敢對,濟進曰:「陛下上不失作慈父,下得盡天年,即為善矣。」帝納其言。俄除考功員外郎。十八年,初置太子司議郎,妙選人望,遂以濟為之,仍兼崇賢館直學士。尋遷中書舍人,與令狐德棻等撰《晉書》。永徽二年,拜中書侍郎,兼弘文館學士,監修國史。四年,同中書門下三品。五年,加銀青光祿大夫,以修國史功封南陽縣男,賜
 物七百段。六年,遷中書令、檢校吏部尚書。時高宗欲立昭儀武氏為宸妃,濟密表諫曰:「宸妃古無此號,事將不可。」武皇后既立,濟等懼不自安;後乃抗表稱濟忠公,請加賞慰,而心實惡之。顯慶元年,兼太子賓客,進爵為侯,中書令如故。二年,又兼太子詹事。尋而許敬宗等奏濟與褚遂良朋黨構扇,左授臺州刺史。五年,徙庭州刺史。龍朔二年,突厥入寇,濟總兵拒之,謂其眾曰:「吾嘗掛刑網,蒙赦性命,當以身塞責,特報國恩。」遂不釋甲胄赴賊,
 沒於陣。時年五十三,贈楚州刺史,給靈輿遞還鄉。有文集三十卷,行於代。



 濟兄亙,有學行,與濟齊名。上元中,官至黃門侍郎、同中書門下三品。



 上官儀,本陜州陜人也。父弘,隋江都宮副監,因家於江都。大業末,弘為將軍陳稜所殺,儀時幼,藏匿獲免。因私度為沙門,游情釋典,尤精《三論》,兼涉獵經史,善屬文。貞觀初,楊仁恭為都督,深禮待之。舉進士。太宗聞其名,召授弘文館直學士。累遷秘書郎。時太宗雅好屬文,每遣
 儀視草,又多令繼和,凡有宴集,儀嘗預焉。俄又預撰《晉書》成,轉起居郎,加級賜帛。高宗嗣位,遷秘書少監。龍朔二年,加銀青光祿大夫、西臺侍郎、同東西臺三品,兼弘文館學士如故。本以詞彩自達,工於五言詩,好以綺錯婉媚為本。儀既貴顯,故當時多有效其體者,時人謂為上官體。儀頗恃才任勢,故為當代所嫉。麟德元年,宦者王伏勝與梁王忠抵罪,許敬宗乃構儀與忠通謀,遂下獄而死,家口籍沒。子庭芝,歷位周王府屬。與儀俱被殺。
 庭芝有女,中宗時為昭容,每侍帝草制誥,以故追贈儀為中書令、秦州都督、楚國公;庭芝黃門侍郎、岐州刺史、天水郡公,仍令以禮改葬。



 史臣曰:褚河南上書言事,亹癖有經世遠略。魏徵、王珪之後,骨鯁風彩,落落負王佐器者,殆難其人。名臣事業,河南有焉。昔齊人饋樂而仲尼去,戎王溺妓而由餘奔,婦人之言,聖哲懼罹其禍,況二佞據衡軸之地,為正人之魑魅乎!古之志士仁人,一言相期,死不之悔,況於君
 臣之間,受托孤之寄,而以利害禍福,忘平生之言哉!而韓、來諸公,可謂守死善道,求福不回者焉。



 贊曰:褚公之言,和樂愔愔,鐘石在虡,動成雅音。二猘雙吠,三賢一心。人皆觀望,我不浮沉。



\end{pinyinscope}