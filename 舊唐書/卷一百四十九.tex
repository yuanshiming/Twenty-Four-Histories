\article{卷一百四十九}

\begin{pinyinscope}

 ○劉玄佐子士寧士幹李萬榮附董晉陸長源劉全諒李忠臣李希烈吳少誠弟少陽少陽子元濟附



 劉玄佐,本名洽,滑州匡城人也。少倜儻,不理生業;為縣
 捕盜吏,違法,為令所笞,僅死,乃亡命從軍。大歷中,為永平軍衙將。李靈曜據汴州,洽將兵乘其無備,徑入宋州,遂詔以州隸永平軍,節度使李勉奏署宋州刺史。建中二年,加兼御史中丞、亳潁節度等使。



 李正己死,子納匿喪謀叛,而李洧以徐州歸順,納遣兵圍之。詔洽與諸軍援洧,與賊接戰,大破之,斬首萬餘級。由是轉輸路通,加御史大夫。又收濮州,降其將楊令暉,分兵挾之,徇濮陽,降其將高彥昭,以通濮陽津。遷尚書,累封四百戶,兼曹濮
 觀察使,尋加淄青兗鄆招討使,又加汴滑都統副使。李希烈攻汴州,德宗在奉天,連戰,賊稍卻。興元初,進加檢校左僕射,加平章事。希烈圍寧陵,洽大將劉昌言堅守不下。希烈攻陳州,洽遣昌言與諸軍救之,大敗賊黨,獲其將翟崇暉。希烈棄汴州,洽率軍收汴,詔加汴宋節度。無幾,授本管及陳州諸軍行營都統,賜名玄佐。是歲來朝,又拜涇原四鎮北庭等道兵馬副元帥,檢校司空,益封八百戶。



 玄佐性豪侈,輕財重義,厚賞軍士,故百姓益
 困。是以汴之卒,始於李忠臣,訖於玄佐,而日益驕恣,多逐殺將帥,以利剽劫。又寵任小吏張士南及養子樂士朝,財物鉅萬。士朝通玄佐嬖妾。玄佐在鎮,李納每使來,必重贈遺,飾美女名樂,從其游娛,故多得其陰事,常先為備,故納憚其心計。貞元三年三月,薨於位,年五十八,廢朝三日,贈太傅。將佐初匿喪,稱疾俟代,帝亦為隱,數日乃發喪。子士寧、士干。



 初,將佐匿喪,既發,帝遣問所欲立:「吳湊可乎?」監軍孟介、行軍盧瑗皆曰「便」。及湊次汜水,
 柩將遷,請備儀;瑗不許,又令留什物俟新使,將士大怒。玄佐子婿及親兵乃以三月晦夜激怒三軍。明晨,衙兵皆甲胄,擁士寧登重榻,衣以墨縗,呼為留後。軍士執城將曹金岸、浚儀令李邁,曰:「爾等皆請吳湊者!」遂臠之,唯盧瑗獲免。士寧乃以財物分賜將士,請之為帥,孟介以聞。帝召宰臣問計,竇參曰:「今汴人挾李納以邀命,若不許,懼合於納。」遂從之,授士寧起復金吾衛將軍同正、汴州刺史、宣武軍節度等使。士寧位未定時,遣使通王武
 俊、劉濟、田緒,以士寧未受詔於國,皆留之。



 士寧初授節制,諸將多不悅服。性忍暴淫亂,或彎弓挺刃,手殺人於杯案間,悉烝父之妓妾,又強取人之婦女,好惈觀婦人。每出畋獵,數日方還,軍府苦之。其大將李萬榮與其父玄佐同里閈,少相善,寬厚得眾心;士寧疑之,去其兵權,令攝汴州事。萬榮深怨之,將伺其隙逐之。十年正月,士寧以眾二萬畋於城南,兵既出,萬榮晨入士寧廨舍,召其所留心腹兵千餘人,矯謂之曰:「有詔征大夫入朝,俾
 吾掌留務,汝輩人賜錢三千貫,無他憂也。」兵士皆拜。萬榮既約親兵於內,又召各營兵於外,以是言令之,軍士皆聽命。萬榮乃分兵閉城門,馳使白士寧曰:「詔征大夫,宜速即路;若遷延不行,當傳首以獻。」士寧知眾不為用,計無所出,乃將五百騎走歸京師。比至中牟,亡走大半;至東都,所餘僮隸婢妾數十人而已。既至京師,詔令歸第服喪,禁絕出入。萬榮乃斬士寧所親之將辛液、白英賢以令於軍,凡賞軍士錢二十萬貫,詔令籍沒士寧家
 財以分賞焉。遂授萬榮宣武軍兵馬留後。



 初,萬榮遣兵三千備秋於京西,有親兵三百,前為劉士寧所驕者,日益橫。萬榮惡之,悉置行籍中,由是深怨萬榮。大將韓惟清、張彥琳請將往,不許;萬榮令其子乃將之,未發。惟清、彥琳不得志,因親兵銜怨,乃作亂,共攻萬榮。萬榮分兵擊之,叛卒兵械少,戰不勝,乃劫轉運財貨及居人而潰,殺傷千餘人。叛兵四出,多投宋州,刺史劉逸準厚撫之。韓惟清走鄭州,張彥琳走東都,以束身歸罪,宥以不死,
 並流竄焉。萬榮悉捕逃叛將卒妻孥數千人,皆誅之。萬榮誅叛卒之後,人心恟々不安,軍卒數人呼於市曰:「今夜大兵四面至,城當破。」眾驚駭。萬榮悉捕得,或云士寧所教,萬榮斬之以聞;遂以士寧廢處郴州。十一年五月,授萬榮宣武軍節度使。其年八月,萬榮病,遂署其子乃為司馬。乃勒大將李湛、伊婁涚、張伾往外鎮,尋皆令殺之。涚、伾皆已死,惟李湛至尉氏,尉氏鎮將郝忠節不肯殺湛。是夜軍士逐出李乃,遂執送京師。萬榮以其日病
 卒。乃至京師,付京兆府杖殺。



 劉士干,玄佐養子,前為太府少卿。有樂士朝者,亦為玄佐養子,因冒劉姓,與士干有隙。及玄佐卒,或云為士朝所鴆。士乾知之,及至京師,遣奴持刀於喪位,語士朝曰:「有吊客至。」因誘殺之。賜士干死。



 董晉,字混成,河中盧鄉人。明經及第。至德初,肅宗自靈武幸彭原,晉上書謁見,授校書郎、翰林待制,再轉衛尉丞,出為汾州司馬。未幾,刺史崔圓改淮南節度,奏晉以
 本官攝殿中侍御史,充判官,尋歸臺,授本官,遷侍御史、主客員外郎、祠部郎中。大歷中,兵部侍郎李涵送崇徽公主使回紇,奏晉為判官。使還,拜司勛郎中。歷秘書太府太常少卿監、左金吾將軍。旬日,德宗嗣位,改太常卿,遷右散騎常侍,兼御史中丞知臺事。以清勤謹慎,故驟遷右職。尋為華州刺史、兼御史中丞、潼關防禦使。久之,加兼御史大夫。硃泚僭逆於京師,使兇黨仇敬、何望之侵逼華州,晉奔遁赴行在,授國子祭酒,尋令往恆州宣
 慰。從車駕還京師,遷左金吾衛大將軍,改尚書左丞。時右丞元琇領度支使,為韓滉所擠貶黜,晉嫉之,見宰相極言非罪,舉朝稱之。復拜太常卿。



 五年,遷門下侍郎、同平章事。時政事決在竇參,晉但奉詔書,領然諾而已。金吾衛將軍沈房有弟喪,公除,衣慘服入閣。上問宰相,對曰:「準式,朝官有周年已下喪者,諸騑縵,不合衣淺色。」帝曰:「南班安得有之?」對曰:「因循而然。」又問晉冠冕之制,對曰:「古人服冠冕者,動有佩玉之響,所以節步也。《禮》云『
 堂上接武,堂下布武』,至恭也;步武有常,君前之禮,進趨而已。今或奔走以致顛僕,非恭慎也。在式,朝官皆是綾袍袱,五品已上金玉帶,取其文彩畫飭,以奉上也。是以禹惡衣食而致美乎黻冕,君親一致。昔尚書郎含香,老萊彩服,皆此義也。服絁縵,非制也。」上深然之,遂詔曰:「常參官入閣,不得趨走;周期已下喪者,禁慘服朝會。」又令服本品綾袍金玉帶。晉明於禮學如此。



 竇參驕滿既甚,帝漸惡之。八年,參諷晉奏其侄給事中竇申為吏部侍
 郎,帝正色曰:「豈不是竇參遣卿奏也?」晉不敢隱。因問參過失,晉具奏之。旬日,參貶官,晉憂懼,累上表辭位。九年夏,改禮部尚書、兵部尚書、東都留守、東都畿汝州都防禦使。



 會汴州節度李萬榮疾甚,其子乃為亂,以晉為檢校左僕射、同平章事,兼汴州刺史、宣武軍節度營田、汴宋觀察使。晉既受命,唯將幕官傔從等十數人,都不召集兵馬。既至鄭州,宣武軍迎候將吏無至者。晉左右及鄭州官吏皆懼,共勸晉云:「鄧惟恭承萬榮疾病之甚,遂
 總領軍州事。今相公到此,尚不使人迎候,其情狀豈可料;即恐須且遲回,以候事勢。」晉曰:「奉詔為汴州節度使,即合準敕赴官,何可妄為逗留!」人皆憂其不測,晉獨恬然。未至汴州十數里,鄧惟恭方來迎候,晉俾其不下馬;既入,乃委惟恭以軍政,眾服晉明於事體機變,而未測其深淺。



 初,萬榮逐劉士寧,代為節度使,委兵於惟恭,以其同鄉里。及疾甚,李乃將為亂,惟恭乃與監軍同謀縛乃,送歸朝廷。惟恭自以當便代居其位,故不遣候吏,以疑
 懼晉心,冀其不敢進。不意晉之速至。晉已近,方遽出迎之。然心常怏怏,竟以驕盈慢法,潛圖不軌,配流嶺南。



 朝廷恐晉柔懦,尋以汝州刺史陸長源為晉行軍司馬。晉謙恭簡儉,每事因循多可,故亂兵粗安。長源好更張云為,數請改易舊事,務從削刻。晉初皆然之,及案牘已成,晉乃命且罷。又委錢穀支計於判官孟叔度。叔度輕佻,好慢易軍人,皆惡之。晉十五年二月卒,年七十六,廢朝三日,贈太傅,賜布帛有差。卒後未十日,汴州大亂,殺長
 源、叔度等。



 陸長源,字泳之,開元、天寶中尚書左丞、太子詹事餘慶之孫,西河太守璪之子。長源淑書史。乾元中,陷河北諸賊,因佐昭義軍節度薛嵩卒後。久之,歷建、信二州刺史。浙西節度韓滉兼領江、淮轉運,奏長源檢校郎中、兼中丞,充轉運副使。罷為都官郎中,改萬年縣令,出為汝州刺史。



 貞元十二年,授檢校禮部尚書、宣武軍行軍司馬,汴州政事,皆決斷之。性輕佻,言論容易,恃才傲物,所在
 人畏而惡之。及至汴州,欲以峻法繩驕兵;而董晉判官楊凝、孟叔度亦縱恣淫湎,眾情共怒。晉性寬緩,事務因循,以收士心。長源每事守法,晉或茍且,長源輒執而正之。



 及晉卒,令長源知留後事。長源揚言曰:「將士多弛慢,不守憲章,當以法繩之。」由是人人恐懼。加以叔度苛刻,多縱聲色,數至樂營與諸婦人嬉戲,自稱孟郎,眾皆薄之。舊例,使長薨,放散布帛於三軍制服。至是,人請服,長源初固不允,軍人求之不已,長源等議給其布直。叔度
 高其鹽價而賤為布直,每人不過得鹽三二斤,軍情大變。或勸長源,故事有大變,皆賞三軍,三軍乃安。長源曰:「不可使我同河北賊,以錢買健兒取旌節。」兵士怨怒滋甚,乃執長源及叔度等臠而食之,斯須骨肉糜散。長源死之日,詔下以為節度使,及聞其死,中外惜之,贈尚書右僕射。



 劉全諒,懷州武涉人也。父客奴,由征行家於幽州之昌平。少有武藝,從平盧軍。開元中,有室韋首領段普恪,恃
 驍勇,數苦邊。節度使薛楚玉以客奴有膽氣,令抗普恪。客奴單騎襲之,斬首以獻,自白身授左驍衛將軍,充游奕使,自是數有戰功。性忠謹,為軍人所信。天寶末,安祿山反,詔以安西節度封常清為範陽節度,以平盧節度副使呂知誨為平盧節度,以太原尹王承業為河東節度。祿山既僭位於東都,遣腹心韓朝陽等招誘知誨;知誨遂受逆命,誘殺安東副都護、保定軍使馬靈詧,祿山遂署知誨為平盧節度使。客奴與平盧諸將同議,取知
 誨殺之;仍遣與安東將王玄志遙相應援,馳以奏聞。十五載四月,授客奴柳城郡太守、攝御史大夫、平盧節度支度營田陸運、押兩蕃、渤海黑水四府經略及平盧軍使,仍賜名正臣。又以王玄志為安東副大都護、攝御史中丞、保定軍及營田使。正臣仍領兵平盧來襲範陽,未至,為逆賊將史思明等大敗之。正臣奔歸,為王玄志所鴆而卒。逆賊署徐歸道平盧節度,王玄志與平盧將侯希逸等又襲殺歸道。大歷九年,追贈正臣工部尚書。



 全
 諒本名逸準,以父勛授別駕、長史。建中初,劉玄佐為宋亳節度使,召署為牙將,以勇果騎射聞。玄佐以宗侄厚遇之,累署都知兵馬使,試太僕卿、兼御史中丞。玄佐卒,子士寧代為節度使,疑宋州刺史翟良佐不附己,陽言出巡,至宋州,遽以逸準代良佐為刺史。及董晉卒,兵亂,殺陸長源,監軍俱文珍與大將密召逸準赴汴州,令知留後。朝廷因授以檢校工部尚書、汴州刺史,兼宣武軍節度觀察等使,仍賜名全諒。貞元十五年二月卒,年四
 十九,廢朝一日,贈右僕射。



 李忠臣,本姓董,名秦,平盧人也,世家於幽州薊縣。自云曾祖文昱,棣州刺史;祖玄獎,安東都護府錄事參軍;父神嶠,河內府折沖。忠臣少從軍,在卒伍之中,材力冠異。事幽州節度薛楚玉、張守珪、安祿山等,頻委征討,積勞至折沖郎將、將軍同正、平盧軍先鋒使。



 及祿山反,與其倫輩密議,殺偽節度呂知誨,立劉正臣為節度,以忠臣為兵馬使。攻長楊,戰獨山,襲榆關、北平,殺賊將申子貢、
 榮先欽,擒周釗送京師,忠臣功多。又從正臣破漁陽,逆將李歸仁、李咸、白秀芝等來拒戰,約數十合,並摧破之;無何,潼關失守,郭子儀、李光弼退師,忠臣乃引軍北歸。奚王阿篤孤初以眾與正臣合,後詐言請以萬餘騎同收範陽,至後城南,中夜反攻,忠臣與戰,遂至溫泉山,破之;擒大首領阿布離,斬以祭纛釁鼓。正臣卒,又與眾議以安東都護王玄志為節度使。



 至德二載正月,玄志令忠臣以步卒三千自雍奴為葦筏過海。賊將石帝庭、烏
 承洽來拒;忠臣與董竭忠退之,轉戰累日,遂收魯城、河間、景城等,大獲資糧,以赴本軍。復與大將田神功率兵討平原、樂安郡,下之;擒偽刺史臧瑜等。防河招討使李銑承制以忠臣為德州刺史。屬史思明歸順,河南節度張鎬令忠臣以兵赴鄆州,與諸軍使收河南州縣。又與裨將陽惠元大破賊將王福德於舒舍口,肅宗累下詔慰諭,仍令鎮濮州,尋移韋城。



 乾元元年九月,改光祿卿同正。其年,與郭子儀等九節度圍安慶緒於相州。明年
 二月,諸軍潰歸,忠臣亦退。至滎陽,賊將敬



 釭來襲官船,忠臣大破之,獲米二百餘艘,以資汴州軍士。尋拜濮州刺史、緣河守捉使,移鎮杏園渡。及史思明陷汴州,節度使許叔冀與忠臣並力屈降賊。思明撫忠臣背曰:「吾比只有左手,今得公,兼有右手矣!」與俱寇河陽。數日,忠臣夜以五百人斫其營,突圍歸。李光弼以聞,詔加開府儀同三司、殿中監同正,賜實封二百戶。召至京師,賜姓李氏,名忠臣,封隴西郡公,賜良馬、莊宅、銀器、彩物等。



 時陜
 西、神策兩節度郭英乂、衛伯玉鎮陜州,以忠臣為兩軍節度兵馬使。魚朝恩亦在陜,俾忠臣與賊將李歸仁、李感義等戰於永寧、莎柵;前後數十陣,皆摧破之。會淮西節度王仲升為賊所擒,寶應元年七月,拜忠臣太常卿同正、兼御史中丞、淮西十一州節度。尋加安州刺史,仍鎮蔡州。其年,令忠臣會元帥諸軍收復東都。二年六月,就加御史大夫。時回紇可汗既歸其國,留判官安恪、石帝庭於河陽守禦財物,因此招聚亡命為寇,道路壅隔,詔
 忠臣討平之。



 永泰元年,吐蕃犯西陲,京師戒嚴。代宗命中使追兵,諸道多不時赴難。使至淮西,忠臣方會鞠,即令整師飾駕。監軍大將固請曰:「軍行須擇吉日。」忠臣奮臂於眾曰:「焉有父母遇寇難,待揀好日,方救患乎!」即日進發。自此方隅有警,忠臣必先期而至。由是代宗嘉其忠節,加本道觀察使,寵賜頗厚。及同華節度周智光舉兵反,詔忠臣與神策將李太清等討平之。大歷三年,加檢校工部尚書,實封通前三百戶。五年,加蔡州刺史。七
 年,檢校右僕射、知省事。李靈曜之叛,田承嗣使侄悅援之,忠臣與諸軍大破悅等,汴州平。十一年十二月,加檢校司空平章事、汴州刺史。



 忠臣性貪殘好色,將吏妻女多被誘脅以通之。又軍無紀綱,所至縱暴,人不堪命。而以妹婿張惠光為衙將,恃勢兇虐,軍中苦之;數有言於忠臣,不之信也。俄以惠光為節度副使,令惠光子為衙將,陵橫甚於其父。忠臣所信任大將李希烈,素善騎射,群情所伏,因眾心之怒,以十四年三月,與少將丁皓、賈
 子華、監軍判官蔣知璋等舉兵斬惠光父子,以脅逐忠臣。單騎赴京師,朝廷方寵武臣,不之責也,依前檢校司空、平章事,留京師奉朝請。



 建中初,嘗因奏對,德宗謂之曰:「卿耳甚大,真貴人也。」忠臣對曰:「臣聞驢耳甚大,龍耳甚小;臣耳雖大,乃驢耳也。」上說之。時常侍張涉承恩用事,坐受財賄事露,帝將以法繩之——涉,即帝在春宮時侍講也。忠臣奏曰:「陛下貴為天子,而先生以乏財抵法,以愚臣觀之,非先生之過也。」帝意解,但令歸田里。前湖南
 觀察辛京杲嘗以忿怒杖殺部曲,有司劾奏京杲殺人當死,從之。忠臣奏曰:「京杲合死久矣!儒生;上問之,對曰:「渠柏叔某於某處戰死,兄弟某於某處戰死,渠嘗從行,獨不死,是以知渠合死久矣。」上亦憫然,不令加罪,改授王傅而已。



 忠臣木強率直,不識書,不喜儒生;及罷兵權,官位崇重,常鬱鬱不得志。及硃泚反,以為偽司空、兼侍中。泚率兵逼奉天,命忠臣京城留守。泚敗,忠臣走樊川別業,李晟下將士擒忠臣至,系之有司。興元元年,並其子並誅
 斬之,時年六十九,籍沒其家。



 李希烈,遼西人。父大定。希烈少從平盧軍,後隨李忠臣過海至河南。寶應初,忠臣為淮西節度,署希烈為偏裨,累授將軍、試光祿卿、殿中監。忠臣兼領汴州,希烈為左廂都虞候,加開府儀同三司。大歷末,忠臣軍政不修,事多委妹婿張惠光,為押衙,弄權縱恣,人怨。與少將丁皓等斬惠光父子,忠臣奔赴朝廷。詔以忻王為淮西節度副大使,授希烈蔡州刺史、兼御史中丞、淮西節度留後,
 令滑亳節度李勉兼領汴州。



 德宗即位後月餘,加御史大夫,充淮西節度支度營田觀察使,又改淮西節度淮寧軍以寵之。建中元年,又加檢校禮部尚書。會山南東道節度梁崇義拒捍朝命,迫脅使臣,二年六月,詔諸軍節度率兵討之;加希烈南平郡王,兼漢北都知諸兵馬招撫處置使。希烈破崇義眾,遂討平之。錄希烈功,加檢校右僕射、同平章事,賜實封五百戶。淄青節度李正己又謀不軌,三年秋,加希烈檢校司空,兼淄青兗鄆登萊
 齊等州節度支度營田、新羅、渤海兩蕃使,令討襲正己。希烈遂率所部三萬人移居許州,聲言遣使往青州招諭李納,其實潛與交通,又移牒汴州令備供擬,將與納同為亂。李勉以其道路合自陳留,乃除道具饌以待之,希烈不從,乃大慢罵。自是志意縱肆,言多悖慢,日遣使交通河北諸賊帥等。是歲長至日,硃滔、田悅、王武俊、李納各僭稱王,滔使至希烈,希烈亦僭稱建興王、天下都元帥。



 四年,希烈遣其將襲陷汝州,執李元平而去,東都
 大擾亂。朝廷猶為含容,遣太子太師顏真卿往宣慰。真卿發後數日,以龍武將軍哥舒曜為東都兼汝州行營兵馬節度。希烈既見真卿,但肆兇言,令左右慢罵,指斥朝廷。又遣逆黨董待名、韓霜露、劉敬宗、陳質、翟暉等四人伺外,侵抄州縣,官軍皆為其所敗,荊南節度張伯儀全軍覆沒。又令周曾、王玢、姚憺、呂從賁、康琳等來襲曜,曾、玢、憺等謀回軍據蔡州襲討希烈,事洩,並遇害。神策軍使白志貞又獻策謀,令嘗為節度、都團練使者各出
 家僮部曲一人及馬,令劉德信總之討希烈。尋詔李勉為淮西招討使,哥舒曜為副。至四月,曜率眾屯襄城,頻與賊戰,皆不勝。八月,希烈率眾二萬圍襄城,李勉又令將唐漢臣率兵與劉德信同為曜之影援,皆望風敗衄。希烈兇逆既甚,帝乃命舒王為荊襄、江西、沔鄂等道節度諸軍行營兵馬都元帥,大開幕府,文武僚屬之盛,前後出師,未有其比。又令涇原諸道出兵,皆赴襄城。軍未發,會涇州兵亂,車駕幸奉天。其日,希烈大破曜軍於襄
 城,曜遁歸東都,賊因乘勝攻陷汴州,李勉奔歸宋州。



 希烈性慘毒酷,每對戰陣殺人,流血盈前,而言笑飲饌自若,以此人畏而服從其教令,盡其死力。其攻汴州,驅百姓,令運木土築壘道,又怒其未就,乃驅以填之,謂之濕梢。既入汴州,於是僭號曰武成,以孫廣、鄭賁、李綬、李元平為宰相;以汴州為大梁府,李清虛為尹,署百官。遣兵東討,至寧陵,竟為劉洽所拒,不得前。又遣將翟暉率精卒襲陳州,為劉洽、李納大破之,生擒暉以獻。諸軍乘勝
 進攻汴州,希烈遁歸蔡州,擒其偽署將相鄭賁、劉敬宗等。李皋、樊澤、曲環、張建封又四面討襲之,累拔其郡縣,希烈敗衄。貞元二年三月,因食牛肉遇疾,其將陳仙奇令醫人陳仙甫置藥以毒之而死。妻男骨肉兄弟共一十七人,並誅之。



 初,希烈於唐州得象一頭,以為瑞應,又上蔡、襄城獲其珍寶,乃是爛車釭及滑石偽印也。



 陳仙奇者,起於行間,性忠果。自希烈死,朝廷授淮西節度,頗竭誠節。未幾,為別將吳少誠所殺,贈太子太保,賻布帛、
 米粟有差,喪事官給。



 吳少誠,幽州潞縣人。父為魏博節度都虞候。少誠以父勛授一子官,釋褐王府戶曹。後至荊南,節度使庾準奇之,留為衙門將。準入覲,從至襄漢,見梁崇義不遵憲度,知有異志,少誠密計有成擒之略,將自陳於闕下。屬李希烈初授節制,銳意立功,見少誠計慮,乃以少誠所見錄奏,有詔慰飭,不次封通義郡王。未幾,崇義違命,希烈受制專征,以少誠為前鋒。崇義平,賜實封五千戶。後希
 烈叛,少誠頗為其用。希烈死,少誠等初推陳仙奇統戎事,朝廷已命仙奇,尋為少誠所殺,眾推少誠知留務。朝廷遂授以申光蔡等州節度觀察兵馬留後,尋正授節度。



 少誠善為治,勤儉無私,日事完聚,不奉朝廷。貞元三年,判官鄭常及大將楊冀謀逐少誠以聽命於朝,試校書郎劉涉假為手詔數十,潛致於大將,欲因少誠之出,閉城門以拒之。屬少誠將出餞中使,常、冀等遂謀舉事;臨發,為人所告,常、冀先遇害。其將李嘉節等各持假詔
 請罪,少誠悉宥之。其大將宋炅、曹齊奔歸京師。



 十五年,陳許節度曲環卒,少誠擅出兵攻掠臨潁縣,節度留後上官涚遣兵赴救,臨潁鎮使韋清與少誠通,救兵三千餘人,悉擒縛而去。九月,遂圍許州。尋下詔削奪少誠官爵,分遣十六道兵馬進討。十二月,官軍敗衄於小溵河。明年正月,夏州節度使韓全義為淮蔡招討處置使,北路行營諸軍將士,並取全義指揮,陳許節度留後上官涚充副使。五月,全義與少誠將吳秀、吳少陽等戰於



 溵
 水南,官軍復敗。七月,全義頓軍於五樓行營,為賊所乘,大潰,全義與都監軍使賈秀英、賈國良等夜遁,遂城守溵水。汴宋、徐泗、淄青兵馬直趣陳州,列營四面。少誠兵逼溵水五、六里下營,韓全義諸軍又退保陳州。其汴州、河陽等兵各私歸本道,陳許將孟元陽與神策兵各率所部留軍溵水。全義斬昭義、滑州、河陽、河中都將凡四人,然竟未嘗整陣交鋒,而王師累挫潰。少誠尋引兵退歸蔡州。遂下詔洗雪,復其官爵,累加檢校僕射。順宗即
 位,加同中書門平章事。元和初,遷檢校司空,依前平章事。元和四年十一月卒,年六十,廢朝三日,贈司徒。



 吳少陽,本滄州清池人。初,吳少誠父翔在魏博軍中,與少陽相愛。及少誠知淮西留守,乃厚以金帛取少陽至,則名以堂弟,署為軍職,累奏官爵,出入少誠家,情旨甚暱。少陽度少誠猜忍,懼為所害,乃請出外以任防捍之任,少誠乃表為申州刺史、兼御史大夫,凡五年。少陽頗寬易,而少誠之眾悅附焉。及少誠病亟,家僮單于熊兒者,
 偽以少誠意取少陽至,時少誠已不知人,乃偽署少陽攝副使、知軍州事。少誠子元慶,年二十餘,先為軍職,兼御史中丞,少陽密害之。及少誠死,少陽自為留後。時王承宗求繼士真,不受詔;憲宗怒,以討承宗,不欲兵連兩河,乃詔遂王宥遙領彰義軍節度大使,以少陽為留後。遂授彰義軍節度使、檢校工部尚書。少陽據蔡州凡五年,不朝覲。汝南多廣野大澤,得豢馬畜,時奪掠壽州茶山之利,內則數匿亡命,以富實其軍。又屢以牧馬來獻,
 詔因善之。元和九年九月卒,贈右僕射。



 吳元濟,少陽長子也。初為試協律郎、兼監察御史、攝蔡州刺史。及父死,不發喪,以病聞,因假為少陽表,請元濟主兵務。帝遣醫工候之,即稱少陽疾愈,不見而還。先是,少陽判官蘇兆、楊元卿及其將侯惟清嘗同為少陽畫朝覲計;及元濟自領軍,兇狠無義,唯暱軍中兇悍之徒。素不便兆,縊殺之,歸其尸於家,械侯惟清而囚之。時朝廷誤聞惟清已死,贈兵部尚書,贈蘇兆以右僕射。楊元卿先奏事在京
 師,得盡言經略淮西事於宰相李吉甫。始,少陽以病聞,元卿請凡淮西使在道路者,所在留止之。及少陽卒,凡四十日,不為輟朝,但易將加兵於外以待。其邸吏無何妄傳董重質已殺元濟,並屠其家;李吉甫遽請對拜賀,乃輟朝。數日,知元濟尚在。時賊陰計已成,群眾四出,狂悍而不可遏,屠舞陽,焚葉縣,攻掠魯山、襄城。汝州、許州及陽翟人多逃伏山谷荊棘間,為其殺傷驅剽者千里,關東大恐。



 十月,以陳州刺史李光顏為忠武軍節度使,
 又以山南東道節度使嚴綬充申光蔡等州招撫使,仍令內常侍崔潭峻監綬軍。十年正月,綬軍臨賊西境。詔曰:



 吳元濟逆絕人理,反易天常;不居父喪,擅領軍政。諭以詔旨,曾無謙恭,熒惑一方之人,迫脅三軍之眾。以少陽嘗經任使,為之軫悼,命申吊祭,臨遣使臣。陵虐封疆,遂致稽阻,絕朝廷之理,忘父子之恩。旋又掩寇舞陽,傷殘吏卒,焚燒葉縣,騷擾閭閻,恣行奪攘,無所畏忌。朕念賞延之義,重傷籓帥之門,尚欲納於忠順之途,處在顯
 榮之地。未能飭怒,猶為包荒,再降詔書,俾申招撫。而毒螫滋甚,奸心靡悛,壽春西南,又陷鎮柵,窮兇稔惡,縱暴延災。覆載之所不容,人神之所共棄,良非獲已,致此興戎。吳元濟在身官爵,並宜令削奪。令宣武、大寧、淮南、宣歙等道兵馬合勢,山南東道及魏博、荊南、江西、劍南東川兵馬與鄂岳許會,東都防禦使與懷鄭汝節度及義成兵馬掎角相應,同期進討。



 二月,綬兵為賊所襲,敗於磁丘,退保唐州。四月,光顏破賊黨,元濟遣人求援於鎮
 州王承宗、淄鄆李師道;二帥上表於朝廷,請赦元濟之罪,朝旨不從。自是兩河賊帥所在竊發,冀以沮撓王師。五月,承宗、師道遣盜燒河陰倉,詔御史中丞裴度於軍前宣喻,觀用兵形勢。度還奏曰:「臣觀諸將,唯光顏勇義盡心,必有成功。」上意甚悅。翌日,光顏奏大破賊於時曲,上曰:「度知光顏,可謂至矣!」乃以度兼刑部侍郎。自是中外相賀,決不赦賊,徵天下兵環申、蔡之郊,大小十餘鎮。六月,承宗、師道遣盜伏於京城,殺宰相武元衡、中丞裴
 度;衡先死,度重傷而免。憲宗特怒,即命度為宰相,淮右用兵之事,一以委之。七月,李師道遣嵩山僧圓凈結山賊與留邸兵,欲焚燒東都,先事敗而禍弭。嚴綬退罷,乃以汴州節度使韓弘為淮右行營兵馬都統;以高霞寓有名,用為唐鄧節度。



 十一年春,諸軍雲合,惟李光顏、懷汝節度烏重胤心無顧望,旦夕血戰,繼獻戎捷。六月,高霞寓為賊所擊,敗於鐵城,退保新興柵。時諸軍勝負皆不實聞,多虛稱克捷;及霞寓敗,中外恟々。宰相諫官屢
 以罷兵為請,唯裴度堅於破賊。尋以袁滋代霞寓為唐鄧帥,滋柔懦不能軍。十二年正月,袁滋復貶,閑廄使李醖表請軍前自效,乃用醖為唐鄧帥以代滋。醖軍壓境,拔賊文城柵,擒柵將吳秀琳,又獲賊將李祐。李光顏亦拔賊郾城。元濟始懼,盡發左右及守城卒,屬董重質以抗光顏、重胤。



 六月,元濟乞降,為群賊所制,不能自拔。上以元兇已蹙,兵未臨於賊城,輓饋日殫,因延英問計於宰相。裴度曰:「賊力已困,但群帥不一,故未能決降。」上曰:「
 卿決能行乎?」曰:「臣誓不與賊偕全。」七月,詔以度為彰義軍節度使,兼申光蔡四面行營招撫使,以郾城為行在,蔡州為節度所。八月,度至郾城,激勵士眾,軍士喜度至,以賞罰必行,皆願輸罄;每出勞,軍士有流涕者。



 時李醖營文城柵,既得吳秀琳、李祐,知其可用,委信無疑,日夜與計事於帳中。祐曰:「元濟勁軍,多在洄曲西境防捍,而守蔡者皆市人疲耄之卒,可以乘虛掩襲,直抵懸匏,比賊將聞之,元濟成擒矣!」醖然之,咨於裴度。度曰:「兵非出
 奇不勝,常侍良圖也。」十一月,醖夜出軍,令李祐率勁騎三千為前鋒,田進誠三千為後軍,醖自率三千為中軍。其月十日夜,至蔡州城下,坎墻而畢登,賊不之覺。十一日,攻衙城,擒元濟並其家屬以聞。



 初,元濟之叛,恃其兇狠,然治軍無紀綱。其將趙昌洪、凌朝江、董重質等各權兵外寇。李師道鄆州之鹽,城往來寧陵、雍丘之間,韓弘知而不禁。淮右自少誠阻兵已來,三十餘年,王師加討,未嘗及其城下,嘗走韓全義,敗于頔,故驕悍無所顧忌。
 且恃城池重固,有陂浸阻回,故以天下兵環攻三年,所克者一縣而已。及黜高霞寓、李遜、袁滋,諸軍始進。又得陰山府沙阤驍騎、邯鄲勇卒,光顏、重胤之奮命,及丞相臨統,破諸將首尾之計,力擒元惡。



 申、蔡之始,人劫於希烈、少誠之虐法,而忘其所歸。數十年之後,長者衰喪,而壯者安於毒暴而恬於搏噬。地既少馬,而廣畜騾,乘之教戰,謂之騾子軍。尤稱勇悍,而甲仗皆畫為雷公星文,以為厭勝;而少誠能以奸謀固眾心。



 初,韓全義敗於溵
 水,蔡兵於全義帳中得公卿間問訊書,少誠束而諭眾曰:「朝廷公卿以此書托全義,收蔡州日,乞一將士妻女以為婢妄。」以此激怒其眾,絕其歸向之心。是以蔡人有老死不聞天子恩宥者,故堅為賊用。地雖中州,人心過於夷貊,乃至搜閱天下豪銳,三年而後屈者,彼非將才而力備,蓋勢驅性習,不知教義之所致也。



 元濟至京,憲宗御興安門受浮,百僚樓前稱賀,乃獻廟社,徇於兩京,斬之於獨柳,時年三十五。其夜失其首。妻沈氏,沒入掖
 庭;弟二人、子三人,流於江陵誅之;判官劉協庶七人皆斬。光、蔡等州平,始復為王土矣。



 史臣曰:治亂,勢也,勢亂不能卒治。長源以法繩驕軍,禍不旋踵;則董公之寬柔不無謂。古之名將,以陰謀怨望,鮮全其族者。董秦始奮忠義,多長者言,宜其顯赫,及失意挾邪,俄被淮陰之戮,惜哉!吳少誠為希烈之亂胎,雖謀奪其軍,及嗣而滅。而元濟效希烈之狂悖,謂無天地,人之兇險,一至於斯!是知王者御治之道,其可忽諸!



 贊曰:聖哲之君,慎名與器。不軌之臣,得寵則戾。董怨而族,吳悖而菹。好亂樂禍,可監前車。



\end{pinyinscope}