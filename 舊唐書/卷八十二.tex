\article{卷八十二}

\begin{pinyinscope}

 ○於
 志寧高季輔張行成族孫易之昌宗



 於志寧,雍州高陵人,周太師燕文公謹之曾孫也。父宣道,隋內史舍人。志寧,大業末為冠氏縣長,時山東群盜
 起,乃棄官歸鄉里。高祖將入關,率群從於長春宮迎接,高祖以其有名於時,甚加禮遇,授銀青光祿大夫。太宗為渭北道行軍元帥,召補記室,與殷開山等參贊軍謀。及太宗為秦王、天策上將,志寧累授天策府從事中郎,每侍從征伐,兼文學館學士。貞觀三年,累遷中書侍郎。太宗命貴臣內殿宴,怪不見志寧,或奏曰:「敕召三品已上,志寧非三品,所以不來。」太宗特令預宴,即加授散騎常侍,行太子左庶子。累封黎陽縣公。時議者欲立七廟,
 以涼武昭王為始祖,房玄齡等皆以為然。志寧獨建議以為武昭遠祖,非王業所因,不可為始祖。太宗又以功臣為代襲刺史,志寧以今古事殊,恐非久安之道,上疏爭之。皆從志寧所議。太宗因謂志寧曰:「古者太子既生,士負之,即置輔弼。昔成王幼小,周、召為師傅,日聞正道,習以成性。今皇太子既幼少,卿當輔之以正道,無使邪僻開其心。勉之無怠,當稱所委,官賞可不次而得也。」志寧以承乾數虧禮度,志在匡救,撰《諫苑》二十卷諷之。太
 宗大悅,賜黃金十斤、絹三百匹。十四年,兼太子詹事。明年,以母憂解。尋起復本官,屢表請終喪禮,太宗遣中書侍郎岑文本就宅敦諭之曰:「忠孝不並,我兒須人輔弼,卿宜抑割,不可徇以私情。」志寧遂起就職。



 時皇太子承乾嘗以盛農之時,營造曲室,累月不止,所為多不法。志寧上書諫曰:



 臣聞克儉節用,實弘道之源;崇侈恣情,乃敗德之本。是以凌雲概日,戎人於是致譏;峻宇雕墻,《夏書》以之作誡。昔趙盾匡晉,呂望師周,或勸之以節財,或
 諫之以厚斂,莫不盡忠以佐國,竭誠以奉君,欲茂實播於無窮,英聲被乎物聽。咸著簡策,以為美談。今所居東宮,隋日營建,睹之者尚譏其侈,見之者猶嘆其華。何容此中更有修造,財帛日費,土木不停,窮斤斧之工,極磨礱之妙?且丁匠官奴入內,比者曾無伏監。此等或兄犯國章,或弟罹王法,往來御苑,出入禁闈,鉗鑿緣其身,槌杵在其手。監門本防非慮,宿衛以備不虞,直長既自不知,千牛又復不見。爪牙在外,廝役在內,所司何以自安,
 臣下豈容無懼?又鄭、衛之樂,古謂淫聲。昔朝歌之鄉,回車者墨翟;夾谷之會,揮劍者孔丘。先聖既以為非,通賢將以為失。頃聞宮內,屢有鼓聲大樂,伎兒入便不出。聞之者股慄,言之者心戰。往年口敕,伏請重尋,聖旨殷勤,明誡懇切。在於殿下,不可不思,至於微臣,不得無懼。臣自驅馳宮闕,已積歲年,犬馬尚解識恩,木石猶能知感,所有管見,敢不盡言?如鑒以丹誠,則臣有生路;若責其忤旨,則臣是罪人。但悅意取容,藏孫方之疾疹;犯顏逆
 耳,《春秋》比之藥石。伏望停工匠之作,罷久役之人,絕鄭、衛之音,斥群小之輩,則三善允備,萬國作貞矣。



 承乾不納。承乾又令閹官多在左右,志寧上書諫曰:



 臣聞堯稱稽古,功著於搜揚;舜曰聰明,績彰於去惡。然開元立極,布政辨方,莫不旌賁英賢,驅除不肖。理亂之本,咸在於茲。況閹宦之徒,體非全氣,更蕃階闥,左右宮闈,托親近以立威權,假出納以為禍福。昔易牙被任,變起齊邦;張讓執鈞,亂生漢室。伊戾為詐,宋國受其殃;趙高作奸,秦
 氏鐘其弊。加以弘、石用事,京、賈則連首受誅;王、曹掌權,何、竇則踵武被戮。遂使縉紳重足,宰司屏氣。然順其情者,則榮逮幼沖;迕其意者,則災及襁褓。爰暨高齊都鄴,亦弊閹官。鄧長顒位至侍中,陳德信爵隆開府,外干朝政,內預宴私,宗枝藉其吹噓,重臣仰其鼻息。罪積山嶽,靡掛於刑書;功無涓塵,已勒於鐘鼎,富逾金穴,財甚銅山。是以家起怨嗟,人懷憤嘆。骨鯁之士,語不見聽;謇諤之臣,言必被斥。齊都顛覆,職此之由。向使任諒直之臣,
 退佞給之士,據趙、魏之地,擁漳、滏之兵,修德行仁,養政施化,何區區周室而敢窺覦者焉!然杜漸防萌,古人所以遠禍;以大喻小,先哲於焉取則。伏惟殿下道茂重離,德光守器,憲章古始,祖述前修,欲使休譽遠聞,英聲遐暢。臣竊見寺人一色,未識上心,或輕忽高班,凌轢貴仕,便是品命失序,綱紀不立,取笑通方之人,見譏有識之士。然典內職掌,唯在門外通傳;給使主司,但緣階闥供奉。今乃往來閣內,出入宮中,行路之人,咸以為怪。伏望
 狎近君子,屏黜小人,上副聖心,下允眾望。



 承乾覽書甚不悅。承乾嘗驅使司馭等,不許分番,又私引突厥達哥支入宮內。志寧上書諫曰:



 臣聞上天蓋高,日月以光其德;明君至聖,輔佐以贊其功。是以周誦升儲,見匡毛、畢;漢盈居震,取資黃、綺。姬旦抗法於伯禽,賈生陳事於文帝。莫不殷勤於端士,懇切於正人。昔鄧禹名臣,方居審諭之任;疏受宿望,始除輔導之官。歷代賢君,莫不丁寧於太子者,良以地膺上嗣,位處副君,善則率土沾其恩,
 惡則海內罹其禍。近聞僕寺、司馭,爰及駕士、獸醫,始自春初,迄茲夏晚,常居內役,不放分番。或家有尊親,闕於溫凊;或室有幼弱,絕於撫養。春則廢其耕墾,夏又妨其播殖。事乖存愛,恐致怨嗟。且突厥達哥支等,人面獸心,豈得以禮教期,不可以仁信待。心則未識於忠孝,言則莫辯其是非,近之有損於英聲,暱之無益於盛德。引之入閣,人皆驚駭,豈臣愚識,獨用不安?臣下為殿下之股肱,殿下為臣下之君父,君父以存撫為務,股肱以匡救
 為心。是以苦口之藥以奉身,逆耳之言以安位。古人樹誹謗之木,以求己愆;懸敢諫之鼓,以思身過。由是從諫之主,鼎祚克昌;愎諫之君,洪業隳墜。



 承乾大怒,陰遣刺客張師政、紇干承基就殺之。二人潛入其第,見志寧寢處苫廬,竟不忍而止。及承乾敗後,推鞫具知其事。太宗謂志寧曰:「知公數有規諫,事無所隱。」深加勉勞。右庶子令狐德棻等以無諫書,皆從貶責。及高宗為皇太子,復授志寧太子左庶子,未幾遷侍中。永徽元年,加光祿大
 夫,進封燕國公。二年,監修國史。時洛陽人李弘泰坐誣告太尉長孫無忌,詔令不待時而斬決。志寧上疏諫曰:



 伏惟陛下情篤功臣,恩隆右戚。以無忌橫遭誣告,事並是虛,欲戮告人,以明賞罰,一以絕誣告之路,二以慰勛戚之心。又以所犯是真,無忌便有破家之罪,今告為妄,弘泰宜戮不待時。且真犯之人,事當罪逆;誣謀之類,罪唯及身。以罪較量,明非惡逆,若欲依律,合待秋分。今時屬陽和,萬物生育,而特行刑罰。此謂傷春。竊案《左傳》聲
 子曰:「賞以春夏,刑以秋冬。」順天時也。又《禮記·月令》曰:「孟春之月,無殺孩蟲。省囹圄,去桎梏,無肆掠,止獄訟。」又《漢書》董仲舒曰:「王者欲有所為,宜求其端於天道。天道之大者在陰陽。陽為德,陰為刑,刑主殺而德主生。陽常居大夏,而以生育養長為事;陰常居大冬,而積於空虛不用之處。以此見天之任德不任刑也。」伏惟陛下纂聖升祚,繼明御極,追連、胥之絕軌,蹈軒、頊之良規。欲使舉動順於天時,刑罰依於律令,陰陽為之式序,景宿於是靡
 差,風雨不愆,雩禜輟祀。方今太蔟統律,青陽應期,當生長之辰,施肅殺之令,伏願暫回聖慮,察古人言,倘蒙垂納,則生靈幸甚。



 疏奏,帝從之。是時,衡山公主欲出降長孫氏,議者以時既公除,合行吉禮。志寧上疏曰:



 臣聞明君馭歷,當俟獻替之臣;聖主握圖,必資鹽梅之佐。所以堯詢四岳,景化洽於區中;舜任五臣,懿德被於無外。左有記言之史,右立記事之官,大小咸書,善惡俱載。著懲勸於簡牘,垂褒貶於人倫,為萬古之範圍,作千齡之龜
 鏡。伏見衡山公主出降,欲就今秋成禮。竊按《禮記》云:「女十五而笄,二十而嫁;有故,二十三而嫁。」鄭玄云:「有故,謂遭喪也。」固知須終三年。《春秋》云:「魯莊公如齊納幣。」杜預云:「母喪未再期而圖婚,二傳不譏失禮,明故也。」此即史策具載,是非歷然,斷在聖情,不待問於臣下。其有議者云:「準制,公除之後,須並從吉。」此漢文創制其儀,為天下百姓。至於公主,服是斬縗,縱使服隨例除,無宜情隨例改。心喪之內,方復成婚,非唯違於禮經,亦是人情不可。
 伏惟陛下嗣膺寶位,臨統萬方,理宜繼美羲、軒,齊芳湯、禹,弘獎仁孝之日,敦崇名教之秋。此事行之苦難,猶須抑而守禮,況行之甚易,何容廢而受譏?此理有識之所共知,非假愚臣之說也。伏願遵高宗之令軌,略孝文之權制,國家於法無虧,公主情禮得畢。



 於是詔公主待三年服闋,然後成禮。其年,拜尚書左僕射、同中書門下三品。三年,以本官兼太子少師。



 顯慶元年,遷太子太傅。嘗與右僕射張行成、中書令高季輔俱蒙賜地,志寧奏曰:「
 臣居關右,代襲箕裘,周魏以來,基址不墜。行成等新營莊宅,尚少田園,於臣有餘,乞申私讓。」帝嘉其意,乃分賜行成及季輔。四年,表請致仕,聽解尚書左僕射,拜太子太師,仍同中書門下三品。高宗之將廢王庶人也。長孫無忌、褚遂良執正不從,而李勣、許敬宗密申勸請,志寧獨無言以持兩端。及許敬宗推鞫長孫無忌詔獄,因誣構志寧黨附無忌,坐是免職,尋降授榮州刺史。麟德元年,累轉華州刺史,年老請致仕,許之。二年,卒於家,年七
 十八。贈幽州都督,謚曰定。上元三年,追復其左光祿大夫、太子太師。志寧雅愛賓客,接引忘倦,後進文筆之士,無不影附,然亦不能有所薦達,議者以此少之。前後預撰格式律令、《五經義疏》及修禮、修史等功,賞賜不可勝計。有集二十卷。子立政,太僕少卿。志寧玄孫休烈,休烈子益,自有傳。



 高季輔,德州蓚人也。祖表,魏安德太守。父衡,隋萬年令。季輔少好學,兼習武藝。居母喪以孝聞。兄元道,仕隋為
 汲令。武德初,縣人翻城從賊,元道被害,季輔率其黨出鬥,竟擒殺其兄者,斬之持首以祭墓,甚為士友所稱。由是群盜多歸附之,眾至數千。尋與武陟人李厚德率眾來降,授陟州總管府戶曹參軍。貞觀初,擢拜監察御史,多所彈糾,不避權要。累轉中書舍人。



 時太宗數召近臣,令指陳時政損益。季輔上封事五條:其略曰:



 陛下平定九州,富有四海,德超邃古,道高前烈。時已平矣,功已成矣,然而刑典未措者,何哉?良由謀猷之臣,不弘簡易之
 政;臺閣之使,昧於經遠之道。執憲者以深刻為奉公,當官者以侵下為益國,未有坦平恕之懷,副聖明之旨。至如設官分職,各有司存。尚書八座,責成斯在,王者司契,義屬於茲。伏願隨方訓誘,使各揚其職。仍須擢溫厚之人,升清潔之吏;敦樸素,革澆浮,先之以敬讓,示之以好惡,使家識孝慈,人知廉恥。醜言過行,見嗤於鄉閭;忘義私暱,取擯於親族。杜其利欲之心,載以清凈之化。自然家肥國富,氣和物阜。禮節於是競興,禍亂何由而作?



 又
 曰:



 竊見聖躬,每存節儉,而凡諸營繕,工徒未息。正丁正匠,不供驅使,和雇和市,非無勞費。人主所欲,何事不成?猶願愛其財而勿殫,惜其力而勿竭。今畿內數州,實惟邦本,地狹人稠,耕植不博,菽粟雖賤,儲蓄未多,特宜優矜,令得休息。強本弱枝,自古常事。關、河之外,徭役全少,帝京、三輔,差科非一;江南、河北,彌復優閑。須為差等,均其勞逸。



 又曰:



 今公主之室,封邑足以給資用;勛貴之家,俸祿足以供器服。乃戚戚於儉約,汲汲於華侈,放息出
 舉,追求什一。公侯尚且求利,黎庶豈覺其非?錐刀必競,實由於此,有黷朝風,謂宜懲革。



 又曰:



 仕以應務代耕,外官卑品,猶未得祿,既離鄉家,理必貧匱。但妻子之戀,賢達猶累其懷;饑寒之切,夷、惠罕全其行。為政之道,期於易從。若不恤其匱乏,唯欲責其清勤,凡在末品,中庸者多,止恐巡察歲去,輶軒繼軌。不能肅其侵漁,何以求其政術?今戶口漸殷,倉廩已實,斟量給祿,使得養親。然後督以嚴科,責其報效,則庶官畢力,物議斯允。



 又曰:



 竊見
 密王元曉等,俱是懿親,陛下友愛之懷,義高古昔,分以車服,委以籓維,須依禮儀,以副瞻望。比見帝子拜諸叔,諸叔亦答拜,王爵既同,家人有禮,豈合如此顛倒昭穆?伏願一垂訓誡,永循彞則。



 書奏,太宗稱善。十七年,授太子右庶子,又上疏切諫時政得失,特賜鐘乳一劑,曰:「進藥石之言,故以藥石相報。」十八年,加銀青光祿大夫,兼吏部侍郎,凡所銓敘,時稱允當。太宗嘗賜金背鏡一面,以表其清鑒焉。二十二年,遷中書令,兼檢校吏部尚書、
 監修國史,賜爵蓚縣公。永徽二年,授光祿大夫,行侍中,兼太子少保。以風疾廢於家,乃召其兄虢州刺史季通為宗正少卿視其疾,又屢降中使,觀其進食,問其增損。尋卒,年五十八。帝為之舉哀,廢朝三日,贈開府儀同三司、荊州都督,謚曰憲。



 子正業,仕至中書舍人,坐與上官儀善,配流嶺外。



 張行成,定州義豐人也。少師事河間劉炫,勤學不倦,炫謂門人曰:「張子體局方正,廊廟才也。」大業末,察孝廉,為
 謁者臺散從員外郎。王世充僭號,以為度支尚書。世充平,以隋資補宋州穀熟尉。又應制舉乙科,授雍州富平縣主簿,理有能名。秩滿,補殿中侍御史。糾劾不避權戚,太宗以為能,謂房玄齡曰:「觀古今用人,必因媒介,若行成者,朕自舉之,無先容也。」太宗嘗言及山東、關中人,意有同異,行成正侍宴,跪而奏曰:「臣聞天子以四海為家,不當以東西為限;若如是,則示人以益狹。」太宗善其言,賜名馬一匹、錢十萬、衣一襲。自是每有大政,常預議焉。
 累遷給事中。太宗嘗臨軒謂侍臣曰:「朕所以不能恣情欲,取樂當年,而勵節苦心,卑宮菲食者,正為蒼生耳。我為人主,兼行將相之事,豈不是奪公等名?昔漢高祖得蕭、曹、韓、彭,天下寧宴;舜、禹、湯、武有稷、契、伊、呂,四海乂安。此事朕並兼之。」行成退而上書諫曰:「有隋失道,天下沸騰,陛下撥亂反正,拯生人於塗炭,何周、漢君臣之所能擬?陛下聖德含光,規模弘遠,雖文武之烈,實兼將相,何用臨朝對眾與其較量,以萬乘至尊,共臣下爭功哉?臣
 聞『天何言哉,四時行焉』;又聞『汝惟不矜,天下莫與汝爭能』。臣備員樞近,非敢知獻替之事,輒陳狂直,伏待菹醢。」太宗深納之。轉刑部侍郎、太子少詹事。太宗東征,皇太子於定州監國,即行成本邑也。太子謂行成曰:「今者送公衣錦還鄉。」於是令有司祀其先人墓。行成因薦鄉人魏唐卿、崔寶權、馬龍駒、張君劼等,皆以學行著聞,太子召見,以其老不任職,皆厚賜而遣之。太子又使行成詣行在所,太宗見之甚悅,賜馬二匹、縑三百匹。駕還京,為
 河南巡察大使。還,稱旨,以本官兼檢校尚書左丞。是歲,太宗幸靈州,太子當從,行成上疏曰:「伏承皇太子從幸靈州。臣愚以為皇太子養德春宮,日月未幾,華夷遠邇,佇聽嘉音。如因以監國,接對百僚,決斷庶務,明習政理,既為京師重鎮,且示四方盛德。與其出陪私愛,曷若俯從公道?」太宗以為忠,進位銀青光祿大夫。二十三年,遷侍中,兼刑部尚書。太宗崩,與高季輔侍高宗即位於太極殿梓宮前。尋封北平縣公,監修國史。時晉州地連震,
 有聲如雷,高宗以問行成。行成對曰:「天,陽也;地,陰也。陽,君象;陰,臣象。君宜轉動,臣宜安靜。今晉州地動,彌旬不休。雖天道玄邈,窺算不測;而人事較量,昭然作戒。恐女謁用事,大臣陰謀,修德禳災,在於陛下。且陛下本封晉也,今地震晉州,下有徵應,豈徒然耳。伏願深思遠慮,以杜未萌。」二年八月,拜尚書左僕射。尋加授太子少傅。四年,自三月不雨至於五月,復抗表請致仕。高宗手制答曰:「密雲不雨,遂淹旬月,此朕之寡德,非宰臣咎。實甘萬
 方之責,用陳六事之過。策免之科,義乖罪己。今敕斷表,勿復為辭。」賜宮女黃金器物。固請乞骸骨,高宗曰:「公,我之故舊腹心,奈何舍我而去?」因愴然流涕。行成不得已,復起視事。九月,卒於尚書省,時年六十七。高宗哭之甚哀,輟朝三日,令九品已上就第哭。比斂,中使三至,賜內衣服,令尚宮宿於家,以視殯斂。贈開府儀同三司、並州都督。所司備禮冊命,祭以少牢,賻絹布八百段、米粟八百石,賜東園秘器,謚曰定。弘道元年,詔以行成配享高
 宗廟庭。子洛客嗣,官至雍州渭南令。



 行成族孫易之、昌宗。易之父希臧,雍州司戶。易之初以門廕,累遷為尚乘奉御,年二十餘,白皙美姿容,善音律歌詞。則天臨朝,通天二年,太平公主薦易之弟昌宗入侍禁中,既而昌宗啟天后曰:「臣兄易之器用過臣,兼工合煉。」即令召見,甚悅。由是兄弟俱侍宮中,皆傅粉施硃,衣錦繡服,俱承闢陽之寵。俄以昌宗為雲麾將軍,行左千牛中郎將;易之為司衛少卿。賜第一區、物五百段、奴婢駝馬等。信宿,加
 昌宗銀青光祿大夫,賜防閣,同京官朔望朝參。仍贈希臧襄州刺史,母韋氏阿臧封太夫人,使尚宮至宅問訊,仍詔尚書李迥秀私侍阿臧。武承嗣、三思、懿宗、宗楚客、宗晉卿候其門庭,爭執鞭轡,呼易之為五郎,昌宗為六郎。俄加昌宗左散騎常侍。聖歷二年,置控鶴府官員,以易之為控鶴監、內供奉,餘官如故。久視元年,改控鶴府為奉宸府,又以易之為奉宸令,引辭人閻朝隱、薛稷、員半千並為奉宸供奉。每因宴集,則令嘲戲公卿以為笑
 樂。若內殿曲宴,則二張、諸武侍坐,樗蒲笑謔,賜與無算。時諛佞者奏云,昌宗是王子晉後身。乃令被羽衣,吹簫,乘木鶴,奏樂於庭,如子晉乘空。辭人皆賦詩以美之,崔融為其絕唱,其句有「昔遇浮丘伯,今同丁令威。中郎才貌是,藏史姓名非。」天后令選美少年為左右奉宸供奉,右補闕硃敬則諫曰:「臣聞志不可滿,樂不可極。嗜欲之情,愚智皆同,賢者能節之不使過度,則前聖格言也。陛下內寵,已有薛懷義、張易之、昌宗,固應足矣。近聞上舍
 奉御柳模自言子良賓潔白美須眉,左監門衛長史侯祥雲陽道壯偉,過於薛懷義,專欲自進堪奉宸內供奉。無禮無儀,溢於朝聽。臣愚職在諫諍,不敢不奏。」則天勞之曰:「非卿直言,朕不知此。」賜彩百段。以昌宗醜聲聞於外,欲以美事掩其跡,乃詔昌宗撰《三教珠英》於內。乃引文學之士李嶠、閻朝隱,徐彥伯、張說、宋之問、崔湜、富嘉謨等二十六人,分門撰集。成一千三百卷,上之。加昌宗司僕卿,封鄴國公,易之為麟臺監,封恆國公,各實封三
 百戶。俄改昌宗為春官侍郎。易之、昌宗皆粗能屬文,如應詔和詩,則宋之問、閻朝隱為之代作。則天春秋高,政事多委易之兄弟。中宗為皇太子,太子男邵王重潤及女弟永泰郡主竊言二張專政。易之訴於則天,付太子自鞫問處置,太子並自縊殺之。又御史大夫魏元忠嘗奏二張之罪,易之懼不自安,乃誣奏元忠與司禮丞高戩云:「天子老矣,當挾太子為耐久朋。」則天曰:「汝何以知之?」易之曰:「鳳閣舍人張說為證。」翌日,則天召元忠及說
 廷詰之,皆妄。則天尚以二張之故,逐元忠為高要尉,張說長流欽州。長安二年,易之贓賂事發,為御史臺所劾下獄,兄司府少卿昌儀、司禮少卿同休皆貶黜。及則天臥疾長生院,宰臣希得進見,唯易之兄弟侍側,恐禍變及己,乃引用朋黨,陰為之備。人有榜其事於路,左臺御史中丞宋璟請按之。則天陽許,尋敕宋璟使幽州按都督屈突仲翔,令司禮卿崔神慶鞫之。神慶希旨雪昌宗兄弟。



 神龍元年正月,則天病甚。是月二十日,宰臣崔玄
 暐、張柬之等起羽林兵迎太子,至玄武門,斬關而入,誅易之、昌宗於迎仙院,並梟首於天津橋南。則天遜居上陽宮。易之兄昌期,歷岐、汝二州刺史,所在苛猛暴橫,是日亦同梟首。朝官房融、崔神慶、崔融、李嶠、宋之問、杜審言、沈佺期、閻朝隱等皆坐二張竄逐,凡數十人。



 史臣曰:於燕公輔導儲皇,高侍中敷陳理行,張北平斥言陰沴,皆人所難言者。茍非金玉貞度,松筠挺操,安能咈人主之意,獻苦口之忠?宜其論道巖廊,克終顯盛。古
 所謂能以義匡主之失,三君有焉。



 贊曰:猗歟於公,獻替兩宮。前修克繼,嗣德彌隆。高酬藥劑,張感宸衷。君臣之義,斯為始終。



\end{pinyinscope}