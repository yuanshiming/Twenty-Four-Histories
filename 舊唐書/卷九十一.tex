\article{卷九十一}

\begin{pinyinscope}

 ○裴炎
 劉禕之魏玄同李昭德



 裴炎,絳州聞喜人也。少補弘文生,每遇休假,諸生多出游,炎獨不廢業。歲餘,有司將薦舉,辭以學未篤而止。在館垂十載,尤曉《春秋左氏傳》及《漢書》。擢明經第,尋為濮
 州司倉參軍。累歷兵部侍郎、中書門下平章事、侍中、中書令。



 永淳元年,高宗幸東都,留太子哲守京師,命炎與劉仁軌、薛元超為輔。明年,高宗不豫,炎從太子赴東都侍疾。十一月,高宗疾篤,命太子監國,炎奉詔與黃門侍郎劉齊賢、中書侍郎郭正一並於東宮平章事。十二月丁巳,高宗崩,太子即位。未聽政,宰臣奏議,天后降令於門下施行。中宗既立,欲以後父韋玄貞為侍中,又欲與乳母子五品,炎固爭以為不可。中宗不悅,謂左右曰:「我
 讓國與玄貞豈不得,何為惜侍中耶?」炎懼,乃與則天定策廢立。炎與中書侍郎劉禕之、羽林將軍程務挺、張虔勖等勒兵入內,宣太后令,扶帝下殿。帝曰:「我有何罪?」太后報曰:「汝若將天下與韋玄貞,何得無罪!」乃廢中宗為盧陵王,立豫王旦為帝。炎以定策功,封河東縣侯。



 太后臨朝,天授初,又降豫王為皇嗣。時太后侄武承嗣請立武氏七廟及追王父祖,太后將許之。炎進諫曰:「皇太后天下之母,聖德臨朝,當存至公,不宜追王祖禰,以示自
 私。且獨不見呂氏之敗乎?臣恐後之視今,亦猶今之視昔。」太后曰:「呂氏之王,權在生人;今者追尊,事歸前代。存歿殊跡,豈可同日而言?」炎曰:「蔓草難圖,漸不可長。殷鑒未遠,當絕其源。」太后不悅而止。時韓王元嘉、魯王靈夔等皆皇屬之近,承嗣與從父弟三思屢勸太后因事誅之,以絕宗室之望。劉禕之、韋仁約並懷畏憚,唯唯無言,炎獨固爭,以為不可,承嗣深憾之。



 文明元年,官名改易,炎為內史。秋,徐敬業構逆,太后召炎議事。炎奏曰:「皇帝
 年長,未俾親政,乃致猾豎有詞。若太后返政,則此賊不討而解矣。」御史崔察聞而上言,曰:「裴炎伏事先朝,二十餘載,受遺顧托,大權在己,若無異圖,何故請太后歸政?」乃命御史大夫騫味道、御史魚承曄鞫之。鳳閣侍郎胡元範奏曰:「炎社稷忠臣,有功於國,悉心奉上,天下所知,臣明其不反。」右衛大將軍程務挺密表申理之,文武之間證炎不反者甚眾,太后皆不納。光宅元年十月,斬炎于都亭驛之前街。炎初被擒,左右勸炎遜詞於使者,炎
 嘆曰:「宰相下獄,焉有更全之理!」竟無折節。及籍沒其家,乃無儋石之蓄。胡元範,申州義陽人,坐救炎流死瓊州。程務挺伏法,納言劉齊賢貶吉州長史,吏部侍郎郭待舉貶岳州刺史,皆坐救炎之罪也。



 先是,開耀元年十月,定襄道行軍大總管裴行儉獻定襄所獲俘囚,除曲赦外,斬阿史那伏念,溫傅等五十四人於都市。初,行儉討伐之時,許伏念以不死,伏念乃降。時炎害行儉之功,奏云:「伏念是程務挺、張虔勖逼逐於營,又磧北回紇南向
 逼之,窘急而降。」乃殺之。行儉嘆曰:「渾、浚之事,古今恥之。但恐殺降之後,無復來者。」行儉因此稱疾不出。炎致國家負義而殺降,妒能害功,構成陰禍,其敗也宜哉!



 睿宗踐祚,下制曰:「飾終追遠,斯乃舊章;表德旌賢,有光恆策。故中書令裴炎,含弘稟粹,履信居貞,望重國華,才稱人秀。唯幾成務,績宣於代工;偶居無猜,義深於奉上。文明之際,王室多虞,保乂朕躬,實著誠節。而危疑起釁,倉卒羅災,歲月屢遷,丘封莫樹。永言先正,感悼良多。宜追賁
 於九原,俾增榮於萬古。可贈益州大都督。」炎長子彥先,後為太子舍人;從子伷先,後為工部尚書。



 劉禕之,常州晉陵人也。祖興宗,陳鄱陽王諮議參軍。父子翼,善吟諷,有學行。隋大業初,歷秘書監,河東柳顧言甚重之。性不容非,朋僚有短,面折之。友人李伯藥常稱曰:「劉四雖復罵人,人都不恨。」貞觀元年,詔追入京,以母老固辭,太宗許其終養。江南大使李襲譽嘉其至孝,恆以米帛賚之,因上表旌其門閭,改所居為孝慈里。母
 卒,服竟,徵拜吳王府功曹,再遷著作郎、弘文館直學士,預修《晉書》,加朝散大夫。永徽初卒,高宗遣使吊贈,給靈輿還鄉。有集二十卷。



 禕之少與孟利貞、高智周、郭正一俱以文藻知名,時人號為劉、孟、高、郭。尋與利貞等同直昭文館。上元中,遷左史、弘文館直學士,與著作郎元萬頃,左史範履冰、苗楚客,右史周思茂、韓楚賓等皆召入禁中,共撰《列女傳》、《臣軌》、《百僚新誡》、《樂書》,凡千餘卷。時又密令參決,以分宰相之權,時人謂之「北門學士」。禕之兄
 懿之,時為給事中,兄弟並居兩省,論者美之。



 儀鳳二年,轉朝議大夫、中書侍郎,兼豫王府司馬,尋加中大夫。禕之有姊在宮中為內職,天后令省榮國夫人之疾,禕之潛伺見之,坐是配流巂州。歷數載,天后表請高宗召還,拜中書舍人。轉相王府司馬,復遷檢校中書侍郎。高宗謂曰:「相王朕之愛子,以卿忠孝之門,藉卿師範,所冀蓬生麻中,不扶自直耳。」禕之居家孝友,甚為士族所稱,每得俸祿,散於親屬,高宗以此重之。則天臨朝,甚見親委。
 及豫王立,禕之參預其謀,擢拜中書侍郎、同中書門下三品,賜爵臨淮男。時軍國多事,所有詔敕,獨出禕之,構思敏速,皆可立待。及官名改易,禕之為鳳閣侍郎、同鳳閣鸞臺三品。



 時有司門員外郎房先敏得罪,左授衛州司馬,詣宰相陳訴。內史騫味道謂曰:「此乃皇太后處分也。」禕之謂先敏曰:「緣坐改官,例從臣下奏請。」則天聞之,以味道善則歸己,過則推君,貶青州刺史。以禕之推善於君,引過在己,加授太中大夫,賜物百段、細馬一匹。因
 謂侍臣曰:「夫為臣之體,在揚君之德,君德發揚,豈非臣下之美事?且君為元首,臣作股肱,情同休戚,義均一體。未聞以手足之疾移於腹背,而得一體安者。味道不存忠赤,已從屏退。禕之竭忠奉上,情甚可嘉。」納言王德真對曰:「昔戴至德每有善事,必推於君。」太后曰:「先朝每稱至德能有此事,逮其終歿,有制褒崇。為臣之道,豈過斯行,傳名萬代,可不善歟!」



 儀鳳中,吐蕃為邊患,高宗謂侍臣曰:「吐蕃小醜,屢犯邊境,我比務在安輯,未即誅夷。而
 戎狄豺狼,不識恩造,置之則疆場日駭,圖之則未聞上策,宜論得失,各盡所懷。」時劉景仙、郭正一、皇甫文亮、楊思徵、薛元超各有所奏。禕之時為中書舍人,對曰:「臣觀自古明王聖主,皆患夷狄。吐蕃時擾邊隅,有同禽獸,得其土地,不可攸居,被其憑凌,未足為恥。願戢萬乘之威,且寬百姓之役。」高宗嘉其言。



 後禕之嘗竊謂鳳閣舍人賈大隱曰:「太后既能廢昏立明,何用臨朝稱制?不如返政,以安天下之心。」大隱密奏其言。則天不悅,謂左右曰:「
 禕之我所引用,乃有背我之心,豈復顧我恩也!」垂拱三年,或誣告禕之受歸州都督孫萬榮金,兼與許敬宗妾有私,則天特令肅州刺史王本立推鞫其事。本立宣敕示禕之,禕之曰:「不經鳳閣鸞臺,何名為敕?」則天大怒,以為拒捍制使,乃賜死於家,時年五十七。



 初,禕之既下獄,睿宗為之抗疏申理,禕之親友咸以為必見原宥,竊賀之。禕之曰:「吾必死矣。太后臨朝獨斷,威福任己,皇帝上表,徒使速吾禍也。」禕之在獄時,嘗上疏自陳。及臨終,既
 洗沐,而神色自若,命其子執筆草謝表,其子將絕,殆不能書。監刑者促之。禕之乃自操數紙,援筆立成,詞理懇至,見者無不傷痛。時麟臺郎郭翰、太子文學周思鈞共稱嘆其文,則天聞而惡之,左遷翰為巫州司法,思鈞為播州司倉。睿宗即位,以禕之宮府舊僚,追贈中書令。有集七十卷,傳於時。



 魏玄同,定州鼓城人也。舉進士。累轉司列大夫。坐與上官儀文章屬和,配流嶺外。上元初赦還。工部尚書劉審
 禮薦玄同有時務之才,拜岐州長史。累遷至吏部侍郎。



 玄同以既委選舉,恐未盡得人之術,乃上疏曰:



 臣聞制器者必擇匠以簡材,為國者必求賢以蒞官。匠之不良,無以成其工;官之非賢,無以致於理。君者,所以牧人也;臣者,所以佐君也。君不養人,失君道矣;臣不輔君,失臣任矣。任人者,誠國家之基本,百姓之安危也。方今人不加富,盜賊不衰,獄公未清,禮義猶闕者,何也?下吏不稱職,庶官非其才也。官之不得其才者,取人之道,有所未
 盡也。臣又聞傳說曰:「明王奉若天道,建邦設都,樹後王君公,承以大夫師長,不惟逸豫,惟以理人。」昔之邦國,今之州縣,士有常君,人有定主,自求臣佐,各選英賢,其大臣乃命於王朝耳。秦並天下,罷侯置守,漢氏因之,有沿有革。諸侯得自置吏四百石以下,其傅相大官,則漢為置之。州郡掾吏、督郵從事,悉任之於牧守。爰自魏、晉,始歸吏部,遞相祖襲,以迄於今。用刀筆以量才,案簿書而察行,法令之弊,其來自久。



 蓋君子重因循而憚改作,有
 不得已者,亦當運獨見之明,定卓然之議。如今選司所行者,非上皇之令典,乃近代之權道,所宜遷徙,實為至要。何以言之?夫尺丈之量,所及者蓋短;鐘庾之器,所積者寧多。非其所及,焉能度之;非其所受,何以容之?況天下之大,士人之眾,而可委之數人之手乎?假使平如權衡,明如水鏡,力有所極,照有所窮,銓綜既多,紊失斯廣。又以比居此任,時有非人。豈直愧彼清通,昧於甄察;亦將竟其庸妄,糅彼棼絲。情故既行,何所不至?臟私一啟,
 以及萬端。至乃為人擇官,為身擇利,顧親疏而下筆,看勢要而措情。悠悠風塵,此焉奔兢;擾擾游宦,同乎市井。加以厚貌深衷,險如溪壑,擇言觀行,猶懼不周。今使百行九能,折之於一面,具僚庶品,專斷於一司,不亦難矣!且魏人應運,所據者乃三分;晉氏播遷,所臨者非一統。逮乎齊、宋,以及周、隋,戰爭之日多,安泰之時少,瓜分瓦裂,各在一方。隋氏平陳,十餘年耳,接以兵禍,繼以饑饉,既德業之不逮,或時事所未遑,非謂是今而非古也。武
 德、貞觀,與今亦異,皇運之初,庶事草創,豈唯日不暇給,亦乃人物常稀。天祚大聖,享國永年,比屋可封,異人間出。咸以為有道恥賤,得時無怠,諸色入流,歲以千計。群司列位,無復新加,官有常員,人無定限。選集之始,霧積雲屯,擢敘於終,十不收一。淄澠雜混,玉石難分,用舍去留,得失相半。撫即事之為弊,知及後之滋失。



 夏、殷已前,制度多闕,周監二代,煥乎可睹。豈諸侯之臣,不皆命於天子,王朝庶官,亦不專於一職。故周穆王以伯冏為太
 僕正,命之曰:「慎簡乃僚,無以巧言令色便僻側媚,唯吉士。」此則令其自擇下吏之文也。太僕正,中大夫耳,尚以僚屬委之,則三公九卿,亦必然矣。《周禮》:太宰、內史,並掌爵祿廢置;司徒、司馬,別掌興賢詔事。當是分任於群司,而統之以數職,各自求其小者,而王命其大者焉。夫委任責成,君之體也,所委者當,所用者精,故能得濟濟之多士,盛芃芃之棫樸。



 裴子野有言曰:「官人之難,先王言之尚矣。居家視其孝友,鄉黨服其誠信,出入觀其志義,憂
 歡取其智謀。煩之以事,以觀其能;臨之以利,以察其廉。《周禮》始於學校,論之州里,告諸六事,而後貢之王庭。其在漢家,尚猶然矣。州郡積其功能,然後為五府所闢,五府舉其掾屬而升於朝,三公參得除署,尚書奏之天子。一人之身,所關者眾;一士之進,其謀也詳。故官得其人,鮮有敗事。魏、晉反是,所失弘多。」子野所論,蓋區區之宋朝耳,猶謂不勝其弊,而況於當今乎!



 又夫從政蒞官,不可以無學。故《書》曰:「學古入官,議事以制。」《傳》曰:「我聞學以
 從政,不聞以政入學。」今貴戚子弟,例早求官,髫齔之年,已腰銀艾,或童草之歲,已襲硃紫。弘文崇賢之生,千牛輦腳之類,課試既淺,藝能亦薄,而門閥有素,資望自高。夫象賢繼父,古之道也。所謂胄子,必裁諸學,修六禮以節其性,明七教以興其德,齊八政以防其淫,舉上賢以崇德,簡不肖以黜惡。少則受業,長而出仕,並由德進,必以才升,然後可以利用賓王,移家事國。少仕則廢學,輕試則無才,於此一流,良足惜也。又勛官三衛流外之徒,
 不待州縣之舉,直取之於書判,恐非先德而後言才之義也。



 臣又以為國之用人,有似人之用財。貧者厭糟糠,思短褐;富者餘糧肉,衣輕裘。然則當衰弊乏賢之時,則可磨策朽鈍而乘馭之;在太平多士之日,亦宜妙選髦俊而任使之。《詩》云:「翹翹錯薪,言刈其楚。」楚,荊也,在薪之翹翹者。方之用才,理亦當爾,選人幸多,尤宜簡練。臣竊見制書,每令三品、五品薦士,下至九品,亦令舉人,此聖朝側席旁求之意也。但以褒貶不甚明,得失無大隔,故
 人上不憂黜責,下不盡搜揚,茍以應命,莫慎所舉。且惟賢知賢,聖人篤論,伊、皋既舉,不仁咸遠。復患階秩雖同,人才異等,身且濫進,鑒豈知人?今欲務得實才,兼宜擇其舉主。流清以源潔,影端由表正,不詳舉主之行能,而責舉人之庸濫,不可得已。《漢書》云:「張耳、陳餘之賓客、廝役,皆天下俊傑。」彼之蕞爾,猶能若斯,況以神皇之聖明,國家之德業,而不建久長之策,為無窮之基,盡得賢取士之術,而但顧望魏、晉之遺風,留意周、隋之末事,臣竊
 惑之。伏願稍回聖慮,時採芻言,略依周、漢之規,以分吏部之選。即望所用精詳,鮮于差失。



 疏奏不納。弘道初,轉文昌左丞,兼地官尚書、同中書門下三品。則天臨朝,遷太中大夫、鸞臺侍郎,依前知政事。垂拱三年,加銀青光祿大夫,檢校納言,封鉅鹿男。玄同素與裴炎結交,能保始終,時人呼為「耐久朋」。而與酷吏周興不協。永昌初,為周興所構,云玄同言:「太后老矣,須復皇嗣。」太后聞之,怒,乃賜死於家。監刑御史房濟謂玄同曰:「何不告事,冀得
 召見,當自陳訴。」玄同嘆曰:「人殺鬼殺,有何殊也,豈能為告人事乎!」乃就刑,年七十三。子恬,開元中為潁王傅。



 李昭德,京兆長安人也。父乾祐,貞觀初為殿中侍御史。時有鄃令裴仁軌私役門夫,太宗欲斬之。乾祐奏曰:「法令者,陛下制之於上,率土尊之於下,與天下共之,非陛下獨有也。仁軌犯輕罪而致極刑,是乖畫一之理。刑罰不中,則人無所措手足。臣忝憲司,不敢奉制。」太宗意解,仁軌竟免。乾祐尋遷侍御史。母卒,廬於墓側,負土成墳,
 太宗遣使就墓吊之,仍旌表其門。後歷長安令、治書御史,皆有能名,擢拜御史大夫。乾祐與中書令褚遂良不協,竟為遂良所構。永徽初,繼受邢、魏等州刺史。乾祐雖強直有器幹,而暱於小人,既典外郡,與令史結友,書疏往返,令伺朝廷之事。俄為友人所發,坐流愛州。乾封中,起為桂州都督,歷拜司刑太常伯。舉京兆功曹參軍崔擢為尚書郎,事既不果,私以告擢。後擢有犯,乃告乾祐洩禁中語以贖罪,乾祐復坐免官。尋卒。



 昭德,即乾祐之
 孽子也。強幹有父風。少舉明經,累遷至鳳閣侍郎。長壽二年,增置夏官侍郎三員,時選昭德與婁師德、侯知一為之。是歲,又遷鳳閣鸞臺平章事,尋加檢校內史。長壽中,神都改作文昌臺及定鼎、上東諸門,又城外郭,皆昭德創其制度,時人以為能。初,都城洛水天津之東,立德坊西南隅,有中橋及利涉橋,以通行李。上元中,司農卿韋機始移中橋置於安眾坊之左街,當長夏門,都人甚以為便,因廢利涉橋,所省萬計。然歲為洛水沖注,常勞
 治葺。昭德創意積石為腳,銳其前以分水勢,自是竟無漂損。



 時則天以武承嗣為文昌左相,昭德密奏曰:「承嗣,陛下之侄,又是親王,不宜更在機權,以惑眾庶。且自古帝王,父子之間猶相篡奪,況在姑侄,豈得委權與之?脫若乘便,寶位寧可安乎?」則天矍然曰:「我未之思也。」承嗣亦嘗返譖昭德,則天曰:「自我任昭德,每獲高臥,是代我勞苦,非汝所及也。」承嗣俄轉太子少保,罷知政事。延載初,鳳閣舍人張嘉福令洛陽人王慶之率輕薄惡少數
 百人詣闕上表,請立武承嗣為皇太子。則天不許,慶之固請不已,則天令昭德詰責之,令散。昭德便杖殺慶之,餘眾乃息。昭德因奏曰:「臣聞文武之道,布在方策,民有侄為天子而為姑立廟乎!以親親言之,則天皇是陛下夫也,皇嗣是陛下子也,陛下正合傳之子孫,為萬代計。況陛下承天皇顧托而有天下,若立承嗣,臣恐天皇不血食矣。」則天寤之,乃止。



 時朝廷諛佞者多獲進用,故幸恩者,事無大小,但近諂諛,皆獲進見。有人於洛水中獲
 白石數點赤,詣闕輒進。諸宰相詰之,對云:「此石赤心,所以來進。」昭德叱之曰:「此石赤心,洛水中餘石豈能盡反耶?」左右皆笑。是時,來俊臣、侯思止等枉撓刑法,誣陷忠良,人皆懾懼,昭德每廷奏其狀,由是俊臣黨與少自摧屈。來俊臣又嘗棄故妻而娶太原王慶詵女,侯思止亦奏娶趙郡李自挹女,敕政事堂共商量。昭德撫掌謂諸宰相曰:「大可笑!往年俊臣賊劫王慶詵女,已大辱國。今日此奴又請索李自挹女,無乃復辱國耶!」尋奏寢之。侯
 思止後竟為昭德所繩,搒殺之。



 既而昭德專權用事,頗為朝野所惡。前魯王府功曹參軍丘愔上疏言其罪狀曰:



 臣聞百王之失,皆由權歸於下。宰臣持政,常以勢盛為殃。魏冉誅庶族以安秦,非不忠也。弱諸候以強國,亦有功也。然以出入自專,擊斷無忌,威震人主,不聞有王,張祿一進深言,卒用憂死。向使昭王不即覺悟,魏冉果以專權,則秦之霸業,或不傳其子孫。陛下創業興王,撥亂英主,總權收柄,司契握圖。天授已前,萬機獨斷,發命
 皆中,舉事無遺,公卿百僚,具職而已。自長壽已來,厭怠細政,委任昭德,使掌機權。然其幹濟小才,不堪軍國大用。直以性好凌轢,氣負剛強,盲聾下人,芻狗同列,刻薄慶賞,矯枉憲章,國家所賴者微,所妨者大。天下杜口,莫敢一言,聲威翕赫,日已熾盛。臣近於南臺見敕日,諸處奏事,陛下已依,昭德請不依,陛下便不依。如此改張,不可勝數。昭德參奉機密,獻可替否,事有便利,不預諮謀,要待畫旨將行,方始別生駁異。揚露專擅,顯示於人,歸
 美引愆,義不如此。州縣列位,臺寺庶官,入謁出辭,望塵習氣。一切奏讞,與奪事宜,皆承旨意,附會上言。今有秩之吏,多為昭德之人。陛下勿謂昭德小心,是我手臂。臣觀其膽,乃大於身,鼻息所沖,上拂雲漢。近者新陷來、張兩族,兼挫侯、王二仇,鋒銳理不可當,方寸良難窺測。書曰:知人亦未易,人亦未易知。漢光武將寵龐萌,可以托孤,卒為戎首。魏明帝期司馬懿以安國,竟肆奸回。夫小家治生,有千百之資,將以托人,尚憂失授。況兼天下之
 重,而可輕忽委任者乎!今昭德作福專威,橫絕朝野,愛憎與奪,旁若無人。陛下恩遇至深,蔽過甚厚。臣聞蟻穴壞堤,針芒寫氣,涓涓不絕,必成江河。履霜堅冰,須防其漸,權重一去,收之極難。臣又聞輕議近臣,犯顏深諫,明君聖主,亦有不容。臣熟知今日言之於前,明日伏誅於後。但使國安身死,臣實不悔。陛下深覽臣言,為萬姓自愛。」



 時長上果毅鄧注又著《碩論》數千言,備述昭德專權之狀,鳳閣舍人逢弘敏遽奏其論。則天乃惡昭德,謂納
 言姚璹曰:「昭德身為內史,備荷殊榮,誠如所言,實負於國。」延載初,左遷欽州南賓尉,數日,又命免死配流。尋又召拜監察御史。時太僕少卿來俊臣與昭德素不協,乃誣構昭德有逆謀,因被下獄,與來俊臣同日而誅。是日大雨,士庶莫不痛昭德而慶俊臣也。相謂曰:「今日天雨,可謂一悲一喜矣。」神龍中,降制曰:「故李昭德勤恪在公,強直自達。立朝正色,不吐剛以茹柔;當軸勵詞,必抗情以歷詆。墉隍府寺,樹勣良多,變更規模,歿而不朽。道淪
 福善,業虧嫉惡,名級不追,風流將沫。式旌壞樹,光被幽明,可贈左御史大夫。」德宗建中三年,加贈司空。



 史臣曰:裴炎位居相輔,時屬艱難,歷覽前蹤,非無忠節。但見遲而慮淺,又遭命以會時。何者,當是時,高宗晏駕尚新,武氏革命未見,炎也唯慮中宗之過失,是其淺也;不見太后之苞藏。是其遲也。及乎承嗣請封祖禰,三思勸殺宗親,然後徒有諫章,何嘗濟事,是辜遺托,豈痛伏誅。時論則然,遲淺須信。況聞睹構逆則示其閑暇,俾殺
 降則彰彼猜嫌,小數有餘,大度何足,又其驗也。



 禕之名父之子,諒知其才,著述頗精,履歷無愧。師範王府,秉執相權,咸有能名,固愜群議。何乃失言於大隱,取金於萬榮,潛見內人,私通嬖妾,使濁跡玷其清譽,淫行污於貞名。若言俗困濫刑,公行誣告,即又自昧周防之道,人非盡戮之冤。賜死於家,猶為多幸,臨終不撓,抑又徒勞。



 玄同富於詞學,公任權衡,當為典選之時,備疏擇才之理。但以高宗棄代之後,則天居位之間,革命是懷,附己為
 愛,茍一言之不順,則赤族以難逃。是以唐之名臣,難忘中興之計;周之酷吏,常謀並進之讒。玄同欲復皇儲,固宜難免,死而無過,人殺何妨。



 昭德強幹為臣,機巧蒞事,凡所制置,動有規模。武承嗣方持左相權,將立為皇太子,尋更所任,復寢其謀,咸由昭德之言,能拒則天之旨。又觀其誅侯思止,法王慶之,挫來俊臣,致朋黨漸衰,諛佞稍退。又則天謂承嗣曰:「我任昭德,每獲高臥,代我勞苦,非汝所及也。」此則強幹機巧之驗焉。公忠之道,亦在
 其中矣。不然,則何以致是哉!若使昭德用謙御下,以柔守剛,不恃專權,常能寡過,則復皇嗣而非晚,保臣節而必終。蓋由道乏弘持,器難苞貯,純剛是失,卷智不全。所以丘愔抗陳,鄧注深論,瓦解而固難收拾,風摧而豈易扶持。自取誅夷,人誰怨懟?



 贊曰:政無刑法,時屬艱危。裴炎之智,慮淺見遲。禕之履行,貸色自欺。昭德強猛,何由不虧?死無令譽,孰謂非宜。玄同不幸,顛殞亦隨。



\end{pinyinscope}