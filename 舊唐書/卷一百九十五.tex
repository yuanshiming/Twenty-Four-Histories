\article{卷一百九十五}

\begin{pinyinscope}

 ○李知本
 張志寬劉君良宋興貴張公藝附王君操周智壽智爽許坦王少玄附趙弘智陳集原元讓裴敬彞裴守真子子餘李日
 知崔沔陸南金弟趙璧張琇兄瑝梁文貞李處恭張義貞呂元簡附崔衍丁公著羅讓



 善父母為孝,善兄弟為友。夫善於父母,必能隱身錫類,仁惠逮於胤嗣矣;善於兄弟,必能因心廣濟,德信被於宗族矣!推而言之,可以移於君,施於有政,承上而順下,令終而善始,雖蠻貊猶行焉,雖窘迫猶亨焉!自昔立身揚名,未有不偕孝友而成者也。前代史官,所傳《孝友傳》,
 多錄當時旌表之士,人或微細,非眾所聞,事出閭里,又難詳究。今錄衣冠盛德,眾所知者,以為稱首。至於州縣薦飾者,必覆其殊尤,可以勸世者,亦載之。



 李知本,趙州元氏人,後魏洛州刺史靈六世孫也。父孝端,隋獲嘉丞。初,孝端與族弟太沖,俱有世閥,而太沖官宦最高,孝端方之為劣。鄉族為之語,曰:「太沖無兄,孝端無弟。」知本頗涉經史,事親至孝,與弟知隱甚稱雍睦。子孫百餘口,財物僮僕,纖毫無間。隋末,盜賊過其閭而不
 入,因相讓曰:「無犯義門。」同時避難者五百餘家,皆賴而獲免。



 知本貞觀初官至夏津令,知隱至伊闕丞。知本孫瑱,開元中為給事中、揚州刺史。知隱孫顒,有文詞,亦歷給事中、太常少卿。從祖兄弟,凡為給事者四人。



 張志寬,蒲州安邑人。隋末喪父,哀毀骨立,為州里所稱。賊帥王君廓屢為寇掠,聞其名,獨不犯其閭,鄰里賴之而免者百餘家。後為里正,詣縣稱母疾,急求歸。縣令問其狀,對曰:「母嘗有所苦,志寬亦有所苦。向患心痛,知母
 有疾。」令怒曰:「妖妄之辭也!」系之於獄。馳驗其母,竟如所言。令異之,慰喻遣去。



 及丁母憂,負土成墳,廬於墓側,手植松柏千餘株。高祖聞之,遣使就吊,授員外散騎常侍,賜物四十段,表其門閭。



 劉君良,瀛州饒陽人也。累代義居,兄弟雖至四從,皆如同氣,尺布斗粟,人無私焉。大業末,天下饑饉,君良妻勸其分析,乃竊取庭樹上鳥鶵,交置諸巢中,令群鳥鬥競。舉家怪之,其妻曰:「方今天下大亂,爭鬥之秋,禽鳥尚不
 能相容,況於人乎!」君良從之。分別後月餘,方知其計。中夜,遂攬妻發大呼曰:「此即破家賊耳!」召諸昆弟,哭以告之。是夜棄其妻,更與諸兄弟同居處,情契如初。



 屬盜起,閭里依之為堡者數百家,因名為義成堡。武德七年,深州別駕楊弘業造其第,見有六院,唯一飼,子弟數十人,皆有禮節,咨嗟而去。貞觀六年,詔加旌表。



 又有宋興貴者,雍州萬年人。累世同居,躬耕致養,至興貴已四從矣。高祖聞而嘉之,武德二年,詔曰:



 人稟五常,仁義為重;士
 有百行,孝敬為先。自古哲王,經邦致治,設教垂範,皆尚於斯。叔世澆訛,人多偽薄,修身克己,事資誘勸。朕恭膺靈命,撫臨四海,愍茲弊俗,方思遷導。宋興貴立操雍和,志情友穆,同居合爨,累代積年,務本力農,崇謙履順。弘長名教,敦勵風俗,宜加褒顯,以勸將來。可表其門閭,蠲免課役。布告天下,使明知之。



 興貴尋卒。



 鄆州壽張人張公藝,九代同居。北齊時,東安王高永樂詣宅慰撫旌表焉。隋開皇中,大使、邵陽公梁子恭亦親慰撫,重表其門。
 貞觀中,特敕吏加旌表。麟德中,高宗有事泰山,路過鄆州,親幸其宅,問其義由。其人請紙筆,但書百餘「忍」字。高宗為之流涕,賜以縑帛。



 王君操,萊州即墨人也。其父隋大業中與鄉人李君則鬥競,因被毆殺。君操時年六歲,其母劉氏告縣收捕,君則棄家亡命,追訪數年弗獲。貞觀初,君則自以世代遷革,不慮國刑,又見君操孤微,謂其無復仇之志,遂詣州府自首。而君操密袖白刃刺殺之,刳腹取其心肝,啖食
 立盡,詣刺史具自陳告。州司以其擅殺戮,問曰:「殺人償死,律有明文,何方自理,以求生路?」對曰:「亡父被殺,二十餘載。聞諸典禮,父仇不可同天。早願圖之,久而未遂,常懼亡滅,不展冤情。今大恥既雪,甘從刑憲。」州司據法處死,列上其狀,太宗特詔原免。



 周智壽者,雍州同官人。其父永徽初被族人安吉所害。智壽及弟智爽乃候安吉於途,擊殺之。兄弟相率歸罪於縣,爭為謀首,官司經數年不能決。鄉人或證智爽先
 謀,竟伏誅。臨刑神色自若,顧謂市人曰:「父仇已報,死亦何恨!」智壽頓絕衢路,流血遍體。又收智爽尸,舐取智爽血,食之皆盡,見者莫不傷焉。



 豫州人許坦,年十歲餘,父入山採藥,為猛獸所噬,即號叫以杖擊之,獸遂奔走,父以得全。太宗聞而謂侍臣曰:「坦雖幼童,遂能致命救親,至孝自中,深可嘉尚。」授文林郎,賜帛五十段。



 博州聊城人王少玄者,父隋末於郡西為亂兵所害。少玄遺腹生,年十餘歲,問父所在。其母告之,因哀泣,便欲求尸以葬。
 時白骨蔽野,無由可辨。或曰:「以子血霑父骨,即滲入焉。」少玄乃刺其體以試之。凡經旬日,竟獲父骸以葬。盡體病瘡,歷年方愈。貞觀中,本州聞薦,拜除王府參軍。



 趙弘智,洛州新安人。後魏車騎大將軍肅孫。父玄軌,隋陜州刺史。弘智早喪母,事父以孝聞。學通《三禮》、《史記》、《漢書》。隋大業中,為司隸從事。武德初,大理卿郎楚之應詔舉之,授詹事府主簿。又預修《六代史》。



 初,與秘書丞令狐德棻、齊王文學袁朗等十數人同修《藝文類聚》,轉太子
 舍人。貞觀中,累遷黃門侍郎,兼弘文館學士。以疾出為萊州刺史。弘智事兄弘安,同於事父,所得俸祿,皆送於兄處。及兄亡,哀毀過禮。事寡嫂甚謹,撫孤侄以慈愛稱。稍遷太子右庶子。及宮廢,坐除名。尋起為光州刺史。



 永徽初,累轉陳王師。高宗令弘智於百福殿講《孝經》,召中書門下三品及弘文館學士、太學儒者,並預講筵。弘智演暢微言,備陳五孝。學士等難問相繼,弘智酬應如響。高宗怡然曰:「朕頗耽墳籍,至於《孝經》,偏所習睹。然孝之
 為德,弘益實深,故云『德教加於百姓,刑於四海,是知孝道之為大也。」顧謂弘智:「宜略陳此經切要者,以輔不逮。」弘智對曰:「昔者天子有諍臣七人,雖無道不失其天下。微臣顓愚,願以此言奏獻。」帝甚悅,賜彩絹二百匹、名馬一匹。尋遷國子祭酒,仍為崇賢館學士。四年卒,年八十二,謚曰宣。有文集二十卷。



 陳集原,瀧州開陽人也。代為嶺表酋長。父龍樹,欽州刺史。集原幼有孝行,父才有疾,即終日不食。永徽中,喪父,
 嘔血數升,枕服苫廬,悲感行路。資財田宅及僮僕三十餘人,並以讓兄弟。則天時,官至左豹韜衛將軍。



 元讓,雍州武功人也。弱冠明經擢第。以母疾,遂不求仕。躬親藥膳,承侍致養,不出閭里者數十餘年。及母終,廬於墓側,蓬發不櫛沐,菜食飲水而已。



 咸亨中,孝敬監國,下令表其門閭。永淳元年,巡察使奏讓孝悌殊異,擢拜太子右內率府長史。後以歲滿還鄉里。鄉人有所爭訟,不詣州縣,皆就讓決焉。聖歷中,中宗居春宮,召拜太子
 司議郎。及謁見,則天謂曰:「卿既能孝於家,必能忠於國。今授此職,須知朕意。宜以孝道輔弼我兒。」尋卒。



 裴敬彞,絳州聞喜人也。曾祖子通,隋開皇中太中大夫。母終,廬於墓側,哭泣無節,目遂喪明。俄有白鳥巢於墳樹。子通弟兄八人,復以友悌著名,詔旌表其門,鄉人至今稱為「義門裴氏」。



 敬彞少聰敏,七歲解屬文。性又端謹,宗族咸重之,號為「甘露頂」。年十四,侍御史唐臨為河北巡察使,敬彞父智周時為內黃令,為部人所訟,敬彞詣
 臨論其冤。臨大奇之,因令作詞賦。智周事得釋,特表薦敬彞,補陳王府典簽。智周在官暴卒,敬彞時在長安,忽泣涕不食,謂所親曰:「大人每有痛處,吾即輒然不安。今日心痛,手足皆廢,事在不測,得無戚乎?」遂請急還,倍道言歸。果聞父喪,羸毀逾禮。事母復以孝聞。



 乾封初,累轉監察御史。時母病,有醫人許仁則,足疾不能乘馬,敬彞每肩輿之以候母焉。及母卒,特詔贈以縑帛,仍官造靈輿。服闋,拜著作郎,兼修國史。儀鳳中,自中書舍人歷吏
 部侍郎、左庶子。則天臨朝,為酷吏所陷,配流嶺南,尋卒。



 裴守真,絳州稷山人也。後魏冀州刺史叔業六世孫也。父慎,大業中為淮南郡司戶。屬郡人楊琳、田瓚據郡作亂,盡殺官吏。以慎素有仁政,相誡不許驚害,仍令人護送慎及妻子還鄉。貞觀中,官至酂令。



 守真早孤,事母至孝。及母終,哀毀骨立,殆不勝喪。復事寡姊及兄甚謹,閨門禮則,士友所推。初舉進士,及應八科舉,累轉乾封尉,屬永淳初關中大饑,守真盡以祿俸供姊及諸甥,身及
 妻子粗糲不充,初無倦色。尋授太常博士。



 守真尤善禮儀之學,當時以為稱職。高宗時封嵩山,詔禮官議射牲之事,守真奏曰:



 據《周禮》及《國語》,郊祀天地,天子自射其牲。漢武唯封太山,令侍中儒者射牲行事。至於餘祀,亦無射牲之文。但親舂射牲,雖是古禮,久從廢省。據封禪祀禮曰:未明十五刻,宰人以鸞刀割牲,質明而行事。比鸞駕至時,宰牲總畢,天皇唯奠玉酌獻而已。今祀前一日射牲,事即傷早;祀日方始射牲,事又傷晚。若依漢武
 故事,即非親射之儀,事不可行。



 又《神功破陣樂》、《功成慶善樂》二舞,每奏,上皆立對。守真又議曰:



 竊唯二舞肇興,謳吟攸屬,贊九功之茂烈,葉萬國之歡心。義均《韶》、《夏》,用兼賓祭,皆祖宗盛德,而子孫享之。詳覽傳記,未有皇王立觀之禮。況升中大事,華夷畢集,九服仰垂拱之安,百蠻懷率舞之慶。甄陶化育,莫匪神功,豈於樂舞,別申嚴敬。臣等詳議,奏二舞時,天皇不合起立。



 時並從守真議。會高宗不豫,事竟不行。及高宗崩,時無大行兇儀,守真
 與同時博士韋叔夏、輔抱素等討論舊事創為之,當時稱為得禮之中。



 守真天授中為司府丞,則天特令推究詔獄,務存平恕,前後奏免數十家。由是不合旨,出為汴州司錄,累轉成州刺史。為政不務威刑,甚為人吏所愛。俄轉寧州刺史,成州人送出境者數千人。長安中卒。



 子子餘,事繼母以孝聞。舉明經,累補鄠縣尉。時同列李朝隱、程行諶皆以文法著稱,子餘獨以詞學知名。



 或問雍州長史陳崇業,子餘與朝隱、行諶優劣,崇業曰:「譬如春
 蘭秋菊,俱不可廢也。」景龍中,為左臺監察御史。時涇、岐二州有隋代蕃戶子孫數千家,司農卿趙履溫奏,悉沒為官戶奴婢,仍充賜口,以給貴幸。子餘以為官戶承恩,始為蕃戶,又是子孫,不可抑之為賤,奏劾其事。時履溫依附宗楚客等,與子餘廷對曲直。子餘詞色不撓,履溫等詞屈,從子餘奏為定。



 開元初,累遷冀州刺史。政存寬惠,人吏稱之。又為岐王府長史,加銀青光祿大夫。十四年卒,謚曰孝。子餘居官清儉,友愛諸兄弟。



 兄弟六人,皆
 有志行。次弟巨卿,衛尉卿;耀卿,別有傳。



 李日知,鄭州滎陽人也。舉進士。天授中,累遷司刑丞。時用法嚴急,日知獨寬平,無冤濫。嘗免一死囚,少卿胡元禮請斷殺之,與日知往復至於數四。元禮怒,曰:「元禮不離刑曹,此囚終無生理。」答曰:「日知不離刑曹,此囚終無死法。」因以兩狀列上,日知果直。



 神龍初,為給事中。日知事母至孝。時母老,嘗疾病,日知取急,調侍數日而鬢發變白。尋加朝散大夫。其母未受命婦邑號而卒,將葬發
 引,吏人齎告身而至,日知於路上即時殞絕,久之乃蘇。左右皆哀慟,莫能仰視。巡察使、衛州司馬路敬潛將聞其孝悌之跡,使求其狀,日知辭讓不報。服闋,累遷黃門侍郎。



 時安樂公主池館新成,中宗親往臨幸,從官皆預宴賦詩。日知獨存規誡,其末章曰:「所願暫思居者逸,莫使時稱作者勞。」論者多之。



 景雲元年,同中書門下平章事,轉御史大夫,知政事如故。明年,進拜侍中。先天元年,轉刑部尚書,罷知政事。頻乞骸骨,請致仕,許之。



 初,日知
 將有陳請,而不與妻謀,歸家而使左右飾裝,將出居別業。妻驚曰:「家產屢空,子弟名宦未立,何為遽辭職也?」日知曰:「書生至此,已過本分。人情無厭,若恣其心,是無止足之日。」及歸田園,不事產業,但葺構池亭,多引後進,與之談宴。開元三年卒。



 初,日知以官在權要,諸子弟年才總角,皆結婚名族,時議以為失禮之中。卒後,少子伊衡,以妾為妻,費散田宅,仍列訟諸兄,家風替矣。



 崔沔,京兆長安人,周隴州刺史士約玄孫也。自博陵徙
 關中,世為著姓。父皚,庫部員外郎、汝州長史。沔淳謹,口無二言,事親至孝,博學有文詞。初應制舉,對策高第。俄被落第者所援,則天令所司重試,沔所對策,又工於前,為天下第一,由是大知名。再轉陸渾主簿。秩滿調遷,吏部侍郎岑羲深賞重之,謂人曰:「此今之卻詵也。」特表薦擢為左補闕,累遷祠部員外郎。沔為人舒緩,訥於造次,當官正色,未嘗撓沮。



 睿宗時,徵拜中書舍人。時沔母老疾在東都,沔不忍舍之,固請閑官,以申侍養,由是改為
 虞部郎中。無何,檢校御史中丞。時監察御史宋宣遠,恃盧懷慎之親,頗犯法,沔舉劾之。又姚崇之子光祿少卿彞,留司東都,頗通賓客,廣納賄賂,沔又將按驗其事。姚、盧時在政事,遽薦沔有史才,轉為著作郎,其實去權也。



 開元七年,為太子左庶子。母卒,哀毀逾禮,常於廬前受吊,賓客未嘗至於靈座之室,謂人曰:「平生非至親者,未嘗升堂入謁,豈可以存亡而變其禮也。」中書令張說數稱薦之。服闋,拜中書侍郎。或謂沔曰:「今之中書,皆是宰
 相承宣制命。侍郎雖是副貳,但署位而已,甚無事也。」沔曰:「不然。設官分職,上下相維,各申所見,方為濟理。豈可俯默偷安,而為懷祿士也!」自是每有制敕及曹事,沔多所異同,張說頗不悅焉。尋出為魏州刺史,奏課第一,徵還朝廷,分掌吏部十銓事。以清直,歷秘書監、太子賓客。



 二十四年,制令禮官議加籩豆之數及服制之紀。太常卿韋縚奏請加宗廟之奠,每坐籩豆各十二。外祖服,請加至大功九月,舅服加至小功五月,堂姨、堂舅、舅母服,
 請加至袒免。時又令百官詳議可否。沔建議曰:



 竊聞識禮樂之情者能作,達禮樂之文者能述。述作之義,聖賢所重;禮樂之本,古今所崇。變而通之,所以久也。所謂變者,變其文也;所謂通者,通其情也。祭祀之興,肇於太古,人所飲食,必先嚴獻。未有火化,茹毛飲血,則有毛血之薦;未有曲糵,污樽抔飲,則有玄酒之奠。施及後王,禮物漸備,作為酒醴,伏其犧牲,以致馨香,以極豐潔,故有三牲八簋之盛,五齊九獻之殷。然以神道至玄,可存而不
 可測也;祭禮主敬,可備而不敢廢也。是以血腥爛熟,玄樽犧象,靡不畢登於明薦矣!



 然而薦貴於新,味不尚褻,雖則備物,猶存節制。故《禮》云:「天之所生,地之所長,茍可薦者,莫不咸在。」備物之情也。「三牲之俎,八簋之實,美物備矣;昆蟲之異,草木之實,陰陽之物備矣。」此則節制之文也。鉶俎、籩豆、簠簋、樽罍之實,皆周人之時饌也,其用通於宴饗賓客。而周公制禮,咸與毛血玄酒同薦於先。晉中郎盧諶,近古之知禮、著《家祭禮》者也。觀其所薦,皆
 晉時常食,不復純用禮經舊文。然則當時飲食,不可闕於祭祀明矣,是變禮文而通其情也!



 我國家由禮立訓,因時制範,考圖史於前典,稽周、漢之舊儀。清廟時享,禮饌畢陳,用周制也,而古式存焉;園寢上食,時膳具設,遵漢法也,而珍味極焉。職貢來祭,致遠物也;有新必薦,順時令也。苑囿之內,躬稼所收,蒐狩之時,親發所中,莫不割鮮擇美,薦而後食,盡誠敬也。若此至矣,復何加焉!但當申敕有司,祭如神在,無或簡怠,勖增虔誠。其進貢珍
 羞,或時物鮮美,考諸祠典,無有漏落。皆詳名目,編諸甲令,因宜而薦,以類相從。則新鮮肥濃,盡在是矣,不必加於籩豆之數也。至於祭器,隨物所宜。故大羹,古食也,盛於。,古器也;和羹,時饌也。盛於鉶。鉶,時器也。亦有古饌而盛於時器,故毛血盛於盤,玄酒盛於樽。未有薦時饌而追用古器者,由古質而今文,便於事也。雖加籩豆十二,未足以盡天下美物,而措諸清廟,有兼倍之名,近於侈矣!魯人丹桓宮之楹,又刻其桷,《春秋》書以「非禮」。御
 孫諫曰:「儉,德之恭也;侈,惡之大也。先君有恭德,而君納諸惡,無乃不可乎!」是不可以越禮而崇侈於宗廟也。又據《漢書·藝文志》:「墨家之流,出於清廟,是以貴儉」。由此觀之,清廟之不尚於奢,舊矣。太常所請,恐未可行。



 又按太常奏狀:「今酌獻酒爵,制度全小,僅未一合,執持甚難,不可全依古制,猶望稍須廣大。」竊據禮文,有以小為貴者,獻以爵,貴其小也。小不及制,敬而非禮,是有司之失其傳也。固可隨失厘正,無待議而後革。然禮失於敬,猶奢
 而寧儉,非大過也。未知今制,何所依準。請兼詳令式,據文而行。



 又按太常奏狀「外祖服請加至大功九月,舅服請加至小功五月,堂姨、堂舅、舅母請加至袒免」者。竊聞大道既隱,天下為家,聖人因之,然後制禮。禮教之設,本於正家,家道正而天下定矣!正家之道,不可以貳;總一之義,理歸本宗。所以父以尊崇,母以厭降,豈亡愛敬,宜存倫序。是以內有齊斬,外服皆緦,尊名所加,不過一等,此先王不易之道。前聖所志,後賢所傳,其來久矣。昔辛
 有適伊川,見被發而祭於野者,曰:「不及百年,此其戎乎!其禮先亡矣!」往修新禮,時改舊章,漸廣《渭陽》之恩,不遵洙、泗之典。及弘道之後,唐元之間,國命再移於外族矣。禮亡徵兆,倘或斯見,天人之際,可不戒哉!



 開元初,補闕盧履冰嘗進狀論喪服輕重,敕令僉議。於時群議紛挐,各安積習,太常禮部奏依舊定。陛下運稽古之明,特降別敕,一依古禮。事符典故,人知向方,式固宗盟,社稷之福。更圖異議,竊所未詳。



 時職方郎中韋述、戶部郎中楊
 伯成、禮部員外郎楊沖昌、監門兵曹劉秩等,亦建議與沔相符。俄又令中書門下參詳為定。於是宗廟之典,籩豆每座各加至六,親姨、舅為小功,舅母加緦麻,堂姨至袒免,餘依舊定,乃下制施行焉。沔既善禮經,朝廷每有疑議,皆取決焉。二十七年卒,時年六十七,贈禮部尚書。



 陸南金,蘇州吳郡人也。祖士季,從同郡顧野王學《左氏傳》,兼通《史記》、《漢書》。隋末,為越王侗記室兼侍讀。侗稱制,授著作郎。時王世充將行篡奪,侗不平之,謂士季曰:「隋
 有天下,三十餘載,朝廷文武,遂無烈者乎?」士季對曰:「見危授命,臣之宿心。請因其啟事,便加手刃。」事頗洩,遂停士季侍讀。



 貞觀初,為太學博士,兼弘文館學士,尋卒。



 南金初為奉禮郎。開元初,太常少卿盧崇道犯罪,流嶺表,逃歸東都。時南金以母喪在家,崇道事急,假稱吊賓,造南金,言其情,南金哀而納焉。崇道俄為仇人所發,詔使侍御史王旭按其事,遂捕獲崇道,連引南金,旭遂繩以重法。



 南金弟趙璧詣旭,自言藏崇道,請代兄死。南金固
 稱:「弟實自誣,身請當罪。」兄弟讓死,旭怪而問其故。趙璧曰:「兄是長嫡,又能幹家事。亡母未葬,小妹未嫁,自惟幼劣,生無所益,身自請死。」旭遂列上狀,上嘉其友義,並特宥之。南金由是大知名。



 南金頗涉經史,言行修謹,左丞相張說及宗人太子少保象先,皆欽重之。累轉庫部員外郎,以疾,固辭不堪繁劇,轉為太子洗馬。卒,年五十餘。



 張琇者,蒲州解人也。父審素,為巂州都督,在邊累載。俄有糾其軍中贓罪,敕監察御史楊汪馳傳就軍按之。汪
 在路,為審素黨與所劫,對汪殺告事者,脅汪令奏雪審素之罪。俄而州人翻殺審素之黨,汪始得還。至益州,奏稱審素謀反,因深按審素,構成其罪。斬之,籍沒其家。琇與兄瑝,以年幼坐徙嶺外。尋各逃歸,累年隱匿。汪後累轉殿中侍御史,改名萬頃。



 開元二十三年,瑝、琇候萬頃於都城,挺刃殺之。瑝雖年長,其發謀及手刃,皆琇為之。既殺萬頃,系表於斧刃,自言報仇之狀。便逃奔,將就江外,殺與萬頃同謀構父罪者。行至汜水,為捕者所獲。時
 都城士女,皆矜琇等幼稚孝烈,能復父仇,多言其合矜恕者。中書令張九齡又欲活之。



 裴耀卿、李林甫固言:「國法不可縱報仇。」上以為然,因謂九齡等曰:「復仇雖禮法所許,殺人亦格律具存。孝子之情,義不顧命,國家設法,焉得容此!殺之成復仇之志,赦之虧律格之條。然道路誼議,故須告示。」乃下敕曰:「張瑝等兄弟同殺,推問款承。律有正條,俱各至死。近聞士庶,頗有誼詞,矜其為父復仇,或言本罪冤濫。但國家設法,事在經久,蓋以濟人,期
 於止殺。各申為子之志,誰非徇孝之夫,展轉相繼,相殺何限!咎由作士,法在必行;曾參殺人,亦不可恕。不能加以刑戮,肆諸市朝,宜付河南府告示決殺。」



 瑝、琇既死,士庶咸傷愍之,為作哀誄,榜於衢路。市人斂錢,於死所造義井,並葬瑝、琇於北邙。又恐萬頃家人發之,並作疑塚數所。其為時人所傷如此。



 梁文貞,虢州閿鄉人。少從征役,比回而父母皆卒。文貞恨不獲終養,乃穿壙為門,磴道出入,晨夕灑掃其中。結
 廬墓側,未嘗暫離。自是不言三十年,家人有所問,但畫字以對。其後山水沖斷驛路,更於原上開道,經文貞墓前。由是行旅見之,遠近莫不欽嘆。有甘露降塋前樹,白兔馴擾,鄉人以為孝感所致。



 開元初,縣令崔季友刊石以紀之。十四年,刺史許景先奏:「文貞孝行絕倫,泣血廬墓,三十餘年,請宣付史官。」是歲,御史大夫崔隱甫廷奏:「恆州鹿泉人李處恭、張義貞兩家,祖父自國初已來,異姓同居,至今三代,百有餘年。又青州北海人呂元簡,四
 代同居,至所畜牛馬羊狗,皆異母共乳。請加旌表,仍編入史館。」制皆許之。



 崔衍,左丞倫之子。繼母李氏,不慈於衍。衍時為富平尉,倫使於吐蕃,久方歸,李氏衣弊衣以見倫。倫問其故,李氏稱:「自倫使於蕃中,衍不給衣食。」倫大怒,召衍責詬,命僕隸拉於地,袒其背,將鞭之。衍涕泣,終不自陳。倫弟殷,聞之趨往,以身蔽衍,杖不得下。因大言曰:「衍每月俸錢,皆送嫂處,殷所具知,何忍乃言衍不給衣食!」倫怒乃解。
 由是倫遂不聽李氏之譖。及倫卒,衍事李氏益謹。李氏所生子郃,每多取子母錢,使其主以契書徵負於衍。衍歲為償之,故衍官至江州刺史,而妻子衣食無所餘。



 後歷蘇、虢二州刺史。虢居陜、華二州之間,而稅重數倍。其青苗錢,華、陜之郊,畝出十有八;而虢之郊,每征十之七。衍乃上其事。時裴延齡領度支,方務聚斂,乃紿衍以前後刺史無言者。衍又上陳人困,曰:「臣所治多是山田,且當郵傳沖要,屬歲不登,頗甚流離。舊額賦租,特望蠲減。
 臣伏見比來諸郡論百姓間事,患在長吏因循不為申請,不詣實,不患朝廷不矜放。有以不言受譴者,未有言而獲罪者。陛下拔臣牧大郡,委臣撫疲民,臣所以不敢顧望,茍求自安,敢罄狂瞽,上干聖覽。」帝以衍詞理切直,乃特敕度支,令減虢州青苗錢。



 遷宣歙池觀察使,政務簡便,人頗懷之。其所擇從事,多得名流。時有位者待賓僚率輕傲,衍獨加禮敬,幕中之士,後多顯達。



 貞元中,天下好進奉以結主恩,徵求聚斂,州郡耗竭,韋皋、劉贊、裴
 肅為之首。贊死而衍代其位。衍雖不能驟革其弊,居宣州十年,頗勤儉,府庫盈溢。及穆贊代衍,宣州歲饉,遂以錢四十二萬貫代百姓稅,故宣州人不至流散。貞元二十一年,詔加工部尚書。



 丁公著,字平子,蘇州吳郡人。祖衷,父緒,皆不仕。公著生三歲,喪所親。七歲,見鄰母抱其子,哀感不食,因請於父,絕粒奉道,冀其幽贊,父憫而從之。年十七,父勉令就學。年二十一,《五經》及第。明年,又通《開元禮》,授集賢校書郎。
 秩未終,歸侍鄉里,不應請闢。居父喪,躬負土成墳,哀毀之容,人為憂之,裏閭聞風,皆敦孝悌。觀察使薛華表其行,詔賜粟帛,旌其門閭。



 淮南節度使李吉甫慕其才行,薦授太子文學,兼集賢殿校理。吉甫自淮南入相,廷薦其行,即日授右補闕。遷集賢直學士,尋授水部員外郎,充皇太子及諸王侍讀。著《皇太子及諸王訓》十卷。轉駕部員外,仍兼舊職。



 穆宗即位,未及聽政,召居禁中,詢訪朝典,以宰相許之。公著陳情,詞意極切,超授給事中,賜
 紫金魚袋。未幾,遷工部侍郎,仍兼集賢殿學士,寵青宮之舊也。知吏部選事。公著知將欲大用,以疾辭退,因求外官,遂授浙江西道都團練觀察使。二年,授河南尹。皆以清靜為理。改尚書右丞,轉兵部、吏部侍郎,遷禮部尚書、翰林侍講學士。上以浙西災寇,詢求良帥,命檢校戶部尚書領之。詔賜米七萬碩以賑給,浙民賴之。改授太常卿,以疾請歸鄉里,未至而終,年六十四。贈右僕射,廢朝一日。著《禮志》十卷。



 公著清儉守道,每得一官,未嘗不
 憂色滿容。年四十四喪室,以至終身,無妓妾聲樂之好。兇問至日,中外痛惜之。



 羅讓,字景宣。祖懷操。父珦,官至京兆尹。讓少以文學知名,舉進士,應詔對策高等,為咸陽尉。丁父憂,服除,尚衣麻茹菜,不從四方之闢者十餘年。李獻為淮南節度使,就其所居,請為從事。除監察御史,轉殿中,歷尚書郎、給事中,累遷至福建觀察使、兼御史中丞,甚著仁惠。有以女奴遺讓者,讓問其所因,曰:「本某寺家人。兄姊九人,皆
 為官所賣,其留者唯老母耳。」讓慘然,焚其券書,以女奴歸其母。入為散騎常侍。未幾,除江西都團練觀察使、兼御史大夫。年七十一卒。贈禮部尚書。



 子劭京,字子峻,進士擢第,又登科。讓再從弟詠。詠子劭權,字昭衡,進士擢第。劭京、劭權知名於時,並歷清貫。



 贊曰:麒麟鳳凰,飛走之類。唯孝與悌,亦為人瑞。表門賜爵,勸乃錫類。彼禽者梟,傷仁害
 義。



\end{pinyinscope}