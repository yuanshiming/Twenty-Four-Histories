\article{卷一百七十五}

\begin{pinyinscope}

 ○李渤
 張仲方裴潾張皋附李中敏李甘高元裕兄少逸李漢李景儉



 李渤,字浚之,後魏橫野將軍申國公發之後。祖玄珪,衛
 尉寺主簿。父鈞,殿中侍御史,以母喪不時舉,流於施州。渤恥其家污,堅苦不仕;勵志於文學,不從科舉,隱於嵩山,以讀書業文為事。



 元和初,戶部侍郎鹽鐵轉運使李巽、諫議大夫韋況更薦之,以山人徵為左拾遺。渤托疾不赴,遂家東都。朝廷政有得失,附章疏陳論。又撰《御戎新錄》二十卷,表獻之。九年,以著作郎征之。詔曰:「特降新恩,用清舊議。」渤於是赴官。歲餘,遷右補闕。連上章疏忤旨,改丹王府諮議參軍,分司東都。十二年,遷贊善大夫,
 依前分司。



 十三年,遣人上疏論時政,凡五事:一禮樂,二食貨,三刑政,四議都,五辯讎。渤以散秩在東都,以上章疏為己任,前後四十五封。再遷為庫部員外郎。



 時皇甫鎛作相,剝下希旨。會澤潞節度使郗士美卒,渤充吊祭使,路次陜西。渤上疏曰:「臣出使經行,歷求利病。竊知渭南縣長源鄉本有四百戶,今才一百餘戶;闃鄉縣本有三千戶;今才有一千戶,其他州縣大約相似。訪尋積弊,始自均攤逃戶。凡十家之內,大半逃亡,亦須五家攤稅。
 似投石井中,非到底不止。攤逃之弊,苛虐如斯,此皆聚斂之臣剝下媚上,唯思竭澤,不慮無魚。乞降詔書,絕攤逃之弊。其逃亡戶以其家產錢數為定,征有所欠,乞降特恩免之。計不數年,人必歸於農矣。夫農者,國之本,本立然後可以議太平。若不由茲,而云太平者,謬矣。」又言道途不修,驛馬多死。憲宗覽疏驚異,即以飛龍馬數百匹,付畿內諸驛。渤既以草疏切直,大忤宰相,乃謝病東歸。



 穆宗即位,召為考功員外郎。十一月,定京官考,不避
 權幸,皆行升黜。奏曰:



 宰臣蕭俛、段文昌、崔植,是陛下君臨之初,用為輔弼,安危理亂,決在此時。況陛下思天下和平,敬大臣禮切,固未有暱比左右、侈滿自賢之心。而宰相之權,宰相之事,陛下一以付之,實君義臣行,千載一遇之時也。此時若失,他更無時。而俛等上不能推至公,申炯誡,陳先王道德,以沃君心;又不能正色匪躬,振舉舊法,復百司之本,俾教化大立。臣聞政之興廢,在於賞罰。俛等作相已來,未聞獎一人德義,舉守官奉公者,
 使天下在官之徒有所激勸;又不聞黜一人職事不理、持祿養驕者,使尸祿之徒有所懼。如此,則刑法不立矣!邪正莫辯,混然無章,教化不行,賞罰之設,天下之事,復何望哉!



 一昨陛下游幸驪山,宰相、翰林學士是陛下股肱心腹,宜皆知之。蕭俛等不能先事未形,忘軀懇諫,而使陛下有忽諫之名流於史冊,是陷君於過也。孔子曰:「所謂大臣者,以道事君,不可則止。」若俛等言行計從,不當如是。若言不行,計不從,須奉身速退,不宜尸素於化
 源。進退戾也,何所避辭?其蕭人免、段文昌、崔植三人並翰林學士杜元穎等,並請考中下。



 御史大夫李絳、左散騎常侍張惟素、右散騎常侍李益等諫幸驪山,鄭覃等諫畋游,是皆恐陛下行幸不息,恣情無度;又恐馬有銜蹶不測之變,風寒生疾之憂,急奏無所詣,國璽委於婦人中幸之手。絳等能率御史諫官論列於朝,有懇激事君之體。其李絳、張惟素、李益三人,伏請賜上下考外,特與遷官,以彰陛下優忠賞諫之美。



 其崔元略冠供奉之首,
 合考上下;緣與於翬上下考,於翬以犯贓處死,準令須降,請賜考中中。大理卿許季同,任使於翬、韋道沖、韋正牧,皆以犯贓,或左降,或處死,合考中下;然頃者陷劉闢之亂,棄家歸朝,忠節明著,今宜以功補過,請賜考中中。少府監裴通,職事修舉,合考中上;以其請追封所生母而舍嫡母,是明罔於君,幽欺其先,請考中下。伏以昔在宰夫入寢,擅飲師曠、李調。今愚臣守官,請書宰相學士中下考。上愛聖運,下振頹綱,故臣懼不言之為罪,不懼
 言之為罪也。其三品官考,伏緣限在今月內進,輒先具如前。其四品以下官,續具條疏聞奏。



 狀入,留中不下。議者以宰輔曠官,自宜上疏論列,而渤越職釣名,非盡事君之道。未幾,渤以墜馬傷足,請告,會魏博節度使田弘正表渤為副使。杜元穎奏曰:「渤賣直沽名,動多狂躁。聖恩矜貸,且使居官。而干進多端,外交方鎮,遠求奏請,不能自安。久留在朝,轉恐生事。」乃出為虔州刺史。



 渤至州,奏還鄰境信州所移兩稅錢二百萬,免稅米二萬斛,減
 所由一千六百人。觀察使以其事上聞。未滿歲,遷江州刺史。張平叔判度支,奏徵久遠逋懸,渤在州上疏曰:「伏奉詔敕,雲度支使所奏,令臣設計徵填當州貞元二年逃戶所欠錢四千四百一十貫。臣當州管田二千一百九十七頃,今已旱死一千九百頃有餘,若更勒徇度支使所為,必懼史官書陛下於大旱中徵三十六年前逋懸。臣任刺史,罪無所逃。臣既上不副聖情,下不忍鞭笞黎庶,不敢輕持符印,特乞放臣歸田。」乃下詔曰:「江州所
 奏,實為懇誠。若不蠲容,必難存濟。所訴逋欠並放。」長慶二年,入為職方郎中。三年,遷諫議大夫。



 敬宗沖年即位,坐朝常晚。一日入閣,久不坐,群臣候立紫宸門外,有耆年衰病者,幾將頓僕。渤出次白宰相曰:「昨日拜疏陳論,今坐益晚,是諫官不能回人主之意,渤之罪也。請先出閣,待罪於金吾仗。」語次喚仗,乃止。渤又以左右常侍,職參觀諷,而循默無言,論之曰:「若設官不責其事,不如罷之,以省經費。茍未能罷,則請責職業。」渤充理匭使,奏曰:「
 事之大者聞奏,次申中書門下,次移諸司。諸司處理不當,再來投匭,即具事奏聞。如妄訴無理,本罪外加一等。準敕告密人付金吾留身待進止。今欲留身後牒臺府,冀止絕兇人。」從之。



 長慶、寶歷中,政出多門,事歸邪幸。渤不顧忠難,章疏論列,曾無虛日。帝雖昏縱,亦為之感悟。轉給事中,面賜金紫。



 寶歷元年,改元大赦。先是,鄠縣令崔發聞門外喧鬥,縣吏言五坊使下毆擊百姓。發怒,命吏捕之。曳挾既至,時已曛黑,不問色目。良久與語,乃知
 是一內官。天子聞之怒,收發系御史臺。御樓之日,放系囚,發亦在雞竿下。時有品官五十餘人,持仗毆發,縱橫亂擊,發破面折齒。臺吏以席蔽之,方免。是日系囚皆釋,發獨不免。渤疏論之曰:「縣令不合曳中人,中人不合毆御囚,其罪一也。然縣令所犯在恩前,中人所犯在恩後。中人橫暴,一至於此,是朝廷馴致使然。若不早正刑書,臣恐四夷之人及籓鎮奏事傳道此語,則慢易之心萌矣。」渤又宣言於朝云:「郊禮前一日,兩神策軍於青城內
 奪京兆府進食牙盤,不時處置,致有毆擊崔發之事。」上聞之,按問左右,皆言無奪食事。以渤黨發,出為桂州刺史、兼御史中丞,充桂管都防禦觀察使。



 渤雖被斥,正論不已,而諫官繼論其屈。後宰相李逢吉、竇易直、李程因延英上語及崔發,逢吉等奏曰:「崔發凌轢中人,誠大不敬。然發母是故相韋貫之姊,年僅八十。自發下獄,積憂成疾。伏以陛下孝治天下,稍垂恩宥。」帝愍然良久,曰:「比諫官論奏,但言發屈,未嘗言不敬之罪,亦不言有老母。
 如卿等言,寧無愍惻!」即遣中使送發至其家,兼撫問發母。韋夫人號哭,對中使杖發四十,拜章謝恩。帝又遣中使慰安之。



 渤在桂管二年,風恙求代,罷歸洛陽。太和五年,以太子賓客徵至京師。月餘卒,時年五十九,贈禮部尚書。渤孤貞,力行操尚,不茍合,而闒茸之流,非其沽激。至於以言擯退,終不息言,以救時病,服名節者重之。



 子祝,會昌中登進士第,闢諸侯府。



 張仲方,韶州始興人。祖九皋,廣州刺史、殿中監、嶺南節
 度使。父抗,贈右僕射。仲方伯祖始興文獻公九齡,開元朝名相。仲方,貞元中進士擢第,宏辭登科,釋褐集賢校理,丁母憂免。服闋,補秘書省正字,調授咸陽尉。出為邠州從事,入朝歷侍御史、倉部員外郎。



 會呂溫、羊士諤誣告宰相李吉甫陰事,二人俱貶。仲方坐呂溫貢舉門生,出為金州刺史。吉甫卒,入為度支郎中。時太常定吉甫謚為「恭懿」,博士尉遲汾請為「敬憲」,仲方駁議曰:



 古者,易名請謚,禮之典也。處大位者,取其巨節,蔑諸細行,垂範
 當代,昭示後人,然後書之,垂於不朽。善善惡惡,不可以誣,故稱一字,則至明矣;定褒貶是非之宜,泯同異紛綸之論。



 贈司徒吉甫,稟氣生材,乘時佐治,博涉多藝,含章炳文。燮贊陰陽,經緯邦國。惜乎通敏資性,便媚取容。故載踐樞衡,疊致臺袞,大權在己,沈謀罕成,好惡徇情,輕諾寡信。諂淚在臉,遇便則流;巧言如簧,應機必發。



 夫人臣之翼戴元後者,端恪致治,孜孜夙夜,絹熙庶績,平章百揆。兵者兇器,不可從我始;及乎伐罪,則料敵以成功。
 至使內有害輔臣之盜,外有懷毒蠆之孽。師徒暴野,戎馬生郊。皇上旰食宵衣,公卿大夫且慚且恥。農人不得在畝,緝婦不得在桑。耗斂賦之常貲,散帑廩之中積;征邊徼之備,竭運挽之勞。殭尸血流,胔骼成嶽,酷毒之痛,號訴無辜,剿絕群生,逮今四載。禍胎之兆,實始其謀;遺君父之憂,而豈謂之先覺者乎?



 夫論大功者,不可以妄取,不可以枉致。為資畫者體理,不顯不競,而豈妨令美?當削平西蜀,乃言語侍從之臣;擒翦東吳,則訏謨廊廟
 之輔。較其功則有異,言其力則不倫。何舍其所重而錄其所輕,收其所小而略其所大?且奢靡是嗜,而曰愛人以儉;受授無守,而曰慎才以補。斥諫諍之士於外,豈不近之蔽聰乎?舉忠烈之廟於內,豈不近之暱愛也?焉有蔽聰暱愛,家範無制,而能垂法作程,憲章百度乎?



 謹按謚法,敬以直內,內而不肅,何以刑於外?憲者,法也。《戴記》曰:「憲章文武。」又曰:「發慮憲。」義以為敬恪終始,載考歷位,未嘗效一法官,議一小獄。及居重位,以安和平易寬柔
 自處。考其名,與其行不類;研其事,與其道不侔。一定之辭,惟精惟審,異日詳制,貽諸史官。請俟蔡寇將平,天下無事,然後都堂聚議,謚亦未遲。



 憲宗方用兵,惡仲方深言其事,怒甚,貶為遂州司馬,量移復州司馬。遷河東少尹。未幾,拜鄭州刺史。



 滎陽大海佛寺,有高祖為隋鄭州刺史日,為太宗疾祈福,於此寺造石像一軀,凡刊勒十六字以志之。歲久剚缺,滎陽令李光慶重加修飾,仲方再刊石記之以聞。



 及敬宗即位,李程作相,與仲方同年
 登進士第,召仲方為右諫議大夫。敬宗童年戲慢,詔淮南王播造上巳競渡船三十只。播將船材於京師造作,計用半年轉運之費方得成。仲方詣延英面論,言甚懇激。帝只令造十只以進。帝又欲幸華清宮,仲方諫曰:「萬乘所幸,出須備儀。無宜輕行,以失威重。」帝雖不從,慰勞之。



 太和初,出為福州刺史、兼御史中丞、福建觀察使。三年,入為太子賓客。五年四月,轉右散騎常侍。七年,李德裕輔政,出為太子賓客分司。八年,德裕罷相,李守閔復
 召仲方為常侍。



 九年十一月,李訓之亂,四宰相、中丞、京兆尹皆死。翌日,兩省官入朝。宣政衙門未開,百官錯立於朝堂,無人吏引接。逡巡,閣門使馬元贄斜開宣政衙門傳宣曰:「有敕召左散騎常侍張仲方。」仲方出班。元贄宣曰:「仲方可京兆尹。」然後衙門大開,喚仗。月餘,鄭覃作相,用薛元賞為京兆尹,出仲方為華州刺史。開成元年五月,入為秘書監。外議以鄭覃黨李德裕,排擯仲方。覃恐涉朋黨,因紫宸奏事,覃啟曰:「丞郎闕人,臣欲用張仲
 方。」文宗曰:「中臺侍郎,朝廷華選。仲方作牧守無政,安可以丞郎處之?」累加銀青光祿大夫、上柱國、曲江縣開國伯,食邑七百戶。二年四月卒。



 仲方貞確自立,綽有祖風。自駁謚之後,為德裕之黨擯斥,坎坷而歿,人士輩之。有文集三十卷。



 兄仲端,位終都昌令。弟仲孚,登進士第,為監察御史。



 裴潾,河東人也。少篤學,善隸書。以門廕入仕。元和初,累遷右拾遺,轉左補闕。元和中,兩河用兵。初,憲宗寵任內
 官,有至專兵柄者,又以內官充館驛使。有曹進玉者,恃恩暴戾,遇四方使多倨,有至捽辱者,宰相李吉甫奏罷之。十二年,淮西用兵,復以內官為使。潾上疏曰:「館驛之務,每驛皆有專知官。畿內有京兆尹,外道有觀察使、刺史迭相監臨;臺中又有御史充館驛使,專察過闕。伏知近有敗事,上聞聖聰。但明示科條,督責官吏,據其所犯,重加貶黜,敢不惕懼,日夜厲精。若令宮闈之臣,出參館驛之務,則內臣外事,職分各殊,切在塞侵官之源,絕出
 位之漸。事有不便,必誡以初;令或有妨,不必在大。當掃靜妖氛之日,開太平至理之風。澄本正名,實在今日。」言雖不用,帝意嘉之,遷起居舍人。



 憲宗季年銳於服餌,詔天下搜訪奇士。宰相皇甫鎛與金吾將軍李道古挾邪固寵,薦山人柳泌及僧大通、鳳翔人田佐元,皆待詔翰林。憲宗服泌藥,日增躁渴,流聞於外。潾上疏諫曰:



 臣聞除天下之害者,受天下之利;共天下之樂者,饗天下之福。故上自黃帝、顓頊、堯、舜、禹、湯,下及周文王、武王,咸以
 功濟生靈,德配天地,故天皆報之以上壽,垂祚於無疆。伏見陛下以大孝安宗廟,以至仁牧黎元。自踐祚已來,刬積代之妖兇,開削平之洪業。而禮敬宰輔,待以終始;內能大斷,外寬小故。夫此神功聖化,皆自古聖主明君所不及,陛下躬親行之,實光映千古矣。是則天地神祇,必報陛下以山嶽之壽;宗廟聖靈,必福陛下以億萬之齡;四海蒼生,咸祈陛下以覆載之永。自然萬靈保祐,聖壽無疆。



 伏見自去年已來,諸處頻薦藥術之士,有韋山
 甫、柳泌等,或更相稱引,迄今狂謬,薦送漸多。臣伏以真仙有道之士,皆匿其名姓,無求於代,潛遁山林,滅影雲壑,唯恐人見,唯懼人聞。豈肯干謁公卿,自鬻其術?今者所有誇炫藥術者,必非知道之士。咸為求利而來,自言飛煉為神,以誘權貴賄賂。大言怪論,驚聽惑時,及其假偽敗露,曾不恥於逃遁。如此情狀,豈可保信其術,親餌其藥哉?《禮》曰:「夫人,食味別聲,被色而生者也。」《春秋左氏傳》曰:「味以行氣,氣以實志。」又曰:「水火醯醢鹽梅,以烹魚
 肉。宰夫和之,齊之以味;君子食之,以平其心。」夫三牲五穀,稟自五行,發為五味,蓋天地生之所以奉人也,是以聖人節而食之,以致康強逢吉之福。若夫藥石者,前聖以之療疾,蓋非常食之物。況金石皆含酷烈熱毒之性,加以燒治,動經歲月,既兼烈火之氣,必恐難為防制。若乃遠征前史,則秦、漢之君,皆信方士,如盧生、徐福、欒大、李少君,其後皆奸偽事發,其藥竟無所成。事著《史記》、《漢書》,皆可驗視。《禮》曰:「君之藥,臣先嘗之;親之藥,子先嘗之。」
 臣子一也,臣願所有金石,煉藥人及所薦之人,皆先服一年,以考其真偽,則自然明驗矣。



 伏惟元和聖文神武法天應道皇帝陛下,合日月照臨之明,稟乾元利貞之德,崇正若指南,受諫如轉規,是必發精金之刃,斷可疑之網。所有藥術虛誕之徒,伏乞特賜罷遣,禁其幻惑。使浮雲盡徹,朗日增輝;道化侔羲、農,悠久配天地,實在此矣。伏以貞觀已來,左右起居有褚遂良、杜正倫、呂向、韋述等,咸能竭其忠誠,悉心規諫。小臣謬參侍從,職奉起
 居,侍從之中,最近左右。傳曰:「近臣盡規。」則近侍之臣,上達忠款,實其本職也。



 疏奏忤旨,貶為江陵令。



 穆宗即位,柳泌等誅,徵潾為兵部員外郎,遷刑部郎中。有前率府倉曹曲元衡者,杖殺百姓柏公成母。法官以公成母死在辜外,元衡父任軍使,使以父廕征銅。柏公成私受元衡資貨,母死不聞公府,法寺以經恩免罪。潾議曰:「典刑者,公柄也。在官者得施於部屬之內;若非在官,又非部屬,雖有私罪,必告於官。官為之理,以明不得擅行鞭捶
 於齊人也。且元衡身非在官,公成母非部屬,而擅憑威力,橫此殘虐,豈合拘於常典?柏公成取貨於讎,利母之死,悖逆天性,犯則必誅。」奏下,元衡杖六十配流,公成以法論至死,公議稱之。轉考功、吏部二郎中。



 寶歷初,拜給事中。太和四年,出為汝州刺史、兼御史中丞,賜紫。坐違法杖殺人,貶左庶子,分司東都。



 七年,遷左散騎常侍,充集賢殿學士。集歷代文章續梁昭明太子《文選》,成三十卷,目曰《大和通選》,並音義、目錄一卷,上之。當時文士,非
 素與潾游者,其文章少在其選,時論咸薄之。



 八年,轉刑部侍郎,尋改華州刺史。九年,復拜刑部侍郎。開成元年,轉兵部侍郎。二年,加集賢院學士,判院事。尋出為河南尹,入為兵部侍郎。三年四月卒,贈戶部尚書,謚曰敬。



 潾以道義自處,事上盡心,尤嫉朋黨,故不為權幸所知。憲宗竟以藥誤不壽,君子以潾為知言。穆宗雖誅柳泌,既而自惑,左右近習,稍稍復進方士。時有處士張皋上疏曰:



 神慮淡則血氣和,嗜欲勝則疾疹作。和則必臻於壽
 考,作則必致於傷殘。是以古之聖賢,務自頤養,不以外物撓耳目,不徇聲色敗性情。由是和平自臻,福慶斯集。故《易》曰:「無妄之疾,勿藥有喜。」《詩》曰:「自天降康,降福穰穰。」此皆理合天人,著在經訓。然則藥以攻疾,無疾固不可餌之也。高宗朝,處士孫思邈者,精識高道,深達攝生,所著《千金方》三十卷,行之於代。其《序論》云:「凡人無故不宜服藥,藥氣偏有所助,令人臟氣不平。」思邈此言,可謂洞於事理也。或寒暑為寇,節宣有乖,事資醫方,尚須重慎。
 故《禮》云:「醫不三代,不服其藥。」施於凡庶,猶且如此,況在天子,豈得自輕?先朝暮年,頗好方士,徵集非一,嘗試亦多;果致危疾,聞於中外,足為殷鑒。皆陛下素所詳知,必不可更踵前車,自貽後悔。今朝野之人,紛紜竊議,直畏忤旨,莫敢獻言。臣蓬艾微生,麋鹿同處,既非邀寵,亦又何求?但泛覽古今,粗知忠義,有聞而默,於理不安。願陛下無怒芻蕘,庶裨萬一。



 穆宗嘆獎其言,尋令訪皋,不獲。



 李中敏,隴西人。父嬰。中敏元和末登進士第,性剛褊敢
 言。與進士杜牧、李甘相善,文章趣向,大率相類。中敏累從府闢,入為監察,歷侍御史。太和中,為司門員外郎。



 六年夏旱,時王守澄方寵鄭注,及誣構宋申錫後,人側目畏之。上以久旱,詔求致雨之方。中敏上言曰:「仍歲大旱,非聖德不至,直以宋申錫之冤濫,鄭注之奸弊。今致雨之方,莫若斬鄭注而雪申錫。」士大夫皆危之,疏留中不下。明年,中敏謝病歸洛陽。及訓、注誅,竟雪申錫,召中敏為司勛員外郎。尋遷刑部郎中,知臺雜。



 其年,拜諫議大
 夫,充理匭使。上言曰:「據舊例,投匭進狀人先以副本呈匭使,或詭異難行者,不令進入。臣檢尋文按,不見本敕,所由但云貞元奉宣,恐是一時之事。臣以為本置匭函,每日從內將出,日暮進入,意在使冤濫無告,有司不為申理者,或論時政,或陳利害;宜開其必達之路,所以廣聰明而慮幽枉也。若令有司先見,裁其可否,即非重密其事,俾壅塞自伸於九重之意。臣伏請今後所有進狀及封事,臣但為引進,取舍可否,斷自中旨。庶使名實在
 茲,以明置匭之本。」從之。尋拜給事中。



 李甘,字和鼎。長慶末,進士擢第,又制策登科。太和中,累官至侍御史。鄭注入翰林侍講,舒元輿既作相,注亦求入中書。甘唱於朝曰:「宰相者,代天理物,先德望而後文藝。注乃何人,敢茲叨竊?白麻若出,吾必壞之。」會李訓亦惡注之所求,相注之事竟寢。訓不獲已,貶甘封州司馬。



 又有李款者,與中敏同時為侍御史。鄭注邠寧入朝,款伏閣彈注云:「內通敕使,外結朝官,兩地往來,卜射財貨。」
 文宗不之省。及注用事,款亦被逐。開成中,累官至諫議大夫,出為蘇州刺史,遷洪州刺史、江西觀察使。杜牧自有傳。



 高元裕,字景圭,渤海人。祖甝。父集,官卑。元裕登進士第,本名允中,太和初,為侍御史,奏改元裕。累遷左司郎中。李宗閔作相,用為諫議大夫,尋改中書舍人。九年,宗閔得罪南遷,元裕出城餞送,為李訓所怒,出為閬州刺史。時鄭注入翰林,元裕草注制辭,言注以醫藥奉召親,注
 怒。會送宗閔,乃貶之。訓、注既誅,復徵為諫議大夫。



 開成三年,充翰林侍講學士。文宗寵莊恪太子,欲正人為師友。乃兼太子賓客。四年,改御史中丞,風望峻整。上言曰:「御史府紀綱之地,官屬選用,宜得實才。其不稱者,臣請出之。」監察御史杜宣猷、柳壞、崔郢、侍御史魏中庸、高弘簡,並以不稱,出為府縣之職。尋而藍田縣人賀蘭,進與里內五十餘人相聚念佛,神策鎮將皆捕之,以為謀逆,當大闢。元裕疑其冤,上疏請出賀蘭進等付臺覆問,
 然後行刑,從之。



 會昌中,為京兆尹。大中初,為刑部尚書。二年,檢校吏部尚書、襄州刺史,加銀青光祿大夫、渤海郡公、山南東道節度使。入為吏部尚書,卒。元裕兄少逸、元恭。



 少逸,長慶末為侍御史,坐弟元裕貶官,左授贊善大夫,累遷左司郎中。元裕為中丞,少逸遷諫議大夫,代元裕為侍講學士。兄弟迭處禁密,時人榮之。會昌中,為給事中,多所封奏。大中初,檢校禮部尚書、華州刺史、潼關防禦、鎮國軍使。入為左散騎常侍、工部尚書,卒。



 元裕子
 璩,登進士第。大中朝,由內外制歷丞郎,判度支。咸通中,守中書侍郎、平章事。



 李漢,字南紀,宗室淮陽王道明之後。道明生景融,景融生務該,務該生思,思生岌。岌已上無名位,及岌為蜀州晉原尉。岌生荊,荊為陜州司馬。荊生漢。



 漢,元和七年登進士第,累闢使府。長慶末,為左拾遺。敬宗好治宮室,波斯賈人李蘇沙獻沈香亭子材。漢上疏論之曰:「若以沈香為亭子,即與瑤臺瓊室事同。」寶歷中,王政日僻,漢與
 同列薛廷老,因入閣,廷奏曰:「近日除授不由中書,擬議多是宣出施行。臣恐自此紀綱大壞,奸邪恣行。願陛下各敕有司,稍存典故。」坐言忤旨,出為興元從事。



 文宗即位,召為屯田員外郎、史館修撰。漢,韓愈子婿,少師愈為文,長於古學,剛訐亦類愈。預修《憲宗實錄》,尤為李德裕所憎。太和四年,轉兵部員外郎。李宗閔作相,用為知制誥,尋遷駕部郎中。



 八年,代宇文鼎為御史中丞。時李程為左僕射,以儀注不定,奏請定制。先是,太和三年,兩省
 官同定左右僕射儀注:御史中丞已下,與僕射相遇,依令致敬,斂馬側立待。僕射謝官日,大夫中丞、三院御史,就幕次參見,其觀象門外立班,既以後至為重。大夫中丞到班後,朝堂所由引僕射就位,傳呼贊導,始大夫就列之儀。班退,贊導亦如之。御史大夫與僕射道途相遇,則分道而行。舊事,左右僕射初上,御史中丞、吏部侍郎已下羅拜。四年,中書奏曰:「僕射受中丞侍郎拜,則似太重;答郎官已下拜,則太輕。起今後,諸司四品已下官,及
 御史臺六品已下並郎官,並望準故事,餘依元和七年敕處分。」可之。至是,因李程奏,漢議曰:「左右僕射初上,受左右丞、諸曹侍郎、諸司四品及御史中丞已下拜。謹按《開元禮》及《六典》,並無此儀注,不知所起之由。或以為僕射師長百僚,此語亦無證據,唯有曹魏時賈詡《讓官表》中一句語耳。且尚書令是正長,尚無受拜之文。故事,與御史中丞、司隸校尉,號三獨坐。伏以朝廷比肩,同事聖主,南面受拜,臣下何安?縱有明文,尚須厘革。故《禮記》曰:『
 君於士不答拜,非其臣則答之。』況御史中丞、殿中御史是供奉官,尤為不可。儀制令雖有隔品之文,不知便是受拜否?及御史大夫,亦曾受御史已下拜,今並不行。蓋以禮數僭逼,非人臣所安。元和六年七月,詔崔邠、段平仲與當時禮官王涇、韋公肅等同議其事,理甚精詳。今請舉而行之,庶為折衷。」時程入省,竟依舊儀,議者以漢奏為是。



 七年,轉禮部侍郎。八年。改戶部侍郎。九年四月,轉吏部侍郎。六月,李宗閔得罪罷相,漢坐其黨,出為汾
 州刺史。宗閔再貶,漢亦改汾州司馬,仍三二十年不得錄用。會昌中,李德裕用事,漢竟淪躓而卒。



 漢弟滻、洗、潘,皆登進士第。潘,大中初為禮部侍郎。漢子貺,亦登進士第。



 李景儉,字寬中,漢中王瑀之孫。父褚,太子中舍。景儉,貞元十五年登進士第。性俊朗,博聞強記,頗閱前史,詳其成敗。自負王霸之略,於士大夫間無所屈降。



 貞元末,韋執誼、王叔文東宮用事,尤重之,待以管、葛之才。叔文竊
 政,屬景儉居母喪,故不及從坐。韋夏卿留守東都,闢為從事。竇君為御史中丞,引為監察御史。群以罪左遷,景儉坐貶江陵戶曹。累轉忠州刺史。



 元和末入朝。執政惡之,出為澧州刺史。與元稹、李紳相善。時紳、稹在翰林,屢言於上前。及延英辭日,景儉自陳己屈,穆宗憐之,追詔拜倉部員外郎。月餘,驟遷諫議大夫。



 性既矜誕,寵擢之後,凌蔑公卿大臣,使酒尤甚。中丞蕭俛、學士段文昌相交輔政,景儉輕之,形於談謔。二人俱訴之,穆宗不獲已,
 貶之。制曰:「諫議大夫李景儉,擢自宗枝,嘗探儒術,薦歷臺閣,亦分郡符。動或違仁,行不由義。附權幸以虧節,通奸黨之陰謀。眾情皆疑,群議難息。據因緣之狀,當置嚴科;順長養之時,特從寬典。勉宜省過,無或徇非。可建州刺史。」未幾元稹用事,自郡召還,復為諫議大夫。



 其年十二月,景儉朝退,與兵部郎中知制誥馮宿、庫部郎中知制誥楊嗣復、起居舍人溫造、司勛員外郎李肇、刑部員外郎王鎰等同謁史官獨孤朗,乃於史館飲酒。景儉乘
 醉詣中書謁宰相,呼王播、崔植、杜元穎名,面疏其失,辭頗悖慢。宰相遜言止之,旋奏貶漳州刺史。是日同飲於史館者皆貶逐。



 景儉未至漳州而元稹作相,改授楚州刺史。議者以景儉使酒,凌忽宰臣,詔令才行,遽遷大郡。稹懼其物議,追還,授少府少監。從坐者皆召還。而景儉竟以忤物不得志而卒。景儉疏財尚議,雖不厲名節,死之日,知名之士咸惜之。



 景儉弟景儒、景信、景仁,皆有藝學,知名於時。景信、景仁,皆登進士第。



 史臣曰:仲尼有言:「不得中行而與之,必也狂狷乎!」若渤論考第,仲方駁謚,誠知後悔,不能息言,可謂狷歟?當賊注挾邪之辰,群公結舌而寢默,而中敏、李甘、元裕,或肆其言,或奮其筆,暴揚醜跡,不憚撩須。謂之為狂,即有遺恨,比夫請劍斷佞,亦可同年而語也。南紀有良史才,足以自立,而協比權幸,顛沛終身。君子慎獨,庸可忽諸。景儉自負太過,蕩而無檢,良驥中年跅弛之患也。



 贊曰:張、李切言,利刃決云。裴諫方士,深誠愛君。言排賊
 注,高、李不群。漢、儉朋比,夫何足云。



\end{pinyinscope}