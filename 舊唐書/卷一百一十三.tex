\article{卷一百一十三}

\begin{pinyinscope}

 ○馮盎阿史那社爾子道真叔祖蘇尼失蘇尼失子忠附契苾何力黑齒常之李多祚李嗣業白孝德



 馮盎,高州良德人也。累代為本部大首領。盎少有武略,
 隋開皇中為宋康令。仁壽初,潮、成等五州獠叛,盎馳至京,請討之。文帝敕左僕射楊素與盎論賊形勢,素曰:「不意蠻夷中有此人,大可奇也。」即令盎發江、嶺兵擊之。賊平,授金紫光祿大夫,仍除漢陽太守。



 武德三年,廣、新二州賊帥高法澄、洗寶徹等並受林士弘節度,殺害隋官吏,盎率兵擊破之。既而寶徹兄子智臣又聚兵於新州,自為渠帥,盎趨往擊之。兵交,盎卻兜鍪大呼曰:「爾等頗識我否?」賊多棄戈肉袒而拜,其徒遂潰,擒寶徹、智臣等,
 嶺外遂定。或有說盎曰:「自隋季崩離,海內騷動。今唐雖應運,而風教未浹,南越一隅,未有所定。公克平五嶺二十餘州,豈與趙佗九郡相比?今請上南越王之號。」盎曰:「吾居南越,於茲五代,本州牧伯,唯我一門,子女玉帛,吾之有也。人生富貴,如我殆難,常恐弗克負荷,以墜先業。本州衣錦便足,餘復何求?越王之號,非所聞也。」



 四年,盎以南越之眾降,高祖以其地為羅、春、白、崖、儋、林等八州,仍授盎上柱國、高羅總管,封吳國公,尋改封越國公。拜
 其子智戴為春州刺史,智或東合州刺史,徙封盎耿國公。貞觀五年,盎來朝,太宗宴賜甚厚。俄而羅竇諸洞獠叛,詔令盎率部落二萬為諸軍先鋒。時有賊數萬屯據險要,不可攻逼。盎持弩語左右曰:「盡吾此箭,可知勝負。」連發七矢,而中七人,賊退走,因縱兵乘之,斬首千餘級。太宗令智戴還慰省之,自後賞賜不可勝數。盎奴婢萬餘人,所居地方二千里,勤於簿領,詰擿奸狀,甚得其情。二十年卒。贈左騎衛大將軍、荊州都督。



 阿史那社爾,突厥處羅可汗子也。年十一,以智勇稱於本蕃,拜為拓設,建牙於磧北,與欲谷設分統鐵勒、紇骨、同羅等諸部。在位十年,無所課斂。諸首領或鄙其不能富貴,社爾曰:「部落既豐,於我便足。」諸首領咸畏而愛之。



 武德九年,延陀、回紇等諸部皆叛,攻破欲谷設,社爾擊之,復為延陀所敗。貞觀二年,遂率其餘眾保於西偏,依可汗浮圖。後遇頡利滅,而西蕃葉護又死,奚利邲咄陸可汗兄弟爭國,社爾揚言降之,引兵西上,因襲破古蕃,
 半有其國,得眾十餘萬,自稱都布可汗。謂其諸部曰:「首為背叛破我國者,延陀之罪也。今我據有西方,大得兵馬,不平延陀而取安樂,是忘先可汗,為不孝也。若天令不捷,死亦無恨。」其酋長咸諫曰:「今新得西方,須留鎮壓。若即棄去,遠擊延陀,只恐葉護子孫必來復國。」社爾不從,親率五萬餘騎討延陀於磧北,連兵百餘日。遇我行人劉善因立同娥設為咥利始可汗,社爾部兵又苦久役,多委之逃。延陀因縱擊敗之,復保高昌
 國。其舊兵在者才萬餘人,又與西蕃結隙。



 九年,率眾內屬,拜左騎衛大將軍。歲餘,令尚衡陽長公主,授駙馬都尉,典屯兵於苑內。十四年,授行軍總管,以平高昌。諸人咸即受賞,社爾以未奉詔旨,秋毫無所取。及降別敕,然後受之。及所取,唯老弱故弊而已。軍還,太宗美其廉慎,以高昌所得寶刀並雜彩千段賜之,仍令檢校北門左屯營,封畢國公。十九年,從太宗征遼,至駐蹕陣,頻遭流矢,拔而又進。其所部兵士,人百其勇,盡獲殊勛。師旋,兼授鴻臚卿。二
 十一年,為昆丘道行軍大總管,征龜茲。明年,軍次西突厥,擊處密,大破之,餘眾悉降。又下龜茲大撥換城,虜龜茲王白訶黎布失畢及大臣那利等百餘人而還。屬太宗崩,請以身殉葬,高宗遣使喻以先旨,不許。遷右衛大將軍。永徽四年,加位鎮軍大將軍。六年卒,贈輔國大將軍、並州都督,陪葬昭陵。起塚以象蔥山,仍為立碑,謚曰元。子道真,位至左屯衛大將軍。


貞觀初,阿史那蘇尼失者,啟民可汗之母弟,社
 \gezhu{
  人小}
 叔祖也。其父始畢可汗以為
 沙缽羅設,督部落五萬家,牙直靈州之西北,驍雄有恩惠,甚得種落之心。及頡利政亂,而蘇尼失所部獨不攜離。突利之來奔也,頡利乃立蘇尼失為小可汗。及頡利為李靖所破,獨騎而投之,蘇尼失遂舉其眾歸國,因令子忠擒頡利以獻。太宗賞賜優厚。拜北寧州都督、右衛大將軍,封懷德郡王。貞觀八年卒。



 忠以擒頡利功,拜左屯衛將軍,妻以宗女定襄縣主,賜名為忠,單稱史氏。貞觀九年,遷右衛大將軍。永徽初,封薛國公,累遷右驍衛
 大將軍。所歷皆以清謹見稱,時人比之金日磾。上元初卒,贈鎮軍大將軍,陪葬昭陵。



 子暕,襲封薛國公,垂拱中,歷位司僕卿。



 契苾何力,其先鐵勒別部之酋長也。父葛,隋大業中繼為莫賀咄特勒,以地逼吐谷渾,所居隘狹,又多瘴癘,遂入龜茲,居於熱海之上。特勤死,何力時年九歲。降號大俟利發。至貞觀六年,隨其母率眾千餘家詣沙州,奉表內附,太宗置其部落於甘、涼二州。何力至京,授左領軍
 將軍。



 七年,與涼州都督李大亮、將軍薛萬均同征吐谷渾。軍次赤水川,萬均率騎先行,為賊所攻,兄弟皆中槍墮馬,徒步而鬥,兵士死者十六七。何力聞之,將數百騎馳往,突圍而前,縱橫奮擊,賊兵披靡,萬均兄弟由是獲免。時吐谷渾主在突淪川,何力復欲襲之,萬均懲其前敗,固言不可。何力曰:「賊非有城郭,逐水草以為生,若不襲其不虞,便恐鳥驚魚散,一失機會,安可傾其巢穴耶!」乃自選驍兵千餘騎,直入突淪川,襲破吐谷渾牙帳,斬
 首數千級,獲駝馬牛羊二十餘萬頭,渾主脫身以免,俘其妻子而還。有詔勞於大斗拔谷。萬均乃排毀何力,自稱己功。何力不勝憤怒,拔刀而起,欲殺萬均,諸將勸止之。太宗聞而責問其故,何力言萬均敗恧之事,太宗怒,將解其官回授,何力固讓曰:「以臣之故而解萬均,恐諸蕃聞之,以為陛下厚蕃輕漢,轉相誣告,馳競必多。又夷狄無知,或謂漢臣皆如此輩,固非安寧之術也。」太宗乃止。尋令北門宿衛,檢校屯營事,敕尚臨洮縣主。



 十四年,
 為蔥山道副大總管,討平高昌。時何力母姑臧夫人、母弟賀蘭州都督沙門並在涼府。十六年,詔許何力觀省其母,兼撫巡部落。時薛延陀強盛,契苾部落皆願從之。何力至,聞而大驚曰:「主上於汝有厚恩,任我又重,何忍而圖叛逆!」諸首領皆曰:「可敦及都督已去,何故不行?」何力曰:「我弟沙門孝而能養,我以身許國,終不能去也。」於是眾共執何力至延陀所,置於可汗牙前。何力箕踞而坐,拔佩刀東向大呼
 曰:「豈有大唐烈士,受辱蕃庭,天地日月,願知我心!」又割左耳以明志不奪也。可汗怒,欲殺之,為其妻所抑而止。初,太宗聞何力之延陀,明非其本意。或曰:「人心各樂其土,何力今入延陀,猶魚之得水也。」太宗曰:「不然,此人心如鐵石,必不背我。」會有使自延陀至,具言其狀,太守泣謂群臣曰:「契苾何力竟如何?」遽遣兵部侍郎崔敦禮持節入延陀,許降公主,求何力。由是還,拜右驍衛大將軍。太宗既許公主於延陀,行有日矣,何力抗表固言不可。太宗曰:「吾聞天子無戲言,既已許
 之,安可廢?」何力曰:「然。臣本請延緩其事,不謂總停。臣聞六禮之內,婿合親迎,宜告延陀親來迎婦,縱不敢至京邑,即當使詣靈州。畏漢必不敢來,論親未可有成日。既憂悶,臣又攜離,不盈一年,自相猜忌。延陀志性狠戾,若死,必兩子相爭,坐而制之,必然之理。」太宗從之。延陀恐有詐,竟不至靈州。自後常悒悒不得志,一年而死,兩子果爭權,各立為主。



 太宗征遼東,以何力為前軍總管,軍次白崖城,為賊所圍,被矛中腰,瘡重疾甚,太宗自為傅
 藥。及拔賊城,敕求傷之者高突勃,付何力自殺之。何力奏言:「犬馬猶為其主,況於人乎?彼為其主,況致命冒白刃而刺臣,是其義勇士也。本不相識,豈是冤仇?」遂舍之。二十二年,為昆丘道總管,擊龜茲,獲其王訶梨布失畢及諸首領等。太宗崩,何力欲殺身以殉,高宗諭而止之。



 永徽二年,處月、處密叛,以何力為弓月道大總管,討平之,擒其渠帥處密時健俟斤、合支賀等以歸。顯慶二年,遷左驍衛大將軍,累封郕國公,兼檢校鴻臚卿。龍朔元
 年,又為遼東道行軍大總管。九月,次于鴨綠水,其地即高麗之險阻,莫離支男生以精兵數萬守之,眾莫能濟。何力始至,會層冰大合,趣即渡兵,鼓噪而進,賊遂大潰,追奔數十里,斬首三萬級,餘眾盡降,男生僅以身免。會有詔班師,乃還。其年,九姓叛,以何力為鐵勒道安撫大使。乃簡精騎五百馳入九姓中,賊大驚,何力乃謂曰:「國家知汝被詿誤,遂有翻動,使我舍汝等過,皆可自新。罪在酋渠,得之則已。」諸姓大喜,共擒偽葉護及設、特勤等同
 惡二百餘人以歸,何力數其罪而誅之。乾封元年,又為遼東道行軍大總管,兼安撫大使。高麗有眾十五萬,屯於遼水,又引靺鞨數萬據南蘇城。何力奮擊,皆大破之。斬首萬餘級,乘勝而進,凡拔七城。乃回軍會英國公李勣于鴨綠水,共攻辱夷、大行二城,破之。勣頓軍於鴨綠柵,何力引蕃漢兵五十萬先臨平壤。勣仍繼至,共拔平壤城,執男建,虜其王還。授鎮軍大將軍,行左衛大將軍,徙封涼國公,仍檢校右羽林軍。儀鳳二年卒,贈輔國大
 將軍、並州都督,陪葬昭陵,謚曰烈。



 有三子:明、光、貞。明,左鷹揚衛大將軍,兼賀蘭都督,襲爵涼國公。光,則天時右豹韜衛將軍,為酷吏所殺。貞,司膳少卿。



 黑齒常之,百濟西部人。長七尺餘,驍勇有謀略。初在本蕃,仕為達率兼郡將,猶中國之刺史也。顯慶五年,蘇定方討平百濟,常之率所部隨例送降款。時定方縶左王及太子隆等,仍縱兵劫掠,丁壯者多被戮。常之恐懼,遂與左右十餘人遁歸本部,鳩集亡逸,共保任存山,
 築柵以自固,旬日而歸附者三萬餘人。定方遣兵攻之,常之領敢死之士拒戰,官軍敗績,遂復本國二百餘城,定方不能討而還。龍朔三年,高宗遣使招諭之,常之盡率其眾降。累轉左領軍員外將軍。



 儀鳳中,吐蕃犯邊,常之從李敬玄擊之。劉審禮之沒賊,敬玄欲抽軍,卻阻泥溝,而計無所出。常之夜率敢死之兵五百人進掩賊營,吐蕃首領跋地設棄軍宵遁,敬玄因此得還。高宗嘆其才略,擢授左武衛將軍,兼檢校左羽林軍,賜金五百兩、絹五
 百匹,仍充河源軍副使。時吐蕃贊婆及素和貴等賊徒三萬餘屯於良非川。常之率精騎三千夜襲賊營,殺獲二千級,獲羊馬數萬,贊婆等單騎而遁。擢常之為大使,又賞物四百匹。常之以河源軍正當賊沖,欲加兵鎮守,恐有運轉之費,遂遠置烽戍七十餘所,度開營田五千餘頃,歲收百餘萬石。開耀中,贊婆等屯於青海,常之率精兵一萬騎襲破之,燒其糧貯而還。常之在軍七年,吐蕃深畏憚之,不敢復為邊患。嗣聖元年,遷左武衛大將
 軍,仍檢校左羽林軍。垂拱二年,突厥犯邊,命常之率兵拒之。躡至兩井,忽逢賊三千餘眾,常之見賊徒爭下馬著甲,遂領二百餘騎,身當先鋒直沖,賊遂棄甲而散。俄頃,賊眾大至。及日將暮,常之令伐木,營中燃火如烽燧,時東南忽有大風起,賊疑有救兵相應,遂狼狽夜遁。以功進封燕國公。三年,突厥入寇朔州,常之又充大總管,以李多祚、王九言為副。追躡至黃花堆,大破之,追奔四十餘里,賊散走磧北。時有中郎將爨寶璧表請窮追餘
 賊,制常之與寶璧會,遙為聲援。寶璧以為破賊在朝夕,貪功先行,竟不與常之謀議,遂全軍而沒。尋為周興等誣構,雲與右鷹揚將軍趙懷節等謀反系獄,遂自縊而死。



 常之嘗有所乘馬為兵士所損,副使牛師獎等請鞭之。常之曰:「豈可以損私馬而決官兵乎!」竟赦之。前後所得賞賜金帛等,皆分給將士;及死,時甚惜之。



 李多祚,代為靺鞨酋長。多祚驍勇善射,意氣感激。少以軍功歷位右羽林軍大將軍,前後掌禁兵,北門宿衛
 二十餘年。



 神龍初,張柬之將誅張易之兄弟,引多祚將籌其事,謂曰:「將軍在北門幾年?」曰:「三十年矣。」柬之曰:「將軍擊鐘鼎食,金章紫綬,貴寵當代,位極武臣,豈非大帝之恩乎?」曰:「然。」又曰:「將軍既感大帝殊澤,能有報乎?大帝之子見在東宮,逆豎張易之兄弟擅權,朝夕危逼。宗社之重,於將軍,誠能報恩,正屬今日。」多祚曰:「茍緣王室,惟相公所使,終不顧妻子性命。」因即引天地神祗為要誓,詞氣感動,義形於色。遂與柬之等定謀誅易之兄弟,以功
 進封遼陽郡王,食實封八百戶,仍拜其子承訓為衛尉少卿。其年,將有事於太廟,特令多祚與安國相王登輦夾侍。監察御史王覿上疏諫曰:「竊惟祔廟之禮,在於尊祖奉先;肅事之儀,豈厭惟親與德。伏見恩敕令安國相王與李多祚參乘,且多祚夷人,有功於國,適可加之寵爵,豈宜逼奉至尊,侍帝弟而連衡,與吾君而共輦?誠恐萬方之人,不允所望。昔文帝引趙談參乘,盎伏車前曰:『臣聞天子所共六尺輿者,皆天下豪英。今漢雖乏人,陛
 下獨奈何與刀鋸之餘共載!』於是斥而下之。多祚雖無趙談之累,亦非卿相之重,不自循省,無聞固讓,豈國乏良輔,更無其人。史官所書,將示於後。何袁盎之強諫,獨微臣之不及。惟陛下詳擇焉。」上謂覿曰:「多祚雖是夷人,緣其有功,委以心腹,特令侍輦,卿勿復言也。」



 節愍太子之殺武三思也,多祚與羽林大將軍李千里等率兵以從。太子令多祚先至玄武樓下,冀上問以殺三思之意,遂按兵不戰。時有宮闈令楊思勖於樓上侍帝,請拒其
 先鋒。多祚子婿羽林中郎將野呼利為先軍總管,思勖挺刃斬之,兵眾大沮。多祚俄為左右所殺,並殺其二子,籍沒其家。睿宗即位,下制曰:「以忠報國,典冊所稱;感義捐軀,名節斯在。故右羽林大將軍、上柱國、遼陽郡王李多祚,三韓貴種,百戰餘雄。席寵禁營,乃心王室,仗茲誠信,翻陷誅夷。賴彼神明,重清奸慝,永言徽烈,深合褒崇。宜追歿後之榮,以復生前之命。可還舊官,仍宥其妻子。」



 李嗣業,京兆高陵人也。身長七尺,壯勇絕倫。天寶初,
 隨募至安西,頻經戰鬥,於時諸軍初用陌刀,咸推嗣業為能。每為隊頭,所向必陷。節度使馬靈察知其勇健,每出師,令嗣業與焉。累遷至中郎將。



 天寶七載,安西都知兵馬使高仙芝奉詔總軍,專征勃律,選嗣業與郎將田珍為左右陌刀將。於時吐蕃聚十萬眾於娑勒城,據山因水,塹斷崖谷,編木為城。仙芝夜引軍渡信圖河,奄至城下。仙芝謂嗣業與田珍曰:「不午時須破此賊。」嗣業引步軍持長刀上,山頭拋櫑蔽空而下,嗣業獨引一旗於絕
 險處先登,諸將因之齊上。賊不虞漢軍暴至,遂大潰,填溪谷,投水溺死,僅十八九。遂長驅至勃律城擒勃律王、吐蕃公主,斬藤橋,以兵三千人戍。於是拂林、大食諸胡七十二國皆歸國家,款塞朝獻,嗣業之功也。由此拜右威衛將軍。十載,又從平石國,及破九國胡並背叛突騎施,以跳蕩加特進,兼本官。初,仙芝紿石國王約為和好,乃將兵襲破之,殺其老弱,虜其丁壯,取金寶瑟瑟駝馬等,國人號哭,因掠石國王東,獻之於闕下。其子逃難奔
 走,告於諸胡國。群胡忿之,與大食連謀,將欲攻四鎮。仙芝懼,領兵二萬深入胡地,與大食戰,仙芝大敗。會夜,兩軍解,仙芝眾為大食所殺,存者不過數千。事窘,嗣業白仙芝曰:「將軍深入胡地,後絕救兵。今大食戰勝,諸胡知,必乘勝而並力事漢。若全軍沒,嗣業與將軍俱為賊所虜,則何人歸報主?不如馳守白石嶺,早圖奔逸之計。」仙芝曰:「爾,戰將也。吾欲收合餘燼,明日復戰,期一勝耳。」嗣業曰:「愚者千慮,或有一得,勢危若此,不可膠柱。」固請行,
 乃從之。路隘,人馬魚貫而奔。會跋汗那兵眾先奔,人及駝馬塞路,不克過。嗣業持大棒前驅擊之,人馬應手俱斃。胡等遁,路開,仙芝獲免。仙芝表其功,加驃騎左金吾大將軍。



 及祿山反,兩京陷,上在靈武,詔嗣業赴行在。嗣業自安西統眾萬里,威令肅然,所過郡縣,秋毫不犯。至鳳翔謁見,上曰:「今日得卿,勝數萬眾,事之濟否,實在卿也。」遂與郭子儀、僕固懷恩等常犄角為先鋒將。嗣業每持大棒沖擊,賊眾披靡,所向無敵。



 祿山之亂,兩京未復,
 肅宗在鳳翔。至德二年九月,嗣業從廣平王收復京城,與賊大戰於香積寺北,西拒灃水,東臨大川,十里間軍容不斷。嗣業時為鎮西、北庭支度行營節度使,為前軍,朔方右行營節度使郭子儀為中軍,關內行營節度王思禮為後軍。戈鋌鼓鞞,震曜山野,距賊軍數里,列長陣而待之。賊將李歸仁初以銳師數來挑戰,我師攢矢而逐之,賊軍大至,逼我追騎,突入我營,我師囂亂。嗣業謂郭子儀曰:「今日之事,若不以身啖寇,決戰於陣,萬死而
 冀其一生。不然,則我軍無孑遺矣。」嗣業乃脫衣徒搏,執長刀立於陣前大呼,當嗣業刀者,人馬俱碎,殺十數人,陣容方駐。前軍之士盡執長刀而出,如墻而進。嗣業先登奮命,所向摧靡。是時,賊先伏兵於營東,偵者知之,元帥廣平王分回紇銳卒,令擊其伏兵,賊將大敗。嗣業出賊營之背,與回紇合勢,表裏夾攻,自午及酉,斬首六萬級,填溝壑而死者十二三。賊帥張通儒、安守忠、李歸仁等收合殘卒,東走保陜郡。慶緒又命嚴莊率眾數萬,赴
 陜助通儒輩以拒官軍。廣平王、郭子儀、王思禮等大軍營於陜西。嗣業與子儀遇賊於新店,與之力戰,數合,我師初勝而後敗,嗣業逐急應接。回紇從南山望見官軍敗,曳白旗而下,徑抵賊背,穿賊陣,賊陣西北角先陷。嗣業又率精騎前擊,表裏齊進,賊眾大敗,走河北。子儀遂收東都。嗣業以功加開府儀同三司、衛尉卿,封虢國公,食實封二百戶。



 乾元二年,諸將同圍相州。是時築堤引漳水灌城,經月餘,城不拔。是時,軍無統帥,諸將自圖全,
 人無鬥志。賊每出戰,嗣業被堅沖突,履鋒冒刃,為流矢所中。數日,瘡欲愈,臥於帳中,忽聞金鼓之聲,因而大叫,瘡中血出數升注地而卒。上聞之震悼,嗟惜久之,詔曰:「臨難忘身,為臣之大節;念功加贈,經國之常典。故衛尉卿、兼懷州刺史、充北庭行營節度使、虢國公李嗣業,植操沉厚,秉心忠烈,懷幹時之勇略,有戡難之遠謀。久仕邊陲,備經任使。自兇渠構亂,中夏不寧,持感激之誠,總驍果之眾,親當矢石,頻立勛庸。壯節可嘉,將謀於百勝;
 忠誠未遂,空恨於九原。言念其功,良深憫悼。死於王事,禮有可加,宜贈裂土之封,用廣飾終之義。可贈武威郡王。其賻贈及緣葬事,所司倍於常式,仍令官給靈輿,遞還所在。以其子佐國襲其官爵,食實封二百戶。」



 白孝德,安西胡人也,驍悍有膽力。乾元中,事李光弼為偏裨。史思明攻河陽,使驍將劉龍仙率鐵騎五千臨城挑戰。龍仙捷勇自恃,舉右足加馬鬣上,嫚罵光弼。光弼登城望,顧諸將曰:「孰可取者?」僕固懷恩請行,光弼曰:「此
 非大將所為。」歷選其次,左右曰:「白孝德可。」光弼乃招孝德前,問曰:「可乎?」曰:「可。」光弼問:「所要幾何兵?」孝德曰:「可獨往耳。」光弼壯之。終問所欲,對曰:「願選五十騎於軍門為繼,兼請大軍鼓噪以增氣勢,他無所用。」光弼撫其背以遣之。孝德挾二矛,策馬截流而渡。半濟,懷恩賀曰:「克矣。」光弼曰:「未及,何知其克?」懷恩曰:「觀其攬轡便闢,可萬全者。」龍仙見其獨來,甚易之,足不降鬣。稍近,將動,孝德搖手示之,若使其不動,龍仙不之測,乃止。孝德呼曰:「侍中
 使餘致辭,非他也。」龍仙去十步與之言,褻罵如初。孝德息馬伺便,因真目曰:「賊識我乎?」龍仙曰:「誰耶?」曰:「我,國之大將白孝德也。」龍仙曰:「是何豬狗!」孝德發聲寔啖,持矛躍馬而搏之。城上鼓噪,五十騎繼進。龍仙矢不暇發,環走堤上。孝德追及,斬首,攜之而歸,賊徒大駭。其後,累戰功至安西北庭行營節度、鄜坊邠寧節度使,歷檢校刑部尚書,封昌化郡王。以家難去職,服闋復舊官。



 大歷十四年九月,轉太子少傅,尋卒,時年六十六,贈太子太保。


史臣曰:歷代武臣,壯勇出眾者有諸,節行勵俗者鮮矣,矧蠻夷之人乎!如馮盎智勇守節,社
 \gezhu{
  人小}
 廉慎知足,蘇尼失恩惠,史忠清謹。凡用兵破吐蕃、谷渾,勇也;心如鐵石,忠也;不解萬均官,恕也;阻延陀之親,智也;舍高突勃之死,識也。立大功,居顯位,夙夜匪懈者,何力有焉。常之以私馬恕官兵,與將士均賞賜,古之名將,無以加焉。多祚忘身許國,孝德壯勇立功,皆三軍之傑也,豈九夷之陋哉!嗣業力贊中興,終歿王事,未可倫而擬也。



 贊曰:君子之居,九夷無陋。壯哉嗣業,孰出其
 右!



\end{pinyinscope}