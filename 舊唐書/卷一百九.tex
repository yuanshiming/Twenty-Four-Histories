\article{卷一百九}

\begin{pinyinscope}

 ○宇文
 融韋堅楊慎矜王鉷



 宇文融,京兆萬年人,隋禮部尚書平昌公弼之玄孫也。祖節,貞觀中為尚書右丞,明習法令,以幹局見稱。時江夏王道宗嘗以私事托於節,節遂奏之,太宗大悅,賜絹
 二百匹,仍勞之曰:「朕所以不置左右僕射者,正以卿在省耳。」永徽初,累遷黃門侍郎、同中書門下三品,代於志寧為侍中。坐房遺愛事配流桂州而卒。父嶠,萊州長史。



 融,開元初累轉富平主簿,明辯有吏乾,源乾曜、孟溫相次為京兆尹,皆厚禮之,俄拜監察御史。時天下戶口逃亡,免役多偽濫,朝廷深以為患。融乃陳便宜,奏請檢察偽濫,搜括逃戶。玄宗納其言,因令融充使推勾。無幾,獲偽濫及諸免役甚眾,特加朝散大夫,再遷兵部員外郎,
 兼侍御史。融於是奏置勸農判官十人,並攝御史,分往天下,所在檢括田疇,招攜戶口。其新附客戶,則免其六年賦調,但輕稅入官。議者頗以為擾人不便,陽翟尉皇甫憬上疏曰:



 臣聞智者千慮,或有一失,愚夫千計,亦有一得。且無益之事繁,則不急之務眾;不急之務眾,則數役;數役,則人疲;人疲,則無聊生矣。是以太上務德,以靜為本;其次化之,以安為上。但責其疆界,嚴之堤防,山水之餘,即為見地。何必聚人阡陌,親遣括量,故奪農時,
 遂令受弊。又應出使之輩,未識大體,所由殊不知陛下愛人至深,務以勾剝為計。州縣懼罪,據牒即徵。逃亡之家,鄰保代出;鄰保不濟,又便更輸。急之則都不謀生,緩之則慮法交及。臣恐逃逸從此更深。至如澄流在源,止沸由火,不可不慎。今之具僚,向逾萬數,蠶食府庫,侵害黎人。國絕數載之儲,家無經月之畜,雖其厚稅亦不可供。戶口逃亡,莫不由此。縱使伊、皋申術,管、晏陳謀,豈息茲弊?若以此給,將何以堪!雖東海、南山盡為粟帛,亦恐不
 足,豈括田稅客能周給也!



 左拾遺楊相如上書,咸陳括客為不便。上方委任融,侍中源乾曜及中書舍人陸堅皆贊成其事,乃貶憬為盈川尉。於是諸道括得客戶凡八十餘萬,田亦稱是。州縣希融旨意,務於獲多,皆虛張其數,亦有以實戶為客者。歲終徵得客戶錢數百萬,融由是擢拜御史中丞。言事者猶稱括客損居人,上令集百僚於尚書省議。公卿已下懼融恩勢,皆雷同不敢有異詞,唯戶部侍郎楊瑒獨建議以括客不利居人,徵籍
 外田稅,使百姓困弊,所得不補所失。無幾,瑒出為外職。



 融乃馳傳巡歷天下,事無大小,先牒上勸農使而後申中書,省司亦待融指捴而後決斷。融之所至,必招集老幼宣上恩命,百姓感其心,至有流淚稱父母者。融使還具奏,乃下制曰:



 人惟邦本,本固邦寧,必在安人,方能固本。永言理道,實獲朕心。思所以康濟黎庶,寵綏華夏,上副宗廟乾坤之寄,下答宇縣貢獻之勤,何嘗不夜分輟寢,日旰忘食。然後以眇眇之身,當四海之貴。雖則長想
 遐邇,不可家至日見。至於宣布政教,安輯逋亡,言念再三,其勤至矣。莫副朕命,實用恧焉,當扆永懷,靜言厥緒。豈人流自久,招諭不還,上情靡通於下,眾心罔達於上。求之明發,想見其人。當屬括地使宇文融謁見於延英殿,朕以人必土著,因議逃亡,嘉其忠讜,堪任以事,乃授其田戶紀綱,兼委之郡縣厘革,便令充使,奉以安人。遂能恤我黎元,克將朕命,發自夏首,及於歲終,巡按所及,歸首百萬。仍聞宣制之日,老幼欣躍,惟令是從,多流淚
 以感朕心,咸吐誠以荷王命。猶恐朕之薄德,未孚於人,撫字安存,更冀良算。遂命百司長吏,方州嶽牧,僉議廟堂,廣徵異見。群詞盈於札翰,環省彌於旬日,庶廣朕意,豈以為勞,稽眾考言,謂斯折衷。欲人必信,期於令行,凡爾司存,勉以遵守。



 夫食為人天,富而後教,經教彞體,前哲至言。故平糴行於昔王,義倉加於近代,所以存九年之蓄,收上中之斂。穰賤則農不傷財,災饉則時無菜色,救人活國,其利博哉!今流戶大來,王田載理,敖庾之務,
 寤寐所懷。其客戶所稅錢,宜均充所在常平倉用,仍許預付價值,任粟麥兼貯。並舊常平錢粟,並委本道判官勾當處置,使斂散及時,務以矜恤。且分災恤患,州黨之常情;損餘濟闕,親鄰之善貸。故木鐸云徇,里胥均功,夜績相從,齊俗以贍。今陽和布澤,丁壯就田,言念鰥煢,事資拯助。宜委使司與州縣商量,勸作農社,貧富相恤,耕耘以時。仍每至雨澤之後,種獲忙月,州縣常務,一切停減。使趨時急於備寇,尺璧賤於寸陰,是則天無虛施,人
 無遺力。



 又政在經遠,功惟久著,今逃亡初復,居業未康,循逃戶及籍外剩田,猶宜勞徠,理資存撫。其十道分判官,三五年內,使就厥功,令有終始。當道覆屯,及須推劾,並以委之,不須廣差餘使,示專其事,不擾於人。政術有能,必行賞罰。其已奏復業歸首,勾當州縣,每季一申,不須挾名,致有勞擾。其歸首戶,各令新首處與本貫計會年戶色役,勿欺隱及其兩處徵科。宣布天下,使明知朕意。



 中書令張說素惡融之為人,又患其權重,融之所奏,
 多建議爭之。融揣其意,先事圖之。中書舍人張九齡言於說曰:「宇文融承恩用事,辯給多詞,不可不備也。」說曰:「此狗鼠輩,焉能為事!」融尋兼戶部侍郎。從東封還,又密陳意見,分吏部為十銓典選事,所奏又為說所抑。融乃與御史大夫崔隱甫連名劾說,廷奏其狀,說由是罷知政事。融恐說復用為己患,數譖毀之。上惡其朋黨,尋出融為魏州刺史。俄轉汴州刺史,又上表請用《禹貢》九河舊道,開稻田以利人,並回易陸運本錢,官收其利。雖
 興役不息,而事多不就。



 十六年,復入為鴻臚卿,兼戶部侍郎。明年,拜黃門侍郎,與裴光庭並兼同中書門下平章事。融既居相位,欲以天下為己任,謂人曰:「使吾居此數月,庶令海內無事矣。」於是薦宋璟為右丞相,裴耀卿為戶部侍郎,許景先為工部侍郎,甚允朝廷之望。然性躁急多言,又引賓客故人,晨夕飲謔,由是為時論所譏。時禮部尚書、信安王禕為朔方節度使,殿中侍御史李宙劾之,驛召將下獄。禕既申訴得理,融坐阿黨李宙,出為汝州
 刺史,在相凡百日而罷。



 裴光庭時兼御史大夫,又彈融交游朋黨及男受贓等事,貶昭州平樂尉。在嶺外歲餘,司農少卿蔣岑舉奏融在汴州回造船腳,隱沒鉅萬,給事中馮紹烈又深文案其事實,融於是配流巖州。地既瘴毒,憂恚發疾,遂詣廣府,將停留未還。都督耿仁忠謂融曰:「明公負朝廷深譴,以至於此,更欲故犯嚴命,淹留他境,仁忠見累,誠所甘心,亦恐朝廷知明公在此,必不相容也。」融遽還,卒於路。上聞之,思其舊功,贈臺州刺史。



 韋堅,京兆萬年人。父元珪,先天中,銀青光祿大夫,開元初,袞州刺史。堅姊為贈惠宣太子妃,堅妻又楚國公姜皎女,堅妹又為皇太子妃,中外榮盛,故早從官敘。二十五年,為長安令,以幹濟聞。與中貴人善,探候主意。見宇文融楊慎矜父子以勾剝財物爭行進奉而致恩顧,堅乃以轉運江淮租賦,所在置吏督察,以裨國之倉廩,歲益鉅萬。玄宗以為能。



 天寶元年三月,擢為陜郡太守、水陸轉運使。自西漢及隋,有運渠自關門西抵長安,以通
 山東租賦。奏請於咸陽擁渭水作興成堰,截灞、滻水傍渭東注,至關西永豐倉下與渭合。於長安東九里長樂坡下、滻水之上架苑墻,東面有望春樓,樓下穿廣運潭以通舟楫,二年而成。堅預於東京、汴、宋取小斛底船三二百只置於潭側,其船皆署牌表之。若廣陵郡船,即於栿背上堆積廣陵所出錦、鏡、銅器、海味;丹陽郡船,即京口綾衫段;晉陵郡船,即折造官端綾繡,會稽郡船,即銅器、羅、吳綾、絳紗;南海郡船,即玳瑁、真珠、象牙、沉香;豫
 章郡船,即名瓷、酒器、茶釜、茶鐺、茶碗;宣城郡船,即空青石、紙筆、黃連;始安郡船,即蕉葛、蚺蛇膽、翡翠。船中皆有米,吳郡即三破糯米、方丈綾。凡數十郡。駕船人皆大笠子、寬袖衫、芒屨,如吳、楚之制。先是,人間戲唱歌詞云:「得丁紇反體都董反紇那也,紇囊得體耶?潭裏船車鬧,揚州銅器多。三郎當殿坐,看唱《得體歌》。」至開元二十九年,田同秀上言「見玄元皇帝,云有寶符在陜州桃林縣古關令尹喜宅」,發中使求而得之,以為殊祥,改桃林為靈寶縣。
 及此潭成,陜縣尉崔成甫以堅為陜郡太守鑿成新潭,又致揚州銅器,翻出此詞,廣集兩縣官,使婦人唱之,言:「得寶弘農野,弘農得寶耶!潭裏船車鬧,揚州銅器多。三郎當殿坐,看唱《得寶歌》。」成甫又作歌詞十首,白衣缺胯綠衫,錦半臂,偏袒膊,紅羅抹額,於第一船作號頭唱之。和者婦人一百人,皆鮮服靚妝,齊聲接影,鼓笛胡部以應之。餘船洽進,至樓下,連檣彌亙數里,觀者山積。京城百姓多不識驛馬船墻竿,人人駭視。



 堅跪上諸郡輕貨,
 又上百牙盤食,府縣進奏,教坊出樂迭奏。玄宗歡悅,下詔敕曰:



 古之善政者,貴於足食,欲求富國者,必先利人。朕關輔之間,尤資殷贍,比來轉輸,未免艱辛,故置比潭,以通漕運。萬代之利,一朝而成,將允葉於永圖,豈茍求於縱觀。其陜郡太守韋堅,始終檢校,夙夜勤勞,賞以懋功,則惟常典。宜特與三品,仍改授一三品京官兼太守,判官等並即量與改轉。其專知檢校始末不離潭所者並孔目官,及至典選日,優與處分,仍委韋堅具名錄
 奏。應役人夫等,雖各酬傭直,終使役日多,並放今年地稅。且啟鑿功畢,舟楫已通,既涉遠途,又能先至,永言勸勵,稍宜甄獎。其押運綱各賜一中上考,準前錄奏。船夫等宜共賜錢二千貫,以充宴樂。外郡進上物,賜貴戚朝官。賜名廣運潭。



 時堅姊故惠宣太子妃亦出寶物供樓上鋪設,進食竟日而罷。



 李林甫以堅姜氏婿,甚狎之。至是懼其詭計求進,承恩日深,堅又與李適之善,益怒之,恐入為相,乃與腹心構成其罪。四月,進銀青光祿大
 夫、左散騎常侍、陜郡太守、水陸轉運使,勾當緣河及江淮南租庸轉運處置使並如故;又以判官元捴、豆盧友除監察御史。三年正月,堅又加兼御史中丞,封韋城男。九月,拜守刑部尚書,奪諸使,以楊慎矜代之。



 五載正月望夜,堅與河西節度、鴻臚卿皇甫惟明夜游,同過景龍觀道士房,為林甫所發,以堅戚里,不合與節將狎暱,是構謀規立太子。玄宗惑其言,遽貶堅為縉雲太守,惟明為播川太守。尋發使殺惟明於黔中,籍其資財。六月,又貶
 堅為江夏員外別駕。又構堅與李適之善,貶適之為宜春太守。七月,堅又長流嶺南臨封郡,堅弟將作少匠蘭、鄠縣令冰、兵部員外郎芝、堅男河南府戶曹諒並遠貶。至十月,使監察御史羅希奭逐而殺之,諸弟及男諒並死。堅妻姜氏,林甫以其久遭輕賤,特放還本宗。倉部員外郎鄭章貶南豐丞,殿中侍御史鄭欽說貶夜郎尉,監察御史豆盧友貶富水尉,監察御史楊惠貶巴東尉,連累者數十人。又敕嗣薛王琄夷陵郡員外別駕長任,其
 母隨男任;女婿新貶巴陵太守盧幼林長流合浦郡。肅宗時為皇太子,恐懼上表,稱與新婦離絕。七載,嗣薛王琄停,仍於夜郎郡安置,其母亦勒隨男。堅貶黜後,林甫諷所司發使於江淮、東京緣河轉運使,恣求堅之罪以聞,因之綱典船夫溢於牢獄,郡縣徵剝不止,鄰伍盡成裸形,死於公府,林甫死乃停。



 楊慎矜,隋煬帝玄孫也。曾祖隋齊王暕,祖正道,大業末,隨宇文化及至河北,為竇建德所破,因與其祖母
 蕭皇后入於建德軍,建德送於突厥處羅可汗牙。貞觀初,李靖擊破頡利可汗,胡酋康蘇密以蕭後及正道歸,授尚衣奉御。父隆禮,長安中天官郎中,神龍後,歷洛、梁、滑、汾、懷五州刺史,皆以清嚴能檢察人吏絕於欺隱聞。景雲中,以名犯玄宗上字,改為崇禮。開元初,擢為太府少卿,雖錢帛充牣,丈尺間皆躬自省閱,時議以為前後為太府者無與為比。擢拜太府卿,加銀青光祿大夫,進封弘農郡公。在職二十年,公清如一。年九十餘,授戶部尚
 書致仕。時太平且久,御府財物山積,以為經楊卿者無不精好,每歲勾剝省便出錢數百萬貫。



 慎矜沉毅有材幹,任氣尚朋執。初,為汝陽令,有能名。崇禮罷太府,玄宗訪其子堪委其父任者。宰臣以慎餘、慎矜,慎名三人皆勤恪清白有父風,而慎矜為其最,因拜監察御史,知太府出納。慎餘先為司農丞,除太子舍人,監京倉。尋丁父憂。二十六年服闋,累遷侍御史,仍知太府出納。慎名授大理評事,攝監察御史,充都含嘉倉出納使,甚承恩顧。慎
 矜於諸州納物者有水漬傷破及色下者,皆令本州征折估錢,轉市輕貨,州縣徵調,不絕於歲月矣。在臺數年,又專知雜事,風格甚高。



 天寶二年,遷權判御史中丞,充京畿採訪使,知太府出納使並如故。時右相李林甫握權,慎矜以遷拜不由其門,懼不敢居其任,固讓之,因除諫議大夫,兼侍御史,仍依舊知太府出納。以鴻臚少卿蕭諒為御史中丞,諒至臺,無所捴讓,頗不相能,竟出為陜郡太守。林甫以慎矜屈於己,復擢為御史中丞,仍充
 諸道鑄錢使,餘如故。



 時散騎常侍、陜郡太守韋堅兼御史中丞,為水陸漕運使,權傾宰相。侍御史王鉷推堅獄,慎矜引身中立以候望,鉷恨之,林甫亦憾焉。慎矜與鉷父瑨中外兄弟,鉷即表侄,少相狎,鉷入臺,慎矜為臺端,亦有推引。及鉷遷中丞,雖與鉷同列,每呼為王鉷,鉷恃與林甫善,漸不平之。五載,慎矜遷戶部侍郎,中丞、使如故。林甫見慎矜受主恩,心嫉之,又知王鉷於慎矜有間,又誘而啖之,鉷乃伺其隙以陷之。慎矜奪鉷職田,背詈
 鉷,詆其母氏,鉷不堪其辱。慎矜性疏快,素暱於鉷,嘗話讖書於鉷,又與還俗僧史敬忠游處,敬忠有學業。鉷於林甫構成其罪,云慎矜是隋家子孫,心規克復隋室,故蓄異書,與兇人來往,而說國家休咎。



 時天寶六載十一月,玄宗在華清宮,林甫令人發之。玄宗震怒,系之於尚書省,詔刑部尚書蕭隱之、大理卿李道邃、少卿楊璹、侍御史楊釗、殿中侍御史盧鉉同鞫之;又使京兆士曹吉溫往東京收慎矜兄少府少監慎餘、弟洛陽令慎名等
 雜訊之;又令溫於汝州捕史敬忠獲之,便赴行在所。先令盧鉉收太府少卿張瑄於會昌驛,系而推之,瑄不肯答辯。鉉百端拷訊不得,乃令不良枷瑄,以手力絆其足,以木按其足間,敝其枷柄向前,挽其身長校數尺,腰細欲絕,眼鼻皆血出,謂之「驢駒拔撅」,瑄竟不肯答。又使鉉與御史崔器入城搜慎矜宅,無所得,拷其小妻韓珠團,乃在豎櫃上作一暗函盛讖書等,鉉於袖中出而納之,詬以示慎矜。慎矜曰:「他日不見,今乃來,是命也。吾死也。」
 及溫以敬忠至戲水驛東十餘里,使證說之:「若至溫湯,即求首陳不可得矣。」去溫湯十餘里,敬忠乞紙筆於桑樹下具吐之。比見慎矜,敬忠證之,慎矜皆引實。二十五日,詔楊慎矜、慎餘、慎名並賜自盡;史敬忠決重杖一百;鮮於賁、範滔並決重杖,配流遠郡;慎矜外甥前通事舍人辛景湊決杖配流。義陽郡司馬、嗣虢王巨與敬忠相識,解官於南賓郡安置;太府少卿張瑄決六十,長流嶺南臨封郡,亦死於流所。慎矜兄弟並史敬忠莊宅官收,
 以男女配流嶺南諸郡;其張瑄、萬俟承暉、鮮於賁等準此配流。乃使臨察御史顏真卿送敕至東京,殿中侍御史崔寓引慎名,令河南法曹張萬頃宣敕示之。慎名見慎矜賜自盡,初尚撫膺,及聞慎餘及身皆爾,遂止。及宣敕了,慎名曰:「今奉聖恩,不敢稽留晷刻,但以寡姊老年,請作數行書以別之。」寓揖真卿,真卿許之。慎名神色不變,入房中作書曰:「拙於謀運,不能靜退。兄弟並命,唯姊尚存,老年孤煢,何以堪此!」書後又數條事。又宅中作一
 板池,池中魚一皆放之,遂縊而死。監察御史平冽齎敕至大理寺,慎餘聞死,合掌指天而縊。



 初,慎矜至溫湯,正食,忽見一鬼物長丈餘,硃衣冠幘,立於門扇後,慎矜叱之,良久不滅,以熱羹投之乃滅。無何,下獄死。兄弟甚友愛,事寡姊如母,皆偉儀形,風韻高朗,愛客喜飲,籍甚於時。慎名嘗覽鏡,見其須面神彩,有過於人,覆鏡嘆惋曰:「吾兄弟三人,盡長六尺餘,有如此貌、如此材而見容當代以期全,難矣!何不使我少體弱耶?」竟如其
 言。



 王鉷,太原祁人也。祖方翼,夏州都督,為時名將,生臣、瑨、珣。臣、瑨,開元初並歷中書舍人。珣,兵部侍郎、秘書監。鉷,即瑨之孽子。開元十年,為鄠縣尉、京兆尹稻田判官。二十四年,再遷監察御史。二十九年,累除戶部員外郎,常兼御史。天寶二年,充京和市和糴使,遷戶部郎中。三載,長安令柳升以賄敗。初,韓朝宗為京兆尹,引升為京令。朝宗又於終南山下為茍家觜買山居,欲以避世亂。玄宗怒,敕鉷推之,朝宗自高平太守貶為吳興別駕。又
 加鉷長春宮使。四載,加勾戶口色役使,又遷御史中丞,兼充京畿採訪使。五載,又為京畿、關內道黜陟使,又兼充關內採訪使。



 時右相李林甫怙權用事。志謀不利於東儲,以除不附己者,而鉷有吏乾,倚之轉深,以為己用。既為戶口色役使,時有敕給百姓一年復。鉷即奏徵其腳錢,廣張其數,又市輕貨,乃甚於不放。輸納物者有浸漬,折估皆下本郡徵納。又敕本郡高戶為租庸腳士,皆破其家產,彌年不了。恣行割剝,以媚於時,人用嗟怨。古制,
 天子六宮,皆有品秩高下,其俸物因有等差。唐法沿于周、隋,妃嬪宮官,位有尊卑,亦隨其品而給授,以供衣服鉛粉之費,以奉於宸極。玄宗在位多載,妃御承恩多賞賜,不欲頻於左右藏取之。鉷探旨意,歲進錢寶百億萬,便貯於內庫,以恣主恩錫齎。鉷云:「此是常年額外物,非征稅物。」玄宗以為鉷有富國之術,利於王用,益厚待之。丁嫡母憂,起復舊職,使如故。



 七載,又加檢察內作事,遷戶部侍郎,仍兼御史中丞,賜紫金魚袋。八載,兼充閑廄使及苑
 內營田五坊宮苑等使、隴右群牧都支度營田使,餘並如故。太白山人李渾言於金星洞見老人,云有玉版石記符,聖上長生久視。玄宗令鉷入山洞求而得之。因上尊號,加鉷銀青光祿大夫、都知總監及栽接等使。九載五月,兼京兆尹,使並如故。



 鉷威權轉盛,兼二十餘使,近宅為使院,文案堆積,胥吏求押一字,即累日不遂。中使賜遺,不絕於門,雖晉公林甫亦畏避之。林甫子岫為將作監,供奉禁中;鉷子準衛尉少卿,亦
 鬥雞供奉,每謔岫,岫常下之。萬年尉韋黃裳、長安尉賈季鄰常於事貯錢數百繩,名倡珍饌,常有備擬,以候準所適。又於宅側自有追歡之所。鉷與弟戶部郎中銲,召術士任海川游其門,問其相命,言有王否。海川震懼,潛匿不出。鉷懼洩其事,令逐之,至馮翊郡,得,誣以他事杖殺之。定安公主男韋會任王府司馬,聞之,話於私庭,乃被侍兒說於傭保者。或有憾於會,告於鉷,鉷遣賈季鄰收於長安獄,入夜縊之,明辰載尸還其家。會皇堂外甥,同產兄
 王繇尚永穆公主,而惕息不敢言。



 十載,封太原縣公,又兼殿中監。十一載四月,銲與故鴻臚少卿邢璹子糸宰情密累年,縡潛構逆謀,引右龍武軍萬騎刻取十一月殺龍武將軍,因燒諸城門及市,分數百人殺楊國忠及右相李林甫、左相陳希烈等。先期二日事發,玄宗臨朝,召鉷,上於玉案前過狀與鉷。鉷好弈棋,縡善棋,鉷因銲與之交故,至是意銲在縡處金城坊,密召之,日晏,始令捕賊官捕之。萬年尉薛榮光、長安尉賈季鄰等捕之,逢銲
 於化度寺門。季鄰為鉷所引用,為赤尉,銲謂之曰:「我與邢縡故舊,縡今反,恐事急妄相引,請足下勿受其言。」榮先等至縡門,縡等十餘人持弓刃突出,榮先等遂與格戰。季鄰以銲語白鉷,鉷肐謂之曰:「我弟何得與之有謀乎!」鉷與國忠共討逐縡,縡下人曰:「勿損太夫人。」國忠為劍南節度使,有隨身官以白國忠曰:「賊有號,不可戰。」須臾,驃騎大將軍、內侍高力士領飛龍小兒甲騎四百人討之,縡為亂兵所斬,擒其黨善射人韋瑤等以獻。國忠以
 白玄宗,玄宗以鉷委任深,必不與之知情,鉷與銲別生,嫉其富貴,故欲陷鉷耳,遂特原銲不問,然意欲鉷請罪之。上密令國忠諷之,國忠不敢洩上意,諷鉷曰:「且主上眷大夫深,今日大夫須割慈存門戶,但抗疏請罪郎中。郎中亦未必至極刑,大夫必存,何如並命!」鉷俯首久曰:「小弟先人餘愛,平昔頻有處分,義不欲舍之而謀存。」乃進狀。十二日,鉷入朝,左相陳希烈言語侵之,鉷恨之,憤訴言氣頗高。鉷朝回,於中書侍郎修表,令人進狀,門
 司已不納矣。須臾,敕希烈推之。鉷以表示宰相,林甫曰:「大夫後之矣。」遂不許。俄銲至,國忠問:「大夫知否?」銲未及應。侍御史裴冕恐銲引之,冕叱詈之曰:「足下為臣不忠,為弟不義。聖上以大夫之故,以足下為戶部郎中,又加五品,恩亦厚矣。大夫豈知縡事乎?」國忠愕然,謂銲曰:「實知,即不可隱;不知,亦不可妄引。」銲方曰:「七兄不知。」季鄰證其罪。及日暮,奏之。銲決杖死於朝堂,賜鉷自盡於三衛廚。明日,移於資聖寺廊下,裴冕言於國忠,令歸宅權
 斂之,又請令妻、女送墓所,國忠義而許之,令鉷判官齊奇營護之。男準除名,長流嶺南承化郡,備長流珠崖郡,至故驛殺之;妻薛氏及在室女並流。初,鉷與御史中丞、戶部侍郎楊慎矜親,且情厚,頗為汲引,及貴盛爭權,鉷附於李林甫,為所誘,陷慎矜家。經五年而鉷至赤族,豈天道歟!



 史臣曰:夫奸佞之輩,惟事悅人;聚斂之臣,無非害物。賈禍招怨,敗國喪身,罕不由斯道也。君人者,中智已降,亦
 心緣利動,言為甘聞,志雖慕於聖明,情不勝於嗜欲,徒有賢佐,無如之何,所以禮經戒其勿畜。宇文融、韋堅、楊慎矜、王鉷,皆開元之幸人也,或以括戶取媚,或以漕運承恩,或以聚貨得權,或以剝下獲寵,負勢自用,人莫敢違。張說、李林甫手握大權,承主恩顧,尚遭凌擯,以身下之,他人即可知也。然天道惡盈,器滿則覆,終雖不令,其弊已多,良可痛也。宋璟、裴耀卿、許景先獲居重任,因融薦之,此亦有鳳之一毛也。玄宗以聖哲之姿,處高明之
 位,未免此累,或承之羞,後之帝王,得不深鑒!



 贊曰:財能域人,聚則民散。如何帝王,志求餘羨。融、堅、矜、鉷,因利乘便。以徼寵榮,宜招後患。



\end{pinyinscope}