\article{卷一百八十九}

\begin{pinyinscope}

 ○韋仁壽陳君賓張允濟李桐客李素立孫至遠至遠子畬薛大鼎賈敦頤弟敦實李君球崔知溫高智周田仁會子歸道韋機孫岳嶽子景駿權懷恩叔祖萬紀馮元常弟元淑蔣儼王方翼薛季昶



 漢宣帝曰:「使政平訟息,民無愁嘆,與我共理,其惟良二千石乎!」故漢代命
 官,
 重外輕內,郎官出宰百里,郡守入作三公。世祖中興,尤深吏術,慎選名儒為輔相,不以吏事責功臣;政優則增秩賜金,績負則論輸左校。選任之道,皇漢其優。



 隋政不綱,彞倫斯紊。天子事巡游而務征伐,具僚逞側媚而竊恩權。是時朝廷無正人,方岳無廉
 吏。跨州連郡,莫非豺虎之流;佩紫懷黃,悉奮爪牙之毒。以至土崩不救,旋踵而亡。



 武德之初,餘風未殄。太宗皇帝削平亂跡,湔洗污風,唯思稼穡之艱,不以珠璣為寶。以是人知恥格,俗尚貞修,太平之基,率由茲道。洎天后、玄宗之代,貞元、長慶之間,或以卿士大夫涖方州,或以御史、郎官宰畿甸,行古道也,所病不能。



 自武德已還,歷年三百,其間岳牧,不乏循良。今錄其政術有聞,為之立傳,所冀表吏師而儆不恪也。



 韋仁壽,雍州萬年人也。大業末,為蜀郡司法書佐,斷獄平恕,其得罪者皆曰:「韋君所斷,死而無恨。」高祖入關,遣使定巴蜀,使者承制拜仁壽巂州都督府長史。時南寧州內附,朝廷每遣使安撫,類皆受賄,邊人患之,或有叛者。高祖以仁壽素有能名,令檢校南寧州都督,寄聽政於越巂,使每歲一至其地以慰撫之。仁壽將兵五百人至西洱河,承制置八州十七縣,授其豪帥為牧宰,法令清肅,人懷歡悅。及將還,酋長號泣曰:「天子遣公鎮撫南
 寧,何得便去?」仁壽以城池未立為辭,諸酋長乃相與築城,立廨舍,旬日而就。仁壽又曰:「吾奉詔但令巡撫,不敢擅住。」及將歸,蠻夷父老各揮涕相送。因遣子弟隨之入朝,貢方物,高祖大悅。仁壽復請徙居南寧,以兵鎮守。有詔特聽以便宜從事,令益州給兵送之。刺史竇軌害其功,托以蜀中山獠反叛,未遑遠略,不時發遣。經歲餘,仁壽病卒。



 陳君賓,陳鄱陽王伯山子也。仕隋為襄國太守。武德初,
 以郡歸款,封東陽公,拜邢州刺史。貞觀元年,累轉鄧州刺史。州邑喪亂之後,百姓流離。君賓至才期月,皆來復業。二年,天下諸州並遭霜澇,君賓一境獨免。當年多有儲積,蒲、虞等州戶口,盡入其境逐食。太宗下詔勞之曰:



 朕以隋末亂離,毒被海內;率土百姓,零落殆盡,州里蕭條,十不存一;寤寐思之,心焉若疚。是以日昃忘食,未明求衣,曉夜孜孜,惟以安養為慮。每見水旱降災,霜雹失所,撫躬責己,自慚德薄。恐貧乏之黎庶,不免饑餒;傾竭
 倉廩,普加賑恤。其有一人絕食,若朕奪之,分命庶僚,盡心匡救。去年關內六州及蒲、虞、陜、鼎等復遭亢旱,禾稼不登,糧儲既少,遂令分房就食。比聞刺史以下及百姓等並識朕懷,逐糧戶到,遞相安養,回還之日,各有贏糧。乃別齎布帛,以申贈遺,如此用意,嘉嘆良深。一則知水旱無常,彼此遞相拯贍,不慮兇年。二則知禮讓興行,輕財重義,四海士庶,皆為兄弟。變澆薄之風,敦仁慈之俗,政化如此,朕復何憂。其安置客口,官人支配得所,並令
 考司錄為功最。養戶百姓,不吝財帛,已敕主者免今年調物。宜知此意,善相勸勉。



 其年,入為太府少卿,轉少府少監。九年,坐事除名。後起授虔州刺史,卒。



 張允濟,青州北海人也。隋大業中為武陽令,務以德教訓下,百姓懷之。元武縣與其鄰接,有人以牸牛依其妻家者八九年,牛孳產至十餘頭;及將異居,妻家不與,縣司累政不能決。其人詣武陽質於允濟。允濟曰:「爾自有令,何至此也?」其人垂泣不止,具言所以。允濟遂令左右
 縛牛主,以衫蒙其頭,將詣妻家村中,云捕盜牛賊,召村中牛悉集,各問所從來處。妻家不知其故,恐被連及,指其所訴牛曰:「此是女婿家牛也,非我所知。」允濟遂發蒙,謂妻家人曰:「此即女婿,可以牛歸之。」妻家叩頭服罪。元武縣司聞之,皆大慚。又嘗道逢一老母種蔥者,結庵守之。允濟謂母曰:「但歸,不煩守也。若遇盜,當來告令。」老母如其言,居一宿而蔥大失。母以告允濟。悉召蔥地十里中男女畢集,允濟呼前驗問,果得盜蔥者。曾有行人候
 曉先發,遺衫於路,行十數里方覺。或謂曰:「我武陽境內,路不拾遺,但能回取,物必當在。」如言果得。遠近稱之。政績尤異。



 遷高陽郡丞,時無郡將,允濟獨統大郡,吏人畏悅。及賊帥王須拔攻圍,時城中糧盡,吏人取槐葉槁節食之,竟無叛者。貞觀初,累遷刑部侍郎,封武城縣男。出為幽州刺史,尋卒。



 李桐客,冀州衡水人也。仕隋為門下錄事。大業末,煬帝幸江都,時四方兵起,謀欲徙都丹陽,召百僚會議。公卿
 希旨,俱言「江右黔黎,皆思望幸,巡狩吳會,勒石紀功,復禹之跡,今其時也。」桐客獨議曰:「江南卑濕,地狹州小,內奉萬乘,外給三軍,吳人力屈,不堪命。且逾越險阻,非社稷之福。」御史奏桐客謗毀朝政,僅而獲免。後隋滅,從宇文化及至黎陽,轉沒竇建德。建德平,太宗召授秦府法曹參軍。貞觀初,累遷通、巴二州。所在清平流譽,百姓呼為慈父。後卒於家。



 李素立,趙州高邑人,北齊梁州刺史義深曾孫也。祖駼,
 散騎常侍。父政藻,隋水部郎中,大業末充使淮南,為盜所殺。素立,武德初為監察御史。時有犯法不至死者,高祖特命殺之,素立諫曰:「三尺之法,與天下共之,法一動搖,則人無所措手足。陛下甫創鴻業,遐荒尚阻,奈何輦轂之下,便棄刑書?臣忝法司,不敢奉旨。」高祖從之。自是屢承恩顧。素立尋丁憂,高祖令所司奪情,授以七品清要官,所司擬雍州司戶參軍。高祖曰:「此官要而不清。」又擬秘書郎。高祖曰:「此官清而不要。」遂擢授侍御史,高祖
 曰:「此官清而復要。」



 貞觀中,累轉揚州大都督府司馬。時突厥鐵勒部相率內附,太宗於其地置瀚海都護府以統之,以素立為瀚海都護。又有闕泥孰別部,猶為邊患。素立遣使招諭降之。夷人感其惠,率馬牛以饋素立,素立唯受其酒一杯,餘悉還之。為建立廨舍,開置屯田。久之,轉綿州刺史。永徽初,遷蒲州刺史,及將之任,所餘糧儲及什物,皆令州司收之,唯齎己之書籍而去。道病卒,高宗聞而特為廢朝一日,謚曰平。



 其孫至遠,有重名。長
 壽中為天官郎中。內史李昭德重其才,薦於則天,擢令知流內選事。或勸至遠謝其私恩,至遠曰:「李公以公見用,豈得以私謁也。」竟不謝,遂為昭德所銜,因事出為壁州刺史卒。



 至遠子畬,初為汜水主簿。處事敏速,有聲稱,雖村童廁養之輩,一閱之後,無不知替代姓名者。累轉國子司業。事母甚謹,閨門邕睦,累代同居。每歲時拜慶,長幼男女,咸有禮節。及妻卒,時母已先病,畬恐傷母意,約家人不令哭聲使聞於母,朝夕定省,不曾見其憂念
 之色,士友甚以此稱之。及母終,過毀,卒於喪。



 至遠弟從遠,景雲中歷黃門侍郎、太府卿。



 素立從兄子游道,則天時官至冬官尚書、同鳳閣鸞臺三品。



 薛大鼎,蒲州汾陽人,周太子少傅博平公善孫也。父粹,隋介州長史。漢王諒謀反,授絳州刺史,諒敗伏誅。大鼎以年幼免死,配流辰州,後得還鄉里。義旗初建,於龍門謁高祖,因說:「請勿攻河東,從龍門直渡,據永豐倉,傳檄遠近,則足食足兵。既總天府,據百二之所,斯亦拊背扼
 喉之計。」高祖深然之。時將士咸請先攻河東,遂從眾議。授大將軍府察非掾。



 貞觀中,累轉鴻臚少卿、滄州刺史。州界有無棣河,隋末填廢。大鼎奏開之,引魚鹽於海。百姓歌之曰:「新河得通舟楫利,直達滄海魚鹽至。昔日徒行今騁駟,美哉薛公德滂被。」大鼎又以州界卑下,遂決長蘆及漳、衡等三河,分洩夏潦,境內無復水害。時與瀛州刺史賈敦頤、曹州刺史鄭德本,俱有美政,河北稱為「鐺腳刺史」。



 永徽四年,授銀青光祿大夫,行荊州大都督
 府長史。明年卒。有二子:克構、克勤。



 克構,天授中官至麟臺監。克勤,歷司農少卿,為來俊臣所陷伏誅。克構坐配流嶺表而死。



 賈敦頤,曹州冤句人也。貞觀中,歷遷滄州刺史。在職清潔,每入朝,盡室而行,唯弊車一乘,羸馬數匹;羈勒有闕,以繩為之,見者不知其刺史也。二十三年,轉瀛州刺史。州界滹沱河及滱水,每歲泛溢,漂流居人,敦頤奏立堤堰,自是無復水患。



 永徽五年,累遷洛州刺史。時豪富之
 室,皆籍外占田;敦頤都括獲三千餘頃,以給貧乏。又發奸摘伏,有若神明。尋卒。弟敦實。



 敦實,貞觀中為饒陽令,政化清靜,老幼懷之。時敦頤復授瀛州刺史。舊制,大功以上不復連官。朝廷以其兄弟在職,俱有能名,竟不遷替。咸亨元年,累轉洛州長史,甚有惠政。時洛陽令楊德幹杖殺人吏,以立威名,敦實曰:「政在養人,義須存撫,傷生過多,雖能亦不足貴也。」常抑止德乾,德干亦為之稍減。四年,遷太子右庶子。



 初敦頤為洛州刺史,百姓共樹
 碑於大市通衢;及敦實去職,復刻石頌美,立於兄之碑側,時人號為「棠棣碑」。敦實後為懷州刺史。永淳初,以年老致仕。及病篤,子孫迎醫視之,敦實曰:「未聞良醫能治老也。」終不服藥。垂拱四年卒,時年九十餘。



 子膺福,先天中,歷左散騎常侍、昭文館學士,坐預竇懷貞等謀逆伏誅。



 李君球,齊州平陵人也。父義滿,屬隋亂,糾合宗黨,保固村閭,外盜不敢侵逼,以功累授齊郡通守。武德初,遠
 申誠款,詔以其宅為譂州,仍拜為總管,封平陵郡公。



 君球少任俠,頗涉書籍。貞觀中,齊州都督齊王據州城舉兵作亂,君球與兄子行均守縣城。事平,太宗聞而嘉之,擢授游擊將軍,仍改其本縣為全節縣。君球累補左驍衛、義全府折沖都尉。



 龍朔三年,高宗將伐高麗,君球上疏諫曰:



 臣聞心之病者,不能緩聲;事之急者,不能安言;性之慈者,不能隱情。且食君之祿者,死君之事。今臣食陛下之祿矣,其敢愛身乎?臣聞《司馬法》曰:「國雖大,好戰必
 亡;天下雖安,忘戰必危。」兵者,兇器,戰者,危事,故聖主明王重行之也。愛人力之盡,恐府庫之殫,懼社稷之危,生中國之患。故古人云:「務廣德者昌,務廣地者亡。」昔秦始皇好戰不已,至於失國,是不愛其內而務其外故也。漢武遠討朔方,殆乎萬里,廣拓南海,分為八郡;終於戶口減半,國用空虛。至於末年,方垂哀痛之詔,自悔其失。



 彼高麗者,闢側小醜,潛藏山海之間,得其人不足以彰聖化;棄其地不足以損天威。何至乎疲中國之人,傾府庫
 之實,使男子不得耕耘,女子不得蠶織!陛下為人父母,不垂惻隱之心,傾其有限之貲,貪於無用之地。設令高麗既滅,即不得不發兵鎮守,少發則兵威不足,多發則人心不安,是乃疲於轉戍,萬姓無聊生也。萬姓無聊,即天下敗矣!天下既敗,陛下何以自安?故臣以為征之不如不征,滅之不如不滅。



 書奏不納。



 尋遷蔚州刺史。未行,改為興州刺史。累遷揚州大都督府長史。政尚嚴肅,人吏憚之,盜賊屏跡,高宗頻降書勞勉。時有吐谷渾犯塞,
 以君球素有威重,轉為靈州都督。尋卒官。



 崔知溫,許州鄢陵人。祖樞,司農卿。父義真,陜州刺史。知溫初為左千牛。麟德中,累轉靈州都督府司馬。州界有渾、斛薛部落萬餘帳,數侵掠居人,百姓咸廢農業,習騎射以備之。知溫表請徙於河北,斛薛不願遷移。時將軍契苾何力為之言於高宗,遂寢其奏。知溫前後十五上詔,竟從之,於是百姓始就耕獲。後斛薛入朝,因過州謝曰:「前蒙奏徙河北,實有怨心。然牧地膏腴,水草不乏,部
 落日富,始荷公恩。」拜伏而去。



 知溫四遷蘭州刺史。會有黨項三萬餘眾來寇州城,城內勝兵既少,眾大懼,不知所為。知溫使開城門延賊,賊恐有伏,不敢進。俄而將軍權善才率兵來救,大破黨項之眾。善才因其降,欲盡坑之,以絕後患,知溫曰:「弗逆克奔,古人之善戰。誅無噍類,禍及後昆。又溪谷崢嶸,草木幽蔚,萬一變生,悔之何及!」善才然其計。又欲分降口五百人以與知溫。知溫曰:「向論安危之策,乃公事也,豈圖私利哉!」固辭不受。黨項餘
 眾由是悉來降附。



 知溫累遷尚書左丞,轉黃門侍郎、同中書門下三品,兼修國史。永隆二年七月,遷中書令。永淳三年三月卒,年五十七,贈荊州大都督。



 子泰之,開元中官至工部尚書。



 少子諤之。諤之,神龍初為將作少匠,預誅張易之有功,封博陵縣侯,賜實封二百戶。開元初,累遷少府監。



 知溫兄知悌。知悌,高宗時官至戶部尚書。



 高智周,常州晉陵人。少好學,舉進士。累補費縣令,與丞、尉均分俸錢,政化大行,人吏刊石以頌之。尋授秘書郎、
 弘文館直學士,預撰《瑤山玉彩》、《文館辭林》等。三遷蘭臺大夫。時孝敬在東宮,智周與司文郎中賀凱、司經大夫王真儒等,俱以儒學詔授為侍讀。總章元年,請假歸葬其父母,因謂所親曰:「知進而不知退,取患之道也。」乃稱疾去職。



 俄起授壽州刺史,政存寬惠,百姓安之。每行部,必先召學官。見諸生,試其講誦,訪以經義及時政得失,然後問及墾田獄訟之事。咸亨二年,召拜正諫大夫,兼檢校禮部侍郎。尋遷黃門侍郎、同中書門下三品,兼修
 國史。俄轉御史大夫,累表固辭煩劇之任,高宗嘉其意,拜右散騎常侍。又請致仕,許之。永淳二年十月,卒於家,年八十二,贈越州都督府。



 智周少與鄉人蔣子慎善,同詣善相者,曰:「明公位極人臣,而胤嗣微弱;蔣侯官祿至薄,而子孫轉盛。」子慎後累年為建安尉卒,其子繪來謁智周。智周已貴矣,曰:「吾與子父有故,子復有才。」因以女妻之。永淳中,為緱氏尉、鄭州司兵卒。



 繪子捷,舉進士。開元中,歷臺省,仕至湖、延二州刺史。子貴,贈揚州大都督。



 捷子冽、渙,並進士及第。冽,歷禮、吏、戶部三侍郎,尚書左丞;渙,天寶末給事中,永泰初右散騎常侍。高氏殄滅已久,果符相者之言。初,冽兄弟在父艱,廬於墓側,植松柏千餘株,又同時榮貴,人推其友愛。



 冽子鏈,渙子銖,亦進士舉。



 田仁會,雍州長安人。祖軌,隋幽州刺史、信都郡公。父弘,陵州刺史,襲信都郡公。仁會,武德初應制舉,授左衛兵曹,累遷左武候中郎將。貞觀十八年,太宗征遼發後,薛
 延陀數萬騎抄河南,太宗令仁會及執失思力率兵擊破之,逐北數百里,延陀脫身走免。太宗嘉其功,降璽書慰勞。



 永徽二年,授平州刺史,勸學務農,稱為善政。轉郢州刺史,屬時旱,仁會自曝祈禱,竟獲甘澤。其年大熟,百姓歌曰:「父母育我田使君,精誠為人上天聞。田中致雨山出雲,倉廩既實禮義申。但願常在不患貧。」五遷勝州都督。州界有山賊阻險,劫奪行李,仁會發騎盡捕殺之。自是外戶不閉,盜賊絕跡。入為太府少卿。



 麟德二年,轉
 右金吾將軍,所得祿俸,估外有餘,輒以納官,時人頗譏其邀名。仁會強力疾惡,晝夜巡警,自宮城至於衢路,絲毫越法,無不立發。每日庭引百餘人,躬自閱罰,略無寬者。京城貴賤,咸畏憚之。



 時有女巫蔡氏,以鬼道惑眾,自云能令死者復生,市里以為神明,仁會驗其假妄,奏請徙邊。高宗曰:「若死者不活,便是妖妄;若死者得生,更是罪過。」竟依仁會所奏。



 仁會,總章二年遷太常正卿,咸亨初又轉右衛將軍,以年老致仕。儀鳳四年卒,年七十八,
 謚曰威。神龍中,以子歸道贈戶部尚書。



 歸道,弱冠明經舉。長壽中累補司賓丞,仍通事舍人內供奉。久之,轉左衛郎將。



 聖歷初,突厥默啜遣使請和,制遣左豹韜衛將軍閻知微入蕃,冊為立功報國可汗。默啜又遣使入朝謝恩,知微遇諸途,便與之緋袍、銀帶,兼表請蕃使入都日,大備陳設。歸道上言曰:「突厥背恩積稔,悔過來朝,宜待聖恩,寬其罪戾,解辮削衽,須稟天慈。知微擅與袍帶,國家更將何物充賜?望反初服,以俟朝恩。且小蕃使到,
 不勞大備之儀。」則天然之。



 及默啜將至單于都護府,乃令歸道攝司賓卿迎勞之。默啜又奏請六胡州及單于都護府之地,則天不許。默啜深怨,遂拘縶歸道,將害之。歸道辭色不撓,更責以無厭求請,兼喻其禍福,默啜意稍解。會有制賜默啜粟三萬石、雜彩五萬段、農器三千事,並許之結婚。於是歸道得還,遂面陳默啜不利之狀,請加防禦,則天納焉。頃之,默啜果叛,挾閻知微入寇趙、定等州。擢拜歸道夏官侍郎,甚見親委。累遷左金吾將
 軍、司膳卿,兼押千騎。未幾,除尚方監,加銀青光祿大夫。轉殿中監,仍令依舊押千騎,宿衛於玄武門。



 敬暉等討張易之、昌宗也,遣使就索千騎。歸道既先不預謀,拒而不與。及事定,暉等將誅之,歸道執辭免,令歸私第。中宗嘉其忠壯,召拜太僕少卿,驟除殿中少監、右金吾將軍。歲餘病卒,贈輔國大將軍,追封原國公,中宗親為文以祭之。



 子賓庭,開元中為光祿卿。



 韋機,雍州萬年人。祖元禮,隋浙州刺史。父恪,洛州別駕。
 機,貞觀中為左千牛胄曹,充使往西突厥,冊立同俄設為可汗。會石國反叛,路絕,三年不得歸。機裂裳錄所經諸國風俗物產,名為《西征記》。及還,太宗問蕃中事,機因奏所撰書。太宗大悅,擢拜朝散大夫,累遷至殿中監。



 顯慶中為檀州刺史。邊州素無學校,機敦勸生徒,創立孔子廟,圖七十二子及自古賢達,皆為之贊述。會契苾何力東討高麗,軍眾至檀州,而灤河泛漲,師不能進,供其資糧,數日不乏。何力全師還,以其事聞。高宗以
 為能,超拜司農少卿,兼知東都營田,甚見委遇。有宦者於苑中犯法,機杖而後奏。高宗嗟賞,賜絹數十疋,謂曰:「更有犯者,卿即鞭之,不煩奏也。」



 上元中,遷司農卿,檢校園苑。造上陽宮,並移中橋從立德坊曲徙於長夏門街,時人稱其省功便事。有道士硃欽遂為天后所使,馳傳至都,所為橫恣。機囚之,因密奏曰:「道士假稱中宮驅使,依倚形勢,臣恐虧損皇明,為禍患之漸。」高宗特發中使慰諭機,而欽遂配流邊州,天后由是不悅。



 儀鳳中,機坐家人犯盜,為憲司所
 劾,免官。永淳中,高宗幸東都,至芳桂宮驛,召機,令白衣檢校園苑。將復本官,為天后所擠而止,俄令檢校司農少卿事,會卒。



 子餘慶。餘慶官至右驍衛兵曹,早卒。餘慶子嶽。



 嶽亦以吏乾著名,則天時,累轉汝州司馬。會則天幸長安,召拜尚舍奉御,從駕還京,因召見。則天謂曰:「卿是韋機之孫,勤幹固有家風也。卿之家事,朕悉知之。」因問家人名,賞慰良久。尋拜太原尹。岳素不習武,固辭邊任。由是忤旨,左遷
 宋州長史,歷海、虢二州刺史,所在皆著威名。睿宗時,入為殿中少監,甚承恩顧。及竇懷貞、李晉等伏誅,以岳嘗與交往,為姜皎所陷,左遷渠州別駕,稍遷陜州刺史。開元中,卒於潁州別駕。嶽子景駿。



 景駿明經舉,神龍中,累轉肥鄉令。縣北界漳水,連年泛溢。舊堤迫近水漕,雖修築不息,而漂流相繼。景駿審其地勢,拓南數里,因高築堤。暴水至,堤南以無患,水去而堤北稱為腴田。漳水舊有架柱長橋,每年修葺,景駿又改造為浮橋。自是無復水患,至今賴焉。時河北饑,景駿躬撫合境村閭,必通贍恤,貧弱獨免流離。及去任,人吏立碑頌德。



 開元中,為貴鄉令。縣人有母子相訟者,景駿謂之曰:「吾少孤,每見人養親,自恨終天無分,汝幸在溫清之地,何得如此?錫類不行,令之罪也。」因垂泣嗚咽,仍取《孝經》付令習讀之。於是母子感悟,各請改悔,遂稱慈孝。



 累轉趙州長史,路由肥鄉,人吏驚喜,競來犒餞,留連經日。有童稚數人,年甫十餘歲,亦在其中,景駿謂曰:「計
 吾為此令時,汝輩未生,既無舊恩,何殷勤之甚也?」咸對曰:「此間長宿傳說,縣中廨宇、學堂、館舍、堤橋,並是明公遺跡。將謂古人,不意親得瞻睹,不覺欣戀倍於常也。」其為人所思如此。



 十七年,遷房州刺史。州帶山谷,俗參蠻夷,好淫祀而不修學校。景駿始開貢舉,悉除淫祀。又通狹路,並造傳館,行旅甚以為便。二十年,轉奉先令,未行而卒。



 權懷恩,雍州萬年人,周荊州刺史、千金郡公景宣玄孫也。其先自天水徙家焉。祖弘壽,大業末為臨汾郡司倉書佐。高祖鎮晉陽,引判留守事。以從義師之
 功,累轉秦王府長史,太宗遇之甚厚。又從平王世充,拜太僕卿。累封盧國公卒,謚曰恭。父知讓,襲爵,官至博州刺史。



 懷恩初以廕授太子洗馬。咸亨初,累轉尚乘奉御,襲爵盧國公。時有奉乘安畢羅善於調馬,甚為高宗所寵。懷恩奏事,遇畢羅在帝左右戲無禮,懷恩退而杖之四十。高宗知而嗟賞之,謂侍臣曰:「懷恩乃能不避強御,真良吏也。」即日拜萬年令。為政清肅,令行禁止,前後京縣令無及之者。後
 歷慶、萊、衛、邢四州刺史,洛州長史。



 懷恩姿狀雄毅,束帶之後,妻子不敢仰視。所歷皆以威名御下,人吏重足而立。俄出為宋州刺史。時汴州刺史楊德干亦以嚴肅與懷恩齊名。至是懷恩路由汴州,德乾送之出郊,懷恩見新橋中途立木以禁車過者,謂德乾曰:「一言處分豈不得,何用此為?」德干大慚,時議以為不如懷恩也。轉益州大都督府長史,尋卒。



 侄楚璧,官至左領軍衛兵曹參軍。開元十年,駕在東都,楚璧乃與故兵部尚書李迥秀男齊損、從祖弟金吾淑、陳倉尉、盧玢及京城左屯營押官長上折沖周履濟、楊楚劍、元令琪等舉
 兵反。立楚璧兄子梁山,年十五,詐稱襄王男,號為光帝。擁左屯營兵百餘人,梯上景風門,逾城而入,踞長樂恭禮門。入宮城,求留守、刑部尚書王志愔,不獲。屬天曉,屯營兵自相翻覆,盡殺梁山等。傳首東都,楚璧並坐籍沒。



 懷恩叔祖萬紀。萬紀性強正,好直言。貞觀中,為治書侍御史,以公事奏劾魏徵、溫彥博等,太宗以為不避豪貴,甚禮之。遷尚書左丞,封冀氏男,再轉齊王祐府長史。祐既失德,數匡正之,竟為祐所殺,語在《祐傳》。祐既死,贈萬紀齊州都督、武都公,謚
 曰敬。



 子玄福,高宗時為兵部侍郎。



 馮元常,相州安陽人,自長樂徙家焉,北齊右僕射子琮曾孫也。舉明經。高宗時,累遷監察御史,為劍南道巡察使,興利除害,蜀土賴焉。永淳中,為尚書左丞。元常清鑒有理識,甚為高宗之所賞。嘗密奏「中宮權重,宜稍抑損」,高宗雖不能用,深以其言為然。則天聞而甚惡之。及臨朝,四方承旨,多獻符瑞。嵩陽令樊文進瑞石,則天命於朝堂示百官。元常奏言:「狀涉諂偽,不可誣罔士庶。」則天不悅,出為隴州
 刺史。



 俄而天下嶽牧集乾陵會葬,則天不欲元常赴陵所,中途改授眉州刺史。劍南先時光火賊夜掠居人,晝潛山谷。元常至,喻以恩信,許其首露,仍切加捕逐,賊徒舍器杖,面縛自陳者相繼。又轉廣州都督,便道之任,不許詣都。



 尋屬安南首領李嗣仙殺都護劉延祐,剽陷州縣,敕元常討之。率士卒濟南海,先馳檄示以威恩,喻以禍福。嗣仙徒黨多相率歸降,因縱兵誅其魁首,安慰居人而旋。雖屢有政績,則天竟不賞之。尋為酷吏周興所陷,追赴都,下獄死。



 元常閨門雍肅,雅有禮度,雖小功之喪,未嘗寢於私室,甚為士類所稱。



 從父弟元淑,則天時為清漳令,政有殊績,百姓號為神明。又歷浚儀、始平二縣令,皆單騎赴職,未嘗以妻子之官。所乘馬,午後則不與芻,雲令其作齋。身及奴僕,每日一食而已。俸祿之餘,皆供公
 用,並給與貧士。人或譏其邀名,元淑曰:「此吾本性,不為苦也。」中宗時,降璽書勞勉,仍令史官編其事跡。卒於祠部郎中。



 蔣儼,常州義興人。貞觀中,為右屯衛兵曹參軍。太宗將征遼東,募使高麗者,眾皆畏憚。儼謂人曰:「主上雄略,華夷畏威,高麗小蕃,豈敢圖其使者。縱其凌虐,亦是吾死所也。」遂出請行。及至高麗,莫離支置於窟室中,脅以兵刃,終不屈撓。會高麗敗,得歸。太宗奇之,拜朝散大夫。再遷幽州司馬。以善政為巡察使劉祥道所薦,擢為會州刺史。再遷殿中少監,數陳意見,高宗每優納之。再轉蒲州刺史。蒲州戶口殷劇,前後刺史,多不稱職。儼下車未幾,令行禁止,稱為良牧。



 永淳元年,拜太僕卿;以父名卿,固辭,乃除太子右衛副率。時征隱士田游巖為太子洗馬,在宮竟無匡輔。儼乃貽書以責之曰:「足下負巢、由之峻節,傲唐、虞之聖主。養煙霞之逸氣,守林壑之遁情,有年載矣!故能聲出區宇,名流海內。主上屈萬乘之重,申三顧之榮,遇子以商山之客,待子以不臣之禮。將以輔導儲貳,漸染芝蘭耳。皇太子春秋鼎盛,聖道未周,拾遺補闕,臣子恆務。僕以不才,猶參廷諜,誠以素非德望,位班卒伍,言以人廢,不蒙採掇。足下受調護之寄,是可言之秋;唯唯而無一談,悠悠以卒年歲。向使不飡周粟,僕何敢言!祿及親矣,將何酬塞?想為不達,謹書起予。」游巖竟不能答。



 儼尋檢校太常卿。文明中,封義興縣子,歷右衛大將軍、太子詹事,以年老致仕。垂拱三年卒於家,年七十八。文集五卷。



 王方翼,並州祁人也,高宗王庶人從祖兄也。祖裕,武德初隋州刺史。裕妻即高祖妹同安大長公主也。太宗時,以公主屬尊年老,特加敬異,數幸其第,賞賜累萬。方翼父仁表,貞觀中為岐州刺史。仁表卒,妻李氏為主所斥,居於鳳泉別業。時方翼尚幼,乃與傭保齊力勤作,苦心計。功不虛棄,數年闢田數十頃,修飾館宇,列植竹木,遂為富室。公主卒後,歸長安。友人趙持滿犯罪被誅,暴尸於城西,親戚莫敢收視。方翼嘆曰:「欒布之哭彭越,大義
 也;周文之掩朽骼,至仁也。絕友之義,蔽主之仁,何以事君?」乃收其尸,具禮葬之。高宗聞而嘉嘆,由是知名。



 永徽中累授安定令。誅大姓皇甫氏,盜賊止息,號為善政。五遷肅州刺史。時州城荒毀,又無壕塹,數為寇賊所乘。方翼發卒浚築,引多樂水環城為壕。又出私財造水碾磑,稅其利以養饑餒,宅側起舍十餘行以居之。屬蝗儉,諸州貧人死於道路,而肅州全活者甚眾,州人為立碑頌美。



 會吏部侍郎裴行儉西討遮匐,奏方翼為副,兼檢校
 安西都護。又築碎葉鎮城,立四面十二門,皆屈曲作隱伏出沒之狀,五旬而畢。西域諸胡競來觀之,因獻方物。



 永隆中,車簿反叛,圍弓月城。方翼引兵救之,至伊麗河。賊前來拒,因縱擊。大破之,斬首千餘級。俄而二姓咽曲悉發眾十萬,與車簿合勢,以拒方翼。屯兵熱海,與賊連戰,流矢貫臂,徐以佩刀截之,左右莫有覺者。既而所將蕃兵懷貳,謀執方翼以應賊。方翼密知之,悉召會議,佯出軍資以賜之。續續引去,便令斬之。會大風,又振金鼓
 以亂其聲,遂誅七千餘人。因遣裨將分道討襲咽曲等。賊既無備,因是大潰,擒首領突騎施等三百人,西域遂定。以功遷夏州都督。屬牛疫,無以營農,方翼造人耕之法,施關鍵,使人推之,百姓賴焉。



 永淳二年,詔徵方翼,將議西域之事,於奉天宮謁見,賜食與語。方翼衣有舊時血漬之處,高宗問其故,方翼具對熱海苦戰之狀。高宗使袒視其瘡,嘆曰:「吾親也。」賞賜甚厚。俄屬綏州白鐵餘舉兵反,乃詔方翼副程務挺討之。賊平,封太原郡公。



 則
 天臨朝,以方翼是庶人近屬,陰欲除之。及程務挺被誅,以方翼與務挺連職素善,追赴都下獄,遂流於崖州而死。



 子寶、珣、瑨,並知名。寶、瑨,開元中皆為中書舍人;珣,至秘書監。



 薛季昶,絳州龍門人也。則天初,上封事,解褐拜監察御史。頻按制獄稱旨,累遷御史中丞。萬歲通天元年,夏官郎中侯味虛統兵討契丹不利,奏言「賊徒熾盛,常有蛇虎導其軍」。則天命季昶按驗其狀,便為河北道按察使。
 季昶先馳至軍,斬味虛以聞。又有槁城尉吳澤者,貪虐縱橫,嘗射殺驛使,截百姓子女發以為髢,州將不能制,甚為人吏所患。季昶又杖殺之。由是威震遠近,州縣望風懾懼。然後布以恩信,旌揚善吏。有汴州孝女李氏,年八歲,父卒,柩殯在堂十餘載,每日哭臨無限。及年長,母欲嫁之。遂截發自誓,請在家終養。及喪母,號毀殆至滅性,家無丈夫,自營棺槨,州里欽其至孝,送葬者千餘人。葬畢,廬於墓側,蓬頭跣足,負土成墳,手植松柏數百株。
 季昶列上其狀,有制特表門閭,賜以粟帛。



 久視元年,季昶自定州刺史入為雍州長史,威名甚著,前後京尹,無及之者。俄遷文昌左丞,歷魏、陜二州刺史。長安末,為洛州長史,所在皆以嚴肅為政。



 神龍初,以預誅張易之兄弟功,加銀青光祿大夫,拜戶部侍郎。時季昶勸敬暉等因兵勢殺武三思。暉等不從,竟以此敗,語在《暉傳》。季昶亦因是累貶,自桂州都督授儋州司馬。初,季昶與昭州首領周慶立及廣州司馬光楚客不協。及將之儋州,懼
 慶立見殺,將往廣州,又惡楚客,乃嘆曰:「薛季昶行事至是耶!」因自制棺,仰藥而死。



 睿宗即位,下制曰:「故儋州司馬薛季昶,剛干義烈。早承先顧,驅策中外,績譽昭宣;有莊、湯之推舉,同汲黯之強直。屬醜正操衡,除其異己,橫加竄責,卒至殂亡。言念忠冤,有懷嘉悼。可贈左御史大夫,仍同敬暉等例,與一子官。」



\end{pinyinscope}