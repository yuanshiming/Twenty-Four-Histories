\article{卷一百五十五}

\begin{pinyinscope}

 ○高崇
 文子承簡伊慎硃忠亮劉昌裔範希朝王鍔子稷閻巨源孟元陽趙昌



 高崇文,其先渤海人。崇文生幽州,樸厚寡言,少從平盧
 軍。貞元中,隨韓全義鎮長武城,治軍有聲。五年夏,吐蕃三萬寇寧州,崇文率甲士三千救之,戰於佛堂原,大破之,死者過半。韓全義入覲,崇文掌行營節度留務,遷兼御史中丞。十四年,為長武城使,積粟練兵,軍聲大振。



 永貞元年冬,劉闢阻兵,朝議討伐,宰臣杜黃裳以為獨任崇文,可以成功。元和元年春,拜檢校工部尚書、兼御史大夫,充左神策行營節度使,兼統左右神策、奉天麟游諸鎮兵以討闢。時宿將專征者甚眾,人人自謂當選,及
 詔出大驚。崇文在長武城,練卒五千,常若寇至。及是,中使至長武,卯時宣命,而辰時出師五千,器用無闕者。軍至興元,軍中有折逆旅之匕箸,斬之以徇。西從閬中入,遂卻劍門之師,解梓潼之圍,賊將邢泚遁歸。屯軍梓州,因拜崇文為東川節度使。先是,劉闢攻陷東川,擒節度使李康;及崇文克梓州,乃歸康求雪己罪,崇文以康敗軍失守,遂斬之。



 成都北一百五十里有鹿頭山,扼兩川之要,闢築城以守,又連八柵,張掎角之勢以拒王師。是
 日,破賊二萬於鹿頭城下,大雨如注,不克登,乃止。明日,又破於萬勝堆。堆在鹿頭之東,使驍將高霞寓親鼓,士扳緣而上,矢石如雨;又命敢死士連登,奪其堆,燒其柵,柵中之賊殲焉。遂據堆下瞰鹿頭城,城中人物可數。凡八大戰皆大捷,賊搖心矣。



 八月,阿跌光顏與崇文約,到行營愆一日。懼誅,乃深入以自贖,故軍於鹿頭西大河之口,以斷賊糧道,賊大駭。是日,賊綿江柵將李文悅以三千人歸順,尋而鹿頭將仇良輔舉城降者眾二萬。闢
 之男方叔、子婿蘇強,先監良輔軍,是日械系送京師,降卒投戈面縛者彌十數里,遂長驅而直指成都。德陽等縣城皆鎮以重兵,莫不望旗率服,師無留行。闢大懼,以親兵及逆黨盧文若齎重寶西走吐蕃。吐蕃素受其賂,且將啟之。崇文遣高霞寓、酈定進倍道追之,至羊灌田及焉。闢自投岷江,擒于湧湍之中。西蜀平,乃檻闢送京師伏法。文若赴水死。王師入成都,介士屯於大逵,軍令嚴肅,珍寶山積,市井不移,無秋毫之犯。



 先是,賊將邢泚
 以兵二萬為鹿頭之援,既降又貳,斬之以徇。衣冠陷逆者,皆匍匐衙門請命,崇文條奏全活之。制授崇文檢校司空,兼成都尹,充劍南西川節度、管內度支營田觀察處置、統押近界諸蠻,西山八國雲南安撫等使。改封南平郡王,食實封三百戶,詔刻石紀功於鹿頭山下。



 崇文不通文字,厭大府案牘諮稟之繁,且以優富之地,無所陳力,乞居塞上以捍邊戍,懇疏累上。二年冬,制加同中書門下平章事、邠州刺史、邠寧慶三州節度觀察等使,
 仍充京西都統。恃其功而侈心大作,帑藏之富,百工之巧,舉而自隨,蜀都一罄。以不習朝儀,憚於入覲,優詔令便道之鎮。居三年,大修戎備。元和四年卒,年六十四,廢朝三日,贈司徒,謚曰威武,配享憲宗廟庭。



 子承簡,少為忠武軍部將,後入神策軍。以父征劉闢,拜嘉王傅。裴度征淮、蔡,奏承簡以本官兼御史中丞,為其軍都押衙。淮西平,詔以郾城、上蔡、遂平三縣為溵州,治郾城,用承簡為刺史。尋轉邢州刺史,值觀察使責時賦急,承簡代數
 百戶出其租。



 遷宋州刺史,屬汴州逐其帥,以部將李絺行帥事。絺遣其將責宋官私財物,承簡執而囚之。自是汴使來者,輒系之,一日並出斬於軍門之外,威震郡中。及絺兵大至,宋州凡三城,已陷南一城,承簡保北兩城以拒,凡十餘戰。會徐州救兵至,絺為汴將李質執之,傳送京師,兵圍宋者即遁去。授承簡檢校左散騎常侍、充海沂密等州節度觀察處置等使。



 俄遷檢校工部尚書、義成軍節度、鄭滑潁等州觀察處置等使。就加檢校尚
 書右僕射。入拜右金吾衛大將軍,充右街使。復出為邠寧慶等州節度觀察處置等使。先是,羌虜多以秋月犯西邊,承簡請軍寧州以備之。因疾,上言乞入覲,即隨表詣闕。太和元年八月,行至永壽縣傳舍卒,贈司空。



 崇文孫駢,歷位崇顯,終淮南節度使,自有傳。



 伊慎,兗州人。善騎射,始為果毅。喪母,將營合祔,不識其父之墓。晝夜號哭,未浹日,夢寐有指導焉。遂發壟,果得舊記驗。



 大歷八年,江西節度使路嗣恭討嶺南哥舒晃
 之亂,以慎為先鋒,直逼賊壘,疾戰破之,斬首三千級,由是復始興之地。未幾,與諸將追斬晃於泔溪,函首獻於闕下。嗣恭表慎功,授連州長史,知當州團練副使,三遷江州別駕。



 討梁崇義之歲,慎以江西牙將從李希烈,摧鋒陷敵,功又居多。江漢既平,希烈愛慎之材,數遺善馬,意欲縻之,慎以計遁,歸命本道。明年,希烈果反。嗣曹王皋始至鐘陵,大集將吏,得慎而壯之。大集兵將,繕理舟師。希烈懼慎為曹王所任,遺慎七屬之甲,詐為慎書行
 間焉。上遣中使即軍以詰之,曹王乃抗疏論雪。上章未報,會賊兵溯江來寇,曹王乃召慎勉之令戰,大破三千餘眾,朝廷始信其不貳。累破蔡山柵,取蘄州,降其將李良。又攻黃梅縣,殺賊將韓霜露,斬首千餘級。優詔褒異,授試太子詹事,封南充郡王,又兼御史中丞、蘄州刺史,充節度都知兵馬使。



 建中末,車駕在梁、洋,鹽鐵使包佶以金幣溯江將進獻,次於蘄口。時賊已屠汴州,遣驍將杜少誠將步騎萬餘來寇黃梅,以絕江道。慎兵七千,遇
 於永安戍。慎列樹三柵,相去數里,偃旗臥鼓。於中柵聲鼓,三柵悉兵以擊,賊軍大亂,少誠脫身以免,斬級不可勝數,江路遂通。又破茍莽柵,進兵圍安州。賊阻溳水,攻之不能下。希烈遣其甥劉戒虛將騎八千來援,慎分兵迎擊,戰於應山,擒戒虛,縛示城下,遂開門請罪。以功拜安州刺史、兼御史大夫,仍賜實封一百戶。希烈又遣將援隋州,慎擊之於厲鄉,走康叔夜,斬首五千級。希烈死,李惠登為賊守隋州,慎飛書招諭,惠登遂以城降。因密
 奏惠登可用,詔授隋州刺史。



 貞元十五年,以慎為安黃等州節度、管內支度營田觀察等使。十六年,吳少誠阻命,詔以本道步騎五千,兼統荊南、湖南、江西三道兵,當其一面。於申州城南前後破賊數千,以例加檢校刑部尚書。二十一年,於安黃置奉義軍額,以為奉義軍節度使、檢校右僕射。憲宗即位,入真拜右僕射。元和二年,轉檢校左僕射,兼右金吾衛大將軍。以賂第五從直求鎮河中,為從直所奏,貶右衛將軍。數月,復為檢校尚書右
 僕射,兼右衛上將軍。元和六年卒,年六十八,贈太子太保。



 硃忠亮,本名士明,沛州浚儀人。初事薛嵩為將。大歷中,詔鎮普潤縣,掌屯田。硃泚之亂,以麾下四十騎奔奉天。德宗嘉之,封東陽郡王,為「奉天定難功臣」。及大駕南幸,為虜騎所獲,系於長安。賊平,李晟釋之,薦於渾瑊,署定平鎮都虞候。鎮使李朝採卒,遂代之。憲宗即位,加御史大夫。築臨涇城有勞,特加檢校工部尚書、涇原四鎮節
 度使,仍賜名。涇土舊俗多賣子,忠亮以俸錢贖而還其親者約二百人。元和八年卒,贈右僕射。



 劉昌裔,太原陽曲人。少游三蜀。楊琳之亂,昌裔說其歸順。及琳授洺州刺史,以昌裔為從事,琳死乃去。



 曲環將幽隴兵收濮州也,闢為判官。詔授監察御史,累加至檢校兵部尚書,賜紫,兼中丞,充營田副使。貞元十五年,環鎮許州,卒,詔上官涚知節度留後。吳少誠攻許州,涚領事,欲棄城走。昌裔追止之曰:「留後既受詔,宜以死守城。
 況城中士馬足以破賊,但堅壁不戰,不過五七日,賊勢必衰,我以全制之可也。」涚然之。賊日夕攻急,堞壞不得修,昌裔令造戰棚木柵以待;募壯士破營,得突將千人,鑿城分出,大破之,因立戰棚木柵於城上,城以故不陷。兵馬使安國寧與涚不善,謀反以城降賊;事洩,昌裔密計斬之。即召其麾下千餘人食之,賞縑二匹,伏兵諸要巷,令持縑者悉斬之,無一人得脫。十六年,以全陳許功,以涚為節度使,昌裔為陳州刺史。



 韓全義之敗溵水也,
 與諸道兵皆走保陳州;求舍,昌裔登城謂曰:「天子命公討蔡州,今來陳州,義不敢納,請舍城外。」而從千騎入全義營,持牛酒勞軍。全義不自意,驚喜嘆服。十八年,改充陳許行軍司馬。明年,涚卒,詔昌裔為許州刺史,充陳許節度使,再加檢校右僕射。



 元和八年五月,許州大水,壞廬舍,漂溺居人。六月,徵昌裔加檢校左僕射,兼左龍武統軍。初,昌裔以老疾而軍府無政,因其水敗軍府,上乃促令韓皋代之。昌裔赴召,至長樂驛,聞有是命,乃上言
 風眩,請歸私第,許之。其年卒,贈潞州大都督。



 範希朝,字致君,河中虞鄉人。建中年,為邠寧虞候,戎政修舉,事節度使韓游瑰。及德宗幸奉天,希朝戰守有功,累加兼中丞,為寧州刺史。游瑰入覲,自奉天歸邠州,以希朝素整肅有聲,畏其逼己,求其過,將殺之。希朝懼,奔鳳翔。德宗聞之,趣召至京師,置於左神策軍中。游瑰歿,邠州諸將列名上請希朝為節度,德宗許之。希朝讓於張獻甫,曰:「臣始逼而來,終代其任,非所以防凱覦安反
 側也。」詔嘉之,以獻甫統邠寧。數日,除希朝振武節度使,就加檢校禮部尚書。



 振武有黨項、室韋,交居川阜,凌犯為盜,日入慝作,謂之「刮城門」。居人懼駭,鮮有寧日。希朝周知要害,置堡柵,斥候嚴密,人遂獲安。異蕃雖鼠竊狗盜,必殺無赦,戎虜甚憚之,曰:「有張光晟,苦我久矣,今聞是乃更姓名而來。」其見畏如此。蕃落之俗,有長帥至,必效奇駝名馬,雖廉者猶曰當從俗,以致其歡,希朝一無所受。積十四年,皆保塞而不為橫。單于城中舊少樹,希
 朝於他處市柳子,命軍人種之,俄遂成林,居人賴之。貞元末,累表請修朝覲。時節將不以他故自述職者,惟希朝一人,德宗大悅。既至,拜檢校右僕射,兼右金吾大將軍。



 順宗時,王叔文黨用事,將授韓泰以兵柄;利希朝老疾易制,乃命為左神策、京西諸城鎮行營節度使,鎮奉天,而以泰為副,欲因代之,叔文敗而罷。憲宗即位,復以檢校僕射為右金吾,出拜檢校司空,充朔方靈鹽節度使。



 突厥別部有沙陀者,北方推其勇勁,希朝誘致之,自
 甘州舉族來歸,眾且萬人。其後以之討賊,所至有功,遷河東節度使。率師討鎮州無功。既耄且疾,事不理,除左龍武統軍,以太子太保致仕。元和九年卒,贈太子太師。



 希朝近代號為名將,人多比之趙充國。及張茂昭擊王承宗,幾覆,希朝玩寇不前,物議罪之。



 王鍔,字昆吾,自言太原人。本湖南團練營將。初,楊炎貶道州司馬,鍔候炎于路,炎與言異之。後嗣曹王皋為團練使,擢任鍔,頗便之。使招邵州武岡叛將王國良有功,
 表為邵州刺史。及皋改江西節度使,李希烈南侵,皋請鍔以勁兵三千鎮尋陽。後皋自以全軍臨九江,既襲得蘄州,盡以眾渡,乃表鍔為江州刺史、兼中丞,充都虞候,因以鍔從。小心習事,善探得軍府情狀,至於言語動靜,巨細畢以白皋。皋亦推心委之,雖家宴妻女之會,鍔或在焉。鍔感皋之知,事無所避。



 後皋攻安州,使伊慎盛兵圍之;賊懼,請皋使至城中以約降,皋使鍔懸而入。既成約,殺不從者以出。明日城開,皋以其眾入。伊慎以賊恟懼,
 由其圍也,不下鍔,鍔稱疾避之。及皋為荊南節度使,表鍔為江陵少尹、兼中丞,欲列於賓倅。馬彞、裴泰鄙鍔請去,乃復以為都虞候。



 明年,從皋至京師,皋稱鍔於德宗曰:「鍔雖文用小不足,他皆可以試驗。」遂拜鴻臚少卿。尋除容管經略使,凡八年,溪洞安之。遷廣州刺史、御史大夫、嶺南節度使。廣人與夷人雜處,地征薄而叢求於川市。鍔能計居人之業而榷其利,所得與兩稅相埒。鍔以兩稅錢上供時進及供奉外,餘皆自入。西南大海中諸
 國舶至,則盡沒其利,由是鍔家財富於公藏。日發十餘艇,重以犀象珠貝,稱商貸而出諸境。周以歲時,循環不絕,凡八年,京師權門多富鍔之財。拜刑部尚書。時淮南節度使杜佑屢請代,乃以鍔檢校兵部尚書,充淮南副節度使。鍔始見佑,以趨拜悅佑,退坐司馬事。數日,詔杜佑以鍔代之。



 鍔明習簿領,善小數以持下,吏或有奸,鍔畢究之。嘗聽理,有遺匿名書於前者,左右取以授鍔,鍔內之靴中,靴中先有他書以雜之。及吏退,鍔探取他
 書焚之,人信其以所匿名者焚也。既歸省所告者,異日乃以他微事連其所告者,固窮按驗之以譎眾,下吏以為神明。鍔長於部領,程作有法,軍州所用竹木,其餘碎屑無所棄,皆復為用。掾曹簾壞,吏以新簾易之,鍔察知,以故者付舡坊以替箬,其他率如此。每有饗宴,輒錄其餘以備後用,或云賣之,收利皆自歸,故鍔錢流衍天下。在鎮四年,累至司空。



 元和二年來朝,真拜左僕射,未幾除檢校司徒、河中節度。居三年,兼太子太傅,移鎮太原。
 時方討鎮州,鍔緝綏訓練,軍府稱理。鍔受符節居方面凡二十餘年。九年,加同平章事。十年卒,年七十六,贈太尉。鍔將卒,約束後事甚明,如知其死日。



 鍔附太原王翃為從子,以婚閥自炫,炫子弟多附鍔以致名宦。又嘗讀《春秋左氏傳》,自稱儒者,人皆笑之。



 子稷,歷官鴻臚少卿。鍔在籓鎮,稷嘗留京師,以家財奉權要,視官高下以進賂,不待白其父而行之。廣治第宅,嘗奏請藉坊以益之,作復垣洞穴,實金錢於其中。貴官清品,溺其賞宴而游,
 不憚清議。及父卒,為奴所告稷換鍔遺表,隱沒所進錢物。上令鞫其奴於內仗,又發中使就東都驗責其家財。宰臣裴度苦諫,於是罷其使而殺奴。稷長慶二年為德州刺史,廣齎金寶僕妾以行。節度使李全略利其貨而圖之,故致本州軍亂,殺稷,其室女為全略所虜,以妓媵處之。



 稷子叔泰。開成四年,滄州節度使劉約上言:「王稷為李全略所殺,家無遺類。稷男叔泰,時年五歲,郡人宋忠獻匿之獲免,乃收養之,今已成長。臣獎其義,忠獻已
 補職,叔泰津送以聞。」文宗詔曰:「王鍔累朝宣力,王稷一旦捐軀,須錄孤遺,微申憫念。王叔泰委吏部與九品官,令奉祭。」



 閻巨源,貞元十九年以勝州刺史攝振武行軍司馬。屬希朝入覲,遂代為節度。以材力進,無他智能。初不知書而好文,其言輒乖誤,時人多摭其談說以為戲,然以寬厚為將卒所懷。後為邠寧節度使、檢校左僕射。元和九年卒。



 孟元陽,起於陳許軍中,理戎整肅,勤事,善部署。曲環之為節度,元陽已為大將,環使董作西華屯。元陽盛夏芒戺立稻田中,須役者退而後就舍,故其田歲無不稔,軍中足食。環卒,吳少誠寇許州,元陽城守;外無救兵,攻圍甚急,而終不能傅其城,賊乃罷兵。韓全義五樓之敗,諸軍多私歸,元陽及神策都將蘇元策、宣州都將王幹各率部留軍溵水,破賊二千餘人。兵罷,加御史大夫。元和初,拜河陽節度、檢校尚書。五年,拜右僕射、昭義節度,入
 為右羽林統軍,封趙國公。俄拜左金吾大將軍,復除統軍。元和九年卒,贈揚州大都督。



 趙昌,字洪祚,天水人。祖不器,父居貞,皆有名於時。李承昭為昭義節度,闢昌在幕府。貞元七年,為虔州刺史。屬安南都護為夷獠所逐,拜安南都護,夷人率化。十年,因屋壞傷脛,懇疏乞還,以檢校兵部郎中裴泰代之,入拜國子祭酒。及泰為首領所逐,德宗詔昌問狀。昌時年七十二,而精健如少年者,德宗奇之,復命為都護,南人相
 賀。



 憲宗即位,加檢校工部尚書,尋轉戶部尚書,充嶺南節度。元和三年,遷鎮荊南,徵為太子賓客。及得見,拜工部尚書、兼大理卿。歲餘,讓卿守本官。六年,除華州刺史,辭於麟德殿。時年八十餘,趨拜輕捷,召對詳明,上退而嘆異,宣宰臣密訪其頤養之道以奏焉。在郡三年,入為太子少保。九年卒,年八十五,贈揚州大都督,謚曰成。



 史臣曰:高崇文以律貞師,勤於軍政,戎麾指蜀,遽立奇功,可謂近朝之良將也。伊慎、硃忠亮、劉昌裔、範希朝、閻
 巨源、孟元陽、趙昌等,各立功立事,亦一時之名臣。王鍔明可照奸,忠能奉主,此乃垂名於後也。至若竹頭木屑,曾無棄遺,作事有程,儉而足用,則又士君子之為也。如賤收貴出,務積珠金,唯利是求,多財為累,則與夫清白遺子孫者遠矣!凡百在位,得不鑒之。



 贊曰:崇文之功,顯於西蜀。伊慎之忠,見乎南服。硃、劉、範、閻,各有其目。元陽、趙昌,不無遺躅。惟彼太原,戰勛可錄。累在多財,子孫不祿。



\end{pinyinscope}