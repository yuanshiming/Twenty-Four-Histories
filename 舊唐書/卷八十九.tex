\article{卷八十九}

\begin{pinyinscope}

 ○唐
 臨孫紹張文瓘兄文琮從弟文收徐有功



 唐臨,京兆長安人,周內史瑾孫也。其先自北海徙關中。伯父令則,開皇末為左庶子,坐諂事太子勇誅死。臨少與兄皎俱有令名。武德初,隱太子總兵東征,臨詣軍獻
 平王世充之策,太子引直典書坊,尋授右衛率府鎧曹參軍。宮殿廢,出為萬泉丞。縣有輕囚十數人,會春暮時雨,臨白令請出之,令不許。臨曰:「明公若有所疑,臨請自當其罪。」令因請假,臨召囚悉令歸家耕種,與之約,令歸系所。囚等皆感恩貸,至時畢集詣獄,臨因是知名。



 再遷侍御史,奉使嶺外,按交州刺史李道彥等申叩冤系三千餘人。累轉黃門侍郎,加銀青光祿大夫。儉薄寡欲,不治第宅,服用簡素,寬於待物。嘗欲吊喪,令家童自歸家
 取白衫,家僮誤將餘衣,懼未敢進。臨察知之,使召謂曰:「今日氣逆,不宜哀泣,向取白衫,且止之也。」又嘗令人煮藥,失制。潛知其故,謂曰:「陰暗不宜服藥,宜即棄之。」竟不揚言其過,其寬恕如此。



 高宗即位,檢校吏部侍郎。其年,遷大理卿。高宗嘗問臨在獄系囚之數,臨對詔稱旨。帝喜曰:「朕昔在東宮,卿已事朕,朕承大位,卿又居近職,以疇昔相委,故授卿此任。然為國之要,在於刑法,法急則人殘,法寬則失罪,務令折中,稱朕意焉。」高宗又嘗親錄
 死囚,前卿所斷者號叫稱冤,臨所入者獨無言。帝怪問狀,囚曰:「罪實自犯,唐卿所斷,既非冤濫,所以絕意耳。」帝嘆息良久曰:「為獄者不當如此耶!」



 永徽元年,為御史大夫。明年,華州刺史蕭齡之以前任廣州都督贓事發,制付群官集議。及議奏,帝怒,令於朝堂處置。臨奏曰:



 臣聞國家大典,在於賞刑,古先聖王,惟刑是釁。《虞書》曰:「罪疑惟輕,功疑惟重,與其殺弗辜,寧失弗經。」《周禮》:「刑平國用中典,刑亂國用重典。」天下太平,應用堯、舜之典。比來有
 司多行重法,敘勛必須刻削,論罪務從重科,非是憎惡前人,止欲自為身計。今議蕭齡之事,有輕有重,重者流死,輕者請除名。以齡之受委大籓,贓罪狼籍,原情取事,死有餘辜。然既遣詳議,終須近法。竊惟議事群官,未盡識議刑本意。律有八議,並依《周禮》舊文,矜其異於眾臣,所以特制議法。禮:王族刑於隱者,所以議親;刑不上大夫,所以議貴。知重其親貴,議欲緩刑,非為嫉其賢能,謀致深法。今既許議,而加重刑,是與堯、舜相反,不可為萬
 代法。



 高宗從其奏,齡之竟得流於嶺外。



 尋遷刑部尚書,加金紫光祿大夫,復歷兵部、度支、吏部三尚書。顯慶四年,坐事,貶為潮州刺史。卒官,年六十。所撰《冥報記》二卷,大行於世。



 兄皎,武德初為秦府記室,從太宗征討,專掌書檄,深見親待。貞觀中,累轉吏部侍郎。先是,選集無限,隨到補職,時漸太平,選人稍眾,皎始請以冬初一時大集,終季春而畢,至今行之。歷遷益州長史。卒,贈太常卿。



 子之奇,調露中為給事中,坐嘗為章懷太子僚屬徙邊。
 文明元年,起為括蒼令,與徐敬業作亂伏誅。



 臨孫紹,博學,善《三禮》。神龍中為太常博士。景龍二年,韋庶人上言:「自妃、主及命婦、宮官,葬日請給鼓吹。」中宗特制許之。紹上疏諫曰:「竊聞鼓吹之樂,本為軍容。昔黃帝涿鹿有功,以為警衛。故㭎鼓曲有《靈夔吼》、《雕鶚爭》、《石墜崖》、《壯士怒》之類,自昔功臣備禮,適得用之。丈夫有四方之功,以恩加寵錫。假如郊天祀地,誠是重儀,惟有宮懸,本無案據。故知軍樂所備,尚不洽於神祇;鉦鼓之音,豈能接於閨
 閫。準式,公主、王妃已下葬禮,惟有團扇、方扇、彩帷、錦鄣之色。加之鼓吹,歷代未聞。又準令,五品官婚葬,元無鼓吹,惟京官五品,得借四品鼓吹為儀。令特給五品以上母妻,五品官則不當給限,便是班秩本因夫子,儀飾乃復過之。事非倫次,難為定制,參詳義理,不可常行。請停前敕,各依常典。」疏奏不納。



 紹尋遷左臺侍御史,兼太常博士。中宗將親拜南郊,國子祭酒祝欽明等希旨皇后為亞獻,紹與博士蔣欽緒固爭以為不可。又則天父母二陵各置守戶五百人,武三思及子崇
 訓墓各置守戶六十人。以武氏外戚乃與昭陵禮同,三思等復逾親王之制,又上疏切諫。當時雖皆不從,深為議者所美。睿宗即位,又數陳時政損益,累轉給事中,仍知禮儀事。



 先天二年冬,今上講武於驪山,紹以修儀注不合旨,坐斬。時今上既怒講武失儀,坐紹於纛下,右金吾將軍李邈遽請宣敕,遂斬之。時人既痛惜紹,而深咎於邈。尋有敕罷邈官,遂擯廢終其身。



 張文瓘,貝州武城人。大業末徙家魏州之昌樂。瓘幼孤,
 事母兄以孝友聞。貞觀初,舉明經,補並州參軍。時英國公李勣為長史,深禮之。累遷水部員外郎。時兄文琮為戶部侍郎,舊制兄弟不許並居臺閣,遂出為雲陽令。龍朔年,累授東西臺舍人、參知政事。尋遷東臺侍郎、同東西臺三品,兼知左史事。



 時初造蓬萊、上陽、合璧等宮,又征討四夷,廄馬有萬匹,倉庫漸虛。文瓘因進諫曰:「人力不可不惜,百姓不可不養,養之逸則富以康,使之勞則怨以叛。秦皇、漢武,廣事四夷,多造宮室,使士崩瓦解,戶
 口減半。臣聞制化於未亂,保邦於未危,人罔常懷,懷於有仁。陛下不制於未亂之前,安能救於既危之後?百姓不堪其弊,必構禍難,殷鑒不遠,近在隋朝。臣願稍安撫之,無使生怨。」上深納其言,於是節減廄馬數千匹,賜文瓘繒錦百段。



 咸亨三年,官名復舊,改授黃門侍郎,兼太子左庶子。俄遷大理卿,依舊知政事。文瓘至官旬日,決遣疑事四百餘條,無不允當,自是人有抵罪者,皆無怨言。文瓘常有疾,系囚相與齋禱,願其視事。當時咸稱其
 執法平恕,以比戴胄。上元二年,拜侍中,兼太子賓客。大理諸囚聞文瓘改官,一時慟哭,其感人心如此。



 文瓘性嚴正,諸司奏議,多所糾駁,高宗甚委之。或時臥疾在家,朝廷每有大事,上必問諸宰臣曰:「與文瓘議未?」奏云未者,則遣共籌之。奏云已議者,皆報可從之。其後,新羅外叛,高宗將發兵討除。時文瓘疾病在家,乃輿疾請見,奏曰:「比為吐蕃犯邊,兵屯寇境,新羅雖未即順,師不內侵。若東西俱事征討,臣恐百姓不堪其弊。請息兵修德以
 安百姓。」高宗從之。儀鳳二年卒,年七十三,贈幽州都督,謚曰懿。以其經事孝敬皇帝,特敕陪葬恭陵。四子:潛、沛、洽、涉。中宗時,潛官至魏州刺史,沛同州刺史,洽衛尉卿,涉殿中監。父子兄弟五人皆至三品官,時人謂之「萬石張家」。及韋溫等被誅之際,涉為亂兵所殺。



 兄文琮,貞觀中為持書侍御史。三遷毫州刺史,為政清簡,百姓安之。永徽初,表獻《太宗文皇帝頌》,優制褒美,賜絹百匹,徵拜戶部侍郎。從母弟房遺愛以罪貶授房州刺史,文琮作
 詩祖餞。及遺愛誅,坐是出為建州刺史。州境素尚淫祀,不修社稷。文琮下教書曰:「春秋二社,蓋本為農,惟獨此州,廢而不立。禮典既闕,風俗何觀?近年已來,田多不熟,抑不祭先農所致乎!神在於敬,何以邀福?」於是示其節限條制,百姓欣而行之。尋卒。文集二十卷。子戩,官至江州刺史,撰《喪儀纂要》七卷,行於時。戩弟錫,則天時為鳳閣侍郎、同鳳閣鸞臺平章事。先是,姊子李嶠知政事,錫拜官,而嶠罷相出為國子祭酒,舅甥相代為相,時人榮
 之。錫與鄭杲俱知天官選事,坐贓,則天將斬之以徇,臨刑而特赦之。中宗時,累遷工部尚書,兼修國史,尋令於東都留守。中宗崩,韋庶人臨朝,詔錫與刑部尚書裴談並同中書門下三品。旬日,出為絳州刺史。累封平原郡公,以年老致仕而卒。



 文琮從父弟文收,隋內史舍人虔威子也。尤善音律,嘗覽蕭吉《樂譜》,以為未甚詳悉,更博採群言及歷代沿革,裁竹為十二律吹之,備盡旋宮之義。時太宗將創制禮樂,召文收於太常,令與少卿祖孝
 孫參定雅樂。太樂有古鐘十二,近代惟用其七,餘有五,俗號啞鐘,莫能通者。文收吹律調之,聲皆響徹,時人咸服其妙。尋授協律郎。十一年,文收表請厘正太樂,上謂侍臣曰:「樂本緣人,人和則樂和。至如隋煬帝末年,天下喪亂,縱令改張音律,知其終不和諧。若使四海無事,百姓安樂,音律自然調和,不藉更改。」竟不依其請。十四年,景雲見,河水清,文收採《硃雁天馬》之義,制《景雲河清》樂,名曰「燕樂」,奏之管弦,為樂之首,今元會第一奏者是也。
 咸亨元年,遷太子率更令,卒官。撰《新樂書》十二卷。



 徐有功,國子博士文遠孫也。舉明經,累轉蒲州司法參軍,紹封東莞男。為政寬仁,不行杖罰。吏人感其恩信,遞相約曰:「若犯徐司法杖者,眾必斥罰之。」由是人爭用命,終於代滿,不戮一人。載初元年,累遷司刑丞。時酷吏周興、來俊臣、丘神勣、王弘義等構陷無辜,皆抵極法,公卿震恐,莫敢正言。有功獨存平恕,詔下大理者,有功皆議出之,前後濟活數十百家。常於殿庭論奏曲直,則天厲
 色詰之,左右莫不悚慄,有功神色不撓,爭之彌切。尋轉秋官員外郎,轉郎中。俄而鳳閣侍郎任知古、冬官尚書裴行本等七人被構陷當死,則天謂公卿曰:「古人以殺止殺,我今以恩止殺。就群公乞知古等,賜以再生,各授以官,佇申來效。」俊臣、張知默等又抗表請申大法,則天不許之。俊臣乃獨引行本,重驗前罪,奏曰:「行本潛行悖逆,告張知蹇與廬陵王反不實,罪當處斬。」有功駁奏曰:「俊臣乖明主再生之賜,虧聖人恩信之道。為臣雖當嫉
 惡,然事君必將順其美。」行本竟以免死。道州刺史李仁褒及弟榆次令長沙,又為唐奉一所構,高宗末私議吉兇,謀復李氏,將誅之。有功又固爭之,不能得。秋官侍郎周興奏有功曰:「臣聞兩漢故事,附下罔上者腰斬,面欺者亦斬。又《禮》云:析言破律者殺。有功故出反囚,罪當不赦,請推按其罪。」則天雖不許系問,然竟坐免官。久之,起為左臺侍御史,則天特褒異之。時遠近聞有功授職,皆欣然相賀。



 有功嘗上疏論天官、秋官及朝堂三司理匭
 使愆失,其略曰:「陛下即位已來,海內職員一定,而天下選人漸多。掌選之曹用舍不平,補擬乖次,囑請公行,顏面罔懼。遂使囂謗滿路,怨讟盈朝,浸以為常,殊無愧憚。又往屬唐朝季年,時多逆節,鞫訊結斷,刑獄至嚴。革命以來,載祀遽積,餘風未殄,用法猶深。今推鞫者猶行酷法,妄劾斷。臣即按驗,奏而劾之,獲其枉狀,請即付法斷罪,亦奪祿貶考,以慚其德。其三司受表及理匭申冤使,不速與奪,致令擁塞,有理不為申者,亦望準前彈奏,貶
 考奪祿。臣昔處法司,緣蒙擢用,臣無以上答至造,願以執法酬恩。無縱詭隨,不避強御,猛噬鷙擊,是臣之分。如蒙允納,請降敕施行,庶不越旬時,亦可以除殘革弊,刑措不用,天下幸甚。」



 後潤州刺史竇孝諶妻龐氏為奴誣告,雲夜解祈福,則天令給事中薛季昶鞫之。季昶鍛練成其罪,龐氏當坐斬。有功獨明其無罪。而季昶等返陷有功黨援惡逆,奏付法,法司結刑當棄市。有功方視事,令史垂泣以告,有功曰:「豈吾獨死,而諸人長不死耶?」乃徐
 起而歸。則天覽奏,召有功詰之曰:「卿比斷獄,失出何多?」對曰:「失出,臣下之小過;好生,聖人之大德。願陛下弘大德,則天下幸甚。」則天默然。於是龐氏減死,流於嶺表,有功除名為庶人。尋起為左司郎中,累遷司刑少卿。有功謂所親曰:「今身為大理,人命所懸,必不能順旨詭辭以求茍免。」故前後為獄官,以諫奏枉誅者,三經斷死,而執志不渝,酷吏由是少衰,時人比漢之於、張焉。或曰:「若獄官皆然,刑措何遠。」久之,轉司僕少卿子。長安二年卒,年六
 十二,贈司刑卿。



 中宗即位,制曰:「忠正之臣,自昔攸尚;褒贈之典,舊章所重。故贈大理卿徐有功,節操貞勁,器懷亮直,徇古人之志業,實一代之賢良,司彼刑書,深存敬慎。周興、來俊臣等性惟殘酷,務在誅夷,不順其情,立加誣害。有功卓然守法,雖死不移,無屈撓之心,有忠烈之議。當其執斷,並遇平反,定國、釋之,何以加此。朕惟新庶政,追想前跡,其人既歿,其德可稱。追往贈終,慰茲泉壤。可贈越州刺史,仍遣使就家吊祭,賜物百段,授一子官。」
 今上踐祚,竇孝諶之子希瑊等請以身之官爵讓有功子惀,以報舊恩。惀由是自太子司議郎、恭陵令累遷申王府司馬,卒。



 史臣曰:文法,理具之大者,故舜命皋陶為士,昌言誡敕,勤亦至焉。蓋人命所懸,一失其平,冤不可復,聖王所以疚心也。如臨之守法,文瓘之議刑,時屬哲王,可以理奪。當賊後遷鼎之際,酷吏羅織之辰,徐有功獨抗群邪,持平不撓,此所以為難也。比釋之、定國,徐又過之。希瑊讓
 爵酬恩,可知遺愛。



 贊曰:聽訟惟明,持法惟平。二者或爽,人何以生?猗歟徐公,獬豸之精。世皆紛濁,不改
 吾清。



\end{pinyinscope}