\article{卷一十七上 本紀第十七上 敬宗 文宗上}

\begin{pinyinscope}

 敬宗睿武昭愍孝皇帝諱湛,穆宗長子,母曰恭僖太后王氏。元和四年六月七日,生於東內之別殿。長慶元年三月,封景王。二年十二月,立為皇太子。四年正月壬申,
 穆宗崩。癸酉,皇太子即位柩前,時年十六。甲戌,左僕射韓皋卒。丙子,群臣準遺詔奏皇帝寶冊禮畢,詔賞神策諸軍士人絹十匹、錢十千、畿內諸軍鎮絹十匹、錢五千,其餘軍鎮頒給有差。內出綾絹三百萬段以助賞給。穆宗初即位,在京軍士人獲五十千,在外軍鎮差降無幾。至是,宰臣奏議請量國力頒賞,故差減於先朝,物議是之。群臣五上章請聽政,從之。二月辛巳朔,上縗服見群臣於紫宸門外。壬午,渤海送備宿衛大聰叡等五十人
 入朝。癸未,貶戶部侍郎李紳為端州司馬。丙戌,貶翰林學士、駕部郎中、知制誥龐嚴為信州刺史,翰林學士、司封員外郎、知制誥蔣防為汀州刺史,皆紳之引用者。以右拾遺吳思為殿中侍御史,充入蕃告哀使。李紳之貶,李逢吉受賀,群官至中書,而思獨不往,逢吉怒而斥為遠使。戊子,河北告哀使、諫議大夫高允恭卒於東都。辛卯,敕沒掖庭宮人、先配內園宮人,並宜放出,任其所適。己亥,冊大行皇帝皇太后為太皇太后。庚子,西川節度
 使杜元穎進罨畫打球衣五百事,非禮也。辛丑,上始御紫宸殿受朝。既退,幸飛龍院,厚賜內官等物有差。以米貴,出太倉粟四十萬石,於兩市賤糶,以惠貧民。癸未夜,太白犯東井北轅。乙巳,上率群臣詣光順門冊皇太后。丁未,御中和殿擊球,賜教坊樂官綾絹三千五百匹。戊申,擊球於飛龍院。己酉,大合樂於中和殿,極歡而罷,內官頒賜有差。



 三月庚戌朔,貶司農少卿李彤吉州司馬,以前為鄧州刺史,坐贓百萬,仍自刻德政碑故也。壬子,
 上御丹鳳樓,大赦天下。京畿夏青苗錢並放,秋青苗錢每貫放二百文。天下常貢之外不得進獻。六宅、十宅諸王女,宜令每年於選人中選擇降嫁。今後戶帳田畝,五年一定稅。是日,風且雨。甲寅,始於延英對宰臣。丙辰,以尚書右丞韋顗為戶部侍郎。戊午,禮儀使奏:「外命婦正旦及四始日舊行起居之禮,伏以禮煩則瀆,請停。」從之。庚申,工部尚書胡證檢校戶部尚書、京兆尹。甲子,故山南東道節度使牛元翼家為王廷湊所害,上惜其冤橫,
 傷悼久之,仍嘆宰執非才,縱奸臣跋扈。翰林學士韋處厚奏曰:「理亂之本,非有他術,順人則理,違人則亂。陛下常嘆息,恨無蕭、曹。今有一裴度,尚不能用,此馮唐所以感悟漢文,雖有頗、牧不能用也。」以太子少保張弘靖為太子少師,分司東都太子賓客令狐楚河南尹。丁卯,以刑部尚書段文昌判左丞事。戊辰,群臣入閣,日高猶未坐,有不任立而踣者。諫議大夫李渤出次白宰相,俄而始坐。班退,左拾遺劉棲楚諫,頭叩龍墀血流,上
 為之動容,仍賜緋魚袋。編氓徐忠信闌入浴堂門,杖四十,配流天德。庚午,賜內教坊錢一萬貫,以備游幸。是夜,太白犯東井北轅。甲戌,夏州節度使李祐奏:於塞外築烏延、宥州、臨塞、陰河、陶子等五城,以備蕃寇。又以黨項為盜,於蘆子關北木瓜嶺築壘,以扼其沖。乙亥,幸教坊,賜伶官綾絹三千五百匹。



 夏四月庚辰朔。甲申,以御史大夫王涯為戶部尚書、兼御史大夫,充鹽鐵轉運等使。壬辰,兵部侍郎武儒衡卒。丙申,賊張韶等百餘人至右
 銀臺門,殺閽者,揮兵大呼,進至清思殿,登御榻而食,攻弓箭庫。左神策軍兵馬使康藝全率兵入宮討平之。是日,上聞其變,急幸左軍。丁酉,上還宮,群臣稱慶。諫議大夫李渤以上輕易致盜,言甚激切。己亥,九仙門等監共三十五人,並笞之。辛丑,染坊使田晟、段政直流天德,以張韶染坊役夫故也。詔雪吐突承璀之罪,令男士曄改葬之。丙午,宰臣李逢吉封涼國公,牛僧孺封奇章縣子。



 五月己酉朔。乙卯,制以正議大夫、尚書吏部侍郎、上柱國、
 渭源縣開國男、食邑三百戶、賜紫金魚袋李程守本官、同中書門下平章事。以朝議郎、守尚書戶部侍郎、兼御史大夫、判度支、上柱國、賜紫金魚袋竇易直為朝散大夫,本官同中書門下平章事。判度支、戶部侍郎韋顗賜金紫。己未,割富平縣之豐水鄉、下邽縣之翟公鄉、澄城縣之撫道鄉、白水縣之會賓鄉,以奉景陵。癸亥,以鹽州刺史傅良弼為夏州節度使。東都、江陵監大轉運留後並改為知院官,從其使王涯請也。



 六月己卯朔,以左神
 策大將康藝全為鄜坊節度使。辛巳,敕以霖雨命疏決京城系囚。庚辰,大風吹壞延喜、景風等門。工部侍郎張惟素卒。壬辰,以左金吾衛大將軍李願檢校司空,兼河中尹、御史大夫,充河中、絳、隰等州節度使。丙申,山南西道節度使、守司空裴度加同中書門下平章事。度之拜興元也,為宰相李逢吉所排,不帶平章事,李程、韋處厚日為度論於上前,故有是命。加陳、許節度使李光顏守司徒。癸卯,太保張弘靖卒。己巳,浙西水壞太湖堤,水入
 州郭,漂民廬舍。丁未,以吏部尚書趙宗儒為太常卿,兵部尚書鄭絪為吏部尚書。



 秋七月戊申朔。己酉,睦州、清溪等六縣大雨,山谷發洪水泛溢,漂城郭廬舍。庚辰,以前河中節度使郭釗為兵部尚書。戊午,太子賓客許季同卒。辛酉,疏靈州特進渠,置營田六百頃。乙丑,鄆、曹、濮暴雨水溢,壞城郭廬舍。丁卯,敕以穀貴,凡給百官俸內一半合給匹段,今宜給粟,每斗折錢五十文。辛未,以大理卿崔元略為京兆尹、兼御史大夫。甲戌,左金吾衛大
 將軍李祐進馬二百五十匹。御史溫造於閣內奏彈祐罷使違敕進奉,祐趨出待罪,詔宥之。襄、均、復等州漢江溢,漂民廬舍。丙子,浙西觀察使李德裕奏:「詔令當道造盝子二十具,計用銀一萬三千兩,金一百三十兩。昨已進兩具,用銀一千三百兩,當道在庫貯備銀無二三百兩,皆百計收市,方成此兩具。臣當道唯有留使錢五萬貫,每事節儉支費,猶欠十三萬貫不足。臣若因循不奏,則負陛下任使之恩;若分外誅求,又累陛下慈儉之德。
 伏乞宣令宰臣商議,何以遣臣得上不違宣索,下不闕軍須,不困疲人,不斂物怨。」時有詔罷進奉,故德裕有是奏。



 八月丁酉朔。是夜,火犯土星。妖賊馬文忠與品官季文德等凡一千四百人,將圖不軌,皆杖一百處死。癸未,火犯東井。甲寅,詔於關內、關東折糴、和糴粟一百五十萬石。陳、許、蔡、鄆、曹、濮等州水害稼。丁亥,火入東井。己丑,以李心妻孫宏為河南府兵曹參軍,蔣清孫禺阜為伊陽令,錄忠臣後也。是夜,金犯軒轅右角。壬辰,江王府長史
 段釗上言,稱前任龍州刺史,近郭有牛心山,山上有仙人李龍遷祠,頗靈應,玄宗幸蜀時,特立祠廟。上遣高品張士謙往龍州檢行,回奏牛心山有掘斷處。群臣言宜須修築。時方冱寒,役民數萬計,東川節度使李絳表訴之。甲子,以太常卿趙宗儒為太子少師。乙巳,宣武軍節度韓充卒。



 九月丙午朔。丁未,波斯大商李蘇沙進沉香亭子材,拾遺李漢諫云:「沉香為亭子,有異瑤臺、瓊室。」上怒,優容之。庚戌,以河南尹令狐楚檢校禮部尚書、汴州
 刺史、宣武軍節度、宋、汴、亳觀察等使。乙卯,罷理匭使。以諫議大夫李渤知匭,奏請置胥吏、添課料故也。戊午,加硃融檢校司空。詔浙西織造可幅盤絳繚綾一千匹。觀察使李德裕上表論諫,不奉詔,乃罷之。己巳,以兵部侍郎王起為河南尹。甲子,吐蕃遣使求《五臺山圖》。己巳,浙西、淮南各進宣索銀莊奩三具。



 冬十月丙子朔,宗正寺選尚書縣主胥和元亮等二十五人,各賜錢三十萬,令備吉禮。辛巳,以吏部侍郎崔從為太常卿。庚子,嶺南節
 度使鄭權卒。辛丑,吐蕃貢BX牛,鑄成銀犀牛、羊、鹿各一。壬寅,以鄂岳觀察使、檢校兵部尚書崔植檢校吏部尚書,兼廣州刺史、御史大夫,充嶺南節度觀察經略使。以戶部侍郎韋顗為御史中丞,兼戶部侍郎;以御史中丞鄭覃權權工部侍郎;以刑部侍郎韋弘景為吏部侍郎;以權權禮部郎李宗閔權權兵部侍郎;以工部侍郎於敖為刑部侍郎。



 十一月丙午朔。戊申,安南都護李元喜奏:黃家賊與環王國合勢陷陸州,殺刺史葛維。蘇、常、
 湖、岳、吉、潭、郴等七州水傷稼。庚申,葬穆宗於光陵。十二月乙亥朔。癸未,回紇、吐蕃、奚、契丹遣使朝貢。襄州柳公綽、滄州李全略、晉州李寰、滑州高承簡並自尚書加檢校右僕射。以前起居舍人劉棲楚為諫議大夫。淮南節度使王播厚賂貴要,求領鹽鐵使,諫議大夫獨孤朗張仲方、起居郎孔敏行柳公權宋申錫、補闕韋仁實劉敦儒、拾遺李景主薛廷老等伏延英抗疏論之。戊子夜,月掩東井。庚寅,加天平軍節度使烏重胤同平章事。乙未,
 徐泗王智興請置僧尼戒壇,浙西觀察使李德裕奏狀論其奸幸。時自憲宗朝有敕禁私度戒壇,智興冒禁陳請,蓋緣久不興置,由是天下沙門奔走如不及。智興邀其厚利,由是致富,時議丑之。丁酉,宰相牛僧孺進封奇章郡公,李程彭原郡公,竇易直晉陽郡公,並食邑三千戶。吏部侍郎韓愈卒。



 寶歷元年春正月乙巳朔。辛亥,觀祀昊天上帝於南郊。禮畢,御丹鳳樓,大赦,改元寶歷元年。先是,雩縣令崔發
 坐誤辱中官下獄,是日,與諸囚陳於金雞竿下俟釋放。忽有內官五十餘人,環發而毆之,發破面折齒,臺吏以度蔽之,方免。有詔復系於臺中,宰相救之,方釋。宰相牛僧孺累表乞解機務,帝許以郊禮後。乙卯,以僧孺檢校禮部尚書、同平章事、鄂州刺史,充武昌軍節度、鄂岳觀察使。淮南節度使王播兼諸道鹽鐵轉運使。於鄂州特置武昌軍額,寵僧孺也。壬申,以給事中李渤為桂州刺史、兼御史中丞、桂管防禦觀察使。李德裕獻《丹扆箴》六
 首,上深嘉之,命學士韋處厚優其答詔。辛卯,以前禮部郎中李翱為廬州刺史,以求知制誥,面數宰相李逢吉過故也。辛丑,江西觀察使薛放卒。癸卯,以職方郎中、知制誥王璠為御史中丞。



 三月乙巳朔,以兵部尚書郭釗為梓州刺史、劍南東川節度使。壬子,宴群臣於三殿。戊辰夜,有流星長三丈,出紫微,入濁滅。辛未,以前桂管觀察使殷侑為江西觀察使。上御宣政殿試制舉人二百九十一人,以中書舍人鄭涵、吏部郎中崔琯、兵部郎中
 李虞仲並充考制策官。



 夏四月甲戌朔,宰相涼國公李逢吉進封鄭國公。以右神策大將軍康志睦檢校工部尚書,兼青州刺史、平盧軍節度使。宣中書,以諫議大夫劉棲楚為刑部侍郎。丞郎宣授,自棲楚始也。鄭涵等考定制舉人。敕下後數日,上謂宰相曰:「韋端符、楊魯士皆涉物議,宜與外官。」乃授端符白水尉,魯士城固尉。宰臣請其罪名,不報。癸巳,群臣上徽號曰文武大聖廣孝皇帝,御宣政殿受冊。禮畢,御丹鳳樓,大赦天下,大闢罪已
 下,無輕重咸赦除之。時李紳貶官。李逢吉惡紳,不欲紳量移,乃於赦書節文內,但言左降官已經量移,宜與量移近處,不言未量移者宜與量移。翰林學士韋處厚上疏論列云:「不可為李紳一人與逢吉相惡,遂令近年流貶官皆不量移,則乖曠蕩之道也。」帝遽命追赦書添改之。乙亥,以劍南東川節度、檢校司空李絳為左僕射。御史蕭徹彈京兆尹、兼御大夫崔元略違詔徵畿內所放錢萬七千貫,付三司勘鞫不虛。辛丑,敕削元略兼御史大
 夫。五月甲辰朔,以前平盧軍節度使薛平檢校左僕射、兼戶部尚書。賜振武軍錢一十四萬貫,修築東受降城。庚戌,幸魚藻宮觀竟渡。庚申,正衙命使冊九姓迥紇登里囉汨沒密施毗伽昭禮可汗。丙寅,太子少傅致仕閻濟美卒。丁卯,湖南觀察使沈傳師奏:「當道先配吐蕃羅沒等一十七人,準赦放還本國,今各得狀,不願還。」從之。庚午,以右金吾將軍李文悅為豐州刺史、天德軍防禦使。安南李元喜奏移都護府於江北岸。



 六月壬申朔。乙酉,詔
 公主、郡主並不得進女口。丙戌,將作監張武均出為洋州刺史,坐贓犯也。諸司白身馮志謀等三百九人,並賜祿。丁亥,命品官田務豐領國信十二車賜迥紇可汗及太和公主。己丑,河中節度使、檢校司空李願卒。乙未,以檢校左僕射、兼戶部尚書薛平檢校司空、河中尹、河中節度使。



 秋七月癸卯朔,以忠武軍節度使、守司徒、兼侍中李光顏為太原尹、北京留守、河東節度使,以兗海節度使王沛為許州刺史、忠武軍節度使。熒惑犯右執法。甲辰,監鐵使
 王播進羨餘,物議欲鳴鼓而攻之。乙酉,鄜坊水壞廬舍。癸丑,以右金吾衛大將軍張茂宗為兗、海、沂、密節度使。乙卯,正衙命使冊司徒李光顏。丙辰,淄王傅分司元錫卒。己未,詔王播造競渡船二十只供進,仍以船材京內造。時計其功,當半年轉運之費。諫議大夫張仲方切諫,乃改進十只。辛酉,萬年縣典賈鎮誣告故統軍王佖男正謨等七人
 謀亂,詔杖殺之。甲子夜,月犯畢。乙丑,侍講學士崔郾、高重進《纂要》十卷,賜錦採二百匹。丁卯,以戶部侍郎韋顗為吏部侍郎,京兆尹崔元略為戶部侍郎。奉天縣水壞廬舍。辛未,以左散騎常侍胡證為戶部尚書、判度支。太子賓客分司廬士玫卒。閏七月壬午朔,以權知工部侍郎鄭覃為京兆尹。甲申,拾遺李漢、舒元褒、薛廷老於閣內論曰:「伏見近日除授,往往不由中書進擬,多是內中宣出。臣恐紀綱浸壞,奸邪恣行,伏希詳察。」上然之。詔度
 支進銅三千斤、金薄十萬翻,修清思院新殿及升陽殿圖障。丙戌,戶部尚書致仕裴堪卒。戊子,以給事中盧元輔為工部侍郎。壬辰,以前河東節度使李聽為義成軍節度使。戊戌,以刑部尚書段文昌為兵部尚書,依前判左丞事。



 八月辛丑朔。戊申,以酅國公楊造男元湊襲酅國公,食邑三千戶。兩京、河西大稔,敕度支和糴折糴粟二百萬石。乙卯夜,太白近房。戊午,遣中使往湖南、江南等道及天臺山採藥。時有道士劉從政者,說以長生久
 視之道,請於天下求訪異人,冀獲靈藥。仍以從政為光祿少卿,號升玄先生。秋九月辛未朔。丁丑,衛尉卿劉遵古役人安再榮告前袁王府長史武昭謀害宰相李逢吉,詔三司鞫之。壬午,昭義節度使劉悟卒。癸未夜,太白犯南斗。丙戌夜,月犯右執法。丁酉,華州暴水傷稼。徐州王智興奏,大將武華等四百人謀亂,並伏誅。十月庚子朔,河南尹王起奏,盜銷錢為佛像者,請以盜鑄錢論。丁巳,振武節度使張惟清以東受降城濱河,歲久雉堞
 壞,乃移置於綏遠烽南,及是功成。己未,以崖州安置人嗣郢王佐為潁王府長史,分司東都,仍賜金紫。壬戌夜,太白近哭星。甲子,三司鞫武昭獄得實,武昭及弟匯、役人張少騰宜付京兆府決,河陽節度掌書記李仲言配流象州,匯流崖州,太學博士李涉流康州,皆坐武昭事也。



 十一月庚午朔。辛未,以御史中丞王璠為工部侍郎,以諫議大夫獨孤朗為御史中丞。癸酉,鎮星近東井。癸未,以殿中少監嚴公素為容管經略使。是夜,月犯東井。
 庚寅,車駕幸溫湯,即日還宮。壬辰,以刑部侍郎劉棲楚為京兆尹。丙申,詔封皇子普為晉王。丁酉,吏部侍郎韋顗卒。十二月己亥朔。辛丑,以晉王普為昭義軍節度副大使;以劉悟子將作監主簿從諫起復雲麾將軍、守金吾衛大將軍同正、檢校左散騎常侍、兼御史大夫,充昭義節度留後。戊申夜,月犯畢。其夜,北方有霧起,須臾遍天,霧上有赤氣,久而方散。甲子,以左僕射李絳為太子少師,分司東都。戊辰,敕:「農功所切,實在耕牛,疲氓多
 乏,須議給賜。委度支往河東、振武、靈、夏等州市耕牛一萬頭,分給畿內貧下百姓。」是歲,淮南、浙西、宣、襄、鄂、潭、湖南等州旱災傷稼。



 二年春正月己巳朔。庚午,貶殿中侍御史王源植為昭州司馬。時源植街行,為教坊樂伎所侮,導從呵之,遂成紛竟。京兆尹劉棲楚決責樂伎,御史中丞獨孤朗論之太切,上怒,遂貶源植。辛未,湖南觀察使沈傳師奏:奉詔校尋葉靖能、羅光遠文案,檢尋不獲。癸酉,右贊善大夫
 李光現與品官李重實爭忿,以笏擊重實流血。,上以宗屬,罰兩月俸料。甲戌,以諸軍丁夫二萬入內穿池修殿。辛巳,興元節度使裴度奏修斜谷路及館驛皆畢功。壬辰,裴度來朝。甲午,以衛尉卿劉遵古為湖南觀察使,以國子祭酒衛中行為福建觀察使。丙申,鹽鐵使王播奏:「揚州城內,舊漕河水淺,舟船澀滯,輸不及期程。今從閶門外古七里港開河,向東屈曲,取禪智寺橋,東通舊官河,計長一十九里。其功役所費,當使自方圓支遣。」從之。
 二月己亥朔。辛丑,容管經略使嚴公素奏:「當州普寧等七縣,請同廣、昭、桂、賀四州例北選。」從之。丙午夜,月犯畢。丁未,以山南西道節度觀察處置等使、光祿大夫、守司空、同中書門下平章事、興元尹、上柱國、晉國公裴度守司空、同平章事,復知政事。丁巳寒食節,三殿宴群臣,自戊午至庚申方止。丙寅,正冊司空裴度。丁卯,以禮部尚書王涯檢校左僕射,為山南西道節度使。



 三月戊辰朔,命興唐觀道士孫準入翰林待詔。辛未,江西觀察使殷
 侑請於洪州寶歷寺置僧尼戒壇,敕殷侑故違制令,擅置戒壇,罰一季俸料。甲戌,賜宰臣百僚上巳宴於曲江亭。乙亥,右散騎常侍李翱卒。戊寅,幸魚藻宮觀競渡。辛巳,以同州刺史蕭俯為太子少保分司。壬午,以工部尚書裴武為同州刺史。癸未,嶺南節度使崔植奏:「廣、湖、封、雷、潘、辯等七州戍軍。除折沖別將外,並請停。」從之。丙戌,昆明夷遣使朝貢。丁亥,敕冊才人郭氏為貴妃。丙申,以吏部侍郎韋弘景為陜虢觀察使。四月戊戌朔,橫海軍
 節度使李全略卒。壬寅,以右金吾衛大將軍高承簡為邠、寧、慶節度使。丙午,王廷湊檢校司空。戊申,昭義節度使留後劉從諫檢校工部尚書,充昭義節度副大使、知節度事。庚戌,鄂岳觀察使牛僧孺奏:「當道沔州與鄂州隔江相對,才一里餘,其州請並省,其漢陽、汊川兩縣隸鄂州。」從之。丙辰,右金吾衛大將軍高霞寓卒。丙寅,先是王廷湊請於當道立聖德碑,是日,內出碑文賜廷湊。



 五月戊辰朔,上御宣和殿,對內人親屬一千二百人,並於
 教坊賜食,各頒錦彩。辛未,秘書省著作郎韋公肅注太宗所撰《帝範》十二篇進,特賜錦彩百匹。甲戌,以涇原節度楊元卿為河陽三城懷州節度使,以金吾衛大將軍李祐為涇原節度使。是夜,月近太微星。浙西送到絕粒女道士施子微。戊寅,幸魚藻宮觀競渡。庚辰,中使自新羅取鷹鷂回。幽州軍亂,殺其帥硃克融及男延齡,軍人立其第二子延嗣為留後。辛巳,神策軍苑內古長安城中修漢未央宮,掘獲白玉床一張,長六尺。癸未,山人杜
 景先於光順門進狀,稱有道術;令中使押杜景先往淮南及江南、湖南、嶺南諸州求訪異人。甲申,以右丞丁公著為兵部侍郎,以前湖南觀察使沈傳師為尚書左丞。辛卯,贈硃克融司徒。甲午夜,熒惑犯昴。賜興唐觀道士劉從政修院錢二萬貫。



 六月丁酉朔,賜御史中丞獨孤朗金紫。丁巳,減放苑內役人二千五百。帝性好土木,自春至冬,興作相繼。庚申,鄆州進驢打球人石定寬等四人。是夜,太白犯昴。辛酉,幸凝碧池,令兵士千餘人於池
 中取大魚,長大者送入新池。癸亥,以旱,命京城諸司疏理系囚。以延康坊官宅一區為諸王府司局。甲子,上御三殿,觀兩軍、教坊、內園分朋驢鞠、角抵。戲酣,有碎首折臂者,至一更二更方罷。



 秋七月丙寅朔。乙亥,河中進力士八人。癸未,衡王絢薨。敕鄠縣渼陂尚食管系,太倉廣運潭復賜司農寺。



 八月丙申朔,以司空、平章事裴度判度支;以工部侍郎王播為河南尹,代王起;以起為吏部侍郎;以前福州觀察使徐晦為工部侍郎。是夜,太
 白近太微。令供奉道士二十人隨浙西處士周息元入內宮之山亭院,上問以道術,言識張果、葉靜能。浙西觀察使李德裕上疏言息元誕妄,無異於人。庚戌,以太府卿李憲為江西觀察使。丁丑夜,月犯輿鬼。加京兆尹劉棲楚兼御史大夫。癸丑,以太常卿崔從檢校吏部尚書、判東都尚書省事、兼御史大夫、東都留守、東畿汝都防禦使。



 九月丁丑朔,大合宴於宣和殿,陳百戲,自甲戌至丙子方已。戊寅,河東節度使、守司徒、兼侍中李光顏卒。
 出內庫錢萬貫,令內園召募力士。幽州鹽軍奏:都知兵馬使李再義與弟再寧同殺硃延嗣並其家屬三百餘人,推再義為留後。壬申,宰相李程為北都留守、河東節度使。敕戶部所管同州長春宮莊宅,宜令內莊宅使管系。



 冬十月乙未朔。乙亥,以幽州衙前都知兵馬使李再義檢校戶部尚書,充盧龍軍節度副大使、知節度事,仍賜名載義。壬戌,以中書舍人崔郾為禮部侍郎。



 十一月甲子朔,以太清宮道士趙歸真充兩街道門都教授博
 士。帝好深夜自捕狐貍,宮中謂之「打夜狐」。中官許遂振、李少端、魚弘志以侍從不及削職。壬申,以戶部尚書胡證檢校兵部尚書,兼廣州刺史,充嶺南節度使。甲申,以右僕射、同平章事李逢吉檢校司空、同平章事,兼襄州刺史,充山南東道節度使、臨漢監牧使。乙酉,同州刺史裴武卒。己丑,詔朝官及方鎮人家不得置私白身。癸巳,以前東都留守楊於陵為太子少傅。中官李奉義、王惟直、成守貞各杖三十,分配諸陵;宣徽使閆弘約、副使劉
 弘逸各杖二十。十二月甲午朔。辛丑,帝夜獵還宮,與中官劉克明、田務成、許文端打球,軍將蘇佐明、王嘉憲、石定寬等二十八人飲酒。帝方酣,入室埂衣,殿上燭忽滅,劉克明等同謀害帝,即時殂於室內,時年十八。群臣上謚曰睿武昭愍孝皇帝,廟號敬宗。大和元年七月十三日葬於莊陵。



 史臣曰:古人謂堯無子,舜無父,言其賢不肖之相遠也。以文惠驕誕之性,繼之以昭愍,固其宜也。而昭獻、昭肅,
 英特不群,文足以緯邦家,武足以平禍亂。三子之操行頓異,其可道哉?寶歷不群,國統幾絕,天未降喪,幸賴裴度,復任弼諧。彼狡童兮,夫何足議!



 文宗元聖昭獻孝皇帝諱昂,穆宗第二子,母曰貞獻皇后蕭氏。元和四年十月十日生。;長慶元年封江王。初名涵。寶歷二年十二月八日,敬宗遇害,賊蘇佐明等矯制立絳王勾當軍國事。樞密使王守澄、中尉梁守謙率禁軍討賊,誅絳王,迎上於江邸。癸卯,見宰臣於閣內,下教
 處分軍國事。甲辰,僧惟真、齊賢、正簡,道士趙歸真,並配流嶺南,擊球軍將於登等六人令本軍處置。宰臣百僚三上表勸進。乙巳,即位於宣政殿。丙午,上赴西宮成服。丁未,宰臣百僚上表請聽政,三表,許之。道士紀處玄、楊沖虛,伎術人李元戢、王信等,並配流嶺南。戊申,尊聖母為皇太后。己酉,敕鳳翔、淮南先進女樂二十四人,並放歸本道。庚戌,以正議大夫、尚書兵部侍郎、知制誥、充翰林學士、柱國、賜紫金魚袋韋處厚為中書侍郎、同中書
 門下平章事。以翰林學士路隨承旨,侍講學士宋申錫充書詔學士。丙辰,以山南道節度使柳公綽為刑部尚書。丁己,為絳王舉哀,廢朝三日。庚申,詔:



 君天下者,莫尚乎崇澹泊,子困窮,遵道以端本,推誠而達下。故聖祖之誡,以慈儉為寶;大《易》明訓,垂簡易之文。未有上約而下不豐,欲寡而求不給。朕以眇薄,遭逢內難,刷君父之仇恥,攄億兆之哀冤。而股肱大臣,群卿庶士,引義抗請,至於再三。以圖宗社之安,以答華夷之望,俯從眾欲,夙
 夜震兢。思所以克己復禮,修政安人,宵興匪寧,旰食勞慮。夫儉過則酌之以禮,文勝則矯之以質。庶乎俗合太古,道洽生靈,儀刑家邦,以化天下。內庭宮人非職掌者,放三千人,任從所適。長春宮斛斗諸物,依前戶部收管。鄠縣、渼陂、鳳翔府駱谷地還府縣。教坊樂官、翰林待詔、伎術官並總監諸色職掌內冗員者共一千二百七十人,並宜停廢。總監中一百二十四人先屬諸軍,並各歸本司。餘七百三人,勒納牒身,放歸本管。先供教坊衣糧
 一百分,廂家及諸司新加衣糧三千分,並宜停給。五方鷹鷂並解放。今年新宣附食度支衣糧小兒一百人,並停給。別詔宣索纂組雕鏤不在常貢內者,並停。度支、鹽鐵、戶部及州府百司應供宮禁年支一物已上,並準貞元元額為定。先造供禁中床榻以金筐瑟瑟寶鈿者,悉宜停造。東頭御馬坊、球場,宜卻還龍武軍。其殿及亭子,所司毀拆,餘舍賜本軍。應行從處張陳,不得用花蠟結彩華飾。今年已來諸道所進音聲女人,各賜束帛放還。城
 外墳墓先有開劚以備行幸處,宜曉示百姓,任其修塞。其大逆魁首蘇佐明等二十八人,並已處斬,宗族籍沒。妖妄僧惟貞、道士趙歸真等或假於卜筮,或托以醫方,疑眾挾邪,已從流竄。其情非奸惡,跡涉詿誤者,一切不問。兇徒既殄,寰宇佇康,載舉令猷,用弘庶績。布告中外,知朕意焉。



 帝在籓邸,知兩朝之積弊,此時厘革,並出宸衷,士民相慶,喜理道之復興矣。壬戌,以前江西觀察使殷侑為大理卿。



 大和元年春正月癸亥朔。庚午,以御史中丞獨孤朗為戶部侍郎,以兵部尚書、權判左丞事段文昌為御史大夫。是夜,月掩畢大星。戊寅,以左散騎常侍李益為禮部尚書致仕,以京兆尹劉桂楚為棲管觀察使。以前戶部侍郎於敖為宣歙觀察使,代崔群;以群為兵部尚書。癸未,以吏部侍郎庾承宣為京兆尹、兼御史大夫。丙申,復置兩輔、六雄、十望、十緊三十四州別駕。其諸色在京及內外諸軍使等職事,並不在挾名限。己亥,以右散騎常
 侍、集賢殿學士、判院事張政甫為工部尚書。辛丑,以前廣州節度使崔植為戶部尚書,以太子少師、分司東都李絳檢校司空,兼太常卿。乙巳,御丹鳳樓,大赦,改元大和。甲寅,敕諸道節度觀察使去任日,宜具交割狀,仍限新使到任一月分析聞奏,以憑0殿最。丙辰,以華州刺史錢徽為尚書右丞,以前河陽節度使崔弘禮為華州鎮國軍使。己未,以太子少保分司蕭俯為檢校右僕射,兼禮部尚書。以虔州刺史韓約為安南都護。



 三月庚
 戌朔,右軍中尉梁守謙請致仕,以樞密使王守澄代。戊寅,以前蘇州刺史白居易為秘書監,仍賜金紫。壬午,幽州李載義奏故張弘靖判官家屬凡一百九十人,並送赴闕。四月壬辰朔。癸巳,以太子少傅楊於陵守右僕射致仕,俸料全給。甲午,鳳翔築臨汧城於汧陽縣西北八十里。壬寅,毀升陽殿東放鴨亭;戊申,毀望仙門側看樓十間:並敬宗所造也。以前亳州刺史張遵為邕管經略使。乙卯,以禮部尚書蕭俯為太子少師分司。己未,忠
 武軍節度使王沛卒。庚申,以太僕卿高瑀檢校左散騎常侍,充忠武軍節度。己巳,貶山南東道節度副使李續為涪州刺史,山南東道行軍司馬張又新為汀州刺史,李逢吉黨也。



 五月壬戌朔。戊辰,詔:「元首股肱,君臣象類,義深同體,理在坦懷。夫任則不疑,疑則不任。然自魏、晉已降,參用霸制,虛議搜索,因習尚存。朕方推表大信,置人心腹,庶使諸侯方岳,鼓洽道化,夷貊飛走,暢泳治功。況吾臺宰,又何間焉。自今已後,紫宸坐朝,眾僚既退,宰
 臣復進奏事,其監搜宜停。」丙子,以天平軍節度使、守司徒、同中書門下平章事烏重胤為橫海軍節度使;以前攝橫海軍節度副使、檢校國子祭酒、侍御史李同捷檢校左散騎常侍,兼兗州刺史,充兗海沂密等州節度使。就加魏博史憲誠同平章事。甲申,淮南節度、鹽鐵、轉運等使王播來朝。丙戌夜,熒惑犯右執法。



 六月辛卯朔,敕文武常參官朝參不到,據料錢多少,每貫罰二十五文。癸巳,以淮南節度副大使、知節度事、管內營田觀察處
 置臨海監牧等使,兼諸道鹽鐵轉運等使、銀青光祿大夫、檢校司空、同中書門下平章事、揚州大都督府長史、上柱國、太原縣開國伯、食邑七百戶王播可尚書左僕射、同中書門下平章事,依前充諸道鹽鐵轉運使。以御史大夫段文昌代播為淮南節度使。丙申,左司郎中、兼侍御史知雜溫造權知御史中丞。癸卯,詔:「元和、長慶中,皆因用兵,權以濟事,所下制敕,難以通行。宜令尚書省取元和已來制敕,參詳刪定訖,送中書門下議定奏。」
 甲寅,以旱放系囚。七月辛酉朔。癸亥,太常卿李絳進封魏國公。李同捷除兗、海,不受詔,結幽鎮謀叛。癸酉,葬敬宗於莊陵。辛巳,敕今年權於東都置舉。徐州王智興請全軍討李同捷。



 八月庚寅朔,以工部侍郎獨孤朗為福建觀察使,以太府卿裴弘泰為黔中經略使、觀察使。左僕射致仕楊於陵讓全給俸料,許之。庚子,詔削奪李同捷在身官爵,復以張茂宗為兗、海、沂、密節度使。辛丑,邠寧節度使高承簡卒。壬寅,以刑部尚書柳公綽檢校左
 僕射,充邠寧節度使。戊申,以諫議大夫張仲方為福建觀察使。癸丑,前福建觀察使獨孤朗卒。



 九月庚申朔。癸亥,以左神軍將軍、知軍事何文哲為鄜、坊、丹、延節度使。甲戌,以左神策軍、知軍事李泳為單于都護,充振武、麟勝節度使。丁丑,浙西觀察使李德裕、浙東觀察使元稹就加檢校禮部尚書。壬午,桂管觀察使劉棲楚卒。丙戌,以諫議大夫蕭裕為桂管觀察使。癸丑,兗州復置萊蕪縣。



 十一月己未朔。丙申,河中薛平奏虞鄉縣有
 白虎入靈峰觀。天平、橫海等軍節度使、守司徒,同中書門下平章事烏童胤卒。庚辰,以保義軍節度、晉慈等察處置等使李寰為橫海軍節度使。癸巳,以晉州、慈州復隸河中。癸巳,以左丞錢徽為華州刺史。丁酉,右金吾衛大將軍王公亮為潭州刺史、湖南觀察使。



 二年春正月戊午朔。壬申,以右散騎常侍孔戢為京兆尹。



 二月丁亥朔,以兵部侍郎王起為陜虢觀察使,代韋弘景;以弘景為尚書左丞。乙己,以刑部侍郎盧元輔為
 兵部侍郎,秘書監白居易為刑部侍郎。庚戌,敕李絳所進則天太后刪定《兆人本業》三卷,宜令所在州縣寫本散配鄉村。



 三月丁巳朔,度支奏:「京兆府奉先縣界鹵池側近百姓,取水柏柴燒灰煎鹽,每一石灰得鹽一十二斤一兩,亂法甚於咸土,請行禁絕。今後犯者據灰計鹽,一如兩池鹽法條例科斷。」從之。辛巳,上御宣政殿親試制策舉人。以左散騎常侍馮宿、太常少卿賈餗、庫部郎中龐嚴為考制策官。閏三月丙戌朔,內出水車樣,令京
 兆府造水車,散給緣鄭白渠百姓,以溉水田。



 夏四月丙辰朔。壬午,以邕管經略使王茂元為容管經略使。



 五月乙酉朔。丁巳,命中使於漢陽公主及諸公主第宣旨:「今後每遇對日,不得廣插釵梳,不須著短窄衣服。乙未,以吏部侍郎丁公著為禮部尚書。庚子,敕:「應諸道進奉內庫,四節及降誕進奉金花銀器並纂組文纈雜物,並折充鋌銀及綾絹。其中有賜與所須,待五年後續有進止。」帝性恭儉,惡侈靡,庶人務敦本,故有是詔。帝與侍講學
 士許康佐語及取蚺蛇膽,生剖其腹,為之惻然。乃詔度支曰:「每年供進蚺蛇膽四兩,桂州一兩、賀州二兩、泉州一兩,宜於數內減三兩,桂、賀、泉三州輪次歲貢一兩。」帝自撰集《尚書》中君臣事跡,命畫工圖於太液亭,朝夕觀覽焉。王廷湊出兵侵鄰籓,欲撓王師,以援李同捷,昭義劉從諫請出軍討之。



 六月乙卯朔,晉王普薨,贈為悼懷太子。陳州水,害秋稼。癸亥,四方館請賜印,其文以「中書省四方館」為名。辛酉,以吏部尚書鄭絪為太子少保。辛
 巳,以靈武節度使李進誠為邠寧節度使,以天德軍使李文悅為靈武節度使。乙酉,以前邠寧節度使柳公綽檢校左僕射,兼刑部尚書。甲辰,詔宰臣集三署四品已上常參官,議討王廷湊可否。是夜,彗西出攝提南,長二尺。



 八月申寅朔。丁巳,以兵部侍郎盧元輔為華州鎮國軍使,以代錢徽;以徽為吏部尚書致仕。壬戌,京畿奉先等十七縣水。



 九月甲申朔。丁亥,王智興拔棣州。以新除橫海軍節度使李寰為夏州節度使。甲午,詔削奪王遷
 湊在身官爵,鄰道接界隨便進討。以前夏州節度使傅良弼為橫海軍節度使。庚戌,安南軍亂,逐都護韓約。



 冬十月癸丑朔。丁巳,罷揚州海陵監牧。以戶部尚書崔植為華州刺史、鎮國軍使。丙寅,嶺南節度使胡證卒。辛未,以江西觀察使李憲為嶺南節度使。癸酉,以尚書右僕射、同平章事竇易直檢校左僕射、同平章事,充山南東道節度使、臨漢監牧等使,代李逢吉;以逢吉為宣武軍節度使,代令狐楚;以楚為戶部尚書。以右丞沈傳師為
 江西觀察使。己卯,以河南尹王璠為右丞,以左散騎常侍馮宿為河南尹。



 十一月癸未朔。乙酉,以右金吾衛大將軍李祐為橫海軍節度使,新除傅良弼赴鎮,卒於陜州故也。甲辰,禁中巳時昭德寺火,直宣政殿之東,至午未間,北風起,火勢益甚,至暮稍息。十二月壬子朔。乙丑,魏博行營都知兵馬使亓志紹率所部兵馬二萬人謀叛,欲殺史憲誠父子。壬申,中書侍郎、同平章事韋處厚暴卒。戊寅,詔以兵部侍郎、知制誥、充翰林學士路隨為
 中書侍郎、同平章事。



 三年春正月壬午朔。丙戌,亓志紹率兵回據永濟縣,其眾分散入諸縣邑。史憲誠告難,詔滄州行營兵士赴之。丁亥,京兆尹孔戢卒。庚寅,吏部尚書致仕錢徽卒。庚子,李聽殺敗亓志紹兵,志紹北走鎮州。甲辰,以太常卿李絳檢校司空,兼興元尹、山南西道節度使。華州刺史、鎮國軍潼關防禦使崔植卒。己酉,以前山南西道節度使王涯為太常卿。



 二月辛亥朔,以兵部尚書崔群為荊南
 節度使。甲寅,荊南節度使王潛卒。



 三月辛巳朔,以戶部尚書令狐楚為東都留守。乙酉,敕兵戈未息,教坊每日祗候樂人宜權停。壬辰,易定節度使柳公濟卒。以前東都留守崔從為戶部尚書。



 夏四月庚午,王智興奏部下將石雄搖扇軍情,請行朝典,乃長流白州。



 五月己卯朔。甲申,柏耆斬李同捷於將陵,滄景平,李祐入滄州。丁亥,御興安樓,受滄州所獻。李祐送李同捷母、妻及男元達等赴闕,詔並宥之,令於湖南安置。貶滄德宣慰使、諫議
 大夫柏耆循州司戶,宣慰判官、殿中侍御史沈亞之虔州南康尉,以擅入滄州取李同捷,諸鎮所怒,奏論之也。丙申,橫海軍節度使李祐卒。以涇原節度使李岵為齊、德等州節度使,改名有裕。丁酉,以前義武軍節度使傅毅為滄州刺史、橫海軍節度使。辛丑,以右金吾衛大將軍張惟清檢校司空,充涇原節度使;以左金吾衛大將軍劉遵古為邠寧節度使。



 六月己酉朔。辛亥,以魏博節度使史憲誠檢校司徒、兼侍中、河中尹,充河中晉絳節
 度使;以義成軍節度使李聽兼充魏博節度使;以魏博節度副使、檢校工部尚書史孝章為相衛節度使。壬申,敕:「元和四年敕禁鉛錫皆納官,許人糾告,一錢賞百錢,此為太過。此後以鉛錫錢交易者,一貫以下,州府常行杖決脊杖二十;十貫以下決六十,徒三年;過十貫已上,集眾決殺。能糾告者,一貫賞錢五十文。」秋七月己卯朔。癸未,中使劉弘逸送史憲誠旌節自魏州還,稱六月二十六日夜,魏博軍亂,殺史憲誠,立大將何進滔為留
 後,其新節度使李聽入城不得。乙丑,河中節度使薛平依前河中節度使。乙未,嶺南節度使李憲卒。兵部侍郎盧元輔卒。丁酉,以京兆尹崔護為御史大夫、廣南節度使。戊戌,以大理卿李諒為京兆尹。乙巳,以禮部尚書、翰林侍講學士丁公著檢校戶部尚書,兼潤州刺史,充浙江西道觀察使;以前浙西觀察使、檢校禮部尚書李德裕為兵部侍郎。辛亥,魏博何進滔奏:準詔割相、衛三州,三軍不受。壬子,詔以魏博衙內都知兵馬使何進滔檢
 校左散騎常侍,充魏博節度使。癸丑,以衛尉卿殷侑檢校工部尚書,為齊德滄節度使。辛酉,京畿、奉先等九縣旱,損田。播州流人衛中行卒,宋、亳水害稼。壬申,詔雪王廷湊,復官爵。甲戌,以吏部侍郎李宗閔同中書門下平章事。



 九月戊寅朔。辛巳,敕兩軍、諸司、內官不得著紗縠綾羅等衣服。帝性儉素,不喜華侈。駙馬韋處仁戴夾羅巾,帝謂之曰:「比慕卿門地清素,以之選尚。如此巾服,從他諸戚為之。唯卿非所宜也。」壬辰,以兵部侍郎李德裕
 檢校永部尚書,兼滑州刺史、義成軍節度使。戊戌,以前睦州刺史陸亙為越州刺史、浙東觀察使,代元稹;以稹為尚書左丞,代韋弘景;以弘景為禮部尚書。



 冬十月戊申朔。己酉,江西沈傳師奏:皇帝誕月,請為僧尼起方等戒壇。詔曰:「不度僧尼,累有敕命。傳師忝為籓守,合奉詔條,誘致愚妄,庸非理道,宜罰一月俸料。」丙辰,以前義成軍節度使李聽為太子少師。癸亥,以戶部侍郎崔元略為戶部尚書、判度支。以中書舍人韋辭為湖南觀察使。



 十一月丁丑朔。庚辰,太子太傅鄭絪卒。丙戌,敕前亳州刺史李繁於京兆府賜死。甲申,帝親祀昊天上帝於南郊,禮畢,御丹鳳門,大赦。節文禁止奇貢,云:「四方不得以新樣織成非常之物為獻,機杼纖麗若花絲布繚綾之類,並宜禁斷。敕到一月,機杼一切焚棄。刺史分憂,得以專達。事有違法,觀察使然後奏聞。」丙申,西川奏南詔蠻入寇。甲辰,王智興來朝。乙巳,以智興守太傅,依前平章事、武寧軍節度使,進封雁門郡王。十二月丁未朔,南蠻
 逼戎州,遣使起荊南、鄂岳、襄鄧、陳許等道兵赴援蜀川。以劍南東川節度使郭釗為西川節度使,仍權東川事。壬子,貶劍南西川節度使杜元穎為韶州刺史。遣中使楊文端齎詔賜南蠻王蒙豐佑。蠻軍陷邛、雅等州。戊午,以右領軍衛大將軍董重質充神策西川行營都知兵馬使。西川奏蠻軍陷成都府。東川奏蠻軍入梓州西郭門下營。又詔促諸鎮兵救援西川。己丑,以東都留守令狐楚檢校右僕射、天平軍節度使,代崔弘禮為東都留
 守。丁卯,貶杜元穎循州司馬。乙巳,郭釗奏蠻軍抽退,遣使賜蠻帥蒙巔國信。辛未,以太子少師李聽為邠寧節度使。癸酉,以中丞溫造為右丞,吏部郎中宇文鼎為中丞。



\end{pinyinscope}