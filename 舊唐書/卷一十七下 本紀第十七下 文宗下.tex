\article{卷一十七下 本紀第十七下 文宗下}

\begin{pinyinscope}

 大和四
 年春正月丙子朔。辛卯,武昌軍節度使牛僧孺來朝。丙戌,以左神策軍大將軍丘直方為鄜坊節度使。戊子,詔封長男永為魯王。辛卯,以武昌節度使、鄂岳蘄黃安申等觀察處置等使、金紫光祿大夫、檢校吏部尚書、同中書門平章事、上柱國、奇章郡開國公牛僧孺為兵部尚書、同中書門下平章事。壬辰,以兵部侍郎崔郾為陜虢觀察使。封魯王母王氏為昭儀。癸巳,以前邠寧節度使劉遵古為劍南東川節度使。甲午,守左僕射、同
 平章事,諸道鹽鐵轉運使王播卒。丙申,以太常卿五涯為吏部尚書,充諸道鹽鐵轉運使。辛丑,以尚書左丞元稹檢校戶部尚書,充武昌軍節度、鄂岳蘄黃安申等州觀察使。癸卯,以前陜虢觀察使王起為左丞。二月丙午朔。戊午,興元軍亂,節度使李絳舉家被害,判官薛齊、趙存約死之。庚申,以左丞溫造為興元節度使。辛未,夏州節
 度使
 李寰卒。壬申,以神策行營節度使董重質為夏、綏、銀、宥節度使。三月乙亥,以河東節度使李程檢校左僕射、同平章事,兼河中尹、晉絳慈隰等州節度使,以刑部尚書柳公綽檢校左僕射、太原尹、北都留守、河東節度使。丁丑,以前河中節度使薛平為太子太保。丁
 亥,以衛尉卿桂仲武為福建觀察使。興元溫造奏:「害李絳賊首丘崟、丘鑄及官健千人,並處斬訖。其親刃絳者斬一百段,號令者三段,餘並斬首。內一百首祭李絳,三十首祭死王事官僚,其餘尸首並投於漢江。」己丑,詔興元監軍使楊叔元宜配流康州百姓,錮身遞於配所。丁酉,監修國史、中書侍郎、平章事路隨進所撰《憲宗實錄》四十卷,優詔答之,賜史官等五人錦繡銀器有差。癸卯,以淮南節度使段文昌檢校尚書左僕射、同中書門下
 平章事,兼江陵尹,充荊南節度使;以前太子賓客崔從檢校右僕射、揚州大都督府長史、淮南節度使。甲辰,以前荊南節度使崔群檢校右僕射、兼太常卿。以中書舍人李虞仲為華州刺史,代嚴休復;以休復為右散騎常侍。



 夏四月乙巳朔。丙午,以右散騎常侍、翰林侍講學士鄭覃為工部尚書。丁未,兵部尚書致仕張賈卒。丁巳,貶前齊德滄景等州節度使李有裕為永州刺史,馳驛赴任。庚申,以尚書左丞王起為戶部尚書、判度支,代崔元
 略;以元略檢校吏部尚書,為東都留守。辛酉,月掩南斗第二星。壬戌,詔曰:「儉以足用,令出惟行,著在前經。斯為理本。朕自臨四海,愍元元之久困,日昃忘食,宵興疚懷。雖絕文繡之飾,尚愧茅茨之儉。亦諭卿士,形於詔條。如聞積習流弊,餘風未革。車服第室,相高以華靡之制;資用貨寶,固啟於貪冒之源。有司不禁,侈俗滋扇。蓋朕教導之未敷,使兆庶昧於恥尚也。其何以足用行令,臻於致理歟!永念慚嘆,迨茲申敕。自今內外班列職位之
 士,各務素樸,弘茲國風。有僭差尤甚者,御史糾上。主者宣示中外,知朕意焉。」文宗承長慶、寶歷奢靡之風,銳意懲革,躬行儉素,以率厲之。辛未,以前東都留守崔弘禮為刑部尚書。鎮州王廷湊請修建初、啟運二陵,從之。



 五月甲戌朔。丁丑,以旱命京城諸司疏理系囚。己卯,通化南北二門鎖不可開,鑰入,如有持之者。上令鐵工破鎖,時日己及辰矣。丁亥,改鄆州東平縣為天平縣。戊子,敕度支每歲於西川織造綾羅錦八千一百六十七匹,令
 數內減二千五百十匹。



 六月癸卯朔。丁未,以守司徒、門下侍郎、平章事、上柱國、晉國公、食邑三千戶、食實封三百戶裴度為守司徒、平章軍國重事;待疾損日,每三日、五日一度入中書。辛未夜,自一更至五更,大小星流旁午,觀者不能數。壬申,詔:如聞諸司刑獄例多停滯,委尚書左右丞及監察御史糾舉以聞。



 秋七月癸酉朔。癸未,詔以朝議郎、尚書右丞、上柱國、賜紫金魚袋宋申錫為正議大夫、行尚書右丞、同中書門下平章事。乙酉,敕:「前
 行郎中知制誥者,約滿一周年,即與正授;從諫議大夫知者,亦宜準此;餘依長慶二年七月二十七日敕處分。」振武置雲伽關,加鎮兵千人。以吏部侍郎王璠為京兆尹、兼御史大夫,代李諒為桂管觀察使。太原饑,賑粟三萬石。賜十六宅諸王綾絹二萬匹。丁酉,守司徒裴度上表辭冊命,言:「臣此官已三度受冊,有靦面目。」從之。



 八月壬寅朔。丙辰,鄜州水,溺居民三百餘家。太原柳公綽奏雲、代、蔚三州山谷間石化為面,人取食之。己未,宣歙觀
 察使於敖卒。甲子,內出綾絹三十萬匹,付戶部充和糴。戊辰,幸梨園亭,會昌殿奏新樂。



 九月壬申朔。丁丑,以大理卿裴誼檢校右散騎常侍,充江西觀察使,代沈傳師;以傳師為宣歙觀察使。內出綾三千匹,賜宥州築城兵士。戊寅,舒州太湖、宿松、望江三縣水,溺民戶六百八十,詔以義倉賑貸。庚辰,吏部尚書王涯為右僕射,依前鹽金歲轉運使。壬午,以守司徒、平章軍國重事、晉國公裴度守司徒、兼待中,充山南東道節度使。以投來奚王茹羯
 為右驍衛將軍同正。丙戌,以前山南東道節度使竇易直為尚書左僕射。戊子,吏部尚書致仕裴向卒。己丑,淮南天長等七縣水,害稼。丁酉,前豐州刺史、天德軍使渾鐵坐贓七千貫,貶哀州司馬。



 冬十月壬寅朔。戊申,以東都留守崔元略檢校吏部尚書,兼滑州刺史、義成軍節度使,代李德裕;以德裕檢校兵部尚書,兼成都尹。充劍南西川節度使、
 檢校司空郭釗為太常卿,代崔群為吏部尚書。丁卯,御史中丞宇文鼎奏:「今月十三日,宰臣宣旨,今後群臣延英奏事,前一日進狀入來者。臣以尋常公事,不暇面論,但見表章,足以陳露。儻臨時忽有公務,文字不足盡言,則咫尺天聽,無路聞達。更俟後坐,動逾數辰,處置之間,便有不及。伏乞重賜宣示,限以狀入者,並在卯前;如在卯後,聽不收覽。自然人各遵守,禮亦得中。」從之。



 十一月辛未朔。是夜,熒惑近左執法。癸巳,以左丞康承宣為
 兗、海、沂、密等州節度使。淮南大水及蟲霜,並傷稼。十二月辛丑朔,滄州殷侑請廢景州為景平縣。己酉,義成軍節度使崔元略卒。壬子,以左金吾衛大將軍段嶷為義成軍節度使。癸丑,湖南觀察使韋辭卒。丙辰,以工部侍郎崔琯為京兆尹,代王璠為尚書左丞。癸亥,東都留守崔弘禮卒。以同州刺史高重為潭州刺史、兼御史中丞,充湖南觀察使。甲子,左僕射致仕楊於陵卒,贈司空。丙寅,以前河南尹馮宿為工部侍郎。戊辰,以太子賓客分
 司白居易為河南尹,以代韋弘景;以弘景守刑部尚書、東都留守。閏十二月辛未朔。壬申,太常卿郭釗卒,贈司徒。壬辰,廢齊州歸化縣地入臨邑縣。廢是州,其縣隸滄州刺史。是歲,京畿、河南、江南、荊襄、鄂岳、湖南等道大水,害稼,出官米賑給。



 五年春正月庚子朔,以積陰浹旬,罷元會。丁巳,賜滄德節度使曰義昌軍。太原旱,賑粟十萬石。己未,詔方鎮節度觀察使請入觀者,先上表奏聞,候充則任進程,庚申,
 幽州軍亂,逐其帥李載義,立後院副兵馬使楊志誠為留後。癸亥詔端午節辰,方鎮例有進奉,其雜彩匹段,許進生白綾絹。己丑,以權知渤海國務大彞震檢校秘書監、忽汗州都督、渤海國王。



 二月庚午朔。壬辰,以盧龍軍節度使、守太保、同平章事李載義守太保、同中書門下平章事。時載義失地入朝,賜第於永寧里,給賜優厚。丙申,以桂管觀察使李諒為嶺南節度使。戊戌,神策中尉王守澄奏得軍虞候豆盧著狀,告宰相宋申錫與漳王
 謀反。即令追捕。庚子,詔貶宋申錫為太子右庶子。壬寅,左常侍崔玄亮及諫官等十四人伏奏玉階:「北軍所告事,請不於內中鞫問,乞付法司。」帝曰:「吾已謀於公卿矣,卿等且退。」崔玄亮泣涕陳諫久之,帝改容勞之曰:「朕即與宰臣商議。」玄亮等方退。癸卯,詔漳王湊可降為巢縣公,右庶子宋申錫開州司馬同正。初,京師忷忷,以宰相實聯親王謀逆,三四日後,方知誣構。人士側目於守澄、鄭注,故諫官號泣論之。申錫方免其禍。己酉,敕以李載
 義入朝,於曲江亭賜宴,仍命宰臣百僚赴會。辛酉,以黔中觀察使裴弘泰為桂管經略使,以前安州刺史陳正儀為黔中觀察使。丁卯,紫宸奏事,宰相路隨至龍墀,僕於地,令中人掖之。翌日,上疏陳退,識者嘉之。



 夏四月己巳,甲戌,以新羅王嗣子金景徽為開府儀同三司、檢校太保,使持節雞林州諸軍事、雞林州大都督、寧海軍使、上柱國,封新羅王;仍封其母樸氏為新羅國太妃。丁亥,詔:「史官記事,用戒時常,先朝舊制,並得隨仗。其後宰
 臣撰時政記,因循斯久,廢墜實多。自今後宰臣奏事,有關獻替及臨時處分稍涉政刑者,委中書門下丞一人隨時撰錄,每季送史館,庶警朕闕,且復官常。」己丑,以李載義為山南西道節度,依前守太保、同平章事,代溫造;以造為兵部侍郎。以幽州盧龍節度留後楊志誠檢校工部尚書,為幽州盧龍節度使。



 五月戊戌朔,太廟第四室、第六室破漏,有司不時修葺,各罰俸。上命中使領工徒及以禁中修營材葺之。右補闕韋溫上疏論曰:「宗廟
 不葺,罪在有司弛慢,宜加重責。今有司止於罰俸,便委內臣葺修,是許百司之官公然廢職。以宗廟之重,為陛下所私,則群官有司便同委棄,此臣竊為聖朝惜也。事關宗廟,皆書史冊,茍非舊典,不可率然。伏乞更下詔書,復委所司營葺,則制度不紊,官業各修矣。」疏奏,帝嘉之,乃追止中使,命有司修奉。戊午,西川李德裕奏:南蠻放還先擄掠百姓、工巧、僧道約四千人還本道。辛酉,東都留守、刑部尚書韋弘景卒。丙寅,以京兆尹崔琯為尚書
 左丞。太常少卿龐嚴權知京兆尹。



 六月丁卯朔。戊寅,以霖雨涉旬,詔疏理諸司系囚。辛卯,蘇、杭、湖南水害稼。甲午,東川奏:玄武江水漲二丈,梓州羅城漂人廬舍。



 秋七月丁酉朔。庚子,贈太子賓客李渤禮部尚書。辛丑,以兵部侍郎溫造檢校戶部尚書,為東都留守。甲辰,以太了少師分司、上柱國、襲徐國公蕭俯守左僕射致仕。劍南東、西兩川水,遣使宣撫賑給。己未,以給事中羅讓為福建觀察使。



 八月丙寅朔。庚午,武昌軍節度使、檢校戶部
 尚書元稹卒。辛未,貶刑部員外郎舒元輿為著作郎。元輿累上表請自效,並進文章,朝議責其躁進也。壬申,以河陽三城懷州節度使楊元卿為宣武軍節度使,代李逢吉;以逢吉檢校司徒、兼太子太師,充東都留守,代溫造;以溫造為河陽三城懷州節度使。戊寅,以陜虢觀察使崔郾為鄂岳安黃觀察使。甲申,以中書舍人崔咸為陜州防禦使。詔陜州舊有都防禦觀察使額宜停,兵馬屬本州防禦使。丙戌,京兆尹龐嚴卒。庚寅,以司農卿、駙
 馬都尉杜忭為京兆尹。



 九月丙申朔。甲辰,貶太子左庶子郭求為婺王府司馬,以其心疾,與同僚忿競也。翰林學士薛廷老、李讓夷皆罷職守本官。廷老在翰林,終日酣醉無儀檢,故罷。讓夷常推薦廷老,故坐累也。己未,以左僕射竇易直判太常卿。西川李德裕奏收復吐蕃所陷維州,差兵鎮守。



 冬十月乙丑朔,以前綿州刺史鄭綽為安南都護。戊寅,蠻寇雋州,陷二縣。辛巳,滄州移清池縣於南羅城內置。



 十一月乙未朔。庚戌,鳳翔節度使王
 承元來朝。己未,以承元檢校司空、青州刺史,充平盧軍節度使。癸亥,以尚書左僕射、判太常卿事竇易直檢校司空,為鳳翔隴右節度使。十二月乙丑朔。戊寅,以左丞王璠兼判太常卿事。甲申,貶新除桂管觀察使裴弘泰為饒州刺史,以除鎮淹程不進,為憲司所糾故也。癸巳,以鄭州刺史李翱為桂管觀察使。是歲,淮南、浙江東西道、荊襄、鄂岳、劍南東川並水,害稼,請蠲秋租。京師大雨雪。



 六年春正月乙未朔,以久雪廢元會。戊戌,振武李泳招收得黑山外契苾部落四百七十三帳。壬子,詔:「朕聞『天聽自我人聽天視自我人視。』朕之菲德,涉道未明,不能調序四時,導迎和氣。自去冬已來,逾月雨雪,寒風尤甚,頗傷於和。念茲庶氓,或罹凍餒,無所假貸,莫能自存。中宵載懷,旰食興嘆,怵惕若厲,時予之辜。思弘惠澤,以順時令。天下死罪囚,除官典犯贓、故意殺人外,並降從流,流已下遞降一等。應京畿諸縣,宜令以常平義他倉斗
 賑恤。京城內鰥寡癃殘無告不能自存者,委京兆尹量事濟恤,具數以聞。言念赤子。視之如傷。天或警予,示此陰沴。扶躬夕惕,予甚悼焉。」群臣拜表上徽號。甲寅,司徒致仕薛平卒。



 二月甲子朔,以前義昌軍節度使殷侑檢校吏部尚書,充天平軍節度、鄆曹濮等州觀察使,代令狐楚;以楚檢校右僕射,兼太原尹、北都留守、河東節度使。戊寅,蘇、湖二州水,賑米二十二萬石。以本州常平義倉斛斗給。庚辰,戶部尚書、判度支王起請於邠寧、靈武
 置營田務,從之。己丑,寒食節,上宴群臣於麟德殿。是日,雜戲人弄孔子,帝曰:「孔子,古今之師,安得侮瀆。」亟命驅出。



 三月甲午朔。辛丑,以武寧軍節度使、守太傅、同平章事王智興兼侍中,充忠武軍節度、陳許蔡觀察等使。以邠寧節度使李聽為武寧軍節度、徐泗濠觀察等使;以金吾衛大將軍孟友亮為邠寧節度使。以前河東節度使柳公綽為兵部尚書。辛酉,以前忠武軍節度使高瑀檢校右僕射,充武寧軍節度、徐泗濠觀察等使。



 夏四月
 癸亥朔。乙丑,兵部尚書柳公綽卒。戊寅,以新除武寧軍節度使李聽為太子太保。



 五月癸巳朔。甲辰,西川修邛崍關城,又移雋州於臺登城。壬子,浙西丁公著奏杭州八縣災疫,賑米七萬石。丁巳,以鹽州刺史王晏平檢校左散騎常侍、御史大夫,充靈鹽節度使。己未,興平縣人上官興因醉殺人而亡竄,官捕其父囚之,興歸,待罪有司。京兆尹杜忭、中丞宇文鼎以興自首免父之囚,其孝可獎,請免死。詔兩省參議,皆言殺人者死,古今共守,興
 不可免。上竟從忭等議免死,決杖八十,配流靈州。庚申,詔:「如聞諸道水旱害人,疾疫相繼,宵旰罪己,興寢疚懷。今長吏奏申,札瘥猶甚。蓋教化未感於蒸人,精誠未格於天地,法令或爽,官吏為非。有一於茲,皆傷和氣。並委中外臣僚,一一具所見聞奏,朕當親覽,無憚直言。其遭災疫之家,一門盡歿者,官給兇器。其餘據其人口遭疫多少,與減稅錢。疫疾未定處,官給醫藥。諸道既有賑賜,國費復慮不充,其供御所須及諸公用,量宜節減,以救
 兇荒。」六月壬戌朔。丙寅,京兆尹杜忭兼御史大夫。戊寅,右僕射王涯奉敕,準令式條疏士庶衣服、車馬、第舍之制度。敕下後。浮議沸騰。杜忭於敕內條件易施行者寬其限,事竟不行,公議惜之。



 秋七月辛卯朔。甲午,以諫議大夫王彥威、戶部郎中楊漢公、祠部員外郎蘇滌、右補闕裴休並充史館修撰。故事,史官不過三員,或止兩員,今四人並命,論者非之。戊申,原王逵薨。癸丑,以前靈武節度使李文悅為兗、海、沂、密節度使。己未,以河中節度
 使李程為左僕射;以戶部尚書、判度支王起檢校吏部尚書,充河中晉、慈、隰節度使;以御史中丞、兼刑部侍郎宇文鼎為戶部侍郎、判度支。



 八月辛酉朔,吏部尚書崔群卒。以駕部郎中、知制誥李漢為御史中丞。乙丑,以尚書右丞、判太常卿王璠檢校禮部尚書、潤州刺史、浙西觀察使。庚午,山南東道節度使裴度來朝。壬申,以前浙西觀察使丁公著為太常卿。甲戌,御史中丞李漢奏論僕射上事儀,不合受四品已下官拜。時左僕射李程將
 赴省上故也。詔曰:「僕射上儀,近定所緣拜禮,皆約令文,已經施行,不合更改,宜準大和四年十一月十六日敕處分。」九月庚寅朔,淄青初定兩稅額,五州一十九萬三千九百八十九貫,自此淄青始有上供。庚子,以太傅趙宗儒守司空致仕。辛丑,涿州置新城縣,古督亢之地也。丁未,太常卿丁公著卒。庚戌,司空致仕趙宗儒卒。壬子,以右金吾衛將軍史孝章為鄜、坊、丹、延節度使。



 冬十月庚子朔。甲子,詔魯王永宜冊為皇太子。壬午,
 以左金吾衛將軍李昌言檢校左散騎常侍,充夏、綏、銀、宥節度使。甲申,以諫議大夫王彥威為河中少尹,以其論上官興獄太徼訐故也。



 十一月己丑朔。丁未,淮南節度使、檢校右僕射崔從卒。乙卯,以荊南節度使段文昌為劍南西川節度使。依前檢校左僕射、同平章事。十二月己未朔。乙丑,以中書侍郎、同平章事牛僧孺檢校右僕射、同平章事、揚州大都督府長史,充淮南節度使。戊辰,內養王宗禹渤海使回,言渤海置左右神策軍事、左右
 三軍一百二十司,畫圖以進。以尚書右丞崔琯為江陵尹、荊南都團練觀察使。珍王諴薨。乙亥,昭義節度使劉從諫來朝。丁未,以前西川節度使李德裕為兵部尚書。責授循州司馬杜元穎卒,贈湖州刺史。



 七年春正月乙丑朔,御含元殿受朝賀。比年以用兵、雨雪,不行元會之儀。故書,吳蜀貢新茶,皆於冬中作法為之,上務恭儉,不欲逆其物性,詔所供新茶,宜於立春後造。甲午,加劉從諫同平章事。襄州裴度奏請停臨漢監
 牧,從之。此監元和十四年置,馬三千二百匹,廢百姓田四百餘頃,停之為便。乙亥,以太府卿崔珙為廣州刺史、嶺南節度使。壬子,詔:「朕承上天之睠佑,荷列聖之丕圖,宵旰憂勞,不敢暇逸,思致康乂,八年於茲。而水旱流行,疫疾作沴,兆庶艱食,札瘥相仍。蓋德未動天,誠未感物,一類失所,其過在予。載懷罪己之心,深軫納隍之嘆。如聞關輔、河東,去年亢旱,秋稼不登,今春作之時,農務又切,若不賑救,懼至流亡。京兆府賑粟十萬石,河南府、河
 中府、絳州各賜七萬石,同、華、陜、虢、晉等州各賜十萬石,並以常平義倉物充。」以新除嶺南節度使崔珙檢校工部尚書,充武寧軍節度使;以右金吾衛將軍王茂元為嶺南節度使。丙辰,以前武寧軍節度使高瑀為刑部尚書。嶺南五管及黔中等道選補使,宜權停一二年。



 二月己未朔。己巳,以吏部侍郎庾承宣為太常卿。癸酉,以宗正卿李詵為陜州防禦使,代崔咸;以咸為右散騎常侍。己卯,麟德殿對吐蕃、渤海、牂柯、昆明等使。辛巳,御史臺
 奏:均王傅王堪男禎,國忌日於私第科決罰人。詔曰:「準令,國忌日禁飲酒、舉樂。決罰人吏,都無明文。起今後從有此類,不須舉奏。王禎宜釋放。」丙戌,詔以銀青光祿大夫、守兵部尚書、上柱國、贊皇縣開國伯、食邑七百戶李德裕以本官同中書門下平章事。



 三月戊子朔。庚寅,以前戶部侍郎楊嗣復為尚書左丞。壬辰,以左散騎常侍張仲方為太子賓客分司。仲方為郎中時,常駁故相李吉甫謚,德裕秉政,仲方請告,因授之。己亥,嶺南節度使
 李諒卒。辛丑,和王綺薨。復於埇橋置宿州,豁徐州符離縣蘄縣、泗州虹縣隸之,以東都鹽鐵院官吳季真為宿州刺史。癸卯,以京兆尹、駙馬都尉杜忭檢校禮部尚書,充鳳翔隴右節度。己酉,安南奏:蠻寇寇當管金龍州,當管生獠國、赤珠落國同出兵擊蠻,敗之。庚戌,出給事中楊虞卿為常州刺史,中書舍人張元夫汝州刺史。以太府卿韋長為京兆尹。丙辰,以散騎常侍嚴休復為河南尹。丁巳,以給事中蕭浣為鄭州刺史。



 夏四月戊午朔。辛酉,九
 姓回紇可汗卒。癸亥,前鳳翔節度使、檢校司空竇易直卒。癸酉,以同州刺史吳士智為江西觀察使。以吏部侍郎高釴為同州刺史。庚辰,以工部侍郎李固言為右丞,中書舍人楊汝士為工部侍郎。壬子,以河南尹白居易為太子賓客,分司東都。甲申,以江西觀察使裴誼為歙池觀察使,代沈傳師;以傳師為吏部侍郎。以右金吾衛將軍唐弘實使回紇,冊九姓回紇愛登里羅汨沒施合句錄毗伽彰信可汗。



 五月丁亥朔。丁酉,以李聽為鳳翔
 隴右節度使,依前檢校司徒、兼太子太保。癸卯,興元李載義來朝。癸丑,以前邛州刺史劉旻為安南都護。



 六月丁巳朔。乙巳,以山南西道節度使李載義為太原尹、北都留守、河東節度使,依前守太保、同平章事。壬申,以御史中丞李漢為禮部侍郎,以工部尚書、翰林侍講學士鄭覃為御史大夫。甲戌,以刑部尚書高瑀為太子少保分司。乙亥,以中書侍郎、平章事李宗閔檢校禮部尚書、同平章事,兼興元尹、山南西道節度使。丁丑,以左金吾
 衛將軍李從易為桂管觀察使。己卯,以右神策大將軍李用為邠寧節度使。河陽修防口堰,役工四萬,溉濟源、河內、溫縣、武德、武陟五縣田五千餘頃。癸未,涇原節度使張惟清卒。乙酉,以前河東節度使令狐楚檢校右僕射,兼吏部尚書。



 秋七月丙戌朔。丁亥,以右龍武統軍康志睦為四鎮北庭行軍、涇原節度使。壬寅,以金紫光祿大夫、守尚書右僕射、諸道鹽鐵轉運使、上柱國、代郡公、食邑二千戶王涯可同中書門下平章事,領使如故。甲
 辰,右丞李固言等奏狀,論僕射省中上事,不合受四品已下拜。敕旨宜準大和四年十一月十六日敕處分。乙巳,虢州刺史崔玄亮卒。以左丞楊嗣復檢校禮部尚書,充劍南東川節度使;以戶部侍郎庾敬休為左丞。己酉,以旱,命京城諸司疏決系囚。壬子,敕應任外官帶一品正京官者,縱不知政事,其俸料宜兼給。癸丑,以左僕射李程檢校司空,兼汴州刺史、宣武軍節度使。甲寅,以旱徙市。左降官開州司馬宋申錫卒,詔許歸葬。閏七月乙
 卯朔,詔曰:「朕嗣守丕圖,覆嫗生類,競業寅畏,上承天休。而陰陽失和,膏澤愆候,害我稼穡,災於黔黎。有過在予,敢忘咎責。從今避正殿,減供膳,停教坊樂,廄馬量減芻粟,百司廚饌亦宜權減。陰陽鬱堙,有傷和氣,宜出宮女千人。五坊鷹犬量須減放。內外修造事非急務者,並停。」時久無雨,上心憂勞。詔下數日,雨澤霑洽,人心大悅。乙丑,以前宣武軍節度楊元卿為太子太保。戊戌,以給事中崔戎為華州刺史。癸未,以太子賓客李紳檢校左散
 騎常侍兼越州刺史,充浙東觀察使,代陸亙;以亙為宣歙觀察使。



 八月甲申朔,御宣政殿,冊皇太子永。是日降詔:「應犯死從流,流已下遞減一等。諸王自今後相次出閣,授緊望已上州刺史佐。其十六宅諸縣主,委吏部於選人中簡擇配匹,具以名聞。皇太子方從師傅傳授《六經》,一二年後,當令齒胄國庠,以興墜典。宜令國子選名儒,置五經博士各一人。其公卿士族子弟,明年已後,不先入國學習業,不在應明經進士限。其進士舉宜
 先試帖經,並略問大義,取經義精通者放及第。卿大夫者,下人之所視,遠方之所仿,若非恭儉克己,廉直任人,而望其服從,固不可得。況朕不寶珠玉,不御纖華,逮於六宮,皆務儉薄。卿大夫得不葉朕此志,率先兆人?比年所頒制度,皆約國家令式,去其甚者,稍謂得中。而士大夫茍自便身,安於習俗,因循未革,以至於今。百官士族,起今年十月,其衣服輿馬,並宜準大和六年十月七日敕,如有固違,重加黜責。文武常參官及諸州府長官子
 為父後者,賜勛兩轉。」癸巳,太子太保楊元卿卒。戊申,以京兆尹韋長兼御史大夫,以刑部尚書高瑀為忠武軍節度使。



 九月甲寅朔。丙寅,侍御史李款閣內奏彈前邠州行軍司馬鄭注,曰:「注內通敕使,外連朝官,兩地往來,卜射財貨,晝伏夜動,干竊化權。人不敢言,道路以目,請付法司推劾情款。」旬日之中,諫章數十上,由是授注通王府司馬、兼侍御史,充神策軍判官,中外駭嘆。甲寅,以前忠武軍節度使王智興依前守太傅、兼侍中、河中尹、
 河中晉絳慈顯節度使,代王起;以起為兵部尚書。



 冬十月癸未朔,揚州江都等七縣水,害稼。壬辰,上降誕日,僧徒、道士講論於麟德殿。翌日,御延英,上謂宰臣曰:「降誕日設齋,起自近代。朕緣相承已久,未可便革,雖置齋會,唯對王源中等暫入殿,至僧道講論,都不監聽。」宰相路隨等奏:「誕日齋會,誠資景福,本非中國教法。臣伏見開元十七年張說、乾源曜以誕日為千秋節,內外宴樂,以慶昌期,頗為得禮。」上深然之,宰臣因請十月十日為
 慶成節上誕日也。從之。辛酉,潤、常、蘇、湖四州水,害稼。



 十一月癸丑朔。乙亥,涇源節度使康志睦卒。己卯,以左神策長武城使硃叔夜為涇州刺史,充涇原節度使。壬午,於銀州置監牧。十二月癸未朔。己亥,刑部詳定大理丞謝登新編《格後敕》六十卷,令刪落詳定為五十卷。庚子,幸望春宮,聖體不康。癸卯,平盧軍節度、檢校司空王承元卒。丁未,以河南尹嚴休復檢校禮部尚書,充平盧軍節度、淄青登萊棣觀察等使。戊甲,以給事中王質權知
 河南尹。以河東節度副使李石為給事中。



 八年春正月癸丑朔。丁巳,聖體痊平,御太和殿見內臣。甲子,御紫宸殿見群臣。丙寅,修太廟。令太常卿庾承宣攝太尉,遍告九室,遷神主於便殿。癸酉,揚、楚、舒、廬、壽、滁、和七州去年水,損田四萬餘頃。



 二月壬午朔,日有蝕之。庚寅,詔以聖躬痊復,赦系囚,放逋賦,移流人。己亥,蔚州飛狐鎮置鑄錢院。



 三月壬子朔。甲寅,上巳,賜群臣宴於曲江亭。庚午,以山南東道節度使裴度充東都留守,依
 前守司徒、兼侍中;以東都留守李逢吉檢校司徒、兼右僕射。癸酉,兗海節度使李文悅卒。丙子,以右丞李固言為華州刺史,代崔戎;以戎為兗海觀察使。四月壬午朔。壬辰,集賢學士裴濆撰《通選》三十卷,以擬昭明太子《文選》,濆所取偏僻,不為時論新稱。甲午,以縮州刺史吳李真為邕管經略使,乙已,乾林學士,兵部侍郎王源中辭內職乃以源中為禮部尚書。



 五月辛亥朔。己巳,修奉太廟畢,以吏部尚書令狐楚攝太尉,遍告神主,復正殿。飛
 龍神駒中廄火。



 六月庚辰朔。辛巳,徙市。壬午,大理卿劉遵古卒。壬辰,陳許節度使高瑀卒。甲午,以旱,詔諸司疏決系囚。丙申,以前鳳翔節度使、駙馬都尉杜忭起復檢校戶部尚書,充忠武軍節度使。戊戌,襯臣王涯、路隨奏請依舊制讀時令。庚子,充海觀察使崔戎卒。辛丑,同州刺史高釴卒。戊申,以將作監、駙馬都尉崔杞為充海沂密觀察使。



 秋七月庚戌朔。丙辰,以工部侍郎楊汝士為同州刺史。戊午,奉先、美原、櫟陽等雨,損夏麥。辛酉,定
 陵臺大雨,震東廊郎下地裂一百三十尺,詔宗正卿李仍叔啟告修塞。癸亥,郯王經薨。己巳夜,月犯昴。壬申,以右金吾衛大將軍段百倫檢校工部尚書,充福建觀察使。堂帖中外臣僚,各舉善《周易》學者。



 八月己卯朔,右龍武統軍董重質卒。庚寅,太白犯熒惑。辛卯,詔故澧王大男漢可封東陽郡王,第二男源可封安陸郡王,第三男演可封臨安郡王;故深王大男潭可封河內郡王,第二男淑可封吳興郡王;故絳王大男洙可封新安郡王,第
 二男滂可封高平郡王;故洋王大男沛可封潁川郡王;淄王大男浣可封許昌郡王;沔王大男瀛可封封晉陵郡王;鄜王大男溥可封平陽郡王:仍並賜光祿大夫。丙申,罷諸色選舉,歲旱故也。己亥,御寫《周易》義五道示群臣,有人明此義者,三日內聞奏。時李仲言以《易》道惑上,及下其義,人皆竊笑,卒無進言者。



 九月乙酉朔。辛亥夜,彗起太微,近郎位,西指,長丈餘,西北行,凡九夜,越郎位西北五尺滅。癸丑,月入南斗。乙亥,宣州觀察使陸亙卒。己
 未,宰臣李德裕進《御臣要略》及《柳氏舊聞》三卷。隨州刺史杜師仁前刺吉州,坐贓計絹三萬匹,賜死於家。故江西觀察使裴誼乖於廉察,削所贈工部尚書。庚申,右軍中尉王守澄宣召鄭注,對於浴堂門,仍賜錦彩銀器。是夜,彗出東方,長三尺,輝耀甚偉。辛酉,以權知河南尹王質為宣歙觀察使。吏部尚書致仕張正甫卒。癸亥,以尚書吏部侍郎鄭浣為河南尹。甲子,鄭注進《藥方》一卷。庚午,安王溶、潁王瀍皆檢校兵部尚書。宰相路隨冊拜太
 子太師。辛巳,幽州節度使楊志誠、監軍李懷仵悉為三軍所逐,立其部將史元忠為留後。陜州、江西旱、無稼。己丑,秘書監崔威卒。庚寅,以山南西道節度使、檢校禮部尚書、同平章事、上柱國、襄武縣開國侯、食邑一千戶李宗閔可中書侍郎、同中書門下平章事。辛卯,以中使田全操充皇太子見太師禮儀使。壬辰,召國子四門助教李仲言對於思政殿,賜緋。河南府、鄧州、同州、揚州並奏旱蟲傷損秋稼。甲午,以銀青光祿大夫、守中書侍郎、平
 章事李德裕檢校兵部尚書、同平章事、興元尹,充山南西道節度使。以助教李仲言為國子《周易》博士,充翰林侍講學士。皇太子見太師路隨於明門。丙申,諫官上疏論李仲言不合獎任,上令中使宣逾諫官曰:「朕留仲言禁中,顧問經義,敕命已行,不可遽改。」淮南、兩浙、黔中水為災,民戶流亡,京師物價暴貴。庚子,詔鄭注對於太和殿。以御史大夫鄭覃為戶部尚書。壬寅,翰林院宴李仲言,賜《法曲》弟子二十人奏樂以寵之。丙午,以新除興
 元節度使李德裕為兵部尚書。



 十一月丁未朔。庚戌,以尚書左僕射致仕蕭府為太子太傅。辛亥,以左金吾衛大將軍蕭洪為河陽三城節度使。襄州水,損田。壬子,滁州奏清流等三縣四月雨至六月,諸山發洪水,漂溺戶萬三千八百。癸丑,以禮部尚書王源中檢校戶部尚書,充山南西道節度使;以戶部侍郎李漢為華州刺史、鎮國軍潼關防禦使。成德軍節度使王廷湊卒。以前河陽節度使溫造為御史大夫。己卯,幽州節度使楊志誠被
 逐入朝,下御史臺訊鞫。志誠在幽州,被服皆為龍鳳,乃流之嶺外,至商州殺之。乙亥,以兵部尚書李德裕檢校右僕射,充鎮海軍節度、浙江西道觀察等使。丙子,李仲言奏請改名訓,從之。十二月丁丑朔。己卯,以昭義節度副使、檢校庫部員外郎、賜紫金魚袋鄭注為太僕卿。辛巳,以棣州刺史韓威為安南都護。癸未,以通王為幽州盧龍節度使,以權勾當幽州兵馬史元忠為留後。甲申,許太子太傅蕭瑀致仕。是夜,月掩昴。己丑,以太子賓客
 分司張仲方為左散騎常侍,常州刺史楊虞卿為工部侍郎。己亥,以尚書左僕射李逢吉守司徒致仕。以宗正卿李仍叔為湖南觀察使,代李翱;以翱為刑部侍郎,代裴濆;以濆為華州鎮國軍潼關防禦使。昭成寺火。



 九年春正月丁未朔。乙卯,以鎮州左司馬王元逵起復定遠將軍、守左金吾衛大將軍、檢校工部尚書,充成德軍節度使、鎮冀深趙觀察等使。以太常卿庾成宣檢校吏部尚書,充天平軍節度使,代殷侑;以侑為刑部尚書。
 癸亥,巢縣公湊薨,追封齊王。壬申,司徒致仕李逢吉卒。癸酉,以右散騎常侍舒元輿為陜州防禦觀察使。以前棣州刺史田早為安南都護。



 二月丙子朔。甲申,以司農卿王彥威兼御史大夫,充平盧軍節度使。丁亥,發神策軍一千五百人修淘曲江。如諸司有力,要於曲江置亭館者,宜給與閑地。辛丑,冀王絿薨。癸卯,京師地震。甲辰,以幽州留後史元忠為盧龍節度使。乙巳,劍南西川節度使、檢校左僕射、同平章事段文昌卒。庚申,以劍南東
 川節度使楊嗣復檢校戶部尚書,兼成都尹、西川節度使。乙丑,以歲饑,河北尤甚,賜魏博六州粟五萬石,陳許、鄆、曹濮三鎮各賜糙米二萬石。庚午,左丞庾敬休卒,廢朝一日。詔曰:「官至丞、郎,朕所親委,不幸雲亡者,宜為之廢朝。自今丞、郎宜準諸司三品官例,罷朝一日。」



 夏四月丙子朔。丙戌,以桂管觀察使李從易為廣州刺史、嶺南節度使。以鎮海軍節度使、浙西觀察等使李德裕為太子賓客,分司東都。辛卯,以京兆尹賈食束為浙西觀察使;
 以工部侍郎楊虞卿為京兆尹,仍賜金紫。以給事中韓佽為桂管觀察使。丙申,以太子太師、門下侍郎、平章事路隨為鎮海軍節度、浙西觀察等使。戊戌,詔以新浙西觀察使賈餗為中書侍郎、同中書門下平章事。庚子,詔銀青光祿大夫、守太子賓客分司東都、上柱國、贊皇縣開國伯、食邑七百戶李德裕貶袁州長史。辛丑,大風,含元殿四鴟吻並皆落。壞金吾仗舍。廢樓觀城四十餘所。壬寅,吏部侍郎沈傳師卒。



 五月乙巳朔。丁未,以浙東觀
 察使李紳為太子賓客,分司東都。乙卯,以給事中高銖為浙東觀察使。戊午,以御史大夫溫造為禮部尚書,以吏部侍郎李固言為御史大夫。辛酉,太和公主進馬射女子七人、沙陀小兒二人。戊辰,以金吾大將軍李玼為黔中觀察使,以尚書右丞王璠為戶部尚書、判度支。己巳,以戶部尚書鄭覃為秘書監。辛未,宰相王涯冊拜司空。癸酉,以河中節度使王智興為宣武軍節度使,依前守太傅、兼侍中。



 六月乙亥朔,西市火。以前宣武軍節度
 使李程為河中節度使。庚寅夜,月掩歲。癸巳,以吏部尚書令狐楚為太常卿。丁酉,禮部尚書溫造卒。京兆尹楊虞卿家人出妖言,下御史臺。虞卿弟司封郎中漢公並男知進等八人撾登聞鼓稱冤,敕虞卿歸私第。己亥,以右神策大將軍劉沔為涇原節度使。壬辰,詔以銀青光祿大夫、守中書侍郎、同平章事、襄武縣開國侯、食邑一千戶李宗閔貶明州刺史,時楊虞卿坐妖言人歸第,人皆以為冤誣,宗閔於上前極言論列,上怒,面數宗閔之
 罪,叱出之,故坐貶。



 秋七月甲申朔,貶京兆尹楊虞卿為虔州司馬同正。丙午,以給事中李石權知京兆尹。戊申,填龍首池為鞠場,曲江修紫雲樓。辛亥,詔以御史大夫李固言為門下侍郎、同平章事。壬子,再貶李宗閔為處州長史。癸丑,以右司郎中、兼侍御史、知雜事舒元輿為御史中丞。貶吏部侍郎李漢為汾州刺史,刑部侍郎蕭浣為遂州刺史。丁巳,詔不得度人為僧尼。戊午,貶工部侍郎、充皇太子侍讀崔侑為洋州刺史,貶吏部郎中張
 諷夔州刺史,考功郎中、皇太子侍讀蘇滌忠州刺史,戶部郎中楊敬之連州刺史。辛酉,以鄂岳觀察使崔郾充浙西觀察使,以國子祭酒高重為鄂岳觀察使。壬戌,鎮海軍節度使路隨卒。癸亥,貶侍御史李甘為封州司馬,殿中侍御史蘇特為潘州司戶。甲子,以《周易》博士李訓為兵部郎中、知制誥,依前充翰林侍講學士。丁卯,天平軍節度使庾承宣卒。以大理卿羅讓為散騎常侍,以汝州刺史郭行餘為大理卿。戊辰,以刑部尚書殷侑為天
 平軍節度使,以吉州刺史裴泰為邕管經略使。



 八月甲戌朔,以戶部侍郎李翱檢校禮部尚書,充山南東道節度使,代王起;以起為兵部尚書,判戶部事。丙子,又貶處州長史李宗閔為潮州司戶。丁丑,以太僕卿鄭注為工部尚書,充翰林侍講學士。上幸左軍龍首殿,因幸梨園,含元殿大合樂。戊寅,以秘書監鄭覃為刑部尚書。貶翰林學士、守尚書戶部侍郎、知制誥李玨為江州刺史,以鄜坊節度使史孝章為義成軍節度使。甲申,以左神策軍
 大將軍趙儋為鄜坊節度使。甲午,貶中書舍人權璩為鄭州刺史。丙申,內官楊承和於驩州安置,韋元素象州安置,王踐言思州安置,仰錮身遞送。言李宗閔為吏部侍郎時,托駙馬沈於宮人宋若憲處求宰相,承和、踐言、元素居中導達故也。宗閔黨楊虞卿、李漢、蕭浣皆再貶。壬寅,貶中書舍人高元裕為閬州刺史。元裕為鄭注除官制,說注醫藥之功,注銜之故也。以蘇州刺史盧周仁為湖南觀察使。



 九月癸卯朔,奸臣李訓、鄭注用事,不
 附己者,即時貶黜,朝廷悚震,人不自安。是日,下詔曰:「朕承天之序,燭理未明,勞虛襟以求賢,勵寬德以容眾。頃者臺輔乖弼諧之道,而具僚扇朋此之風,翕然相從,實斁彞憲。致使薰蕕共器,賢不肖並馳,退跡者咸後時之夫,登門者有迎吠之客。繆盭之氣,堙鬱未平,而望陰陽順時,疵癘不作,朝廷清肅,班列和安,自古及今,未嘗有也。今既再申朝典,一變澆風,掃清朋附之徒,匡飭貞廉之俗,凡百卿士,惟新令猷。如聞周行之中,尚蓄疑懼,或
 有妄相指目,令不自安,今茲曠然,明喻朕意。應與宗閔、德裕或新或故及門生舊吏等,除今日已前放黜之外,一切不問。」辛亥,以太子賓客分司東都白居易為同州刺史,代楊汝士;以汝士為駕部侍郎。乙亥,以涇原節度使劉沔為振武麟勝節度使。丙辰,以權知御史中丞舒元輿為御史中丞,兼判刑部侍郎。庚申,以鳳翔節度使李聽為忠武軍節度使。癸亥,令內養齊抱真將杖於青泥驛決殺前襄州監軍陳弘志,以有殺逆之罪也。丁卯,
 以門下侍郎、同平章事李固言為興元尹、山南西道節度使;以翰林侍講學士、工部尚書鄭注檢校右僕射,充鳳翔隴右節度使。戊辰,以右軍中尉王守澄為左右神策觀軍容使,兼十二衛統軍。己巳,詔以朝議郎、守御史中丞、兼刑部侍郎、賜紫金魚袋舒元輿本官同中書門下平章事。朝議郎、守兵部郎中、知制誥、充翰林侍講學士、賜緋魚袋李訓可守尚書禮部侍郎、同中書門下平章事,仍賜金紫。壬申,以刑部郎中、兼侍御史、知雜李孝
 本權知御史中丞。



 冬十月癸酉。乙亥,杜忭復為陳許節度使,李聽為太子太保分司。內出曲江新造紫雲樓彩霞亭額,左軍中尉仇士良以百戲於銀臺門迎之。時鄭注言秦中有災,宜興土功厭之,乃浚昆明、曲江二池。上好為詩,每誦杜甫《曲江行》云:「江頭宮殿鎖千門,細柳新蒲為誰綠?」乃知天寶已前,曲江四岸皆有行宮臺殿、百司廨署,思復升平故事,故為樓殿以壯之。王涯獻榷茶之利,乃以涯為榷茶使。茶之有榷稅,自涯始也。京兆、
 河南兩畿旱。以吏部尚書令狐楚為左僕射,以刑部尚書鄭覃為右僕射。辛巳,遣中使李好古齎CG賜王守澄,是日,守澄卒。壬午,賜群臣宴於曲江亭。癸未,以前廣州節度使王茂元為涇原節度使。丁亥,禮部郎中錢可復、兵部員外郎李敬彞、駕部員外郎盧簡能、主客員外郎蕭傑、左拾遺盧茂弘等皆授鳳翔使府判官,從鄭注奏請也。乙未,以新受同州刺史白居易為太子少傅分司,以汝州刺史劉禹錫為同州刺史。己亥,以前河陽節度
 使蕭洪為鄜坊節度使。淄青觀察使王彥威請停管內縣丞一十九員,從之。庚子,東都留守、特進、守司徒、侍中裴度進位中書令,餘如故。以前山南西道節度使王源中為刑部尚書。



 十一月壬寅朔。乙巳,令內養馮叔良殺前徐州監軍王守涓於中牟縣。以左神策軍胡沐為容管經略使,以大理卿郭行餘為邠寧節度使。丁未,鄜坊節度使趙儋卒。乙酉,左金吾衛大將軍崔鄯卒。癸丑,以左僕射令狐楚判太常卿事,右僕射鄭覃判國子祭
 酒事。丁巳,以戶部尚書、判度支王璠為太原尹、北都留守、河東節度使。戊午,以京兆尹李右為戶部侍郎、判度支,以京兆少尹羅立言權知府事。己未,以太府卿韓約為左金吾衛大將軍。壬戌,中尉仇士良率兵誅宰相王涯、賈餗、舒元輿、李訓,新除太原節度王璠,郭行餘、鄭注、羅立言、李孝本,韓約等十餘家,皆族誅。時李訓、鄭注謀誅內官,詐言金吾仗舍石榴樹有甘露,請上觀之。內官先至金吾仗,見幕下伏甲,遽扶帝輦入內,故訓等敗,流
 血塗地。京師大駭,旬日稍安。癸亥,詔以銀青光祿大夫、尚書左僕射、上柱國、滎陽郡開國公鄭覃以本官同中書門下平章事。乙丑,詔以朝議郎、守尚書戶部侍郎、判度支李石可朝議大夫、本官同平章事。丁卯,以左神刺大將軍陳君奕為鳳翔節度使。戊辰,以給事中李翊為御史中丞,左右軍尉仇士良、魚志弘並兼上將軍。十二月壬申朔,諸道鹽鐵轉運榷茶使令狐楚奏榷茶不便於民,請停,從之。癸丑。太子太保張茂宗卒。甲子,敕左
 右省起居齎筆硯及紙於螭頭下記言記事。丙子,以刑部尚書王源中為天平軍節度使。丁丑,敕諸道府不得私置歷日板。己卯,鳳翔監軍奏鄭注判官錢可復等四人並處斬訖。庚辰,上御紫宸,謂宰相曰:「坊市之間,人漸安未?」李石奏曰:「人情雖安,然刑殺過多,致此陰沴。又聞鄭注在鳳翔招致兵募不少,今皆被刑戮,臣恐乘此生事,切宜原赦以安之。」上曰:「然」鄭覃又陳理道。上曰:「我每思貞觀、開元之時,觀今日之事,往往憤氣填膺耳。」癸未,
 儀仗使田全操巡邊回,馳馬入金光門,街市訛言相驚,縱橫散走。賴金吾大將軍陳君賞以其徒立望仙門下,至晚方定。丁亥,以權知京兆尹張仲方為華州防禦使,以司農卿薛元賞權知京兆。左僕射令狐楚奏:「方鎮節度使等,具弩帓,帶器仗,就尚書省兵部參辭,伏乞停罷。如須參謝,令具公服。」從之。時楚引訓、注奸謀,用王璠、郭行餘兵仗,遂云不宜以兵仗入省參辭,殊乖事體也。物方尤之。先是,宰相武元衡被害,憲宗出內庫弓箭、陌刀
 賜左右街使,俟宰相入朝,以為翼從,及建福門退。至是亦停之。辛卯,置諫院印。



 開成元年正月辛丑朔,帝常服御宣政殿受賀,遂宣詔大赦天下,改元開成。乙巳,御紫宸殿,宰臣李石奏曰:「陛下改元御殿,人情大悅,全放京兆一年租賦,又停四節進奉,恩澤所該,實當要切。」帝曰:「朕務行其實,不欲崇長空文。」石曰:「赦書須內留一本,陛下時看之。又十道黜陟使發日,更付與公事根本,令向外與長吏詳擇施行,方
 盡利害之要。」丁未,以秘書監韋縝為工部尚書。敕:「楊承和、韋元素、王踐言、崔潭峻頃遭誣陷,每用追傷,宜復官爵,聽其歸葬。」以銀州刺史劉源為夏、綏、銀、宥節度使。丙辰望,日有蝕之。



 二月辛未朔,以左散騎常侍羅讓為江西觀察使。乙亥夜四更,京師地震,屋瓦皆墜。丙申,左武衛大將軍硃叔夜賜死於藍田關。天德奏生退渾部落三千帳來投豐州。



 三月庚子朔。壬寅以袁州長史李德裕為滁州刺史。庚申,幸龍首池,觀內人賽雨,因賦《暮春
 喜雨詩》。昭義節度使劉從諫三上疏,問王涯罪名,內官仇士良聞之惕懼。是日,從諫遣焦楚長入奏,於客狀誹謗,請面對。上召楚長慰諭遣之。



 夏四月庚午朔,以河南尹鄭浣為左丞,以太子賓客分司東都李紳為河南尹。癸酉,以亳州刺史裴弘泰為義成軍節度使,以諫議大夫李讓夷兼權知起居舍人事。乙卯,以潮州司戶李宗閔為衡州司馬,以江州刺史李玨為太子賓客分司。癸未,吏部侍郎李虞仲卒。辛卯,淄王協薨。甲午,詔以山南
 西道節度使、檢校兵部尚書李固言門下侍郎、同中書門下平章事;以左僕射、諸道鹽鐵轉運使令狐楚檢校左僕射,為山南西道節度使。丙申,李固言判戶部事;李石判度支,兼諸道鹽鐵轉運使。



 五月乙亥朔。癸卯,以翰林學士歸融為御史中丞。丁未,以給事中郭承嘏為華州防禦使。給事中盧載以承嘏公正守道,屢有封駁,不宜置之外郡,乃封還詔書。詡日,復以承嘏為給事中,乃以給事中盧鈞代嘏守華州。乙卯,御紫宸,上謂宰臣
 曰:「為政之道,自古所難。」李石對曰:「朝廷法令行,則易。」丁巳,以尚書右丞鄭肅為陜虢都防禦觀察使。前罷觀察,復置之。以中書舍人唐扶為福建觀察使。庚申,判國子祭酒宰臣鄭覃奏:「太學新置五經博士各一人,請依王府官例,賜以祿粟。」從之。丙寅,昭義奏開夷儀山路,通太原、晉州,從之。閏五月己巳朔。甲申,以河中節度使李程為左僕射、判太常卿事。乙酉,以太子太保分司李聽為河中節度使。丙戌,烏集唐安寺,逾月方散。己丑,以神策
 大將軍魏仲卿為朔方靈鹽節度。湖南觀察使盧周仁進羨餘錢二萬貫、雜物八萬段;不受,還之,使貸貧下戶徵稅。



 六月戊戌朔。癸亥,以河南尹李紳檢校禮部尚書、汴州刺史,充宣武軍節度使。



 秋七月戊辰朔,御史臺奏:「秘書省管新舊書五萬六千四百七十六卷,長慶二年已前並無文案。大和五年已後,並不納新書。今請創立簿籍,據闕添寫卷數,逐月申臺。」從之。辛未,以左金吾衛將軍傅毅為鄜坊節度使。癸酉,宣武軍節度使王智興
 卒。辛卯,刑部尚書殷侑檢校右僕射,充山南東道節度使。壬午,以滁州刺史李德裕為太子賓客。甲午,以金吾衛大將軍陳君賞為平盧軍節度使,代王彥威;以彥威為戶部侍郎、判度支。丙申,湖南觀察使盧周仁進羨餘錢一十萬貫,御史中丞歸融彈其違制進奉,詔以周仁所進錢於河陰院收貯。



 八月戊戌朔。甲辰,詐稱國舅人前鄜坊節度使蕭洪宜長流驩州。戊申,以皇太后親弟蕭本為右贊善大夫。



 九月丁卯朔。庚辰,詔復故左降開
 州司馬宋申錫正議大夫、尚書右丞、同平章事,仍以其子慎徽為城固尉。以饒州刺史馬植為安南都護。辛巳,以壽州刺史高承恭為邕管經略使。辛卯,敕秘書省,集賢院應欠書四萬五千二百六十一卷,配諸道繕寫。



 冬十月丁酉朔。己酉,揚州江都七縣水旱,損田。



 十一月丙寅朔。庚辰,浙西觀察使崔郾卒。以太子賓客分司東都李德裕檢校戶部尚書,充浙西觀察使。壬午,以兵部尚書、皇太子侍讀王起兼判太常卿。甲申,以左僕射李程
 兼吏部尚書。忠武帥杜忭、天平帥王源中奏:當道常平義倉斛斗,除元額外,請別置十萬石。十二月丙申朔,以京兆尹、兼御史大夫薛元賞為武寧節度、徐泗宿濠觀察等使,以戶部侍郎、兼御史中丞歸融為京兆尹,以給事中狄兼謨為御史中丞。己酉,嶺南節度使李從易卒。庚戌,以華州刺史盧鈞為廣州刺史,充嶺南節度使;以中書舍人崔龜從為華州防禦使。辛亥,劍南東川節度使馮宿卒。壬子,太僕卿段伯倫卒,癸丑,以兵部侍郎湯
 汝士檢校禮部尚書,充劍南東川節度使。己未,漵王縱薨。



 二年春正月乙丑朔。丙寅,宣州觀察使王質卒。乙亥,以吏部侍郎崔鄲為宣歙觀察使,以右丞鄭浣為刑部尚書、判左丞事。庚寅,戶部侍郎、判度支王彥威進所撰《供軍圖》,略序曰「至德、乾元之後,迄於貞元、元和之際,天下有觀察者十,節度二十有九,防御者四,經略者三。掎角之師,犬牙相制,大都通邑,無不有兵,約計中外兵額至
 八十八萬。長慶戶口凡三百三十五萬,而兵額又約九十九萬,通計三戶資奉一兵。今計天下租賦,一歲所入,總不過三千五百餘萬,而上供之數三之一焉。三分之中,二給衣賜,自留州留使兵士衣食之外,其餘四十萬眾,仰給度支焉。」二月乙未朔。丙申,刑部侍郎郭承嘏卒。丙午夜,彗出東方,長七尺,在危初,西指。戊申,王彥威進所撰《唐典》七十卷,起武德,終永貞。庚戌,均王緯薨。辛酉夜,彗長丈餘,直西行,稍南指,在虛九度半。壬戌夜,彗長
 二丈餘,廣三尺,在女九度,自是漸長闊。



 三月甲子朔,內出音聲女妓四十八人,令歸家。乙丑夜,彗星長五丈,歧分兩尾,其一指氐,其一掩房。丙寅,罷曲江宴。是夜,彗長六丈,尾無歧,在亢七度。敕尚食使,自今每一日御食料分為十日,停內修造。戊辰夜,彗長八丈有餘,西北行,東指,在張十四度。辛未,宣徽院《法曲》樂官放歸。壬申,詔曰:



 朕嗣丕構,對越上玄,虔恭寅畏,於今一紀。何嘗不宵衣念道,昃食思愆,師周文之小心,慕《易·乾》之夕惕,懼德不
 類,貽列聖羞。將欲致和平,時無殃咎,然誠未格物,謫見於天,仰愧三靈,俯慚庶匯,思獲攸濟,浩無津涯。昔宋景發言,星因退舍;魯僖納諫,饑不害人。取鑒往賢,深惟自勵。載軫在予之責,宜降恤辜之恩,式表殷憂,冀答昭誡。天下死罪降從流,流已下並放,唯故殺人、官典犯贓、主掌錢穀賊盜,不在此限。諸州遭水旱處,並蠲租稅。中外修造並停。五坊鷹隼悉解放。朕今素服避殿,徹樂減膳。近者內外臣僚,繼貢章表,欲加徽號。夫道大為帝,
 朕膺此稱,祗愧已多,矧鐘星變之時,敢議名揚之美?非懲既往,且儆將來,中外臣僚,更不得上表奏請。表已在路,並宜速還。在朝群臣,方岳長吏,宜各上封事,極言得失,弼違納誨,副我虛懷。



 甲戌,以左僕射李程為山南東道節度使。壬午,以楚州刺史嚴譽為桂管觀察使。甲申,以山南東道節度使殷侑為太子賓客分司。貞興門外鵲巢於古塚。丁亥,邠寧節度使李用卒。戊子,以河南尹李玨為戶部侍郎。乙丑,以金吾大將軍李直臣為邠寧
 節度使。壬辰,桂管觀察使韓佽卒。以兵部侍郎裴濆為河南尹。



 夏四月甲午朔。戊戌,詔將仕郎、守尚書工部侍郎、知制誥,充翰林學士,兼皇太子侍讀、上騎都尉、賜紫金魚袋陳夷行可本官同中書門下平章事。丙子,以中書舍人敬昕為江西觀察使戊申,前江西觀察使羅讓卒。辛酉,詔置終南山神祠。蓬州復置蓬池、朗池二縣。



 五月癸亥朔。乙丑,以東都留守裴
 度為太原尹、北都留守、河東節度使,依前守司徒、中書令。丙寅,戶部侍郎李玨判本司事。以浙西觀察使李德裕檢校戶部尚書,兼揚州大都督府長史,充淮南節度使。辛未,詔以前淮南節度使牛僧孺為檢校司空、東都留守,以蘇州刺史盧商為浙西觀察使。壬申,上幸十六宅,與諸王宴樂。決十六宅宮市內官範文喜等三人,以供諸王食物不精故也。



 六月癸巳朔。丁酉,以成德軍節度使王元逵為駙馬都尉,尚壽安公主。己亥,以鴻臚卿
 李逵為天德軍都防禦使。庚子,吏部奏長定選格,請加置南曹郎中一人,別置印一面,以「新置南曹之印」為文,從之。丙午,河陽軍亂,逐節度使李泳。戊申,以左金吾衛將軍李執方為河陽三城懷州節度使。庚戌,以右金吾衛大將軍崔珙為京兆尹。魏、博、澤、潞、淄、青、滄德、、兗、海、河南府等州並奏蝗害稼。鄆州奏蝗得雨自死。丁亥,以御史中丞狄兼謨為刑部侍郎,以前京兆尹歸融為秘書監,以給事中李翊為湖南觀察使。



 秋七月壬戌朔。乙亥,
 以久旱徙市,閉坊門。甲申,以太府卿張賈為兗海觀察使。詔除河北三鎮外,諸州府不得以試銜奏官。鄆州奏:「當州先廢天平、平陰兩縣,請復置平陰縣,以制盜賊。」從之。乙酉,以蝗旱,詔諸司疏決系囚。己丑,遣使下諸道巡覆蝗蟲。是日,京畿雨,群臣表賀。外州李紳奏蝗蟲入境,不食田苗,詔書褒美,仍刻石於相國寺。



 八月壬辰朔。丁酉,彗出虛、危之間。振武奏突厥入寇營田。庚戌,詔昭儀王氏冊為德妃,昭容楊氏冊為賢妃。又詔:「敬宗皇帝第
 二子休復、第三子執中、第四子言揚、第六子成美等,宜開列土之封,用申睦族之典。休復可封梁王,執中可封襄王,言揚可封紀王,成美可封陳王。皇第二男宗儉可封蔣王。」乙丑,房州刺史盧行簡坐贓杖殺。己巳,以前湖南觀察使盧行術為陜虢觀察使。甲申,詔曰:「慶成節朕之生辰,天下錫宴,庶同歡泰。不欲屠宰,用表好生,非是信尚空門,將希無妄之福。恐中外臣庶不諭朕懷,廣置齋筵,大集僧眾,非獨凋耗物力,兼恐致惑生靈。自今宴會
 蔬食,任陳脯醢,永為常例。」又敕:「慶成節宜令京兆尹準上已,重陽例,於曲江會斌百僚。延英奉觴宜權停。」戊子,以尚書戶部侍郎、判度支王彥威為衛尉卿,分司東都。



 冬十月辛卯朔,詔改天后所撰《三教珠英》為《海內珠英》。戊戌,詔嘉王運、循王遹、通王諶並可光祿大夫、檢校司空,賜勛上柱國,仍依百官例給料錢。安王溶、潁王瀍並給料錢。庚子,慶成節,賜群臣宴於曲江,上幸十六宅,與諸王宴樂。癸卯,宰臣判國子祭酒鄭覃進《石壁九經》
 一百六十卷。時上好文,鄭覃以經義啟導,稍折文章之士,遂奏置五經博士,依後漢蔡伯喈刊碑列於太學,創立《石壁九經》,諸儒校正訛謬。上又令翰林勒字官唐玄度復校字體,又乖師法,故石經立後數十年,名儒皆不窺之,以為蕪累甚矣。戊申,以門下侍郎、同平章事李固言為劍南西川節度使,依前同門下侍郎、平章事。甲寅,敕鹽鐵、戶部、度支三使下監院官,皆郎官、御史為之。使雖更改,院官不得移替,如顯有曠敗,即具事以聞。己未,
 以前西川節度使楊嗣復為戶部尚書,充諸道鹽鐵轉運使。



 十一月辛酉朔。壬戌,以太子賓客分司東都殷侑為忠武軍節度使。癸亥,狂病人劉德廣突入含元殿,付京兆府杖殺。乙丑,京師地震。丁丑,興元節度使令狐楚卒。丁亥,以刑部尚書鄭浣為山南西道節度使。己丑,契丹朝貢。十二月庚寅朔。丙申,內閣對左右史裴素等。上自開成初復故事,每入閣,左右史執筆立於螭頭之下,君臣論奏,得以備書,故開成政事最詳於近代。壬寅,以
 前忠武軍節度使杜悰為工部尚書、判度支。時忭既除官,久未謝恩,戶部侍郎李玨奏杜為岐陽公主服假內。玨因言:「比來駙馬為公主行服三年,所以士族之家不願為國戚者以此。」帝大駭其奏,即日詔曰:「制服輕重,必資典禮,如聞往者駙馬為公主行服三年,緣情之義,殊非故實,違經之制,今乃聞知。宜行期周,永為定制。」



 三年春正月庚申朔。甲子,宰臣李石遇盜於親仁里,中劍,斷其馬尾,又中流矢,不甚傷。是時,京城大恐,捕盜不
 獲,既而知仇士良新為。乙丑,常參官入朝者九人而已,餘皆潛竄,累日方安。丁卯,詔故齊王湊贈懷懿太子。戊申,以諸道鹽鐵轉運使、正議大夫、守戶部尚書、上柱國、宏農郡開國伯、食邑七百戶、賜紫金魚袋楊嗣復可本官同中書門下平章事,朝議郎、戶中侍郎、判戶部事、上柱國、賜紫金魚袋李玨可本官同中書門下平章事,依前判戶部事。丙子,以中書侍郎、同中書門下平章事李石為荊南節度使,依前中書侍郎、平章事。丁丑,以前荊
 南節度使韋長為河南尹。癸未,詔去秋蝗蟲害稼處放逋賦,仍以本處常平倉賑貸。是日大雪。



 二月己丑朔。乙未,上謂宰臣曰:「李宗閔在外數年,可別與一官。」鄭覃、陳夷行曰:「宗閔養成鄭注,幾覆朝廷,其奸邪甚於李林甫。」楊嗣復、李玨奏曰:「大和未,宗閔、德裕同時得罪,二年之間,德裕再量移為淮南節度使,而宗閔尚在貶所。凡事貴得中,不可但徇私情。」上曰:「與一郡可也。」丁酉,以衡州司馬李宗閔為杭州刺史。庚子,吏部奏:「去年所修長定
 選格,或乖往例,頗不便人,不可久行,請卻用舊格。」從之。乙巳,詔僕射、尚書、侍郎、左右丞、大卿監每遇坐日,宜令兩人循次進對。丁未,以同州刺史孫簡為陜虢觀察使,代盧行術;以術為福王傅,分司東都。乙酉,禮部尚書許康佐卒。辛亥,左丞盧載為同州防禦使。



 三月己未朔。庚午,封故陳王第十九男儼為宣城郡王,故襄王第三男寀為樂平郡王。



 夏四月戊子朔。己丑,禮部尚書致仕徐晦卒。辛卯,戶部侍郎崔龜從判本司事。詔曰:「戶部侍郎兩
 員,今後先授上者,宜令判本司錢穀;如帶平章事,判鹽鐵度支,兼中丞學士不在此限。」壬辰,以給事中裴袞為華州防禦使。乙酉,改《法曲》為《仙韶曲》,仍以伶官親處為仙韶院。兵部侍郎裴濆卒。癸丑,屯田郎中李衢、沔王府長史林贊等進所修《皇唐玉牒》一百五十卷。



 五月丁巳朔,敕禮部:貢院進士、舉人,歲限放三十人及第。辛酉,詔:前江西觀察使吳士規坐贓,長流端州。庚午,月犯天心大星。癸未,以吏部侍郎高鍇為鄂岳觀察使,代高重;以
 重為兵部侍郎。



 六月丁未朔。辛酉,出宮人四百八十,送兩街寺觀安置。廢晉州平陽院礬官,並歸州縣。癸丑,上御紫宸,對宰臣曰:「幣輕錢重如何?」楊嗣復曰:「此事已久,不可遽變其法,法變則擾人。但禁銅器,斯得其要。」秋七月丙辰朔。壬戌,陳許節度使殷侑卒。甲子,以衛尉卿王彥威檢校禮部尚書,充忠武軍節度使;以右金吾衛大將軍史孝章為邠寧節度使。戊辰,西川節度使李固言再上表,讓門下侍郎及檢校右僕射。



 八月丙戌朔。甲午,
 山南東道諸州大水,田稼漂盡。丁酉,詔:「大河而南,幅員千里,楚澤之北,連亙數州。以水潦暴至,堤防潰溢,既壞廬舍,復損田苗。言念黎元,羅此災沴,或生業蕩盡,農功索然,困餧雕殘,豈能自濟。宜令給事中盧弘宣往陳許、鄭滑、曹濮等道宣慰,刑部郎中崔瑨往山南東道、鄂岳、蘄黃道宣慰。」己亥,嘉王運薨。魏博六州蝗食秋苗並盡。



 九月丙辰朔。辛酉,荊南李石讓中書侍郎,乃改授檢校兵部尚書。壬戌,上以皇太子慢游敗度,欲廢之,中丞狄
 兼謨垂涕切諫。是夜,移太子於少陽院,殺太子宮人左右數十人。戊辰,詔梁王等五人,先於北內,可卻歸十六宅。辛未,易定節度使張璠卒。壬申,以易州刺史李仲遷為定州刺史,充義武軍節度使。戊寅,以東都留守牛僧孺為左僕射。辛巳,詔皇太子侍讀竇宗直隔日入少陽院。



 冬十月乙酉朔,以尚書左丞崔琯檢校戶部尚書,充東都留守。易定軍亂,不納新使李仲遷,立張璠子元益為留後。己丑,以少府監張沼為黔中觀察使。壬辰,以右
 金吾衛將軍高霞寓為夏、綏、銀、宥節度使。癸巳,以中書舍人李景讓為華州防禦使。甲午,慶成節,命中人以酒酺、《仙韶樂》賜群臣宴於曲江亭。丁酉,夏州節度使劉源卒。庚子,皇太子薨於少陽院,謚曰莊恪。乙巳,以左金吾將軍郭旼為邠、寧、慶節度使。是夜,彗起於軫,其長三丈,東西指。乙酉,前邠寧節度使史孝章卒。



 十一月乙卯朔,是夜,慧孛東西竟天。壬戌,詔曰:「上天蓋高,感應必由乎人事;寰宇雖廣,理亂盡系於君心。從古已來,必然之義。
 朕嗣膺寶位,十有三年,常克己以恭虔,每推誠於眾庶。將以導迎休應,漸致輯熙,期克荷於宗祧,思保寧於華夏。而德有所未至,信有所未孚。災氣上騰,天文謫見,再周期月,重擾星躔。當求衣之時,睹垂象之變,兢懼惕厲,若蹈泉穀。是用舉成湯之六事,念宋景之一言,詳求譴告之端,採聽銷禳之術。必有精理,蘊於眾情,冀屈法以安人,爰恤刑而原下。應京城諸道見系囚,自十二月八日已前,死罪降流,已下遞減一等,十惡大逆、殺人劫盜、
 官典犯贓不在此限。今年遭水蝗蟲處,並宜存撫賑給。」以滄州節度使李彥佐為鄆、曹、濮節度使,以德州刺史、滄景節度副使劉約為義昌軍節度使。癸亥,以宋州刺史唐弘實為邕管經略使。乙丑,天平軍節度使王源中卒。庚午,以翰林學士丁居晦為御史中丞。壬申,以蔡州刺史韓威為定州刺史、義武軍節度、北平軍等使。十二月乙酉朔。辛丑,詔以河東節度使、開府儀同三司、守司徒、兼中書令、太原尹、北都留守、上柱國、晉國公、食邑三
 千戶裴度可守司徒、中書令。以兵部侍郎狄兼謨為河東節度使。丙午,守太子太師、尚書右僕射、門下侍郎、國子祭酒、同平章事鄭覃罷太子太師,仍三五日入中書。日本國貢珍珠絹。



 四年春正月甲寅朔。丁巳,熒惑太白辰聚於南斗。丁卯夜,於威泰殿觀燈作樂,三宮太后諸公主等畢會。上性節儉,延安公主衣裾寬大,即時斥歸,駙馬竇浣待罪。詔曰:「公主入參,衣服逾制,從夫之義,過有所歸。浣宜奪兩月
 俸錢。」閏月甲申朔,以吏部侍郎鄭肅檢校禮部尚書、河中晉絳慈隰等州節度使,以蘇州刺史李道樞為浙東觀察使,以諫議大夫高元裕為御史中丞。丙申,以前河中節度使李聽為太子太保。己亥,裴度自太原至,上令中人就第問疾。辛丑,以司農卿李為福建觀察使,諫官論其不可,乃罷之。丙午,以大理卿盧貞為福建觀察使。丁未,興元節度使鄭浣卒。戊申,闍婆國朝貢。二月癸酉朔。辛酉,以吏部侍郎歸融檢校禮部尚書,充山南西
 道節度使。丙寅,寒食節,上禦通化門以觀游人。戊辰,幸勤政樓觀角抵、蹴鞠。



 三月癸未朔。乙酉,賜群臣上巳宴於曲江。是夜,月掩東井第三星。丙申,司徒、中書令裴度卒。癸酉,浙東觀察使李道樞卒。以戶部侍郎崔龜從為宣歙觀察使,代崔鄲;以鄲為太常卿。以楚州刺史蕭俶為浙東觀察使。



 夏四月壬子朔,以右羽林統軍李昌言為鄜坊節度使。壬戌,有麞出太廟。



 五月辛丑朔。丁亥,閣內上謂宰臣曰:「新修《開元政要》如何?」楊嗣復曰:「臣等未
 見。陛下欲以此書傳示子孫,則宣付臣等,參定可否。緣開元政事與貞觀不同,玄宗或好畋游,或好聲色,選賢任能,未得盡美。撰述示後,所貴作程,豈容易哉!」丙申,鄭覃、陳夷行罷知政事,覃守左僕射,夷行為吏部侍郎。丙午,邠寧節度使郭旼卒。天平、魏博、易定等管內蝗食秋稼。



 六月辛亥朔,以長武城使苻澈為邠寧節度。庚申,上幸十六宅安王、潁王院宴樂,賜與頗厚。戊辰,以久旱,分命祠禱,每憂動於色。宰相等奏曰:「水旱時數使然,乞
 不過勞聖慮。」上改容言曰:「朕為人主,無德及天下,致茲災旱,又謫見於天。若三日不雨,當退歸南內,更選賢明以主天下。」宰臣嗚咽流涕,各請策免。是夜,大雨霑霈。丁丑,襄陽山竹結實,其米可食。



 秋七月庚辰朔,西蜀水,害稼。乙未夜,月犯熒惑。壬寅,以河南尹韋長為平盧軍節度使,以刑部侍郎高鍇為河南尹。甲辰,以大中大夫、守太常卿、上柱國、賜紫金魚袋崔鄲可本官同中書門下平章事。滄景、淄青大水。



 八月庚戌朔,以給事中姚合為
 陜虢觀察使。辛亥,鄜王憬薨。丙辰,邢州廢青山縣,磁州移昭義縣於固鎮驛。癸亥,以左僕射牛僧孺檢校司空、同平章事,兼襄州刺史,充山南東道節度使。辛未夜,流星出羽林,尾長八十餘尺,滅後有聲如雷。壬申,鎮、冀四州蝗食稼,至於野草樹葉皆盡。



 九月己卯朔。辛卯,以劍南東川節度楊汝士為吏部侍郎。丁酉夜,月掩東井第三星。辛丑,以吏部侍郎陳夷行為華州鎮國軍防禦使,以蘇州刺史李穎為江西觀察使,以諫議大夫馮定為
 桂管觀察使。甲辰,以京兆尹鄭復為劍南東川節度使。丙午,以前江西觀察使敬昕為京兆尹。



 冬十月己酉朔。戊午,慶成節,賜群臣宴於曲江亭。辛酉夜,星入斗魁。前桂管觀察使嚴謇卒。丙寅,制以敬宗第六男陳王成美為皇太子。丁丑,太子太保李聽卒。



 十一月己卯朔。壬申,前福建觀察使唐扶卒。己亥,曲赦京城系囚。十二月己酉朔。癸丑,貶光祿卿、駙馬都尉韋讓為澧州長史。乙卯,乾陵火。以杭州刺史李宗閔為太子賓客,分司東都。辛
 酉,上不康,百僚赴延英起居。乙亥,宰臣入謁,見上於太和殿。是歲,戶部計見管戶四百九十九萬六千七百五十二。



 五年春正月戊寅朔,上不康,不受朝賀。己卯,詔立親弟潁王瀍為皇太弟,權勾當軍國事。皇太子成美復為陳王。辛巳,上崩於大明宮之太和殿,壽享三十三。群臣謚曰元聖昭獻皇帝,廟號文宗。其年八月十七日,葬於章陵。



 史臣曰:昭獻皇帝恭儉儒雅,出於自然,承父兄奢弊之餘,當閽寺撓權之際,而能以治易亂,代危為安。大和之初,可謂明矣。初,帝在籓時,喜讀《貞觀政要》,每見太宗孜孜政道,有意於茲。洎即位之後,每延英對宰臣,率漏下十一刻。故事,天子雙日視事,帝謂宰輔曰:「朕欲與卿等每日相見,其輟朝、放朝、用雙日可也。」時憲宗郭后居興慶宮,曰太皇太后,敬宗寶歷太后及上母蕭太后,時呼「三宮太后」。帝性仁孝,三宮問安,其情如一。嘗內園
 進櫻桃,所司啟曰:「別賜三宮太后。」帝曰:「太后宮送物,焉得為賜。」遽取筆改賜為奉。宗正寺以祭器杇敗,請易之,及有司呈進,命陳於別殿,具冠帶面閱之,容色淒然。尤勒於政理凡選內外群官,府進名,帝必面訊其行能,然後補除。中書用鴻臚卿張賈為衢州刺史,賈好博,朝辭日,帝謂之曰:「聞卿善長行。」對曰:「政事之餘,聊與賓客為戲,非有所妨。」帝曰:「豈有好之而無妨也!」內外聞之悚息。而帝以累世變起禁闈,尤側目於中官,欲盡除之。然訓、
 注狂狡之流,制御無術,矢謀既誤,幾致顛危。所謂「有帝王之道,而無帝王之才」,雖旰食焦憂,不能弭患,異哉!



 贊曰:昭獻統天,洪惟令德。心憤仇恥,志除兇慝。未殄夔魖,又生鬼蜮。天未好治,亂何由息。



\end{pinyinscope}