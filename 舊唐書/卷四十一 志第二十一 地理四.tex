\article{卷四十一 志第二十一 地理四}

\begin{pinyinscope}

 ○劍南道東
 西道九嶺南道五管十



 △劍南道



 成都府隋蜀郡。武德四年,改為益州,置總管府,管益、綿、陵、遂、資、雅、嘉、滬、戎、會、松、翼、巂、南寧、昆、恭十七州。益州領成都、雒、九隴、郫、雙流、新津、晉原、青城、陽安、金水、平泉、玄武、綿竹等十三縣。又置唐隆、導江二縣。二年,分置邛、眉、普、榮、登五州,屬總管府。又置新都、什邡二縣。三年,
 罷總管,置西南道行臺。仍分綿竹、導江、九龍三縣立濛州,陽安、金水、平泉三縣立簡州,割玄武屬梓州。又析置德陽、新繁、萬春三縣。九年,罷行臺,置都督府,督益、綿、簡、嘉、陵、眉、犍、邛十州,並督巂、南寧、會都督府。貞觀二年,廢濛州之九隴、綿竹、導江來屬,仍改萬春為溫江。六年。罷南寧都督,更置戎州都督,屬益州。八年,兼領南金州都督。十年,又督益、綿、簡、嘉、陵、雅、眉、邛八州,茂、巂二都督。十七年,置蜀縣。龍朔二年,升為大都督府,仍置廣都縣。咸亨二年,
 置金堂。儀鳳二年,又置唐昌、濛陽二縣。垂拱三年,分雒、九隴等十三縣置彭、蜀二州。其年,又置犀浦縣。聖歷三年,又置東陽縣。
 天寶元年,改益州為蜀郡,依舊大都督府,督劍南三十八郡。十五載,玄宗幸蜀,駐蹕成都。至德二年十月,駕回西京,改蜀郡為成都府,長史為尹。又分為劍南東川、西川,各置節度使。廣德元年,黃門侍郎嚴武為成都尹,復並東、西川為一節度。自崔寧鎮蜀後,分為西川,自後不改。舊
 領縣
 十六,戶
 十一萬七千
 八百八十九,口七十四萬三百一十二。漢朝蜀郡,戶二十六萬八千二百七十,口一百二十四萬。天寶領縣十,戶十六萬九百五十,口九十二萬八千一百九十九。在京師西南二千三百七十九里,至東都三千二百一十六里。



 成都漢縣,屬蜀郡。漢朝成都一縣,管戶一萬六千二百五十六。蜀,三代之時西南夷國,或臣或否。至秦惠王既霸西戎,欲廣其地,乃令其相張儀、司馬錯伐蜀。取其地,立漢中、巴、蜀三郡。蜀王本都廣都之樊鄉,張儀平蜀後,自赤里街移治於少城,今州城是也。蜀城,張
 儀所築



 華陽貞觀十七年,分成都縣置蜀縣,在州郭下,與成都分理。乾元元年二月,改為華陽。新都漢縣,屬廣漢郡



 新繁漢繁縣,屬蜀郡。劉禪時加「新」字



 犀浦垂拱二年,分成都縣置



 雙流漢廣都縣地,屬蜀郡。隋置雙流縣



 廣都龍朔三年,分雙流置,取隋舊名



 郫漢縣,屬蜀郡。隋置濛州,大業省為郫縣。溫江漢郫縣地,魏蜀郡治於此。隋為萬春縣。貞觀元年,改為溫江



 靈池久視元年,分蜀縣置東陽縣。天寶元年,改為靈池。



 漢州上垂拱二年,分益州五縣置漢州。天寶元年,改為德陽郡。乾元元年,復為漢州。領縣五,戶六萬九千五,口三十萬八千二百三。至京師二千二百里,至東都三千一百一十六里。



 雒漢縣,屬廣漢郡。後漢置益州,治於雒。晉置新都郡,宋、齊為廣漢郡。垂拱二年,置漢州。皆治雒縣也



 德陽後周廢縣。
 武德三年,分雒置



 什邡漢縣,屬廣漢郡。後周改為方寧。武德三年,改為什邡。雍齒侯邑,在縣北四十步。綿竹漢縣,屬廣漢郡。隋開皇二年,置晉熙縣。十八年,又改為孝水縣。大業三年,改為綿竹。武德三年,屬濛州。州廢,來屬之



 金堂咸亨二年,分雒縣、新都置,屬益州。垂拱二年,來屬。



 彭州上垂拱二年,分益州四縣置彭州,天寶元年,改為蒙陽郡。乾元元年,復為彭州。領縣四,戶五萬五千九百二十二,口三十五萬七千三百八十七。至京師二千三百三十九里,至東都三千一百六十九里。



 九隴州所治。漢繁縣地,宋置晉壽郡,古城在縣西北三里。梁置東益州。後魏為天水郡,仍改為九隴。初於縣東三里置濛州,大業省。武德三年,復置濛州,領九隴、綿竹、導江三縣,置彭州之名也。三縣置,屬益州。垂拱二年,屬彭州。長壽二年,改為周昌。神龍初復置



 濛陽儀鳳二年,分九隴、雒、什邡三縣置,屬益州。垂拱三年,來屬



 導江蜀置都安縣,後周改為汶山。武德元年,改為盤龍,尋改為導江。三年,割屬濛州。州廢,屬益州。舊治灌口城,武德元年,移治導江郡。垂拱二年,來屬。



 蜀州垂
 拱二年,分益州四縣置。天寶元年,改為唐安郡。乾元元年,復為蜀州也。領縣四,戶五萬六千五百七十七,口三十九萬六百九十四。至京師三千三百三十二里,至東都三千一百七十二
 里。



 晉原漢江源地,屬蜀州。李雄立江源郡,晉改為多融縣,又改為晉原。鶴鳴山,在西北十里



 青城漢江源縣地。南齊置齊基縣,後周改為青城。山在西北三十二里。舊「青」字
 加水,開元十八年,去「水」為「青」



 唐安本漢江源縣地,後魏於此立犍為郡及僰道縣。隋省。武德元年復置,改為唐隆。長壽二年,為武隆。先天元年,改為唐安



 新津漢武陽縣,屬犍為郡。後周改為新津,屬益州。垂拱二年,屬蜀州也。



 眉州上隋眉山郡之通義縣。武德二年,割嘉州之通義、丹棱、洪雅、青神、南安五縣置眉州。五年,省南安。貞觀二年,置隆山縣。天寶元年,改
 為通義郡。乾元元年,復為眉州也。舊領縣五,戶三萬六千九,口十六萬九千七百五十五。天寶,戶四萬三千五百二十九,口一十七萬五千二百五十六。至京師二千五百五十里,至東都三千二百八十九里。



 通義後漢置通義縣,屬齊通郡。梁改為青州,後魏改為眉州。後改通義為安洛,又復通義。隋初為廣通,尋
 改為通義。武德元年,於縣置唐眉州也



 彭山漢武陽縣地,屬犍為。晉於郡置西江
 陽郡。後魏增置隆山郡,以界內有鼎鼻山,地形隆故也。隋改為陵州隆山縣。先天元年,改為彭山也



 丹棱本南齊齊樂郡,後周改為洪雅縣。隋改為丹棱,屬嘉州。武德二年,來屬
 也



 洪雅後周洪雅鎮,隋改為縣。武德九年,置犍州。貞觀初,州廢,屬眉州也。



 青神漢南安縣,屬犍為郡。縣臨青衣江,西魏置青衣縣。本治思蒙水口,武德八年,移
 於今治,屬眉州也。



 綿州上隋金山郡。武德元年,改為綿州,領巴西、昌隆、涪城、魏城、金山、萬安、神泉七縣。三年,分置顯武、龍安、文義、鹽泉四縣。七年,省金山縣。貞觀元年,又省文義縣。舊領縣九,戶四萬三千九百四,口十九萬五千五百
 六十三。天
 寶領縣九,戶六萬五千六十六,口二十六萬三千三百五十二。至京師二千五百九里,至東都三千二百五十九里。



 巴西漢涪縣,屬廣漢郡。晉置梓潼郡,西魏置潼州。隋改為綿州,煬帝改為金山郡。隋改涪為巴西縣也。



 涪城漢涪縣地,東晉置始平郡。後魏改為涪城及潼縣。隋
 改潼為涪城。



 昌明漢涪縣地,晉置漢昌縣,後魏為昌隆。先天元年,改為昌明。舊有顯武縣,神龍元年,改為興聖。開元二年廢,並入昌明,仍分巴西、涪城、萬安三縣地置興聖縣。二十七年廢,地各還本屬。



 魏城隋置



 羅江漢涪縣地。晉於梓潼水尾萬安故城置萬安縣。後魏置萬安郡,隋廢。天寶元年,改萬安為羅江。廉泉、讓水,出縣北平地也。



 神泉漢涪縣地。晉置西充國縣,隋改為神泉,以縣西泉能愈疾故也。



 鹽泉武德三
 年,分魏城置也。



 龍安隋金山縣。武德三年,復置,改為龍安。



 西昌隋金山縣。隋末廢。永淳元年,復置,改為西昌也。



 劍州隋普安郡。武德元年,改為始州,領縣七。聖歷二年,置劍門縣。先天二年,改始州為劍州。天寶五年,改為普安郡。乾元元年,復為劍州也。舊領縣七,戶三萬六千七百一十四,口十九萬九十六。天寶領縣八,戶二萬三千五百一十,口一十萬四百五十。至京師一千六百六
 十二里,至東都二千五百六十里。



 普安漢梓潼縣,廣漢郡治也。宋置南安郡,梁置南梁州,又改為安州。西魏改為始州,兼置普安郡。武德元年,復為始州。皆治於普安也。



 黃安梁分梓潼縣置梁安縣,尋改為黃安。



 永歸隋分梓潼縣置。



 梓潼漢縣。蜀先分廣漢置梓潼,西魏改為潼川郡,隋為梓潼縣。後魏自涪縣移梓潼郡於今縣,屬始州,仍改郡為縣也。



 陰平晉流人入蜀,於縣置北陰平郡。山北有十八隴山,山
 有隴十八也。



 武連漢梓潼縣地。宋置武都郡及下辨縣,又改下辨為武功。後魏改為武連也。



 臨津漢梓潼縣地。南齊置相厚縣,隋改為臨津也。



 劍門聖歷二年,分普安、永歸、陰平三縣地,於方期驛城置劍門,縣界大劍山,即梁山也。其北三十里所,有小劍山。大劍山有劍閣道,三十里至劍處,張載刻銘之所。劍山東西二百三十一里。



 梓州上隋新城郡。武德元年,改為梓州,領郪、射洪、鹽
 亭、飛烏四縣。三年,又以益州玄武來屬。四年,又置永泰縣。調露元年,置銅山縣。天寶元年,改為梓潼郡。乾元元年,復為梓州。乾元後,分蜀為東、西川,梓州恆為東川節度使治所。舊領縣七,戶四萬五千九百二十九,口二十四萬八千三百九十四。天寶領縣八,戶六萬一千八百二十四,口二十四萬六千六百五十二。至京師二千九十里,至東都二千九百里。



 郪漢縣,屬廣漢郡,歷晉、宋、齊不改。梁於縣置新州,西
 魏改為昌城郡。隋改為梓州,煬帝改為新城郡。郡城左帶涪水,右挾中江,鄰居水陸之要。梓州所治,以梓潼水為名也。



 射洪漢郪縣地,後魏分置射洪縣。婁縷灘東六里,有射江,語訛為「洪」。



 通泉漢廣漢縣地,隋縣也。



 玄武漢底道縣,屬蜀郡。晉改為玄武。武德元年,屬益州。三年,割屬梓州也。



 鹽亭漢廣漢縣地,梁置鹽亭縣也。



 飛烏漢郪縣地,隋置飛烏鎮,又改為縣,取飛烏山為名也。



 永泰武德四年,分鹽亭、武安
 二縣置。



 銅山調露元年,分郪、飛烏二縣地置也。



 閬州隋巴西郡。武德元年,改為隆州,領閬中、南部、蒼溪、南充、相如、西水、三城、奉國、儀隴、大寅十縣。其年,又立新井、思恭二縣。四年,以南充、相如屬果州,儀隴、大寅屬蓬州。又置新政。七年,又以奉國屬西平州。還以奉國來屬。又省思恭入閬中縣。先天元年,改為閬州。天寶元年,改為閬中郡。乾元元年,復為閬州。舊領縣八,戶三萬八千九百四十九,口二十七萬三千五百四十三。
 今領縣九,戶二萬五千五百八十八,口十三萬二千一百九十二。至京師一千九百一十五里,至東都二千七百六十里。



 閬中漢縣,屬巴郡。梁置北巴州。西魏置隆州及盤龍郡。煬帝改為巴西郡。武德為隆州。皆治閬中。閬水迂曲經郡三面,故曰閬中,隋為閬內也。



 晉安漢閬中縣地。梁置金匱二。又為金遷郡。隋省郡,改為晉城。武德改為晉安也。



 南部後漢分閬中置充國縣,屬巴郡。又
 分置南充國郡。梁改為南充郡,隋改為南部也。



 蒼溪後漢分宕渠置漢昌縣,屬巴郡。隋改漢昌為蒼溪也。



 西水漢閬中縣地。梁置掌夫城,後周改為西水縣。



 奉國後漢分閬中置。武德七年,屬西平州。貞觀元年,還屬隆州。



 新井漢充國縣地。武德元年,分南部、晉安二縣置。界內有鹽井。



 新政武德四年,分南部、相如兩縣置。



 岐坪舊屬利州,開元二十三年來屬也。



 果州中隋巴西郡之南充縣。武德四年,割隆州之南充、相如二縣置果州,因果山為名。又置西充、郎池二縣。天寶元年,為南充郡。乾元元年,復為果州也。舊領縣四,戶一萬三千五百一十,口七萬五千八百一十一。天寶領縣六,戶三萬三千九百四,口八萬九千二百二十五。至京師二千五百五十八里,至東都三千四百二十三里。



 南充漢安漢縣,屬巴郡。宋於安漢故城置南宕渠郡。
 隋改安漢為南充。果山,在縣南八里。



 相如漢安漢縣地,梁置梓潼郡。周省郡,立相如縣,以縣城南二十里,有相如故宅二。相如坪,有琴臺。



 流溪開耀元年,析南充縣於溪水側置也。



 西充武德四年,分南充置。有西充山。



 郎池武德四年,分相如置。



 岳池萬歲通天二年,分南充、相如二縣置。初治思岳池,開元二十年,移治今所。



 遂州中隋遂寧郡。武德元年,改為遂州,領方義、長江、
 青石三縣。二年,置總管府,管遂、梓、資、普四州。貞觀罷總管。十年,復置都督,督遂、果、普、合四州。十七年,罷都督府。天寶元年,改為遂寧郡。乾元元年,復為遂州。舊領縣三,戶一萬二千九百七十七,口六萬六千四百六十九。天寶領縣五,戶三萬五千六百三十二,口十萬七千七百一十六。至京師二千三百二十九里,至東京三千一百六十六里。



 方義漢廣漢縣,屬廣漢郡。宋置遂寧郡,齊、梁加「東」字。後
 周改東遂寧為遂州。後魏改廣漢為方義。



 長江東晉巴興縣,魏改為長江。舊治靈鷲山,上元二年,移治白桃川也



 蓬溪永淳元年,分方義縣置唐興縣。長壽二年,改為武豐。神龍初復。景龍二年,分唐興置唐安縣。先天二年,廢唐安縣,移唐安廢縣置。天寶元年,改唐興為蓬溪也。



 青石東晉晉興縣。後魏改為始興。隋改始興為青石,以縣界有青石祠也。



 遂寧景龍元年分置。



 普州中隋資陽郡之安岳縣。武德二年,分資州之安岳、隆康、安居、普慈四縣置普州。三年,又置樂至、隆龕二縣。天寶元年,改為安岳郡。乾元元年,復為普州。舊領縣六,戶二萬五千八百四十,口六萬七千三百二十。天寶領縣四,戶二萬五千六百九十三,口七萬四千六百九十二。至京師二千三百六十里,至東都三千二百三里。



 安岳漢犍為、巴郡地,資中、牛鞞、墊江三縣地。李雄亂後,為獠所據。梁招撫之,置普慈郡。後周置普州,隋省。武
 德二年,復置,安岳為治所。



 安居後周柔剛縣,屬安居郡。隋改柔剛為安居。柔剛山,在縣東二十步。舊治柔剛山,天授二年,移理張柵也。



 普康後周永唐縣,隋改為永康,移治伏強城,尋改為隆康。先天元年,改為普康也。



 崇龕後周隆龕城,隋隆龕縣。舊治整瀨川,久視元年,移治波羅川。先天元年,為崇龕。隆龕山,在縣西三里也。



 陵州中隋隆山郡。武德元年,改為陵州,領仁壽、貴平、
 井研、始建、隆山五縣。貞觀元年,割隆山屬眉州。天寶元年,改為仁壽郡。乾元元年,復為陵州也。舊領縣四,戶一萬七千四百四十一,口八萬一百一十。天寶領縣五,戶三萬四千七百二十八,口一十萬一百二十八。至京師二千五百一十里,至東都三千四百八十四里。



 仁壽漢武陽縣東境,屬犍為郡。晉置西城戍,以為井防。後魏平蜀,改為普寧縣。後周置陵州,以州南陵井為名。隋改普寧為仁壽,所治也



 貴平漢廣都縣之東
 南地,屬蜀郡。後魏置和仁郡,仍立平井、貴平、可曇三縣。舊治和仁城,開元十四年,移治祿川也



 井研漢武陽縣地。東晉置西江陽郡。魏置蒲亭縣,隋改為井研。武德四年,自擁思茫水移治今所也



 始建漢武陽縣地。隋開皇十年,於此置始建鎮。大業五年,改鎮為始建縣。舊治擁思茫水,聖歷二年,移治榮祉山。籍梁席郡,一名漢陽戍。永徽四年,分貴平置。



 資州上隋資陽郡。武德元年,改為資州,領盤石、內江、安後、普慈、安居、隆康、資陽、大牢、威遠。其年,割大牢、威遠屬榮州。二年,分安居、
 隆康、普慈、安岳四縣屬普州。貞觀四年,置丹山縣。天寶元年,改為資陽郡。乾元元年,復為資州。乾元二年正月,分置昌州,尋廢也。舊領縣八,戶二萬九千三百四十七,口十五萬二千一百三十九。天寶,戶二萬九千六百三十五,口十萬四千七百七十五。至京師二千五百六十里,至東都三千五百一十里。



 盤石漢資中縣,屬犍為郡。後周於今簡州陽安縣移資州於漢資中故城為治所。仍改資中為盤石,今州治



 資陽後周分資中置縣,在資水之陽也



 牛鞞漢資中縣為盤地。隋分置牛鞞縣。漢有牛鞞縣,屬犍為郡,此非也。洛水,一名牛鞞水



 內江漢資中縣地,後漢於中江水濱置漢安戍。其年,改為中江縣,因其北江,乃雲中。隋改為內江。漢安故城,今縣治也



 月山資中地,義寧二年置



 龍水資中地,義寧二年置



 銀山資中地,義寧二年置



 丹山漢資中地,貞觀四年置。六年,並入內江。七年,又置。



 榮州中隋資陽郡之大牢縣。武德元年,置榮州,領大牢、威遠二縣。貞觀元年,置旭川、婆日、至如三縣。二年,割瀘州之隆越來屬。六年,自公井移州治大牢,仍割嘉州資官來屬。八年,又割瀘州之和義來屬。廢婆日、至如、隆越三縣。永徽二年,移州治旭川。天寶元年,改為和義郡。乾元元年,復為榮州。舊領縣六,戶一萬二千二百六十二,口五萬六千六百一十四。天寶,戶五千六百三十九,口一萬八千二十四。至京師二千九百七十二里,至東都
 二千七百四十九里。



 大牢漢南安縣,屬犍為郡。隋置大牢鎮,尋改為縣。武德元年,割資州之大牢、威遠二縣,於公井鎮置榮州,取界內榮德山為名。又改公井為縣。貞觀六年,自公井移州治於大牢縣也



 公井漢江陽縣,屬犍為郡。後周置公井鎮。武德元年,鎮置榮州,改為公井縣。貞觀六年,漢移於大牢也



 威遠漢安縣地,屬犍為郡。隋於舊威遠戍置縣。武德初,屬資州。其年,割屬榮州也



 旭川貞觀
 元年,分大牢縣置



 資官漢南安縣地,晉置資官縣。武德初,屬嘉州。貞觀六年,來屬



 和義漢安縣地,隋置和義縣。



 簡州隋蜀郡之陽安縣。武德三年,分益州置。天寶元年,改為陽安郡。乾元元年,復為簡州。舊領縣三,戶一萬三千八百五,口七萬五千一百三十三。天寶,戶二萬三千六十六,口十四萬三千一百九十。在京師西南二千七百里,至東都三千六百里。



 陽安漢牛鞞縣,屬犍為郡。後魏置陽安縣,又分陽安、平泉、資陽三縣置簡州,取界內賴簡池為名



 金水漢新都縣,屬廣漢郡。晉將硃齡石於東山立金泉戍。後魏立金泉郡,分置金泉、白牟二縣。隋改為金潤,屬蜀郡。武德初,為金水。三年,屬簡州。縣有金堂山



 平泉漢牛鞞縣地,後魏置婆潤縣。隋移縣治於賴黎池,仍改為平泉縣,縣之旁地湧泉故也。



 嘉州中隋眉山郡。武德元年,改為嘉州,領龍游、平羌、
 夾江、峨眉、玉津、綏山、通義、洪雅、丹巂、青神、南安五縣置眉州。貞觀六年,改資官,屬榮州。上元元年,以戎州之犍為來屬。天寶元年,改為犍為郡。乾元元年,復為嘉州。三月,劍南節度使盧元裕請升為中都督府。尋罷。舊領縣六,戶二萬五千八十五,口七萬五千三百九十一。天寶領縣八,戶三萬四千二百八十九,口九萬九千五百九十一。至京師二千七百二十里。至東都三千五百里。



 龍游漢南安縣地,屬犍為郡。後周置平羌縣。隋初,為
 峨眉縣,又改為青衣縣。隋伐陳時,龍見於江中引舟,乃改為龍游縣也,州臨大江為名



 平羌後周置也



 峨眉漢南安縣。隋置峨眉縣,取西山名也



 夾江漢南安縣地。隋分龍流、平羌三縣,於涇上置夾江縣。今北八十里,有夾江廢戍,即涇上地也。舊治涇上,武德元年,移於今治也



 玉津漢南安縣。隋置玉津縣,江中出璧故也



 綏山隋招致生獠,於榮樂城置綏山縣,取旁山名也



 羅目麟德二年,開生獠置沐州及
 羅目縣。上元三年,俱廢。儀鳳三年,又置,治沲和城,屬嘉州。如意元年,又自峨眉縣界移羅目治於今所也



 犍為本漢都,因山立名。舊屬戎州。上元元年,改屬嘉州。



 邛州上隋臨邛郡之依政縣。武德元年,割雅州之依政、臨邛、臨溪、蒲江、火井五縣,置邛州於依政縣。三年,又置安仁縣。顯慶二年,移州治於臨邛。天寶元年,改為臨邛郡。乾元元年,復為邛州。舊領縣六,戶一萬五千八百八十六,口七萬二千八百五十九。天寶領縣七,戶四萬二千
 一百七,口十九萬三百二十七。在京師西南二千五百一十五里,至東都三千三百七十一里。



 臨邛漢縣,屬蜀郡。邛水,出嚴道邛來山,入青衣江,故云臨邛。晉於益州唐隆縣置臨邛縣。後魏平蜀,自唐隆移臨邛縣治於漢臨邛縣西,立臨邛郡。隋罷郡,移臨邛縣於今所治。有火井、銅官山也



 依政秦蒲陽縣。漢臨邛縣。梁置蒲口鎮及邛州。後魏改為蒲陽郡,置依政縣。隋改為臨邛郡,治依政。梁、魏邛州,在今縣西南二里,
 後周移治於今所,後移治於臨邛



 安仁秦臨邛縣地。武德三年,置安仁縣。貞觀十七年廢。咸亨初,復置



 大邑咸亨二年,分益州晉原縣置也



 蒲江漢臨邛縣地。後魏置廣定縣,隋改為蒲江,南枕蒲水故也



 臨溪後魏分臨邛縣置也



 火井漢臨邛縣地。周置火井鎮,隋改鎮為縣也。



 雅州下都督府隋臨邛郡。武德元年,改為雅州,領嚴道、名山、盧山、依政、臨邛、蒲江、臨溪、蒙陽、漢源、火井、長松、
 靈關、楊啟、嘉良、大利、陽山十六縣。其年,割依政、臨邛、蒲江、臨溪、火井五縣置邛州;漢原、陽山二縣置登州。二年,置榮經縣。六年,省嘉良、楊啟、大利、靈關、蒙陽、長松六縣。九年,廢登州,還以陽山、漢源來屬。貞觀二年,又以陽山、漢源屬巂州。八年,又置百丈縣。永徽五年,以巂州漢源來屬。儀鳳四年,置飛越、大渡二縣。大足元年,又割漢源、飛越二縣置黎州。神龍三年,廢黎州,漢源、飛越屬雅州。開元三年,又割二縣置黎州,又置都督府。天寶元年,
 改為盧山郡。乾元元年,復為雅州,都督羈縻一十九州也,舊領縣五,戶一萬三百六十二,口四萬一千七百二十三。天寶,戶一萬八百九十二,口五萬四千四百一十九。在京師西南二千七百二十三里,至東都三千五百一里。



 嚴道漢縣,屬蜀郡。晉末大亂,夷獠據之。後魏開生獠,於此置蒙山郡,領始陽、蒙山二縣。隋改始陽為嚴道,蒙山為名山。仁壽四年,置雅州,煬帝改為嚴道



 盧山
 漢嚴道地。隋置盧山鎮,又改為縣。盧山,在縣西北六十里章盧山下,有山硤,口開三丈,長二百步,俗呼為盧關。關外即生獠也



 名山嚴道縣地。魏置蒙山縣,隋改為名山也



 百丈漢嚴道縣地,在漢臨邛南百二十里。有百丈山。武德置百丈鎮。貞觀八年,改鎮為縣



 榮經漢嚴道縣地。武德三年,置榮經縣。縣界有邛來山、九折阪、銅山也



 雅州都督一十九州,並生羌、生獠羈縻州,無州縣



 嘉梁州東石孔州西石孔州林波州涉邛州汶東州金林州費林州徐渠州會野州雉州中川州鉗矢州強雞州長臂州楊常州林燒州當仁州當馬州皆天寶已前,歲時貢奉,屬雅州都督。



 黎州下雅州之漢源縣。大足元年,割漢源、飛越二縣及巂州之陽山置黎州。天寶元年,改為洪源郡。乾元元
 年,復為黎州,領羈縻五十四州也。領縣三,戶一千七百三十一,口七千六百七十八。至京師二千九百五十里,至東都三千七百里。



 漢源越巂郡之地。隋漢源縣。長安四年,巡察使奏置黎州,後使宋乾徽奏廢入雅州。大足元年,又置黎州。神龍三年廢。開元三年,又置黎州,取蜀南沈黎地為名,州所治



 飛越儀鳳四年,分漢源於飛越水置縣,屬雅州。大足元年,屬黎州。長安二年,廢大渡縣,並入。神龍三
 年,屬雅州。開元三年,又屬黎州也



 通望舊陽山縣,屬巂州。大足元年,屬黎州。神龍二年,又屬巂州。開元元年,卻屬黎州。天寶元年,改為通望也



 黎州,統制五十四州,皆徼外生獠。無州,羈縻而已。羅巖州索古州秦上州輒榮州劇川州合欽州蓬州柏坡州博盧州明川州肔皮州蓬矢州大渡州米川州木屬州河東州
 諾莋州甫嵐州昌明州歸化州象川州叢夏州和良州和都州附樹州東川州上貴州滑川州比川州吉川州甫鏚州比地州蒼榮州野川州邛陳州貴林州護川州牒琮州浪彌州郎郭州上欽州時蓬州儼馬州橛查州邛川州護邛州腳川州開望州
 上蓬州比蓬州剝重州久護州瑤劍州明昌州。



 瀘州下都督府隋瀘川郡。武德元年,改為瀘州,領富世、江安、綿水、合江、來鳳、和義七縣。武德三年,置總管府,一州。九年,省來鳳。貞觀元年,置思隸、思逢、施陽三縣。仍置涇南縣。又省施陽縣。十三年,省思隸、思逢二縣。十七年,置溱、珍二州。儀鳳二年,又置晏、納、奉、浙、鞏、薛六州。載初二年,置順州。天授元年,置思峨州。久視元年,置淯州。二
 年罷州。並屬瀘州都督,凡十州。天寶元年,改為瀘川郡,依舊都督。乾元元年,復為瀘州。舊領縣六,戶一萬九千一百一十六,口六萬六千八百二十八。天寶,戶一萬六千五百九十四,口六萬五千七百一十一。在京師西南三千三百里,至東都四千一百九十六里。



 瀘川漢江陽縣地,屬犍為郡。梁置瀘州,故以江陽為瀘川縣,州所治也



 富義隋富世縣。貞觀二十三年,改為富義縣。界有富世鹽井,井深二百五十尺,以達鹽
 泉,俗呼玉女泉。以其井出鹽最多,人獲厚利,故云富世



 江安漢江陽縣地。晉時,生獠攻郡,破之,又置漢安縣。隋改為江安也



 合江漢符縣地,屬犍為郡。晉置安樂縣,後周改為合江也



 綿水漢江陽縣地,晉置綿水縣,當綿水入江之口也



 涇南貞觀八年,分瀘川置,在涇水之南



 瀘州都督十州,皆招撫夷獠置,無戶口、道里,羈縻州



 納州儀鳳二年,開山洞置。天寶元年,改為都寧郡。乾
 元元年,復為納州,領縣八,並與州同置



 羅圍播羅施陽都寧羅當羅藍都闕胡茂薛州儀鳳二年,招生獠置。天寶元年,改為黃池郡。乾元元年,復為薛州也。領縣三,與州同置



 枝江黃池播陵晏州儀鳳二年,開山洞置。天寶改為羅陽郡。乾元元年,復為晏州也。領縣七,與州同置。思峨柯陰新
 賓扶來思晏多岡羅陽鞏州儀鳳二年,開山洞置。天寶改為因忠郡。乾元元年,復為鞏州也。領縣四,與州同置。多樓波員比求播郎順州載初二年置,領縣五,與州同置。



 曲水順山靈巖來猿龍池奉州儀鳳二年置,領縣三,與州同置。柯理柯巴羅蓬
 思峨州天授元年置,領縣二,與州同置。多溪洛溪能州大足元年置,領縣四,與州同置。長寧來銀菊池猿山淯州久視元年置,領縣四,與州同置。新定淯川固城居牢浙州儀鳳二年置,領縣四,與州同置。浙源越賓洛川鱗山



 茂州都督府隋汶山郡。武德元年,改為會川,領汶山、北山、汶川、左封、通化、翼針、交川、翼水九縣。其年,割翼針、左封、翼水三縣置翼州,以交川屬松州。三年,置總管府,管會、翼二州。四年,改為南會州。七年,改為都督府,督南會、翼、向、維、塗、冉、穹、炎、徹、笮十州。貞觀八年,改為茂州,以郡界茂濕山為名。仍置石泉縣。天寶元年,改為通化郡。乾元元年,復為茂州也。舊領縣四,戶三千三百八十六,口五萬三千七百六十一。天寶,戶二千五百一十,口一萬
 三千二百四十二。至京師西南二千七百九十四里,至東都三千一十四里。



 汶山漢汶江縣,屬蜀郡。故城在今縣北二里,舊冉駹地。晉汶山郡,宋廣陽縣。周為汶州,置汶山縣。隋初,改為蜀州,又改為會州。貞觀八年,改為茂州



 汶川漢綿皦縣地,屬蜀郡。晉置汶山縣,後周移汶川於廣陽縣齊州置,即今治也。玉壘山,在縣東北四里。石紐山,亦在縣界。永徽二年,廢汶川縣並入



 石泉漢岷山縣,屬蜀
 郡。貞觀八年,置石泉縣也



 通化漢廣柔縣地,屬蜀郡。後周置石門鎮,陳改為金山鎮,尋改為通化也



 茂州都督府,羈縻州十。維、翼兩州,後進為正州。相次為正者七,今附於都督之下。



 翼州下隋汶山郡之翼針縣。武德元年,分置翼州。六年,自左封移州治於翼針。咸亨三年,置都督府,移就悉州城內。上元二年。罷都督,移還舊治。天寶元年,改為臨翼郡。乾元元年,復為翼州也。舊領縣三,戶一
 千六百二,口三千八百九十八。天寶領縣二,戶七百一十一,口三千六百一十八。在京師西南二千九百三十里,至東都三千二百七十八里。



 衛山漢蠶陵縣,屬蜀郡。故城在縣西,有蠶陵山。隋改為翼針縣,治七頃城。貞觀十七年,移治七里溪。天寶元年,改為衛山縣



 翼水漢蠶陵縣,隋置翼水縣也



 溪川昭德二縣開生獠新置。



 維州下武德元年,白茍羌降附,乃於姜維故城置維
 州,領金川、定廉二縣。貞觀元年,羌叛,州縣俱罷。二年,生羌首領董屈占者,請吏復立維州,移治於姜維城東,始屬茂州,為羈縻州。麟德二年,進為正州。尋叛,羌降,為羈縻州。垂拱三年,又為正州。天寶元年,改為維川郡。乾元元年,復為維州。上元元年後,河西、隴右州縣,皆陷吐蕃。贊普更欲圖蜀川,累急攻維州,不下,乃以婦人嫁維州門者。二十年中,生二子。及蕃兵攻城,二子內應,城遂陷。吐蕃得之,號無憂城。累入兵寇擾西川。韋皋在蜀二十年,
 收復不遂。至大中末,杜忭鎮蜀,維州首領內附,方復隸西川。舊領縣三,,戶二千一百四十二,無口。天寶領縣二。戶二千一百七十九,口三千一百九十八。至京師二千八百三十里,至東都三千五百六十三里。



 薛城漢已前,徼外羌冉駹之地。蜀劉禪時,蜀將姜維、馬忠等,討汶山叛羌,即此地也。今州城,即姜維故壘也。隋初,蜀師討叛羌,於其地置薛城戍。大業末,又沒於羌。武德七年,白茍羌酋鄧賢佐內附,乃於姜維城置維州,
 領金川、定廉二縣。貞觀元年,賢佐叛,罷郡縣。三年,左上封生羌酋董屈占等,舉族內附,復置維州及二縣。薛城,在州西南二百步也



 小封咸亨二年,剌史董弄招慰生羌置也。



 塗州下武德元年,臨塗羌歸附,置塗州,領端源、婆覽二縣。貞觀二年,州縣俱省。五年,又分茂州之端源戍置塗州也。領縣三,與州同置



 端源臨塗悉憐戶二千三百三十四,口四千二百六十一。至京師西南二
 千六百八十九里。



 炎州下貞觀五年,生羌歸附,置西封州。八年,改為炎州。領縣三,與州同置:大封慕仙義川



 領戶五千七百,無口數。在京師西南三千三百七十六里。



 徹州下貞觀五年,西羌首領董凋貞歸化置。領縣三,與州同置:



 文徹俄耳文進



 領戶三千三百,無口數,在京師西南三千四百一十八里。



 向州下貞觀五年,生羌歸化置也。領縣二,與州同置



 貝左向貳領戶一千六百二,口三千八百九十八,在京師西南二千八百六十九里。



 冉州下,本徼外斂才羌地。貞觀五年,置西冉州。九年,去「西」字。領縣四,與州同置。領戶一千六百二,口三千八百九十八。在京師西南二千八百六十九里



 冉州磨山玉溪金水



 領戶一千三百七十,無口。在京師西南三千七百三十九里。



 穹州下貞觀五年,生羌歸附,置西博州。八年,改為穹州。領縣五,與州同置。領戶三千四百三十六,無口。在京師
 西南三千二百六十七里。



 笮州下貞觀七年,白茍羌降附,置西恭州。八年,改為笮州也。領縣三,與州同置:遂都亭勸北思



 無口戶。在京師西南二千九百四十五里。



 右九州,皆屬茂州都督。永徽後,又析為三十一州,今不錄其餘也。



 戎州中都督府隋犍為郡。武德元年,改為戎州,領僰道、犍為、南溪、開邊、存阜馬阜五縣。貞觀四年,以開邊屬南通州。於州置都督府,督戎、郎、昆、曲、協、黎、盤、曾、鉤、髳、尹、匡、
 裒、宗、靡、姚、微十七州。八年,置撫來縣。仍改南通州為賢州,以開邊來屬。天寶元年,改為南溪郡,依舊都督,羈縻三十六州,一百三十七縣。並荒梗,無戶口。乾元元年,復為戎州。舊領縣六,戶三萬一千六百七十,口六萬一千二十六。天寶領縣五,戶四千三百五十九,口一萬六千三百七十五。在京師西南三千一百四里,至東都四千四百八十里。



 僰道漢縣,犍為郡治所。故僰侯國,梁置戎州也



 南
 溪漢南廣縣,屬犍為郡。後周於廢郡置南武戍。隋改龍源戍,又置為南溪縣也



 義賓本漢南安縣,屬犍為郡。隋改為存阜馬阜縣。天寶元年,改為義賓



 開邊漢僰道地,隋置開邊縣也



 歸順聖歷二年,分存阜馬阜縣置,以處生獠也



 戎州都督府,羈縻州十六,武德、貞觀後招慰羌戎開置也。



 協州下隋犍為郡之地。古夜郎侯國。武德元年,開南中置也。領縣三,與州同置。東安西安湖津



 領戶三百二十九。在京師西南四千里。北接戎州。



 曲州下武德元年,開南中置恭州。八年,改為曲州。領縣二,與州同置



 硃提武德元年,置安上縣。七年,改為硃提



 唐興



 領戶一千九十四。在京師西南四千三百三十里。北接協州。



 郎州下武德元年,開南中置南寧州,乃立味、同樂、升麻、同起、新豐、隴堤、泉麻、梁水、降九縣。武德四年,置總管府,管南寧、恭、協、昆、尹、曾、姚、西濮、西宗九州。五年,罷總管。其年冬,復置,寄治益州。七年,改為都督,督西寧、豫、西利、
 南雲、磨、南籠七州。並前九州,合十六州。仍割南寧州之降縣屬西寧州。八年,自益州移都督於今治。貞觀六年,罷都督,置刺史。八年,改南寧為郎州也。領縣七。



 味隋廢同樂縣,武德元年復置,改名



 同樂升麻同起新豐隴堤泉麻並與州同置。戶六千九百四十二。在京師西南五千六百七十里。北接曲州。



 昆州下漢益州郡地。武德初,招慰置。領縣四,與州同置。



 益寧



 晉寧有滇池,周三百里



 安寧秦臧
 漢縣。領戶一千二百六十七。在京師西南五千三百七十里。北接巂州。



 盤州下武德七年,開置西平州。貞觀八年,改為盤州。領縣三,與州同置。



 附唐平夷盤水即舊興古郡也。領戶一千九百六十。在京師西南五千三十里。北接郎州,南接交州。



 黎州下武德七年,析南寧州置西寧州。貞觀八年,改為黎州。領縣二,二縣本屬南寧。梁水絳。領戶一
 千。至京師無里數。北接昆州。



 匡州下武德七年,開置南雲州。貞觀三年,改為匡州也。領縣二,與州同置。



 勃弄匡川縣界有永昌故城也。領戶四千八百。在京師西南五千一百六十五里。



 髳州下武德四年,置西濮州。貞觀十一年,改為髳州也。領縣四,與州同置。



 濮水青蛉舊屬越巂郡。



 歧星銅山



 領戶一千三百九十。在京師西南四千
 八百五十里。南接姚州。



 尹州下武德四年置。領縣五,與州同置。



 馬邑天池鹽泉甘泉湧泉



 領戶一千七百。無里數。接髳州。



 曾州下武德四年置。領縣五,與州同置。



 曾三部神泉龍亭長和



 領戶一千二百七。在京師西南五千一百四十五里。西接匡州。



 鉤州下武德七年,置南龍州。貞觀十一年,改為鉤州
 也。領縣二,與州同置。



 望水唐封



 領戶一千。在京師西南五千六百五十里。北接昆州。



 靡州下武德七年,置西豫州。貞觀三年,改為靡州。領縣二,與州同置



 磨豫七部



 領戶一千二百。在京師西南四千九百四十五里。南接姚州。



 裒州下武德四年置。領縣二,與州同置



 揚彼強樂



 領戶一千四百七十。在京師西南四千九百七十里。南接姚州。



 宗州武德四年,置西宗州。貞觀十一年,去「西」字。領縣三,與州同置



 宗居石塔河西



 領戶一千九百三十。在京師西南五千一十里。北接姚州。



 微州下武德四年,置利州。貞觀十一年,改為微州。領縣二,與州同置



 深利十部



 領戶一千一百五十。在京師西南四千九百七十里。東接靡州。



 姚州武德四年置,在姚府舊城北百餘步。漢益州郡之云南縣。古滇王國。楚頃襄王使大將莊蹻溯沅水,出
 且蘭,以伐夜郎。屬秦奪楚黔中地,蹻無路能還,遂自王之。秦並蜀,通五尺道,置吏。漢武開西南夷,置益州郡,雲南即屬邑也。後置永昌郡,雲南、哀牢、博南皆屬邑也。蜀劉氏分永昌為建寧郡,又分永昌、建寧置雲南郡,而治於弄棟。晉改為晉寧郡,又置寧州。武德四年,安撫大使李英以此州內人多姓姚,故置姚州,管州三十二。麟德元年,移姚州治於弄棟川。自是朝貢不絕。天寶末,楊國忠用事,蜀帥撫慰不謹,蠻王閣羅鳳不恭,國忠命鮮於
 仲通興師十萬,渡瀘討之,大為羅鳳所敗。鎮蜀,蠻帥異牟尋歸國,遂以韋皋為雲南安撫大使,命使冊拜,謂之南詔。大和中,杜元穎鎮蜀,蠻王鹺顛侵蜀,自是或臣或否。咸通中,結構南海蠻,深寇蜀部。西南夷之中,南詔蠻最大也。領縣二。



 瀘南縣在瀘水之南



 長明。戶三千七百。至京師四千九百里。



 右上十六州,舊屬戎州都督府。天寶已前,朝貢不絕。



 巂州中都督府隋越巂郡。武德元年,改為巂州,領巂、
 邛部、可泉、蘇祁、臺登六縣。二年,又置昆明縣。三年,置總管府,管一州。貞觀二年,割雅州陽山、漢源二縣來屬。八年,又置和集縣。天寶元年,越巂郡,依舊都督府。乾元元年,復為巂州也。舊領縣十,戶二萬三千五十四,口五萬三千六百一十八。天寶領縣七,戶四萬七百二十一,口十七萬五千二在八十。在京師西南三千六百五十四里。



 越巂漢郡名,武帝置。今縣,漢邛都縣地,屬越巂郡。有越水、巂水。後周於越城置嚴州。隋改為西寧州,尋改巂
 州,仍分邛都置越巂縣,州所治也



 邛部後漢屬越巂郡。漢闌縣地,屬沈黎郡。後周置節部縣也



 臺登漢縣,屬越巂郡



 蘇祁漢蘇夷縣,屬越巂郡。後周平南夷,於故城復置也



 西瀘漢邛都縣地,梁置可泉縣。隋治姜磨戍。武德七年,移於今。天寶末年。改為西瀘也



 昆明漢定莋縣,屬越巂郡。後周置定莋鎮。武德二年,鎮為昆明縣,蓋南接昆明之地故也



 會川上元二年,移邛都縣於會川置,因改為會川也。



 松州下都督府隋同昌郡之嘉誠縣。武德元年,置松州。貞觀二年,置都督府。督崌、懿、嵯、闊、麟、雅、叢、可、遠、奉、嚴、諾、蛾、彭、軌、蓋、直、肆、位、玉、璋、祐、臺、橋、序二十五羈縻等州。永徽之後,生羌相繼忽叛,屢有廢置。儀鳳二年,復加整比,督文、扶、當、柘、靜、翼六州。都督羈縻三十州:研州、劍州、探那州、忋州、毗州、河州、乾州、瓊州、犀州、拱州、龕州、陪州、如州、麻州、霸州、闌州、光州、至涼州、蠶州、曄州、梨州、思帝州、戍州、統州、穀州、邛州、樂客州、達違州、卑州、慈州。據天
 寶十二載簿,松州都督府,一百四州,其二十五州有額戶口,但多羈縻逃散,餘七十九州皆生羌部落,或臣或否,無州縣戶口,但羈縻統之。天寶元年,改松州為交川郡。乾元元年,復為松州。據貞觀初分十道:松、文、扶、當、悉、柘、靜等屬隴右道。永徽之後,據梁州之境,割屬劍南道也。舊領縣三,戶六百一十二,口六千三百五。天寶,戶一千七十六,口五千七百四十二。南至翼州一百八十里,東至扶州三百三十八里,東至茂州三百里,西南至當
 州三百里,西北至吐蕃界九十里。至京師二千二百五十里,至東都三千五十里。



 嘉誠歷代生羌之地,漢帝招慰之,置護羌校尉,別無州縣。至後魏,白水羌象舒治自稱鄧至王,據此地。其子舒彭遣使朝貢,乃拜龍驤將軍、甘松縣子,始置甘松縣。魏末大亂,又絕。後周復招慰之,於此置龍涸防。天和六年,改置扶州,領龍涸郡。隋改甘松為嘉誠縣,屬同昌郡。武德元年,於縣置松州,取州界甘松嶺為名



 交川後
 周置龍涸郡,隋廢為交川縣也



 平康垂拱元年,割交川及當州通軌、翼針三縣置平康縣,屬當州。天寶元年,改交川郡也。



 文州隋武都郡之曲水縣。義寧二年,置陰平郡,領曲水、長松、正西三縣。武德改文州。貞觀元年,省正西入曲水。天寶元年,改為陰平郡。乾元元年,復為文州。舊屬隴右道,隸松州都督。永徽中,改屬劍南道也。舊領縣二,戶一千九百八,口八千一百四十七。天寶,戶一千六百八
 十六,口九千二百五。在京師西南一千四百九十里,至東都二千二百九十里。



 曲水漢陰平道,屬廣漢。晉亂,楊茂搜據為仇池,氐、羌相傳疊代。後魏平氐、羌,始置文州。隋為曲水縣。武德後,置文州,治於曲水也。



 長松後魏置蘆北郡,郡置建昌縣。後周移郡縣於此置。隋廢郡,改縣為長松。白馬水在縣北也。



 扶州隋同昌郡。天寶元年,改為扶州。天寶元年,復為
 同昌郡。乾元元年,復為扶州。舊屬隴右道,隸松州都督。永徽後,改為劍南道。舊領縣四,戶一千九百二十八,口八五百五十六。天寶,戶二千四百一十八,口一萬四千二百八十五。在京師西南一千六百九十里,至東都二千四百四十九里。



 同昌歷代吐谷渾所據。西魏逐吐谷渾,於此置鄧州及鄧寧郡,蓋以平定鄧至羌為名。隋初,改置扶州及同昌縣。煬帝又為同昌郡。流於此也。



 帖夷後魏置帖
 夷郡。隋罷為縣。萬歲通天二年,改為武進。神龍依舊為帖夷。



 萬全後魏置武進郡,又改為上安郡。隋廢郡為尚安縣。舊治刺利村,長安二年,移治黑水堡。至德二年八月,改為萬全也。



 鉗川後魏置鉗川郡。隋罷郡,復為縣。



 龍州下隋平武郡。武德元年,改為龍門郡。其年,加「西」字。貞觀元年,改為龍州。天寶元年,改為江油郡。乾元元年,復為龍州。舊屬隴右道,永徽後,割屬劍南也。舊領
 縣二,戶一千一十七,口六千一百四十九。天寶,戶二千九在九十二,口四千二百二十八。在京師西南二千六百六十里,至東都三千一十五里。



 江油秦、漢、曹魏為無人之境。鄧艾伐蜀,由陰道景谷,行無人之地七百里,鑿山通道,攀木緣崖,魚貫而進,以至江油,即此城也。晉始置陰平郡,於此置平武縣。至梁有楊、李二姓大豪,分據其地。後魏平蜀,置龍州。隋初廢郡,改平武為江油。縣界有石門山。



 清川後魏馬盤縣。
 天寶元年,改為清川也。



 當州下本松州之通軌縣。貞觀二十一年,析置當州,以土出當歸為名。州治利川,領通軌、左封二縣。顯慶二年,又析左封置悉州。儀鳳二年,移治逢臼橋。天寶元年,改為江源郡。乾元元年,復為當州。本屬隴右道也。領縣三,戶二千一百四十六,口六千七百一十三。至京師三千一百里,至東都三千九百里。東北至松州九百里。



 通軌本屬松州,歷代生羌之地。貞觀二十年,松州首
 領董和那蓬固守松府,特敕於通軌縣置當州,以蓬為刺史。顯慶元年,蓬嫡子屈寧襲繼為刺史。又置和利、谷利、平康三縣也



 和利顯慶二年,分通軌置。谷利文明元年,開生羌置也。



 悉州本翼州之左封縣。顯慶元年,置悉州,領悉唐、左封、識臼三縣,治悉唐城。咸亨元年,移治左封。儀鳳二年,羌叛,又寄治當州城內,尋歸舊治。垂拱二年,置歸誠縣。載初元年,移治匪平川。天寶元年,改為歸誠郡,割識臼屬
 臨翼郡。乾元元年,復為悉州。舊屬隴右道松州都督,後屬劍南道。領縣二,戶八百一十六,口三千九百一十四。至京師二千七百五十里,至東都三千八百里。至西靜州六十里,西北至當州八十里也。



 左封本屬翼州,在當州東南四十里。顯慶元年,生羌首領董系比射內附,乃於地置悉州,在悉唐川故也。以董系比射為刺史,領左封、歸城二縣。載初元年,又移州理東南五十里匪平川置也。歸誠垂拱二年,分左
 封置。



 靜州本當州之悉唐縣。顯慶元年,於縣置悉州。咸亨元年,於悉州置翼州都督府,移悉州理左封置。儀鳳二年,罷都督府,翼州卻還治於翼針縣,於悉唐縣置南和州。天授二年,改為靜州,比屬隴右道,隸松州都督。後割屬劍南。領縣二,戶一千五百七十七,口六千六百六十九。東北至當州六十里,東至悉州八十里。至京師與當州道里數同也。



 悉唐縣置在悉唐川。舊屬當州,顯慶中來屬也。



 靜居縣界有靜川也。



 恭州下開元二十四年,分靜州廣平縣置恭州,仍置博恭、烈山二縣。天寶元年,改為恭化郡。乾元元年,復為恭州。本屬隴右道,後割屬劍南。領縣三,戶一千一百八十九,口六千二百二十二。東至柘州一百里,東北至靜州界。至京師三千一百二十里。



 和集舊廣平縣,屬靜州。開元二十四年,於縣置恭州。
 天寶元年,改為和集。



 博恭開元二十四年,分廣平置也。



 烈山開元二十四年,分廣平置。



 柘州下永徽後置。天寶元年,改為蓬山郡。乾元元年,復為柘州。本屬隴右道松州都督,後割屬劍南也。



 保州下本維州之定廉縣。開元二十八年,置奉州,以董晏立為刺史。領定廉一縣。天寶元年,改為雲山郡。八載,移治所於天保軍,乃改為天保郡。乾元元年二月,西山子弟兵馬使嗣歸誠王董嘉俊以西山管內天保郡
 歸附,乃為保州,以嘉俊為刺史。領縣三,戶一千二百四十五,口四千五百三十六。至京師二千九百四十里,至東都三千七百九十里。東至維州風流鎮四十五里也。



 定廉隋置定廉鎮。隋末陷羌。武德七年,招白茍羌,置維州及定廉縣,以界水名。永徽元年,廢鹽城並入。開元二十八年,改屬奉州。天寶八載,改為天保郡也



 歸順雲山天寶八年,分定廉置此二縣也。



 真州下天寶五載,分臨翼郡之昭德、雞川兩縣置昭
 德郡。乾元元年,改為真州,取真符縣為名也。領縣三,戶六百七十六,口三千一百四十七。至京師三千里,至東都三千八百五十里。



 真符天寶五載,分雞川、昭德二縣置,州所治也



 雞川先天二年,割翼州翼水縣置,屬翼州。天寶五載,改真州



 昭德本識臼縣,屬悉州。天寶元年,改屬翼州,仍改名昭德縣。五年,改屬真州也。



 霸州下天寶元年,因招附生羌置靜戎郡。乾元元年,
 改為霸州也。領縣一,戶一百七十一,口一千八百六十一。至京師二千六百三十二里,至東都三千二百七十一里



 安信與郡同置,州所治也。



 已上十二州,舊屬隴右道,永徽已後,割屬松州都督,入劍南道。諸州隸松州都督,相繼屬劍南也。



 松州都督府,督羈縻二十五州。舊督一百四州,領州,無縣戶口,惟二十五有名額,皆招撫生羌置也。



 崌州下貞觀元年,招慰黨項置州處也。領縣二,與州同置:江源洛稽。領戶一百五十五。至京師西南二千二百四十六里。



 懿州下貞觀五年,置西吉州。八年,改為懿州,處黨項也。領縣二,與州同置:吉當唐位。無戶口。至京師西南二千二百五十里。



 闊州下貞觀五年置,處黨項。領縣二,與州同置:闊源落吳。無戶口。至京師西南二千五百一十里。



 麟州下貞觀五年,置西麟州,處生羌歸附。八年,去「西」字。領縣七,與州同置:硤川和善斂具硤源三交利恭東陵無戶口。至京師四千五百里。



 雅州下貞觀五年,處生羌置西雅州。八年,去「西」字。領縣三,與州同置:新城三泉石隴無戶口。至京師西南二千六百六十里。



 叢州貞觀五年,黨項發歸附置也。領縣五,與州同置。都流厥調湊般匐器邇率鐘,並為諸羌部落,遙立,無縣。
 寧遠臨泉臨河無戶口。至京師西南一千八百里。



 可州貞觀四年,處黨項,置西義州。八年,改為可州也。領縣三,與州同置:



 義誠清化靜方無戶口。至京師西南一千四十里。



 遠州貞觀四年,生羌歸附置也。領縣二,與州同置。羅水、小部川。無戶口。至京師西南二千三百六十
 里。



 奉州貞觀三年,處生羌置西仁州。八年,改為奉州也。領縣三,與州同置:奉德思安永慈無戶口。至京師西南二千一百六里。



 巖州貞觀五年,置西金州。八年,改為巖州。領縣三,與州同置:金池甘松丹巖



 無戶口。至京師西南二千一百里。



 諾州貞觀五年,處降羌置。領縣三,與州同置:諾川歸德籬渭



 無戶口。至京師西南二千六百四十
 三里。



 蛾州貞觀五年,處降羌置。領縣二,與州同置:常平那川



 無戶口。至京師二千七百里。



 彭州貞觀三年,處降黨項置洪州。七年,改為彭州。領縣四,與州同置:洪川歸遠臨津歸正



 無戶口。至京師西南二千七百八十里。



 軌州都督府,貞觀二年,處黨項置。領縣四,與州同置:通川玉城金原俄徹



 無戶口。至京師西南
 二千三百九十里。



 盍州貞觀四年,置西唐州。八年,改為盍州,處降羌也。領縣四,與州同置:湘水河唐曲嶺枯川



 戶二百二十,無口。至京師西南二千六百三十里。



 直州貞觀五年,置西集州。八年,改為直州,處降羌。領縣二,與州同置:集川新川



 戶一百,無口。至京師二千五百里。



 肆州貞觀五年,處降羌置。領縣四,與州同置:歸唐
 芳叢鹽水磨山



 無戶口。至京師二千六百里。



 位州貞觀四年,降生羌置西鹽州。八年,改為位州。領縣二,與州同置:位豐西使



 戶一百,無口。至京師二千四百一十里。



 玉州貞觀五年,處降羌置。領縣二,與州同置:玉山帶河



 戶二百一十五,無口。至京師二千八百七十八里。



 嶂州貞觀四年,處降羌置。領縣四,與州同置:洛平
 顯川桂川顯平



 戶二百,無口。至京師二千九百里。



 祐州貞觀四年,處降羌置。領縣二,與州同置:廓川歸定



 無戶口。至京師二千一百九十里。



 臺州貞觀六年,處黨項置西滄州。八年,改為臺州。無縣。至京師二千一百三十五里。



 橋州貞觀六年,處降羌置。無縣。至京師二千四百里。



 序州貞觀十年,處黨項置。無縣。至京師二千四百里。



 右二十五州,舊屬隴右道,隸松州都督府。貞觀中,招慰黨項羌漸置。永徽已後,羌戎叛臣,制置不一。今存招降之始,以表太平之所至也。



 嶺南道



 南海節度使,領是十七州也。



 廣州中都督府隋南海郡。武德四年,討平蕭銑,置廣州總管府,管廣、東衡、洭、南綏、岡五州,並南康總管。其廣州領南海、增城、清遠、政賓、寶安五縣。六年,又置高、循二
 總管,隸廣州。七年,改總管為大都督。九年,廢南康都督,以端、封、宋、洭、瀧、建、齊、威、扶、議、勤十一州隸廣府。其年,又省勤州。貞觀改中都督府,省威、齊、宋、洭、四州,仍以廢洭州之湞陽、浛洭二縣來屬。改東衡為韶州,仍以南康州及崖州都督,並隸廣州。二年,省循州都督,以循、潮二州隸廣府。八年,改建州為藥川、南綏州為湞州、南會州為竇州。十二年,改南康州。十三年,省湞州,以四會、化蒙、懷集、水存安四縣來屬。省岡州,以義寧、新會二縣並屬廣州。
 其年,又以義寧、新會二縣立岡州。今督廣、韶、端、康、封、岡、新、藥、瀧、竇、義、雷、循、潮十四州。永徽後,以廣、桂、容、邕、安南府,皆隸廣府都督統攝,謂之五府節度使,名嶺南五管。天寶元年,改為南海郡。乾元元年,復為廣州。州內有經略軍,管鎮兵五千四百人,其衣糧輕稅,本道自給。廣州刺史,充嶺南五府經略使。舊領縣十,戶一萬二千四百六十三,口五萬九千一百一十四。天寶領縣十三,戶四萬二千二百三十五。在京師東南五千四百四十七里,至東都四千九百里。



 南海五嶺之南,漲海之北,三代已前,是為荒服。秦滅六國,始開越置三郡,曰南海、桂林、象郡,以謫戍守之。秦
 亡,南海尉任囂病且死,召南海龍川令趙佗,付以尉事。佗乃聚兵守五嶺,擊並桂林、象郡,自稱南越武王。子孫相傳五代九十三年。漢武帝命伏波將軍路博德、樓船將軍楊僕兵逾嶺南,滅之。其地立九郡,曰南海、蒼梧、鬱林、合浦、交止、九真、日南、儋耳、珠崖。後漢廢珠崖、儋耳入合浦郡。交州刺史領七郡而已。今南海縣即漢番禺縣,南海郡。隋分番禺置南海縣。番山,在州東三百步。禺山,在北一里。貪泉,州西三十里。越王井,州北四里。番禺
 漢縣名,秦屬南海郡。後漢置交州,領郡七。吳置廣州。皆治番禺也



 增城後漢番禺縣地。吳於縣置東宮。有增江



 四會漢縣,屬南海。武德五年,於縣治北置南綏州,領四會、化蒙、新招、化穆、化注五縣。貞觀元年,省新招、化注二縣,以廢威州之懷集、廢齊州之水存安二縣來屬。八年,改為湞州。十三年,省州及化穆縣,以四會、化蒙、懷集、洊安四縣屬廣州也



 化蒙隋縣。武德五年,屬南綏州。貞觀元年,省化注入。八年,改綏州為湞州,縣
 仍屬。十三年,改屬廣州



 懷集晉懷化縣,隋為懷集。武德五年,於縣置威州,領興平、懷集、霍清、威成四縣。貞觀元年,州廢,以懷集屬南綏,省興平、霍清、威成三縣。八年,改綏州為湞州,縣仍屬。十三年,屬廣州



 東莞隋寶安縣。至德二年九月,改為東莞。郡,於嶺外其為名也



 清遠隋縣。武德六年,廢政賓縣並入,所治也



 洊水漢封陽縣,屬蒼梧郡。南齊改為洊安。武德四年,於縣置齊州,領洊安、宣樂、宋昌三縣。貞觀元年,省齊州及
 宣樂、宋宣二縣,以洊安屬綏州。八年,改綏州為湞州,縣仍屬。十三年,湞州廢,屬賓州。至德二年九月,改為洊水也



 湞陽漢縣,屬桂陽郡。隋為真陽。五年,屬洭州。貞觀初,州廢,改真陽湞陽,屬廣州。湞山,在縣北三十里。



 韶州隋南海郡之曲江縣。武德四年,平蕭銑,置番州,領曲江、始興、樂昌、臨瀧、良化五縣。貞觀元年,改為韶州,仍割洭州之翁源來屬。八年,廢臨瀧、良化二縣。天寶元年,改為始興郡。乾元元年,復為韶州。舊領縣四,戶六千
 九百六十,口四萬四百一十六。天寶領縣六,戶三萬一千,口十六萬八千九百四十八。南至廣州八百里,西至郴州五百里,東南至虔州七百里。至京師四千九百三十二里,至東都四千一百四十二里。



 曲江漢縣,屬桂陽郡。在曲江川,州所治也



 始興漢南野縣地,屬豫章郡。孫皓分南康郡之南鄉,始興縣置。縣界東嶠,一名大庾嶺,南越之北塞。漢討南越時,有將軍姓庾,城於此。五嶺之最東,故曰東嶠也



 樂昌
 隋置



 翁源翁水在縣界。隋縣。武德五年,置洭州。貞觀初廢,以屬韶州



 仁化湞昌已上二縣,天寶後新置。



 循州隋龍川郡。武德五年,改為循州總管府,管循、潮二州。循州領歸善、河源、博羅、興寧、海豐、羅陽。省龍川入歸善、石城入河源、齊昌入興寧。貞觀二年,廢都督府。天寶元年,改為海豐郡。乾元元年,復為循州。舊領縣五,戶六千八百九十一,口三萬六千四百三十六。天寶領縣六,戶九
 千五百二十五,無口數。南至廣州四百里,東至潮州五百一十七里,北至虔州隔山嶺一千六百五十里。至東都四千八百里。



 歸善秦、漢龍川縣地,屬南海郡。宋置歸善縣,縣界羅浮山。貞觀元年,省龍川縣並入。博羅漢舊縣,屬南海郡也。河源隋縣。循江,一名河源水,自虔州雩都縣流入。龍川,在河源縣,云有龍穿地而出,即水流,漢因置龍川縣。貞觀元年,省石城並入。海豐宋縣,屬東
 莞郡。南海在海豐縣南五十里,即漲海,渺漫無際。武德五年,分置陸安縣。貞觀初並入也。興寧漢龍川縣地。貞觀元年省齊昌並入。雷鄉新置。



 岡州隋南海郡之新會縣。武德四年,平蕭銑,置岡州,領新會、封平、義寧三縣。貞觀五年,州廢,以新會、義寧屬廣州,省封平、封樂二縣。其年,又立岡州,割廣州之新會、義寧來屬。又立封樂縣。天寶元年改為義寧郡。乾元元年,復為岡州也。舊領縣二,戶二千三百五十八,口八千
 六百六十二。天寶,戶五千六百五十,無口數。在京師西南六千三百五里。



 新會漢南海縣地。晉置新會郡。改置封州,又改為允州,又改為岡州。隋末廢,並入廣州。武德四年,復為岡州。舊治盆源城。貞觀十三年,廢岡州,縣屬廣州。其年,復置州於今治也



 義寧漢番禺縣地。宋置義寧縣,屬新會郡。



 賀州隋蒼梧郡之臨賀縣。武德四年,平蕭銑,置賀州。
 天寶元年,改為臨賀郡。乾元元年,復為賀州也。舊領縣五,戶六千七百一十三,口一萬八千六百二十八。天寶領縣六,戶四千五百,無口數。在京師東南四千一百三十里,至東都三千五百七十二里。東南至廣州八百七十六里,東至連州二百六十里,南至封州三百六十六里,北至道州四百里,北至富州三百二十里,西南至梧州四百二十二里也。



 臨賀州所治。漢縣,屬蒼梧郡。臨、賀水。吳置臨賀郡。宋
 改為臨慶國,齊復為臨賀郡。隋置賀州,隋末廢為縣。武德四年,復置賀州



 桂嶺漢臨賀縣地,隋舊也



 馮乘漢縣,屬蒼梧郡。有荔平關



 封陽漢縣,屬蒼梧郡



 富川漢富川縣。天寶改為富水,後復為富川也



 蕩山新置。



 端州隋信安郡。武德元年,置端州,領高要、樂城、銅陵、平興、博林五縣。其年,以樂城屬康州,銅陵屬春州。七年,置清泰縣。貞觀十三年,省博林、清泰二縣。天寶元年,為
 高要郡。乾元元年,復為端州。舊領縣二,戶四千四百九十一,口二萬四千三百三。天寶,戶九千五百,口二萬一千一百二十。東至廣州二百四十里,南至新州一百四十里,西至康州一百六里。至京師四千九百三十五里,至東都四千七百里。



 高要州所治。漢縣,屬蒼梧郡。宋、齊屬南海郡。陳置高要郡,隋置端州。縣北五里有石室山。縣西有鵠奔亭,即漢交州刺史行部到鵠亭,夜,女子鬼訴冤之亭



 平興
 漢高要縣地,隋分置。武德七年,分置清泰縣。貞觀十三年,省清泰並入。



 新州隋信安郡之新興縣。武德四年,平蕭銑,置新州。天寶元年,改為新興郡。乾元元年,復為新州。舊領縣四,戶七千三百八十八,口三萬五千二十五。天寶領縣三,戶九千五百。東至廣州義寧縣四十一里,北至端州一百四十里,西北至康州二百七十里,西南至勤州一百七十里。至京師五千五十二里,至東都五千里。



 新興漢臨允縣,屬合浦郡。晉置新寧郡,梁置新州



 索盧武德四年,析新興縣置



 永順新置。



 康州隋信安郡之端溪縣。武德四年,置康州都督府,督端、康、封、新、宋、瀧等州。九年,廢都督府及康州。貞觀元年,又置南康州。十一年廢,十二年又置康州。天寶元年,改為晉康郡。乾元元年,復為康州。舊領縣四,戶四千一百二十四,口一萬三千五百四。天寶,戶一萬五百一十,口一萬七千二百一十九。東北至廣州三百四十里,西
 南至梧州二百八十四里,東至端州一百六十里,南至瀧州二百三十里,西至封州一百三十里,南至新州二百七十里。至京師五千七百五十里,至東都五千一百五十里。



 端溪漢縣,屬蒼梧郡。晉於縣分置晉康郡。隋廢郡,並入信安郡。武德復置康州。縣界有端山,山下有溪也



 晉康隋安遂縣。至德二年,改為晉康縣



 悅城隋樂城縣。武德五年,屬端州。又割屬康州,改為悅城



 都
 城漢端溪縣。東百步有程溪,亦名零溪,溫嫗養龍之溪也。



 封州下隋蒼梧郡之封川縣。武德四年,平蕭銑,置封州。天寶元年,改為臨封郡。乾元元年,復為封州。舊領縣四,戶二千五百五十五,口一萬三千四百七十七。天寶領縣二,戶三千九百,口一萬一千八百二十七。東北至廣州九十五里,西北至梧州五十五里,東至康州一百三十里,北至賀州三百六十六里。至京師水陸四千五百
 一十里也。



 封川州所治。漢廣信縣地,屬蒼梧郡。在封水之陽。梁置梁信郡。隋平陳,改為成州。,又改為封州。隋末,州廢為封川縣,屬蒼梧郡。武德初,置封州。隋移州於封川口,即今縣治也



 開建漢封陽縣地,屬蒼梧郡,隋舊也。



 瀧州隋永熙郡之瀧水縣。武德四年,平蕭銑,置瀧州。天寶元年,改為開陽郡。乾元元年,復為瀧州。舊領縣四,戶三千六百二十七,口九千四百三十九。天寶領縣五。



 瀧水州所治。漢端溪縣地,屬蒼梧郡。晉分端溪立龍鄉,即今州治。梁分廣熙郡置建州,又分建州之雙頭洞立雙州。隋改龍鄉為平原縣,又改為瀧水



 開陽隋廢縣。武德四年,分瀧水置



 永寧武德四年,於安遂縣置藥州,領安遂、永寧、安南、永業四縣。貞觀中,廢藥州,以永寧屬瀧州。本隋永熙縣,武德五年,改為永寧縣



 鎮南隋安南縣。至德二年九月,改為鎮南



 建水新置。



 恩州隋高涼郡。武德四年,平蕭銑,置高州都督府,管高、春、羅、辯、雷、崖、儋、新八州。七年,割崖、儋、雷、新屬廣州。貞觀二十三年,廢高州都督府,置恩州。天寶元年,改為恩平郡。乾元元年,復為恩州,內有清海軍,管戍兵三千人也。領縣三,戶九千,無口數。至京師東南六千五百里。西北六十里接廣州界。



 恩平州所治。漢合浦郡也,隋置海安縣。武德五年,改為齊安。至德二年九月,改為恩平也



 杜陵隋杜原縣。
 武德五年,改為杜陵也



 陽江隋舊置也。



 春州隋高涼郡之陽春縣。武德四年,平蕭銑,置春州。天寶元年,改為南陵郡。乾元元年,復為春州。舊領一,戶五千七百一十四,口二萬一千六十一。天寶領縣二,戶一萬一千二百一十八。至京師東南六千四百四十八里。東至廣州六百四十二里,南至恩州九十三里,西至高州三百三十里,東北至新州二百六十里,西北至瀧州界也。



 陽春州所治。漢高涼縣地,屬合浦郡,至隋不改也



 羅水天寶後置。



 高州隋高涼郡。舊治高涼縣,後改為西平縣。貞觀二十三年,分西平、杜陵置恩州,高州移治良德縣。天寶元年,改為高涼郡。乾元元年,復為高州。領縣三,戶一萬二千四百。西北至竇州九十二里,北至瀧州界三百五十里,西南至潘州九十里,東至春州三百三十里。至京師六千二百六十二里,至東都五千五百二十里。



 良德漢合浦縣地,屬合浦郡。吳置高涼郡,宋、齊不改



 電白梁置電白郡,隋改為縣也



 保定舊保安縣。至德二年,改為保定。



 藤州下隋永平郡。武德四年,置藤州,領永平、猛陵、安基、武林、隋建、陽安、普寧、戎城、寧人、淳人、大賓、賀川十二縣。貞觀七年,以武林屬龔州、安普屬燕州、普寧屬容州。八年,以猛陵屬梧州。十二年,以隋建屬龔州。天寶元年,改為感義郡。乾元元年,復為藤州也。舊領縣六,戶九千
 二百三十六,口一萬三百七十二。天寶領縣三,戶三千九百八十。至京師五千五百九十六里,至東都五千二百里。南至義州二百里,西至龔州一百四十九里,北至梧州九十七里。



 鐔津漢猛陵縣,屬蒼梧郡。晉置永平郡。隋置藤州及鐔津。



 感義



 義昌本安昌縣。至德二年九月,改為義昌。



 義州下隋永熙郡之永業縣。武德五年,置南義州及
 四縣。貞觀元年,州廢,以所領縣入南建州。二年,復置義州,還以故縣來屬。五年,廢義州,縣屬南建州。六年,復置義州。又改縣來屬。天寶元年,改為連城郡。乾元元年,復為義州。舊領縣四,戶三千二百二十五,無口。天寶領縣三,戶一千一百一十,口七千三百三。至京師五千七百五十里,至東都四六百九十里,東至梧州隔鄣嶺一百七十里,北至藤州二百里,西至容州九十里,東南至竇州一百七十二里,東北至瀧州二百七里。



 嶺溪州所治。漢猛城縣,屬蒼梧郡。武德四年,置龍城縣,置南義州。貞觀初廢,二年復置義州,領龍城、安義、連城、義城四縣。至德中,改安義為永業,龍城為嶺溪



 永業舊安義縣,至德年改



 連城武德五年,分瀧州之正義縣置。



 竇州下隋永熙郡懷德縣。武德四年,置南扶州及五縣。以獠反寄瀧州。貞觀元年廢,以所管縣並屬瀧州。二年,獠平,復置南扶州,自瀧州還其故縣。五年復廢,縣隸
 瀧州。六年復置,以故縣來屬。其年,改南扶為竇州。天寶元年,改為懷德郡。乾元元年,復為竇州。舊領縣五,戶三千五百五十。天寶領縣四,戶一千一十九。至京師水陸六千一百二里,至東都水陸五千四百里。西至容州二百里,東至瀧州一百八十里,南至潘州一百五十里,東南至高州九十二里,北至義州二百三十里,西南至禺州一百九十里。



 信義漢端溪縣地,屬蒼梧。隋為懷德縣。武德四年,析
 懷德縣置信義縣,仍置南扶州。貞觀中,改為寶州,取州界有羅竇洞為名也



 懷德本屬瀧州,後來屬也



 潭峨武德四年,分信義縣置也



 特亮武德四年,分信義置也。



 勤州隋信安郡之高梁縣地。武德四年,置勤州,隸南康州總管。九年,改隸廣州,其年廢,縣屬春州。後置勤州,以銅陵來屬。仍析置富林縣。領縣三,戶六百八十二,口一千九百三十三。至京師五千三百九十里,至東都五
 千里。東至新州一百七十里,西至瀧州二百六十里,南至廣州六百三十五里,西北至康州二百七十三里。



 富林州所治,析銅陵置



 銅陵漢臨允縣地,屬合浦郡。宋立瀧潭縣。隋改為銅陵,以界內有銅山也。



 桂管十五州在廣州西。



 桂州下都督府隋始安郡。武德四年,平蕭銑,置桂州總管府,管桂、象、靜、融、賀、樂、荔、南昆、龍九州,並定州一總管。其桂州領始安、福祿、純化、興安、臨源、永福、陽朔、歸義、
 宣風、象十縣。尋改定州為南尹州。其年,又置欽州總管,隸桂府。五年,置南恭、燕、梧三州,隸桂府。九年,置晏州,隸桂府。貞觀元年,以欽、玉、南亭三州隸桂府。二年,省玉州、南亭州。五年,置賓州,隸桂府。六年,又以尹、藤、越、白、相、繡、鬱、姜、南宕、南方、南簡、南晉十二州隸桂府。其年,置龔州都督,亦隸桂府。其年,廢龍、鬱二州。八年,改越州為廉州,南簡為橫州,南方為澄州,南宕為潘州,南晉為邕州,尹州為貴州,靜州為富州,樂州為昭州,南昆為柳州,銅州
 為容州。廢福祿、歸義二縣。十年,廢姜州。十二年,廢晏州,以建陵縣來屬。廢荔州,以荔浦、崇仁二縣來屬。省宣風縣。今督桂、昭、賀、富、梧、藤、容、潘、白、廉、繡、欽、橫、邕、融、柳、貴十七州。天寶元年,改為始安郡,依舊都督府。至德二年九月,改為建陵郡。乾元元年,復為桂州,刺史充經略軍使,管戍兵千人,衣糧稅本管自給也。舊領縣十,戶三萬二千七百八十一,口五萬六千五百二十六。天寶領戶一萬七千五百,口七萬一千一十八。至京師水陸路四千
 七百六十里,至東都水陸路四千四十里。東至道州五百里,西至容州四百九十三里,南至昭州二百一十里,北至邵州六百八十五里,東南至賀州五百三十里,西南至柳州八百里,東北至永州五百五十里。



 臨桂州所治。漢始安縣地,屬零陵郡。吳分置始安郡,宋改為始建國,南齊始安郡,梁置桂州。隋末,復為始安郡。江源多桂,不生雜木,故秦時立為桂林郡也



 理定漢始安縣。隋分置興安,近改為理定



 靈川武德
 四年,分始安置



 陽朔隋舊。貞觀元年,廢歸義縣並入



 荔浦漢縣,屬蒼梧郡。武德四年,置荔浦、建陵、隋化、崇仁、純義。五年,以隋化屬南恭州。貞觀元年,以建陵屬晏州。十三年,廢荔州,以荔浦、崇仁屬桂州,純義屬蒙州也



 豐水舊永豐縣。元和初,改為豐水縣



 修仁隋置建陵縣。貞觀元年,於縣置晏州,領武龍、建陵二縣。十二年,廢晏州及武龍縣,以建陵屬桂州。長慶元年,改為修仁縣



 恭化武德四年,分始安置純化縣。元
 和初,改為恭化也



 永福武德四年,分始安置



 臨源武德四年,分始安置



 全義新置。



 昭州隋始安郡之平樂縣。武德四年,平蕭銑,置樂州,領平樂、永豐、恭城、沙亭四縣,貞觀七年,省沙亭縣。八年,改為昭州,以昭岡潭為名。天寶元年,改為平樂郡。乾元元年,復為昭州也。舊領縣三,戶四千九百一十八,口一萬二千六百九十一。天寶,戶三千五百。至京師四千四百三十六里,至東都四千二百一十九里。西至桂州二
 百二十里,東北至道州四百里,北至永州六百三十九里,南至富州一百六十六里也。



 平樂州所治。漢荔浦地,屬蒼梧郡。晉置平樂縣。貞觀七年,省沙亭並入也。



 恭城武德四年,析平樂置。



 永平隋縣,舊屬藤州。



 富州下隋始安郡之龍平縣。武德四年,平蕭銑,置靜州,領龍平、博勞、歸化、安樂、開江、豪靜、蒼梧七縣。尋又分蒼梧、豪靜、開江三縣置梧州。九年,省安樂縣。貞觀八
 年,改為富州,以富川水為名。天寶元年,改為開江郡。乾元元年,復為富州。舊領縣三,戶三千三百四十九,口四千三百一十九。天寶,戶一千二百九十。至京師五千一百三十里,至東都四千八百五十里。西北至桂州界八十里,東南至梧州界九十里,北至昭州一百六十六里。



 龍平漢臨賀縣地,屬蒼梧郡。吳置臨賀郡,梁分臨賀置南靜郡,又改為靜州,改南靜郡為龍平縣。貞觀八年,改為富州,以富川水為名也



 思勤新置



 馬江隋開
 江縣。長慶元年,改為馬江。皆漢臨賀縣地。



 梧州下隋蒼梧郡。武德四年,平蕭銑,置梧州,領蒼梧、豪靜、開江三縣。貞觀八年,割藤州之孟陵、賀州之綏越來屬。十三年,廢豪靜縣。天寶元年,改為蒼梧郡。乾元元年,復為梧州也。舊領縣四,戶三千八十四,口五千四百二十三。天寶領縣三,戶五千。至京師五千五百里,至東都五千一百里。東至封州八十里,東北至賀州四百一十里,北接富州界,正西至藤州一百九十里。



 蒼梧漢蒼梧郡,治廣信縣,即今治。隋立蒼梧縣,於此置郡



 戎城隋縣舊屬藤州,今來屬



 孟陵漢猛陵縣,屬蒼梧郡。



 蒙州隋始安郡之隋化縣。武德四年,置南恭州。割荔州之立山、東區、純義三縣分置嶺政縣。貞觀八年,改為蒙州,取州東蒙山為名。十二年,省嶺政入立山。天寶元年,改為蒙山郡。乾元元年,復為蒙州。舊領縣三,戶一千六十九。天寶,戶一千五十九。至京師五千一百里,至東
 都四千七百里。南至桂州二百四十九里,東至富州九十七里,西南至象州一百七十六里。



 立山州所治。漢荔浦縣,屬蒼梧郡。隋分荔浦置隋化縣。武德四年,改為立山,於縣置荔州,尋改為恭州。貞觀八年,改為蒙州。州東蒙山,山下有蒙水,居人多姓蒙故也



 東區武德五年,分立山置,屬荔州。貞觀六年,屬燕州。十年,改為蒙州



 正義貞觀五年,置純義縣,屬荔州。乾元初改為正義也。



 龔州下隋永平郡之武林縣。貞觀三年,置燕州。七年,移燕州於今州東。仍於燕州之舊所置龔州都督府,督龔、潯、蒙、賓、澄、燕七州。割藤州之武林、燕州之泰川來屬。又立平南、西平、歸政、大同四縣。十二年,廢潯州,以桂平、陵江、大賓、皇化四縣來屬。其年,省泰川入平南,省陵江入桂平,省歸政入西平。又割藤州之隋建來屬。天寶元年,改為臨江郡。乾元元年,復為龔州。舊領縣八,戶一萬三千八百二十一,口一萬一千一百二十八。天寶領縣
 六,戶九千,口二萬一千。至京師五千七百二十里,至東都五千三百六十一里。東至藤州一百四十九里,南至繡州九十五里,西至潯州一百三十里,北至蒙州二百四十里。



 平南州所治。漢猛陵縣地,屬蒼梧郡。晉分蒼梧置永平郡,仍置武林縣。貞觀七年,分置平南縣。後自武林移龔州治於此也



 武林猛陵縣地。隋分置武林縣,屬藤州。貞觀七年,屬龔州



 隋建猛陵縣地。武德年,屬藤
 州。貞觀年,屬龔州也



 大同貞觀元年分置



 陽川本陽建縣,後改為陽川也。



 潯州下隋鬱林郡之桂平縣。貞觀七年,置潯州,領桂平、陵江、大賓、皇化四縣。十二年,廢潯州,以四縣屬龔州。後復置潯州,以桂平、大賓、皇化來屬,又省陵江入桂平。天寶元年,改為潯江郡。乾元元年,復為潯州也。舊領縣三,戶二千五百,口六千八百三十六。至京師五千九百六十里,至東都五千七百里。東至龔州一百三十里,西
 至潘州二百五十里,西南至貴州一百五十里,西北至蒙州三百六十里。西南接鬱林州界。



 桂平漢布山縣,鬱林郡所治也。隋為桂平縣。武德年,屬貴州。貞觀初,屬燕州。七年,屬潯州。十二年,州廢,屬龔州。復置潯州



 皇化漢阿林縣,屬鬱林郡。隋置皇化縣,後廢。貞觀六年,復置,屬潯州。州廢,屬龔州。又復屬潯州。



 鬱林州下隋鬱林郡之石南縣。貞觀中置鬱林州,領
 石南、興德。天寶元年,改為鬱林郡。乾元元年,復為鬱林州也。領縣五,戶一千九百一十八,口九千六百九十九。至京師五千五百七里,至東都五千一百六十里。東至平琴州九十里,南至牢州一百二里,西南至昭州一百一十里,北至貴州一百五十里。



 石南州所治。漢鬱林郡之地。梁置定州,隋改尹州,煬帝為鬱林郡,皆治於此。陳時置石南郡,隋改為縣也



 鬱林隋縣,屬貴州,後來屬



 興業興德武德四年,
 分鬱林置



 潭慄



 平琴州下漢鬱林郡地。唐置平琴州,無年月。領縣四。天寶元年,改為平琴郡。乾元元年,復為州。建中並入黨州。今存。領縣四,戶一千一百七十四。至京師六千四百八十里,至東都五千八百三十里。西至鬱林州九十里,東南至牢州一百一十里,北至貴州一百五十里,北至繡州九十二里,東至黨州二十二里。



 容山州所治。本名安仁,至德年改也



 懷義福陽
 古符三縣與州同置。



 賓州下隋鬱林郡之嶺方縣。貞觀五年,析南方州之嶺方、思乾、瑯邪、南尹州之安城置賓州。十二年省思乾縣。天寶元年,改為安城郡。至德二年九月,改為嶺方郡。乾元元年,復為賓州。舊領縣三,戶七千四百八十五。天寶,戶一千九百七十六,口八千五百八十。至京師四千三百里,至東都四千一百里。南至淳州二百里,東南至貴州一百七十里,西至邕州二百五十七里,東南至蒙州
 三百二十里,西北至澄州一百二十里也。



 嶺方漢縣,屬鬱林郡。武德四年,屬南方州。貞觀五年,改為賓州



 瑯邪武德四年,析嶺方縣置



 保城梁置安城縣。至德二年,改為保城也。



 澄州下隋鬱林郡之嶺方縣地。武德四年,平蕭銑,置南方州,領無虞、瑯邪、思乾、上林、止戈、賀水、嶺方七縣。貞觀五年,以上林、止戈、瑯邪、嶺方屬賓州。八年,改南方州為澄州。天寶元年,改為賀水郡。乾元元年,復為澄州。舊領
 縣四,戶一萬八百六十八。天寶後,戶一千三百六十八,口八千五百八十。至京師四千六百里,至東都四千三百里。南至邕州三百里,北至竇州四百三十里,東南至賓州一百二十里,西至古州五百七十九里。



 上林州所治。漢嶺方縣地。武德四年,析置上林縣也



 無虞武德四年,析嶺方置



 賀水武德四年,析柳州馬平縣置。



 繡州下隋鬱林郡之阿林縣。武德四年,置林州,領常
 林、阿林、皇化、歸誠、羅繡、盧越等縣。六年,改為繡州。貞觀六年,省歸誠、盧越。七年,以皇化屬潯州。天寶元年,改為常林郡。乾元元年,復為繡州,領縣三,戶九千七百七十三。至京師六千九十里,至東都五千五百里。南至黨州五十里,北至貴州一百里也。



 常林漢阿林縣地,屬鬱林郡。武德四年,析貴州之鬱平縣,置林州及常林縣。貞觀六年,省歸誠縣入常林縣,移治廢歸誠縣故城。又改林州為繡州



 阿
 林漢縣,屬鬱林郡



 羅繡武德四年,析阿林置。



 象州下隋始安郡之桂林縣。武德四年,平蕭銑,置象州,領陽壽、西寧、桂林、武仙、武德五縣。貞觀十二年,省西寧縣,割廢晏州武化、長風來屬。天寶元年,改為象山郡。乾元元年,復為象州。舊領縣六,戶一萬一千八百四十五,口一萬二千五百二十一。天寶領縣三,戶五千五百,口一萬八百九十。至京師四千九百八十九里。北至桂州四百里,東至象州一百七十六里,南至費州三百里,
 西北至柳州二百里,東南至潯州三百六十里,西南至嚴州二百九十里也。



 武化州所治。漢潭中縣地,屬鬱林郡。隋建陵縣,屬桂州。武德四年,析建陵置武化縣,屬晏州。貞觀十二年,廢晏州來屬,仍自武德縣移象州於縣置。非秦之象郡,秦象郡今合浦縣



 武德漢中留縣地,屬鬱林郡。吳於縣置鬱林郡,仍分中留置桂林縣。武德四年,改為武德,於縣界置象州



 陽壽隋縣。武仙武德四年,析
 桂林置。



 柳州隋始安郡之馬平縣。武德四年,平蕭銑,置昆州,領馬平、新平、文安、賀水、歸德五縣。其年,改歸德為修德,改文安為樂沙,仍加昆州為南昆州。八年,以賀水屬澄州。貞觀七年,省樂沙入新平縣,以廢龍州之龍城來屬。八年,改南昆為柳州。九年,置崖山縣。十二年,省新平入馬平。天寶元年,改為龍城郡。乾元元年,復為柳州,以州界柳嶺為名。舊領縣四,戶六千六百七十四,口七千六
 百三十七。天寶領縣五,戶二千二百三十二,口一萬一千五百五十。至京師水陸相乘五千四百七十里,至東都水陸相乘五千六百里。東至桂州四百七里,至粵州二百九十里,北至融州二十里,東南至象州二百里,北至柳州三十里。



 馬平州所治。漢潭中縣地,屬鬱林郡。隋置馬平縣。武德四年,於縣置昆州,又改為柳州也



 龍城隋縣。武德四年,置龍州,領龍城、柳嶺二縣。貞觀七年,廢龍州,省
 柳嶺縣



 象貞觀中置



 洛曹舊洛封縣,元和十三年改



 洛容皆漢潭中地。貞觀後析置。



 融州下隋始安郡之義熙縣。武德四年,平蕭銑,置融州,復開皇舊名,領義熙、臨牂黃水、安修四縣。六年,改義熙為融水。貞觀十三年,省安修入臨牂。天寶元年,改為融水郡。乾元元年,復為融州。舊領縣三,戶二千七百九十四,口三千三百三十五。天寶,戶一千二百三十二。至京師五千二百七十里,至東都四千四百七十里。東至
 桂州四百九十一里,南至柳州三十里,至武零山二百里也。



 融水漢潭中地,與柳州同。隋置義熙縣。武德四年,改為融水,州所治也



 武陽舊黃水、臨牂二縣。析融水置。後並入,改為武陽。



 邕管十州在桂府西南。



 邕州下都督府隋鬱林郡之宣化縣。武德五年,置南晉州,領宣化一縣。貞觀六年,改為邕州都督府。天寶元
 年改為朗寧郡。乾元元年復為邕州。上元後,置經略使,領邕、貴、黨、橫等州。後又罷。長慶二年六月,復置經略使,以刺史領之。刺史充經略使,管戍兵一千七百人,衣糧稅本管自給。舊領縣五,戶八千二百二十五。天寶後,戶二千八百九十三,口七千三百二。至京師五千六百里,至東都五千三百二十七里。東南至欽州三百五十里,東北至賓州二百五十里,西南至羈縻左州五百里。



 宣化州所治。漢嶺方縣地。屬鬱林郡。秦為桂林郡地。
 驩水在縣北,本牂柯河,俗呼鬱林江,即駱越水也,亦名溫水。古駱越地也



 武緣隋廢縣。武德五年復置也



 晉興晉於此置晉興郡,隋廢為縣



 朗寧武德五年分置



 思龍如和封陵三縣,開磎洞漸置也。



 貴州下隋鬱林郡。武德四年,平蕭銑,置南尹州總管府,管南尹、南晉、南簡、南方、白、藤、南容、越、繡九州。南尹州領鬱林、馬嶺、安城、鬱平、石南、桂平、嶺山、興德、潮水、懷澤十
 一縣。五年,以桂平屬燕州,嶺山屬南橫州。貞觀五年,以安城屬賓州。七年,罷都督府,九年,改南尹為貴州。天寶元年,改為懷澤郡。乾元元年,復為貴州也。舊領縣八,戶二萬八千九百三十,口三萬一千九百九十六。天寶後,領縣四,戶三千二十六,口九千三百。至京師五千三百八十里,至東都五千一百二十里。東至繡州一百里,南至鬱林州一百五十里,西至橫州二百里,北至象州三百里,西南至賓州九十四里,東北至潯州一百五十里。



 鬱平漢廣鬱縣地,屬鬱林郡。古西甌、駱越所居。後漢谷永為鬱林太守,降烏滸人十萬,開七縣,即此也,烏滸之俗:男女同川而浴;生首子食之,云宜弟;娶妻美讓兄;相習以鼻飲。秦平天下,始招慰之,置桂林郡。漢改為鬱林郡。地在廣州西南安南府之地,邕州所管郡縣是也。隋分鬱平縣。鬱江,在州東也



 懷澤宋廢縣。武德四年又置



 潮水武德四年分鬱林置



 義山新置。



 黨州下古西甌所居。秦置桂林郡,漢為鬱林郡。唐置
 黨州,失起置年月。與平琴州同土俗。西至平琴治所二十二里。天寶元年,以黨州為寧仁郡。乾元元年,復為黨州。建中二年二月,廢平琴州並入。領縣四,戶一千三百,口七千四百。至京師地理,與平琴州同。南至牢州一百里,北至繡州五十里,東南至容州一百五十里,北接繡州界百餘里也。



 橫州下隋鬱林郡之寧浦縣。武德四年,置簡州,領寧浦、樂山、蒙澤、淳風、嶺山五縣。六年,改為南簡州。貞觀八
 年,改橫州。天寶元年,改為寧浦郡。乾元元年,復為橫州也。舊領縣四,戶一千一百二十八,口一萬七百三十四。天寶領縣三,戶一千九百七十八,口八千三百四十二。至京師五千五百三十九里,至東都四千七百五里。南至欽州三百五十里,西至巒州一百五十里,北至貴州一百六十里也。



 寧浦州所治。漢廣鬱縣地,屬鬱林郡。吳分置寧浦郡,晉、宋、齊不改。梁分置簡陽郡。隋平陳,郡並廢,置簡
 州,又改為緣州。煬帝廢州,置寧浦縣,鬱林郡。武德復置,改為橫州。



 從化漢高涼縣地,屬合浦郡。武德四年,分寧浦置淳風縣。貞觀元年,改為從化也



 樂山漢高涼縣地,隋置樂山縣。



 田州土地與邕州同,失廢置年月,疑是開元中置。天寶元年,改為橫山郡。乾元元年復為田州。舊領縣五,戶四千一百六十八。舊圖無四至州郡及兩京道里數。



 都救惠佳武籠橫山如賴並與州同置也。



 嚴州秦桂林郡地,後為獠所據。乾封元年,招致生獠,置嚴州及三縣。天寶元年,改為修德郡。乾元元年,復為嚴州。領縣三,戶一千八百五十九,口七千五十一。至京師五千三百二十七里,至東都四千八百九十三里。東北至柳州二百四十里,東南接象州界,西北接澄州界也。



 來賓州所治也。



 循德歸化,與州同置。



 山州失起置年月。天寶元年,改為龍池郡。乾元元年,
 復為山州。領縣二,戶一千三百二十。無四至及京洛里數。



 龍池州所治也



 盆山



 巒州秦桂林郡。唐置淳化,失起置年月。天寶元年,改為永定縣。乾元元年,復為淳州。永貞元年,改為巒州也。領縣三,戶七百七十,口三千八百三。至京師五千三百里,至東都四千九百里。南至橫州一百四十里,西至邕州三百里,北至賓州二百五十五里。



 永定州所治也



 武羅靈竹二縣與州同置。



 羅州隋高涼郡之石龍縣地。武德五年,於縣置羅州,領石龍、吳川、陵羅、龍化、羅辯、南河、石城、招義、零綠、慈廉、羅肥十一縣。六年,移羅州於石城縣,於舊所置南石州,割石龍、陵羅、龍化、羅辯、慈廉、羅肥屬南石州。天寶元年,改羅州為招義郡。乾元元年,復為羅州。舊領縣五,戶五千四百六十,口八千四十一。至京師六千五百二十二里,至東都五千七百五里。東至大海一百三十九里,南至
 雷州二百五十里,西至廉州二百五十里,北至辯州一百五十里,西南至零綠縣大海一百二十里,西北至白州二百三十里,東北至新州五十里。



 石城州所治。漢合浦郡地。宋將檀道濟於陵羅江口築石城,因置羅州,屬高涼郡。唐復置羅州於縣



 吳川隋縣



 招義武德五年析石龍縣置也



 南河武德五年析石龍縣置也。



 潘州下隋合浦郡之定川縣。武德四年,置南宕州,領
 南昌、定川、陸川、思城、溫水、宕川六縣,治南昌縣。貞觀六年,移治定川。八年,改為潘州,仍廢思城縣。天寶元年,改為南潘郡。乾元元年,復為潘州也。舊領縣五,戶一萬七百四十八。天寶後,領縣三,戶四千三百,口八千九百六十七。至西京七千一百六十一里,至東都六千三百八十九里。至高州九十里,南至大海五十六里,至辯州一百二十里,北至竇州一百五十一里。



 茂名州所治。古西甌、駱越地,秦屬桂林郡,漢為合浦
 郡之地。隋置定川縣。武德四年,平嶺表,於縣置南宕州,改為潘州,仍改縣茂名也



 南巴隋廢縣。武德五年置



 潘水以縣水為名。武德五年分置也。



 容管十州在桂管西南



 容州下都督府隋合浦郡之北流縣。武德四年,平蕭銑,置銅州,領北流、豪石、宕昌、渭龍、南流、陵城、普寧、新安八縣。貞觀元年,改為容州,以容山為名。十一年,省新安縣。開元中,升為都督府。天寶元年,改為普寧郡。乾元元
 年,復為容州都督府。仍舊置防禦、經略、招討等使,以刺史領之。刺史充經略軍使,管鎮兵一千一百人,衣糧稅本管自給。舊領縣七,戶八千八百九十。天寶後,領縣五,戶四千九百七十,口一萬七千八十七。至京師五千九百一十里,至東都五千四百八十五里。東至藤州二百五十九里,南至竇州二百里,西至禺州十五里,北至龔州二百里,西至隋建縣一百九十里,西北至黨州一百五十里,東北接義州界。



 北流州所治。漢合浦縣地,隋置北流縣。縣南三十里,有兩石相對,其間闊三十步,俗號鬼門關。漢伏波將軍馬援討林邑蠻,路由於此,立碑石龜尚在。昔時趨交趾,皆由此關。其南尤多瘴癘,去者罕得生還,諺曰:「鬼門關,十人九不還。」其土少鐵,以睟石燒為器,以烹魚鮭,北人名「五侯燋石。」一經火,久之不冷,即今之滑石也,亦名冷石



 普寧隋置



 陵城武德四年,析北流置



 渭龍武德四年,析普寧置



 欣道新置。



 辯州下隋高涼郡之石龍縣。武德五年,置羅州,移治石城。於舊所置南石州,領石龍、陵羅、龍化、羅辯、慈廉、羅肥六縣。貞觀九年,改南石州為辯州,省慈廉、羅肥二縣。天寶元年,改陵水郡。乾元元年復為辯州也。舊領縣四,戶一萬三百五十。天寶後,領縣三,戶四千八百五十八,口一萬六千二百九。至京師五千七百一十八里,至東都五千三百七十里。東至廣州一千一百四十四里,南至羅州吳川縣界五十里,南至白州博白縣二百三十
 里,北至禺州三百八十二里,南至潘州四十里,西南至羅州一百五十里,西北至白州三百里。



 石龍州所治。漢高涼縣地,屬合浦郡。秦象郡地。武德五年屬羅州,六年改屬辯州。



 陵羅武德五年,置羅州。六年,改為南石州也



 龍化武德五年,分置也。



 白州下隋合浦郡之合浦縣地。武德四年,置南州,領博白、朗平、周羅、龍豪、淳良、建寧六縣。六年,改為白州。貞觀十二年,省朗平、淳良二縣。天寶元年,改為南昌郡。乾
 元元年,復為白州。舊領縣四,戶八千二百六。天寶領縣五,戶二千五百七十四,口九千四百九十八。至京師六千一百七十五里,至東都五千九百一十九里。東至辯州二百里,南至羅州二百二十里,西至州界朗平山八十里,北至牢州一百里,西南至廣州二百里,東北至禺州二百里。



 博白州所治。漢合浦縣地,屬合浦郡。武德五年,析合浦縣置博白縣也



 建寧武德四年,析合浦縣置。貞觀十二年,省淳良並
 入



 周羅武德四年,析合浦置



 龍豪武德四年,析合浦置



 南昌隋縣。舊屬潘州,又來屬也。



 牢州下本巴、蜀徼外蠻夷地,漢牂柯郡地。武德二年,置義州。五年,改為智州。貞觀十二年,改為牢州,以牢石為名。天寶元年,改為定川郡。乾元元年復為牢州也。舊領縣三,戶一千六百四十一,口一萬一千七百五十六。去京師與容州道里同。東至容州一百二十五里,南至白州一百里,西至隋林州一百一十里,北至黨州一百
 里。



 南流武德四年,析容州北流縣置,屬容州。貞觀十一年,改智州為牢州,以牢石為名。牢石高四十丈,周二十里,在州界也。定川



 宕川貞觀十一年,分南流置也。



 欽州下隋寧越郡。武德四年,平蕭銑,改為欽州總管府,管一州,領欽江、安京、南賓、遵化、內亭五縣。五年,置如和縣。其年,置玉州、南亭州,並隸欽府,以內亭、遵化二
 縣屬亭州。貞觀元年,罷都督府。二年,廢亭州,復以內亭、遵化並來屬。十年,省海平縣。天寶元年,改為寧越郡。乾元元年,復為欽州也。舊領縣七,戶一萬四千七十二,口一萬八千一百二十七。天寶領縣五,戶二千七百,口一萬一百四十六。至京師五千二百五十一里。東至嚴州四百里,南至大海二百五十里,西至瀼州六百三十里,至橫州三百五十里,東南至廣州七百里,西南至陸州六百里,西至容州三百五十里,東北至貴州四百里。



 欽江州所治。漢合浦縣地,宋分置寧壽郡及寧壽縣。梁置安州,隋改為欽州,仍改宋壽縣為欽江。煬帝改為寧越郡。皆治欽江也



 保京隋安京縣。至德二年,改為保京。縣北十里安京山,下有如和山,似循州羅浮山形勢



 遵化隋舊置



 內亭隋縣。武德五年,於縣置南亭州。貞觀元年,州廢,復屬欽州也



 靈山已上縣,並漢合浦縣也。



 禺州隋合浦郡之定川縣。武德四年,置南宕州,領南
 昌、定川、陸川、思城、溫水、宕川六縣,治南昌縣。貞觀六年,移治定川。八年,改為潘州,仍廢思城。總章元年改為東峨州,移治峨石縣。二年,改為禺州。天寶元年改為溫水郡。乾元元年復為禺州。舊領縣五,戶一萬七百四十八。天寶領縣四,戶三千一百八十。至京師五千三百五里,至東都五千里。至義州一百九十里,南至辯州三百里,西至白州二百里,北至容州一百一十里。



 峨石秦象郡地、晉南昌郡之邊邑,為禺州所治也



 溫水武德四年,析南昌置



 陸川隋廢縣。武德四年置



 扶桑武德四年置。



 湯州下秦象郡地。唐置湯州,失起置年月。天寶元年改為溫泉郡。乾元元年復為湯州也。領縣三,無戶口及無兩京道里、四至州府。



 湯泉州所治也



 淥水羅韶與州同置。



 瀼州下貞觀十二年,清平公李弘節遣欽州首領寧師京,尋劉方故道,行達交趾,開拓夷獠,置瀼州。天寶元
 年,改為臨潭郡。乾元元年,復為瀼州。領縣四,戶一千六百六十六,無兩京地里。東至欽州六百三十里,北至容州二百八十二里。在安南府之東北、鬱林之西南。



 臨江州所治也



 波零鵠山弘遠與州同置。



 巖州下土地與合浦郡同。唐置巖州,失起置年月。天寶元年,改為安樂郡。至德二年,改為常樂郡。乾元元年,復為巖州。領縣四,戶一千一百一十,無兩京道里、四至州府也。



 常樂本安樂縣。至德二年改,州所治



 思封高城石巖與州同置



 古州土地與瀼州同年置。天寶元年,改為樂古郡。乾元元年,復為古州。



 安南府在邕管之西



 安南都督府隋交趾郡。武德五年,改為交州總管府,管交、峰、愛、仙、鳶、宋、慈、險、道、龍十州。其交州領交趾、懷德、南定、宋平四縣。六年,澄、慈、道、宋並加「南」字。七年,又置玉
 州,隸交府。貞觀元年,省南宋州以宋平縣,省隆州以陸平縣,省鳶州以硃鳶縣,省龍州以龍編縣,並隸交府。仍省懷德縣及南慈州。二年,廢玉州入欽州。六年,改南道州為仙州。十一年,廢仙州,以平道縣來屬。今督交、峰、愛、驩四州。調露元年八月,改交州都督府為安南都護府。大足元年四月,置武安州、南登州,並隸安南府。至德二年九月,改為鎮南都護府,後為安南府。刺史充都護,管兵四千二百。舊領縣八,戶一萬七千五百二十三,口八萬八千
 七百八十八。天寶領縣七,戶二萬四千二百三十,口九萬九千六百五十二。至京師七千二百五十三里,至東都七千二百二十五里。西至愛州界小黃江口,水路四百一十六里,西南至長州界文陽縣靖江鎮一百五十里,西北至峰州嘉寧縣論江口水路一百五十里,東至硃鳶縣界小黃江口水路五百里,北至硃鳶州阿勞江口水路五百四十九里,北至武平縣界武定江二百五十二里,東北至交趾縣界福生去十里也。



 宋平漢西手卷音拳縣地,屬日南郡。自漢至晉猶為西手卷縣。宋置宋平郡及宋平縣。隋平陳,置交州。煬帝改為交趾,刺史治龍編,交州都護制諸蠻。其海南諸國,大抵在交州南及西南,居大海中州上,相去或三五百里,三五千里,遠者二三萬里。乘舶舉帆,道里不可詳知。自漢武已來朝貢,必由交趾之道。武德四年於宋平置宋州,領宋平、弘教、南定三縣。五年,又分宋平置交趾、懷德二縣。自貞觀元年廢南宋州,以弘教、懷德、交趾三縣省入
 宋平縣,移交趾縣名於漢故交趾城置。以宋平、南定二縣屬交州。交趾漢交趾郡之羸婁二字並音來口反地。隋為交趾縣,取漢郡名。武德四年,置慈廉、烏延、武立三縣。六年,改為南慈州。貞觀初,州廢,並廢三縣,並入交趾



 硃鳶漢縣名,交趾郡。今縣,吳軍平縣地。舊置武平郡。龍編漢交趾郡守治羸婁。後漢周敞為交趾太守。乃移治龍編。言立城之始,有蛟龍盤編津之間,因為城名。武德四年於縣置龍州,領龍編、武寧、平樂三縣。貞觀初廢龍州,
 以武寧、平樂入龍編,割屬仙州。十年,廢仙州,以龍編屬交州也



 平道漢封溪縣地,南齊置昌國縣。《南越志》:交趾之地,最為膏腴。舊有君長曰雄王,其佐曰雄侯。後蜀王將兵三萬討雄王,滅之。蜀以其子為安陽王,治交趾。其國地,在今平道縣東。其城九重,周九里,士庶蕃阜。尉佗在番禺,遣兵攻之。王有神弩,一發殺越軍萬人,趙佗乃與之和,仍以其子始為質。安陽王以媚珠妻之,子始得弩毀之。越兵至,乃殺安陽王,兼其地。武德四年於
 縣置道州,領平道、昌國、武平三縣。六年,改為南道州,又改為仙州。貞觀十年廢仙州,以昌國入平道,屬交州



 武平吳置武平郡。隋為縣。本漢封溪縣。後漢初,赩泠縣女子徵側叛,攻陷交趾,馬援率師討之,三年方平。光武乃增置望海、封溪二縣,即此也。隋曰隆平。武德四年,改為武平



 太平



 武峨州下土地與交州同。置武峨州,失起置年月。天寶元年,改為武峨郡。乾元元年,復為武峨州。領縣五,戶
 一千八百五十,無口。無兩京道里及四至州府也。



 武峨州所治也



 武緣武勞梁山皆與州同置也



 如馬



 粵州下土地與交州同。唐置粵州,失起置年月。天寶元年,改為龍水郡。乾元元年,復為粵州。領縣四,無戶口數,亦無兩京道里及四至州府也。



 龍水州所治也。崖山、東璽、天河皆與州同置。



 芝州下土地與交州同。唐置芝州,失起置年月。天寶元年,改為忻城郡。乾元元年,復為芝州。領縣一。



 忻城州所治。無戶口及兩京道里、四至州府。最遠惡處。



 愛州隋九真郡。武德五年,置愛州,領九真、松源、楊山、安順四縣。又於州界分置積、順、安、永、胥、前真、山七州。改永州為都州。九年,改積州為南陵州。貞觀初,廢都州入前真州。其年,廢前真、胥二州入南陵州。又廢安州以隆安縣,廢山州以建初縣,並屬州。又廢楊山、安順二縣入九真
 縣。改南陵州復為真州。八年,廢建初入隆安。九年,廢松源入九真。十年,廢真州,以胥浦、軍安、日南、移風四縣屬愛州。天寶元年,改為九真郡。乾元元年,復為愛州。九真南與日南接界,西接牂柯界,北與巴蜀接,東北與鬱林州接,山險溪洞所居。舊領縣七,戶九千八十,口三萬六千五百一十九。天寶領縣六,戶一萬四千七百。至京師八千八百里,至東都八千一百里。在交州西,不詳道里遠近。其南即驩州界。



 九真漢武帝開置九真郡,治於胥浦縣。領居風、都龐、餘發、咸驩、無切、無編等七縣。今九真縣,即漢居風縣地。吳改為移風。隋改為九真,州所治。自漢至南齊為九真郡。梁置愛州,隋為九真郡



 安順隋舊武德三年,置順州,又分置東河、建昌、邊河,並屬順州。州廢,及三縣皆並入安順,屬愛州也



 崇平隋隆安縣。武德五年,於縣置安州及山州,又分隆安立教山、建道、都握三縣,並屬安州,領四縣。又置岡山、真潤、古安、西安、建初五縣,屬山州。貞
 觀元年,廢安州及三縣,又廢山州及五縣,以隆安隸愛州。先天元年,改為崇安。至德二年,改為崇平



 軍寧隋軍安縣。武德五年,於縣界置永州。七年,改為都州。貞觀元年,改為前真州。十年,改屬愛州。至德二年,改為軍寧



 日南漢居風地。縣界有居風山,上有風門,常有風。其山出金牛,往往夜見,照耀十里。時鬥,則海水沸溢,有霹靂,人家牛皆怖,號曰「神牛」。隋為日南縣



 無編漢舊縣,屬九真郡。又有漢西於縣,故城在今縣東所置
 也。



 福祿州下土俗同九真郡之地,後為生獠所據。龍朔三年,智州刺史謝法成招慰生獠昆明、北樓等七千餘落。總章二年,置福祿州以處之。天寶元年,改為福祿郡。至德二年,改為唐林郡。乾元元年,復為福祿州。領縣二,無戶口及兩京道里、四至州郡。



 柔遠州所治。與州同置。本名安遠,至德二年,改為柔遠也



 唐林



 長州土俗與九真同。唐置長州,失起置年月。天寶元年,改為文陽郡。乾元元年,復為長州。領縣四,戶六百四十八,無口及兩京道里、四至州府也。



 文陽銅蔡長山其常皆與州同置。



 驩州陳日南郡。武德五年,置南德州總管府,領德、明、智、驩、林、源、景、海八州。南德州領六縣。八年,改為德州。貞觀初,改為驩州,以舊驩州為演州。二年,置驩州都督府,領驩、演、明、智、林、源、景、海八州。十二年,廢明、源、海三州。天
 寶元年,改為日南郡。乾元元年,復為驩州也。舊領縣六,戶六千五百七十九,口一萬六千六百八十九。天寶領縣四,戶九千六百一十九,口五萬八百一十八。至京師陸路一萬二千四百五十二里,水路一萬七千里,至東都一萬一千五百九十五里,水路一萬六千二百二十里。東至大海一百五十里,南至林州一百五十里,西至環王國界八百里,北至愛州界六百三里,南至盡當郡界四百里,西北到靈跋江四百七十里,東北至辯州五
 百二里。



 九德州所治。古越裳氏國,秦開百越,此為象郡。漢武元鼎六年開交趾已南,置南郡,治於硃吾,領比景、盧容、西手卷、象林五縣。吳分日南置九德郡,晉、宋、齊因之。隋改為驩州,廢九德郡為縣,今治也。後漢遣馬援討林邑蠻,援自交趾循海隅,開側道以避海,從蕩昌縣南至九真郡,自九真至其國,開陸路,至日南郡,又行四百餘里,至林邑國。又南行二千餘里,有西屠夷國,鑄二銅柱於象
 林南界,與西屠夷分境,以紀漢德之盛。其時,以不能還者數十人,留於其銅柱之下。至隋乃有三百餘家,南蠻呼為「馬留人」。其水路,自安南府南海行三千餘里至林邑,計交趾至銅柱五千里。



 浦陽晉置



 懷驩隋為咸驩縣,屬九真郡。武德五年,於縣置驩州,領安人、扶演、相景、西源四縣,治安人。貞觀九年,改為演州。十三年,省相景縣入扶演。十六年,廢演州。其所管四縣,廢入咸驩。後改為懷驩



 越裳吳置。武德五年,於縣置
 明州,析置萬安、明弘、明定三縣隸之。又分日南郡文谷、金寧二縣置智州,領文谷、新鎮、闍員、金寧四縣。貞觀十三年,廢明州,越裳屬智州。後又廢智州,以越裳屬驩州。



 林州隋林邑郡。貞觀九年,綏懷林邑置林州,寄治於驩州南界,今廢無名,領縣三,無戶口。去京師一萬二千里。



 林邑州所治。漢武帝開百越,於交趾郡南三千里置日南郡,領縣四,治於硃吾。其林邑,即日南郡之象林縣。
 縣在南,故曰日南,郡南界四百里。後漢時,中原喪亂,象林縣人區連殺縣令,自稱林邑王。後有範熊者,代區連,相傳累世,遂為林邑國。其地皆開北戶以向日。晉武時,範氏入貢。東晉末,範攻陷日南郡,告交州刺史硃蕃,求以日南郡北界橫山為界。其後,又陷九真郡。自是,屢寇交趾南界。至貞觀中,其主修職貢,乃於驩州南僑置林邑郡以羈縻之,非正林邑國



 金龍隋文帝時,遣大將劉方率兵萬人,自交趾南伐林邑國,敗之。其王梵志
 遁走,方收其廟主一十八人,皆鑄金為之。方盡虜其人,空其地,乃班師。因方得其龍,乃為縣名



 海界三縣並貞觀九年置。



 景州隋北景郡。貞觀二年,置南景州,寄治驩州南界。八年,改為景州。後亦廢,無其名。領縣三,無戶口。至京師一萬一千五百里。



 北景漢縣名,屬日南郡,在安南府南三千里。北景在南。晉將灌邃攻林邑王範佛,破其國,遂於其國五月五日立表,北景在表南九寸一分,故
 自北景已南,皆北戶以向日也。「北」字或單為「匕」



 由文貞觀二年置也



 硃吾漢日南郡所治之縣也。前志曰:「硃吾人不粒食,依魚資魚為生。」記云:「硃吾,在日南郡,此僑立名也。」



 峰州下隋交趾郡之嘉寧縣。武德四年,置峰州,領嘉寧、新昌、安仁、竹輅、石堤、封溪六縣。貞觀元年,廢石堤、封溪入嘉寧,竹輅入新昌。天寶元年,改為承化郡。乾元元年,復為峰州也。舊領縣三,戶五千四百四十四,口六千
 四百三十五。天寶領縣五,戶一千九百二十。州在安南府西北,至京師七千七百一十里。



 嘉寧州所治。漢赩泠縣地,屬交趾郡。古文朗夷之地。秦屬象郡。吳分交趾置新興郡。晉改為新昌。宋、齊因之,改為興州。隋初改為峰州。煬帝廢,並入交趾。武德復置峰州也



 承化新昌嵩山珠綠嵩山珠綠新置。



 陸州隋寧越郡之玉山縣。武德五年,置玉山州,領安海、
 海平二縣。貞觀二年,廢玉山州。上元二年,復置,改為陸州,以州界山為名。天寶元年,改為玉山郡。乾元元年,復為陸州。領縣三,戶四百九十四,口二千六百七十四。至京師七千二十六里,至東都七千里。東至廉州界三百里,南至大海,北至思州七百六十二里,東南際大海,西南至當州寧海二百四十里也。



 烏雷州所治也



 華清舊玉山縣,天寶年改



 寧海舊安海縣。至德二年改為寧海縣也。



 廉州下隋合浦郡。武德五年,置越州,領合浦、安昌、高城、大廉、大都五縣。貞觀六年,置珠池。其年,改大都屬白州。八年,改越州為廉州。十年,廢姜州,以封山、東羅、蔡龍三縣來屬。十二年,廢安昌、珠池二縣入合浦,廢高城入蔡龍。天寶元年,改為合浦郡。乾元元年,復為廉州。舊領縣五,戶一千五百二十二。天寶,戶三千三十二,口一萬三千二十九。至京師六千五百四十七里,至東都五千八百三十六里。東至白州二百里,南至羅州
 三百五十里,西北至安南府一千里,北至欽州七百里。



 合浦漢縣,屬合浦郡。秦之象郡地。吳改為珠官。宋分置臨漳郡及越州,領郡三,治於此。時西江都護陳伯紹為刺史,始立州鎮,鑿山為城,以威俚、獠。隋改為祿州。及為合州,又改為合浦。唐置廉州。大海,在西南一百六十里,有珠母海,郡人採珠之所,雲合浦也。州界有瘴江,名合浦江也



 封山隋縣。武德五年,置姜州,領封山、東羅、蔡龍三縣。貞觀十年,廢州,以三縣入廉州



 蔡龍
 武德五年,分置也



 大廉武德五年置。四縣皆漢合浦縣地。



 雷州下隋合浦郡之海康縣。武德四年,平蕭銑,置南合州,領海康、隋康、鐵杷、椹川四縣。貞觀元年,改為東合州。二年,改隋康為徐聞縣。八年,改東合州為雷州。天寶元年,改為海康郡。乾元元年,復為雷州也。舊領縣四,戶二千四百五十八。天寶領縣三,戶四千三百二十,口二萬五百七十二。至京師六千五百一十二里,至東都五
 千九百三十一里。東至大海二十里,西至大海一百里,東南至大海十五里,西南至大海一百里,隔海至崖州四百三十里,東北及西北與羅州接界。



 海康漢徐聞縣地,屬合浦郡。秦象郡地。梁分置南合州,隋去「南」字,煬帝廢合州,置海康縣



 遂溪舊齊鐵杷、椹川二縣,後廢,改為遂溪也



 徐聞漢縣名。隋置隋康縣。貞觀二年,改為徐聞。《漢志》曰合浦郡徐聞南入海,達珠崖郡,即此縣。



 籠州貞觀十二年,清平公李弘節遣龔州大同縣人龔固興招慰生蠻。置籠州。天寶元年,改為扶南郡。乾元元年,復為籠州。領縣七,戶三千六百六十七。無四至州縣、兩京道里。扶南國,在日南郡之南海西大島中,去日南郡約七千里,在林邑國西三千里,其王,貞觀中遣使朝貢,故立籠州招置之。遙取其名,非正扶南國也。



 武勒州所治



 武禮羅龍扶南龍賴武觀武江皆與州同置。



 環州下貞觀十二年,清平公李弘節開拓生蠻,置環州,以環國為名。天寶元年,改為正平郡。乾元元年,復為環州。領縣八,無戶口及兩京道里、並四至州府。



 正平州所治



 福零龍源饒勉思恩武石歌良蒙都與州同置。



 德化州永泰二年四月,於安南府西界、牂柯南界置。領縣二。



 德化、歸義與州同置。



 郎茫州永泰二年四月,於安南府西界置,領縣二。



 龍然福守與州同置。



 崖州下隋珠崖郡。武德四年平蕭銑,置崖州,領舍城、平昌、澄邁、顏羅、臨機五縣。貞觀元年置都督府,督崖、儋、振三州。其年,改顏羅為顏城,平昌為文昌。三年,割儋州屬廣府。五年,又置瓊州。十三年,廢瓊州,以臨機、容瓊、萬安三縣來屬。天寶元年改為珠崖郡。乾元元年復為崖州,在廣府東南。舊領縣七,戶六千六百四十六。天寶,戶十一鄉。至京師七千四百六十里,至東都六千三百里,廣府東南二
 千餘里。雷州徐聞縣南舟行,渡大海,四百三十里達崖州。漢武帝元封元年,遣使自徐聞南入海,得大洲,東西南北方一千里,略以為珠崖、儋耳二郡。民以布如單被,穿中從頭穿之。民種禾稻、糸寧麻,女子蠶織。無馬與虎,有牛、羊、豕、雞、犬。兵則矛、盾、木弓、竹矢、骨鏃。郡縣吏卒,多侵凌之,故率數歲一反。昭帝省儋耳,並珠崖。元帝用賈捐之之言,乃棄之。唐武德初,復析珠崖郡置崖、儋、瓊、振、萬安五州,於崖州同置都督府領之。後廢都督,隸廣州經
 略使。後又改隸安南都護府也。



 舍城州所治。隋舊縣。其崖、儋、振、瓊、萬安五州,都在海中洲之上,方千里,四面抵海。北渡海,揚帆一日一夜,至雷州也



 澄邁隋縣



 文昌武德五年置平昌縣。貞觀元年改為文昌。



 儋州下隋儋耳郡。武德五年置儋州,領義倫、昌化、感恩、富羅四縣。貞觀元年,分昌化置普安。天寶元年,改為昌化郡。乾元元年復為儋州也。舊領縣五,戶三千九百
 五十六。天寶,戶三千三百九。至京師七千四百四十二里。與崖州同在海中洲上,東至振州四百里。



 義倫本漢儋耳郡城,即此縣。隋為義倫縣,州所治也



 昌化隋縣



 感恩



 洛場新置



 富羅隋之毗善縣。武德五年,改置。



 瓊州本隋珠崖郡之瓊山縣。貞觀五年,置瓊州,領瓊山、萬安二縣。其年,又割崖州臨機來屬。十三年,廢瓊州,以屬崖州。尋復置瓊州,領瓊山、容瓊、曾口、樂會、顏羅五縣。天寶元
 年,改為瓊山郡。乾元元年,復為瓊州。貞觀五年十月,嶺南節度使李復奏曰:「瓊州本隸廣府管內,乾封年,山洞草賊反叛,遂茲淪陷,至今一百餘年。臣令判官姜孟京、崖州刺史張少逸,並力討除,今已收復舊城,且令降人權立城相保,以瓊州控壓賊洞,請升為下都督府,加瓊、崖、振、儋、萬安等五州招討游弈使。其崖州都督請停。」從之。領縣五,戶六百四十九。兩京與崖州道里相類。西南至振州四百五十里,與崖州同在大海中也。



 瓊山州所治。貞元七年十一月省容瓊縣並入。臨高本屬崖州,貞元七年割屬瓊州。曾口樂會顏羅後漸析置。



 振州隋臨振郡。武德五年置振州。天寶元年改為臨振郡。乾元元年,復為振州也。領縣四,戶八百一十九,口二千八百二十一。至京師八千六百六里,至東都七千七百九十七里。東至萬安州陵水縣一百六十里,南至大海,西北至儋州四百二十里,北至瓊州四百五十里,
 東南至大海二十七里,西南至大海千里,西北至延德縣九十里,與崖州同在大海洲中。



 寧遠州所治。隋舊



 延德隋縣



 吉陽貞觀二年,分延德置



 臨川隋縣



 落屯新置。



 萬安州與崖、儋同在大海洲中。唐置萬安州,失起置年月。天寶元年,改為萬安郡。至德二年改為萬全郡。乾元元年復為萬安州。領縣四,無戶口。西接振州界。兩京道里,與振州相類也。



 萬安州所治。至德二年,改為萬全,後復置



 陵水富云博遼與州同置。



 赤土國州南渡海,便風十四日,至雞籠島,即至其國。亦海中之一洲。



 丹丹國振州東南海中之一洲,舟行十日至。



\end{pinyinscope}