\article{卷一百八十八}

\begin{pinyinscope}

 ○楊思勖高力士李輔國程元振魚朝恩劉希暹賈明觀竇文場霍仙鳴俱文珍吐突承璀
 王守澄田令孜楊復光楊復恭



 唐制有內侍省,其官員:內侍四人;內常侍六人;內謁者監六人;內給事八人;謁者十二人;典引十八人;寺伯二人;寺人六人。別有五局:掖廷局掌宮人簿籍;宮闈局掌宮內門禁,其屬有掌扇、給使等員;奚官局掌宮人疾病死喪;內僕局掌宮中供帳燈燭;內府局主中藏給納。五局有令丞,皆內官為之。



 貞觀中,太宗定制,內侍省不置
 三品官,內侍是長官,階四品。至永淳末,向七十年,權未假於內官,但在閣門守御,黃衣廩食而已。則天稱制,二十年間,差增員位。中宗性慈,務崇恩貸,神龍中,宦官三千餘人,超授七品以上員外官者千餘人,然衣硃紫者尚寡。



 玄宗在位既久,崇重宮禁,中官稍稱旨者,即授三品、左右監門將軍,得門施棨戟。開元、天寶中,長安大內、大明、興慶三宮,皇子十宅院,皇孫百孫院。東都大內、上陽兩宮,大率宮女四萬人,品官黃衣已上三千人,衣硃
 紫者千餘人。後李輔國從幸靈武,程元振翼衛代宗,怙寵邀君,乃至守三公,封王爵,干預國政,亦未全握兵權。代宗時,子儀北伐,親王東討,遂特立觀軍容宣慰使,命魚朝恩為之,然自有統帥,亦監領而已。



 德宗避涇師之難,幸山南,內官竇文場、霍仙鳴擁從。賊平之後,不欲武臣典重兵,其左右神策、天威等軍,欲委宦者主之。乃置護軍中尉兩員、中護軍兩員,分掌禁兵,以文場、仙鳴為兩中尉,自是神策親軍之權,全歸於宦者矣。自貞元之
 後,威權日熾,蘭錡將臣,率皆子蓄;籓方戎帥,必以賄成;萬機之與奪任情,九重之廢立由己。元和之季,毒被乘輿。長慶纘隆,徒鬱枕干之憤;臨軒暇逸,旋忘塗地之冤。而易月未除,滔天盡怒。甲第名園之賜,莫匪伶官;硃袍紫綬之榮,無非巷伯。是時高品白身之數,四千六百一十八人,內則參秉戎權,外則臨監籓嶽。文宗包祖宗之恥,痛肘腋之仇,思翦厲階,去其太甚。宋申錫言未出口,尋以破家;李仲言謀之不臧,幾乎敗國。何、竇之徒轉蹙,
 讓、珪之勢尤狂,五十餘年,禍胎逾煽,昭宗之季,所不忍聞。



 臣遍覽前書,考茲覆轍,試言大較,庶竭其源。何者?自書契已來,不無閽寺,況垂之天象,備見職官。即如秦皇、漢武,宮闈之內,宦官以侍宴游。但英睿之君,措置斯得;及荒僻之主,奢蕩是求。委番、棸、蹶、楀之徒,飾姬姜狗馬之玩,外言不入,惟欲是從。雖並列五侯,猶為賞薄;遍封萬戶,尚嫌恩疏。茍思捧日之勤,遂據回天之勢。及三綱錯亂,四海崩離。袁本初之入北宮,無須殆盡;石冉閔之
 攻鄴下,內豎咸誅。旋至殄瘁邦家,不獨感傷和氣,淫刑斯逞,可為傷心。向使不假威權,但趨帷扆,何止四星終吉,抑亦萬乘延洪!昔賢為社鼠之喻,不其然乎?



 今錄楊思勖已下所行事,以為鑒誡云。



 楊思勖,本姓蘇,羅州,石城人。為內官楊氏所養,以閹,從事內侍省。預討李多祚功,超拜銀青光祿大夫,行內常侍。思勖有膂力,殘忍好殺。從臨淄王誅韋氏,遂從王為爪士,累遷右監門衛將軍。



 開元初,安南首領梅玄成叛,
 自稱「黑帝」。與林邑、真臘國通謀,陷安南府。詔思勖將兵討之。思勖至嶺表,鳩募首領子弟兵馬十餘萬,取伏波故道以進,出其不意。玄成遽聞兵至,惶惑計無所出,竟為官軍所擒,臨陣斬之,盡誅其黨與,積尸為京觀而還。



 十二年,五溪首領覃行璋作亂,思勖復受詔率兵討之,生擒行璋,斬其黨三萬餘級。以軍功累加輔國大將軍。後從東封,又加驃騎大將軍,封虢國公。



 十四年,邕州賊帥梁大海擁賓、橫等數州反叛。思勖又統兵討之,生擒
 梁大海等三千餘人,斬餘黨二萬餘級,復積尸為京觀。



 十六年,瀧州首領陳行範、何游魯、馮璘等聚徒作亂,陷四十餘城。行範自稱帝,游魯稱定國大將軍,璘稱南越王,割據嶺表。詔思勖率永、連、道等兵及淮南弩手十萬人進討。兵至瀧州,臨陣擒游魯、馮璘,斬之。行範潛竄深州,投雲際、盤遼二洞。思勖悉眾攻之,生擒行範,斬之。斬其黨六萬級,獲口馬金玉巨萬計。思勖性剛決,所得俘囚,多生剝其面,或剺發際,掣去頭皮;將士已下,望風懾
 憚,莫敢仰視,故所至立功。內給事牛仙童使幽州,受張守珪厚賂。玄宗怒,命思勖殺之。思勖縛架之數日,乃探取其心,截去手足,割肉而啖之,其殘酷如此。二十八年卒,時年八十餘。



 高力士,潘州人,本姓馮。少閹,與同類金剛二人,聖歷元年嶺南討擊使李千里進入宮。則天嘉其黠惠,總角修整,令給事左右。後因小過,撻而逐之。內官高延福收為假子。延福出自武三思家,力士遂往來三思第。歲餘,則
 天復召入禁中,隸司宮臺,廩食之。長六尺五寸,性謹密,能傳詔敕,授宮闈丞。



 景龍中,玄宗在籓,力士傾心奉之,接以恩顧。及唐隆平內難,升儲位,奏力士屬內坊,日侍左右,擢授朝散大夫、內給事。先天中,預誅蕭、岑等功,超拜銀青光祿大夫,行內侍同正員。開元初,加右監門衛將軍,知內侍省事。



 玄宗尊重宮闈,中官稍稱旨,即授三品將軍,門施棨戟,故楊思勖、黎敬仁、林招隱、尹鳳祥等,貴寵與力士等。楊則持節討伐,黎、林則奉使宣傳,尹則
 主書院。其餘孫六、韓莊、楊八、牛仙童、劉奉廷、王承恩、張道斌、李大宜、硃光輝、郭全、邊令誠等,殿頭供奉、監軍、入蕃、教坊、功德主當,皆為委任之務。監軍則權過節度,出使則列郡闢易。其郡縣豐贍,中官一至軍,則所冀千萬計,修功德,市鳥獸,詣一處,則不啻千貫,皆在力士可否。故帝城中甲第,畿甸上田、果園池沼,中官參半於其間矣。



 每四方進奏文表,必先呈力士,然後進御,小事便決之。玄宗常曰:「力士當上,我寢則穩。」故常止於宮中,稀出
 外宅。若附會者,想望風彩,以冀吹噓,竭肝膽者多矣。宇文融、李林甫、李適之、蓋嘉運、韋堅、楊慎矜、王鉷、楊國忠、安祿山、安思順、高仙芝因之而取將相高位,其餘職不可勝紀。肅宗在春宮,呼為二兄,諸王公主皆呼「阿翁」,駙馬輩呼為「爺」。力士於寢殿側簾帷中休息,殿側亦有一院,中有修功德處,雕瑩璀璨,窮極精妙。力士謹慎無大過,然自宇文融已下,用權相噬,以紊朝綱,皆力士之由。又與時消息,觀其勢候,雖至親愛,臨覆敗皆不之救。



 力
 士義父高延福夫妻,正授供奉。嶺南節度使於潘州求其本母麥氏送長安,令兩媼在堂,備於甘脆。金吾大將軍程伯獻與力士結為兄弟,麥氏亡,伯獻於靈筵散發,具縗絰,受賓吊答。十七年,贈力士父廣州大都督,麥氏越國夫人。



 開元初,瀛州呂玄晤作吏京師,女有姿色,力士娶之為婦,擢玄晤為少卿、刺史,子弟皆為王傅。呂夫人卒,葬城東,葬禮甚盛。中外爭致祭贈,充溢衢路;自第至墓,車馬不絕。



 天寶初,加力士冠軍大將軍、右監門衛
 大將軍,進封渤海郡公。七載,加驃騎大將軍。力士資產殷厚,非王侯能擬。於來庭坊造寶壽佛寺、興寧坊造華封道士觀,寶殿珍臺,侔於國力。於京城西北截澧水作碾,並轉五輪,日破麥三百斛。初,寶壽寺鐘成,力士齋慶之,舉朝畢至。凡擊鐘者,一擊百千;有規其意者,擊至二十杵,少尚十杵。



 其後又有華州袁思藝,特承恩顧。然力士巧密,人悅之;思藝驕倨,人士疏懼之。十四載,置內侍省內侍監兩員,秩正三品,以力士、思藝對任之。玄宗幸
 蜀,思藝走投祿山,力士從幸成都,進封齊國公。從上皇還京,加開府儀同三司,賜實封五百戶。



 上元元年八月,上皇移居西內甘露殿,力士與內官王承恩、魏悅等,因侍上皇登長慶樓,為李輔國所構,配流黔中道。力士至巫州,地多薺而不食,因感傷而詠之曰:「兩京作芹賣,五溪無人採。夷夏雖不同,氣味終不改。」



 寶應元年三月,會赦歸,至朗州,遇流人言京國事,始知上皇厭代。力士北望號慟,嘔血而卒。代宗以其耆宿,保護先朝,贈揚州大
 都督,陪葬泰陵。



 李輔國,本名靜忠,閑廄馬家小兒。少為閹,貌陋,粗知書計。為僕,事高力士,年且四十餘,令掌廄中簿籍。天寶中,閑廄使五鉷嘉其畜牧之能,薦入東宮。祿山之亂,玄宗幸蜀;輔國侍太子扈從,至馬嵬,誅楊國忠。輔國獻計太子,請分玄宗麾下兵,北趨朔方,以圖興復。輔國從至靈武,勸太子即帝位,以系人心。肅宗即位,擢為太子家令,判元帥府行軍司馬事,以心腹委之。仍賜名護國,四方
 奏事,御前符印軍號,一以委之。輔國不茹葷血,常為僧行,視事之隙,手持念珠,人皆信以為善。從幸鳳翔,授太子詹事,改名輔國。



 肅宗還京,拜殿中監,閑廄、五坊、宮苑、營田、栽接、總監等使。又兼隴右群牧、京畿鑄錢、長春宮等使,勾當少府、殿中二監都使。至德二年十二月,加開府儀同三司,進封郕國公,食實封五百戶。



 宰臣百司,不時奏事,皆因輔國上決。常在銀臺門受事,置察事子數十人,官吏有小過,無不伺知,即加推訊。府縣按鞫,三
 司制獄,必詣輔國取決,隨意區分,皆稱制敕,無敢異議者。每出則甲士數百人衛從。中貴人不敢呼其官,但呼五郎。宰相李揆,山東甲族,位居臺輔,見輔國執子弟之禮,謂之五父。肅宗又為輔國娶故吏部侍郎元希聲侄擢女為妻。擢弟挹,時並引入臺省,擢為梁州長史。輔國判元帥行軍司馬,專掌禁兵,賜內宅居止。



 上皇自蜀還京,居興慶宮,肅宗自夾城中起居。上皇時召伶官奏樂,持盈公主往來宮中,輔國常陰候其隙而間之。上元元
 年,上皇嘗登長慶樓,與公主語。劍南奏事官過朝謁,上皇令公主及如仙媛作主人。



 輔國起微賤,貴達日近,不為上皇左右所禮,慮恩顧或衰,乃潛畫奇謀以自固。因持盈待客,乃奏云:「南內有異謀。」矯詔移上皇居西內,送持盈於玉真觀,高力士等皆坐流竄。



 二年八月,拜兵部尚書,餘官如故。詔群臣於尚書省送上,賜御府酒饌、太常樂,武士戎服夾道,朝列畢會。輔國驕恣日甚,求為宰臣,肅宗曰:「以公勛力,何官不可,但未允朝望,如何?」輔國
 諷僕射裴冕聯章薦己。肅宗密謂宰臣蕭華曰:「輔國欲帶平章事,卿等欲有章薦,信乎?」華不對。問裴冕,曰:「初無此事,吾臂可截,宰相不可得也。」華復入奏,上喜曰:「冕固堪大用。」輔國銜之。寶應元年四月,肅宗寢疾,宰臣等不可謁見,輔國誣奏華專權,請黜之。上不許,輔國固請不已。乃罷華知政事,守禮部尚書。及帝崩,華竟被斥逐。



 代宗即位,輔國與程元振有定策功,愈恣橫。私奏曰:「大家但內裏坐,外事聽老奴處置。」代宗怒其不遜,以方握禁
 軍,不欲遽責。乃尊為尚父,政無巨細,皆委參決。五月,加司空、中書令,食實封八百戶。程元振欲奪其權,請上漸加禁制,乘其有間,乃罷輔國判元帥行軍事,其閑廄已下使名,並分授諸貴,仍移居外。輔國始懼,茫然失據。詔進封博陸王,罷中書令,許朝朔望。輔國欲入中書修謝表,閽吏止之曰:「尚父罷相,不合復入此門。」乃氣憤而言曰:「老奴死罪,事朗君不了,請於地下事先帝。」上猶優詔答之。十月十八日夜,盜入輔國第,殺輔國,攜首臂而去。
 詔刻木首葬之,仍贈太傅。



 程元振,以宦者直內侍省,累遷至內射生使。寶應末,肅宗晏駕,張皇后與太子有怨,恐不附己,引越王系入宮,欲令監國。元振知其謀,密告李輔國,乃挾太子,誅越王並其黨與。代宗即位,以功拜飛龍副使、右監門將軍、上柱國,知內侍省事。尋代輔國判元帥行軍司馬,專制禁兵,加鎮軍大將軍、右監門衛大將軍,封保定縣侯,充寶應軍使。九月,加驃騎大將軍,封邠國公,贈其父元貞司
 空。母卻氏,趙國夫人。是時元振之權,甚於輔國,軍中呼為「十郎」。



 元振常請托於襄陽節度使來瑱,瑱不從。及元振握權,徵瑱入朝。瑱遷延不至。廣德元年,破裴,遂入朝,拜兵部尚書。元振欲報私憾,誣瑱之罪,竟坐誅。宰臣裴冕為肅宗山陵使,有事與元振相違,乃發小吏贓私,貶冕施州刺史。來瑱名將,裴冕元勛,二人既被誣陷,天下方鎮皆解體。元振猶以驕豪自處,不顧物議。



 九月,吐蕃、黨項入犯京畿,下詔徵兵,諸道卒無至者。十月,蕃軍
 至便橋,代宗蒼黃出幸陜州;賊陷京師,府庫蕩盡。及至行在,太常博士柳伉上疏切諫誅元振以謝天下,代宗顧人情歸咎,乃罷元振官,放歸田里,家在三原。



 十二月,車駕還京。元振服縗麻於車中,入京城,以規任用。與御史大夫王昇飲酒,為御史所彈。詔曰:



 族談錯立,法尚不容;同惡陰謀,議當從重。有一於此,情實難原。程元振性惟兇愎,質本庸愚,蕞爾之身,合當萬死。頃已寬其嚴典,念其微勞,屈法伸恩,放歸田里。仍乖克己,尚未知非;既
 忘含煦之仁,別貯覬覦之望。敢為嘯聚,仍欲動搖,不令之臣,共為睥睨;妄談休咎,仍懷怨望。束兵裹甲,變服潛行,無顧君親,將圖不軌。按驗皆是,無所逃刑,首足異門,未雲塞責。朕猶不忘薄效,再舍罪人;特寬斧鉞之誅,俾正投荒之典。宜長流榛州百姓,委京兆府差綱遞送;路次州縣,差人防援,至彼捉拘,勿許東西。縱有非常之赦,不在會恩之限。凡百僚庶,宜體朕懷。



 魚朝恩,天寶末以宦者入內侍省,初為品官,給事黃
 門。性黠惠,善宣答,通書計。至德中,常令監軍事。九節度討安慶緒於相州,不立統帥,以朝恩為觀軍容宣慰處置使。觀軍容使名,自朝恩始也。以功累加左監門衛大將軍。時郭子儀頻立大功,當代無出其右;朝恩妒其功高,屢行間諜;子儀悉心奉上,殊不介意。肅宗英悟,特察其心,故朝恩之間不行。自相州之敗,史思明再陷河洛,朝恩常統禁軍鎮陜,以殿東夏。廣德元年,西蕃入犯京畿,代宗幸陜。時禁軍不集,徵召離散,比至華陰,朝恩大軍
 遽至迎奉,六師方振。由是深加寵異,改為天下觀軍容宣慰處置使。時四方未寧,萬務事殷,上方注意勛臣,朝恩專典神策軍,出入禁中,賞賜無算。



 朝恩性本凡劣,恃勛自伐,靡所忌憚。時引腐儒及輕薄文士於門下,講授經籍,作為文章,粗能把筆釋義,乃大言於朝士之中,自謂有文武才幹,以邀恩寵。上優遇之,加判國子監事,光祿、鴻臚、禮賓、內飛龍、閑廄等使。赴國子監視事,特詔宰臣、百僚、六軍將軍送上,京兆府造食,教坊賜樂。大臣群
 官二百餘人,皆以本官備章服充附學生,列於監之廊下,侍詔給錢萬貫充食本,以供學生廚料。朝恩恣橫,求取無厭,凡有奏請,以先允為度,幸臣未有其比。



 大歷二年,朝恩獻通化門外賜莊為寺,以資章敬太后冥福;仍請以章敬為名,復加興造,窮極壯麗。以城中材木不足充費,乃奏壞曲江亭館、華清宮觀樓及百司行廨、將相沒官宅給其用,土木之役,僅逾萬億。三年,讓判國子監事,加韓國公。



 章敬太后忌日,百僚於興唐寺行香,朝恩
 置齋饌於寺外之車坊,延宰臣百僚就食。朝恩恣口談時政,公卿惕息。戶部郎中相里造、殿中侍御史李衎以正言折之。朝恩不悅,乃罷會。



 後嘗釋奠於國子監。宰臣百僚皆會,朝恩講《易》,征《鼎卦》「覆餗」之義,以譏元載。載心銜之,陰圖除去之。上以朝恩太橫,亦惡之。載欲伺其便,巧中傷之;乃用腹心崔昭為京兆尹,伺朝恩出處。昭不吝財賂,潛與朝恩黨陜州觀察使皇甫溫相結,溫與昭協。自是朝恩動靜,載皆知之,巨細悉以聞。上益怒,朝恩
 未之察,日以驕橫。載奏加朝恩實封,又加皇甫溫權位,以肆其欲。



 五年,朝恩所暱武將劉希暹微有過忤,上諷之。詔罷朝恩觀軍容使,加實封通前一千戶。朝恩始疑,然每朝謁,恩顧如常,亦不以載為意。會寒食宴近臣,朝恩入謁。先是,每宴罷,必出還營,是日有詔留之。朝恩始懼,言頗悖慢,上亦以舊恩不之責。是日朝恩還第,自經而卒。劉希暹亦下獄賜死。



 希暹,出自戎伍,有膂力,形貌光偉,以騎射聞。朝恩用之
 為神策都虞候,封交河郡王。善候朝恩意旨,深被委信。累遷至太僕卿,與兵馬使王駕鶴同掌禁兵,所為不法。諷朝恩於北軍置獄,召坊市兇惡少年,羅織城內富人,誣以違法,捕置獄中,忍酷考訊,錄其家產,並沒於軍。或有舉選之士,財貨稍殷,客於旅舍,遇橫死者非一。坊市苦之,謂之「入地牢」。捕賊吏有賈明觀者,尤兇蠹,以屢置大獄,家產巨萬。希暹黨之,地在禁密,人無敢言者。朝恩死,上寬宥之。以素志非順,慮不見容,常自疑懼。與王駕
 鶴聯職,希暹辭多不遜。駕鶴純謹,上信任之,至是以希暹語上聞,乃誅之。



 賈明觀者,本萬年縣捕賊吏。事希暹,恣為兇惡,毒甚豺狼。朝恩、希暹既死,元載復受明觀奸謀,潛容之,特奏令江西效力。明觀將出城,百姓數萬人懷磚石候之,載令市吏止約。明觀在洪州二年,觀察使魏少游容之。及路嗣恭代少游,至郡之日,召明觀笞殺之。識者減魏之名,多路之正。



 朝恩素待禮部尚書裴士淹,戶部侍郎、判度
 支第五琦,二人亦坐貶官。



 竇文場、霍仙鳴者,始在東宮事德宗。初魚朝恩誅後,內官不復典兵,德宗以親軍委白志貞。志貞多納豪民賂,補為軍士,取其傭直,身無在軍者,但以名籍請給而已。涇師之亂,帝召禁軍禦賊,志貞召集無素,是時並無至者,唯文場、仙鳴率諸宦者及親王左右從行。志貞貶官,左右禁旅,悉委文場主之。從幸山南,兩軍漸集。



 德宗還京,頗忌宿將,凡握兵多者,悉罷之。禁旅文場、仙鳴分統
 焉。貞元十二年六月,特立護軍中尉兩員、中護軍兩員,以帥禁軍。乃以文場為左神策護軍中尉,仙鳴為右神策護軍中尉,右神威軍使張尚進為右神策中護軍,內謁者監焦希望為左神策中護軍,自文場等始也。



 時竇、霍之權,振於天下,籓鎮節將,多出禁軍,臺省清要,時出其門。文場累加驃騎大將軍。是歲仙鳴病,帝賜馬十匹,令於諸寺為僧齋以祈福。久病不愈,十四年,倉卒而卒。上疑左右小使正將食中加毒,配流者數十人。仙鳴死
 後,以開府內常侍第五守亮為右軍中尉。文場連表請致仕,許之。



 十五年已後,楊志廉、孫榮義為左右軍中尉,亦踵竇、霍之事,怙寵驕恣。貪利冒寵之徒,利其納賄,多附麗之。至於貞元末,宦官復盛。順宗即位,王叔文用事,與韋執誼謀奪神策軍權,乃用宿將範希朝為京西北禁軍都將。事未行,為內官俱文珍等所排,叔文貶而止。



 俱文珍,貞元末宦官,後從義父姓,曰劉貞亮。性忠正,剛而蹈義。順宗即位,風疾不能視朝政,而宦官李忠言與
 牛美人侍病。美人受旨於帝,復宣之於忠言;忠言授之王叔文。叔文與朝士柳宗元、劉禹錫、韓日華圖議,然後下中書,俾韋執誼施行,故王之權振天下。叔文欲奪宦者兵權,每忠言宣命,內臣無敢言者,唯貞亮建議與之爭。知其朋徒熾,慮隳朝政,乃與中官劉光琦、薛文珍、尚衍、解玉等謀,奏請立廣陵王為皇太子,勾當軍國大事。順宗可之。貞亮遂召學士衛次公、鄭絪、李程、王涯入金鑾殿,草立儲君詔。及太子受內禪,盡逐叔文之黨,政
 事悉委舊臣,時議嘉貞亮之忠藎。累遷至右衛大將軍,知內侍省事。元和八年卒,憲宗思其翊戴之功,贈開府儀同三司。



 吐突承璀,幼以小黃門直東宮,性敏慧,有才幹。憲宗即位,授內常侍,知內省事,左監門將軍。俄授左軍中尉、功德使。四年,王承宗叛,詔以承璀為河中、河南、浙西、宣歙等道赴鎮州行營兵馬招討等使,內侍省常侍宋惟澄為河南、陜州、河陽已來館驛使,內官曹淮玉、劉國珍、馬
 江朝等分為河北行營糧料館驛等使。諫官、御史上疏相屬,皆言自古無中貴人為兵馬統帥者,補闕獨孤鬱、段平仲尤激切。憲宗不獲已,改為充鎮州已來招撫處置等使。及承璀率禁軍上路,帝禦通化門樓,慰諭遣之。出師經年無功,乃遣密人告王承宗,令上疏待罪,許以罷兵為解。仍奏昭義節度使盧從史素與賊通,許為承宗求節鉞。乃誘潞州牙將烏重胤謀執從史送京師。及承宗表至,朝廷議罷兵,承璀班師,仍為禁軍中尉。段平
 仲抗疏極論承璀輕謀弊賦,請斬之以謝天下,憲宗不獲已,降為軍器使。俄復為左衛上將軍,知內侍省事。



 時弓箭庫使劉希先取羽林大將軍孫璹錢二十萬,以求方鎮,事發賜死,辭相告訐,事連承璀,乃出為淮南節度監軍使。



 太子通事舍人李涉,性狂險,投匭上書,論希先、承璀無罪,不宜貶戮。諫議大夫、知匭事孔戣,見涉疏之副本,不受其章。涉持疏於光順門欲進之,戣上疏論其纖邪,貶涉硤州司倉。上待承璀之意未已,而宰相李絳
 在翰林,時數論承璀之過,故出之。八年,欲召承璀還,乃罷絳相位。承璀還,復為神策中尉。惠昭太子薨,承璀建議請立澧王寬為太子,憲宗不納,立遂王宥。穆宗即位,銜承璀不佑己,誅之。敬宗時,中尉馬存亮論承璀之冤,詔雪之,仍令假子士曄以禮收葬。



 王守澄,元和末宦者。憲宗疾大漸,內官陳弘慶等弒逆。憲宗英武,威德在人,內官秘之,不敢除討,但云藥發暴崩。時守澄與中尉馬進潭、梁守謙、劉承偕、韋元素等定
 冊立穆宗皇帝。長慶中,守澄知樞密事。



 初,元和中,守澄為徐州監軍,遇翼城醫人鄭注,出入節度使李醖家。注敏悟過人,博通典藝,棋奕醫卜,尤臻於妙,人見之者,無不歡然。注嘗為李醖煮黃金,服一刀圭,可愈痿弱重膇之疾,復能反老成童。醖與守澄服之,頗效。守澄知樞密,薦引入禁中,穆宗待之亦厚。注多奇詭,每與守澄言必通夕。



 文宗即位,守澄為驃騎大將軍,充右軍中尉。注復得幸於文宗,後依倚守澄,大為奸弊。文宗以元和逆黨
 尚在,其黨大盛,心常憤惋,端居不怡。翰林學士宋申錫嘗獨對探知,上略言其意,申錫請漸除其逼。帝亦以申錫沉厚有方略,為其事可成,乃用為宰相。申錫謀未果,為注所察,守澄乃令軍吏豆盧著誣告申錫與漳王謀逆,申錫坐貶。



 宰相李逢吉從子訓,與注交通,訓亦機詭萬端,二人情義相得,俱為守澄所重。復引訓入禁中,為上講《周易》。既得幸,又探知帝旨,復以除宦官謀中帝意。帝以訓才辯縱橫,以為其事必捷,待以殊寵,自流人中
 用為學官,充侍進學士。時仇士良有翌上之功,為守澄所抑,位未通顯。訓奏用士良分守澄之權,乃以士良為左軍中尉;守澄不悅,兩相矛盾。訓因其惡。



 太和九年,帝令內養李好古齏鴆賜守澄,秘而不發,守澄死,仍贈揚州大都督。其弟守涓為徐州監軍,召還,至中牟,誅之。守澄豢養訓、注,反罹其禍,人皆快其受佞,而惡訓、注之陰狡。



 李訓既殺守澄,復惡鄭注,乃奏用注為鳳翔節度使。訓欲盡誅宦官,乃與金吾將軍韓約、新除太原節度使
 王璠、新除邠寧節度使郭行餘、權御史中丞李孝本、權京兆尹羅立言謀。其年十一月二十一日,上御宣政殿,百僚班定,韓約不奏平安,乃奏曰:「臣當仗廨內石榴樹,夜來降甘露,請陛下幸仗舍觀之。」帝乘輦趨金吾仗。中尉仇士良與諸官先往石榴樹觀之,伺知其詐;又聞幕下兵仗聲,蒼黃而還,奏曰:「南衙有變。」遂扶帝輦入閣門。李訓從輦大呼曰:「邠寧、太原之兵,何不赴難?衛乘輿者,人賞百千!」於是誰何之卒,及御史臺從人,持兵入宣政
 殿院,宦官死者甚眾。輦既入閣門,內官呼萬歲。俄而士良等率禁兵五百餘人,露刃出東上閣逢人即殺,王涯、賈餗、舒元輿、李訓等四人宰相及王璠、郭行餘等十一人,尸橫闕下。自是權歸士良與魚弘志。至宣宗即位,復誅其太甚者,而閽寺之勢,仍握軍權之重焉。



 田令孜,本姓陳。咸通中,從義父入內侍省為宦者。頗知書,有謀略,自諸司小使監諸鎮用兵,累遷神策中尉、左監門衛大將軍。乾符中,盜起關東。諸軍誅盜,以令孜為
 觀軍容、制置左右神策、護駕十軍等使。京師不守,從僖宗幸蜀。鸞輿返正,令孜頗有匡佐之功,時令孜威權振天下。



 時關中寇亂初平,國用虛竭,諸軍不給。令孜請以安邑、解縣兩池榷鹽課利,全隸神策軍。詔下,河中王重榮抗章論列,言使名久例隸當道,省賦自有常規。令孜怒,用王處存為河中節度使,重榮不奉詔。令孜率禁兵討之。重榮引太原軍為援,戰於沙苑,禁軍大敗。京師復亂,僖宗出幸寶雞,又移幸山南,方鎮皆憾令孜生事。令
 孜懼,引前樞密楊復恭代己,從幸梁州,求為西川監軍。西川節度使陳敬瑄,即令孜之弟也。



 昭宗即位,三川大亂。詔宰相韋昭度鎮西川,陳敬瑄不受代。令孜引閬州刺史王建為援,建素以父事令孜。時建方亂東川,聞其召也,以西蜀可圖,欣然赴之。建以所領千餘兵至漢州,陳敬瑄以建雄豪難制,辭而遣之。建曰:「十軍阿父召予,及門而拒,鄰籓聞之,孰肯相容?為予報令公,建至此,無所歸也。」遂遣使上表,請討陳敬瑄以自效。朝廷嘉之,即
 命昭度為招討,入蜀加兵,經年無功,昭度還京。建遂絕棧道,不通詔使。歲中急擊成都,陳敬瑄計窘,遣令孜出城,與建通和。建竟自為蜀帥,令孜以義父之故,依倚仍舊監軍事。既而陳敬瑄遇鴆,令孜亦為建所殺。



 楊復光,內常侍楊玄價之養子也。幼以宦者入內侍省,慷慨負節義,有籌略,為小黃門,監鎮兵征討。乾符中,賊渠黃巢之犯江西,復光為排陣使,遣判官吳彥弘入城喻朝旨,巢即令其將尚君長奉表歸國。招討使宋威害
 其功,並兵擊賊,巢怒,復作剽。朝廷誅尚君長,怨怒愈深。宋威戰敗,復光總其兵權,進攻洪州,擒賊將徐唐莒。詔以荊南節度使王鐸為招討,代宋威。復光監忠武軍,屯於鄧州,以遏賊沖。



 京師陷賊,節度使周岌受偽命,賊使往來旁午。岌嘗夜宴,急召復光。左右曰:「周公歸賊,必謀害內侍,不如勿往。」復光曰:「事勢如此,義不圖全。」即赴之。酒酣,岌言本朝事,復光因泣下。良久曰:「丈夫所感者恩義,而規利害,非丈夫也。公自匹夫享公侯之貴,豈舍十
 八葉天子而北面臣賊,何恩義利害之可言乎!」聲淚俱發,岌亦為之流涕。岌曰:「吾不能獨力拒賊,貌奉而心圖之,故召公。」瀝酒為盟。是夜,復光遣其養子守亮殺賊使於傳舍。



 時秦宗權叛岌,據蔡州。復光得忠武之師三千入蔡州,說宗權,俾同義舉。宗權遣將王淑率眾萬人從復光收荊襄。次鄧州,王淑逗留不進,復光斬之,並其軍,分為八都。鹿晏弘、晉暉、李師泰、王建、韓建等,皆八都之大將也。進攻南陽,賊將硃溫、何勤來逆戰。復光敗之,進
 收鄧州,獻捷行在,中和元年五月也。復光乘勝追賊,至藍橋,丁母憂還。尋起復,受詔充天下兵馬都監,押諸軍入定關輔。王重榮為東面招討使,復光以兵會之。



 二年七月,至河中。賊將硃溫守同州,復光遣使諭之。九月,溫以所部來降。時賊將李翔守華州,巢寇益盛,王重榮憂之。謂復光曰:「臣賊則負國,拒戰則兵微,今日成敗,未可知也,公其圖之。」復光曰:「雁門李僕射以雄武振北陲,其家尊與吾先世同患難。李雁門奮不顧身,自播遷已來,
 徵兵未至者,蓋太原阻路也。如以朝旨諭鄭公,詔到,其軍必至。」重榮曰:「善!」王鐸遣使奉墨詔之太原,太原以兵從之。及收京城,三敗巢賊,復光與其子守亮、守宗等身先犯難,功烈居多。其年六月,卒於河中,時年四十二。



 復光雖黃門近幸,然慷慨有大志,善撫士卒;及死之日,軍中慟哭累日。身後平賊立功者,多是復光部下門人故將也。



 諸假子:守亮,興元節度使;守宗,忠武節度使:守信,商州防禦使;守忠,洋州節度使;其餘以守為名者數十
 人,皆為牧守將帥。



 楊復恭,貞元末中尉楊志廉之後。志廉子欽義,大中朝為神策中尉。欽義子三人:玄翼、玄價、玄寔。



 玄翼,咸通中掌樞密;玄寔乾符中為右軍中尉;玄價,河陽監軍。



 復恭,即玄翼子也。以父,幼為宦者,入內侍省。知書,有學術,每監諸鎮兵。龐勛之亂,監陣有功,自河陽監軍入為宣徽使。咸通十年,玄翼卒,起復為樞密使。時黃巢犯闕,左軍中尉田令孜為天下觀軍容制置使,專制中外。復恭每
 事力爭得失,令孜怒,左授復恭飛龍使,乃稱疾退於藍田。



 僖宗自蜀還京,田令孜出師失律,車駕再幸山南,復用復恭為樞密使,尋代令孜為右軍中尉。時行在制置,內外經略,皆出於復恭。車駕還京,授觀軍容使,封魏國公。



 僖宗晏駕,迎壽王踐祚。文德元年,加開府、金吾上將軍,專典禁兵,既軍權在手,頗擅朝政。昭宗惡之,政事多訪於宰臣。故韋昭度、張浚、杜讓能每有陳奏,即舉大中故事,稍抑宦者之權。上性明察,由是偏聽之釁生焉。國
 舅王瑰,頗居中任事,復恭惡之,奏授黔南節度。至吉柏江,覆舟而沒,物議歸咎於復恭,上每切齒道復恭。復恭假子天威軍使守立,權勇冠於六軍,人皆避之。上欲罪復恭,懼守立為亂,乃謂復恭曰:「吾要卿家守立在左右,可進來。」乃賜姓李,名順節,恩寵特異,勢侔樞要。乃與復恭爭權,每中傷其陰事,授順節鎮海軍節度使、同平章事。



 大順二年九月,詔復恭致仕,賜杖履。復恭既失勢,欲退止商山別居,第在昭化里,近玉山營。假子守信為玉
 山軍使,守信時候復恭於其第,或誣告云玉山軍使與復恭謀亂,詔李順節率禁軍攻之。昭宗御延喜樓。守信以兵拒之,順節屢敗。際晚,守信、復恭挈其族出通化門,趨興元。守信令部將張綰殿其後,綰戰敗,被擒。復恭至興元,節度使楊守亮乃糾合諸守義兄弟舉兵,以討順節為名。天子詔李茂貞、王行瑜討之。



 明年,守亮兵敗,復恭與守亮挈其族,將奔太原,入商山。至乾元縣,為華州兵所獲,執送京師,皆梟首於市。李茂貞收興元,進復恭
 前後與守亮私書六十紙,內訴致仕之由云:「承天是隋家舊業,大侄但積粟訓兵,不要進奉。吾於荊榛中援立壽王,有如此負心,門生天子,既得尊位,乃廢定策國老。」其不遜如是。後復恭假子彥博奔太原,收復恭骸骨,葬於介休縣之抱腹山。



 復恭之後,宦者西門重遂為右軍中尉。李茂貞初並山南,兵眾強盛,干預朝政,宰相杜讓能與重遂等謀誅之。師興,為茂貞所敗,重遂被誅,乃以內官駱全瓘、劉景宣為左右軍中尉。



 乾寧二年春,李茂
 貞、王行瑜以兵入朝,殺宰相韋昭度、李溪。河東節度使李克用率師渡河,討邠、岐二帥,軍於渭北。駱全瓘與茂貞宿衛將閻圭,脅天子幸岐州,昭宗蒼黃幸莎城。茂貞以太原問罪,乃誅全瓘、閻圭以自解。昭宗幸華州,宦官稍微。



 及光化還宮,內官景務修、宋道弼復專國政,宰相崔胤深惡之,中外不睦。宰相徐彥若、王搏有度量,見其陰險相傾,懼危時事,嘗奏曰:「人君當務大體,平心御物,無有偏私。偏任偏聽,古人所患。今中官怙寵,道路目之,
 皆知此弊,然未能卒改。俟多難漸平,以道消息之。陛下勿洩聖謨,啟其奸詐。」崔胤知搏所奏,頗銜之,他日見上,曰:「王搏奸邪,已為敕使外應,不可在相位。」二年六月,貶搏官,賜死於藍田。道弼、務修亦賜死。以樞密使劉季述、王奉先為兩軍中尉,出徐彥若鎮南海。



 崔胤秉政而排擯宦官,季述等外結籓侯,以為黨援。十一月六日,季述矯詔以皇太子監國,遂廢昭宗。居東內,奪傳國寶授太子。昭宗以何皇后宮。數人隨行,幽於東宮。季述手持
 銀禋,於上前以禋畫地數上罪狀,云:「某時某事,你不從我言,其罪一也。」其悖逆如此。乃令李師虔以兵圍之。鎔錫錮其扃鎔鐍。時方凝冽,嬪御無被,哭聲聞於外。穴墻通食者兩月。十二月晦,崔胤等謀反正,誅季述、奉先,復迎昭宗即位,改元天復元年。



 其歲十一月,硃全忠寇河中華州,陷之;京師震恐。中尉韓全誨請上且幸鳳翔。全忠追逼乘輿,兵圍鳳翔者累年。三年正月,茂貞殺兩軍中尉韓全誨、張弘彥、樞密使袁易簡、周敬容等二十二人,
 皆斬首,以布囊貯之,令學士薛貽矩送於全忠求和。是月,全忠迎駕還長安,詔以崔胤為宰相,兼判六軍諸衛。



 胤奏曰:「高祖、太宗承平時,無內官典軍旅。自天寶以後,宦官浸盛。貞元、元和,分羽林衛為左、右神策軍,使衛從,令宦官主之,唯以二千人為定制。自是參掌樞密。由是內務百司,皆歸宦者,上下彌縫,共為不法:大則傾覆朝政,小則構扇籓方。車駕頻致播遷,朝廷漸加微弱,原其禍作,始自中人。自先帝臨御已來,陛下纂承之後,朋儕
 日熾,交亂朝綱,此不翦其本根,終為國之蟊賊。內諸司使務宦官主者,望一切罷之,諸道監軍使,並追赴闕廷,即國家萬世之便也。」詔曰:



 宦官之興,肇於秦、漢。趙高、閻樂,竟滅嬴宗;張讓、段珪,遂傾劉祚。肆其志則國必受禍,悟其事則運可延長。朕所以斷在不疑,祈天永命者也。



 先皇帝嗣位之始,年在幼沖,群豎相推,奄專大政。於是毒流宇內,兵起山東,遷幸三川,幾淪神器。回鑾之始,率土思安,而田令孜妒能忌功,遷搖近鎮,陳倉播越,患難
 相仍。洎朕纂承,益相侮慢,復恭、重遂逞其禍,道弼、季述繼其兇;幽辱朕躬,凌脅孺子。天復返正,罪己求安,兩軍內樞,一切假借。韓全誨等每懷憤惋,曾務報仇;視將相若血仇,輕君上如木偶。未周星歲,竟致播遷;及在岐陽,過於羈紲。上憂宗社傾墜,下痛民庶流離,茫然孤居,無所控告。



 全忠位兼二柄,深識朕心,駐兵近及於三年,獨斷方誅於元惡。今謝罪郊廟,即宅宮闈,正刑當在於事初,除惡宜絕其根本。先朝及朕,五致播遷,王畿之氓,減
 耗大半;父不能庇子,夫不能室妻。言念於茲,痛深骨髓,其誰之罪?爾輩之由!



 帝王之為治也,內有宰輔卿士,外有籓翰大臣,豈可令刑餘之人,參預大政?況此輩皆朕之家臣也,比於人臣之家,則奴隸之流。恣橫如此,罪惡貫盈,天命誅之,罪豈能舍?橫尸伏法,固不足矜,含容久之,亦所多愧。其第五可範已下,並宜賜死。其在畿甸同華、河中,並盡底處置訖。諸道監軍使已下,及管內經過並居停內使,敕到並仰隨處誅夷訖聞奏。已令準國朝
 故事,量留三十人,各賜黃絹衫一領,以備宮內指使,仍不得輒有養男。其左右神策軍,並令停廢。



 是日,諸司宦官百餘人,及隨駕鳳翔群小又二百餘人,一時斬首於內侍省,血流塗地。及宮人宋柔等十一人,兩街僧道與內官相善者二十餘人,並笞死於京兆府。內諸司一切罷之,皆歸省寺。自是京城並無宦宮,天子每宣傳詔命,即令宮人出入。崔胤雖復仇快志,國祚旋亦覆亡,悲夫!



 贊曰:崇墉大廈,壯其楹磶。殿邦禦侮,亦俟明德。宵人意
 褊,動不量力。投鼠敗器,良堪太息。



\end{pinyinscope}