\article{卷二 本紀第二 太宗上}

\begin{pinyinscope}

 太
 宗文
 武大聖大廣孝皇帝諱世民,高祖第二子也。母曰太穆順聖皇后竇氏。隋開皇十八年十二月戊午,生於武功之別館。時有二龍戲於館門之外,三日而去。高
 祖之臨岐州,太宗時年四歲。有書生自言善相,謁高祖曰:「公貴人也,且有貴子。」見太宗,曰:「龍鳳之姿,天日之表,年將二十,必能濟世安民矣。」高祖懼其言洩,將殺之,忽失所在,因採「濟世安民」之義以為名焉。太宗幼聰睿,玄鑒深遠,臨機果斷,不拘小節,時人莫能測也。



 大業末,煬帝於雁門為突厥所圍,太宗應募救援,隸屯衛將軍云定興營。將行,謂定興曰:「必齎旗鼓以設疑兵。且始畢可汗舉國之師,敢圍天子,必以國家倉卒無援。我張軍容,
 令數十里幡旗相續,夜則鉦鼓相應,虜必謂救兵雲集,望塵而遁矣。不然,彼眾我寡,悉軍來戰,必不能支矣。」定興從焉。師次崞縣,突厥候騎馳告始畢曰:王師大至。由是解圍而遁。及高祖之守太原,太宗時年十八。有高陽賊帥魏刀兒,自號歷山飛。來攻太原,高祖擊之,深入賊陣。太宗以輕騎突圍而進,射之,所向皆披靡,拔高祖於萬眾之中。適會步兵至,高祖與太宗又奮擊,大破之。時隋祚已終,太宗潛圖義舉,每折節下士,推財養客,群盜
 大俠,莫不願效死力。及義兵起,乃率兵略徇西河,克之。拜右領大都督,右三軍皆隸焉,封燉煌郡公。



 大軍西上賈胡堡,隋將宋老生率精兵二萬屯霍邑,以拒義師。會久雨糧盡,高祖與裴寂議,且還太原,以圖後舉。太宗曰:「本興大義以救蒼生,當須先入咸陽,號令天下;遇小敵即班師,將恐從義之徒一朝解體。還守太原一城之地,此為賊耳,何以自全!」高祖不納,促令引發。太宗遂號泣於外,聲聞帳中。高祖召問其故,對曰:「今兵以義動,進戰
 則必克,退還則必散。眾散於前,敵乘於後,死亡須臾而至,是以悲耳。」高祖乃悟而止。



 八月己卯,雨霽,高祖引師趣霍邑。太宗恐老生不出戰,乃將數騎先詣其城下,舉鞭指麾,若將圍城者,以激怒之。老生果怒,開門出兵,背城而陣。高祖與建成合陣於城東,太宗及柴紹陣於城南。老生麾兵疾進,先薄高祖,而建成墜馬,老生乘之,高祖與建成軍咸卻。太宗自南原率二騎馳下峻阪,沖斷其軍,引兵奮擊,賊眾大敗,各舍仗而走。懸門發,老生引
 繩欲上,遂斬之,平霍邑。至河東,關中豪傑爭走赴義。太宗請進師入關,取永豐倉以賑窮乏,收群盜以圖京師,高祖稱善。太宗以前軍濟河,先定渭北。三輔吏民及諸豪猾詣軍門請自效者日以千計,扶老攜幼,滿於麾下。收納英俊,以備僚列,遠近聞者,咸自托焉。師次於涇陽,勝兵九萬,破胡賊劉鷂子,並其眾。留殷開山、劉弘基屯長安故城。太宗自趣司竹,賊帥李仲文、何潘仁、向善志等皆來會,頓於阿城,獲兵十三萬。長安父老齎牛酒詣
 旌門者不可勝紀,勞而遣之,一無所受。軍令嚴肅,秋毫無所犯。尋與大軍平京城。高祖輔政,受唐國內史,改封秦國公。會薛舉以勁卒十萬來逼渭濱,太宗親擊之,大破其眾,追斬萬餘級,略地至於隴坻。



 義寧元年十二月,復為右元帥,總兵十萬徇東都。及將旋,謂左右曰:「賊見吾還,必相追躡。」設三伏以待之。俄而隋將段達率萬餘人自後而至,度三王陵,發伏擊之,段達大敗,追奔至於城下。因於宜陽、新安置熊、穀二州,戍之而還。徙封趙國
 公。高祖受禪,拜尚書令、右武候大將軍,進封秦王,加授雍州牧。



 武德元年七月,薛舉寇涇州,太宗率眾討之,不利而旋。九月,薛舉死,其子仁杲嗣立。太宗又為元帥以擊仁杲,相持於折墌城,深溝高壘者六十餘日。賊眾十餘萬,兵鋒甚銳,數來挑戰,太宗按甲以挫之。賊糧盡,其將牟君才、梁胡郎來降。太宗謂諸將軍曰:「彼氣衰矣,吾當取之。」遣將軍龐玉先陣於淺水原南以誘之,賊將宗羅並軍來拒,玉軍幾敗。既而太宗親御大軍,奄自原
 北,出其不意。羅望見,復回師相拒。太宗將驍騎數十入賊陣,於是王師表裏齊奮,羅大潰,斬首數千級,投澗穀而死者不可勝計。太宗率左右二十餘騎追奔,直趣折墌以乘之。仁杲大懼,嬰城自守。將夕,大軍繼至,四面合圍。詰朝,仁杲請降,俘其精兵萬餘人、男女五萬口。既而諸將奉賀,因問曰:「始大王野戰破賊,其主尚保堅城,王無攻具,輕騎騰逐,不待步兵,徑薄城下,咸疑不克,而竟下之,何也?」太宗曰:「此以權道迫之,使其計不暇發,
 以故克也。羅恃往年之勝,兼復養銳日久,見吾不出,意在相輕。今喜吾出,悉兵來戰,雖擊破之,擒殺蓋少。若不急躡,還走投城,仁杲收而撫之,則便未可得矣。且其兵眾皆隴西人,一敗披退,不及回顧,散歸隴外,則折墌自虛,我軍隨而迫之,所以懼而降也。此可謂成算,諸君盡不見耶?」諸將曰:「此非凡人所能及也。」獲賊兵精騎甚眾,還令仁杲兄弟及賊帥宗羅、翟長孫等領之。太宗與之游獵馳射,無所間然。賊徒荷恩懾氣,咸願效死。時
 李密初附,高祖令密馳傳迎太宗於豳州。密見太宗天姿神武,軍威嚴肅,驚悚嘆服,私謂殷開山曰:「真英主也。不如此,何以定禍亂乎?」凱旋,獻捷於太廟。拜太尉、陜東道行臺尚書令,鎮長春宮,關東兵馬並受節度。尋加左武候大將軍、涼州總管。



 宋金剛之陷澮州也,兵鋒甚銳。高祖以王行本尚據蒲州,呂崇茂反於夏縣,晉、澮二州相繼陷沒,關中震駭,乃手敕曰:「賊勢如此,難與爭鋒,宜棄河東之地,謹守關西而已。」太宗上表曰:「太原王業所
 基,國之根本,河東殷實,京邑所資。若舉而棄之,臣竊憤恨。願假精兵三萬,必能平殄武周,克復汾、晉。」高祖於是悉發關中兵以益之,又幸長春宮親送太宗。二年十一月,太宗率眾趣龍門關,履冰而渡之,進屯柏壁,與賊將宋金剛相持。尋而永安王孝基敗於夏縣,於筠、獨孤懷恩、唐儉並為賊將尋相、尉遲敬德所執,將還澮州。太宗遣殷開山、秦叔寶邀之於美良川,大破之,相等僅以身免,悉虜其眾,復歸柏壁。於是諸將咸請戰,太宗曰:「金剛懸
 軍千里,深入吾地,精兵驍將,皆在於此。武周據太原,專倚金剛以為捍。士卒雖眾,內實空虛,意在速戰。我堅營蓄銳以挫其鋒,糧盡計窮,自當遁走。」



 三年二月,金剛竟以眾餒而遁,太宗追之至介州。金剛列陣,南北七里,以拒官軍。太宗遣總管李世勣、程咬金、秦叔寶當其北,翟長孫、秦武通當其南。諸軍戰小卻,為賊所乘。太宗率精騎擊之,沖其陣後,賊眾大敗,追奔數十里。敬德、相率眾八千來降,還令敬德督之,與軍營相參。屈突通懼其為
 變,驟以為請。太宗曰:「昔蕭王推赤心置人腹中,並能畢命,今委任敬德,又何疑也。」於是劉武周奔於突厥,並、汾悉復舊地。詔就軍加拜益州道行臺尚書令。



 七月,總率諸軍攻王世充於洛邑,師次穀州。世充率精兵三萬陣於慈澗,太宗以輕騎挑之。時眾寡不敵,陷於重圍,左右咸懼。太宗命左右先歸,獨留後殿。世充驍將單雄信數百騎夾道來逼,交搶競進,太宗幾為所敗。太宗左右射之,無不應弦而倒,獲其大將燕頎。世充乃拔慈澗之鎮
 歸於東都。太宗遣行軍總管史萬寶自宜陽南據龍門,劉德威自太行東圍河內,王君廓自洛口斷賊糧道。又遣黃君漢夜從孝水河中下舟師襲回洛城,克之。黃河已南,莫不響應,城堡相次來降。大軍進屯邙山。九月,太宗以五百騎先觀戰地,卒與世充萬餘人相遇,會戰,復破之,斬首三千餘級,獲大將陳智略,世充僅以身免。其所署筠州總管楊慶遣使請降,遣李世勣率師出轘轅道安撫其眾。滎、汴、洧、豫九州相繼來降。世充遂求救於
 竇建德。



 四年二月,又進屯青城宮。營壘未立,世充眾二萬自方諸門臨谷水而陣。太宗以精騎陣於北邙山,令屈突通率步卒五千渡水以擊之,因誡通曰:「待兵交即放煙,吾當率騎軍南下。」兵才接,太宗以騎沖之,挺身先進,與通表裏相應。賊眾殊死戰,散而復合者數焉。自辰及午,賊眾始退。縱兵乘之,俘斬八千人,於是進營城下。世充不敢復出,但嬰城自守,以待建德之援。太宗遣諸軍掘塹,匝布長圍以守之。吳王杜伏威遣其將陳正通、
 徐召宗率精兵二千來會於軍所。偽鄭州司馬沈悅以武牢降,將軍王君廓應之,擒其偽荊王王行本。會竇建德以兵十餘萬來援世充,至於酸棗。蕭瑀、屈突通、封德彞皆以腹背受敵,恐非萬全,請退師穀州以觀之。太宗曰:「世充糧盡,內外離心,我當不勞攻擊,坐收其敝。建德新破孟海公,將驕卒惰,吾當進據武牢,扼其襟要。賊若冒險與我爭鋒,破之必矣。如其不戰,旬日間世充當自潰。若不速進,賊入武牢,諸城新附,必不能守。二賊並力,
 將若之何?」通又請解圍就險以候其變,太宗不許。於是留通輔齊王元吉以圍世充,親率步騎三千五百人趣武牢。



 建德自滎陽西上,築壘於板渚,太宗屯武牢,相持二十餘日。諜者曰:「建德伺官軍芻盡,候牧馬於河北,因將襲武牢。」太宗知其謀,遂牧馬河北以誘之。詰朝,建德果悉眾而至,陳兵氾水,世充將郭士衡陣於其南,綿互數里,鼓噪,諸將大懼。太宗將數騎升高丘以望之,謂諸將曰:「賊起山東,未見大敵。今度險而囂,是無政令;逼城
 而陣,有輕我心。我按兵不出,彼乃氣衰,陣久卒饑,必將自退,追而擊之,無往不克。吾與公等約,必以午時後破之。」建德列陣,自辰至午,兵士饑倦,皆坐列,又爭飲水,逡巡斂退。太宗曰:「可擊矣!」親率輕騎追而誘之,眾繼至。建德回師而陣,未及整列,太宗先登擊之,所向皆靡。俄而眾軍合戰,囂塵四起。太宗率史大奈、程咬金、秦叔寶、宇文歆等揮幡而入,直突出其陣後,張我旗幟。賊顧見之,大潰。追奔三十里,斬首三千餘級,虜其眾五萬,生擒建
 德於陣。太宗數之曰:「我以干戈問罪,本在王世充,得失存亡,不預汝事,何故越境,犯我兵鋒?」建德股慄而言曰:「今若不來,恐勞遠取。」高祖聞而大悅,手詔曰;「隋氏分崩,崤函隔絕。兩雄合勢,一朝清蕩。兵既克捷,更無死傷。無愧為臣,不憂其父,並汝功也。」乃將建德至東都城下。世充懼,率其官屬二千餘人詣軍門請降,山東悉平。太宗入據宮城,令蕭瑀、竇軌等封守府庫,一無所取,令記室房玄齡收隋圖籍。於是誅其同惡段達等五十餘人,枉
 被囚禁者悉釋之,非罪誅戮者祭而誄之。大饗將士,班賜有差。高祖令尚書左僕射裴寂勞於軍中。



 六月,凱旋。太宗親披黃金甲,陣鐵馬一萬騎,甲士三萬人,前後部鼓吹,俘二偽主及隋氏器物輦輅獻於太廟。高祖大悅,行飲至禮以享焉。高祖以自古舊官不稱殊功,乃別表徽號,用旌勛德。



 十月,加號天策上將、陜東道大行臺,位在王公上。增邑二萬戶,通前三萬戶。賜金輅一乘,袞冕之服,玉璧一雙,黃金六千斤,前後部鼓吹及九部之樂,
 班劍四十人。於時海內漸平,太宗乃銳意經籍,開文學館以待四方之士。行臺司勛郎中杜如晦等十有八人為學士,每更直閣下,降以溫顏,與之討論經義,或夜分而罷。未幾,竇建德舊將劉黑闥舉兵反,據洺州。



 十二月,太宗總戎東討。五年正月,進軍肥鄉,分兵絕其糧道,相持兩月。黑闥窘急求戰,率步騎二萬,南渡洺水,晨壓官軍。太宗親率精騎,擊其馬軍,破之,乘勝蹂其步卒,賊大潰,斬首萬餘級。先是,太宗遣堰洺水上流使淺,令黑闥
 得渡。及戰,乃令決堰,水大至,深丈餘,賊徒既敗,赴水者皆溺死焉。黑闥與二百餘騎北走突厥,悉虜其眾,河北平。時徐圓朗阻兵徐、兗,太宗回師討平之,於是河、濟、江、淮諸郡邑皆平。十月,加左右十二衛大將軍。



 七年秋,突厥頡利、突利二可汗自原州入寇,侵擾關中。有說高祖云:「只為府藏子女在京師,故突厥來,若燒卻長安而不都,則胡寇自止。」高祖乃遣中書侍郎宇文士及行山南可居之地,即欲移都。蕭瑀等皆以為非,然終不敢犯顏正諫。
 太宗獨曰:「霍去病,漢廷之將帥耳,猶且志滅匈奴。臣忝備籓維,尚使胡塵不息,遂令陛下議欲遷都,此臣之責也。幸乞聽臣一申微效,取彼頡利。若一兩年間不系其頸,徐建移都之策,臣當不敢復言」。高祖怒,仍遣太宗將三十餘騎行刬。還日,固奏必不可移都,高祖遂止。八年,加中書令。



 九年,皇太子建成、齊王元吉謀害太宗。六月四日,太宗率長孫無忌、尉遲敬德、房玄齡、杜如晦、宇文士及、高士廉、侯君集、程知節、秦叔寶、段志玄、屈突通、張
 士貴等於玄武門誅之。甲子,立為皇太子,庶政皆斷決。太宗乃縱禁苑所養鷹犬,並停諸方所進珍異,政尚簡肅,天下大悅。又令百官各上封事,備陳安人理國之要。己巳,令曰:「依禮,二名不偏諱。近代已來,兩字兼避,廢闕已多,率意而行,有違經典。其官號、人名、公私文籍,有『世民』兩字不連續者,並不須諱。」罷幽州大都督府。辛未,廢陜東道大行臺,置洛州都督府,廢益州道行臺,置益州大都督府。壬午,幽州大都督廬江王瑗謀逆,廢為庶人。
 乙酉,罷天策府。七月壬辰,太子左庶子高士廉為侍中,右庶子房玄齡為中書令,尚書右僕射蕭瑀為尚書左僕射,吏部尚書楊恭仁為雍州牧,太子左庶子長孫無忌為吏部尚書,右庶子杜如晦為兵部尚書,太子詹事宇文士及為中書令,封德彞為尚書右僕射。



 八月癸亥,高祖傳位於皇太子,太宗即位於東宮顯德殿。遣司空、魏國公裴寂柴告於南郊。大赦天下。武德元年以來責情流配者並放還。文武官五品已上先無爵者賜爵一
 級,六品已下加勛一轉。天下給復一年。癸酉,放掖庭宮女三千餘人。甲戌,突厥頡利、突利寇涇州。乙亥,突厥進寇武功,京師戒嚴。丙子,立妃長孫氏為皇后。己卯,突厥寇高陵。辛巳,行軍總管尉遲敬德與突厥戰於涇陽,大破之,斬首千餘級。癸未,突厥頡利至於渭水便橋之北,遣其酋帥執失思力入朝為覘,自張形勢,太宗命囚之。親出玄武門,馳六騎幸渭水上,與頡利隔津而語,責以負約。俄而眾軍繼至,頡利見軍容既盛,又知思力就拘,
 由是大懼,遂請和,詔許焉。即日還宮。乙酉,又幸便橋,與頡利刑白馬設盟,突厥引退。九月丙戌,頡利獻馬三千匹、羊萬口,帝不受,令頡利歸所掠中國戶口。丁未,引諸衛騎兵統將等習射於顯德殿庭,謂將軍已下曰:「自古突厥與中國更有盛衰。若軒轅善用五兵,即能北逐獯鬻;周宣驅馳方、召,亦能制勝太原。至漢、晉之君,逮於隋代,不使兵士素習干戈,突厥來侵,莫能抗禦,致遺中國生民塗炭於寇手。我今不使汝等穿池築苑,造諸淫費,
 農民恣令逸樂,兵士唯習弓馬,庶使汝鬥戰,亦望汝前無橫敵。」於是每日引數百人於殿前教射,帝親自臨試,射中者隨賞弓刀、布帛。朝臣多有諫者,曰:「先王制法,有以兵刃至御所者刑之,所以防萌杜漸,備不虞也。今引裨卒之人,彎弧縱矢於軒陛之側,陛下親在其間,正恐禍出非意,非所以為社稷計也。」上不納。自是後,士卒皆為精銳。壬子,詔私家不得輒立妖神,妄設淫祀,非禮祠禱,一皆禁絕。其龜易五兆之外,諸雜占卜,亦皆停斷。長
 孫無忌封齊國公,房玄齡邢國公,尉遲敬德吳國公,杜如晦蔡國公,侯君集潞國公。



 冬十月丙辰朔,日有蝕之。癸亥,立中山王承乾為皇太子。癸酉,裴寂食實封一千五百戶,長孫無忌、王君廓、尉遲敬德、房玄齡、杜如晦一千三百戶,長孫順德、柴紹、羅藝、趙郡王孝恭一千二百戶,侯君集、張公謹、劉師立一千戶,李世勣、劉弘基九百戶,高士廉、宇文士及、秦叔寶、程知節七百戶,安興貴、安修仁、唐儉、竇軌、屈突通、蕭瑀、封德彞、劉義節六百戶,錢
 九隴、樊世興、公孫武達、李孟常、段志玄、龐卿惲、張亮、李藥師、杜淹、元仲文四百戶,張長遜、張平高、李安遠、李子和、秦行師、馬三寶三百戶。十一月庚寅,降宗室封郡王者並為縣公。十二月癸酉,親錄囚徒。是歲,新羅、龜茲、突厥、高麗、百濟、黨項並遣使朝貢。



 貞觀元年春正月乙酉,改元。辛丑,燕郡王李藝據涇州反,尋為左右所斬,傳首京師。庚午,以僕射竇軌為益州大都督。三月癸巳,皇后親蠶。尚書左僕射、宋國公蕭瑀
 為太子少師。丙午,詔:「齊故尚書僕射崔季舒、給事黃門侍郎郭遵、尚書右丞封孝琰等,昔仕鄴中,名位通顯,志存忠讜,抗表極言,無救社稷之亡,遂見龍逢之酷。其季舒子剛、遵子雲、孝琰子君遵,並以門遭時譴,淫刑濫及。宜從褒獎,特異常倫,可免內侍,量才別敘。」



 夏四月癸巳,涼州都督、長樂王幼良有罪伏誅。六月辛巳,尚書右僕射、密國公封德彞薨。壬辰,太子少保宋國公蕭瑀為尚書左僕射。是夏,山東諸州大旱,令所在賑恤,無出今年
 租賦。秋七月壬子,吏部尚書、齊國公長孫無忌為尚書右僕射。八月戊戌,貶侍中、義興郡公高士廉為安州大都督。戶部尚書裴矩卒。是月,關東及河南、隴右沿邊諸州霜害秋稼。



 九月辛酉,命中書侍郎溫彥博、尚書右丞魏徵等分往諸州賑恤。中書令、郢國公宇文士及為殿中監。御史大夫、檢校吏部尚書、參預朝政、安吉郡公杜淹署位。十二月壬午,上謂侍臣曰:「神仙事本虛妄,空有其名。秦始皇非分愛好,遂為方士所詐,乃遣童男女數
 千人隨徐福入海求仙藥,方士避秦苛虐,因留不歸。始皇猶海側踟躕以待之,還至沙丘而死。漢武帝為求仙,乃將女嫁道術人,事既無驗,便行誅戮。據此二事,神仙不煩妄求也。」尚書左僕射、宋國公蕭瑀坐事免。戊申,利州都督義安王孝常、右武衛將軍劉德裕等謀反,伏誅。是歲,關中饑,至有鬻男女者。



 二年春正月辛丑,尚書右僕射、齊國公長孫無忌為開府儀同三司。徙封漢王屬為恪王,衛王泰為越王,楚王
 祐為燕王。復置六侍郎,副六尚書事,並置左右司郎中各一人。前安州大都督、趙王元景為雍州牧,蜀王恪為益州大都督,越王泰為揚州大都督。二月丙戌,靺鞨內屬。三月戊申朔,日有蝕之。丁卯,遣御史大夫杜淹巡關內諸州。出御府金寶,贖男女自賣者還其父母。庚午,大赦天下。



 夏四月己卯,詔骸骨暴露者,令所在埋瘞。丙申,契丹內屬。初詔天下州縣並置義倉。夏州賊帥梁師都為其從父弟洛仁所殺,以城降。五月,大雨雹。六月庚寅,
 皇子治生,宴五品以上,賜帛有差,仍賜天下是日生者粟。辛卯,上謂侍臣曰:「君雖不君,臣不可以不臣。裴虔通,煬帝舊左右也,而親為亂首。朕方崇獎敬義,豈可猶使宰民訓俗。」詔曰:



 天地定位,君臣之義以彰;卑高既陳,人倫之道斯著。是用篤厚風俗,化成天下。雖復時經治亂,主或昏明,疾風勁草,芬芳無絕,剖心焚體,赴蹈如歸。夫豈不愛七尺之軀,重百年之命?諒由君臣義重,名教所先,故能明大節於當時,立清風於身後。至如趙高之殞
 二世,董卓之鴆弘農,人神所疾,異代同憤。況凡庸小豎,有懷兇悖,遐觀典策,莫不誅夷。辰州刺史、長蛇縣男裴虔通,昔在隋代,委質晉籓,煬帝以舊邸之情,特相愛幸。遂乃志蔑君親,潛圖弒逆,密伺間隙,招結群醜,長戟流矢,一朝竊發。天下之惡,孰云可忍!宜其夷宗焚首,以彰大戮。但年代異時,累逢赦令,可特免極刑,除名削爵,遷配驩州。



 秋七月戊申,詔:「萊州刺史牛方裕、絳州刺史薛世良、廣州都督府長史唐奉義、隋武牙郎將高元禮,並
 於隋代俱蒙任用,乃協契宇文化及,構成弒逆。宜依裴虔通,除名配流嶺表。」太宗謂侍臣曰:「天下愚人,好犯憲章,凡赦宥之恩,唯及不軌之輩。古語曰:『小人之幸,君子之不幸。』『一歲再赦,好人喑啞。』凡養稂莠者傷禾稼,惠奸宄者賊良人。昔文王作罰,刑茲無赦。又蜀先主嘗謂諸葛亮曰:『吾周旋陳元方、鄭康成間,每見啟告理亂之道備矣,曾不語赦也。』夫小人者,大人之賊,故朕有天下已來,不甚放赦。今四海安靜,禮義興行,非常之恩,施不可
 數,將恐愚人常冀僥幸,唯欲犯法,不能改過。」八月甲戌朔,幸朝堂,親覽冤屈。自是,上以軍國無事,每日視膳於西宮。癸巳,公卿奏曰:「依禮,季夏之月,可以居臺榭。今隆暑未退,秋霖方始,宮中卑濕,請營一閣以居之。」帝曰:「朕有氣病,豈宜下濕。若遂來請,糜費良多。昔漢文帝將起露臺,而惜十家之產。朕德不逮於漢帝,而所費過之,豈謂為民父母之道也。」竟不許。是月,河南、河北大霜,人饑。



 九月丙午,詔曰:「尚齒重舊,先王以之垂範;還章解組,朝
 臣於是克終。釋菜合樂之儀,東膠西序之制,養老之義,遺文可睹。朕恭膺大寶,憲章故實,乞言尊事,彌切深衷。然情存今古,世踵澆季,而策名就列,或乖大體。至若筋力將盡,桑榆且迫,徒竭夙興之勤,未悟夜行之罪。其有心驚止足,行堪激勵,謝事公門,收骸閭里,能以禮讓,固可嘉焉。內外文武群官年高致仕、抗表去職者,參朝之日,宜在本品見任之上。」丁未,謂侍臣曰:「婦人幽閉深宮,情實可愍。隋氏末年,求採無已,至於離宮別館,非幸御
 之所,多聚宮人,皆竭人財力,朕所不取。且灑掃之餘,更何所用?今將出之,任求伉儷,非獨以惜費,亦人得各遂其性。」於是遣尚書左丞戴胄、給事中杜正倫等,於掖庭宮西門簡出之。



 冬十月庚辰,御史大夫、安吉郡公杜淹卒。戊子,殺瀛州刺史盧祖尚。十一月辛酉,有事於圓丘。十二月壬午,黃門侍郎王珪為侍中。



 三年春正月辛亥,契丹渠帥來朝。戊午,謁太廟。癸亥,親耕籍田。辛未,司空、魏國公裴寂坐事免。二月戊寅,中書
 令、邢國公房玄齡為尚書左僕射,兵部尚書、檢校侍中、蔡國公杜如晦為尚書右僕射,刑部尚書、檢校中書令、永康縣公李靖為兵部尚書,右丞魏徵為守秘書監,參預朝政。



 夏四月辛巳,太上皇徙居大安宮。甲子,太宗始於太極殿聽政。五月,周王元方薨。六月戊寅,以旱,親錄囚徒。遣長孫無忌、房玄齡等祈雨於名山大川,中書舍人杜正倫等往關內諸州慰撫。又令文武官各上封事,極言得失。已卯,大風折木。秋八月己巳朔,日有蝕之。薛
 延陀遣使朝貢。



 九月癸丑,諸州置醫學。冬十一月丙午,西突厥、高昌遣使朝貢。庚申,以並州都督李世勣為通漢道行軍總管,兵部尚書李靖為定襄道行軍總管,以擊突厥。十二月戊辰,突利可汗來奔。癸未,杜如晦以疾辭位,許之。癸丑,詔建義以來交兵之處,為義士勇夫殞身戎陣者各立一寺,命虞世南、李伯藥、褚亮、顏師古、岑文本、許敬宗、硃子奢等為之碑銘,以紀功業。是歲,戶部奏言:中國人自塞外來歸及突厥前後內附、開四夷為
 州縣者,男女一百二十餘萬口。



\end{pinyinscope}