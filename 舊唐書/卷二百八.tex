\article{卷二百八}

\begin{pinyinscope}

 永泰二年二月,命大理少卿、兼御史中丞楊濟修好於吐蕃。四月,吐蕃遣首領論泣藏等百餘人隨濟來朝,且謝申好。大歷二年十月,靈州破吐蕃二萬餘眾,生擒五
 百人,獲馬一千五百匹。十一月,和蕃使、檢校戶部尚書、兼御史大夫薛景仙自吐蕃使還,首領論泣陵隨景仙來朝。景仙奏云:「贊普請以鳳林關為界。」俄又遣使路悉等十五人來朝。三年八月,吐蕃十萬寇靈武,大將尚悉摩寇邠州。邠寧節度使馬璘破二萬餘眾,擒其俘以獻之。九月,寇靈州,朔方騎將白元光破之。俄又復破二萬眾於婁武,獲羊馬數千計。關內副元帥郭子儀於靈州破吐蕃六萬餘眾。十二月,以蕃寇歲犯西疆,增修鎮守,
 乃移馬璘鎮涇州,仍為涇原節度使。劍南西川亦破吐蕃萬餘眾。五年五月,徙置安、悉、拓、靜、恭五州於山陵要害之地,以備吐蕃。



 八年秋,吐蕃六萬騎寇靈武,蹂踐我禾稼而去。十月,寇涇、邠等州,郭子儀遣先鋒將渾瑊與賊戰於宜祿,我師不利,副將史籍等三人死之,村墅居人為驅掠者凡千餘人。是夜,瑊收合散卒襲賊營,會馬璘亦襲其輜重,凡殺數千人,賊遂潰。子儀大破吐蕃十餘萬眾。



 初,吐蕃犯我邠郊,馬璘以精卒二千餘人潛夜
 掩賊營,射賊豹皮將中目,賊眾扶之號泣,遂舉營遁去。璘因收獲朔方兵健二百餘人,百姓七百餘人,駝馬數百匹。



 九年四月,以吐蕃侵擾,預為邊備,乃降敕:



 宜令子儀以上郡、北地、四塞、五原、義渠、稽胡、鮮卑雜種步馬五萬眾,嚴會栒邑,克壯舊軍。抱玉以晉之高都,韓之上黨,河、湟義徒,汧、隴少年,凡三萬眾,橫絕高壁,斜界連營。馬璘以西域前庭,車師後部,兼廣武之戍,下蔡之徭,凡三萬眾,屯於泗中,張大軍之援。忠誠以武落別授,右地奇
 鋒,凡二萬眾,出岐陽而北會。希讓以三輔太常之徒,六郡良家之子,自渭上而西合汴宋、淄青、河陽、幽薊,總四萬眾,分列前後。魏博、成德、昭義、永平總六萬眾,大舒左右。朕內整禁旅,親誓諸將,資以千金之費,錫以六牧之馬。其戎裝戰器,軍用邊儲,各有司存,素皆精辦。咨爾將相文武宣力之臣,夫師克在和,善戰不陣,各宜保據疆界,屯據要沖,斥堠惟明,首尾相應。若既悔過,何必勞人;如或不恭,自當伐罪。然後眷求統一,以制諸軍。進取之宜,
 俟於後命。



 十一年正月,劍南節度使崔寧大破吐蕃故洪等四節度兼突厥、吐渾、氐、蠻、羌、黨項等二十餘萬眾,斬首萬餘級,生擒葛城兵馬使一千三百五十人,獻於闕下。牛羊及軍資器械,不可勝紀。十二年九月,入寇坊州,掠黨項羊馬而去。十月,崔寧破吐蕃望漢城。十四年八月,命太常少卿韋倫持節使吐蕃,統蕃俘五百人歸之。十月,吐蕃率南蠻眾二十萬來寇:一入茂州,過汶川及灌口;一入扶、文,過方維、白壩;一自黎、雅過邛峽關,連
 陷郡邑。乃發禁兵四千人及幽州兵五千人同討,大破之。



 建中元年四月,韋倫至。自大歷中聘使前後數輩,皆留之不遣。俘獲其人,必遣中官部統徙江、嶺,因緣求財及給養之費,不勝其弊。去年冬,吐蕃大興師以三道來侵,會德宗初即位;以德綏四方,徵其俘囚五百餘人,各給衣一襲,使倫統還其國,與之約和,敕邊將無得侵伐。吐蕃始聞歸其人,不之信,及蕃俘入境,部落皆畏威懷惠。其贊普乞立贊謂倫曰:「不知是來也,而有三恨,奈何?」
 倫曰:「未達所謂。」乞立贊曰:「不知大國之喪,而吊不及哀,一也。不知山陵之期,而賻不成禮。二也。不知皇帝舅聖明繼立,已發眾軍三道連衡。今靈武之師,聞命輒已;而山南之師已入扶、文,蜀師已趨灌口,追且不及,是三恨也。」乃發使奉贄,不二旬而復命。蜀帥上所獲戎俘,有司請準舊事頒為徒隸,上曰:「要約著矣,言庸二乎?」乃各給縑二匹、衣一襲而歸之。五月,以韋倫為太常卿,復使吐蕃。其冬,遣宰相論欽明思等五十五人隨倫至,且獻方
 物。吐蕃見倫再至,甚歡。既就館,聲樂以娛之,留九日而還,兼遣其渠帥報命。



 二年十二月,入蕃使判官常魯與吐蕃使論悉諾羅等至自蕃中。初,魯與其使崔漢衡至列館,贊普令止之,先命取國信敕。既而使謂漢衡曰:「來敕云:『所貢獻物,並領訖;今賜外甥少信物,至領取。』我大蕃與唐舅甥國耳,何得以臣禮見處?又所欲定界,雲州之西,請以賀蘭山為界。其盟約,請依景龍二年敕書云:『唐使到彼,外甥先與盟誓;蕃使到此,阿舅亦親與盟。』」乃
 邀漢衡遣使奏定。魯使還奏焉,為改敕書,以「貢獻」為「進」,以「賜」為「寄」,以「領取」為「領之」。且謂曰:「前相楊炎不循故事,致此誤爾。」其定界盟,並從之。



 三年四月,放先沒蕃將士僧尼等八百人歸還,報歸蕃俘也。九月,和蕃使、殿中少監、兼御史中丞崔漢衡與蕃使區類贊至。時吐蕃大相尚結息忍而好殺,以嘗覆敗於劍南,思刷其恥,不肯約和。其次相尚結贊有材略,因言於贊普,請定界明約,以息邊人。贊普然之,竟以結贊代結息為大相,終約和好,
 期以十月十五日會盟於境上。以崔漢衡為鴻臚卿,以都官員外郎樊澤兼御史中丞、充入蕃計會使。初,漢衡與吐蕃約定月日盟誓,漢衡到,商量未決,已過其期,遂命澤詣結贊復定盟會期,且告遣隴右節度使張鎰與之同盟,澤至故原州,與結贊相見,以來年正月十五日會盟於清水西。



 四年正月,詔張鎰與尚結贊盟於清水。將盟,鎰與結贊約,各以二千人赴壇所,執兵者半之,列於壇外二百步,散從者半之,分立壇下。鎰與賓佐齊映、
 齊抗及會盟官崔漢衡、樊澤、常魯、於



 頔等七人皆朝服;結贊與其本國將相論悉頰藏、論臧熱、論利陀、斯官者、論力徐等亦七人,俱升壇為盟。初約漢以牛,蕃以馬,鎰恥與之盟,將殺其禮,乃謂結贊曰:「漢非牛不田,蕃非馬不行,今請以羊、豕、犬三物代之。」結贊許諾。塞外無豕,結贊請出羝羊,鎰出犬及羊,乃於壇北刑之,雜血二器而歃盟。文曰:



 唐有天下,恢奄禹跡,舟車所至,莫不率俾。以累聖重光,歷年惟永,彰王者之丕業,被四海之聲教。與吐
 蕃贊普,代為婚姻,固結鄰好,安危同體,甥舅之國,將二百年。其間或因小忿,棄惠為讎,封疆騷然,靡有寧歲。皇帝踐祚,愍茲黎元,俾釋俘隸,以歸蕃落。蕃國展禮,同茲葉和,行人往復,累布成命。是必詐謀不起,兵車不用矣。彼猶以兩國之要,求之永久,古有結盟,今請用之。國家務息邊人,外其故地,棄利蹈義,堅盟從約。今國家所守界:涇州西至彈箏峽西口,隴州西至清水縣,鳳州西至同谷縣,暨劍南西山大渡河東,為漢界。蕃國守鎮在蘭、
 渭、原、會,西至臨洮,東至成州,抵劍南西界磨些諸蠻,大渡水西南,為蕃界。其兵馬鎮守之處,州縣見有居人,彼此兩邊見屬漢諸蠻,以今所分見住處,依前為定。其黃河以北,從故新泉軍,直北至大磧,直南至賀蘭山駱駝嶺為界,中間悉為閑田。盟文有所不載者,蕃有兵馬處蕃守,漢有兵馬處漢守,並依見守,不得侵越。其先未有兵馬處,不得新置,並築城堡耕種。今二國將相受辭而會,齊戒將事,告天地山川之神,惟神照臨,無得愆墜。其
 盟文藏於宗廟,副在有司,二國之成,其永保之。



 結贊亦出盟文,不加於坎,但埋牲而已。盟畢,結贊請鎰就壇之西南隅佛幄中焚香為誓。誓畢,復升壇飲酒。獻酬之禮,各用其物,以將厚意而歸。



 二月,命崔漢衡持節答蕃,遣區頰贊等歸。上初令宰相、尚書與蕃相區頰贊盟於豐邑里壇所。將盟,以清水之會疆埸不定,遂罷。因留頰贊未遣,復令漢衡使於贊普。六月,答蕃使判官于頔與蕃使論頰沒藏等至自青海。七月,以禮部尚書李揆加御史大
 夫,為入蕃會盟使。又命宰相李忠臣、盧杞、關播、右僕射崔寧、工部尚書喬琳、御史大夫於頎、太府卿張獻恭、司農卿段秀實、少府監李昌夔、京兆尹王翃、左金吾衛將軍渾瑊等與區頰贊等會盟於壇所。初,于頔至自蕃中,與尚結贊約:「疆場既定,請歸其使。」從之。以豐邑坊盟壇在京城之內非便,請卜壇於京城之西。其禮如清水之儀。先盟二日,命有司告太廟,監官致齋。三日,朝服升壇,關播跪讀盟文。盟畢,宴賜而遣之。



 興元元年二月,以右
 散騎常侍兼御史大夫於頎往涇州已來宣慰吐蕃,仍與州府計會頓遞。時吐蕃款塞請以兵助平國難,故遣使焉。四月,命太常少卿、兼御史中丞沈房為入蕃計會及安西、北庭宣慰使。是月,渾瑊與吐蕃論莽羅率眾大破硃泚將韓旻、張廷芝、宋歸朝等於武功之武亭川,斬首萬餘級。



 貞元二年,命倉部郎中、兼侍御史趙聿為入吐蕃使。八月,吐蕃寇涇、隴、邠、寧數道,掠人畜,取禾稼,西境騷然。諸道節度及軍鎮,咸閉壁自守而已。京師戒嚴。
 上遣左金吾將軍張獻甫與神策將李升曇、蘇清沔等統兵屯於咸陽,召河中節度駱元光率眾戍咸陽以援之。九月,以吐蕃游騎及於好畤,上復遣張獻甫等統兵屯於咸陽,又詔遣左監門將軍康成使於吐蕃。初,吐蕃大相尚結贊累遣使請盟會定界,乃命成使之。至上砦原,與結贊相見,令其使論乞陀與成同來。



 是月,鳳翔節度使李晟以吐蕃侵軼,遣其將王佖夜襲賊營,率驍勇三千人入汧陽。誡之曰:「賊之大眾,當過城下,無擊其首
 尾。首尾雖敗,中軍力全,若合勢攻之,汝必受其弊。但候其前軍已過。見五方旗、虎豹衣,則其中軍也。出其不意,乃是奇功。」佖如其言出擊之,賊眾果敗,副將史廷玉力戰死之。又寇鳳翔城下,李晟出兵御之,一夕而退。十月,李晟遣兵襲吐蕃之沙堡,大破之。焚其歸積,斬蕃酋扈屈律設贊等七人,傳首京師。



 十一月,吐蕃陷鹽州。初,賊之來也,刺史杜彥光使以牛酒犒之。吐蕃謂曰:「我欲州城居之,聽爾率其人而去。」彥光乃悉眾奔鄜州。十二月,
 陷夏州,刺史拓拔乾暉率眾而去,復據其城。又寇銀州,素無城壁,人皆奔散。



 三年春,命檢校左庶子、兼御史中丞崔浣為入吐蕃使,相次又遣左庶子李銛使之。河東、保寧等道節度使馬燧來朝。初,尚結贊既陷鹽、夏等州,各留千餘人守之,結贊大眾屯於鳴沙。自去冬及春,羊馬多死,糧餉不給。時詔遣華州、潼關節度駱元光、邠寧節度韓游瑰統眾與鳳翔、鄜、邠及諸道戍卒,屯於塞上,又命燧率師次於石州,分兵隔河與元光等掎角討之。
 結贊聞而大懼,累遣使請和,仍約盟會。上皆不許。又遣其大將論頰熱厚禮卑詞求燧請盟,燧以奏焉,上又不許。惟促其合勢討逐。燧喜賂信詐,乃與頰熱俱入朝,盛言其可保信,許盟約,上於是從之。燧既赴朝也,諸軍但閉壁而已。結贊遽悉其眾棄夏州而歸,馬既多死,有徒行者。及是夏平涼之會,竟渝盟,馬燧亦由此失兵柄而奉朝請矣。



 四月,崔浣至自鳴沙。初,浣至鳴沙,與尚結贊相見,詢問其違約陷鹽、夏州之故。對曰:



 本以定界碑被
 牽倒,恐二國背盟相侵,故造境上請修舊好。又蕃軍頃年破硃泚之眾於武功,未獲酬人賞,所以來耳。及徙涇州,其節度使閉城自守,音問莫達。又徙鳳翔,請通使於李令公,亦不見納。及遣康成、王真之來,皆不能達大國之命。日望大臣充使,兼展情禮,實無至者,乃引軍還。及鹽、夏二州之師,二州懼我之眾,請以城與我,求全而歸,非我所攻陷也。今君以國親將命,若結好復盟。蕃之願也。盟會之期及定界之所,唯命是聽。君歸奏決,定當以鹽、
 夏相還也。



 又云:



 清水之會,同盟者少,是以和好輕慢不成。今蕃相及元帥已下凡二十一人赴。靈州節度使杜希全稟性和善,外境所知,請令主盟會。涇州節度李觀,亦請同主之。



 又同章表上聞。浣誘賂蕃中給役者,求其人馬真數,凡五萬九千餘人、馬八萬六千餘匹,可戰者僅三萬人,餘悉童幼,備數而已。



 是日,改崔浣為鴻臚卿,再入吐蕃。令浣報尚結贊曰:「杜希全職在靈州,不可出境。李觀今已改官,以侍中渾瑊充盟會使。」約以五月二
 十四日復盟於清水。又令告以鹽、夏二州歸於我,才就盟會。上疑蕃情不實,以得州為信焉。五月,渾瑊以充盟會使來辭,且受命。以兵部尚書崔漢衡為盟會副使,司勛員外郎鄭叔矩為判官。渾瑊赴會盟所,上令瑊統眾二萬餘人,遣華州潼關節度駱元光赴之。上令宰臣召吐蕃使論泣贊等於中書議會盟之所。



 初崔浣與尚結贊約復會於清水,且先歸我鹽、夏二州,結贊云:「清水非吉地,請會於原州之土梨樹。」又請盟畢歸二州。浣遣使
 與泣贊等同奏,上務懷柔遠人,皆從之。約以五月十五日盟於土梨樹,上召宰臣謀之。先是左神策將馬有麟奏:「土梨樹地多險隘,恐蕃軍隱伏,不利於我。平涼川四隅坦平,且近涇州。就之為便。」由是乃定盟所於平涼川。時蕃使論泣贊已復命,遽追還,告而遣之。



 渾瑊與尚結贊會於平涼。初,瑊與結贊約,以兵三千人列於壇之東西,散手四百人至壇下。及將盟,又約各益游軍相覘伺。結贊擁精騎數萬於壇西,蕃之游軍貫穿我師。瑊之將
 梁奉貞率六十騎為游軍,才至蕃中,皆被執留,瑊不虞也。結贊又遣人請瑊曰:「請侍中以下服衣冠劍珮以俟命。」蓋誘其下馬,將劫持之。瑊與崔漢衡、監軍特進宋鳳朝等皆入幕次,坦無他慮,結贊命伐鼓三聲,其眾呼噪而至。瑊遽出自幕後,偶得他馬,跨而奔歸。時馬不加銜,瑊伏於鬣而手加之,凡馳十餘里,銜方及口,故追騎之矢,過而不傷焉。唯瑊之裨將辛榮招合數百人,據北阜與賊接戰,須臾賊眾四合,榮力屈而降。鳳朝及瑊判官
 韓弇,並為亂兵所殺。漢衡及中官劉延邕、俱文珍、李清朝,漢衡判官鄭叔矩、路泌,掌書記袁同直,大將扶餘準、馬寧及神策、鳳翔、河東大將孟日華、李至言、樂演明、範澄、馬弇等六十餘人皆陷焉。餘將士及夫役死者四五百人,驅掠者千餘人,咸被解奪其衣。



 初,漢衡為亂軍所擊,其從吏呂溫以身蔽之,刃中溫而漢衡獲免。漢衡乃夷言謂執者曰:「我漢使崔尚書也,結贊與我善,如若殺我,結贊亦殺汝。」乃拾之,盡驅而西。既已面縛,各以一木
 自領至趾約於身,以毛繩三束之,又以毛繩連其發而約之。夜皆踣於地,以發繩各系一橛,又以毛罽都覆之,守衛者臥其上,以防其亡逸也。至故原州,結贊坐於帳中,召與相見,數讓國家,因怒渾瑊曰:「武功之捷,皆我之力,許以涇州、靈州相報,皆食其言。負我深矣,舉國所忿。本劫是盟,在擒瑊也。吾遣以金飾桎梏待瑊,將獻贊普。既以失之,虛致君等耳,當遣君輩三人歸也。呂溫帶瘡亦至,結贊嘉其義,厚給齎之。結贊率其眾於石門,遣中
 官俱文珍、渾瑊之將馬寧、馬燧之將馬弇歸於我。遂送漢衡、叔矩等囚於河州,辛榮、扶餘準等於故廓州、鄯州分囚之。結贊本請杜希全、李觀問盟,將執二節將,率其銳師來犯京師,希全等既不行,又欲執渾瑊長驅入寇,其謀也如此。上遣中官王子恆齎詔書以遺結贊,蕃界不納而還。



 初,瑊與駱元光將發涇州,元光謂瑊曰:「本奉詔令營於潘原堡,以應援侍中。竊以潘原去盟所六七十里,蕃情多詐,侍中倘有急,何由知之?請次侍中為營,
 以虞其變。」瑊以非詔旨,固止之。元光與同進。瑊之營西去盟所二十餘里,元光之營次之。其濠柵頗深固,瑊之濠柵可逾越焉。及瑊單騎奔歸,未及其營,守將李朝彩不能整眾,多已奔散。瑊至,空營而已,器械資糧悉棄之,賴元光之眾陣於營中,瑊既入,賊追騎方退。元光乃先遣輜重,次與瑊俱申其號令,嚴其部伍而還。瑊復鎮於奉天。



 六月,鹽、夏二州吐蕃焚城門及廬舍,毀城壁而歸。七月,詔曰:



 乃者吐蕃犯塞,毒我生靈,俶擾隴東,深入河
 曲。朕以兵戈粗定,傷夷未瘳,務息戰伐之謀,遂從通和之請。亦知戎醜,志在貪婪,重違修睦之辭,乃允尋盟之會。果為隱匿,變發壝宮,縱犬羊兇狡之群,乘文武信誠之眾,蒼黃淪陷,深用惻然。此皆由朕之不明,致其至此。既無德於萬眾,亦有愧於四方,宵旰貽憂,何嗟而及。今兵部尚書崔漢衡等,皆國之良士,朝之藎臣,嬰縶窮廬,眇然殊域。念其家室,或未周於屢空;錄以息男,庶或資於薄俸。漢衡宜與一子七品官,司勛員外郎鄭叔矩、檢
 校戶部郎中路泌、殿中侍御史韓弇及大將孟日華、辛榮、李至言、範澄、王良賁、樂演明、陽昔、權交成等,各與一子八品官;試左金吾兵曹參軍袁同直、榆次尉裴頲及副兵馬使以下,各與一子九品官。仍並與正員官。餘將幹各與一子官,仍委本使即具名銜聞奏。



 於是遣決勝軍使唐良臣以眾六百人戍潘原堡,神策副將蘇太平率其眾五百人戍隴州。



 八月,崔漢衡至自吐蕃。初,漢衡與同陷者並至河州,尚結贊令召漢衡與神策將孟日
 華、中官劉延邕,俱至石門而遣之。結贊令五十騎送至境上,且齎表請進。及潘原,李觀使止曰:「有詔不許更納蕃使。」受其表而返其人。自是吐蕃率羌、渾之眾犯塞,分屯於潘口及青石嶺。先是,吐蕃之眾自潘口東分為三道:其一趨隴州,其一趨汧陽之東,其一趨釣竿原。是日,相次屯於所趨之地,連營數十里。其汧陽賊營,距鳳翔四十里,京師震恐,士庶奔駭。賊遣羌、渾之眾,衣漢戎服,偽稱邢君牙之眾,奄至吳山及寶雞北界,焚燒廬舍,驅
 掠人畜,斷吳山神之首,百姓丁壯者驅之以歸,羸老者咸殺之,或斷手鑿目,棄之而去。初,李晟在鳳翔,令伐大木塞安化峽,及是,賊並焚之。



 九月,詔神策軍將石季章以眾三千戍武宮,召唐良臣自潘原戍百里城。是月,吐蕃大掠汧陽、吳山、華亭等界人庶男女萬餘口,悉送至安化峽西,將分隸羌、渾等。乃曰:「從爾輩東向哭辭鄉國。」眾遂大哭。其時一慟而絕者數百人,投崖谷死傷者千餘人,聞者為之痛心焉。渾瑊遣其將任蒙主以眾三千
 戍好疇。是月,吐蕃之眾復至,分屯於豐義及華亭。百僚入計以破吐蕃圍。隴州刺史韓清沔與蘇太平夜出兵伏於大像龕。及夜半,令城中及龕各舉火相應,賊大驚,因襲其營,賊乃退散。時吐蕃攻陷華亭。



 初,賊之圍華亭也,先絕其汲水道。其守將王仙鶴及鎮兵百姓凡三千人,皆在圍中。使人間道請救於隴州,刺史韓清沔令蘇太平率一千五百人赴之。及中路,其游騎百餘沒於賊,太平素懦怯寡謀,遽引眾退歸。賊自是每日令游騎千
 餘至隴州,州兵不敢復出。凡四日,圍中絕水,援軍不至,賊又積柴城下,將焚之,仙鶴遂降於賊。賊並焚廬舍,毀城壁,虜士眾十三四,收丁壯棄老而去。北攻連雲堡,又陷。堡之三面頗峭峻,唯北面連原,以濠為固。賊自其北建拋樓七具,擊堡中,堡中唯一井,投石俄而滿焉。又飛梁架濠而過,苦攻之。堡將張明遂與其眾男女千餘口東向慟哭而降。涇州之西,唯有連雲堡每偵候賊之進退,及是堡陷,涇州不敢啟西門,西門外皆為賊境,樵蘇
 殆絕,收刈禾稼,必布陣於野而收獲之。獲既失時,所得多空穗。於是涇人有饑憂焉。吐蕃驅掠連雲堡之眾及邠、涇編戶逃竄山谷者,並牛畜萬計,悉其眾送至彈箏峽。自是涇、隴、邠等賊之所至,俘掠殆盡。是秋,數州人無積聚者,邊將唯遣使表賀賊退而已。



 十月,吐蕃數千騎復至長武城,韓全義率眾御之。韓游瑰之將請以眾助之,游瑰不許。及暮,賊退,全義亦引還。自是賊之騎常往來涇、邠之間,諸城西門莫敢啟者。賊又修故原州城,其
 大眾屯焉。



 四年五月,吐蕃三萬餘騎犯塞,分入涇、邠、寧、慶、麟等州,焚彭原縣廨舍,所至燒廬舍,人畜沒者約二三萬,計凡二旬方退。陳許行營將韓全義自長武城率眾抗之,無功而還。游瑰素無軍政,且疾不能興,閉城自守,莫敢御也。先是,吐蕃入寇,恆以秋冬,及春則多遇疾疫而退。是來也,方盛暑而無患。蓋華人陷者,厚其資產,質其妻子,為戎虜所將而侵軼焉。九月,吐蕃將尚悉董星、論莽羅等寇寧州,節度使張獻甫率眾御之,斬首百
 餘級,賊轉寇麟坊等州,縱掠而去。



 五年十月,劍南節度使韋皋遣將王有道等與東蠻兩林苴那時、勿鄧夢沖等帥兵於故巂州臺登北穀大破吐蕃青海、獵城二節度,殺其大兵馬使乞臧遮遮、悉多楊硃,斬首二千餘級,其投崖谷赴水死者不可勝數,生擒籠官四十五人,收獲器械一萬餘事、馬牛羊一萬餘頭匹。遮遮者,吐蕃驍勇者也,或云尚結贊之子,頻為邊患。自其死也,官軍所攻城柵,無不降下。蕃眾日卻,數年間,盡復巂州之境。



 六
 年,吐蕃陷我北庭都護府。初,北庭、安西,既假道於回紇朝奏,因附庸焉。蕃性貪狠,徵求無度。北庭近羌,凡服用食物所資,必強取之,人不卿生矣。又有沙陀部六千餘帳與北庭相依,亦屬於回紇。回紇肆其抄奪,尤所厭苦。其葛祿部及白服突厥素與回紇通和,亦憾其奪掠,因吐蕃厚賂見誘。遂附之。於是吐蕃率葛祿、白服之眾,去歲各來寇北庭,回紇大相頡乾迦斯率眾援之,頻戰敗績,吐蕃攻圍頗急。北庭之人既苦回紇,是歲乃舉城降
 於吐蕃,沙陀部落亦降焉。北庭節度使楊襲古與麾下二千餘人出奔西州,頡乾迦斯不利而還。



 七年秋,又悉其丁壯五六萬人,將復北庭,仍召襲古偕行,俄為吐蕃、葛祿等所擊,大敗,死者大半。頡乾迦斯紿之曰:「且與我同至牙帳,當送君歸本朝也。」襲古從之,及牙帳,留而不遣,竟殺之。自是安西阻絕,莫知存否。唯西州之人,猶固守焉。頡乾迦斯既敗恤,慕祿之眾乘勝取回紇之浮圖川。回紇震恐,悉遷西北部落羊馬於牙帳之南以避之。



 八年四月,吐蕃寇靈州,掠人畜,攻陷水口城,進圍州城,塞水口及支渠以營田。詔河東、振武分兵為援,又分神策六軍之卒三千餘人戍於定遠、懷遠二城,上御神武樓勞遣之。吐蕃引去。六月,吐蕃數千騎由青石嶺寇涇州,掠田軍千餘人還,及連雲堡,守捉使唐朝臣遣兵出戰,大將王進用死之。九月,西川節度使韋皋攻吐蕃之維州,獲大將論贊熱及首領獻於京師。十一月,山南西道節度嚴震擊破吐蕃於芳州及黑水堡,焚其積聚,並
 獻首虜。



 九年二月,詔城鹽州。是州先為吐蕃所毀,自此塞外無堡障。靈武勢隔,西逼鄜坊,甚為邊患,故命城之,二旬而畢。又詔兼御史大夫紇干遂統兵五千與兼御史中丞杜彥光之眾戍之。是役也,上念將士之勞,厚令度支供給。又詔涇原、湖南、山南諸軍深討吐蕃,以分其力。由是板築之際,虜無犯塞者。及畢,中外咸稱賀焉。是月,西川韋皋獻獲吐蕃首虜。器械、旗幟、牛馬於闕下。



 初,將城鹽州,上命皋出師以分吐蕃之兵,皋遣大將董勔、
 張芬出西山及南道,破俄和城、通鶴軍。吐蕃南道元帥論莽熱率眾來援,又破之,殺傷數千人,焚定廉故城。凡平柵堡五十餘所。



 十年,南詔蠻蒙異牟尋大破吐蕃於神川,使來獻捷,語在《南詔傳》。十一年八月,黃少卿攻陷欽、橫、潯、貴四州,吐蕃渠帥論乞髯蕩沒藏悉諾律以其家屬來降。明年,並以為歸德將軍。十二年九月,吐蕃寇慶州及華池縣,殺傷頗甚。



 十三年正月,邢君牙奏請於隴州西七十里築城以備西戎,名永信城。吐蕃贊普遣
 使農桑昔齎表請修和好,邊將以聞。上以其豺狼之性,數負恩背約,不受表狀,任其使卻歸。五月十七日,吐蕃於劍山、馬嶺三處開路,分軍下營,僅經一月,進軍逼臺登城。巂州刺史曹高任率領諸軍將士並東蠻子弟合勢接戰,自朝至午,大破之,生擒大籠官七人,陣上殺獲三百人,余被刀箭者不可勝紀,收獲馬畜五百餘頭匹、器械二千餘事。十四年十月,夏州節度使韓全義破吐蕃於鹽州西北。十六年六月,鹽州破吐蕃於烏蘭橋下。



 十七年七月,吐蕃寇鹽州,又陷麟州,殺刺史郭鋒,毀城隍,大掠居人,驅黨項部落而去。次鹽州西九十里橫槽烽頓軍,呼延州僧延素輩七人,稱徐舍人召。其火隊吐蕃沒勒遽引延素等疾趨至帳前,皆馬革梏手,毛繩縲頸。見一吐蕃年少,身長六尺餘,赤髭大目,乃徐舍人也。命解縛,坐帳中,曰:「師勿懼。餘本漢人,司空英國公五代孫也。屬武後斫喪王室,高祖建義中泯,子孫流播絕域,今三代矣。雖代居職位,世掌兵要,思本之心無涯,顧血
 族無由自拔耳!此蕃、漢交境也,復九十里至安樂州,師無由歸東矣。」延素曰:「僧身孤親老,懇祈全活。」悲不自勝。又曰:「餘奉命率師備邊,因求資食,遂涉漢疆,展轉東進至麟州。城既無備,援兵又絕,是以拔之。知郭使君是勛臣子孫,必將活之,不幸為亂兵所害。」適有飛鳥使至,飛鳥,猶中國驛騎也,云:「術者上變,召軍亟還。」遂歸之。時詔韋皋分遣偏將勒步騎合二萬,出成都西山,南北九道並進,逼棲雞、老翁、故維州、保州、松州諸城,以紓北邊故
 也。九月,韋皋大破吐蕃於維州。



 十八年正月,韋皋擒吐蕃大首領論莽熱來獻,賜崇仁里宅以居之。莽熱,吐蕃內大相也。先貞元十六年,韋皋累破吐蕃二萬餘眾於黎州、巂州,吐蕃遂大搜閱,築壘造舟,潛謀寇邊,皋悉挫之。於是吐蕃酋帥兼監統曩貢、臘城等九節度嬰嬰、籠官馬定德與其大將八十七人,舉部落來降。定德有計畫,嬰嬰習知兵法及山川地形,吐蕃每用兵,定德常乘驛計議,諸將稟其成算。至是自以邊功不立,懼得罪而歸心焉。
 其明年,吐蕃昆明城管磨些蠻千餘戶又來降。吐蕃以其眾外潰,遂北寇靈、朔,陷麟州。詔韋皋出兵成都西山以紓北邊。皋遂命鎮靜軍兵馬使陳洎等,統兵萬人出三奇路,威戎軍使崔堯臣率兵一千出龍溪石門路南,維保二州兵馬使仇冕、保霸兩州刺史董振等,率兵二千進逼吐蕃維州城中,北路兵馬使邢玼並諸州刺史董懷愕等率兵四千進攻棲雞、老翁等城,都將高倜、王英俊等率兵二千進逼故松州,隴東路兵馬使元膺並
 諸將郝宗等復分兵八千出南道雅、邛、黎、巂等路。又令邛州鎮南軍使、御史大夫韋良金發鎮兵一千三百續進,雅州經略使路惟明與三部落主趙日進等率兵三千進攻逋租、偏松等城,黎州經略使王有道率三部落郝金信等二千過大渡河深入吐蕃界,巂州經略使陳孝陽與行營兵馬使何大海、韋義等及磨些蠻三部落主苴那時率兵四千進攻昆明、諾濟城。自八月至於十二月,累破十六萬眾,拔其七城、五軍鎮,受降三千餘戶,
 生擒六千餘人,斬首一萬餘級,遂圍維州。救軍再至,轉戰千餘里,吐蕃連敗,靈、朔之寇引眾南下。於是贊普遣莽熱以內大相兼東境五道節度兵馬使、都統群牧大使率雜虜十萬眾,來解維州之圍。王師萬餘眾,據險設伏以待之。先以千人挑戰,莽熱見我師之少也,悉眾來追,入於伏中,請將四面疾擊,遂擒莽熱,虜眾大潰。



 十九年五月,吐蕃使論頰熱至。六月,以右龍武大將軍薛伾兼御史大夫,使於吐蕃。二十年三月上旬,贊普卒,廢朝
 三日,命工部侍郎張薦吊祭之。贊普以貞元十三年四月卒,長子立;一歲卒,次子嗣立。命文武三品以上官吊其使。四月,吐蕃使臧河南觀察使論乞冉及僧南撥特計波等五十四人來朝。十二月,遣使論襲執、郭志崇來朝。



 二十一年二月,順宗命佐金吾衛將軍、兼御史中丞田景度持節告哀於吐蕃,以庫部員外郎、兼御史中丞熊執易為副使。七月,吐蕃使論悉諾等來朝。永貞元年十月,贊普使論乞縷勃藏來貢,助德宗山陵金銀、衣服、
 牛馬等。十一月,以衛尉少卿、兼御史中丞侯幼平充入蕃告冊立等使。



 元和元年正月,福建道送到吐蕃生口十七人,詔給遞乘放還蕃。六月,遣使論勃藏來朝。五年五月,遣使論思耶熱來朝,並歸鄭叔矩、路泌之柩及叔矩男文延等一十三人。叔矩、泌,平涼之盟陷焉,凡二十餘年,竟不屈節,因沒於蕃中,至是請和,故歸之。六月,命宰相杜佑等與吐蕃使議事中書令,且言歸我秦、原、安樂州地。七月,遣鴻臚少卿、攝御史中丞李銘為入蕃
 使,丹王府長史、兼侍御史吳暈副之。六年至十年,遣使朝貢不絕。十二年四月,吐蕃以贊普卒來告,以右衛將軍烏重兼御史中丞,充吊祭使,殿中侍御史段鈞副之。



 十三年十月,吐蕃圍我宥州、鳳翔,上言遣使修好。是月,靈武於定遠城破吐蕃二萬人,殺戮二千人,生擒節度副使一人、判官長行三十九人,獲羊馬甚眾。平涼鎮遏使郝玼破二萬餘眾,收復原州城,獲羊馬不知其數。夏州節度田縉於靈武亦破三千餘人。十一月,鹽州上
 言:吐蕃入河曲,夏州破五萬餘人。靈武破長樂州羅城,焚其屋宇器械。西川節度使王播攻拔峨和、棲雞等城。



 十四年正月,敕曰:



 朕臨御萬邦,推布誠信。西戎納款,積有歲時,中或虧違,亦嘗苞貸。我有殊德,寧不是思,重譯貢珍,道途相繼,申恩示禮,曾無闕焉。昨者蕃使奉章,又至京輦,將君長之命,陳和好之誠。臨軒召見,館氣加厚,復以信幣,諭之簡書。亦既言旋,才及近甸,遽聞蟻聚,來犯封陲,河曲之間,頗為暴擾。背惠棄約,斯謂無名,公議
 物情,咸請誅絕。朕深惟德化之未被,豈慮夷俗不之賓,其國失信,其使何罪!釋其維縶以遂性,示之弘覆以忘懷。予衷茍孚,庶使知感。其蕃使論矩立藏等並後般來使,並宜放歸本國。仍委鳳翔節度使以此意曉喻。



 八月,吐蕃營於慶州方渠,大軍至河州界。十月,吐蕃節度論三摩及宰相尚塔藏、中書令尚綺心兒共領軍約十五萬眾,圍我鹽州數重。黨項首領亦發兵驅羊馬以助。閱歷三旬,賊以飛梯、鵝車、木驢等四面齊攻,城欲陷者數
 四。刺史李文悅率兵士乘城力戰,城穿壞不可守,撤屋版以御之,晝夜防拒,或潛兵斫營,開城出戰,約殺賊萬餘眾。諸道救兵無至者。凡二十七日,賊乃退。



 十五年二月,以秘書少監兼御史中丞田洎入吐蕃告哀,並告冊立。三月,攻掠我青塞堡。七月,遣使來吊祭。十月,侵逼涇州。命右軍中尉梁守謙充左右神策、京西、京北行營都監,統神策兵四千人,並發八鎮全軍往救援。以太府少卿、兼御史中丞邵同持節入吐蕃,充答請和好使。貶前
 入吐蕃使、秘書少監田洎郴州司戶。



 初,洎入蕃為吊祭使,蕃請於長武城下會盟。洎懦怯,恐不得還,唯唯而已。至是西戎入寇,且曰:「田洎許我統兵馬赴盟誓。」遂貶之。戎人實以邊將擾之致忿,徒假洎為辭也。涇州上言:「吐蕃大將並退。」於是罷神策行營兵。自田縉統夏州,以貪狠侵擾,黨項苦之,屢引西戎犯塞。及是大兵入寇,邊將郝玼數襲擊蕃壘,殺戮甚眾,邠州李光顏復以全師而至,戎人懼而退。蓋田縉始生國患,而賴光顏、郝玼之驅
 戮也。十一月,夏州節度使李佑自領兵赴長澤鎮,靈武節度使李聽自領兵赴長樂州,並奉詔討吐蕃也。十二月,吐蕃千餘人圍烏、白池。



 長慶元年六月,犯青塞堡,以我與回紇和親故也。鹽州刺史李文悅發兵進擊之。九月,吐蕃遣使請盟,上許之。宰相欲重其事,請告太廟,太常禮院奏曰:「謹按肅宗、代宗故事,與吐蕃會盟,並不告廟。唯德宗建中末,與吐蕃會盟於延平門,欲重其誠信,特令告廟。至貞元三年,會於平涼,亦無告廟之文。伏以
 事出一時,又非經制,求之典禮,亦無其文。今謹參詳,恐不合告。」從之。乃命大理卿、兼御史大夫劉元鼎充西蕃盟會使,以兵部郎中、兼御史中丞劉師老為副,尚舍奉御、兼監察御史李武、京兆府奉先縣丞兼監察御史李公度為判官。十月十日,與吐蕃使盟,宰臣及右僕射、六曹尚書、中執法、太常、司農卿、京兆尹、金吾大將軍皆預焉。其詞曰:



 維唐承天,撫有八紘,聲教所臻,靡不來廷。兢業齊慄,懼其隕顛,纘武紹文,疊慶重光,克彰浚哲,罔忝
 洪緒,十有二葉,二百有四載。則我太祖,權明號而建不拔,鋪鴻名而垂永久。類上帝以答嘉應,享皇靈以酬景福,曷有怠已?越歲在癸丑,冬十月癸酉,文武孝德皇帝詔丞相臣植、臣播、臣元穎等,與大將和蕃使禮部尚書訥羅論等,會盟於京師,壇於城之西郊,坎於壇北。凡讀誓、刑牲、加書、復壤、陟降、周旋之禮,動無違者,蓋所以偃兵息人,崇姻繼好,懋建遠略,規恢長利故也。



 原夫昊穹上臨,黃祗下載,茫茫蠢蠢之類,必資官司,為厥宰臣,茍
 無統紀,則相滅絕。中夏見管,維唐是君;西裔一方,大蕃為主。自今而後,屏去兵革,宿忿舊惡,廓焉消除,追崇舅甥,曩昔結援。邊堠撤警,戍烽韜煙,患難相恤,暴掠不作,亭障甌脫,絕其交侵。襟帶要害,謹守如故,彼無此詐,此無彼虞。嗚呼!愛人為仁,保境為信,畏天為智,是神為禮,有一不至,拘災於躬。塞山崇崇,河水湯湯,日吉辰良,奠其兩疆,西為大蕃,東實巨唐。大臣執簡,播告秋方。



 大蕃贊普及宰相缽闡布、尚綺心兒等,先寄盟文要節云:「蕃、
 漢兩邦,各守見管本界,彼此不得征,不得討,不得相為寇讎,不得侵謀境上。若有所疑,或要捉生問事,便給衣糧放還。」今並依從,更無添改。



 預盟之官十七人,皆列名焉。其劉元鼎等與論訥羅同赴吐蕃本國就盟,仍敕元鼎到彼,令宰相已下各於盟文後自書名。靈武節度使李進誠於太石山下破吐蕃三千騎。



 二年二月,遣使來請定界。六月,復遣使來朝。鹽州奏:「吐蕃千餘人入靈武界,遣兵逐便邀截。」又言:「擒得與黨項送書信吐蕃一百
 五十人。」是月劉元鼎自吐蕃使回,奏云:「去四月二十四日到吐蕃牙帳,以五月六日會盟訖。」



 初,元鼎往來蕃中,並路經河州,見其都元帥、尚書令尚騎心兒云:「回紇,小國也。我以丙申年逾磧討逐,去其城郭二日程,計到即破滅矣,會我聞本國有喪而還。回紇之弱如此,而唐國待之厚於我,何哉?」元鼎云:「回紇於國家有救難之勛,而又不曾侵奪分寸土地,豈得不厚乎!」是時元鼎往來,渡黃河上流,在洪濟橋西南二千餘里,其水極為淺狹,春
 可揭涉,秋夏則以船渡。其南三百餘里有三山,山形如金敖,河源在其間,水甚清冷,流經諸水,色遂赤,續為諸水所注,漸既黃濁。又其源西去蕃之列館約四驛,每驛約二百餘里。東北去莫賀延磧尾,闊五十里,向南漸狹小,北自沙州之西。乃南入吐渾國,至此轉微,故號磧尾。計其地理,當劍南之直西。元鼎初見贊普於悶懼盧川,蓋贊普夏衙之所,其川在邏娑川南百里,臧河之所流也。時吐蕃遣使論悉諾息等隨元鼎來謝,命太僕少卿杜
 載使以答之。



 三年正月,遣使論答熱來朝賀。四年九月,遣使求《五臺山圖》。十月,貢犛牛及銀鑄成犀牛、羊、鹿各一。寶歷元年三月,遣使尚綺立熱來朝。且請和好。九月,遣光祿卿李銳為使以答之。太和五年至八年。遣使朝貢不絕,我亦時遣使報之。開成元年、二年,皆遣使來。



 會昌二年,贊普卒。十二月,遣論贊等來告哀,詔以將作少監李璟吊祭之。大中三年春,宰相尚恐熱殺東道節度使,以秦、原、安樂等三州並石門、木硤等七關款塞,涇原節
 度使康季榮以聞,命太僕卿陸耽往勞焉。其年七月,河、隴耆老率長幼千餘人赴闕,上御延喜樓觀之,莫不歡呼抃舞,更相解辮,爭冠帶於康衢,然後命善地以處之,觀者咸稱萬歲。



 史臣曰:戎狄之為患也久矣!自秦、漢已還,載籍大備,可得而詳也。但世罕小康,君無常聖,我衰則彼盛,我盛則彼衰,盛則侵我郊圻,衰則服我聲教。懷柔之道,備預之方,儒臣多議於和親,武將唯期於戰勝,此其大較也。彼
 吐蕃者,西陲開國,積有歲年,蠶食鄰蕃,以恢土宇。高宗朝,地方萬里,與我抗衡;近代以來,莫之與盛。至如式遏邊境,命制出師,一彼一此,或勝或負,可謂勞矣!迨至幽陵盜起,乘輿播遷,戍卒咸歸,河、湟失守,此又天假之也。自茲密邇京邑,時縱寇掠,雖每遣行人,來修舊好,玉帛才至於上國,烽燧已及於近郊,背惠食言,不顧禮義,即可知也。夫要以神明,貴其誠信,平涼之會,畜其詐謀,此又不可以忠信而御也。孔子曰:「夷狄之有君,不如諸夏
 之亡也。」誠哉是言!



 贊曰:西戎之地,吐蕃是強。蠶食鄰國,鷹揚漢疆。乍叛乍服,或弛或張。禮義雖攝。其心豺狼。



\end{pinyinscope}