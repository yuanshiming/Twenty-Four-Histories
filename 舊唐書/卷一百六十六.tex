\article{卷一百六十六}

\begin{pinyinscope}

 ○潘孟陽李翛王遂曹華韋綬鄭權盧士玫韓全義高霞寓高瑀崔戎陸亙張正甫子毅夫毅夫子禕



 潘孟陽,禮部侍郎炎之子也。孟陽以父廕進,登博學宏
 辭科。累遷殿中侍御史,降為司議郎。孟陽母,劉晏女也。公卿多父友及外祖賓從,故得薦用,累至兵部郎中。



 德宗末,王紹以恩幸,數稱孟陽之材,因擢授權知戶部侍郎,年未四十。順宗即位,永貞內禪,王叔文誅,杜佑始專判度支,請孟陽代叔文為副。時憲宗新即位,乃命孟陽巡江淮省財賦,仍加鹽鐵轉運副使,且察東南鎮之政理。時孟陽以氣豪權重,領行從三四百人,所歷鎮府,但務游賞,與婦女為夜飲。至鹽鐵轉運院,廣納財賄,補吏
 職而已。及歸,大失人望,罷為大理卿。三年,出為華州刺史,遷梓州刺史、劍南東川節度使。與武元衡有舊,元衡作相,復召為戶部侍郎、判度支,兼京北五城營田使,以和糴使韓重華為副。太府卿王遂與孟陽不協,議以營田非便,持之不下,孟陽忿憾形於言。二人俱請對,上怒不許,乃罷孟陽為左散騎常侍。明年,復拜戶部侍郎。



 孟陽氣尚豪俊,不拘小節。居第頗極華峻。憲宗微行至樂游原,見其宏敞,工猶未已,問之。左右以孟陽對,孟陽懼
 而罷工作。性喜宴,公卿朝士多與之游,時指怒者不一。俄以風緩不能行,改左散騎常侍。元和十年八月卒,贈兵部尚書。憲宗每事求理,常發江淮宣慰使,左司郎中鄭敬奉使。辭,上誡之曰:「朕宮中用度,一匹已上皆有簿籍,唯賑恤貧民,無所計算。卿經明行修,今登車傳命,宜體吾懷,勿學潘孟陽奉使,所至但務酣飲、游山寺而已。」其為人主所薄如此!



 李翛,不知何許人。起於寒賤,以莊憲皇后妹婿,元和已
 來驟階仕進。以恩澤至坊州、絳州刺史。無他才,性纖巧承迎。常飾廚傳以奉往來中使及禁軍中尉賓客,以求善譽。治民蒞事,粗有政能。上以為才,召拜司農卿,遷京兆尹。



 十年,莊憲太后崩,翛為山陵橋道置頓使。恃能惜費,每事減損。靈駕至灞橋頓,從官多不得食。及至渭城北門,門壞。先是,橋道司請改造渭城北門,計錢三萬。翛以勞費不從,令深鑿軌道以通靈駕。掘土既深,旁柱皆懸,因而頓壞,所不及邅輬車者數步而已。初欲壞城之
 東北墉,以出靈駕,中人皆不可,乃停駕,徹去壞門土木而後行。翛懼,誣奏邅輬軸折,山陵使李逢吉令御史封其車軸,自陵還,奏請免翛官。上以用兵務集財賦,以翛前後進奉,不之責,但罰俸而已。逢吉極言其罪,乃削銀青階。翌日,復賜金紫。自此,朝廷端士,多遭譖毀,義士為之側目。時宿師於野,饋運不集。浙西重鎮,號為殷阜,乃以翛為潤州刺史、浙西觀察使,令設法鳩聚財貨。淮西用兵,頗賴其賦。十四年,以病求還京師,未朝謁而卒。



 王遂,宰相方慶之孫也。以吏能聞於時。尤長於興利,銳於操下,法頗嚴酷。累遷至鄧州刺史。以曉達錢穀,入為太府卿。潘孟陽判度支,與遂私憾,互有爭論。遂為西北供軍使,言營田非便,與孟陽會議相非,各求請對。上怒,俱不見,出遂為柳州刺史。遂親吏韋行素、柳季常請課料於兩池務。屬遂罷務,季常等為吏所誣,各笞四十。遂柳州制出,左丞呂元膺執奏曰:「遂以補吏犯贓,法當從坐。其除官制云『清能業官』,據遂犯狀,不宜有『清』字。柳州
 大郡,出守為優。謹封還制書。」上令喻之,方行。數年,用兵淮西。天子藉錢穀吏以集財賦,知遂強幹,乃用為宣州刺史、宣歙觀察使。淮、蔡平,王師東討,召拜光祿卿,充淄青行營諸軍糧料使。以光祿職當祠祭,改檢校左散騎常侍、兼御史大夫。



 初,師之出也,歲計兵食三百萬石。及鄆賊誅,遂進羨餘一百萬,上以為能。時分師道所據十二州為三鎮,乃以遂為沂州刺史、沂兗海等州觀察使。



 遂性狷忿,不存大體。而軍州民吏,久染污俗,率多獷戾,
 而遂數因公事訾詈將卒曰「反虜」,將卒不勝其忿。牙將王弁乘人心怨怒,十四年七月,遂方宴集,弁噪集其徒,害遂於席,判官張實、李甫等同遇害。及曹華代遂至鎮,盡擒亂黨王弁等誅之。



 遂器用不弘,僻於聚斂,而非兼撫之才。但峻威刑,以繩亂俗。其所制笞杖,率逾常制。遂既死,監軍使封其杖進呈。上令出示於朝,以誡廉使。



 曹華,宋州楚丘人,仕宣武軍為牙校。貞元末,吳少誠叛,本軍以華驍果有智算,用為襄城戍將。蔡賊攻襄城,華
 屢敗之,德宗特賜旗甲。元和九年,以功授寧州刺史。未行而吳元濟叛,朝廷命河陽帥烏重胤討賊。重胤請華為懷汝節度行營副使。前後數十戰,大破賊於青陵城。賊平,授棣州刺史,封陳留郡王。棣鄰於鄆,賊屢侵逼,華招募群盜之勁者,補之軍卒,分據要路。其後,賊至皆擊敗之,鄆人不敢北顧。及李師道誅,分所管十二州為三鎮。王遂為沂兗海觀察使,褊刻不能馭眾,為牙將王弁所害,朝廷遂授華左散騎常侍、沂州刺史、沂海兗觀察
 使。



 華至鎮,視事三日,宴將吏,伏甲士千人於幕下。群校既集,華喻之曰:「吾受命廉問,奉聖旨,以鄆州將士分割三處,有道途轉徙之勞。今有頒給,北州兵稍厚。鄆州士卒處右,州兵處左,冀易以區別。」分定,並令州兵出外。既出闔門,乃謂鄆卒曰:「天子深知鄆人之勞,然前害主帥者,不能免罪。」甲士自幕中出,周環之,凡鄆一千二百人,立斬於庭,血流成渠。是日,門屏之間,有赤霧高丈餘,久之方散。自是海、沂之人,重足股慄,無敢為盜者。



 華惡沂
 之地褊,請移理於兗,許之。初,李正己盜有青、鄆十二州,傳襲四世,垂五十年,人俗頑驁,不知禮教。華令將吏曰:「鄒、魯儒者之鄉,不宜忘於禮義。」乃躬禮儒士,習俎豆之容,春秋釋奠於孔子廟,立學講經,儒冠四集。出家財贍給,俾成名入仕,其往者如歸。



 及鎮州軍亂,殺田弘正,華表請以本軍進討,就加檢校工部尚書,升兗海為武寧節度,賜之節鉞。李絺叛於大梁,華不俟命赴討。絺方遣兵三千人取宋州,華逆擊敗之。由是,宋、亳不從絺亂。絺
 平,以功加檢校尚書右僕射。以河朔拒命,移華為滑州刺史、義成軍節度使。長慶三年七月,卒於鎮,時年六十九。



 華雖出自戎行,而動必由禮。尤重士大夫,未嘗以富貴驕人;下迨僕隸走使之徒,必待之以誠信,人以為難。贈司空。



 韋綬,字子章,京兆人。少有至性,喪父,刺血寫佛經。初為長安縣尉,遭硃泚之亂,變服乘驢赴奉天。于頔鎮襄陽,闢為賓佐。嘗因言政,面刺頔之縱恣。入朝為工部員外,
 轉屯田郎中。元和十年,改職方郎中,充太子諸王侍讀,再遷諫議大夫。



 時穆宗在東宮,方幼好戲。綬講書之隙,頗以嘲誚悅之。嘗密齎家所造食,入宮餉太子。憲宗嘗召對,綬奏曰:「太子學書,至『依』字,輒去旁『人』。臣問之,太子云:『君父以此字可天下奏事,臣子不合全書。』」上益嘉太子之賢,賜綬錦彩。綬無威儀,時以人間鄙說戲言以取悅太子。太子因入侍,道綬語。憲宗不悅,謂侍臣曰:「凡侍讀者,當以經義輔導太子,納之輒物,而綬語及此,予何
 望耶?」乃罷侍讀,出為虔州刺史。



 穆宗即位,以師友之恩,召為尚書右丞,兼集賢院學士,甚承恩顧,出入禁中。綬以七月六日是穆宗載誕節,請以是日百官詣光順門賀太后,然後上皇帝壽。時政道頗僻,敕出,人不敢議。久之,宰臣奏古無生日稱賀之儀,其事終寢。綬在集賢,遇重陽,賜宰臣百官曲江宴;綬請與集賢學士別為一會,從之。長慶元年三月,轉禮部尚書,判集賢院事。



 帝嘗問:「禳災祈福,其可必乎?」綬對曰:「昔宋景公以一善言而法
 星退之三舍,此禳災以德也。漢文帝除秘祝,每於祠祭,盡敬而已,言無所祈,以明福不可以求致也。而二君卒能變已變之災,享自致之福,著於史傳,其理甚明。如失德以祈災消,媚神以祈福至,神茍有知,當因以致譴,非祈禳之道也。」時人主失德,綬因以諷之。



 二年十月,檢校戶部尚書、興元尹、山南西道節度使。辭日,請門戟十二,自將赴鎮。又訴家貧,請賜錢二百萬。又面乞授子元弼官。上皆可之。綬御事無術,洎臨戎鎮,庶政隳紊。二年八
 月卒,贈尚書右僕射。博士劉端夫請謚為「通」,殿中侍御史孟琯上言以為非當。博士權安請謚為「繆」,竟不施行。



 鄭權,滎陽開封人也。登進士第,釋褐涇原從事。節度使劉昌符病亟,請入覲,度軍情必變,以權寬厚容眾,俾主留務。及昌符上路,兵果亂。權挺身入白刃中,抗辭喻以逆順。因殺其首亂者數人,三軍畏伏。德宗聞而嘉之。時天子厭兵,籓鎮將吏得軍情者,多超授官爵。自試衛佐擢授行軍司馬、御史中丞。入朝為倉部郎中,累遷至河
 南尹。十一年,代李遜為襄州刺史、山南東道節度使。十二年,轉華州刺史、潼關防禦、鎮國軍使。十三年,遷德州刺史、德棣滄景節度使。



 時朝廷用兵討李師道,權以德、棣之兵臨境。奏於平原、安德二縣之間置歸化縣,以集降民。滄州刺史李宗奭與權不協,每事多違,不稟節制。權奏之,上令中使追之。宗奭諷州兵留己,上言懼亂,未敢離郡,乃以烏重胤鎮橫海,代權歸朝。滄州將吏懼,共逐宗奭。宗奭方奔歸京師。詔以悖慢之罪,斬於獨柳之
 下。其弟宗爽,長流汀州。授權邠寧節度。會天德軍使上章論宗奭之冤,為權誣奏,權降授原王傅。尋遷右金吾衛大將軍,充左街使。



 穆宗即位,改左散騎常侍,充入回鶻告哀使。憚其遠役,辭以足疾,不獲免,肩輿而行。權器度魁偉,有辭辯。既至虜廷,與虜主爭論曲直,言辭激壯,可汗深敬異之。



 長慶元年使還。出為河南尹,入拜工部侍郎,遷本曹尚書。以家人數多,俸入不足,求為鎮守。旬月,檢校右僕射、廣州刺史、嶺南節度使。初,權出鎮,有中
 人之助。南海多珍貨,權頗積聚以遺之,大為朝士所嗤。四年十月卒。



 盧士玫,山東右族,以文儒進。性端厚,與物無競,雅有令聞。始為吏部員外郎,稱職,轉郎中、京兆少尹。奉憲宗園寢,刑簡事集,時論推其有才,權知京兆尹事。會幽州劉總願釋兵柄入朝,請用張弘靖代己。復請析瀛、漠兩州,用士玫為帥,朝廷一皆從之。士玫遂授檢校右常侍,充瀛、漠兩州都防禦觀察使。



 無何,幽州亂,害賓佐,縶弘靖,取
 裨將硃克融領軍務,遣兵襲瀛、漠。朝廷慮防禦之名不足抗兇逆,即日除士玫檢校工部尚書,充瀛漠節度使。士玫亦罄家財助軍用,堅拒叛徒者累月。竟以官軍救之不至,又瀛漠之卒親愛多在幽州,遂為其下陰導克融之兵以潰。士玫及從事皆被拘執,送幽州,囚於賓館。及朝廷宥克融之罪,士玫方得歸東洛。尋拜太子賓客,留司洛中,旋除虢州刺史,復為賓客。寶歷元年七月卒,贈工部尚書。



 韓全義,出自行間,少從禁軍,事竇文場。及文場為中尉,用全義為帳中偏將,典禁兵在長武城。貞元十三年,為神策行營節度、長武城使,代韓潭為夏綏銀宥節度,詔以長武兵赴鎮。全義貪而無勇,短於撫御。制未下,軍中知之,相與謀曰:「夏州沙磧之地,無耕蠶生業。盛夏移徙,吾所不能。」是夜,戍卒鼓噪為亂,全義逾城而免,殺其親將王棲巖、趙虔曜等。賴都虞候高崇文誅其亂首而止之,全義方獲赴鎮。



 明年,吳少誠拒命,詔征十七鎮之師
 討之。時軍無統帥,兵無多少,皆以內官監之,師之進退不由主將。十五年冬,王師為賊所敗於小溵河。德宗以文場素待全義,乃用為蔡州四面行營招討使,仍以陳許節度使上官涚副之。諸鎮之師,皆取全義節度。全義將略非所長,能以巧佞財賄結中貴人,以被薦用。及師臨賊境,又制在監軍,每議兵出,一帳之中,中人十數,紛然爭論莫決。蔡賊聞之,屢求決戰。十六年五月,遇賊於溵水南廣利城。旗鼓未交,諸軍大潰,為賊所乘。全義退
 保五樓,賊對壘相望。潰兵未集,乃與監軍賈英秀、賈國良等保溵水縣。賊距溵水五六里而軍,全義懼其凌突,退保陳州。其汴宋、河北之軍,皆亡歸本鎮,唯陳許將孟元陽、神策將蘇光榮等數千人守溵水。全義誘潞州大將夏侯仲宣、滑將時昂、河陽將權文度、河中將郭湘等誅之。由是軍情稍固。少誠知王師無能為,致書幣以告監軍,願求昭洗。德宗召大臣議,宰相賈耽曰:「昨全義五樓退軍,賊不追擊者,應望國家恩貸。臣伏恐須開生路。」
 上然之。又得監軍等奏,即下制洗滌,加其爵秩。



 十七年,全義自陳州班師,而中人掩其敗跡,上待之如初。全義武臣,不達朝儀,托以足疾,不任謁見。全義司馬崔放入對,德宗勞問,放引過,言招撫無功。德宗曰:「全義為招討使,招得吳少誠歸國,其功大矣。何必殺人乃為功耶!」旋命還鎮,令中使就第賜宴,錫齎頗厚。自還至辭,都不謁見而去。議者以隳敗法制,從古已還,未如貞元之甚。憲宗在籓,常惡其事。及即位,全義懼,求入覲,詔以太子太
 保致仕。其年七月卒。



 高霞寓,範陽人。祖仙,父棲鶴,皆以孝聞。凡五代同爨。德宗朝,採訪使洪經綸奏旌表其門閭,鄉里稱美其事。霞寓少讀《左氏春秋》及孫、吳《兵法》,好大言,頗以節概自許。



 貞元中,徒步造長武城使高崇文,待以猶子之分,擢授軍職,累奏憲宗,甚見委信。元和初,詔授兼御史大夫,從崇文將兵擊劉闢,連戰皆克,下鹿頭城,降李文悅、仇良輔。蜀平,以功拜彭州刺史,尋繼崇文為長武城使,封感
 義郡王。元和五年,以左威衛將軍隨吐突承璀擊王承宗,又加左散騎常侍。明年,改豐州刺史、三城都團練防禦使。六遷至檢校工部尚書。



 元和十年,朝廷討吳元濟,以霞寓宿將,乃析山南東道為兩鎮,以霞寓為唐鄧隋節度使。



 霞寓雖稱勇敢,素昧機略;至於統制,尤非所長。及達所部,乃率兵趣蕭陂,與賊決戰。既小勝,又進至文城柵。賊軍偽敗而退,霞寓逐之不已,因為伏兵所掩,王師大衄,霞寓僅以身免。坐貶歸州刺史。後以恩例,徵為
 右衛大將軍。



 十三年,出為振武節度使,入為左武衛大將軍。長慶元年,授邠寧節度使。三年,就加檢校右僕射。四年,加檢校司空,又加司徒。



 寶歷二年,疽發首,不能理事,求歸闕下。其夏,授右金吾衛大將軍、檢校司徒,途次奉天而卒,年五十五,贈太保。



 霞寓卒伍常材,始因宦官進用,遂階節將。位望既高,言多不遜。朝廷知之,欲議移罷。霞寓頗懷憂恐,舍私第為佛寺,上言請額為「懷恩」,用資聖福。大率奸妄兇狡如此。又非斥朝列,侮慢僚屬,鄙
 辭俚語,日聞於時。



 高瑀,渤海蓚人。少好論兵,釋褐右金吾胄曹,累闢諸府從事,歷陳、蔡二郡刺史,入為太僕卿。太和初,忠武節度使王沛卒,物議以陳許軍四征有功,必自擇帥;或以禁軍之將得之。宰相裴度、韋處厚議瑀深沉方雅,曾刺陳、蔡,人懷良政,又熟忠武軍情,欲請用瑀。事未聞,陳許表至,果請瑀為帥,乃授檢校左散騎常侍、許州刺史、忠武節度使。自大歷已來,節制之除拜,多出禁軍中尉。凡命
 一帥,必廣輸重賂。禁軍將校當為帥者,自無家財,必取資於人;得鎮之後,則膏血疲民以償之。及瑀之拜,以內外公議,搢紳相慶曰:「韋公作相,債帥鮮矣!」



 三年,就加檢校工部尚書。比年水旱,人民薦饑。瑀召集州民,繞郭立堤塘一百八十里,蓄洩既均,人無饑年。加檢校右僕射。六年,移授徐州刺史、武寧軍節度等使。議者以徐泗王智興之後,軍士驕恣,宜得雄帥鎮之。乃以太府卿崔珙代瑀,徵為刑部尚書。以疾求分司,拜太子少傅。其月,復
 授檢校右僕射、陳許蔡節度使。八年六月卒,贈司空。



 瑀性寬和,有體量,為官雖無赫赫之譽,所至皆理,尤得士心,論者美之。



 崔戎,字可大。高伯祖玄暐,神龍初有大功,封博陵郡王。祖嬰,郢州刺史。父貞固,太原榆次尉。戎舉兩經登科,授太子校書,調判入等,授藍田主薄,為籓鎮名公交闢。



 裴度領太原,署為參謀。時王承宗據鎮州叛,度請戎單車往諭之,承宗感泣受教。入為殿中侍御史,累拜吏部郎
 中,遷諫議大夫。尋為劍南東、西兩川宣慰使。西州承蠻寇之後,戎既宣撫,兼再定徵稅,廢置得所,公私便之。還,拜給事中,駁奏為當時所稱。改華州刺史,遷兗海沂密都團練觀察等使。將行,州人戀惜遮道,至有解靴斷登者。理兗一年,大和八年五月卒,贈禮部尚書。



 陸亙,字景山,吳郡人。祖元明,睦州司馬。父持詮,惠陵臺令。亙以書判授集賢殿正字、華原縣尉。應制舉,授萬年縣丞。自京兆府兵曹參軍拜太常博士。寺有禮生孟真,
 久於其事,凡吉兇大儀,禮官不能達,率訪真。真亦賴是須要姑息。元和七年,冊皇太子,將撰儀注,真亦欲參預;亙笞之,由是禮儀不專於胥吏。自虞部員外郎出為鄧州刺史。其後入為戶部郎中、秘書少監、太常少卿,歷刺兗、蔡、虢、蘇四郡。遷越州刺史、浙東團練觀察等使。移宣歙觀察使,加御史大夫。太和八年九月卒,年七十一,贈禮部尚書。



 亙強明嚴毅,所至稱理。初赴兗州,延英面奏曰:「凡節度使握兵分屯屬郡者,刺史不能制,遂為一州
 之弊,宜有處分。」因詔天下兵分屯屬郡者,隸於刺史。越之永喜郡,城於海閆,常陷寇境,集官吏廩祿之半,以代常賦,因循相踵,吏返為幸。亙按舉贓罪,表請郡守以降,增給其俸,人皆賴之。



 張正甫,字踐方,南陽人。曾祖大禮,坊州刺史。祖紹貞,尚書右丞。父泚,蘇州司馬。正甫登進士第,從樊澤為襄陽從事,累轉監察御史。于頔代澤,闢留正甫。正甫堅辭之,遂誣奏貶郴州長史。後由邕府徵拜殿中侍御史,遷戶
 部員外郎,轉司封員外、兼侍御史知雜事。遷戶部郎中,改河南尹。由尚書右丞為同州刺史,入拜左散騎常侍、集賢殿學士判院事。轉工部尚書。五年,檢校兵部尚書、太子詹事。明年,以吏部尚書致仕。正甫仁而端亮,蒞官清強。居外任,所至稱理。太和八年九月卒,年八十三,累贈太師。子毅夫。



 毅夫,登進士第。初正甫兄式,大歷中進士登第,繼之以正甫,式子元夫、傑夫、征夫又相次登科。太和中,文章之盛,世共稱之。元夫,太和初兵部郎中、知
 制誥,遷中書舍人,出為汝州刺史。毅夫位至戶部侍郎、弘文館學士判院事。諸群從登第者數人,而毅夫子禕最知名。



 禕,字冠章,釋褐汴州從事、戶部判官,入為藍田尉、集賢校理。趙隱鎮浙西,劉鄴鎮淮南,皆闢為賓佐。入為監察御史,遷左補闕。乾符中,詔入翰林為學士,累官至中書舍人。黃巢犯京師,從僖宗幸蜀,拜工部侍郎,判戶部事。奉使江淮還,為當塗者不協,改太子賓客、左散騎常侍,轉吏部侍郎,歷刑部、兵部尚書。從昭宗在華,為
 韓建所構,貶衡州司馬。昭宗還京,徵拜禮部尚書、太常卿,充禮儀使,遷兵部尚書。



 禕苦心為文,老而益壯。為刑部時,劉鄴子覃,當巢寇時避禍於金吾將軍張直方之第,被害。僖宗還京,而惡覃者以托附逆黨,死不以義,下三司詳罪。禕上章申理,言覃父子並命於賊廷,豈附逆耶?其家竟獲洗雪,覃亦贈官。其行義始終,皆如此類。



 史臣曰:孟陽、王遂儒雅之曹,才有可稱,竟以財媚時君,陷為俗吏。蹈道之論,可不懼耶!全義官由妄進,霞寓位
 以卒升,勇毅不足以啟行,謀慮不足以應變,敗亡之辱,不亦宜乎?朝無責帥之刑,蓋自恥也。權、瑀長者,末塗喪真,雖牽於食貧,純則偽矣。



 贊曰:蘊仁則哲,蘊利則狂。搢紳之胤,勿效潘、王。全義逃責,貞元失策。霞寓薄刑,元和復興。



\end{pinyinscope}