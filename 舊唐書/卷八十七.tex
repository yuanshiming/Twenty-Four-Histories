\article{卷八十七}

\begin{pinyinscope}

 ○郭孝恪張儉蘇定方薛仁貴程務挺張士貴趙道興



 郭孝恪,許州陽翟人也,少有志節。隋末,率鄉曲數百人附於李密,密大悅之,謂曰:「昔稱汝潁多奇士,故非謬也。」
 令與徐勣守黎陽。後密敗,勣令孝恪入朝送款,封陽翟郡公,拜宋州刺史。令與徐勣經營武牢已東,所得州縣,委以選補。其後,竇建德率眾來援王世充,孝恪於青城宮進策於太宗曰:「世充日踧月迫,力盡計窮,懸首面縛,翹足可待。建德遠來助虐,糧運阻絕,此是天喪之時。請固武牢,屯軍汜水,隨機應變,則易為克殄。」太宗然其計。及破建德,平世充,太宗於洛陽置酒高會諸將曰:「郭孝恪謀擒建德之策,王長先龍門下米之功,皆出諸人之
 右也。」歷遷貝、趙、江、涇四州刺史,所在有能名。入為太府少卿,轉左驍衛將軍。貞觀十六年,累授金紫光祿大夫,行安西都護、西州刺史。其地高昌舊都,士流與流配及鎮兵雜處,又限以沙磧,與中國隔絕。孝恪推誠撫御,大獲其歡心。初,王師之滅高昌也,制以高昌所虜焉耆生口七百盡還之。焉耆王尋叛歸欲谷可汗,朝貢稀至。令孝恪伺其機便,因表請擊之。以孝恪為安西道行軍總管,率步騎三千出銀山道以伐焉耆。孝恪夜襲其城,虜其
 王龍突騎支。太宗大悅,璽書勞之曰:「卿破焉耆,虜其偽王,功立威行,深副所委。但焉耆絕域,地阻天山,恃遠憑深,敢懷叛逆。卿望崇位重,報效情深,遠涉沙場,龔行罰罪。取其堅壁,曾不崇朝;再廓游魂,遂無遺寇。糸面思竭力,必大艱辛。超險成功,深足嘉尚。」俄又以孝恪為昆丘道副大總管以討龜茲,破其都城。孝恪自留守之,餘軍分道別進,龜茲國相那利率眾遁逃。孝恪以城外未賓,乃出營於外,有龜茲人來謂孝恪曰:「那利為相,人心素歸,
 今亡在野,必思為變。城中之人,頗有異志,公宜備之。」孝恪不以為虞。那利等果率眾萬餘,陰與城內降胡表裏為應。孝恪失於警候,賊將入城鼓噪,孝恪始覺之,乃率部下千餘人入城,與賊合戰。城中人復應那利,攻孝恪。孝恪力戰而入,至其王所居,旋復出,戰於城門,中流矢而死,孝恪子待詔亦同死於陣。賊竟退走,將軍曹繼叔復拔其城。太宗聞之,初責孝恪不加警備,以致顛覆;後又憐之,為其家舉哀。高宗即位,追贈安西都護、陽翟郡
 公,待詔贈游擊將軍,仍賻物三百段。孝恪性奢侈,僕妾器玩,務極鮮華,雖在軍中,床帳完具。嘗以遺行軍大總管阿史那社爾,社爾一無所受。太宗聞之曰:「三將優劣之不同也。郭孝恪今為寇虜所屠,可謂自貽伊咎耳。」次子待封,高宗時,官至左豹韜衛將軍。咸亨中,與薛仁貴率兵討吐蕃,於大非川戰敗,減死除名。少子待聘,長安中官至宋州刺史。



 張儉,雍州新豐人,隋相州刺史、皖城公威之孫也。父植,
 車騎將軍、連城縣公。儉即高祖之從甥也。貞觀初,以軍功累遷朔州刺史。時頡利可汗自恃強盛,每有所求,輒遣書稱敕。緣邊諸州,遞相承稟。及儉至,遂拒不受,太宗聞而嘉之。儉又廣營屯田,歲致穀十萬斛,邊糧益饒。及遭霜旱,勸百姓相贍,遂免饑餒,州境獨安。後檢校勝州都督,以母憂去職。儉前在朔州,屬李靖平突厥之後,有思結部落,貧窮離散,儉招慰安集之。其不來者,或居磧北,既親屬分住,私相往還,儉並不拘責,但存綱紀,羈縻
 而已。及儉移任,州司謂其將叛,遽以奏聞。朝廷議發兵進討,仍起儉為使,就觀動靜。儉單馬推誠,入其部落,召諸首領,布以腹心,咸匍匐啟顙而至,便移就代州。即令檢校代州都督。儉遂勸其營田,每年豐熟。慮其私蓄富實,易生驕侈,表請和糴,擬充貯備,蕃人喜悅。邊軍大收其利。遷營州都督,兼護東夷校尉。太宗將征遼東,遣儉率蕃兵先行抄掠。儉軍至遼西,為遼水汛漲,久而未渡,太宗以為畏懦,召還。儉詣洛陽謁見,面陳利害,因說水
 草好惡,山川險易,太宗甚悅,仍拜行軍總管,兼領諸蕃騎卒,為六軍前鋒。時有獲高麗候者,稱莫離支將至遼東,詔儉率兵自新城路邀擊之,莫離支竟不敢出。儉因進兵渡遼,趨建安城,賊徒大潰,斬首數千級。以功累封皖城郡公,賞賜甚厚。其後,改東夷校尉為東夷都護,仍以儉為之。永徽初,加金紫光祿大夫。四年,卒於官,年六十,謚曰密。儉兄大師,累以軍功仕至太僕卿、華州刺史、武功縣男。儉弟延師,永徽初,累授左衛大將軍,封範陽
 郡公。延師廉謹周慎,典羽林屯兵前後三十餘年,未嘗有過,朝廷以此稱之。龍朔三年,卒官,贈荊州都督,謚曰敬,陪葬昭陵。唐制三品已上,門列棨戟,儉兄弟三院,門皆立戟,時人榮之,號為「三戟張家」。



 蘇定方,冀州武邑人也。父邕,大業末,率鄉閭數千人為本郡討賊。定方驍悍多力,膽氣絕倫,年十餘歲,隨父討捕,先登陷陣。父卒,郡守又令定方領兵,破賊首張金稱於郡南,手斬金稱,又破楊公卿於郡西,追奔二十餘里,
 殺獲甚眾,鄉黨賴之。後仕竇建德,建德將高雅賢甚愛之,養以為子。雅賢俄又為劉黑闥攻陷城邑,定方每有戰功。及黑闥、雅賢死,定方歸鄉里。貞觀初,為匡道府折沖,隨李靖襲突厥頡利於磧口。靖使定方率二百騎為前鋒,乘霧而行,去賊一里許,忽然霧歇,望見其牙帳,馳掩殺數十百人。頡利及隋公主狼狽散走,餘眾俯伏,靖軍既至,遂悉降之。軍還,授左武候中郎將。永徽中,轉左衛勛一府中郎將,從左衛大將軍程知節徵賀魯,為前
 軍總管。至鷹娑川,突厥有二萬騎來拒,總管蘇海政與戰,互有前卻。既而突厥別部鼠尼施等又領二萬餘騎續至。定方正歇馬,隔一小嶺,去知節十許里,望見塵起,率五百騎馳往擊之,賊眾大潰,追奔二十里,殺千五百餘人,獲馬二千匹,死馬及所棄甲仗,綿亙山野,不可勝計。副大總管王文度害其功,謂知節曰:「雖雲破賊,官軍亦有死傷,蓋決成敗法耳,何為此事?自今正可結為方陣,輜重並納腹中,四面布隊,人馬被甲,賊來即戰,自保
 萬全。無為輕脫,致有傷損。」又矯稱別奉聖旨,以知節恃勇輕敵,使文度為其節制,遂收軍不許深入。終日跨馬被甲結陣,由是馬多瘦死,士卒疲勞,無有戰志。定方謂知節曰:「本來討賊,今乃自守,馬餓兵疲,逢賊即敗。怯懦如此,何功可立!又公為大將,閫外之事,不許自專,別遣軍副,專其號令,理必不然。須囚縶文度,飛表奏之。」知節不從。至恆篤城,有胡降附,文度又曰:「比我兵回,彼還作賊,不如盡殺,取其資財。」定方曰:「如此,自作賊耳,何成伐
 叛?」文度不從。及分財,唯定方一無所取。師還,文度坐處死,後得除名。明年,擢定方為行軍大總管,又徵賀魯,以任雅相、回紇婆潤為副。自金山之北,指處木昆部落,大破之。其俟斤懶獨祿以眾萬餘帳來降,定方撫之,發其千騎進至突騎施部。賀魯率胡祿屋闕啜、懾舍提暾啜、鼠尼施處半啜、處木昆屈律啜、五努失畢兵馬,眾且十萬,來拒官軍,定方率回紇及漢兵萬餘人擊之。賊輕定方兵少,四面圍之,定方令步卒據原,攢槊外向,親領漢
 騎陣於北原。賊先擊步軍,三沖不入,定方乘勢擊之,賊遂大潰,追奔三十里,殺人馬數萬。明日,整兵復進。於是胡祿屋等五努失畢悉眾來降,賀魯獨與處木昆屈律啜數百騎西走。餘五咄六聞賀魯敗,各向南道降於步真,於是西蕃悉定。唯賀魯及咥運率其牙內餘眾而奔,定方追之,復大戰於伊麗水上,殺獲略盡。賀魯及咥運十餘騎逼夜亡走,定方遣副將蕭嗣業追捕之,至於石國,擒之而還。高宗臨軒,定方戎服操賀魯以獻,列其地
 為州縣,極於西海。定方以功遷左驍衛大將軍,封刑國公,又封子慶節為武邑縣公。俄有思結闕俟斤都曼先鎮諸胡,擁其所部及疏勒、硃俱般、蔥嶺三國復叛,詔定方為安撫大使,率兵討之。至葉葉水,而賊保馬頭川,於是選精卒一萬人、馬三千匹馳掩襲之,一日一夜行三百里,詰朝至城西十里。都曼大驚,率兵拒戰於城門之外,賊師敗績,退保馬保城,王師進屯其門。入夜,諸軍漸至,四面圍之,伐木為攻具,布列城下。都曼自知不免,面
 縛開門出降。俘還至東都,高宗御乾陽殿,定方操都曼特勒獻之,蔥嶺以西悉定。以功加食邢州鉅鹿真邑五百戶。顯慶五年,從幸太原,制授熊津道大總管,率師討百濟。定方自城山濟海,至熊津江口,賊屯兵據江。定方升東岸,乘山而陣,與之大戰,揚帆蓋海,相續而至。賊師敗績,死者數千人,自餘奔散。遇潮且上,連舳入江,定方於岸上擁陣,水陸齊進,飛楫鼓噪,直趣真都。去城二十許里,賊傾國來拒,大戰破之,殺虜萬餘人,追奔入郭。其
 王義慈及太子隆奔於北境,定方進圍其城。義慈次子泰自立為王,嫡孫文思曰:「王與太子雖並出城,而身見在,叔總兵馬,即擅為王,假令漢兵退,我父子當不全矣。」遂率其左右投城而下,百姓從之,泰不能止。定方命卒登城建幟,於是泰開門頓顙。其大將禰植又將義慈來降,太子隆並與諸城主皆同送款。百濟悉平,分其地為六州。俘義慈及隆、泰等獻於東都。定方前後滅三國,皆生擒其主。賞賜珍寶,不可勝計,仍拜其子慶節為尚輦
 奉御,定方俄遷左武衛大將軍。乾封二年卒,年七十六。高宗聞而傷惜,謂侍臣曰:「蘇定方於國有功,例合褒贈,卿等不言,遂使哀榮未及。興言及此,不覺嗟悼。」遽下詔贈幽州都督,謚曰莊。



 薛仁貴,絳州龍門人。貞觀末,太宗親征遼東,仁貴謁將軍張士貴應募,請從行。至安地,有郎將劉君昂為賊所圍甚急,仁貴往救之,躍馬徑前,手斬賊將,懸其頭於馬鞍,賊皆懾伏,仁貴遂知名。及大軍攻安地城,高麗莫離
 支遣將高延壽、高惠真率兵二十五萬來拒戰,依山結營,太宗分命諸將四面擊之。仁貴自恃驍勇,欲立奇功,乃異其服色,著白衣,握戟,腰鞬張弓,大呼先入,所向無前,賊盡披靡卻走。大軍乘之,賊乃大潰。太宗遙望見之,遣馳問先鋒白衣者為誰,特引見,賜馬兩匹、絹四十匹,擢授游擊將軍、雲泉府果毅,仍令北門長上,並賜生口十人。及軍還,太宗謂曰:「朕舊將並老,不堪受閫外之寄,每欲抽擢驍雄,莫如卿者。朕不喜得遼東,喜得卿也。」尋
 遷右領軍郎將,依舊北門長上。永徽五年,高宗幸萬年宮,甲夜,山水猥至,沖突玄武門,宿衛者散走。仁貴曰:「安有天子有急,輒敢懼死?」遂登門桄叫呼,以驚宮內。高宗遽出乘高,俄而水入寢殿,上使謂仁貴曰:「賴得卿呼,方免淪溺,始知有忠臣也。」於是賜御馬一匹。蘇定方之討賀魯也,於是仁貴上疏曰:「臣聞兵出無名,事故不成,明其為賊,敵乃可伏。今泥熟仗素幹,不伏賀魯,為賊所破,虜其妻子。漢兵有於賀魯諸部落得泥熟等家口,將充
 賤者,宜括取送還,仍加賜賚。即是矜其枉破,使百姓知賀魯是賊,知陛下德澤廣及也。」高宗然其言,使括泥熟家口送還之,於是泥熟等請隨軍效其死節。顯慶二年,詔仁貴副程名振於遼東經略,破高麗於貴端城,斬首三千級。明年,又與梁建方、契苾何力於遼東共高麗大將溫沙門戰於橫山,仁貴匹馬先入,莫不應弦而倒。高麗有善射者,於石城下射殺十餘人,仁貴單騎直往沖之,其賊弓矢俱失,手不能舉,便生擒之。俄又與辛文陵
 破契丹於黑山,擒契丹王阿卜固及諸首領赴東都。以功封河東縣男。尋又領兵擊九姓突厥於天山,將行,高宗內出甲,令仁貴試之。上曰:「古之善射,有穿七札者,卿且射五重。」仁貴射而洞之,高宗大驚,更取堅甲以賜之。時九姓有眾十餘萬,令驍健數十人逆來挑戰,仁貴發三矢,射殺三人,自餘一時下馬請降。仁貴恐為後患,並坑殺之。更就磧北安撫餘眾,擒其偽葉護兄弟三人而還。軍中歌曰:「將軍三箭定天山,戰士長歌入漢關。」九姓
 自此衰弱,不復更為邊患。乾封初,高麗大將泉男生率眾內附,高宗遣將軍龐同善、高等迎接之。男生弟男建率國人逆擊同善等,詔仁貴統兵為後援。同善等至新城,夜為賊所襲。仁貴領驍勇赴救,斬首數百級。同善等又進至金山,為賊所敗,高麗乘勝而進。仁貴橫擊之,賊眾大敗,斬首五萬餘級。遂拔其南蘇、木底、蒼巖等三城,始與男生相會。高宗手敕勞之曰:「金山大陣,兇黨實繁。卿身先士卒,奮不顧命,左沖右擊,所向無前,諸軍賈勇,
 致斯克捷。宜善建功業,全此令名也。」仁貴乘勝領二千人進攻扶餘城,諸將咸言兵少,仁貴曰:「在主將善用耳,不在多也。」遂先鋒而行,賊眾來拒,逆擊大破之,殺獲萬餘人,遂拔扶餘城。扶餘川四十餘城,乘風震懾,一時送款。仁貴便並海略地,與李勣大會軍於平壤城。高麗既降,詔仁貴率兵二萬人與劉仁軌於平壤留守,仍授右威衛大將軍,封平陽郡公,兼檢校安東都護。移理新城,撫恤孤老;有幹能者,隨才任使;忠孝節義,咸加旌表。高
 麗士眾莫不欣然慕化。



 咸亨元年,吐蕃入寇,又以仁貴為邏娑道行軍大總管。率將軍阿史那道真、郭待封等以擊之。待封嘗為鄯城鎮守,恥在仁貴之下,多違節度。軍至大非川,將發赴烏海,仁貴謂待封曰:「烏海險遠,車行艱澀,若引輜重,將失事機,破賊即回,又煩轉運。彼多瘴氣,無宜久留。大非嶺上足堪置柵,可留二萬人作兩柵,輜重等並留柵內,吾等輕銳倍道,掩其未整,即撲滅之矣。」仁貴遂率先行,至河口遇賊,擊破之,斬獲略盡,收
 其牛羊萬餘頭,回至烏海城,以待後援。待封遂不從仁貴之命,領輜重繼進。比至烏海,吐蕃二十餘萬悉眾來救,邀擊,待封敗走趨山,軍糧及輜重並為賊所掠。仁貴遂退軍屯於大非川。吐蕃又益眾四十餘萬來拒戰,官軍大敗,仁貴遂與吐蕃大將論欽陵約和。仁貴嘆曰:「今年歲在康午,軍行逆歲,鄧艾所以死於蜀,吾知所以敗也。」仁貴坐除名。尋而高麗眾相率復叛,詔起仁貴為雞林道總管以經略之。上元中,坐事徙象州,會赦歸。高宗
 思其功,開耀元年,復召見,謂曰:「往九成宮遭水,無卿已為魚矣。卿又北伐九姓,東擊高麗,漢北、遼東咸遵聲教者,並卿之力也。卿雖有過,豈可相忘?有人雲卿烏海城下自不擊賊,致使失利,朕所恨者,唯此事耳。今西邊不靜,瓜、沙路絕,卿豈可高枕鄉邑,不為朕指揮耶?」於是起授瓜州長史,尋拜右領軍衛將軍,檢校代州都督,又率兵擊突厥元珍等於雲州,斬首萬餘級,獲生口二萬餘人、駝馬牛羊三萬餘頭。賊聞仁貴復起為將,素憚其名,
 皆奔散,不敢當之。其年,仁貴病卒,年七十,贈左驍衛將軍,官造靈輿,並家口給傳還鄉。子訥,別有傳。



 程務挺,洺州平恩人也。父名振,大業末,仕竇建德為普樂令,甚有能名,諸賊不敢犯其境。尋棄建德歸國,高祖遙授永年令,仍令率兵經略河北。名振夜襲鄴縣,俘其男女千餘人以歸。去鄴八十里,閱婦人有乳汁者九十餘人,悉放遣之。鄴人感其仁恕,為之設齋,以報其恩。及建德敗,始之任。俄而劉黑闥陷洺州,名振復與刺史陳
 君賓自拔歸朝。母潘、妻李,在路為賊所掠,沒於黑闥。名振又從太宗討黑闥,時黑闥於冀、貝、滄、瀛等州水陸運糧,以拒官軍,名振率千餘人邀擊之,盡毀其舟車。黑闥聞之大怒,遂殺名振母、妻。及黑闥平,名振請手斬黑闥,以其首祭母。名振以功拜營州都督府長史,封東郡公,賜物二千段、黃金三百兩。累轉洺州刺史。太宗將征遼東,召名振問以經略之事,名振初對失旨;太宗動色詰之,名振酬對逾辯,太宗意解,謂左右曰:「房玄齡常在我
 前,每見別嗔餘人,猶顏色無主。名振生平不見我,向來責讓,而詞理縱橫,亦奇士也。」即日拜右驍衛將軍,授平壤道行軍總管。前後攻沙卑城,破獨山陣,皆以少擊眾,稱為名將。永徽六年,累除營州都督,兼東夷都護。又率兵破高麗於貴端水,焚其新城,殺獲甚眾。後歷晉、蒲二州刺史。龍朔二年卒,贈右衛大將軍,謚曰烈。



 務挺少隨父征討,以勇力聞,遷右領軍衛中郎將。永隆中,突厥史伏念反叛,定襄道行軍總管李文暕、曹懷舜、竇義昭等
 相次戰敗。又詔禮部尚書裴行儉率兵討之,務挺為副將,仍檢校豐州都督。時伏念屯於金牙山,務挺與副總管唐玄表引兵先逼之。伏念懼不能支,遂間道降於行儉,許伏念以不死。中書令裴炎以伏念懼務挺等兵勢而降,非行儉之功,伏念遂伏誅。務挺以功遷右衛將軍,封平原郡公。永淳二年,綏州城平縣人白鐵余率部落稽之黨據縣城反,偽稱尊號,署百官,又進寇綏息,殺掠人吏,焚燒村落,詔務挺與夏州都督王方翼討之。務挺
 進攻其城,拔之,生擒白鐵餘,盡平其餘黨。又以功拜左驍衛大將軍、檢校左羽林軍。嗣聖初,與右領軍大將軍、檢校右羽林軍張虔勖同受則天密旨,帥兵入殿庭,廢中宗為廬陵王,立豫王為皇帝。則天臨朝,累受賞賜,特拜其子齊之為尚乘奉御。務挺泣請回授其弟,則天嘉之,下制褒美,乃拜其弟原州司馬務忠為太子洗馬。又明年,以務挺為左武衛大將軍、單于道安撫大使,督軍以御突厥。務挺善於綏御,威信大行,偏裨已下,無不盡
 力;突厥甚憚之,相率遁走,不敢近邊。及裴炎下獄,務挺密表申理之,由是忤旨。務挺素與唐之奇、杜求仁友善,或構言務挺與裴炎、徐敬業皆潛相應接。則天遣左鷹揚將軍裴紹業就軍斬之,籍沒其家。突厥聞務挺死,所在宴樂相慶,仍為務挺立祠,每出師攻戰,即祈禱焉。



 貞觀、永徽間,軍將又有張士貴、趙道興,狀跡可錄。



 張士貴者,虢州盧氏人也。本名忽峍,善騎射,膂力過人。大業末,聚眾為盜,攻剽城邑,遠近患之,號為「忽峍賊」。高
 祖降書招懷之,士貴以所統送款,拜右光祿大夫。累有戰功,賜爵新野縣公。從平東都,授虢州刺史。高祖謂之曰:「欲卿衣錦晝游耳。」尋入為右武候將軍。貞觀七年,破反獠而還,太宗勞之曰:「聞公親當矢石,為士卒先,雖古名將,何以加也!朕嘗聞以身報國者,不顧性命,但聞其語,未聞其實,於公見之矣。」後累遷左領軍大將軍,改封虢國公。顯慶初卒,贈荊州都督,陪葬昭陵。



 趙道興者,甘州酒泉人。隋右武候大將軍才之子也。道
 興,貞觀初歷遷左武候中郎將,明閑宿衛,號為稱職。太宗嘗謂之曰:「卿父為隋武候將軍,甚有當官之譽。卿今克傳弓冶,可謂不墜家聲。」因授右武候將軍,賜爵天水縣子。其父時廨宇,仍舊不改,時人以為榮。道興嘗自指其事曰:「此是趙才將軍,還使趙才將軍兒坐。」為朝野所笑,傳為口實。儀鳳中,累遷左金吾衛大將軍。文明年,以老病致仕於家。子晈,亦為金吾將軍,凡三代執金吾,為時所稱。



 史臣曰:孝恪機鈐果毅,協草昧之際;樹勛建策,有傑世之風。然而務奢為恆,既未盡善,舉眾失律,不其惑與!張公經略,有天然才度,務穡勸分,董和成績,惜哉中壽,其才未盡。刑國公神略翕張,雄謀戡定,輔平屯難,始終成業。疏封陟位,未暢茂典,蓋闕如也。仁貴驍悍壯勇,為一時之傑,至忠大略,勃然有立。噫,待封不協,以敗全略。孔子曰:「可與立,未可與權。」上加明命,竟致立功,知臣者君,信哉!務挺勇力驍果,固有父風,英概輔時,克繼洪烈。然
 而茍預廢立,竟陷讒構。古之言曰:「惡之來也,如火之燎於原,不可向邇。」其是之謂乎!士貴、道興,逢時立效,得盡義勇,以觀厥成;而繼父風概,三代執金,不亦美乎!



 贊曰:五將雄雄,俱立邊功。張、蘇二族,功名始終。郭、薛、務挺,徼功奮命。垂則窮邊,兵無常
 勝。



\end{pinyinscope}