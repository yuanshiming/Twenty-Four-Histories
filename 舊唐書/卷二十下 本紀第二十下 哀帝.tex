\article{卷二十下 本紀第二十下 哀帝}

\begin{pinyinscope}

 哀皇帝諱柷,
 昭宗第九子,母曰積善太后何氏。景福元年九月三日,生於大內。乾寧四年二月,封輝王,名祚。天復三年二
 月,拜開府儀同三司,充
 諸道兵馬元帥。天祐
 元年八月十二日,昭宗遇弒。翌日,蔣玄暉矯宣遺詔,曰:「我國家化隋為唐,奄有天下,三百年之睹兵戈之屢起,賴勛賢協力,宗社再安。豈意宮闈之間,禍亂忽作,昭儀李漸榮、河東夫人裴貞一潛懷逆節,輒肆狂謀,傷疻既深,已及危革。萬機不可以久曠,四海不可以乏君,神鼎所歸,須有纘繼。輝王祚幼彰岐嶷,長實端良,裒然不群,予所鐘愛,必能克奉丕
 訓,以安兆人。宜立為皇太子,仍改名柷,監軍國事。於戲!孝愛可以承九廟,恭儉可以安萬邦,無樂逸游,志康寰宇。百闢卿士,佑茲沖人,載揚我高祖、太宗之休烈。」是日遷神柩於西宮,文武百僚班慰于延和門外。其日午時,又矯宣皇太后令曰:「予遭家不造,急變爰臻,禍生女職之徒,事起宮奚之輩。皇帝自罹鋒刃,已至彌留,不及顧遺,號慟徒切。定大計者安社稷,纂丕圖者擇賢明,議屬未亡人,須示建長策。承高祖之寶運,醫元勛之忠規,伏
 示股肱,以匡沖昧。皇太子柷宜於柩前即皇帝位,其哀制並依祖宗故事,中書門下準前處分。於戲!送往事居,古人令範,行今報舊,前哲格言。抆淚敷宣,言不能喻。」帝時年十三,乞且監國,柩前即位,宜差太常卿王溥充禮儀使,又令太子家令李能告哀於十六宅。丙午,大行皇帝大殮,皇太子柩前即皇帝位。己酉,矯制曰:「昭儀李漸榮、河東夫人裴貞一,今月十一日夜持刃謀逆,懼罪投井而死,宜追削為悖逆庶人。」蔣玄暉夜既弒逆,詰旦宣
 言於外曰:「夜來帝與昭儀博戲,帝醉,為昭儀所害。」歸罪宮人,以掩弒逆之跡。然龍武軍官健備傳二夫人之言於市人。尋用史太為棣州刺史,以酬弒逆之功。



 庚戌,群臣上表請聽政。甲寅,中書奏:「皇帝九月三日降誕,請以其日為乾和節。」從之。乙丑,百僚赴西宮,殮訖,釋服。皇帝見群臣於崇勛殿西廊下。中書帖:今月二十四日釋服後,三日一度進名起居。丙辰,敕:「朕奉太后慈旨,以兩司綱運未來,百官事力多闕,旦夕霜冷,深軫所懷。令於內
 庫方圓銀二千一百七十二兩,充見任文武常參官救接,委御史臺依品秩分俵。」是日,皇帝聽政。丁巳,敕:乾和節方在哀疚,其內道場宜停。戊午,遣刑部尚書張禕告哀於河中,全忠號哭盡哀。庚申,敕:「乾和節文武百僚諸軍諸使諸道進奏官準故事於寺觀設齋,不得宰殺,只許酒果脯醢。」辛酉,敕:「三月二十三日嘉會節。伏以大行皇帝仙駕上升,靈山將卜,神既游於天際,節宜輟於人間。準故事,嘉會節宜停。」



 九月壬戌朔,百官素服赴西內
 臨,進名奉慰。戊辰,大行皇帝大祥,百官素服赴西內臨。己巳,敕右僕射、門下侍郎、禮部尚書、平章事裴樞宜充大行皇帝山陵禮儀使,門下侍郎、平章事獨孤損宜充大行皇帝山陵使,兵部侍郎李燕充鹵簿使,權知河南尹韋震充橋道使,宗正卿李克勤充按行使。庚午,皇帝釋服從吉。中書門下奏:「伏以陛下光繼寶圖,纂承丕緒,教道克申於先訓,保任實自於慈顏。今則正位宸居,未崇徽號。伏以大行皇帝皇后母臨四海,德冠六宮,推尊
 宜正於鴻名,敬上式光於睿孝,望上尊號曰皇太后。」奉敕宜依。又敕輝王府官屬宜停。辛巳,山陵橋道使改差權河南尹張廷範,其頓遞陵下應接等使,並令廷範兼之。庚寅,中書奏:太常寺止鼓兩字「敔」上字犯御名,請改曰「肇」。從之。



 十月辛卯朔,日有蝕之,在心初度。壬辰,全忠自河中來朝,赴西內臨祭訖,對於崇勛殿。甲午,敕檢校太保、左龍武統軍硃友恭可復本姓名李彥威,貶崖州司戶同正。檢校司徒、右龍武統軍氏叔琮可貶貝州司
 戶同正。又敕:「彥威等主典禁兵,妄為扇動,既有彰於物論,兼亦系於軍情。謫掾遐方,安能塞責?宜配充本州長流百姓,仍令所在賜自盡。」河南尹張廷範收彥威等殺之。臨刑,大呼曰:「賣我性命,欲塞天下之謗,其如神理何!操心若此,欲望子孫長世,可乎?」呼廷範,謂曰:「公行當及此,勉自圖之。」是日,全忠歸大梁。丙申,制天平軍節度使、檢校太師、中書令,兼鄆州刺史、上柱國、東平王、食邑七千戶張全義本官兼河南尹、許州刺史、忠武軍節度觀
 察等使、判六軍諸衛事。皇帝即位行事官、左丞楊涉進封開國伯,加食邑四百戶。吏部侍郎趙光逢進開國公,加食邑三百戶。右散騎常侍竇回、給事中孫續、戶部郎中知制誥封舜卿等加勛階。禮儀使、太常卿王溥與一子八品正員官。書寶冊官吏部尚書陸扆、刑部尚書張禕,扆與一子八品正員官,禕加階。太子太保盧紹卒。魏博羅紹威進救接百官絹千匹、綿三千兩。



 十一月辛酉朔。癸酉午時,日有黃白暈,旁有青赤紉。楊行密攻光州,
 又急攻鄂州,杜洪遣使求援,全忠率師五萬自潁州渡淮,至霍丘大掠以紓之,行密分兵來拒。乙酉,敕:「據太常禮院奏,於十二月內擇日冊太后者。朕近奉慈旨,以山陵未畢,哀感方纏。凡百有司,且虔充奉,吉兇之禮,難以並施。太后冊禮,宜俟山陵畢日,庶得橋山攀慕,彰盡節於群臣;蘭殿承榮,展盛儀於朕志。情既獲遂,禮實宜之。付所司。」己丑,嶺南東道辨州宜改為勛州。十二月辛卯朔。癸卯,權知河南府尹、和王傅張廷範宜復本官。光祿
 大夫、檢校司徒河東縣開國子、食邑五百戶、充山陵副使、權知河南尹、天平軍節度副使韋震權知鄆州軍州事。



 二年春正月庚申朔,楊行密陷鄂州,執節度使杜洪,斬於揚州市。鄂、岳、蘄、黃等州入行密。全忠自霍丘還大梁。甲子,太常卿王溥上大行皇帝謚號、廟號,乃敕右僕射、平章事裴樞撰謚冊,中書侍郎柳璨撰哀冊。辛未,敕:「朕祗荷丕圖,仰惟元訓,方迫遺弓之痛,俯臨同軌之期。將
 展孝思,親扶護衛。皇太后義深鳴鳳,痛切攀龍,亦欲專奉靈輿,躬及園寢,兼盡追摧之道,用終克敬之儀。其大行皇帝山陵發引日,朕隨太后親至陵所,付中書門下,宜體至懷。」群臣三表論諫,乃止。



 二月庚寅朔。壬辰,制以前知鄜州軍州事、檢校尚書左僕射劉鄩為右金吾衛大將軍,充右街使。檢校左僕射硃漢賓為右羽林統軍。丙申,群臣告謚於西宮。己亥,敕:「今月十一日,大行皇帝啟攢宮。準故事,坊市禁音樂,至二十日掩玄宮畢,如舊。」
 庚子,啟攢宮,文武百僚夕臨於西宮。丁未,靈駕發引,濮王已下從,皇帝、太后長樂門外祭畢歸大內。己酉,葬昭宗皇帝於和陵。庚戌,制以太常卿王溥為工部尚書。壬子,制以汝州刺史裴迪為刑部尚書。泰寧軍節度、檢校司空、兗州刺史、御史大夫葛從周檢校司徒、兼右金吾上將軍致仕,從周病風,不任朝謁故也。以左金吾上將軍盧彥威為左威衛上將軍。是月社日,樞密使蔣玄暉宴德王裕已下九王於九曲池,既醉,皆絞殺之,竟不知
 其瘞所。丙辰,左僕射裴贄等議遷廟,合遷順宗一室,從之。己未,昭宗皇帝神主祔太廟,禮院奏昭宗廟樂,曰《咸寧之舞》。



 三月庚申朔。壬戌,制以前平盧軍節度使、檢校太傅、同平章事、兼青州刺史、上柱國、瑯邪郡公、食邑二千五百戶王師範為孟州刺史、河陽三城懷孟節度觀察等使,從全忠奏也。甲子,制以特進、尚書右僕射、門下侍郎、同平章事、太清宮使、弘文館大學士、延資庫使、諸道鹽鐵轉運使、判度支、上柱國、河東郡開國公、食邑二
 千戶裴樞可守尚書左僕射。光祿大夫、門下侍郎、戶部尚書、同平章事、監修國史、河南縣開國子、食邑五百戶獨孤損可檢校尚書左僕射、同平章事,兼安南都護,充靜海軍節度、安南管內觀察處置等使。以光祿大夫、中書侍郎、同平章事、集賢殿大學士、上柱國、博陵郡開國公、食邑一千五百戶崔遠可守尚書右僕射。以正議大夫、中書侍郎、同平章事,判戶部事、上柱國、河東縣男、食邑三百戶柳璨為門下侍郎、兼戶部尚書、同平章事、太
 清宮使、弘文館大學士、延資庫使、諸道鹽鐵轉運等使。以正議大夫、尚書吏部侍郎、上柱國、賜紫金魚袋張文蔚為中書侍郎、同平章事、監修國史、判度支。以銀青光祿大夫、行尚書左丞、上柱國、弘農縣伯、食邑七百戶楊涉為中書侍郎、同平章事、集賢殿大學士、判戶部事。庚午,敕:「朕以宰臣學士,文武百僚,常拘官局,空逐游從。今膏澤不愆,豐年有望,當茲韶景,宜示優恩。自今月十二日後至十六日,各令取便選勝追游。付所司。」壬申,以檢
 校司徒、和王傅張廷範為太常卿。丁亥,敕:「翰林學士、戶部侍郎楊注是宰臣楊涉親弟,兄既秉於樞衡,弟故難居宥密,可守本官,罷內職。」四月己丑朔。壬辰,敕河南府緱氏縣令宜兼充和陵臺令,仍升為赤縣。癸巳,敕曰:「文武二柄,國家大綱,東西兩班,官職同體。咸匡聖運,共列明廷,品秩相對於高卑,祿俸皆均於厚薄。不論前代,祗考本朝。太宗皇帝以中外臣僚,文武參用,或自軍衛而居臺省,亦由衣冠而秉節旄,足明於武列文班,不令分
 清濁優劣。近代浮薄相尚,凌蔑舊章,假偃武以修文,競棄本而逐末。雖藍衫魚簡,當一見而便許升堂;縱拖紫腰金,若非類而無令接席。以是顯揚榮辱,分別重輕,遽失人心,盡隳朝體。致其今日,實此之由,須議改更,漸期通濟。文武百官,自一品以下,逐月所給料錢並須均勻,數目多少,一般支給。兼差使諸道,亦依輪次,既就公平,必期開泰。凡百臣庶,宜體朕懷。」和王傅張廷範者,全忠將吏也,以善音律,求為太常卿,全忠薦用之。宰相裴樞
 以廷範非樂卿之才,全忠怒,罷樞相位。柳璨希旨,又降此詔斥樞輩,故有白馬之禍。丙午,前棣州刺史劉仁遇檢校司空,兼兗州刺史、御史大夫,充泰寧軍節度使。乙未,制左僕射裴樞、新除清海軍節度使獨孤損、河南尹張全義、工部尚書王溥、司空致仕裴贄、刑部尚書張禕,並賜一子八品正員官,以奉山陵之勞也。敕曰:「朕以宿麥未登,時陽久亢,慮闕粢盛之備,軫予宵旰之懷。所宜避正位於宸居,減珍羞於常膳,諒惟眇質,深合罪躬。自
 今月八日已後,不御正殿,減常膳。付所司。」辛丑,侍御史李光庭郗殷象、殿中丞張升崔昭矩、起居舍人盧仁烱盧鼎蘇楷、吏部員外郎崔協、左補闕崔咸休、右補闕杜承昭羅兗、右拾遺韋彖路德延,並宜賜緋魚袋;兵部郎中韋乾美、比部郎中楊煥,皆賜紫金魚袋:並以奉山陵之勞也。壬寅,敕:「朕獲荷丕圖,仰遵慈訓,爰崇徽號,已定禮儀,冀申為子之心,以展奉親之敬。昨所司定今月二十五日行皇太后冊禮。再奉慈旨,以宮殿未停工作,蒸
 暑不欲勞人,宜改吉辰,固難違命。冊禮俟修大內畢功日,所司以聞。」癸卯,太清宮使柳璨奏修上清宮畢,請改為太清宮,從之。甲辰夜,彗起北河,貫文昌,其長三丈,在西北方。丁未,敕:「設官分職,各有司存,銓衡既任於吏曹,除授寧煩於宰職。但所司注擬申到,中書過驗酌量,茍或差舛,難可盡定。近年除授,其徒實繁,占選部之闕員,擇公當之優便,遂致三銓注擬之時,皆曠職務。且以宰相之任,提舉百司,唯務公平無私,方致漸臻有道。應天
 下州府令錄,並委吏部三銓注擬。自天祐二年四月十一日已後,中書並不除授,或諸薦奏量留,即度可否施行。庶各司其局,免致紊隳,宰相提綱,永存事體。付所司。」辛亥,以彗孛謫見,德音放京畿軍鎮諸司禁囚,常赦不原外,罪無輕重,遞減一等,限三日內疏理聞奏。壬子,敕:「朕以沖幼,克嗣丕基,業業兢兢,勤恭夕惕。彗星謫見,罪在朕躬。雖已降赦文,特行恩宥,起今月二十四日後,避正殿,減常膳,以明思過。付所司。」丙辰,敕:「準向來事例,每
 貫抽除外,以八百五十文為貫,每陌八十五文。如聞坊市之中,多以八十為陌,更有除折,頓爽舊規。付河南府,市肆交易,並以八十五文為陌,不得更有改移。」戊午,敕:「東上閣門,西上閣門,比常出入,以東上為先。大忌進名,即西上閣門為便。比因閹官擅權,乃以陰陽取位,不思南面,但啟西門。邇來相承,未議更改,詳其稱謂,似爽舊規。自今年五月一日後,常朝出入,取東上閣門,或遇奉慰,即開西上閣門,永為定制。付所司。」又敕:「朕以上天謫
 見,避殿責躬,不宜朔會朝正殿。其五月一日朝會,宜權停。」五月己未朔,以星變不視朝。敕曰:「天文變見,合事祈禳,宜於太清宮置黃籙道場,三司支給齋料。」壬戌,敕:「法駕遷都之日,洛京再建之初,慮懷土有類於新豐,權更名以變於舊制。妖星既出於雍分,高閎難效於秦餘,宜改舊門之名,以壯卜年之永。延喜門改為宣仁門,重明門改為興教門,長樂門改為光政門,光範門曰應天門,乾化門曰乾元門,宣政門曰敷政門,宣政殿曰貞觀殿,
 日華門曰左延福門,月華門曰右延福門,萬壽門曰萬春門,積慶門曰興善門,含章門曰膺福門,含清門曰延義門,金鑾門曰千秋門,延和門曰章善門,保寧殿曰文思殿。其見在門名,有與西京門同名者,並宜復洛京舊門名。付所司。」乙酉夜,西北彗星長六七十丈,自軒轅大角及天市西垣,光輝猛怒,其長竟天。丙寅,有司修皇太后宮畢。中書奏:「皇太后慈惠臨人,寬仁馭物,早葉伣天之兆,克彰誕聖之符。今輪奐新宮,規摹舊典,崇訓既征
 於信史,積善宜顯於昌期。太后宮請以積善為名。」從之。又以將卜郊禋,預調雅樂,宜以太常卿張廷範充修樂懸使。丁卯,荊襄節度使趙匡凝奏為故使成汭立祠宇,從之。己巳,太清宮使柳璨奏:「近敕改易宮殿門名,竊以玄元皇帝廟,西京曰太清宮,東京曰太微宮,其太清宮請復為太微宮,臣便給入官階。」從之。庚午,敕:「所司定今年十月九日有事郊丘,其修制禮衣祭服宜令宰臣柳璨判,祭器宜令張文蔚、楊涉分判,儀仗車輅宜令太常
 卿張廷範判。」壬申,制新除靜海軍節度使、銀青光祿大夫、檢校尚書左僕射、同平章事、兼安南都護、河南郡開國侯、食邑一千戶獨孤損可責授朝散大夫、棣州刺史,仍令御史臺發遣出京訖聞奏。敕曰:「朕謬將眇質,叨荷丕圖,常懷馭朽之心,每軫泣辜之念。諒於黜責,豈易施行。左僕射裴樞、右僕射崔遠,雖罷機衡,尚居揆路,既處優崇之任,未傷進退之規。不能秉志安家,但恣流言謗國,頗興物論,難抑朝章。須離八座之榮,尚付六條之政,
 勉思咎己,無至尤人。樞可責授朝散大夫、登州刺史,遠可責授朝散大夫、萊州刺史,便發遣出京。」兵部郎中韋乾美貶沂州司戶。甲戌,敕中書舍人封渭貶齊州司戶,右補闕鄭輦密州莒縣尉,兵部員外盧協祁州司戶,並員外置。乙亥,敕吏部尚書陸扆貶濮州司戶,工部尚書王溥淄州司戶。司天奏:「旬朔已前,星文變見,仰觀垂象,特軫聖慈。自今月八日夜已後,連遇陰雨,測候不得。至十三日夜一更三點,天色暫晴,景緯分明,妖星不見於
 碧虛,災沴潛消於天漢者。」敕曰:「上天謫見,下土震驚,致夙夜之沈憂,恐生靈之多難。不居正殿,盡輟常羞,益務齋虔,以申禳禱。果致玄穹覆祐,孛彗消除,豈罪己之感通,免貽人於災沴。式觀陳奏,深慰誠懷。」丙子,敕戶部郎中李仁儉貶和王府咨議,起居舍人盧仁烱安州司戶,壽安尉、直弘文館盧晏滄州東光尉。丁丑,陳許節度使張全義奏:「得許州留後狀申,自多事以來,許州權為列郡,今特創鼓角樓訖,請復為軍額。」敕旨依舊置忠武軍
 牌額。戊寅,宴群臣於崇勛殿,全忠與王鎔、羅紹威置宴也。庚辰,敕特進、檢校司徒、守太保致仕趙崇可曹州司戶,銀青光祿大夫、兵部侍郎王贊可濮州司戶。辛巳,敕責授登州刺史裴樞可隴州司戶,責授棣州刺史獨孤損可瓊州司戶,責授萊州刺史崔遠可白州司戶。壬午,敕司勛員外韋甄責授和王友,洛陽縣令李光序責授左春坊典設郎。甲申,秘書監崔仁魯可密州司戶,國子祭酒崔澄陳州司戶,太府少卿裴金咸徐州司戶,衛尉少
 卿裴紓曹州南華尉,左補闕崔咸休寧陵尉,司封員外薛水高輝州司戶,前鹽鐵推官獨孤憲臨沂尉,秘書少監裴鉥鄆州司戶,長安尉、直史館裴格符離尉,兵部郎中李象鄭州司戶,刑部員外盧薦範縣尉。丙戌,潁州汝陰縣人彭文妻產三男。丁亥,敕以翰林學士、尚書職方郎中張策兼充史館修撰,修國史。



 六月戊子朔,敕:「責授隴州司戶裴樞、瓊州司戶獨孤損、白州司戶崔遠、濮州司戶陸扆、淄州司戶王溥、曹州司戶趙崇,濮州司戶王贊
 等,皆受國恩,咸當重任。罔思罄謁,唯貯奸邪,雖已謫於遐方,尚難寬於國典。委御史臺差人所在州縣各賜自盡。」時樞等七人已至滑州,皆並命於白馬驛,全忠令投尸於河。己丑,敕:「君臣之間,進退以禮,矧於求舊,欲保初終,茍自掇於悔尤,亦須行於黜責。特進、守司空致仕、上柱國、河東縣開國公、食邑二千戶裴贄早以公望,常踐臺司,靡聞竭力以匡時,每務養恬而避事。洎從請老,不謂無恩,合慎樞機,動循規矩。雖雲勇退,乃有後言,自
 為簿從之酋,頗失人臣之禮。謫居郡掾,用正朝綱,可責授青州司戶。刑部郎中李煦可萊州司戶。」辛卯,太微宮使柳璨奏:「前使裴樞充宮使日,權奏請玄元觀為太清宮,又別奏在京弘道觀為太清宮,至今未有制置。伏以今年十月九日陛下親事南禋,先謁聖祖廟,弘道觀既未修葺,玄元觀又在北山,若車駕出城,禮非便穩。今欲只留北邙山上老君廟一所,其玄元觀請拆入都城,於清化坊內建置太微宮,以備車駕行事。」從之。壬辰,敕:「諸道
 節度、觀察、防禦、刺史等,部內有新除朝官、前資朝官,敕到後三日內發遣赴闕,仍差人監送。所在州縣不得停住,茍或稽違,必議貶黜。付所司。」癸巳,敕:「衛尉少卿敬沼是裴贄之甥。常累於舅,或以明經撓文柄,或以私事竊化權。贄已左遷,爾又何追!可貶徐州蕭縣尉。」丙申,敕:「福建每年進橄欖子,比因閹豎出自閩中,牽於嗜好之間,遂成貢奉之典。雖嘉忠藎,伏恐煩勞。今後只供進蠟面茶,其進橄欖子宜停。」戊戌,敕:「密縣令裴練貶登州牟平
 尉,長水令崔仁略淄州高苑尉,福昌主簿陸珣沂州新泰尉,泥水令獨孤韜範縣尉,並員外置,皆裴樞、崔遠、陸扆宗黨也。壬寅,湖南馬殷奏,岳州洞庭、青草之側,有古祠四所,先以荒圮,臣復修廟了畢,乞賜名額者。敕旨黃陵二妃祠曰懿節,洞庭君祠曰利涉侯,青草祠曰安流侯;三閭大夫祠,先以澧朗觀察使雷滿奏,已封昭靈侯,宜依天祐元年九月二十九日敕處分。丙午,全忠奏:「得宰相柳璨記事,欲拆北邙山下玄元觀移入都內,於清化
 坊取舊昭明寺基,建置太微宮,準備十月九日南郊行事。緣延資庫鹽鐵並無物力,令臣商量者。臣已牒判六軍諸衛張全義指揮工作訖。」優詔嘉之。丁未,敕:「太子賓客柳遜嘗為張浚租庸判官,又王溥監修日奏充判官,授工部侍郎,又與趙崇、裴贄為刎頸之交。昨裴樞等得罪之時,合當連坐,尚矜暮齒,且俾懸車,可本官致仕。」戊申,敕前司勛員外郎、賜緋魚袋李延古責授衛尉寺主簿。七月戊午朔。辛酉,賜全忠《迎鑾記功碑文》,立於都內。
 全忠進助郊禮錢三萬貫。癸亥,再貶柳遜曹州司馬。辛巳,敕全忠請鑄河中、晉、絳諸縣印,縣名內有「城」字並落下,如密鄭、絳、蒲例,單名為文。壬午,宰臣柳璨、禮部尚書蘇循充皇太后冊禮使。是日,於積善宮行禮畢,帝乘輦赴太后宮稱賀。丙戌,太常禮院奏:「每月朔望,皇帝赴積善宮起居,文武百官於宮門進名起居。」從之。



 八月丁亥朔。戊子,制中書舍人姚洎可尚書戶部侍郎,充元帥府判官,從全忠奏也。洛苑使奏谷水屯地內嘉禾合穎。乙
 未,敕:「偽稱官階人泉州晉江縣應鄉貢明經陳文巨招伏罪款,付河南府決殺。庚子,敕:「漢代元勛,鄧禹冠諸侯之上;晉朝重位,王導居百闢之先。皆道著匡扶,功宣寰宇,其於崇寵,迥異等倫。朕獲以眇躬,重興丕運,凡關制度,必法舊章,實仗勛賢,永安宗社。副元帥梁王正守太尉、中書令,忠武軍節度使、河南尹張全義亦正守中書令,俱深倚注,咸正臺衡。其朝廷冊禮、告祀天地宗廟,其司空則差官攝行,太尉、侍中、中書令即宰臣攝行。今太
 尉副元帥任冠籓垣,每遇行禮之時,或不在京國,即事須差攝太尉行事。全義見居闕下,任正中樞,不可更差別官又攝中書令事。其太尉官,如梁王朝覲在京,便委行事,如卻赴鎮,即依前攝行。所合差中書令,便委全義以本官行禮。其侍中、司空、司徒即臨時差官。付所司。」壬寅,敕:「前太中大夫、尚書兵部侍郎、賜紫金魚袋司空圖俊造登科,硃紫升籍,既養高以傲代,類移山而釣名。志樂漱流,心輕食祿。匪夷匪惠,難居公正之朝;載省載思,
 當徇幽棲之志。宜放還中條山。」癸卯,敕太常卿張廷範宜充南郊禮儀使。丁未,制削奪荊襄節度使趙匡凝在身官爵。是月乙未,全忠遣大將楊師厚討匡凝,收唐、鄧、復、郢、隨等州,全忠自率親軍赴之。荊襄之軍,陣於漢水之陰。



 九月丁巳朔。辛酉,楊師厚於襄州西六十里陰谷江口伐竹木為浮梁。癸亥,梁成,引軍渡江。甲子,趙匡凝率勁兵二萬,陣於江之湄。師厚一戰敗之,遂乘勝躡之,陣於城下。是夜,匡凝挈其孥潰圍遁去。乙丑,師厚入襄
 陽。丙寅,全忠繼至。壬申,匡凝牙將王建武遣押牙常質以荊南降。言權知荊南軍府事趙匡明今月十一日棄城上峽,奔蜀川。敕曰:「梁王躬臨貔武,收復荊、襄,拔峴首若轉丸,平荊門如沃雪,連收兩鎮,並走二兇。乃睠勛庸,載深嘉注,宜賜詔獎飾。」內出宣旨:「妳婆楊氏可賜號昭儀,妳婆王氏可封郡夫人,第二妳婆王氏先帝已封郡夫人,準楊氏例改封。」中書奏議言:「乳母古無封夫人賜內職之例,近代因循,殊乖典故。昔漢順帝以乳母宋氏為山
 陽君,安帝乳母王氏曰野王君,當時朝議非之。今國祚中興,禮宜求舊。臣等商量,楊氏望賜號安聖君,王氏曰福聖君,第二王氏曰康聖君。」從之。己巳,敕武成王廟宜改為武明王。乙酉,敕先擇十月九日有事郊丘,備物之間,有所未辦,宜改用十一月十九日。十月丙戌朔,制梁王全忠可充諸道兵馬元帥,別開府幕,加食邑通前一萬五千戶,實封一千五百戶。金州馮行襲奏當道昭信軍額內一字,與元帥全忠諱字同,乃賜號戎昭軍。制削
 奪荊南留後趙匡凝官爵。丁亥,敕:「洛城坊曲內,舊有朝臣諸司宅舍,經亂荒榛。張全義葺理已來,皆已耕墾,既供軍賦,即系公田。或恐每有披論,認為世業,須煩按驗,遂啟幸門。其都內坊曲及畿內已耕植田土,諸色人並不得論認。如要業田,一任買置。凡論認者,不在給還之限。如有本主元自差人勾當,不在此限。如荒田無主,即許識認。付河南府。」甲午,起居郎蘇楷駁昭宗謚號曰:「帝王御宇,由理亂以審污隆;宗祀配天,資謚號以定升降。
 故臣下君上皆不得而私也。伏以陛下順考古道,昭彰至公,既當不諱之朝,寧阻上言之路。伏以昭宗皇帝睿哲居尊,恭儉垂化,其於善美,孰敢蔽虧。然而否運莫興,至理猶鬱,遂致四方多事,萬乘頻遷。始則閹豎猖狂,受幽辱於東內;終則嬪嬙悖亂,罹夭閼於中闈。其於易名,宜循考行。有司先定尊謚曰聖穆景文孝皇帝,廟號昭宗,敢言溢美,似異直書。按後漢和、安、順帝,緣非功德,遂改宗稱,以允臣下之請。今郊禋有日,祫祭惟時。將期允
 愜列聖之心,更下詳議新廟之稱。庶使葉先朝罪己之德,表聖主無私之明。」楷,禮部尚書循之子,凡劣無藝。乾寧二年應進士登第後,物論以為濫,昭宗命翰林學士陸扆、秘書監馮渥覆試黜落,永不許入舉場,楷負愧銜怨。至是,全忠弒逆君上,柳璨陷害朝臣,乃與起居郎羅袞、起居舍人盧鼎連署駁議。楷目不知書,手僅能執筆,其文羅袞作也。時政出賊臣,哀帝不能制。太常卿張廷範改謚曰恭靈莊閔孝皇帝,廟號曰襄宗。全忠雄猜物
 鑒,自楷駁謚後,深鄙之,既傳代之後,循、楷父子皆斥逐,不令在朝。丁未,所司改題昭宗神主,輟朝一日,癸丑,敕成德軍宜改為武順,管內槁城縣曰槁平,信都曰堯都,欒城曰欒氏,阜城曰漢阜,臨城為房子,避全忠祖、父名也。



 十一月乙卯朔,敕潞州潞城縣改為潞子,黎城曰黎亭。全忠平荊襄後,遂引軍將攻淮南。行次棗陽,阻雨,比至光州,道險塗潦,人馬饑乏。休止十餘日,乃趨固始。進軍距壽州三十里,壽人閉壁不出,左右言師老不可用。
 是月丙辰,全忠自正陽渡淮而北,至汝陰。全忠深悔此行無益。丁卯,至大梁。時哀帝以此月十九日親祠圓丘,中外百司禮儀法物已備。戊辰,宰相已下於南郊壇習儀,而裴迪自大梁回,言全忠怒蔣玄暉、張廷範、柳璨等謀延唐祚,而欲郊天改元。玄暉、柳璨大懼。庚午,敕曰:「先定此月十九日親禮南郊,雖定吉辰,改卜亦有故事。宜改取來年正月上辛。付所司。」辛巳,制:「回天再造竭忠守正功臣、諸道兵馬元帥、宣武宣義天平護國等軍節度
 觀察處置、修宮闕制置、度支解縣池場、亳州太清宮等使、開府儀同三司、守太尉、中書令、河中尹、汴滑鄆等州刺史、上柱國、梁王、食邑一萬五千戶、實封一千五百戶硃全忠可授相國,總百揆,其以宣武、宣義、天平、護國、天雄、武順、忠武、佑國、河陽、義武、昭義、保義、戎昭、武定、泰寧、平盧、匡國、鎮國、武寧、忠義、荊南二十一道為魏國,仍進封魏王,依前充諸道兵馬元帥、太尉、中書令、宣武宣義天平護國等軍節度觀察處置等使,加食邑五千戶,實
 封八千五百戶,入朝不趨,劍履上殿,贊拜不名,兼備九錫之命,仍擇日備禮冊命。又制以楊師厚為襄州兵馬留後,左龍武統軍張慎思為武寧軍兵馬留後。壬午,中書門下奏:「相國魏王總百揆,百司合呈納本司印。其中書門下印,堂候王仁珪呈納,中書公事,權追中書省印行遣。」從之。甲申,敕河南告成縣改為陽邑,蔡州襄城改為苞孚,同州韓城改為韓原,絳州翼城改為澮川,鄆州鄆城改為萬安,慈州文城改為屈邑,澤州晉城改高都,
 陽城改為濩澤,安州應城改為應陽,洪州豐城改為吳高。全忠令判官司馬鄴讓相國總百揆之命。十二月乙酉朔。戊子,詔蔣玄暉齎手詔赴魏國,不許陳讓錫命。辛卯,制:正議大夫、門下侍郎,兼戶部尚書、同平章事、太微宮使、弘文館大學士、延資庫使,充諸道鹽鐵轉運等使、上柱國、河東縣開國男、食邑三百戶柳璨可光祿大夫、守司空,兼門下侍郎、同平章事、太微宮使、弘文館大學士、延資庫使,充諸道鹽鐵轉運等使,進封河東縣開國
 伯,通前食邑七百戶,充魏國冊禮使。制:「相國魏王曾祖贈太傅茂琳追封魏王,謚宣憲;祖贈太師信追封魏王,謚武元;父贈尚書令誠追封魏王,謚文明。敕右常侍王鉅、太常卿張廷範、給事中崔沂、工部尚書李克助、祠部郎中知制誥張茂樞、膳部員外知制誥杜曉、吏部郎中李光嗣、駕部郎中趙光胤、戶部郎中崔協、比部郎中楊煥、左常侍孔拯、右諫議蕭頎、左拾遺裴彖、右拾遺高濟、職方郎中牛希逸、主客郎中蕭蘧等,隨冊禮使柳璨魏
 國行事。先是,北院宣徽使王殷使壽州行營,構蔣玄暉於全忠,全忠怒,急歸大梁。上令刑部尚書裴迪齎詔慰勞全忠,全忠忿恨,語極不遜,故行相國百揆之命以悅其心。蔣玄暉自至大梁陳訴,全忠怒猶不解。帝憂之。甲午,上召三宰相議其事,柳璨曰:「人望歸元帥,陛下揖讓釋負,今其時也。」帝曰:「運祚去唐久矣,幸為元帥所延。今日天下,非予之天下,神器大寶,歸於有德,又何疑焉。他人傳予意不盡,卿自往大梁,備言此懷。」乃賜璨茶、藥,便
 令進發。乙未,敕:樞密使蔣玄暉宜削在身官爵,送河南府處斬。豐德庫使應頊、尚食使硃建武送河南府決殺。庚子,敕:樞密使及宣徽南院北院並停。其樞密公事,令王殷權知。其兩院人吏,並勒歸中書。其諸司諸道人,並不得到宣徽院。凡有公事,並於中書論請。其延義、千秋兩門,只差小黃門三人勾當,其官健勒歸本軍。敕:「魏王堅辭寵命,過示捴謙。朕以國史所書元帥之任,並以天下為名,爰自近年,改為諸道,既非舊制,須在正名。宜追
 制改為天下兵馬元帥,餘準詔旨處分。」辛丑,敕:「漢宣帝中興,五日一聽朝,歷代通規,永為常式。近代不循舊儀,輒隳制度,既奸邪之得計,致臨視之失常,須守舊規,以循定制。宜每月只許一、五、九日開延英,計九度。其入閣日,仍於延英日一度指揮;如有大段公事,中書門下具榜子奏請開延英,不計日數。付所司。」又敕:「宮嬪女職,本備內任,近年已來,稍失儀制。宮人出內宣命,寀御參隨視朝,乃失舊規,須為永制。今後每遇延英坐朝日,只令
 小黃門祗候引從,宮人不得擅出內門,庶循典儀,免至紛雜。」壬寅,戎昭軍奏收復金州,兵火之後,井邑殘破,請移理所於均州,從之。仍改為武定軍。乙巳,汴州別駕蔣仲伸決殺,玄暉季父也。又敕:「蔣玄暉身居密近,擅弄威權,鬻爵賣官,聚財營第,而苞藏悖逆,稔浸奸邪。雖都市已處於極刑,而屈法尚慊於眾怒,更示焚棄之典,以懲顯負之蹤。宜追削為兇逆百姓,仍委河南府揭尸於都門外,聚眾焚燒。」玄暉死後,王殷、趙殷衡等又譖於全忠
 云:「內人相傳,玄暉私侍積善宮,與柳璨、張廷範為盟誓之交,求興唐祚。」戊申,全忠令知樞密王殷害皇太后何氏於積善宮,又殺宮人阿秋、阿虔,言通導蔣玄暉。己酉,敕以太后喪,廢朝三日。百官奉慰訖。又敕曰:「皇太后位承坤德,有愧母儀。近者兇逆誅夷,宮闈詞連醜狀,尋自崩變,以謝萬方。朕以幼沖,君臨區宇,雖情深號慕,而法難徇私,勉循秦、漢之規,須示追降之典。其遣黃門收所上皇太后寶冊,追廢為庶人,宜差官告郊廟。」庚戌,敕:「朕
 以謬荷丕圖,禮合親謁郊廟,先定來年正月上辛用事。今以宮圍內亂,播於丑聲,難以慚恧之容,入於祖宗之廟。其明年上辛親謁郊廟宜停。」壬子,敕積善宮安福殿宜廢。癸丑,敕光祿大夫、守司空、門下侍郎、平章事、太微宮使、弘文館大學士、延資庫使、諸道鹽鐵轉運使柳璨責授朝議郎,守登州刺史。又敕:「太常卿張廷範、太常少卿裴磵溫鑾、祠部郎中知制誥張茂樞等,蔣玄暉在樞密之時,與柳璨、張廷範共為朋扇,日相往來,假其游宴
 之名,別貯傾危之計。茍安重位,酷陷朝臣,既此陰謀,難寬大闢。柳璨已從別敕處分,廷範可責授萊州司戶。裴磵等常同聚會,固共苞藏,磵可青州北海尉,鑾臨淄尉,茂樞博昌尉,並員外置。」甲寅,敕:「責授登州刺史柳璨,素矜憸巧,每務回邪。幸以庸才,驟居重位,曾無顯效,孤負明恩。詭譎多端,苞藏莫測,但結連於兇險,獨陷害於賢良。罪既貫盈,理須竄殛。可貶密州司戶,再貶長流崖州百姓,委御史臺賜自盡。」是日斬於上東門外。又敕:「張廷
 範性唯庸妄,志在回邪,不能保慎寵榮,而乃苞藏兇險。密交柳璨,深結玄暉,晝議宵行,欺天負地。神祇共怒,罪狀難原。宜除名,委河南府於都市集眾,以五車分裂。溫鑾、裴磵、張茂樞並除名,委於御史臺所在賜自盡。柳璨弟瑀、瑊,送河南府決殺。」



 三年春正月乙卯朔,全忠以四鎮之師七萬,會河北諸軍,屯於深州樂城。戊午,敕右拾遺柳瑗貶洺州雞澤尉,璨疏屬也。乙丑,全忠自汴河赴魏州。丙寅,制:「定亂安國
 功臣、鎮海鎮東軍節度、浙江東西道觀察處置等使、淮南東面行營招討營田安撫兩浙鹽鐵制置發運等使、開府儀同三司、守侍中、兼中書令、杭越兩州刺史、上柱國、吳王、食邑九千戶、實封五百戶錢鏐,總臨兩鎮,制撫三吳。道途阻艱,未行冊命,宜令所司擇日備禮。」己巳夜,魏博節度使羅紹威殺其衙內親軍八千人。戊午,全忠自內黃入魏州。是月,魏博衙外兵五萬自歷亭還,分據紹威貝、博等州,汴軍攻圍之。壬申,敕:「相國總百揆魏王
 頃辭冊命,宜令所司再行冊禮。」辛巳,國子監奉:「奉去年十一月五日敕文,應國學每年與諸道等一例解送兩人,今監生郭應圖等六十人連狀論訴。」敕旨:「取士之科,明經極重,每年人數,已有舊規,去夏條疏,蓋防渝濫。今國子監、河南府俱有論奏,所試明經,宜令準常年例解送禮部,放人多少,酌量施行。但不徇囑求,無致僥幸。付所司。」二月甲申朔,魏博節度使羅紹威宜許於本鎮置三代私廟。癸卯。敕今年禮部所放進士,據依去年人數
 外,更放兩人。



 三月甲寅朔。甲戌,敕:「河中、昭義管內,俱有慈州,地里相去不遠,稱謂時聞錯誤,其昭義管內慈州宜改為惠州。」壬戌,全忠奏河中判官劉崇子匡圖,今年進士登第,遽列高科,恐涉群議,請禮部落下。戊寅,制元帥梁王可兼領諸道鹽鐵轉運等使,判度支戶部事,充三司都制置使。辛巳,敕貶西都留守判官、左諫議大夫鄭賨崖州司戶,尋賜死。四月甲申朔,日有蝕之,在胃十二度。戊申,魏博羅紹威奏:「臣當管博州聊城縣、武陽莘
 縣武水博平高堂等五縣,皆於黃河東岸,其鄉村百姓渡河輸稅不便,與天平軍管界接連,請割屬鄆。」從之。



 五月癸酉朔,追贈故荊南節度使成汭、鄂岳節度使杜洪官爵,仍於本州立祠廟,從全忠奏也。丙申,敕:「天祐二年九月二十日於金州置戎昭軍,割均、房二州為屬郡。比因馮行襲葉贊元勛,克宣丕績,用獎濟師之效,遂行割地之權。今命帥得人,疇庸有秩,其戎昭軍額宜停,其均、房二州卻還山南東道收管。」六月癸未朔,甲申,敕:「襄州
 近因趙匡凝作帥,請別立忠義軍額,既非往制,固是從權。忠義軍額宜停廢,依舊為山南東道節度使。」己亥,權右唐州事衛審符奏,州郭凋殘,又不居要路,請移理所於泌陽縣,從之。制以京兆尹、佑國軍節度使韓建為青州節度使,代王重師;以重師代建為京兆尹。壬寅,敕:「文武百僚每月一度入閣於貞觀殿。貞觀大殿,朝廷正衙,遇正至之辰,受群臣朝賀。比來視朔,未正規儀,今後於崇勛殿入閣。付所司。」左拾遺、充史館修撰裴彖以堂叔
 母危疾在濟源,無兄弟侍疾,乞假寧省,從之。七月壬子朔。己未,全忠始自魏州歸大梁,魏博六州平定。檢校工部尚書、守宗正卿、嗣邠王震停見任,落下襲封,以請告於外也。辛未,皇妹永明公主薨,罷朝三日。



 八月甲辰,全忠復自汴州北渡河,攻滄州。乙未,魏博奏割貝州永濟、廣宗,相州臨河、內黃、洹水、斥丘等六縣隸魏州,從之。



 九月辛亥朔。丁卯,全忠在軍至滄州,軍於長蘆。是月積陰霖雨不止,差官宗禜都門。十月乙未,兩浙錢鏐請於本鎮
 立三代私廟,從之。



 十一月庚戌朔。丙子,廢牛羊司。御廚肉河南府供進,所有進到牛羊,便付河南府收管。十二月己卯朔,淮南偽署宣歙觀察使、檢校司徒王茂章可金紫光祿大夫、檢校太保,從錢鏐奏也。茂章背楊渥,以宣州降錢鏐故也。己丑,全忠奏文武兩班一、五、九朝日,元帥府排比廊飧。敕曰:「百官入朝,兩廊賜食,遷都之後,有司官闕供。元帥梁王欲整大綱,復行故事,俾其班列,益認優隆,宜賜詔獎飾。」甲辰,河陽節度副使孫乘貶崖
 州司戶,尋賜自盡。閏十二月己酉朔,福建百姓僧道詣闕,請為節度使王審知立德政碑,從之。乙丑,華州鎮國節度觀察處置等使額及興德府名,並宜停廢,復為華州刺史,充本州防禦使,仍隸同州為支郡,所管華、商兩州諸縣,先升次赤,次畿並罷,宜依舊名。西都佑國軍作鎮已來,未有屬郡,其金州、商州宜隸為屬郡。京兆府奉先縣本屬馮翊,櫟陽連接下邽,奉先縣宜卻隸同州,櫟陽宜隸華州。丙寅,奪西川節度使王建在身官爵。戊辰,
 李克用與幽州之眾同攻潞州,全忠守將丁會以澤、潞降太原,克用以其子嗣昭為留後。甲戌,全忠燒長蘆營旋軍,聞潞州陷故也。乙亥,貶興唐府少尹孫秘長流愛州,尋賜死,孫乘弟也。



 四年春正月戊寅朔。壬寅,全忠自長蘆至大梁,天子遣御史大夫薛貽矩齎詔慰勞。全忠自弒昭宗之後,岐、蜀、太原,連兵牽制,關西日削。幸羅紹威殺牙軍,全獲魏博六州。將行篡代,欲威臨河朔,乃再興師臨幽、滄,冀仁恭
 父子乞盟,則與之相結,以固王鎔、紹威之心。而自秋迄冬,攻滄州無功,及聞丁會失守,燒營遽還。路由魏州,羅紹威知失勢,恐兵襲己,深贊篡奪之謀,他日如王受禪,必罄六州軍賦以助大禮,全忠深感之。至大梁,會薛貽矩來,乃以臣禮見全忠。貽矩承間密陳禪代之謀,全忠心德之。貽矩還奏曰:「元帥有受代意,陛下深體時事,去茲重負。」帝曰:「此吾素懷也。」乃降詔元帥以二月行傳禪之禮,全忠偽辭。



 二月壬子,詔文武百官以今月七日齊
 赴元帥府。癸丑,宰相百官辭,全忠以未斷表為詞。



 三月戊寅朔,全忠令大將李思安率兵三萬,合魏博之眾,攻掠幽州。思安頓兵臨其郛,會仁恭子守光率兵赴援,思安乃還。庚寅,詔薛貽矩再使大梁,達傳位之旨。甲辰,詔曰:



 敕宰臣文武百闢,籓岳庶尹,明聽朕言。夫大寶之尊,神器之重,儻非德充宇宙,功濟黔黎,著重華納麓之功,彰文命導川之績,允熙帝載,克代天工,則何以統御萬邦,照臨八極。元帥梁王,龍顏瑞質,玉理奇文,以英謀睿
 武定寰瀛,以厚澤深仁撫華夏。神功至德,絕後光前,緹油罕紀其鴻勛,謳誦顯歸於至化。二十年之功業,億兆眾之推崇,邇無異言,遠無異望。朕惟王聖德,光被八紘,宜順玄穹,膺茲寶命。況天文符瑞,雜沓宣明,虞夏昌期,顯於圖籙。萬機不可以久曠,天命不可以久違,神祇葉心,歸於有德。朕敬以天下,傳禪聖君,退居舊籓,以備三恪。今敕宰臣張文蔚、楊涉等率文武百僚,備法駕奉迎梁朝,勉厲肅恭,尊戴明主。沖人釋茲重負,永為虞賓,獲
 奉新朝,慶泰兼極。中外列闢,宜體朕懷。



 乙酉,乃以中書侍郎、平章事張文蔚充冊使,禮部尚書蘇循為副。中書侍郎、平章事楊涉押傳國寶使,翰林學士、中書舍人張策為副。御史大夫薛貽矩為押金寶使,左丞趙光逢為副。甲午,文蔚押文武百僚赴大梁。甲子,行事。冊曰:



 皇帝若曰:咨爾天下兵馬元帥、相國總百揆梁王,朕每觀上古之書,以堯舜為始者,蓋以禪讓之典垂於無窮。故封泰山,禪梁父,略可道者七十二君,則知天下至公,非一
 姓獨有。自古明王聖帝,焦思勞神,惴若納隍,坐以待旦,莫不居之則兢畏,去之則逸安。且軒轅非不明,放勛非不聖,尚欲游於姑射,休彼大庭。矧乎歷數尋終,期運久謝,屬於孤藐,統御萬方者哉!況自懿祖之後,嬖幸亂朝,禍起有階,政漸無象。天綱幅裂,海水橫流,四紀於茲,群生無庇。洎乎喪亂,誰其底綏。洎於小子,粵以幼年,繼茲衰緒。豈茲沖昧,能守洪基?惟王明聖在躬,體於上哲。奮揚神武,戡定區夏,大功二十,光著冊書。北越陰山,南逾
 瘴海,東至碣石,西暨流沙,懷生之倫,罔不悅附。矧予寡昧,危而獲存。今則上察天文,下觀人願,是土德終極之際,乃金行兆應之辰。況十載之間,彗星三見,布新除舊,厥有明徵,謳歌所歸,屬在睿德。今遣持節、銀青光祿大夫、守中書侍郎、同中書門下平章事張文蔚等,奉皇帝寶綬,敬遜於位。於戲!天之歷數在爾躬,允執其中,天祿永終。王其祗顯大禮,享茲萬國,以肅膺天命。



 全忠建國,奉帝為濟陰王,遷於曹州,處前刺史氏叔琮之第。時太
 原、幽州、鳳翔、西川猶稱天祐正朔。天祐五年二月二十一日,帝為全忠所害,時年十七,仍謚曰哀皇帝,以王禮葬於濟陰縣之定陶鄉。中興之初,方備禮改卜,遇國喪而止。明宗時就故陵置園邑,有司請謚曰昭宣光烈孝皇帝,廟號「景宗」。中書覆奏少帝行事,不合稱宗,存謚而已。知禮者亦以宣、景之謚非宜,今只取本謚,載之於紀。



 史臣曰:悲哉!土運之將亡也,五常殆盡,百怪斯呈,宇縣瓜分,皇圖瓦解。昭宗皇帝英猷奮發,志憤陵夷,旁求奇
 傑之才,欲拯淪胥之運。而世途多僻,忠義俱亡,極爵位以待賢豪,罄珍奇而托心腹。殷勤國士之遇,罕有托孤之賢,豢豐而犬豕轉獰,肉飽而虎狼逾暴。五侯九伯,無非問鼎之徒;四岳十連,皆畜無君之跡。雖蕭屏之臣扼腕,巖廊之輔痛心,空銜毀室之悲,寧救喪邦之禍?及扶風西幸,洛邑東遷,如寄珠於盜跖之門,蓄水於尾閭之上,往而不返,夫何言哉!至若川竭山崩,古今同嘆;虎爭龍戰,興替無常。縱胠篋之不仁,亦攫金之有道。曹操請
 刑於椒壺,蓋迫陰謀;馬昭拒命於凌雲,窘於見討。誠知醜跡,得以為詞,而全忠所行,止於殘忍。況自岐遷洛,天子塊然,六軍盡斥於秦人,四面皆環於汴卒。冕旒如寄,纖芥為疑,迎鑾未及於崇朝,剚刃已聞於塗地。立嗣君於南面,斃母後於中闈,黃門與禁旅皆殲,宗室共衣冠並殪。復又盜鐘掩耳,嫁禍於人。何九六之數窮,偶天人之道盡,目擊斯亂,言之傷心。哀帝之時,政由兇族。雖揖讓之令,有類於山陽;而凌逼之權,過逾於侯景。人道浸
 薄,陰騭難征,然以此受終,如何延永!



 贊曰:勛華受命,揖讓告終。逆取順守,仁道已窮。暴則短祚,義則延洪。虞賓之禍,非止
 一宗。



\end{pinyinscope}