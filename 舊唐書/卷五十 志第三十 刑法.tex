\article{卷五十 志第三十 刑法}

\begin{pinyinscope}

 古之聖人,為人父母,莫不制禮以崇敬,立刑以明威,防閑於未然,懼爭心之將作也。故有輕重三典之異,宮墨五刑之差,度時而施宜,因事以議制。大則陳之原野,小
 則肆諸市朝,以御奸宄,用懲禍亂。興邦致理,罔有弗由於此者也。暨淳樸既消,澆偽斯起,刑增為九,章積三千,雖有凝脂次骨之峻,而錐刀之末,盡爭之矣。自漢迄隋,世有增損,而罕能折衷。隋文帝參用周、齊舊政,以定律令,除苛慘之法,務在寬平。比及晚年,漸亦滋虐。煬帝忌刻,法令尤峻,人不堪命,遂至於亡。



 高祖初起義師於太原,即布寬大之令。百姓苦隋苛政,競來歸附。旬月之間,遂成帝業。既平京城,約法為十二條。惟制殺人、劫盜、背
 軍、叛逆者死,餘並蠲除之。及受禪,詔納言劉文靜與當朝通識之士,因開皇律令而損益之,盡削大業所用煩峻之法。又制五十三條格,務在寬簡,取便於時。尋又敕尚書左僕射裴寂、尚書右僕射蕭瑀及大理卿崔善為、給事中王敬業、中書舍人劉林甫顏師古王孝遠、涇州別駕靖延、太常丞丁孝烏、隋大理丞房軸、上將府參軍李桐客、太常博士徐上機等,撰定律令,大略以開皇為準。於時諸事始定,邊方尚梗,救時之弊,有所未暇,惟正
 五十三條格,入於新律,餘無所改。至武德七年五月奏上,乃下詔曰:



 古不云乎,「萬邦之君,有典有則。」故九疇之敘,興於夏世,兩觀之法,大備隆周。所以禁暴懲奸,弘風闡化,安民立政,莫此為先。自戰國紛擾,恃詐任力,苛制煩刑,於茲競起。秦並天下,隳滅禮教,恣行酷烈,害虐蒸民,宇內騷然,遂以顛覆。漢氏撥亂,思易前軌,雖復務從約法,蠲削嚴刑,尚行菹醢之誅,猶設錙銖之禁。字民之道,實有未弘,刑措之風,以茲莫致。爰及魏、晉,流弊相沿,
 寬猛乖方,綱維失序。下凌上替,政散民凋。皆由法令湮訛,條章混謬。自斯以後,宇縣瓜分,戎馬交馳,未遑典制。有隋之世,雖云厘革,然而損益不定,疏舛尚多,品式章程,罕能甄備。加以微文曲致,覽者惑其淺深,異例同科,用者殊其輕重,遂使奸吏巧詆,任情與奪,愚民妄觸,動陷羅網,屢聞厘革,卒以無成。



 朕膺期受籙,寧濟區宇,永言至治,興寐為勞。補千年之墜典,拯百王之餘弊,思所以正本澄源,式清流末,永垂憲則,貽範後昆。爰命群才,
 修定科律。但今古異務,文質不同,喪亂之後,事殊曩代,應機適變,救弊斯在。是以斟酌繁省,取合時宜,矯正差遺,務從體要。迄茲歷稔,撰次始畢,宜下四方,即令頒用。庶使吏曹簡肅,無取懸石之多;奏讞平允,靡競錐刀之末。勝殘去殺,此焉非遠。



 於是頒行天下。



 及太宗即位,又命長孫無忌、房玄齡與學士法官,更加厘改。戴胄、魏徵又言舊律令重,於時議絞刑之屬五十條。免死罪,斷其右趾,應死者多蒙全活。太宗尋又愍其受刑之苦,謂侍
 臣曰:「前代不行肉刑久矣,今忽斷人右趾,意甚不忍。」諫議大夫王珪對曰:「古行肉刑,以為輕罪。今陛下矜死刑之多,設斷趾之法,格本合死,今而獲生。刑者幸得全命,豈憚去其一足?且人之見者,甚足懲誡。」上曰:「本以為寬,故行之。然每聞惻愴,不能忘懷。」又謂蕭瑀、陳叔達等曰:「朕以死者不可再生,思有矜愍,故簡死罪五十條,從斷右趾。朕復念其受痛,極所不忍。」叔達等咸曰:「古之肉刑,乃在死刑之外。陛下於死刑之內,改從斷趾,便是以生
 易死,足為寬法。」上曰:「朕意以為如此,故欲行之。又有上書言此非便,公可更思之。」其後蜀王法曹參軍裴弘獻又駁律令不便於時者四十餘事,太宗令參掌刪改之。弘獻於是與玄齡等建議,以為古者五刑,刖居其一。及肉刑廢,制為死、流、杖、笞凡五等,以備五刑。今復設刖足,昌為六刑。減死在於寬弘,加刑又加煩峻。乃與八座定議奏聞,於是又除斷趾法,改為加役流三千里,居作二年。



 又舊條疏,兄弟分後,廕不相及,連坐俱死,祖孫配
 沒。會有同州人房強,弟任統軍於岷州,以謀反伏誅,強當從坐。太宗嘗錄囚徒,憫其將死,為之動容。顧謂侍臣曰:「刑典仍用,蓋風化未洽之咎。愚人何罪,而肆重刑乎?更彰朕之不德也。用刑之道,當審事理之輕重,然後加之以刑罰。何有不察其本而一概加誅,非所以恤刑重人命也。然則反逆有二:一為興師動眾,一為惡言犯法。輕重有差,而連坐皆死,豈朕情之所安哉?」更令百僚詳議。於是玄齡等復定議曰:「案禮,孫為王父尸。案令,祖
 有廕孫之義。然則祖孫親重而兄弟屬輕,應重反流,合輕翻死,據禮論情,深為未愜。今定律,祖孫與兄弟緣坐,俱配沒。其以惡言犯法不能為害者,情狀稍輕,兄弟免死,配流為允。」從之。自是比古死刑,殆除其半。



 玄齡等遂與法司定律五百條,分為十二卷:一曰名例,二曰衛禁,三曰職制,四曰戶婚,五曰廄庫,六曰擅興,七曰賊盜,八曰鬥訟,九曰詐偽,十曰雜律,十一曰捕亡,十二曰斷獄。有笞、杖、徒、流、死,為五刑。笞刑五條,自笞十至五十;杖刑五條,
 自杖六十至杖一百;徒刑五條,自徒一年,遞加半年,至三年;流刑三條,自流二千里,遞加五百里,至三千里;死刑二條:絞、斬。大凡二十等。又有議請減贖當免之法八:一曰議親,二曰議故,三曰議賢,四曰議能,五曰議功,六曰議貴,七曰議賓,八曰議勤。八議者,犯死罪者皆條所坐及應議之狀奏請,議定奏裁。流罪已下,減一等。若官爵五品已上,及皇太子妃大功已上親,應議者周以上親,犯死罪者上請。流罪已下,亦減一等。若七品已上官,
 及官爵得請者之祖父母、父母、兄弟、姊妹、妻、子孫,犯流罪已下,各減一等。若應議請減及九品已上官,若官品得減者之祖父母、父母、妻、子孫,犯流罪已下,聽贖。其贖法:笞十,贖銅一斤,遞加一斤,至杖一百,則贖銅十斤。自此已上,遞加十斤,至徒三年,則贖銅六十斤。流二千里者,贖銅八十斤;流二千五百里者,贖銅九十斤;流三千里者,贖銅一百斤。絞斬者,贖銅一百二十斤。又許以官當罪。以官當徒者,五品已上犯私罪者,一官當徒二年;九品已上,一官當徒一年。若犯公
 罪者,各加一年。以官當流者,三流同比徒四年,仍各解見任。除名者,比徒三年。免官者,比徒二年。免所居官者,比徒一年。又有十惡之條:一曰謀反,二曰謀大逆,三曰謀叛,四曰謀惡逆,五曰不道,六曰大不敬,七曰不孝,八曰不睦,九曰不義,十曰內亂。其犯十惡者,不得依議請之例。年七十以上、十五以下及廢疾,犯流罪以下,亦聽贖。八十已上、十歲以下及篤疾,犯反逆殺人應死者,上請,盜及傷人,亦收贖,餘皆勿論。九十以上、七歲以下,雖
 有死罪,不加刑。比隋代舊律,減大闢者九十二條,減流入徒者七十一條。其當徒之法,唯奪一官,除名之人,仍同士伍。凡削煩去蠹,變重為輕者,不可勝紀。



 又定令一千五百九十條,為三十卷。貞觀十一年正月,頒下之。又刪武德、貞觀已來敕格三千餘件,定留七百條,以為格十八卷,留本司施行。斟酌今古,除煩去弊,甚為寬簡,便於人者。以尚書省諸曹為之目,初為七卷。其曹之常條,但留本司者,別為《留司格》一卷。蓋編錄當時制敕,永為
 法則,以為故事。《貞觀格》十八卷,房玄齡等刪定。《永徽留司格》十八卷,《散頒格》七卷,長孫無忌等刪定,永徽中,又令源直心等刪定,惟改易官號曹局之名,不易篇目。《永徽留司格後本》,劉仁軌等刪定。《垂拱留司格》六卷,《散頒格》三卷,裴居道刪定。《太極格》十卷,岑羲等刪定。《開元前格》十卷,姚崇等刪定。《開元後格》十卷,宋璟等刪定。皆以尚書省二十四司為篇目。凡式三十有三篇,亦以尚書省列曹及秘書、太常、司農、光祿、太僕、太府、少府及監門、
 宿衛、計帳名其篇目,為二十卷。《永徽式》十四卷,《垂拱》、《神龍》、《開元式》並二十卷,其刪定格令同。



 太宗又制在京見禁囚,刑部每月一奏,從立春至秋分,不得奏決死刑。其大祭祀及致齋、朔望、上下弦、二十四氣、雨未晴、夜未明、斷屠日月及假日,並不得奏決死刑。其有赦之日,武庫令設金雞及鼓於宮城門外之右,勒集囚徒於闕前,撾鼓千聲訖,宣詔而釋之。其赦書頒諸州,用絹寫行下。又系囚之具,有枷、杻鉗、鎖,皆有長短廣狹之制,量罪輕重,
 節級用之。其杖皆削去節目,長三尺五寸。訊囚杖,大頭徑三分二厘,小頭二分二厘。常行杖,大頭二分七厘,小頭一分七厘。笞杖,大頭二分,小頭一分半。其決笞者,腿分受。決杖者,背、腿、臀分受。及須數等拷訊者,亦同。其拷囚不過三度,總數不得過二百。杖罪已下,不得過所犯之數。諸斷罪而無正條,其應出罪者,則舉重以明輕;其應入罪者,則舉輕以明重。稱加者,就重次;稱減者,就輕次。惟二死三流,同為一減,不得加至於死。斷獄而失於
 出入者,以其罪罪之。失入者,各減三等;失出者,各減五等。



 初,太宗以古者斷獄,必訊於三槐九棘之官,乃詔大闢罪,中書、門下五品已上及尚書等議之。其後河內人李好德,風疾瞀亂,有妖妄之言,詔按其事。大理丞張蘊古奏,好德癲病有征,法不當坐。治書侍御史權萬紀,劾蘊古貫相州,好德之兄厚德,為其刺史,情在阿縱,奏事不實。太宗曰:「吾常禁囚於獄內,蘊古與之弈棋,今復阿縱好德,是亂吾法也。」遂斬於東市。既而悔之。又交州都
 督盧祖尚,以忤旨斬於朝堂,帝亦追悔。下制,凡決死刑,雖令即殺,仍三覆奏。尋謂侍臣曰:「人命至重,一死不可再生。昔世充殺鄭頲,既而悔之,追止不及。今春府史取財不多,朕怒殺之,後亦尋悔,皆由思不審也。比來決囚,雖三覆奏,須臾之間,三奏便訖,都未得思,三奏何益?自今已後,宜二日中五覆奏,下諸州三覆奏。又古者行刑,君為徹樂減膳。朕今庭無常設之樂,莫知何徹,然對食即不啖酒肉。自今已後,令與尚食相知,刑人日勿進酒
 肉。內教坊及太常,並宜停教。且曹司斷獄,多據律文,雖情在可矜,而不敢違法,守文定罪,或恐有冤。自今門下覆理,有據法合死而情可宥者,宜錄狀奏。」自是全活者甚眾。其五覆奏,以決前一日、二日覆奏,決日又三覆奏。惟犯惡逆者,一覆奏而已,著之於令。



 太宗既誅張蘊古之後,法官以出罪為誡,時有失入者,又不加罪焉,由是刑網頗密。帝嘗問大理卿劉德威曰:「近來刑網稍密,何也?」德威對曰:「律文失入減三等,失出減五等。今失入則無
 辜,失出則便獲大罪,所由吏皆深文。」太宗然其言。由是失於出入者,令依律文,斷獄者漸為平允。十四年,又制流罪三等,不限以里數,量配邊惡之州。其後雖存寬典,而犯者漸少。



 高宗即位,遵貞觀故事,務在恤刑。嘗問大理卿唐臨在獄系囚之數,臨對曰:「見囚五十餘人,惟二人合死。」帝以囚數全少,怡然形於顏色。永徽初,敕太尉長孫無忌、司空李勣、左僕射於志寧、右僕射行成、侍中高季輔、黃門侍郎宇文節柳奭、右丞段寶玄、太常少卿
 令狐德棻、吏部侍郎高敬言、刑部侍郎劉燕客、給事中趙文恪,中書舍人李友益、少府丞張行實、大理丞元紹、太府丞王文端、刑部郎中賈敏行等,共撰定律令格式。舊制不便者,皆隨刪改。遂分格為兩部:曹司常務為《留司格》,天下所共者為《散頒格》。其《散頒格》下州縣,《留司格》但留本司行用焉。三年,詔曰:「律學未有定疏,每年所舉明法,遂無憑準。宜廣召解律人條義疏奏聞。仍使中書、門下監定。」於是太尉趙國公無忌、司空英國公勣、尚書
 左僕射兼太子少師監修國史燕國公志寧、銀青光祿大夫刑部尚書唐臨、太中大夫守大理卿段寶玄、朝議大夫守尚書右丞劉燕客、朝議大夫守御史中丞賈敏行等,參撰《律疏》,成三十卷,四年十月奏之,頒於天下。自是斷獄者皆引疏分析之。永徽五年五月,上謂侍臣曰:「獄訟繁多,皆由刑罰枉濫,故曰刑者成也,一成而不可變。末代斷獄之人,皆以苛刻為明,是以秦氏網密秋荼,而獲罪者眾。今天下無事,四海乂安,欲與公等共行寬
 政。今日刑罰,得無枉濫乎?」無忌對曰:「陛下欲得刑法寬平,臣下猶不識聖意。此法弊來已久,非止今日。若情在體國,即共號癡人,意在深文,便稱好吏。所以罪雖合杖,必欲遣徒,理有可生,務入於死,非憎前人,陷於死刑。陛下矜而令放,法司亦宜固請,但陛下喜怒不妄加於人,刑罰自然適中。」上以為然。永徽六年七月,上謂侍臣曰:「律通比附,條例太多。」左僕射志寧等對:「舊律多比附斷事,乃稍難解。科條極眾,數至三千。隋日再定,惟留五百。
 以事類相似者,比附科斷。今日所停,即是參取隋律修易。條章既少,極成省便。」



 龍朔二年,改易官號,因敕司刑太常伯源直心、少常伯李敬玄、司刑大夫李文禮等重定格式,惟改曹局之名,而不易篇第。麟德二年奏上。至儀鳳中,官號復舊,又敕左僕射劉仁軌、右僕射戴至德、侍中張文瓘、中書令李敬玄、右庶子郝處俊、黃門侍郎來恆、左庶子高智周、右庶子李義琰、吏部侍郎裴行儉馬載、兵部侍郎蕭德昭裴炎、工部侍郎李義琛、刑部侍
 郎張楚、金部郎中盧律師等,刪緝格式。儀鳳二年二月九日,撰定奏上。先是詳刑少卿趙仁本撰《法例》三卷,引以斷獄,時議亦為折衷。後高宗覽之,以為煩文不便。因謂侍臣曰:「律、令、格、式,天下通規,非朕庸虛所能創制。並是武德之際,貞觀已來,或取定宸衷,參詳眾議,條章備舉,軌躅昭然,臨事遵行,自不能盡。何為更須作例,致使觸緒多疑。計此因循,非適今日,速宜改轍,不得更然。」自是,《法例》遂廢不用。



 則天臨朝,初欲大收人望。垂拱初年,
 令熔銅為匭,四面置門,各依方色,共為一室。東面名曰延恩匭,上賦頌及許求官爵者封表投之。南面曰招諫匭,有言時政得失及直言諫諍者投之。西面曰申冤匭,有得罪冤濫者投之。北面曰通玄匭,有玄象災變及軍謀秘策者投之。每日置之於朝堂,以收天下表疏。既出之後,不逞之徒,或至攻訐陰私,謗訕朝政者。後乃令中書、門下官一人,專監其所投之狀,仍責識官,然後許進封,行之至今焉。則天又敕內史裴居道、夏官尚書岑長
 倩、鳳閣侍郎韋方質與刪定官袁智弘等十餘人,刪改格式,加計帳及勾帳式,通舊式成二十卷。又以武德已來、垂拱已前詔敕便於時者,編為《新格》二卷,則天自製序。其二卷之外,別編六卷,堪為當司行用,為《垂拱留司格》。時韋方質詳練法理,又委其事於咸陽尉王守慎,又有經理之才,故《垂拱格》、《式》,議者稱為詳密。其律令惟改二十四條,又有不便者,大抵依舊。



 然則天嚴於用刑,屬徐敬業作亂,及豫、博兵起之後,恐人心動搖,欲以威制
 天下,漸引酷吏,務令深文,以案刑獄。長壽年有上封事言嶺表流人有陰謀逆者,乃遣司刑評事萬國俊攝監察御史就案之,若得反狀,斬決。國俊至廣州,遍召流人,擁之水曲,以次加戮。三百餘人,一時並命,然後鍛煉曲成反狀。乃更誣奏云:「諸道流人,多有怨望。若不推究,為變不遙。」則天深然其言。又命攝監察御史劉光業、王德壽、鮑思恭、王處貞、屈貞筠等,分往劍南、黔中、安南、嶺南等六道,按鞫流人。光業所在殺戮。光業誅九百人,德壽
 誅七百人,其餘少者不減數百人。亦有雜犯及遠年流人,亦枉及禍焉。時周興、來俊臣等,相次受制推究大獄。乃於都城麗景門內,別置推事使院,時人謂之「新開獄」。俊臣又與侍御史侯思止王弘義郭霸李敬仁、評事康暐衛遂忠等,招集告事數百人,共為羅織,以陷良善。前後枉遭殺害者,不可勝數。又造《告密羅織經》一卷,其意旨皆網羅前人,織成反狀。俊臣每鞫囚,無問輕重,多以醋灌鼻。禁地牢中,或盛之於甕,以火圍繞炙之。兼絕其
 糧餉,至有抽衣絮以啖之者。其所作大枷,凡有十號:一曰定百脈,二曰喘不得,三曰突地吼,四曰著即承,五曰失魂膽,六曰實同反,七曰反是實,八曰死豬愁,九曰求即死,十曰求破家。又令寢處糞穢,備諸苦毒。每有制書寬宥囚徒,俊臣必先遣獄卒,盡殺重罪,然後宣示。是時海內懾懼,道路以目。麟臺正字陳子昂上書曰:



 臣聞古之御天下者,其政有三:王者化之,用仁義也;霸者威之,任權智也;強國脅之,務刑罰也。是以化之不足,然後威
 之,威之不足,然後刑之。故至於刑,則非王者之所貴矣。況欲光宅天下,追功上皇,專任刑殺以為威斷,可謂策之失者也。



 臣伏睹陛下聖德聰明,游心太古,將制靜宇宙,保乂黎民,發號施令,出於誠慊。天下蒼生,莫不懸望聖風,冀見神化,道德為政,將侍於陛下矣。臣聞之,聖人出,必有驅除,蓋天人之符,應休命也。日者東南微孽,敢謀亂常。陛下順天行誅,罪惡咸伏,豈非天意欲彰陛下威武之功哉!而執事者不察天心,以為人意,惡其首亂
 唱禍,法合誅屠,將息奸源,窮其黨與。遂使陛下大開詔獄,重設嚴刑,冀以懲奸,觀於天下。逆黨親屬及其交游,有涉嫌疑,辭相連及,莫不窮捕考校,枝葉蟠拿。大或流血,小御魑魅。至有奸人熒惑,乘險相誣,糾告疑似,冀圖爵賞,叫於闕下者,日有數矣。於時朝廷徨徨,莫能自固,海內傾聽,以相驚恐。賴陛下仁慈,憫其危懼,賜以恩詔,許其大功已上,一切勿論。人時獲泰,謂生再造。愚臣竊以忻然,賀陛下聖明,得天之機也。不謂議者異見,又執
 前圖,比者刑獄,紛紛復起。陛下不深思天意,以順休期,尚以督察為理,威刑為務,使前者之詔,不信於人。愚臣昧焉,竊恐非五帝、三王伐罪吊人之意也。



 臣竊觀當今天下百姓,思安久矣。曩屬北胡侵塞,西戎寇邊,兵革相屠,向歷十載。關、河自北,轉輸幽、燕;秦、蜀之西,馳騖湟、海。當時天下疲極矣!重以大兵之後,屬遭兇年,流離饑餓,死喪略半。幸賴陛下以至聖之德,撫寧兆人,邊境獲安,中國無事,陰陽大順,年穀累登,天下父子,始得相養矣。
 揚州構禍,殆有五旬,而海中晏然,纖塵不動,豈非天下蒸庶厭兇亂哉?臣以此卜之,百姓思安久矣。今陛下不務玄默,以救疲民,而又任威刑以失其望,欲以察察為政,肅理寰區。愚臣暗昧,竊有大惑。且臣聞刑者,政之末節也。先王以禁暴厘亂,不得已而用之。今天下幸安,萬物思泰,陛下乃以末節之法,察理平人,愚臣以為非適變隨時之義也。頃年以來,伏見諸方告密。囚累百千輩。大抵所告,皆以揚州為名,及其窮竟,百無一實。陛下仁
 恕,又屈法容之,傍訐他事,亦為推劾。遂使奸臣之黨,快意相讎,睚眥之嫌,即稱有密。一人被告,百人滿獄。使者推捕,冠蓋如市。或謂陛下愛一人而害百人,天下喁喁,莫知寧所。



 臣聞自非聖人,不有外患,必有內憂,物理自然也。臣不敢以古遠言之,請指隋而說。臣聞長老云:隋之末世,天下猶平。煬帝不恭,窮毒威武,厭居皇極,自總元戎,以百萬之師,觀兵遼海,天下始騷然矣。遂使楊玄感挾不臣之勢,有大盜之心,欲因人謀,以竊皇業。及稱
 兵中夏,將據洛陽,哮寔之勢傾宇宙矣。然亂未逾月,而頭足異處。何者?天下之弊,未有土崩,蒸人之心,猶望樂業。煬帝不悟,暗忽人機。自以為元惡既誅,天下無巨猾也,皇極之任,可以刑罰理之。遂使兵部尚書樊子蓋專行屠戮,大窮黨與,海內豪士,無不罹殃。遂至殺人如麻,流血成澤,天下靡然思為亂矣。於是蕭銑、硃粲起於荊南,李密、竇建德亂於河北。四海雲搖,遂並起而亡隋族矣。豈不哀哉!長老至今談之,委曲如是。



 觀三代夏、殷興
 亡,已下至秦、漢、魏、晉理亂,莫不皆以毒刑而致敗壞也。夫大獄一起,不能無濫。何者?刀筆之吏,寡識大方,斷獄能者,名在急刻。文深網密,則共稱至公,爰及人主,亦謂其奉法。於是利在殺人,害在平恕,故獄吏相誡,以殺為詞。非憎於人也,而利在己。故上以希人主之旨,以圖榮身之利。徇利既多,則不能無濫,濫及良善,則淫刑逞矣。夫人情莫不自愛其身,陛下以此察之,豈非無濫矣!冤人籲嗟,感傷和氣;和氣悖亂,群生癘疫;水旱隨之,則有
 兇年。人既失業,則禍亂之心怵然而生矣。頃來亢陽愆候,雲而不雨,農夫釋耒,瞻望嗷嗷,豈不由陛下之有聖德而不降澤於人也?儻旱遂過春,廢於時種,今年稼穡,必有損矣。陛下可不敬承天意,以澤恤人?臣聞古者明王重慎刑罰,蓋懼此也。《書》不云乎,「與其殺不辜,寧失不經。」陛下奈何以堂堂之聖,猶務強國之威。愚臣竊為陛下不取。



 且愚人安則樂生,危則思變。故事有招禍,法有起奸。倘大獄未休,支黨日廣,天下疑惑,相恐無辜,人情
 之變,不可不察。昔漢武帝時巫蠱獄起,江充行詐,作亂京師,至使太子奔走,兵交宮闕,無辜被害者以萬千數。當時劉宗幾覆滅矣,賴武帝得壺關三老上書,幡然感悟,夷江充三族,餘獄不論,天下少以安耳。臣讀書至此,未嘗不為戾太子流涕也。古人云:「前事不忘,後事之師。」伏願陛下念之。今臣不避湯鑊之罪,以螻蟻之命,輕觸宸嚴。臣非不惡死而貪生也,誠以負陛下恩遇,以微命蔽塞聰明,亦非敢欲陛下頓息嚴刑,望在恤刑耳。乞與
 三事大夫,圖其可否。夫往者不可諫,來者猶可追,無以臣微而忽其奏,天下幸甚。



 疏奏不省。



 時司刑少卿徐有功常駁酷吏所奏,每日與之廷爭得失,以雪冤濫,因此全濟者亦不可勝數,語在《有功傳》。及俊臣、弘義等伏誅,刑獄稍息。前後宰相王及善、姚元崇、硃敬則等,皆言垂拱已來身死破家者,皆是枉濫,則天頗亦覺悟。於是監察御史魏靖上言曰:



 臣聞國之綱紀,在乎生殺。其周興、來俊臣、丘神勣、萬國俊、王弘義、侯思止、郭弘霸、李敬仁、
 彭先覺、王德壽、張知默者,即堯年四兇矣。恣騁愚暴,縱虐含毒,讎嫉在位,安忍朝臣,罪逐情加,刑隨意改。當其時也,囚囹如市,朝廷以目。既而素虛不昧,冤魂有托,行惡其報,禍淫可懲,具嚴天刑,以懲亂首。竊見來俊臣身處極法者,以其羅織良善,屠陷忠賢,籍沒以勸將來,顯戮以謝天下。臣又聞之道路,上至聖主,傍洎貴臣,明明知有羅織之事矣,俊臣既死,推者獲功,胡元禮超遷,裴談顯授,中外稱慶,朝廷載安。破其黨者,即能賞不逾時;
 被其陷者,豈可淹之累歲。且稱反徒,須得反狀。惟據片辭,即請行刑,拷楚妄加,款答何限。故徐有功以寬平而見忌,斛瑟羅以妓女而受拘,中外具知,枉直斯在,借以為喻,其餘可詳。臣又聞之,郭弘霸自刺而唱快,萬國俊被遮而遽亡。霍獻可臨終,膝拳於項;李敬仁將死,舌至於臍。皆眾鬼滿庭,群妖橫道,惟徵集應,若響隨聲。備在人謠,不為虛說,伯有晝見,殆無以過。此亦羅織之一據也。臣以至愚,不識大體,儻使平反者數人,眾共詳覆來
 俊臣等所推大獄,庶鄧艾獲申於今日,孝婦不濫於昔時,恩渙一流,天下幸甚。



 疏奏,制令錄來俊臣、丘神勣等所推鞫人身死籍沒者,令三司重推勘,有冤濫者,並皆雪免。



 中宗神龍元年,制以故司僕少卿徐有功,執法平恕,追贈越州都督,特授一子官。又以丘神勣、來子珣、萬國俊、周興、來俊臣、魚承曄、王景昭、索元禮、傅游藝、王弘義、張知默、裴籍、焦仁亶、侯思止、郭霸、李敬仁、皇甫文備、陳嘉言、劉光業、王德壽、王處貞、屈貞筠、鮑思恭二十三
 人,自垂拱已來並枉濫殺人,所有官爵,並令追奪。天下稱慶。時既改易,制盡依貞觀、永徽故事。敕中書令韋安石、禮部侍郎祝欽明、尚書右丞蘇瑰、兵部郎中狄光嗣等,刪定《垂拱格》後至神龍元年已來制敕,為《散頒格》七卷。又刪補舊式,為二十卷,頒於天下。景雲初,睿宗又敕戶部尚書岑羲、中書侍郎陸象先、右散騎常侍徐堅、右司郎中唐紹、刑部員外郎邵知與、刪定官大理寺丞陳義海、右衛長史張處斌、大理評事張名播、左衛率府倉
 曹參軍羅思貞、刑部主事閻義顓凡十人,刪定格、式、律、令。太極元年二月奏上,名為《太極格》。



 開元初,玄宗敕黃門監盧懷慎、紫微侍郎兼刑部尚書李乂、紫微侍郎蘇頲、紫微舍人呂延祚、給事中魏奉古、大理評事高智靜、同州韓城縣丞侯郢璡、瀛州司法參軍閻義顓等,刪定格、式、令,至三年三月奏上,名為《開元格》。六年,玄宗又敕吏部侍郎兼侍中宋璟、中書侍郎蘇頲、尚書左丞盧從願、吏部侍郎裴漼慕容珣、戶部侍郎楊滔、中書舍人劉
 令植、大理司直高智靜、幽州司功參軍侯郢璡等九人,刪定律、令、格、式,至七年三月奏上。律、令、式仍舊名,格曰《開元後格》。十九年,侍中裴光庭、中書令蕭嵩,又以格後制敕行用之後,頗與格文相違,於事非便,奏令所司刪撰《格後長行敕》六卷,頒於天下。二十二年,戶部尚書李林甫又受詔改修格令。林甫遷中書令,乃與侍中牛仙客、御史中丞王敬從,與明法之官前左武衛胄曹參軍崔見、衛州司戶參軍直中書陳承信、酸棗尉直刑部俞
 元杞等,共加刪緝舊格、式、律、令及敕,總七千二十六條。其一千三百二十四條於事非要,並刪之。二千一百八十條隨文損益,三千五百九十四條仍舊不改。總成律十二卷,《律疏》三十卷,《令》三十卷,《式》二十卷,《開元新格》十卷。又撰《格式律令事類》四十卷,以類相從,便於省覽。二十五年九月奏上,敕於尚書都省寫五十本,發使散於天下。其年刑部斷獄,天下死罪惟有五十八人。大理少卿徐嶠上言:大理獄院,由來相傳殺氣太盛,鳥雀不棲,至
 是有鵲巢其樹。於是百僚以幾至刑措,上表陳賀。玄宗以宰相變理、法官平允之功,封仙客為邠國公,林甫為晉國公,刑部大理官共賜帛二千匹。



 自明慶至先天六十年間,高宗寬仁,政歸宮閫。則天女主猜忌,果於殺戮,宗枝大臣,鍛於酷吏,至於移易宗社,幾亡李氏。神龍之後,後族干政,景雲繼立,歸妹怙權。開元之際,刑政賞罰,斷於宸極,四十餘年,可謂太平矣。及塚臣懷邪,邊將內侮,乘輿幸於巴、蜀,儲副立於朔方,曾未逾年,載收京邑,
 書契以來,未有克復宗社若斯之速也。而兩京衣冠,多被脅從,至是相率待罪闕下。而執事者務欲峻刑以取威,盡誅其族,以令天下。議久不定,竟置三司使,以御史大夫兼京兆尹李峴、兵部侍郎呂諲、戶部侍郎兼御史中丞崔器、刑部侍郎兼御史中丞韓擇木、大理卿嚴向等五人為之。初,西京文武官陸大鈞等陷賊來歸,崔器草儀,盡令免冠徒跣,撫膺號泣,以金吾府縣人吏圍之,於朝謝罪,收付大理京兆府獄系之。及陳希烈等大臣
 至者數百人,又令朝堂徒跣如初,令宰相苗晉卿、崔圓、李麟等百僚同視,以為棄辱,宣詔以責之。朝廷又以負罪者眾,獄中不容,乃賜楊國忠宅鞫之。器、諲多希旨深刻,而擇木無所是非,獨李峴力爭之,乃定所推之罪為六等,集百僚尚書省議之。肅宗方用刑名,公卿但唯唯署名而已。於是河南尹達奚珣等三十九人,以為罪重,與從共棄。珣等十一人,於子城西伏誅。陳希烈、張垍、郭納、獨孤朗等七人,於大理寺獄賜自盡。達奚摯、張岯、李
 有孚、劉子英、冉大華二十一人,於京兆府門決重杖死。大理卿張均引至獨柳樹下刑人處,免死配流合浦郡,而達奚珣、韋恆乃至腰斬。先是,慶緒至相州,史思明、高秀巖等皆送款請命,肅宗各令復位,便領所管,至是懼不自安,各率其黨叛。其後三司用刑,連年不定,流貶相繼。及王璵為相,素聞物議,請下詔自今已後,三司推勘未畢者,一切放免,大收人望。後蕭華拔魏州歸國,嘗話於朝云:「初河北官聞國家宣詔放陳希列等脅從官一
 切不問,各令復位,聞者悔歸國之晚,舉措自失。及後聞希烈等死,皆相賀得計,無敢歸者。於是河北將吏,人人益堅,大兵不解。」



 後有毛若虛、敬羽之流,皆深酷割剝,驟求權柄,殺人以逞刑,厚斂以資國。六七年間,大獄相繼,州縣之內,多是貶降人。肅宗復聞三司多濫,嘗悔云:「朕為三司所誤,深恨之。」及彌留之際,以元載為相,乃詔天下流降人等一切放歸。



 代宗寶應元年,回紇與史朝義戰,勝,擒其將士妻子老幼四百八十人。上以婦人雖為
 賊家口,皆是良家子女,被賊逼略,惻然愍之,令萬年縣於勝業佛寺安置,給糧料。若有親屬認者,任還之;如無親族者,任其所適,仍給糧遞過。於是人情莫不感戴忻悅。大歷十四年六月一日,德宗御丹鳳樓大赦。赦書節文:「律、令、格、式條目有未折衷者,委中書門下簡擇理識通明官共刪定。自至德已來制敕,或因人奏請,或臨事頒行,差互不同,使人疑惑,中書門下與刪定官詳決,取堪久長行用者,編入格條。」三司使,準式以御史中丞、中
 書舍人、給事中各一人為之,每日於朝堂受詞,推勘處分。建中二年,罷刪定格令使並三司使。先是,以中書門下充刪定格令使,又以給事中、中書舍人、御史中丞為三司使。至是中書門下奏請復舊,以刑部、御史臺、大理寺為之。其格令委刑部刪定。元和四年九月敕:「刑部大理決斷系囚,過為淹遲,是長奸幸。自今已後,大理寺檢斷,不得過二十日,刑部覆下,不得過十日。如刑部覆有異同,寺司重加不得過十五日,省司量覆不得過本日。如
 有牒外州府節目及於京城內勘,本推即日以報。牒到後計日數,被勘司卻報不得過五日。仍令刑部具遣牒及報牒月日,牒報都省及分察使,各準敕文勾舉糾訪。」



 六年九月,富平縣人梁悅,為父殺仇人秦果,投縣請罪。敕:「復仇殺人,固有彞典。以其申冤請罪,視死如歸,自詣公門,發於天性。志在徇節,本無求生之心,寧失不經,特從減死之法。宜決一百,配流循州。」職方員外郎韓愈獻議曰:



 伏奉今月五日敕:復仇,據禮經則義不同天,徵
 法令則殺人者死。禮法二事,皆王教之端,有此異同,必資論辯,宜令都省集議聞奏者。伏以子復父仇,見於《春秋》,見於《禮記》,又見於《周官》,又見於諸子史,不可勝數,未有非而罪之者也。最宜詳於律,而律無其條,非闕文也。蓋以為不許復仇,則傷孝子之心,而乖先王之訓;許復仇,則人將倚法專殺,無以禁止其端矣。夫律雖本於聖人,然執而行之者,有司也。經之所明者,制有司也。丁寧其義於經,而深沒其文於律者,其意將使法吏一斷於
 法,而經術之士,得引經而議也。《周官》曰:「凡殺人而義者,令勿仇,仇之則死。」義,宜也,明殺人而不得其宜者,子得復仇也。此百姓之相仇者也。《公羊傳》曰:「父不受誅,子復仇可也。」不受誅者,罪不當誅也。又《周官》曰:「凡報仇讎者,書於士,殺之無罪。」言將復仇,必先言於官,則無罪也。今陛下垂意典章,思立定制。惜有司之守,憐孝子之心,示不自專,訪議群下。臣愚以為復仇之名雖同,而其事各異。或百姓相仇,如《周官》所稱,可議於今者;或為官吏所
 誅,如《公羊》所稱,不可行於今者。又《周官》所稱,將復仇,先告於士則無罪者。若孤稚羸弱,抱微志而伺敵人之便,恐不能自言於官,未可以為斷於今也。然則殺之與赦,不可一例。宜定其制曰:凡有復父仇者,事發,具其事由,下尚書省集議奏聞。酌其宜而處之,則經律無失其指矣。



 元和十三年八月,鳳翔節度使鄭餘慶等詳定《格後敕》三十卷,右司郎中崔郾等六人修上。其年,刑部侍郎許孟容、蔣乂等奉詔刪定,復勒成三十卷。刑部侍郎劉
 伯芻等考定,如其舊卷。



 長慶元年五月,御史中丞牛僧孺奏:「天下刑獄,苦於淹滯,請立程限。大事,大理寺限三十五日詳斷畢,申刑部,限三十日聞奏。中事,大理寺三十日,刑部二十五日。小事,大理寺二十五日,刑部二十日。一狀所犯十人以上,所斷罪二十件以上,為大。所犯六人以上,所斷罪十件以上,為中。所犯五人以下,所斷罪十件以下,為小。其或所抵罪狀並所結刑名並同者,則雖人數甚多,亦同一人之例。違者罪有差。」二年四月,
 刑部員外郎孫革奏:「京兆府雲陽縣人張蒞,欠羽林官騎康憲錢米。憲征之,蒞承醉拉憲,氣息將絕。憲男買得,年十四,將救其父。以蒞角牴力人,不敢捴解,遂持木鍤擊蒞之首見血,後三日致死者。準律,父為人所毆,子往救,擊其人折傷,減凡鬥三等。至死者,依常律。即買得救父難是性孝,非暴;擊張蒞是心切,非兇。以髫丱之歲,正父子之親,若非聖化所加,童子安能及此?《王制》稱五刑之理,必原父子之親以權之,慎測淺深之量以別之。《春秋》
 之義,原心定罪。周書所訓,諸罰有權。今買得生被皇風,幼符至孝,哀矜之宥,伏在聖慈。臣職當讞刑,合分善惡。」敕:「康買得尚在童年,能知子道,雖殺人當死,而為父可哀。若從沉命之科,恐失原情之義,宜付法司,減死罪一等。」



 大和七年十二月,刑部奏:「先奉敕詳定前大理丞謝登《新編格後敕》六十卷者。臣等據謝登所進,詳諸理例,參以格式,或事非久要,恩出一時,或前後差殊,或書寫錯誤,並已落下及改正訖。去繁舉要,列司分門,都為五
 十卷。伏請宣下施行。」可之。八年四月,詔應犯輕罪人,除情狀巨蠹,法所難原者,其他過誤罪愆,及尋常公事違犯,不得鞭背。遵太宗之故事也。俄而京兆尹韋長奏:「京師浩穰,奸豪所聚。終日懲罰,抵犯猶多,小有寬容,即難禁戢。若恭守敕旨,則無以肅清;若臨事用刑,則有違詔使。伏望許依前據輕重處置。」從之。



 開成四年,兩省詳定《刑法格》一十卷,敕令施行。



 會昌元年九月,庫部郎中、知制誥紇干泉等奏:「準刑部奏,犯贓官五品已上,合抵死
 刑,請準獄官令死於家者,伏請永為定格。」從之。大中五年四月,刑部侍郎劉彖等奉敕修《大中刑法總要格後敕》六十卷,起貞觀二年六月二十日,至大中五年四月十三日,凡二百二十四年雜敕,都計六百四十六門,二千一百六十五條。七年五月,左衛率倉曹參軍張戣進《大中刑法統類》一十二卷,敕刑部詳定奏行之。



\end{pinyinscope}