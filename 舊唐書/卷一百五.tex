\article{卷一百五}

\begin{pinyinscope}

 ○李乂薛
 登韋湊從子虛心虛舟韓思復曾孫佽張廷珪王求禮辛替否



 李乂,本名尚真,趙州房子人也。少與兄尚一、尚貞俱以文章見稱,舉進士。景龍中,累遷中書舍人。時中宗遣使
 江南分道贖生,以所在官物充直。乂上疏曰:「江南水鄉,採捕為業,魚鱉之利,黎元所資,土地使然,有自來矣。伏以聖慈含育,恩周動植,布天下之大德,及鱗介之微品。雖雲雨之私,有霑於末類;而生成之惠,未洽於平人。何則?江湖之饒,生育無限;府庫之用,支供易殫。費之若少,則所濟何成;用之倘多,則常支有闕。在於拯物,豈若憂人。且鬻生之徒,唯利斯視,錢刀日至,網罟年滋,施之一朝,營之百倍,未若回救贖之錢物,減困貧之徭賦,活國
 愛人,其福勝彼。」



 乂知制誥凡數載。景雲元年,遷吏部侍郎,與宋璟、盧從願同時典選,銓敘平允,甚為當時所稱。尋轉黃門侍郎。時睿宗令造金仙、玉真二觀,乂頻上疏諫,帝每優容之。開元初,特令乂與中書侍郎蘇頲纂集起居注,錄其嘉謨昌言可體國經遠者,別編奏之。乂在門下,多所駁正。開元初,姚崇為紫微令,薦乂為紫微侍郎,外托薦賢,其實引在己下,去其糾駁之權也。俄拜刑部尚書。乂方雅有學識,朝廷稱其有宰相之望,會病卒。兄
 尚一,清源尉,早卒;尚貞,官至博州刺史。兄弟同為一集,號曰《李氏花萼集》,總二十卷。



 薛登,本名謙光,常州義興人也。父士通,大業中為鷹揚郎將。江都之亂,士通與鄉人聞人嗣安等同據本郡,以禦寇賊。武德二年,遣使歸國,高祖嘉之,降璽書勞勉,拜東武州刺史。俄而輔公祏於江都構逆,遣其將西門君儀等寇常州,士通率兵拒戰,大破之,君儀等僅以身免。及公祏平,累功封臨汾侯。貞觀初,歷遷泉州刺史,卒。



 謙
 光博涉文史,每與人談論前代故事,必廣引證驗,有如目擊。少與徐堅、劉子玄齊名友善。文明中,解褐閬中主簿。天授中,為左補闕,時選舉頗濫,謙光上疏曰:



 臣聞國以得賢為寶,臣以舉士為忠。是以子皮之讓國僑,鮑叔之推管仲,燕昭委兵於樂毅,苻堅托政於王猛。子產受國人之謗,夷吾貪共賈之財,昭王錫輅馬以止讒,永固戮樊世以除譖。處猜嫌而益信,行間毀而無疑,此由默而識之,委而察之深也。至若宰我見愚於宣尼,逢萌被
 知於文叔,韓信無聞於項氏,毛遂不齒於平原,此失士之故也。是以人主受不肖之士則政乖,得賢良之佐則時泰,故堯資八元而庶績其理,周任十亂而天下和平。由是言之,則士不可不察,而官不可妄授也。何者?比來舉薦,多不以才,假譽馳聲,互相推獎,希潤身之小計,忘臣子之大猷,非所以報國求賢,副陛下翹翹之望者也。



 臣竊窺古之取士,實異於今。先觀名行之源,考其鄉邑之譽,崇禮讓以勵己,明節義以標信,以敦樸為先最,以
 雕蟲為後科。故人崇勸讓之風,士去輕浮之行。希仕者必修貞確不拔之操,行難進易退之規。眾議以定其高下,郡將難誣於曲直。故計貢之賢愚,即州將之榮辱;穢行之彰露,亦鄉人之厚顏。是以李陵降而隴西慚,干木隱而西河美。故名勝於利,則小人之道消;利勝於名,則貪暴之風扇。是以化俗之本,須擯輕浮。昔冀缺以禮讓升朝,則晉人知禮;文翁以儒林獎俗,則蜀士多儒。燕昭好馬,則駿馬來庭;葉公好龍,則真龍入室。由是言之,未
 有上之所好而下不從其化者也。自七國之季,雖雜縱橫,而漢代求才,猶徵百行。是以禮節之士,敏德自修,閭里推高,然後為府寺所闢。魏氏取人,尤愛放達;晉、宋之後,祗重門資。獎為人求官之風,乖授職惟賢之義。有梁薦士,雅愛屬詞;陳氏簡賢,特珍賦詠。故其俗以詩酒為重,不以修身為務。逮至隋室,餘風尚在,開皇中李諤論之於文帝曰:「魏之三祖,更好文詞,忽君人之大道,好雕蟲之小藝。連篇累牘,不出月露之形;積案盈箱,唯是風
 雲之狀。代俗以此相高,朝廷以茲擢士,故文筆日煩,其政日亂」。帝納李諤之策,由是下制禁斷文筆浮詞。其年,泗洲刺史司馬幼之以表不典實得罪。於是風俗改勵,政化大行。煬帝嗣興,又變前法,置進士等科。於是後生之徒,復相放效,因陋就寡,赴速邀時,緝綴小文,名之策學,不以指實為本,而以浮虛為貴。



 有唐纂歷,雖漸革於故非;陛下君臨,思察才於共理。樹本崇化,惟在旌賢。今之舉人,有乖事實。鄉議決小人之筆,行修無長者之論。
 策第喧競於州府,祈恩不勝於拜伏。或明制才出,試遣搜易又,驅馳府寺之門,出入王公之第。上啟陳詩,唯希咳唾之澤;摩頂至足,冀荷提攜之恩。故俗號舉人,皆稱覓舉。覓為自求之稱,未是人知之辭。察其行而度其材,則人品於茲見矣。徇己之心切,則至公之理乖;貪仕之性彰,則廉潔之風薄。是知府命雖高,異叔度勤勤之讓;黃門已貴,無秦嘉耿耿之辭。縱不能抑己推賢,亦不肯待於三命。豈與夫白駒皎皎,不雜風塵,束帛戔戔,榮高物
 表,校量其廣狹也!是以耿介之士,羞自拔而致其辭;循常之人,舍其疏而取其附。故選司補署,喧然於禮闈;州貢賓王,爭訟於階闥。謗議紛合,浸以成風。夫競榮者必有競利之心,謙遜者亦無貪賄之累。自非上智,焉能不移;在於中人,理由習俗。若重謹厚之士,則懷祿者必崇德以修名;若開趨競之門,邀仕者皆戚施而附會。附會則百姓罹其弊,潔己則兆庶蒙其福。故風化之漸,靡不由茲。今訪鄉閭之談,唯祇歸於里正。縱使名虧禮則,罪
 掛刑章,或冒籍以偷資,或邀勛而竊級,假其不義之賂,則是無犯鄉閭。豈得比郭有道之銓量,茅容望重,裴逸人之賞拔,夏少名高,語其優劣也!



 祇如才應經邦之流,唯令試策;武能制敵之例,只驗彎弧。若其文擅清奇,便充甲第,藻思微減,便即告歸。以此收人,恐乖事實。何者?樂廣假筆於潘岳,靈運詞高於穆之,平津文劣於長卿,子建筆麗於荀彧。若以射策為最,則潘、謝、曹、馬必居孫、樂之右;若使協贊機猷,則安仁、靈運亦無裨附之益。由
 此言之,不可一概而取也。至如武藝,則趙雲雖勇,資諸葛之指捴;周勃雖雄,乏陳平之計略。若使樊噲居蕭何之任,必失指縱之機;使蕭何入戲下之軍,亦無免主之效。鬥將長於摧鋒,謀將審於料事。是以文泉聚米,知隗囂之可圖;陳湯屈指,識烏孫之自解。八難之謀設,高祖追慚於酈生;九拒之計窮,公輸息心於伐宋。謀將不長於弓馬,良相寧資於射策。豈與夫元長自表,妄飾詞鋒,曹植題章,虛飛麗藻,校量其可否也!



 伏願陛下降明制,
 頒峻科。千里一賢,尚不為少,僥幸冒進,須立堤防。斷浮虛之飾詞,收實用之良策,不取無稽之說,必求忠告之言。文則試以效官,武則令其守御,始既察言觀行,終亦循名責實,自然僥幸濫吹之伍,無所藏其妄庸。故晏嬰云:「舉之以語,考之以事;寡其言而多其行,拙於文而工於事。」此取人得賢之道也。其有武藝超絕,文鋒挺秀,有效伎之偏用,無經國之大才,為軍鋒之爪牙,作詞賦之標準。自可試凌雲之策,練穿札之工,承上命而賦《甘泉》,
 稟中軍而令赴敵,既有隨才之任,必無負乘之憂。臣謹案吳起臨戰,左右進劍,吳子曰:「夫提鼓揮桴,臨難決疑,此將事也。一劍之任,非將事也。」謹案諸葛亮臨戎,不親戎服,頓蜀兵於渭南,宣王持劍,卒不敢當。此豈弓矢之用也!謹案楊得意誦長卿之文,武帝曰:「恨不得與此人同時。」及相如至,終於文園令,不以公卿之位處之者,蓋非其所任故也。



 謹案漢法,所舉之主,終身保任。楊雄之坐田儀,責其冒薦;成子之居魏相,酬於得賢。賞罰之令
 行,則請謁之心絕;退讓之義著,則貪競之路消。自然朝廷無爭祿之人,選司有謙捴之士。仍請寬立年限,容其採訪簡汰,堪用者令其試守,以觀能否;參驗行事,以別是非。不實免王丹之官,得人加翟璜之賞,自然見賢不隱,食祿不專。荀彧進鐘繇、郭嘉,劉隱薦李膺、硃穆,勢不云遠。有稱職者受薦賢之賞,濫舉者抵欺罔之罪,自然舉得賢行,則君子之道長矣。



 尋轉水部員外郎,累遷給事中、檢校常州刺史。屬宣州狂寇硃大目作亂,百姓奔
 走,謙光嚴備安輯,闔境肅然。轉刑部侍郎,加銀青光祿大夫,再遷尚書左丞。景雲中,擢拜御史大夫。時僧惠範恃太平公主權勢,逼奪百姓店肆,州縣不能理。謙光將加彈奏,或請寢之,謙光曰:「憲臺理冤滯,何所回避,朝彈暮黜,亦可矣。」遂與殿中慕容玽奏彈之,反為太平公主所構,出為岐州刺史。惠範既誅,遷太子賓客,轉刑部尚書,加金紫光祿大夫、昭文館學士。開元初,為東都留守,又轉太子賓客。以與太子同名,表請行字,特敕賜名登。
 尋以孽子悅千牛為憲司所劾,放歸田里。朝廷以其家貧,又特給致仕祿。七年卒,年七十三,贈晉州刺史。撰《四時記》二十卷。



 韋湊,京兆萬年人。曾祖瓚,隋尚書右丞。祖叔諧,蒲州刺史。父玄,桂州都督府長史。湊,永淳二年,解褐授婺州參軍,累轉揚府法曹參軍。州人前仁壽令孟神爽豪縱,數犯法,交通貴戚,前後官吏莫敢繩按,湊白長史張潛,請因事除之。會神爽坐事推問,湊無所假借,神爽妄稱有
 密旨,究問引虛,遂杖殺之,遠近稱伏。湊,景龍中歷遷將作少匠、司農少卿。嘗以公事忤宗楚客,出為貝州刺史。



 睿宗即位,拜鴻臚少卿,加銀青光祿大夫。景雲二年,轉太府少卿,又兼通事舍人。時改葬節愍太子,優詔加謚;又雪李多祚等罪,還其官爵,仍議更加贈官。湊上書曰:



 臣聞王者發號施令,必法乎天道,使三綱攸敘,十等咸若者,善善明,惡惡著也。善善者,懸爵賞以勸之也;惡惡者,設刑罰以懲之也。其賞罰所不加者,則考行立謚以褒貶
 之,所以勸誡將來也。斯並至公之大猷,非私情之可徇。故箕、微獲用,管、蔡為戮。謚者,臣議其君,子議其父,而曰「靈」曰「厲」者,不敢以私而亂大猷也,則其餘安可失衷哉!



 臣竊見節愍太子與李多祚等擁北軍禁旅,上犯宸居,破扉斬關,突禁而入,兵指黃屋,騎騰紫微。孝和皇帝移御玄武門,親降德音,諭以逆順,而太子據鞍自若,督眾不停。俄而其黨悔非,轉逆為順,或回兵討賊,或投狀自拘。多祚等伏誅,太子方事逃竄。向使同惡相濟,天道無
 征,賊徒闕倒戈之人,侍臣虧陛戟之衛,其為禍也,胡可忍言!於時臣任將作少匠,賜通事舍人內供奉。其明日,孝和皇帝引見供奉官等,雨淚謂曰:「幾不與卿等相見!」其為危懼,不亦甚乎!而今聖朝雪罪禮葬,謚為節愍,以臣愚識,竊所惑焉。



 夫臣子之禮,嚴敬斯極,故過位必趨,蹙路馬芻有誅。昔漢成之為太子也,行不敢絕馳道。當周室之衰微也,秦師過周北門,左右免胄而下,王孫滿猶以其不卷甲束兵,譏其無禮,知其必敗。由是言之,則
 太子稱兵宮內,跨馬御前,悖禮已甚矣,況將更甚乎。而可褒謚,此臣所未諭也。以其斬武三思父子而嘉之乎?然弄兵討逆以安君父,則可嘉也,而乃因欲自取之,是競為逆,可褒謚乎?此又臣所未諭也。將廢韋氏而嘉之乎?然韋氏逆彰義絕,雖誅之亦可也。當此時也,韋氏未有逆彰,未有義絕,於太子為母,豈有廢母之理乎!且既非中宗之命而廢之,是劫父廢母,亦悖逆也,可褒謚乎?此又臣所未諭也。夫君或不君,臣安可不臣?父或不父,
 子安可不子?借如君父有桀、紂之行,臣子無廢殺之理。況先帝功格宇宙,德被生靈,廟號中宗,謚曰孝和皇帝,而逆命之子,可褒謚乎?此又臣所未諭也。



 昔獻公惑驪姬之譖,將殺其太子申生,公子重耳謂之曰:「子盍言子之志於公乎?」太子曰:「不可,君安驪姬,是我傷君之心也。」曰:「然則盍行乎?」曰:「不可,君謂我欲弒君也,天下豈有無父之國哉!吾何行之!」使人辭於狐突曰:「申生不敢愛其死。雖然,吾君老矣,子少,國家多難。伯氏茍出而圖吾君,
 申生受賜而死。」再拜稽首,乃自縊。其行如是,其謚僅可為恭。今太子之行反是,可謚為節愍乎?此又臣所未諭也。



 昔漢武帝末年,江充與太子有隙,恐帝晏駕後為太子所誅。會巫蠱事起,充典理其事。因此為奸,遂至太子宮掘蠱,得桐木以誣太子。時武帝避暑甘泉宮,獨皇后、太子在,太子不能自明,納其少傅石德謀,遂矯節斬充,因敗逃匿。非稱兵詣闕,無逆謀於父,然身死於湖,不葬無謚。至昭帝時,有男子詣北闕自稱衛太子,制使公卿
 識視,至者莫敢發言。京兆尹雋不疑後至,叱從吏收縛之。或曰:「是非未可知,且安之。」不疑曰:「諸君何患於衛太子。昔蒯聵出奔,輒拒而不納,《春秋》是之。衛太子得罪先帝,亡不即死,今來自詣,此罪人也。」遂送制獄。天子聞而嘉之曰:「公卿大臣,當用經術明於大義者。」及後太子孫立為天子,是曰孝宣皇帝,太子方獲禮葬,而謚曰戾。今節愍太子之行比之,豈可同年而語。其於陛下,又猶子也,而謚為節愍乎?此又臣所未諭也。



 昔項羽之臣丁公,
 常將危漢高祖,高祖謂之曰:「二賢豈相厄哉!」丁公乃止。及高祖滅項氏,遂戮丁公以徇,曰:「使項王失天下者,丁公也。」夫戮之,大義至公也,不私德之,所以誡其後之事君者。今節愍太子之為逆,復非欲保護陛下,其可褒謚乎?此又臣之所未諭也。



 陛下天縱聖哲,所任賢明,以臣至愚,寧可幹議?然臣又惟堯、舜,聖君也,八凱、五臣,良佐也,猶廣聽芻蕘之言者,蓋為智者千慮,或有一失,愚者千慮,或有一得也。故曰:「狂夫之言,聖人擇焉。」臣輒緣斯
 義,敢以陳聞,願得與議謚者對議於御前。若臣言非也,甘受謗聖政之罪,赴鼎鑊之誅。仍請申明義以示天下,使臣輩愚惑者咸蒙冰釋,則無復異議矣。若所謚未當,奈何施之聖朝,垂之史冊,使後代逆臣賊子因而引譬,資以為辭,是開悖亂之門,豈示將來之法!伏望改定其謚,務合禮經。其李多祚等罪,請從宥免,不謂為雪,以順天下之心,則盡善盡美矣。



 書奏,睿宗引湊謂曰:「誠如卿言。事已如此,如何改動?」湊曰:「太子實行悖逆,不可褒美,
 請稱其行,改謚以一字。多祚等以兵犯君,非曰無罪,只可云放,不可稱雪。」帝然其言。當時執政以制令已行,難於改易,唯多祚等停贈官而已。



 明年春,起金仙、玉真兩觀,用工巨億。湊進諫曰:「陛下去夏,以妨農停兩觀作,今正農月,翻欲興功。雖知用公主錢,不出庫物,但土木作起,高價雇人,三輔農人,趨目前之利,舍農受雇,棄本逐末。臣聞一夫不耕,天下有受其饑者,臣竊恐不可。」帝不應。湊又奏曰:「日陽和布氣,萬物生育,土木之間,昆蟲無
 數。此時興造,傷殺甚多,臣亦恐非仁聖本旨。」睿宗方納其言,令在外詳議。中書令崔湜、侍中岑羲謂湊曰:「公敢言此,大是難事。」湊曰:「叨食厚祿,死且不辭,況在明時,必知不死。」尋出為陜州刺史,無幾,轉汝州刺史。開元二年夏,敕靖陵建碑,徵料夫匠。湊以自古園陵無建碑之禮,又時正旱儉,不可興功,飛表極諫,工役乃止。尋遷岐州剌史。



 四年,入為將作大匠。時有敕復孝敬廟為義宗,湊上書曰:



 臣聞王者制禮,是曰規模,規模之興,實由師古。
 師古之道,必也正名,名之與實,故當相副。其在宗廟,禮之大者,豈可失哉!禮,祖有功而宗有德,祖宗之廟,百代不毀。故殷太甲為太宗,太戊曰中宗,武丁曰高宗;周宗文王、武王;漢則文帝為太宗,武帝為世宗。其後代有稱宗者,皆以方制海內,德澤可宗,列於昭穆,期於不毀。稱宗之義,不亦大乎!伏惟孝敬皇帝位止東宮,未嘗南面,聖道誠冠於儲副,德教不被於寰瀛,立廟稱宗,恐非合禮。況別起寢廟,不入昭穆,稽諸祀典,何義稱宗?而廟號
 義宗,稱之萬代,以臣庸識,竊謂不可。陛下率循典禮,以闢大猷,有司所議,以致此失,或虧盡善,豈不惜哉!望更詳議,務合於禮。



 於是敕太常議,遂停義宗之號。



 湊前後上書論時政得失,多見採納。再遷河南尹,累封彭城郡公。以公事左授杭州刺史,轉汾州刺史。十年,拜太原尹兼節度支度營田大使。其年卒官,年六十五。贈幽州都督,謚曰文。子見素,自有傳。湊從子虛心。



 虛心父維,少習儒業,博涉文史,舉進士。自大理丞累至戶部郎中,善於
 剖判,時員外郎宋之問工於詩,時人以為戶部有二妙。終於左庶子。虛心舉孝廉,為官嚴整,累至大理丞、侍御史。神龍年,推按大獄,時僕射竇懷貞、侍中劉幽求意欲寬假,虛心堅執法令,有不可奪之志。景龍中,西域羌胡背叛,時並擒獲,有敕盡欲誅之。虛心論奏,但罪元首,其所全者千餘人。虛心有孝行,及丁父憂,哀毀過禮,須鬢盡白,朝廷深所嗟尚。後遷御史中丞、左右丞、兵部侍郎、荊揚潞長史兼採訪使,所在官吏振肅,威令甚舉,中外
 以為標準。歷戶部尚書、東京留守,卒,年六十七。



 季弟虛舟,亦以舉孝廉,自御史累至戶部、司勛、左司郎中,歷荊州長史,洪、魏州刺史兼採訪使,多著能政。入為刑部侍郎,終大理卿。家有禮則,父子兄弟更踐郎署,時稱「郎官家」。



 韓思復,京兆長安人也。祖倫,貞觀中為左衛率,賜爵長山縣男。思復少襲祖爵。初為汴州司戶參軍,為政寬恕,不行杖罰。在任丁憂,家貧,鬻薪終喪制。時姚崇為夏官
 侍郎,知政事,深嘉嘆之,擢授司禮博士。



 景龍中,累遷給事中。時左散騎常侍嚴善思坐譙王重福事下制獄,有司言:「善思昔嘗任汝州刺史,素與重福交游,召至京師,竟不言其謀逆,唯奏云『東都有兵氣』。據狀正當匿反,請從絞刑。」思復駁奏曰:「議獄緩死,列聖明規;刑疑從輕,有國常典。嚴善思往在先朝,屬韋氏擅內,恃寵宮掖,謀危宗社。善思此時遂能先覺,因詣相府有所發明,進論聖躬必登宸極。雖交游重福,蓋謀陷韋氏。及其謁見,猶不
 奏聞,將此行藏,即從極法。且敕追善思,書至便發,向懷逆節,寧即奔命?一面疏網,誠合順生;三驅取禽,來而可宥。惟刑是恤,事合昭詳。請付刑部集群官議定奏裁,以符慎獄。」是時議者多云善思合從原宥,有司仍執前議請誅之。思復又駁曰:「臣聞刑人於市,爵人於朝,必僉謀攸同,始行之無惑。謹按諸司所議,嚴善思十才一入,抵罪惟輕。夫帝閽九重,塗遠千里。故借天下之耳以聽,聽無不聰;借天下之目以視,視無不接。今群言上聞,採擇
 宜審,若棄多就少,臣實懼焉。輿誦一乖,下情不達,雖欲從眾,其可及乎!凡百京司,逢時之泰,列官分職,有賢有親。親則列籓諸王,陛下愛子;賢則胙茅開國,陛下名臣。見無禮於君,寧肯雷同不異?今措詞多出,法令從輕。」上納其奏,竟免善思死,配流靜州。思復尋轉中書舍人,數上疏陳得失,多見納用。



 開元初,為諫議大夫。時山東蝗蟲大起,姚崇為中書令,奏遣使分往河南、河北諸道殺蝗蟲而埋之。思復以為蝗蟲是天災,當修德以禳之,恐
 非人力所能翦滅。上疏曰:「臣聞河南、河北蝗蟲,頃日更益繁熾,經歷之處,苗稼都損。今漸翾飛河西,游食至洛,使命來往,不敢昌言,山東數州,甚為惶懼。且天災流行,埋瘞難盡。望陛下悔過責躬,發使宣慰,損不急之務,召至公之人,上下同心,君臣一德,持此誠實,以答休咎。前後驅蝗使等,伏望總停。《書》云:『皇天無親,惟德是輔;人心無親,惟惠是懷。』不可不收攬人心也。」上深然之,出思復疏以付崇。崇乃請遣思復往山東檢蝗蟲所損之處,及
 還,具以實奏。崇又請令監察御史劉沼重加詳覆,沼希崇旨意,遂箠撻百姓,回改舊狀以奏之。由是河南數州,竟不得免。思復遂為崇所擠,出為德州刺史,轉絳州刺史。入為黃門侍郎,加銀青光祿大夫,代裴漼為御史大夫。思復性恬澹,好玄言,安仁體道,非紀綱之任。無幾,轉太子賓客。十三年卒,年七十餘。



 子朝宗,天寶初為京兆尹。



 曾孫佽,字相之,少有文學,性尚簡澹。舉進士,累闢籓方。自襄州從事徵拜殿中侍御史,遷刑部員外。求為澧
 州刺史。歲滿受代,宰相牛僧孺鎮鄂渚,闢為從事,徵拜刑部郎中,轉京兆少尹,遷給事中。出為桂州觀察使。桂管二十餘郡,州掾而下至邑長三百員,由吏部而補者什一,他皆廉吏量其才而補之。佽既至桂,吏以常所為官者數百人引謁,一吏執籍而前曰:「具員請補其闕。」佽戒曰:「在任有政者,不奪所理;有過者,必繩以法。缺者當俟稽諸故籍,取其可者,然後補之。」會春衣使內官至,求賄於郵吏,三豪家因厚其資以求邑宰,佽悉諾之。使去,
 坐以撓法,各笞其背。自是豪猾斂跡,皆得清廉吏以蘇活其人。未幾,詔置五管都監,計所費盡一境地征,不足飽其意,佽特用儉約處之,遂為定制,君子以為難。開成二年,卒於官,贈工部侍郎。



 張廷珪,河南濟源人,其先自常州徙焉。廷珪少以文學知名,性慷慨,有志尚。弱冠應制舉。長安中,累遷監察御史。則天稅天下僧尼出錢,欲於白司馬阪營建大像。廷珪上疏諫曰:



 夫佛者,以覺知為義,因心而成,不可以諸
 相見也。經云:「若以色見我,以音聲求我,是人行邪道,不能見如來。」此真如之果不外求也。陛下信心歸依,發弘誓願,壯其塔廟,廣其尊容,已遍於天下久矣。蓋有住於相而行布施,非最上第一希有之法。何以言之?經云:「若人滿三千大千世界七寶以用布施,及恆河沙等身命布施,其福甚多。若人於此經中受持及四句偈等為人演說,其福勝彼。」如佛所言,則陛下傾四海之財,殫萬人之力,窮山之木以為塔,極冶之金以為像,雖勞則甚矣,
 費則多矣,而所獲福不愈於一禪房之匹夫。



 菩薩作福德,不應貪著,蓋有為之法不足高也。況此營建,事殷木土,或開發盤礡,峻築基階,或塞穴洞,通轉採斫,輾壓蟲蟻,動盈巨億。豈佛標坐夏之義,愍蠢動而不忍害其生哉!又役鬼不可,唯人是營,通計工匠,率多貧窶,朝驅暮役,勞筋苦骨,簞食瓢飲,晨炊星飯,饑渴所致,疾疹交集。豈佛標徒行之義,愍畜獸而不忍殘其力哉!又營築之資,僧尼是稅,雖乞丐所致,而貧闕猶多。州縣徵輸,星火
 逼迫,或謀計靡所,或鬻賣以充,怨聲載路,和氣未洽。豈佛標隨喜之義,愍愚蒙而不忍奪其產哉!且邊朔未寧,軍裝日給,天下虛竭,海內勞弊。伏惟陛下慎之重之,思菩薩之行為利益一切眾生,應如是布施,則其福德若南西北方四維上下虛空不可思量。夫何必勤於住相,凋蒼生之業,崇不急之務乎!臣以時政論之,則宜先邊境,蓄府庫,養人力;臣以釋教論之,則宜救苦厄,滅諸相,崇無為。伏願陛下察臣之愚,行佛之意,務以理為上,不
 以人廢言,幸甚幸甚。



 則天從其言,即停所作,仍於長生殿召見,深賞慰之。景龍末,為中書舍人,再轉洪州都督,仍為江南西道按察使。



 開元初,入為禮部侍郎。時久旱,關中饑儉,下制求直諫昌言、弘益政理者。廷珪上疏曰:



 臣聞古有多難興王、殷憂啟聖者,皆以事危則志銳,情迫則思深,故能自下登高,轉禍為福者也。伏見景龍之末,中宗遇禍,先天之際,兇黨構謀,社稷有危於綴旒,國朝將均於絕綖。陛下神武超代,精誠動天,再掃氛沴,六
 合清朗。而後上順皇旨,俯念黔黎,高運璿衡,光膺寶籙。日月所燭之地,書軌未通之鄉,無不霑濡渥恩,被服淳化。十堯、九舜,未足稱也。明明上帝,照臨下土,宜錫介祉,以答鴻休。



 然屬頃歲已來,陰陽愆候,九穀失稔,萬姓阻饑,關輔之間,更為尤劇。至有樵蘇莫爨,糧籺靡資,不復聊生,方憂轉死。偶會昌運,遘茲難否者,臣竊思之,皇天之意,將恐陛下春秋鼎盛,神聖在躬,不崇朝而建大功,自籓邸而陟元後,或簡下濟之道,獨滿雄圖之志,輕虞
 舜而不法,思漢武以自高。是故昭見咎徵,載加善誘,將欲大君日慎一日,雖休勿休,永保太和,以固邦本也。斯皇天於陛下睠顧深矣,陛下焉可不奉若休旨而寅畏哉!臣愚誠願陛下約心削志,澄思勵精,考羲、農之書,敦素樸之道。登庸端士,放黜佞人,屏退後宮,減徹外廄,場無蹴𧾷匊之玩,野絕從禽之賞。休石田之遠境,罷金甲之懸軍,矜恤煢嫠,蠲薄徭賦。去奇伎淫巧,捐和璧隋珠,不見可欲,使心不亂。自然波清四海,塵銷九域,農夫樂其
 業,餘糧棲於畝。則和氣上通於天,雖五星連珠,兩曜合璧,未足多也;珍祥下降於地,雖鳳皇巢閣,麒麟在郊,未足奇也。或謂天之炯戒不足畏者,則將上帝憑怒,風雨迷錯,荒饉日甚,無以濟下矣。或謂人之窮乏不足恤者,則將齊甿沮志,億兆攜離,愁苦勢極,無以奉上矣。斯蓋安危所系,禍福之源,奈何朝廷曾不是察!況今陛下受命伊始,敷政惟新,卿士百僚,華夷萬族,莫不清耳以聽,刮目而視,延頸企踵,冀有所聞見,顒顒如也。何可怠棄
 典則,坐辜其望哉!



 再遷黃門侍郎。時監察御史蔣挺以監決杖刑稍輕,敕朝堂杖之,廷珪奏曰:「御史憲司,清望耳目之官,有犯當殺即殺,當流即流,不可決杖。士可殺,不可辱也。」時制命已行,然議者以廷珪之言為是。俄坐洩禁中語,出為沔州刺史,又歷蘇、宋、魏三州刺史。入為少府監,加金紫光祿大夫,封範陽男。四遷太子詹事,以老疾致仕。二十二年卒,年七十餘,贈工部尚書,謚曰貞穆。廷珪素與陳州刺史李邕親善,屢上表薦之,邕所撰
 碑碣之文,必請廷珪八分書之。廷珪既善楷隸,甚為時人所重。



 王求禮,許州長社人。則天朝為左拾遺,遷監察御史。性忠謇敢言,每上封彈事,無所畏避。時契丹李盡忠反叛,其將孫萬榮寇陷河北數州,河內王武懿宗擁兵討之,畏懦不敢進。既而賊大掠而去,懿宗條奏滄、瀛百姓為賊詿誤者數百家,請誅之。求禮執而劾之曰:「此詿誤之人,比無良吏教習,城池又不完固,為賊驅逼,茍徇圖全,
 豈素有背叛之心哉!懿宗擁強兵數十萬,聞賊將至,走保城邑,罪當誅戮。今乃移禍於詿誤之人,豈是為臣之道?請斬懿宗以謝河北百姓。」懿宗大懼,則天竟降制赦之。



 契丹陷幽州,饋輓不給,左相豆盧欽望請輟京官兩月俸料以助軍,求禮謂欽望曰:「公祿厚俸優,輟之可也。國家富有四海,足以儲軍國之用,何藉貧官薄俸。公此舉豈宰相法邪?」欽望作色拒之,乃奏曰:「秦、漢皆有稅算以贍軍,求禮不識大體,妄有訟辭。」求禮對曰:「秦皇、漢武
 稅天下,虛中以事邊,奈何使聖朝則效?不知欽望此言是大體耶!」事遂不行。



 時三月雪,鳳閣侍郎蘇味道等以為瑞,草表將賀,求禮止之曰:「宰相調燮陰陽,而致雪降暮春,災也,安得為瑞?如三月雪為瑞雪,則臘月雷亦瑞雷也。」舉朝嗤笑,以為口實。求禮竟以剛正,名位不達而卒。



 辛替否,京兆人也。景龍年為左拾遺。時中宗置公主府官屬,安樂公主府所補尤多猥濫。又駙馬武崇訓死後,
 棄舊宅別造一宅,侈麗過甚。時又盛興佛寺,百姓勞弊,帑藏為之空竭。替否上疏諫曰:



 臣聞古之建官,員不必備,九卿以下,皆有其位而闕其選。賞一人謀乎三事,職一人訪乎群司,負寵者畏權勢之在躬,知榮者避權門而不入。故稱賞不僭,官不濫,士皆完行,家有廉節,朝廷有餘俸,百姓有餘食。下忠於上,上禮於下,委裘而無倉卒之危,垂拱而無顛沛之患。夫事有惕耳目,動心慮,作不師古,以行於今者,蓋有之矣。伏惟陛下百倍行賞,十
 倍增官,金銀不供其印,束帛無充於錫,何愧於無用之臣,何慚於無力之士!至於公府補授,罕有推擇,遂使富商豪賈,盡居纓冕之流,鬻伎行巫,咸涉膏腴之地。



 臣聞古人曰:「福生有基,禍生有胎。」伏惟公主陛下之愛女,選賢良以嫁之,設官職以輔之,傾府庫以賜之,壯第觀以居之,廣池膋以嬉之,可謂之至重也,可謂之至憐也。然而用不合於古義,行不根於人心,將恐變愛成憎,轉福為禍。何者?竭人之力,人怨也;費人之財,人怨也;奪人之
 家,人怨也。愛數子而取三怨於天下,使邊疆之士不盡力,朝廷之士不盡忠,人之散矣,獨持所愛,何所恃乎?向者魯王賞同諸婿,禮等朝臣,則亦有今日之福,無曩時之禍。人徒見其禍,不知禍之所來。所以禍者,寵愛過於臣子也。去年七月五日,已見其徵矣。而今事無改,更尚因循,棄一宅而造一宅,忘前禍而忽後禍。臣竊謂陛下憎之矣,非愛之也。



 臣聞君以人為本,本固則邦寧。邦寧則陛下夫婦、母子長相保也。伏惟外謀宰臣,為久安之
 計以存之,不使奸臣賊子以伺之。臣聞微不可不防,遠不可不慮。當今疆場危駭,倉廩空虛,揭竿守禦之士賞不及,肝腦塗地之卒輸不充。而方大起寺舍,廣造第宅,伐木空山,不足充梁棟,運土塞路,不足充墻壁。誇古耀今,逾章越制,百僚鉗口四海傷心。夫釋教者,以清凈為基,慈悲為主,故當體道以濟物,不欲利己以損人,故常去己以全真,不為榮身以害教。三時之月,掘山穿池,損命也;殫府虛帑,損人也;廣殿長廊,榮身也。損命則不慈
 悲,損人則不濟物,榮身則不清凈,豈大聖大神之心乎!臣以為非真教,非佛意,違時行,違人欲。自像王西下,佛教東傳,青螺不入於周前,白馬方行於漢後。風流雨散,千帝百王,飾彌盛而國彌空,役彌重而禍彌大。覆車繼軌,曾不改途,晉臣以佞佛取譏,梁主以舍身構隙。若以造寺必為其理體,養人不足以經邦,則殷、周已往皆暗亂,漢、魏已降皆聖明;殷、周已往為不長,漢、魏已降為不短。臣聞夏為天子二十餘代而殷受之,殷為天子二十
 餘代而周受之,周為天子三十餘代而秦受之,自漢已後歷代可知也。何者?有道之長,無道之短,豈因其窮金玉、修塔廟,方得久長之祚乎!



 臣聞於經曰:「菩薩心住於法而行布施,如人入暗,即無所見。」又曰:「一切有為法,如夢幻泡影,如露亦如電。」臣以減雕琢之費以賑貧下,是有如來之德;息穿掘之苦以全昆蟲,是有如來之仁;罷營構之直以給邊陲,是有湯、武之功;回不急之祿以購廉清,是有唐、虞之理。陛下緩其所急,急其所緩,親未來
 而疏見在,失真實而冀虛無,重俗人之所為而輕天子之功業,臣竊痛之矣。當今出財依勢者盡度為沙門,避役奸訛者盡度為沙門;其所未度,唯貧窮與善人。將何以作範乎?將何以役力乎?臣以為出家者,舍塵俗,離朋黨,無私愛。今殖貨營生,非舍塵俗;拔親樹知,非離朋黨;畜妻養孥,非無私愛。是致人以毀道,非廣道以求人。伏見今之宮觀臺榭,京師之與洛陽,不增修飾,猶恐奢麗。陛下尚欲填池塹,捐苑囿,以賑貧人無產業者。今天下
 之寺蓋無其數,一寺當陛下一宮,壯麗之甚矣!用度過之矣!是十分天下之財而佛有七八,陛下何有之矣!百姓何食之矣!雖以陰陽為炭,萬物為銅,役不食之人,使不衣之士,猶尚不給。況資於天生地養,風動雨潤,而後得之乎!臣聞國無九年之儲,國非其國。伏計倉廩,度府庫,百僚供給,百事用度,臣恐卒歲不充,況九年之積乎!一旦風塵再擾,霜雹薦臻,沙門不可擐干戈,寺塔不足攘饑饉,臣竊痛之矣!



 疏奏不納。歲餘,安樂公主被誅。



 睿
 宗即位,又為金仙、玉真公主廣營二觀。先是,中宗時斜封受官人一切停任,凡數百千人,又有敕放令卻上。替否時為左補闕,又上疏陳時政曰:



 臣嘗以為古之用度不時,爵賞不當,破家亡國者,口說不如身逢,耳聞不如眼見,臣請以有唐已來理國之得失,陛下之所眼見者以言之。惟陛下審之聽之,擇善而從之,則萬歲之業,自可致矣,何憂乎黎庶之不康,福祚之不永!



 伏以太宗文武聖皇帝,陛下之祖,撥亂反正,開階立極,得至理之體,
 設簡要之方。省其官,清其吏,舉天下職司無一虛授,用天下財帛無一枉費。賞必俟功,官必得俊,所為無不成,所征無不伏。不多造寺觀而福德自至,不多度僧尼而殃咎自滅。道合乎天地,德通乎神明。故天地憐之,神明祐之,使陰陽不愆,風雨合度。四人樂其業,五穀遂其成,腐粟爛帛,填街委巷。千里萬里,貢賦於郊;九夷百蠻,歸款於闕。自有帝皇已來,未有若斯之神聖者也,故得享國久長,多歷年所,陛下何不取而則之?



 中宗孝和皇帝,
 陛下之兄,居先人之業,忽先人之化,不取賢良之言,而恣子女之意。官爵非擇,虛食祿者數千人;封建無功,妄食土者百餘戶。造寺不止,枉費財者數百億;度人不休,免租庸者數十萬。是使國家所出加數倍,所入減數倍。倉不停卒歲之儲,庫不貯一時之帛。所惡者逐,逐多忠良;所愛者賞,賞多讒慝。朋佞喋喋,交相傾動。容身不為於朝廷,保位皆由於黨附。奪百姓之食,以養殘兇;剝萬人之衣,以塗土木。於是人怨神怒,親忿眾離,水旱不調,
 疾疫屢起。遠近殊論,公私罄然。五六年間,再三禍變,享國不永,受終於兇婦人。寺舍不能保其身,僧尼不能護妻子,取譏萬代,見笑四夷。此陛下之所眼見也,何不除而改之。



 依太宗之理國,則百官以理,百姓無憂,故太山之安立可致矣;依中宗之理國,則萬人以怨,百事不寧,故累卵之危立可致矣。頃自夏已來,霪雨不解,穀荒於壟,麥爛於場。入秋已來,亢旱成災,苗而不實,霜損蟲暴,草葉枯黃。下人咨嗟,未知賙賑;而營寺造觀,日繼於時,
 檢校試官,充臺溢署。伏惟陛下愛兩女,為造兩觀,燒瓦運木,載土填坑,道路流言,皆雲計用錢百餘萬貫。惟陛下,聖人也,無所不知;陛下,明君也,無所不見。既知且見,知倉有幾年之儲,庫有幾年之帛?知百姓之間可存活乎?三邊之上可轉輸乎?當今發一卒以御邊陲,遣一兵以衛社稷,多無衣食,皆帶饑寒。賞賜之間,迥無所出,軍旅驟敗,莫不由斯。而乃以百萬貫錢造無用之觀,以受六合之怨乎!以違萬人之心乎!伏惟陛下續阿韋之醜
 跡,而不改阿韋之亂政。忍棄太宗之理本,不忍棄中宗之亂階;忍棄太宗久長之謀,不忍棄中宗短促之計。陛下又何以繼祖宗、觀萬國。



 昔陛下為皇太子,在阿韋之時,危亡是懼,常切齒於群兇。今貴為天子,富有海內,而不改群兇之事,臣恐復有切齒於陛下者也,陛下又何以非群兇而誅之?臣往見明敕,自今已後,一依貞觀故事。且貞觀之時,豈有今日之造寺營觀,加僧尼道士,益無用之官,行不急之務,而亂政者也!臣以為棄其言而
 不行其信,慕其善而不遷其惡,陛下又何以刑於四海?往者,和帝之憐悖逆也,為奸人之所誤,宗晉卿勸為第宅,趙履溫勸為園亭,損數百家之居,侵數百家之地。工徒斫而未息,義兵紛以交馳,卒使亭不得游,宅不得坐。信邪佞之說,成骨肉之刑,此陛下之所眼見也。今茲造觀,臣必知非陛下、公主之本意,得無趙履溫之徒將勸為之,冀誤其骨肉,不可不明察也。



 臣聞出家修道者,不預人事,專清其身心,以虛泊為高,以無為為妙,依兩卷《
 老子》,視一軀天尊,無欲無營,不損不害。何必璇臺玉榭,寶像珍龕,使人困窮,然後為道哉!且舊觀足可歸依,無造無營,以取窮竭。若此行之三年,國不富,人不安,朝廷不清,陛下不樂,則臣請殺身於朝,以令天下言事者。伏惟陛下行非常之惠,權停兩觀,以俟豐年。以兩觀之財,為公主施貧窮,填府庫,則公主福德無窮矣。不然,臣恐下人怨望,不減於前朝之時。前朝之時,賢愚知敗,人雖有口而不敢言,言未發聲,禍將及矣。韋月將受誅於丹
 徼,燕欽融見殺於紫庭,此人皆不惜其身而納忠於主,身既死矣,朝亦危矣。故先朝誅之,陛下賞之,是陛下知直言之士有裨於國。臣今直言,亦先代之直,惟陛下察之。



 疏奏,睿宗嘉其公直。稍遷為右臺殿中侍御史。開元中,累轉潁王府長史。天寶初卒,年八十餘。



 史臣曰:夫好聞其善,惡聞其過,君人者之常情也;寧諂媚以取容,不逆耳以招禍,臣人者之常情也。能反此者,不亦善乎!李、薛等六君,吐忠讜之言,補朝廷之失,有犯
 無隱,不愧古人,有唐之良臣也。



 贊曰:臣之事君,有邪有正。君之使臣,從諫則聖。李、薛輸忠,救人之命。韋、韓讜言,醫國之病。辛、王章疏,犯顏竦聽。張子法言,實裨時政。



\end{pinyinscope}