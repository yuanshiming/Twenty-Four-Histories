\article{卷一百七十九}

\begin{pinyinscope}

 ○憲宗
 二十
 子:穆宗皇帝、宣宗皇帝、惠昭太子寧、澧王惲、
 深王悰、洋王忻、絳王悟、建王恪、鄜王憬、瓊王悅、沔王恂、婺王懌、茂王愔、淄王協、衡王憺、澶王心充、棣王惴、彭王惕、信王憻、榮王心責。



 惠昭太子寧,憲宗長子也。母曰紀美人。貞元二十一年四月,封平原郡王。元和元年八月,進封鄧王。四年閏三月,立為皇太子,改名宙,尋復今名。其年有司將行冊禮,以孟夏、孟秋再卜日,臨事皆以雨罷,至十月方行冊禮。元和六年十二月薨,年十九,廢朝十三日。時敕國子司
 業裴茝攝太常博士,西內勾當。茝通習古今禮儀,嘗為太常博士。及官至郎中,每兼其職,至改司業,方罷兼領。國典無皇太子薨禮,故又命茝領之。廢朝十三日,蓋用期服以日易月之制也。謚曰惠昭。



 澧王惲,憲宗第二子也,本名寬。貞元二十一年,封同安郡王。元和元年,進封澧王。七年,改今名。時吐突承璀恩寵特異,惠昭太子薨,議立儲副,承璀獨排群議,屬澧王,欲以威權自樹,賴憲宗明斷不惑。上將冊拜太子,詔翰
 林學士崔群代澧王作讓表一章。群奏曰:「凡事己合當之而不為,則有退讓焉。」上深納之。及憲宗晏駕,承璀死,王亦薨於其夕。以元和十五年四月丁丑發喪,廢朝三日。長子漢,東陽郡王。次子源,安陸郡王。第三子演,臨安郡王。



 深王悰,本名察,憲宗第四子也。貞元二十一年,封彭城郡王。元和元年,進封深王,改今名。長子潭,河內郡王。次子淑,吳興郡王。



 洋王忻,本名寰,憲宗第五子也。貞元二十一年,封為高密郡王。元和元年,進封洋王。七年,改今名。太和二年薨。長子沛,太和八年,封潁川郡王。



 絳王悟,本名寮,憲宗第六子也。貞元二十一年,封文安郡王。元和元年,進封絳王。七年,改今名。寶歷二年冬遇害。長子洙,太和八年,封新安郡王。第二子滂,封高平郡王。



 建王恪,本名審,憲宗第十子也。元和元年八月,淄青節
 度李師古卒,其弟師道擅領軍務,以邀符節。朝廷方興討罰之師,不欲分兵兩地,乃封審為建王。間一日,授開府儀同三司、鄆州大都督,充平盧軍淄青等州節度營田觀察處置、陸運海運、押新羅渤海兩蕃等使,而以師道為節度留後。不出閣。七年,改今名。長慶元年薨。



 鄜王憬,長慶元年封,開成四年七月薨。長子溥,平陽郡王。



 瓊王悅,長慶元年封。第二子津,河間郡王。



 沔王恂,長慶元年封。長子瀛,晉陵郡王。



 婺王懌,長慶元年封。長子清,新平郡王。



 茂王愔,長慶元年封。長子潓,武功郡王。



 淄王協,憲宗第十四子也。長慶元年封,開成元年薨。長子浣,太和八年八月封許昌郡王。第三子渙,馮翊郡王。



 衡王憺,長慶元年封。長子涉,晉平郡王。



 澶王心充,長慶元年封。長子濘,雁門郡王。



 棣王惴,大中六年封,咸通三年薨。



 彭王惕,大中三年封。



 信王憻,大中十四年封,咸通八年薨。



 榮王心責,咸通三年封,廣明元年八月十九日,授開府儀同三司,守司空,其年十月九日薨。其子令平嗣王。



 穆宗五子:敬宗皇帝、文宗皇帝、武宗皇帝、懷懿太子湊、安王溶。



 懷懿太子湊,穆宗第六子。少寬和溫雅,齊莊有度。長慶初,封漳王。文宗以王守澄恃權,深怒閹官,欲盡誅之,密
 令宰相宋申錫與外臣謀畫其計。守澄門人鄭注伺知其事,欲先事誅申錫。以漳王賢而有望,乃令神策虞候豆盧著告變言:「十六宅宮市典宴敬則、硃訓與申錫親吏王師文同謀不軌,硃訓與王師文言聖上多病,太子年小,若立兄弟,次是漳王,要先結托,乃於師文處得銀五鋌、絹八百匹;又晏敬則於十六宅將出漳王吳綾汗衫一領、熟線綾一匹,以答申錫。」其事皆鄭注憑虛結構,而擒硃訓等於黃門獄,鍛煉偽成其款。居三四日,朝臣
 方悟其誣構。諫官崔玄亮等閣中極諫,叩頭出血,請出申錫獄付外勘鞫。鄭注輩恐其偽跡敗露,乃請行貶黜。制曰:「王者教先入愛,義不遺親。豈於同氣之中,可致異詞之間。如或慎修不至,詿誤有聞,構為厲階,犯我邦紀,未加殛竄,尚屈彞章。漳王湊手足之親,盤石是固,居崇寵秩,列在戚籓。頃多克順之心,亦有尚賢之志。而滿盈生患,敗覆是圖,奸兇會同,謀議聯及。污我皇化,彰於外朝,初駭予衷,再驚群聽。尚以未具獄詞,猶資審慎,建侯
 之命,姑務從寬。可降封巢縣公。」制下,上令中使齎巢縣官告,就十宅賜湊。言國法須此,爾宜寬勉。八年薨,贈封齊王。



 鄭注伏誅。帝思湊被陷而心傷之,開成三年正月制曰:



 褒善飾終,王者常典。況我友于之愛,手足之親,永言痛悼之懷,用錫元良之命。故齊王湊孕靈天宇,擢秀本枝,孝敬知於孩提,惠和洽於親愛。將固磐石,遂分茅社。學探蟻術之精,智有象舟之妙。好書樂善,造次不失其清規;置醴尊師,風雨不忘其至敬。方期臺耇,以保怡
 怡,天胡不仁,殲我同氣。念周宣好愛之分,長慟莫追;覽魏文榮樂之言,軫懷無已。由是稽諸前典,式殿追榮,特峻彞章,表恩泉壞。雖禮命之儀則爾,而天倫之恨何攄?遐想幽魂,宜膺寵數。可贈懷懿太子,有司擇日冊命。



 安王溶,穆宗第八子。母楊賢妃,長慶元年封。太和八年,授開府儀同三司、檢校吏部尚書。開成初,敕安王、潁王,並以百官例,逐月給料錢。武宗即位,李德裕秉政,或告文宗崩時,楊嗣復以與賢妃宗家,欲立安王為嗣,故王
 受禍,嗣復貶官。



 敬宗五子:悼懷太子普、梁王休復、襄王執中、紀王言揚、陳王成美。



 悼懷太子普,敬宗長子也。母曰郭妃。實歷元年,封晉王。太和二年薨,年五歲。上撫念之甚厚,冊贈悼懷太子。



 梁王休復。開成二年八月詔曰:「王者胙土畫疆,封建子弟,所以承衛帝室,蕃茂本枝,祖宗成式,朕曷敢廢?況天付正性,夙奉至訓,尊賢好善,體仁由禮,是可舉建侯之
 命,膺分社之榮。親親賢賢,於是乎在。敬宗皇帝第二子休復、第三子執中、第四子言揚、第六子成美,皆氣蘊中和,行推敬慎,游泳《墳》、《索》,佩服師言。宜開土宇之封,用申睦族之典。休復可封梁王,執中可封襄王,言揚可封紀王,成美可封陳王。宜令有司擇日備禮冊命。」



 襄王執中,與梁王同時受封。第三男採,樂平郡王。



 紀王,與襄王同時受封。



 陳王成美,與紀王言揚同時受封。開成四年十月,詔曰:「
 古先哲王之有天下也,何嘗不正國本而承天序,建儲貳而主重離?朕以寡昧,祗荷丕圖。虔恭寅畏,思固鴻業,慎擇全懿,曠於旬時。而卿士獻謀,龜筮告吉,以為少陽虛位,願舉盛儀。列聖垂休,俾合予志,選賢而立,式表無私。敬宗皇帝第六男陳王成美,天假忠孝,日新道德;溫文合雅,謙敬保和。裕端明之體度,尚《詩》、《書》之辭訓,言皆中禮,行不違仁。是可以訓考舊章,欽若成命,授之匕鬯,以奉粢盛。宜回硃邸之榮,俾踐青宮之重,可立為皇太
 子。宜令所司擇日備禮冊命。」自莊恪太子薨,將相大臣洎職言者,拜章面陳凡累月,上遂命立陳王。未行冊禮,復降仍舊,其年殂於籓邸。第十九男儼,宣城郡王。



 文宗二子:莊恪太子永、蔣王宗儉。



 莊恪太子永,文宗長子也。母曰王德妃。太和四年正月,封魯王。六年,上以王年幼,思得賢傅輔導之。時王傅和元亮,因待制召問。元亮出於卒吏,不知書,一不能對。後宰相延英奏事,上從容曰:「魯王質性可教,宜擇賢士大
 夫為官屬,不可復用和元亮之輩。」因以戶部侍郎庾敬休守本官,兼魯王傅;太常卿鄭肅守本官,兼王府長史;戶部郎中李踐方守本官,兼王府司馬。其年十月,降詔冊為皇太子。



 上自即位,承敬宗盤游荒怠之後,恭儉惕慎,以安天下。以晉王謹願,且欲建為儲貳。未幾,晉王薨,上哀悼甚,不復言東宮事久之。今有是命,中外慶悅。後以王起、陳夷行為侍讀。



 開成三年,上以皇太子宴游敗度,不可教導,將議廢黜。特開延英,召宰臣及兩省御史
 臺五品已上、南班四品已上官對。宰臣及眾官以為儲後年小,可俟改過,國本至重,願寬宥。御史中丞狄兼謨上前雪涕以諫,詞理懇切。翌日,翰林學士六人洎神策六軍軍使十六人又進表陳論,上意稍解。



 其日一更,太子歸少陽院,以中人張克己、柏常心充少陽院使;如京使王少華、判官袁載和及品官、白身、內園小兒、官人等數十人,連坐至死及剝色、流竄。尋詔侍讀竇宗直、周敬慎依前隔日入少陽院。



 其年薨,敕兵部尚書王起撰哀
 冊文曰:



 維大唐開成三年,歲次戊午,十月乙酉朔,十六日庚子,皇太子薨於少陽院。十七日辛丑,遷座於大吉殿。十一月乙卯朔,二十四日戊寅,命冊使太子太師兼右僕射、門下侍郎、國子祭酒、平章事鄭覃,副使中書侍郎、平章事楊嗣復,持節冊謚曰莊恪。十二月乙酉朔,十二日丙申,葬於驪山之北原莊恪陵,禮也。玉琯歲窮,金壺漏盡,祖奠告徹,哀笳將引。庭滅燎而月寒,路搖旍而風緊。皇帝念主鬯之缺位,悼佩觿之夭年。銅樓已閉,銀
 牒徒懸。方追思於對日,遽冥寞而賓天。典冊具舉,文物咸備。爰詔侍臣,顯揚上嗣,其詞曰:



 皇矣帝緒,肇基綿古;種德尊道,宗文祖武。上聖開成,天下和平;儲祉發祥,是生元良。覃訏之初,岐嶷用彰;蘊才游藝,玉裕金相。既免孩提,是加封殖;俾維城於東魯,錫介珪於上國。辭榮硃邸,正位青宮;尊師重傅,養德含聰。畏馳道而不絕,問寢門而益恭。招賢警戒,齒胄謙沖;冀日躋於三善,奉天慈於九重。漢莊好學,既顯於外;魏丕能文,方循於內。美不
 二於顏過,嘉得三於鯉退。焜耀甲觀,鏗鏘瑜珮。方積善於為山,何反真而游岱。嗚呼哀哉!



 憂兢損壽,沉痾始遘;群望並走,百靈宜祐。吳客之問徒為,越人之方靡救。占前星之掩曜,知東朝之降咎;天垂象而則然,人由己而何有?嗚呼哀哉!稅駕乘華兮即宮夜臺,鳳笙長絕兮蜃輅徐來。啟青宮而右出,歷玄灞而左回;度凋林兮魂斷,入曠野兮心摧。水助挽而幽咽,雲帶翣而徘徊;悲佳城之已掩,見新廟之方開。嗚呼哀哉!授經兮曷期,執紼兮
 增欷;九原作兮何嗟及,七日還兮安可希。有少海之波逝,無西園之蓋飛;商山之羽翼已散,望苑之賓客咸歸。瑟彼玉簡,閟於泉扉;用傳信於文字,願不昧於音徽。嗚呼哀哉!



 初,上以太子稍長,不循法度,暱近小人,欲加廢黜。迫於公卿之請,乃止。太子終不悛改,至是暴薨。時傳云:太子德妃之出也,晚年寵衰。賢妃楊氏,恩渥方深。懼太子他日不利於己,故日加誣譖,太子終不能自辨明也。太子既薨,上意追悔。四年,因會寧殿宴。小兒緣橦,有
 一夫在下,憂其墮地,有若狂者。止問之,乃其父也。上因感泣,謂左右曰:「朕富有天下,不能全一子。」遂召樂官劉楚材、宮人張十十等責之,曰:「陷吾太子,皆爾曹也。今已有太子,更欲踵前耶?」立命殺之。



 蔣王宗儉,文宗第二子,開成二年封。



 武宗五子:杞王峻,開成五年封;益王峴、兗王岐、德王嶧、昌王嵯,皆會昌二年封。



 宣宗十一子:懿宗皇帝,餘並封王。



 靖懷太子漢,會昌六年封雍王,大中六年薨,冊贈靖懷太子。



 雅王涇,宣宗第二子。大中元年封。



 衛王灌,大中十一年封,十四年薨。



 夔王滋,宣宗第三子也。會昌六年封,咸通四年薨。



 慶王沂,第四子也。會昌六年封,大中四十年薨。



 濮王澤,第五子也。大中二年封。



 鄂王潤,第六子也。大中五年封,乾符三年薨。



 懷王洽,第七子也。大中八年封。



 昭王汭,第八子也。大中八年封,乾符三年薨。



 康王汶,大中八年封。



 廣王澭,大中十一年封。



 懿宗八子:僖宗皇帝、昭宗皇帝,餘並封王。



 魏王佾,咸通三年封。



 涼王健,咸通三年封,乾符六年薨。



 蜀王佶,咸通三年封。



 咸王侃,咸通六年封郢王,十年改封今王。



 吉王保,咸通十三年封,文德元年八月九日授開府儀同三司、檢校太傅,仍加食邑三百戶。



 睦王倚,咸通十三年封。



 僖宗二子:



 建王震,中和元年九月十六日封。



 益王升,光啟三年十一月十四日封。



 昭宗十子:哀帝,餘並封王。



 德王裕,昭宗長子也。大順二年六月二十八日封,乾寧
 四年二月十四日冊為皇太子。時駕在華州,韓建畏諸王主兵,誘防城卒張行思、花重武相次告通王以下欲殺建。建他日又造訛言云:諸王欲劫遷車駕,別幸籓鎮。諸王懼,詣建自陳。建乃延入臥內,密遣人奏云:「今日睦王、濟王、韶王、通王、彭王、韓王、儀王、陳王等八人到臣理所,不測事由。臣竊量事體,不合與諸王相見,兼恐久在臣所,於事非宜。忽然及門,意不可測。」又上疏抗請歸十六宅。如是者數四,帝不允。建懼為諸王所圖,乃以精甲
 數千圍行宮,請誅定州護駕軍都將李筠。帝懼甚,詔斬筠於大雲橋。其三都軍士,尋放還本道。殿後都,亦與三都元繞行宮扈蹕。至昌,並急詔散之。罷諸王兵柄。建慮上不悅,乃上表請立德王為皇太子。其年八月,嗣延王戒丕自太原還,詔與通王已下八王並賜死於石堤穀。



 光化末,樞密使劉季述、王仲先等幽昭宗於東門,冊裕為帝。及天復初誅季述、仲先,與寺人藏於右軍。群臣請殺之,昭宗曰:「太子沖幼,為賊輩所立。」依舊令歸少陽院。
 及硃全忠自鳳翔迎駕還京,以德王眉目疏秀,春秋漸盛,常惡之。謂崔胤曰:「德王曾竊居寶位,天下知之。大義滅親,何得久留?是教後代以不孝也。請公密啟。」胤然之,昭宗不納。他日言於全忠,全忠曰:「此國家大事,臣安敢竊議?乃崔胤賣臣也。」尋以哀帝為天下兵馬元帥。



 後昭宗至洛下,一日幸福先寺,謂樞密使蔣玄暉曰:「德王,朕之愛子,全忠何故須令廢之,又欲殺之?」言訖淚下,因嚙其中指血流。玄暉具報全忠,由是轉恚。昭宗遇弒之日,
 蔣玄暉於西內置社筵;酒酣,德王已下六王皆為玄暉所殺,投尸九曲池。



 棣王祤。乾寧元年十月八日封。



 虔王禊、沂王禋、遂王禕,並與棣王同時封冊。



 景王秘,乾寧四年十月二十二日封。



 祁王祺與景王同時封冊。



 雅王禛、瓊王祥,並光化元年十一月九日封。



 嗣襄王襜,性柔善,無他能。光啟二年春,車駕在寶雞,西軍
 逼請幸岐隴;帝以數十騎自大散關幸興元。時襜有疾,不能從,因為硃玫所挾至鳳翔。有臺省官從行未及者僅百人。四月,玫乃與宰相蕭遘、裴澈率群僚冊襜為監國。襜以鄭昌圖判度支,而鹽鐵、戶部各置副使,三司之事,一以委焉,目曰「廢置相公」。五月,襜遣偽戶部侍郎柳陟等十餘人,分諭關東、河北諸道,納偽命者甚眾。十月,硃玫率蕭遘等冊襜為帝,改元曰永貞,遙尊僖宗為太上元皇聖帝。



 初,河中王重榮表率東諸侯進貢,唯蔡賊
 與太原不順。秦宗權自僭號,太原不協於硃玫故也。及王行瑜殺硃玫,襜奔至渭上,王重榮使人迎之,襜與偽百官泣別,謂曰:「朕見重榮,當令與卿等各備所服以接卿。」殺硃玫之翌日,襜為鄜州亂軍所殺,行瑜遂函首送行在。襜四月監國,至十二月死,凡在偽位九月矣。



 硃玫者,邠州人也。少從邊,以功歷郡守。乾符末,領邠寧節制。中和中,收復京師,與太原李克用、東方達同制加使相。光啟元年冬,受詔招討河中,軍敗。以軍容使田令
 孜失策,時諸軍皆怒,乃徇人情,表請誅令孜。令孜與楊復恭挾帝西幸,玫又失策。乃虜嗣襄王襜,與蕭遘等同立為帝,大行封拜,以啖諸侯;而天下之人,歸者十五六焉。與李昌符始謀冊立,及後,玫自稱大丞相,吐握在己。昌符怒之。乃以表送款行在,復密結樞密使楊復恭,人心乃離。



 時行在出令,有能斬硃玫首者,則授以邠帥。賊將王行瑜以大唐峰不利,退保鳳州。終慮得罪,與腹心密謀,徑入京師。時玫有第在和善里,行瑜率兵仗入見。
 玫猶責以擅還,行瑜曰:「我要代爾領邠州節制,何復多言?」遂斬之。



 王行瑜者,邠州人也。少隸本軍,事硃玫為偏將,平巢寇有功。光啟二年,玫冊嗣襄王襜為偽帝,授天平軍節度使。領兵守大散關,攻大唐峰,為李鋌所敗,乃送款行在。以部下反攻硃玫於闕下,斬之,因授邠州節度使。後平楊守亮於山南,以功累加至中書令。景福中,逼朝廷加尚書令。宰臣韋昭度密奏不可。會韓建、李茂貞稱兵入
 覲,欲行廢立。不果,乃請殺昭度與李磎。是歲,又遣弟行約攻河中;河中引太原軍至,由是大敗。行約、行實劫駕不獲,遂歸邠州。行瑜率兵屯梨園,王師圍急。行實、行約先敗,次保龍泉。行瑜又遁至邠州,不能守。乾寧二年十一月,挈族至慶州,為部下所殺。



 史臣曰:自天寶已降,內官握禁旋,中闈纂繼,皆出其心。故手才攬於萬機,目己睨於六宅;防閑禁錮,不近人情。文守好古睦親,至敦友悌。悔前非於齊湊,褒以儲闈;付
 後事於陳王,歸其胄席。或降輿硃邸,對食瓊筵,怡怡申肺腑之情,穆穆盡棣華之義;近朝盛美,可洽風謠。昭肅惑讒,毒流安邸。雖覽大臣之議,欲使磐維;竟無出閣之儀,終身幽枉。《穀風》之怨,可為傷心。大中、咸通已來,寶圖世及。犬牙麟趾,雖不迨於姬周;平什布謠,未甚悲於宗籍。於姬不足,比魏有餘。



 贊曰:周封子弟,運祚綿長。管、蔡剿絕,魯、魏克昌。誅叛賞順,王者大綱。法不私親,棣萼其芳。



\end{pinyinscope}