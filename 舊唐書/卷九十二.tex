\article{卷九十二}

\begin{pinyinscope}

 ○韋思謙子承慶嗣立陸元方子
 象先蘇瑰子頲



 韋思謙,鄭州陽武人也。本名仁約,字思謙,以音類則天父諱,故稱字焉。其先自京兆南徙,家於襄陽。舉進士,累補應城令,歲餘調選。思謙在官,坐公事微殿,舊制多未
 敘進。吏部尚書高季輔曰:「自居選部,今始得此一人,豈以小疵而棄大德。」擢授監察御史,由是知名。嘗謂人曰:「御史出都,若不動搖山嶽,震懾州縣,誠曠職耳。」時中書令諸遂良賤市中書譯語人地,思謙奏劾其事,遂良左授同州刺史。及遂良復用,思謙不得進,出為清水令。謂人曰:「吾狂鄙之性,假以雄權,觸機便發,固宜為身災也。大丈夫當正色之地,必明目張膽以報國恩,終不能為碌碌之臣保妻子耳。左肅機皇甫公義檢校沛王府長
 史,引思謙為同府倉曹,謂思謙曰:「公豈池中之物,屈公為數旬之客,以望此府耳。」累遷右司郎中。



 永淳初,歷尚書左丞、御史大夫。時武候將軍田仁會與侍御史張仁禕不協而誣奏之。高宗臨軒問仁禕,仁禕惶懼,應對失次。思謙歷階而進曰:「臣與仁禕連曹,頗知事由。仁禕懦而不能自理。若仁會眩惑聖聰,致仁禕非常之罪,即臣亦事君不盡矣。請專對其狀。」辭辯縱橫,音旨明暢,高宗深納之。思謙在憲司,每見王公,未嘗行拜禮。或勸之,答
 曰:「雕鶚鷹鸇,豈眾禽之偶,奈何設拜以狎之?且耳目之官,固當獨立也。」初拜左丞,奏曰:「陛下為官擇人,非其人則闕。今不惜美錦,令臣制之,此陛下知臣之深,亦微臣盡命之秋。」振舉綱目,朝廷肅然。



 則天臨朝,轉宗正卿,會官名改易,改為司屬卿。光宅元年,分置左、右肅政臺,復以思謙為右肅政大夫。大夫舊與御史抗禮,思謙獨坐受其拜。或以為辭,思謙曰:「國家班列,自有差等,奈何以姑息為事耶?」垂拱初,賜爵博昌縣男,遷鳳閣鸞臺
 三品。二年,代蘇良嗣為納言。三年,上表告老請致仕。許之,仍加太中大夫。永昌元年九月,卒於家,贈幽州都督。二子:承慶、嗣立。



 承慶,字延休。少恭謹,事繼母以孝聞。弱冠舉進士,補雍王府參軍。府中文翰,皆出於承慶,辭藻之美,擅於一時。累遷太子司議郎。儀鳳四年五月,詔皇太子賢監國。時太子頗近聲色,與戶奴等款狎,承慶上書諫曰:



 臣聞太子者,君之貳,國之本也。所以承宗廟之重,系億兆之心,萬國以貞,四海屬望。殿下以仁孝之德,明睿
 之姿,岳峙泉渟,金貞玉裕。天皇升殿下以儲副,寄殿下以監撫,欲使照無不及,恩無不覃,百僚仰重曜之暉,萬姓聞瀳雷之響。



 夫君無民,無以保其位;人非食,無以全其生。故孔子曰:「百姓足,君孰與不足;百姓不足,君孰與足?」自頃年已來,頻有水旱,菽粟不能豐稔,黎庶自致煎窮。今夏亢陽,米價騰踴,貧窶之室,無以自資,朝夕遑遑,唯憂餒饉。下人之瘼,實可哀矜,稼穡艱難,所宜詳悉。天皇所以垂衣北極,殿下所以守器東宮,為天下之所尊,
 得天下之所利者,豈唯上玄之幽贊,亦百姓之力也。百姓危,則社稷不得獨安。百姓亂,則帝王不能獨理。故古之明君,飽而知人饑,溫而知人寒,每以天下為憂,不以四海為樂。今關、隴之外,兇寇憑凌,西土編甿,凋喪將盡,干戈日用,烽柝薦興,千里有勞於饋糧,三農不遑於稼穡。殿下為臣為子,乃國乃家。為臣在於竭忠,為子期於盡孝。在家不可以自逸,在國不可以自康。一物有虧,聖上每留神念;三邊或梗,殿下豈不兢懷。況當養德之秋,
 非是任情之日。



 伏承北門之內,造作不常,玩好所營,或有煩費。倡優雜伎,不息於前,鼓吹繁聲,亟聞於外,既喧聽覽,且黷宮闈。兼之僕隸小人,緣此得親左右,亦既奉承顏色,能不恃托恩光。作福作威,莫不由此,不加防慎,必有愆非。儻使微累德音,於後悔之何及?《書》云:「不作無益害有益。」此皆無益之事,固不可耽而悅之。



 臣又聞「高而不危,所以長守貴;滿而不溢,所以長守富。」是知高危不可不慎,滿溢不可不持。《易》曰:「君子終日乾乾,夕惕若
 厲,無咎。」敬慎之謂也。在於凡庶,參守而行之,猶可以高振聲華,坐致榮祿。況殿下有少陽之位,有天挺之姿,片善而天下必聞,小能而天下咸服,豈可不為盡善盡美之道,以取可大可久之名哉!伏願博覽經書以廣其德,屏退聲色以抑其情。靜默無為,恬虛寡欲,非禮勿動,非法不言。居處服玩,必循節儉;畋獵游娛,不為縱逞。正人端士,必引而親之;便僻側媚,必斥而遠之。使惠聲溢於遠近,仁風翔於內外,則可以克享終吉,長保利貞,為上
 嗣之稱首,奉聖人之鴻業者矣。



 又嘗為《諭善箴》以獻太子。太子善之,賜物甚厚。承慶又以人之用心,多擾濁浮躁,罕詣沖和之境,乃著《靈臺賦》以廣其志,辭多不載。



 調露初,東宮廢,出為烏程令,風化大行。長壽中,累遷鳳閣舍人,兼掌天官選事。承慶屬文迅捷,雖軍國大事,下筆輒成,未嘗起草。尋坐忤大臣旨,出為沂州刺史。未幾,詔復舊職,依前掌天官選事。久之,以病免,改授太子諭德。後歷豫、虢等州刺史,頗著聲績,制書褒美。長安初,入為
 司僕少卿,轉天官侍郎,兼修國史。承慶自天授以來,三掌天官選事,銓授平允,海內稱之。尋拜鳳閣侍郎、同鳳閣鸞臺平章事,仍依舊兼修國史。神龍初,坐附推張易之弟昌宗失實,配流嶺表。時易之等既伏誅,承慶去巾解帶而待罪。時欲草赦書,眾議以為無如承慶者,乃召承慶為之。承慶神色不撓,援筆而成,辭甚典美,當時咸嘆服之。歲餘,起授辰州刺史,未之任,入為秘書員外少監,兼修國史。尋以修《則天實錄》之功,賜爵扶陽縣子,賚
 物五百段。又制撰《則天皇后紀聖文》,中宗稱善,特加銀青光祿大夫。俄授黃門侍郎,仍依舊兼修國史,未拜而卒。中宗傷悼久之,乃召其弟相州刺史嗣立令赴葬事,仍拜黃門侍郎,令繼兄位,其見用如此。贈秘書監,謚曰溫。子長裕,膳部員外郎。



 嗣立,承慶異母弟也。母王氏,遇承慶甚嚴,每有杖罰,嗣立必解衣請代,母不聽,輒私自杖,母察知之,漸加恩貸,議者比晉從王祥、王覽。少舉進士,累補雙流令,政有殊績,為蜀中之最。三遷萊蕪令。會
 承慶自鳳閣舍人以疾去職,則天召嗣立謂曰:「卿父往日嘗謂朕曰『臣有兩男忠孝,堪事陛下。』自卿兄弟效職,如卿父言。今授卿鳳閣舍人,令卿兄弟自相替代。」即日遷鳳閣舍人。時學校頹廢,刑法濫酷,嗣立上疏諫曰:



 臣聞古先哲王立學官,掌教國子以六德、六行、六藝,三教備而人道畢矣。《禮記》曰:「化人成俗,必由學乎。」學之於人,其用蓋博。故立太學以教於國,設庠序以化於邑,王之諸子、卿大夫士之子及國之俊選皆造焉。八歲入小學,
 十五入太學。春秋教以《禮》、《樂》,冬夏教以《詩》、《書》。是以教洽而化流,行成而不悖。自天子以至於庶人,未有不須學而成者也。



 國家自永淳已來,二十餘載,國學廢散,胄子衰缺,時輕儒學之官,莫存章句之選。貴門後進,競以僥幸升班;寒族常流,復因凌替弛業。考試之際,秀茂罕登,驅之臨人,何以從政?又垂拱之後,文明在辰,盛典鴻休,日書月至,因藉際會,入仕尤多。加以讒邪兇黨來俊臣之屬,妄執威權,恣行枉陷,正直之伍,死亡為憂,道路以
 目,人無固志,罕有執不撓之懷,殉至公之節,偷安茍免,聊以卒歲。遂使綱領不振,請托公行,選舉之曹,彌長渝濫。隨班少經術之士,攝職多庸瑣之才,徒以猛暴相誇,罕能清惠自勖。使海內黔首,騷然不安,州縣官僚,貪鄙未息,而望事必循理,俗致康寧,不可得也。陛下誠能下明制,發德音,廣開庠序,大敦學校,三館生徒,即令追集。王公已下子弟,不容別求仕進,皆入國學,服膺訓典。崇飾館廟,尊尚儒師,盛陳奠菜之儀,宏敷講說之會,使士
 庶觀聽,有所發揚,弘獎道德,於是乎在。則四海之內,靡然向風,延頸舉足,咸知所向。然後審持衡鏡,妙擇良能,以之臨人,寄之調俗。則官無侵暴之政,人有安樂之心,居人則相與樂業,百姓則皆戀桑梓,豈復憂其逃散而貧窶哉!今天下戶口,亡逃過半,租調既減,國用不足。理人之急,尤切於茲。故知務學之源,豈唯潤身進德而已?將以誨人利國,可不務之哉!



 臣聞堯、舜之日,畫其衣冠;文、景之時,幾致刑措。歷茲千載,以為美談。臣伏惟陛下
 睿哲欽明,窮神知化,自軒、昊已降,莫之與京。獨有往之論法,或未盡善,皆由主司奸兇,惑亂視聽。尋而陛下聖察,具詳之矣,然竟未能顯其本源,明其前事,令天下萬姓識陛下本心,尚使四海多銜冤之人,九泉有抱痛之鬼。臣誠愚暗,不識大綱,請為陛下始末而言其事。



 揚、豫之後,刑獄漸興,用法之伍,務於窮竟,連坐相牽,數年不絕。遂使巨奸大猾伺隙乘間,內苞豺狼之心,外示鷹鸇之跡,陰圖潛結,共相影會,構似是之言,成不赦之罪。皆
 深為巧詆,恣行楚毒,人不勝痛,便乞自誣,公卿士庶,連頸受戮。道路籍籍,雖知非辜,而鍛練已成,辯占皆合。縱皋陶為理,於公定刑,則謂污宮毀柩,猶未塞責。雖陛下仁慈哀念,恤獄緩死,及覽辭狀,便已周密,皆謂勘鞫得情,是其實犯,雖欲寬舍,其如法何?於是小乃身誅,大則族滅,相緣共坐者,不可勝言。此豈宿構仇嫌,將申報復,皆圖茍成功效,自求官賞。當時稱傳,謂為羅織。其中陷刑得罪者,雖有敏識通材,被告言者便遭枉抑,心徒痛
 其冤酷,口莫能以自明。或受誅夷,或遭竄殛,並甘心引分,赴之如歸。故知弄法徒文,傷人實甚。賴陛下特回聖察,昭然詳究。周興、丘勣之類,弘義、俊臣之徒,皆相次伏誅,事暴遐邇,而朝野慶泰,若再睹陽和。且如仁傑、元忠,俱罹枉陷,被勘鞫之際,亦皆已自誣。向非陛下至明,垂以省察,則菹醢之戮,已及其身,欲望輸忠聖代,安可復得!陛下擢而升之,各為良輔,國之棟幹,稱此二人。何乃前非而後是哉?誠由枉陷與甄明爾。但恐往之得罪者
 多並此流,則向時之冤者其數甚眾。昔殺一孝婦,尚或降災。而濫者蓋多,寧無怨氣!怨氣上達則水旱所興,欲望歲登,不可得也。



 倘陛下弘天地之大德,施雷雨之深仁,歸罪於削刻之徒,降恩於枉濫之伍。自垂拱已來,大闢罪已下,常赦所不原者,罪無輕重,一皆原洗,被以昭蘇。伏法之輩,追還官爵,緣累之徒,普沾恩造。如此則天下知此所陷罪,元非陛下之意,咸是虐吏之辜。幽明歡欣,則感通和氣;和氣下降,則風雨以時;風雨以時,則五
 穀豐稔;歲既稔矣,人亦安矣。太平之美,亦何遠哉!伏願陛下深察。



 尋遷秋官侍郎,三過鳳閣侍郎、同鳳閣鸞臺平章事。長安中,則天嘗與宰臣議及州縣官吏,納言李嶠、夏官尚書唐休璟等奏曰:「臣等謬膺大任,不能使兵革止息,倉府殷盈,戶口尚有逋逃,官人未免貪濁,使陛下臨朝軫嘆,屢以為言,夙夜慚惶,不知啟處。伏思當今要務,莫過富國安人。富國安人之方,在擇刺史。竊見朝廷物議,莫不重內官,輕外職,每除授牧伯,皆再三披訴。
 比來所遣外任,多是貶累之人,風俗不澄,實由於此。今望於臺閣寺監,妙簡賢良,分典大州,共康庶績。臣等請輟近侍,率先具僚,務在憂國濟人,庶當有所補益。」則天曰:「卿等處鸞臺鳳閣,誰為此行?」嗣立率先對曰:「臣以庸愚,謬膺獎擢,內掌機密,非臣所堪。承乏外臺,庶當盡節,倘垂採錄,臣願此行。」於是嗣立帶本官檢校汴州刺史。



 無幾,嗣立兄承慶入知政事,嗣立轉成均祭酒,兼檢校魏州刺史。又徙洺州刺史。尋坐承慶左授饒州長史。歲
 餘,徵為太僕少卿,兼掌吏部選事。神龍二年,為相州刺史。及承慶卒,代為黃門侍郎,轉太府卿,加修文館學士。景龍三年,轉兵部尚書、同中書門下三品。時中宗崇飾寺觀,又濫食封邑者眾,國用虛竭。嗣立上疏諫曰:



 臣聞國無九年之儲,家無三年之蓄,家非其家,國非其國。故知立國立家,皆資於儲蓄矣。夫水旱之災,關之陰陽運數,非人智力所能及也。堯遭大水,湯遭大旱,則知仁聖之君所不能免,當此時不至於困弊者,積也。今陛下倉
 庫之內,比稍空竭,尋常用度,不支一年。倘有水旱,人須賑給,徵發時動,兵要資裝,則將何以備之?其緣倉庫不實,妨於政化者,觸類而是。



 臣竊見比者營造寺觀,其數極多,皆務取宏博,競崇環麗。大則費耗百十萬,小則尚用三五萬餘,略計都用資財,動至千萬已上。轉運木石,人牛不停,廢人功,害農務,事既非急,時多怨咨。故《書》曰:「不作無益害有益,功乃成;不貴異物賤用物,民乃足。」誠哉此言,非虛談也。且玄旨秘妙,歸於空寂,茍非修心定
 慧,諸法皆涉有為。至如土木雕刻等功,唯是殫竭人力,但學相誇壯麗,豈關降伏身心。且凡所興功,皆須掘鑿,蟄蟲在土,種類實多。每日殺傷,動盈萬計,連年如此,損害可知。聖人慈悲為心,豈有須行此事,不然之理,皎在目前。世俗眾僧,未通其旨,不慮府庫空竭,不思聖人憂勞,謂廣樹福田,即是增修法教。倘水旱為災,人至饑餒,夷狄作梗,兵無資糧,陛下雖有龍象如云,伽藍概日,豈能裨萬分之一,救元元之苦哉!於道法既有乖,在生人
 極為損,陛下豈可不深思之!



 臣竊見食封之家,其數甚眾。昨略問戶部,云用六十餘萬丁,一丁兩匹,即是一百二十萬已上。臣頃在太府,知每年庸調絹數,多不過百萬,少則七八十萬已來,比諸封家,所入全少。倘有蟲霜旱澇,曾不半在,國家支供,何以取給?臣聞自封茅土,裂山河,皆須業著經綸,功申草昧,然後配宗廟之享,承帶礪之恩。皇運之初,功臣共定天下,當時食封才上三二十家,今以尋常特恩,遂至百家已上。國家租賦,太半私
 門,私門則資用有餘,國家則支計不足。有餘則或致奢侈,不足則坐致憂危,制國之方,豈謂為得?封戶之物,諸家自徵,或是官典,或是奴僕,多挾勢騁威,凌突州縣。凡是封戶,不勝侵擾,或輸物多索裹頭,或相知要取中物,百姓怨嘆,遠近共知。復有因將貨易,轉更生釁,徵打紛紛,曾不寧息,貧乏百姓,何以克堪!若必限丁物送太府,封家但於左藏請受,不得輒自徵催,則必免侵擾,人冀蘇息。



 臣又聞設官分職,量事置吏,此本於理人而務安
 之也。故《書》曰「在官人,在安人。官人則哲,安人則惠。能哲而惠,何憂乎歡兜,何畏乎有苗」者也!是明官得其人,而天下自理矣。古者取人,必先採鄉曲之譽,然後闢於州郡;州郡有聲,然後闢於五府;才著五府,然後升之天朝。此則用一人所擇者甚悉,擢一士所歷者甚深。孔子曰:「譬有美錦,不可使人學制。」此明用人不可不審擇也。用得其才則理,非其才則亂,理亂所設,焉可不深擇之哉!今之取人,有異此道。多未甚試效,即頓至遷擢。夫趨競
 者人之常情,僥幸人之所趣。而今務進不避僥幸者,接踵比肩,布於文武之列。有文者用理內外,則有回邪贓污上下敗亂之憂;有武者用將軍戎,則有庸懦怯弱師旅喪亡之患。補授無限,員闕不供,遂至員外置官,數倍正闕。曹署典吏,困於祗承,府庫倉儲,竭於資奉。國家大事,豈甚於此!古者懸爵待士,唯有才者得之,若任用無才,則有才之路塞,賢人君子所以遁跡銷聲,常懷嘆恨者也。且賢人君子,守於正直之道,遠於僥幸之門,若
 僥幸開,則賢者不可復出矣。賢者遂退,若欲求人安化洽,復不可得也。人若不安,國將危矣,陛下安可不深慮之!又刺史、縣令,理人之首。近年已來,不存簡擇。京官有犯及聲望下者,方遣牧州;吏部選人,暮年無手筆者,方擬縣令。此風久扇,上下同知,將此理人,何以率化?今歲非豐稔,戶口流亡,國用空虛,租調減削。陛下不以此留念,將何以理國乎?臣望下明制,具論前事,使有司改換簡擇,天下刺史、縣令,皆取才能有稱望者充。自今已往,
 應有遷除諸曹侍郎、兩省、兩臺及五品已上清望官,先於刺史、縣令中選用。牧宰得人,天下大理,萬姓欣欣然,豈非太平樂事哉!唯陛下詳擇。



 疏奏不納。



 嗣立與韋庶人宗屬疏遠,中宗特令編入屬籍,由是顧賞尤重。賞於驪山構營別業,中宗親往幸焉,自制詩序,令從官賦詩,賜絹二千匹。因封嗣立為逍遙公,名其所居為清虛原幽棲谷。韋氏敗,幾為亂兵所害,寧王憲以嗣立是從母之夫,救護免之。睿宗踐祚,拜中書令。尋日,出為許州刺
 史。以定冊尊立睿宗之功,賜實封一百戶。開元初,入為國子祭酒。先是,中宗遺制睿宗輔政,宗楚客、韋溫等改削槁草,嗣立時在政事府,不能正之。至是為憲司所劾,左遷岳州別駕。久之,遷陳州刺史。時河南道巡察使、工部尚書劉知柔奏嗣立清白可陟之狀,詔命未下,開元七年卒,贈兵部尚書,謚曰孝。中書門下又奏:「嗣立衣冠之內,夙表才名;兄弟之間,特稱和睦。承恩歷事,位列宰臣。中年以不能正身,頗近兇戚,為憲司糾劾,因茲出貶。
 若循其始,終是吉人,宜棄其瑕,以從眾望。請贈物一百段。」從之。



 嗣立、承慶俱以學行齊名。長壽中,嗣立代承慶為鳳閣舍人。長安三年,承慶代嗣立為天官侍郎,頃之又代嗣立知政事。及承慶卒,嗣立又代為黃門侍郎,前後四職相代。又父子三人,皆至宰相。有唐已來,莫與為比。嗣立三子:孚、恆、濟,皆知名。孚,累遷至左司員外郎。恆,開元初為碭山令。為政寬惠,人吏愛之。會車駕東巡,縣當供帳,時山東州縣皆懼不辦,務於鞭撲,恆獨不杖罰
 而事皆濟理,遠近稱焉。御史中丞宇文融,即恆之姑子也,嘗密薦恆有經濟之才,請以己之官秩回授,乃擢拜殿中侍御史。歷度支左司等員外、太常少卿、給事中。二十九年,為隴右道河西黜陟使。恆至河西時,節度使蓋嘉運恃托中貴,公為非法,兼偽敘功勞,恆抗表請劾之,人代其懼。因出為陳留太守,未行而卒,時人甚傷惜之。濟,早以辭翰聞。開元初,調補鄄城令。時有人密奏玄宗:「今歲吏部選敘太濫,縣令非材,全不簡擇。」及縣令謝官
 日,引入殿庭,問安人策一道,試者二百餘人,獨濟策第一,或有不書紙者。擢濟為醴泉令,二十餘人還舊官,四五十人放歸習讀,侍郎盧從願、李朝隱貶為刺史。濟至醴泉,以簡易為政,人用稱之。三遷為庫部員外郎。二十四年,為尚書戶部侍郎。累歲轉太原尹。制《先德詩》四章,述祖、父之行,辭致高雅。天寶七載,又為河南尹,遷尚書左丞。三代為省轄,衣冠榮之。濟從容雅度,所蒞人推善政,後出為馮翊太守。



 陸元方,蘇州吳縣人。世為著姓。曾祖琛,陳給事中黃門侍郎。伯父柬之,以工書知名,官至太子司議郎。元方舉明經,又應八科舉,累轉監察御史。則天革命,使元方安輯嶺外。將涉海,時風濤甚壯,舟人莫敢舉帆。元方曰:「我受命無私,神豈害我?」遽命之濟,既而風濤果息。使還稱旨,除殿中侍御史。即以其月擢拜鳳閣舍人,仍判侍郎事。俄為來俊臣所陷,則天手敕特赦之。長壽二年,再遷鸞臺侍郎、同鳳閣鸞臺平章事。延載初,又加鳳閣侍郎。
 證聖初,內史李昭德得罪,以元方附會昭德,貶綏州刺史。尋復為春官侍郎,又轉天官侍郎、尚書左丞,尋拜鸞臺侍郎、平章事。則天嘗問以外事,對曰:「臣備位宰臣,有大事即奏,人間碎務,不敢以煩聖覽。」由是忤旨,責授太子右庶子,罷知政事。尋轉文昌左丞,病卒。



 元方在官清謹,再為宰相,則天將有遷除,每行以訪之,必密封以進,未嘗露其私恩。臨終,取前後草奏悉命焚之,且曰:「吾陰德於人多矣,其後庶幾福不衰矣。」又有書一匣,常自緘
 封,家人莫有見者,及卒視之,乃前後敕書,其慎密如此。贈越州都督。開元十八年,又贈揚州大都督。子象先。



 象先,本名景初。少有器量,應制舉,拜揚州參軍。秩滿調選,時吉頊為吏部侍郎,擢授洛陽尉,元方時亦為吏部,固辭不敢當。頊曰:「為官擇人,至公之道。陸景初才望高雅,非常流所及,實不以吏部之子妄推薦也。」竟奏授之。遷左臺監察御史,轉殿中,歷授中書侍郎。



 景雲二年冬,同中書門下平章事,監修國史。初,太平公主將引中書侍
 郎崔湜知政事,密以告之,湜固讓象先,主不許之,湜因亦請辭。主遽言於睿宗,乃並拜焉。象先清凈寡欲,不以細務介意,言論高遠,雅為時賢所服。湜每謂人曰:「陸公加於人一等。」太平公主時既用事,同時宰相蕭至忠、岑義及湜等咸傾附之,唯象先孤立,未嘗造謁。先天二年,至忠等伏誅,象先獨免其難。以保護功封兗國公,賜實封二百戶,加銀青光祿大夫。時窮討至忠等枝黨,連累稍眾,象先密有申理,全濟甚多,然未嘗言及,當時無知
 之者。



 其年,出為益州大都督府長史,仍為劍南道按察使。在官務以寬仁為政,司馬韋抱真言曰:「望明公稍行杖罰,以立威名。不然,恐下人怠墮,無所懼也。」象先曰:「為政者理則可矣,何必嚴刑樹威。損人益己,恐非仁恕之道。」竟不從抱真之言。歷遷河中尹。六年,廢河中府,依舊為蒲州,象先為刺史,仍為河東道按察使。嘗有小人犯罪,但示語而遣之。錄事白曰:「此例當合與杖。」象先曰:「人情相去不遠,此豈不解吾言?若必須行杖,即當自汝為
 始。」錄事慚懼而退。象先嘗謂人曰:「天下本自無事,祗是庸人擾之,始為繁耳。但當靜之於源,則亦何憂不簡。」前後為刺史,其政如一,人吏咸懷思之。按察使停,入為太子詹事,歷工部尚書。十年冬,知吏部選事,又加刑部尚書,以繼母憂免官。十三年,起復同州刺史,尋遷太子少保。二十四年卒,年七十二,贈尚書左丞相,謚曰文貞。



 象先弟景倩,歷監察御史。景融,歷大理正、滎陽郡太守、河南尹、兵吏部侍郎、左右丞、工部尚書、東都留守、襄陽郡
 太守、陳留郡太守,並兼採訪使。景獻,歷殿中侍御史、屯田員外郎。景裔,河南令、庫部郎中。皆有美譽。僧一行少時,嘗與象先昆弟相善,常謂人曰:「陸氏兄弟皆有才行,古之荀、陳,無以加也。」其為當時所稱如此。



 元方從叔餘慶,陳右軍將軍珣孫也。少與知名之士陳子昂、宋之問、盧藏用、道士司馬承禎、道人法成等交游,雖才學不逮子昂等,而風流強辯過之。累遷中書舍人。則天嘗引入草詔,餘慶惶惑,至晚竟不能措一辭,責授左司郎中。累
 除大理卿、散騎常侍、太子詹事。以老疾致仕,尋卒。象先四代孫,文宗太和四年,除釋褐參軍文學。



 蘇瑰,字昌容,就兆武功人,隋尚書右僕射威曾孫也。祖夔,隋鴻臚卿。父亶,貞觀中臺州刺史。瑰弱冠本州舉進士,累授豫王府錄事參軍。長史王德真、司馬劉禕之皆器重之。長安中,累遷揚州大都督府長史。揚州地當沖要,多富商大賈,珠翠珍怪之產,前長史張潛、於辯機皆致之數萬,唯瑰挺身而去。神龍初,入為尚書右丞,以明
 習法律,多識臺閣故事,特命刪定律、令、格、式。尋加銀青光祿大夫。是歲,再遷戶部尚書,奏計帳,所管戶時有六百一十五萬六千一百四十一。



 尋加侍中。封淮陽縣子,充西京留守。時秘書員外監鄭普思謀為妖逆,雍、岐二州妖黨大發,瑰收普思系獄考訊之。普思妻第五氏以鬼道為韋庶人所寵,居止禁中,由是中宗特敕慰諭瑰,令釋普思之罪。瑰上言普思幻惑,罪當不赦。中宗至京,又面陳其狀。尚書左僕射魏元忠奏曰:「蘇瑰長者,其
 忠懇如此,願陛下察之。」帝乃配流普思於儋州,其黨並誅。瑰遷吏部尚書,進封淮陽縣侯。



 景龍三年,轉尚書右僕射、同中書門下三品,進封許國公。是歲,將拜南郊,國子祭酒祝欽明希庶人旨,建議請皇后為亞獻,安樂公主為終獻。瑰深非其議,嘗於御前面折欽明,帝雖悟,竟從欽明所奏。公卿大臣初拜官者,例許獻食,名為「燒尾」。瑰拜僕射無所獻。後因侍宴,將作大匠宗晉卿曰:「拜僕射竟不燒尾,豈不喜耶?」帝默然。瑰奏曰:「臣聞宰相者,主
 調陰陽,代天理物。今粒食踴貴,百姓不足,臣見宿衛兵至有三日不得食者。臣愚不稱職,所以不敢燒尾。」是歲六月,與唐休璟並加監修國史。



 四年,中宗崩,秘不發喪,韋庶人召諸宰相韋安石、韋巨源、蕭至忠、宗楚客、紀處訥、韋溫、李嶠、韋嗣立、唐休璟、趙彥昭及瑰等十九人入禁中會議。初,遺制遣韋庶人輔少主知政事,授安國相王太尉參謀輔政。中書令宗楚客謂溫曰:「今須請皇太后臨朝,宜停相王輔政。且皇太后於相王居嫂叔不通問
 之地,甚難為儀注,理全不可。」瑰獨正色拒之,謂楚客等曰:「遺制是先帝意,安可更改!」楚客及韋溫大怒,遂削相王輔政而宣行焉。是月,韋氏敗,相王即帝位,下詔曰:「尚書右僕射、同中書門下三品、監修國史、許國公蘇瑰,自周旋近密,損益樞機,謀猷有成,匡贊無忌。頃者遺恩顧托,先意昭明,奸回動搖,內外危逼,獨申讜議,實挫邪謀。況籓邸僚屬,念殷惟舊,無德不報,抑惟令典。可尚書左僕射,餘如故。」



 景雲元年,以老疾轉太子少傅。是歲十一
 月薨,贈司空、荊州大都督,謚曰文貞。瑰臨終遺令薄葬,及祖載之日,官給儀仗外,唯有布車一乘,論者稱焉。開元二年,下詔曰:「疇庸賞善,百王攸先;追還飾終,千載同德,故尚書左丞相、太子少傅、贈司空、荊州大都督、許國文貞公,瑰履正體道,外方內直,悉心奉上,卑身率禮。協贊帷幄,三朝有鹽梅之任;燮諧臺袞,九命為社稷之臣。先朝晏駕,釁起宮掖,國擅稱制之奸,人懷綴旒之懼。兇威孔熾,宗祀幾傾。顧命遺恩,太皇輔政,逆臣刊削,韋氏
 臨朝。遂能首發昌言,侃然正色,列諸視聽,暴於朝野。松檟已遠,風烈猶存,糸面懷誠節,良深耿嘆。可賜實封一百戶。」四年,詔與徐國公劉幽求配享睿宗廟庭。十七年,加贈司徒。



 瑰子頲,少有俊才,一覽千言。弱冠舉進士,授烏程尉,累遷左臺監察御史。長安中,詔頲按覆來俊臣等舊獄,頲皆申明其枉,由此雪冤者甚眾。



 神龍中,累遷給事中,加修文館學士,俄拜中書舍人。尋而頲父同中書門下三品,父子同掌樞密,時以為榮。機事填委,文誥皆
 出頲手。中書令李嶠嘆曰:「舍人思如湧泉,嶠所不及也。」俄遷太常少卿。景雲中,瑰薨,詔頲起復為工部侍郎,加銀青光祿大夫。頲抗表固辭,辭理懇切,詔許其終制。服闋就職,襲父爵許國公。玄宗謂宰臣曰:「有從工部侍郎得中書侍郎否?」對曰:「任賢用能,非臣等所及。」玄宗曰:「蘇頲可中書侍郎,仍供政事食。」明日,加知制誥。有政事食,自頲始也。頲入謝,玄宗曰:「常欲用卿,每有好官闕,即望宰相論及。宰相皆卿之故人,卒無言者,朕為卿嘆息。中
 書侍郎,朕極重惜,自陸象先歿後,朕每思之,無出卿者。」時李乂為紫微侍郎,與頲對掌文誥。他日,上謂頲曰:「前朝有李嶠、蘇味道,謂之蘇、李;今有卿及李乂,亦不讓之。卿所制文誥,可錄一本封進,題云『臣某撰』,朕要留中披覽。」其禮遇如此。玄宗欲於靖陵建碑,頲諫曰:「帝王及後,無神道碑,且事不師古,動皆不法。若靖陵獨建,陛下祖宗之陵皆須追造。」玄宗從其言而止。



 開元四年,遷紫微侍郎、同紫微黃門平章事,與侍中宋璟同知政事。璟剛
 正,多所裁斷,頲皆順從其美;若上前承旨、敷奏及應對,則頲為之助,相得甚悅。璟嘗謂人曰:「吾與蘇家父子,前後同時為宰相。僕射長厚,誠為國器;若獻可替否,罄盡臣節,斷割吏事,至公無私,即頲過其父也」。八年,除禮部尚書,罷政事。俄知益州大都督府長史事。前司馬皇甫恂破庫物織新樣錦以進,頲一切罷之。或謂頲曰:「公今在遠,豈得忤聖意?」頲曰:「明主不以私愛奪至公,豈以遠近間易忠臣節也!」竟奏罷之。巂州蠻酋苴院私與吐蕃
 連謀,將為內寇,頲獲其間諜,將士咸請出兵討之,頲不從,乃作書並間諜以送苴院,苴院慚悔,竟不敢入寇。



 十三年,從駕東封,玄宗令頲撰朝覲碑文。俄又知吏部選事。頲性廉儉,所得俸祿,盡推與諸弟,或散之親族,家無餘資。十五年卒,年五十八。初,優贈之制未出,起居舍人韋述上疏曰:「臣伏見貞觀、永徽之時,每有公卿大臣薨卒,皆輟朝舉哀,所以成終始之恩,厚君臣之義。上有旌賢錄舊之德,下有生榮死哀之美,列於史冊,以示將來。
 昔智悼子卒,平公宴樂,杜蒯一言,言始感悟。《春秋》載其盛烈,禮經以為美談,今古舊事,昭然可睹。臣伏見故禮部尚書蘇頲,累葉輔弼,代傳忠清。頲又伏事軒陛二十餘載,入參謀猷,出總籓牧。誠績斯著,操履無虧,天不憖遺,奄違聖代。伏願陛下思帷蓋之舊,念股肱之親,修先朝之盛典,鑒晉平之遠跡,為之輟朝舉哀,以明同體之義。使歿者荷德於泉壤,存者盡節於周行,凡百卿士,孰不幸甚。臣官忝記事,君舉必書,敢申舊典,上黷宸扆,希
 降恩貸,俯垂詳擇。」即日於洛城南門舉哀,輟朝兩日,贈尚書右丞相,謚曰文憲。及葬日,玄宗游咸宜宮,將出獵,聞頲喪出,愴然曰:「蘇頲今日葬,吾寧忍娛游。」中路還宮。頲弟詵、冰、乂。



 詵,歷授右司郎中、給事中、徐州刺史。先是,拜給事中時,頲為中書侍郎,上表讓詵所授。玄宗曰:「古來有內舉不避親乎?」頲曰:「晉祁奚是也。」玄宗曰:「若然,則朕用蘇詵,何得屢言?近日卿父子猶同在中書,兄弟有何不得?卿言非至公也。」冰,為虞部郎中。乂,為職方郎中。



 乾,瑰從父兄也。父勖,武德中為秦王府文學館學士。貞觀中,尚南康公主,拜駙馬都尉,累選魏王泰府司馬。勖既博學有美名,甚為泰所重。因勸泰請開文學館,引才名之士,撰《括地志》。後歷吏部郎、太子左庶子,卒。乾少以明經累授徐王府記室參軍。徐王好畋獵,干每諫止之。垂拱中,歷遷魏州刺史。時河北饑饉,舊吏苛酷,百姓多有逃散。干乃督察奸吏,務勸農桑,由是逃散者皆來復業,稱為良牧。召拜右羽林將軍,尋遷冬官尚書。酷吏來
 俊臣素忌嫉之,遂誣奏乾在魏州與瑯邪王沖私書往復,因系獄鞫訊,幹發憤而卒。



 瑰四代孫翔,文宗太和四年,釋褐文學參軍。



 史臣曰:韋思謙始以州縣,奮於煙霄,持綱不避於權豪,報國能忘於妻子。自強不息,剛毅近仁,信有之矣!高季輔、皇甫公義,可謂知人矣!且福善餘慶,不謂無徵,二子構堂,俱列相輔,文皆經濟,政盡明能。加以承慶方危,染翰而曾非恐悚;嗣立見用,襲封而罔墜逍遙。無忝父風,
 寧慚祖德,謚溫謚孝,何愧易名?陸元方博學大度,再踐鈞衡,當則天時,非有忠貞,應無黜責,綏州之任,抑又何慚!觀其濟海無私,狂風自止,臨終焚槁,溫樹始彰。故知正可以動神明,德可以延家代。象先益高人品,尤著相才,全濟有名,孤立無禍。景倩、景融、景獻、景裔等咸居清列,得非有後於魯乎?蘇瑰,孔子云:「居其室,出其言善,則千里之外應之,況其邇者乎!」又「言行君子之樞機,樞機之發,榮辱之主也」。當中宗棄代,韋氏奪權,預謀者十有
 九人,咸生異議,瑰志存大節,獨發讜言。其後善惡顯彰,黜陟明著。聖人之言,驗於斯矣。頲唯公是相,以儉承家,李嶠許之湧泉,宋璟稱其過父。艱難之際,節操不回,善始令終,先後無愧。



 贊曰:善人君子,懷忠秉正。盡富文章,咸推諫諍。豈愧明廷,無慚重柄。子子孫孫,演承餘慶。



\end{pinyinscope}