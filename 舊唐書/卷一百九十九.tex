\article{卷一百九十九}

\begin{pinyinscope}

 ○郭正一元萬頃範履冰苗神客周思茂胡楚賓附喬知之弟侃備劉希夷附劉允濟富嘉謨吳少微谷倚附員半千丘悅附
 劉憲王適司馬鍠梁載言附沈佺期陳子昂閭丘均附宋之問閻朝隱王無競李適尹元凱附賈曾子至許景先賀知章賀朝萬齊融張若虛邢巨包融李登之附席豫徐安貞附齊浣王浣李邕孫逖子成



 郭正一,定州彭城人。貞觀中舉進士。累轉中書舍人、弘
 文館學士。永隆二年,遷秘書少監,檢校中書侍郎,與魏玄同、郭待舉並同中書門下平章事。宰相以平章事為名,自正一等始也。永淳二年,正除中書侍郎。正一在中書累年,明習舊事,兼有詞學,制敕多出其手,當時號為稱職。則天臨朝,轉國子祭酒,罷知政事。尋出為晉州刺史,入為麟臺監,又檢校陜州刺史。永昌元年,為酷史所陷,流配嶺南而死,家口籍沒,文集多遺失。



 先是,儀鳳中,吐蕃入寇,工部尚書劉審禮率兵十八萬,與蕃將倫欽
 陵戰於青海,王師大敗,審禮沒於陣。高宗駭然,乃召侍臣問以御戎之策,正一對曰:「吐蕃作梗,年歲已深,命將興師,相繼不絕,空勞士馬,虛費糧儲,近討則徒損兵威,深入則未窮巢穴。臣望少發兵募,且遣備邊,明立烽候,勿令侵擾。伺國用豐足,人心葉同,寬之數年,可一舉而滅。」給事中劉齊賢、皇甫文亮等亦以為嚴守為便。正一才略,率多此類。



 元萬頃,洛陽人,後魏景穆皇帝之胤。祖白澤,武德中總
 管。萬頃善屬文,起家拜通事舍人。乾封中,從英國公李勣征高麗,為遼東道總管記室。別帥馮本以大軍援裨將郭待封,船破失期。待封欲作書與勣,恐高麗知其救兵不至,乘危迫之,乃作離合詩贈勣。勣不達其意,大怒曰:「軍機急切,何用詩為?必斬之!」萬頃為解釋之,乃止。



 勣嘗令萬頃作文檄高麗,其語有譏高麗「不知守鴨綠之險」,莫離支報云「謹聞命矣」,遂移兵固守鴨綠,官軍不得入,萬頃坐是流於嶺外。後會赦得還,拜著作郎。



 時天后
 諷高宗廣召文詞之士入禁中修撰,萬頃與左史範履冰、苗神客,右史周思茂、胡楚賓咸預其選,前後撰《列女傳》、《臣軌》、《百僚新誡》、《樂書》等凡千餘卷。朝廷疑議及百司表疏,皆密令萬頃等參決,以分宰相之權,時人謂之「北門學士」。



 萬頃屬文敏速,然性疏曠,不拘細節,無儒者之風。則天臨朝,遷鳳閣舍人。無幾,擢拜鳳閣侍郎。



 萬頃素與徐敬業兄弟友善,永昌元年為酷吏所陷,配流嶺南而死。時神客、楚賓已卒,履冰、思茂相次為酷吏所殺。



 範履冰者,懷州河內人。自周王府戶曹召入禁中,凡二十餘年。垂拱中,歷鸞臺、天官二侍郎。尋遷春官尚書、同鳳閣鸞臺平章事,兼修國史。載初元年,坐嘗舉犯逆者被殺。



 苗神客者,滄州東光人。官至著作郎。



 周思茂者,貝州漳南人。少與弟思鈞,俱早知名。自右史轉太子舍人。與範履冰在禁中最蒙親遇,至於政事損益,多參預焉。累遷麟臺少監、崇文館學士。垂拱四年,下
 獄死。



 胡楚賓者,宣州秋浦人。屬文敏速,每飲半酣而後操筆。高宗每令作文,必以金銀杯盛酒令飲,便以杯賜之。楚賓終日酣宴,家無所藏,費盡復入待詔,得賜又出。然性慎密,未嘗言禁中事,醉後人或問之,答以他事而已。自殷王文學拜右史、崇賢直學士而卒。



 喬知之,同州馮翊人也。父師望,尚高祖女廬陵公主,拜駙馬都尉,官至同州刺史。知之與弟侃、備,並以文詞知
 名。知之尤稱俊才,所作篇詠,時人多諷誦之。則天時,累除右補闕,遷左司郎中。知之有侍婢曰窈娘,美麗善歌舞,為武承嗣所奪。知之怨惜,因作《綠珠篇》以寄情,密送與婢,婢感憤自殺。承嗣大怒,因諷酷吏羅織誅之。



 侃,開元初為兗州都督。



 備,預修《三教珠英》,長安中卒於襄陽令。



 時又有汝州人劉希夷,善為從軍閨情之詩,詞調哀苦,為時所重,志行不修,為奸人所殺。



 劉允濟,洛州鞏人,其先自沛國徙焉。南齊彭城郡丞愬
 六代孫也。少孤,事母甚謹。博學善屬文,與絳州王勃早齊名,特相友善。弱冠,本州舉進士,累除著作佐郎。允濟嘗採摭魯哀公後十二代至於戰國遺事,撰《魯後春秋》二十卷。表上之,遷左史,兼直弘文館。垂拱四年,明堂初成,允濟奏上《明堂賦》以諷,則天甚嘉嘆之,手制褒美,拜著作郎。



 天授中,為來俊臣所構,當坐死,以其母老,特許終其餘年,仍留系獄。久之,會赦免,貶授大庾尉。長安中,累遷著作佐郎,兼修國史。未幾,擢拜鳳閣舍人。中興初,
 坐與張易之款狎,左授青州長史,為吏清白,河南道巡察使路敬潛甚稱薦之。尋丁母憂,服闋而卒。



 富嘉謨,雍州武功人也。舉進士。長安中,累轉晉陽尉,與新安吳少微友善,同官。先是,文士撰碑頌,皆以徐、庾為宗,氣調漸劣。嘉謨與少微屬詞,皆以經典為本,時人欽慕之,文體一變,稱為富吳體。嘉謨作《雙龍泉頌》、《千蠋穀頌》,少微撰《崇福寺鐘銘》,詞最高雅,作者推重。並州長史張仁亶待以殊禮,坐必同榻。嘉謨後為壽安尉,預修《三
 教珠英》。中興初,為左臺監察御史,卒。有文集五卷。



 少微亦舉進士,累至晉陽尉。中興初,調於吏部,侍郎韋嗣立稱薦,拜右臺監察御史。臥病,聞嘉謨死,哭而賦詩,尋亦卒。有文集五卷。



 嘉謨與少微在晉陽,魏郡穀倚為太原主簿,皆以文詞著名,時人謂之「北京三傑」。倚後流寓客死,文章遺失。



 微子鞏,開元中,為中書舍人。



 員半千,本名餘慶,晉州臨汾人。少與齊州人何彥先同師事學士王義方,義方嘉重之,嘗謂之曰:「五百年一賢,
 足下當之矣!」因改名半千。及義方卒,半千與彥先皆制服,喪畢而去。



 上元初,應八科舉,授武陟尉。屬頻歲旱饑,勸縣令殷子良開倉以賑貧餒,子良不從。會子良赴州,半千便發倉粟以給饑人。懷州刺史郭齊宗大驚,因而按之。時黃門侍郎薛元超為河北道存撫使,謂齊宗曰:「公百姓不能救之,而使惠歸一尉,豈不愧也!」遽令釋之。尋又應岳牧舉。



 高宗御武成殿,召諸州舉人,親問曰:「兵書所云天陣、地陣、人陣,各何謂也?」半千越次而進曰:「臣
 觀載籍,此事多矣。或謂:天陣,星宿孤虛;地陣,山川向背;人陣,偏伍彌縫。以臣愚見,謂不然矣。夫師出以義,有若時雨,得天之時,此天陣也;兵在足食,且耕且戰,得地之利,此地陣也;善用兵者,使三軍之士,如父子兄弟,得人之和,此人陣也。三者去矣,其何以戰!」高宗甚嗟賞之。及對策,擢為上第。



 垂拱中,累補左衛胄曹,仍充宣慰吐蕃使。及引辭,則天曰:「久聞卿名,謂是古人,不意乃在朝列。境外小事,不足煩卿,宜留待制也。」即日使入閣供奉。嗣
 聖元年,半千為左衛長史,與鳳閣舍人王處知、天官侍郎石抱忠,並為弘文館直學士,仍與著作佐郎路敬淳分日於顯福門待制。半千因撰《明堂新禮》三卷,上之。則天封中嶽,半千又撰《封禪四壇碑》十二首以進,則天稱善。前後賜絹千餘匹。



 長安中,五遷正諫大夫,兼右控鶴內供奉。半千以控鶴之職,古無其事,又授斯任者率多輕薄,非朝廷進德之選,上疏請罷之。由是忤旨,左遷水部郎中,預修《三教珠英》。



 中宗時,為濠州刺史。睿宗即位,
 徵拜太子右諭德,兼崇文館學士,加銀青光祿大夫,累封平原郡公。開元二年卒。文集多遺失。半千同時學士丘悅。



 丘悅者,河南陸渾人也。亦有學業。景龍中,為相王府掾,與文學韋利器、典簽裴耀卿俱為王府直學士。睿宗在籓甚重之,官至岐王傅。開元初卒。撰《三國典略》三十卷,行於時。



 劉憲,宋州寧陵人也。父思立,高宗時為侍御史。屬河南、
 河北旱儉,遣御史中丞崔謐等分道存問賑給,思立上疏諫曰:「今麥序方秋,蠶功未畢,三時之務,萬姓所先。敕使撫巡,人皆竦抃,忘其家業,冀此天恩,踴躍參迎,必難抑止,集眾既廣,妨廢亦多。加以途程往還,兼之晨夕停滯。既緣賑給,須立簿書,本欲安存,卻成煩擾。又無驛之處,其馬稍難。簡擇公私,須預追集。雨後農務,特切常情,暫廢須臾,即虧歲計。每為一馬,遂勞數家,從此相乘,恐更滋甚。望且委州縣賑給,待秋閑時出使褒貶。」疏奏,謐
 等遂不行。後遷考功員外郎,始奏請明經加帖、進士試雜文,自思立始也。尋卒官。



 憲弱冠舉進士,累除冬官員外郎。



 天授中,受詔推按來俊臣。憲嫉其酷暴,欲因事繩之,反為俊臣所構,貶濆水令。再遷司僕丞。及俊臣伏誅,擢憲為給事中,尋轉鳳閣舍人。



 神龍初,坐嘗為張易之所引,自吏部侍郎出為渝州刺史。俄復入為太僕少卿,兼修國史,加修文館學士。景雲初,三遷太子詹事



 玄宗在東宮,留意經籍,憲因上啟曰:「自古及今,皆重於學。至
 於光耀盛德,發揚令問,安靜身心,保寧家國,無以加焉。殿下居副君之位,有絕人之才,豈假尋章摘句,蓋資略知大意,用功甚少,為利極多。伏願克成美志,無棄暇日,上以慰至尊之心,下以答庶僚之望。侍讀褚無量,經明行修,耆年宿望,時賜召問,以察其言,幸甚!」玄宗甚嘉納之。明年,憲卒,贈兗州都督。有集三十卷。



 初則天時,敕吏部糊名考選人判,以求才彥,憲與王適、司馬鍠、梁載言相次判入第二等。



 王適,幽州人。官至雍州司功。



 司馬皪,洛州溫人也。神龍中,卒於黃門侍郎。



 梁載言,博州聊城人。歷鳳閣舍人,專知制誥。撰《具員故事》十卷,《十道志》十六卷,並傳於時。中宗時為懷州刺史。



 沈佺期,相州內黃人也。進士舉。長安中,累遷通事舍人,預修《三教珠英》。



 佺期善屬文,尤長七言之作,與宋之問齊名,時人稱為沈宋。再轉考功員外郎,坐贓配流嶺表。神龍中,授起居郎,加修文館直學士。後歷中書舍人、太
 子詹事。開元初卒。有文集十卷。



 弟全交及子,亦以文詞知名。



 陳子昂,梓州射洪人。家世富豪。子昂獨苦節讀書,尤善屬文。初為《感遇詩》三十首,京兆司功王適見而驚曰:「此子必為天下文宗矣!」由是知名。舉進士。會高宗崩,靈駕將還長安,子昂詣闕上書,盛陳東都形勝,可以安置山陵,關中旱儉,靈駕西行不便。曰:



 梓州射洪縣草莽愚臣子昂,謹頓首冒死獻書闕下。



 臣聞明王不惡切直之言
 以納忠,烈士不憚死亡之誅以極諫。故有非常之策者,必待非常之時;得非常之時者,必待非常之主。然後危言正色,抗義直辭,赴湯鑊而不回,至誅夷而無悔!豈徒欲詭世誇俗,厭生樂死者哉!實以為殺身之害小,存國之利大。故審計定議而甘心焉。況乎得非常之時,遇非常之主,言必獲用,死亦何驚!千載之跡,將不朽於今日矣!



 伏惟大行皇帝遺天下,棄群臣,萬國震驚,百姓屠裂。陛下以徇齊之聖,承宗廟之重,天下之望,喁喁如也。莫
 不冀蒙聖化,以保餘年;太平之主,將復在於茲矣!況皇太后又以文母之賢,協軒宮之耀,軍國大事,遺詔決之;唐、虞之際,於斯盛矣!



 臣伏見詔書,梓宮將遷西京,鸞輿亦欲陪幸。計非上策,智者失圖;廟堂未聞有骨鯁之謨,朝廷多見有順從之議;臣竊惑以為過矣!伏自思之,生聖日,沐皇風,摩頂至踵,莫非亭育;不能歷丹鳳,抵濯龍,北面玉階,東望金屋,抗音而正諫者,聖王之罪人也!所以不顧萬死,乞獻一言,願蒙聽覽,甘就鼎鑊,伏惟陛下
 察之。



 臣聞秦都咸陽之時,漢都長安之日,山河為固,天下服矣。然猶北取胡、宛之利,南資巴蜀之饒。自渭入河,轉關東之粟;逾沙絕漠,致山西之儲。然後能削平天下,彈壓諸侯,長轡利策,橫制宇宙。今則不然。燕、代迫匈奴之侵,巴、隴嬰吐蕃之患;西蜀疲老,千里贏糧;北國丁男,十五乘塞;歲月奔命,其弊不堪。秦之首尾,今為闕矣!即所餘者,獨三輔之間耳。頃遭荒饉,人被薦饑。自河已西,莫非赤地;循隴已北,罕逢青草。莫不父兄轉徙,妻子流
 離,委家喪業,膏原潤莽,此朝廷之所備知也。賴以宗廟神靈,皇天悔禍,去歲薄稔,前秋稍登,使羸餓之餘,得保性命,天下幸甚,可謂厚矣!然而流人未返,田野尚蕪,白骨縱橫,阡陌無主。至於蓄積,尤可哀傷。陛下不料其難,貴從先意,遂欲長驅大駕,按節秦京,千乘萬騎,何方取給?況山陵初制,穿復未央;土木工匠,必資徒役。今欲率疲弊之眾,興數萬之軍,徵發近畿,鞭撲羸老,鑿山採石,驅以就功。春作無時,秋成絕望,凋瘵遺噍,再罹艱苦。倘
 不堪弊,必有逋逃,「子來」之頌,將何以述之?此亦宗廟之大機,不可不審圖也!況國無兼歲之儲,家鮮匝時之蓄。一旬不雨,猶可深憂,忽加水旱,人何以濟?陛下不深察始終,獨違群議,臣恐三輔之弊,不止如前日矣!



 且天子以四海為家,聖人包六合為宇。歷觀邃古,以至於今,何嘗不以三王為仁,五帝為聖!雖周公制作,夫子著明,莫不祖述堯、舜,憲章文、武,為百王之鴻烈,作千載之雄圖!然而舜死陟方,葬蒼梧而不返;禹會群後,歿稽山而永
 終。豈其愛蠻夷之鄉而鄙中國哉?實將欲示聖人無外也。故能使墳籍以為美談,帝王以為高範。況我巍巍大聖,轢帝登皇,日月所照,莫不率俾。何獨秦、豐之地,可置山陵;河、洛之都,不堪園寢?陛下豈不察之,愚臣竊為陛下惜也!且景山崇麗,秀冠群峰,北對嵩、邙,西望汝海,居祝融之故地,連太昊之遺墟。帝王圖跡,縱橫左右;園陵之美,復何加焉!陛下曾未察之,謂其不可;愚臣鄙見,良足尚矣!況瀍、澗之中,天地交會,北有太行之險,南有宛、
 葉之饒,東壓江、淮,食湖淮之利,西馳崤、澠,據關河之寶。以聰明之主,養純粹之人,天下和平,恭己正南面而已。陛下不思瀍、洛之壯觀,關、隴之荒蕪,乃欲棄太山之安,履焦原之險,忘神器之大寶,徇曾、閔之小節。愚臣暗昧,以為甚也!陛下何不覽爭臣之策,採行路之謠,諮謨太后,平章宰輔,使蒼生之望,知有所安,天下豈不幸甚!



 昔者平王遷都,光武都洛,山陵寢廟,不在東京;宗社墳塋,並居西土。然而《春秋》美為始王,《漢書》載為代祖,豈其不
 願孝哉?何聖賢褒貶於斯濫矣?實以時有不可,事有必然。蓋欲遺小存大,去禍歸福,聖人所以貴也。夫小不忍,亂大謀,仲尼之至誡,願陛下察之。若以臣愚不用,朝議遂行,臣恐關、隴之憂,無時休也!



 臣又聞太原蓄鉅萬之倉,洛口積天下之粟,國家之資,斯為大矣!今欲舍而不顧,背以長驅,使有識驚嗟,天下失望。倘鼠竊狗盜,萬一不圖,西入陜州之郊,東犯武牢之鎮,盜敖倉一抔之粟,陛下何以遏之?此天下之至機,不可不深懼也。雖則盜
 未旋踵,誅刑已及,滅其九族,焚其妻子,泣辜雖恨,將何及焉!故曰:「先謀後事者逸,先事後謀者失。」「國之利器,不可以示人。」斯言豈徒設也,固願陛下念之!



 則天召見,奇其對,拜麟臺正字。則天將事雅州討生羌,子昂上書曰:



 麟臺正字臣子昂昧死上言。



 臣聞道路云:國家欲開蜀山,自雅州道入討生羌,因以襲擊吐蕃。執事者不審圖其利害,遂發梁、鳳、巴蜒兵以徇之。臣愚以為西蜀之禍,自此結矣!



 臣聞亂生,必由於怨。雅州邊羌,自國初已來,
 未嘗一日為盜。今一旦無罪受戮,其怨必甚。怨甚懼誅,必蜂駭西山。西山盜起,則蜀之邊邑,不得不連兵備守。兵久不解,則蜀之禍構矣!昔後漢末西京喪敗,蓋由此諸羌。此一事也。



 且臣聞吐蕃桀黠之虜,君長相信,而多奸謀。自敢抗天誅,邇來向二十餘載,大戰則大勝,小戰則小勝,未嘗敗一隊,亡一夫。國家往以薛仁貴、郭待封為虓武之將,屠十一萬眾於大非之川,一甲不返。又以李敬玄、劉審禮為廊廟之器,辱十八萬眾於青海之澤,
 身囚虜庭。是時精甲勇士,勢如雲雷,然竟不能擒一戎,馘一丑,至今而關、隴為空。今乃欲以李處一為將,驅憔悴之兵,將襲吐蕃。臣竊憂之,而為此虜所笑。此二事也。



 且夫事有求利而得害者。則蜀昔時不通中國,秦惠王欲帝天下而並諸侯,以為不兼幹不取蜀,勢未可舉,乃用張儀計,飾美女,譎金牛,因間以啖蜀侯。蜀侯果貪其利,使五丁力士鑿通谷,棧褒斜,置道於秦。自是險阻不關,山谷不閉,張儀躡踵乘便,縱兵大破之,蜀侯誅,幹邑
 滅。至今蜀為中州,是貪利而亡。此三事也。



 且臣聞吐蕃羯虜,愛蜀之珍富,欲盜之,久有日矣。然其勢不能舉者,徒以山川阻絕,障隘不通,此其所以頓餓狼之喙而不得侵食也。今國家乃撤邊羌,開隘道,使其收奔亡之種,為向導以攻邊。是乃借寇兵而為賊除道,舉全蜀以遺之。此四事也。



 臣竊觀蜀為西南一都會,國家之寶庫,天下珍貨聚出其中。又人富粟多,順江而下,可以兼濟中國。今執事者乃圖僥幸之利,悉以委事西羌。地不足以
 富國,徒殺無辜之眾,以傷陛下之仁;糜費隨之,無益聖德。又況僥幸之利,未可圖哉!此五事也。



 夫蜀之所恃,有險也;人之所安,無役也。今國家乃開其險,役其人;險開則便寇,人役則傷財。臣恐未見羌戎,已有奸盜在其中矣!往年益州長史李崇真圖此奸利,傳檄稱吐蕃欲寇松州,遂使國家盛軍師、大轉餉以備之。未二三年,巴蜀二十餘州,騷然大弊,竟不見吐蕃之面,而崇真贓錢已計鉅萬矣。蜀人殘破,幾不堪命。此之近事,猶在人口,陛
 下所親知。臣愚意者不有奸臣欲圖此利,復以生羌為計者哉!此六事也。



 且蜀人尪劣,不習兵戰,一虜持矛,百人莫敢當。又山川阻曠,去中夏精兵處遠。今國家若擊西羌,掩吐蕃,遂能破滅其國,奴虜其人,使其君長系首北闕,計亦可矣!若不到如此,臣方見蜀之邊陲不守,而為羌夷所橫暴。昔辛有見被發而祭伊川者,以為不出百年,此其為戎。臣恐不及百年而蜀為戎。此七事也。



 且國家近者廢安北,拔單于,棄龜茲,放疏勒,天下翕然,謂
 之盛德。所以者何?蓋以陛下務在仁,不在廣;務在養,不在殺。將以此息邊鄙,休甲兵,行三皇、五帝之事者也!今又徇貪夫之議,謀動兵戈,將誅無罪之戎,而遺全蜀之患,將何以令天下乎?此愚臣所以不甚悟者也。況當今山東饑,關、隴弊,歷歲枯旱,人有流亡。誠是聖人寧靜,思和天人之時,不可動甲兵,興大役,以自生亂。臣又流聞西軍失守,北軍不利,邊人忙動,情有不安。今者復驅此兵,投之不測。臣聞自古亡國破家,未嘗不由黷兵。今小
 人議夷狄之利,非帝王之至德也,又況弊中夏哉!



 臣聞古之善為天下者,計大而不計小,務德而不務刑;圖其安則思其危,謀其利則慮其害;然後能長享福祿。伏願陛下熟計之!



 再轉右拾遺。數上疏陳事,詞皆典美。時有同州下邽人徐元慶,父為縣尉趙師韞所殺。後師韞為御史,元慶變姓名於驛家傭力,候師韞,手刃殺之。議者以元慶孝烈,欲舍其罪。子昂建議以為:「國法專殺者死,元慶宜正國法,然後旌其閭墓,以褒其孝義可也。」當時
 議者,咸以子昂為是。俄授麟臺正字。武攸宜統軍北討契丹,以子昂為管記,軍中文翰皆委之。



 子昂父在鄉,為縣令段簡所辱,子昂聞之,遽還鄉里。簡乃因事收系獄中,憂憤而卒,時年四十餘。



 子昂褊躁無威儀,然文詞宏麗,甚為當時所重。有集十卷,友人黃門侍郎盧藏用為之序,盛行於代。



 子昂卒後,益州成都人閭丘均,亦以文章著稱。景龍中,為安樂公主所薦,起家拜太常博士。而公主被誅,均坐貶為循州司倉,卒。有集十卷。



 宋之問,虢州弘農人。父令文,有勇力,而工書,善屬文。高宗時,為左驍衛郎將、東臺詳正學士。之問弱冠知名,尤善五言詩,當時無能出其右者。初徵令與楊炯分直內教,俄授洛州參軍,累轉尚方監丞、左奉宸內供奉。易之兄弟雅愛其才,之問亦傾附焉。預修《三教珠英》,常扈從游宴。則天幸洛陽龍門,令從官賦詩,左史東方虯詩先成,則天以錦袍賜之。及之問詩成,則天稱其詞愈高,奪虯錦袍以賞之。



 及易之等敗,左遷瀧州參軍。未幾,逃還,
 匿於洛陽人張仲之家。仲之與駙馬都尉王同皎等謀殺武三思,之問令兄子發其事以自贖。及同皎等獲罪,起之問為鴻臚主薄,由是深為義士所譏。



 景龍中,再轉考功員外郎。時中宗增置修文館學士,擇朝中文學之士,之問與薛稷、杜審言等首膺其選,當時榮之。及典舉,引拔後進,多知名者。尋轉越州長史。



 睿宗即位,以之問嘗附張易之、武三思,配徙欽州。先天中,賜死於徙所。之問再被竄謫,經途江、嶺,所有篇詠,傳布遠近。友人武平
 一為之纂集,成十卷,傳於代。



 世人以之問父為三絕,之問以文詞知名,弟之悌有勇力,之遜善書,議者云各得父之一絕。



 之悌,開元中自右羽林將軍出為益州長史、劍南節度兼採訪使。尋遷太原尹。



 閻朝隱,趙州欒城人也。少與兄鏡幾、弟仙舟俱知名。朝隱文章雖無《風》、《雅》之體,善構奇,甚為時人所賞。累遷給事中,預修《三教珠英》。張易之等所作篇什,多是朝隱及宋之問潛代為之。聖歷二年,則天不豫,令朝隱往少室
 山祈禱。朝隱乃曲申悅媚,以身為犧牲,請代上所苦。及將康復,賜絹彩百匹、金銀器十事。俄轉麟臺少監。易之伏誅,坐徙嶺外。尋召還。先天中,復為秘書少監。又坐事貶為通州別駕,卒官。



 朝隱修《三教珠英》時,成均祭酒李嶠與張昌宗為修書使,盡收天下文詞之士為學士,預其列者,有王無競、李適、尹元凱,並知名於時。自餘有事跡者,各見其本傳。



 王無競者,字仲烈。其先瑯邪人,因官徙居東萊,宋太尉
 弘之十一代孫。父侃,棣州司馬。



 無競有文學,初應下筆成章舉及第,解褐授趙州欒城縣尉,歷秘書省正字,轉右武衛倉曹、洛陽縣尉,遷監察御史,轉殿中。舊例,每日更直於殿前。正班時,宰相宗楚客、楊再思常離班偶語,無競前曰:「朝禮至敬,公等大臣,不宜輕易以慢恆典。」楚客等大怒,轉無競為太子舍人。神龍初,坐訶詆權幸,出為蘇州司馬。及張易之等敗,以嘗交往,再貶嶺外,卒於廣州,年五十四。



 李適者,雍州萬年人。景龍中,為中書舍人,俄轉工部侍郎。睿宗時,天臺道士司馬承禎被徵至京師。及還,適贈詩,序其高尚之致,其詞甚美,當時朝廷之士,無不屬和,凡三百餘人。徐彥伯編而敘之,謂之《白雲記》,頗傳於代。尋卒。



 尹元凱者,瀛州樂壽人。初為磁州司倉,坐事免,乃棲遲山林,不求仕進,垂三十年。與張說、盧藏用特相友善,徵拜右補闕。卒於並州司馬。



 賈曾,河南洛陽人也。父言忠,乾封中為侍御史。時朝廷有事遼東,言忠奉使往支軍糧。及還,高宗問以軍事,言忠畫其山川地勢,及陳遼東可平之狀,高宗大悅。又問諸將優劣,言忠曰:「李勣先朝舊臣,聖鑒所悉。龐同善雖非鬥將,而持軍嚴整。薛仁貴勇冠三軍,名可振敵。高侃儉素自處,忠果有謀。契苾何力沉毅持重,有統御之才,然頗有忌前之癖。諸將夙夜小心,忘身憂國,莫過於李勣者。」高宗深然之。累轉吏部員外郎。坐事左遷邵州司
 馬,卒。



 曾少知名。景雲中,為吏部員外郎。玄宗在東宮,盛擇宮僚,拜曾為太子舍人。時太子頻遣使訪召女樂,命宮臣就率更署閱樂,多奏女妓。曾啟諫曰:



 臣聞作樂崇德,以感人神,《韶》、《夏》有容,《咸》、《英》有節,婦人媟黷,無豫其間。昔魯用孔子,幾至於霸,齊人懼之,饋以女樂,魯君既受,孔子所以行。戎有由餘,兵強國富,秦人反間,遺之女妓,戎王耽悅,由余乃奔。斯則大聖名賢,嫉之已久。良以婦人為樂,必務冶容,哇姣動心,蠱惑喪志,上行下效,淫俗
 將成,敗國亂人,實由茲起。



 伏惟殿下神武命代,文思登庸,宇內顒顒,瞻仰德化。而渴賢之美,未被於民心;好妓之聲,或聞於人聽。豈所以追啟、誦之徽烈,襲堯、舜之英風者哉!至若監撫餘閑,宴私多豫,後庭妓樂,古或有之,非以風人,為弊猶隱。至於所司教習,章示群僚,慢伎淫聲,實虧睿化。伏願下教令,發德音,屏倡優,敦《雅》、《頌》,率更女樂,並令禁斷,諸使採召,一切皆停。則朝野內外,皆知殿下放鄭遠佞,輝光日新,凡在含生,孰不欣戴。



 太子手
 令答曰:「比嘗聞公正直,信亦不虛。寡人近日頗尋典籍,至於政化,偏所留心,女樂之徒,亦擬禁斷。公之所言,雅符本意。」俄特授曾中書舍人。曾以父名忠,固辭。乃拜諫議大夫、知制誥。



 明年,有事於南郊,有司立議,唯祭昊天上帝,而不設皇地祇之位。曾奏議:「請於南郊方丘,設皇地祇及從祀等坐,則禮惟稽古,義得緣情。」睿宗令宰相及禮官詳議,竟依曾所奏。開元初,復拜中書舍人,曾又固辭,議者以為中書是曹司名,又與曾父音同字別,於
 禮無嫌,曾乃就職。與蘇晉同掌制誥,皆以詞學見知,時人稱為蘇賈。曾後坐事,貶洋州刺史。開元六年,玄宗念舊,特恩甄敘,繼歷慶、鄭等州刺吏,入拜光祿少卿,遷禮部侍郎。十五年卒。



 子至。至,天寶末為中書舍人。祿山之亂,從上皇幸蜀。時肅宗即位於靈武,上皇遣至為傳位冊文。上皇覽之,嘆曰:「昔先帝遜位於朕,冊文則卿之先父所為。今朕以神器大寶付儲君,卿又當演誥。累朝盛典,出卿父子之手,可謂難矣!」至伏於御前,嗚咽感涕。



 寶
 慶二年,為尚書左丞。時禮部侍郎楊綰上疏,請依古制。縣令舉孝廉於刺史,試其所通之學,送名於省;省試每經問義十條、對策三道,取其通否。詔令左右丞、諸司侍郎、大夫、中丞、給、舍等參議,議者多與綰同。至議曰:



 夏之政尚忠,殷之政尚敬,周之政尚文,然則文與忠、敬,皆統人之行也。是故前代以文取士,本行也;由詞以觀行,則及詞也。宣父稱「顏子不遷怒,不貳過」,謂之「好學」。至乎修《春秋》,則游、夏不能措一辭,不亦明乎!間者,禮部取人,有
 乖斯義。試學者以帖字為精通,而不窮旨義,豈能知「遷怒」、「貳過」之道乎?考文者以聲病為是非,唯擇浮艷,豈能知移風易俗化天下之事乎?是以上失其源,下襲其流,乘流波蕩,不知所止,先王之道,莫能行也。夫先王之道消,則小人之道長;小人之道長,則亂臣賊子由是出焉。臣弒其君,子弒其父,非一朝一夕之故,其所由來者漸矣!漸者何?儒道不舉,取士之失也。夫一國之事,系一人之本,謂之風。贊揚其風,系卿大夫也,卿大夫何嘗不出
 於士乎?今取士,試之小道,不以遠者大者,使干祿之徒,趨馳末術,是誘導之差也。所以祿山一呼,四海震蕩;思明再亂,十年不復。向使禮讓之道弘,仁義之風著,則忠臣孝子,比屋可封,逆節不得而萌也,人心不得而搖也。



 且夏有天下四百載,禹之道喪,而殷始興焉。殷有天下六百祀,湯之法棄,而周始興焉。周有天下八百年,文、武之政弊,而秦始並焉。觀三代之選士任賢,皆考實行,故能風俗淳一,運祚長遠。秦坑儒士,二代而亡。漢興,雜用
 三代之政,弘四科之舉,終彼四百,豈非學行道扇,化行於鄉里哉!自魏至隋,僅四百載,竊號僭位,德義不修,是以子孫速顛,享國咸促。



 國家革魏、晉、隋、梁之弊,承夏、殷、周、漢之業,四隩既宅,九州攸同,覆幬生育,德合天地。安有舍皇王舉士之道,從亂代取人之術!此公卿大夫之辱也。



 今西京有太學,州縣有小學,兵革一動,生徒流離,儒臣師氏,祿廩無由,貢士不稱行實,胄子何嘗講習。禮部每歲擢甲乙之第,謂弘獎勸,不其謬歟!只足以長浮
 薄之風,啟僥幸之路矣!其國子博士等,望加員數,厚其祿秩,通儒碩生,間居其職。十道大郡,量置太學館,令博士出外,兼領郡官,召置生徒,依乎故事,保桑梓者,鄉里舉焉,在流寓者,闍序推焉。朝而行之,夕見其利。



 議者然之。宰臣等奏以舉人舊業已成,難於速改。其今歲舉人,望且依舊。賈至所議,來年允之。



 廣德二年,轉禮部侍郎。是歲,至以時艱歲歉,舉人赴省者,奏請兩都試舉人,自至始也。永泰元年,加集賢院待制。大歷初,改兵部侍郎。
 五年,轉京兆尹、兼御史大夫,卒。



 許景先,常州義興人,後徙家洛陽。少舉進士,授夏陽尉。神龍初,東都起聖善寺報慈閣。景先詣闕獻《大像閣賦》,詞甚美麗,擢拜左拾遺。累遷給事中。開元初,每年賜射,節級賜物,屬年儉,甚費府庫。景先奏曰:



 近臣三九之辰,頻賜宴射,已著格令,猶降綸言。但古制不存,禮章多闕,官員累倍,帑藏未充,水旱相仍,繼之師旅。既不足以觀德,又未足以威邊;耗國損人,且為不急。夫古之天子,以
 射選諸侯,以射飾禮樂,以射觀容志,故有《騶虞》、《貍首》之奏,《採蘩》、《採蘋》之樂。天子則以備官為節,諸侯則以時會為節,卿大夫以循法為節,士以不失職為節,皆審志固行,德美事成,陰陽克和,暴亂不作。故諸侯貢士,亦試於射宮;容體有虧,則絀其地。是諸侯君臣皆盡志於射,射之禮也大矣哉!今則不然。眾官既多,鳴鏑亂下,以茍獲為利,以偶中為能,素無五善之容,頗失三侯之禮。冗官厚秩,禁衛崇班,動盈累千,其算無數。近河南、河北,水澇處
 多,林胡小蕃,見寇郊壘,軍書日至,河朔騷然。命將除兇,未圖克捷;興師十萬,日費千金。去歲豫、亳兩州,微遭旱損,庸賦不辦,以致流亡。聖人憂勤,降使招恤,流離歲月,猶未能安,人之困窮,以至於此。今一箭偶中,是一丁庸調,用之既無惻隱,獲之固無恥慚。考古循今,則為未可。且禁衛武官,隨番許射,能中的者,必有賞焉。此則訓武習戎,時習不闕,待寇寧歲稔,率由舊章,則愛禮養人,幸甚!幸甚!



 自是乃停賜射之禮。



 俄轉中書舍人。自開元初,
 景先與中書舍人齊浣、王丘、韓休、張九齡掌知制誥,以文翰見稱。中書令張說嘗稱曰:「許舍人之文,雖無峻峰激流嶄絕之勢,然屬詞豐美,得中和之氣,亦一時之秀也。」十年夏,伊、汝泛溢,漂損居人廬舍,溺死者甚眾。景先言於侍中源乾曜曰:「災眚所降,必資修德以禳之。《左傳》所載『降服出次』,即其事也。誠宜發德音,遣大臣存問,憂人罪己,以答天譴。明公位存輔弼,當發明大體,以啟沃明主,不可緘默也。」乾曜然其言,遽以聞奏,乃下詔遣戶
 部尚書陸象先往賑給窮乏。



 十三年,玄宗令宰臣擇刺史之任,必在得人,景先首中其選,自吏部侍郎出為虢州刺史。後轉岐州,入拜吏部侍郎,卒。



 賀知章,會稽永興人,太子洗馬德仁之族孫也。少以文詞知名,舉進士。初授國子四門博士,又遷太常博士,皆陸象先在中書引薦也。開元十年,兵部尚書張說為麗正殿修書使,奏請知章及秘書員外監徐堅、監察御史趙冬曦,皆入書院,同撰《六典》及《文纂》等,累年,書竟不就。
 後轉太常少卿。



 十三年,遷禮部侍郎,加集賢院學士,又充皇太子侍讀。是歲,玄宗封東嶽,有詔應行從群臣,並留於谷口,上獨與宰臣及外壇行事官登於嶽上齋宮之所。



 初,上以靈山清潔,不欲喧繁,召知章講定儀注,因奏曰:「昊天上帝君位,五方諸帝臣位,帝號雖同,而君臣異位。陛下享君位於山上,群臣祀臣位於山下,誠足垂範來葉,為變禮之大者也。然禮成於三獻,亞終合於一處。」上曰:「朕正欲如是,故問卿耳。」於是敕:「三獻於山上行
 事,五方帝及諸神座於下壇行事。」



 俄屬惠文太子薨,有詔禮部選挽郎,知章取舍非允,為門廕子弟喧訴盈庭。知章於是以梯登墻,首出決事,時人咸嗤之,由是改授工部侍郎,兼秘書監同正員,依舊充集賢院學士。俄遷太子賓客、銀青光祿大夫兼正授秘書監。



 知章性放曠,善談笑,當時賢達皆傾慕之。工部尚書陸象先,即知章之族姑子也,與知章甚相親善。象先常謂人曰:「賀兄言論倜儻,真可謂風流之士。吾與子弟離闊,都不思之,一
 日不見賀兄,則鄙吝生矣。」



 知章晚年尤加縱誕,無復規檢,自號四明狂客,又稱「秘書外監」,遨游里巷。醉後屬詞,動成卷軸,文不加點,咸有可觀。又善草隸書,好事者供其箋翰,每紙不過數十字,共傳寶之。



 時有吳郡張旭,亦與知章相善。旭善草書,而好酒,每醉後號呼狂走,索筆揮灑,變化無窮,若有神助,時人號為張顛。



 天寶三載,知章因病恍惚,乃上疏請度為道士,求還鄉里,仍舍本鄉宅為觀。上許之,仍拜其子典設郎曾為會稽郡司馬,仍
 令侍養。禦制詩以贈行,皇太子已下咸就執別。至鄉無幾壽終,年八十六。



 肅宗以侍讀之舊,乾元元年十一月詔曰:「故越州千秋觀道士賀知章,器識夷淡,襟懷和雅,神清志逸,學富才雄,挺會稽之美箭,蘊昆崗之良玉。故飛名仙省,侍講龍樓,常靜默以養閑,因談諧而諷諫。以暮齒辭祿,再見款誠,願追二老之蹤,克遂四明之客。允葉初志,脫落朝衣,駕青牛而不還,狎白衣而長往。丹壑非昔,人琴兩亡,惟舊之懷,有深追悼,宜加縟禮,式展哀
 榮。可贈禮部尚書。」



 先是,神龍中,知章與越州賀朝、萬齊融,揚州張若虛、邢巨,湖州包融,俱以吳、越之士,文詞俊秀,名揚於上京。朝萬止山陰尉,齊融昆山令,若虛兗州兵曹,巨監察御史。融遇張九齡,引為懷州司戶、集賢直學士。數子人間往往傳其文,獨知章最貴。



 神龍中,有尉氏李登之,善五言詩,蹉跌不偶,六十餘,為宋州參軍卒。



 席豫,襄陽人,湖州刺史固七世孫,徙家河南。豫進士及第。開元中,累官至考功員外郎,典舉得士,為時所稱。三
 遷中書舍人,與韓休、許景先、徐安貞、孫逖相次掌制誥,皆有能名。轉戶部侍郎,充江南東道巡撫使,兼鄭州刺史。入為吏部侍郎,玄宗謂之曰:「卿以前為考功,職事修舉,故有此授。」豫典選六年,復有令譽。天寶初,改尚書左丞。尋檢校禮部尚書,封襄陽縣子。玄宗幸溫泉宮,登朝元閣賦詩,群臣屬和。帝以豫詩為工,手制褒美曰:「覽卿所進,實詩人之首出,作者之冠冕也。」



 豫與弟晉,俱以詞藻見稱。而豫性尤謹,雖與子弟書疏及吏曹簿領,未嘗草
 書。謂人曰:「不敬他人,是自不敬也。」或曰:「此事甚細,卿何介意?」豫曰:「細猶不謹,而況巨耶!」七載,卒於位,時年六十九。



 疾篤,謂其子曰:「吾亡三日斂,斂日即葬,勿更久留,貽公私之煩。家無餘財,可賣所居,聊備葬禮。」人嘉其達。贈江陵大都督,謚曰文。



 徐安貞者,信安龍丘人。尤善五言詩。嘗應制舉,一歲三擢甲科,人士稱之。開元中,為中書舍人、集賢院學士。上每屬文及作手詔,多命安貞視草,甚承恩顧。累遷中書
 侍郎。天寶初卒。



 齊浣,定州義豐人。少以詞學稱。弱冠以制科登第,釋褐蒲州司法參軍。景雲二年,中書令姚崇用為監察御史。彈劾違犯,先於風教,當時以為稱職。開元中,崇復用為給事中,遷中書舍人。論駁書詔,潤色王言,皆以古義謨誥為準的。侍中宋璟、中書侍郎蘇頲並重之。秘書監馬懷素、右常侍元行沖受詔編次四庫群書,乃奏浣為編修使,改秘書少監。尋丁憂免。



 十二年,出為汴州刺史。河
 南,汴為雄郡,自江、淮達於河、洛,舟車輻輳,人庶浩繁。前後牧守,多不稱職,唯倪若水與浣皆以清嚴為治,民吏歌之。中書令張說擇左右丞之才,舉懷州刺史王丘為左丞,以浣為右丞。李元紘、杜暹為相,以開府、廣平公宋璟為吏部尚書,又用戶部侍郎蘇晉與浣為吏部侍郎,當時以為高選。



 時開府王毛仲寵幸用事,與龍武將軍葛福順為姻親,故北門官見毛仲奏請,無不之允,皆受毛仲之惠,進退隨其指使。浣惡之,乘間論之曰:「福順典
 兵馬,與毛仲婚姻,小人寵極則奸生,若不預圖,恐後為患,惟陛下思之。況腹心之委,何必毛仲,而高力士小心謹慎,又是閹官,便於禁中驅使。臣雖過言,庶裨萬一。臣聞君不密則失臣,臣不密則失身,惟聖慮密之。」玄宗嘉其誠,諭之曰:「卿且出。朕知卿忠義,徐俟其宜。」會大理丞麻察坐事出為興州別駕,浣與察善,出城餞之,因語禁中諫語。察性譐誻,遽以浣語奏之。玄宗怒,令中書門下鞫問。又召浣於內殿,謂之曰:「卿向朕道『君不密則失臣,
 臣不密則失身』,而疑朕不密,而翻告麻察,是何密耶?麻察輕險無行,常游太平之門,此日之事,卿豈不知耶?」浣免冠頓首謝罪,乃貶高州良德丞。又貶察為潯州皇化尉。浣數年量移常州刺史。



 二十五年,遷潤州刺史,充江南東道採訪處置使。潤州北界隔吳江,至瓜步沙尾,紆匯六十里,船繞瓜步,多為風濤之所漂損。浣乃移其漕路,於京口塘下直渡江二十里,又開伊婁河二十五里,即達揚子縣。自是免漂損之災,歲減腳錢數十萬。又立
 伊婁埭,官收其課,迄今利濟焉。數年,復為汴州刺史。淮、汴水運路,自虹縣至臨淮一百五十里,水流迅急,舊用牛曳竹索上下,流急難制。浣乃奏自虹縣下開河三十餘里,入於清河,百餘里出清水,又開河至淮陰縣北岸入淮,免淮流湍險之害。久之,新河水復迅急,又多殭石,漕運難澀,行旅弊之。



 浣因高力士中助,連為兩道採訪使。遂興開漕之利,以中人主意,復勾剝貨財,賂遣中貴,物議薄之。又納劉戒之女為妾,凌其正室,專制家政。李
 林甫惡之,遣人掎摭其失。會浣判官犯贓,浣連坐,遂廢歸田里。



 天寶初,起為員外少詹事,留司東都。時絳州刺史嚴挺之為林甫所構,除員外少詹事,留司東都。與浣皆朝廷舊德,既廢居家巷,每園林行樂,則杖屢相過,談宴終日。林甫聞而患之,欲離其勢。五年,用浣為平陽太守。卒於郡。肅守即位,為林甫所陷者皆得雪,浣受褒贈。



 王浣,並州晉陽人。少豪蕩不羈。登進士第,日以蒱酒為事。並州長史張嘉貞奇其才,禮接甚厚。浣感之,撰樂詞
 以敘情,於席上自唱自舞,神氣豪邁。張說鎮並州,禮浣益至。會說復知政事,以浣為秘書正字,擢拜通事舍人,遷駕部員外。櫪多名馬,家有妓樂。浣發言立意,自比王侯;頤指儕類,人多嫉之。



 說既罷相,出浣為汝州長史,改仙州別駕。至郡,日聚英豪,從禽擊鼓,恣為歡賞,文士祖詠、杜華常在座,於是貶道州司馬,卒。有文集十卷。



 李邕,廣陵江都人。父善,嘗受《文選》於同郡人曹憲。後為左侍極賀蘭敏之所薦引,為崇賢館學士,轉蘭臺郎。敏
 之敗,善坐配流嶺外。會赦還,因寓居汴、鄭之間,以講《文選》為業。年老疾卒。所注《文選》六十卷,大行於時。



 邕少知名。長安初,內史李嶠及監察御史張廷珪,並薦邕詞高行直,堪為諫諍之官,由是召拜左拾遺。俄而御史中丞宋璟奏侍臣張昌宗兄弟有不順之言,請付法推斷。則天初不應,邕在階下進曰:「臣觀宋璟之言,事關社稷,望陛下可其奏。」則天色稍解,始允宋璟所請。



 既出,或謂邕曰:「吾子名位尚卑,若不稱旨,禍將不測,何為造次如是?」
 邕曰:「不願不狂,其名不彰。若不如此,後代何以稱也?」



 及中宗即位,以妖人鄭普思為秘書監,邕上書諫曰:



 蓋人有感一餐之惠,殞七尺之身;況臣為陛下官,受陛下祿,而目有所見,口不言之,是負恩矣!自陛下親政日近,復在九重,所以未聞在外群下竊議。道路籍籍,皆云普思多行詭惑,妄說妖祥。唯陛下不知,尚見驅使。此道若行,必撓亂朝政。臣至愚至賤,不敢以胸臆對揚天威,請以古事為明證。孔丘云:「《詩》三百,一言以蔽之,曰:思無邪。」陛
 下今若以普思有奇術,可致長生久視之道,則爽鳩氏久應得之,永有天下,非陛下今日可得而求。若以普思可致仙方,則秦皇、漢武久應得之,永有天下,亦非陛下今日可得而求。若以普思可致佛法,則漢明、梁武久應得之,永有天下,亦非陛下今日可得而求。若以普思可致鬼道,則墨翟、干寶,各獻於至尊矣,而二主得之,永有天下,亦非陛下今日可得而求!此皆事涉虛妄,歷代無效,臣愚不願陛下復行之於明時。唯堯、舜二帝,自古稱
 聖,臣觀所得,故在人事,敦睦九族,平章百姓,不聞以鬼神之道理天下。伏願陛下察之,則天下幸甚!



 疏奏不納。以與張柬之善,出為南和令,又貶富州司戶。



 唐隆元年,玄宗清內難,召拜左臺殿中侍御史,改戶部員外郎,又貶崖州舍城丞。開元三年,擢為戶部郎中。



 邕素與黃門侍郎張廷珪友善。時姜皎用事,與廷珪謀引邕為憲官。事洩,中書令姚崇嫉邕險躁,因而構成其罪,左遷括州司馬。後徵為陳州刺史。



 十三年,玄宗車駕東封回,邕於
 汴州謁見,累獻詞賦,甚稱上旨。由是頗自矜炫,自云當居相位。張說為中書令,甚惡之。俄而陳州贓污事發,下獄鞫訊,罪當死,許州人孔璋上書救邕曰:



 臣聞明主御宇,舍過舉能,取材棄行;烈士抗節,勇不避死,見危授命。晉用林父,豈念過乎?漢用陳平,豈念行乎?禽息殞身,北郭碎首,豈愛死乎?向若林父誅,陳平死,百里不用,晏嬰見逐,是晉無赤狄之士,漢無皇極之尊,秦不並西戎,齊不霸東海矣!



 臣伏見陳州刺史李邕,學成師範,文堪經
 國;剛毅忠烈,難不茍免。往者張易之用權,人畏其口,而邕折其角;韋氏恃勢,言出禍應,而邕挫其鋒。雖身受謫屈,而奸謀中損,即邕有大造於我邦家也。且斯人所能者,拯孤恤窮,救乏賑惠,積而便散,家無私聚。今聞坐贓下吏,鞫訊待報,將至極刑,死在朝夕。



 臣聞生無益於國,不若殺身以明賢。臣朽賤庸夫,輪轅無取,獸息禽視,雖生何為!況賢為國寶,社稷之衛,是臣痛惜深矣!臣願六尺之軀,甘受膏斧,以代邕死。臣之死,所謂落一毛;邕之
 生,有足照千里。然臣與邕,生平不款,臣知有邕,邕不知有臣。臣不逮邕,明矣!夫知賢而舉,仁也;代人任患,義也。臣獲二善而死。且不朽,則又何求!陛下若以臣之賤不足以贖邕,雁門縫掖有效矣。伏惟陛下寬邕之生,速臣之死。令邕率德改行,想林父之功;使臣得瞑目黃泉,附北郭之跡,臣之大願畢矣!陛下即以陽和之始,難於用鉞,俟天成命,敢忘伏劍,豈煩大刑,然後歸死。皇天后土,實照臣之心。



 昔吳、楚七國叛,因亞夫得劇孟,則寇不足
 憂。夫以一賢之能,敵七國之眾。伏惟敷含垢之道,存棄瑕之義;遠思劇孟,近取李邕,豈惟成愷悌之澤,實亦歸天下之望!況大禮之後,天地更新,赦而復論,人誰無罪?惟明主圖之。臣聞士為知己者死。且臣不為死者所知,甘於死者,豈獨為惜邕之賢,亦成陛下矜能之德。惟明主圖之!



 疏奏,邕已會減死,貶為欽州遵化縣尉,璋亦配流嶺南而死。邕後於嶺南從中官楊思勖討賊有功,又累轉括、淄、滑三州刺史,上計京師。



 邕素負美名,頻被貶
 斥,皆以邕能文養士,賈生、信陵之流,執事忌勝,剝落在外。人間素有聲稱,後進不識,京、洛阡陌聚觀,以為古人。或將眉目有異,衣冠望風,尋訪門巷。又中使臨問,索其新文,復為人陰中,竟不得進。



 天寶初,為汲郡、北海二太守。邕性豪侈,不拘細行,所在縱求財貨,馳獵自恣。五載,奸贓事發。又嘗與左驍衛兵曹柳勣馬一匹,及勣下獄,吉溫令勣引邕議及休咎,厚相賂遺,詞狀連引,敕刑部員外郎祁順之、監察御史羅希奭馳往就郡決殺之,時
 年七十餘。



 初,邕早擅才名,尤長碑頌。雖貶職在外,中朝衣冠及天下寺觀,多齎持金帛,往求其文。前後所制,凡數百首,受納饋遺,亦至鉅萬。時議以為自古鬻文獲財,未有如邕者。有文集七十卷。其《張韓公行狀》、《洪州放生池碑》、《批韋巨源謚議》,文士推重之。後因恩例,得贈秘書監。



 孫逖,潞州涉縣人。曾祖仲將,壽張丞。祖希莊,韓王府典簽。父嘉之,天冊年進士擢第,又以書判拔萃,授蜀州新津主簿。歷曲周、襄邑二縣令,以宋州司馬致仕,卒,年八
 十三。



 逖幼而英俊,文思敏速。始年十五,謁雍州長史崔日用。日用小之,令為《土火爐賦》。逖握翰即成,詞理典贍。日用覽之駭然,遂為忘年之交,以是價譽益重。開元初,應哲人奇士舉,授山陰尉。遷秘書正字。十年,應制登文藻宏麗科,拜左拾遺。張說尤重其才,逖日游其門,轉左補闕。黃門侍郎李暠出鎮太原,闢為從事。



 暠在鎮,與蒲州刺史李尚隱游於伯樂川,逖為之記,文士盛稱之。二十一年,入為考功員外郎、集賢修撰。逖選貢士二年,多
 得俊才。初年則杜鴻漸至宰輔,顏真卿為尚書。後年拔李華、蕭穎士、趙驊登上第,逖謂人曰:「此三人便堪掌綸誥。」



 二十四年,拜逖中書舍人。逖自以通籍禁闈,其父官才邑宰,乃上表陳情曰:「臣父嘉之,雖當暮齒,幸遇明時,綿歷驅馳,才及令長。臣夙荷嚴訓,累登清秩,頻遷省闥,又拜掖垣。地近班榮,臣則過量;途遙日暮,父乃後時。在公府有偷榮之責,於私庭無報德之效,反慚烏鳥,徒廁鴛鴻。伏望降臣一外官,特乞微恩,稍沾臣父。」玄宗優詔
 獎之,授嘉之宋州司馬致仕,尋卒。



 丁父喪免。二十九年服闋,復為中書舍人。其年充河東黜陟使。天寶三載,權判刑部侍郎。五載,以風病求散秩,改太子左庶子。逖掌誥八年,制敕所出,為時流嘆服。議者以為自開元已來,蘇頲、齊浣、蘇晉、賈曾、韓休、許景先及逖,為王言之最。逖尤善思,文理精練,加之謙退不伐,人多稱之。以疾沉廢累年,轉太子詹事。上元中卒。廣德二年,詔贈尚書右僕射,謚曰文。有集三十卷。



 子宿、絳、成。逖弟遹、遘、造。



 遹終左
 武衛兵曹。宿歷河東掌記。代宗朝歷刑部郎中、中書舍人,出為華州刺史,卒。



 成,字退思,以父廕累授雲陽、長安尉,歷監察御史,轉殿中。隴右副元帥李抱玉奏充掌書記,入為屯田、司勛二員外郎。丁母喪免,終制,出為洛陽令,轉長安令。時兄宿為華州刺史,因失火驚懼成喑病。成素孝悌,蒼黃請急,不俟報而趨華。代宗嘉之,嘆曰:「急難之切,觀過知仁。」歷倉部郎中、京兆少尹。出為信州刺史,有惠政,郡人請立碑頌德,優詔褒美。轉蘇州刺史。貞
 元四年,改桂州刺史、桂管觀察使。五年卒。



 宿子公器,官至信州刺史、邕管經略使。



 公器子簡、範,並舉進士。會昌後,兄弟繼居顯秩,歷諸道觀察使。簡,兵部尚書。子紓、徽,並登進士第。



\end{pinyinscope}