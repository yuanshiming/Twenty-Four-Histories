\article{卷二百九}

\begin{pinyinscope}

 ○林
 邑婆利盤盤真臘陀洹訶陵墮和羅墮婆登東謝蠻
 西趙蠻牂牁蠻南平獠東女國南詔蠻驃國



 林邑國,漢日南象林之地,在交州南千餘里。其國延袤數千里,北與皛州接。地氣冬溫,不識冰雪,常多霧雨。其王所居城,立木為柵。王著日氈古貝,斜絡膊,繞腰,上加真珠金鎖,以為瓔珞,卷發而戴花。夫人服朝霞古貝以為短裙,首戴金花,身飾以金鎖真珠瓔珞。王之侍衛,有兵五千人,能用弩及,以藤為甲,以竹為弓,乘象而戰。
 王出則列象千頭,馬四百匹,分為前後。其人拳發色黑,俗皆徒跣,得麝香以塗身,一日之中,再塗再洗。拜謁皆合掌頓顙。嫁娶之法,得取同姓。俗有文字,尤信佛法,人多出家。父母死,子則剔發而哭,以棺盛尸,積柴燔柩,收其灰,藏於金瓶,送之水中。俗以十二月為歲首,稻歲再熟。自此以南,草木冬榮,四時皆食生菜,以檳榔汁為酒。有結遼鳥,能解人語。



 武德六年,其王範梵志遣使來朝。八年,又遣使獻方物。高祖為設《九部樂》以宴之,及賜其
 王錦彩。貞觀初,遣使貢馴犀。四年,其王範頭黎遣使獻火珠,大如雞卵,圓白皎潔,光照數尺,狀如水精,正午向日。以艾蒸之,即火燃。五年,又獻五色鸚鵡。太宗異之,詔太子右庶子李百藥為之賦。又獻白鸚鵡,精識辯慧,善於應答。太宗憫之,並付其使,令放還於林藪。自此朝貢不絕。頭黎死,子範鎮龍代立。太宗崩,詔於陵所刊石圖頭黎之形,列於玄闕之前。十九年,鎮龍為其臣摩訶漫多伽獨所殺,其宗族並誅夷,範氏遂絕。國人乃立頭黎
 之女婿婆羅門為王。後大臣及國人感思舊主。乃廢婆羅門而立頭黎之嫡女為王。



 自林邑以南,皆卷發黑身,通號為「昆侖」。



 婆利國,在林邑東南海中洲上。其地延袤數千里,自交州南渡海,經林邑、扶南、赤土、丹丹數國乃至焉。其人皆黑色,穿耳附榼。王姓剎利耶伽,名護路那婆,世有其位。王戴花形如皮弁,裝以真珠瓔珞,身坐金床。侍女有金花寶縷之飾,或持白拂孔雀扇。行則駕象,鳴金擊鼓吹
 蠡為樂。男子皆拳發,被古貝,布橫幅以繞腰。風氣暑熱,恆如中國之盛夏。穀一歲再熟。有古貝草,緝其花以作布,粗者名古貝,細者名白赩。貞觀四年,其王遣使隨林邑使獻方物。



 盤盤國,在林邑西南海曲中,北與林邑隔小海,自交州船行四十日乃至,其國與狼牙修國為鄰,人皆學婆羅門書,甚敬佛法。貞觀九年,遣使來朝,貢方物。



 貞臘國,在林邑西北,本扶南之屬國,「昆侖」之類。在京師
 南二萬七百里,北至愛州六十日行。其王姓剎利氏。有大城三十餘所,王都伊奢那城,風俗被服與林邑同。地饒瘴癘毒。海中大魚有時半出,望之如山。每五六月中,毒氣流行,即以牛豕祠之,不者則五穀不登。其俗東向開戶,以東為上。有戰象五千頭,尤好者飼以飯肉。與鄰國戰,則象隊在前,於背上以木作樓,上有四人,皆持弓箭。國尚佛道及天神,天神為大,佛道次之。



 武德六年,遣使貢方物。貞觀二年,又與林邑國俱來朝獻。太宗嘉其
 陸海疲勞,錫賚甚厚。南方人謂真臘國為吉蔑國。自神龍以後,真臘分為二半:以南近海多陂澤處,謂之水真臘半;以北多山阜,謂之陸真臘,亦謂之文單國。高宗、則天、玄宗朝,並遣使朝貢。



 水真臘國,其境東西南北約員八百里,東至奔陀浪州,西至墮羅缽底國,南至小海,北即陸真臘。其王所居城號婆羅提拔。國之東界有小城,皆謂之國。其國多象,元和八年,遣李摩那等來朝。



 陀洹國,在林邑西南大海中,東南與墮和羅接,去交趾
 三月餘日行。賓服於墮和羅。其王姓察失利,字婆末婆那。土無蠶桑,以白氈朝霞布為衣。俗皆樓居,謂之「干欄」。貞觀十八年,遣使來朝。二十一年,又遣使獻白鸚鵡及婆律膏,仍請馬及銅鐘,詔並給之。



 訶陵國,在南方海中洲上居,東與婆利、西與墮婆登、北與真臘接,南臨大海。豎木為城,作大屋重閣,以棕櫚皮覆之。王坐其中,悉用象牙為床。食不用匙箸,以手而撮。亦有文字,頗識星歷。俗以椰樹花為酒,其樹生花,長三
 尺餘,大如人膊,割之取汁以成酒,味甘,飲之亦醉。



 貞觀十四年,遣使來朝。大歷三年、四年皆遣使朝貢。元和十年,遣使獻僧祗僮五人、鸚鵡、頻伽鳥並異種名寶。以其使李訶內為果毅,訶內請回授其弟,詔褒而從之。十三年,遣使進僧祗女二人、鸚鵡、玳瑁及生犀等。



 墮和羅國,南與盤盤、北與迦羅舍佛、東與真臘接,西鄰大海。去廣州五月日行。貞觀十二年,其王遣使貢方物。二十三年,又遣使獻象牙、火珠,請賜好馬,詔許之。



 墮婆登國,在林邑南,海行二月,東與訶陵、西與迷黎車接,北界大海。風俗與訶陵略同。其國種稻,每月一熟。亦有文字,書之於貝多葉。其死者,口實以金,又以金釧貫於四肢,然後加以婆律膏及龍腦等香,積柴以燔之。貞觀二十一年,其王遣使獻古貝、象牙、白檀,太宗璽書報之,並賜以雜物。



 東謝蠻,其地在黔州之西數百里,南接守宮獠,西連夷子,北至蠻。土宜五穀,不以牛耕,但為畬田,每歲易。俗無
 文字,刻木為契。散在山洞間,依樹為層巢而居,汲流以飲。皆自營生業,無賦稅之事。謁見貴人,皆執鞭而拜;有功勞者,以牛馬銅鼓賞之。有犯罪者,小事杖罰之,大事殺之,盜物倍還其贓。婚姻之禮,以牛酒為聘。女婦夫家,皆母自送之。女夫慚,逃避經旬乃出。宴聚則擊銅鼓,吹大角,歌舞以為樂。好帶刀劍,未嘗舍離。丈夫衣服,有衫襖大口褲,以綿綢及布為之。右肩上斜束皮帶,裝以螺殼、虎豹猿狖及犬羊之皮,以為外飾。坐皆蹲踞。男女椎
 髻,以緋束之,後垂向下。其首領謝元深,既世為酋長,其部落皆尊畏之。謝氏一族,法不育女,自云高姓不可下嫁故也。



 貞觀三年,元深入朝,冠烏熊皮冠,若今之髦頭,以金銀絡額,身披毛帔,韋皮行滕而著履,中書侍郎顏師古奏言:「昔周武王時,天下太平,遠國歸款,周史乃書其事為《王會篇》。今萬國來朝,至於此輩章服,實可圖寫,今請撰為《王會圖》。」從之。以其地為應州,仍拜元深為刺史,領黔州都督府。又有南謝首領謝強,與西謝鄰,共元
 深俱來朝見,為南壽州刺史。後改為莊州。



 貞元十三年正月,西南蕃大酋長、正議大夫、檢校蠻州長史,繼襲蠻州刺史,資陽郡開國公、賜紫金魚袋宋鼎,左右大首領、朝散大夫、前檢校邛州刺史、賜紫金魚袋謝汕,左右大首領、繼襲攝蠻州巴江縣令、賜紫金魚袋宋萬傳,界首子弟大首領、朝散大夫、牂州錄事參軍謝文經。黔中經略招討觀察使王礎奏:「前件刺史,建中三年一度朝貢,自後更不許隨例入朝。今年懇訴稱州接牂牁,同被聲
 教,獨此排擯,竊自慚恥,謹遣隨牂牁等朝賀。伏乞特賜優諭,兼同牂牁刺史授官。其牂牁兩州,戶口殷盛,人力強大,鄰側諸蕃,悉皆敬憚。請比兩州每年一度朝貢,仍依牂牁輪環差定,並以才干位望為眾推者充。」敕旨曰:「宋鼎等已改官訖,餘依舊。」



 西趙蠻,在東謝之南,其界東至夷子,西至昆明,南至西洱河。山洞阻深,莫知道里。南北十八日行,東西二十三日行。其風俗物產與東謝同。首領趙氏。世為酋長。有戶
 萬餘。貞觀三年,遣使入朝。二十一年,以其地置明州,以首領趙磨為刺史。



 牂牁蠻,首領亦姓謝氏。其地北去兗州一百五十里,東至辰州二千四百里,南至交州一千五百里,西至昆明九百里。無城壁,散為部落而居。土氣鬱熱,多霖雨。稻粟再熟。無徭役,唯征戰之時,乃相屯聚。刻木為契。其法:劫盜者二倍還贓;殺人者出牛馬三十頭,乃得贖死,以納死家。風俗物產,略與東謝同。其首領謝龍羽,大業末據
 其地,勝兵數萬人。



 武德三年,遣使朝貢,授龍羽牂州刺史,封夜郎郡公。貞觀四年十二月,遣使朝貢。開元十年閏五月,大酋長謝元齊死,詔立其嫡孫嘉藝襲其官封。二十五年,大酋長趙君道來朝,且獻方物,大歷中、貞元初,數遣使朝貢。七年二月,授其酋長趙主俗官,以其歲初朝貢不絕,褒之也。自七年至十八年,凡五遣使來。



 元和三年五月敕:「自今以後,委黔南觀察使差本道軍將充押領牂牁、昆明等使。」四年正月,遣使來朝。是月,遣中
 使魏德和領其使,並齎國信物,降璽書賜其王焉。七年、九年、十一年,凡三遣使來。其年十二月,又遣使來賀正。長慶中,亦朝貢不絕。寶歷元年十二月,遣使謝良震來朝。太和五年至會昌二年,凡七遣使來。



 南平獠者,東與智州、南與渝州、西與南州、北與涪州接。部落四千餘戶。土氣多瘴癘,山有毒草及沙虱、蝮蛇。人並樓居,登梯而上。號為「干欄」。男子左衽露發徒跣;婦人橫布兩幅,穿中而貫其首,名為「通裙」。其人美發,為髻鬟垂於後。以
 竹筒如筆,長三四寸,斜貫其耳,貴者亦有珠榼。土多女少男,為婚之法,女氏必先貨求男族,貧者無以嫁女,多賣與富人為婢。俗皆婦人執役。其王姓硃氏,號為劍荔王,遣使內附,以其地隸於渝州。



 東女國,西羌之別種,以西海中復有女國,故稱東女焉。俗以女為王。東與茂州、黨項接,東南與雅州接,界隔羅女蠻及白狼夷。其境東西九日行,南北二十日行。有大小八十餘城。其王所居名康延川,中有弱水南流,用牛
 皮為船以渡。戶四萬餘眾,勝兵萬餘人,散在山谷間。女王號為「賓就」。有女官,曰「高霸」,平議國事。在外官僚,並男夫為之。其王侍女數百人,五日一聽政。女王若死,國中多斂金錢,動至數萬,更於王族求令女二人而立之。大者為王,其次為小王。若大王死,即小王嗣立,或姑死而婦繼,無有篡奪。其所居,皆起重屋,王至九層,國人至六層。其王服青毛綾裙,下領衫,上披青袍,其袖委地。冬則羔裘,飾以紋錦。為小鬟髻,飾之以金。耳垂榼,足履索
 蜺。俗重婦人而輕丈夫。文字同於天竺。以十一月為正。其俗每至十月,令巫者齎楮詣山中,散糟麥於空,大咒呼鳥。俄而有鳥如雞,飛入巫者之懷,因剖腹而視之,每有一穀,來歲必登,若有霜雪,必多災異。其俗信之,名為鳥卜。其居喪,服飾不改,為父母則三年不櫛沐。貴人死者,或剝其皮而藏之,內骨於瓶中,糅以金屑而埋之。國王將葬,其大臣親屬殉死者數十人。



 隋大業中,蜀王秀遣使招之,拒而不受。武德中,女王湯滂氏始遣使貢方物,
 高祖厚資而遣之。還至隴右,會突厥入寇,被掠於虜庭。及頡利平,其使復來入朝。太宗送令反國,並降璽書慰撫之。垂拱二年,其王斂臂遣大臣湯劍左來朝,仍請官號。則天冊拜斂臂為左玉鈐衛員外將軍,仍以瑞錦制蕃服以賜之。



 天授三年,其王俄琰兒來朝。萬歲通天元年,遣使來朝。開元二十九年十二月,其王趙曳夫遣子獻方物。天寶元年,命有司宴於曲江,令宰臣已下同宴。又封曳夫為歸昌王,授左金吾衛大將軍,賜其子帛八
 十匹,放還。後復以男子為王。



 貞元九年七月,其王湯立悉與哥鄰國王董臥庭、白狗國王羅陀忽、逋租國王弟鄧吉知、南水國王侄薛尚悉曩、弱水國王董闢和、悉董國王湯息贊、清遠國王蘇唐磨、咄霸國王董藐蓬,各率其種落詣劍南西川內附。其哥鄰國等,皆散居山川。弱水王即國初女國之弱水部落。其悉董國,在弱水西,故亦謂之弱水西悉董王。舊皆分隸邊郡,祖、父例授將軍、中郎、果毅等官,自中原多故,皆為吐蕃所役屬。其部落,
 大者不過三二千戶,各置縣令十數人理之。土有絲絮,歲輸於吐蕃。至是悉與之同盟,相率獻款,兼齎天寶中國家所賜官誥共三十九通以進。西川節度使韋皋處其眾於維、霸、保等州,給以種糧耕牛,咸樂生業。立悉等數國王自來朝,召見於麟德殿。授立悉銀青光祿大夫、歸化州刺史;鄧吉知試太府少卿兼丹州長史;薛尚悉曩試少府少監兼霸州長史;董臥庭行至綿州卒,贈武德州刺史,命其子利囉為保守都督府長史,襲哥鄰王。
 立悉妹乞悉漫頗有才智,從其兄來朝,封和義郡夫人。其大首領董臥卿等,皆授以官。俄又授女國王兄湯厥銀青光祿大夫、試太府卿;清遠王弟蘇歷顛銀青光祿大夫、試衛尉卿;南水國王薛莫庭及湯息贊、董藐蓬,女國唱後湯拂庭、美玉缽、南郎唐,並授銀青光祿大夫、試太僕卿。



 其年,西山松州生羌等二萬餘戶,相繼內附。其粘信部落主董夢蔥,龍諾部落主董闢忽,皆授試衛尉卿。立悉等並赴明年元會訖,錫以金帛,各遣還。尋詔加韋
 皋統押近界羌、蠻及西山八國使。其部落代襲刺史等官,然亦潛通吐蕃,故謂之「兩面羌」。



 南詔蠻,本烏蠻之別種也,姓蒙氏。蠻謂王為「詔。」自言哀牢之後,代居蒙舍州為渠帥,在漢永昌故郡東,姚州之西。其先渠帥有六,自號「六詔」,兵力相埒,各有君長,無統帥。蜀時為諸葛亮所征,皆臣服之。國初有蒙舍龍,生迦獨龐。迦獨生細奴邏,高宗時來朝。細奴邏生邏盛,武后時來朝。其妻方娠,邏盛次姚州,聞妻生子,曰:「吾且有子,
 死於唐地足矣。」子名曰盛邏皮。邏盛至京師,賜錦袍金帶歸國。



 開元初,邏盛死,子盛邏皮立。盛邏皮死,子皮邏閣立。二十六年,詔授特進,封越國公,賜名曰歸義。其後破洱河蠻,以功策授雲南王。歸義漸強盛,餘五詔浸弱。先是,劍南節度使王昱受歸義賂,奏六詔合為一詔。歸義既並五詔,服群蠻,破吐蕃之眾,兵日以驕大。每入覲,朝廷亦加禮異。



 二十七年,徙居大和城。天寶四載,歸義遣孫鳳迦異來朝,授鴻臚卿。歸國,恩賜甚厚,歸義意望
 亦高。時劍南節度使章仇兼瓊遣使至雲南,與歸義言語不相得,歸義常銜之。



 七年,歸義卒,詔立子閣羅鳳襲雲南王。無何,鮮於仲通為劍南節度使,張虔陀為雲南太守。仲通褊急寡謀,虔陀矯詐,待之不以禮。舊事,南詔常與其妻子謁見都督,虔陀皆私之。有所徵求,閣羅鳳多不應,虔陀遣人罵辱之,仍密奏其罪惡。閣羅鳳忿怨,因發兵反攻,圍虔陀,殺之,時天寶九年也。



 明年,仲通率兵出戎、巂州。閣羅鳳遣使謝罪,仍與雲南錄事參軍姜
 如芝俱來,請還其所虜掠,且言:「吐蕃大兵壓境,若不許,當歸命吐蕃,雲南之地,非唐所有也。」仲通不許,囚其使,進兵逼大和城,為南詔所敗。自是閣羅鳳北臣吐蕃。吐蕃令閣羅鳳為贊普鐘,號曰東帝,給以金印。蠻謂弟為「鐘」,時天寶十一年也。十二年,劍南節度使楊國忠執國政,仍奏徵天下兵,俾留後、侍御史李宓將十餘萬,輦餉者在外。涉海,瘴死者相屬於路,天下始騷然苦之。宓復敗於大和城北,死者十八、九。會安祿山反,閣羅鳳乘釁
 攻陷巂州及會同軍,西復降尋傳蠻。



 大歷十四年,閣羅鳳子鳳迦異先閣羅鳳死,立迦異子,是為異牟尋。頗知書,有才智,善撫其眾。吐蕃役賦南蠻重數,又奪諸蠻險地立城堡,歲徵兵以助鎮防,牟尋益厭苦之。有鄭回者,本相州人,天寶中舉明經,授巂州西瀘縣令。巂州陷,為所虜。閣羅鳳以回有儒學,更名曰蠻利。甚愛重之,命教鳳迦異。及異牟尋立,又命教其子尋夢湊。回久為蠻師,凡授學,雖牟尋、夢湊,回得箠撻,故牟尋以下皆嚴憚
 之。蠻謂相為清平官,凡置六人。牟尋以回為清平官,事皆咨之,秉政用事。餘清平官五人,事回卑謹,或有過,回輒撻之。回嘗言於牟尋曰:「自昔南詔嘗款附中國,中國尚禮義,以惠養為務,無所求取。今棄蕃歸唐,無遠戍之勞、重稅之困,利莫大焉。」牟尋善其言,謀內附者十餘年矣。會劍南西川節度使韋皋招撫諸蠻,苴烏星、虜望等歸化,微聞牟尋之意,因令蠻寓書於牟尋,且招懷之,時貞元四年也。



 七年,又遣間使持書喻之。道出磨些蠻,其魁
 主潛告吐蕃,使至雲南。吐蕃已知之,令詰牟尋。牟尋懼,因紿吐蕃曰:「唐使,本蠻也,韋皋許其求歸,無他謀。」遂執送吐蕃。吐蕃益疑之,多召南詔大臣之子為質,牟尋愈怨。



 九年四月,牟尋乃與酋長定計遣使:趙莫羅眉由兩川,楊大和堅由黔中,或由安南。使凡三輩,致書與韋皋,各齎生金丹砂為贄。三分前皋所與牟尋書,各持其一為信。歲中,三使皆至京師,且曰:「牟尋請歸大國,永為籓國。所獻生金,以喻向北之意如金也;丹砂,示其赤心耳。」
 上嘉之,乃賜牟尋詔書,因命韋皋遣使以觀其情。皋遂命巡官崔佐時至牟尋所都陽苴咩城,南去太和城十餘里,東北至成都二千四百里,東至安南如至成都,通水陸行。是時也,吐蕃使數百人,先佐時在南詔。牟尋悉召諸種落與議歸化,或未畢至,未敢公言,密令佐時稱牂牁使,衣以牂牁服而入。佐時不肯,曰:「我大唐使,安得服小夷之服。」牟尋不得已,乃夜迎佐時,設位陳燈燭。佐時乃大宣詔書。牟尋恐吐蕃知,顧左右無色,而業已歸
 唐,久之,歔欷流涕,皆俯伏受命。



 其明年正月,異牟尋使其子閣勸及清平官等與佐時,盟於點蒼山神祠。盟書一藏於神室,一沉於西洱河,一置祖廟,一以進天子。閣勸即尋夢湊也。鄭回見佐時,多所指導,故佐時探得其情。乃請牟尋斬吐蕃使數人,以示歸唐。又得其吐蕃所與金印。牟尋尋遣佐時歸,仍刻金契以獻。閣勸賦詩以餞之。牟尋乃去吐蕃所立帝號,私於佐時,請復南詔舊名。佐時與盟訖,留二旬有六日而歸。



 初,吐蕃因爭北庭,
 與回鶻大戰,死傷頗眾。乃徵兵於牟尋,須萬人。牟尋既定計歸我,欲因徵兵以襲之。乃示寡弱,謂吐蕃曰:「蠻軍素少,僅可發三千人。」吐蕃少之,請益至五千,乃許。牟尋遽遣兵五千人戍吐蕃,乃自將數萬踵其後,晝夜兼行,乘其無備,大破吐蕃於神川。遂斷鐵橋,遣使告捷。且請韋皋使閱其所虜獲及城堡,以取信焉。時韋皋上言:「牟尋收鐵橋已來城壘一十六,擒其王五人,降其眾十餘萬。」以祠部郎中兼御史中丞袁滋持節冊南詔,仍賜牟
 尋印,鑄用黃金,以銀為窠。文曰:「貞元冊南詔印。」先是,韋皋奏南詔前遣清平官尹仇寬獻所受吐蕃印五,二用黃金,今賜請以黃金,從蠻夷所重,傳示無窮。從皋之請也。



 十年八月,遣使蒙湊羅棟及尹仇寬來獻鐸槊、浪人劍及吐蕃印八紐。湊羅棟,牟尋之弟也,錫賚甚厚,以尹仇寬為檢校左散騎常侍,餘各授官有差。俄又封尹仇寬為高溪郡王。十一年三月,遣清平官尹輔酋隨袁滋來朝。又得先沒蕃將衛景升、韓演等,並南詔所獲吐蕃
 將帥俘馘百人至京師。湊羅棟歸國,在道而卒,贈右散騎常侍。授尹輔酋檢校太子詹事兼御史中丞,餘亦差次授官。又降敕書賜異牟尋及子閣勸,清平官鄭回、尹仇寬等各一書,書左列中書三官宣奉行,復舊制也。九月,異牟尋遣使獻馬六十匹。



 十二年,韋皋於雅州會野路招收得投降蠻首領高萬唐等六十九人,戶約七千,兼萬唐等先受吐蕃金字告身五十片。十四年,異牟尋遣酋望大將軍王丘各等賀正,兼獻方物。十九年正月
 旦,上御含元殿受南詔朝賀。以其使楊鏌龍武為試太僕少卿,授黎州廓清道蠻首領襲恭化郡王劉志寧試太常卿。二十年,南詔遣使朝貢。



 元和二年八月,遣使鄧傍傳來朝,授試殿中監。三年十二月,以異牟尋卒,廢朝三日。四年正月,以太常少卿武少儀充吊祭使,仍冊牟尋之子驃信苴蒙閣勸為南詔王,仍命鑄「元和冊南詔印」。七年十月,皆遣使朝貢。



 十一年五月,以龍蒙盛卒,廢朝三日。遣使來請冊立其君長。以少府少監李銑充冊立
 吊祭使,左贊善大夫許堯佐副之。十二年至十五年,比年遣使來朝,或年內二三至者。



 寶歷三年,大和元年,亦遣使來。三年,杜元穎鎮西川,以文儒自高,不練戎事。南蠻乘我無備,大舉諸部入寇。牧守屢陳,亦不之信。十一月,蜀川出軍與戰,不利。陷我邛州,逼成都府,入梓州西郭,驅劫玉帛子女而去。上聞之,大怒,再貶元穎為循州司馬。



 明年正月,其王蒙嵯顛以表自陳請罪,兼疏元穎過失。國家方事柔遠,尋釋其罪,復遣使來朝。五年、八年,
 亦遣使來貢方物。開成四年、五年,會昌二年,皆遣使來朝。



 驃國,在永昌故郡南二千餘里,去上都一萬四千里。其國境,東西三千里,南北三千五百里。東鄰真臘國,西接東天竺國,南盡溟海,北通南詔些樂城界,東北拒陽苴咩城六千八百里。往來通聘迦羅婆提等二十國,役屬者道林王等九城,食境土者羅君潛等二百九十部落。



 其王姓困沒長,名摩羅惹。其國相名摩訶思那。其王近
 適則舁以金繩床,遠適則乘象。嬪妹甚眾,常數百人。其羅城構以專甃,周一百六十里,濠岸亦構專,相傳本是舍利佛城。城內有居人數萬家,佛寺百餘區。其堂宇皆錯以金銀,塗以丹彩,地以紫鑛,覆以錦罽。其俗好生惡殺。其土宜菽粟稻梁,無麻麥。其理無刑名桎梏之具,犯罪者以竹五十本束之,復犯者撻其背,數止五,輕者止三,殺人者戮之。男女七歲則落發,止寺舍,依桑門,至二十不悟佛理,乃復長發為居人。其衣服悉以白赩為朝
 霞,繞腰而已。不衣繒帛,雲出於蠶,為其傷生故也。君臣父子長幼有序。華言謂之驃,自謂突羅成闍婆,人謂之徒裡掘。



 古未嘗通中國。貞元中,其王聞南詔異牟尋歸附,心慕之。八年,乃遣其弟悉利移因南詔重譯來朝,又獻其國樂凡十曲,與樂工三十五人俱。樂曲皆演釋氏經論之詞意。尋以悉利移為試太僕卿。



 史臣曰:禹畫九州,周分六服,斷長補短,止方七千。國賦之所均,王教之所備,此謂華夏者也。以圓蓋方輿之廣,
 廣穀大川之多,民生其間,胡可勝道,此謂蕃國者也。西南之蠻夷不少矣,雖言語不通,嗜欲不同,亦能候律瞻風,遠修職貢。但患己之不德,不患人之不來。何以驗之?貞觀、開元之盛,來朝者多也!



 贊曰:五方異氣,所稟不同。維南極海,曰蠻與戎。惡我則叛,好我則通。不可不德,使其瞻風。



\end{pinyinscope}