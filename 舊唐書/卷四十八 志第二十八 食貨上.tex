\article{卷四十八 志第二十八 食貨上}

\begin{pinyinscope}

 先
 王之制,度地以居人,均其沃瘠,差其貢賦,蓋斂之必以道也。量入而為出,節用而愛人,度財省費,蓋用之必有度也,是故既庶且富,而教化行焉。周有井田之制,秦有阡陌之法,二世發閭左而海
 內崩離,漢武稅舟車而國用以竭。自古有國有家,興亡盛衰,未嘗不由此也。隋文帝因周氏平齊之後,府庫充實,庶事節儉,未嘗虛費。開皇之初,議者以比漢代文、景,有粟陳貫朽之積。煬帝即位,大縱奢靡,加以東西行幸,輿駕不息,征討四夷,兵車屢動。西失律於沙徼,東喪師於遼、碣,數年之間,公私罄竭,財力既殫,國遂亡矣。



 高祖發跡太原,因晉陽宮留守庫物,以供軍用。既平京城,先封府庫,賞賜給用,皆有節制,徵斂賦役,務在寬簡。未及逾年,遂成帝業。其後掌財賦者,世有人焉。開元已前,事歸尚書省,開元已後,權移他官。由是有轉運使、租庸使、鹽鐵使、度支鹽鐵轉運使、常平鑄錢鹽鐵使、租庸青苗使、水陸運鹽鐵租庸使、兩稅使,隨事立名,沿革不一。設
 官分職,選賢任能,得其人則有益於國家,非其才則貽患於黎庶,此又不可不知也。如裴耀卿、劉晏、李巽數君子,便時利物,富國安民,足為世法者也。



 開元中,有御史宇文融獻策,括籍外剩田:色役偽濫,及逃戶許歸首,免五年征賦。每丁量稅一千五百錢,置攝御史,分路檢括隱審。得戶八十餘萬,田亦稱是,得錢數百萬貫,玄宗以為能,數年間拔為御
 史中丞、戶部侍郎。融又畫策開河北王莽河,溉田數千頃,以營稻田,事未果而融敗。時又楊崇禮為太府卿,清嚴善勾剝,分寸錙銖,躬親不厭。轉輸納欠,折估漬損,必令徵送。天下州縣徵財帛,四時不止。及老病致仕,以其子慎矜為御史,專知太府出納。其弟慎名又專知京倉,皆以苛刻害人,承主恩而征責。又有韋堅,規宇文融、楊慎矜之跡,乃請於江淮轉運租米,取州縣義倉粟,轉市輕貨,差富戶押船,若遲留損壞,皆征船戶。關中漕渠,鑿廣運潭以挽山東之粟,歲四百萬石,帝以為能,又至貴盛。又王鉷進計,奮身自為戶口色役使,徵剝財貨,每歲進錢百億,寶貨稱是。雲非正額租庸,便入百寶大盈庫,以供人主宴私賞賜之用。玄宗日益眷之,數年間亦為御史大夫、京兆尹、帶二十餘使。又楊國忠藉椒房之勢,承恩幸,帶四十
 餘使,云經其聽覽,必數倍弘益,又見寵貴。太平既久,天下至安,人不願亂。而此數人,設詭計以侵擾之,凡二十五人,同為剝喪,而人無敢言之者。及安祿山反於範陽,兩京倉庫盈溢而不可名。楊國忠設計,稱不可耗正庫之物,乃使御史崔眾於河東納錢度僧、尼、道士,旬日間行錢百萬。玄
 宗幸巴蜀,鄭昉使劍南,請於江陵稅鹽麻以資國,官置吏以督之。肅宗建號於靈武,
 後用雲間鄭叔清為御史,於江淮間豪族富商率貸及賣官爵,以裨國用。德宗朝討河朔及
 李希烈,物力耗竭。趙贊司國計,纖瑣刻剝,以為國用不足,宜賦取於下,以資軍蓄。與諫官陳京等更陳計
 策,贊請稅京師居人屋宅,據其間架差等計入。陳京又請籍列肆商賈資產,以分數借之。宰相同為欺罔,遂行其計。中外沸騰,人懷怨望。時又配王公已下及嘗在方鎮之家出家僮及
 馬以助征行,公私囂然矣。後又張滂、裴延齡、王涯等,剝下媚上,此皆足為世戒者也。



 先是興元克復京師後,府藏盡
 虛,諸道初有進奉,以資經費,復時有宣索。其後諸賊既平,朝廷無事,常賦之外,進奉不息。韋皋劍南有日進,李兼江西有月進。杜亞揚州、劉贊宣州、王緯李錡浙西,皆競為進奉,以固恩澤。貢入之奏,皆白臣於正稅外方圓,亦曰「羨餘」。節度使或托言密旨,乘此盜貿官物。諸道有謫罰官吏入其財者,刻祿廩,通津達道者稅之,蒔蔬藝果者稅之,死亡者稅之。節度觀察交代,或先期稅入以為進奉。然十獻其二三耳,其餘沒入,不可勝紀。此節度
 使進奉也。其後裴肅為常州刺史,乃鬻貨薪炭案牘,百賈之上,皆規利焉。歲餘又進奉。無幾,遷浙東觀察使。天下刺史進奉,自肅始也。劉贊死於宣州,嚴綬為判官,傾軍府資用進奉。無幾,拜刑部員外郎。天下判官進奉,自綬始也。習以為常,流宕忘返。



 大抵有唐之御天下也,有兩稅焉,有鹽鐵焉,有漕運焉,有倉廩焉,有雜稅焉。今考其本末,敘其否臧,以為《食貨志》云。



 武德七年,始定律令。以度田之制:五尺為步,步二百四十為畝,畝百為頃。丁
 男、中男給一頃,篤疾、廢疾給四十畝,寡妻妾三十畝。若為戶者加二十畝。所授之田,十分之二為世業,八為口分。世業之田,身死則承戶者便授之;口分,則收入官,更以給人。賦役之法:每丁歲入租粟二石。調則隨鄉土所產,綾、絹、絁各二丈,布加五分之一。輸綾、絹、絁者,兼調綿三兩;輸布者,麻三斤。凡丁,歲役二旬。若不役,則收其傭,每日三尺。有事而加役者,旬有五日免其調,三旬則租調俱免。通正役,並不過五十日。若嶺南諸州則稅米,上
 戶一石二斗,次戶八斗,下戶六斗。若夷獠之戶,皆從半輸。蕃胡內附者,上戶丁稅錢十文,次戶五文,下戶免之。附經二年者,上戶丁輸羊二口,次戶一口,下,三戶共一口。凡水旱蟲霜為災,十分損四已上免租,損六已上免調,損七已上課役俱免。



 凡天下人戶,量其資產,定為九等。每三年,縣司注定,州司覆之。百戶為里,五里為鄉。四家為鄰,五家為保。在邑居者為坊,在田野者為村。村坊鄰里,遞相督察。士農工商,四人各業。食祿之家,不得與
 下人爭利。工商雜類,不得預於士伍。男女始生者為黃,四歲為小,十六為中,二十一為丁,六十為老。每歲一造計帳,三年一造戶籍。州縣留五比,尚書省留三比。神龍元年,韋庶人為皇后,務欲求媚於人,上表請以二十二為丁,五十八為老,制從之。及韋氏誅,復舊。至天寶三年,又降優制,以十八為中男,二十二為丁。天下籍始造四本,京師及東京尚書省、戶部各貯一本,以備車駕行幸,省於載運之費焉。



 凡權衡度量之制:度,以北方秬黍中
 者一黍之廣為分,十分為寸,十寸為尺,十尺為丈。量,以秬黍中者容一千二百為龠,二龠為合,十合為升,十升為斗;三升為大升,三斗為大斗,十大斗為斛。權衡:以秬黍中者百黍之重為銖,二十四銖為兩,三兩為大兩,十六兩為斤。調鐘律,測晷景,合湯藥及冠冕,制用小升小兩,自餘公私用大升大兩。又山東諸州,以一尺二寸為大尺,人間行用之。其量制,公私又不用龠,合內之分,則有抄撮之細。



 天寶九載二月,敕:「車軸長七尺二寸,面三
 斤四兩,鹽斗,量除陌錢每貫二十文。」先是,開元八年正月,敕:「頃者以庸調無憑,好惡須準,故遣作樣以頒諸州,令其好不得過精,惡不得至濫,任土作貢,防源斯在。而諸州送物,作巧生端,茍欲副於斤兩,遂則加其丈尺,至有五丈為疋者,理甚不然。闊一尺八寸,長四丈,同文共軌,其事久行,立樣之時,亦載此數。若求兩而加尺,甚暮四而朝三。宜令所司簡閱,有逾於比年常例,丈尺過多,奏聞。」



 二十二年五月,敕:「定戶口之時,百姓非商戶郭外
 居宅及每丁一牛,不得將入貨財數。其雜匠及幕士並諸色同類,有蕃役合免征行者,一戶之內,四丁已上,任此色役不得過兩人,三丁已上,不得過一人。」其年七月十八日,敕:「自今已後,京兆府關內諸州,應徵庸調及資課,並限十月三十日畢。」至天寶三載二月二十五日赦文:「每載庸調八月征,以農功未畢,恐難濟辦。自今已後,延至九月三十日為限。」二十五年三月,敕:「關輔庸調,所稅非少,既寡蠶桑,皆資菽粟,常賤糶貴買,損費逾深。又
 江淮等苦變造之勞,河路增轉輸之弊,每計其運腳,數倍加錢。今歲屬和平,庶物穰賤,南畝有十千之獲,京師同水火之饒,均其餘以減遠費,順其便使農無傷。自今已後,關內諸州庸調資課,並宜準時價變粟取米,送至京逐要支用。其路遠處不可運送者,宜所在收貯,便充隨近軍糧。其河南、河北有不通水利,宜折租造絹,以代關中調課。所司仍明為條件,稱朕意焉。」



 天寶元年正月一日赦文:如聞百姓之內,有戶高丁多,茍為規避,父母
 見在,乃別籍異居。宜令州縣勘會。其一家之中,有十丁已上者,放兩丁征行賦役。五丁已上,放一丁。即令同籍共居,以敦風教。其侍丁孝假,免差科。」廣德元年七月,詔:「一戶之中,三丁放一丁庸調。地稅依舊每畝稅二升。天下男子,宜二十三成丁,五十八為老。」永泰元年五月,京兆麥大稔,京兆尹第五琦奏請每十畝官稅一畝,效古什一之稅。從之。二年五月,諸道稅地錢使、殿中侍御史韋光裔等自諸道使還,得錢四百九十萬貫。乾元以來,
 屬天下用兵,京師百僚俸錢減耗。上即位,推恩庶僚,下議公卿。或以稅畝有苗者,公私咸濟。乃分遣憲官,稅天下地青苗錢,以充百司課料。至是,仍以御史大夫為稅地錢物使,歲以為常,均給百官。



 大歷四年正月十八日,敕有司:「定天下百姓及王公已下每年稅錢,分為九等:上上戶四千文,上中戶三千五百文,上下戶三千文。中上戶二千五百文,中中戶二千文,中下戶一千五百文。下上戶一千文,下中戶七百文,下下戶五百文。其見官,
 一品準上上戶,九品準下下戶,餘品並準依此戶等稅。若一戶數處任官,亦每處依品納稅。其內外官,仍據正員及占額內闕者稅。其試及同正員文武官,不在稅限。其百姓有邸店行鋪及爐冶,應準式合加本戶二等稅者,依此稅數勘責徵納。其寄莊戶,準舊例從八等戶稅,寄住戶從九等戶稅,比類百姓,事恐不均,宜各遞加一等稅。其諸色浮客及權時寄住戶等,無問有官無官,各所在為兩等收稅。稍殷有者準八等戶,餘準九等戶。如數
 處有莊田,亦每處稅。諸道將士莊田,既緣防御勤勞,不可同百姓例,並一切從九等輸稅。」其年十二月,敕:「今關輔墾田漸廣,江淮轉漕常加,計一年之儲,有太半之助,其於稅地,固可從輕。其京兆來秋稅,宜分作兩等,上下各半,上等每畝稅一斗,下等每畝稅六升。其荒田如能佃者,宜準今年十月二十九日敕,一切每畝稅二升。仍委京兆尹及令長一一存撫,令知朕意。」五年三月,優詔定京兆府百姓稅。夏稅,上田畝稅六升,下田畝稅四升。
 秋稅,上田畝稅五升,下田畝稅三升。荒田開佃者,畝率二升。八年正月二十五日,敕:「青苗地頭錢,天下每畝率十五文。以京師煩劇,先加至三十文,自今已後,宜準諸州,每畝十五文。」



 建中元年二月,遣黜陟使分行天下,其詔略曰:「戶無主客,以見居為簿。人無丁中,以貧富為差。行商者,在郡縣稅三十之一。居人之稅,秋夏兩征之。各有不便者,三之。餘徵賦悉罷,而丁額不廢。其田畝之稅,率以大歷十四年墾數為準。征夏稅無過六月。秋稅無
 過十一月。違者進退長吏。令黜陟使各量風土所宜、人戶多少均之,定其賦,尚書度支總統焉。」三年五月,淮南節度使陳少游請於本道兩稅錢每千增二百,因詔他州悉如之。八年四月,劍南西川觀察使韋皋奏請加稅什二,以增給官吏,從之。



 元和十五年八月,中書門下奏:「伏準今年閏正月十七日敕,令百僚議錢貨輕重者,今據群官楊於陵等議,『伏請天下兩稅榷鹽酒利等,悉以布帛絲綿,任土所產物充稅,並不徵見錢,則物漸重,錢
 漸輕,農人見免賤賣匹帛』者。伏以群臣所議,事皆至當,深利公私。請商量付度支,據諸州府應徵兩稅,供上都及留州留使舊額。起元和十六年已後,並改配端匹斤兩之物為稅額,如大歷已前租庸課調,不計錢,令其折納。使人知定制,供辦有常。仍約元和十五年徵納布帛等估價。其舊納虛估物,與依虛估物回計,如舊納實估物並見錢,即當於端匹斤兩上量加估價回計。變法在長其物價,價長則永利公私。初雖微有加饒,法行即當就
 實。比舊給用,固利而不害。仍作條件處置,編入旨符。其鹽利酒利,本以榷率計錢,有殊兩稅之名,不可除去錢額。中有令納見錢者,亦請令折納時估匹段。上既不專以錢為稅,人得以所產輸官,錢貨必均其重輕,隴畝自廣於蠶織。便時惠下,庶得其宜。其土乏絲麻,或地連邊塞,風俗更異,賦入不同,亦請商量,委所司裁酌,隨便宜處置。」詔從之。大和四年五月,劍南西川宣撫使、諫議大夫崔戎奏:「準詔旨制置西川事條。今與郭釗商量,兩稅
 錢數內三分,二分納見錢,一分折納匹段,每二貫加饒百姓五百文,計一十三萬四千二百四十三貫文。依此曉諭百姓訖。經賊州縣,準詔三分減放一分,計減錢六萬七千六百二十貫文。不經賊處,先徵見錢,今三分一分折納雜物,計優饒百姓一十三萬貫。舊有稅姜芋之類,每畝至七八百。徵斂不時,今並省稅名,盡依諸處為四限等第,先給戶帖,餘一切名目勒停。」



 高祖即位,仍用隋之五銖錢。武德四年七月,廢五銖錢,行開元通寶錢,
 徑八分,重二銖四絫,積十文重一兩。一千文重六斤四兩。仍置錢監於洛、並、幽、益等州。秦王、齊王各賜三爐鑄錢,右僕射裴寂賜一爐。敢有盜鑄者身死,家口配沒。五年五月,又於桂州置監。議者以新錢輕重大小最為折衷,遠近甚便之。後盜鑄漸起,而所在用錢濫惡。顯慶五年九月,敕以惡錢轉多,令所在官私為市取,以五惡錢酬一好錢。百姓以惡錢價賤,私自藏之,以候官禁之弛。高宗又令以好錢一文買惡錢兩文,弊仍不息。至乾封
 元年封嶽之後,又改造新錢,文曰「乾封泉寶」,徑一寸,重二銖六分,仍與舊錢並行。新錢一文當舊錢之十。周年之後,舊錢並廢。



 初,開元錢之文,給事中歐陽詢制詞及書,時稱其工。其字含八分及隸體,其詞先上後下,次左後右讀之。自上及左回環讀之,其義亦通。流俗謂之開通元寶錢。及鑄新錢,乃同流俗,「乾」字直上,「封」字在左。尋寤錢文之誤,又緣改鑄,商賈不通,米帛增價,乃議卻用舊錢。二年正月,下詔曰:「泉布之興,其來自久。實古今之
 要重,為公私之寶用。年月既深,偽濫斯起,所以採乾封之號,改鑄新錢。靜而思之,將為未可。高祖撥亂反正,爰創軌模。太宗立極承天,無所改作。今廢舊造新,恐乖先旨。其開元通寶,宜依舊施行,為萬代之法。乾封新鑄之錢,令所司貯納,更不須鑄。仍令天下置爐之處,並鑄開元通寶錢。」既而私鑄更多,錢復濫惡。



 高宗嘗臨軒謂侍臣曰:「錢之為用,行之已久,公私要便,莫甚於斯。比為州縣不存檢校,私鑄過多。如聞荊、潭、宣、衡,犯法尤甚。遂有
 將船筏宿於江中,所部官人不能覺察。自今嚴加禁斷,所在追納惡錢,一二年間使盡。」當時雖有約敕,而奸濫不息。儀鳳四年四月,令東都出遠年糙米及粟,就市給糶,斗別納惡錢百文。其惡錢令少府司農相知,即令鑄破。其厚重徑合斤兩者,任將行用,時米粟漸貴,議者以為鑄錢漸多,所以錢賤而物貴。於是權停少府監鑄錢,尋而復舊。則天長安中,又令懸樣於市,令百姓依樣用錢。俄又簡擇艱難,交易留滯。又降敕非鐵錫、銅蕩、穿穴
 者,並許行用。其有熟銅、排斗、沙澀、厚大者,皆不許簡。自是盜鑄蜂起,濫惡益眾。江淮之南,盜鑄者或就陂湖、巨海、深山之中,波濤險峻,人跡罕到,州縣莫能禁約。以至神龍、先天之際,兩京用錢尤濫。其郴、衡私鑄小錢,才有輪郭,及鐵錫五銖之屬,亦堪行用。乃有買錫熔銷,以錢模夾之,斯須則盈千百,便齎用之。



 開元五年,車駕往東都,宋璟知政事,奏請一切禁斷惡錢。六年正月,又切斷天下惡錢,行二銖四絫錢。不堪行用者,並銷破復鑄。至
 二月又敕曰:「古者聚萬方之貨,設九府之法,以通天下,以便生人。若輕重得中,則利可知矣;若真偽相雜,則官失其守。頃者用錢,不論此道。深恐貧窶日困,奸豪歲滋。所以申明舊章,懸設諸樣,欲其人安俗阜,禁止令行。」時江淮錢尤濫惡,有官爐、偏爐、棱錢、時錢等數色。璟乃遣監察御史蕭隱之充江淮使。隱之乃令率戶出錢,務加督責。百姓乃以上青錢充惡錢納之,其小惡者或沉之於江湖,以免罪戾。於是市井不通,物價騰起,流聞京師。
 隱之貶官,璟因之罷相,乃以張嘉貞知政事。嘉貞乃弛其禁,人乃安之。



 開元二十二年,中書侍郎張九齡初知政事,奏請不禁鑄錢,玄宗令百官詳議。黃門侍郎裴耀卿李林甫、河南少尹蕭炅等皆曰:「錢者通貨,有國之權,是以歷代禁之,以絕奸濫。今若一啟此門,但恐小人棄農逐利,而濫惡更甚,於事不便。」左監門錄事參軍劉秩上議曰:



 伏奉今月二十一日敕,欲不禁鑄錢,令百僚詳議可否者。夫錢之興,其來尚矣,將以平輕重而權本末。
 齊桓得其術而國以霸,周景失其道而人用弊。考諸載籍,國之興衰,實系於是。陛下思變古以濟今,欲反經以合道,而不即改作,詢之芻堯,臣雖蠢愚,敢不薦其聞見。古者以珠玉為上幣,黃金為中幣,刀布為下幣。管仲曰:「夫三幣,握之則非有補於暖也,舍之則非有損於飽也。先王以守財物,以御人事,而平天下也。」是以命之曰衡。衡者,使物一高一下,不得有常。故與之在君,奪之在君,貧之在君,富之在君。是以人戴君如日月,親君如父母,
 用此術也。是為人主之權。



 今之錢,即古之下幣也。陛下若舍之任人,則上無以御下,下無以事上,其不可一也。夫物賤則傷農,錢輕則傷賈。故善為國者,觀物之貴賤,錢之輕重。夫物重則錢輕,錢輕由乎物多,多則作法收之使少;少則重,重則作法布之使輕。輕重之本,必由乎是,奈何而假於人?其不可二也。夫鑄錢不雜以鉛鐵則無利,雜以鉛鐵則惡,惡不重禁之,不足以懲息。且方今塞其私鑄之路,人猶冒死以犯之,況啟其源而欲人之
 從令乎?是設陷阱而誘之入,其不可三也。夫許人鑄錢,無利則人不鑄,有利則人去南畝者眾。去南畝者眾,則草不墾,草不墾,又鄰於寒餒,其不可四也。夫人富溢則不可以賞勸,貧餒則不可以威禁。法令不行,人之不理,皆由貧富之不齊也。若許其鑄錢,則貧者必不能為。臣恐貧者彌貧而服役於富室,富室乘之而益恣。昔漢文之時,吳濞,諸侯也,富埒天子;鄧通,大夫也,財侔王者。此皆鑄錢之所致也。必欲許其私鑄,是與人利權而舍其
 柄,其不可五也。



 陛下必以錢重而傷本,工費而利寡,則臣願言其失,以效愚計。夫錢重者,猶人日滋於前,而爐不加於舊。又公錢重,與銅之價頗等,故盜鑄者破重錢以為輕錢。錢輕,禁寬則行,禁嚴則止,止則棄矣,此錢之所以少也。夫鑄錢用不贍者,在乎銅貴,銅貴,在採用者眾。夫銅,以為兵則不如鐵,以為器則不如漆,禁之無害,陛下何不禁於人?禁於人,則銅無所用,銅益賤,則錢之用給矣。夫銅不布下,則盜鑄者無因而鑄,則公錢不破,
 人不犯死刑,錢又日增,末復利矣。是一舉而四美兼也,惟陛下熟察之。



 時公卿群官,皆建議以為不便。事既不行,但敕郡縣嚴斷惡錢而已。



 至天寶之初,兩京用錢稍好,米粟豐賤。數載之後,漸又濫惡,府縣不許好者加價回博,好惡通用。富商奸人,漸收好錢,潛將往江淮之南,每錢貨得私鑄惡者五文,假托官錢,將入京私用。京城錢日加碎惡,鵝眼、鐵錫、古文、綖環之類,每貫重不過三四斤。十一載二月,下敕曰:「錢貨之用,所以通有無;輕重
 之權,所以禁逾越。故周立九府之法,漢備三官之制。永言適便,必在從宜。如聞京師行用之錢,頗多濫惡,所資懲革,絕其訛謬。然安人在於存養,化俗期於變通,法若從寬,事堪持久。宜令所司即出錢三數十萬貫,分於兩市,百姓間應交易所用錢不堪久行用者,官為換取,仍限一月日內使盡。庶單貧無患,商旅必通。其過限輒違犯者,一事已上,並作條件處分。」是時京城百姓,久用惡錢,制下之後,頗相驚擾。時又令於龍興觀南街開場,出
 左藏庫內排斗錢,許市人博換。貧弱者又爭次不得。俄又宣敕,除鐵錫、銅沙、穿穴、古文,餘並許依舊行用,久之乃定。



 乾元元年七月,詔曰:「錢貨之興,其來久矣,代有沿革,時為重輕。周興九府,實啟流泉之利;漢造五銖,亦弘改鑄之法。必令小大兼適,母子相權。事有益於公私,理宜循於通變。但以干戈未息,帑藏猶虛,卜式獻助軍之誠,弘羊興富國之算,靜言立法,諒在便人。御史中丞第五琦奏請改錢,以一當十,別為新鑄,不廢舊
 錢,冀實三官之資,用收十倍之利,所謂於人不擾,從古有經。宜職於諸監別鑄一當十錢,文曰「乾元重寶」。其開元通寶者依舊行用。所請採鑄捉搦處置,即條件聞奏。」二年三月,琦入為相,又請更鑄重輪乾元錢,一當五十,二十斤成貫。詔可之。於是新錢與乾元、開元通寶錢三品並行。尋而穀價騰貴,米斗至七千,餓死者相枕於道。乃抬舊開元錢以一當十,減乾元錢以一當三十。緣人厭錢價不定,人間抬加價錢為虛錢。長安城中,競為盜鑄,寺觀鐘
 及銅象,多壞為錢。奸人豪族犯禁者不絕。京兆尹鄭叔清擒捕之,少不容縱,數月間搒死者八百餘人。人益無聊矣。



 上元元年六月,詔曰:「因時立制,頃議新錢,且是從權,知非經久。如聞官爐之外,私鑄頗多,吞並小錢,逾濫成弊。抵罪雖眾,禁奸未絕。況物價益起,人心不安。事藉變通,期於折衷。其重棱五十價錢,宜減作三十文行用。其開元舊時錢,宜一當十文行用。其乾元十當錢,宜依前行用。仍令京中及畿縣內依此處分,諸州待進止。」七
 月敕:「重棱五十價錢,先令畿內減至三十價行,其天下諸州,並宜準此。」寶應元年四月,改行乾元錢,一以當二,乾元重棱小錢,亦以一當二;重棱大錢,一以當三。尋又改行乾元大小錢,並以一當一。其私鑄重棱大錢,不在行用之限。



 大歷四年正月,關內道鑄錢等使、戶部侍郎第五琦上言,請於絳州汾陽、銅原兩監,增置五爐鑄錢,許之。



 建中元年九月,戶部侍郎韓洄上言:「江淮錢監,歲共鑄錢四萬五千貫,輸於京師,度工用轉送之費,每貫
 計錢二千,是本倍利也。今商州有紅崖冶出銅益多,又有洛源監,久廢不理。請增工鑿山以取銅,興洛源錢監,置十爐鑄之,歲計出錢七萬二千貫,度工用轉送之費,貫計錢九百,則利浮本也。其江淮七監,請皆停罷。」從之。貞元九年正月,張滂奏:「諸州府公私諸色鑄造銅器雜物等。伏以國家錢少,損失多門。興販之徒,潛將銷鑄。錢一千為銅六斤,造寫器物,則斤直六百餘。有利既厚,銷鑄遂多,江淮之間,錢實減耗。伏請準從前敕文,除鑄鏡
 外,一切禁斷。」元和三年五月,鹽鐵使李巽上言:「得湖南院申,郴州平陽,高亭兩縣界,有平陽冶及馬跡、曲木等古銅坑,約二百八十餘井,差官檢覆,實有銅錫。今請於郴州舊桂陽監置爐兩所,採銅鑄錢,每日約二十貫,計一年鑄成七千貫,有益於人。」從之。其年六月,詔曰:「泉貨之法,義在通流。若錢有所壅,貨當益賤。故藏錢者得乘人之急,居貨者必損己之資。今欲著錢令以出滯藏,加鼓鑄以資流布,使商旅知禁,農桑獲安,義切救時,情非
 欲利。若革之無漸,恐人或相驚。應天下商賈先蓄見錢者,委所在長吏,令收市貨物,官中不得輒有程限,逼迫商人,任其貨易,以求便得。計周歲之後,此法遍行,朕當別立新規,設蓄錢之禁。所以先有告示,許有方圓,意在他時行法不貸。又天下有銀之山,必有銅礦。銅者,可資於鼓鑄,銀者,無益於生人。權其重輕,使條專一。其天下自五嶺以北,見採銀坑,並宜禁斷。恐所在坑戶,不免失業,各委本州府長吏勸課,令其採銅,助官中鑄作。仍委
 鹽鐵使條流聞奏。」



 四年閏三月,京城時用錢每貫頭除二十文、陌內欠錢及有鉛錫錢等,準貞元九年三月二十六日敕:「陌內欠錢,法當禁斷,慮因捉搦,或亦生奸,使人易從,切於不擾。自今已後,有因交關用欠陌錢者,宜但令本行頭及居停主人牙人等檢察送官。如有容隱,兼許賣物領錢人糾告,其行頭、主、人、牙人,重加科罪。府縣所由祗承人等,並不須干擾。若非因買賣自將錢於街衢行者,一切勿問。」其年六月,敕:「五嶺已北,所有銀坑,依
 前任百姓開採,禁見錢出嶺。」



 六年二月,制:「公私交易,十貫錢已上,即須兼用匹段。委度支鹽鐵使及京兆尹即具作分數,條流聞奏。茶商等公私便換見錢,並須禁斷。」其年三月,河東節度使王鍔奏請於當管蔚州界加置爐鑄銅錢,廢管內錫錢。許之,仍令加至五爐。七年五月,戶部王紹、度支盧坦、鹽鐵王播等奏:「伏以京都時用多重見錢,官中支計,近日殊少。蓋緣比來不許商人便換,因茲家有滯藏,所以物價轉高,錢多不出。臣等今商量,
 伏請許令商人於三司任便換見錢,一切依舊禁約。伏以比來諸司諸使等,或有便商人,錢多留城中,逐時收貯,積藏私室,無復通流。伏請自今已後,嚴加禁約。」從之。八年四月,敕:「以錢重貨輕,出內庫錢五十萬貫,令兩市收市布帛,每端匹估加十之一。」



 十二年正月,敕:「泉貨之設,故有常規,將使重輕得宜,是資斂散有節,必通共變,以利於人。今繒帛轉賤,公私俱弊。宜出見錢五十萬貫,令京兆府揀擇要便處開場,依市價交易。選清強官吏,
 切加勾當。仍各委本司,先作處置條件聞奏。必使事堪經久,法可通行。」又敕:「近日布帛轉輕,見錢漸少,皆緣所在壅塞,不得通流。宜令京城內自文武官僚,不問品秩高下,並公、郡、縣主、中使等,下至士庶、商旅、寺觀、坊市,所有私貯見錢,並不得過五千貫。如有過此,許從敕出後,限一月內任將市別物收貯。如錢數較多,處置未了,任於限內於地界州縣陳狀,更請限。縱有此色,亦不得過兩個月。若一家內別有宅舍店鋪等,所貯錢並須計用
 在此數。其兄弟本來異居曾經分析者,不在此限。如限滿後有違犯者,白身人等,宜付所司,決痛杖一頓處死。其文武官及公主等,並委有司聞奏,當重科貶。戚屬中使,亦具名銜聞奏。其剩貯錢,不限多少,並勒納官。數內五分取一分充賞錢,止於五千貫。此外察獲,及有人論告,亦重科處分,並量給告者。」時京師裏閭區肆所積,多方鎮錢,王鍔、韓弘、李惟簡,少者不下五十萬貫。於是競買第屋以變其錢,多者竟裏巷傭僦以歸其直。而高貲
 大賈者,多依倚左右軍官錢為名,府縣不得窮驗,法竟不行。



 十四年六月,敕:「應屬諸軍諸使,更有犯時用錢每貫除二十文、足陌內欠錢及有鉛錫錢者,宜令京兆府枷項收禁,牒報本軍本使府司,差人就軍及看決二十。如情狀難容,復有違拒者,仍令府司聞奏。」十五年八月,中書門下奏:「伏準群官所議鑄錢,或請收市人間銅物,令州郡鑄錢。當開元以前,未置鹽鐵使,亦令州郡勾當鑄造。今若兩稅盡納匹段,或慮兼要通用見錢。欲令諸道
 公私銅器,各納所在節度、團練、防禦、經略使,便據元敕給與價直,並折兩稅。仍令本處軍人熔鑄。其鑄本,請以留州留使年支未用物充,所鑄錢便充軍府、州、縣公用。當處軍人,自有糧賜,亦較省本,所資眾力,並收眾銅,天下並功,速濟時用。待一年後鑄器物盡,則停。其州府有出銅鉛可以開爐處,具申有司,便令同諸監冶例,每年與本充鑄。其收市銅器期限,並禁鑄造買賣銅物等,待議定便令有司條流聞奏。其上都鑄錢及收銅器,續處
 分。將欲頒行,尚資周慮,請令中書門下兩省、御史臺並諸司長官商量,重議聞奏。」從之。



 長慶元年九月,敕:「泉貨之義,所貴通流。如聞比來用錢,所在除陌不一。與其禁人之必犯,未若從俗之所宜,交易往來,務令可守。其內外公私給用錢,從今以後,宜每貫一例除墊八十,以九百二十文成貫,不得更有加除及陌內欠少。」大和三年六月,中書門下奏:「準元和四年閏三月敕,應有鉛錫錢,並合納官,如有人糾得一錢,賞百錢者。當時敕條,貴在
 峻切,今詳事實,必不可行。只如告一錢賞百錢,則有人告一百貫錫錢,須賞一萬貫銅錢,執此而行,事無畔際。今請以鉛錫錢交易者,一貫已下,以州府常行決脊杖二十;十貫已下,決六十,徒三年;過十貫已上,所在集眾決殺。其受鉛錫錢交易者,亦準此處分。其用鉛錫錢,仍納官。其能糾告者,每一貫賞五千文,不滿貫者,準此計賞,累至三百千,仍且取當處官錢給付。其所犯人罪不死者,徵納家資,充填賞錢。」可之。四年十一月,敕:「應私貯
 見錢家,除合貯數外,一萬貫至十萬貫,限一周年內處置畢;十萬貫至二十萬貫以下者,限二周年處置畢。如有不守期限,安然蓄積,過本限,即任人糾告,及所由覺察。其所犯家錢,並準元和十二年敕納官,據數五分取一分充賞。糾告人賞錢,數止於五千貫。應犯錢法人色目決斷科貶,並準元和十二年敕處分。其所由覺察,亦量賞一半。」事竟不行。五年二月,鹽鐵使奏:「湖南管內諸州百姓私鑄造到錢。伏緣衡、道數州,連接嶺南,山洞深
 邃,百姓依模監司錢樣,競鑄造到脆惡奸錢,轉將賤價博易,與好錢相和行用。其江西、鄂岳、桂管鑄濫錢,並請委本道觀察使條流禁絕。」敕旨宜依。



 會昌六年二月,敕:「緣諸道鼓鑄佛像鐘磬等新錢,已有次第,須令舊錢流布。絹帛價稍增。文武百僚俸料,宜起三月一日,並給見錢。其一半先給虛估匹段,對估價支給。」敕:「比緣錢重幣輕,生人坐困,今加鼓鑄,必在流行。通變救時,莫切於此。宜申先甲之令,以誡居貨之徒。京城及諸道,起今年十
 月以後,公私行用,並取新錢,其舊錢權停三數年。如有違犯,同用鉛錫惡錢例科斷,其舊錢並納官。」事竟不行。



 開元元年十一月,河中尹姜師度以安邑鹽池漸涸,師度開拓疏決水道,置為鹽屯,公私大收其利。其年十一月五日,左拾遺劉彤上表曰:「臣聞漢孝武為政,廊馬三十萬,後宮數萬人,外討戎夷,內興宮室,殫費之甚,實百當今,而古費多而貨有餘,今用少而財不足,何也?豈非古取山澤,而今取貧民哉。取山澤,則公利厚而人歸於
 農;取貧民,則公利薄而人去其業。故先王作法也,山海有官,虞衡有職,輕重有術,禁發有時。一則專農,二則饒國,濟人盛事也。臣實為今疑之。夫煮海為鹽,採山鑄錢,伐木為室。農餘之輩,寒而無衣,饑而無食,傭賃自資者,窮苦之流也。若能以山海厚利,資農之餘人,厚斂重徭,免窮苦之子,所謂損有餘而益不足,帝王之道,可不謂然乎?臣願陛下詔鹽鐵木等官收興利,貿遷於人,則不及數年,府有餘儲矣。然後下寬貸之令,蠲窮獨之徭,可
 以惠群生,可以柔荒服。雖戎狄、猾夏,堯、湯水旱,無足虞也。奉天適變,惟在陛下行之。」上令宰臣議其可否,咸以鹽鐵之利,甚益國用,遂令將作大匠姜師度、戶部侍郎強循俱攝御史中丞,與諸道按察使檢責海內鹽鐵之課。「比令使人勾當,除此外更無別求。在外不細委知,如聞稱有侵刻,宜令本州刺史上佐一人檢校,依令式收稅。如有落帳欺沒,仍委按察使糾覺奏聞。其姜師度除蒲州鹽池以外,自餘處更不須巡檢。」



 貞元十六年十
 二月,史牟奏:「澤、潞、鄭等州,多是末鹽,請禁斷。」從之。元和五年正月,度支奏:「鄜州、邠州、涇原諸將士,請同當處百姓例,食烏、白兩池鹽。」六年閏十二月,度支盧坦奏:「河中兩池顆鹽,敕文只許於京畿、鳳翔、陜、虢、河中澤潞、河南許汝等十五州界內糶貨。比來因循,兼越興、鳳、文、成等六州。臣移牒勘責,得山南西道觀察使報,其果、閬兩州鹽,本土戶人及巴南諸郡市糴,又供當軍士馬,尚有懸欠,若兼數州,自然闕絕。又得興元府諸耆老狀申訴。臣
 今商量,河中鹽請放入六州界糶貨。」從之。十年七月,度支使皇甫鎛奏,加峽內四監、劍南東西川、山南西道鹽估,以利供軍。從之。十三年,鹽鐵使程異奏:「應諸州府先請置茶鹽店收稅。伏準今年正月一日赦文,其諸州府因用兵已來,或慮有權置職名,及擅加科配,事非常制,一切禁斷者。伏以榷稅茶鹽,本資財賦,贍濟軍鎮,蓋是從權。昨兵罷,自合便停,事久實為重斂。其諸道先所置店及收諸色錢物等,雖非擅加,且異常制,伏請準赦文
 勒停。」從之。



 十四年三月,鄆、青、兗三州各置榷鹽院。



 長慶元年三月,敕:「河朔初平,人希德澤,且務寬泰,使之獲安。其河北榷鹽法且權停。仍令度支與鎮冀、魏博等道節度審察商量,如能約計課利錢數,分付榷鹽院,亦任穩便。」自天寶末兵興以來,河北鹽法,羈縻而已。暨元和中,皇甫鎛奏置稅鹽院,同江、淮兩池榷利,人苦犯禁,戎鎮亦頻上訴,故有是命。其月,鹽鐵使王播奏:「揚州、白沙兩處納榷場,請依舊為院。」又奏:「諸道鹽院糶鹽付商人,請
 每斗加五十,通舊三百文價;諸處煎鹽停場,置小鋪糶鹽,每斗加二十文,通舊一百九十文價。」又奏:「應管煎鹽戶及鹽商,並諸鹽院停場官吏所由等,前後制敕,除兩稅外,不許差役追擾。今請更有違越者,縣令、刺史貶黜罰俸。」從之。二年五月,詔曰:「兵革初寧,亦資榷筦,閭閻重困,則可蠲除。如聞淄青、兗、鄆三道,往來糶鹽價錢,近取七十萬貫,軍資給費,優贍有餘。自鹽鐵使收管已來,軍府頓絕其利。遂使經行陣者有停糧之怨,服隴畝者有加
 稅之嗟,犯鹽禁者困鞭撻之刑,理生業者乏蠶醬之具。雖縣官受利,而郡府益空。俾人獲安寧,我因節用。其鹽鐵先於淄青、兗、鄆等道管內置小鋪糶鹽,巡院納榷,起今年五月一日已後,一切並停。仍各委本道約校比來節度使自收管充軍府逐急用度,及均減管內貧下百姓兩稅錢數。至年終,各具糶鹽所得錢,並均減兩稅。奏聞。」



 安邑、解縣兩池,舊置榷鹽使,仍各別置院官。元和三年七月,復以安邑、解縣兩池留後為榷鹽使。先是,兩池
 鹽務隸度支,其職視諸道巡院。貞元十六年,史牟以金部郎中主池務,恥同諸院,遂奏置使額。二十一年,鹽鐵、度支合為一使,以杜佑兼領。佑以度支既稱使,其所管不宜更有使名,遂與東渭橋使同奏,罷之。至是,裴均主池務,職轉繁劇,復有是請。大和三年四月,敕安邑、解縣兩池榷課,以實錢一百萬貫為定額。至大中二年正月,敕但取匹段精好,不必計舊額錢數。及大中年,度支奏納榷利一百二十一萬五千餘貫。



 女鹽池在解縣,朝邑
 小池在同州,鹵池在京兆府奉先縣,並禁斷不榷。烏池在鹽州,舊置榷稅使。長慶元年三月,敕烏池每年糶鹽收博榷米,以一十五萬石為定額。溫池,大中四年三月因收復河隴,敕令度支收管。溫池鹽仍差靈州分巡院官勾當。至六年三月,敕令割屬威州,置榷稅使。緣新制置,未立榷課定額。胡落池在豐州界,河東供軍使收管。每年採鹽約一萬四千餘石,供振武、天德兩軍及營田水運官健。自大中四年黨項叛擾,饋運不通,供軍使請
 權市河東白池鹽供食。其白池屬河節度使,不系度支。初,玄宗已前,亦有鹽池使。景雲四年三月,蒲州刺史充關內鹽池使。先天二年九月,強循除豳州刺史,充鹽池使,此即鹽州池也。開元十五年五月,兵部尚書蕭嵩除關內鹽池使。此是朔方節度常帶鹽池使也。



\end{pinyinscope}