\article{卷三十 志第十 音樂三}

\begin{pinyinscope}

 貞觀二年,
 太常少卿祖孝孫既定雅樂,至六年,詔褚亮、虞世南、魏徵等分制樂章。其後至則天稱制,多所改易,歌辭皆是內出。開元初,則中書令張說奉制所作,然雜
 用貞觀舊詞。自後郊廟歌工樂師傳授多缺,或祭用宴樂,或郊稱廟詞。二十五年,太常卿韋縚令博士韋逌、直太樂尚沖、樂正沈元福、郊社令陳虔申懷操等,銓敘前後所行用樂章,為五卷,以付太樂、鼓吹兩署,令工人習之。時太常舊相傳有宮、商、角、徵、羽《宴樂》五調歌詞各一卷,或云貞觀中侍中楊恭仁妾趙方等所銓集,詞多鄭、衛,皆近代詞人雜詩,至縚又令太樂令孫玄成更加整比為七卷。又自開元已來,歌者雜用胡夷里巷之曲,其
 孫玄成所集者,工人多不能通,相傳謂為法曲。今依前史舊例,錄雅樂歌詞前後常行用者,附於此志。其五調法曲,詞多不經,不復載之。



 冬至祀昊天於圓丘樂章八首貞觀二年,祖孝孫定雅樂。貞觀六年,褚亮、虞世南、魏徵等作此詞,今行用。



 降神用《豫和》:



 上靈眷命兮膺會昌,盛德殷薦葉辰良。景福降兮聖德遠。玄化穆兮天歷長。



 皇帝行用《太和》:



 穆穆我後,道應千齡。登三處大,得一居貞。禮唯崇德,樂以和聲。百神仰止,天下文明。



 登歌奠玉帛用《肅和》:



 陽播氣,甄耀垂明。有赫圓宰,深仁曲成。日麗蒼璧,煙開紫營。聿遵虔享,式降鴻禎。



 迎俎入用《雍和》:



 欽惟大帝,載仰皇穹。始命田燭,爰啟郊宮。《雲門》駭聽,雷
 鼓鳴空。神其介祀,景祚斯融。



 酌獻飲福用《壽和》:



 八音斯奏,三獻畢陳。寶祚惟永,暉光日新。



 送文舞出迎武舞入用《舒和》:



 迭璧凝影皇壇路,編珠流彩帝郊前。已奏黃鐘歌大呂,還符寶歷祚昌年。



 武舞作用《凱安》:



 昔在炎運終,中華亂無象。酆郊赤烏見,邙山黑雲上。大
 賚下周車,禁暴開殷網。幽明同葉贊,鼎祚齊天壤。



 送神用《豫和》:



 歌奏畢兮禮獻終,六龍馭兮神將升。明德感兮非黍稷,降福簡兮祚休徵。



 又郊天樂章一首太樂舊有此辭,不詳所起。



 送神用《豫和》:



 蘋蘩禮著,黍稷誠微。音盈鳳管,彩駐龍旗。洪歆式就,介福攸歸。送樂有闋,靈馭遄飛。



 則天大聖皇后大享昊天樂章十二首御撰



 第一:



 太陰凝至化,貞耀蘊軒儀。德邁娥臺敞,仁高姒幄披。捫天遂啟極,夢日乃升曦。



 第二:



 贍紫極,望玄穹。翹至懇,罄深衷。聽雖遠,誠必通。垂厚澤,降雲宮。



 第三:



 乾儀混成沖邃,天道下濟高明。陽晨披紫闕,太一曉降黃庭。圓壇敢申昭報,方璧冀展虔情。丹襟式敷衷懇,玄鑒庶察微誠。



 第四:



 巍巍睿業廣,赫赫聖基隆。菲德承先顧,禎符萃眇躬。銘開武巖側,圖薦洛川中。微誠詎幽感,景命忽昭融。有懷慚紫極,無以謝玄穹。



 第五:



 朝壇霧卷,曙嶺煙沉。爰設筐幣,式表誠心。筵輝麗璧,樂暢和音。仰惟靈鑒,俯察翹襟。



 第六:



 昭昭上帝,穆穆下臨。禮崇備物,樂奏鏘金。蘭羞委薦,桂醑盈斟。敢希明德,聿罄莊心。



 第七:



 金尊浮九醖,禮備三周。陳誠菲奠,契福神猷。



 第八:



 奠璧郊壇昭大禮,鏘金拊石表虔誠。始奏《承云》娛帝賞,復歌《調露》暢《韶英》。



 第九:



 荷恩承顧托,執契恭臨撫。廟略靜邊荒,天兵曜神武。有截資先化,無為遵舊矩。禎符降昊穹,大業光寰宇。



 第十:



 肅肅祀典,邕邕禮秩。三獻已周,九成斯畢。爰撤其俎,載遷其實。或升或降,唯誠唯質。



 第十一:



 禮終肆類,樂闋九成。仰惟明德,敢薦非馨。顧慚菲奠,久駐雲軿。瞻荷靈澤,悚戀兼盈。



 第十二:



 式乾路,闢天扉。回日馭,動雲衣。登金闕,入紫微。望仙駕,仰恩徽。



 景龍三年中宗親祀昊天上帝樂章十首



 降神用《豫和》:



 天之歷數歸睿唐,顧惟菲德欽昊蒼。選吉日兮表殷薦,冀神鑒兮降陽。



 皇帝行用《太和》圓鐘宮



 恭臨寶位,肅奉瑤圖。恆思解網,每軫泣辜。德慚巢燧,化劣唐虞。期我良弼,式贊嘉謨。



 告謝圓鐘宮:



 得一流玄澤,通三御紫宸。遠葉千齡運,遐銷九域塵。絕瑞駢闐集,殊祥絡繹臻。年登慶西畝,稔歲賀盈囷。



 登歌用《肅和》無射均之林鐘羽:



 悠哉廣覆,大矣曲成。九玄著象,七曜甄明。珪璧是奠,醖酎斯盈。作樂崇德,爰暢《咸英》。



 迎俎用《雍和》圓鐘均之黃鐘羽:



 郊壇展敬,嚴配因心。孤竹簫管,空桑瑟琴。肅穆大禮,鏗鏘八音。恭惟上帝,希降靈歆。



 酌獻用《福和》圓鐘宮:



 九成爰奏,三獻式陳。欽承景福,恭托明禋。



 中宮助祭升壇用函鐘宮:



 坤元光至德,柔訓闡皇風。《芣苡》芳聲遠,《螽斯》美化隆。睿範超千載,嘉猷備六宮。肅恭陪盛典,欽若薦禋宗。



 亞獻用函鐘宮:



 三靈降饗,三后配神。虔敷藻奠,敬展郊禋。



 送文舞出迎武舞入用《舒和》:圓鐘均之中呂商



 已陳粢盛敷嚴祀,更奏笙鏞協雅聲。璇圖寶歷欣寧謐,晏俗淳風樂太平。



 武舞作用《凱安》圓鐘均之無射徵:



 堂堂聖祖興,赫赫昌基泰。戎車盟津偃,玉帛塗山會。舜日啟祥暉,堯雲卷征旆。風猷被有截,聲教覃無外。



 開元十一年玄宗祀昊天於圓丘樂章十一首



 降神用《豫和》圓鐘宮三成,黃鐘角一成,太簇徵一成,姑洗羽一成,已上六變詞同。:



 至矣丕構,蒸哉太平。授犧膺籙,復禹繼明。草木仁化,《鳧鷖》頌聲。祀宗陳德,無愧斯誠。



 迎神用《歆和》:



 崇禋已備,粢盛聿修。潔誠斯展,鐘石方遒。



 皇祖光皇帝室酌獻用《長發》黃鐘宮。詞同貞觀《長發》。



 太祖景皇帝室酌獻用《大基》太簇宮。詞同貞觀《大基》。



 代祖元皇帝室酌獻用《大成》姑洗宮。詞同貞觀《大成》。



 高祖神堯皇帝室酌獻用《大明》蕤賓宮。詞同貞觀《大明》。



 太宗文武聖皇帝室酌獻用《崇德》夷則宮。詞同貞觀《崇德》。:



 高宗天皇大帝室酌獻用《鈞天》黃鐘宮。詞同光宅鈞天。



 義宗孝敬皇帝室酌獻用《承光》黃鐘宮:



 金相載穆,玉裕重暉。養德清禁,承光紫微。乾宮候色,震象增威。監國方永,賓天不歸。孝友自衷,溫文性與。龍樓正啟,鶴駕斯舉。丹扆流念,鴻名式序。中興考室,永陳彞俎。



 皇帝飲福用《延和》黃鐘宮:



 巍巍累聖,穆穆重光。奄有區夏,祚啟隆唐。百蠻飲澤,萬國來王。本枝億載,鼎祚逾長。



 皇帝行用《太和》:



 郊壇齊帝,禮樂祠天。丹青寰宇,宮徵山川。神祇畢降,行止重旋。融融穆穆,納祉洪延。



 登歌奠玉帛用《肅和》:



 止奏潛聆,登儀宿轉。大玉躬奉,參鐘首奠。簠簋聿升,犧牲遞薦。昭事顒若,存存以伣。



 迎俎入用《雍和》:



 爛雲普洽,律風無外。千品其凝,九賓斯會。禋樽晉燭,純犧滌汰。玄覆攸廣,鴻休汪濊。



 皇帝酌獻天神用《壽和》:



 六變爰闋,八階載虔。祐我皇祚,於萬斯年。



 酌獻配座用《壽和》:



 於赫聖祖,龍飛晉陽。底定萬國,奄有四方。功格上下,道冠農黃。郊天配享,德合無疆。



 飲福酒用《壽和》:



 崇崇太畤,肅肅嚴禋。粢盛既潔,金石畢陳。上帝來享,介福爰臻。受厘合福,寶祚惟新。



 送文舞出迎武舞入用《舒和》:



 祝史正辭,人神慶葉。福以德昭,享以誠接。六變雲備,百禮斯浹。祀事孔明,祚流萬葉。



 武舞用《凱安》:



 馨香惟後德,明命光天保。肅和崇聖靈,陳信表皇道。玉鏚初蹈厲,金匏既靜好。



 禮畢送神用《豫和》:



 大號成命,《思文》配天。神光肸蚃,龍駕言旋。眇眇閶闔,昭
 昭上玄。俾昌而大,於萬斯年。



 皇帝還大次用《太和》:



 六成既闋,三薦雲終。神心具醉,聖敬愈崇。受厘皇邸,回蹕帷宮。穰穰之福,永永無窮。



 玄宗開元十三年封泰山祀天樂章十四首中書令燕國公張說作,今行用。



 降神用《豫和》六變夾鐘宮之一:



 款泰壇,柴泰清。受天命,報天成。竦皇心,薦樂聲。志上達,
 歌下迎。



 夾鐘宮之二:



 億上帝,臨下庭。騎日月,陪列星。嘉祝信,大糦馨。澹神心,醉皇靈。



 夾鐘宮之三:



 相百闢,貢八荒。九歌敘,萬舞翔。肅振振,鏘皇皇。帝欣欣,福穰穰。



 黃鐘宮:



 高在上,道光明。物資始,德難名。承眷命,牧蒼生。寰宇謐,太階平。



 太簇徵:



 天道無親,至誠與鄰。山川偏禮,宮徵惟新。玉帛非盛,聰明會貞。正斯一德,通乎百神。



 姑洗羽:



 饗帝饗親,維孝維聖。緝熙懿德,敷揚成命。華夷志同,笙鏞禮盛。明靈降止,感此誠敬。



 迎送皇帝用《太和》:



 孝敬中發,和容外彰。騰華照宇,如升太陽。貞璧就奠,玄靈垂光。禮樂具舉,濟濟洋洋。



 登歌奠玉帛用《肅和》羽調:



 奠祖配天,承天享帝。百靈咸秩,四海來祭。植我蒼璧,布我玄制。華日徘徊,神靈容裔。



 迎俎入用《雍和》:



 俎豆有馝,潔粢豐盛。亦有和羹,既戒既平。鼓鐘管磬,肅
 唱和鳴。皇皇后祖,賚我思成。



 酌獻用《壽和》黃鐘宮調:



 蒸蒸我後,享獻惟夤。躬酌鬱鬯,跪奠明神。孝莫孝乎配上帝以親,敬莫敬乎教天下為臣。



 皇帝飲福用《壽和》:



 皇祖嚴配,配享皇天。皇皇降嘏,天子萬年。



 送文舞出迎武舞入用《舒和》商調:



 六鐘翕協六變成,八佾倘佯八風生。樂《九韶》兮人神感,
 美《七德》兮天地清。



 終獻亞獻用《凱安》:



 列祖順三靈,文宗威四海。黃鉞誅群盜,硃旗掃多罪。戢兵天下安,約法人心改。大哉幹羽意,長見風雲在。



 送神用《豫和》夾鐘宮調:



 禮樂終,煙燎上。懷靈惠,結皇想。歸風疾,回風爽。百福來,眾神往。



 正月上辛祈穀於南郊樂章八首貞觀中褚亮作,今
 行用。



 降神用《豫和》詞同冬至圓丘。



 皇帝行用《太和》詞同冬至圓丘。



 登歌奠玉帛用《肅和》《貞觀禮》,祀感帝用此詞,顯慶已後,詞同冬至圓丘。:



 履艮斯繩,居中體正。龍運垂祉,昭符啟聖。式事嚴禋,聿懷嘉慶。惟帝永錫,時皇休命。



 迎俎用《雍和》:



 殷薦乘春,太壇臨曙。八簋盈和,六瑚登御。嘉稷匪歆,德馨斯飫。祝嘏無易,靈心有豫。



 皇帝酌獻飲福酒用《壽和》詞同冬至圓丘。



 送文舞出迎武舞入用《舒和》:



 玉帛犧牲申敬享,金絲戚羽盛音容。庶俾億齡禔景福,長欣萬宇洽時邕。



 武舞用《凱安》詞同冬至圓丘。



 送神用《豫和》詞同冬至圓丘。



 季秋享上帝於明堂樂章八首貞觀中褚亮等作,今行用。



 降神用《豫和》詞同冬至圓丘。



 皇帝行用《太和》詞同冬至圓丘。



 登歌奠玉帛用《肅和》:



 象天御宇,乘時布政。嚴配申虔,宗禋展敬。樽罍盈列,樹
 羽交映。玉幣通誠,祚隆皇聖。



 迎俎用《雍和》:



 八牖晨披,五精朝奠。霧凝璇篚,風清金縣。神滌備全,明粢豐衍。載結彞俎,陳誠以薦。



 皇帝酌獻飲福用《壽和》詞同冬至圓丘。



 送文舞出迎武舞入用《舒和》:



 御扆合宮承寶歷,席圖重館奉明靈。偃武修文九圍泰,沉烽靜柝八荒寧。



 武舞用《凱安》詞同冬至圓丘。



 送神用《豫和》詞同冬至圓丘。



 則天大聖皇后享明堂樂章十二首御撰



 外辦將出:



 總章陳昔典,衢室禮惟神。宏規則天地,神用葉陶鈞。負扆三春旦,充庭萬宇賓。顧己誠虛薄,空慚馭兆人。



 皇帝行用黃鐘宮:



 仰膺歷數,俯順謳歌。遠安邇肅,俗阜時和。化光玉鏡,訟息金科。方興典禮,永戢干戈。



 皇嗣出入升降:



 至人光俗,大孝通神。謙以表性,恭惟立身。洪規載啟,茂典方陳。譽隆三善,祥開萬春。



 迎送王公:



 千官肅事,萬國朝宗。載延百闢,爰集三宮。君臣得合,魚水斯同。睿圖方永,周歷長隆。



 登歌大呂均無射羽:



 禮崇宗祀,志表嚴禋。笙鏞合奏,文物惟新。敬遵茂典,敢
 擇良辰。潔誠斯著,奠謁方申。



 配饗:



 笙鏞間鳴玉,文物昭清暉。粹影臨芳奠,休光下太微。孝思期有感,明潔庶無違。



 宮音:



 履艮苞群望,居中冠百靈。萬方資廣運,庶品荷裁成。神功諒匪測,盛德實難名。藻奠申誠敬,恭祀表惟馨。



 角音:



 出震位,開平秩。扇條風,乘甲乙。龍德盛,鳥星出。薦珪篚,陳誠實。



 徵音:



 赫赫離精禦炎陸,滔滔熾景開隆暑。冀延神鑒俯蘭樽,式表虔襟陳桂俎。



 商音:



 律中夷則,序應收成。功宣建武,儀表惟明。爰申禮奠,庶展翹誠。九秋是式,百穀斯盈。



 羽音:



 葭律肇啟隆冬,蘋藻攸陳饗祭。黃鐘既陳玉燭,紅粒方殷稔歲。



 孟夏雩祀上帝於南郊樂章八首貞觀中褚亮等作,今行用。



 降神用《豫和》詞同冬至圓丘。



 皇帝行用《太和》詞同冬至圓丘。



 登歌奠玉帛用《肅和》:



 硃鳥開辰,蒼龍啟映。大帝昭饗,群生展敬。禮備懷柔,功宣舞詠。旬液應序,年祥葉慶。



 迎俎用《雍和》:



 紺筵分彩,瑤圖葉絢。風管晨凝,雲歌曉囀。肅事蘋藻,虔申桂奠。百穀斯登,萬箱攸薦。



 皇帝酌獻飲福酒用《壽和》詞同冬至圓丘。



 送文舞出迎武舞入用《舒和》:



 鳳曲登歌調令序,龍雩集舞泛祥風。彩FK雲回昭睿德,硃幹電發表神功。



 武舞用《凱安》詞同冬至圓丘。



 送神用《豫和》詞同冬至圓丘。



 又雩祀樂章二首太樂舊有此詞,不詳所起,或云開元初造。



 降神用《豫和》:



 鳥緯遷序,龍星見辰。純陽在律,明德崇禋。五方降帝,萬宇安人。恭以致享,肅以迎神。



 送神用《豫和》:



 祀遵經設,享緣誠舉。獻畢於樽,撤臨於俎。舞止干戚,樂停柷敔。歌以送神,神還其所。



 祀五方上帝於五郊樂章四十首貞觀中魏徵等作,今行用。



 祀黃帝降神奏宮音:



 黃中正位,含章居貞。既彰六律,兼和五聲。畢陳萬舞,乃薦斯牲。神其下降,永祚休平。



 皇帝行用《太和》詞同冬至圓丘。



 登歌奠玉帛用《肅和》:



 渺渺方輿,蒼蒼圓蓋。至哉樞紐,宅中圖大。氣調四序,風和萬籟。祚我明德,時雍道泰。



 迎俎用《雍和》:



 金縣夕肆,玉俎朝陳。饗薦黃道,芬流紫辰。乃誠乃敬,載享載禋。崇薦斯在,惟皇是賓。



 皇帝酌獻飲福用《壽和》詞同冬至圓丘。



 送文舞出迎武舞入用《舒和》:



 禦征乘宮出郊甸,安歌率舞遞將迎。自有《雲門》符帝賞,猶持雷鼓答天成。



 武舞用《凱安》詞同冬至圓丘。



 送神用《豫和》詞同冬至圓丘。



 祀青帝降神用角音:



 鶴雲旦起,鳥星昏集。律候新風,陽開初蟄。至德可饗,行潦斯挹。錫以無疆,蒸人乃粒。



 皇帝行用《太和》詞同冬至圓丘。



 登歌奠玉帛用《肅和》:



 玄鳥司春,蒼龍登歲。節物變柳,光風轉蕙。瑤席降神,硃弦饗帝。誠備祝嘏,禮殫珪幣。



 迎俎用《雍和》:



 大樂稀音,至誠簡禮。文物斯建,聲名濟濟。六變有成,三登無體。乃眷豐潔,恩覃愷悌。



 皇帝酌獻飲福用《壽和》詞同冬至圓丘。



 送文舞出迎武舞入用《舒和》:



 笙歌龠舞屬年韶,鷺鼓鳧鐘展時豫。調露初迎綺春節,承雲遽踐蒼霄馭。



 武舞用《凱安》詞同冬至圓丘。



 送神用《豫和》詞同冬至圓丘。



 祀赤帝降神用徵音:



 青陽告謝,硃明戒序。延長是祈,敬陳椒醑。博碩斯薦,笙鏞備舉。庶盡肅恭,非馨稷黍。



 皇帝行用《太和》詞同冬至圓丘。



 登歌奠玉帛用《肅和》:



 離位克明,火中宵見。峰雲暮起,景風晨扇。木槿初榮,含桃可薦。芬馥百品,鏗鏘三變。



 迎俎用《雍和》:



 昭昭丹陸,奕奕炎方。禮陳牲幣,樂備篪簧。瓊羞溢俎,玉浮觴。恭惟正直,歆此馨香。



 皇帝酌獻飲福用《壽和》詞同冬至圓丘。



 送文舞出迎武舞入用《舒和》:



 千里溫風飄絳羽,十枚炎景勝硃幹。陳觴薦俎歌三獻,拊石摐金會七盤。



 武舞用《凱安》詞同冬至圓丘。



 送神用《豫和》詞同冬至圓丘。



 祀白帝降神用商音:



 白藏應節,天高氣清。歲功既阜,庶類收成。萬方靜謐,九土和平。馨香是薦,受祚聰明。



 皇帝行用《太和》詞同冬至圓丘。



 登歌奠玉帛用《肅和》:



 金行在節,素靈居正。氣肅霜嚴,林凋草勁。豺祭隼擊,潦收川鏡。九穀已登,萬箱流詠。



 迎俎用《雍和》:



 律應西成,氣躔南呂。珪幣咸列,笙竽備舉。苾苾蘭羞,芬芬桂醑。式資宴貺,用調霜序。



 皇帝酌獻飲福用《壽和》詞同冬至圓丘。



 送文舞出迎武舞入用《舒和》:



 璿儀氣爽驚緹龠,玉呂灰飛含素商。鳴鞞奏管芳羞薦,會舞安歌葆眊揚。



 武舞用《凱安》詞同冬至圓丘。



 送神用《豫和》詞同冬至圓丘。



 祀黑帝降神用羽音:



 嚴冬季月,星回風厲。享祀報功,方祈來歲。



 皇帝行用《太和》詞同冬至圓丘。



 登歌奠玉帛用《肅和》:



 律周玉琯,星回金度。次極陽烏,紀窮陰兔。火林溯雪,湯
 泉凝冱。八蠟已登,三農息務。



 迎俎用《雍和》:



 陽月斯紀,應鐘在候。載潔牲牷,爰登俎豆。既高既遠,無聲無臭。靜言格思,惟神保佑。



 皇帝酌獻飲福用《壽和》詞同冬至圓丘。



 送文舞出迎武舞入用《舒和》:



 執龠持羽初終曲,硃干玉鏚始分行。《七德》、《九功》咸已暢,明靈降福具穰穰。



 武舞用《凱安》詞同冬至圓丘。



 送神用《豫和》詞同冬至圓丘。



 又五郊樂章十首太樂舊有此詞,不詳所起。



 黃郊迎神:



 硃明季序,黃郊王辰。厚以載物,甘以養人。毓金為體,稟火成身。宮音式奏,奏以迎神。



 送神:



 春末冬暮,徂夏杪秋。土王四月,時季一周。黍稷已享,籩豆宜收。送神有樂,神其賜休。



 青郊迎神:



 緹幕移候,青郊啟蟄。淑景遲遲,和風習習。璧玉宵備,旌旄曙立。張樂以迎,帝神其入。



 送神:



 文物流彩,聲明動色。人竭其恭,靈昭其飭。歆薦無巳,垂禎不極。送禮有章,惟神還軾。



 赤郊迎神:



 青陽節謝,硃明候改。靡草雕華,含桃流彩。列鐘磬,筵
 陳脯醢。樂以迎神,神其如在。



 送神:



 炎精式降,蒼生攸仰。羞列豆籩,酒陳牲象。昭祀有應,宜其不爽。送樂張音,惟靈之往。



 白郊迎神:



 序移玉律,節應金商。天嚴殺氣,吹警秋方。燎既積,稷奠並芳。樂以迎奏,庶降神光。



 送神:



 祀遵五禮,時屬三秋。人懷肅敬,靈降禎休。奠歆旨酒,薦享珍羞。載張送樂,神其上游。



 黑郊迎神:



 玄英戒序,黑郊臨候。掌禮陳彞,司筵執豆。寒雰斂色,冱泉凝漏。樂以迎神,八音斯奏。



 送神:



 北郊時冽,南陸輝處。奠本虔誠,獻彌恭慮。上延祉福,下承歡豫。廣樂送神,神其整馭。



 祀朝日樂章八首貞觀中作,今行用。



 降神用《豫和》詞同冬至圓丘。



 皇帝行用《太和》詞同冬至圓丘。



 登歌奠玉帛用《肅和》:



 惟聖格天,惟明饗日。帝郊肆類,王宮戒吉。珪奠春舒,鐘歌曉溢。禮云克備,斯文有秩。



 迎俎用《雍和》:



 晨儀式薦,明祀惟光。神物爰止,靈暉載揚。玄端肅事,紫幄興祥。福履攸假,於昭令王。



 皇帝酌獻飲福用《壽和》詞同冬至圓丘。



 送文舞出迎武舞入用《舒和》:



 崇牙樹羽延《調露》,旋宮扣律掩《承云》。誕敷懿德昭神武,載集豐功表睿文。



 武舞用《凱安》詞同冬至圓丘。



 送神用《豫和》詞同冬至圓丘。



 又祀朝日樂章二首太樂舊有此辭,不詳所起。



 迎神:



 太陽朝序,王宮有儀。蟠桃彩駕,細柳光馳。軒祥表合,漢
 歷彰奇。禮和樂備,神其降斯。



 送神:



 五齊兼飭,百羞具陳。樂終廣奏,禮畢崇禋。明鑒萬宇,昭臨兆人。永流洪慶,式動曦輪。



 祀夕月樂章八首貞觀中作,今行用。



 降神用《豫和》詞同冬至圓丘。



 皇帝行用《太和》詞同冬至圓丘。



 登歌奠玉帛用《肅和》:



 測妙為神,通微曰聖。坎祀貽則,郊禋展敬。璧薦登光,金
 歌動映。以載嘉德,以流曾慶。



 迎俎用《雍和》:



 朏晨爭舉,天宗禮闢。夜典涼秋,陰明湛夕。有斯旨,有牲斯碩。穆穆其暉,穰穰是積。



 皇帝酌獻飲福用《壽和》詞同冬至圓丘。



 送文舞出迎武舞入用《舒和》:



 合吹八風金奏動,分容萬舞玉鞘驚。詞昭茂典光前烈,夕曜乘功表盛明。



 武舞用《凱安》詞同冬至圓丘。



 送神用《豫和》詞同冬至圓丘。



 蠟百神樂章八首貞觀中作,今行用。



 降神用《豫和》詞同冬至圓丘。



 皇帝行用《太和》詞同冬至圓丘。



 登歌奠玉帛用《肅和》:



 序迫歲陰,日躔星紀。爰稽茂典,聿崇清祀。綺幣霞舒,瑞珪虹起。百禮垂裕,萬靈薦祉。



 迎俎用《雍和》:



 緹龠勁序,玄英晚候。姬蠟開儀,豳歌入奏。蕙馥雕俎,蘭
 芬玉酎。大饗明祇,永綏多祐。



 皇帝酌獻飲福用《壽和》詞同冬至圓丘。



 送文舞出迎武舞入用《舒和》:



 經緯兩儀文化洽,削平萬域武功成。瑤弦自樂乾坤泰,玉鏚長歡區縣寧。



 武舞用《凱安》詞同冬至圓丘。



 送神用《豫和》詞同冬至圓丘。



 又蠟百神樂章二首太樂舊有此詞,不詳所起。



 迎神今不行用:



 八蠟開祭,萬物咸祀。上極天維,下窮坤紀。鼎俎流馥,樽彞薦美。有靈有祇,咸希來止。



 送神今不行用:



 十旬歡洽,一日祠終。澄彞拂俎,報德酬功。慮虔容肅,禮縟儀豐。神其降祉,整馭隨風。



 夏至祭皇地祇於方丘樂章八首貞觀中褚亮等作



 迎神用《順和》:



 萬物資以化,交泰屬升平。易從業惟簡,得一道斯寧。具
 儀光玉帛,送舞變《咸英》。黍稷良非貴,明德信惟馨。



 皇帝行用《太和》詞同冬至圓丘。



 登歌奠玉帛用《肅和》:



 至矣坤德,皇哉地祇。開元統紐,合大承規。九宮肅列,六典相儀。永言配命,長保無虧。



 迎俎用《雍和》:



 柔而能方,直而能敬。厚載以德,大亨以正。有滌斯牷,有馨斯盛。介茲景福,祚我休慶。



 皇帝酌獻飲福用《壽和》詞同冬至圓丘。



 送文舞出迎武舞入用《舒和》:



 玉幣牲牷分薦享,羽旄幹鏚遞成容。一德惟寧兩儀泰,三才保合四時邕。



 武舞用《凱安》詞同冬至圓丘。



 送神用《順和》:



 陰祇葉贊,厚載方貞。牲幣具舉,簫管備成。其禮惟肅,其德惟明。神之聽矣,式鑒虔誠。



 則天皇后永昌元年大享拜洛樂章十五首御撰



 設禮用《昭和》:



 九玄眷命,三聖基隆。奉成先旨,明臺畢功。宗祀展敬,冀表深衷。永昌帝業,式播淳風。



 《致和》:



 神功不測兮運陰陽。包藏萬宇兮孕八荒。天符既出兮帝業昌。願臨明祀兮降禎祥。



 《咸和》:



 坎澤祠容備舉,坤壇祭典爰伸。靈眷遙行秘躅,嘉貺薦委殊珍。肅禮恭禋載展,翹襟懇志逾殷。方期交際懸應,末一句逸。



 乘輿初行用《九和》:



 祇荷坤德,欽若乾靈。慚惕罔置,興居匪寧。恭崇禮則,肅奉儀形。惟憑展敬,敢薦非馨。



 拜洛用《顯和》:



 菲躬承睿顧,薄德忝坤儀。乾乾遵後命,翼翼奉先規。撫
 俗勤雖切,還淳化尚虧。未能弘至道,何以契明祇?



 受圖用《顯和》:



 顧德有慚虛菲,明祇屢降禎符。汜水初呈秘象,溫洛薦表昌圖。玄澤流恩載洽,丹襟荷渥增愉。



 登歌用《昭和》:



 舒陰至養,合大資生。德以恆固,功由永貞。升歌薦序,垂幣翹誠。虹開玉照,鳳引金聲。



 迎俎用《敬和》:



 蘭俎既升,蘋羞可薦。金石載設,《咸英》已變。林澤斯總,山川是遍。敢用敷誠,實惟忘倦。



 酌獻用《欽和》:



 送文舞出迎武舞入用《齊和》:



 沉潛演貺分三極,廣大凝禎總萬方。既薦羽旌文化啟,還呈幹戚武威揚。



 武舞用《德和》:



 夕惕司龍契,晨兢當鳳扆。崇儒習舊規,偃霸循先旨。絕
 壤飛冠蓋,遐區麗山水。幸承三聖餘,忻屬千年始。



 撤俎用《禋和》:



 百禮崇容,千官肅事。靈降舞兆,神凝有粹。奠享咸周,威儀畢備。奏《夏》登列,歌《雍》撤肆。



 辭神用《通和》:



 皇皇靈眷,穆穆神心。暫動凝質,還歸積陰。功玄樞紐,理寂高深。銜恩佩德,聳志翹襟。



 送神用《歸和》:



 言旋雲洞兮躡煙途。永寧中宇兮安下都。苞涵動植兮順榮枯,長貽寶貺兮贊璇圖。



 又《歸和》:



 調雲闋兮神座興,驂雲駕兮儼將升。騰絳霄兮垂景祐,翹丹懇兮荷休徵。



 睿宗太極元年祭皇地祇於方丘樂章八首不詳撰者



 迎神用《順和》黃鐘宮三變,太簇角一變,姑洗徵一變,南呂羽一變。:



 坤厚載物,德柔垂祉。九域咸雍,四溟為紀。敬因良節,虔
 修陰祀。廣樂式張,靈其降止。



 金奏新加太簇宮:



 坤元至德,品物資生。神凝博厚,道葉高明。列鎮五岳,環流四瀛。於何不載,萬寶斯成。



 皇帝行用《太和》詞同貞觀冬至圓丘,黃鐘宮。



 登歌奠玉帛用《肅和》詞同貞觀太廟《肅和》,應鐘均之夷則。



 迎俎及酌獻用《雍和》詞同貞觀太廟《雍和》。



 送文舞出迎武舞入用《舒和》詞同皇帝朝群臣《舒和》。



 武舞用《凱安》詞同貞觀冬至圓丘。



 送神用《順和》林鐘宮:



 樂備金石,禮光樽俎。大享爰終,洪休是舉。雨零感節,雲飛應序。纓紱載辭,皇靈具舉。



 玄宗開元十一年祭皇地祇於汾陰樂章十一首



 迎神用《順和》林鐘以下各再變林鐘宮黃門侍郎韓思復作:



 大樂和暢,殷薦明神。一降通感,八變必臻。有求斯應,無
 德不親。降靈醉止,休征萬人。



 太簇角中書侍郎廬從願作:



 坤元載物,陽樂發生。播殖資始,品匯咸亨。列俎棋布,方壇砥平。神歆禋祀,後德惟明。



 姑洗徵司勛郎中劉晃作:



 大君出震,有事郊禋。齋戒既肅,馨香畢陳。樂和禮備,候暖風春。恭惟降福,實賴明神。



 南呂羽禮部侍郎韓休作:



 於穆浚哲,維清緝熙。肅事昭配,永言孝思。滌濯靜嘉,馨香在茲。神之聽之,用受福釐。



 皇帝行用《太和》黃鐘宮吏部尚書王鷿作:



 於穆聖皇,六葉重光。太原刻頌,後土疏場。寶鼎呈符,歊雲降祥。禮樂備矣,降福穰穰。



 登歌奠玉帛用《肅和》蕤賓均之夾鐘羽刑部侍郎崔玄暐作:



 聿修嚴配,展事禋宗。祥符寶鼎,禮備黃琮。祝詞以信,明德惟聰。介茲景福,永永無窮。



 迎俎用《雍和》黃鐘均之南呂羽徐州刺史賈曾作:



 蠲我餴饎,潔我膋薌。有豆孔碩,為羞既臧。至誠無昧,精意惟芳。神其醉止,欣欣樂康。



 酌獻飲福用《壽和》黃鐘宮禮部尚書蘇頲作:



 禮物斯備,樂章乃陳。誰其作主,皇考聖真。對越在天,聖明佐神。窅然汾上,厚澤如春。



 送文舞出迎武舞入用《舒和》太簇宮太常少卿何鸞作:



 樂奏雲闋,禮章載虔。禋宗於地,昭假於天。惟馨薦矣,既醉歆焉。神之降福,永永萬年。



 武舞用《凱安》黃鐘均之林鐘徵主爵郎中蔣挺作:



 維歲之吉,維辰之良。聖君紱冕,肅事壇場。大禮已備,大樂斯張。神其醉止,降福無疆。



 送神用《順和》尚書右丞源光裕作:



 方丘既膳,嘉饗載謐。齊敬畢誠,陶匏貴質。秀簠豐薦,芳俎盈實。永永福流,其升如日。



 玄宗開元十三年禪社首山祭地祇樂章八首



 迎神用《順和》太常少卿賀知章作:



 至哉含柔德,萬物資以生。常順稱厚載,流謙通變盈。聖
 心事能察,層廟陳厥誠。黃祇儼如在,泰折俟咸亨。



 皇帝行用《太和》:



 肅我成命,於昭黃祇。裘冕而祀,陟降在斯。五音克備,八變聿施。緝熙肆靖,厥心匪離。



 登歌奠玉帛用《肅和》:



 黃祇是祗,我其夙夜。夤畏誠潔,匪遑寧舍。禮以琮玉,薦厥茅藉。念茲降康,胡寧克暇。



 迎俎入用《雍和》:



 夙夜宥密,不敢寧宴。五齊既陳,八音在縣。粢盛以潔,房俎斯薦。惟德惟馨,尚茲克遍。



 皇帝酌獻用《壽和》:



 惟以明發,有懷載殷。樂盈而反,禮順其禋。立清以獻,薦欲是親。於穆不已,裒對斯臻。



 皇帝飲福用《福和》:



 穆穆天子,告成岱宗。大裘如濡,執珽有顒。樂以平志,禮以和容。上帝臨我,云胡肅邕。



 皇帝還宮用《太和》:



 昭昭有唐,天俾萬國。列祖應命,四宗順則。申錫無疆,宗我同德。曾孫繼緒,享神配極。



 迎神用《靈具醉》代順和,侍中源乾曜作。:



 靈具醉,杳熙熙。靈將往,眇禗禗。顧明德,吐正詞。爛遺光,流禎祺。



 祭神州於北郊樂章八首貞觀中褚亮作



 送神用《順和》詞同夏至方丘:



 皇帝行用《太和》詞同冬至圓丘。



 登歌奠玉帛用《肅和》:



 大矣坤儀,至哉神縣。包含日域,牢籠月HC。露潔三清,風調六變。皇祇屆止,式歆恭薦。



 迎俎用《雍和》:



 泰折嚴享,陰郊展敬。禮以導神,樂以和性。黝牲在列,黃琮俯映。九土既平,萬邦貽慶。



 皇帝酌獻飲福用《壽和》詞同冬至圓丘。



 送文舞出迎武舞入用《舒和》:



 坤道降祥和庶品,靈心載德厚群生。水土既調三極泰,文武畢備九區平。



 武舞用《凱安》詞同冬至圓丘。



 送神用《順和》詞同冬至圓丘。



 又祭神州樂章二首太樂舊有此詞,不詳所起。



 迎神:



 黃輿厚載,赤寰歸德。含育九區,保安萬國。誠敬無怠,禋祀有則。樂以迎神,其儀不忒。



 送神:



 神州陰祀,洪恩廣濟。草樹沾和,飛沉沐惠。禮修鼎俎,奠歆瑤幣。送樂有章,靈軒其逝。



 祭太社樂章八首貞觀中褚亮等作



 迎神用《順和》詞同夏至方丘。



 皇帝行用《太和》詞同冬至圓丘。



 登歌奠玉帛用《肅和》:



 後土凝德,神功葉契。九域底平,兩儀交際。戊期應序,陰墉展幣。靈車少留,俯歆樽桂。



 迎俎用《雍和》:



 美報崇本,嚴恭展事。受露疏壇,承風啟地。潔粢登俎,醇犧入饋。介福遠流,群生畢遂。



 皇帝酌獻飲福用《壽和》詞同冬至圓丘。



 送文舞出迎武舞入用《舒和》:



 神道發生敷九稼,陰陽乘仁暢八埏。緯武經文陶景化,登祥薦祉啟豐年。



 武舞用《凱安》詞同冬至圓丘。



 送神用《順和》詞同冬至圓丘。



 又太社樂章二首太樂舊有此詞,
 不詳所起。



 迎神:



 烈山有子,後土有臣。播種百穀,濟育兆人。春官緝禮,宗伯司禋。戊為吉日,迎享茲辰。



 送神:



 告祥式就,酬功載畢。親地尊天,禮文經術。貺征令序,福流初日。神馭爰歸,祠官其出。



 享先農樂章貞觀中褚亮等作



 迎神用《咸和》:



 粒食伊始,農之所先。古今攸賴,是曰人天。耕斯帝藉,播厥公田。式崇明祀,神其福焉。



 皇帝行用《太和》詞同冬至圓丘。



 登歌奠玉帛用《肅和》:



 尊彞既列,瑚簋有薦。歌工載登,幣禮斯奠。肅肅享祀,顒顒纓弁。神之聽之,福流寰縣。



 迎俎用《雍和》:



 前夕親牲,質明奉俎。沐芳整弁,其儀式序。盛禮畢陳,嘉
 樂備舉。歆我懿德,非馨稷黍。



 皇帝酌獻飲福用《壽和》詞同冬至圓丘。



 送文舞出迎武舞入用《舒和》:



 羽龠低昂文綴已,干戚蹈厲武行初。望歲祈農神所聽,延祥介福豈云虛。



 武舞用《凱安》詞同冬至圓丘。



 送神用《承和》:



 又享先農樂章一首太樂舊有此詞,不詳所起。



 送神用《承和》:



 三推禮就,萬庾祈凝。夤賓志遠,藨濆惟興。降歆肅薦,垂祐祗膺。送神有樂,神其上升。



 享先蠶樂章五首顯慶中,皇后親蠶,奉敕內出此詞。



 迎神用《永和》亦曰《順德》:



 芳春開令序,韶苑暢和風。惟靈申廣祐,利物表神功。綺會周天宇,黼黻藻寰中。庶幾承慶節,歆奠下帷宮。



 皇后升壇用《肅和》:



 明靈光至德,深功掩百神。祥源應節啟,福緒逐年新。萬
 宇承恩覆,七廟佇恭禋。於茲申至懇,方期遠慶臻。



 登歌奠幣用《展敬》:



 霞莊列寶衛,雲集動和聲。金卮薦綺席,玉幣委芳庭。因心罄丹款,先己勵蒼生。所冀延明福,於茲享至誠。



 迎俎用《潔誠》:



 桂筵開玉俎,蘭圃薦瓊芳。八音調鳳律,三獻奉鸞觴。潔粢申大享,庭宇冀降祥。神其覃有慶,錫福永無疆。



 飲福送神用《昭慶》:



 仙壇禮既畢,神駕儼將升。佇屬深祥啟,方期庶績凝。虔誠資宇內,務本勖黎蒸。靈心昭備享,率土洽休徵。



 皇太子親釋奠樂章五首



 迎神用承和亦曰《宣和》:



 聖道日用,神機不測。金石以陳,弦歌載陟。爰釋其菜,匪馨於稷。來顧來享,是宗是極。



 皇太子行用《承和》:



 萬國以貞光上嗣,三善茂德表重輪。視膳寢門遵要道,
 高闢崇賢引正人。



 登歌奠幣用《肅和》:



 粵惟上聖,有縱自天。旁周萬物,俯應千年。舊章允著,嘉贄孔虔。王化茲首,儒風是宣。



 迎俎用《雍和》:



 堂獻瑤篚,庭敷璆縣。禮備其容,樂和其變。肅肅親享,雍雍執奠。明禮惟馨,蘋蘩可薦。



 送文舞出迎武舞入用《舒和》:



 隼集龜開昭聖列,龍蹲鳳歭肅神儀。尊儒敬業宏圖闡,緯武經文盛德施。



 武舞用《凱安》詞同冬至圓丘。



 送神用《承和》詞同迎神:



 又享孔廟樂章二首太樂舊有此詞,不詳所起。



 迎神:



 通吳表聖,問老探貞。三千弟子,五百賢人。億齡規法,萬載祠禋。潔誠以祭,奏樂迎神。



 送神:



 醴溢犧象,羞陳俎豆。魯壁類聞,泗川如覯。裏校覃福,胄筵承祐。雅樂清音,送神其奏。



 享龍池樂章十首



 第一章紫微令姚崇作也:



 恭聞帝裏生靈沼,應報明君鼎業新。既葉翠泉光寶命,還符白水出真人。此時舜海潛龍躍,北地堯河帶馬巡。獨有前池一小雁,叨承舊惠入天津。



 第二章左拾遺蔡孚作:



 帝宅王家大道邊,神馬龍龜湧聖泉。昔日昔時經此地,看來看去漸成川。歌臺舞榭宜正月,柳岸梅洲勝往年。莫言波上春雲少,只為從龍直上天。



 第三章太府少卿沈佺期作:



 龍池躍龍龍已飛,龍德先天天不違。池開天漢分黃道,龍向天門入紫微。邸第樓臺多氣色,君王鳧雁有光輝。為報寰中百川水,來朝上地莫東歸。



 第四章黃門侍郎盧懷慎作:



 代邸東南龍躍泉,清漪碧浪遠浮天。樓臺影就波中出,日月光疑鏡裏懸。雁沼回流成舜海,龜書薦祉應堯年。大川既濟慚為楫,報德空思奉細涓。



 第五章殿中監姜皎作:



 龍池初出此龍山,常經此地謁龍顏。日日芙蓉生夏水,年年楊柳變春灣。堯壇寶匣餘煙霧,舜海漁舟尚往還。願以飄颻五雲影,從來從去九天間。



 第六章吏部尚書崔日用作:



 龍興白水漢興符,聖主時乘運斗樞。岸上裛茸五花樹,波中的皪千金珠。操環昔聞迎夏啟,發匣先來瑞有虞。風色雲光隨隱見,亦云神化象江湖。



 第七章紫微侍郎蘇頲作:



 西京鳳邸躍龍泉,佳氣休光鐘在天。軒後霧圖今已得,秦王水劍昔常傳。恩魚不似昆明釣,瑞鶴長如太液仙。願侍巡游同舊里,更聞簫鼓濟樓船。



 第八章黃門侍郎李乂
 作:



 星分邑里四人居,水洊源流萬頃餘。魏國君王稱象處,晉家籓邸化龍初。青蒲暫似游梁馬,綠藻還疑宴鎬魚。自有神靈滋液地,年年雲物史官書。



 第九章工部侍郎姜晞作:



 靈沼縈回邸第前,浴日涵春寫曙天。始見龍臺升鳳闕,應如霄漢起神泉。石匱渚傍還啟聖,桃李初開更有仙。欲化帝圖從此受,正同河變一千年。



 第十章兵部郎中裴璀作:



 乾坤啟聖吐龍泉,泉水年年勝一年。始看魚躍方成海,即睹龍飛利在天。洲渚遙將銀漢接,樓臺直與紫微連。休氣榮光常不散,懸知此地是神仙。



\end{pinyinscope}