\article{卷一百四十五}

\begin{pinyinscope}

 ○田承嗣侄悅子緒緒子季安田弘正子布牟布子在宥張孝忠子茂昭茂昭子克勤弟茂宗茂和陳楚附



 田承嗣,平州人,世事盧龍軍為裨校。祖璟,父守義,以豪俠聞於遼、碣。承嗣,開元末為軍使安祿山前鋒兵馬使,
 累俘斬奚、契丹功,補左清道府率,遷武衛將軍。祿山構逆,承嗣與張忠志等為前鋒,陷河洛。祿山敗,史朝義再陷洛陽,承嗣為前導,偽授魏州刺史。代宗遣朔方節度使僕固懷恩引回紇軍討平河朔。帝以二兇繼亂,郡邑傷殘,務在禁暴戢兵,屢行赦宥,凡為安、史詿誤者,一切不問。時懷恩陰圖不軌,慮賊平寵衰,欲留賊將為援,乃奏承嗣及李懷仙、張忠志、薛嵩等四人分帥河北諸郡,乃以承嗣檢校戶部尚書、鄭州刺史。俄遷魏州刺史、貝
 博滄瀛等州防禦使。居無何,授魏博節度使。



 承嗣不習教義,沉猜好勇,雖外受朝旨,而陰圖自固。重加稅率,修繕兵甲;計戶口之眾寡,而老弱事耕稼,丁壯從征役,故數年之間,其眾十萬。仍選其魁偉強力者萬人以自衛,謂之衙兵。郡邑官吏,皆自署置。戶版不籍於天府,稅賦不入於朝廷,雖曰籓臣,實無臣節。代宗以黎元久罹寇虐,姑務優容,累加檢校尚書僕射、太尉、同中書門下平章事,封雁門郡王,賜實封千戶。及升魏州為大都督府,
 以承嗣為長史,仍以其子華尚永樂公主,冀以結固其心,庶其悛革。而生於朔野,志性兇逆,每王人慰安,言詞不遜。



 大歷八年,相衛節度使薛嵩卒,其弟崿欲邀旄節;及用李承昭代嵩,衙將裴志清謀亂逐崿,崿率眾歸於承嗣。十年,薛崿歸朝,承嗣使親黨扇惑相州將吏謀亂,遂將兵襲擊,謬稱救應。代宗遣中使孫知在使魏州宣慰,令各守封疆。承嗣不奉詔,遣大將盧子期攻洺州,楊光朝攻衛州。殺刺史薛雄,仍逼知在令巡磁、相二州,諷
 其大將割耳剺面,請承嗣為帥,知在不能詰。四月,詔曰:



 田承嗣出自行間,策名邊戍,早參戎秩,效用無聞,嘗輔兇渠,驅馳有素。洎再平河朔,歸命轅門。朝廷俯念遺黎,久罹兵革。自祿山召禍,瀛、博流離;思明繼釁,趙、魏堙厄;以至農桑井邑,靡獲安居,骨肉室家,不能相保。念其凋瘵,思用撫寧,以其先布款誠,寄之為理。所以委授旄鉞之任,假以方面之榮,期爾知恩,庶能自效。崇資茂賞,首冠朝倫,列異姓之苴茅,登上公之禮命。子弟童稚,皆聯
 臺閣之華;妻妾僕媵,並受國邑之號。人臣之寵,舉集其門;將相之權,兼領其職。



 夫宰相者,所以盡忠,而乃據國家之封壤,仗國家之兵戈,安國家之黎人,調國家之征賦。掩有資實,憑竊寵靈,內包兇邪,外示歸順。且相、衛之略,所管素殊,而逼脅軍人,使之翻潰。因其驚擾,便進軍師,事跡暴彰,奸邪可見。不然,豈志清之亂,曾未崇朝;子期、光朝,會於明日。足知先有成約,指期而來,是為蔑棄典刑,擅興戈甲。既云相州騷擾,鄰境救災,旋又更取磁
 州,重行威虐。此實自矛盾,不究始終。三州既空,遠邇驚陷,更移兵馬,又赴洺州,實為暴惡不仁,窮極殘忍。



 薛雄乃衛州刺史,固非本籓,忿其不附,橫加凌虐,一門盡屠,非復噍類,酷烈無狀,人神所冤。又四州之地,皆列屯營,長史屬官,任情補署。精甲利刃,良馬勁兵,全實之資裝,農藏之積實,盡收魏府,罔有孑遺。其為蓋在無赦,欲行討問,正厥刑書。猶示含容,冀其遷善,抑於典憲,務在慰安。乃遣知在遠奉詔書,諭以深旨,乃命承昭副茲麾下,
 撫彼舊封。而承昭又遣親將劉渾先傳詔命。承嗣逡巡磁、相,仍劫知在偕行,先令侄悅權扇軍吏,至使引刀自割,抑令騰口相稽,當眾喧嘩,請歸承嗣。論其奸狀,足以為憑,此而可容,何者為罪?



 承嗣宜貶永州刺史,仍許一幼男女從行,便路赴任。委河東節度使薛兼訓、成德軍節度使李寶臣、幽州節度留後硃滔、昭義節度李承昭、淄青節度李正己、淮西節度李忠臣、永平軍節度使李勉、汴宋節度田神玉等,掎角進軍。如承嗣不時就職,所
 在加討,按軍法處分。



 詔下,承嗣懼;而麾下大將,復多攜貳,倉黃失圖。乃遣牙將郝光朝奉表請罪,乞束身歸朝。代宗重勞師旅,特恩詔允,並侄悅等悉復舊官,仍詔不須入覲。



 十一年,汴將李靈曜據城叛,詔近鎮加兵。靈曜求援於魏。承嗣令田悅率眾五千赴之,為馬燧、李忠臣逆擊敗之;悅僅而獲免,兵士死者十七八,復詔誅之。十二年,承嗣復上章請罪,又赦之,復其官爵。承嗣有貝、博、魏、衛、相、磁、洺等七州,復為七州節度使,於是承嗣弟廷
 琳及從子悅、承嗣子綰、緒等皆復本官,仍令給事中杜亞宣諭,賜鐵券。



 十三年九月,卒,時年七十五。有子十一人:維、朝、華、繹、綸、綰、緒、繪、純、紳、縉等。維為魏州刺史;朝,神武將軍;華,太常少卿、駙馬都尉,尚永樂公主,再尚新都公主;餘子皆幼。而悅勇冠軍中,承嗣愛其才,及將卒,命悅知軍事,而諸子佐之。



 悅初為魏博中軍兵馬使、檢校右散騎常侍、魏府左司馬。大歷十三年,承嗣卒,朝廷用悅為節度留後。驍勇有膂力,性殘忍好亂,而能外飾行
 義,傾財散施,人多附之,故得兵柄。尋拜檢校工部尚書、御史大夫,充魏博七州節度使。大歷末,悅尚恭順。建中初,黜陟使洪經綸至河北,方聞悅軍七萬。經綸素昧時機,先以符停其兵四萬,令歸農畝。悅偽亦順命,即依符罷之。既而大集所罷將士,激怒之曰:「爾等久在軍戎,各有父母妻子,既為黜陟使所罷,如何得衣食自資?」眾遂大哭。悅乃盡出其家財帛衣服以給之,各令還其部伍。自此魏博感悅而怨朝廷。



 居無何,或謬稱車駕將東封,
 而李勉增廣汴州城。李正己聞而猜懼,以兵萬人屯曹州,遣使說悅,同為拒命。悅乃與正己、梁崇義等謀各阻兵,以判官王侑、扈萼、許士則為腹心,邢曹俊、孟希祐、李長春、符璘、康愔為爪牙。建中二年,鎮州李寶臣卒,子惟岳求襲節鉞。俄而淄青李正己卒,子納亦求節鉞。朝廷皆不允,遂與惟岳、李納同謀叛逆。時朝廷遣張孝忠等討恆州,悅將孟希祐率兵五千援之。又遣將康愔率兵八千圍邢州,楊朝光五千人營於邯鄲西北盧家砦,絕
 昭義糧餉之路,悅自將兵甲數萬繼進。邢州刺史李洪、臨洺將張伾,為賊所攻,御備將竭,詔河東節度使馬燧、河陽李芃,與昭義軍討悅。七月三日,師自壺關東下,收賊盧家砦,大破賊於雙岡;邢州解圍,悅眾遁走,保洹水。馬燧等三帥距悅軍三十里為壘,李納遣兵八千人助悅。



 魏將邢曹俊者,承嗣之舊將,老而多智,頗知兵法,悅暱于扈萼,以曹俊為貝州刺史。及悅拒官軍於臨洺,大為王師所破,悅乃召曹俊而問計焉。曹俊曰:「兵法十倍
 則攻,尚書以逆犯順,勢且不侔。宜於郭口置兵萬人以遏西師,則河北二十四州悉為尚書有矣。今於臨洺、武安設攻城之計,糧竭卒盡,危兇立至,未見其可也。」祐等以其異己,咸譖毀,悅復令守貝州。



 悅與淄青兵三萬餘人陣於洹水,馬燧等三帥與神策將李晟等來攻,悅之眾復敗,死傷二萬計。悅收合殘卒奔魏州,至南郭外,大將李長春拒關不內,以俟官軍。三帥雖進,頓兵於魏州南平邑浮圖,咸遲留不進,長春乃開門內之。悅持佩刀
 立於軍門,謂軍士百姓曰:「悅藉伯父餘業,久與卿等同事,今既敗喪相繼,不敢圖全。然悅所以堅拒天誅者,特以淄青、恆冀二大人在日,為悅保薦於先朝,方獲承襲。今二帥雲亡,子弟求襲,悅既不能報效,以至興師。今軍旅敗亡,士民塗炭,此皆悅之罪也。以母親之故,不能自剄,公等當斬悅首以取功勛,無為俱死也!」乃自馬投地,眾皆憐之。或前撫持悅曰:「久蒙公恩,不忍聞此!今士民之眾,猶可一戰,生死以之。」悅收涕言曰:「諸公不以悅喪
 敗,猶願同心,悅縱身死,寧忘厚意於地下乎!」悅乃自割一髻,以為要誓,於是將士自斷其髻,結為兄弟,誓同生死。其將符璘、李再春、李瑤,悅從兄昂,相次以郡邑歸國。璘等家在魏州者,無少長悉為悅所害。悅觀城內兵仗罄乏,士眾衰減,甚為惶駭,乃復召邢曹俊與之謀。既至,完整徒旅,繕修營壁,人心復堅。經旬餘日,馬燧等進至城下。向使燧等乘勝長驅,襲其未備,則魏城屠之久矣!識者痛惜之。



 會王武俊殺李惟岳,硃滔攻深州,下之,朝
 廷以武俊為恆州刺史,又以寶臣故將康日知為深趙二州觀察使。是以武俊怨賞功在日知下,硃滔怨不得深州,二將有憾於朝廷。悅知其可間,遣判官王侑、許士則使於北軍,說硃滔曰:「昨者司徒奉詔征伐,徑趨賊境。旬朔之內,拔束鹿,下深州,惟嶽勢蹙,故王大夫獲殄兇渠,皆因司徒勝勢。又聞司徒離幽州日,有詔得惟岳郡縣,使隸本鎮;今割深州與日知,是國家無信於天下也。且今上英武獨斷,有秦皇、漢武之才,誅夷豪傑,欲掃除
 河朔,不令子孫嗣襲。又朝臣立功立事如劉晏輩,皆被屠滅。昨破梁崇義,殺三百餘口,投之漢江,此司徒之所明知也。如馬燧、抱真等破魏博後,朝廷必以儒德大臣以鎮之,則燕、趙之危可翹足而待也。若魏博全,則燕、趙無患,田尚書必以死報恩義。合從連衡,救災恤患,《春秋》之義也。春秋時諸侯有危者,桓公不能救則恥之。今司徒聲振宇宙,雄略命世,救鄰之急,非徒立義,且有利也。尚書以貝州奉司徒,命某送孔目,惟司徒熟計之。」滔既
 有貳於國,欣然從之。乃命判官王郢與許士則同往恆州說王武俊,仍許還武俊深州。武俊大喜,即令判官王巨源報滔,仍知深州事。武俊又說張孝忠同援悅,孝忠不從,恐為後患,乃遣小校鄭朅築壘於北境,以拒孝忠;仍令其子士真為恆、冀、深三州留後,以兵圍趙州。



 三年五月,悅以救軍將至,率其眾出戰於御河之上,大敗而還。四月,硃滔、武俊蒐軍於寧晉縣,共步騎四萬。五月十四日,起軍南下,次宗城,滔判官鄭雲逵及弟方逵背滔
 歸馬燧。六月二十八日,滔、武俊之師至魏州,會神策將李懷光軍亦至。懷光銳氣不可遏,堅欲與賊戰,遂徑薄硃滔陣,殺千餘人。王武俊與騎將趙琳、趙萬敵等二千騎橫擊懷光陣,滔軍繼踵而進,禁軍大敗,人相蹈藉,投尸於河三十里,河水為之不流。馬燧等收軍保壘。是夜,王武俊決河水入王莽故河,欲隔官軍,水已深三尺,糧餉路絕。王師計無從出,乃遣人告硃滔曰:「鄙夫輒不自量,與諸人合戰。王大夫善戰,天下無敵;司徒五郎與王
 君圖之,放老夫歸鎮,必得聞奏,以河北之事委五郎。」時武俊戰勝,滔心忌之,即曰:「大夫二兄敗官軍,馬司徒卑屈若此,不宜迫人於險也。」武俊曰:「燧等連兵十萬,皆是國之名臣,一戰而北,貽國之恥,不知此等何面見天子耶!然吾不惜放還,但不行五十里,必反相拒。」燧等至魏縣,軍於河西;武俊等三將,壁於河東。兩軍相持,自七月至十月,勝負未決。



 悅感硃滔救助,欲推為盟主。滔判官李子牟、武俊判官鄭儒等議曰:「古有戰國,連衡誓約以
 抗秦,請依周末七雄故事,並建國號為諸侯,用國家正朔。今年號不可改也。」於是硃滔稱冀王,悅稱魏王,武俊稱趙王,又請李納稱齊王。十一月一日,築壇於魏縣中,告天受之。滔為盟主,稱孤;武俊、悅、納稱寡人。滔以幽州為範陽府,恆州為真定府,魏州為大名府,鄆州為東平府,皆以長子為元帥。偽冊之日,其軍上有雲物稍異,馬燧等望而笑曰:「此云無知,乃為賊瑞。」又其營地前三年土長高三尺餘,魏州戶曹韋稔為《土長頌》曰:「益土之兆
 也。」



 四年十月,涇師犯闕,諸師各還本鎮。悅、滔、武俊互相疑惑,各去王號,遣使歸國。悅亦致書於抱真,遣使聞奏。興元元年正月,加悅檢校尚書右僕射,封濟陽王,使並如故。仍令給事中、兼御史大夫孔巢父往魏州宣慰。時悅阻兵四年,身雖驍猛,而性愎無謀。以故頻致破敗,士眾死者十七八。魏人苦於兵革,願息肩焉;聞巢父至,莫不舞忭。悅方宴巢父,為其從弟緒所殺。



 緒,承嗣第六子。大歷末,授京兆府參軍。承嗣卒時,緒年幼稚。承嗣慮諸
 子不任軍政,以從子悅便弓馬,性狡黠,故任遇之,俾代為帥守。及緒年長,悅以承嗣委遇之厚,待緒等無間,令主衙軍。緒兇險多過,悅不忍,嘗笞而拘之。緒頗怨望,常俟釁隙。會興元元年,朝廷宥悅,仍令孔巢父往宣慰。悅既順命,門階徹警。悅宴巢父夜歸,緒率左右數十人先殺悅腹心蔡濟、扈崿、許士則等,挺劍而入。其兩弟止之;緒斬止者,遂徑升堂。悅方沉醉,緒手刃悅並悅妻高氏,又入別院殺悅母馬氏。自河北諸盜殘害骨肉,無酷於
 緒者。緒懼眾不附,奔出北門。邢曹俊、孟希祐等領徒數百追及之。遙呼之曰:「節度使須郎君為之,他人固不可也。」乃以緒歸衙,推為留後。明日,歸罪於扈崿,以其首徇;然後稟於孔巢父,遣使以聞。時緒兄綸居長,為亂兵所殺,遂以緒為留後。朝廷授緒銀青光祿大夫、魏州大都督府長史、兼御史大夫、魏博節度使。時硃滔率兵兼引回紇之眾南侵,緒遣兵助王武俊、李抱真,大破硃滔於涇城,以功授檢校工部尚書。貞元元年,以嘉誠公主出
 降緒,加駙馬都尉。尋遷檢校左僕射,封常山郡王,食邑三千戶。改封雁門郡王,食實封五百戶。尋加同平章事。



 初,田悅性儉嗇,衣服飲食,皆有節度;而緒等兄弟,心常不足。緒既得志,頗縱豪侈,酒色無度。貞元十二年四月,暴卒,時年三十三,贈司空,賻賚加等。



 子三人:季和、季直、季安。季和為澶州刺史;季直為衙將;季安最幼,為嫡嗣。



 季安,字夔。母微賤,嘉誠公主蓄為己子,故寵異諸兄。年數歲,授左衛胄曹參軍,改著作佐郎、兼侍御史,充魏博
 節度副大使。累加至試光祿少卿、兼御史大夫。緒卒時,季安年才十五,軍人推為留後,朝廷因授起復左金吾衛將軍,兼魏州大都督府長史、魏博節度營田觀察處置等使。服闕,拜銀青光祿大夫、檢校尚書右僕射,進位檢校司空,襲封雁門郡王。未幾,加金紫光祿大夫,以本官同中書門下平章事。



 季安幼守父業,懼嘉誠之嚴,雖無他才能,亦粗修禮法。及公主薨,遂頗自恣,擊鞠、從禽色之娛。其軍中政務,大抵任徇情意,賓僚將校,言皆不
 從。免公主喪,加檢校司徒。元和中,王承宗擅襲戎帥,憲宗命吐突承璀為招撫使,會諸軍進討。季安亦遣大將率兵赴會,仍自供糧餉。師還,加太子太保。



 季安性忍酷,無所畏懼。有進士丘絳者,嘗為田緒從事,及季安為帥,絳與同職侯臧不協,相持爭權。季安怒,斥絳為下縣尉;使人召還,先掘坎於路左,既至坎所,活排而瘞之,其兇暴如此!元和七年卒,時年三十二,贈太尉。子懷諫、懷禮、懷詢、懷讓。



 懷諫母,元誼女。及季安卒,元氏召諸將欲立
 懷諫,眾皆唯唯。懷諫幼,未能御事,軍政無巨細皆取決於私白身蔣士則,數以愛憎移易將校。衙軍怒,取前臨清鎮將田興為留後,遣懷諫歸第,殺蔣士則等十餘人。田興葬季安畢,送懷諫於京師,乃起復授右監門衛將軍,賜第一區,芻米甚厚。田氏自承嗣據魏州至懷諫,四世相傳襲四十九年,而田興代焉。



 田弘正,本名興。祖延惲,魏博節度使承嗣之季父也,位終安東都護府司馬。延惲生廷玠,幼敦儒雅,不樂軍職,
 起家為平舒丞。遷樂壽、清池、束城、河間四縣令,所至以良吏稱。大歷中,累官至太府卿、滄州別駕,遷滄州刺史、兼御史中丞,充橫海軍使。承嗣與淄青李正己、恆州李寶臣不協,承嗣既令廷玠守滄州,而寶臣、硃滔兵攻擊,欲兼其土宇。廷玠嬰城固守,連年受敵,兵盡食竭,人易子而食,卒無叛者,卒能保全城守。朝廷嘉之,遷洺州刺史,又改相州。屬薛崿之亂,承嗣蠶食薛嵩所部。廷玠守正字民,不以宗門回避而改節。建中初,族侄悅代承
 嗣領軍政,志圖兇逆,慮廷玠不從,召為節度副使。悅奸謀頗露,廷玠謂悅曰:「爾藉伯父遺業,可稟守朝廷法度,坐享富貴,何苦與恆、鄆同為叛臣?自兵亂已來,謀叛國家者,可以歷數,鮮有保完宗族者。爾若狂志不悛,可先殺我,無令我見田氏之赤族也。」乃謝病不出。悅過其第而謝之;廷玠杜門不納,將吏請納。建中三年,鬱憤而卒。



 弘正,廷玠之第二子。少習儒書,頗通兵法,善騎射,勇而有禮,伯父承嗣愛重之。當季安之世,為衙內兵馬使。季
 安惟務侈靡,不恤軍務,屢行殺罰;弘正每從容規諷,軍中甚賴之。季安以人情歸附,乃出為臨清鎮將,欲捃摭其過害之。弘正假以風痺請告,灸灼滿身,季安謂其無能為。及季安病篤,其子懷諫幼騃,乃召弘正署其舊職。



 季安卒,懷諫委家僮蔣士則改易軍政,人情不悅,咸曰:「都知兵馬使田興,可為吾帥也!」衙兵數千詣興私第陳請,興拒關不出,眾呼噪不已。興出,眾環而拜,請入府署。興頓僕於地,久之。度終不免,乃令於軍中曰:「三軍不以
 興不肖,令主軍務,欲與諸軍前約,當聽命否?」咸曰:「惟命是從!」興曰:「吾欲守天子法,以六州版籍請吏,勿犯副大使,可乎?」皆曰:「諾!」是日,入府視事,殺蔣士則十數人而已。晚自府歸第,其兄融責興曰:「爾卒不能自晦,取禍之道也!」翌日,具事上聞。憲宗嘉之,加興銀青光祿大夫、檢校工部尚書、魏州大都督府長史、兼御史大夫、上柱國、沂國公,充魏、博等州節度觀察、處置、支度、營田等使,仍賜名弘正。仍令中書舍人裴度使魏州宣慰,賜魏博三軍
 賞錢一百五十萬貫。



 弘正既受節鉞,上表曰:



 臣聞君臣父子,是謂大倫,爰立紀綱,以正上下。其或子不為子,臣不為臣,覆載莫可得容,幽明所宜共殛。臣家本邊塞,累代唐人;從乃祖乃父以來,沐文子文孫之化。臣幸因宗族,早列偏裨,驅馳戎馬之鄉,不睹朝廷之禮。惟忠與孝,天與臣心。常思奮不顧生,以身殉國,無由上達,私自感傷。豈意命偶昌時,事緣難故,白刃之下,謬見推崇。天慈遽臨,免書罪累,朝章薦及,仍委旂旄。錫封壤於全籓,列
 班榮於八座;君父之恩已極,絲毫之效未伸,但以靦冒知羞,低回自愧。是知功榮所著,必俟危亂之時;徼幸之來,卻在清平之日。循涯揣分,以寵為憂。伏自天寶已還,幽陵肇亂,山東奧壤,悉化戎墟。外撫車馬,內懷梟獍,官封代襲,刑賞自專,國家含垢匿瑕,垂六十載。臣每思此事,當食忘餐。若稍假天年,得奉宸算,兼弱攻昧,批亢搗虛;竭鷹犬之資,展獲禽之用,導揚和氣,洗滌偽風,然後退歸田園,以避賢路。臣懷此志,陛下察之!



 優詔褒美。



 弘
 正樂聞前代忠孝立功之事,於府舍起書樓,聚書萬餘卷,視事之隙,與賓佐講論古今言行可否。今河朔有《沂公史例》十卷,弘正客為弘正所著也。魏州自承嗣已來,館宇服玩有逾常制者,悉命徹毀之,以正大侈不居,乃視事於採訪使。賓僚參佐,請之於朝。頗好儒書,尤能史氏,《左傳》、《國史》,知其大略。



 自弘正歸國,幽、恆、軍阜、蔡有齒寒之懼,屢遣客間說,多方誘阻,而弘正終始不移其操。裴度明理體,詞說雄辯;弘正聽其言,終夕不倦。遂深
 相結納,由是奉上之意逾謹。元和十年,朝廷用兵討吳元濟,弘正遣子布率兵三千進討,屢戰有功。李師道以弘正效忠,又襲其後,不敢顯助元濟,故絕其掎角之援,王師得致討焉。俄而王承宗叛,詔弘正以全師壓境。承宗懼,遣使求救於弘正,遂表其事,承宗遂納二子,獻德、棣二州以自解。



 十三年,王師加兵於鄆,詔弘正與宣武、義成、武寧、橫海等五鎮之師會軍齊進。十一月,弘正自帥全師自楊劉渡河築壘,距鄆四十里。師道遣大將劉
 悟率重兵以抗弘正,結壘相望。前後合戰,魏軍大捷。而李醖、李光顏三面進攻,賊皆挫敗,其勢將危。十四年三月,劉悟以河上之眾倒戈入鄆,斬師道首,詣弘正請降。淄青十二州平,論功加檢校司徒、同中書門下平章事。是年八月,弘正入覲,憲宗待之隆異,對於麟德殿,參佐將校二百餘人皆有頒錫,進加檢校司徒、兼侍中,實封三百戶。仍以其兄檢校刑部尚書、相州刺史融為太子賓客,東都留司。弘正三上章,願留闕下,憲宗勞之曰:「昨
 韓弘至朝,稱疾懇辭戎務,朕不得不從。今卿復請留,意誠可尚,然魏土樂卿之政,鄰境服卿之威,為我長城,不可辭也。可亟歸籓。」弘正每懼有一旦之憂,嗣襲之風不革,兄弟子侄,悉仕於朝,憲宗皆擢居班列,硃紫盈庭,當時榮之。



 十五年十月,鎮州王承宗卒,穆宗以弘正檢校司徒、兼中書令、鎮州大都督府長史,充成德軍節度、鎮冀深趙觀察等使。弘正以新與鎮人戰伐,有父兄之怨,乃以魏兵二千為衛從。十一月二十六日,至鎮州,時賜
 鎮州三軍賞錢一百萬貫,不時至,軍眾喧騰以為言。弘正親自撫喻,人情稍安。仍表請留魏兵為紀綱之僕,以持眾心,其糧賜請給於有司。時度支使崔倰不知大體,固阻其請,凡四上表不報。明年七月,歸卒於魏州,是月二十八日夜軍亂,弘正並家屬、參佐、將吏等三百餘口並遇害。穆宗聞之震悼,冊贈太尉,賵賻加等。弘正孝友慈惠,骨肉之恩甚厚。兄弟子侄在兩都者數十人,競為崇飾,日費約二十萬,魏、鎮州之財,皆輦屬於道。河北將
 卒心不平之,故不能盡變其俗,竟以此致亂。弘正子布、群、牟。



 布,弘正第三子。始,弘正為田季安裨將,鎮臨清,布年尚幼,知季安身世必危,密白其父帥其所鎮之眾歸朝,弘正甚奇之。及弘正節制魏博,布掌親兵,國家討淮、蔡,布率偏師隸嚴綬,軍於唐州,授檢校秘書監、兼殿中侍御史。前後十八戰,破凌雲柵,下郾城,布皆有功,擢授御史中丞。時裴度為宣撫使,嘗觀兵於沱口,賊將董重質領驍騎遽至,布以二百騎突出溝中擊之;俄而諸軍
 大集,賊乃退去。淮西平,拜左金吾衛將軍、兼御史大夫。十三年,丁母憂,起復舊官。十五年冬,弘正移鎮成德軍,仍以布為河陽三城懷節度使,父子俱擁節旄,同日拜命。時韓弘亦與子公武俱為節度使,然人以忠勤多田氏。



 長慶元年春,移鎮涇原。其秋,鎮州軍亂,害弘正,都知兵馬使王廷湊為留後。時魏博節度使李醖病不能軍,無以捍廷湊之亂;且以魏軍田氏舊旅,乃急詔布至,起復為魏博節度使,仍遷檢校工部尚書,令布乘傳之鎮。
 布喪服居堊室,去旌節導從之飾;及入魏州,居喪御事,動皆得禮。其祿俸月入百萬,一無所取,又籍魏中舊產,無巨細計錢十餘萬貫,皆出之以頒軍士。牙將史憲誠出己麾下,謂必能輸誠報效,用為先鋒兵馬使,精銳悉委之。時屢有急詔促令進軍。十月,布以魏軍三萬七千討之,結壘於南宮縣之南。十二月,進軍,下賊二柵。時硃克融囚張弘靖,據幽州,與廷湊掎角拒命。河朔三鎮,素相連衡,憲誠陰有異志。而魏軍驕侈,怯於格戰,又屬雪
 寒,糧餉不給,以此愈無鬥志,憲誠從而間之。俄有詔分布軍與李光顏合勢,東救深州,其眾自潰,多為憲誠所有,布得其眾八千。是月十日,還魏州。十一日,會諸將復議興師,而將卒益倨,咸曰:「尚書能行河朔舊事,則死生以之;若使復戰,皆不能也。」布以憲誠離間,度眾終不為用,嘆曰:「功無成矣!」即日,密表陳軍情,且稱遺表,略曰:「臣觀眾意,終負國恩,臣既無功,不敢忘死。伏願陛下速救光顏、元翼,不然,則義士忠臣,皆為河朔屠害。」奉表號哭,
 拜授其從事李石。乃入啟父靈,抽刀自刺,曰:「上以謝君父,下以示三軍。」言訖而絕。時議以布才雖不足,能以死謝家國,心志決烈,得燕、趙之古風焉。穆宗聞之駭嘆,廢朝三日,詔曰:



 故魏博節度使、起復寧遠將軍、檢校工部尚書、兼魏州大都督府長史、御史大夫、賜紫金魚袋田布,朕以寡昧,臨御萬邦,威刑不能禁干紀之徒,道化不能馴多僻之俗,致使上公罹禍,田氏銜冤。爰整旅以徂征,每終食而浩嘆,自茲吊伐,驟歷寒暄。雖良將銳師,率
 皆協力;而俟時觀釁,未即齊驅。嗟我誠臣,結其哀憤,引遷延之咎以自刻責,奮決烈之志以謝君親。白刃置於肝心,鴻毛論其生死,忠臣孝子,一舉兩全。晉稱卞氏之門,漢表尸鄉之節,比方於布,今古為鄰。況其臨命須臾,處之不撓;載形章表,益深衷悃。間使發緘,悼心疾首。從先臣於厚載,爾則無愧;睹遺像於麟閣,予何所堪!端拱崇名,職垂彞典,據斯以報,聊攄永懷。可贈尚書右僕射。



 布子在宥,大中年為安南都護,頗立邊功。



 群,太和八年
 為少府少監,充入吐蕃使,歷棣州刺史、安南都護。



 牟,會昌初為豐州刺史、天德軍使,歷武寧軍節度使。大中朝為兗海節度使,移鎮天平軍。諸子皆以邊上立功,累更籓鎮,以忠義為談者所稱。



 張孝忠,本奚之種類。曾祖靖,祖遜,代乙失活部落酋帥。父謐,開元中以眾歸國,授鴻臚卿同正,以孝忠貴,贈戶部尚書。孝忠以勇聞於燕、趙。時號張阿勞、王沒諾干,二人齊名。阿勞,孝忠本字;沒諾干,王武俊本字。孝忠形體
 魁偉,長六尺餘,性寬裕,事親恭孝。天寶末,以善射授內供奉。安祿山奏為偏將,破九姓突厥,先登陷陣,以功授果毅折沖。祿山、史思明繼陷河洛,孝忠皆為其前鋒。史朝義敗,入李寶臣帳下。上元中,奏授左領軍郎將,累加左金吾衛將軍同正、試殿中監,仍賜名孝忠,歷飛狐、高陽二軍使。李寶臣以孝忠謹重驍勇,甚委信之,以妻妹昧谷氏妻焉,仍悉以易州諸鎮兵馬令其統制。前後居城鎮十餘年,甚著威惠。



 田承嗣之寇冀州也,寶臣俾孝
 忠以精騎數千御之。承嗣見其整肅,嘆曰:「張阿勞在焉,冀州未易圖也!」乃焚營宵遁。及寶臣與硃滔戰於瓦橋,常慮滔來攻,故以孝忠為易州刺史,選精騎七千配焉,使捍幽州。奏授太子賓客、兼御史中丞,封範陽郡王。既而寶臣疑忌大將,殺李獻誠等四五人,使召孝忠,孝忠懼不往。寶臣使孝忠弟孝節召焉。孝忠命孝節復命曰:「諸將無狀,連頸受戮,孝忠懼死不敢往,亦不敢叛,猶公之不覲於朝,慮禍而已,無他志也。」孝節泣曰:「兄不行,吾
 歸死矣!」孝忠曰:「偕往則並命,吾留無患也。」乃歸,果無患。



 無幾,寶臣死,其子惟岳阻兵不受命,朝廷詔幽州節度使討之。滔以孝忠宿將善戰,有精兵八千在易州,慮軍興則撓其後,乃使判官蔡雄說孝忠曰:「惟岳小子驕貴,不達人事,輒拒朝命。滔奉命伐罪,使君何用助逆,不自求多福耶!今昭義、河東攻破田悅,淮西李僕射收下襄陽,梁崇義投井而卒,臨漢江而誅者五千人,即河南軍計日北首,趙、魏滅亡可見也。使君誠能去逆效順,必受
 重任,有先歸國之功矣!」孝忠然之,乃遣衙官隨雄報滔,又遣易州錄事參軍董稹入朝。德宗嘉之,授孝忠檢校工部尚書、恆州刺史、兼御史大夫,充成德軍節度使,便令與滔合兵攻惟岳,仍賜實封二百戶。其弟孝義及孝忠三女已適人在恆州者,悉為惟岳所害。孝忠甚德滔之保薦,以其子茂和聘滔之女,契約甚密,遂合兵破惟岳之師於束鹿,惟岳遁歸恆州。滔請乘勝襲之,孝忠仍引軍西北,還營義豐,滔大駭。孝忠將佐曰:「尚書布赤心
 於硃司徒,相信至矣。今逆寇已潰,不終其功,竊所未喻。」孝忠曰:「本求破賊,賊已破矣。然恆州宿將尚多,迫之則困獸猶鬥,緩之必翻然改圖。又硃滔言大識淺,可以慮始,難與守成。吾壁義豐,坐待惟岳之殄滅耳!」既而硃滔屯束鹿,不敢進軍。月餘,王武俊果斬惟嶽首以獻,如孝忠所料。後定州刺史楊政義以州降,孝忠遂有易、定之地。時既誅惟岳,分四州各置觀察使,武俊得恆州,康日知得深、趙二州,孝忠得易州。以成德軍額在恆州,孝忠
 既降政義,朝廷乃於定州置義武軍,以孝忠檢校兵部尚書,為義武軍節度、易定滄等州觀察等使。



 及硃滔、王武俊謀叛,將救田悅於魏州,慮孝忠踵後,滔軍將發,復遣蔡雄往說之。孝忠曰:「李惟岳背國作逆,孝忠歸國,今為忠臣。孝忠性直,業已效忠,不復助逆矣!往與武俊同行,且孝忠與武俊俱出蕃部,少長相狎,深知其心僻,能翻覆語,司徒當記鄙言,忽有蹉跌,始相憶也!」滔又啖以金帛,終拒而不從。易定居二兇之間,四面受敵,孝忠修
 峻溝壘,感勵將士,竟不受二兇之熒惑,議者多之。又加檢校左僕射,實封至三百戶。後孝忠為硃滔侵逼,詔神策兵馬使李晟、中官竇文場率師援之。孝忠以女妻晟子憑,與晟戮力同心,整訓士眾,竟全易定,賊不敢深入。及上幸奉天,令大將楊榮國提銳卒六百從晟入關赴難,收京城,榮國有功。



 興元元年正月,詔以本官同平章事。滄州本隸成德軍,既移隸義武,其刺史李固烈者,惟岳妻兄也,請還恆州。是歲,孝忠遣牙將程華往滄州交
 檢府藏。固烈輜車數十乘上路,滄州軍士呼曰:「士皆菜色,刺史不垂賑恤,乃稇載而歸,官物不可得也!」殺固烈而剽之。程華聞亂,由竇而遁,將士追之,謂曰:「固烈貪暴,已誅之矣,押牙且知州務。」孝忠即令攝刺史事。及硃滔、王武俊稱偽國,華與孝忠阻絕,不能相援。華嬰城拒賊,一州獲全,朝廷嘉之,乃拜華滄州刺史、御史中丞,充橫海軍使,仍改名日華,令每歲以滄州稅錢十二萬貫供義武軍。



 貞元二年,河北蝗旱,米斗一千五百文。復大兵
 之後,民無蓄積,餓殍相枕。孝忠所食,豆䜺而已,其下皆甘粗糲,人皆服其勤儉,孝忠為一時之賢將也。三年,加檢校司空,仍以其子茂宗尚義章公主。孝忠遣其妻鄧國夫人昧谷氏入朝,執親迎之禮。上嘉之,賞賚隆厚。五年七月,為將佐所惑,以兵入蔚州。尋詔歸鎮,仍以擅興削檢校司空。七年三月卒,時年六十二,廢朝三日,追封上谷郡王,贈太傅,再贈魏州大都督,冊贈太師,謚曰貞武。子茂昭、茂宗、茂和。



 茂昭,本名升雲。幼有志氣,好儒書,
 以父廕累官至檢校工部尚書。貞元七年,孝忠卒,德宗以邕王諒為義武軍節度大使、易定觀察使;以升雲為定州刺史,起復左金吾衛大將軍,充節度觀察留後,仍賜名茂昭。九年正月,授節度使,累遷檢校僕射、司空。二十年十月,入朝,累陳奏河北及西北邊事,詞情忠切,德示聳聽,嘆曰:「恨見卿之晚!」錫宴於麟德殿,賜良馬、甲第、器用、珍幣甚厚,仍以其第三男克禮尚晉康郡主。德宗方欲委之以邊任,明年晏駕,茂昭入臨於太極殿,每朝
 晡預列,聲哀氣咽,人皆獎其忠懇。順宗聽政,加中書門下平章事,且令還鎮,賜女樂二人,三表辭讓。及中使押犢車至第,茂昭立謂中使曰:「女樂出自禁中,非臣下所宜目睹。昔汾陽、咸寧、西平、北平嘗受此賜,不讓為宜。茂昭無四賢之功,述職入覲,人臣常禮,奈何當此寵賜!後有立功之臣,陛下何以加賞?」順宗聞之,深加禮異,允其所讓。又錫安仁里第,亦固讓不受。元和二年,又請入覲,五上章懇切,憲宗許之。冬十月,至京師,留數月,詔令歸
 鎮。茂昭願奉朝請於闕下,不許;加太子太保,復令還鎮。



 四年,王承宗叛,詔河東、河中、振武三鎮之師,合義武軍,為恆州北道招討。茂昭創廩廄,開道路,以待西軍。屬正月望夜,軍吏請曰:「舊例,上元前後三夜,不止行人,不閉里門。今外道軍戎方集,請如軍令。」茂昭曰:「三鎮兵馬,官軍也,安得言外道!放燈一如常歲。」使長男克讓與諸軍分道並進。克讓渡木刀溝,與賊接戰屢勝。茂昭親擐甲胄,為諸軍前鋒,累獻戎捷,幾覆承宗。會朝廷洗雪承宗,
 乃詔班師,加檢校太尉,兼太子太傅。



 自安、史之亂,兩河籓帥多阻命自固,父死子代,唯茂昭表請舉族還朝。鄰籓累遣游客間說,茂昭志意堅決,拜表求代者數四。上乃命左庶子任迪簡為其行軍司馬,乘驛赴之。以兩郡之簿書、管鑰、符印付迪簡,遣其妻季氏、男克讓、克恭等先就路。將行,誡之曰:「吾使爾曹侍親出易者,庶後之子孫不為風俗所染,則吾無恨矣!」時五年冬也。行及晉州,拜檢校太尉、兼中書令,充河中晉絳慈隰等州節度
 觀察等使。十二月十二日,至京師。故事雙日不坐,是日特開延英殿對茂昭,五刻乃罷。又上表請遷祖考之骨墓於京兆。在朝兩月,未之鎮。六年二月,疽發於首,卒,時年五十。廢朝五日,冊贈太師,賻絹三千匹、布一千端、米粟三千碩,喪事所須官給,詔京兆尹監護,謚曰獻武。



 憲宗念其忠藎,諸昆仲子侄皆居職秩,仍詔每年給絹二千匹,春秋分給。克讓、克恭官至諸衛大將軍。小男克勤,長慶中左武衛大將軍。時有赦文許一子五品官,克勤以
 子幼,請準近例回授外甥。狀至中書,下吏部員外郎判廢置,裴夷直斷曰:「一子官,恩在念功,貴於延賞;若無己子,許及宗男。今張克勤自有息男,妄以外甥奏請,移於他族,知是何人!儻涉賣官,實為亂法。雖援近日敕例,難破著定格文,國章既在必行,宅相恐難虛授。具狀上中書門下,克勤所請,望宜不允。」遂為定例。



 茂宗以父廕累官至光祿少卿同正。貞元三年,許尚公主,拜銀青光祿大夫、本官駙馬都尉,以公主幼,待年十三。屬茂宗母亡,
 遺表請終嘉禮。德宗念茂昭之勛,即日授雲麾將軍,起復授左衛將軍同正、駙馬都尉。諫官蔣乂等論曰:「自古以來,未聞有駙馬起復而尚公主者。」上曰:「卿所言,古禮也;如今人家往往有借吉為婚嫁者,卿何苦固執?」又奏曰:「臣聞近日人家有不甚知禮教者,或女居父母服,家既貧乏,且無強近至親,即有借吉以就親者。至於男子借吉婚娶,從古未聞,今忽令駙馬起復成禮,實恐驚駭物聽。況公主年幼,更俟一年出降,時既未失,且合禮經。」
 太常博士韋彤、裴堪曰:「伏見駙馬都尉張茂宗猶在母喪,聖恩念其亡母遺表所請,許公主出降,仍令茂宗即吉就婚者。伏以夫婦之義,人倫大端,所以《關雎》冠於《詩》首者,王化所先也。天屬之親,孝行為本,所以齊斬五服之重者,人道之厚也。聖人知此二端為訓人之本,不可變也,故制婚禮,上以承宗廟,下以繼後嗣,至若墨衰奪情,事緣金革。若使茂宗釋衰服而衣冕裳,去堊室而為親迎,雖云輟哀借吉,是亦以兇瀆嘉。伏願抑茂宗亡母
 之請,顧典章不易之義,待其終制,然後賜婚。」德宗不納,竟以義章公主降茂宗。自是以戚里之親,頗承恩顧。



 元和中,為閑廄使。國家自貞觀中至於麟德,國馬四十萬匹在河、隴間。開元中尚有二十七萬,雜以牛羊雜畜,不啻百萬,置八使四十八監,占隴右、金城、平涼、天水四郡,幅員千里,自長安至隴右,置七馬坊,為會計都領。岐、隴間善水草及腴田,皆屬七馬坊。至麟德以後,西戎陷隴右,國馬盡散,監牧使與七馬坊名額盡廢,其地利因歸
 於閑廄使。寶應中,鳳翔節度使請以監牧賦給貧民為業,土著相承,十數年矣。又有別敕賜諸寺觀凡千餘頃。及茂宗掌閑廄,與中尉吐突承璀善,遂恃恩舉舊事,並以監牧地租歸閑廄司。茂宗又奏麟游縣有岐陽馬坊,按舊圖地方三百四十頃,制下閑廄司檢計。百姓紛紜論訴,節度使李惟簡具事上聞,詔監察御史孫革往按問之。革還奏曰:「天興縣東五里有隋故岐陽馬坊,地在其側,蓋因監為名,與今岐陽所指百姓侵占處不相接,
 皆有明驗。」茂宗怒,恃有中助,誣革所奏不實。又令侍御史範傳式覆按,乃附茂宗,盡翻前奏,遂奪居人田業,皆屬閑廄,乃罷革官。長慶初,岐人論訴不已,詔御史按驗明白,乃復以其地還百姓,貶傳式官。



 茂宗俄授左金吾衛大將軍。長慶二年,檢校工部尚書,兼兗州刺史、御史大夫,充兗海沂節度等使,加檢校兵部尚書。太和五年,入為左津吾衛大將軍,充左衛使,轉左龍武統軍卒。



 茂和,元和中為左武衛將軍。裴度為淮西行營處置,用兵
 討吳元濟,建牙赴行營,奏用茂和為都押衙。茂和嘗以膽氣才略自贊於相府,故度奏用之。茂和慮度無功,淮、蔡不可平,乃辭之以疾。度怒甚,奏請斬茂和以勵行者。憲宗曰:「予以其家門忠順,為卿遠貶。」後復用為諸衛將軍,卒。



 陳楚者,定州人,茂昭之甥。少有武干,為義勇牙將,事茂昭,每出征伐,必令典精卒。隨茂昭入朝,授諸衛大將軍。元和十二年,義武軍節度使渾鎬喪師,定州兵亂,乃除
 楚易定節度,令馳傳赴任。亂猶未彌,楚夜馳入州城。楚家世久在定州,軍中部校皆楚之舊卒,人情大悅,軍卒帖然。轉河陽三城懷節度使。前後亟立戰功,入為龍武統軍。長慶三年卒。



 史臣曰:朝廷治亂,在法制當否,形勢得失而已。秦人叛上,法制失也;漢道勃興,形勢得也。臣觀開元之政舉,坐制百蠻;天寶之法衰,遂淪四海。玄宗一失其勢,橫流莫救,地分於群盜,身播於九夷。河朔二十餘州,竟為盜穴,
 諸田兇險,不近物情。而弘正、孝忠,頗達人臣之節,沂國力善無報,殆天意之好亂惡治歟!茂昭忠梗有禮,明禍福大端,近代之賢侯也!



 贊曰:田宗不令,禍淫無應。謂天輔仁,胡覆弘正。茂昭知止,終以善勝。孰生厲階,上失威柄。



\end{pinyinscope}