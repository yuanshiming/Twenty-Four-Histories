\article{卷一百八十二}

\begin{pinyinscope}

 ○趙隱弟騭子光逢光裔光胤張裼子文蔚
 濟美貽憲李蔚崔彥昭鄭畋盧攜王徽



 趙隱,字大隱,京兆奉天人也。祖植。建中末硃泚之亂,德
 宗幸奉天,時倉卒變起,羽衛不集,數日間賊來攻城,植以家人奴客奮力拒守,仍獻家財以助軍賞,天子嘉之。賊平,咸寧王渾瑊闢為推官,累遷殿中侍御史。貞元初,遷鄭州刺史。鄭滑節度使李融奏兼副使。十年,融病,軍府之政委於植。大將宋朝晏構三軍為亂,中夜火發,植與監軍列卒待之。遲明,亂卒自潰,即日誅斬皆盡。帝優詔嘉之,入為衛尉少卿,三遷尚書工部侍郎。十七年,出為廣州刺史、兼御史大夫、嶺南東道節度觀察等使,卒
 於鎮。子存約、滂。



 存約,太和三年為興元從事。是時軍亂,存約與節度使李絳方宴語,吏報:「新軍亂,突入府廨,公宜避之。」絳曰:「吾為帥臣,去之安往?」麾存約令遁,存約曰:「荷公厚德,獲奉賓階。背恩茍免,非吾志也。」即欲部分左右拒賊,是日與絳同遇害。



 隱以父罹非禍,泣守松楸,十餘年杜門讀書,不應闢命。會昌中,父友當權要,敦勉仕進,方應弓招,累為從事。大中三年,應進士登第,累遷郡守、尚書郎、給事中、河南尹,歷戶、兵二侍郎,領鹽鐵轉運
 等使。咸通末,以本官同平章事,加中書侍郎,兼禮部尚書,進階特進、天水伯、食邑七百戶。



 隱性仁孝,與弟騭尤稱友悌。少孤貧,弟兄力耕稼以奉親,造次不干親戚。既居宰輔,不以權位自高。退朝易衣,弟兄侍母左右。歲時伏臘,公卿大臣盈門通訊,而大臣及母之榮,無如其比。乾符中罷相,檢校兵部尚書、潤州刺史、浙西觀察等使。入為太常卿,轉吏部尚書,累加尚書左僕射。廣明中卒。子光逢、光裔、光胤。



 弟騭,亦以進士登第。大中末,與兄隱
 並踐省閣。咸通初,以兵部員外郎知制誥,轉郎中,正拜中書舍人。六年,權知貢舉。七年,選士,多得名流,拜禮部侍郎、御史中丞,累遷華州刺史、潼關防禦、鎮國軍等使,卒。



 光逢,乾符五年登進士第,釋褐鳳翔推官。入朝為監察御史,丁父憂免。僖宗還京,授太常博士,歷禮部、司勛、吏部三員外郎,集賢殿學士,轉禮部郎中。景福中,以祠部郎中知制誥,尋召充翰林學士,正拜中書舍人、戶部侍郎、學士承旨。改兵部侍郎、尚書左丞,學士如故。乾寧
 三年,從駕幸華州,拜御史中丞,改禮部侍郎。



 劉季述廢立之後,宰相崔胤與黃門爭權,衣冠道喪。光逢移疾,退居洛陽,閉關卻掃六七年。昭宗遷洛,起為吏部侍郎,復為左丞,歷太常卿。鼎沒於梁,累官至宰輔,封齊國公。



 光裔,光啟三年進士擢第。乾寧中,累遷司勛郎中、弘文館學士,改膳部郎中、知制誥,賜金紫之服。兄弟對掌內外制命,時人榮之。季述廢立之後,光逢歸洛。光裔旅游江表以避患。嶺南劉隱深禮之,奏為副使,因家嶺外。



 光胤,
 大順二年進士登第。天祐初,累官至駕部郎中。入梁,歷顯位。中興用為宰輔。



 張裼,字公表,河間人。父君卿,元和中舉進士,詞學知名,累歷郡守。裼,會昌四年進士擢第,釋褐壽州防禦判官。於琮布衣時,客游壽春,郡守待之不厚。裼以琮衣冠子,異禮遇之。琮將別,謂裼曰:「吾餉逆旅翁五十千,郡將之惠不登其數,如何?」裼方奉母,家貧,適得俸絹五十匹,盡以遺琮,約曰:「他時出處窮達,交相恤也。」裼累闢太原掌
 書記。大中朝,琮為翰林學士,俄登宰輔,判度支。琮召裼為司勛員外郎、判度支。尋用為翰林學士,轉郎中、知制誥,拜中書舍人、戶部侍郎、學士承旨。咸通末,琮為韋保衡所構譴逐,裼坐貶封州司馬。保衡誅,琮得雪。裼量移入朝,為太子賓客,遷吏部侍郎、京兆尹。乾符三年,出為華州刺史。其年冬,檢校吏部尚書、鄆州刺史、天平軍節度觀察等使。四年,卒於鎮,時年六十四。子文蔚、濟美、貽憲。



 文蔚,乾符二年進士擢第,累佐使府。龍紀初,入朝為
 尚書郎。乾寧中,以祠部郎中知制誥,正拜中書舍人,賜紫。崔胤擅朝政,與蔚同年進士,尤相善,用為翰林學士、戶部侍郎,轉兵部。從昭宗遷洛陽。輝王時,拜中書侍郎、平章事。入梁,卒。



 濟美、貽憲,相繼以進士登第。貽憲覆試落籍,為戶部巡官、集賢校理。



 李蔚,字茂休,隴西人。祖上公,位司農卿,元和初為陜虢觀察使。父景素,太和中進士。蔚,開成末進士擢第,釋褐襄陽從事。會昌末調選,又以書判拔萃,拜監察御史,轉
 殿中監。大中七年,以員外郎知臺雜,尋知制誥,轉郎中,正拜中書舍人。咸通五年,權知禮部貢舉。六年,拜禮部侍郎,轉尚書右丞。



 懿宗奉佛太過,常於禁中飯僧,親為贊唄。以旃檀為二高座,賜安國寺僧徹,逢八飯萬僧。蔚上疏諫曰:



 臣聞孔丘,聖者也,言則引周任之言;苻融,賢者也,諫必稱王猛之議。誠以事求師古,詞貴達情。陛下自纘帝圖,克崇佛事,止當修外,未甚得中。臣略採本朝名臣啟奏之言,以證奉佛初終之要。



 天后時,曾營大像,
 功費百萬,狄仁傑諫曰:「夫寶鉸殫於綴飾,瑰材竭於輪奐。功不使鬼,必在役人;物不天來,皆從地出;非苦百姓,物何以求?物生有時,用之無度;臣每思惟,實所悲痛。至如往在江表,像法盛興;梁武、簡文,施舍無限。及乎三淮沸浪,五嶺騰煙,列剎盈衢,無救危亡之禍;緇衣蔽路,豈益勤王之師?況近年以來,風塵屢擾;水旱失節,征役稍繁。必若多費官財,又苦人力,一隅有難,將何以救?」此切當之言一也。



 中宗時,公主外戚,奏度僧尼,姚崇諫曰:「佛
 不在外,求之於心。佛圖澄最賢,無益於後趙;羅什多藝,不救於姚秦。何充、苻融,皆遭敗滅;齊襄、梁武,未免災殃。但志發慈悲,心行利益,若蒼生安樂,即是佛身。」此切當之言二也。



 睿宗為金仙、玉真二公主造二道宮,辛替否諫曰:「自夏以來,淫雨不解,穀荒於壟,麥爛於場。入秋以來,亢旱為災,苗而不實,霜損蟲暴,草菜枯黃;下人咨嗟,未加賑貸。陛下愛兩女而造兩觀,燒瓦運木,載土填沙。道路流言,皆云用錢百萬。陛下聖人也,遠無不知;陛下
 明君也,細無不見!既知且見,知倉有幾年之儲?庫有幾年之帛?知百姓之間可存活乎?三邊之士可轉輸乎?今發一卒捍以邊陲,追一兵以衛社稷,多無衣食,皆帶饑寒;賞賜之間,迥無所出。軍旅驟敗,莫不由斯。而陛下破百萬貫錢,造不急之觀,以賈六合之怨,以違萬人之心。」此切當之言三也。



 替否又諫造寺曰:「釋教以清凈為基,慈悲為主。常體道以濟物,不利己而害人。每去己以全真,不營身以害教。今三時之月,築山穿池,損命也;殫府
 虛藏,損人也;廣殿長廊,營身也。損命,則不慈悲,損人,則不濟物,營身,則不清凈。豈大聖至神之心乎?佛書曰:『一切有為法,如夢幻泡影,如露亦如電。』臣以為減雕琢之費以賑貧人,是有如來之德;息穿掘之苦以全昆蟲,是有如來之仁;罷營葺之直以給邊陲,是有湯武之功;回不急之祿以購清廉,是有唐虞之治。陛下緩其所急,急其所緩;親未來而疏見在,失真實而冀虛無。重俗人之所為,輕天子之功業,臣實痛之!」此切當之言四也。



 臣觀
 仁傑,天后時上公也;姚崇,開元時賢相也;替否,睿宗之直臣也。臣每覽斯言,未嘗不廢卷而太息,痛其言之不行也。



 伏以陛下深重緇流,妙崇佛事,其為樂善,實邁前蹤。但細詳時代之安危,渺鑒昔賢之敷奏,則思過半矣,道遠乎哉!臣過忝渥恩,言虧匡諫,但舉從繩之義,少裨負扆之明。營繕之間,稍宜停減。



 優詔嘉之。尋拜京兆尹、太常卿。



 尋以本官同平章事,加中書侍郎,與盧攜、鄭畋同輔政。罷相,出為襄州刺史、山南東道節度使。入為吏
 部尚書,加檢校尚書右僕射、汴州刺史、宣武軍節度觀察等使。咸通十四年,轉揚州大都督府長史、淮南節度副大使知節度事。乾符三年受代,百姓詣闕乞留一年,從之。四年,復為吏部尚書,尋遷檢校司空、東都留守、東畿汝都防禦使。六年,河東軍亂,殺崔季康,詔以邠寧李侃鎮太原,軍情不伏。以蔚嘗為太原從事,軍民懷之。八月,以蔚為太原尹、北都留守、河東節度觀察等使。其年十月到鎮,下車三日,暴病卒。



 弟綰,從兄繪,累官至刺史。



 蔚三子:渥、洵、澤。



 渥,咸通末進士及第,釋褐太原從事,累拜中書舍人、禮部侍郎。光化三年,選貢士。洵至福建觀察使。



 崔彥昭,字思文,清河人。父豈。彥昭,大中三年進士擢第,釋褐諸侯府。咸通初,累遷兵部員外郎,轉郎中、知制誥,拜中書舍人,再遷戶部侍郎,判本司事。



 彥昭長於經濟,儒學優深,精於吏事。前治數郡,所蒞有聲,動多遺愛。十年,檢校禮部尚書、孟州刺史、河陽懷節度使,進階金紫。
 十二年正月,加檢校刑部尚書、太原尹、北都留守、河東節度管內觀察等使。



 時徐、泗用兵之後,北戎多寇邊,沙阤諸部動干紀律。彥昭柔以恩惠,來以兵威,三年之間,北門大治,軍民歌之。考滿受代,耆老數千詣闕乞留。詔報曰:「彥昭早著令名,累更劇任。入司邦計,開張用經緯之文;出統籓維,撫馭得韜鈐之術。自臨並部,隱若長城。但先和眾安人,不欲恃險與馬。遂致三軍百姓,瀝懇同詞,備述政能,唯恐罷去。顧茲重鎮,方委長材。既獲便安,
 未議移替,想當知悉。」



 僖宗即位,就加檢校吏部尚書。時趙隱、高璩知政事,與彥昭同年進士,薦彥昭長於治財賦。十五年三月,召為吏部侍郎,充諸道鹽鐵轉運使。乾符初,以本官同平章事、判度支。



 先是,楊收、路巖、韋保衡皆以朋黨好賂得罪。蕭人放秉政,頗革前弊。而彥昭輔政數月,百職斯舉,察而不煩,士君子稱之。二年,因其轉官,僖宗誡曰:



 彥昭歷試有勞,僉諧無愧。涉於六月,秉是一心。修乃文可以興文教,勵乃武可以成武功。重整前規,
 兩司大計,清能壁立,政乃風行。奸欺屏絕於多歧,請托銷摧於正議。不煩內庫,有助涓毫;不假外籓,有進絲發。軍食所入,餘剩於明年;郊廟所供,克辦於今歲。頗符神化,真謂廟謀。不有良臣,安能富國?宜酬勛於黃閣,俾正位於紫垣。敬服誡詞,永堅茂業,鳴呼!秉鈞之道,何所難哉;覆軍之塗,近已多矣!與其樹黨,不若修身;與其收恩,不如秉直。買暫勝者;貽其永敗,沽小智者,囊其大愚。不貴及人,唯爭自我;初誠潤屋,尋以危家。金玉滿堂,莫之
 能守,縱經營而得位,用枉撓而當辜。唯爾選自朕心,採於人望。宣詔既畢,閑門未知,來遂奔車,退無私謝。獨推元老,曾請急徵;以守道而自臻,實榮親之最重。爾其堅持正直,允執規程。但畏幽陰,必歸公當。甘言可憚,敘往可嗤。獎善須明,懲奸須銳。利於人者,雖難必舉;利於己者,雖易勿為。頻念孤寒,每思耕織,常自勤於數事,便有望於中興。彰朕知臣,在卿匡國,必使恩從下布,法自上行。但立直標,終無曲影。茍致我於堯、舜,亦比爾於皋、夔。
 可中書侍郎,依前判度支事。



 彥昭事母至孝,雖位居宰輔,退朝侍膳,與家人雜處,承奉左右,未嘗高言。歲時慶賀,公卿拜席,時人榮之。累遷門下侍郎,兼刑部尚書,充太清宮使、弘文館大學士。與鄭畋、李蔚同知政事,三加兼官,皆領度支如故。進階特進,累兼尚書右僕射。罷相,歷方鎮,以太子太保分司卒。子保謙。



 鄭畋,字臺文,滎陽人也。曾祖鄰,祖穆,父亞,並登進士第。亞,字子佐,元和十五年擢進士第,又應賢良方正、直言
 極諫制科。吏部調選,又以書判拔萃,數歲之內,連中三科。聰悟絕倫,文章秀發。李德裕在翰林,亞以文幹謁,深知之。出鎮浙西,闢為從事。累屬家艱,人多忌嫉,久之不調。會昌初,始入朝為監察御史,累遷刑部郎中。中丞李回奏知雜,遷諫議大夫、給事中。五年,德裕罷相鎮渚宮,授亞正議大夫,出為桂州刺史、御史中丞、桂管都防禦經略使。大中二年,吳汝納訴冤,德裕再貶潮州,亞亦貶循州刺史,卒。



 畋年十八,登進士第,釋褐汴宋節度推官,
 得秘書省校書郎。二十二,吏部調選,又以書判拔萃。授渭南尉、直史館事。未行,亞出桂州,畋隨侍左右。大中朝,白敏中、令狐綯相繼秉政十餘年,素與德裕相惡。凡德裕親舊多廢斥之,畋久不偕於士伍。咸通中,令狐綯出鎮,劉瞻鎮北門,闢為從事。入朝為虞部員外郎。右丞鄭薰,令狐之黨也,摭畋舊事覆奏,不放入省,畋復出為從事。五年,入為刑部員外郎,轉萬年令。九年,劉瞻作相,薦為翰林學士,轉戶部郎中。



 畋以久罹擯棄,幸承拔擢,因
 授官自陳曰:「臣十八進士及第,二十二書判登科。此時結綬王畿,便貯青雲之望。洎一沉風水,久換星霜,厭外府之樽罍,渴明庭之禮樂。咸通五年,方始登朝。若匪遭逢聖君,無以發揚幽跡。臣任刑部員外郎日,累於閣內對揚。去冬蒙擢宰萬年,又得延英中謝。傾藿幸依於白日,舍盆終睹於青天。昨以京縣浩穰,苦心為政,疲羸粗息,強御無蹤。方專宰字之心,用副憂勤之化。陛下過垂採聽,超授恩榮,擢於百里之中,致在三清之上。才超翰
 苑,遽改郎曹。」



 尋加知制誥,又自陳曰:「臣會昌二年,進士及第,大中首歲,書判登科。其時替故昭義節度使沈詢作渭南縣尉;兩考罷免,楊收以結綬替臣。詢則備歷顯榮,歿經數載;收則寵極臺輔,絀已三年。臣則外困賓筵,內甘散秩,仰窺霄漢,空嘆云泥。雖雲賦命屯奇,實以遭人排忌。」其因事自洗滌如此。



 俄遷中書舍人。十年,王師討徐方,禁庭書詔旁午。畋灑翰泉湧,動無滯思,言皆破的,同僚閣筆推之。尋遷戶部侍郎。龐勛平,以本官充承
 旨。畋以德望先達,淪滯久之。既冠禁庭,當為宰輔,因謝承旨自陳曰:「禁林素號清嚴,承旨尤稱峻重。偏膺顧問,首冠英賢。今之宰輔四人,三以此官騰躍,其為盛美,更異尋常。豈謂凡流,繼茲芳躅,臣所以憂不稱承旨之任也。至若繼劉瞻之慎密,守保衡之規程,瀝懇事君,披肝翊聖。以貞方為介胄,用忠信作籓籬。丹青帝文,金玉王度,臣亦不敢讓承旨之職。況沉舟墜羽,因聖主發揚,有薄藝微才,受鴻恩知遇。再周寒暑,六忝官榮,由郎吏以
 至於貳卿,自末僚而遷於上列。」其切於大用如此。



 其年八月,劉瞻以諫囚醫工宗族,罷相,出為荊南節度使。畋草制過為美詞。懿宗省之甚怒,責之曰:「畋頃以行跡玷穢,為時棄捐,朝籍周行,無階踐歷。竟因由徑,遂致叨居,塵忝既多,狡蠹尤甚。且居承旨,合體朕懷。一昨劉蟾出籓,朕豈無意?爾次當視草,過為美詞。逞譎詭於筆端,籠愛憎於形內。徒知報瞻欬唾之惠,誰思蔑我拔擢之恩?載詳言偽而堅,果明同惡相濟。人之多僻,一至於斯!宜
 行竄逐之科,用屏回邪之黨。可梧州刺史。」



 僖宗即位,召還。授右散騎常侍,改兵部侍郎。乾符四年,遷吏部侍郎。尋降制曰:「頃者時鬱正途,權歸邪幸。爾畋執心無惑,秉節被讒,徵復鴛行,愈洽人望。既負彌綸之業,宜居輔弼之司。可本官同平章事。」僖宗上尊號禮畢,進加中書侍郎,進階特進,轉門下侍郎,兼禮部尚書、集賢殿大學士。



 五年,黃巢起曹、鄆,南犯荊、襄,東渡江、淮,眾歸百萬,所經屢陷郡邑。六年,陷安南府據之。致書與浙東觀察使崔
 璆,求鄆州節鉞。璆言賊勢難圖,宜因授之,以絕北顧之患。天子下百僚議。初,黃巢之起也,宰相盧攜以浙西觀察使高駢素有軍功,奏為淮南節度使,令扼賊沖。尋以駢為諸道行營都統。及崔璆之奏,朝臣議之。有請假節以紓患者。畋採群議,欲以南海節制縻之。攜以始用高駢,欲立奇功以圖勝。攜曰:「高駢將略無雙,淮士甲兵甚銳。今諸道之師方集,蕞爾纖寇,不足平殄。何事舍之示怯,而令諸軍解體耶!」畋曰:「巢賊之亂,本因饑歲。人以利
 合,乃至實繁。江、淮以南,薦食殆半。國家久不用兵,士皆忘戰;所在節將,閉門自守,尚不能枝。不如釋咎包容,權降恩澤。彼本以饑年利合,一遇豐歲,孰不懷思鄉土?其眾一離,則巢賊幾上肉耳,此所謂不戰而屈人兵也!若此際不以計攻,全恃兵力,恐天下之憂未艾也。」



 群議然之,而左僕射於琮曰:「南海有市舶之利,歲貢珠璣。如今妖賊所有,國藏漸當廢竭。」上亦望駢成功,乃依攜議。及中書商量制敕,畋曰:「妖賊百萬,橫行天下,高公遷延玩
 寇,無意翦除,又從而保之,彼得計矣。國祚安危,在我輩三四人畫度。公倚淮南用兵,吾不知稅駕之所矣!」攜怒,拂衣而起,袂染於硯,因投之。僖宗聞之怒,曰:「大臣相詬,何以表儀四海?」二人俱罷政事,以太子賓客分司東都。



 廣明元年,賊自嶺表北渡江、浙,虜崔璆,陷淮南郡縣。高駢止令張璘控制沖要,閉壁自固。天子始思畋前言,二人俱徵還,拜畋禮部尚書。尋出為鳳翔隴右節度使。是冬,賊陷京師,僖宗出幸。畋聞難作,候駕於斜谷迎謁,垂
 泣曰:「將相誤陛下,以至於此。臣實罪人,請死以懲無狀。」上曰:「非卿失也。朕以狂寇凌犯,且駐蹕興元。卿宜堅扼賊沖,勿令滋蔓。」畋對曰:「臣心報國,死而後已,請陛下無東顧之憂。然道路艱虞,奏報梗澀,臨機不能遠稟聖旨,願聽臣便宜從事。」,上曰:「茍利宗社,任卿所行。」畋還鎮,搜乘補卒,繕修戎仗,浚飾城壘。盡出家財以散士卒。晝夜如臨大敵。



 中和元年二月,賊將尚讓、王璠率眾五萬,欲攻鳳翔。畋預知賊至,令大將李昌言等伏於要害。賊以
 畋儒者,必不能拒,步騎長驅,部伍不整。畋以銳卒數千,陳於高岡,虛立旗幟,延袤數里。距賊十餘里,伐鼓而陣。賊不之測眾寡,始欲列卒而陣,後軍未至,而昌言等發伏擊之,其眾大撓。日既晡矣,岐軍四合,追擊於龍尾陂,賊委兵仗自潰,斬馘萬計,得其鎧仗,岐軍大振。天子聞之,謂宰相曰:「予知畋不盡儒者之勇,甚慰予懷。」即授畋檢校尚書左僕射、同平章事,充京西諸道行營都統。



 時畿內諸鎮禁軍尚數萬,賊巢污京師後,眾無所歸。畋承
 制招諭,諸鎮將校皆萃岐陽。畋分財以結其心,與之盟誓,期匡王室。又傳檄天下曰:



 鳳翔隴右節度使、檢校尚書左僕射、同中書門下平章事、充京西諸道行營都統、上柱國、滎陽郡開國公、食邑二千戶鄭畋,移檄告諸籓鎮、郡縣、侯伯、牧守、將吏曰:夫屯亨有數,否泰相沿,如日月之蔽虧,似陰陽之愆伏。是以漢朝方盛,則莽、卓肆其奸兇;夏道未衰,而羿、浞騁其殘酷。不無僭越,尋亦誅夷。即知妖孽之生,古今難免。代有忠貞之士,力為匡復之
 謀。我國家應五運以承乾,躡三王之垂統,綿區飲化,匝宇歸仁。十八帝之鴻猷,銘於神鼎;三百年之睿澤,播在人謠。加以政尚寬弘,刑無枉濫,翼翼勤行於王道,孜孜務恤於生靈。足可傳寶祚於無窮,御瑤圖於不朽。



 近歲螟蝗作害,旱延災,因令無賴之徒,遽起亂常之暴。雖加討逐,猶肆猖狂。草賊黃巢,奴僕下才,豺狼醜類。寒耕熱耨,不勵力於田疇;媮食靡衣,務偷生於剽奪。結連兇黨,驅迫平人,始擾害於里閭,遂侵凌於郡邑。屬以籓臣
 不武,戎士貪財,徒加討逐之名,竟作遷延之役。致令滋蔓,累有邀求。聖上愛育情深,含弘道廣,指萬方而罪己,用百姓以為心。假以節旄,委之籓鎮,冀其悛革,免困疲羸。而殊無犬馬之誠,但恣蟲蛇之毒。剽掠我征鎮,覆沒我京都,凌辱我衣冠,屠殘我士庶。視人命有同於草芥,謂大寶易取如弈棋。而乃竊據宮闈,偽稱名號。爛羊頭而拜爵,續狗尾以命官。燕巢幕以誇安,魚在鼎而猶戲。殊不知五侯拗怒,期分項羽之尸;四塚既成,待葬蚩尤
 之骨。猶復廣侵田宅,濫瀆貨財,比溪壑以難盈,類烏鳶而縱攫。茫茫赤縣,僅同夷貊之鄉;惴惴黔黎,若在狴牢之內。固已人神共怒,行路傷心。



 畋謬領籓垣,榮兼將相,每枕戈而待旦,常泣血以忘餐;誓與義士忠臣,共翦狐鳴狗盜。近承詔命,會合諸軍。皇帝親御六師,即離三蜀;霜戈萬隊,鐵馬千群;雕虎嘯以風生,應龍驤而雲起。淮南高相公,會關東諸道百萬雄師,計以夏初,會於關內。畋與涇原節度使程宗楚、秦州節度使仇公遇等,已驅
 組練,大集關畿;爭麾隴右之蛇矛,待掃關中之蟻聚。而吐蕃、黨項以久被皇化,深憤國讎,願以沙漠之軍,共獻蕩平之捷。此際華戎合勢,籓鎮連衡,旌旗煥爛於雲霞,劍戟晶熒於霜雪。莫不持繩待試,賈勇爭先;思垂竹帛之功,誓雪朝廷之恥。矧茲殘孽,不足殄除。況諸道世受國恩,身縻好爵,皆貯匡邦之略,咸傾致主之誠。自函、洛構氛,鑾輿避狄,莫不指銅駝而皆裂,望玉壘以魂銷。聞此勤王,固宜投袂。更希憤激,速殄寇讎。永圖社稷之勛,
 以報君親之德,迎鑾反正,豈不休哉。



 時駕在坤維,音驛阻絕,以為朝廷無能復振。及畋傳檄,諸籓聳動,各治勤王之師,巢賊聞之大懼。自是賊騎不過京西。當時非畋扼賊之沖,褒、蜀危矣。尋進位檢校司空。



 其年冬,畋暴病,以岐山方禦賊沖,宜須驍將鎮守,表薦大將李昌言,詔可之。詔畋赴行在。二年正月至成都,以王鐸代畋將兵收復。畋尋以僕射平章事,以疾,久之不拜,累表乞解機務。二年冬,罷相,授太子少保。僖宗以畋子給事中凝績
 為隴州刺史,詔侍畋就郡養疾,薨於郡舍,時年五十九。



 光啟末,李茂貞授鳳翔節度使。畋會兵時,茂貞為博野軍小校在奉天,畋盡召其軍至岐下,以茂貞勤於軍旅,甚奇之,委以游邏之任。至是,茂貞思畋獎待之恩,上表論之曰:



 臣伏見當道故檢校司空、同平章事鄭畋,瑞應星精,祥開月角;建洪爐於聖代,成庶績於明昌。鳳毛方浴於春池,龍節忽移於右輔。旋以群鴟嘯聚,萬蝟鋒攢,蒼黃而玉輅省方,次第而金門徹鑰。九州相望,初猶豫
 以從風;百闢無歸,半狐疑而委質。而畋沖冠怒發,投袂治兵;羅劍戟於樽前,練貔貅於閫外。坎牲誓眾,釁鼓出師;馳羽檄於四方,暢皇威於萬里。身維地軸,決橫流而盡入東溟;手正天關,掃妖星而重尊北極。及至囊沙減灶,伐鼓揚旌;四兇方侈於獸心,一陣盡塗於龍尾。大振建瓴之捷,只於反掌之間。不期天柱朝摧,將星夜隕;竹帛徒書於茂烈,松楸未煥於易名。臣始仕從戎,爰承指顧,稟三令五申之戒,預一匡九合之謀。今則謬以微功,
 獲居重鎮。尋武侯之遺愛,城壘宛然;念叔子之高蹤,涕零何極?伏冀特加贈謚,以慰泉扃。



 昭宗嘉之,詔贈司徒,謚曰文昭。



 畋文學優深,器量弘恕。美風儀,神彩如玉,尤能賦詩。與人結交,榮悴如一。始為員外郎,為鄭薰不放省上,畋不以為憾。及畋作相,薰子為郎,畋特獎拔為給事中,列曹侍郎。其以德報怨,多此類也。



 子凝績,景福中歷刑部、戶部侍郎。



 盧攜,字子升,範陽人。祖損。父求,寶歷初登進士第,應諸
 府闢召。位終郡守。攜,大中九年進士擢第,授集賢校理,出佐使府。咸通中,入朝為右拾遺、殿中侍御史,累轉員外郎中、長安縣令、鄭州刺史。召拜諫議大夫。乾符初,以本官召充翰林學士,拜中書舍人。乾符末,加戶部侍郎、學士承旨。四年,以本官同中書門下平章事,累加門下侍郎,兼兵部尚書、弘文館大學士。



 五年,黃巢陷荊南、江西外郛及虔、吉、饒、信等州,自浙東陷福建,遂至嶺南,陷廣州,殺節度使李岧,遂抗表求節鉞。初,王仙芝起河南,
 攜舉宋威、齊克讓、曾袞等有將略,用為招討使。及宋威殺尚君長,致賊充斥。朝廷遂以宰臣王鐸為都統,攜深不悅。浙帥崔璆等上表,請假黃巢廣州節鉞,上令宰臣議。攜以王鐸為統帥,欲激怒黃巢,堅言不可假賊節制,止授率府率而已。與同列鄭畋爭論,投硯於地。由是兩罷之,為太子賓客分司。



 六年,高駢大將張麟頻破賊。攜素待高駢厚,常舉可為統帥。天子以駢立功,復召攜輔政。及王鐸失守,罷都統,以高駢代之。由是自潼關以東,
 汝、陜、許、鄧、汴、滑、青、兗皆易帥。王鐸、鄭畋所授任者,皆易之。攜內倚田令孜,外以高駢為援,朝廷大政,高下在心。時攜病風,精神恍惚。政事可否,皆決於親吏溫季修,貨賄公行。及賊擾淮南,張麟被殺,而許州逐帥,溵水兵潰。朝廷震懼,皆歸罪于攜。及賊陷潼關,罷攜相,為太子賓客,是夜仰藥而死。



 子晏,天祐初,為河南縣尉,為柳璨所殺。



 王徽,字昭文,京兆杜陵人,其先出於梁魏。魏為秦滅,始
 皇徙關東豪族實關中,魏諸公子徙於霸陵。以其故王族,遂為王氏。後周同州刺史熊,徽之十代祖,葬咸陽之鳳岐原,子孫因家焉。曾祖擇,從兄易從,天后朝登進士第。從弟明從、言從,睿宗朝並以進士擢第。昆仲四人,開元中三至鳳閣舍人,故時號「鳳閣王家」。其後,易從子定,定子逢,逢弟仲周,定兄密,密子行古,行古子收,收子超,皆以進士登第。王氏自易從已降,至大中朝登進士科者,一十八人;登臺省,歷牧守、賓佐者,三十餘人。擇從,大
 足三年登進士第,先天中,又應賢良方正制舉,升乙第,再遷京兆士曹參軍,充麗正殿學士。祖察,至德二年登進士第,位終連州刺史。父自立,位終緱氏令。



 徽大中十一年進士擢第,釋褐秘書省校書郎。戶部侍郎沈詢判度支,闢為巡官。宰相徐商領鹽鐵,又奏為參佐。時宣宗詔宰相於進士中選子弟尚主,或以徽籍上聞。徽性沖淡,遠勢利,聞之憂形於色。徽登第時,年逾四十,見宰相劉彖哀祈,具陳年已高矣,居常多病,不足以塵污禁臠。
 彖於上前言之方免。從令狐綯歷宣武、淮南兩鎮掌書記,得大理評事。召拜右拾遺,前後上疏論事二十三,人難言者必犯顏爭之,人士翕然稱重。



 會徐商罷相鎮江陵,以徽舊僚,欲加奏闢而不敢言。徽探知其旨,即席言曰:「僕在進士中,荷公重顧,公佩印臨戎,下官安得不從?」商喜甚,奏授殿中侍御史,賜緋,荊南節度判官。



 高湜時持憲綱,奏為侍御史知雜,兼職方員外郎,轉考功員外。時考簿上中下字硃書,吏緣為奸,多有揩改。徽白僕射,
 請以墨書,遂絕奸吏之弊。宰相蕭人放以徽明於吏術,尤重之。乾封初,遷司封郎中、長安縣令。學士闕人,人放用徽為翰林學士,改職方郎中、知制誥,正拜中書舍人。延英中謝,面賜金紫,遷戶部侍郎、學士承旨。改兵部侍郎、尚書左丞,學士承旨如故。



 廣明元年十二月三日,改戶部侍郎、同平章事。是日,黃巢入潼關,其夜僖宗出幸。徽與同列崔沆、豆盧彖、僕射於琮,至曙方知車駕出幸,遂相奔馳赴行在。徽夜落荊榛中,墜於崖谷,為賊所得,迫還
 京師。將授之偽命,徽示以足折口喑,雖白刃環之,終無懼色。賊令輿歸第,命醫工視之。月餘,守視者稍怠,徽乃雜於負販,竄之河中,遣人間道奉絹表入蜀。



 天子嘉之,詔授光祿大夫,守兵部尚書。將赴行在,尋詔徽以本官充東面宣慰催陣使。時王鐸都統行營兵馬在河中,累年未能破賊。徽與行營都監楊復光謀,赦沙陁三部落,令赴難。其年夏,代北軍至,決戰累捷,收復京師,以功加尚書右僕射。



 光啟中,潞州軍亂,殺其帥成麟,以兵部侍
 郎鄭昌圖權知昭義軍事。時孟方立割據山東三州,別為一鎮。上黨支郡,唯澤州耳,而軍中之人多附方立,昌圖不能制。宰相奏請以重臣鎮之,乃授徽檢校尚書左僕射、同平章事、潞州大都督府長史、澤潞邢洺磁觀察等使。時鑾輅未還,關東聚盜。而河東李克用與孟方立方爭澤潞。以朝廷兵力必不能加,上表訴之曰:



 臣聞量才授任,本切於安人;奉上推忠,莫先於體國。臣早逢昌運,備歷華資,止仗竭誠,幸無躁跡。六年內置,雖叨侍從
 之榮;一日臺司,未展匡扶之志。敢忘急病,用副憂勤。況重鎮兵符,元戎相印,特膺寵寄,出自宸衷,豈合憚勞,更陳衷款。但以鄭昌圖主留累月,將結深根;孟方立專據三州,轉成積釁。招其外則潞人胥怨,撫其內則邢將益疑。禍方熾於既焚,計奈何於已失。須觀勝負,乃決安危。欲遵命而勇行,則寢與百慮;思奉身而先退,則事體兩全。伏乞聖慈,博求廷議,擇其可付,理在從長。免微臣負懷寵之譏,使上黨破必爭勢。觸籓知難,庶無愧於前
 言;報國圖功,豈無伸於此日。



 天子乃以昌圖鎮之,以徽為諸道租庸供軍等使,餘官如故。



 時京師收復之後,宮寺焚燒,園陵毀廢,故車駕久而未還。乃以徽為大明宮留守、京畿安撫制置、修奉園陵等使。徽方治財賦,又兼制置,王畿之人,大半流喪,乃招合遺散,撫之如子。數年之間,版戶稍葺,東內齋閣,繕完有序。徽拜表請車駕還京,曰:「昨者狂寇將逃,延災方甚。而端門鳳畤,鎮福地而獨存;王氣龍盤,鬱祥煙而不散。足表宗祧降祉,臨御非
 遙。今雖初議修崇,未全壯麗,式示卑宮之儉,更凝馭道之尊。且肅宗才見捷書,便離岐下;德宗雖當盛暑,不駐漢中。故事具存,昌期難緩,願回鑾輅,早復京師。臣謬以散材,叨膺重寄,閉閣深念,拜章累陳。審時事之安危,系廟謀之得失。臣雖隨宜制置,竭力撫綏,如或鑾駕未回,必恐人心復散。縱成微效,終負殊私。勢有必然,理宜過慮。以茲淹駐,轉失機宜。實希永掛宸聰,亟還清蹕。」帝深嘉納,進位檢校司空、御史大夫,權知京兆尹事。



 中外權
 臣,遣人治第京師。因其亂後,多侵犯居人,百姓告訴相繼。徽不避權豪,平之以法。由是殘民安業,而權幸側目惡其強。乃以其黨薛杞為少尹,知府事。杞方居父喪,徽執奏不令入府。權臣愈怒,奏罷徽使務,以本官征赴行在。尋授太子少師,移疾退居蒲州。滿十旬,請罷。僖宗還宮,復授太子少師,疾,未任朝謁。宰相以徽怨望,奏貶集州刺史,徽乃輿疾赴貶所。不旬日,沙陁逼京師,僖宗出幸寶雞,而軍容田令孜得咎。天子以徽無罪,召拜吏部
 尚書,封瑯邪郡侯,食邑千戶。徽將赴行在,而襄王僭偽。邠、岐兵士,追逼乘輿。天子幸漢中,徽不能進。李襜偽制至河中府,召徽赴闕。徽托以風疾,不能步履。襜將僭號,逼內外臣僚署誓狀。徽稱臂緩,不能秉筆,竟不署名。



 硃玫既誅,天子自褒中還,至鳳翔,召徽拜御史大夫。車駕還宮,徽上章,以足膝風痺,不任朝拜,乞除散秩,復授太子少師。及便殿中謝,昭宗顧瞻進對,曰:「王徽神氣尚強,安可自便?」乃改授吏部尚書。大亂之後,銓選失緒,吏為
 奸蠹,有重疊補擬者。徽從初注授,便置手歷,一一檢視,人無擁滯,內外稱之。進位檢校司空,守尚書右僕射。大順元年十二月卒,贈太尉,謚曰貞。



 子三人:椿、樗、松。



 史臣曰:議兵之難,古無百勝,蓋以行權制變,法斷在於臨機;出奇無窮,聲實懸於中的。昔晉國之平孫皓,賈公閭堅沮渡江;吳人欲拒曹瞞,張輔吳終慚失策。彼之賢俊,未免悔尤。況盧子升平代書生,素迷軍志,只保高駢之平昔,不料高駢之苞藏;以至力困黃巢,毒流赤縣,絕
 吭仰藥,何所補焉?臺文氣激壯圖,志攄宿憤,慷慨誓眾,叱吒臨戎;竟扼賊喉,以康天步,謂之不武,斯焉取斯?崔、趙以鼎職奉親,天倫並達,積慶垂裕,播美士林。徽志吐盜泉,脫身虎口,功名不墜,君子多之。



 贊曰:武以伸威,謀以制敵。何必臨戎,陳師衽席。高駢玩寇,盧攜保奸。聖斷一誤,崎嶇劍山。



\end{pinyinscope}