\article{卷二十三 志第三 禮儀三}

\begin{pinyinscope}

 封禪之禮,自漢光武之後,曠世不修。隋開皇十四年,晉王廣率百官抗表,固請封禪。文帝令牛弘、辛彥之、許善心等創定儀注。至十五年,行幸兗州,遂於太山之下,為
 壇設祭,如南郊之禮,竟不升山而還。



 貞觀六年,平突厥,年穀屢登,群臣上言請封泰山。太宗曰:「議者以封禪為大典。如朕本心,但使天下太平,家給人足,雖闕封禪之禮,亦可比德堯、舜;若百姓不足,夷狄內侵,縱修封禪之儀,亦何異於桀、紂?昔秦始皇自謂德洽天心,自稱皇帝,登封岱宗,奢侈自矜。漢文帝竟不登封,而躬行儉約,刑措不用。今皆稱始皇為暴虐之主,漢文為有德之君。以此而言,無假封禪。禮云,『至敬不壇』,掃地而祭,足表至誠,
 何必遠登高山,封數尺之土也!」侍中王珪對曰:「陛下發德音,明封禪本末,非愚臣之所及。」秘書監魏徵曰:「隨末大亂,黎民遇陛下,始有生望。養之則至仁,勞之則未可。升中之禮,須備千乘萬騎,供帳之費,動役數州。戶口蕭條,何以能給?」太宗深嘉徵言,而中外章表不已。上問禮官兩漢封山儀注,因遣中書侍郎杜正倫行太山上七十二帝壇跡。是年兩河水潦,其事乃寢。至十一年,群臣復勸封山,始議其禮。於是國子博士劉伯莊、睦州刺史
 徐令言等,各上封祀之事,互設疑議,所見不同。多言新禮中封禪儀注,簡略未周。太宗敕秘書少監顏思古、諫議大夫硃子奢等,與四方名儒博物之士參議得失。議者數十家,遞相駁難,紛紜久不決。於是左僕射房玄齡、特進魏徵、中書令楊師道,博採眾議堪行用而與舊禮不同者奏之。



 其議昊天上帝壇曰:「將封先祭,義在告神,且備謁敬之儀,方展慶成之禮。固當於壇下止,預申齊潔。贊饗已畢,然後登封。既表重慎之深,兼示行事有漸。
 今請祭於泰山下,設壇以祀上帝,以景皇帝配享。壇長一十二丈,高一丈二尺。」又議制玉牒曰:「金玉重寶,質性貞堅,宗祀郊禋,皆充器幣,豈嫌華美,實貴精確。況乎三神壯觀,萬代鴻名,禮極殷崇,事資藻縟。玉牒玉檢,式韞靈奇。,傳之無窮,永存不朽。今請玉牒長一尺三寸,廣厚各五寸。玉檢厚二寸,長短闊狹一如玉牒。其印齒請隨璽大小,仍纏以金繩五周。」又議玉策曰:「封禪之祭,嚴配作主,皆奠玉策,肅奉虔誠。今玉策四枚,各長一尺三寸,廣
 一寸五分,厚五分。每策五簡,俱以金編。其一奠上帝,一奠太祖座,一奠皇地祇,一奠高祖座。」又議金匱曰:「登配之策,盛以金匱,歸格藝祖之廟室。今請長短令容玉策,高廣各六寸。形制如今之表函。纏以金繩,封以金泥,印以受命璽。」又議方石再累曰:「舊藏玉牒,止用石函,亦猶盛書篋笥,所以或呼石篋。今請方石三枚,以為再累。其十枚石檢,刻方石四邊而立之。纏以金繩,封以石泥,印以受命璽。」又議泰山上圓壇曰:「四出開道,壇場通義,
 南面入升,於事為允。今請介丘上圓壇廣五丈,高九尺,用五色土加之。四面各設一階。御位在壇南,升自南階,而就上封玉牒。」又議圓壇上土封曰:「凡言封者,皆是積土之名。利建分封,亦以班社立號。謂之封禪,厥義可知。今請於圓壇之上,安置方石,璽緘既畢,加土築以為封。高一丈二尺,而廣二丈,以五色土益封,玉牒藏於其內。祀禪之土,其封制亦同此。」又議玉璽曰:「謹詳前載方石緘封,玉檢金泥,必資印璽,以為秘固。今請依令用受命璽
 以封石檢。其玉檢既與石檢大小不同,請更造璽一枚,方一寸二分,文同受命璽,以封玉牒。石檢形制,依漢建武故事。」又議立碑曰:「勒石紀號,顯揚功業,登封降禪,肆覲之壇,立碑紀之。」又議設告至壇曰:「既至山下,禮行告至,柴於東方上帝,望秩遍禮群神。今請其壇方八十一尺,高三尺,陛仍四出。其禪方壇及餘儀式,請從今禮。仍請柴祭、望秩,同時行事。」又議廢石闕及大小距石曰:「距石之設,意取牢固,本資實用,豈云雕飾。今既積土厚封,足
 與天長地久。其小距環壇,石闕回建,事非經誥,無益禮義,煩而非要,請從減省。」



 太宗從其議,仍令附之於禮。



 十五年,下詔,將有事於泰山,復令公卿諸儒詳定儀注。太常卿韋挺、禮部侍郎令狐德棻為封禪使,參考其議。時論者又執異見,顏師古上書申明前議。太宗覽其奏,多依師古所陳為定。車駕至洛陽宮,會有彗星之變,乃下詔罷其事。



 高宗即位,公卿數請封禪,則天既立為皇后,又密贊之。麟德二年二月,車駕發京,東巡狩,詔禮官、博
 士撰定封禪儀注:



 有司於乾封元年正月戊辰朔。先是,有司齋戒。於前祀七日平旦,太尉誓百官於行從中臺,云:「來月一日封祀,二日登封泰山,三日禪社首,各揚其職,不供其事,國有常刑。」上齋於行宮四日,致齋三日。近侍之官應從升者,及從事群官、諸方客使,各本司公館清齋一宿。前祀一日,諸衛令共屬:未後一刻,設黃麾半仗於外壝之外,與樂工人俱清齋一宿。



 有司於太嶽南四里為圓壇,三成、十二階,如圓丘之制。壇上飾以青,四
 面各依方色,並造燎壇及壝三重。又造玉策三枚,皆以金繩連編玉簡為之。每簡長一尺二寸,廣一寸二分,厚三分,刻玉填金為字。又為玉匱一,以藏正座玉策,長一尺三寸。並玉檢方五寸,當繩處刻為五道,當封璽處刻深二分,方一寸二分。又為金匱二,以藏配座玉策,制度如玉匱。又為黃金繩以纏金玉匱,各五周。為金泥、玉匱、金匱。為玉璽一枚,方一寸二分,文同受命璽,封玉匱、金匱。又為石感,以藏玉匱。用方石再累,各方五尺,厚一尺,刻方石
 中令容玉匱。感旁施檢處,皆刻深三寸三分,闊一尺。當繩處皆刻深三分,闊一寸五分。為石檢十枚,以檢石感,皆長三尺,闊一尺,厚七寸。皆刻為印齒三道,深四寸。當封璽處方五寸,當通繩處闊一寸五分。皆有小石蓋,制與檢刻處相應,以檢擫封泥。其檢立於感旁,南方、北方各三,東方、西方各二,去感隅皆七寸。又為金繩以纏石感,各五周,徑三分。為石泥以泥石感,其泥,末石和方色土為之。為距石十二枚,分距感隅,皆再累,各闊二尺,長
 一丈,斜刻其首,令與感隅相應。



 泰山之上,設登封之壇,上徑五丈,高九尺,四出陛。壇上飾以青,四面依方色。一壝,隨地之宜。其玉牒、玉匱、石感、石檢、距石,皆如封祀之制。又為降禪壇於社首山上,方壇八隅,一成八陛,如方丘之制。壇上飾以黃,四面依方色。三壝,隨地之宜。其玉策、玉匱、石感、石檢、距石等,亦同封祀之制。



 至其年十二月,車駕至山下。及有司進奏儀注,封祀以高祖、太宗同配,禪社首以太穆皇后、文德皇后同配,皆以公卿充亞獻、
 終獻之禮。於是皇后抗表曰:



 伏尋登封之禮,遠邁古先,而降禪之儀,竊為未允。其祭地祇之日,以太后昭配,至於行事,皆以公卿。以妾愚誠,恐未周備。何者?乾坤定位,剛柔之義已殊;經義載陳,中外之儀斯別。瑤壇作配,既合於方祇;玉豆薦芳,實歸於內職。況推尊先後,親饗瓊筵,豈有外命宰臣,內參禋祭?詳於至理,有紊徽章。但禮節之源,雖興於昔典;而升降之制,尚缺於遙圖。且往代封嶽,雖云顯號,或因時俗,意在尋仙;或以情覬名,事
 深為己。豈如化被乎四表,推美於神宗;道冠乎二儀,歸功於先德。寧可仍遵舊軌,靡創彞章?



 妾謬處椒闈,叨居蘭掖。但以職惟中饋,道屬於蒸、嘗;義切奉先,理光於蘋、藻。罔極之思,載結於因心;祗肅之懷,實深於明祀。但妾早乖定省,已闕侍於晨昏;今屬崇禋,豈敢安於帷帟。是故馳情夕寢,睠嬴里而翹魂;疊慮宵興,仰深郊而聳念。伏望展禮之日,總率六宮內外命婦,以親奉奠。冀申如在之敬,式展虔拜之儀。積此微誠,已淹氣序。既屬鑾輿
 將警,奠璧非賒,輒效丹心,庶裨大禮。冀聖朝垂則,永播於芳規;螢燭末光,增輝於日月。



 於是祭地祇、梁甫,皆以皇后為亞獻,諸王大妃為終獻。



 丙辰,前羅文府果毅李敬貞論封禪須明水實樽:「《淮南子》云:『方諸見月,則津而為水。』高誘注云:『方諸,陰燧,大蛤也。熟摩拭令熱,以向月,則水生。以銅盤受之,下數石。』王充《論衡》云:『陽燧取火於日,方諸取水於月,相去甚遠,而火至水來者,氣感之驗也。』《漢舊儀》云:『八月飲酎,車駕夕牲,以鑒諸取水於月,以
 陽燧取火於日。』《周禮·考工記》云:『金有六齊。金錫半,謂之鑒燧之齊。』鄭玄注云:『鑒燧,取水火於日月之器也。』準鄭此注,則水火之器,皆以金錫為之。今司宰有陽燧,形如圓鏡,以取明火;陰鑒形如方鏡,以取明水。但比年祠祭,皆用陽燧取火,應時得;以陰鑒取水,未有得者,常用井水替明水之處。」奉敕令禮司研究。敬貞因說先儒是非,言及明水,乃云:「周禮金錫相半,自是造陽燧法,鄭玄錯解以為陰鑒之制。仍古取明水法,合用方諸,引《淮南子》
 等書,用大蛤也。」又稱:「敬貞曾八九月中,取蛤一尺二寸者依法試之。自人定至夜半,得水四五斗。」奉常奏曰:「封禪祭祀,即須明水實樽。敬貞所陳,檢有故實。」又稱:先經試驗確執,望請差敬貞自取蚌蛤,便赴太山與所司對試。」



 是日,制曰:「古今典制,文質不同,至於制度,隨世代沿革,唯祀天地,獨不改張,斯乃自處於厚,奉天以薄。又今封禪,即用玉牒金繩,器物之間,復有瓦樽秸席,一時行禮,文質頓乖,駁而不倫,深為未愜。其封祀、降禪所設上帝、后土位,先設颭秸、瓦甒、瓢杯等物,
 並宜改用裀褥罍爵,每事從文。其諸郊祀,亦宜準此。」於是昊天上帝之座褥以蒼,皇地祇褥以黃,配帝及後褥以紫,五方上帝及大明、夜明席皆以方色,內官已下席皆以莞。



 三年正月,帝親享昊天上帝於山下,封禮之壇,如圓丘之儀。祭訖,親封玉策,置石感,聚五色土封之。圓徑一丈二尺,高九尺。其日,帝率侍臣已下升泰山。翌日,就山上登封之壇封玉策訖,復還山下之齋宮。其明日,親祀皇地祇於社首山上,降禪之壇,如方丘之儀。皇后
 為亞獻,越國太妃燕氏為終獻。翌日,上御朝覲壇以朝群臣,如元日之儀。禮畢,宴文武百僚,大赦改元。初,上親享於降禪之壇,行初獻之禮畢,執事者皆趨而下。宦者執帷,皇后率六宮以升,行禮。帷帟皆以錦繡為之。百僚在位瞻望,或竊議焉。於是詔立登封、降禪、朝覲之碑,各於壇所。又詔名封祀壇為舞鶴臺,介丘壇為萬歲臺,降禪壇為景雲臺,以紀當時所見之瑞焉。



 高宗既封泰山之後,又欲遍封五嶽。至永淳元年,於洛州嵩山之南,置
 崇陽縣。其年七月,敕其所造奉天宮。二年正月,駕幸奉天宮。至七月,下詔將以其年十一月封禪於嵩嶽。詔國子司業李行偉、考工員外郎賈大隱、太常博士韋叔夏裴守貞輔抱素等詳定儀注。於是議:



 立封祀壇,如圓丘之制。上飾以玄,四面依方色。為圓壇,三成,高二丈四尺,每等高六尺。壇上徑一十六步,三等各闊四步。設十二陛,陛皆上闊八尺,下闊一丈四尺。為三重壝,距外壝三十步,內壝距五十步。燎壇在壇東南外壝之內,高三尺,
 方一丈五尺,南出陛。登封壇,圓徑五丈,高九尺。四出陛,為一壝,飾以五色,準封祀。禪祭壇,上飾以金,四面依方色,為八角方壇,再成,高一丈二尺,每等高四尺。壇上方十六步,每等廣四步,設八陛。其上壇陛皆廣八尺,中等陛皆廣一丈,下等陛皆廣一丈二尺。為三重壝之大小,準封祀。為埋坎,在壇之未地外壝之內,方深取足容物,南出陛。朝覲壇,於行宮之前為壇。宮方三分。壝二,在南。壇方二十四丈,高九尺,南面兩陛,餘三百各一陛。封祀、
 登封,五色土封石感為圓封,上徑一丈二尺,下徑三丈,高九尺。禪祭,五色土封為八角方封,大小準封祀制度。所用尺寸,準歷東封,並用古尺。諸壇並築土為之,禮無用石之文。並度影以定方位。登封、降禪,四出陛各當四方之中,陛各上廣七尺,下廣一丈二尺。封祀玉帛料,有蒼璧,四圭有邸,圭璧。禪祭有黃琮,兩圭有邸,無圭璧。



 又定登封、降禪、朝覲等日。準禮,冬至祭天於圓丘,其封祀請用十二日。準東封祀故事,十二日登封,
 十三日禪祭,十四日朝覲。若有故,須改登封已下期日,在禮無妨。



 又輦輿料云:封祀、登封,皇帝出乘玉輅,還乘金輅。皇太子往還金輅。禪祭,皇帝、太子如封祀。又衣服料云:「東封祠祭日,天皇服袞冕,近奉制,依《貞觀禮》服大裘。又云:袞冕服一具,齋服之;通天冠服一具,回服之;翼善冠服一具,馬上服之。皇太子袞冕服。又齋則服遠游冠,受朝則公服遠游冠服,馬上則進德冠服。



 當時又令詳求射牛之禮。行偉、守貞等議曰:「據《周禮》及《國語》,郊祀天地,天子自
 射其牲。漢武唯封太山,令侍中儒者射牛行事。至於餘祀,亦無射牲之文。但親舂射牲,雖是古禮,久從廢省。據封禪禮,祀日,未明十五刻,宰人以鑾刀割牲,質明而行事。比鑾駕至時,牢牲總畢,天皇唯奠玉酌獻而已。今若祀前一日射牲,事即傷早。祀日方始射牲,事又傷晚。若依漢武故事,即非親射之儀,事不可行。」詔從之。尋屬高宗不豫,遂罷封禪之禮。



 則天證聖元年,將有事於嵩山,先遣使致祭以祈福助,下制,號嵩山為神岳,尊嵩山神為
 天中王,夫人為靈妃。嵩山舊有夏啟及啟母、少室阿姨神廟,咸令預祈祭。至天冊萬歲二年臘月甲申,親行登封之禮。禮畢,便大赦,改元萬歲登封,改嵩陽縣為登封縣,陽成縣為告成縣。粵三日丁亥,禪於少室山。又二日己丑,御朝覲壇朝群臣,咸如乾封之儀。則天以封禪日為嵩嶽神祇所祐,遂尊神嶽天中王為神嶽天中皇帝,靈妃為天中皇后,夏后啟為齊聖皇帝;封啟母神為玉京太后,少室阿姨神為金闕夫人;王子晉為升仙太子,
 別為立廟。登封壇南有槲樹,大赦日於其杪置金雞樹。則天自制《升中述志碑》,樹於壇之丙地。



 玄宗開元十二年,文武百僚、朝集使、皇親及四方文學之士,皆以理化升平,時穀屢稔,上書請修封禪之禮並獻賦頌者,前後千有餘篇。玄宗謙沖不許。中書令張說又累日固請,乃下制曰:



 自古受命而王者,曷嘗不封泰山,禪梁父,答厚德,告成功。三代之前,罔不由此。越自魏、晉,以迄周、隋,帝典闕而大道隱,王綱弛而舊章缺,千載寂寥,封崇莫嗣。
 物極而復,天祚我唐,武、文二後,應圖受籙。洎於高宗,重光累盛,承至理,登介丘,懷百神,震六合,紹殷、周之統,接虞、夏之風。中宗弘懿鑠之休,睿宗沐粹精之道,巍巍蕩蕩,無得而稱者也。



 朕昔戡多難,稟略先朝,虔奉慈旨,嗣膺丕業。是用創九廟以申孝敬,禮二郊以展嚴禋,寶菽粟於水火,捐珠玉於山谷。兢兢業業,非敢追美前王;日慎一日,實以奉遵遺訓。至於巡狩大典,封禪鴻名,顧惟寡薄,未遑時邁,十四載於茲矣。今百穀有年,五材無眚,
 刑罰不用,禮義興行,和氣氤氳,淳風澹泊。蠻夷戎狄,殊方異類,重譯而至者,日月於闕廷;奇獸神禽,甘露嘉醴,窮祥極瑞,朝夕於林御。王公卿士,罄乃誠於中;鴻生碩儒,獻其書於外。莫不以神祇合契,億兆同心。斯皆烈祖聖孝,垂裕餘慶。故朕賴宗廟之介福,敢以眇身,顓其克讓。是以敬奉群議,弘此大猷,以光我高祖之丕圖,以紹我高祖之鴻烈。永言陟配,追感載深。可以開元十三年十一月十日,式遵故實,有事太山。所司與公卿諸儒詳
 擇典禮,預為備具,勿廣勞人,務存節約,以稱朕意。



 於是詔中書令張說、右散騎常侍徐堅、太常少卿韋絳、秘書少監康子元、國子博士侯行果等,與禮官於集賢書院刊撰儀注。



 玄宗初以靈山好靜,不欲喧繁,與宰臣及侍講學士對議,用山下封祀之儀。於是張說謂徐堅、韋絳等曰:「乾封舊儀,禪社首,享皇地祇,以先後配饗。王者父天而母地,當今皇母位,亦當往帝之母也,子配母饗,亦有何嫌?而以皇后配地祇,非古之制也。天監孔明,福善
 如響。乾封之禮,文德皇后配皇地祇,天後為亞獻,越國太妃為終獻。宮闈接神,有乖舊典。上玄不祐,遂有天授易姓之事,宗社中圮,公族誅滅,皆由此也。景龍之季,有事圓丘,韋氏為亞獻,皆以婦人升壇執籩豆,渫黷穹蒼,享祀不潔。未及逾年,國有內難,終獻皆受其咎,掌座齋郎及女人執祭者,多亦夭卒。今主上尊天敬神,事資革正。斯禮以睿宗大聖貞皇帝配皇地祇,侑神作主。」乃定議奏聞。上從之。



 舊禮:郊祀既畢,收取玉帛牲體,置於柴
 上,然後燔於燎壇之上,其壇於神壇之左。顯慶中,禮部尚書許敬宗等因修改舊禮,乃奏曰:



 謹按祭祀之禮,周人尚臭,祭天則燔柴,祭地則瘞血,宗廟則焫蕭灌鬯,皆貴氣臭,同以降神。禮經明白,義釋甚詳。委柴在祭神之初,理無所惑。是以《三禮義宗》等並云:「祭天以燔柴為始,然後行正祭。祭地以瘞血為先,然後行正祭。」又《禮論》說太常賀循上言:「積柴舊在壇南,燎祭天之牲,用犢左胖,漢儀用頭,今郊用脅之九個。太宰令奉牲脅,太祝令
 奉圭璧,俱奠燎薪之上。」此即晉氏故事,亦無祭末之文。既云漢儀用牲頭,頭非神俎之物,且祭末俎皆升右胖之脅。唯有《三禮》、賀循既云用祭天之牲左胖,復云今儀用脅九個,足明燔柴所用,與升俎不同。是知自在祭初,別燔牲體,非於祭末,燒神餘饌。此則晉氏以前,仍遵古禮。唯周、魏以降,妄為損益。緣告廟之幣,事畢瘞埋,因改燔柴,將為祭末。事無典實,禮闕降神。



 又燔柴、正祭,牲、玉皆別。蒼璧蒼犢之流,柴之所用;四圭騂犢之屬,祀之所須。
 故郊天之有四圭,猶祀廟之有圭瓚。是以《周官典瑞》,文勢相因,並事畢收藏,不在燔例。而今新禮引用蒼璧,不顧圭瓚,遂亦俱燔,義既有乖,理難因襲。又燔柴作樂,俱以降神,則處置之宜,須相依準。柴燔在左,作樂在南,求之禮情,實為不類。且《禮論》說積柴之處在神壇之南,新體以為壇左,文無典故。請改燔為祭始,位樂懸之南,外壝之內。其陰祀瘞埋,亦請準此。



 制可之。自是郊丘諸祀,並先焚而後祭。



 及玄宗將作封禪之禮,張說等參定儀
 注,徐堅、康子元等建議曰:



 臣等謹按顯慶年修禮官長孫無忌等奏改燔柴在祭前狀稱「祭祀之禮,必先降神。周人尚臭,祭天則燔柴」者。臣等按禮,迎神之義,樂六變則天神降,八變則地祇出,九變則鬼神可得而禮矣。則降神以樂,《周禮》正文,非謂燔柴以降神也。案尚臭之義,不為燔之先後。假如周人尚臭,祭天則燔柴,容或燔臭先以迎神。然則殷人尚聲,祭天亦燔柴,何聲可燔先迎神乎?又按顯慶中無忌等奏稱「晉氏之前,猶遵古禮。周、
 魏以降,妄為損益」者。今按郭璞《晉南郊賦》及注《爾雅》:「祭後方燔。」又按《宋志》所論,亦祭後方燔。又檢南齊、北齊及梁郊祀,亦飲福酒後方燔。又檢後周及隋郊祀,亦先祭後燔。據此,即周遵後燔,晉不先燎。無忌之事,義乃相乖。



 又按《周禮大宗伯》職:「以玉作六器,以禮天地四方。」《注》云:「禮為始告神時薦於神也。」下文云:「以蒼璧禮天,以黃琮禮地,皆有牲幣,各如其器之色。」又《禮器》云:以少為貴者,祭天特牲。」是知蒼璧之與蒼牲,俱各奠之神座,理
 節不惑。又云:「四圭有邸,以祀天、旅上帝。」即明祀昊天上帝之時,以旅五方天帝明矣。其青圭、赤璋、白琥、玄璜,自是立春、立夏、立秋、立冬之日,各於其方迎氣所用,自分別矣。今按顯慶所改新禮,以蒼璧與蒼牲、蒼幣,俱用先燔。蒼璧既已燔矣,所以遂加四圭有邸,奠之神座。蒼牲既已燔矣,所以更加騂牲,充其實俎。混昊天於五帝,同用四圭;失特牲之明文,加為二犢。深乖禮意,事乃無憑。



 考功員外郎趙冬曦、太學博士侯行果曰:「先焚者本以
 降神,行之已久。若從祭義,後焚為定。」中書令張說執奏曰:「徐堅等所議燔柴前後,議有不同。據祭義及貞觀。顯慶已後,既先燔,若欲正失禮,求祭義,請從《貞觀禮》。如且因循不改,更請從《顯慶禮》。凡祭者,本以心為主,心至則通於天地,達於神祇。既有先燔、後燎,自可斷於聖意,聖意所至,則通於神明。燔之先後,臣等不敢裁定。」玄宗令依後燔及先奠之儀。是後太常卿寧王憲奏請郊壇時祭,並依此先奠璧而後燔柴、瘞埋,制從之。



 時又有四門助教
 施敬本駁奏舊封禪禮八條,其略曰:



 舊禮,侍中跪取匜沃盥,非禮也。夫盥手洗爵,人君將致潔而尊神,故能使小臣為之。今侍中,大臣也,而沃盥於人君;太祝,小臣也,乃詔祝於天神。是接天神以小臣,奉人君以大臣,故非禮。按《周禮·大宗伯》曰:「鬱人,下士二人,贊裸事。」則沃盥此職也。漢承秦制,無鬱人之職,故使近臣為之。魏、晉至今,因而不改。然則漢禮,侍中行之則可矣,今以侍中為之,則非也。漢侍中,其始也微。高帝時籍孺為之,惠帝時閎
 孺為之,留侯子闢強年十五為之。至後漢,樓堅以議郎拜侍中,邵闔自侍中遷步兵校尉,秩千石,少府卿之屬也。少府卿秩中二千石,丞秩千石,侍中與少府丞班同。魏代蘇則為之。舊侍中親省起居,故謂之「執獸子。」吉茂見謂之曰,「仕進不止執獸子」,是言其為褻臣也。今侍中,名則古官,人非昔任,掌同燮理,寄實鹽梅,非復漢、魏「執獸子」之班,異乎《周禮》鬱人之職。行舟不息,墜劍方遙,驗刻而求,可謂謬矣。



 夫祝以傳命,通主人之意以薦於神
 明,非賤職也。故兩君相見,則卿為上儐。況天人之際,其肅恭之禮,以兩君為喻,不亦大乎!今太祝,下士也,非所以重命而尊神之義也。然則周、漢太祝,是禮矣。何者?按《周禮·大宗伯》曰:「太祝,下大夫二人,上士四人,掌六祝之辭。」大宗伯為上卿,今禮部尚書、太常卿比也;小宗伯中大夫,今侍郎、少卿比也;太祝下大夫,今郎中、太常丞比也;上士四人,今員外郎、太常博士之比也。故可以處天人之際,致尊極之辭矣。又漢太祝令,秩六百石,與太常
 博士同班。梁太祝令,與南臺御史同班。今太祝下士之卑,而居下大夫之職,斯又刻舟之論,不異於前矣。



 又曰:



 舊禮,謁者引太尉升壇亞獻,非禮也。謁者已賤,升壇已重,是微者用之於古,而大體實變之於今也。按《漢官儀》:「尚書御史臺官屬有謁者僕射一人,秩六百石,銅印青綬;謁者三十五人,以郎中歲稱給事,未滿歲稱灌謁者。又按《漢書百官公卿表》:光祿勛官屬有郎中、員外,秩比二千石;有謁者,掌賓贊受事,員七十人,秩比六百石。
 古之謁者,秩異等,今謁者班微,以之從事,可謂疏矣。



 又曰:



 舊禮,尚書令奉玉牒,今無其官,請以中書令從事。按漢武帝時,張安世為尚書令,游宴後宮,以宦者一人出入帝命,改為中書謁者令。至成帝,罷宦者,用士人。魏黃初改秘書,置中書監令。舊尚書並掌制誥,既置中書官,而制誥樞密皆掌焉。則自魏以來,中書是漢朝尚書之職。今尚書令奉玉牒,是用漢禮,其官既闕,故可以中書令主之。



 議奏,玄宗令張說、徐堅召敬本與之對議詳定,說等
 奏曰:「敬本所議,其中四條,先已改定。有不同者,望臨時量事改攝。」制從之。



 十三年十一月丙戌,至泰山,去山趾五里,西去社首山三里。丁亥,玄宗服袞冕於行宮,致齋於供帳前殿。己丑,日南至,大備法駕,至山下。玄宗御馬而登,侍臣從。先是玄宗以靈山清潔,不欲多人上,欲初獻於山上壇行事,亞獻、終獻於山下壇行事。因召禮官學士賀知章等入講儀注,因問之,知章等奏曰:「昊天上帝,君位;五方時帝,臣位;帝號雖同,而君臣異位。陛下享
 君位於山上,群臣祀臣位於山下,誠足以垂範來葉,為變禮之大者也。禮成於三,初獻、亞、終,合於一處。」玄宗曰:「朕正欲如是,故問卿耳。」於是敕三獻於山上行事,其五方帝及諸神座於山下壇行事。玄宗因問:「玉牒之文,前代帝王,何故秘之?」知章對曰:「玉牒本是通於神明之意。前代帝王,所求各異,或禱年算,或思神仙,其事微密,是故莫知之。」玄宗曰:「朕今此行,皆為蒼生祈福,更無秘請。宜將玉牒出示百僚;使知朕意。」其辭曰:「有唐嗣天子臣
 某,敢昭告於昊天上帝。天啟李氏,運興土德。高祖、太宗,受命立極。高宗升中,六合殷盛。中宗紹復,繼體不足。上帝眷祐,錫臣忠武。底綏內難,推戴聖父。恭承大寶,十有三年。敬若天意,四海晏然。封祀岱嶽,謝成於天。子孫百祿,蒼生受福。」



 庚寅,祀昊天上帝於山上封臺之前壇,高祖神堯皇帝配享焉。邠王守禮亞獻,寧王憲終獻。皇帝飲福酒。癸巳,中書令張說進稱:「天賜皇帝太一神策,周而復始,永綏兆人。」帝拜稽首。山上作圓臺四階,謂之封
 壇。臺上有方石再累,謂之石感玉牒、玉策,刻玉填金為字,各盛以玉匱,束以金繩,封以金泥,皇帝以受命寶印之。納二玉匱於感中,金泥堿際,以「天下同文」之印封之。壇東南為燎壇,積柴其上。皇帝就望燎位,火發,群臣稱萬歲,傳呼下山下,聲動天地。山下壇祀,群臣行事已畢,皇帝未離位,命中書門下曰:「朕以薄德,恭膺大寶。今封祀初建,雲物休祐,皆是卿等輔弼之力。君臣相保,勉副天心,長如今日,不敢矜怠。」中書令張說跪言:「聖心誠懇,宿
 齋山上。昨夜則息風收雨,今朝則天清日暖,復有祥風助樂,卿雲引燎,靈跡盛事,千古未聞。陛下又思慎終如初。長福萬姓,天下幸甚。」



 先是車駕至岳西來蘇頓,有大風從東北來,自午至夕,裂幕折柱,眾恐。張說倡言曰:「此必是海神來迎也。」及至嶽下,天地清晏。玄宗登山,日氣和煦。至齋次日入後,勁風偃人,寒氣切骨。玄宗因不食,次前露立,至夜半,仰天稱:「某身有過,請即降罰。若萬人無福,亦請某為當罪。兵馬辛苦,乞停風寒。」應時風止,山
 氣溫暖。時從山上布兵至於山壇,傳呼辰刻及詔命來往,斯須而達。夜中燃火相屬,山下望之,有如連星自地屬天。其日平明,山上清迥,下望山下,休氣四塞,登歌奏樂,有祥風自南而至,絲竹之聲,飄若天外。及行事,日揚火光,慶雲紛鬱,遍滿天際。群臣並集於社首山帷宮之次,以候鑾駕,遙望紫煙憧憧上達,內外歡噪。玄宗自山上便赴社首齋次,辰巳間至,日色明朗,慶雲不散。百闢及蕃夷爭前迎賀。辛卯,享地祇於社首之泰折壇,睿
 宗大聖貞皇帝配祀。五色雲見,日重輪。藏玉策於石感,如封壇之儀。壬辰,玄宗御朝覲之帳殿,大備陳布。文武百僚,二王後,孔子後,諸方朝集使,岳牧舉賢良及儒生、文士上賦頌者,戎狄夷蠻羌胡朝獻之國,突厥頡利發,契丹、奚等王,大食、謝褷、五天十姓,昆侖、日本、新羅、靺鞨之侍子及使,內臣之番,高麗朝鮮王,百濟帶方王,十姓摩阿史那興昔可汗,三十姓左右賢王,日南、西竺、鑿齒、雕題、牂柯、烏滸之酋長,咸在位。制曰:



 朕聞天監唯後,後
 克奉天,既合德以受命,亦推功而復始。厥初作者七十二君,道洽跡著,時至符出,皆用事於介丘,升中於上帝。人神之望,蓋有以塞之,皇王之序,可得而言。朕接統千歲,承光五葉,惟祖宗之德在人,惟天地之靈作主。往者內難,幽贊而集大勛;間無外虞,守成而纘舊服。未嘗不乾乾終日,思與公卿大夫上下協心,聿求至理,以弘我烈聖,其庶乎馨香。今九有大寧,群氓樂業,時必敬授而不奪,物亦順成而無夭。懋建皇極,幸致太和。洎乃幽遐,
 率由感被。戎狄不至,唯文告而來庭;麟鳳已臻,將覺情而在藪。以故凡百執事,亟言大封。顧惟不德,切欲勿議。伏以先聖儲祉,與天同功,荷傳符以在今,敢侑神而無報。大篇斯在,朕何讓焉。遂奉遵高宗之舊章,憲乾封之令典,時邁東土,柴告岱嶽,百神群望,莫不懷柔,四方諸侯,莫不來慶,斯是天下之介福,邦家之耿光也。無窮之休祉,豈獨在予;非常之惠澤,亦宜逮
 下。可大赦天下。封泰山神為天齊王,禮秩加三公一等,仍令所管崇飾祠廟,環山十里,禁其樵採。給近山二十戶復,以奉祠神。



 玄宗制《紀太山銘》,御書勒於山頂石壁之上。其辭曰:



 朕宅帝位,十有四載,顧惟不德,懵於至道,任夫難任,安夫難安,茲朕未知獲戾於上下,心之浩蕩,若涉大川。賴上帝垂休,先後儲慶,宰相庶尹,交修皇極,四海會同,五典敷暢,歲云嘉熟,人用大和。百闢僉謀,唱餘封禪,謂孝莫大於嚴父,禮莫盛於告天,天符既至,人望既
 積,固請不已,固辭不獲。肆餘與夫二三臣,稽虞《典》,繹漢制,張皇六師,震讋九宇。旌旗有列,士馬無嘩,肅肅邕邕,翼翼溶溶,以至岱宗,順也。



 《爾雅》曰:「泰山為東嶽。」《周官》曰:「兗州之鎮山。」實萬物之始,故稱岱焉;其位居五岳之伯,故稱宗焉。自昔王者受命易姓,於是乎啟天地,薦成功,序圖錄,紀氏號。朕統承先王,茲率厥典,實欲報玄天之眷命,為蒼生而祈福,豈敢高祝千古,自比九皇哉!故設壇場於山下,受群方之助祭;躬封燎於山上,冀一獻之
 通神。斯亦因高崇天,就廣增地之義也。



 乃仲冬庚寅,有事東嶽,類於上帝,配我高祖。在天之神,罔不畢降。粵翌日,禪於社首,佑我聖考,祀於皇祇。在地之神,罔不咸舉。暨壬辰,覲群後,上公進曰:「天子膺天符,納介福。群臣拜稽首,呼萬歲。慶合歡同,乃陳誡以德。大渾協度,彞倫攸敘,三事百揆,時乃之功。萬物由庚,兆人允植,列牧眾宰,時乃之功。一二兄弟,篤行孝友,錫類萬國,時唯休哉!我儒制禮,我史作樂,天地擾順,時唯休哉!蠻夷戎狄,重譯
 來貢,累聖之化,朕何慕焉。五靈百寶,日來月集,會昌之運,朕何惑焉。凡今而後,儆乃在位,一王度,齊象法,權舊章,補缺政,存易簡,去煩苛。思立人極,乃見天則。



 於戲!天生蒸人,惟後時乂,能以美利利天下,事天明矣。地德載物,惟後時相,能以厚生生萬人,事地察矣。天地明察,鬼神著矣。惟我藝祖文考,精爽在天,其曰「懿爾幼孫,克享上帝。惟帝時若,馨香其下」,丕乃曰「有唐氏文武之曾孫隆基,誕錫新命,纘我舊業,永保天祿,子孫其承之」。餘小
 子敢對揚上帝之休命,則亦與百執事尚綏兆人,將多於前功,而毖彼後患。一夫不獲,萬方其罪予。一心有終,上天其知我。朕惟寶行三德,曰慈、儉、謙。慈者,覆無疆之言;儉者,崇將來之訓;自滿者人損,自謙者天益。茍如是,則軌跡易循,基構易守。磨石璧,刻金石,冀後人之聽辭而見心,觀末而知本。銘曰:



 維天生人,立君以理,維君受命,奉天為子。代去不留,人來無已,德涼者滅,道高斯起。赫赫高祖,明明太宗,爰革隋政,奄有萬邦。罄天張宇,盡
 地開封,武稱有截,文表時邕。高宗稽古,德施周溥,茫茫九夷,削平一鼓。禮備封禪,功齊舜禹,巖巍岱宗,衛我神主。中宗紹運,舊邦惟新,睿宗繼明,天下歸仁。恭己南面,氤氳化淳,告成之禮,留諸後人。緬餘小子,重基五聖,匪功伐高,匪德矜盛。欽若祀典,丕承永命,至誠動天,福我萬姓。古封太山,七十二君,或禪亭亭,或禪云云。其跡不見,其名可聞,祗遹文祖,光昭舊勛。方士虛誕,儒書不足,佚後求仙,誣神檢玉。秦災風雨,漢污編錄,德未合天,或承之辱。道在觀政,
 名非從欲,銘心絕巖,播告群岳。



 於是中書令張說撰《封祀壇頌》、侍中源乾曜撰《社首壇頌》、禮部尚書蘇頲撰《朝覲壇頌》以紀德。



 玄宗乙酉歲生,以華岳當本命。先天二年七月正位,八月癸丑,封華嶽神為金天王。開元十年,因幸東都,又於華嶽祠前立碑,高五十餘尺。又於嶽上置道士觀,修功德。至天寶九載,又將封禪於華嶽,命御史大夫王鉷開鑿險路以設壇場,會祠堂災
 而止。



\end{pinyinscope}