\article{卷八十五}

\begin{pinyinscope}

 ○崔敦禮盧承慶劉祥道李敬玄李義琰孫處約樂彥瑋趙仁本



 崔敦禮,雍州咸陽人,隋禮部尚書仲方孫也。其先本居博陵,世為山東著姓,魏末徙關中。敦禮本名元禮,高祖
 改名焉。頗涉文史。重節義,嘗慕蘇子卿之為人。武德中,拜通事舍人。九年,太宗使敦禮往幽州召廬江王瑗。瑗舉兵反,執敦禮,問京師之事,敦禮竟無異詞。太宗聞而壯之,遷左衛郎將,賜以良馬及黃金雜物。貞觀元年,擢拜中書舍人,遷兵部侍郎,頻使突厥。累轉靈州都督。二十年,徵為兵部尚書。又奉詔安撫回紇、鐵勒部落。時延陀寇邊,敦禮與英國公李勣瀚海都督回紇吐迷度為其下所殺,詔敦禮往就部落綏輯之,因立
 其嗣子而還。敦禮深識蕃情,凡所奏請,事多允會。永徽四年,代高季輔為侍中,累封固安縣公,仍修國史。六年,加光祿大夫,代柳奭為中書令,尋又兼檢校太子詹事。敦禮以老疾屢陳乞請退。顯慶元年,拜太子少師,仍同中書門下三品。敕召其子定襄都督府司馬餘慶,使侍其疾。尋卒,年六十餘。高宗舉哀於東雲龍門,賜東園秘器,贈開府儀同三司、並州大都督,陪葬昭陵,賻絹布八百段、米粟八百碩,謚曰昭。子餘慶,官至兵部尚書。敦禮
 孫貞慎,神龍初為兵部侍郎。



 盧承慶,幽州範陽人。隋武陽太守思道孫也。父赤松,大業末為河東令。與高祖有舊,聞義師至霍邑,棄縣迎接,拜行臺兵部郎中。武德中,累轉率更令,封範陽郡公,尋卒。承慶美風儀,博學有才幹,少襲父爵。貞觀初,為秦州都督府戶曹參軍。因奏河西軍事,太宗奇其明辯,擢拜考功員外郎。累遷民部侍郎。太宗嘗問歷代戶口多少之數,承慶敘夏、殷以後迄於周、隋,皆有依據,太宗嗟賞
 久之。尋令兼檢校兵部侍郎,仍知五品選事。承慶辭曰:「選事職在尚書,臣今掌之,便是越局。」太宗不許,曰:「朕今信卿,卿何不自信也?」俄歷雍州別駕、尚書左丞。永徽初,為褚遂良所構,出為益州大都督府長史。遂良俄又求索承慶在雍州舊事奏之,由是左遷簡州司馬。歲餘,轉洪州長史。會高宗將幸汝州之溫湯,擢承慶為汝州刺史,入為光祿卿。顯慶四年,代杜正倫為度支尚書,仍同中書門下三品。尋坐度支失所,出為潤州刺史,再遷雍
 州長史,加銀青光祿大夫。總章二年,代李乾祐為刑部尚書,以年老請致仕,許之,仍加金紫光祿大夫。三年,病卒,年七十六。臨終誡其子曰:「死生至理,亦猶朝之有暮。吾終,斂以常服;晦朔常食巽,不用牲牢;墳高可認,不須廣大;事辦即葬,不須卜擇;墓中器物,瓷漆而已;有棺無槨,務在簡要;碑志但記官號、年代,不須廣事文飾。」贈幽州都督,謚曰定。



 弟承業,亦有學識。貞觀末,官至雍州長史、檢校尚書左丞。兄弟相次居此任,時人榮之。俄坐承慶
 事左遷忠州刺史。顯慶初,復為雍州長史。前後皆有能名。三遷左肅機,兼掌司列選事,賜爵魏縣子。總章中,卒於揚州大都督府長史,贈洺州刺史,謚曰簡。承業弟承泰,齊州長史。



 承泰子齊卿,長安初,為雍州錄事參軍。時則天令雍州長史薛季旭擇僚吏堪為御史者,季旭以聞,齊卿薦長安尉盧懷慎、李休光、萬年尉李乂、崔湜、咸陽丞倪若水、盩厔尉田崇闢、新豐尉崔日用,後皆至大官。齊卿,開元初為豳州刺史。時張守珪為果毅,齊卿禮
 接之,謂曰:「十年內當知節度。」果如其言,時人謂齊卿有人倫之鑒。齊卿好酒,飲至斗餘不亂,寬厚可親,士友以此善之。累遷太子詹事,封廣陽縣公,尋卒。承慶弟孫藏用,別有傳。



 劉祥道,魏州觀城人也。父林甫,武德初為內史舍人,時兵機繁速,庶事草創,高祖委林甫專典其事,以才幹見稱。尋詔與中書令蕭瑀等撰定律令,林甫因著《律議》萬餘言。久之,擢拜中書侍郎,賜爵樂平男。貞觀初,再遷吏
 部侍郎。初,隋代赴選者,以十一月為始,至春即停,選限既促,選司多不究悉。時選人漸眾,林甫奏請四時聽選,隨到注擬,當時甚以為便。時天下初定,州府及詔使多有赤牒授官,至是停省,盡來赴集,將萬餘人。林甫隨才銓擢,咸得其宜。時人以林甫典選,比隋之高孝基。三年,病卒,臨終上表薦賢,太宗甚嘉悼之,賜絹二百五十匹。祥道少襲父爵。永徽初,歷中書舍人、御史中丞、吏部侍郎。顯慶二年,遷黃門侍郎,仍知吏部選事。祥道以銓綜
 之術猶有所闕,乃上疏陳其得失。其一曰:



 今之選司取士,傷多且濫,每年入流,數過一千四百,傷多也。雜色入流,不加銓簡,是傷濫也。經明行修之士,猶或罕有正人,多取胥徒之流,豈能皆有德行?即知共厘務者,善人少而惡人多。有國以來,已四十載,尚未刑措,豈不由此乎?但服膺先王之道者,奏第然始付選;趨走幾案之間者,不簡便加祿秩。稽古之業,雖則難知,斗筲之材,何其易進?其雜色應入流人,望令曹司試判訖,簡為四等奏聞。
 第一等付吏部,第二等付兵部,次付主爵,次付司勛。其行署等私犯公坐情狀可責者,雖經赦降,亦量配三司;不經赦降者,放還本貫。冀入流不濫,官無冗雜,且令胥徒之輩,漸知勸勉。



 其二曰:



 古之選者,為官擇人,不聞取人多而官員少。今官員有數,入流無限,以有數供無限,遂令九流繁總,人隨歲積。謹約準所須人,量支年別入流者。今內外文武官一品以下,九品已上,一萬三千四百六十五員,略舉大數,當一萬四千人。壯室而仕,耳順
 而退,取其中數,不過支三十年。此則一萬四千人,三十年而略盡。若年別入流者五百人,經三十年便得一萬五千人,定須者一萬三千四百六十五人,足充所須之數。況三十年之外,在官者猶多,此便有餘,不慮其少。今年常入流者,遂逾一千四百,計應須數外,其餘兩倍。又常選放還者,仍停六七千人,更復年別新加,實非處置之法。



 其三曰:



 儒為教化之本,學者之宗。儒教不興,風俗將替。今庠序遍於四海,儒生溢於三學。誘掖之方,理實
 為備,而獎進之道,事或未周。但永徽已來,於今八載,在官者以善政粗聞,論事者以一言可採,莫不光被綸音,超升不次。而儒生未聞恩及,臣故以為獎進之道未周。



 其四曰:



 國家富有四海,已四十年,百姓官僚,未有秀才之舉。豈今人之不如昔人,將薦賢之道未至?寧可方稱多士,遂間斯人。望六品已下,爰及山谷,特降綸言,更審搜訪,仍量為條例,稍加優獎。不然,赫赫之辰,斯舉遂絕,一代盛事,實為朝廷惜之。



 其五曰:



 唐、虞三載考績,黜陟
 幽明。兩漢用人,亦久居其職。所以因官命氏,有倉、庾之姓。魏、晉以來,事無可紀。今之在任,四考即遷。官人知將秩滿,必懷去就;百姓見有遷代,能無茍且?以去就之人,臨茍且之輩,責以移風易俗,其可得乎!望經四考,就任加階,至八考滿,然後聽選。還淳反樸,雖未敢必期;送故迎新,實稍減勞弊。



 其六曰:



 尚書省二十四司及門下中書都事、主書、主事等,比來選補,皆取舊任流外有刀筆之人。縱欲參用士流,皆以儔類為恥,前後相承,遂成
 故事。且掖省崇峻,王言秘密,尚書政本,人物攸歸,而多用胥徒,恐未盡銓衡之理。望有厘革,稍清其選。



 明年,中書令杜正倫亦言入流人多,為政之弊。高宗遣祥道與正倫詳議其事。時公卿已下,憚於改作,事竟不行。祥道尋以修禮功,進封陽城縣侯。四年,遷刑部尚書,每覆大獄,必歔欷累嘆,奏決之日,為之再不食。龍朔元年,權檢校蒲州刺史。三年,兼檢校雍州長史,俄遷右相。祥道性謹慎,既居宰相,深懷憂懼。數自陳老疾,請退就閑職。俄
 轉司禮太常伯,罷知政事。麟德二年,將有事於泰山。有司議依舊禮,皆以太常卿為亞獻,光祿卿為終獻。祥道駁曰:「昔在三代,六卿位重,故得佐祠。漢、魏以來,權歸臺省,九卿皆為常伯屬官。今登封大禮,不以八座行事,而用九卿,無乃徇虛名而忘實事乎!」高宗從其議,竟以司徒徐王元禮為亞獻,祥道為終獻。事畢,進爵廣平郡公。乾封元年,又上表乞骸骨,優制加金紫光祿大夫,聽致仕。其年卒,年七十一,贈幽州都督,謚曰宣。子齊賢襲爵。



 齊賢,初自侍御史出為晉州司馬,高宗聞其方正,甚禮之。時將軍史興宗嘗從帝於苑中弋獵,因言晉州出好鷂,劉齊賢見為司馬,請使捕之。帝曰:「劉齊賢豈是覓鷂人耶!卿何以此待之?」遂止。齊賢後避章懷太子名,改名景先。永淳中,累遷黃門侍郎、同中書門下平章事。則天臨朝,代裴炎為侍中。及裴炎下獄,景先與鳳閣侍郎胡元範抗詞明其不反,則天甚怒之。炎既誅死,景先左遷普州刺史,未到,又貶授吉州長史。永昌年,為酷吏所陷,
 系於獄,自縊死,仍籍沒其家。景先自祖、父三代皆為兩省侍郎及典選,又叔父吏部郎中應道、從父弟禮部侍郎令植等八人,前後為吏部郎中員外,有唐已來,無有其比云。



 李敬玄,亳州譙人也。父孝節,穀州長史。敬玄博覽群書,特善五禮。貞觀末,高宗在東宮,馬周啟薦之,召入崇賢館,兼預侍讀,仍借禦書讀之。敬玄雖風格高峻,有不可犯之色,然勤於造請,不避寒暑,馬周及許敬宗等皆推
 薦延譽之。乾封初,歷遷西臺舍人、弘文館學士。總章二年,累轉西臺侍郎,兼太子右中護、同東西臺三品,兼檢校司列少常伯。時員外郎張仁禕有時務才,敬玄以曹事委之。仁禕始造姓歷,改修狀樣、銓歷等程式,處事勤勞,遂以心疾而卒。敬玄因仁禕之法,典選累年,銓綜有序。自永徽以後,選人轉多,當其任者,罕聞稱職,及敬玄掌選,天下稱其能。預選者歲有萬餘人,每於街衢見之,莫不知其姓名。其被放有訴者,即口陳其書判失錯及
 身負殿累,略無差殊。時人咸服其強記,莫之敢欺。選人有杭州參軍徐太玄者,初在任時,同僚有張惠犯贓至死,太玄哀其母老,乃詣獄自陳與惠同受。惠贓數既少,遂得減死,太玄亦坐免官,不調十餘年。敬玄知而大嗟賞之,擢授鄭州司功參軍,太玄由是知名,後官至秘書少監、申王師,以德行為時所重。敬玄賞鑒,多此類也。咸亨二年,授中書侍郎,餘並如故。三年,加銀青光祿大夫,行吏部侍郎,依舊兼太子右庶子、同中書門下三品。四
 年,監修國史。上元二年,拜吏部尚書,仍依舊兼太子左庶子,監修國史、同中書門下三品。敬玄久居選部,人多附之。前後三娶,皆山東士族。又與趙郡李氏合譜,故臺省要職,多是其同族婚媾之家。高宗知而不悅,然猶不彰其過。儀鳳元年,代劉仁軌為中書令。調露二年,吐蕃入寇,仁軌先與敬玄不協,遂奏請敬玄鎮守西邊。敬玄自以素非邊將之才,固辭。高宗謂曰:「仁軌若須朕,朕即自往,卿不得辭也。」竟以敬玄為洮河道大總管,兼安撫
 大使,仍檢校鄯州都督,率兵以御吐蕃。及將戰,副將工部尚書劉審禮先鋒擊之。敬玄聞賊至,狼狽卻走。審禮既無繼援,遂沒於陣。俄有詔留敬玄於鄯州防禦,敬玄累表稱疾,乞還醫療。許之。既入見,驗疾不重,高宗責其詐妄,又積其前後愆失,貶授衡州刺史。稍遷揚州大都督府長史。永淳元年卒,年六十八,贈兗州都督。撰《禮論》六十卷、《正論》三卷、文集三十卷。子思沖,神龍初,歷工部侍郎、左羽林軍將軍,從節愍太子誅武三思,事敗見殺,
 籍沒其家。敬玄弟元素,亦有吏才,初為武德令。時懷州刺史李文暕將調率金銀造常滿尊以獻,百姓甚弊之,官吏無敢異議者。元素抗詞固執,文暕乃損其制度,以家財營之。延載元年,自文昌左丞遷鳳閣侍郎、鳳閣鸞臺平章事,加銀青光祿大夫。萬歲通天二年,坐與洛州錄事參軍綦連耀交結,為武懿宗所陷,被殺,神龍初雪免。



 李義琰,魏州昌樂人,常州刺史玄道族孫也。其先自隴
 西徙山東,世為著姓。父玄德,癭陶令。義琰少舉進士,累補太原尉。時李勣為並州都督,僚吏皆望風懾懼,義琰獨廷折曲直,勣甚禮之。義琰,麟德中為白水令,有能名,拜司刑員外郎。上元中,累遷中書侍郎,又授太子右庶子、同中書門下三品。時天后預知國政,高宗嘗欲下詔令後攝知國事,義琰與中書令郝處俊固爭,以為不可,事竟寢。義琰身長八尺,博學多識,高宗每有顧問,言皆切直。章懷太子之廢也,高宗慰勉官僚,盡舍罪,令復其
 位。庶子薛元超等皆舞蹈謝恩,義琰獨引罪涕泣,時論美之。義琰宅無正寢,弟義璡為司功參軍,乃市堂材送焉。及義璡來覲,義琰謂曰:「以吾為國相,豈不懷愧?更營美室,是速吾禍,此豈愛我意哉!」義璡曰:「凡人仕為丞尉,即營第宅,兄官高祿重,豈宜卑陋以逼下也?」義琰曰:「事難全遂,物不兩興。既有貴仕,又廣其宇,若無令德,必受其殃。吾非不欲之,懼獲戾也。」竟不營構,其木為霖雨所腐而棄之。義琰後改葬父母,使舅氏移其舊塋,高宗知
 而怒曰:「豈以身在樞要,凌蔑外家,此人不可更知政事。」義琰聞而不自安,以足疾上疏乞骸骨,乃授銀青光祿大夫,聽致仕。乃將歸東都田里,公卿已下祖餞於通化門外,時人以比漢之二疏。垂拱初,起為懷州刺史。義琰自以失則天意,恐禍及,固辭不拜。四年,卒於家。義琰從祖弟義琛,永淳初,為雍州長史。時關輔大饑,高宗令貧人散於商、鄧逐食。義琛恐黎人流轉,因此不還,固爭之。由是忤旨,出為梁州都督,轉岐州刺史,稱為良吏。卒官。



 高宗時宰相,又有孫處約、樂彥瑋、趙仁本。並有名跡。



 孫處約者,汝州郟城人也。貞觀中,為齊王祐記室。祐既失德,處約數上書諫之。祐既誅,太宗親檢其家文疏,得處約諫書,甚嗟賞之。累轉中書舍人。其年,中書令杜正倫奏請更授一舍人,與處約同知制誥,高宗曰:「處約一人足辦我事,何須多也。」處約以預修《太宗實錄》成,賜物七百段。三遷中書侍郎,與李勣、許敬宗同知國政。尋避中宮諱,改名茂道。坐事左轉司禮少常伯。顯慶中,拜少
 司成,以老疾請致仕,許之,尋卒。子佺,睿宗時為左羽林大將軍,徵契丹戰歿。



 樂彥瑋者,雍州長安人。顯慶中,為給事中。時故侍中劉洎之子詣闕上言,洎貞觀末為褚遂良所譖枉死,稱冤請雪,中書侍郎李義府又左右之。高宗以問近臣,眾希義府之旨,皆言其枉。彥瑋獨進曰:「劉洎大臣,舉措須合軌度,人主暫有不豫,豈得即擬負國?先朝所責,未是不愜。且國君無過舉,若雪洎之罪,豈可謂先帝用刑不當
 乎?」然其言,遂寢其事。彥瑋尋丁憂,起為唐州刺史。及入辭,高宗記其言直,復拜東臺舍人。累遷西臺侍郎、同東西臺三品。乾封元年,代劉仁軌為大司憲,官名復舊,改為御史大夫。上元三年卒,贈秦州都督,永昌年,以子思晦貴,重贈揚州大都督。思晦,則天時官至鸞臺侍郎,兼檢校天官尚書、同鳳閣鸞臺三品,為酷吏所殺。



 趙仁本者,陜州河北人也。貞觀中,累轉殿中侍御史。自義寧已來,詔敕皆手自纂錄,臨事皆暗記之,甚為當時
 所伏。會有敕差一御史遠使,同列遞相辭托,仁本越次請行,言於治書侍御史馬周曰:「食君之祿,死君之事。雖復跋涉艱險,所不敢辭也。」及回,事又稱旨,擢吏部員外郎。乾封中,歷遷東臺侍郎、同東西臺三品,尋轉司列少常伯,知政事如故。時許敬宗為右相,頗任權勢,仁本拒其請托,遂為敬宗所構,俄授尚書左丞,罷知政事。咸亨初卒官。



 史臣論曰:崔、盧數公,皆以忠清文行,致位樞要。恪恭匪
 懈,以保名位,誠所謂持盈守成,太平之君子。然敬玄之擢太玄,可謂能舉善者矣。義琰腐材而不營第舍,可謂有儉德矣。彥瑋獨遏奸臣,仁本請當遠使,終升輔相,不亦宜乎!



 贊曰:盧、劉兩族,奕世名卿。二李、二樂,俱號公清。權臣獨抗,美第不營。以茲輔弼,無愧德聲。



\end{pinyinscope}