\article{卷七十一 列傳第二十一 魏徵}

\begin{pinyinscope}

 ○魏徵



 魏徵,字玄成,鉅鹿曲城人也。父長
 賢,北齊屯留令。徵少孤貧,落拓有大志,不事生業,出家為道士。好讀書,多所通涉,見天下漸亂,尤屬意縱橫之說。大業末,武陽郡丞
 元寶藏舉兵以應李密,召徵使典書記。密每見寶藏之疏,未嘗不稱善,既聞徵所為,遽使召之。徵進十策以乾密,雖奇之而不能用。及王世充攻密於洛口,徵說密長史鄭頲曰:「魏公雖驟勝,而驍將銳卒死傷多矣;又軍無府庫,有功不賞。戰士心惰,此二者難以應敵。未若深溝高壘,曠日持久,不過旬月,敵人糧盡,可不戰而退,追而擊之,取勝之道。且東都食盡,世充計窮,意欲死戰,可謂窮寇難與爭鋒,請慎無與戰。」頲曰:「此老生之常談耳!」徵
 曰:「此乃奇謀深策,何謂常談?」因拂衣而去。及密敗,徵隨密來降,至京師,久不見知。自請安輯山東,乃授秘書丞,驅傳至黎陽。時徐世勣尚為李密擁眾,徵與世勣書曰:



 自隋末亂離,群雄競逐,跨州連郡,不可勝數。魏公起自叛徒,奮臂大呼,四方響應,萬里風馳,雲合霧聚,眾數十萬。威之所被,將半天下,破世充於洛口,摧化及於黎山。方欲西蹈咸陽,北凌玄闕,揚旌瀚海,飲馬渭川,翻以百勝之威,敗於奔亡之虜。固知神器之重,自有所歸,不可
 以力爭。是以魏公思皇天之乃睠,入函谷而不疑。公生於擾攘之時,感知己之遇。根本已拔,確乎不動,鳩合遺散,據守一隅。世充以乘勝餘勇,息其東略;建德因侮亡之勢,不敢南謀。公之英聲,足以振於今古。然誰無善始,終之慮難。去就之機,安危大節。若策名得地,則九族廕其餘輝;委質非人,則一身不能自保。殷鑒不遠,公所聞見。孟賁猶豫,童子先之,知幾其神,不俟終日。今公處必爭之地,乘宜速之機,更事遲疑,坐觀成敗,恐兇狡之輩,
 先人生心,則公之事去矣。



 世勣得書,遂定計遣使歸國,開倉運糧,以饋淮安王神通之軍。俄而建德悉眾南下,攻陷黎陽,獲徵,署為起居舍人。及建德就擒,與裴矩西入關。隱太子聞其名,引直洗馬,甚禮之。徵見太宗勛業日隆,每勸建成早為之所。及敗,太宗使召之,謂曰:「汝離間我兄弟,何也?」徵曰:「皇太子若從徵言,必無今日之禍。」太宗素器之,引為詹事主簿。及踐祚,擢拜諫議大夫,封鉅鹿縣男,使安輯河北,許以便宜從事。徵至磁州,遇前
 宮千牛李志安、齊王護軍李思行錮送詣京師。徵謂副使李桐客曰:「吾等受命之日,前宮、齊府左右,皆令赦原不問。今復送思行,此外誰不自疑?徒遣使往,彼必不信,此乃差之毫厘,失之千里。且公家之利,知無不為,寧可慮身,不可廢國家大計。今若釋遣思行,不問其罪,則信義所感,無遠不臻。古者,大夫出疆,茍利社稷,專之可也。況今日之行,許以便宜從事,主上既以國士見待,安可不以國士報之乎?」即釋遣思行等,仍以啟聞,太宗甚悅。



 太宗新即位,勵精政道,數引徵入臥內,訪以得失。徵雅有經國之才,性又抗直,無所屈撓。太宗與之言,未嘗不欣然納受。徵亦喜逢知己之主,思竭其用,知無不言。太宗嘗勞之曰:「卿所陳諫,前後二百餘事,非卿至誠奉國,何能若是?」其年,遷尚書左丞。或有言徵阿黨親戚者,帝使御史大夫溫彥博案驗無狀,彥博奏曰:「徵為人臣,須存形跡,不能遠避嫌疑,遂招此謗。雖情在無私,亦有可責。」帝令彥博讓徵,且曰:「自今後不得不存形跡。」他日,徵
 入奏曰:「臣聞君臣協契,義同一體。不存公道,唯事形跡,若君臣上下,同遵此路,則邦之興喪,或未可知。」帝瞿然改容曰:「吾已悔之。」徵再拜曰:「願陛下使臣為良臣,勿使臣為忠臣。」帝曰:「忠、良有異乎?」徵曰:「良臣,稷、契、咎陶是也。忠臣,龍逢、比干是也。良臣使身獲美名,君受顯號,子孫傳世,福祿無疆。忠臣身受誅夷,君陷大惡,家國並喪,空有其名。以此而言,相去遠矣。」帝深納其言,賜絹五百匹。貞觀三年,遷秘書監,參預朝政。徵以喪亂之後,典章紛
 雜,奏引學者校定四部書。數年之間,秘府圖籍,粲然畢備。時高昌王麴文泰將入朝,西域諸國咸欲因文泰遣使貢獻,太宗令文泰使人厭怛紇干往迎接之。徵諫曰:「中國始平,瘡痍未復,若微有勞役,則不自安。往年文泰入朝,所經州縣,猶不能供,況加於此輩。若任其商賈來往,邊人則獲其利;若為賓客,中國即受其弊矣。漢建武二十二年,天下已寧。西域請置都護、送侍子,光武不許,蓋不以蠻夷勞弊中國也。今若許十國入貢,其使不下
 千人,欲使緣邊諸州何以取濟?人心萬端,後雖悔之,恐無所及。」上善其議。時厭怛紇干已發,遽追止之。後太宗幸九成宮,因有宮人還京,憩於湋川縣之官舍。俄又右僕射李靖、侍中王珪繼至,官屬移宮人於別所而舍靖等。太宗聞之,怒曰:「威福之柄,豈由靖等?何為禮靖而輕我宮人!」即令案驗湋川官屬及靖等。徵諫曰:「靖等,陛下心膂大臣;宮人,皇后掃除之隸。論其委付,事理不同。又靖等出外,官吏訪朝廷法式,歸來,陛下問人間疾苦。靖
 等自當與官吏相見,官吏亦不可不謁也。至於宮人,供食之外,不合參承。若以此罪責縣吏,恐不益德音,徒駭天下耳目。」帝曰:「公言是也。」乃釋官吏之罪,李靖等亦寢而不問。尋宴於丹霄樓,酒酣。太宗謂長孫無忌曰:「魏徵、王珪,昔在東宮,盡心所事,當時誠亦可惡。我能拔擢用之,以至今日,足為無愧古人。然徵每諫我不從,發言輒即不應,何也?」對曰:「臣以事有不可,所以陳論,若不從輒應,便恐此事即行。」帝曰:「但當時且應,更別陳論,豈不得
 耶?」徵曰:「昔舜誡群臣:『爾無面從,退有後言。』若臣面從陛下方始諫,此即『退有後言』,豈是稷、契事堯、舜之意耶?」帝大笑曰:「人言魏徵舉動疏慢,我但覺嫵媚,適為此耳。」徵拜謝曰:「陛下導之使言,臣所以敢諫,若陛下不受臣諫,豈敢數犯龍鱗?」是月,長樂公主將出降,帝以皇后所生,敕有司資送倍於永嘉長公主。徵曰:「不可。昔漢明欲封其子,云『我子豈與先帝子等?可半楚、淮陽。』前史以為美談。天子姊妹為長公主,子為公主,既加『長』字,即是有所
 尊崇。或可情有淺深,無容禮相逾越。」上然其言,入告長孫皇后,後遣使齎錢四十萬、絹四百匹,詣徵宅以賜之。尋進爵郡公。七年,代王珪為侍中,尚書省滯訟有不決者,詔徵評理之。徵性非習法,但存大體,以情處斷,無不悅服。初,有詔遣令狐德棻、岑文本撰《周史》,孔穎達、許敬宗撰《隋史》,姚思廉撰《梁》、《陳史》,李百藥撰《齊史》。徵受詔總加撰定,多所損益,薦在簡正。《隋史》序論,皆徵所作、《梁》、《陳》、《齊》各為總論,時稱良史。史成,加左光祿大夫,進封鄭國
 公,賜物二千段。



 徵自以無功於國,徒以辯說,遂參帷幄,深懼滿盈,後以目疾頻表遜位。太宗曰:「朕拔卿於讎虜之中,任公以樞要之職,見朕之非,未嘗不諫。公獨不見金之在礦也,何足貴哉?良冶鍛而為器,便為人所寶,朕方自比於金,以卿為良匠。卿雖有疾,未為衰老,豈得便爾?」其年,徵又面請遜位,太宗難違之,乃拜徵特進,仍知門下事。其後又頻上四疏,以陳得失。其一曰:



 臣觀自古受圖膺運,繼體守文,控御英傑,南面臨下,皆欲配厚德
 於天地,齊高明於日月,本枝百代,傳祚無窮。然而克終者鮮,敗亡相繼,其故何哉?所以求之失其道也。殷鑒不遠,可得而言。昔在有隋,統一寰宇,甲兵強盛,四十餘年,風行萬里,威動殊俗;一旦舉而棄之,盡為他人之有。彼煬帝豈惡天下之治安,不欲社稷之長久,故行桀虐,以就滅亡哉?恃其富強,不虞後患。驅天下以從欲,罄萬物以自奉,採域中之子女,求遠方之奇異。宮宇是飾,臺榭是崇,徭役無時,干戈不戢。外示威重,內多險忌。讒邪者
 必受其福,忠正者莫保其生。上下相蒙,君臣道隔,人不堪命,率土分崩。遂以四海之尊,殞於匹夫之手,子孫殄滅,為天下笑,深可痛哉!聖哲乘機,拯其危溺,八柱傾而復正,四維絕而更張。遠肅邇安,不逾於期月;勝殘去殺,無待於百年。今宮觀臺榭,盡居之矣;奇珍異物,盡收之矣;姬姜淑媛,盡侍於側矣;四海九州,盡為臣妾矣。若能鑒彼之所以亡,念我之所以得,日慎一日,雖休勿休。焚鹿臺之寶衣,毀阿房之廣殿,懼危亡於峻宇,思安處於
 卑宮,則神化潛通,無為而理,德之上也。若成功不毀,即仍其舊,除其不急,損之又損。雜茅茨於桂棟,參玉砌以土階,悅以使人,不竭其力,常念居之者逸,作之者勞,億兆悅以子來,群生仰而遂性,德之次也。若惟聖罔念,不慎厥終,忘締構之艱難,謂天命之可恃。忽彩椽之恭儉,追雕墻之侈靡,因其基以廣之,增其舊而飾之。觸類而長,不思止足,人不見德,而勞役是聞,斯為下矣。譬之負薪救火,揚湯止沸,以亂易亂,與亂同道,莫可則也,後嗣
 何觀,則人怨神怒;人怨神怒,則災害必下,而禍亂必作。禍亂既作,而能以身名令終者,鮮矣!順天革命之後,隆七百之祚,貽厥孫謀,傳之萬世,難得易失,可不念哉!



 其二曰:



 臣聞求木之長者,必固其根本;欲流之遠者,必浚其泉源;思國之安者,必積其德義。源不深而豈望流之遠,根不固而何求木之長?德不厚而思國之治,雖在下愚,知其不可,而況於明哲乎!人君當神器之重,居域中之大,將崇極天之峻,永保無疆之休。不念於居安思危,
 戒貪以儉;德不處其厚,情不勝其欲,斯亦伐根以求木茂,塞源而欲流長者也。凡百元首,承天景命,莫不殷憂而道著,功成而德衰。有善始者實繁,能克終者蓋寡,豈其取之易而守之難乎?昔取之而有餘,今守之而不足,何也?夫在殷憂必竭誠以待下,既得志則縱情以傲物。竭誠則胡越為一體,傲物則骨肉為行路。雖董之以嚴刑,振之以威怒,終茍免而不懷仁,貌恭而不心服。怨不在大,可畏惟人。載舟覆舟,所宜深慎。奔車朽索,其可忽
 乎?君人者,誠能見可欲則思知足以自戒,將有所作則思知止以安人,念高危則思謙沖而自牧,懼滿溢則思江海而下百川,樂盤游則思三驅以為度,恐懈怠則思慎始而敬終,慮壅蔽則思虛心以納下,想讒邪則思正身以黜惡,恩所加則思無因喜以謬賞,罰所及則思無因怒而濫刑。總此十思,弘茲九德,簡能而任之,擇善而從之。則智者盡其謀,勇者竭其力,仁者播其惠,信者效其忠。文武爭馳,君臣無事,可以盡豫游之樂,可以養松
 喬之壽,鳴琴垂拱,不言而化。何必勞神苦思,代下司職,役聰明之耳目,虧無為之大道哉!



 其三曰:



 臣聞《書》曰:「明德慎罰,惟刑恤哉!」《禮》云:「為上易事,為下易知,則刑不煩矣。上多疑則百姓惑,下難知則君長勞矣。」夫上易事,下易知,君長不勞,百姓不惑。故君有一德,臣無二心;上播忠厚之誠,下竭股肱之力,然後太平之基不墜,「康哉」之詠斯起。當今道被華夷,功高宇宙,無思不服,無遠不臻。然言尚於簡大,志在於明察,刑賞之本,在乎勸善而懲
 惡。帝王之所以與天下為畫一,不以親疏貴賤而輕重者也。今之刑賞,未必盡然。或申屈在乎好惡,輕重由乎喜怒。遇喜則矜其刑於法中,逢怒則求其罪於事外;所好則鉆皮出其毛羽,所惡則洗垢求其瘢痕。瘢痕可求,則刑斯濫矣;毛羽可出,則賞典謬矣。刑濫則小人道長,賞謬則君子道消。小人之惡不懲,君子之善不勸,而望治安刑措,非所聞也。且夫豫暇清談,皆敦尚於孔、老;威怒所至,則取法於申、韓。直道而行,非無三黜,危人自安,
 蓋亦多矣。故道德之旨未弘,刻薄之風已扇。夫上風既扇,則下生百端,人競趨時,則憲章不一,稽之王度,實虧君道。昔州黎上下其手,楚國之法遂差;張湯輕重其心,漢朝之刑以弊。人臣之頗僻,猶莫能申其欺罔,況人君之高下,將何以措其手足乎!以睿聖之聰明,無幽微而不燭,豈神有所不達,智有所不通哉?安其所安,不以恤刑為念;樂其所樂,遂忘先笑之變。禍福相倚,吉兇同域,唯人所召,安可不思?頃者責罰稍多,威怒微厲,或以供
 給不贍,或以人不從欲,皆非致治之所急,實乃驕奢之攸漸。是知貴不與驕期而驕自來,富不與奢期而奢自至,非徒語也。且我之所代,實在有隋,隋氏亂亡之源,聖明之所臨照。以隋氏之甲兵,況當今之士馬;以隋氏之府儲藏,譬今日之資儲;以隋氏之戶口,校今時之百姓。度長計大,曾何等級?然隋氏以富強而喪敗,動之也;我以貧寡而安寧,靜之也。靜之則安,動之則亂,人皆知之,非隱而難見也,微而難察也。鮮蹈平易之途,多遵覆車之
 轍,何哉?在於安不思危,治不念亂,存不慮亡之所致也。昔隋氏之未亂,自謂必無亂;隋氏之未亡,自謂必不亡。所以甲兵屢動,徭役不息,至於身將戮辱,竟未悟其滅亡之所由也,可不哀哉!



 夫鑒形之美惡,必就於止水;鑒國之安危,必取於亡國。《詩》曰:「殷鑒不遠,在夏后之世。」又曰:「伐柯伐柯,其則不遠。」臣願當今之動靜,思隋氏以為鑒,則存亡治亂,可得而知。若能思其所以危,則安矣;思其所以亂,則治矣;思其所以亡,則存矣。存亡之所在,節
 嗜欲以從人。省畋游之娛,息靡麗之作,罷不急之務,慎偏聽之怒。近忠厚,遠便佞,杜悅耳之邪說,聽苦口之忠言。去易進之人,賤難得之貨。採堯、舜之誹謗,追禹、湯之罪己,惜十家之產,順百姓之心。近取諸身,恕以待物。思勞謙以受益,不自滿以招損。有動則庶類以和,出言而千里斯應,超上德於前載,樹風聲於後昆。此聖哲之宏規,帝王之盛業,能事斯畢,在乎慎守而已。夫守之則易,取之實難,既得其所以難,豈不能保其所以易?其或保
 之不固,則驕奢淫泆動之也。慎終如始,可不勉歟!《易》云:「君子安不忘危,存不忘亡,治不忘亂,是以身安而國家可保。」誠哉斯言,不可以不深察也。伏惟陛下欲善之志,不減於昔時,聞過必改,少虧於曩日。若能以當今之無事,行疇昔之恭儉,則盡善盡美,固無得而稱焉。



 其四曰:



 臣聞為國之基,必資於德禮;君子所保,惟在於誠信。誠信立則下無二心,德禮形則遠人斯格。然則德禮誠信,國之大綱,在於父子君臣,不可斯須而廢也。故孔子曰:「
 君使臣以禮,臣事君以忠。」又曰:「自古皆有死,人無信不立。」文子曰:「同言而信,信在言前;同令而行,誠在令外。」然則言而不行,言不信也;令而不從,令無誠也。不信之言,無誠之令,為上則敗國,為下則危身,雖在顛沛之中,君子所不為也。自王道休明,十有餘載,威加海外,萬國來庭,倉稟日積,土地日廣。然而道德未益厚,仁義未益博者,何哉?由乎待下之情未盡於誠信,雖有善始之勤,未睹克終之美故也。其所由來者漸,非一朝一夕之故。昔
 貞觀之始,聞善若驚,暨五六年間,猶悅以從諫。自茲厥後,漸惡直言,雖或勉強,時有所容,非復曩時之豁如也。謇諤之士,稍避龍鱗;便佞之徒,肆其巧辯。謂同心者為朋黨,謂告訐者為至公,謂強直者為擅權,謂忠讜者為誹謗。謂之朋黨,雖忠信而可疑;謂之至公,雖矯偽而無咎。強直者畏擅權之議,忠讜者慮誹謗之尤。至於竊斧生疑,投杼致惑,正人不得盡其言,大臣莫能與之諍。熒惑視聽,鬱於大道,妨化損德,其在茲乎?故孔子惡利口
 之覆邦家,蓋為此也。且君子小人,貌同心異。君子掩人之惡,揚人之善,臨難無茍免,殺身以成仁。小人不恥不仁,不畏不義,唯利之所在,危人以自安。夫茍在危人,則何所不至。今將求致治,必委之於君子;事有得失,或訪之於小人。其待君子也,則敬而疏;遇小人也,必輕而狎。狎則言無不盡,疏則情或不通。是譽毀在於小人,刑罰加於君子,實興喪所在,亦安危所系,可不慎哉!夫中智之人,豈無小慧,然才非經國,慮不及遠,雖竭力盡誠,猶
 未免於傾敗;況內懷奸利,承顏順旨,其為患禍,不亦深乎?故孔子曰:「君子或有不仁者焉,未見小人而仁者。」然則君子不能無小惡,惡不積,無妨於正道;小人或時有小善,善不積,不足以立忠。今謂之善人矣,復慮其有不信,何異夫立直木而疑其影之不直乎?雖竭精神,勞思慮,其不可亦已明矣。



 夫君能盡禮,臣得竭忠,必在於內外無私,上下相信。上不信則無以使下,下不信則無以事上。信之為義,大矣哉!故自天祐之,吉無不利。昔齊桓
 公問於管仲曰:「吾欲酒腐於爵,肉腐於俎,得無害於霸乎?」管仲曰:「此極非其善者,然亦無害霸也。」公曰:「何如而害霸乎?」曰:「不能知人,害霸也;知而不能用,害霸也;用而不能信,害霸也;既信而又使小人參之,害霸也。」晉中行穆伯攻鼓,經年而不能下,饋間倫曰:「鼓之嗇夫,間倫知之,請無疲士大夫而鼓可得。」穆伯不應。左右曰:「不折一戟,不傷一卒,而鼓可得,君奚為不取?」穆伯曰:「間倫之為人也,佞而不仁。若間倫下之,吾不可以不賞。賞之,是賞
 佞人也。佞人得志,是使晉國之士舍仁而為佞,雖得鼓,將何用之?」夫穆伯列國大夫,管仲霸者之佐,猶慎於信任,遠避佞人也如此,況乎為四海之大君,應千齡之上聖,而可使巍巍之盛德,復將有所間然乎?若欲令君子小人是非不雜,必懷之以德,待之以信,厲之以義,節之以禮,然後善善而惡惡,審罰而明賞,則小人絕其佞邪,君子自強不息。無為之化,何遠之有?善善而不能進,惡惡而不能去,罰不及於有罪,賞不加於有功,則危亡之
 期,或未可保。永錫祚胤,將何望哉?



 太宗手詔嘉美,優納之。嘗謂長孫無忌曰:「朕即位之初,上書者或言『人主必須威權獨運,不得委任群下』;或欲耀兵振武,懾服四夷。唯有魏徵勸朕『偃革興文,布德施惠,中國既安,遠人自服』。朕從其語,天下大寧。絕域君長,皆來朝貢,九夷重譯,相望於道。此皆魏徵之力也。」



 太宗嘗嫌上封者眾,不近事實,欲加黜責。徵奏曰:「古者立誹謗之木,欲聞己過。今之封事,謗木之流也。陛下思聞得失,祗可恣其陳道。若
 所言衷,則有益於陛下;若不衷,無損於國家。」太宗曰:「此言是也。」並勞而遣之。後太宗在洛陽宮,幸積翠池,宴群臣,酒酣各賦一事。太宗賦《尚書》曰:「日昃玩百篇,臨燈披《五典》。夏康既逸豫,商辛亦流湎。恣情昏主多,克己明君鮮。滅身資累惡,成名由積善。」徵賦西漢曰:「受降臨軹道,爭長趣鴻門。驅傳渭橋上,觀兵細柳屯。夜宴經柏谷,朝游出杜原。終藉叔孫禮,方知皇帝尊。」太宗曰:「魏徵每言,必約我以禮也。」尋以修定《五禮》,當封一子為縣男,請讓
 孤兄子叔慈。太宗愴然曰:「卿之此心,可以勵俗。」遂許之。十二年,禮部尚書王珪奏言:「三品以上遇親王於途,皆降乘,違法申敬,有乖儀準。」太宗曰:「卿輩皆自崇貴,卑我兒子乎?」徵進曰:「自古迄茲,親王班次三公之下。今三品皆曰天子列卿及八座之長,為王降乘,非王所宜當也。求諸故事,則無可憑;行之於今,又乖國憲。」太宗曰:「國家所以立太子者,擬以為君也。然則人之修短,不在老少,設無太子,則母弟次立。以此而言,安得輕我子耶?」徵曰:「
 殷家尚質,有兄終弟及之義;自周以降,立嫡必長,所以絕庶孽之窺覦,塞禍亂之源本,有國者之所深慎。」於是遂可珪奏。會皇孫誕育,召公卿賜宴,太宗謂侍臣曰:「貞觀以前,從我平定天下,周旋艱險,玄齡之功,無所與讓。貞觀之後,盡心於我,獻納忠讜,安國利民,犯顏正諫,匡朕之違者,唯魏徵而已。古之名臣,何以加也!」於是親解佩刀以賜二人。



 徵以戴聖《禮記》編次不倫,遂為《類禮》二十卷,以類相從,削其重復,採先儒訓注,擇善從之,研精
 覃思,數年而畢。太宗覽而善之,賜物一千段,錄數本以賜太子及諸王,仍藏之秘府。



 先是,遣使詣西域立葉護可汗,未還,又遣使多齎金銀帛歷諸國市馬。徵諫曰:「今以立可汗為名,可汗未定,即詣諸國市馬,彼必以為意在市馬,不為專意立可汗。可汗得立,則不甚懷恩。諸蕃聞之,以為中國薄義重利,未必得馬,而失義矣。昔漢文有獻千里馬者,曰:吾兇行日三十里,吉行五十里,鑾輿在前,屬車在後,吾獨乘千里馬將安之?乃賞其道里所
 費而返之。漢光武有獻千里馬及寶劍者,馬以駕鼓車,劍以賜騎士。陛下凡所施為,皆邈逾三王之上,奈何至於此事,欲為孝文、光武之下乎?又魏文帝欲求市西域大珠,蘇則曰:『若陛下惠及四海,則不求自至,求而得之,不足為貴也。』陛下縱不能慕漢文之高行,可不畏蘇則之言乎?」太宗納其言而止。時公卿大臣並請封禪,唯徵以為不可。太宗曰:「朕欲卿極言之。豈功不高耶?德不厚耶?諸夏未治安耶?遠夷不慕義耶?嘉瑞不至耶?年穀不
 登耶?何為而不可?」對曰:「陛下功則高矣,而民未懷惠;德雖厚矣,而澤未滂流;諸夏雖安,未足以供事;遠夷慕義,無以供其求;符瑞雖臻,羅猶密;積歲豐稔,倉廩尚虛,此臣所以竊謂未可。臣未能遠譬,且借喻於人。今有人十年長患,療治且愈,此人應皮骨僅存,便欲使負米一石,日行百里,必不可得。隋氏之亂,非止十年,陛下為之良醫,疾苦雖已乂安,未甚充實,告成天地,臣竊有疑。且陛下東封,萬國咸萃,要荒之外,莫不奔走。今自伊、洛以
 東,暨乎海岱,灌莽巨澤,蒼茫千里,人煙斷絕,雞犬不聞,道路蕭條,進退艱阻,豈可引彼夷狄,示以虛弱?竭財以賞,未厭遠人之望;重加給復,不償百姓之勞。或遇水旱之災,風雨之變,庸夫橫議,悔不可追。豈獨臣之懇誠,亦有輿人之誦。」太宗不能奪。是後,右僕射缺,欲拜之,徵固讓乃止。及皇太子承乾不修德業,魏王泰寵愛日隆,內外庶僚,並有疑議。太宗聞而惡之,謂侍臣曰:「當今朝臣忠謇,無逾魏徵,我遣傅皇太子,用絕天下之望。」十六年,
 拜太子太師,知門下省事如故。徵自陳有疾,詔答曰:「漢之太子,四皓為助,我之賴公,即其義也。知公疾病,可臥護之。」其年,稱綿惙,中使相望。徵宅先無正寢,太宗欲為小殿,輟其材為徵營構,五日而成,遣中使齎素褥布被而賜之,遂其所尚也。及病篤,輿駕再幸其第,撫之流涕,問所欲言,徵曰:「嫠不恤緯而憂宗周之亡。」後數日,太宗夜夢徵若平生,及旦而奏徵薨,時年六十四。太宗親臨慟哭,廢朝五日,贈司空、相州都督,謚曰文貞。給羽葆鼓
 吹、班劍四十人,賻絹布千段、米粟千石,陪葬昭陵。及將祖載,徵妻裴氏曰:「徵平生儉素,今以一品禮葬,羽儀甚盛,非亡者之志。」悉辭不受,竟以布車載柩,無文彩之飾。太宗登苑西樓,望喪而哭,詔百官送出郊外。帝親制碑文,並為書石。其後追思不已,賜其實封九百戶。嘗臨朝謂侍臣曰:「夫以銅為鏡,可以正衣冠;以古為鏡,可以知興替;以人為鏡,可以明得失。朕常保此三鏡,以防己過。今魏徵殂逝,遂亡一鏡矣!徵亡後,朕遣人至宅,就其書
 函得表一紙,始立表草,字皆難識,唯前有數行,稍可分辯,云:『天下之事,有善有惡,任善人則國安,用惡人則國亂。公卿之內,情有愛憎,憎者唯見其惡,愛者唯見其善。愛憎之間,所宜詳慎,若愛而知其惡,憎而知其善,去邪勿疑,任賢勿貳,可以興矣。』其遺表如此,然在朕思之,恐不免斯事。公卿侍臣,可書之於笏,知而必諫也。」徵狀貌不逾中人,而素有膽智,每犯顏進諫,雖逢王赫斯怒,神色不移。嘗密薦中書侍郎杜正倫及吏部尚書侯君集
 有宰相之材。徵卒後,正倫以罪黜,君集犯逆伏誅,太宗始疑徵阿黨。徵又自錄前後諫諍言辭往復以示史官起居郎褚遂良,太宗知之,愈不悅。先許以衡山公主降其長子叔玉,於是手詔停婚,顧其家漸衰矣。徵四子,叔琬、叔璘、叔瑜。叔玉襲爵國公,官至光祿少卿;叔瑜至潞州刺史,叔璘禮部侍郎,則天時為酷吏所殺。神龍初,繼封叔玉子膺為鄭國公。



 叔瑜子華,開元初太子右庶子。



 史臣曰:臣嘗讀漢史《劉更生傳》,見其上書論王氏擅權,
 恐移運祚,漢成不悟,更生徘徊伊鬱,極言而不顧禍患,何匡益忠盡也如此!當更生時,諫者甚多。如谷永、楊興之上言,圖為奸利,與賊臣為鄉導,梅福、王吉之言,雖近古道,未切事情。則納諫任賢,詎宜容易!臣嘗閱《魏公故事》,與文皇討論政術,往復應對,凡數十萬言。其匡過弼違,能近取譬,博約連類,皆前代諍臣之不至者。其實根於道義,發為律度,身正而心勁,上不負時主,下不阿權幸,中不侈親族,外不為朋黨,不以逢時改節,不以圖位
 賣忠。所載章疏四篇,可為萬代王者法。雖漢之劉向、魏之徐邈、晉之山濤、宋之謝朏,才則才矣,比文貞之雅道,不有遺行乎?前代諍臣,一人而已。



 贊曰:智者不諫,諫或不智。智者盡言,國家之利。鄭公達節,才周經濟。太宗用之,子孫長世。



\end{pinyinscope}