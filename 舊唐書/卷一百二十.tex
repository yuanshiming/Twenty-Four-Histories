\article{卷一百二十}

\begin{pinyinscope}

 ○肅宗代宗諸子肅宗十三子代宗二十子



 越王人系承天皇帝倓衛王佖彭王僅兗王僴涇王侹鄆王榮襄王僙杞王倕召王偲恭懿太子佋定王侗
 淮陽王僖昭靖太子邈均王遐睦王述丹王逾恩王連韓王迥簡王遘益王乃隋王迅荊王選蜀王溯忻王造韶王暹嘉王運端王遇循王遹恭王通原王逵雅王逸



 肅宗皇帝十四子:章敬皇后生代宗皇帝,宮人孫氏生越王人系,張氏生承天皇帝,王氏生衛王佖,陳婕妤生彭
 王僅,韋妃生兗王僴,張美人生涇王侹,裴昭儀生襄王僙,段婕妤生杞王倕,崔妃生召王偲,張皇后生恭懿太子佋、定王侗,宮人生鄆王榮、宋王僖。



 越王人系,本名儋,肅宗第二子也。天寶中,封南陽郡王,授特進。至德二年十二月,進封趙王。乾元二年三月,九節度之兵潰於河北,史思明僭號於相州,王師未集,朝廷震駭。詔以李光弼握兵關東以代子儀。光弼請以親賢統師,七月,詔曰:



 握兵之要,古先為重;命帥之道,心膂攸
 憑。是知靖難夷兇,必資於金革;總戎授律,實仗於親賢。蓋將底寧邦家,保息黎獻者矣。朕以薄德,纘承鴻緒,往屬元兇暴亂,中夏不寧。上憑宗社之靈,下藉熊羆之力,由是廓清咸、洛,拯此生人。頃以河朔殘妖,尚稽天討,蛇豕竊依於城堡,塗炭久被於齊氓,朕為人父母,寧忘閔念。雖好生息戰,每冀其歸降;而餘孽昧恩,靡聞於悔禍。所以軒後親征於獯鬻,周文致役於昆夷,古之用兵,蓋非獲已。趙王人系幼稟異操,夙懷韜略,負東平之文學,蘊
 任城之智勇。性惟忠孝,持愛敬以立身;志尚權謀,有經通之遠智。知子者父,方有屬於維城;擇能而授,俾克申於戎律。且兇徒嘯聚,頗歷歲時,惡既貫盈,理當撲滅。君親有命,可不敬乎!俾展龍豹之韜,永清梟獍之類。可充天下兵馬元帥,仍令司空、兼侍中、薊國公光弼副知節度行營事。應緣軍司署置,所司準式。



 九月,史思明陷洛陽,光弼以副元帥董兵守河陽,王不出京師。十月,下詔車駕親征,諫官論奏乃止;王請行,不許。三年四月,改封
 越王。寶應元年四月,肅宗寢疾彌留。皇后張氏與中官李輔國有隙,因皇太子監國,謀誅輔國,使人以肅宗命召太子入宮。皇后謂太子曰:「賊臣輔國,久典禁軍,四方詔令,皆出其口。頃矯制命,逼徙聖皇。今聖體彌留,心懷怏怏,常忌吾與汝。又聞射生內侍程元振結托黃門,將圖不軌,若不誅之,禍在頃刻。」太子泣而對曰:「此二人是陛下勛舊內臣,今聖躬不康,重以此事驚撓聖慮,情所難任。若決行此命,當出外徐圖之。」後知太子難與共事,
 乃召人系謂之曰:「皇太子仁惠,不足以圖平禍難。」復以除輔國謀告之,曰:「汝能行此事乎?」人系曰:「能。」後令內謁者監段恆俊與越王謀,召中官有武勇者二百餘人,授甲於長生殿。是月乙丑,皇后矯詔召太子,程元振伺知之,告輔國。元振握兵於凌霄門候之,太子既至,以難告。太子曰:「必無此事。聖恙危篤,吾豈懼死不赴召乎?」元振曰:「為社稷計,行則禍及矣。」遂以兵護太子匿於飛龍廄。丙寅夜,元振、輔國勒兵於三殿前,收捕越王及同謀內侍硃
 光輝、段恆俊等百餘人。禁系幽皇后於別殿,侍者十數人隨之。是日,皇后、越王俱為輔國所害



 人系子:建、逌、逾。建,建中元年十一月,封武威郡王,授殿中監同正員;逌封興道郡王,授殿中監同正員;逾封齊國公,光祿卿同正員。



 承天皇帝倓,肅宗第三子也。天寶中,封建寧郡王,授太常卿同正員。英毅有才略,善射。祿山之亂,玄宗幸蜀,倓兄弟典親兵扈從。車駕渡渭,百姓遮道乞留太子,太子諭之曰:「至尊奔播,吾不忍違離左右,俟吾見上奏聞。」倓
 於行宮謂太子曰:「逆胡犯順,四海分崩,不因人情,何以興復?夫有國家者,大孝莫若存社稷。今從至尊入蜀,則散關已東,非皇家所有,何以維屬人情?殿下宜購募豪傑,暫往河西,收拾戎馬,點集防邊將卒,不下十萬人,光弼、子儀,全軍河朔,謀為興復,計之上也。」廣平王亦贊成之,於是令李輔國奏聞。玄宗欣然聽納,乃分從官、士卒以遣之。時敗卒膽破,兵仗不完,太子既北上,渡渭,一日百戰。倓自選驍騎數百衛從,每蒼黃顛沛之際,血戰在
 前。太子或過時不得食,倓涕泗不自勝,上尤憐之,軍士屬目歸於倓。至靈武,太子即帝位。廣平既為元子,欲以倓為天下兵馬元帥。侍臣曰:「廣平王塚嗣,有君人之量。」上曰:「廣平地當儲貳,何假更為元帥?」左右曰:「廣平今未冊立,艱難時人尤屬望於元帥。況太子從曰撫軍,守曰監國。今之元帥,撫軍也,廣平為宜。」遂以廣平為元帥,倓典親軍,李輔國為元帥府司馬。



 時張良娣有寵,倓性忠謇,因侍上屢言良娣頗自恣,輔國連結內外,欲傾動皇
 嗣。自是,日為良娣、輔國所構,云「建寧恨不得兵權,頗畜異志。」肅宗怒,賜倓死。既而省悟,悔之。



 明年冬,廣平王收復兩京,遣判官李泌入朝獻捷。泌與上有東宮之舊,從容語及建寧事,肅宗改容謂泌曰:「倓于艱難時實得氣力,無故為下人之所間,欲圖害其兄,朕以社稷大計,割愛而為之所也。」泌對曰:「爾時臣在河西,豈不知其故。廣平兄弟,天倫篤睦,至今廣平言及建寧,則嗚咽不已。陛下之言,出於讒口也。」帝因泣下曰:「事已及此,無如之何!」
 泌因奏曰:「臣幼稚時念《黃臺瓜辭》,陛下嘗聞其說乎?高宗大帝有八子,睿宗最幼。天后所生四子,自為行第,故睿宗第四。長曰孝敬皇帝,為太子監國,而仁明孝悌。天後方圖臨朝,乃鴆殺孝敬,立雍王賢為太子。賢每日憂惕,知必不保全,與二弟同侍於父母之側,無由敢言。乃作《黃臺瓜辭》,令樂工歌之,冀天后聞之省悟,即生哀愍。辭云:『種瓜黃臺下,瓜熟子離離。一摘使瓜好,再摘令瓜稀,三摘猶尚可,四摘抱蔓歸。』而太子賢終為天后所逐,
 死於黔中。陛下有今日運祚,已一摘矣,慎無再摘。」上愕然曰:「公安得有是言!」時廣平王立大功,亦為張皇后所忌,潛構流言,泌因事諷動之。



 及代宗即位,深思建寧之冤,追贈齊王。大歷三年五月,詔曰:「故齊王人炎,承天祚之慶,保鴻名之光。降志尊賢,高才好學,藝文博洽,智略宏通。斷必知來,謀皆先事,識無不達,理至逾精。乃者寇盜橫流,鑾輿南幸。先聖以宸扆之戀,將侍君親;惟王以宗廟之重,誓寧家國。克協朕志,載符天時,立辨群議之非,
 同獻五原之計。中興之盛,實藉奇功。景命不融,早從厚穸,天倫之愛,震惕良深。流涕追封,胙於東海,頃加表飾,未極哀榮。夫以參舊邦再造之勤,成天下一家之業,而存未峻其等,歿未尊其稱,非所以旌徽烈,明至公也。朕以眇身,纘膺大寶,不及讓王之禮,莫申太弟之嗣,所懷靡殫,邈想逾切,非常之命,寵錫攸宜。敬用追謚曰承天皇帝,與興信公主第十四女張氏冥婚,謚曰恭順皇后。有司準式,擇日冊命,改葬於順陵,仍祔於奉天皇帝廟,
 同殿異室焉。」



 衛王佖,肅宗第四子。天寶中,封西平郡王,授殿中監同正員。早薨。寶應元年五月,追贈衛王。



 彭王僅,肅宗第五子。天寶中,封新城郡王,授鴻臚卿同正員。至德二年十二月,進封彭王。乾元二年冬,史思明再陷河洛,關東用兵,人情震懼,群臣請以親王遙統兵柄。三年四月詔曰:



 古之哲王,宅中御宇,莫不內封子弟,外建籓維。故周稱百代,抑聞麟趾之美;漢命六官,亦樹
 犬牙之制。歷考前載,率由舊章。朕以薄德,纘承鴻緒,屬豺狼未殄,金革猶虞。賴文武藎臣,協心同德,庶克清於玄昆,期永保於皇圖。且授鉞分符,義已先於用武;又維城作翰,道方弘於建親。咨爾分閫之崇,成予磐石之固。彭王僅等,銀潢毓慶,璿萼分輝,忠孝稟於天成,文武稱其備用。今三秦之地,萬國來庭,誠宜列皇子以建封,崇懿籓而制勝,是資固本,委以臨戎。彭王僅可充河西節度大使,兗王僴可充北庭節度大使,涇王侹可充隴右
 節度大使,杞王倕可充陜西節度大使,興王佋可充鳳翔節度大使。



 僅,是歲薨。子鎮,授太僕卿同正員,封常山郡王。



 兗王僴,肅宗第六子。母韋妃,刑部尚書堅之妹。肅宗在東宮,選為太子妃,生僴及永和公主。堅後為李林甫誣構被誅,太子懼,奏請與妃離異,於別宮安置,僴,天寶中封潁川郡王,授太子詹事同正員。至德二年十二月,進封兗王。乾元三年,領北庭節度大使。寶應元年薨。



 涇王侹,肅宗第七子。天寶中,封東陽郡王,授光祿卿同正員。至德二載十二月,進封涇王。乾元三年,領隴右節度大使。興元元年薨。



 鄆王榮,肅宗第八子。天寶中,封靈昌郡王。早世。寶應元年五月,追贈鄆王。



 襄王僙,肅宗第九子。至德二載十二月,封襄王。貞元七年正月薨。



 杞王倕,肅宗第十子。母段婕妤,貞元六年六月贈為昭
 儀。倕,至德二載封,貞元十四年薨。



 召王偲,肅宗第十一子。至德二載十二月封,元和元年薨。



 恭懿太子佋,肅宗第十二子。至德二載封興王。上元元年六月薨。佋,皇后張氏所生,上尤鐘受。後屢危太子,欲以興王為儲貳,會薨而止。七月丁亥,詔曰:



 厚禮所以飾終,易名所以表行。況情鐘天屬,寵及褒封,載疇加等之美,式備元儲之贈,永懷軫念,有惻彞章。第十二子故興
 王佋,毓慶璿源,分華若木,天資純孝,神假聰明。河間聚書,幼聞樂善之旨;延陵聽樂,早得知音之妙。頃以暫嬰沉瘵,殆積旬時,而資敬益彰,穎晤逾爽。愛親之戀,言不間於斯須;告訣之辭,事先符於夢寐。顧惟至性,實切深哀。將胙土析珪,載崇籓翰,聞《詩》對《易》,爰就琢磨。方冀成立,豈期天喪。瑤英始茂,遽摧於當春;隙駟俄遷,忽沉於厚夜。興言痛悼,閔惜良深。宜賁寵於青宮,俾哀榮於玄穸。可贈太子,謚曰恭懿。應緣喪葬,所司準式,仍令京兆
 尹劉晏充監護使。



 詔宰臣李揆持節冊命。十一月,葬於高陽原。其哀冊曰:



 維上元元年,太歲庚子,六月己未朔,二十六日甲申,皇第十二子、持節鳳翔等四州節度觀察大使興王佋,薨於中京內邸,殯於寢之西階。粵八月丁亥,冊贈皇太子,廟號恭懿。冬十一月庚寅,詔葬於長安之高陽原,禮也。燕隧開封,龍轀進轍,陳祖載而就位,儼塗芻以成列。皇帝哀玉林之悶景,憫璿萼之罹霜,瞻龍綍而增思,懷雁池而永傷。考謚惟古,褒崇有式。爰詔
 史司,恭宣懿德。其辭曰:



 惟天祚唐,累葉重光,中興宸景,再紐乾綱。本枝建國,磐石疏疆,克開龍胤,實曰賢王。驪源孕彩,日乾騰芳,深仁廣孝,蘊藝含章。秀發童年,惠彰齔齒,蹈禮知方,承尊葉旨。對日流辯,占鳳擅美,魯、衛後塵,間、平絕軌。胡孽初構,王師未班,爰從襁褓,載歷險艱。愛備中掖,名崇懿籓,居常稟訓,動不違顏。禮及佩觿,朝加分器,胙土延渥,登壇受帥。玉質金聲,文經武緯,樂善為寶,崇儒是貴。浚哲外朗,溫文內深,閱書成誦,觀樂表
 音。《五經》在口,六律諧心,才優藝洽,絕古超今。蛇豕猶梗,寰區未乂。滌慮祈真,焚香演偈。食去葷血,心依定惠。庶福邦家,俾清兇穢。霧露嬰疾,聰明害神,沉彖始遘,彌曠盈旬。止慮無擾,發言有倫,在膏方亟,問膳逾勤。雲物告徵,星辰變象,楚藥無救,秦醫莫仗。靈儀窅而上賓,徽音邈其長往。違舊邸於青社,即幽陵於黃壤。嗚呼哀哉!魂氣奪兮去何之,精靈存兮孝有思。念君親之永隔,托夢寐而來辭。延桂宮而震悼,貫椒壼而纏悲。旌遺芳於碣
 館,賁新命於儲闈。鳴呼哀哉!先遠戒候,占龜獻吉。指鶉野而西臨,背鳳城而右出。天慘慘而苦霧,山蒼蒼而曀日。望馳道而長辭,赴幽塗而永畢。嗚呼哀哉!生為寵王兮宸愛所鐘,歿追上嗣兮朝典斯崇。升玉笙於洞府,閱銀棨於泉宮。金石誰固,人生有終,簡冊攸記兮德音無窮。敢直詞於篆美,庶永代而成風。鳴呼哀哉!



 佋薨時年八歲。既薨之夕,肅宗、張後俱夢佋有如平昔,拜辭流涕而去。帝方寢疾,追念過深,故特以儲闈之贈寵之。上疾
 累月方平。



 定王侗,肅宗第十三子。亦張後所生,佋之母弟。至德二載,封定王。寶應初薨,時年甚幼。



 宋王僖,肅宗第十四子。初封淮陽王,早夭,追封宋王。



 代宗皇帝二十子:睿真皇後沈氏生德宗皇帝,崔妃生昭靖太子,獨孤皇后生韓王迥;餘十七王,舊史不載母氏所出。



 昭靖太子邈,代宗第二子。寶應元年,封鄭王。大歷初,代
 皇太子為天下兵馬元帥。王好讀書,以儒行聞。大歷九年薨,廢朝三日,由是罷元帥之職。上惜其才早夭,冊贈昭靖太子,葬於萬年縣界。



 均王遐,代宗第三子。早夭,貞元八年追封。



 睦王述,代宗第四子。大歷九年冬,田承嗣謀亂河朔,時鄭王居長,典兵師,不幸薨落,諸王皆幼,多未封建。大臣奏議請封親王,分領戎師,以威天下。十年二月,詔曰:



 虞、夏之制,諸子疏封;漢、魏以還,十連授律。是用錫珪班瑞,
 盤石開疆,信通邑之紀綱,為中都之屏翰。然則旌鉞之寄,推擇攸難,因親之任,各膺其命。第四子述、第五子逾、第六子連、第七子迥、第八子遘、第十三子造、第十四子暹、第十五子運、第十六子遇、第十七子遹、第十八子通、第十九子逵、第二十子逸等,並敏茂純懿,稟於衷誠,溫良孝恭,形於進對,動皆合義,居必有常。可以理眾靖人,撫封宣化,而總列城之賦,繕分閫之謀,克勤公家,允輔王室。今則均茅社之寵,盛槐庭之儀,授鉞登車,嗣茲朝
 典,維城之固,爾其懋哉。述可封睦王,充嶺南節度支度營田、五府經略觀察處置等大使;逾可封郴王,充渭北鄜、坊等州節度大使;連可封恩王;韓王迥可汴、宋等節度觀察處置等大使;遘可封鄜王;造可封忻王,充昭義軍節度觀察處置等大使;暹可封韶王,運可封嘉王,遇可封端王,遹可封循王,通可封恭王,逵可封原王,逸可封雅王:仍並可封開府儀同三司。



 是時,皇子勝衣者盡加王爵,不出閣。德宗朝,述為諸王之長。時分命中使周
 行天下,求訪沈太后,詔以睦王為奉迎太后使,以工部尚書喬琳副之。貞元七年薨。



 丹王逾,代宗第五子。大歷十年,封郴王,領渭北鄜坊節度大使。建中四年,改丹王。元和十五年薨。



 恩王連,代宗第六子。大歷十年封,元和十二年薨。



 韓王迥,代宗第七子。以母寵,既生而受封,雖沖幼,恩在鄭王之亞。寶應元年,封韓王。貞元十二年薨,時年四十七。



 簡王遘,代宗第八子。大歷十年,封鄜王,建中四年,改封簡王。元和四年薨。



 益王乃,代宗第九子。大歷四年封。



 隋王迅,代宗第十子。大歷十年封,興元元年薨。



 荊王選,代宗第十一子,早世。建中二年正月,追封荊王,贈開府儀同三司。



 蜀王溯,代宗第十二子。大歷十四年封。本名遂,建中二年改今名。



 忻王造,代宗第十三子。大歷十年封,仍領昭義軍節度觀察大使。元和六年薨。



 韶王暹,代宗第十四子。大歷十年封,貞元十二年薨。



 嘉王運,代宗第十五子。大歷十年封,貞元十七年薨。



 端王遇,代宗第十六子。大歷十年封,貞元七年薨。



 循王遹,代宗第十七子。大歷十年封。



 恭王通,代宗第十八子。大歷十年封。



 原王逵,代宗第十九子。大歷十年封。大和六年薨。



 雅王逸,代宗第二十子。大歷十年封,貞元十五年薨。



 史臣曰:艷妻破國,孽子敗宗。前代英傑之君,率不免於斯累者,何也?良以愛惡不由於義斷,毀譽遽逐於情移。雖申生孝己之仁,卒不能回君父之愛,悲哉!孝宣皇帝當屯剝之運,收忠義之心,忍行愛子之刑,終宥奸閹之罪,大雅君子,為之痛心。張後卒以兇終,固其宜矣。



 贊曰:床簀之愛,人情易惑。以義制情,哲王令德。李侯悟主,韻諧金石。褒謚建寧,良堪太息。



\end{pinyinscope}