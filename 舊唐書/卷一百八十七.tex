\article{卷一百八十七}

\begin{pinyinscope}

 ○獨孤懷恩竇德明侄懷貞族弟孝諶孝諶子希瑊希球希瓘希瓘從父弟維鍌長孫敞從父弟操趙持滿附武承嗣子延秀從父弟三思三思子崇訓從祖弟懿宗攸暨攸暨妻太平公主從父弟攸緒薛懷義附
 韋溫王仁皎子守一吳漵弟湊竇覦柳晟王子顏



 自古後族,能以德禮進退、全宗保名者,鮮矣。蓋恃宮掖之寵,接宴私之歡,高爵厚祿驕其內,聲色服玩惑於外;莫知師友之訓,不達危亡之道。故以中才處之,罕不覆敗;亦由重植之木,自然顛披也。明哲之君,知驕侈之易滿,榮寵之難保;授任各當其才,祿位不過其量;告之以天命不易,誡之以大義滅親;使居無過之地,永享不貲
 之福,與國終始,不失其所以親也。《易》曰:「震來虩虩,恐致福也。」又曰:「婦子嘻嘻,失家節也。」與其愛而失節,曷若懼而致福?魏氏懲漢人之敗,著矯枉之法:幼主嗣位,母後不得臨朝;外氏無功,時主不得封爵。雖曰刻薄,而卞、甄之族,竟無大過。皇唐受命,長孫、竇氏以勛賢任職,而武氏、韋氏以盈滿致覆。夫廢興者,豈天命哉,蓋人事也!竇威、長孫無忌各自有傳,其餘載其得失,為《外戚傳》,以存鑒誡焉。###
 獨孤懷恩,元貞皇后弟之子也。父整,隋涿郡太守。懷恩幼時,以獻皇后之侄,養於宮中。後仕為鄠縣令。高祖平京城,授長安令。在職嚴明,甚得時譽。及高祖受禪,擢拜工部尚書。時虞州刺史韋義節擊堯君素於蒲州,而義節文吏怯懦,頻戰不利。高祖遣懷恩代總其眾。懷恩督兵城下,為賊所拒,頻戰不利。高祖切讓之,因是怨望。高祖嘗戲之曰:「弟姑子悉為天子,次當舅子乎?」懷恩遂自以為符命,每扼腕曰:「我家豈女獨富貴耶?」由是陰圖異
 計。



 時虞鄉南山多群盜。劉武周將宋金剛寇陷澮州,高祖悉發關中卒以隸太宗,屯於柏壁。懷恩遂與解縣令榮靜、前五原縣主簿元君寶謀引王行本兵及武周連和,與山賊劫永豐倉而斷柏壁糧道,割河東地以啗武周。事臨發,會夏縣人呂崇茂殺縣令,據縣起兵,應武周。高祖遣懷恩與永安王孝基、陜州總管於筠、內史侍郎唐儉攻崇茂。宋金剛潛兵來襲,諸將盡沒。君寶與開府劉讓亦同陷於賊中,遂洩懷恩之謀。既而懷恩逃歸,高
 祖復令率師攻蒲州。唐儉在賊中,說賊將尉遲敬德,請使讓還,連和罷兵,遂使發其事。會堯君素為其下所殺,小帥王行本以蒲州降,懷恩勒兵入據其城。高祖將濟河,已御舟矣,會讓至,乃使召懷恩,懷恩不知事已洩,輕舟來赴。及中流而執之,收其黨按驗,遂誅之,時年三十六,籍沒基家。



 竇德明,太穆順聖皇后兄之孫也。祖照,尚後魏文帝女義陽公主,封鉅鹿公。父彥,襲父封,仕隋為西平郡守。德
 明少師事陳留王孝逸,頗涉文史。會漢王諒作亂,遣其將綦良攻黎州。德明時年十八,募得五千人,倍道而進,號令嚴整,一戰破之。以功累拜齊王府屬,坐事免。及義師圍長安,永安王孝基、襄邑王神符、江夏王道宗及高祖之婿竇誕、趙慈景並系獄,隋將衛文升、陰世師欲殺之。德明謂文升曰:「罪不在此輩,殺之無傷於彼,適足招怨。」文升乃止。及謁見高祖,竟不自言,時人稱其長者。武德初,拜考功郎中。從太宗擊王世充,頻有戰功,封顯武
 男。貞觀初,歷常、愛二州刺史。尋卒。



 弟德玄,高宗時為左相。德玄子懷貞。



 懷貞,少有名譽,時兄弟宗族,並以輿馬為事,懷貞獨折節自修,衣服儉素。聖歷中為清河令,治有能名。俄歷越州都督、揚州大都督府長史,所在皆以清乾著稱。



 神龍二年,累遷御史大夫,兼檢校雍州長史。時韋庶人及安樂公主等干預朝政。懷貞每諂順委曲取容,改名從一,以避后父之諱,自是名稱日損。庶人微時乳母王氏,本蠻婢也,特封莒國夫人,嫁為懷貞妻。俗
 謂乳母之婿為阿沄,懷貞每因謁見之次及進表疏,列在官位,必曰「皇后阿沄」,時人或以「國沄」呼之,初無慚色。宦官用權,懷貞尤所畏敬,每視事聽訟,見無須者,誤以接之。監察御史魏傳弓嘗以內常侍輔信義尤縱暴,將奏劾之。懷貞曰:「輔常侍深為安樂公主所信任,權勢甚高,言成禍福,何得輒有彈糾?」傳弓曰:「今王綱漸壞,君子道消,正由此輩擅權耳!若得今日殺之,明日受誅,無所恨。」懷貞無以答,但固止之。



 韋庶人敗,左遷濠州司馬。尋
 擢授益州大都督府長史。以附會太平公主,累拜侍中、兼御史大夫,代韋安石為尚書左僕射,監修國史,賜爵魏國公。



 睿宗為金仙、玉真二公主創立兩觀,料功甚多,時議皆以為不可。唯懷貞贊成其事,躬自監役。懷貞族弟詹事司直維鍌謂懷貞曰:「兄位極臺袞,當思獻可替否,以輔明主。奈何校量瓦木,廁跡工匠之間,欲令海內何所瞻仰也?」懷貞不能對,而監作如故。時人為之語曰:「竇僕射前為韋氏國沄,後作公主邑丞。」言懷貞伏事公
 主,同於邑官也。



 先天二年,太平公主逆謀事洩,懷貞懼罪,投水而死。追戮其尸,改姓毒氏。



 德明族弟孝諶。



 孝諶,刑部尚書誕之子,昭成順聖皇后父也。則天時,歷太常少卿、潤州刺史。長壽二年,後母龐氏被酷吏所陷,誣與後咒詛不道,孝諶左遷羅州司馬而卒。



 子希瑊、希球、希瓘,並流嶺南。神龍初,隨例雪免。景雲年,追贈李諶太尉、邠國公,希瑊襲爵。玄宗即位,加贈孝諶太保,希瑊等以舅氏,甚見優寵。



 希瑊累遷太子少傅、豳國公,尋卒。



 希球
 官至太子賓客,封冀國公,開元二十七年卒。及卒,謚曰靖。



 希瓘初賜爵畢國公,後改名曳。初為左散騎常侍,及希球卒,因授開府儀同三司。玄宗以早失太后,尤重外家,曳兄弟三人皆國公,食實封。



 曳子鍔,又尚玄宗女永昌長公主,恩寵賜賚,實為厚矣。而兄弟皆貪鄙,過自封植,曳又甚之。



 天寶七年,有竇勉潛交巫祝,勉犯法,曳坐信其詭說,被停官,放歸田園。尋以尊老,又授開府儀同三司,依舊朝會。十三載十二月卒,玄宗哭於行在,贈司
 徒。財貨鉅萬。



 瑨曳從父弟維鍌,好學,以撰著為業。時宗族咸以外戚,崇飾輿馬,維鍌獨清儉自守。中書令張說、黃門侍郎盧藏用、給事中裴子餘皆與之親善。官至水部郎中卒。撰《吉兇禮要》二十卷,行於代。



 長孫敞,文德順聖皇后之叔父也。仕隋為左衛郎將。煬帝幸江都,留敞守京城禁苑。及義旗入關,率子弟迎謁於新豐,從平京城,以功除將作少監。出為杞州刺史。貞觀初,坐贓免。太宗以後親,常令內給絹以供私費。尋拜
 宗正少卿致仕,加金紫光祿大夫,累封平原郡公。卒,贈幽州都督,謚曰良,陪葬昭陵。



 敞從父弟操,周大司徒、薛國公覽之子也。武德中,為陜東道行臺金部郎中,出為陜州刺史。自州東引水入城,以代井汲,百姓於今利之。貞觀中,歷洺州刺史、益揚二州都督府長史,並有善政。二十三年,以子詮尚太宗女新城公主,拜岐州刺史。永徽初,加金紫光祿大夫,賜爵樂壽男。尋卒,贈吏部尚書、並州都督,謚曰安。



 詮官至尚書奉御。詮即侍中韓瑗妻
 弟也。及瑗得罪,事連於詮,減死配流巂州。詮至流所,縣令希旨杖殺之。



 詮之甥有趙持滿者,工書善射,力搏猛獸,捷及奔馬;而親仁愛眾,多所交結,京師無貴賤皆愛慕之。初為涼州長史,嘗逐野馬,自後射之,無不洞於胸腋,邊人深伏之。許敬宗懼其作難,誣與詮及無忌同反。及拷訊,終無異詞,且曰:「身可殺,辭不可奪。」吏竟代為款以殺之。



 武承嗣,荊州都督士矱之孫,則天順聖皇后兄子也。初,
 士矱娶相里氏,生元慶、元爽。又娶楊氏,生三女:長適越王府功曹賀蘭越石,次則天,次適郭氏。士矱卒後,兄子惟良、懷運及元爽等遇楊氏失禮。及則天立為皇后,追贈士矱為司徒、周忠孝王,封楊氏代國夫人。賀蘭越石早卒,封其妻為韓國夫人。尋又加贈士矱為太尉,楊氏改封為榮國夫人。時元慶仕為宗正少卿,元爽為少府少監,惟良為衛尉少卿。榮國夫人恨其疇日薄己,諷皇后抗疏請出元慶等為外職,佯為退讓,其實惡之也。於
 是元慶為龍州刺史,元爽為濠州刺史,惟良為始州刺史。元慶至州病卒,元爽自濠州又配流振州而死。



 乾封年,惟良與弟淄州刺史懷運,以岳牧例集於泰山之下。時韓國夫人女賀蘭氏在宮中,頗承恩寵。則天意欲除之,諷高宗幸其母宅,因惟良等獻食,則天密令人以毒藥貯賀蘭氏食中,賀蘭氏食之,暴卒,歸罪於惟良、懷運,乃誅之。仍諷百僚抗表請改其姓為蝮氏,絕其屬籍。元爽等緣坐配流嶺外而死,乃以韓國夫人之子敏之為
 士矱嗣,改姓武氏,累拜左侍極、蘭臺太史,襲爵周國公。仍令鳩集學士李嗣真、吳兢之徒,於蘭臺刊正經史,並著撰傳記。



 敏之既年少色美,烝於榮國夫人,恃寵多愆犯,則天頗不悅之。咸亨二年,榮國夫人卒,則天出內大瑞錦,令敏之造佛像追福,敏之自隱用之。又司衛少卿楊思儉女有殊色,高宗及則天自選以為太子妃,成有定日矣,敏之又逼而淫焉。及在榮國服內,私釋衰絰,著吉服,奏妓樂。時太平公主尚幼,往來榮國之家,宮人侍
 行,又嘗為敏之所逼。俄而奸污事發,配流雷州,行至韶州,以馬韁自縊而死。



 承嗣,元爽子也。敏之死後,自嶺南召還,拜尚衣奉御,襲祖爵周國公。俄遷秘書監。則天臨朝,追尊士矱為忠孝太皇,置崇先府官屬,五代祖已下,皆為王。嗣聖元年,以承嗣為禮部尚書。尋除太常卿、同中書門下三品。垂拱中,轉春官尚書,依舊知政事。載初元年,代蘇良嗣為文昌左相、同鳳閣鸞臺三品,兼知內史事。



 天授元年,於東都創置武氏七廟,追尊周文王為
 始祖文皇帝,王子武為睿祖康皇帝,云武氏之先也。後五代祖贈太原靖王居常為嚴祖成皇帝,高祖贈趙肅恭王克己為肅祖章敬皇帝,曾祖贈魏康王儉為烈祖昭安皇帝,祖贈周安成王華為顯祖文穆皇帝,考忠孝太皇為太祖孝明高皇帝,妣皆隨帝號曰皇后。元慶為梁憲王,元爽為魏德王。又追封伯父及兄弟俱為王,諸姑姊為長公主。於是封承嗣為魏王,元慶子夏官尚書三思為梁王,後從父兄子納言攸寧為建昌王,太子通
 事舍人攸歸為九江王,司禮卿重規為高平王,左衛親府中郎將載德為潁川王,右衛將軍攸暨為千乘王,司農卿懿宗為河內王,左千牛中郎將嗣宗為臨川王,右衛勛二府中郎將攸宜為建安王,尚乘直長攸望為會稽王,太子通事舍人攸緒為安平王,攸止為恆安王。又封承嗣男延基為南陽王,延秀為淮陽王,三思男崇訓為高陽王,崇烈為新安王,後兄子贈陳王承業男延暉為嗣陳王,延祚為咸安王。



 承嗣嘗諷則天革命,盡誅皇
 室諸王及公卿中不附己者,承嗣從父弟三思又盛贊其計,天下於今冤之。俄又賜承嗣實封千戶,仍監修國史。承嗣自為次當為皇儲,令鳳閣舍人張嘉福諷諭百姓抗表陳請,則天竟不許。如意元年,授特進。尋拜太子太保,罷知政事。承嗣以不得立為皇太子,怏怏而卒,贈太尉、並州牧,謚曰宣。



 子延基襲爵,則天避其父名,封為繼魏王。尋與其妻永泰郡主及懿德太子等,話及張易之兄弟出入宮中,恐有不利,後忿爭不協,洩之,則天聞
 而大怒,咸令自殺。復以承嗣次子延義為繼魏王。



 中宗即位,侍中敬暉等以唐室中興,武氏諸王宜削其王爵,乃率群官上表曰:



 臣聞神器者,天下之至公,必歸乎有德;皇極者,域中之大寶,必順乎天命。歷考前古,祥觀帝業,皆不並興,莫有二主。故三皇氏沒而五帝氏興,夏、商氏衰而周、漢氏作。何則?帝王之歷數,心應乎五行,水盛則火衰,木衰則金盛。天地之運也,合乎四時,春往則夏來,暑退則寒集。則知五行之數也,帝王不可違,違之則
 宗社不安,生人不理。四時之序,天地不能變,變之則霜露不均,水旱交錯。



 自有隋失御,海內崩離,天歷之重,歸於唐室。萬方樂業,荷撥亂之功;三聖重光,布生成之德。可謂有功於四海,有德於蒸人。自弘道遏密,生靈降禍,百闢哀號,如喪考妣。



 則天皇后臨御帝圖,明目達聰,躬親庶績。則有讒邪兇孽,誣惑睿德,構害宗枝,誅夷殆盡。英籓賢戚,百不一存,餘類在者,投竄荒裔。冤酷人神,感傷天地,忠臣義士,實所痛心。自天授之際,時稱改革,武
 家子侄,咸樹封建,十餘年間,實亦榮極。於時唐室籓屏,豈得並封,故知事有升降,時使然也。



 今則天皇帝厭倦萬機,神器大寶,重歸陛下。百姓謳歌,欣復唐業,上至卿士,下及蒼生,黃發之倫,童兒之輩,莫不歡欣舞忭,如見父母。豈不以唐家恩德,感幽祇之心;陛下仁明,順天下之望?今皇業重構,聖祚中興,神祇之道,有助於先德矣!黎人之誠,無負於陛下矣!臣又聞之,業不兩盛,事不兩大,故天無二日,土無二王,前聖之格言,先哲之明誡。自
 皇明反正,天命惟新,武家諸王,封建依舊,生者既加茅土,死者仍追賦邑,萬姓失望,卿士寒心,何則?開闢以來,罕有斯理,帝王之道,實無此法。陛下縱欲開恩,以行私惠,豈可違五行之歷數乎?乖四時之寒暑乎?



 又海內眾情,朝廷竊議,為武氏諸王身計,亦適將有損。何則?處之未得其所,居之實恐不安,陛下雖欲寵之,翻乃陷之,不遵古典故也。且唐歷有歸,周命已去;爵重則難保,祿薄則易全。又則天皇帝親政之時,武氏諸王,亦分外職。今
 居京輦,不降舊封,天下之心,竊將不可。陛下縱欲敦崇外戚,曲流恩貸,奈宗廟社稷之計何?奈卿士黎庶之議何?



 伏願陛下為社稷之遠圖,割私情之小愛,內崇經邦之要,外順遐邇之心,豈不固宗社之基,允人靈之願?則陛下巍巍之業,貫三光而洞九泉。親親之義,上有倫而下有序。臣特承榮寵,思竭丹赤,既為唐臣,實為唐計,伏乞聖慈,俯垂矜納。



 中書舍人岑義之詞也。上答曰:



 朕嘗因暇景,博覽前修,帝籍皇圖,略稽其跡。至若二靈肇判,
 三才聿興,驪連粟陸之辰,尊盧大庭之日,時猶樸略,未著圖書。洎乎出震應期,畫八卦而成象;炎皇御歷,播百穀以興農。車服創於軒轅之朝,歷象建於唐堯之代,封建之事,闕爾無聞。自周漢已來,方崇蕃屏。至於三微更王,五運迭興,以古揆今,事跡有爽。



 比者別宗撫歷,異姓興邦,伏以則天大聖皇帝,內輔外臨,將五十載,在朕躬則為慈母,於士庶即是明君。往者垂拱之中,嗣皇臨政,當此之際,魯衛並存。及乎全節興妖,瑯邪構逆,災連七
 國,釁結三監,既行大義之懷,遂有泣誅之事。周唐革命,蓋為從權,子侄封王,國之常典。卿等表云「天授之際,武家封建,唐家籓屏,豈得並封」者,至如千里一房,不預逆謀,還依姓李,無改舊惠,豈非善惡區分,申明逆順矣?今以聖上乖豫,高枕怡神,委政朕躬,纂承丕緒。昨者二月之首,攸暨等屢請削封,朕獨斷襟懷,不依來請。昔漢祖以布衣取天下,猶封異姓為王,況朕以累聖開基,豈可削封外族?群公等以「天無二日,土無二王」,抗表紫庭,用
 申丹懇者。然以賞罰之典,經國大綱,攸暨、三思,皆悉預告兇豎,雖不親冒白刃,而亦早獻丹誠,今若卻除舊封,便慮有功難勸。



 於是降封梁王三思為德靜郡王,量減實封二百戶,定王、駙馬都尉攸暨為樂壽郡王,河內郡王懿宗為耿國公,建昌郡王攸寧為江國公,會稽郡王攸望為鄴國公,臨川郡王嗣宗為管國公,建安郡王攸宜為息國公,高平郡王重規為鄶國公,繼魏王延義為魏國公,安平郡王攸緒為巢國公,高陽郡王、駙馬都尉
 崇訓為酆國公,淮陽郡王延秀為桓國公,咸安郡王延祚為咸安郡公。



 中宗時,嗣宗至曹州刺史,攸宜工部尚書,重規岐州刺史,相次病卒。攸望至太常卿,左遷春州司馬而死。延秀伏誅後,武氏宗屬緣坐誅死及配流,殆將盡矣。先天二年,制削士獲帝號,依舊追贈太原王,妻楊氏亦削後號,依舊為太原王妃。



 延秀,承嗣第二子也。則天時,突厥默啜上言有女請和親,制延秀與閻知微俱往突厥,將親迎默啜女為妻。既而默啜執知微,入冠
 趙、定等州,故延秀久不得還。神龍初,默啜更請通和,先令延秀送款,始得歸,封桓國公,又授左衛中郎將。時武崇訓為安樂公主婿,即延秀從父兄,數引至主第。延秀久在蕃中,解突厥語,常於主第,延秀唱突厥歌,作胡旋舞,有姿媚,主甚喜之。及崇訓死,延秀得幸,遂尚公主。



 主,韋後所生男女中最小。初,中宗遷於房州,欲達州境,生於路次。性惠敏,容質秀絕。中宗、韋后愛龐日深,恣其所欲,奏請無不允許。恃寵橫縱,權傾天下,自王侯宰相已
 下,除拜多出其門。所營第宅並造安樂佛寺,擬於宮掖,巧妙過之。令楊務廉於城西造定昆池於其莊,延袤數里。出降之時,以皇后仗發於宮中,中宗與韋後御安福門觀之,燈燭供擬,徹明如晝。延秀拜度日,授太常卿,兼右衛將軍、駙馬都尉,改封恆國公,實封五百戶。廢休祥宅,於金城坊造宅,窮極壯麗,帑藏為之空竭。崇訓子數歲,因加金紫光祿大夫、太常卿同正員、左衛將軍,封鎬國公,賜實封五百戶,以嗣其父。公主產男滿月,中宗、韋
 後幸其第,就第放赦,遣宰臣李嶠、文士宋之問、沈佺期、張說、閻朝隱等數百人賦詩美之。



 延秀既恃恩,放縱無所忌憚。又公主府倉曹符鳳知延秀有不臣之心,遂說曰:「今天下蒼生,猶以武氏為念,大周必可再興。按讖書云『黑衣神孫披天裳』,駙馬即神皇之孫也。」每勸令著皁襖子以應之。及韋庶人敗,延秀與公主在內宅,格戰良久。皆斬之。後追貶為悖逆庶人。



 三思,元慶子也。少以後族累轉右衛將軍。則天臨朝,擢拜夏官尚書。及革命,封
 梁王,賜實封一千戶。尋拜天官尚書。證聖元年,轉春官尚書,監修國史。聖歷元年,檢校內史。二年,進拜特進、太子賓客,仍並依舊監修國史。



 三思略涉文史,性傾巧便僻,善事人,由是特蒙信任。則天數幸其第,賞賜甚厚。時薛懷義、張易之、昌宗皆承恩顧。三思與承嗣每折節事之。懷義欲乘馬,承嗣、三思必為之執轡。又贈昌宗詩,盛稱昌宗才貌是王子晉後身,仍令朝士遞相屬和。三思又以則天厭居深宮,又欲與張易之、昌宗等扈從馳騁,
 以弄其權。乃請創造三陽宮於嵩高山,興泰宮於萬壽山,請則天每歲臨幸,前後工役甚眾,百姓怨之。



 神龍初,進拜司空、同中書門下三品,加實封五百戶,固辭不受。未幾,隨例降封為德靜郡王,量減實封二百戶。尋拜左散騎常侍,則天遺制令復其所減實封。



 初,敬暉等立功後,掌知國政,三思慮其更為己患,而令其子崇訓因安樂公主構誣敬暉等,並流於嶺表而死。自是三思威權日盛,軍國政事,多所參綜。敬暉等所斥黜者,皆能引復
 舊職,令百官復修則天之法。時人皆言其陰懷篡逆,以比曹孟德、司馬仲達。



 雍州人韋月將、高軫等並上疏言三思父子必為逆亂。三思知而求索其罪。有司希旨,奏:「月將坐當棄市,軫配流嶺外」。黃門侍郎宋璟執奏云:「月將所犯,不合至死。」三思怒,竟斥宋璟為外職。三思既猜嫉正士,嘗言「不知何等名作好人,唯有向我好者,是好人耳。」又與其所親兵部尚書宗楚客、將作大匠宗晉卿、太府卿紀處訥、鴻臚卿甘元柬遞相引致,干黷時政。侍
 御史周利用、冉祖雍,太僕丞李悛,光祿丞宋之遜,監察御史姚紹之等五人,常為其耳目,時人呼為「三思五狗」。



 中宗尋又制:武氏崇恩廟,一依天授時舊禮享祭,其吳陵、順陵,並置官員,皆三思意也。



 三思既與韋庶人及上官昭容私通,嘗忌節愍太子,又因安樂公主密謀廢黜之。三年七月,太子率羽林大將軍李多祚等,發左右羽林兵,殺三思及其子崇訓於其第,並殺其親黨十餘人。俄而事變,太子既死,中宗為三思舉哀,廢朝五日,贈太
 尉,追封梁王,謚曰宣。安樂公主又以節愍太子首致祭於三思及崇訓靈柩前。睿宗踐祚,以三思父子俱有逆節,制令斫棺暴尸,平其墳墓。



 崇訓,三思第二子也。則天時,封為高陽郡王。長安中,尚安樂郡主。時三思用事於朝,欲寵其禮。中宗為太子在東宮,三思宅在天津橋南,自重光門內行親迎禮,歸於其宅。三思又令宰臣李嶠、蘇味道,詞人沈佺期、宋之問、徐彥伯、張說、閻朝隱、崔融、崔湜、鄭愔等賦《花燭行》以美之。其時張易之、昌宗、宗楚
 客兄弟貴盛,時假詞於人,皆有新句。崇訓授左衛中郎將。神龍元年,拜駙馬都尉,遷太常卿,兼左衛將軍。降封酆國公,仍賜實封五百戶,尋徙封鎬國公。二年,兼太子賓客,攝左衛將軍。及為節愍太子所殺,優制贈開府儀同三司,追贈魯王,謚曰忠。



 懿宗,則天伯父士逸之孫也。父元忠,高宗時仕至倉部郎中。天授年,封士逸為蜀王,懿宗封為河內郡王,歷遷洛州長史、左金吾衛大將軍。萬歲通天年中,契丹賊帥孫萬榮寇河北,命懿宗為大
 總管討之。軍次趙州,用聞賊將至冀州,懿宗懼,便欲棄軍而遁。人或謂曰:「賊眾極多,然其軍無輜重,以抄掠為資,若按兵以守,勢必離散,因而擊之。可有大功也。」懿宗不聽,遂退據相州,時人嗤其怯懦。由是賊眾進屠趙州而去。尋又令懿宗安撫河北諸州。



 先是,百姓有脅從賊眾,後得歸來者。懿宗以為同反,總殺之。仍生刳取其膽,後行刑,流血盈前,言笑自若。初。孫萬榮別帥何阿小攻陷冀州,亦多屠害士女。至是,時人號懿宗與阿小為兩
 何,為之語曰:「唯此兩何,殺人最多。」懿宗又自天授已來,嘗受中旨,推鞫制獄,王公大臣,多被陷成其罪,時人以為周興、來俊臣之亞焉。神龍初,隨例降爵,封耿國公,累轉懷州刺史,尋卒。



 攸暨,則天伯父士讓孫也。天授中,封士讓為楚王,攸暨封千乘郡王。賜爵實封三百戶。兄攸寧為建昌郡王,實封四百戶。攸寧歷遷鳳閣侍郎、納言、冬官尚書,病卒。



 攸暨初為右衛中郎將,尚太平公主,授駙馬都尉。累遷右衛將軍,進封定王,又加實封三百戶。
 俄又改安定郡王,歷遷司禮卿、左散騎常侍,加特進。神龍中,拜司徒,復封定王,實封滿一千戶,固辭不拜。尋而隨例降封樂壽郡王,拜右散騎常侍,加開府儀同三司。延秀等誅後,又降封楚國公。延和元年卒,贈太尉、並州大都督,追封定王。尋以公主謀逆,令平毀其墓。



 太平公主者,高宗少女也。以則天所生,特承恩寵。初,永隆年降駙馬薛紹。紹,垂拱中被誣告與諸王連謀伏誅,則天私殺攸暨之妻以配主焉。公主豐碩,方額廣頤,多權略,則
 天以為類己,每預謀議,宮禁嚴峻,事不令洩。公主亦畏懼自檢,但崇飾邸第。二十餘年,天下獨有太平一公主,父為帝,母為後,夫為親王,子為郡王,貴盛無比。永淳已前朝制,親王食實封八百戶,有至一千戶;公主出降三百戶,公主加五十戶。太平食湯沐之邑一千二百戶,聖歷初加至三千戶。



 神龍元年,預誅張易之謀有功,進號鎮國太平公主,相王加號安國相王,並食實封通前五千戶,賞賜不可勝紀。公主孽氏二男二女,武氏二男一
 女,並食實封。又相王、衛王重俊、成王千里宅,遣衛士宿衛,環其所居,十步置一仗舍,持兵巡徼,同於宮禁。太平、長寧、安樂三公主,置鋪一如親王。二年正月,置公主府。景龍二年,公主男崇簡、崇敏、崇行,同授三品,與漁陽王兄弟四人同制。時中宗仁善,韋后、上官昭容用事禁中,皆以為智謀不及公主,甚憚之。公主日益豪橫,進達朝士,多至大官,詞人後進造其門者,或有貧窘,則遺之金帛,士亦翕然稱之。



 及唐隆元年六月,韋後作逆稱制,偽
 尊溫王。玄宗居臨淄邸,憤之,將清內難。公主又預其謀,令男崇簡從之。及立溫王,數日,天下之心歸於相府,難為其議。公主入啟幼主,以王室多故,資於長君,乃提下幼主,因與玄宗、大臣尊立睿宗。公主頻著大勛,益尊重,乃加實封五千戶,通前滿一萬戶。公主子崇行、崇敏、崇簡三人,封異姓王;崇行國子祭酒,四人九卿三品。每入奏事,坐語移時,所言皆聽。薦人或驟歷清職,或至南北衙將相,權移人主。軍國大政,事必參決,如不朝謁,則宰
 臣就第議其可否。



 公主由是滋驕,田園遍於近甸膏腴,而市易造作器物,吳、蜀、嶺南供送,相屬於路。綺疏寶帳,音樂輿乘,同於宮掖。侍兒披羅綺,常數百人,蒼頭監嫗,必盈千數。外州供狗馬玩好滋味,不可紀極。有胡僧惠範,家富於財寶,善事權貴,公主與之私,奏為聖善寺主,加三品,封公,殖貨流於江劍。公主懼玄宗英武,乃連結將相,專謀異計。其時宰相七人,五出公主門,常元楷、李慈掌禁兵,常私謁公主。



 先天二年七月,玄宗在武德殿,
 事漸危逼,乃勒兵誅其黨竇懷貞、蕭至忠、岑羲等,公主遽入山寺,數日方出,賜死於家。公主諸子及黨與死者數十人。籍其家,財貨山積,珍奇寶物,侔於御府,馬牧羊牧田園質庫,數年征斂不盡。惠範家產亦數十萬貫。



 攸緒,惟良子也。少有志行。天授中封安平郡王,歷遷殿中監,出為揚州大都督府長史。聖歷中,棄官隱於嵩山,以琴書藥餌為務。中宗即位,以車安備禮徵之,降書曰:



 朕聞大隱忘情,不去朝市,至人無跡,何所凝滯。王高標峻
 尚,雅操孤貞;有咸一之用,弘體二之德;學究深遠,理實精微。草芥貂蟬,錙銖纓紱;廕松山而辭竹苑,去硃邸而臥清溪;逍遙林壑,傲睨箕潁,有年歲矣。



 朕虔膺聖歷,重闡皇基;保乂邦家,寧輯區宇;求賢採彥,俯穀窺山。王之所居,接近嵩岳,長望高烈,思滿風煙。駐驆喬巖,追尋大隗;鳴鑾峒岫,詢訪廣成;機務殷繁,有懷莫遂。今遣國子司業杜慎盈以禮命征闢,掃夔、龍之弟,虛稷、契之筵,神化丹青,朕之志也。豈以黃屋之貴,傾彼白雲之心?通變
 之宜,希從降志;延貯閶闔,若在汾陽。



 攸緒應召至都,授太子賓客。尋請歸嵩山,制從之,令京官五品已上餞送於定鼎門外。



 及三思、延秀等構逆,諸武多坐誅戮,唯攸緒以隱居不預其禍,時論美之。睿宗即位,又降敕曰:「頃以賊臣結黨,後族擅權,扇動宮闈,肆行鴆毒。靈祇所感,奸惡伏誅;今得宗社乂安,天地交泰。卿久厭簪紱,早慕林泉,守道不回,見幾而作,興言高尚,有足嘉稱。但怒用不遷,罪無相及,為善有驗,卿之謂與!或慮驚疑,故令慰
 謝。」其見重如此。尋徵為太子賓客,不就。開元二年,攸緒又請就廬山居止,制不許。仍令州縣數加存問,不令外人侵擾。十一年卒,年六十九。



 薛懷義者,京兆鄠縣人,本姓馮,名小寶。以鬻臺貨為業,偉形神,有膂力,為市於洛陽,得幸於千金公主侍兒。公主知之,入宮言曰:「小寶有非常材用,可以近侍。」因得召見,恩遇日深。則天欲隱其跡,便於出入禁中,乃度為僧。又以懷義非士族,乃改姓薛,令與太平公主婿薛紹合
 族,令紹以季父事之。自是與洛陽大德僧法明、處一、惠儼、稜行、感德、感知、靜軌、宣政等在內道場念誦。懷義出入乘廄馬,中官侍從,諸武朝貴,匍匐禮謁,人間呼為薛師。



 垂拱初,說則天於故洛陽城西修故白馬寺,懷義自護作。寺成,自為寺主。頗恃恩狂蹶,其下犯法,人不敢言。右臺御史馮思勖屢以法劾之;懷義遇勖於途,令從者毆之,幾死。又於建春門內敬愛寺別造殿宇,改名佛授記寺。



 垂拱四年,拆乾元殿,於其地造明堂,懷義充使督
 作。凡役數萬人,曳一大木千人,置號頭,頭一,千人齊和。明堂大屋凡三層,計高二百尺。又於明堂北起天堂,廣袤亞於明堂。懷義以功拜左威衛大將軍,封梁國公。



 永昌中,突厥默啜犯邊,以懷義為清平道大總管,率軍擊之,至單于臺,刻石紀功而還。加輔國大將軍,進右衛大將軍,改封鄂國公、柱國,賜帛二千段。



 懷義與法明等造《大雲經》,陳符命,言則天是彌勒下生,作閻浮提主,唐氏合微。故則天革命稱周,懷義與法明等九人並封縣
 公,賜物有差,皆賜紫袈娑、銀龜袋。其偽《大雲經》頒於天下寺,各藏一本,令升高座講說。則天將革命,誅殺宗屬諸王,唯千金公主以巧媚善進奉獨存;抗疏請以則天為母,因得曲加恩寵,改邑號為延安大長公主,加實封,賜姓武氏。以子克乂娶魏王武承嗣女,內門參問,不限早晚,見則盡歡。



 長壽二年,默啜復犯塞,又以懷義為代北道行軍大總管,以李多祚、蘇宏暉為將。未行,改朔方道行軍大總管,以內史李昭德為行軍長史,鳳閣侍郎、
 平章事蘇味道為行軍司馬,契苾明、曹仁師、沙吒忠義等十八將軍以討之。未行虜退,乃止。



 懷義後厭入宮中,多居白馬寺,刺血畫大像,選有膂力白丁度為僧,數滿千人。侍御史周矩疑其奸,奏請劾之,不許。固請之,則天曰:「卿且退,朕即令去。」矩至臺,薛師亦至,乘馬蹋階而下,便坦腹於床。矩召臺吏,將按之,遽乘馬而去。矩具以聞,則天曰:「此道人風病,不可苦問。所度僧任卿勘當。」矩按之,窮其狀以聞,諸僧悉配遠州。遷矩天官員外郎,竟為
 薛師所構,下獄,免官。



 後有御醫沈南璆得幸,薛師恩漸衰,恨怒頗甚。證聖中,乃焚明堂、天堂,並為灰燼,則天愧而隱之,又令懷義充使督作。乃於明堂下置九州鼎,鑄銅為十二屬形象,置於本辰位,皆高一丈,懷義率人作號頭安置之。



 其後益驕倨,則天惡之,令太平公主擇膂力婦人數十,密防慮之。人有發其陰謀者,太平公主乳母張夫人令壯士縛而縊殺之,以輦車載尸送白馬寺。其侍者僧徒,皆流竄遠惡處。



 韋溫,中宗韋庶人從父兄也。父玄儼,高宗末官至許州刺史。玄儼弟玄貞,初為普州參軍,以女為皇太子妃,擢拜豫州刺史。中宗嗣位,妃為後。及帝降為廬陵王,玄貞配流欽州而死。後母崔氏,為欽州首領寧承兄弟所殺。



 玄貞有四子:洵、浩、洞、泚,亦死於容州。後二妹,逃竄獲免,間行歸長安。



 及中宗復位,韋氏復為皇后。其日,追贈玄貞為上洛郡王。左拾遺賈虛己上疏諫曰:「孔子曰:『惟名與器,不可以假人。』且非李氏而王,自古盟書所棄。今陛
 下創制謀始,垂範將來,為皇王令圖,子孫明鏡。匡復未幾,後族有私,臣雖庸愚,尚知未可;史官執簡,必是直書。今萬姓顒然,聞一善令,莫不途歌裏頌,延頸向風,欣然慕化,日恐不及。陛下奈何行私惠,使樵夫議之。即先朝贈太原王,殷鑒不遠。同雲生於膚寸,尋木起於蘗栽,誠可惜也。渙汗既行,難改成命,臣望請皇后抗表固辭,使天下知弘讓之風,彤管著沖謙之德,是則巍巍聖鑒,無得而稱。」疏奏不省。



 尋又追贈玄貞為太師、雍州牧、益州
 大都督;玄儼為特進、並州大都督、魯國公。遣使迎玄貞及崔氏喪柩歸京師。又遣廣州都督周仁軌率兵討斬寧承兄弟,以其首祭於崔氏。擢拜仁軌左羽林大將軍,賜爵汝南郡公,食實封五百戶。及玄貞等柩將至,上與後登長樂宮,望喪而泣。加贈玄貞為酆王,謚曰文獻,仍號其廟曰褒德,陵曰榮先。各置官員,並給戶一百人守衛灑掃。又贈玄貞子洵為吏部尚書、汝南郡王,浩太常卿、武陵郡王,洞衛尉卿、淮南郡王,泚太僕卿、上蔡郡王,
 亦遣使迎其喪柩於京師。



 溫,神龍中累遷禮部尚書,封魯國公。弟湑,左羽林將軍,封曹國公。後妹夫陸頌為國子祭酒,馮太和為太常少卿,太和尋卒,又適嗣虢王邕。湑子捷,尚成安公主,溫從祖弟濯,尚定安公主,皆拜駙馬都尉。



 景龍三年,溫遷太子少保、同中書門下三品,仍遙授揚州大都督。溫等既居榮要,燻灼朝野,時人比之武氏。湑及陸頌相次病卒,賻贈甚厚。及中宗崩,後令溫總知內外兵馬,守援宮掖。又引從子播、族弟璇、弟捷、濯
 等,分掌屯營及左右羽林軍。臨淄王討韋氏,溫等皆坐斬,宗族無少長皆死,語在《韋庶人傳》。睿宗即位,仍令削平玄貞及洵等墳墓。



 王仁皎,玄宗王庶人父也。景龍中,官至長上果毅。玄宗即位,以後父,歷將作大匠、太僕卿,遷開府儀同三司,封祁國公。仁皎不預朝政,但厚自奉養,積子女財貨而已。開元七年卒,贈太尉,官供葬事。柩車既發,上於望春亭遙望之,令張說為其碑文,玄宗親書石焉。子守一。



 守一與後雙生。守一與玄宗有舊,及上登極,以清陽公主妻之。從討蕭至忠、岑羲等有功,自尚乘奉御遷殿中少監,特封晉國公,累轉太子少保。父卒,襲爵祁國公。十一年,坐與庶人潛通左道,左遷柳州司馬,行至藍田驛,賜死。守一性貪鄙,積財巨萬,及籍沒其家,財帛不可勝計。



 吳漵,章敬皇后之弟也,濮州濮陽人。祖神泉,位終縣令。父令珪,益州郫縣丞。寶歷二年,代宗始封拜外族,贈神泉司徒,令珪太尉,令珪母弟前宣城令令瑤為開府儀
 同三司、太子家令,封濮陽郡公;中郎將令瑜為開府儀同三司、太子諭德、濟陽郡公。漵時為盛王府錄事參軍,拜開府儀同三司、太子詹事、濮陽郡公。以元舅遷鴻臚少卿、金吾將軍。建中初,遷大將軍。漵雖居戚屬,恭遜謙和,人皆重之。



 涇師之亂,從幸奉天,盧杞、白志貞謂德宗曰:「臣細觀硃泚心跡,必不至為戎首,佇當效順。宜擇大臣一人,入京師慰諭,以觀其心。」上召從幸群臣言之,皆憚其行。漵起奏曰:「不以臣才望無堪,臣願北行。」德宗甚
 悅。漵退而謂人曰:「人臣食君之祿,死君之難,臨危自計,非忠也。吾忝戚屬,今日委身於賊,誠知必死,不欲聖情慊於無人犯難也。」即日齎詔見泚,深陳上待屬之意。時泚逆謀已定,貌雖從命,而心已異,乃留漵於客省,竟被害。上聞之,悲悼不已,贈太子太傅,賜其家實封二百戶,一子五品正員官,敕收城日葬事官給。弟湊。



 湊,寶歷中與兄漵同日開府,授太子詹事,俱封濮陽郡公。湊以兄弟三品,固辭太過,乞授卑官。乃以湊檢校太子賓客,兼
 太子家令,充十宅王使。累轉左金吾衛大將軍。



 湊小心謹慎,智識周敏,特承顧問,偏見委信。大歷中,滑帥令狐彰、汴帥田神功相次歿於理所,時籓方兵驕,乘戎帥喪亡,人情多梗。代宗命湊銜命撫慰,至必委曲說諭,隨所欲為之奏請,皆得軍民和協,帝深重之。



 宰臣元載弄權,招致賄賂,醜跡日彰。帝惡之,將加之法,恐左右洩漏,無與言者,唯與湊密計圖之。及收載於內侍省,同列王縉,其黨楊炎、王昂、韓洄、包佶、韓會等,皆當從坐籍沒。湊諫
 救百端,言「法宜從寬,縉等從坐,理不至死。若不降以等差,一例極刑,恐虧損聖德。」由是縉等得減死,流貶之。



 大歷末,丁繼母喪免。建中初,起為右衛將軍,兼通州刺史。貞元初,入為太子賓客,出為福州刺史、御史中丞、福建觀察使。為政勤儉清苦,美譽日聞。宰相竇參以私怨惡之,數加譖毀,又言湊風病,不任趨馳。德宗召湊至京師,對於別殿,上令殿上行走,以驗其病否,由是悟參之誣,因是惡參。尋以湊為陜州大都督府長史、陜虢觀察使,
 以代參之黨李翼。會劉玄佐卒,以湊檢校兵部尚書、汴州刺史、御史大夫、宣武軍節度使。



 時汴州軍亂,殺牙將曹金岸、縣令李邁,謀立玄佐子士寧。上將遣兵送湊赴鎮,召宰臣議。竇參深沮其行,恐軍中拒命,乃召湊回,授右金吾衛大將軍,而以梁宋節鉞授士寧。



 貞元十四年春夏旱,穀貴,人多流亡。京兆尹韓皋以政事不理黜官。上召湊,面授京兆尹,即日令視事,經宿方下制。湊孜孜為理,以勤儉為務,人樂其政。時宮中選內官買物於市,
 倚勢強買物,不充價,人畏而避之,呼為「宮市」。掌賦者多與中貴人交結假借,不言其弊。湊為京尹,便殿從容論之,曰:「物議以中人買物於市,稍不便於人,此事甚細,虛掇流議。凡宮中所須,責臣可辦,不必更差中使。若以臣府縣外吏,不合預聞宮中所須,則乞選內官年高謹重者,充宮市令,庶息人間論議。」又奏:「掌閑彍騎、飛龍內園、芙蓉及禁軍諸司等使,雜供手力資課太多,量宜減省。」上多從之。



 初,府掾吏以湊起自戚籓,不諳簿領,凡有疑
 獄難決之事,多候湊將出時方呈,冀免指擿瑕病。湊雖倉卒閱視,必指其奸幸之處,下筆決斷,無毫厘之差。掾吏非大過,不行笞責;而召面按問,詰責而釋之。吏尤惕厲,庶務咸舉。



 文敬太子、義章公主相繼薨歿,上深追念,葬送之儀頗厚。召集工役,載土築墳,妨民農務。湊候上顧問,極言之。宗屬門吏以湊論諫太繁,恐上厭苦,每以簡約規之。湊曰:「聖上明哲,憂勞四海,必不以公主、太子之鐘念而忽疲民。但人多順旨不言,若再三啟諫,必動
 宸情,則生民受賜。長吏不言,是為阿旨。如窮民上訴,罪在何人?」議者重之。以能政,兼兵部尚書。官街樹缺,所司植榆以補之,湊曰:「榆非九衢之玩。」亟命易之以槐。及槐陰成而湊卒,人指樹而懷之。



 湊於德宗為老舅,漢魏故事,多退居散地,才免罪戾而已。湊自貞元已來,特承恩顧,歷中外顯貴,雖聖獎隆深,亦由湊小心辦事,奉職有方故也。



 湊既疾,不召巫醫,藥不入口,家人泣而勉之。對曰:「吾以凡才,濫因外戚進用,起家便授三品,歷顯位四
 十年,壽登七十,為人足矣,更欲何求?古之以親戚進用者,罕有善終,吾得歸全以侍先人,幸也。」德宗知之,令御醫進藥,不獲已,服之。貞元十六年四月卒,時年七十一,贈尚書左僕射,罷朝一日。



 竇覦,昭成皇后族侄。父光,華原尉。覦以親廕,釋褐右衛率府兵曹參軍。鄜坊節度臧希讓奏為判官,累授監察殿中侍御史、檢校工部員外郎、坊州刺史。興元元年,討李懷光於河中,詔覦以坊州兵七百人屯邰陽。賊平,以
 功兼御史中丞。遷同州刺史,入朝為戶部侍郎。



 覦無他才伎,為吏有計數,又以韓滉子婿,故籓府闢召,遂歷牧守。宰相竇參,覦再從侄。參少依覦,及參秉政,力薦於朝,故有貳卿之拜。數月,為揚州大都督府長史、御史大夫、充淮南節度副大使、知節度事,既非德舉,人咸薄之。赴鎮旬日,暴卒,詔贈禮部尚書。



 柳晟者,肅宗皇后之甥。母和政公主。父潭,官至太僕卿、駙馬都尉。晟少無檢操,代宗於諸甥之中,特加撫鞠,俾
 與太子、諸王同學,授詩書,恩寵罕比。累試太常卿。



 德宗即位,以與晟幼同硯席,尤親之。涇師之亂,從幸奉天,晟密啟曰:「願受詔入京城,游說群賊,冀其攜貳。」德宗壯而許之。晟與賊帥多有舊,出入其門說誘之。事洩,為硃泚所擒,械之於獄。晟有力,乃於獄中穿垣破械而遁,落發為僧,間道歸行在。遷將作少監。元和初,檢校工部尚書、興元尹、山南西道節度使。罷鎮入朝,以違詔進奉,為御史元稹所劾,詔宥之。俄充入回鶻冊立使,復命,遷左金
 吾衛大將軍。元和十三年卒,贈太子少保。



 王子顏,瑯邪臨沂人,莊憲皇后之父也。祖思敬,少從軍,累試太子賓客。父難得,有勇決,善騎射,天寶初為河源軍使。吐蕃贊普王子郎支都有勇,乘諳真馬,寶鈿裝鞍,出陣求斗,無敢與校者。難得挾槍奮馬突前,刺殺郎支都,斬其首,傳於京師。軍還,玄宗召見之,令於殿前乘馬挾槍作刺郎支都之狀。賜以錦袍金帶,累拜金吾將軍同正員。



 天寶七載,從哥舒翰擊吐蕃於積石,虜吐谷渾
 王子悉弄參及子婿悉頰藏而還,累拜左武衛將軍、關西游奕使。九載,擊吐蕃,收五橋,拔樹敦城,補白水軍使。十三載,從收九曲,加特進。



 祿山之叛,從哥舒翰戰於潼關;關門不守,從肅宗幸靈武。時行在闕軍賞,難得進絹三千疋及金銀器等。至德初,試衛尉卿、興平軍使,兼鳳翔都知兵馬使。進收京城,與賊軍戰。其下靳元曜戰酣墮馬,難得馳救之,賊射之中眉,皮穿披下鄣目。難得自拔去箭,並皮掣洛,馳馬復戰,血流被面,而抗賊不已。肅宗深嘉之。從郭子儀攻安慶緒於相州,累封瑯邪郡公,英武軍使。寶應二年卒,贈潞州大都督。



 子顏少從父征役,累官金紫光祿大夫、檢校衛尉卿,生後而卒。順宗內禪,以後生憲宗皇帝,褒贈先代:思敬司徒,難得太傅,子顏太師。



 顏子重榮,官至福王傅;用,官至太子賓客、金吾將軍。



 贊曰:戚里之賢,避寵畏權。不恤禍患,鮮能保全。福盈者敗,勢壓者顛。武之惟良,明於自然。



\end{pinyinscope}