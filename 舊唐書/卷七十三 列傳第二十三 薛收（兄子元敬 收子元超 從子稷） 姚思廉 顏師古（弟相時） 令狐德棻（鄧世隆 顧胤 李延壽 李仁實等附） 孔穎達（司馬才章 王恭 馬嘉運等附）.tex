\article{卷七十三 列傳第二十三 薛收(兄子元敬 收子元超 從子稷) 姚思廉 顏師古(弟相時) 令狐德棻(鄧世隆 顧胤 李延壽 李仁實等附) 孔穎達(司馬才章 王恭 馬嘉運等附)}

\begin{pinyinscope}

 ○薛收
 兄子元敬收子元超從子稷姚思廉顏師古弟相時令狐德棻鄧世隆顧胤李延壽李仁實等附孔穎達司馬才章王恭馬嘉運等附



 薛收,字伯褒,蒲州汾陰人,隋內史侍郎道衡子也。事繼
 從父孺以孝聞。年十二,解屬文。以父在隋非命,乃潔志不仕。大業末,郡舉秀才,固辭不應。義旗起,遁於首陽山,將協義舉。蒲州通守堯君素潛知收謀,乃遣人迎收所生母王氏置城內,收乃還城。後君素將應王世充,收遂逾城歸國。秦府記室房玄齡薦之於太宗,即日召見,問以經略,收辯對縱橫,皆合旨要。授秦府主簿,判陜東道大行臺金部郎中。時太宗專任征伐,檄書露布,多出於收。言辭敏速,還同宿構,馬上即成,曾無點竄。太宗討王
 世充也,竇建德率兵來拒,諸將皆以為宜且退軍,以觀賊形勢。收獨建策曰:「世充據有東都,府庫填積,其兵皆是江淮精銳,所患者在於乏食,是以為我所持,求戰不可。建德親總軍旅,來拒我師,亦當盡彼驍雄,期於奮決。若縱其至此,兩寇相連,轉河北之糧以相資給,則伊、洛之間戰鬥不已。今宜分兵守營,深其溝防,即世充欲戰,慎勿出兵。大王親率猛銳,先據成皋之險,訓兵坐甲,以待其至。彼以疲弊之師,當我堂堂之勢,一戰必克。建德
 即破,世充自下矣。不過兩旬,二國之君,可面縛麾下。若退兵自守,計之下也。」太宗納之,卒擒建德。東都平,太宗入觀隋氏宮室,嗟後主罄人力以逞奢侈。收進曰:「竊聞峻宇雕墻,殷辛以滅;土階茅棟,唐堯以昌。秦帝增阿房之飾,漢後罷露臺之費,故漢祚延而秦禍速,自古如此。後主曾不能察,以萬乘之尊,困一夫之手,使土崩瓦解,取譏後代,以奢虐所致也。」太宗悅其對。及軍還,授天策府記室參軍。太宗初授天策上將、尚書令,命收與世南
 並作第一讓表,竟用收者。太宗曾侍高祖游後園中,獲白魚,命收為獻表,收援筆立就,不復停思,時人推其二表贍而速。從平劉黑闥,封汾陰縣男。武德六年,以本官兼文學館學士,與房玄齡、杜如晦特蒙殊禮,受心腹之寄。又嘗上書諫獵,太宗手詔曰:「覽讀所陳,實悟心膽,今日成我,卿之力也。明珠兼乘,豈比來言,當以誡心,書何能盡!今賜卿黃金四十鋌,以酬雅意。」七年,寢疾,太宗遣使臨問,相望於道。尋命輿疾詣府,太宗親以衣袂撫收,
 論敘生平,潸然流涕。尋卒,年三十三。太宗親自臨哭。哀慟左右。與收從父兄子元敬書曰:「吾與卿叔共事,或軍旅多務,或文詠從容。何嘗不驅馳經略,款曲襟抱?比雖疾苦,日冀痊除,何期一朝,忽成萬古!追尋痛惋,彌用傷懷。且聞其兒子幼小,家徒壁立,未知何處安置?宜加安撫,以慰吾懷。」因使人吊祭,贈物三百段。及後,遍圖學士等形像。太宗嘆曰:「薛收遂成故人,恨不早圖其像。」及登極,顧謂房玄齡曰:「薛收若在,朕當以中書令處之。」又嘗
 夢收如平生,又敕有司特賜其家粟帛。貞觀七年,贈定州刺史。永徽六年,又贈太常卿,陪葬昭陵。文集十卷。



 元敬,隋選部侍郎邁子也。有文學,少與收及收族兄德音齊名,時人謂之「河東三鳳」。收為長雛,德音為鸑鷟,元敬以年最小為鵷雛。武德中,元敬為秘書郎,太宗召為天策府參軍,兼直記室。收與元敬俱為文學館學士。時房、杜等處心腹之寄,深相友托,元敬畏於權勢,竟不之狎,如晦常云:「小記室不可得而親,不可得而疏。」太宗入東
 宮,除太子舍人。時軍國之務,總於東宮,元敬專掌文翰,號為稱職。尋卒。



 收子元超。元超早孤,九歲襲爵汾陰男。及長,好學,善屬文。太宗甚重之,令尚巢剌王女和靜縣主,累授太子舍人,預撰《晉書》。高宗即位,擢拜給事中,時年二十六。數上書陳君臣政體及時事得失,高宗皆嘉納之。俄轉中書舍人,加弘文館學士,兼修國史。中書省有一盤石,初,道衡為內史侍郎,嘗踞而草制,元超每見此石,未嘗不泫然流涕。永徽五年,丁母憂解。明年,起授
 黃門侍郎,兼檢校太子左庶子。元超既擅文辭,兼好引寒俊,嘗表薦任希古、高智周、郭正一、王義方、孟利貞等十餘人,由是時論稱美。後以疾出為饒州刺史。三年,拜東臺侍郎。右相李義府以罪配流巂州,舊制,流人禁乘馬,元超奏請給之,坐貶為簡州刺史。歲餘,西臺侍郎上官儀伏誅,又坐與文章款密,配流巂州。上元初,遇赦還,拜正諫大夫。三年,遷中書侍郎,尋同中書門下三品。時高宗幸溫泉校獵,諸蕃酋長亦持弓矢而從。元超以為
 既非族類,深可為虞,上疏切諫,帝納焉。時元超特承恩遇,常召入與諸王同預私宴。又重其文學政理之才,曾謂元超曰:「長得卿在中書,固不藉多人也。」永隆二年,拜中書令,兼太子左庶子。高宗幸東都,太子於京師監國,因留元超以侍太子。帝臨行謂元超曰:「朕之留卿,如去一臂。但吾子未閑庶務,關西之事,悉以委卿。所寄既深,不得默爾。」於是元超表薦鄭祖玄、鄧玄挺、崔融為崇文館學士。又數上疏諫太子,高宗知而稱善,遣使慰諭,賜
 物百段。弘道元年,以疾乞骸,加金紫光祿大夫,聽致仕。其年冬卒,年六十二。贈光祿大夫、秦州都督,陪葬乾陵。文集四十卷。子曜,亦以文學知名,聖歷中,修《三教珠英》,官至正諫大夫。元超從子稷。



 稷舉進士,累轉中書舍人。時從祖兄曜為正諫大夫,與稷俱以辭學知名,同在兩省,為時所稱。景龍末,為諫議大夫、昭文館學士。好古博雅,尤工隸書。自貞觀、永徽之際,虞世南、褚遂良時人宗其書跡,自後罕能繼者。稷外祖魏徵家富圖籍,多有虞、
 褚舊跡,稷銳精模仿,筆態遒麗,當時無及之者。又善畫,博探古跡。睿宗在籓,留意於小學,稷於是特見招引,俄又令其子伯陽尚仙源公主。及踐祚,累拜中書侍郎,與蘇頲等對掌制誥。俄與中書侍郎崔日用參知政事。睿宗以鐘紹京為中書令,稷勸令禮讓,因入言於帝曰:「紹京素無才望,出自胥吏,雖有功勛,未聞令德。一朝超居元宰,師長百僚,臣恐清濁同貫,失於聖朝具瞻之美。」帝然其言,因紹京表讓,遂轉為戶部尚書。稷又於帝前面
 折崔日用,遞相短長,由是罷知政事,遷左散騎常侍,歷工部、禮部二尚書。以翊贊睿宗功封晉國公,賜實封三百戶,除太子少保。睿宗常召稷入宮中參決庶政,恩遇莫與為比。及竇懷貞伏誅,稷以知其謀,賜死於萬年縣獄中。子伯陽,以尚公主拜右千牛衛將軍、駙馬都尉,亦以功封安邑郡公,別食實封四百戶。及父死,特免坐,左遷晉州員外別駕。尋而配徙嶺表,在道自殺。伯陽子談,開元十六年,尚常山公主,拜駙馬都尉、光祿員外卿,旬
 日暴卒。



 姚思廉,字簡之,雍州萬年人。父察,陳吏部尚書;入隋,歷太子內舍人、秘書丞、北絳公,學兼儒史,見重於三代。陳亡,察自吳興始遷關中。思廉少受漢史於其父,能盡傳家業,勤學寡欲,未嘗言及家人產業。在陳為揚州主簿,入隋為漢王府參軍,丁父憂解職。初,察在陳嘗修梁、陳二史,未就,臨終令思廉續成其志。丁繼母憂,廬於墓側,毀瘠加人。服闋,補河間郡司法書佐。思廉上表陳父遺
 言,有詔許其續成《梁》、《陳史》。煬帝又令與起居舍人崔祖浚修《區宇圖志》。後為代王侑侍讀。會義師克京城,侑府僚奔駭,唯思廉侍王,不離其側。兵將升殿,思廉厲聲謂曰:「唐公舉義,本匡王室,卿等不宜無禮於王。」眾服其言,於是布列階下。高祖聞而義之,許其扶侑至順陽閣下,泣拜而去。觀者咸嘆曰:「忠烈之士也。仁者有勇,此之謂乎!」高祖受禪,授秦王文學。後太宗征徐圓朗,思廉時在洛陽,太宗嘗從容言及隋亡之事,慨然嘆曰:「姚思廉不
 懼兵刃,以明大節,求諸古人,亦何以加也!」因寄物三百段以遺之,書曰:「想節義之風,故有斯贈。」尋引為文學館學士。太宗入春宮,遷太子洗馬。貞觀初,遷著作郎、弘文館學士。寫其形像,列於《十八學士圖》,令文學褚亮為之贊,曰:「志苦精勤,紀言實錄。臨危殉義,餘風勵俗。」三年,又受詔與秘書監魏徵同撰梁、陳二史。思廉又採謝炅等諸家梁史續成父書,並推究陳事,刪益博綜顧野王所修舊史,撰成《梁書》五十卷、《陳書》三十卷。魏徵雖裁其總
 論,其編次筆削,皆思廉之功也,賜彩絹五百段,加通直散騎常侍。思廉以籓邸之舊,深被禮遇,政有得失,常遣密奏之,思廉亦直言無隱。太宗將幸九成宮,思廉諫曰:「離宮游幸,秦皇、漢武之事,固非堯、舜、禹、湯之所為也。」言甚切至。太宗諭曰:「朕有氣疾,熱便頓劇,固非情好游賞也。」因賜帛五十匹。九年,拜散騎常侍,賜爵豐城縣男。十一年卒。太宗深悼惜之,廢朝一日,贈太常卿,謚曰康,賜葬地於昭陵。子處平,官至通事舍人。處平子璹、珽,別有
 傳。



 顏籀,字師古,雍州萬年人,齊黃門侍郎之推孫也。其先本居瑯邪,世仕江左。及之推,歷事周、齊,齊滅,始居關中。父思魯,以學藝稱,武德初為秦王府記室參軍。師古少傳家業,博覽群書,尤精詁訓,善屬文。隋仁壽中,為尚書左丞李綱所薦,授安養尉。尚書左僕射楊素見師古年弱貌羸,因謂曰:「安養劇縣,何以克當?」師古曰:「割雞焉用牛刀。」素奇其對。到官果以干理聞。時薛道衡為襄州總
 管,與高祖有舊,又悅其才,有所綴文,嘗使其掎摭疵病,甚親暱之。尋坐事免,歸長安,十年不得調,家貧,以教授為業。



 及起義,師古至長春宮謁見,授朝散大夫。從平京城,拜敦煌公府文學,轉起居舍人,再遷中書舍人,專掌機密。於時軍國多務,凡有制誥,皆成其手。師古達於政理,冊奏之工,時無及者。太宗踐祚,擢拜中書侍郎,封瑯邪縣男。以母憂去職。服闋,復為中書侍郎。歲餘,坐事免。太宗以經籍去聖久遠,文字訛謬,令師古於秘書省考
 定《五經》,師古多所厘正,既成,奏之。太宗復遣諸儒重加詳議,於時諸儒傳習已久,皆共非之。師古輒引晉、宋已來古今本,隨言曉答,援據詳明,皆出其意表,諸儒莫不嘆服。於是兼通直郎、散騎常侍,頒其所定之書於天下,令學者習焉。貞觀七年,拜秘書少監,專典刊正。所有奇書難字,眾所共惑者,隨疑剖析,曲盡其源。是時多引後進之士為讎校,師古抑素流,先貴勢,雖富商大賈亦引進之,物論稱其納賄,由是出為郴州刺史。未行,太宗惜
 其才,謂之曰「卿之學識,良有可稱,但事親居官,未為清論所許。今之此授,卿自取之。朕以卿曩日任使,不忍遐棄,宜深自誡勵也。」於是復以為秘書少監。師古既負其才,又早見驅策,累被任用,及頻有罪譴,意甚喪沮。自是闔門守靜,杜絕賓客,放志園亭,葛巾野服。然搜求古跡及古器,耽好不已。俄又奉詔與博士等撰定《五禮》,十一年,《禮》成,進爵為子。時承乾在東宮,命師古注班固《漢書》,解釋詳明,深為學者所重。承乾表上之,太宗令編之秘
 閣,賜師古物二百段、良馬一匹。十五年,太宗下詔,將有事於泰山,所司與公卿並諸儒博士詳定儀注。太常卿韋挺、禮部侍郎令狐德棻為封禪使,參考其儀,時論者競起異端。師古奏曰:「臣撰定《封禪儀注書》在十一年春,於時諸儒參詳,以為適中。」於是詔公卿定其可否,多從師古之說,然而事竟不行。師古俄遷秘書監、弘文館學士。十九年,從駕東巡,道病卒,年六十五,謚曰戴。有集六十卷。其所注《漢書》及《急就章》,大行於世。永徽三年,師古
 子揚庭為符璽郎,又表上師古所撰《匡謬正俗》八卷。高宗下詔付秘書閣,仍賜揚庭帛五十匹。



 師古弟相時,亦有學業。武德中,與房玄齡等為秦府學士。貞觀中,累遷諫議大夫,拾遺補闕,有諍臣之風。尋轉禮部侍郎。相時羸瘠多疾病,太宗常使賜以醫藥。性仁友,及師古卒,不勝哀慕而卒。師古叔父游秦,武德初累遷廉州刺史,封臨沂縣男。時劉黑闥初平,人多以強暴寡禮,風俗未安,游秦撫恤境內,敬讓大行。邑里歌曰:「廉州顏有道,性行
 同莊、老。愛人如赤子,不殺非時草。」高祖璽書勞勉之。俄拜鄆州刺史,卒於官。撰《漢書決疑》十二卷,為學者所稱,後師古注《漢書》,亦多取其義耳。



 令狐德棻,宜州華原人,隋鴻臚少卿熙之子也。先居燉煌,代為河西右族。德棻博涉文史,早知名。大業末,為藥城長,以世亂不就職。及義旗建,淮安王神通據太平宮,自稱總管,以德棻為記室參軍。高祖入關,引直大丞相府記室。武德元年,轉起居舍人,甚見親待。五年,遷秘書
 丞,與侍中陳叔達等受詔撰《藝文類聚》。高祖問德棻曰:「比者,丈夫冠、婦人髻,競為高大,何也?」對曰:「在人之身,冠為上飾,所以古人方諸君上。昔東晉之末,君弱臣強,江左士女,皆衣小而裳大。及宋武正位之後,君德尊嚴,衣服之制,俄亦變改。此即近事之徵。」高祖然之。時承喪亂之餘,經籍亡逸,德棻奏請購募遺書。重加錢帛,增置楷書,令繕寫。數年間,群書略備。德棻嘗從容言於高祖曰:「竊見近代已來,多無正史,梁、陳及齊,猶有文籍。至周、隋
 遭大業離亂,多有遺闕。當今耳目猶接,尚有可憑,如更十數年後,恐事跡湮沒。陛上既受禪於隋,復承周氏歷數,國家二祖功業,並在周時。如文史不存,何以貽鑒今古?如臣愚見,並請修之。」高祖然其奏,下詔曰:



 司典序言,史官記事,考論得失,究盡變通。所以裁成義類,懲惡勸善,多識前古,貽鑒將來。伏羲以降,周、秦斯及,兩漢傳緒,三國受命,迄於晉、宋,載籍備焉。自有魏南徙,乘機撫運,周、隋禪代,歷世相仍。梁氏稱邦,跨據淮海;齊遷龜鼎,陳
 建皇宗,莫不自命正朔,綿歷歲祀,各殊徽號,刪定禮儀。至於發跡開基,受終告代,嘉謀善政,名臣奇士,立言著績,無乏於時。然而簡牘未編,紀傳咸闕,炎涼已積,謠俗遷訛。餘烈遺風,倏焉將墜。朕握圖馭宇,長世字人,方立典謨,永垂憲則。顧彼湮落,用深軫悼,有懷撰次,實資良直。中書令蕭瑀、給事中王敬業、著作郎殷聞禮可修魏史,侍中陳叔達、秘書丞令狐德棻、太史令庾儉可修周史,兼中書令封德彞、中書舍人顏師古可修隋史,大理
 卿崔善為、中書舍人孔紹安、太子洗馬蕭德言可修梁史,太子詹事裴矩、兼吏部郎中祖孝孫、前秘書丞魏徵可修齊史,秘書監竇璡、給事中歐陽詢、秦王文學姚思廉可修陳史。務加詳核,博採舊聞,義在不刊,書法無隱。



 瑀等受詔,歷數年,竟不能就而罷。貞觀三年,太宗復敕修撰,乃令德棻與秘書郎岑文本修周史,中書舍人李百藥修齊史,著作郎姚思廉修梁、陳史,秘書監魏徵修隋史,與尚書左僕射房玄齡總監諸代史。眾議以魏史
 既有魏收、魏彥二家,已為詳備,遂不復修。德棻又奏引殿中侍御史崔仁師佐修周史,德棻仍總知類會梁、陳、齊、隋諸史。武德已來創修撰之源,自德棻始也。六年,累遷禮部侍郎,兼修國史,賜爵彭陽男。十年,以修周史賜絹四百匹。十一年,修《新禮》成,進爵為子。又以撰《氏族志》成,賜帛二百匹。十五年,轉太子右庶子。承乾敗,隨例除名。十八年,起為雅州刺史,以公事免。尋有詔改撰《晉書》,房玄齡奏德棻令預修撰,當時同修一十八人,並推德
 棻為首,其體制多取決焉。書成,除秘書少監。



 永徽元年,又受詔撰定律令,復為禮部侍郎,兼弘文館學士,監修國史及《五代史志》。尋遷太常卿,兼弘文館學士。時高宗初嗣位,留心政道,嘗召宰臣及弘文館學士於中華殿而問曰:「何者為王道;霸道?又孰為先後?」德棻對曰:「王道任德,霸道任刑。自三王已上,皆行王道;唯秦任霸術,漢則雜而行之;魏、晉已下,王、霸俱失。如欲用之,王道為最,而行之為難。」高宗曰:「今之所行,何政為要?」德棻對曰:「古
 者為政,清其心,簡其事,以此為本。當今天下無虞,年穀豐稔,薄賦斂,少徵役,此乃合於古道。為政之要道,莫過於此。」高宗曰:「政道莫尚於無為也。」又問曰:「禹、湯何以興?桀、紂何以亡?」德棻對曰:「《傳》稱:『禹、湯罪己,其興也勃焉;桀、紂罪人,其亡也忽焉。』二主惑於妹喜、妲己,誅戮諫者,造砲烙之刑,是其所以亡也。」高宗甚悅,既罷,各賜以繒彩。四年,遷國子祭酒,以修貞觀十三年以後實錄功,賜物四百段,兼授崇賢館學士。尋又撰《高宗實錄》三十卷,進
 爵為公。龍朔二年,表請致仕,許之,仍加金紫光祿大夫。乾封元年,卒於家,年八十四,謚曰憲。德棻暮年尤勤於著述,國家凡有修撰,無不參預。



 自武德已後,有鄧世隆、顧胤、李延壽、李仁實前後修撰國史,頗為當時所稱。



 鄧世隆者,相州人也。大業末,王世充兄子太,守河陽,引世隆為賓客,大見親遇。及太宗攻洛陽,遣書諭太,世隆為復書,言辭不遜。洛陽平後,世隆懼罪,變姓名,自號隱玄先生,竄於白鹿山。貞觀初,徵授國子主簿,與崔仁師、
 慕容善行、劉顗、庾安禮、敬播等俱為修史學士。世隆負宿罪,猶不自安。太宗聞之,遣房玄齡諭之曰:「爾為王太作書,誠合重罪,但各為其主,於朕豈有惡哉?朕今為天子,何能追責匹夫之過?爾宜坦然,勿懷危懼也。」擢授著作佐郎,歷衛尉丞。初,太宗以武功定海內,櫛風沐雨,不暇於詩書。暨於嗣業,進引忠良,銳精思政。數年之後,道致隆平,遂於聽覽之暇,留情文史。敘事言懷,時有構屬,天才宏麗,興托玄遠。貞觀十三年,世隆上疏請編錄御
 集,太宗竟不許之。世隆又採隋代舊事,撰為《東都記》三十卷。遷著作郎。尋卒。



 顧胤者,蘇州吳人也。祖越,陳給事黃門侍郎。父覽,隋秘書學士。胤,永徽中歷遷起居郎,兼修國史。撰《太宗實錄》二十卷成,以功加朝散大夫,授弘文館學士。以撰武德、貞觀兩朝國史八十卷成,加朝請大夫,封餘杭縣男,賜帛五百段。龍朔三年,遷司文郎中。尋卒。胤又撰《漢書古今集》二十卷,行於代。子琮,長安中為天官侍郎、同鳳閣
 鸞臺平章事。



 李延壽者,本隴西著姓,世居相州。貞觀中,累補太子典膳丞、崇賢館學士,嘗受詔與著作佐郎敬播同修《五代史志》,又預撰《晉書》,尋轉御史臺主簿,兼直國史。延壽嘗撰《太宗政典》三十卷表上之。歷遷符璽郎,兼修國史,尋卒。調露中,高宗嘗觀其所撰《政典》,嘆美久之,令藏於秘閣,賜其家帛五十段。延壽又嘗刪補宋、齊、梁、陳及魏、齊、周、隋等八代史,謂之《南北史》,凡一百八十卷,頗行於代。



 李仁實,魏州頓丘人。官至左史。嘗著《格論》三卷、《通歷》八卷、《戎州記》,並行於時。



 孔穎達,字仲達,冀州衡水人也。祖碩,後魏南臺丞。父安,齊青州法曹參軍。穎達八歲就學,日誦千餘言。及長,尤明《左氏傳》、《鄭氏尚書》、《王氏易》、《毛詩》、《禮記》,兼善算歷,解屬文。同郡劉焯名重海內,穎達造其門。焯初不之禮,穎達請質疑滯,多出其意表,焯改容敬之。穎達固辭歸,焯固留不可。還家,以教授為務。隋大業初,舉明經高第,授河
 內郡博士。時煬帝徵諸郡儒官集於東都,令國子秘書學士與之論難,穎達為最。時穎達少年,而先輩宿儒恥為之屈,潛遣刺客圖之。禮部尚書楊玄感舍之於家,由是獲免。補太學助教。屬隋亂,避地於武牢。太宗平王世充,引為秦府文學館學士。武德九年,擢授國子博士。貞觀初,封曲阜縣男,轉給事中。時太宗初即位,留心庶政,穎達數進忠言,益見親待。太宗嘗問曰:「《論語》云:『以能問於不能,以多問於寡,有若無,實若虛。』何謂也?」穎達對曰:「
 聖人設教,欲人謙光。己雖有能,不自矜大,仍就不能之人求訪能事。己之才藝雖多,猶以為少,仍就寡少之人更求所益。己之雖有,其狀若無。己之雖實,其容若虛。非唯匹庶,帝王之德,亦當如此。夫帝王內蘊神明,外須玄默,使深不可測,度不可知。《易》稱『以蒙養正,以明夷蒞眾』,若其位居尊極,炫耀聰明,以才凌人,飾非拒諫,則上下情隔,君臣道乖。自古滅亡,莫不由此也。」太宗深善其對。六年,累除國子司業。歲餘,遷太子右庶子,仍兼國子司
 業。與諸儒議歷及明堂,皆從穎達之說。又與魏徵撰成《隋史》,加位散騎常侍。十一年,又與朝賢修定《五禮》,所有疑滯,咸諮決之。書成,進爵為子,賜物三百段。庶人承乾令撰《孝經義疏》,穎達因文見意,更廣規諷之道,學者稱之。太宗以穎達在東宮數有匡諫,與左庶子於志寧各賜黃金一斤、絹百匹。十二年,拜國子祭酒,仍侍講東宮。十四年,太宗幸國學觀釋奠,命穎達講《孝經》,既畢,穎達上《釋奠頌》,手詔褒美。後承乾不循法度,穎達每犯顏進
 諫。承乾乳母遂安夫人謂曰:「太子成長,何宜屢致面折?」穎達對曰:「蒙國厚恩,死無所恨。」諫諍逾切,承乾不能納。先是,與顏師古、司馬才章、王恭、王琰等諸儒受詔撰定《五經》義訓,凡一百八十卷,名曰《五經正義》。太宗下詔曰:「卿等博綜古今,義理該洽,考前儒之異說,符聖人之幽旨,實為不朽。」付國子監施行,賜穎達物三百段。時又有太學博士馬嘉運駁穎達所撰《正義》,詔更令詳定,功竟未就。十七年,以年老致仕。十八年,圖形於凌煙閣,贊曰:「
 道光列第,風傳闕里。精義霞開,掞辭飆起。」二十二年卒,陪葬昭陵,贈太常卿,謚曰憲。



 司馬才章者,魏州貴鄉人也。父烜,博涉《五經》,善緯候。才章少傳其業。隋末為郡博士,貞觀六年,左僕射房玄齡薦之,屢蒙召問,擢授國子助教,論議該洽,學者稱之。



 王恭者,滑州白馬人也。少篤學,博涉《六經》。每於鄉閭教授,弟子自遠方至數百人。貞觀初,徵拜太學博士,其所講《三禮》,皆別立義證,甚為精博。蓋文懿、文達等皆當時
 大儒,罕所推借,每講《三禮》,皆遍舉先達義,而亦暢恭所說。



 馬嘉運者,魏州繁水人也。少出家為沙門,明於《三論》。後更還俗,專精儒業,尤善論難。貞觀初,累除越王東閣祭酒。頃之,罷歸,隱居白鹿山。十一年,召拜太學博士,兼弘文館學士,預修《文思博要》。嘉運以穎達所撰《正義》頗多繁雜,每掎摭之,諸儒亦稱為允當。高宗居春宮,引為崇賢館學士。數與洗馬秦暐侍講殿中,甚蒙禮異。十九年,
 遷國子博士卒。



 史臣曰:唐德勃興,英儒間出,佐命協力,實有其人。薛收左右厥猷,經謀雅道,不幸短命,殲我良士。上言「恨不圖形,若在,當以中書令處之」,才可知矣。元敬藻翰明敏,而畏權勢,竟不狎房、杜,深沉至慎,不亦優哉!元超藉父風望,弼亮宏略,諒非其罪,而再遷流。及登大任,益有嘉謀,汲引多才,以隆弘納,其感恩之重,時其聞諸?有始有卒,其殆庶幾乎!稷出自名家,涉於大用,及自貽謀釁,如貞
 亮何?姚思廉篤學寡欲,受漢史於家尊,果執明義,臨大節而不可奪。及筆削成書,箴規翊聖,言其命世,亦當仁乎!師古家籍儒風,該博經義,至於詳注史策,探測典禮,清明在躬,天有才格。然而三黜之負,竟在時譏,孔子曰「才難」,不其然乎?令狐德棻貞度應時,待問平直。征舊史,修新禮,以暢國風;辨治亂,談王霸,以資帝業。「元首明哉,股肱良哉」,其斯之謂歟!鄧世隆國史時譽,固有諒直。其復書不遜,何不知之甚也!上疏請編御集,其弼直乎!顧
 胤清芬,可觀彞範,積善餘慶,其有子哉!李延壽研考史學,修撰刪補,克成大典,方之班、馬,何代無人?仁實據摭,抑又次焉。孔穎達風格高爽,幼而有聞,探賾明敏,辨析應對,天有通才。人道惡盈,必有毀訐,及《正義》炳煥,乃異人也,雖其掎摭,亦何損於明?司馬才章藉時崇儒,明核致業;王恭弘闡聲教,禮學研詳;馬嘉運達識自通,克成典雅。並符才用,潤色丹青,其掎摭繁雜,蓋求備者也。



 贊曰:河東三鳳,俱瑞黃圖。棻為良史,穎實名儒。解經不
 窮,希顏之徒。登瀛入館,不其盛乎!



\end{pinyinscope}