\article{卷二十七 志第七 禮儀七}

\begin{pinyinscope}

 貞觀十四年,太宗因修禮官奏事之次,言及喪服,太宗曰:「同爨尚有緦麻之恩,而嫂叔無服。又舅之與姨,親疏相似,而服紀有殊,理未為得。宜集學者詳議。餘有親重
 而服輕者,亦附奏聞。」於是侍中魏徵、禮部侍郎令狐德棻等奏議曰:



 臣聞禮所以決嫌疑,定猶豫,別同異,明是非者也。非從天降,非從地出,人情而已矣。夫親族有九,服術有六,隨恩以薄厚,稱情以立文。然舅之與姨,雖為同氣,論情度義,先後實殊。何則?舅為母之本族,姨乃外戚他族,求之母族,姨不在焉,考之經典,舅誠為重。故周王念齊,每稱舅甥之國;秦伯懷晉,實切《渭陽》之詩。在舅服止一時,為姨居喪五月,循名喪實,逐末棄本。蓋古人
 之情,或有未達,所宜損益,實在茲乎!



 《記》曰:「兄弟之子,猶子也。蓋引而進之也;嫂叔之不服,蓋推而遠之也。」禮:繼父同居,則為之期;未嘗同居,則不為服。從母之夫,舅之妻,二夫人相為服。或曰,同爨緦。然則繼父之徒,並非骨肉,服重由乎同爨,恩輕在乎異居。故知制服雖系於名;亦緣恩之厚薄者也。或有長年之嫂,遇孩童之叔,劬勞鞠養,情若新生,分饑共寒,契闊偕老。譬同居之繼父,方他人之同爨,情義之深淺,寧可同日而言哉!在其生也,
 愛之同於骨肉;及其死也,則曰推而遠之。求之本原,深所未諭。若推而遠之為是,則不可生而共居;生而共居之為是,則不可死同行路。重其生而輕其死,厚其始而薄其終,稱情立文,其義安在?且事嫂見稱,載籍非一。鄭仲虞則恩禮甚篤,顏弘都則竭誠致感,馬援則見之必冠,孔伋則哭之為位。此並躬踐教義,仁深孝友,察其所尚之旨,豈非先覺者歟?但於其時,上無哲王,禮非下之所議,遂使深情鬱乎千載,至理藏於萬古,其來久矣,豈
 不惜哉!



 今屬欽明在辰,聖人有作,五禮詳洽,一物無遺。猶且永念慎終,凝神遐想。以為尊卑之敘,雖煥乎大備;喪紀之制,或情理未周。爰命秩宗,更詳考正。臣等奏遵明旨,觸類旁求,採摭群經,討論傳記。或引兼名實,無文之禮咸秩,敦睦之情畢舉,變薄俗於既往,垂篤義於將來,信六籍所不能談,超百王而獨得者也。諸儒所守,互有異同,詳求厥中,申明聖旨。



 謹按曾祖父母舊服齊衰三月,請加為齊衰五月。嫡子婦舊服大功,請加為期。眾子婦
 小功,今請與兄弟子婦同為大功九月。嫂叔舊無服,今請服小功五月報。其弟妻及夫兄,亦小功五月。舅服緦麻,請與從母同服小功。



 制可之。



 顯慶二年九月,修禮官長孫無忌等又奏曰:「依古喪服,甥為舅緦麻,舅報甥亦同此制。貞觀年中,八座議奏:『舅服同姨,小功五月。』而今律疏,舅報於甥,服猶三月。謹按旁尊之服,禮無不報,已非正尊,不敢降也。故甥為從母五月,從母報甥小功,甥為舅緦麻,舅亦報甥三月,是其義矣。今甥為舅使同從
 母之喪,則舅宜進甥以同從母之報。修律疏人不知禮意,舅報甥服,尚止緦麻,於例不通,禮須改正。今請修改律疏,舅報甥亦功。」又曰:「庶母古禮緦麻,新禮無服。謹按庶母之子,即是己昆季,為之杖期,而己與之無服。同氣之內,吉兇頓殊,求之禮情,深非至理。請依典故,為服緦麻。」制又從之。



 龍朔二年八月,所司奏:「司文正卿蕭嗣業,嫡繼母改嫁身亡,請申心制。據令,繼母改嫁及為長子,並不解官。」既而有敕:「雖云嫡母,終是繼母,據禮緣情,
 須有定制。付所司議定奏聞。」司禮太常伯隴西郡王博乂等奏稱:



 緬尋《喪服》,母名斯定,嫡、繼、慈、養,皆在其中。惟出母制,特言出妻之子,明非生己,則皆無服。是以令雲母嫁,又云出妻之子。出言其子,以著所生,嫁即言母,通包養、嫡,俱當解任,並合心喪。其不解者,惟有繼母之嫁。繼母為名,正據前妻之子;嫡於諸孽,禮無繼母之文。甲令今既見行,嗣業理申心制。然奉敕議定,方垂永則,令有不安,亦須厘正。竊以嫡、繼、慈、養,皆非所生,並同行路。
 嫁雖比出稍輕,於父終為義絕。繼母之嫁,既殊親母,慈、嫡義絕,豈合心喪?望請凡非所生,父卒而嫁,為父後者無服,非承重者杖期,並不心喪,一同繼母。有符情禮,無玷舊章。又心喪之制,惟施服屈,杖期之服,不應解官。而令文三年齊斬,亦入心喪之例;杖期解官,又有妻喪之舛。又依禮,庶子為其母緦麻三月。既是所生母服,準例亦合解官。令文漏而不言,於事終須修附。既與嫡母等嫁同一令條,總議請改,理為允愜者。



 依集文武官九品已
 上議。得司衛正卿房仁裕等七百三十六人議,請一依司禮狀,嗣業不解官。得右金吾衛將軍薛孤吳仁等二十六人議,請解嗣業官,不同司禮狀者。母非所生,出嫁義絕,仍令解職,有紊緣情。杖期解官,不甄妻服,三年齊斬,謬曰心喪。庶子為母緦麻,漏其中制。此並令文疏舛,理難因襲。依房仁裕等議,總加修附,垂之不朽。其禮及律疏有相關涉者,亦請準此改正。嗣業既非嫡母改醮,不合解官。



 詔從之。



 上元元年,天後上表曰:「至如父在為母服
 止一期,雖心喪三年,服由尊降。竊謂子之於母,慈愛特深,非母不生,非母不育。推燥居濕,咽苦吐甘,生養勞瘁,恩斯極矣!所以禽獸之情,猶知其母,三年在懷,理宜崇報。若父在為母服止一期,尊父之敬雖周,報母之慈有闕。且齊斬之制,足為差減,更令周以一期,恐傷人子之志。今請父在為母終三年之服。」高宗下詔,依議行焉。開元五年,右補闕盧履冰上言:「準禮,父在為母一周除靈,三年心喪。則天皇后請同父沒之服,三年然始除靈。雖
 則權行,有紊彞典。今陛下孝理天下,動合禮經,請仍舊章,庶葉通典。」於是下制令百官詳議;並舅及嫂叔服不依舊禮,亦合議定。刑部郎中田再思建議曰:



 乾尊坤卑,天一地二,陰陽之位分矣,夫婦之道配焉。至若死喪之威,隆殺之等,禮經五服之制,齊斬有殊,考妣三年之喪,貴賤無隔,以報免懷之慈,以酬罔極之恩者也。



 稽之上古,喪期無數,暨乎中葉,方有歲年。《禮》云:「五帝殊時,不相沿樂;三王異代,不相襲禮。」《白虎通》云:「質文再變,正朔
 三而復。」自周公制禮之後,孔父刊經已來,爰殊厭降之儀,以標服紀之節。重輕從俗,斟酌隨時。故知禮不從天而降,不由地而出也,在人消息,為適時之中耳。春秋諸國,魯最知禮,以周公之後,孔子之邦也。晉韓起來聘,言「周禮盡在魯矣。」齊仲孫來盟,言「魯猶秉周禮。」尚有子張問高宗諒陰三年,子思不聽其子服出母,子游謂同母異父昆弟之服大功,子夏謂合從齊衰之制。此等並四科之數,十哲之人,高步孔門,親承聖訓,及遇喪事,猶此
 致疑,即明自古已來,升降不一者也。



 三年之制,說者紛然。鄭玄以為二十七月,王肅以為二十五月。又改葬之服,鄭云服緦三月,王云訖葬而除。又繼母出嫁,鄭云皆服,王云從於繼育,乃為之服。又無服之殤,鄭云子生一月,哭之一日;王云以哭之一日易服之月。鄭、王祖經宗傳,各有異同;荀摯採古求遺,互為損益。方知去聖漸遠,殘缺彌多。故曰會禮之家,名為聚訟,寧有定哉!而父在為母三年,行之已逾四紀,出自高宗大帝之代,不從則
 天皇后之朝。大帝御極之辰,中宮獻書之日,往時參議,將可施行,編之於格,服之已久。前王所是,疏而為律;後王所是,著而為令。何必乖先帝之旨,阻人子之情,虧純孝之心,背德義之本?有何妨於聖化,有何紊於彞倫,而欲服之周年,與伯叔母齊焉,與姑姊妹同焉?夫三年之喪,如白駒之過隙,君子喪親,有終身之憂,何況再周乎!夫禮者,體也,履也,示之以跡。孝者,畜也,養也,因之以心。小人不恥不仁,不畏不義。服之有制,使愚人企及;衣之
 以衰,使見之摧痛。以此防人,人猶有朝死而夕忘者;以此制人,人猶有釋服而從吉者。方今漸歸古樸,須敦孝義,抑賢引愚,理資寧戚,食稻衣錦,所不忍聞。



 若以庶事朝儀,一依周禮,則古之人臣見君也,公卿大夫贄羔雁、珪璧,今何故不依乎?周之用刑也,墨、劓、宮、刖,今何故不行乎?周則侯、甸、男、衛,朝聘有數,今何故不行乎?周則不五十不仕,七十不入朝,今何故不依乎?周則井、邑、丘、甸,以立征稅,今何故不行乎?周則三老五等,父死子及,今
 何故不行乎?周則冠冕衣裘,乘車而戰,今何故不行乎?周則分土五更,膠序養老,今何故不行乎?諸如此例,不可勝述。何獨孝思之事,愛一年之服於其母乎?可為痛心,可為慟哭者!



 《詩》云:「哀哀父母,生我劬勞。」《禮》云:「父之親子也,親賢而下無能;母之親子也,賢則親之,無能則憐之。」阮嗣宗晉代之英才,方外之高士,以為母重於父。據齊斬升數,粗細已降,何忍服之節制,減至於周?豈後代之士,盡慚於古。循古未必是,依今未必非也。又同爨服
 緦,禮經明義。嫂叔遠別,同諸路人。引而進之,觸類而長。猶子咸衣苴枲,季父不服緦麻,推遠之情有餘,睦親之義未足。又母之昆弟,情切渭陽,翟酺訟舅之冤,甯氏宅甥之相,我之出也,義亦殷焉。不同從母之尊,遂降小功之服,依諸古禮,有爽俗情。今貶舅而宗姨,是陋今而榮古。此並太宗之制也,行之百年矣,輒為刊復,實用有疑。



 於是紛議不定。履冰又上疏曰:「《禮》:父在,為母十一月而練,十三月而祥,十五月而禫,心喪三年。上元中,則天皇
 后上表,請同父沒之服,亦未有行。至垂拱年中,始編入格,易代之後,俗乃通行。臣開元五年,頻請仍舊。恩敕並嫂叔舅姨之服,亦付所司詳議。諸司所議,同異相參。所司惟執齊斬之文,又曰亦合典禮。竊見新修之格,猶依垂拱之偽,致有祖父母安存,子孫之妻亡沒,下房筵幾,亦立再周,甚無謂也。據《周易·家人》卦云:『利女貞女正位於內,男正位於外。男女正,天地之大義。家人有嚴君焉,父母之謂也。父父、子子、兄兄、弟弟、夫夫、婦婦,家道正而天
 下正矣。』《禮》:『女在室,以父為天;出嫁,以夫為天。』又:『在家從父,出嫁從夫,夫死從子。』本無自專抗尊之法。即《喪服四制》云:『天無二日,土無二王,國無二君,家無二尊,以一理之也。故父在為母服周者,避二尊也。』伏惟陛下正持家國,孝理天下,而不斷在宸衷,詳正此禮,無隨末俗,顧念兒女之情。臣恐後代復有婦奪夫政之敗者。」



 疏奏未報。履冰又上奏曰:



 臣聞夫婦之道,人倫之始。尊卑法於天地,動靜合於陰陽,陰陽和而天地生成,夫婦正而人倫式序。自家刑國,
 牝雞無晨,四德之禮不愆,三從之義斯在。即《喪服四制》云:「天無二日,土無二王,國無二君,家無二尊,以一理之也。故父在為母服周者,見無二尊也。」準舊儀,父在為母一周除靈,再周心喪。父必三年而後娶者,達子之志焉。豈先聖無情於所生,固有意於家國者矣。原夫上元肇年,則天已潛秉政,將圖僭篡,預自崇先。請升慈愛之喪,以抗尊嚴之禮,雖齊斬之儀不改,而幾筵之制遂同。數年之間,尚未通用。天皇晏駕,中宗蒙塵。垂拱之末,果行聖
 母之偽符;載初之元,遂啟易代之深釁。孝和雖名反正,韋氏復效晨鳴。孝和非意暴崩,韋氏旋即稱制。不蒙陛下英算,宗廟何由克復?《易》云:「臣弒其君,子弒其父,非一朝一夕之故。」其斯之謂矣。臣謹尋禮意,防杜實深,若不早圖刊正,何以垂戒於後?所以薄言禮教,請依舊章,恩敕通明,蒙付所司詳議。



 且臣所獻者,蓋請正夫婦之綱,豈忘母子之道。諸議多不討其本源,所非議者,大凡只論罔極之恩;喪也寧戚;禽獸識母而不識父;秦燔書後
 禮經殘缺,後儒纘集,不足可憑;豈得與伯叔母服同,豈得與姑姊妹制等;三王不相襲禮,五帝不相沿樂;齊斬足為升降,歲年何忍不同:此並道聽途說之言,未習先王之旨,又安足以議經邦理俗之禮乎?臣請據經義以明之。所云「罔極之恩」者,春秋祭祀,以時思之。君子有終身之憂,霜露之感,豈止一二周之服哉!故聖人恐有朝死而夕忘,曾鳥獸之不若,為立中制,使賢不肖共成文理而已。所云「喪也寧戚」者,孔子答林放之問。至如太奢
 太儉,太易太戚,皆非禮中。茍不得中,名為俱失,不如太儉太戚焉。毀而滅性,猶愈於朝死夕忘焉。此論臨喪哀毀之容,豈比於同宗異姓之服?所云「禽獸識母而不識父」者,禽獸群居而聚簹,而無家國之禮,少雖知親愛其母,長不解尊嚴其父。引此為諭則亦禽獸之不若乎!所云「秦燔書後禮經殘缺,後儒纘集,不足可憑」者,人間或有遺逸,豈亦家戶到而燔之」假若盡燔,茍不可信,則墳黃都謬,庠序徒立,非聖之談,復雲安屬?所云「與伯叔姑
 姊服同」者,伯叔姑姊有筵杖之制、三年心喪乎?所云「五帝不相沿樂,不相襲禮」,誠哉是言!此是則天懷私苞禍之情,豈可復相沿樂襲禮乎?所云「齊斬足為升降」者,母齊父斬,不易之禮。



 按《三年問》云:「將由修飾之君子與,三年之喪,若駟之過隙,遂之,則是無窮也。然則何以周也?曰:至親以周斷。是何也?曰:天地則已易矣,四時則已變矣,其在天地之者,莫不更始焉,以是象之也。然則何以三年?曰:加重焉耳。」故父加至再周,父在為母加
 三年心喪。今者還同父沒之制,則尊厭之律安施?《喪服四制》又曰:「凡禮之大體,體天地,法四時,則陰陽,順人情,故謂之禮。」訾之者是不知禮之所由生。非徒不識禮之所由制,亦恐未達孝子之通義。



 臣謹按《孝經》,以明陛下孝治之合至德要道,請論世欲訾禮之徒。夫至德謂孝悌,要道謂禮樂。「移風易俗,莫善於樂,安上治民,莫善於禮。」又《禮》有「無體之禮,無聲之樂。」按《孝經援神契》云:「天子孝曰就,就之為言成也。天子德被天下,澤及萬物,始終
 成就,則其親獲安,故曰就也。諸侯孝曰度,度者法也。諸侯居國,能奉天子法度,得不危溢,則其親獲安,故曰度也。卿大夫孝曰譽,譽之為言名也。卿大夫言行布滿,能無惡稱,譽達遐邇,則其親獲安,故曰譽也。士孝曰究,究者以明審為義。士始升朝,辭親入仕,能審資父事君之禮,則其親獲安,故曰究也。庶人孝曰畜,畜者含畜為義。庶人含情受樸,躬耕力作,以畜其德,則其親獲安,故曰畜也。」陛下以韋氏構逆,中宗降禍,宸衷哀憤,睿情卓烈。
 初無一旅之眾,遂殄九重之妖,定社稷於阽危,拯宗枝於塗炭。此陛下孝悌之至,通於神明,光於四海,無所不通。使諸侯得守其法度,卿大夫得盡其言行,士得資親以事君,庶人得用天而分地。此陛下無體之禮,以安上理人也。上元以來,政由武氏,文明之後,法在兇人。賊害宗親,誅滅良善,勛階歲累,酺赦年頻。佞之則榮華,正之則遷謫。神龍、景雲之際,其事尤繁;先天、開元之間,斯弊都革。此陛下之無聲之樂,以移風易俗也。



 臣前狀單略,
 議者未識臣之懇誠。謹具狀重進,請付中書門下商量處分。臣言若讜,然敢側足於軒墀;臣言不忠,伏請竄跡於荒裔。



 左散騎常侍元行沖奏議曰:「天地之性,惟人最靈者,蓋以智周萬物,惟睿作聖,明貴賤,辨尊卑,遠嫌疑,分情理也。是以古之聖人,徵性識本,緣情制服,有申有厭。天父、天夫,故斬衰三年,情理俱盡者,因心立極也。生則齊體,死則同穴,比陰陽而配合,同兩儀而成化。而妻喪杖期,情禮俱殺者,蓋以遠嫌疑,尊乾道也。父為嫡子
 三年斬衰,而不去職者,蓋尊祖重嫡,崇禮殺情也。資於事父以事君,孝莫大於嚴父。故父在,為母罷職齊周而心喪三年,謂之尊厭者,則情申而禮殺也。斯制也,可以異於飛走,別於華夷。羲、農、堯、舜,莫之易也;文、武、周、孔,同所尊也。今若舍尊厭之重,虧嚴父之義,略純素之嫌,貽非聖之責,則事不師古,有傷名教矣。姨兼從母之名,即母之女黨,加於舅服,有理存焉。嫂叔不服,避嫌疑也。若引同爨之緦,以忘推遠之跡,既乖前聖,亦謂難從。謹詳
 三者之疑,並請依古為當。」自是百僚議意不決。



 至七年八月,下敕曰:「惟周公制禮,當歷代不刊;況子夏為《傳》,乃孔門所受。格條之內,有父在為母齊衰三年,此有為而為,非尊厭之義。與其改作,不如師古,諸服紀宜一依《喪服》文。」自是卿士之家,父在為母行服不同:或既周而禫,禫服六十日釋服,心喪三年者;或有既周而禫服終三年者;或有依上元之制,齊衰三年者。時議者是非紛然,元行沖謂人曰:「聖人制厭降之禮,豈不知母恩之深也,
 以尊祖貴禰,欲其遠別禽獸,近異夷狄故也。人情易搖,淺識者眾。一紊其度,其可止乎!」二十年,中書令蕭嵩與學士改修定五禮,又議請依上元敕,父在為母齊衰三年為定。及頒禮,乃一依行焉。



 二十三年,藉田禮畢,正制曰:「服制之紀,或有所未通,宜令禮官學士詳議聞奏。」太常卿韋縚奏曰:「謹按《儀禮喪服》:舅,緦麻三月。從母,小功五月。《傳》曰:可以小功,以名加也。堂姨舅、舅母,恩所不及。外祖父母。小功五月。《傳》曰:何以小功,以尊加也。舅,緦麻
 三月,並是情親而服屬疏者也。外祖正尊,同於從母之服。姨舅一等,服則輕重有殊。堂姨舅親即未疏,恩絕不相為服。親舅母來承外族,同爨之禮不加。竊以古意猶有所未暢者也。且為外祖小功,此則正尊情甚親而服屬疏者也,請加至大功九月。姨舅儕類,親既無別,服宜齊等,請為舅加至小功五月。堂姨舅疏降一等,親舅母從服之例,先無制服之文,並望加至袒免。臣聞禮以飾情,服從義制,或有沿革,損益可明。事體既大,理資詳審。望付
 尚書省集眾官吏詳議,務從折衷,永為典則。」



 於是太子賓客崔沔建議曰:「竊聞大道既隱,天下為家。聖人因之,然後制禮。禮教之設,本為正家,家道正而天下定矣。正家之道,不可以貳,總一定議,理歸本宗。父以尊崇,母以厭降,豈忘愛敬,宜存倫序。是以內有齊斬,外服皆緦麻,尊名所加,不過一等,此先王不易之道也。前聖所志,後賢所傳,其來久矣。昔辛有適伊川,見被發而祭於野者,曰:『不及百年,此其戎乎?其禮先亡矣」!貞觀修禮,時改舊
 章,漸廣渭陽之恩,不遵洙、泗之典。及弘道之後,唐隆之間,國命再移於外族矣。禮亡徵兆,儻或斯見,天人之際,可不誡哉!開元初,補闕盧履冰嘗進狀論喪服輕重,敕令僉議。於時群議紛拏,各安積習,太常禮部,奏依舊定。陛下運稽古之思,發獨斷之明,至開元八年,特降別敕,一依古禮。事符故實,人知向方,式固宗盟,社稷之福。更圖異議,竊所未詳。願守八年明旨,以為萬代成法。」



 職方郎中韋述議曰:



 天生萬物,惟人最靈。所以尊尊親親,別
 生分類,存則盡其愛敬,沒則盡其哀戚。緣情而制服,考事而立言,往聖討論,亦已勤矣。上自高祖,下至玄孫,以及其身,謂之九族。由近而及遠,稱情而立文,差其輕重,遂為五服。雖則或以義降,或以名加,教有所從,理不逾等。百王不易,三代可知,日月同懸,咸所仰也。自微言既絕,大義復乖,雖文質有遷,而必遵此制。



 謹按《儀禮·喪服傳》曰:「外親之服皆緦麻。」鄭玄謂:「外親,異姓。正服不過緦麻。」外祖父母,小功五月,以尊加也。從母,小功五月,以名
 加也。舅甥外孫、中外昆弟,依本服緦麻三月。若以匹敵,外祖則祖也,舅則伯叔父之別也。姨舅伯叔,則父母之恩不殊,而獨殺於外氏,聖人之心,良有以也。《喪服傳》曰:「禽獸知母而不知父。」野人曰,父母何算焉。都邑之士,則知尊禰矣。大夫及學士,則知尊祖也。諸侯及其太祖,天子及其始祖。聖人究天道而厚於祖禰,系姓族而親其子孫,近則別其賢愚,遠則異於禽獸。由此言之,母黨比於本族,不可同貫明矣。且家無二尊,喪無二斬,人之所
 奉,不可貳也。特重於大宗者,降其小宗;為人後者,減其父母之服;女子出嫁,殺其本家之喪。蓋所存者遠,所抑者私也。今若外祖及舅更加服一等,堂舅及姨列於服紀之內,則中外之制,相去幾何?廢禮徇情,所務者末。古之制作者知人情之易搖,恐失禮之將漸,別其同異,輕重相懸,欲使後來之人,永不相雜。微旨斯在,豈徒然哉!且五服有上殺之義,必循源本,方及條流。伯叔父母本服大功九月,從父昆弟亦大功九月,並以上出於祖,其
 服不得過於祖也。從祖祖父母、從祖父母、從祖昆弟,皆小功五月;以出於曾祖,服不得過於曾祖也。族祖祖父母、族祖父母、族祖昆弟,皆緦麻三月,以其出於高祖,其服不得過於高祖也。堂舅姨既出於外曾祖,若為之制服,則外曾祖父母及外伯叔祖父母,亦宜制服矣。外祖加至大功九月,則外曾祖合至小功,外高祖合至緦麻。若舉此而舍彼,事則不均;棄親而錄疏,理則不順。推而廣之,是與本族無異矣。服皆有報,則堂外甥、外曾孫、侄
 女之子,皆須制服矣。



 聖人豈薄其骨肉,背其恩愛。情之親者,服制乃輕,蓋本於公者薄於私,存其大者略其細,義有所斷,不得不然。茍可加也,亦可減也,往聖可得而非,則禮經可得而隳矣。先王之制,謂之彞倫,奉以周旋,猶恐失墜,一紊其敘,庸可止乎?且舊章淪胥,為日已久矣。所存者無幾,又欲棄之,雖曰未達,不知其可。請依《儀禮·喪服》為定。



 禮部員外郎楊仲昌議曰:「謹按《儀禮》曰:『外服皆緦。』又曰:『外祖父母以尊加,從母以名加,並為小功
 五月。』其為舅緦,鄭文貞公魏徵已議同從母例,加至小功五月訖。今之所加,豈異前旨?雖文貞賢也,而周、孔聖也,以賢改聖,後學何從?堂舅姨、堂舅母,並升為袒免,則何以祖述禮經乎?如以外祖父母加至大功,則豈無加報於外孫乎?如外孫為報,服大功,則本宗庶孫,何同等而相淺乎?儻必如是,深所不便。竊恐內外乖序,親疏奪倫、情之所沿,何所不至,理必然也。昔子路有姊之喪而不除,孔子問之,子路對曰:『吾寡兄弟而不忍也。』子曰:『先
 王制禮,行道之人皆不忍也。』子路聞而除之。此則聖人因言以立訓,援事抑情之明例也。禮不云乎,無輕議禮。明共蟠於天地,並彼日月,賢者由之,安敢小有損益也!況夫《喪服》之紀,先王大猷,奉以周旋,以匡人道。一辭寧措,千載是遵,涉於異端,豈曰弘教。伏望各依正禮,以厚儒風。太常所謂增加,愚見以為不可。」又戶部郎中楊伯成、左監門錄事參軍劉秩並同是議,與沔等略同。議奏,上又手敕侍臣等曰:「朕以為親姨舅既服小功,則舅母
 於舅有三年之服,服是受我而厚,以服制情,則舅母之服,不得全降於舅也,宜服緦麻。堂姨舅古今未制服,朕思敦睦九族,引而親之,宜服袒免。又鄭玄注《禮記》云『同爨緦』,若比堂姨舅於同爨,親則厚矣。又《喪服傳》云,『外親之服皆緦』,是亦不隔於堂姨舅也。若以所服不得過本,而須為外曾祖父母及外伯叔祖父母制服,亦何傷乎?是皆親親敦本之意,卿等更熟詳之。」



 侍中裴耀卿、中書令張九齡、禮部尚書李林甫等奏曰:「外族之親,禮無厭
 降。外甥既為舅母制服,舅母還合報之。夫外甥既為報服,則與夫之姨舅,以類是同,外甥之妻,不得無服。所增者頗廣,所引者漸疏。微臣愚蒙,猶有未達。」玄宗又手制答曰:「從服有六,此其一也。降殺之制,禮無明文。此皆自身率親,用為制服。所有存抑,盡是推恩。朕情有未安,故令詳議,非欲茍求變古,以示不同。卿等以為『外族之親,禮無厭降,報服之制,所引甚疏』。且姨舅者,屬從之至近也,以親言之,則亦姑伯之匹敵也。豈有所引者疏,而降
 所親者服?又婦,從夫者也。夫之姨舅,夫既有服,從夫而服,由是睦親。實欲令不肖者企及,賢者俯就。卿等宜熟詳之。」耀卿等奏曰:「陛下體至仁之德,廣推恩之道,將弘引進,以示睦親,再發德音,更令詳議。臣等按《大唐新禮》:親舅加至小功,與從母同服。此蓋當時特命,不以輕重遞增,蓋不欲參於本宗,慎於變禮者也。今聖制親姨舅小功,更制舅母緦麻,堂姨舅袒免等服,取類《新禮》,垂示將來,通於物情,自我作則。群儒風議,徒有稽留。並望準
 制施行。」制從之。天寶六載正月,出嫁母宜終服三年。



\end{pinyinscope}