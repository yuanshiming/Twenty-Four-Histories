\article{卷一百七十四 裴度}

\begin{pinyinscope}

 ○裴度



 裴度,字中立,河東聞喜人。祖有鄰,濮州濮陽令。父漵,河南府澠池丞。度,貞元五年進士擢第,登宏辭科。應制舉賢良方正、能直言極諫科,對策高等,授河陰縣尉。遷監
 察御史,密疏論權幸,語切忤旨,出為河南府功曹。遷起居舍人。元和六年,以司封員外郎知制誥,尋轉本司郎中。



 七年,魏博節度使田季安卒。其子懷諫幼年不任軍政,牙軍立小將田興為留後。興布心腹於朝廷,請守國法,除吏輸常賦,憲宗遣度使魏州宣諭。興承僭侈之後,車服垣屋,有逾制度,視事齋閣,尤加宏敞。興惡之,不於其間視事,乃除舊採訪使居之,請度為壁記,述興謙降奉法,魏人深德之。興又請度遍至屬郡,宣述詔旨,魏
 人郊迎感悅。使還,拜中書舍人。



 九年十月,改御史中丞。宣徽院五坊小使,每歲秋按鷹犬於畿甸,所至官吏必厚邀供餉,小不如意,即恣其須索,百姓畏之如寇盜。先是,貞元末,此輩暴橫尤甚,乃至張網羅於民家門及井,不令出入汲水,曰:「驚我供奉鳥雀。」又群聚於賣酒食家,肆情飲啖。將去,留蛇一篋,誡之曰:「吾以此蛇致供奉鳥雀,可善飼之,無使饑渴。」主人賂而謝之,方肯攜蛇篋而去。至元和初,雖數治其弊,故態未絕。小使嘗至下邽縣,
 縣令裴寰性嚴刻,嫉其兇暴,公館之外,一無曲奉。小使怒,構寰出慢言。及上聞,憲宗怒,促令攝寰下獄,欲以大不敬論。宰相武元衡等以理開悟,帝怒不解。度入延英奏事,因極言論列,言寰無罪。上愈怒曰:「如卿之言,寰無罪即決五坊小使;如小使無罪,即決裴寰。」度對曰:「按罪誠如聖旨,但以裴寰為令長,憂惜陛下百姓如此,豈可加罪?」上怒色遽霽。翌日,令釋寰。尋以度兼刑部侍郎,奉使蔡州行營,宣諭諸軍。既還,帝問諸將之才,度曰:「臣觀
 李光顏見義能勇,終有所成。」不數日,光顏奏大破賊軍於時曲,帝尤嘆度之知人。



 十年六月,王承宗、李師道俱遣刺客刺宰相武元衡,亦令刺度。是日,度出通化里,盜三以劍擊度,初斷靴帶,次中背,才絕單衣,後微傷其首,度墮馬。會度帶氈帽,故創不至深。賊又揮刃追度,度從人王義乃持賊連呼甚急,賊反刃斷義手,乃得去。度已墮溝中,賊謂度已死,乃舍去。居三日,詔以度為門下侍郎、同中書門下平章事。



 度勁正而言辯,尤長於政體,凡
 所陳諭,感動物情。自魏博使還,宣達稱旨,帝深嘉屬。又自蔡州勞軍還,益聽其言。尚以元衡秉政,大用未果,自盜發都邑,便以大計屬之。



 初,元衡遇害,獻計者或請罷度官以安二鎮之心,憲宗大怒曰:「若罷度官,是奸計得行,朝綱何以振舉?吾用度一人,足以破此二賊矣。」度亦以平賊為己任。度以所傷請告二十餘日,詔以衛兵宿度私第,中使問訊不絕。未拜前一日,宣旨謂度曰:「不用宣政參假,即延英對來。」及度入對,撫諭周至。時群盜乾
 紀,變起都城,朝野恐駭。及度命相制下,人情始安,以為必能殄寇。自是誅賊之計,日聞獻替,用軍愈急。



 十一年,莊憲皇后崩,度為禮儀使。上不聽政,欲準故事置塚宰,以總百司。度獻議曰:「塚宰是殷、周六官之首,既掌邦理,實統百司。故王者諒闇,百官有權聽之制。後代設官,既無此號,不可虛設。且國朝故事,或置或否,古今異制,不必因循。」敕旨曰:「諸司公事,宜權取中書門下處分。」識者是之。



 六月,蔡州行營唐鄧節度使高霞寓兵敗於鐵城,
 中外恟駭。先是,詔群臣各獻誅吳元濟可否之狀。朝臣多言罷兵赦罪為便,翰林學士錢徽、蕭俛語尤切,唯度言賊不可赦。及霞寓敗,宰相以上必厭兵,欲以罷兵為對。延英方奏,憲宗曰:「夫一勝一負,兵家常勢。若帝王之兵不合敗,則自古何難於用兵,累聖不應留此兇賊。今但論此兵合用與否,及朝廷制置當否,卿等唯須要害處置。將帥有不可者,去之勿疑;兵力有不足者,速與應接。何可以一將不利,便沮成計?」於是宰臣不得措言,朝
 廷無敢言罷兵者,故度計得行。



 王稷家二奴告稷換父遺表,隱沒進奉物。留其奴於仗內,遣中使往東都檢責稷之家財。度奏曰:「王鍔身歿之後,其家進奉已多。今因其奴告檢責其家事,臣恐天下將帥聞之,必有以家為計者。」憲宗即日遣中使還,二奴付京兆府決殺。



 十二年,李醖、李光顏屢奏破賊,然國家聚兵淮右四年,度支供餉,不勝其弊,諸將玩寇相視,未有成功,上亦病之。宰相李逢吉、王涯等三人,以勞師弊賦,意欲罷兵,見上互陳
 利害。度獨無言。帝問之,對曰:「臣請身自督戰。」明日延英重議,逢吉等出,獨留度,謂之曰:「卿必能為朕行乎?」度俯伏流涕曰:「臣誓不與此賊偕全。」上亦為之改容。度復奏曰:「臣昨見吳元濟乞降表,料此逆賊勢實窘蹙。但諸將不一,未能迫之,故未降耳。若臣自赴行營,則諸將各欲立功以固恩寵,破賊必矣!」上然之。翌日,詔曰:



 輔弼之臣,軍國是賴。興化致理,秉鈞以居。取威定功,則分閫而出。所以同君臣之體,一中外之任焉。屬者問罪汝南,致誅
 淮右,蓋欲刷其污俗,吊彼頑人。雖挈地求生者實繁有徒,而嬰城執迷者未翦其類,何獸困而猶鬥,豈鳥窮之無歸歟?由是遙聽鼓鼙,更張琴瑟,煩我臺席,董茲戎旃。朝議大夫、守中書侍郎、同平章事、飛騎尉、賜紫金魚袋裴度,為時降生,協朕夢卜,精辨宣力,堅明納忠。當軸而才謀老成,運籌而智略有定。司其樞務,備知四方之事;付以兵要,必得萬人之心。是用禱於上玄,揀此吉日,帶丞相之印綬,所以尊其名;賜諸侯之斧鉞,所以重其命。
 爾宜宣布清問,恢壯皇猷,感勵連營,蕩平多壘,招懷孤疾,字撫夷傷。況淮西一軍,素效忠節,過海赴難,史冊書勛。建中初,攻破襄陽,擒滅崇義。比者脅於兇逆,歸命無由。每念前勞,常思安撫。所以內輟輔臣,俾為師率,實欲保全慰諭,各使得宜。汝往欽哉!無越我丕訓。可門下侍郎、同中書門下平章事、蔡州刺史,充彰義軍節度、申光蔡觀察等使,仍充淮西宣慰招討處置使。



 詔出,度以韓弘為淮西行營都統,不欲更為招討,請只稱宣慰處置
 使。又以此行既兼招撫,請改「翦其類」為「革其志」。又以弘已為都統,請改「更張琴瑟」為「近輟樞衡」,請改「煩我臺席」為「授以成算」,皆從之。仍奏刑部侍郎馬總為宣慰副使,太子右庶子韓愈為彰義行軍司馬,司勛員外郎李正封、都官員外郎馮宿、禮部員外郎李宗閔等為兩使判官書記,皆從之。



 初,德宗朝政多僻,朝官或相過從,多令金吾伺察密奏,宰相不敢於私第見賓客。及度輔政,以群賊未誅,宜延接奇士,共為籌畫,乃請於私居接延賓
 客,憲宗許之。自是天下賢俊,得以效計議於丞相,接士於私第,由度之請也。



 自討淮西,王師屢敗。論者以殺傷滋甚,轉輸不逮,擬議密疏,紛紜交進。度以腹心之疾,不時去之,終為大患,不然,兩河之盜,亦將視此為高下。遂堅請討伐,上深委信,故聽之不疑。



 度既受命,召對於延英,奏曰:「主憂臣辱,義在必死。賊滅,則朝天有日;賊在,則歸闕無期。」上為之惻然流涕。



 十二年八月三日,度赴淮西,詔以神策軍三百騎衛從,上禦通化門慰勉之。度樓
 下銜涕而辭,賜之犀帶。度名雖宣慰,其實行元帥事,仍以郾城為治所。上以李逢吉與度不協,乃罷知政事,出為劍南東川節度。



 既離京,淮西行營大將李光顏、烏重胤謂監軍梁守謙曰:「若俟度至而有功,即非我利。可疾戰,先事立功。」是月六日,將出兵,與賊戰於賈店,為賊所敗。度二十七日至郾城,巡撫諸軍,宣達上旨,士皆賈勇。時諸道兵皆有中使監陣,進退不由主將,戰勝則先使獻捷,偶衄則凌挫百端。度至行營,並奏去之,兵柄專制
 之於將,眾皆喜悅。軍法嚴肅,號令畫一,以是出戰皆捷。度遣使入蔡州,元濟與度書曰:「比密有降款,而索日進隔河大呼,遂令三軍防元濟,故歸首無路。」



 十月十一日,唐鄧節度使李醖,襲破懸瓠城,擒吳元濟。度先遣宣慰副使馬總入城安撫。明日,度建彰義軍節,領洄曲降卒萬人繼進。李愬具櫜鞬以軍禮迎度,拜之路左。度既視事,蔡人大悅。舊令:途無偶語,夜不燃燭,人或以酒食相過從者,以軍法論。度乃約法,唯盜賊、鬥殺外,餘盡除之,
 其往來者,不復以晝夜為限。於是蔡之遺黎,始知有生人之樂。



 初,度以蔡卒為牙兵。或以為反側之子,其心未安,不可自去其備。度笑而答曰:「吾受命為彰義軍節度使,元惡就擒,蔡人即吾人也。」蔡之父老,無不感泣。申、光之民,即時平定。



 十一月二十八日,度自蔡州入朝,留副使馬總為彰義軍留後。初,度入蔡州,或譖度沒入元濟婦女珍寶。聞,上頗疑之。上欲盡誅元濟舊將,封二劍以授梁守謙,使往蔡州。度回至郾城遇之,乃復與守謙入
 蔡州,量罪加刑,不盡如詔。守謙固以詔止,度先以疏陳,乃徑赴闕下。二月,詔加度金紫光祿大夫、弘文館大學士,賜勛上柱國,封晉國公,食邑三千戶,復知政事。



 憲宗以淮西賊平,因功臣李光顏等來朝,欲開內宴,詔六軍使修麟德殿之東廊。軍使張奉國以公費不足,出私財以助用,訴於執政。度從容啟曰:「陛下營造,有將作監等司局,豈可使功臣破產營繕?」上怒奉國洩漏,乃令致仕。其浚龍首渠,起凝暉殿,雕飾綺煥,徙佛寺花木以植於
 庭。有程異、皇甫鎛者,奸纖用事,二人領度支鹽鐵,數貢羨餘錢,助帝營造。帝又以異、鎛平蔡時供饋不乏,二人並命拜同平章事。度延英面論曰:「程異、皇甫鎛,錢穀吏耳,非代天理物之器也。陛下徇耳目之欲,拔置相位,天下人騰口掉舌,以為不可,於陛下無益。願徐思其宜。」帝不省納。度三上疏論之,請罷己相位,上都不省。事見《鎛傳》。



 又賈人張陟負五坊使楊朝汶息利錢潛匿,朝汶於陟家得私簿記,有負錢人盧載初,云是故西川節度使
 盧坦大夫書跡,朝汶即捕坦家人拘之。坦男不敢申理,即以私錢償之。及徵驗書跡,乃故鄭滑節度盧群手書也。坦男理其事,朝汶曰:「錢已進過,不可復得。」御史中丞蕭俛及諫官上疏陳其暴橫之狀,度與崔群因延英對,極言之。憲宗曰:「且欲與卿商量東軍,此小事我自處置。」度奏曰:「用兵,小事也;五坊追捕平人,大事也。兵事不理,只憂山東;五坊使暴橫,恐亂輦轂。」上不悅。帝久方省悟,召楊朝汶數之曰:「向者為爾使我羞見宰相。」遽命誅之。



 初,淮、蔡既平,鎮、冀王承宗甚懼。度遣辯士游說,客於趙、魏間。使說承宗,令割地入質以效順。故承宗求援於田弘正,由度使客諷動之,故兵不血刃,而承宗鼠伏。



 十三年,李師道翻覆違命,詔宣武、義成、武寧、橫海四節度之師與田弘正會軍討之。弘正奏請取黎陽渡河,會李光顏等軍齊進。帝召宰臣於延英議可否,皆曰:「閫外之事,大將制之,既有奏陳,宜遂其請。」度獨以為不可,奏曰:「魏博一軍,不同諸道。過河之後,卻退不得,便須進擊,方見
 成功。若取黎陽渡河,既才離本界,便至滑州,徒有供餉之勞,又生顧望之勢。況弘正、光顏並少威斷,更相疑惑,必恐遷延。然兵事不從中制一定處分。或慮不可。若欲於河南持重,則不如河北養威。不然,則且秣馬厲兵,候霜降水落,於楊劉渡河,直抵鄆州。但得至陽穀已來下營,則兵勢自盛,賊形自撓。」上曰:「卿言是矣。」乃詔弘正取楊劉渡河。及弘正軍既濟河而南,距鄆州四十里築壘,賊勢果蹙。頃之,誅師道。



 度執性不回,忠於事上,時政或
 有所闕,靡不極言之,故為奸臣皇甫鎛所構,憲宗不悅。十四年,檢校左僕射、同中書門下平章事、太原尹、北都留守、河東節度使。



 穆宗即位,長慶元年秋,張弘靖為幽州軍所囚,田弘正於鎮州遇害,硃克融、王廷湊復亂河朔,詔度以本官充鎮州四面行營招討使。時驕主荒僻,輔相庸才,制置非宜,致其復亂。雖李光顏、烏重胤等稱為名將,以十數萬兵擊賊,無尺寸之功。蓋以勢既橫流,無能復振。然度受命之日,搜兵補卒,不遑寢息。自董西
 師,臨於賊境,屠城斬將,屢以捷聞。穆宗深嘉其忠款,中使撫諭無虛月,進位檢校司空,兼充押北山諸蕃使。



 時翰林學士元稹,交結內官,求為宰相,與知樞密魏弘簡為刎頸之交。稹雖與度無憾,然頗忌前達加於己上。度方用兵山東,每處置軍事,有所論奏,多為稹輩所持。天下皆言稹恃寵熒惑上聽,度在軍上疏論之曰:



 臣聞主聖臣直。今既遇聖主,輒為直臣,上答殊私,下塞群謗,誓除國蠹,無以家為。茍獻替之可行,何性命之足惜?伏惟
 皇帝陛下恭承丕業,光啟雄圖,方殄頑人之風,以立太平之事。而逆豎構亂,震驚山東;奸臣作朋,撓敗國政。陛下欲掃蕩幽、鎮,宜肅清朝廷。何者?為患有大小,議事有先後。河朔逆賊,只亂山東;禁闈奸臣,必亂天下。是則河朔患小,禁闈患大。小者,臣等與諸戎臣必能翦滅;大者,非陛下制斷,非陛下覺悟,無計驅除。今文武百僚,中外萬品,有心者無不憤忿,有口者無不咨嗟。直以威權方重,獎用方深,無所畏避,不敢抵觸,恐事未行禍已及,
 不為國計,且為身謀。



 臣比者猶思隱忍,不願發明。一則以罪惡如山,怨謗如雷,伏料聖明,必自誅殛;一則以四方無事,萬樞且過,雖紀綱潛壞,賄賂公行,俟其貫盈,必自顛覆。今屬兇徒擾攘,宸衷憂軫,凡有制命,計於安危。痛此奸邪,恣行欺罔,干亂聖略,非止一途。又翰苑舊臣,結為朋黨,陛下聽其所說,更訪於近臣,私相計會,更唱迭和,蔽惑聰明。所以臣自兵興已來,所陳章疏,事皆要切,所奉書詔,多有參差。惜陛下委付之意不輕,被奸臣
 抑損之事不少。



 臣素知佞幸,亦無讎嫌,只是昨者,臣請乘傳詣闕,面陳戎事,奸臣之徒,最所畏懼。知臣若到御坐之前,必能悉數其過,以此百計止臣此行。臣又請領兵齊進,逐便攻討,奸臣之黨,曲加阻礙。恐臣統率諸道,或有成功,進退皆受羈牽,意見悉遭蔽塞。復共一二憸狡,同辭合力。或兩道招撫,逗留旬時;或遣蔚州行營,拖曳日月。但欲令臣失所,使臣無成,則天下理亂,山東勝負,悉不顧矣。為臣事君,一至於此。且陛下左右前後,忠
 良至多,亦有熟會典章,亦有飽諳師旅,足得任使,何獨斯人?以臣愚見,若朝中奸臣盡去,則河朔逆賊,不討而自平;若朝中奸臣尚在,則逆賊縱平無益。



 臣讀國史,知代宗朝蕃戎侵軼,直犯都城。代宗不知,蓋被程元振蒙蔽,幾危社稷。當時柳伉,乃太常一博士耳,猶能抗表歸罪,為國除害。今臣年處,兼總將相,豈肯坐觀兇邪,有曀日月。不勝感憤嫉惡之至!謹附中使趙奉國以聞。倘陛下未信忠言,猶惑奸黨,伏乞出臣此表,令三事大夫與
 百僚集議。彼不受責,臣合伏辜,天鑒孔明,照臣肝血。但得天下之人,知臣不負陛下,則雖死之日,猶生之年。



 繼上三章,辭情激切。穆宗雖不悅,雖懼大臣正議,乃以魏弘簡為弓箭庫使,罷元稹內職。然寵稹之意未衰。俄拜稹平章事,尋罷度兵權,守司徒、同平章事,充東都留守。諫官相率伏閣詣延英門者日二三。帝知其諫,不即被召,皆上疏言:時未偃兵,度有將相全才,不宜置之散地。帝以章疏旁午,無如之何,知人情在度,遂詔度自太原
 由京師赴洛。及元稹為相,請上罷兵,洗雪廷湊、克融,解深州之圍,蓋欲罷度兵柄故也。



 二年三月,度至京師。既見,先敘克融、廷湊暴亂河朔,受命討賊無功;次陳除職東都,許令入覲。辭和氣勁,感動左右。度伏奏龍墀,涕泗鳴咽,帝為之動容,口自諭之曰:「所謝知,朕於延英待卿。」



 初,人以度無左右之助,為奸邪排擯,雖度勛德,恐不能感動人主。及度奏河北事,慷慨激切,揚於殿廷,在位者無不聳動。雖武夫貴介,亦有咨嗟出涕者。翌日,以度守
 司徒、揚州大都督府長史,充淮南節度使,進階光祿大夫。



 時硃克融、王廷湊雖受朝廷節鉞,未解深州之圍。度初發太原,與二鎮書,諭以大義。克融解圍而去,廷湊亦退舍。有中使自深州來言之,穆宗甚喜。即日又遣中使往深州取牛元翼,更命度致書與廷湊。度沿路奉詔,中使得度書云:「朝謝後,即歸留務。恐廷湊知度無兵權,即背前約,請度易之。」中使乃進度書草,具奏其事。及度至京師,進退明辯,帝方憂深州之圍,遂授度淮南節度使。



 先是,監軍使劉承偕恃寵凌節度使劉悟,三軍憤發大噪,擒承偕,欲殺之。已殺其二傔,悟救之獲免,而囚承偕。詔遣歸京,悟托以軍情,不時奉詔。至是,宰臣延英奏事,度亦在列。上顧謂度曰:「劉悟拘承偕而不遣,如何處置?」度辭以蕃臣不合議軍國事。上固問之,且曰:「劉悟負我,我以僕射寵之,近又賜絹五百萬疋,不思報功,翻縱軍眾凌辱監軍,我實難奈此事。」度對曰:「承偕在昭義不法,臣盡知之,昨劉悟在行營與臣書,數論其事。是時有中
 使趙弘亮在臣軍,仍持悟書將去,欲自奏,不知奏否?」上曰:「我都不知,悟何不密奏其事,我豈不能處置?」度曰:「劉悟武臣,不知大臣體例。雖然,臣竊以悟縱有密奏,陛下必不能處置。今日事狀如此,臣等面論,陛下猶未能決,悟單辭豈能動聖聽哉?」上曰:「前事勿論,直言此時如何處置?」度曰:「陛下必欲收忠義之心,使天下戎臣為陛下死節,唯有下半紙詔書,言任使不明,致承偕亂法如此,令悟集三軍斬之。如此,則萬方畢命,群盜破膽,天下無
 事矣。茍不能如此,雖與劉悟改官賜絹,臣亦恐於事無益。」上俛首良久,曰:「朕不惜承偕。緣是太后養子,今被囚縶,太后未知,如卿處置未得,可更議其宜。」度與王播等復奏曰:「但配流遠惡處,承偕必得出。」上以為然,承偕果得歸。



 度方受冊司徒,徐州奏節度副使王智興自河北行營率師還,逐節度使崔群,自稱留後。朝廷駭懼,即日宣制,以度守司徒、同平章事,復知政事。乃以宰相王播代度鎮淮南。度與李逢吉素不協。度自太原入朝,而惡
 度者以逢吉善於陰計,足能構度,乃自襄陽召逢吉入朝,為兵部尚書。度既復知政事,而魏弘簡、劉承偕之黨在禁中。逢吉用族子仲言之謀,因醫人鄭注與中尉王守澄交結,內官皆為之助。五月,左神策軍奏告事人李賞稱和王府司馬於方受元稹所使,結客欲刺裴度。詔左僕射韓皋、給事中鄭覃與李逢吉三人鞫於方之獄。未竟,罷元稹為同州刺史,罷度為左僕射,李逢吉代度為宰相。自是,逢吉之黨李仲言、張又新、李續等,內結中
 官,外扇朝士,立朋黨以沮度,時號「八關十六子」,皆交結相關之人數也。而度之醜譽日聞,俄出度為山南西道節度使,不帶平章事。



 長慶四年,襄陽節度使牛元翼卒。其家先在鎮州,朝廷累遣中使取之,王廷湊遷延不遣。至是,聞元翼卒,乃盡屠其家。昭愍皇帝聞之,嗟惋累日,因嘆宰輔非才,致奸臣悖逆如此。翰林學士韋處厚上言曰:



 臣聞汲黯在朝,淮南不敢謀叛;干木處魏,諸侯不敢加兵。王霸之理,皆以一士而止百萬之師,以一賢而
 制千里之難。臣伏以裴度勛高中夏,聲播外夷,廷湊、克融皆憚其用,吐蕃、回鶻悉服其名。今若置之巖廊,委其參決,西夷北虜,未測中華;河北山東,必稟廟算。況幽、鎮未靜,尤資重臣。管仲曰:「人離而聽之則愚,合而聽之則聖。」理亂之本,非有他術,順人則理,違人則亂。伏承陛下當食嘆息,恨無蕭、曹。今有一裴度尚不留驅使,此馮生所以感悟漢文,雲雖有廉頗、李牧不能用也。



 夫御宰相,當委之信之,親之禮之。如於事不效,於國無勞,則置之
 散僚,黜之遠郡。如此,則在位者不敢不勵,將進者不敢茍求。陛下存終始之分,但不永棄,則君臣之厚也。今進皆負四海責望,退不失六部尚書,不肖者無因而勸。臣與李逢吉素無讎嫌,臣嘗被裴度因事貶黜。今之所陳,上答聖明,下達君議,披肝感激,伏地涕流。伏望鑒臣愛君,矜臣體國,則天下幸甚。



 昭愍愕然省悟,見度奏狀不帶平章事,謂處厚曰:「度曾為宰相,何無平章事?」處厚因奏:「為逢吉所擠,度自僕射出鎮興元,遂於舊使銜中減
 落。」帝曰:「何至是也。」翌日下制,復兼同平章事。



 然逢吉之黨,巧為毀沮,恐度復用。有陳留人武昭者,性果敢而辯舌。度之討淮西也,昭求進於軍門,乃令入蔡州說吳元濟。元濟臨之以兵,昭氣色自若,善待而還。度以為可用,署之軍職,隨度鎮太原,奏授石州刺史。罷郡,除袁王府長史。昭既在散位,心微悒鬱,而有怨逢吉之言。而奸邪之黨,使衛尉卿劉遵古從人安再榮告事,言武昭欲謀害李逢吉。獄具,而武昭死,蓋欲訐度舊事以污之也。然
 士君子公論,皆佑度而罪逢吉。天子漸明其端,每中使過興元,必傳密旨撫諭,且有徵還之約。



 寶歷元年十一月,度疏請入覲京師。明年正月,度至,帝禮遇隆厚,數日,宣制復知政事。而逢吉黨有左拾遺張權輿者,尤出死力。度自興元請入朝也,權輿上疏曰:「度名應圖讖,宅據岡原,不召自來,其心可見。」先是奸黨忌度,作謠辭云:「非衣小兒坦其腹,天上有口被驅逐。」「天口」言度嘗平吳元濟也。又帝城東西,橫亙六崗,合《易象乾》卦之數。度平樂
 里第,偶當第五崗,故權輿取為語辭。昭愍雖少年,深明其誣謗,獎度之意不衰,奸邪無能措言。



 時昭愍欲行幸洛陽,宰相李逢吉及兩省諫官,累疏論列,帝正色曰:「朕去意已定。其從官宮人,悉令自備糗糧,不勞百姓供饋。」逢吉頓首言曰:「東都千里而近,宮闕具存,以時巡游,固亦常典。但以法駕一動,事須備儀,千乘萬騎,不可減省。縱不費用絕廣,亦須豐儉得宜,豈可自備糗糧,頓失大體?今干戈未甚戢,邊鄙未甚寧,恐人心動搖,伏乞稍回
 宸慮。」帝不聽,令度支員外郎盧貞往東都已來,檢計行宮及洛陽大內。朝廷方懷憂恐,會度自興元來,因延英奏事,帝語及巡幸。度曰:「國家營創兩都,蓋備巡幸。然自艱難已來,此事遂絕。東都宮闕及六軍營壘、百司廨署,悉多荒廢。陛下必欲行幸,亦須稍稍修葺。一年半歲後,方可議行。」帝曰:「群臣意不及此,但云不合去。若如卿奏,不行亦得止後期。」旋又硃克融、史憲誠各請以丁匠五千,助修東都,帝遂停東幸。



 幽州硃克融執留賜春衣使
 楊文端,奏稱衣段疏薄;又奏今歲三軍春衣不足,擬於度支請給一季春衣,約三十萬端匹;又請助丁匠五千修東都。上憂其不遜,問宰臣曰:「克融所奏,如何處分?我欲遣一重臣往宣慰,便索春衣使,可乎?」度對曰:「克融家本兇族,無故又行凌悖,必將滅亡,陛下不足為慮。譬如一豺虎,於山林間自吼自躍,但不以為事,則自無能為。此賊只敢於巢穴中無禮,動即不得。今亦不須遣使宣慰,亦不要索所留敕使,但更緩旬日已來,與一詔云:『聞
 中官到彼稍失去就,待到,我當有處分。所賜卿春衣,有司製造不謹,我甚要知之,已令科處。』所請丁匠五千人及兵馬赴東都,固是虛語。臣料賊中,必出不得,今欲直挫其奸意,即報云:『卿所請丁匠修宮闕,可速遣來,已敕魏博等道,令所在排比供擬。』料得此詔,必章惶失計。若未能如此,猶示含容,則報云:『東都宮闕,所要修葺,事在有司,不假卿遣丁匠遠來。又所言三軍春衣,自是本道常事。比來朝廷或有事賜與,皆緣徵發,須是優恩,若尋
 常則無此例。我固不惜三二十萬端疋,只是事體不可獨與範陽。卿宜知悉。』只如此處分即得,陛下更不要介意。」上從之,遂進詔章,至皆如度所料。不旬日,幽州殺克融並其二子。



 時帝童年驕縱,倦接群臣。度從容奏曰:「比者,陛下每月約六七度坐朝。天下人心,無不知陛下躬親庶政,乃至河北賊臣遠聞,亦皆聳聽。自兩月已來,入閣開延英稍稀,或恐大段公事須稟睿謀者,有所擁滯。伏冀陛下乘涼數坐,以廣延問。伏以頤養聖躬,在於順
 適時候。若飲食有節,寢興有常,四體唯和,萬壽可保。道書云:『春夏早起,取雞鳴時;秋冬晏起,取日出時。』蓋在陽則欲及陰涼,在陰則欲及溫暖。今陛下憂勤庶政,親覽萬機,每御延英,召臣等奏對,方屬盛夏,宜在清晨。如至巳午之間,即當炎赫之際,雖日昃忘食,不憚其勞,仰瞻扆旒,亦似煩熱。臣等已曾陳論,切望聽納。」自後,視事稍頻。



 未幾,兼領度支。屬盜起禁闈,宮車晏駕,度與中貴人密謀,誅劉克明等,迎江王立為天子。以功加門下侍郎、
 集賢殿大學士、太清宮使,餘如故。以贊導之勛,進階特進。



 時滄景節度使李全略死,其子同捷竊弄兵柄,以求繼襲。度請行誅伐,逾年而同捷誅。因拜疏上陳調兵食非宰相事,請歸諸有司。詔從之。賜實封三百戶。



 度年高多病,上疏懇辭機務,恩禮彌厚。文宗遣御醫診視,日令中使撫問。四年六月,詔曰:



 昔漢以孔光降置幾之詔,晉以鄭沖申奉冊之命。雖優隆耆德,顯重元臣,而議政不及於咨詢,用禮止在於安逸。朕勤求至理,所寶唯賢,顧
 諟舊勞,敢不加敬。由是委宰制於大政,釋參決於繁務。時因聽斷,誠望弼諧,遷秩上公,式是殊寵。特進、守司徒、兼門下侍郎、同中書門下平章事,充集賢殿大學士、上柱國、晉國公、食邑三千戶、食實封三百戶裴度,稟河岳之英靈,受乾坤之間氣;珪璋特達,城府洞開。外茂九功,內苞一德。器為社稷之鎮,才實邦國之楨。故能祗事累朝,宣融景化。



 在憲宗時,掃滌區宇,爾則有出車殄寇之勛;在穆宗時,混同文軌,爾則有參戎入輔之績;在敬宗
 時,阜康兆庶,爾則有活國庇人之勤。迨弼朕躬,總齊方夏,爾則有吊伐底寧之力。皆不遺廟算,布在簡編,功利及人,不可悉數。而朝論益重,我心實知。方用皋陶之謨,適值留侯之疾,瀝懇牢讓,備列奏章,塞詔上言,動形顏色。果聞勿藥之喜,更俟調鼎之功,而體力未和,音容尚阻。不有優崇之命,孰彰寵待之恩?宜其協贊機衡,弘敷教典;論道而儀刑卿士,宣德而鎮撫華夷。嗇養精神,保綏福履,為國元老,毗予一人。可司徒、平章軍國重事,待
 疾損日,每三日、五日一度入中書。散官勛封實封如故。仍備禮冊命。



 度表辭曰:「伏以公臺崇禮,典冊盛儀,庸臣當之,實謂忝越。況累承寵命,亦為便蕃,前後三度,已行此禮。令臣猶參樞近,竊懼無以弼諧,重此勞煩,有靦面目。伏乞天恩且課臣效官,責臣實事,冊命之儀,特賜停罷。則素餐高位,空負恥於中心;弁冕輕車,免譏誚於眾口。」優詔從之。九月,加守司徒、兼侍中、襄州刺史,充山南東道節度觀察、臨漢監牧等使。



 度素稱堅正,事上不回,
 故累為奸邪所排,幾至顛沛。及晚節,稍浮沉以避禍。初,度支鹽鐵使王播,廣事進奉以希寵,度亦掇拾羨餘以效播,士君子少之。復引韋厚叔、南卓為補闕拾遺,俾彌縫結納,為目安之計。而後進宰相李宗閔、牛僧孺等不悅其所為,故因度謝病罷相位,復出為襄陽節度。



 初,元和十四年,於襄陽置臨漢監牧。廢百姓田四百頃,其牧馬三千二百餘匹。度以牧馬數少,虛廢民田,奏罷之,除其使名。八年三月,以本官判東都尚書省事,充東都留
 守。九年十月,進位中書令。十一月,誅李訓、王涯、賈餗、舒元輿等四宰相,其親屬門人從坐者數十百人;下獄訊劾,欲加流竄。度上疏理之,全活者數十家。



 自是,中官用事,衣冠道喪。度以年及懸輿,王綱版蕩,不復以出處為意。東都立第於集賢里,築山穿池,竹木叢萃,有風亭水榭,梯橋架閣,島嶼回環,極都城之勝概。又於午橋創別墅,花木萬株;中起涼臺暑館,名曰「綠野堂」。引甘水貫其中,釃引脈分,映帶左右。度視事之隙,與詩人白居易、劉
 禹錫酣宴終日,高歌放言,以詩酒琴書自樂,當時名士,皆從之游。每有人士自都還京,文宗必先問之曰:「卿見裴度否?」



 上以其足疾,不便朝謁,而年未甚衰,開成二年五月,復以本官兼太原尹、北都留守、河東節度使。詔出,度累表固辭老疾,不願更典兵權。優詔不允。文宗遣吏部郎中盧弘往東都宣旨曰:「卿雖多病,年未甚老,為朕臥鎮北門可也。」促令上路,度不獲已,之任。三年冬,病甚,乞還東都養病。四年正月,詔許還京,拜中書令。以疾未
 任朝謝。詔曰:「司徒、中書令度,綽有大勛,累居臺鼎。今以疾恙,未任謝上,其本官俸料,宜自計日支給。」又遣國醫就第診視。



 屬上巳曲江賜宴,群臣賦詩,度以疾不能赴。文宗遣中使賜度詩曰:「注想待元老,識君恨不早。我家柱石衰,憂來學丘禱。」仍賜御札曰:「朕詩集中欲得見卿唱和詩,故令示此。卿疾恙未痊,固無心力,但異日進來。春時俗說難於將攝,勉加調護,速就和平。千百胸懷,不具一二。藥物所須,無憚奏請之煩也。」御札及門,而度已
 薨,四年三月四日也。上聞之,震悼久之,重令繕寫,置之靈座。時年七十五,冊贈太傅,輟朝四日,賵賻加等。詔京兆尹鄭復監護喪事,所須皆官給。



 上怪度無遺表。中使問之,家人進其稿草。其旨以未定儲貳為憂,言不及家事。



 度始自書生以辭策中科選,數年之間,翔泳清切。逢時艱否,而能奮命決策,橫身討賊,為中興宗臣。當元和、長慶間,亂臣賊子,蓄銳喪氣,憚度之威稜。度狀貌不逾中人,而風彩俊爽,占對雄辯,觀聽者為之聳然。時有奉
 使絕域者,四夷君長必問度之年齡幾何,狀貌孰似,天子用否?其威名播於憬俗,為華夷畏服也如此。時威望德業,侔於郭子儀,出入中外,以身系國之安危、時之輕重者二十年。凡命將相,無賢不肖,皆推度為首,其為士君子愛重也如此。雖江左王導、謝安坐鎮雅俗,而訏謨方略,度又過之。



 有子五人:識、譔、讓、諗、議。



 識以廕授官,累遷至通議大夫、檢校右散騎常侍、壽州刺史、本州團練使、上柱國、襲晉國公、食邑三千戶、實封一百五十戶,賜
 紫金魚袋。大中初,改潭州刺史、御史中丞,充河南都團練觀察使。八年,加檢校戶部尚書、鳳翔尹、鳳翔隴右節度使。十一年,本官移許州刺史、忠武軍節度、陳許觀察等使。



 譔,長慶元年登進士第。



 讓初任京光府參軍,太和中度鎮襄陽,奏乞讓從行。



 諗,大中五年,自大中大夫檢校右散騎常侍、御史大夫、宣州刺史、宣歙觀察使、上柱國、河東男、食邑三百戶,賜紫金魚袋,入朝權知刑部侍郎。兄弟並列方鎮,時人榮之。



 史臣曰:德宗懲建中之難,姑息籓臣,貞元季年,威令衰削。章武皇帝志據宿憤,廷訪嘉猷。始得杜邠公,用高崇文誅劉闢。中得武丞相,運籌訓戎,贊成睿斷。終得裴晉公,耀武伸威,竟殄兩河宿盜。雄哉,章武之果斷也!晉公以書生素業,致位臺衡,逢進遘屯,扼腕兇醜,誓以身徇,不亦壯乎!夫人臣事君,唯忠與義。大則以訏謨排禍難,小則以讜正匡過失,內不慮身計,外不恤人言,古人所難也。晉公能之,誠社稷之良臣,股肱之賢相;元和中興
 之力,公胡讓焉!昔仲尼嘆周室陵遲,齊桓霸翼,而有微管之論。嘗承宗、師道之濟惡也,奸人遍四海,刺客滿京師。乃至關吏禁兵,附賊陰計,議臣言未出口,刃已揕胸。茍非死義之臣,孰肯橫身冒難,以輔天子者?茍裴令不用,元和之世則時運未可知也。臣所以明左衽之嘆,宣聖獎賢之深。



 贊曰:晉公伐叛,以身犯難。用之則治,舍之則亂。公去巖廊,復失冀方。穎、植之謀,信為不臧。



\end{pinyinscope}