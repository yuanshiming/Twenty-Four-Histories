\article{卷九十六}

\begin{pinyinscope}

 ○魏元忠韋安石子陟斌斌子況從父兄子抗從祖兄子巨源趙
 彥昭附
 蕭至
 忠宗楚客紀處訥附



 魏元忠,宋州宋城人也。本名真宰,以避則天母號改焉。初,為太學生,志氣倜儻,不以舉薦為意,累年不調。時有
 左史盩厔人江融撰《九州設險圖》,備載古今用兵成敗之事,元忠就傳其術。儀鳳中,吐蕃頻犯塞,元忠赴洛陽上封事,言命將用兵之工拙,曰:



 臣聞理天下之柄,二事焉,文與武也。然則文武之道,雖有二門,至於制勝禦人,其歸一揆。方今王略遐宣,皇威遠振,建禮樂而陶士庶,訓軍旅而懾生靈。然論武者以弓馬為先,而不稽之以權略;談文者以篇章為首,而不問之以經綸。而奔競相因,遂成浮俗。臣嘗讀魏、晉史,每鄙何晏、王衍終日談空。
 近觀齊、梁書,才士亦復不少,並何益於理亂哉?從此而言,則陸士衡著《辨亡論》,而不救河橋之敗,養由基射能穿札,而不止鄢陵之奔,斷可知矣。昔趙岐撰御寇之論,山濤陳用兵之本,皆坐運帷幄,暗合孫、吳。宣尼稱「有德者必有言,仁者必有勇」,則何平叔、王夷甫豈得同日而言載!



 臣聞才生於代,代實須才,何代而不生才,何才而不生代。故物有不求,未有無物之歲;士有不用,未有無士之時。夫有志之士,在富貴之與貧賤,皆思立於功名,
 冀傳芳於竹帛。故班超投筆而嘆,祖逖擊楫而誓,此皆有其才而申其用矣。且知己難逢,英哲罕遇,士之懷琬琰以就埃塵,抱棟梁而困溝壑者,則悠悠之流,直睹此士之貧賤,安知此士之方略哉。故漢拜韓信,舉軍驚笑;蜀用魏延,群臣觖望。嗟乎,富貴者易為善,貧賤者難為功,至於此也!



 亦有位處立功之際,而不展其志略,身為時主所知,竟不能盡其才用,則貧賤之士焉足道哉!漢文帝時,魏尚、李廣並身任邊將,位為郡守。文帝不知魏
 尚之賢而囚之,不知李廣之才而不能用之。常嘆李廣恨生不逢時,令當高祖日,萬戶侯豈足道哉。夫以李廣才氣,天下無雙,匈奴畏之,號為「飛將」,爾時胡騎憑凌,足伸其用。文帝不能大任,反嘆其生不逢時。近不知魏尚、李廣之賢,而乃遠想廉頗、李牧。故馮唐曰,雖有頗、牧而不能用,近之矣。從此言之,疏斥賈誼,復何怪哉。此則身為時主所知,竟不能盡其才用。晉羊祜獻計平吳,賈充、荀勖沮其策,祜嘆曰:「天下不如意恆十居七八。」緣荀、賈
 不同,竟不大舉。此則位處立功之際,而不得展其志略。而布衣韋帶之人,懷一奇,抱一策,上書闕下,朝進而望夕召,何可得哉。



 臣請歷訪內外文武職事五品已上,得不有智計如羊祜、武藝如李廣,在用與不用之間,不得騁其才略。伏願降寬大之詔,使各言其志。無令汲黯直氣,臥死於淮陽。仲舒大才,位屈於諸候相。



 又曰:



 臣聞帝王之道,務崇經略。經略之術,必仗英奇。自國家良將,可得言矣。李靖破突厥,侯君集滅高昌,蘇定方開西域,李
 勣平遼東,雖奉國威靈,亦其才力所致。古語有之,「人無常俗,政有理亂,兵無強弱,將有能否」。由此觀之,安邊境,立功名,在於良將也。故趙充國征先零,馮子明討南羌,皆計不空施,機不虛發,則良將立功之驗也。然兵革之用,王者大事,存亡所系。若任得其才,則摧兇而扼暴。茍非其任,則敗國而殄人。北齊段孝玄云:「持大兵者,如擎盤水。傾在俯仰間,一致蹉跌,求止豈得哉!」從此而言,周亞夫堅壁以挫吳、楚,司馬懿閉營而困葛亮,俱為上策。
 此皆不戰而卻敵,全軍以制勝。是知大將臨戎,以智為本。漢高之英雄大度,尚曰「吾寧鬥智」;魏武之綱神冠絕,猶依法孫、吳。假有項籍之氣,袁紹之基,而皆泯智任情,終以破滅,何況復出其下哉!



 且上智下愚,明暗異等,多算少謀,眾寡殊科。故魏用柏直以拒漢,韓信輕為豎子;燕任慕容評以抗秦,王猛謂之奴才。即柏直、慕容評智勇俱亡者也。夫中材之人,素無智略,一旦居元帥之任,而意氣軒昂,自謂當其鋒者無不摧碎,豈知戎昭果毅、
 敦《詩》說《禮》之事乎!故李信求以二十萬眾獨舉鄢郢,其後果辱秦軍;樊噲願得十萬眾橫行匈奴,登時見折季布,皆其事也。



 當今朝廷用人,類取將門子弟,亦有死事之家而蒙抽擢者。此等本非幹略見知,雖竭力盡誠,亦不免於傾敗,若之何使當閫外之任哉?後漢馬賢討西羌,皇甫規陳其必敗;宋文帝使王玄謨收復河南,沈慶之懸知不克。謝玄以書生之姿,拒苻堅天下之眾,郗超明其必勝;桓溫提數萬之兵,萬里而襲成都,劉真長期
 於決取。雖時有今古,人事皆可推之,取驗大體,觀其銳志與識略耳。明者隋分而察,成敗之形,昭然自露。京房有言,「後之視今,亦猶今之視古。」則昔賢之與今哲,意況何殊。當事機之際也。皆隨時而立功,豈復取賢於往代,待才於未來也?即論知與不知,用與不用。夫建功者,言其所濟,不言所起;言其所能,不言所藉。若陳湯、呂蒙、馬隆、孟觀,並出自貧賤,勛濟甚高,未聞其家代為將帥。董仲舒曰:「為政之用,譬之琴瑟,不調甚者,必解弦而更張
 之,乃可鼓也。」故陰陽不和,擢士為相;蠻夷不龔,拔卒為將,即更張之義也。以四海之廣,億兆之眾,其中豈無卓越奇絕之士?臣恐未之思也,夫何遠之有。



 又曰:



 臣聞賞者禮之基,罰者刑之本。故禮崇謀夫竭其能,賞厚義士輕其死,刑正君子勖其心,罰重小人懲其過。然則賞罰者,軍國之綱紀,政教之藥石。綱紀舉而眾務自理,藥石行而文武用命。彼吐蕃蟻結蜂聚,本非勍敵,薛仁貴、郭待封受閫外之寄,奉命專征,不能激勵熊羆,乘機掃撲。
 敗軍之後,又不能轉禍為福,因事立功,遂乃棄甲喪師,脫身而走。幸逢寬政,罪止削除,國家綱漏吞舟,何以過此。天皇遲念舊恩,收其後效,當今朝廷所少,豈此一二人乎?且賞不勸謂之止善,罰不懲謂之縱惡。仁貴自宣力海東,功無尺寸,坐玩金帛,瀆貨無厭,今又不誅,縱惡更甚。臣以疏賤,乾非其事,豈欲間天皇之君臣,生厚薄於仁貴?直以刑賞一虧,百年不復,區區所懷,實在於此。



 古人云:「國無賞罰,雖堯、舜不能為化。」今罰不能行,賞亦
 難信,故人間議者皆言,「近日征行,虛有賞格而無其事。」良由中才之人不識大體,恐賞賜勛庸,傾竭倉庫,留意錐刀,將此益國。徇目前之近利,忘經久之遠圖,所謂錯之毫厘,失之千里者也。且黔首雖微,不可欺以得志,瞻望恩澤,必因事而生心。既有所因,須應之以實,豈得懸不信之令,設虛賞之科?比者師出無功,未必不由於此。文子曰:「同言而信,信在言前;同令而行,誠在令外。」故商君移木以表信,曹公割發以明法,豈禮也哉,有由然也。
 自蘇定方定遼東,李勣破平壤,賞絕不行,勛仍淹滯,數年紛紜,真偽相雜,縱加沙汰,未至澄清。臣以吏不奉法,慢自京師,偽勛所由,主司之過。其則不遠,近在尚書省中,不聞斬一臺郎,戮一令史,使天下知聞,天皇何能照遠而不照近哉!神州化首,萬國共尊,文昌政本,四方是則,軌物宣風,理亂攸在。臣是以披露不已,冒死盡言。



 且明鏡所以照形,往事所以知今,臣識不稽古,請以近事言之。貞觀年中,萬年縣尉司馬玄景舞文飾智,以邀乾
 沒,太宗審其奸詐,棄之都市。及征高麗也,總管張君乂擊賊不進,斬之旗下。臣以偽勛之罪,多於玄景;仁貴等敗,重於君乂。向使早誅薛仁貴、郭待封,則自餘諸將,豈敢失利於後哉?韓子云:「慈父多敗子,嚴家無格虜。」此言雖小,可以喻大。公孫弘有言:「人主病不廣大,人臣病不節儉。」臣恐天皇病之於不廣大,過在於慈父,斯亦日月之一蝕也。又今之將吏,率多貪暴,所務唯狗馬,所求唯財物,無趙奢、吳起散金養士之風,縱使行軍,悉是此屬。
 臣恐吐蕃之平,未可旦夕望也。



 帝甚嘆異之,授秘書省正字,令直中書省,仗內供奉。尋除監察御史。



 文明年,遷殿中侍御史。其年,徐敬業據揚州作亂,左玉鈐衛大將軍李孝逸督軍討之,則天詔元忠監其軍事。孝逸至臨淮,而偏將雷仁智為敬業先鋒所敗,敬業又攻陷潤州,回兵以拒孝逸。孝逸懼其鋒,按甲不敢進。元忠謂孝逸曰:「朝廷以公王室懿親,故委以閫外之事,天下安危,實資一決。且海內承平日久,忽聞狂狡,莫不注心傾耳,以
 俟其誅。今大軍留而不進,則解遠近之望,萬一朝廷更命他將代公,其將何辭以逃逗撓之罪?幸速進兵以立大效,不然,則禍難至矣。」孝逸然其言,乃部勒士卒以圖進討。



 時敬業屯於下阿溪,敬業弟敬猷率偏師以逼淮陰。元忠請先擊敬猷,諸將咸曰:「不如先攻敬業,敬業敗,則敬猷不戰而擒矣。若擊敬猷,則敬業引兵救之,是腹背受敵也。」元忠曰:「不然,賊之勁兵精卒,盡在下阿,蟻聚而來,利在一決,萬一失捷,則大事云矣。敬猷本出博徒,
 不習戰鬥,其眾寡弱,人情易搖,大軍臨之,其勢必克。既克敬猷,我之乘勝而進。彼若引救淮陰,計程則不及,又恐我之進掩江都,必邀我於中路。彼則勞倦,我則以逸待之,破之必矣。譬之逐獸,弱者先擒,豈可舍必擒之弱獸,趨難敵之強兵?恐未可也。」孝逸從之,乃引兵擊敬猷,一戰而破之,敬猷脫身而遁。孝逸乃進軍,與敬業隔溪相拒。前軍總管蘇孝祥為賊所破,孝逸又懼,欲引退。初,敬業至下阿,有流星墜其宮,及是,有群烏飛噪於陣
 上,元忠曰:「驗此,即賊敗之兆也。風順荻乾,火攻之利。」固請決戰,乃平敬業。元忠以功擢司刑正,稍遷洛陽令。



 尋陷周興獄,詣市將刑,則天以元忠有討平敬業功,特免死配流貴州。時承敕者將至市,先令傳呼,監刑者遽釋元忠令起,元忠曰:「未知敕虛實,豈可造次。」徐待宣敕,然始起謝,觀者咸嘆其臨刑而神色不撓。聖歷元年,召授侍御史,擢拜御史中丞。又為來俊臣、侯思止所陷,再被流於嶺表。復還,授御史中丞。元忠前後三被流,於時人多
 稱其無罪。則天嘗謂曰:「卿累負謗鑠,何也?」對曰:「臣猶鹿也,羅織之徒,有如獵者,茍須臣肉作羹耳。此輩殺臣以求達,臣復何辜。」



 聖歷二年,擢拜鳳閣侍郎、同鳳閣鸞臺平章事,檢校並州長史。未幾,加銀青光祿大夫,遷左肅政臺御史大夫,兼檢校洛州長史。政號清嚴。長安中,相王為並州元帥,元忠為副。時奉宸令張易之嘗縱其家奴凌暴百姓,元忠笞殺之,權豪莫不敬憚。時突厥與吐蕃數犯塞,元忠皆為大總管拒之。元忠在軍,唯持重自
 守,竟無所克獲,然亦未嘗敗失。



 中宗在春宮時,元忠檢校太子左庶子。時張易之、昌宗權寵日盛,傾朝附之。元忠嘗奏則天曰:「臣承先帝顧眄,受陛下厚恩,不徇忠死節,使小人得在君側,臣之罪也。」則天不悅。易之、昌宗由是含怒。因則天不豫,乃譖元忠與司禮丞高戩潛謀曰:「主上老矣,吾屬當挾太子而令天下。」則天惑其言,乃下元忠詔獄,召太子、相王及諸宰相,令昌宗與元忠等殿前參對,反復不決。昌宗又引鳳閣舍人張說令執證元
 忠。說初偽許之,及則天召說驗問,說確稱元忠實無此語。則天乃悟元忠被誣,然以昌宗之故,特貶授端州高要尉。



 中宗即位,其日驛召元忠,授衛尉卿、同中書門下三品。旬日,又遷兵部尚書,知政事如故。尋進拜侍中,兼檢校兵部尚書。時則天崩,中宗居諒暗,多不視事,軍國大政,獨委元忠者數日。未幾,遷中書令,加授光祿大夫,累封齊國公,監修國史。神龍二年,元忠與武三思、祝欽明、徐彥伯、柳沖、韋承慶、崔融、岑羲、徐堅等撰《則天皇後
 實錄》二十卷。編次文集一百二十卷奏之。中宗稱善,賜元忠物千段,仍封其子衛王府諮議參軍升為任城縣男。時元忠特承寵榮,當朝用事。初元忠作相於則天朝,議者以為公清。至是再居政事,天下莫不延首傾屬,冀有所弘益。元忠乃親附權豪,抑棄寒俊,竟不能賞善罰惡,勉修時政,議者以此少之。四年秋,代唐璟為尚書右僕射,兼中書令,仍知兵部尚書事,監修國史。未幾,元忠請歸鄉拜掃,特賜錦袍一領、銀千兩,並給千騎四人,充
 其左右,手敕曰:「衣錦晝游,在乎茲日;散金敷惠,諒屬斯辰。」元忠至鄉里,竟自藏其銀,無所賑施。及還,帝又幸白馬寺以迎勞之,其恩遇如此。



 是時,安樂公主嘗私請廢節愍太子,立己為皇太女。中宗以問元忠,元忠固稱不可,乃止。尋遷左僕射,餘並如故。元忠又嫉武三思專權用事,心常憤嘆,思欲誅之。三年秋,節愍太子起兵誅三思,元忠及左羽林大將軍李多祚等皆潛預其事。太子既斬三思,又率兵詣闕,將請廢韋後為庶人,遇元忠子
 太僕少卿升於永守門,協令從己。太子兵至玄武樓下,多祚等猶豫不戰,元忠又持兩端,由是不克,升為亂兵所殺。中宗以元忠有平寇之功,又素為高宗、天后所禮遇,竟不以升為累,委任如初。



 是時,三思之黨兵部尚書宗楚客與侍中紀處訥等又執證元忠及升,雲素與節愍太子同謀構逆,請夷其三族,中宗不許。元忠懼不自安,上表固請致仕。手制聽解左僕射,以特進、齊國公致仕於家,仍朝朔望。楚客等又引右衛郎將姚庭筠為御
 史中丞,令劾奏元忠,由是貶渠州員外司馬。侍中楊再思、中書令李嶠皆依楚客之旨,以致元忠之罪,唯中書侍郎蕭至忠正議云當從寬宥。楚客大怒,又遣給事中冉祖雍與楊再思奏言:「元忠既緣犯逆,不合更授內地官。」遂左遷思州務川尉。頃之,楚客又令御史袁守一奏言:「則天昔在三陽宮不豫,內史狄仁傑奏請陛下監國,元忠密進狀云不可。據此,則知元忠懷逆日久,伏請加以嚴誅。」中宗謂楊再思等曰:「以朕思之,此是守一大錯。
 人臣事主,必在一心,豈有主上少有不安,即請太子知事?乃是狄仁傑樹私惠,未見元忠有失。守一假借前事羅織元忠,豈是道理。」楚客等遂止。元忠行至涪陵而卒,年七十餘。景龍四年,追贈尚書左僕射、齊國公、本州刺史,仍令所司給靈輿送至鄉里。睿宗即位,制令陪葬定陵。景雲三年,又降制曰:「故左僕射、齊國公魏元忠,代協人望,時稱國良。歷事三朝,俱展誠效。晚年遷謫,頗非其罪。宜特還其子著作郎晃實封一百戶。」開元六年,謚曰
 貞。二子升、晃。



 韋安石,京兆萬年人,周大司空、鄖國公孝寬曾孫也。祖津,大業末為民部侍郎。煬帝之幸江都,敕津與段達、元文都等於洛陽留守,仍檢校民部尚書事。李密逼東都,津拒戰於上東門外。兵敗,為密所囚,及王世充殺文都等,津獨免其難。密敗,歸東都,世充僭號,深被委遇。及洛陽平,高祖與津有舊,徵授諫議大夫,檢校黃門侍郎。出為陵州刺史,卒。父琬,成州刺史。叔琨,戶部侍郎。琨弟璲,
 倉部員外。



 安石應明經舉,累授乾封尉,蘇良嗣甚禮之。永昌元年,三遷雍州司兵,良嗣時為文昌左相,謂安石曰:「大材須大用,何為徒勞於州縣也。」特薦於則天,擢拜膳部員外郎、永昌令、並州司馬。則天手制勞之曰:「聞卿在彼,庶事存心,善政表於能官,仁明彰於鎮撫。如此稱職,深慰朕懷。」俄拜並州刺史,又歷德、鄭二州刺史。安石性持重,少言笑,為政清嚴,所在人吏咸畏憚之。久視年,遷文昌右丞,尋拜鸞臺侍郎、同鳳閣鸞臺平章事,兼
 太子左庶子。長安三年,為神都留守,兼判天官、秋官二尚書事。後與崔神慶等同為侍讀,尋知納言事。是歲,又加檢校中臺左丞,兼太子左庶子、鳳閣鸞臺三品如故。



 時張易之兄弟及武三思皆恃寵用權,安石數折辱之,甚為易之等所忌。嘗於內殿賜宴,易之引蜀商宋霸子等數人於前博戲。安石疏奏曰:「蜀商等賤類,不合預登此筵。」因顧左右令逐出之,座者皆為失色。則天以安石辭直,深慰勉之。時鳳閣侍郎陸元方在座,退而告人曰:「此
 真宰相,非吾等所及也。」則天嘗幸興泰宮,欲就捷路,安石奏曰:「千金之子,且有垂堂之誡,萬乘之尊,不宜輕乘危險。此路板築初成,無自然之固,鑾駕經之,臣等敢不請罪。」則天登時為之回輦。安石俄又舉奏易之等罪狀,初有敕付安石及夏官尚書唐休璟推問,未竟而事變。四年,出為揚州大都督府長史。



 神龍初,徵拜刑部尚書。是歲,又遷吏部尚書,復知政事。俄代張柬之為中書令,封鄖國公,以嘗為宮僚,賜實封三百戶,又兼相王府長
 史。俄轉戶部尚書,復為侍中,監修國史。中宗與庶人嘗因正月十五日夜幸其第,賜賚不可勝數。又中宗嘗幸安樂公主城西池館,公主具舟楫,請御樓船,安石諫曰:「御輕舟,乘不測,臣恐非帝王之事。」乃止。



 睿宗踐祚,拜太子少保,改封郇國公。俄又歷侍中、中書令。景雲二年,加開府儀同三司。時太平公主與竇懷貞等潛有異圖,將引安石預其事,公主屢使子婿唐晙邀安石至宅,安石竟拒而不往。睿宗嘗密召安石,謂曰:「聞朝廷傾心東宮,
 卿何不察也?」安石對曰:「陛下何得亡國之言,此必太平之計。太子有大功於社稷,仁明孝友,天下所稱,願陛下無信讒言以致惑也。」睿宗矍然曰:「朕知之矣,卿勿言也。」太平於簾中竊聽之,乃構飛語,欲令鞫之,賴郭元振保護獲免。俄而遷尚書左僕射,兼太子賓客,依舊同中書門下三品,雖假以崇寵,實去其權。其冬,罷知政事,拜特進,充東都留守。太常主簿李元澄,即安石之子婿,其妻病死,安石夫人薛氏疑元澄先所幸婢厭殺之。其婢久
 已轉嫁,薛氏使人捕而捶之致死。由是為御史中丞楊茂謙所劾,出為蒲州刺史。無幾,轉青州刺史。



 安石初在蒲州時,太常卿姜皎有所請托,安石拒之,皎大怒。開元二年,皎弟晦為御史中丞,以安石等作相時,同受中宗遺制,宗楚客、韋溫削除相王輔政之辭,安石不能正其事,令侍御史洪子輿舉劾之。子輿以事經赦令,固稱不可。監察御史郭震希皎等意,越次奏之,於是下詔曰:「青州刺史韋安石、太子賓客韋嗣立、刑部尚書趙彥昭等,
 往在先朝,曲蒙厚賞,因緣幸會,久在廟堂,朋黨比周,聞於行路。景龍之末,長蛇縱禍,倉卒之間,人神憤怨,未聞舍生取義,直道昌言,遂削太上皇輔政之辭,用韋氏臨朝之策。比常隱忍,復以崇班,將期愧畏,稍懲前惡,而尚款回邪,茍安榮寵。宜從謫官之典,以勵事君之節。安石可沔州別駕,嗣立可岳州別駕,彥昭可袁州別駕,並員外置。」安石既至沔州,晦又奏云:「安石嘗檢校定陵造作,隱官物入己。」敕符下州征贓。安石嘆曰:「此祇應須我死
 耳!」憤激而卒,年六十四。開元十七年,贈蒲州刺史。天寶初,以子貴,追贈開府儀同三司、尚書左僕射、郇國公,謚曰文貞。二子陟、斌,並早知名。



 陟字殷卿,代為關中著姓,人物衣冠,弈世榮盛。安石晚有子,及為並州司馬,始生陟及斌,俱少聰敏,頗異常童。陟自幼風標整峻,獨立不群,安石尤愛之。神龍二年,安石為中書令,陟始十歲,拜溫王府東閣祭酒,加朝散大夫,累遷秘書太堂丞,有文彩,善隸書,辭人、秀士已游其門矣。開元初,丁父憂,居喪
 過禮。自此杜門不出八年,與弟斌相勸勵,探討典墳,不舍晝夜,文華當代,俱有盛名。於時才名之士王維、崔顥、盧象等,常與陟唱和游處。廣平宋公見陟嘆曰:「盛德遺範,盡在是矣。」歷洛陽令,轉吏部郎中。張九齡一代辭宗,為中書令,引陟為中書舍人,與孫逖、梁涉對掌文誥,時人以為美談。



 後為禮部侍郎。陟好接後輩,尤鑒於文,雖辭人後生,靡不諳練。曩者主司取與,皆以一場之善,登其科目,不盡其才。陟先責舊,仍令舉人自通所工詩筆,
 先試一日,知其所長,然後依常式考核,片善無遺,美聲盈路。後為吏部侍郎,常病選人冒名接腳,闕員既少,取士良難,正調者被擠,偽集者冒進。陟剛腸嫉惡,風彩嚴正,選人疑其有瑕,案聲盤詰,無不首伏。每歲皆贖得數百員闕,以待淹滯,常謂所親曰:「使陟知銓衡一二年,則無人可選矣。」



 陟門地豪華,早踐清列,侍兒閹閽,列侍左右者十數,衣書藥食,咸有典掌,而輿馬僮奴,勢侔於王家主第。自以才地人物,坐取三公,頗以簡貴自處,善誘
 納後進,其同列朝要,視之蔑如也。如道義相知,靡隔貴賤,而布衣韋帶之士,恆虛席倒屣以迎之,時人以此稱重。



 李林甫忌之,出為襄陽太守,兼本道採訪使,又改陳留採訪使,復加銀青光祿大夫。天寶中襲封郇國公,以親累貶鐘離太守,重貶義陽太守。尋移河東太守,充本道採訪使。



 十二年入考,在華清宮。右相楊國忠惡其才望,恐踐臺衡,乃引河東人吳象之謂曰:「子能使人告陟乎?吾以子為御史。」象之曰:「能。」乃告陟與御史中丞吉溫
 結托,欲謀陷朝廷,又誘陟侄韋元志證之。陟坐貶為桂州桂嶺尉,未之任,再貶昭州平樂尉。



 會祿山反,陷洛陽,陟愛弟斌為賊所得。國忠欲構陟與賊通應,潛令吏卒伺其所居,欲協之令陟憂死。其士豪人勸陟曰:「昔張燕公竄逐,藏於陳氏,以免危亡。詔命儻來,誰敢申覆?未若輕舟千里,且泛溪洞,候事清徐出,豈不美也!」陟慨然應之曰:「我積信於國朝,非一代也。況素所秉心,無負神理,命之合爾,其敢逃刑?燕公之謀,誠愧厚意,不能從也。」因
 謝遣之,乃堅臥不動。



 經歲餘,潼關失守,肅宗即位於靈武,起為吳郡太守,兼江南東道採訪使。未到郡,肅宗使中官賈游嚴手詔追之。未至鳳翔,會江東永王擅起兵,令陟招諭,除御史大夫,兼江東節度使。陟以季廣琛從永王下江,非其本意,懼罪出奔,未有所適,乃有表請拜廣琛為丹陽太守、兼御史中丞、緣江防禦使,以安反側。因與淮南節度使高適、淮西節度使來瑱等同至安州。陟謂適、瑱曰:「今中原未復,江淮動搖,人心安危,實在茲
 日。若不齊盟質信,以示四方,令知三帥協心,萬里同力,則難以集事矣。」陟推瑱為地主,乃為載書,登壇誓眾曰:「淮西節度使、兼御史大夫瑱,江東節度使、御史大夫陟,淮南節度使、御史大夫適等,銜國威命,各鎮方隅,糾合三垂,翦除兇慝,好惡同之,無有異志。有渝此盟,墜命亡族。皇天後士,祖宗神明,實鑒斯言。」陟等辭旨慷慨,血淚俱下,三軍感激,莫不隕泣。其後江表樹碑以紀忠烈。



 無何,有詔令陟赴行在。陟以廣琛雖承恩命,猶且遲回,恐
 後變生,禍貽於陟,欲往招慰,然後赴征,乃發使上表,懇言其急。陟馳至歷陽,見廣琛,且宣恩旨,勞徠行賞,陟自以私馬數匹賜之,安其疑懼。即日便赴行在,謁見肅宗,肅宗深器之,拜御史大夫。拾遺杜甫上表論房琯有大臣度,真宰相器,聖朝不容,辭旨迂誕,肅宗令崔光遠與陟及憲部尚書顏真卿同訊之。陟因入奏曰:「杜甫所論房琯事,雖被貶黜,不失諫臣大體。」上由此疏之。時朝臣立班多不整肅,至有班頭相吊哭者,乃罷陟御史大夫,
 顏真卿代,授吏部尚書。自後任事寵臣,皆後來初用,望風畏忌,道竟不行。因宗人伐墓柏,坐不能禁,出為絳州刺史。乾元二年,入為太常卿。呂諲再入相,薦為禮部尚書、東京留守,判尚書省事,兼東京畿觀察處置等使。逆賊史思明寇逼河洛,副元帥李光弼議守河陽,令陟率東京官屬入關回避,乃令兵守陜州。有詔遷吏部尚書,留守如故,令止於永樂,不許至京,候光弼收復河洛,令陟依前居守。



 陟早有臺輔之望,間被李林甫、楊國忠所
 擠。及中原兵起,天下事殷,陟常自謂負經緯之器,遭後生騰謗,明主見疑,常鬱鬱不得志,乃嘆曰:「吾道窮於此乎,有志不伸,得非天命乎!」因遘疾,上元元年八月,卒於虢州,時年六十五,贈荊州大都督。永泰元年,詔曰:「竭忠之臣,歿不廢命,奉上之節,行固無私,言念飾終,抑惟恆典。故金紫光祿大夫、吏部尚書、兼御史大夫、充東京留守、兼判留司尚書省事、東京畿觀察處置使、上柱國、郇國公韋陟,敦敏直方,端嚴峻整,弘敷典禮,表正人倫,學
 冠通儒,文含大雅。頃者詢謨舊德,保厘成周,眷彼郊圻,資其慎固。而兇胡殘醜,密邇河洛,命居陜、虢,時俟翦除。才加喉舌之榮,遽嬰霜露之疾。方期克享眉壽,冀其有瘳,奄此殂歿,良深震悼。升車而復,以申三禭之恩;在牖加紳,宜崇八座之寵。可贈尚書左僕射。」太常博士程皓議謚為「忠孝」。刑部尚書顏真卿以為忠則以身許國,見危致命,孝則晨昏色養,取樂庭闈,不合二行殊難,以成「忠孝」。主客員外郎歸崇敬又駁之,紛議不已。右僕射郭
 英乂不達其體,請從太常之狀而奏。陟子允。



 斌,景雲初安石為宰輔時,授太子通事舍人。早修整,尚文藝,容止嚴厲,有大臣體,與兄陟齊名。開元十七年,司徒薛王業為女平恩縣主求婚,以斌才地奏配焉。遷秘書丞。天寶初,轉國子司業,徐安貞、王維、崔顥,當代辭人,特為推挹。天寶中,拜中書舍人,兼集賢院學士。兄陟先為中書舍人,未幾遷禮部侍郎。陟在南省,斌又掌文誥。改太常少卿。天寶五載,右相李林甫構陷刑部尚書韋堅,斌以親
 累貶巴陵太守,移臨安太守,加銀青光祿大夫。斌授五品時,兄陟為河東太守,堂兄由為右金吾將軍,縚為太子少師,四人同時列戟,衣冠之盛,罕有其比。



 十四載,安祿山反,陷洛陽,斌為賊所得,偽授黃門侍郎,憂憤而卒。及克復兩京,肅宗乾元元年,贈秘書監。安石兄叔夏別有傳。從父兄子抗,從祖兄子巨源。



 抗,弱冠舉明經,累轉吏部郎中,以清謹著稱。景雲初,為永昌令,不務威刑而政令肅一。都輦繁劇,前後為政,寬猛得中,無如抗者。無
 幾,遷右臺御史中丞,人吏詣闕請留,不許,因立碑於通衢,紀其遺惠。開元三年,自左庶子出為益州長史。四年,入為黃門侍郎。



 八年,河曲叛胡康待賓擁徒作亂,詔抗持節慰撫。抗素無武略,不為寇所憚。在路遲留不敢進,因墜馬稱疾,竟不至賊所而還。俄以本官檢校鴻臚卿,代王晙為御史大夫,兼按察京畿。時抗弟拯為萬年令,兄弟同領本部,時人榮之。尋以薦御史非其人,出為安州都督,轉蒲州刺史。十一年,入為大理卿,其年代陸象
 先為刑部尚書,尋又分掌吏部選事。十四年卒。抗歷職以清儉自守,不務產業,及終,喪事殆不能給。玄宗聞其貧,特令給靈輿,遞送還鄉。贈太子少傅,謚曰貞。抗為京畿按察使時,舉奉天尉梁升卿、新豐尉王倕、金城尉王冰、華原尉王燾為判官及度支使,其後升卿等皆名位通顯,時人以抗有知人之鑒。



 巨源,周京兆尹總曾孫也。祖匡伯,襲祖爵鄖國公,入隋改封舒國公,官至尚衣奉御。巨源則天時累遷司賓少卿,轉司府卿、文昌右丞、同
 鳳閣鸞臺平章事。三年,轉夏官侍郎,依前平章事。有吏才,勾覆省內文案,下符剝徵,雖為下所怨苦,然亦頗收其利。證聖初,出為鄜州刺史,尋拜地官尚書、神都留守。長安二年,詔入轉刑部尚書,又加太子賓客,再為神都留守。



 神龍初,入拜工部尚書,封同安縣子。又遷吏部尚書、同中書門下三品,進封郇縣伯。時安石為中書令,以是巨源近屬,罷知政事。巨源尋遷侍中、中書令,進封舒國公,附入韋後三等親,敘為兄弟,編在屬籍。是歲,巨源
 奉制與唐休璟、李懷遠、祝欽明、蘇環等定《垂拱格》及《格後敕》,前後計二十卷,頒下施行。時武三思先有實封數千戶在貝州,時屬大水,刺史宋璟議稱租庸及封丁並合捐免;巨源以為穀稼雖被湮沉,其蠶桑見在,可勒輸庸調,由是河朔戶口頗多流散。



 景龍二年,順天翊聖皇后衣箱中裙上有五色雲起,久則方歇,巨源以為非常佳瑞,請布告天下,許之。中宗又令畫工圖其狀以示百僚,仍大赦天下,內外五品已上官母妻各加封邑。時中
 宗即雅信符瑞,巨源又贊成其妖妄。是歲星墜如雷,野雉皆雊,咎徵若此,不聞巨源有言,蓋與韋皇後繼敘源流,佞媚官爵,疑其開導,以踵則天。時有驍衛將軍迦葉志忠、太常少卿鄭愔、兵部尚書宗楚客、右補闕趙延禧等,或相諷諭,或上表章,謬說符祥,朋黨取媚,識者嗟憤。



 景龍三年,拜尚書左僕射,依舊知政事。未幾,又拜尚書令、同中書門下三品,仍舊監修國史。時國家將有事於南郊,而巨源希韋後之旨,協同祝欽明之議,言皇後合
 助郊祀,竟以皇后為亞獻,巨源為終獻,又以大臣女為齋娘。及韋庶人之難,家人令巨源逃匿,巨源曰:「吾國之大臣,豈得聞難不赴?」乃出,至都街,為亂兵所殺,時年八十。



 睿宗即位,贈特進、荊州大都督。太常博士李處直議巨源謚曰「昭」。戶部員外郎李邕駁之曰:「三思引之為相,阿韋托之為親,無功而封,無德而祿,同族則醜正安石,他人則附邪楚客,謚之曰『昭』,良恐不當。」初,巨源與安石迭為宰相,時人以為情不相協,故邕以此稱之。處直仍
 固請依前謚為定。邕又駁曰:



 夫古之謚,在乎勸沮,將杜小人之業,冀長君子之風。故為善者雖存不貴仕,而沒有餘名,此賢達所以砥節也;為惡者雖生有所幸,死懷所懲,此回邪所以易心也。嗚呼!巨源嘗未斯察,而乃聞義不從,與惡相濟,蓄罔上之志,協群兇之謀,茍容聖朝,貪昧厚祿。自以宰臣之貴,不崇朝而賈害者,固鬼得而誅之也。彼則匹夫之微,未受命而行刑者,固人得而誅之也。幽明之憤,斷焉可知,天地之心,自此而見矣。



 頃者
 皇運中興,功臣翼政。時序未幾,邪逆執權,奸慝者拜爵於私門,忠正者黜降於籓郡。巨源此際,用事方殷。且於阿韋何親,而結為昆季;於國家何力,而累忝大官。此則暗通中人,附會武氏,托城社之固,亂皇家之基。其罪一也。



 又國之大事,在祀與戎,酌於禮經,陳於郊祭。將以對越天地,光揚祖宗,即告成功,以觀海內。惟昔亞獻,不聞婦人,阿韋蓄無君之忱,懷自達之意,潛圖帝位,議啄皇孫,升壇擬儀,拜賜明命,將預家事,無守國章。巨源創跡
 於前,悖逆演成於後。時有禮部侍郎徐堅、太常博士唐紹、蔣欽緒、彭景直並言之莫從。其罪二也。



 又上天不吊,先帝遇毒,悔禍無徵,阿韋將篡。畫計未果,逆心尚搖,周章夷猶,倉卒迷謬。於是太平公主矯為陳謨,上官昭容紿草遺詔,故得今上輔政,阿韋參謀。將大業垂成,而休命中輟者,職由巨源躡韋溫之足,楚客附巨源之耳,梟聲遽發,狼顧相驚,以阿韋臨朝,以韋溫當國。其罪三也。



 又人為邦本,財實聚人,奪其財則人心自離,無其人則
 國本何恃。巨源屢踐臺輔,專行勾徵,廢越條章,崇尚侵刻,樹怨天下,剝害生靈,兆庶流離,戶口減耗。況以三思食邑,往在貝州,時屬久陰,災逢多雨。租庸捐免,申令昭明,匪今獨然,自古不易。三思慮其封物,巨源啟此異端,以為稼穡湮沉,雖無菽粟,蠶桑織糸任,可輸庸調。致使河朔黎人,海隅士女,去其鄉井,鬻其子孫,饑寒切身,朝夕奔命。其罪四也。



 但巨源長於華宗,仕於累代,作萬國之相,處具瞻之地,蔽日月之屋輝,負丘山之重責,今乃妄
 加褒述,安能分謗者哉!



 當時雖不從邕議,而論者是之。巨源與安石及則天時文昌右相待價,並是五服之親,自餘近屬至大官者數十人。



 趙彥昭者,甘州張掖人也。父武孟,初以馳騁佃獵為事。嘗獲肥鮮以遺母,母泣曰:「汝不讀書而佃獵如是,吾無望矣。」竟不食其膳。武孟感激勤學,遂博通經史。舉進士,官至右臺侍御史,撰《河西人物志》十卷。



 彥昭少以文辭知名。中宗時,累遷中書侍郎、同中書門下三品,兼修國
 史,充修文館學士。景龍四年,金城公主出降吐蕃贊普,中宗命彥昭為使,彥昭以既充外使,恐失其寵,殊不悅。司農卿趙履溫私謂曰:「公國之宰輔,而為一介之使,不亦鄙乎?」彥昭曰:「計將安出?」履溫因為陰托安樂公主密奏留之,中宗乃遣左驍衛大將軍楊矩代彥昭而往。



 睿宗時,出為涼州都督,為政清嚴,將士已下皆動足股慄。又為宋州刺史,入為吏部侍郎,又為刑部尚書、關內道持節巡邊使、檢校左御史臺大夫。



 彥昭素與郭元振、張
 說友善。及蕭至忠等伏誅,元振、說等稱彥昭先嘗密圖其事,乃以功遷刑部尚書,封耿國公,賜實封一百戶。殿中侍御史郭震奏:「彥昭以女巫趙五娘左道亂常,托為諸姑,潛相影援。既因提挈,乃踐臺階。驅車造門,著婦人之服;攜妻就謁,申猶子之情。於時南憲直臣,劾以霜憲,暫加微貶,旋登寵秩。同惡相濟,一至於此。乾坤交泰,宇宙再清,不加貶削,法將安措?請付紫微黃門,準法處分。」俄而姚崇入相,甚惡彥昭之為人,由是累貶江州別駕,
 卒。



 蕭至忠,秘書少監德言曾孫也。少仕為畿尉,以清謹稱。嘗與友人期於路隅,會風雪凍冽,諸人皆奔避就宇下。至忠曰:「寧有與人期而求安失信乎?」獨不去,眾咸嘆服。神龍初,武三思擅權,至忠附之,自吏部員外擢拜御史中丞。遷吏部侍郎,仍兼御史中丞。恃武三思勢,掌選無所忌憚,請謁杜絕,威風大行。尋遷中書侍郎,兼中書令。



 節愍太子誅武三思後,有三思黨與宗楚客、紀處訥令
 侍御史冉祖雍奏言:「安國相王及鎮國太平公主亦與太子連謀舉兵,請收付制獄。」中宗召至忠令按其事,至忠泣而奏曰:「陛下富有四海,貴為天子,豈不能保一弟一妹,受人羅織?宗社存亡,實在於此。臣雖愚昧,竊為陛下不取。《漢書》云:『一尺布,尚可縫,一斗粟,尚可舂,兄弟二人不相容。』願陛下詳察此言。且往者則天皇后欲令相王為太子,王累日不食,請迎陛下。固讓之誠,天下傳說,足明冉祖雍等所奏,咸是構虛。」帝深納其言而止。尋轉
 黃門侍郎、同中書門下平章事。至忠上疏陳時政,曰:



 臣聞王者列職分司,為人求理,求理之道,必在用賢。得其人則公務克修,非其才則厥官如曠。官曠則事廢,事廢則人殘,漸至凌遲,率由於此。頃者選曹授職,政事官人,或異才升,多非德進。皆因依貴要,互為粉飾,茍得即是,曾無遠圖,上下相蒙,誰肯言及?臣聞官爵者公器也,恩幸者私惠也,祇可金帛富之,粱肉食之,以存私澤也。若以公器為私用,則公議不行,而勞人解體;以小私而妨
 至公,則私謁門開,而正言路絕,儉人遞進,君子道消,日削月朘,卒見凋弊者,為官非其人也。昔漢館陶公主為子求郎,明帝謂曰:「郎官上應列宿,出宰百里,茍非其人,則人受其殃。」賜錢十萬而已。此即至公之道不虧,恩私之情無替,良史直筆,將為美談,於今稱之,不輟其口者也。當今列位已廣,冗員倍多,祈求未厭,日月增數。陛下降不貲之澤,近戚有無涯之請,賣官利己,鬻法徇私。臺寺之內,硃紫盈滿,官秩益輕,恩賞彌數。儉利之輩,冒進
 而莫識廉隅;方雅之流,知難而斂分丘隴。才者莫用,用者不才,二事相形,十有其五。故人不效力而官匪其人,欲求其理,實亦難哉。



 臣竊見宰相及近侍要官子弟,多居美爵,此並勢要親戚,罕有才藝,遞相囑托,虛踐官榮。《詩》云:「東人之子,職勞不賚。西人之子,粲粲衣服。私人之子,百僚是試。或以其酒,不以其漿。廛廛佩璲,不以其長。」此言王政不平,眾官廢職,私家之子,列試於榮班,非任之人,徒長其飾佩。臣愚伏願陛下想居安思危之義,行
 改弦易張之道。愛惜爵賞,審量材識,官無虛授,人必為官,進大雅於樞近,退小子於閑僻,政令惟一,威恩以信,私不害公,情不撓法,則天下幸甚。臣伏見永徵故事,宰相子弟多居外職者,非直抑強宗、分大族,亦以退不肖、擇賢才。伏願陛下遠稽舊典,近遵先聖,特降明敕,令宰相已下及諸司長官子弟,並改授外官,庶望分職四方,共寧百姓,表裏相統,遐邇人安。



 疏奏不納。



 明年,代韋巨源為侍中,仍依舊修史。尋遷中書令。時宗楚客、紀處訥
 潛懷奸計,自樹朋黨,韋巨源、世再思、李嶠皆唯諾自全,無所匡正。至忠處於其間,頗存正道,時議翕然重之。中宗亦曰:「諸宰相中,至忠最憐我。」韋庶人又為亡弟贈汝南王洵與至忠亡女為冥婚合葬。及韋氏敗,至忠發墓,持其女柩歸,人以此譏之。至忠又以女適庶人舅崔從禮之子。成禮日,中宗為蕭氏婚主,韋庶人為崔氏婚主,時人謂之「天子嫁女,皇後娶婦」。



 睿宗即位,景雲初,出為晉州刺史,甚有能名。時太平公主用事,至忠潛遣間使
 申意,求入為京職。誅韋氏之際,至忠一子任千牛,為亂兵所殺,公主冀至忠以此怨望,可與謀事,即納其請。召拜刑部尚書、右御史大夫,再遷吏部尚書。先天二年,復為中書令。是歲,至忠與竇懷貞、魏知古、崔湜、陸象先、柳沖、徐堅、劉子玄等撰成《姓族系錄》二百卷,有制加爵賜物各有差。未幾,左僕射竇懷貞、侍中岑羲及至忠並戶部尚書李晉、太子少保薛稷、左散騎常侍賈膺福、左羽林大將軍常元楷、右羽林將軍李慈等與太平公主謀
 逆事洩,至忠遽遁入山寺,數日,捕而伏誅,籍沒其家。至忠雖清儉刻己,然簡約自高,未嘗接待賓客,所得俸祿,亦無所賑施。及籍沒,財帛甚豐,由是頓絕聲望矣。弟元,工部侍郎。廣微,工部員外。



 宗楚客者,蒲州河東人,則天從父姊之子也。兄秦客,垂拱中潛勸則天革命稱帝,由是累遷內史。後與楚客及弟晉卿並以奸贓事發,配流嶺外。秦客死,楚客等尋復追還。楚客累遷夏官侍郎、同鳳閣鸞臺平章事。神龍初,
 為太僕卿。武三思用事,引楚客為兵部尚書、同中書門下三品,晉卿累遷將作大匠。節愍太子既殺武三思,兵敗,逃於鄠縣,楚客遣使追斬之,仍令以其首祭三思及崇訓喪柩。韋庶人及安樂公主尤加親信,未幾,遷中書令。楚客雖跡附韋氏,而嘗別有異圖,與侍中紀處訥共為朋黨,故時人呼為宗、紀。



 景龍中,西突厥娑葛與阿史那忠節不和,屢相侵擾,西陲不安。安西都護郭元振奏請徒忠節於內地,楚客與晉卿、處訥等各納忠節重賂,
 奏請發兵以討娑葛,不納元振所奏。娑葛知而大怒,舉兵入寇,甚為邊患。於是監察御史崔琬劾奏楚客等曰:



 臣聞四牡項領,良御不乘;二心事君,明罰無舍。謹案宗楚客、紀處訥等,性惟險詖,志越溪壑,幸以遭逢聖主,累忝殊榮,承愷悌之恩,居弼諧之地。不能刻意砥操,憂國如家,微效涓塵,以裨川岳。遂乃專作威福,敢樹朋黨,有無君之心,闕大臣之節。潛通獫狁,納賄不貲;公引頑兇,受賂無限。丑問充斥,穢行昭彰。且境外之交,情狀難測,
 今娑葛反叛,邊鄙不寧,由此賊臣,取怨中國。論之者懼禍以結舌,語之者避罪以鉗口。但晉卿昔居榮職,素闕忠誠,屢抵嚴刑,皆由黷貨。今又叨忝,頻沐殊恩,厚祿重權,當朝莫比。曾無悛改,仍徇贓私,此而可容,孰不可恕?臣謬參直指,義在觸邪,請除巨蠹,用答天造。楚客、處訥、晉卿等驕恣跋扈,人神同疾,不加天誅,詎清王度。並請收禁,差三司推鞫。



 舊制,大臣有被御史對仗劾彈者,即俯僂趨出,立於朝堂待罪。楚客更吒鰓作色而進,自言
 以執性忠鯁,被琬誣奏。中宗竟不能窮核其事,遽令琬與楚客等結為義兄弟以和解之。韋氏敗,楚客與晉卿等皆伏誅。



 紀處訥者,秦州上邽人也。娶武三思妻之姊,由是累遷太府卿。神龍中,嘗因穀貴,中宗召處訥親問其故。武三思諷知太史事右驍衛將軍迦葉志忠、太史令傅孝忠奏言,「其夜有攝提星入太微,至帝座。此則王者與大臣私相接,大臣能納忠,故有斯應。」帝以為然,降敕褒述處
 訥,賜衣一副、彩六十段。無幾,進拜侍中,與楚客等同時伏誅。



 史官曰:大帝、孝和之朝,政不由己,則天在位,已絕綴旒,韋後司晨,前蹤覆轍。當是時,奸邪有黨,宰執求容,順之則惡其名彰,逆之則憂其禍及,欲存身致理者,非中智常才之所能也。況元忠、安石、巨源、至忠、彥昭等行非純一,識昧存亡,徇利貪榮,有始無卒,不得其死,宜哉!楚客、晉卿、處訥等讒諂並進,威虐貫盈,不使逃刑,可謂政正。



 贊曰:為唐重臣,食唐重祿。顛危不持,富貴何足。二宗、一紀,讒邪酷毒。與前數公,死不知辱。



\end{pinyinscope}