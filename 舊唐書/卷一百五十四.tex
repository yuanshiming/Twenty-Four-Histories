\article{卷一百五十四}

\begin{pinyinscope}

 ○舒王誼通王諶虔王諒肅王詳文敬太子謜資王謙代王諲昭王誡欽王諤珍王諴郯王經
 均王緯漵王縱莒王紓密王綢郇王綜邵王約宋王結集王緗冀王絿和王綺衡王絢欽王績會王纁福王綰珍王繕撫王曨岳王緄袁王紳桂王綸翼王綽蘄王緝



 德宗皇帝十一子:昭德皇后王氏生順宗皇帝;舒王誼,昭靖太子之子;文敬太子,順宗之子;諸妃生通王已下
 八王,本錄不載母氏。



 舒王誼,本名謨,代宗第三子昭靖太子邈之子也。以其最幼,德宗憐之,命之為子。大歷十四年六月,封舒王,拜開府儀同三司,與通王、虔王同日封。仍詔所司,其開府俸料,逐月進內,尋以軍興罷支。建中元年,領四鎮北庭行軍、涇原節度大使;以涇州刺史孟皞為節度留後。以誼愛弟之子,諸王之長,軍國大事,欲其更踐,必委試之。



 明年,尚父郭子儀病篤,上御紫宸,命誼持制書省之。誼
 冠遠游冠,絳紗袍,乘象輅,駕駟馬,飛龍騎士三百人隨之。國府之官,皆褲褶騎而導前,鹵簿備引而不樂,在遏密故也。及門,郭氏子弟迎拜於外,王不答拜。子儀臥不能興,以手叩頭謝恩已。王解冠珮,以常服傳詔勞問之。



 三年,蔡帥李希烈叛,詔哥舒曜討之。八月,希烈自帥眾三萬,圍哥舒曜於襄城,又詔河南都統李勉援之。勉舍襄城,令大將唐漢臣等選勁兵,徑襲許州以解圍。漢臣未至許,上遣中使追之,責以違詔,亟旋師,為賊所乘,
 漢臣之眾大敗。勉恐東都危急,乃分兵數千赴洛,又為賊所隔。賊眾急攻汴、滑,勉走宋州,朝廷大聳,乃詔誼為揚州大都督,持節荊襄、江西、沔鄂等道節度,兼諸軍行營兵馬元帥,改名誼。又以哥舒翰聲近,士卒竊議,改封普王,令統攝諸軍,進攻希烈。仍以兵部侍郎蕭復為戶部尚書、兼御史大夫、元帥府統軍長史。舊例有行軍長史,以復父名衡,特更之。又以新除潭州觀察使孔巢父為右庶子、兼御史大夫,充行軍司馬;以山南東道節度
 行軍司馬、檢校兵部郎中、兼御史中丞樊澤為諫議大夫、兼御史中丞、行軍右司馬。刑部員外郎劉從一為吏部郎中、兼中丞;侍御史韋儹為工部郎中、兼中丞,並充元帥府判官。兵部員外郎高參為本司郎中,充元帥府掌書記。以右金吾大將軍渾瑊檢校工部尚書、兼御史大夫,為中軍虞候。江西節度使嗣曹王皋為前軍兵馬使,鄂岳團練使李兼為之副。山南東道節度使賈耽為中軍兵馬使。荊南節度使張伯儀充後軍兵馬使。以左
 神武軍使王價檢校太子賓客;左衛將軍高承謙檢校太子詹事;前司農少卿郭曙檢校左庶子,前秘書省著作郎常願為秘書少監,並充元帥府押衙。制下未行,涇原兵亂而止。



 德宗初聞兵士出怨言,不得賞設,乃令誼與翰林學士姜公輔傳詔安撫,許以厚賞。行及內門,兵已陣於闕前;誼狼狽而還,遂奉德宗出幸奉天。賊之攻城,誼晝夜傳詔,慰勞諸軍,僅不解帶者月餘。從車駕還宮,復封舒王、開府儀同三司,揚州大都督如故。永貞元
 年十月薨,廢朝三日。



 通王諶,德宗第三子也。大歷十四年封,制授開府儀同三司。貞元九年十月,領宣武軍節度大使、汴宋等州觀察支度營田等使,以宣武都知兵馬使李萬榮為留後,王不出閣。十一年,河東帥李自良卒,以諶為河東節度大使,以行軍司馬李說知府事,充留後,亦不出閣。



 虔王諒,德宗第四子。大歷十四年封,授開府儀同三司。貞元二年,領蔡州節度大使、申光蔡觀察等使,以大將
 吳少誠為留後。十年,領朔方靈鹽節度大使、靈州大都督,以朔方行軍司馬李欒為靈府左司馬,知府事,朔方留後。十一年九月,橫海大將程懷信逐其帥懷直。十月,以諒領橫海節度大使、滄景觀察等使,以都知兵馬使程懷信為留後,王不出閣。十六年,徐帥張建封卒,徐軍亂,又以諒領徐州節度大使、徐泗濠觀察處置等使,以建封子愔為留後。



 肅王詳,德宗第五子。大歷十四年六月封。建中三年十
 月薨,時年四歲,廢朝三日,贈揚州大都督。性聰惠,上尤憐之,追念無已,不令起墳墓,詔如西域法,議層磚造塔。禮儀使判官、司門郎中李巖上言曰:「墳墓之義,經典有常,自古至今,無聞異制。層磚起塔,始於天竺,名曰『浮圖』,行之中華,竊恐非禮。況肅王天屬,名位尊崇,喪葬之儀,存乎簡冊,舉而不法,垂訓非輕。伏請準令造墳,庶遵典禮。」詔從之。



 文敬太子謜,順宗之子。德宗愛之,命為子。貞元四年,封
 邕王,授開府儀同三司。七年,定州張孝忠卒,以謜領義武軍節度大使、易定觀察等使,以定州刺史張茂昭為留後。十年六月,潞帥李抱真卒,又以謜領昭義節度大使、澤潞邢洺名磁觀察等使,以潞將王虔休為潞府司馬、知留後。十五年十月薨,時年十八,廢朝三日,贈文敬太子,所司備禮冊命。其年十二月,葬於昭應,有陵無號。發引之日,百官送於通化門外,列位哭送。是日風雪寒甚,近歲未有。詔置陵署令丞。



 資王謙,德宗第七子。大歷十四年封。



 代王諲,德宗第八子。本封縉雲郡王,早薨。建中二年,追封代王。



 昭王誡,德宗第九子。貞元二十一年封。



 欽王諤,德宗第十子。順宗即位,詔曰:「王者之制,子弟畢封,所以固籓輔而重社稷,古今之通義也。第十弟諤等,寬簡忠厚,生知孝敬,行皆由禮,志不違仁。樂善本於性情,好賢宗於師傅。纘修六藝,達人倫風化之源;博習群
 言,知惠和睦友之道。溫恭朝夕,允茂厥猷,克有嘉聞,宜封土宇。諤可封欽王。第十一弟可封珍王。



 珍王諴,德宗第十一子,與欽王同制封。



 德宗仁孝,動循法度,雖子弟姑妹之親,無所假借。建中初,詔親王子弟帶開府朝秩者,出就本班。又以公主、郡縣主出降,與舅姑抗禮。詔曰:「冠婚之義,人倫大經。昔唐堯降嬪,帝乙歸妹。迨於漢氏,同姓主之。爰自近古,禮教陵夷,公郡法度,僭差殊制。姻族闕齒序之義,舅姑有拜下之禮,自家刑
 國,多愧古人。今縣主有行,將俟嘉令,俾親執棗慄,以見舅姑;敬遵宗婦之儀,降就家人之禮。事資變革,以抑浮華。其令禮儀使與禮官博士,約古今舊儀及《開元禮》,詳定公主、郡縣主出降、覿見之文儀以聞。」



 初,開元中置禮會院於崇仁里。自兵興已來,廢而不修,故公、郡、縣主不時降嫁,殆三十年,至有華發而猶丱者,雖居內館,而不獲覲見十六年矣。凡皇族子弟,皆散棄無位,或流落他縣,湮沉不齒錄,無異匹庶。及德宗即位,敘用枝屬,以時
 婚嫁,公族老幼,莫不悲感。初即位,將謁太廟,始與公、郡、縣主相見於大次中,尊者展其敬,幼者申其愛,歔欷哭泣之聲聞於朝,公卿陪列者為之淒然。每將有大禮,必與諸父昆弟同其齋次。及岳陽、信寧、宜芳、永順、朗陵、陽安、襄城、德清、南華、元城、新鄉等十一縣主同月出降,敕所司大小之物,必周其用。至於櫛、纚、笄、總,皆經於心,各給錢三百萬,使中官主之,以買田業,不得侈用。其衣服之飾,使內司計造,不在此數。是時所司度人用一籠花,
 計錢七十萬。帝曰:「籠花首飾,婦禮不可闕,然用費太廣,即無謂也。宜損之又損之。」及三萬而止。帝謂主等曰:「吾非有所愛,但不欲無益之費耳。」各以餘錢六十萬賜之,以備他用。



 舊例,皇姬下嫁,舅姑返拜而婦不答。及是制下,禮官定制曰:「既成婚於禮會院,明晨,舅坐於堂東階西向,姑南向,婦執笄,盛以棗慄,升自西階,再拜,跪奠於舅席前。退降受,盛以腶修。升,北面再拜,跪奠於姑席前。降,東面拜婿之伯叔兄弟姊妹。已而謝恩於光順門,
 婿之親族亦隨之,然後會宴於十六宅。」是日,縣主皆如其制。初,贈司徒沈易良之妻崔氏,即太后之季父母也,帝每見之,方屣而靴,召王、韋二美人出拜。敕崔氏坐受勿答。故戚屬之間,罔不憚其敬,不肅而遵禮法焉。



 順宗二十三子:莊憲皇后王氏生憲宗皇帝;王昭儀生郯王經;趙昭儀生宋王結;王昭儀生郇王綜;王昭訓生衡王絢;餘十八王,本錄不載母氏。



 郯王經,本名渙,順宗次子。始封建康郡王,貞元二十一
 年封。太和八年薨。



 均王緯,本名沔,順宗第三子。始封洋川郡王,貞元二十一年進封。



 漵王縱,本名洵,順宗第四子。初授殿中監,封臨淮郡王,貞元二十一年進封。



 莒王紓,本名浼,順宗第五子。初授秘書監,封弘農郡王。貞元二十一年進封。太和八年薨。



 密王綢,本名泳,順宗第六子。始封漢東郡王,貞元二十
 一年進封。元和二年九月薨。



 郇王綜,本名湜,順宗第七子。初授少府監,封晉陵郡王,貞元二十一年進封。元和三年四月薨。



 邵王約,本名漵,順宗第八子。初授國子祭酒,封高平郡王,貞元二十一年進封。



 宋王結,本名滋,順宗第九子。始封雲安郡王,貞元二十一年進封。長慶二年薨。



 集王緗,貞元二十一年封。長慶二年薨。



 冀王絿,本名淮,順宗第十子。初授太常卿,封宣城郡王,貞元二十一年進封。太和九年薨。



 和王綺,本名湑,順宗第十一子。始封德陽郡王,貞元二十一年進封。太和七年薨。



 衡王絢,順宗第十二子。貞元二十一年封。寶歷二年薨。



 欽王績,順宗第十三子。貞元二十一年封。



 會王纁,順宗第十四子。貞元二十一年封。元和五年十一月薨。



 福王綰,本名浥,順宗第十五子。母莊憲王皇后,憲宗同出。初授光祿卿,封河東郡王,貞元二十一年進封。咸通元年,特冊拜司空。明年薨。



 珍王繕,本名況,順宗第十六子。初授衛尉卿,封洛交郡王,貞元二十一年進封。



 撫王曨,順宗第十七子。貞元二十一年封。咸通四年,特冊拜司空。五年,冊司徒。乾符三年,冊太尉。其年薨。



 岳王緄,順宗第十八子。貞元二十一年封。太和二年薨。



 袁王紳,順宗第十九子。貞元二十一年封。太和十四年薨。



 桂王綸,順宗第二十子。貞元二十一年封。太和九年薨。



 翼王綽,順宗第二十一子。貞元二十一年封。咸通二年薨。



 蘄王緝,順宗第二十二子。咸通八年封。



 史臣曰:夫聖人君臨宇縣,肇啟邦基,莫不受命上玄,膺名帝籙。自太昊已降,五運相推,迄於殷湯,歷數綿永。但
 設均平之化,未聞封建之名。洎乎周、漢,始以子弟建侯樹屏,以作維城。及王室浸微,遂有莽、卓之亂。唐室自艱難已後,兩河兵革屢興,諸王雖封,竟不出閣,夫帝王居寰宇之尊,撫億兆之眾,但能平一理道,夙夜嚴恭,任賢使能,設官分職,自然四海樂推。天命所祐,縱無封建,亦鴻基永固,安俟嬰孺鎮重哉?



 贊曰:孝文秉禮,道弘籓邸。睦族展親,儀刑戚里。自閣臨籓,所謂周爰。無如惡鳥,終懷籠樊。



\end{pinyinscope}