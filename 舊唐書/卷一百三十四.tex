\article{卷一百三十四}

\begin{pinyinscope}

 ○王璵道士李國禎附李泌子繁顧況附崔造關播李元平附



 王璵,少習禮學,博求祠祭儀注以干時。開元末,玄宗方尊道術,靡神不宗。璵抗疏引古今祀典,請置春壇,祀青
 帝於國東郊,玄宗甚然之,因遷太常博士、侍御史,充祠祭使。璵專以祀事希幸,每行祠禱,或焚紙錢,禱祈福祐,近於巫覡,由是過承恩遇。肅宗即位,累遷太常卿,以祠禱每多賜賚。乾元三年七月,兼蒲州刺史,充蒲、同、絳等州節度使。中書令崔圓罷相,乃以璵為中書侍郎、同中書門下平章事。人物時望,素不為眾所稱,及當樞務,聲問頓減。璵又奏置太一神壇於南郊之東,請上躬行祀事。肅宗嘗不豫,太卜云:「崇在山川。」璵乃遣女巫分行天
 下,祈祭名山大川。巫皆盛服乘傳而行,上令中使監之,因緣為奸,所至乾托長吏,以邀賂遺。一巫盛年而美,以惡少年數十自隨,尤為蠹弊,與其徒宿於黃州傳舍。刺史左震晨至,驛門扃鐍,不可啟,震破鎖而入,曳女巫階下斬之,所從惡少年皆斃。閱其贓賂數十萬,震籍以上聞,仍請贓錢代貧民租稅,其中使發遣歸京,肅宗不能詰。肅宗親謁九宮神,殷勤於祠禱,皆璵所啟也。歲餘,罷知政事,為刑部尚書。上元二年,兼揚州長史、御史大夫。
 兗淮南節度使。肅宗南郊禮畢,以璵使持節都督越州諸軍事、越州刺史,充浙江東道節度觀察處置使,本官兼御史大夫,祠祭使如故。入為太子少保,轉少師。。大歷三年六月卒。



 璵以祭祀妖妄致位將相,時以左道進者,往往有之。廣德二年八月,道士李國禎以道術見,因奏皇室仙系,宜修崇靈跡。請於昭應縣南三十里山頂置天華上宮露臺、大地婆父、三皇、道君、太古天皇、中古伏羲媧皇等祠堂,並置掃灑宮戶一百戶。又於縣之東義
 扶穀故湫置龍堂,並許之。時歲饑荒,人甚不安,昭應縣令梁鎮上表曰:



 臣聞國以人為本,害其本則非國;神以人為主,虐其主則非神。故昔之聖王,所以極陳理道,明著祀典,將愛其人而慎用其財力,敬其神而虔恭於祠祭。故神享其明德而降之福,人受其大賚而盡其力,然後神人以和,而國家可保也。一昨蟊賊作孽,水旱為災,雖王畿皆遍,而臣縣最苦。此則神之不能御大災明矣,又何力於陛下而得列祀典哉!用以殘弊之餘,當兇荒
 之歲,丁壯素出家入仕,羸老方飛芻輓粟,令但供億王事,已不堪命,更奔走鬼道,何以聊生?



 臣又聞天地之神,尊之極者,掃地可祭,精意可饗。陛下亦何必廢先王之典,崇俗巫之說,走南畝之客,殺東鄰之牛,而後冀非妄之福。陛下雖欲為人祈福,福未至而人已困矣!其不可一也。陛下不視昔者有道之君,至德之後,曷不卑宮室,惡飲食,恭己以遂萬物之性哉!陛下今違神亭育之心,竭人疲困之力,如是又何從而致其福哉?此又不可二
 也。又陛下宗廟之敬極矣,尚無一月三祭之禮;今此獨為,則宗廟之靈,將等以親疏,校以厚薄,陛下又何以言哉?此又不可三也。又大地婆父,祀典無文,言甚不經,義無可取。若陛下待與大地建祖宗之廟,必上天貽向背之責,陛下又何以為詞哉?此又不可四也。夫湫者,龍之所居也。龍得水則神,無水則螻蟻之匹也。故知水存則龍在,水竭則龍亡,此愚智之所同知矣。今湫竭已久,龍安所存?陛下又崇飾祠宇,豐潔薦奠,為去龍之穴,破生
 人之產,人且怨矣,神何歆哉!此又不可五也。其道君、三皇、五帝,則兩京及所都之處,皆建宮觀祠廟,時設齋醮饗祀,國有彞典,官有常禮,蓋無闕失,何勞神役靈?此又不可六也。臣稽先王之典禮,觀前聖之軌躅,休咎豐兇,災祥禍福,必主帝王五事,不在山川百神。此又不可七也。



 臣伏察此弊,頗知其由。蓋以道士李國禎等動眾則得人,興工則獲利,祭祀則受胙,主執則弄權。是以鼓動禁中,熒惑天聽,逾越險阻,負荷粢盛,以日系年,無時而
 息。曾不謂神功力,空止竭人膏血,以使人神胥怨,災孽並生。罔上害人,左道亂政,原情定罪,非殺而何!



 臣昨受命之時,親承聖旨,務存安緝,許逐權宜。誠願沉鄴縣之巫,安流弊之俗,其所興兩祠土木之功、丹青之役、三六之祭、灑掃之戶,謹明宣旨,並以權宜停訖。人吏百姓等,知陛下以從善為心,嫉惡為務,蠲除不急,劃革煩苛,皆喧呼於庭,抃躍於路,所徵糧糗,無不樂輸。臣伏以國禎等並交結中貴,狡蠹成性,臣雖忘身許國,不懼讒構,終
 恐賄及豪右,復為奸惡。其國禎等見據狀推勘,如獲贓狀,伏望許臣徵收,便充當縣郵館本用。其湫既竭,不可更置祠堂,又不當為大地建立祖廟,臣並請停。其三皇、道君、天皇、伏羲、女媧等,既先各有宮廟,望請並於本所依禮齋祭。



 上從之。



 李泌,字長源,其先遼東襄平人,西魏太保、八柱國司徒徒何弼之六代孫。今居京兆吳房令承休之子。少聰敏,博涉經史,精究《易象》,善屬文,尤工於詩,以王佐自負。張九
 齡、韋虛心、張廷珪皆器重之。泌操尚不羈,恥隨常格仕進。天寶中,自嵩山上書論當世務,玄宗召見,令侍詔翰林,仍東宮供奉。楊國忠忌其才辯,奏泌嘗為《感遇詩》,諷刺時政,詔於蘄春郡安置,乃潛遁名山,以習隱自適。天寶末,祿山構難,肅宗北巡,至靈武即位,遣使訪召。會泌自嵩、潁間冒難奔赴行在,至彭原郡謁見,陳古今成敗之機,甚稱旨,延致臥內,動皆顧問。泌稱山人,固辭官秩,特以散官寵之,解褐拜銀青光祿大夫,俾掌樞務。至於
 四言文狀、將相遷除,皆與泌參議,權逾宰相,仍判元帥廣平王軍司馬事。肅宗每謂曰:「卿當上皇天寶中,為朕師友,下判廣平行軍,朕父子三人,資卿道義。」其見重如此。尋為中書令崔圓、幸臣李輔國害其能,將有不利於泌。泌懼,乞游衡山,優詔許之,給以三品祿俸,遂隱衡岳,絕粒棲神。



 數年,代宗即位,召為翰林學士,頗承恩遇。及元載輔政,惡其異己,因江南道觀察都團練使魏少游奏求參佐,稱泌有才,拜檢校秘書少監,充江南西道
 判官,幸其出也。尋改為檢校郎中,依前判官。元載誅,乃馳傳入謁,上見悅之。又為宰相常袞所忌,出為楚州刺史。及謝恩,具陳戀闕,上素重之,留京數月。會澧州刺史闕,袞盛陳泌理行,以荊南凋瘵,遂輟泌理之。詔曰:「荊南都會,粵在澧陽,俾人歸厚,惟賢是牧。以泌文可以代成風俗,政可以全活惸嫠。爰命頒條,期乎共理,地薄淮陽之守,勉思渤海之功。可檢校御史中丞,充澧朗硤團練使。」重其禮而遣之。無幾,改杭州刺史,以理稱。



 興元初,征
 赴行在,遷左散騎常侍。貞元元年,除陜州長史,充陜、虢都防禦觀察使。二年六月,泌奏:「虢州盧氏山冶,近出瑟瑟,請充獻,禁人開採。」詔曰:「瑟瑟之寶,中土所無今產於近甸,實為靈貺。朕不飾器玩,不尚珍奇,常思返樸之風,用明躬儉之節。其出瑟瑟之處,任百姓求採,不宜禁止。」就加泌檢校禮部尚書。時陳、許戍邊卒三千自京西逃歸,至州境,泌潛師險隘,左右攻擊,盡誅之。尋拜中書侍郎、平章事、集賢崇文館學士、修國史。初,張延賞大減官
 員,人情咨怨,泌請復之,以從人欲,因是奏罷兼試額內占闕等官,加百官俸料,隨閑劇加置手力課,上從之,人人以為便。而竇參旁奏,遂改易,使同品之內,月俸多少累等。泌又奏請罷拾遺、補闕,上雖不從,亦不授人,故諫司惟韓皋、歸登而已。泌仍命收其署湌錢,令登等寓食於中書舍人,故時戲云:「韓諫議雖分左右,歸拾遺莫辨存亡。」如是者三年。至貞元五年,以前東都防禦判官、殿中侍御史、內供奉韋綬為左補闕,監察御史梁肅右補
 闕。既復置,人心忻然。順宗在春宮,妃蕭氏母郜國公主交通外人,上疑其有他,連坐貶黜者數人,皇儲亦危。泌百端奏說,上意方解。



 泌頗有讜直之風,而談神仙詭道,或云嘗與赤松子、王喬、安期、羨門游處,故為代所輕,雖詭道求容,不為時君所重。德宗初即位,尤惡巫祝怪誕之士。初,肅宗重陰陽祠祝之說,用妖人王璵為宰相,或命巫媼乘驛行郡縣以為厭勝。凡有所興造功役,動牽禁忌。而黎幹用左道位至尹京,嘗內集眾工,編刺珠繡
 為御衣,既成而焚之,以為禳禬,且無虛月。德宗在東宮,頗知其事,即位之後,罷集僧於內道場,除巫祝之祀。有司言宣政內廊壞,請修繕。而太卜云:「孟冬為魁岡,不利穿築,請卜他月。」帝曰:「《春秋》之義,啟塞從時,何魁岡之有?」卒命修之。又代宗山陵靈駕發引,上號送於承天門,見轀輬不當道,稍指午未間。問其故,有司對曰:「陛下本命在午,故不敢當道。」上號泣曰:「安有枉靈駕而謀身利。」卒命直午而行。及建中末,寇戎內梗,桑道茂有城奉天之
 說,上稍以時日禁忌為意,而雅聞泌長於鬼道,故自外征還,以至大用,時論不以為愜。及在相位,隨時俯仰,無足可稱。復引顧況輩輕薄之流,動為朝士戲侮,頗貽譏誚。年六十八薨,贈太子太傅,賻禮有加。泌放曠敏辯,好大言,自出入中禁,累為權幸忌嫉恆由智免;終以言論縱橫,上悟聖主,以躋相位。有文集二十卷。



 子繁,少聰警,有才名,無行義。泌為相,嘗引薦夏縣處士北平陽城為諫議大夫。城道直,既遇知己,深德之。及泌歿,戶部尚書
 裴延齡巧佞奉上,德宗信任,竊弄威權,舉朝側目。城中正之士,尤忿嫉之。一日盡疏其過惡,欲密論奏,以繁故人子,為可親信,遂示其疏草,兼請繁繕寫。繁既寫,悉能記之,其夕乃徑詣延齡,具述其事。延齡聞之,即時請對,盡以城章中欲論事件,一一先自解。及城疏入,德宗以為妄,不之省。泌與右補闕、翰林學士梁肅友善,嘗命繁持所著文請肅潤色。繁亦自有學術,肅待之甚厚,因許師事,日熟其門。及肅卒,繁亂其配,士君子無不嘆駭,積
 年委棄。後起為太常博士,太常卿權德輿奏斥之,除河南府士曹掾。以其警悟異常,泌之故人為宰相,左右援拯,後得累居郡守,而力學不倦。罷隨州刺史,歸京師,久不承恩。



 韋處厚入相,厚待之。寶歷二年六月,敬宗降誕日,御三殿,特詔兵部侍郎丁公著、太常少卿陸旦與繁等三人抗浮圖道士講論。九月,除大理少卿,復加弘文館學士。時諫官御史章疏相繼,宰臣不得已,出為亳州刺史。州境嘗有群賊,剽人廬舍,劫取貨財,累政擒捕不
 獲。繁潛設機謀,悉知賊之巢穴,出兵盡加誅斬。時議責繁以不先啟聞廉使,涉於擅興之罪,朝廷遣監察御史舒元輿按問。元輿素與繁有隙,復以初官,銳於生事,乃盡反其獄辭,以為繁濫殺無辜,狀奏,敕於京兆府賜死,時人冤之。其後元輿被禍,人以為有報應焉。



 初,泌流放江南,與柳渾、顧況為人外之交,吟詠自適。而渾先達,故泌復得入官於朝。



 顧況者,蘇州人。能為歌詩,性詼諧,雖王公之貴與之交
 者,必戲侮之,然以嘲誚能文,人多狎之。柳渾輔政,以校書郎征。復遇李泌繼入,自謂己知秉樞要。當得達官,久之方遷著作郎。況心不樂,求歸於吳。而班列群官,咸有侮玩之目,皆惡嫉之。及泌卒,不哭,而有調笑之言,為憲司所劾,貶饒州司戶。有文集二十卷。其《贈柳宜城》辭句,率多戲劇,文體皆此類也。



 子非熊,登進士第,累佐使府,亦有詩名於時。



 崔造,字玄宰,博陵安平人。少涉學,永泰中,與韓會、盧東
 美、張正則為友,皆僑居上元,好談經濟之略,嘗以王佐自許,時人號為「四夔」。浙西觀察使李棲筠引為賓僚,累至左司員外郎。與劉晏善,及晏遭楊炎、庾準誣奏伏誅,造累貶信州長史。



 硃泚之逆,造為建州刺史,聞難作,馳檄鄰州,請齊舉義兵,遂調發所部,得二千人,德宗聞而嘉之。及收京師,詔徵造至藍田,以舅源休明逆伏誅,上疏請罪,不敢即赴闕。上以為知禮,優詔慰勉,拜吏部郎中、給事中。貞元二年正月,與中書舍人齊映各守本官,
 同平章事。時京畿兵亂之後,仍歲蝗旱,府無儲積。德宗以造敢言,為能立事,故不次登用。



 造久從事江外,嫉錢穀諸使罔上之弊,乃奏天下兩稅錢物,委本道觀察使、本州刺史選官典部送上都;諸道水陸運使及度支、巡院、江淮轉運使等並停;其度支、鹽鐵,委尚書省本司判;其尚書省六職,令宰臣分判。乃以戶部侍郎元琇判諸道鹽鐵、榷酒等事;戶部侍郎吉中孚判度支及諸道兩稅事;宰臣齊映判兵部承旨及雜事;宰臣李勉判刑部;
 宰臣劉滋判吏部、禮部;造判戶部、工部。又以歲饑,浙江東西道入運米每年七十五萬石,今更令兩稅折納米一百萬石,委兩浙節度使韓滉運送一百萬石至東渭橋;其淮南濠壽旨米、洪潭屯米,委淮南節度使杜亞運送二十萬石至東渭橋。諸道有鹽鐵處,依舊置巡院勾當;河陰見在米及諸道先付度支、巡院般運在路錢物,委度支依前勾當,其未離本道者,分付觀察使發遣,仍委中書門下年終類例諸道課最聞奏。造與元琇素厚,
 罷使之後,以鹽鐵之任委之。而韓滉方司轉運,朝廷仰給其漕發。滉以司務久行,不可遽改。德宗復以滉為江淮轉運使,餘如造所條奏。元琇以滉性剛難制,乃復奏江淮轉運,其江南米自江至揚子凡十八里,請滉主之;揚子已北,琇主之。滉聞之怒,掎摭琇鹽鐵司事論奏。德宗不獲已,罷琇判使,轉尚書右丞。其年秋初,江淮漕米大至京師,德宗嘉其功,以滉專領度支、諸道鹽鐵轉運等使,造所條奏皆改。物議亦以造所奏雖舉舊典,然兇
 荒之歲,難為集事,乃罷造知政事,守太子右庶子,貶琇雷州司戶。造初奏太銳,及琇改官,憂懼成疾,數月不能視事。明年九月卒,年五十一。



 關播,字務元,衛州汲人也。天寶末,舉進士。鄧景山為淮南節度使,闢為從事,累授衛佐評事,遷右補闕。善言物理,尤精釋氏之學。大歷中,神策軍使王駕鶴妻關氏以播與同宗,深遇之。元載惡其交往,出為河南府兵曹,攝職數縣,皆有政能。陳少游領浙東、淮南,又闢為判官,歷
 檢校金部員外,攝滁州刺史。李靈曜阻兵,跋扈於梁汴。少游自總兵鎮淮上,所在盜賊峰起。播調閱州兵,令其守備。又為政清凈簡惠,既無盜賊,人甚安之。楊綰、常袞知政事,薦播為都官員外郎。



 德宗登極,湖南山洞中有王國良者,聚眾為盜,令播往宣撫之。臨行,召對於別殿,上問政理之要,播奏云:「為政之本,須求有道賢人,乃可得理。」上謂播云:「朕下詔求賢良,當躬新閱試,亦遣使臣黜陟,廣加搜訪聞薦,擢其能者用之,冀以傅理。」播奏曰:「
 下詔求賢黜陟舉薦,唯得求名詞之士,安有有道賢人肯隨牒舉選乎?」上悅其言,謂播曰:「卿且使去,回日當與卿論政事。」播又奏曰:「臣今奏詔招撫,國良不受命,臣請便宣恩命,語鄰境速出兵翦除。」上曰:「卿言深合朕意。」使回,改兵部員外,遷河中少尹。



 建中初,張鎰為河中少尹。鎰尋入相,二年七月,遷播給事中。舊例,諸司甲庫,皆是胥吏掌知,為弊頗久,播始建議並以士人知之,至今稱當。轉刑部侍郎、奉迎皇太后副使。盧杞以播柔緩,冀
 其易制,驟稱薦之。尋遷吏部侍郎,轉刑部尚書、知刪定。奏上元中,詔擇古今名將十人於武成王廟配享,如文宣王廟之儀。播以「太公古稱大賢,今其下稱亞聖,於義不安。又孔子十哲,皆是當時弟子,今所擇名將,年代不同,於義既乖,於事又失。臣請刪去名將配享之儀及十哲之稱。」從之。



 建中三年十月,拜銀青光祿大夫、中書侍郎、同中書門下平章事、集賢殿崇文館大學士、修國史。時政事決在盧杞,播但斂衽取容而已。乏於知人之鑒,
 好大言虛誕者,播必悅而親信之。有李元平、陶公達、張愻、劉承誡,皆言談詭妄,言誇大可立功名,亦有微材薄藝。播累奏雲元平等皆可將相也,請閱試用之,上以為然,以元平為補闕。會淮西節度李希烈叛亂,上以汝州要鎮,令選擇刺史。播薦元平為汝州刺史,尋加檢校吏部郎中、汝州別駕,知州事。元平至州旬日,為希烈所擒,汝州陷賊,中外哂之。由是公達等未克任用。播與盧杞等從駕幸奉天,既而杞、白志貞等並貶黜,播尚知政事,
 中外囂然,以為不可,遂罷相,改刑部尚書。大臣韋倫等泣於朝曰:「宰相不能謀猷翊贊,以至今日,而尚為尚書,可痛心也!」



 貞元四年,回紇請和親,以咸安公主出降可汗,令播以本官加檢校右僕射、兼御史大夫,持節充送咸安公主及冊可汗使,奉使往來,皆清儉謹慎,蕃人悅之。使回,遷兵部尚書,固辭疾,請罷官,改太子少師致仕。播致仕之後,減去僮僕車騎,閉關守靜,不縈外事,士君子重之。貞元十三年正月卒,時年七十九,廢朝一日,贈太
 子太保。



 李元平者,宗室子。始為湖南觀察使蕭復判官,試大理評事。性疏傲,敢大言,好論兵,天下賢士大夫無可其意者,以是人多銜怒。關播奇重之,許以將帥。時希烈反叛,朝廷以汝州與賊接壤,刺史韋光裔懦弱不任職,播乃盛稱元平,特召見,超左補闕,不數日,擢為檢校吏部郎中,兼汝州別駕,知州事。既至部,募工徒繕理郛郭,希烈乃使勇士應募,執役板築,凡入數百人,元平不之覺。希
 烈遣偽將李克誠以數百騎突至其城,先應募執役者應於內,縛元平馳去。既見希烈,遺下污地。希烈見其無須眇小,戲謂克誠曰:「使汝取李元平,何得將元平兒來?」因嫚罵曰:「盲宰相使汝當我,何待我淺耶!」偽署為御史中丞。播聞元平得用,仍欺於人曰:「李生功業濟矣。」言必能覆希烈而建功也。居無何,希烈用為宰相,或告其有二者,乃斷一指以自誓。希烈既死,或有人言在賊中微有謀慮,貸死流於珍州。會赦得歸剡中,浙東觀察使皇
 甫政表聞其到,以發上怒,復流賀州而死。



 史臣曰:蒸嘗礿祀,前王制以奉先;怪力亂神,宣聖鄙而不語。凡云左道,固有舊章。璵假於鬼神,乃至將相,既處代天之位,爰滋亂政之源。國禎妖人疑眾,妄恢其祀典;梁鎮正士抗疏,方悟其上心。泌見可進而知難退,足為高率智辯之士;居相位而談鬼神,乃見狂妄浮薄之蹤。《王制》云:「執左道以亂政,殺。」寧無畏乎!繁之醜行,棄於當時,竟陷非辜,諒由素履。造為臣得禮,蒞事非能;播居位
 取容,舉人敗事。皆非國器,咸歷臺司,失人者亡,國其危矣。



 贊曰:璵、泌、造、播,俱非相材。國禎左道,梁生直哉!



\end{pinyinscope}