\article{卷一百八十五}

\begin{pinyinscope}

 ○史憲誠子孝章何進滔子弘敬韓允忠子簡樂彥禎子從訓羅弘信子威



 史憲誠,其先出於奚虜,今為靈武建康人。祖道德,開府儀同三司、試太常卿、上柱國、懷澤郡王。父周洛,為魏博
 軍校,事田季安,至兵馬大使、銀青光祿大夫、檢校太子賓客、兼御史中丞、柱國、北海郡王。憲誠始以材勇,隨父歷軍中右職,兼監察御史。元和中,田弘正討李師道,令憲誠以先鋒四千人濟河,累下其城柵。復以大軍齊進,乘勢逐北,魏之全師迫於鄆之城下。師道窮蹙,劉悟斬首投魏軍。錄功,超授憲誠兼中丞。



 鎮州王承宗死,弘正自魏移領鎮州。居數月,為王廷湊所殺,遂以兵叛。朝廷以弘正子布為魏博節度使,領兵討伐,俾復父冤。時幽
 州硃克融援助廷湊,布不能制,因自引決,軍情囂然。



 憲誠為中軍都知兵馬使,乘亂以河朔舊事動其人心,諸軍即擁而歸魏,共立為帥,國家因而命之。時克融、廷湊並據兵為亂,憲誠喜得旄節,雖外順朝旨,而中與硃、王為輔車之勢,長慶二年正月也。



 尋遣司門郎中韋文恪宣慰。時李絺為亂,與憲誠書問交通。憲誠表請與絺節鉞,仍於黎陽艤舟,示欲渡河。及見文恪,舉止驕倨,其言甚悖。旋聞絺為帳下所殺,乃從改過,謂文恪曰:「憲誠蕃
 人,猶狗也,唯能識主。雖被棒打,終不忍離。」其狡譎如此。朝廷每為優容,尋加左僕射。敬宗即位,進秩司空。



 太和二年,滄景節度使李全略卒,其子同捷竊據軍城,表邀符節,舉兵伐之。先是,憲誠與全略婚媾,及同捷叛,復潛以糧餉為助。上屢發使申諭,尋又就加平章事。憲誠嘗遣驍將至闕下,恣為張大,宰相韋處厚以語折銼之,憲誠不敢復與同捷為應。時憲誠示出師共討同捷。及滄景平,加司徒。憲誠心不自安,乃遣子孝章入覲,又飛章
 願以所管奉命。上嘉之。乃加侍中,移鎮河中。憲誠素懷向背,不能以忠誠感激其眾。未及出城,太和三年六月二十六日夜,為軍眾所害,冊贈太尉。



 孝章,幼聰悟好學。元和中,李醖為魏帥,取大將子弟列於軍籍。孝章倡言願效文職,醖奇之,令攝府參軍。及憲誠領節鉞,改士曹參軍、兼監察御史,賜緋。孝章以父在鎮多違朝旨,嘗雪涕極諫,備陳逆順之理。朝廷聞而嘉之,乃授檢校太子左諭德、兼侍御史,充節度副使。累遷至散騎常侍、兼御
 史大夫,賜紫。領本道兵同平滄景,加工部尚書。尋請赴闕,文宗慰勞甚厚,憲誠亦因懇乞朝覲。上知憲誠之入覲,自孝章之謀,遂加禮部尚書,分相、衛、澶三州別為一鎮,俾孝章領之。孝章未到鎮,憲誠遇害。上以孝章有忠節,起復為右金吾衛將軍。間歲,授鄜坊節度使。居四年,遷於滑。一歲,入為右領軍大將軍,改右金吾大將軍,俄授邠寧節度。



 孝章歷三鎮,雖無異績,而謹身畏法,以保初終。開成三年十月卒,贈右僕射。



 何進滔,靈武人也。曾祖孝物,祖俊,並本州軍校。父默,夏州衙前兵馬使,檢校太子賓客,試太常卿。以進滔之貴,贈左散騎常侍。進滔客寄於魏,委質軍門,事節度使田弘正。弘正奉詔討鄆州,破李師道,時進滔為衙內都知兵馬使,以功授兼侍御史。太和三年,軍眾害史憲誠,連聲而呼曰:「得衙內都知兵馬使何端公知留後,即三軍安矣。」推而立之。朝廷因授進滔左散騎常侍、魏博等州節度觀察處置等使。為魏帥十餘年,大得民情,累官至
 司徒、平章事卒。



 子弘敬襲其位。朝廷時遣河中帥李執方、滄州帥劉約各遣使勸令歸闕,別俟朝旨。弘敬不從,竟就加節制。及劉稹反,不時起兵。鎮州王元逵下邢、洺二州,兵次上黨,弘敬方出師壓境。大中後,宣宗務其姑息,繼加官爵,亦至使相。咸通初,卒。子全皞嗣之。朝廷尋降符節,累官亦至同平章事。十一年,為軍人所害。子孫相繼,四十餘年。



 韓允忠,魏州人也。舊名君雄,懿宗改賜今名。父國昌,歷
 本州右職。會昌中,從何弘敬破劉稹,以功為貝州刺史、兼御史中丞。以允忠故,累贈兵部尚書。允忠少仕軍門,繼升裨校。潞州之役,亦與其行。咸通十一年,何全皞為軍眾所殺,推允忠為帥。時僖宗為普王,即降詔遙領節度,授允忠左散騎常侍、兼御史中丞,充節度觀察留後。不數月,轉檢校工部尚書、魏州大都督府長史、充魏博節度觀察等使。累加至檢校司空、同平章事。乾符元年十一月卒,年六十一。累贈太尉。



 子簡,自允忠初授戎帥,
 便為節度副使。乾符初,累官至檢校工部尚書。允忠卒,即起復為節度觀察留後。逾月,加檢校右僕射。其後累加至侍中,封昌黎郡王。



 賊巢之亂,諸葛爽受其偽命河陽節度使。時僖宗在蜀,寇盜蜂起,簡據有六州,甲兵強盛,竊懷僭亂之志,且欲啟其封疆,乃舉兵攻河陽,爽棄城而走。簡遂留兵保守,因北掠邢、洺而歸,遂移軍攻鄆。鄆帥曹全晸出戰,為簡所敗,死之。鄆將崔君裕收合殘眾,保鄆州。簡進攻其城,半年不下,河陽復為諸葛爽所
 襲。簡因欲先討君裕,次及河陽,乃舉兵至鄆,君裕請降。尋移軍復攻河陽,行及新鄉,為爽軍逆擊,敗之。簡單騎奔回,憂憤。疽發背而卒,時中和元年十一月也。



 樂彥禎,魏州人也。父少寂,歷澶、博、貝三州刺史,贈工部尚書。彥禎少為本州軍校。韓簡之領節旄也,以彥禎為馬步軍都虞候,轉博州刺史。下河陽,走諸葛爽。有功,遷澶州刺史。簡再討河陽之敗也。彥禎以一軍先歸,魏人遂共立之。朝廷尋授檢校工部尚書,知魏博留後。俄加
 戶部尚書,充節度觀察處置等使。中和四年,累加至尚書左僕射、同平章事。僖宗自蜀回,加開府儀同三司,冊拜司徒。



 彥禎志滿驕大,動多不法。一旦征六州之眾,板築羅城,約河門舊堤,周八十里,月餘而畢,人用怨咨。



 又其子從訓天資悖逆。王鐸自滑移鎮滄州,過魏郊,從訓見其女妓,利之,先伏兵於漳南高雞泊,俟鐸之至,圍而害之,掠其所有。時朝廷微弱,不能詰。魏人素知鐸名望,議者惜之,而罪從訓。從訓又召亡命之徒五百餘輩,出
 入臥內,號為「子將」,委以腹心。軍人籍籍,各有異議。從訓聞而忌之,易服遁出,止於近縣。彥禎因命為六州都指揮使。未幾,又兼相州刺史。到任之後,般輦軍器,取索錢帛,使人來往,交午塗路,軍府疑貳。



 彥禎危憤而卒,眾推都將趙文弁知留後事。從訓自相州領兵三萬餘人至城下,文弁按兵不出。眾懷疑懼,復害文弁,推羅弘信為帥。弘信以兵出戰,敗之。從訓招集餘眾,次於洹水。弘信遣將程公佐領兵討擊,大敗之,梟從訓首於軍門,時文
 德元年春也。



 羅弘信,字德孚,魏州貴鄉人。曾祖秀,祖珍,父讓,皆為本州軍校。弘信少從戎役,歷事節度使韓簡、樂彥禎。光啟末,彥禎子從訓忌牙軍,出居於外,軍眾廢彥禎,推趙文弁權主軍州事。眾復以為不便,因推弘信為帥。先是,有鄰人密謂弘信曰:「某嘗夜遇一白須翁,相告云,君當為土地主。如是者再三。」弘信竊異之。及廢文弁,軍人聚呼曰:「孰願為節度使者?」弘信即應之曰:「白須翁早以命我。」
 眾乃環而視之,曰:「可也。」由是立之。僖宗聞之,文德元年四月,詔加工部尚書,權知節度留後。七月,復加金紫光祿大夫、檢校尚書右僕射,充魏博節度觀察處置等使。龍紀中,加檢校司空、同平章事,封豫章郡公。



 乾寧中,硃全忠急攻兗鄆,硃瑄求援於太原。太原發軍,假道於魏,令大將李存信屯莘縣。存信御軍無法,侵魏之芻牧,弘信不平之。全忠復遣人謂之曰:「太原志吞河朔,回戈之日,貴道堪憂。」弘信乃托好於汴,出師三萬攻存信,敗之。
 太原怒,舉兵攻魏,營於觀音門外。汴將葛從周援之,屯於洹水。李克用子落落時為鐵林軍使,為從周所擒,乃退歸。自是太原之師,每歲侵擾相、魏,魏人患之。



 硃全忠方事兗鄆,懼弘信離貳,每歲時賂遺,必卑辭厚禮答貺。全忠對魏使北面拜而受之,曰:「六兄比予倍年已上,兄弟之國,安得以常鄰遇之。」弘信以為厚己,亦推心焉。弘信累官至檢校太師、守侍中、臨清王。光化元年九月卒,年六十三,贈太師,追封北平王,謚曰莊肅。子威。



 威,字端
 己。文德初,授左散騎常侍,充天雄軍節度副使。自龍紀至乾寧,十年之中,累加官爵。弘信卒,襲父位為留後,朝廷從而命之。天復末,累加至檢校太傅、兼侍中、長沙王。天祐初,授檢校太尉、守侍中,進封鄴王,賜號「忠勤宣力致理功臣」。



 魏之牙中軍者,自至德中,田承嗣盜據相、魏、澶、博、衛、貝等六州,召募軍中子弟置之部下,遂以為號。皆豐給厚賜,不勝驕寵。年代浸遠,父子相襲,親黨膠固。其兇戾者,強買豪奪,逾法犯令,長吏不能禁。變易主帥,
 有同兒戲,如史憲誠、何進滔、韓君雄、樂彥禎,皆為其所立。優獎小不如意,則舉族被害。威懲其往弊,雖以貨賂姑息,而心銜之。



 威嗣世之明年,正月,幽州劉仁恭擁兵十萬,謀亂河朔,進陷貝州,長驅攻魏。威求援於汴。硃全忠遣將李思安屯於洹水。葛從周自邢、洺引軍入魏。燕將劉守文、單可及攻汴軍於內黃。思安逆戰,大敗之,乘勝追躡。從周出會掩擊,復敗燕軍,斬首三萬。三年,威引汴軍攻滄州以報之。自是,威感全忠援助之恩,合從景
 附。



 天祐二年七月十三日夜,牙軍裨校李公佺作亂,威僅以身免。公佺出奔滄州。自是愈懼,遣使求援於全忠,密謀破之。全忠遣李思安會魏博軍,再攻滄州。全忠女妻威子廷規,先是卒。全忠遣長直軍校馬嗣勛選兵千人,密於輿中實兵甲入魏,言助女葬事。三年正月五日,嗣勛至,全忠親率大軍濟河,言視行營於滄景。威欲因而出迎,至期,即假全忠帳下銳卒入而夾攻之。牙軍頗疑,堅請不出。威恐洩其事,慰納之。是月十四日夜,率廝
 養百十輩,與嗣勛合攻之。時宿於牙城者千人,遲明殺之殆盡;凡八千家,皆破其族。魏軍攻滄州者,在歷亭聞有變,其將史仁遇擁之,保於高唐。六州之內,皆為讎敵,累月平之。威仕梁數年後卒,年三十四,位至守太師、兼中書令,贈尚書令,謚曰貞壯。



 威性明敏,達於吏道。伏膺儒術,招納文人,聚書至萬卷。每花朝月夕,與賓佐賦詠,甚有情致。錢塘人羅隱者,有當世詩名,自號「江東生」。威遣使賂遺,敘其宗姓,推為叔父。隱亦集其詩寄之。威酷
 嗜其作,目己所為曰《偷江東集》,凡五卷,今鄴中人士諷詠之。



 史臣曰:魏、鎮、燕三鎮,不能制之也久矣。兵強地廣,合從連衡。爵命雖假於朝廷,群臣自謀於元帥。如史憲誠等五家,其初皆因此而得之,其後亦因此而失之。蓋不知取之以權,守之以仁,則遠矣。若善繼者,史氏、羅氏之二子有焉,其餘不足觀也。



 贊曰:逆取順守,古亦有之。如其逆守,滅亡必隨。史、何、韓、
 樂,世數盛衰。足以為鑒,念茲在茲。



\end{pinyinscope}