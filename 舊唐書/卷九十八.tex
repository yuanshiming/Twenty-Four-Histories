\article{卷九十八}

\begin{pinyinscope}

 ○蘇味道李嶠崔融
 盧藏用徐彥伯



 蘇味道,趙州欒城人也。少與鄉人李嶠俱以文辭知名,時人謂之蘇李。弱冠,本州舉進士。累轉咸陽尉。吏部侍郎裴行儉先知其貴,甚加禮遇,及徵突厥阿史那都支,
 引為管記。孝敬皇帝妃父裴居道再登左金吾將軍,訪當時才子為謝表,托於味道,援筆而成,辭理精密,盛傳於代。



 延載初,歷遷鳳閣舍人、檢校鳳閣侍郎、同鳳閣鸞臺平章事,尋加正授,證聖元年,坐事,出為集州刺史,俄召拜天官侍郎。聖歷初,遷鳳閣侍郎、同鳳閣鸞臺三品。味道善敷奏,多識臺閣故事,然而前後居相位數載,竟不能有所發明,但脂韋其間,茍度取容而已。嘗謂人曰「處事不欲決斷明白,若有錯誤,必貽咎譴,但模棱以持
 兩端可矣。」時人由是號為「蘇模棱」。



 長安中,請還鄉改葬其父,優制令州縣供其葬事。味道因此侵毀鄉人墓田,役使過度,為憲司所劾,左授坊州刺史。未幾,除益州大都督府長史。神龍初,以親附張易之,昌宗貶授郿州剌史。俄而復為益州大都督府長史,未行而卒,年五十八,贈冀州刺史。味道與其弟太子洗馬味玄甚相友愛,味玄若請托不諧,輒面加凌折,味道對之怡然,不以為忤,論者稱焉。有文集行於代。



 李嶠,趙州贊皇人,隋內史侍郎元操從曾孫也。代為著姓,父鎮惡,襄城令。嶠早孤,事母以孝聞。為兒童時,夢有神人遺之雙筆,自是漸有學業。弱冠舉進士,累轉監察御史。時嶺南邕、嚴二州首領反叛,發兵討擊,高宗令嶠往監軍事。嶠乃宣朝旨,特赦其罪,親入獠洞以招諭之。叛者盡降,因罷兵而還,高宗甚嘉之。累遷給事中。時酷吏來俊臣構陷狄仁傑、李嗣真、裴宣禮等三家,奏請誅之,則天使嶠與大理少卿張德裕、侍御史劉憲覆其獄。
 德裕等雖知其枉,懼罪,並從俊臣所奏,嶠曰:「豈有知其枉濫而不為申明哉!孔子曰:『見義不為,無勇也。』」乃與德裕等列其枉狀,由是忤旨,出為潤州司馬。詔入,轉鳳閣舍人。則天深加接待,朝廷每有大手筆,皆特令嶠為之。



 時初置右御史臺,巡按天下,嶠上疏陳其得失曰:



 陛下創置右臺,分巡天下,察吏人善惡,觀風俗得失,斯政途之綱紀,禮法之準繩,無以加也。然猶有未折衷者,臣請試論之。夫禁綱尚疏,法令宜簡,簡則法易行而不煩雜,
 疏則所羅廣而無苛碎。竊見垂拱二年諸道巡察使所奏科目,凡有四十四件,至於別準格敕令察訪者,又有三十餘條。而巡察使率是三月已後出都,十一月終奏事,時限迫促,簿書填委,晝夜奔逐,以赴限期。而每道所察文武官,多至二千餘人,少者一千已下,皆須品量才行,褒貶得失,欲令曲盡行能,則皆不暇。此非敢墮於職而慢於官也,實才有限而力不及耳。臣望量其功程,與其節制,使器周於用,力濟於時,然後進退可以責成,得
 失可以精核矣。



 又曰:



 今之所察,但準漢之六條,推而廣之,則無不包矣。無為多張科目,空費簿書。且朝廷萬機,非無事也,機事之動,恆在四方,是故冠蓋相望,郵驛繼踵。今巡使既出,其他外州之事,悉當委之,則傳驛大減矣。然則御史之職,故不可得閑,自非分州統理,無由濟其繁務。有大小相兼,率十州置御史一人,以周年為限,使其親至屬縣,或入閭里,督察奸訛,觀採風俗,然後可以求其實效,課其成功。若此法果行,必大裨政化。且御
 史出持霜簡,入奏天闕,其於勵已自修,奉職存憲,比於他吏,可相百也。若其按劾奸邪,糾擿欺隱,比於他吏,可相十也。陛下試用臣言,妙擇賢能,委之心膂,假溫言以制之,陳賞罰以勸之,則莫不盡力而效死矣。何政事之不理,何禁令之不行,何妖孽之敢興?



 則天善之。乃下制分天下為二十道,簡擇堪為使者。會有沮議者,竟不行。尋知天官侍郎事,遷麟臺少監。



 聖歷初,與姚崇偕遷同鳳閣鸞臺平章事,俄轉鸞臺侍郎,依舊平章事,兼修國
 史。久視元年,嶠舅天官侍郎張錫入知政事,嶠轉成均祭酒,罷知政事及修史,舅甥相繼在相位,時人榮之。嶠尋檢校文昌左丞、東都留守。長安三年,嶠復以本官平章事,尋知納言事。明年,遷內史。嶠後固辭煩劇,復拜成均祭酒,平章事如故。



 長安末,則天將建大像於白司馬阪,嶠上疏諫之,其略曰:「臣以法王慈敏,菩薩護持,唯擬饒益眾生,非要營修土木。伏聞適像,稅非戶口,錢出僧尼,不得州縣祗承,必是不能濟辦,終須科率,豈免勞擾!
 天下編戶,貧弱者眾,亦有傭力客作以濟餱糧,亦有賣舍貼田以供王役。造像錢見有一十七萬餘貫,若將散施,廣濟貧窮,人與一千,濟得一十七萬餘戶。拯饑寒之弊,省勞役之勤,順諸佛慈悲之心,沾聖君亭育之意,人神胥悅,功德無窮。」疏奏不納。



 中宗即位,嶠以附會張易之兄弟,出為豫州刺史。未行,又貶為通州刺史。數月,徵拜吏部侍郎,封贊皇縣男。無幾,遷吏部尚書,進封縣公。神龍二年,代韋安石為中書令。初,嶠在吏部時,志欲曲
 行私惠。冀得復居相位奏置員外官數千人。至是官僚倍多,府庫減耗,乃抗表引咎辭職,並陳利害十餘事。中宗以嶠昌言時政之失,輒請罷免,手制慰諭而不允,尋令復居舊職。三年,又加修文館大學士,監修國史,封趙國公。景龍三年,罷中書令,以特進守兵部尚書、同中書門下三品。



 睿宗即位,出為懷州刺史,尋以年老致仕。初,中宗崩,嶠密表請處置相王諸子,勿令在京。及玄宗踐祚,宮內獲其表,以示侍臣。或請誅之,中書令張說曰:「嶠
 雖不辯逆順,然亦為當時之謀,吠非其主,不可追討其罪。」上從其言,乃下制曰:「事君之節,危而不變,為臣則忠,貳乃無赦。特進、趙國公李嶠,往緣宗、韋弒逆,朕恭行戡定,揖讓之際,天命有歸,嶠有窺覦,不知逆順,狀陳詭計,朕親覽焉。以其早負辭學,累居臺輔,妨而莫言,,特掩其惡。今忠邪既辨,具物惟新,賞罰倘乖,下人安勸?雖經赦令,猶宜放斥,矜其老疾,俾遂餘生,宜聽隨子虔州刺史暢赴任。」尋起為盧州別駕而卒。有文集五十卷。



 崔融,齊州全節人。初,應八科舉擢第。累補宮門丞,兼直崇文館學士。中宗在春宮,制融為侍讀,兼侍屬文,東朝表疏,多成其手。聖歷中,則天幸嵩岳,見融所撰《啟母廟碑》,深加嘆美,及封禪畢,乃命融撰朝觀碑文。自魏州司功參軍擢授著作佐郎,尋轉右史。聖歷二年,除著作郎,仍兼右史內供奉。四年,遷鳳閣舍人。久視元年,坐忤張昌宗意,左授婺州長史。頃之,昌宗怒解,又請召為春官郎中,知制誥事。長安二年,再遷鳳閣舍人。三年,兼修國
 史。時有司表稅關市,融深以為不可,上疏諫曰:



 伏見有司稅關市事條,不限工商,但是行人盡稅者,臣謹按《周禮》九賦,其七日「關市之賦」。竊惟市縱繁巧,關通末游,欲令此徒止抑,所以咸增賦稅。臣謹商度今古,料量家國,竊將為不可稅。謹件事跡如左,伏惟聖旨擇焉。



 往古之時,淳樸未散,公田籍而不稅,關防譏而不征。中代已來,澆風驟進,桑麻疲弊,稼穡辛勤,於是各徇通財,爭趨作巧,求徑捷之欲速,忘歲計之無餘。遂使田萊日荒,倉廩
 不積,蠶織休廢。弊縕闕如,饑寒猥臻,亂離斯起。先王懲其若此,所以變古隨時,依本者恆科,占末者增稅。夫關市之稅者,謂市及國門,關門者也,唯斂出入之商賈,不稅來往之行人。今若不論商人,通取諸色,事不師古,法乃任情。悠悠末代,於何瞻仰;濟濟盛朝,自取嗤笑。雖欲憲章姬典,乃是違背《周官》。臣知其不可者一也。



 臣謹案《易》《系辭》稱:「庖羲氏沒,神農氏作,日中為市,致天下之人,聚天下之貨,交易而退,各得其所。」《班志》亦云:「財者,帝王
 聚人守位,養成群生,奉順天德,理國安人之本也。仕農工商,四人有業。學以居位曰仕,闢士殖穀曰農,作巧成器曰工,通財鬻貨曰商。聖王量能授事,四人陳力受職。」然則四人各業久矣。今復安得動而搖之!蕭何云:「人情一定,不可復動。」班固又云:曹參相齊,齊國安集,大稱賢相。參去,屬其後相曰:「以齊獄市為寄,慎勿擾也。」後相曰:「理無大於此者乎?」參曰:「不然。夫獄市者,所以並容也,今若擾之,奸人安所容乎?吾是以先之。」夫獄市,兼受善惡。
 若窮極,奸人無所容竄;奸人無所容竄,久且為亂。秦人極刑而天下叛,孝武峻法而刑獄繁,此其效也。老子曰:「我無為而人自化,我好靜而人自正。」參欲以道化其本,不欲擾其末。臣知其不可者二也。



 四海之廣,九州之雜。關必據險路,市必憑要津。若乃富商大賈,豪宗惡少,輕死重義,結黨連群,喑鳴則彎弓,睚眥則挺劍。小有失意,且猶如此,一旦變法,定是相驚。乘茲困窮,或致騷動,便恐南走越,北走胡,非唯流逆齊人,亦自攪亂殊俗。又如
 邊徼之地,寇賊為鄰,興胡之旅,歲月相繼,倘同科賦,致有猜疑,一從散亡,何以制禁?求利雖切,為害方深。而有司上言,不識大體,徒欲益帑藏,助軍國,殊不知軍國益擾,帑藏逾空。臣知其不可者三也。



 孟軻又云:「古之為關也,將以禦暴;今之為關也,將以為暴。」今行者皆稅,本末同流。且如天下諸津,舟航所聚,旁通巴、漢,前指閩、越,七澤十藪,三江五湖,控引河洛,兼包淮海。弘舸巨艦,千軸萬艘,交貿往還,昧旦永日。今若江津河口,置鋪納稅,納
 稅則檢覆,檢覆則遲留。此津才過,彼鋪復止,非唯國家稅錢,更遭主司僦賂。船有大小,載有少多,量物而稅,觸途淹久。統論一日之中,未過十分之一,因此壅滯,必致籲嗟。一朝失利,則萬商廢業,萬商廢業,則人不聊生。其間或有輕訬任俠之徒,斬龍刺蛟之黨,鄱陽暴謔之客,富平悍壯之夫,居則藏鏹,出便竦劍。加之以重稅,因之以威脅,一旦獸窮則搏,鳥窮則攫,執事者復何以安之哉?臣知其不可者四也。



 五帝之初,不可詳已;三王之後,
 厥有著雲;秦、漢相承,典章大備至如關市之稅,史籍有文。秦政以雄圖武力,舍之而不用也;漢武以霸略英才,去之而勿取也。何則?關為御暴之所,市為聚人之地,稅市則人散,稅關則暴興,暴興則起異圖,人散則懷不軌。夫人心莫不背善而樂禍,易動而難安。一市不安,則天下之市心搖矣;一關不安,則天下之關心動矣。況澆風久扇,變法為難,徒欲禁末流、規小利,豈知失玄默、亂大倫。魏、晉眇小,齊、隋齷齪,亦所不行斯道者也。臣知其不
 可者五也。



 今之所以稅關市者,何也?豈不以國用不足,邊寇為虞,一行斯術,冀有殷贍然也!微臣敢借前箸以籌之。伏惟陛下當聖期,御玄籙,沉璧於洛,刻石於嵩,鑄寶鼎以窮奸,坐明堂而布政,神化廣洽,至德潛通。東夷暫驚,應時平殄;南蠻才動,計日歸降。西域五十餘國,廣輸一萬餘里,城堡清夷,亭堠靜謐。比為患者,唯苦二蕃。今吐蕃請命,邊事不起,即目雖尚屯兵,久後疑成馳柝。獨有默啜,假息孤恩,惡貫禍盈,覆亡不暇。征役日已省
 矣,繁費日已稀矣,然猶下明制,遵太樸,愛人力,惜人財,王侯舊封,妃主新禮,所有支料,咸令減削。此陛下以躬率先,堯、舜之用心也。且關中、河北,水旱數年,諸處逃亡,今始安輯,,倘加重稅,或慮相驚。況承平歲積,薄賦日久,俗荷深恩,人知自樂。卒有變法,必多生怨,生怨則驚擾,驚擾則不安,中既不安,外何能禦?文王曰:「帝王富其人,霸王富其地,理國若不足,亂國若有餘。」古人有言:「帝王藏於天下,諸侯藏於百姓,農夫藏於庾,商賈藏於篋。」惟
 陛下詳之。必若師興有費,國儲多窘,即請倍算商客,加斂平人。如此則國保富強,人免憂懼,天下幸甚。臣知其不可者六也。



 陛下留神系表,屬想政源,冒茲炎熾,早朝晏坐。一日二日,機務不遺,先天後天,虛心密應。時政得失,小子何知,率陳瞽辭,伏紙惶懼。



 疏奏,則天納之,乃寢其事。



 四年,除司禮少卿,仍知制誥。時張易之兄弟頗招集文學之士,融與納言李嶠、鳳閣侍郎蘇味道、麟臺少監王紹宗等俱以文才降節事之。及易之伏誅,融左授
 袁州刺史。尋召拜國子司業,兼修國史。神龍二年,以預修《則天實錄》成,封清河縣子,賜物五百段,璽書褒美。融為文典麗,當時罕有其比,朝廷所須《洛出寶圖頌》、《則天哀冊文》及諸大手筆,並手敕付融。撰哀冊文,用思精苦,遂發病卒,時年五十四。以侍讀之恩,追贈衛州刺史,謚曰文。有集六十卷。二子禹錫、翹,開元中,相次為中書舍人。



 盧藏用,字子潛,度支尚書承慶之侄孫也。父璥,有名於
 時,官至魏州司馬。藏用少以辭學著稱。初舉進士選,不調,乃著《芳草賦》以見意。尋隱居終南山,學闢谷、練氣之術。



 長安中,徵拜左拾遺。時則天將營興泰宮於萬安山,藏用上疏諫曰:



 臣愚雖不達時變,竊嘗讀書,見自古帝王之跡眾矣。臣聞土階三尺,茅茨不翦,採椽不斫者,唐堯之德也;卑宮室,菲飲食,盡力於溝洫者,大禹之行也;惜中人十家之產,而罷露臺之制者,漢文之明也。並能垂名無窮,為帝皇之烈。豈不以克念徇物,博施濟眾,以
 臻於仁恕哉!今陛下崇臺邃宇,離宮別館,亦已多矣。更窮人之力以事土木,臣恐議者以陛下為不憂人、務奉已也。



 且頃歲已來,雖年穀頗登,而百姓未有儲蓄。陛下西幸東巡,人未休息,土木之役,歲月不空。陛下不因此時施德布化,復廣造宮苑,臣恐人未易堪。今左右近臣,多以順意為忠;朝廷具僚,皆以犯忤為患。至今陛下不知百姓失業,亦不知左右傷陛下之仁也。臣聞忠臣不避死亡之患,以納君於仁;明主不惡切直之言,以
 垂名千載。陛下誠能發明恕之制,以勞人為辭,則天下必以陛下為惜人力而苦己也。小臣固陋,不識忌諱,敢冒死上聞。乞下臣此章,與執事者議其可否,則天下幸甚。



 神龍中,累轉起居舍人,兼知制誥,俄遷中書舍人。藏用常以俗多拘忌,有乖至理,乃著《析滯論》以暢其事,辭曰:



 客曰:「天道玄微,神理幽化,聖人所以法象,眾庶由其運行。故大撓造甲子,容成著律歷,黃公裁變,玄女啟謨,八門御時,六神直事。從之者則兵強國富,違之者則將
 弱朝危,有同影響,若合符契。先生亦嘗聞之乎?



 主人曰:「何為其然也?子所謂曲學所習,曘昧所守,徒識偏方之詭說,未究亨衢之通論。蓋《易》曰「先天不違」,《傳》稱「人神之主」。範圍不過,三才所以虛中;進退非邪,百王所以無外。故曰:「國之將興聽於人,將亡聽於神。」又曰:「禍福無門,唯人所召。人無釁焉,妖不自作。」由是言之,得喪興亡,並關人事;吉兇悔吝,無涉天時。且皇天無親,唯德是輔,為不善者,天降之殃。高宗修德,桑穀以變;宋君引過,法星退
 舍,此天道所以從人者也。古之為政者,刑獄不濫則人壽,賦斂蠲省則人富,法令有常則國靜,賞罰得中則兵強。所以禮者,士之所歸,賞者,士之所死,禮賞不倦,則士爭先。茍違此途,雖卜時行刑,擇日出令,必無成功矣。自叔世遷訛,俗多徼幸,競稱怪力,爭誦詭言,屈政教而就孤虛,棄信賞而從推步。附會前史,變易舊經,依托空文,以為徵據。覆軍敗將者,則隱秘無聞;偶同幸中者,則共相文飾。豈唯德之增惑,亦乃學人自是。嗚呼,習俗訛謬,
 一至此焉!



 昔者,甲子興師,非成功之日;往亡用事,異制勝之辰。人事茍修,何往不濟?至若環城自守,接陣重圍,無闕地形,不乖天道。若兵強將智,粟積城堅,雖復屢轉魁剛,頻移太歲,坐推白虎,行計貪狼,自符難斗之祥,多貽蟻附之困。故曰,任賢使能,則不時日而事利;明法審令,則不卜筮而事吉;養勞賞功,則不禱祠而得福。此所謂天時不如地利,地利不如人和。太公犯雨,逆天時也,韓信背水,乖地利也,並存人事,俱成大業。削樹而斬龐
 涓,舉火而屠張郃,未必暗同歲德,冥會日游,俱運三門,並占四殺。杜郵齒劍,抑唯計沮,垓下悲歌,實階剚印。若以並資厭勝,不事良圖,則長平盡坑,固須恆濟,襄城無噍,亦可常保。是知拘而多忌,終喪大功;百姓與能,必遺小數。金雞玉鶴,方為楚國之殃;《萬畢》、《枕中》,適構淮南之禍。刻符指盜,反更亡身;被發邀神,翻招夷族。嗟乎,威斗赭鞭,不禳赤伏之運;築城斷罔,何救素靈之哭!火災不驗,裨灶無力以窺天;超乘階兇,王孫取監於觀德。九徵
 九變,是曰長途;人謀鬼謀,良歸有道。此並經史陳跡,賢聖通規,仁遠乎哉,詎宜滯執?



 客乃蹙然避席曰:「鄙人困蒙,不階至道,請事斯語,歸於正途。而今而後,焚蓍龜,毀律歷,廢六合,斥五行,浩然清慮,則將奚若?」答曰:「此所謂過猶不及也。夫甲子所以配日月,律歷所以通歲時,金木所以備法象,蓍龜所以筮吉兇。聖人以此神明德行,輔助謀猷,存之則協贊成功,執之則凝滯於物。消息之義,其在茲乎」!客於是循墻匍匐,帖然無氣,口去心醉,不
 知所以答矣。



 景龍中,為吏部侍郎。藏用性無挺特,多為權要所逼,頗隳公道。又遷黃門侍郎,兼昭文館學士,轉工部侍郎、尚書右丞。先天中,坐托附太平公主,配流嶺表。開元初,起為黔州都督府長史,兼判都督事,未行而卒,年五十餘。有集二十卷。



 藏用工篆隸,好琴棋,當時稱為多能之士。少與陳子昂、趙貞固友善,二人並早卒,藏用厚撫其子,為時所稱。然初隱居之時,有貞儉之操,往來於少室、終南二山,時人稱為「隨駕隱士」;及登朝,趑趄
 詭佞,專事權貴,奢靡淫縱,以此獲譏於世。



 徐彥伯,兗州瑕丘人也。少以文章擅名,河北道安撫大使薛元超表薦之,對策擢第,累轉蒲州司兵參軍。時司戶韋暠善判事,司士李亙工於翰札,而彥伯以文辭雅美,時人謂之「河中三絕」。



 彥伯聖歷中累除給事中,時王公卿士多以言語不慎,密為酷吏周興、來俊臣等所陷,彥伯乃著《樞機論》以誡於代,其辭曰:



 《書》曰:「唯口起羞,惟甲胃起戎。」又云:「齊乃位,度乃口。」《易》曰:「慎言語,節飲食。」又
 云:「出其言善,千里應之;出其言不善,千里違之。」《禮》亦云:「可言也,不可行也,君子不言也;可行也,不可言也,君子不行也。」嗚呼!先聖知言之為大也,知言之為急也,精微以勸之,典謨以告之,禮經以防之。守名教者,何可不修其詁訓而服其糟粕乎?故曰:「言語者,君子之樞機,動則物應,物應則得失之兆見也。得之者江海比鄰,失之者肝膽楚、越,然後知否泰榮辱,系於言乎!



 夫言者,德之柄也,行之主也,志之端也,身之文也,既可以濟身,亦可以覆
 身。故中庸鏤其心,右階銘其背,南容復於白圭,箕子疇於《洪範》,良有以也。是以掎摭瑕玷,參詳躁競,審無常以階亂,將不密以致危。利生於口,森然覆邦之說;道不由衷,變彼如簧之刺。。可不懼之哉!其有識暗邪正,慮微形朕,破金湯之龠,封禍亂之根,用詀讘為全計,以號詉為令德。至若梧宮問答,荊、齊所以奔命;韓、魏加肘,智伯所以危殘。蔡侯繩息媯也,亟招甲兵之罰;鄭曼圖宗卿也,而受鼎鑊之誅。史遷輕議,終下蠶室;張紘說,更齒龍
 淵。凡此過言,其流匪一。或穢猶糞土,或動成刀劍,或茍且其心,或脂膏其吻。挾邪作蠱,守之而不懈;往輒破的,去之而彌遠。亦可異韓廬聚音,釐也群吠,得死為幸,何循名之立乎?雖復伯玉沮顏,追謝於元凱,蔣濟貽恨,失譽於王陵,犀首沒齒於季章,曹瞞齚舌於劉主,當何及哉!孔子曰:「予欲無言。」又云:「終身為善一言敗之,惜也。」老子亦云:「多言數窮。」又云「聰明深察而近於死者,議人者也。」何聖人之深思偉慮,杜漸防萌之至乎!



 夫不可言而
 言者曰狂,可言而不言者曰隱。鉗舌拱默,曷通彼此之懷;括囊而處,孰啟謨明之訓?則上言者,下聽也;下言者,上用也。睿哲之言,猶天地也,人覆燾而生焉;大雅之言,猶鐘鼓也,人考擊而樂焉。作以龜鏡,姬公之言也;出為金石,曾子之言也;存其家邦,國僑之言也;立而不朽,臧孫之言也。是謂德音,詣我宗極,滿於天下,貽厥後昆。殷宗甘之於酒醴,孫卿諭之以琴瑟,闕里重於四時,郢都輕其千乘。豈不韙哉,豈不休哉!但楙探世猷,克念丕訓,
 審思而應,精慮而動。謀其心以後發,擇其交以後談,不蹙趨於非黨,不屏營於詭遇。非先王之至德不敢行,非先王之法言不敢道,翦其諜諜之緒,撲其炎炎之勢。自然介爾景福,錫茲純嘏,則悔吝何由而生,怨惡何由而至哉?孔子曰:「終日行,不遺已患;終日言,不遺已憂。」如此乃可以言也。戒之哉,戒之哉!



 神龍元年,遷太常少卿,兼修國史,以預修《則天實錄》成,封高平縣子,賜物五百段。未幾,出為衛州刺史,以善政聞,璽書勞勉。俄轉蒲州刺
 史,入為工部侍郎,尋除衛尉卿,兼昭文館學士。景龍三年,中宗親拜南郊,彥伯作《南郊賦》以獻,辭甚典美。景雲初,加銀青光祿大夫,遷右散騎常侍、太子賓客,仍兼昭文館學士。先天元年,以疾乞骸骨,許之。開元二年卒。彥伯事寡嫂甚謹,撫諸侄同於己子。自晚年屬文,好為強澀之體,頗為後進所效焉。有文集二十卷,行於時。



 史臣曰:才出於智,行出於性。故文章巧拙,由智之深淺也;行義詭實,由性之善惡也。然則智性稟之於氣,不可
 使之強也。蘇味道、李嶠等,俱為輔相,各處穹崇。觀其章疏之能,非無奧贍;驗以弼諧之道,罔有貞純。故狄仁傑有言曰:「蘇、李足為文吏矣。」得非齷齪者乎!模棱之病,尤足可譏。崔融、盧藏用、徐彥伯等,文學之功,不讓蘇、李,知有守常之道,而無應變之機。規諫之深,崔比盧、徐,稍為優矣。



 贊曰:房、杜、姚、宋,俱立大功。咸以二族,譚為美風。蘇、李文學,一代之雄。有慚輔弼,稱之豈同。凡人有言,未必有德。
 崔與盧、徐,皆攻翰墨。文雖堪尚,義無可則。備位守常,斯言罔忒。



\end{pinyinscope}