\article{卷九十七}

\begin{pinyinscope}

 ○婁
 師德王孝傑唐休璟張仁願薛訥王晙



 婁師德,鄭州原武人也。弱冠,進士擢第,授江都尉。揚州長史盧承業奇其才,嘗謂之曰:「吾子臺輔之器,當以子
 孫相托,豈可以官屬常禮待也?」



 上元初,累補監察御史。屬吐蕃犯塞,募猛士以討之,師德抗表請為猛士。高宗大悅,特假朝散大夫,眾軍西討,頻有戰功,遷殿中侍御史,兼河源軍司馬,並知營田事。天授初,累授左金吾將軍,兼檢校豐州都督,仍依舊知營田事。則天降書勞曰:「卿素積忠勤,兼懷武略,朕所以寄之襟要,授以甲兵。自卿受委北陲,總司軍任,往還靈、夏,檢校屯田,收率既多,京坻遽積。不煩和糴之費,無復轉輸之艱,兩軍及北鎮
 兵數年咸得支給。勤勞之誠,久而彌著,覽以嘉尚,欣悅良深。」



 長壽元年,召拜夏官侍郎、判尚書事。明年,同鳳閣鸞臺平章事。則天謂師德曰:「王師外鎮,必藉邊境營田,卿須不憚劬勞,更充使檢校。」又以為河源、積石、懷遠等軍及河、蘭、鄯、廓等州檢校營田大使。稍遷秋官尚書。萬歲登封元年,轉左肅政御史大夫,仍並依舊知政事。證聖元年,吐蕃寇洮州,令師德與夏官尚書王孝傑討之,與吐蕃大將論飲陵、贊婆戰於素羅汗山,官軍敗績,師
 德貶授原州員外司馬。



 萬歲通天二年,入為鳳閣侍郎、同鳳閣鸞臺平章事。是歲,兼檢校右肅政御史大夫,仍知左肅政臺事,以與王懿宗、狄仁傑分道安撫河北諸州。神功元年,拜納言,累封譙縣子。尋詔師德充隴右諸軍大使,仍檢校河西營田事。聖歷二年,突厥入寇,復令檢校並州長史,仍充天兵軍大總管。是歲九月卒,贈涼州都督,謚曰貞。



 初,狄仁傑未入相時,師德嘗薦之,及為宰相,不知師德薦已,數排師德,令充外使。則天嘗出師
 德舊表示之,仁傑大慚,謂人曰:「吾為婁公所含如此,方知不逮婁公遠矣。」師德頗有學涉,器量寬厚,喜怒不形於色。自專綜邊任,前後三十餘年,恭勤接下,孜孜不怠。雖參知政事,深懷畏避,竟能以功名始終,甚為識者所重。



 王孝傑,京兆新豐人也。高宗末,為副總管,從工部尚書劉審禮西討吐蕃,戰於大非川,為賊所獄。吐蕃贊普見孝傑,垂泣曰:「貌類吾父。」厚加敬禮,由是免死,尋得歸。則
 天時,累遷右鷹揚衛將軍。孝傑久在吐蕃中,悉其虛實。長壽元年,為武威軍總管,與左武衛大將軍阿史那忠節率眾以討吐蕃,乃克復龜茲、于闐、疏勒、碎葉四鎮而還。則天大悅,謂侍臣曰:「昔貞觀中貝綾,得此蕃城,其後西陲不守,並陷吐蕃。今既盡復於舊,邊境自然無事。孝傑建斯功效,竭此款誠,遂能裹足徒行,身與士卒齊力。如此忠懇,深是可嘉。」乃拜孝傑為左衛大將軍。明年,遷夏官尚書、同鳳閣鸞臺三品,封清源男。延載初,入為瀚
 海道行軍總管,餘如故。證聖初,又為朔方道總管,尋坐與吐蕃戰敗免官。



 萬歲通天年,契丹李盡忠、孫萬榮反叛,復詔孝傑白衣起為清邊道總管,統兵十八萬以討之。孝傑軍至東峽石谷遇賊,道隘,虜甚眾,孝傑率精銳之士為先鋒,且戰且前,及出谷,布方陣以捍賊。後軍總管蘇宏暉畏賊眾,棄甲而遁。孝傑既無後繼,為賊所乘,營中潰亂,孝傑墮穀而死,兵士為賊所殺及奔踐而死殆盡。時張說為節度管記,馳奏其事。則天問孝傑敗亡
 之狀,說曰:「孝傑忠勇敢死,乃誠奉國,深入寇境,以少御眾,但為後援不至,所以致敗。」於是追贈孝傑夏官尚書,封耿國公。拜其子無擇為朝散大夫。遣使斬宏暉以徇。使未至幽州,而宏暉已立功贖罪,竟免誅。開元中,無擇官至左驍衛將軍,以恩例贈孝傑特進。



 唐休璟,京兆始平人也。曾祖規,周驃騎大將軍、安邑縣公。祖宗,隋大業末為朔方郡丞。時為梁師都舉兵,將據城,宗抗節不從,乃為所害。



 休璟少以明經擢第。永徽中,
 解褐吳王府典簽,無異材,調授營州戶曹。調露中,單於突厥背叛,誘扇奚、契丹侵掠州縣,後奚、羯胡又與桑乾突厥同反。都督周道務遣休璟將兵擊破之於獨護山,斬獲甚眾,超拜豐州司馬。永淳中,突厥圍豐州,都督崔智辯戰歿。朝議欲罷豐州,徙百姓於靈、夏,休璟以為不可,上書曰:「豐州控河遏賊,實為襟帶,自秦、漢已來,列為郡縣,田疇良美,尤宜耕牧。隋季喪亂,不能堅守,乃遷徙百姓就寧、慶二州,致使戎羯交侵,乃以靈、夏為邊界。貞
 觀之末,始募人以實之,西北一隅,方得寧謐。今若廢棄,則河傍之地復為賊有,靈、夏等州人不安業,非國家之利也。」朝廷從其言,豐州復存。



 垂拱中,遷安西副都護。會吐蕃攻破焉耆,安息道大總管、文昌右相韋待價及副使閻溫古失利,休璟收其餘眾,以安西土。遷西州都督,上表請復取四鎮。則天遣王孝傑破吐蕃,拔四鎮,亦休璟之謀也。聖歷中,為司衛卿,兼涼州都督、右肅政御史大夫,持節隴右諸軍州大使。



 久視元年秋,吐蕃大將麴
 莽布支率騎數萬寇涼州,入自洪源谷,將圍昌松縣。休璟以數千人往擊之,臨陣登高,望見賊衣甲鮮盛,謂麾下曰:「自欽陵死,贊婆降,麴莽布支新知賊兵,欲曜威武,故其國中貴臣酋豪子弟皆從之。人馬雖精,不習軍事,吾為諸君取之。」乃被甲先登,與賊六戰六克,大破之,斬其副將二人,獲首二千五百級,築京觀而還。是後休璟入朝,吐蕃亦遣使來請和,因宴屢覘休璟。則天問其故,對曰:「往歲洪源戰時,此將軍雄猛無比,殺臣將士甚眾,
 故欲識之。」則天大加嘆異,擢拜右武威、右金吾二衛大將軍。



 休璟尤諳練邊事,自碣石西逾四鎮,綿亙萬里,山川要害,皆能記之。長安中,西突厥烏質勒與諸蕃不和,舉兵相持,安西道絕,表奏相繼。則天令休璟與宰相商度事勢,俄頃間草奏,便遣施行。後十餘日,安西諸州表請兵馬應接,程期一如休璟所畫。則天謂休璟曰:「恨用卿晚。」因遷夏官尚書、同鳳閣鸞臺三品。又謂魏元忠及楊再思、李嶠、姚元崇、李迥秀等曰:「休璟諳練邊事,卿等
 十不當一也。」



 尋轉太子右庶子,依舊知政事。以契丹入寇,復拜夏官尚書,兼檢校幽,營等州都督,兼安東都護。時中宗在春宮,將行,進啟於皇太子曰:「張易之兄弟幸蒙寵遇,數侍宴禁中,縱情失禮。非人臣之道,惟加防察。」中宗即位,召拜輔國大將軍、同中書門下三品,封酒泉郡公,顧謂曰:「卿曩日直言,朕今不忘。初欲召卿計事,但以遐遠,兼懷北狄之憂耳。」未幾,加特進,拜尚書右僕射。是歲秋,大水,休璟兩上表自咎,請免官甚切,辭多不載。
 中宗竟不允,手制答曰:「陰陽乖爽,事屬在予,待罪私門,難依來表。」尋遷中書令,充京師留守,俄加檢校吏部尚書。又以宮僚之舊,賜實封三百戶,累封宋國公。休璟在任,無所弘益。



 景龍二年,致仕於家,年力雖衰,進取彌銳。時尚宮賀婁氏頗關預國政,憑附者皆得寵榮,休璟乃為其子娶賀婁氏養女為妻,因以自達。由是起為太子少師、同中書門下三品,監修國史,仍封宋國公。休璟年逾八十,而不知止足,依托求進,為時所譏。景雲元年,又
 拜特進,充朔方道行軍大總管,以備突厥,停其舊封,別賜實封一百戶。二年,表請致仕。許之。祿及一品子課並令全給。休璟初得封時,以絹數千匹分散親族,又以家財數十萬大開塋域,備禮葬其五服之親,時人稱之。延和元年七月薨,年八十六,贈荊州大都督,謚曰忠。子先慎襲爵,官至陳州刺史。次子先擇,開元中為右金吾衛將軍。



 張仁願,華州下邽人也。本名仁亶,以音類睿宗諱改焉。
 少有文武材幹,累遷殿中侍御史。時有御史郭霸上表稱則天是彌勒佛身,鳳閣舍人張嘉福與洛州人王慶之等請立武承嗣為皇太子,皆請仁願連名署表,仁願正色拒之,甚為有識所重。尋而夏官尚書王孝傑為吐刺軍總管,統眾以御吐蕃,詔仁願往監之。仁願與孝傑不協,因人奏事,稱孝傑軍誣罔之狀。孝傑由是免為庶人,仁願遽遷侍御史。



 萬歲通天二年,監察御史孫承景監清邊軍,戰還,書戰圖以奏。每陣必畫承景躬當矢
 石、先鋒禦賊之狀,則天嘆曰:「御史乃能盡誠如此!」擢拜右肅政臺中丞,令仁願敘錄承景下立功人。仁願未發都,先問承景對陣勝負之狀。承景身實不行,問之皆不能對,又虛增功狀。仁願廷奏承景罔上之罪,於是左遷崇仁令,擢仁願為肅政臺中丞、檢校幽州都督。會突厥默啜入寇,攻陷趙、定,擁眾回至幽州,仁願勒兵出城邀擊之,流矢中手,賊亦引退。則天遣使勞問,賜以醫藥。累遷並州大都督府長史。



 神龍二年,中宗還京,以仁願為
 左屯衛大將軍,兼檢校洛州長史。時都城穀貴,盜竊甚眾,仁願一切皆捕獲杖殺之。積尸府門,遠近震慴,無敢犯者。初,高宗時賈敦頤為洛州刺史,亦有政績,與仁願皆為一時之最。故時人為之語曰:「洛州有前賈後張,可敵京兆三王。」其見稱如此。



 三年,突厥入寇。朔方軍總管沙吒忠義為賊所敗。詔仁願攝御史大夫,代忠義統眾。仁願至軍而賊眾已退,乃躡其後,夜掩大破之。先,朔方軍北與突厥以河為界,河北岸有拂雲神祠。突厥將入
 寇,必先詣祠祭酹求福,因牧馬料兵而後渡河。時突厥默啜盡眾西擊突騎施娑葛,仁願請乘虛奪取漠南之地,於河北築三受降城,首尾相應,以絕其南寇之路。太子少師唐休璟以為兩漢已來,皆北守黃河,今於寇境築城,恐勞人費功,終為賊虜所有,建議以為不便。仁願固請不已,中宗竟從之。仁願表留年滿鎮兵以助其功。時咸陽兵二百餘人逃歸,仁願盡擒之,一時斬於城下,軍中股心慄,役者盡力,六旬而三城俱就。以拂雲祠為中
 城,與東、西兩城相去各四百餘里,皆據津濟,遙相應接,北拓地三百餘里,於牛頭朝那山北置烽候一千八百所。自是突厥不得度山放牧,朔方無復寇掠,減鎮兵數萬人。



 仁願初建三城,不置壅門及卻敵、戰格之具。或問曰:「此邊城御賊之所,不為守備,何也?」仁願曰:「兵貴在攻取,不宜退守。寇若至此,即當並力出戰,回顧望城,猶須斬之,何用守備生其退恧之心也?」其後常元楷為朔方軍總管,始築壅門以備寇,議者以此重仁願而輕元楷
 焉。仁願在朔方,奏用監察御史張敬忠、何鸞、長安尉寇、泚、鄠縣尉王易從、始平主簿劉體微分判軍事,太子文學柳彥昭為管記,義烏尉晁良貞為隨機。敬忠等皆以文吏著稱,多至大官,時稱仁願有知人之鑒。



 景龍二年,拜左衛大將軍、同中書門下三品,累封韓國公。春還朝,秋復督軍備邊。中宗賦詩祖餞,賞賜不可勝紀。尋加鎮軍大將軍。睿宗即位,以老致仕,特全給祿俸,又拜兵部尚書,加光祿大夫,依舊致仕。開元二年卒,贈太子少傅,
 傅物二百段,命五品官一人為監護使。子之輔,開元初為趙州刺史。



 薛訥,絳州萬泉人也,左武衛大將軍仁貴子也。為藍田令,有富商倪氏於御史臺理其私債,中丞來俊臣受其貨財,斷出義倉米數千石以給之。訥曰:「義倉本備水旱,以為儲蓄,安敢絕眾人之命,以資一家之產?」竟報上不與。會俊臣得罪,其事乃不行。其後突厥入寇河北,則天以訥將門,使攝左武威衛將軍、安東道經略。臨行,於同
 明殿召見與語,訥因奏曰:「醜虜恁凌,以盧陵為辭。今雖有制升儲,外議猶恐未定。若此命不易,則狂賊自然款伏。」則天深然其言。尋拜幽州都督,兼安東都護。轉並州大都督府長史,兼檢校左衛大將軍。久當邊鎮之任,累有戰功。



 玄宗即位,於新豐講武,訥為左軍節度。時元帥與禮官得罪,諸部頗亦失序。唯訥及解琬之軍不動。玄宗令輕騎召訥等,至軍門,皆不得入。禮畢,上甚加慰勞。



 時契丹及奚與突厥連和,屢為邊患,訥建議請出師討
 之。開元二年夏,詔與左監門將軍杜賓客、定州刺吏崔宣道等率眾二萬,出檀州道以討契丹等。杜賓客以為時屬炎暑,將土負戈甲,齎資糧,深入寇境,恐難為制勝。中書令姚元崇亦以為然。訥獨曰:「夏月草茂,羔犢生息之際,不費糧儲,亦可漸進。一舉振國威靈,不可失也。」時議咸以為不便。玄宗方欲威服四夷,特令訥同紫微黃門三品,總兵擊奚、契丹,議者乃息。六月,師至灤河,遇賊,時既蒸暑,諸將失計會,盡為契丹等所覆。訥脫身走免,
 歸罪於崔宣道及蕃將李思敬等八人,詔盡令斬之,特免杜賓客之罪。下制曰:「並州大都督府長史兼檢校左衛大將軍、和戎大武等諸軍州節度大使、同紫微黃門三品薛訥,總戎御邊,建議為首。暗於料敵,輕於接戰,,張我王師,衄之虜境。觀其疇昔,頗常輸罄,每欲資忠報主,見義忘身。特緩嚴刑,俾期來效,宜赦其罪,所有官爵等並從除削。」



 其年八月,吐蕃大將坌達延、乞力徐等率眾十萬寇臨洮軍,又進寇蘭州及渭州之渭源縣,掠群牧
 而去。詔訥白衣攝左羽林將軍,為隴右防禦使,與大僕少卿王晙等率兵邀擊之。十月,訥領眾至渭源,遇賊戰於武階驛,與王晙掎角夾攻之,大破賊眾。追奔至洮水,又戰於長城堡,豐安軍使王海賓先鋒力戰死之。將士乘勢進擊,又敗之,殺獲萬人,擒其將六指鄉彌洪,盡收其所掠羊馬,並獲其器械,不可勝數。時有詔將以十二月親征吐蕃,及聞訥等克捷,玄宗大悅,乃停親征。追贈王海賓左金吾衛大將軍,賜物三百段、粟三百石,名其
 稚子為忠嗣,拜朝散大夫。命紫微舍人倪若水往,即便敘錄功狀,拜訥為左羽林軍大將軍,復封平陽郡公,仍拜子暢朝散大夫。俄又充涼州鎮軍大總管。尋以年老,特聽致仕。八年卒,年七十餘,贈太常卿,謚曰昭定。訥沉勇寡言,臨大敵而益壯。訥弟楚玉,開元中,為幽州大都督府長史,以不稱職見代而卒。



 王晙,滄州景城人,徙家於洛陽。祖有方,岷州刺史。晙弱冠明經擢第,歷遷殿中侍御史,加朝散大夫。時朔方軍
 元帥魏元忠討賊失利,歸罪於副將韓思忠,奏請誅之。晙以思忠既是偏裨,制不由已,又有勇智可惜,不可獨殺非辜,乃廷議爭之。思忠竟得釋,而晙亦由是出為渭南令。



 景龍未,累轉為桂州都督。桂州舊有屯兵,常運衡、永等州糧以饋之,晙始改築羅郭,奏罷屯兵及轉運。又堰江水,開屯田數千頃,百姓賴之。尋上疏請歸鄉拜墓,州人詣闕請留晙,乃下敕曰:「彼州往緣寇盜,戶口凋殘,委任失材,乃令至此。卿處事強濟,遠邇寧靜,築城務農,
 利益已廣,隱括綏緝,復業者多。宜須政成,安此黎庶,百姓又有表請,不須來也。」晙在州又一年,州人立碑以頌其政。再轉鴻臚大卿,充朔方軍副大總管,兼安西大都護,豐安、定遠、三城及側近軍並受晙節度。後轉太僕少卿、隴右群牧使。



 開元二年,吐蕃精甲十萬寇臨洮軍,晙率所部二千人卷甲倍程,與臨洮兩軍合勢以拒之。賊營於大來谷口,吐蕃將坌達延又率兵繼至。晙乃出奇兵七百人,衣之蕃服,夜襲之。相去五里,置鼓角,令前者
 遇寇大呼,後者擊鼓以應之。賊眾大懼,疑有伏兵,自相殺傷,死者萬計。俄而攝右羽林將軍薛訥率眾邀擊吐蕃,至武階谷,去大來穀二十里,為賊所隔。晙率兵迎訥之軍,賊置兵於兩軍之間,連亙數十里。晙夜出壯士銜枚擊之,賊又大潰。乃與訥合軍,掩其餘眾,追奔至洮水,殺獲不可勝數,盡收所掠牧馬而還。以功加銀青光祿大夫,封清源縣男,兼原州都督,仍拜其子班為朝散大夫。尋除並州大都督府長史。明年,突厥默啜為九姓所
 殺,其下酋長多款塞投降,置之河曲之內。俄而小殺繼立,降者漸叛。晙上疏曰:



 突厥時屬亂離,所以款塞降附。其與部落,非有仇嫌,情異北風,理固明矣,養成其釁,雖悔何追。今者,河曲之中,安置降虜,此輩生梗,實難處置。日月漸久,奸詐逾深,窺邊間隙,必為患難。今有降者部落,不受軍州進止,輒動兵馬,屢有傷殺。詢問勝州左側,被損五百餘人。私置烽鋪,潛為抗拒,公私行李,頗實危懼。北虜如或南牧,降戶必與連衡。臣問沒蕃歸人云,卻
 逃者甚眾,南北信使,委曲通傳,此輩降人,翻成細作。倘收合餘燼,來逼軍州,虜騎恁凌,胡兵應接,表裏有敵,進退無援。雖復韓、彭之勇,孫、吳之策,令其制勝,其可必乎!



 望至秋冬之際,令朔方軍盛陳兵馬,告其禍福,啗以繒帛之利,示以麋鹿之饒,說其魚米之鄉,陳其畜牧之地。並分配淮南、河南寬鄉安置,仍給程糧,送至配所。雖復一時勞弊,必得久長安穩。二十年外,漸染淳風,將以充兵,皆為勁卒。若以北狄降者不可南中安置,則高麗俘
 虜置之沙漠之曲,西域編氓散在青、徐之右,唯利是視,務安疆埸,何獨降胡,不可移徙。



 近者,在邊將士,爰及安蕃使人,多作諛辭,不為實對。或言北虜破滅,或言降戶安靜,志欲自言功效,非有以徇邦家。伏願察斯利口,行茲遠慮,邊荒清晏,黎元幸甚。



 臣料留住之議,謀者云遵故事,必言降戶之輩,舊置河曲之中,昔年既得康寧,今日還應穩便。但同時異事,先典攸傳。往者頡利破亡,邊境寧謐,降戶之輩,無復他心,所以多歷歲年,此類皆無
 動靜。今虜見未破滅,降戶私使往來,或畏北虜之威,或懷北虜之惠,又是北虜戚屬,夫豈不識親疏,將比昔年,安可同日!



 臣料其中頗有三策。若盛陳兵馬,散令分配,內獲精兵之實,外袪黠虜之謀,暫勞永安,此上策也。若多屯士卒,廣為備擬,亭障之地,蕃、漢相參,費甚人勞,此下策也。若置之朔塞,任之來往,通傳信息,結成禍胎,此無策也。伏願察斯三者,詳其善惡,利害之狀,長短可尋。縱因遷移,或致逃叛,但有移得之者,即是今日良圖,留
 待河冰,恐即有變。臣蒙天澤,叨居重鎮,逆耳利行,敢不盡言。



 疏奏未報,降虜果叛,敕晙帥並州兵西濟河以討之。晙乃間行倍道,以夜繼晝,卷甲舍幕而趨之。夜於山中忽遇風雪甚盛,晙恐失期,仰天誓曰:「晙若事君不忠,不討有罪,明靈所殛,固自當之,而士眾何辜,令其勞苦!若誠心忠烈,天監孔明,當止雪回風,以濟戎事。」言訖,風回而雪止。時叛者分為兩道,其在東者,晙追及之,殺一千五百餘人,生獲一千四百餘人,駝馬牛羊甚眾。晙以
 功遷左散騎常侍、持節朔方道行軍大總管,尋遷御史大夫。



 時突厥𧾷夾跌部落及僕固都督勺磨等散在受降城左右居止,且謀引突厥共為表裏,陷軍城而叛。晙因入奏,密請誅之。八年秋,晙誘𧾷夾跌等黨與八百餘人於中受降城誅之,由是乃授晙兵部尚書,復充朔方軍大總管。



 九年,蘭池州胡苦於賦役,誘降虜餘燼,攻夏州反叛,詔隴右節度使、羽林將軍郭知運與晙相知討之。晙奏:「朔方軍兵自有餘力,其郭知運請還本軍。」未報,而知運兵
 至,與晙頗不相協。晙所招撫降者,知運縱兵擊之,賊以為晙所賣,皆相率叛走。晙進封清源縣公,,仍兼御史大夫。俄而賊眾復相結聚,晙坐左遷梓州刺史。十年,拜太子詹事,累封中山郡公。屬車駕北巡,以晙為吏部尚書,兼太原尹。十一年夏,,代張說為兵部尚書、同中書門下三品,追錄破胡之功,加金紫光祿大夫,仍充朔方軍節度大使。其年冬,上親郊祀,追晙赴京,以會大禮。晙以時屬冰壯,恐虜騎乘隙入寇,表辭不赴,手敕慰勉,仍賜衣
 一副。會許州刺史王喬家奴告喬與晙潛謀構逆,敕侍中源乾曜、中書令張說鞫其狀。晙既無反狀,乃以違詔追不到,左遷蘄州刺史。十四年,累遷戶部尚書,復為朔方軍節度使。二十年卒,年七十餘,贈尚書左丞相,謚曰忠烈。



 往歲,魏元忠為張易之、昌宗所構,左授高要尉,晙密狀申明之。宋璟時為鳳閣舍人,謂晙曰:「魏公且全矣,子須威嚴而坐理,恐子之狼狽也。」晙曰:「魏公忠而獲罪,晙為義所激,顛沛無恨。」璟嘆曰:「璟不能申魏公之枉,深
 負朝廷矣。」晙氣貌雄壯,時人謂之有熊虎之狀。然慕義激勵,有古人之風,御下整肅,人吏畏而愛之。晙卒後,信安王禕於幽州討奚告捷,奏稱軍士咸見晙與蕃將高昭領兵馬先軍討賊,上聞而嗟異久之。戶部郎中楊伯城上疏,請晙等墳特乞增修封域,量加表異,降使饗祭,優其子孫。玄宗乃遣使就其家廟祭,仍如其子官秩。



 史臣曰:婁師德應召而慷愾,勇也;薦仁傑而入用,忠也;不使仁傑知之,公也;營田贍軍,智也;恭勤接下,和也;參
 知政事,功名有卒,是人之難也,又何愧於將相乎!王孝傑,唐休璟、張仁願、薛訥、王晙等,皆韜武干,亟立邊功。然孝傑失於再擒,休璟虧於餘行。先敗後勝,薛訥何慚;止雪回風,王晙難掩;仁願操履,中否相兼。



 贊曰:拯物之心,不形於色。將相之材,人何以測。臣有始終,功無爽忒。多忌梁公,自招慚德。唐、張、訥、晙,善陣能師。共服戎虜,不憂邊陲。



\end{pinyinscope}