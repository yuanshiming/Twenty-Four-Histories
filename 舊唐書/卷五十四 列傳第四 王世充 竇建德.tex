\article{卷五十四 列傳第四 王世充 竇建德}

\begin{pinyinscope}

 ○王世充竇建德



 王世充,字行滿,本姓支,西域胡人也。寓居新豐。祖支頹耨,早死。父收,隨母嫁霸城王氏,因冒姓焉,仕至汴州長史。世充頗涉經史,尤好兵法及龜策、推步之術。開皇
 中,以軍功拜儀同,累轉兵部員外郎。善敷奏,明習法律,然舞弄文法,高下其心。或有駁難之者,世充利口飾非,辭議鋒起,眾雖知其不可而莫能屈。



 大業中,累遷江都丞,兼領江都宮監。時煬帝數幸江都,世充善候人主顏色,阿諛順旨,每入言事,帝必稱善。乃雕飾池臺,陰奏遠方珍物,以媚於帝,由是益暱之。世充知隋政將亂,陰結豪俊,多收群心,有系獄抵罪,皆枉法出之,以樹私恩。及楊玄感作亂,吳人硃燮、晉陵人管崇起兵江南以應之,自
 稱將軍,擁眾十餘萬。隋遣將軍吐萬緒、魚俱羅等討之,不克。世充為其偏將,募江都萬餘人,頻擊破之。每有克捷,必歸功於下,所獲軍實,皆推與士卒,由此人爭為用,功最居多。



 十年,齊郡賊帥孟讓自長白山寇掠諸郡,至盱眙,有眾十餘萬。世充以兵拒之,保都梁山,為五柵,相持不戰,乃倡言兵走,羸師示弱。讓笑曰:「王世充文法小吏,安能領兵?吾令生縛取之,鼓行而入江都。」時百姓皆入壁,野無所掠,賊眾漸餒,又苦柵當其道,不得南侵,即
 分兵圍五柵。世充每日擊之,陽不利,走還入柵。如是數日,讓益輕之,乃稍分人於南方抄,留兵才足以圍柵。世充知其懈,乃於營中夷灶撤幕,投方陣,四面外向,毀柵而出,奮擊,大破之,讓以數十騎遁去,斬首萬餘級,俘虜十餘萬人。煬帝以世充有將帥才略,復遣領兵討諸小盜,所向盡平。



 十一年,突厥圍煬帝於雁門。世充盡發江都人將往赴難,在軍中蓬首垢面,悲泣無度,曉夜不解甲,藉草而臥。煬帝聞之,以為忠,益信任之。十二年,遷江
 都通守。時厭次人格謙為盜數年,兵十餘萬在豆子中,為太僕卿楊義臣所殺,世充帥師擊其餘眾,破之。又擊盧明月於南陽,虜獲數萬。後還江都,煬帝大悅,自執杯酒以賜之。及李密攻陷洛口倉,進逼東都,煬帝特詔世充大發兵,於洛口拒密,前後百餘戰,未有勝負。又遣就軍拜世充為將軍,趣令破賊。世充引軍渡洛水,與李密戰,世充軍敗績,溺死者萬餘人,乃率餘眾歸河陽。時天寒大雪,兵士在道凍死者又數萬人,比至河陽,才以
 千數。世充自系獄請罪,越王侗遣使赦之,徵還洛陽,置營於含嘉倉城,收合亡散,復得萬餘人。



 俄而宇文化及作難,太府卿元文都、武衛將軍皇甫無逸、右司郎中盧楚,奉越王侗嗣位於東都,拜世充為吏部尚書,封鄭國公。文都謂楚等曰:「今化及弒逆,仇恥未報,吾雖志在枕戈,而力所不及。為國計者,莫如以尊官寵李密,以庫物權啖之,使擊化及,令兩賊自斗,化及既破,而密之兵固亦疲矣。又其士卒得我之賞,居我之官,內外相親,易為
 反間,我師養力以乘其弊,則密亦可圖也。」楚等以為然。即日遣使拜密為太尉、尚書令,令討化及。密遂稱臣奉制,以兵拒化及於黎陽。每戰勝,則遣使告捷,眾皆悅。世充獨謂其麾下諸將曰:「文都之輩,刀筆吏耳,吾觀其勢,必為李密所擒。且吾軍人每與密戰,殺其父兄子弟,前後已多,一旦為之下,吾屬無類矣!」出言以激怒其眾。文都知而大懼,與楚等謀,因世充入內,伏甲而殺之,期有日矣。納言段達庸懦,恐事不果,遣其女婿張志以楚等
 謀告世充。其夜,勒兵圍宮城,將軍費曜、田闍等拒戰於東太陽門外,曜軍敗,世充遂攻門而入,無逸以單騎遁走,獲楚殺之。時宮門閉,世充遣人扣門言於侗曰:「元文都等欲執皇帝降於李密,段達知而告臣,臣非敢反,誅反者耳。」初,文都聞變,入奉侗於乾陽殿,陳兵衛之,令將帥乘城以拒難。段達矯侗命,執文都送於世充,至則亂擊而死。達又矯侗命,開門以納世充。世充悉遣人代宿衛者,然後入謁陳謝曰:「文都等無狀,謀相屠害,事急為
 此,不敢背國。」侗與之盟。其日,進拜尚書左僕射,總督內外諸軍事。世充去含嘉城,移居尚書省,專宰朝政。以其兄世惲為內史令,入居禁中,子弟咸擁兵馬,鎮諸城邑。



 未幾,李密破化及還,其勁兵良馬多戰死,士卒疲倦。世充欲乘其弊而擊之,恐人心不一,乃假托鬼神,言夢見周公。乃立祠於洛水,遣巫宣言周公欲令僕射急討李密,當有大功,不則兵皆疫死。世充兵多楚人,俗信妖言,眾皆請戰。世充簡練精勇,得二萬餘人,馬二千餘匹,軍
 於洛水南。密軍偃師北山上。時密新破化及,有輕世充之心,不設壁壘。世充夜遣三百餘騎潛入北山,伏溪谷中,令軍人秣馬蓐食,遲明而薄密。密出兵應之,陳未成列而兩軍合戰。其伏兵發,乘高而下,馳壓密營,又縱火焚其廬舍,密軍潰,降其將張童仁、陳智略,進下偃師,密走保洛口。初,世充兄世偉及子玄應隨化及至東郡,密得而囚之於城中,至是盡獲之。又執密長史邴元真妻子、司馬鄭虔象之母及諸將子弟,皆撫慰之,各令潛呼
 其父兄。世充進兵,次洛口,邴元真、鄭虔象等舉倉城以應之。密以數十騎走河陽,率餘眾入朝。世充盡收其眾,振旅而還。侗進拜世充太尉,以尚書省為其府,備置官屬。世充立三榜於府門之外:一求文才學識堪濟世務者,一求武藝絕人摧鋒陷陣者,一求能理冤枉擁抑不申者。於是上書陳事,日有數百,世充皆躬自省覽,殷勤慰勞。好行小惠,下至軍營騎士,皆飾辭以誘之。當時有識者見其心口相違,頗以懷貳。世充嘗於侗前賜食,還
 家大嘔吐,疑遇毒所致,自是不復朝請,與侗絕矣。遣云定興、段達入奏於侗,請加九錫之禮。二年三月,遂策授相國,總百揆,封鄭王,加九錫備物。有道士桓法嗣者,自言解圖讖,乃上《孔子閉房記》,畫作丈夫持一竿以驅羊。釋云:「隋,楊姓也。乾一者,王字也。王居羊後,明相國代隋為帝也。」又取《莊子人間世》、《德充符》二篇上之,法嗣釋曰:「上篇言『世』,下篇言『充』,此即相國名矣,明當德被人間,而應符命為天子也。」世充大悅曰:「此天命也。」再拜受之,即
 以法嗣為諫議大夫。世充又羅取雜鳥,書帛系其頸,自言符命而散放之。有彈射得鳥來而獻者,亦拜官爵。段達、雲定興等入見於侗曰:「天命不常,鄭王功德甚盛,願陛下揖讓告禪,遵唐、虞之跡。」侗怒曰:「天下者,高祖之天下,若隋德未衰,此言不可發,必天命有改,亦何論於禪讓?公等皆是先朝舊臣,忽有斯言,朕復當何所望!」段達等莫不流涕。世充又使人謂曰:「今海內未定,須得長君,待四方乂安,復子明闢。必若前盟,義不違負。」四月,假為
 侗詔策禪位,遣兄世惲廢侗於含涼殿,世充僭即皇帝位,建元曰開明,國號鄭。先封同姓王隆為淮陽王,整為東郡王,楷為馮翊王,素為樂安王。次封叔瓊為陳王,兄世衡為秦王,世偉為楚王,世惲為齊王。又封瓊子辯為杞王,衡子虔壽為蔡王,偉子弘烈為魏王,行本為荊王,琬為代王;惲子仁則為唐王,道誠為衛王,道詢為趙王,道夌為燕王;兄世師子太為宋王,君度為越王。立子玄應為皇太子,封子玄恕為漢王。世充每聽朝,必殷勤誨
 諭,言辭重復,千端萬緒,百司奉事,疲於聽受。或輕騎游歷街衢,亦不清道,百姓但避路而已,按轡徐行,謂百姓曰:「昔時天子深坐九重,在下事情,無由聞徹。世充非貪寶位,本欲救時,今當如一州刺史,每事親覽,當與士庶共評朝政。恐門禁有限,慮致壅塞,今止順天門外置座聽朝。」又令西朝堂受抑屈,東朝堂受直諫。於是獻書上事,日有數百,條疏既煩,省覽難遍,數日後不復更出。



 五月,世充禮部尚書裴仁基及其子左輔大將軍行儼、尚
 書左丞宇文儒童等數十人謀誅世充,復尊立侗。事洩,皆見害,夷其三族。六月,世惲因勸世充害侗,以絕眾望。世充遣其侄行本鴆殺侗,謚曰恭皇帝。其將軍羅士信率其眾千餘人來降。十月,世充率眾東徇地,至於滑州,仍以兵臨黎陽。十一月,竇建德入世充之殷州,殺掠居人,焚燒積聚,以報黎陽之役。



 三年二月,世充殿中監豆盧達來降。世充見眾心日離,乃嚴刑峻制,家一人逃者,無少長皆坐為戮,父子、兄弟、夫妻許其相告而免之。又
 令五家相保,有全家叛去而鄰人不覺者,誅及四鄰。殺人相繼,其逃亡益甚。至於樵採之人,出入皆有限數,公私窘急,皆不聊生。又以宮城為大獄,意有所忌,即收系其人及家屬於宮中。又每使諸將出外,亦收其親屬質於宮內。囚者相次,不減萬口,既艱食,餒死者日數十人。世充屯兵不散,倉粟日盡,城中人相食。或握土置甕中,用水淘汰,沙石沉下,取其上浮泥,投以米屑,作餅餌而食之,人皆體腫而腳弱,枕倚於道路。其尚書郎盧君業、
 郭子高等皆死於溝壑。七月,秦王率兵攻之,師至新安,世充鎮堡相次來降。八月,秦王陳兵於青城宮,世充悉兵來拒,隔澗而言曰:「隋末喪亂,天下分崩,長安、洛陽,各有分地,世充唯願自守,不敢西侵。計熊、穀二州,相去非遠,若欲取之,豈非度內?既敦鄰好,所以不然。王乃盛相侵軼,遠入吾地,三崤之道,千里饋糧,以此出師,未見其可。」太宗謂曰:「四海之內,皆承正朔,唯公執迷,獨阻聲教。東都士庶,亟請王師,關中義勇,感恩致力。至尊重違眾
 願,有斯吊伐。若轉禍來降,則富貴可保;如欲相抗,無假多言。」世充無以報。太宗分遣諸將攻其城鎮,所至輒下。九月,王君廓攻拔世充之轘轅縣,東徇地至管城而還,於是河南州縣相次降附。竇建德自侵殷州之後,與世充遂結深隙,信使斷絕。十一月,竇建德又遣人結好,並陳救援之意。世充乃遣其兄子琬及內史令長孫安世報聘,且乞師。



 四年二月,世充率兵出方諸門,與王師相抗,世充軍敗。因乘勝追之,屯其城門。世充步卒不得入,
 驚散南走,追斬數千級,虜五千餘人。世充從此不復敢出,但嬰城自守,以待建德之援。三月,秦王擒建德並王琬、長孫安世等於武牢,回至東都城下以示之,且遣安世入城,使言敗狀。世充惶惑,不知所為,將潰圍而出,南走襄陽,謀於諸將,皆不答,乃率其將吏詣軍門請降。於是收其府庫,頒賜將士。世充黃門侍郎薛德音以文檄不遜,先誅之,次收世充黨與段達、楊注、單雄信、楊公卿、郭士衡、郭什柱、董浚、張童仁、硃粲等十餘人,皆戮於洛
 渚之上。



 秦王以世充至長安,高祖數其罪,世充對曰:「計臣之罪,誠不容誅,但陛下愛子秦王許臣不死。」高祖乃釋之。與兄苪、妻、子同徙於蜀,將行,為仇人定州刺史獨孤修所殺。子玄應及兄世偉等在路謀叛,伏誅。世充自篡位,凡三年而滅。



 竇建德,貝州漳南人也。少時,頗以然諾為事。嘗有鄉人喪親,家貧無以葬,時建德耕於田中,聞而嘆息,遽輟耕牛,往給喪事,由是大為鄉黨所稱。初,為里長,犯法亡去,
 會赦得歸。父卒,送葬者千餘人,凡有所贈,皆讓而不受。



 大業七年,募人討高麗,本郡選勇敢尤異者以充小帥,遂補建德為二百人長。時山東大水,人多流散,同縣有孫安祖,家為水所漂,妻子餒死。縣以安祖驍勇,亦選在行中。安祖辭貧,白言漳南令,令怒笞之。安祖刺殺令,亡投建德,建德舍之。是歲,山東大饑,建德謂安祖曰:「文皇帝時,天下殷盛,發百萬之眾以伐遼東,尚為高麗所敗。今水潦為災,黎庶窮困,而主上不恤,親駕臨遼,加以往
 歲西征,瘡痍未復,百姓疲弊,累年之役,行者不歸,今重發兵,易可搖動。丈夫不死,當立大功,豈可為逃亡之虜也?我知高雞泊中廣大數百里,莞蒲阻深,可以逃難,承間而出,虜掠足以自資。既得聚人,且觀時變,必有大功於天下矣。」安祖然其計。建德招誘逃兵及無產業者,得數百人,令安祖率之,入泊中為群盜,安祖自稱將軍。鄃人張金稱亦結聚得百人,在河阻中。蓚人高士達又起兵得千餘人,在清河界中。時諸盜往來漳南者,所過皆
 殺掠居人,焚燒舍宅,獨不入建德之閭。由是郡縣意建德與賊徒交結,收系家屬,無少長皆殺之。建德聞其家被屠滅,率麾下二百人亡歸。士達自稱東海公,以建德為司兵。後安祖為張金稱所殺,其兵數千人又盡歸於建德。自此漸盛,兵至萬餘人,猶往來高雞泊中。每傾身接物,與士卒均執勤苦,由是能致人之死力。



 十二年,涿郡通守郭絢率兵萬餘人來討士達。士達自以智略不及建德,乃進為軍司馬,咸以兵授焉。建德既初董眾,欲
 立奇功以威群賊,請士達守輜重,自簡精兵七千人以拒絢,詐為與士達有隙而叛之。士達又宣言建德背亡,而取虜獲婦人紿為建德妻子,於軍中殺之。建德偽遣人遺絢書請降,願為前驅,破士達以自效。約信之,即引兵從建德至長河界,期與為盟,共圖士達。絢兵益懈而不備,建德襲之,大破絢軍,殺略數千人,獲馬千餘匹,絢以數十騎遁走,遣將追及於平原,斬其首以獻士達。由是建德之勢益振。



 隋遣太僕卿楊義臣率兵萬餘人討
 張金稱,破之於清河,所獲賊眾皆屠滅,餘散在草澤間者復相聚而投建德。義臣乘勝至平原,欲入高雞泊中,建德謂士達曰:「歷觀隋將,善用兵者,唯義臣耳。新破金稱,遠來襲我,其鋒不可當。請引兵避之,令其欲戰不得,空延歲月,將士疲倦,乘便襲擊,可有大功。今與爭鋒,恐公不能敵也。」士達不從其言,因留建德守壁,自率精兵逆擊義臣。戰小勝,而縱酒高宴,有輕義臣之心。建德聞之曰:「東海公未能破賊而自矜大,此禍至不久矣。隋兵
 乘勝,必長驅至此,人心驚駭,吾恐不全。」遂留人守壁,自率精銳百餘據險,以防士達之敗。後五日,義臣果大破士達,於陣斬之,乘勢追奔,將圍建德。守兵既少,聞士達敗,眾皆潰散。建德率百餘騎亡去,行至饒陽,觀其無守備,攻陷之,撫循士眾,人多願從,又得三千餘兵。初,義臣既殺士達,以為建德不足憂。建德復還平原,收士達敗兵之死者,悉收葬焉。為士達發喪,三軍皆縞素。招集亡卒,得數千人,軍復大振,始自稱將軍。初,群盜得隋官及
 山東士子皆殺之,唯建德每獲士人,必加恩遇。初得饒陽縣長宋正本,引為上客,與參謀議。此後隋郡長吏稍以城降之,軍容益盛,勝兵十餘萬人。



 十三年正月,築壇場於河間樂壽界中,自稱長樂王,年號丁丑,署置官屬。七月,隋遣右翊衛將軍薛世雄率兵三萬來討之,至河間城南,營於七里井。建德聞世雄至,選精兵數千人伏河間南界澤中,悉拔諸城偽遁,云亡入豆子中。世雄以為建德畏己,乃不設備。建德覘知之,自率敢死士一
 千人襲擊世雄。會雲霧晝晦,兩軍不辨,隋軍大潰,自相踏藉,死者萬餘,世雄以數百騎而遁,餘軍悉陷。於是建德進攻河間,頻戰不下。其後城中食盡,又聞煬帝被弒,郡丞王琮率士吏發喪,建德遣使吊之,琮因使者請降,建德退舍具饌以待焉。琮率官屬素服面縛詣軍門,建德親解其縛,與言隋亡之事,琮俯伏裴哀,建德亦為之泣。諸賊帥或進言曰;「琮拒我久,殺傷甚眾,計窮方出,今請烹之。」建德曰:「此義士也。方加擢用,以勵事君者,安可
 殺之!往在泊中共為小盜,容可恣意殺人,今欲安百姓以定天下,何得害忠良乎?」因令軍中曰:「先與王琮有隙者,今敢動搖,罪三族。」即日授琮瀛州刺史。始都樂壽,號曰金城宮,自是郡縣多下之。



 武德元年冬至日,於金城宮設會,有五大鳥降於樂壽,群鳥數萬從之,經日而去,因改年為五鳳。有宗城人獻玄珪一枚,景城丞孔德紹曰:「昔夏禹膺籙,天錫玄珪。今瑞與禹同,宜稱夏國。」建德從之。先是,有上谷賊帥王須拔自號漫天,擁眾數萬,入
 掠幽州,中流矢而死。其亞將魏刀兒代領其眾,自號歷山飛,入據深澤,有徒十萬。建德與之和,刀兒因弛守備,建德襲破之,又盡並其地。



 二年,宇文化及僭號於魏縣,建德謂其納言宋正本、內史侍郎孔德紹曰:「吾為隋之百姓數十年矣,隋為吾君二代矣。今化及殺之,大逆無道,此吾仇矣,請與諸公討之,何如?」德紹曰:「今海內無主,英雄競逐,大王以布衣而起漳浦,隋郡縣官人莫不爭歸附者,以大王仗順而動,義安天下也。宇文化及與國
 連姻,父子兄弟受恩隋代,身居不疑之地,而行弒逆之禍,篡隋自代,乃天下之賊也。此而不誅,安用盟主!」建德稱善。即日引兵討化及,連戰,大破之。化及保聊城,建德縱撞車拋石,機巧絕妙,四面攻城,陷之。建德入城,先謁隋蕭皇后,與語稱臣。悉收弒煬帝元謀者宇文智及、楊士覽、元武達、許弘仁、孟景,集隋文武官,對而斬之,梟首轅門之外。化及並其二子同載以檻車,至大陸縣斬之。



 建德每平城破陣,所得資財,並散賞諸將,一無所取。又
 不啖肉,常食唯有菜蔬、脫粟之飯。其妻曹氏不衣紈綺,所使婢妾才十數人。至此,得宮人以千數,並有容色,應時放散。得隋文武官及驍果尚且一萬,亦放散,聽其所去。又以隋黃門侍郎裴矩為尚書左僕射,兵部侍郎崔君肅為侍中,少府令何稠為工部尚書,自餘隨才拜授,委以政事,其有欲往關中及東都者亦恣聽之,仍給其衣糧,以兵援之,送出其境。攻陷洺州,虜刺史袁子幹。遷都於洺州,號萬春宮。遣使往灌津,祠竇青之墓,置守塚
 二十家。又與王世充結好,遣使朝隋越王侗於洛陽。後世充廢侗自立,乃絕之,始自尊大,建天子旌旗,出警入蹕,下書言詔。追謚隋煬帝為閔帝,封齊王暕子政道為鄖公。然猶依倚突厥。隋義城公主先嫁突厥,及是遣使迎蕭皇后,建德勒兵千餘騎送之入蕃,又傳化及首以獻公主。既與突厥相連,兵鋒益盛。



 九月,南侵相州,河北大使淮安王神通不能拒,退奔黎陽。相州陷,殺刺史呂氏。又進攻衛州,陷黎陽,左武衛大將軍李世勣、皇妹同
 安長公主及神通並為所虜。滑州刺史王軌為奴所殺,攜其首以奔建德,曰:「奴殺主為大逆,我何可納之!」命立斬奴,而返軌首於滑州。吏人感之,即日而降。齊、濟二州及兗州賊帥徐圓朗皆聞風而下。建德釋李世勣,使其領兵以鎮黎州。



 三年正月,世勣舍其父而逃歸,執法者請誅之,建德曰:「勣本唐臣,為我所虜,不忘其主,逃還本朝,此忠臣也,其父何罪!」竟不誅。舍同安長公主及神通於別館,待以客禮。高祖遣使與之連和,建德即遣公主
 與使俱歸。嘗破趙州,執刺史張昂、邢州刺史陳君賓、大使張道源等,以侵軼其境,建德將戮之。其國子祭酒凌敬進曰:「夫犬各吠非其主,今鄰人堅守,力屈就擒,此乃忠確士也。若加酷害,何以勸大王之臣乎?」建德盛怒曰:「我至城下,猶迷不降,勞我師旅,罪何可赦?」敬又曰:「今大王使大將軍高士興於易水抗禦羅藝,兵才至,士興即降,大王之意復為可不?」建德乃悟,即命釋之。其寬厚從諫,多此類也。又遣士興進圍幽州,攻之不克,退軍旅籠
 火城,為藝所襲,士興大潰。先是,其大將王伏寶多勇略,功冠等倫,群帥嫉之。或言其反,建德將殺之,伏寶曰:「我無罪也,大王何聽讒言,自斬左右手乎?」既殺之,後用兵多不利。



 九月,建德自帥師圍幽州,藝出兵與戰,大破之,斬首千二百級。藝兵頻勝而驕,進襲其營,建德列陣於營中,填塹而出,擊藝敗之。建德薄其城,不克,遂歸洺州。其納言宋正本好直諫,建德又聽讒言殺之。是後人以為誡,無復進言者,由此政教益衰。



 先,曹州濟陰人孟海
 公擁精兵三萬,據周橋城以掠河南之地。其年十一月,建德自率兵渡河以擊之。時秦王攻王世充於洛陽,建德中書舍人劉斌說建德曰:「今唐有關內,鄭有河南,夏居河北,此鼎足相持之勢也。聞唐兵悉眾攻鄭,首尾二年,鄭勢日蹙而唐兵不解。唐強鄭弱,其勢必破鄭,鄭破則夏有齒寒之憂。為大王計者,莫若救鄭,鄭拒其內,夏攻其外,破之必矣。若卻唐全鄭,此常保三分之勢也。若唐軍破後而鄭可圖,則因而滅之,總二國之眾,乘唐軍
 之敗,長驅西入,京師可得而有,此太平之基也。」建德大悅曰:「此良策矣。」適會世充遣使乞師於建德,即遣其職方侍郎魏處繪入朝,請解世充之圍。



 四年二月,建德克周橋,虜海公,留其將範願守曹州,悉發海公及徐圓朗之眾來救世充。軍至滑州,世充行臺僕射韓洪開城納之,遂進逼元州、梁州、管州,皆陷之,屯於滎陽。三月,秦王入武牢,進薄其營,多所傷殺,並擒其將殷秋、石瓚。時世充弟世辨為徐州行臺,遣其將郭士衡領兵數千人從
 之,合眾十餘萬,號為三十萬,軍次成皋,築宮於板渚,以示必戰。又遣間使約世充共為表裏。經二月,迫於武牢,不得進。秦王遣將軍王君廓領輕騎千餘抄其糧運,獲其大將張青特,虜獲甚眾。建德數不利,人情危駭,將帥已下破孟海公,皆有所獲,思歸洺州。凌敬進說曰:「宜悉兵濟河,攻取懷州河陽,使重將居守。更率眾鳴鼓建旗,逾太行,入上黨,先聲後實,傳檄而定。漸趨壺口,稍駭蒲津,收河東之地,此策之上也。行此必有三利:一則入無
 人之境,師有萬全;二則拓土得兵;三則鄭圍自解。」建德將從之,而世充之使長孫安世陰齎金玉,啖其諸將,以亂其謀。眾咸進諫曰:「凌敬,書生耳,豈可與言戰乎?」建德從之,退而謝敬曰:「今眾心甚銳,此天贊我矣。因此決戰,必將大捷。已依眾議,不得從公言也。」敬固爭,建德怒,扶出焉。其妻曹氏又言於建德曰:「祭酒之言可從,大王何不納也?請自滏口之道,乘唐國之虛,連營漸進,以取山北,又因突厥西抄關中,唐必還師以自救,此則鄭圍解
 矣。今頓兵武牢之下,日月淹久,徒為自苦,事恐無功。」建德曰:「此非女子所知也。且鄭國懸命朝暮,以待吾來,既許救之,豈可見難而退,示天下以不信也?」於是悉眾進逼武牢,官軍按甲挫其銳。及建德結陣於汜水,秦王遣騎挑之,建德進軍而戰,竇抗當之。建德少卻,秦王馳騎深入,反覆四五合,然後大破之。建德中槍,竄於牛口渚,車騎將軍白士讓、楊武威生獲之。先是,軍中有童謠曰:「豆入牛口,勢不得久。」建德行至牛口渚,甚惡之,果敗於
 此地。建德所領兵眾,一時奔潰,妻曹氏及其左僕射齊善行將數百騎遁於洺州。餘黨欲立建德養子為主,善行曰:「夏王平定河朔,士馬精強,一朝被擒如此,豈非天命有所歸也?不如委心請命,無為塗炭生人。」遂以府庫財物悉分士卒,各令散去。善行乃與建德右僕射裴矩、行臺曹旦及建德妻率偽官屬,舉山東之地,奉傳國等八璽來降。七月,秦王俘建德至京師,斬於長安市,年四十九。自起軍至滅,凡六歲,河北悉平。其年,劉黑闥復盜
 據山東。



 史臣曰:世充奸人,遭逢昏主,上則諛佞詭俗以取榮名,下則強辯飾非以制群論。終行篡逆,自恣陸梁,安忍殺人,矯情馭眾,凡所委任,多是叛亡,出降秦王,不致顯戮,其為幸也多矣。建德義伏鄉閭,盜據河朔,撫馭士卒,招集賢良。中絕世充,終斬化及,不殺徐蓋,生還神通,沉機英斷,靡不有初。及宋正本、王伏寶被讒見害,凌敬、曹氏陳謀不行,遂至亡滅,鮮克有終矣。然天命有歸,人謀不
 及。



 贊曰:世充篡逆,建德愎諫,二兇即誅,中原弭亂。



\end{pinyinscope}