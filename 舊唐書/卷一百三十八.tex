\article{卷一百三十八}

\begin{pinyinscope}

 ○馬燧子暢燧兄炫渾瑊子鎬金歲



 馬燧,字洵美,汝州郟城人,其先自右扶風徙焉。祖氏,官至左玉鈐衛倉曹。父季龍,嘗舉明《孫》《吳》,俶儻善兵法,官至嵐州刺史、幽州經略軍使。燧少時,嘗與諸兄讀書,乃
 輟卷嘆曰:「天下將有事矣,丈夫當建功於代,以濟四海,安能矻矻為一儒哉!」燧姿度魁異,長六尺二寸,沉勇多智略,該涉群書,尤善兵法。



 安祿山反,俾光祿卿賈循守範陽。燧說循曰:「祿山負恩首亂,雖陷洛城,必當夷滅。公盍建不代之功,誅其逆將向潤客、牛廷玠,拔其根柢,祿山西不能入關,則坐而受擒,天下可定也。」循雖善之,計不時決,事洩,祿山果遣韓朝陽來召循。朝陽至範陽,與循語,陰伏壯士以弓弦縊殺之。燧脫身走西山,隱者徐
 遇匿之。逾月,間行歸平原。平原不守,復走魏郡。



 寶應中,澤潞節度使李抱玉署奏趙城尉。是時回紇大軍還國,恃復東都之功,倔強恣睢,所過或虜掠廩粟,供餼小不如意,恣行殺害。抱玉具供辦,賓介皆憚不敢行,燧自贊請主郵驛。比回紇至,則先賂其渠帥,與明要約,回紇乃授燧旗幟為識,犯令者命燧戮之。取死囚給左右廝役,小違令,輒殺之。回紇相顧失色,虜涉其境,無敢暴掠。抱玉益奇之。燧因說抱玉曰:「屬者與回紇言,燧得其情。
 今僕固懷恩恃功樹黨,李懷仙、張忠志、薛嵩、田承嗣分授疆土,皆出於懷恩,其子瑒佻勇不義。以燧度之,將必窺太原西山以為亂,公宜深備之。」無何,懷恩果與太原都將李竭誠通謀,將取太原,其帥辛云京覺之,斬竭誠,固城自守,懷恩遣其子瑒率兵圍之。初,回紇北歸,遣其將安恪、石常庭將兵數百及誘募附麗者復數千人以守河陽,東都所虜掠重貨,悉積河陽。是時,懷恩遺薛嵩自相、衛饋糧以絕河津。抱玉令燧詣薛嵩說之,嵩乃絕
 懷恩從順。署奏左武衛兵曹。歷太子通事舍人,遷著作郎、營田判官。無幾,遷秘書少監、兼殿中侍御史,為節度判官、承務郎,遷鄭州刺史。燧乃勸課農畝,總其戶籍,歲一稅之,州人以為便。大歷四年,改懷州刺史。乘亂兵之後,其夏大旱,人失耕稼;燧乃務修教化,將吏有父母者,燧輒造之施敬,收葬暴骨,去其煩苛。至秋,界中生魯谷,人頗賴之。



 抱玉移鎮鳳翔,以汧陽被邊,署奏隴州刺史、兼御史中丞。州西有通道,廣二百餘步,上連峻山,山與
 吐蕃相直,虜每入寇,皆出於此。燧乃按行險易,立石種樹以塞之,下置二門,設籬櫓,八日而功畢。會抱玉入覲,與燧俱行。久之,代宗知其能,召見,拜商州刺史、兼御史中丞、防禦水陸運使。



 大歷十年,河陽三城兵亂,逐鎮將常休明,以燧檢校左散騎常侍、御史大夫、河陽三城使。十一年五月,汴州大將李靈耀反,因據州城,絕運路,以邀節制。代宗務姑息人,因授靈耀汴、宋等八州節度留後。靈耀不受命。乃潛結魏博;田承嗣乃遣兄子悅將兵
 援靈耀,破永平軍將劉洽。詔燧與淮西節度使李忠臣合軍討靈耀。忠臣懼賊,焚廬舍西走。燧勸其還兵,請為前鋒,擊破田悅,進逼汴州。忠臣行汴南,燧引軍行汴北,又敗靈耀將張清於西梁固。靈耀選銳兵八千,號為「餓狼軍」;燧獨引軍擊破之,進至浚儀。是時,河陽兵冠諸軍。承嗣又遣悅將兵二萬救靈耀,破永平軍將杜如江,略曹州,又敗李正己游軍,擊走劉洽、長孫全緒等軍,乘勝去汴州一舍,方陣而進。忠臣會宋州、淮南、浙西兵,與戰
 不利,請救於燧,燧引四千人為奇兵擊破之,田悅匹馬遁去。靈耀知悅敗,明日以百騎夜走,汴州悉降,燧讓功於忠臣。忠臣素暴戾,燧不欲入汴城,乃引軍退舍於板橋。忠臣入城,果專其功,因會擊殺宋州刺史李僧惠。燧還河陽。



 大歷十四年六月,檢校工部尚書、太原尹、北都留守、河東節度留後,尋為節度使。太原承前政鮑防百井敗軍之後,兵甲寡弱,燧乃悉召將吏牧馬廝役,得數千人,悉補騎卒,教之數月,為精騎。造甲者必令長短三
 等,稱其所衣,以便進趨。又造戰車,蒙以狻猊象,列戟於後,行則載兵甲,止則為營陣,或塞險以遏奔沖,器械無不犀利。居一年,陳兵三萬,開廣場以習戰陣,教其進退坐作之勢。



 建中二年六月,朝於京師,加檢校兵部尚書,令還太原。初,田悅新代承嗣統兵,恐人不附己,詐效誠款,燧上疏明其必反,宜先備之。其年,悅果與淄青、恆冀通謀,自將兵三萬圍邢州,次臨洺,築重城,絕其內外,以拒救兵。邢州將李洪、臨洺將張伾,皆堅守不拔。昭義軍
 告急,乃詔燧將步騎二萬與昭義節度使李抱真、神策行營兵馬使李晟合軍救臨洺。燧軍出郭口,兵未過險,乃遣持書喻悅,且示之好,悅謂燧畏之。十一月,師次邯鄲,悅遣使至,燧皆斬之以徇;遣兵擊破其支軍,射殺其將成炫之。悅自攻臨洺,遣大將楊朝光將兵萬人,於臨洺南雙岡東西列二柵以御燧。燧乃率李抱真、李晟進軍,營於二柵之中。其夜,東柵走歸悅。明日,燧進軍營明山,取其棄柵以置輜重。悅謂將吏曰:「朝光堅柵不下萬
 人,假令燧等盡銳攻之,比數日,計不能下,殺傷必甚。吾此必拔臨洺,賞勞軍士而與之戰,必勝之術也。」悅乃分恆州李惟岳救兵五千以助朝光,燧率軍攻朝光,田悅將萬餘人救之。燧乃令大將李自良、李奉國將騎兵合神策軍於雙岡御之,令曰:「令悅得過,當斬爾!」自良等擊卻悅。燧乃令推火車以焚其柵,斬朝光及大將盧子昌,斬首五千餘級,生虜八百餘人。居五日,進軍至臨洺,田悅悉軍戰。燧自將銳兵扼其沖口,凡百餘合,士皆決
 死,悅兵大敗,斬首萬餘級,生虜九百人,得穀三十萬斛,器甲稱是。悅收敗兵夜遁,邢州圍亦解。以功加右僕射。先戰,燧誓軍中,戰勝請以家財行賞,既勝,盡出其私財以頒將士。德宗嘉之,詔度支出錢五萬貫行賞,還燧家財。尋加魏博招討使。



 三年正月,田悅求救於淄青、恆冀,李納遣大將衛俊將兵萬人救悅,李惟嶽亦遣兵三千赴援。悅收合散卒二萬餘人,壁於洹水,淄青軍其東,恆冀軍其西,首尾相應。燧率諸軍進屯於鄴,奏請益河陽
 兵,詔河陽節度使李芃將兵會之。軍次於漳,悅遣將王光進以兵守長橋,築月城以為固,軍不得渡。燧乃於下流以車數百乘,維以鐵鎖,鎖絕中流,實以土囊以遏水,水稍淺,諸軍畢渡。是時軍糧少,悅深壁不戰,欲老燧軍。燧令諸軍持十日糧,進次倉口,與悅夾洹水而軍。抱真與李芃問:「糧少而深入,何也?」燧曰:「糧少利速戰,兵法善於致人,不致於人。今田悅與淄青、恆三軍為首尾,計欲不戰,以老我師;若分軍擊其左右,兵少未可必破,悅
 且來救,是前後受敵也。兵法所謂攻其必救,彼固當戰也,燧為諸軍合而破之。」燧乃造三橋道逾洹水,日挑戰,悅不敢出。恆州兵以軍少,懼為燧所並,引軍合於悅。悅謂燧明日復挑戰,乃伏兵萬人,欲邀燧。燧乃令諸軍半夜皆食,先雞鳴時擊鼓吹角,潛師傍洹水徑趨魏州,令曰:「聞賊至,則止為陣。」又令百騎吹鼓角,皆留於後,仍抱薪持火,待軍畢發,止鼓角匿其旁,伺悅軍畢渡,焚其橋。軍行十數里,悅乃率淄青、恆州步騎四萬餘人逾橋掩
 其後,乘風縱火,鼓噪而進。燧乃坐,申令無動,命前除草斬榛棘廣百步以為陣;燧出陣,募勇力得五千餘人,分為前列,以俟賊至。比悅軍至,則火止氣乏,力少衰,乃縱兵擊之,悅軍大敗。時神策、昭義、河陽軍小卻,河東軍既勝,諸軍還鬥,合擊又大破之。迫洹水,悅軍走橋,橋已焚矣。悅軍亂,赴水,斬首二萬餘級,殺大將孫晉卿、安墨啜,生獲三千餘人,溺死者不可勝紀。淄青軍殆盡,死者相枕藉三十餘里。悅收敗卒千餘人走魏州,至門,州將李
 長春閉門不納。久之,追兵不至,比明,乃納悅。悅既入,殺長春,嬰城自守。數日,李再春以博州降,悅兄昂以洺州降,王光進以長橋降。悅遣符璘、李瑤將五百騎送淄青兵還鎮,璘、瑤因來降燧。魏州先引御河入城南流,燧令塞其領口,河流絕,城中益恐。悅乃遣許士則、侯臧徒步間行說硃滔、王武俊,借兵求救。時王武俊已殺李惟岳,傳首京師,授武俊恆冀觀察都防禦使;時武俊同列張孝忠已為易定節度使,武俊獨為防禦使,又割趙、深二州
 為一鎮,以康日知為觀察使,甚為怨望,且素輕孝忠,恥名在下。時硃滔討李惟岳,拔深州,求隸幽州不得,亦怨望。由是滔、武俊同謀救悅。悅恃燕、趙之援,又出兵二萬背城而陣,燧復與諸軍擊破之。五月,加燧同中書門下平章事。六月,硃滔、王武俊聯兵五萬來救悅,至於城下。諸帥議退兵,燧固不可,德宗遣朔方節度使李懷光將朔方軍步騎萬五千人赴燧。是月晦,懷光亦至。懷光勇而無謀,軍至之日,未休息,堅請與滔等戰,王師不利。悅
 等決水灌燧等軍,燧兵屈糧少,七月,燧與諸軍退次魏縣。是月,詔加燧魏州大都督府長史,兼魏、博、貝四州節度、觀察、招討等使。田悅、硃滔、王武俊軍亦至魏縣,與官軍隔河對壘。十一月,三盜於魏縣軍中遞相推獎王號:硃滔稱冀王,田悅稱魏王,王武俊稱趙王;又遣使於李納,納稱齊王。四道共推淮西李希烈為天下兵馬元帥、太尉、建興王,皆偽署官號,如國初行臺之制,而名目頗有妖僻者,然未敢偽稱年號。而五盜合從圖傾社稷,兩
 河鼎沸,寇盜橫行;燧等雖志在勤王,竟莫能驅攘患難。



 四年十月,涇師犯關,帝幸奉天,燧引軍還太原。議者云:「燧若乘田悅洹水之敗,並力攻之,時城中敗卒無三二千人,皆夷傷未起,日夕俟降;燧與抱真不和,遷延於擊賊,乃致三盜連結,至今為梗,職燧之由。」燧至太原,遣行軍司馬王權將兵五千赴奉天,又遣男匯及大將之子與俱來,壁於中渭橋。及帝幸梁州,權、匯領兵還鎮。燧以晉陽王業所起,度都城東面平易受敵,時天下騷動,北
 邊數有警急,乃引晉水架汾而注城之東,瀦以為池,寇至計省守陴者萬人;又決汾水環城,多為池沼,樹柳以固堤。尋兼保寧軍節度使。



 興元元年正月,加檢校司徒,封北平郡王。七月,德宗還京,加燧奉誠軍及晉、絳、慈、隰節度並管內諸軍行營副元帥,令與侍中渾瑊、鎮國軍節度使駱元光同討河中。初,李懷光據河中,燧遣使招諭之,懷光妹婿要廷珍守晉州,衙將毛朝易又守隰州,鄭抗守慈州,皆相次降燧。初,王武俊自魏縣還鎮,雖去偽
 號,而攻圍趙州不解,康日知窘蹙,欲棄趙州,燧奏曰:「可詔武俊與抱真同擊硃滔,以深、趙隸武俊,請改日知為晉、慈、隰節度使。」日知未至而三州降燧,故又加燧晉、慈、隰節度使。燧乃表讓三州於日知,且言因降而授之,恐後有功者踵以為常。上嘉而許之。燧乃遣使迎日知,既至,籍府庫而歸之,日知喜且過望。



 九月十五日,燧帥步騎三萬次於絳,分兵收夏縣,略稷山,攻龍門,降其將馮萬興、任象玉。燧以兵攻絳州,十月,拔其外城,其夜偽刺
 史王克同與大將達奚小進棄城走,降其眾四千人。又遣大將李自良、穀秀分兵略定聞喜、夏縣、萬泉、虞鄉、永樂、猗氏六縣,降其將辛兟及兵五千人。穀秀以犯令虜士女,斬之以徇。



 貞元元年,軍次寶鼎,敗賊騎兵於陶城,前鋒將李黯追擊之,射殺賊將徐伯文,斬首萬餘級,獲馬五百匹。是歲,天下蝗旱,物價騰踴,軍乏糧餉,而京師言事多請手舍懷光,上意未決。燧以懷光逆節尤甚,河中密邇京邑,反覆不可保信,舍之無以示天下,慮上為左
 右所惑,且兵事尚密。六月,燧乃舍軍以數百騎朝於京師。比召見,燧曰:「臣雖不武,得芻糧支一月,足以平河中。」上許之。



 七月,燧因朝京師,乃與渾瑊、駱元光、韓游瑰合軍,次於長春宮。懷光將徐廷光以兵六千守宮城,御備甚嚴。燧度長春不下,則懷光自固,攻之曠日持久,所傷必甚,乃挺身至城下呼廷光。廷光素憚燧威名,則拜於城上。燧度廷光心已屈,乃徐謂之曰:「我來自朝廷,可西面受命。」廷光復拜。燧乃喻之曰:「公等皆朔方將士,祿山
 以來,首建大動,四十餘年,功伐最高,奈何棄祖父之動力,背君上,為族滅之計耶!從吾,非止免禍,富貴可圖也。」賊徒皆不對。燧又曰:「爾以吾言不誠,今相去不遠數步,爾當射我!」乃披襟示之。廷光感泣俯伏,軍士亦泣下。先一日,賊焦籬堡守將尉珪以兵二千因堡降燧;廷光東道既絕,乃率眾出降。燧以數騎徑入城,處之不疑,莫不畏服,眾大呼曰:「吾輩復得為王人矣!」渾瑊由是服燧,私謂參佐曰:「予嘗謂馬公用兵與予不相遠,但警怪累敗
 田悅;今觀其行兵料敵,吾不迨遠矣!」八月,燧移軍於焦籬堡。其夜,賊太原堡守將吳冏棄堡而遁,其下皆降。燧率諸軍濟河,兵凡八萬,陣於城下。是日,賊將牛名俊斬懷光首以城降。其守兵猶一萬六千人,斬賊將閻晏、孟寶、張清、吳冏等七人以徇,為懷光脅虜者皆舍之。



 燧自朝京師還行營,凡二十七日而河中平。詔書褒美,遷光祿大夫,兼侍中,仍與一子五品正員官。宴賜畢,還太原。是行也,德宗賜燧《宸扆》、《臺衡》二銘。序曰:



 朕每覽上古之
 書,用及唐、虞之際,君臣相得,聖賢同時,日夕孜孜,講論至道,或陳其鑒誡,或諷以詠歌,煥乎典謨,百代是式,有以見啟沃之道,理化之端,意甚慕之,而未能迨也。頃靈監節度使杜希全著書上獻,多所規諫,聊為《君臣箴》,用答其意。河東等道副元帥、司徒燧固請勒石,貽厥後人。朕以文既非工,義又非備,垂諸來裔,良所恧焉。起予者商,因之有作,庶乎朝夕自儆,且俾後代知我文武殿邦之臣歟。



 《宸扆銘》曰:



 天生蒸人,性命元淳,嗜欲交馳,利害糾
 紛。無主乃亂,樹之以君,九域茫茫,萬情雲云。目不備睹,耳難遍聞,睹之聞之,矧又非真。事失其源,道遠莫親,理行其要,化行如神。失源維何,不自正身,正身之方,先誠其意。罔從爾欲,罔載爾偽,體道崇德,本仁率義。必信若寒暑,無私象天地,感而遂通,百慮一致。任人之術,各當其器,舍短從長,理無求備。事多總集,眾才咸遂,知而必任,任而勿貳。以天下之目為鑒,我鑒斯明;以天下之心為謀,我謀則智。求賢惟廣,辯理惟精,逆耳咈心,必嘉乃
 誠。順旨茍容,亦察其情,斥去奸諛,全度忠貞。先人立言,為代作程,諤諤者昌,唯唯者傾,系以興亡,曷云其輕。承天子人,夫豈不貴?伊昔哲王,夙夜祗畏。馭朽為戒,納隍為志,神將害盈,天匪假易。四海為家,夫豈不富?伊昔哲王,勤儉固陋。土階罔飾,露臺罷構,遠奇伎淫巧,放珍禽怪獸。敬之慎之,天命可祐。欲令必行,順人之情,欲誠必著,清己之慮,心無億詐,事必忠恕。凡將有為,靡不三思,喜怒以節,動靜以時。毫厘或差,禍害亦隨,慢易厥初,悔
 其曷追。刑不可長,武不可恃,作威逞力,厲階斯起。垂旒蔽聰,黈纊塞耳,含弘光大,是亦為美。覆之如天,愛之如子,仁心感人,率土自理。嗟予寡昧,嗣守丕圖,寇戎薦興,德化未孚。大業兢兢,其敢以渝,俯察物情,仰稽典謨,作誡斯言,置於坐隅。



 《臺衡銘》曰:



 天列臺星,垂象於人,聖人則天,亦建輔臣。以翼以弼,為衡為鈞,如耳目應心,如股肱連身,是則同體,孰云非親?陰陽相推,四序成歲,君臣相得,萬邦作乂。感同風雲,合若符契,以道匡救,盡規獻
 替,木必從繩,金其用礪。帝者之盛,時惟陶唐,乃聞疇咨,仄陋明易又。洎乎有虞,二八騰芳。爰迨伊尹,相於成湯。載生姜牙,諒彼武王。道無不行,謀無不臧,君聖臣賢,運泰時康。漢高既興,蕭、曹亦彰。烈烈我祖,膺期而昌,剷滅群兇,砥平四方。惟衛及英,啟闢封疆;曰房與杜,振理維綱;亦有魏徵,忠謇昂昂。偉茲眾材,為棟為梁,蕩蕩巍巍,邦家有光。是知道之廢興,系於時主,主之得失,資於臺輔。經之以文,緯之以武,出為方伯,入作申、甫,絕維載張,闕
 袞斯補。惟德是倚,惟才是求,人不易知,德亦難周。傅說板築,夷吾射鉤,任之不疑,千載垂休,體於至公,何鄙何讎。追惟哲主,必賴良弼,矧予不德,暗於理術。師旅繁起,政刑多失,遘茲艱屯,夙夜祗慄。翊我戴我,實惟勛賢,內熙庶績,外總十連,威武載揚,謀猷日宣。長城壓境,巨艦濟川,同德同心,扶危持顛。予嘉爾誠,爾相予理,惟後失道,亦臣之恥。自昔格言,慎終如始,功藏鼎彞,道冠圖史。無俾伊、傅,克專厥美,作鑒勒銘,永世是紀。



 燧至太原,乃
 勒二銘於起義堂西偏,帝為題額,其崇寵如此。



 二年冬,吐蕃大將尚結贊陷鹽、夏二州,各留兵守之,結贊大軍屯於鳴沙,自冬及春,羊馬多死,糧餉不繼。德宗以燧為綏、銀、麟勝招討使,令與華帥駱元光、邠帥韓游瑰及鳳翔諸鎮之師會於河西進討。燧出師,次石州。結贊聞之懼,遣使請和,仍約盟會,上皆不許。又遣其大將論頰熱厚禮卑辭申情於燧請和,燧頻表論奏,上堅不許。三年正月,燧軍還太原。四月,燧與論頰熱俱入朝,燧盛言蕃
 情可保,請許其盟,上然之。燧既入朝,結贊遽自鳴沙還蕃。是歲閏五月十五日,侍中渾瑊與蕃相尚結贊盟於平涼,為蕃軍所劫,狼狽僅免,陷將吏六十餘員,由燧之謬謀也,坐是奪兵權。六月,以燧守司徒,兼侍中、北平王如故,仍賜妓樂,奉朝請而已。



 五年九月,燧與太尉李晟召見於延英殿,上嘉其有大勛力,皆圖形凌煙閣,列於元臣之次。九年七月,燧對於延英。初,上以燧足疾,不令朝謁;是日,燧以冬首入朝,敕許不拜而坐。時太尉晟初
 薨,帝謂燧曰:「常時卿與太尉晟同來,今獨見卿,不覺悲慟。」上歔欷久之。燧既退,足疾,僕於地,上親掖起之,送及於陛,燧頓首泣謝。累上表乞骸,陳讓侍中,優詔不許。貞元十一年八月薨,時年七十。先是,司天頻奏熒惑太白犯太微上將,間一月而燧薨。廢朝四日,詔京兆尹韓皋監護喪事,嗣吳王獻為吊祭贈賵使,冊贈太尉,謚曰莊武。子匯、暢。



 暢以父廕累遷至鴻臚少卿,留京師。建中三年,燧討田悅於山東,時歲旱,京師括率商戶,人心甚搖。
 鳳翔留鎮幽州兵,多離散入南山為盜。殿中丞李雲端與其黨袁封、單超俊、李誠信、冀信等與暢善,因飲食聚會,言時事將危;暢乃遣家人溫靖與父書,具陳利害,可班師還鎮。燧怒,執靖具奏其狀,令兄炫執暢請罪。德宗以燧方討賊,不竟其事,誅雲端等十一人,敕炫就第杖暢三十,上於是罷括率之令。燧貲貨甲天下,燧既卒,暢承舊業,屢為豪幸邀取。貞元末,中尉楊志廉諷暢令獻田園第宅,順宗復賜暢。初為匯妻所訴,析其產,中貴又
 逼取,仍指使施於佛寺,暢不敢吝;晚年財產並盡,身歿之後,諸子無室可居,以至凍餒。今奉誠園亭館,即暢舊第也。暢終少府監,贈工部尚書。



 子繼祖,以祖廕,四歲為太子舍人,累遷至殿中少監,年三十七卒。



 炫,字弱翁,燧之仲兄,少以儒學聞於時,隱居蘇門山,不應闢召。至德中,李光弼鎮太原,闢為掌書記、試大理評事、監察御史,歷侍御史。常參謀議,光弼甚重之,奏授比部、刑部郎中。田神功鎮汴州,奏授節度判官、檢校兵部郎中。轉連州
 刺史,徵拜吏部郎中,又出為閬州刺史,入為大理少卿。建中初,為潤州刺史,黜陟使柳載以清白聞,徵拜太子右庶子,遷左散騎常侍。弟燧為司徒,以親比拜刑部侍郎,以疾辭,改兵部尚書致仕。貞元七年卒,時年七十九。



 史臣曰:燧雄勇強力,常先計後戰,又善誓師,將戰,親自號令,士無不慷慨感動,戰皆決死,未嘗折北,謀得兵勝,冠於一時。然力能擒田悅而不取,納蕃帥之偽款而保其必盟;平涼之會,大臣幾陷,關畿搖動,此謂才有餘而
 心不至,議者惜而恨之。



 渾瑊,皋蘭州人也,本鐵勒九姓部落之渾部也。高祖大俟利發渾阿貪支,貞觀中為皋蘭州刺史。曾祖元慶、祖大壽、父釋之,皆代為皋蘭都督。大壽,開元初歷左領衛中郎將、太子僕同正。釋之,少有武藝,從朔方軍,積戰功於邊上,累遷至開府儀同三司、試太常卿、寧朔郡王。廣德中,與吐蕃戰,沒於靈武,年四十九。



 瑊本名曰進,年十餘歲即善騎射,隨父戰伐,破賀魯部,下石保城,收龍駒
 島,勇冠諸軍,累授折沖果毅。後節度使安思順遣瑊提偏師深入葛祿部,經狐媚磧,略特羅斯山,大破阿布思部;又與諸軍城永清柵、天安軍,遷中郎將。



 安祿山構逆,瑊從李光弼出師河北,定諸郡邑。賊將有李立節者,素稱驍勇,與瑊格鬥,臨陣斬之,遷右驍衛將軍。既而肅宗即位於靈武,瑊統兵赴行在,至天德,遇蕃軍入寇,瑊擊敗之。從郭子儀收兩京,瑊討安慶緒,破賊於新鄉。改檢校太僕卿,充武鋒軍使。又從僕固懷恩討史朝義,前後數
 十戰。朝義平,加開府儀同三司、太常卿,賜實封二百戶。



 及懷恩謀亂,令子歊與瑊率軍圍榆次,朔方將殺歊,瑊率所部歸郭子儀。會瑊父釋之戰死,又起復本官,為朔方行營左廂兵馬使。從子儀討吐蕃於邠州。以功加御史中丞。軍還,盛秋於邠。會吐蕃大入寇,至奉天,瑊拒戰於漠谷,大破蕃軍,以功加太子賓客,復屯於奉天。華州周智光反,子儀奉詔討之,令瑊領馬步萬人攻下同州。智光平,詔以邠、寧、慶三州隸朔方軍,子儀領之;子儀令
 瑊先率兵至邠州,便於宜祿縣防秋。歲餘,加兼御史大夫。



 大歷七年,吐蕃大寇邊,瑊與涇原節度使馬璘會兵,大破蕃賊於黃菩原。自是,每年常戍於長武城,臨盛秋。十一年,領邠州刺史。其年,吐蕃入寇州方渠、懷安等鎮,瑊擊卻之。十二年,子儀入朝,令瑊知邠寧慶三州兵馬留後。十三年,回紇侵太原,破鮑防軍,北歸,頗為邊患。以瑊為石嶺關已南諸軍都知兵馬使,率兵掎角逐之,虜騎引退。其年八月,加檢校工部尚書、單于副都護、振
 武軍使。十四年,郭子儀拜太尉,號尚父,分所管內別置三節度,以瑊兼單于大都護,充振武軍、東受降城、鎮北大都護府、綏銀麟勝等軍州節度副大使知節度使事、管內支度營田等使。其年,復以崔寧為朔方節度使,領子儀舊管,徵瑊為左金吾衛大將軍,兼左街使。



 建中四年,李希烈遣間諜詐為瑊書與希烈交通,瑊奏其狀,上特保證之,仍賜瑊馬一匹並鞍轡,錦採二百匹。時以普王為荊襄等道兵馬元帥討李希烈,大開府幕,以瑊檢
 校戶部尚書、御史大夫,充中軍都虞候。會涇師亂,德宗幸奉天,後三日,瑊率家人子弟自京城至,乃署為行在都虞候、檢校兵部尚書、京畿渭北節度觀察使。居數日,邠寧節度使韓游瑰與慶州刺史論惟明統兵三千,自乾陵北過,赴醴泉以拒朱泚。會諜報泚已出兵,帝遽令追游瑰兵,才至奉天,賊軍果至。游瑰等戰於城東,王師不利,遂乘勝奔突,將入,官軍與賊隔門相持,自卯至午,殺傷頗甚。門內有草車數乘,瑊令推車塞門,焚之以外
 御,乘火力戰,賊方解去,然重圍已合。賊大修攻具,以僧法堅為匠師,毀佛寺房宇以為梯櫓。是月,賊自丁未至辛未,四面攻城,晝夜矢石不絕,瑊隨機應敵,僅能自固。



 十一月,靈武節度使杜希全、鹽州刺史戴休顏、夏州刺史常春合兵六千人赴難。將至,上議其所向,宰相盧杞、白志貞以漠谷路為便。瑊曰:「漠谷險隘,必為賊所邀,不若取乾陵北過,附柏城而行,便取城東北雞子堆下營,與城中掎角相應,且分賊勢,朱泚必不更於陵寢往來。」
 杞曰:「漠谷路近,若慮逆賊邀擊,即出兵應接,若取乾陵路,恐驚陵寢。」瑊曰:「今朱泚圍城,斬伐柏城,以夜繼晝,驚動已多。今城中危急,佇望救軍,唯希全等率先赴難,安危是賴,所繫非輕,制置不宜差跌。但令希全等於雞子堆下營,固守善地,賊泚可以計破也。」盧杞等曰:「陛下以順討逆,不可自驚陵寢。」白志貞從而贊之,上從杞議。希全等進至漠谷,果為賊軍邀擊,奪據水口,乘高以大弩、巨石左右夾擊,殺傷頗甚;城中出兵應援,亦為賊挫銳
 而退。希全等各歸還本鎮,賊攻城逾急,壕塹環之。旬日,復偏攻東北角,矢石亂入,晝夜如雨,城中死傷者甚眾。重圍救絕,芻粟俱盡,城中伺賊休息,輒遣人城外捃拾樵採以進御。人心危蹙,上與瑊對泣。賊泚北據乾陵,下瞰城內,身衣黃衣,蔽以翟扇,前後左右,皆朱紫閹官,宴賜拜舞,紛紜旁午。城中動息,賊俯窺之,慢辭戲侮,以為破在漏刻之頃,時令騎將環城招公卿、士庶,責以不識天命。十五日,賊造雲橋成,闊數十丈,以巨輪為腳,推之
 使前,施濕氈生牛革,多懸水囊以為障,直指城東北隅,兩旁構木為廬,冐以牛革,回環相屬,負土運薪於其下,以填壕塹,矢石不能傷。城中恟懼,相顧失色。上召瑊勉諭之,令齎空名告身自御史大夫、實封五百已下者千餘軸,募諸軍突將敢死之士以當之;兼賜瑊御筆一管,當戰勝,量其功伐,即署其名授之,不足者,筆書其身,因命以位。仍謂瑊曰:「朕便與卿別,更不用對來,縱有急切,令馬承倩在卿處,但令附奏。」瑊俯伏嗚咽,上亦悲慟
 不自勝,撫瑊背而遣之。前一日,瑊與防城使侯仲莊揣雲橋來路,先鑿地道,下可深丈餘,上積馬糞,深五六尺。次二日,即令爇火,次一日復下柴薪夜燒之,平明,火焰高於城壘。是時,北風正急,賊乃隨風推橋以薄城下,賊三千餘人相繼而登。城上士卒皆久寒餒,又少甲胄,瑊但感激誠厲之。以饑弱之眾,當劇賊之鋒,雖力戰應敵,人憂不濟,公卿已下,仰首祝天。賊徒至地道所,橋腳偏陷,不能進。須臾,風回焰轉,雲橋焚為灰燼,賊焚死者數
 千,城中歡噪振地。時瑊中流矢,遽自拔之,血流沾沫,格鬥不已,初不言瘡痛,以激士心。是日,上先授瑊二子官,餘授將校有差。賊又別造雲橋,周以重鐵,方就,而朔方節度使李懷光自魏縣行營赴難,先遣兵馬使張韶入奏。韶至奉天,與賊填塹者相雜,臨城忽大呼,謂城上曰:「我李懷光使也,懷光自河北領大軍至矣。」即繩引而登。城中得懷光表,歡聲振動,賊眾不之測,乃令舁韶巡於城上。翌日,懷光大軍次醴泉,是夜,賊解圍而去。



 興元元
 年正月,以瑊為行在都知兵馬使。二月,賜實封五百戶。是月,德宗移幸山南。時懷光叛逆,二賊連結,寇盜縱橫,瑊分布諸軍,以為翼衛,才入谷口,而懷光追騎遽至,瑊令侯仲莊以後軍擊敗之。三月,加檢校左僕射、同中書門下平章事,兼靈州都督、靈鹽豐夏等州、定遠西城天德軍節度等使,仍充朔方邠寧振武等道兼永平軍奉天行營兵馬副元帥,上臨軒授鉞,用漢拜韓信故事。是月,瑊將諸軍赴京畿,賊將韓旻、張廷芝、宋歸朝等拒我
 師於武功,瑊與吐蕃將論莽羅之眾大破賊於武亭川,斬首萬餘級。瑊便赴奉天應接李晟,抗京城西面。五月,李晟自東渭橋抵京城攻賊,瑊亦與韓游瑰、戴休顏西面諸軍會合。晟破賊之日,瑊亦進收咸陽。尋聞朱泚、姚令言奔敗,命諸軍分道邀擊,其眾離潰,相率來降。選勁騎三千急追泚至涇州,賊將誅泚,傳首來獻。六月,加瑊侍中。論收京城之功,加實封李晟一千戶,瑊八百戶,韓游瑰、戴休顏四百戶,駱元光、尚可孤五百戶。七月,德宗
 還宮,以瑊守本官,兼河中尹、河中絳慈隰節度使,仍充河中同陜虢節度及管內諸軍行營兵馬副元帥,改封咸寧郡王。九月,賜瑊大寧里甲第、女樂五人,入第之日,宰臣、節將送之,一如李晟入第之儀。以李懷光未平,又加朔方行營兵馬副元帥,與河東節度使馬燧會兵進討。貞元元年八月,河中平,以功加檢校司空,與一子五品正員官。是冬望,皇帝親郊昊天上帝,瑊入朝陪祀畢,還鎮河中。



 三年,吐蕃入寇,至鳳翔,為李晟邀擊之,又襲
 破其摧沙堡,吐蕃深恨之。尚結贊入寇,陷我鹽、夏二州,以兵守之。欲長驅犯京師,而畏瑊與李晟、馬燧,欲陰計圖之。乃卑詞遜禮告馬燧,請重立盟誓,則蕃軍引去,德宗不許。馬燧自入朝言之,上乃令崔翰入蕃報結贊,言還我鹽、夏,則許同盟。結贊謂翰曰:「清水之會,同盟人少,是以和好輕慢不成;今蕃相及元帥已下凡二十一人赴盟,靈州節度使杜希全、涇原節度使李觀皆和善守信,境外重之,此時須請預盟。」翰約盟於清水,且先歸我
 鹽、夏二州,結贊曰:「清水非吉地,請會盟於原州土梨樹。」又請盟畢歸二州。翰歸,備奏其事,神策將馬有麟奏曰:「土梨樹地多險,恐蕃軍隱伏不利,不如於平涼,其地坦平,且近涇州,就之為便。」乃定盟於平涼川。初,結贊請李觀、杜希全預盟,欲執之,徑犯京師。詔報之曰:「杜希全職在靈州,不可出境,李觀又已改官;今遣侍中渾瑊充盟會使。」五月,瑊自咸陽入朝,詔授平涼盟會使,兵部尚書崔漢衡副之,司勛郎中鄭叔矩為判官。瑊統兵二萬,又
 詔華州節度使駱元光以本鎮兵從瑊。閏月十五日,瑊與結贊會平涼。初,約以兵三千列於壇之東西,散手四百人至壇下,各遣游軍相覘伺。是時,蕃軍精騎數萬列於壇西,蕃之游軍貫穿我軍之中。瑊將梁奉貞率六十騎為游軍,才至壇所,為蕃軍所執。結贊又謂瑊曰:「請侍中已下具衣冠劍珮。」瑊與監軍宋鳳朝、崔漢衡等入幕次,坦無他慮。結贊命伐鼓三通,其眾呼噪而至。瑊遽出自幕後,偶得他馬,跨而奔馳,追騎雲合,流矢雨集而不
 傷。會瑊將辛榮以數百人入據北阜,與賊血戰,追騎方止,瑊僅得免,辛榮兵盡矢窮,力屈而降。宋鳳朝、瑊判官鄭弇,為追兵所殺;崔漢衡、中官俱文珍、劉延、李清朝,漢衡判官鄭叔矩、瑊判官路泌、袁同直,大將軍扶餘準、馬寧、神策將孟日華、李至言、樂演明、範澄、馬弇等六十餘人,皆陷於賊。尚結贊至原州,列坐帳中,召陷蕃將吏讓之,因怒瑊曰:「武功之捷,吐蕃之力,許以涇州、靈州相報,竟食其言,負我深矣,舉國同怨。本劫是盟,志在擒瑊。吾已
 為金枷待瑊,將獻贊普;既已失之,虛致君等何為?」乃放俱文珍、馬寧、馬弇歸朝。七月,瑊自奉天入朝,素服待罪,詔釋之而後見。俄而吐蕃入寇京畿,瑊鎮奉天。十月,還河中。四年七月,加邠、寧、慶副元帥。十二年二月,加檢校司徒,兼中書令,諸使、副元帥如故。十五年十二月二日,薨於鎮。廢朝五日,群臣於延英奉慰。詔贈太師,謚曰忠武,賻絹布四千匹、米粟三千石。及喪車將至,又為廢朝。應緣喪事,所司準式支給,命京兆尹監護。葬日,賜絹五
 百匹。



 瑊忠勤謹慎,功高不伐,在籓方歲時貢奉,必躬親閱視;每有頒錫,雖居遠地,如在帝前。位極將相,無忘謙抑,物論方之金日磾,故深為德宗委信,猜間不能入,君子多之。子練、鎬、金歲。



 鎬,瑊第二子。性謙謹,多與士大夫游。歷延、唐二州刺史,軍政吏職,有可稱者。及元和中,諸道出師討王承宗,屬義武軍節度使任迪簡病不能軍,以鎬藉父威名,足以鎮定,乃以鎬檢校右散騎常侍,充義武軍節度副使。九月六日,加檢校工部尚書,代迪簡為
 節度使。鎬治兵練卒,頗有威望,然不能觀釁養銳,以期必勝。鎮、定相去九十里,元和十一年冬,鎬率全師壓賊境而軍,距賊壘三十里。鎬謀慮不周,但耀兵鋒,無所控制,賊乃分兵潛入定州界焚燒驅掠。鎬怒,進攻賊壘,交鋒而敗,師徒殆喪其半,餘眾還定州,亂不可遏,朝廷乃除陳楚代之。楚聞亂,馳入定州。鎬為亂兵所劫,以至裸露。楚既整戢,於亂兵處率斂衣服還鎬,方得歸朝,坐貶韶州刺史。後代州刺史韓重華奏收得鎬供軍錢絹十
 餘萬貫匹,再貶循州刺史。歲餘卒。



 金歲,瑊第三子,以廕起家為諸衛參軍,歷諸衛將軍。元和初,出為豐州刺史、天德軍使,坐贓貶袁州司戶,憲宗思咸寧之勛,比例從輕。五年,徵為袁王傅,復賜金紫,遷殿中監。開成初,宰相擬壽州刺史,文宗曰:「金歲,勛臣子弟,豈可委以牧民?仲尼有言,『不如多與之邑』,今我念其先人之功,與之致富可也。」宰臣曰:「金歲常歷名郡,有政能。」乃從之。三年,入為右金吾衛大將軍、知街事,歷諸衛大將軍,卒。



 史臣曰:馬司徒之方略,渾咸寧之忠藎,各奮節義,為時名臣。然元城之師,失策於田悅;平涼之會,幾陷於吐蕃,此亦術有所不至也。緬思建中之亂,四海波騰,賊泚竊發之辰,宗祀不絕如線,茍非忠臣致命,化危為安,則李氏之宗社傾矣。



 贊曰:北平之勛,排難解紛。咸寧蹈義,感慨匡君。再隆基構,克殄昏氛。回天捧日,實賴將軍。



\end{pinyinscope}