\article{卷六十七 列傳第十七 李靖(客師 令問 彥芳) 李勣(孫敬業)}

\begin{pinyinscope}

 ○李
 靖客師令問彥芳李勣孫敬業



 李靖,本名藥師,雍州三原人也。祖崇義,後魏殷州刺史、永康公。父詮,隋趙郡守。靖姿貌瑰偉,少有文武材略,每謂所親曰:「大丈夫若遇主逢時,必當立功立事,以取富
 貴。」其舅韓擒虎,號為名將,每與論兵,未嘗不稱善,撫之曰:「可與論孫、吳之術者,惟斯人矣。」初仕隋為長安縣功曹,後歷駕部員外郎。左僕射楊素、吏部尚書牛弘皆善之。素嘗拊其床謂靖曰:「卿終當坐此。」大業末,累除馬邑郡丞。會高祖擊突厥於塞外,靖察高祖,知有四方之志,因自鎖上變,將詣江都,至長安,道塞不通而止。高祖克京城,執靖將斬之,靖大呼曰:「公起義兵,本為天下除暴亂,不欲就大事,而以私怨斬壯士乎!」高祖壯其言,太宗
 又固請,遂舍之。太宗尋召入幕府。武德三年,從討王世充,以功授開府。時蕭銑據荊州,遣靖安輯之。輕騎至金州,遇蠻賊數萬,屯聚山谷。廬江王瑗討之,數為所敗。靖與瑗設謀擊之,多所克獲。既至硤州,阻蕭銑,久不得進。高祖怒其遲留,陰敕硤州都督許紹斬之。紹惜其才,為之請命,於是獲免。會開州蠻首冉肇則反,率眾寇夔州,趙郡王孝恭與戰,不利。靖率兵八百,襲破其營,後又要險設伏,臨陣斬肇則,俘獲五千餘人。高祖甚悅,謂公卿
 曰:「朕聞使功不如使過,李靖果展其效。」因降璽書勞曰:「卿竭誠盡力,功效特彰。遠覽至誠,極以嘉賞,勿憂富貴也。」又手敕靖曰:「既往不咎,舊事吾久忘之矣。」四年,靖又陳十策以圖蕭銑。高祖從之,授靖行軍總管,兼攝孝恭行軍長史。高祖以孝恭未更戎旅,三軍之任,一以委靖。其年八月,集兵於夔州。銑以時屬秋潦,江水泛漲,三峽路險,必謂靖不能進,遂休兵不設備。九月,靖乃率師而進,將下峽,諸將皆請停兵以待水退,靖曰:「兵貴神速,機
 不可失。今兵始集,銑尚未知,若乘水漲之勢,倏忽至城下,所謂疾雷不及掩耳,此兵家上策。縱彼知我,倉卒徵兵,無以應敵,此必成擒也。」孝恭從之,進兵至夷陵。銑將文士弘率精兵數萬屯清江,孝恭欲擊之,靖曰:「士弘,銑之健將,士卒驍勇,今新失荊門,盡兵出戰,此是救敗之師,恐不可當也。宜自泊南岸,勿與爭鋒,待其氣衰,然後奮擊,破之必矣。」孝恭不從,留靖守營,率師與賊合戰。孝恭果敗,奔於南岸。賊舟大掠,人皆負重。靖見其軍亂,縱
 兵擊破之,獲其舟艦四百餘艘,斬首及溺死將萬人。孝恭遣靖率輕兵五千為先鋒,至江陵,屯營於城下。士弘既敗,銑甚懼,始徵兵於江南,果不能至。孝恭以大軍繼進,靖又破其驍將楊君茂、鄭文秀,俘甲卒四千餘人,更勒兵圍銑城。明日,銑遣使請降,靖即入據其城,號令嚴肅,軍無私焉。時諸將咸請孝恭云:「銑之將帥與官軍拒戰死者,罪狀既重,請籍沒其家,以賞將士。」靖曰:「王者之師,義存吊伐。百姓既受驅逼,拒戰豈其所願?且犬吠非
 其主,無容同叛逆之科,此蒯通所以免大戮於漢祖也。今新定荊、郢,宜弘寬大,以慰遠近之心,降而籍之,恐非救焚拯溺之義。但恐自此已南城鎮,各堅守不下,非計之善。」於是遂止。江、漢之域,聞之莫不爭下。以功授上柱國,封永康縣公,賜物二千五百段。詔命檢校荊州刺史,承制拜授。乃度嶺至桂州,遣人分道招撫,其大首領馮盎、李光度、寧真長等皆遣子弟來謁,靖承制授其官爵。凡所懷輯九十六州,戶六十餘萬。優詔勞勉,授嶺南道
 撫慰大使,檢校桂州總管。十六年,輔公祏於丹陽反,詔孝恭為元帥、靖為副以討之,李勣、任瑰、張鎮州、黃君漢等七總管並受節度。師次舒州,公祏遣將馮惠亮率舟師三萬屯當塗,陳正通、徐紹宗領步騎二萬屯青林山,仍於梁山連鐵鎖以斷江路,築卻月城,延袤十餘里,與惠亮為犄角之勢。孝恭集諸將會議,皆云:「惠亮、正通並握強兵,為不戰之計,城柵既固,卒不可攻。請直指丹陽,掩其巢穴,丹陽既破,惠亮自降。」孝恭欲從其議。靖曰:「公
 祏精銳,雖在水陸二軍,然其自統之兵,亦皆勁勇。惠亮等城柵尚不可攻,公祏既保石頭,豈應易拔?若我師至丹陽,留停旬月,進則公祏未平,退則惠亮為患,此便腹背受敵,恐非萬全之計。惠亮、正通皆是百戰餘賊,必不憚於野戰,止為公祏立計,令其持重,但欲不戰,以老我師。今欲攻其城柵,乃是出其不意,滅賊之機,唯在此舉。」孝恭然之。靖乃率黃君漢等先擊惠亮,苦戰破之,殺傷乃溺死者萬餘人,惠亮奔走。靖率輕兵先至丹陽,公祏
 大懼。先遣偽將左游仙領兵守會稽以為引援,公祏擁兵東走,以趨游仙,至吳郡,與惠亮、正通並相次擒獲,江南悉平。於是置東南道行臺,拜靖行臺兵部尚書,賜物千段、奴婢百口、馬百匹。其年,行臺廢,又檢校揚州大都督府長史。丹陽連罹兵寇,百姓凋弊,靖鎮撫之,吳、楚以安。八年,突厥寇太原,以靖為行軍總管,統江淮兵一萬,與張瑾屯大谷。時諸軍不利,靖眾獨全。尋檢校安州大都督。高祖每云:「李靖是蕭銑、輔公祏膏肓,古之名將韓、
 白、衛、霍,豈能及也!」九年,突厥莫賀咄設寇邊,征靖為靈州道行軍總管。頡利可汗入涇陽,靖率兵倍道趨豳州,邀賊歸路,既而與虜和親而罷。



 太宗嗣位,拜刑部尚書,並錄前後功,賜實封四百戶。貞觀二年,以本官兼檢校中書令。三年,轉兵部尚書。突厥諸部離叛,朝廷將圖進取,以靖為代州道行軍總管,率驍騎三千,自馬邑出其不意,直趨惡陽嶺以逼之。突利可汗不虞於靖,見官軍奄至,於是大懼,相謂曰:「唐兵若不傾國而來,靖豈敢孤
 軍而至?」一日數驚。靖候知之,潛令間諜離其心腹,其所親康蘇密來降。四年,靖進擊定襄,破之,獲隋齊王暕之子楊正道及煬帝蕭后,送於京師,可汗僅以身遁。以功進封代國公,賜物六百段及名馬、寶器焉。太宗嘗謂曰:「昔李陵提步卒五千,不免身降匈奴,尚得書名竹帛。卿以三千輕騎深入虜庭,克復定襄,威振北狄,古今所未有,足報往年渭水之役。」自破定襄後,頡利可汗大懼,退保鐵山,遣使入朝謝罪,請舉國內附。又以靖為定襄道
 行軍總管,往迎頡利。頡利雖外請朝謁,而潛懷猶豫。其年二月,太宗遣鴻臚卿唐儉、將軍安修仁慰諭,靖揣知其意,謂將軍張公謹曰:「詔使到彼,虜必自寬。遂選精騎一萬,齎二十日糧,引兵自白道襲之。」公謹曰:「詔許其降,行人在彼,未宜討擊。」靖曰:「此兵機也,時不可失,韓信所以破齊也。如唐儉等輩,何足可惜。」督軍疾進,師至陰山,遇其斥候千餘帳,皆俘以隨軍。頡利見使者,大悅,不虞官兵至也。靖軍將逼其牙帳十五里,虜始覺。頡利畏威
 先走,部眾因而潰散。靖斬萬餘級,俘男女十餘萬,殺其妻隋義成公主。頡利乘千里馬將走投吐谷渾,西道行軍總管張寶相擒之以獻。俄而突利可汗來奔,遂復定襄、常安之地,斥土界自陰山北至於大漠。太宗初聞靖破頡利,大悅,謂侍臣曰:「朕聞主憂臣辱,主辱臣死。往者國家草創,太上皇以百姓之故,稱臣於突厥,朕未嘗不痛心疾首,志滅匈奴,坐不安席,食不甘味。今者暫動偏師,無往不捷,單于款塞,恥其雪乎!」於是大赦天下,酺五
 日。御史大夫溫彥博害其功,譖靖軍無綱紀,致令虜中奇寶,散於亂兵之手。太宗大加責讓,靖頓首謝。久之,太宗謂曰:「隋將史萬歲破達頭可汗,有功不賞,以罪致戮。朕則不然,當赦公之罪,錄公之勛。」詔加左光祿大夫,賜絹千匹,真食邑通前五百戶。未幾,太宗謂靖曰:「前有人讒公,今朕意已悟,公勿以為懷。」賜絹二千匹,拜尚書右僕射。靖性沉厚,每與時宰參議,恂恂然似不能言。八年,詔為畿內道大使,伺察風俗。尋以足疾上表乞骸骨,言
 甚懇至。太宗遣中書侍郎岑文本謂曰:「朕觀自古已來,身居富貴,能知止足者甚少。不問愚智,莫能自知,才雖不堪,強欲居職,縱有疾病,猶自勉強。公能識達大體,深足可嘉,朕今非直成公雅志,欲以公為一代楷模。」乃下優詔,加授特進,聽在第攝養。賜物千段、尚乘馬兩匹,祿賜、國官府佐,並依舊給,患若小瘳,每三兩日至門下、中書平章政事。九年正月,賜靖靈壽杖,助足疾也。未幾,吐谷渾寇邊,太宗顧謂侍臣曰:「得李靖為帥,豈非善也!」靖
 乃見房玄齡曰:「靖雖年老,固堪一行。」太宗大悅,即以靖為西海道行軍大總管,統兵部尚書、任城王道宗、涼州都督李大亮、右衛將軍李道彥、利州刺史高甑生等三總管征之。九年,軍次伏俟城,吐谷渾燒去野草,以餧我師,退保大非川,諸將咸言春草未生,馬已羸瘦,不可赴敵。唯靖決計而進,深入敵境,遂逾積石山。前後戰數十合,殺傷甚眾,大破其國。吐谷渾之眾遂殺其可汗來降,靖又立大寧王慕容順而還。初,利州刺史高甑生為鹽
 澤道總管,以後軍期,靖薄責之,甑生因有憾於靖。及是,與廣州都督府長史唐奉義告靖謀反。太宗命法官按其事,甑生等竟以誣罔得罪。靖乃闔門自守,杜絕賓客,雖親戚不得妄進。十一年,改封衛國公,授濮州刺史,仍令代襲,例竟不行。十四年,靖妻卒,有詔墳塋制度,依漢衛、霍故事;築闕象突厥內鐵山、吐谷渾內積石山形,以旌殊績。十七年,詔圖畫靖及趙郡王孝恭等二十四人於凌煙閣。十八年,帝幸其第問疾,仍賜絹五百匹,進位衛
 國公、開府儀同三司。太宗將伐遼東,召靖入閣,賜坐御前,謂曰:「公南平吳會,北清沙漠,西定慕容,唯東有高麗未服,公意如何?」對曰:「臣往者憑藉天威,薄展微效,今殘年朽骨,唯擬此行。陛下不棄,老臣病期瘳矣。」太宗愍其羸老,不許。二十三年,薨於家,年七十九。冊贈司徒、并州都督,給班劍四十人、羽葆鼓吹,陪葬昭陵,謚曰景武。子德謇嗣,官至將作少匠。



 靖弟客師,貞觀中,官至右武衛將軍,以戰功累封丹陽郡公。永徽初,以年老致仕,性好
 馳獵,四時從禽,無暫止息。有別業在昆明池南,自京城之外,西際澧水,鳥獸皆識之,每出則鳥鵲隨逐而噪,野人謂之「鳥賊」。總章中卒,年九十餘。



 客師孫令問,玄宗在籓時與令問款狎,及即位,以協贊功累遷至殿中少監。先天中,預誅竇懷貞等功,封宋國公,實封五百戶。令問固辭實封,詔不許。開元中,轉殿中監、左散騎常侍,知尚食事。令問雖特承恩寵,未嘗干預時政,深為物論所稱。然厚於自奉,食饌豐侈,廣畜芻豢,躬臨宰殺。時方奉佛,
 其篤信之士或譏之。令問曰:「此物畜生,與果菜何異?胡為強生分別,不亦遠於道乎?」略不以恩眄自恃,閑適郊野,從禽自娛。十五年,涼州都督王君跂奉回紇部落叛,令問坐與連姻,左授撫州別駕,尋卒。



 大和中,令問孫彥芳,鳳翔府司錄參軍,詣闕進高祖、太宗所賜衛國公靖官告、敕書、手詔等十餘卷,內四卷太宗文皇帝筆跡,文宗寶惜不能釋手。其佩筆尚堪書,金裝木匣,制作精巧。帝並留禁中,令書工模寫本還之,賜芳絹二百匹、衣服、
 靴笏以酬之。



 李勣,曹州離狐人也。隋末徙居滑州之衛南。本姓徐氏,名世勣,永徽中,以犯太宗諱,單名勣焉。家多僮僕,積粟數千鍾,與其父蓋皆好惠施,拯濟貧乏,不問親疏。大業末,韋城人翟讓聚眾為盜,勣往從之,時年十七,謂讓曰:「今此土地是公及勣鄉壤,人多相識,不宜自相侵掠。且宋、鄭兩郡,地管御河,商旅往還,船乘不絕,就彼邀截,足以自相資助。」讓然之,於是劫公私船取物,兵眾大振。隋
 遣齊郡通守張須陀率師二萬討之,勣與頻戰,竟斬須陀於陣。初,李密亡命在雍丘,浚儀人王伯當匿於野,伯當共勣說翟讓奉密為主。隋令王世充討密,勣以奇計敗世充於洛水之上,密拜勣為東海郡公。時河南、山東大水,死者將半,隋帝令饑人就食黎陽,開倉賑給。時政教已紊,倉司不時賑給,死者日數萬人。勣言於密曰:「天下大亂,本是為饑,今若得黎陽一倉,大事濟矣。」密乃遣勣領麾下五千人自原武濟河掩襲,即日克之,開倉恣
 食,一旬之間,勝兵二十萬餘。經歲餘,宇文化及於江都弒逆,擁兵北上,直指東郡。時越王侗即位於東京,赦密之罪,拜為太尉,封魏國公;授勣右武候大將軍,命討化及。密遣勣守倉城,勣於城外掘深溝以固守,化及設攻具,四面攻倉,阻塹不得至城下,勣於塹中為地道,出兵擊之,大敗而去。



 武德二年,密為王世充所破,擁眾歸朝。其舊境東至於海,南至於江,西至汝州,北至魏郡,勣並據之,未有所屬,謂長史郭孝恪曰:「魏公既歸大唐,今此
 人眾土地,魏公所有也。吾若上表獻之,即是利主之敗,自為己功,以邀富貴,吾所恥也。今宜具錄州縣名數及軍人戶口,總啟魏公,聽公自獻,此則魏公之功也。」乃遣使啟密。使人初至,高祖聞其無表,惟有啟與密,甚怪之。使者以勣意聞奏,高祖大喜曰:「徐世勣感德推功,實純臣也。」詔授黎陽總管、上柱國,萊國公。尋加右武候大將軍,改封曹國公,賜姓李氏,賜良田五十頃,甲第一區。封其父蓋為濟陰王,蓋固辭王爵,乃封舒國公,授散騎常
 侍、陵州刺史。令勣總統河南、山東之兵以拒王世充。及李密反叛伏誅,高祖以勣舊經事密,遣使報其反狀。勣表請收葬,詔許之。勣服衰絰,與舊僚吏將士葬密於黎山之南,墳高七仞,釋服而散,朝野義之。而竇建德擒化及於魏縣,復進軍攻勣,力屈降之。建德收其父,從軍為質,令勣復守黎陽。三年,自拔歸京師。四年,從太宗伐王世充於東都,累戰大捷。又東略地至武牢,偽鄭州司兵沈悅請翻武牢,勣夜潛兵應接,克之。擒其偽刺史荊
 王行本。又從太宗平竇建德,降王世充,振旅而還。論功行賞,太宗為上將,勣為下將,與太宗俱服金甲,乘戎輅,告捷於太廟。其父自洺州與裴矩入朝,高祖見之大喜,復其官爵。勣又從太宗破劉黑闥、徐圓朗,累遷左監門大將軍。圓朗重據兗州反,授勣河南大總管以討之,尋獲圓朗,斬首以獻,兗州平。七年,詔與趙郡王孝恭討輔公祏,孝恭領舟師巡江而下,勣領步卒一萬渡淮,拔其壽陽,至硤石。公祏之將陳正通率兵十萬屯於梁山,又
 遣其大將馮惠亮帥水軍十萬,鎖連大艦以斷江路,仍於江西結壘,分守水陸,以禦王師。勣攻其壘,尋克之。惠亮單舼而遁。勣乘勝逼正通,大潰,以十餘騎奔於丹陽。公祐棄城夜遁,勣縱騎追斬之於武康,江南悉定。八年,突厥寇并州,命勣為行軍總管,擊之於太谷,走之。太宗即位,拜并州都督,賜實封九百戶。貞觀三年,為通漠道行軍總管。至雲中,與突厥頡利可汗兵會,大戰於白道。突厥敗,屯營於磧口,遣使請和。詔鴻臚卿唐儉往赦之。
 勣時與定襄道大總管李靖軍會,相與議曰:「頡利雖敗,人眾尚多,若走渡磧,保於九姓,道遙阻深,追則難及。今詔使唐儉至彼,其必弛備,我等隨後襲之,此不戰而平賊矣。」靖扼腕喜曰:「公之此言,乃韓信滅田橫之策也。」於是定計。靖將兵逼夜而發,勣勒兵繼進。靖軍既至,賊營大潰,頡利與萬餘人欲走渡磧。勣屯軍於磧口,頡利至,不得渡磧,其大酋長率其部落並降於勣,虜五萬餘口而還。時高宗為晉王,遙領并州大都督,授勣光祿大夫,
 行并州大都督府長史。父憂解,尋起復舊職。十一年,改封英國公,代襲蘄州刺史,時並不就國,復以本官遙領太子左衛率。勣在并州凡十六年,令行禁止,號為稱職。太宗謂侍臣曰:「隋煬帝不能精選賢良,安撫邊境,惟解築長城以備突厥,情識之惑,一至於此!朕今委任李世勣於并州,遂使突厥畏威遁走,塞垣安靜,豈不勝遠築長城耶?」



 十五年,徵拜兵部尚書,未赴京,會薛延陀遣其子大度設帥騎八萬南侵李思摩部落。命勣為朔州行
 軍總管,率輕騎三千追及延陀於青山,擊大破之,斬其名王一人,俘獲首領,虜五萬餘計,以功封一子為縣公。勣時遇暴疾,驗方云,鬚灰可以療之,太宗乃自剪鬚,為其和藥。勣頓首見血,泣以懇謝,帝曰:「吾為社稷計耳,不煩深謝。」十七年,高宗為皇太子,轉勣太子詹事兼左衛率,加位特進,同中書門下三品。太宗謂曰:「我兒新登儲貳,卿舊長史,今以宮事相委,故有此授。雖屈階資,可勿怪也。」太宗又嘗閑宴,顧勣曰:「朕將屬以幼孤,思之無越
 卿者。公往不遺於李密,今豈負於朕哉!」勣雪涕致辭,因噬指流血。俄而沉醉,乃解御服覆之,其見委信如此。十八年,太宗將親征高麗,授勣遼東道行軍大總管,攻破蓋牟、遼東、白崖等數城,又從太宗摧殄駐蹕陣,以功封一子為郡公。二十年,延陀部落擾亂,詔勣將二百騎便發突厥兵討擊。至烏德鞬山,大戰破之。其大首領梯真達于率眾來降,其可汗咄摩支南竄於荒谷,遣通事舍人蕭嗣業招慰部領,送於京師,磧北悉定。二十二年,轉
 太常卿,仍同中書門下三品。旬日,復除太子詹事。二十三年,太宗寢疾,謂高宗曰:「汝於李勣無恩,我今將責出之。我死後,汝當授以僕射,即荷汝恩,必致其死力。」乃出為疊州都督。高宗即位,其月,召拜洛州刺史,尋加開府儀同三司,令同中書門下,參掌機密。是歲,冊拜尚書左僕射。永徽元年,抗表求解僕射,仍令以開府儀同三司依舊知政事。四年,冊拜司空。初,貞觀中,太宗以勳庸特著,嘗圖其形於凌煙閣,至是,帝又命寫形焉,仍親為之
 序。顯慶三年,從幸東都,在路遇疾,帝親臨問。麟德初,東封泰山,詔勣為封禪大使,乃從駕。次滑州,其姊早寡,居勣舊閭,皇后親自臨問,賜以衣服,仍封為東平郡君。又墜馬傷足,上親降問,以所乘賜之。



 乾封元年,高麗莫離支男產為其弟男建所逐,保於國內城,遣子獻城詣闕乞師。總章元年,命勣為遼東道行軍總管,率兵二萬略地至鴨綠水。賊遣其弟來拒戰,勣縱兵擊敗之,追奔二百里,至於平壤城。男建閉門不敢出,賊中諸城駭懼,
 多拔人眾遁走,降款者相繼。勣又引兵圍平壤,遼東道副大總管劉仁軌、郝處俊、將軍薛仁貴並會於平壤,犄角圍之。經月餘,克其城,虜其王高藏及男建、男產,裂其諸城,並為州縣,振旅而旋。令勣便道以高藏及男建獻於昭陵,禮畢,備軍容入京城,獻太廟。二年,加太子太師,增食實封通前一千一百戶。其年寢疾,詔以勣為司衛正卿,使得視疾。尋薨,年七十六。帝為之舉哀,輟朝七日,贈太尉、揚州大都督,謚曰貞武,給東園
 秘器,陪葬昭陵。令司平太常伯楊昉攝同文正卿監護。及葬日,帝幸未央古城,登樓臨送,望柳車慟哭,並為設祭。皇太子亦從駕臨送,哀慟悲感左右。詔百官送至故城西北,所築墳一準衛、霍故事,象陰山、鐵山及烏德鞬山,以旌破突厥、薛延陀之功。光宅元年,詔勣配享高宗廟庭。



 勣前後戰勝所得金帛,皆散之於將士。初得黎陽倉,就倉者數十萬人。魏徵、高季輔、杜正倫、郭孝恪皆遊其所,一見於眾人中,即加禮敬,引之臥內,談謔忘倦。及
 平武牢,獲偽鄭州長史戴胄,知其行能,尋釋於竟,推薦咸至顯達,當時稱其有知人之鑒。又,初平王世充,獲其故人單雄信,依例處死,勣表稱其武藝絕倫,若收之於合死之中,必大感恩,堪為國家盡命,請以官爵贖之。高祖不許,臨將就戮,勣對之號慟,割股肉以啖之,曰:「生死永訣,此肉同歸於土矣。」仍收養其子。每行軍用師,頗任籌算,臨敵應變,動合事機。與人圖計,識其臧否,聞其片善,扼腕而從。事捷之日,多推功於下,以是人皆為用,所
 向多克捷。洎勣之死,聞者莫不淒愴。與弟弼特存友愛,閨門之內,肅若嚴君。自遇疾,高宗及皇太子送藥,即取服之;家中召醫巫,皆不許入門。子弟固以藥進,勣謂曰:「我山東一田夫耳,攀附明主,濫居富貴,位極三臺,年將八十,豈非命乎?修短必是有期,寧容浪就醫人求活!」竟拒而不進。忽謂弼曰:「我似得小差,可置酒以申宴樂。」於是堂上奏女妓,簷下列子孫。宴罷,謂弼曰:「我自量必死,欲與汝一別耳。恐汝悲哭,誑言似差,可未須啼泣,聽我
 約束。我見房玄齡、杜如晦、高季輔辛苦作得門戶,亦望垂裕後昆,並遭癡兒破家蕩盡。我有如許豚犬,將以付汝,汝可防察,有操行不倫、交遊非類,急即打殺,然後奏知。又見人多埋金玉,亦不須爾。惟以布裝露車,載我棺柩,棺中斂以常服,惟加朝服一副,死倘有知,望著此奉見先帝。明器惟作馬五六匹,下帳用幔布為頂,白紗為裙,其中著十個木人,示依古禮芻靈之義,此外一物不用。姬媼已下,有兒女而願住自養者,聽之;餘並放出。事
 畢,汝即移入我堂,撫恤小弱。違我言者,同於戮屍。」此後略不復語,弼等遵行遺言。



 勣少弟感,幼有志操。李密之敗也,陷於王世充,世充逼令以書召勣,感曰:「家兄立身,不虧名節,今已事主,君臣分定,決不以感造次改圖。」卒不肯。世充怒,遂害焉,時年十五。勣長子震,顯慶初官至桂州刺史,先勣卒。



 勣孫敬業。高宗崩,則天太后臨朝,既而廢帝為廬陵王,立相王為皇帝,而政由天后,諸武皆當權任,人情憤怨。時給事中唐之奇貶授括蒼令,長安
 主簿駱賓王貶授臨海丞,詹事司直杜求仁黝縣丞,敬業坐事左授柳州司馬,其弟盩厔令敬猷亦坐累左遷,俱在揚州。敬業用前盩厔尉魏思溫謀,據揚州。嗣聖元年七月,敬業遣其黨監察御史薛璋先求使江都,又令雍州人韋超詣璋告變,云「揚州長史陳敬之與唐之奇謀逆」,璋乃收敬之繫獄。居數日,敬業矯制殺敬之,自稱揚州司馬,詐言「高州首領馮子猷叛逆,奉密詔募兵進討。」是日開府庫,令士曹參軍李宗臣解繫囚及丁役、工
 匠,得數百人,皆授之以甲。錄事參軍孫處行拒命,敬業斬之以徇。遂據揚州,鳩聚民眾,以匡復廬陵為辭。乃開三府:一曰匡復府,二曰英公府,三曰揚州大都督府。敬業自稱匡復府上將,領揚州大都督,以杜求仁、唐之奇、駱賓王為府屬,餘皆偽署職位。旬日之間,勝兵有十餘萬。仍移檄諸郡縣曰:



 偽臨朝武氏者,人非溫順,地實寒微。昔充太宗下陳,嘗以更衣入侍。洎乎晚節,穢亂春宮。密隱先帝之私,陰圖後庭之嬖。入門見嫉,蛾眉不肯讓
 人;掩袖工讒,狐媚偏能惑主。踐元后於翬翟,陷吾君於聚麀。加以虺蠍為心,豺狼成性,近狎邪僻,殘害忠良,殺姊屠兄,弒君鴆母。人神之所同嫉,天地之所不容。猶復包藏禍心,窺竊神器。君之愛子,幽之於別宮;賊之宗盟,委之以重任。嗚呼!霍子孟之不作,朱虛侯之已亡。燕啄皇孫,知漢祚之將盡;龍漦帝后,識夏廷之遽衰。



 敬業皇唐舊臣,公侯冢胤,奉先君之成業,荷本朝之舊恩。宋微子之興悲,良有以也;袁君山之流涕,豈徒然哉!是用氣
 憤風雲,志安社稷,因天下之失望,順宇內之推心。爰舉義旗,誓清妖孽。南連百越,北盡三河,鐵騎成群,玉舳相接。海陵紅粟,倉儲之積靡窮;江浦黃旗,匡復之功何遠!班聲動而北風起,劍氣沖而南斗平。喑嗚則山嶽崩頹,叱吒則風雲變色。以此制敵,何敵不摧?以此圖功,何功不克?



 公等或家傳漢爵,或地協周親,或膺重寄於爪牙,或受顧命於宣室。言猶在耳,忠豈忘心?一抔之土未乾,六尺之孤何託?倘能轉禍為福,送往事居,共立勤王之
 師,無廢舊君之命,凡諸爵賞,同裂山河。請看今日之域中,竟是誰家之天下!



 則天命左玉鈐衛大將軍李孝逸將兵三十萬討之,追削敬業祖、父官爵,剖墳斫棺,復本姓徐氏。初,敬業兵集,圖其所向,薛璋曰:「金陵王氣猶在,大江設險,可以自固。且取常、潤等州,以為霸基,然後治兵北渡。」魏思溫曰:「兵貴神速,但宜早渡淮而北,招合山東豪傑,乘其未集,直取東都,據關決戰,此上策也。」敬業不從。十月,率眾渡江,攻拔潤州,殺刺史李思文。先是,太
 子賢為天后所廢,死於巴州,敬業乃求狀貌似賢者,置於城中,奉之為主,云賢本不死。孝逸軍渡淮,至楚州,敬業之眾狼狽還江都,屯兵高郵以拒之。頻戰大敗,孝逸乘勝追躡。敬業奔至揚州,與唐之奇、杜求仁等乘小舸,將入海投高麗。追兵及,皆捕獲之。初,敬業傳檄至京師,則天讀之微哂,至「一抔之土未乾」,遽問侍臣曰:「此語誰為之?」或對曰:「駱賓王之辭也。」則天曰:「宰相之過,安失此人?」中宗返正,詔曰:「故司空勣,往因敬業,毀廢墳塋。朕追
 想元勳,永懷佐命。昔竇憲干紀,無累安豐之祠;霍禹亂常,猶全博陸之祀。罪不相及,國之通典。宜特垂恩禮,令所司速為起墳,所有官爵,並宜追復。」勣諸子孫坐敬業誅殺,靡有遺胤,偶脫禍者,皆竄跡胡越。貞元十七年,吐蕃陷麟州,驅掠民畜而去。至鹽州西橫槽烽,蕃將號徐舍人者,環集漢俘於呼延州,謂僧延素曰:「師勿甚懼,予本漢五代孫也。屬武太后斫喪王室,吾祖建義不果,子孫流落絕域,今三代矣。雖代居職任,掌握兵要,然思本
 之心,無忘於國。但族屬已多,無由自拔耳。此地蕃漢交境,放師還鄉。」數千百人,解縛而遣之。



 史臣曰:近代稱為名將者,英、衛二公,誠煙閣之最。英公振彭、黥之跡,自拔草莽,常能以義籓身,與物無忤,遂得功名始終。賢哉,垂命之誡!敬業不蹈貽謀,至於覆族,悲夫!衛公將家子,綽有渭陽之風。臨戎出師,凜然威斷。位重能避,功成益謙。銘之鼎鍾,何慚耿、鄧。美哉!



 贊曰:功以懋賞,震主則危。辭祿避位,除猜破疑。功定華
 夷,志懷忠義。白首平戎,賢哉英、
 衛。



\end{pinyinscope}