\article{卷二百一十二  鐵勒契丹奚室韋靺鞨渤海靺鞨霫烏羅渾}

\begin{pinyinscope}

 ○鐵勒契丹奚室韋靺鞨渤海靺鞨霫烏羅渾



 鐵勒,本匈奴別種。自突厥強盛,鐵勒諸郡分散,眾漸寡
 弱。至武德初,有薛延陀、契苾、回紇、都播、骨利幹、多覽葛、僕骨、拔野古、同羅、渾部、思結、斛薛、奚結、阿跌、白霫等,散在磧北。薛延陀者,自云本姓薛氏,其先擊滅延陀而有其眾,因號為薛延陀部。其官方兵器及風俗,大抵與突厥同。



 初,大業中,西突厥處羅可汗始強大,鐵勒諸部皆臣之,而處羅征稅無度,薛延陀等諸部皆怨,處羅大怒,誅其酋帥百餘人。鐵勒相率而叛,共推契苾哥楞為易勿真莫賀可汗,居貪汗山北;又以薛延陀乙失缽為也
 咥小可汗,居燕末山北。西突厥射匱可汗強盛,延陀、契苾二部並去可汗之號以臣之。回紇等六部在鬱督軍山者,東屬於始畢,乙失缽所部在金山者,西臣於葉護。



 貞觀二年,葉護可汗死,其國大亂。乙失缽之孫曰夷男。率其部落七萬餘家附於突厥。遇頡利之政衰,夷男率其徒屬反攻頡利,大破之。於是頡利部諸姓多叛頡利,歸於夷男,共推為主,夷男不敢當。時太宗方圖頡利,遣游擊將軍喬師望從間道齎冊書拜夷男為真珠毗伽
 可汗,賜以鼓纛。夷男大喜,遣使貢方物,復建牙於大漠之北鬱督軍山下,在京師西北六千里。東至靺鞨,西至葉護,南接沙磧,北至俱倫水,回紇、拔野古、阿跌、同羅、僕骨、霫諸大部落皆屬焉。



 三年,夷男遣其弟統特勒來朝,太宗厚加撫接,賜以寶刀及寶鞭。謂曰:「汝所部有大罪者鞭之。」夷男甚喜。



 四年,平突厥頡利之後,朔塞空虛,夷男率其部東返故國,建庭於都尉揵山北,獨邏河之南,在京師北三千三百里;東至室韋,西至金山,南至突厥,
 北臨瀚海,即古匈奴之故地。勝兵二十萬,立其二子為南北部。太宗亦以其強盛,恐為後患。



 十二年,遣使備禮冊命,拜其二子皆為小可汗,外示優崇,實欲分其勢也。會朝廷立李思摩為可汗,處其部眾於漠南之地。夷男心惡思摩,甚不悅。



 十五年,太宗幸洛陽,將有事於太山。夷男謀於其國曰:「天子封太山,萬國必會,士馬皆集,邊境空虛,我於此時取思摩如拉朽耳。」因命其子大度設勒兵二十萬,屯白道川,據善陽嶺以擊思摩之部。思摩
 遣使請救,詔英國公李勣、蒲州刺史薛萬徹率步騎數萬赴之。逾白道川至青山,與大度設相及。追之累月,至諾真水,大度設知不脫,乃互十里而陳兵。



 先是,延陀擊沙缽羅及阿史那社爾等,以步戰而勝。及其將來寇也,先講武於國中,教習步戰;每五人,以一人經習戰陣者使執馬,而四人前戰;克勝即授馬以追奔,失應接罪至於死,沒其家口,以賞戰人,至是遂行其法。突厥兵先合輒退,延陀乘勝而逐之。勣兵拒擊,而延陀萬矢俱發,傷
 我戰馬。乃令去馬步陣,率長槊數百為隊,齊奮以沖之,其眾潰散。副總管薛萬徹率數千騎收其執馬者。其眾失馬,莫知所從,因大縱,斬首三千餘級,獲馬萬五千匹,甲仗輜重不可勝計。大度設跳身而遁,萬徹將數百騎追之,弗及。其餘眾大奔走,相騰踐而死者甚眾,伏尸被野。夷男因乞與突厥和,並遣使謝罪。



 十六年,遣其叔父沙缽羅泥敦策斤來請婚,獻馬三千匹。太宗謂侍臣曰:「北狄世為寇亂,今延陀崛強,須早為之所。朕熟思之,唯
 有二策:選徒十萬,擊而虜之,滅除兇醜,百年無事,此一策也;若遂其來請,結以婚姻,緩轡羈縻,亦足三十年安靜,此亦一策也。未知何者為先?」司空房玄齡對曰:「今大亂之後,瘡痍未復,且兵兇戰危,聖人所慎。和親之策,實天下幸甚。」太宗曰:「朕為蒼生父母,茍可以利之,豈惜一女?」遂許以新興公主妻之。因征夷男備親迎之禮。仍發詔將幸靈州與之會。夷男大悅,謂其國中曰:「我本鐵勒之小帥也,天子立我為可汗,今復嫁我公主,車駕親至
 靈州,斯亦足矣!」於是稅諸部羊馬以為聘財。或說夷男曰:「我薛延陀可汗與大唐天子俱一國主,何有自往朝謁?如或拘留,悔之無及!」夷男曰:「吾聞大唐天子聖德遠被,日月所照,皆來賓服。我歸心委質,冀得睹天顏,死無所恨!然磧北之地,必當有主,舍我別求,固非大國之計。我志決矣,勿復多言!」於是言者遂止。太宗乃發使受其羊馬。然夷男先無府藏,調斂其國,往返且萬里,既涉沙磧,無水草,羊馬多死,遂後期。太宗於是停幸靈州。既而
 其聘羊馬來至,所耗將半。議者以為夷狄不可禮義畜,若聘財未備而與之婚,或輕中國,當須要其備禮,於是下詔絕其婚。既而李思摩數遣兵侵掠之。延陀復遣突利失擊思摩,至定襄,抄掠而去。太宗遣英國公李勣援之,見虜已出塞而還。太宗以其數與思摩交兵,璽書責讓之。



 十九年,謂其使人曰:「語爾可汗,我父子並東征高麗,汝若能寇邊者,但當來也!」夷男遣使致謝,復請發兵助軍,太宗答以優詔而止。其冬,太宗拔遼東諸城,破駐
 蹕陳,而高麗莫離支潛令靺鞨誑惑夷男,啖以厚利,夷男氣懾不敢動。俄而夷男卒,太宗為之舉哀。夷男少子肆葉護拔灼襲殺其兄突利失可汗而自立,是為頡利俱利薛沙多彌可汗。拔灼性褊急,馭下無恩,多所殺戮,其下不附。是時復以太宗尚在遼東,遂發兵寇夏州,將軍執失思力擊敗之,虜其眾數萬,拔灼輕騎遁去,尋為回紇所殺,宗族殆盡。其餘眾尚五六萬,竄於西域,又諸姓俟斤遞相攻擊,各遣使歸命。



 二十年,太宗遣使江夏
 王道宗、左衛大將軍阿史那社爾為瀚海道安撫大使;右領軍大將軍執失思力領突厥兵,代州都督薛萬徹、營州都督張儉、右驍衛大將軍契苾何力各統所部兵分道並進,太宗親幸靈州,為諸軍聲援。既而道宗渡磧,遇延陀餘眾數萬來拒戰。道宗擊破之,斬首千餘級。萬徹又與回紇相遇,二將各遣使諭以綏懷之意。其酋帥見使者,皆頓顙歡呼,請入朝。太宗至靈州,其鐵勒諸部相繼至數千人,仍請列為州縣,北荒悉平。詔曰:



 惟天為
 大,合其德者弗違;謂地蓋厚,體其仁者光被。故能彌倫八極,輿蓋二儀,振絕代之英聲,畢天下之能事。彼匈奴者,與開闢而俱生;奄有龍庭,共上皇而並列。僭稱驕子,分天街於紫宸;仰應旄頭,抗大禮於皇極。糸面窺邃古,能無力制。自朕臨御天下,二紀於茲,粵以眇身,一匡寰宇。始勤勞於昧旦,終致治於升平。曩者聊命偏師,遂擒頡利;今茲始弘廟略,已滅延陀。雖麾駕出征,未逾郊甸;前驅所轥,才掩塞垣。長策風行,已振金徽之表,揚威電發,
 遠璟沙場之外。鐵勒諸姓、回紇胡祿俟利發等,總百餘萬戶,散處北溟,遠遣使人,委身內屬,請同編列,並為州郡。收其瀚海,盡入提封;解其辮發,並垂冠帶。上變星昴,歸於東井之躔;下掩蹛林,袪入南山之囿。混元已降,殊未前聞;無疆之業,永貽來裔。古人所不能致,今既吞之;前王所不能屈,今咸滅之。斯實書契所未有,古今之壯觀,豈朕一人獨能宣力!蓋由上靈儲祉,錫以太康;宗廟威靈,成茲克定。即宜備禮,告於清廟,仍頒示普天。



 其後延陀
 西遁之眾,共推夷男兄子咄摩支為伊特勿失可汗,率部落七萬餘口,西歸故地。乃去可汗之號,遣使奉表,請居鬱督軍山北。詔兵部尚書崔敦禮就加綏撫。而諸部鐵勒素服薛延陀之眾,及咄摩支至,九姓渠帥莫不危懼。朝議恐為磧北之患,復令英國公李勣進加討擊。勣率九姓鐵勒二萬騎至於天山。咄摩支見官軍奄至,惶駭不知所為;且聞詔使蕭嗣業在回紇中,因而請降。嗣業與之俱至京師,詔授右武衛將軍,賜以田宅。咄摩支
 入國後,鐵勒酋帥潛知其部落,仍持兩端。李勣因縱兵追擊,前後斬五千餘級,虜男女三萬計。



 二十二年,契苾、回紇等十餘部落以薛延陀亡散殆盡,乃相繼歸國。太宗各因其地土,擇其部落,置為州府:以回紇部為瀚海都督府,僕骨為金徽都督府,多覽葛為燕然都督府,拔野古部為幽陵都督府,同羅部為龜林都督府,思結部為盧山都督府,渾部為皋蘭州,斛薛部為高闕州,奚結部為雞鹿州,阿跌部為雞田州,契苾部為榆溪州,思結
 別部為蹛林州,白霫部為寘顏州,凡一十三州。拜其酋長為都督、刺史,給玄金魚以為符信,又置燕然都護以統之。是歲,太宗以鐵勒諸部並皆內屬,詔賜京城百姓大酺三日。



 永徽元年,延陀首領先逃逸者請歸國,高宗更置溪彈州以安恤之。至則天時,突厥強盛,鐵勒諸部在漠北者漸為所並。回紇、契苾、思結、渾部徙於甘、涼二州之地。



 其骨利幹北距大海,去京師最遠,自古未通中國。貞觀中遣使來朝貢,遣雲麾將軍康蘇密往慰撫之,
 仍列其地為玄闕州。俄又遣使隨蘇密使入朝,獻良馬十匹。太宗奇其駿異,為之制名。號為十驥:一曰騰霜白,二曰皎雪驄,三曰凝露驄,四曰懸光驄,五曰決波騟,六曰飛霞驃,七曰發電赤,八曰流金瓜,九曰翱麟紫,十曰奔虹赤。又為文以敘其事。自延陀叛後,朝貢遂絕。



 契丹,居潢水之南,黃龍之北,鮮卑之故地,在京城東北五千三百里。東與高麗鄰,西與奚國接,南至營州,北至室韋。冷陘山在其國南,與奚西山相崎,地方二千里。逐
 獵往來,居無常處。其君長姓大賀氏。勝兵四萬三千人,分為八部,若有徵發,諸部皆須議合。不得獨舉。獵則別部,戰則同行。本臣突厥,好與奚斗,不利則遁保青山及鮮卑山。其俗死者不得作塚墓,以馬駕車送入大山,置之樹上,亦無服紀。子孫死,父母晨夕哭之;父母死,子孫不哭。其餘風俗與突厥同。



 武德初,數抄邊境。二年,入寇平州。六年,其君長咄羅遣使貢名馬豐貂。貞觀二年,其君摩會率其部落來降。突厥頡利遣使請以梁師都易
 契丹,太宗謂曰:「契丹、突厥,本是別類,今來降我,何故索之?師都本中國人,據我州城,以為盜竊,突厥無故容納之,我師往討,便來救援。計不久自當擒滅。縱其不得,終不以契丹易之。」



 太宗伐高麗,至營州,會其君長及老人等,賜物各有差,授其蕃長窟哥為左武衛將軍。



 二十二年,窟哥等部咸請內屬,乃置松漠都督府,以窟哥為左領軍將軍兼松漠都督府、無極縣男,賜姓李氏。顯慶初,又拜窟哥為左監門大將軍。其曾孫祜莫離,則天時歷
 左衛將軍兼檢校彈汗州刺史,歸順郡王。



 又契丹有別部酋帥孫敖曹,初仕隋為金紫光祿大夫。武德四年,與靺鞨酋長突地稽俱遣使內附,詔令於營州城傍安置,授雲麾將軍,行遼州總管。至曾孫萬榮,垂拱初累授右玉鈐衛將軍、歸誠州刺史,封永樂縣公。萬歲通天中,萬榮與其妹婿松漠都督李盡忠,俱為營州都督趙翽所侵侮,二人遂舉兵殺翽,據營州作亂。盡忠即窟哥之胤,歷位右武衛大將軍兼松漠都督。則天怒其叛亂,下詔
 改萬榮名為萬斬,盡忠為盡滅。盡滅尋自稱無上可汗,以萬斬為大將,前鋒略地,所向皆下,旬日兵至數萬,進逼檀州。詔令右金吾大將軍張玄遇、左鷹揚衛將軍曹仁師、司農少卿麻仁節率兵討之。與萬斬戰於西硤石谷,官軍敗績,玄遇、仁節並為賊所虜。又令夏官尚書王孝傑、左羽林將軍蘇宏暉領兵七萬以繼之。與萬斬戰於東硤石谷,孝傑在陣陷沒。宏暉棄甲而遁。萬斬乘勝度其眾入幽州,殺略人吏。清邊道大總管、建安郡王武
 攸宜遣裨將討之,不能克。又詔左金吾大將軍、河內王武懿宗為大總管,御史大夫婁師德為副大總管,右武衛將軍沙吒忠義為前軍總管,率兵三十萬以討之。俄而李盡滅死,萬斬代領其眾。萬斬又遣別帥駱務整、何阿小為游軍前鋒,攻陷冀州,殺刺史陸寶積,屠官吏子女數千人。俄而奚及突厥之眾掩擊其後,掠其幼弱。萬斬棄其眾,以輕騎數千人東走。前軍副總管張九節率數百騎設伏以邀之。萬斬窮蹙,乃將其家奴輕騎宵遁,
 至潞河東,解鞍憩於林下,其奴斬之。張九節傳其首於東都,自是其餘眾遂降突厥。



 開元三年,其首領李失活以默啜政衰,率種落內附。失活,即盡忠之從父弟也。於是復置松漠都督府。封失活為松漠郡王,拜左金吾衛大將軍兼松漠都督。其所統八部落,各因舊帥拜為刺史,又以將軍薛泰督軍以鎮撫之。明年,失活入朝,封宗室外甥女楊氏為永樂公主以妻之。



 六年,失活死,上為之舉哀,贈特進。失活從父弟娑固代統其眾,遣使冊立,
 仍令襲其兄官爵。娑固大臣可突於驍勇,頗得眾心,娑固謀欲除之。可突於反攻娑固,娑固奔營州。都督許欽淡令薛泰帥驍勇五百人,又徵奚王李大輔者及娑固合眾以討可突於。官軍不利,娑固、大輔臨陣皆為可突於所殺,生拘薛泰。營府震恐,許欽澹移軍西入渝關。可突於立娑固從父弟鬱於為主。俄又遣使請罪,上乃令冊立鬱於,令襲娑固官爵,仍赦可突於之罪。



 十年,鬱於入朝請婚。上又封從妹夫率更令慕容嘉賓女為燕郡
 公主以妻之,仍封鬱於為松漠郡王,授左金吾衛員外大將軍、兼靜析軍經略大使,賜物千段。鬱於還蕃,可突於來朝,拜左羽林將軍,從幸並州。



 明年,鬱於病死,弟吐於代統其眾,襲兄官爵,復以燕郡公主為妻。吐於與可突於復相猜阻。



 十三年,攜公主來奔,便不敢還,改封遼陽郡王,因留宿衛。可突於立李盡忠弟邵固為主。其冬,車駕東巡,邵固詣行在所,因從至嶽下,拜左羽林軍員外大將軍、靜析軍經略大使,改封廣化郡王,又封皇從
 外甥女陳氏為東華公主以妻之。



 邵固還蕃,又遣可突於入朝,貢方物,中書侍郎李元紘不禮焉,可突於怏怏而去。左丞相張說謂人曰:「兩蕃必叛。可突於人面獸心,唯利是視,執其國政,人心附之,若不優禮縻之,必不來矣!」十八年,可突於殺邵固,率部落並脅奚眾降於突厥,東華公主走投平盧軍。於是詔中書舍人襲寬、給事中薛侃等於京城及關內、河東、河南、河北分道募壯勇之士,以忠王浚為河北道行軍元帥以討之,師竟不行。



 二
 十年,詔禮部尚書信安王禕為行軍副大總管,領眾與幽州長史趙含章出塞擊破之,俘獲甚眾。可突於率其麾下遠遁,奚眾盡降,禕乃班師。明年,可突於又來抄掠。幽州長史薛楚玉遣副將郭英傑、吳克勤、鄔知義、羅守忠率精騎萬人,並領降奚之眾追擊之。軍至渝關都山之下,可突於領突厥兵以拒官軍。奚眾遂持兩端,散走保險。官軍大敗,知義、守忠率麾下遁歸,英傑、克勤沒於陣,其下六千餘人,盡為賊所殺。詔以張守珪為幽州長
 史兼御史中丞以經略之。可突於漸為守珪所逼,遣使偽降。俄又回惑不定,引眾漸向西北,將就突厥。守珪遣管記王悔等就部落招諭之。時契丹衙官李過折與可突於分掌兵馬,情不葉,悔潛誘之,過折夜勒兵斬可突於及其支黨數十人。



 二十三年正月,傳首東都。詔封過折為北平郡王,授特進,檢校松漠州都督,賜錦衣一副、銀器十事、絹彩三千疋。其年,過折為可突於餘黨泥禮所殺,並其諸子,唯一子刺乾走投安東得免,拜左驍衛
 將軍。



 天寶十年,安祿山誣其酋長欲叛,請舉兵討之。八月,以幽州、雲中、平盧之眾數萬人,就潢水南契丹衙與之戰,祿山大敗而還,死者數千人。至十二年,又降附。迄於貞元,常間歲來修籓禮。



 貞元四年,與奚眾同寇我振武,大掠人畜而去。九年、十年,復遣使來朝,大首領悔落拽何已下,各授官放還。十一年,大首領熱蘇等二十五人來朝。自後至元和、長慶、寶歷、太和、開成時遣使來朝貢。會昌二年九月,制:「契丹新立王屈戍,可雲麾將軍,守
 右武衛將軍員外置同正員。」幽州節度使張仲武上言:「屈戍等云,契丹舊用回紇印,今懇請聞奏,乞國家賜印。」許之,以「奉國契丹之印」為文。



 奚國,蓋匈奴之別種也,所居亦鮮卑故地,即東胡之界也,在京師東北四千餘里。東接契丹,西至突厥,南拒白狼河,北至霫國。自營州西北饒樂水以至其國。勝兵三萬餘人,分為五部,每部置俟斤一人。風俗並於突厥。每隨逐水草,以畜牧為業,遷徙無常。居有氈帳,兼用車為
 營,牙中常五百人持兵自衛。此外部落皆散居山谷,無賦稅。其人善射獵,好與契丹戰爭。



 武德中,遣使朝貢。貞觀二十二年,酋長可度者率其所部內屬,乃置饒樂都督府,以可度者為右領軍兼饒樂都督,封樓煩縣公,賜姓李氏。顯慶初,又授右監門大將軍。萬歲通天年,契丹叛後,奚眾管屬突厥,兩國常遞為表裏,號曰「兩蕃」。景雲元年,其首領李大輔遣使貢方物,睿宗嘉之,宴賜甚厚。



 延和元年,左羽林將軍、檢校幽州大都督孫儉,率兵十
 二萬以襲其部落,師次冷硎,前軍左驍衛將軍李楷洛等與大輔會戰,我師敗績。儉懼,不敢進救,遣使矯報大輔云:「我奉敕來此招諭蕃將,李楷洛等不受節度而輒用兵,請斬以謝。」大輔曰:「若奉敕招諭,有何國信物?」儉率軍中繒帛萬餘段並袍帶以與之。大輔曰:「將軍可南還,無相驚擾。」儉軍漸失部伍,大輔乃率眾逼之,由是大敗,兵士死傷者數萬。儉及副將周以悌為大輔所擒,送於突厥默啜,並遇害。



 開元三年,大輔遣其大臣粵蘇梅落
 來請降,詔復立其地為饒樂州,封大輔為饒樂郡王,仍拜左金吾員外大將軍、饒樂州都督。五年,大輔與契丹首領松漠郡王李失活咸請於柳城依舊置營州都督府,上從之。敕太子詹事姜師度充使督工作,役八千餘人。其年,大輔入朝,詔封從外甥女辛氏為固安公主以妻之,賜物一千五百疋,遣右領軍將軍李濟持節送還蕃。



 八年,大輔率兵救契丹,戰死,其弟魯蘇嗣立。



 十年,入朝,詔令襲其兄饒樂郡王、右金吾員外大將軍、兼保塞
 軍經略大使,賜物一千段,仍以固安公主為妻。而公主與嫡母未和,遞相論告,詔令離婚,復以成安公主之女韋氏為東光公主以妻之。



 十四年,又改封魯蘇為奉誠王,授右羽林軍員外將軍。



 十八年,奚眾為契丹衙官可突於所脅,復叛降突厥。魯蘇不能制,走投渝關,東光公主奔歸平盧軍。其秋,幽州長史趙含章發清夷軍兵擊奚。破之,斬首二百級。自是奚眾稍稍歸降。



 二十年,信安王禕奉詔討叛奚。奚酋長李詩、瑣高等以其部落五千
 帳來降。詔封李詩為歸義王、兼特進、左羽林軍大將軍同正。仍充歸義州都督,賜物十萬段,移其部落於幽州界安置。天寶五載,又封其王娑固為昭信王,仍授饒樂都督。



 自大歷後,朝貢時至。貞元四年七月,奚及室韋寇振武。十一年四月,幽州奏卻奚六萬餘眾。元和元年,其王饒樂府都督、襲歸誠王梅落來朝,加檢校司空,放還蕃。三年,以奚首領索低為右武威衛將軍同正,充檀、蘇兩州游奕兵馬使,仍賜姓李氏。八年,遣使來朝。



 十一年,
 遣使獻名馬。爾後每歲朝貢不絕,或歲中二三至。故事,常以範陽節度使為押奚、契丹兩蕃使。自至德之後,籓臣多擅封壤,朝廷優容之,彼務自完,不生邊事,故二蕃亦少為寇。其每歲朝賀,常各遣數百人至幽州,則選其酋渠三五十人赴闕,引見於麟德殿,錫以金帛遣還,餘皆駐而館之,率為常也。



 室韋者,契丹之別類也。居越河北,其國在京師東北七千里。東至黑水靺鞨,西至突厥,南接契丹,北至於海。
 其國無君長,有大首領十七人,並號「莫賀弗」,世管攝之,而附於突厥。兵器有角弓楛矢,尤善射,時聚弋獵,事畢而散。其人土著,無賦斂。或為小室,以皮覆上,相聚而居,至數十百家。剡木為犁,不加金刃,人牽以種,不解用牛。夏多霧雨,冬多霜霰。畜宜犬豕,豢養而敢之,其皮用以為韋,男子女人通以為服。被發左衽,其家富者項著五色雜珠。婚嫁之法,男先就女舍,三年役力,因得親迎其婦。役日已滿,女家分其財物,夫婦同車而載,鼓舞共歸。



 武德中,獻方物。貞觀三年,遣使貢豐貂,自此朝貢不絕。



 又云:室韋,我唐有九部焉。所謂嶺西室韋、山北室韋、黃頭室韋、大如者室韋、小如者室韋、婆萵室韋、訥北室韋、駱駝室韋,並在柳城郡之東北,近者三千五百里,遠者六千二百里。今室韋最西與回紇接界者,烏素固部落,當俱輪泊之西南。次東有移塞沒部落。次東又有塞曷支部落,此部落有良馬,人戶亦多,居啜河之南,其河彼俗謂之燕支河。次又有和解部落,次東又有烏羅護部
 落,又有那禮部落。又東北有山北室韋,又北有小如者室韋,又北有婆萵室韋,東又有嶺西室韋,又東南至黃頭室韋,此部落兵強,人戶亦多,東北與達姤接。嶺西室韋北又有訥北支室韋,此部落較小。烏羅護之東北二百餘里,那河之北有古烏丸之遺人,今亦自稱烏丸國。武德、貞觀中,亦遣使來朝貢。其北大山之北有大室韋部落,其部落傍望建河居。其河源出突厥東北界俱輪泊,屈曲東流,經西室韋界,又東經大室韋界,又東經蒙
 兀室韋之北,落俎室韋之南,又東流與那河、忽汗河合,又東經南黑水靺鞨之北,北黑水靺鞨之南,東流注于海。烏丸東南三百里,又有東室韋部落,在越河之北。其河東南流,與那河合。開元、天寶間,比年或間歲入貢。大歷中,亦頻遣使來貢。貞元八年閏十二月,室韋都督和解熱素等一十人來朝。太和五年至八年,凡三遣使來。九年十二月,室韋大都督阿成等三十人來朝。開成、會昌中,亦遣使來朝貢不絕。



 靺鞨,蓋肅慎之地,後魏謂之勿吉,在京師東北六千餘里。東至於海,西接突厥,南界高麗,北鄰室韋。其國凡為數十部,各有酋帥,或附於高麗,或臣於突厥。而黑水靺鞨最處北方,尤稱勁捷,每恃其勇,恆為鄰境之患。俗皆編發,性兇悍,無憂戚,貴壯而賤老。無屋宇,並依山水掘地為穴,架木於上,以土覆之,狀如中國之塚墓,相聚而居。夏則出隨水草,冬則入處穴中。父子相承,世為君長。俗無文字。兵器有角弓及楛矢。其畜宜豬,富人至數百
 口,食其肉而衣其皮。死者穿地埋之,以身襯土,無棺斂之具,殺所乘馬於尸前設祭。



 有酋帥突地稽者,隋末率其部千餘家內屬,處之於營州,煬帝授突地稽金紫光祿大夫、遼西太守。武德初,遣間使朝貢,以其部落置燕州,仍以突地稽為總管。劉黑闥之叛也,突地稽率所部赴定州,遣使詣太宗請受節度,以戰功封蓍國公。又徙其部落於幽州之昌平城。會高開道引突厥來攻幽州,突地稽率兵邀擊,大破之。



 貞觀初,拜右衛將軍,賜姓李
 氏。尋卒。子謹行,偉貌,武力絕人。麟德中,歷遷營州都督。其部落家僮數千人,以財力雄邊,為夷人所憚。累拜右領軍大將軍,為積石道經略大使。吐蕃論欽陵等率眾十萬人入寇湟中,謹行兵士樵採,素不設備,忽聞賊至,遂建旗伐鼓,開門以待之。吐蕃疑有伏兵,竟不敢進。



 上元三年,又破吐蕃數萬眾於青海,降璽書勞勉之。累授鎮軍大將軍,行右衛大將軍,封燕國公。永淳元年卒,贈幽州都督,陪葬乾陵。自後或有酋長自來,或遣使來朝
 貢,每歲不絕。



 其白山部,素附於高麗,因收平壤之後,部眾多入中國。汨咄、安居、骨室等部,亦因高麗破後奔散微弱,後無聞焉。縱有遺人,並為渤海編戶。唯黑水部全盛,分為十六部,部又以南北為稱。



 開元十三年,安東都護薛泰請於黑水靺鞨內置黑水軍。續更以最大部落為黑水府,仍以其首領為都督,諸部刺史隸屬焉。中國置長史,就其部落監領之。十六年,其都督賜姓李氏,名獻誠,授雲麾將軍兼黑水經略使,仍以幽州都督為其
 押使,自此朝貢不絕。



 渤海靺鞨大祚榮者,本高麗別種也。高麗既滅,祚榮率家屬徙居營州。萬歲通天年,契丹李盡忠反叛,祚榮與靺鞨乞四比羽各領亡命東奔,保阻以自固。盡忠既死,則天命右玉鈐衛大將軍李楷固率兵討其餘黨,先破斬乞四比羽,又度天門嶺以迫祚榮。祚榮合高麗、靺鞨之眾以拒楷固;王師大敗,楷固脫身而還。屬契丹及奚盡降突厥,道路阻絕,則天不能討,祚榮遂率其眾東保
 桂婁之故地,據東牟山,築城以居之。



 祚榮驍勇善用兵,靺鞨之眾及高麗餘燼,稍稍歸之。聖歷中,自立為振國王,遣使通於突厥。其地在營州之東二千里,南與新羅相接。越熹靺鞨東北至黑水靺鞨,地方二千里,編戶十餘萬,勝兵數萬人。風俗瑟高麗及契丹同,頗有文字及書記。



 中宗即位,遣侍御史張行岌往招慰之。祚榮遣子入侍,將加冊立,會契丹與突厥連歲寇邊,使命不達。睿宗先天二年,遣郎將崔往冊拜祚榮為左驍衛員外
 大將軍、渤海郡王,仍以其所統為忽汗州,加授忽汗州都督,自是每歲遣使朝貢。



 開元七年,祚榮死,玄宗遣使吊祭。乃冊立其嫡子桂婁郡王大武藝襲父為左驍衛大將軍、渤海郡王、忽汗州都督。



 十四年,黑水靺鞨遣使來朝,詔以其地為黑水州,仍置長史,遣使鎮押。武藝謂其屬曰:「黑水途經我境,始與唐家相通。舊請突厥吐屯,皆先告我同去。今不計會,即請漢官,必是與唐家通謀,腹背攻我也。」遣母弟大門藝及其舅任雅發兵以擊
 黑水。門藝曾充質子至京師,開元初還國,至是謂武藝曰:「黑水請唐家官史,即欲擊之,是背唐也。唐國人眾兵強,萬倍於我,一朝結怨,但自取滅亡。昔高麗全盛之時,強兵三十餘萬,抗敵唐家,不事賓伏,唐兵一臨,掃地俱盡。今日渤海之眾,數倍少於高麗,乃欲違背唐家,事必不可。」



 武藝不從。門藝兵至境,又上書固諫。武藝怒,遣從兄大壹夏代門藝統兵,徵門藝,欲殺之。門藝遂棄其眾,間道來奔,詔授左驍衛將軍。武藝尋遣使朝貢,仍上表
 極言門藝罪狀,請殺之。上密遣門藝往安西,仍報武藝云:「門藝遠來歸投,義不可殺。今流向嶺南,已遣去訖。」乃留其使馬文軌、蔥勿雅,別遣使報之。俄有洩其事者,武藝又上書云:「大國示人以信,豈有欺誑之理!今聞門藝不向嶺南,伏請依前殺卻。」由是鴻臚少卿李道邃、源復以不能督察官屬,致有漏洩,左遷道邃為曹州刺史,復為澤州刺史。遣門藝暫向嶺南以報之。



 二十年,武藝遣其將張文休率海賊攻登州刺史韋俊。詔遣門藝往幽
 州徵兵以討之,仍令太僕員外卿金思蘭往新羅發兵以攻其南境。屬山阻寒凍,雪深丈餘,兵士死者過半,竟無功而還。武藝懷怨不已,密遣使至東都,假刺客刺門藝於天津橋南,門藝格之,不死。詔河南府捕獲其賊,盡殺之。



 二十五年,武藝病卒,其子欽茂嗣立。詔遣內侍段守簡往冊欽茂為渤海郡王,仍嗣其父為左驍衛大將軍、忽汗州都督。欽茂承詔赦其境內,遣使隨守簡入朝貢獻。



 大歷二年至十年,或頻遣使來朝,或間歲而至,或歲內二三至者。十二年正月,遣使獻日本國舞女一十
 一人及方物。四月、十二月,使復來。建中三年五月、貞元七年正月,皆遣使來朝,授其使大常靖為衛尉卿同正,令還蕃。八月,其王子大貞翰來朝,請備宿衛。十年正月,以來朝王子大清允為右衛將軍同正,其下三十餘人,拜官有差。



 十一年二月,遣內常侍殷志贍冊大嵩璘為渤海郡王。十四年,加銀青光祿大夫、檢校司空,進封渤海國王。



 嵩璘父欽茂,開元中,襲父位為郡王、左金吾大將軍。天寶中,累加特進、太子詹事、賓客。寶應元年,進封
 國王。大歷中,累加拜司空、太尉。及嵩璘襲位,但授其郡王、將軍而已。嵩璘遣使敘理,故再加冊命。十一月,以王侄大能信為左驍衛中郎將、虞候、婁蕃長,都督茹富仇為右武衛將軍,放還。



 二十一年,遣使來朝。順宗加嵩璘金紫光祿大夫、檢校司空。元和元年十月,加檢校太尉。十二月,遣使朝貢。



 四年,以嵩璘男元瑜為銀青光祿大夫、檢校秘書監、忽汗州都督,依前渤海國王。五年,遣使朝貢者二。七年,亦遣使來朝。八年正月,授元瑜弟權知
 國務言義銀青光祿大夫、檢校秘書監、都督、渤海國王,遣內侍李重旻使焉。



 十三年,遣使來朝,且告哀。五月,以知國務大仁秀為銀青光祿大夫、檢校秘書監、都督、渤海國王。十五年閏正月,遣使來朝,加大仁秀金紫光祿大夫、檢校司空。十二月,復遣使來朝貢。長慶二年正月,又遣使來。四年二月,大睿等五人來朝,請備宿衛。寶歷中,比歲修貢。太和元年、四年,皆遣使來朝。



 五年,大仁秀卒,以權知國務大彞震為銀青光祿大夫、檢校秘書監、
 都督、渤海國王。六年,遣王子大明俊等來朝。七年正月,遣同中書右平章事高寶英來謝冊命,仍遣學生三人,隨寶英請赴上都學問。先遣學生三人,事業稍成,請歸本國,許之。二月,王子大先晟等六人來朝。開成後,亦修職貢不絕。



 霫,匈奴之別種也,居於潢水北,亦鮮卑之故地,其國在京師東北五千里。東接靺鞨,西至突厥,南至契丹,北與烏羅渾接。地周二千里,四面有山,環繞其境。人多善射
 獵,好以赤皮為衣緣,婦人貴銅釧,衣襟上下懸小銅鈴,風俗略與契丹同。有都倫紇斤部落四萬戶,勝兵萬餘人。貞觀三年,其君長遣使貢方物。



 烏羅渾國,蓋後魏之烏洛侯也,今亦謂之烏羅護,其國在京師東北六千三百里,東與靺鞨,西與突厥,南與契丹,北與烏丸接。風俗與靺鞨同。貞觀六年,其君長遣使獻貂皮焉。



 史臣曰:北狄密邇中華,侵邊蓋有之矣;東夷隔礙瀛海,
 作梗罕常聞之。非惟勢使之然,抑亦稟於天性。太平之人仁,空峒之人武,信矣。隨煬帝縱欲無厭,興兵遼左,急斂暴欲,由是而起。亂臣賊子,得以為資,不戢自焚,遂亡其國。我太宗文皇帝親馭戎輅,東征高麗,雖有成功,所損亦甚。及凱還之日,顧謂左右曰:「使朕有魏徵在,必無此行矣!」則是悔於出師也可知矣。何者?夷狄之國,猶石田也,得之無益,失之何傷?必務求虛名,以勞有用。但當修文德以來之,被聲教以服之,擇信臣以撫之,謹邊備
 以防之,使重譯來庭,航海入貢,茲庶得其道也!



 贊曰:東夷之人,北狄之俗。爰考《周官》,是稱蠻服。未得無傷,已得何足!宜務懷柔,謂之羈束。



\end{pinyinscope}