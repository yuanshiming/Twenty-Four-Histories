\article{卷九 本紀第九 玄宗下}

\begin{pinyinscope}

 開元二十五年春正月壬午,制:「朕猥集休運,多謝哲王,然而哀矜之情,小大必慎。自臨寰宇,子育黎烝,未嘗行極刑,起大獄。上玄降鑒,應以祥和,思協平邦之典,致之仁壽之域。自今有犯死刑,除十惡罪,宜令中書門下與法官詳所犯
 輕重,具狀奏聞。崇德尚齒,三代丕義;敦風勸俗,五教攸先。其曾任五品已上清資官以禮去職者,所司具錄名奏,老疾不堪厘務者與致仕。道士、女冠宜隸宗正寺,僧尼令祠部檢校。百司每旬節休假,並不須入曹司,任游勝為樂。宣示中外,知朕意焉。」癸卯,道士尹愔為諫議大夫、集賢學士兼知史館事。二月,新羅王金興光卒,其子承慶嗣位,遣贊善大夫邢璹攝鴻臚少卿,
 往吊祭,冊立之。壬子,加宗正丞一員。戊午,罷江淮運,停河北運。癸酉,張守珪破契丹餘眾於祿山,殺獲甚眾。



 三月乙卯,河西節度使崔希逸自涼州南率眾入吐蕃界二千餘里。己亥,希逸至青海西郎佐素文子觜,與賊相遇,大破之,斬首二千餘級。



 夏四月庚戌,陳、許、豫、壽四州開稻田。辛酉,監察御史周子諒上書忤旨,Ξ之殿庭,朝堂決杖死之。甲子,尚書右丞相張九齡以曾薦引子諒,左授荊州長史。乙丑,皇太子瑛、鄂王瑤、光王琚並廢
 為庶人。太子妃兄駙馬都尉薛鏽長流瀼州,至藍田驛賜死。



 六月壬戌,熒惑犯房,至心星越度而過。秋七月己卯,大理少卿徐岵奏:「天下今歲斷死刑五十八,幾致刑措,鳥巢寺之獄。」上特推功元輔,庚辰,封李林甫為晉國公,牛仙客為豳國公。己卯,敕諸陵廟並隸宗正寺,其宗正寺官員,自今並以宗枝為之。九月壬申,頒新定《令》、《式》、《格》及《事類》一百三十卷於天下。冬十月,制自今年每年立春日迎春於東郊,其夏及秋冬如常。以十二月朔日
 於正殿受朝,讀時令。十一月壬申,幸溫泉宮。丁丑,開府儀同三司、廣平郡公宋璟薨。十二月丙午,惠妃武氏薨,追謚為貞順皇后,葬於敬陵。吐蕃使其大臣屬盧論莽藏來朝貢。



 二十六年春正月乙亥,工部尚書牛仙客為侍中。丁丑,親迎氣於東郊,祀青帝。制天下系囚,死罪流嶺南,餘並放免。鎮兵部還。京兆府新開稻田,並散給貧人。百官賜勛絹。長安、萬年兩縣各與本錢一千貫,收利供馹,仍付
 雜馹。天下州縣,每鄉一學,仍擇師資,令其教授。諸鄉貢每年令就國子監謁先師,明經加口試。內外八品已下及草澤有博學文辭之士,各委本司本州聞薦。



 二月辛卯,以李林甫遙領隴右節度使。甲辰,禁大寒食以雞卵相饋送。庚申,葬貞順皇后於敬陵。乙卯,以牛仙客遙領河東道節度使。辛酉,廢仙州,分其屬縣隸許、汝等州。三月己巳朔,減秘書省校書、正字官員。丙子,有星孛於紫微垣中,歷斗魁十餘日,因陰雲不見。己酉,河南、洛陽兩
 縣亦借本錢一千貫,收利充人吏課役。癸未,京兆地震。吐蕃寇河西,左散騎常侍崔希逸擊破之;鄯州都督杜希望又攻拔新羅城,制以其城為威戎軍。夏四月己亥朔,始令太常卿韋絳讀時令於宣政殿,百僚於殿上列坐而聽之。五月乙酉,以李林甫遙領河西節度使,兼判梁州事。庚寅,幸咸宜公主宅。六月庚子,立忠王璵為皇太子。秋七月己巳,冊皇太子,大赦天下,常赦所不免者咸赦除之。內外文武官及五品已上為父後者各賜勛
 一轉。忠王府官及侍講加一階。賜酺三日。庚辰,分越州置明州。九月丙申朔,日有蝕之。庚子,於舊六胡州地置宥州。益州長史王昱率兵攻吐蕃安戎城,為賊所據,官軍大敗,昱棄甲而遁,兵士死者數千人。



 冬十月戊寅,幸溫泉宮。是歲渤海靺鞨王大武藝死,其子欽茂嗣立,遣使吊祭,冊立之。其冬,兩京建行宮,造殿宇各千餘間。潤州刺史齊浣開伊婁河於揚州南瓜洲浦。析左右羽林軍置左右龍武軍,以左右萬騎營隸焉。



 二十七年春正月乙巳,大雨雪。二月己巳,加尊號開元聖文神武皇帝,大赦天下,常赦所不免者咸赦除之,開元已來諸色痕瘕人咸從洗滌,左降官量移近處。百姓免今年租稅。三品已上賜爵一級,四品已上加一階。宗廟薦饗,自今已後並用宗子。賜酺五日。



 夏四月丁丑,廢洮州隸蘭州,改臨州為洮州。乙酉,太子少傅竇曳為開府儀同三司,吏部尚書李暠為太子少傅。丁酉,侍中牛仙客為兵部尚書兼侍中;兵部尚書兼中書令李林甫
 為吏部尚書,依舊兼中書令。以東宮內侍隸內侍省為署。五月癸卯,置龍武軍官員。先是,鄎國公主之子薛諗與其黨李談、崔洽、石如山同於京城殺人,或利其財,或違其志,即白日椎殺,煮而食之。其夏事發,皆決殺於京兆府門,諗以國親流瀼州,賜死於城東驛。



 六月甲戌,內常侍牛仙童坐贓,決殺之。幽州節度使、兼御史大夫張守珪以賄貶為括州刺史。太子太師、徐國公蕭嵩以嘗賂仙童,左授青州刺史。秋七月辛丑,熒惑犯南斗。北庭
 都護蓋嘉運以輕騎襲破突騎施於碎葉城,殺蘇祿,威震西陲。八月,吐蕃寇白草、安人等。甲申,制追贈孔宣父為文宣王,顏回為兗國公,餘十哲皆為侯,夾坐。後嗣褒聖侯改封為文宣公。九月,皇太子改名紹。汴州刺史齊浣請開汴河下流,自虹縣至淮陰北合於淮,逾時而功畢。因棄沙壅舊路,行者弊之,尋而新河之水勢淙急,遂填塞矣。前刑部尚書致仕崔隱甫卒。冬十月,將改作明堂。偽言官取小兒埋於明堂之下,以為厭勝。村野
 童兒藏於山谷,都城騷然,咸言兵至。上惡之,遣主客郎中王佶往東都及諸州宣慰百姓,久之定。冬十月,毀東都明堂之上層,改拆下層為乾元殿。戊戌,幸溫泉宮。辛丑,至自溫泉宮。十二月,東都副留守、太子賓客崔沔卒。以益州司馬章仇兼瓊權劍南節度等使。是歲,蓋嘉運大破突騎施之眾,擒其王吐火仙,送於京師。二十八年春正月,兩京路及城中苑內種果樹。癸巳,幸溫泉宮。庚子,至自溫泉宮。壬寅,以望日御勤政樓宴群
 臣,連夜燒燈,會大雪而罷,因命自今常以二月望日夜為之。三月丁亥朔,日有蝕之。壬子,權判益州長史章仇兼瓊拔吐蕃安戎城,分兵鎮守之。夏五月乙未,太子少師韓休、太子少傅李暠卒。六月,懷州刺史、信安王禕為太子少師。庚寅,太子賓客李尚隱卒。秋七月壬寅,追尊宣皇帝陵名曰建初,光皇帝陵名曰啟運,仍置官員。九月,魏州刺史盧暉開通濟渠,自石灰窠引流至州城而西,卻注魏橋。九月庚寅,封皇孫俶等十九人為郡王。冬
 十月甲子,幸溫泉宮。辛巳,至自溫泉宮。乙酉夜,東都新殿後佛光寺災。吐蕃寇安戎城。十一月,牛仙客停遙兼朔方、河東節度使。十二月乙卯,突騎施酋長莫賀達干率眾內屬。己未,禮部尚書杜暹卒。是歲,金城公主薨,吐蕃遣使來告喪。其時頻歲豐稔,京師米斛不滿二百,天下乂安,雖行萬里不持兵刃。



 二十九年春正月丁丑,制兩京、諸州各置玄元皇帝廟並崇玄學,置生徒,令習《老子》、《莊子》、《列子》、《文子》,每年準
 明經例考試。內外官有伯叔兄弟子侄堪任刺史、縣令,所司親自保薦。禁九品已下清資官置客舍邸店車坊、士庶厚葬。三月,吐蕃、突厥各遣使來朝。丙午,風霾,日色無影。夏四月庚戌朔。丙辰,以太原裴伷先為工部尚書。韋虛心卒。親王已下及內外官各賜錢令宴樂。壬午,以左右金吾大將軍裴寬為太原尹、北都留守。秋七月乙卯,洛水泛漲,毀天津橋及上陽宮仗舍。洛、渭之間,廬舍壞,溺死者千餘人。突厥登利可汗死。北州刺史王斛斯
 為幽州節度使;幽州節度副使安祿山為營州刺史,充平廬軍節度副使,押兩番、渤海、黑水四府經略使。



 九月,大雨雪,稻禾偃折,又霖雨月餘,道途阻滯。是秋,河北博、洺等二十四州言雨水害稼,命御史中丞張倚往東都及河北賑恤之。壬申,御興慶門,試明《四子》人姚子產、元載等。冬十月丙申,幸溫泉宮。戊戌,分遣大理卿崔翹等八人往諸道黜陟官吏。十一月庚戌,司空、邠王守禮薨。辛酉,至自溫泉宮。己巳,雨木冰,凝寒凍冽,數日不解。辛
 未,太尉、寧王憲薨,謚為讓皇帝,葬於惠陵。十二月丁酉,吐蕃入寇,陷廓州達化縣及振武軍石堡城,節度使蓋嘉運不能守。女國王趙曳夫及佛逝國王、日南國王遣其子來朝獻。



 天寶元年春正月丁未朔,大赦天下,改元,常赦不原咸赦除之。百姓所欠負租稅及諸色並免之。前資官及白身人有儒學博通、文辭秀逸及軍謀武藝者,所在具以名薦。京文武官才堪為刺史者各令封狀自舉。改黃鉞
 為金鉞。內外官各賜勛兩轉。甲寅,陳王府參軍田同秀上言:「玄元皇帝降見於丹鳳門之通衢,告賜靈符在尹喜之故宅。」上遣使就函谷故關尹喜臺西發得之,乃置玄元廟於大寧坊。陜郡太守李齊物先鑿三門,辛未,渠成放流。



 二月丁亥,上加尊號為開元天寶聖文神武皇帝。辛卯,親享玄元皇帝於新廟。甲午,親享太廟。丙申,合祭天地於南郊。制天下囚徒,罪無輕重並釋放。流人移近處,左降官依資敘用,身死貶處者量加追贈。枉法贓
 十五疋當絞,今加至二十疋。莊子號為南華真人,文子號為通玄真人,列子號為沖虛真人,庚桑子號為洞虛真人。其四子所著書改為真經。崇玄學置博士、助教各一員,學生一百人。桃林縣改為靈寶縣。改侍中為左相,中書令為右相,左右丞相依舊為僕射,又黃門侍郎為門下侍郎。東都為東京,北都為北京,天下諸州改為郡,刺史改為太守。陜州河北縣為平陸縣。老幼版授,文武官三品已上加一爵,四品已下加一階。庚子,平盧節度使安祿山
 進階驃騎大將軍。夏六月庚寅,武功山水暴漲,壞人廬舍,溺死數百人。秋七月癸卯朔,日有蝕之。辛未,左相、豳國公牛仙客卒。



 八月丁丑,刑部尚書、兼御史大夫李適之為左相。丁亥,突厥阿布思及默啜可汗之孫、登利可汗之女相與率其黨屬來降。壬辰,吏部尚書兼右相李林甫加尚書左僕射,左相李適之兼兵部尚書,左僕射裴耀卿為尚書右僕射。九月辛卯,上御花萼樓,出宮女宴毗伽可汗妻可登及男女等,賞賜不可勝紀。丙寅,改
 天下縣名不穩及重名一百一十處。兩京玄元廟改為太上玄元皇帝宮,天下準此。冬十月丁酉,幸溫泉宮。辛丑,改驪山為會昌山,仍於秦坑儒之所立祠宇,以祀遭難諸儒。新成長生殿名曰集靈臺,以祀天神。十一月己巳,至自溫泉宮。是歲,命陜郡太守韋堅引滻水開廣運潭於望春亭之東,以通河、渭;京兆尹韓朝宗又分渭水入自金光門,置潭於西市之兩衙,以貯材木。是冬無冰。其年,天下郡府三百六十二,縣一千五百二十八,鄉一
 萬六千八百二十九。戶部進計帳,今年管戶八百五十二萬五千七百六十三,口四千八百九十萬九千八百。



 二年春正月丙辰,追尊玄元皇帝為大聖祖玄元皇帝,兩京崇玄學改為崇玄館,博士為學士。三月壬子,親祀玄元廟以冊尊號。制追尊聖祖玄元皇帝父周上御史大夫敬曰先天太上皇,母益壽氏號先天太后,仍於譙郡本鄉置廟。尊咎繇為德明皇帝。改西京玄元廟為太清宮,東京為太微宮,天下諸郡為紫極宮。韋堅開廣運
 潭畢功,盛陳舟艦。丙寅,上幸廣運樓以觀之,即日還宮。夏六月甲戌夜,雷震東京應天門觀災,延燒至左、右延福門,經日不滅。七月癸丑,致仕禮部尚書王丘卒。丙辰,尚書右僕射裴耀卿薨。九月,太子少保崔琳卒。辛酉,譙郡紫極宮改為太清宮。冬十月戊辰,太子太保、信安王禕卒。戊寅,幸溫泉宮。十一月乙卯,至自溫泉宮。十二月己亥,東京應天門改為乾元門。戊申,幸溫泉宮。丙辰,至自溫泉宮。十二月乙酉,太子賓客賀知章請度為道士
 還鄉。是冬無雪。



 三載正月丙辰朔,改年為載。赦見禁囚徒。庚子,遣左右相已下祖別賀知章於長樂坡,上賦詩贈之。壬寅,幸溫泉宮。二月己巳,還京。丁丑,封讓皇帝男琳為嗣寧王,故邠王守禮男承寧為嗣邠王,讓帝男璹為嗣申王,惠宣太子男珍為嗣岐王,員為嗣薛王。庚寅,皇太子紹改名亨。是月,河南尹裴敦復卒。閏月辛亥,有星如月,墜於東南,墜後有聲。京師訛言官遣棖捕人肝以祭天狗。人相
 恐,畿縣尤甚,發使安之。



 三月庚午,武威郡上言:番禾縣天寶山有醴泉湧出,嶺石化為瑞麰,遠近貧乏者取以給食。改番禾為天寶縣。癸酉,制天下見禁囚徒死罪降流,流已下並原之。夏四月,南海太守劉巨鱗擊破海賊吳令光,永嘉郡平。敕兩京、天下州郡取官物鑄金銅天尊及佛各一軀,送開元觀、開元寺。五月戊寅,長安令柳升坐贓,於朝堂決殺之。秋八月丙午,九姓拔悉密葉護攻殺突厥烏蘇米施可汗,傳首京師。庚申,內外文武官
 六品已下,自今已後,赴任之後,計載終滿二百日已上,許其成考。



 冬十月癸巳,幸溫泉宮。丁未,改史國為來威國。十一月癸卯,還京。癸丑,每載依舊取正月十四日、十五日、十六日開坊市門燃燈,永以為常式。玉真公主先為女道士,讓號及實封,賜名持盈。十二月甲午,分新豐縣置會昌縣。甲寅,親祀九宮貴神於東郊,禮畢,大赦天下。百姓十八已上為中男,二十三已上成丁。每歲庸調,八月起征,可延至九月。詔天下民間家藏《孝經》一本。



 四載春三月甲申,宴群臣於勤政樓。壬申,封外孫獨孤氏女為靜樂公主,出降契丹松漠都督李懷節;封外孫楊氏女為宜芳公主,出降奚饒樂都督李延寵。秋八月甲辰,冊太真妃楊氏為貴妃。是月,河南睢陽、淮陽、譙等八郡大水。九月,契丹及奚酋長各殺公主,舉部落叛。隴右節度使皇甫惟明與吐蕃戰於石堡城,官軍不利,副將褚直廉等死之。冬十月,於單于都護府置金河縣,安北都護府置陰山縣。丁酉,幸溫泉宮。壬子,以會昌縣為
 同京縣。十二月戊戌,還京。



 五載春正月癸酉,刑部尚書韋堅貶括蒼太守;隴右節度使皇甫惟明貶播川太守,尋決死於黔中。乙亥,敕大小縣令並準畿官吏三選聽集。《禮記月令》改為《時令》。封中岳為中天王,南岳為司天王,北岳為安天王。天下山水,名稱或同,義且不經,多因於里諺,宜令所司各據圖籍改定。丙子,遣禮部尚書席豫、左丞崔翹、御史中丞王鉷等七人分行天下,黜陟官吏。夏四月庚寅,左相、渭源
 伯李適之為太子少保,罷知政事。丁酉,門下侍郎陳希烈同中書門下平章事。五月庚申,敕今後每至旬節休假,中書門下文武百僚不須入朝,外官不須衙集。癸卯,停郡縣差丁白直課錢。



 六月,敕三伏內令宰相辰時還宅。秋七月丙子,韋堅為李林甫所構,配流臨封郡,賜死。堅妹皇太子妃聽離,堅外甥嗣薛王員貶夷陵郡別駕,女婿巴陵太守盧幼臨長流合浦郡。太子少保李適之貶宜春太守,到任,飲藥死。八月,以戶部侍郎郭虛己為
 御史大夫、劍南節度使。九月壬子,於太清宮刻石為李林甫、陳希烈像,侍於聖容之側。冬十月丁酉,幸溫泉宮。改臨淄郡為濟南郡。十一月己巳,還京。十二月辛未,贊善大夫杜有鄰、著作郎王曾、左驍衛兵曹柳勣等為李林甫所構,並下獄死。



 六載正月辛巳朔,北海太守李邕、淄川太守裴敦復並以事連王曾、柳勣,遣使就殺之。丁亥,親享太廟。戊子,親祀圜丘,禮畢,大赦天下,除絞、斬刑,但決重杖。於京城置
 三皇、五帝廟,以時享祭。其章懷、節愍、惠莊、惠文、惠宣等太子,宜與隱太子、懿德太子同為一廟。每日立仗食及設仗於庭,此後並宜停廢。五岳既已封王,四瀆當升公位,封河瀆為靈源公,濟瀆為清源公,江瀆為廣源公,淮瀆為長源公。三月戊戌,南海太守彭果坐贓,決杖,長流溱溪郡,死於路。



 夏四月戊午,門下侍郎陳希烈為左相兼兵部尚書。癸酉,復置軍器監。自五月不雨至秋七月。乙酉,以旱,命宰相、臺寺、府縣錄系囚,死罪決杖配流,徒
 已下特免。庚寅始雨。



 冬十月戊申,幸溫泉宮,改為華清宮。十一月乙亥,戶部侍郎楊慎矜及兄少府少監慎餘與弟洛陽令慎名,並為李林甫及御史中丞王鉷所構,下獄死。十二月丙辰,工部尚書陸景融卒。壬戌,還京。



 七載春正月己卯,禮部尚書席豫卒。己亥,韋絳奏御案褥袱帷等望去紫用赤黃,從之。三月乙酉,大同殿柱產玉芝,有神光照殿。群臣請加皇帝尊號曰開元天寶聖文神武應道,許之。夏四月辛丑,以高力士為驃騎大將
 軍。五月壬午,上御興慶宮,受冊徽號,大赦天下,百姓免來載租庸。三皇以前帝王,京城置廟,以時致祭。其歷代帝王肇跡之處未有祠守者,所在各置一廟。忠臣、義士、孝婦、烈女德行彌高者,亦置祠宇致祭。賜酺三日。六月,範陽節度使安祿山賜實封及鐵券。秋八月己亥朔,改千秋節為天長節。壬子,改萬年縣為咸寧縣。



 冬十月庚午,幸華清宮,封貴妃姊二人為韓國、虢國夫人。十二月戊戌,言玄元皇帝見於華清宮之朝元閣,乃改為降聖
 閣。改會昌縣為昭應縣,會昌山為昭應山;封山神為玄德公,仍立祠宇。辛酉,還京。



 八載春正月甲申,賜京官絹,備春時游賞。二月戊申,引百官於左藏庫縱觀錢幣,賜絹而歸。三月,朔方節度使張齊丘於中受降城北築橫塞城。夏四月,咸寧太守趙奉璋決杖而死,著作郎韋子春貶端溪尉,李林甫陷之也。幸華清宮觀風樓。五月辛巳,於開遠門外作振旅亭。戊子,南海太守劉巨鱗坐贓,決死之。六月,大同殿又產
 玉芝一莖。隴右節度使哥舒翰攻吐蕃石堡城,拔之。閏月己丑,改石堡城為神武軍。劍南索磨川新置都護府,宜以保寧為名。丙寅,上親謁太清宮,冊聖祖玄元皇帝尊號為聖祖大道玄元皇帝。高祖、太宗、高宗、中宗、睿宗五帝,皆加「大聖皇帝」之字;太穆、文德、則天、和思、昭皇后,皆加「順聖皇后」之字。群臣上皇帝尊號為開元天地大寶聖文神武應道皇帝。丁卯,上御含元殿受冊,大赦天下。自今後每至禘祫,並於太清宮聖祖前序昭穆。初,太白山
 人李渾言太白山金星洞有帝福壽玉版石記,求得之,乃封太白山為神應公,金星洞為嘉祥公,所管華陽縣為貞符縣。戊辰,太子太師、徐國公蕭嵩薨。丁亥,南衙立仗馬宜停,省進馬官。秋八月戊子,郡別駕宜停,下郡置長史。冬十月丙寅,幸華清宮。十一月丁巳,幸御史中丞楊釗莊。



 九載春正月庚寅朔,與歲次同始,受朝於華清宮。己亥,還京。庚戌,群臣請封西嶽,從之。二月壬午,御史中丞宋
 渾坐贓及奸,長流高要郡。三月庚戌,改匭使為獻納。辛亥,西嶽廟災。時久旱,制停封西嶽。夏五月庚寅,以旱,錄囚徒。乙卯,安祿山進封東平郡王。節度使封王,自此始也。秋七月己亥,國子監置廣文館,領生徒為進士業者。



 九月乙卯,處士崔昌上《五行應運歷》,以國家合承周、漢,請廢周、隋不合為二王後。冬十一月庚寅,幸華清宮。己丑,制自今告獻太清宮及太廟改為朝獻,巡陵為朝拜,告宗廟為奏,天地享祀文改昭告為昭薦,以告者臨下
 之義故也。辛卯,幸楊國忠亭子。辛丑,立周武王、漢高祖廟於京城,司置官吏。十二月乙亥,還京。



 十載春正月乙酉朔。壬辰,朝獻太清宮。癸巳,朝饗太廟。甲午,有事於南郊,合祭天地,禮畢,大赦天下。太廟置內官,供灑掃諸陵廟。己亥,改傳國寶為承天大寶。丁未,李林甫領安北副大都護、朔方節度使。庚戌,大風,陜郡運船失火,燒米船二百餘隻,人死者五百計。癸丑,分遣嗣吳王祇等十三人祭岳瀆海鎮。二月丁巳,安祿山兼雲
 中太守、河東節度使。夏四月,劍南節度使鮮於仲通將兵六萬討雲南,與雲南王閣羅鳳戰於瀘川,官軍大敗,死於瀘水者不可勝數。五月丁亥,改諸衛幡旗緋色者為赤黃,以符土運。秋八月乙卯,廣陵郡大風,潮水覆船數千艘。丙辰,京城武庫災,燒器械四十七萬事。是秋,霖雨積旬,墻屋多壞,西京尤甚。



 冬十月辛亥,幸華清宮。十一月乙未,幸楊國忠宅。丙午,兵部侍郎、兼御史中丞楊國忠兼領劍南節度使。



 十一載春正月辛亥,還京。二月癸酉,禁惡錢,官出好錢以易之。既而商旅不便,訴於國忠,乃止之。三月,朔方節度副使、奉信王阿布思與安祿山同討契丹,布思與祿山不協,乃率其部下叛歸漠北。丙午,制今後每月朔望,宜令薦食於太廟,每室一牙盤,仍五日一開室門灑掃。改吏部為文部,兵部為武部,刑部為憲部,其部內諸司有部字者並改,將作大匠、少匠為大、少二監。夏四月,御史大夫兼京兆尹王鉷賜死,坐弟銲與兇人邢縡謀逆
 故也。楊國忠兼京兆尹。五月戊申,慶王琮薨,贈靖德太子。六月戊子,東京大風,拔樹發屋。八月己丑,幸左藏庫,賜群臣帛有差。九月甲寅,改諸衛士為武士。冬十月戊寅,幸華清宮。



 十一月乙卯,尚書左僕射兼右相、晉國公李林甫薨於行在所。庚申,御史大夫兼蜀郡長史楊國忠為右相兼文部尚書。十二月甲戌,楊國忠奏請兩京選人銓日便定留放,無長名。己亥,還京。



 十二載春正月壬子,楊國忠於尚書省注官,注訖,於都
 堂對左相與諸司長官唱名。二月庚辰,選人鄭懟等二十餘人以國忠銓注無滯,設齋於勤政殿下,立碑於尚書省門。癸未,追削故右相李林甫在身官爵,男將作監岫、宗黨李復道等五十人皆流貶,國忠誣奏林甫陰結叛胡阿布思故也。夏五月乙酉,以魏、周、隋依舊為三恪及二王後,復封韓、介、酅等公。辛亥,太廟諸陵署依舊隸太常寺。



 七月壬子,天下齊人不得鄉貢,須補國子學生然後貢舉。八月,京城霖雨,米貴,令出太倉米十萬石,減
 價糶與貧人。仍令中書門下就京兆、大理疏決囚徒。



 九月己亥朔,隴右節度使、涼國公哥舒翰進封西平郡王,食實封五百戶。冬十月戊申,幸華清宮。和雇京城丁戶一萬三千人築興慶宮墻,起樓觀。至十二月,改橫塞城為天德軍。庚寅,行從官憲部尚書張筠等請上尊號為開元天地大寶聖文神武孝德證道皇帝。



 十三載春正月丁酉朔,上御華清宮之觀風樓,受朝賀。己亥,安慶緒獻俘於行在,帝引見於禁中,賞賜巨萬。乙
 巳,加安祿山尚書左僕射,賜實封千戶,奴婢十房,莊、宅各一區;又加閑廄、五坊、宮苑、隴右群牧都使,以武部侍郎吉溫為副。丙午,還京。二月癸酉,上親朝獻太清宮,上玄元皇帝尊號曰大聖祖高上大廣道金闕玄元天皇大帝。甲戌,親饗太廟,上高祖謚曰神堯大聖大光孝皇帝,太宗謚曰太宗文武大聖大孝皇帝,高宗謚曰高宗天皇大聖大弘孝皇帝,中宗謚曰中宗太和大聖大昭孝皇帝,睿宗謚曰睿宗玄真大聖大興孝皇帝。乙亥,御興慶
 殿受徽號,禮畢,大赦天下。左降官遭父母憂,放歸。獻陵等五署改為臺,令、丞各升一階。文武三品已上賜爵一級,四品已下加一階。賜酺三日。戊寅,右相兼文部尚書楊國忠守司空,餘如故。甲申,司空楊國忠受冊,天雨黃土,沾於朝服。祿山奏前後討契丹立功將士跳蕩等,請超三資,告身仍望好寫;於是超授將軍者五百餘人,中郎將者二千餘人。



 三月丁酉,太常卿張垍貶盧溪郡司馬,垍兄憲部尚書均貶建安太守。丙午,御躍龍殿門張樂
 宴群臣,賜右相絹一千五百疋,彩羅三百疋,彩綾五百疋;左相絹三百疋,彩羅綾各五十疋;餘三品八十疋,四品五品六十疋,六品七品四十疋,極歡而罷。壬戌,御勤政樓大酺。北庭都護程千里生擒阿布思獻於樓下,斬之於硃雀街。乙丑,左羽林上將軍封常清權北庭都護、伊西節度使。萬春公主出降楊朏。夏五月,熒惑守心五十餘日。六月乙丑朔,日有蝕之,不盡如鉤。侍御史、劍南留後李宓率兵擊雲南蠻於西洱河,糧盡軍旋,馬足陷
 橋,為閣羅鳳所擒,舉軍皆沒。廢濟陽郡,以所領五縣隸東平郡。秋八月丁亥,以久雨,左相、許國公陳希烈為太子太師,罷知政事;文部侍郎韋見素為武部尚書,同中書門下平章事。是秋,霖雨積六十餘日,京城垣屋頹壞殆盡,物價暴貴,人多乏食,令出太倉米一百萬石,開十場賤糶以濟貧民。東都瀍、洛暴漲,漂沒一十九坊。上御勤政樓試四科制舉人,策外加詩賦各一首。制舉加詩賦,自此始也。冬十月壬寅,幸華清宮。貶河東太守韋陟
 為桂嶺尉,武部侍郎吉溫為澧陽郡長史。乙巳,開府儀同三司、畢國公竇曳薨。戊午,還京。其載,戶部計今年見管州縣戶口:管郡總三百二十一,縣一千五百三十八,鄉一萬六千八百二十九;戶九百六十一萬九千二百五十四,三百八十八萬六千五百四不課,五百三十萬一千四十四課;口五千二百八十八萬四百八十八,四千五百二十一萬八千四百八十不課,七百六十六萬二千八百課。



 十四載春三月丙寅,宴群臣於勤政樓,奏《九部樂》,上賦詩斅柏梁體。癸未,遣給事中裴士淹等巡撫河南、河北、淮南等道。八月壬辰,上親錄囚徒。



 冬十月壬辰,幸華清宮。甲午,頒《御注老子》並《義疏》於天下。十一月戊午朔,始寧太守羅希奭以停止張博濟決杖而死,吉溫自縊於獄。丙寅,範陽節度使安祿山率蕃、漢之兵十餘萬,自幽州南向詣闕,以誅楊國忠為名,先殺太原尹楊光翽於博陵郡。壬申,聞於行在所。癸酉,以郭子儀為靈武太守、
 朔方節度使。封常清自安西入奏,至行在。甲戌,以常清為範陽、平盧節度使、兼御史大夫,令募兵三萬以御逆胡。戊寅,還京。以羽林大將軍王承業為太原尹,以衛尉卿張介然為陳留太守、河南節度採訪使,以金吾將軍程千里為潞州長史,並令討賊。甲申,以京兆牧、榮王琬為元帥,命高仙芝副之,於京城召募,號曰天武軍,其眾十萬。丙戌,高仙芝等進軍,上御勤政樓送之。十二月丙戌朔,祿山於靈昌郡渡河。辛卯,陷陳留郡,殺張介然。甲
 午,陷滎陽郡,殺太守崔無詖。丙申,封常清與賊戰於成皋罌子穀,官軍敗績,常清奔於陜郡。丁酉,祿山陷東京,殺留守李憕、中丞盧奕、判官蔣清。時高仙芝鎮陜郡,棄城西保潼關。常山太守顏杲卿與長史袁履謙、賈深等殺賊將李欽湊,執賊將何千年、高邈送京師。辛丑,詔皇太子統兵東討。以永王璘為山南節度使,以江陵長史源洧副之;潁王璬為劍南節度使,以蜀郡長史崔圓副之。二王不出閤。丙午,斬封常清、高仙芝於潼關,以哥舒
 翰為太子先鋒兵馬元帥,領河、隴兵募守潼關以拒之。辛亥,榮王琬薨,贈靖恭太子。



 十五載春正月乙卯,御宣政殿受朝。其日,祿山僭號於東京。庚申,以李光弼為雲中太守、河東節度使。壬戌,賊將蔡希德陷常山郡,執太守顏杲卿、長史袁履謙,殺民吏萬餘,城中流血。甲子,哥舒翰進位尚書左僕射、同中書門下平章事。乙丑,賊將安慶緒犯潼關,哥舒翰擊退之。乙巳,加平原太守顏真卿戶部侍郎,獎守城也。



 二月
 丙戌,李光弼、郭子儀將兵東出井陘,與賊將史思明戰,大破之,進取郡縣十餘。丙辰,誅工部尚書安思順。三月壬午朔,以河東節度使李光弼為御史大夫、範陽節度使。乙酉,以平原太守顏真卿為河北採訪使。己亥,改常山郡為平山郡,房山縣為平山縣,鹿泉縣為獲鹿縣,鹿成縣為束鹿縣。夏四月丙午,以贊善大夫來瑱為潁川太守、招討使。



 五月戊午,南陽太守魯炅與賊將武令珣戰於滍水上,官軍大敗,為賊所虜,進寇我南陽。詔嗣虢
 王巨自藍田出師救南陽。六月癸未朔,顏真卿破賊將袁知泰於堂邑,北海太守賀蘭進明收信都。庚寅,哥舒翰將兵八萬與賊將崔乾祐戰於靈寶西原,官軍大敗,死者十六七。其日,李光弼與賊將史思明戰於常山東嘉山,大破之,斬獲數萬計。辛卯,哥舒翰至潼關,為其帳下火拔歸仁以左右數十騎執之降賊,關門不守,京師大駭,河東、華陰、上洛等郡皆委城而走。甲午,將謀幸蜀,乃下詔親征,仗下,從士庶恐駭,奔走於路。乙未,凌晨自
 延秋門出,微雨沾濕,扈從惟宰相楊國忠、韋見素、內侍高力士及太子,親王,妃主、皇孫已下多從之不及。平明渡便橋,國忠欲斷橋。上曰:「後來者何以能濟?」命緩之。辰時,至咸陽望賢驛置頓,官吏駭散,無復儲供。上憩於宮門之樹下,亭午未進食。俄有父老獻鋋,上謂之曰:「如何得飯?」於是百姓獻食相繼。俄又尚食持御膳至,上頒給從官而後食。是夕次金城縣,官吏已遁,令魏方進男允招誘,俄得智藏寺僧進芻粟,行從方給。丙辰,次馬嵬驛,
 諸衛頓軍不進。龍武大將軍陳玄禮奏曰:「逆胡指闕,以誅國忠為名,然中外群情,不無嫌怨。今國步艱阻,乘輿震蕩,陛下宜徇群情,為社稷大計,國忠之徒,可置之於法。」會吐蕃使二十一人遮國忠告訴於驛門,眾呼曰:「楊國忠連蕃人謀逆!」兵士圍驛四合。及誅楊國忠、魏方進一族,兵猶未解。上令高力士詰之,回奏曰:「諸將既誅國忠,以貴妃在宮,人情恐懼。」上即命力士賜貴妃自盡。玄禮等見上請罪,命釋之。丁酉,將發馬嵬驛,朝臣唯韋見
 素一人,乃命見素子京兆府司錄諤為御史中丞,充置頓使。議其所向,軍士或言河、隴,或言靈武、太原,或言還京為便。韋諤曰:「還京,須有捍賊之備,兵馬未集,恐非萬全,不如且幸扶風,徐圖所向。」上詢於眾,咸以為然。及行,百姓遮路乞留皇太子,願戮力破賊,收復京城,因留太子。戊戌,次扶風縣。己亥,次扶風郡。軍士各懷去就,咸出醜言,陳玄禮不能制。會益州貢春彩十萬匹,上悉命置於庭,召諸將諭之曰:「卿等國家功臣,陳力久矣,朕之優
 獎,常亦不輕。逆胡背恩,事須回避。甚知卿等不得別父母妻子,朕亦不及親辭九廟。」言發涕流。又曰:「朕須幸蜀,路險狹,人若多往,恐難供承。今有此彩,卿等即宜分取,各圖去就。朕自有子弟中官相隨,便與卿等訣別。」眾咸俯伏涕泣曰:「死生願從陛下。」上曰:「去住任卿。」自此悖亂之言稍息。庚子,以司勛郎中、劍南節度留後崔圓為蜀郡長史、劍南節度副大使。以潁王璬為劍南節度大使,以監察御史宋若思為御史中丞充置頓使,韋諤充巡
 閣道使,並令先發。辛丑,發扶風郡,是夕,次陳倉。壬寅,次散關。分部下為六軍,潁王璬先行,壽王瑁等分統六軍,前後左右相次。丙午,次河池郡,崔圓奏劍南歲稔民安,儲供無闕,上大悅,授圓中書侍郎、同中書門下平章事,蜀郡長史、劍南節度如故。以前華州刺史魏犀為梁州長史。秋七月癸丑朔。壬戌,次益昌縣,渡吉柏江,有雙魚夾舟而躍,議者以為龍。甲子,次普安郡,憲部侍郎房琯自後至,上與語甚悅,即日拜為吏部尚書、同中書門下平
 章事。丁卯,詔以皇太子諱充天下兵馬元帥,都統朔方、河東、河北、平盧等節度兵馬,收復兩京;永王璘江陵府都督,統山南東路、黔中、江南西路等節度大使;盛王琦廣陵郡大都督,統江南東路、淮南、河南等路節度大使;豐王珙武威郡都督,領河西、隴石、安西、北庭等路節度大使。初,京師陷賊,車駕倉皇出幸,人未知所向,眾心震駭,及聞是詔,遠近相慶,咸思效忠於興復。庚午,次巴西郡,太守崔渙奉迎。即日以渙為門下侍郎、同中書門下
 平章事。以韋見素為左相。庚辰,車駕至蜀郡,扈從官吏軍士到者一千三百人,宮女二十四人而已。



 八月癸未朔,禦蜀都府衙,宣詔曰:「朕以薄德,嗣守神器,每乾乾惕厲,勤念生靈,一物失所,無忘罪己。聿來四紀,人亦小康,推心於人,不疑於物。而奸臣兇豎,棄義背恩,割剝黎元,擾亂區夏,皆朕不明之過也。今巡撫巴蜀,訓厲師徒,仍令太子諸王蒐兵重鎮,誅夷兇醜,以謝昊穹;思與群臣重弘理道,可大赦天下。」癸巳,靈武使至,始知皇太子即
 位。丁酉,上用靈武冊稱上皇,詔稱誥。己亥,上皇臨軒冊肅宗,命宰臣韋見素、房琯使靈武,冊命曰:「朕稱太上皇,軍國大事先取皇帝處分,後奏朕知。候克復兩京,朕當怡神姑射,偃息大庭。」



 明年九月,郭子儀收復兩京。十月,肅宗遣中使啖廷瑤入蜀奉迎。丁卯,上皇發蜀郡。十一月丙申,次鳳翔郡。肅宗遣精騎三千至扶風迎衛。十二月丙午,肅宗具法駕至咸陽望賢驛迎奉。上皇御宮之南樓,肅宗拜慶樓下,嗚咽流涕不自勝,為上皇徒步控
 轡,上皇撫背止之,即騎馬前導。丁未,至京師,文武百僚、京城士庶夾道歡呼,靡不流涕。即日御大明宮之含元殿,見百僚,上皇親自撫問。人人感咽。時太廟為賊所焚,權移神主於大內長安殿,上皇謁廟請罪,遂幸興慶宮。三載二月,肅宗與群臣奉上皇尊號曰太上至道聖皇帝。乾元三年七月丁未,移幸西內之甘露殿。時閹宦李輔國離間肅宗,故移居西內。高力士、陳玄禮等遷謫,上皇浸不自懌。



 上元二年四月甲寅,崩於神龍殿,時年七
 十八。群臣上謚曰至道大聖大明孝皇帝,廟號玄宗。初,上皇親拜五陵,至橋陵,見金粟山崗有龍盤鳳翥之勢,復近先塋,謂侍臣曰:「吾千秋後宜葬此地,得奉先陵,不忘孝敬矣。」至是,追奉先旨以創寢園,以廣德元年三月辛酉葬於泰陵。



 史臣曰:孔子稱「王者必世而後仁」。李氏自武後移國三十餘年,朝廷罕有正人,附麗無非險輩。持苞苴而請謁,奔走權門;效鷹犬以飛馳,中傷端士。以致斷喪王室,屠
 害宗枝。骨鯁大臣,屢遭誣陷,舞文酷吏,坐致顯榮。禮儀無復興行,刑政壞於犬馬,端揆出阿黨之語,冕旒有和事之名,朋比成風,廉恥都盡。



 我開元之有天下也,糾之以典刑,明之以禮樂,愛之以慈儉,律之以軌儀。黜前朝徼幸之臣,杜其奸也;焚後庭珠翠之玩,戒其奢也;禁女樂而出宮嬪,明其教也;賜酺賞而放哇淫,懼其荒也;敘友于而敦骨肉,厚其俗也;蒐兵而責帥,明軍法也;朝集而計最,校吏能也。廟堂之上,無非經濟之才;表著之中,
 皆得論思之士。而又旁求宏碩,講道藝文。昌言嘉謨,日聞於獻納;長轡遠馭,志在於升平。貞觀之風,一朝復振。於斯時也,烽燧不驚,華戎同軌。西蕃君長,越繩橋而競款玉關;北狄酋渠,捐毳幕而爭趨雁塞。象郡、炎州之玩,雞林、鯷海之珍,莫不結轍於象胥,駢羅於典屬。膜拜丹墀之下,夷歌立仗之前,可謂冠帶百蠻,車書萬里。天子乃覽雲臺之義,草泥金之札,然後封日觀,禪云亭,訪道於穆清,怡神於玄牝,與民休息,比屋可封。於時垂髫之
 倪,皆知禮讓;戴白之老,不識兵戈。虜不敢乘月犯邊,士不敢彎弓報怨。「康哉」之頌,溢於八紘。所謂「世而後仁」,見於開元者矣。年逾三紀,可謂太平。



 於戲!國無賢臣,聖亦難理;山有猛虎,獸不敢窺。得人者昌,信不虛語。昔齊桓公行同禽獸,不失霸主之名;梁武帝靜比桑門,竟被臺城之酷。蓋得管仲則淫不害霸,任硃異則善不救亡。開元之初,賢臣當國,四門俱穆,百度唯貞,而釋、老之流,頗以無為請見。上乃務清凈,事薰修,留連軒後之文,舞詠
 伯陽之說,雖稍移於勤倦,亦未至於怠荒。俄而朝野怨咨,政刑紕繆,何哉?用人之失也。自天寶已還,小人道長。如山有朽壞,雖大必虧;木有蠹蟲,其榮易落。以百口百心之讒諂,蔽兩目兩耳之聰明,茍非鐵腸石心,安得不惑!而獻可替否,靡聞姚、宋之言;妒賢害功,但有甫、忠之奏。豪猾因茲而睥睨,明哲於是乎卷懷,故祿山之徒,得行其偽。厲階之作,匪降自天,謀之不臧,前功並棄。惜哉!



 贊曰:開元握圖,永鑒前車。景氣融朗,昏氛滌除。政才勤
 倦,妖集廷除。先民之言,「靡不有初」。



\end{pinyinscope}