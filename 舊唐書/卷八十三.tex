\article{卷八十三}

\begin{pinyinscope}

 ○祖
 孝孫傅仁均傅弈李淳風呂才



 祖孝孫,幽州範陽人也。父崇儒,以學業知名,仕至齊州長史。孝孫博學,曉歷算,早以達識見稱。初,開皇中,鐘律多缺,雖何妥、鄭譯、蘇夔、萬寶常等亟共討詳,紛然不定。
 及平江左,得陳樂官蔡子元、于普明等,因置清商署。時牛弘為太常卿,引孝孫為協律郎,與子元、普明參定雅樂。時又得陳陽山太守毛爽,妙知京房律法,布琯飛灰,順月皆驗。爽時年老,弘恐失其法,於是奏孝孫從其受律。孝孫得爽之法,一律而生五音,十二律而為六十音,因而六之,故有三百六十音,以當一歲之日。又祖述洗重,依淮南本數,用京房舊術求之,得三百六十律,各因其月律而為一部。以律數為母,以一中氣所有日為子,
 以母命子,隨所多少,分直一歲,以配七音,起於冬至。以黃鐘為宮,太簇為商,林鐘為徵,南呂為羽,姑洗為角,應鐘為變宮,蕤賓為變徵。其餘日建律,皆依運行。每日各以本律為宮。旋宮之義,由斯著矣。然牛弘既初定樂,難復改張。至大業時,又採晉、宋舊樂,唯奏《皇夏》等十有四曲,旋宮之法,亦不施用。高祖受禪,擢孝孫為著作郎,歷吏部郎、太常少卿,漸見親委,孝孫由是奏請作樂。時軍國多務,未遑改創,樂府尚用隋氏舊文。武德七年,始命
 孝孫及秘書監竇璡修定雅樂。孝孫又以陳、梁舊樂雜用吳、楚之音,周、齊舊樂多涉胡戎之伎,於是斟酌南北,考以古音,作《大唐雅樂》。以十二月各順其律,旋相為宮,制十二樂,合三十二曲、八十四調。事具《樂志》。旋宮之義,亡絕已久,世莫能知,一朝復古,自孝孫始也。孝孫尋卒。其後,協律郎張文收復採《三禮》,增損樂章,然因孝孫之本音。



 傅仁均,滑州白馬人也。善歷算、推步之術。武德初,太史
 令庾儉、太史丞傅弈表薦之,高祖因召令改修舊歷。仁均因上表陳七事:其一曰:「昔洛下閎以漢武太初元年歲在丁丑,創歷起元,元在丁丑。今大唐以戊寅年受命,甲子日登極,所造之歷,即上元之歲,歲在戊寅,命日又起甲子,以三元之法,一百八十去其積歲,武德元年戊寅為上元之首,則合璧連珠,懸合於今日。」其二曰:「《堯典》為『日短星昴,以正仲冬』,前代造歷,莫能允合。臣今創法,五十餘年,冬至輒差一度,則卻檢周、漢,千載無違。」其三
 曰:「經書日蝕,《毛詩》為先,『十月之交,朔日辛卯』。臣今立法,卻推得周幽王六年辛卯朔蝕,即能明其中間,並皆符合。」其四曰:「《春秋命歷序》云:『魯僖公五年壬子朔旦冬至。』諸歷莫能符合。臣今造歷,卻推僖公五年正月壬子朔旦冬至則同,自斯以降,並無差爽。」其五曰:「古歷日蝕或在於晦,或在二日;月蝕或在望前,或在望後。臣今立法,月有三大三小,則日蝕常在於朔,月蝕在望前。卻驗魯史,並無違爽。」其六曰:「前代造歷,命辰不從子半,命度不
 起虛中。臣今造歷,命辰起子半,度起於虛六,度命合辰,得中於子,符陰陽之始,會歷術之宜。」其七曰:「前代諸歷,月行或有晦猶東見、朔已西朓。臣今以遲疾定朔,永無此病。」經數月,歷成奏上,號曰《戊寅元歷》,高祖善之。武德元年七月,詔頒新歷,授仁均員外散騎常侍,賜物二百段。



 後中書令封德彞奏歷術差謬,敕吏部郎中祖孝孫考其得失。又太史丞王孝通執《甲辰歷法》以駁之曰:



 案《堯典》云:「日短星昴,以正仲冬。」孔氏云,七宿畢見,舉中者
 言耳。是知中星無定,故互舉一分兩至之星以為成驗也。昴西方處中之宿,虛為北方居中之星,一分各舉中者,即余六星可知。若乃仲冬舉鳥,仲夏舉火,此一至一分又舉七星之體,則餘二方可見。今仁均專守昴中而為定朔,執文害意,不亦謬乎!又案《月令》:仲冬「昏在東壁」。明知昴中則非常準。若言陶唐之代,定是昴中,後代漸差,遂至東壁。然則堯前七千餘載,冬至之日,即便合翼中,逾遠彌卻,尤成不隱。且今驗東壁昏中,日體在斗十
 有三度;若昏於翼中,日應在井十有三度。夫井極北,去人最近,而鬥極南,去人最遠,在井則大熱,在斗乃大寒。然堯前冬至,即應翻熱,及於夏至,便應反寒。四時倒錯,寒暑易位,以理推尋,必不然矣。又,鄭康成,博達之士也。對弟子孫皓云:日永星火,只是大火之次二十度有其中者,非謂心之火星也,實正中也。又平朔、定朔,舊有二家;平望、定望,由來兩術。然三大三小,是定朔、定望之法;一大一小,是平朔、平望之義。且日月之行,有遲有疾,每
 月一相及,謂之合會。故晦朔無定,由人消息。若定大小合朔者,合會雖定,而蔀元紀首,三端並失。若上合履端之始,下得歸餘於終,合會時有進退,履端又皆允協,則《甲辰元歷》為通術矣。



 仁均對曰:



 宋代祖沖之久立差術,至於隋代張胄玄等,因而修之,雖差度不同,各明其意。今孝通不達宿度之差移,未曉黃道之遷改,乃執南斗為冬至之恆星,東井為夏至之常宿,率意生難,豈為通理?夫太陽行於宿度,如郵傳之過逆旅,宿度每歲既差,
 黃道隨而變易,豈得以膠柱之說而為斡運之難乎?又案《易》云:「治歷明時。」《禮》云:「天子玄端,聽朔於南門之外。」《尚書》云:「正月上日,受終於文祖。」孔氏云:「上日,朔日也。」又云:「季秋月朔,辰不集於房。」孔氏云:「集,合也。不合,則日蝕隨可知矣。」又云:「先時、不及時,皆殺無赦。」先時,謂朔日不及時也。若有先後之差,是不知定朔之道矣。《詩》云:「十月之交,朔日辛卯。」又,《春秋》日蝕三十有五,左丘明云:「不書朔,官失之也。」明聖人之教,不論於晦,唯取朔耳。自春秋以
 後,去聖久遠,歷術差違,莫能詳正。故秦、漢以來,多非朔蝕,而宋代御史中丞何承天微欲見意,不能詳究,乃為太史令錢樂之、散騎侍郎皮延宗所抑止。孝通今語,乃是延宗舊辭。承天既非甄明,故有當時之屈。今略陳梗概,申以明之。夫理歷之本,必推上元之歲,日月如合璧,五星如連珠,夜半甲子朔旦冬至。自此以後,既行度不同,七曜分散,不知何年更得餘分普盡,還復總會之時也?唯日分氣分,得有可盡之理,因其得盡,即有三端之
 元。故造經立法者,小餘盡即為元首,此乃紀其日數之元,不關合璧之事矣。時人相傳,皆云大小餘俱盡,即定夜半甲子朔旦冬至者,此不達其意故也。何者?冬至自有常數,朔名由於月起,既月行遲疾無常,三端豈得即合?故必須日月相合,與冬至同日者,始可得名為合朔冬至耳。故前代諸歷,不明其意,乃於大餘正盡之年而立其元法,將以為常,而不知七曜散行,氣朔不合。今法唯取上元連珠合璧,夜半甲子朔旦冬至,合朔之始以
 定,一九相因,行至於今日,常取定朔之宜,不論三端之事。皮延宗本來不知,何承天亦自未悟,何得引而相難耶?



 孝孫以仁均之言為然。



 貞觀初,有益州人陰弘道,又執孝通舊說以駁之,終不能屈。李淳風復駁仁均歷十有八事,敕大理卿崔善為考二家得失,七條改從淳風,餘一十一條並依舊定。仁均後除太史令,卒官。



 傅奕,相州鄴人也。尤曉天文歷數。隋開皇中,以儀曹事漢王諒。及諒舉兵,謂奕曰:「今茲熒惑入井,是何祥也?」奕
 對曰:「天上東井,黃道經其中,正是熒惑行路,所涉不為怪異;若熒惑入地上井,是為災也。」諒不悅。及諒敗,由是免誅,徙扶風。高祖為扶風太守,深禮之。及踐祚,召拜太史丞。太史令庾儉以其父質在隋言占候忤煬帝意,竟死獄中,遂懲其事,又恥以數術進,乃薦奕自代,遂遷太史令。奕既與儉同列,數排毀儉,而儉不之恨,時人多儉仁厚而稱奕之率直。奕所奏天文密狀,屢會上旨,置參旗、井鉞等十二軍之號,奕所定也。武德三年,進《漏刻新
 法》,遂行於時。七年,奕上疏請除去釋教,曰:



 佛在西域,言妖路遠,漢譯胡書,恣其假托。故使不忠不孝,削發而揖君親;游手游食,易服以逃租賦。演其妖書,述其邪法,偽啟三途,謬張六道,恐嚇愚夫,詐欺庸品。凡百黎庶,通識者稀,不察根源,信其矯詐。乃追既往之罪,虛規將來之福。布施一錢,希萬倍之報;持齋一日,冀百日之糧。遂使愚迷,妄求功德,不憚科禁,輕犯憲章。其有造作惡逆,身墜刑網,方乃獄中禮佛,口誦佛經,晝夜忘疲,規免其罪。
 且生死壽夭,由於自然;刑德威福,關之人主。乃謂貧富貴賤,功業所招,而愚僧矯詐,皆云由佛。竊人主之權,擅造化之力,其為害政,良可悲矣!案《書》云:「惟闢作福威,惟闢玉食。臣有作福、作威、玉食,害於而家,兇於而國,人用側頗僻。」降自羲、農,至於漢、魏,皆無佛法,君明臣忠,祚長年久。漢明帝假托夢想,始立胡神,西域桑門,自傳其法。西晉以上,國有嚴科,不許中國之人,輒行髡發之事。洎於苻、石,羌胡亂華,主庸臣佞,政虐祚短,皆由佛教致災
 也。梁武、齊襄,足為明鏡。昔褒姒一女,妖惑幽王,尚致亡國;況天下僧尼,數盈十萬,翦刻繒彩,裝束泥人,而為厭魅,迷惑萬姓者乎!今之僧尼,請令匹配,即成十萬餘戶。產育男女,十年長養,一紀教訓,自然益國,可以足兵。四海免蠶食之殃,百姓知威福所在,則妖惑之風自革,淳樸之化還興。且古今忠諫,鮮不及禍。竊見齊朝章仇子他上表言:「僧尼徒眾,糜損國家,寺塔奢侈,虛費金帛。」為諸僧附會宰相,對朝讒毀;諸尼依托妃主,潛行謗讟。子
 他竟被囚執,刑於都市。及周武平齊,制封其墓。臣雖不敏,竊慕其蹤。



 又上疏十一首,詞甚切直。高祖付群官詳議,唯太僕卿張道源稱奕奏合理。中書令蕭瑀與之爭論曰:「佛,聖人也。奕為此議,非聖人者無法,請置嚴刑。」奕曰:「禮本於事親,終於奉上,此則忠孝之理著,臣子之行成。而佛逾城出家,逃背其父,以匹夫而抗天子,以繼體而悖所親。蕭瑀非出於空桑,乃遵無父之教。臣聞非孝者無親,其瑀之謂矣!」瑀不能答,但合掌曰:「地獄所設,正
 為是人。」高祖將從奕言,會傳位而止。



 奕武德九年五月密奏,太白見秦分,秦王當有天下,高祖以狀授太宗。及太宗嗣位,召奕賜之食,謂曰:「汝前所奏,幾累於我,然今後但須盡言,無以前事為慮也。」太宗常臨朝謂奕曰:「佛道玄妙,聖跡可師,且報應顯然,屢有徵驗,卿獨不悟其理,何也?」奕對曰:「佛是胡中桀黠,欺誑夷狄,初止西域,漸流中國。遵尚其教,皆是邪僻小人,模寫莊、老玄言,文飾妖幻之教耳。於百姓無補,於國家有害。」太宗頗然之。貞
 觀十三年卒,年八十五。臨終誡其子曰:「老、莊玄一之篇,周、孔《六經》之說,是為名教,汝宜習之。妖胡亂華,舉時皆惑,唯獨竊嘆,眾不我從,悲夫!汝等勿學也。古人裸葬,汝宜行之。」奕生平遇患,未嘗請醫服藥,雖究陰陽數術之書,而並不之信。又嘗醉臥,蹶然起曰:「吾其死矣!」因自為墓志曰:「傅奕,青山白雲人也。因酒醉死,嗚呼哀哉!」其縱達皆此類。注《老子》,並撰《音義》,又集魏、晉已來駁佛教者為《高識傳》十卷,行於世。



 李淳風,岐州雍人也。其先自太原徙焉。父播,隋高唐尉,以秩卑不得志,棄官而為道士。頗有文學,自號黃冠子。注《老子》,撰《方志圖》,文集十卷,並行於代。淳風幼俊爽,博涉群書,尤明天文、歷算、陰陽之學。貞觀初,以駁傅仁均歷議,多所折衷,授將仕郎,直太史局。尋又上言曰:「今靈臺候儀,是魏代遺範,觀其制度,疏漏實多。臣案《虞書》稱,舜在璇璣玉衡,以齊七政。則是古以混天儀考七曜之盈縮也。《周官》大司徒職,以土圭正日景,以定地中。此亦
 據混天儀日行黃道之明證也。暨於周末,此器乃亡。漢孝武時,洛下閎復造混天儀,事多疏闕。故賈逵、張衡各有營鑄,陸績、王蕃遞加修補,或綴附經星,機應漏水,或孤張規郭,不依日行,推驗七曜,並循赤道。今驗冬至極南,夏至極北,而赤道當定於中,全無南北之異,以測七曜,豈得其真?黃道渾儀之闕,至今千餘載矣。」太宗異其說,因令造之,至貞觀七年造成。其制以銅為之,表裏三重,下據準基,狀如十字,末樹鰲足,以張四表焉。第一
 儀名曰六合儀,有天經雙規、渾緯規、金常規,相結於四極之內,備二十八宿、十干、十二辰,經緯三百六十五度。第二名三辰儀,圓徑八尺,有璇璣規道,月游天宿矩度,七曜所行,並備於此,轉於六合之內。第三名四游儀,玄樞為軸,以連結玉衡游筒而貫約規矩;又玄樞北樹北辰,南距地軸,傍轉於內;又玉衡在玄樞之間而南北游,仰以觀天之辰宿,下以識器之晷度。時稱其妙。又論前代渾儀得失之差,著書七卷。名為《法象志》以奏之。太宗
 稱善,置其儀於凝暉閣,加授承務郎。十五年,除太常博士。尋轉太史丞,預撰《晉書》及《五代史》,其《天文》、《律歷》、《五行志》皆淳風所作也。又預撰《文思博要》。二十二年,遷太史令。初,太宗之世有《秘記》云:「唐三世之後,則女主武王代有天下。」太宗嘗密召淳風以訪其事,淳風曰:「臣據象推算,其兆已成。然其人已生,在陛下宮內,從今不逾三十年,當有天下,誅殺唐氏子孫殲盡。」帝曰:「疑似者盡殺之,如何?」淳風曰:「天之所命,必無禳避之理。王者不死,多恐枉
 及無辜。且據上象,今已成,復在宮內,已是陛下眷屬。更三十年,又當衰老,老則仁慈,雖受終易姓。其於陛下子孫,或不甚損。今若殺之,即當復生,少壯嚴毒,殺之立讎。若如此,即殺戮陛下子孫,必無遺類。」太宗然竟善其言而止。淳風每占候吉兇,合若符契,當時術者疑其別有役使,不因學習所致,然竟不能測也。顯慶元年,復以修國史功封樂昌縣男。先是,太史監候王思辯表稱《五曹》、《孫子》十部算經理多踳駁。淳風復與國子監算學博士梁述、
 太學助教王真儒等受詔注《五曹》、《孫子》十部算經。書成,高宗令國學行用。龍朔二年,改授秘閣郎中。時《戊寅歷法》漸差,淳風又增損劉焯《皇極歷》,改撰《麟德歷》奏之,術者稱其精密。咸亨初,官名復舊,還為太史令。年六十九卒。所撰《典章文物志》、《乙巳占》、《秘閣錄》,並《演齊人要術》等凡十餘部,多傳於代。子諺,孫仙宗,並為太史令。



 呂才,博州清平人也。少好學,善陰陽方伎之書。貞觀三年,太宗令祖孝孫增損樂章,孝孫乃與明音律人王長
 通、白明達遞相長短。太宗令侍臣更訪能者,中書令溫彥博奏才聰明多能,眼所未見,耳所未聞,一聞一見,皆達其妙,尤長於聲樂,請令考之。侍中王珪、魏徵又盛稱才學術之妙,徵曰:「才能為尺十二枚,尺八長短不同,各應律管,無不諧韻。」太宗即征才,令直引文館。太宗嘗覽周武帝所撰《三局象經》,不曉其旨。太子洗馬蔡允恭年少時嘗為此戲,太宗召問,亦廢而不通,乃召才使問焉。才尋繹一宿,便能作圖解釋,允恭覽之,依然記其舊法,
 與才正同,由是才遂知名。累遷太常博士。太宗以陰陽書近代以來漸致訛偽,穿鑿既甚,拘忌亦多。遂命才與學者十餘人共加刊正,削其淺俗,存其可用者。勒成五十三卷,並舊書四十七卷,十五年書成,詔頒行之。才多以典故質正其理,雖為術者所短,然頗合經義,今略載其數篇。



 其敘《宅經》曰:



 《易》曰:「上古穴居而野處,後世聖人易以宮室,蓋取諸大壯。」迨於殷、周之際,乃有卜宅之文,故《詩》稱「相其陰陽」,《書》云「卜惟洛宅」,此則卜宅吉兇,其來
 尚矣。至於近代師巫,更加五姓之說。言五姓者,謂宮、商、角、徵、羽等。天下萬物,悉配屬之,行事吉兇,依此為法。至如張、王等為商,武、庾等為羽,欲似同韻相求。及其以柳姓為宮,以趙姓為角,又非四聲相管。其間亦有同是一姓,分屬宮商,後有復姓數字,徵羽不別。驗於經典,本無斯說,諸陰陽書,亦無此語,直是野俗口傳,竟無所出之處。唯《堪輿經》,黃帝對於天老,乃有五姓之言。且黃帝之時,不過姬、姜數姓,暨於後代,賜族者多。至如管、蔡、成、霍、
 魯、衛、毛、聃、郜、雍、曹、滕、畢、原、酆、郇,並是姬姓子孫;孔、殷、宋、華、向、蕭、亳、皇甫,並是子姓苗裔。自餘諸國,準例皆然。因邑因官,分枝布葉,未知此等諸姓,是誰配屬?又檢《春秋》,以陳、衛及秦並同水姓,齊、鄭及宋皆為火姓,或承所出之祖,或系所屬之星,或取所居之地,亦非宮、商、角、徵,共相管攝。此則事不稽古,義理乖僻者也。



 敘《祿命》曰:



 謹案《史記》,宋忠、賈誼譏司馬季主云:「夫卜筮者,高人祿命以悅人心,矯言禍福以盡人財。」又案王充《論衡》云:「見骨體
 而知命祿,睹命祿而知骨體。」此即祿命之書,行之久矣。多言或中,人乃信之。今更研尋,本非實錄。但以積善餘慶,不假建祿之吉;積惡餘殃,豈由劫殺之災?皇天無親,常與善人,禍福之應,其猶影響。故有夏多罪,天命剿絕;宋景修德,妖孛夜移。學也祿在,豈待生當建學。文王勤憂損壽,不關月值空亡。長平坑卒,未聞共犯三刑;南陽貴士,何必俱當六合?歷陽成湖,非獨河魁之上;蜀郡炎燎,豈由災厄之下?今時亦有同年同祿,而貴賤懸殊;共
 命共胎,而夭壽更異。



 案《春秋》,魯桓公六年七月,魯莊公生。今檢《長歷》,莊公生當乙亥之歲,建申之月。以此推之,莊公乃當祿之空亡。依祿命書,法合貧賤,又犯勾絞六害,背驛馬三刑,當此三者,並無官爵。火命七月,生當病鄉,為人尪弱,身合矬陋。今案《齊詩》譏莊公「猗嗟昌兮,頎若長兮。美目揚兮,巧趨蹌兮。」唯有向命一條,法當長命。依檢《春秋》,莊公薨時計年四十五矣。此則祿命不驗一也。又案《史記》,秦莊襄王四十八年,始皇帝生,宋忠注云:「
 因正月生,乃名政。」依檢襄王四十八年,歲在壬寅。此年正月生者,命當背祿,法無官爵,假得祿合,奴婢尚少。始皇又當破驛馬三刑,身克驛馬,法當望官不到,金命正月,生當絕下,為人無始有終,老而彌吉。今檢《史記》,始皇乃是有始無終,老更彌兇。唯建命生,法合長壽,計其崩時,不過五十。祿命不驗二也。又《漢武故事》,武帝以乙酉之歲七月七日平旦時生。亦當祿空亡下,法無官爵,雖向驛馬,尚隔四辰。依祿命法,少無官榮,老而方盛。今檢《
 漢書》,武帝即位,年始十六,末年已後,戶口減半。祿命不驗三也。又按《後魏書》云:孝文皇帝皇興元年八月生。今按《長歷》,其年歲在丁未。以此推之,孝文皇帝背祿命並驛馬三刑,身克驛馬。依祿命書,法無官爵,命當父死中生,法當生不見父。今檢《魏書》,孝文皇帝身受其父顯祖之禪。禮云:「嗣子位定於初喪,逾年之後,方始正號。是以天子無父,事三老也。孝文受禪,異於常禮,躬率天下,以事其親,而祿命云不合識父。祿命不驗四也。又按沈約《
 宋書》云:「宋高祖癸亥歲三月生。依此而推,祿之與命,並當空亡。依祿命書,法無官爵;又當子墓中生,唯宜嫡子,假有次子,法當早卒。今檢《宋書》,高祖長子先被篡弒,次子義隆,享國多年。高祖又當祖祿下生,法得嫡孫財祿。今檢《宋書》其孫劉劭、劉浚並為篡逆,幾失宗祧。祿命不驗五也。



 敘《葬書》曰:



 《易》曰:「古之葬者,衣之以薪,不封不樹,喪期無數。」後世聖人易之以棺槨,蓋取諸《大過》。《禮》云:「葬者,藏也,欲使人不得見之。」然《孝經》云:「卜其宅兆而安厝
 之。」以其顧復事畢,長為感慕之所;窀穸禮終,永作魂神之宅。朝市遷變,不得豫測於將來,泉石交侵,不可先知於地下。是以謀及龜筮,庶無後艱,斯乃備於慎終之禮,曾無吉兇之義。暨乎近代以來,加之陰陽葬法,或選年月便利,或量墓田遠近,一事失所,禍及死生。巫者利其貨賄,莫不擅加妨害。遂使葬書一術,乃有百二十家。各說吉兇,拘而多忌。且天覆地載,乾坤之理備焉;一剛一柔,消息之義詳矣。或成於晝夜之道,感於男女之化,三
 光運於上,四氣通於下,斯乃陰陽之大經,不可失之於斯須也。至於喪葬之吉兇,乃附此為妖妄。《傳》云:「王者七日而殯,七月而葬;諸侯五日而殯,五月而葬;大夫經時而葬;士及庶人逾月而已。」此則貴賤不同,禮亦異數。欲使同盟同軌,赴吊有期,量事制宜,遂為常式。法既一定,不得違之。故先期而葬,謂之不懷;後期而不葬,譏之殆禮。此則葬有定期,不擇年月,一也。《春秋》又云:丁巳,葬定公,雨,不克葬,至於戊午襄事。禮經善之。《禮記》云「卜葬先
 遠日」者,蓋選月終之日,所以避不懷也。今檢葬書,以己亥之日用葬最兇。謹按春秋之際,此日葬者凡有二十餘件。此則葬不擇日,二也。《禮記》又云:「周尚赤,大事用平旦;殷尚白,大事用日中;夏尚黑,大事用昏時。」鄭玄《注》云:「大事者何?謂喪葬也。」此則直取當代所尚,不擇時之早晚。《春秋》云,鄭卿子產及子太叔葬鄭簡公,於時司墓大夫室當葬路。若壞其室,即平旦而窆;不壞其室,即日中而窆。子產不欲壞室,欲待日中。子太叔云:「若至日中而
 窆,恐久勞諸侯大夫來會葬者。」然子產既云博物君子,太叔乃為諸侯之選,國之大事,無過喪葬,必是義有吉兇,斯等豈得不用?今乃不問時之得失,唯論人事可否。《曾子問》云:「葬逢日蝕,舍於路左,待明而行,所以備非常也。」若依葬書,多用乾、艮二時,並是近半夜,此即文與禮違。今檢《禮傳》,葬不擇時,三也。葬書云,富貴官品,皆由安葬所致;年命延促,亦曰墳壟所招。然今按《孝經》云:「立身行道,則揚名於後世,以顯父母。」《易》曰:「聖人之大寶曰位,
 何以守位曰仁。」是以日慎一日,則澤及於無疆;茍德不建,則人而無後,此則非由安葬吉兇而論福祚延促。臧孫有後於魯,不關葬得吉日,若敖絕祀於荊,不由遷厝失所。此則安葬吉兇不可信用,其義四也。今之喪葬吉兇,皆依五姓便利。古之葬者,並在國都之北,域兆既有常所,何取姓墓之義?趙氏之葬,並在九原;漢之山陵,散在諸處。上利下利,蔑爾不論,大墓小墓,其義安在?及其子孫富貴不絕,或與三代同風,或分六國而王。此則五
 姓之義,大無稽古;吉兇之理,何從而生?其義五也。且人臣名位,進退何常,亦有初賤而後貴,亦有始泰而終否。是以子文三已令尹,展禽三黜士師。卜葬一定,更不回改,塚墓既成,曾不革易,則何因名位無時暫安。故知官爵弘之在人,不由安葬所致。其義六也。野俗無識,皆信葬書,巫者詐其吉兇,愚人因而徼幸。遂使擗踴之際,擇葬地而希官品;荼毒之秋,選葬時以規財祿。或云辰日不宜哭泣,遂莞爾而對賓客受吊;或云同屬忌於臨壙,
 乃吉服不送其親。聖人設教,豈其然也?葬書敗俗,一至於斯,其義七也。



 太宗又令才造《方域圖》及《教飛騎戰陣圖》,皆稱旨,擢授太常丞。永徽初,預修《文思博要》及《姓氏錄》。顯慶中,高宗以琴曲古有《白雪》,近代頓絕,使太常增修舊曲。才上言曰:「臣按《禮記》及《家語》云,舜彈五弦之琴,歌《南風》之詩。是知琴操曲弄,皆合於歌。又張華《博物志》云:《白雪》是天帝使素女鼓五十弦瑟曲名。又楚大夫宋玉對襄王云,有客於郢中歌《陽春白雪》,國中和者數十
 人。是知《白雪》琴曲,本宜合歌,以其調高,人和遂寡。自宋玉已來,迄今千祀,未有能歌《白雪》曲者。臣今準敕,依琴中舊曲,定其宮商,然後教習,並合於歌,輒以禦制《雪詩》為《白雪》歌詞。又案古今樂府,奏正曲之後,皆別有送聲,君唱臣和,事彰前史。今取太尉長孫無忌、僕射於志寧、侍中許敬宗等《奉和雪詩》以為送聲,合十六節,今悉教訖,並皆合韻。」高宗大悅,更作《白雪歌詞》十六首,付太常編於樂府。時右監門長史蘇敬上言,陶弘景所撰《本草》,
 事多舛謬。詔中書令許敬宗與才及李淳風、禮部郎中孔志約,並諸名醫,增損舊本,仍令司空李勣總監定之,並圖合成五十四卷,大行於代。才龍朔中為太子司更大夫,麟德二年卒。著《隋記》二十卷,行於時。



 子方毅,七歲能誦《周易》、《毛詩》。太宗聞其幼敏,召見,甚奇之,賜以縑帛。後為右衛鎧曹參軍。母終,哀慟過禮,竟以毀卒。布車載喪,隨轜車而葬。友人郎餘令以白粥、玄酒,生芻一束,於路隅奠祭,甚為時人之所哀惜。



 史臣曰:孝孫定音律,仁均正歷數,淳風候象緯,呂才推陰陽,訂於其倫,咸以為裨、梓、京、管之流也。然旋宮三代之法,秦火籍煬,歷代缺其正音,而云孝孫復始,大可嘆也。淳風精於術數,能知女主革命,而不知其人,則所未喻矣。呂才核拘忌之曲學,皆有經據,不亦賢乎!古人所以存而不議,蓋有意焉。



 贊曰:祖、傅、淳、才,彰往考來。裁筠嶰谷,運箸清臺。推迎斡運,圖寫昭回。重黎之後,諸子賢哉!



\end{pinyinscope}