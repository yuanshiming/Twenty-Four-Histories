\article{卷四十二 志第二十二 職官一}

\begin{pinyinscope}

 高祖發跡太原,官名稱位,皆依隋舊。及登極之初,未
 遑改作,隨時署置,務從省便。武德七年定令:以太尉、司徒、司空為三公。尚書、門下、中書、秘書、殿中、內侍為六省。次御史臺;次太常、光祿、衛尉、宗正、太僕、大理、鴻臚、司農、太府,為九寺;次將作監;次
 國子學;次天策上將府;次左右衛、左右驍衛、左右領軍、左右武候、左右監門、左右屯、左右領,為十四衛府。東宮置三師、三少、詹事府、門下典書兩坊。次內坊;次家令、率更、僕三寺;次左右衛率府、左右宗衛率府、左右虞候率府、左右監門率府、左右內率府,為十率府。王公以下置府佐國官。公主置邑司已下。並為京職事官。州縣、鎮戍、岳瀆、關津為外職事官。又以開府儀同三司、從一品。特進、正二品。左光祿大夫、
 從一品。右光祿大夫、正二品。散騎常侍、從三品。太中大夫、正四品。通直散騎常侍、正四品。中大夫、從四品上。員外散騎常侍、從四品下。中散大夫、正五品上。散騎侍郎正五品下。通直散騎侍郎、從五品上。



 員外散騎侍郎、從五品下。朝議郎、承議郎、正六品。通議郎、通直郎、從六品。朝請郎、宣德郎、正七品。朝散郎、宣義郎、從七品。給事郎、徵事郎、正八品。承奉郎、承務郎、從八品。儒林郎、登仕郎、正九品。文林郎、將仕郎,
 從九品。並為文散官。



 輔國、正二品。鎮軍從二品。二大將軍,冠軍、正三品。雲麾、從三品。忠
 武、壯武、宣威、明威、信遠、游騎、游擊自正四品上到從五品下。



 十將軍,為散號將軍,以加武士之無職事者。改上開府儀同三司為上輕車都尉,
 開府儀同三司為輕車都尉,儀同三司為騎都尉,秦王、齊王下統軍為護軍,副統軍為副護軍,上大都督為驍騎尉,大都督為飛騎尉,帥都督為雲騎尉,都督為武騎尉,車騎將軍為游騎將軍,親衛驃騎將軍為親衛中郎將,其勛衛驃騎準此。親衛車騎將軍為親衛郎將,其勛衛翊衛車騎並準此。監門府郎將為監門中郎將,領左右郎將,準此。諸軍驃騎將軍為統軍,其秦王、齊王下領三衛及庫直、驅咥直、車騎並準此。諸軍車騎將
 軍為別將。其散官文騎尉為承議郎,屯騎尉為通直郎,雲騎尉為登仕郎,羽騎尉為將仕郎。武德九年,罷天策上將府。



 貞觀元年,改國子學為國子監,分將作為少府監,通將作為三監。八年七月,始以雲麾將軍為從三品階。九月,
 以統軍正四品下,別將正五品上。十
 一年,改令置太師、太傅、太保為三師。其三公已下,六
 省、一臺、九寺、三監、十二衛、東宮諸司,並從舊定。又改以光祿大夫為從二品,金紫光祿大夫為正三品,銀青光祿大夫為從三品,正議大夫為正四品上,通議大夫為正四品下,太中大夫為從四品上,中大夫為從四品下,中散大夫為正五品上,朝議大夫為正五品下,朝請大夫為從五品上,朝散大夫為從五品下。其六品下,唯改通議郎為奉議郎,自餘依舊。更置驃騎大將軍為從一品武散官;輔國、鎮軍二大將軍為從二品武散官。冠軍
 將軍加大字。及雲麾已下,游擊已上,改為五品已上武散官。又置昭武、振威、致果、翊麾、宣節、御武、仁勇、陪戎八校尉副尉,自正六品至從九品,上階為校尉,下為副尉。為六品已下武散官。



 凡九品已上職事,皆帶散位,謂之本品。職事則隨才錄用,或從閑入劇,或去高就卑,遷從出入,參差不定。散位則一切以門廕結品,然後勞考進敘。《武德令》,職事高者解散官,欠一階不到為「兼」。職事卑者,不解散官。《貞觀令》,以職事高者為「守」,職事卑者為「行」,仍各帶散位。其欠一階,依
 舊為「兼」,與當階者,皆解散官。永徽已來,欠一階者,或為兼,或帶散官,或為守,參而用之。其兩職事者亦為「兼」,頗相錯亂。



 其欠一階之「兼」,古念反。其職事之兼,古恬反。字同音異耳。咸亨二年,始一切為「守」。



 自高宗之後,官名品秩,屢有改易。今錄永泰二年官品。其改易品秩者,注於官品之下。若改官名及職員有加減者,則各附之於本職云。



 唐初因隋號,武德三年三月,改納言為侍中,內史令為中書令,給事郎為給事中,內書省為中書省。貞觀二十三年六月,改民部尚
 書為戶部尚書。七月,改治書侍御史為御史中丞,改諸州治中為司馬,別駕為長史,治禮郎為奉禮郎。顯慶元年,改戶部尚書為度支尚書,侍郎為度支侍郎。又置驃騎大將軍員,從一品。龍朔二年二月甲子,改百司及官名。改尚書省為中臺,僕射為匡政,左右丞為肅機,左右司郎中為丞務,吏部為司列,主爵為司封,考功為司績,禮部為司禮,祠部為司禋,膳部為司膳,主客為司蕃,戶部為司元,度支為司度,倉部為司倉,金部為司珍,兵部
 為司戎,職方為司域,駕部為司輿,庫部為司庫,刑部為司刑,都官為司僕,比部為司計,工部為司平,屯田為司田,虞部為司虞,水部為司川,餘司依舊。尚書為太常伯,侍郎為少常伯,郎中為大夫。中書門下為東西臺。侍中為左相,黃門侍郎為東臺侍郎,給事中為東臺舍人,散騎常侍為左右侍極,諫議大夫為正諫大夫。中書令為右相,侍郎為西臺侍郎,舍人為西臺舍人。秘書省為蘭臺,監為太史,少監為侍郎,丞為大夫。著作郎為司文郎
 中,太史令為秘閣郎中。御史臺為憲臺,御史大夫為大司憲,御史中丞為司憲大夫。殿中省為中御府,丞為大夫。尚食為奉膳,尚藥為奉醫,尚衣為奉冕,尚舍為奉扆,尚乘為奉駕,尚輦為奉御,並為大夫。內侍省為內侍監。太常為奉常,光祿為司宰,衛尉為司衛,宗正為司宗,太僕為司馭,大理為詳刑,正為大夫。鴻臚為司文,司農為司稼,太府為外府,卿並為正卿。少府監為內府監。將作監為繕工監,大匠為大監,少匠為少監。國子監為司成館,
 國子祭酒為大司成,司業為少司成,博士為宣業。都水為司津監。左、右衛府、左、右驍衛府、左、右武衛府,並除「府」字。左、右屯衛府為左右威衛,左、右領軍衛為左右戎衛,武候為金吾衛,千牛為奉宸衛,屯營為羽林軍。詹事為端尹府,門下、典書為左右春坊,左右庶子為左右中護。中允為左贊善大夫,洗馬為司經大夫,中舍人為右贊善大夫。家令寺為宮府寺,率更寺為司更寺,僕寺為馭僕寺,長官並為大夫。左、右衛率府為典戎衛,左、右宗衛率府為司
 御衛,左右虞候率府為清道衛,監門率府為崇掖衛,內率府為奉裕衛。七日,又制廢尚書令,改起居郎為左史,起居舍人為右史,著作佐郎為司文郎,太史丞為秘閣郎,左右千牛為奉宸,司議郎為左司議郎,太子舍人為右司議郎。典膳、藥藏、內直監、宮門大夫,並改為郎。太子千牛為奉裕。



 總章二年置司列、司戎少常伯各兩員。咸亨元年十二月詔:「龍朔二年新改尚書省百司及僕射已下官名,並依舊。其東宮十率府,有異上臺諸衛,各宜
 依舊為率府。其左司議郎除「左」字。其左、右金吾、左、右威衛,依新改」。永淳元年七月,置州別駕。



 光宅元年九月,改尚書省為文昌臺,左、右僕射為文昌左、右相。吏部為天官,戶部為地官,禮部為春官,兵部為夏官,刑部為秋官,工部為冬官。門下省為鸞臺,中書省為鳳閣,侍中為納言,中書令為內史。太常為司禮,鴻臚為司賓,宗正為司屬,光祿為司膳,太府為司府,太僕為司僕,衛尉為司衛,大理為司刑。司農依舊。左、右驍衛為左右威衛,左、右武
 衛為左、右鷹揚衛,左、右威衛為左右豹衛,左、右領軍衛為左右玉鈴衛。左、右金吾衛依舊。御史臺改為左肅政臺,專知京百官及監諸軍旅,並承詔出使。更置右肅政臺,專知諸州案察。



 垂拱元年二月,改黃門侍郎為鸞臺侍郎,文昌都省為都臺,主爵為司封,秘書省為麟臺,內侍省為司宮臺,少府監為尚方監。其左、右尚方兩署除「方」字。將作監為營繕監,國子監為成均監,都水監為水衡監。其詹事府為宮尹府,詹事為太尹,少詹事為少尹。
 左、右內率府為左右奉裕率府,千牛為左右奉裕,左、右監門率府為左右控鶴禁率府,諸衛鎧曹改為胄曹,司膳寺肴藏署改為珍羞署。十月,增置天官侍郎二員。又置左、右補闕、拾遣各二員。三年,加秋官侍郎一員。



 永昌元年,置左、右司員外郎各一員。天授二年,增置左、右補闕、拾遣各三員,通滿五員。長壽二年,增夏官侍郎三員。大足元年,加營繕少匠一員,左右羽林衛各增置將軍一員。洛、雍、並、荊、揚、益六州,置左、右司馬各一員。長安三
 年,增置司勛員外郎一員,地官依舊置侍郎一員,洛、並及三大都督府司馬宜依舊置一員。神龍元年二月,臺閣官名,並依永淳已前故事。廢左、右司員外郎。左右千牛衛各置大將軍一員。東都置太廟官吏,增置太常、大理少卿各一員。二年,又置員外官凡二千餘人。超授閹官七品已上員外者,又千餘人。十二月,復置左右司員外郎各一員。景雲二年,復置太子左、右諭德、太子左、右贊善大夫各兩員。雍、洛及大都督府長史加為三品階,
 別駕致敬,依前。太極元年,光祿、大理、鴻臚、太府、衛尉、宗正,各增置少卿一員。秘書少監、國子司業、少府少監、將作少匠、左右臺中丞,各增置一員。雍、洛二州及益、並、荊、揚四大都督府,各增置司馬一員,分為左、右司馬。



 開元元年十二月,改尚書左右僕射為左右丞相,中書省為紫微省,門下省為黃門省,侍中為監。雍州為京兆府,洛州為河南府。長史為尹,司馬為少尹,錄事參軍為司錄參軍,餘司改司為曹。五年九月,紫微省依舊為中書省,
 黃門省為門下省,黃門監為侍中。二十四年九月,改主爵為司封。天寶元年二月,侍中改為左相,中書令改為右相,左、右丞相依舊為僕射,黃門侍郎為門下侍郎。改州為郡,刺史為太守。十一載正月,改吏部為文部,兵部為武部,刑部為憲部。其行內諸司有部者並改。改駕部為司駕,改庫部為司庫,金部為司金,倉部為司儲,比部為司計,祠部為司禋,膳部為司膳,虞部為司虞,水部為司水。將作大匠為監,少匠為少監。至德二載十二月敕:「
 近日所改百司額及郡名並官名,一切依故事。」於是侍中、中書令、兵吏部等並仍舊。罷郡為州,復以太守為刺史。



 正第一品



 太師、太傅、太保、太尉、司徒、司空、已上職事官。王。爵。《武德令》有天策上將,九年省。



 從第一品



 開府儀同三司、文散官。開府儀同三司及特進不帶職事官者,朝參祿俸並同職事,仍隸吏部
 也。太子太師、太子太傅、太子太保、已上職事官。



 驃騎大將軍、武散官。嗣王、郡王、國公。爵。



 正第二品



 特進、文散官。輔國大將軍、武散官。開國郡公、爵。《武德令》唯有公、侯、伯、子、男,貞觀十一年加開國之稱也。上柱國。



 勛官。《武德令》有尚書令,龍朔二年省。自是正第二品無職事官。



 從第二品



 尚書左右僕射、太子少師、太子少傅、太子少保、京兆河南太原等七府牧、大都督、揚、幽、潞、陜、靈。大都護、單于、安西,已上職事官。



 光祿大夫、文散官,鎮軍大將軍、武散官。開國縣公、爵。柱國。勛官。



 正第三品



 侍中、中書令、吏部尚書、舊班在左相上,《開元令》移在下。門下侍郎、中書侍郎、舊班正四品上,大歷二年升。左右衛、左右驍衛、左右武衛、左右威衛、左右領軍衛、左右金吾衛、左右監門衛、左右羽林軍、左右龍武、左右英武六軍大將軍、左右千牛衛大將軍、自左右衛已下,並為武職事官。戶部、禮部、兵部、刑部、工部尚書、《武德令》,禮部次吏部,兵部次之,民部次之。貞觀年改以民部次禮部,兵部次之。則天初又改以戶部次吏部,禮部次之,兵
 部次之。



 太子賓客、舊兼職無品,《開元前令》定入官品也。太常卿、宗正卿、天寶初升入正三品也。太子詹事、左右散騎常侍、舊班從三品,廣德年升。



 內侍監、唐初舊制,內侍省無三品官,內侍四員,秩四品。天寶十三年十二月,玄宗以中官高力士、袁思藝承恩遇,特置內侍監兩員,秩三品,以授之。



 中都督、上都護、已上除八大將軍,並為文職事官。金紫光祿大夫、文散官。冠軍大將軍、武散官。懷化大將軍、顯慶三年置,以授初附首領,仍隸諸衛也。



 上護軍。勛官。



 從第三品



 御史大夫、舊班在秘書監九卿下,《開元令》移在上。秘書監、光祿、衛尉、太僕、
 大理、鴻臚、司農、太府卿、國子祭酒、殿中監、少府監、將作監、諸衛羽林,入正三品。千牛龍武將軍、下都督、上州刺史、京兆河南太原等七尹、舊雍、洛長史從四品上,景雲二年加秩為從三品也。五大都督府長史、舊從四品上,景雲二年加秩為從三品。大都護府副都護、舊正四品上,《開元令》加入從三品。



 親王傅、巳上並職事官。諸衛羽林、千牛龍武將軍為武,餘並為文。銀青光祿大夫、文散官。開國侯、爵。雲麾將軍、武散官。



 歸德將軍、顯慶三年置,以授初附首領,仍隸諸衛也。護軍。勛官。《武德令》有天策上將府長史、司馬,九年省也。



 正第四品上階



 門下侍郎、中書侍郎、舊正四品下階,《開元令》加入上階也。尚書左丞、永昌元年進為正三品,如意元年復舊,吏部侍郎、武德七年省諸司侍郎,吏部郎中為正四品上。貞觀三年復置侍郎,其吏部郎中復舊為五品下。



 太常少卿、太子左庶子、太子少詹事、太子左右衛、左右司御、左右清道、左右內率、左右監門率府率、中州刺史、軍器監、武德初為正三品,七年省,八年復置,九年又省,十年復置北都軍器監。



 上都護府副都護、上府折沖都尉、《武德令》統軍正四品下,後改為折沖都尉。《垂拱令》始分為上中下府,改定官品。自此已上職事官。率及折沖為武,餘並為文也。



 正議大夫、文散官也。開國伯、爵。忠武將軍、武散官。上輕車都尉。勛官。



 正第四品下階



 尚書右丞、永昌元年進為從三品,如意元年復舊。諸司侍郎、太子右庶子、左右諭德、左右千牛衛、左右監門衛中郎將、親勛翊衛羽林中郎將、下州刺史、《武德令》,中州刺史,正四品,下州刺史,從四品上。《貞觀令》,一切為下州,加入正四品下。自此已上職事官。中郎將為武,餘並為文也。



 通議大夫、文散官。壯武將軍。武散官。



 從第四品上階



 秘書少監、八寺少卿、殿中少監、太子左右衛、司御、清道、
 內率、監門副率、太子親勛翊衛中郎將、太子家令、太子率更令、太子僕、內侍、大都護親王府長史、已上職事官。府率、中郎將為武,餘並為文。太中大夫、文散官。宜威將軍、武散官。輕車都尉。勛官。



 從第四品下階



 國子司業、少府少監、將作少匠、京兆河南太原府少尹、大都督府大都護府親王府司馬、上州別駕、已上職事文官。《武德令》,上州別駕正五品上。二十三年為長史,前上元年,復置別駕,定入從四品也。



 中府折沖都尉、武職事官。中大夫、文散官。明威將軍。武散官。《武德令》有天策上將府從事中郎,九年省。



 正第五品上階



 諫議大夫、御史中丞、《武德令》,從五品上。《貞觀令》,加入正五品上,五年又加入四品。如意元年復舊也。國子博士、給事中、中書舍人、太子中允、太子左右贊善大夫、都水使者、萬年長安河南洛陽太原晉陽奉先會昌縣令、武德元年,敕萬年、長安令為正五上。七年定令,改為從五品。貞觀初復舊也。親勛翊衛羽林郎將、中都督府上都護府長史、親王府諮議參軍事、《武德令》,正五品下也。軍器少監、太史少監、親王府典軍、已上職事官。郎將、典軍為武,餘並為文。《永徽令》,親王典軍從四品下。《垂拱令》改入五品也。



 中散大夫、
 文散官。開國子、爵。定遠將軍、武散官。上騎都尉。勛官。



 正第五品下階



 太子中舍人、尚食尚藥奉御、太子親勛翊衛郎將、內常侍、中都督上都護府司馬、中州別駕、下府折沖都尉、已上職事官。郎將、折沖為武,餘並為文也。



 朝議大夫、文散官。寧遠將軍。武散官。《武德令》有天策上將軍諮祭酒,九年省。



 從第五品上階



 尚書左右諸司郎中、《武德令》,吏部郎中正四品上,諸司郎中正五品上。貞觀二年,並改為
 從五品上也。秘書丞、《武德令》,正五品上。《永徽令》改也。



 著作郎、太子洗馬、殿中丞、尚衣尚舍尚乘尚輦奉御、獻陵昭陵恭陵橋陵八陵令、《武德》,諸陵令從七品下,永徽二年加獻、昭二陵令,為從五品。已後諸陵並相承依獻、昭二陵也。



 親王府副典軍、下都督府上州長史、下州別駕、已上職事官。典軍為武,餘並為文也。朝請大夫、文散官。開國男、爵。游擊將軍、武散官。騎都尉。勛官。舊有太公廟令,武德年七品下,永徽二年加從五品上,開元二十四年省也。



 從第五品下階



 大理正、太常丞、太史令、內給事太子典內、舊正六品上,《開元令》改。
 下都督府上州司馬、《武德令》,上治中正五品下。貞觀初改。



 親王友、《武德令》,正五品下也。宮苑總監、上牧監、上府果毅都尉、已上職事官。果毅為武散,餘並為文。駙馬都尉、奉車都尉、並武散官。駙馬自近代已來,唯尚公主者授之。奉車,有唐已來無其人。朝散大夫、文散官。游擊將軍。武散官。《武德令》有天策上將府主簿、記室、參軍,九年省。《神龍令》有庫谷、斜谷監也。



 正第六品上階



 太學博士、《武德令》,從六品上,貞觀年改。太子詹事府丞、太子司議郎、太子舍人、中郡長史、《武德令》,中州別駕從五品上,貞觀年改也。



 太子典膳藥
 藏郎、京兆河南太原府諸縣令、武德元年敕,雍州諸縣令階從五品上,七年定令改。親王府掾屬、《武德令》,從五品下也。



 武庫中尚署令、《武德令》依上署令,從七品下,太極年改武庫令階,開元年改中尚令階。諸衛左右司階、中府果毅都尉、鎮軍兵滿二萬人已上司馬、已上職事官。司階、果毅為武,餘並為文也。親勛翊衛校衛、衛官。朝議郎、文散官。昭武校尉、武散官。驍騎尉。勛官。



 正第六品下階



 千牛備身左右、衛官已上、王公已下高品子孫起家為之。太子文學、下州長史、武德中,下州別駕,正六品,貞觀二十三年,改為長史丞。永淳元年,諸州置別駕官。天寶八載停別駕,下郡
 置長史。後上元二年,諸州置別駕,不廢下府長史也。



 中州司馬、《武德令》,中州治中,從五品下,《貞觀令》改。內謁者監、中牧監、上牧副監、已上文職事官。上鎮將、武職事官。《武德令》,從四品下也。



 承議郎、文散官。昭武副尉。武散官。《武德令》有天策上將府諸曹參軍事,九年省也。



 從第六品上階



 起居郎、起居舍人、尚書諸司員外郎、《武德令》,吏部員外郎正六品上,諸司員外郎正六品下。貞觀二年改。八寺丞、大理司直、國子助教、《武德令》,從七品上。城門符寶郎、通事舍人、秘書郎、《武德令》,正七品上。著作佐郎、《武
 德令》,正七品下。侍御醫、《武德》、《乾封令》,正七品上。《神龍令》,從六品下。開元改。



 諸衛羽林長史、兩京市署令、武德四年進為從五品上,七年定令,復舊也。下州司馬、《武德令》,中下州治中,正六品下。親王文學、主簿、記室、錄事參軍、《武德令》,親王府文學已上,並正六品下也。諸州上縣令、已上交職事官。諸率府左右司階、武職事官。鎮軍兵不滿二萬人司馬、文職事官。左右監門校尉、親勛翊衛旅帥、衛官。奉議郎、文散官。振威校尉、武散官。飛騎尉。勛官。



 從第六品下階



 侍御史、舊從七品上,《垂拱令改》。少府將作國子監丞、太子內直典
 設宮門郎、太公廟令、司農寺諸園苑監、沙苑監、下牧監、宮苑總監副、互市監、中牧副監、已上文職事官。下府果毅都尉、武職事官。親王府校尉、衛官。通直郎、文散官。振威副尉。武散官。



 正第七品上階



 四門博士、詹事司直、左右千牛衛長史、尚食尚藥直長、太子左右衛司禦清道率府長史、軍器監丞、諸州中縣令、京兆河南太原府司錄參軍事、大都督大都護府錄事參軍事、親王府諸曹參軍、已上文職事官。《武德令》,親王府功曹、倉曹、戶曹、兵曹
 參軍事,從五品下;騎曹、鎧曹、田曹、士曹、水曹參軍事等,七品下也。中鎮將、武職事官。《武德令》,從五品下。太子千牛、親勛翊衛隊正副隊正、已上衛官。朝請郎、文散官。致果校尉武散官。雲騎尉。



 勛官。



 正第七品下階



 尚衣尚舍尚乘尚輦直長、太子通事舍人、內寺伯、京兆河南太原府大都督大都護府諸曹參軍、中都督上都護府錄事參軍事、諸倉諸冶司竹溫湯監、諸衛左右中候、上府別將、《武德令》,別將正五品上,後改為果毅。聖歷三年復置別將。上府長史、《武德
 令》,統軍長史正八品下也。上鎮副、《武德令》,從五品下。



 下鎮將、《武德令》,正六品下。下牧副監、已上職事官。中候、別將、鎮副、鎮將為武,餘並為文也。宣德郎、文散官。致果副尉。



 武散官。《武德令》又有天策上將府參軍事,九年省。又有鹽池鹽井鹽、諸王百司問事謁者。



 從第七品上階



 殿中侍御史、《武德》至《乾封令》,並正八品上,垂拱年改。左右補闕、太常博士、太學助教、《武德令》,從八品下也。門下錄事、中書主書、尚書都事、九寺主簿、太子詹事主簿、太子左右內率監門率府長史、太子侍醫、太子三寺丞、都水監丞、諸州中下縣令、親王
 府東西閣祭酒、《武德令》,正六品下。京縣丞、萬年、長安、河南、洛陽、奉先、會昌、太原、晉陽。下都督府上州錄事參軍、中都督上都護府諸曹參軍事、中府別將長史、中鎮副、《武德令》,正六品下。已上職事官。別將、鎮副為武,餘並為文。左右監門直長、勛衛、太子親衛、已上衛官。朝散郎、文散官。翊麾校尉、武散官。武騎尉。勛官。



 從第七品下階



 太史丞、監局同。御史臺少府將作國子監主簿、御史臺、國子監主簿、舊正八品,《垂拱令》改。掖庭令、宮闈令、上署令、郊社、太樂、鼓吹、太醫、太官、左藏令、乘黃、典
 客、上林、太倉、平準、常平、左尚、右尚、典牧。《武德令》有太廟、諸陵、典農、中尚、都水、常平。其左尚、典牧本中署,右尚本下署,開元初改之也。



 諸州下縣令、天寶五載,一切為中下縣。諸陵署丞、永徽二年加秩。舊有太廟署丞,武德為九品,永徽二年加秩,從七品上,開元年省也。司農寺諸園郤副監神龍令有諸冶副監宮苑總監丞,下都督府諸曹參軍,太子內坊丞舊正八品上開元初改親王國令舊規,流內正九品,太極年改。公主家令舊規,流內正八品,太極年改。上州諸參軍事、下府別將長史、下鎮副、《武德令》,從六品下。諸屯監、《武德令》有芳醖監,《神龍令》有漆園監。



 諸率府左右中候、鎮軍滿二萬人以上諸曹判司、已上職事官。別將、鎮副、中候為武,餘並為文也。太
 子左右監門直長、親王府旅帥、諸折沖府校尉、已上衛官。《武德令》,諸府校尉,正六品下也。宣議郎、文散官。翊麾副尉。武散官。



 正第八品上階



 監察御史、舊從八品上,《垂拱令》改。協律郎、諸衛羽林龍武軍錄事參軍事、中署令、鉤盾、右藏、職染、掌治,《武德令》有衣冠署令。



 中州錄事參軍事、太醫博士、太子典膳藥藏丞、軍器監主簿、武庫署丞舊從八品下,開元初改。兩京市署丞、上牧監丞、《武德令》,從八品下,《神龍令》有庫谷、斜谷、太陰伊陽監丞。



 鎮軍不滿二萬人以上諸曹判司、已上文職事官。翊衛、
 太子勛衛、親王府執仗執乘親事、(已上衛官。給事郎、文散官。宣節校尉。



 武散官。《武德令》有天策上將府典簽,九年省。



 正第八品下階



 奚官內僕內府局令、下署令、太卜、廩犧、珍羞、良醖、掌醢、守宮、武器、車府、司儀、崇玄、導官、中右校、左校、甄官、河渠、弩坊、甲坊。《神龍令》又有乾、楫二署令也。



 諸衛羽林龍武諸曹參軍事、中州諸司參軍事、親王府京兆河南太原府大都督大都護府參軍事、《武德令》,親王府參軍,從七品下,《雍州》行參軍,正八品上。



 尚藥局司醫、京兆河南太原府諸縣丞、太子內直宮門丞、
 太公廟丞、諸宮農圃監、互市監丞、司竹副監、司農寺諸園苑監丞、靈臺郎、已上文職事官。



 諸衛左右司戈、上戍主、已上武職事官。《武德令》有中鎮長史。備身、衛官。徵事郎、文散官。宣節副尉。武散官。



 從第八品上階



 左右拾遣、太醫署針博士、四門助教、《武德令》,以九品上。左右千牛衛錄事參軍、下州錄事參軍、《武德令》有中下州諸司參軍事。



 諸州上縣丞、中牧監丞、《武德令》,正八品上。京縣主簿、太子左右衛司禦清道率府錄事參軍、中都督府上都護府參軍、親王府
 行參軍、《武德令》,正八品上。京兆河南太原大都督府博士、《武德令》,雍州博士,從八品下。諸倉諸冶司竹溫湯監丞、《武德令》有鹽池鹽井監丞,《神龍令》有太和監丞。



 保章正、已上文職事官。太子翊衛諸府旅帥、已上衛官。《武德》、《乾封令》,諸府旅帥,正七品下。承奉郎、文散官。禦侮校尉。



 武散官。



 從第八品下階



 大理評事、律學博士、太醫署丞、醫監、太子左右春坊錄事、左右千牛衛諸曹參軍、內謁者、太子左右衛司禦清道率府諸曹參軍事、太子諸署令、掖庭宮圍局丞、太史都
 水監主簿、太史為局則省主簿。中書門下尚書都省兵吏部考功禮部主事、舊從九品上,開元二十四年改七司入八品,其省內諸司依舊。上署丞、《武德令》有芳醖監丞。



 下都督府上州參軍事、中都督府上州博士、諸州中縣丞、諸王府典簽、《武德令》,正八品下。京縣尉、親王國大司農、舊規,流內正第七品,開元初改。



 公主家丞、舊規,流內正第九品,開元初改。諸屯監丞、上關令、上府兵曹、上鎮倉曹兵曹參軍事、《武德令》有下鎮長史。挈壺正、已上文職事官。中戌主、上戌副、永府左右司戈、已上武職事官。太子備身、親王府隊正、已上衛官。承務郎、文散官。禦侮副尉。



 武散官。



 正第九品上階



 校書郎、《永徽令》加入從八品下,《垂拱令》復舊。太祝、太子左右內率監門府錄事參軍、太子內方典直、中署丞、典客署掌客、親勛翊衛府羽林兵曹參軍事、岳瀆令、諸津令下牧監丞、《武德令》,正八品下。《神龍令》有漆園丞,《開元前令》有沙苑丞。諸州中下丞、中郡博士、《武德令》,正九品下。京兆河南太原府諸縣主簿、武庫署監事、已上並文職事官。《武德令》有天策上將府錄事。其武庫監事,從九品下,太極年改也。儒林郎、文散官。仁勇校尉。武散官。



 正第九品下階



 正字、《永徽令》改入上階,《垂拱令》復舊。太子校書、《永徽令》改入上階、《垂拱令》復舊。奚官內僕內府局丞、下署丞、尚食局食醫、尚藥局醫佐、尚乘局奉乘司庫司廩、太史局司辰、典廄署主乘、太子左右內率監門率府諸曹參軍事、太子三寺主簿、詹事府錄事、龍朔年置桂坊錄事,咸亨年省。太子親勛翊府兵曹參軍事、諸州下縣丞、諸州上縣中縣主簿、中州參軍事、《武德令》,正九品上。下州博士、《武德令》,中下州博士,從九品上,下州博士,從九品下。京兆河南太原府諸縣尉、
 上牧主簿、諸宮農圃監丞、中關令、中府兵曹、親王國尉、舊規,流內正八品,開元初改。《武德令》有親王府鎮事及司閣。上關丞、《武德令》有上津尉。諸衛左右執戟、中鎮兵曹參軍、下戍主、已上職事官。執戟戍主為武,餘並為文。諸折沖府隊正、衛官。登仕郎、文散官。仁勇副尉。武散官。



 從第九品上階



 尚書諸司御史臺秘書省殿中省主事、奉禮郎、律學助教、太子正字、弘文館校書、太史司歷、太醫署醫助教、京兆河南太原府九寺少府將作監錄事、都督都護府上
 州錄事市令、宮匹總監主簿、中牧監主簿、《永徽令》有監漕。諸州中下縣主簿、上縣中縣尉、下府兵曹、已上並職事文官。文林郎、文散官。陪戎校尉。武散官。



 從第九品下階



 內侍省主事、國子監親王府錄事、太子左右春坊主事、崇文館校書、書學博士、算學博士、門下典儀、太醫署按摩咒禁博士、太卜署博士、太醫署針助教、太醫署醫正、太卜署卜正、太史局監候、親王國丞、舊規,流內正第九品,開元初改從正
 流內。掖庭局宮教博士、太子諸署丞、太子典食署丞、太子廄牧署典乘、諸監作諸監事計官、太官署監膳、太樂鼓吹署樂正、大理寺獄丞、下州參軍事、《武德令》,中下州行參軍,正九品,下州參軍,從九品上。中州下州醫博士、諸州中縣下縣尉、京縣錄事、下牧監主簿、下關令、中關丞、諸衛羽林長上、公主邑司錄事、諸津丞、、下鎮兵曹參軍、《武德令》有諸橋諸堰丞。諸率府左右執戟、已上職事官。長上、執戟為武,餘並為文。親王府隊副、諸折沖府隊副、已上衛官。將仕郎、文散官。陪戎副尉。武散官。



 流內九品三十階之內,又有視流內起居,五品至從九品。初以薩寶府、親王國官及三師、三公、開府、嗣郡王、上柱國已下護軍勛官帶職事者府官等品。開元初,一切罷之。今唯有薩寶、祅正二官而已。又有流外自勛品以至九品,以為諸司令史、贊者、典謁、亭長、掌固等品。視流外亦自勛品至九品,開元初唯留薩寶祅祝及府史,餘亦罷之。



 職事者,諸統領曹事,供命王命,上下相攝,以持庶績。近代已來,又分為文武二職,分曹置員,各理
 所掌。五品已下,舊制吏部尚書進用。自隋已後,則中書門下知政事官訪擇聞奏,然後下制授之。三品已上,德高委重者,亦有臨軒冊授。自神龍之後,冊禮廢而不用,朝遷命官,制敕而已。六品已上,吏部選擬錄奏,書旨授之。



 有唐已來,出身入仕者,著令有秀才、明經、進士、明法、書算。其次以流外入流。若以門資入仕,則先授親勛翊衛,六番隨文武簡入選例。又有齋郎、品子、勛官及五等封爵、屯官之屬,亦有番第,許同揀選。天寶三載,又置崇
 玄學,習《道德》等經,同明經例。自餘或臨時聽敕,不可盡載。其秀才,有唐已來無其人。



 職事官資,則清濁區分,以次補授。又以三品已上官,及門下中書侍郎、尚書左右丞、諸司侍郎、太常少卿太子少詹事、左右庶子、秘書少監、國子司業為清望官。太子左右諭德、左右衛左右千牛衛中郎將、太子左右率府左右內率府率及副、太子左右衛率府中朗將、已上四品。諫議大夫、御史中丞、給事中、中書舍人、太子中允、中舍人、左右贊善大夫、洗馬、國子
 博士、尚書諸司郎中、秘書丞、著作郎、太常丞、左右衛郎將、左右衛率府郎將、已上五品。起居郎、起居舍人、太子司議郎、尚書諸司員外郎、太子舍人、侍御史、秘書郎、著作佐郎、太學博士、詹事丞、太子文學、國子助教、已上六品。左右補闕、殿中侍御史、太常博士、四門博士、詹事司直、太學助教、已上七品。左右拾遺、監察御史、四門助教已上八品。



 為清官。自外各以資次遷授。開元中,裴光庭為吏部尚書,始用循資格以注擬六品已下選人。其後每年雖小有移改,然
 相承至今用之。



 武散官,舊謂之散位,不理職務,加官而已。後魏及梁,皆以散號將軍記其本階,自隋改用開府儀同三司已下。貞觀年,又分文武,入仕者皆帶散位,謂之本品。



 以門資出身者,諸嗣王郡王出身從四品下,親王諸子封郡公者從五品上,國公正六品上,郡公正六品下,縣公從六品上,侯正七品上,伯正七品下,子從七品上,男從七品下。皇帝緦麻以上親、皇太后周親出身六品上。皇太后大功親、皇後周親從六品上。皇帝袒免
 親、皇太后小功緦麻親、皇后大功親正七品上。皇后小功緦麻親、皇太子妃周親從七品上。其外戚各依服屬降宗親二階敘。諸娶郡主者出身六品上,娶縣主者正七品上,郡主子出身從七品上。縣主子從八品上。一品子正七品上,二品子正七品下,三品子從七品上,從三品子從七品下,正四品子正八品上,從四品子正八品下,正五品子從八品上,從五品及國公子從八品下。三品以上廕曾孫,五品以上廕孫。孫降子一等,曾孫降孫
 一等。



 諸秀才出身,上上第,正八品上;上中第,正八品下;上下第,從九品上。明經出身,上上第,從八品下;上中第,從九品上。進士、明法出身,甲第,從九品上;乙第,從九品下。若通二經已外,每一經加一等。



 勛官預文武選者,上柱國正六品上敘,以下遞降一階。凡入仕之後,遷代則以四考為限。四考中中,進年勞一階敘。每一考中上,進一階;一考上下,進二階。五品已上非恩制所加,更無進之令。



 自武德至乾封,未有泛階之恩。應入三品者,皆以
 恩舊特拜,入五品者多依選敘,計階至朝散大夫已上,奏取進止,每年量多少進敘。餘並依本品授官。若滿三計至,即一切聽入。至乾封元年,文武普加二階。永淳元年二月敕:「文武官累積勞效,計至五品。一計至者,多未甄擢。再計至者,隨例必升,賢愚一貫。自今已後,一計至已上,有在官清慎,狀跡灼然,材堪應務者,所司具狀錄奏,當與進階。若公正無聞,循默自守,及未經任州縣官者,雖頻經計至,不在加階之限。即為恆例。」弘道元年,又普
 加一階。乃有九品職事及三衛階高者,並入五品。則天朝,泛階漸多,始令仕經八考,職事六品者許入。萬歲通天元年敕:「自今已後,文武官加階應入五品者,並取出身,已歷十二考已上,進階之時,見居六品官。其應入三品人,出身已二十五考以上,進階見居三品官。」無幾,入五品又加至十六考。神功元年制「勛官、品子、流外國官出身,不得任清資要官。應入三品,不得進階」。開元已來,伎術者經二十考,三省都事及主事、錄事十八考,亦聽
 敘。吏部檢歷任階考,判成錄奏。每制之日,應入三品五品者,皆令人參趁。或是遠方牧宰、諸司閑職,齎持金帛贈遣主典,知加階令史,乃有受納萬數者。臺省要職,以加位為榮,亦有遣主典錢帛者。



 舊例,開府及特進,雖不職事,皆給俸祿,預朝會,行立在於本品之次。光祿大夫已下,朝散大夫已上,衣服依本品,無祿俸,不預朝會。朝議郎已下,黃衣執笏,於吏部分番上下承使及親驅使,甚為猥賤。每當上之時,至有為主事令史守扃鑰執
 鞭帽者。兩番已上,則隨番許簡,通時務者始令參選。一登職事已後,雖官有代滿,即不復番上。



 勛官者,出於周、齊交戰之際。本以酬戰士,其後漸及朝流。階爵之外,更為節級。周置上開府儀同三司、開府儀同三司、上儀同三司、儀同三司等十一號。隋文帝因周之舊,更增損之。有上柱國、柱國、上大將軍、大將軍、上開府儀同三司、開府儀同三司,上儀同三司、儀同三司,大都督、帥都督、都督,起正二品,至七品,總十一等,用賞勛勞。煬帝又改為
 左光祿大夫、右光祿大夫、金紫光祿大夫、銀青光祿大夫、正議大夫、朝請大夫、朝散大夫、建節奮武尉、宣惠尉十一等,以代都督已上。又增置綏德、懷仁、守義、奉誠、立信等五尉,以至從九品。武德初,雜用隋制,至七年頒令,定用上柱國、柱國、上大將軍、大將軍、上輕車都尉、輕車都尉、上騎都尉、騎都尉、驍騎尉、飛騎尉、雲騎尉、武騎尉,凡十二等,起正二品,至從七品。貞觀十一年,改上大將軍為上護軍,大將軍為護軍,自外不改,行之至今。



 永微
 已後,以國初勛名與散官名同,年月既久,漸相錯亂。咸亨五年三月,更下詔申明,各以類相比。武德初光祿大夫比今日上柱國,左光祿大夫比柱國,右光祿大夫及上大將軍比上護軍,金紫光祿大夫及將軍比護軍,銀青光祿大夫及上開府比上輕車都尉,正議大夫及開府比輕車都尉,通議大夫及上儀同三司比上騎都尉,朝請大夫及儀同比騎都尉,
 授勛者動盈萬計。每年納課,亦分番於兵部及本郡,當上省司。又分支諸曹,身應役使,有類僮僕,據令乃與公卿齊班,論實在於胥吏之下。蓋以其猥多,又出自兵卒,所以然也。



 武德初,以諸道軍各事繁,分置行臺尚司省。其陜東道大行臺尚書省,令一人,正第二品。掌管內軍人,總判省事。僕射一人,從第二品,三品任置。



 掌貳令事。左丞一人,正第四品下。右丞一人,正第四品下。掌分司糾正省內。都事一人,從第七品上。主事四人,從第九品上,諸司主事並同。



 並掌同京省。兵部
 尚書一人,正第四品,諸尚書並同。兼掌吏部事。司勛郎中一人,正第五品上,諸郎中並同。主事一人。考功郎中一人,主事一人。兵部郎中一人,主事二人。駕部郎中一人,主事二人。民部尚書一人,兼掌禮部事。禮部郎中一人,主事一人。膳部郎中一人,主事一人。度支郎中一人,主事二人。倉部郎中一人,主事二人。工部郎中一人,主事一人。屯田郎中一人,主事一人。每郎中兼京省二
 司。



 各有令史、書令史及掌固,並流外。食貨監一人,正第八品下,諸監同。掌膳羞、財物、賓客、鋪設、音樂、醫藥事。丞二人。正第九品下,諸監丞同。



 農圃監一人,掌倉廩、園圃、柴炭、芻槁、運漕之事。丞四人。武器監一人,掌兵仗、廄牧之事。丞二人。百工監一人,掌舟車及營造雜作之事。丞四人。



 各有錄事及府史、典事、掌固等,並流外。諸道行臺尚書省,益州道、襄州道、東南道、河東道、河北道。令一人,從第二品。掌同陜東道大行臺。僕射一人,正第三品,左右任置。丞一人,左右任置。左丞從四品上,右丞從四品下。都事二人,正第八上。主事二人。兵部尚書一人,從第三品,諸尚
 書同。



 兼掌吏部、禮部事。考功郎中一人,從第五品上,諸郎中並同。主事二人,從第九品下,諸主事同。膳部郎中一人,主事二人。兵部郎中二人,主事二人。民部尚書一人,兼掌刑部、工部。倉部郎中二人,主事二人。刑部郎中一人,主事二人。屯田郎中一人,主事二人。



 每郎中兼掌京省三司,各有令史、書令史、掌固,並流外也。食貨監一人,從八品上,武器監同。兼掌農圃監事,丞一人。兼掌百工監事,丞二人。



 兩監各有錄事、府史、典事、掌固等,並流外。



 時秦王、齊王府官之外,又各置左右六護軍府及左右親事帳內府。其左一右一護
 軍府護軍各一人,正第四品下。掌率統軍已下侍衛陪從。副護軍各二人,從四品下。長史各一人,從七品下。錄事參軍各一人,從八品,有錄事及府史,並流外。倉曹參軍事各一人,兵曹參軍事各一人,鎧曹參軍事各一人。



 並正九品下,各有府史,並流外。統軍各五人,別將各十人,分掌領親勛衛及外軍。左二右二護軍府、左三右三護軍府,各減統軍三人,別將六人。餘職員同左一右一府。其左右親事府統軍各一人,正四品下。掌率左右別將、侍衛陪從。長史一人,正八品下。錄事參軍事各一人,正九
 品上,有錄事及府史,並流外。兵曹參軍事各一人,鎧曹參軍事各一人。



 並正九品下,各有府史,並流外。左別將各一人,右別將各一人,正五品下。掌率親事以上侍衛陪從。其帳內府職員品秩,與統軍府同。又有庫直及驅咥直,庫直隸親事府,驅咥直隸帳內府。各於左右內選才堪者,量事置之。



 武德四年,太宗平洛陽之後,又置天策上將府官員。天策上將一人,掌國之征討,總判府事。長史、司馬各一人,從事中郎二人,並掌通判府事。軍諮祭酒二人,謀軍事,贊相禮儀,宴接賓客。典簽四人,
 掌宣傳導引之事。主簿二人,掌省復教命。錄事二人,記室參軍事二人,掌書疏表啟,宣行教命。功曹參軍事二人,掌官員假使、儀式、醫藥、選舉、考課、祿恤、鋪設等事。倉曹參軍二人,掌糧廩、公廨、田園、廚膳、過所等事。兵曹參軍事二人,掌兵士簿帳、差點等事。騎曹參軍事二人,掌馬驢雜畜簿帳及牧養支料草粟等事。鎧曹參軍事二人,掌戎仗之事。士曹參軍事二人,掌營造及罪罰之事。



 六曹並有令史。書令史。參軍事六人,掌出使及雜檢校之事。其陜
 東道大行臺尚書令及天策上將,太宗在籓為之。及升儲,並省之。山東道行臺,武德五年省。餘道九年省。



\end{pinyinscope}