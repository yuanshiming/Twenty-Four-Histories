\article{卷一百九十六}

\begin{pinyinscope}

 ○徐文遠陸德明曹憲許淹李善公孫羅附歐陽詢子通硃子奢張士衡賈公彥李玄植附張後胤蓋文達宗人文
 懿谷那律蕭德言許叔牙子子儒敬播劉伯莊子之宏秦景通羅道琮



 古稱儒學家者流,本出於司徒之官,可以正君臣,明貴賤,美教化,移風俗,莫若於此焉。故前古哲王,咸用儒術之士;漢家宰相,無不精通一經。朝廷若有疑事,皆引經決定,由是人識禮教,理致升平。近代重文輕儒,或參以法律,儒道既喪,淳風大衰,故近理國多劣於前古。自隋
 氏道消,海內版蕩,彞倫攸篸,戎馬生郊,先代之舊章,往聖之遺訓,掃地盡矣!



 及高祖建義太原,初定京邑,雖得之馬上,而頗好儒臣。以義寧三年五月,初令國子學置生七十二員,取三品已上子孫;太學置生一百四十員,取五品已上子孫;四門學生一百三十員,取七品已上子孫。上郡學置生六十員,中郡五十員,下郡四十員。上縣學並四十員,中縣三十員,下縣二十員。武德元年,詔皇族子孫及功臣子弟,於秘書外省別立小學。二年,詔
 曰:



 盛德必祀,義存方策,達人命世,流慶後昆。建國君人,弘風闡教,崇賢彰善,莫尚於茲。自八卦初陳,九疇攸敘,徽章互垂,節文不備。爰始姬旦,匡翊周邦,創設禮經,尤明典憲。啟生人之耳目,窮法度之本源,化起《二南》,業隆八百;豐功茂德,冠於終古。暨乎王道既衰,頌聲不作,諸侯力爭,禮樂陵遲。粵若宣父,天資睿哲;經綸齊、魯之內,揖讓洙、泗之間;綜理遺文,弘宣舊制。四科之教,歷代不刊;三千之文,風流無歇。



 惟茲二聖,道著群生,守祀不修,
 明褒尚闕。朕君臨區宇,興化崇儒,永言先達,情深紹嗣。宜令有司於國子學立周公、孔子廟各一所,四時致祭。仍博求其後,具以名聞,詳考所宜,當加爵土。是以學者慕向,儒教聿興。



 至三年,太宗討平東夏,海內無事,乃銳意經籍,於秦府開文學館。廣引文學之士,下詔以府屬杜如晦等十八人為學士,給五品珍膳,分為三番更直,宿於閣下。



 及即位,又於正殿之左,置弘文學館,精選天下文儒之士虞世南、褚亮、姚思廉等,各以本官兼署學
 士,令更日宿直。聽朝之暇,引入內殿,講論經義,商略政事,或至夜分乃罷。又召勛賢三品已上子孫,為弘文館學士。



 貞觀二年,停以周公為先聖,始立孔子廟堂於國學,以宣父為先聖,顏子為先師。大徵天下儒士,以為學官。數幸國學,令祭酒、博士講論。畢,賜以束帛。學生能通一大經已上,咸得署吏。又於國學增築學舍一千二百間,太學、四門博士亦增置生員,其書算合置博士、學生,以備藝文,凡三千二百六十員。其玄武門屯營飛騎,亦
 給博士,授以經業;有能通經者,聽之貢舉。是時四方儒士,多抱負典籍,雲會京師。俄而高麗及百濟、新羅、高昌、吐蕃等諸國酋長,亦遣子弟請入於國學之內。鼓篋而升講筵者,八千餘人。濟濟洋洋焉,儒學之盛,古昔未之有也。



 太宗又以經籍去聖久遠,文字多訛謬,詔前中書侍郎顏師古考定《五經》,頒於天下,命學者習焉。又以儒學多門,章句繁雜,詔國子祭酒孔穎達與諸儒撰定《五經》義疏,凡一百七十卷,名曰《五經正義》,令天下傳習。



 十
 四年,詔曰:「梁皇侃、褚仲都,周熊安生、沈重,陳沈文阿、周弘正、張譏,隋何妥、劉炫等,並前代名儒,經術可紀。加以所在學徒,多行其疏,宜加優異,以勸後生。可訪其子孫見在者,錄名奏聞,當加引擢。」



 二十一年,又詔曰:「左丘明、卜子夏、公羊高、穀梁赤、伏勝、高堂生、戴聖、毛萇、孔安國、劉向、鄭眾、杜子春、馬融、盧植、鄭玄、服虔、何休、王肅、王弼、杜元凱、範寧等二十一人,並用其書,垂於國胄。既行其道,理合褒崇。自今有事太學,可與顏子俱配享孔子廟
 堂。」其尊重儒道如此。



 高宗嗣位,政教漸衰,薄於儒術,尤重文吏。於是醇醲日去,畢競日彰,猶火銷膏而莫之覺也。及則天稱制,以權道臨下,不吝官爵,取悅當時。其國子祭酒,多授諸王及駙馬都尉,準貞觀舊事。祭酒孔穎達等赴上日,皆講《五經》題。至是,諸王與駙馬赴上,唯判祥瑞按三道而已。至於博士、助教,唯有學官之名,多非儒雅之實。是時復將親祠明堂及南郊,又拜洛,封嵩岳,將取弘文國子生充齊郎行事,皆令出身放選,前後不
 可勝數。因是生徒不復以經學為意,唯茍希僥幸。二十年間,學校頓時隳廢矣。



 玄宗在東宮,親幸太學,大開講論,學官生徒,各賜束帛。及即位,數詔州縣及百官薦舉經通之士。又置集賢院,招集學者校選,募儒士及博涉著實之流。以為《儒學篇》。



 徐文遠,洛州偃師人,陳司空孝嗣玄孫,其先自東海徙家焉。父徹,梁秘書郎,尚元帝女安昌公主而生文遠。屬江陵陷,被虜於長安,家貧無以自給。其兄休,鬻書為事,
 文遠日閱書於肆,博覽《五經》,尤精《春秋左氏傳》。時有大儒沈重講於太學,聽者常千餘人。文遠就質問,數日便去。或問曰:「何辭去之速?」答曰:「觀其所說,悉是紙上語耳,僕皆先已誦得之。至於奧賾之境,翻似未見。」有以其言告重者,重呼與議論,十餘反,重甚嘆服之。



 文遠方正純厚,有儒者風。竇威、楊玄感、李密皆從其受學。開皇中,累遷太學博士。詔令往並州,為漢王諒講《孝經》、《禮記》。及諒反,除名。大業初,禮部侍郎許善心舉文遠與包愷、褚徽、
 陸德明、魯達為學官,遂擢授文遠國子博士,愷等並為太學博士。時人稱文遠之《左氏》、褚徽之《禮》、魯達之《詩》、陸德明之《易》,皆為一時之最。文遠所講釋,多立新義,先儒異論,皆定其是非,然後詰駁諸家,又出己意,博而且辨,聽者忘倦。



 後越王侗署為國子祭酒。時洛陽饑饉,文遠出城樵採,為李密軍所執。密令文遠南面坐,備弟子禮北面拜之。文遠曰:「老夫疇昔之日,幸以先王之道,仰授將軍。時經興替,倏焉已久。今將軍屬風雲之際,為義眾
 所歸,權鎮萬物,威加四海,猶能屈體弘尊師之義,此將軍之德也,老夫之幸也!既荷茲厚禮,安不盡言乎!但未審將軍意耳!欲為伊、霍繼絕扶傾,雖遲暮,猶願盡力;若為莽、卓乘危迫險,則老夫耄矣,無能為也。」密頓首曰:「昨奉朝命,垂拜上公,冀竭庸虛,匡奉國難。所以未朝見者,不測城內人情。且欲先征化及,報復冤恥,立功贖罪,然後凱旋,入拜天闕。此密之本意,惟先生教之。」文遠曰:「將軍名臣之子,累顯忠節,前受誤於玄感,遂乃暫墜家聲。
 行迷未遠,而回車復路,終於忠孝,用康家國,天下之人,是所望於將軍也。」密又頓首曰:「敬聞命矣,請奉以周旋。」



 及徵化及還,而王世充已殺元文都等,權兵專制。密又問計於文遠,答曰:「王世充亦門人也,頗得識之。是人殘忍,意又褊促,既乘此勢,必有異圖。將軍前計為不諧矣,非破王世充,不可朝覲。」密曰:「嘗謂先生儒者,不學軍旅之事,今籌大計,殊有明略。」



 及密敗,復入東都,王世充給其廩食,而文遠盡敬,見之先拜。或問曰:「聞君踞見李密,
 而敬王公,何也?」答曰:「李密,君子也,能受酈生之揖;王公,小人也,有殺故人之義。相時而動,豈不然歟!」後王世充僭號,復以為國子博士。因出樵採,為羅士信獲之,送於京師,復授國子博士。



 武德六年,高祖幸國學,觀釋奠,遣文遠發《春秋》題,諸儒設難蜂起,隨方占對,皆莫能屈。封東莞縣男。年七十四,卒官。撰《左傳音》三卷、《義疏》六十卷。孫有功,自有傳。



 陸德明,蘇州吳人也。初受學於周弘正,善言玄理。陳大
 建中,太子征四方名儒,講於承先殿。德明年始弱冠,往參焉。國子祭酒徐克開講,恃貴縱辨,眾莫敢當;德明獨與抗對,合朝賞嘆。解褐始興王國左常侍,遷國子助教。陳亡,歸鄉里。隋煬帝嗣位,以為秘書學士。大業中,廣召經明之士,四方至者甚眾。遣德明與魯達、孔褒俱會門下省,共相交難,無出其右者。授國子助教。王世充僭號,封其子為漢王,署德明為師,就其家,將行束脩之禮。德明恥之,因服巴豆散,臥東壁下。王世充子入,跪床前,對之
 遺痢,竟不與語。遂移病於成皋,杜絕人事。



 王世充平,太宗徵為秦府文學館學士,命中山王承乾從其受業。尋補太學博士。後高祖親臨釋奠,時徐文遠講《孝經》,沙門惠乘講《波若經》,道士劉進喜講《老子》,德明難此三人,各因宗指,隨端立義,眾皆為之屈。高祖善之,賜帛五十匹。



 貞觀初,拜國子博士,封吳縣男。尋卒。撰《經典釋文》三十卷、《老子疏》十五卷、《易疏》二十卷,並行於世。太宗後嘗閱德明《經典釋文》,甚嘉之,賜其家束帛二百段。



 子敦信,龍朔
 中官至左侍極,同東西臺三品。



 曹憲,揚州江都人也。仕隋為秘書學士。每聚徒教授,諸生數百人。當時公卿已下,亦多從之受業。憲又精諸家文字之書,自漢代杜林、衛宏之後,古文泯絕,由憲,此學復興。



 大業中,煬帝令與諸學者撰《桂苑珠叢》一百卷,時人稱其該博。憲又訓注張揖所撰《博雅》,分為十卷,煬帝令藏於秘閣。



 貞觀中,揚州長史李襲譽表薦之,太宗徵為弘文館學士。以年老不仕,乃遣使就家拜朝散大夫,
 學者榮之。



 太宗又嘗讀書有難字,字書所闕者,錄以問憲,憲皆為之音訓及引證明白,太宗甚奇之。年一百五歲卒。所撰《文選音義》,甚為當時所重。初,江、淮間為《文選》學者,本之於憲,又有許淹、李善、公孫羅復相繼以《文選》教授,由是其學大興於代。



 許淹者,潤州句容人也。少出家為僧,後又還俗。博物洽聞,尤精詁訓。撰《文選音》十卷。



 李善者,揚州江都人。方雅清勁,有士君子之風。明慶中,
 累補太子內率府錄事參軍、崇賢館直學士,兼沛王侍讀。嘗注解《文選》,分為六十卷,表上之。賜絹一百二十匹,詔藏於秘閣。除潞王府記室參軍,轉秘書郎。乾封中,出為經城令。坐與賀蘭敏之周密,配流姚州。後遇赦得還,以教授為業,諸生多自遠方而至。又撰《漢書辯惑》三十卷。載初元年卒。子邕,亦知名。



 公孫羅,江都人也。歷沛王府參軍,無錫縣丞。撰《文選音義》十卷,行於代。



 歐陽詢,潭州臨湘人,陳大司空頠之孫也。父紇,陳廣州刺史,以謀反誅。詢當從坐,僅而獲免。陳尚書令江總與紇有舊,收養之,教以書計。雖貌甚寢陋,而聰悟絕倫,讀書即數行俱下,博覽經史,尤精《三史》。仕隋為太常博士。高祖微時,引為賓客。及即位,累遷給事中。



 詢初學王羲之書,後更漸變其體,筆力險勁,為一時之絕。人得其尺牘文字,咸以為楷範焉。高麗甚重其書,嘗遣使求之。高祖嘆曰:「不意詢之書名,遠播夷狄,彼觀其跡,固謂其形
 魁梧耶!」



 武德七年,詔與裴矩、陳叔達撰《藝文類聚》一百卷。奏之,賜帛二百段。



 貞觀初,官至太子率更令、弘文館學士,封渤海縣男。年八十餘卒。



 子通,少孤,母徐氏教其父書。每遺通錢,紿云:「質汝父書跡之直。」通慕名甚銳,晝夜精力無倦,遂亞於詢。儀鳳中,累遷中書舍人。丁母憂,居喪過禮。起復本官,每入朝,必徒跣至皇城門外。直宿在省,則席地藉槁。非公事不言,亦未嘗啟齒。歸家必衣縗絰,號慟無恆。自武德已來,起復後而能哀戚合禮者,
 無與通比。年兇未葬,四年居廬不釋服,家人冬月密以氈絮置所眠席下,通覺,大怒,遽令徹之。



 五遷,垂拱中至殿中監,賜爵渤海子。天授元年,封夏官尚書。二年,轉司禮卿,判納言事。為相月餘,會鳳閣舍人張嘉福等請立武承嗣為皇太子,通與岑長倩固執以為不可,遂忤諸武意,為酷吏所陷,被誅。神龍初,追復官爵。



 硃子奢,蘇州吳人也。少從鄉人顧彪習《春秋左氏傳》,後博觀子史,善屬文。隋大業中,直秘書學士。及天下大亂,
 辭職歸鄉里,尋附於杜伏威。武德四年,隨伏威入朝,授國子助教。貞觀初,高麗、百濟同伐新羅,連兵數年不解,新羅遣使告急。乃假子奢員外散騎侍郎充使,喻可以釋三國之憾,雅有儀觀,東夷大欽敬之,三國王皆上表謝罪,賜遣甚厚。



 初,子奢之出使也,太宗謂曰:「海夷頗重學問,卿為大國使,必勿藉其束脩,為之講說。使還稱旨,當以中書舍人待卿。」子奢至其國,欲悅夷虜之情,遂為發《春秋左傳》題,又納其美女之贈。使還,太宗責其違旨,
 猶惜其才,不至深譴,令散官直國子學。轉諫議大夫、弘文館學士,遷國子司業,仍為學士。



 子奢風流蘊藉,頗滑稽,又輔之以文義,由是數蒙宴遇,或使論難於前。十五年卒。



 張士衡,瀛州樂壽人也。父之慶,齊國子助教。士衡九歲喪母,哀慕過禮。父友齊國博士劉軌思見之,每為掩泣。謂其父曰:「昔伯饒號『張曾子』,亦豈能遠過!吾聞君子不親教,當為成就之。」及長,軌思授以《毛詩》、《周禮》,又從熊安
 生及劉焯受《禮記》,皆精究大義。此後遍講《五經》,尤攻《三禮》。仕隋為餘杭令,後以年老歸鄉里。



 貞觀中,幽州都督、燕王靈夔備玄纁束帛之禮,就家迎聘,北面師之。庶人承乾在東宮,又加旌命。及至洛陽宮謁見,太宗延之升殿,賜食,擢授朝散大夫、崇賢館學士。承乾見之,問以齊氏滅亡之由緒,對曰:「齊後主悖虐無度,暱近小人。至如高阿那瑰、駱提婆、韓長鸞等,皆奴僕下才,兇險無賴,是信是使,以為心腹。誅害忠良,疏忌骨肉。窮極奢靡,剝喪
 黎元。所以周師臨郊,人莫為用,以至覆滅,實此之由。」承乾又問曰:「布施營功德,有果報不?」對曰:「事佛在於清凈無欲,仁恕為心。如其貪婪無厭,驕虐是務,雖復傾財事佛,無救目前之禍。且善惡之報,若影隨形,此是儒書之言,豈徒佛經所說。是為人君父,當須仁慈;為人臣子,宜盡忠孝。仁慈忠孝,則福祚攸永;如或反此,則殃禍斯及。此理昭然,願殿下勿為憂慮。」及承乾廢黜,敕給乘傳,令歸本鄉。十九年卒。



 士衡既禮學為優,當時受其業擅名
 於時者,唯賈公彥為最焉。



 賈公彥,洺州永年人。永徽中,官至太學博士。撰《周禮義疏》五十卷、《儀禮義疏》四十卷。



 子大隱,官至禮部侍郎。



 時有趙州李玄植,又受《三禮》於公彥,撰《三禮音義》行於代。玄植兼習《春秋左氏傳》於王德韶,受《毛詩》於齊威,博涉漢史及老、莊諸子之說。貞觀中,累遷太子文學、弘文館直學士。高宗時,屢被召見。與道士、沙門在御前講說經義,玄植辨論甚美,申規諷,帝深禮之。後坐事左遷汜水
 令,卒官。



 張後胤,蘇州昆山人也。父中,有儒學,隋漢王諒出牧並州,引為博士。後胤從父在並州,以學行見稱。時高祖鎮太原,引居賓館。太宗就受《春秋左氏傳》。武德中,累除燕王諮議參軍。



 貞觀中,後胤上言:「陛下昔在太原,問臣:『隋氏運終,何族當得天下?』臣奉對:『李姓必得。公家德業,天下系心,若於此首謀,長驅關右,以圖帝業,孰不幸賴!』此實微臣早識天命。」太宗曰:「此事並記之耳。」因詔入賜宴,
 言及平昔,從容謂曰:「今弟子何如?」後胤對曰:「昔孔子領徒三千,達者無子男之位。臣翼贊一人,為萬乘主,計臣功逾於先聖。」太宗甚悅,賜良馬五匹,拜燕王府司馬。遷國子祭酒,轉散騎常侍。



 永徽初,請致仕,加金紫光祿大夫,給賜並同職事。卒,贈禮部侍郎,陪葬昭陵。



 蓋文達,冀州信都人也。博涉經史,尤明《三傳》。性方雅,美須貌,有士君子之風。刺史竇抗嘗廣集儒生,令相問難,其大儒劉焯、劉軌思、孔穎達咸在坐,文達亦參焉。既論
 難,皆出諸儒意表,抗大奇之,問曰:「蓋生就誰受學?」劉焯對曰:「此生岐嶷,出自天然。以多問寡,焯為師首。」抗曰:「可謂冰生於水而寒於水也。」



 武德中,累授國子助教。太宗在籓,召為文學館直學士。貞觀十年,遷諫議大夫,兼弘文館學士。十三年,除國子司業。俄拜蜀王師,以王有罪,坐免。十八年,授崇賢館學士。尋卒。其宗人文懿,亦以儒業知名,當時稱為「二蓋」焉。



 文懿者,貝州宋城人也。武德初,歷國子助教。時高祖別於秘書省置學,教授王公之
 子,時以文懿為博士。文懿嘗開講《毛詩》,發題,公卿咸萃,更相問難,文懿發揚風雅,甚得詩人之致。貞觀中,卒於國子博士。



 谷那律,魏州昌樂人也。貞觀中,累補國子博士。黃門侍郎褚遂良稱為「九經庫」。尋遷諫議大夫,兼弘文館學士。嘗從太宗出獵,在途遇雨,因問:「油衣若為得不漏?」那律曰:「能以瓦為之,必不漏矣。」意欲太宗不為畋獵。太宗悅,賜帛二百段。永徽初,卒官。



 蕭德言,雍州長安人,齊尚書左僕射思話玄孫也。本蘭陵人,陳亡,徙關中。祖介,梁侍中、都官尚書。父引,陳吏部侍郎。並有名於時。德言博涉經史,尤精《春秋左氏傳》,好屬文。貞觀中,除著作郎,兼弘文館學士。



 德言晚年尤篤志於學,自晝達夜,略無休倦。每欲開《五經》,必束帶盥濯,危坐對之。妻子候間請曰:「終日如是,無乃勞乎?」德言曰:「敬先聖之言,豈憚如此!」時高宗為晉王,詔德言授經講業。及升春宮,仍兼侍讀。尋以年老,請致仕,太宗不許。又
 遺之書曰:



 朕歷觀前代,詳覽儒林,至於顏、閔之才,不終其壽;游、夏之德,不逮其學。惟卿幼挺珪璋,早標美譽。下帷閉戶,包括《六經》;映雪聚螢,牢籠百氏。自隋季版蕩,闍序無聞,儒道墜泥塗,《詩書》填坑穽。眷言墳典,每用傷懷。頃年已來,天下無事,方欲建禮作樂,偃武修文。卿年齒已衰,教將何恃!所冀才德猶茂,臥振高風,使濟南伏生,重在於茲日;關西孔子,故顯於當今。令問令望,何其美也!念卿疲朽,何以可言!



 尋賜爵封陽縣侯。十七年,拜秘
 書少監。兩宮禮賜甚厚。二十三年,累表請致仕,許之。高宗嗣位,以師傅恩,加銀青光祿大夫。永徽五年,卒於家,年九十七。高宗為之輟朝,贈太常卿。文集三十卷。



 曾孫至忠,自有傳。



 許叔牙,潤州句容人。少精於《毛詩》、《禮記》,尤善諷詠。貞觀初,累授晉王文學兼侍讀,尋遷太常博士。升春宮,加朝散大夫,遷太子洗馬,兼崇賢館學士,仍兼侍讀。嘗撰《毛詩纂義》十卷,以進皇太子。太子賜帛百段,兼令寫本付
 司經局。御史大夫高智周嘗謂人曰:「凡欲言《詩》者,必須先讀此書。」貞觀二十三年卒。子子儒。



 子儒,亦以學藝稱。長壽中,官至天官侍郎、弘文館學士。子儒居選部,不以藻鑒為意,委令史句直,以為腹心。注官之次,子儒但高枕而臥,時云「句直平配」。由是補授失序,無復綱紀,道路以為口實。其所注《史記》,竟未就而終。



 敬播,蒲州河東人也。貞觀初,舉進士。俄有詔詣秘書內省佐顏師古、孔穎達修《隋史》,尋授太子校書。史成,遷著
 作郎,兼修國史。與給事中許敬宗撰《高祖》、《太宗實錄》,自創業至於貞觀十四年,凡四十卷。奏之,賜物五百段。太宗之破高麗,名所戰六山為駐蹕,播謂人曰:「聖人者,與天地合德,山名駐蹕,此蓋以鑾輿不復更東矣。」卒如所言。



 時梁國公房玄齡深稱播有良史之才,曰:「陳壽之流也。」玄齡以顏師古所注《漢書》,文繁難省,令播撮其機要,撰成四十卷,傳於代。尋以撰實錄功,遷太子司議郎。時初置此官,極為清望。中書令馬周嘆曰:「所恨資品妄高,
 不獲歷居此職。」參撰《晉書》,播與令狐德棻、陽仁卿、李嚴等四人總其類。



 會刑部奏言:「準律:謀反大逆,父子皆坐死,兄弟處流。此則輕而不懲,望請改從重法。」制遣百僚詳議。播議曰:「昆季孔懷,天倫雖重,比於父子,性理已殊。生有異室之文,死有別宗之義。今有高官重爵,本廕唯逮子孫;祚土錫珪,餘光不及昆季。豈有不沾其廕,輒受其辜,背禮違情,殊為太甚!必期反茲春令,踵彼秋荼,創次骨於道德之辰,建深文於措刑之日,臣將以為不可。」
 詔從之。



 永徽初,拜著作郎。與許敬宗等撰《西域圖》。後歷諫議大夫、給事中,並依舊兼修國史。又撰《太宗實錄》,從貞觀十五年至二十三年,為二十卷。奏之,賜帛三百段。後坐事出為越州都督府長史。龍朔三年,卒官。播又著《隋略》二十卷。



 劉伯莊,徐州彭城人也。貞觀中,累除國子助教。與其舅太學博士侯孝遵齊為弘文館學士,當代榮之。尋遷國子博士,其後又與許敬宗等參修《文思博要》及《文館詞
 林》。龍朔中,兼授崇賢館學士。撰《史記音義》、《史記地名》、《漢書音義》各二十卷,行於代。



 子之宏,亦傳父業。則天時,累遷著作郎,兼修國史。卒於相王府司馬。睿宗即位,以故吏贈秘書少監。



 秦景通,常州晉陵人也。與弟肸,尤精《漢書》,當時習《漢書》者皆宗師之,常稱景通為大秦君,暐為小秦君。若不經其兄弟指授,則謂之「不經師匠,無足採也」。景通,貞觀中累遷太子洗馬,兼崇賢館學士。為《漢書》學者,又有劉納
 言,亦為當時宗匠。



 納言,乾封中,歷都水監主簿,以《漢書》授沛王賢。及賢為皇太子,累遷太子洗馬,兼充侍讀。常撰《俳諧集》十五卷,以進太子。及東宮廢,高宗見而怒之。詔曰:「劉納言收其餘藝,參侍經史,自府入宮,久淹歲月,朝游夕處,竟無匡贊。闕忠孝之良規,進詼諧之鄙說,儲宮敗德,抑有所由。情在好生,不忍加戮,宜從屏棄,以勵將來。可除名。」後又坐事配流振州而死。



 羅道琮,蒲州虞鄉人也。祖順,武德初,為興州刺史。勤於
 學業,而慷慨有節義。貞觀末,上書忤旨,配流嶺表。時有同被流者,至荊、襄間病死,臨終,泣謂道琮曰:「人生有死,所恨委骨異壤。」道琮曰:「我若生還,終不獨歸,棄卿於此!」瘞之路左而去。歲餘,遇赦得還,至殯所,屬霖潦瀰漫,柩不復可得。道琮設祭慟哭,告以欲與俱歸之意,若有靈者,幸相警示。言訖,路側水中,忽然湧沸。道琮又咒云:「若所沸處是,願更令一沸。」咒訖,又沸。道琮便取得其尸,銘志可驗,遂負之還鄉。當時識者稱道琮誠感所致。道
 琮尋以明經登第。高宗末,官至太學博士。每與太學助教康國安、道士李榮等講論,為時所稱。尋卒。



\end{pinyinscope}