\article{卷二十五 志第五 禮儀五}

\begin{pinyinscope}

 唐禮:四時
 各以孟月享太廟,每室用太牢,季冬蠟祭之後,以辰日臘享於太廟,用牲如時祭。三年一祫,以孟冬。五年一禘,以孟夏。又時享之日,修七祀於太廟西門內
 之道南:司命,戶以春,灶以夏;門,厲以秋,行以冬,中溜則於季夏迎氣日祀之。若品物時新堪進御者,所司先送太常,與尚食相知,簡擇精好者,以滋味與新物相宜者配之。太常卿奉薦於太廟,不出神主。仲春薦冰,亦如之。



 武德元年五月,備法駕迎宣簡公,懿王、景皇帝、元皇帝神主,祔於太廟,始享四室。貞觀九年,高祖崩,將行遷祔之禮,太宗命有司詳議廟制。諫議大夫硃子奢建議曰:



 按漢丞相韋玄成奏立五廟,諸侯亦同五。劉子駿議開
 七祖,邦君降二。鄭司農踵玄成之轍,王子雍揚國師之波,分塗並驅,各相師祖,咸玩共所習,好同惡異。遂令歷代祧祀,多少參差,優劣去取,曾無畫一。《傳》稱「名位不同,禮亦異數。」《易》云「卑高以陳,貴賤位矣」。豈非別嫌疑,慎微遠,防陵僭,尊君卑佐,升降無舛,所貴禮者,義在茲乎!若使天子諸侯,俱立五廟,便是賤可以同貴,臣可以濫主,名器無準,冠屨同歸,禮亦異數,義將安設?《戴記》又稱:禮有以多為貴者,天子七廟,諸侯五廟。」若天子五廟,才與
 子男相埒,以多為貴,何所表乎?愚以為諸侯立高祖以下,並太祖五廟,一國之貴也。天子立高祖以上,並太祖七廟,四海之尊也。降殺以兩,禮之正焉。前史所謂「德厚者流光,德薄者流卑」,此其義也。伏惟聖祖在天,山陵有日,祔祖嚴配,大事在斯。宜依七廟,用崇大禮。若親盡之外,有王業之所基者,如殷之玄王,周之後稷,尊為始祖。倘無其例,請三昭三穆,各置神主,太祖一室,考而虛位。將待七百之祚,遞遷方處,庶上依晉、宋,傍愜人情。



 於是
 八座奏曰:



 臣聞揖讓受終之後,革命創制之君,何嘗不崇親親之義,篤尊尊之道,虔奉祖宗,致敬郊廟。自義乖闕里,學滅秦庭,儒雅既喪,經籍湮殄。雖兩漢纂修絕業,魏、晉敦尚斯文,而宗廟制度,典章散逸,習所傳而競偏說,執淺見而起異端。自昔迄茲,多歷年代,語其大略,兩家而已。祖鄭玄者則陳四廟之制,述王肅者則引七廟之文,貴賤混而莫辯,是非紛而不定。



 陛下至德自然,孝思罔極,孺慕逾匹夫之志,制作窮聖人之道,誠宜定一
 代之宏規,為萬世之彞則。臣奉述睿旨,討論往載,紀七廟者實多,稱四祖者蓋寡。校其得失,昭然可見。《春秋穀梁傳》及《禮記》、《王制》、《祭法》、《禮器》《孔子家語》,並云:「天子七廟,諸侯五廟,大夫三廟,士二廟。」《尚書》曰:「七世之廟,可以觀德。」至於孫卿、孔安國、劉歆、班彪父子、孔晁、虞喜、干寶之徒,或學推碩儒,或才稱博物,商較今古,咸以為然。故其文曰:「天子三昭三穆,與太祖之廟而七。」晉、宋、齊、梁,皆依斯義,立親廟六,豈非有國之茂典,不刊之休烈乎?若使
 違群經之明文,從累代之疑議,背子雍之篤論,尊康成之舊學,則天子之禮,下逼於人臣,諸侯之制,上僭於王者,非所謂尊卑有序,名位不同者也。況復禮由人情,自非天墜,大孝莫重於尊親,厚本莫先於嚴配。數盡四廟,非貴多之道;祀逮七世,得加隆之心。是知德厚者流光,乃可久之高義;德薄者流卑,實不易之令範。臣等參議,請依晉、宋故事,立親廟六,其祖宗之制,式遵舊典。庶承宗之道,興於理定之辰;尊祖之義,成於孝治之日。



 制從
 之。於是增修太廟,始崇祔弘農府君及高祖神主,並舊四室為六室。



 二十三年,太宗崩,將行崇祔之禮,禮部尚書許敬宗奏言:「弘農府君廟應迭毀。謹按舊儀,漢丞相韋玄成以為毀主瘞埋。但萬國宗饗,有所從來,一旦瘞埋,事不允愜。晉博士範宣意欲別立廟宇,奉征西等主安置其中。方之瘞埋,頗葉情理,事無典故,亦未足依。又議者或言毀主藏於天府,祥瑞所藏,本非斯意。今謹準量,去祧之外,猶有壇墠,祈禱所及,竊謂合宜。今時廟制,
 與古不同,共基別室,西方為首。若在西夾之中,仍處尊位,祈禱則祭,未絕祗享,方諸舊儀,情實可安。弘農府君廟遠親殺,詳據舊章,禮合迭毀。臣等參議,遷奉神主,藏於夾室,本情篤教,在理為弘。」從之。其年八月庚子,太宗文皇帝神主祔於太廟。



 文明元年八月,奉高宗神主祔於太廟中,始遷宣皇帝神主於夾室。垂拱四年正月,又於東都立高祖、太宗、高宗三廟,四時享祀,如京廟之儀。別立崇先廟以享武氏祖考。則天尋又令所司議立崇
 先廟室數,司禮博士、崇文館學士周忭希旨,請立崇先廟為七室,其皇室太廟,減為五室。春官侍郎賈大隱奏曰:「臣竊準秦、漢皇太后臨朝稱制,並據禮經正文,天子七廟,諸侯五廟。蓋百王不易之義,萬代常行之法,未有越禮違古而擅裁儀注者也。今周悰別引浮議,廣述異文,直崇臨朝權儀,不依國家常度,升崇先之廟而七,降國家之廟而五。臣聞皇圖廣闢,實崇宗社之尊;帝業弘基,實等山河之固。伏以天步多艱,時逢遏密,代天理物,
 自古有之。伏惟皇太后親承顧托,憂勤黎庶,納孝慈之請,垂矜撫之懷,實所謂光顯大猷,恢崇聖載。其崇先廟室,合同諸侯之數,國家宗廟,不合輒有移變。臣之愚直,並依正禮,周忭之請,實乖古儀。」則天由是且止。



 天授二年,則天既革命稱帝,於東都改制太廟為七廟室,奉武氏七代神主,祔於太廟。改西京太廟為享德廟,四時唯享高祖已下三室,餘四室令所司閉其門,廢其享祀之禮。又改西京崇先廟為崇尊廟,其享祀如太廟之儀。萬
 歲登封元年臘月,封嵩山回,親謁太廟。明年七月,又改京崇尊廟,為太廟,仍改太廟署為清廟臺,加官員,崇其班秩。聖歷二年四月,又親祀太廟,曲赦東都城內。



 中宗即位,神龍元年正月,改享德廟依舊為京太廟。五月,遷武氏七廟神主於西京之崇尊廟,東都創置太廟。太常博士張齊賢建議曰:



 昔孫卿子云:「有天下者事七代,有一國者事五代。」則天子七廟,古今達禮。故《尚書》稱「七代之廟,可以觀德」。《祭法》稱「王立七廟,一壇一墠」。王制云:「天子七廟,三昭三穆,與太祖之廟而七。」莫不
 尊始封之君謂11111之陳」陳於太祖,未毀廟之主,皆升合食於太祖之室。太祖東向,昭南向,穆北向。太祖者,商之玄王、周之後稷是也。太祖之外,更無始祖。但商自玄王以後,十有四代,至湯而有天下。周自後稷已後,十有七代,至武王而有天下。其間代數既遠,遷廟親廟,皆出太祖之後,故得合食有序,尊卑不差。其後漢高祖受命,無始封祖,即以高皇帝為太祖。太上皇高帝之父,立廟享祀,不在昭穆合食之列,為
 尊於太祖故也。魏武創業,文帝受命,亦即以武帝為太祖。其高皇、太皇、外處君等並為屬尊,不在昭穆合食之列。晉宣創業,武帝受命,亦即以宣帝為太祖。其征西、豫章、潁川、京兆府君等並為屬尊,不在昭穆合食之列。歷茲已降,至於有隋,宗廟之制,斯禮不改。故宇文氏以文皇帝為太祖,隋室以武元皇帝為太祖。國家誕受在命,累葉重光。景皇帝始封唐公,實為太祖。中間代數既近,列在三昭三穆之內,故皇家太廟,唯有六室。其弘農府
 君、宣、光二帝,尊於太祖,親盡則遷,不在昭穆合食之數。



 今皇極再造,孝思匪寧。奉二月二十九日敕:「七室已下,依舊號尊崇。」又奉三月一日敕:「既立七廟,須尊崇始祖,速令詳之」者。伏尋禮經,始祖即是太祖,太祖之外,更無始祖。周朝太祖之外,以周文王為始祖,不合禮經。或有引《白虎通義》云「後稷為始祖、文王為太祖、武王為太宗」,及鄭玄注《詩·雍》序云「太祖謂文王」以為說者。其義不然。何者?彼以禮「王者祖有功,宗有德,周人祖文王而宗武
 王」,故謂文王為太祖耳,非袷祭群主合食之太祖。



 今之議者,或有欲立涼武昭王為始祖者,殊為不可。何者?昔在商、周、稷、珣始封,湯、武之興,祚由稷、珣,故以稷、珣為太祖,即皇家之景帝是也。涼武昭王勛業未廣,後主失國,土宇不傳。景皇始封,實基明命。今乃舍封唐之盛烈,崇西涼之遠構,考之前古,實乖典禮。魏氏不以曹參為太祖,晉氏不以殷王仰為太祖,宋氏不以楚元王為太祖,齊、梁不以蕭何為太祖,陳、隋不以胡公、楊震為太祖,則
 皇家安可以涼武昭王為太祖乎?漢之東京,大議郊祀,多以周郊後稷,漢當郊堯。制上公卿議,議者多同,帝亦然之。杜林正議,獨以為「周室之興,祚由后稷。漢業特起,功不緣堯。祖宗故事,所宜因循。」竟從林議。又傳稱,「欲知天上,事問長人」,以其近之。武德、貞觀之時,主聖臣賢,其去涼武昭王蓋亦近於今矣。當時不立者,必不可立故也。今既年代浸遠,方復立之,是非三祖二宗之意。實恐景皇失職而震怒,武昭虛位而不答,非社稷之福也。



 宗
 廟事重,禘祫禮崇,先王以之觀德。或者不知其說,既灌而往,孔子不欲觀之。今朝命惟新,宜應慎禮,祭如神在,理不可誣。請準敕加太廟為七室,享宣皇帝以備七代,其始祖不合別有尊崇。



 太常博士劉承慶、尹知章又議云:



 謹按《王制》:「天子七廟,在昭三穆,與太祖之廟而七。」此載籍之明文,古今之通制。皇唐稽考前範,詳採列闢,崇建宗靈,式遵斯典。但以開基之主,受命之君,王跡有淺深,太祖有遠近。湯、文祚基稷、珣,太祖代遠,出乎昭穆之
 上,故七廟可全。若夏繼唐、虞,功非由鯀;漢除秦、項,力不因堯。及魏、晉經圖,周、隋撥亂,皆勛隆近代,祖業非遠,受命始封之主,不離昭穆之親,故肇立宗祊,罕聞全制。夫太祖以功建,昭穆以親崇,有功百代而不遷,親盡七葉而當毀。或以太祖代淺,廟數非備,更於昭穆之上,遠立合遷之君,曲從七廟之文,深乖迭毀之制。



 皇家千齡啟旦,百葉重光。景皇帝浚德基唐,代數猶近,號雖崇於太祖,親尚列於昭穆,且臨六室之位,未申七代之尊。是知
 太廟當六,未合有七。故先朝惟有宣、光、景、元、神、堯、文武六代親廟。大帝登遐,神主升祔於廟室,以宣後帝代數當滿,準禮復遷。今止有光皇帝已下六代親廟,非是天子之廟數不當有七,要由太祖有遠近之異,故初建有多少之殊。敬惟三后臨朝,代多儒雅,神祊事重,禮豈虛存,規模可沿,理難變革。宣皇既非始祖,又廟無祖宗之號,親盡既遷,其在不合重立。若禮終運往,建議復崇,實違《王制》之文,不合先朝之旨。請依貞觀之故事,無改三
 聖之宏規,光崇六室,不虧古議。



 時有制令宰相更加詳定,禮部尚書祝欽明等奏言:「博士三人,自分兩議:「張齊賢以始同太祖,不合更祖昭王;劉承慶以《王制》三昭三穆,不合重崇宣帝。臣等商量,請依張齊賢以景皇帝為太祖,依劉承慶尊崇六室。」制從之。尋有制以孝敬皇帝為義宗,升祔於太廟。其年八月,崇祔光皇帝、太祖景皇帝、代祖元皇帝、高祖神堯皇帝、太宗文武聖皇帝、皇考高宗天皇大帝、皇兄義宗孝敬皇帝於東都之太廟,躬
 行享獻之禮。



 二年,駕還京師,太廟自是亦崇享七室,仍改武氏崇尊廟為崇恩廟。明年二月,復令崇恩廟一依天授時享祭。時武三思用事,密令安樂公主諷中宗,故有此制。尋又特令武氏崇恩廟齋郎取五品子充。太常博士楊孚奏言:「太廟齋郎,承前只七品已下子。今崇恩廟齋郎既取五品子,即太廟齋郎作何等級?」上曰:太廟齋郎亦準崇恩廟置。」孚奏曰:「崇恩廟為太廟之臣,太廟為崇恩廟之君,以臣準君,猶為僭逆,以君準臣,天下疑懼。孔子曰:『
 名不正則言不順,言不順則事不成,事不成則禮樂不興,禮樂不興則刑罰不中,刑罰不中則人無所措手足。故君子名之必可言也。』伏願無惑邪言,以為亂始。」其事乃寢。崇恩廟至睿宗踐祚,乃廢毀之。



 景雲元年冬,將葬中宗孝和皇帝於定陵,中書令姚元之、吏部尚書宋璟奏言:「準禮,大行皇帝山陵事終,即合祔廟。其太廟第七室,先祔皇兄義宗孝敬皇帝、哀皇后裴氏神主。伏以義宗未登大位,崩後追尊,神龍之初,乃特令遷祔。《春秋》之
 義,國君即位未逾年者,不合列敘昭穆。又古者祖宗各別立廟,孝敬皇帝恭陵既在洛州,望於東都別立義宗之廟,遷祔孝敬皇帝、哀皇后神主,命有司以時享祭,則不違先旨,又協古訓,人神允穆,進退得宜。在此神主,望入夾室安置。伏願陛下以禮斷恩。」制從之。及既葬,祔中宗孝和皇帝、和思皇后趙氏神主於太廟。其義宗即於東都從善里建廟享祀。時又追尊昭成、肅明二皇后,於親仁里別置儀坤廟,四時享祭。



 開元四年,睿宗崩,及行
 祔廟之禮,太常博士陳貞節、蘇獻等奏議曰:「謹按孝和皇帝在廟,七室已滿。今睿宗大聖真皇帝是孝和之弟,甫及仲冬,禮當祔遷。但兄弟入廟,古則有焉,遞遷之禮,昭穆須正。謹按《禮論》,太常賀循議云:『兄弟不相為後也。故殷之盤庚,不序於陽甲,而上繼於先君;漢之光武,不嗣於孝成,而上承於元帝。』又曰:『晉惠帝無後,懷帝承統,懷帝自繼於世祖,而不繼於惠帝。其惠帝當同陽甲、孝成,別出為廟。』又曰:『若兄弟相代,則共是一代,昭穆位同。
 至其當遷,不可兼毀二廟。』此蓋禮之常例也。《荀卿子》曰,『有天下者事七代』,謂從禰已上也。尊者統廣,故恩及遠祖。若旁容兄弟,上毀祖考,此則天子有不得全事於七代之義矣。孝和皇帝有中興之功,而無後嗣,請同殷之陽甲、漢之成帝,出為別廟,時祭不虧,大祫之辰,合食太祖。奉睿宗神主升祔太廟,上繼高宗,則昭穆永貞,獻祼長序。」制從之。初令以儀坤廟為中宗廟,尋又改造中宗廟於太廟之西。貞節等又以肅明皇后不合與昭成皇
 后配祔睿宗,奏議曰:「禮,宗廟父昭子穆,皆有配座,每室一帝一後,禮之正儀。自夏、殷而來,無易茲典。伏惟昭成皇后,有太姒之德,已配食於睿宗;則肅明皇后無啟母之尊,自應別立一廟。謹按《周禮》云『奏夷則,歌小呂,以享先妣』者,姜嫄是也。姜嫄是帝嚳之妃,後稷之母,特為立廟,名曰閟宮。又《禮論》云,晉伏系之議云:『晉簡文鄭宣后既不配食,乃築宮於外,歲時就廟享祭而已。』今肅明皇后無祔配之位,請同姜嫄、宣後,別廟而處,四時享祭如舊儀。」
 制從之。於是遷昭成皇后神主祔於睿宗之室,惟留肅明神主於儀坤廟。



 時太常卿姜皎復與禮官上表曰:「臣聞敬宗尊祖,享德崇恩,必也正名,用光時憲,禮也。伏見太廟中則天皇后配高宗天皇大帝,題云『天后聖帝武氏』。伏尋昔居寵秩,親承顧托,因攝大政,事乃從權。神龍之初,已去帝號。岑羲等不閑政體,復題帝名。若又使帝號長存,恐非聖朝通典。夫七廟者,高祖神堯皇帝之廟也。父昭子穆,祖德宗功,非夫帝子天孫,乘乾出震者,不
 得升祔於斯矣。但皇后祔廟,配食高宗,位號舊章,無宜稱帝。今山陵日近,升祔非遙,請申陳告之儀,因除『聖帝』之字,直題云『則天皇后武氏』。」詔從之。時既另造義宗廟,將作大匠韋湊上疏曰:「臣聞王者制體,是曰規模;規模之興,實資師古;師古之道,必也正名;惟名與實,固當相副。其在宗廟,禮之大者,豈可失哉!禮,祖有功而宗有德。祖宗之廟,百代不毀。故殷太甲曰太宗,太戊曰中宗,武丁曰高宗。周宗文王、武王。漢則文帝為太宗,武帝為世
 宗。其後代有稱宗,皆以方制海內,德澤可宗,列於昭穆,期於不毀。祖宗之義,不亦大乎!況孝敬皇帝位止東宮,未嘗南面,聖道誠冠於儲副,德教不被於寰瀛,立廟稱宗,恐非合體。況別起寢廟,不入昭穆,稽諸祀典,何義稱宗?而廟號義宗,稱之萬代。以臣庸識,竊謂不可。望更令有司詳定,務合於禮。」於是太常請以本謚「孝敬」為廟稱。從之。



 五年正月,玄宗將行幸東都,而太廟屋壞,乃奉七廟神主於太極殿。玄宗素服避正殿,輟朝三日,親謁神
 主於太極殿,而後發幸東都。乃敕有司修太廟。明年,廟成,玄宗還京,行親祔之禮。時有司撰儀注,以祔祭之日車駕發宮中,玄宗謂宋璟、蘇頲曰:「祭必先齋,所以齊心也。據儀注,祭之日發大明宮,又以質明行事,縱使侵星而發,猶是移辰方到,質明之禮,其可及乎?又朕不宿齋宮,即安正殿,情所不敢。宜於廟所設齋宮,五日赴行宮宿齋,六日質明行事,庶合於禮。」璟等稱聖情深至,請即奉行。詔有司改定儀注。六日,玄宗自齋宮步詣太廟,
 入自東門,就立位。樂奏九成,升自阼階,行祼獻之禮。至睿宗室,俯伏鳴咽,侍臣莫不流涕。



 有河南府人孫平子詣闕上言:「中宗孝和皇帝既承大統,不合遷於別廟。」玄宗令宰相召平子與禮官對定可否,太常博士蘇獻等固執前議。平子口辯,所引咸有經據,獻等不能屈。時蘇頲知政事,以獻是其從祖之兄,頗黨助之,平子之議竟不得行。平子論竟不已,遂謫平子為康州都城尉,仍差使領送至任,不許東西。平子之任,尋卒。時雖貶平子,議
 者深以其言為是。至十年正月,下制曰:「朕聞王者乘時以設教,因事以制禮,沿革以從宜為本,取舍以適會為先。故損益之道有殊,質文之用斯異。且夫至德之謂孝,所以通乎神明;大事之謂祀,所以虔乎宗廟。國家握紀命歷,重光累盛,四方由其繼明,七代可以觀德。朕嗣守丕業,祗奉睿圖,聿懷昭事,罔不恤祀。嘗覽古典,詢諸舊制,遠則夏、殷事異,近則漢、晉道殊,雖禮文之不一,固嚴敬之無二。朕以為立愛自親始,教人睦也;立敬自長始,
 教人順也。是知朕率於禮,緣於情,或教以道存,或禮從時變,將因宜以創制,豈沿古而限今。況恩以降殺而疏,廟以遷毀而廢。雖式瞻古訓,禮則不違;而永言孝思,情所未足。享嘗則止,豈愛崇而禮備;有禱而祭,非德盛而流永。其祧室宜列為正室,使親而不盡,遠而不禰,廟以貌存,宗猶尊立。俾四時式薦,不間於毀主;百代靡遷,匪惟於始廟。所謂變以合禮,動而得中,嚴配之典克崇,肅雍之美茲在。又兄弟繼及,古有明文。今中宗神主,猶居
 別處,詳求故實,當寧不安,移就正廟,用章大典。仍創立九室,宜令所司擇日啟告移遷。」



 十一年春,玄宗還京師,下制曰:「崇建宗廟,禮之大者;聿追孝饗,德莫至焉。今宗以立尊,親無遷序,永惟嚴配,致用蠲潔,棟宇式崇,祼奠斯授。顧茲薄德,獲承禋祀,不躬不親,曷展誠敬?宜用八月十九日祗見九室。」於是追尊宣皇帝為獻祖,復列於正室,光皇帝為懿祖,並還中宗神主於太廟。及將親祔,會雨而止。乃令所司行事。其京師中宗舊廟,便毀拆之。
 東都舊廟,始移孝敬神主祔焉。其從善里孝敬舊廟,亦令毀拆。二十一年,玄宗又特令遷肅明皇后神主祔於睿宗之室,仍以舊儀坤廟為肅明觀。



 大歷十四年十月,代宗神主將祔,禮儀使顏真卿以元皇帝代數已遠,準禮合祧,請遷於西夾室。其奏議曰:



 《王制》:「天子七廟,三昭三穆,與太祖之廟而七。」又《禮器》云:有以多為貴者,天子七廟。」又《伊尹》曰:「七代之廟,可以觀德。」此經典之明證也。七廟之外,則曰:「去祧為壇,去壇為墠」。故歷代儒者,制迭
 毀之禮,皆親盡宜毀。伏以太宗文皇帝,七代之祖;高祖神堯皇帝,國朝首祚,萬葉所承;太祖景皇帝,受命於天,始封於唐,元本皆在不毀之典。代祖元皇帝,地非開統,親在七廟之外。代宗皇帝升祔有日,元皇帝神主,禮合祧遷。或議者以祖宗之名,難於迭毀。昔漢朝近古,不敢以私滅公,故前漢十二帝,為祖宗者四而已。至後漢漸違經意,子孫以推美為先。自光武已下,皆有廟號,則祖宗之名,莫不建也。安帝信讒,害大臣,廢太子,及崩,無上
 宗之奏,後自建武以來無毀者,因以陵號稱宗。至桓帝失德,尚有宗號。故初平中,左中郎蔡邕以和帝以下,功德無殊,而有過差,不應為宗。餘非宗者,追尊三代,皆奏毀之。是知祖有功,宗有德,存至公之義,非其人不居,蓋三代立禮之本也。自東漢已來,則此道衰矣。魏明帝自稱烈祖,論者以為逆自稱祖宗。故近代此名悉為廟號,未有子孫踐祚而不祖宗先王者。以此明之,則不得獨據兩字而為不合祧遷之證。假令傳祚
 百代,豈可上崇百代以為孝乎?請依三昭三穆之義,永為通典。



 寶應二年,升祔玄宗、肅宗,則獻祖、懿祖已從迭毀。伏以代宗睿文孝皇帝卒哭而祔,則合上遷一室。元皇帝代數已遠,其神主準禮當祧,至禘祫之時,然後享祀。



 於是祧元皇帝於西夾室,祔代宗神主焉。



 永貞元年十一月,德宗神主將祔,禮儀使杜黃裳與禮官王涇等請遷高宗神主於西夾室。其議曰:「自漢、魏已降,沿革不同。古者祖有功,宗有德,皆不毀之名也。自東漢、魏、晉,迄於陳、隋,漸違經
 意,子孫以推美為先,光武已下,皆有祖宗之號。故至於迭毀親盡,禮亦迭遷,國家九廟之尊,皆法周制。伏以太祖景皇帝受命於天,始封元本,德同周之後稷也。高祖神堯皇帝國朝首祚,萬葉所承,德同周之文王也。太宗文皇帝應天靖亂,垂統立極,德同周武王也。周人郊後稷而祖文王、宗武王,聖唐郊景皇帝、祖高祖而宗太宗,皆在不遷之典。高宗皇帝今在三昭三穆之外,謂之親盡,新主入廟,禮合迭遷,藏於從西第一夾室,每至禘祫
 之月,合食如常。」於是祧高宗神主於西夾室,祔德宗神主焉。



 元和元年七月,順宗神主將祔,有司疑於遷毀,太常博士王涇建議曰:



 禮經「祖有功,宗有德」,皆不毀之名也。惟三代行之。漢、魏已降,雖曰祖宗,親盡則遷,無功亦毀,不得行古之道也。昔夏后氏十五代,祖顓頊而宗禹。殷人十七代,祖契而宗湯。周人三十六王,以後稷為太祖,祖文王而宗武王。聖唐德厚流廣,遠法殷、周,奉景皇帝為太祖,祖高祖而宗太宗,皆在百代不遷之典。故代宗升
 祔,遷代祖也;德宗升祔,遷高宗也。今順宗升祔,中宗在三昭三穆之外,謂之親盡,遷於太廟夾室,禮則然矣。



 或諫者以則天太后革命,中宗復而興之,不在遷藏之例,臣竊未諭也。昔者高宗晏駕,中宗奉遺詔,自儲副而陟元後。則天太后臨朝,廢為盧陵王。聖歷元年,太后詔復立為皇太子。屬太后聖壽延長,御下日久,奸臣擅命,紊其紀度。敬暉、桓彥範等五臣,俱唐舊臣,匡輔王室,翊中宗而承大統。此乃子繼父業,是中宗得之而且失之;母
 授子位,是中宗失之而復得之。二十年間,再為皇太子,復踐皇帝位,失之在己,得之在己,可謂革命中興之義殊也。又以周、漢之例推之,幽王為犬戎所滅,平王東遷,周不以平王為中興不遷之廟,其例一也。漢呂后專權,產、祿秉政,文帝自代邸而立之,漢不以文帝為中興不遷之廟,其例二也。霍光輔宣帝,再盛基業,而不以宣帝為不遷之廟,其例三也。伏以中宗孝和皇帝,於聖上為六代伯祖,尊非正統,廟亦親盡。爰及周、漢故事,是與中興功德之主不同,奉遷夾室,固無
 疑也。



 是月二十四日,禮儀使杜黃裳奏曰:「順宗皇帝神主已升祔太廟,告祧之後,即合遞遷。中宗皇帝神主,今在三昭三穆之外,準禮合遷於太廟從西第一夾室,每至禘祫之日,合食如常。」於是祧中宗神主於西夾室,祔順宗神主焉。



 有司先是以山陵將畢,議遷廟之禮。有司以中宗為中興之君,當百代不遷之位。宰臣召史官蔣武問之,武對曰:「中宗以弘道元年於高宗柩前即位,時春秋已壯矣。及母後篡奪,神器潛移。其後賴張柬之等同
 謀,國祚再復。此蓋同於反正,恐不得號為中興之君。凡非我失之,自我復之,謂之中興,漢光武、晉元帝是也。自我失之,因人復之,晉孝惠、孝安是也。今中宗於惠、安二帝事同,即不可為不遷之主也。」有司又云:「五王有再安社稷之功,今若遷中宗廟,則五王永絕配享之例。」武曰:凡配享功臣,每至禘祫年方合食太廟,居常即無享禮。今遷中宗神主,而禘祫之年,毀廟之主並陳於太廟,此同五王配食,與前時如一也。」有司不能答。



 十五年四月,
 禮部侍郎李建奏上大行皇帝謚曰聖神章武孝皇帝,廟號憲宗。先是,河南節度使李夷簡上議曰:「王者祖有功,宗有德。大行皇帝戡翦寇逆,累有武功,廟號合稱祖。陛下正當決在宸斷,無信齷齪書生也。」遂詔下公卿與禮官議其可否。太常博士王彥威奏議:「大行廟號,不宜稱祖,宜稱宗。」從之。其月,禮部奏:「準貞觀故事,遷廟之主,藏於夾室西壁南北三間。第一間代祖室,第二間高宗室,第三間中宗室。伏以山陵日近,睿宗皇帝祧遷有期,
 夾室西壁三室外,無置室處。準《江都集禮》:『古者遷廟之主,藏於太室北壁之中。』今請於夾室北壁,以西為上,置睿宗皇帝神主石室。」制從之。



 長慶四年正月,禮儀使奏:「謹按《周禮》:『天子七廟,三昭三穆,與太祖之廟而七。』《荀卿子》曰:『有天下者祭七代,有一國者祭五代。』則知天子上祭七廟,典籍通規。祖功宗德,不在其數。國朝九廟之制,法周之文。太祖景皇帝,始為唐公,肇基天命,義同周之後稷。高祖神堯皇帝,創業經始,代隋為唐,義同周之文王。
 太宗文皇帝,神武應期,造有區夏,義同周之武王。其下三昭三穆,謂之親廟,四時常饗,自如禮文。今以新主入廟,玄宗明皇帝在三昭三穆之外,是親盡之祖,雖有功德,禮合祧遷,禘祫之歲,則從合食。」制從之。



 開成五年,禮儀使奏:「謹按天子七廟,祖功宗德,不在其中。國朝制度,太廟九室。伏以太祖景皇帝受封於唐,高祖、太宗,創業受命,有功之主,百代不遷。今文宗元聖昭獻皇帝升祔有時,代宗睿文孝武皇帝是親盡之祖,禮合祧遷,每至禘祫,合食如常。」從之。



 會昌元年六月,制曰:「朕近因載誕之日,展承顏之敬,
 太皇太后謂朕曰:『天子之孝,莫大於丕承;人倫之義,莫大於嗣續。穆宗睿聖文惠孝皇帝厭代已久,星霜屢遷,禰宮曠合食之禮,惟帝深濡露之感。宣懿皇太后,長慶之際,德冠後宮,夙表沙麓之祥,實茂河洲之範。先朝恩禮之厚,中壺莫偕。況誕我聖君,纘承昌運,已協華於先帝,方延祚於後昆。思廣貽謀,庶弘博愛,爰從舊典,以慰孝思。當以宣懿皇太后祔太廟穆宗睿聖文惠孝皇帝之室。率是彞訓,其敬承之。』朕祇奉慈旨,載深感咽。宜令
 宣示中外,咸使聞知。」



 會昌六年五月,禮儀使奏:



 武宗昭肅皇帝祔廟,並合祧遷者。伏以自敬宗、文宗、武宗兄弟相及,已歷三朝。昭穆之位,與承前不同。所可疑者,其事有四:一者,兄弟昭穆同位,不相為後;二者,已祧之主,復入舊廟;三者,廟數有限,無後之主,則宜出置別廟;四者,兄弟既不相為後,昭為父道,穆為子道,則昭穆同班,不合異位。



 據《春秋》「文公二年,躋僖公」。何休云:躋,升也,謂西上也。惠公與莊公當同南面西上,隱、桓與閔、僖當同北面西
 上。」孔穎達亦引此義釋經。又賀循云:「殷之盤庚,不序陽甲;漢之光武,上繼元帝。」晉元帝、簡文,皆用此義毀之,蓋以昭穆位同,不可兼毀二廟故也。《尚書》曰:「七代之廟,可以觀德。」且殷家兄弟相及,有至四帝不及祖禰,何容更言七代,於理無矣。二者,今已兄弟相及,同為一代,矯前之失,則合復祔代宗神主於太廟。或疑已祧之主,不合更入太廟者。按晉代元、明之時,已遷豫章、潁川矣,及簡文即位,乃元帝之子,故復豫章、潁川二神主於廟。又國
 朝中宗已祔太廟,至開元四年,乃出置別廟,至十年,置九廟,而中宗神主復祔太廟。則已遷復入,亦可無疑。三者,廟有定數,無後之主,出置別廟者。按魏、晉之初多同廟,蓋取上古清廟一宮,尊遠神祗之義。自後晉武所立之廟,雖云七主,而實六代,蓋景、文同廟故也。又按魯立姜嫄、文王之廟,不計昭穆,以尊尚功德也。晉元帝上繼武帝,而惠、懷、愍三帝,時賀循等諸儒議,以為別立廟,親遠義疏,都邑遷異,於理無嫌也。今以文宗棄代才六七
 年,武宗甫邇復土,遽移別廟,不齒祖宗,在於有司,非所宜議。四者,添置廟之室。按《禮論》,晉太常賀循云:「廟以容主為限,無拘常數。」故晉武帝時,廟有七主六代。至元帝、明帝,廟皆十室。及成、康、穆三帝,皆至十一室。自後雖遷故祔新,大抵以七代為準,而不限室數。伏以江左大儒,通賾睹奧,事有明據,固可施行。今若不行是議,更以迭毀為制,則當上不及高曾未盡之親,下有忍臣子恩義之道。



 今備討古今,參校經史,上請復代宗神主於太廟,以
 存高曾之親。下以敬宗、文宗、武宗同為一代,於太廟東間添置兩室,定為九代十一室之制,以全臣子恩敬之義,庶協大順之宜,得變禮之正,折古今之紛互,立群疑之杓指。俾因心廣孝,永燭於皇明;昭德事神,無虧於聖代。



 敕曰:「宗廟事重,實資參詳。宜令尚書省、兩省、御史臺四品以上官、大理卿、京兆尹等集議以聞。」尚書左丞鄭涯等奏議曰:「夫禮經垂則,莫重於嚴配,必參損益之道,則合典禮之文。況有明徵,是資折衷。伏自敬宗、文宗、武
 宗三朝嗣位,皆以兄弟,考之前代,理有顯據。今謹詳禮院所奏,並上稽古文,旁摭史氏,協於通變,允謂得宜。臣等商議,請依禮官所議。」從之。



 大中三年十一月,制追尊憲宗、順宗謚號,事下有司。太常博士李稠奏請別造憲宗、順宗神主,改題新謚。上疑其事,詔都省集議。右司郎中楊發、都官員外郎劉彥模等奏:「考尋故事,無別造神主改題之例。」事在《楊發傳》。時宰臣奏:「改造改題,並無所據,酌情順理,題則為宜。況今士族之家,通行此例,雖尊
 卑有異,而情理則同。望就神主改題,則為通允。」依之。



 黃巢犯長安,僖宗避狄於成都府。中和元年夏四月,有司請享太祖已下十一室,詔公卿議其儀。太常卿牛叢與儒者同議其事。或曰:「王者巡狩,以遷廟主行。如無遷廟之主,則祝奉幣帛皮珪告於祖禰,遂奉以出,載於齋車,每舍奠焉。今非巡狩,是失守宗廟。夫失守宗廟,則當罷宗廟之事。」叢疑之。將作監王儉、太子賓客李匡乂、虞部員外郎袁皓建議同異。及左丞崔厚為太常卿,遂議立
 行廟。以玄宗幸蜀時道宮玄元殿之前,架幄幕為十一室。又無神主,題神版位而行事。達禮者非之,以為止之可也。明年,乃特造神主以祔行廟。



 光啟元年十二月二十五日,僖宗再幸寶雞。其太廟十一室並祧廟八室及孝明太皇太后等別廟三室等神主,緣室法物,宗正寺官屬奉之隨駕鄠縣,為賊所劫,神主、法物皆遺失。三年二月,車駕自興元還京,以宮室未備,權駐鳳翔。禮院奏:皇帝還宮,先謁太廟。今宗廟焚毀,神主失墜,請準禮例
 修奉者。禮院獻議曰:「按《春秋》:『新宮災,三日哭。』《傳》曰:『新宮,宣公廟也。三日哭,禮也。』按《國史》,開元五年正月二日,太廟四室摧毀,時神主皆存,迎奉於太極殿安置,玄宗素服避正殿。寶應元年,肅宗還京師,以宗廟為賊所焚,於光順門外設次,向廟哭。歷檢故事,不見百官奉慰之儀。然上既素服避殿,百官奉慰,亦合情禮。竊循故事,比附參詳,恐須宗正寺具宗廟焚毀及神主失墜事由奏,皇帝素服避殿,受慰訖,輟朝三日,下詔委少府監擇日依
 禮新造列聖神主。如此方似合宜。伏緣採慄須十一月,漸恐遲晚。」修奉使宰相鄭延昌具議,中書門下奏曰:「伏以前年冬再有震驚,俄然巡寺,主司宗祝,迫以蒼黃。伏緣移蹕鳳翔,未敢陳奏。今則將回鑾輅,皆舉典章,清廟再營,孝思咸備。伏請降敕,命所司參詳典禮修奉。」敕曰:「朕以涼德,祗嗣寶圖,不能上承天休,下正人紀,兵革競興於宇縣,車輿再越於籓垣,宗廟震驚,烝嘗廢闕。敬修典禮,倍切哀摧。宜付所司。」又修奉太廟使宰相鄭延昌
 奏:「太廟大殿十一室、二十三間、十一架,功績至大,計料支費不少。兼宗廟制度有數,難為損益。今不審依元料修奉,為復更有商量?請下禮官詳議。」太常博士殷盈孫奏議言:「如依元料,難以速成,況幣藏方虛,須資變禮。竊以至德二年,以新修太廟未成,其新造神主,權於長安殿安置,便行饗告之禮,如同宗廟之儀,以俟廟成,方為遷祔。今京城除充大內及正衙外,別無殿宇。伏聞先有詔旨,欲以少府監大權充太廟。其五間,伏緣十一
 室於五間之中陳設隘狹,請更接續修建,成十一間,以備十一室薦饗之所。其三太后廟,即於少府監取西南屋三間,以備三室告饗所。」敕旨從之。



 大順元年,將行禘祭,有司請以三太后神主祔饗於太廟。三太后者,孝明太皇太后鄭氏,宣宗之母也;恭僖皇太后王氏,敬宗之母也;貞獻皇太后蕭氏,文宗之母也。三後之崩,皆作神主,有故不當入太廟。當時禮官建議並置別廟,每年五享,及三年一祫,五年一禘,皆於本廟行事,無奉神主
 入太廟之文。至是亂離之後,舊章散失,禮院憑《曲臺禮》,欲以三太后祔享太廟。博士殷盈孫獻議非之,曰:



 臣謹按三太后,憲宗、穆宗之後也。二帝已祔太廟,三後所以立別廟者,不可入太廟故也。與帝在位,皇后別廟不同。今有司悮用王彥威《曲臺禮》,禘別廟太后於太廟,乖戾之甚。臣竊究事體,有五不可。



 《曲臺禮》云:「別廟皇后,禘祫於太廟,祔於祖姑之下。」此乃皇后先崩,已造神主,夫在帝位,如昭成、肅明、元獻、昭德之比。昭成、肅明之崩也,睿
 宗在位。元獻之崩也,玄宗在位。昭德之崩也,肅宗在位。四后於太廟未有本室,故創別廟,當為太廟合食之主,故禘祫乃奉以入饗。其神主但題云「某謚皇后」,明其後太廟有本室,即當遷祔,帝方在位,故皇后暫立別廟耳。本是太廟合食之祖,故禘祫乃升,太廟未有位,故祔祖姑之下。今恭僖、貞獻二太后,皆穆宗之後。恭僖,會昌四年造神主,合祔穆宗廟室。時穆宗廟已祔武宗母宣懿皇后神主,故為恭僖別立廟,其神主直題云皇太后,明
 其終安別廟,不入太廟故也。貞獻太后,大中元年作神主,立別廟,其神主亦題為太后,並與恭僖義同。孝明,咸通五年作神主,合祔憲宗廟室。憲宗廟已祔穆宗之母懿安皇后,故孝明亦別立廟,是懿宗祖母,故題其主為太皇太后。與恭僖、貞獻亦同,帝在位,後先作神主之例。今以別廟太后神主,禘祭升享太廟,一不可也。《曲臺禮別廟皇后禘祫於太廟儀注》云:「內常侍奉別廟皇后神主,入置於廟庭,赤黃褥位。奏云『某謚皇后禘祫祔享太
 廟』,然後以神主升。」今即須奏云「某謚太皇太后」。且太廟中皇后神主二十一室,今忽以太皇太后入列於昭穆,二不可也。若但云「某謚皇后」,則與所題都異,神何依憑?此三不可也。《古今禮要》云:「舊典,周立姜嫄別廟,四時祭薦,及禘祫於七廟,皆祭。惟不入太祖廟為別配。魏文思甄後,明帝母,廟及寢依姜嫄之廟,四時及禘皆與諸廟同。」此舊禮明文,得以為證。今以別廟太后禘祫於太廟,四不可也。所以置別廟太后,以孝明不可與懿安並祔憲
 宗之室,今禘享乃處懿安於舅姑之上,此五不可也。



 且祫,合祭也。合猶不入太祖之廟,而況於禘乎?竊以為並皆禘於別廟為宜。且恭僖、貞獻二廟,比在硃陽坊,禘、祫赴太廟,皆須備法駕,典禮甚重,儀衛至多。咸通之時,累遇大饗,耳目相接,歲代未遙,人皆見聞,事可詢訪,非敢以臆斷也。



 或曰:「以三廟故禘、袷於別廟,或可矣,而將來有可疑焉。謹案睿宗親盡已祧,今昭成、肅明二後同在夾室,如或後代憲宗、穆宗親盡而祧,三太后神主其得
 不入夾室乎?若遇禘、袷,則如之何?對曰:此又大誤也。三太后廟若親盡合祧,但當閟而不享,安得處於夾室。禘、祫則就別廟行之,歷代已來,何嘗有別廟神主復入太廟夾室乎?禘、袷,禮之大者,無宜錯失。



 宰相孔緯曰:「博士之言是也。昨禮院所奏儀注,今已敕下,大祭日迫,不可遽改,且依行之。」於是遂以三太后祔祫太廟。達禮者譏其大謬,至今未正。



 會昌六年十一月,太常博士任疇上言:「去月十七日,饗德明、興聖廟,得廟直候論狀,稱懿祖
 室在獻祖室之上,當時雖以為然,便依行事,猶牒報監察使及宗正寺,請過示詳窺玉牒,如有不同,即相知聞奏。爾後伏檢《高祖神堯皇帝本紀》,伏審獻祖為懿祖之昭,懿祖為獻祖之穆,昭穆之位,天地極殊。今廟室奪倫,不即陳奏,然尚為茍且,罪不容誅。仍敕修撰硃儔、檢討王皞研精詳復,得報稱:『天寶二年,制追尊咎繇為德明皇帝,涼武昭王為興聖皇帝。十載,立廟。至貞元十九年,制從給事中陳京、右僕射姚南仲等一百五十人之議,
 以為禘、袷是祖宗以序之祭,凡有國者必尊太祖。今國家以景皇帝為太祖,太祖之上,施於禘、袷,不可為位。請按德明、興聖廟共成四室,祔遷獻、懿二祖。』謹尋儔等所報,即當時表奏,並獻居懿上。伏以國之大事,宗廟為先,禘、祫之禮,不當失序。四十餘載,理難尋詰。伏祈聖鑒,即垂詔敕,具禮遷正。」其月,疇又奏曰:「伏聞今月十三日敕,以臣所奏獻、懿祖二室倒置事,宜令禮官集議聞奏者。臣去月十七日,緣遇太廟祫饗太祖景皇帝
 已下群主,準貞元十九年所祔獻、懿祖於德明廟,共為四室。準元敕,各於本室行享禮。審知獻祖合居懿祖之上,昭穆方正。其時親見獻祖之室,倒居懿祖之下。於後遍校圖籍,實見差殊,遂敢聞奏。今奉敕宜令禮官集議聞奏者。臣得奉禮郎李岡、太祝柳仲年、協律郎諸葛畋李潼、檢討官王皞、修撰硃儔、博士閔慶之等七人伏稱:『謹按《高祖神堯皇帝本紀》及皇室圖譜,並武德、貞觀、永徽、開元已來諸禮著在甲令者,並云獻祖宣皇帝是神
 堯之高祖,懿祖光皇帝是神堯皇帝之曾祖,以高曾辨之,則獻祖是懿祖之父,懿祖是獻祖之子。即博士任疇所奏倒祀不虛。臣等伏乞即垂詔敕,具禮遷正。』」。其事遂行。



 僖宗自興元還京,夏四月,將行禘祭,有司引舊儀:「禘德明、興聖二廟,及懿祖、獻祖神主祔興聖、德明廟,通為四室。」黃巢之亂,廟已焚毀,及是將禘,俾議其儀。博士殷盈孫議曰:「臣以德明等四廟,功非創業,義止追封,且於今皇帝年代極遙,昭穆甚遠。可依晉韋泓『屋毀乃已』之
 例,因而廢之。」敕下百僚都省會議,禮部員外薛昭緯奏議曰:



 伏以禮貴從宜,過猶不及,祀有常典,理當據經。謹按德明追尊,實為遐遠,徵諸歷代,莫有其倫。自古典禮該詳,無逾周室。後稷實始封之祖,文王乃建極之君,且不聞後稷之前,別議立廟。以至二漢則可明征劉累,梁、魏則近有蕭、曹,稽彼簡書,並無追號。迨於興聖,事非有據。蓋以始王於涼,遂列為祖。類長沙於後漢之代,等楚元於宋高之朝,悉無尊禮之名,足為憲章之驗。重以獻
 祖、懿祖,皆非宗有德而祖有功,親盡宜祧,理當毀瘞,行於二廟,亦出一時。且武德之初,議宗廟之事,神堯聽之,太宗參之,碩學通儒,森然在列,而不議立皋陶、涼武昭之廟,蓋知其非所宜立也。尊太祖、代祖為帝,而以獻祖為宣簡公,懿祖為懿王,卒不加帝號者,謂其親盡則毀明矣。《春秋左氏傳》:孔子在陳,聞魯廟災。曰:『其桓僖乎?』已而果然。」蓋以親盡不毀,宜致天災,炯然之徵,不可忽也。據太常禮院狀所引至德二年克復後不作弘農府君
 廟神主,及晉韋泓「屋朽乃已」之議,頗為明據,深協禮經。其興聖等四室,請依禮院之議。



 奉敕敬依典禮,付所司。



 開元二十二年正月,制以籩、豆之薦,或未能備物,宜令禮官學士詳議具奏。太常卿韋縚請「宗廟之奠,每室籩、豆各加十二。又今之酌獻酒爵,制度全小,僅無一合,執持甚難,請稍令廣大。其郊祀奠獻,亦準此。仍望付尚書省集眾官詳議,務從折衷。」於是兵部侍郎張均及職方郎中韋述等建議曰:



 謹按《禮祭統》曰:「凡天之所生,地之
 所長,茍可薦者,莫不咸在。水草陸海,三牲八簋,昆蟲之異,草木之實,陰陽之物,皆備薦矣。」聖人知孝子之情深,而物類之無限,故為之節制,使祭有常禮,物有其品,器有其數。上自天子,下至公卿,貴賤差降,無相逾越,百代常行無易之道也。又按《周禮膳夫》,「掌王之食飲膳羞:食用六谷,膳用六牲,飲用六清,羞用百有二十品,珍用八物,醬用百有二十甕」,則與祭祀之物,豐省本殊。《左傳》曰:』享以訓恭儉,宴以示慈惠,恭儉以行禮,慈惠以布政。」又
 曰:「享有體薦,宴有折俎。杜預曰:「享有體薦,爵盈而不飲,豆干而不食,宴則相與食之。」享之與宴,猶且異文,祭奠所陳,固不同矣。又按《周禮》,籩人、豆人,各掌四籩、四豆之實,供祭祀與賓客,所用各殊。據此數文,祭奠不同常時,其來久矣。



 且人之嗜好,本無憑準,宴私之饌,與時遷移。故聖人一切同歸於古,難平生所嗜,非禮亦不薦也;平生所惡,是禮即不去也。《楚語》曰:「屈到嗜芰,有疾,召宗老而屬曰:『祭我必以芰。』及卒,宗老將薦芰,屈建命去之,曰:『
 祭典有之,國君有牛享,大夫有羊饋,士有豚犬之奠,庶人有魚炙之薦,籩豆脯醢,則上下安之。不羞珍異,不陳庶侈,不以私欲干國之典』遂不用。」此則禮外之食,前賢不敢薦也。今欲取甘旨之物,肥濃之味,隨所有者皆充祭用,茍逾舊制,其何限焉。雖籩豆有加,豈能備也?



 《傳》曰:「大羹不致,粢食不鑿,昭其儉也。」《書》曰:「黍稷非馨,明德惟馨。」事神在於虔誠,不求厭飫。三年一禘,不欲黷也。三獻而終,禮有成也。《風》有《採蘋》、《採蘩》,《雅》有《行葦》《泂酌》,守以忠
 信,神其舍諸!若以今之珍饌,平生所習,求神無方,何必師古。簠簋可去,而盤盂杯案當在御矣。《韶》《頀》可息,而箜篌笛笙當在奏矣。凡斯之流,皆非正物,或興於近代,或出於蕃夷,耳目之娛,本無則象,用之宗廟,後嗣何觀?欲為永式,恐未可也。且自漢已降,諸陵皆有寢宮,歲時朔望,薦以常饌,此既常行,亦足盡至孝之情矣。宗廟正禮,宜仍典故,率情變革,人情所難。



 又按舊制,一升曰爵,五升曰散。《禮器》稱:「宗廟之祭,貴者獻以爵,賤者獻以散。」此
 明貴小賤大,示之節儉。又按《國語》,觀射父曰:郊禘不過繭慄,蒸嘗不過把握。」夫神,以精明臨人者也,所求備物,不求豐大。茍失於禮,雖多何為?豈可舍先王之遺法,徇一時之所尚,廢棄禮經,以從流欲。裂冠毀冕,將安用之!且君子愛人以禮,不求茍合,況在宗廟,敢忘舊章。請依古制,庶可經久。



 禮部員外郎楊仲昌議曰:「謹按《禮》曰:『夫祭不欲煩,煩則黷;亦不欲簡,簡則怠。』又鄭玄云:『人生尚褻食,鬼神則不然。神農時雖有黍稷,猶未有酒醴。及後
 聖作為醴酪,猶存玄酒,求不忘古。』《春秋》曰:『蘋蘩、藻之菜,潢污行潦之水,可羞於王公,可薦於鬼神。



 』又曰:『大羹不和,粢食不鑿。』此明君人者,有國奉先,敬神嚴享,豈肥濃以為尚,將儉約以表誠。則陸海之物,鮮肥之類,既乖禮文之情,而變作者之法,皆充祭用,非所詳也。《易》曰:『樽酒簋貳,用缶,納約自牖。』此明祭存簡易,不在繁奢。所以一樽之酒,貳簋之奠,為明祀也。抑又聞之,夫義以出禮,禮以體政,違則有紊,是稱不經。薦肥濃則褻味有登,加籩爵
 則事非師古。與其別行新制,寧如謹守舊章?」時太子賓客崔沔、戶部郎中楊伯成、左衛兵曹劉秩等皆建議以為請依舊禮,不可改易。於是宰臣等具沔、述等議以奏。玄宗曰:「朕承祖宗休德,至於享祀粢盛,實思豐潔,禮物之具,諒在昭忠。其非芳潔不應法制者,亦不可用。」以是更令太常量加品味。韋縚又奏:「請每室加籩、豆各六,每四時異品,以當時新果及珍羞同薦。」則可之。又酌獻酒爵,玄宗令用龠升一升,合於古義,而多少適中。自是常
 依行焉。



 後漢世祖光武皇帝葬於原陵,其子孝明帝追思不已。永平元年,乃率諸侯王、公卿,正月朝於原陵,親奉先後陰氏妝奩篋笥悲慟,左右侍臣,莫不嗚咽。梁武帝父丹陽尹順之,追尊為太祖文帝,先葬丹徒,亦尊為建陵。武帝即大位後,大同十五年,亦朝於建陵,有紫雲廕覆陵上,食頃方滅。梁主著單衣介幘,設次而拜,望陵流哭,淚之所沾,草皆變色。陵傍有枯泉,至時而水流香潔。因謂侍臣曰,陵陰石虎,與陵俱創二百餘年,恨小,可
 更造碑石柱麟,並二陵中道門為三闥。園陵職司,並賜一級。奉辭諸陵,哭踴而拜。周太祖文帝葬於成陵,其子明帝初立,元年十二月,謁於成陵。



 高祖神堯葬於獻陵,貞觀十三年正月乙巳,太宗朝於獻陵。先是日,宿衛設黃麾仗周衛陵寢,至是質明,七廟子孫及諸侯百僚、蕃夷君長皆陪列於司馬門內。皇帝至小次,降輿納履,哭於闕門,西面再拜,慟絕不能興。禮畢,改服入於寢宮,親執饌,閱視高祖及先後服御之物,匍匐床前悲慟。左右侍
 御者莫不歔欷。初,甲辰之夜,大雨雪。及皇帝入陵院,悲號哽咽,百闢哀慟,是時雪益甚,寒風暴起,有蒼云出於山陵之上,俄而流布,天地晦冥。至禮畢,皇帝出自寢宮,步過司馬門北,泥行二百餘步,於是風靜雪止,雲氣歇滅,天色開霽。觀者竊議,以為孝感之所致焉。是日曲赦三原縣及從官衛士等,大闢已下,已發覺,未發覺,皆釋其罪。免民一年租賦。有八十已上,及孝子順孫、義夫節婦、鰥寡孤獨、有篤疾者,賜物各有差。宿衛陵邑中郎將、
 衛士齋員及三原令以下,各賜爵一級。丁未,至自獻陵。己酉,朝於太極殿。庚子,會群臣,奏《功成慶善》及《破陣》之樂。



 玄宗開元十七年十一月丙申,親謁橋陵。皇帝望陵涕泣,左右並哀感。進奉先縣同赤縣,以所管萬三百戶供陵寢,三府兵馬供衛,曲赦縣內大闢罪已下。戊戌,謁定陵。己亥,謁獻陵。壬寅,謁昭陵。己巳,謁乾陵。戊申,車駕還宮。大赦天下,流移人並放還,左降官移近處,百姓無出今年地稅之半。每陵取側近六鄉以供陵寢。皇帝初至
 橋陵,質明,柏樹甘露降,曙後祥煙遍空。皇帝謁昭陵,陪葬功臣盡來受饗,鳳吹釭釭,若神祇之所集。陪位文武百僚皆聞先聖嘆息、功臣蹈舞之聲,皆以為至孝所感。天寶二年八月,制:「自今已後,每至九月一日,薦衣於陵寢。」十三載,改獻、昭、乾、定、橋五陵署為臺,其署令改為臺令,加舊一級。



\end{pinyinscope}