\article{卷一百四十七}

\begin{pinyinscope}

 ○李懷
 仙硃希彩附硃滔劉怦子濟澭濟子總程日華子懷直懷直子權李全略子同捷



 李懷仙,柳城胡人也。世事契丹,降將,守營州。祿山之叛,懷仙以裨將從陷河洛。安慶緒敗,又事史思明。善騎射,
 有智數。朝義時,偽授為燕京留守、範陽尹。寶應元年,元帥雍王統回紇諸兵收復東都,朝義渡河北走,乃令副元帥僕固懷恩率兵追之。時群兇瓦解,國威方振,賊黨聞懷恩至,望風納款。朝義以餘孽數千奔範陽,懷仙誘而擒之,斬首來獻。屬懷恩私欲樹黨以固兵權,乃保薦懷仙可用。代宗復授幽州大都督府長史、檢校侍中、幽州盧龍等軍節度使,與賊將薛嵩、田承嗣、張忠志等分河朔而帥之。既而懷恩叛逆,西蕃入寇,朝廷多故,懷仙
 等四將各招合遺孽,治兵繕邑;部下各數萬勁兵,文武將吏,擅自署置;貢賦不入於朝廷,雖稱籓臣,實非王臣也。朝廷初集,姑務懷安,以是不能制。懷仙大歷三年為其麾下兵馬使硃希彩所殺。



 希彩自稱留後。恆州節度使張忠志以懷仙世舊,無辜覆族,遣將率眾討之;為希彩所敗。朝廷不獲已,宥之。以河南副元帥、黃門侍郎、同平章事王縉為幽州節度使,授希彩御史中丞,充幽州節度副使,權知軍州事,詔縉赴鎮。希彩聞縉之來,搜選
 卒伍,大陳戎備以逆之。縉晏然建旌節,而希彩迎謁甚恭。縉知終不可制,勞軍旬日而還。尋加希彩御史大夫,充幽州節度留後。十二月,加希彩幽州大都督府長史、幽州盧龍軍節度使。五年,封高密郡王。既得位,暴橫自恣,無禮於朝廷。七年,孔目官李瑗因人之怒,伺隙斬之,軍人立其兵馬使硃泚為留後。泚自有傳。



 硃滔,賊泚之弟也。平州刺史硃希彩為幽州節度,以滔同姓,甚愛之,常令將腹心親兵。及泚為節度使,遂使滔
 將勁兵三千赴京師,請率先諸軍備塞。自祿山反後,山東範陽,外雖示順,實皆倔強不庭。泚首效臣節,代宗喜甚,命滔勒兵東入長安通化門,西出開遠門,出師勞還;未有兵還王城者,今而許之,蓋示優異。召滔對於三殿,代宗臨軒勞問。既而曰:「卿材孰與泚多?」滔曰:「各有長短。統御士眾,方略明辨,臣不及泚;臣年二十八,獲謁龍顏,泚長臣五歲,未朝鳳闕,此不及臣。」代宗愈喜。



 大歷九年,泚朝覲,因乞留西征吐蕃。以滔試殿中監,權知幽州盧
 龍節度留後、兼御史大夫。及田承嗣反,與李寶臣、李正己等解磁州圍。建中二年,寶臣死,其子惟岳謀襲父位。滔與成德軍節度張孝忠征之,大破惟岳於束鹿。滔命偏師守束鹿,進圍深州。惟岳乃統萬餘眾及田悅援兵圍束鹿。惟岳將王武俊以騎三千方陳橫進。滔繪帛為狻猊象,使猛士百人蒙之,鼓噪奮馳,賊為驚亂,隨擊,大破之,惟岳焚營而遁。以功加檢校司徒,為幽州盧龍軍節度使,以德、棣二州隸焉。朝廷以康日知為深趙二州
 團練使,王武俊為恆冀二州團練使。滔怒失深州,武俊怒失寶臣故地,滔構武俊同己反。馬燧圍田悅於魏州,悅告急,滔與武俊遂連兵救悅,敗李懷光於愜山。三年十一月,滔僭稱大冀王,偽署百官,與李納、田悅、王武俊並稱王,南結李希烈。興元初,田悅、王武俊以硃泚據京師,滔兵強盛,首尾相應,田悅常謂武俊曰:「硃滔心險,不可堤防。」遂相率歸順。



 泚既僭號,立滔為皇太弟,仍令以重賂招誘回紇,南攻魏、貝,即西入關。興元元年正月,滔
 驅率燕、薊之眾及回紇雜虜,號五萬,次南河,攻圍貝州。三月,田緒殺田悅,魏州亂。滔令大將馬實分兵逼魏州,營於王莽河。德宗在山南,慮二兇兵合,遣使授王武俊平章事,令與李抱真葉力擊滔。四月,恆、潞兩軍次涇城北,行營相距十里;抱真自率二百騎徑入武俊軍,面申盟約,結為兄弟。五月四日,進軍距貝州三十里而軍。翌日,滔令大將馬實、盧南史引回紇、契丹來挑戰,武俊遣騎將趙珍提精騎三百當之,抱真將王虔休掎角待之。
 武俊與其子士清自當回紇、契丹部落。兩軍既合,鼓噪震地,回紇恃捷,穿武俊陣而過。武俊乘騎勒馬不動,俟回紇引退,因而薄之,回紇勢不能止。武俊父子縱馬急擊,獲回紇三百騎。滔陣亂,東走,兩邊追斬,俘馘數萬計。遇夜,夾滔壘而軍。是夜,滔以殘眾千人奔德州,委棄戈甲山積。滔至瀛州,殺騎將蔡雄、揚布。以其前鋒先敗,又殺陰陽人尹少伯,以其言舉兵必勝故也。



 六月,李晟收京城,硃泚、姚令言死。滔還幽州,為武俊所攻,僅不能軍,
 上章待罪。九月,詔曰:「硃滔累獻款疏,深效懇誠,省之惻然,良用憫嘆!宜委武俊、抱真開示大信,深加曉諭。若誠心益固,善跡克彰,朕當掩釁錄勛,與之昭雪。」貞元元年,尋卒於位,時年四十,贈司徒。



 劉怦,幽州昌平人也。父貢,嘗為廣邊大斗軍使。怦即硃滔姑之子,積軍功為雄武軍使,廣屯田,節用,以辦理稱。稍遷涿州刺史。居數年,硃滔將兵討田承嗣,奏署怦領留府事,以寬緩得眾心。時李寶臣為田承嗣間說,與之
 通謀。承嗣又以滄州與寶臣,乃以兵劫硃滔於瓦橋關,滔脫身走,乘勝欲襲取幽州。怦設方略鎮撫,寶臣不敢進,以功加御史中丞。



 寶臣死,子惟岳拒朝命,德宗令滔與張孝忠同力討之。及惟岳平,滔怨朝廷違約不與深州,含怒不已。會王武俊亦怨割地深、趙,相謀叛,欲救田悅。怦時知幽州留後事,遣人齎書謂滔曰:「司徒位崇太尉,尊居宰相,恩寵冠籓臣之右,榮遇極矣!今昌平故里,朝廷改為尉卿、司徒里,此亦大夫不朽之名也。但以忠
 順自持,則事無不濟。竊思近日,務大樂戰,不顧成敗,而家滅身屠者,安、史是也。暴亂易亡,今復何有?怦忝密親,世荷恩遇,默而無告,是負重知。惟司徒圖之,無貽後悔也!」滔雖不用其言,亦嘉其盡言,卒無疑貳。凡出征伐,必以怦總留後事。及僭稱大冀王,偽署怦為右僕射、範陽留守。及泚據京邑,召滔南河,至貝州,挫敗而還,兵甲盡喪。怦聞滔將至,悉蒐範陽兵甲,夾道排列二十餘里,以迎滔歸於府第,人皆嘉怦忠義。



 貞元元年,滔卒,三軍推
 怦權撫軍府事。怦為眾所服,卒有其地。朝廷因授怦幽州大都督府長史、兼御史大夫、幽州盧龍節度副大使、知節度事、管內營田觀察、押奚契丹、經略盧龍軍使。居位三月,以貞元元年九月卒,年五十九,廢朝三日,贈兵部尚書,賜布帛有差。子濟繼為幽州節度使。



 濟,怦之長子。初,母難產;既產,侍者初見濟是一大蛇,黑氣勃勃,莫不驚走。及長,頗異常童。所居室焚,人皆驚救,濟從容而出,眾異之。累歷本管州縣牧宰。及怦為節度使,以濟兼
 御史中丞,充行軍司馬。怦卒,軍人習河朔舊事,請濟代父為帥,朝廷姑務便安,因而從之。累加至檢校兵部尚書。



 貞元五年,遷左僕射,充幽州節度使。時烏桓、鮮卑數寇邊,濟率軍擊走之;深入千餘里,虜獲不可勝紀,東北晏然。貞元中,朝廷優容籓鎮方甚,兩河擅自繼襲者,尤驕蹇不奉法。惟濟最務恭順,朝獻相繼,德宗亦以恩禮接之。尋加同中書門下平章事。順宗即位,再遷檢校司徒。元和初,加兼侍中。及詔討王承宗,諸軍未進,濟獨率
 先前軍擊破之,生擒三百餘人,斬首千餘級,獻逆將於闕,優詔褒之。又為詩四韻上獻,以表忠憤之志。明年春,將大軍次瀛州,累攻樂壽、博陸、安平等縣,前後大獻俘獲。賞功頗厚,仍與子孫六品官者凡四人。未幾,有疾,會赦承宗,錄功拜兼中書令。濟在鎮二十餘年,雖輸忠款,竟不入覲。又謀殺其弟澭,澭歸國為信臣。及濟疾,次子總與濟親吏唐弘實通謀鴆殺濟,數日,乃發喪。時年五十四,詔贈太師,廢朝三日,賻禮有加,謚曰莊武。



 弟源,貞
 元十六年八月,為檢校工部尚書,兼左武衛將軍。初,為涿州刺史,不受兄教令,濟奏之,貶漠州參軍,復不受詔。濟帥師至涿州,源出兵拒之,未合而自潰。濟擒源至幽州,上言請令入覲,故授官以征之。



 澭,濟之異母弟也。喜讀書,工武藝,輕財愛士,得人死力。事硃滔,常陳逆順之理。後怦為盧龍軍節度使,病將卒,澭在父側,即以父命召兄濟自漠州至,竟得授節度使。濟常感澭奉己,



 澭為瀛州刺史,亦許以澭代己任;其後濟乃以其子為副大
 使。澭既怒濟,遂請以所部西捍隴塞,拔其所部兵一千五百人、男女萬餘口直趨京師,在道無一人犯令者。德宗寵遇,特授秦州刺史,以普潤縣為理所。



 及順宗傳位,稱太上皇,有山人羅令則詣澭言異端數百言,皆廢立之事,澭立命系之。令則又云某之黨多矣,約以德宗山陵時伺便而動。澭械令則送京師,杖死之。後錄功,賜其額曰保義。其軍蕃戎畏之,不敢為寇,常有復河湟之志,議者壯之。元和二年十二月,卒。



 總,濟之第二子也,性陰
 賊險譎。元和五年,濟奉詔討王承宗,使長子緄假為副使,領留務。時總為瀛州刺史,濟署為行營都兵馬使,屯軍饒陽,師久無功。總潛伺其隙,與判官張、孔目官成國寶及帳內小將為謀,使詐自京至,曰:「朝廷以相公逗留不進,除副大使為節度使矣。」明日,又使人曰:「副大使旌節已到太原。」又使人走而呼曰:「旌節過代州。」舉軍驚恐。濟驚惶憤怒,不知所為,因殺主兵大將數十人及與緄素厚者。乃追緄,以張兄皋代知留務。濟自朝至日
 晏不食,渴索飲,總因置毒而進之。濟死,緄行至涿州,總矯以父命杖殺之,總遂領軍務。朝廷不知其事,因授以斧鉞,累遷至檢校司空。



 及王承宗再拒命,總遣兵取賊武強縣,遂駐軍持兩端,以利朝廷供饋賞賜。是時吳元濟尚存,王承宗方跋扈,易定孤危,憲宗暫務姑息,加總同中書門下平章事。及元濟就擒,李師道梟首,王承宗憂死,田弘正入鎮州,總既無黨援,懷懼,每謀自安之計。初,總弒逆後,每見父兄為祟,甚慘懼,乃於官署後置數
 百僧,厚給衣食,令晝夜乞恩謝罪。每公退,則憩於道場,若入他室,則恟惕不敢寐。晚年恐悸尤甚,故請落發為僧,冀以脫禍,乃以判官張皋為留後。總以落發,上表歸朝,穆宗授天平軍節度使;既聞落發,乃賜紫,號大覺師。總行至易州界,暴卒。輟朝五日,贈太尉,擇日備禮冊命,賻絹布一千五百段、米粟五百石。



 先是,元和初,王承宗阻兵,總父濟備陳征伐之術,請身先之。及出軍,累拔城邑,旋屬被病,不克成功。總既繼父,願述先志,且欲盡更
 河朔舊風。長慶初,累疏求入覲,兼請分割所理之地,然後歸朝。其意欲以幽、涿、營州為一道,請弘靖理之;瀛州、漠州為一道,請盧士玫理之;平、薊、媯、檀為一道,請薛平理之。仍籍軍中宿將盡薦於闕下,因望朝廷升獎,使幽薊之人皆有希羨爵祿之意。及疏上,穆宗且欲速得範陽,宰臣崔植、杜元穎又不為久大經略,但欲重弘靖所授,而未能省其使局,惟瀛、漠兩州許置觀察使,其他郡縣悉命弘靖統之。時總所薦將校,又俱在京師旅舍中,
 久而不問。如硃克融輩,僅至假衣丐食,日詣中書求官,不勝其困。及除弘靖,又命悉還本軍。克融輩雖得復歸,皆深懷觖望,其後果為叛亂。



 總既以土地歸國,授其弟約及男等一十一人,領郡符,加命服者五人,升朝班,佐宿衛者六人。



 程日華,定州安喜人,本單名華。父元皓,事安祿山為帳下將,從陷兩京,頗稱勇力,史思明時為定州刺史。華少事本軍,為張孝忠牙將。



 初,李寶臣授恆州節度,吞削籓
 鄰,有恆、冀、深、趙、易、定、滄、德等八州。寶臣既卒,惟岳拒朝命,以圖繼襲。寶臣部將張孝忠以定州歸國,授成德軍節度使,令與硃滔討惟岳。及惟岳誅,朝廷以恆、冀授王武俊,深、趙授康日知,易、定、滄授張孝忠,分為三帥。時惟岳將李固烈守滄州,孝忠令華詣固烈交郡。固烈將歸真定,悉取滄州府藏,累乘而還。軍人怒,殺固烈,皆奪其財,相與詣華曰:「李使君貪鄙而死,軍州請押牙權領。」不獲已,從之。孝忠因授華知滄州事。未幾,硃滔合武俊謀
 叛,滄、定往來艱阻,二盜遂欲取滄州,多遣人游說,又加兵攻圍,華俱不聽從,乘城自固。久之,錄事參軍李宇為華謀曰:「使君受圍累年,張尚書不能致援,論功獻捷,須至中山,所謂勞而無功者也。請為足下至京師,自以一州為使。」華即遣之。宇入闕,備陳華當二盜之間,疲於矢石。德宗深嘉之,拜華御史中丞、滄州刺史。復置橫海軍,以華為使。尋加工部尚書、御史大夫,賜名日華,仍歲給義武軍糧餉數萬。自是別為一使,孝忠唯有易、定二州
 而已。



 武俊遣人說華歸己,華曰:「相公欲敝邑仍舊隸恆州,且借騎二百以抗賊,俟道路通即從命。」武俊喜,即以二百騎助之。華乃留其馬,遣人皆還。武俊怒其背約,又以硃滔方攻圍,慮為所有而止。及武俊歸國,河朔無事,日華即遣所留馬還武俊,別陳珍幣謝過,武俊歡然而釋。貞元四年卒,贈兵部尚書。子懷直。



 懷直習河朔事,父卒,自知留後事。朝廷嘉父之忠,起復授檢校工部尚書、兼御史大夫,升橫海軍為節度,以懷直為留後。又於弓
 高縣置景州,管東光、景城二縣,以為屬郡。累加至檢校尚書右僕射。五年,起復正授節度觀察使。



 懷直荒於畋獵,數日方還,不恤軍政,軍士不勝寒餒。其帳下將從父兄懷信因眾怒閉門不內,懷直因來朝覲,貞元九年也。德宗優容之,依前檢校右僕射,兼龍武統軍,賜安業里甲第,妓女一人。既而懷信死,懷直子執恭知留後事,乃遣懷直歸滄州。十六年卒,年四十九,廢朝一日,贈揚州大都督。



 執恭代襲父位,朝廷因而授之。元和六年入朝,
 憲宗禮遇遣之,加尚書左僕射。嘗夢滄州衙門樓額悉帖「權」字,遂奏請改名權。十三年,淮西賊平,籓方惕息,權以父子世襲如三鎮事例,心不自安,乃請入朝。十三年,至京師,表辭戎帥,因命華州刺史鄭權代之,以靖安里私第側狹,賜地二十畝,令廣其居。尋遷檢校司空、邠州刺史、邠寧節度使。十四年十一月卒,贈司徒。權兄弟子侄在朝列宿衛者三十餘人。



 李全略者,本姓王,名日簡。為鎮州小將,事王武俊。元和
 中,節度使王承宗沒,軍情不安,自拔歸朝,授代州刺史。及長慶初,鎮州軍亂,殺田弘正;穆宗為之旰食,以日簡嘗為鎮將,召問其計。日簡遂於御前極言利害,兼願有以自效,因授德州刺史,經略其事。明年,擢拜橫海軍節度使,賜姓李氏,名全略,以崇樹之。未幾,令子同捷入侍,兼進錢千萬。逾歲,同捷歸覲,乃奏請授滄州長史、知州事,兼主中軍兵馬;朝廷初不之許,後慮其有奇策,將副經略之旨,遂從之。及得請,全略乃陰結軍士,潛為久計,
 外示忠順,內畜奸謀。棣州刺史王稷善撫眾,且得其心,全略忌而殺之,仍孥戮其屬。凡所為事,大率類此。寶歷二年四月卒。



 子同捷,初為副大使,居喪,擅領留後事,仍重賂籓鄰以求纘襲,朝廷知其所為,經年不問。屬昭愍晏駕,文宗即位,同捷冀易世之後,稍行恩貸,即令母弟同志、同巽入朝,令掌書記崔長奉表,備達懇誠,請從朝旨。詔授同捷檢校左散騎常侍、兗州刺史、兗海節度使;以天平節度使烏重胤為滄州節度以代之。詔下,同捷
 托以三軍乞留,拒命。乃命烏重胤率鄆、齊兵加討。又詔徐帥王智興、滑帥李聽、平盧康志睦、魏博史憲誠、易定張璠、幽州李載義等四面進攻。



 同捷世行奸詐,自以嘗在成德軍為將校,燕、趙之師,可結為城社,乃以玉帛子女賂河北三鎮,以求旄鉞。李載義初受朝命,堅於效順,乃囚同捷侄及所賂玉帛妓女四十七人表獻。又表朝廷加載義左僕射、王廷湊司徒,以悅其心事。廷湊本蓄狼心,欲吞橫海,乃出兵於境以赴同捷。



 王智興師次棣
 州,詔曰:「李同捷幸襲舊勛,不思纘緒,斬麻未幾,私行墨縗。毒殺忠良,擾惑部校,稽之國憲,難逭常刑。朕以頃在先朝,己稽中旨,實遵成命,未議改圖。乃由留務之權,授以戎帥;拔負海之陋,置之中華,推恩含垢,斯亦至矣!而同捷益懷迷執,閉境練兵,大詬鄰封,拒捍中使。遐邇憤怨,中外驚嗟,叛命既彰,大義當絕,事非獲已,良用憮然。其同捷在身官爵,並宜削奪,令諸軍進討。」俄而烏重胤卒,授神策節度使李寰代重胤出師,無功召還,乃加王
 智興平章事,充行營招撫使。史憲誠遣大將丌志沼與子唐帥兵二萬五千攻德州。太和二年九月,智興收棣州,因割隸淄青。時諸軍在野,朝廷特置供軍糧料使,日費浸多。兩河諸帥每有小捷,虛張俘級,以邀賞賚,實欲困朝廷而緩賊也;繒帛征馬,賜之無算。



 同捷既窘,王廷湊援之不及,乃令人誘丌志沼,俾倒戈攻憲誠,許以代為魏博節度。志沼信其言而叛。憲誠告難,詔李聽以諸道兵攻之。志沼敗,奔於鎮州。李寰赴闕,又以李祐代為
 橫海節度。三年三月,詔諫議大夫柏耆軍前慰撫。四月,李祐收德州。同捷乞降於祐,祐疑其詐;柏耆請以騎兵三百入滄州,祐從之。耆徑入滄州,取同捷與其家屬赴京師。其月二十六日,至德州界,諜言廷湊兵來劫篡,耆乃斬同捷首,傳而獻捷,百僚稱賀。同捷母孫、妻崔、兒元逵等既獻,詔悉宥之,配於湖南安置。



 史臣曰:國家崇樹籓屏,保界山河,得其人則區宇以寧,失其授則干戈勃起。若懷仙之輩,習亂河朔,志深狡蠹,
 忠義之談,罔經耳目;以暴亂為事業,以專殺為雄豪,或父子弟兄,或將帥卒伍,迭相屠滅,以成風俗。斯乃王道浸微,教化不及。惜哉蒸民,陷彼虎吻!其間劉總,粗貯臣誠,然而殺父兄以圖榮,落鬢發而避禍;未旋踵而暴卒他境,斯謂報應之驗與!



 贊曰:國法不綱,賊臣鴟張。雖曰父子,兇如虎狼。惡稔族滅,身屠地亡。蠢茲伏莽,污我彞章。



\end{pinyinscope}