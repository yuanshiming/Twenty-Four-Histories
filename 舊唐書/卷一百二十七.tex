\article{卷一百二十七}

\begin{pinyinscope}

 ○劉晏第五琦班宏王紹李巽



 劉晏,字士安,曹州南華人。年七歲,舉神童,授秘書省正字。累授夏縣令,有能名。歷殿中侍御史,遷度支郎中、杭隴華三州刺史,尋遷河南尹。時史朝義盜據東都,寄理
 長水。入為京兆尹,頃之,加戶部侍郎、兼御史中丞,判度支,委府事於司錄張群、杜亞,綜大體,議論號為稱職。無何,為酷吏敬羽所構,貶通州刺史。復入為京兆尹、戶部侍郎,判度支。時顏真卿以文學正直出為利州刺史,晏舉真卿自代為戶部,乃加國子祭酒。寶應二年,遷吏部尚書、平章事,領度支鹽鐵轉運租庸使。坐與中官程元振交通,元振得罪,晏罷相,為太子賓客。尋授御史大夫,領東都、河南、江淮、山南等道轉運租庸鹽鐵使如故。時
 新承兵戈之後,中外艱食,京師米價鬥至一千,官廚無兼時之積,禁軍乏食,畿縣百姓乃挼穗以供之。晏受命後,以轉運為己任,凡所經歷,必究利病之由。至江淮,以書遺元載曰:



 浮於淮、泗,達於汴,入於河,西循底柱、硤石、少華,楚帆越客,直抵建章、長樂,此安社稷之奇策也。晏賓於東朝,猶有官謗,相公終始故舊,不信流言,賈誼復召宣室,弘羊重興功利,敢不悉力以答所知。驅馬陜郊,見三門渠津遺跡。到河陰、鞏、洛,見宇文愷置梁公堰,分
 黃河水入通濟渠;大夫李傑新堤故事,飾像河廟,凜然如生。涉滎郊、浚澤,遙瞻淮甸,步步探討,知昔人用心,則潭、衡、桂陽必多積穀,關輔汲汲,只緣兵糧。漕引瀟、湘、洞庭,萬里幾日,淪波卦席,西指長安。三秦之人,待此而飽;六軍之眾,待此而強。天子無側席之憂,都人見泛舟之役;四方旅拒者可以破膽,三河流離者於茲請命。相公匡戴明主,為富人侯,此今之切務,不可失也。使僕湔洗瑕穢,率罄愚懦,當憑經義,請護河堤,冥勤在官,不辭水
 死。



 然運之利病,各有四五焉。晏自尹京入為計相,共五年矣。京師三輔百姓,唯苦稅畝傷多,若使江、湖米來每年三二十萬,即頓減徭賦,歌舞皇澤,其利一也。東都殘毀,百無一存。若米運流通,則饑人皆附,村落邑厘,從此滋多。受命之日引海陵之倉以食鞏、洛,是計之得者,其利二也。諸將有在邊者,諸戎有侵敗王略者,或聞三江、五湖,貢輸紅粒,雲帆桂楫,輸納帝鄉,軍志曰:「先聲後實,可以震耀夷夏。」其利三也。自古帝王之盛,皆云書
 同文,車同軌,日月所照,莫不率俾。今舟車既通,商賈往來,百貨雜集,航海梯山,聖神輝光,漸近貞觀、永徽之盛,其利四也。



 所可疑者,函、陜凋殘,東周尤甚。過宜陽、熊耳,至武牢、成皋,五百里中,編戶千餘而已。居無尺椽,人無煙爨,蕭條淒慘,獸游鬼哭。牛必羸角,輿必說輹,棧車輓漕,亦不易求。今於無人之境,興此勞人之運,固難就矣,其病一也。河、汴有初,不修則毀澱,故每年正月發近縣丁男,塞長茭,決沮淤,清明桃花已後,遠水自然安流,陽侯、宓妃,
 不復太息。頃因寇難,總不掏拓,澤滅水,岸石崩,役夫需於沙,津吏旋於濘,千里洄上,罔水舟行,其病二也。東垣、底柱,澠池、二陵,北河運處五六百里,戍卒久絕,縣吏空拳。奪攘奸宄,窟穴囊橐。夾河為藪,豺狼狺狺,舟行所經,寇亦能往,其病三也。東自淮陰,西臨蒲阪,亙三千里,屯戍相望。中軍皆鼎司元侯,賤卒儀同青紫,每云食半菽,又云無挾纊,輓漕所至,船到便留,即非單車使折簡書所能制矣,其病四也。惟小子畢其慮奔走之,惟中書詳
 其利病裁成之。



 晏累年已來,事缺名毀,聖慈含育,特賜生全。月餘家居,遽即臨遣,恩榮感切,思殞百身。見一水不通,願荷鍤而先往;見一粒不運,願負米而先趨。焦心苦形,期報明主,丹誠未克,漕引多虞,屏營中流,掩泣獻狀。



 自此每歲運米數十萬石以濟關中。



 又至德初,為國用不足,令第五琦於諸道榷鹽以助軍用,及晏代其任,法益精密,官無遺利。初,歲入錢六十萬貫,季年所入逾十倍,而人無厭苦。大歷末,通計一歲征賦所入總一
 千二百萬貫,而鹽利且過半。累遷吏部尚書。大歷四年六月,與右僕射裴遵慶同赴本曹視事,敕尚食增置儲供,許內侍魚朝恩及宰臣已下常朝官咸詣省送上。八年,知三銓選事。十二年三月,誅宰臣元載,晏奉詔訊鞫。晏以載居任樹黨,布於天下,不敢專斷,請他官共事。敕御史大夫李涵、右散騎常侍蕭昕、兵部侍郎袁傪、禮部侍郎常袞、諫議大夫杜亞同推,載皆款伏。初,晏承旨,門下侍郎、同平章事王縉亦處極法,晏謂涵等曰:重刑再
 覆,國之常典,況誅大臣,得不覆奏?又法有首從,二人同刑,亦宜重取進止。」涵等從命。及晏等覆奏,代宗乃減縉罪從輕。縉之生,晏平反之力也。



 十三年十二月,為尚書左僕射。時宰臣常袞專政,以晏久掌銓衡,時議平允,兼司儲蓄,職舉功深,慮公望日崇,上心有屬。竊忌之,乃奏晏朝廷舊德,宜為百吏師長,外示崇重,內實去其權。及奏上,以晏使務方理,代其任者難其人,使務、知三銓並如故。李靈曜之亂也,河南節帥所據,多不奉法令,征賦亦
 隨之;州縣雖益減,晏以羨餘相補,人不加賦,所入仍舊,議者稱其能。自諸道巡院距京師,重價募疾足,置遞相望,四方物價之上下,雖極遠不四五日知,故食貨之重輕,盡權在掌握,朝廷獲美利而天下無甚貴甚賤之憂,得其術矣。凡所任使,多收後進有幹能者。其所總領,務乎急促,趨利者化之,遂以成風。當時權勢,或以親戚為托,晏亦應之,俸給之多少,命官之遲速,必如其志,然未嘗得親職事。其所領要務,必一時之選,故晏沒後二十餘
 年,韓洄、元琇、裴腆、包佶、盧征、李衡繼掌財賦,皆晏故吏。其部吏居數千里之外,奉教令如在目前,雖寢興宴語,而無欺紿,四方動靜,莫不先知,事有可賀者,必先上章奏。江淮茶、橘,晏與本道觀察使各歲貢之,皆欲其先至。有土之官,或封山斷道,禁前發者,晏厚以財力致之,常先他司,由是甚不為籓鎮所便。



 晏理家以儉約稱,而重交敦舊,頗以財貨遺天下名土,故人多稱之。善訓諸子,咸有學藝。任事十餘年,權勢之重,鄰於宰相,要官重職,
 頗出其門。既有材力,視事敏速,乘機無滯,然多任數,挾權貴,固恩澤,有口者必利啖之。當大歷時,事貴因循,軍國之用,皆仰於晏,未嘗檢轄。



 德宗嗣位,言事者稱轉運可罷多矣。初,楊炎為吏部侍郎,晏為尚書,各恃權使氣,兩不相得。炎坐元載貶,晏快之,昌言於朝。及炎入相,追怒前事,且以晏與元載隙憾,時人言載之得罪,晏有力焉。炎將為載復仇,又時人風言代宗寵獨孤妃而又愛其子韓王迥,晏密啟請立獨孤為皇后。炎因對易又流涕奏言:「
 賴祖宗福祐,先皇與陛下不為賊臣所間。不然,劉晏、黎幹之輩,搖動社稷,兇謀果矣。今乾以伏罪,晏猶領權,臣為宰相,不能正持此事,罪當萬死。」崔祐甫奏言:「此事曖昧,陛下以廓然大赦,不當究尋虛語。」硃泚、崔寧又從傍與祐甫救解之,寧言頗切,炎大怒,故斥寧令出鎮鄜坊以摧挫之。遂罷晏轉運等使,尋貶為忠州刺史。炎欲誣構其罪,知庾準與晏素有隙,舉為荊南節度,以伺晏動靜。準乃奏晏與硃泚書祈救解,言多怨望,炎又證成其
 事,上以為然。是月庚午,晏已受誅,使回奏報,誣晏以忠州謀叛,下詔暴言其罪,時年六十六,天下冤之。家屬徙嶺表,連累者數十人。貞元五年,上悟,方錄晏子執經,授太常博士;少子宗經,秘書郎。執經上請削官贈父,特追贈鄭州刺史。



 第五琦,京兆長安人。少孤,事兄華,敬順過人。及長,有吏才,以富國強兵之術自任。天寶初,事韋堅,堅敗貶官。累至須江丞,時太守賀蘭進明甚重之。會安祿山反,進明遷
 北海郡太守,奏琦為錄事參軍。祿山已陷河間、信都等五郡,進明未有戰功,玄宗大怒,遣中使封刀促之,曰:「收地不得,即斬進明之首。」進明惶懼,莫知所出,琦乃勸令厚以財帛募勇敢士,出奇力戰,遂收所陷之郡。令琦奏事,至蜀中,琦得謁見,奏言:「方今之急在兵,兵之強弱在賦,賦之所出,江淮居多。若假臣職任,使濟軍須,臣能使賞給之資,不勞聖慮。」玄宗大喜,即日拜監察御史,勾當江淮租庸使。尋拜殿中侍御史。尋加山南等五道度
 支使,促辦應卒,事無違闕。遷司金郎中、兼御史中丞,使如故。於是創立鹽法,就山海井灶收榷其鹽,官置吏出糶。其舊業戶並浮人願為業者,免其雜徭,隸鹽鐵使,盜煮私市罪有差。百姓除租庸外,無得橫賦,人不益稅而上用以饒。遷戶部侍郎、兼御史丞,專判度支,領河南等道支度都勾當轉運租庸鹽鐵鑄錢、司農太府出納、山南東西江西淮南館驛等使。



 乾元二年,以本官加同中書門下平章事。初,琦以國用未足,幣重貨輕,乃請鑄乾
 元重寶錢,以一當十行用之。及作相,又請更鑄重輪乾元錢,一當五十,與乾元錢及開元通寶錢三品並行。既而穀價騰貴,餓殣死亡,枕藉道路,又盜鑄爭起,中外皆以琦變法之弊,封奏日聞。乾元二年十月,貶忠州長史,既在道,有告琦受人黃金二百兩者,遣御史劉期光追按之。琦對曰:二百兩金十三斤重,忝為宰相,不可自持。若其付受有憑,即請準法科罪。」期光以為此是琦伏罪也,遽奏之,請除名,配流夷州,馳驛發遣,仍差綱領送至
 彼。寶應初,起為朗州刺史,甚有能政,入遷太子賓客。屬吐蕃寇陷京師,代宗幸陜,關內副元帥郭子儀請琦為糧料使、兼御史大夫,充關內元帥副使。未幾,改京兆尹。車駕克復,專判度支,兼諸道鑄錢鹽鐵轉運常平等使。累封扶風郡公。又加京兆尹,改戶部侍郎,判度支。前後領財賦十餘年。魚朝恩伏誅,琦坐與款狎,出為處州刺史,歷饒、湖二州。入為太子賓客、東都留司。上以其材,將復任用,召還京師,信宿而卒,年七十,贈太子少保。



 子峰,
 峰婦鄭氏女,皆以孝著,旌表其門。



 班宏,衛州汲人也。祖思簡,春官員外郎。父景倩,秘書監。宏少舉進士,授右司御胄曹,後為薛景先鳳翔掌書記,又為高適劍南觀察判官,累拜大理司直,攝監察御史。時青城山有妖賊張安居以左道惑眾,事覺,多誣引大將,冀以緩死,宏驗理而速殺之,人心乃安。既而郭英乂代適,以厭人望,奏署秘書郎,兼雒令,以疾免。大歷三年,遷起居舍人,尋兼理匭使,四遷至給事中。時李寶臣卒
 於其位,子惟岳匿喪求位,上遣宏使成德問疾,且喻之。惟岳厚賂宏,皆不受,還報合旨,遷刑部侍郎,兼京官考使。時右僕射崔寧考兵部侍郎劉乃上下,宏駁曰:「夷荒靖難,專在節制,尺籍伍符,不校省司。夫上行宣美之名,則下開趨競之路;上行阿容,下必朋黨。」因削去之。乃知而謝曰:「乃雖不敏,敢掠一美以徼二罪乎?」尋除吏部侍郎,為汪蕃會盟使李揆之副。



 貞元初,仍歲旱蝗,上以賦調為急,改戶部侍郎,為度支使韓滉之副。遷尚書,復副竇
 參。參初為大理司直,宏已為刑部侍郎,及參為相,領度支,上以宏久司國計,因令副之。且曰:「朕藉參宰相以臨遠,眾務悉委於卿,勿以辭也。」參以先貴,常私解悅之曰:「參後來,一朝居尚書之上,甚不自安,一年之後,當歸此使。」宏心喜,歲餘,參絕不復言。宏性剛愎,為人間之,且怒食言,公事多異。揚子院,鹽鐵轉運委藏也,宏以御史中丞徐粲主之,既不理,且以賄聞,參欲代之,宏執不可。參又選諸院吏,未嘗訪宏,乃疏參所用者過惡以聞,事
 輒留中。無何,參以使勞加吏部尚書,而宏進封蕭國公,怨參以虛號寵之,間惡愈甚。每奉詔營建,宏必極壯麗,親程課役,又厚結權幸以傾參。



 張滂先善於宏,宏薦為司農少卿,及參欲以滂分掌江淮鹽鐵,詢之於宏,宏以滂嫉惡,慮以法繩徐粲,因曰:「滂強戾難制,不可用。」滂知之。八年三月,參遂為上所疏,乃讓度支使,遂以宏專判,而參不欲使務悉歸於宏,問計京兆尹薛玨,玨曰:「二子交惡,而滂剛決,若分鹽鐵轉運於滂,必能制宏。」參乃薦
 滂為戶部侍郎、鹽鐵使、判轉運,尚隸於宏以悅之。江淮兩稅,悉宏主之,置巡院,然令宏、滂共擇其官。滂請鹽鐵舊簿書於宏,宏不與之。每署院官,宏、滂更相是非,莫有用者。滂乃奏曰:「班宏與臣相戾,巡院多闕官。臣掌財賦,國家大計,職不修,無所逃罪。今宏若此,何以輯事?」遂令分掌之。無幾,宏言於宰相趙憬、陸贄曰:「宏職轉運,年運江淮米五十萬斛,前年增七十萬斛,以實太倉,幸無過。今職移於人,不知何謂?」滂時在側,忿然曰:「尚書失言
 甚矣!若運務畢舉,朝廷固不奪之,蓋由喪公錢、縱奸吏故也。且凡為度支胥吏,不一歲,資累鉅萬,僮馬第宅,僭於王公,非盜官財何以致是?道呼喧喧,無不知之,聖上故令滂分掌。公向所言,無乃歸怨於上乎、」宏默然不對。是日,宏稱疾於第,滂往問之,宏不見,憬、贄乃以宏、滂之言上聞。由是遵大歷故事,如劉晏、韓滉所分。滂至揚州按徐粲,逮僕妾子侄,得贓鉅萬,乃徙嶺表。故參得罪,宏頗有力焉。勤恪官署,晨入夕歸,下吏勞而未嘗厭苦,清白
 勤幹,稱之於時。貞元八年七月卒,年七十三,廢朝,加贈,謚曰敬。



 王紹,本家於太原,今為京兆萬年人。舊名與憲宗同,永貞年改焉。少時,顏真卿器重之,因紹舊名,字之曰德素,奏授武康尉。蕭復為常州刺史,闢為從事;包佶領租庸鹽鐵,亦以紹為判官。時李希烈阻兵,江淮租輸,所在艱阻,特移運路自潁入汴。紹奉佶表詣闕,屬德宗西幸,紹乃督緣路輕貨,趣金、商路,倍程出洋州以赴行在。德宗
 親勞之,謂紹曰:「六軍未有春服,我猶衣裘。」紹俯伏流涕,奏曰:「包佶令臣間道進奉數約五十萬。」上曰:道路回遠,經費懸急,卿之所奏,豈可望耶?」後五日而所督繼至,上深賴焉。



 貞元中,為倉部員外郎。時屬兵革旱蝗之後,令戶部收闕官俸,兼稅茶及諸色無名之錢,以為水旱之備。紹自拜倉部,便準詔主判,及遷戶部、兵部郎中,皆獨司其務。擢拜戶部侍郎,尋判度支。後二年,遷戶部尚書。德宗臨馭歲久,機務不由臺司,自竇參、陸贄已後,宰臣
 備位而已。德宗以紹謹密,恩遇特異,凡主重務八年,政之大小,多所訪決。紹未賞洩漏,亦不矜衒。順宗即位,王叔文始奪其權,拜兵部尚書,尋除檢校吏部尚書、東都留守。元和初,遷檢校尚書右僕射、徐州刺史、武寧軍節度,復以濠、泗二州隸焉。時承張愔之後,兵驕難治,紹修輯軍政,人甚安之。六年,徵拜兵部尚書,兼判戶部事。九年卒,年七十二,贈左僕射,謚曰敬。



 李巽,字令叔,趙郡人。少苦心為學,以明經調補華州參
 軍,拔萃登科,授鄠縣尉。周歷臺省,由左司郎中出為常州刺史。逾年,召為給事中,出為湖南觀察使,銳於為理。五年,改江西觀察使,加檢校散騎常侍、兼御史大夫。巽持下以法,吏不敢欺,而動必察之。順宗即位,入為兵部侍郎。司徒杜佑判度支鹽鐵轉運使,以巽乾治,奏為副使。佑辭重位,巽遂專領度支鹽鐵使。榷筦之法,號為難重,唯大歷中僕射劉晏雅得其術,賦入豐羨。巽掌使一年,徵課所入,類晏之多歲,明年過之,又一年加一百八
 十萬貫。舊制,每歲運江淮米五十萬斛抵河陰,久不盈其數,唯巽三年登焉。遷兵部尚書,明年改吏部尚書,使任如故。



 巽精於吏職,蓋性使然也。雖在私家,亦置案牘簿書,勾檢如公署焉。人吏有過,絲毫無所貸,雖在千里外,其恐慄如在巽前。初,程異附王叔文貶竄,巽知其吏才明辯,奏而用之,憲宗不違其請。異勾檢簿籍,又精於巽,故課最加衍,亦異之助焉。巽為吏部尚書,臥疾,郎官相率省問,巽初不言其病,與之考校程課,商略功利,至
 其夕而卒。然性強很狡惡,忌刻頗甚,乘德宗之怒,謀殺竇參,物論冤之。初,參為宰相,不悅於巽,自左司郎中出為常州刺史,仍促其行。不數月,參貶郴州司馬。久之,巽自給事中為湖南觀察使,郴即屬郡也。宣武軍節度使劉士寧以擅襲父任,物議不可,朝廷不得已而授之。及參之貶,士寧嘗以絹數千匹賂參,巽在湖南具奏其事,言參與籓鎮交通,德宗怒,遂賜參死,議者冤之。巽廉察江西,徇喜怒之情,而無罪被戮者多矣。元和四年四月
 卒,時年七十一,贈尚書左僕射。



 史臣曰:歷代操利柄為國計者,莫不損下益上,危人自安,變法以弄權,斂怨以構禍,皆有之矣。如劉晏通擁滯,任才能,富其國而不勞於民,儉於家而利於眾。或問曰:鄭子產吏不能欺,宓子賤吏不忍欺,西門豹吏不敢欺。三子者,古之賢人也,吏皆懷其欺而不能、不忍、不敢也。晏之吏,遠近自不欺者何也?答曰:蓋任其才而得其人也。晏歿,故吏二十餘年繼掌財賦,不其是哉!《史記貨殖》
 云:「平糶齊物,關市不乏,治國之道也。」晏治天下,無甚貴甚賤之物,泛言治國者,其可及乎!舉真卿才,忠也,減王縉罪,正也,忠正之道,復出於人,嗚呼!本秀於林,風必摧之,常袞見忌於前,楊炎致冤於後,可為長嘆息矣!時譏有口者以利啖之,茍不塞讒口,何以持重權?即無以展其才,濟其國矣。是其術也,又何譏焉。第五琦促辦應卒,民不加賦,而國豐饒,亦庶幾矣。然鑄錢變法,物貴身危,其何陋哉!凡利國者,農商之外,不可為也。宏、滂爭權樹
 黨,皆非令人。紹之謹密幹事,巽之皦察精辨,亦足可稱。



 贊曰:豐財忠良,晏道為長。琦、宏、滂、巽,咸以利彰。



\end{pinyinscope}