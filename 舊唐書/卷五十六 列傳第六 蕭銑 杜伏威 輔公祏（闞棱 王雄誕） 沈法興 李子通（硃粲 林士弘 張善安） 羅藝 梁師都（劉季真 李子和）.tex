\article{卷五十六 列傳第六 蕭銑 杜伏威 輔公祏(闞棱 王雄誕) 沈法興 李子通(硃粲 林士弘 張善安) 羅藝 梁師都(劉季真 李子和)}

\begin{pinyinscope}

 ○蕭銑杜伏威輔公祏闞棱王雄誕沈法興李子通硃粲林士弘張善安羅藝梁師都劉季真李子和



 蕭銑,後梁宣帝曾孫也。祖巖,隋開皇初叛隋降於陳,陳亡,為文帝所誅。銑少孤貧,傭書自給,事母以孝聞。煬帝
 時,以外戚擢授羅川令。



 大業十三年,岳州校尉董景珍、雷世猛,旅帥鄭文秀、許玄徹、萬瓚、徐德基、郭華,沔州人張繡等同謀叛隋。郡縣官屬眾欲推景珍為主,景珍曰:「吾素寒賤,雖假名號,眾必不從。今若推主,當從眾望。羅川令蕭銑,梁氏之後,寬仁大度,有武皇之風。吾又聞帝王膺籙,必有符命,而隋氏冠帶,盡號『起梁』,斯乃蕭家中興之兆。今請以為主,不亦應天順人乎?」眾乃遣人諭意,銑大悅,報景珍書曰:「我之本國,昔在有隋,以小事大,朝
 貢無闕。乃貪我土宇,滅我宗祊,我是以痛心疾首,無忘雪恥。今天啟公等,協我心事,若合符節,豈非上玄之意也!吾當糾率士庶,敬從來請。」即日集得數千人,揚言討賊而實欲相應。遇潁川賊帥沈柳生來寇羅川縣,銑擊之,不利,因謂其眾曰:「岳州豪傑首謀起義,請我為主。今隋政不行,天下皆叛,吾雖欲獨守,力不自全。且吾先人昔都此地,若從其請,必復梁祚,遣召柳生,亦當從我。」眾皆大悅,即日自稱梁公,改隋服色,建梁旗幟。柳生以眾
 歸之,拜為車騎大將軍,率眾往巴陵。自起軍五日,遠近投附者數萬人。



 景珍遣徐德基、郭華率州中首領數百人詣軍迎謁,未及見銑,而前造柳生。柳生謂其下曰:「我先奉梁公,勛居第一。今岳州兵眾,位多於我,我若入城,便出其下。不如殺德基,質其首領,獨挾梁王進取州城。」遂與左右殺德基,方詣中軍白銑。銑大驚曰:「今欲撥亂,忽自相殺,我不能為汝主矣。」乃步出軍門。柳生大懼,伏地請罪,銑責而赦之,令復舊位。銑陳兵入城,景珍進言
 於銑曰:「徐德基丹誠奉主,柳生兇悖擅殺之,若不加誅,何以為政?且其為賊,兇頑已久,今雖從義,不革此心,同處一城,必將為變。若不預圖,後悔無及。」銑又從之。景珍遂斬柳生於城內。其下將帥皆潰散。銑於是築壇於城南,燔燎告天,自稱梁王。以有異鳥之瑞,建元為鳳鳴。義寧二年,僭稱皇帝,署置百官,一準梁故事。偽謚其從父琮為孝靖帝,祖巖為河間忠烈王,父璇為文憲王。封董景珍為晉王,雷世猛為秦王,鄭文秀為楚王,許玄徹為
 燕王,萬瓚為魯王,張繡為齊王,楊道生為宋王。隋將張鎮州、王仁壽擊之,不能克。及聞隋滅,鎮州因與寧長真等率嶺表諸州盡降於銑。九江鄱陽,初有林士弘僭號,俄自相誅滅,士弘逃於安成之山洞,其郡亦降於銑。遣其將楊道生攻陷南郡,張繡略定嶺表,東至三硤,南盡交址,北拒漢川,皆附之,勝兵四十餘萬。



 武德元年,遷都江陵,修復園廟。引岑文本為中書侍郎,令掌機密。銑又遣楊道生攻硤州,刺史許紹出兵擊破之,赴水死者大
 半。高祖詔夔州總管趙郡王孝恭率兵討之,拔其通、開二州,斬偽東平郡王蕭闍提。時諸將橫恣,多專殺戮,銑因令罷兵,陽言營農,實奪將帥之權也。其大司馬董景珍之弟為偽將軍,怨銑放其兵,遂謀為亂,事洩,為銑所誅。時景珍出鎮長沙,銑下書赦之,召還江陵,景珍懼,遣間使詣孝恭送款。銑遣其齊王張繡攻之,景珍謂繡曰:「『前年醢彭越,往年殺韓信』,卿豈不見之乎?奈何今日相攻!」繡不答,進兵圍之。景珍潰圍而走,為其麾下所殺。銑
 以繡為尚書令,繡恃勛驕慢,專恣弄權,銑又惡而殺之。既大臣相次誅戮,故人邊將皆疑懼,多有叛者,銑不能復制,以故兵勢益弱。



 四年,高祖命趙郡王孝恭及李靖率巴蜀兵發自夔州,沿流而下;廬江王瑗從襄州道,黔州刺史田世康趣辰州道,黃州總管周法明趣夏口道以圖銑。及大軍將至,銑江州總管蓋彥舉以五州降。又遣其將文士弘等率兵拒戰,孝恭與李靖皆擊破之,進逼其都。初,銑之放兵散也,自留宿衛兵士數千人,忽聞
 孝恭至而倉卒追兵,並江、嶺之南,道里遼遠,未能相及。孝恭縱兵入郭,布長圍以守之。數日,克其水城,獲其舟船數千艘。其交州總管丘和、長史高士廉、司馬杜之松等先來謁銑,聞兵敗,便詣李靖來降。銑自度救兵不至,謂其群下曰:「天不祚梁,數歸於滅。若待力屈,必害黎元,豈以我一人致傷百姓?及城未拔,宜先出降,冀免亂兵,幸全眾庶。諸人失我,何患無君?」乃巡城號令,守陴者皆慟哭。銑以太牢告於其廟,率官屬緦縗布幘而詣軍門,
 曰:「當死者唯銑,百姓非有罪也,請無殺掠。」孝恭囚之,送於京師。銑降後數日,江南救兵十餘萬一時大至,知銑降,皆送款於孝恭。銑至,高祖數其罪,銑對曰:「隋失其鹿,英雄競逐,銑無天命,故至於此。亦猶田橫南面,非負漢朝。若以為罪,甘從鼎鑊。」竟斬於都市,年三十九。銑自初起,五年而滅。



 杜伏威,齊州章丘人也。少落拓,不治產業,家貧無以自給,每穿窬為盜。與輔公祏為刎頸之交。公祏姑家以牧
 羊為業,公祏數攘羊以饋之,姑有憾焉,因發其盜事。郡縣捕之急,伏威與公祏遂俱亡命,聚眾為群盜,時年十六。常營護諸盜,出則居前,入則殿後,故其黨咸服之,共推為主。



 大業九年,率眾入長白山,投賊帥左君行,不被禮,因舍去,轉掠淮南,自稱將軍。時下邳有苗海潮,亦聚眾為盜,伏威使公祏謂曰:「今同苦隋政,各興大義,力分勢弱,常恐見擒,何不合以為強,則不患隋軍相制。若公能為主,吾當敬從,自揆不堪,可來聽命,不則一戰以決
 雄雌。」海潮懼,即以其眾歸於伏威。江都留守遣校尉宋顥率兵討之,伏威與戰,陽為奔北,引入葭蘆中,而從上風縱火,迫其步騎陷於大澤,火至皆燒死。有海陵賊帥趙破陣,聞伏威兵少而輕之,遣使召伏威,請與並力。伏威令公祏嚴兵居外以待變,親將十人持牛酒入謁。破陣大悅,引伏威入幕,盡集其酋帥縱酒高會。伏威於坐斬破陣而並其眾。由此兵威稍盛,復屠安宜。



 煬帝遣右御衛將軍陳棱以精兵八千討之,棱不敢戰,伏威遺棱
 婦人之服以激怒之,並致書號為「陳姥」,棱大怒,悉兵而至。伏威逆拒,自出陣前挑戰,棱部將射中其額,伏威怒,指之曰:「不殺汝,我終不拔箭。」遂馳之。棱部將走奔其陣,伏威因入棱陣,大呼沖擊,所向披靡,獲所射者,使其拔箭,然後斬之,攜其首復入棱軍奮擊,殺數十人。棱陣大潰,僅以身免。乘勝破高郵縣,引兵據歷陽,自稱總管,分遣諸將略屬縣,所至輒下,江淮間小盜爭來附之。伏威嘗選敢死之士五千人,號為「上募」,寵之甚厚,與同甘苦。
 有攻戰,輒令上募擊之,及戰罷閱視,有中在背,便殺之,以其退而被擊也。所獲貲財,皆以賞軍士,有戰死者,以其妻妾殉葬,故人自為戰,所向無敵。



 宇文化及之反也,署為歷陽太守,伏威不受。又移居丹陽,進用人士,大修器械,薄賦斂,除殉葬法,其犯奸盜及官人貪濁者,無輕重皆殺之。仍上表於越王侗,侗拜伏威為東道大總管,封楚王。太宗之圍王世充,遣使招之,伏威請降。高祖遣使就拜東南道行臺尚書令、江淮以南安撫大使、上柱
 國,封吳王,賜姓李氏,預宗正屬籍,封其子德俊為山陽公,賜帛五千段、馬三百匹。伏威遣其將軍陳正通、徐紹宗率兵來會。武德四年,遣其將軍王雄誕討李子通於杭州,擒之以獻。又破汪華於歙州,盡有江東、淮南之地,南接於嶺,東至於海。尋聞太宗平劉黑闥,進攻徐圓朗,伏威懼而來朝,拜為太子太保,仍兼行臺尚書令。留於京師,禮之甚厚,位在齊王元吉之上,以寵異之。初,輔公祏之反也,詐稱伏威之令,以紿其眾,高祖遣趙郡王孝
 恭討之。時伏威在長安暴卒。及公祏平,孝恭收得公祏反辭,不曉其詐,遽以奏聞,乃除伏威名,籍沒其妻子。貞觀元年,太宗知其冤,赦之,復其官爵,葬以公禮。



 輔公祏,齊州臨濟人。隋末,從杜伏威為群盜。初,伏威自稱總管,以公祏為長史。李子通之敗沈法興也,伏威使公祏以精卒數千渡江討之。子通率眾數萬以拒公祏,兵鋒甚銳。公祏簡甲士千人,皆使執長刀,仍令千餘人隨後,令之曰:「有卻者斬。」公祏自領餘眾,復居其後。俄而
 子通方陣而前,公祏所遣千人皆殊死決戰,公祏乃縱左右翼攻之,子通大潰,降其眾數千人。公祏尋與伏威遣使歸國,拜為淮南道行臺尚書左僕射,封舒國公。初,伏威與公祏少相愛狎,公祏年長,伏威每兄事之,軍中咸呼為伯,畏敬與伏威等。伏威潛忌之,為署其養子闞棱為左將軍,王雄誕為右將軍,推公祏為僕射,外示尊崇,而陰奪其兵權。公祏知其意,怏怏不平,乃與故人左游仙偽學道闢穀以遠其事。武德五年,伏威將入朝,留
 公祏居守,復令雄誕典兵以副公祏,陰謂曰:「吾入京,若不失職,無令公祏為變。」其後左游仙乃說公祏令反。會雄誕屬疾於家,公祏奪其兵,詐言伏威不得還江南,貽書令其起兵。因僭即偽位,自稱宋國,於陳故都築宮以居焉。署置百官,以左游仙為兵部尚書、東南道大使、越州總管。大修兵甲,轉漕糧饋。時吳興賊帥沈法興據毗陵,公祏擊破之。又遣其將馮惠亮屯於博望山,陳正通、徐紹宗屯於青林山以拒官軍。高祖命趙郡王孝恭率
 諸將奮擊,大破之。紹宗、正通以五騎奔於丹陽。公祏懼而遁走,欲就左游仙於會稽,至武康,為野人所執,送於丹陽,孝恭斬之,傳首京師。公祏與伏威同起,至滅凡十三載,江東悉平。初,伏威養壯士三十餘人為假子,分領兵馬,唯闞棱、王雄誕知名。



 闞棱,齊州臨濟人。善用大刀,長一丈,施兩刃,名為陌刃,每一舉,輒斃數人,前無當者。及伏威據有江淮之地,棱數有戰功,署為左將軍。伏威步兵皆出自群賊,類多放
 縱,有相侵奪者,棱必殺之,雖親故無所舍,令行禁止,路不拾遺。後從伏威入朝,拜左領軍將軍,遷越州都督。及公祏僭號,棱從軍討之,與陳正通相遇。陣方接,棱脫兜鍪謂賊眾曰:「汝不識我邪?何敢來戰!」其眾多棱舊之所部,由是各無鬥志,或有還拜者。公祏之破,棱功居多,頗有自矜之色。及擒公祏,誣棱與己通謀。又杜伏威、王雄誕及棱家產在賊中者,合從原放,孝恭乃皆籍沒。棱訴理之,有忤於孝恭,孝恭怒,遂以謀反誅之。



 王雄誕者,曹州濟陰人。初,伏威之起也,用其計,屢有克獲,署為驃騎將軍。伏威後率眾渡淮,與海陵賊李子通合。後子通惡伏威雄武,使騎襲之,伏威被重瘡墮馬,雄誕負之,逃於葭蘆中。伏威復招集餘黨,攻劫郡縣,隋將來整又擊破之,亡失餘眾。其部將西門君儀妻王氏勇決多力,負伏威而走,雄誕率麾下壯士十餘人衛護。隋軍追至,雄誕輒還御之,身被數槍,勇氣彌厲,竟脫伏威。時闞棱年長於雄誕,故軍中號棱為大將軍,雄誕為小
 將軍。



 後伏威令輔公祏擊李子通於江都,使雄誕與棱為副,戰於溧水,子通大敗。公祏乘勝追之,卻為子通所破,軍士皆堅壁不敢出。雄誕謂公祏曰:「子通軍無營壘,且狃於初勝而不設備,若擊之,必克。」公祏不從。雄誕以其私屬數百人銜枚夜擊之,因順風縱火,子通大敗,走渡太湖,復破沈法興,居其地。高祖聞伏威據有吳、楚,遣使諭之。雄誕率眾討之,子通以精兵守獨松嶺,雄誕遣其部將陳當率千餘人,出其不意,乘高據險,多張旗幟,
 夜則縛炬火於樹上,布滿山澤間。子通大懼,燒營而走,保於杭州。雄誕追擊敗之,擒子通於陣,送於京師。歙州首領汪華,隋末據本郡稱王十餘年,雄誕回軍擊之。華出新安洞口以拒雄誕,甲兵甚銳。雄誕伏精兵於山谷間,率羸弱數千人當之,戰才合,偽退歸本營。華攻之不能克,會日暮欲還,雄誕伏兵已據其洞口,華不得入,窘急面縛而降。蘇州賊帥聞人遂安據昆山縣而無所屬,伏威又命雄誕攻之。雄誕以昆山險隘,難以力勝,遂單
 騎詣其城下,陳國威靈,示以禍福,遂安感悅,率諸將出降。以前後功授歙州總管,封宜春郡公。伏威之入朝也,留輔公祏鎮江南,而兵馬屬於雄誕。公祏將為逆,奪其兵,拘之別室,遣西門君儀諭以反計,雄誕曰:「當今方太平,吳王又在京輦,國家威靈,無遠不被,公何得為族滅事耶!雄誕有死而已,不敢聞命。」公祏知不可屈,遂縊殺之。雄誕善撫恤將士,皆得其死力,每破城鎮,約勒部下,絲毫無犯,故死之日,江南士庶莫不為之流涕。高祖嘉
 其節,命其子果襲封宜春郡公。太宗即位,追贈左衛大將軍、越州都督,謚曰忠。



 果,垂拱初官至廣州都督,安西大都護。



 沈法興,湖州武康人也。父恪,陳特進、廣州刺史。法興,隋大業末為吳興郡守。東陽賊帥樓世乾舉兵圍郡城,煬帝令法興與太僕丞元祐討之。俄而宇文化及弒煬帝於江都,法興自以代居南土,宗族數千家,為遠近所服,乃與祐部將孫士漢、陳果仁執祐於坐,號令遠近。以誅
 化及為名,發自東陽,行收兵,將趨江都,下餘杭郡,比至烏程,精卒六萬。毗陵郡通守路道德率兵拒之,法興請與連和,因會盟襲殺道德,進據其城。時齊郡賊帥樂伯通據丹陽,為化及城守,法興使果仁攻陷之,於是據有江表十餘郡,自署江南道總管。復聞越王侗立,乃上表於侗,自稱大司馬、錄尚書事、天門公。承制置百官,以陳果仁為司徒,孫士漢為司空,蔣元超為尚書左僕射,殷芊為尚書左丞,徐令言為尚書右丞,劉子翼為選部侍
 郎,李百藥為府掾。



 法興自克毗陵後,謂江淮已南可指捴而定,專立威刑,將士有小過,便即誅戮,而言笑自若,由是將士解體。稱梁,建元曰延康,改易隋官,頗依陳氏故事。是時,杜伏威據歷陽,陳棱據江都,李子通據海陵,並握強兵,俱有窺覦江表之志。法興三面受敵,軍數挫衄。陳棱尋被李子通圍於江都,棱窘急,送質求救,法興使其子綸領兵數萬救之。子通率眾攻綸,大敗,乘勝渡江,陷其京口。法興使蔣元超拒之於庱亭,元超戰死。法
 興與左右數百人投吳郡賊帥聞人遂安,遣其將葉孝辯迎之。法興至中路而悔,欲殺孝辯,更向會稽。孝辯覺之,法興懼,乃赴江死。初,法興以義寧二年起兵,至武德三年而滅。



 李子通,東海丞人也。少貧賤,以魚獵為事。居鄉里,見班白提挈者,必代之。性好施惠,家無蓄積,睚眥之怨必報。隋大業末,有賊帥左才相,自號博山公,據齊郡之長白山,子通歸之,以武力為才相所重。有鄉人陷於賊者,必
 全護之。時諸賊皆殘忍,唯子通獨行仁恕,由是人多歸之,未半歲,兵至萬人。才相稍忌之,子通自引去,因渡淮,與杜伏威合。尋為隋將來整所敗,子通擁其餘眾奔海陵,得眾二萬,自稱將軍。初,宇文化及以隋將軍陳棱為江都太守,子通率師擊之。棱南求救於沈法興,西乞師於杜伏威,二人各以兵至,伏威屯清流,法興保楊子,相去數十里間。子通納言毛文深進計,募江南人詐為法興之兵,夜襲伏威。伏威不悟,恨法興之侵己,又遣兵襲
 法興。二人相疑,莫敢先動。子通遂得盡銳攻陷江都,陳棱奔於伏威。子通入據江都,盡虜其眾,因僭即皇帝位,國稱吳,建元為明政。



 丹陽賊帥樂伯通率眾萬餘來降,子通拜尚書左僕射。更進擊法興於庱亭,斬其僕射蔣元超,法興棄城宵遁,遂有晉陵之地。獲法興府掾李百藥,引為內史侍郎,使典文翰;以法興尚書左丞殷芊為太常卿,使掌禮樂。由是隋郡縣及江南人士多歸之。後伏威遣輔公祏攻陷丹陽,進屯溧水,子通擊之,反為公
 祏所敗。又屬糧盡,子通棄江都,保於京口,江西之地盡歸伏威。子通又東走太湖,鳩集亡散,得二萬人,襲沈法興於吳郡,破之,率其官屬都於餘杭。東至會稽,南至千嶺,西距宣城,北至太湖,盡有其地。



 未幾,杜伏威遣其將王雄誕攻之,大戰於蘇州,子通敗績,退保餘杭。雄誕進逼之,戰於城下,軍復敗,子通窮蹙請降。伏威執之,並其左僕射樂伯通送於京師,盡收其地。高祖不之罪,賜宅一區、公田五頃,禮賜甚厚。及伏威來朝,子通謂伯通曰:「
 伏威既來,東方未靜,我所部兵,多在江外,往彼收之,可有大功於天下矣。」遂相與亡,至藍田關,為吏所獲,與伯通俱伏誅。時又有硃粲、林士弘、張善安,皆僭號於江、淮之間。



 硃粲者,亳州城父人也。初為縣佐史。大業末,從軍討長白山賊,遂聚結為群盜,號「可達寒賊」,自稱迦樓羅王,眾至十餘萬。引軍渡淮,屠竟陵、沔陽,後轉掠山南,郡縣不能守,所至殺戮,噍類無遺。義寧中,招慰使馬元規擊破
 之。俄而收輯餘眾,兵又大盛,僭稱楚帝於冠軍,建元為昌達,攻陷鄧州,有眾二十萬。粲所克州縣,皆發其藏粟以充食,遷徙無常,去輒焚餘貲,毀城郭,又不務稼穡,以劫掠為業。於是百姓大餒,死者如積,人多相食。軍中罄竭,無所虜掠,乃取嬰兒蒸而啖之,因令軍士曰:「食之美者,寧過於人肉乎!但令他國有人,我何所慮?」即勒所部,有略得婦人小兒皆烹之,分給軍士,乃稅諸城堡,取小弱男女以益兵糧。隋著作佐郎陸從典、通事舍人顏愍
 楚因譴左遷,並在南陽,粲悉引之為賓客,後遭饑餒,合家為賊所啖。又諸城懼稅,皆相攜逃散。顯州首領楊士林、田瓚率兵以背粲,諸州響應,相聚而攻之,大戰於淮源。粲敗,以數千兵奔於菊潭縣,遣使請降。高祖令假散騎常侍段確迎勞之,確因醉,侮粲曰:「聞卿啖人,作何滋味?」粲曰:「若啖嗜酒之人,正似糟藏豬肉。」確怒,慢罵曰:「狂賊,入朝後一頭奴耳,更得啖人乎!」粲懼,於坐收確及從者數十人,奔於王世充,拜為龍驤大將軍。東都平,獲之,
 斬於洛水之上。士庶嫉其殘忍,競投瓦礫以擊其尸,須臾封之若塚。



 林士弘者,饒州鄱陽人也。大業十二年,與其鄉人操師乞起為群盜。師乞自號元興王,攻陷豫章郡而據之,以士弘為大將軍。隋遣持書侍御史劉子翊率師討之,師乞中矢而死。士弘代董其眾,復與子翊大戰於彭蠡湖,隋師敗績,子翊死之。士弘大振,兵至十餘萬。大業十三年,徙據虔州,自稱皇帝,國號楚,建元太平,以其黨王戎
 為司空。攻陷臨川、廬陵、南康、宜春等諸郡,北至九江,南洎番禺,悉有其地。其黨張善安保南康郡,懷貳於士弘,以舟師循江而下,擊破豫章。士弘尚有南昌、虔、循、潮數州之地。及蕭銑破後,散兵稍往歸之,士弘復振。荊州總管趙王孝恭遣使招慰之,其循、潮二州並來降。武德五年,士弘遣其弟鄱陽王藥師率兵二萬攻圍循州,刺史楊略與戰,大破之。士弘懼而遁走,潛保於安城之山洞。王戎亦以南昌來降,拜為南昌州刺史。戎於是召士弘
 藏之於宅,招誘舊兵,更謀作亂。其年,洪州總管張善安密知其事,發兵討之,會士弘死,部兵潰散,戎為善安所虜。



 張善安者,兗州方與人也。年十七便為劫盜,轉掠淮南,有眾百餘人。會孟讓為王世充所破,其散卒稍歸之,得八百人。襲破廬江郡,因渡江,附林士弘於豫章。士弘不信之,營於南塘上。善安憾之,襲擊士弘,焚其郛郭。而士弘後去豫章,善安復來據之,仍以其地歸國,授洪州總
 管。輔公祏之反也,善安亦舉兵相應,公祏以為西南道大行臺。安撫使李大亮以兵擊之,兩軍隔水而陣,大亮諭以禍福。答曰:「善安無背逆之心,但為將士所誤。今欲歸降,又恐不免於死。」大亮謂曰:「張總管既有降心,吾亦不相疑阻。」因獨身逾澗就之,入其陣,與善安握手交言,示無猜意。善安大喜,因許降,將數十騎至大亮營,大亮引之而入,因令武士執之,從者遁走。既而送善安於長安,稱不與公祏交通,高祖初善遇之。及公祏敗,搜得其
 書,與相往復,遂誅之。



 羅藝,字子延,本襄陽人也,寓居京兆之雲陽。父榮,隋監門將軍。藝性桀黠,剛愎不仁,勇於攻戰,善射,能弄槊。大業時,屢以軍功官至虎賁郎將,煬帝令受右武衛大將軍李景節度,督軍於北平。藝少習戎旅,分部嚴肅,然任氣縱暴,每凌侮於景,頻為景所辱,藝深銜之。後遇天下大亂,涿郡物殷阜,加有伐遼器仗,倉粟盈積。又臨朔宮中多珍產,屯兵數萬,而諸賊競來侵掠。留守官虎賁郎
 將趙什住、賀蘭誼、晉文衍等皆不能拒,唯藝獨出戰,前後破賊不可勝計,威勢日重。什住等頗忌藝,藝陰知之,將圖為亂,乃宣言於眾曰:「吾輩討賊,甚有功效,城中倉庫山積,制在留守之官,而無心濟貧,此豈存恤之意也!」以此言激怒其眾,眾人皆怨。既而旋師,郡丞出城候藝,藝因執之陳兵,而什住等懼,皆來聽命。於是發庫物以賜戰士,開倉以賑窮乏,境內咸悅。殺渤海太守唐禕等不同己者數人,威振邊朔,柳城、懷遠並歸附之。藝黜柳
 城太守楊林甫,改郡為營州,以襄平太守鄧暠為總管,藝自稱幽州總管。宇文化及至山東,遣使召藝,藝曰:「我隋室舊臣,感恩累葉,大行顛覆,實所痛心。」乃斬化及使者,而為煬帝發喪,大臨三日。竇建德、高開道亦遣使於藝,藝謂官屬曰:「建德、開道,皆劇賊耳,化及弒逆,並不可從。今唐公起兵,皆符人望,入據關右,事無不成。吾率眾歸之,意已決矣,有沮眾異議者必戮之。」會我使人張道源綏輯山東,遣人諭意,藝大悅。武德三年,奉表歸國,詔
 封燕王,賜姓李氏,預宗正屬籍。



 太宗之擊劉黑闥也,藝領本兵數萬,破黑闥弟什善於徐河,俘斬八千人。明年,黑闥引突厥俱入寇,藝復將兵與隱太子建成會於洺州,因請入朝,高祖遇之甚厚,俄拜左翊衛大將軍。藝自以功高位重,無所降下,太宗左右嘗至其營,藝無故毆擊之。高祖怒,以屬吏,久而乃釋,待之如初。時突厥屢為寇患,以藝素有威名,為北夷所憚,令以本官領天節軍將鎮涇州。



 太宗即位,拜開府儀同三司,而藝懼不自安,
 遂於涇州詐言閱武,因追兵,矯稱奉密詔勒兵入朝,率眾軍至於豳州。治中趙慈皓不知藝反,馳出謁之,藝遂入據豳州。太宗命吏部尚書長孫無忌、右武候大將軍尉遲敬德率眾討藝。王師未至,慈皓與統軍楊岌潛謀擊之,事洩,藝執慈皓系獄。岌時在城外,覺變,遽勒兵攻之,藝大潰,棄妻子,與數百騎奔於突厥。至寧州界,過烏氏驛,從者漸散,其左右斬藝,傳首京師,梟之於市。復其本姓羅氏。藝弟壽,時為利州都督,緣坐伏誅。先是,曹州
 女子李氏為五戒,自言通於鬼物,有病癩者,就療多愈,流聞四方,病人自遠而至,門多車騎。高祖聞之,詔赴京師。因往來藝家,謂藝妻孟氏曰:「妃骨相貴不可言,必當母儀天下。」孟篤信之,命密觀藝,又曰:「妃之貴者,由於王;王貴色發矣,十日間當升大位。」孟氏由是遽勸反,孟及李皆坐斬。



 梁師都,夏州朔方人也。代為本郡豪族,仕隋鷹揚郎將。大業末,罷歸。屬盜賊群起,師都陰結徒黨數千人,殺郡
 丞唐宗,據郡反。自稱大丞相,北連突厥。隋將張世隆擊之,反為所敗。師都因遣兵掠定雕陰、弘化、延安等郡,於是僭即皇帝位,稱梁國,建元為永隆。突厥始畢可汗遺以狼頭纛,號為大度毗伽可汗。師都乃引突厥居河南之地,攻破鹽川郡。



 武德二年,高祖遣延州總管段德操督兵討之。師都與突厥之眾數千騎來寇延安,營於野豬嶺。德操以眾寡不敵,按甲以挫其銳。後伺師都稍怠,遣副總管梁禮率眾擊之,德操以輕騎出其不意。師都
 與禮酣戰久之,德操多張旗幟,奄至其後,師都大潰,逐北二百餘里,虜男女二百餘口。經數月,師都又以步騎五千來寇,德操擊之,俘斬略盡。及劉武周之敗,師都大將張舉、劉旻相次來降,師都大懼,遣其尚書陸季覽說處羅可汗曰:「比者中原喪亂,分為數國,勢均力弱,所以北附突厥。今武周既滅,唐國益大,師都甘從亡破,亦恐次及可汗。願可汗行魏孝文之事,遣兵南侵,師都請為鄉導。」處羅從之。謀令莫賀咄設入自原州,泥步設與師
 都入自延州,處羅入自並州,突利可汗與奚霫契丹、靺鞨入自幽州,合於竇建德,經滏口道來會於晉、絳。兵臨發,遇處羅死,乃止。高祖又令德操悉發邊兵進擊師都,拔其東城。師都退據西城,又求救於突厥頡利可汗,頡利以勁兵萬騎救援之。時稽胡大帥劉GC成率眾降師都,師都信讒殺之,於是群情疑懼,多叛師都來降。師都勢蹙,乃往朝頡利,為陳入寇之計。自此頻致突厥之寇,邊州略無寧歲。頡利可汗之寇渭橋,亦師都計也。頡利
 政亂,太宗知師都勢危援孤,以書諭之,不從。遣夏州長史劉旻、司馬劉蘭經略之。有得其生口者,輒縱遣令為反間,離其君臣之計。頻選輕騎踐其禾稼,城中漸虛,歸命者相繼,皆善遇之。由是益相猜阻。有李正寶、辛獠兒者,皆其名將,謀執師都,事洩不果,正寶竟來降。貞觀二年,太宗遣右衛大將軍柴紹、殿中少監薛萬均討之,又使劉旻、劉蘭率勁卒直據朔方東城以逼之。頡利可汗遣兵來援師都,紹逆擊破之,進屯城下。師都兵勢日蹙,
 其從父弟洛仁斬師都,詣紹降,拜洛仁為右驍衛將軍,封朔方郡公。師都自起至滅,凡十二歲。以其地為夏州。時又有劉季真、李子和,屯據北邊,與劉武周、梁師都遞為表裏。



 劉季真者,離石胡人也。父龍兒,隋末擁兵數萬,自號劉王,以季真為太子。龍兒為虎賁郎將梁德所斬,其眾漸散。及義師起,季真與弟六兒復舉兵為盜,引劉武周之眾攻陷石州。季真北連突厥,自稱突利可汗,以六兒為
 拓定王,甚為邊患。時西河公張綸、真鄉公李仲文俱以兵臨之,季真懼而來降,授石州總管,賜姓李氏,封彭城郡王。季真見宋金剛與官軍相持於澮州,久而未決,遂親伏武周,與之合勢。及金剛敗,季真亡奔高滿政,尋為所殺。



 李子和者,同州蒲城人也。本姓郭氏。大業末,為左翊衛,犯罪徙榆林,見郡內大饑,遂潛引敢死士,得十八人,攻郡門,執郡丞王才,數以不恤百姓,斬之,開倉以賑窮乏。
 自稱永樂王,建元為正平,尊其父為太公,以弟子政為尚書令,子端、子升為左、右僕射。有眾二千餘騎,南連梁師都,北附突厥始畢可汗,並送子為質以自固。始畢先署劉武周為定楊天子,梁師都為解事天子,又以子和為平楊天子,子和固辭不敢當,始畢乃更署子和為屋利設。武德元年,遣使歸款,授榆林郡守。尋就拜雲州總管,封金河郡公。二年,進封郕國公。時師都強暴,子和慮為所攻,尋勒兵襲師都寧朔城,克之。子和既絕師都,又
 伺突厥間釁,遣使以聞,為處羅可汗候騎所獲,處羅大怒,囚其弟子升。子和自以孤危,甚懼。四年,拔戶口南徙,詔以延州故城居之。五年,從太宗平劉黑闥,陷陣有功。高祖嘉其誠節,賜姓李氏,拜右武衛將軍。貞觀元年,賜實封三百戶。十一年,除婺州刺史,改封夷國公。顯慶元年,累轉黔州都督。以年老乞骸骨,許之,加金紫光祿大夫。麟德九年卒。



 史臣曰:蕭銑聚烏合之眾,當鹿走之時,放兵以奪將權,
 殺舊以求位定,洎大軍奄至,束手出降,宜哉!杜伏威恃勇聚徒,見機歸國,或致疑於高祖,竟見雪於太宗。輔公祏竊兵為叛,王雄誕守節不回,訓子孫以忠貞,感士庶之流涕。子通修仁馭眾,終懷貳以伏誅;羅藝歸國立功,信妖言而為叛。善始令終者,鮮矣!沈法興狂賊,梁師都兇人,皆至覆亡,殊無改悔。自隋朝維絕,宇縣瓜分,小則鼠竊狗偷,大則鯨吞虎據。大唐舉義,兆庶歸仁,高祖運應瑤圖,太宗天資神武,群兇席卷,寰海鏡清,祚享永年,
 功宣後代,謚曰神堯、文武,豈不韙哉!



 贊曰:失政資盜,圖王僭號。真主勃興,風驅電掃。



\end{pinyinscope}