\article{卷四十五 志第二十五 輿服}

\begin{pinyinscope}

 昔黃帝造車服,為
 之屏蔽,上古簡儉,未立等威。而三、
 五之君,不相沿習,乃改正朔,易服色,車有輿輅之別,服有裘冕之差,文之以染繢,飾之以絺繡,華蟲象物,龍火分形,於是典章興矣。周自夷王削弱,諸侯自恣。窮孔翬之羽毛,無
 以供其侈;極隨和之掌握,不足慊其華。則皮弁革舄之容,非珠履鷸冠之玩也。迨秦誅戰國,斟酌舊儀,則有鹵簿、金根、大駕、法駕,備千乘萬騎,異《舜典》、《周官》。漢氏因之,號乘輿三駕,儀衛之盛,無與比隆。東京帝王,博雅好古,明帝始令儒者考《曲臺》之說,依《周官》五輅六冕之文,山龍藻火之數,創為法服。雖有制作,竟
 寢不行。輿駕乘金根而已。服則袞冕,冠則通天。其後所御,多從袍服。事具前志。而裘冕之服,歷代不行。後魏、北齊,輿服奇詭,至隋氏一統,始復舊儀。



 隋制,車有四等,有亙憲、通憲、軺車、輅車。初制五品以上乘偏憲,其後嫌其不美,停不行用,以亙車代之。三品以上通憲車,則青壁。一品軺車,油憲硃網。唯輅車一等,聽敕始得乘之。馬珂,一品以下九子,四品七子,五品五子。



 衣裳有常服、公服、朝服、祭服四等之制。



 平巾幘,牛角簪,紫衫,白袍、靴,起梁帶。五品已上,金玉鈿飾,用犀為簪,是為常服,武官盡服之。六品已下,衫以緋。至於大仗陪立,五品已上及親侍加兩襠滕蛇,其勛侍去兩襠。



 弁冠,硃衣裳,素革帶,烏皮履,是為公服。其弁通用烏漆紗為之,象牙為簪導。五品已上,亦以鹿胎為弁,犀為簪導者。加玉琪之飾:一品九琪,二
 品八琪,三品七琪,四品六琪。三品兼有紛、鞶囊,佩於革帶之後,上加玉珮一。鞶囊:二品以上金縷,三品以上銀縷,五品以上彩縷,文官尋常入內及在本司常服之。



 親王,遠游三梁冠,金附蟬,犀簪導,白筆。三師三公、太子三師三少、尚書秘書二省、九寺、四監、太子三寺、諸郡縣關市、親王文學、籓王嗣王、公侯,進賢冠。三品以上三梁,五品以上兩梁,犀簪導。九品以上一梁,牛角簪導。門下、內書、殿內三省,諸衛府,長秋監,太子左右庶子、內坊、諸率,宮門內坊,親王府都尉,府鎮防戍九品以上,散官一品已下,武弁幘。侍中、中書令,加貂蟬,珮紫綬。散官者,白筆。御史、司隸二臺,法冠。



 一名獬豸冠。謁者臺大夫以下,高山冠。並絳紗單衣,白紗內單,皁領、褾、襈、裾,白練裙襦,絳蔽膝,革帶,金飾鉤暐,方心曲領,紳帶,玉鏢金飾劍,亦通用金鏢,山玄玉佩,綬,襪,烏皮舄。是為朝服。玉佩,纁硃綬,施二玉環。三品以上綠綬,四品、五品青綬。二品以下去玉環,六品以下去
 劍、珮、綬。八品以下,冠去白筆,衣省內單及曲領、蔽膝,著烏皮履。五品加紛、鞶囊。其綬纁硃者,用四彩,赤、紅、縹、紺紅。硃質,纁文織,長一丈八尺,二百四十首,闊九寸。綠綬用四彩,綠、紫、黃、硃紅。綠質,長一丈八尺,二百四十首,闊九寸。紫綬用四彩,紫、黃、赤、紅。紫質,長一丈六尺,一百八十首,闊八寸。青綬三彩,白、青、紅。青質,長一丈四尺,一百四十首,闊七寸。



 玄衣纁裳冕而旒者,是為祭服,綬、珮、劍各依朝服之數。其章逢七品以下,降二為差,六品以下
 無章。



 文武之官皆執笏,五品以上,用角牙為之,六品以下,用竹木。



 是時,內外群官,文物有序,僕御清道,車服以庸。於是貴賤士庶,較然殊異。越王侗於東都嗣位,下詔停廢。自茲以後,浸以不章,以至於亡。



 唐制,天子車輿有玉輅、金輅、象輅、革輅、木輅,是為五輅,耕根車、安車、四望車,已上八等,並供服乘之用。其外有指南車、記里鼓車、白鷺車、鸞旗車、闢惡車、軒車、豹尾車、羊車、黃鉞車,豹尾、黃鉞二車,武德中無,自貞觀已後加焉。其黃鉞,天寶元年制改為金鉞。屬車十二乘,並為儀仗之用。大駕行
 幸,則分前後,施於鹵簿之內。若大陳設,則分左右,施於儀衛之內。



 玉輅,青質,以玉飾諸末。重輿,左青龍,右白虎,金鳳翅,畫虡文鳥獸,黃屋左纛。金鳳一在軾前,十二鑾在衡,正縣鑾數,皆其副輅,及耕根則八。



 二鈴在軾,
 龍輈前設鄣塵,青蓋黃裏,繡飾,博山鏡子,樹羽,輪皆硃班重牙。左建旗十有二旒,皆畫升龍,其長曳地。
 右載闟戟,長四尺,廣三尺,黻文。旗首金龍頭銜結綬及鈴綏。駕蒼龍,金鍐方釳,插翟尾五焦,鏤錫,鞶纓十有二就。



 錫,馬當顱,鏤金為之。鞶纓鞍皆以五彩飾之。就,成也,一匝為一就也。祭祀、納後則供之。



 金輅,赤質,以金飾諸末,餘與玉輅同,駕赤昚,鄉射、祀還、飲至則供之。



 象輅,黃質,以象飾諸末,餘與玉輅同,駕黃昚,行道則供之。



 革輅,白質,鞔之以革,餘與玉輅同,駕白駱,巡狩、臨兵事則供之。



 木輅,黑質,漆之,餘與玉略同,駕黑昚,畋獵則供之。



 五輅之蓋,旌旗之質及鞶纓,皆從輅
 色,蓋之里皆用黃。其鏤錫,五輅同。



 耕根車,青質,蓋三重,餘與玉輅同,耕籍則供之。



 安車,金飾,重輿,曲壁,八鑾在衡,紫油纁,硃里通AW,硃絲絡網,硃鞶纓,硃覆閤朆,貝絡,駕赤昚,臨幸則供之。



 四望車,制同犢車,金飾。八鑾在衡,青油纁,硃里通AW,硃絲絡網,拜陵、臨吊則供之。



 自高宗不喜乘輅,每有大禮,則御輦以來往。爰洎則天以後,遂以為常。玄宗又以輦不中禮,又廢而不用。開元十一年冬,將有事於南郊,乘輅而往,禮畢,騎而還。自此行幸及
 郊祀等事,無遠近,皆騎於儀衛之內。其五輅及腰輿之屬,但陳於鹵簿而已。



 皇后車則有重翟、厭翟、翟車、安車、四望車、金根車六等。



 重翟車,青質,金飾諸末,輪畫硃,金根車牙,其箱飾以重翟羽,青油纁,硃里通AW,繡紫帷,硃絲絡網,繡紫絡帶,八鑾在衡,鏤錫,鞶纓十二就,金鍐方釳,插翟尾,硃絲,總以硃為之,如馬纓而小,著馬勒,在兩耳與兩鑣也。駕蒼龍,受冊、從祀、享廟則供之。厭翟,赤質,金飾諸末,輪畫硃牙,其箱飾以次翟羽,紫油纁,硃里通幰,紅錦帷,硃絲絡網,紅錦
 絡帶,餘如重翟車。駕赤昚,採桑而供之。翟車,黃質,金飾諸末,輪畫硃牙,其車側飾以翟羽,黃油纁,黃裏通幰,白紅錦帷,硃絲絡網,白紅錦絡帶,餘如重翟。駕黃昚,歸寧則供之。諸鞶纓之色,皆從車質。安車,赤質,金飾,紫通幰硃里。駕四馬,臨幸及吊則供之。四望車,硃質,紫油通幰,油畫絡帶。拜陵、臨吊則供之。金根車,硃質,紫油通幰,油畫絡帶,硃絲網。常行則供之。



 皇太子車輅,有金輅、軺車、四望車。



 金輅,赤質,金飾諸末,重較,箱畫虡文鳥獸,黃屋,伏
 鹿軾,龍輈,金鳳一在軾,前設鄣塵,硃蓋黃裏,輪畫硃牙,左建旗九旒,右載闟戟,旗首金龍頭銜結綬及鈴綏。駕赤昚四,八鑾在衡,二鈴在軾,金鍐方釳,插翟尾五焦,鏤錫,鞶纓九就。從祀享、正冬大朝、納妃則供之。軺車,金飾諸末,紫幰幟硃里,駕一馬。五日常服及朝享宮臣、出入行道則供之。四望車,金飾諸末,紫油纁,通幰硃里,硃絲絡網,駕一馬。吊臨則供之。



 王公已下車輅,親王及武職一品,象飾輅。自餘及二品、三品,革輅。四品,木輅。五品,軺
 車。



 象輅,以象飾諸末,硃班輪,八鑾在衡,左建旗,旗、畫龍,一升一降。右載闟戟。



 革輅,以革飾諸末,左建FJ,(通帛為FJ,餘同象輅。木輅,以漆飾之,餘同革輅。軺車,曲壁,青通AW。諸輅皆硃質硃蓋,硃旗FJ。一品九旒,二品八旒,三品七旒,四品六旒,其鞶纓就數皆準此。



 內命婦夫人乘厭翟車,嬪乘翟車,婕妤已下乘安車,各駕二馬。外命婦、公主、王妃乘厭翟車,駕二馬。自餘一品乘白銅飾犢車,青通AW,硃里油纁,硃絲絡網,駕以牛。二品已下去油纁、絡網,四品青偏
 AW。



 有唐已來,三公已下車輅,皆太僕官造貯掌。若受制行冊命及二時巡陵、婚葬則給之。自此之後,皆騎馬而已。



 唐制,天子衣服,有大裘之冕、袞冕、鷩冕、毳冕、繡冕、玄冕、通天冠、武弁、黑介幘、白紗帽、平巾幘、白帢,凡十二等。



 大裘冕,無旒,廣八寸,長一尺六寸,玄裘纁里,已下廣狹準此。金飾,玉簪導,以組為纓,色如其綬。裘以黑羔皮為之,玄領、褾、襟緣。硃裳,白紗中單,皁領,青褾、襈、裾、革帶,玉鉤、暐,大帶,素帶硃里,紺其外,上以硃,下以綠,紐用組也。蔽漆隨裳。鹿盧玉具劍,火珠鏢首。
 白玉雙珮,玄組雙大綬,六彩,玄、黃、赤、白、縹、綠、純玄質,長二丈四尺,五百首,廣一尺。



 小雙綬長二尺一寸,色同大綬而首半之,間施三玉環。硃襪,赤舄。祀天神地祇則服之。



 袞冕,金飾,垂白珠十二旒,以組為纓,色如其綬,黈纊充耳,玉簪導。玄衣,纁裳,十二章,八章在衣,日、月、星、龍、山、華蟲、火、宗彞;四章在裳,藻、粉米、黼、黻,衣褾、領為升龍,織成為之也。



 各為六等,龍、山以下,每章一行,十二。白紗中單,黼領,青褾、襈、裾,黻。繡龍、山、火三章,餘同上。



 革帶、大帶、劍、珮、綬與上同。舄加金飾。諸祭祀及廟、遣上將、徵還、飲至、踐阼、加元服、納后、若元
 日受朝,則服之。



 鷩冕,服七章,三章在衣,華蟲、火、宗彞;四章在裳,藻、粉米、黼、黻。餘同袞冕。有事還主則服之。毳冕,服五章,三章在衣,宗彞、藻、粉米;二章在裳,黼、黻也。餘同鷩冕。祭海岳則服之。繡冕,服三章,一章在衣,粉米;二章在裳,黼、黻。餘同毳冕,祭社稷、帝社則服之。玄冕服,衣無章,裳刺黼一章。



 餘同繡冕。蠟祭百神、朝日夕月則服之。通天冠,加金博山,附蟬十二首,施珠翠,黑介幘,發纓翠綏,玉若犀簪導。絳紗里,白紗中單,領,褾,飾以織成。硃襈、裾,白裙,白裙襦。亦裙衫也。絳紗蔽漆,白假帶,方心曲領。其革帶、珮、劍、綬、衣蔑、舄與
 上同。若未加元服,則雙童髻,空頂黑介幘,雙玉導,加寶飾。諸祭還及冬至朔日受朝、臨軒拜王公、元會、冬會則服之。武弁,金附蟬,平巾幘,餘同前服。講武、出征、四時蒐狩、大射、祃、類、宜社、賞祖、罰社、纂嚴則服之。弁服,弁以鹿皮為也。十有二琪,琪以白玉珠為之。



 玉簪導,絳紗衣,素裳,革帶,白玉雙珮,鞶囊,小綬,白襪,烏皮履。朔日受朝則服之。黑介幘,白紗單衣,白裙襦,革帶,素襪,烏皮履。拜陵則服之。白紗帽,亦烏紗也。白裙襦,亦裙衫也。



 白襪,烏皮履。視朝聽訟及宴見賓客則服
 之。平巾幘,金寶飾。導簪冠文皆以玉,紫褶,亦白褶。白褲,玉具裝,真珠寶細帶。乘馬則服之。白帢,臨大臣喪則服之。



 太宗又制翼善冠,朔、望視朝,以常服及帛練裙襦通著之。若服褲褶,又與平巾幘通用。著於令。其常服,赤黃袍衫,折上頭巾,九環帶,六合靴,皆起自魏、周,便於戎事。自貞觀已後,非元日、冬至受朝及大祭祀,皆常服而己。



 顯慶元年九月,太尉長孫無忌與修禮官等奏曰:



 準武德初撰《衣服令》,天子祀天地,服大裘冕,無旒。臣無忌、志寧、敬
 宗等謹按《郊特牲》云:「周之始郊,日以至。」「被袞以象天,戴冕藻十有二旒,則天數也。」而此二禮,俱說周郊,袞與大裘,事乃有異。按《月令》:「孟冬,天子始裘。」明以禦寒,理非當暑,若啟蟄祈穀,冬至報天,行事服裘,義歸通允。至於季夏迎氣,龍見而雩,炎熾方隆,如何可服?謹尋歷代,唯服袞章,與《郊特牲》義旨相協。按周遷《輿服志》云,漢明帝永平二年,制採《周官》、《禮記》,始制祀天地服,天子備十二章。沈約《宋書志》云:「魏、晉郊天,亦皆服袞。」又王智深《宋紀》曰:「
 明帝制云,以大冕純玉藻、玄衣、黃裳郊祀天地。」後魏、周、齊,迄於隋氏,勘其禮令,祭服悉同。斯則百王通典,炎涼無妨,復與禮經事無乖舛。今請憲章故實,郊祀天地,皆服袞冕,其大裘請停,仍改禮令。又檢《新禮》,皇帝祭社稷服繡冕,四旒,三章。祭日月服玄冕,三旒,衣無章。謹按:令文是四品五品之服,此則三公亞獻,皆服袞衣,孤卿助祭,服毳及鷩,斯乃乘輿章數,同於大夫,君少臣多,殊為不可。據《周禮》云:「祀昊天上帝則服大裘而冕,五帝亦如
 之。享先王則袞冕,享先公則鷩冕,祀四望山川則毳冕,祭社稷五祀則糸希冕,諸小祀則玄冕。」又云:「公侯伯子男孤卿大夫之服,袞冕以下,皆如王之服。」所以《三禮義宗》,遂有二釋。一云公卿大夫助祭之日,所著之服,降王一等。又云悉與王同。求其折衷,俱未通允。但名位不同,禮亦異數。天子以十二為節,義在法天,豈有四旒三章,翻為御服。若諸臣助祭,冕與王同,便是貴賤無分,君臣不別。如其降王一等,則王著玄冕之時,群臣次服爵弁,既
 屈天子,又貶公卿。《周禮》此文,久不施用。亦猶祭祀之立尸侑,君親之拜臣子,覆巢設硩蔟之官,去曈置蟈氏之職,唯施周代,事不通行。是故漢、魏以來,下迄隋代,相承舊事,唯用袞冕。今《新禮》親祭日月,仍服五品之服。臨事施行,極不穩便。請遵歷代故實,諸祭並用袞冕。



 制可之。



 無忌等又奏曰:「皇帝為諸臣及五服親舉哀,依禮著素服。今令用雲白帢,禮令乘舛,須歸一塗。且白帢出自近代,事非稽古,雖著令文,不可行用。請改從素服,以會禮
 文。」制從之。自是鷩冕已下,乘輿更不服之,白帢遂廢,而令文因循,竟不改削。



 開元十一年冬,玄宗將有事於南郊,中書令張說又奏稱:「準令,皇帝祭昊天上帝,服大裘之冕,事出《周禮》,取其質也。永徽二年,高宗親享南郊用之。明慶年修禮,改用袞冕,事出《郊特牲》,取其文也。自則天已來用之。若遵古制,則應用大裘,若便於時,則袞冕為美。」令所司造二冕呈進,上以大裘樸略,冕又無旒,既不可通用於寒暑,乃廢不用之。自是元正朝會依禮令
 用袞冕及通天冠,大祭祀依《郊特牲》亦用袞冕。自餘諸服,雖在於令文,不復施用。十七年,朝拜五陵,但素服而已。朔、望常朝,亦用常服,其翼善冠亦廢。



 《武德令》:皇太子衣服,有袞冕、具服遠游三梁冠、公服遠游冠、烏紗帽、平巾幘五等。貞觀已後,又加弁服、進德冠之制。



 袞冕,白珠九旒,以組為纓,色如其綬,青纊充耳,犀簪導。玄衣,纁裳,九章。五章在衣,龍、山、華蟲、火、宗彞;四章在裳,藻、粉米,黼、黻。織成為之。



 白紗中單,黼領,青褾、襈、裾。革帶,金鉤暐,大帶,素帶硃里,亦紕以硃綠,皆用組,黻。隨裳色,火、山二章也。
 玉具劍,金寶飾也。



 玉鏢首。瑜玉雙珮,硃組雙大綬,四彩,赤、白、縹、紺,純硃質,長一丈八尺,三百二十首,廣九寸。小雙綬長二尺六寸,色同大綬而首半之,施二玉環也。



 硃襪赤舄。舄加金飾。侍從皇帝祭祀及謁廟、加元服、納妃則服之。



 具服遠游三梁冠,加金附蟬九首,施珠翠,黑介幘,發纓翠綏,犀簪導。絳紗袍,白紗中單,皁領、褾、襈、裙,白裙襦,白假帶,方心曲領,絳紗蔽膝。其革帶、劍、珮、綬、襪、舄與上同。後改用白襪、黑舄。未冠則雙單髻,空頂黑介幘,雙玉導,加寶飾。謁廟還宮、元日冬至
 朔日入朝、釋奠則服之。公服遠游冠,簪導以下並同前也。絳紗單衣,白裙襦,革帶,金鉤暐,假帶,方心,紛,鞶囊,長六尺四寸,廣二寸四分,色同大綬。



 白襪,烏皮履。五日常服、元日冬至受朝則服之。平巾幘,紫褶,白褲,寶細起梁帶。乘馬則服之。弁服,弁以鹿皮為之。



 犀簪導,組纓,玉琪九,絳紗衣,素裳,革帶,鞶囊,小綬,雙珮,白襪,烏皮履。朔望及視事則兼服之。進德冠,九琪,加金飾,其常服及白練裙襦通著之。若服褲褶,則與平巾幘
 通著。



 自永徽已後,唯服袞冕、具服、公服而已。若乘馬褲,則著進德冠,自餘並廢。若宴服、常服,紫衫袍與諸王同。



 開元二十六年,肅宗升為皇太子,受冊,太常所撰儀注有服絳紗袍之文。太子以為與皇帝所稱同,上表辭不敢當,請有以易之。玄宗令百官詳議。尚書左丞相裴耀卿、太子太師蕭嵩等奏曰:「謹按《衣服令》,皇太子具服,有遠游冠,三梁,加金附蟬九首,施珠翠,黑介幘,發纓綏,犀簪導,絳紗袍,白紗中單,皁領、褾、襈,白裙襦,方心曲領,
 絳紗蔽膝,革帶,劍,珮,綬等,謁廟還宮、元日冬至朔日入朝、釋奠則服之。其絳紗袍則是冠衣之內一物之數,與裙襦、劍、珮等無別。至於貴賤之差,尊卑之異,則冠為首飾,名制有殊,並珠旒及裳彩章之數,多少有別,自外不可事事差異。亦有上下通服,名制是同,禮重則具服,禮輕則從省。今以至敬之情,有所未敢,衣服不可減省,稱謂須更變名。望所撰儀注,不以絳紗袍為稱,但稱為具服,則尊卑有差,謙光成德。」議奏上,手敕改為硃明服,下
 所司行用焉。



 《武德令》,侍臣服有袞、鷩、毳、繡、玄冕,及爵弁,遠游、進賢冠,武弁,獬豸冠,凡十等。



 袞冕,垂青珠九旒,以組為纓,色如其綬,以下旒、纓皆如之也。青纊充耳,簪導。青衣,纁裳,服九章。五章在衣,龍、山、華蟲、火、宗彞,為五等。四章在裳,藻、粉米、黼、黻。皆絳為繡,遍衣而已,下皆如之。



 白紗中單,黼領,繡冕以下,中單青領。青褾、襈裙。革帶,鉤暐,大帶,三品已上,素帶硃里,皆紕其外,上以綠。五品帶,紕其垂,外以玄黃。紐皆用青組之。



 黻凡黻皆隨裳色。毳冕以上,山、火二章,繡冕山一章,玄冕無章。劍,珮,綬,硃襪,赤舄,第一品服之。



 鷩冕,七旒,服七章,三章在衣,華蟲、火、宗彞;四章在裳,藻、粉米、黼、黻也。



 餘同袞冕,第二
 品服之。



 毳冕,五旒,服五章,三章在衣,宗彞、藻、粉米;二章在裳,黼、黻也。餘同鷩冕,第三品服之。



 繡冕,四旒,服三章,一章在衣,粉米;二章在裳,黼,黻。餘並同毳冕,第四品服之。



 玄冕,衣無章,裳刻黻一章,餘同繡冕,第五品服之。



 爵弁,色同爵,無旒無章。玄纓,簪導,青衣,纁裳,白紗中單,青領、褾、裙,革帶,鉤暐,大帶,練帶,紕其垂,內外以繡,紐約用青組。爵韠,襪,赤履,九品已上服之。凡冕服,助祭及親迎若私家祭祀皆服之,爵弁亦同。凡冕,制皆以羅為之,其服以紬。爵弁用紬為之,其服用繒。



 遠游三梁冠,黑介幘,青綏。凡文
 官皆青綏,以下準此也。皆諸王服之,親王則加金附蟬。



 進賢冠,三品以上三梁,五品以上兩梁,九品以上一梁。皆三公、太子三師三少、五等爵、尚書省、秘書省、諸寺監學、太子詹事府、三寺及散官,親王師友、文學、國官,若諸州縣關津岳瀆等流內九品以上服之。



 武弁,平巾幘,侍中、中書令則加貂蟬,侍左者左珥,侍右者右珥。皆武官及門下、中書、殿中、內侍省、天策上將府、諸衛領軍武候監門、領左右太子諸坊諸率及鎮戍流內九品已上服之。其親王府佐九品以上,亦準此。
 法冠,一名獬豸冠,以鐵為柱,其上施珠兩枚,為獬豸之形。左右御史臺流內九品以上服之。高山冠者,內侍省內謁者及親王下司閤等服之。卻非冠者,亭長、門僕服之。諸應冠而未冠者,並雙童髻,空頂幘。五品已上雙玉導,金飾,三品以上加寶飾,六品以下無飾。朝服,亦名具服。冠,幘,纓,簪導,絳紗單衣,白紗中單,皁領、襈、裙,白裙襦,亦裙衫也。



 革帶,鉤暐,假帶,曲領方心,絳紗蔽膝,襪,舄,劍,珮,綬。一品已下,五品以下,陪祭、朝饗、拜表大事則服之。七品已上,
 去劍、珮、綬,餘並同。公服,亦名從省服。冠,幘,纓,簪導,絳紗單衣,白裙襦,亦裙衫也。



 革帶,鉤暐,假帶,方心,襪,履,粉,鞶囊。一品以下,五品以上,謁見東宮及餘公事則服之。其六品以下,去紛、鞶囊,餘並同。諸珮綬者,皆雙綬。親王纁硃綬,四彩,赤、黃,縹、紺。純硃質,纁文織。長一丈八尺,二百四十首,廣九寸。一品綠綟綬,四彩,紫、黃、赤。純綠質,長一丈八尺,二百四十首,廣九寸。二品、三品紫綬,三彩,紫、黃、赤。純紫質。長一丈六尺,一百八十首,廣八寸。四品青綬,三彩,青、白、
 紅。純青質。長一丈四尺,一百四十首,廣七寸。五品黑綬,二彩,青、紺。純紺質。長一丈二尺,一百首,廣六寸。



 自王公以下皆有小雙綬,長二尺六寸,色同大綬而首半之。正第一品佩二玉環,自外不同也。



 有綬者則有紛,皆長六尺四寸,廣二尺四分,各隨綬色。諸鞶囊,二品以上金縷,三品金銀鏤,四品銀鏤,五品彩鏤。諸珮,一品珮山玄玉,二品以下、五品以上,佩水蒼玉。



 諸文官七品以上朝服者,簪白筆,武官及爵則不簪。諸舄履並烏色,舄重皮底,履單皮底。別注色者,不用此色。



 諸勛官及爵任職事官者,
 散官、散號將軍同職事。正衣本服,自外各從職事服。諸致仕及以理去官,被召謁見,皆服前官從省服。平巾幘,簪導,冠支,五品以上紫褶,六品以下緋褶,加兩襠滕蛇,並白褲,起梁帶。五品以上,金玉雜細。六品以下,金飾隱起。靴,武官及衛官陪立大仗則服之。若文官乘馬,亦通服之,去兩襠滕蛇。諸視品府佐,武弁,平巾幘。國官,進賢一梁冠,黑介幘,簪導。其服各準正品,其流外官,亦依正品流外之例。參朝則服之。若謁見府公,府佐平巾黑幘,國官黑介幘,皆白紗單衣,烏皮履。



 諸流外官
 行署,三品以上黑介幘,絳公服,用緋為之,制同絳紗單衣。方心,革帶,鉤暐,假帶,襪,烏皮履。九品以上絳戺衣,制同絳公服,袖狹,形直如溝,不垂。去方心、假帶,餘同絳公服。其非行署者,太常寺謁者、卜博士、醫助教、祝史、贊引,鴻臚寺掌儀、諸典書、典學,內侍省內典引,太子門下坊典儀、內坊導客舍人、諸贊者,王公以下舍人,公主謁者等,各準行署,依品服。自外及民任雜掌無官品者,皆平巾幘,緋衫,大口褲。朝集從事則服之。諸典謁,武弁,絳公服。其齋郎,介幘,涘衣。自
 外品子任雜掌者,皆平巾幘,緋衫,大口褲。朝集從事則服之。黑介幘,簪導,深衣,青褾、領,革帶,烏皮履。未冠則雙童髻,空頂黑介幘,去革帶。國子、太學、四門學生參見則服之。書算學生、州縣學生,則烏紗帽,白裙襦,青領。諸外官拜表受詔皆服。



 本品無朝服者則服之。其餘公事及初上,並公服。諸州大中正,進賢一梁冠,絳紗公服,若有本品者,依本品參朝服之。諸州縣佐史、鄉正、里正、岳瀆祝史、齋郎,並介幘,絳涘衣。



 平巾幘,緋褶,大口褲,紫附涘,尚食局主食、
 典膳局主食、太官署食官署掌膳服之。平巾綠幘,青布褲,尚食局主膳、典膳局典食、太官署食官署供膳服之。平巾五辮髻,青褲褶,青耳屩,羊車小史服之。總角髻,青褲褶,漏刻生、漏童服之。



 龍朔二年九月戊寅,司禮少常伯孫茂道奏稱:「諸臣九章服,君臣冕服,章數雖殊,飾龍名袞,尊卑相亂。望諸臣九章衣以云及麟代龍,升山為上,仍改冕。」當時紛議不定。儀鳳年,太常博士蘇知機又上表,以公卿以下冕服,請別立節文。敕下有司詳議。
 崇文館學士校書郎楊炯奏議曰;



 古者太昊庖犧氏,仰以觀象,俯以察法,造書契而文籍生。次有黃帝軒轅氏,長而敦敏,成而聰明,垂衣裳而天下理。其後數遷五德,君非一姓。體國經野,建邦設都,文質所以再而復,正朔所以三而改。夫改正朔者,謂夏后氏建寅,殷人建丑,周人建子。至於以日系月,以月系時,以時系年,此則三王相襲之道也。夫易服色者,謂夏后氏尚黑,殷人尚白,周人尚赤。至於山、龍、華蟲、宗彞、藻、火、粉米、黼、黻,此又百代
 可知之道也。謹按《虞書》曰:「予欲觀古人之象,日、月、星辰、山、龍、華蟲作繪,宗彞、藻、火、粉米、黼、黻絺繡。」由此言之,則其所從來者尚矣。



 夫日月星辰者,明光照下土也。山者,布散雲雨,象聖王澤沾下人也。龍者,變化無方,象聖王應機布教也。華蟲者,雉也,身被五採,象聖王體兼文明也。宗彞者,武蜼也,以剛猛制物,象聖王神武定亂也。藻者,逐水上下,象聖王隨代而應也。火者,陶冶烹飪,象聖王至德日新也。米者,人恃以生,象聖王物之所賴也。黼
 能斷割,象聖王臨事能決也。黻者,兩己相背,象君臣可否相濟也。逮有周氏,乃以日月星辰為旌旗之飾,又登龍於山,登火於宗彞,於是乎制袞冕以祀先王也。九章者,法於陽數也。以龍為首章者,袞者卷也,龍德神異,應變潛見,表聖王深沈遠智,卷舒神化也。又制鷩冕以祭先公也。鷩者雉也,有耿介之志,表公有賢才,能守耿介之節也。又制毳冕以祭四望也。四望者,岳瀆之神也。武蜼者,山林所生也,明其象也。制絺冕以祭社稷也。社稷,
 土穀之神也。粉米由之成也,象其功也。又制玄冕以祭群小祀也。百神異形,難可遍擬,但取黻之相背異名也。夫以周公之多才也,故化定制禮,功成作樂。夫以孔宣之將聖也,故行夏之時,服周之冕。先王之法服,乃此之自出矣;天下之能事,又於是乎畢矣。



 今表狀「請制大明冕十二章,乘輿服之」者。謹按,日月星辰者,已施旌旗矣;龍武山火者,又不逾於古矣。而云麟鳳有四靈之名,玄龜有負圖之應,云有紀官之號,水有感德之祥,此蓋別
 表休徵,終是無逾比象。然則皇王受命,天地與興符,仰觀則璧合珠連,俯察則銀黃玉紫。盡南宮之粉壁,不足寫其形狀;罄東觀之鉛黃,無以紀其名實。固不可畢陳於法服也。雲也者,從龍之氣也,水也者,藻之自生也,又不假別為章目也。此蓋不經之甚也。



 又「鷩冕八章,三公服之」者。鷩者,太平之瑞也,非三公之德也。鷹鸇者,鷙鳥也,適可以辨祥刑之職也。熊羆者,猛獸也,適可以旌武臣之力也。又稱藻為水草,無所法象,引張衡賦云,「蒂倒茄
 於藻井,披江葩之狎獵。」謂為蓮花,取其文採者。夫茄者蓮也,若以蓮花代藻,變古從今,既不知草木之名,亦未達文章之意。此又不經之甚也。



 又「毳冕六章,三品服之」者。按此王者祀四望服之名也。今三品乃得同王之毳冕,而三公不得同王之袞名。豈惟顛倒衣裳,抑亦自相矛盾。此又不經之甚也。



 又「黼冕四章,五品服之」。考之於古,則無其名;驗之於今,則非章首。此又不經之甚也。



 若夫禮惟從俗,則命為制,令為詔,乃秦皇之故事,猶可以
 適於今矣。若乃義取隨時,則出稱警,入稱蹕,乃漢國之舊儀,猶可以行於代矣。亦何取於變周公之軌物,改宣尼之法度者哉!



 由是竟寢知機所請。



 景龍二年七月,皇太子將親釋奠於國學,有司草儀注,令從臣皆乘馬著衣冠。太子左庶子劉子玄進議曰:



 古者自大夫已上皆乘車,而以馬為騑服。魏、晉已降,迄於隋代,朝士又駕牛車,歷代經史,具有其事,不可一二言也。至如李廣北征,解鞍憩息;馬援南伐,據鞍顧盼。斯則鞍馬之設,行於軍
 旅,戎服所乘,貴於便習者也。案江左官至尚書郎而輒輕乘馬,則為御史所彈。又顏延之罷官後,好騎馬出入閭里,當代稱其放誕。此則專車憑軾,右擐朝衣;單馬御鞍,宜從褻服。求之近古,灼然之明驗矣。



 自皇家撫運,沿革隨時。至如陵廟巡幸,王公冊命,則盛服冠履,乘彼輅車。其士庶有衣冠親迎者,亦時以服箱充馭。在於他事,無復乘車,貴賤所行,通鞍馬而已。臣伏見比者鑾輿出幸,法駕首途,左右侍臣皆以朝服乘馬。夫冠履而出,止可
 配車而行,今乘車即停,而冠履不易,可謂唯知其一而未知其二。何者?褒衣博帶,革履高冠,本非馬上所施,自是車中之服。必也襪而升鐙,跣以乘鞍,非惟不師古道,亦自取驚今俗,求諸折中,進退無可。且長裙廣袖,襜如翼如,鳴珮紆組,鏘鏘弈弈,馳驟於風塵之內,出入於旌棨之間,儻馬有驚逸,人從顛墜,遂使屬車之右,遺履不收,清道之傍,絓驂相續,固以受嗤行路,有損威儀。



 今議者皆雲秘閣有《梁武南郊圖》,多有衣冠乘馬者,此則
 近代故事,不得謂無其文。臣案此圖是後人所為,非當時所撰。且觀當今有古今圖畫者多矣,如張僧繇畫《群公祖二疏》,而兵士有著芒屩者;閻立本畫《昭君入匈奴》,而婦人有著帷帽者。夫芒屩出於水鄉,非京華所有;帷帽創於隋代,非漢宮所作。議者豈可徵此二畫以為故實者乎!由斯而言,則《梁武南郊之圖》,義同於此。又傳稱義惟因俗,禮貴緣情。殷輅周冕,規模不一;秦冠漢珮,用舍無恆。況我國家道軼百王,功高萬古,事有不便,資於
 變通。其乘馬衣冠,竊謂宜從省廢。臣此異議,其來自久,日不暇給,未及榷揚。今屬殿下親從齒胄,將臨國學,凡有衣冠乘馬,皆憚此行,所以輒進狂言,用申鄙見。



 皇太子手令付外宣行,仍編入令,以為恆式。



 宴服,蓋古之褻服也,今亦謂之常服。江南則以巾褐裙襦,北朝則雜以戎夷之制。爰至北齊,有長帽短靴,合褲襖子,硃紫玄黃,各任所好。雖謁見君上,出入省寺,若非元正大會,一切通用。高氏諸帝,常服緋袍。隋代帝王貴臣,多服黃文綾
 袍,烏紗帽,九環帶,烏皮六合靴。百官常服,同於匹庶,皆著黃袍,出入殿省。天子朝服亦如之,惟帶加十三環以為差異,蓋取於便事。其烏紗帽漸廢,貴賤通服折上巾,其制周武帝建德年所造也。晉公宇文護始命袍加下襴。及大業元年,煬帝始制詔吏部尚書牛弘、工部尚書宇文愷、兼內史侍郎虞世基、給事郎許善心、儀曹郎袁朗等憲章古則,創造衣冠,自天子逮於胥吏,章服皆有等差。始令五品以上,通服硃紫。是後師旅務殷,車駕多
 行幸,百官行從,雖服褲褶,而軍間不便。六年,復詔從駕涉遠者,文武官等皆戎衣,貴賤異等,雜用五色。五品已上,通著紫袍,六品已下,兼用緋綠。胥吏以青,庶人以白,屠商以皁,士卒以黃。



 武德初,因隋舊制,天子宴服,亦名常服,唯以黃袍及衫,後漸用赤黃,遂禁士庶不得以赤黃為衣服雜飾。四年八月敕:「三品已上,大科紬綾及羅,其色紫,飾用玉。五品已上,小科紬綾及羅,其色硃,飾用金。六品已上,服絲布,雜小綾,交梭,雙紃,其色黃。六品、七
 品飾銀。八品、九品鍮石。流外及庶人服紬、絁、布,其色通用黃。飾用銅鐵。」五品已上執象笏。三品已下前挫後直,五品已上前挫後屈。自有唐已來,一例上圓下方,曾不分別。六品已下,執竹木為笏,上挫下方。其折上巾,烏皮六合靴,貴賤通用。貞觀四年又制,三品已上服紫,五品已下服緋,六品、七品服綠,八品、九品服以青,帶以鍮石。婦人從夫色。雖有令,仍許通著黃。五年八月敕,七品已上,服龜甲雙巨十花綾,其色綠。九品已上,服絲布及雜
 小綾,其色青。十一月,賜諸衛將軍紫袍,錦為褾袖。八年五月,太宗初服翼善冠,貴臣服進德冠。



 龍朔二年,司禮少常伯孫茂道奏稱:「舊令六品、七品著綠,八品、九品著青,深青亂紫,非卑品所服。望請改八品、九品著碧。朝參之處,聽兼服黃。」從之。總章元年,始一切不許著黃。上元元年八月又制:「一品已下帶手巾、算袋,仍佩刀子、礪石,武官欲帶者聽之。文武三品已上服紫,金玉帶。四品服深緋,五品服淺緋,並金帶。六品服深綠,七品服淺綠,並
 銀帶。八品服深青,九品服淺青,並鍮石帶。庶人並銅鐵帶。」文明元年七月甲寅詔:「旗幟皆從金色,飾之以紫,畫以雜文。八品已下舊服者,並改以碧。京文官五品已上,六品已下,七品清官,每日入朝,常服褲褶。諸州縣長官在公衙,亦準此。」景雲中又制,令依上元故事,一品已下帶手巾、算袋,其刀子、礪石等許不佩。武官五品已上佩䪓韘七事。七謂佩刀、刀子、礪石、契苾真、噦厥針筒、火石袋等也。至開元初復罷之。則天天授二年二月,朝集使
 刺史賜繡袍,各於背上繡成八字銘。長壽三年四月,敕賜岳牧金字銀字銘袍。延載元年五月,則天內出緋紫單羅銘襟背衫,賜文武三品已上。左右監門衛將軍等飾以對師子,左右衛飾以麒麟,左右武威衛飾以對虎,左右豹韜衛飾以豹,左右鷹揚衛飾以鷹,左右玉鈐衛飾以對鶻,左右金吾衛飾以對豸,諸王飾以盤龍及鹿,宰相飾以鳳池,尚書飾以對雁。



 武德已來,始有巾子,文官名流,上平頭小樣者。則天朝,貴臣內賜高頭巾子,呼
 為武家諸王樣。中宗景龍四年三月,因內宴賜宰臣已下內樣巾子。開元已來,文官士伍多以紫皁官絁為頭巾、平頭巾子,相效為雅制。玄宗開元十九年十月,賜供奉官及諸司長官羅頭巾及官樣巾子,迄今服之也。



 天寶十載五月,改諸衛旗幡隊仗,先用緋色,並用赤黃色,以符土德。



 高祖武德元年九月,改銀菟符為銀魚符。高宗永徽二年五月,開府儀同三司及京官文武職事四品、五品,並給隨身魚。咸亨三年五月,五品已上賜新魚
 袋,並飾以銀。三品已上各賜金裝刀子、礪石一具。垂拱二年正月,諸州都督刺史,並準京官帶魚袋。天授元年九月,改內外所佩魚並作龜。久視元年十月,職事三品已上龜袋,宜用金飾,四品用銀飾,五品用銅飾。上守下行,皆從官給。神龍元年二月,內外官五品已上依舊佩魚袋。六月,郡王、嗣王特許佩金魚袋。景龍三年八月,令特進佩魚。散職佩魚,自此始也。自武德已來,皆正員帶闕官始佩魚袋,員外、判試、檢校自則天、中宗後始有之,
 皆不佩魚。雖正員官得佩,亦去任及致仕即解去魚袋。至開元九年,張嘉貞為中書令,奏諸致仕許終身佩魚,以為榮寵。以理去任,亦聽佩魚袋。自後恩制賜賞緋紫,例兼魚袋,謂之章服,因之佩魚袋、服硃紫者眾矣。



 梁制云,褲褶,近代服以從戎,今纘嚴則文武百官咸服之。車駕親戎,則縛褲不舒散也。中官紫褶,外官絳褶,舄用皮。服冠衣硃者,紫衣用赤舄,烏衣用烏舄。唯褶服以靴。靴,胡履也,取便於事,施於戎服。



 舊制,乘輿案褥、床褥、床帷,皆
 以紫為飾。天寶六載,禮儀使太常卿韋糸舀奏請依禦袍色,以赤黃為飾。從之。



 《武德令》:皇后服有褘衣、鞠衣、鈿釵禮衣三等。



 褘衣,首飾花十二樹,並兩博鬢,其衣以深青織成為之,文為翬翟之形。素質,五色,十二等。



 素紗中單,黼領,羅縠褾、襈,褾、襈皆用硃色也。蔽膝,隨裳色,以緅為領,用翟為章,三等。



 大帶,隨衣色,硃里,紕其外,上以硃錦,下以綠錦,紐約用青組。以青衣,革帶,青襪、舄,舄加金飾。白玉雙珮,玄組雙大綬。章彩尺寸與乘輿同。



 受冊、助祭、朝會諸大事則服之。鞠衣,黃羅為之。其蔽膝、大帶及衣革帶、舄隨衣色。餘與褘衣同,唯無雉也。親蠶則服之。
 鈿釵禮衣,十二鈿,服通用雜色,制與上同,唯無雉及珮綬,去舄,加履。宴見賓客則服之。



 皇太子妃服,首飾花九樹,小花如大花之數,並兩博鬢也。褕翟。青織成為之,文為搖翟之形,青質、五色、九等也。素紗中單,黼領,羅縠褾、襈,褾、襈皆用硃也。蔽膝,隨裳色,用緅為領緣,以搖翟為章,二等也。大帶,隨衣色,硃里,紕其外,上以硃錦,下以綠錦,紐用青組。以青衣,革帶,青襪、舄,舄加金飾瑜玉珮,紅硃雙大綬。章彩尺寸與皇太子同。受冊、助祭、朝會諸大事則服之。鞠衣,黃羅為之,其蔽膝、大帶及衣革帶隨衣色。餘褕翟同,唯無雉也。從蠶則服之。鈿釵禮衣,九鈿,服通用雜色,制與上同,唯無雉及珮、綬,
 去舄,加履。宴見賓客則服之。



 內外命婦服花釵,施兩博鬢,寶鈿飾也。翟衣青質,羅為之,繡為雉,編次於衣及裳,重為九等而下。第一品花鈿九樹,寶鈿準花數,以下準此也。



 翟九等。第二品花鈿八樹,翟八等。第三品花鈿七樹,翟七等。第四品花鈿六樹,翟六等。第五品花鈿五樹,翟五等。並素紗中單,黼領,硃褾、襈亦通用羅縠也。蔽膝,隨裳色,以緅為領緣,加以文繡,重雉為章二事,一品已下皆同也。大帶,隨衣色,緋其外,上以硃錦,下以綠錦,紐同青組。



 青衣,革帶,青襪、舄,珮,綬。內命婦受冊、從蠶、朝會則服之;其外命婦嫁及受冊、從蠶、大朝會亦準此。鈿釵禮衣,
 通用雜色,制與上同,唯無雉及珮綬。



 去舄,加履。第一品九鈿,第二品八鈿,第三品七鈿,第四品六鈿,第五品五鈿。內命婦尋常參見,外命婦朝參辭見及禮會則服之。六尚、寶林、御女、採女、女官等服,禮衣通用雜色,制與上同,惟無首飾。七品已上,有大事服之,尋常供奉則公服。



 公服去中單、蔽膝、大帶。九品已上,大事及尋常供奉,並公服。東宮準此。女史則半袖裙襦。諸公主、王妃珮綬同,諸王縣主、內命婦準品。外命婦五品已上,皆準夫、子,即非因夫、子別加邑
 號者,亦準品。婦人宴服,準令各依夫色,上得兼下,下不得僭上。既不在公庭,而風俗奢靡,不依格令,綺羅錦繡,隨所好尚。上自宮掖,下至匹庶,遞相仿效,貴賤無別。



 武德、貞觀之時,宮人騎馬者,依齊、隋舊制,多著FD旂,雖發自戎夷,而全身障蔽,不欲途路窺之。王公之家,亦同此制。永徽之後,皆用帷帽,拖裙到頸,漸為淺露。尋下敕禁斷,初雖暫息,旋又仍舊,咸亨二年又下敕曰:「百官家口,咸預士流,至於衢路之間,豈可全無障蔽。比來多著帷
 帽,遂棄FD旂,曾不乘車,別坐簷子。遞相仿效,浸成風俗,過為輕率,深失禮容。前者已令漸改,如聞猶未止息。又命婦朝謁,或將馳駕車,既入禁門,有虧肅敬。此並乖於儀式,理須禁斷,自今已後,勿使更然。」則天之後,帷帽大行,FD旂漸息。中宗即位,宮禁寬弛,公私婦人,無復冪旂之制。開元初,從駕宮人騎馬者,皆著胡帽,靚妝露面,無復障蔽。士庶之家,又相仿效,帷帽之制,絕不行用。俄又露髻馳騁,或有著丈夫衣服靴衫,而尊卑內外,斯一貫
 矣。



 奚車,契丹塞外用之,開元、天寶中漸至京城。兜籠,巴蜀婦人所用,今乾元已來,蕃將多著勛於朝,兜籠易於擔負,京城奚車、兜籠、代於車輿矣。



 武德來,婦人著履,規制亦重,又有線靴。開元來,婦人例著線鞋,取輕妙便於事,侍兒乃著履。臧獲賤伍者皆服襴衫。太常樂尚胡曲,貴人御饌,盡供胡食,士女皆竟衣胡服,故有範陽羯胡之亂,兆於好尚遠矣。



 太極元年,左司郎中唐紹上疏曰:



 臣聞王公已下,送終明器等物,具標甲令,品秩高下,各
 有節文。孔子曰,明器者,備物而不可用,以芻靈者善,為俑者不仁。傳曰,俑者,謂有面目機發,似於生人也。以此而葬,殆將於殉,故曰不仁。近者王公百官,競為厚葬,偶人像馬,雕飾如生,徒以眩耀路人,本不因心致禮。更相扇慕,破產傾資,風俗流行,遂下兼士庶。若無禁制,奢侈日增。望諸王公已下,送葬明器,皆依令式,並陳於墓所,不得衢路行。



 又士庶親迎之儀,備諸辨禮,所以承宗廟,事舅姑,當須昏以為期,詰朝謁見。往者下俚庸鄙,時有
 障車,邀其酒食,以為戲樂。近日此風轉盛,上及王公,乃廣奏音樂,多集徒侶,遮擁道路,留滯淹時,邀致財物,動逾萬計。遂使障車禮貺,過於聘財,歌舞喧嘩,殊非助感。即虧名教,實蠹風猷,違紊禮經,須加節制。望請婚姻家障車者,並須禁斷。其有犯者,有廕家請準犯名教例附簿,無廕人決杖六十,仍各科本罪。



\end{pinyinscope}