\article{卷二百五}

\begin{pinyinscope}

 西突厥本與北突厥同祖。初,木桿與沙缽略可汗有隙,因分為二。其國即烏孫之故地,東至突厥國,西至雷翥海,南至疏勒,北至瀚海,在長安北七千里。自焉耆國西
 北七日行,至其南庭;又正北八日行,至其北庭。鐵勒、龜茲及西域諸胡國,皆歸附之。其人雜有都陸及弩失畢、歌邏祿、處月、處密,伊吾等諸種。風俗大抵與突厥同,唯言語微差。其官有葉護,有特勒,常以可汗子弟及宗族為之;又有乙斤、屈利啜、閻洪達、頡利發、吐屯、俟斤等官,皆代襲其位。



 處羅可汗,隋煬帝大業中與其弟闕達設及特勒大奈入朝。仍從煬帝征高麗,賜號為曷薩那可汗。遇江都之
 亂,從宇文化及至河北。化及敗,歸長安,高祖為之降榻,引與同坐,封歸義郡王。獻大珠於高祖。高祖勞之曰:「珠信為寶,朕所重者赤心,珠無所用。」竟不受之。先與始畢有隙,及在京師,始畢遣使請殺之,高祖不許。群臣諫曰:「今若不與,則是存一人而失一國也,後必為患。」太宗曰:「人窮來歸我,殺之不義。」驟諫於高祖,由是遲回者久之。不得已,乃引曷薩那於內殿,與之縱酒,既而送至中書省,縱北突厥使殺之。太宗即位,令以禮改葬。



 闕達設初
 居於會寧,有部落三千餘騎。至隋末,自稱闕達可汗。武德初,遣使內屬,拜吐烏過拔闕可汗,厚加撫慰。尋為李軌所滅。



 特勒大奈,隋大業中與曷薩那可汗同歸中國。及從煬帝討遼東,以功授金紫光祿大夫。後分其部落於樓煩。會高祖舉兵,大奈率其眾以從。隋將桑顯和襲義軍於飲馬泉,諸軍多已奔退,大奈將數百騎出顯和後,掩其不備,擊,大破之,諸軍復振。拜光祿大夫。及平京城,以力戰功,賞物五千段,賜姓史氏。武德初,從太宗
 破薛舉。又從平王世充,破竇建德、劉黑闥,並有殊功。賜宮女三人,雜彩萬餘段。貞觀三年,累遷右武衛大將軍、檢校豐州都督,封竇國公,實封三百戶。十二年卒,贈輔國大將軍。初,曷薩那之朝隋也,為煬帝所拘,其國人遂立薩那之叔父,曰射匱可汗。



 射匱可汗者,達頭可汗之孫也。既立後,始開土宇,東至金山,西至海,自玉門已西諸國皆役屬之。遂與北突厥為敵,乃建庭於龜茲北三彌山,尋卒。弟統葉護可汗代
 立。



 統葉護可汗,勇而有謀,善攻戰。遂北並鐵勒,西拒波斯,南接罽賓,悉歸之。控弦數十萬,霸有西域,據舊烏孫之地。又移庭於石國北之千泉。其西域諸國王悉授頡利發,並遣吐屯一人監統之,督其征賦。西戎之盛,未之有也。



 武德三年,遣使貢條支巨卵。時北突厥作患,高祖厚加撫結,與之並力以圖北蕃,統葉護許以五年冬。大軍將發,頡利可汗聞之,大懼,復與統葉護通和,無相征伐。
 統葉護尋遣使來請婚。高祖謂侍臣曰:「西突厥去我懸遠,急疾不相得力,今請婚,其計安在?」封德彞對曰:「當今之務,莫若遠交而近攻,正可權許其婚,以威北狄。待之數年後,中國盛全,徐思其宜。」高祖遂許之婚,令高平王道立至其國,統葉護大悅。遇頡利可汗頻歲入寇,西蕃路梗,由是未果為婚。



 貞觀元年,遣真珠統俟斤與高平王道立來獻萬釘寶鈿金帶,馬五千匹。時統葉護自負強盛,無恩於國,部眾咸怨,歌邏祿種多叛之。頡利可汗不
 悅中國與之和親,數遣兵入寇,又遣人謂統葉護曰:「汝若迎唐家公主,要須經我國中而過。」統葉護患之,未克婚。為其伯父所殺而自立,是為莫賀咄侯屈利俟毗可汗。太宗聞統葉護之死,甚悼之,遣齎玉帛至其死所祭而焚之。會其國亂,不果至而止。



 莫賀咄侯屈利俟毗可汗,先分統突厥種類為小可汗,及此自稱大可汗,國人不附。弩失畢部共推泥孰莫賀設為可汗,泥孰不從。時統葉護之子咥力特勒避莫賀
 咄之難,亡在康居,泥孰遂迎而立之,是為乙毗缽羅肆葉護可汗。連兵不息,俱遣使來朝,各請婚於我。太宗答之曰:「汝國擾亂,君臣未定,戰爭不息,何得言婚!」竟不許。仍諷令各保所部,無相征伐。其西域諸國及鐵勒先役屬於西突厥者,悉叛之,國內虛耗。



 肆葉護既是舊主之子,為眾心所歸,其西面都陸可汗及莫賀咄可汗部豪帥,多來附之。又興兵以擊莫賀咄,大敗之。莫賀咄遁於金山,尋為咄陸可汗所害,國人乃奉肆葉護為大可汗。
 肆葉護可汗立,大發兵北征鐵勒,薛延陀逆擊之,反為所敗。肆葉護性猜狠信讒,無統馭之略。有乙利可汗者,於肆葉護功最多,由是授小可汗,以非罪族滅之。群下震駭,莫能自固。肆葉護素憚泥孰,而陰欲圖之,泥孰遂適焉耆。其後沒卑達干與突厥弩失畢二部豪帥潛謀擊之,肆葉護以輕騎遁於康居,尋卒。國人迎泥孰於焉耆而立之,是為咄陸可汗。



 咄陸可汗泥孰者,亦稱大渡可汗。父莫賀設,本隸統葉
 護。武德中,嘗至京師。時太宗居籓,務加懷輯,與之結盟為兄弟。既被推為可汗,遣使詣闕請降。太宗遣使賜以名號及鼓纛。貞觀七年,遣鴻臚少卿劉善因至其國,冊授為吞阿婁拔奚利邲咄陸可汗。明年,泥孰卒,其弟同娥設立,是為沙缽羅咥利失可汗。



 沙缽羅咥利失可汗,以貞觀九年上表請婚,獻馬五百疋。朝廷唯厚加撫慰,未許其婚。俄而其國分為十部,每部令一人統之,號為十設。每設賜以一箭,故稱十箭焉。
 又分十箭為左右廂,一廂各置五箭。其左廂號五咄六部落,置五大啜,一啜管一箭;其右廂號為五弩失畢,置五大俟斤,一俟斤管一箭,都號為十箭。其後或稱一箭為一部落,大箭頭為大首領。五咄六部落居於碎葉已東,五弩失畢部落居於碎葉已西,自是都號為十姓部落。



 咥利失既不為眾所歸,部眾攜貳,為其統吐屯所襲,麾下亡散。咥利失以左右百餘騎拒之,戰數合,統吐屯不利而去。咥利失奔其弟步利設,與保焉耆。其阿悉吉
 闕俟斤與統吐屯等召國人,將立欲谷設為大可汗。以咥利失為小可汗。統吐屯為人所殺,欲谷設兵又為其俟斤所破,咥利失復得舊地,弩失畢、處密等並歸咥利失。



 十二年,西部竟立欲谷設為乙毗咄陸可汗。乙毗咄陸可汗既立,與咥利失大戰,兩軍多死,各引去。因與咥利失中分,自伊列河已西屬咄陸,已東屬咥利失。咄陸可汗又建庭於鏃曷山西,謂為北庭。自厥越失、拔悉彌、駁馬、結骨、火燅、觸水昆諸國皆臣之。



 十三年,咥利失為
 其吐屯俟利發與欲谷設通謀作難,咥利失窮蹙,奔拔汗那而死。弩失畢部落酋帥迎咥利失弟伽那之子薄布特勒而立之,是為乙毗沙缽羅葉護可汗。



 乙毗沙缽羅葉護可汗既立,建庭於睢合水北,謂之南庭。東以伊列河為界,自龜茲、鄯善,且末、吐火羅、焉耆、石國、史國、何國、穆國、康國,皆受其節度。累遣使朝貢,太宗降璽書慰勉。



 貞觀十五年,令左領軍將軍張大師往授焉,賜以鼓纛。於時咄陸可汗與葉護頗相攻擊。會咄陸
 遣使詣闕,太宗諭以敦睦之道。咄陸於時兵眾漸強,西域諸國復來歸附。未幾,咄陸遣石國吐屯攻葉護,擒之,送於咄陸,尋為所殺。



 咄陸可汗既並其國,弩失畢諸姓心不服咄陸,皆叛之。咄陸復率兵擊吐火羅,破之。自恃其強,專擅西域。遣兵寇伊州,安西都護郭恪率輕騎二千自烏骨邀擊,敗之。咄陸又遣處月、處密等圍天山縣,郭恪又擊走之。恪乘勝進拔處月俟斤所居之城,追奔及於遏索山,斬首千餘級,降其處密之眾而歸。咄陸初
 以泥孰啜自擅取所部物,斬之以徇;尋為泥孰啜部將胡祿居所襲,眾多亡逸,其國大亂。



 貞觀十五年,部下屋利啜等謀欲廢咄陸,各遣使詣闕,請立可汗。太宗遣使齎璽書立莫賀咄乙毗可汗之子,是為乙毗射匱可汗。



 乙毗射匱可汗立,乃發弩失畢兵就白水擊咄陸。自知不為眾所附,乃西走吐火羅國。中國使人先為咄陸所拘者,射匱悉以禮資送歸長安,復遣使貢方物,請賜婚。太宗許之,詔令割龜茲、于闐、疏勒、硃俱波、蔥嶺等五國
 為聘禮。及太宗崩,賀魯反叛,射匱部落為其所並。



 阿史那賀魯者,曳步利設射匱特勒之子也。初,阿史那步真既來歸國,咄陸可汗乃立賀魯為葉護,以繼步真。居於多邏斯川,在西州直北一千五百里,統處密、處月、姑蘇、歌羅祿、弩失畢五姓之眾。其後,咄陸西走吐火羅國,射匱可汗遣兵迫逐,賀魯不常厥居。貞觀二十二年,乃率其部落內屬,詔居廷州。尋授左驍衛將軍、瑤池都督。高宗即位,進拜左驍衛大將軍,瑤池都督如故。



 永徽
 二年,與其子咥運率眾西遁,據咄陸可汗之地,總有西域諸郡,建牙於雙河及千泉,自號沙缽羅可汗,統攝咄陸、弩失畢十姓。其咄陸有五啜:一曰處木昆律啜;二曰胡祿居闕啜,賀魯以女妻之;三曰攝舍提暾啜;四曰突騎施賀邏施啜;五曰鼠尼施處半啜。弩失畢有五俟斤:一曰阿悉結闕俟斤,最為強盛;二曰哥舒闕俟斤;三曰拔塞幹暾沙缽俟斤,四曰阿悉結泥孰俟斤;五曰哥舒處半俟斤。各有所部,勝兵數十萬,並羈屬賀魯。西域諸
 國,亦多附隸焉。



 賀魯尋立咥運為莫賀咄葉護,數侵擾西蕃諸部,又進寇廷州。三年,詔遣左武候大將軍梁建方、右驍衛大將軍契苾何力率燕然都護所部回紇兵五萬騎討之,前後斬首五千級,虜渠帥六十餘人。四年,咄陸可汗死,其子真珠護與五弩失畢請擊賀魯,破其牙帳,斬首千餘級。



 顯慶二年,遣右屯衛將軍蘇定方,燕然都護任雅相,副都護蕭嗣業,左驍衛大將軍、瀚海都督回紇婆閏等率師討擊,仍使右武衛大將軍阿史那
 彌射、左屯衛大將軍阿史那步真為安撫大使。定方行至曳咥河西,賀魯率胡祿居闕啜等二萬餘騎列陣而待。定方率副總管任雅相等與之交戰,賊眾大敗,斬大首領都搭達干等二百餘人。賀魯及闕啜輕騎奔竄,渡伊麗河,兵馬溺死者甚眾。嗣業至千泉賀魯下牙之處,彌射進軍至伊麗水,處月、處密等部各率眾來降。彌射又進次雙河,賀魯先使步失達干鳩集散卒,據柵拒戰。彌射、步真攻之,大潰;又與蘇定方攻賀魯於碎葉水,大
 破之。



 賀魯與咥運欲投鼠耨設,至石國之蘇咄城傍,人馬饑乏,城主伊涅達干詐將酒食出迎,賀魯信其言入城,遂被拘執。蕭嗣業既至石國,鼠耨設乃以賀魯屬之。賀魯謂嗣業曰:「我破亡虜耳!先帝厚我,而我背之,今日之敗,天怒我也。舊聞漢法,殺人皆於都市,至京殺我,請向昭陵,使得謝罪於先帝,是本願也。」高宗聞而愍之。及俘賀魯至京師,令獻於昭陵及太廟,詔特免死。分其種落置昆陵、濛池二都護府,其所役屬諸國,皆分置州府,
 西盡於波斯,並隸安西都護府。四年,賀魯卒。詔葬於頡利墓側,刻石以紀其事。



 阿史那彌射者,室點密可汗五代孫也。初,室點密從單于統領十大首領,有兵十萬眾,往平西域諸胡國,自為可汗,號十姓部落,世統其眾。彌射在本蕃為莫賀咄葉護。貞觀六年,詔遣鴻臚少卿劉善因就蕃立為奚利邲咄陸可汗,賜以鼓纛、彩帛萬段。其族兄步真欲自立為可汗,遂謀殺彌射弟侄二十餘人。彌射既與步真有隙,以貞
 觀十三年率所部處月、處密部落入朝,授右監門大將軍。其後步真遂自立為咄陸葉護,其部落多不服,委之遁去。步真復攜家屬入朝,授左屯衛大將軍。



 彌射後從太宗征高麗有功,封平襄縣伯。顯慶二年,轉右武衛大將軍。及討平賀魯,乃冊立彌射為興昔亡可汗兼右衛大將軍、昆陵都護,分押賀魯下五咄六部落。步真授繼往絕可汗,兼右衛大將軍、濛池都護,仍分押五弩失畢部落。因下詔曰:



 自西蕃罹亂,三十餘年。比者賀魯猖狂,
 百姓重被劫掠。朕君臨四海,情均養育。不可使兇狡之虜,恣行侵漁,無辜之氓,久遭塗炭。故遣右屯衛將軍蘇定方等統率騎勇,北路討逐。卿等宣暢朝風,南道撫育。遂使兇渠畏威,夷人慕德,伐叛柔服,西域總平。賀魯父子既已擒獲,諸頭部落須有統領。卿早歸闕庭,久參宿衛,深感恩義,甚知法式,所以冊立卿等各為一部可汗。但諸姓從賀魯,非其本情,卿等才至即降,亦是赤心向國。卿宜與盧承慶等準其部落大小,位望高下,節級授
 刺史以下官。



 龍朔中,又令彌射、步真率所部從釭海道大總管蘇海政討龜茲。步真嘗欲並彌射部落,遂密告海政云:「彌射欲謀反,請以計誅之。」時海政兵才數千,懸師在彌射境內,遂集軍吏而謀曰:「彌射若反,我輩即無噍類。今宜先舉事,則可克捷。」乃偽稱有敕,令大總管齎物數百萬段分賜可汗及諸首領。由是彌射率其麾下,隨例請物,海政盡收斬之。共後西蕃盛言彌射非反,為步真所誣,而海政不能審察,濫行誅戮。



 則天臨朝,十姓
 無主數年,部落多散失。垂拱初,遂擢授彌射子左豹韜衛翊府中郎將元慶為左玉鈐衛將軍兼昆陵都護,令襲興昔亡可汗,押五咄六部落。步真子斛瑟羅為右玉鈐衛將軍,兼濛池都護,押五弩失畢部落。尋進授元慶左衛大將軍。



 如意元年,為來俊臣誣謀反被害。其子獻,配流崖州。長安三年,召還。累授右驍衛大將軍,襲父興昔亡可汗,充安撫招慰十姓大使。獻本蕃漸為默啜及烏質勒所侵。遂不敢還國。開元中,累遷右金吾大將軍。
 卒於長安。



 阿史那步真者,在本蕃授左屯衛大將軍。與彌射討平賀魯,加授驃騎大將軍、行右衛大將軍、濛池都護、繼往絕可汗,押五弩失畢部落。尋卒。其子斛瑟羅,本蕃為步利設,垂拱初,授右玉鈐衛將軍兼濛池都護、襲繼往絕可汗,押五弩失畢部落。天授元年,拜左衛大將軍,改封竭忠事主可汗,仍賜濛池都護。尋卒。子懷道,神龍年累授右屯衛大將軍、光祿卿,轉太僕卿兼濛池都護、十姓
 可汗。自垂拱已後,十姓部落頻被突厥默啜侵掠,死散殆盡。及隨斛瑟羅才六七萬人,徙居內地,西突厥阿史那氏於是遂絕。



 突騎施烏質勒者,西突厥之別種也。初隸在斛瑟羅下,號為莫賀達干。後以斛瑟羅用刑嚴酷,眾皆畏之,尤能撫恤其部落,由是為遠近諸胡所歸附。其下置都督二十員,各統兵七千人。嘗屯聚碎葉西北界,後漸攻陷碎葉,徙其牙帳居之。東北與突厥為鄰,西南與諸胡相接,
 東南至西廷州。斛瑟羅以部眾削弱,自則天時入朝,不敢還蕃,其地並為烏質勒所並。



 景龍二年,詔封為西河郡王,令攝御史大夫,解琬就加冊立。未至,烏質勒卒。其長子娑葛代統其眾,詔便立娑葛為金河郡王,仍賜以宮女四人。



 初,娑葛代父統兵,烏質勒下部將闕啜忠節甚忌之,以兵部尚書宗楚客當朝任勢,密遣使齎金七百兩以賂楚客,請停娑葛統兵。楚客乃遣御史中丞馮嘉賓充使至其境,陰與忠節籌其事,並自致書以申意。
 在路為娑葛游兵所獲,遂斬嘉賓,仍進兵攻陷火燒等城,遣使上表以索楚客頭。



 景龍三年,娑葛弟遮弩恨所分部落少於其兄,遂叛入突厥,請為鄉導,以討娑葛。默啜乃留遮弩,遣兵二萬人與其左右來討娑葛,擒之而還。默啜顧謂遮弩曰:「汝於兄弟尚不和協,豈能盡心於我。」遂與娑葛俱殺之。默啜兵還,娑葛下部將蘇祿鳩集餘眾,自立為可汗。



 蘇祿者,突騎施別種也。頗善綏撫,十姓部落漸歸附之,
 眾二十萬,遂雄西域之地,尋遣使來朝。開元三年,制授蘇祿為左羽林軍大將軍、金方道經略大使,進為特勒,遣侍御史解忠順齎璽書冊立為忠順可汗。自是每年遣使朝獻。上乃立史懷道女為金河公主以妻之。



 時杜暹為安西都護,公主遣牙官齎馬千疋詣安西互市。使者宣公主教與暹,暹怒曰:「阿史那氏女,豈合宣教與吾節度耶!」杖其使者,留而不遣,其馬經雪,寒死並盡。蘇祿大怒,發兵分寇四鎮。會杜暹入知政事,趙頤貞代為安
 西都護,城守久之,由是四鎮貯積及人畜並為蘇祿所掠,安西僅全。蘇祿既聞杜暹入相,稍引退,俄又遣使入朝獻方物。



 十八年,蘇祿使至京師,玄宗御丹鳳樓設宴。突厥先遣使入朝,是日亦來預宴,與蘇祿使爭長。突厥使曰:「突騎施國小,本是突厥之臣,不宜居上。」蘇祿使曰:「今日此宴,乃為我設,不合居下。」於是中書門下及百僚議,遂於東西幕下兩處分坐,突厥使在東,突騎施使在西。宴訖,厚齎而遣之。



 蘇祿性尤清儉,每戰伐,有所克獲,盡
 分與將士及諸部落。其下愛之,甚為其用。潛又遣使南通吐蕃,東附突厥。突厥及吐蕃亦嫁女與蘇祿。既以三國女為可敦,又分立數子為葉護,費用漸廣。先既不為積貯,晚年抄掠所得者,留不分之。又因風病,一手攣縮,其下諸部,心始攜貳。



 有大首領莫賀達干、都摩度兩部落,最為強盛。百姓又分為黃姓、黑姓兩種,互相猜阻。



 二十六年夏,莫賀達干勒兵夜攻蘇祿,殺之。都摩度初與莫賀達干連謀,俄又相背,立蘇祿之子咄火仙為可汗,以
 輯其餘眾,與莫賀達干自相攻擊。莫賀達干遣使告安西都護蓋嘉運。嘉運率兵討之,大敗都摩度之眾,臨陣擒咄火仙,並收得金河公主而還。又欲立史懷道之子昕為可汗以鎮撫之,莫賀達干不肯,曰:「討平蘇祿,本是我之元謀,若立史昕為主,則國家何以酬賞於我?」乃不立史昕,便令莫賀達干統眾。



 二十七年二月,嘉運率將士詣闕獻俘,玄宗御花萼樓以宴之,仍令將吐火仙獻於太廟。俄又黃姓、黑姓自相屠殺,各遣使降附。



 史臣曰:中原多事,外國窺邊,周獫狁、漢匈奴之後,其類實繁,前史論之備矣。突厥自隋文修王道,肅軍容,示恩威以羈縻之;煬帝失政教,生戎心,肇亂離以啟發之。高祖借其力而入平京師,群賊附其強而迭據河朔。高祖同御榻以延其使,太宗幸便橋以約其和。當其時焉,不其盛矣!竟滅其族而身死於國者,何也?咸謂太宗有馭夷狄之道,李勣著戡定之功。殊不知突厥之始也,賞罰明而將士戮力。遇煬帝之亂,亡命蓄怒者既附之,其興
 也宜哉!頡利之衰也,兄弟手勾隙而部族離心。當太宗之理,謀臣猛將討逐之,其亡也宜哉!洎武后亂朝,默啜犯塞,玄宗纂嗣,傳首京師,東封太山,西戎扈蹕,開元之代,繼踵來降。西突厥諸族,遇其理,則眾心悅附而甲兵興焉;遇其亂,則族類怨怒而本根破矣!理亂二道,華夷一途。或質言於盛衰倚伏,未為確論。



 贊曰:中國失政,邊夷幸災。理亂之道,取鑒將來。



\end{pinyinscope}