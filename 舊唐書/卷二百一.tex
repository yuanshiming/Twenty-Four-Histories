\article{卷二百一}

\begin{pinyinscope}

 ○崔善為薛頤甄權弟立
 言宋俠
 許胤宗乙弗弘禮袁天綱孫思邈明崇儼張憬藏李嗣真張文仲李
 虔縱韋慈藏附尚獻甫裴知古附孟詵嚴善思金梁鳳張果葉法善僧玄奘神秀慧能普寂義福附一行泓師附桑道茂



 夫術數占相之法,出於陰陽家流。自劉向演《洪範》之言,京房傳焦贛之法,莫不望氣視祲,懸知災異之來;運策揲蓍,預定吉兇之會。固已詳於魯史,載彼《周官》。其弊者肄業非精,順非行偽,而庸人不修德義,妄冀遭逢。如魏豹之納薄姬,孫皓之邀青蓋,王莽隨式而移坐,劉歆聞
 讖而改名。近者綦連耀之構異端,蘇玄明之犯宮禁,皆因占候,輔此奸兇。聖王禁星緯之書,良有以也。國史載袁天綱前知武後,恐匪格言,而李淳風刪方伎書,備言其要。舊本錄崔善為已下,此深於其術者,兼桑門道士方伎等,並附此篇。



 崔善為,貝州武城人也。祖顒,後魏員外散騎侍郎。父權會,齊丞相府參軍事。善為好學,兼善天文算歷,明達時務。弱冠州舉,授文林郎。屬隋文帝營仁壽宮,善為領
 丁匠五百人。右僕射楊素為總監,巡至善為之所,索簿點人,善為手持簿暗唱之,五百人一無差失,素大驚。自是有四方疑獄,多使善為推按,無不妙盡其理。



 仁壽中,稍遷樓煩郡司戶書佐。高祖時為太守,甚禮遇之。善為以隋政傾頹,乃密勸進,高祖深納之。義旗建,引為大將軍府司戶參軍,封清河縣公。武德中,歷內史舍人、尚書左丞,甚得譽。諸曹令史惡其聰察,因其身短而傴,嘲之曰:「崔子曲知鉤,隨例得封侯。髆上全無項,胸前別有頭。」高
 祖聞之,勞勉之曰:「澆薄之人,醜正惡直。昔齊末奸吏歌斛律明月,而高緯愚暗,遂滅其家。朕雖不德,幸免斯事。」因購流言者,使加其罪。時傅仁均所撰《戊寅元歷》,議者紛然,多有同異,李淳風又駁其短十有八條。高祖令善為考校二家得失,多有駁正。



 貞觀初,拜陜州刺史。時朝廷立議,戶殷之處,得徙寬鄉。善為上表稱:「畿內之地,是謂戶殷,丁壯之人,悉入軍府。若聽移轉,便出關外。此則虛近實遠,非經通之議。」其事乃止。後歷大理、司農二卿,
 名為稱職。坐與少卿不協,出為秦州刺史,卒,贈刑部尚書。



 薛頤,滑州人也。大業中,為道士。解天文律歷,尤曉雜占。煬帝引入內道場,亟令章醮。武德初,追直秦府。頤嘗密謂秦王曰:「德星守秦分,王當有天下,願王自愛。」秦王乃奏授太史丞,累遷太史令。貞觀中,太宗將封禪泰山,有彗星見,頤因言「考諸玄象,恐未可東封」。會褚遂良亦言其事,於是乃止。



 頤後上表請為道士,太宗為置紫府觀
 於九秬山,拜頤中大夫,行紫府觀主事。又敕於觀中建一清臺,候玄象,有災祥薄蝕謫見等事,隨狀聞奏。前後所奏,與京臺李淳風多相符契。後數歲卒。



 甄權,許州扶溝人也。嘗以母病,與弟立言專醫方,得其旨趣。隋開皇初,為秘書省正字,後稱疾免。隋魯州刺史庫狄鏚苦風患,手不得引弓,諸醫莫能療。權謂曰:「但將弓箭向垛,一針可以射矣。」針其肩隅一穴,應時即射。權之療疾,多此類也。



 貞觀十七年,權年一百三歲,太宗幸
 其家,視其飲食,訪以藥性,因授朝散大夫,賜幾杖衣服。其年卒。撰《脈經》、《針方》、《明堂人形圖》各一卷。



 弟立言,武德中累遷太常丞。御史大夫杜淹患風毒發腫,太宗令立言視之。既而奏曰:「從今更十一日午時必死。」果如其言。時有尼明律,年六十餘,患心腹鼓脹,身體嬴瘦,已經二年。立言診脈曰:「其腹內有蟲,當是誤食發為之耳。」因令服雄黃,須臾吐一蛇,如人手小指,唯無眼,燒之,猶有發氣,其疾乃愈。立言尋卒。撰《本草音義》七卷,《古今錄驗方》
 五十卷。



 宋俠者,洺州清漳人,北齊東平王文學孝正之子也。亦以醫術著名。官至朝散大夫、藥藏監。撰《經心錄》十卷,行於代。



 許胤宗,常州義興人也。初事陳,為新蔡王外兵參軍。時柳太后病風不言,名醫治皆不愈,脈益沉而噤。胤宗曰:「口不可下藥,宜以湯氣薰之。令藥入腠理,周理即差。」乃造黃蓍防風湯數十斛,置於床下,氣如煙霧,其夜便得
 語。由是超拜義興太守。陳亡入隋,歷尚藥奉御。武德初,累授散騎侍郎。



 時關中多骨蒸病,得之必死,遞相連染,諸醫無能療者。胤宗每療,無不愈。或謂曰:「公醫術若神,何不著書以貽將來?」胤宗曰:「醫者,意也,在人思慮。又脈候幽微,苦其難別,意之所解,口莫能宣。且古之名手,唯是別脈;脈既精別,然後識病。夫病之於藥,有正相當者,唯須單用一味,直攻彼病,藥力既純,病即立愈。今人不能別脈,莫識病源,以情臆度,多安藥味。譬之於獵,未知
 兔所,多發人馬,空地遮圍,或冀一人偶然逢也。如此療疾,不亦疏乎!假令一藥偶然當病,復共他味相和,君臣相制,氣勢不行,所以難差,諒由於此。脈之深趣,既不可言,虛設經方,豈加於舊。吾思之久矣,故不能著述耳!」年九十餘卒。



 乙弗弘禮,貝州高唐人也。隋煬帝居籓,召令相己。弘禮跪而賀曰:「大王骨法非常,必為萬乘之主,誠願戒之在得。」煬帝即位,召天下道術人,置坊以居之,仍令弘禮統
 攝。帝見海內漸亂,玄象錯謬,內懷憂恐,嘗謂弘禮曰:「卿昔相朕,其言已驗。且占相道術,朕頗自知。卿更相朕,終當何如?」弘禮逡巡不敢答。帝迫曰:「卿言與朕術不同,罪當死。」弘禮曰:「臣本觀相書,凡人之相,有類於陛下者,不得善終。臣聞聖人不相,故知凡聖不同耳。」自是帝嘗遣使監之,不得與人交言。



 初,泗州刺史薛大鼎隋時嘗坐事沒為奴,貞觀初,與數人詣之,大鼎次至,弘禮曰:「君奴也,欲何所相?」咸曰:「何以知之?」弘禮曰:「觀其頭目,直是賤
 人,但不知餘處何如耳?」大鼎有慚色,乃解衣視之,弘禮曰:「看君面,不異前言。占君自腰已下,當為方岳之任。」其占相皆此類也。貞觀末卒。



 袁天綱,益州成都人也。尤工相術。隋大業中,為資官令。武德初,蜀道使詹俊赤牒授火井令。初,天綱以大業元年至洛陽。時杜淹、王珪、韋挺就之相。天綱謂淹曰:「公蘭臺成就,學堂寬博,必得親糾察之官,以文藻見知。」謂王曰:「公三亭成就,天地相臨,從今十年已外,必得五品要
 職。」謂韋曰:「公面似大獸之面,交友極誠,必得士友攜接,初為武職。」復謂淹等「二十年外,終恐三賢同被責黜,暫去即還。」淹尋遷侍御史,武德中為天策府兵曹、文學館學士。王珪為太子中允。韋挺,隋末與隱太子友善,後太子引以為率。至武德六年,俱配流巂州。淹等至益州,見天綱曰:「袁公洛邑之言,則信矣。未知今日之後何如?」天綱曰:「公等骨法,大勝往時,終當俱受榮貴。」至九年,被召入京,共造天綱。天綱謂杜公曰:「即當得三品要職,年壽
 非天綱所知。王、韋二公,在後當得三品官,兼有年壽,然晚途皆不稱愜,韋公尤甚。」淹至京,拜御史大夫、檢校吏部尚書。王珪尋授侍中,出為同州刺史。韋挺歷御史大夫、太常卿,貶象州刺史。皆如天綱之言。



 大業末,竇軌客游德陽,嘗問天綱。天綱謂曰:「君額上伏犀貫玉枕,輔角又成、必於梁、益州大樹功業。」武德初,軌為益州行臺僕射,引天綱,深禮之。天綱又謂軌曰:「骨法成就,不異往時之言。然目氣赤脈貫瞳子,語則赤氣浮面。如為將軍,恐多
 殺人。願深自誡慎。」武德九年,軌坐事被徵,將赴京,謂天綱曰:「更得何官?」曰:「面上家人坐仍未見動,輔角右畔光澤,更有喜色,至京必承恩,還來此任。」其年果重授益州都督。



 則天初在襁褓,天綱來至第中,謂其母曰:「唯夫人骨法,必生貴子。」乃召諸子,令天綱相之。見元慶、元爽曰:「此二子皆保家之主,官可至三品。」見韓國夫人曰:「此女亦大貴,然不利其夫。」乳母時抱則天,衣男子之服,天綱曰:「此郎君子神色爽徹,不可易知,試令行看。」於是步
 於床前,仍令舉目,天綱大驚曰:「此郎君子龍睛鳳頸,貴人之極也。」更轉側視之,又驚曰:「必若是女,實不可窺測,後當為天下之主矣!」



 貞觀八年,太宗聞其名,召至九成宮。時中書舍人岑文本令視之。天綱曰:「舍人學堂成就,眉覆過目,文才振於海內,頭又生骨,猶未大成,若得三品,恐是損壽之徵。」文本官至中書令,尋卒。其年,侍御史張行成、馬周同問天綱,天綱曰:「馬侍御伏犀貫腦,兼有玉枕,又背如負物,當富貴不可言。近古已來,君臣道合,
 罕有如公者。公面色赤,命門色暗,耳後骨不起,耳無根,只恐非壽者。」周後位至中書令、兼吏部尚書,年四十八卒。謂行成曰:「公五岳四瀆成就,下亭豐滿,得官雖晚,終居宰輔之地。」行成後至尚書右僕射。天綱相人所中,皆此類也。申國公高士廉嘗謂曰:「君更作何官?」天綱曰:「自知相命,今年四月盡矣。」果至是月而卒。



 孫思邈,京兆華原人也。七歲就學,日誦千餘言。弱冠,善談莊、老及百家之說,兼好釋典。洛州總管獨孤信見而
 嘆曰:「此聖童也。但恨其器大,難為用也。」周宣帝時,思邈以王室多故,隱居太白山。隋文帝輔政,徵為國子博士,稱疾不起。嘗謂所親曰:「過五十年,當有聖人出,吾方助之以濟人。」及太宗即位,召詣京師,嗟其容色甚少,謂曰:「故知有道者誠可尊重,羨門、廣成,豈虛言哉!」將授以爵位,固辭不受。顯慶四年,高宗召見,拜諫議大夫,又固辭不受。



 上元元年,辭疾請歸,特賜良馬,及鄱陽公主邑司以居焉。當時知名之士宋令文、孟詵、盧照鄰等,執師
 資之禮以事焉。思邈嘗從幸九成宮,照鄰留在其宅。時庭前有病梨樹,照鄰為之賦,其序曰:「癸酉之歲,餘臥疾長安光德坊之官舍。父老云:『是鄱陽公主邑司。昔公主未嫁而卒,故其邑廢。』時有孫思邈處士居之。邈道合古今,學殫數術。高談正一,則古之蒙莊子;深入不二,則今之維摩詰。其推步甲乙,度量乾坤,則洛下閎、安期先生之儔也。」照鄰有惡疾,醫所不能愈,乃問思邈:「名醫愈疾,其道何如?」思邈曰:



 吾聞善言天者,必質之於人,善言人者,
 亦本之於天。天有四時五行,寒暑迭代,其轉運也,和而為雨,怒而為風,凝而為霜雪,張而為虹蜺,此天地之常數也。人有四支五藏,一覺一寢,呼吸吐納,精氣往來,流而為榮衛,彰而為氣色,發而為音聲,此人之常數也。陽用其形,陰用其精,天人之所同也。及其失也,蒸則生熱,否則生寒,結而為瘤贅,陷而為癰疽,奔而為喘乏,竭而為焦枯,診發乎面,變動乎形。推此以及天地亦如之。故五緯盈縮,星辰錯行,日月薄蝕,孛彗飛流,此天地之危
 診也。寒暑不時,天地之蒸否也;石立土踴,天地之瘤贅也;山崩土陷,天地之癰疽也;奔風暴雨,天地之喘乏也;川瀆竭涸,天地之焦枯也,良醫導之以藥石,救之以針劑,聖人和之以至德,輔之以人事,故形體有可愈之疾,天地有可消之災。



 又曰:



 膽欲大而心欲小,智欲圓而行欲方。《詩》曰:「如臨深淵,如履薄冰」,謂小心也;「糾糾武夫,公侯干城」,謂大膽也。「不為利回,不為義疚」,行之方也;「見機而作,不俟終日」,智之圓也。



 思邈自雲開皇辛酉歲生,至
 今年九十三矣;詢之鄉里,咸云數百歲人。話周、齊間事,歷歷如眼見。以此參之,不啻百歲人矣。然猶視聽不衰,神採甚茂,可謂古之聰明博達不死者也。



 初,魏徵等受詔修齊、梁、陳、周、隋五代史,恐有遺漏,屢訪之,思邈口以傳授,有如目睹。東臺侍郎孫處約將其五子侹、儆、俊、佑、佺以謁思邈,思邈曰:「俊當先貴;佑當晚達;佺最名重,禍在執兵。」後皆如其言。太子詹事盧齊卿童幼時,請問人倫之事,思邈曰:「汝後五十年位登方伯,吾孫當為屬吏,
 可自保也。」後齊卿為徐州刺史,思邈孫溥果為徐州蕭縣丞。思邈初謂齊卿之時,溥猶未生,而預知其事。凡諸異跡,多此類也。



 永淳元年卒。遺令薄葬,不藏冥器,祭祀無牲牢。經月餘,顏貌不改,舉尸就木,猶若空衣,時人異之。自注《老子》、《莊子》,撰《千金方》三十卷,行於代。又撰《福祿論》三卷,《攝生真錄》及《枕中素書》、《會三教論》各一卷。



 子行,天授中為鳳閣侍郎。



 明崇儼,洛州偃師人。其先平原士族,世仕江左。父恪,豫
 州刺史。崇儼年少時,隨父任安喜令,父之小吏有善役召鬼神者,崇儼盡能傳其術。乾封初,應封嶽舉,授黃安丞。會刺史有女病篤,崇儼致他方殊物以療之,其疾乃愈。高宗聞其名,召與語。悅之,擢授冀王府文學。儀鳳二年,累遷正諫大夫,特令入閣供奉。崇儼每因謁見,輒假以神道,頗陳時政得失,帝深加允納。潤州棲霞寺,是其五代祖梁處士山賓故宅,帝特為制碑文,親書於石,論者榮之。



 四年,為盜所殺。時語以為崇儼密與天後為厭
 勝之法,又私奏章懷太子不堪承繼大位,太子密知之,潛使人害之。優制贈侍中,謚曰莊,仍拜其子珪為秘書郎。



 珪,開元中仕至懷州刺史。



 張憬藏,許州長社人。少工相術,與袁天綱齊名。太子詹事蔣儼年少時,嘗遇憬藏,因問祿命。憬藏曰:「公從今二年,當得東宮掌兵之官,秩未終而免職。免職之後,厄在三尺土下。又經六年,據此合是死徵。然後當享富貴,名位俱盛,即又不合中,年至六十一,為蒲州刺史,十月三
 十日午時祿絕。」儼後皆如其言。嘗奉使高麗,被莫離支囚於地窖中,經六年,然後得歸。及在蒲州,年六十一矣,至期,召人吏妻子與之告別,自云當死。俄而有敕,許令致仕。



 左僕射劉仁軌微時,嘗與鄉人靖思賢各齎絹贈憬藏以問官祿。憬藏謂仁軌曰:「公居五品要官,雖暫解黜,終當位極人臣。」仁軌後自給事中坐事,令白衣向海東效力。固辭思賢之贈,曰:「公當孤獨客死。」及仁軌為僕射,思賢尚存,謂人曰:「張憬藏相劉僕射,則妙矣。吾今已
 有三子,田宅自如,豈其言亦有不中也?」俄而三子相繼而死,盡貨田宅,寄死於所親園內。憬藏相人之妙,皆此類。竟不仕,以壽終。



 李嗣真,滑州匡城人也。父彥琮,趙州長史。嗣真博學曉音律,兼善陰陽推算之術。弱冠明經舉,補許州司功。時左侍極賀蘭敏之受詔於東臺修撰,奏嗣真弘文館參預其事。嗣真與同時學士劉獻臣、徐昭俱稱少俊,館中號為「三少」。敏之既恃寵驕盈,嗣真知其必敗,謂所親曰:「
 此非庇身之所也。」因咸亨年京中大饑,乃求出,補義烏令。無何,敏之敗,修撰官皆連坐流放,嗣真獨不預焉。調露中,為始平令,風化大行。時章懷太子居春宮,嗣真嘗於太清觀奏樂,謂道士劉概、輔儼曰:「此曲何哀思不和之甚也?」概、儼曰:「此太子所作《寶慶樂》也。」居數日,太子廢為庶人。概等以其事聞奏,高宗大奇之,徵拜司禮丞,仍掌五禮儀注,加中散大夫,封常山子。



 永昌中,拜右御史中丞,知大夫事。時酷吏來俊臣構陷無罪,嗣真上書諫
 曰:「臣聞陳平事漢祖,謀疏楚君臣,乃用黃金五萬斤,行反間之術。項王果疑臣下,陳平反間果行。今告事紛紜,虛多實少,焉知必無陳平先謀疏陛下君臣,後謀除國家良善,臣恐為社稷之禍!伏乞陛下特回天慮,察臣狂瞽,然後退就鼎鑊,實無所恨!」疏奏不納。尋被俊臣所陷,配流嶺南。



 萬歲通天年,徵還,至桂陽,自筮死日,預託桂陽官屬備兇器。依期暴卒。則天深加憫惜,敕州縣遞靈輿還鄉,贈濟州刺史。神龍初,又贈御史大夫。



 撰《明堂新
 禮》十卷,《孝經指要》、《詩品》、《書品》、《畫品》各一卷。



 張文仲,洛州洛陽人也。少與鄉人李虔縱、京兆人韋慈藏並以醫術知名。文仲,則天初為侍御醫。時特進蘇良嗣於殿庭因拜跪便絕倒,則天令文仲、慈藏隨至宅候之。文仲曰:「此因憂憤邪氣激也。若痛沖脅,則劇難救。」自朝候之。未及食時,即苦沖脅絞痛。文仲曰:「若入心,即不可療。」俄頃心痛,不復下藥,日旰而卒。文仲尤善療風疾。其後則天令文仲集當時名醫共撰療風氣諸方,仍令
 麟臺監王方慶監其修撰。文仲奏曰:「風有一百二十四種,氣有八十種。大抵醫藥雖同,人性各異,庸醫不達藥之性使冬夏失節,因此殺人。唯腳氣頭風上氣,常須服藥不絕。自餘則隨其發動,臨時消息之。但有風氣之人,春末夏初及秋暮,要得通洩,即不困劇。」於是撰四時常服及輕重大小諸方十八首表上之。文仲久視年終於尚藥奉御。撰《隨身備急方》三卷,行於代。



 虔縱,官至侍御醫。慈藏,景龍中光祿卿。自則天、中宗已後,諸醫咸推文
 仲等三人為首。



 尚獻甫,衛州汲人也。尤善天文。初,出家為道士。則天時召見,起家拜太史令,固辭曰:「臣久從放誕,不能屈事官長。」則天乃改太史局為渾儀監,不隸秘書省,以獻甫為渾儀監。數顧問災異,事皆符驗。又令獻甫於上陽宮集學者撰《方域圖》。長安二年,獻甫奏曰:「臣本命納音在金,今熒惑犯五諸候、太史之位。熒,火也,能克金,是臣將死之徵。」則天曰:「朕為卿禳之。」遽轉獻甫為水衡都尉,謂曰:「
 水能生金,今又去太史之位,卿無憂矣。」其秋,獻甫卒,則天甚嗟異惜之。復以渾儀監為太史局,依舊隸秘書監。



 時又有雍州人裴知古,善於音律。長安中為太樂丞。神龍元年正月,春享西京太廟,知古預其事。謂萬年令元行沖曰:「金石諧和,當有吉慶之事,其在唐室子孫乎?」其月,中宗即位,復改國為唐。知古又能聽婚夕環佩之聲,知其夫妻終始。後卒於太樂令。



 孟詵,汝州梁人也。舉進士。垂拱初,累遷鳳閣舍人。詵少
 好方術,嘗於鳳閣侍郎劉禕之家,見其敕賜金,謂禕之曰:「此藥金也。若燒火其上,當有五色氣。」試之果然。則天聞而不悅,因事出為臺州司馬。後累遷春官侍郎。



 睿宗在籓,召充侍讀。長安中,為同州刺史,加銀青光祿大夫。神龍初致仕,歸伊陽之山第,以藥餌為事。詵年雖晚暮,志力如壯,嘗謂所親曰:「若能保身養性者,常須善言莫離口,良藥莫離手。」睿宗即位,召赴京師,將加任用,固辭衰老。景雲二年,優詔賜物一百段,又令每歲春秋二時,
 特給羊酒糜粥。開元初,河南尹畢構以詵有古人之風,改其所居為子平里。尋卒,年九十三。



 詵所居官,好勾剝為政,雖繁而理。撰《家》、《祭禮》各一卷,《喪服要》二卷,《補養方》、《必效方》各三卷。



 嚴善思,同州朝邑人也。少以學涉知名,尤善天文歷數及卜相之術。初應消聲幽藪科舉擢第。則天時為監察御史,權右拾遺、內供奉。數上表陳時政得失,多見納用。稍遷太史令。



 聖歷二年,熒惑入輿鬼,則天以問善思。善
 思對曰:「商姓大臣當之。」其年,文昌左相王及善卒。長安中,熒惑入月,鎮星犯天關。善思奏曰:「法有亂臣伏罪,且有臣下謀上之象。」歲餘,張柬之、敬暉等起兵誅張易之、昌宗。其占驗皆此類也。



 神龍初,遷給事中。則天崩,將合葬乾陵,善思奏議曰:



 謹按《天元房錄葬法》云:「尊者先葬,卑者不合於後開入。」則天太后,卑於天皇大帝,今欲開乾陵合葬,即是以卑動尊。事既不經,恐非安隱。臣又聞乾陵玄闕,其門以石閉塞,其石縫隙,鑄鐵以固其中。今
 若開陵,必須鐫鑿。然以神明之道,體尚幽玄;今乃動眾加功,誠恐多所驚黷。又若別開門道,以入玄宮,即往者葬時,神位先定,今更改作,為害益深。又以修築乾陵之後,國頻有難,遂至則天太后權總萬機,二十餘年,其難始定。今乃更加營作,伏恐還有難生。



 但合葬非古,著在禮經,緣情為用,無足依準。況今事有不安,豈可復循斯制!伏見漢時諸陵,皇后多不合葬;魏、晉已降,始有合者。然以兩漢積年,向餘四百;魏、晉之後,祚皆不長。雖受命
 應期,有因天假,然以循機享德,亦在天時。但陵墓所安,必資勝地,後之胤嗣,用托靈根,或有不安,後嗣亦難長享。伏望依漢朝之故事,改魏、晉之頹綱,於乾陵之傍,更擇吉地,取生墓之法,別起一陵,既得從葬之儀,又成固本之業。



 臣伏以合葬者,人緣私情;不合者,前修故事。若以神道有知,幽途自得通會;若以死者無知,合之復有何益!然以山川精氣,土為星象,若葬得其所,則神安後昌;若葬失其宜,則神危後損。所以先哲垂範,具之葬經,
 欲使生人之道必安,死者之神必泰。伏望少回天眷,俯覽臣言,行古昔之明規,割私情之愛欲,使社稷長享,天下乂安。凡在懷生,孰不慶幸!



 疏奏不納。



 景龍中,遷禮部侍郎,出為汝州刺史。睿宗在籓,善思嘗謂姚元之曰:「相王必登帝位。」及踐祚,元之以事聞奏,由是召拜右散騎常侍。



 唐隆元年,鄭愔謀冊譙王重福為帝,乃草偽制,除善思為禮部尚書,知吏部選事。及譙王下獄,景雲元年,大理寺奏:「善思與逆人重福通謀,合從極法。」給事中韓
 思復奏曰:「議獄緩死,列聖明規;刑疑惟輕,有國恆典。嚴善思往在先朝,屬韋氏擅內,恃寵宮掖,謀危社稷。善思此時,乃能先覺,因詣相府,有所發明,進論聖躬,必登宸極。雖交游重福,謀陷韋氏,敕追善思,書至便發,向懷逆節,寧即奔命?一面疏綱,誠合順生;三驅取禽,來而有宥。唯刑是恤,理合昭詳。請付刑部集群官議定奏裁,以符慎獄。」時議者多云:「善思合從原宥。」有司仍執前議請誅之。思復又駁奏懇直。睿宗納其奏,竟免善思死,配流靜
 州。無幾,遇赦還。年八十五,開元十七年卒。



 初,善思為御史時,中書舍人劉允濟為酷吏所陷,當死。善思愍其老,密表奏請,允濟乃得免誅。善思後見允濟,竟不自言其事。韓思復奏免善思之罪,亦未曾有所言謝。時人稱其長者。



 善思子向,乾元中為鳳翔尹,寶應中授太常員外卿。始善思父徐州長史延及善思,俱年八十五而卒;廣德二年,向卒,又年八十五。向兄前趙郡司馬宙,長向十歲,向卒時,宙並無恙。



 金梁鳳,不知何許人也。天寶十三載,客於河西。善相人,又言玄象。時哥舒翰為節度使,詔入京師。裴冕為祠部郎中,知河西留後,在武威。梁鳳謂冕曰:「玄象有變,半年間有兵起,郎中此時當得中丞,不拜中丞,即得宰相,不離天子左右,大富貴。」冕曰:「公乃狂言,冕何至此?」梁鳳曰:「有一日向東京,一日入蜀川,一日來向朔方,此時公得相。」冕懼其言,深謝絕之。其後安祿山反,南犯洛陽,僭稱偽位。哥舒翰東守潼關,累月,奏冕為御史中丞,追赴京。
 冕又詰曰:「事驗也。」冕又問三日之兆,梁鳳曰:「東京日即自磨滅,蜀川日亦不能久,此間日何轉分明,不可說。」冕志之。既潼關失守,玄宗幸蜀,肅宗北如靈武,冕會之,勸成策立,改元為至德元年。冕果為中書侍郎、平章事。冕奏之,肅宗召拜都水使者。



 梁鳳在河隴,謂呂諲曰:「判官骨相,合得宰相。須得一大驚怖,即得。」諲後至驛,責讓驛長,搒之。驛吏武將,性粗猛,持弓矢突入,射諲,矢兩發,幾中諲面,諲逾墻得免。以報梁鳳,梁鳳曰:「此必入相。」逾年,
 諲自黃門侍郎知政事。



 梁鳳在鳳翔,李揆、盧允二人同見之,俱素服,自稱選人。梁鳳謂之曰:「公等並至清望官,那得雲無官。」揆、允以實對。梁鳳遣二人行,謂揆曰:「公從舍人即入相,一年內事。」謂允曰:「公好即是吏部郎中。」及克復兩京,揆自中書舍人知禮部侍郎事,入為中書侍郎、平章事,乃以允為吏部郎中。其驗多此類。爾後佯聾以自晦。冕為右僕射、兼御史大夫、成都尹、劍南節度使,有進止,令將梁鳳行。後乃病卒。



 張果者,不知何許人也。則天時,隱於中條山,往來汾、晉間,時人傳其有長年秘術,自云年數百歲矣。嘗著《陰符經玄解》,盡其玄理。則天遣使召之,果佯死不赴。後人復見之,往來恆州山中。開元二十一年,恆州刺史韋濟以狀奏聞。玄宗令通事舍人裴晤往迎之。果對使絕氣如死,良久漸蘇。晤不敢逼,馳還奏狀。又遣中書舍人徐嶠齎璽書以邀迎之。果乃隨嶠至東都,肩輿入東宮中。



 玄宗初即位,親訪理道及神仙方藥之事,及聞變化不測
 而疑之。有邢和璞者,善算人而知夭壽善惡。玄宗令算果,則懵然莫知其甲子。又有師夜光者,善視鬼。玄宗召果與之密坐,令夜光視之。夜光進曰:「果今安在?」夜光對面終莫能見。玄宗謂力士曰:「吾聞飲堇汁無苦者,真奇士也。」會天寒,使以堇汁飲果。果乃引飲三卮,醺然如醉所作,顧曰:「非佳酒也。」乃寢。頃之,取鏡視齒,則盡燋且黧。命左右取鐵如意擊齒墜,藏於帶。乃懷中出神仙藥,微紅,傅墜齒之齗。復寐良久,齒皆出矣,粲然潔白,玄宗方
 信之。



 玄宗好神仙,而欲果尚公主。果固未知之,謂秘書少監王迥質、太常少卿蕭華曰:「諺云娶婦得公主,真可畏也。」迥質與華相顧,未曉其言。即有中使至,宣曰:「玉真公主早歲好道,欲降先生。」果大笑,竟不奉詔。迥質等方悟向來之言。



 後懇辭歸山,因下制曰:「恆州張果先生,游方外者也。跡先高尚,深入窈冥。是渾光塵,應召城闕。莫詳甲子之數,且謂羲皇上人。問以道樞,盡會宗極。今特行朝禮,爰畀寵命。可銀青光祿大夫,號曰通玄先生。」其
 年請入恆山,錫以衣服及雜彩等,便放歸山。乃入恆山,不知所之。玄宗為造棲霞觀於隱所,在蒲吾縣,後改為平山縣。



 道士葉法善,括州括蒼縣人。自曾祖三代為道士,皆有攝養占卜之術。法善少傳符籙,尤能厭劾鬼神。顯慶中,高宗聞其名,徵詣京師,將加爵位,固辭不受。求為道士,因留在內道場,供待甚厚。時高宗令廣徵諸方道術之士,合煉黃白。法善上言:「金丹難就,徒費財物,有虧政理,
 請核其真偽。」帝然其言,因令法善試之,由是乃出九十餘人,因一切罷之。法善又嘗於東都凌空觀設壇醮祭,城中士女競往觀之。俄頃數十人自投火中,觀者大驚,救之而免。法善曰:「此皆魅病,為吾法所攝耳。」問之果然。法善悉為禁劾,其病乃愈。



 法善自高宗、則天、中宗歷五十年,常往來名山,數召入禁中,盡禮問道。然排擠佛法,議者或譏其向背。以其術高,終莫之測。



 睿宗即位,稱法善有冥助之力。先天二年,拜鴻臚卿,封越國公,仍依舊
 為道士,止於京師之景龍觀,又贈其父為歙州刺史。當時尊寵,莫與為比。



 法善生於隋大業之丙子,死於開元之庚子,凡一百七歲。八年卒。詔曰:



 故道士鴻臚卿、員外置、越國公葉法善,天真精密,妙理玄暢,包括秘要,發揮靈符,固以冥默難源,希夷罕測。而情棲蓬閬,跡混朝伍,保黃冠而不杖,加紫綬而非榮,卓爾孤秀,冷然獨往。勝氣絕俗,貞風無塵,金骨外聳,珠光內應。斯乃體應中仙,名升上德。朕當聽政之暇,屢詢至道;公以理國之法,數
 奏昌言。謀參隱諷,事宣弘益。嘆徽音之未泯,悲形解之俄留,曾莫憖遺,殲良奄及。永惟平昔,感愴於懷,宜申禮命,式旌泉壤。可贈越州都督。



 僧玄奘,姓陳氏,洛州偃師人。大業末出家,博涉經論。嘗謂翻譯者多有訛謬,故就西域,廣求異本以參驗之。貞觀初,隨商人往游西域。玄奘既辯博出群,所在必為講釋論難,蕃人遠近咸尊伏之。在西域十七年,經百餘國,悉解其國之語,仍採其山川謠俗,土地所有,撰《西域記》
 十二卷。貞觀十九年,歸至京師。太宗見之,大悅,與之談論。於是詔將梵本六百五十七部於弘福寺翻譯,仍敕右僕射房玄齡、太子左庶子許敬宗,廣召碩學沙門五十餘人,相助整比。



 高宗在宮,為文德太后追福,造慈恩寺及翻經院,內出大幡,敕《九部樂》及京城諸寺幡蓋眾伎,送玄奘及所翻經像、諸高僧等入住慈恩寺。顯慶元年,高宗又令左僕射於志寧、侍中許敬宗、中書令來濟、李義府、杜正倫、黃門侍郎薛元超等,共潤色玄奘所定
 之經,國子博士範義碩、太子洗馬郭瑜、弘文館學士高若思等,助加翻譯。凡成七十五部。奏上之。後以京城人眾競來禮謁,玄奘乃奏請逐靜翻譯,敕乃移於宜君山故玉華宮。六年卒,時年五十六,歸葬於白鹿原,士女送葬者數萬人。



 僧神秀,姓李氏,汴州尉氏人。少遍覽經史,隋末出家為僧。後遇蘄州雙峰山東山寺僧弘忍,以坐禪為業,乃嘆伏曰:「此真吾師也。」便往事弘忍,專以樵汲自役,以求其
 道。



 昔後魏末,有僧達摩者,本天竺王子,以護國出家,入南海,得禪宗妙法,雲自釋迦相傳,有衣缽為記,世相付授。達摩齎衣缽航海而來,至梁,詣武帝。帝問以有為之事,達摩不說。乃之魏,隱於嵩山少林寺,遇毒而卒。其年,魏使宋云於蔥嶺回,見之,門徒發其墓,但有衣履而已。達摩傳慧可,慧可嘗斷其左臂,以求其法,慧可傳璨,璨傳道信,道信傳弘忍。



 弘忍姓周氏,黃梅人。初,弘忍與道信並住東山寺,故謂其法為東山法門。神秀既師事弘
 忍,弘忍深器異之,謂曰:「吾度人多矣,至於懸解圓照,無先汝者。」



 弘忍以咸亨五年卒,神秀乃往荊州,居於當陽山。則天聞其名,追赴都,肩輿上殿,親加跪禮,敕當陽山置度門寺以旌其德。時王公已下及京都士庶,聞風爭來謁見,望塵拜伏,日以萬數。中宗即位,尤加敬異。中書舍人張說嘗問道,執弟子之禮,退謂人曰:「禪師身長八尺,龐眉秀耳,威德巍巍,王霸之器也。」



 初,神秀同學僧慧能者,新州人也。與神秀行業相埒。弘忍卒後,慧能住韶
 州廣果寺。韶州山中,舊多虎豹,一朝盡去,遠近驚嘆,咸歸伏焉。神秀嘗奏則天,請追慧能赴都,慧能固辭。神秀又自作書重邀之,慧能謂使者曰:「吾形貌短陋,北土見之,恐不敬吾法。又先師以吾南中有緣,亦不可違也。」竟不度嶺而死。天下乃散傳其道,謂神秀為北宗,慧能為南宗。



 神秀以神龍二年卒,士庶皆來送葬。有詔賜謚曰「大通禪師」。又於相王舊宅置報恩寺,岐王範、張說及徵士盧鴻一皆為其碑文。



 神秀卒後,弟子普寂、義福,並為
 時人所重。



 普寂姓馮氏,蒲州河東人也。年少時遍尋高僧,以學經律。時神秀在荊州玉泉寺,普寂乃往師事,凡六年,神秀奇之,盡以其道授焉。久視中,則天召神秀至東都,神秀因薦普寂,乃度為僧。及神秀卒,天下好釋氏者咸師事之。中宗聞其高年,特下制令普寂代神秀統其法眾。



 開元十三年,敕普寂於都城居止。時王公士庶,競來禮謁。普寂嚴重少言,來者難見其和悅之容,遠近尤以此重之。二十七年,終於都城興唐寺,年八十九。時
 都城士庶曾謁者,皆制弟子之服。有制賜號為「大照禪師」。及葬,河南尹裴寬及其妻子,並衰麻列於門徒之次,士庶傾城哭送,閭里為之空焉。



 義福姓姜氏,潞州銅鞮人。初止藍田化感寺,處方丈之室,凡二十餘年,未嘗出宇之外。後隸京城慈恩寺。開元十一年,從駕往東都,途經蒲、虢二州,刺史及官吏士女,皆齎幡花迎之,所在途路充塞。以二十年卒,有制賜號「大智禪師」。葬於伊闕之北,送葬者數萬人。中書侍郎嚴挺之為制碑文。



 神秀,禪
 門之傑,雖有禪行,得帝王重之,而未嘗聚徒開堂傳法。至弟子普寂,始於都城傳教,二十餘年,人皆仰之。



 僧一行,姓張氏,先名遂,魏州昌樂人,襄州都督、郯國公公謹之孫也。父擅,武功令。



 一行少聰敏,博覽經史,尤精歷象、陰陽、五行之學。時道士尹崇博學先達,素多墳籍。一行詣崇,借揚雄《太玄經》,將歸讀之。數日,復詣崇,還其書。崇曰:「此書意指稍深,吾尋之積年,尚不能曉,吾子試更研求,何遽見還也?」一行曰:「究其義矣。」因出所撰《大衍
 玄圖》及《義決》一卷以示崇。崇大驚,因與一行談其奧賾,甚嗟伏之。謂人曰:「此後生顏子也。」一行由是大知名。武三思慕其學行,就請與結交,一行逃匿以避之。尋出家為僧,隱於嵩山,師事沙門普寂。睿宗即位,敕東都留守韋安石以禮徵。一行固辭以疾,不應命。後步往荊州當陽山,依沙門悟真以習梵律。



 開元五年,玄宗令其族叔禮部郎中洽齎敕書就荊州強起之。一行至京,置於光太殿,數就之,訪以安國撫人之道,言皆切直,無有所隱。
 開元十年,永穆公主出降,敕有司優厚發遣,依太平公主故事。一行以為高宗末年,唯有一女,所以特加其禮。又太平驕僭,竟以得罪,不應引以為例。上納其言,遽追敕不行,但依常禮。其諫諍皆此類也。



 一行尤明著述,撰《大衍論》三卷,《攝調伏藏》十卷,《天一太一經》及《太一局遁甲經》、《釋氏系錄》各一卷。時《麟德歷經》推步漸疏,敕一行考前代諸家歷法,改撰新歷,又令率府長史梁令瓚等與工人創造黃道游儀,以考七曜行度,互相證明。於是
 一行推《周易》大衍之數,立衍以應之,改撰《開元大衍歷經》。至十五年卒,年四十五,賜謚曰「大慧禪師」。



 初,一行從祖東臺舍人太素,撰《後魏書》一百卷,其《天文志》未成,一行續而成之。上為一行制碑文,親書於石,出內庫錢五十萬,為起塔於銅人之原。明年,幸溫湯,過其塔前,又駐騎徘徊,令品官就塔以告其出豫之意;更賜絹五十匹,以蒔塔前松柏焉。



 初,一行求訪師資,以窮大衍,至天臺山國清寺,見一院,古松十數,門有流水。一行立於門屏
 間,聞院僧於庭布算聲,而謂其徒曰:「今日當有弟子自遠求吾算法,已合到門,豈無人導達也?」即除一算。又謂曰:「門前水當卻西流,弟子亦至。」一行承其言而趨入,稽首請法,盡受其術焉,而門前水果卻西流。道士邢和璞嘗謂尹愔曰:「一行其聖人乎?漢之洛下閎造歷,云:『後八百歲當差一日,必有聖人正之。』今年期畢矣,而一行造《大衍》正其差謬,則洛下閎之言,信矣!非聖人而何?



 時又有黃州僧泓者,善葬法。每行視山原,即為之圖,張說深
 信重之。



 桑道茂者,大歷中游京師,善太一遁甲五行災異之說,言事無不中。代宗召之禁中,待詔翰林。建中初,神策軍修奉天城,道茂請高其垣墻,大為制度,德宗不之省。及硃泚之亂,帝蒼卒出幸,至奉天,方思道茂之言。時道茂已卒,命祭之。



 贊曰:術數之精,事必前知。粲如垂象,變告無疑。怪誕之夫,誣罔蓍龜。致彼庸妄,幸時艱危。



\end{pinyinscope}