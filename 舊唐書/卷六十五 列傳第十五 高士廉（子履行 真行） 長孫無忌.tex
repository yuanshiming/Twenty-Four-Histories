\article{卷六十五 列傳第十五 高士廉(子履行 真行) 長孫無忌}

\begin{pinyinscope}

 ○高士廉子履行真行長孫無忌



 高儉,字士廉,渤海蓚人。曾祖飛雀,後魏贈太尉。祖岳,北齊侍中、左僕射、太尉、清河王。父勵,字敬德,北齊樂安王、尚書左僕射、隋洮州刺史。士廉少有器局,頗涉文史。隋
 司隸大夫薛道衡、起居舍人崔祖浚並稱先達,與士廉結忘年之好,由是公卿藉甚。大業中,為治禮郎。士廉妹先適隋右驍衛將軍長孫晟,生子無忌及女。晟卒,士廉迎妹及甥于家,恩情甚重。見太宗潛龍時非常人,因以晟女妻焉,即文德皇后也。隋軍伐遼,時兵部尚書斛斯政亡奔高麗,士廉坐與交游,謫為硃鳶主簿。事父母以孝聞,嶺南瘴癘,不可同行,留妻鮮于氏侍養,供給不足。又念妹無所庇,乃賣大宅,買小宅以處之,分其餘資,輕
 裝而去。尋屬天下大亂,王命阻絕,交趾太守丘和署為司法書佐。士廉久在南方,不知母問,北顧彌切。嘗晝寢,夢其母與之言,宛如膝下,既覺而涕泗橫集。明日果得母訊,議者以為孝感之應。時欽州寧長真率眾攻和,和欲出門迎之,士廉進說曰:「長真兵勢雖多,懸軍遠至,內離外蹙,不能持久。且城中勝兵,足以當之,奈何而欲受人所制?」和從之,因命士廉為行軍司馬,水陸俱進,逆擊破之,長真僅以身免,餘眾盡降。及蕭銑敗,高祖使徇嶺
 南。武德五年,士廉與和上表歸國,累遷雍州治中。時太宗為雍州牧,以士廉是文德皇后之舅,素有才望,甚親敬之。及將誅隱太子,士廉與其甥長孫無忌並預密謀。六月四日,士廉率吏卒釋系囚,授以兵甲,馳至芳林門,備與太宗合勢。太宗升春宮,拜太子右庶子。



 貞觀元年,擢拜侍中,封義興郡公,賜實封九百戶。士廉明辯,善容止,凡有獻納,搢紳之士莫不屬目。時黃門侍郎王珪有密表附士廉以聞,士廉寢而不言,坐是出為安州都督,
 轉益州大都督府長史。蜀土俗薄,畏鬼而惡疾,父母病有危殆者,多不親扶侍,杖頭掛食,遙以哺之。士廉隨方訓誘,風俗頓改。秦時李冰守蜀,導引汶江,創浸灌之利,至今地居水側者,須直千金,富強之家,多相侵奪。士廉乃於故渠外別更疏決,蜀中大獲其利。又因暇日汲引辭人,以為文會,兼命儒生講論經史,勉勵後進,蜀中學校粲然復興。蜀人硃桃椎者,淡泊為事,隱居不仕,披裘帶索,沉浮人間。竇軌之鎮益州也,聞而召見,遺以衣服,
 逼為鄉正。桃椎口竟無言,棄衣於地,逃入山中,結庵澗曲。夏則裸形,冬則樹皮自覆,人有贈遺,一無所受。每為芒履,置之於路,人見之者,曰:「硃居士之履也」。為鬻米置於本處,桃椎至夕而取之,終不與人相見。議者以為焦先之流。士廉下車,以禮致之,及至,降階與語,桃椎不答,直視而去。士廉每令存問,桃椎見使者,輒入林自匿。近代以來,多輕隱逸,士廉獨加褒禮,蜀中以為美談。五年,入為吏部尚書,進封許國公,仍封一子為縣公。獎鑒人
 倫,雅諳姓氏,凡所署用,莫不人地俱允。高祖崩,士廉攝司空,營山陵制度。事畢,加特進、上柱國。是時,朝議以山東人士好自矜誇,雖復累葉陵遲,猶恃其舊地,女適他族,必多求聘財。太宗惡之,以為甚傷教義,乃詔士廉與御史大夫韋挺、中書侍郎岑文本、禮部侍郎令狐德棻等刊正姓氏。於是普責天下譜諜,仍憑據史傳,考其真偽,忠賢者褒進,悖逆者貶黜,撰為《氏族志》。士廉乃類其等第以進。太宗曰:「我與山東崔、盧、李、鄭,舊既無嫌,為其
 世代衰微,全無冠蓋,猶自雲士大夫,婚姻之間,則多邀錢幣。才識凡下,而偃仰自高,販鬻松檟,依托富貴。我不解人間何為重之?祗緣齊家惟據河北,梁、陳僻在江南,當時雖有人物,偏僻小國,不足可貴,至今猶以崔、盧、王、謝為重。我平定四海,天下一家。凡在朝士,皆功效顯著,或忠孝可稱,或學藝通博,所以擢用。見居三品以上,欲共衰代舊門為親,縱多輸錢帛,猶被偃仰。我今特定族姓者,欲崇重今朝冠冕,何因崔干猶為第一等?昔漢高
 祖止是山東一匹夫,以其平定天下,主尊臣貴。卿等讀書,見其行跡,至今以為美談,心懷敬重。卿等不貴我官爵耶?不須論數世以前,止取今日官爵高下作等級。」遂以崔乾為第三等。及書成,凡一百卷,詔頒於天下。賜士廉物千段,尋同中書門下三品。十二年,與長孫無忌等以佐命功,並代襲刺史,授申國公。其年,拜尚書右僕射。士廉既任遇益隆,多所表奏,成輒焚稿,人莫知之。攝太子少師,特令掌選。十六年,加授開府儀同三司,尋表請
 致仕,聽解尚書右僕射,令以開府儀同三司依舊平章事。又正受詔與魏徵等集文學之士,撰《文思博要》一千二百卷,奏之,賜物千段。十七年二月,詔圖形凌煙閣。十九年,太宗伐高麗,皇太子定州監國,士廉攝太子太傅,仍典朝政。皇太子下令曰:「攝太傅、申國公士廉,朝望國華,儀刑攸屬,寡人忝膺監守,實資訓導。比聽政,常屈同榻,庶因諮白,少祛蒙滯。但據案奉對,情所未安,已約束不許更進。太傅誨諭深至,使遵常式,辭不獲免,輒復敬
 從。所司亦宜別以一案供太傅。」士廉固讓不敢當。二十年,遇疾,太宗幸其第問之,因敘說生平,流涕歔欷而訣。二十一年正月壬辰,薨於京師崇仁里私第,時年七十二。太宗又命駕將臨之,司空玄齡以上餌藥石,不宜臨喪,抗表切諫,上曰:「朕之此行,豈獨為君臣之禮,兼以故舊情深,姻戚義重,卿勿復言也。」太宗從數百騎出興安門,至延喜門,長孫無忌馳至馬前諫曰:「餌石臨喪,經方明忌。陛下含育黎元,須為宗社珍愛。臣亡舅士廉知將
 不救,顧謂臣曰:『至尊覆戴恩隆,不遺簪履,亡歿之後,或致親臨。內省凡才,無益聖日,安可以死亡之餘,輒回宸駕,魂而有靈,負譴斯及。』陛下恩深故舊,亦請察其丹誠。」其言甚切,太宗猶不許。無忌乃伏於馬前流涕,帝乃還宮。贈司徒、並州都督,陪葬昭陵,謚曰文獻。士廉祖、父洎身,並為僕射,子為尚書,甥為太尉,當代榮之。六子:履行、至行、純行、真行、審行、慎行。及喪柩出自橫橋,太宗登故城西北樓望而慟。高宗即位,追贈太尉,與房玄齡、屈突
 通並配享太宗廟庭。



 子履行,貞觀初歷祠部郎中。丁母憂,哀悴逾禮。太宗遣使諭之曰:「孝子之道,毀不滅性。汝宜強食,不得過禮。」服闋,累遷滑州刺史。尚太宗女東陽公主,拜駙馬都尉。十九年,除戶部侍郎,加銀青光祿大夫。無幾,遭父艱,居喪復以孝聞,太宗手詔敦喻曰:「古人立孝,毀不滅身。聞卿絕粒,殊乖大體,幸抑摧裂之情,割傷生之累。」俄起為衛尉卿,進加金紫光祿大夫,襲爵申國公。永徽元年,拜戶部尚書、檢校太子詹事、太常卿。顯
 慶元年,出為益州大都督府長史。先是,士廉居此職,頗著能名。至是,履行繼之,亦有善政,大為人吏所稱。三年,坐與長孫無忌親累,左授洪州都督,轉永州刺史,卒於官。



 履行弟真行,官至右衛將軍。其子典膳丞岐,坐與章懷太子陰謀,事洩,詔付真行令自懲誡。真行遂手刃之,仍棄其尸於衢路。高宗聞而鄙之,貶真行為睦州刺史,卒。



 長孫無忌,字輔機,河南洛陽人。其先出自後魏獻文帝
 第三兄。初為拓拔氏,宣力魏室,功最居多,世襲大人之號,後更跋氏,為宗室之長,改姓長孫氏。七世祖道生,後魏司空、上黨靖王。六世祖旃,後魏特進、上黨齊王。五世祖觀,後魏司徒、上黨定王。高祖稚,西魏太保、馮翊文宣王。曾祖子裕,西魏衛尉卿、平原郡公。祖光,周開府儀同三司,襲平原公。父晟,隋右驍衛將軍。無忌貴戚好學,該博文史,性通悟,有籌略。文德皇后即其妹也。少與太宗友善,義軍渡河,無忌至長春宮謁見,授渭北道行軍典
 簽。常從太宗征討,累除比部郎中,封上黨縣公。武德九年,隱太子建成、齊王元吉謀,將害太宗,無忌請太宗先發誅之。於是奉旨密召房玄齡、杜如晦等共為籌略。六月四日,無忌與尉遲敬德、侯君集、張公謹、劉師立、公孫武達、獨孤彥雲、杜君綽、鄭仁泰、李孟嘗等九人,入玄武門討建成、元吉,平之。太宗升春宮,授太子左庶子。及即位,遷左武候大將軍。貞觀元年,轉吏部尚書,以功第一,進封齊國公,實封千三百戶。太宗以無忌佐命元勛,地
 兼外戚,禮遇尤重,常令出入臥內。其年,拜尚書右僕射。時突厥頡利可汗新與中國和盟,政教紊亂,言事者多陳攻取之策。太宗召蕭瑀及無忌問曰:「北番君臣昏亂,殺戮無辜。國家不違舊好,便失攻昧之機;今欲取亂侮亡,復爽同盟之義。二途不決,孰為勝耶?」蕭瑀曰:「兼弱攻昧,擊之為善。」無忌曰:「今國家務在戢兵,待其寇邊,方可討擊。彼既已弱,必不能來。若深入虜廷,臣未見其可。且按甲存信,臣以為宜。」太宗從無忌之議。突厥尋政衰而
 滅。



 或有密表稱無忌權寵過盛,太宗以表示無忌曰:「朕與卿君臣之間,凡事無疑。若各懷所聞而不言,則君臣之意無以獲通。」因召百僚謂之曰:「朕今有子皆幼,無忌於朕,實有大功,今者委之,猶如子也。疏間親,新間舊,謂之不順,朕所不取也。」無忌深以盈滿為誡,懇辭機密,文德皇后又為之陳請,太宗不獲已,乃拜開府儀同三司,解尚書右僕射。是歲,太宗親祠南郊,及將還,命無忌與司空裴寂同升金輅。五年,與房玄齡、杜如晦、尉遲敬德
 四人,以元勛各封一子為郡公。七年十月,冊拜司空,無忌固辭,不許。又因高士廉奏曰:「臣幸居外戚,恐招聖主私親之誚,敢以死請。」太宗曰:「朕之授官,必擇才行。若才行不至,縱朕至親,亦不虛授,襄邑王神符是也;若才有所適,雖怨仇而不棄,魏徵等是也。朕若以無忌居後兄之愛,當多遺子女金帛,何須委以重官,蓋是取其才行耳。無忌聰明鑒悟,雅有武略,公等所知,朕故委之臺鼎。」無忌又上表切讓,詔報之曰:「昔黃帝得力牧而為五帝
 先,夏禹得咎繇而為三王祖,齊桓得管仲而為五伯長。朕自居籓邸,公為腹心,遂得廓清宇內,君臨天下。以公功績才望,允稱具瞻,故授此官,無宜多讓也。」太宗追思王業艱難,佐命之力,又作《威鳳賦》以賜無忌。其辭曰:



 有一威鳳,憩翮朝陽。晨游紫霧,夕飲玄霜。資長風以舉翰,戾天衢而遠翔。西翥則煙氛閉色,東飛則日月騰光。化垂鵬於北裔,馴群鳥於南荒。殄亂世而方降,應明時而自彰。俯翼雲路,歸功本樹。仰喬枝而見猜,俯修條而抱
 蠹。同林之侶俱嫉,共乾之儔並忤。無恆山之義情,有炎洲之兇度。若巢葦而居安,獨懷危而履懼。鴟鴞嘯乎側葉,燕雀喧乎下枝。慚己陋之至鄙,害他賢之獨奇。或聚咮而交擊,乍分羅而見羈。戢凌雲之逸羽,韜偉世之清儀。遂乃蓄情宵影,結志晨暉,霜殘綺翼,露點紅衣。嗟憂患之易結,嘆矰繳之難違。期畢命於一死,本無情於再飛。幸賴君子,以依以恃,引此風雲,濯斯塵滓。披蒙翳於葉下,發光華於枝裏。仙翰屈而還舒,靈音摧而復起。眄
 八極以遐翥,臨九天而高峙。庶廣德於眾禽,非崇利於一己。是以徘徊感德,顧慕懷賢。憑明哲而禍散,托英才而福全。答惠之情彌結,報功之志方宣。非知難而行易,思令後而終前。俾賢德之流慶,畢萬葉而芳傳。



 十一年,令與諸功臣世襲刺史。詔曰:



 周武定業,胙茅土於子弟;漢高受命,誓帶礪於功臣。豈止重親賢之地,崇其典禮,抑亦固磐石之基,寄以籓翰。魏、晉已降,事不師古,建侯之制,有乖名實。非所謂作屏王室,永固無窮者也。隋氏
 之季,四海沸騰,朕運屬殷憂,戡翦多難。上憑明靈之祐,下賴英賢之輔,廓清宇縣,嗣膺寶歷,豈予一人,獨能致此!時迍共資其力,世安專享其利,乃睠於斯,甚所不取。但今刺史,即古之諸侯,雖立名不同,監統一也。故申命有司,斟酌前代,宣條委共理之寄,象賢存世及之典。司空、齊國公無忌等,並策名運始,功參締構,義貫休戚,效彰夷險,嘉庸懿績,簡於朕心,宜委以籓鎮,改錫土宇。無忌可趙州刺史,改封趙國公;尚書左僕射、魏國公玄齡
 可宋州刺史,改封梁國公;故司空、蔡國公杜如晦可贈密州刺史,改封萊國公;特進、代國公靖可濮州刺史,改封衛國公;特進、吏部尚書、許國公士廉可申州刺史,改封申國公;兵部尚書、潞國公侯君集可陳州刺史,改封陳國公;刑部尚書、任城郡王道宗可鄂州刺史,改封江夏郡王;晉州刺史、趙郡王孝恭可觀州刺史,改封河間郡王;同州刺史、吳國公尉遲敬德可宣州刺史,改封鄂國公;並州都督府長史、曹國公李勣可蘄州刺史,改封
 英國公;左驍衛大將軍、楚國公段志玄可金州刺史,改封褒國公;左領軍大將軍、宿國公程知節可普州刺史,改封盧國公;太僕卿、任國公劉弘基可朗州刺史,改封夔國公;相州都督府長史、鄅國公張亮可澧州刺史,改封鄖國公。餘官食邑並如故,即令子孫奕葉承襲。



 無忌等上言曰:「臣等披荊棘以事陛下,今海內寧一,不願違離,而乃世牧外州,與遷徙何異。」乃與房玄齡上表曰:



 臣等聞質文迭變,皇王之跡有殊;今古相沿,致理之方乃
 革。緬惟三代,習俗靡常,爰制五等,隨時作教。蓋由力不能制,因而利之,禮樂節文,多非己出。逮於兩漢,用矯前違,置守頒條,蠲除曩弊。為無益之文,覃及四方;建不易之理,有逾千載。今曲為臣等,復此奄荒,欲其優隆,錫之茅社,施於子孫,永貽長世。斯乃大鈞播物,毫發並施其生;小人逾分,後世必嬰其禍。何者?違時易務,曲樹私恩,謀及庶僚,義非僉允。方招史冊之誚,有紊聖代之綱。此其不可一也。又臣等智效罕施,器識庸陋。或情緣右戚,
 遂陟臺階;或顧想披荊,便蒙夜拜。直當今日,猶愧非才,重裂山河,愈彰濫賞。此其不可二也。又且孩童嗣職,義乖師儉之方,任以褰帷,寧無傷錦之弊?上干天憲,彞典既有常科,下擾生民,必致餘殃於後,一掛刑網,自取誅夷。陛下深仁,務延其世,翻令剿絕,誠有可哀。此其不可三也。當今聖歷欽明,求賢分政,古稱良守,寄在共理。此道之目,為日滋久,因緣臣等,或有改張。封植兒曹,失於求瘼,百姓不幸,將焉用之?此其不可四也。在茲一舉,為
 損實多,曉夕深思,憂貫心髓。所以披丹上訴,指事明心,不敢浮辭,同於矯飾。伏願天澤,諒其愚款,特停渙汗之旨,賜其性命之恩。



 太宗覽表謂曰:「割地以封功臣,古今通義,意欲公之後嗣,翼朕子孫,長為籓翰,傳之永久。而公等薄山河之誓,發言怨望,朕亦安可強公以土宇耶?」於是遂止。十二年,太宗幸其第,凡是親族,班賜有差。十六年,冊拜司徒。十七年,令圖畫無忌等二十四人於凌煙閣,詔曰:



 自古皇王,褒崇勛德,既勒銘於鐘鼎,又圖形
 於丹青。是以甘露良佐,麟閣著其美;建武功臣,雲臺紀其跡。司徒、趙國公無忌,故司空、揚州都督、河間元王孝恭,故司空、萊國成公如晦,故司空、相州都督、太子太師、鄭國文貞公征,司空、梁國公玄齡,開府儀同三司、尚書右僕射、申國公士廉,開府儀同三司、鄂國公敬德,特進、衛國公靖,特進、宋國公瑀,故輔國大將軍、揚州都督、褒忠壯公志玄,輔國大將軍、夔國公弘基,故尚書左僕射、蔣忠公通,故陜東道行臺右僕射、鄖節公開山,故荊州都督、譙襄公柴紹,故荊州都督、邳襄公順
 德,洛州都督、鄖國公張亮,光祿大夫、吏部尚書、陳國公侯君集,故左驍衛大將軍、郯襄公張公謹,左領軍大將軍、盧國公程知節,故禮部尚書、永興文懿公虞世南,故戶部尚書、渝襄公劉政會,光祿大夫、戶部尚書、莒國公唐儉,光祿大夫、兵部尚書、英國公勣,故徐州都督、胡壯公秦叔寶等,或材推棟梁,謀猷經遠,綢繆帷帳,經綸霸圖;或學綜經籍,德範光茂,隱犯同致,忠讜日聞;或竭力義旗,委質籓邸,一心表節,百戰標奇;或受脤廟堂,闢土
 方面,重氛載廓,王略遐宣。並契闊屯夷,劬勞師旅。贊景業於草昧,翼淳化於隆平。茂績殊勛,冠冕列闢;昌言直道,牢籠搢紳。宜酌故實,弘茲令典,可並圖畫於凌煙閣。庶念功之懷,無謝於前載;旌賢之義,永貽於後昆。



 其年,太子承乾得罪,太宗欲立晉王,而限以非次,回惑不決。御兩儀殿,群官盡出,獨留無忌及司空房玄齡、兵部尚書李勣,謂曰:「我三子一弟,所為如此,我心無憀。」因自投於床,抽佩刀欲自刺。無忌等驚懼,爭前扶抱,取佩刀以
 授晉王。無忌等請太宗所欲,報曰:「我欲立晉王。」無忌曰:「謹奉詔。有異議者,臣請斬之。」太宗謂晉王曰:「汝舅許汝,宜拜謝。」晉王因下拜。太宗謂無忌等曰:「公等既符我意,未知物論何如?」無忌曰:「晉王仁孝,天下屬心久矣。伏乞召問百僚,必無異辭。若不蹈舞同音,臣負陛下萬死。」於是建立遂定,因加授無忌太子太師。尋而太宗又欲立吳王恪,無忌密爭之,其事遂輟。太宗嘗謂無忌等曰:「朕聞主賢則臣直,人苦不自知,公宜面論,攻朕得失。」無忌
 奏言:「陛下武功文德,跨絕古今,發號施令,事皆利物。《孝經》云:『將順其美。』臣順之不暇,實不見陛下有所愆失。」太宗曰:「朕冀聞己過,公乃妄相諛悅。朕今面談公等得失,以為鑒誡。言之者可以無過,聞之者可以自改。」因目無忌曰:「善避嫌疑,應對敏速,求之古人,亦當無比;而總兵攻戰,非所長也。高士廉涉獵古今,心術聰悟,臨難既不改節,為官亦無朋黨;所少者骨鯁規諫耳。唐儉言辭便利,善和解人,酒杯流行,發言啟齒;事朕三十載,遂無一
 言論國家得失。楊師道性行純善,自無愆過;而情實怯懦,未甚任事,緩急不可得力。岑文本性道敦厚,文章是其所長;而持論常據經遠,自當不負於物。劉洎性最堅貞,言多利益;然其意上然諾於朋友,能自補闕,亦何以尚。馬周見事敏速,性甚貞正,至於論量人物,直道而行,朕比任使,多所稱意。褚遂良學問稍長,性亦堅正,既寫忠誠,甚親附於朕,譬如飛鳥依人,自加憐愛。」十九年,太宗征高麗,令無忌攝侍中。還,無忌固辭師傅之位,優詔
 聽罷太子太師。二十一年,遙領揚州都督。二十三年,太宗疾篤,引無忌及中書令褚遂良二人受遺令輔政。太宗謂遂良曰:「無忌盡忠於我,我有天下,多是此人力。爾輔政後,勿令讒毀之徒損害無忌。若如此者,爾則非復人臣。」



 高宗即位,進拜太尉,兼揚州都督,知尚書及門下二省事並如故。無忌固辭知尚書省事,許之,仍令以太尉同中書門下三品。永徽二年,監修國史。高宗嘗謂公卿:「朕開獻書之路,冀有意見可錄,將擢用之。比者上疏
 雖多,而遂無可採者。」無忌對曰:「陛下即位,政化流行,條式律令,固無遺闕。言事者率其鄙見,妄希僥幸,至於裨俗益教,理當無足可取。然須開此路,猶冀時有讜言,如或杜絕,便恐下情不達。」帝曰:「又聞所在官司,猶自多有顏面。」無忌曰:「顏面阿私,自古不免。然聖化所漸,人皆向公,至於肆情曲法,實謂必無此事。小小收取人情,恐陛下尚亦不免,況臣下私其親戚,豈敢頓言絕無?」時無忌位當元舅,數進謀議,高宗無不優納之。明年,以旱上疏
 辭職,高宗頻降手詔敦喻不許。五年,親幸無忌第,見其三子,並擢授朝散大夫。又命圖無忌形像,親為畫贊以賜之。六年,帝將立昭儀武氏為皇后,無忌屢言不可,帝乃密遣使賜無忌金銀寶器各一車、綾錦十車,以悅其意。昭儀母楊氏復自詣無忌宅,屢加祈請。時禮部尚書許敬宗又屢申勸請,無忌嘗厲色折之。帝後又召無忌、左僕射於志寧、右僕射褚遂良,謂曰:「武昭儀有令德,朕欲立為皇后,卿等以為如何?」無忌曰:「自貞觀二十三年
 後,先朝付托遂良,望陛下問其可否。」帝竟不從無忌等言而立昭儀為皇后。皇后以無忌先受重賞而不助己,心甚銜之。顯慶元年,無忌與史官國子祭酒令狐德棻綴集武德、貞觀二朝史為八十卷,表上之,無忌以監領功,賜物二千段,封其子潤為金城縣子。四年,中書令許敬宗遣人上封事,稱監察御史李巢與無忌交通謀反,帝令敬宗與侍中辛茂將鞠之。敬宗奏言無忌謀反有端,帝曰:「我家不幸,親戚中頻有惡事。高陽公主與朕同
 氣,往年遂與房遺愛謀反,今阿舅復作惡心。近親如此,使我慚見萬姓。」敬宗曰:「房遺愛乳臭兒,與女子謀反,豈得成事?且無忌與先朝謀取天下,眾人服其智,作宰相三十年,百姓畏其威,可謂威能服物,智能動眾。臣恐無忌知事露,即為急計,攘袂一呼,嘯命同惡,必為宗廟深憂。誠願陛下斷之,不日即收捕,準法破家。」帝泣曰:「我決不忍處分與罪,後代良史道我不能和其親戚,使至於此。」敬宗曰:「漢文帝漢室明主,薄昭即是帝舅,從代來日,
 亦有大勛,與無忌不別。於後惟坐殺人,文帝惜國之法,令朝臣喪服就宅,哭而殺之,良史不以為失。今無忌忘先朝之大德,舍陛下之至親,聽受邪謀,遂懷悖逆,意在塗炭生靈。若比薄昭罪惡,未可同年而語,案諸刑典,合誅五族。臣聞當斷不斷,反受其亂,大機之事,間不容發,若少遲延,恐即生變,惟請早決!」帝竟不親問無忌謀反所由,惟聽敬宗誣構之說,遂去其官爵,流黔州,仍遣使發次州府兵援送至流所。其子秘書監、駙馬都尉沖等
 並除名,流於嶺外。敬宗尋與吏部尚書李義府遣大理正袁公瑜就黔州重鞫無忌反狀,公瑜逼令自縊而死,籍沒其家。無忌既有大功,而死非其罪,天下至今哀之。上元元年,優詔追復無忌官爵,特令無忌孫延主齊獻公之祀。無忌從父兄安世,仕王世充,署為內史令,東都平,死於獄中。安世子祥,以文德皇后近屬,累除刑部尚書,坐與無忌通書見殺。



 史臣曰:士廉才望素高,操秉無玷,保君臣終始之義,為
 子孫襲繼之謀。社稷之臣,功亦隆矣;獎遇之恩,賞亦厚矣。及子真行,手刃其子,何兇忍也?若是積慶之道,不其惑哉!無忌戚里右族,英冠人傑,定立儲闈,力安社稷,勛庸茂著,終始不渝。及黜廢中宮,竟不阿旨,報先帝之顧托,為敬宗之誣構。嗟乎!忠信獲罪,今古不免;無名受戮,族滅何辜!主暗臣奸,足貽後代。



 贊曰:嚴嚴申公,功名始終。文皇題品,信謂酌中。趙公右戚,兩朝宣力。功成不去,竟逢鬼域。



\end{pinyinscope}