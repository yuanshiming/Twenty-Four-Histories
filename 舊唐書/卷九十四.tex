\article{卷九十四}

\begin{pinyinscope}

 ○王及善杜景儉硃敬則楊再思李懷遠子景伯景伯子彭年附豆盧欽望張光輔史務滋崔元綜周允元附



 王及善,洺州邯鄲人也。父君愕。隋大業末,並州人王君廓掠邯鄲,君愕往說君廓曰:「方今萬乘失御,英雄競起,誠宜撫納遺氓,保全形勝,按甲以觀時變,擁眾而歸真主,此富貴可圖也。今足下居無尺土之地,守無兼旬之糧,恣行殘忍,所過攘敚,竊為足下寒心矣。」君廓曰:「計將安出?」君愕為陳井陘之險,可先往據之。君廓從其言,乃屯井陘山。歲餘,會義師入定關中,乃與君廓率所部萬餘人來降,拜大將軍。頻以戰功封新興縣公,累遷左武衛將軍。從太宗征遼東,兼領左屯營兵馬。與高麗戰於駐蹕山,君愕先鋒陷陣,力戰而死。太宗深痛悼之,贈左衛大將軍、幽州都督、邢國公。賜東園秘器,陪葬昭陵。



 杜景儉,冀州武邑人也。少舉明經,累除殿中侍御史。出為益州錄事參軍。時隆州司馬房嗣業除益州司馬,除書未到,即欲視事,又鞭笞僚吏,將以示威,景儉謂曰:「公
 雖受命為此州司馬,而州司未受命也。何藉數日之祿,而不待九重之旨,即欲視事,不亦急耶?」嗣業益怒。景儉又曰:「公今持咫尺之制,真偽未知,即欲攬一州之權,誰敢相保?揚州之禍,非此類耶。」乃叱左右各令罷散,嗣業慚赧而止。俄有制除嗣業荊州司馬,竟不如志,人吏為之語曰:「錄事意,與天通,益州司馬折威風。」景儉由是稍知名。入為司賓主簿,轉司刑丞。



 天授中,與徐有功、來俊臣、侯思止專理制獄,時人稱云:「遇徐、杜者必生,遇來、侯
 者必死。」累遷洛州司馬。尋轉鳳閣侍郎、同鳳閣鸞臺平章事。則天嘗以季秋內出梨花一枝示宰臣曰:「是何祥也?」諸宰臣曰:「陛下德及草木,故能秋木再花,雖周文德及行葦,無以過也。」景儉獨曰:「謹按《洪範五行傳》:『陰陽不相奪倫,瀆之即為災。』又《春秋》云:『冬無愆陽,夏無伏陰,春無淒風,秋無苦雨。』今已秋矣,草木黃落,而忽生此花,瀆陰陽也。臣慮陛下布教施令,有虧禮典。又臣等忝為宰臣,助天理物,理而不和,臣之罪也。」於是再拜謝罪,則天
 曰:「卿真宰相也!」



 延載初,為鳳閣侍郎周允元奏景儉黨於李昭德,左遷溱州刺史。後累除司刑卿。聖歷二年,復拜鳳閣侍郎、同鳳閣鸞臺平章事。時契丹入寇,河北諸州多陷賊中。及事定,河內王武懿宗將盡論其罪。景儉以為皆是驅逼,非其本心,請悉原之。則天竟從景儉議。歲餘,轉秋官尚書。坐漏洩禁中語,左授司刑少卿,出為並州長史。道病卒,贈相州刺史。子澄,頗以文藻著名,官至鞏縣尉。



 硃敬則,字少連,毫州永城人也。代以孝義稱,自周至唐,三代旌表,門標六闕,州黨美之。敬則倜儻重節義,早以辭學知名。與三從兄同居,財產無異。又與左史江融、左僕射魏元忠特相友善。咸亨中,高宗聞而召見,與語甚奇之,將加擢用,為中書舍人李敬玄所毀,乃授洹水尉。



 長壽中,累除右補闕。敬則以則天初臨朝稱制,天下頗多流言異議,至是既漸寧晏,宜絕告密羅織之徒,上疏曰:



 臣聞李斯之相秦也,行申、商之法,重刑名之家,杜私
 門,張公室,棄無用之費,損不急之官,惜日愛功,疾耕急戰,人繁國富,乃屠諸侯。此救弊之術也。故曰:刻薄可施於進趨,變詐可陳於攻戰。兵猶火也,不戢將自焚。況鋒鏑已銷,石城又毀,諒可易之以寬泰,潤之以淳和,八風之樂以柔之,三代之禮以導之。秦既不然,淫虐滋甚,往而不返,卒至土崩,此不知變之禍也。



 陸賈、叔孫通之事漢王也,當滎陽、成皋之間,糧饋已窮,智勇俱困,不敢開一說,效一奇,唯進豪猾之材,薦貪暴之客。及區宇適平,
 干戈向戢,金鼓之聲未歇,傷痍之痛尚聞,二子顧眄,綽有餘熊,乃陳《詩書》,說《禮樂》,開王道,謀帝圖。高皇帝忿然曰:「吾以馬上得之,安事《詩事》乎!」對曰:「馬上得之,可馬上理之乎?」高皇默然。於是陸賈著《新語》,叔孫通定禮儀,始知天子之尊,此知變之善也。向使高皇排二子而不用,置《詩書》而不顧,重攻戰之吏,尊首級之材,復道爭功,張良已知其變,拔劍擊柱,吾屬不得無謀。即晷漏難逾,何十二帝乎?亡秦之續,何二百年乎?故曰:仁義者,聖人之
 蘧廬;禮經者,先王之陳跡。然則祝祠向畢,芻狗須投;淳精已流,糟粕可棄。仁義尚舍,況輕此者乎?自文明草昧,天地屯蒙,三叔流言,四兇構難。不設鉤距,無以應天順人;不切刑名,不可摧奸息暴。故置神器,開告端,曲直之影必呈,包藏之心盡露。神道助直,無罪不除;人心保能,無妖不戮。以茲妙算,窮造化之幽深;用此神謀,入天人之秘術。故能計不下席,聽不出闈,蒼生晏然,紫宸易主。大哉偉哉,無得而稱也!豈比造攻鳴條,大戰牧野,血變
 草木,頭折不周,可同年而語乎?然而急趨無善跡,促柱少和聲,拯溺不規行,療饑非鼎食。即向時之妙策,乃當今之芻狗也。伏願覽秦、漢之得失,考時事之合宜,審糟粕之可遺,覺蘧廬之須毀。見機而作,豈勞終日乎?陛下必不可偃蹇太平,徘徊中路。伏願改法制,立章程,下恬愉之辭,流曠蕩之澤,去萋菲之牙角,頓奸險之鋒芒,窒羅織之源,掃朋黨之跡,使天下蒼生坦然大悅,豈不樂哉!



 則天甚善之。



 長安三年,累遷正諫大夫,尋同鳳閣鸞
 臺平章事。時御史大夫魏元忠、鳳閣舍人張說為張易之兄弟所誣構,將陷重闢,諸宰相無敢言者,敬則獨抗疏申理曰:「元忠、張說素稱忠正,而所坐無名。若令得罪,豈不失天下之望也?」乃得減死。四年,以老疾請罷知政事,許之,累轉冬官侍郎,仍依舊兼修國史。張易之、昌宗嘗命畫工圖寫武三思及納言李嶠、鳳閣侍郎蘇味道、夏官侍郎李迥秀、麟臺少監王紹宗等十八人形像,號為《高士圖》,每引敬則預其事,固辭不就,其高潔守正如
 此。



 神龍元年,出為鄭州刺史,尋以老致仕。二年,侍御史冉祖雍素與敬則不協,乃誣奏云與王同皎親善,貶授廬州刺史。經數月,洎代到,還鄉里,無淮南一物,唯有所乘馬一匹,諸子侄步從而歸。敬則重然諾,善與人交,每拯人急難,不求其報。又嘗與三從兄同居四十餘年,財產無異。雅有知人之鑒,凡在品論者,後皆如其言。景龍三年五月,卒於家,年七十五。



 敬則嘗採魏、晉已來君臣成敗之事,著《十代興亡論》。又以前代文士論廢五等者,
 以秦為失,事未折衷,乃著《五等論》曰:



 昔秦廢五等,崔實、仲長統、王朗、曹冏等皆以為秦之失,予竊異之,試通其志云。



 蓋明王之理天下也,先之以博愛,本之以仁義,張四維,尊五美,懸禮樂於庭宇,置軌範於中衢。然後決玄波使橫流,揚薰風以高扇,流愷悌之甘澤,浸曠蕩之膏腴,正理革其淫邪,淳風柔其骨髓。使天下之人,心醉而神足。其於忠義也,立則見其參於前;其於進趨也,若章程之在目。禮經所及,等日月之難逾;聲教所行,雖風雨
 之不輟。聖人知俗之漸化也,王道之已行也,於是體國經野,庸功勛親。分山裂河,設磐石之固,內守外御,有維城之基。連絡遍於域中,膠葛盡於封內。雖道昏時喪,澤竭政塞,鄭伯逐王,申侯弒主,魯不供物,宋不成周,吳徵伯牢,楚問九鼎,小白之一匡天下,重耳之一戰諸侯,無君之跡顯然,篡奪之謀中寢者,直以周禮尚存,簡書不隕。故曰:「不敢失墜,天威在顏。」



 自春秋之後,禮義漸頹,風俗塵昏,愧恥心盡,疾走先得者為上,奪攘投會者為
 能。加以八世專齊,三家分晉,子貢之亂五國,蘇秦之斗七雄,苛刻繁興,經籍道息,莫不長詐術,貴攻戰,萬姓皆戴爪牙,無人不屬觜距。所以商鞅欺故友,李斯囚舊交,孫臏喪足於龐涓,張儀得志於陳軫。一旅之眾,便欲稱王;再戰之雄,爭來奉帝。先王會盟之禮,昔時樽俎之容,三代玄風,掃地至盡。況始皇削平區宇,殊非至公,李斯之作股肱,罕循大道,人無見德,唯虐是聞。當此時也,主猜於上,人駭於下,父不能保之於子,君不能得之於臣。欲
 使始皇分土奸雄,建侯薄俗,若喻晉、鄭之可依,便借賊兵而資盜糧,寄龍魚而助風雨,不可行也。是以秦鑒周德之綿深,懼己圖之不遠,罷侯置守,高下在心,天下制在一人,百姓不聞二主。直是不得行其世封,非薄功臣而賤骨肉也。



 高皇帝揭日月之明,懷天地之量,算財不足以分賞,論地不足以受封。邑皆百城,土有千里,人殷國富,地廣兵強。五十年間,七國同反,賈誼憂失其國,晁錯請削其地。若言由大而反也,不若召陵之師、踐土之
 眾也;若言有材而起也,劉濞非王霸之材,田祿無先、管之略也。是齊、晉以逆禮為慚,吳、楚以犯上非愧,釁由教起,其所由來遠矣。自此之後,雜霸又衰,中興不能改物創圖,黃初不能深謀遠慮。糸面觀漢、魏之際,尋其經緯之初,未有積德重光,澤及萬物。觀其教,偷薄於秦風;察其人,豺狼於漢日。故魏太祖曰:「若使無孤,天下幾人稱帝,幾人稱王!」明竊號議者,觸目皆是。欲以此時開四賜之祚,垂萬代之封,必有通車三川以窺周室,介馬汾、濕而
 逐翼侯。而王司徒屢請於當時,曹元首又勤於宗室,皆不知時也。



 當時賢者是之。



 敬則知政事時,每以用人為先。桂州蠻叛,薦裴懷古;鳳閣舍人缺,薦魏知古;右史缺,薦張思敬。則天以為知人。



 睿宗即位,嘗謂侍臣曰:「神龍已來,李多祚、王同皎並復舊官,韋月將、燕欽融咸有褒贈,不知更有何人,尚抱冤抑?」吏部尚書劉幽求對曰:「故鄭州刺史硃敬則,往在則天朝任正諫大夫、知政事,忠貞義烈,為天下所推。神龍時,被宗楚客、冉祖雍等誣構,
 左授廬州刺史。長安年中,嘗謂臣云:『相王必膺期受命,當須盡節事之。』及韋氏篡逆干紀,臣遂見危赴難,翼戴興歷,雖則天誘其事,亦是敬則先啟之心。今陛下龍興寶位,兇黨就戮,敬則尚銜冤泉壤,未蒙昭雪。況復事符先覺,誠即可嘉。」睿宗然之,贈敬則秘書監,謚曰元。



 楊再思,鄭州原武人也。少舉明經,授玄武尉。充使詣京師,止於客舍。會盜竊其囊裝,再思邂逅遇之,盜者伏罪,再思謂曰:「足下當苦貧匱,至此無行。速去勿作聲,恐為
 他人所擒。幸留公文,餘財盡以相遺。」盜者齋去,再思初不言其事,假貸以歸。累遷天官員外郎,歷左右肅政臺御史大夫。延載初,守鸞臺侍郎、同鳳閣鸞臺平章事。證聖初,轉鳳閣侍郎,依前同平章事,兼太子右庶子。尋遷內史,自弘農縣男累封至鄭國公。



 再思自歷事三主,知政十餘年,未嘗有所薦達。為人巧佞邪媚,能得人主微旨,主意所不欲,必因而毀之,主意所欲,必因而譽之。然恭慎畏忌,未嘗忤物。或謂再思曰:「公名高位重,何為屈
 折如此?」再思曰:「世路艱難,直者受禍。茍不如此,何以全其身哉!」長安末,昌宗既為法司所鞫,司刑少卿桓彥範斷解其職。昌宗俄又抗表稱冤,則天意將申理昌宗,廷問宰臣曰:「昌宗於國有功否?」再思對曰:「昌宗往因合練神丹,聖躬服之有效,此實莫大之功。」則天甚悅,昌宗竟以復職。時人貴彥範而賤再思也。時左補闕戴令言作《兩腳野狐賦》以譏刺之,再思聞之甚怒,出令言為長社令,朝士尤加嗤笑。再思為御史大夫時,張易之兄司禮
 少卿同休嘗奏請公卿大臣宴於司禮寺,預其會者皆盡醉極歡。同休戲曰:「楊內史面似高麗。」再思欣然,請剪紙自貼於巾,卻披紫袍,為高麗舞,縈頭舒手,舉動合節,滿座嗤笑。又易之弟昌宗以姿貌見寵幸,再思又諛之曰:「人言六郎面似蓮花;再思以為蓮花似六郎,非六郎似蓮花也。」其傾巧取媚也如此。



 長安四年,以本官檢校京兆府長史,又遷檢校揚州大都督府長史。中宗即位,拜戶部尚書,兼中書令,轉侍中,以宮僚封鄭國公,賜實
 封三百戶。又為冊順天皇后使,賜物五百段,鞍馬稱是。時武三思將誣殺王同皎,再思與吏部尚書李嶠、刑部尚書韋巨源並受制考按其獄,竟不能發明其枉,致同皎至死,眾冤之。再思俄復為中書令、吏部尚書。景龍三年,遷尚書右僕射,加光祿大夫。其年薨,贈特進、並州大都督,陪葬乾陵,謚曰恭。子植、植子獻,並為司勛員外郎。再思弟季昭為考功郎中,溫玉為戶部侍郎。



 李懷遠,邢州柏仁人也。早孤貧好學,善屬文。有宗人欲
 以高廕相假者,懷遠竟拒之,退而嘆曰:「因人之勢,高士不為;假廕求官,豈吾本志?」未幾,應四科舉擢第,累除司禮少卿。出為邢州刺史,以其本鄉,固辭不就,改授冀州刺史。俄歷揚、益等州大都督府長史,未行,又授同州刺史。在職以清簡稱。入為太子左庶子,兼太子賓客,歷遷右散騎常侍、春官侍郎。大足年,遷鸞臺侍郎,尋同鳳閣鸞臺平章事。歲餘,加銀青光祿大夫,拜秋官尚書,兼檢校太子左庶子,賜爵平鄉縣男。長安四年,以老辭職,聽
 解秋官尚書,正除太子左庶子,尋授太子賓客。神龍初,除左散騎常侍、兵部尚書、同中書門下三品,加金紫光祿大夫,進封趙郡公,特賜實封三百戶。俄以疾請致仕,許之。中宗將幸京師,又令以本官知東都留守。



 懷遠雖久居榮位,而彌尚簡率,園林宅室,無所改作。常乘款段馬,左僕射豆盧欽望謂曰:「公榮貴如此,何不買駿馬乘之?」答曰:「此馬幸免驚蹶,無假別求。」聞者莫不嘆美。神龍二年八月卒,中宗特賜錦被以充斂,輟朝一日,親為文
 以祭之,贈侍中,謚曰成。子景伯。



 景伯,景龍中為給事中,又遷諫議大夫。中宗嘗宴侍臣及朝集使,酒酣,令各為《回波辭》。眾皆為謅佞之辭,及自要榮位。次至景伯,曰:「回波爾時酒卮,微臣職在箴規。侍宴既過三爵,喧嘩竊恐非儀。」中宗不悅,中書令蕭至忠稱之曰:「此真諫官也。」景雲中,累遷右散騎常侍,尋以老疾致仕。開元中卒。子彭年。彭年有吏才,工於剖析,當時稱之。開元中,歷考功員外郎、知舉,又遷中書舍人、給事中、兵部侍郎。天寶初,又
 為吏部侍郎,與右相李林甫善。慕山東著姓為婚姻,引就清列,以大其門。典銓管七年,後以贓污為御史中丞宋渾所劾,長流領南臨賀郡。累月,渾及第恕又以贓下獄,詔渾流嶺南高要郡,恕流南康郡。天寶十二載,起彭年為濟陰太守,又遷馮翊太守,入為中書舍人、給事中、吏部侍郎。十五載,玄宗幸蜀,賊陷西京。彭年沒於賊,脅授偽官,憂憤忽忽不得志,與韋斌相次而卒。及克復兩京,優制贈彭年為禮部尚書。



 豆盧欽望,京兆萬年人也。曾祖通,隋相州刺史、南陳郡公。祖寬,即隋文帝之甥也。大業末,為梁泉令。及高祖定關中,寬與郡守蕭瑀率豪右赴京師,由是累授殿中監,仍詔其子懷讓尚萬春公主。高祖以寬曾祖萇魏太和中例稱單姓,至是改寬為盧氏。貞觀中,歷遷禮部尚書、左衛大將軍,封芮國公。永徽元年卒,贈特進、並州都督,陪葬昭陵,謚曰定。又復其姓為豆盧氏。父仁業,高宗時為左衛將軍。



 欽望,則天時累遷司賓卿。長壽二年,代
 宗秦客為內史。時李昭德亦為內史,執權用事,欽望與同時宰相韋巨源、陸元方、蘇味道、杜景儉等並委曲從之。證聖元年,昭德坐事,左遷涪陵尉,則天以欽望等不能執正,又為司刑少卿皇甫文備奏欽望附會昭德,罔上附下,乃左遷欽望為趙州刺史,韋巨源自右丞為鄜州刺史,陸元方自秋官侍郎為綏州刺史,蘇味道自鳳閣侍郎為集州刺史。其年,欽望入為司禮卿,遷秋官尚書,封芮國公。出為河北道宣勞使。俄而廬陵王復為皇太
 子,以欽望為皇太子宮尹。聖歷二年,拜文昌右相、同鳳閣鸞臺三品,尋授太子賓客,停知政事。中宗即位,以欽望宮僚舊臣,拜尚書左僕射、知軍國重事,兼檢校安國相王府長史,兼中書令、知兵部事、監修國史。



 欽望作相兩朝,前後十餘年,張易之兄弟及武三思父子皆專權驕縱,圖為逆亂。欽望獨謹其身,不能有所匡正,以此獲譏於代。神龍二年,拜開府儀。景龍三年五月,表請氣骸,不許。十一月卒,年八十餘。贈司空、並州大都督,
 謚曰元,賜東園秘器,陪葬乾陵。則天時,宰相又有張光輔、史務滋、崔元綜、周允元等,並有名績。



 張光輔者,京兆人也。少明辯,有吏乾。累遷司農少卿、文昌右丞。以討平越王貞之功,拜鳳閣侍郎、知政事。永昌元年,遷納言。旬日,又拜內史。皆有名。其年,洛州司馬房嗣業、洛陽令張嗣明坐與徐敬業弟敬真陰相交結。敬真自流所繡州逃歸,將北投突厥,引虜入寇。途經洛下,嗣業、嗣明二人給其衣糧而遣之。行至定州,為人所覺。
 嗣業於獄中自縊死。嗣明與敬真多引海內相識,冀緩其死。嗣明稱光輔征豫州日,私說圖識天文,陰懷兩端,顧望以觀成敗。光輔由是被誅,家口籍沒。



 史務滋者,宣州溧陽人。累至內史。天授中,雅州刺史劉行實及弟渠州刺史行瑜、尚衣奉御行感,並兄子左鷹揚將軍虔通,並為侍御史來子珣誣以謀反誅。又於盱眙毀其父左監門大將軍伯英棺柩。初,務滋素與行感周密,意俗寢其反狀。則天怒,令俊臣鞫之,務滋恐被陷
 刑,乃自殺。



 崔元綜者,鄭州新鄭人也。祖君肅,武德中黃門侍郎、鴻臚卿。元綜,天授中累轉秋官侍郎。長壽元年,遷鸞臺侍郎、同鳳閣鸞臺平章事。元綜勤於政事,每在中書,必束帶至晚,未嘗休偃。好潔細行,薰辛不歷口者二十餘年。雖外示謹厚,而情深刻薄,每受制鞫獄,必披毛求疵,陷於重闢。以此故人多畏而鄙之。明年,犯罪配流振州,朝野莫不稱慶。尋赦還,復拜監察御史。中宗時,累遷尚書
 左丞、蒲州刺史,以老疾致仕。晚年好攝養導引之術,年九十餘卒。



 周允元者,豫州人也。弱冠舉進士。延戴初,累轉左肅政御史中丞,俄除鳳閣鸞臺平章事。嘗與諸宰臣侍宴,則天令各述書傳中善言。允元曰:「恥其君不如堯、舜。」武三思以為語有指斥,糾而駁之。則天曰:「聞此言足以為誡,豈特將為過耶?」證聖元年卒,贈貝州刺史。則天為七言詩以傷之,又自繕寫,時以為榮。



 史官曰:王及善在孝敬東宮,誠能奉職。當俊臣下獄,力諫除兇,是憂濫及賢良,而欲明彰羽翼,興復之志,不謂無心。杜景儉五刑有濫,濟活為心,四氣不和,歸罪在己,則天謂曰「真宰相。」然奈柔順李昭德,不無吐剛之過也。硃敬則文學有稱,節行無愧,諫諍果決,推擇精真,茍非洞鑒古今,深識王霸,何由立其高論哉?惜乎相不得時矣!楊再思佞而取貴,茍以全身,掩不善而自欺,謂無十目十手也。李懷遠名不茍於假廕,貴不衒於故鄉,無改
 陋居,常乘劣駟,亦一時之善矣。然匪躬之道,未之聞也。豆盧欽望、張光輔、史務滋、崔元綜、周允元等,或有片言,非無小善,登於大用,可謂具臣。



 贊曰:及善奉職,非無智力。景儉當權,不謂不賢。雄文高節,少連為絕。守道安貧,懷遠當仁。欽望之屬,片善何足。蹈媚再思,祇宜遄速。



\end{pinyinscope}