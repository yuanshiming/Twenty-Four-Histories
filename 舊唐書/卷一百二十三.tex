\article{卷一百二十三}

\begin{pinyinscope}

 ○楊綰崔祐甫子植植再從兄俊常袞



 楊綰,字公權,華州華陰人也。祖溫玉,則天朝為戶部侍郎、國子祭酒。父侃,開元中醴泉令,皆以儒行稱。綰生聰惠,年四歲,處群從之中,敏識過人。嘗夜宴親賓,各舉坐
 中物以四聲呼之,諸賓未言,綰應聲指鐵燈樹曰:「燈盞柄曲。」眾咸異之。及長,好學不倦,博通經史,九流七略,無不該覽,尤工文辭,藻思清贍。而宗尚玄理,沉靜寡欲,常獨處一室,左右經書,凝塵滿席,澹如也。含光晦用,不欲名彰,每屬文,恥於自白,非知己不可得而見。早孤家貧,養母以孝聞,甘旨或闕,憂見於色。親友諷令干祿,舉進士。調補太子正字。天寶十三年,玄宗御勤政樓,試博通墳典、洞曉玄經、辭藻宏麗、軍謀出眾等舉人,命有司供
 食,既暮而罷。取辭藻宏麗外,別試詩賦各一首。制舉試詩賦,自此始也。時登科者三人,綰為之首,超授右拾遺。



 天寶末,安祿山反,肅宗即位於靈武。綰自賊中冒難,披榛求食,以赴行在。時朝廷方急賢,及綰至,眾心咸悅,拜起居舍人、知制誥。歷司勛員外郎、職方郎中,掌誥如故。遷中書舍人,兼修國史。故事,舍人年深者謂之「閣老」,公廨雜料,歸閣老者五之四。綰以為品秩同列,給受宜均,悉平分之,甚為時論歸美。再遷禮部侍郎,上疏條奏貢
 舉之弊曰:



 國之選士,必藉賢良。蓋取孝友純備,言行敦實,居常育德,動不違仁。體忠信之資,履謙恭之操,藏器則未嘗自伐,虛心而所應必誠。夫如是,故能率己從政,化人鎮俗者也。自叔葉澆詐,茲道浸微,爭尚文辭,互相矜炫。馬卿浮薄,竟不周於任用;趙壹虛誕,終取擯於鄉閭。自時厥後,其道彌盛,不思實行,皆徇空名,敗俗傷教,備載前史,古人比文章於鄭、衛,蓋有由也。



 近煬帝始置進士之科,當時猶試策而已。至高宗朝,劉思立為考功
 員外郎,又奏進士加雜文,明經填帖,從此積弊,浸轉成俗。幼能就學,皆誦當代之詩;長而博文,不越諸家之集。遞相黨與,用致虛聲,《六經》則未嘗開卷,《三史》則皆同掛壁。況復徵以孔門之道,責其君子之儒者哉。祖習既深,奔競為務。矜能者曾無愧色,勇進者但欲凌人,以毀讟為常談,以向背為己任。投刺幹謁,驅馳於要津;露才揚己,喧騰於當代。古之賢良方正,豈有如此者乎!朝之公卿,以此待士,家之長老,以此垂訓。欲其返淳樸,懷禮讓,
 守忠信,識廉隅,何可得也!譬之於水,其流已濁,若不澄本,何當復清。方今聖德御天,再寧寰宇,四海之內,顒顒向化,皆延頸舉踵,思聖朝之理也。不以此時而理之,則太平之政又乖矣。



 凡國之大柄,莫先擇士。自古哲後,皆側席待賢;今之取人,令投牒自舉,非經國之體也。望請依古制,縣令察孝廉,審知其鄉閭有孝友信義廉恥之行,加以經業,才堪策試者,以孝廉為名,薦之於州。刺史當以禮待之,試其所通之學,其通者送名於省。自縣至省,
 不得令舉人輒自陳牒。比來有到狀保辯識牒等,一切並停。其所習經,取《左傳》、《公羊》、《穀梁》、《禮記》、《周禮》、《儀禮》、《尚書》、《毛詩》、《周易》,任通一經,務取深義奧旨,通諸家之義。試日,差諸司有儒學者對問,每經問義十條,問畢對策三道。其策皆問古今理體及當時要務,取堪行用者。其經義並策全通為上第,望付吏部便與官;其經義通八、策通二為中第,與出身;下第罷歸。其明經比試帖經,殊非古義,皆誦帖括,冀圖僥幸。並近有道舉,亦非理國之體,
 望請與明經、進士並停。其國子監舉人,亦請準此。如有行業不著,所由妄相推薦,請量加貶黜。所冀數年之間,人倫一變,既歸實學,當識大猷。居家者必修德業,從政者皆知廉恥,浮競自止,敦龐自勸,教人之本,實在茲焉。事若施行,即別立條例。



 詔左右丞、諸司侍郎、御史大夫、中丞、給、舍同議奏聞。給事中李廣、給事中李棲筠、尚書左丞賈至、京兆尹兼御史大夫嚴武所奏議狀與綰同。尚書左丞至議曰:



 謹按夏之政尚忠,殷之政尚敬,周之
 政尚文,然則文與忠敬,皆統人之行也。且夫謚號述行,美極人文,人文興則忠敬存焉。是故前代以文取士,本文行也,由辭以觀行,則及辭也。宣父稱顏子不遷怒,不貳過,謂之好學。至乎修《春秋》,則游、夏之徒不能措一辭,不亦明乎!間者禮部取人,有乖斯義。《易》曰:「觀乎人文以化成天下。」《關雎》之義曰:「先王以是經夫婦,成孝敬,厚人倫,美教化,移風俗,蓋王政之所由廢興也。」故延陵聽《詩》,知諸侯之存亡。今試學者以帖字為精通,不窮旨義,豈能知
 遷怒貳過之道乎?考文者以聲病為是非,唯擇浮艷,豈能知移風易俗化天下之事乎?是以上失其源而下襲其流,波蕩不知所止,先王之道,莫能行也。夫先王之道消,則小人之道長;小人之道長,則亂臣賊子生焉。臣弒其君,子弒其父,非一朝一夕之故,其所由來者漸矣。漸者何?謂忠信之凌頹,恥尚之失所,末學之馳騁,儒道之不舉,四者皆取士之失也。



 夫一國之事,系一人之本謂之風。贊揚其風,系卿大夫也,卿大夫何嘗不出於士乎?
 今取士試之小道,而不以遠者大者,使干祿之徒,趨馳末術,是誘導之差也。夫以蝸蚓之餌雜垂滄海,而望吞舟之魚,不亦難乎!所以食垂餌者皆小魚,就科目者皆小藝。四人之業,士最關於風化。近代趨仕,靡然向風,致使祿山一呼而四海震蕩,思明再亂而十年不復。向使禮讓之道弘,仁義之道著,則忠臣孝子比屋可封,逆節不得而萌也,人心不得而搖也。



 且夏有天下四百載,禹之道喪而殷始興焉;殷有天下六百祀,湯之法棄而周
 始興焉;周有天下八百年,文、武之政廢而秦始並焉。觀三代之選士任賢,皆考實行,故能風化淳一,運祚長遠。秦坑儒士,二代而亡。漢興,雜三代之政,弘四科之舉,西京始振經術之學,東都終持名節之行。至有近戚竊位,強臣擅權,弱主孤立,母后專政,而社稷不隕,終彼四百,豈非興學行道、扇化於鄉里哉?厥後文章道弊,尚於浮侈,取士術異,茍濟一時。自魏至隋,僅四百載,三光分景,九州阻域,竊號僭位,德義不修,是以子孫速顛,享國咸
 促。國家革魏、晉、梁、隋之弊,承夏、殷、周、漢之業,四隩既宅,九州攸同,覆燾亭育,合德天地。安有舍皇王舉士之道,蹤亂代取人之術?此公卿大夫之辱也。楊綰所奏,實為正論。



 然自典午覆敗,中原版蕩,戎狄亂華,衣冠遷徙,南北分裂,人多僑處。聖朝一平區宇,尚復因循,版圖則張,閭井未設,士居鄉士,百無一二,累緣官族,所在耕築,地望系之數百年之外,而身皆東西南北之人焉。今欲依古制鄉舉里選,猶恐取士之未盡也,請兼廣學校,以
 弘訓誘。今京有太學,州縣有小學,兵革一動,生徒流離,儒臣師氏,祿廩無向。貢士不稱行實,胄子何嘗講習,獨禮部每歲擢甲乙之第,謂弘獎擢,不其謬歟?祗足長浮薄之風,啟僥幸之路矣。其國子博士等,望加員數,厚其祿秩,選通儒碩生,間居其職。十道大郡,量置太學館,令博士出外,兼領郡官,召置生徒。依乎故事,保桑梓者鄉里舉焉,在流寓者庠序推焉。朝而行之,夕見其利。如此則青青不復興刺,擾擾由其歸本矣。人倫之始,王化之先,
 不是過也。



 李暠等議與綰協,文多不載。宰臣等奏以舉人舊業已成,難於速改,其今歲舉人,望且許應舊舉,來歲奉詔,仍敕禮部即具條例奏聞。代宗以廢進士科問翰林學士,對曰:「進士行來已久,遽廢之,恐失人業。」乃詔孝廉與舊舉兼行。綰又奏歲貢孝悌力田及童子科等,其孝悌力田,宜有實狀,童子越眾,不在常科,同之歲貢,恐長僥幸之路。詔停之。再遷吏部侍郎,歷典舉選,精核人物,以公平稱。



 時元載秉政,公卿多附之,綰孤立中道,
 清貞自守,未嘗私謁。載以綰雅望素高,外示尊重,心實疏忌。會魚朝恩死,載以朝恩嘗判國子監事,塵污太學,宜得名儒,以清其秩,乃奏為國子祭酒,實欲以散地處之。載貪冒日甚,天下清議,亦歸於綰,上深知之,以載久在樞衡,未即罷遣。仍遷綰為太常卿,充禮儀使,以郊廟禮久廢,藉綰振起之也,亦以觀其效用。是年三月,載伏誅,上乃拜綰中書侍郎、同中書門下平章事、集賢殿崇文館大學士,兼修國史。綰久積公輔之望,及詔出,朝野
 相賀。綰累表懇讓,上屬意稍重,綰不敢辭。



 綰素以德行著聞,質性貞廉,車服儉樸,居廟堂未數月,人心自化。御史中丞崔寬,劍南西川節度使寧之弟,家富於財,有別墅在皇城之南,池館臺榭,當時第一,寬即日潛遣毀拆。中書令郭子儀在邠州行營,聞綰拜相,座內音樂減散五分之四。京兆尹黎幹以承恩,每出入騶馭百餘,亦即三日減損車騎,唯留十騎而已。其餘望風變奢從儉者,不可勝數,其鎮俗移風若此。



 綰有宿痼疾,居職旬日,中風,
 優詔令就中書省攝養,每引見延英殿,特許扶入。時厘革舊弊,唯綰是瞻,恩遇莫二。綰累抗疏辭位,頻詔敦勉不許。及綰疾亟,上日發中使就第存問,尚書御醫,旦夕在側,上聞其有間,喜見容色。數日而薨,中使在門,馳奏於上,代宗震悼久之,輟朝三日。詔曰:



 王者之於大臣也,存則寄其腹心,均於肢體,參於軍國之重,敘以陰陽之和;歿則誄其事功,加之命數,告於宗廟之祭,襚以紱冕之章,則九原可歸,百闢知勸。故朝議大夫、守中書侍郎、
 同中書門下平章事、集賢殿崇文館大學士、監修國史、上柱國、賜紫金魚袋楊綰,性合元和,身齊律度,道匡雅俗,器重宗彞。寬柔敬恭,協於九德;文行忠信,弘於四教。內無耳目之役,以孝悌傳於家;外無車服之容,以貞實形於代。西掖專宥密之地,南宮領選舉之源。以儒術首於國庠,以禮度掌於高廟,簡廉其質,條職同休。頃以任非其才,毒流於政,爰登清凈之輔,庶諧至理之期。道風既穆於朝班,儉德已行於海內。雖賢人之業,冀於可久;
 而夫子之命,末如之何。方有憑依,遽此淪謝,屏予之嘆,震悼良深。所懷莫從,長想何及。況歷官有素絲之節,居家無匹帛之餘,故飾以華袞,增其法賻,備膺典策,載賁朝經。可贈司徒。



 又詔文武百僚臨於其第,遣內常侍吳承倩會吊,贈絹千匹、布三百端。上深惜之,顧謂朝臣曰:「天不使朕致太平,何奪我楊綰之速也!俯及大斂,與卿等悲悼同之。」宰輔賻贈恩遇哀榮之盛,近年未有其比。太常初謚曰:「文貞」。詔曰:「褒德勸善,《春秋》之舊章;考行易
 名,禮經之通典。垂範作則,存乎格言。朝議大夫、中書侍郎、同中書門下平章事、集賢殿崇文館大學士、修國史、上柱國、賜紫金魚袋、贈司徒楊綰,履道居貞,含和毓德,行為人紀,文合典謨。清而晦名,無自伐之善;約以師儉,有不矜之謙。方冊直書,秩宗相禮,辭稱良史,學茂醇儒。委在樞衡,掌茲密命,彌契沃心之道,累陳造膝之誠。將以布天下五行之和,同君臣一德之運,遽軫藏舟之嘆,未展濟川之才。素業久而彌彰,清風歿而可尚。自古飾
 終之義,皆錫以美名。謚法曰:『忠信愛人曰文,平易不懈曰簡。』宜謚曰文簡。」比部郎中蘇端,性疏狂,嫉其賢,乃肆毀黷,異同其議。上怒,貶端為廣州員外司馬。



 綰儉薄自樂,未嘗留意家產,口不問生計,累任清要,無宅一區,所得俸祿,隨月分給親故。清識過人,至如往哲微言,《五經》奧義,先儒未悟者,綰一覽究其精理。雅尚玄言,宗釋道二教,嘗著《王開先生傳》以見意,文多不載。凡所知友,皆一時名流。或造之者,清談終日,未嘗及名利。或有客欲
 以世務乾者,見綰言必玄遠,不敢發辭,內愧而退。大歷中,德望日崇,天下雅正之士爭趨其門,至有數千里來者。以清德坐鎮雅俗,時比之楊震、邴吉、山濤、謝安之儔也。



 崔祐甫,字貽孫。祖晊,懷州長史。父沔,黃門侍郎,謚曰孝公。家以清儉禮法,為士流之則。祐甫舉進士,歷壽安尉。安祿山陷洛陽,士庶奔迸,祐甫獨崎危於矢石之間,潛入私廟,負木主以竄。歷起居舍人、司勛吏部員外郎,累
 拜御史中丞、永平軍行軍司馬,尋知本軍京師留後。性剛直,無所容受,遇事不回。累遷中書舍人。時中書侍郎闕,祐甫省事,數為宰相常袞所侵,祐甫不從;袞怒之,奏令分知吏部選,每有擬官,袞多駁下,言數相侵。時硃泚上言,隴州將趙貴家貓鼠同乳,不相為害,以為禎祥。詔遣中使以示於朝,袞率百僚慶賀,祐甫獨否。中官詰其故,答曰:「此物之失常也,可吊不可賀。」中使徵其狀,祐甫上奏言:



 臣聞天生萬物,剛柔有性,聖人因之,垂訓作
 則。《禮記郊特牲》曰:「迎貓,為其食田鼠也。」然則貓之食鼠,載在禮典,以其除害利人,雖微必錄。今此貓對鼠不食,仁則仁矣,無乃失於性乎!鼠之為物,晝伏夜動,詩人賦之曰:「相鼠有體,人而無禮。」又曰:「碩鼠碩鼠,無食我黍。」其序曰:「貪而畏人,若大鼠也。」臣旋觀之,雖雲動物,異於麋鹿麝兔,彼皆以時殺獲,為國之用。貓受人養育,職既不修,亦何異於法吏不勤觸邪,疆吏不勤捍敵?又按禮部式具列三瑞,無貓不食鼠之目,以茲稱慶,臣所未詳。
 伏以國家化洽理平,天符洊至,紛綸雜沓,史不絕書。今茲貓鼠,不可濫廁。若以劉向《五行傳》論之,恐須申命憲司,察聽貪吏,誡諸邊候,無失徼巡。貓能致功,鼠不為害。



 代宗深嘉之。袞益惡祐甫。



 代宗初崩,發哀於西宮,袞以獨受任遇,哀逾等禮。例,晨夕臨者,皆十五舉音,而袞輒哀慟涕泗,或中墀返哭,顧慕若不能去,同列者皆不悅。及袞與禮司議群臣喪服,曰:「案《禮》,為君斬衰三年。漢文權制,猶三十六日。國家太宗崩,遺詔亦三十六日,而群臣
 延之,既葬而除,約四月也。高宗崩,服絕輕重,如漢故事,武太后崩亦然。及玄宗、肅宗崩,始變天子喪為二十七日,且當時遺詔雖曰:『天下吏人三日釋服』在朝群臣實服二十七日而除,則朝臣宜如皇帝之制。」祐甫執曰:「伏準遺詔,無朝臣庶人之別,但言『天下人吏,敕到後出臨,三日皆釋服』,則朝野中外,何非天下?凡百執事,誰非吏職?則皇帝宜二十七日而群臣當三日也。」袞曰:「案賀循注義,吏者謂官長所署,則今胥吏耳,非公卿百僚之例。」
 祐甫曰:「《左傳》云:『委之三吏。』則三公也。史稱循吏、良吏者,豈胥徒歟?」袞曰:「禮非天降地出,人情而已。且公卿大臣,榮受殊寵,故宜異數。今與黔首同制,信宿而除之,於爾安乎?」祐甫曰:「若遺詔何?詔旨可改,孰不可?」袞堅諍不服,而聲色甚厲,不為禮節。又袞方哭於鉤陳之前,而袞從吏或扶之,祐甫指示於眾曰:「臣哭於君前,有扶禮乎?」袞聞之,不堪其怒。乃上言祐甫率情變禮,輕議國典,請謫為潮州刺史。內議太重,改為河南少尹。



 初,肅宗時天下事
 殷,而宰相不減三四員,更直掌事。若休沐各在第,有詔旨出入,非大事不欲歷抵諸第,許令直事者一人假署同列之名以進,遂為故事。是時,中書令郭子儀、檢校司空平章事硃泚,名是宰臣,當署制敕,至於密勿之議,則莫得聞。時德宗踐祚未旬日,居不言之際,袞循舊事,代署二人之名進。貶祐甫敕出,子儀及泚皆表明祐甫不當貶謫,上曰:「向言可謫,今言非罪,何也?」二人皆奏實未嘗有可謫之言,德宗大駭,謂袞誣罔。是日,百僚苴絰序
 立於月華門,立貶袞為河南少尹,以祐甫為門下侍郎、平章事,兩換其職。祐甫出至昭應縣,徵還。尋轉中書侍郎,修國史,仍平章事。



 上初即位,庶務皆委宰司。自至德、乾元中,天下多戰伐,啟奏填委,故官賞紊雜。及永泰之後,四方既定,而元載秉政,公道隘塞,官由賄成。中書主書卓英倩、李待榮輩用事,勢傾朝列,天下官爵,大者出元載,小者自倩、榮。四方齎貨賄求官者,道路相屬,靡不稱遂而去,於是綱紀大壞。及元載敗,楊綰尋卒,常袞當
 國,杜絕其門,四方奏請,莫有過者,雖權勢與匹夫等。非以辭賦登科者,莫得進用。雖賄賂稍絕,然無所甄異,故賢愚同滯。及祐甫代袞,薦延推舉,無復疑滯,日除十數人,作相未逾年,凡除吏幾八百員,多稱允當。上嘗謂曰:「有人謗卿所除擬官,多涉親故,何也?」祐甫奏曰:「臣頻奉聖旨,令臣進擬庶官,進擬必須諳其才行。臣若與其相識,方可粗諳,若素不知聞,何由知其言行?獲謗之由,實在於此。」上以為然。



 神策軍使王駕鶴掌禁兵十餘年,權
 傾中外,德宗初登極,將令白琇珪代之,懼其生變。祐甫召駕鶴與語,留連之,琇珪已赴軍視事矣。時李正己畏懼德宗威德,乃表獻錢三十萬貫。上欲納其奏,慮正己未可誠信,以計逗留止之,未有其辭,延問宰相。祐甫對曰:「正己奸詐,誠如聖慮。臣請因使往淄青,便令宣尉將士,因正己所獻錢錫齎諸軍人,且使深荷聖德,又令外籓知朝廷不重財貨。」上悅,從之,正己大慚,而心畏服焉。祐甫謀猷啟沃,多所弘益,天下以為可復貞觀、開
 元之太平也。



 至冬被疾,肩輿入中書,臥而承旨。或休假在第,大事必令中使咨決。薨時年六十,上甚悼惜之,廢朝三日,冊贈太傅,賻布帛米粟有差,謚曰文貞。無子,遺命猶子植為嗣。有文集三十卷。故事,門下侍郎未嘗有贈三師者,德宗以祐甫謇謇有大臣節,故特寵異之。硃泚之亂,祐甫妻王氏陷於賊中,泚以嘗與祐甫同列,雅重其為人,乃遺王氏繒帛菽粟,王氏受而緘封之,及德宗還京,具陳其狀以獻。士君子益重祐甫家法,宜其享
 令名也。



 植字公修,祐甫弟廬江令嬰甫子。植既為相,上言出繼伯父胤,推恩不及於父,詔贈嬰甫吏部侍郎。植潛心經史,尤精《易象》。累歷清要,為給事中,時稱舉職。時皇甫鎛以宰相判度支,請減內外官俸祿,植封還敕書,極諫而止。鎛復奏諸州府鹽院兩稅、榷酒、鹽利、匹段等加估定數,及近年天下所納鹽酒利抬估者一切徵收,詔皆可之。植抗疏論奏,令宰臣召植宣旨嘉諭之,物議罪鎛而美植。尋除御史中丞,入閣彈事,頗振綱紀。



 長慶
 初,拜中書侍郎、同中書門下平章事。穆宗嘗謂侍臣曰:「國家貞觀中,文皇帝躬行帝道,治致昇平。及神龍、景龍之間,繼有內難,玄宗平定,興復不易,而聲明最盛,歷年長久,何道而然?」植對曰:「前代創業之君,多起自人間,知百姓疾苦。初承丕業,皆能厲精思理。太宗文皇帝特稟上聖之資,同符堯、舜之道,是以貞觀一朝,四海寧晏。有房玄齡、杜如晦、魏徵、王珪之屬為輔佐股肱,君明臣忠,事無不理,聖賢相遇,固宜如此。玄宗守文繼體,嘗經天
 后朝艱危,開元初得姚崇、宋璟,委之為政。此二人者,天生俊傑,動必推公,夙夜孜孜,致君於道。璟嘗手寫《尚書·無逸》一篇,為圖以獻。玄宗置之內殿,出入觀省,咸記在心,每嘆古人至言,後代莫及,故任賢戒欲,心歸沖漠。開元之末,因《無逸圖》朽壞,始以山水圖代之。自後既無座右箴規,又信奸臣用事,天寶之世,稍倦於勤,王道於斯缺矣。建中初,德宗皇帝嘗問先臣祐甫開元、天寶治亂之殊,先臣具陳本末。臣在童丱,即聞其說,信知古人以韋、
 弦作戒,其益弘多。陛下既虛心理道,亦望以《無逸》為元龜,則天下幸甚。」穆宗善其對。



 他日,復謂宰臣曰:「前史稱漢文帝惜十家之產而罷露臺。又云身衣弋綈,履革舄,集上書囊以為殿帷,何太儉也!信有此乎?」植對曰:「良史所記,必非妄言。漢興,承亡秦殘酷之後,項氏戰爭之餘,海內凋弊,生人力竭。漢文仁明之主,起自代邸,知稼穡之艱難,是以即位之後,躬行儉約。繼以景帝,猶遵此風。由是海內黔首,咸樂其生,家給戶足。迨至武帝,公私
 殷富,用能出師征伐,威行四方,錢至貫朽,穀至紅腐。上務侈靡,資用復竭,末年稅及舟車六畜,人不聊生,戶口減半,乃下哀痛之詔,封丞相為富人侯。皆漢史明徵,用為事實。且耕蠶之勸,出自人力,用既無度,何由以至富強!據武帝嗣位之初,物力阜殷,前代無比,固當因文帝儉約之致也。」上曰:「卿言甚善,患行之為難耳。」



 憲宗皇帝削平群盜,河朔三鎮復入提封。長慶初,幽州節度使劉總表以幽、薊七州上獻,請朝廷命帥。總仍懼部將構亂,乃
 籍其豪銳者先送京師。時硃克融在籍中。植與同列杜元潁素不知兵,且無遠慮。克融等在京羈旅窮餓,日詣中書乞官,殊不介意。及張弘靖赴鎮,令克融等從還。不數月,克融囚弘靖,害賓佐,結王廷湊,國家復失河朔,職植兄弟之由。乃罷知政事,守刑部尚書,出為華州刺史。大和三年正月卒,年五十八。植雖器量謹厚,而無開物成務之才,及喪師異方,天下尤其失策。



 倰,字德長。祖濤,大理卿孝公沔之弟也。濤生儀甫,終大理丞,即俊之父。
 以門廕由太廟齋郎調授太平、東陽二主簿。李衡廉察湖南、江西,闢為賓佐,坐事沉廢。久之,復以選授宣州錄事參軍。觀察使崔衍奇其才,奏加章服,倰辭而不受。李巽鎮江西,奏為副使,得監察裏行,又從巽領使,為河陰院鹽鐵留後。入為侍御史,尋改膳部員外,充轉運判官。入為膳部郎中,充荊襄十道兩稅使,賜金紫。遷蘇州刺史,理行為第一。轉潭州刺史、湖南都團練觀察使。湖南舊法,豐年貿易不出境,鄰部災荒不相恤。倰至,謂屬吏
 曰:「此非人情也,無宜閉糶,重困於民也。」自是商賈通流。入為戶部侍郎、判度支。



 時倰再從弟植為宰相,倰性剛褊,恃其權寵,與奪任情。時朝廷以王承元歸國,命田弘正移帥鎮州。弘正之行,以魏卒二千為帳下,又以常山之人久隔朝化,人情易為變擾,累表請留魏卒為綱紀,其糧賜請度支歲給。穆宗下宰臣議,倰固言魏、鎮各有鎮兵,朝廷無例支給,恐為事例,不可聽從。弘正不獲已,遣魏卒還籓,不數日而鎮州亂,弘正遇害。穆宗失德,倰
 黨方盛,人不敢糾其罪。罷領度支,檢校禮部尚書,出為鳳翔節度等使。不期歲,召為河南尹,時年七十,抗疏致仕,詔以戶部尚書歸第。明年暴卒,輟朝一日,贈太子少保,謚曰肅。倰居官清嚴,所至必理,然性介急,待僚屬不以禮節,恃己之廉,見贓污者如仇焉。



 子巖,登進士第,闢襄陽掌書記、監察御史,方雅有父風。



 常袞,京兆人也。父無為,三原縣丞,以袞累贈僕射。袞,天寶末舉進士,歷太子正字,累授補闕、起居郎。寶應二年,
 選為翰林學士、考功員外郎中、知制誥,依前翰林學士。永泰元年,遷中書舍人。袞文章俊拔,當時推重,與楊炎同為舍人,時稱為常、楊。性清直孤潔,不妄交游。內侍魚朝恩恃權寵,兼領國子監事,袞上疏以為不可。時朝廷多事,西北邊虜,連為寇盜,袞累上章陳其利害,代宗甚顧遇之,加集賢院學士。大歷元年,遷禮部侍郎,仍為學士。時中官劉忠翼權傾內外,涇原節度馬璘又累著功勛,恩寵莫二,各有親戚乾貢部及求為兩館生,袞皆執
 理,人皆畏之。



 元載之得罪,令袞與劉晏、李涵等鞫之,獄竟,拜袞門下侍郎、同平章事,太清、太微宮使,崇文、弘文館大學士,與楊綰同掌樞務。代宗尤信重綰。綰弘通多可,袞頗務苛細,求清儉之稱,與綰之道不同。先是,百官俸料寡薄,綰與袞奏請加之。時韓滉判度支,袞與滉各騁私懷,所加俸料,厚薄由己。時少列各定月俸為三十五千,滉怒司業張參,唯止給三十千;袞惡少詹事趙期,遂給二十五千。太子洗馬,實司經局長官,文學為之
 貳。袞有親戚任文學者給十二千,而給洗馬十千。其輕重任情,不通時政,多如此類。



 無幾,楊綰卒,袞獨當政。故事,每日出內廚食以賜宰相,饌可食十數人,袞特請罷之,迄今便為故事。又將故讓堂封,同列以為不可而止。議者以為厚祿重賜,所以優賢崇國政也,不能,當辭位,不宜辭祿食。政事堂有後門,蓋宰相時到中書舍人院,咨訪政事,以自廣也,袞又塞絕其門,以示尊大,不相往來。既懲元載為政時公道梗澀,賄賂朋黨大行,不以財勢
 者無因入仕。袞一切杜絕之。中外百司奏請,皆執不與,權與匹夫等,尤排擯非文辭登科第者。雖窒賣官之路,政事大致壅滯。



 代宗既素重楊綰,欲以政事委之。綰尋卒,袞與綰志尚素異,嫉而怒之。有司議謚綰為文貞,袞微諷比部郎中蘇端令駁之,毀綰過甚,端坐黜官。時既無中書侍郎,舍人崔祐甫領省事,袞以為同中書門下平章事兼得總中書省,遂管綜中書胥吏、省事去就及其案牘,祐甫不能平之,累至忿競。遂令祐甫分知吏部選
 事,所擬官又多駁下。時袞散官尚朝議,又無封爵,郭子儀因入朝奏之,遂特加銀青光祿大夫,封河內郡公。及代宗崩,與祐甫爭論喪服輕重,代相署奏。初換祐甫河南少尹,再貶為潮州刺史。楊炎入相,素與袞善,建中元年,遷福建觀察使。四年正月卒,時年五十五。久之。贈左僕射。有文集六十卷。



 史臣曰:善人為邦百年,即可勝殘去殺,楊綰入相數日,遽致移風易俗。周、召、伊、傅,蕭、張、房、杜,歷代為相之顯者,
 蔑聞斯道也。嘗讀諸集,賞善多溢美,書罪多溢惡;如楊綰拜相之麻,贈官之制,改謚之詔,則當時秉筆者無愧色矣。昔趙文子薦士七十,古為美談;崔祐甫除吏八百,人無間言。開物成務之才,滅私徇公之道可知也。噫!公權餘旬日而薨,貽孫未期年而逝,邃古已來,理世少而亂世多,其義在茲矣。常袞之輩,不足云爾。



 贊曰:公權儒道,貽孫相才。命乎不永,時哉可哀。



\end{pinyinscope}