\article{卷一百四十二}

\begin{pinyinscope}

 ○趙憬
 韋
 倫賈耽姜公輔



 趙憬,字退翁,天水隴西人也。總章中吏部侍郎、同東西臺三品仁本之曾孫。祖諠歷左司郎中。父道先,洪州錄事參軍。



 憬少好學,志行修潔,不求聞達。寶應中,玄宗、肅
 宗梓宮未祔,有司議山陵制度。時西蕃入寇,天下饑饉,憬以褐衣上疏,宜遵儉制,時人稱之。後連為州從事,試江夏尉。累遷監察御史,隨牒籓府,歷殿中侍御史、太子舍人。居母憂,哀毀幾絕。服除,建中初,擢授水部員外郎。未拜,會湖南觀察使李承請為副使、檢校工部郎中,充職。歲餘,承卒,遂知留後事。尋授潭州刺史、兼御史中丞、湖南觀察使,仍賜金紫。居二歲,受代歸京師,闔門靜居,不與人交。久之,特召對於別殿。憬多學問,有辭辯,敷奏
 稱旨,上悅,拜給事中。



 貞元四年,回紇請結和親。詔以咸安公主降回紇,命檢校右僕射關播充使。憬以本官兼御史中丞為副。前後使回紇者,多私齎繒絮,蕃中市馬回以規利。憬一無所市,人嘆美之。使還,遷尚書左丞,綱轄省務,清勤奉職。竇參為宰相,惡其能,請出為同州刺史,上不從。



 八年四月,竇參罷黜,憬與陸贄並拜中書侍郎、同中書門下平章事。憬深於理道,常言:「為政之本,在於選賢能,務節儉,薄賦斂,寬刑罰。」對揚之際,必以此為
 言,乃獻《審官六議》曰:



 臣謬登宰府,四年於茲,恭承德音,未嘗不以求賢為切。至於延薦,職在愚臣,雖當代天之工,且乏知人之鑒;漸積歲月,負於聖明,無補王猷,有妨賢路。況多疾恙,兼慮闕遺,頃奉表章,備陳肝膈。陛下以臣性拙直,身病可矜,不棄孱微,尚加委任。自此思省,報效尤難,莫副堯、舜之心,空懷尸素之懼。伏惟陛下法象應期,聖神廣運,雲行雨施,皆發自然,訓誥典謨,悉經睿覽。臣所以不敢援引古昔,上煩天聰,且以用人之要,願
 伸鄙見。復念稽顙丹陛,仰對宸嚴,謇訥易窮,遽數難辯,理詳則塵瀆頗甚,言略則利害未宣。若默以求容,茍而竊位,縱天地之仁幸免,而中外之責何逃!非陛下用臣之意也。其所欲言者,皆陛下聖慮之內。臣以頂戴恩造,不知所為,身被風毒,漸覺沉痼,是以勤勤懇懇,切於愚誠也。



 臣聞貞觀、開元之際,宰輔論事,或多上書,所冀獲盡情理。今臣酌前代之損益,體當時之通變,謹獻《審官六議》,伏惟閑宴時賜省覽。



 其大指,議相,則曰:「宜博採眾
 賢,用為輔弼。今中外知其賢者,伏願陛下用之,識其能者任之,求其全材,恐不可得。」



 議進用庶官,則曰:「異同之論,是非難辨。由考課難於實效,好惡雜於眾聲,所以訪之彌多,得之彌少。選士古今為難,拔十得五,賢愚猶半。陛下謂臣曰:『何必五也?十得二三斯可矣!』聖主思賢至是,而宰臣不能進之,臣之罪也。進賢在於廣任用,明殿最,舉大節,棄其小瑕,隨其所能,試之以事,用人之大綱也。」



 議京諸司闕官,則曰:「當今要官多闕,閑官十無一二。
 文武任用,資序遞遷,要官本以材行,閑官多由恩澤。朝廷或將任,多擬要官則人少闕多,閑官則人多闕少;明當選拔者轉少,在優容者轉多,宜補闕員,務育材用。大廈永固,是棟梁榱桷之全也;聖朝致理,亦庶官群吏之能也。」



 議中外考課官,則曰:「漢以數易長吏,謂之弊政。其有能理者,輒增秩賜金,或八九年、十餘年,乃入為九卿,或遷三輔。功績茂異,遂至丞相,其間不隔數官。今陛下內選庶僚,外委州府,課績高者,不次超升,致理之法,無
 逾於此。臣愚以為黜陟且立年限,若所居要重,未當遷移,就加爵秩。其餘進退,令知褒貶之必應,遲速之有常。如課績在中,年考及限,與之平轉。中外迭處,歷試其能,使無茍且之心,又無滯淹之慮。」



 議舉遺滯,則曰:「官司既廣,必委宰輔以舉之;宰輔不能遍知,又詢於庶官;庶官不能遍知,又訪於眾人。眾聲囂然,互有臧否,十人舉之未信,一人毀之可疑,迨至於今,茲弊未改。其所以然者,非盡為愛憎也,苦於不審實而承聲言之。大凡常人之
 心,以稱人之善為清,以攻人之過為直,茍有除授,多生橫議。由是宰臣每將薦用,亦自重難,日往月來,未副聖意。宜須採聽時論,以所舉多者先用,必非大故,皆不棄之。」



 議擢用諸使府僚屬,則曰:「諸使闢吏,各自精求,務於得人,將重府望。既經試效,能否可知,擢其賢能,置之朝列。或曰外使須才,固不可奪。臣知必不然也。屬者使府賓介,每有登朝,本使殊以為榮,自喜知人,且明公選。大凡才能之士,名位未達,多在方鎮。日月在上,誰不知之,
 思登闕庭,如望霄漢,宜須博採,無宜久滯。」上優詔答之。



 時吏部侍郎杜黃裳為中貴讒譖及他過犯,御史中丞穆贊、京兆少尹韋武、萬年縣令李宣、長安令盧云,皆為裴延齡構陷,將加斥逐。憬保護救解之,故多從輕貶。



 初,憬廉察湖南,令狐峘、崔儆並為巡屬刺史。峘嘗歷中書舍人、禮部侍郎,儆久在朝列,所為或虧法令,憬每以正道制之。峘、儆密遣人數憬罪狀,毀之於朝。及憬為相,拔儆自大理卿為尚書右丞,峘先貶官為別駕,又擢為吉
 州刺史,時人多之。



 憬與陸贄同知政事。贄恃久在禁庭,特承恩顧,以國政為己任,才周歲,轉憬為門下侍郎。憬由是深銜之,數以目疾請告,不甚當政事,因是不相協。裴延齡奸詐恣睢,滿朝側目。憬初與贄約於上前論之,及延英奏對,贄極言延齡奸邪誑誕之狀,不可任用。德宗不悅,形於顏色。憬默然無言,由是罷贄平章事,而憬當國矣。



 時宰相賈耽、盧邁與憬三人。十二年春正月,耽、邁皆有假,故憬獨對於延英。上問曰:「近日起居注記何
 事?」憬對曰:「古者左史記言,人君動止,有實言隨即記錄,起居注是也。國朝永徽中,起居唯得對仗承旨,仗下後謀議皆不得聞,其記注唯編制敕,更無他事。所以長壽中姚璹知政事,以為親承德音謨訓,若不宣旨,宰相、史官無以得書。璹請宰相一人記錄所論軍國政事,謂之時政記,每月送史館。既而時政記又廢。」上曰:「君舉必書,義存勸誡。既嘗有時政記,宰臣宜依故事為之。」無何,憬卒,時政記亦不行。



 憬特承恩顧,性清儉,雖為宰輔,居第
 僕使,類貧士大夫之家,所得俸入,先置私廟,而竟不立第舍田產。



 其年八月,遇暴疾,信宿而卒,時年六十一。子元亮進憬遺表草曰:「臣叨荷聖慈,竊塵臺鼎,年序頗久,績用無聞,負乘之敗已彰,覆餗之咎俄及。而天與之疾,福過生災,自今日卯時以來,稍加困重,針灸不及,藥餌奚施。奄然游魂,終當就木,冥冥殘喘,豈忍辭天!號呼涕零,側息心斷,反風結草,誓報深恩,雖死猶生,豈孤素願。無任感恩,嗚咽痛恨之至。」德宗尤悼惜之,廢朝三日,冊
 贈太子太傅,賻帛五百端、米粟四百石,令鴻臚卿王權充冊吊使。



 元亮官至左司郎中、侍御史知雜事卒。次子全亮,官至侍御史、桂管防禦判官。元亮兄宣亮、弟承亮,皆以門廕授官。



 韋倫,開元、天寶中朔方節度使光乘之子。少以廕累授藍田縣尉。以吏事勤恪,楊國忠署為鑄錢內作使判官。國忠恃權寵,又邀名稱,多徵諸州縣農人令鑄錢。農夫既非本色工匠,被所由抑令就役,多遭箠罰,人不聊生。
 倫白國忠曰:「鑄錢須得本色人,今抑百姓農人為之,尤費力無功,人且興謗。請厚懸市估價,募工曉者為之。」由是役使減少,而益鑄錢之數。天寶末,宮內土木之功無虛日,內作人吏因緣為奸,倫乃躬親閱視,省費減倍。改大理評事。



 會安祿山反,車駕幸蜀,拜倫監察御史、劍南節度行軍司馬,兼充置頓使判官,尋改屯田員外、兼侍御史。時內官禁軍相次到蜀,所在侵暴,號為難理;倫清儉,率身以化之,蜀川咸賴其理。竟遭中官毀譖,貶衡州
 司戶。屬東都、河南並陷賊,漕運路絕,度支使第五琦薦倫有理能,拜商州刺史,充荊襄等道租庸使。會襄州裨將康楚元、張嘉延聚眾為叛,兇黨萬餘人,自稱東楚義王。襄州刺史王政棄城遁走。嘉延又南襲破江陵,漢、沔饋運阻絕,朝廷旰食。倫乃調發兵甲駐鄧州界,兇黨有來降者,必厚加接待。數日後,楚元眾頗怠,倫進軍擊之。生擒楚元以獻,餘眾悉走散,收租庸錢物僅二百萬貫,並不失墜。荊、襄二州平。詔除崔光遠為襄州節度使,徵
 倫為衛尉卿。旬日,又以本官兼寧州刺史、招討處置等使,尋又兼隴州刺史。



 乾元三年,襄州大將張瑾殺節度使史翽作亂,乃以倫為襄州刺史、兼御史大夫、山南東道襄鄧等十州節度使。時李輔國秉權用事,節將除拜,皆出其門。倫既為朝廷公用,又不私謁輔國。倫受命未行,改秦州刺史、兼御史中丞、本州防禦使。時吐蕃、黨項歲歲入寇,邊將奔命不暇。倫至秦州,屢與虜戰。兵寡無援,頻致敗衄,連貶巴州長史、思州務川縣尉。



 代宗即位,
 起為忠州刺史,歷臺、饒二州。以中官呂太一於嶺南矯詔募兵為亂,乃以倫為韶州刺史、兼御史中丞、韶連柳三州都團練使。竟遭太一用賂反間,貶信州司馬、虔州司戶、隋州司戶、隨州司馬。遇赦,旅寓於洪州十數年。



 德宗即位,選堪使絕域者,徵倫拜太常少卿、兼御史中丞,持節充通和吐蕃使。倫至蕃中,初宣諭皇恩,次述國威德遠振,蕃人大悅,贊普入獻方物。使還,遷太常卿、兼御史大夫,加銀青光祿大夫。再入吐蕃,奉使稱旨,西蕃敬
 服。朝廷得失,數上疏言之。又為宰相盧杞所惡,改太子少保,累加開府儀同三司。涇師之亂,駕幸奉天。及盧杞、白志貞、趙贊等貶官,關播罷相為刑部尚書,倫於朝堂嗚咽而言曰:「宰相不能弼諧啟沃,使天下一至於此。仍為尚書,天下何由致理?」聞者敬憚之。從駕梁州,還京,又欲擢用盧杞為饒州刺史。倫又上表切言不可,深為忠正之士所稱嘆。以年逾七十,表請休官,改太子少師致仕,封郢國公。時李楚琳以僕射兼衛尉卿,李忠誠以尚
 書兼少府監,倫上言曰:「楚琳兇逆,忠誠蕃戎醜類,不合廁列清班。」又表請置義倉以防水旱,擇賢良任之左右。又言吐蕃必無信約,專須防備,不可輕易。上每善遇之。



 倫居家孝友,撫弟侄以慈愛稱。貞元十四年十二月卒,時年八十三,贈揚州都督。



 賈耽,字敦詩,滄州南皮人。以兩經登第,調授貝州臨清縣尉。上疏論時政,授絳州正平尉。從事河東,檢校膳部員外郎、太原少尹、北都副留守。又檢校禮部郎中、節度
 副使,改汾州刺史。在郡七年,政績茂異。入為鴻臚卿,時左右威遠營隸鴻臚,耽仍領其使。大歷十四年十一月,檢校左散騎常侍、兼梁州刺史、御史大夫、山南西道節度使。



 建中三年十一月,檢校工部尚書、兼御史大夫、山南東道節度使。德宗移幸梁州。興元元年二月,耽使行軍司馬樊澤奏事於行在,澤既復命,方大宴諸將,有急牒至,言澤代耽為節度使,而召耽為工部尚書。耽得牒內懷中,宴飲不改容。及散,召樊澤,以詔授之曰:「詔以行
 軍為節度使,耽今即上路。」因告將吏使謁澤。牙將張獻甫曰:「天子巡幸山南,尚書使行軍奉表起居,而行軍敢自圖節鉞,潛奪尚書土地,此可謂事人不忠。軍中皆不伏,請殺樊澤。」耽曰:「公是何言歟!天子有命,即為節度使矣。耽今赴行在,便與公偕行。」即日離鎮,以獻甫自隨,軍中乃安。尋以本官為東都留守、東畿汝南防禦使。



 貞元二年,改檢校右僕射、兼滑州刺史、義成軍節度使。是時淄青節度使李納雖去偽王號,外奉朝旨,而心常蓄並
 吞之謀。納兵士數千人自行營歸,路由滑州,大將請城外館之。耽曰:「與人鄰道,奈何野處其兵?」命館之城內,淄青將士皆心服之。耽善射好獵,每出畋不過百騎,往往獵於李納之境。納聞之,大喜,心畏其度量,不敢異圖。九年,徵為右僕射、同中書門下平章事。



 耽好地理學,凡四夷之使及使四夷還者,必與之從容,訊其山川土地之終始。是以九州之夷險,百蠻之土俗,區分指畫,備究源流。自吐蕃陷隴右積年,國家守於內地,舊時鎮戍,不可
 復知。耽乃畫隴右、山南圖,兼黃河經界遠近,聚其說為書十卷,表獻曰:



 臣聞楚左史倚相能讀《九丘》,晉司空裴秀創為六體;《九丘》乃成賦之古經,六體則為圖之新意。臣雖愚昧,夙嘗師範,累蒙拔擢,遂忝臺司。雖歷踐職任,誠多曠闕,而率土山川,不忘寤寐。其大圖外薄四海,內別九州,必藉精詳,乃可摹寫,見更纘集,續冀畢功。然而隴右一隅,久淪蕃寇,職方失其圖記,境土難以區分。輒扣課虛微,採掇輿議,畫《關中隴右及山南九州等圖》一
 軸。伏以洮、湟舊墟,連接監牧;甘、涼右地,控帶朔陲。岐路之偵候交通,軍鎮之備御沖要,莫不匠意就實,依稀像真。如聖恩遣將護邊,新書授律,則靈、慶之設險在目,原、會之封略可知。諸州諸軍,須論里數人額;諸山諸水,須言首尾源流。圖上不可備書,憑據必資記注,謹撰《別錄》六卷。又黃河為四瀆之宗,西戎乃群羌之帥,臣並研尋史牒,翦棄浮詞,罄所聞知,編為四卷,通錄都成十卷。文義鄙樸,伏增慚悚。



 德宗覽之稱善,賜廄馬一匹、銀採百
 匹、銀瓶盤各一。



 至十七年,又撰成《海內華夷圖》及《古今郡國縣道四夷述》四十卷,表獻之,曰:



 臣聞地以博厚載物,萬國棋布;海以委輸環外,百蠻繡錯。中夏則五服、九州,殊俗則七戎、六狄,普天之下,莫非王臣。昔毋丘出師,東銘不耐;甘英奉使,西抵條支;奄蔡乃大澤無涯,罽賓則懸度作險。或道理回遠,或名號改移,古來通儒,罕遍詳究。臣弱冠之歲,好聞方言,筮仕之辰,注意地理,究觀研考,垂三十年。絕域之比鄰,異蕃之習俗,梯山獻琛之
 路,乘舶來朝之人,咸究竟其源流,訪求其居處。闤闠之行賈,戎貊之遺老,莫不聽其言而掇其要。閭閻之瑣語,風謠之小說,亦收其是而芟其偽。



 然殷、周以降,封略益明,承歷數者八家,渾區宇者五姓,聲教所及,惟唐為大。秦皇罷侯置守,長城起於臨洮;孝武卻地開邊,障塞限於雞鹿;東漢則哀牢請吏;西晉則裨離結轍;隋室列四郡於卑和海西,創三州於扶南江北,遼陽失律,因而棄之。高祖神堯皇帝誕膺天命,奄有四方。太宗繼明重熙,
 柔遠能邇,逾大磧通道,北至仙娥,於骨利幹置玄闕州。高宗嗣守丕績,克廣前烈,遣單車齎詔,西越蔥山,於波刺斯立疾陵府。中宗復配天之業,不失舊物。睿宗含先天之量,惟新永圖。玄宗以大孝清內,以無為理外,大宛驥錄,歲充內廄,與貳師之窮兵黷武,豈同年哉!肅宗掃平氛昆,潤澤生人。代宗刬除殘孽,彞倫攸敘。



 伏惟皇帝陛下,以上聖之姿,當太平之運,敦信明義,履信包元,惠養黎蒸,懷柔遐裔。故瀘南貢麗水之金,漠北獻餘吾之
 馬,玄化洋溢,率士沾濡。



 臣幼切磋於師友,長趨侍於軒墀,自揣孱愚,叨榮非據,鴻私莫答,夙夜兢惶。去興元元年,伏奉進止,令臣修撰國圖,旋即充使魏州、汴州,出鎮東洛、東都,間以眾務,不遂專門,績用尚虧,憂愧彌切。近乃力竭衰病,思殫所聞見,叢於丹青。謹令工人畫《海內華夷圖》一軸,廣三丈,從三丈三尺,率以一寸折成百里。別章甫左衽,奠高山大川。縮四極於纖縞,分百郡於作繢。宇宙雖廣,舒之不盈庭;舟車所通,覽之咸在目。並撰《
 古今郡國縣道四夷述》四十卷,中國以《禹貢》為首,外夷以《班史》發源;郡縣紀其增減,蕃落敘其衰盛。前地理書以黔州屬酉陽,今則改入巴郡;前西戎志以安國為安息,今則改入康居。凡諸疏舛,悉從厘正。隴西、十地,播棄於永初之中;遼東、樂浪,陷屈於建安之際。曹公棄陘北,晉氏遷江南,緣邊累經侵盜,故墟日致堙毀。舊史撰錄,十得二三,今書搜補,所獲太半。《周禮職方》,以淄、時為幽州之浸,以華山為荊河之鎮,既有乖於《禹貢》,又不出於
 淹中,多聞闕疑,詎敢編次。其古郡國題以墨,今州縣題以硃,今古殊文,執習簡易。臣學謝小成,才非博物。伏波之聚米,開示眾軍;酂侯之圖書,方知厄塞。企慕前哲,嘗所寄心,輒罄庸陋,多慚紕繆。



 優詔答之,賜錦彩二百匹、袍段六、錦帳二、銀瓶盤各一、銀榼二、馬一匹,進封魏國公。



 順宗即位,檢校司空,守左僕射,知政事如故。時王叔文用事,政出群小,耽惡其亂政,屢移病乞骸,不許。耽性長者,不喜臧否人物。自居相位,凡十三年,雖不能以安
 危大計啟沃於人主,而常以檢身厲行以律人。每自朝歸第,接對賓客,終日無倦。至於家人近習,未嘗見其喜慍之色,古之淳德君子,何以加焉!



 永貞元年十月卒,時年七十六。廢朝四日,冊贈太傅,謚曰元靖。



 姜公輔,不知何許人。登進士第,為校書郎。應制策科高等,授左拾遺,召入翰林為學士。歲滿當改官,公輔上書自陳,以母老家貧,以府掾俸給稍優,乃求兼京兆尹戶曹參軍,特承恩顧。才高有器識,每對見言事,德宗多從
 之。



 建中四年十月,涇師犯闕。德宗蒼黃自苑北便門出幸,公輔馬前諫曰:「硃泚嘗為涇原帥,得士心。昨以硃滔叛,坐奪兵權,泚常憂憤不得志。不如使人捕之,使陪鑾駕,忽群兇立之,必貽國患。臣頃曾陳奏,陛下茍不能坦懷待之,則殺之,養獸自貽其患,悔且無益。」德宗曰:「已無及矣!」從幸至奉天,拜諫議大夫,俄以本官同中書門下平章事。



 從幸山南,車駕至城固縣,唐安公主薨。上之長女,昭德皇后所生,性聰敏仁孝,上所鐘愛。初,詔尚韋宥,
 未克禮會而遇播遷;及薨,上悲悼尤甚,詔所司厚其葬禮。公輔諫曰:「非久克復京城,公主必須歸葬,今於行路,且宜儉薄,以濟軍士。」德宗怒,謂翰林學士陸贄曰:「唐安夭亡,不欲於此為塋壟,宜令造一磚塔安置,功費甚微,不合關宰相論列。姜公輔忽進表章,都無道理,但欲指朕過失,擬自取名。朕比擢拔為腹心,乃負朕如此!」贄對曰:「公輔官是諫議,職居宰衡,獻替固其職分。本立輔臣,置之左右,朝夕納誨,意在防微,微而弼之,乃其所也。陛
 下以造塔役費微小,非宰相所論之事。但問理之是非,豈論事之大小!若造塔為是,役雖大而作之何傷!若造塔為非,費雖小而言者何罪!」帝又曰:「卿未會朕意。朕以公輔才行,共宰相都不相當,在奉天時已欲罷免,後因公輔辭退,朕已面許。尋屬懷光背叛,遂且因循,容至山南。公輔知朕擬改官,所以固論造塔,賣直取名。據此用心,豈是良善!朕所惆悵者,只緣如此。」贄再三救護,帝怒不已,乃罷為左庶子。尋丁母憂,服闕,授右庶子,久之不
 遷。



 洎陸贄知政事,以有翰林之舊,數告贄求官。贄密謂公輔曰:「予嘗見郴州竇相,言為公奏擬數矣,上旨不允,有怒公之言。」公輔恐懼,上疏乞罷官為道士,久之未報。後又廷奏,德宗問其故,公輔不敢洩贄,便以參言為對。帝怒,貶公輔為泉州別駕,又遣中使齎詔責竇參。順宗即位,起為吉州刺史,尋卒。憲宗朝,贈禮部尚書。



 史臣曰:賈魏公以溫克長者,致位丞相,拒獻甫之請,畋李納之郊,則器略可知矣!韋郢公慷慨節義,困於讒邪,
 命矣夫!趙丞相區分檢裁,求為雅士,以爭權而陷陸贄,則前時以德報怨,其可信乎!公輔一言悟主,驟及臺司;一言不合,禮遽疏薄,則加膝墜泉之間,君道可知矣!



 贊曰:元靖訏謨,真謂純儒。手調鼎飪,心運地圖。姜躁趙險,並躍天衢。哀哉韋公,終困讒夫。



\end{pinyinscope}