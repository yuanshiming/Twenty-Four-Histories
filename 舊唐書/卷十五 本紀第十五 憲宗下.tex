\article{卷十五 本紀第十五 憲宗下}

\begin{pinyinscope}

 元
 和
 七年春正月辛酉朔,己巳,以刑部尚書趙宗儒檢校吏部尚書、興元尹、山南西道節度使。庚午,以兵部尚書王紹判戶部事。辛未,以京兆尹元義方為鄜州刺史、
 鄜坊丹延觀察使,以司農卿李銛為京兆尹。是夜,月掩熒惑。壬申,廢信州永豐縣、越州山陰縣、衢州盈川縣。癸酉,振武河溢,毀東受降城。



 二月庚寅朔。壬辰,詔以去秋旱歉,賑京畿粟三十萬石;其元和六年春賑貸百姓粟二十四萬石,並宜放免。辛丑,尚書省重定左、右僕射上事儀注。壬寅,以兵部侍郎許孟容為河南尹。辛亥,山南西道節度使裴玢卒。癸丑,入蕃使不得與私覿正員官,量別支給以充私覿。舊使絕域者,許賣正員官十餘員,
 取貨以備私覿,雖優假遠使,殊非典法,故革之。敕:「錢重物輕,為弊頗甚,詳求適變,將以便人。所貴緡貨通行,裏閭寬恤。宜令群臣各隨所見利害狀以聞。」三月己未。辛酉,以惠昭太子葬,罷曲江上巳宴。庚午,以旱,敕諸司疏決系囚。



 夏四月戊子朔。癸巳,敕天下州府民戶,每田畝,種桑二樹,長吏逐年檢計以聞。辛亥,鹽鐵使王播奏元和六年賣鹽鐵,除峽內井鹽外,計收六百八十五萬九千二百貫。



 五月戊午朔。庚申,上謂宰臣曰:「卿等累
 言吳越去年水旱,昨有御史自江淮回,言不至為災,人非甚困。」李絳對曰:「臣得兩浙、淮南狀,繼言歉旱。方隅授任,皆朝廷信重之臣。御史非良,或容希媚,此正當奸佞之臣。況推誠之道,君人大本,任大臣以事,不可以小臣言間之。伏望明示御史姓名,正之典刑。」上曰:「卿言是也。朝廷大體,以恤人為本,一方不稔,即宜賑救,濟其饑寒,況可疑之也!向者不思而有此問,朕言過矣。」絳等拜賀。癸亥,熒惑近太微右執法。



 六月丁亥朔,舒州桐城梅天
 陂內,有黃白二龍,自陂中乘風雷躍起,高二百尺,行六里,入浮塘陂。癸巳,以金紫光錄大夫、守司徒、同平章事、崇文館大學士,太清宮使、上柱國、岐國公杜佑為光祿大夫,守太保致仕,宜朝朔望,佑累表懇請故也。己亥,月近南斗魁第四星。鎮州甲仗庫一十三間災,兵仗都盡。王承宗常蓄叛謀,至是始懼天罰,兇氣稍奪,仍殺主庫吏百餘人。乙丑,以兵部員外郎王涯知制誥。乙亥,制立遂王宥為皇太了,改名恆。己卯,以新羅大宰相金彥升
 為開府儀同三司、檢校太尉、使持節、大都督雞林州諸軍事、雞林州刺史,兼寧海軍使、上柱國,封新羅國王;仍冊彥升妻貞氏為妃。



 八月丁亥朔,新除新羅國大宰相金崇斌等三人,宜令本國準例賜戟。戊戌,魏博節度使田季安卒。辛丑,廢蓬州宕渠縣。甲辰,宣歙觀察使房式卒。丙午,以蘇州刺史範傳正為宣歙觀察使。戊申,制:「諸州府五品已上官替後,委本道長官量其才行、官業、資歷,每年冬季一度聞薦。其罷使郎官、御史,許朝臣每年
 冬季準此聞薦,諸使府參佐、檢校官,從元授官月日計,如是五品已上官及臺省官,經三十個月外,任與轉改;餘官經三十六個月奏轉改。如未經考便有事故及停替官,本限之外更加十個月,即任申奏。」辛亥,以左龍武大將軍薛平為滑州刺史、義成軍節度使。



 冬十月乙未,魏博三軍舉其衙將田興知軍州事。時田季安死,子懷諫年十一,為副大使、知軍府事,軍政一決於家僮蔣士則,數易大將,軍情不安。因田興入衙,兵環而劫請,興頓
 僕於地,軍眾不散。興曰:「欲聽吾命,勿犯副大使。」眾曰:「諾。」但殺蔣士則等十數人而止。即日移懷諫於外,令朝京師。甲辰,以魏博都知兵馬使、兼御史中丞、沂國公田興為銀青光祿大夫、檢校工部尚書,兼魏州大都督府長史,充魏博節度使。庚戌,澧王寬改名惲,深王察改名忭,洋王寰改名忻,絳王寮改名悟,建王審改名恪。以鄭滑節度使袁滋為戶部尚書。



 十一月丙辰朔。乙丑,詔:「田興以魏博請命,宜令司封郎中、知制誥裴度往彼宣慰,賜
 三軍賞錢一百五十萬貫,以河陰院諸道合進內庫物充。六州諸縣宣達朝旨。辛未,太保致仕杜祐卒。東川觀察使潘孟陽奏龍州武安縣嘉禾生,有麟食之。麟食來,群鹿環之,光彩不可正視。使畫工圖之以獻。乙亥,以給事中李逢吉、司勛員外郎李巨並充皇太子諸王侍讀。戊寅,吏部尚書鄭餘慶請復置吏部考官三員,吏部郎中楊於陵執奏以為不便。乃
 詔考官韋顗等三人祇考及第科目人,其餘吏部侍郎自定,己卯,江西觀察伎崔芃卒。辛巳前魏情節度副使田懷諫為右監門衛將軍,賜宅一區、芻粟等。甲申,以同州刺史裴堪為江西觀察使。十二月丙戌朔,以吏部尚書鄭餘慶為太子少傅。丙辰,左拾遺楊歸厚以自娶婦,進狀借禮會院,貶國子主簿分司。戊戌,以京兆尹裴向為同州防禦使。己亥,魏博奏管內州縣官員二百五十三員,請吏部銓注。



 八年春正月乙卯朔。庚午,冊大言義為渤海國王,授秘書監、忽汗州都督。辛未,制以正議大夫、守禮部尚書、同平章事、上柱國、扶風郡開國公權德輿守禮部尚書,罷知政事。癸未,以山南東道節度使李夷簡檢校戶部尚書、成都尹,充劍南西川節度使。以戶部尚書袁滋檢校兵部尚書、襄州刺史,充山南東道節度使。



 二月乙酉朔。辛卯,田興改名弘正。宰相李吉甫進所撰《元和郡國圖》三十卷,又進《六代略》三十卷,又為《十道州郡圖》五十四
 卷。宰相于頔男太常丞敏專殺梁正言奴,棄溷中。事發,頔與男季友素服待罪。貶頔恩王傅。於敏長流雷州,錮身發遣。殿中少監、駙馬都尉於季友誑罔公主,藏隱內人,轉授兇兄,移貯外舍,傷風黷禮,莫大於茲,宜削奪所任官,令在家修省。贊善大夫於正、秘書丞於方並停見任,皆頔祿之子也。捕獲受于頔賂為致出鎮人梁正言,及交構權貴僧鑒虛,並付京兆府杖死。甲子,以劍南西川節度使、銀青光祿大夫、檢校吏部尚書、兼門下侍郎、同
 平章事、上柱國、臨淮郡開國公、食邑二千戶武元衡復入中書知政事,兼崇玄館大學士、太清宮使。辛未,上以久旱,親於禁中求雨,是夜,澍雨沾足。丙子,大風壞崇陵寢殿鴟尾,折門戟六。



 夏四月癸未朔。乙酉,以邕管經略使房啟為桂管觀察使,以開州刺史竇群為邕管經略使。丙戌,以錢重貨輕,出庫錢五十萬貫,令兩常平倉收市布帛,每段匹於舊估加十之一。鄜坊觀察使元義方卒,辛卯,以將作監薛伾為鄜坊觀察使。乙未,長安西市
 豕生三耳八足二尾。僧鑒虛為高崇文納賂四萬五千貫與宰相杜黃裳,共引致人永樂縣仿吳憑,付錢與黃裳男載。敕吳賃配流昭州,黃裳、崇文已薨歿,所用錢不須勘問,杜載釋放。辛亥,賜魏博田弘正錢二十萬貫,收市軍糧。庚申,河中尹張弘靖奏修古舜城。



 六月辛巳朔。時積雨,延英不開十五日。是日,上謂宰臣曰:「今後每三日,雨亦對來。」乙酉,工部尚書致仕裴佶卒。丙戌,以東都留守韓皋檢校吏部尚書,兼許州刺史,充忠武軍節度
 使。庚寅,京師大風雨,毀屋飄瓦,人多壓死。所在川瀆暴漲,行人不通。辛丑,出宮人二百車,任從所適,以水災故也。壬寅,宰臣武元衡李吉甫李絳、舊相鄭餘慶權德輿各奉詔令進舊詩。



 秋七月辛亥朔。癸丑,以權德輿檢校吏部尚書、東都留守。丁卯,以振武節度使李光進為靈州大都督府長史、靈武節度使。癸酉,命中尉彭中獻修興唐觀,壯其規制,北拒禁城,開復道以通行幸。是夜,月近五諸侯。丁丑,新授桂管觀察使房啟降為太僕少卿。
 啟初拜桂管,啟吏賂吏部主者,私得官告以授啟。俄有詔命中使齎告牒與啟,曰:「受之五日矣。」上怒,杖吏部令史,罰郎官,啟亦即降之。以安南都護馬總為桂管觀察使,以江州刺史張勔為安南都護、本管經略招討使。鄜坊觀察使薛伾卒。



 八月辛巳朔。癸未,以蘄州刺史裴行立為安南都護、本管經略招討使,以張勔耄年也。丁亥,以司農卿裴武為鄜坊觀察使。庚寅,詔毀家徇國故徐州刺史李洧等一十家子孫,並宜甄獎。甲午,太白近軒
 轅。辛丑,以東川節度使潘孟陽為戶部侍郎、判度支,盧坦為梓州刺史、劍南東川節度使。乙巳,廢天武軍,並入神策軍。



 九月庚戌朔。丙辰,淄青李師道進鶻十二,命還之。戊午,賜群臣宴於曲江。乙丑,詔:「比聞嶺南五管並福建、黔中等道,多以南口餉遺,及於諸處博易,骨肉離析,良賤難分。此後嚴加禁止,如違,長吏必當科罰。」淮西吳少陽獻馬三百匹。丙寅,詔:「減死戍邊,前代美政,量其遠邇,亦有便宜。今後兩京、關內、河南、河東、河北、淮南、山南
 東西道州府,除大闢罪外,輕犯不得配流天德五城。,」戊辰,以給事中竇易直為陜虢防禦使,仍賜金紫。壬申,以恩王傅于頔為太子賓客。以前朔方靈鹽節度使王佖為右衛將軍。將相出入,翰林草制,謂之白麻。至佖,奏罷中書草制,因為例也。太常習樂,始復用大鼓。



 冬十月庚辰朔。己丑,熒惑近太微西垣南首星。庚寅,以湖南觀察使柳公綽為嶽、鄂、沔、蘄、安、黃觀察使。辛卯,涇原節度使硃忠亮卒。壬辰,汴州韓弘進所撰《聖朝萬歲樂譜》,共三百首。
 己巳,以宗正少卿李道古為黔中觀察使,以蘇州刺史張正甫為湖南觀察使。丙申,以大雪放朝,人有凍踣者,雀鼠多死。戊戌,以神策普潤鎮使蘇光榮為涇州刺史、四鎮北庭行軍涇原節度使。翰林學士、司封員外郎韋弘景守本官,以草光榮詔漏敘功勛故也。壬辰,振武奏回紇千騎至鷿鵜泉。



 十一月庚戌朔。丙辰,以福建觀察使裴次元為河南尹。丙寅,以鹽州隸夏州。自夏州至豐州,初置八驛。丁卯,以泗州刺史薛謇為福建觀察使。右
 龍武統軍劉昌裔卒。癸酉,昭義郗士美奏諸軍就食於臨洺。京畿水、旱、霜,損田三萬八千頃。十二庚辰朔,以京兆尹李銛為鄜坊觀察使,以代裴武入為京兆尹。辛巳,敕:「應賜王公、公主、百官等莊宅、碾磑、店鋪、車坊、園林等,一任貼典貨賣,其所緣稅役,便令府縣收管。」敕:「張茂昭立功河朔,舉族歸朝,義烈之風,史冊收載。如聞身歿之後,家無餘財,追懷舊勛,特越常典,宜歲賜絹二千匹,春秋二時支給。」群臣上表,請立德妃郭氏為皇后。丙戌,
 以桂官觀察使馬總為廣州刺史、嶺南節度使,以邕管經略使崔詠為桂管觀察使。庚寅,以夔州刺史馬平陽為邕管經略使。振武軍亂,逐其帥李進賢,屠其家。乃以夏州節度使張煦代進賢,率兵二千赴鎮,許便宜擊斷。丙午,以金吾衛將軍田進為夏州刺史、夏綏銀節度使。以河溢浸滑州羊馬城之半,滑州薛平、魏博田弘正征役萬人,於黎陽界開古黃河道南北長十四里,東西闊六十步,深一丈七尺,決舊河水勢,滑人遂無水患。



 九年春正月己酉朔。乙卯,大霧而雪。李吉甫累表辭相位,不許。乙亥,張煦入單于都護府,誅作亂軍士蘇國珍等二百五十二人。



 二月巳卯朔,戶部侍郎、判度支潘孟陽兼京北五城營田使。丁丑,貶前振武節度使顧進賢為通州刺史,監軍路朝見配役於定陵。丁未,詔以歲饑,放關內元和八年己前逋租錢粟,賑常平義倉粟三十萬石。丙申,賜振武軍絹二萬匹。丁酉,月近心大星。癸卯,制朝議大夫、守中書侍郎、同平章事、上柱國、高邑男李絳守禮部
 尚書,累表辭相位故也。



 三月己酉朔。丙辰,巂州地震,晝夜八十震,壓死者百餘人。庚申,妖人梁叔高自廣州來,授書與吏部侍郎楊於陵,使為己輔。於陵執之以告,殺之。辛酉,以太子少傅鄭餘慶檢佼右僕射、興元尹、山南西道節度使,代趙宗儒為御史大夫。丁卯,隕霜殺桑。召大理卿裴棠棣男損、前昭應令杜式方男忭見於麟德殿前,各賜緋,許尚公主。



 夏四月戊寅朔。庚寅,詔贈太師咸寧王渾瑊宜配享德宗廟庭。



 五月丁未朔,以嶺南節
 度使鄭絪為工部尚書。庚申,移宥州於經略軍,郭下置延恩縣,隸夏州觀察使。是月旱,穀貴,出太倉粟七十萬石,開六場糶以惠饑民。乙丑,桂王綸薨。以旱,免京畿夏稅十三萬石、青苗錢五萬貫。



 六月丙子朔。戊寅,以天德軍經略使周懷乂卒,廢朝一日。經略使廢朝,自懷乂始也。庚辰,以義武軍節度副使渾鎬檢校式工尚書,兼定州大都督府長史,充義武軍節度使、易定觀察使、北平軍等使。丙戌,以在龍武將軍燕重旰為豐州刺史、天德
 軍豐州西城中城都防禦押蕃落等使。乙未,置禮賓院於長興里之北。丙申,以左丞孔戣為華州刺史、潼關防禦、鎮國軍等使。壬寅,制河中晉絳慈隰等州節度使張弘靖守刑部尚書、同中書門下平章事。



 秋七月丙午朔。乙未,以御史大夫趙宗儒檢校尚書右僕射,兼河中尹、河中晉絳等州節度使。戊辰,以太子司議郎杜忭為銀青光祿大夫、殿中少監、駙馬都尉,尚岐陽公主。閏八月乙巳朔。辛酉,以河陽節度使烏重胤兼汝州刺史。壬戌,
 以中書舍人王涯、屯田郎中韋綬為皇太子諸王侍讀。己巳,加田弘正檢校右僕射,賞三軍錢二十萬貫。九月甲戌朔,以洺州刺史李光顏為陳州刺史、忠武軍都知兵馬使。丙戌,以山南東道節度使袁滋檢校兵部尚書,兼江陵尹、荊南節度使。以荊南節度使嚴綬檢校司空、襄州刺史、山南東道節度使。己丑,月掩軒轅。淮西節度使吳少陽卒,其子元濟匿喪,自總兵柄乃焚劫舞陽等四縣。朝廷遣使吊祭,拒而不納。壬辰,真臘國朝貢。戊戌,
 加河東節度使王鍔檢校司空、同平章事,以給事中孟簡為越州刺史、浙東觀察使。贈吳少陽尚書右僕射。



 冬十月甲辰朔。丙午,金紫光錄大夫、中書侍郎、同平章事、集賢大學士、監修國史、上柱國、趙國公李吉甫卒。甲寅,以刑部員外郎令狐楚為職方員外郎、知制誥。壬戌,以忠武軍節度使韓皋為吏部尚書,以忠武軍節度副使兼陳州刺史李光顏為許州刺史、忠武軍節度使。甲子,制:「朕嗣膺寶位,於茲十年。每推至誠,以御方夏,庶以仁
 化,臻於太和,宵衣旰食,意屬於此。今淮西一道,未達朝經。擅自繼襲,肆行寇掠。將士等迫於受制,非是本心。思去三面之羅,庶遵兩階之義。宜以山南東道節度使嚴綬兼充申光蔡等州招撫使。」仍命內常侍崔潭峻為監軍。戊辰,以尚書左丞呂元膺檢校工部尚書、東都留守。舊例,命留守賜旗甲,與方鎮同,及元膺受命,不賜。諫官援華、汝、壽三州例有賜,居守之重,不宜獨闕,上曰:「此三處亦宜停賜。」



 十一月甲戌朔。甲申,以吏部尚書韓皋為
 太子賓客。甲午,以御史中丞胡證為單于大都護、振武麟勝等軍節度使。丁酉,太子太傅範希朝卒。戊戌,以中書舍人裴度為御史中丞;以左金吾大將軍郭釗檢校工部尚書、邠州刺史,充邠寧節度使;以職方員外郎、知制誥令狐楚為翰林學士。十二月甲辰朔。丁未,振武節度使張煦卒。辛亥,邠寧節度使、檢校右僕射閻巨源卒。癸丑,兵部尚書王紹卒。己未,右羽林統軍孟元陽卒。丙寅,太子少保趙昌卒。戊辰,制以中大夫、守尚書右丞、上
 騎都尉、賜紫金魚袋韋貫之本官同中書門下平章事。



 十年春正月癸酉朔。乙酉,宣武軍節度使韓弘守司徒,平章事並如故。丙申,嚴綬帥師次蔡州界。己亥,制削奪吳元濟在身官爵。庚子,桂管奏移富州治於故城。



 二月癸卯朔。甲辰,嚴綬軍為賊所襲,敗於磁丘,退守唐州。田弘正子布、韓弘子公武各率師隸李光顏討賊。辛亥,以禮部尚書李絳為華州潼關防禦鎮軍等使。壬戌,河東防秋將劉輔殺豐州刺史燕重旰。己巳,以羽林將軍李
 匯為涇原節度使。



 三月壬申朔,以右金吾將軍李奉仙為豐州刺史、天德軍西城中城都防禦使。己卯,以劍南西川節度行軍司馬李程為兵部郎中、知制誥。乙酉,以虔州司馬韓泰為漳刑刺史,以永州司馬柳宗元為柳州刺史、饒州司馬韓曄為汀州刺史,朗州司馬劉禹錫為播州刺史,臺州司馬陳諫為封州刺史。御史中丞裴度以禹錫母老,請移近處,乃改授連州刺史。贈故太常卿崔邠禮部尚書。李光顏破賊於南頓。辛亥,盜焚河陰
 轉運院,凡燒錢帛二十萬貫匹、米二萬四千八百石、倉室五十五間。防院兵五百人營於縣南,盜火發而不救,呂元膺召其將殺之。自盜火發河陰,人情駭擾。壬戌,以長安縣令徐俊為邕管經略使。



 五月辛未朔。辛巳,御史中丞裴度兼刑部侍郎。時度自淮西行營宣慰還,所言軍機,多合上旨,故以兼官寵之。丙申,李光顏大破賊黨於洄曲。自徵兵討賊,凡十餘鎮之師,環於申、蔡,未立戰功。裴度使還,奏曰:「臣觀諸將,惟光顏見義能勇,必能立
 功。」至是告捷,京師相賀,上尤賞度之知人。



 六月辛丑朔。癸卯,鎮州節度使王承宗盜夜伏於靖安坊,刺宰相武元衡,死之;又遣盜於通化坊刺御史中丞裴度,傷首而免。是日,京城大駭,自京師至諸門加衛兵;宰相導從加金吾騎士,出入則彀弦露刃,每過里門,訶索甚諠;公卿持事柄者,以家僮兵仗自隨。武元衡死數日,未獲賊。兵部侍郎許孟容請見,奏曰:「豈有國相橫尸路隅,不能擒賊!」因灑泣極言,上為之憤嘆。乃詔京城諸道,能捕賊
 者賞錢萬貫,仍與五品官,敢有蓋藏,全家誅戮。乃積錢二萬貫於東西市。京城大索,公卿節將復壁重尞者皆搜之。庚戌,神策將士王士則、王士平以盜名上言,且言王承宗所使,乃捕得張晏等八人誅之。乙丑,制以朝議郎、守御史丞、兼刑部侍郎、飛騎尉、賜紫金魚袋裴度為朝請大夫、守刑部侍郎、同中書門下平章事。



 秋七月庚午朔,靈武節度使李光進卒。辛未,以神策軍長武城使杜叔良為朔方、靈鹽、定遠城節度觀察使。甲戌,詔:「成
 德軍節度使王承宗,自滌瑕疵,累加獎拔,列在維籓之任,待以忠正之徒。謂懷君父之恩,克勵人臣之節。而動思棄命,恣逞非心,傲狠反常,橫辱無畏。以其先祖,嘗立忠動,每念含容,庶聞悛革。曾不知陰謀逆狀,久則逾彰;兇德禍機,盈而自覆。乃敢輕肆指斥,妄陳表章,潛遣奸人,內懷兵刃,賊殺元輔,毒傷憲臣,縱其兇殘,無所顧望。推窮事跡,罪狀昭明,周覽讞詞,良用驚嘆。宜令絕其朝貢,其所部博野、樂壽兩縣本屬範陽,宜卻隸劉總。駙馬
 都尉王承系、太子贊善王承迪、丹王府司馬王承榮等,並宜遠郡安置。」先是,承宗上表怨咎武元衡,留中不報。又肆指斥,上使持其表以示百官,群臣皆請問罪。丙戌,涇原節度使李匯卒,以將作監王潛為涇州刺史、四鎮北庭涇原節度使。乙未,以京兆尹裴武為司農卿,以捕賊弛慢故也。



 八月己亥朔,日有蝕之。丙寅,訶陵國遣使獻僧祗僮及五色鸚鵡、頻伽鳥並異香名寶。丁未,淄青節度使李師道陰與嵩山僧圓凈謀反,勇士數百人伏
 於東都進奏院,乘洛城無兵,欲竊發焚燒宮殿而肆行剽掠。小將場進、李再興告變,留守呂元膺乃出兵圍之,賊突圍而出,入嵩岳,山棚盡擒之。訊其首,僧圓凈主謀也。僧臨刑嘆曰:「誤我事,不得使洛城流血!」九月癸酉,以宣武軍節度使韓弘充淮西行營兵馬都統。丁酉,以太子賓客韓皋為兵部尚書。冬十月庚子,始析山南東道為兩節度,以戶部侍郎李遜為襄州刺史,充襄、復、郢、均、房節度使;以右羽林將軍高霞寓為唐州刺史,充唐、隨、
 鄧節度使。刑部尚書權德輿奏請行用新刪定《敕格》三十卷,從之。壬子,以太子賓客于頔為戶部尚書。



 十一月戊辰,詔出內庫繒絹五十五萬匹供軍。乙亥,以山南東道節度使嚴綬為太子少保。戊寅,盜焚獻陵寢宮。詔發振武兵二千,會義武軍以討王承宗。十二月壬寅夜,太白犯鎮星。甲辰,李願擊敗李師道之眾九千,斬首二千級。壬子,東都留守呂元膺請募置三河子弟以衛宮城。甲寅,越州復置山陰縣。庚申,新造指南車、記里鼓。出宮
 人七十二人置京城寺觀,有家者歸之。乙丑,河東節度使王鍔卒。是歲,渤海、新羅、奚、契丹、黑水、南詔、牂柯並遣使朝貢。



 十一年春正月丁卯朔,以宿師於野,不受朝賀。己巳,以中書侍郎、平章事張弘靖檢校吏部尚書,兼太原尹、北都留守、河東節度使。戊寅,詔群臣曰:「今用兵已久,利害相半。其攻守之宜,罰宥之要,宜各具議狀以聞。」庚辰,翰林學士錢徽、蕭俯各守本官,以上疏請罷兵故也。癸未,
 削奪王承宗在身官爵,所襲封邑賜武俊子金吾將軍士平。令河東、河北道諸鎮加兵進討。甲申,盜斷建陵門戟四十七竿。甲子,李光顏奏破賊。



 二月癸卯,吐蕃贊普卒。以中書舍人、權知禮部貢舉、賜緋魚袋李逢吉為門下侍郎、同平章事,賜紫金魚袋。以內庫絹四萬匹賞幽、魏將士。甲寅,以華州刺史李絳為兵部尚書。丙辰,月掩心。戊午,南詔蠻酋龍蒙盛卒。



 三月庚午,皇太后崩於興慶宮之咸寧殿。是日,群臣發喪於西宮兩儀殿,以宰臣
 裴度為禮儀使,吏部尚書韓皋為大明宮留守,設次於中書。辛未,敕諸司公事,宜權取中書門下處分。癸酉,分命朝臣告哀於天下。甲戌,見群臣於紫宸門外廡下。己卯,以宰臣李逢吉充大行皇太后山陵使。出內庫繒帛五萬匹充奉山陵。己丑,月近鎮星。



 夏四月壬寅,西川節度使李夷簡遣使告哀於南詔。後喪,邊鎮告四夷,舊制也。庚戌,貶戶部侍郎、判度支楊於陵為郴州刺史,坐供軍有闕也。丁巳,以徐、宿饑,賑粟八萬石。



 五月丁卯夜,辰、
 歲二宿合於東井。宥州軍亂,逐刺史駱怡。壬申,李光顏破賊於凌雲柵。六月甲辰,高霞寓敗於鐵城,退保新興柵,是日人情悚駭,宰相奏對,多請罷兵。上曰:「勝負兵家常勢,不可以一將失利,便沮成計。今但議用兵方略,朝廷庶務,制置可否耳。」是夜,月掩心後星。庚戌,田弘正軍討王承宗,次於南宮。辛酉,群臣上大行皇太后謚曰莊憲。



 秋七月丁丑,貶隨、唐節度使高寓為歸州刺史。以河南尹鄭權為襄州刺史,充山南東道節度使;以荊南
 節度使袁滋為唐州刺史、彰義軍節度使、申光唐蔡隨鄧州觀察使,權以唐州理為所;以華州刺史裴武為江陵尹,充荊南節度使。戊寅,以隨州刺史楊旻為唐州刺史,充行營都知兵馬使。以滋儒者,故復以旻將其兵。壬午,宣武軍奏破賊。



 八月壬寅,以宰臣韋貫之為吏部侍郎,罷知政事。貫之以淮西、河北兩處用兵,勞於供餉,請緩承宗而專討元濟,與裴度爭論上前故也。戊申,容州奏颶風海水毀州城。甲申,祔莊憲皇后於豐陵。九月丁
 卯,饒州奏浮梁、樂平二縣,五月內暴雨水溢,失四千七百戶,溺死者一百七十人。丙子,新除吏部侍郎韋貫之再貶湖南觀察使。辛未,貶吏部侍郎韋顗為陜州刺史,刑部郎中李正為金州刺史,度支郎中薛公幹為房州刺史,屯田郎中李宣為忠州刺史,考功郎中韋處厚為開州刺史,禮部員外郎崔韶為果州刺史,並為補闕張宿所構,言與貫之朋黨故也。乙酉,蔡州軍前奏拔凌雲柵。



 冬十月丁巳,以刑部尚書權德輿檢校吏部尚書,
 兼興元尹,充山南西道節度使。丙寅,幽州劉總加平章事,鄆州李師道加檢校司空。師道聞拔凌雲柵,乃懼,偽貢款誠,故有是令。庚午,以司農卿王遂為宣州刺史、宣歙池觀察使,以京兆尹要翛為潤州刺史、浙西觀察使。以遂、翛常歷計司,能聚斂,方藉供軍,故有斯授。壬申,敕諸道奏事官,非急切不得乘驛馬。丁丑,出內庫錢五十萬貫供軍。戊寅夜,月犯歲。辛巳,命內常侍梁守監淮西行營諸軍。仍以空名告身五百通及金帛付之。戊子
 夜,土、火合於虛、危。十二月丙午,以易州刺史陳楚為定州刺史、義武軍節度使。丁未,以翰林學士、尚書工部侍郎、知制誥王涯為中書侍郎、同平章事。甲寅,以閑廄宮苑使李醖檢校左散騎常侍,兼鄧州刺史,充唐、隨、鄧等州節度使。初置潁水運使。運揚子院米,自淮陰溯流至壽州,四十里入潁口,又溯流至潁州沈丘界,五百里至於項城,又溯流五百里入溵河,又三百里輸於郾城。得米五十萬石,茭一千五百萬束。省汴運七萬六千貫。
 己未,邕管奏黃洞賊屠巖州。未央宮及飛龍草場火。京畿水害田,潤、常、湖、衢、陳、許大水。是歲冬雷,桃、杏花。回鶻、奚、契丹、牂柯、渤海等朝貢。



 十二年春正月辛酉朔,以用兵不受朝賀。癸未,貶義武軍節度使渾鎬為循州刺史,坐討賊失律也。甲申,貶唐、鄧節度袁滋為撫州刺史,以上疏請罷兵故也。乙酉夜,星見而雨。戊子夜,彗出畢南,長丈餘,指西南,凡三日,南近參旗而沒。



 二月壬申,以內庫絹布六十九萬段匹、銀
 五千兩,付度支供軍。庚子,敕京城居人五家相保,以搜奸慝。時王承宗、李師道欲阻用兵之勢,遣人折陵廟之戟,焚芻槁之積,流矢飛書,恐駭京國,故搜索以防奸。及賊平,復得淄青簿領,中有賞蒲、潼關吏案,乃知容奸者關吏也,搜索不足以為防。庚申,敕宜於許汝行營側近置行郾城,以處賊中歸降人戶。甲寅,岳鄂團練使李道古師攻申州,克羅城,賊力戰,道古之眾大敗。



 三月壬戌,昭義郗士美兵敗於柏鄉,兵士死者千人。戊辰,滄州程
 執恭改名權。太常定李吉甫謚曰「敬憲」,度支郎中張仲方非之,上怒,貶為遂州司馬。賜吉甫謚曰忠。丁丑,月犯心後星。癸未,賊將吳秀琳以文城柵兵三千降李醖。



 夏四月辛卯,李光顏破賊三萬於郾城,殺其卒什二三,獲馬千匹、器甲三萬。辛丑,駙馬都尉於季友居嫡母喪,與進士劉師服歡宴夜飲。季友削官爵,苔四十,忠州安置;師服笞四十,配流連州;于頔不能訓子,削階。己酉,出太倉粟二十五萬石糶於西京,以惠饑民。庚戌,敕改蔡州
 吳房縣為遂平縣,移置於文城柵南新城內。丁卯,賊郾城守將鄧懷金與縣令董昌以郾城降。甲戌,渭南雨雹,中人有死者。丙子,詔權罷河北行營,專討淮、蔡。



 五月庚寅朔。癸巳,隨唐節度使李醖奏敗賊於吳房,獲賊將李祐。己亥,以尚書左丞許孟容為東都留守,充都畿防禦使。時東畿民戶供軍尤苦,車數千乘相錯於路,牛皆饋軍,民戶多以驢耕。己酉,作蓬萊池周廊四百間。



 六月己未朔,以衛尉卿程異為鹽鐵使,代王播。時異為鹽鐵
 使副,自江南收拾到供軍錢一百八十五萬以進,故得代播。壬戌,賊吳元濟上表,請束身歸朝。時連破三柵,賊勢迫蹙,實欲歸朝,而制於左右,故不果行。乙酉,京師大雨,含元殿一柱傾,市中水深三尺,壞坊民二千家。



 秋七月戊子朔。壬辰,詔以定州饑,募人入粟受官及減選、超資。河北水災,邢、洺尤甚,平地或深二丈。甲辰,戶部尚書于頔請致仕,不允。嶺南節度使崔詠卒。乙酉,敕:「今後左降官及責授正員官等,宜從到任經五考滿,許量移;如
 未滿五考遇恩赦者,從節文處分;如犯十惡大逆、贓賄緣坐,奏取進止。」庚戌,以國子祭酒孔戣為廣州刺史、嶺南節度使。丙辰,制以中書侍郎、平章事裴度守門下侍郎同平章事、使持節蔡州諸軍事、蔡州刺史,充彰義軍節度、申光蔡觀察處置等使,仍充淮西宣慰處置使。以朝散大夫、守尚書戶部侍郎、上護軍、賜紫金魚袋崔群為中書侍郎、同中書門下平章事。以刑部侍郎馬總兼御史大夫,充淮西行營諸軍宣慰副使;以太了右庶子
 韓愈兼御史中丞,充彰義軍行軍司馬;以司勛員外郎李正封、都官員外郎馮宿、禮部員外郎李宗閔皆兼侍御史,為判官書記:從度出征。詔以郾城為行蔡州治所。



 八月戊午朔。庚申,裴度發赴行營,敕神策軍三百人衛從,上禦通化門勞遣之。度望門再拜,銜涕而辭,上賜之犀帶。以河南尹辛秘為潞府長史、昭義軍節度使,代郗士美。以士美為工部尚書,孟簡為戶部侍郎。戊辰,以同州刺史張正甫為河南尹。甲申,裴度至郾城。



 九月丁亥
 朔。戊子,出內庫羅綺、犀玉、金帶之具,送度支估計供軍。甲午,御史臺奏;「同制除官,承前以名字高下為班位先後。或名在前身在外,及到,卻在舊人之上。今請以上日為先後。」敕曰:「名在前,上日在後,未逾月,不在此限。行立班次,即宜以敕內前後為定。」戊戌,劍南東川節度盧坦卒。己亥,貶京兆尹竇易直為金州刺史,以鞫獄得贓不實故也。辛丑,以御史中丞為京兆尹。壬寅,以湖南觀察使韋貫之為太子詹事分司。乙巳,以刑部郎中知雜崔
 元略為御史中丞。丁未,以朝議大夫、門下侍郎、同平章事李逢吉檢校兵部尚書、使持節梓州諸軍事、梓州刺史,充劍南東川節度副大使,知節度事。庚子,以撫州刺史袁滋為湖南觀察使。



 冬十月壬申,裴度往沲口觀板築五溝,賊遽至,注弩挺刃將及度,面李光顏,田布扼其歸路,大敗之。是日,度幾陷。癸酉,內出《元和辯謗略》三卷付史館。甲申,以淮南節度使、檢校左僕射李鄘為門下侍郎、同中書門下平章事,以左丞衛次公代鄘為淮南
 節度使。己卯、隨、唐節度使李醖率師入蔡州,執吳元濟以獻,淮西平。甲申詔:「淮西立功將出,委韓弘、裴度條疏奏聞。淮西軍人,一切不問。宜準元敕給復二年。」十一月丙戌朔,御興安門受淮西之俘。以吳元濟徇兩市,斬於獨柳樹;妻沈氏,沒入掖庭;弟二人、子三人,配流,尋誅之;判官劉協等七人處斬。錄平淮西功:隨唐節度使、檢校左散騎常侍李醖檢校尚書左僕射、襄州刺史,充山南東道節度、襄鄧隨唐復郢均房等州觀察等使;加宣武
 軍節度使韓弘兼侍中;忠武軍節度使李光顏、河陽節度使烏重胤並檢校司空。以宣武軍都虞候韓公武檢校左散騎常侍、鄜坊丹延節度使,以魏博行營兵馬使田布為右金吾衛將軍,皆賞破賊功也。甲午,恩王連薨。以蔡州郾城為溵,析上蔡、西平、遂平三縣隸焉。戊申,以淮西宣慰副使、門下侍郎、同平章事裴度守本官,賜上柱國、晉國公、
 食邑三千戶;以蔡州留後馬總檢校工部尚書、蔡州刺史、彰義軍節度使、溵州潁陳許節度使。丙子,以右庶子韓愈為刑部侍郎。是歲,河南、河北水。



 十三年春正月己酉朔,御含元殿受朝賀,禮畢,御丹鳳樓,大赦天下。己丑,以文宣王三十八代孫孔惟晊襲文宣公。庚寅,敕李師道頻獻表章,披露懇誠,宜令諫議大夫張宿往彼宣慰。辛亥,以禮部尚書王播為成都尹、劍南西川節度使。二月乙亥,御麟德殿,宴群臣,大合樂,凡
 三日而罷,頒賜有差。



 三月庚寅,以前劍南西川節度使李夷簡為御史大夫。兩丙,以同州刺史鄭絪為東都留守、都畿汝防禦使。庚子,以御史大夫李夷簡為門下侍郎、同平章事。宰相李鄘守戶部尚書,罷知政事。丁未,以太子少師鄭餘慶為左僕射。辛亥,詔:「百司職田,多少不均,為弊日久,宜令逐司各收職田草粟都數,自長官以下,除留闕官物外分給。」至銀臺待罪,請獻德、棣二州,兼入管內租稅。壬戌,前東都留守許孟容卒。庚辰,詔復王承宗官爵。以
 華州刺史鄭權為德州刺史、橫海軍節度、德棣滄景等州觀察使。



 五月乙酉,鳳翔節度使李惟簡卒。乙未,月近心後星。丙辰,以忠武軍節度使李光顏為滑州刺史、義成軍節度使,以彰義軍節度使馬總為許州刺史、忠武軍節度使、陳許溵蔡觀察使。戊戌,以山南東道節度使李醖為鳳翔尹、鳳翔隴右節度使,辛丑,知渤海國務大仁秀檢校秘書監、忽汗州都督,冊為渤海國王。丙午,以戶部侍郎孟簡檢校工部尚書、襄州刺史、山南東道
 節度使。



 六月癸丑朔,日有食之。乙丑,湖南觀察使袁滋卒。丁丑,以滄景節程權為邠州刺史、邠寧節度使。出內庫絹三十萬匹、錢三十萬貫,付度支供軍。



 秋七月癸未,以新除鳳翔節度使李醖為徐州刺史、武寧軍節度使。甲申,以田弘正檢校司空。乙酉,詔削奪淄青節度使李師道在身官爵,仍令宣武、魏博、義成、武寧、橫海等五鎮之師,分路進討。辛丑,以門下侍郎、同平章事李夷簡檢校左僕射、同平章事、揚州大都督府長史、淮南節
 度使。己酉,詔諸道節度使先帶度支營田使名者,並罷之。庚戌,以左僕射鄭餘慶為鳳翔隴右節度使。八月壬子,以中書侍郎平章事王涯為兵部侍郎,罷知政事。戊午,以尚書右丞崔從為興元尹、山南西道節度使。甲戌,太白近左執法。乙亥,敕應同司官有大功已上親者,但非連判及勾檢之官並官長,則不在回避改換之限。時刑部員外郎楊嗣復以父於陵除戶部侍郎,遂以近例避嫌,請出省,不從,因有是敕。丁丑,木、金、水、三宿聚於軫。
 戊寅,前山南西道節度使權德輿卒。



 九月甲申,以左衛將軍高霞寓為單于大都護,、振武麟勝節度使。甲辰,以戶部侍郎、判度支皇甫鎛同中書門下平章事,依前判度支。以衛尉卿充諸道鹽鐵轉運使程異為工部侍郎、同中書門下平章事,依前充使。是時,上切於財賦,故用聚斂之臣居相位。詔下,群情驚駭,宰臣裴度、崔群極諫,不納。二人請退。熒惑近哭星。丁未,出內庫絹十萬匹給東軍。



 冬十月甲寅,吐蕃寇宥州。壬戌,靈武奏破吐蕃二
 萬於定遠城。癸亥,前淮南節度使衛次公卒。甲子,平涼鎮遏兵馬使郝玼奏收復原州,破吐蕃二萬。是夜,月近昴。丙子,以左金吾衛大將軍薛平檢校刑部尚書、滑州刺史,充義成軍節度使;以義成軍節度使李光顏為許州刺史,充忠武軍節度使、陳許觀察等使。



 十一月辛巳朔,夏州破吐蕃五萬。靈武奏攻破吐蕃長樂州羅城。丁亥,以山人柳泌為臺州刺史,為上於天臺山採仙藥故也。制下,諫官論之,不納。壬寅,以河陽節度使鳥重胤為
 滄州刺史、橫海軍節度、滄景德棣觀察等使。丁未,以華州刺史令狐楚為懷州刺史,充河陽三城、懷、孟節度使。十二月辛亥,敕左右龍武軍六軍及威遠營應納課戶共一千八百人衣糧並停,仍付府縣收管。戊寅,軍前擒到李師道將夏侯澄等四十七人,詔並釋付魏博及義成軍收管,要還賊中者,則量事優給放還。上顧謂宰臣曰:「人臣事君,但力行善事,自致公望,何乃好樹朋黨。」
 上曰:「他人之言,亦與卿等相似,豈易辯之哉?」度曰:「君子小人,觀其所行,當自區別矣。」上曰:「凡好事口說則易,躬行則難。卿等即言之,須行之,勿空口說。」度等謝曰:「陛下處分,可謂至矣,臣等敢不激勵。然天下之人,從陛下所行,不從陛下所言,臣等亦願陛下每言之則行之。」上頗欣納。是歲,回紇、南詔蠻、渤海、高麗、吐蕃、奚、契丹、訶陵國並朝貢。



 十四年春正月庚辰朔,以東師宿野,不受朝賀。壬午,復
 置仗內教坊於延政里。丁亥徐州軍破賊二萬於金鄉。迎鳳翔法門寺佛骨至京師,留禁中三日,乃送詣寺,王公士庶奔走舍施如不及。刑部侍郎韓愈上疏極陳其弊。癸巳,貶愈為潮州刺史。丙申,魏博軍破賊五萬於東阿。辛巳,斬前滄州刺史李宗奭於獨柳樹。朝廷初除鄭權滄州,宗奭拒詔不受代,既而為三軍所逐,乃入朝,故誅之。癸卯夜,月近南斗魁。丙午,魏博軍破賊萬人於陽谷。



 二月己酉朔,以商州刺史嚴謨為黔中觀察使。乙卯,
 敕淄青行營諸軍,所至收下城邑,不得妄行傷殺,及焚燒廬舍,掠奪民財,開發墳墓,宜嚴加止絕。以鎮、冀水災,賜王承宗綾絹萬匹。辛酉,襄陽節度使孟簡舉鄖鄉鎮遏使趙潔為鄖鄉縣令,有虧常式,罰一月俸料。壬戌,田弘正奏,今月九日,淄青都知兵馬使劉悟斬李師道並男二人首請降,師道所管十二州平。甲了,上御宣政殿受賀。己巳,上御興安門受田弘正所獻賊俘,群臣賀於樓下。庚午,制以淄青兵馬使、金紫光錄大夫、試殿中監、
 兼監察御史劉悟檢校工部尚書、滑州刺史,充義成軍節度使,封彭城郡王,食邑三千戶,賜錢二萬貫、莊宅各一區。癸酉,田弘正加檢校司徒、同中書門下平章事。



 三月己卯朔。丁酉,上以齊、魯初平,宴群臣於麟德殿,賜物有差。戊子,以華州刺史馬總鄆、濮、曹等州觀察等使;己丑,以義成軍節度使薛平為青州刺史,充平盧軍節度、淄青齊登萊等州觀察等使;以淄青四面行營供軍使王遂為沂州刺史,充沂、海、兗、密等州都團練觀察等使:
 析李師道所據十二州為三鎮也。庚寅,浙西觀察使李翛卒。辛卯,李師道妻魏氏並男沒入掖庭,堂弟師賢師智、侄弘巽配流。乙未,以中書舍人衛中行華州刺史、潼關防禦、鎮國軍等使。辛丑,上顧謂宰臣曰:「聽受之間,大是難事。推誠選任,所謂委寄,必合盡心;及至所行,臨事不無偏黨。朕臨御已來,歲月斯久,雖不明不敏,然漸見物情,每於行為,務欲詳審。比令學士集前代昧政之事,為《辯謗略》,每欲披閱,以為鑒誡耳。」崔群對曰:「無情曲直,
 辯之至易;稍懷欺詐,審之實難。故孔子有眾好眾惡之論,浸潤膚受之說,蓋以曖昧難辯故也。若擇賢而任之,待之以誠,糾之以法,則人自歸公,孰敢行偽?陛下詳觀載籍,以廣聰明,實天下幸甚。」以撫州司馬令狐通為右衛將軍。給事中崔植封還制書,言通前刺史壽州,用兵失律,未宜獎用。上令宰臣諭植,以通父彰有功,不忍遂棄其子,其制方行。



 夏四月戊申朔。乙卯,太白順行近東井。戊午,以刑部尚書李願為鳳翔尹,充鳳翔、隴右
 節度使。丙寅,詔:「諸道節度、都團練、防禦、經等使所管支郡,除本軍州外,別置鎮遏、守捉、兵馬者,並合屬刺史。如刺史帶本州團練、防禦、鎮遏等使,其兵馬額便隸此使。如無別使,即屬軍事。其有邊於溪洞連接蕃蠻之處,特建城鎮,不關州郡者,不在此限。」辛未,工部侍郎、同平章事、諸道鹽鐵轉運等使程異卒。丙子,制金紫光錄大夫、門下侍郎、同中書門下平章事,兼弘文館大學士、上柱國、晉國公、食邑三千戶裴度可檢校左僕射,兼門下
 侍郎、平章事、太原尹、北都留守,充河東節度、觀察、處置等使。



 五月戊寅朔,以刑部侍郎柳公綽充鹽鐵轉運等使。庚辰,以楚州刺史李聽為夏州刺史、夏綏銀宥等州節度使。丙戌,以河東節度使、檢校吏部尚書、同平章事張弘靖為吏部尚書;以忠武軍節度使李光顏為邠、寧、慶節度使,仍以忠武軍六千人赴鎮。庚寅,以工部尚書郗士美檢校刑部尚書、許州刺史,充忠武軍節度使。是夜,月近心大星。己亥,置臨海監牧,命淮南節度使兼之。
 敕李師古妻裴氏、女宜娘於鄧州安置,李宗奭妻韋氏放出掖庭:坐李師道族人籍沒,上愍之,宥以輕典。以宣歙觀察使竇易直為潤州刺史,充浙西觀察使韓弘進助平淄青絹二十萬匹,女樂十人。女樂還之。



 六月丁未朔。癸丑,以福建觀察使元錫為宣州刺史、宣歙池觀察使。庚申,以戶部侍郎歸登為工部尚書。以鄭州刺史裴乂為福州刺史、福建觀察使。辛酉,敕定州大都督府復上州。甲子,以前兵部尚書李絳檢校吏部尚書、河中尹,
 充河中晉張慈隰觀察使。癸酉,詔左金吾大將軍胡證充京西北巡邊使,所經鎮戍,與守將審量利害,具事實奏聞。



 秋七月丁丑。戊寅,汴州韓弘來朝。己卯,左散騎常侍致仕薛蘋卒。乙酉夜,月掩心大星。辛巳,群臣上尊號曰元和聖文神武法天應道皇帝。是日,御宣政殿受冊,禮畢,御丹鳳樓,大赦天下。京畿今年秋稅、青苗、榷酒等錢,每貫量放四百文;元和五年已前逋租賦並放。甲午,韓弘進糸也絹二十八萬匹,銀器二百七十事。丁酉,以
 河陽三城懷州節度使、朝議郎、使持節懷州諸軍事、守懷州刺史、兼御史大夫、賜紫金魚袋令狐楚可朝議大夫、過中書侍郎、同中書門下平章事。壬寅,以永州刺史韋正武為邕管經略使。癸卯,以前黔中觀察使魏義通為懷州刺史、河陽三城懷孟節度使。沂州軍亂,殺節度使王遂。甲辰,以棣州刺史曹華為沂州刺史,充沂、海、兗密等州都團練觀察使。乙巳,罷晉州防禦使。



 八月丁未朔。乙酉,制宣武軍節度副大使、知節度事、汴宋亳潁等
 州觀察處置等使、開府儀同三司、守司徒、兼侍中、汴州刺史、上柱國、許國公、食邑三千戶韓弘可守司徒、兼中書令,宣武軍節度使。甲寅,於襄州穀城縣置臨漢監以牧馬,仍令山南東道節度使兼充監牧使。戊午,王承宗進位檢校左僕射。己未,田弘正來朝。上謂宰臣曰:「天下事重,一日不可曠廢。若遇連假不坐,有事即詣延英請對。」崔群以殘暑方甚,目
 同列將退。上止之曰:「數日一見卿等,時雖暑熱,朕不為勞。」久之方罷。丁亥,宴田弘正與大將判官二百人於麟德殿,賜物有差。戊辰,陳許節度使、檢校刑部尚書郗士美卒。



 九月丙子朔。戊寅,考功郎中蕭祐進古畫、古書二十卷。斬沂州亂首王弁於東市。癸未,以國子祭酒李遜檢校禮部尚書、許州刺史、忠武軍節度、陳許溵蔡等觀察使。庚寅,貶右衛大將軍田縉為衡王傅。縉前鎮夏州,私用軍糧四萬石,強取黨項羊馬,致黨項引吐蕃入寇
 故也。甲午,以太子少師鄭餘慶兼判國子祭酒。辛丑,以田弘正兄相州刺史田融檢校刑部尚書,兼太子賓客,分司東都。甲辰,以魏博節度使、光祿大夫、檢校司徒、同平章事、兼魏州大都督長史、上柱國、沂國公、食邑三午戶田弘正依前檢校司徒、兼侍中,賜實封三百戶。時弘正三上表乞留闕庭,不許。乙巳,上顧謂宰臣曰:「朕讀《玄宗實錄》,見開元初銳意求理,至十六年已後,稍似懈倦,開元未又不及中年,何也?」崔群對曰:「玄宗少歷民間,身
 經迍難,故即位之初,知人疾苦,躬勤庶政。加之姚崇、宋璟、蘇頲、盧懷慎等守正之輔,孜孜獻納,故致治平。及後承平日久,安於逸樂,漸遠端士,而近小人。宇文融以聚斂媚上心,李林甫以奸邪惑上意,加之以國忠,故及於亂。願陛下以開元初為法,以天寶未為戒,即社稷無疆之福也。」時皇甫鎛以諂刻欺蔽在相位,故群因奏以諷之。



 冬十月丙午朔。壬戌,安南軍亂,殺都護李象古並家屬,部曲千餘人皆遇害。丙寅,以唐州刺史桂仲武為安
 南都護,潮州刺史韓愈為袁州刺史。是月,吐蕃寇鹽州。



 十一月乙亥朔,以戶部尚書李鄘為太子賓客、東都留守。辛卯,靈武大將史敬奉破吐蕃於鹽州城下,賜敬奉實封五十戶賞之。丁酉,以原王傅鄭權為右金吾大將軍,充右街使。上服方士柳泌金丹藥,起居舍人裴濆上表切諫,以「金石含酷烈之性,加燒煉則火毒難制。若金丹已成,且令方士自服一年,觀其效用,則進御可也。」上怒。己亥,貶裴濆為江陵令。十二月乙巳朔。庚戌,國子祭
 酒鄭餘慶奏見任文官一品至九品,外使兼京正員官者,每月於所請秋錢每貫抽十文,修國子監,從之。乙卯,以諫議大夫、守中書侍郎、同中書門下平章事、上柱國、賜紫金魚袋崔群為潭州刺史、兼御史大夫,充湖南觀察使。為皇甫鎛所譖。及群被貶,人皆切齒於鎛。



 十五年春正月甲戌朔,上以餌金丹小不豫,罷元會。庚辰,鎮冀觀察使王承宗奏鎮冀深等州,每州請置錄事參軍一員,判司三員,每縣請置令一員,從之。壬午,以
 前湖南觀察使崔倰權知戶部侍郎、判度支。丙戌,沂、海四州觀察使府移置於兗州,改觀察使曹華為兗州刺常陰晦,微雨雪,夜則晴明,凡十七日方霽。丙申,月犯心大星,光彩相及。廢齊州豐齊縣入長清,廢全節縣入歷城,廢亭山縣入章丘縣。義成軍節度使劉悟來朝。戊戌,上對悟於麟德殿。上自服藥不佳,數不視朝,人情恟懼,及悟出道上語,京城稍安。庚子,以少府監韓璀為鄜
 州刺史、鄜坊丹延節度使。是夕,上崩於大明宮之中和殿,享年四十三。時以暴崩,皆言內官陳弘志弒逆,史氏諱而不書。辛丑,宣遺詔。壬寅,移仗西內。



 五月丁酉,群臣上謚曰聖神章武孝皇帝,廟號憲宗。庚申,葬於景陵。



 史臣蔣系曰:憲宗嗣位之初,讀列聖實錄,見貞觀、開元故事,竦慕不能釋卷,顧謂丞相曰:「太宗之創業如此,玄宗之致理如此,既覽國史,乃知萬倍不如先聖。當先聖之代,猶須宰執臣僚同心輔助,豈朕今日獨為理哉!」
 自是延英議政,晝漏率下五六刻方退。自貞元十年已後,朝廷威福日削,方鎮權重。德宗不委政宰相,人間細務,多自臨決,奸佞之臣,如裴延齡輩數人,得以錢穀數術進,宰相備位而已。及上自籓邸監國,以至臨御,訖於元和,軍國樞機,盡歸之於宰相。由是中外咸理,紀律再張,果能剪削亂階,誅除群盜。睿謀英斷,近古罕儔,唐室中興,章武而已。任異、鎛之聚斂,逐群、度於籓方,政道國經,未至衰紊。惜乎服食過當,閹豎竊發,茍天假之年,庶
 幾於理矣!



 贊曰:貞元失馭,群盜箕踞。章武赫斯,削平嘯聚。我有宰衡,耀德觀兵。元和之政,聞於頌聲。



\end{pinyinscope}