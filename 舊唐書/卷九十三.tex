\article{卷九十三}

\begin{pinyinscope}

 ○狄仁傑族曾孫兼謨王方慶姚璹弟班



 狄仁傑字懷英,並州太原人也。祖孝緒,貞觀中尚書左丞。父知遜,夔州長史。仁傑兒童時,門人有被害者,縣吏就詰之,眾皆接對,唯仁傑堅坐讀書。吏責之,仁傑曰:「黃
 卷之中,聖賢備在,猶不能接對,何暇偶俗吏,而見責耶!」後以明經舉,授汴州判佐。時工部尚書閻立本為河南道黜陟使,仁傑為吏人誣告,立本見而謝曰:「仲尼云:『觀過知仁矣』足下可謂海曲之明珠,東南之遺寶。」薦授並州都督府法曹。其親在河陽別業,仁傑赴並州,登太行山,南望見白雲孤飛,謂左右曰:「吾親所居,在此云下。」瞻望佇立久之,雲移乃行。仁傑孝友絕人,在並州,有同府法曹鄭崇質,母老且病,當充使絕域。仁傑謂曰:「太夫人
 有危疾,而公遠使,豈可貽親萬里之憂!」乃詣長史藺仁基,請代崇質而行。時仁基與司馬李孝廉不協,因謂曰:「吾等豈獨無愧耶?」由是相待如初。



 仁傑,儀鳳中為大理丞,周歲斷滯獄一萬七千人,無冤訴者。時武衛大將軍權善才坐誤斫昭陵柏樹,仁傑奏罪當免職。高宗令即誅之,仁傑又奏罪不當死。帝作色曰:「善才斫陵上樹,是使我不孝,必須殺之。」左右矚仁傑令出,仁傑曰:「臣聞逆龍鱗,忤人主,自古以為難,臣愚以為不然。居桀、紂時則
 難,堯、舜時則易。臣今幸逢堯、舜,不懼比千之誅。昔漢文時有盜高廟玉環,張釋之廷諍,罪止棄市。魏文將徙其人,辛毗引裾而諫,亦見納用。且明主可以理奪,忠臣不可以威懼。今陛下不納臣言,瞑目之後,羞見釋之、辛毗於地下。陛下作法,懸之象魏,徒流死罪,俱有等差。豈有犯非極刑,即令賜死?法既無常,則萬姓何所措其手足?陛下必欲變法,請從今日為始。古人云:『假使盜長陵一抔土,陛下何以加之?』今陛下以昭陵一株柏殺一將軍,
 千載之後,謂陛下為何主?此臣所以不敢奉制殺善才,陷陛下於不道。」帝意稍解,善才因而免死。居數日,授仁傑侍御史。時司農卿韋機兼領將作、少府二司,高宗以恭陵玄宮狹小,不容送終之具,遣機續成其功。機於埏之左右為便房四所,又造宿羽、高山、上陽等宮,莫不壯麗。仁傑奏其太過,機竟坐免官。左司郎中王本立恃寵用事,朝廷懾懼,仁傑奏之,請付法寺,高宗特原之。仁傑奏曰:「國家雖乏英才,豈少本立之類,陛下何惜罪人而
 虧王法?必欲曲赦本立,請棄臣於無人之境,為忠貞將來之誡。」本立竟得罪,由是朝廷肅然。



 尋加朝散大夫,累遷度支郎中。高宗將幸汾陽宮,以仁傑為知頓使。並州長史李沖玄以道出妒女祠,俗云盛服過者必致風雷之災,乃發數萬人別開御道。仁傑曰:「天子之行,千乘萬騎,風伯清塵,雨師灑道,何妒女之害耶?」遽令罷之。高宗聞之,嘆曰:「真大丈夫也!」



 俄轉寧州刺史,撫和戎夏,人得歡心,郡人勒碑頌德。御史郭翰巡察隴右,所至多所按
 劾。及入寧州境內,耆老歌刺史德美者盈路。翰既授館,召州吏謂之曰:「入其境,其政可知也。願成使君之美,無為久留。」州人方散。翰薦名於朝,徵為冬官侍郎,充江南巡撫使。吳、楚之俗多淫祠,仁傑奏毀一千七百所,唯留夏禹、吳太伯、季札、伍員四祠。



 轉文昌右丞,出為豫州刺史。時越王貞稱兵汝南事敗,緣坐者六七百人,籍沒者五千口,司刑使逼促行刑。仁傑哀其詿誤,緩其獄,密表奏曰:「臣欲顯奏,似為逆人申理;知而不言,恐乖陛下存
 恤之旨。表成復毀,意不能定。此輩咸非本心,伏望哀其詿誤。」特敕原之,配流豐州。豫囚次於寧州,父老迎而勞之曰:「我狄使君活汝輩耶!」相攜哭於碑下,齋三日而後行。豫囚至流所,復相與立碑頌狄君之德。



 初,越王之亂,宰相張光輔率師討平之。將士恃功,多所求取,仁傑不之應。光輔怒曰:「州將輕元帥耶?」仁傑曰:「亂河南者,一越王貞耳。今一貞死而萬貞生。」光輔質其辭,仁傑曰:「明公董戎三十萬,平一亂臣,不戢兵鋒,縱其暴橫,無罪之
 人,肝腦塗地,此非萬貞何耶?且兇威協從,勢難自固,及天兵暫臨,乘城歸順者萬計,繩墜四面成蹊。公奈何縱邀功之人,殺歸降之眾?但恐冤聲騰沸,上徹於天。如得尚方斬馬劍加於君頸,雖死如歸。」光輔不能詰,心甚銜之。還都,奏仁傑不遜,左授復州刺史。入為洛州司馬。



 天授二年九月丁酉,轉地官侍郎、判尚書、同鳳閣鸞臺平章事。則天謂曰:「卿在汝南時,甚有善政,欲知譖卿者乎?」仁傑謝曰:「陛下以臣為過,臣當改之;陛下明臣無過,臣之
 幸也。臣不知譖者,並為善友,臣請不知。」則天深加嘆異。



 未幾,為來俊臣誣構下獄。時一問即承者例得減死,來俊臣逼協仁傑,令一問承反。仁傑嘆曰:「大周革命,萬物唯新,唐朝舊臣,甘從誅戮。反是實!」俊臣乃少寬之。判官王德壽謂仁傑曰:「尚書必得減死。德壽意欲求少階級,憑尚書牽楊執柔,可乎?」仁傑曰:「若何牽之?」德壽曰:「尚書為春官時,執柔任其司員外,引之可也。」仁傑曰:「皇天后土,遣仁傑行此事!」以頭觸柱,流血被面,德壽懼而謝焉。
 既承反,所司但待日行刑,不復嚴備。仁傑求守者得筆硯,拆被頭帛書冤,置綿衣中,謂德壽曰:「時方熱,請付家人去其綿。」德壽不之察。仁傑子光遠得書,持以告變。則天召見,覽之而問俊臣。俊臣曰:「仁檔不免冠帶,寢處甚安,何由伏罪?」則天使人視之,俊臣遽命仁傑巾帶而見使者。乃令德壽代仁傑作謝死表,附使者進之。則天召仁傑,謂曰:「承反何也?」對曰:「向若不承反,已死於鞭笞矣。」「何為作謝死表?」曰「臣無此表。」示之,乃知代署也。故得免死。貶
 彭澤令。武承嗣屢奏請誅之,則天曰:「朕好生惡殺,志在恤刑。渙汗已行,不可更返。」



 萬歲通天年,契丹寇陷冀州,河北震動,徵仁傑為魏州刺史。前刺史獨孤思莊懼賊至,盡驅百姓入城,繕修守具。仁傑既至,悉放歸農畝,謂曰:「賊猶在遠,何必如是。萬一賊來,吾自當之,必不關百姓也。」賊聞之自退,百姓咸歌誦之,相與立碑以紀恩惠。俄轉幽州都督。



 神功元年,入為鸞臺侍郎、同鳳閣鸞臺平章事,加銀青光祿大夫,兼納言。仁傑以百姓西戍疏
 勒等四鎮,極為凋弊,乃上疏曰:



 臣聞天生四夷,皆在先王封疆之外。故東拒滄海,西隔流沙,北橫大漠,南阻五嶺,此天所以限夷狄而隔中外也。自典籍所紀,聲教所及,三代不能至者,國家盡兼之矣。此則今日之四境,已逾於夏、殷者也。詩人矜薄伐於太原,美化行於江、漢,則是前代之遠裔,而國家之域中。至前漢時,匈奴無歲不陷邊,殺掠吏人。後漢則西羌侵軼漢中,東寇三輔,入河東上黨,幾至洛陽。由此言之,則陛下今日之士宇,過於
 漢朝遠矣。若其用武荒外,邀功絕域,竭府庫之實,以爭磽確不毛之地,得其人不足以增賦,獲其土不可以耕織。茍求冠帶遠夷之稱,不務固本安人之術,此秦皇、漢武之所行,非五帝、三皇之事業也。若使越荒外以為限,竭資財以騁欲,非但不愛人力,亦所以失天心也。昔始皇窮兵極武,以求廣地,男子不得耕於野,女子不得蠶於室,長城之下,死者如亂麻,於是天下潰叛。漢武追高、文之宿憤,藉四帝之儲實,於是定朝鮮,討西域,平南越,
 擊匈奴,府庫空虛,盜賊蜂起,百姓嫁妻賣子,流離於道路者萬計。末年覺悟,息兵罷役,封丞相為富民侯,故能為天所祐也。昔人有言:「與覆車同軌者未嘗安。」此言雖小,可以喻大。



 近者國家頻歲出師,所費滋廣,西戍四鎮,東戍安東,調發日加,百姓虛弊。開守西域,事等石田,費用不支,有損無益,轉輸靡絕,杼軸殆空。越磧逾海,分兵防守,行役既久,怨曠亦多。昔詩人云:「王事靡盬不能藝稷黍。」「豈不懷歸,畏此罪罟。念彼蒸人,涕零如雨。」此則前
 代怨思之辭也。上不是恤,則政不行而邪氣作。邪氣作,則蟲螟生而水旱起。若此,雖禱祀百神,不能調陰陽矣。方今關東饑饉,蜀、漢逃亡,江、淮以南,徵求不息。人不復業,則相率為盜,本根一搖,憂患不淺。其所以然者,皆為遠戍方外,以竭中國,爭蠻貊不毛之地,乖子養蒼生之道也。



 昔漢元納賈捐之之謀而罷珠崖郡,宣帝用魏相之策而棄車師之田,豈不欲慕尚虛名,蓋憚勞人力也。近貞觀年中,克平九姓,冊李思摩為可汗,使統諸部者,
 蓋以夷狄叛則伐之,降則撫之,得推亡固存之義,無遠戍勞人之役。此則近日之令典,經邊之故事。竅見阿史那斛瑟羅,陰山貴種,代雄沙漠,若委之四鎮,使統諸蕃,封為可汗,遣御寇患,則國家有繼絕之美,荒外無轉輸之役。如臣所見,請捐四鎮以肥中國,罷安東以實遼西,省軍費於遠方,並甲兵於塞上,則恆、代之鎮重,而邊州之備實矣。況綏撫夷狄,蓋防其越逸,無侵侮之患則可矣。何必窮其窟穴,與螻蟻計校長短哉!



 且王者外寧必
 有內憂,蓋為不勤修政故也。伏惟陛下棄之度外,無以絕域未平為念。但當敕邊兵謹守備,蓄銳以待敵,待其自至,然後擊之,此李牧所以制匈奴也。當今所要者,莫若令邊城警守備,遠斥候,聚軍實,蓄威武。以逸待勞,則戰士力倍;以主御客,則我得其便。堅壁清野,則冠無所得。自然賊深入必有顛躓之慮,淺入必無虜獲之益。如此數年,可使二虜不擊而服矣。



 仁傑又請廢安東,復高氏為君長,停江南之轉輸,慰河北之勞弊,數年之後,可
 以安人富國。事雖不行,識者是之。尋檢校納言,兼右肅政臺御史大夫。



 聖歷初,突厥侵掠趙、定等州,命仁傑為河北道元帥,以便宜從事。突厥盡殺所掠男女萬餘人,從五回道而去。仁傑總兵十萬追之不及。便制仁傑河北道安撫大使。時河朔人庶,多為突厥逼脅,賊退後懼誅,又多逃匿。仁傑上疏曰:



 臣聞朝廷議者,以為契丹作梗,始明人之逆順,或因迫脅,或有願從,或受偽官,或為招慰,或兼外賊,或是土人,跡雖不同,心則無別。誠以山
 東雄猛,由來重氣,一顧之勢,至死不回。近緣軍機,調發傷重,家道悉破,或至逃亡,剔屋賣田,人不為售,內顧生計,四壁皆空。重以官典侵漁,因事而起,取其髓腦,曾無心媿。修築池城,繕造兵甲,州縣役使,十倍軍機。官司不矜,期之必取,枷杖之下,痛切肌膚。事迫情危,不循禮義,愁苦之地,不樂其生。有利則歸,且圖賒死,此乃君子之愧辱,小人之常行。人猶水也,壅之則為泉,疏之則為川,通塞隨流,豈有常性。昔董卓之亂,神器播遷,及卓被誅,
 部曲無赦,事窮變起,毒害生人,京室丘墟,化為禾黍。此由恩不普洽,失在機先。臣一讀此書,未嘗不廢卷嘆息。今以負罪之伍,必不在家,露宿草行,潛竄山澤。赦之則出,不赦則狂,山東群盜,緣茲聚結。臣以邊塵暫起,不足為憂,中土不安,以此為事。臣聞持大國者不可以小道,理事廣者不可以細分。人主恢弘,不拘常法,罪之則眾情恐懼,恕之則反側自安。伏願曲赦河北諸州,一無所問。自然人神道暢,率土歡心,諸軍凱旋,得無侵擾。



 制從
 之。軍還,授內史。



 聖歷三年,則天幸三陽宮,王公百僚咸經侍從,唯仁傑特賜宅一區,當時恩寵無比。是歲六月,左玉鈐衛大將軍李楷固、右武威衛將軍駱務整討契丹餘眾,擒之,獻俘於含樞殿。則天大悅,特賜楷固姓武氏。楷固、務整,並契丹李盡忠之別帥也。初,盡忠之作亂,楷固等屢率兵以陷官軍,後兵敗來降,有司斷以極法。仁傑議以為楷固等並有驍將之才,若恕其死,必能感恩效節。又奏請授其官爵,委以專征。制並從之。及楷固
 等凱旋,則天召仁傑預宴,因舉觴親勸,歸賞於仁傑。授楷固左玉鈐衛大將軍,賜爵燕國公。



 則天又將造大像,用功數百萬,令天下僧尼每日人出一錢,以助成之。仁傑上疏諫曰:



 臣聞為政之本,必先人事。陛下矜群生迷謬,溺喪無歸,欲令像教兼行,睹相生善。非為塔廟必欲崇奢,豈令僧尼皆須檀施?得伐尚舍,而況其餘。今之伽藍,制過宮闕,窮奢極壯,畫繢盡工,寶珠殫於綴飾,環材竭於輪奐。工不使鬼,止在役人,物不天來,終須地出,不
 損百姓,將何以求?生之有時,用之無度,編戶所奉,常若不充,痛切肌膚,不辭箠楚。游僧一說,矯陳禍福,翦發解衣,仍慚其少。亦有離間骨肉,事均路人,身自納妻,謂無彼我。皆托佛法,詿誤生人。里陌動有經坊,闤闠亦立精舍。化誘倍急,切於官徵;法事所須,嚴於制敕。膏腴美業,倍取其多;水碾莊園,數亦非少。逃丁避罪,並集法門,無名之僧,凡有幾萬,都下檢括,已得數千。且一夫不耕,猶受其弊,浮食者眾,又劫人財。臣每思惟,實所悲痛。



 往在
 江表,像法盛興,梁武、簡文,舍施無限。及其三淮沸浪,五嶺騰煙。列剎盈衢,無救危亡之禍;緇衣蔽路,豈有勤王之師!比年已來,風塵屢擾,水旱不節,征役稍繁。家業先空,瘡痍未復,此時興役,力所未堪,伏惟聖朝,功德無量,何必要營大像,而以勞費為名。雖斂僧錢,百未支一。尊容既廣,不可露居,覆以百層,尚憂未遍,自餘廓廡,不得全無。又云不損國財,不傷百姓,以此事主,可謂盡忠?臣今思惟,兼採眾議,咸以為如來設教,以慈悲為主,下濟
 群品,應是本心,豈欲勞人,以存虛飾?當今有事,邊境未寧,宜寬征鎮之徭,省不急之費。設令雇作,皆以利趨,既失田時,自然棄本。今不樹稼,來歲必饑,役在其中,難以取給。況無官助,義無得成,若費官財,又盡人力,一隅有難,將何救之!



 則天乃罷其役。是歲九月,病卒,則天為之舉哀,廢朝三日,贈文昌右相,謚曰文惠。



 仁傑常以舉賢為意,其所引拔桓彥範、敬暉、竇懷貞、姚崇等,至公卿者數十人。初,則天嘗問仁傑曰:「朕要一好漢任使,有乎?」仁
 傑曰:「陛下作何任使?」則天曰:「朕欲待以將相。」對曰:「臣料陛下若求文章資歷,則今之宰臣李嶠、蘇味道亦足為文吏矣。豈非文士齷齪,思得奇才用之,以成天下之務者乎?」則天悅曰:「此朕心也。」仁傑曰:「荊州長史張柬之,其人雖老,真宰相才也。且久不遇,若用之,必盡節於國家矣。」則天乃召拜洛州司馬。他日,又求賢。仁傑曰:「臣前言張柬之,猶未用也。」則天曰:「已遷之矣。」對曰:「臣薦之為相,今為洛州司馬,非用之也。」又遷為
 秋官侍郎,後竟召為相。柬之果能興復中宗,蓋仁傑之推薦也。



 仁傑嘗為魏州刺史,人吏為立生祠。及去職,其子景暉為魏州司功參軍,頗貪暴,為人所惡,乃毀仁傑之祠。長子光嗣,聖歷初為司府丞,則天令宰相各舉尚書郎一人,仁傑乃薦光嗣。拜地官員外郎,蒞事稱職,則天喜而言曰:「祁奚內舉,果得其人。」開元七年,自汴州刺史轉揚州大都督府長史,坐贓貶歙州別駕卒。



 初,中宗在房陵,而吉頊、李昭德皆有匡復讜言,則天無復闢意。唯仁傑每從容奏對,
 無不以子母恩情為言,則天亦漸省悟,竟召還中宗,復為儲貳。初,中宗自房陵還宮,則天匿之帳中,召仁傑以廬陵為言。仁傑慷慨敷奏,言發涕流,遽出中宗謂仁傑曰:「還卿儲君。」仁傑降階泣賀,既已,奏曰:「太子還宮,人無知者,物議安審是非?」則天以為然,乃復置中宗於龍門,具禮迎歸,人情感悅。仁傑前後匡復奏對,凡數萬言,開元中,北海太守李邕撰為《梁公別傳》,備載其辭。中宗返正,追贈司空;睿宗追封梁國公。仁傑族曾孫兼謨。



 兼謨,
 登進士第。祖郊、父邁,仕官皆微。兼謨元和末解褐襄陽推官,試校書郎,言行剛正,使府知名。憲宗召為左拾遺,累上書言事,歷尚書郎。長慶、太和中,歷鄭州刺史,以治行稱,入為給事中。開成初,度支左藏庫妄破漬污縑帛等贓罪,文宗以事在赦前不理。兼謨封還敕書,文宗召而諭之曰:「嘉卿舉職,然朕已赦其長官,典吏亦宜在宥。然事或不可,卿勿以封敕為艱。」遷御史中丞。謝日,文宗顧謂之曰:「御史臺朝廷綱紀,臺綱正則朝廷理,朝廷正
 則天下理。凡執法者,大抵以畏忌顧望為心,職業由茲不舉。卿梁公之後,自有家法,豈復為常常之心哉!」兼謨謝曰:「朝法或未得中,臣固悉心彈奏。」會江西觀察使吳士矩違額加給軍士,破官錢數十萬計。兼謨奏曰:「觀察使守陛下土地,宣陛下詔條,臨戎賞軍,州有定數。而士矩與奪由己,盈縮自專,不唯貽弊一方,必致諸軍援例。請下法司,正行朝典。」士矩坐貶蔡州別駕。兼謨尋轉兵部侍郎。明年,檢校工部尚書、太原尹,充河東節度使。會
 昌中,累歷方鎮,卒。



 王方慶,雍州咸陽人也,周少司空石泉公褒之曾孫也。其先自瑯邪南度,居於丹陽,為江左冠族。褒北徙入關,始家咸陽焉。祖軿,隋衛尉丞。伯父弘讓,有美名,貞觀中為中書舍人。父弘直,為漢王元昌友,畋獵無度,乃上書切諫,其略曰:「夫宗子維城之托者,所以固邦家之業也。大王功無任城戰克之效,行無河間樂善之譽,爵高五等,邑富千室,當思答極施之洪慈,保無疆之永祚。其為
 計者,在乎修德,冠屨《詩》《禮》,畋獵史傳。覽古人成敗之所由,鑒既往存亡之異跡,覆前戒後,居安慮危。奈何列騎齊驅,交橫壟畝,野有游客,巷無居人。貽眾庶之憂,逞一情之樂,從禽不息,實用寒心。」元昌覽書而遽止。漸見疏斥,轉荊王友。龍朔中卒。



 方慶年十六,起家越王府參軍。嘗就記室任希古受《史記》、《漢書》。希古遷為太子舍人,方慶隨之卒業。永淳中,累遷太僕少卿。則天臨朝,拜廣州都督。廣州地際南海,每歲有昆侖乘舶以珍物與中國
 交市。舊都督路元睿冒求其貨,昆侖懷刃殺之。方慶在任數載,秋毫不犯。又管內諸州首領,舊多貪縱,百姓有詣府稱冤者,府官以先受首領參餉,未嘗鞫問。方慶乃集止府僚,絕其交往,首領縱暴者悉繩之,由是境內清肅。當時議者以為有唐以來,治廣州者無出方慶之右。有制褒之曰:「朕以卿歷職著稱,故授此官,既美化遠聞,實副朝寄。令賜卿雜採六十段,並瑞錦等物,以彰善政也。」



 證聖元年,召拜洛州長史,尋加銀青光祿大夫,封石
 泉縣男。萬歲登封元年,轉並州長史,封瑯邪縣男。未行,遷鸞臺侍郎、同鳳閣鸞臺平章事。俄轉鳳閣侍郎,依舊知政事。



 神功元年七月,清邊道大總管建安王攸宜破契丹凱還,欲以是月詣闕獻俘。內史王及善以為將軍入城,例有軍樂,既今上孝明高皇帝忌月,請備而不奏。方慶奏曰:「臣按禮經,但有忌日,而無忌月。晉穆帝納後,用九月九日,是康帝忌月,於時持疑不定。下太常,禮官荀訥議稱:『禮只有忌日,無忌月。若有忌月,即有忌時、忌
 歲,益無理據。』當時從訥所議。軍樂是軍容,與常不等,臣謂振作於事無嫌。」則天從之。則天嘗幸萬安山玉泉寺,以山逕危懸,欲御腰輿而上。方慶諫曰:「昔漢元帝嘗祭廟,出便門,御樓船,光祿勛張猛奏曰:『乘船危,就橋安。』元帝乃從橋,即前代舊事。今山徑危險,石路曲狹,上瞻駭目,下視寒心,比於樓船,安危不等。陛下蒸人父母,奈何踐此畏塗?伏望停輿駐蹕。」則天納其言而止。是歲,改封石泉子。



 時有制,每月一日於明堂行告朔之禮。司禮博
 士闢閭仁住奏議,其略曰:「經史正文,無天子每月告朔之事,唯《禮記玉藻》云:『天子聽朔於南門之外。』其每月告朔者,諸侯之禮也。臣謹按《禮論》及《三禮義宗》、《江都集禮》、《貞觀禮》、《顯慶禮》及《祠令》,無天子每月告朔之事。若以為無明堂故無告朔之禮,有明堂即合告朔,則周、秦有明堂而無天子每月告朔之事。臣等參求,既無其禮,不可習非,以天子之尊而用諸侯之禮。」方慶又奏議,其略曰:「明堂,天子布政之宮也。謹按《穀梁傳》云:『閏者,附月之餘
 日,天子不以告朔。』『非禮也。閏以正時,時以作事,事以厚生,生人之道,於是乎在矣。不告閏朔,棄時政也。』臣據此文,則天子閏月亦告朔矣。寧有他月而廢其禮乎?先儒舊說,天子行事,一年十八度入明堂矣。大享不問卜,一入也;每月告朔,十二入也;四時迎氣,四入也;巡狩之年,一入也。今禮官議唯歲首一入耳,與先儒既異,在臣不敢同。宋朝何承天纂集其文,以為《禮論》,雖加編次,事則闕如。梁代崔靈恩撰《三禮義宗》,但捃摭前儒,因循故事
 而已。隋煬帝命學士撰《江都集禮》,只抄撮舊禮,更無異文。《貞觀》、《顯慶禮》及《祠令》不言告朔者,蓋為歷代不傳,所以其文乃闕。各有緣由,不足依據。今禮官引為明證,在臣誠實有疑。」則天又令春官廣集眾儒,取方慶、仁住所奏議,以定得失。時成均博士吳揚善、太學博士郭山惲等奏:「按《周禮》及《三傳》,皆有天子告朔之禮,秦滅《詩》、《書》,由是告朔禮廢。望依方慶議。」有制從之。



 則天以方慶家多書籍,嘗訪求右軍遺跡。方慶奏曰:「臣十代從伯祖羲之
 書,先有四十餘紙,貞觀十二年,太宗購求,先臣並已進之。唯有一卷見今在。又進臣十一代祖導、十代祖洽、九代祖珣、八代祖曇首、七代祖僧綽、六代祖仲寶、五代祖騫、高祖規、曾祖褒,並九代三從伯祖晉中書令獻之已下二十八人書,共十卷。」則天御武成殿示群臣,仍令中書舍人崔融為《寶章集》,以敘其事,復賜方慶,當時甚以為榮。



 方慶又舉:「令杖『期喪、大功未葬,不預朝賀;未終喪,不預宴會。』比來朝官不遵禮法,身有哀容,陪預朝會,手
 舞足蹈,公違憲章,名教既虧,實玷皇化。伏望申明令式,更禁斷。」從之。方慶漸以老疾,乞從閑逸,乃授麟臺監修國史。及中宗立為東宮,方慶兼檢校太子左庶子。



 聖歷二年一日,則天欲季冬講武,有司稽緩,延入孟春。方慶上疏曰:「謹按《禮記月令》:『孟冬之月,天子命將帥講武,習射御角力。』此乃三時務農,一時講武,以習射御,角校才力,蓋王者常事,安不忘危之道也。『孟春之月,不可以稱兵。』兵者,甲胄干戈之總名。兵金性,克木,春盛德在木,而
 舉金以害盛德,逆生氣。『孟春行冬令,則水潦為敗,雪霜大摯,首種不入。』蔡邕《月令章句》云:『太陰新休,少陽尚微,而行冬令以導水氣,故水潦至而敗生物也。雪霜大摯,折陽者也。太陰乾時,雨雪而霜,故大傷首種。首種,謂宿麥也,麥以秋種,故謂之首種。入,收也,春為沍寒所傷,故至夏麥不成長也。』今孟春講武,是行冬令,以陰政犯陽氣,害發生之德。臣恐水潦敗物,霜雪損稼,夏麥不登,無所收入也。伏望天恩不違時令,至孟冬教習,以順天道。」
 手制答曰:「比為久屬太平,多歷年載,人皆廢戰,並悉學文。今者用整兵威,故令教習。卿以春行冬令,則水潦為敗,舉金傷木,則便害發生。循覽所陳,深合典禮,若違此請,乃月令虛行。佇啟直言,用依來表。」是歲,正授太子左庶子,封石泉公,餘並如故,俸料同職事三品,兼侍皇太子讀書。方慶又上言:「謹按史籍所載,人臣與人主言及上表,未有稱皇太子名者。當為太子皇儲,其名尊重,不敢指斥,所以不言。晉尚書僕射山濤啟事,稱皇太子而
 不言名。濤中朝名士,必詳典故,其不稱名,應有憑準。朝官尚猶如此,宮臣歸則不疑。今東宮殿及門名,皆有觸犯,臨事論啟,回避甚難。孝敬皇帝為太子時,改弘教門為崇教門;沛王為皇太子,改崇賢館為崇文館。皆避名諱,以遵典禮。此即成例,足為軌模。伏望天恩因循舊式,付司改換。」制從之。



 長安二年五月卒,贈袞州都督,謚曰貞。中宗即位,以宮僚之舊,追贈吏部尚書。方慶博學好著述,所撰雜書凡二百餘卷。尤精《三禮》,好事者多詢訪
 之。每所酬答,咸有典據,故時人編次,名曰《禮雜答問》。聚書甚多,不減秘閣,至於圖畫,亦多異本。諸子莫能守其業,卒後尋亦散亡。長子光輔,開元中官至潞州刺史。少子晙,工書知名,尤善琴棋,而性多嚴整,官至殿中侍御史。



 姚璹,字令璋,散騎常侍思廉之孫也。少孤,撫弟妹以友愛稱。博涉經史,有才辯。永徽中明經擢第。累補太子宮門郎。與司議郎孟利貞等奉令撰《瑤山玉彩》書,書成,遷
 秘書郎。調露中,累遷至中書舍人,封吳興縣男。則天臨朝,遷夏官侍郎。坐從父弟敬節同徐敬業之亂,貶桂州都督府長史。時則天雅好符瑞,璹至嶺南,訪諸山川草樹,其名號有「武」字者,皆以為上膺國姓,列奏其事。則天大悅,召拜天官侍郎。善於選補,時人稱之。



 長壽二年,遷文昌左丞、同鳳閣鸞臺平章事。自永徽以後,左、右史雖得對仗承旨,仗下後謀議,皆不預聞。璹以為帝王謨訓,不可暫無紀述,若不宣自宰相,史官無從得書。乃表請
 仗下所言軍國政要,宰相一人專知撰錄,號為時政記,每月封送史館。宰相之撰時政記,自璹始也。是歲九月,坐事轉司賓少卿,罷知政事。延載初,擢拜納言。有司以璹從父弟犯法,奏言不合更為侍臣。璹上言:「昔王敦稱兵犯順,王導仍典樞機;嵇康戮於晉朝,嵇紹忠於晉室。竊惟前古,尚不為疑;今奉聖恩,豈由臣下。必以體例有乖,伏請甘從屏退。」則天曰:「此乃我意,卿復何言!但當盡忠,無聽浮說。」



 時武三思率蕃夷酋長,請造天樞於端門
 外,刻字紀功,以頌周德,璹為督作使。證聖初,璹加秋官尚書、同平章事。是歲,明堂災,則天欲責躬避正殿,璹奏曰:「此實人火,非曰天災。至如成周宣榭,卜代愈隆;漢武建章,盛德彌永。臣又見《彌勒下生經》云,當彌勒成佛之時,七寶臺須臾散壞。睹此無常之相,便成正覺之因。故知聖人之道,隨緣示化,方便之利,博濟良多。可使由之,義存於此。況今明堂,乃是布政之所,非宗廟之地,陛下若避正殿,於禮未為得也。」左拾遺劉承慶廷奏云:「明堂
 宗祀之所,今既被焚,陛下宜輟朝思過。」璹又持前議以爭之,則天乃依璹奏。先令璹監造天樞,至是以功當賜爵一等。璹表請回贈父一官,乃追贈其父豫州司戶參軍處平為博州刺史。天後將封嵩岳,命璹總知撰儀注,並充封禪副使。及重造明堂,又令璹充使督作,以功加銀青光祿大夫。



 時有大石國使請獻獅子,璹上疏諫曰:「獅子猛獸,唯止食肉,遠從碎葉,以至神都,肉既難得,檢為勞費。陛下以百姓為心,慮一物有失,鷹犬不蓄,漁獵總
 停。運不殺以闡大慈,垂好生以敷至德,凡在翾飛蠢動,莫不感荷仁恩。豈容自菲薄於身,而厚資給於獸,求之至理,必不然乎」。疏奏,遽停來使。又九鼎初成,制令黃金千兩塗之。璹進諫曰:「夫鼎者神器,貴在質樸自然,無假別為浮飾。臣觀其狀,先有五彩輝煥,錯雜其間,豈待金色,方為炫耀?」則天又從之。



 尋屬契丹犯塞,命梁王武三思為榆關道安撫大使、璹為副使以備之。及還,坐事,神功初左授益州大都督府長史。蜀中官吏多貪暴,璹屢
 有發手適,奸無所容。則天嘉之,降璽書勞之曰:「夫嚴霜之下,識貞松之擅奇,疾風之前,知勁草之為貴。物既有此,人亦宜哉。卿早荷朝恩,委任斯重。居中作相,弘益已多,防邊訓兵,心力俱盡。歲寒無改,終始不渝。乃眷蜀中,氓俗殷雜,久缺良守,弊於侵漁,政以賄成,人無措足。是用命卿出鎮,寄茲存養。果能攬轡澄清,下車整肅。吏不敢犯,奸無所容,前後糾手適,蓋非一緒。貪殘之伍,屏跡於列城;剽奪之儔,遁形於外境。詎勞期月,康此黎元,言念德
 聲,良深嘉尚。宜布瑯邪之化,當以豫州為法。」則天又嘗謂侍臣曰:「凡為長官,能清自身者甚易,清得僚吏者甚難。至於姚璹,可謂兼之矣。」



 時新都丞硃待闢坐贓至死,逮捕系獄。待闢素善沙門理中,陰結諸不逞,因待闢以殺璹為名,擬據巴蜀為亂。人密表告之者,制令璹按其獄。璹深持之,事涉疑似引而誅死者,僅以千數。則天又令洛州長史宋元爽、御史中丞霍獻可等重加詳覆,亦無所發明。逮系獄數百人,不勝酷毒,遞相附會,以就反
 狀。因此籍沒者復五十餘家,其餘稱知反配流者亦十八九,道路冤之。監察御史袁恕己劾奏其事。則天初令璹與恕己對定,又尋令罷推。俄拜地官尚書。歲餘,轉冬官尚書,仍西京留守。長安中,累表乞骸骨,制聽致仕,進爵為伯。遇官名復舊,為工部尚書。神龍元年卒,遺令薄葬,贈越州都督,謚曰成。



 弟班,少好學,以勤苦自立。舉明經,累除定、汴、滄、虢、豳等五州刺史,加銀青光祿大夫,轉秦州刺史。以善政有聞,璽書褒美,賜絹百匹。神龍元年,
 累封宣城郡公,三遷太子詹事,仍兼左庶子。時節愍太子舉事不法,班前後上書進諫。今載四事:



 其一曰:臣聞賈誼曰:「選天下之端士,孝悌博聞有道術者,使與太子居處出入。故太子見正事,聞正言,行正道,左右前後皆正人也。夫習與正人居之,不能無正;習與不正人居之,不能無不正。太子既冠成人,免於保傅之嚴,則有記過之史。徹膳之宰,進善之旌,誹謗之木,敢諫之鼓,瞽史誦箴,大夫進謀,故習與智長,化與心成。夫教得而左右正,
 則太子正矣;太子正而天下定矣。」臣又聞之,木從繩則正,後從諫則聖。善言古者,所以驗於今。伏惟殿下睿德洪深,天姿聰敏,近代成敗,前古安危,莫不懸鑒在心,動合典禮。臣以庸朽,濫居輔弼,虛備耳目,叨預股肱,輒薦塵露,庶裨山海。伏以內置作坊,工巧得入宮闈之內、禁衛之所,或言語內出,或事狀外通,小人無知,不識輕重,因為詐偽,有玷徽猷。臣望並付所司,以停宮內造作。如或要須役造,猶望宮外安置,庶得工匠不於宮禁出入。



 其二曰:臣聞漢文帝身衣弋綈,足履革舄;齊高帝欄檻用銅者,皆易以鐵。經侯帶玉具劍環珮以過魏,太子不視,經侯曰:「魏國亦有寶乎?」太子曰:「主信臣忠,魏之寶也。」經侯委劍珮而去。太子使追還之,謂曰:「珠玉珍玩,寒不可衣,饑不可食,無遺我賊。」經侯杜門不出。臣觀聖賢經籍,務以簡素為貴;皇王政化,皆以菲薄為德。伏惟殿下留心恭儉,靡尚浮奢。臣愚猶望損之又損之,居簡以行簡,減省造作,節量用度。



 其三曰:臣聞銀牖銅樓,宮闈嚴
 秘,門閤來往,皆有簿歷。殿下時有所須,唯門司宣令,或恐奸偽之輩,因此妄為增減,脫有文狀舛錯,事理便即差違。且近日呂升之便乃代署宣敕,伏賴殿下睿敏,當即覺其奸偽,自餘臣下庸淺,豈能深辨真虛?望墨令及覆事行下,並用內印印畫署之後,冀得免有詐假,乃是長久規模。臣又聞之,忠臣事君,有犯而無隱;明主馭下,納諫以進德。故《書》云:「有言逆於志,必求諸道;有言順於心,必求諸非道。」伏惟殿下仁明昭著,聖敬日躋,探幽洞
 微,窮神索隱。事之善惡,毫厘靡差;理有危疑,錙銖無爽。臣以庸謬,叨侍春闈,職居獻替,豈敢緘默!



 其四曰:臣聞聖人不專其德,賢智必有所師。故曰:與善人言,如入芝蘭之室,久自芬芳;與不善人言,如火銷膏,不覺而盡。今司經見無學士,供奉未有侍讀,伏望時因視膳,奏請置人。所冀講席談筵,務盡忠規之道;披文手適句,方資審諭之勤。臣又聞臣之事主,必盡乃誠;君之進賢,務求忠讜。伏惟殿下養德儲闈,以端靜為務;恭膺守器,以學業為
 先。經所以立行修身,史所以諳識成敗。雅誥既習,忠孝乃成,傳記方通,安危斯辨。知父子君臣之道,識古今鑒戒之規,經史為先,斯乃急務。至於工巧造作,僚吏直司,實為末事,無足勞慮。臣以庸淺,獻替是司,臣而不言,負譴聖日,言而獲罪,是所甘心。伏願留意經書,簡略細事,一蒙採納,萬殞無辭。乞降儲明,俯矜狂瞽。



 疏奏,太子雖稱善,竟不悛革。太子敗,詔遣索其宮中,得班諫書,中宗嘉其切直。時宮臣皆貶黜,唯班擢拜右散騎常侍。歲餘,
 遷秘書監。



 睿宗即位,累授戶部尚書,轉太子賓客。先天二年,加金紫光祿大夫,復拜戶部尚書。班與兄弟璹,數年間俱為定州刺史、戶部尚書,時人榮之。開元二年卒,年七十四。班嘗以其曾祖察所撰《漢書訓纂》,多為後之注《漢書》者隱沒名氏,將為己說;班乃撰《漢書紹訓》四十卷,以發明舊義,行於代。



 史臣曰:天子有諍臣七人,雖無道不失其天下。致廬陵復位,唐祚中興,諍由狄公,一人以蔽。或曰:許之太甚。答
 曰:當革命之時,朋邪甚眾,非推誠竭力,致身忘家者,孰能與於此乎!仁傑流死不避,骨鯁有彰,雖逢好殺無辜,能使終畏大義。竟存天下,豈不然乎!王方慶乾城南海,羽冀東宮,臺閣樞機,無不功濟,所謂君子不器者也。茍非文學,斯焉取斯。璹成都布政,始卒不侔;相國上章,或否或中。且焚明堂而避正殿,固諍何多;黜唐頌而立天樞,一言非措。矧乃妄求符瑞,已失忠貞;精擇楚茅,難裨過咎。不常其德,罔畏承羞。班規諫有才,牧守多善,儲幄
 之任,可謂得人。



 贊曰:犯顏忤旨,返政扶危。是人雜事,狄能有之。終替武氏,克復唐基。功之莫大,人無以師。方慶之才,周旋特立。璹也無常,班能操執。



\end{pinyinscope}