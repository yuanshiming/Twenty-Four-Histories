\article{卷一百二十八}

\begin{pinyinscope}

 ○薛嵩弟崿嵩子平嵩族子雄令狐彰子建運通田神功弟神玉侯希逸李正己子納納子師古師道宗人洧附



 薛嵩,絳州萬泉人。祖仁貴,高宗朝名將,封平陽郡公。父楚玉,為範陽、平盧節度使。嵩少以門廕,落拓不事家產,
 有膂力,善騎射,不知書。自天下兵起,束身戎伍,委質逆徒。廣德元年,東都平,時皇太子為天下兵馬元帥,遣僕固懷恩東收河朔。嵩為賊守相州,聞賊朝義兵潰,王師至,嵩惶惑迎拜於懷恩馬前,懷恩釋之,令守舊職。時懷恩二心已萌。懷恩平河朔旋,乃奏嵩及田承嗣、張忠志、李懷仙分理河北道;詔遂以嵩為相州刺史,充相、衛、洺、邢等州節度觀察使,承嗣鎮魏州,忠志鎮恆州,懷仙鎮幽州,各據數州之地。時多事之後,姑欲安人,遂以重寄委嵩。
 嵩感恩奉職,數年間,管內粗理,累遷檢校右僕射。大歷八年正月卒。詔遣弟崿知留後,累加崿太子少師。大歷十年正月丁酉,昭義軍兵馬使裴志清盜所將兵逐崿,舉眾歸田承嗣以叛。崿奔於洺州,上表乞入朝,許之。至京,素服於銀臺門待罪,詔釋之。



 嵩子平,年十二,為磁州刺史。嵩卒,軍吏欲用河北故事,脅平知留後務,平偽許之,讓於叔父崿,一夕以喪歸。及免喪,累授右衛將軍,在南衙凡三十年。宰相杜黃裳深器之,薦為汝州刺
 史、兼御史中丞,理有能名。元和七年,淮西用兵,自左龍武大將軍授兼御史大夫、滑州刺史、鄭滑節度觀察等使,累有戰功。滑州城西距黃河二里,每歲常為水患。平詢訪得古河道,接衛州黎陽縣界。平率魏博節度使田弘正同上聞,開古河南北長十四里,決舊河以分水勢,滑人遂無水患。居鎮六年,入為左金吾大將軍。未幾,復為鄭滑節度觀察使。及平李師道,朝廷以東平十二州析為三道,以淄、青、齊、登、萊五州為平盧軍,以平為節度、
 觀察等使,仍押新羅、渤海兩蕃使。



 長慶元年,幽鎮叛,杜叔良統橫海全軍討伐不勝,王庭湊圍牛元翼於深州。棣州為賊所窘,朝廷乃委平以偏師援棣州,平即遣將李叔佐以兵五百救之。居數月,刺史王稷饋給稍薄,兵士怨怒,叔佐不能戢,宵潰而歸。仍推突將馬狼兒為帥,行及青城鎮,劫鎮將李自勸,並其眾;次至博昌鎮,復劫其鎮兵,共得七千餘人,徑逼青州城。城中兵士不敵,平悉府庫並家財募二千精卒,逆擊之,仍先以騎兵掩其
 家屬輜重,賊眾惶惑反顧,因大敗。狼兒與其同惡十數輩脫身竄匿,餘黨降,稍後者斬於鞠場。明日,狼兒亦就擒戮,脅從者放歸田里。詔加右僕射,進封魏國公,由是遠近畏伏平之威略。



 在鎮六周歲,兵甲完利,井賦均一。至是入覲,百姓遮道乞留,數日乃得出。時人以為近日節制,罕有其比。寶歷元年,歸朝,進加檢校左僕射、兼戶部尚書。逾月,復檢校司空,兼河中絳隰節度觀察等使。大和二年,復以晉州,慈州隸河中,益兵三千人,加平檢校
 司徒。在河中凡六年,召拜太子太保。明年,上疏乞老,以司徒致仕,居一年卒,冊贈太傅。嵩族子雄,初為嵩屬吏,知衛州事,嵩歿,特詔授衛州刺史。魏博節度田承嗣誘為亂,雄不從,承嗣遣刺客盜殺之。



 令狐彰,京兆富平人也。遠祖自燉煌徙家焉,代有冠冕。父濞,天寶中任鄧州錄事參軍,以清白聞,本道採訪使宋鼎引為判官。初任範陽縣尉,通幽州人女,生彰,及秩滿,留彰於母氏,彰遂少長範陽。倜儻有膽氣,涉獵書傳,
 粗知文義,善弓矢,乃策名從軍,事安祿山。天福中,以軍功累遷至左衛員外郎將。



 安祿山叛逆,以本官隨賊黨張通儒赴京師,通儒偽署為城內左街使。王師收復二京,隨通儒等遁走河朔,又陷逆賊史思明,偽署為博州刺史及滑州刺史,令統數千兵戍滑臺。彰感激忠義,思立名節,乃潛謀歸順。會中官楊萬定監滑州軍,彰遂募勇士善於水者,俾乘夜涉河,達表奏於萬定,請以所管賊一將兵馬及州縣歸順,萬定以聞。自祿山構逆,為賊守者,未有
 舉州向化,肅宗得彰表,大悅,賜書慰勞。時彰移鎮杏園渡,遂為思明所疑,思明乃遣所親薛岌統精卒圍杏園攻之。彰乃明示三軍,曉以逆順,眾心感附,咸悉力為用。與賊兵戰,大破之,潰圍而出,遂以麾下將士數百人隨萬定入朝。肅宗深獎之,禮甚優厚,賜甲第一區、名馬數匹,並帷帳什器頗盛,拜御史中丞,兼滑州刺史、滑毫魏博等六州節度,仍加銀青光祿大夫,鎮滑州,委平殘寇。及史朝義滅,遷御史大夫,封霍國公,尋加檢校工部尚
 書。未幾,檢校右僕射,餘並如故。



 彰在職,風化大行。滑州瘡痍未復,城邑為墟,彰以身勵下,一志農戰,內檢軍戎,外牧黎庶,法令嚴酷,人不敢犯。數年間,田疇大闢,庫藏充積,歲奉王稅及修貢獻,未嘗暫闕。時犬戎犯邊,徵兵防秋。彰遣屬吏部統營伍,自滑至京之西郊,向二千餘里,甲士三千人,率自賚糧,所過州縣,路次供擬,皆讓而不受,經閭里不犯秋毫,識者稱之。然性識猜阻,人有忤意,不加省察,輒至斃踣,此其短也。臨終,手疏辭表,誡子
 以忠孝守節,又舉能自代。表曰:



 臣自事陛下,得備籓守,受恩則重,效節未終,長辭聖朝,痛入心骨,臣誠哀懇,頓首頓首。臣受性剛拙,亦能包含。頃因魚朝恩將掠亳州,遂與臣結怨,當其縱暴,臣不敢入朝,專聽天誅,即欲奔謁。及魚朝恩死,即臣屬疾苦,又遭家艱,力微眼暗,行動須人,拜舞不能,數月有闕。欲請替辭退,即日望稍瘳,冀得康強,榮歸朝覲。自冬末舊疾益重,瘡腫又生,氣息奄奄,遂期殞歿。不遂一朝天闕,一拜龍顏,臣禮不終,忠誠
 莫展,臣之大罪,下慚先代,仰愧聖朝。臣竭誠事上,誓立大節,天地神明,實知臣心。心不遂行,言發自痛。當使倉糧錢絹羊馬牛畜一切已上,並先有部署;三軍兵士,州縣官吏等,各恭舊職,祗待聖恩。臣伏見吏部尚書劉晏及工部尚書李勉,知識忠貞,堪委大事,伏願陛下速令檢校,上副聖心。臣男建等,性不為非,行亦近道,今勒歸東都私第,使他年為臣報國,下慰幽魂。臨歿昏亂,伏表哀咽。



 上覽表,嗟悼久之。特下詔褒美曰:



 中衛社稷,外修
 疆事,合於一體,以靖庶邦,其在有終,謂之不朽。觀前代文武通賢,有匡時戡難,撻於大化,不忘時君,未嘗不嘉尚而流嘆也。今有忠烈之臣彰,剛直形外,純和積中,本於孝敬,輔以才略,統制籓閫,服勞王家。往以母老,躬於就養,豈不戀闕,以茲曠年。及苴麻在艱,優諭權奪,踴絕傷足,淚盡喪明,入覲之期,良願莫遂。想其風彩,久軫顧懷,遽見淪沒,用深追悼。嗟乎!方疾之時,以情自疏,無所有隱,見之於詞。復節守常,條上軍簿,請擇良帥,命於中
 朝。乃令遺胤,爰歸東洛,教忠以報國,約禮以居喪。古人所謂生不交利,死不屬其子,夫豈遠哉!節概誠亮,高絕無鄰,喟然感傷,鑒寐增慟。有以見東州士大夫勤王尊主之志,用嘉其休,可以垂範,宣付史館,式昭名臣。



 子建、運、通。



 建,大歷四年十二月,彰遣入朝,特加兼御史中丞,歸滑州。及彰卒,滑三軍逼奪情禮,建守死不從,舉家歸京師。服闋,累轉至右龍虎軍使。德宗以涇原兵亂,出幸奉天,建方教射於軍中,遂以四百人隨駕為後殿。至奏
 天,以建為行在中軍鼓角使。幸梁州,轉行在右廂兵馬使、右羽林大將軍、兼御史大夫。興元元年六月,加檢校左散騎常侍、行在都知兵馬使、左神武大將軍。建妻李氏,恆帥寶臣女也,建惡,將棄之,乃誣與人庸教生邢士倫奸通。建召士倫榜殺之,因逐其妻。士倫母聞,不勝其痛,卒。李氏奏請按劾,詔令三司詰之。李氏及奴婢款證,被誣頗明白,建方自首伏。建會赦免坐。德宗詔曰:「子育黎元,未能禁暴,在予之責,用軫於懷。宜輟常膳五百千文,
 充葬士倫母子。其父既衰耄,至無所歸,良深矜念,委京兆尹厚加存恤。」貞元四年七月,以前官為右領軍大將軍。五年三月,以專殺不辜,德宗念舊勛,特容貸之;復陳訴,詞甚虛罔,遂貶旋州別駕同正,卒於貶所。貞元六年九月,贈右領軍大將軍。十年,贈揚州大都督。



 運為東都留守將,逐賊出郊,其日有劫轉運絹於道者,杜亞以運豪家子,意其為之,乃令判官穆員及從事張弘靖同鞠其事。員與弘靖皆以運職在牙門,必不為盜,抗請不按。
 亞不聽,而怒斥逐員等,令親事將武金鞫之。金笞箠運從者十餘人,一人笞死,九人不勝考掠自誣,竟無贓狀。亞具以聞,請流運於嶺表。德宗令侍御史李元素、刑部員外崔從質、大理司直盧士瞻三司覆按運獄,既竟,明運跡非行盜,以曾捕掠人於家,配流歸州。武金肆虐作威,教人通款,配流建州。後歲餘,齊抗捕得劫轉運絹賊郭鵠、硃瞿曇等七人及贓絹,詔令杜亞與留臺同劾之,皆首伏。然終不原運,運死於歸州,眾冤之。



 通,元和中,宰
 相李吉甫奏曰:「臣伏見代宗朝滑州節度使令狐彰臨終上表,悉以土地兵甲籍上朝廷,遣諸子隨表歸闕。代宗以彰遺表宣示百僚,當時在位者聞之,無不感嘆。今有次子通在。臣每感彰同進河朔諸鎮,付子傳孫,無不燻灼數代;唯彰忠義感激,奉國忘家,遣子入朝,以土地歸於先帝。貞元中,長子建坐事死於施州,幼子運亦無罪流於歸州,欲使忠義之人,何所激勸?今通幸存,得遇明聖,伏乞陛下召之與語,如堪用,望垂獎錄。」憲宗念彰
 之忠,即授通贊善大夫,出為宿州刺史。時討淮、蔡,用為泗州刺史。歲中改壽州團練使、檢校御史中丞。每與賊戰,必虛張虜獲,得賊數人,即為露布上之。宰相武元衡笑而不奏;如有敗衄,即不敢上聞。後為賊所攻,境上城柵並陷,通走固州城,閉壁不出。憲宗遣李文通往宣慰,度其將至,遂令代通,貶為昭州司戶,移撫州司馬。十四年,徵為右衛將軍,制下,給事中崔植封還制書,言通前刺壽州失律,不宜遽加獎任。憲宗令宰相宣喻門下,言
 通父有功於國,不宜逐棄其子,制命方行。歲餘,出為淄州刺史。長慶初,入為左衛大將軍,卒。



 田神功,冀州人也,。家本微賤。天寶末,為縣裏胥,會河朔兵興,從事幽、薊。上元元年,為平盧節度都知兵馬使,兼鴻臚卿,於鄭州破賊四千餘眾,生擒逆賊大將四人,牛馬器械不可勝數。尋為鄧景山所引,至揚州,大掠百姓商人資產,郡內比屋發掘略遍,商胡波斯被殺者數千人。二年二月,生擒逆賊劉展,送於闕下。以擒展功,累遷
 檢校工部尚書、兼御史大夫、汴宋等八州節度使。大歷三年三月,朝京師,獻馬十匹、金銀器五十件、繒彩一萬匹。時郭子儀入朝,請宴宰臣等於私第,神功效其請,亦以許之。尋加檢校右僕射,赴尚書省視事,特詔宰臣已下百官送上,仍加知省事以寵之。神功忠樸干勇,當時所稱。八年冬,復覲闕廷,遘疾,信宿而終。上悼惜,為之徹樂,廢朝三日;贈司徒,賻絹一千匹、布五百端;特許百官吊喪,賜屏風茵褥於靈座,並賜千僧齋以追福,至德已
 來,將帥不兼三事者,哀榮無比。



 弟神玉,自曹州刺史權汴州留後。大歷十年正月,加檢校兵部郎中、兼御史中丞,為汴州刺史,知汴州節度觀察留後事並河陽、澤潞等兵馬,直據淇門,會李承昭討魏博田承嗣。十一年卒,詔滑州李勉代之。



 侯希逸,平盧人也。少習武藝。天寶末,安祿山反,署其腹心徐歸道為平盧節度。希逸時為平盧裨將,率兵與安東都護王玄志襲殺歸道,
 使以聞,詔以玄志為平盧節度使。乾元元年冬,玄志病卒,軍人共推立希逸為平盧軍使,朝廷因授節度使。既數為賊所迫,希逸率勵將士,累破賊徒向潤客、李懷仙等。既淹歲月,且無救援,又為奚虜所侵,希逸拔其軍二萬餘人,且行且戰,遂達於青州。會田神功、能元皓於兗州,青州遂陷於希逸,詔就加希逸為平盧、淄青節度使。自是迄今,淄青節度皆帶平盧之名也。



 希逸初領淄青,甚著聲稱,理兵務農,遠近美之。寶應元年,與諸節度同討襲史朝義,平之,加檢校工部尚書,
 賜實封,圖形凌煙閣。以私艱去職。大歷十一年九月,起復檢校尚書右僕射、上柱國,封淮陽郡王。後漸縱恣,政事怠惰,尤崇奉釋教,且好畋游,興功創寺宇,軍州苦之。永泰元年,因與巫者夜宿於城外,軍士乃閉之不納。希逸奔歸朝廷,拜檢校右僕射,久之,加知省事,遷司空。詔出而卒,廢朝三日,贈太保。



 李正己,高麗人也。本名懷玉,生於平盧。乾元元年,平盧節度使王玄志卒,會有敕遣使來存問,懷玉恐玄志子
 為節度,遂殺之,與軍人共推立侯希逸為軍帥。希逸母即懷玉姑也。後與希逸同至青州,累至折沖將軍,驍健有勇力。寶應中,眾軍討史朝義,至鄭州。回紇方強暴恣橫,諸節度皆下之,正己時為軍候,獨欲以氣吞之。因與其角逐,眾軍聚觀,約曰:「後者批之。」既逐而先,正己擒其領而批其背,回紇尿液俱下,眾軍呼笑,虜慚,由是不敢為暴。



 節度使侯希逸即其外兄也,用為兵馬使。正己沉毅得眾心,希逸因事解其職,軍中皆言其非罪,不當廢。
 會軍人逐希逸,希逸奔走,遂立正己為帥,朝廷因授平盧淄青節度觀察使、海運押新羅渤海兩蕃使、檢校工部尚書、兼御史大夫、青州刺史,賜今名。尋加檢校尚書右僕射,封饒陽郡王。大歷十一年十月,檢校司空、同中書門下平章事。十三年,請入屬籍,從之。為政嚴酷,所在不敢偶語。初有淄、青、齊、海、登、萊、沂、蜜、德、棣等州之地,與田承嗣、令狐彰、薛嵩、李寶臣、梁崇義更相影響。大歷中,薛嵩死,及李靈曜之亂,諸道共攻其地,得者為己邑,正
 己復得曹、濮、徐、兗、鄆,共十有五州,內視同列,貨市渤海名馬,歲歲不絕。法令齊一,賦稅均輕,最稱強大。嘗攻田承嗣,威震鄰辭。歷檢校司空、左僕射、兼御史大夫,加平章事、太子太保、司徒。



 後自青州徙居鄆州,使子納及腹心之將分理其地。建中後,畏懼朝廷,多不自安。聞將築汴州,乃移兵屯濟陰,晝夜教習為備。河南騷然,天下為憂,羽檄馳走,徵兵以益備。又於徐州增兵,以扼江淮,於是運輸為之改道。未幾,發疽卒,時年四十九。子納擅總
 兵政,秘之數月,乃發喪。納阻兵,興元元年四月,歸順,方贈正己太尉。



 納少時,正己遣將兵備秋,代宗召見,嘉之,自奉禮郎超拜殿中丞、兼侍御史,賜紫金魚袋。歷檢校倉部郎中,兼總父兵,奏署淄州刺史。正己將兵擊田承嗣,奏署節度觀察留後。尋遷青州刺史,又奏署行軍司馬,兼曹州刺史、曹濮徐兗沂海留後,又加御史大夫。



 建中初,正己、田悅、梁崇義、張惟岳皆反。二年,正己卒,納秘喪,統父眾,仍復為亂。比會悅於濮陽,遣大將衛俊將兵
 一千救悅,為河東節度使馬燧敗於洹水,殺傷殆盡。詔諸軍誅之,納從叔父洧以徐州,李士真以德州,及棣州李長卿,皆以州歸順。納以彭城險厄,又怒洧背宗,乃悉兵圍之。詔宣武軍節度劉洽與諸軍救之,大敗納兵於城下。後將兵於濮陽,洽攻破其城外。納自城上見洽,涕泣悔罪,遣判官房說以其弟經、男成務朝京師,請因洽從順。會中使宋鳳朝見之,謂納計蹙,欲誅破之以為己功,奏請無舍,上乃械說等系禁中。納遂歸鄆州,復與李
 希烈、硃滔、王武俊、田悅合謀皆反,偽稱齊王,建置百官。及興元之降罪己詔,納乃效順,詔加檢校工部尚書、平盧軍節度、淄青等州觀察使。無幾,檢校右僕射、同中書門下平章事。時希烈圍陳州,納遣兵與諸軍奮擊,大破之,因解圍。加檢校司空,封五百戶。貞元初,升鄆州為大都督府,改授長史。年三十四,薨於位,廢朝三日,贈賻有差。



 子師古,累奏至青州刺史。貞元八年,納死,軍中以師古代其位而上請,朝廷因而授之。起復右金吾大將
 軍同正、平盧及青淄齊節度營田觀察、海運陸運押新羅渤海兩蕃使。成德軍節度王武俊率師次於德、棣二州,將取蛤朵及三汊城。棣州之鹽池與蛤朵歲出鹽數十萬斛,棣州之隸淄青也,其刺史李長卿以城入硃滔,而蛤朵為納所據,因城而戍之,以專鹽利。其後武俊以敗硃滔功,以德、棣二州隸之,蛤朵猶為納戍。納初於德州南跨河而城以守之,謂之三汊,交田緒以通魏博路,而侵掠德州,為武俊患。及納卒,師古繼之。武俊以其年
 弱初立,舊將多死,心頗易之,乃率眾兵以取蛤朵、三汊為名,其實欲窺納之境。師古令棣州降將趙鎬拒之。武俊令其子士清將兵先濟於滴河,會士清營中火起,軍驚,惡之,未進。德宗遣使諭旨,武俊即罷還。師古毀三汊口城,從詔旨。師古雖外奉朝命,而嘗畜侵軼之謀,招集亡命,必厚養之,其得罪於朝而逃詣師古者,因即用之。其有任使於外者,皆留其妻子,或謀歸款於朝,事洩,族其家,眾畏死而不敢異圖。



 貞元十年五月,師古服闋,加
 檢校禮部尚書。十二年正月,檢校尚書右僕射。十一月,師古丁母憂,起復左金吾上將軍同正。十五年正月,師古、杜佑、李欒妾媵並為國夫人。十六年六月,與淮南節度使杜佑同制加中書門下平章事。及德宗遺詔下,告哀使未至,義成軍節度使李元素以與師古鄰道,錄遺詔報師古,以示無外。師古遂集將士,引元素使者謂曰:師古近得邸吏狀,具承聖躬萬福。李元素豈欲反,乃忽偽錄遺詔以寄。師古三代受國恩,位兼將相,見賊不可
 以不討。」遂杖元素使者,遽出後以討元素為名,冀因國喪以侵州縣。俄聞順宗即位,師古乃罷兵。,後累官至檢校司徒、兼侍中。卒贈太傅。



 師道,師古異母弟。其母張忠志女。師道時知密州事,師古死,其奴不發喪,潛使迎師道於密而奉之。朝命久未至,師道謀於將吏,或欲加兵於四境,其判官高沐固止之。乃請進兩稅,守鹽法,申官員,遣判官崔承寵、孔目官林英相繼奏事。時杜黃裳作相,欲乘其未定也,以計分削之,憲宗以蜀川方擾,不能
 加兵於師道。元和元年七月,遂命建王審遙領節度,授師道檢校左散騎常侍、兼御史大夫,權知鄆州事,棄淄青節度留後。十月,加檢校工部尚書,兼鄆州大都督府長史,充平盧軍及淄青節度副大使,知節度事、管內支度營田觀察處置、陸運海運押新羅渤海兩蕃等使。自正己至師道,竊有鄆、曹等十二州,六十年矣。懼眾不附己,皆用嚴法制之。大將持兵鎮於外者,皆質其妻子;或謀歸款於朝,事洩,其家無少長皆殺之。以故能劫其眾,
 父子兄弟相傳焉。五年七月,檢校尚書右僕射。



 十年,王師討蔡州,師道使賊燒河陰倉,斷建陵橋。初,師道置留邸於河南府,兵諜雜以往來,吏不敢辨。因吳元濟北犯汝、鄭,郊畿多警,防禦兵盡戍伊闕,師道潛以兵數十百人內其邸,謀焚宮闕而肆殺掠。既烹牛饗眾矣,明日將出,會有小將楊進、李再興者詣留守呂元膺告變,元膺追伊闕兵圍之,半日不敢進攻。防禦判官王茂元殺一人而後進,或有毀其墉而入者。賊眾突出殺人,圍兵奔
 駭,賊得結伍中衢,內其妻子於囊橐中,以甲胄殿而行,防禦兵不敢追。賊出長夏門,轉掠郊墅,東濟伊水,入嵩山。元膺誡境上兵重購以捕之。數月,有山棚鬻鹿於市,賊遇而奪之,山棚走而徵其黨,或引官軍共圍之谷中,盡獲之。窮理得其魁首,乃中嶽寺僧圓靜,年八十餘,嘗為史思明將,偉悍過人。初執之,使巨力者奮錘,不能折脛。圓靜罵曰:「鼠子,折人腳猶不能,敢稱健乎!」乃自置其足教折之。臨刑,乃曰:「誤我事,不得使洛城流血。」死者
 凡數十人。留守御將二人、都亭驛卒五人、甘水驛卒三人,皆潛受其職署,而為之耳目,自始謀及將敗,無知者。初,師道多買田於伊闕、陸渾之間,凡十所處,欲以舍山而衣食之。有訾嘉珍、門察者,潛部分之,以屬圓靜,以師道錢千萬偽理嵩山之佛光寺,期以嘉珍竊發時舉火於山中,集二縣山棚人作亂。及窮按之,嘉珍、門察,乃賊武元衡者,元膺具狀以聞。及誅吳元濟,師道恐懼,上表乞聽朝旨,請割三州並遣長子入侍宿衛,詔許之。



 師
 道識暗,政事皆決於群婢。婢有號蒲大姊、袁七娘者,為謀主,乃言曰:「自先司徒以來,有此十二州,奈何一日無苦而割之耶!今境內兵士數十萬人,不獻三州,不過發兵相加,可以力戰,戰不勝,乃議割地,未晚也。」師道從之而止,表言軍情不葉,乃詔諸軍討伐。十年十二月,武寧軍節度使李願遣將王智興擊破師道之眾九千,斬首二千餘級,獲牛馬四千,遂至平陰。十一年十一月,加師道司空,仍遣給事中柳公綽往宣慰,且觀所為,欲寬
 容之。師道茍以遜順為辭,長惡不悛。十三年七月,滄州節度使鄭權破淄青賊於齊州福城縣,斬首五百餘級。十月,徐州節度使李愬、兵馬使李祐於兗州魚臺縣破賊三千餘人。魏博節度使田弘正率本軍自陽劉渡河,距鄆州九十里下營,再接戰,破賊三萬餘眾,生擒三千人,收器械不可勝紀。陳許節度使李光顏於濮陽縣界破賊,收斗門城、杜莊柵。田弘正復於故東阿縣界破賊五萬。諸軍四合,累下城柵。



 師道使劉悟將兵當魏博軍,既
 敗,數令促戰。師未進,乃使奴召悟計事。悟知其來殺己,乃稱病不出,召將吏謀曰:「魏博兵強,乘勝出戰,必敗吾師,不出則死。今天子所誅,司空一人而已。悟與公等皆被驅逐就死地,何如轉禍為福,殺其來使,以兵趣鄆州,立大功以求富貴。」眾皆曰:「善。」乃迎其使而斬之,遂賚師道追牒,以兵趣鄆州。及夜,至門,示以師道追牒,乃得入。兵士繼進,至球場,因圍其內城,以火攻之,擒師道而斬其首,送於魏博軍,元和十四年二月也。是月,弘正獻於
 京師,天子命左右軍如受馘儀,先獻於太廟效社,憲宗御興安門受之,百僚稱賀。



 初,東軍諸道行營節度擒逆賊將夏侯澄等共四十七人,詔曰:「附麗兇黨,拒抗王師,國有常刑,悉合誅戮。朕以久居污俗,皆被脅從,況討伐已來,時日不幾,縱懷轉禍之計,未有效款之由,情似可矜,朕不忍殺。況三軍百姓,孰非吾人,詔令頒行,罪止師道。方欲拯於塗炭,是用活其性命,誠為屈法,庶使知恩。並宜特從釋放,仍令卻遞送至魏博及義成行營,各委
 節度收管驅使。如父母血屬猶在賊中,或羸老疾病情切歸還者,仍量事優當放去,務備相全貸,何所疑留。」及澄等至行營,賊覘知傳告,叛徒皆感朝恩,由是劉悟得行其謀焉。



 師道妻魏氏及小男並配掖庭。堂弟師賢、師智配流春州,侄弘巽配流雷州。詔分其十二州為三節度,俾馬總、薛平、王遂分鎮焉。仍命宰臣崔群撰碑以紀其績。國家自天寶末安祿山首亂兩河,至寶應元年王師平史朝義,其將薛嵩、李懷仙、田承嗣、李寶臣等受偽
 命分領州郡,朝廷厭兵,因僕固懷恩請,就加官爵。及侯希逸為軍人逐出,正己又據齊、魯之地,既而遞相膠固,聯結姻好,職貢不入,法令不加,率以為常。仍皆署其子為副大使,父死子立,則以三軍之請聞,亦有為大將所殺而自立者。自安、史以後,迄至於貞元,朝廷多務優容,每聞擅襲,因而授之,以故六十餘年,兩河號為反側之俗。憲宗知人善任,削平亂跡,兩河復為王土焉。師道妻魏氏,元和十五年出家為尼。



 洧,正己從父兄也。正己用為
 徐州刺史。正己死,子納犯宋州,洧以其州歸順,加御史大夫,封潮陽郡王,食實封二百戶,充招諭使。初,洧遣攝巡官崔程奉表至京師,令口奏並白宰相:「徐州恐不能獨當賊,若得徐、海、沂三州節度都團練使,即必立功。況海、沂兩州,亦並為賊納所據,非國家州縣。其刺史王涉、馬萬通等,洧並素與之約,若有詔命,冀必成功。」程乍自外到闕,以為宰相一也,乃先以其言白張鎰,鎰言於盧杞。杞怒程不先白己,故洧所請不行,杞妨公害私,皆此
 類也。及李納遣兵攻徐州,劉洽與諸將擊退之,賊勢未衰,始加洧徐、海、沂都團練觀察使,尋加密州。時海、密州皆為賊所據,不受洧命。旋加洧檢校戶部尚書。未幾,疽發背,稍平,乃大具糜餅,飯僧於市,洧乘平肩輿自臨其場,市人歡呼,洧驚,疽潰於背而卒,贈左僕射。



 史臣曰:自安、史亂離,河朔割據,雖外尊朝旨,而內蓄奸謀。薛嵩祖父,國之名將,及身濡足賊廷,既沐國恩,尚存家法,守土奉職,終身一心,果有令人,克全餘慶。彰居喪
 循禮,有士子之風;馭眾權謀,著將軍之業。中外善政,終始令名,成功不居,告老致仕,方之者鮮矣。背逆歸國,治兵牧民,上表推誠,舉賢代己,時稱能善始善終者也。建志稟遺訓,克全令名,不能終保功業,惜哉!神功忠勇,竟著勛名;希逸荒狂,自失茅土。師道祖父弟兄,盜據青、鄆,得計則潛圖兇逆,失勢則偽奉朝旨,向背任情,數十年矣。或問曰:師古之前,三帥而不滅;師道繼立,數年而亡者,何哉?答曰:納與師古,自運奸謀,躬臨戎事;朝廷任盧
 杞,以私妨公,致懷光變忠為逆,李納父子,宜其茍延。洎憲宗當朝,裴度為相,君臣道合,中外情通;師道外任諸奴,內聽群婢,軍民攜貳,家族滅亡,不亦宜乎!假息數年,猶為多矣,何所疑焉?



 贊曰:田神功勇能立勛,令狐彰死不失節。薛平振家世以顯揚,師道任臧獲則亡
 滅。



\end{pinyinscope}