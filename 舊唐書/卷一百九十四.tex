\article{卷一百九十四}

\begin{pinyinscope}

 ○李憕子源彭彭孫景讓張介然崔無詖盧奕蔣清顏杲卿子泉明薛願龐堅附張巡姚掞附許遠
 程千里袁光庭邵真符璘趙曄石演芬張名振附張伾甄濟劉敦儒高沐賈直言庾敬休辛讜



 李憕,太原文水人。父希倩,中宗神龍初,右臺監察御史。



 憕早聰敏,以明經舉,開元初為咸陽尉。時張說自紫微令、燕國公出為相州刺史、河北按察使,有洺州劉行善相人,說問:「寮採後誰貴達?」行乃稱憕及臨河尉鄭巖。說
 乃以女妻巖,妹婿陰行真女妻於憕。及說為並州長史、天兵軍大使,引憕常在幕下。九年,入為相,心妻又為長安尉。屬宇文融為御史,括田戶,奏知名之士崔希逸、咸暠業、宇文順、於孺卿、李宙及心妻為判官,攝監察御史,分路檢察,以課並遷監察御史。心妻驟歷兵、吏部郎中,給事中。憕有吏乾,明於幾案,甚有當官之稱。



 二十八年,為河南少尹。時蕭炅為尹,依倚權貴,蒞事多不法。憕以公直正之,人用系賴。又道士孫甑生以左道求進,托以修功德,
 往來嵩山,求請無度,心妻必挫之。炅及甑生患之,而構於朝廷。天寶初,出為清河太守。十一載,累轉河東太守、本道採訪。謁於行在所,改尚書右丞、京兆尹。十四載,轉光祿卿、東京留守,判尚書省事。



 其載十一月,安祿山反於範陽,人心震懼。玄宗遣安西節度封常清兼御史大夫為將,召募於東京以御之。憕與留臺御史中丞盧奕、河南尹達奚珣,綏輯將士,完繕城郭,遏其侵逼。遷心妻禮部尚書,依前留守。自逆徒發範陽,至渡河,令嚴,覘候計絕。
 及渡河,陷陳留、滎陽二郡,殺張介然、崔無詖,數日間已至都城下。祿山所統,皆蕃漢精兵,訓練已久;常清之眾,多市井之人,初不知戰。及兵交之後,被鐵騎唐突,飛矢如雨,皆魂懾色沮,望賊奔散。心妻謂奕曰:「吾曹荷國重寄,誓無避死,雖力不敵,其若官守何!」奕亦便許願守本司。於是憕居留守宅,奕獨居臺中。



 及常清西奔,祿山領其眾,椎鼓大呼,以入都城,殺掠數千人,箭及宮闕。然後住居於閑廄中,令擒心妻及奕、判官蔣清等三人,害之,以威
 於眾。祿山傳心妻、奕、清三人之首,以徇河北。信宿,至平原,太守顏真卿斬其使,浴其首,殮以木函,祭而瘞之,以聞。玄宗贈心妻司徒,仍與一子五品官。奕武部尚書,崔無詖工部尚書,各與一子官。蔣清文部郎中。



 心妻豐於產業,伊川膏腴,水陸上田,修竹茂樹,自城及闕口,別業相望,與吏部侍郎李彭年皆有地癖。鄭巖,天寶中仕至絳郡太守,入為少府監,田產亞於心妻。



 心妻有子十餘人,二子為僧,與心妻同遇害;二子彭、源,存焉。



 源,時年八歲,為賊所俘,轉
 徙流離,凡七八年。及史朝義走河北,洛陽故吏有義源者,贖之於民家。代宗聞之,授河南府參軍,轉司農寺主簿。以父死禍難,無心祿仕,誓不婚妻,不食酒肉。洛陽之北惠林寺,心妻之舊堅墅也,源乃依寺僧,寓居一室,依僧齋戒,人未嘗見其所習。先穴地為墓,預為終制,時時偃仰於穴中。



 長慶三年,御史中丞李德裕表薦之曰:「處士李源,即故禮部尚書、東都留守、贈司徒、忠烈公李心妻之少子。天與忠孝,嗣茲貞烈。以父死國難,哀纏終身,自司農
 寺主簿,絕心祿仕,垂五十年。暨於衰暮,多依惠林佛寺,本心妻之墅也。寺之正殿,即心妻之寢室,源過殿必趨,未嘗登踐。隨僧一食,已五十年。其端心執孝,無有不至。抱此貞節,棄於清朝,臣竊為陛下惜之。」詔曰:



 《禮》著死綏,《傳》稱握節,捐生守位,取重人倫。為義甚明,其風或替,言念於此,慨然興懷。而朝之公卿,有上言者,雲天寶之季,盜起幽陵,振蕩生靈,噬吞河洛。贈司徒、忠烈公心妻,處難居首,正色受屠,兩河聞風,再固危壁,首立殊節,到今稱之。其
 子源,有曾、閔之行,可貫於神明;有巢、由之風,可希於太古。山林以寄其跡,爵祿不入於心,泊然無營,五十餘載。夫褒忠可以勸臣節,旌孝可以激人倫,尚義可以警澆浮,敬老可以厚風俗。舉茲四者,大儆于時。是用擢自衡門,立於文陛,處以諫職,冀聞讜言,仍加印紱,式示光寵。可守左諫議大夫,賜緋魚袋。仍敕河南尹差官就所居敦諭遣發。



 穆宗尋令中使齎手詔、緋袍、牙笏、絹二百匹,往洛陽惠林寺宣賜。源受詔,對中使苦陳疾甚年高,不
 能趨拜,附表謝恩,其官告服色絹,皆辭不受。竟卒於寺。



 彭,以一子官累歷州縣令長。子宏,仕官愈卑。生三子:景讓、景莊、景溫,自元和後,相繼以進士登第。



 景讓,太和中為尚書郎,出為商州刺史。開成二年,入朝為中書舍人。二年十月,出為華州刺史、潼關防禦、鎮國軍使。四年,入為禮部侍郎。五年,選貢士李蔚,後至宰相;楊知退為尚書。大中朝,為襄州刺史、山南道節度使,入為吏部尚書。十一年,轉御史大夫。



 景讓有大志,事親以孝聞,正色立
 朝,言無避忌。為大夫時,宣宗舅鄭光卒,詔贈司徒,罷朝三日。景讓曰:「國舅雖親,朝典有素,無容過越。」乃上言曰:



 鄭光是陛下親舅,外族之愛,誠軫聖心,況皇太后哀切之時,理合加等,而賜之粟帛,隆其第宅,自家刑國,允謂合宜。今以輟朝之數,比於親王公主,則前例所無。縱有,亦不可施用。何者?先王制禮,所以防微。大凡人情,於外族則深,於宗屬則薄。所以先王制禮,割愛厚親,士庶猶然,況當萬乘!親王公主,宗屬也;舅氏,外族也。今朝廷公
 卿以至庶人,據《開元禮》,外祖父母及親舅喪服,小功五月,若親伯叔親兄弟即服齊縗周年。所以疏其外而密於內也。有天下者,尤不可使外戚強盛。故西漢有呂氏之侈,幾滅劉氏;國朝有則天之篡,殆革唐命。皆非一朝一夕,其所由來漸也。今鄭光輟朝日數,與親王公主同,設使陛下速改詔命,輟朝一日或兩日,示其升降有差,恩禮無僭,使四方見陛下欽明之德,青史傳陛下制度之文,垂之百王,播之芳烈。



 臣愚不肖,謬竊恩私,實願陛
 下處於堯、舜之上,羲、軒之列,所以甘心鼎鑊,伏進危言!



 優詔報之,乃罷兩日。景讓復為吏部尚書,卒,謚曰孝。



 景溫,登第後踐歷臺閣。咸通中,自工部侍郎出為華州刺史、潼關防禦、鎮國軍使。景莊,亦至達官。



 張介然者,蒲州猗氏人也。本名六朗。謹慎善籌算,為郡守在河、隴。及天寶中,王忠嗣、皇甫惟明、哥舒翰相次為節將,並委以營田支度等使。進位衛尉卿,仍兼行軍司馬,使如故。及加銀青光祿大夫,帶上柱國,因入奏稱旨,
 特加賜齎。介然乘間奏曰:「臣今三品,合列棨戟。若列於帝城,鄉里不知臣貴。臣,河東人也,請列戟於故鄉。」玄宗曰:「所給可列故鄉,京城佇當別賜。」介然拜謝而出,仍賜絹五百匹,令宴集閭里,以寵異之。本鄉列戟,自介然始也。哥舒翰追在西京,薦為少府監。



 安祿山將犯河洛,以介然為河南防禦使,令守陳留。陳留水陸所湊,邑居萬家,而素不習戰。介然至任數日,賊已渡河。雖率兵登城,兼守要害,虜騎十萬,所過殺戮,煙塵亙天,彌漫數十里。
 介然之眾,聞吹角鼓噪之聲,授甲不得,氣已奪矣,故至覆敗。



 初,玄宗以祿山起逆,於河南要路懸榜以購其首,又諭已殺其子慶宗等。祿山入陳留北郭,安慶緒見榜,白於祿山。祿山於輿中兩手撫胸,大哭數聲,曰:「我有何罪,已殺我兒?」便縱兇毒。前有陳留兵將降者向萬人,行列於路,祿山命其牙將殺戮皆盡,流血如川。乃斬介然於軍門,祿山氣乃稍解。頓軍於陳留郭下,以其將李庭望為節度鎮之。十五載,玄宗贈介然工部尚書,與一
 子五品官。



 崔無詖者,京兆長安人也。本博陵舊族。父從禮,中宗韋庶人之舅,景龍中衛尉卿。時中書令、酂國公蕭至忠才位素高,甚承恩顧,敕亡先女冥婚韋庶人亡弟。無詖婚至忠女,後為女家,中宗為兒家,供擬甚厚,時人為之語曰:「皇后嫁女,天子娶婦。」及韋庶人敗,至忠女亦死,無詖坐累久貶在外。



 開元中,為益州司馬。會楊國忠為新都尉,與之歡甚,國忠因事引用之,累轉陜郡太守、少府監、
 滎陽郡太守。安祿山率眾南向,無詖召募拒之。及賊陷陳留郡後,兇威轉盛,戈矛鼓角,驚駭城邑,兩宿及滎陽。乘城自墜如雨,故無詖及官吏,盡為賊所虜。賊以其將武令珣鎮之。



 盧奕,黃門監懷慎之少子也。與其兄奐齊名。大腹豐下,眉目疏朗。謹願寡欲,不尚輿馬,克己自勵。開元中,任京兆司錄參軍。天寶初,為鄠縣令、兵部郎中。所歷有聲,皆如奐之所治也。天寶八載,轉給事中。十一載,為御史中
 丞。始懷慎及奐並為中丞,父子三繼,清節不易,時人美之。奕留臺東都,又分知東都武部選事。



 十四載,安祿山犯東都,人吏奔散;奕在臺獨居,為賊所執,與李憕同見害。玄宗聞而愍之,贈兵部尚書。太常議謚,博士獨孤及議曰:



 盧奕剛毅樸忠,直方而清,勵精吏事,所居可紀。天寶十四載,洛陽陷沒。於時東京人士,狼狽鹿駭,猛虎磨牙而爭其肉,居位者皆欲保命而全妻子。或先策高足,爭脫羿彀;或不恥茍活,甘飲盜泉。奕獨正身守位義不
 去,以死全節誓不辱。勢窘力屈,以朝服就執,猶慷慨感憤,數賊梟獍之罪。觀者股慄,奕不變其色,而北面辭君,然後受害。雖古烈士,方之者鮮矣!



 或曰:「洛陽之存亡,操兵者實任其咎,非執法吏所能抗。師敗將奔,去之可也。委身寇仇,以死誰懟?」及以為不然。勇者御而忠者守,必社稷是衛,則死生以之。危而去之,是智免也,於忠何有?昔荀息殺身於晉,不食其言也;仲由結纓於衛,食焉不避其難也;玄冥勤其官而水死,守位而忘軀也;伯姬待
 保姆而火死,先禮而後身也。彼四人者,死之日,皆於事無補,夫豈愛死而賈禍也!以為死輕於義,故蹈義而捐生。古史書之,使事君者勸。然則祿山之亂,大於里克、孔悝;奕廉察之任,切於玄冥之官。分命所系,不啻於保姆;逆黨兵威,甚於水火。於斯時也,能與執干戈者同其戮力,挽之不來,推之不去,豈不以師可虧,免不可茍,身可殺,節不可奪。故全其特操於白刃之下,孰與夫懷安偷生者同其風哉!



 謹按謚法,圖國忘身曰「貞」,秉德遵業曰「
 烈」。奕執憲戎馬之間,志籓王室,可謂圖國;國危不能拯,而繼之以死,可謂忘身;歷官一十任,言必正,事必果,而清節不撓,去之若始至,可謂秉德;先黃門以直道佐時,奕嗣之以忠純,可謂遵業。請謚曰「貞烈」。



 從之。



 蔣清者,故吏部侍郎欽緒之子。舉明經,調補太子校書郎、鞏縣丞,盧奕留之憲府。清與諸兄溢、演、沇,知名於時。奕之被害,清亦死焉。



 顏杲卿,瑯邪臨沂人。世仕江左。五代祖之推,北齊黃門
 侍郎、修文館學士。齊亡入周,始家關內,遂為長安人焉。曾伯祖師古,貞觀中秘書監,自有傳。曾祖勤禮,崇文館學士。祖甫,曹王侍讀。父元孫,垂拱初登進士第,考功員外郎劉奇榜其詞策,文瑰俊拔,多士聳觀。歷官長安尉、太子舍人、亳州刺史卒。



 杲卿以廕受官,性剛直,有吏乾。開元中,為魏州錄事參軍,振舉綱目,政稱第一。天寶十四載,攝常山太守。時安祿山為河北、河東採訪使,常山在其部內。其年十一月,祿山舉範陽之兵詣闕。十二月
 十二日,陷東都。杲卿忠誠感發,懼賊遂寇潼關,即危宗社。時從弟真卿為平原太守,初聞祿山逆謀,陰養死士,招懷豪右,為拒賊之計。至是遣使告杲卿,相與起義兵,掎角斷賊歸路,以紓西寇之勢。杲卿乃與長史袁履謙、前真定令賈深、前內丘丞張通幽等,謀開土門以背之。時祿山遣蔣欽湊、高邈率眾五千守土門。杲卿欲誅欽湊,開土門之路。時欽湊軍隸常山郡,屬欽湊遣高邈往幽州未還,杲卿遣吏召欽湊至郡計事。是月二十二日
 夜,欽湊至,舍之於傳舍。會飲既醉,令袁履謙與參軍馮虔、縣尉李棲默、手力翟萬德等殺欽湊。中夜,履謙以欽湊首見杲卿,相與垂泣,喜事交濟也。是夜,稾城尉崔安石報高邈還至蒲城,即令馮虔、翟萬德與安石往圖之。詰朝,高邈之騎從數人至稾城驛,安石皆殺之。俄而邈至,安石紿之曰:「太守備酒樂於傳舍。」邈方據下馬,馮虔等擒而縶之。是日,賊將何千年自東都來趙郡,馮虔、萬德伏兵於醴泉驛,千年至,又擒之。即日縛二賊將還
 郡。杲卿遣子安平尉泉明及賈深、張通幽、翟萬德,函欽湊之首,械二賊,送於京師。



 至太原,節度使王承業留泉明、賈深等,寢杲卿之表。承業自上表獻之,以為己功。玄宗不之知,擢拜承業大將軍,牙官獲賞者百數。玄宗尋知杲卿之功,乃加衛尉卿、兼御史大夫,以袁履謙為常山太守,賈深為司馬。



 杲卿既斬賊將,收兵練卒,乃檄告河北郡縣,言朝廷以榮王為河北兵馬大元帥,哥舒翰為副,統眾三十萬,即出土門。郡縣聞之,皆殺賊守將,遠
 近響應,時十五郡皆為國家所守。時安祿山遣使傳李心妻、盧奕之首徇河北。至平原,真卿殺賊使,收藏心妻等首。清池尉賈載亦斬偽署景城守劉玄道,傳首於平原。饒陽郡守盧全誠亦據郡舉兵,會於真卿。時常山、平原二郡兵威大振。祿山方自率眾而西,已至陜,聞河北有變而還,乃命史思明、蔡希德率眾渡河。



 十五年正月,思明攻常山郡。城中兵少,眾寡不敵,御備皆竭。其月八日,城陷,杲卿、履謙為賊所執,送於東都。思明既陷常山,遂攻
 諸郡,鄴、廣平、鉅鹿、趙郡、上谷、博陵、文安、魏郡、信都,復為賊守。祿山見杲卿,面責之曰:「汝昨自範陽戶曹,我奏為判官,遂得光祿、太常二丞,便用汝攝常山太守,負汝何事而背我耶?」杲卿瞋目而報曰:「我世為唐臣,常守忠義,縱受汝奏署,復合從汝反乎!且汝本營州一牧羊羯奴耳,叨竊恩寵,致身及此,天子負汝何事而汝反耶?」祿山怒甚,令縛於中橋南頭從西第二柱,節解之,比至氣絕,大罵不息。



 是日杲卿幼子誕、侄詡及袁履謙,皆被先截
 手足,何千年弟在傍,含血噴其面,因加割臠,路人見之流涕。其年二月,李光弼、郭子儀之師自土門東下,復收常山郡。杲卿、履謙等妻女數百人,系之獄中,光弼破械出之,令行喪服,給遣周厚。



 至德二年冬,廣平王收復兩京,史思明以河朔歸國。時真卿為蒲州刺史,乃令泉明於河北求訪血屬。杲卿妹先適故榆次令張景儋,妹女流落賊中,泉明一女亦落賊中,俱索購錢三萬。泉明悉索所費,購姑女而還,比復納購,己女遂失。而袁履謙已
 下,父之將吏妻子奴隸三百餘人,轉徙賊中,窮窘無告。泉明悉以歸蒲州,真卿贍給久之,隨其所詣而資送之。泉明求其父尸於東都,得其行刑者,言杲卿被害時,先斷一足,與履謙同坎瘞之。及發瘞得尸,果無一足,即日與履謙之尸,各為一柩,扶護還長安。初,履謙妻疑夫柩殮衣儉薄,發棺視之,一與杲卿等,履謙妻號踴感嘆,待之如父。泉明之志行仁義如此。



 乾元元年五月,詔曰:「故衛尉卿、兼御史中丞、恆州刺史顏杲卿,任彼專城,志梟
 狂虜,艱難之際,忠義在心。憤群兇而慷慨,臨大節而奮發,遂擒元惡,成此茂勛。屬胡虜憑陵,流毒方熾,孤城力屈,見陷寇仇,身歿名存,實彰忠烈。夫仁者有勇,驗之於臨難;臣之報國,義存於捐軀。嘉其死節之誠,未備飾終之禮,可贈太子太保。」



 薛願,河東汾陰人。父縚,禮部郎中。兄崇一,尚惠宣太子女宜君縣主。女弟為廢太子瑛妃。願坐宮廢貶官。祿山之亂,南陽節度使魯炅奏用願為潁川太守、本郡防禦
 使。時賊已陷陳留、滎陽、汝南等郡,方圍南陽。潁川當其來往之路,願與防禦副使龐堅同力固守,城中儲蓄無素,兵卒單寡。自至德元年正月至十一月,賊晝夜攻之不息,距城百里,廬舍墳墓林樹開發斬徹殆盡,而外救無至。賊將阿史那承慶悉以銳卒並攻,為木驢木鵝,雲梯沖棚,四面雲合,鼓噪如雷,矢石如雨,力攻十餘日,城中守備皆竭,賊夜半乘梯而入。願、堅俱被執,送於東都,將支解之。或說祿山曰:「薛願、龐堅,義士也。人各為其主,
 屠之不祥。」乃系於洛水之濱,屬苦寒,一夕凍死。



 堅,武德功臣玉之玄孫。初娶邠王守禮女建寧縣主。魯炅奏為潁川郡長史,兼防禦副使。



 張巡,蒲州河東人。兄曉,開元中監察御史。兄弟皆以文行知名。巡聰悟有才幹,舉進士,三以書判拔萃入等。天寶中,調授清河令。有能名,重義尚氣節,人以危窘告者,必傾財以恤之。



 祿山之亂,巡為真源令。說譙郡太守,令完城,募市人,為拒賊之勢。時吳王祗為靈昌太守,奉詔
 糾率河南諸郡,練兵以拒逆黨,濟南太守李隨副之。巡與單父尉賈賁各召募豪傑,同為義舉。



 時雍丘令令狐潮欲以其城降賊,民吏百餘人不從命,潮皆反接,僕之於地,將斬之。會賊來攻城,潮遽出鬥,而反接者自解其縛,閉城門拒潮召賁。賁與巡引眾入雍丘,殺潮妻子,嬰城守備。吳王祗承制授賁監察御史。數日,賊來攻城,賁出鬥而死,巡乃合賁之眾城守。令狐潮引賊將李廷望攻圍累月,賊傷夷大半。祿山乃於雍丘北置杞州,築城
 壘以絕餉路,自是內外隔絕。又相持累月,賊鋒轉熾,城中益困。



 時許遠為睢陽守,與城父令姚摐同守睢陽城,賊攻之不下。初祿山陷河洛,許叔冀守靈昌,薛願守潁川,許遠守睢陽,皆城孤無援。願守一年而城陷,督冀一年而自拔,獨睢陽堅守。賊將尹子奇攻圍經年。巡以雍丘小邑,儲備不足,大寇臨之,必難保守,乃列卒結陣詐降,至德二年正月也。玄宗聞而壯之,授巡主客郎中、兼御史中丞。尹子奇攻圍既久,城中糧盡,易子而食,析骸
 而爨,人心危恐,慮將有變。巡乃出其妾,對三軍殺之,以饗軍士。曰:「諸公為國家戮力守城,一心無二,經年乏食,忠義不衰。巡不能自割肌膚,以啖將士,豈可惜此婦,坐視危迫。」將士皆泣下,不忍食,巡強令食之。乃括城中婦人;既盡,以男夫老小繼之,所食人口二三萬,人心終不離變。



 時賀蘭進明以重兵守臨淮,巡遣帳下之士南霽雲夜縋出城,求援於進明。進明日與諸將張樂高會,無出師意。霽雲泣告之曰:「本州強寇凌逼,重圍半年,食
 盡兵窮,計無從出。初圍城之日,城中數萬口,今婦人老幼,相食殆盡,張中丞殺愛妾以啖軍人,今見存之數,不過數千,城中之人,分當餌賊。但睢陽既拔,即及臨淮,皮毛相依,理須援助。霽雲所以冒賊鋒刃,匍匐乞師,謂大夫深念危亡,言發響應,何得宴安自處,殊無救恤之心?夫忠臣義士之所為,豈宜如此!霽雲既不能達主將之意,請嚙一指,留於大夫,示之以信,歸報本州。」霽雲自臨淮還睢陽,繩城而入。城中將吏知救不至,慟哭累日。



 十月,
 城陷。巡與姚摐、南霽雲、許遠,皆為賊所執。巡神氣慷慨,每與賊戰,大呼誓師,眥裂血流,齒牙皆碎。城將陷,西向再拜,曰:「臣智勇俱竭,不能式遏強寇,保守孤城。臣雖為鬼,誓與賊為厲,以答明恩。」及城陷,尹子奇謂巡曰:「聞君每戰眥裂,嚼齒皆碎,何至此耶?」巡曰:「吾欲氣吞逆賊,但力不遂耳!」子奇以大刀剔巡口,視其齒,存者不過三數。巡大罵曰:「我為君父義死。爾附逆賊,犬彘也,安能久哉!」子奇義其言,將禮之,左右曰:「此人守義,必不為我用。
 素得士心,不可久留。」是日,與姚摐、霽雲同被害,唯許遠執送洛陽。



 姚摐者,浹州平陸人,故相梁國公崇之侄孫。父弇,開元初歷處州刺史。摐性豪蕩,好飲謔,善絲竹。歷壽安尉、城父令,與張巡素相親善。以守睢陽之功,至德二年春,加檢校尚書侍郎。



 賈賁者,故閬州刺史璿之子也。



 許遠者,杭州鹽官人也。世仕江右。曾祖高陽公敬宗,龍朔中宰相,自有傳。遠清幹,初從軍河西,為磧西支度判
 官。章仇兼瓊鎮劍南,又闢為從事。慕其門,欲以子妻之。遠辭,兼瓊怒,積他事中傷,貶為高要尉。後遇赦得還。



 祿山之亂,不次拔將帥,或薦遠素練戎事。玄宗召見,拜睢陽太守,累加侍御史、本州防禦使。及賊將尹子奇攻圍,遠與張巡、姚摐嬰城拒守經年,外救不至,兵糧俱盡而城陷。尹子奇執送洛陽,與哥舒翰、程千里,俱囚之客省。及安慶緒敗,渡河北走,使嚴莊皆害之。



 初,賀蘭進明與房琯素不相葉。及琯為宰相,進明時為御史大夫。琯奏
 用進明為彭城太守、河南節度使、兼御史大夫,代嗣虢王巨;復用靈昌太守許叔冀為進明都知兵馬、兼御史大夫,重其官以挫進明。虢王巨受代之時,盡將部曲而行,所留者揀退羸兵數千人、劣馬數百匹,不堪捍賊。叔冀恃部下精銳,又名位等於進明,自謂匹敵,不受進明節制。故南霽雲之乞師,進明不敢分兵,懼叔冀見襲。兩相觀望,坐視危亡,致河南郡邑為墟,由執政之乖經制也。



 程千里,京兆人。身長七尺,骨相魁岸,有勇力。本磧西募人,累以戎勛,官至安西副都讓。天寶十一載,授御史中丞。十二載,兼北庭都讓,充安西北庭節度使。突厥首領阿布思先率眾內附,隸朔方軍,玄宗賜姓名曰李獻忠。李林甫遙領朔方節度,用獻忠為副將。後有詔移獻忠部落隸幽州,獻忠素與祿山有隙,懼不奉詔,乃叛歸磧北,數為邊患。玄宗憤之,命千里將兵討之。



 十二載十一月,千里兵至磧西,以書喻葛祿,令其相應。獻忠勢窮,歸
 葛祿部。葛祿縛獻忠並其妻子及帳下數千人,送之千里,飛表獻捷,天子壯之。十三載三月,千里獻俘於勤政樓,斬之於硃雀街,以功授右金吾衛大將軍同正,仍留佐羽林軍。祿山之亂,詔千里於河東召募,充河東節度副使、雲中太守。



 十五載正月,遷上黨郡長史、特進,攝御史中丞,以兵守上黨。賊來攻城,屢為千里所敗,以功累加開府儀同三司、禮部尚書、兼御史大夫。



 至德二年九月,賊將蔡希德圍城,數以輕騎挑戰。千里恃其驍果,開
 懸門,率百騎,欲生擒希德。勁騎搏之,垂將擒而希德救兵至,千里斂騎而退,橋壞墜坑,反為希德所執。仰首告諸騎曰:「非吾戰之過,此天也!為我報諸將士,乍可失帥,不可失城。」軍人聞之泣下,晝夜嚴兵城守,賊竟不能拔。千里至東都,安慶緒舍之,偽署特進,囚之客省。及慶緒敗走,為嚴莊所害。



 其年十二月,上御丹鳳樓大赦,節文曰:「忠臣事君,有死無貳;烈士徇義,雖歿如存。其李心妻、盧奕、袁履謙、張巡、許遠、張介然、蔣清、龐堅等,即與追贈,訪
 其子孫,厚其官爵,家口深加優恤。」自是赦恩,無不該於節義,而程千里終以生執賊庭,不沾褒贈。



 袁光庭者,河西戍將,天寶末為伊州刺史。祿山之亂,西北邊戎兵入赴難,河、隴郡邑,皆為吐蕃所拔。唯光庭守伊州累年,外救不至。虜百端誘說,終不之屈,部下如一。及矢石既盡,糧儲並竭,城將陷沒,光庭手殺其妻子,自焚而死。朝廷聞之,贈工部尚書。



 邵真者,恆州節度使李寶臣之判官也。累加檢校司封
 郎中、兼御史中丞,專掌文翰,寶臣深所信任。寶臣死,其子惟岳擅領父眾。李正己、田悅遣人說惟岳同叛,真泣諫曰:「先公位兼將相,受國厚恩,大夫縗絰之中,遽欲違命,同鄰道之惡,違先公之志,必不可也!田悅與我密邇,絕之又恐速禍;正己稍遠,絕之易耳。但令悅使還報,請徐思其宜;執正己使送京師,因請致討,朝廷必嘉大夫之忠,而旌節可得。」惟岳然之,令真草奏。將發,孔目吏胡震謂惟岳曰:「此事非細,請與將吏會議。」長史畢華曰:「先
 公與二道親好,二十餘年,一朝背之,伏恐生事。今執其來使,送於京師,大善。脫未為朝廷所信,正己兵強,忽來襲城,孤軍無援,何以敵之?不若仍舊勿絕,徐觀其變。」惟岳又從之。真又勸惟岳遣其弟惟簡入朝,仍遣軍吏薛廣嗣詣河東節度馬燧軍求保薦。田悅屯兵束鹿,聞其謀,遣人謂惟岳曰:「邵真惑亂軍政,必速殺之。不然,吾且討其罪矣。」惟岳懼,遂殺真。朝廷聞而嘉之,贈戶部尚書。



 符璘者,田悅之將。初,馬燧、李抱真、李芃等破田悅於洹
 水,燧等進屯魏州。時悅與李納會於濮陽,因請助兵,納分麾下數千人隨之。至是納為河南諸軍所逼,自濮陽奔歸濮州,徵兵於悅,悅遣璘將三百騎護送之。納兵既歸,遂悉其眾降於燧。遷璘試太子詹事、兼御史中丞,封義陽郡王,實封一百戶。



 璘父令奇,初為悅部將,至是因璘之出,遂令三子同降於燧。悅怒,執令奇,令奇大呼慢罵之,悅族其家。贈令奇戶部尚書。



 趙曄,字雲卿,鄧州穰人,其先自天水徙焉。貞觀中,主客
 員外郎德言曾孫也。父敬先,殿中侍御史。



 曄志學,善屬文。開元中,舉進士,連擢科第,補太子正字,累授大理評事,貶北陽尉,移雷澤、河東二丞。河東採訪使韋陟以曄履操清直,頗推敬之,表為賓僚。陟罷,陳留採訪使郭納復奏曄為支使。及安祿山陷陳留,因沒於賊。時有京兆韋氏,夫任畿官,以不供賊軍遇害,韋被逆賊沒入為婢。江西觀察使韋儇,族兄弟也。曄哀其冤抑,以錢贖之,俾其妻置之別院,厚供衣食,而曄竟不面其人。明年,收復
 東都,曄以家財資給,而訪其親屬歸之,識者咸重焉。



 乾元初,三司議罪,貶晉江尉。數年,改錄事參軍。徵拜左補闕,未至。福建觀察使李承昭奏為判官,授試大理司直、兼監察御史。試司議郎、兼殿中侍御史。入為膳部、比部二員外,膳部、倉部二郎中,秘書少監。



 曄性孝悌,敦重交友,雖經艱危,不改其操。少時與殷寅、顏真卿、柳芳、陸據、蕭穎士、李華、邵軫、同志友善,故天寶中語曰:「殷、顏、柳、陸、蕭、李、邵、趙」,以其重行義,敦交道也。而曄早擅高名,在宦途
 五十年,累經貶謫,蹇躓備至。入仕三十年,方霑省官,身在郎署,子常徒步。官既散曹,俸祿單寡,衣食不充,以至亡歿,服名檢者為之嘆息。建中四年冬,涇原兵叛,曄竄於山谷。尋以疾終,追贈華州刺史。



 子宗儒,別有傳。



 石演芬,本西城胡人也。以武勇為朔方邠寧節度兵馬使、兼御史大夫。李懷光養為子,累至右武鋒都將。時懷光軍屯三橋,將與硃泚通謀。演芬乃使門客郜成義密疏,具言懷光無狀,請罷其總統。成義至奉天,乃反以其
 言告懷光子琟,琟密告其父。懷光乃召演芬責之曰:「以爾為子,奈何欲破我家!今死可乎?」演芬對曰:「天子以公為腹心,公上負天子,安可責演芬!且演芬胡人,不解異心,欲守事一人,幸免呼為賊。死,常分也!」懷光使左右臠食之,皆曰:「此忠烈士也!可令速死。」乃以刀斷其頸。德宗追思義烈,贈兵部尚書,仍賜錢三百千。又捕得郜成義於朔方,戮之。



 先是,詔賜懷光鐵券。懷光奉詔倨慢,左都將張名振大呼軍門曰:「太尉見賊不擊,天使到不迎,固
 將反耶!且安史兩賊,僕固懷恩,今皆族滅,公欲何為?是資忠義之士立功勛耳!」懷光聞之,召謂曰:「我不反,為賊強盛,須蓄銳俟時耳。」無幾,懷光引軍入咸陽,名振曰:「公乃言不反,今此來何也?何不急攻硃泚,收復京城,以圖富貴?」懷光曰:「名振病狂。」使左右殺之。



 張伾,建中初,以澤潞將鎮臨洺。田悅攻之,伾度兵力不能出戰,嚴設守備,嬰城拒守,賊不能拔。累月,攻之益急,士多死傷,糧儲漸乏,救兵未至。伾知事不濟,無以激士
 心,乃悉召將卒於軍門,命其女出拜之,謂曰:「將士辛苦守戰,伾之家無尺寸物與公等,獨有此女,幸未嫁人,願出賣之,為將士一日之費。」眾皆大哭,曰:「誓為將軍死戰,幸無慮也!」會馬燧與太原之師至,與眾合擊悅於城下,大敗之。伾乘勢出戰,士卒無不一當百。圍解,以功遷泗州刺史。在州十餘年,拜右金吾衛大將軍。詔未至,病卒。貞元二十一年,贈尚書右僕射。



 有子重政,軍吏欲立為郡將,重政母徐氏固拒不從。詔曰:「前昭義軍泗州行營
 衙前兵馬使、大中大夫、試太子賓客、兼監察御史張重政,門有勛力,惟推義勇。夙聞克家之美,常稱撫眾之才。近者其父初亡,群小扇惑,誘以奇計,俾執軍麾。而重政與其母兄,號泣固拒,遂全懇願,奔告元戎,不為利回,成其先志。於家為孝子,在國為忠臣,軍政乂安,行義昭著。念茲名節,感嘆良深,宜洽恩榮,俾弘激勸。禮無避於金革,理當由於權奪,戎章憲府,式示兼崇。可起復雲麾將軍,守金吾衛大將軍、員外置同正員,檢校太子詹事、兼
 御史中丞,仍委淮南節度使與要職事任使。」



 又詔曰:「張重政母高平郡夫人徐氏,族茂姻閥,行表柔明,懷正家之美,有擇鄰之識。頃當變故,曾不詭隨,保其門宗,訓成忠孝,雖圖史所載,何以加之!念其令子,已申獎用,特彰母儀之德,俾崇封國之榮。可封魯國太夫人。」



 甄濟,字孟成,中山無極人,家於衛州。少孤,天寶中隱居衛州青巖山。人伏其操行,約不畋漁。採訪使安祿山表薦之,授試大理評事,充範陽郡節度掌記。



 天寶末,安
 祿山有異志,謀以智免。衛縣令齊誠信可托,乃求使至衛,具以誠告。弟心妻密求羊血以為備,至夜,偽嘔血疾不能支,遂舁歸。及祿山反,使偽節度使蔡希德領行戮者李掞等二人,封刀來召,察濟詐不起,即就戮之。濟以左手書云:「去不得!」李掞持刀而前,濟引首以待,希德歔欷嗟嘆之,曰:「李掞退。」以實病報祿山。後安慶緒亦使人至縣,強舁至東都安國觀。經月餘,代宗收東京。濟起,詣軍門上謁,乃送上都。肅宗館之於三司使,令受偽命官
 瞻望,以愧其心。授秘書郎,轉太子舍人。寶應初,拜刑部員外郎。魏少游奏授著作郎、兼侍御史,終於襄州。



 元和中,襄州節度使袁滋奏其節行,詔曰:「符風樹節,謂之立名;歿加褒贈,所以誘善。故朝散大夫、秘書省著作郎、兼侍御史甄濟,早以文雅,見稱於時。嘗因闢召,亦佐戎府。而能保堅貞之正性,不履危機;睹逆亂之潛萌,不從脅污。義聲可傳於竹帛,顯贈未賁於松楸。籓方所陳,允葉彞典,追加命秩,以獎忠魂。可贈秘書少監。」



 劉敦儒,開元朝史官左散騎常侍子玄之孫。敦儒母有心疾,非日鞭人不安,子弟僕使,不勝其苦,皆逃遁他處,唯敦儒侍養不息,體常流血。及母亡,居喪毀瘠骨立,洛中謂之劉孝子。



 元和中,東都留守權德輿具奏其至行,詔曰:「孝子劉敦儒,生於儒門,稟此至性。王祥篤行,起孝敬而不移;曾參養志,積歲年而罔怠。用弘勸獎,而服官常,分曹洛師,俾遂私志。可左龍武軍兵曹參軍,分司東都。」



 高沐,渤海人。父憑,從事於宣武軍,知曹州事。李靈曜作亂,憑密遣使奏賊中事狀,詔除曹州刺史。無何,李正己盜有曹、濮,憑遂陷於賊,數年卒。



 沐,貞元中進士及第。以家族在鄆,李師古置為判官。居數年,師道擅襲,每謀不順,沐與同列郭昈、李公度等,必廣引古今成敗諭之,前後說師道為善者凡千言。其判官李文會、孔目官林英,皆為師道信用,乘間相與涕泣於師道前曰:「文會等血誠憂尚書家事,反為高沐輩所嫉。尚書奈何不惜十二
 州之城,成高沐等百代之名乎!」復日夜讒構,由是漸見疑忌,令沐知萊州事。林英因奏事至京,逼邸吏密報師道云:「高沐潛有誠款至朝廷矣!」師道大怒,李文會從而構成之。沐遂遇害於遷所,而囚郭昈於萊州,其血屬皆徙遠地。



 及淮西平,師道漸懼。李公度與其將李英曇乘其懼也,說師道獻三州及入質長子。初,甚然之,中悔,將殺公度。賈直言聞之,謂師道用事奴曰:「今大禍將至,豈非高沐冤氣所為!又殺公度,是益其疾也!」乃止。逐英曇
 於萊州。未至,縊殺之。又有崔承寵、楊偕、陳佑、崔清,皆以仗順為賊所惡,李文會呼為高沐之黨。沐遇害,承寵等同被囚放。郭昈名亞於沐,雖不死,備嘗困辱矣。及劉悟平賊,遽召李公度,執手歔欷。既除滑州節度,首闢昈及公度為從事。



 元和十四年四月,詔曰:



 圖難忘死,為臣之峻節;顯忠旌善,有國之令猷。日者妖豎反覆,侮我朝章,而濮州刺史高沐,劫在兇威,潛輸忠款。諷其不庭之咎,將冀革心;數其煮海之饒,聿求利國。伏奏必陳於逆節,
 漏師常破其陰謀。竟以盜憎,遂死王事,歿而不朽,風聲凜然。式表漏泉之澤,且彰勁草之節。可贈吏部尚書。仍委馬總訪其遺骸,以禮收葬,優恤其家。若有子孫,具名聞奏。



 賈直言者,父道沖,以伎術得罪,貶之,賜鴆於路。直言偽令其父拜四方,辭上下神祗,伺使者視稍怠,即取其鴆以飲,遂迷僕而死。明日鴆洩於足而復蘇。代宗聞之,減父死,直言亦自此病蹙。後從事於李師道。師道不恭朝
 命,直言冒刃說者二,輿櫬說者一。師道訖不從。及劉悟斬師道,節制鄭滑,得直言於禁錮之間,又嘉其所為,因奏置幕中。後遷於潞,亦與之俱行。悟纖微乖失,直言必盡理箴規,以是美譽日聞於朝。穆宗以諫議大夫征之,悟拜章乞留,復授檢校右庶子、兼御史大夫,依前充昭義軍行軍司馬。悟用其言,終身不虧臣節。後歷太子賓客。太和九年三月卒,廢朝一日,贈工部尚書。



 庾敬休,字順之,其先南陽新野人。祖光烈,與仲弟光先,
 祿山迫以偽官,皆潛伏奔竄。光烈為大理少卿,光先為吏部侍郎。父河,當賊泚盜據宮闕,與季弟倬逃竄山谷。河終兵部郎中。



 敬休舉進士,以宏詞登科,授秘書省校書郎,從事宣州。旋授渭南尉、集賢校理。遷右拾遺、集賢學士。歷右補闕,稱職,轉起居舍人,俄遷禮部員外郎。入為翰林學士,遷禮部郎中,罷職歸官。又遷兵部郎中、知制誥。丁憂,服闋,改工部侍郎,權知吏部選事,遷吏部侍郎。



 上將立魯王為太子,慎選師傅,改工部侍郎,兼魯王
 傅。奏:「劍南西川、山南西道每年稅茶及除陌錢,舊例委度支巡院勾當,榷稅當司於上都召商人便換。太和元年,戶部侍郎崔元略與西川節度使商量,取其穩便,遂奏請茶稅事使司自勾當,每年出錢四萬貫送省。近年已來,不依元奏,三道諸色錢物,州府逗留,多不送省。請取江西例,於歸州置巡院一所,自勾當收管諸色錢物送省,所冀免有逋懸。欲令巡官李濆專往與德裕、遵古商量制置,續具奏聞。」從之。又奏:「兩川米價騰踴,百姓流亡。
 請糶兩川闕官職田祿米,以救貧人。」從之。再為尚書左丞。太和九年三月,卒於家。



 敬休姿容溫雅,襟抱夷曠,不飲酒茹葷,不邇聲色。著《諭善錄》七卷。贈吏部尚書。



 辛讜,故太原尹云京之孫,壽州刺史晦之猶子也。性慷慨,重然諾,專務賑人之急。年五十,不求茍進,有濟時匡難之志。



 咸通十年,龐勛亂徐泗。時杜慆守泗州,賊以郡當江淮要害,極力攻之。時兩淮郡縣皆陷。慆守臨淮久之,援軍雖集,賊未解圍。時讜寓居廣陵,乃仗劍拏小艇
 趨泗口,貫城柵入城見慆。慆素聞有義而不相面,喜讜至,握手謝曰:「判官李延樞方話子為人,何遽至耶?吾無憂矣!」時賊三面攻城,王師結壘於洪源驛。相顧不前。讜夜以小舟穿賊壘至洪源驛。見監軍郭厚本,論泗州危急,且宜速救,厚本然之。淮南都將王公弁謂厚本曰:「賊眾我寡,無宜輕舉,當俟可行。」讜坐中拔劍瞋目謂公弁曰:「賊百道攻城,陷在旦夕。公等奉詔赴援,而逗留不進,更欲何為?不唯有負國恩,丈夫氣義,亦宜感發!假如臨
 淮陷賊,淮南即是寇場,公何獨存耶!」即欲揮刃向公弁,厚本持之。讜望泗州大哭經日,帳下為之流涕。厚本義其心,選勇士三百,隨讜入泗州。夜半斬賊柵,大呼,由水門而入,賊軍大駭。既知援兵入,賊乃退舍,人心遂固。



 浙西觀察使杜審權遣大將翟行約率軍三千赴援,屯蓮塘驛。慆欲遣人勞之,將吏皆憚其行。讜曰:「杜相公以大夫宗盟,急難相赴,安得令使者無言而還!」即齎慆書幣,犒其使。淮南大將李湘率師五千來援,賊詐降,敗於淮
 口,湘與郭厚本皆為賊所執,自是無援。賊並兵急攻,以鐵鎖斷淮流,梯沖雲合,凡周七月,晝夜不息。乘城之士,不遑寢寐,面目生瘡,軍儲漸少,分食稀粥。賴讜犯難仗義,求救於淮北諸軍。既而馬舉以大軍至,賊解圍而去。



 讜無子,猶子山僧、元老等寄在廣陵。每出城,則書二姓名,謂慆曰:「志之,得嗣為幸。」慆益感之。賊平,授讜泗州團練判官、侍御史。慆遷鄭滑節度,讜亦從之,為賓佐。慆卒,乃退歸江東,以隱居為事。



 贊
 曰:獸解觸邪,草能指佞。烈士徇義,見危致命。國有忠臣,亡而復存。何以喪邦?奸邪受恩。



\end{pinyinscope}