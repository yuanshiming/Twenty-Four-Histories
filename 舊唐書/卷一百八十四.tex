\article{卷一百八十四}

\begin{pinyinscope}

 ○硃克融李載義楊志誠
 張仲武子直方張允伸張公素李可舉李全忠子匡威匡籌



 硃克融,賊泚之從孫也。祖滔,父洄。克融少為幽州軍校,事
 節度使劉總。總將歸朝,慮其有變,籍軍中素有異志者,薦之闕下,時克融亦在籍中。宰相崔植、杜元穎不知兵,且無遠略,謂兩河無虞,遂奏勒歸鎮。長慶初,幽州軍亂,囚其帥張弘靖。時洄廢疾於家,軍中素伏其謀略,至是眾欲立之。洄自以老且病,推克融統軍務焉。朝廷尋加檢校左散騎常侍,授以符節。



 寶歷二年,遣使送方鎮及三軍時服,克融怒所賜疏弱,執中使以聞。上特優容,別命中使宣諭,仍改賜衣物,流其使楊文端等。先是,克
 融執中使,奏稱:「竊聞陛下欲幸東都,請將兵馬並丁匠五千人,修理宮闕,迎候車駕。」又上言無衣,擬於朝廷請三十萬端疋,以備一歲所費,不然則三軍不安。天子怒其悖慢,取宰臣裴度謀,優容之,語見別卷。克融官至檢校司空、吳興郡王。



 其年五月,本州軍亂,殺之,子延齡亦遇害。次子延嗣竊立,尋為大將李載義所殺。



 李載義,字方谷,常山愍王之後。代以武力稱,繼為幽州屬郡守。載義少孤,與鄉曲之不令者游。有勇力,善挽強
 角牴。劉濟為幽州節度使,見而偉之,致於親軍,從征伐。以功遷衙前都知兵馬使,檢校光祿大夫、兼監察御史。寶歷中,幽師殺硃克融。其子延嗣竊襲父位,不遵朝旨,虐用其人;載義遂殺之,數其罪以聞。敬宗嘉之,拜檢校戶部尚書、兼御史大夫,封武威郡王,充幽州盧龍等軍節度副大使,知節度事。



 未幾,李同捷據滄景以邀襲父爵。載義上表,請討同捷以自效。上嘉其誠懇,特加檢校右僕射。累破賊軍,以功加司空,進階金紫。太和三年,平
 滄景,策勛加平章事,仍賜實封三百戶。四年,契丹寇邊,以兵擊走之,仍虜其名王,就加太保。五年春,為其部下楊志誠所逐,因入覲。上以載義有平滄景之功,又能恭順朝旨,再拜太保、同平章事。其年,改山南西道節度、觀察等使,兼興元尹。七年,遷北都留守,兼太原尹,充河東節度觀察處置等使。尋加開府儀同三司。丁母憂,起復驃騎大將軍,餘如故。



 回鶻每遣使入朝,所至強暴。邊城長吏多務茍安,不敢制之以法。但嚴兵防守,虜益驕悍,
 或突入市肆,暴橫無所憚。至是,有回鶻將軍李暢者,曉習中國事,知不能以法制馭,益驕恣。鞭捶驛吏,貪求無已。載義因召李暢與語曰:「可汗使將軍朝貢,以固舅甥之好,不當使將軍暴踐中華。今朝廷饔餼至厚,所以禮蕃客也。茍有不至,吏當坐死。若將軍之部伍不戢,凌侮上國,剽掠廬舍,載義必殺為盜者。將軍勿以法令可輕而不戒勵之!」遂罷防守之兵,而使兩卒司其門。虜知其心為下,無敢犯令。九年,加侍中。開成二年卒,年五十,贈
 太尉。



 載義晚年驕恣,慘暴一方。以楊志誠復為部下所逐,過太原,載義躬身毆擊,遂欲殺之,賴從事救解以免。然而擅殺志誠之妻孥及將卒。朝廷錄其功,屈法不問。



 楊志誠,太和五年為幽州後院副兵馬使,事李載義。時朝廷賜載義德政碑文。載義延中使擊鞠,志誠亦與焉,遂於鞠場叫呼謀亂。載義奔於易州,志誠乃為本道馬步都知兵馬使。



 文宗聞之驚,急召宰臣。時牛僧孺先至,上謂曰:「幽州今日之事可奈何?」僧孺曰:「此不足煩聖慮,
 臣被召疾趨氣促,容臣稍緩息以對。」上良久曰:「卿以為不足憂,何也?」僧孺對曰:「陛下以範陽得失系國家休戚耶?且自安、史之後,範陽非國家所有。前時劉總向化,以土地歸闕,朝廷約用錢八十萬貫,而未嘗得範陽尺布斗粟上供天府;則今日志誠之得,猶前日載義之得也。陛下但因而撫之,亦事之宜也。且範陽,國家所賴者,以其北捍突厥,不令南寇。今若假志誠節鉞,惜其土地,必自為力。則爪牙之用,固不計於逆順。臣固曰不足煩聖
 慮。」上大喜曰:「如卿之言,吾洗然矣。」尋以嘉王運遙領節度,以志誠為節度觀察留後,檢校左散騎常侍,兼幽州左司馬。尋改檢校工部尚書、節度副大使,知節度事。



 七年,轉檢校吏部尚書。詔下,進奏官徐迪詣中書白宰相曰:「軍中不識朝廷體位,只知自尚書改僕射為遷,何知工部轉吏部為美?且軍士盛飾以待新恩,一旦復為尚書,軍中必慚。今中使往彼,其勢恐不得出。」及使至,其傔奔還,奏曰:「楊志誠怒不得僕射,三軍亦有怨言。春衣使
 魏寶義、兼他使焦奉鸞,尹士恭,並為志誠縶留矣。」志誠遣將王文穎謝恩,並讓官,復賜官告批答,文穎不受而歸。朝廷納裴度言,務以含垢,下詔諭之,因再遣使加尚書右僕射。



 八年,為三軍所逐,則立史元忠。元忠進志誠所造袞龍衣二副及被服鞍韉,皆繡飾鸞鳳日月之形,或為王字。因付御史臺按問,流嶺南。行至商州,殺之。



 初,元忠既逐志誠,詔以通王淳遙領節度,授元忠左散騎常侍、幽州大都督府左司馬、知府事,充節度留後。明年,
 轉檢校工部尚書、節度副大使,知節度事。後為偏將陳行泰所殺。



 張仲武,範陽人也。仲武少業《左氏春秋》,擲筆為薊北雄武軍使。會昌初,陳行泰殺節度使史元忠,權主留後。俄而,行泰又為次將張絳所殺,令三軍上表,請降符節。時仲武遣軍吏吳仲舒表請以本軍伐叛。上遣宰臣詢其事,仲舒曰:「絳與行泰,皆是游客,主軍人心不附。仲武是軍中舊將張光朝之子,年五十餘,兼曉儒書,老於戎事,
 性抱忠義,願歸心闕廷。」李德裕因奏:「陳行泰、張絳皆令大將上奏,邀求節旄,所以必不可與。今仲武上表布誠,先陳密款,因而拔用,即似有名。」許之,乃授兵馬留後,詔撫王紘遙領節度。尋改仲武節度副大使、知節度事,檢校工部尚書、幽州大都督府長史、兼御史大夫、蘭陵郡王。俄而回鶻擾邊。



 時回鶻有將勒那頡啜擁赤心宰相一族七千帳,東逼漁陽。仲武遣其弟仲至與裨將游奉寰、王如清等,率銳兵三萬人大破之。前後收其侯王貴
 族千餘人,降三萬人,獲牛馬、橐駝、旗纛、罽幕不可勝計。遣從事李周瞳、牙門將國從,相次獻捷。詔加檢校兵部尚書,兼東面招撫回鶻使。先是,奚、契丹皆有回鶻監護使,督以歲貢,且為漢諜。至是,遣裨將石公緒等諭意兩部,凡戮八百餘人。又回鶻初遣宣門將軍等四十七人,詭詞結歡,潛伺邊隙。仲武使密賂其下,盡得陰謀。且欲馳入五原,驅掠雜虜。遂逗遛其使,緩彼師期。人馬病死,竟不遣之。回鶻烏介可汗既敗,不敢近邊,乃依康居
 求活,盡徙餘種,寄托黑車子部。



 仲武由是威加北狄,表請於薊北立《紀聖功銘》,敕李德裕為之文,其銘曰:



 太和之初,赤氣宵興;開成之末,彤雲暮凝。異鳥南來,胡滅之徵。北夷飆掃,厥國土崩。逼迫遷徙,震我邊鄙;長蛇去穴,奔鯨失水。上都薊門,兵連千里;曾不畏天,猶為驕子。丐我邊穀,邀我王師,假我一城,建彼幡旗。歸計強漢,郅支嫚辭;狼顧朔野,伏莽見羸。雁門之北,羌戎雜處,濈々群羊,茫茫大鹵。縱其梟騎,驚我牧圉;暴若豺狼,疾如風雨。
 皇赫斯怒,羽檄徵兵;謀而泉默,斷乃霆聲。沉機變化,動合神明,沙漠之外,虜無隱情。漁陽突騎,燕歌壯氣,赳赳元戎,眈眈虎視。金鼓誓眾,干旄蔽地,爰命其弟,屬之大事。翩翩飛將,董我三軍;稟兄之制,代帥之勤。威略火烈,胡馬星分,戈回白日,劍薄浮雲。天街之北,旄頭已落;絕轡之野,蚩尤未縛。俾我元侯,恢弘遠略;終取單于,系之徽索。陰山寢鋒,亭徼弢弓,萬里昆夷,九譯而通。蠻夷既同,天子之功;儒臣篆美,刊石垂鴻。



 仲武歷官至司徒、中
 書門下平章事。大中年卒,謚曰莊。



 子直方,以幽州節度副使襲父位。動多不法,慮為將卒所圖。三年冬,托以游獵,奔赴闕庭,尋授金吾將軍。直方性率暴,行豪奪之事,以罪累貶柳州司馬。十一年,遷右驍衛將軍,分司東都。咸通中,位至羽林統軍。中和歲,賊巢犯闕,公卿恃其豪,多隱藏於第。直方納招亡命,謀欲劫巢。或有告者,由是以兵圍而害之。



 張允伸,字逢昌,範陽人也。曾祖秀,檀州刺史。祖巖,納降
 軍使。父朝掖,贈太尉。允申世仕幽州軍門,累職至押衙,兼馬步都知兵馬使。大中四年,戎帥周綝寢疾,表允伸為留後,朝廷可其奏,加右散騎常侍。其年冬,詔賜旌節,遷檢校工部尚書。咸通九年,累加至光祿大夫、檢校司徒、兼太傅、同中書門下平章事、燕國公。



 十年,徐人作亂,請以弟允皋領兵伐叛,懿宗不允。進助軍米五十萬石,鹽二萬石。詔嘉之,賜以錦彩、玉帶、金銀器等。冬,又加特進,兼侍中。十二年,以風恙拜章請就醫藥,詔許之。以子
 簡會檢校工部尚書,充節度副大使。十三年,允伸再上表進納所賜旌節。朝命未至,其年正月二十五日卒,年八十八。再贈太尉,謚曰忠烈。



 允伸領鎮凡二十三年,克勤克儉,比歲豐登。邊鄙無虞,軍民用乂。至今談者美之。有子十四人。



 簡真,幽府左司馬,先允伸卒。簡壽,右領軍衛大將軍。餘或升朝籍,或為刺史、郡佐。



 張公素,範陽人。咸通中,為幽州軍校。事張允伸,累遷至平州刺史。允伸卒,子簡會權主留後事,公素領本郡兵
 赴焉。三軍素畏公素威望,簡會知力不能制,即時出奔,遂立為帥。朝廷尋授旌節,累加至中書門下平章事。無幾,李茂勛奪其位,公素歸闕,貶復州司戶參軍。



 李可舉,本回鶻阿布思之族也。張仲武破回鶻,可舉父茂勛與本部侯王降焉。茂勛善騎射,性沉毅,仲武器之。常遣拓邊,以功封郡王,賜姓名。



 咸通末,納降軍使陳貢言者,幽之宿將,人所信服。茂勛密謀劫而殺之,聲云貢言舉兵。張公素以兵逆擊不利,公素走,茂勛入城,軍民
 方知其非貢言也。既有其眾,遂推而立之,朝廷即降符節。無幾,以疾告老,授右僕射致仕,表可舉自節度副使、幽州左司馬加右散騎常侍,為節度留後。中和中,累官至檢校太尉。



 中和末,以太原李克用兵勢方盛,與定州王處存密相締結。可舉慮其窺伺山東,終為己患,遂遣使構雲中赫連鐸乘其背,則與鎮州合謀舉兵,兼言易、定是燕、趙之餘,雲得其地則正其疆理而分之。時可舉遣將李全忠攻易州。有次將劉仁恭者,多權數;攻之彌
 月不下,乃穴地道以入。其城既下,易州士卒稍驕。王處存引輕軍三千,以羊皮蒙之,夜伏於城外,仍別於間道以騎士伺之。燕軍望見,謂之群羊,爭趨焉。處存乘其無部伍,一擊大敗之,尋復其城。全忠遁歸,懼可舉罪之,收其餘眾,反攻幽州。可舉危急,收集其族,登樓自燔而死。



 李全忠,範陽人。廣明中,為棣州司馬。有蘆生於室,一尺三節,心惡之。謂別駕張建曰:「吾室生蘆,無乃怪歟?」建曰:「蘆,茅類,得澤而滋,公家有茅土之慶,殆天意乎!其生三
 節,必傳節鉞者三人。公勉樹功名,無忘斯言。」全忠秩滿還鄉里,事節度使李可舉為牙將。時可舉兵鋒方盛,欲與鎮人分易、定,遣全忠將兵攻之,為定州軍大敗於易水。全忠懼,率其餘眾掩攻幽州。可舉死。三軍推全忠為留後,朝廷因以節鉞授之,光啟元年春也。



 全忠卒,子匡威自襲父位,稱留後。匡威素稱豪爽,屬遇亂離,繕甲燕薊,有吞四海之志。赫連鐸據雲中,屢引匡威與河東爭雲、代,並兵積年。景福初,鎮州王鎔誘河東將李存孝。克
 用怒,加兵討之。時鎔童幼,求援於燕;匡威親率軍應之。二年春,河東復出師井陘,再乞師,匡威來援。



 匡威弟匡籌,妻張氏有國色。師將發,家人會別,匡威酒酣,留張氏報之。匡籌私懷忿怒,匡威軍至博野,匡籌乃據城自為節度。匡威部下聞之,亡歸者半。匡威退無歸路,將入覲京師。時匡威留於深州,遣判官李抱貞奉章以聞。屬京師大亂之後,聞匡威來朝,市人震恐,咸曰「金頭王來謀社稷」,士庶有亡竄山谷者。匡威其實不行,欲圖鎮州,示
 無留意。熔以匡威再來援己,致其失師,遣使迎歸府第,父事之。匡威為鎔城郛繕甲,指陳方略,視鎔如子。每陰謀驟施,以悅人心。鎮之三軍,素忠於王氏,惡其所為。會鎔過匡威第慰忌辰,匡威縞衣裹甲,伏兵劫鎔入牙城。鎔兵逆戰,燔東偏門,軍士呼噪登屋,矢下如雨。鎔僕墨君和亂中扶鎔登屋免難,而斬匡威以徇。



 是歲,匡籌出師攻鎮之樂壽、武強以報恥。匡威部曲劉仁恭歸於河東。乾寧元年冬,河東聽仁恭之謀,出師進討。二月,敗燕
 軍於居庸,匡籌挈其族遁去,將赴京師。至景城,為滄州節度使盧彥威所殺,掠其輜車、妓妾。匡籌妻張氏產於路,不能進,劉仁恭獲之,獻於李克用,後立為夫人,嬖寵專房。李氏父子三葉,十年而亡。



 史臣曰:大都偶國,亂之本也。故古先哲王建國,公侯之封,不過千乘,所以強幹弱枝,防其悖慢。彼幽州者,列九圍之一,地方千里而遙,其民剛強,厥田沃壤。遠則慕田光、荊卿之義,近則染祿山、思明之風。二百餘年,自相崇
 樹,雖朝廷有時命帥,而土人多務逐君。習苦忘非,尾大不掉,非一朝一夕之故也。若李載義、張仲武、張允伸因利乘便,獲領旌旗,以仁守之,恭順朝旨,亦足多也。如硃克融、楊志誠、史元忠、張公素、李可舉、李全忠,以不仁得之,靡更曩志。或尋為篡奪,或僅傳子孫,咸非令終,蓋其宜也。



 贊曰:蠍石之野,氣勁人豪。二百餘載,自相尊高。載義、仲武,亦多忠勞。餘因篡得,不仁何逃?



\end{pinyinscope}