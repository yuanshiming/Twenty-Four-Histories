\article{卷一百 姚崇宋璟}

\begin{pinyinscope}

 ○姚崇宋璟



 姚崇,本名元崇,陜州硤石人也。父善意,貞觀中,任巂州都督。元崇為孝敬挽郎,應下筆成章舉,授濮州司倉,五遷夏官郎中。時契丹寇陷河北數州,兵機填委,元崇剖
 析若流,皆有條貫。則天甚奇之,超遷夏官侍郎,又尋同鳳閣鸞臺平章事。



 聖歷初,則天謂侍臣曰:「往者周興、來俊臣等推勘詔獄,朝臣遞相牽引,咸承反逆,國家有法,朕豈能違。中間疑有枉濫,更使近臣就獄親問,皆得手狀,承引不虛,朕不以為疑,即可其奏。近日周興、來俊臣死後,更無聞有反逆者,然則以前就戮者,不有冤濫耶?」元崇對曰:「自垂拱已後,被告身死破家者,皆是枉酷自誣而死。告者特以為功,天下號為羅織,甚於漢之黨錮。
 陛下令近臣就獄問者,近臣亦不自保,何敢輒有動搖?被問者若翻,又懼遭其毒手,將軍張虔勖、李安靜等皆是也。賴上天降靈,聖情發寤,誅鋤兇豎,朝廷乂安。今日已後,臣以微軀及一門百口保見在內外官更無反逆者。乞陛下得告狀,但收掌,不須推問。若後有徵驗,反逆有實,臣請受知而不告之罪。」則天大悅曰:「以前宰相皆順成其事,陷朕為淫刑之主。聞卿所說,甚合朕心。」其日,遣中使送銀千兩以賜元崇。



 時突厥叱利元崇構逆,則
 天不欲元崇與之同名,乃改為元之。俄遷鳳閣侍郎,依舊知政事。



 長安四年,元之以母老,表請解職侍養,言甚哀切,則天難違其意,拜相王府長史,罷知政事,俾獲其養。其月,又令元之兼知夏官尚書事、同鳳閣鸞臺三品。元之上言:「臣事相王,知兵馬不便。臣非惜死,恐不益相王。」則天深然其言,改為春官尚書。是時,張易之請移京城大德僧十人配定州私置寺,僧等苦訴,元之斷停,易之屢以為言,元之終不納。由是為易之所譖,改為司僕
 卿,知政事如故,使充靈武道大總管。



 神龍元年,張柬之、桓彥範等謀誅易之兄弟,適會元之自軍還都,遂預謀,以功封梁縣侯,賜實封二百戶。則天移居上陽宮,中宗率百官就閤起居,王公已下皆欣躍稱慶,元之獨嗚咽流涕。彥範、柬之謂元之曰:「今日豈是啼泣時!恐公禍從此始。」元之曰:「事則天歲久,乍此辭違,情發於衷,非忍所得。昨預公誅兇逆者,是臣子之常道,豈敢言功;今辭違舊主悲泣者,亦臣子之終節,緣此獲罪,實所甘心。」無幾,
 出為亳州刺史,轉常州刺史。



 睿宗即位,召拜兵部尚書、同中書門下三品,尋遷中書令。時玄宗在東宮,太平公主干預朝政,宋王成器為閑廄使,岐王範、薛王業皆掌禁兵,外議以為不便。元之同侍中宋璟密奏請令公主往就東都,出成器等諸王為刺史,以息人心。睿宗以告公主,公主大怒。玄宗乃上疏以元之、璟等離間兄弟,請加罪,乃貶元之為申州刺史。再轉揚州長史、淮南按察使,為政簡肅,人吏立碑紀德。俄除同州刺史。先天二年,
 玄宗講武在新豐驛,召元之代郭元振為兵部尚書、同中書門下三品,復遷紫微令。避開元尊號,又改名崇,進封梁國公。固辭實封,乃停其舊封,特賜新封一百戶。



 先是,中宗時,公主外戚皆奏請度人為僧尼,亦有出私財造寺者,富戶強丁,皆經營避役,遠近充滿。至是,崇奏曰:「佛不在外,求之於心。佛圖澄最賢,無益於全趙;羅什多藝,不救於亡秦。何充、苻融,皆遭敗滅;齊襄、梁武,未免災殃。但發心慈悲,行事利益,使蒼生安樂,即是佛身。何用
 妄度奸人,令壞正法?」上納其言,令有司隱括僧徒,以偽濫還俗者萬二千餘人。



 開元四年,山東蝗蟲大起,崇奏曰:「《毛詩》云:『秉彼蟊賊,以付炎火。』又漢光武詔曰:『勉順時政,勸督農桑,去彼蝗蜮,以及蟊賊。』此並除蝗之義也。蟲既解畏人,易為驅逐。又苗稼皆有地主,救護必不辭勞。蝗既解飛,夜必赴火,夜中設火,火邊掘坑,且焚且瘞,除之可盡。時山東百姓皆燒香禮拜,設祭祈恩,眼看食苗,手不敢近。自古有討除不得者,只是人不用命,但使齊
 心戮力,必是可除。」乃遣御史分道殺蝗。汴州刺史倪若水執奏曰:「蝗是天災,自宜修德。劉聰時除既不得,為害更深。」仍拒御史,不肯應命。崇大怒,牒報若水曰:「劉聰偽主,德不勝妖;今日聖朝,妖不勝德。古之良守,蝗蟲避境,若其修德可免,彼豈無德致然!今坐看食苗,何忍不救,因以饑饉,將何自安?幸勿遲回,自招悔吝。」若水乃行焚瘞之法,獲蝗一十四萬石,投汴渠流下者不可勝紀。時朝廷喧議,皆以驅蝗為不便,上聞之,復以問崇。崇曰:「庸
 儒執文,不識通變。凡事有違經而合道者,亦有反道而適權者。昔魏時山東有蝗傷稼,緣小忍不除,致使苗稼總盡,人至相食;後秦時有蝗,禾稼及草木俱盡,牛馬至相啖毛。今山東蝗蟲所在流滿,仍極繁息,實所稀聞。河北、河南,無多貯積,倘不收獲,豈免流離,事系安危,不可膠柱。縱使除之不盡,猶勝養以成災。陛下好生惡殺,此事請不煩出敕,乞容臣出牒處分。若除不得,臣在身官爵,並請削除。」上許之。黃門監盧懷慎謂崇曰:「蝗是天災,
 豈可制以人事?外議咸以為非。又殺蟲太多,有傷和氣。今猶可復,請公思之。」崇曰:「楚王吞蛭,厥疾用瘳;叔敖殺蛇,其福乃降。趙宣至賢也,恨用其犬;孔丘將聖也,不愛其羊。皆志在安人,思不失禮。今蝗蟲極盛,驅除可得,若其縱食,所在皆空。山東百姓,豈宜餓殺!此事崇已面經奏定訖,請公勿復為言。若救人殺蟲,因緣致禍,崇請獨受,義不仰關。」懷慎既庶事曲從,竟亦不敢逆崇之意,蝗因此亦漸止息。



 是時,上初即位,務修德政,軍國庶務,多
 訪於崇,同時宰相盧懷慎、源乾曜等,但唯諾而已。崇獨當重任,明於吏道,斷割不滯。然縱其子光祿少卿彞、宗正少卿異廣引賓客,受納饋遺,由是為時所譏。時有中書主書趙誨為崇所親信,受蕃人珍遺,事發,上親加鞫問,下獄處死。崇結奏其罪,復營救之,上由是不悅。其冬,曲赦京城,敕文時標誨名,令決杖一百,配流嶺南。崇自是憂懼,頻面陳避相位,薦宋璟皆獲進見。有人於洛水中獲自代。俄授開府儀同三司,罷知政事。



 居月餘,玄宗將幸東都,而太廟屋壞,上召
 宋璟、蘇頲問其故,璟等奏言:「陛下三年之制未畢,誠不可行幸。凡災變之發,皆所以明教誡。陛下宜增崇大道,以答天意,且停幸東都。」上又召崇問曰:「朕臨發京邑,太廟無故崩壞,恐神靈誡以東行不便耶?」崇對曰:「太廟殿本是苻堅時所造,隋文帝創立新都,移宇文朝故殿造此廟,國家又因隋氏舊制,歲月滋深,朽蠹而毀。山有朽壞,尚不免崩,既久來枯木,合將摧折,偶與行期相會,不是緣行乃崩。且四海為家,兩京相接,陛下以關中不甚
 豐熟,轉運又有勞費,所以為人行幸,豈是無事煩勞?東都百司已作供擬,不可失信於天下。以臣愚見,舊廟既朽爛,不堪修理,望移神主於太極殿安置,更改造新廟,以申誠敬。車駕依前徑發。」上曰:「卿言正合朕意。」賜絹二百匹,令所司奉七廟神主於太極殿,改新廟,車駕乃幸東都。因令崇五日一參,仍入閤供奉,甚承恩遇。後又除太子少保,以疾不拜。九年薨,年七十二,贈揚州大都督,謚曰文獻。



 璟崇先分其田園,令諸子侄各守其分,仍為遺
 令以誡子孫,其略曰:



 古人云:富貴者,人之怨也。貴則神忌其滿,人惡其上;富則鬼瞰其室,虜利其財。自開闢已來,書籍所載,德薄任重而能壽考無咎者,未之有也。故範蠡、疏廣之輩,知止足之分,前史多之。況吾才不逮古人,而久竊榮寵,位逾高而益懼,恩彌厚而增憂。往在中書,遘疾虛憊,雖終匪懈,而諸務多闕。薦賢自代,屢有誠祈,人欲天從,竟蒙哀允。優游園沼,放浪形骸,人生一代,斯亦足矣。田巴云:「百年之期,未有能至。」王逸少云:「俯仰
 之間,已為陳跡。」誠哉此言。



 比見諸達官身亡以後,子孫既失覆廕,多至貧寒,鬥尺之間,參商是競。豈唯自玷,乃更辱先,無論曲直,俱受嗤毀。莊田水碾,既眾有之,遞相推倚,或致荒廢。陸賈、石苞,皆古之賢達也,所以預為定分,將以絕其後爭,吾靜思之,深所嘆服。



 昔孔子亞聖,母墓毀而不修;梁鴻至賢,父亡席卷而葬。昔楊震、趙咨、盧植、張奐,皆當代英達,通識今古,咸有遺言,屬以薄葬。或濯衣時服,或單帛幅巾,知真魂去身,貴於速朽,子孫皆
 遵成命,迄今以為美談。凡厚葬之家,例非明哲,或溺於流俗,不察幽明,咸以奢厚為忠孝,以儉薄為慳惜,至令亡者致戮尸暴骸之酷,存者陷不忠不孝之誚。可為痛哉!可為痛哉!死者無知,自同糞土,何煩厚葬,使傷素業。若也有知,神不在柩,復何用違君父之令,破衣食之資。吾身亡後,可殮以常服,四時之衣,各一副而已。吾性甚不愛冠衣,必不得將入棺墓,紫衣玉帶,足便於身,念爾等勿復違之。且神道惡奢,冥塗尚質,若違吾處分,使吾
 受戮於地下,於汝心安乎?念而思之。



 今之佛經,羅什所譯,姚興執本,與什對翻。姚興造浮屠於永貴里,傾竭府庫,廣事莊嚴,而興命不得延,國亦隨滅。又齊跨山東,周據關右,周則多除佛法而修繕兵威,齊則廣置僧徒而依憑佛力。及至交戰,齊氏滅亡,國既不存,寺復何有?修福之報,何其蔑如!梁武帝以萬乘為奴,胡太后以六宮入道,豈特身戮名辱,皆以亡國破家。近日孝和皇帝發使贖生,傾國造寺,太平公主、武三思、悖逆庶人、張夫人
 等皆度人造寺,竟術彌街,咸不免受戮破家,為天下所笑。經云:「求長命得長命,求富貴得富貴」,「刀尋段段壞,火坑變成池。」比求緣精進得富貴長命者為誰?生前易知,尚覺無應,身後難究,誰見有征。且五帝之時,父不葬子,兄不哭弟,言其致仁壽、無夭橫也。三王之代,國祚延長,人用休息,其人臣則彭祖、老聃之類,皆享遐齡。當此之時,未有佛教,豈抄經鑄像之力,設齋施佛之功耶?《宋書》《西域傳》,有名僧為《白黑論》,理證明白,足解沈疑,宜觀而
 行之。



 且佛者覺也,在乎方寸,假有萬像之廣,不出五蘊之中,但平等慈悲,行善不行惡,則佛道備矣。何必溺於小說,惑於凡僧,仍將喻品,用為實錄,抄經寫像,破業傾家,乃至施身亦無所吝,可謂大惑也。亦有緣亡人造像,名為追福,方便之教,雖則多端,功德須自發心,旁助寧應獲報?遞相欺誑,浸成風俗,損耗生人,無益亡者。假有通才達識,亦為時俗所拘。如來普慈,意存利物,損眾生之不足,厚豪僧之有餘,必不然矣。且死者是常,古來不
 免,所造經像,何所施為?



 夫釋迦之本法,為蒼生之大弊,汝等各宜警策,正法在心,勿效兒女子曹,終身不悟也。吾亡後必不得為此弊法。若未能全依正道,須順俗情,從初七至終七,任設七僧齋。若隨齋須布施,宜以吾緣身衣物充,不得輒用餘財,為無益之枉事,亦不得妄出私物,徇追福之虛談。



 道士者,本以玄牝為宗,初無趨競之教,而無識者慕僧家之有利,約佛教而為業。敬尋老君之說,亦無過齋之文,抑同僧例,失之彌遠。汝等勿拘
 鄙俗,輒屈於家。汝等身沒之後,亦教子孫依吾此法云。



 十七年,重贈崇太子太保。崇長子彞,開元初光祿少卿。次子異,坊州刺史。少子弈,少而修謹,開元末,為禮部侍郎、尚書右丞。天寶元年,右相牛仙客薨,彞男閎為侍御史、仙客判官,見仙客疾亟,逼為仙客表,請以弈及兵部侍郎盧奐為宰相代己。其妻因中使奏之,玄宗聞而怒之,閎決死,弈出為永陽太守,奐為臨淄太守。玄孫合,登進士第,授武功尉,遷監察御史,位終給事中。



 宋璟,邢州南和人,其先自廣平徙焉,後魏吏部尚書弁七代孫也。父玄撫,以璟貴,贈邢州刺史。璟少耿介有大節,博學,工於文翰。弱冠舉進士,累轉鳳閣舍人。當官正色,則天甚重之。長安中,幸臣張易之誣構御史大夫魏元忠有不順之言,引鳳閣舍人張說令證之。說將入於御前對覆,惶惑迫懼,璟謂曰:「名義至重,神道難欺,必不可黨邪陷正,以求茍免。若緣犯顏流貶,芬芳多矣。或至不測,吾必叩閤救子,將與子同死。努力,萬代瞻仰,在此
 舉也。」說感其言。及入,乃保明元忠,竟得免死。



 璟尋遷左御史臺中丞。張易之與弟昌宗縱恣益橫,傾朝附之。昌宗私引相工李弘泰觀占吉兇,言涉不順,為飛書所告。璟廷奏請窮究其狀,則天曰:「易之等已自奏聞,不可加罪。」璟曰:「易之等事露自陳,情在難恕,且謀反大逆,無容首免。請勒就御史臺勘當,以明國法。易之等久蒙驅使,分外承恩,臣必知言出禍從,然義激於心,雖死不恨。」則天不悅。內史楊再思恐忤旨,遽宣敕令璟出。璟曰:「天顏
 咫尺,親奉德音,不煩宰臣擅宣王命。」則天意稍解,乃收易之等就臺,將加鞫問。俄有特敕原之,仍令易之等詣璟辭謝,璟拒而不見,曰:「公事當公言之,若私見,則法無私也。」



 璟嘗侍宴朝堂,時易之兄弟皆為列卿,位三品,璟本階六品,在下。易之素畏璟,妄悅其意,虛位揖璟曰:「公第一人,何乃下座?」璟曰:「才劣品卑,張卿以為第一人,何也?」當時朝列,皆以二張內寵,不名官,呼易之為五郎,昌宗為六郎。天官侍郎鄭善果謂璟曰:「中丞奈何呼五
 郎為卿?」璟曰:「以官言之,正當為卿;若以親故,當為張五。足下非易之家奴,何郎之有?鄭善果一何懦哉!」其剛正皆此類也。自是易之等常欲因事傷之,則天察其情,竟以獲免。



 神龍元年,遷吏部侍郎。中宗嘉璟正直,仍令兼諫議大夫、內供奉,仗下後言朝廷得失。尋拜黃門侍郎。時武三思恃寵執權,嘗請托於璟,璟正色謂之曰:「當今復子明闢,王宜以侯就第,何得尚干朝政?王獨不見產、祿之事乎?」俄有京兆人韋月將上書訟三思潛通宮掖,
 將為禍患之漸,三思諷有司奏月將大逆不道,中宗特令誅之。璟執奏請按其罪狀,然後申明典憲,月將竟免極刑,配流嶺南而死。



 中宗幸西京,令璟權檢校並州長史,未行,又帶本官檢校貝州刺史。時河北頻遭水潦,百姓饑餒,三思封邑在貝州,專使徵其租賦,璟又拒而不與,由是為三思所擠。又歷杭、相二州刺史,在官清嚴,人吏莫有犯者。



 中宗晏駕,拜洛州長史。睿宗踐祚,遷吏部尚書、同中書門下三品。玄宗在春宮,又兼右庶子,加銀
 青光祿大夫。先是,外戚及諸公主干預朝政,請托滋甚。崔湜、鄭愔相次典選,為權門所制,九流失敘,預用兩年員闕注擬,不足,更置比冬選人,大為士庶所嘆。至是,璟與侍郎李乂、盧從願等大革前乂弊,取舍平允,銓綜有敘。



 時太平公主謀不利於玄宗,嘗於光範門內乘輦伺執政以諷之,眾皆失色。璟昌言曰:「東宮有大功於天下,真宗廟社稷之主,安得有異議!」乃與姚崇同奏請令公主就東都。玄宗懼,抗表請加璟罪於等,乃貶璟為楚州刺
 史。無幾,歷魏、兗、冀三州刺史,河北按察使。遷幽州都督、兼御史大夫。尋拜國子祭酒,兼東都留守。歲餘,轉京兆尹,復拜御史大夫,坐事出為睦州刺史,轉廣州都督,仍為五府經略使。廣州舊俗,皆以竹茅為屋,屢有火災。璟教人燒瓦,改造店肆,自是無復延燒之患,人皆懷惠,立頌以紀其政。



 開元初,徵拜刑部尚書。四年,遷吏部尚書,兼黃門監。明年,官名改易,為侍中,累封廣平郡公。其秋,駕幸東都,次永寧之崤谷,馳道隘狹,車騎停擁,河南尹
 李朝隱、知頓使王怡並失於部伍,上令黜其官爵。璟入奏曰:「陛下富有春秋,方事巡狩,一以墊隘,致罪二臣,竊恐將來人受艱弊。」於是遽令舍之。璟曰:「陛下責之,以臣言免之,是過歸於上而恩由於下。請且使待罪於朝,然後詔復其職,則進退得其度矣。」上深善之。俄又令璟與中書侍郎蘇頲為皇子制名及封邑,並公主等邑號。璟等奏曰:「王子將封三十餘國,周之麟趾,漢之犬牙,彼何足雲,於斯為盛。竊以郯、郟王等傍有古邑字,臣等以類
 推擇,謹件三十國名。又王子先有名者,皆上有『嗣』字,又公主邑號,亦選擇三十美名,皆文不害意,言足定體。又令臣等別撰一佳名及一美邑號者。七子均養,百王至仁,今若同等別封,或緣母寵子愛,骨肉之際,人所難言,天地之中,典有常度。昔袁盎降慎夫人之席,文帝竟納之,慎夫人亦不以為嫌,美其得久長之計。臣等故同進,更不別封,上彰覆載無偏之德。」上稱嘆之。



 七年,開府儀同三司王皎卒,及將築墳,皎子駙馬都尉守一請同昭
 成皇后父竇孝諶故事,其墳高五丈一尺。璟及蘇頲請一依禮式,上初從之。翌日,又令準孝諶舊例。璟等上言曰:



 夫儉,德之恭;侈,惡之大。高墳乃昔賢所誡,厚葬實君子所非。古者墓而不墳,蓋此道也。凡人子於哀送之際,則不以禮制為思。故周、孔設齊斬緦免之差,衣衾棺郭之度,賢者俯就,私懷不果。且蒼梧之野,驪山之徒,善惡分區,圖史所載。眾人皆務奢靡而獨能革之,斯所謂至孝要道也。中宮若以為言,則此理固可敦諭。



 在外或云
 竇太尉墳甚高,取則不遠者。縱令往日無極言,其事偶行,令出一時,故非常式。又貞觀中文德皇后嫁所生女長樂公主,奏請儀注加於長公主,魏徵諫云:「皇帝之姑姊為長公主,皇帝之女為公主,既有『長』字,合高於公主。若加於長公主,事甚不可。」引漢明故事云:「群臣欲封皇子為王,帝曰:『朕子豈敢與先帝子等。』」時太宗嘉納之。文德皇后奏降中使致謝於征。此則乾坤輔佐之間,綽有餘裕。豈若韋庶人父追加王位,擅作邦陵,禍不旋踵,為
 天下笑。則犯顏逆耳,阿意順旨,不可同日而言也。



 況令之所載,預作紀綱,情既無窮,故為之制度,不因人以搖動,不變法以愛憎。頃謂金科玉條,蓋以此也。比來蕃夷等輩及城市閑人,遞以奢靡相高,不將禮儀為意。今以後父之寵,開府之榮,金穴玉衣之資,不憂少物;高墳大寢之役,不畏無人。百事皆出於官,一朝亦可以就。而臣等區區不已以聞,諒欲成朝廷之政,崇國母之德,化浹寰區,聲光竹素。倘中宮情不可奪,陛下不能苦違,即準
 一品合陪陵葬者,墳高三丈已上,四丈已下,降敕將同陪陵之例,即極是高下得宜。



 上謂璟等曰:「朕每事常欲正身以成綱紀,至於妻子,情豈有私?然人所難言,亦在於此。卿等乃能再三堅執,成朕美事,足使萬代之後,光揚我史策。」乃遣使齎彩絹四百匹分賜之。



 先是,朝集使每至春將還,多有改轉,率以為常,璟奏請一切勒還,絕其僥求之路。又禁斷惡錢,發使分道檢括銷毀之,頗招士庶所怨。俄授璟開府儀同三司,罷知政事。明年,京兆
 人權梁山構逆伏誅,制河南尹王怡馳傳往長安窮其枝黨。怡禁系極眾,久之未能決斷,乃詔璟兼京兆留守,並按覆其獄。璟至,惟罪元謀數人,其餘緣梁山詐稱婚禮因假借得罪及脅從者,盡奏原之。十二年,駕又東巡,璟復為留守。上臨發,謂璟曰:「卿國之元老,為朕股肱耳目。今將巡洛邑,為別歷時,所有嘉謨嘉猷,宜相告也。」璟因極言得失,特賜彩絹等,仍手制曰:「所進之言,書之座右,出入觀省,以誡終身。」其見重如此。俄又兼吏部尚書。
 十七年,遷尚書右丞相,與張說、源乾曜同日拜官。敕太官設饌,太常奏樂,於尚書都省大會百僚。玄宗賦詩褒述,自寫與之。



 二十年,以年老上表曰:「臣聞力不足者,老則更衰;心無主者,疾而尤廢。臣昔聞其語,今驗諸身,況且兼之,何能為也。臣自拔跡幽介,欽屬盛明,才不逮人,藝非經國。復以久承驅策,歷參試用,命偶時來,榮因歲積。遂使再升臺座,三入塚司,進階開府,增封本郡。所更中外,已紊彞章,逮居端揆,左叨名職。何者?丞相官師之
 長,任重昔時;愚臣衰朽之餘,用慚他日。位則愈盛,人則浸微,盡知其然,何居而可?頃僶俯從政,蒼黃不言,實懷覆載之德,冀竭涓塵之效。今積羸成憊,沈錮莫瘳,耳目更昏,手足多廢。顧惟殞越,寧遂宿心?安可以茍徇大名,仍尸重祿,且留章綬,不上闕庭。儀刑此乖,禮法何設?伏惟陛下審能以授,為官而擇,察臣之懇詞,矜臣之不逮,使罷歸私第,養疾衡門,上弭官謗,下知死所。則歸全之望,獲在愚臣;養老之恩,成於聖代。日暮途遠,天高聽卑,
 瞻望軒墀,伏深感戀。謹奉表陳乞以聞。」手敕許之,仍令全給祿俸。璟乃退歸東都私第,屏絕人事,以就醫藥。二十二年,駕幸東都,璟於路左迎謁,上遣榮王親勞問之,自是頻遣使送藥餌。二十五年薨,年七十五,贈太尉,謚曰文貞。



 子昇,天寶初太僕少卿。次尚,漢東太守。次渾,與右相李林甫善,引為諫議大夫、平原太守、御史中丞、東京採訪使。次恕,都官郎中、劍南採訪判官,依倚權勢,頗為貪暴。渾在平原,重征一年庸調。作東畿採訪使,又
 使河南尉楊朝宗影娶妻鄭氏。鄭氏即薛稷外孫,姊為宗婦,孀居有色,渾有妻,使朝宗聘而渾納之,奏朝宗為赤尉。恕在劍南,有雒縣令崔珪,恕之表兄,妻美,恕誘而私之,而貶珪官。又養刺客李晏。至九載,並為人所發,贓私各數萬貫。林甫奏稱璟子渾就東京臺推,恕就本使劍南推,皆有實狀,渾流領南高要郡,恕流海康郡。尚,其載又為人訟其贓,貶臨海長史。其子華、衡,居官皆坐贓,相次流貶。其後渾會赦,量移至東陽郡下,請托過求,及役
 使人吏,求其資課,人不堪其弊,訟之,配流潯江郡。然兄弟盡善飲謔,俳優雜戲,衡最粗險,廣平之風教,無復存矣。廣德後,渾除太子諭德,為物議薄之,乃留寓於江嶺卒。



 史臣曰:履艱危則易見良臣,處平定則難彰賢相。故房、杜預創業之功,不可儔匹。而姚、宋經武、韋二後,政亂刑淫,頗涉履於中,克全聲跡,抑無愧焉。



 贊曰:姚、宋入用,刑政多端。為政匪易,防刑益難。諫諍以
 猛,施張用寬。不有其道,將何以安?



\end{pinyinscope}