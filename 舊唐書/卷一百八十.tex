\article{卷一百八十}

\begin{pinyinscope}

 ○李宗閔楊嗣復子授損技拭捴楊虞卿弟漢公從兄汝士馬植李讓夷魏鷿周墀崔龜從鄭肅盧商



 李宗閔,字損之,宗室鄭王元懿之後。祖自仙,楚州別駕。
 父䎖,宗正卿,出為華州刺史、鎮國軍潼關防禦等使。䎖兄夷簡,元和中宰相。宗閔,貞元二十一年進士擢第,元和四年,復登制舉賢良方正科。



 初,宗閔與牛僧孺同年登進士第,又與僧孺同年登制科。應制之歲,李吉甫為宰相當國,宗閔、僧孺對策,指切時政之失,言甚鯁直,無所回避。考策官楊於陵、韋貫之、李益等又第其策為中等,又為不中第者注解牛、李策語,同為唱誹。又言翰林學士王涯甥皇甫湜中選,考核之際,不先上言。裴垍時
 為學士,居中覆視,無所異同。吉甫泣訴於上前,憲宗不獲已,罷王涯、裴垍學士。垍守戶部侍郎,涯守都官員外郎,吏部尚書楊於陵出為嶺南節度使,吏部員外郎韋貫之出為果州刺史。王涯再貶虢州司馬,貫之再貶巴州刺史;僧孺、宗閔亦久之不調,隨牒諸侯府。七年,吉甫卒,方入朝為監察御史,累遷禮部員外郎。



 元和十二年,宰相裴度出征吳元濟,奏宗閔為彰義軍觀察判官。賊平,遷駕部郎中,又以本官知制誥。穆宗即位,拜中書舍
 人。時䎖自宗正卿出刺華州,父子同時承恩制,人士榮之。長慶元年,子婿蘇巢於錢徽下進士及第,其年,巢覆落。宗閔涉請托,貶劍州刺史。時李吉甫子德裕為翰林學士,錢徽榜出,德裕與同職李紳、元稹連衡言於上前,云徽受請托,所試不公,故致重覆。比相嫌惡,因是列為朋黨,皆挾邪取權,兩相傾軋。自是紛紜排陷,垂四十年。



 復入為中書舍人。三年冬,權知禮部侍郎。四年,貢舉事畢,權知兵部侍郎。寶歷元年,正拜兵部侍郎,父憂免。太
 和二年,起為吏部侍郎,賜金紫之服。三年八月,以本官同平章事。



 時裴度薦李德裕,將大用。德裕自浙西入朝,為中人助宗閔者所沮,復出鎮。尋引牛僧孺同知政事,二人唱和,凡德裕之黨皆逐之。累轉中書侍郎、集賢大學士。七年,德裕作相。六月,罷宗閔知政事,檢校禮部尚書、同平章事、興元尹、山南西道節度使。



 宗閔為吏部侍郎時,因駙馬都尉沈結托女學士宋若憲及知樞密楊承和,二人數稱之於上前,故獲徵用。及德裕秉政,群
 邪不悅,而鄭注、李訓深惡之。文宗乃復召宗閔於興元,為中書侍郎、平章事,命德裕代宗閔為興元尹。既再得權位,輔之以訓、注,尤恣所欲,進封襄武侯,食邑千戶。



 九年六月,京兆尹楊虞卿得罪,宗閔極言救解,文宗怒叱之曰:「爾嘗謂鄭覃是妖氣,今作妖,覃耶、爾耶?」翌日,貶明州刺史,尋再貶處州長史。七月,鄭注發沈、宋若憲事,內官楊承和、韋元素、沈及若憲姻黨坐貶者十餘人,又貶宗閔潮州司戶。時訓、注竊弄威權,凡不附己者,目
 為宗閔、德裕之黨,貶逐無虛日,中外震駭,連月陰晦,人情不安。九月詔曰:



 朕承天纘歷,燭理不明,勞虛襟以求賢,勵寬德以容眾。頃者,或臺輔乖弼違之道,而具僚扇朋附之風;翕然相從,實篸彞憲。致使薰蕕共器,賢不肖並馳;退跡者成後時之夫,登門者有迎吠之客。繆戾之氣,堙鬱和平,而望陰陽順時,疵癘不作;朝廷清肅,班列和安,自古及今,未嘗有也。今既再申朝典,一變澆風,掃清朋比之徒,匡飭貞廉之俗。凡百卿士,惟新令猷。如聞
 周行之中,尚蓄疑懼,或有妄相指目,令不自安,今斯曠然,明喻朕意。應與宗閔、德裕或親或故及門生舊吏等,除今日已前黜遠之外,一切不問。各安職業,勿復為嫌。



 文宗以二李朋黨,繩之不能去,嘗謂侍臣曰:「去河北賊非難,去此朋黨實難。」宗閔雖驟放黜,竟免李訓之禍。



 開成元年,量移衢州司馬。三年,楊嗣復輔政,與宗閔厚善,欲拔用之,而畏鄭覃沮議,乃托中人密諷於上。上以嗣復故,因紫宸對,謂宰相曰:「宗閔在外四五年,宜別授一
 官。」鄭覃曰:「陛下憐其地遠,宜移近內地三五百里,不可再用奸邪。陛下若欲用宗閔,臣請先退。」陳夷行曰:「比者,宗閔得罪,以朋黨之故,恕死為幸。寶歷初,李續之、張又新、蘇景胤等,朋比奸險,幾傾朝廷,時號『八關十六子』。」李玨曰:「主此事者,罪在逢吉。李續之居喪服闋,不可不與一官,臣恐中外衣冠,交興議論,非為續之輩也。」夷行曰:「昔舜逐四兇天下治。朝廷求理,何惜此十數纖人?」嗣復曰:「事貴得中,不可但徇憎愛。」上曰:「與一郡可也。」鄭覃曰:「
 與郡太優,止可洪州司馬耳。」夷行曰:「宗閔養成鄭注之惡,幾覆邦家,國之巨蠹也。」嗣復曰:「比者,陛下欲加鄭注官,宗閔不肯,陛下亦當記憶。」覃曰:「嗣復黨庇宗閔。臣觀宗閔之惡,甚於李林甫。」嗣復曰:「覃語大過。昔玄宗季年,委用林甫,妒賢害能,破人家族。宗閔在位,固無此事。況太和末,宗閔、德裕同時得罪。二年之間,德裕再領重鎮,而宗閔未離貶所。陛下懲惡勸善,進退之理宜均,非臣獨敢黨庇。昨殷侑與韓益奏官及章服,臣以益前年犯
 贓,未可其奏;鄭覃托臣云『幸且勿論。』孰為黨庇?」翌日,以宗閔為杭州刺史。四年冬,遷太子賓客,分司東都。時鄭覃、陳夷行罷相,嗣復方再拔用宗閔知政事,俄而文宗崩。



 會昌初,李德裕秉政,嗣復、李玨皆竄嶺表。三年,劉稹據澤潞叛。德裕以宗閔素與劉從諫厚,上黨近東都,宗閔分司非便,出為封州刺史。又發其舊事,貶郴州司馬,卒於貶所。



 子琨、瓚,大中朝皆進士擢第。令狐綯作相,特加獎拔。瓚自員外郎知制誥,歷中書舍人、翰林學士。綯
 罷相,出為桂管觀察使。御軍無政,為卒所逐,貶死。



 自天寶艱難之後,宗室子弟,賢而立功者,唯鄭王、曹王子孫耳。夷簡再從季父汧國公勉,德宗朝宰相。夷簡諸弟夷亮、夷則、夷範,皆登進士第。宗閔弟宗冉。宗冉子深、湯。湯累官至給事中,咸通中踐更臺閣,知名於時。



 楊嗣復,字繼之,僕射於陵子也。初,於陵十九登進十第,二十再登博學宏詞科,謂補潤州句容尉。浙西觀察使韓滉有知人之鑒,見之甚悅。滉有愛女,方擇佳婿,謂其
 妻柳氏曰:「吾閱人多矣,無如楊生貴而有壽,生子必為宰相。」於陵秩滿,寓居揚州而生嗣復。後滉見之,撫其首曰:「名位果逾於父,楊門之慶也。」因字曰慶門。



 嗣復七八歲時已能秉筆為文。年二十,進士擢第。二十一,又登博學宏詞科,釋褐秘書省校書郎。遷右拾遺,直史館。以嗣復深於禮學,改太常博士。元和十年,累遷至刑部員外郎。鄭餘慶為詳定禮儀使,奏為判官,改禮部員外郎。時父於陵為戶部侍郎,嗣復上言與父同省非便,請換他
 官。詔曰:「應同司官有大功以下親者,但非連判及勾檢之官並官長,則不在回避之限。如官署同,職司異,雖父子兄弟無所避嫌。」再遷兵部郎中。長慶元年十月,以庫部郎中知制誥,正拜中書舍人。



 嗣復與牛僧孺、李宗閔皆權德輿貢舉門生,情義相得,進退取舍,多與之同。四年,僧孺作相,欲薦拔大用,又以於陵為東都留守。未歷相位,乃令嗣復權知禮部侍郎。寶歷年元二月,選貢士六十八人,後多至達官。文宗即位,拜戶部侍郎。以父於
 陵太子少傅致仕,年高多疾,懇辭侍養,不之許。太和四年,丁父憂免。七年三月,起為尚書左丞。其年宗閔罷相,德裕輔政。七月,以嗣復檢校禮部尚書、梓州刺史、劍南東川節度觀察等使。九年,宗閔復知政事。三月,以嗣復檢校戶部尚書、成都尹、劍南西川節度副大使知節度事、觀察處置等使。



 開成二年十月,入為戶部侍郎,領諸道鹽鐵轉運使。三年正月,與同列李玨並以本官同平章事,領使如故,進階金紫,弘農伯,食邑七百戶。上以幣
 輕錢重,問鹽鐵使何以去其太甚?嗣復曰:「此事累朝制置未得,但且禁銅,未可變法。法變擾人,終亦未罷去弊。」李玨曰:「禁銅之令,朝廷常典,但行之不嚴,不如無令。今江淮已南,銅器成肆,市井逐利者,銷錢一緡,可為數器,售利三四倍。遠民不知法令,率以為常。縱國家加爐鑄錢,何以供銷鑄之弊?所以禁銅之令,不得不嚴。」



 八月,紫宸奏事,曰:「聖人在上,野無遺賢。陸洿上疏論兵,雖不中時事,意亦可獎。閑居蘇州累年,宜與一官。」李玨曰:「
 士子趨競者多,若獎陸洿,貪夫知勸矣。昨竇洵直論事,陛下賞之以幣帛,況與陸洿官耶?」帝曰:「洵直獎其直心,不言事之當否。」鄭覃曰:「若苞藏則不可知。」嗣復曰:「臣深知洵直無邪惡,所奏陸洿官,尚未奉聖旨。」鄭覃曰:「陛下須防朋黨。」嗣復曰:「鄭覃疑臣朋黨,乞陛下放臣歸去。」因拜乞罷免。李玨曰:「比來朋黨,近亦稍弭。」覃曰:「近有小朋黨生。」帝曰:「此輩凋喪向盡。」覃曰:「楊漢公、張又新、李續之即今尚在。」玨曰:「今有邊事論奏。」覃曰:「論邊事安危,臣不如玨;
 嫉惡則玨不如臣。」嗣復曰:「臣聞左右佩劍,彼此相笑。臣今不知鄭覃指誰為朋黨。」因當香案前奏曰:「臣待罪宰相,不能申夔、龍之道,唯以朋黨見譏,必乞陛下罷臣鼎職。」上慰勉之。文宗方以政事委嗣復,惡覃言切。



 帝延英謂宰臣曰:「人傳符讖之語,自何而來?」嗣復對曰:「漢光武好以讖書決事,近代隋文帝亦信此言,自是,此說日滋,只如班彪《王命論》所引,蓋矯意以止賊亂,非所重也。」李玨曰:「喪亂之時,佐命者務神符命;理平之代,只合推諸
 人事。」上曰:「卿言是也。」帝又曰:「天後用人,有自布衣至宰相者,當時還得力否?」嗣復曰:「天後重行刑闢,輕用官爵,皆自圖之計耳。凡用人之道,歷試方見其能否。當艱難之時,或須拔擢,無事之日,不如且循資級。古人拔卒為將,非治平之時,蓋不獲已而用之也。」上又問新修《開元政要》,敘致何如。嗣復曰:「臣等未見。陛下若欲遺之子孫,則請宣付臣等,參詳可否。玄宗或好游畋,或好聲色,與貞觀之政不同,故取舍須當,方堪流傳。」



 四年五月,上問
 延英政事,逐日何人記錄監修?李玨曰:「是臣職司。」陳夷行曰:「宰相所錄,必當自伐,聖德即將掩之。臣所以頻言,不欲威權在下。」玨曰:「夷行此言,是疑宰相中有賣威權、貨刑賞者。不然,何自為宰相而出此言?臣累奏求退,若得王傅,臣之幸也。」鄭覃曰:「陛下開成元年、二年政事至好,三年、四年漸不如前。」嗣復曰:「元年、二年是鄭覃、夷行用事,三年、四年臣與李玨同之。臣蒙聖慈擢處相位,不能悉心奉職。鄭覃云『三年之後,一年不如一年』,臣之罪
 也。陛下縱不誅夷,臣合自求泯滅。」因叩頭曰:「臣今日便辭玉階,不敢更入中書。」即趨去。上令中使召還,勞之曰:「鄭覃失言,卿何及此?」覃起謝曰:「臣性愚拙,言無顧慮。近日事亦漸好,未免些些不公,亦無甚處。臣亦不獨斥嗣復,遽何至此。所為若是,乃嗣復不容臣耳。」嗣復曰:「陛下不以臣微才,用為中書侍郎。時政善否,其責在臣。陛下月費俸錢數十萬,時新珍異,必先賜與,蓋欲輔佐聖明,臻於至理。既一年不如一年,非惟臣合得罪,亦上累聖
 德。伏請別命賢能,許臣休退。」上曰:「鄭覃之言偶然耳,奚執咎耶?」嗣復數日不入,上表請罷。帝方委用,乃罷鄭覃、夷行知政事。自是,政歸嗣復,進加門下侍郎。明年正月,文宗崩。



 先是,以敬宗子陳王為皇太子。中尉仇士良違遺令立武宗。武宗之立,既非宰相本意,甚薄執政之臣。其年秋,李德裕自淮南入輔政。九月,出嗣復為湖南觀察使。明年,誅樞密薛季稜、劉弘逸。中人言:「二人頃附嗣復、李玨,不利於陛下。」武宗性急,立命中使往湖南、桂管,
 殺嗣復與玨。宰相崔鄲、崔珙等亟請開延英,因極言國朝故事,大臣非惡逆顯著,未有誅戮者,願陛下復思其宜。帝良久改容曰:「朕纘嗣之際,宰相何嘗比數。李玨、季稜志在扶冊陳王,嗣復、弘逸志在樹立安王。立陳王猶是文宗遺旨,嗣復欲立安王,全是希楊妃意旨。嗣復嘗與妃書云:『姑姑何不敩則天臨朝?』」珙等曰:「此事曖昧,真虛難辨。」帝曰:「楊妃曾臥疾,妃弟玄思,文宗令入內侍疾月餘,此時通導意旨。朕細問內人,情狀皎然,我不欲宣
 出於外。向使安王得志,我豈有今日?然為卿等恕之。」乃追潭、桂二中使,再貶嗣復潮州刺史。



 宣宗即位,徵拜吏部尚書。大中二年,自潮陽還,至岳州病,一日而卒,時年六十六。贈左僕射,謚曰孝穆。



 子損、授、技、拭、捴,而授最賢。



 授,字得符,大中九年進士擢第,釋褐從事諸侯府,入為鄠縣尉、集賢校理。歷監察御史、殿中,分務東臺。再遷司勛員外郎、洛陽令、兵部員外郎。李福為東都留守,奏充判官,改兵部郎中,由吏部拜左諫議大夫、給事中,出為
 河南尹。盧攜作相,召拜工部侍郎。黃巢犯京師,僖宗幸蜀,徵拜戶部侍郎。以母病,求散秩,改秘書監分司。車駕還,拜兵部侍郎。宰相有報怨者,改左散騎常侍、國子祭酒,又轉太子賓客。從昭宗在華下,改刑部尚書、太子少保。卒,贈左僕射。



 子煚,字公隱,進士及第,再遷左拾遺。昭宗初即位,喜游宴,不恤時事,煚上疏極諫,帝面賜緋袍象笏。崔安潛出鎮青州,闢為支使。不至鎮,改太常博士。歷主客、戶部二員外郎。關中亂,崔胤引硃全忠入京師,
 乃挈家避地湖南,官終諫議大夫。



 損,字子默,以廕受官,為藍田尉。三遷京兆府司錄參軍,入為殿中侍御史。家在新昌里,與宰相路巖第相接。巖以地狹,欲易損馬廄廣之,遣人致意。時損伯叔昆仲在朝者十餘人,相與議曰:「家門損益恃時相,何可拒之?」損曰:「非也。凡尺寸地,非吾等所有。先人舊業,安可以奉權臣?窮達,命也。」巖不悅。會差制使鞫獄黔中,乃遣損使焉。逾年而還,改戶部員外郎、洛陽縣令。入為吏部員外,出為絳州刺史。路巖罷
 相,徵拜給事中,遷京兆尹。盧攜作相,有宿憾,復拜給事中,出為陜虢觀察使。時軍亂,逐前使崔蕘。損至,盡誅其亂首。逾年,改青州刺史、御史大夫、淄青節度使。又檢校刑部尚書、鄆州刺史、天平軍節度使。未赴鄆,復留青州,卒於鎮。



 技進士及第,位至中書舍人。



 拭官終考功員外郎。捴終兵部郎中。拭、捴並進士擢第。



 楊虞卿,字師皋,虢州弘農人。祖燕客。父寧,貞元中為長安尉。少有棲遁之志,以處士徵入朝。有口辯,優游公卿
 間。竇參尤重之,會參貶,仕進不達而卒。



 虞卿,元和五年進士擢第,又應博學宏辭科。元和末,累官至監察御史。穆宗初即位,不修政道,盤游無節,虞卿上疏諫曰:



 臣聞鳶烏遭害則仁鳥逝,誹謗不誅則良言進。況詔旨勉諭,許陳愚誠,故臣不敢避誅,以獻狂瞽。



 竊聞堯、舜受命,以天下為憂,不聞以位為樂。況北虜猶梗,西戎未賓,兩河之瘡磐未平,五嶺之妖氛未解。生人之疾苦盡在,朝廷之制度莫修,邊儲屢空,國用猶屈。固未可以高枕無虞
 也。



 陛下初臨萬宇,有憂天下之志。宜日延輔臣公卿百執事,凝旒而問,造膝以求,使四方內外,有所觀焉。自聽政已來,六十日矣,八開延英,獨三數大臣仰龍顏,承聖問。其餘侍從詔誥之臣,偕入而齊出,何足以聞政事哉!諫臣盈廷,忠言未聞於聖聽,臣實羞之。蓋由主恩尚疏,而眾正之路未啟也。



 夫公卿大臣,宜朝夕接見論道,賜與從容,則君臣之情相接,而理道備聞矣。今自宰相已下四五人,時得頃刻侍坐,天威不遠,鞠躬隕越,隨旨上
 下,無能往來。此由君太尊、臣太卑故也。自公卿已下,雖歷踐清地,曾未祗奉天睠,以承下問,鬱塞正路,偷安幸門。況陛下神聖如五帝,臣下莫能望清光。所宜周遍顧問,惠其氣色,使支體相輔,君臣喻明。陛下求理於公卿,公卿求理於臣輩,自然上下孜孜相問,使進忠若趨利,論政若訴冤。如此而不聞過失、不致升平者,未之有也。



 自古帝王,居危思安之心不相殊,而居安慮危之心不相及,故不得皆為聖帝明王。



 小臣疏賤,豈宜及此,獨不
 忍冒榮偷祿,以負聖朝。惟陛下圖之。



 帝深獎其言。尋令奉使西北邊,犒賞戍卒,遷侍御史,再轉禮部員外郎、史館修撰。長慶四年八月,改吏部員外郎。



 太和二年,南曹令史李幹等六人,偽出告身簽符,賣鑿空偽官,令赴任者六十五人,取受錢一萬六千七百三十貫。虞卿按得偽狀,捕幹等移御史臺鞫劾。幹稱六人共率錢二千貫,與虞卿典溫亮,求不發舉偽濫事跡。乃詔給事中嚴休復、中書舍人高鉞、左丞韋景休充三司推案,而溫亮
 逃竄。幹等既伏誅,虞卿以檢下無術,停見任。



 及李宗閔、牛僧孺輔政,起為左司郎中。五年六月,拜諫議大夫,充弘文館學士,判院事。六年,轉給事中。七年,宗閔罷相,李德裕知政事,出為常州刺史。



 虞卿性柔佞,能阿附權幸以為奸利。每歲銓曹貢部,為舉選人馳走取科第,占員闕,無不得其所欲;升沉取舍,出其脣吻。而李宗閔待之如骨肉,以能朋比唱和,故時號黨魁。八年,宗閔復入相,尋召為工部侍郎。九年四月,拜京兆尹。其年六月,京師
 訛言鄭注為上合金丹,須小兒心肝,密旨捕小兒無算。民間相告語,扃鎖小兒甚密,街肆洶洶。上聞之不悅,鄭注頗不自安。御史大夫李固言素嫉虞卿朋黨,乃奏曰:「臣昨窮問其由,此語出於京兆尹從人,因此扇於都下。」上怒,即令收虞卿下獄。虞卿弟漢公並男知進等八人自系,撾鼓訴冤,詔虞卿歸私第。翌日,貶虔州司馬,再貶虔州司戶,卒於貶所。



 子知進、知退、堪,弟漢公,皆登進士第。知退歷都官、戶部二郎中;堪庫部、吏部二員外郎。



 漢
 公,太和八年擢進士第,又書判拔萃,釋褐為李絳興元從事。絳遇害,漢公遁而獲免。累遷戶部郎中、史館修撰。太和七年,遷司封郎中。



 漢公子範、籌,皆登進士第,累闢使府。



 虞卿從兄汝士。汝士,字慕巢,元和四年進士擢第,又登博學宏詞科,累闢使府。長慶元年為右補闕。坐弟殷士貢舉覆落,貶開江令。入為戶部員外,再遷職方郎中。太和三年七月,以本官知制誥。時李宗閔、牛僧孺輔政,待汝士厚。尋正拜中書舍人,改工部侍郎。八年,出為
 同州刺史。九年九月,入為戶部侍郎。開成元年七月,轉兵部侍郎。其年十二月,檢校禮部尚書、梓州刺史、劍南東川節度使。時宗人嗣復鎮西川,兄弟對居節制,時人榮之。四年九月,入為吏部侍郎,位至尚書,卒。



 子知溫、知遠、知權,皆登進士第。



 知溫累官至禮部郎中、知制誥,入為翰林學士、戶部侍郎,轉左丞。出為河南尹、陜虢觀察使。遷檢校兵部尚書、襄州刺史、山南東道節度使。



 知溫弟知至,累官至比部郎中、知制誥。坐故府劉瞻罷相,貶
 官。知至亦貶瓊州司馬。入為諫議大夫,累遷京兆尹、工部侍郎。知溫、知至皆位至列曹尚書。



 汝士弟魯士。魯士,字宗尹,本名殷士。長慶元年,進士擢第,其年詔翰林覆試。殷士與鄭朗等覆落,因改名魯士。復登制科,位不達而卒。



 初汝士中第,有時名,遂歷清貫。其年諸子皆至正卿,鬱為昌族。所居靜恭里,知溫兄弟,並列門戟。咸通中,昆仲子孫,在朝行方鎮者十餘人。



 馬植,扶風人。父曛。植,元和十四年進士擢第,又登制策
 科,釋褐壽州團練副使。得秘書省校書郎,三遷饒州刺史。開成初,遷安南都護、御史中丞、安南招討使。



 植文雅之餘,長於吏術。三年,奏:「當管羈縻州首領,或居巢穴自固,或為南蠻所誘,不可招諭,事有可虞。臣自到鎮,約之以信誠,曉之以逆順。今諸首領,總發忠言,願納賦稅。其武陸縣請升為州,以首領為刺史。」從之。又奏陸州界廢珠池復生珠。以能政,就加檢校左散騎常侍,加中散大夫,轉黔中觀察使。會昌中,入為大理卿。



 植以文學政事
 為時所知。久在邊遠,及還朝,不獲顯官,必微有望,李德裕素不重之。宣宗即位,宰相白敏中與德裕有隙,凡德裕所薄者,必不次拔擢之。乃加植金紫光祿大夫,行刑部侍郎,充諸道鹽鐵轉運使。轉戶部侍郎,領使如故。俄以本官同平章事,遷中書侍郎,兼禮部尚書。敏中罷相,植亦罷為太子賓客,分司東都。數年,出為許州刺史、檢校刑部尚書、忠武軍節度觀察等使。大中末,遷汴州刺史、宣武軍節度觀察等使。卒於鎮。



 李讓夷,字達心,隴西人。祖悅,父應規。讓夷,元和十四年擢進士第,釋褐諸侯府。太和初入朝,為右拾遺,召充翰林學士,轉左補闕。三年,遷職方員外郎、左司郎中,充職。九年,拜諫議大夫。



 開成元年,以本官兼知起居舍人事。時起居舍人李褒有痼疾,請罷官。宰臣李石奏闕官,上曰:「褚遂良為諫議大夫,嘗兼此官,卿可盡言今諫議大夫姓名。」石遂奏李讓夷、馮定、孫簡、蕭俶。帝曰:「讓夷可也。」李固言欲用崔球、張次宗。鄭覃曰:「崔球游宗閔之門,赤
 墀下秉筆記注,為千古法,不可用朋黨。如裴中孺、李讓夷,臣不敢有纖芥異論。」其為人主大臣知重如此。二年,拜中書舍人。以鄭覃此言,深為李玨、楊嗣復所惡,終文宗世,官不達。



 及德裕秉政,驟加拔擢,歷工、戶二侍郎,轉左丞。累遷檢校尚書右僕射,俄拜中書侍郎,同平章事。宣宗即位罷相,以太子賓客分司卒。



 魏抃,字申之,鉅鹿人。五代祖文貞公征,貞觀朝名相。曾祖殷,汝陽令。祖明,亦為縣令。父馮,獻陵臺令。抃,太和七
 年登進士第。楊汝士牧同州,闢為防禦判官,得秘書省校書郎。汝士入朝,薦為右拾遺。文宗以抃魏徵之裔,頗奇待之。



 前邕管經略使董昌齡枉殺錄事參軍衡方厚,坐貶漵州司戶。至是量移硤州刺史,抃上疏論之曰:「王者施渙汗之恩以赦有罪,唯故意殺人無赦。昌齡比者錄以微效,授之方隅,不能祗慎寵光,恣其狂暴,無辜專殺,事跡顯彰。妻孥銜冤,萬里披訴。及按鞫伏罪,貸以微生,中外議論,以為屈法。今若授之牧守,以理疲人,則殺
 人者拔擢,而冤苦者何伸?交紊憲章,有乘至理。」疏奏,乃改為洪州別駕。



 御史中丞李孝本,皇族也,坐李訓誅,有女沒人掖廷。抃諫曰:



 臣聞治國家者,先資於德義;德義不修,家邦必壞。故王者以德服人,以義使人。服使之術,要在修身;修身之道,在於孜孜。夫一失百虧之戒,存乎久要之源。前志曰:「勿以小惡而為之,勿以小善而不為。」斯則懼於漸也!臣又聞,君如日焉,顯晦之微,人皆瞻仰;照臨之大,何以掩藏?前代設敢諫之鼓,立誹謗之木,貴
 聞其過也。陛下即位以來,誕敷文德,不悅聲色,出後宮之怨婦,配在外之鰥夫。洎今十年,未嘗採擇。自數月已來,天睠稍回,留神妓樂,教坊百人、二百人,選試未已;莊宅司收市,亹癖有聞。昨又宣取李孝本之女入內。宗姓不異,寵幸何名?此事深累慎修,有虧一簣。陛下九重之內,不得聞知。凡此之流,大生物議,實傷理道之本,未免塵穢之嫌。夫欲人不知,莫若勿為。諺曰:「止寒莫若重裘,止謗莫若自修。」伏希陛下照鑒不惑;崇千載之盛德,去
 一旦之玩好。教坊停息,宗女遣還,則大正人倫之風,深弘王者之體。



 疏奏,帝即日出孝本女,遷抃右補闕。詔曰:「昔乃先祖貞觀中諫書十上,指事直言,無所避諱。每覽國史,未嘗不沉吟伸卷,嘉尚久之。爾為拾遺,其風不墜,屢獻章疏,必道其所以。至於備灑掃於諸王,非自廣其聲妓也;恤髫齔之宗女,固無嫌於征取也。雖然,疑似之間,不可家至而戶曉。爾能詞旨深切,是博我之意多也。噫!人能匪躬謇諤,似其先祖;吾豈不能虛懷延納,仰希
 貞觀之理歟?而抃居官日淺,未當敘進,吾豈限以常典,以待直臣!可右補闕。」帝謂宰臣曰:「昔太宗皇帝得魏徵,裨補闕失,弼成聖政。我得魏抃,於疑似之間,必能極諫。不敢希貞觀之政,庶幾處無過之地矣。」



 教坊副使雲朝霞善吹笛,新聲變律,深愜上旨。自左驍衛將軍宣授兼揚府司馬。宰臣奏曰:「揚府司馬品高,郎官刺史迭處,不可授伶官。」上意欲授之,因宰臣對,亟稱朝霞之善。抃聞之,累疏陳論,乃改授潤州司馬。荊南監軍使呂令琮從
 人,擅入江陵縣,毀罵縣令韓忠,觀察使韋長申狀與樞密使訴之。抃上疏曰:「伏以州縣侵屈,只合上聞。中外關連,須存舊制。韋長任膺廉使,體合精詳,公事都不奏聞,私情擅為逾越。況事無巨細,不可將迎。縣令官業有乖,便宜理罪;監軍職司侵越,即合聞天。或以慮煩聖聽,何不但申門下?今則首紊常典,理合糾繩。伏望聖慈,速加懲戒!」疏奏不出,時論惜之。



 三年,轉起居舍人。紫宸中謝,帝謂之曰:「以卿論事忠切,有文貞之風,故不循月限,授
 卿此官。」又謂之曰:「卿家有何舊書詔?」對曰:「比多失墜,惟簪笏見存。」上令進來。鄭覃曰:「在人不在笏。」上曰:「鄭覃不會我意,此即《甘棠》之義,非在笏而已。」抃將退,又召誡之曰:「事有不當,即須奏論。」抃曰:「臣頃為諫官,合伸規諷。今居史職,職在記言,臣不敢輒逾職分。」帝曰:「凡兩省官並合論事,勿拘此言。」尋以本官直弘文館。



 四年,拜諫議大夫,仍兼起居舍人,判弘文館事。紫宸入閣,遣中使取抃起居注,欲視之。抃執奏曰:「自古置史官,書事以明鑒
 誡。陛下但為善事,勿畏臣不書。如陛下所行錯忤,臣縱不書,天下之人書之。臣以陛下為文皇帝,陛下比臣如褚遂良。」帝又曰:「我嘗取觀之。」抃曰:「由史官不守職分,臣豈敢陷陛下為非法?陛下一覽之後,自此書事須有回避。如此,善惡不直,非史也。遺後代,何以取信?」乃止。



 初立朝,為李固言、李玨、楊嗣復所引,數年之內,至諫議大夫。武宗即位,李德裕用事,抃坐楊、李之黨,出為汾州刺史。楊、李貶官,抃亦貶信州長史。宣宗即位,白敏中當國,量
 移郢州刺史,尋換商州。二年,內徵為給事中,遷御史中丞。謝日,面賜金紫之服。彈駙馬都尉杜中立贓罪,貴戚憚之。兼戶部侍郎,判本司事。抃奏曰:「御史臺紀綱之地,不宜與泉貨吏雜處,乞罷中司,專綜戶部公事。」從之。



 尋以本官同平章事,判使如故。謝日,奏曰:「臣無夔、契之才,驟叨夔、契之任,將何以仰報鴻私?今邊戍粗安,海內寧息,臣愚所切者,陛下未立東宮,俾正人傳導,以存副貳之重。」因泣下。上感而聽之。



 先是,累朝人君不欲人言立
 儲貳,若非人主己欲,臣下不敢獻言。宣宗春秋高,嫡嗣未辨,抃作相之日,率先啟奏,人士重之。尋兼集賢大學士。詹毗國獻象,抃以其性不安中土,請還其使,從之。太原節度使李業殺降虜,北邊大擾。業有所恃,人不敢非。抃即奏其事,乃移業滑州。加中書侍郎。大理卿馬曙從人王慶告曙家藏兵甲。曙坐貶官,而慶無罪。抃引法律論之,竟杖殺慶。



 進階銀青光祿大夫,兼禮部尚書、監修國史。修成《文宗實錄》四十卷,上之。其修史官給事中盧
 耽、太常少卿蔣偕、司勛員外郎王諷、右補闕盧告、膳部員外郎牛叢,皆頒賜錦彩、銀器,序遷職秩。抃轉門下侍郎,兼戶部尚書。大中十年,以本官平章事、成都尹、劍南西川節度副大使知節度事。十一年,以疾求代,徵拜吏部尚書。以疾未痊,乞授散秩,改檢校右僕射,守太子少保。十二年十二月卒,時年六十六,贈司徒。



 抃儀容魁偉,言論切直,與同列上前言事,他宰相必委曲規諷,唯抃讜言無所畏避。宣宗每曰:「魏謨綽有祖風,名公子孫,我
 心重之。」然竟以語辭太剛,為令狐綯所忌,罷之。



 抃嘗鈔撮子書要言,以類相從,二十卷,號曰《魏氏手略》。有文集十卷。



 子潛、滂。潛登進士第。潛子敖,韋琮甥。後琮為相,潛歷顯官。



 周墀,字德升,汝南人。祖頲,父霈。墀,長慶二年擢進士第,太和末,累遷至起居郎。墀能為古文,有史才。文宗重之,補集賢學士,轉考功員外郎,仍兼起居舍人事。開成二年冬,以本官知制誥,尋召充翰林學士。三年,遷職方郎
 中。四年十月,正拜中書舍人,內職如故。武宗即位,出為華州刺史、鎮國軍潼關防禦等使,改鄂州刺史、御史中丞、鄂岳觀察使。會昌六年十一月,遷洪州刺史、江南西道觀察使。大中初,檢校禮部尚書、滑州刺史、義成軍節度、鄭滑觀察等使、上柱國、汝南男,食邑三百戶。入朝為兵部侍郎、判度支。尋以本官同平章事,累遷銀青光祿大夫、中書侍郎、監修國史,兼刑部尚書。罷相,檢校刑部尚書、梓州刺史、御史大夫、劍南東川節度使。未行,追制
 改檢校右僕射,加食邑五百戶。歷方鎮卒。



 崔龜從,字玄告,清河人。祖璜,父誠,官微。龜從,元和十二年擢進士第,又登賢良方正制科,及書判拔萃二科,釋褐拜右拾遺。太和二年,改太常博士。



 龜從長於禮學,精歷代沿革,問無不通。時饗宗廟於敬宗室,祝板稱皇帝孝弟。龜從議曰:「臣審祥孝字,載考禮文,義本主於子孫,理難施於兄弟。按《禮記》卜虞之文,子孫曰哀,兄弟曰某。然則虞之稱哀,與祭之稱孝,其義一也。於祖禰則理宜
 稱孝,於伯仲則止可稱名。又東晉溫嶠議宗廟祝辭,於孝字非子者則不稱,傍親直言敢告。當時朝議,咸以為宜。今臣上考禮經,無兄弟稱孝之義;下征晉史,有不稱傍親之文。臣謂饗敬宗廟,宜去孝弟兩字。」



 又以祀九宮壇,舊是大祠。龜從議曰:「九宮貴神,經典不載。天寶中,術士奏請,遂立祠壇。事出一時,禮同郊祀。臣詳其圖法,皆主星名,縱司水旱兵荒,品秩不過列宿。今者,五星悉是從祀,日月猶在中祠,豈容九宮獨越常禮,備列王事,誡
 誓百官?尊卑乖儀,莫甚於此。若以嘗在祀典,不可廢除,臣請降為中祠。」制從之。



 龜從又以大臣薨謝,不於聞哀日輟朝,奏議曰:「伏以廢朝軫悼,義重君臣,所貴及哀,尤宜示信。自頃已來,輟朝非奏報之時,備禮於數日外。雖遵常制,似不本情。臣不敢遠征古書,請引國朝故事:貞觀中任瑰卒,有司對仗奏闕聞,太宗責其乖禮;岑文本既歿,其夕為罷警嚴;張公謹之亡,哭之不避辰日。是知閔悼之意,不宜過時。臣謂大臣薨,禮合輟朝。縱有機務
 急速,便殿須召宰臣,不臨正朝,無爽事體。如此,則由衷之信,載感於幽明;稱情之文,無虧於典禮。」又奏:「文武三品官薨卒輟朝。有未經親重之官,今任又是散列者,為之變禮,誠恐非宜。自今後,文武三品以上官,非曾任將相,及曾在密近,宜加恩禮者,餘請不在輟朝之限。」從之。



 累轉考功郎中、史館修撰。九年,轉司勛郎中、知制誥。十二月,正拜中書舍人。開成初,出為華州刺史。三年三月,人為戶部侍郎,判本司事。四年,權判吏部尚書銓事。大
 中四年,為中書侍郎、同平章事,兼吏部尚書。五年七月,撰成《續唐歷》三十卷,上之。六年,罷相,檢校吏部尚書,汴州刺史、宣武軍節度觀察等使,累歷方鎮卒。



 鄭肅,滎陽人。祖烈,父閱,世儒家。肅苦心力學。元和三年,擢進士第,又以書判拔萃,歷佐使府。太和初,入朝為尚書郎。六年,轉太常少卿。肅能為古文,長於經學,左丘明、《三禮》、儀注疑議,博士以下必就肅決之。



 時魯王永有寵,文宗擇名儒為其府屬,用戶部侍郎庾敬休兼王傅,戶
 部郎中李踐方兼司馬,以肅本官兼長史,由是知名。明年,魯王為太子,肅加給事中。九年,改刑部侍郎,尋改尚書右丞,權判吏部西銓事。開成初,出為陜虢都防禦觀察使、兼御史大夫。二年九月,召拜吏部侍郎。帝以肅嘗侍太子,言論典正,復令兼太子賓客,為東宮授經。既而太子失寵,上不悅,有廢斥意。肅因召見,深陳邦國大本、君臣父子之義。上改容嘉之。而太子竟以楊妃故得罪。乃以肅檢校禮部尚書,兼河中尹、河中節度、晉絳觀察
 等使。會昌初,武宗思太子永之無罪,盡誅陷永之黨。朝議稱肅忠正,有大臣之節。召拜太常卿,累遷戶部、兵部尚書。



 五年,以本官同平章事,加中書、門下二侍郎,監修國史,兼尚書右僕射。素與李德裕親厚。宣宗即位,德裕罷知政事,肅亦罷相,復為河中節度使。以疾辭,拜太子太保,卒。



 子洎,咸通中累官尚書郎,出為刺史。洎子仁規、仁表,俱有俊才,文翰高逸。



 仁規累遷拾遣、補闕、尚書郎、湖州刺史、尚書郎知制誥,正拜中書舍人,卒。



 仁表擢第
 後,從杜審權、趙騭為華州、河中掌書記,入為起居郎。仁表文章尤稱俊拔,然恃才傲物,人士薄之。自謂門地、人物、文章具美,嘗曰:「天瑞有五色雲,人瑞有鄭仁表。」劉鄴少時,投文於洎,仁表兄弟嗤鄙之。咸通末,鄴為宰相,仁表竟貶死南荒。



 盧商,字為臣,範陽人。祖昂,灃州刺史。父廣,河南縣尉。商,元和四年擢進士第,又書判拔萃登科。少孤貧力學,釋褐秘書省校書郎。範傅式廉察宣歙,闢為從事。王播、段
 文昌相繼鎮西蜀,商皆佐職為記室,累改禮部員外郎。入朝為工部員外郎、河南縣令,歷工部、度支、司封三郎中。太和九年,改京兆少尹,權大理卿事。



 開成初,出為蘇州刺史。中謝日,賜金紫之服。



 初,郡人苦鹽法太煩,奸吏侵漁。商至,籍見戶,量所要自售,無定額。蘇人便之,歲課增倍。宰相領鹽鐵,以其績上,遷潤州刺史、浙西團練觀察使。入為刑部侍郎,轉京兆尹。三年,朝廷用兵上黨,飛挽越太行者,環地六七鎮,以商為戶部侍郎,判度支,兼
 供軍使,軍用無闕。逆稹蕩平,加檢校禮部尚書、梓州刺史、劍南東川節度使。



 宣宗即位,入為兵部侍郎。尋以本官同平章事、範陽郡開國公,食邑二千戶,加兼工部尚書。數年,檢校工部尚書,出為鄂岳觀察使,就加檢校兵部尚書。大中十三年,以疾求代,徵拜戶部尚書。其年八月,卒於漢陰驛,時年七十一。



 子知遠、知微、知宗、僧朗、蕘。



 史臣曰:宗閔、嗣復,承宗室世家之地胄,有文學政事之美名,徊翔清華,出入隆顯。茍能義以為上,群而不黨,議
 太平於稷、契之列,致人主於勛、華之盛,遭時得位,誰曰不然?而舍披鴻猷,狎茲鼠輩,養虞卿而射利,抗德裕以報仇。矛盾相攻,幾傾王室,沒身蠻瘴,其利伊何?古者,廉、藺解仇,冀全國體,而邀歡釋憾,實亂大倫。世道銷刓,一至於此!崔、魏二丞相,嘉言啟奏,無忝正人。墀、讓史才,肅之禮學,商之長者,或登三事,或踐六卿,以道始終,夫何不韙。



 贊曰:漢誅鉤黨,魏破疽囊。何鄧之後,二李三楊。偷權報
 怨,任國存亡。書茲覆轍,敢告巖廊!



\end{pinyinscope}