\article{卷六十二 列傳第十二 李綱(子少植 少植子安仁) 鄭善果(從兄元璹) 楊恭仁(子思訓 思訓孫睿交 恭仁從孫執柔 恭仁少弟師道) 皇甫無逸(孫忠) 李大亮(族孫迥秀)}

\begin{pinyinscope}

 ○李綱子少植少植子安仁鄭善果從兄元璹楊恭仁子思訓思訓孫睿交恭仁從孫執柔恭仁少弟師道皇甫無逸孫忠李大亮族孫迥秀



 李綱,字文紀,觀州蓚人也。祖元則,後魏清河太守。父制,
 周車騎大將軍。綱少慷慨有志節,每以忠義自許。初名瑗,字子玉,讀《後漢書·張綱傳》,慕而改之。周齊王憲引為參軍。宣帝將害憲,召僚屬證成其罪,綱誓之以死,終無撓辭。及憲遇害,露車載尸而出,故吏皆散,唯綱撫棺號慟,躬自埋瘞,哭拜而去。



 隋開皇末,為太子洗馬。皇太子勇嘗以歲首宴宮臣,左庶子唐令則自請奏琵琶,又歌《武媚娘》之曲。綱自勇曰:「令則身任宮卿,職當調護,乃於宴座自比倡優,進淫聲,穢視聽。事若上聞,令則罪不測,
 豈不累於殿下?臣請遽正其罪。」勇曰:「我欲為樂耳,君勿多事。」綱趨而出。及勇廢黜,文帝召東宮官屬切讓之,無敢對者。綱對曰:「今日之事,乃陛下之過,非太子罪也。勇器非上品,性是常人,若得賢明之士輔導之,足堪繼嗣皇業。方今多士盈朝,當擇賢者居其任,奈何以弦歌鷹犬之才侍側,至令致此?乃陛下訓導不足,豈太子之罪耶!」辭氣凜然,左右皆為之失色。文帝曰:「令汝在彼,豈非擇人?」綱曰:「臣在東宮,非得言者。」帝奇其對,擢拜尚書右
 丞。時左僕射楊素、蘇威當朝用事,綱每固執所見,不與之同,由是二人深惡之。會遣大將軍劉方誅討林邑,楊素言於文帝曰:「林邑多珍寶,自非正人不可委。」因言綱可任,文帝以為行軍司馬。劉方承素之意,屈辱綱,幾至於死。及軍還,久不得調。後拜齊王府司馬。未幾,蘇威復令綱詣南海應接林邑,久而不召。綱後自來奏事,威復言綱擅離所職,以之屬吏。綱見善卜者,令筮之,遇《鼎》,因謂綱曰:「公易姓之後,方可得志而為卿輔。宜早退;不然,
 有折足之敗也。」尋會赦免,屏居於鄠。



 大業末,賊帥何潘仁以綱為長史。義師至京城,綱來謁見。高祖大悅,授丞相府司錄,封新昌縣公,專掌選。高祖踐祚,拜禮部尚書,兼太子詹事,典選如故。



 先是,巢王元吉授並州總管,於是縱其左右攘奪百姓,宇文歆頻諫不納,乃上表曰:「王在州之日,多出微行,常共竇誕游獵,蹂踐穀稼,放縱親暱,公行攘奪,境內獸畜,取之殆盡。當衢而射,觀人避箭以為笑樂。分遣左右,戲為攻戰,至相擊刺疻傷致死。夜
 開府門,宣淫他室。百姓怨毒,各懷憤嘆。以此守城,安能自保!」元吉竟坐免。又諷父老詣闕請之,尋令復職。時劉武周率五千騎至黃蛇嶺,元吉遣車騎將軍張達以步卒百人先嘗之。達以步卒少,固請不行。元吉強遣之,至則盡沒於賊。達憤怒,因引武周攻陷榆次,進逼並州。元吉大懼,紿其司馬劉德威曰:「卿以老弱守城,吾以強兵出戰。」因夜出兵,攜其妻孥,棄軍奔還京師,並州遂陷。高祖怒甚,謂綱曰:「元吉幼小,未習時事,故遣竇誕、宇文歆
 輔之。強兵數萬,食支十年,起義興運之資,一朝而棄。宇文歆首畫此計,我當斬之。」綱曰:「賴歆令陛下不失愛子,臣以為有功。」高祖問其故,綱對曰:「罪由竇誕不能規諷,致令軍人怨憤。又齊王年少,肆行驕逸放縱,左右侵漁百姓,誕曾無諫止,乃隨順掩藏,以成其釁,此誕之罪。宇文歆論情則疏,向彼又淺,王之過失,悉以聞奏。且父子之際,人所難言,歆言之,豈非忠懇?今欲誅罪,不錄其心,臣愚竊以為過。」翌日,高祖召綱入,升御坐謂曰:「今我有
 公,遂使刑罰不濫。元吉自惡,結怨於人。歆既曾以表聞,誕亦焉能制禁?」



 時高祖拜舞人安叱奴為散騎常侍,綱上疏諫曰:「謹案《周禮》,均工、樂胥不得預於仕伍。雖復才如子野,妙等師襄,皆身終子繼,不易其業。故魏武使禰衡擊鼓,衡先解朝服,露體而擊之,云不敢以先王法服為伶人之衣。雖齊高緯封曹妙達為王,授安馬駒為開府,既招物議,大絜彞倫,有國有家者以為殷鑒。方今新定天下,開太平之基。起義功臣,行賞未遍;高才碩學,猶
 滯草萊。而先令舞胡,致位五品;鳴玉曳組,趨馳廊廟,顧非創業垂統貽厥子孫之道也。」高祖不納。尋令參詳律令。



 綱在東宮,隱太子建成初甚禮遇。建成常往溫湯,綱時以疾不從。有進生魚於建成者,將召饔人作鱠。時唐儉、趙元楷在座,各自贊能為鱠,建成從之,既而謂曰:「飛刀鱠鯉,調和鼎食,公實有之;至於審諭弼諧,固屬於李綱矣。」於是遣使送絹二百匹以遺之。建成後漸狎無行之徒,有猜忌之謀,不可諫止。又思筮者之言,頻乞骸骨。
 高祖漫罵之曰:「卿為潘仁長史,何乃羞為朕尚書?且建成在東宮,遣卿輔導,何為屢致辭乎?」綱頓首陳謝曰:「潘仁,賊也,誠在殺害,每諫便止,所活極多,為其長史,故得無愧。陛下功成業泰,頗自矜伐,臣以凡劣,才乖元凱,所言如水投石,安敢久為尚書?兼以愚臣事太子,所懷鄙見,復不探納,既無補益,所以請退。」高祖謝曰:「知公直士,勉弼我兒。」於是擢拜太子少保,尚書、詹事並如故。綱又上書諫太子曰:「綱耄矣,日過時流,墳樹已拱,幸未就土,
 許傅聖躬,無以酬恩,請效愚直,伏願殿下詳之。竊見飲酒過多,誠非養生之術。且凡為人子者,務於孝友,以慰君父之心,不宜聽受邪言,妄生猜忌。」建成覽書不懌,而所為如故。綱以數言事忤太子旨,道既不行,鬱鬱不得志。武德二年,以老表辭職,優詔解尚書,仍為太子少保。高祖以綱隋代名臣,甚加優禮,每手敕未嘗稱名,其見重如此。



 貞觀四年,拜太子少師。時綱有腳疾,不堪踐履,太宗特賜步輿,令綱乘至閣下,數引入禁中,問以政道。
 又令輿入東宮,皇太子引上殿,親拜之。綱於是陳君臣父子之道、問寢視膳之方,理順辭直,聽者忘倦。太子每親政事,太宗必令綱及左僕射房玄齡、侍中王珪侍坐。太子嘗商略古來君臣名教竭忠盡節之事,綱凜然曰:「托六尺之孤,寄百里之命,古人以為難,綱以為易。」每吐論發言,皆辭色慷慨,有不可奪之志。及遇疾,太宗遣尚書左僕射房玄齡詣宅存問,賜絹二百匹。五年卒,年八十五。贈開府儀同三司,謚曰貞。太子為之立碑。初,周齊
 王憲女孀居孑立,綱自以齊王故吏,贍恤甚厚。及綱卒,其女被發號哭,如喪所生焉。



 子少植,隋武陽郡同功書佐,先綱卒。少植子安仁,永徽中為太子左庶子。屬太子被廢,歸於陳邸,宮僚皆逃散,無敢辭送者,安仁獨涕泣拜辭而去,朝野義之。後卒於恆州刺史。



 鄭善果,鄭州滎澤人也。祖孝穆,西魏少司空、岐州刺史。父誠,周大將軍、開封縣公。大象初,討尉遲迥,力戰遇害。善果年九歲,以父死王事,詔令襲其官爵。家人以其嬰
 孺,弗之告也,受冊悲慟,擗踴不能勝,觀者莫不為之流涕。隋開皇初,改封武德郡公,拜沂州刺史。大業中,累轉魯郡太守。善果篤慎,事親至孝。母崔氏,賢明曉於政道,每善果理務,崔氏嘗於閣內聽之。聞其剖斷合理,歸則大悅;若處事不允,母則不與之言,善果伏於床前,終日不敢食。崔氏謂之曰:「吾非怒汝,反愧汝家耳。汝先君在官清恪,未嘗問私,以身徇國,繼之以死。吾亦望汝繼父之心。自童子承襲茅土,今位至方伯,豈汝身能致之耶?
 安可不思此事而妄加嗔怒?內則墜爾家風,或亡官爵;外則虧天子之法,以取罪戾。吾寡婦也,有慈無威,使汝不知教訓,以負清忠之業,吾死之日,亦何面以事汝先君乎!」善果由此遂勵己為清吏,所在有政績,百姓懷之。及朝京師,煬帝以其居官儉約,蒞政嚴明,與武威太守樊子蓋者為天下第一,各賞物千段,黃金百兩,再遷大理卿。後突厥圍煬帝於雁門,以守御功,拜右光祿大夫。從幸江都。宇文化及弒逆,署為民部尚書,隨化及至遼
 城。淮安王神通圍化及,善果為化及守禦督戰,為流矢所中。及神通退還,竇建德進軍克之。建德將王琮獲善果,誚之曰:「公隋室大臣也,自尊夫人亡後而清稱益衰,又忠臣子,奈何為弒君之賊殉命苦戰而傷痍若此?」善果深愧赧,欲自殺,偽中書令宋正本馳往救止之。建德又不為之禮,乃奔相州。淮安王神通送於京師,高祖遇之甚厚,拜太子左庶子,檢校內史侍郎,封滎陽郡公。善果在東宮,數進忠言,多所匡諫。未幾,檢校大理卿,兼民
 部尚書。正身奉法,甚有善績。制與裴寂等十人,每奏事及侍立,並令升殿,與從兄元璹在其數,時以為榮。尋坐事免。及山東平,持節為招撫大使,坐選舉不平除名。後歷禮部、刑部二尚書。貞觀元年,出為岐州刺史,復以公事免。三年,起為江州刺史,卒。



 元璹,隋岐州刺史、沛國公譯子也。少以父功拜儀同大將軍,襲爵沛國公,累轉右武候將軍,改封莘國公。大業中,出為文城郡守。義師至河東,元璹以郡來降,徵拜太常卿。及定京城,以本官兼
 參旗將軍。元璹少在戎旅,尤明軍法,高祖常令巡諸軍,教其兵事。突厥始畢可汗弟乙力設代其兄為叱羅可汗,又劉武周將宋金剛與叱羅共為掎角,來寇汾、晉。詔元璹入蕃,諭以禍福,叱羅竟不納,乃欲總其部落,入寇太原,以為武周聲援。未幾,叱羅遇疾,療之弗愈,其下疑元璹令人毒之,乃囚執元璹,不得歸,叱羅竟死。頡利嗣立,留元璹,每隨其牙帳,經數年。頡利後聞高祖遺其財物,又許結婚,始放元璹來還。高祖勞之曰:「卿在虜庭,累
 載拘系,蘇武弗之過也。」拜鴻臚卿。尋而突厥又寇並州,時元璹在母喪,高祖令墨絰充使招慰。突厥從介休至晉州,數百里間,數騎數十萬,填映山谷。及見元璹,責中國違背之事,元璹隨機應對,竟無所屈,因數突厥背誕之罪,突厥大慚,不能報。元璹又謂頡利曰:「漢與突厥,風俗各異,漢得突厥,既不能臣,突厥得漢,復何所用?且抄掠資財,皆入將士,在於可汗,一無所得。不如早收兵馬,遣使和好,國家必有重賚,幣帛皆入可汗,免為劬勞,坐
 受利益。大唐初有天下,即與可汗結為兄弟,行人往來,音問不絕。今乃舍善取怨,違多就少,何也?」頡利納其言,即引還。太宗致書慰之曰:「知公已共可汗結和,遂使邊亭息警,爟火不然。和戎之功,豈唯魏絳,金石之錫,固當非遠。」元璹自義寧已來,五入蕃充使,幾至於死者數矣。貞觀三年,又使入突厥,還奏曰:「突厥興亡,唯以羊馬為準。今六畜疲羸,人皆菜色,又其牙內炊飯,化而為血。徵祥如此,不出三年,必當覆滅。」太宗然之。無幾,突厥果敗。
 元璹後累轉左武候大將軍,坐事免。尋起為宜州刺史,復封沛國公。元璹有幹略,所在頗著聲譽。然其父譯事繼母失溫凊之禮,隋文帝曾賜以《孝經》;至元璹事親,又不以孝聞,清論鄙之。二十年卒,贈幽州刺史,謚曰簡。



 弟孫杲知名,則天時為天官侍郎。



 楊恭仁,本名綸,弘農華陰人,隋司空、觀王雄之長子也。隋仁壽中,累除甘州刺史。恭仁務舉大綱,不為苛察,戎夏安之。文帝謂雄曰:「恭仁在州,甚有善政,非唯朕舉得
 人,亦是卿義方所致也。」大業初,轉吏部侍郎。楊玄感作亂,煬帝制恭仁率兵經略,與玄感戰於破陵,大敗之。玄感兄弟挺身遁走,恭仁與屈突通等追討獲之。軍旋,煬帝召入內殿,謂曰:「我聞破陵之陣,唯卿力戰,功最難比。雖知卿奉法清慎,都不知勇決如此也。」納言蘇威曰:「仁者必有勇,固非虛也。」時蘇威及左衛大將軍宇文述、御史大夫裴蘊、黃門侍郎裴矩等皆受詔參掌選事,多納賄賂,士流嗟怨。恭仁獨雅正自守,不為蘊等所容,由是
 出為河南道大使,討捕盜賊。時天下大亂,行至譙郡,為硃粲所敗,奔還江都。宇文化及弒逆,署吏部尚書,隨至河北,為化及守魏縣。時元寶藏據有魏郡,會行人魏徵說下寶藏,執恭仁送於京師。高祖甚禮遇之,拜黃門侍郎,封觀國公。尋為涼州總管。恭仁素習邊事,深悉羌胡情偽,推心馭下,人吏悅服,自蔥嶺已東,並入朝貢。未幾,遙授納言,總管如故。俄而突厥頡利可汗率眾數萬奄至州境,恭仁隨方備御,多設疑兵,頡利懼而退走。屬瓜
 州刺史賀拔威擁兵作亂,朝廷憚遠,未遑征討。恭仁乃募驍勇,倍道兼進,賊不虞兵至之速,克其二城。恭仁悉放俘虜,賊眾感其寬惠,遂相率執威而降。久之,徵拜吏部尚書,遷左衛大將軍、鼓旗將軍。貞觀初,拜雍州牧,加左光祿大夫,行揚州大都督府長史。五年,遷洛州都督。太宗曰:「洛陽要重,古難其人。朕之子弟多矣,恐非所任,特以委公也。」恭仁性虛澹,必以禮度自居,謙恭下士,未嘗忤物,時人方之石慶。恭仁弟師道,尚桂陽公主,從侄
 女為巢剌王妃,弟子思敬,尚安平公主,連姻帝室,益見崇重。後以老病乞骸骨,聽以特進歸第。十三年卒,冊贈開府儀同三司、潭州都督,陪葬昭陵,謚曰孝。



 子思訓襲爵。顯慶中,歷右屯衛將軍。時右衛大將軍慕容寶節有愛妾,置於別宅,嘗邀思訓就之宴樂。思訓深責寶節與其妻隔絕,妾等怒,密以毒藥置酒,思訓飲盡便死。寶節坐是配嶺表。思訓妻又詣闕稱冤,制遣使就斬之。仍改《賊盜律》,以毒藥殺人之科,更從重法。



 思訓孫睿交,本
 名璬,少襲爵觀國公,尚中宗女長寧公主。預誅張易之有功,賜實封五百戶。神龍中,為秘書監。後被貶,卒於絳州別駕。



 恭仁弟續,頗有辭學。貞觀中,為鄆州刺史。續孫執柔,則天時為地官尚書,則天以外氏近屬,甚優寵之。時武承嗣、攸寧相次知政事,則天嘗曰:「我今當宗及外家,常一人為宰相。」由是執柔同中書門下三品,尋卒。執柔子滔,開元中官至吏部侍郎、同州刺史。執柔弟執一,神龍初,以誅張易之功封河東郡公,累至右金吾衛大
 將軍。



 恭仁少弟師道,隋末自洛陽歸國,授上儀同,為備身左右。尋尚桂陽公主,超拜吏部侍郎,累轉太常卿,封安德郡公。貞觀七年,代魏徵為侍中。性周慎謹密,未嘗漏洩內事,親友或問禁中之言,乃更對以他語。嘗曰:「吾少窺漢史,至孔光不言溫室之樹,每欽其餘風,所庶幾也。」師道退朝後,必引當時英俊,宴集園池,而文會之盛,當時莫比。雅善篇什,又工草隸,酣賞之際,援筆直書,有如宿構。太宗每見師道所制,必吟諷嗟賞之。十三年,轉
 中書令。太子承乾逆謀事洩,與長孫無忌、房玄齡同按其獄。師道妻前夫之子趙節與承乾通謀,師道微諷太宗,冀活之,由是獲譴,罷知機密。轉吏部尚書。師道貴家子,四海人物,未能委練,所署用多非其才,而深抑貴勢及其親黨,以避嫌疑,時論譏之。太宗嘗從容謂侍臣曰:「楊師道性行純善,自無愆過。而情實怯懦,未甚更事,緩急不可得力。」未幾,從征高麗,攝中書令。及軍還,有毀之者,稍貶為工部尚書,尋轉太常卿。二十一年卒,贈吏部
 尚書、並州都督,陪葬昭陵,賜東園秘器,並為立碑。子豫之,尚巢剌王女壽春縣主。居母喪,與永嘉公主淫亂,為主婿竇奉節所擒,具五刑而殺之。師道兄子思玄,高宗時為吏部侍郎、國子祭酒。玄弟思敬,禮部尚書。師道從兄子崇敬,太子詹事。



 始恭仁父雄在隋,以同姓寵貴,自武德之後,恭仁兄弟名位尤盛,則天時,又以外戚崇寵。一家之內,駙馬三人,王妃五人,贈皇后一人,三品已上官二十餘人,遂為盛族。



 皇甫無逸,字仁儉,安定烏氏人。父誕,隋並州總管府司馬。其先安定著姓,徙居京兆萬年。仁壽末,漢王諒於並州起兵反,誕抗節不從,為諒所殺。無逸時在長安,聞諒反,即同居喪之禮。人問其故,泣而對曰:「大人平生徇節義,既屬亂常,必無茍免。」尋而兇問果至。在喪柴毀過禮,事母以孝聞。煬帝以誕死節,贈柱國、弘義郡公,令無逸襲爵。時五等皆廢,以其時忠義之後,特封平輿侯。拜涓陽太守,甚有能名,差品為天下第一。再轉右武衛將軍,
 甚見親委。帝幸江都,以無逸留守洛陽。及江都之變,與段達、元文都尊立越王侗為帝。王世充作難,無逸棄老母妻子,斬關而走,追騎且至,無逸曰:「吾死而後已,終不能同爾為逆。」因解所服金帶投之於地,曰:「以此贈卿,無為相迫。」追騎競下馬取帶,自相爭奪,由是得免。高祖以隋代舊臣,甚尊禮之,拜刑部尚書,封滑國公,歷陜東道行臺民部尚書。明年,遷御史大夫。時益部新開,刑政未洽,長吏橫恣,贓污狼藉;令無逸持節巡撫之,承制除授。
 無逸宣揚朝化,法令嚴肅,蜀中甚賴之。有皇甫希仁者,見無逸專制方面,徼幸上變,云:「臣父在洛陽,無逸為母之故,陰遣臣與王世充相知。」高祖審其詐,數之曰:「無逸逼於世充,棄母歸朕。今之委任,異於眾人。其在益州,極為清正。此蓋群小不耐,欲誣之也。此乃離間我君臣,惑亂我視聽。」於是斬希仁於順天門,遣給事中李公昌馳往慰諭之。俄而又告無逸陰與蕭銑交通者,無逸時與益州行臺僕射竇璡不協,於是上表自理,又言璡罪狀。
 高祖覽之曰:「無逸當官執法,無所回避,必是邪佞之徒,惡直醜正,共相構扇也。」因令劉世龍、溫彥博將按其事,卒無驗而止,所告者坐斬,竇璡亦以罪黜。無逸既返命,高祖勞之曰:「公立身行己,朕之所悉。比多譖訴者,但為正直致邪佞所憎耳。」尋拜民部尚書,累轉益州大都督府長史。閉門自守,不通賓客,左右不得出門。凡所貨易,皆往他州;每按部,樵採不犯於人。嘗夜宿人家,遇燈炷盡,主人將續之,無逸抽佩刀斷衣帶以為炷,其廉介如
 此。然過於審慎,所上表奏,懼有誤失,必讀之數十遍,仍令官屬再三披省;使者就路,又追而更審,每遣一使,輒連日不得上道。議者以此少之。母在長安疾篤,太宗令驛召之。無逸性至孝,承問惶懼,不能飲食,因道病卒。贈禮部尚書,太常考行,謚曰「孝」。禮部尚書王珪駁之曰:「無逸入蜀之初,自當扶侍老母,與之同去,申其色養,而乃留在京師,子道未足,何得為孝?」竟謚為良。孫忠,開元中為衛尉卿。



 李大亮,雍州涇陽人。後魏度支尚書琰之曾孫也。其先本居隴西狄道,代為著姓。祖綱,後魏南岐州刺史。父充節,隋朔州總管、武陽公。大亮少有文武才幹,隋末,署韓國公龐玉行軍兵曹。在東都與李密戰,敗,同輩百餘人皆就死,賊帥張弼見而異之,獨釋與語,遂定交於幕下。義兵入關,大亮自東都歸國,授土門令。屬百姓饑荒,盜賊侵寇,大亮賣所乘馬分給貧弱,勸以墾田,歲因大稔。躬捕寇盜,所擊輒平。時太宗在籓,巡撫北境,聞而嗟嘆,
 下書勞之,賜馬一匹、帛五十段。其後,胡賊寇境,大亮眾少不敵,遂單馬詣賊營,召其豪帥,諭以禍福,群胡感悟,相率請降。大亮又殺所乘馬,以與之宴樂,徒步而歸。前後降者千餘人,縣境以清。高祖大悅,超拜金州總管府司馬。時王世充遣其兄子弘烈據襄陽,令大亮安撫樊、鄧,以圖進取。大亮進兵擊之,所下十餘城。高祖下書勞勉,遷安州刺史。又令徇廣州巴東,行次九江,會輔公祏反,大亮以計擒公示石將張善安。公祏尋遣兵圍猷州,刺
 史左難當嬰城自守,大亮率兵進援,擊賊破之。以功賜奴婢百人,大亮謂曰:「汝輩多衣冠子女,破亡至此,吾亦何忍以汝為賤隸乎!」一皆放遣。高祖聞而嗟異,復賜婢二十人,拜越州都督。貞觀元年,轉交州都督,封武陽縣男。在越州寫書百卷,及徙職,皆委之廨宇。尋召拜太府卿,出為涼州都督,以惠政聞。嘗有臺使到州,見有名鷹,諷大亮獻之。大亮密表曰:「陛下久絕畋獵,而使者求鷹。若是陛下之意,深乖昔旨;如其自擅,便是使非其人。」太
 宗下之書曰:「以卿兼資文武,志懷貞確,故委籓牧,當茲重寄。比在州鎮,聲績遠彰,念此忠勤,無忘寤寐。使遣獻鷹,遂不曲順,論今引古,遠獻直言,披露腹心,非常懇到,覽用嘉嘆,不能便已。有臣若此,朕復何憂!宜守此誠,終始若一。古人稱一言之重,侔於千金,卿之此言,深足貴矣。今賜卿胡瓶一枚,雖無千鎰之重,是朕自用之物。」又賜荀悅《漢紀》一部,下書曰:「卿立志方直,竭節至公,處職當官,每副所委,方大任使,以申重寄。公事之閑,宜尋典
 籍。然此書敘致既明,論議深博,極為治之體,盡君臣之義,今以賜卿,宜加尋閱也。」時頡利可汗敗亡,北荒諸部相率內屬。有大度設、拓設、泥熟特勒及七姓種落等,尚散在伊吾,以大亮為西北道安撫大使以綏之,多所降附。朝廷愍其部眾凍餒,遣於磧石貯糧,特加賑給。大亮以為於事無益,上疏曰:



 臣聞欲綏遠者,必先安近。中國百姓,天下本根;四夷之人,猶於枝葉。擾於根本,以厚枝附,而求久安,未之有也。自古明王,化中國以信,馭夷狄
 以權。故《春秋》云:「戎狄豺狼,不可厭也;諸夏親暱,不可棄也。」自陛下君臨區宇,深根固本,人逸兵強,九州殷盛,四夷自服。今者招致突厥,雖入提封,臣愚稍覺勞費,未悟其有益也。然河西氓庶,積禦蕃夷,州縣蕭條,戶口鮮少,加因隋亂,減耗尤多。突厥未平之前,尚不安業;匈奴微弱已來,始就農畝。若即勞役,恐致妨損。以臣愚惑,請停招慰。且謂之荒服者,故臣而不內。是以周室愛人攘狄,竟延七百之齡;秦王輕戰事胡,四十載而遂絕。漢文
 養兵靜守,天下安豐;孝武揚威遠略,海內虛耗。雖悔輪臺,追已不及。至於隋室,早得伊吾,兼統鄯善,既得之後,勞費日甚,虛內致外,竟損無益。遠尋秦、漢,近觀隋室,動靜安危,昭然備矣。伊吾雖已臣附,遠在蕃磧,人非中夏,地多沙鹵。其自豎立稱籓附庸者,請羈縻受之,使居塞外,必畏威懷德,永為蕃臣,蓋行虛惠,而收實福矣。近日突厥傾國入朝,既不能俘之江淮,以變其俗;置於內地,去京不遠,雖則寬仁之義,亦非久安之計也。每見一人初
 降,賜物五匹、袍一領,酋帥悉授大官,祿厚位尊,理多縻費。以中國之幣帛,供積惡之兇虜,其眾益多,非中國之利也。



 太宗納其奏。八年,為劍南道巡省大使。大亮激濁揚清,甚獲當時之譽。及討吐谷渾,以大亮為河東道行軍總管。與大總管李靖等出北路,涉青海,歷河源,遇賊於蜀渾山,接戰破之,俘其名王,虜雜畜五萬計。以功進爵為公,賜物千段、奴婢一百五十人,悉遺親戚。仍罄其家資,收葬五葉宗族無後者三十餘喪,送終之禮,一時
 稱盛。後拜左衛大將軍。十七年,晉王為皇太子,東宮僚屬,皆盛選重臣。以大亮兼領太子右衛率,俄兼工部尚書,身居三職,宿衛兩宮,甚為親信。大亮每當宿直,必通宵假寐。太宗嘗勞之曰:「至公宿直,我便通夜安臥。」其見任如此。太宗每有巡幸,多令居守。房玄齡甚重之,每稱大亮有王陵、周勃之節,可以當大位。大亮雖位望通顯,而居處卑陋,衣服儉率。至性忠謹,雖妻子不見其惰容。事兄嫂有同於父母。每懷張弼之恩,而久不能得。弼時
 為將作丞,自匿不言。大亮嘗遇諸途而識之,持弼而泣,恨相得之晚。多推家產以遺弼,弼拒而不受。大亮言於太宗曰:「臣有今日之榮,張弼力也。有官爵請回。」太宗遂遷弼為中郎將,俄代州都督。時人皆賢大亮不背恩,而多弼不自伐也。十八年,太宗幸洛陽,令大亮副司空玄齡居中。尋遇疾,太宗親為調藥,馳驛賜之。臨終上表,請停遼東之役,又言京師宗廟所在,願深以關中為意。表成而嘆曰:「吾聞禮,男子不死婦人之手。」於是命屏婦人,
 言終而卒,時五十九。死之日,家無珠玉可以為唅,唯有米五石、布三十端。親戚孤遺為大亮所鞠養,服之如父者十五人。太宗為舉哀於別次,哭之甚慟,廢朝三日。贈兵部尚書、秦州都督,謚曰懿,陪葬昭陵。



 兄子道裕,永徽中為大理卿。



 迥秀,大亮族孫也。祖玄明,濟州刺史。父義本,宣州刺史。迥秀弱冠應英材傑出舉,拜相州參軍,累轉考功員外郎。則天雅愛其材,甚寵待之。掌舉數年,遷鳳閣舍人。迥秀母氏庶賤而色養過人,其妻崔氏嘗叱
 其媵婢,母聞之不悅,迥秀即時出之。或止云:「賢室雖不避嫌疑,然過非出狀,何遽如此?」迥秀曰:「娶妻本以承順顏色,顏色茍違,何敢留也?」竟不從。長安初,歷天官、夏官二侍郎,俄同鳳閣鸞臺平章事。則天令宮人參問其母,又嘗迎入宮中,待之甚優。迥秀雅有文才,飲酒斗餘,廣接賓朋,當時稱為風流之士。然頗托附權幸,傾心以事張易之、昌宗兄弟,由是深為讜正之士所譏。俄坐贓,出為廬州刺史。景龍中,累轉鴻臚卿、修文館學士,又持節
 為朔方道行軍大總管。所居宅中生芝草數莖,又有貓為犬所乳,中宗以為孝感所致,使旌其門閭。俄代姚崇為兵部尚書,病卒。子齊損,開元十年,與權梁山等構逆伏誅,籍沒其家也。



 史臣曰:孔子云,「邦有道,危言危行。」如李綱直道事人,執心不回。始對隋文,慷慨獲免;終忤楊素,屈辱尤深。及高祖臨朝,諫舞胡鳴玉,懷不吐不茹之節,存有始有卒之規,可謂危矣。非逢有道,焉能免諸?《易》曰,「王臣蹇蹇,匪躬
 之故」,李綱有焉。善果幼事賢母,長為正人。元璹於國有功,祗練邊事,承家不孝,終為匪人。恭仁仕隋忠厚,馭眾謙恭。破賊立功,方見仁者有勇;掌選被斥,所謂獨正者危。自偽歸朝,懷才遇主,連婚帝室,列位籓宣,始終無玷者鮮矣!師道慎密純善,怯懦無更事之名;抑勢避嫌,署用致非才之誚。無逸知父守節陷難,離母避逆終吉,忠信之道著矣;絕賓客以閉府門,斷衣帶以續燈炷,廉介之志彰矣。於乎,蜀道初開,親老地梗,至孝滅性,子道可
 知,不得謚為「孝」也,惜哉!大亮文武兼才,貞確成性。賣馬勸農,是為政也;投身諭賊,略也;放奴婢從良者,仁也;因鷹諫獵,臨終上表,忠也;論伊吾之眾,智也;葬五葉無後,報張弼恩,義也;侍兄嫂如父母,孝也;不死婦人之手,禮也;無珠玉為唅,廉也。房玄齡云,大亮有王陵、周勃之節,名下無虛士矣!迥秀諂事權幸,爰至臺司,餘不足觀,清風替矣。



 贊曰:李綱守道,言行俱危。善果母訓,清貞是資。元璹
 父子,要道何虧。恭仁獨正,令德無違。師道慎密,抑勢見機。無逸廉介,終於孝思。大亮才德,陵、勃名隨。迥秀托附,實污臺司。



\end{pinyinscope}