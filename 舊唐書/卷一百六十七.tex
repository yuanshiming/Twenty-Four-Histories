\article{卷一百六十七}

\begin{pinyinscope}

 ○孟
 簡胡證崔元略子鉉鉉子沆元略弟元受元式元儒杜元穎崔弘禮李虞仲王質盧簡辭兄簡能弟弘正簡求簡能子知猷簡求子嗣業汝弼



 孟簡,字幾道,平昌人。天后時同州刺史詵之孫。工詩有
 名。擢進士第,登宏辭科,累官至倉部員外郎。戶部侍郎王叔文竊政,簡為子司,多不附之;叔文惡之雖甚,亦不至擯斥。尋遷司封郎中。元和四年,超拜諫議大夫,知匭事。簡明於內典。六年,詔與給事中劉伯芻、工部侍郎歸登、右補闕蕭俛等,同就醴泉佛寺翻譯《大乘本生心地觀經》,簡最擅其理。



 王承宗叛,詔以吐突承璀為招討使。簡抗疏論之,坐語訐,出為常州刺史。八年,就加金紫光祿大夫。簡始到郡,開古孟瀆,長四十一里,灌溉沃壤四
 千餘頃,為廉使舉其課績,是有就加之命。是歲,徵拜為給事中。九年,出為越州刺史、兼御史中丞、浙東觀察使。承李遜抑遏士族、恣縱編戶之後,及簡為政,一皆反之,而農估多受其弊,當時以為兩未可也。十二年,入為戶部侍郎。十三年,代崔元略為御史中丞,仍兼戶部侍郎。是歲,出為襄州刺史、山南東道節度使。十四年,敕於穀城縣置群牧,命曰「臨漢監」,令簡充使。簡奏請均州鄖鄉縣鎮遏使趙潔充本縣令,臺司奏有虧刑典,罰一月俸。
 是歲,改授太子賓客,分司東都。十五年,穆宗即位,貶吉州司馬員外置同正員。初,簡在襄陽,以腹心吏陸翰知上都進奏,委以關通中貴。翰持簡陰事,漸不可制。簡怒,追至州,以土囊殺之,且欲滅口。翰子弟詣闕,進狀訴冤,且告簡贓狀。御史臺按驗,獲簡賂吐突承璀錢帛等共計七千餘貫匹,事狀明白,故再貶之。長慶元年大赦,量移睦州刺史。二年,移常州刺史。三年,入為太子賓客,分司東都。其年十二月卒。



 簡性俊拔尚義。早歲交友先歿
 者,視其孤,每厚於周恤,議者以為有前輩風。然溺於浮圖之教,為儒曹所誚。



 胡證,字啟中,河東人。父瑱,伯父玫,登進士第。證,貞元中繼登科,咸寧王渾瑊闢為河中從事。自殿中侍御史拜韶州刺史,以母年高不可適遠,改授太子舍人。襄陽節度使于頔請為掌書記,檢校祠部員外郎。



 元和四年,由侍御史歷左司員外郎、長安縣令、戶部郎中。田弘正以魏博內屬,請除副貳,乃兼御史中丞,充魏博節度副使,
 仍兼左庶子。入遷左諫議大夫。



 九年,以黨項寇邊,以證有安邊才略,乃授單于都護、御史大夫、振武軍節度使。前任將帥非統馭之才,邊事曠廢,朝廷故特用證以鎮。十三年,徵為金吾大將軍,依前兼御史大夫。十四年,充京西、京北巡邊使,訪其利害以聞。



 長慶元年,太和公主出降回紇,詔以本官檢校工部尚書充和親使。舊制,以使車出境,有行人私覿之禮,官不能給,召富家子納貲於使者而命之官。及證將行,首請厘革,儉受省費,以絕
 鬻官之門。行及漠南,虜騎繼至,狼心犬態,一日千狀,欲以戎服變革華服,又欲以王姬疾驅徑路。證抗志不拔,守漢儀,黜夷法,竟不辱君命。使還,拜工部侍郎。



 敬宗即位之初,檢校戶部尚書,守京兆尹。數月,遷左散騎常侍。寶歷初,拜戶部尚書、判度支,上表乞免,願效籓服。二年,檢校兵部尚書、廣州刺史,充嶺南節度使。太和二年,以疾上表求還京師。是歲十月卒於嶺南,時年七十一,廢朝一日,贈左僕射。



 廣州有海之利,貨貝狎至。證善蓄積,
 務華侈,厚自奉養,童奴數百,於京城修行里起第,連亙閭巷。嶺表奇貨,道途不絕,京邑推為富家。證素與賈餗善,及李訓事敗,禁軍利其財,稱證子溵匿餗,乃破其家。一日之內,家財並盡。軍人執溵入左軍,仇士良命斬之以徇。時溵弟湘為太原從事,忽白晝見綠衣人無首,血流被地,入於室,湘惡之。翌日,溵兇問至,而湘獲免。



 崔元略,博陵人。祖渾之。父儆,貞元中官至尚書左丞。元略舉進士,歷佐使府。元和八年,拜殿中侍御史。十二年,
 遷刑部郎中、知臺雜事,擢拜御史中丞。元和十三年,以李夷簡自西川征拜御史大夫,乃命元略留司東臺。尋除京兆少尹,知府事,仍加金紫。數月,真拜京兆尹。明年,改左散騎常侍。



 穆宗即位,命元略使黨項宣撫。辭疾不行,出為黔南觀察使、兼御史中丞。初,元略受命使黨項,意宰臣以私憾排斥,頗出怨言。宰相崔植奏曰:「比以聖意切在安撫黨項,乃差元略往使。受命之後,苦不樂行,言辭之間,頗乖去就。豈有身忝重恩,不思報效?茍非便
 己,即不肯行。須有薄懲,以肅在位,請出為黔中觀察使。」初,崔植任吏部郎中,元略任刑部郎中知雜。時中丞改京兆尹,物議以植有風憲之望。元略因入閣,妄稱植失儀,命御史彈之。時二人皆進擬為中丞,中旨果授元略,植深銜之。及植為相,元略以左散騎常侍使於黨項;元略意植之見排,辭疾不行。被譴出。逾年,轉鄂州刺史、鄂岳都團練觀察使。長慶四年,入為大理卿。



 敬宗即位,復為京兆尹,尋兼御史大夫。以誤征畿甸經赦免放緡錢
 萬七千貫,為侍御史蕭澈彈劾。有詔刑部郎中趙元亮、大理正元從質、侍御史溫造充三司覆理。元略有中助,止於削兼大夫。初,元略有宰相望,及是事,望益減。



 寶歷元年,遷戶部侍郎。議者以元略版圖之拜,出於宣授。時諫官有疏,指言內常侍崔潭峻方有權寵,元略以諸父事之,故雖被彈劾,而遽遷顯要。元略亦上章自辨,且曰:「一昨府縣條疏,臺司舉劾,孤立無黨,謗言益彰,不謂詔出宸衷,恩延望外。處南宮之重位,列左戶之清班,豈臣
 庸虛,敢自干冒。天心所擇,雖驚特進之恩;眾口相非,乃致因緣之說。」詔答之曰:「朕所命官,豈非公選?卿能稱職,奚恤人言!」然元略終不能逃父事潭峻之名。



 寶歷二年四月,京兆府以元略前任尹日為橋道使,造東渭橋時,被本典鄭位、判官鄭復虛長物價,抬估給用,不還人工價直,率斂工匠破用,計贓二萬一千七百九貫。敕云:「元略不能檢下,有涉慢官,罰一月俸料。」時劉棲楚自為京兆尹,有覬覦相位之意。元賂方在次對,又多游裴度門,
 棲楚恐礙己,以計摧之,乃按舉山陵時錢物以污之。



 太和三年,轉戶部尚書。四年,判度支。五年,檢校吏部尚書。出為東都留守、畿汝等防禦使。是歲,又遷滑州刺史、義成軍節度使。十二月卒,廢朝三日,贈尚書左僕射。子鉉。



 鉉,字臺碩,登進士第。三闢諸侯府,荊南、西蜀掌書記。會昌初,入為左拾遺,再遷員外郎,知制誥,召入翰林,充學士。累遷戶部侍郎承旨。會昌末,以本官同平章事。為同列李德裕所嫉,罷相,為陜虢觀察使、檢校刑部尚書。



 宣
 宗即位,遷檢校兵部尚書、河中尹、博陵縣開國子,食邑五百戶。大中三年,召拜御中大夫,尋加正議大夫、中書侍郎、同平章事。累遷金紫光祿大夫,守左僕射、門下侍郎、太清宮使、弘文館大學士、博陵縣開國公,食邑至二千戶。七年,以館中學士崔彖、薛逢等撰《續會要》四十卷,獻之。九年,檢校司徒、揚州大都督長史,進封魏國公、淮南節度使。宣宗於太液亭賦詩宴餞,有「七載秉鈞調四序」之句,儒者榮之。



 咸通初,移鎮襄州。咸通八年,徐州戍
 將龐勛自桂管擅還,道途剽掠。鉉時為荊南節度,聞徐州軍至湖南,盡率州兵,點募丁壯,分扼江、湘要害,欲盡擒之。徐寇聞之,逾嶺自江西、淮右北渡,朝議壯之。卒於江陵。



 子沆、汀、潭、沂。



 沆,登進士第,官至員外郎,知制誥,拜中書舍人。坐事貶循州司戶。乾符初,復拜舍人,尋遷禮部侍郎,典貢舉。選名士十數人,多至卿相。乾符末,本官同平章事。遇京國盜據,從駕不及而卒。沂後官亦隆顯。



 元略弟元受、元式、元儒。



 元受登進士第,高陵尉,直史館。
 元和初,於皋謨為河北行營糧料使。元受與韋岵、薛巽、王湘等皆為皋謨判官,分督供饋。既罷兵,或以皋謨隱沒贓罪,除名賜死。元受從坐,皆逐嶺表,竟坎壈不達而卒。子鈞、鉶、銖相繼登進士第,闢諸侯府。



 元式,會昌三年檢校左散騎常侍、河中尹、河中晉絳觀察使。四年,檢校禮部尚書、太原尹、北都留守、河東節度使。六年,入為刑部尚書。宣宗朝領度支,以本官同平章事。



 元儒,元和五年登進士第。



 元式子鍇,仕至京兆尹。



 杜元穎,萊公如晦裔孫也。父佐,官卑。元穎,貞元末進士登第,再闢使府。元和中為左拾遺、右補闕,召入翰林,充學士。手筆敏速,憲宗稱之。吳元濟平,以書詔之勤,賜緋魚袋。轉司勛員外郎,知制誥。穆宗即位,召對思政殿,賜金紫,超拜中書舍人。其年冬,拜戶部侍郎承旨。長慶元年三月,以本官同平章事,加上柱國、建安男。元穎自穆宗登極,自補闕至侍郎,不周歲居輔相之地。辭臣速達,未有如元穎之比也。



 三年冬,帶平章事出鎮蜀州,穆宗
 御安福門臨餞。昭愍即位,童心多僻,務為奢侈,而元穎求蜀中珍異玩好之具,貢奉相繼,以固恩寵。以故箕斂刻削,工作無虛日,軍民嗟怨,流聞於朝。太和三年,南詔蠻攻陷戎、巂等州,徑犯成都。兵及城下,一無備擬,方率左右固牙城而已。蠻兵大掠蜀城玉帛、子女、工巧之具而去。是時,蠻三道而來,東道攻梓州,郭釗御之而退。時元穎幾陷,賴郭釗擊敗其眾,方還。蠻驅蜀人至大渡河,謂之曰;「此南吾境,放爾哭別鄉國。」數萬士女,一時慟哭,
 風日為之慘淒。哭已,赴水而死者千餘。怨毒之聲,累年不息。蠻首領泬顛遣人上表曰:「蠻軍比修職貢,遽敢侵邊?但杜元穎不恤三軍,令入蠻疆作賊;移文報彼,都不見信。故蜀部軍人,繼為鄉導,蓋蜀人怨苦之深,祈我此行,誅虐帥也。誅之不遂,無以慰蜀士之心,願陛下誅之。」監軍小使張士謙至,備言元穎之咎。坐貶循州司馬,判官崔璜連州司馬,紇干臮郢州長史,盧並唐州司馬,皆以佐元穎無狀也。六年,卒於貶所。臨終,上表乞贈官,贈
 湖州刺史。



 元穎弟元絳,位終太子賓客。絳子審權,位至宰相,自有傳。



 崔弘禮,字從周,博陵人。北齊懷遠之七伐孫。祖育,常州江陰令。父孚,湖州長城令。弘禮風貌魁偉,磊落有大志。舉進士,累佐蕃府,官至侍御史。



 元和中,呂元膺為東都留守,以弘禮為從事。時淮西吳少陽初死,吳元濟阻兵拒命,山東反側之徒,為之影援;東結李師道,謀襲東洛,以脅朝廷。弘禮為元膺籌畫,部分兵眾,以固東都,卒亦
 無患。累除汾州、棣州刺史。會田弘正請入覲,請副使,乃授弘禮衛州刺史,充魏博節度副使,歷鄭州刺史。b####長慶元年,劉總入覲,張弘靖移鎮範陽,復加弘禮檢校左散騎常侍,充幽州盧龍軍節度副使。未及境,幽、鎮兵亂,改為絳州刺史。明年,汴州李絺反,急詔追弘禮為河南尹、兼御史大夫、東都畿汝都防禦副使。絺平,遷河陽節度使。整練戈矛,頗壯戎備。又上言請於秦渠下闢荒田三百頃,歲收粟二萬斛,詔皆從之。以疾連表請代。數歲,拜
 檢校戶部尚書、華州刺史。會天平軍節度使烏重胤卒,朝廷難其人,復以弘禮為天平軍節度使,仍詔即日乘遞赴鎮。



 文宗即位,就加檢校左僕射。理鄆三載,改授東都留守,仍遷刑部尚書。詔赴闕,以疾未至。太和四年十月,復除留守。是歲十二月卒,年六十四,贈司空。



 弘禮少時,專以倜儻意氣自任;通涉兵書,留心軍旋之要,用此累更選用,歷踐籓鎮。所居無可尚之績,雖繕完有素,然善治生蓄積,物議少之。



 李虞仲,字見之,趙郡人。祖震,大理丞。父端,登進士第,工詩。大歷中,與韓翃、錢起、盧綸等文詠唱和,馳名都下,號「大歷十才子」。時郭尚父少子曖尚代宗女升平公主,賢明有才思,尤喜詩人,而端等十人,多在曖之門下。每宴集賦詩,公主坐視簾中,詩之美者,賞百縑。曖因拜官,會十子曰:「詩先成者賞。」時端先獻,警句云:「薰香荀令偏憐小,傅粉何郎不解愁。」主即以百縑賞之。錢起曰:「李校書誠有才,此篇宿構也。願賦一韻正之,請以起姓為韻。」端
 即襞箋而獻曰:「方塘似鏡草芊芊,初月如鉤未上弦。新開金埒教調馬,舊賜銅山許鑄錢。」曖曰:「此愈工也。」起等始服。端自校書郎移疾江南,授杭州司馬而卒。



 虞仲亦工詩。元和初,登進士第,又以制策登科,授弘文校書。從事荊南,入為太常博士,遷兵部員外、司勛郎中。寶歷中,考制策甚精,轉兵部郎中,知制誥,拜中書舍人。太和四年,出為華州刺史、兼御史大夫。入拜左散騎常侍,兼秘書監。八年,轉尚書右丞。九年,為兵部侍郎,尋改吏部。開
 成元年四月卒,時年六十五。



 虞仲簡淡寡欲,立性方雅,奕代文學,達而不矜,士友重之。



 王質,字華卿,太原祁人。五代祖通,字仲淹,隋末大儒,號文中子。通生福祚,終上蔡主簿。福祚生勉,登進士第,制策登科,位終寶鼎令。勉生怡,終渝州司戶。怡生潛,揚州天長丞。質則潛之第五子。少負志操,以家世官卑,思立名於世,以大其門。寓居壽春,躬耕以養母,專以講學為事,門人受業者大集其門。年甫強仕,不求聞達,親友規
 之曰:「以華卿之才,取名位如俯拾地芥耳,安自苦於亹茸者乎?揚名顯親,非耕稼可致也。」質乃白於母,請赴鄉舉。元和六年,登進士甲科。釋褐嶺南管記,歷佐淮蔡、許昌、梓潼、興元四府,累奏兼監察御史。入朝為殿中,遷侍御史、戶部員外郎。為舊府延薦、檢校司封郎中,賜金紫,充興元節度副使。入為戶部郎中,遷諫議大夫。



 太和中,王守澄構陷宰相宋申錫。文宗怒,欲加極法。質與常侍崔玄亮雨泣切諫,請付外推,申錫方從輕典。質為中人
 側目,執政出為虢州刺史。質射策時,深為李吉甫所器;及德裕為相,甚禮之,事必咨決。尋召為給事中、河南尹。八年,為宣州刺史、兼御史中丞、宣歙團練觀察使。在政三年。開成元年十二月,無疾暴卒,時年六十八,贈左散騎常侍,謚曰定。



 質清廉方雅,為政有聲。雖權臣待之厚,而行己有素,不涉朋比之議。在宣城闢崔珦、劉濩、裴夷直、趙丱為從事,皆一代名流。視其所與,人士重之。子曰慶存。



 盧簡辭,字子策,範陽人,後徙家於蒲。祖翰。父綸,天寶末舉進士,遇亂不第,奉親避地於鄱陽,與郡人吉中孚為林泉之友。大歷初,還京師,宰相王縉奏為集賢學士、秘書省校書郎。王縉兄弟有詩名於世,縉既官重,凡所延闢,皆辭人名士,以綸能詩,禮待逾厚。會縉得罪,坐累。久之,調陜府戶曹、河南密縣令。建中初,為昭應令。硃泚之亂,咸寧王渾瑊充京城西面副元帥,乃拔綸為元帥判官、檢校金部郎中。貞元中,吉中孚為翰林學士、戶部侍
 郎,典邦賦,薦綸於朝。會丁家艱,而中孚卒。太府卿韋渠牟得幸於德宗,綸即渠牟之甥也,數稱綸之才。德宗召之內殿,令和禦制詩,超拜戶部郎中。方欲委之掌誥,居無何,卒。



 初,大歷中,詩人李端、錢起、韓翃輩能為五言詩;而辭情捷麗,綸作尤工。至貞元末,錢、李諸公凋落,綸嘗為《懷舊詩》五十韻,敘其事曰:「吾與吉侍郎中孚、司空郎中曙、苗員外發、崔補闕峒、耿拾遺湋、李校書端,風塵追游,向三十載。數公皆負當時盛稱榮耀,未幾,俱沉下泉。
 傷悼之際,常暢博士追感前事,賦詩五十韻見寄。輒有所酬,以申悲舊,兼寄夏侯審侍御。」其歷言諸子云:「侍郎文章宗,傑出淮楚靈。掌賦若吹籟,司言如建瓴。郎中善慶餘,雅韻與琴清。鬱鬱松帶雪,蕭蕭鴻入冥。員外貞貴儒,弱冠被華纓。月香飄桂實,乳溜瀝瓊英。補闕思沖融,巾拂藝亦精。彩蝶戲方圃,瑞雲滋翠屏。拾遺興難侔,逸調曠無程。九醖貯彌潔,三花寒轉馨。校書才智雄,舉世一娉婷。賭墅鬼神變,屬辭鸞鳳驚。差肩曳長裾,總轡奉
 和鈴。共賦瑤臺雪,同觀金谷笙。倚天方比劍,沉水忽如瓶。君持玉盤珠,寫我懷袖盈。讀罷涕交頤,願言躋百齡。」綸之才思,皆此類也。文宗好文,尤重綸詩,嘗問侍臣曰:「《盧綸集》幾卷?有子弟否?」李德裕對曰:「綸有四男,皆登進士第,今員外郎簡能、侍御史簡辭是也。」即遣中使詣其家,令進文集。簡能盡以所集五百篇上獻,優詔嘉之。



 簡辭,元和六年登第,三闢諸侯府。長慶末,入朝為監察,轉侍御史。文雅之餘,尤精法律,歷朝簿籍,靡不經懷。寶歷
 中,故京兆尹黎幹男煟詣臺治父葉縣舊業,臺司莫知本末。簡辭曰:「乾坐魚朝恩黨誅,田產籍沒。大歷已來,多少赦令,豈有雪朝恩、黎幹節文?況其田產分給百姓,將及百年,而煟恃中助而冒論耶!」乃移汝州刺史裴通,準大歷元年敕給百姓。又福建鹽鐵院官盧昂坐贓三十萬,簡辭按之,於其家得金床、瑟瑟枕大如斗。昭愍見之曰:「此宮中所無,而盧昂為吏可知也!」尋轉考功員外郎,轉郎中。太和中,坐事自太僕卿出為衢州刺史。會昌中,
 入為刑部侍郎,轉戶部。大中初,轉兵部侍郎、檢校工部尚書、許州刺史、御史大夫、忠武軍節度使,遷檢校刑部尚書、襄州刺史、山南東道節度使,卒。簡辭兄簡能。



 簡能,字子拙,登第後再闢籓府,入為監察御史。太和九年,由駕部員外檢校司封郎中,充鳳翔節度判官。時鄭注得幸,李訓與之謀誅宦官,俾注鎮鳳翔,仍妙選當時才俊以為賓佐。簡能與蕭俛弟傑、錢起子可復,皆為訓所選,從注。及訓敗,注誅。簡能、蕭傑等四人皆為監軍使所害。



 簡辭弟弘正、簡求。



 弘正,字子強,元和末登進士第,累闢使府掌書記。入朝為監察御史、侍御史。太和中,華州刺史宇文鼎、戶部員外盧允中坐贓,弘正按之。文宗怒,將殺鼎,弘正奏曰:「鼎歷持綱憲,繩糾之官,今為近輔刺史,以贓污聞,死固常典。但取受之首,罪在允中,監司之責,鼎當連坐。」文宗釋之,鼎方減等。三遷兵部郎中、給事中。



 會昌末,王師討劉稹。時詔河北三帥收山東州郡。俄而何弘敬、王元逵得邢、洺、磁三郡。宰臣奏議曰:「山東三郡,
 以賊稹未誅,宜且立留後。如弘敬、元逵有所陳請,則朝廷難以依違。」上曰:「然,誰可任者?」李德裕曰:「給事中盧弘正嘗為昭義判官,性又通敏,推擇攸宜。」即命為邢洺磁團練觀察留後。未行而稹誅,乃令弘正銜命宣諭河北三鎮。使還,拜工部侍郎。



 大中初,轉戶部侍郎,充鹽鐵轉運使。前是,安邑、解縣兩池鹽法積弊,課入不充。弘正令判官司空輿至池務檢察,特立新法,仍奏輿為兩池使。三年,課入加倍,其法至今賴之。檢校戶部尚書,出為徐
 州刺史、武寧軍節度使、徐泗濠觀察等使。徐方自智興之後,軍士驕怠,有銀刀都,尤勞姑息,前後屢逐主帥。弘正在鎮期年,皆去其首惡,喻之忠義。訖於受代,軍旋無譁。鎮徐四年,遷檢校兵部尚書、汴州刺史、宣武軍節度、宋亳潁觀察等使,卒於鎮。



 簡求,字子臧,長慶元年登進士第,釋褐江西王仲舒從事。又從元稹為浙東、江夏二府掌書記。裴度鎮襄陽,保厘洛都,皆闢為賓佐,奏殿中侍御史。入朝,拜監察。裴度鎮太原,復奏為記室。入為殿
 中,賜緋。牛僧孺鎮襄漢,闢為觀察判官。入為水部、戶部二員外郎。會昌末,討劉稹,詔以許帥李彥佐為招討使。朝廷以簡求累佐使府,達於機略,乃以簡求為忠武節度副使知節度事、本道供軍使。入為吏部員外,轉本司郎中,求為蘇州刺史。



 時簡辭鎮漢南,弘正為侍郎,領使務,昆仲皆居顯列,時人榮之。既而宰執不協,弘正出鎮,罷簡求為左庶子分司。數年,出為壽州刺史。九年,黨項叛,以簡求為四鎮北庭行軍、涇州刺史、涇原渭武節度
 押蕃落等使、檢校左散騎常侍、上柱國、範陽縣男、食邑三百戶。十一年,遷檢校工部尚書、定州刺史、御史大夫、義武軍節度、北平軍等使。十三年,檢校刑部尚書、鳳翔尹、鳳翔隴西節度觀察等使。十四年八月,代裴休為太原尹、北都留守,充河東節度觀察等使。



 簡求辭翰縱橫,長於應變,所歷四鎮,皆控邊陲。屬雜虜寇邊,因之移授,所至撫御,邊鄙晏然。太原軍素管退渾、契苾、沙陁三部落,或撫納不至,多為邊患。前政或要之詛盟,質之子弟,
 然為盜不息。簡求開懷撫待,接以恩信,所質子弟,一切遣之。故五部之人,欣然聽命。咸通初,以疾辭,表章瀝懇。制以太子太師致仕,還於東都。都城有園林別墅,歲時行樂,子弟侍側,公卿在席,詩酒賞詠,竟日忘歸,如是者累年。五年十月卒,時年七十六。贈尚書左僕射。



 簡能子知猷。知猷登進士第,釋褐秘書省正字。宰臣蕭鄴鎮江陵、成都,闢為兩府記室。入拜左拾遺,改右補闕、史館修撰,轉員外郎。出為饒州刺史。入拜兵部郎中,賜緋魚,改
 吏部郎中、太常少卿。出為商州刺史。徵拜給事中,轉中書舍人。僖宗幸山南,襄王偽署,乃避地金州。駕還,徵拜工部侍郎,轉戶部,判史館,遷尚書右丞、兵部侍郎。歷太常卿,工部、戶部尚書,復領太常卿。昭宗在華下,加檢校右僕射,守太子少師。進位太子太師,檢校司空,卒於華下。知猷器度長厚,文辭美麗。尤工書,落簡措翰,人爭模仿。子文度,位亦至丞郎。



 簡辭無子,以簡求子貽殷、玄禧入繼。貽殷終光祿少卿。玄禧登進士第,終國子博士。



 弘正
 子虔灌,有俊才,進士登第。所著文筆,為時所稱。位終秘書監。



 簡求十子,而嗣業、汝弼最知名。



 嗣業進士登第,累闢使府。廣明初,以長安尉直昭文館、左拾遺、右補闕。王鐸徵兵收兩京,闢為都統判官、檢校禮部郎中,卒。



 汝弼登進士第,累遷至祠部員外郎、知制誥,從昭宗遷洛。屬柳璨黨附賊臣,誣陷士族,汝弼懼,移疾退居,客游上黨。遇潞府為太原所攻,節度使丁會歸降,從會至太原,李克用奏為節度副使,累奏戶部侍郎。太原使府有龍
 泉亭,簡求節制時手書詩一章,在亭之西壁。汝弼復為亞帥,每亭中宴集,未嘗居賓位,西向俯首而已,人士嘉之。



 盧氏兩世貴盛,六卿方鎮相繼,而未有居輔相者。至中興,嗣業子文紀,仕至尚書中書侍郎、平章事。



 史臣曰:孟襄陽之清節,胡廣州之堅正,卒以結權幸而敗,積貨賄而亡。人如面焉,固難知也。二崔以綱憲相傾,元穎以獻奇取媚,雖遭時多僻,位至鼎司。言之正人,亦孔之醜,而父事宦者,何所逃譏?以端、綸之才,任不逾元
 士,而盧簡辭之昆仲,雲摶水擊,鬱為鼎門,非德積慶鐘,安能及此?辭人之後,不亦休哉!



 贊曰:君子喻義,小人近利。孟譴胡亡,家財掃地。聲勢相傾,崔、杜醜名。端綸諸子,奕葉光榮。



\end{pinyinscope}