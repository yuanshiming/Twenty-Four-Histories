\article{卷二百一十}

\begin{pinyinscope}

 ○泥婆
 羅黨項羌高昌吐谷渾焉耆龜茲疏勒于闐天竺罽賓康國婆斯
 拂菻大食



 泥婆羅國,在吐蕃西。其俗翦發與眉齊,穿耳,揎以竹桶牛角,綴至肩者以為姣麗。食用手,無匕箸。其器皆銅。多商賈,少田作。以銅為錢,面文為人,背文為馬牛,不穿孔。衣服以一幅蔽布身,日數盥浴。以板為屋,壁皆雕畫。俗重博戲,好吹蠡擊鼓。頗解推測盈虛,兼通歷術。事五天神,鐫石為像,每日清水浴神,烹羊而祭。其王那陵提婆,身著真珠、玻璃、車渠、珊瑚、琥珀、瓔珞,耳垂金鉤玉榼,佩
 寶裝伏突,坐獅子床,其堂內散花燃香。大臣及諸左右並坐於地,持兵數百列侍其側。宮中有七層之樓,覆以銅瓦,欄檻楹栿皆飾珠寶。樓之四角,各懸銅槽,下有金龍,激水上樓,注於槽中,從龍口而出,狀若飛泉。那陵提婆之父,為其叔父所篡,那陵提婆逃難於外,吐蕃因而納焉,克復其位,遂羈屬吐蕃。



 貞觀中,衛尉丞李義表往使天竺,塗經其國,那陵提婆見之,大喜,與義表同出觀阿耆婆沴池。周回二十餘步,水恆沸,雖流潦暴集,爍石焦
 金,未嘗增減。以物投之,即生煙焰,懸釜而炊,須臾而熟。其後王玄策為天竺所掠,泥婆羅發騎與吐蕃共破天竺有功。永徽二年,其王尸利那連陀羅又遣使朝貢。



 黨項羌,在古析支之地,漢西羌之別種也。魏、晉之後,西羌微弱,或臣中國,或竄山野。自周氏滅宕昌、鄧至之後,黨項始強。其界東至松州,西接葉護,南雜舂桑、迷桑等羌,北連吐谷渾,處山谷間,互三千里。其種每姓別自為部落,一姓之中復分為小部落,大者萬餘騎,小者數千
 騎,不相統一。有細封氏、費聽氏、往利氏、頗超氏、野辭氏、房當氏、米擒氏、拓拔氏,而拓拔最為強族。俗皆土著,居有棟宇,其屋織犛牛尾及羊毛覆之,每年一易。俗尚武,無法令賦役。其人多壽,年一百五六十歲。不事產業,好為盜竊,互相凌劫。尤重復仇,若仇人未得,必蓬頭垢面跣足蔬食,要斬仇人而後復常。男女並衣裘褐,仍被大氈。畜犛牛、馬、驢、羊,以供其食。不知稼穡,土無五穀。氣候多風寒,五月草始生,八月霜雪降。求大麥於他界,醖以
 為酒。妻其庶母及伯叔母、嫂、子弟之婦,淫穢烝褻,諸夷中最為甚,然不婚同姓。老死者以為盡天年,親戚不哭;少死者則云夭枉,乃悲哭之。死則焚尸,名為火葬。無文字,但候草木以記歲時。三年一相聚,殺牛羊以祭天。自周及隋,或叛或朝,常為邊患。



 貞觀三年,南會州都督鄭元璹遣使招諭,其酋長細封步賴舉部內附,太宗降璽書慰撫之。步賴因來朝,宴賜甚厚,列其地為軌州,拜步賴為刺史。仍請率所部討吐谷渾。其後諸姓酋長相次
 率部落皆來內屬。請同編戶,太宗厚加撫慰,列其地為崌、奉、巖、遠四州,各拜其首領為刺史。



 有羌酋拓拔赤辭者,初臣屬吐谷渾,甚為渾主伏允所暱,與之結婚。及貞觀初,諸羌歸附,而赤辭不至。李靖之擊吐谷渾,赤辭屯狼道坡以抗官軍。廓州刺史久且洛生遣使諭以禍福,赤辭曰:「我被渾主親戚之恩,腹心相寄,生死不貳,焉知其他。汝可速去,無令污我刀也。」洛生知其不悟,於是率輕騎襲之,擊破赤辭於肅遠山,斬首數百級,虜雜畜六
 千而還。太宗又令岷州都督李道彥說諭之,赤辭從子思頭密送誠款,其黨拓拔細豆又以所部來降。赤辭見其宗黨離,始有歸化之意。後岷州都督劉師立復遣人招誘,於是與思頭並率眾內屬,拜赤辭為西戎州都督,賜姓李氏。自此職貢不絕。其後吐蕃強盛,拓拔氏漸為所逼,遂請內徙,始移其部落於慶州,置靜邊等州以處之。其故地陷於吐蕃,其處者為其役屬,吐蕃謂之「弭藥」。



 又有黑黨項,在於赤水之西。李靖之擊吐谷渾也,渾主
 伏允奔黑黨項,居以空閑之地。及吐谷渾舉國內屬,黑黨項酋長號敦善王因貢方物。又有雪山黨項,姓破丑氏,居於雪山之下,及白狗、舂桑、白蘭等諸羌,自龍朔已後,並為吐蕃所破而臣屬焉。



 其在西北邊者,天授三年內附,凡二十萬口,分其地置朝、吳、浮、歸等十州,仍散居靈、夏等界內。自至德已後,常為吐蕃所誘,密以官告授之,使為偵道,故時或侵叛,尋亦底寧。寶應初,其首領來朝,請助國供靈州軍糧,優詔褒美。



 其在涇、隴州界者,上
 元元年率其眾十餘萬,詣鳳翔節度使崔光遠請降。寶應元年十二月,其歸順州部落、乾封州部落、歸義州部落、順化州部落、和寧州部落、和義州部落、保善州部落、寧定州部落、羅雲州部落、朝鳳州部落,並詣山南西道都防禦使、梁州刺史臧希讓請州印。希讓以聞,許之。



 貞元三年十二月,初禁商賈以牛、馬、器械於黨項部落貿易。十五年二月,六州黨項自石州奔過河西。黨項有六府部落,曰野利越詩、野利龍兒、野利厥律、兒黃、野海、野
 窣等。居慶州者號為東山部落,居夏州者號為平夏部落。永泰、大歷已後,居石州,依水草。至是永安城鎮將阿史那思昧擾其部落,求取駝馬無厭,中使又贊成其事,黨項不堪其弊,遂率部落奔過河。元和九年五月,復置宥州以護黨項。



 十五年十一月,命太子中允李寮為宣撫黨項使。以部落繁富,時遠近商賈,齎繒貨入貿羊馬。至太和、開成之際,其籓鎮統領無緒,恣其貪婪,不顧危亡,或強市其羊馬,不酬其直,以是部落苦之,遂相率為
 盜,靈、鹽之路小梗。會昌初,上頻命使安撫之,兼命憲臣為使,分三印以統之。在邠、寧、延者,以侍御史、內供奉崔君會主之;在鹽、夏、長、澤者,以侍御史、內供奉李鄠主之;在靈、武、麟、勝者,以侍御史、內供奉鄭賀主之,仍各賜緋魚以重其事。久而無狀,尋皆罷之。



 高昌者,漢車師前王之庭,後漢戊己校尉之故地。在京師西四千三百里。其國有二十一城,王都高昌。其交河城,前王庭也;田地城,校尉城也。勝兵且萬人。厥土良沃,
 穀麥歲再熟;有蒲萄酒,宜五果;有草名白疊,國人採其花,織以為布。有文字,知書計,所置官亦採中國之號焉。其王麴伯雅,即後魏時高昌王嘉之六世孫也。隋煬帝時入朝,拜左光祿大夫、車師太守、封弁國公,仍以戚屬宇文氏女為華容公主以妻之。



 武德二年,伯雅死,子文泰嗣,遣使來告哀,高祖遣前河州刺史硃惠表往吊之。七年,文泰又獻狗雄雌各一,高六寸,長尺餘,性甚慧,能曳馬銜燭,云本出拂菻國。中國有拂菻狗,自此始也。太
 宗嗣位,復貢玄狐裘,因賜其妻宇文氏花鈿一具。宇文氏復貢玉盤。西域諸國所有動靜,輒以奏聞。貞觀四年冬,文泰來朝,及將歸蕃,賜遺甚厚。其妻宇文氏請預宗親,詔賜李氏,封常樂公主,下詔慰諭之。



 時西戎諸國來朝貢者,皆塗經高昌,文泰後稍壅絕之。伊吾先臣西突厥,至是內屬,文泰又與葉護連結,將擊伊吾。太宗以其反覆,下書切讓,徵其大臣冠軍阿史那矩入朝,將與議事。文泰竟不遣,乃遣其長史麴雍來謝罪。



 初,大業之亂,
 中國人多投於突厥。及頡利敗,或有奔高昌者,文泰皆拘留不遣。太宗詔令括送,文泰尚隱蔽之。又尋與西突厥乙毗設擊破焉耆三城,虜其男女而去。焉耆王上表訴之,太宗遣虞部郎中李道裕往問其狀。十三年,太宗謂其使曰:「高昌數年來朝貢脫略,無籓臣禮,國中署置官號,準我百僚,稱臣於人,豈得如此!今茲歲首,萬國來朝,而文泰不至。增城深塹,預備討伐。日者我使人至彼,文泰云:『鷹飛於天,雉竄於蒿,貓游於堂,鼠安於穴,各得
 其所,豈不活耶!』又西域使欲來者,文泰悉拘留之。又遣使謂薛延陀云:『既自為可汗,與漢天子敵也,何須拜謁其使。』事人闕禮,離間鄰好,惡而不誅,善者何勸?明年,當發兵馬以擊爾。」是時薛延陀可汗表請為軍向導,以擊高昌,太宗許之。令民部尚書唐儉至延陀,與謀進取。太宗冀其悔過,復下璽書,示以禍福,征之入朝。文泰稱疾不至。太宗乃命吏部尚書侯君集為交河道大總管,率左屯衛大將軍薛萬均及突厥、契、苾之眾,步騎數萬眾
 以擊之。時公卿近臣,皆以行經沙磧,萬里用兵,恐難得志;又界居絕域,縱得之,不可以守,競以為諫。太宗皆不聽。文泰謂所親曰:「吾往者朝覲,見秦、隴之北,城邑蕭條,非復有隋之比。設今伐我,發兵多則糧運不給;若發三萬以下,吾能制之。加以磧路艱險,自然疲頓,吾以逸待勞,坐收其弊,何足為憂也?」及聞王師臨磧口,惶駭計無所出,發病而死。



 其子智盛嗣立。既而君集兵奄至柳谷,進趨田地城,將軍契苾何力為前軍,與之接戰而退。大
 軍繼之,攻拔其城,虜男女七千餘口。進逼其都。智盛移君集書曰:「有罪於天子者,先王也,咎深譴積,身已喪亡。智盛襲位無幾,君其赦諸?」君集謂曰:「若能悔禍,當面縛軍門也。」又命諸軍引沖車、拋車以逼之,飛石雨下,城中大懼。智盛窮蹙,出城降。君集分兵掠地,下其三郡、五縣、二十二城。戶八千,口三萬七千七百,馬四千三百匹。其界東西八百里,南北五百里。先是,其國童謠云:「高昌兵馬如霜雪,漢家兵馬如日月。日月照霜雪,回手自消滅。」
 文泰使人捕其初唱者,不能得。



 初,文泰與西突厥欲谷設通和,遺其金帛,約有急相為表裏。及聞君集兵至,欲谷設懼而西走,不敢救。君集尋遣使告捷,太宗大悅,宴白僚,班賜各有差。曲赦高昌部內從軍兵士已上,父子犯死罪已下,期親犯流已下,大功犯徒已下,小功緦麻犯杖罪,悉原之。



 時太宗欲以高昌為州縣,特進魏徵諫曰:「陛下初臨天下,高昌夫婦先來朝謁。自後數月,商胡被其遏絕貢獻,加之不禮大國,遂使五誅載加。若罪止
 文泰,斯亦可矣,未若撫其人而立其子,所謂伐罪吊民,威德被於遐外,為國之善者也。今若利其土壤,以為州縣,常須千餘人鎮守,數年一易,每及交蕃,死者十有三四,遣辦衣資,離別親戚,十年之後,隴右空虛。陛下終不得高昌撮穀尺布以助中國,所謂散有用而事無用,臣未見其可。」太宗不從,竟以其地置西州,又置安西都護府,留兵以鎮之。初,西突厥遣其葉護,屯兵於可汗浮圖城,與高昌相影響,至是懼而來降,以其地為庭州。於是
 勒石紀功而旋。其智盛君臣及其豪右,皆徙中國。



 麴氏有國,至智盛凡九世,一百三十四年而滅。尋拜智盛為左武衛將軍,封金城郡公;弟智湛為右武衛中郎將,天山縣公。及太宗崩,刊石像智盛之形,列於昭陵玄闕之下。智湛,麟德中終於左驍衛大將軍、西州刺史。天授初,其子崇裕授左武衛大將軍,交阿郡王。卒,封襲遂絕。



 吐谷渾,其先居於徒河之清山,屬晉亂,始度隴,止於甘松之南,洮水之西,南極白蘭,地數千里。有城郭而不居,
 隨逐水草,廬帳為室,肉酪為糧。其官初有長史、司馬、將軍。近代已來,有王公、僕射、尚書、郎中。其俗頗識文字。男子通服長裙繒帽,或戴冪苾,婦人以金花為首飾,辮發縈後,綴以珠貝。其婚姻富家厚出聘財,貧人竊女而去。父卒,妻其庶母;兄亡,妻其諸嫂。喪有服制,葬訖而除。國無常稅,用度不給,輒斂富室商人,以取足而止。殺人及盜馬者罪死,他犯則徵物以贖罪。氣候多寒,土宜大麥、蔓菁,頗有菽粟。出良馬、犛牛、銅、鐵、硃砂之類。有青海,周回
 八百里,中有小山,至冬,放牝馬於其上,言得龍種。嘗得波斯馬,放入海,因生驄駒,能日行千里,故代稱「青海驄」焉。地兼鄯善、且沫。西北有流沙數百里,夏有熱風,傷弊行旅,風之將至,老駝便知之,則引項而鳴,以口鼻埋沙中。人以為候,即以氈擁蔽口鼻而避其患。



 隋煬帝時,其王伏允來犯塞,煬帝親總六軍以討之,伏允以數十騎潛於泥嶺而遁,其仙頭王率男女十餘萬口來降。煬帝立其質子順為王,送之本國,令統餘眾,尋復追還。大業
 末,伏允悉收故地,復為邊患。高祖受禪,順自江都來歸長安。時李軌猶據涼州,高祖遣使與伏允通和,令擊軌以自效,當放順返國。伏允大悅,興兵擊之,戰於庫門,交綏而退。頻遣使朝貢,以順為請,高祖乃遣之。



 太宗即位,伏允遣其洛陽公來朝。使未返,大掠鄯州而去。太宗遣使責讓之,徵伏允入朝,稱疾不至。仍為其子尊王求婚,於是責其親迎以羈縻之。尊王又稱疾不肯入朝,有詔停婚,遣中郎將康處直諭以禍福。伏允遣兵寇蘭、廓二
 州。



 時鄯州刺史李玄運上言:「吐谷渾良馬悉牧青海,輕兵掩之,可致大利。」於是遣左驍衛大將軍段志玄率邊兵及契苾、黨項之眾以擊之。去青海三十里,志玄與左驍衛將軍梁洛仁不欲戰,頓軍遲留不進,吐谷渾遂驅青海牧馬而遁。亞將李君羨率精騎別路,及賊於青海之南懸水鎮,擊破之,虜牛羊二萬餘頭而還。時伏允年老昏耄,其邪臣天柱王惑亂之,拘我行人鴻臚丞趙德楷。太宗頻遣宣諭,使者十餘返,竟無悛心。



 貞觀九年,詔
 特進李靖為西海道行軍大總管;兵部尚書侯君集為積石道行軍總管,任城王道宗為鄯州道行軍總管,仍為靖副;涼州都督李大亮為且沫道行軍總管,岷州都督李道彥為赤水道行軍總管,利州刺史高甑生為鹽澤道行軍總管,並突厥、契苾之眾以擊之。諸將頻與賊遇,連戰破之,獲其高昌王慕容孝雋。孝雋有雄略,伏允心膂之臣也。靖等進至赤海,遇其天柱三部落,擊大破之,遂歷於河源。李大亮又俘其名王二十人,雜畜數萬,
 至且沫西境,或傳伏允西走,渡圖倫磧,欲入于闐。將軍薛萬均率輕銳追奔,入磧數百里,及其餘黨,破之。磧中乏水,將士皆刺馬血而飲之。侯君集與江夏王道宗趣南路,登漢哭山,飲馬烏海,獲其名王梁屈忽,經塗二千餘里空虛之地,盛夏降霜,多積雪,其地乏水草,將士啖冰,馬皆食雪。又達於柏梁,北望積石山,觀河源之所出焉。兩軍會於大非川,至破邏貞谷,伏允子大寧王順窮蹙,斬其國相天柱王,舉國來降。伏允大懼,與千餘騎遁
 於磧中,眾稍亡散,能屬之者才百餘騎,乃自縊而死。國人乃立順為可汗,稱臣內附。



 順,即伏允之嫡子也。初為侍子於隋,拜金紫光祿大夫,久不得歸,伏允遂立他子為太子,及得返國,意常怏怏。會李靖等諸軍所向克捷,自以失位,欲因此立功,由是遂降。乃詔曰:



 吐谷渾擅相君長,竊據荒裔,志在兇德,政出權門。酋渠攜貳,種落怨憤,長惡不悛,野心彌熾。莫顧籓臣之禮,曾無事上之節,草竊疆場,虐割兆庶,積惡既稔,天亡有征。朕君臨四海,
 含育萬類,一物失所,責深在予。所以爰命六軍,申茲九伐,義存活國,情非黷武。其子大寧王慕容順,隋氏之甥,志懷明悟,長自中土,幸慕華風,爰見時機,深識逆順。以其愎諫違眾,獨陷迷途,遂誅邪臣,存茲大計。翻然改轍,代父歸罪,忠孝之美,深有可嘉。子能立功,足以補過,既往之釁,特宜原免。然其建國西鄙,已歷年代,即從廢絕,情所未忍,繼其宗祀,允歸命胤。可封順為西平郡王,仍授趉胡呂烏甘豆可汗。



 太宗恐順不能靜其國,仍遣李
 大亮率精兵數千,為其聲援。順既久質於隋,國人不附,未幾為臣下所殺。其子燕王諾曷缽嗣立。



 諾曷缽既幼,大臣爭權,國中大亂。太宗遣兵援之,封為河源郡王。仍授烏地也拔勒豆可汗,遣淮陽王道明持節冊拜,賜以鼓纛。諾曷缽因入朝請婚。十四年,太宗以弘化公主妻之,資送甚厚。十五年,諾曷缽所部丞相王專權,陰謀作難。將征兵,詐言祭山神,因欲襲擊公主,劫諾曷缽奔於吐蕃,期有日矣。諾曷缽知而大懼,率輕騎走鄯善城,其
 威信王以兵迎之。鄯州刺史杜鳳舉與威信王合軍擊丞相王,破之,殺其兄弟三人,遣使言狀。太宗命民部尚書唐儉持節撫慰之。太宗崩,刻石圖諾曷缽之形,列於昭陵之下。



 高宗嗣位,以其尚主,拜駙馬都尉,賜物四十段。其後與吐蕃互相攻伐,各遣使請兵救援,高宗皆不許之。吐蕃大怒,率兵以擊吐谷渾。諾曷缽既不能御,脫身及弘化公主走投涼州。高宗遣右威衛大將軍薛仁貴等救吐谷渾,為吐蕃所敗,於是吐谷渾遂為吐蕃所並。諾曷缽以親
 信數千帳來內屬,詔左武衛大將軍蘇定方為安置大使,始徙其部眾於靈州之地,置安樂州,以諾曷缽為刺史,欲其安而且樂也。



 垂拱四年,諾曷缽卒,子忠嗣。忠卒,子宣趙嗣。聖歷三年,授宣趙左豹韜衛員外大將軍,仍襲父烏地也拔勒豆可汗。宣趙卒,子曦皓嗣。曦皓卒,子兆嗣,及吐蕃陷我安樂州,其部眾又東徙,散在朔方、河東之境。今俗多謂之退渾,蓋語急而然。貞元十四年十二月,以朔方節度副使、左金吾衛大將軍同正慕容復為襲長樂州都督、
 青海國王、烏地也拔勒豆可汗。未幾,卒,其封襲遂絕。



 葉谷渾自晉永嘉之末,始西渡洮水,建國於群羌之故地,至龍朔三年為吐蕃所滅,凡三百五十年。



 焉耆國,在京師西四千三百里,東接高昌,西鄰龜茲,即漢時故地。其王姓龍氏,名突騎支。勝兵二千餘人,常役屬於西突厥。其地良沃,多蒲萄,頗有魚鹽之利。



 貞觀六年,突騎支遣使貢方物,復請開大磧路以便行李,太宗許之。自隋末罹亂,磧路遂閉,西域朝貢者皆由高昌。及
 是,高昌大怒,遂與焉耆結怨,遣兵襲焉耆,大掠而去。西突厥莫賀設與咄陸、弩失畢不協,奔於焉耆,咄陸復來攻之。



 六年,遣使言狀,並貢名馬。時西突厥國亂,太宗遣中郎將桑孝彥領左右胄曹韋弘機往安撫之,仍冊立咥利失可汗。可汗既立,素善焉耆,令與焉耆為援。十二年,處月、處密與高昌攻陷焉耆五城,掠男女一千五百人,焚其廬舍而去。十四年,侯君集討高昌,遣使與之相結,焉耆王大喜,請為聲援。及破高昌,其王詣軍門稱謁。
 焉耆人先為高昌所虜者,悉歸之。由是遣使謝恩,並貢方物。



 其年,西突厥重臣屈利啜為其弟娶焉耆王女,由是相為脣齒,朝貢遂闕。安西都護郭孝恪請擊之,太宗許焉。會焉耆王弟頡鼻葉護兄弟三人來至西州,孝恪選步騎三千出銀山道,以頡鼻弟慄婆準為鄉導。焉耆所都城,四面有水,自恃險固,不虞於我。孝恪倍道兼行,夜至城下,潛遣將士浮水而渡。至曉,一時攀堞,鼓角齊震,城中大擾。孝恪縱兵擊之,虜其王突騎支,首虜千餘
 級。以慄婆準導軍有功,留攝國事而還。時駕幸洛陽宮,孝恪鎖突騎支並其妻子送行在所,詔宥之。初,西突厥屈利啜將兵來援焉耆,孝恪還師三日,屈利啜乃囚慄婆準,而西突厥處般啜令其吐屯來攝焉耆,遣使朝貢。太宗數之曰:「焉耆者,我兵擊得,汝何人,輒來統攝。」吐屯懼而返國。焉耆又立慄婆準從父兄薛婆阿那支為王。處般啜乃執慄婆準送於龜茲,為所殺。薛婆阿那支既得處般啜為援,遂有國。及阿史那社爾之討龜茲,阿那
 支大懼,遂奔龜茲,保其東城,以御官軍。社爾擊擒之,數其罪而斬焉。求得阿那支從父弟先那準,立為王,以修職貢。及太宗葬昭陵。乃刻石像龍突騎支之形,列於玄闕之下。自是朝貢不絕。



 龜茲國,即漢西域舊地也。在京師西七千五百里。其王姓白氏。有城郭屋宇,耕田畜牧為業。男女皆翦發,垂與項齊,唯王不翦發。學胡書及婆羅門書、算計之事,尤重佛法。其王以錦蒙項,著錦袍金寶帶,坐金獅子床。有良
 馬、封牛。饒蒲萄酒,富室至數百碩。



 高祖即位,其主蘇伐勃駃遣使來朝。勃駃尋卒,子蘇伐疊代立,號時健莫賀俟利發。貞觀四年,又遣使獻馬,太宗賜以璽書,撫慰甚厚,由此歲貢不絕,然臣於西突厥。安西都護郭孝恪來伐焉耆,龜茲遣兵援助,自是職貢頗闕。伐疊死,其弟訶黎布失畢代立,漸失籓臣禮。



 二十年,太宗遣左驍衛大將軍阿史那社爾為昆山道行軍大總管,與安西都護郭孝恪、司農卿楊弘禮率五將軍,又發鐵勒十三部兵
 十餘萬騎,以伐龜茲。社爾既破西蕃處月、處密,乃進師趨其北境,出其不意,西突厥所署焉耆王棄城而遁,社爾遣輕騎追擒之。龜茲大震,守將多棄城而走。社爾進屯積石,去其都城三百里。遣伊州刺史韓威率千餘騎為前鋒,右驍衛將軍曹繼叔次之。西至多褐城,與龜茲王相遇,及其相那利、將羯獵顛等,有眾五萬,逆拒王師。威乃偽遁而引之,其王俟利發見威兵少,悉眾而至。威退行三十里,與繼叔軍會,合擊大破之。其王退保都城,
 社爾進軍逼之,王乃輕騎而走,遂下其城,令孝恪守之。遣沙州刺史蘇海政、尚輦奉御薛萬備以精騎逼之,行六百里,其王窘急,退保於撥換城。社爾等進軍圍之,擒其王及大將羯獵顛等。其相那利僅以身免,潛引西突厥之眾並其國兵萬餘人,來襲孝恪,殺之,官軍大擾。倉部郎中崔義起與曹繼叔、韓威等擊之,那利敗走。尋為龜茲人所執以詣軍。前後破其大城五所,虜男女數萬口。社爾因立其王之弟葉護為王,勒石紀功而旋。俘其
 王訶黎布失畢及那利、羯獵顛等獻於社廟。尋以訶黎布失畢為左武翊衛中郎將,那利已下授官各有差。太宗之葬昭陵,乃刻石像其形,列於玄闕之前。永徽元年,又以訶黎布失畢為右驍衛大將軍,尋放還蕃,撫其餘眾,依舊為龜茲王,賜物一千段。



 先是,太宗既破龜茲,移置安西都護府於其國城,以郭孝恪為都護,兼統於闐、疏勒、碎葉,謂之「四鎮」。高宗嗣位,不欲廣地勞人,復命有司棄龜茲等四鎮,移安西依舊於西州。其後吐蕃大入,
 焉耆已西四鎮城堡,並為賊所陷。則天臨朝,長壽元年,武威軍總管王孝傑、阿史那忠節大破吐蕃,克復龜茲、于闐等四鎮,自此復於龜茲置安西都護府,用漢兵三萬人以鎮之。既徵發內地精兵,遠逾沙磧。並資遣衣糧等,甚為百姓所苦。言事者多請棄之,則天竟不許。其安西都護,則天時有田揚名,中宗時有郭元振,開元初則張孝暠、杜暹,皆有政績,為夷人所伏。



 疏勒國,即漢時舊地也。西帶蔥嶺,在京師西九千三百
 里。其王姓裴氏。貞觀中,突厥以女妻王。勝兵二千人。俗事祅神,有胡書文字。貞觀九年,遣使獻名馬,自是朝貢不絕。開元十六年,玄宗遣使冊立其王裴安定為疏勒王。



 于闐國,西南帶蔥嶺,與龜茲接,在京師西九千七百里。勝兵四千人。其國出美玉。俗多機巧,好事祅神,崇佛教。先臣於西突厥。其王姓尉遲氏,名屈密。



 貞觀六年,遣使獻玉帶,太宗優詔答之。十三年,又遣子入侍。及阿史那
 社爾伐龜茲,其王伏闍信大懼,使其子以駝萬三百匹饋軍。及將旋師,行軍長史薛萬備請社爾曰:「今者既破龜茲,國威已振,請因此機,願以輕騎羈取於闐之王。」社爾乃遣萬備率五十騎抵於闐之國,萬備陳國威靈,勸其入見天子,伏闍信於是隨萬備來朝。



 高宗嗣位,拜右驍衛大將軍,又授其子葉護玷為右驍衛將軍,並賜金帶、錦袍、布帛六十段,並宅一區,留數月而遣之,因請留子弟以宿衛。太宗葬昭陵,刻石像其形,列於玄闕之下。



 垂拱三年,其王伏闍雄復來入朝。天授三年,伏闍雄卒,則天封其子璥為於闐國王。開元十六年,復冊立尉遲伏師為於闐王,數遣使朝貢。乾元三年,以於闐王尉遲勝弟守左監門衛率葉護曜為太僕員外卿,仍同四鎮節度副使。權知本國事。以勝至德初領兵赴國難,因堅請留宿衛,故有是命,事有勝傳。



 天竺國,即漢之身毒國,或云婆羅門地也。在蔥嶺西北,周三萬餘里。其中分為五天竺:其一曰中天竺,二曰東
 天竺,三曰南天竺,四曰西天竺,五曰北天竺。地各數千里,城邑數百。南天竺際大海,北天竺拒雪山,四周有山為壁,南面一穀,通為國門;東天竺東際大海,與扶南、林邑鄰接;西天竺與罽賓、波斯相接;中天竺據四天竺之會,其都城周回七十餘里,北臨禪連河。雲昔有婆羅門領徒千人,肄業於樹下,樹神降之,遂為夫婦。宮室自然而立,僮僕甚盛。於是使役百神,築城以統之,經日而就。此後有阿育王,復役使鬼神,累石為宮闕,皆雕文刻鏤。
 非人力所及。阿育王頗行茍政,置砲烙之刑,謂之地獄,今城中見有其跡焉。



 中天竺王姓乞利咥氏,或云剎利氏,世有其國,不相篡弒。厥土卑濕暑熱,稻歲四熟,有金剛,似紫石英,百煉不銷,可以切玉。又有旃檀、鬱金諸香。通於大秦,故其寶物或至扶南、交趾貿易焉。百姓殷樂,俗無簿籍,耕王地者輸地利。以齒貝為貨。人皆深目長鼻。致敬極者,氐足摩踵。家有奇樂倡伎。其王與大臣多服錦罽。上為螺髻於頂,餘發翦之使拳。俗皆徒跣。衣重
 白色,唯梵志種姓披白疊以為異。死者或焚尸取灰,以為浮圖;或委之中野,以施禽獸;或流之於河,以飼魚鱉。無喪紀之文。謀反者幽殺之,小犯罰錢以贖罪。不孝則斷手刖足,截耳割鼻,放流邊外。有文字,善天文算歷之術。其人皆學《悉曇章》,云是梵天法。書於貝多樹葉以紀事。不殺生飲酒。國中往往有舊佛跡。



 隋煬帝時,遣裴矩應接西蕃,諸國多有至者,唯天竺不通,帝以為恨。當武德中,其國大亂。其嗣王尸邏逸多練兵聚眾,所向無敵。
 象不解鞍,人不釋甲,居六載而四天竺之君皆北面以臣之,威勢遠振,刑政甚肅。



 貞觀十五年,尸羅逸多自稱摩伽陀王,遣使朝貢。太宗降璽書慰問,尸羅逸多大驚,問諸國人曰:「自古曾有摩訶震旦使人至吾國乎?」皆曰:「未之有也。」乃膜拜而受詔書,因遣使朝貢。太宗以其地遠,禮之甚厚,復遣衛尉丞李義表報使。尸羅逸多遣大臣郊迎,傾城邑以縱觀,焚香夾道,逸多率其臣下東面拜受敕書,復遣使獻火珠及鬱金香、菩提樹。



 貞觀十年,
 沙門玄奘至其國,將梵本經論六百餘部而歸。先是遣右率府長史王玄策使天竺,其四天竺國王咸遣使朝貢。會中天竺王尸羅逸多死,國中大亂,其臣那伏帝阿羅那順篡立,乃盡發胡兵以拒玄策。玄策從騎三十人與胡御戰,不敵,矢盡,悉被擒。胡並掠諸國貢獻之物。玄策乃挺身宵遁,走至吐蕃,發精銳一千二百人,並泥婆羅國七千餘騎,以從玄策。玄策與副使蔣師仁率二國兵進至中天竺國城,連戰三日,大破之,斬首三千餘級,
 赴水溺死者且萬人,阿羅那順棄城而遁,師仁進擒獲之。虜男女萬二千人,牛馬三萬餘頭匹。



 於是天竺震懼,俘阿羅那順以歸。二十二年至京師,太宗大悅,命有司告宗廟,而謂群臣曰:「夫人耳目玩於聲色,口鼻耽於臭味,此乃敗德之源。若婆羅門不劫掠我使人,豈為俘虜耶?昔中山以貪寶取弊,蜀侯以金牛致滅,莫不由之。」拜玄策朝散大夫。是時就其國得方土那邇娑婆寐,自言壽二百歲,云有長生之術。太宗深加禮敬,館之於金
 飆門內。造延年之藥。令兵部尚書崔敦禮監主之,發使天下,採諸奇藥異石,不可稱數。延歷歲月,藥成,服竟不效,後放還本國。太宗之葬昭陵也,刻石像阿羅那順之形,列於玄闕之下。



 五天竺所屬之國數十,風俗物產略同。有伽沒路國,其俗開東門以向日。王玄策至,其王發使貢以奇珍異物及地圖,因請老子像及《道德經》。那揭陀國,有醯羅城,中有重閣,藏佛頂骨及錫杖。貞觀二十年,遣使貢方物。天授二年,東天竺王摩羅枝摩、西天竺
 王尸羅逸多、南天竺王遮婁其拔羅婆、北天竺王婁其那那、中天竺王地婆西那,並來朝獻。景龍四年,南天竺國復遣使來朝。景雲元年,復遣使貢方物。開元二年,西天竺復遣使貢方物。八年,南天竺國遣使獻五色能言鸚鵡。其年,南天竺國王尸利那羅僧伽請以戰象及兵馬討大食及吐蕃等,仍求有及名其軍。玄宗甚嘉之,名軍為懷德軍。九月,南天竺王尸利那羅僧伽寶多枝摩為國造寺,上表乞寺額,敕以歸化為名賜之。十一月,遣
 使冊利那羅伽寶多為南天竺國王,遣使來朝。十七年六月,北天竺國藏沙門僧密多獻質汗等藥。十九年十月,中天竺國王伊沙伏摩遣其大德僧來朝貢。



 二十九年三月,中天竺王子李承恩來朝,授游擊將軍,放還。天寶中,累遣使來。



 罽賓國,在蔥嶺南,去京師萬二千二百里。常役屬於大月氏。其地暑濕,人皆乘象,土宜秔稻,草木凌寒不死。其俗尤信佛法。隋煬帝時,引致西域,前後至者三十餘國,
 唯罽賓不至。



 貞觀十一年,遣使獻名馬,太宗嘉其誠款,賜以繒彩。十六年,又遣使獻褥特鼠,喙尖而尾赤,能食蛇,有被蛇螫者,鼠輒嗅而尿之,其瘡立愈。顯慶三年,訪其國俗,云「王始祖馨孽,至今曷擷支,父子傳位,已十二代。」其年,改其城為修鮮都督府。龍朔初,授其王修鮮等十一州諸軍事兼修鮮都督。



 開元七年,遣使來朝,進天文經一夾、秘要方並蕃藥等物,詔遣冊其王為葛羅達支特勒。二十七年,其王烏散特勒灑以年老,上表請以
 子拂菻罽婆嗣位,許之,仍降使冊命。天寶四年,又冊其子勃匐準為襲罽賓及烏萇國王,仍授左驍衛將軍。乾元元年,又遣使朝貢。



 又有勃律國,在罽賓、吐蕃之間。開元中頻遣使朝獻。八年,冊立其王蘇麟陀逸之為勃律國王,朝貢不絕。二十二年,為吐蕃所破。



 康國,即漢康居之國也。其王姓溫,月氏人也。先居張掖祁連山北昭武城,為突厥所破,南依蔥嶺,遂有其地。枝庶皆以昭武為姓氏,不忘本也。其人皆深目高鼻,多須髯。丈
 夫翦發或辮發。其王冠氈帽。飾以金寶。婦人盤髻,幪以皁巾,飾以金花。人多嗜酒,好歌舞於道路。生子必以石蜜內口中,明膠置掌內,欲其成長口常甘言,掌持錢如膠之黏物。俗習胡書。善商賈,爭分銖之利。男子年二十,即遠之旁國,來適中夏,利之所在,無所不到。以十二月為歲首,有婆羅門為之占星候氣,以定吉兇。頗有佛法。至十一月,鼓舞乞寒,以水相潑,盛為戲樂。



 隋煬帝時,其王屈術支娶西突厥葉護可汗女,遂臣於西突厥。武德
 十年,屈術支遣使獻名馬。貞觀九年,又遣使貢獅子,太宗嘉其遠至,命秘書監虞世南為之賦,自此朝貢歲至。十一年,又獻金桃、銀桃,詔令植之於苑囿。



 萬歲通天年,則天封其大首領篤婆缽提為康國王,仍拜左驍衛大將軍。缽提尋卒,又冊其子泥涅師師為康國王。師師以神龍中卒,國人又立突昏為王。開元六年,遣使貢獻鎖子甲、水精杯、馬腦瓶、駝鳥卵及越諾之類。十九年,其王烏勒上表,請封其子咄曷為曹國王,默啜為米國王,許
 之。二十七年,烏勒卒,遣使冊咄曷襲父位。天寶三年,又封為欽化王,其母可敦封為郡夫人。十一載、十三載,並遣使朝貢。



 波斯國,在京師西一萬五千三百里,東與吐火羅、康國接,北鄰突厥之可薩部,西北拒拂菻,正西及南俱臨大海。戶數十萬。其王居有二城,復有大城十餘,猶中國之離宮。其王初嗣位,便密選子才堪承統者,書其名字,封而藏之。王死後,大臣與王之群子共發封而視之,奉所
 書名者為主焉。其王冠金花冠,坐獅子床,服錦袍,加以瓔珞。俗事天地日月水火之諸神,西域諸胡事火祅者,皆詣波斯受法焉。其事神,以麝香和蘇塗須點額,及於耳鼻,用以為敬,拜必交股。文字同於諸胡。男女皆徒跣。丈夫翦發,戴白皮帽,衣不開襟,並有巾帔,多用蘇方青白色為之,兩邊緣以織成錦。婦人亦巾帔裙衫,辮發垂後,飾以金銀,其國乘象而戰,每一象,戰士百人,有敗恤者則盡殺之。國人生女,年十歲已上有姿貌者,其王收而
 養之,以賞有功之臣。俗右尊而左卑。以六月一日為歲首。斷獄不為文書約束,口決於庭。其系囚無年限,唯王者代立則釋之。其叛逆之罪,就火祅燒鐵灼其舌,瘡白者為理直,瘡黑者為有罪。其刑有斷手、刖足、髡鉗、劓刖,輕罪翦須,或系牌於項以志之,經時月而釋焉。其強盜一入獄,至老更不出,小盜罰以銀錢。死亡則棄之於山,制服一月而即吉。氣候暑熱,土地寬平,知耕種,多畜牧,有鳥形如橐駝,飛不能高,食草及肉,亦能敢犬攫羊,土
 人極以為患。又多白馬、駿犬,或赤日行七百里者駮,金犬今所謂波斯犬也。出婁及大驢、師子、白象、珊瑚樹高一二尺,琥珀、車渠、瑪瑙、火珠、玻璃、琉璃、無食子、香附子、訶黎勒、胡椒、蓽撥、石蜜、千年棗、甘露桃。



 隋大業末,西突厥葉護可汗頻擊破其國,波斯王庫薩和為西突厥所殺,其子施利立,葉護因分其部帥,監統其國,波斯竟臣於葉護。及葉護可汗死,其所令監統者因自擅於波斯,不復役屬於西突厥。施利立一年卒,乃立庫薩和之女為
 王,突厥又殺之。施利之子單羯方奔拂菻,於是國人迎而立之,是為尹恆支,在位二年而卒。兄子伊嗣候立。



 二十一年,伊嗣候遣使獻一獸,名活褥蛇,形類鼠而色青,身長八九寸,能入穴取鼠。伊嗣候懦弱,為大首領所逐,遂奔吐火羅,未至,亦為大食兵所殺。其子名卑路斯,又投吐火羅葉護,獲免。卑路斯龍朔元年奏言頻被大食侵擾,請兵救援。詔遣隴州南由縣令王名遠充使西域,分置州縣,因列其地疾陵城為波斯都督府,授卑路斯
 為都督。是後數遣使貢獻。咸亨中,卑路斯自來入朝,高宗甚加恩賜,拜右武衛將軍。



 儀鳳三年,令吏部侍郎裴行儉將兵冊送卑路斯為波斯王,行儉以其路遠,至安西碎葉而還,卑路斯獨返,不得入其國,漸為大食所侵,客於吐火羅國二十餘年,有部落數千人,後漸離散。至景龍二年,又來入朝,拜為左威衛將軍,無何病卒,其國遂滅,而部眾猶存。



 自開元十年至天寶六載,凡十遣使來朝,並獻方物。四月,遣使獻瑪瑙床。九年四月,獻火毛
 繡舞筵、長毛繡舞筵、無孔真珠。乾元元年,波斯與大食同寇廣州,劫倉庫,焚廬舍,浮海而去。大歷六年,遣使來朝,獻真珠等。



 拂菻國,一名大秦,在西海之上,東南與波斯接,地方萬餘里,列城四百,邑居連屬。其宮宇柱櫳,多以水精琉璃為之。有貴臣十二人共治國政,常使一人將囊隨王車,百姓有事者,即以書投囊中,王還宮省發,理其枉直。其王無常人,簡賢者而立之。國中災異及風雨不時,輒廢
 而更立。其王冠形如鳥舉翼,冠及瓔珞,皆綴以珠寶,著錦繡衣,前不開襟,坐金花床。有一鳥似鵝,其毛綠色,常在王邊倚枕上坐,每進食有毒,其鳥輒鳴。其都城疊石為之,尤絕高峻,凡有十萬餘戶,南臨大海。城東面有大門,其高二十餘丈,自上及下,飾以黃金,光輝燦爛,連曜數里。自外至王室,凡有大門三重,列異寶雕飾。第二門之樓中,懸一大金秤,以金丸十二枚屬於衡端,以候日之十二時焉;為一金人,其大如人,立於側,每至一時,其
 金丸輒落,鏗然發聲,引唱以紀日時,毫厘無失。其殿以瑟瑟為柱,黃金為地,象牙為門扇,香木為棟梁。其俗無瓦,搗白石為末,羅之塗屋上,其堅密光潤,還如玉石。至於盛暑之節,人厭囂熱,乃引水潛流,上遍於屋宇,機制巧密,人莫之知。觀者惟聞屋上泉鳴,俄見四簷飛溜,懸波如瀑,激氣成涼風,其巧妙如此。



 風俗,男子翦發,披帔而右袒,婦人不開襟,錦為頭巾。家資滿億,封以上位。有羊羔生於土中,其國人候其欲萌,乃築墻以院之,防外
 獸所食也。然其臍與地連,割之則死,唯人著甲走馬及擊鼓以駭之,其羔警鳴而臍絕,便遂水草。俗皆髡而衣繡,乘輜軿白蓋小車,出入擊鼓,建旌旗幡幟。土多金銀奇寶,有夜光璧、明月珠、駭雞犀、大貝、車渠、瑪瑙、孔翠、珊瑚、琥珀,凡西域諸珍異多出其國。隋煬帝常將通拂菻,竟不能致。



 貞觀十七年,拂菻王波多力遣使獻赤玻璃、綠金精等物,太宗降璽書答慰,賜以綾綺焉。自大食強盛,漸陵諸國,乃遣大將軍摩栧伐其都城,因約為和好,
 請每歲輸之金帛,遂臣屬大食焉。乾封二年,遣使獻底也伽。大足元年,復遣使來朝。開元七年正月,其主遣吐火羅大首領獻獅子、羚羊各二。不數月,又遣大德僧來朝貢。



 大食國,本在波斯之西。大業中,有波斯胡人牧駝於俱紛摩地那之山,忽有獅子人語謂之曰:「此山西有三穴,穴中大有兵器,汝可取之。穴中並有黑石白文,讀之便作王位。」胡人依言,果見穴中有石及槊刃甚多,上有文,
 教其反叛。於是糾合亡命,渡恆曷水,劫奪商旅,其眾漸盛,遂割據波斯西境,自立為王。波斯、拂菻各遣兵討之,皆為所敗。



 永徽二年,始遣使朝貢。其王姓大食氏,名敢密莫末膩,自云有國已三十四年,歷三主矣。其國男兒色黑多須,鼻大而長,似婆羅門;婦人白皙。亦有文字。出駝馬,大於諸國。兵刃勁利。其俗勇於戰鬥,好事天神。土多沙石,不堪耕種,唯食駝馬等肉。俱紛摩地那山在國之西南,鄰於大海,其王移穴中黑石置之於國。又嘗遣人
 乘船,將衣糧入海,經八年而未及西岸。海中見一方石,石上有樹,乾赤葉青,樹上總生小兒;長六七寸,見人皆笑,動其手腳,頭著樹枝,其使摘取一枝,小兒便死,收在大食王宮。又有女國,在其西北,相去三月行。



 龍朔初,擊破波斯,又破拂菻,始有米面之屬。又將兵南侵婆羅門,吞並諸胡國,勝兵四十餘萬。長安中,遣使獻良馬。景雲二年,又獻方物。開元初,遣使來朝,進馬及寶鈿帶等方物。其使謁見,唯平立不拜,憲司欲糾之,中書令張說奏
 曰:「大食殊俗,慕義遠來,不可置罪。」上特許之。」尋又遣使朝獻,自云在本國惟拜天神,雖見王亦無致拜之法,所司屢詰責之,其使遂請依漢法致拜。其時西域康國、石國之類,皆臣屬之。其境東西萬里,東與突騎施相接焉。



 一云隋開皇中,大食族中有孤列種代為酋長,孤列種中又有兩姓:一號盆泥奚深,一號盆泥末換。其奚深後有摩訶末者,勇健多智,眾立之為主,東西征伐,開地三千里,兼克夏臘,一名釤城釤音所鑒反)。摩訶末後十四代,至
 末換。末換殺其兄伊疾而自立,復殘忍,其下怨之。有呼羅珊木粗人並波悉林舉義兵,應者悉令著黑衣,旬月間眾盈數萬。鼓行而西,生擒末換,殺之。遂求得奚深種阿蒲羅拔,立之。末換已前謂之白衣大食,自阿蒲羅拔後改為黑衣大食。阿蒲羅拔卒,立其弟阿蒲恭拂。至德初遣使朝貢,代宗時為元帥,亦用其國兵以收兩都。



 寶應、大歷中頻遣使來。恭拂卒,子迷地立。迷地卒,子牟棲立,牟棲卒,弟訶論立。貞元中,與吐蕃為勍敵。蕃軍太半
 西御大食,故鮮為邊患,其力不足也。十四年,詔以黑衣大食使含嵯、焉雞、沙北三人並為中郎將,各放還蕃。



 史臣曰:西方之國,綿亙山川,自張騫奉使已來,介子立功之後,通於中國者多矣。有唐拓境,遠極安西,弱者德以懷之,強者力以制之。開元之前,貢輸不絕。天寶之亂,邊徼多虞,邠郊之西,即為戎狄,槁街之邸,來朝亦稀。故古先哲王,務寧華夏,語曰:「近者悅,遠者來。」斯之謂矣!



 贊曰:大蒙之人,西方之國,與時盛衰,隨世通塞。勿謂戎
 心,不懷我德;貞觀、開元,槁街充斥。



\end{pinyinscope}