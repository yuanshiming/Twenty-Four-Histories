\article{卷一百五十三}

\begin{pinyinscope}

 ○於休烈子肅肅子敖敖
 子琮令狐峘歸崇敬子登登子融奚陟張薦子又新希復希復子讀蔣乂子系伸柳登弟冕子璟沈傳師子詢



 於休烈,河南人也。高祖志寧,貞觀中任左僕射,為十八學士。父默成,沛縣令,早卒。休烈至性貞愨,機鑒敏悟。自幼好學,善屬文,與會稽賀朝、萬齊融、延陵包融為文詞之友,齊名一時。舉進士,又應制策登科,授秘書省正字。累遷右補闕、起居郎、集賢殿學士,轉比部員外郎,郎中。楊國忠輔政,排不附己者,出為中部郡太守。



 值祿山構難,肅宗踐祚,休烈自中部赴行在,擢拜給事中。遷太常少卿,知禮儀事,兼修國史。肅宗自鳳翔還京,勵精聽受,
 嘗謂休烈曰:「君舉必書,良史也。朕有過失,卿書之否?」對曰:「禹、湯罪己,其興也勃焉。有德之君,不忘規過,臣不勝大慶。」時中原蕩覆,典章殆盡,無史籍檢尋。休烈奏曰:「《國史》一百六卷,《開元實錄》四十七卷,起居注並餘書三千六百八十二卷,並在興慶宮史館。京城陷賊後,皆被焚燒。且《國史》、《實錄》,聖朝大典,修撰多時,今並無本。伏望下御史臺推勘史館所由,令府縣招訪。有人別收得《國史》、《實錄》,如送官司,重加購賞。若是史官收得,仍赦其罪。得
 一部,超授官資,得一卷賞絹十匹。」數月之內,唯得一兩卷。前修史官工部侍郎韋述陷賊,入東京,至是以其家藏《國史》一百一十三卷送於官。



 肅宗以太常鐘磬,自隋已來,所傳五音,或有不調,乾元初謂休烈曰:「古者聖人作樂,以應天地之和,以合陰陽之序,則人不夭扎,物不疵癘。且金石絲竹,樂之器也。比親享郊廟,每聽懸樂,宮商不備,或鐘磬失度。可盡將鐘磬來,朕當於內自定。」太常集樂工考試,數日審知差錯,然後令別鑄造磨刻。及
 事畢,上臨殿親試考擊,皆合五音,群臣稱慶。



 休烈尋轉工部侍郎、修國史,獻《五代帝王論》,帝甚嘉之。宰相李揆矜能忌賢,以休烈修國史與己齊列,嫉之,奏為國子祭酒,權留史館修撰以下之。休烈恬然自持,殊不介意。舊儀,元正冬至,百官不於光順門朝賀皇后,乾元元年,張皇后遂行此禮。休烈奏曰:「《周禮》有命夫朝人主,命婦朝女君。自顯慶已來,則天皇后始行此禮。其日,命婦又朝光順門,與百官雜處,殊為失禮。」肅宗詔停之。



 代宗即位,
 甄別名品,宰臣元載稱之,乃拜右散騎常侍,依前兼修國史,尋加禮儀使。遷工部侍郎。又改檢校工部尚書,兼判太常卿事,正拜工部尚書,累封東海郡公,加金紫光祿大夫。在朝凡三十餘年,歷掌清要,家無提石之蓄。恭儉溫仁,未嘗以喜慍形於顏色。而親賢下士,推轂後進,雖位崇年高,曾無倦色。篤好墳籍,手不釋卷,以至於終。大歷七年卒,年八十一。有集十卷行於代。



 嗣子益,次子肅,相繼為翰林學士。



 是歲春,休烈妻韋氏卒。上以休烈
 父子儒行著聞,特詔贈韋氏國夫人,葬日給鹵簿鼓吹。及聞休烈卒,追悼久之,褒贈尚書左僕射,賻絹百匹、布五十端,遣謁者內常侍吳承倩就私第宣慰。儒者之榮,少有其比。



 肅官至給事中。肅子敖。



 敖字蹈中,以家世文史盛名。少為時彥所稱,志行修謹。登進士第,釋褐秘書省校書郎。湖南觀察使楊憑闢為從事;府罷,鳳翔節度使李鄘、鄂岳觀察使呂元膺相繼闢召。自協律郎、大理評事試監察御史,元和六年,真拜監察御史,轉殿中,歷
 倉部司勛二員外、萬年令,拜右司郎中,出為商州刺史。長慶四年,入為吏部郎中。其年,遷給事中。



 昭愍初即位,李逢吉用事,與翰林學士李紳素不葉,遂誣紳以不測之罪,逐於嶺外。紳同職駕部郎中知制誥龐嚴、司封員外郎知制誥蔣防,坐紳黨左遷信、汀等州刺史。黜詔下,敖封還詔書。時人以為與嚴相善,訴其非罪,皆曰:「於給事犯宰執之怒,伸龐、蔣之屈,不亦仁乎?」及駁奏出,乃是論龐嚴貶黜太輕,中外無不大噱,而逢吉由是獎之。尋
 轉工部侍郎,遷刑部,出為宣歙觀察使、兼御史中丞。



 敖溫裕長者,與物無忤,居官亦未嘗有立。周踐臺閣,三為列曹侍郎,謹順自容而已。太和四年八月卒,年六十六,贈禮部尚書。



 四子:球、珪、瑰、琮,皆登進士第。



 琮,落拓有大志,雖以門資為吏,久不見用。大中朝,駙馬都尉鄭顥以琮世故,獨以器度奇之。會有詔於士族中選人才尚公主,衣冠多避之。顥謂琮曰:「子人才甚佳,但不護細行,為世譽所抑,久而不調,能應此命乎?」琮然之。會李籓知貢
 舉,顥托之登第;其年遂升諫列,尚廣德公主,拜駙馬都尉。累踐臺閣,揚歷籓府。乾符中同平章事。



 黃寇犯京師,僖宗出幸,琮病不能從。既僭號,起琮為相。琮以疾辭。迫脅不已,琮曰:「吾病亟矣,死在旦夕。加以唐室親姻,義不受命,死即甘心。」竟為賊所害,而赦公主。主視琮受禍,謂賊曰:「妾李氏女也,義不獨存,願與於公並命。」賊不許,公主入室自縊而卒。廣德閨門有禮,咸通、乾符中譽在人口。於族內外冠婚喪祭,主必自預行禮,諸婦班而見之,
 尊卑答勞,咸有儀法,為時所稱。珪、球皆至清顯。



 令狐峘,德棻之玄孫。登進士第。祿山之亂,隱居南山豹林谷,谷中有亙別墅。司徒楊綰未仕時,避亂南山,止於峘舍。峘博學,貫通群書,有口辯,綰甚稱之。及綰為禮部侍郎,修國史,乃引峘入史館。自華原尉拜右拾遺,累遷起居舍人,皆兼史職,修《玄宗實錄》一百卷、《代宗實錄》四十卷。著述雖勤,屬大亂之後,起居注亡失,峘纂開元、天寶事,雖得諸家文集,編其詔策,名臣傳記十無三四,後
 人以漏落處多,不稱良史。大歷八年,改刑部員外郎。



 德宗即位,將厚奉元陵,峘上疏諫曰:



 臣聞《傳》曰:「近臣盡規」,《禮記》曰:「事君有犯而無隱」。臣幸偶昌運,謬參近列,敢竭狂愚,庶裨分寸,伏惟陛下詳察。



 臣讀《漢書·劉向傳》,見論王者山陵之誡,良史稱嘆,萬古芬芳。何者?聖賢之心,勤儉是務,必求諸道,不作無益。故舜葬蒼梧,不變其肆;禹葬會稽,不改其列。周武葬於畢陌,無丘壟之處;漢文葬於霸陵,因山谷之勢。禹非不忠也,啟非不順也,周公非
 不悌也,景帝非不孝也,其奉君親,皆從微薄。昔宋文公始為厚葬,用蜃炭,益車馬,其臣華元、樂舉,《春秋》書為不臣。秦始皇葬驪山,魚膏為燈燭,水銀為江海,珍寶之藏,不可勝計,千載非之。宋桓魋為石槨,夫子曰:「不如速朽」。子游問喪具,夫子曰:「稱家之有無」。張釋之對孝文曰:「使其中無可欲,雖無石槨,又何戚焉?」漢文帝霸陵皆以瓦器,不以金銀為飾。由是觀之,有德者葬逾薄,無德者葬逾厚,昭然可睹矣!



 陛下自臨御天下,聖政日新。進忠去
 邪,減膳節用,不珍雲物之瑞,不近鷹犬之娛。有司給物,悉依元估,利於人也。遠方底貢,唯供祀事,薄於己也。故澤州奏慶雲,詔曰:「以時和為嘉祥」;邕州奏金坑,詔曰:「以不貪為寶」。恭惟聖慮,無非至理。而獨六月一日制節文云「應緣山陵制度,務從優厚,當竭帑藏,以供費用」者,此誠仁孝之德,切於聖衷。伏以尊親之義,貴於合禮。陛下每下明詔,發德音,皆比蹤唐、虞,超邁周、漢。豈取悅凡常之目,有違賢哲之心,與失德之君競其奢侈者也?臣又
 伏讀遺詔曰:「其喪儀制度,務從儉約,不得以金銀錦彩為飾。」陛下恭順先志,動無違者。若制度優厚,豈顧命之意耶?



 伏惟陛下遠覽虞、夏、周、漢之制,深惟夫子、張釋之之誡,虔奉先旨,俯遵禮經,為萬代法,天下幸甚!今赦書雖已頒行,諸條尚猶未出,此時奉遺制,敷聖理,固未晚也。伏望速詔有司,悉從古禮。臣聞愚夫之言,明主擇焉。況臣忝職史官,親逢睿德,恥同華元、樂舉之為不臣也,願以舜、禹之理,紀聖猷也。夙夜懇迫,不敢不言,抵犯聖
 明,實憂罪譴。言行身黜,雖死猶生。



 優詔答曰:「朕頃議山陵,心方迷謬,忘遵先旨,遂有優厚之文。卿聞見該通,識度弘遠,深知不可,形於至言。援引古今,依據經禮,非唯中朕之病,抑亦成朕之躬。免朕獲不子之名,皆卿之力也。敢不聞義而徙,收之桑榆,奉以始終,期無失墜。古之遺直,何以加焉!」



 初,大歷中,劉晏為吏部尚書,楊炎為侍郎,晏用峘判吏部南曹事。峘荷晏之舉,每分闕,必擇其善者送晏,不善者送炎,炎心不平之。及建中初,亙為禮
 部侍郎,炎為宰相,不念舊事。有士子杜封者,故相鴻漸子,求補弘文生。炎嘗出杜氏門下,托封於峘。峘謂使者曰:「相公誠憐封,欲成一名,乞署封名下一字,峘得以志之。」炎不意峘賣,即署名托封。峘以炎所署奏論,言宰相迫臣以私,臣若從之,則負陛下,不從則炎當害臣。德宗出疏問炎,炎具言其事,德宗怒甚,曰:「此奸人,無可奈何!」欲決杖流之,炎苦救解,貶衡州別駕。遷衡州刺史。



 貞元中,李泌輔政,召拜右庶子、史館修撰。性既僻異,動失人
 和。在史館,與同職孔述睿等爭忿細故,數侵述睿。述睿長者,讓而不爭。無何,泌卒,竇參秉政,惡其為人,貶吉州別駕。久之,授吉州刺史。



 齊映廉察江西,行部過吉州。故事,刺史始見觀察使,皆戎服趨庭致禮;映雖嘗為宰相,然驟達後進,峘自恃前輩,有以過映,不欲以戎服謁。入告其妻韋氏,恥抹首趨庭。謂峘曰:「卿自視何如人,白頭走小生前,卿如不以此禮見映,雖黜死,我亦無恨。」



 峘曰「諾」,即以客禮謁之。映雖不言,深以為憾。映至州,奏峘糾
 前政過失,鞫之無狀,不宜按部臨人,貶衢州別駕。衢州刺史田敦,峘知舉時進士門生也。初峘當貢部,放榜日貶逐,與敦不相面。敦聞峘來,喜曰:「始見座主。」迎謁之禮甚厚。敦月分俸之半以奉峘。峘在衢州殆十年。順宗即位,以秘書少監征,既至而卒。



 元和三年,峘子太僕寺丞丕,始獻峘所撰《代宗實錄》四十卷。初,亙坐李泌貶,監修國史奏峘所撰實錄一分,請於貶所畢功。至是方奏,以功贈工部尚書。



 歸崇敬,字正禮,蘇州吳郡人也。曾祖奧,以崇敬故,追贈秘書監。祖樂,贈房州刺史。父待聘,亦贈秘書監。



 崇敬少勤學,以經業擢第。遭喪哀毀,以孝聞,調授四門助教。天寶末,對策高第,授左拾遺,改秘書郎。遷起居郎、贊善大夫,兼史館修撰,又加集賢殿校理。以家貧求為外職,歷同州、潤州長史,會玄宗、肅宗二帝山陵,參掌禮儀,遷主客員外郎。又兼史館修撰,改膳部郎中。



 崇敬以百官朔望朝服褲褶非古,上疏云:「按三代典禮,兩漢史籍,並無
 褲褶之制,亦未詳所起之由。隋代已來,始有服者。事不師古,伏請停罷。」從之。



 又諫:「東都太廟,不合置木主。謹按典禮,虞主用桑,練主用慄。作桑主則埋慄主,作慄主則埋桑主,所以神無二主,天無二日,土無二王也。東都太廟,是則天皇后所建,以置武氏木主。中宗去其主而存其廟,蓋將以備行幸遷都之置也。且殷人屢遷,前八後五,則前後遷都一十三度,不可每都而別立神主也。議者或云:『東都神主已曾虔奉而禮之,豈可一朝廢之乎?』
 且虞祭則立桑主而虔祀,練祭則立慄主而埋桑主,豈桑主不曾虔祀而乃埋之?又所闕之主,何須更作?作之不時,恐非禮也。」



 又議云:「每年春秋二時釋奠文宣王,祝板御署訖,北面揖,臣以為禮太重。謹按《大戴禮》,師尚父授周武王丹書,武王東面而立。今署祝板,伏請準武王東面之禮,輕重庶得其中。」



 時有術士巨彭祖上疏云:「大唐土德,千年合符,請每四季郊祀天地。」詔禮官儒者議之。崇敬議曰:「按舊禮,立春之日,迎春於東郊,祭青帝。立
 夏之日,迎夏於南郊,祭赤帝。先立秋十八日,迎黃靈於中地,祀黃帝。秋、冬各於其方。黃帝於五行為土王,在四季生於火,故火用事之末而祭之,三季則否。漢、魏、周、隋,共行此禮。國家土德乘時,亦以每歲六月土王之日,祀黃帝於南郊,以後土配,所謂合禮。今彭祖請用四季祠祀,多憑緯候之說,且據陰陽之說。事涉不經,恐難行用。」又議祭五人帝不稱臣云:「太昊五帝,人帝也,於國家即為前後之禮,無君臣之義。若於人帝而稱臣,則於天帝
 復何稱也?議者或云:『五人帝列於《月令》,分配五時。』則五神、五音、五祀、五蟲、五臭、五穀皆備,以備其時之色數,非謂別有尊崇也。」又請太祖景皇帝配天,事已具《禮儀志》。自是國典大禮,崇敬常參議焉。



 大歷初,以新羅王卒,授崇敬倉部郎中、兼御史中丞,賜紫金魚袋,充吊祭、冊立新羅使。至海中流,波濤迅急,舟船壞漏,眾咸驚駭。舟人請以小艇載崇敬避禍,崇敬曰:「舟中凡數十百人,我何獨濟?」逡巡,波濤稍息,竟免為害。故事,使新羅者,至海東
 多有所求,或攜資帛而往,貿易貨物,規以為利。崇敬一皆絕之,東夷稱重其德。使還,授國子司業,兼集賢學士。與諸儒官同修《通志》,崇敬知《禮儀志》,眾稱允當。



 時皇太子欲以仲秋之月,於國學行齒胄之禮。崇敬以國學及官名不稱,請改國學之制,兼更其名,曰:



 《禮記·王制》曰,天子學曰「闢雍」。又《五經通義》云:「闢雍,養老教學之所也。」以形制言之,雍,壅也;闢,璧也,壅水環之,圓如璧形。以義理言之,闢,明也;雍,和也,言以禮樂明和天下。《禮記》亦謂之
 澤宮。《射義》云:天子將祭,必先習射於澤宮。故前代文士,亦呼云璧池,亦曰璧沼,亦謂之學省。後漢光武立明堂、闢雍、靈臺,謂之三雍宮。至明帝,躬行養老於其中。晉武帝亦作明堂、闢雍、靈臺,親臨闢雍,行鄉飲酒之禮。又別立國子學,以殊士庶。永嘉南遷,唯有國子學,不立闢雍。北齊立國子寺,隋初亦然。至煬帝大業十三年,改為國子監。今國家富有四海,聲明文物之盛,唯闢雍獨闕,伏請改國子監為闢雍省。



 又以:



 祭酒之名,非學官所宜。按《
 周禮》:「師氏掌以義詔王,教國子。」請改祭酒為太師氏,位正三品。又司業者,義在《禮記》,云「樂正司業」。正,長也,言樂官之長,司主此業。《爾雅》云:「大板謂之業。」按《詩·周頌》:「設業設虡,崇牙樹羽。」則業是懸鐘磬之栒虡也。今太學既不教樂,於義則無所取,請改司業一為左師,一為右師,位正四品。



 又以:



 《五經》六籍,古先哲王致理之式也。國家創業,制取賢之法,立明經,發微言於眾學,釋回增美,選賢與能。自艱難已來,取人頗易,考試不求其文義,及第先
 取於帖經,遂使專門業廢,請益無從,師資禮虧,傳受義絕。今請以《禮記》、《左傳》為大經;《周禮》、《儀禮》、《毛詩》為中經;《尚書》、《周易》為小經,各置博士一員。其《公羊》、《穀梁》文疏少,請共準一中經,通置博士一員。所擇博士,兼通《孝經》、《論語》,依憑章疏,講解分明,注引旁通,問十得九;兼德行純潔,文詞雅正,儀形規範,可為師表者。令四品以上各舉所知。在外者給驛,年七十已上者蒲輪。其國子、太學、四門、三館,各立五經博士,品秩上下,生徒之數,各有差。其舊博士、
 助教、直講、經直及律館、算館助教,請皆罷省。



 其教授之法,學生至監,謁同業師。其所執贄,脯脩一束、清酒一壺,衫布一段,其色隨師所服。師出中門,延入與坐,割脩斟酒,三爵而止。乃發篋出經,摳衣前請。師為依經辨理,略舉一隅,然後就室。每朝、晡二時請益,師亦二時居講堂,說釋道義,發明大體,兼教以文行忠信之道,示以孝悌睦友之義。旬省月試,時考歲貢。以生徒及第多少,為博士考課上下。其有不率教,者,則檟楚撲之。國子不率教
 者,則申禮部,移為太學。太學之不變者,移之四門。四門之不變者,歸本州之學。州學之不變者,復本役,終身不齒。雖率教九年而學不成者,亦歸之州學。



 其禮部考試之法,請無帖經,但於所習經中問大義二十,得十八為通;兼《論語》、《孝經》各問十得八,兼讀所問文注義疏,必令通熟者為一通。又於本經問時務策三道,通二為及第。其中有孝行聞於鄉閭者,舉解具言於習業之下。省試之日,觀其所實,義少兩道,亦請兼收。其天下鄉貢,亦如
 之。習業考試,並以明經為名。得第者,授官之資與進士同。若此,則教義日深,而禮讓興;禮讓興,則強不犯弱,眾不暴寡。此由太學而來者也。



 詔下尚書集百僚定議以聞。議者以為省者,禁也,非外司所宜名。《周禮》代掌其職者曰氏,國學非代官,不宜曰太師氏。其餘大抵以習俗既久,重難改作,其事不行。



 會國學胥吏以餐錢差舛,御史臺按問,坐貶饒州司馬。建中初,又拜國子司業。尋選為翰林學士,遷左散騎常侍,加銀青光祿大夫。尋兼普
 王元帥參謀,累加光祿大夫。以兩河叛渙之徒初稟朝命,令崇敬以本官兼御史大夫持節宣慰,奉使稱旨。及還,上表請歸拜墓,許之,賜以繒帛,儒者榮之。尋加特進、檢校戶部尚書,遷工部尚書,並依前翰林學士,充皇太子侍讀。累表辭,以年老乞骸骨,改兵部尚書致仕。貞元十五年卒,時年八十,廢朝一日,贈左僕射。子登嗣。



 登,字沖之。雅實弘厚,事紀母以孝稱。大歷七年,舉孝廉高第,補四門助教。貞元初,復登賢良科,自美原尉拜右拾遺。
 時裴延齡以奸佞有恩,欲為相,諫議大夫陽城上疏切直,德宗赫怒。右補闕熊執易等亦以危言忤旨。初執易草疏成,示登,登愕然曰:「願寄一名。雷電之下,安忍令足下獨當!」自是同列切諫。登每聯署其奏,無所回避,時人稱重。轉右補闕、起居舍人,三任十五年。同列嘗出其下者,多以馳騖至顯官,而登與右拾遺蔣武,退然自守,不以淹速介意。後遷兵部員外郎,充皇子侍讀,尋加史館修撰。



 順宗初,以東朝舊恩,超拜給事中,旋賜金紫,仍錫
 衫笏焉。遷工部侍郎。與孟簡、劉伯芻、蕭俛受詔同翻譯《大乘本生心地觀經》。又為東宮及諸王侍讀,獻《龍樓箴》以諷。久之,改左散騎常侍。因中謝,憲宗問時所切,登以納諫為對,時論美之。轉兵部侍郎,兼判國子祭酒事,遷工部尚書。元和十五年卒,年六十七,贈太子少保。



 登有文學,工草隸。寬博容物。嘗使僮飼馬,馬蹄踶,僮怒,擊折馬足,登知而不責。晚年頗好服食,有饋金石之藥者,且云先嘗之矣,登服之不疑。藥發毒幾死,方訊雲未之嘗;
 他人為之怒,登視之無慍色。常慕陸象先之為人,議者亦以為近之。子融嗣。



 融,進士擢第,自監察拾遺入省,拜工部員外郎,遷考功員外。六年,轉工部郎中,充翰林學士。八年,正拜舍人。九年,轉戶部侍郎。開成元年,兼御史中丞。湖南觀察使盧周仁違敕進羨餘錢十萬貫。融奏曰:「天下一家,何非君土?中外財賦,皆陛下府庫也。周仁輒陳小利,妄設異端,言南方火災,恐成灰燼,進於京國,姑徇私誠。入財貨以希恩,待朝廷而何淺!臣恐天下放
 效,以羨餘為名,因緣刻剝,生人受弊。周仁請行重責,以例列籓。其所進錢,請還湖南,代貧下租稅。」詔周仁所進於河陰院收貯,以備水旱。金部員外郎韓益判度支案,子弟受人賂三千餘貫,半是擬贓。上問融曰:「韓益所犯,與盧元中、姚康孰甚?」對曰:「元中與康枉破官錢三萬餘貫,益所取受人事,比之殊輕。」乃貶梧州司戶。



 尋遷京兆尹。時府司物力不充,特敕賜錢五萬貫;府司以所賜之半還司農寺菜錢,融因對言之。上以融學家,因問「『蔬糲』
 字有賴音,何也?『糲』是飯之極粗者耶?」融以義類對之。時兩公主出降,府司供帳事殷,又俯近上巳,曲江賜宴奏請改日。上曰:「去年重陽,取九月十九日,未失重陽之意,今改取十三日可也。」既而李固言作相,素不悅融,罷尹。月餘,授秘書監。俄而固言罷,楊嗣復輔政,以融權知兵部侍郎。一年內拜吏部。三年檢校禮部尚書、興元尹、兼御史大夫,充山南西道節度使。



 融子仁晦、仁翰、仁憲、仁召、仁澤,皆登進士第。咸通中並至達官。



 奚陟,字殷卿,亳州人也。祖翰繹,天寶中弋陽郡太守。陟少好讀書,登進士第,又登制舉文詞清麗科,授弘文館校書,尋拜大理評事。佐入吐蕃使,不行,授左拾遺。丁父母憂,哀毀過禮,親朋愍之。車駕幸興元,召拜起居郎、翰林學士。辭以疾病,久不赴職,改太子司議郎。歷金部、吏部員外郎、左司郎中,彌綸省闥。又累奉使,皆稱旨。



 貞元八年,擢拜中書舍人。是歲,江南、淮西大雨為災,令陟勞問巡慰,所在人安悅之。中書省故事,姑息胥徒,以常在
 宰相左右也,陟皆以公道處之。先是右省雜給,率分等第,皆據職田頃畝,即主書所受與右史等。陟乃約以料錢為率,自是主書所得減拾遺。時中書令李晟所請紙筆雜給,皆不受;但告雜事舍人,令且貯之,他日便悉以遺舍人。前例,雜事舍人自攜私入,陟以所得均分省內官。又躬親庶務,下至園蔬,皆悉自點閱,人以為難,陟處之無倦。遷刑部侍郎。



 裴延齡惡京兆尹李充有能政,專意陷害之,誣奏充結陸贄,數厚賂遺金帛。充既貶官,又
 奏充比者妄破用京兆府錢穀至多,請令比部勾覆,以比部郎中崔元翰陷充,怨惡贄也。詔許之。元翰曲附延齡,劾治府史。府史到者,雖無過犯,皆笞決以立威,時論喧然。陟乃躬自閱視府案,具得其實,奏言:「據度支奏,京兆府貞元九年兩稅及已前諸色羨餘錢,共六十八萬餘貫,李充並妄破用。今所勾勘,一千二百貫已來是諸縣供館驛加破,及在諸色人戶腹內合收,其斛斗共三十二萬石,唯三百餘石諸色輸納所由欠折,其餘並是
 準敕及度支符牒,給用已盡。」陟之寬平守法,多如此類。元翰既不遂其志,因此憤恚而卒。



 陟尋以本官知吏部選事,銓綜平允,有能名,遷吏部侍郎。所蒞之官,時以為稱職。貞元十五年卒,年五十五,贈禮部尚書。



 張薦,字孝舉,深州陸澤人。祖翾,字文成,聰警絕綸,書無不覽。為兒童時,夢紫色大鳥,五彩成文,降於家庭。其祖謂之曰:「五色赤文,鳳也;紫文,趯翾也,為鳳之佐,吾兒當以文章瑞於明廷。」因以為名字。初登進士第,對策尤工,
 考功員外郎謇味道賞之曰:「如此生,天下無雙矣!」調授岐王府參軍。又應下筆成章及才高位下、詞標文苑等科。翾凡應八舉,皆登甲科。再授長安尉,遷鴻臚丞。凡四參選,判策為銓府之最。員外郎員半千謂人曰:「張子之文如青錢,萬簡萬中,未聞退時。」時流重之,目為「青錢學士」。然性褊躁,不持士行,尤為端士所惡,姚崇甚薄之。開元初,澄正風俗,翾為御史李全交所糾,言翾語多譏刺,時坐貶嶺南。刑部尚書李日知奏論,乃追敕移於近處。
 開元中,入為司門員外郎卒。翾下筆敏速,著述尤多,言頗詼諧。是時天下知名,無賢不肖,皆記誦其文。天后朝,中使馬仙童陷默啜,默啜謂仙童曰:「張文成在否?」曰:「近自御史貶官。」默啜曰:「國有此人而不用,漢無能為也。」新羅、日本東夷諸蕃,尤重其文,每遣使入朝,必重出金貝以購其文,其才名遠播如此。



 薦少精史傳,顏真卿一見嘆賞之。天寶中,浙西觀察使李涵表薦其才可當史任,乃詔授左司禦率府兵曹參軍。既至闕下,以母老疾,竟
 不拜命。母喪闋,禮部侍郎於邵舉前事以聞,召充史館修撰,兼陽翟尉。硃泚之亂,變姓名伏匿城中,因著《史遁先生傳》。德宗還宮,擢拜左拾遺。貞元元年冬,上親郊。時初克復,簿籍多失,禮文錯亂,乃以薦為太常博士,參典禮儀。四年,回紇和親,以檢校右僕射、刑部尚書關播充使,送咸安公主入蕃,以薦為判官,轉殿中侍御史。使還,轉工部員外郎,改戶部本司郎中。十一年,拜諫議大夫,仍充中館修撰。



 時裴延齡恃寵,譖毀士大夫。薦欲上書
 論之,屢揚言未果。延齡聞之怒,奏曰:「諫官論朝政得失,史官書人君善惡,則領史職者不宜兼諫議。」德宗以為然。薦為諫議月餘,改秘書少監。延齡排擯不已,會差使冊回紇毗伽懷信可汗及吊祭,乃命薦兼御史中丞,入回紇。二十年,吐蕃贊普死,以薦為工部侍郎、兼御史大夫,充入吐蕃吊祭使。涉蕃界二千餘里,至赤嶺東被病,歿於紇壁驛,吐蕃傳其柩以歸。順宗即位,兇問至,詔贈禮部尚書。



 薦自拾遺至侍郎,僅二十年,皆兼史館修撰。
 三使絕域,皆兼憲職。以博洽多能,敏於占對被選。有文集三十卷,及所撰《五服圖》、《宰輔略》、《靈怪集》、《江左寓居錄》等,並傳於時。子又新、希復,皆登進士第。



 又新,幼工文,善於傅會。長慶中,宰相李逢吉用事,翰林學士李紳深為穆宗所寵,逢吉惡之;求朝臣中兇險敢言者掎摭紳陰事,俾暴揚於搢紳間。又新與拾遺李續之、劉棲楚,尤蒙逢吉睠待,指為鷹犬。穆宗崩,昭愍初即位,又新等構紳,敗端州司馬,朝臣表賀,又至中書賀宰相。及門,門者止
 之曰:「請少留,緣張補闕在齋內與相公談。」俄而又新揮汗而出,旅揖群官曰:「端溪之事,又新不敢多讓。」人皆闢易憚之。與續之等七人,時號「八關十六子」。



 寶歷三年,逢吉出為山南東道節度使,請又新為副使,李續之為行軍司馬。逢吉為宰相時,用門下省主事田伾。伾犯贓亡命,逢吉保之於外。及罷相,裴度發其事,逢吉坐罰俸。又詔曰:「朕在億兆人之上,不令而人化,不言而人信者,法也。法行則君主重,法廢則朝廷輕。田伾常掛亡命之章,
 偷請養賢之祿,跡在搜捕,公行人間,而更冒選吏曹,顯擬郡佐。及黃樞覆驗,烏府追擒,證逮皆明,奸狀盡得。三移憲牒,一無申陳。眾狀滿前,群議溢耳,終則步健不至,瑯璫空來。蔑視紀綱,頗同侮謔,顧茲參畫,負我上臺。閱視連名,伊爾二子,又新可汀州刺史,李續之可涪州刺史。」及逢吉致仕,李訓用事,復召二子為尚書郎。訓貶,復貶而卒。



 希復子讀,登進士第,有俊才。累官至中書舍人、禮部侍郎,典貢舉,時稱得士。位終尚書左丞。



 蔣乂,字德源,常州義興人也。祖瑰,太子洗馬,開元中弘文館學士。父將明,累遷至左司郎中、國子司業、集賢殿學士、副知院事,代為名儒。而乂,史官吳兢之外孫,以外舍富墳史,幼便記覽不倦。七歲時,誦庾信《哀江南賦》,數遍而成誦在口,以聰悟強力,聞於親黨間。弱冠博通群籍,而史才尤長。其父在集賢時,以兵亂之後,圖籍溷雜,乃白執政,請攜乂入院,令整比之。宰相張鎰見而奇之,乃署為集賢小職。乂編次逾年,於亂中勒成部帙,得二
 萬餘卷,再遷王屋尉,充太常禮院修撰。貞元九年,轉右拾遺,充史館修撰。



 十三年,以故河中節度使張茂昭弟光祿少卿同正茂宗尚義章公主,茂宗方居母喪,有詔起復雲麾將軍成禮。詔下,乂上疏諫曰:「墨縗之禮,本緣金革。從古已來,未有駙馬起復尚主者。既乖典禮,且違人情,切恐不可。」上令中使宣諭云:「茂宗母臨亡有請,重違其心。」乂又拜疏,辭逾激切。德宗於延英特召入對,上曰:「卿所言,古禮也。朕聞如今人家,往往有借吉為婚嫁
 者,卿何苦固執?」對曰:「臣聞里俗有不甚知禮法者,或女居父母服內,家既貧匱,旁無至親,即有借吉以就禮者。男子借吉而娶,臣未嘗聞之。況陛下臨御已來,每事憲章典禮。建中年郡縣主出降,皆詔有司依禮,不用俗儀,天下慶戴。忽今駙馬起復成禮,實恐驚駭物聽。臣或聞公主年甚幼小,即更俟一年出降,時既未失,且合禮經,實天下幸甚!」上曰:「卿言甚善,更俟商量。」俄而韋彤、裴堪諫疏繼入,上不悅,促令奉行前詔,然上心頗重乂。



 上嘗
 登凌煙閣,見左壁頹剝,文字殘缺,每行僅有三五字,命錄之以問宰臣。宰臣遽受宣,無以對;即令召乂至,對曰:「此聖歷中《侍臣圖贊》,臣皆記憶。」即於御前口誦,以補其缺,不失一字。上嘆曰:「虞世南暗寫《列女傳》,無以加也。」十八年,遷起居舍人,轉司勛員外郎,皆兼史職。時集賢學士甚眾,會詔問神策軍建置之由。相府討求,不知所出,諸學士悉不能對,乃訪於乂。乂徵引根源,事甚詳悉,宰臣高郢、鄭珣瑜相對曰:「集賢有人矣!」翌日,詔兼判集賢
 院事。父子代為學士,儒者榮之。時順宗祔廟,將行祧遷之禮,詔公卿議。咸云:「中宗中興之主,不當遷。」乂建議云:「中宗既正位柩前,乃受母後篡奪,五王翼戴,方復大業。此乃由我失之,因人得之,止可同於返正,不得號為中興。」群議紛然,竟依乂所執。



 元和二年,遷兵部郎中。與許孟容、韋貫之等受詔刪定制敕,成三十卷,奏行用。改秘書少監,復兼史館修撰。尋奉詔與獨孤鬱、韋處厚同修《德宗實錄》。五年,書成奏御,以功拜右諫議大夫。明年監
 修國史裴垍罷相,李吉甫再入,以乂垍之修撰,改授太常少卿。久之,遷秘書監。



 乂性樸直,不能事人,或遇權臣專政,輒數歲不遷官。在朝垂三十年,前後每有大政事、大議論,宰執不能裁決者,必召以咨訪。乂徵引典故,以參時事,多合其宜,然亦以此自滯。而好學不倦,老而彌篤,雖甚寒暑,手不釋卷。旁通百家,尤精歷代沿革。家藏書一萬五千卷。本名武,因憲宗召對,奏曰,「陛下已誅群寇,偃武修文,臣名於義未允,請改名乂。」上忻然從之。時
 帝方用兵兩河,乂亦因此諷諭耳。乂居史任二十年,所著《大唐宰輔錄》七十卷、《凌煙閣功臣》、《秦府十八學士》、《史臣》等傳四十卷。長慶元年卒,年七十五,贈禮部尚書,謚曰懿。子系、伸、偕、仙、佶。



 系,太和初授昭應尉,直史館。二年,拜右拾遺、史館修撰,典實有父風。與同職沈傳師、鄭浣、陳夷行、李漢等受詔撰《憲宗實錄》。四年,書成奏御,轉尚書工部員外,遷本司郎中,仍兼史職。宰相宋申錫為北軍羅織,罪在不測,系與諫官崔玄亮泣諫於玉階之下,
 申錫亦減死,時論稱之。開成中,轉諫議大夫。武宗朝,李德裕用事,惡李漢,以系與漢僚婿,出為桂管都護禦觀察使。中宗即位,徵拜給事中、集賢殿學士、判院事。轉吏部侍郎,改左丞。出為興元節度使,入為刑部尚書。俄檢校戶部尚書、鳳翔尹,充鳳翔隴節度使,入為兵部尚書。以弟伸為丞相,懇辭朝秩,檢校尚書左僕射、襄州刺史、山南東道節度使,封淮陽縣開國公,食邑五百戶。



 伸,登進士第,歷佐使府。大中初入朝,右補闕、史館修撰,轉中
 書舍人,召入翰林為學士。自員外郎中,至戶部侍郎、學士承旨,轉兵部侍郎。大中末,中書侍郎、平章事。



 仙、佶,皆至刺史。



 偕,有史才,以父任歷官左拾遺、史館修撰,轉補闕。咸通中,與同職盧耽、牛叢等受詔修《文宗實錄》。



 蔣氏世以儒史稱,不以文藻為事,唯伸及系子兆有文才,登進士第,然不為文士所譽。與柳氏、沈氏父子相繼修國史實錄,時推良史,京師云《蔣氏日歷》,士族靡不家藏焉。



 柳登,字成伯,河東人。父芳,肅宗朝史官,與同職韋述受
 詔添修吳兢所撰《國史》;殺青未竟而述亡,芳緒述凡例,勒成《國史》一百三十卷。上自高祖,下止乾元,而敘天寶後事,絕無倫類,取舍非工,不為史氏所稱。然芳勤於記注,含毫罔倦。屬安、史亂離,國史散落,編綴所聞,率多闕漏。上元中坐事徙黔中,遇內官高力士亦貶巫州,遇諸途。芳以所疑禁中事,咨於力士。力士說開元、天寶中時政事,芳隨口志之。又以《國史》已成,經於奏御,不可復改,乃別撰《唐歷》四十卷,以力士所傳,載於年歷之下。芳自
 永寧尉、直史館,轉拾遺、補闕、員外郎,皆居史任,位終右司郎中、集賢學士。



 登少嗜學,與弟冕咸以該博著稱。登年六十餘,方從宦游,累遷至膳部郎中。元和初,為大理少卿,與刑部侍郎許孟容等七人,奉詔刪定開元已後敕格。再遷右庶子,以衰病改秘書監,不拜,授右散騎常侍致仕。長慶二年卒,時九十餘,輟朝一日,贈工部尚書。弟冕。



 冕,文史兼該,長於吏職。貞元初,為太常博士。二年,昭德王皇后之喪,論皇太子服紀。左補闕穆質請依禮
 周期而除,冕與同職張薦等奏議曰:



 準《開元禮》,子為母齊衰三年,此王公已下服紀。皇太子為皇后喪服,國禮無聞。昔晉武帝元皇后崩,其時亦疑太子所服。杜元凱奏議曰:「古者天子三年之喪,既葬除服。魏氏革命,亦以既葬為節。故天子諸侯之禮,嘗已具矣,惡其害己而削去其節。今其存者唯《士喪禮》一篇,戴勝之記錯雜其內,亦難以取正。皇太子配二尊,與國為體,固宜卒哭而除服。」於是山濤、魏舒並同其議,晉朝從之。歷代遵行,垂之
 不朽。



 臣謹按實錄,文德皇后以貞觀十年九月崩,十一月葬,至十一年正月,除晉王,治為並州都督。晉王即高宗在籓所封,文德皇后幼子,據其命官,當已除之義也。今請皇太子依魏、晉故事,為大行皇后喪服,葬而虞,虞而卒哭,卒哭而除,心喪終制,庶存厭降之禮。



 事下中書,宰臣召問禮官曰:「《語》云:『子食於有喪者之側,未嘗飽也。』今豈可令皇太子衰服侍膳,至於既葬乎?準令,群臣齊衰,給假三十日即公除。約於此制,更審議之。」張薦曰:「請
 依宋、齊間皇后為父母服三十日公除例,為皇太子喪服之節。」三十日公除詣於正內,則服墨慘,歸至本院,衰麻如故。穆質曰:「杜元凱既葬除服之論,不足為法。臣愚以為遵三年之制則太重,從三十日之變太輕,唯行古之道,以周年為定。」詔宰臣與禮官定可否。宰臣以穆質所奏問博士,冕對曰:「準《禮》,三年喪,無貴賤一也。豈有以父母貴賤而差降喪服之節乎?且《禮》有公門脫齊衰,《開元禮》皇后為父母服十三月,其稟朝旨,十三日而除;皇
 太子為外祖父母服五月,其從朝旨,則五日而除。所以然者,恐喪服侍奉,有傷至尊之意也。故從權制,昭著國章,公門脫衰,義亦在此,豈皆為金革乎?皇太子今若抑哀,公除墨慘朝覲,歸至本院,依舊衰麻,酌於變通,庶可傳繼。」宰臣然其議,遂命太常卿鄭叔則草奏,以冕議為是。而穆質堅執前義,請依古禮,不妨太子墨衰於內也。宰臣齊映、劉滋參酌群議,請依叔則之議,制從之。及董晉為太常卿,德宗謂之曰:「皇太子所行周服,非朕本意,
 有諫官橫論之。今熟計之,即禮官請依魏、晉故事,斯甚折衷。」明年冬,上以太子久在喪,合至正月晦受吉服,欲以其年十一月釋衰麻,以及新正稱慶。有司皆論不可,乃止。



 六年十一月,上親行郊享。上重慎祀典,每事依禮。時冕為吏部郎中,攝太常博士,與司封郎中徐岱、倉部郎中陸質、工部郎中張薦,皆攝禮官,同修郊祀儀注,以備顧問。初,詔以皇太子亞獻終獻,當受誓戒否,冕對曰:「準《開元禮》有之,然誓詞云『不供其職,國有常刑』,今太子
 受誓,請改云『各揚其職,肅奉常儀』。」上又問升郊廟去劍履,及象劍尺寸之度,祝文輕重之宜,冕據禮經沿革聞奏,上甚嘉之。



 冕言事頗切,執政不便之,出為婺州刺史。十三年,兼御史中丞、福州刺史,充福建都團練觀察使。冕在福州,奏置萬安監牧,於泉州界置群牧五,悉索部內馬五千七百匹、驢騾牛八百頭、羊三千口,以為監牧之資。人情大擾,期年,無所滋息,詔罷之。以政無狀,詔以閻濟美代歸而卒。子璟,登進士第,亦以著述知名。



 璟,寶
 歷初登進士第,三遷監察御史。時郊廟告祭,差攝三公行事,多以雜品;璟時監察,奏曰:「準開元二十三年敕,宗廟大祠,宜差左右丞相、嗣王、特進、少保、少傅、尚書、賓客、御史大夫。又二十五年敕,太廟五享,差丞相、師傅、尚書、嗣、郡王通攝,餘司不在差限。又元和四年敕,太廟告祭攝官,太尉以宰相充,其攝司空、司徒,以僕射、尚書、師傅充,餘司不在差限。比來吏部因循,不守前後敕文,用人稍輕。請自今年冬季,勒吏部準開元、元和敕例差官。」從
 之。再遷度支員外郎,轉吏部。開成初,換庫部員外郎、知制誥,尋以本官充翰林學士。



 初,璟祖芳精於譜學,永泰中按宗正譜牒,自武德已來宗枝昭穆相承,撰皇室譜二十卷,號曰《永泰新譜》,自後無人修續。



 璟因召對,言及圖譜事,文宗曰:「卿祖嘗為皇家圖譜,朕昨觀之,甚為詳悉。卿檢永泰後試修續之。」璟依芳舊式,續德宗後事,成十卷,以附前譜,仍詔戶部供紙筆廚料。五年,拜中書舍人充職。武宗朝,轉禮部侍郎,再司貢籍,時號得人。子韜
 亦以進士擢第。



 沈傳師,字子言,吳人。父既濟,博通群籍,史筆尤工,吏部侍郎楊炎見而稱之。建中初,炎為宰相,薦既濟才堪史任,召拜左拾遺、史館修撰。既濟以吳兢撰《國史》,以則天事立本紀,奏議非之曰:



 史氏之作,本乎懲勸,以正君臣,以維家邦。前端千古,後法萬代,使其生不敢差,死不妄懼。緯人倫而經世道,為百王準的;不止屬辭比事,以日系月而已。故善惡之道,在乎勸誡;勸誡之柄,存乎褒貶。
 是以《春秋》之義,尊卑輕重升降,幾微仿佛,雖一字二字,必有微旨存焉。況鴻名大統,其可以貸乎?



 伏以則天皇后,初以聰明睿哲,內輔時政,厥功茂矣。及弘道之際,孝和以長君嗣位,而太后以專制臨朝;俄又廢帝,或幽或徙。既而握圖稱籙,移運革名,牝司燕啄之蹤,難乎備述。其後五王建策,皇運復興,議名之際,得無降損。必將義以親隱,禮從國諱,茍不及損,當如其常,安可橫絕彞典,超居帝籍?昔仲尼有言,必也正名,故夏、殷二代為帝者
 三十世矣,而周人通名之曰王;吳、楚、越之君為王者百餘年,而《春秋》書之為子。蓋高下自乎彼,而是非稽乎我。過者抑之,不及者援之,不為弱減,不為僭奪。握中持平,不振不傾,使其求不可得,而蓋不可掩,斯古君子所以慎其名也。



 夫則天體自坤順,位居乾極,以柔乘剛,天紀倒張,進以強有,退非德讓。今史臣追書,當稱之太后,不宜曰「上」。孝和雖迫母後之命,降居籓邸,而體元繼代,本吾君也,史臣追書,宜稱曰「皇帝」,不宜曰「廬陵王」。睿宗在
 景龍已前,天命未集,徒稟後制,假臨大寶,於倫非次,於義無名,史臣書之,宜曰「相王」,未宜曰「帝」。若以得失既往,遂而不舉,則是非褒貶,安所辨正,載筆執簡,謂之何哉?則天廢國家歷數,用周正朔,廢國家太廟,立周七廟。鼎命革矣,徽號易矣,旂裳服色,既已殊矣!今安得以周氏年歷而列為《唐書》帝紀?征諸禮經,是謂亂名。且孝和繼天踐祚,在太后之前,而敘年制紀,居太后之下;方之躋僖。是謂不智,詳今考古,並未為可。



 或曰:班、馬良史也,編
 述漢事,立高后以續帝載,豈有非之者乎?答曰:昔高后稱制,因其曠嗣,獨有分王諸呂,負於漢約,無遷鼎革命之甚。況其時孝惠已歿,孝文在下,宮中二子,非劉氏種,不紀呂后,將紀誰焉?雖云其然,議者猶為不可,況遷鼎革命者乎?



 或曰:若天后不紀,帝緒缺矣,則二十二年行事,何所系乎?曰:孝和以始年登大位,以季年復舊業,雖尊名中奪,而天命未改,足以首事,足以表年,何所拘閡,裂為二紀?昔魯昭之出也,《春秋》歲書其居,曰「公在乾侯」。
 且君在,雖失位,不敢廢也。今請並《天后紀》合《孝和紀》,每於歲首,必書孝和所在以統之,書曰某年春正月,皇帝在房陵,太后行某事,改某制云云。則紀稱孝和,而事述太后,俾名不失正,而禮不違常;名禮兩得,人無間矣!其姓氏名諱,入宮之由,歷位之資,才藝智略,年辰崩葬,別纂錄入《皇后傳》,列於廢後王庶人之下,題其篇曰「則天順聖武后」云。



 事雖不行,而史氏稱之。



 德宗初即位,銳於求理。建中二年夏,敕中書、門下兩省,分置待詔官三十
 員,以見官前任及同正試攝九品已上,擇文學理道、韜鈐法度之深者為之,各準品秩給俸錢,廩餼、幹力、什器、館宇之設,以公錢為之本,收息以贍用。物論以為兩省皆名侍臣,足備顧問,無勞別置冗員。既濟上疏論之曰:



 臣伏以陛下今日之理,患在官煩,不患員少;患在不問,不患無人。且中書、門下兩省常侍、諫議、補闕、拾遺,總四十員,及常參待制之官,日有兩人,皆備顧問,亦不少矣。中有二十一員,尚闕人未充,他司缺職,累倍其數。陛下
 若謂見官非才,不足與議,則當選求能者,以代其人。若欲務廣聰明,畢收淹滯,則當擇其可者,先補缺員。則朝無曠官,俸不徒費。且夫置錢息利,是有司權宜,非陛下經理之法。今官三十員,皆給俸錢,幹力及廚廩什器、建造宇,約計一月不減百萬,以他司息例準之,當以錢二千萬為之本,方獲百萬之利。若均本配人,當復除二百戶,或許其入流。反覆計之,所損滋甚。當今關輔大病,皆為百司息錢,傷人破產,積於府縣。實思改革,以正本
 源。又臣嘗計天下財賦耗篸之大者,唯二事焉:最多者兵資,次多者官俸。其餘雜費,十不當二事之一。所以黎人重困,杼軸猶空。方期緝熙,必藉裁減。今四方形勢,兵罷未得,資費之廣,蓋非獲已。陛下躬行儉約,節用愛人,豈俾閑官,復為冗食?籍舊而置,猶可省也,若之何加焉?陛下必以制出不可改,請重難慎擇,遷延寢罷。



 其事竟不得行。既而楊炎譴逐,既濟坐貶處州司戶。後復入朝,位終禮部員外郎。



 傳師擢進士,登制科乙第,授太子校
 書郎、鄠縣尉,直史館,轉左拾遺、左補闕,並兼史職。遷司門員外郎、知制誥,召充翰林學士。歷司勛、兵部郎中,遷中書舍人。性恬退無競,時翰林未有承旨,次當傳師為之,固稱疾,宣召不起,乞以本官兼史職。俄兼御史中丞,出為潭州刺史、湖南觀察使。入為尚書右丞。出為洪州刺史、江南西道觀察使,轉宣州刺史、宣歙池觀察使。入為吏部侍郎。太和元年卒,年五十九,贈吏部尚書。



 初,傳師父既濟撰《建中實錄》十卷,為時所稱。傳師在史館,預
 修《憲宗實錄》未成,廉察湖南,特詔齎一分史稿,成於理所。



 有子樞、詢,皆登進士第。



 詢歷清顯,中書舍人、翰林學士、禮部侍郎。咸通中,檢校戶部尚書、潞州長史、昭義節度使。為政簡易,性本恬和。奴歸秦者,通詢侍者,詢將戮之未果;奴結牙將為亂,夜攻府第,詢舉家遇害。



 史臣曰:前代以史為學者,率不偶於時,多罹放逐,其故何哉?誠以褒貶是非在於手,賢愚輕重系乎言,君子道微,俗多忌諱,一言切己,嫉之如仇。所以峘、薦坎壈於仕
 塗,沈、柳不登於顯貫,後之載筆執簡者,可以為之痛心!道在必伸,物不終否,子孫藉其餘祐,多至公卿者,蓋有天道存焉!



 贊曰:褒貶以言,孔道是模。誅亂以筆,亦有董狐。邦家大典,班、馬何辜?懲惡勸善,史不可無。



\end{pinyinscope}