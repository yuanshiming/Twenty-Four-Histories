\article{卷三十九 志第十九 地理二}

\begin{pinyinscope}

 ○河東道三河北道四山南道五



 河東
 道



 河中府隋河東郡。武德元年,置蒲州,治桑泉縣,領河東、桑泉、猗氏、虞鄉四縣。二年,置蒲州總管府,管蒲、虞、泰、絳、邵、澮六州。三年,移蒲治河東縣,依舊總管府。其年,置溫泉縣。九年,又置都督府,管蒲、虞、芮、邵、泰五州,仍省溫泉縣。其年,罷都督府。貞觀八年,割虢州之永樂來屬。十七年,以廢虞州之安邑解縣、廢泰州之汾陰來屬。開元八年,置中都,改蒲州為河中府。其年,罷中都,依舊為蒲州,又與陜,鄭、汴、懷、魏為「六雄」。十二年,昇為「四輔」。天寶元年,
 改為河東郡。乾元元年,復為蒲州,割安邑屬陜州。三年四月,置河中府,析同州之朝邑,於河西鹽坊置河西縣,來屬。元年建卯月,又為中都。元和三年,復為河中府。舊領縣五,戶三萬六千四百九十九,口十七萬三千七百八十四。天寶領縣八,戶七萬八百,口四十六萬九千二百一十三。元和領縣十一。在京師東北三百二十四里,去東都五百五十里。



 河東隋縣。州理所。
 開元八年,分置河西縣。其年,罷中都,乃省,乾元三年,復置



 河西舊朝邑縣,屬同州,管長春宮。乾元元年,置河中府,割朝邑來屬,改為河西縣,以鹽坊為理所



 臨晉隋分猗氏置桑泉縣。武德三年,分置溫泉縣。九年,省溫泉並入桑泉。天寶
 十
 三年,改為臨晉縣



 解隋虞鄉縣。武德元年,改為解縣,屬虞州。蒲州別置
 虞鄉縣。貞觀十七年,省解縣並入虞鄉。二十二年,復析置解縣,屬蒲州



 猗氏漢縣,古郇國也



 虞鄉漢解縣地,後魏分置虞鄉縣。貞觀十七年,省解縣,並入虞鄉縣。二十年,復置解縣,省虞鄉。天授二年,復分解縣置虞鄉縣



 永樂武德元年,分芮城縣置屬芮州。九年,廢芮州,改屬鼎州。貞觀八年,改屬蒲州,又割屬虢州。神龍元年,復來屬



 寶鼎漢汾陰縣。隋屬泰州。貞觀十七年,廢泰州,縣來屬。開元十一年,玄宗祀后土,獲寶鼎,因改為寶鼎



 龍門漢皮氏縣,後魏改為龍門。武德元年,於縣
 置泰州,領龍門、萬泉、汾陰四縣。貞觀十七年,廢泰州及芮縣,以龍門、萬泉屬絳州,汾陰屬蒲州



 聞喜漢縣。隋為桐鄉縣。武德元年,分置聞喜縣



 萬泉武德三年,分稷山界於薛通城置萬泉縣,屬泰州。州
 廢,入絳州,後又隸河中府。



 絳州隋絳郡。武德元年,置絳州總管府,管絳、潞、蓋、建、澤、沁、韓、晉、呂、舉、泰、蒲、虞、芮、邵十五州。絳州領正平、太平、曲沃、聞喜、稷山五縣。三年,廢總管府。其年,以廢北澮州之翼城置翼城縣。領翼城、絳、小鄉三縣。武德元年,改為澮州。二年,改為北澮州。四年,州廢,三縣並入絳州。置南絳州,又置絳縣。



 曲沃漢絳縣地,後魏置曲沃縣



 絳漢聞喜縣,後魏置南絳州,又置絳縣



 稷山後魏高涼縣,隋改名稷山



 垣隋縣。義寧元年,
 置
 邵原,領垣、王屋,又置清廉、亳城四縣。武德元年,改為邵州。二年,又置長泉縣。五年,廢亳城。九年,省邵州,省清廉入垣縣,王屋屬懷州,垣屬絳州



 襄陵後魏擒盛縣。改為襄陵,取漢舊名。屬晉州。元和十四年,屬絳州。



 晉州隋臨汾郡。義旗初,改為平陽郡,領臨汾、襄陵、岳
 陽、冀氏、楊五縣。其年,改楊縣為洪洞。武德元年,改為晉州,分襄陵置浮山縣,分洪洞置西河縣。三年,置總管府,管晉、絳、沁、呂四州。移治白馬城。改浮山為神山縣。貞觀六年,廢都督。十二年,移治所於平陽古城。十七年,省西河縣,以廢呂州之霍邑、趙城、汾西三縣來屬。天寶元年,改州為平陽郡。乾元元年,復為晉州。元和十四年,割襄陵屬絳州。大和元年,改屬河中府。舊領縣七,戶二萬一千六百一十七,口九萬七千五百五。天寶領縣九,戶六
 萬四千八百三十六,口四十二萬九千二百二十一。元和領縣八。在京師東北七百二十五里,至東都七百三十九里。



 臨汾漢平陽縣,隋改為臨汾。貞觀十七年,省西河縣,並入臨汾



 洪洞漢楊縣,至隋不改。義寧元年,改為洪洞,取縣北嶺名



 神山武德二年,分襄陵置浮山縣。四年,改為神山,以縣東南羊角山神見為名



 岳陽後魏安澤縣,隋改為岳陽



 霍邑漢彘縣,後漢改
 為永安。隋於此置汾州,尋改為呂州,領霍邑、趙城、汾西、靈石四縣。貞觀十七年,廢呂州,以霍邑等三縣來屬,以靈石屬汾州



 趙城國初,分霍邑縣置



 汾西後漢汾西郡,隋廢為縣,屬呂州。隋末陷賊。武德初,權於今城南五十里申村堡置。貞觀六年,移於今所



 冀氏漢猗氏縣地,後於古猗氏縣地南置冀氏。



 隰州下隋龍泉郡。武德元年,改為隰州,領隰川、溫泉、大寧、石樓四縣。二年,置總管府,領隰、中、昌、南汾、東和、西
 德六州。三年,又置北溫州屬焉。貞觀元年,省中、昌、西德、北溫四州,又以廢昌州蒲縣來屬,仍督隰、南汾、東和三州。三年,廢都督府。又以廢東和州永和縣來屬。天寶元年,改為大寧郡。乾元元年,復為隰州。舊領縣六,戶八千二百二十二,口三萬八千三百九十五。天寶,戶一萬九千四百五十五,口十二萬四千四百二十。在京師東北九百六里,至東都八百八十里。



 隰川州所理。漢蒲子縣地,隋為隰川縣。



 蒲漢縣。
 武德二年,置昌州,領蒲、仵城、常武、昌原四縣。貞觀元年,省昌州及昌原、仵城、常武三縣,以蒲屬隰州。



 大寧漢北屈縣地,隋為仵城。武德二年,置中州於隋大寧故城,因改名大寧。貞觀元年,廢中州及大義、白龍二縣,以大寧隸隰州。



 永和漢狐水聶縣,隋為永和。武德二年,移治於仙芝谷西,屬東和州,又分置樓山縣。貞觀元年,廢東和州及樓山縣,以永和隸隰州。



 石樓漢土軍縣,隋改為石樓。武德二年,於縣置西德州,領長壽、臨河、
 石樓三縣。貞觀元年,廢西德州,省長壽、臨河二縣,以石樓屬東和州。二年,又省東和州,以石樓來屬。



 溫泉隋新城縣。武德二年,分置溫泉縣,仍置北溫州,領溫泉、新城、高堂三縣,屬隰州總管府。貞觀元年,省北溫州及新城、高堂二縣,以溫泉來屬。



 汾州上隋西河郡。義旗初,依舊領隰城、介休、孝義、平遙四縣。其年,割介休、平遙二縣屬介休郡。武德元年,以介休郡為介州,西河郡為浩州。三年,改浩州為汾州,仍
 割並州之文水來屬。貞觀元年,省介州,以介休、平遙二縣來屬。文水還並州。十七年,以廢呂州之靈石來屬。天寶元年,改為西河郡。乾元元年,復為汾州。舊領縣四,戶三萬四千九,口十萬六千三百八十四。天寶領縣五,戶五萬九千四百五十,口三十二萬二百三十三。去京師一千二百六里,東都九百三十七里。



 西河漢美稷縣,隋為隰城縣。上元元年九月,改為西河縣。



 孝義漢中陽縣,後魏曰永安。貞觀元年,改為
 孝義。



 介休漢縣。武德元年,於縣置介州。貞觀元年,州廢,以介休、平遙屬汾州。



 平遙漢平陶縣。後魏廟諱,改「陶」為「遙」。武德屬介州。州廢來屬



 靈石隋分介休縣置,屬呂州。州廢來屬。



 慈州下元魏曰南汾州,隋改為耿州,又為文成郡。武德元年,改為汾州。五年,改為南汾州。八年,改為慈州,以郡近慈烏戍故也。舊領縣五,戶五千二百四十五,口二萬二千六百五十一。天寶,戶一萬一千六百一十六,口
 六萬二千四百八十六。在京師東北六百八十三里,去東都七百二十七里。



 吉昌隋縣



 文城元魏曰斤城縣,隋改為文城。顯慶三年,移斤城縣東北文城村置。



 昌寧漢臨汾縣地,後魏分置太平縣,又分太平置昌寧縣。



 呂香義寧元年,分仵城縣置平昌縣。貞觀元年,改為呂香,因舊鎮為名。上元三年,移治所於故平昌府南置,今縣是也。



 仵城後魏置縣,取鎮戍名也。



 潞州大都督府隋上黨郡。武德元年,改為潞州,領上黨、長子、屯留、潞城四縣。二年,置總管府,管潞、澤、沁、韓、蓋五州。四年,分上黨置壺關縣。貞觀元年,廢都督府。八年,置大都督府。十年,又改為都督府。貞觀十七年,廢韓州,以所管襄垣等五縣屬潞州。開元十七年,以玄宗歷職此州,置大都督府,管慈、儀、石、沁四州。天寶元年,改為上黨郡。乾元元年,依舊為潞州大都督府。舊領縣五,戶一萬八千六百九十,口八萬三千四百五十五。舊於襄垣
 置韓州,領縣五,戶七千一十七,口三萬二千九百三十六。天寶領縣十,戶六萬八千三百九十一,口三十八萬八千六百六十。在京師東北一千一百里,至東都四百八十七里。



 上黨漢壺關縣。隋分置上黨,州所治。



 壺關武德四年,分上黨置,治於高望堡。貞觀十七年,移治進流川。



 長子漢縣



 屯留隋舊。武德五年,自霍壁移於今所。



 潞城古邑。隋特置潞城縣。



 襄垣隋縣。武德元
 年,於縣置韓州,領襄垣、黎城、涉、銅鞮、武鄉五縣,又割並州之榆社來屬。三年,置甲水縣,仍以榆社屬榆州。六年,割沁州之銅鞮來屬。九年,省甲水縣。貞觀十七年,廢韓州,以襄垣等五縣隸潞州。



 黎城舊刈陵縣,隋改曰黎城州。



 涉漢縣。隋屬韓州。州廢來屬。



 銅鞮隋屬韓州。武德元年,屬沁州。三年,分置甲水縣。五年,移治<角亥>水堡。六年,移於今所,屬韓州。省甲水縣。韓州廢,屬潞州。



 武鄉漢垣縣,後魏曰沮城,移治於南亭川。
 改為鄉縣,屬韓州。州廢,屬潞州。則天加「武」字。神龍年,去「武」字,復為鄉縣。後又加「武」字。



 澤州上隋長平郡。武德元年,改為蓋州,領高平、丹川、陵川,又置蓋城四縣。又於濩澤縣置澤州,領濩澤、沁水、端氏三縣。三年,於今理置晉城縣。六年,廢建州,自高平移蓋州治之。八年,移澤州治端氏。九年,省丹川、蓋城。貞觀元年,廢蓋州,自端氏縣移澤州於今治。天寶元年,改澤州為高平郡。乾元元年,復為澤州。舊領縣六,戶一萬
 六百六十,口四萬六千七百三十二。天寶,戶二萬七千八百二十二,口二十五萬七千九十。在京師東北一千三十里,至東都六百六十七里。



 晉城漢高都縣,隋改為丹川。武德元年,移丹川於源澤水北,屬蓋州。二年,於古高都城置晉城縣,屬建州。六年,廢建州,縣屬蓋州。九年,省丹川縣。貞觀元年,廢蓋州,縣屬澤州



 端氏漢縣。武德八年,移澤州於此縣。貞觀元年,又移於晉城



 陵川漢泫氏縣,隋改陵川。武
 德初,屬蓋州。貞觀元年,隸澤州



 陽城隋濩澤縣。武德元年,於縣置澤州。八年,移州治於端氏。天寶元年,改為陽城



 沁水元魏置東永安縣,隋改為沁水,屬蓋州。州廢來屬



 高平漢泫氏縣地。武德元年,於縣置蓋州,領高平、丹川、陵川、蓋城四縣。貞觀元年,廢蓋州,來屬。



 沁州下隋上黨郡之沁源縣。義寧元年,置義寧郡,領沁源、銅鞮、綿上,仍分沁源置和川,凡四縣。武德元年,改為沁州。二年,分沁源置招遠縣。三年,省招遠縣。六年,以
 銅鞮屬韓州。天寶元年,改沁州為陽城郡。乾元元年,復為沁州。舊領縣三,戶三千九百五十六,口一萬六千一百七。天寶,戶六千三百八,口三萬四千九百六十三。在京師東北一千二十五里,去東都六百三十五里。



 沁源漢穀遠縣。州所治。後魏改為沁源



 和川義寧元年,分沁源置



 綿上隋分介休之南界,置綿上縣。



 遼州隋太原郡之遼山縣。武德三年,分並州之樂平、和順、平城、石艾四縣置遼州,治樂平。其年,置義興縣。六
 年,自樂平移於遼山,仍以石艾、樂平二縣屬受州,省義興縣,以廢榆州之榆社、平城二縣來屬。八年,改遼州為箕州。先天元年,又改為儀州。天寶元年,改為樂平郡。乾元元年,復為儀州。中和三年八月,復為遼州。舊領縣四,戶四千三百六十五,口八萬八千六百四十。天寶,戶九千八百八十二,口五萬四千五百八十。在京師東北一千四百五十九里,至東都七百九十七里。



 遼山漢垣縣地,魏改尞陽縣。隋改遼山縣,屬並州。武
 德三年,屬遼州。



 榆社晉武鄉縣。義寧元年,分置榆社縣。武德三年,於此置榆州,割並州平城來屬。仍置偃武縣。六年,廢榆州及偃武縣,以平城、榆社屬遼州



 和順漢沾縣地。隋為和順縣。武德初,屬並州,三年,改為遼州



 平城隋縣。武德初,屬並州。三年,改屬榆州,六年,改為遼州。



 北京太原府隋為太原郡。武德元年,改為並州總管,領晉陽、太原、榆次、太谷、祁、陽直、壽陽、盂、樂平、交城、石艾、
 文水、遼山、平城、烏河、榆社十六縣。其年,置清源縣,仍以榆社屬韓州。三年,廢總管。其年,置汾陽;仍以盂、壽陽二縣置受州,治盂縣;樂平、遼山、平城、石艾四縣置遼州,治樂平;太谷、祁二縣置太州,治太谷;仍以文水屬汾州。四年,又置總管,管並、介、受、遼、太、榆、汾七州。其年,改為上總管。五年,又改代、石二總管。其年,改上總管為大總管。六年,又改朔州總管,仍割汾州之文水來屬。其年,廢太州,以太谷、祁二縣來屬。七年,改為大都督府。其年,置羅陰
 縣,仍省陽直縣,改汾陽為陽曲縣,又以文水屬汾州。貞觀元年,省烏河、羅陰二縣,又以文水來屬。八年,以廢受州之壽陽、盂、樂平、石艾,又割順州之燕然,凡五縣來屬。督並、汾、箕、嵐四州。十四年,廢燕然縣。龍朔二年,進為大都督府。天授元年,置北都兼都督府。開元十一年,又置北都,改並州為太原府。天寶元年,改北都為北京。舊領縣十四,戶九萬七千八百七十四,口二十萬九百三十六。天寶領縣十三,戶十二萬八千九百五,口七十七萬
 八千二百七十八。在京師東北一千三百六十里,至東都八百八里。



 太原漢晉陽縣。隋文又移於州城內古晉陽城置,今州所治。



 晉陽隋新移於州內



 太谷隋縣。武德三年,置太州。六年,州廢,以太谷、祁屬並州



 文水隋縣。武德三年,屬汾州。六年,屬並州。七年,又屬汾州。貞觀初,還屬並州。天授元年,改為武興縣,以天后鄉里縣,與太原、晉陽並為京師。神龍元年,依舊為文水



 榆次
 漢縣



 盂隋縣。武德三年,置受州,領盂、壽陽二縣。六年,移受州於壽陽。貞觀八年,省受州,盂復屬並州



 清源隋於古梗陽城置清源縣,以水為名



 交城隋分晉陽縣置,取縣西北古交城為名。初治交山,天授元年,移治郤波村。先天二年,於故縣分置靈川縣,開元二年後省



 陽曲隋陽直縣。武德三年,分置汾陽縣。七年,省陽直縣,改汾陽為陽曲縣,仍移治陽直廢縣。其年,又分置羅陰縣。貞觀元年省。十七年,又省燕然並入。



 壽陽隋舊縣。武德三年,屬受州。六年,移受州於此,領壽陽、盂二縣。其年,又割遼州之樂平、石艾二縣來屬。貞觀八年,廢受州,以所管四縣隸並州



 廣陽漢上艾縣,後漢改為石艾縣。武德三年,屬遼州。六年,屬受州。八年,州廢,屬並州。天寶元年,改為廣陽。



 樂平隋縣。武德三年,於縣置遼州。六年,移遼州治於箕州,以樂平屬受州。州廢,縣來屬。



 祁漢縣,至隋不改。武德三年,屬太州,州廢來屬。



 代州中都督府隋為雁門郡。武德元年,置代州總管,管代、忻、蔚三州。代州領雁門、繁畤、崞、五臺四縣。五年,廢總管。六年,又置,管代、蔚、忻、朔四州。貞觀四年,又督靈州。六年,又督順州。十二年,省順州,以懷化縣來屬。今督代、忻、蔚、朔、靈五州。高宗廢懷化縣。證聖元年,置武延縣。天寶元年,改為雁門郡。依舊為都督府。乾元元年,復為代州。舊領縣五,戶九千二百五十九,口三萬六千二百三十四。天寶,戶二萬一千二百八十,口十萬三百五十。在京
 師東北一千五百五十里,去東都一千二百二十三里。



 雁門漢廣武縣,隋為雁門縣



 五臺漢慮縣,隋改為五臺



 繁畤漢縣



 崞漢縣。東魏置廓州,又廢



 唐林證聖元年,分五臺、崞縣置武延縣,唐隆元年,改唐林。



 蔚州隋雁門郡之靈丘縣。武德四年,平劉武周。六年,置蔚州,寄治並州陽典縣,仍置靈丘、飛狐二縣。七年,寄治代州繁畤縣。八年,又寄治忻州秀容之北恆州城。貞
 觀五年,移於今治。天寶元年,改為安邊郡。至德二年九月,改為興唐郡。乾元元年,置蔚州。舊領縣二,戶九百四十二,口三千七百四十八。天寶領縣三,戶五千五十二,口二萬九百五十八。在京師東北一千八百一十里,去東都一千六百四十里。



 靈丘隋縣。隋末陷賊,寄治陽曲。自此,隨州寄治。貞觀五年,移於今所



 飛狐隋縣,隋末陷賊,武德六年,復置,寄治於易州遂城縣。貞觀五年,移治於今所



 興唐
 隋安邊縣。至德二年,改為興唐。



 忻州隋樓煩郡之秀容縣。義旗初,置新興郡,領秀容一縣。武德元年,改為忻州。四年,又置定襄縣。天寶元年,改為定襄郡。乾元元年,復為忻州。舊領縣二,戶四千九百八十七,口一萬七千一百三十。天寶,戶一萬四千八百六,口八萬二千三十二。在京師東北一千三百八十里,去東都一千六十三里。



 秀容漢汾陽縣地,治郭下。隋朝自秀容故城移於此,
 因改為秀容縣



 定襄漢陽曲縣地。後漢末,移陽曲於太原界置,乃於陽曲古城置定襄縣。復廢。武德四年,分秀容縣復置。



 嵐州下隋樓煩郡之嵐城縣。武德四年,平劉武周,置東會州,領嵐城縣;又以北和州之太和縣來屬。其年,分嵐城置合會、豐潤二縣,仍自故郡城移嵐州於廢東會州,置嵐州。舊領岢嵐一縣,縣移舊嵐州。其年,又以北管州之靜樂縣來屬。七年,置臨津縣。九年,省合會、岢嵐、太
 和三縣。貞觀元年,改臨津為合河。三年,又置太和縣。八年,又省。天寶元年,復為樓煩郡。乾元元年,復為嵐州。舊領縣三,戶二千八百四十二,口一萬一千五百四十一。天寶領縣四,戶一萬六千七百四十八,口八萬四千六。在京師東北一千二百九十五里,去東都一千一百四十四里。



 宜芳隋嵐城縣。武德四年,改為宜芳,屬東會州。四年,分置豐潤、合會二縣。五年,省豐潤並入。六年,改屬嵐州。
 九年,省合會並入



 靜樂漢汾陽縣地,有隋汾陽宮。武德四年,置管州,領靜樂,又分置汾陽、六度二縣。五年,改管州為北管州。六年,省北管州及汾陽、六度二縣。以靜樂屬嵐州



 合河隋臨泉縣。武德四年,置臨津縣。貞觀元年,改為合河



 嵐谷舊岢嵐軍也,在宜芳縣北界。長安三年,分宜芳於岢嵐舊軍置嵐谷縣。神龍二年,廢縣置軍。開元十二年,復置縣。



 憲州下舊樓煩監牧也。先隸隴右節度使,至德後,屬
 內飛龍使。舊樓煩監牧,嵐州刺史兼領。貞元十五年,楊缽為監牧使,遂專領監司,不系州司。龍紀元年,特置憲州於樓煩監,仍置樓煩縣。郡城,開元四年王毛仲築。州新置,未記戶口帳籍。



 樓煩龍紀元年,於監西一里置



 玄池州東六十里置



 天池州西南五十里置。本置於孔河館,乾元後移於安明谷口道人堡下。



 石州隋離石郡。武德元年,改為石州。五年,置總管府,
 管石、北和、北管、東會、嵐、西定六州。貞觀二年,廢都督府。三年,復置都督。六年,又廢。天寶元年,改為昌化郡。乾元元年,復為石州。舊領縣五,戶三千七百五十八,口一萬七千四百二。天寶,戶一萬四千二百九十四,口六萬六千九百三十五。在京師東北一千二百九十一里,至東都一千二百二十八里。



 離石漢縣。周改為昌化郡,隋復為離石,州所治



 平夷後周析離石縣置



 定胡隋縣。武德三年,置西
 定州。貞觀二年廢,分置孟門縣。七年,廢孟門入定胡



 臨泉隋太和縣。武德三年,置北和州,改太和縣為臨泉縣。貞觀三年,省北和州,縣屬石州。方山隋縣。武德二年,置方州。三年,州廢,縣屬石州。



 朔州隋馬邑縣。武德四年,置朔州,領善陽、常寧二縣。其年,省常寧縣。天寶元年,改為馬邑郡。乾元元年,復改為朔州。舊領縣一,戶一千二百五十七,口四千九百一十三。天寶領縣二,戶五千四百九十三,口二萬四千五
 百三十三。在京師東北一千七百七十四里,至東都一千三百四十三里。



 善陽漢定襄地,有秦時馬邑城、武周塞。後魏置桑乾郡。隋為善陽縣



 馬邑秦漢舊名,久廢。開元五年,分善陽縣於大同軍城置。



 雲州隋馬邑郡之云內縣界恆安鎮也。武德四年,平劉武周。六年,置北恆州。七年,州廢,貞觀十四年,自朔州北定襄城,移雲州及定襄縣置於此。永淳元年,為賊所
 破,因廢,乃移百姓於朔州。開元二十年,復為雲州。天寶元年,改為雲中郡。乾元元年,復為雲州,領縣一,戶七十三,口五百六十一。在京師東北一千九百四十里,去東都一千六百四十二里。



 雲中隋雲內縣之恆安鎮。武德六年,置北恆州。貞觀十四年,自朔州北定襄城移雲州於此置,因為定襄縣。今治,即後魏所都平城也。永淳元年,為賊所破,因廢雲州及縣。開元二十年,與州復置。仍改定襄為雲中縣。



 單于都護府秦漢時雲中郡城也。唐龍朔三年,置雲中都護府。麟德元年,改為單于大都護府。東南至朔州三百五十七里。振武軍在城內置。天寶,戶二千一百,口一萬三千。在京師東北二千三百五十里,去東都二千里。



 金河與府同置。



 河北道



 懷州雄隋河內郡。武德二年,於濟源西南柏崖城置懷州,領大基、河陽、集城、長泉四縣。其年,於濟源立西濟
 州,於武德縣立北義州,修武縣東北故濁鹿城立陟州,置總管府,管懷、西濟、北義、陟四州。三年,懷州又置太行、忠義、紫陵、谷只、溫五縣。四年,移懷州於今治野王城。其年,又於溫縣置平州,以溫縣屬之。又省穀只、太行、忠義、紫陵四縣。後省平州,仍於隋河陽宮置盟州,領河陽、集城、溫三縣。又省西濟、北義、陟三州入懷州。又於獲嘉縣置殷州。其懷州總管,管懷、盟、殷三州。懷州領河內、武德、軹、濟源五縣。八年,廢盟州,省集城入河陽縣,以河陽、溫
 二縣來屬。貞觀元年,罷都督府,以廢殷州修武、獲嘉、武陟,廢邵州之王屋四縣來屬。仍省懷、軹二縣。顯慶二年,割河陽、溫、濟源、王屋四縣屬洛州。天授元年,改為河內郡。乾元元年,復為懷州。舊領縣九:河內、武德、修武、獲嘉、武陟、溫、河陽、濟源、王屋。戶三萬九十,口十二萬六千九百一十六。天寶領縣五,戶五萬五千三百四十九,口三十一萬八千一百二十六。在京師東九百六十九里,至東都一百四十里。



 河內漢野王縣,隋為河內縣。武德四年,省太行、忠義、紫陵三縣並入



 武德隋為安昌縣。武德三年,改為武德



 武陟、漢懷縣地,故城在今縣西



 修武漢山陽縣地。修武,古名也,隋因之。武德二年,李原德以縣東北濁鹿城歸順,因置陟州及修武縣。四年,賊平,改為武陟,廢陟州,以修武屬殷州,仍移縣治於隋故修武城。貞觀元年,省殷州,修武屬懷州



 獲嘉漢縣名。武德四年,於縣置殷州,領獲嘉、武德、武陟、修武、新鄉、共城
 五縣。貞觀元年,省殷州,以獲嘉、武陟、修武屬懷州,新鄉、共城屬衛州。



 衛州望隋汲郡,本治衛縣。武德元年,改為衛州。二年,陷竇建德。四年,賊平,仍舊領衛、清淇、湯陰三縣。其年,廢義州,以汲縣來屬。六年,以湯陰屬相州。貞觀元年,州移治於汲縣,又廢殷州,以共城、新鄉、博望三縣來屬。六年,廢博望縣。十七年,廢清淇縣。其年,又以廢黎州之黎陽縣來屬。天寶元年,改為汲郡。乾元元年,復為衛州。舊領縣
 五,戶一萬一千九百三,口四萬三千六百八十二。天寶,戶四萬八千五十六,口二十八萬四千六百三十。在京師東一千二百二十二里,去東都三百九十里。



 汲漢縣,隋因之。武德元年,置義州,領汲縣。四年,廢義州,縣屬衛州。貞觀元年,衛州自衛縣徙治所於汲縣



 新鄉隋割汲、獲嘉二縣地,於古新樂城置新鄉縣。武德初,屬義州。州廢,來屬殷州。州廢,屬衛州



 衛漢朝歌縣。紂所都朝歌城,在今縣西。隋大業二年,改為衛縣,
 仍置汲郡於縣治。貞觀初,移於汲縣。初屬義州。州廢,屬衛州。十七年,省清淇縣入衛縣。長安三年,又置清淇縣。神龍元年,又省入衛縣



 共城漢共縣,隋因之。武德元年,置共州,領共城、凡城二縣。四年,廢共州,省凡城入共城縣。初屬殷州。貞觀初,來屬



 黎陽隋黎陽縣。武德二年,置黎州總管府,管殷、衛、洹、澶四州。尋陷賊。四年,平竇建德,復置黎州,領臨河、內黃、湯陰、觀城、頓丘、繁陽、澶水八縣。其年,以澶水、觀城、頓丘三縣置澶州,又以湯
 陰屬相州。貞觀元年,省繁陽,又以澶水來屬。十七年,廢黎州及澶水縣,以黎陽屬衛州,內黃、臨河屬相州。



 相州漢魏郡也。後魏道武改為相州,隋為魏郡。武德元年,置相州總官府,領安陽、鄴、林慮、零泉、相、臨漳、洹水、堯城八縣。二年,割林慮置巖州。四年,廢總管府,仍省零泉縣。五年,廢巖州,以林慮來屬,仍省相縣。六年,割衛州之湯源來屬。其年,復置總管府,管磁、洺、黎、衛、邢六州。九年,廢都督府。貞觀元年,改湯源為湯陰,以廢磁州之淦
 陽、成安二縣來屬。十年,復置都督,管相、衛、黎、魏、洺、邢、貝七州。十六年,罷都督府。十七年,以廢黎州之內黃、臨河來屬。天寶元年,改為鄴郡。乾元元年,復為相州。舊領縣九,戶一萬一千四百九十,口七萬四千七百六十六。天寶縣十一,戶十萬一千一百四十二,口五十九萬一百九十六。在京師東北一千四百二十一里,至東都六百六里。



 安陽漢侯國,故城在湯陰東。曹魏時,廢安陽,並入鄴。
 後周移鄴,置縣於安陽故城,仍為鄴縣。隋又改為安陽縣,州所治。漢魏郡城,在縣西北七里



 鄴漢縣,屬魏郡。後魏於此置相州,東魏改為司州。周平齊,復為相州。周大象二年,隋文輔政,相州刺史尉遲迥舉兵不順,楊堅令韋孝寬討迥,平之,乃焚燒鄴城,徙其居人,南遷四十五里。以安陽城為相州理所,仍為鄴縣。煬帝初,於鄴故都大慈寺置鄴縣。貞觀八年,始築今治所小城



 湯陰漢蕩陰縣也,並入安陽。武德四年,分安陽置湯源
 縣,屬衛州。六年,改屬相州。貞觀元年,改為湯陰



 林慮漢隆慮縣。武德三年,置巖州,領林慮一縣。五年,巖州廢,縣屬相州。



 堯城隋縣



 洹水漢長樂縣地,屬魏郡。周建德六年,分臨漳東北界置洹水縣。



 臨漳後周建德六年,分鄴縣置。成安漢斥丘縣,屬魏郡。後廢,北齊復置,改為成安。



 內黃漢縣名。舊屬黎州,貞觀十七年,改屬相州。



 臨河隋分黎陽縣置。貞觀十七年,改屬相州,廢澶水縣並入。



 魏州雄漢魏郡元城縣之地。後魏天平二年,分館陶西界,於今州西北三十里古趙城置貴鄉縣。後周建德七年,以趙城卑濕,東南移三十里,就孔思集寺為貴鄉縣。大象二年,於縣置魏州。隋改名武陽郡。武德四年,平竇建德,復為魏州。又分置漳陰縣,領貴鄉、昌樂、元城、莘、武陽、臨黃、觀城、頓丘、繁水、魏、冠氏、館陶、漳陰十三縣。其年,割頓丘、觀城二縣置澶州,又割莘、臨黃、武陽三縣置莘州,又割冠氏、館陶置毛州。魏州置總管府,管魏、黎、
 澶、莘、毛五州。魏州領貴鄉、昌樂、繁水、漳陰、元城、魏六縣。貞觀元年,罷都督府,仍省漳陰縣。其年,廢莘、毛、澶三州,盡以所領縣屬魏州。十七年,省元城、武陽、觀城三縣。十八年,省繁水縣。龍朔二年,改為冀州大都督府,以冀王為都督,管冀、貝、德、相、棣、滄、魏七州。咸亨三年,依舊為魏州,罷都督府。永昌元年,置武聖縣。聖歷二年,又置元城縣。天寶元年,改為魏郡。乾元元年,復為魏州。舊領縣十三,戶三萬四百四十,口十三萬六千六百一十二。天寶領
 縣十,戶十五萬一千五百九十六,口一百一十萬九千八百七十。在京師東北一千五百九十里,去東都七百五十里。



 貴鄉後魏分館陶西界,置貴鄉縣於趙城。周建德七年,自趙城東南移三十里,以孔思集寺為縣廨。大象二年,於縣置魏州。武德八年,移縣入羅城內。開元二十八年,刺史盧暉移於羅城西百步。大歷四年,又移於河南岸置



 元城隋縣,治古殷城。貞觀十七年,並入貴鄉。
 聖歷二年,又分貴鄉、莘縣置,治王莽城。開元十三年,移治州郭下。古殷城,在朝城東北十二里



 魏漢舊縣,在今縣南。天寶三年,移於今所



 館陶漢縣,隋因之。武德五年,置毛州,割魏州之館陶、冠氏、堂邑,貝州之臨清、清水。又分置沙丘縣。貞觀元年,廢毛州,省沙丘、清水二縣,以堂邑屬博州,臨清屬貝州,館陶、冠氏屬魏州



 冠氏春秋邑名。隋分館陶縣東界置。武德四年,屬毛州。州廢來屬



 莘漢陽平縣地,隋置新州。武德五年,
 改為莘州,領莘、臨黃、武陽、武水四縣。貞觀元年,廢莘州,以莘、臨黃、武陽屬魏州,武水屬博州



 臨黃漢觀縣地,隋為臨黃縣。武德四年,屬莘州。州廢來屬



 朝城隋武陽縣。貞觀十七年,廢武陽入臨黃、莘二縣。開元七年復置,改為朝城



 昌樂晉置,屬陽平郡。後魏置昌州,今縣西古城是也。隋廢昌樂縣入繁水。武德五年復置,隸魏州。今治所,武德六年築也。



 澶州漢頓丘縣,屬東郡。今縣北古陰安城是也。武德
 四年,分魏州之頓丘、觀城置澶州,領頓丘、觀城,又特置澶水縣。貞觀元年,廢澶州,以澶水屬黎州,頓丘、觀城屬魏州。大歷七年正月敕,又於頓丘縣置澶州,領頓丘、清豐、觀城、臨黃四縣。州新置,元未計戶口帳籍。在京師東北一千四百八十五里,至東都六百八十五里。



 頓丘漢縣,屬東郡,後移治所於陰安城,隋屬魏郡,今縣地北陰安城是也



 清豐大歷七年,割頓丘、昌樂二縣界四鄉置。以縣界有孝子張清豐門闕,魏州田承
 嗣請為縣名



 觀城隋縣。唐初,屬澶州。州廢,亦省觀城。大歷七年,割昌樂、臨黃二縣四鄉,置縣於舊觀城店



 臨黃隋舊縣。武德四年,屬莘州。州廢,屬魏州。大歷七年,置澶州,割之來屬。



 博州上隋武陽郡之聊城縣。武德四年,平竇建德,置博州,領聊城、武水、堂邑、茌平,仍置莘亭、靈泉、清平、博平、高唐凡九縣。五年,省莘亭、靈泉二縣。貞觀元年,省茌平縣。天寶元年,改為博平郡。乾元元年,復為博州。舊領縣六,戶
 七千立方百八十二,口三萬七千三百九十四。天寶,戶五萬二千六百三十一,口四十萬八千二百五十二。在京師東北一千七百一里。至東都九百四十七里。



 聊城漢縣。治郭下。武德四年,分置茌平縣。貞觀元年,省入聊城



 博平漢縣。隋因之。武德四年,分置靈縣。五年省,並入博平。貞觀十七年,省博平入聊城。天授二年,析聊城復置



 武水漢陽平縣地,屬東郡。隋改為清邑,又分清邑置武水縣。武德四年,屬莘州。貞觀元年,
 屬博州



 清平漢貝丘縣。隋改為清平,屬博州



 堂邑漢縣。後魏廢。隋分清陽縣復置。初屬毛州,州廢,屬博州



 高唐隋縣。長壽二年,改為崇武。神龍元年,復為高唐。



 貝州隋為清河郡。武德四年,平竇建德,置貝州,領清河、武城、漳南、歷亭、清陽、鄃、夏津七縣。六年,移治所於歷亭。八年,還於舊治。九年,以廢宗州之宗城、經城來屬,又以廢毛州之臨清來屬。天寶元年,改為清河郡。乾元元
 年,復為貝州。舊領縣九,戶一萬七千七百一十九,口九萬七十九。天寶,戶十一萬一十五,口八十三萬四千七大量五十七。在京師東北一千七百八十二里,至東都九百九十三里。



 清陽武德四年,分置夏津縣。九年,復省。舊治甘陵城。永昌元年,移治於孔橋。開元二十三年,移就州治



 清河漢縣,後漢恆帝改為甘陵,後省。隋復分置清河縣,在郭下



 武城漢曰東武城。舊治古夏城。調露元年,
 移於今治



 宗城隋舊。武德四年,置宗州,領宗城、府城、南宮、斌強四縣。九年,廢宗州及府城、斌強二縣,以經城、宗城屬貝州,南宮屬冀州



 臨清漢清泉縣,後魏改為臨清。武德四年,屬毛州。州廢,屬貝州



 經城漢縣。武德四年,屬宗州。州廢來屬



 漳南漢東陽縣,後魏省。隋分棗強、清平二縣地,復置於古東陽城,仍改為漳南縣



 歷亭漢東陽地。隋分鄃縣置歷亭縣



 夏津舊鄃縣。天寶元年,改為夏津。



 洺州望隋武安郡。武德元年,改為洺州,領永年、洺水、平恩、清漳四縣。二年,陷竇建德。四年,建德平,立山東道大行臺,又立曲周、雞澤二縣。五年,罷行臺,置洺州大總管府,管洺、衛、巖、相、磁、邢、趙八州。六年,罷總管府。以磁州之武安、臨洺、肥鄉三縣來屬。貞觀元年,又以廢磁州之邯鄲來屬。天寶元年,改為廣平郡。乾元元年,復為洺州。永泰之後,復以武安、邯鄲屬磁州。會昌元年,省清漳、洺水二縣入肥鄉、平恩、曲周等縣。舊領縣七,戶二萬二千
 九百三十三,口十萬一千三十。天寶領縣十,戶九萬一千六百六十六,口六十八萬三千二百八十。省清漳、洺水。今領縣六。在京師東北一千五百八十五里,至東都八百五十七里。



 永年州所治。本漢曲梁縣,屬廣平郡。改廣平為永年



 平恩漢縣。隋自斥漳城移於平恩故城置



 臨洺漢易陽縣,隋改為臨洺。武德元年,置紫州,領臨洺、武安、肥鄉、邯鄲等縣。四年,罷紫州,臨洺屬磁州。五年,改屬
 洺州



 雞澤漢廣平縣地。武德四年,置雞澤縣



 肥鄉漢邯溝縣地。曹魏立肥鄉縣,屬廣平郡。會昌三年,省清漳縣入



 曲周隋廢縣。武德四年,復置。會昌三年,省洺水縣入。



 磁州隋魏郡之淦陽縣。武德元年,置磁州,領淦陽、臨水、成安三縣。四年,割洺州之臨洺、武安、邯鄲、肥鄉來屬。六年,置磁州總管府,領磁、邢、洺、黎、相、衛六州。其年,廢總管府。以臨洺、武安、肥鄉三縣屬洺州,磁州領滏陽、成安、邯鄲
 三縣。貞觀元年,廢磁州,滏陽、成安屬相州,以邯鄲屬洺州。永泰元年六月,昭義節度使薛嵩請於淦陽復置磁州,領滏陽、武安、昭義、邯鄲四縣。州新置,未計戶口帳籍。在京師東北一千四百八十五里,至東都六百六十五里。



 滏陽漢武安縣地。隋置滏陽縣,州所治



 邯鄲漢縣,屬廣平郡。隋屬磁州。州廢,屬洺州。永泰初,復置磁州,來屬。



 武安漢縣。隋復置,隸磁州。



 昭義永泰元
 年,廉察使薛嵩特置於滏口之右故臨水縣城。



 邢州上隋襄國郡。武德元年,改為邢州總管府,管邢、溫、和、封、蓬、東龍六州。邢州領龍崗、堯山、內丘三縣。四年,平竇建德,罷總管府。割內丘屬趙州,仍省和、溫、封三州,以其所領南和、沙河、平縣三縣來屬。又立任縣。五年,割趙州之內丘、柏仁來屬。天寶元年,改為鉅鹿郡。乾元元年,復為邢州。舊領縣九,戶二萬一千九百八十五,,口九萬九百六十。天寶,戶七萬一百八十九,口三十八萬二
 千七百九十八。在京師東北一千六百五十五里,至東都八百五十七里。



 龍岡漢襄國縣,隋改為龍岡,州所治也



 沙河隋分龍岡縣置。武德元年,置溫州。四年,州廢,屬邢州



 南和漢縣,後周置南和郡,隋廢州為縣。武德元年,置和州。四年州廢,縣屬邢州



 鉅鹿隋於漢南涘故城置鉅鹿縣。武德元年,置起州並白起縣。四年,廢起州,鉅鹿屬趙州。仍省白起,並入鉅鹿。貞觀元年,屬邢州。舊治東
 府亭城。嗣聖元年,移於今所。



 平鄉漢鉅鹿郡,故郡城在今縣北十一里。古鉅鹿城,即今治也。隋改平鄉縣。



 任漢南地。晉置任縣,後廢。武德四年,復置。舊治苑鄉城。



 堯山漢柏仁縣,至隋不改。武德元年,置東龍州,領柏仁縣。四年,平竇建德,縣屬趙州。貞觀初,屬邢州。天寶元年,改為堯山。



 內丘漢中丘縣。隋改為內丘縣,屬趙州。貞觀初,還屬邢州。



 趙州漢平棘縣,故城在今縣南。後魏於昭慶縣置殷
 州,齊改為趙州。隋廢,尋復置趙郡於平棘縣。武德元年,張志昂以郡歸國,改為趙州,領平棘、高邑、贊皇、元氏、廮陶、欒城、大陸、柏鄉、房子、禋城、鼓城十二縣,其年,以禋城屬廉州,以鼓城屬深州。四年,改大陸為象城。天寶元年,改為趙郡。乾元元年,復為趙州。舊領縣九,戶二萬一千四百二十七,口八萬五千九百九十二。天寶,戶六萬三千四百五十四,口三十九萬五千二百三十八。去京師東北一千八百四十三里,至東都一千三十三里。



 平棘漢平棘縣,屬常山郡。隋自象城移趙州治所於縣置。



 寧晉漢楊氏縣,屬鉅鹿郡。今治即楊氏城也。後改為廮陶,元魏改為癭遙,隋復為陶。天寶元年,改為寧晉。



 昭慶漢廣阿縣,屬鉅鹿郡。後魏置殷州,北齊改為趙州。隋改廣阿為大陸。武德四年,改為象城。天寶元年,改為昭慶,以有建初、啟運二陵故也。



 柏鄉漢縣,屬鉅鹿郡,故城在今縣西南十七里。後廢。隋於今治彭水之陽,復置。



 高邑漢鄗縣,屬常山郡。世祖更名
 高邑,晉代不改。



 臨城漢房子縣,屬常山郡。天寶元年,改為臨城。



 贊皇古無其名,隋置,取贊皇山為名。



 元氏漢常山郡所治,故城在今縣西。



 鎮州秦東垣縣。漢高改名真定,置恆山郡,又為真定國。歷代為常山郡。治元氏,後魏道武登常山郡,北望安樂壘美之,遂移郡治於安樂城,今州城是也。周、隋改為恆州,後廢。義旗初,復置恆州,領真定、石邑、行唐、九門、滋陽五縣,州治石邑。武德元年,陷竇建德。四年,賊平,徙治
 所於真定,省滋陽縣,又割廉州之禋城來屬。天寶元年,改為常山郡。乾元元年,復為恆州。興元元年,升為都督府。元和十五年,改為鎮州。舊領縣六,戶二萬六千一百一十三,口五萬四千五百四十三。天寶領縣九,戶五萬四千六百三十三,口三十四萬二千二百三十四。今領縣十一。在京師東北一千七百六十里,至東都一千一百三十六里。



 真定隋屬高陽郡。武德四年,自石邑移恆州於縣為
 治所。載初元年,改為中山縣。神龍元年,復為真定縣。



 禋城漢縣。唐初,置鉅鹿郡,領禋城、桓肆、新豐、宜安四縣。武德元年,改為廉州。其年,陷竇建德。四年,賊平,復置廉州,領禋城、鼓城、毋極四縣。省桓肆、、新豐、宜安,並入禋城。貞觀元年,廢廉州,以鹿城屬深州,鼓城、毋極屬定州,禋城屬恆州。



 石邑漢縣,屬常山郡。



 九門漢縣,屬常山郡。至隋不改。國初置九門郡,領九門、新市、信義三縣。武德元年,改為觀州。五年,州廢,省信義、新市二縣。
 以九門隸恆州。



 靈壽漢縣,屬常山郡。義寧元年,置燕州。武德四年,州廢,縣屬井州。七年州廢,屬恆州。



 行唐漢南行唐縣,屬常山郡。武德四年,置王城縣,屬常山郡。武德五年,省滋陽縣並入。長壽二年,改為章武。神龍元年,復為行唐。



 井陘漢縣,屬常山郡。義寧元年,置井陘郡,並葦澤縣。武德元年,改為井州。四年,又以廢岳州之房山、蒲吾二縣,恆州之鹿泉來屬。五年,又以恆州之靈壽來屬。貞觀元年,廢蒲吾、葦澤二縣入井陘。十
 七年,廢井州,以井陘等三縣屬恆州。



 獲鹿漢石邑縣地。隋置鹿泉縣,屬井州。貞觀十七年,來屬。至德元年,改為獲鹿。



 平山漢蒲吾縣,屬常山郡。隋改為房山縣。義寧元年,置房山郡。武德元年,置岳州,領房山一縣。四年,廢岳州,房山屬恆州。至德元年,改為平山縣,仍以恆州為平山郡。



 鼓城漢臨平、下曲陽兩縣之地,屬鉅鹿郡。隋分禋城於下曲陽故城東五里置昔陽縣,尋改為鼓城。武德四年,屬廉州。州廢,屬定州。大歷三年,割
 屬恆州。



 欒城漢關縣,屬常山郡。後魏於關縣古城置欒城縣,屬趙州。大歷三年,割屬恆州。



 冀州上隋信都郡。武德四年,改為冀州,領信都、衡水、武邑、棗強、南宮、堂陽、下博、武強八縣。六年,置總管府,移治所於下博,管冀、貝、深、宗四州。貞觀元年,廢都督府,移州治於信都。又以下博、武強二縣屬深州。十七年,以廢深州之下博、武強、鹿城,廢觀州之阜城來屬。龍朔二年,改為魏州都督府。咸亨三年,復舊。先天二年,割下博、武
 強、鹿城三縣屬深州。開元二年,復以下博、武強還冀州。天寶元年,改為信都。乾元元年,復為冀州。舊領縣六:信都、南宮、堂陽、棗強、武邑、衡水。戶一萬六千二十三,口七萬二千七百三十三。天寶領縣九,戶一十萬三千八百八十五,口八十三萬五百二十。在京師東北一千九百七十八里,至東都一千一百里。



 信都漢信都國城,今州所治也。後漢改為樂成國,又改安平國。魏、晉後為冀州所治



 南宮漢縣,屬信都
 國,至隋不改。武德四年,屬宗州。貞觀元年,屬冀州



 堂陽漢縣,屬鉅鹿郡。隋舊屬冀州



 棗強漢縣,屬清河郡。隋舊也



 武邑漢縣,屬信都國。隋舊。武德四年,分置昌亭縣。貞觀初省



 衡水古無此名,隋開皇十七年,河北大使郎蔚之分信都北界、武邑西界、下博南界、置衡水縣,特築此城



 阜城漢縣,屬渤海郡。隋屬冀州。故城在今縣東二十里,今城隋築



 蓚漢縣,屬渤海郡。隋舊隸觀州。州廢,屬德州。故城在今縣南十里。
 貞觀元年,分置觀津縣,尋省。永泰後,屬冀州。



 深州武德四年,平竇建德,於河間郡之饒陽縣置深州,領安平、饒陽、蕪蔞三縣。初治安平,其年,移治饒陽。貞觀元年,割故廉州之鹿城,冀州之武強、下博來屬。省蕪蔞縣。十七年,廢深州,以饒陽屬瀛州,安平屬定州,鹿城、下博、武強屬冀州。先天二年,復割饒陽、安平、鹿城置深州,仍分置陸澤縣。天寶元年,改深州為饒陽郡。乾元元年,復為深州。舊領縣五,戶二萬一百五十六,口八萬七
 千。天寶,縣四,戶萬八千八百二十五,口三十四萬六千四百七十二。在京師東北二千一十三里,至東都一千二百五十里。



 陸澤、先天二年,分饒陽、鹿城界置陸澤縣於古鄡城。鄡,漢縣,屬鉅鹿郡



 饒陽漢縣,屬涿郡。武德四年,分置蕪蔞縣,貞觀元年省。十七年,割屬瀛州。先天二年,遷深州。武德初,為深州所治



 束鹿漢安定侯國,今縣西七里故城是也。周、齊為安定縣,隋改為鹿城。唐至德
 元年,改為束鹿



 下博漢縣,屬信都國。隋舊。武德四年,屬冀州。貞觀元年,改屬深州。十七年,屬冀州。先天二年,還深州



 安平漢縣,屬涿郡。武德初,置深州,以縣屬。十七年,州廢,屬定州。先天二年,來屬



 武強漢武隧縣,屬河間國。晉改為武強。武德四年,屬冀州。貞觀元年,屬深州



 博野漢蠡吾縣,屬涿郡。後漢分置博陵縣,後魏改為博野。武德五年,置蠡州,領博野、清苑,割定州之義豐三縣。八年,州廢,三縣各還本屬。九年,復立
 蠡州,領博野、清苑二縣。貞觀元年,廢蠡州,博野、清苑屬瀛州。永泰中,屬深州



 樂壽漢樂成縣,屬河間國。城在今縣東南十六里。後魏移縣東北,近古樂壽亭,因改為樂壽。隋屬河間郡。永泰中,割屬深州。



 滄州上漢渤海郡,隋因之。武德元年,改為滄州,領清池、饒安、無棣三縣,治清池。其年,移治饒安。四年,平竇建德,分饒安置鬲津縣。五年,以清池屬東鹽州。六年,以觀州胡蘇縣來屬,州仍徙治之。其年,又省棣州,以滴河、厭
 次、陽信、樂陵四縣來屬。貞觀元年,以瀛州之景城,廢景州之長蘆、南皮、魯城三縣,廢東鹽州之鹽山、清池二縣,並來屬。又以滴河、厭次二縣屬德州,以胡蘇屬觀州,仍移治於清池。又省鬲津入樂陵,省無棣入陽信。八年,復置無棣縣。十七年,以廢觀州之弓高、東光、胡蘇來屬。割陽信屬棣州。天寶元年,改為景城郡。乾元元年,復為滄州。舊領縣十,戶二萬五十二,口九萬五千七百九十六。天寶領縣十一,戶十二萬四千二十四,口八十二萬五
 千七百五。在京師東北二千二百一十八里,去東都一千三百八十二里。



 清池漢浮陽縣,渤海郡所治。隋改為清池縣,治郭下。武德四年,屬景州。五年,改屬東鹽州。貞觀元年,改屬滄州



 鹽山漢高城,古縣在南。隋改為鹽山。武德四年,置東鹽州,領縣一。五年,又割景州之清池來屬,仍置浮水縣。貞觀元年,省東鹽州及浮水縣,以清池屬滄州



 南皮漢縣,屬渤海郡。至隋不改。武德四年,屬景州。貞
 觀元年,改屬滄州。



 長蘆漢參戶縣,屬渤海郡。後周改為長蘆。武德四年,割滄州之清池、南皮二縣,瀛州之魯城、平舒、長蘆三縣,於此置景州。其年,陷劉黑闥。五年,賊平,置景州總管府,管滄、瀛、東鹽、景四州。又分清池縣屬東鹽州。貞觀元年,廢景州,以平舒屬瀛州,南皮、魯城、長蘆三縣屬滄州。舊治永濟河西,開元十六年,移於今治。



 樂陵漢舊縣,屬平原郡。隋不改。武德四年,屬棣州。六年,省棣州,以縣屬滄州。



 饒安漢千童縣,屬渤海
 郡。後漢改為饒安,隋因之。武德元年,移治故千童城,仍移州治於此。六年,州移治胡蘇。貞觀十二年,移縣治故浮水城。



 無棣漢陽信縣,屬渤海郡。改為無棣。貞觀元年,並入陽信。八年,復置。大和二年,屬棣州,又復還滄州。



 臨津漢東光縣地。隋於故胡蘇亭置胡蘇縣。武德四年,屬觀州。貞觀十七年,屬滄州。天寶元年,改為臨津。



 乾符隋魯城縣。武德四年,屬景州。貞觀元年,改屬滄州。乾符年,改為乾符。



 景州漢鬲縣地,屬平原郡。隋置弓高縣,屬渤海郡。武德四年,於縣置觀州,領弓高、蓚、阜城、東光、安陵、胡蘇、觀津七縣。六年,以胡蘇屬滄州。貞觀元年,省觀津縣,復以胡蘇來屬。十七年,廢觀州,以東光、胡蘇屬滄州,蓚縣、安陵屬德州,阜城屬冀州。貞觀二年,又於弓高縣置景州,又以弓高、東光、胡蘇來屬。長慶元年,廢景州,四縣亦還本屬。二年,復於弓高置景州。大和四年廢,縣屬滄州。景福元年,復於弓高置景州,管東光、安陵三縣。天祐五年,
 移州治於東光縣。領縣六,戶一萬一千三,口五萬七千五百三十二。在京師東北二千九百里,至東都一千三百里。



 弓高漢鬲縣,屬平原郡。隋置弓高縣,後於縣治置觀州、景州。興替不常,事在《州說》中



 東光漢縣,屬渤海郡。歷代不改



 安陵隋宣府鎮。武德四年,置安陵縣,屬觀州。貞觀十七年,廢觀州,改屬德州。永徽二年,移治白社橋。景福元年,改屬景州。



 德州漢平原郡。隋置德州,又為平原郡。武德四年,平竇建德後,置德州,領安德、般、平原、長河、將陵、平昌六縣。其年,置總管府,管博、德、棣、觀四州。貞觀元年,廢都督府,割滄州之滴河、厭次來屬。十七年,廢般縣,以滴河、厭次二縣屬棣州。又以廢觀州之蓚縣、安陵來屬。天寶元年,改為平原郡。乾元元年,復為德州。舊領縣八,戶一萬一百三十五,口五萬二千一百四十一。天寶領縣七,戶八萬三千三百一十一,口六十五萬九千八百五十五。至
 京師一千九百八十二里,去東都一千一百三十八里。



 安德漢縣,屬平原郡。今州治,至隋不改



 平原漢舊平源郡所治,故城在今縣西南二十五里。今縣治城,北齊所築



 長河漢廣川縣,屬信都國,後廢。隋於舊廣川縣東八十里置新縣,今治是也。尋改為長河縣,為水所壞。元和四年十月,移就白橋,於水濟河西岸置縣,東去故城十三里。十年,又置河東小胡城



 將陵漢安德縣。隋分安德於將陵故城置此縣



 平昌漢縣,
 屬平原郡。故城在今縣東三十里。大和二年,割屬齊州,又還德州。



 定州上後漢中山國。後魏置安州,尋改為定州。隋改博陵郡,又復為高陽郡。武德四年,平竇建德,復置定州,領安喜、義豐、北平、深澤、毋極、唐昌、新樂、恆陽、唐、望都等十縣。其年,置總管府,領定、恆、井、滿、廉五州。六年,昇為大總管府,管定、洺、相、磁、黎、冀、深、蠡、滄、瀛、魏、貝、景、博、趙、宗、觀、廉、井、邢、欒、德、衛、滿、
 幽、易、燕、檀、平、營等三十二州。七年,改為都督府,管定、恆、滿、井、趙、廉、欒、蠡等八州。貞觀元年,以廢廉州之鼓城來屬。五年,廢都督府。十七年,以廢深州之安平來屬。先天二年,以安平還深州。天寶元年,改為博陵郡。乾元元年,復為定州。大歷三年,以鼓城隸恆州,曲陽隸洹州。九年,廢洹州,曲陽復來屬。貞觀十三年,復為大都督府,十四年廢,依舊為上州。舊領縣十一,戶二萬五千六百三十七,口八萬六千八百六十九。天寶,戶七萬八千九十,口
 四十九萬六千六百七十六。在京師東北二千九百六里,至東都一千二百里。



 安喜漢盧奴縣,屬中山國。慕容垂改為不連,北齊改為安喜,隋改為鮮虞縣。武德四年,復為安喜,州所治也



 義豐漢安國縣,屬中山國。隋自皇阜城移於鄭德堡置,今縣治。後仍改為義豐。萬歲通天二年,契丹攻之不下,則天改為立節縣。神龍中,復舊名



 北平漢縣,屬中山國。萬歲通天二年,契丹攻之不下,乃改為徇忠縣。
 神龍元年,復舊名



 望都武德四年,分安喜、北平二縣置。初治安險故城,貞觀八年,移於今治。



 安險漢縣,屬中山國。



 曲陽漢上曲陽縣,屬常山郡。隋改為恆陽。大歷三年,屬洹州。九年,復來屬。元和十五年,改為曲陽。



 陘邑漢苦陘縣,屬中山國。章帝改為漢昌,曹魏改為魏昌,隋改為隋昌。武德四年,改為唐昌。天寶元年,改為陘邑。



 唐漢縣,屬中山國。舊治古公城,聖歷元年,移於今所。新樂古鮮虞子國。漢新市縣,屬中
 山郡。隋改為新樂。



 祁州中景福二年,定州節度使王處存奏請於本部無極縣置祁州。州新置,未計戶口帳籍。在京師東北二千二百一十里,至東都一千三百二十里。



 無極漢縣,屬中山國。「無」本作「毋」字。,武德四年,屬廉州。貞觀元年,屬定州。萬歲通天二年,改「毋」字為「無」



 深澤漢縣,屬中山國。至隋不改。屬定州。隋徙治滹沱北,本縣治也,隋末陷城。武德四年,復立縣。景福二年,割屬祁
 州。



 易州中隋上谷郡。武德四年,討平竇建德,改為易州,領易、淶水、永樂、遂城、乃五縣。五年,割乃縣置北義州。州廢,以乃來屬。開元二十三年,分置五回、樓亭、板城三縣。天寶元年,改為上谷郡,復隋舊名。乾元元年,復為易州。舊領縣五,戶一萬二千八百二十,口六萬三千四百五十七。天寶領縣八,戶四萬四千二百三十,口二十五萬八千七百七十九。今領縣六。在京師東北二千三百三
 十四里,至東都一千四百六十三里。



 易漢故安縣,屬涿郡。隋為易縣



 容城漢縣,屬涿郡。改為乃縣。武德五年,置北義州,領乃,又割幽州之固安、歸義屬之。貞觀元年,廢北義州,三縣各還本屬,聖歷二年,契丹入寇,固守得全,因改名全忠縣。天寶元年,改為容城



 遂城漢北新城縣,屬中山國。後魏改為新昌,隋末為遂城



 淶水漢乃縣,屬涿郡。隋屬上谷郡



 滿城漢北平縣地,後魏置永樂縣,隋不改。天寶元
 年,改為滿城。五回開元二十三年,剌史盧暉奏分易縣置城於五回山下,因名之。二十四年,遷於五公城。暉又奏置樓亭、板城二縣。天寶後廢。



 瀛州上隋河間郡。武德四年,討平竇建德,改為瀛州,領河間、樂壽、景城、文安、束城、豐利六縣,五年,又置武垣、任丘二縣。貞觀元年,省豐利入文安,省武垣入河間,割蒲州之高陽、鄚,故景州之平舒,故蠡州之博野、清苑五縣來。又以景城屬滄州。景雲二年,割鄚、任丘、文安、清
 苑四縣屬靺州。天寶元年,改為河間郡。乾元元年,復為瀛州。舊領縣十:河間、高陽、樂壽、博野、清苑、靺、任丘、文安、平舒、束城。景雲二年,分靺、文安、任丘、清苑置靺州。大歷後,割博野、樂壽隸深州。舊戶三萬五千六百五,口十六萬四千。天寶領縣六,戶九萬八千一十八,口六十六萬三千一百七十一。今領縣五。在京師東北二千二百里,至東都一千三百二里。



 河間漢州鄉縣地,屬涿郡。隋為河間縣



 高陽漢
 縣,屬涿郡。隋舊。武德四年,於縣置蒲州,領高陽、博野、清苑三縣,屬蠡州。八年,二縣又割屬蒲州。九年,復隸蠡州。貞觀元年,廢蒲州,以靺、高陽二縣屬瀛州



 平舒漢東平舒縣,屬渤海郡。後去「東」字,隋不改。武德四年,屬景州,貞觀元年,屬瀛州



 束城漢束州縣,屬渤海郡。隋曰束城,屬河間郡



 景城漢縣,屬渤海郡。武德四年,屬瀛州。貞觀元年,屬滄州。大中後,割屬瀛州。



 莫州上本瀛州之鄚縣。景雲二年,於縣置鄚州,割瀛
 州之鄚、任丘、文安、清苑,幽州之歸義等五縣屬之。其年,歸義復還幽州。開元十三年,以「鄚」字類「鄭」字,改為莫。天寶元年,改為文安郡。乾元元年,復為莫州。管縣六:莫、文安、任丘、清苑、長豐、唐興。天寶領縣六,戶五萬三千四百九十三,口三十三萬九千九百七十二。去京師二千三百一十里,至東都一千四百三十里。



 莫漢縣,屬涿郡,至隋不改。武德四年,屬蒲州。貞觀元年,改屬瀛州。景雲二年,割屬莫州



 清苑漢樂鄉縣,
 屬信都國。隋為清苑。武德四年,屬蒲州,貞觀元年,改屬瀛州。景雲二年,屬莫州



 文安漢縣,屬渤海郡,至隋不改,故城在今縣東北。舊屬瀛州,景雲二年來屬



 任丘隋縣,後廢,武德五年,分莫縣復置



 長豐開元十九年,分文安、任丘二縣置



 唐興如意元年,分河間縣置武昌縣,屬瀛州。長安四年,改屬易州。其年,還隸瀛州。神龍元年,改為唐興縣。景雲二年,改屬莫州。



 幽州大都督府隋為涿郡。武德元年,改為幽州總管
 府,管幽、易、平、檀、燕、北燕、營、遼等八州。幽州領薊、良鄉、潞、涿、固安、雍奴、安次、昌平等八縣。二年,又分潞縣置玄州,領一縣,隸總管。四年,竇建德平,固安縣屬北義州。六年,改總管為大總管,管三十九州。七年,改為大都督府,又改涿縣為範陽。九年,改大都督為都督。幽、易、景、瀛、東鹽、滄、蒲、蠡、北義、燕、營、遼、平、檀、玄、北燕等十七州。貞觀元年,廢玄州,以漁陽、潞二縣來屬。又廢北義州,以固安來屬。八年,又置歸義縣。都督幽、易、燕、北燕、平、檀六州。乾封三
 年,置無終縣。如意元年,分置武隆縣。景龍三年,分置三河縣。開元十三年,升為大都督府。十八年,割漁陽、玉田、三河置薊州。天寶元年,改範陽郡。屬範陽、上谷、媯川、密雲、歸德、漁陽、順義、歸化八郡。乾元元年,復為幽州。舊領縣十:薊、潞、雍奴、漁陽、良鄉、固安、昌平、範陽、歸義也。戶二萬一千六百九十八,口十萬二千七十九。天寶,縣十,戶六萬七千二百四十二,口十七萬一千三百一十二。今領縣九。在京師東北二千五百二十里,至東都一千六
 百里。



 薊州所治。古之燕國都。漢為薊縣,屬廣陽國。晉置幽州,慕容雋稱燕,皆治於此。自晉至隋,幽州刺史皆以薊為治所



 幽都管郭下西界,與薊分理。建中二年,取羅城內廢燕州廨署,置幽都縣,在府北一里。



 廣平天寶元年,分薊縣置。三載復廢。至德後,復分置



 潞後漢縣,屬漁陽郡,隋不改。武德二年,於縣置玄州,仍置臨泃縣。玄州領潞、臨泃、漁陽、無終四縣。貞觀元年,廢玄
 州,省臨泃、無終二縣,以潞、漁陽屬幽州



 武清後漢雍奴縣,屬漁陽郡。歷代不改。天寶元年,改為會昌縣。天寶元年,改為永清



 永清如意元年,分安次縣置武隆縣。景雲元年,改為武清



 安次漢縣,屬渤海郡,至隋不改。隋屬幽州



 良鄉漢縣,屬涿郡,至隋不改



 昌平後漢縣,屬廣陽國,故城在今縣東南。隋屬涿郡。



 涿州本幽州之範陽縣。大歷四年,幽州節度使硃希彩奏請於範陽縣置涿州,仍割幽州之範陽、歸義、固安
 三縣以隸涿,屬幽州都督。州新置,未計戶口帳籍。至京師二千四百里,至東都一千四百八十里。



 範陽漢涿郡之涿縣也,郡所治。曹魏文帝改為範陽郡。晉為範陽國,後魏為範陽郡,隋為涿縣。武德七年,改為範陽縣。大歷四年,復於縣置涿州



 新昌漢縣名,後廢。大歷四年,復析固安縣置



 歸義治易縣地,屬涿郡。北齊省入鄚縣。武德五年,於縣置北義州。貞觀元年,與州同省。八年,復置,改屬幽州。分置涿州,又來屬



 固安漢縣,屬涿郡。武德四年,屬北義州,移治章信城。貞觀元年,省義州,以縣屬幽州,乃移於今治。今治城,漢方城縣地,屬廣陽國



 新城大歷四年,析置。



 薊州開元十八年,分幽州之三縣置薊州。天寶元年,改為漁陽郡。乾元元年,復為薊州。天寶領縣三,戶五千三百一十七,口二萬八千五百二十一。至京師二千八百二十三里,至東都一千二十三里。



 漁陽後漢縣。屬漁陽國。秦右北平郡所治也。隋為漁
 陽縣。武德元年,屬幽州。二年,改屬玄州,又分置無終縣。貞觀元年,屬幽州,省無終。神龍元年,改屬營州。開元四年,還屬幽州。十八年於縣置薊州,乃隸之



 三河開元四年,分潞縣置,屬幽州。十八年,改隸薊州



 玉田漢無終縣,屬右北平郡。乾封二年,於廢無終縣置,名無終,屬幽州。萬歲通天二年,改為玉田縣。神龍元年,割屬營州。開元四年,還屬幽州。八年,又割屬營州。十一年,又屬薊州。



 檀州後漢人虒奚縣,屬漁陽郡。隋置安樂郡,分幽州燕樂、密雲二縣隸之。武德元年,改為檀州。天寶元年,改為密雲郡。乾元元年,復為檀州。舊領縣二,戶一千七百三十七,口六千四百六十八。天寶,戶六千六十四,口三萬二百四十六。在京師東北二千六百五十七里,至東都一千八百四十四里。



 密雲隋縣。州所治



 燕樂隋縣。後魏於縣置廣陽郡。後廢。舊治白檀故城,長壽二年,移治新城。即今治也。



 媯州隋涿郡之懷戎縣。武德七年,討平高開道,置北燕州,復北齊舊名。貞觀八年,改名媯州,取媯水為名。長安二年,移治舊清夷軍城。天寶元年,改名媯川郡。乾元元年,復為媯州。舊領縣一,戶四百七十六,口二千四百九十。天寶,戶二千二百六十三,口一萬一千五百八十四。在京師東北二千八百四十二里,至東都一千九百一十里。



 懷戎後漢潘縣,屬上谷郡。北齊改為懷戎。媯水經其
 中,州所治也



 媯川天寶後析懷戎縣置,今所。



 平州隋為北平郡。武德二年,改為平州,領臨渝、肥如二縣。其年,自臨渝移治肥如,改為盧龍縣,更置撫寧縣。七年,省臨渝、撫寧二縣。天寶元年,改為北平郡。乾元元年,復為平州。舊領縣一,戶六百三,口二千五百四十二。天寶領縣三,戶三千一百一十三,口二萬五千八十六。在京師東北二千六百五十里,至東都一千九百里。



 盧龍後漢肥如縣,屬遼西郡,至隋不改。武德二年,改
 為盧龍縣,復開皇舊名



 石城漢縣,屬右北平。貞觀十五年,於故臨渝縣城置臨渝。萬歲通天二年,改為石城,取舊名



 馬城開元二十八年,分盧龍縣置。



 順州下貞觀六年置,寄治營州南五柳城。天寶元年,改為順義郡。乾元元年,復為順州。舊領縣一,戶八十一,口二百一十九。天寶,戶一千六十四,口五千一百五十七。



 賓義郡所理,在幽州城內。



 歸順州開元四年置,為契丹松漠府彈汗州部落。天寶元年,改為歸化郡。乾元元年,復為歸順州。天寶領縣一,戶一千三十七,口四千四百六十九。在京師二千六百里,至東都一千七百一十里。



 懷柔州所理也。



 營州上都督府隋柳城郡。武德元年,改為營州總管府,領遼、燕二州,領柳城一縣。七年,改為都督府,管營、遼二州,貞觀二年,又督昌州。三年,又督師、崇二州。六年,又
 督順州。十年,又督慎州。今督七州。萬歲通天二年,為契丹李萬榮所陷。神龍元年,移府於幽州界置,仍領漁陽、玉田二縣。開元四年,復移還柳城。八年,又往就漁陽。十一年,又還柳城舊治。天寶元年,改為柳城郡。乾元元年,復為營州。舊領縣一,戶一千三十一,口四千七百三十二。天寶,戶九百九十七,口三千七百八十九。在京師東北三千五百八十九里,至東都二千九百一十里。



 柳城漢縣,屬遼西郡。室韋、靺鞨諸部,並在東北。遠者
 六千里,近者二千里。西北與奚接界,北與契丹接界。



 燕州隋遼西郡,寄治於營州。武德元年,改為燕州總管府,領遼西、瀘河、懷遠三縣。其年,廢瀘河縣。六年,自營州南遷,寄治於幽州城內。貞觀元年,廢都督府,仍省懷遠縣。開元二十五年,移治所於幽州北桃谷山。天寶元年,改為歸德郡。乾元元年,復為燕州。舊領縣一,無實土戶。所領戶出粟皆靺鞨別種,戶五百。天寶,戶二千四十五,口一萬一千六百三。兩京道里,與幽州同。



 遼西州所治縣也。



 威州武德二年,置遼州總管,自燕支城徙寄治營州城內。七年,廢總管府。貞觀元年,改為威州,隸幽州大都督。所領戶,契丹內稽部落。舊領縣一,戶七百二十九,口四千二百二十二。天寶,戶六百一十一,口一千八百六十九。兩京道里,與涿州同。



 威化後契丹陷營州乃南遷,寄治於良鄉縣石窟堡,為威化縣,州治也。



 慎州武德初置,隸營州,領涑沫靺鞨烏素固部落。萬歲通天二年,移於淄、青州安置。神龍初,復舊,隸幽州。天寶領縣一,戶二百五十,口九百八十四。



 逢龍契丹陷營州後南遷,寄治良鄉縣之故都鄉城,為逢龍縣,州所治也。



 玄州隋開皇初置,處契丹李去閭部落。萬歲通天二年,移於徐、宋州安置。神龍元年,復舊。今隸幽州。天寶領縣一,戶六百一十八,口一千三百三十三。



 靜蕃州治所,範陽縣之魯泊村。



 崇州武德五年,分饒樂郡都督府置崇州、鮮州,處奚可汗部落,隸營州都督。舊領縣一,戶一百四十,口五百五十四。天寶,戶二百,口七百一十六。



 昌黎貞觀二年,置北黎州,寄治營州東北廢楊師鎮。八年,改為崇州,置昌黎縣。契丹陷營州,徙治於潞縣之古潞城,為縣。



 夷賓州乾封中,於營州界內置,處靺鞨愁思嶺部落,
 隸營州都督。萬歲通天二年,遷於徐州。神龍初,還隸幽州都督。領縣一,戶一百三十,口六百四十八。



 來蘇自徐州還寄於良鄉縣之古廣陽城,為縣。



 師州貞觀三年置,領契丹室韋部落,隸營州都督。萬歲通天元年,遷於青州安置。神龍初,改隸幽州都督。舊領縣一,戶一百三十八,口五百六十八。天寶,戶三百一十四,口三千二百一十五。



 陽師初,貞觀置州於營州東北廢陽師鎮,故號師州。
 神龍中,自青州還寄治於良鄉縣之故東閭城,為州治,縣在焉。



 鮮州武德五年,分饒樂郡都督府奚部落置,隸營州都督。萬歲通天元年,遷於青州安置。神龍初,改隸幽州。天寶領縣一,戶一百七,口三百六十七。



 賓從初置營州界,自青州還寄治潞縣之古潞城。



 帶州貞觀十九年,於營州界內置,處契丹乙失革部落,隸營州都督。萬歲通天元年,遷於青州安置。神龍初,
 放還,隸幽州都督。天寶領縣一,戶五百六十九,口一千九百九十。



 孤竹舊治營州界。州陷契丹後,寄治於昌平縣之清水店,為州治。



 黎州載初二年,析慎州置,處浮渝靺鞨烏素固部落,隸營州都督。萬歲通天元年,遷於宋州管治。神龍初還,改隸幽州都督。天寶領縣一,戶五百六十九,口一千九百九十一。



 新黎自宋州遷寄治於良鄉縣之故都鄉城。



 沃州載初中,析昌州置,處契丹松漠部落,隸營州。州陷契丹,乃遷於幽州,隸幽州都督。天寶領縣一,戶一百五十九,口六百一十九。



 濱海沃州本寄治營州城內,州陷契丹,乃遷於薊縣東南回城,為治所。



 昌州貞觀二年置,領契丹松漠部落,隸營州都督。萬歲通天二年,遷於青州安置。神龍初還,隸幽州。舊領縣
 一,戶一百三十二,口四百八十七。天寶,戶二百八十一,口一千八十八。



 龍山貞觀二年,置州於營州東北廢靜蕃戍。七年,移治於三合鎮。營州陷契丹,乃遷於安次縣古常道城,為州治。



 歸義州總章中置,處海外新羅,隸幽州都督。舊領縣一,戶一百九十五,口六百二十四。



 歸義在良鄉縣之古廣陽城,州所治也。



 瑞州貞觀十年,置於營州界,隸營州都督,處突厥烏突汗達干部落。咸亨中,改為瑞州。萬歲通天二年,遷於宋州安置。神龍初還,隸幽州都督。舊領縣一,戶六十,口三百六十五。天寶,戶一百九十五,口六百二十四。



 來遠舊縣在營州界。州陷契丹,移治於良鄉縣之故廣陽城。信州萬歲通天元年置,處契丹失活部落,隸營州都督。二年,遷於青州安置。神龍初還,隸幽州都督。天寶
 領縣一,戶四百一十四,口一千六百。



 黃龍州所治,寄治範陽縣。



 青山州景雲元年,析玄州置,隸幽州都督。領縣一,戶六百二十二,口三千二百一十五。



 青山寄治於範陽縣界水門村。



 凜州天寶初置於範陽縣界,處降胡。領縣一,戶六百四十八,口二千一百八十七。



 安東都護府總章元年九月,司空李勣平高麗。高麗
 本五部,一百七十六城,戶六十九萬七千。其年十二月,分高麗地為九都督府,四十二州,一百縣,置安東都護府於平壤城以統之。用其酋渠為都督、刺史、縣令,令將軍薛仁貴以兵二萬鎮安東府。上元三年二月,移安東府於遼東郡故城置。儀鳳二年,又移置於新城。聖歷元年六月,改為安東都督府。神龍元年,復為安東都護府。開元二年,移安東都護於平州置。天寶二年,移於遼西故郡城置。至德後廢,初置領羈縻州十四,戶一千五百
 八十二。去京師四千六百二十五里,至東都三千八百二十里。



 新城州都督府、遼城州都督府、哥勿州都督府、建安州都督府、南蘇州、木底州、蓋牟州、代那州、倉巖州、磨米州、積利州、黎山州、延津州、安市州凡此十四州,並無城池。是高麗降戶散此諸軍鎮,以其酋渠為都督、刺史羈縻之。天寶,領戶五千七百一十八,
 口一萬八千一百五十六。



 自燕以下十七州,皆東北蕃降胡散諸處幽州、營州界內,以州名羈縻之,無所役屬。安祿山之亂,一切驅之為寇,遂擾中原。至德之後,入據河朔,其部落之名無存者。今記天寶承平之地理焉。



 山南道



 山南西道



 梁州興元府隋漢川郡。武德元年,置梁州總管府,管梁、洋、集、興四州。梁州領南鄭、褒中、城固、西四縣。二年,
 改城固為唐固,割西縣置褒州。三年,置白雲縣。七年,改總管為都督,督梁、洋、集、興、褒五州。梁州領南鄭、褒中、白雲四縣。八年,廢褒州,以西、金牛二縣來屬。九年,省白雲縣入城固。貞觀三年,復改唐固為城固。五年,改褒中為褒城。六年,廢都督府。八年又置,依舊督梁、洋、集、壁四州。十七年又罷。顯慶元年,復置都督府,督梁、洋、集、壁四州。開元十三年,改梁州為褒州,依舊都督府。二十年,又為梁州。天寶元年,改為漢中郡,仍為都督府。乾元元年,復為
 梁州。興元元年六月,昇為興元府。官員資序,一切同京兆、河南二府。舊領縣五,戶六千六百二十五,口二萬七千五百七十六。武德領縣六,戶三萬七千四百七十,口十五萬三千七百一十七。至京師一千二百二十三里,至東京二千七百八里。



 南鄭州所理。漢縣,屬漢中郡。隋不改。褒城漢褒中縣,屬漢中郡。義寧二年,改為褒中。貞觀三年,復為褒城



 城固隋舊。武德二年,改為唐固。貞觀二年,復為
 城固



 西隋舊。天寶二年,置褒州,割金牛來屬,領西、金牛二縣。八年,廢褒州,以縣屬梁州



 金牛漢葭萌縣地。武德二年,分綿谷縣置,屬褒州。八年,州廢,屬梁州



 三泉武德四年,分綿谷縣置南安州,領三泉、嘉平二縣。八年,廢南安州及嘉平縣,以三泉屬利州。天寶元年,改屬梁州,移治沙溪之東。



 鳳州下隋河池郡。武德元年,改為鳳州。天寶元年,復為河池郡。乾元元年,復為鳳州。舊領縣四,戶一千九百
 五十七,口九千七百九十四。天寶,戶五千九百一十八,口二萬七千八百七十七。在京師西南六百里,至東都一千四百五十里。



 梁泉漢故道縣地。後魏置梁泉縣。晉仇池所處地。後魏廢帝於縣置鳳州



 兩當漢故道縣地。晉改兩當,取水名



 河池後漢縣,屬武都郡。以川為名



 黃花武德四年,分梁泉縣置,以川為名。



 興州下隋順政郡。武德元年,改為興州。天寶元年,改
 為順政郡。乾元元年,復為興州。舊領縣三,戶一千二百二十五,口四千九百一十三。天寶,戶二千二百二十四,口一萬一千四十六。至京師九百四十八里,至東都一千七百八十一里。



 順政漢沮縣,屬武都郡。後魏改為略陽,晉置武興蕃以處互市,後魏於武興蕃置興州,仍以略陽為順政



 長舉漢沮縣地,隋為長舉縣。本治盤頭城,貞觀三年移於今所



 鳴水漢沮縣地,隋為鳴水縣。舊治落蕃水南,永隆元年,移治水北。



 利州下隋義城郡。武德元年,改為利州,領綿谷、葭萌、益昌、義清、岐坪、嘉川、景谷七縣。二年,置總管府,管利、龍、隆、始、蓬、靜六州。三年,割綿谷之東界置南安州。四年,割景谷縣置沙州。七年,又割岐坪、義清二縣置南平州。其年,改總管府為都督府,督利、龍、隆、始、沙、南安、南平、靜八州。利州領綿谷、葭萌、益昌、嘉川四縣。八年,廢南安州,割三泉縣來屬。貞觀元年,廢沙州。二年,廢南平州,復以景谷、岐坪、義清等縣來屬。其年,以嘉川屬靜州。六年,罷都督府。
 以州當劍口,戶不滿萬,移為中州,又降為下州。天寶元年,改為益昌郡,仍割三泉屬梁州。乾元元年,復為利州。舊領縣七,戶九千六百二十八,口三萬一千九十三。天寶領縣六,戶二萬三千九百一十,口四萬四千六百。在京師西南一千四百八十八里,至東都二千一百九十七里。



 綿谷漢葭萌縣地,蜀為漢壽縣。晉改晉壽縣,又分晉壽置興安縣。隋改興安為綿谷。南齊於壽縣置西益州,
 後梁改為利州



 胤山隋義清縣。天寶元年八月,改為胤山



 嘉川隋屬靜州。貞觀十七年,割屬利州



 葭萌漢縣。蜀為漢壽,晉改晉壽,江左改晉安。隋改為葭萌,取漢舊名



 益昌後魏分晉壽縣置京兆縣,後周改為益昌



 景谷漢白水縣地。宋置平興縣,隋改為景谷。武德四年,置沙州,割龍州之方維來屬。沙州領景谷、方維二縣。,貞觀元年,廢沙州,以景谷屬利州,仍省方維縣並入。



 通州上隋通川郡。武德元年,改為通州,領通川、宣漢、三岡、石鼓、東鄉五縣。以宣漢屬南並州。二年,置新寧、思來二縣。三年,以東鄉屬南石州。又為通川總管府,管通、開、蓬、渠、萬、南並、南石、南鄰八州。通州領通川、三岡、石鼓、新寧、思來五縣。八年,以廢南石州之東鄉縣來屬。貞觀元年,以廢南並州之宣漢來屬,又省思來入通川。其年,廢萬州,以永穆來屬。貞觀五年,廢都督府為下州。長安二年,升為中州。開元二十三年,升為上州。天寶元年,改為
 通川郡。乾元元年,復為通州。舊領縣七,戶七千八百九十八,口三萬八千一百二十三。天寶,戶四萬七百四十三,口十一萬八百四。在京師西南二千三百里,去東都二千八百七十五里。



 通川漢宕渠縣地,分置宣漢縣,屬巴郡。後魏改為石城縣。梁於縣置萬州,元魏改為通州。隋為通川縣



 永穆宕渠地,梁置永康縣,隋改為永穆。武德元年,屬巴州。二年,置萬州,蜀割巴州之歸仁,置諾水、廣納、太平、恆
 豐四縣,並屬萬州。七年,省諾水縣。貞觀元年,廢萬州,以歸仁屬巴州,廣納屬壁州,永穆屬通州。廢太平、恆豐二縣入永穆



 三岡隋舊縣



 石鼓後魏置



 東鄉武德三年,置南石州,又分置下蒲、昌樂二縣屬之。八年,廢南石州,省昌樂入石鼓、下蒲入東鄉



 宣漢隋舊。武德元年,置南並州,又置東關縣隸之。貞觀元年,廢南並州,省東關入宣漢。自和昌城移治新安,屬通州



 新寧武德二年,分通川縣置,治新寧故城。貞觀八年,移治賨
 城。巴渠永泰元年六月,分石鼓縣四鄉置巴渠。



 洋州下隋漢川郡之西鄉縣。武德元年,割梁州三縣置洋州。四年,又置洋源縣。天寶元年,改為洋川郡。乾元元年,復為洋州。舊領縣四,戶二千二百二十六,口一萬八千六十。天寶領縣五,戶二萬三千八百四十九,口八萬八千三百二十七。在京師南八百里,至東都二千里。



 西鄉本漢成固縣地,蜀立西鄉縣。後魏於此置洋州,
 以水為名



 黃金漢安陽縣地,屬漢中郡。後魏置黃金縣,水名也。隋縣治巴嶺鎮,貞觀三年,移於今治



 興道隋興勢縣。貞觀二十三年,改為興道



 洋源武德七年,分西鄉縣置



 真符開元十八年,分興道置華陽縣。天寶七年,改屬京兆,仍改為真符。十一年,還屬洋川郡。



 合州中隋涪陵郡。武德元年,改為合州,領石鏡、漢初、赤水三縣。三年,又置新明縣。天寶元年,改為巴川郡。乾
 元元年,復為合州。舊領縣四,戶一萬四千九百三十四,口五萬二百一十。天寶領縣六,戶六萬六千八百一十四,口十萬七千二百二十。在京師南二千四百五十里,至東都三千三百里。



 石鏡漢墊江縣,屬巴郡。宋改名宕渠,宋置東宕渠郡及石鏡縣,又改郡為合州,涪、漢二水合流處為名



 新明武德二年,分石鏡置



 漢初後魏清居縣,隋改漢初



 赤水隋分石鏡置



 巴川開元二十三,割石
 鏡、銅梁二縣置



 銅梁長安三年置。初治奴侖山南,開元三年,移治於武金坑。



 集州下隋漢川郡之難江縣。武德元年,置集州,仍割巴州之符陽、長池、白石三縣來屬。又置平桑縣,凡領五縣。八年,以符陽、白石屬壁州。貞觀元年,廢平桑縣。二年,又置。六年,又省平桑、長池二縣。八年,又割壁州之符陽來屬。十七年,又割廢靜州之地平來屬。天寶元年,改為符陽郡。乾元元年,復為集州。舊領縣一,戶一千一百二
 十六,口四千一十七。天寶領縣三,戶四千三百五十三,口二萬五千七百二十六。在京師西南一千四百二十五里,至東都二千六百里。



 難江漢宕渠縣地,後周改為難江。梁立東巴州,恭帝改為集州。以水為名



 符陽漢縣。武德元年,屬集州。三年,改屬壁州。貞觀八年,復還集州



 地平武德元年,分清化縣置狄平縣。二年,改狄平為地平。其年,置靜州,領地平、嘉川、大牟、清化四縣。貞觀十七年,廢靜州,嘉
 川屬利州,大牟、清化屬巴州,地平屬集州。



 巴州中隋清化郡。武德元年,改為巴州,領化城、清化、曾口、盤道、永穆、歸仁、始寧、奇章、安固、伏虞、恩陽、白石、符陽、長池十四縣。其年以符陽、長池、白石屬集州,以安固、伏虞屬蓬州,清化屬靜州。二年,割歸仁、永穆置萬州。貞觀元年,廢萬州,以歸仁來屬。天寶元年,改為清化郡。乾元元年,復為巴州。舊領縣七,戶一萬九百三十三,口四萬七千八百九十。天寶領縣十,戶三萬二百一十,口九萬一千五十一。至京師二
 千三百六十里,至東都二千五百八十二里。



 化城後漢漢昌縣。梁改為梁大縣,後周改為化城縣。後魏置大谷郡。隋置巴州於縣理



 盤道後魏置



 清化隋屬巴州。武德元年,於清化縣界木門故地置靜州。領清化、大牟二縣。其年,又置地平縣。六年,移靜州於地平縣。又割利州之嘉川,皆隸靜州。貞觀十七年,廢靜州,以清化縣屬巴州



 曾口梁置。隋縣治戴公山。神龍元年,移治曾溪



 歸仁梁置平州,隋改為歸仁縣。
 武德二年,屬萬州。貞觀元年,屬巴州



 始寧梁置,以山為名



 奇章梁置,縣東八里有奇章山



 恩陽梁置義陽縣,隋改為恩陽。貞觀十七年廢。萬歲通天元年,復置



 大牟武德元年,分清化縣置,縣東三里有大牟山



 七盤久視元年分置。



 蓬州下武德元年,割巴州之安固、伏虞,隆州之儀隴、大寅,渠州之宕渠、咸安等六縣,置蓬州,因周舊名。三年,以儀隴屬萬州。尋復來屬。天寶元年,改為咸安郡。至德
 二年,改為蓬山郡。乾元元年,復為蓬州。舊領縣六,戶九千二百六十八,口三萬五千五百六十六。天寶,縣七,戶一萬五千五百七十六,口五萬三千三百五十二。至京師二千二百一十里;至東都二千九百九十五里。



 良山漢宕渠地,梁置伏虞郡安固縣。後周改伏虞為蓬州,安固為良山。開元初,蓬州移治大寅縣,至後不改



 大寅梁置。舊治斗子山,後移治斗壇口,今為蓬州所治



 儀隴梁置。武德二年,屬萬州。州廢,還蓬州。舊
 領金城山,開元二十三年,移治平溪。



 伏虞梁宣漢縣。隋改為伏虞,屬蓬州



 宕渠梁置,取漢縣名。舊治長樂山,長安三年,移治羅獲水



 咸安梁置綏安縣,隋改為咸安。至德二年,改為蓬山



 大竹久視元年,分宕渠縣置。至德二年,割屬濆山郡。



 壁州下武德八年,分巴州始寧縣,改置壁州並諾水縣。又割集州之符陽、白石二縣來屬。貞觀元年,廢萬州,割廣納縣來屬。八年,復以符陽屬集州。天寶元年,改為
 始寧郡。乾元元年,復為壁州。舊領縣三,戶一千四百九十二,口七千四百四十九。天寶,縣四,戶一萬二千三百六十八,口五萬四千七百五十七。在京師西南一千八百二十二里,至東都二千九百四十二里。



 諾水後漢宣漢縣,梁分宣漢置始寧縣。元魏分始寧置諾水縣。武德八年,分巴州始寧之東境,置壁州及諾水縣。今州所治。廣納武德三年,割始寧、歸仁二縣地置,以廣納溪為名



 白石後魏置,以白石水為名。
 武德初,屬巴州,又改屬集州。八年,還壁州。



 巴東開元二十三年六月,置太平縣。天寶元年八月二十四日,改為巴東縣。



 商州隋上洛郡。武德元年,改為商州。其年,於上津縣置上州。貞觀十年,州廢,上津來屬。天寶元年,改為上洛郡。乾元元年,復為商州。舊領縣五,戶四千九百一,口二萬一千五十。天寶,縣六,戶八千九百二十六,口五萬二千八十。至京師二百八十一里,至東都八百八十六
 里。



 上洛漢縣,屬弘農郡。言在洛水之上,故為縣名。隋於縣置上洛郡。



 豐陽漢商縣地。晉分商縣置豐陽,以川為名。舊治吉川城,麟德元年,移理豐陽川。



 洛南漢上洛縣地。晉分置拒陽縣,隋改拒陽為洛南。舊治拒陽川,顯慶三年,移治清川。



 商洛漢商縣,屬弘農郡。隋文加「洛」字



 上津漢長利縣地,屬漢中郡。梁置南洛州,後魏改為上州,隋廢州為上津縣。義寧二年,置上
 津郡。武德元年,改為上州,領上津、豐利、黃土、長利四縣。貞觀初,省長利縣。十年,廢上州,以黃土屬金州,豐利屬均州,上津屬商州。



 安業萬歲通天元年,分豐陽置。景龍三年,改屬雍州。景雲元年,還屬商州。乾元元年正月,改為乾元縣,割屬京兆府。



 金州隋西城郡



 武德元年,改為金州,領洵陽、石泉、安康等縣。其年,割洵陽、驢川二縣置洵州,領三縣。又置西安州。又立寧都、廣德二縣隸西安州。為直州。三年,金州
 置總管府,管金、井、直、洵、洋、南豐、均、漸、遷、房、重、順十二州。七年,廢洵州,以洵城、洵陽、驢川三縣來屬。貞觀元年,廢直州,又省寧都、廣德,以安康來屬,仍省驢川縣。八年,省洵城縣,又以廢上州之黃土縣來屬。天寶元年,改為安康郡。至德二年二月,改為漢南郡。乾元元年,復為金州。舊領縣六,戶一萬四千九十一,口五萬三千二十九。天寶,戶九千六百七十四,口五萬七千九百八十一。在京師南七百三十七里,至東都一千七百里。



 西城州所理。漢西城縣,屬漢中郡。後魏置安康郡,尋改為東梁州。又以其地出金,改為金州。皆以西城為治所。隋末廢。義寧二年,復置



 洵陽漢縣名。武德元年,置洵州,又分洵陽置洵城、驢川二縣。七年,廢洵州,三縣屬金州。貞觀二年,省驢川。八年,省洵城,並入洵陽



 淯陽後魏黃土縣。義寧二年,屬上州。貞觀八年,屬金州。天寶元年,改為淯陽



 石泉隋縣。聖歷元年,改為武安。神龍初,復為石泉。永貞元年,省入漢陰縣,復置



 漢
 陰漢安陽縣,屬漢中郡。晉武改為安康,置安康郡。隋改為縣。武德元年,置西安州,立寧都、廣德二縣。改西安州為直州。州廢,省寧都、廣德二縣入安康。至德二年二月,改為漢陰縣



 平利後周於平利川置吉陽縣,隋改為安吉。武德元年,改為平利。



 開州隋巴東郡之盛山縣。義寧二年,分置萬州,仍割巴東郡之新浦,通川郡之萬世、西流三縣來屬。武德元年,改為開州,領四縣。貞觀初,省西流入盛山。天寶元年,
 改為盛山郡。乾元元年,復為開州。舊領縣三,戶二千一百二十二,口一萬五千五百四。天寶,戶五千六百六十,口三萬四百二十一。在京師南一千四百六十里,至東都二千六百七十里。



 盛山漢朐縣,屬巴郡。蜀分置漢豐縣,周改漢豐為永寧。隋改永寧為盛山。以山為名



 新浦宋分漢豐縣置



 萬歲後周之萬縣,隋加「世」字。貞觀二十三年,改萬世為萬歲縣。



 渠州下隋宕渠郡。武德元年,改為渠州,領流江、賨城、宕渠、咸安、潾水、墊江六縣。其年,改賨城為始安。又分置賨城、義興、豐樂三縣。以宕渠、咸安三縣屬蓬州。又分濆水、墊江、濆山、鹽泉四縣置濆州。三年,割濆州之濆水來屬。八年,省義興、豐樂、賨城三縣。其年,廢潾州,以潾山來屬。天寶元年,改為濆山郡。乾元元年,復為渠州。舊領縣四,戶九千七百二十六,口二萬一千五百五十二。天寶,戶九千九百五十七,口二萬六千五百二十四。在京
 師西南二千一百七十里,至東都三千一百九十里。



 流江漢宕渠縣地,屬巴郡。梁置渠州,周改為北宕渠郡,又改為流江郡。仍於郡內置流江縣。武德元年,改為渠州。又並賨城、義興二縣入流江



 濆水梁置。義寧元年,屬濆州。武德三年,屬渠州



 渠江梁置始安縣,隋不改。天寶元年八月,改為渠江縣



 濆山梁置。濆山,在縣西四十里,重疊濆比為名。隋末,縣廢。武德元年,分置濆山縣,又置濆州。八年,州廢,縣隸渠州。



 渝州隋之巴郡。武德元年,置渝州,因開皇舊名,領江津、涪陵二縣。其年,以涪陵屬涪州。三年,置萬春縣。改萬春為萬壽縣。貞觀十三年,以廢霸州之南平縣來屬。天寶元年,改為南平郡。乾元初,復為渝州。舊領縣四,戶一萬二千七百一十,口五萬七百一十三。天寶,戶六千九百九十五,口二萬七千六百八十五。在京師西南二千七百四十八里,至東都三千四百三十里。



 巴漢江州縣,屬巴郡。古巴子國地。梁置楚州。隋改為
 渝州,以水為名



 江津漢江州縣分置



 萬壽武德三年,分江津縣置萬春縣。五年,改為萬壽



 南平貞觀四年,分巴縣置。於縣南界置南平州,領南平、清穀、周泉、昆川、和山、白溪、瀛山七縣。八年,改南平州為霸州。十三年,州廢,省清穀等縣,以南平縣屬渝州。



 山南東道



 鄧州隋南陽郡。武德二年,改為鄧州,領穰縣、冠軍、深陽三縣。三年,立順陽縣。州置總管,管鄧、淅、酈、宛、淯、新、弘
 等七州。四年,廢總管,隸山南行臺。廢新州,以新野縣來屬。又置平晉縣。六年,省順陽入冠軍,省平晉入穰縣。八年,廢宛州,以南陽來屬,廢酈州,以新城來屬。貞觀元年,省冠軍入新城。天寶元年,改為南陽郡。乾元元年,復為鄧州。舊領縣六,戶三千七百五十四,口一萬八千二百一十二。天寶領縣七,戶四萬三千五十五,口十六萬五千二百五十七。在京師東南九百二十里,至東都六百七十里。



 穰漢縣,屬南陽郡。漢南陽郡以宛為理所,後魏移治於穰。隋改為南陽郡,尋改為鄧州,取漢鄧縣為名



 南陽漢南陽郡所治宛縣也。武德三年,置宛州,領南陽、上宛、上馬、安固四縣,並寄治宛城。八年,州廢,以上馬入唐州,餘三縣入南陽縣,屬鄧州



 新野漢縣,屬南陽郡。晉於縣置義州。武德四年,分置新州,領一縣。其年,新州廢,縣屬鄧州



 向城漢西鄂縣地,屬鄧州。後魏於古向城置縣,乃改立



 臨湍後魏割冠軍縣北境置
 新城縣。武德二年,移治虎遙城,屬酈州。八年,廢酈州,縣屬鄧州。貞觀三年,移治故臨湍聚。天寶元年,改為臨湍縣



 內鄉漢淅縣地,屬弘農郡。後周改為中鄉,隋改為內鄉。武德元年,置淅州,又分內鄉置默水縣,後復改為內鄉



 菊潭漢沮陽縣地。隋改沮水縣,後廢。開元二十四年,割新城復置,改為菊潭。



 唐州上隋淮安郡。武德四年,改為顯州,仍置總管,領顯、北澧、純三州。顯州領比陽、慈丘、平氏、顯岡四縣。五年,
 又分置唐州,屬顯州總管。七年,改為都督府,州不改。貞觀元年,罷都督,仍以廢純州桐柏縣來屬。三年,省顯岡縣。九年,改顯州為唐州,以廢唐州之棗陽、湖陽及廢魯州之方城三縣來屬。十年,以棗陽屬隋州。開元五年,以方城來屬仙州。十三年,置上馬縣。二十六年,以方城來屬。天寶元年,改為淮安郡。乾元元年,復為唐州。舊屬河南道,至德後,割屬山南東道。舊領縣六,戶四千七百二十六。口二萬二千二百九十九。天寶領縣七,戶四萬二
 千六百四十三,口十八萬三千三百六十。至京師一千四百八十里,至東都六百四十六里。



 比陽漢縣,屬南陽郡。後魏置東荊州於漢比陽古城,改為淮州。隋改淮州為顯州,取界內顯望岡為名。貞觀元年,改為唐州。比水出縣東,今縣,州所治也。



 慈丘隋分比陽縣置,取界內慈丘山為名。



 桐柏漢平氏縣地,屬南陽郡。梁置華州,西魏改淮州,又為純州。後周為大義郡,陳廢郡為桐柏縣。



 平氏漢縣,屬南陽
 郡。



 湖陽漢縣,屬南陽郡。隋不改,屬舂陵郡。武德四年,於縣置湖州,領湖陽、上馬二縣。貞觀元年,廢湖州,省上馬,以湖陽屬唐州。



 方城前漢堵陽縣,屬南陽郡。後漢改為順陽。隋改為方城縣,屬淯陽郡。武德二年,於縣置北澧州,領方城、真昌二縣。貞觀初,省真昌縣。八年,改北澧州為魯州,領縣不改。九年,省魯州,以方城屬唐州。



 泌陽後魏石馬縣,後訛為上馬縣。貞觀三年廢。開元十三年,割湖陽復置上馬縣。天寶元年,改為泌陽縣。



 均州下隋淅陽郡之武當縣。義寧二年,割淅陽之武當、均陽二縣置武當郡。又置平陵縣。武德元年,改為均州。七年,省平陵縣。八年,省均陽入武當。其年,以南豐州之鄖鄉、堵陽、安福三縣來屬。貞觀元年,廢均州,又省堵陽、安福二縣。以武當、鄖鄉二縣屬淅州。八年,廢淅州,又以武當、鄖鄉二縣置均州。又廢上州,割豐利縣來屬。天寶元年,改為武當郡。乾元元年,復為均州。舊領縣三,戶二千八百二十九,口一萬二千五百九十三。天寶,戶九千
 六百九十八,口五萬八百九。在京師東南九百三十里,至東都九百一十七里。



 武當州所治。漢縣,南陽郡。梁置南始平郡,後魏改為豐州,隋改為均州,皆治武當縣。縣舊治延岑城,顯慶四年,移於今所



 鄖鄉漢錫縣地,屬漢中郡。晉改為鄖鄉。武德元年,置南豐州,領鄖鄉、安福、堵陽三縣。屬均州。貞觀元年,廢均州,以鄖鄉、武當屬淅州。又省安福、堵陽,並入鄖鄉。八年,復置均州,二縣來屬



 豐利漢長利
 縣地。後魏置豐利郡,分錫縣置豐利縣。武德初,屬上州。州廢,屬均州。



 房州下隋房陵郡。武德元年,改為遷州,領光遷、永清,又置受陽、淅川、房陵,凡領五縣。其年,又於竹山縣置房州,領竹山、上庸,又置武陵,凡領三縣。五年,廢遷州之淅川。。七年,又廢房陵、受陽二縣。貞觀十年,廢遷州,自竹山移房州治於廢州城。其年,省武陵縣。改光遷為房陵縣。天寶元年,改為房陵郡。乾元元年,復為房州。舊領縣四,
 戶四千五百三十三,口二萬一千五百七十九。天寶,戶一萬四千四百二十二,口七萬一千七百八。在京師南一千二百九十五里,至東都一千一百八十五里。



 房陵漢縣,屬漢中郡。後魏為新城郡,又改為光遷國。武德初,改為遷州。置光遷縣。又改為房州,兼改光遷為房陵縣



 永清後魏分房陵縣置大洪縣,周改為永清



 竹山分上庸縣置。武德元年,置房州。貞觀十年,州移治房陵縣



 上庸漢縣,屬漢中郡。



 隋州下隋為漢東郡。武德三年,改為隋州,領隋縣、光化、安貴、平林、順義五縣。五年,省安貴縣。八年,省平林、順義二縣。貞觀十年,割唐州棗陽來屬。天寶元年,改為漢東郡。乾元元年,復為隋州。舊領縣三,戶二千三百五十三,口一萬一千八百九十八。天寶,縣四,戶二萬三千九百一十七,口十萬五千七百二十二。在京師東南一千三百八十八里,至東都一千八里。



 隋漢縣,屬南陽郡。後魏於縣置隋州,隋為漢東郡,皆
 治隋州



 光化隋縣



 棗陽漢舂陵縣,屬南陽郡。隋置舂陵郡。武德三年,改為昌州,領棗陽、舂陵、清潭、湖陽、上馬五縣。其年,分湖陽、上馬置湖州。五年,廢昌州及清潭縣。貞觀元年,省舂陵入棗陽。其年,以廢湖州之上馬、湖陽來屬。九年,廢顯州。自此移唐州於廢顯州,仍屬焉。十年,改屬隋州



 唐城開元二十六年,分棗陽置。



 郢州後魏置溫州。武德四年,置郢州於長壽縣,置京山、藍水二縣屬焉。貞觀元年,省藍水入長壽。又廢郢州,
 以長壽屬鄀州,章山屬荊州。十七年,廢溫州,依舊置郢州,治京山。天寶元年,改為富水郡。乾元元年,復為郢州。舊溫州領縣三,戶一千五百八十,口七千一百七十三。天寶改郢州,戶一萬二千四十六,口五萬七千三百七十五。在京師東南一千四百四十里,至東都一千一百四十九里。



 京山隋縣,屬安陸郡。武德四年,置溫州,因後魏。領京山、富水二縣。貞觀八年,廢鄀州,以長壽來屬。十七年,復
 於縣置郢州。長壽漢竟陵縣地,屬江夏郡。武德四年,於縣置郢州。貞觀元年,廢郢州,以長壽屬鄀州。八年,又屬溫州。十七年,又屬郢州。富水隋舊。武德初,屬溫州。州廢,屬郢州。



 復州隋沔陽郡。武德五年,改為復州,治竟陵縣,貞觀七年,移治沔陽。天寶元年,改為竟陵郡。乾元元年,復為復州。舊領縣三,戶一千四百九十四,口六千二百一十八。天寶,戶八千二百一十,口四萬四千八百八十五。在
 京師東南一千八百里,至東都一千五百一十八里。



 沔陽漢竟陵縣地,屬江夏郡。隋置沔陽郡,武德初,改為復州,皆治此縣



 竟陵漢縣,後廢。晉復置,至隋不改



 監利漢華容縣地,屬南郡。晉置監利縣。



 襄州緊上隋襄陽郡。武德四年,平王世充,改為襄州,因隋舊名。領襄陽、安養、漢南、義清、南漳、常平六縣。州置山南道行臺,統交、廣、安、黃、壽等二百五十七州。五年,省酂州,以陰城、穀城二縣來屬。七年,罷行臺為都督府,督
 襄、鄧、唐、均、淅、重七州。貞觀元年,廢重州,以荊山縣來屬。六年,廢都督府。八年,廢鄀州,以率道、樂鄉二縣來屬。又省常平入襄陽,省陰城入穀城,省南津入義清,省漢南入率道。天寶元年,改為襄陽郡。十四載,置防禦使。乾元元年,復為襄州。上元二年,置襄州節度使,領襄、鄧、均、房、金、商等州,自後為山南東道節度使治所。舊領縣七,戶八千九百五十七,口四萬五千一百九十五。天寶,戶四萬七千七百八十,口二十五萬二千一。在京師東南一
 千一百八十二里,至東都八百五十三里。



 襄陽漢縣,屬南郡。建安十三年,置襄陽郡。晉入為荊州治所。梁置南雍州,西魏改為襄州,隋為襄陽郡,皆以此縣為治所。



 鄧城漢鄧縣,屬南陽郡,古樊城也。宋故安養縣。天寶元年,改為臨漢縣,貞元二十一年,移縣古鄧城置,乃改臨漢為鄧城縣



 穀城漢築陽縣地,屬南陽郡。隋為穀城縣



 義清漢中戶縣地,屬南郡。元魏改為義清縣。舊治柘林,永徽元年,移治清良



 南
 漳漢臨沮縣,屬南郡。晉立上黃縣。後魏改為重陽縣,隋改為南漳。武德二年,分南漳置荊山縣。又於縣治西一百五里置重州,領荊山、重陽、平陽、渠陽、士門、歸義六縣。七年,省渠陽入荊山,省平陽入重陽,又省土門、歸義二縣並房州之永清。貞觀元年,廢重州,以荊山屬襄州。移重陽入州城,改屬遷州。八年,省重陽入荊山。開元十八年,省荊山,移治於南漳故城,乃改為南漳



 宜城漢已阜縣。屬南郡。宋立華山郡於大堤村。即今縣。後
 魏改為宜城郡。分華山、新野置陽立率道縣。周省宜城郡,入率道縣。武德四年,率道屬鄀州。貞觀八年,改隸襄州。天寶七載,改為宜城縣



 樂鄉漢鄀縣,屬南郡。晉於合城郡置樂鄉縣。武德四年,置鄀州,領樂鄉、長壽、率道、上洪四縣。貞觀元年,省上洪縣。八年,廢鄀州,以長壽屬溫州,以樂鄉、率道屬襄州。



 荊州江陵府隋為南郡。武德初,蕭銑所據。四年,平銑,改為荊州,領江陵、枝江、長林、安興、石首、松滋、公安七縣。
 五年,荊州置大總管,管荊、辰、朗、澧、東松、沈、基、復、巴、睦、崇、硤、平等十三州,統潭、桂、交、循、夔、高、康、欽、尹九州。六年,改平州為玉州,改巴州為岳州。七年,廢基州入郢州。其年,改大總管為大都督,督荊、辰、澧、朗、東松、岳、硤、玉八州,仍統潭、桂、交、夔、高、欽、尹等七州。其沈、復、睦、崇四州,循、康二州都督並不統。八年,廢玉州,以當陽縣來屬。貞觀元年,廢郢州,以章山來屬。二年,降為都督府,惟督前七州而已。其桂、潭等七州,不統也。八年,廢東松州入硤州,又省
 章山入長林。十年,辰州改隸黔州。都督硤、澧、朗、岳四州,都督從三品。荊州領江陵、枝江、當陽、長林、安興、石首、松滋、公安等八縣。龍朔二年,升為大都督,督硤、岳、復、郢四州。天寶元年,改為江陵郡。乾元元年三月,復為荊州大都督府。自至德後,中原多故,襄、鄧百姓,兩京衣冠,盡投江、湘,故荊南井邑,十倍其初,乃置荊南節度使。上元元年九月,置南都,以荊州為江陵府,長史為尹,觀察、制置,一準兩京。以舊相呂諲為尹,充荊南節度使,領澧、朗、硤、
 夔、忠、歸、萬等八州,又割黔中之涪,湖南之岳、潭、衡、郴、邵、永、道、連八州,增置萬人軍,以永平為名。二年,置長寧縣於郭內,與江陵並治。其年,省枝江縣入長寧。至德二年,江陵尹衛伯玉以湖南闊遠,請於衡州置防禦使。自此,八州置使,改屬江南西道。舊領縣八,戶一萬二百六十,口四萬九百五十八。天寶領縣七,戶三萬一百九十二葉四萬八千一百四十九。在京師東南一千七百三十里,至東都一千三百一十五里。



 江陵漢縣,南郡治所也。故楚都之郢城,今縣北十里紀南城是也。後治於郢,在縣東南。今治所,晉桓溫所築城也



 長寧上元元年,分江陵縣置,治郭下。二年,又廢枝江並入



 當陽漢縣,屬南郡。武德四年,於縣置平州,領當陽、臨沮二縣。六年,改屬玉州。又省臨沮入當陽,屬荊州



 長林晉分編縣置長林縣,以其有櫟林長阪故也。武德四年,於縣東北百二十里置基州及章山縣。七年,廢基州,以章山屬郢州。州廢,屬荊州。八年,省入
 長林



 石首漢華容縣,屬南郡。武德四年,分華容縣置,取縣北石首山為名。舊治石首山,顯慶元年,移治陽支山下



 松滋漢高城縣地,屬南郡。松滋,亦漢縣名。屬廬江郡。晉時松滋縣人避亂至此,乃僑立松滋縣,因而不改



 公安吳孱縣地。漢末左將軍劉備,自襄陽來鎮此,時號左公,乃改名公安。



 硤州下隋夷陵郡。武德四年,平蕭銑,置硤州,領夷陵、夷道、遠安三縣。貞觀八年,廢東松州,以宜都、長陽、巴山
 三縣來屬。其年,省夷道入宜都。九年,自下牢鎮移治陸抗故壘。天寶元年,改為夷陵郡。乾元元年,復為硤州。舊領縣五,戶四千三百,口一萬七千一百二十七。天寶,戶八千九十八,口四萬五千六十六。在京師東南一千八百八十八里,至東都一千六百四十六里。



 夷陵漢縣,屬南郡。有夷山在西北,因為名。蜀置宜都郡。梁改為宜州,後魏改為拓州,又改為硤州。隋縣治石皋城。武德四年,移治夷陵府。貞觀九年,移治陸抗故壘



 宜都漢夷道縣,屬南郡。陳改為宜都,隋改為宜昌,屬荊州。武德二年,置江州,領宜昌一縣,尋改為宜都。六年,改江州為東松州。八年,廢睦州。以長陽、巴山來屬。貞觀八年,廢東松州,盡以三縣屬硤州



 長陽漢佷山縣,屬武陵郡。隋改為長陽,以溪水為名。隋屬荊州,武德四年,置睦州,領長陽、巴山二縣。八年,廢睦州,以二縣屬東松州。貞觀八年,屬硤州



 遠安漢臨沮縣地,屬南郡。晉改高安縣。後周改為遠安,屬硤州



 巴山隋分
 佷山縣置巴山縣。武德二年,置江州,領巴山、鹽水二縣。四年,廢江州及鹽水縣,以巴山屬睦州。八年,屬東松州。貞觀八年,屬硤州。



 歸州隋巴東郡之秭歸縣。武德二年,割夔州之秭歸、巴東二縣,分置歸州。三年,分秭歸置興山縣,治白帝城。天寶元年,改為巴東郡。乾元元年,復為歸州。舊領縣三,戶三千五百三十一,口二萬一十一。天寶,戶四千六百四十五,口二萬三千四百二十七。在京師南二千二百
 六十八里,至東都一千八百四十三里。



 秭歸漢縣,屬南郡。魏改為臨江郡。吳、晉為建平郡。隋屬巴東郡。武德二年,置歸州。



 巴東漢巫縣地,屬南郡。周置樂鄉縣,隋改為巴東縣



 興山武德三年,分秭歸縣置。舊治高陽城,貞觀十七年,移治太清鎮,天授二年,移治古夔子城。



 夔州下隋巴東郡。武德元年,改為信州。領人復、巫山、雲安、南浦、梁山、大昌、武寧七縣。二年,以武寧、南浦、梁山
 屬浦州。又改信州為夔州,仍置總管,管夔、硤、施、業、浦、涪、渝、穀、南、智、務、黔、克、思、巫、平十九州。八年,以浦州之南浦、梁山來屬。九年,又以南浦、梁山屬浦州。貞觀十四年,為都督府,督歸、夔、忠、萬、涪、渝、南七州。後罷都督府。天寶元年,改為雲安郡。至德元年,於雲安置七州防禦使。乾元元年,復為夔州。二年,刺史唐論請升為都督府。尋罷之。舊領縣四,戶七千八百三十,口三萬九千五百五十。天寶,戶一萬五千六百二十九,口六萬五十。在京師南二
 千四百四十三里,至東都二千一百七十五里。



 奉節漢魚復縣,屬巴郡,今縣北三里赤甲城是也。梁置信州,周為永安郡,隋為巴東郡,仍改為人復縣。貞觀二十三年,改為奉節



 雲安漢朐縣,屬巴郡。故城曰萬戶城。縣西三十里,有鹽官



 巫山漢巫縣,屬南郡。隋加「山」字,以巫山硤為名。舊治巫子城



 大昌晉分巫、秭歸縣置建昌縣,又改為大昌。隋不改。



 萬州隋巴東郡之南浦縣。武德二年,割信州之南浦
 置南浦州,領南浦、梁山、武寧三縣。八年,廢南浦州,以南浦、梁山屬夔州,武寧屬臨州。其年,復立浦州,依舊領三縣。貞觀八年,改為萬州。天寶元年,改為南浦郡。乾元元年,復為萬州,舊領縣三,戶五千三百九十六,口三萬八千八百六十七。天寶,戶五千一百七十九,口二萬五千七百四十六。在京師西南二千六百二十四里,至東都二千四百六十五里。



 南浦後魏分朐縣置魚泉縣,周改為萬川,隋改為
 南浦。武德二年,置浦州。貞觀八年,改為萬州,以此縣為治所



 武寧漢臨江縣地,周分置源陽縣,隋改為武寧,治巴子故城



 梁山後周分朐縣置,治後魏萬川郡故城。



 忠州隋巴東郡之臨江縣。義寧二年,置臨州,又分置豐都縣。武德二年,分浦州之武寧置南賓縣,又分臨江置清水縣,並屬臨州。八年,又以浦州之武寧來屬。其年,又隸浦州。九年,以廢濆州之墊江來屬。貞觀八年,改臨
 州為忠州。天寶元年,改為南賓郡。乾元元年,復為忠州。舊領縣五,戶八千三百一十九,口四萬九千四百七十八。天寶,戶六千七百二十二,口四萬三千二十六。在京師南二千二百二十二里,至東都二千七百四十七里。



 臨江漢縣,屬巴郡。後魏置萬川郡。貞觀八年,改臨州為忠州,治於此縣



 豐都漢枳縣地,屬巴郡。後漢置平都縣。義寧二年,分臨江置豐都縣



 南賓武德二年,分武寧縣置



 墊江漢縣,屬巴郡,後廢。後魏分臨
 江復置。周改為魏安,隋復為墊江。武德初,屬潾州。州廢,屬臨川



 桂溪武德二年,分臨江置清水縣。天寶元年,改為桂溪。



\end{pinyinscope}