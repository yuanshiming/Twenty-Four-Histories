\article{卷五十三列傳第四十一}

\begin{pinyinscope}

 萬
 俟普子洛可朱渾元劉豐破六韓常金祚劉貴蔡俊韓賢尉長命王懷任祥子胄莫多婁貸文子敬顯厙狄迴洛厙狄盛張保洛賀拔仁曲珍段琛尉摽摽子相貴康德韓建業封輔相范舍樂牒舍樂侯莫陳相薛孤延
 斛律羌舉子孝卿張瓊宋顯王則慕容紹宗叱列平步大汗薩薛脩義慕容儼厙狄伏連潘樂彭樂暴顯皮景和綦連猛元景安獨孤永業鮮于世榮傅伏萬俟普,字普撥,太平人,其先匈奴之別也。少雄果,有武力。正光中,破六韓拔陵構逆,逼授太尉。後歸魏,累遷第二鎮人酋長。孝武帝初,封清水郡公。隨入關,拜司空。神武平夏州,普自覆靺城率部歸齊神武。神武躬自迎接,
 封河西郡公,位太尉,薨。贈太師、大司馬、錄尚書事。子洛。



 洛,字受洛干,隨孝武入關,除尚書左僕射。天平中,隨父東歸,封建昌郡公,再遷領軍將軍。初,神武以其父普尊老,特崇禮之,嘗親扶上馬。洛免冠稽首,願出萬死力以報深恩。及河陰之戰,諸軍北度橋,洛以一軍不動,謂西人曰:「萬俟受洛干在此,能來可來也!」西人畏而去之。神武名其所營地為迴洛。洛慷慨有氣節,勇銳冠世。卒,贈太師、大司馬、太尉、錄尚書,謚曰武。



 可朱渾元,字道元,自云遼東人也。曾祖護野肱,為懷朔鎮將,遂家焉。元寬仁有武略,少與神武相知。爾朱榮以
 為別將,隸爾朱天光。平萬俟醜奴等,以功封東縣伯。孝武帝立,累遷渭州刺史。元既早為神弄知遇,兼其母兄在東,恒表疏與神武往來。周文帝有疑心,元乃率所部三千戶,發渭州,西北度烏蘭津,歷河、源二州境,乃得東出。靈州刺史曹掞待元甚厚。掞女婿劉豐生與元深相結,遂資遣元。



 元從靈州東北入雲州界。周文每遣兵邀元,元戰必摧之。神武聞其來,遣平陽太守高崇持金環一枚賜元,並運資糧候接。元至,引見執手。後進並州刺史,以貪污劾,特見原。累以軍功拜司空。天保初,封扶風郡王,位太傅、太師。薨,贈假黃鉞、太宰、太師。錄尚書。元用
 兵務持重,未嘗敗。皇建初,配享文襄廟庭。子長舉襲。



 道元弟天元,亦有將略,便弓馬,封昌陽縣伯。天保初,位殿中、七兵二尚書。



 卒,贈都督、滄州刺史,謚曰恭武。



 天元弟天和,以道元勛重,尚東平長公主,賜爵宜安鄉男。文宣受禪,加駙馬都尉,位開府儀同三司,封成皋郡公。濟南即位,加特進,改封博陵郡公。與楊愔同被殺。追贈司空。



 劉豐,字豐生,普樂人也。有雄姿壯氣,果毅絕人。破六韓拔陵之亂,以守城功,除普樂太守,山鹿縣公,靈州鎮城大都督。賀拔岳與靈州刺史曹掞不睦,豐助掞守。岳將自討掞,為侯莫陳悅所殺。周文帝遣行臺趙善、大都督
 萬俟受洛干復來攻圍,引河灌之,掞與豐堅守不下。豐乃東奔神武,神武以豐為南汾州刺史。河陰之役,豐功居先,神武執其手嗟賞之。及王思政據長社,豐與高岳等攻之。先是訛言大魚道上行,百姓苦之。豐建水攻策,遏洧水灌城。水長,魚鱉皆游焉。城將陷,豐與行臺慕容紹宗見忽有暴風從東北來,正晝昏暗,飛沙走礫,船纜忽絕,漂至城下。豐拍浮向土山,為浪激,不時至。西人鉤之,並為敵所害。豐壯勇善戰,死日,朝野駭惋。贈大司馬、司徒公、尚書令,謚武忠。子曄嗣。



 第三子龍,有巧思,位亦通顯。隋開皇中,歷將作大匠,卒於領軍大將軍。



 八子俱
 非嫡妻所生。每一子所生喪,諸子皆為制服三年。武平中,所生喪,諸弟並請解官,朝廷義而不許。



 破六韓常,單于之裔也。初呼廚貌入朝漢,為魏武所留,遣其叔父右賢王去卑監本國戶。魏氏方興,率部南轉,去卑遣弟右谷蠡王潘六奚率軍北禦。軍敗,奚及五子俱沒於魏,其子孫遂以潘六奚為氏。後人訛誤,以為破六韓。世領部落。父孔雀,少驍勇,背其宗人拔陵,率部降爾朱榮。詔封永安縣侯,第一領人酋長。常,孔雀少子,沉敏有膽略,善騎射。爾朱榮死,常居河西。天平中,與冀州刺史萬俟受洛干等東歸,神武上為武衛將軍。齊受禪,
 封廣川縣公,拜太子太保。卒於滄州刺史。贈尚書令、司徒公、太傅、第一領人酋長、假王,謚曰忠武。



 金祚,字神敬,安定人也。性驍雄,尚氣俠。魏末,以軍功至太中大夫,隨元天穆討平邢杲。歷涇、岐二州刺史。後大行臺賀拔岳表授東雍州刺史,令討仇池氏楊紹先於百頃。未還,岳為侯莫陳悅所殺。祚克仇池還,莫知所歸。俄而神武遣行臺侯景慰諭,祚遂解甲而還,封安定縣公。後隨魏孝武西入,周文帝以祚為兗州刺史。歷太僕、衛尉二卿。尋除東北道大都督、晉州刺史,入據東雍州。神武遣尉景攻降之。芒山之戰,以大都督從破西軍,除
 華州刺史。文宣受禪,加開府儀同三司,別封臨濟縣子。卒,贈司空公。



 劉貴,秀容陽曲人也。剛格有氣斷。歷爾朱榮府騎兵參軍。榮性猛急,貴尤嚴峻,任使多愜榮心。普泰初,行汾州事,棄戍歸齊神武。累遷御史中尉、肆州大中正,加開府、西道行臺僕射。貴所歷,莫不肆其威酷,非理殺害,視下如草芥。性峭直,攻訐無所回避。雖非佐命元功,然與神武布衣舊,特見親重。卒,贈太保、太尉公、錄尚書事,謚忠武。齊受禪,詔祭告其墓。皇建中,配享神武廟庭。



 次子洪徽嗣樂縣男。卒,贈都督、燕州刺史。



 蔡俊,廣寧石門人也。父普,北方擾亂,走奔五原,守戰有功,拜寧朔將軍。



 卒,贈燕州刺史。俊豪爽有膽略,齊神武微時,深相親附。俊初為杜洛周所虜,時神武亦在洛周軍中。神武謀誅洛周,俊預其計。事泄奔葛榮。仍背榮歸爾朱榮。從入洛。及從破葛榮,平元顥,封烏洛縣男。隨神武舉義,及平鄴,破韓陵,並有戰功,進爵為侯。出為齊州刺史。為政嚴暴,又多受納。然亦明解,有部分,吏人畏服之。性好賓客,頗稱施惠。天平中,卒於揚州刺史,贈尚書令、司空公,謚曰威武。齊受禪,詔祭告其墓。皇建初,配享神武廟庭。



 韓賢,字普賢,廣寧石門人也。壯健有武用。初,隨葛榮作逆,榮破後,爾朱榮擢充左右。榮死,爾朱度律以賢為帳內都督,封汾陽縣伯。後為廣州刺史。及齊神武起義,度律以賢素為神武所知,恐有變,遣使徵之。不願去,乃密遣群蠻多舉烽,若有寇至。使者遂為啟,得停。賢仍潛使人通誠於神武。後拜建州刺史。天平初,為洛州刺史。州人韓木蘭等起兵,賢破之。親自案檢收甲仗,有一賊窘迫藏屍間,見將至,忽起斫賢,斷其脛而卒。始漢明帝時,西域以白馬負佛經送洛,因立白馬寺。其經函傳於此寺,形制厚朴,世以古物,歷代寶之。賢知,故斫破之,未幾
 而死。論者謂因此致禍。贈尚書令、司空。子裔嗣。



 尉長命,太安狄那人也。父顯,魏代郡太守。長命性和厚,有器識。參預齊神武起兵,破爾朱氏於韓陵,拜安南將軍。樊子鵠據袞州反,除東南道大都督,與諸軍討平之。徙幽州刺史,督安、平二州。雖多聚斂,然以恩撫人,少得安集。卒,贈司空,謚曰武壯。



 子興,字敬興。便弓馬,有武藝,位冠軍將軍。



 王懷,字懷周,不知何許人也。少好弓馬,頗有氣尚。隨齊神武於冀州起兵,討破爾朱兆於廣阿,又從破四胡何於韓陵,以功封盧鄉縣侯。天平中,為都督、廣州刺史。後從
 神武襲剋西夏州。還,為大都督,鎮下館。除車騎大將軍、儀同三司。



 卒,贈司徒公、尚書僕射。



 懷以武藝勳誠,為神武所知。志力未申,論者惜其不遂。皇建初,配饗神武廟庭。



 任祥,字延敬,廣寧人也。少和厚,有器度,初從葛榮,榮署為王。榮敗,擁所部先降。拜廣寧太守,賜爵西河縣公。隨齊神武起兵,封魏郡公。後兼尚書左僕射,進位開府儀同三司。祥位望既重,能以寬和接物,人士稱之。及斛斯椿釁發,祥棄官北走,歸神武。天平初,拜侍中,遷徐州刺史。在州大有受納,然政不殘,不為人所疾苦。潁川長史
 賀若徽執刺史田迅,據城降西魏。祥戰失利,還北。與行臺侯景、司徒高昂共攻拔潁川。元象元年,卒於鄴。贈太尉公、錄尚書事。



 子胄,性輕俠,頗敏慧,少在神武左右。天平中,擢為東郡太守。家本豐財,又多聚斂,動極豪華;賓客往來,將迎至厚。興和未,神武攻王壁還,留清河公岳為行臺,鎮守晉州,以胄隸之。胄飲酒游縱,不勤防守,神武責之。懼,遂潛遣使送款於周。為人所糾,推勘未得實,神武特免之。胄內不自安,乃與儀同爾朱文暢、參軍房子遠、鄭仲禮等陰圖弒逆,伏誅。



 莫多婁貸文,太安狄那人也。驍果有膽氣。從神武起兵,
 破爾朱兆於廣阿,封石城縣子。從破四胡於韓陵,進爵為侯。從平爾朱兆於赤谼嶺,兆自縊,貸文獲其屍。天平中,進爵為公,晉州刺史。元象初,除車騎大將軍、儀同三司、南道大都督,與行臺侯景攻獨孤信於金墉城。周文帝出函谷,景與高昂議待其至。貸文請率所部擊其前鋒,景等固不許。貸文性勇而專,不受命,以輕騎一千,軍前斥堠,死於周軍。贈尚書左僕射、司徒公。



 子敬顯嗣,彊直勤幹,少以武力見知。恆從斛律光征討,數有戰功。光每令敬顯前驅置營,中夜巡察,或達旦不眠。臨敵置陣,亦命部分將士,深見重。位至開府儀同三司。武平七年,
 從後主平陽敗歸,在並州與唐邕等推立安德王稱尊號。安德敗,武將皆投周軍,唯敬顯走還鄴,授司徒。周武帝平鄴,執之,斬於閶闔門外,責其不留晉陽也。



 厙狄迴洛,代人也。少有武力,儀貌魁偉。初事爾朱榮。榮死,隸爾朱兆。神武舉兵於信都,迴洛擁眾來歸。從破四胡於韓陵,以軍功封順陽縣子,累遷夏州刺史。昭帝即位,封順陽郡王。大寧初,為朔州刺史,轉太子太師。卒,贈太尉、定州刺史。



 厙狄盛,字安盛,懷朔人也。性和柔,少有武用。初為神武親信都督,從征伐。



 累遷幽州刺史,封長廣縣公。齊受禪,
 改封華陽縣公,後拜特進。卒,贈太尉公。



 張保洛,自云本出南陽西鄂,家世好賓客,尚氣俠,頗為北土所知。保洛少便弓馬。初從葛榮。榮敗,仍為爾朱榮統軍。後隸齊神武。神武起兵,保洛為帳內,從破爾朱兆於廣阿及韓陵戰。元象初,為西夏州刺史,以前後功,封安武縣伯。又從戰芒山,進爵為侯。文襄嗣事,歷梁州刺史,進爵為公。齊受禪,加開府,仍為刺史。聚斂,為百姓所患。濟南初,兼侍中,尋出為滄州刺史,封敷城郡王。以聚斂免官,奪王爵。卒,贈前官,追復本封。



 從神武出山東,又有賀拔仁、曲珍、段琛、尉摽、子相貴、康德、韓建業、封輔相、
 范舍樂、牒舍樂,並以軍功至大官,史失其事。



 仁字天惠,無善人。以帳內都督從神武破爾朱氏於韓陵,力戰有功。天保初,封安定郡王,歷數州刺史、太保、太師、右丞相、錄尚書事。武平元年薨,贈假黃鉞、相國、太尉、錄尚書、十二州諸軍事、朔州刺史,謚曰武。



 珍字舍洛,西平酒泉人。壯勇騎射,以帳內從神武。天統中,封安康郡王。武平初,為豫州道行臺尚書令、豫州刺史。卒,贈太尉。



 琛字懷寶,代人。少有武用,從起兵。天保中,開府儀同三司、兗州刺史。



 摽,代人,大寧初,位司徒,封海昌王。卒,子相貴嗣。



 相貴,武平末,開府儀同三司、晉州道行臺尚書僕射、晉州刺
 史。及行臺左丞侯子欽等密啟周武帝請師,求為內應。周武自率眾至城下。子欽等夜開城門,引軍入,鎖相貴送長安,卒。



 弟相願,彊幹有膽略,武平末,開府儀同三司、領軍大將軍。自平陽至並州及到鄴,每立計將殺高阿那肱,廢後主立廣寧王。事竟不果。及廣寧被出,相願拔佩刀斫柱而歎曰:「大事去矣,知復何言!」



 德,代人。歷數州刺史、並省尚書左僕射、開府儀同三司,封新蔡王。



 建業、輔相,俱不知所從來。建業位領軍大將軍、並州刺史。以輔相為朔州總管。



 范舍樂,代人。有武藝,筋力絕人。位東雍州刺史、開府儀同三司,封平舒侯。



 牒舍樂,武威人。開
 府儀同三司、營州刺史,封漢中郡公,戰歿關中。



 侯莫陳相,代人也。祖社伏頹,魏第一領人酋長。父斛古提,朔州刺史,白水公。相七歲喪父,號慕過人。及長,性雄傑。後從神武起兵,破四胡於韓陵,力戰有功,封陽平縣伯,後改封白水郡公。天保初,累遷司空公,進爵白水王,又遷大將軍,拜太尉公,兼瀛州刺史。歷太保、朔州刺史,又授太傳,別封義寧郡公。薨於州,贈假黃鉞、右丞相、太宰、太尉、都督、朔州刺史。



 次子晉貴,嚴重有文武幹略,襲爵白水王,武衛將軍、開府儀同三司、梁州刺史。歸周,授上大將軍,封信安縣公。子仲宣,太常丞。子弘穎、弘信,雍
 州司士參軍。子行方、行儉、行恭。



 薛孤延,代人也。少驍果,從神武起兵,以功累加儀同三司。從西征,至蒲津。



 及竇泰失利,神武班師,延後殿,且戰且行,一日斫折十五刀。神武嘗閱馬於北牧,道逢暴雨,大雷震地,火燒浮圖。神武令延視之。延案槊直前,大呼繞浮圖走,火遂滅。延還,鬚及馬鬃尾皆焦。神武歎其勇決,曰:「延乃能與霹靂鬥!」後封平秦公,與諸將討潁川。延專監造土山,以酒醉,為敵所襲據。潁川平,諸將還京師,宴華林園。文襄啟魏帝,坐延階下以辱之。齊受禪,別賜爵都昌縣公。延性好酒,率多昏醉。以善戰,每大軍征討,
 常為前鋒。位太子太保、太傅。



 斛律羌舉,太安人也。世為部落酋長。羌舉少驍果,從爾朱兆。兆破,乃歸誠神武。神武以其忠於所事,亦加嗟賞。天平中,除大都督。後從神武戰於沙苑,時議進趣計,羌舉曰:「黑獺若欲固守,無糧援可恃。今揣其情,欲一死決,有同狾犬,或能噬人。且渭曲土濘,無所用力。若不與戰,徑趣咸陽,咸陽空虛,可不戰而剋。拔其根本,則黑獺之首,可懸軍門。」神武欲縱火焚之,侯景曰:「當禽以示百姓,燒殺誰復信之?」諸將議既有異同,遂戰於渭曲,大軍敗績。後封密縣侯,為東夏州刺史。有疫疾,刺胸,竹筒吮之,
 垂愈。因怒,創裂而卒。贈儀同三司。



 子孝卿嗣。



 孝卿少聰敏,機悟有風檢。武平末,位侍中、開府儀同三司,封義寧王,知內省事,典外兵、騎兵機密。時政由群豎,自趙彥深死後,朝貴典機密者,唯孝卿一人差居雅道,不至貪穢。後主至齊州,以孝卿為尚書令,又以中書侍郎薛道衡為侍中,封北海王。二人勸後主作承光詔,禪位任城王。令孝卿齎詔策及傳國璽往瀛州,孝卿便詣鄴。仍從周武帝入關,授儀同大將軍、宣納上士。隋開皇中,位太府卿、戶部尚書。



 張瓊,字連德,代人也。少壯健,有武用,初隨葛榮為亂。榮
 敗,爾朱榮以為都督。後歷位濟州刺史。及爾朱氏敗,歸神武,拜滄州刺史,加驃騎大將軍、開府儀同三司。天平中,神武襲剋夏州,以瓊為慰勞大使,留鎮之。尋為周文帝所陷,卒。贈司徒、都督、恆州刺史。



 瓊子欣,尚魏平陽公主,除駙馬都尉、驃騎大將軍、開府儀同三司、建州刺史,南鄭伯。瓊常憂其太盛,每謂親識曰:「凡人官爵,莫若處中。欣位秩太高,深為憂慮。」而欣豪險,遂與公主情好不篤。尋為孝武所害。時人稱瓊先見。



 宋顯,字仲華,敦煌效穀人也。性果毅,有乾用。初事爾朱榮,稍遷為記室參軍。榮死,世隆等以為晉州刺史。後歸
 神武為行臺左丞,拜西袞州刺史。在州多所受納,然勇決有氣幹,檢御左右,咸得其心力。及河陰之戰,深入,沒于行陣。贈司徒公。



 王則,字元軌,自云太原人也。少驍果,有武藝。初隨叔父魏廣平內史老生征討,每有戰功。老生為朝廷所知,則頗有力。初以軍功賜白水子。元顥入洛,則與老生俱降顥。顥疑老生,遂殺之。則奔廣州刺史鄭先護,與同拒顥。顥敗,為東徐州防城都督。爾朱榮之死也,東徐州刺史斛斯椿是其枝黨,內懷憂懼。時梁立魏汝南王悅為魏主,資其士馬,送之境上,椿遂降悅。則與蘭陵太守李義
 擊其偏師,破之。魏因以則行北徐州事,隸爾朱仲遠。仲遠敗,乃歸神武。天平初,頻以軍功,都督、荊州刺史。則有威武,邊人畏伏之。渭曲之役,則為西師圍逼,棄城奔梁。



 梁尋放還,神武恕而不責。元象初,洛州刺史。以前後勛,封太原縣伯。則性貪,在州不法,舊京諸像,毀以鑄錢,於時號河陽錢,皆出其家。以武用,除徐州刺史,取受狼籍。令送晉陽,文襄怒其罪。卒,贈司空,謚烈懿。



 則弟敬寶,位東廣州刺史,與蕭軌攻建業,不剋,死焉。



 慕容紹宗,字紹宗,晃第四子太原王恪之後也。曾祖騰,歸魏,遂居代。祖郁,岐州刺史。父遠,恆州刺史。紹宗容貌
 恢毅,少言,深沉有膽略。爾朱榮即其從舅子也。榮入洛,私告曰:「洛中人士繁盛,驕侈成俗,不除翦,恐難制。吾欲因百官出迎,悉誅之,若何?」對曰:「太后淫虐,天下共棄。公既執忠義,忽欲殲夷多士,實謂非策。」不從。後以軍功封索盧侯,遷爾朱兆長史。及兆敗,紹宗於烏突城見神武,遂攜爾朱榮妻子并兆餘眾自歸神武。神武仍加恩禮,所有官爵並如故,軍謀兵略,時參預焉。



 及遷鄴,令紹宗與高隆之共知府庫、圖籍諸事。累遷青州刺史。時丞相記室孫搴屬紹宗,以其兄為州主簿,紹宗不用。搴譖之神武曰:「紹宗嘗登廣固城長歎,謂所親云:大丈夫有復
 先業理不?」由是徵還。元象初,以軍功進爵為公,累遷御史中尉。屬梁人劉烏黑入寇徐方,授徐州刺史。執烏黑殺之。還,除尚書左僕射。



 侯景反,命紹宗為東南道行臺,加開府,改封燕郡公,又與大都督高岳禽梁貞陽侯蕭明於寒山。迴軍討侯景於渦陽。時景軍甚盛,初聞韓軌往討之,曰:「啖豬腸小兒。」聞高岳往,曰:「此兵精人凡爾。」諸將被輕。及聞紹宗至,扣鞍曰:「誰教鮮卑小兒解遣紹宗來?若然,高王未死邪?」及與景戰,諸將頻敗,無肯先者。紹宗麾兵徑進,諸將從之,因大捷。



 西魏遣王思政據潁川,又以紹宗為南道行臺,與太尉高岳、儀同劉豐圍擊之,
 堰洧水灌城。時紹宗數有凶夢,每惡之,私謂左右曰:「吾自數年已還,恆有蒜髮,昨來忽盡。蒜者算也,其算盡乎!」未幾,與劉豐臨堰,見北有塵氣,乃入艦同坐。



 暴風從東北來,纜斷飄艦,徑向敵城。紹宗自度不免,遂投水卒。三軍將士,莫不悲惋,朝廷嗟傷焉。贈太尉,謚曰景惠。



 長子士肅,以謀反伏法。朝廷以紹宗功,罪止士肅身。皇建初,配享文襄廟庭。



 士肅弟三藏。



 三藏幼聰敏,多武略,頗有父風。武平初,襲爵燕郡公。以軍功,歷位武衛大將軍。周師入鄴,齊後主東遁,委三藏留守鄴宮。齊王公已下皆降,三藏猶拒戰。



 及齊平,武帝引見,恩禮甚厚,授儀同大
 將軍。隋開皇元年,授吳州刺史。九年,副襄陽公韋洸討平嶺南。至廣州,洸中流矢卒,詔三藏檢校廣州道行軍事。以功授大將軍。後遷廓州刺史,人歌頌之,文帝數有勞問。又畜產繁滋,獲醍醐奉獻,賚物百段。十三年,州界連雲山響,稱萬歲者三,詔頒郡國。仍遣使醮山所。其日景雲浮於上,雉兔馴壇側。使還以聞,上大悅,改封河內縣男。歷疊州總管、和州刺史、淮南郡太守,所在有惠政。改授金紫光祿大夫。大業七年卒。



 叱列平,字殺鬼,代郡西部人,世為酋帥。平有容貌,美鬚髯,善射馭。襲第一領人酋長、臨江伯。魏末,以軍功至武
 衛將軍。隨爾朱榮破葛榮,平元顥,封癭陶縣伯。榮死,爾硃氏陵僭。平懼禍,後歸神武。從破四胡於韓陵。以軍功,天保初累遷袞州刺史,開府儀同三司。卒,贈都督、瀛州刺史,謚曰莊惠。子孝沖嗣。



 孝沖弟長叉,武平末,侍中、開府儀同三司,封新寧王。隋開皇中,位上柱國,卒於涇州刺史。長叉無他才技,在官以清乾稱。



 步大汗薩,代郡西部人。祖榮,代郡太守。父居,龍驤將軍、領人別將。薩初從爾朱榮入洛。及平葛榮,累功為都督。榮死,又從兆入洛。及韓陵之敗,以所部降神武。稍遷車騎大將軍,封行唐縣公,晉州刺史。齊受禪,改封義陽郡
 公。



 薛脩義,字公讓,河東汾陰人也。曾祖紹,魏七兵尚書。祖壽仁,秦州刺史、汾陰公。父寶集,定陽太守。



 脩義少而姦俠,輕財重氣。魏正光末,天下兵起,特詔募能得三千人者,用為別將。脩義得七千餘人,假安北將軍、西道別將。以軍功,拜龍門鎮將。



 後宗人鳳賢等作亂,圍鎮城,脩義以天下紛擾,遂為逆,自號黃鉞大將軍。詔都督宗正珍孫討之,軍未至,脩義慚悔,遣表乞一大將招慰,乃降。鳳賢等猶據險不降,脩義與書,降之。乃授鳳賢龍驤將軍,陽夏子,改封汾陰縣侯。爾朱榮以脩義反覆,錄送晉陽,
 與高昂等並見拘防。榮赴洛,並以自隨,置於駝牛署。榮死,魏孝莊以脩義為弘農、河北、河東、正平四郡大都督。時神武為晉州刺史,見之,相待甚厚。及韓陵之捷,以脩義行并州事。孝武帝入關,神武以脩義為關右行臺,自龍門濟河,招下西魏北華州刺史薛崇禮。



 初,神武欲大城晉,中外府司馬房毓曰:「若使賊到此處,雖城何益?」乃止。



 及沙苑之敗,徙秦、南汾、東雍三州人於并州,又欲棄晉,以遣家屬向英雄城。脩義諫曰:「若晉州敗,定州亦不可保。」神武怒曰:「爾輩皆負我,前不聽我城并州城,使我無所趣。」脩義曰:「若失守,則請誅。」斛律金曰:「還仰漢小兒
 守,收家口為質,勿與兵馬。」神武從之,以脩義行晉州事。及西魏儀同長孫子彥圍逼城下,脩義開門伏甲侍之。子彥不測虛實,於是遁去。神武嘉之,就拜晉州刺史。



 後除齊州刺史,以黷貨除名。追其守晉州功,復其官爵。俄以軍功,進正平郡公,加開府。天保中,卒於太子太保,贈司空。子文殊嗣。



 脩義從弟嘉族,性亦豪爽。從神武平四胡於韓陵。歷華、陽二州刺史,卒官。



 子震,字文雄,位廉州刺史,亦著軍功。又歷南汾、譙二州刺史。



 慕容儼,字恃德,清都人,廆之後也。容貌出群,衣冠甚偉,不好讀書,頗學兵法。爾朱氏敗,歸神武。以勳,累遷五城
 太守。見東雍州刺史潘相樂,長揖而已。



 丞尉已下,數罹其罪。乃謂儼曰:「府君,少為群下屈節。」儼攘袂曰:「吾狀貌如此,行望人拜,豈可拜人!」神武聞二人在邊不和,徵相樂還朝,以儼代為刺史。



 遷東荊州刺史。行次長社,忽為其部下人所執,將投山賊張儉,為守人王崇祖私放,獲免。神武仍授以軍司,共擊平儉,始得達州。沙苑之敗,時諸州多翻陷,唯儼獲全。



 天保初,以軍功,除開府儀同三司。六年,梁司徒陸法和、儀同宋茝等以郢州城內附。時清河王岳帥師江上,議以城在江外,求忠勇過人者守之。眾推儼,遂遣鎮郢城。始入而梁大都督侯瑱、任約率
 水陸軍奄至城下,於上流鸚鵡洲造荻葓,竟數里,以塞船路。眾懼,儼悅以安之。城中先有神祠一所,俗號城隍神,儼於是順士卒心祈請,須臾,衝風驚波,漂斷荻葓。約復以鐵鎖連緝,防禦彌切。儼還,共祈請,風浪夜驚,葓復斷絕。如此再三,城人大喜,以為神助。儼出城奮擊,大破之。瑱、約又併力圍城。唯煮槐楮葉並糸寧根、水葒、葛、艾等及靴、皮帶、筋角等食之。人死,即火別分食,唯留骸骨。儼猶信賞必罰,分甘同苦。自正月至六月,人無異志。後蕭方智立,請和。文宣以城在江表,有詔還之。及至,望帝悲不自勝。



 帝親執其手,捋儼須,脫帽看髮,歎自久之。曰:「自
 古忠烈,豈過此也!」除趙州刺史。天統四年,別封寄氏縣公,並賜金銀酒鐘各一枚、胡馬一疋。五年,進爵為義安王。武平元年,為光州刺史。儼少從征討,經略雖非所長,而有將帥之節。



 所歷諸州,雖不能清白守道,亦不貪殘害物。卒,贈司徒。



 子子會,位郢州刺史。周武帝平鄴,使其子送敕喻之,子會枷其子,付獄。尋赦書至,云行臺武王已降,子會乃與僚屬北面慟哭,然後奏命。



 爾朱氏將帥歸神武者,又有代人厙狄伏連,字仲山,本名伏憐,語音連。事爾朱榮至直閣將軍。後從神武,賜爵蛇丘男。天保初,儀同三司,尋加開府。性質朴,勤公事,直衛官闕,曉夕
 不離帝所,頗以此見知。然鄙吝愚狠。為鄭州刺史。好聚斂,又嚴酷,居室患蠅,杖門者曰:「何故聽入!」其妻病,以百錢買藥,每日恨之。不識士流。開府參軍,多是衣冠士族,皆加捶撻,逼遣築牆。武平中,封宜都郡王,除領軍大將軍。尋與瑯邪王矯殺和士開,伏誅,被支解。



 伏連家口百餘,盛夏,人料倉米二升,不給鹽菜,常有飢色。冬至日,親表稱賀,其妻為設豆餅。問豆餅得處,云於馬豆中分減。伏連大怒,典馬、掌食人並加杖罰。積年賜物,藏在別庫,遣一婢專掌管籥。每入庫檢閱,必語妻子;此官物,不得輒用。至死時,唯著敝褲;而積絹至二萬疋,簿錄並歸天
 府。



 潘樂,字相貴,廣寧石門人也。本廣宗大族,魏世分鎮北邊,因家焉。父永,有技藝,襲爵廣宗男。樂初生,有一雀止其母左肩,占者咸言富貴之徵,因名相貴,後始為字。及長,寬厚有膽略。初歸葛榮,榮授京兆王,時年十九。榮敗,隨爾朱榮,為別將討元顥,以功封敷城縣男。



 齊神武出牧晉州,引樂為鎮城都將。後從破爾朱兆於廣阿,進爵廣宗縣伯。累以軍功,拜東雍州刺史。神武嘗議欲廢州,樂以東雍地帶山河,境連胡、蜀,形勝之會,不可棄也,遂如故。後從破周師於河陰,議欲追之,令追者在西,不願
 者東,唯樂與劉豐居西。神武善之,以眾之不同而止。改封金門郡公。



 文宣嗣事,鎮河陽,破西將楊等。時帝以懷州刺史平鑒等所築城深入敵境,欲棄之。樂以軹關要害,必須防固,乃更修理,增置兵將而還。還鎮河陽,拜司空。



 齊受禪,樂進璽綬,進封河東郡王,遷司徒。周文東至崤、陜,遣其行臺侯莫陳崇齊子嶺趣軹關;儀同楊從鼓鐘道出建州,陷孤公戍。詔樂總大眾禦之。樂晝夜兼行,至長子,遣儀同韓永興從建州西趣崇,崇遂遁。又為南道大都督討侯景。樂發石鱉,南度百餘里,至梁涇州。涇州舊在石梁,侯景改為淮州。樂獲其地,仍立涇州。
 又克安州之地。除瀛州刺史,仍略淮、漢。天保六年,薨於懸瓠。贈假黃鉞、太師、大司馬、尚書令。



 子子晃嗣。諸將子弟,率多驕縱,子晃沈密謹愨,以清靖自居。尚公主,拜駙馬都尉。武平末,為幽州道行臺右僕射、幽州刺史。周師將入鄴,子晃率突騎數萬赴援。至博陵,知鄴城不守,詣冀州降周齊王軍。授上開府,隋大業初卒。



 彭樂,字興,安定人也。驍勇善騎射。初隨杜洛周賊,知其不立,降爾朱榮。



 從破葛榮於滏口。又為都督,從神武與行臺僕射于暉討破羊侃于瑕丘。後叛投逆賊韓樓,封北平王。及爾朱榮遣大都督侯深擊樓,樂又叛樓降深。
 神武出山東,樂又隨從。韓陵之役,樂先登陷陣,賊眾大崩,封樂城縣公。後以軍功,進爵汨陽郡公,除肆州刺史。天平四年,從神武西討,與周文相拒。神武欲緩持之,樂氣奮請決戰,曰:「我眾賊少,百人取一,差不可失也。」神武從之。樂因醉入深,被刺腸出,內之不盡,截去復戰,身被數創,軍勢遂挫,不利而還。神武每追諭以戒之。高仲密之叛也,周文援之,神武迎擊於芒山。候騎言賊去洛州四十里,蓐食乾飯,神武曰:「自應渴死,保待我殺!」乃勒陣以待之。西軍至皆喉參。樂以數千精騎為右甄,衝西軍北垂,所向奔退,遂馳入周文營。人告樂叛,神武曰:「樂棄
 韓樓事爾朱榮,背爾朱歸我,又叛入西。事成敗豈在一樂?但念小人反覆爾。」俄而西北塵起,樂使告捷,虜西魏臨洮王東、蜀郡王榮宗、江夏王昇、鉅鹿王闡、譙郡王亮、詹事趙善,督將僚佐四十八人,皆係頸反接手,臨以刃,歷兩陣而唱名焉。諸將乘勝,斬首三萬餘。西軍退,神武使樂追之。周文大窘而走,曰:「癡男子!今日無我,明日豈有汝邪?何不急還前營收金寶?」樂從其言,獲周文金帶一束以歸,言周文漏刃破膽矣。神武詰之,樂以周文言對。且曰:「不為此語放之。」神武雖喜其勝,且怒,令伏諸地,親稱其頭,連頓之,並數沙苑之失,舉刀將下者三,噤齘
 良久,乃止。更請五千騎取周文。神武曰:「爾何放而復言捉邪?」取絹三千疋壓樂,因賜之。累遷司徒。天保初,封陳留王,遷太尉。二年,謀反,為前行襄州事劉章等告,伏誅。



 暴顯,字思祖,魏郡斥丘人也。祖喟,仕魏,為朔州刺史,因家焉。父誕,恆州刺史、樂安公。顯幼時,見一沙門指之曰:「此郎子好相表,大必為良將,貴極人臣。」語終失之。顯善騎射,曾從魏孝莊獵,一日中,手獲禽獸七十三。後從齊神武起義信都,累遷北徐州刺史,封屯留公。天保中,以贓貨解州,大理禁止。處判未訖,為合肥被圍,遣顯與步大汗薩等攻梁北徐州,禽其刺史王彊。天統中,累遷,位
 特進,封定陽王,卒。



 皮景和,瑯邪下邳人也。父慶賓,魏淮南王開府中兵參軍。正光中,因使遇亂,遂家廣寧之石門縣。景和少通敏,善騎射。初以親信事神武。後征步落稽,疑賊有伏,令景和將五六騎深入一谷。遇賊百餘人,便戰。景和射數十人,莫不應弦而倒。



 神武嘗令景和射一野豕,一箭獲之,深見賞異。除庫直正都督。天保初,授通州刺史,封永寧縣子。景和趫捷,有武用,從襲庫莫奚,度黃龍,徵契丹,定稽胡,討蠕蠕,每有戰功。累遷殿中尚書、侍中。景和於武職中兼長吏事,又懷識平均,故頻有美授。周通好後,冠
 蓋往來,常令景和對接。每與同射,百發百中,甚見推重。



 武平中,詔獄多令中黃門等監之,恆令景和案覆,據理執正,由是過無枉濫。



 後除特進,封廣漢郡公,遷領軍將軍。瑯邪王之殺和士開,兵指西闕,內外莫知所為。景和請後主出千秋門,自號令。事平,除尚書右僕射。



 陳將吳明徹寇淮南,令景和拒之。除領軍大將軍。封文城郡王。又有平陽人鄭子饒,詐依佛道設齋會,用米面不多,供贍甚廣,密從地藏,漸出餅飯。愚人以為神力,見信於魏、衛之間。將為逆亂,謀泄。乃潛度河聚眾,自號長樂王,已破乘氏縣。景和遣騎擊破之,禽子饒,送鄴烹之。及吳明
 徹圍壽陽,敕景和與賀拔伏恩救之。是時,拒明徹者多傾覆,唯景和全軍而還。除尚書令。武平六年,卒。贈太尉、錄尚書。



 長子信,機悟有風神。位開府儀同三司、武衛將軍,於勳貴子弟中,稱其識鑒。



 降周軍,授上開府、軍正中大夫。隋開皇中,卒於洮州刺史。



 少子宿達,開皇中,通事舍人。母憂起復,將赴京,辭靈,慟哭而絕。久而獲蘇,不能下食,三日而死。



 綦連猛,字武兒,代人也。其先姬姓,六國末,避亂出塞,保祈連山,因以山為姓。北人語訛,故曰綦連。父元成,燕郡太守。猛少有志氣,便弓馬。初為爾朱榮親信。榮被害,從
 爾朱兆入洛。猛父母兄弟皆在山東,爾朱京纏欲投神武,召之與俱。舉槊謂曰:「不從我者死!」乃從之。去城五十餘里,猛以素蒙兆恩,即背京纏復歸兆。兆敗,猛與斛律羌舉、乞伏貴和逃亡。及見獲,各杖一百。以猛配尉景,貴和配婁昭。羌舉以故酋長子,故無所配。既而三人並為神武親信。後都督爾朱文暢將為逆,猛曰:「昔事其父兄,寧今日受死,不忍告而殺之。」神武聞之曰:「事人當如此。」舍其罪而益親之。以軍功,封廣興縣侯。梁使來聘,云求角武藝。



 文襄遣猛就館接之,雙帶兩鞬,左右馳射。校挽彊弓,梁人引弓兩張,皆三石;猛遂併取四張,疊挽之,過
 度。梁人嗟服。天保初,除東秦州刺史。河清三年,加開府。突厥侵逼晉陽,敕猛覘賊。中一騎將超出來鬥,猛即斬之。



 天統五年,除並省尚書令、領軍大將軍,封山陽王。猛自和士開死後,漸預朝政,疑議與奪,咸亦咨稟。趙彥深以猛武將之中,頗疾姦佞,言議時有可采,故引知機事。祖珽奏言猛與彥深前推瑯邪王,事有意故。於是出猛為定州刺史。彥深為西兗州刺史,即日首途。先是,謠曰:「七月刈禾太早,九月啖糕未好,本欲尋山射虎,激箭旁中趙老。」至是,其言乃驗。猛行至牛蘭,有人告和士開被害時,猛亦知情,遂被追還,削王爵,以開府赴州。在任寬
 惠清慎,吏人稱之。淮陰王阿那肱與猛有舊,每欲攜引之,韓長鸞等沮難。復授膠州刺史。後除大將軍。齊亡入周,卒。



 初,猛與尉興慶、謝猥餒並善射小心,給事神武左右。神武使相者視之,曰:「猛大貴,尉、謝無官。」及芒山之役,興慶救神武之窘,為軍所殺。神武嘆曰:「富貴定在天也!」猛竟如相者言,卒以榮寵自畢。



 興慶事見《齊本紀》。興慶每入陣,常自署名於背。神武使求其尸,祭之。於死處立浮圖,世謂高王浮圖云。於是超贈儀同、涇州刺史,謚曰閔壯。



 元景安,河南洛陽人,魏昭成皇帝之五世孫也。高祖虔,
 陳留王。景安沈敏有幹局,少工騎射,善於事人。父永啟迴代郡公授之。隨魏孝武帝西入關。天平末,周、齊交戰,景安臨陣東歸。芒山之戰,以功賜爵西華縣男,代郡公如故。景安妙閑馳騁,有容則,每梁使至,恆與斛律光、皮景和等對客騎射,見者稱善。天保初,別封興勢伯,帶定襄縣令,賜姓高氏,累遷兼七兵尚書。



 時初築長城,鎮戍未立,詔景安與諸將緣塞以備守。督領既多,且所部軍人富於財物,遂賄貨公行。文宣聞之,遣使推檢,唯景安纖毫無犯。帝深嘉歎,乃以所斂贓絹五百匹賜,以彰清節。孝昭嘗與功臣西園宴射,侯去堂一百三十步,中的
 者賜以良馬及金玉錦彩。有一人射中獸頭,去鼻寸餘。唯景安最後,有矢未發。帝令景安解之。景安引滿,正中獸鼻。帝嗟異稱善,特賞馬二匹,玉帛雜物,又加常等。



 天統四年,除豫州刺史,加開府儀同三司。武平三年,授行臺尚書令,刺史如故。封歷陽郡王。景安久在邊州,人物安之。又管內蠻多華少,景安被以恩威,咸得寧輯。武平末,徵拜領軍大將軍。入周,以大將軍、義寧郡公討稽胡,戰沒。



 初,永兄祚襲爵陳留王。祚卒,子景皓嗣。天保時誅諸元親近者,如景安之徙疏宗,議請姓高氏。景皓云:「豈得棄本宗,遂他姓?大丈夫寧可玉碎,不能瓦全。」



 景安以
 白文宣,乃收景皓誅之,家屬徙彭城。由是景安獨賜姓高氏,自外聽從本姓。



 永弟種子豫,字景豫,美容儀,有器幹。景安告景皓慢言,引豫,云相應和。



 豫占云:「爾時以衣袖掩景皓口,云:莫妄言。」問景皓,與豫同,獲免。卒於東徐州刺史。



 獨孤永業,字世基。本姓劉,中山人也,母改適獨孤氏,永業幼隨母,為獨孤家養,遂從其姓。天保初,除中書舍人。永業解書計,善歌舞,甚為文宣所知。後為洛州刺史、河陽行臺左丞,甚有威信。遷行臺尚書。永業久在河陽,善於招撫,周人憚之。姓鯁直,不交權勢。斛律光求二婢,弗
 得,毀之於朝廷。河清末,徵為太僕卿,以乞伏貴和代之,於是西境蹙弱,河洛人情騷動。武平三年,遣永業取斛律豐洛,因以為北道行臺僕射、幽州刺史。河洛人庶多思永業,又除河陽道行臺、洛州刺史。周武帝親攻金墉,永業出兵禦之,問是何達官,作何行動。周人曰:「至尊自來,主人何不出看客?」永業曰:「客行匆匆,故不出看。」乃通夜辦馬槽二千。周人聞之,以為大軍至,乃去。進位開府、臨川王。有甲士三萬。聞晉州敗,請出兵北討,奏寢不報,永業慨憤。又聞並州亦陷,乃遣子須達告降於周。授上柱國、應公。宣政末,為襄州總管。大象二年,為行軍總管
 崔彥睦所殺。



 鮮于世榮,漁陽人也。父寶業,懷朔鎮將。武平初,贈儀同三司、祠部尚書。



 世榮少沈敏,有器幹。興和二年,為神武親信都督,稍遷平西將軍,賜爵石門縣子。



 天統二年,累加開府儀同三司,除鄭州刺史。武平中,以領軍從平高思好,封義陽郡王、領軍大將軍、太子太傅。及周武帝入代,送馬腦酒鐘與之,得便撞破。周兵入鄴,諸將皆降,世榮在三臺之前,獨鳴鼓不輟。及被執不屈,乃見殺。世榮雖武人,無文藝,以朝危政亂,每常竊歎。見徵稅無厭,賞賜過度,發言歎息焉。



 子貞,武平末,假儀同三司。



 傅伏,太安人也。少從戎,以戰功。稍至開府、永橋領人大都督。周武帝前攻河陰,伏自永橋夜入中水單城。南城陷,被圍二旬不下。救兵至,周師還。後除東雍州刺史。周剋晉州,執行臺尉相貴。招伏,伏不從。周剋并州,遺韋孝寬以伏子世寬來招伏,授上大將軍、武鄉郡公,即給告身,以金馬腦二酒鐘為信。伏不受曰:「事君,有死無二。此兒為臣不能竭忠,為子不能盡孝,人所仇疾,願即斬之,以號令天下。」周武自鄴還至晉州,遣高阿那肱等臨汾召伏。伏聞後主已被獲,仰天大哭,率眾入城,於事前,北面哭良久,然後降。周武見曰:「何不早降?」伏流涕曰:「臣三
 世衣食齊家,被任如此,革命不能自死,羞見天地。」周武親執手曰:「為臣當若此。朕平齊,唯見公一人。」乃自食一羊肋,以骨賜伏,曰:「骨親肉疏,所以相付。」遂引與同食。令於侍伯色宿衛,授上儀同,敕之曰:「若即與公高官,恐歸投者心動。勿慮不富貴。」又問:「前救河陰得何官?」曰:「蒙一轉,授特進、永昌郡公。」周武謂後主曰:「朕前三年,決意取河陰,正為傅伏不可動。公當時賞授,何其薄也?」賜伏金酒卮。後以為岷州刺史,尋卒。



 齊軍晉州敗後,兵將罕有全節。有其殺身成仁者,有儀同叱于茍生。鎮南兗州。



 周武破鄴,赦書至,茍生自縊死。



 又有開府、中侍中、宦者田
 敬宣,本字鵬,蠻人也。年十四五,便好讀書。既為閽寺,伺隙便周章詢請。每至文林館,氣喘汗流,問書之外,不暇他語。及視古人節義事,未嘗不感激沈吟。顏之推重其勤學,甚加開獎。後遂通顯。後主之奔青州,遣其西出參伺動靜,為周軍所獲。問齊主何在,紿云已去。毆捶服之。每折一肢,辭色愈厲,竟斷四體而卒。



 又有雷顯和,晉州敗後,為建州道行臺左僕射。周武帝使其子招焉,顯和禁其子而不受。聞鄴城敗,乃降。



 後主失並州,使開府紇奚永安告急於突厥他缽略可汗。及聞齊滅,他缽處永安於吐谷渾使下。永安抗言曰:「本國既敗,永安豈惜賤
 命?欲閉氣自絕,恐天下不知大齊有死節臣。唯乞一刀,以顯示遠近。」他缽嘉之,贈馬七十疋,歸之。



 又有代人高寶寧,武平末,為營州刺史,鎮黃龍。夷夏重其威信。周武帝平齊,遺信招慰,不受敕書。范陽王紹義在突厥中,寶寧上表勸進。范陽王署寶寧為丞相。



 及盧昌期據范陽起兵,寶寧引紹義集夷夏兵數萬救之。至潞河,知周將宇文神舉屠范陽,還據黃龍。



 論曰:爾朱殘逆,遠效誠款,知神武陵逼,隨帝西遷,去就之途,未為失節。



 道元感母兄之戀,荷知遇之恩,思親懷舊,固其宜矣。生不屈西朝,歸誠河朔;保年之於開,義異
 策名。並乘幾獨運,異夫盜寶竊邑者也。神武招攜,理殊納叛;諸將擇木,情非背恩,故能各立功名,終極榮寵。神敬力屈東維,未虧臣節,其被恩化,蓋亦明主之仁焉。劉貴、蔡俊有先見之明,匡贊霸業,配饗清廟,豈徒然也。



 韓賢、尉長命、王懷、任祥、莫多婁貸文、厙狄回洛,厙狄盛、張保洛、侯莫陳相、薛孤延、斛律羌舉、張瓊、宋顯、王則等,並運屬時來。或因羈旅,馮附末光,申其志力,化為王侯,固為宜矣。孝卿功臣之胤,自致公卿,立履之地,亦足稱也。



 慕容紹宗兵機武略,在世見重,昔事爾朱,固執忠義,不用范曾之言,終見烏江之禍。侯景狼戾,固非後主之臣;
 神武遺言,實表知人之鑒。寒山、渦水,往若摧枯,算盡數奇,逢斯禍酷,悲夫!三藏連屬危亡,貞概自處,可謂不隕門節矣。叱列平、步大汗薩、薛脩義、慕容儼、潘樂、彭樂、暴顯、皮景和、綦連猛、元景安等,策名戎幕,備開夷險,位高任重,咸遂本誠。永業、世榮之徒,國危方見忠節,不然,則丹青簡冊,安所貴乎。



\end{pinyinscope}