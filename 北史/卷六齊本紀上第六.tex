\article{卷六齊本紀上第六}

\begin{pinyinscope}

 齊高祖神武皇帝姓高氏,諱歡,字賀六渾,勃海蓚人也。六世祖隱,晉玄菟太守。隱生慶,慶生泰,泰生湖,三世仕慕容氏。及慕容寶敗,國亂。湖率眾歸魏,為右將軍。湖生四子。第三子謐,仕魏,位至侍御史,坐法徙居懷朔鎮。謐生皇考樹生,性通率,不事家業。住居白道南,數有赤光紫氣之異。鄰人以為怪,勸徙居以避之。皇考曰:「安知非吉?」居之自若。及神武生而皇妣韓氏殂,養於同產姊婿鎮獄隊尉景家。神武既累世北邊,故習其俗,遂同鮮卑。長而深沈有大度,輕
 財重士,為豪俠所宗。目有精光,長頭高權,齒白如玉,少有人傑表。家貧,及娉武明皇后,始有馬,得給鎮為隊主。鎮將遼西段長常奇神武貌,謂曰:「君有康濟才,終不徒然。」便以子孫為托。及貴,追贈長司空,擢其子寧而用之。神武自隊主轉為函使。嘗乘驛過建興,雲務書晦,雷聲隨之,半日乃絕,若有神應者。每行道路,往來無風塵
 之
 色。又嘗夢履眾星而行,覺而內喜。為函使六年,每至洛陽,給令史麻祥使。祥嘗以肉啖神武。神武性不立食,坐而進之。祥以為慢己,笞神武四十。



 及自洛陽還,傾產以結客。親故怪問之,答曰:「吾至洛陽,宿衛羽林相率焚領軍張彝宅,朝廷懼其亂而不問,為政若此,事可知也。財物豈可常守邪?」自是乃有澄清天下之志。與懷朔省事雲中司馬子如及秀容人劉貴、中山人賈顯智為奔走之友。



 懷朔戶曹史孫騰、外兵史侯景亦相友結。劉貴嘗得一白鷹,與神武及尉景、蔡俊、子如、賈顯智等獵於沃野。見一赤兔,每搏輙逸,遂至迥澤。澤中有茅屋,將奔入,有狗自屋中出噬之,鷹兔俱死。神武怒,以鳴鏑射之,狗斃。屋中乃有二人出,持神武襟甚急。其母兩目盲,曳杖,呵其二子,曰:「何故觸大家!出甕中
 酒,烹羊以待客。因自言善暗相,遍捫諸人,言皆貴,而指麾俱由神武。又曰:「子如歷位顯,智不善終。」飲竟,出行數里,還更訪之。則本無人居,乃向非人也。由是諸人益加敬異。



 孝昌元年,柔玄鎮人杜洛周反於上谷,神武乃與同志從之。醜其行事,私與尉景、段榮、蔡俊圖之,不果而逃,為其騎所追。文襄及魏永熙后皆幼,武明后於牛上抱負之。文襄屢落牛,神武彎弓將射之以決去,后呼榮求救,賴榮透下取之以免。



 遂奔葛榮,又亡歸爾朱榮於秀容。先是劉貴事榮,盛言神武美,至是始得見。以憔悴故,未之奇也。貴乃為神武更衣,復求見焉。因隨榮之廄,
 廄有惡馬,榮命剪之,神武乃不加羈絆而剪,竟不蹄齧。已而起曰:「御惡人亦如此馬矣。」榮遂坐神武於床下,屏左右而訪時事。神武曰:「聞公有馬十二谷,色別為群,將此竟何用也?」



 榮曰:「但言爾意。」神武曰:「方今天子愚弱,太后淫亂,孽寵擅命,朝政不行。



 以明公雄武,乘時奮發,討鄭儼、徐紇而清帝側,霸業可舉鞭而成。此賀六渾之意也。」榮大悅,語自日中至夜半乃出。自是每參軍謀。後從榮徙據並州,抵揚州邑人龐蒼鷹,止圍焦中。每從外歸,主人遙聞行響動地。蒼鷹母數見圍焦,上赤氣赫然屬天。又蒼鷹嘗夜欲入,有青衣人拔刀叱曰:「何故觸王?」言
 訖不見。始以為異,密覘之。唯見赤蛇蟠床上,乃益驚異。因殺牛分肉,厚以相奉。蒼鷹母求以神武為義子。及得志,以其宅為第,號為南宅。雖門巷開廣,堂宇崇麗,其本所住團焦,以石堊塗之,留而不毀。至文宣時,遂為宮。既而榮以神武為親信都督。于時魏明帝銜鄭儼、徐紇,逼靈太后。未敢制,私使榮舉兵內向。榮以神武為前鋒。至上黨,明帝又私詔停之。及帝暴崩,榮遂入洛。因將篡位,神武諫恐不聽,請鑄像卜之。鑄不成,乃止。孝莊帝立,以定策勛,封銅鞮伯。及爾朱榮擊葛榮,令神武喻下賊別稱王者七人。後與行臺于暉破羊侃于太山。尋與元天
 穆破邢杲于濟南。累遷第三鎮人酋長。嘗在榮帳內,榮嘗問左右曰:「一日無我,誰可主軍?」皆稱爾朱兆。榮曰:「此正可統三千騎以還。堪代我主眾者,唯賀六渾耳。」因誡兆曰:「爾非其匹,終當為其子穿鼻。」乃以神武為晉州刺史。於是大聚斂,因劉貴貨榮下要人,盡得其意。時州庫角無故自鳴,神武異之,無幾而孝莊誅榮。



 及爾朱兆自晉陽將舉兵赴洛,召神武。神武使長史孫騰辭以絳蜀、汾胡欲反,不可委去。兆恨焉。騰復命,神武曰:「兆舉兵犯上,此大賊也,吾不能久事之。」



 自是始有圖兆計。及兆入洛,執莊帝以北。神武聞之大驚。又使孫騰偽賀兆,因密
 覘孝莊所在,將劫以舉義,不果。乃以書喻之,言不宜執天子以受惡名於海內。兆不納,殺帝而與爾朱世隆等立長廣王曄。改元建明,封神武為平陽郡公。及費也頭紇豆陵步籓入秀容,逼晉陽,兆徵神武。神武將往,賀拔焉過兒請緩行以弊之。神武乃往逗留,辭以河無橋,不得渡。步籓軍盛,敗走。初,孝莊之誅爾朱榮,知其黨必有逆謀,乃密敕步籓,令襲其後。步籓既敗兆等,以兵勢日盛,兆又請救於神武。神武內圖兆,復慮步籓後之難除,乃與兆悉力破之,籓死。深德神武,誓為兄弟。時世隆、度律、彥伯共執朝政,天光據關右,兆據並州。仲遠據東
 郡,各擁兵為暴,天下苦之。葛榮眾流入并、肆者二十餘萬,為契胡陵暴,皆不聊生。大小二十六反,誅夷者半,猶草竊不止。兆患之,問計於神武。神武曰:「六鎮反殘,不可盡殺,宜選王素腹心者,私使統焉。若有犯者,罪其帥,則所罪者寡。」兆曰:「善!誰可行也?」賀拔允時在坐,請神武。神武拳毆之,折其一齒,曰:「生平天柱時,奴輩伏處分如鷹犬。今日天下安置在王,而阿鞠泥敢誣下罔上,請殺之。」



 兆以神武為誠,遂以委焉。神武以兆醉,恐醒後或致疑貳,遂出,宣言:「受委統州鎮兵,可集汾東受令。」乃建牙陽曲川,陳部分。有款軍門者,絳巾袍,自稱梗楊驛子,願廁
 左右。訪之,則以力聞,嘗於並州市搤殺人者,乃署為親信。兵士素惡兆而樂神武,於是莫不皆至。



 居無何,又使劉貴請兆。以並、肆頻歲霜旱,降戶掘黃鼠而食之,皆面無穀色,徒污人國土。請令就食山東,待溫飽而處分之。兆從其議。其長史慕容紹宗諫曰:「不可,今四方擾擾,人懷異望,況高公雄略,又握大兵,將不可為。」兆曰:「香火重誓,何所慮邪?」紹宗曰:「親兄弟尚難信,何論香火!」時兆左右已受神武金,因譖紹宗與神武舊隙,兆乃禁紹宗而催神武發。神武乃自晉陽出滏口。路逢爾朱榮妻鄉郡長公主自洛陽來,馬三百匹,盡奪易之。兆聞,乃釋紹宗
 而問焉。



 紹宗曰:「猶掌握中物也。」於是自追神武,至襄垣。會漳水暴長,橋壞。神武隔水拜曰:「所以借公主馬,非有他故,備山東盜耳。王受公主言,自來賜追,今渡河而死,不辭,此眾便叛。」兆自陳無此意,因輕馬渡,與神武坐幕下,陳謝,遂授刀引頭,使神武斫己。神武大哭,曰:「自天柱薨背,賀六渾更何所仰!願大家千萬歲,以申力用。今旁人構間至此,大家何忍復出此言?」兆投刀於地,遂刑白馬而盟,誓為兄弟,留宿夜飲。尉景伏壯士欲執兆,神武嚙臂止之,曰:「今殺之,其黨必奔歸聚結。兵饑馬瘦,不可相支。若英雄屈起,則為害滋甚。不如且置之。



 兆雖勁捷,
 而凶狡無謀,不足圖也。」旦日,兆歸營,又召神武。神武將上馬詣之,孫騰牽衣乃止。隔水肆罵,馳還晉陽。兆心腹念賢領降戶家累別為營。神武偽與之善,觀其佩刀,因取之以殺其從者,盡散。於是士眾咸悅,倍願附從。



 初,魏真君中,內學者奏言上黨有天子氣,云在壺關大王山。武帝於是南巡以厭當之。累石為三封,斬其北鳳皇山以毀其形。後上黨人居晉陽者號上黨坊,神武實居之。及是行,舍大王山,六旬而進。將出滏口,倍加約束,纖毫之物,不聽侵犯。將過麥地,神武輙步牽馬。遠近聞之,皆稱高儀同將兵整肅,益歸心焉。遂前行屯鄴北,求糧於
 相州刺史劉誕,誕不供。有軍營租米,神武自取之。



 魏晉泰元年二月,神武軍次信都,高乾、封隆之開門以待,遂據冀州。是月,爾朱度律廢元曄而立節閔帝。欲羈縻神武,三月,乃白節閔帝,封神武為勃海王,徵使入覲。神武辭。四月癸巳,又加授東道大行臺、第一鎮人酋長。龐蒼鷹自太原來奔,神武以為行臺郎,尋以為安州刺史。神武自向山東,養士繕甲。禁兵侵掠,百姓歸心。乃詐為書,言爾朱兆將以六鎮人配契胡為部曲,眾皆愁。又為並州符,徵兵討步落稽。發萬人將遣之,孫騰、尉景偽請留五日,如此者再。神武親送之郊,雪涕執別。人號慟,哭聲
 動地。神武乃喻之,曰:「與爾俱失鄉客,義同一家,不意在上乃爾徵召!直向西已當死,後軍期又當死,配國人又當死,奈何?」眾曰:「唯有反耳!」神武曰:「反是急計,須推一人為主。」眾願奉神武。神武曰:「爾鄉里難制,不見葛榮乎?雖百萬眾,無刑法,終自灰滅。今以吾為主,當與前異。不得欺漢兒,不得犯軍令,生死任吾,則可。不爾,不能為取笑天下。」眾皆頓顙,死生唯命。神武曰:「若不得已,明日,椎牛饗士,喻以討爾朱兆之意。」



 封隆之進曰:「千載一時,普天幸甚。」神武曰:「討賊,大順也。拯時,大業也。



 吾雖不武,以死繼之,何敢讓焉。」六月庚子,建義於信都,尚未顯背爾朱
 氏。及李元忠與高乾平殷州,斬爾朱羽生首來謁,神武撫膺曰:「今日反決矣!」乃以元忠為殷州刺史。是時,兵威既振,乃抗表罪狀爾朱氏。世隆等祕表不通。八月,爾朱兆攻陷殷州,李元忠來奔。孫騰以為朝廷隔絕,不權立天子,則眾望無所係。十月壬寅,奉章武王融子勃海太守朗為皇帝,年號中興,是為廢帝。時度律、仲遠軍次晉陽,爾朱兆會之。神武用竇泰策,縱反間。度律、仲遠不戰而還,神武乃敗兆於廣阿。十一月,攻鄴。相州刺史劉誕嬰城固守。神武起土山,為地道,往往建大柱,一時焚之,城陷入地。麻祥時為湯陰令,神武呼之曰麻都,祥慚而
 逃。



 永熙元年正月壬午,拔鄴城,據之。廢帝進神武大丞相、柱國大將軍、太師。



 是時,青州建義大都督崔靈珍、大都督耿翔皆遣使歸附,行汾州事劉貴棄城來降。



 閏三月,爾朱天光自長安,兆自並州,度律自洛陽,仲遠自東郡,同會鄴。眾號二十萬,挾洹水而軍。節閔以長孫承業為大行臺,總督焉。神武令封隆之守鄴,自出頓紫陌。時馬不滿二千,步兵不至三萬,眾寡不敵。乃於韓陵為圓陣,連牛驢以塞歸道。於是將士皆為死志,四面赴擊之。爾朱兆責神武以背己。神武曰:「本戮力者,共輔王室,今帝何在?」兆曰:「永安枉害天柱,我報仇耳。」神武曰:「我昔日
 親聞天柱計,汝在戶前立,豈得言不反邪?且以君殺臣,何報之有?今日義絕矣。」乃合戰,大敗之。爾朱兆對慕容紹宗叩心曰:「不用公言,以此。」將輕走,紹宗反旗鳴角,收聚散卒,成軍容而西上。高季式以七騎追奔,度野馬崗,與兆遇。



 高昂望之不見,哭曰:「喪吾弟矣!」夜久,季式還,血滿袖。斛斯椿倍道先據河橋。初,普泰元年十月,歲星、熒惑、鎮星、太白聚於觜、參,色甚明。太史占云,當有王者興。是時,神武起於信都,至是而破兆等。四月,斛斯椿執天光、度律以送洛陽。長孫承業遣都督賈顯智、張歡入洛陽。執世隆、彥伯斬之。兆奔並州。仲遠奔梁州,遂死焉。時
 凶蠹既除,朝廷慶悅。初,未戰之前月,章武人張紹夜中忽被數騎將踰城至一大將軍前,敕紹為軍導向鄴。云佐受命者除殘賊。紹回視之,兵不測,整疾無聲。將至鄴,乃放焉。及戰之日,爾朱氏軍人見陣外士馬四合,蓋神助也。既而神武至洛陽,廢節閔及中興主而立孝武。孝武既即位,授神武大丞相、天柱大將軍、太師,世襲定州刺史,增封并前十五萬戶。神武辭天柱,減戶五萬。



 壬辰,還鄴,魏帝餞於乾脯山,執手而別。七月壬寅,神武帥師北伐爾朱兆。封隆之言,侍中斛斯椿、賀拔勝、賈顯智等往事爾朱,普皆反噬。今在京師寵任,必構禍隙。神武深
 以為然。乃歸天光、度律於京師,斬之。遂自滏口入。爾朱兆大掠晉陽,北保秀容,並州平。神武以晉陽四塞,乃建大丞相府而定居焉。爾朱兆既至秀容,分兵守險,出入寇抄。神武揚聲討之,師出止者數四,兆意怠。神武揣其歲首當宴會,遣竇泰以精騎馳之。一日一夜行三百里,神武以大軍繼之。



 二年正月,竇泰奄至爾朱兆庭。軍人因宴休惰,忽見泰軍,驚走。追破之於赤洪嶺。兆自縊。神武親臨,厚葬之。慕容紹宗以爾朱榮妻子及餘眾自保烏突城。降,神武以義故待之甚厚。



 神武之入洛也,爾朱仲遠部下都督橋
 寧、張子期自滑臺歸命。神武以其助亂,且數反覆,皆斬之。斛斯椿由是內不自安,乃與南陽王寶炬及武衛將軍元毗、魏光祿、王思政構神武於魏帝。舍人元士弼又奏神武受敕大不敬,故魏帝心貳於賀拔岳。



 初,孝明之時,洛下以兩拔相擊,謠言:「銅拔打鐵拔,元家世將末」。好事者以二拔謂拓拔、賀拔,言俱將衰敗之兆。」



 時司空高乾密啟神武,言魏帝之貳。神武封呈,魏帝殺之。又遣東徐州刺史潘紹業密敕長樂太守龐蒼鷹,令殺其弟昂。昂先聞其兄死,以槊刺柱,伏壯士執紹業於路。得敕書於袍領,遂來奔。神武抱其首哭曰:「天子枉害司空。」遽使
 以白武幡勞其家屬。時乾次弟慎在光州,為政嚴猛。又縱部下取納,魏帝使代之。慎聞難,將奔梁。其屬曰:「公家勛重,必不兄弟相反。」乃弊衣推鹿車歸勃海。逢使者,亦來奔。於是魏帝與神武隙矣。阿至羅虜正光以前常稱籓,自魏朝多事,皆叛。神武遣使招納,便附款。先是,詔以寇賊平,罷行臺。至是以殊俗歸降,復授神武大行臺,隨機處分。神武齎其粟帛,議者以為徒費無益。神武不從。撫慰如初。其酋帥吐陳等感恩,皆從指麾。救曹泥,取萬俟受洛干,大收其用。河西費也頭虜紇豆陵伊利居苦池河,恃險擁眾。神武遣長史侯景屢招不從。



 天平元年正月壬辰,神武西伐費也頭虜紇豆陵伊利於河西,滅之。遷其部落於河東。二月,永寧寺九層浮屠災。既而人有從東萊至。云及海上人咸見之於海中,俄而霧起,乃滅。說者以為天意。若曰:「永寧見災,魏不寧矣。飛入東海,勃海應矣。」魏帝既有異圖,時侍中封隆之與孫騰私言。隆之喪妻,魏帝欲妻以從妹。



 騰亦未之信。心害隆之,泄其言於斛斯椿。椿以白魏帝。又孫騰帶仗入省,擅殺御史。並亡來奔。稱魏帝撾舍人梁續於前。光祿少卿元子乾攘臂擊之,謂騰曰:「語爾高王,元家兒拳正如此。」領軍婁昭辭疾歸晉陽。魏帝於是以斛斯椿兼領
 軍,分置督將及河南、關西諸刺史。華山王鷙在徐州,神武使邸珍奪其管籥。建州刺史韓賢、濟州刺史蔡俊皆神武同義,魏帝忌之。故省建州以去賢,使御史中尉綦俊察俊罪,以開府賈顯智為濟州,俊拒之。魏帝逾怒。五月,下詔,云將征句吳,發河南諸州兵,增宿衛,守河橋。六月丁巳,密詔神武曰:「宇文黑獺自平破秦、隴,多求非分,脫有變非常,事資經略。但表啟未全背戾,進討事涉匆匆。遂召群臣,議其可否。僉言假稱南伐,內外戒嚴。一則防黑獺不虞,二則可威吳楚。」時魏帝將伐神武。神武部署將帥,慮疑,故有此詔。神武乃表曰:「荊州綰接蠻左,密
 邇畿服。關隴恃遠,將有逆圖。臣今潛勒兵馬三萬,擬從河東而渡。又遣恆州刺史庫狄干,瀛州刺史郭瓊,汾州刺史斛律金,前武衛大將軍彭樂擬兵四萬,從其來違津渡。



 遣領軍將軍婁昭,相州刺史竇泰,前瀛州刺史堯雄,並州刺史高隆之擬兵五萬,以討荊州。遣冀州刺史尉景,前冀州刺史高敖曹,濟州刺史蔡俊,前侍中封隆之,擬山東兵七萬,突騎五萬,以征江左。皆約勒所部,伏聽處分。」魏帝知覺其變,乃出神武表,命群官議之,欲止神武諸軍。神武乃集在並僚佐,令其博議。還以表聞,仍以信誓自明忠款曰:「臣為嬖佞所間,陛下一旦賜疑,令
 猖狂之罪,爾朱時計。



 臣若不盡誠竭節,敢負陛下,則使身受天殃,子孫殄絕。陛下若垂信赤心,使干戈不動,佞臣一二人,願斟量廢出。」



 辛未,帝復錄在京文武議意,以答神武。使舍人溫子升草敕,子昇逡巡未敢作。



 帝據胡床拔劍作色,子昇乃為敕曰:前持心血,遠以示王,深冀彼此共相禮悉。而不良之徒,坐生間貳。近孫騰倉卒向彼,致使聞者疑有異謀。故遣御史中尉綦俊,具申朕懷。今得王啟,言誓懇惻。



 反覆思之,猶所未解。以朕眇身,遇王武略,不勞尺刃,坐為天子。所謂生我者父母,貴我者高王。今若無事背王,規相攻討,則使身及子孫。還如王
 誓。皇天后土,實聞此言。近慮宇文為亂,賀拔勝應之。故纂嚴,欲與王俱為聲援。宇文今日使者相望,觀其所為,更無異迹。賀拔在南,開拓邊境,為國立功,念無可責。君若欲分討,何以為辭?東南不賓,為日己久。先朝已來,置之度外,今天下戶口減半,未宜窮兵極武。



 朕既闇昧,不知佞人是誰。可列其姓名,令朕知也。如聞庫狄乾語王云:「本欲取懦弱者為主,無事立此長君,使其不可駕御。今但作十五日行,自可廢之,更立餘者。」如此議論,自是王間勳人,豈出佞臣之口?去歲封隆之背叛,今年孫騰逃走,不罪不送,誰不怪王?騰既為禍始,曾無愧懼。王若
 事君盡誠,何不斬送二首?王雖啟圖西去,而四道俱進。或欲南度洛陽,或欲東臨江左。言之者猶應自怪,聞之者寧能不疑?王若守誠不貳,晏然居北,在此雖有百萬之眾,終無圖彼之心。



 王脫信邪棄義,舉旗南指,縱無匹馬隻輪,猶欲奮空拳而爭死。朕本寡德,王已立之,百姓無知,或謂實可。若為他所圖,則彰朕之惡。假令還為王殺,幽辱齏粉,了無遺恨。何者?王既以德見推,以義見舉,一朝背德舍義,便是過有所歸。本望君臣一體,若合符契,不圖今日,分疏到此!古語云:「越人射我,笑而道之;吾兄射我,泣而道之。」朕既親王,情如兄弟,所以投筆拊膺,
 不覺歔欷。



 初,神武自京師將北,以為洛陽久經喪亂,王氣衰盡。雖有山河之固,土地褊狹,不如鄴,請遷都。魏帝曰:「高祖定鼎河洛,為永永之基。經營制度,至世宗乃畢。王既功在社稷,宜遵太和舊事。」神武奉詔。至是,復謀焉。遣兵千騎鎮建興,益河東及濟州兵,於白溝虜船,不聽向洛,諸州和糴粟,運入鄴城。魏帝又敕神武曰:「王若厭伏人情,杜絕物議,唯有歸河東之兵,罷建興之戍,送相州之粟,追濟州之軍,令蔡俊受代,使邸珍出徐。止戈散馬,各事家業。脫須糧廩,別遣轉輸。則讒人結舌,疑悔不生。王高枕太原,朕垂拱京洛,終不舉足渡河,以干戈相
 指。王若馬首南向,問鼎輕重,朕雖無武,欲止不能。必為社稷宗廟,出萬死之策。



 決在於王,非朕能定。為山止簣,相為惜之。」



 魏帝時以任祥為兼尚書左僕射,加開府。祥棄官走至河北,據郡待神武。魏帝乃敕文武官,北來者任去留。下詔罪狀神武,為北伐經營。神武亦勒馬宣告曰:「孤遇爾朱擅權,舉大義於四海。奉戴主上,義貫幽明。橫為斛斯椿讒構,以誠節為逆首。昔趙鞅興晉陽之甲,誅君側惡人。今者南邁,誅椿而已。」以高昂為前鋒,曰:「若用司空言,豈有今日之舉!」司馬子如答神武曰:「本欲立小者,正為此耳。」魏帝徵兵關右。召賀拔勝赴行在所,遣
 大行臺長孫承業、大都督潁川王斌之、斛斯椿共鎮武牢。汝陽王暹鎮石濟,行臺長孫子彥帥前恆農太守元洪略鎮陜,賈顯智率豫州刺史斛斯元壽伐蔡俊。神武使竇泰與左箱大都督莫多婁貸文逆顯智,韓賢逆暹。元壽軍降泰。貸文與顯智遇於長壽津,顯智陰約降,引軍退。軍司元玄覺之,馳還請益師。魏帝遣大都督侯幾紹赴之。戰於滑臺東。顯智以軍降,紹死之。



 七月,魏帝躬率大眾屯河橋。神武至河北十餘里,再遣口申誠款,魏帝不報。



 神武乃引軍度河。魏帝問計於群臣。或云南依賀拔勝,或云西就關中,或云守洛口死戰,未決。而元斌
 之與斛斯椿爭權不睦,斌之棄椿徑還,紿帝云神武兵至。即日,魏帝遜於長安。己酉,神武入洛,停於永寧寺。八月甲寅,召集百官謂曰:「為臣奉主,匡救危亂。若處不諫爭,出不陪隨,緩則耽寵爭榮,急便竄失,臣節安在!」



 遂收開府儀同三司叱列延慶、兼尚書左僕射辛雄、兼吏部尚書崔孝芬、都官尚書劉廞、兼度支尚書楊機、散騎常侍元士弼,並殺之,誅其貳也。士弼籍沒家口。



 神武以萬機不可曠廢,乃與百僚議。以清河王亶為大司馬,居尚書下舍而承制決事焉。王稱警蹕,神武醜之。神武尋至弘農,遂西剋潼關,執毛洪賓。進軍長城,龍門都督薛崇
 禮降。神武退舍河東,命行臺尚書長史薛瑜守潼關。大都督庫狄溫守封陵。於蒲津西岸築城守華州。以薛紹宗為刺史。高昂行豫州事。神武自發晉陽至此,凡四十啟,魏帝皆不答。



 九月庚寅,神武還至洛陽。乃遣僧道榮奉表關中,又不答。乃集百寮沙門耆老,議所推立。以為自孝昌衰亂,國統中絕,神主靡依,昭穆失序。永安以孝文為伯考。



 永熙遷孝明於夾室。業喪祚短,職此之由。遂議立清河王世子善見。議定,白清河王。王曰:「天子無父,茍使兒立,不惜餘生。」乃立之,是為孝靜帝。魏於是始分為二。



 神武以孝武既西,恐逼崤陜,洛陽復在河外,接近
 梁境。如向晉陽,形勢不能相接。依議遷鄴。護軍祖瑩贊焉。詔下三日,車駕便發,戶四十萬,狼狽就道。神武留洛陽部分,事畢還晉陽。自是軍國政務,皆歸相府。先是童謠曰:「可憐青雀子,飛來鄴城裏。羽翮垂欲成,化作鸚鵡子。」好事者竊言,雀子謂魏帝清河王,鸚鵡謂神武也。初,孝昌中,山胡劉蠡升自稱天子,年號神嘉,居雲陽谷。西土歲被其寇,謂之胡荒。



 二年正月,西魏渭州刺史可朱渾道元擁眾內屬,神武迎納之。壬戌,神武襲擊劉蠡升,大破之。己巳,魏帝褒詔,以神武為相國,假黃鉞,劍履上殿,入朝不趨。



 神武固辭。
 三月,神武欲以女妻蠡升太子,候其不設備。辛酉,潛師襲之。其北部王斬蠡升首以送。其眾復立其子南海王。神武進擊之,又獲南海王,及其弟西海王、北海王、皇后、公卿已下四百餘人,胡、魏五萬戶。壬申,神武朝于鄴。四月,神武請給遷人廩各有差。九月甲寅,神武以州、郡、縣官多乖法,請出使問人疾苦。



 三年正月甲子,神武帥庫狄乾等萬騎襲西魏夏州。身不火食,四日而至。縛槊為悌,夜入其城。擒其刺史費也頭賀拔俄彌突,因而用之。留都督張瓊以鎮守,遷其部落五千戶以歸。西魏靈州刺史曹泥,與其婿涼州刺史
 劉豐,遣使請內屬。周文圍泥,水灌其城,不沒者四尺。神武命阿至羅發騎三萬,徑度靈州,繞出西軍後。



 獲馬五十匹。西師乃退。神武率騎迎泥、豐生,拔其遺戶五千以歸,復泥官爵。魏帝詔加神武九錫,固讓,乃止。二月,神武令阿至羅逼西魏秦州刺史建忠王萬俟普撥,神武以眾應之。六月甲午,普撥與其子太宰受洛干、豳州刺史叱干寶樂、右衛將軍破六韓常及督將三百餘人,擁部來降。八月丁亥,神武請均斗尺,班於天下。



 九月辛亥,汾州胡王迢觸、曹貳龍聚眾反。署立百官,年號平都,神武討平之。十二月丁丑,神武自晉陽西討,遣兼僕射行臺、
 汝陽王暹、司徒高昂等趣上洛。大都督竇泰入自潼關。



 四年正月癸丑,竇泰軍敗自殺。神武軍次蒲津,以冰薄不得赴救。乃班師。高昂攻剋上洛。二月乙酉,神武以并、肆、汾、建、晉、東雍、南汾、秦、陜九州霜旱,人饑流散,請所在開倉振給。六月壬申,神武如天池。獲瑞石,隱起成文曰「六王三川」十一月壬辰,神武西討。自蒲津濟,眾二十萬。周文軍於沙苑。神武以地厄少卻,西人鼓噪而進。軍大亂,棄器甲十有八萬。神武跨駝,候船以歸。



 元象元年三月辛酉,神武固請解丞相,魏帝許之。四月庚寅,神武朝於鄴。壬辰,還晉陽。請開酒禁,并振恤宿衛
 武官。七月壬午,行臺侯景、司徒高昂圍西魏將獨孤信於金墉。西魏帝及周文並來赴救。大都督庫狄乾帥諸將前驅,神武總眾繼進。八月辛卯,戰於河陰,大破西魏軍,俘獲數萬。司徒高昂、大都督李猛、宗顯死之。西師之敗,獨孤信先入關,周文留其都督長孫子彥守金墉,遂燒營以遁。神武遣兵追奔至崤,不及而還。初,神武知西師來侵,自晉陽率眾馳赴。至孟津,未濟,而軍有勝負。既而神武渡河,子彥亦棄城走。神武遂毀金墉而還。十一月庚午,神武朝於京師。十二月壬辰,還晉陽。



 興和元年七月丁丑,魏帝進神武為相國、錄尚書事。固
 讓,乃止。十一月乙丑,神武以新宮成,朝於鄴。魏帝與神武宴射,神武降階下稱賀。又辭勃海王及都督中外諸軍事,詔不許。十二月戊戌,神武還晉陽。



 二年十二月,阿至羅別部遣使請降,神武帥眾迎之,出武州塞,不見。大獵而還。



 三年五月,神武巡北境,使使與蠕蠕通和。



 四年五月辛巳,神武朝于鄴。請令百官,每月面敷政事。明揚仄陋,納諫屏邪。



 親理獄訟,褒黜勤怠。牧守有愆,節級相坐。椒掖之內,進御以序。後園鷹犬,悉皆棄之。六月甲辰,神武還晉陽。九月,神武西征。十月己亥,圍西魏儀
 同三司王思政於玉壁城。欲以致敵,西師不敢出。十一月癸未,神武以大雪,士卒多死,乃班師。



 武定元年二月壬申,北豫州刺史高慎據武牢西叛。三月壬辰,周文率眾援高慎,圍河橋南城。戊申,神武大敗之於芒山。禽西魏督將以下四百餘人,俘斬六萬計。



 是時,軍士有盜殺驢者,軍令應死。神武弗殺,將至並州決之。明日,復戰,奔西軍,告神武所在,西師盡銳來攻。眾潰,神武失馬,赫連陽順下馬,以授神武,與蒼頭馮文洛扶上,俱走。從者步騎六七人。追騎至,親信都督尉興慶曰:「王去矣,興慶腰邊百箭,足殺百人。」神武勉之曰:「事濟,以
 爾為懷州;若死,則用爾子。」



 興慶曰:「兒小,願用兄。」許之。興慶斗,矢盡而死。西魏太師賀拔勝以十三騎逐神武,河州刺史劉洪徽射中其二。勝槊將中神武,段孝先橫射勝馬殪,遂免。豫、洛二州平,神武使劉豐追奔徇地,至恆農而還。七月,神武貽周文書,責以殺孝武之罪。八月辛未,魏帝詔神武為相國、錄尚書事、大行臺,餘如故。固辭,乃止。



 是月,神武命於肆州北山築城,西自馬陵戍,東至土隥,四十日罷。十二月己卯,神武朝於京師。庚辰,還晉陽。



 二年三月癸巳,神武巡行冀、定二州,因朝京師。以冬春
 亢旱,請蠲縣責,振窮乏,宥死罪以下。又請授老人板職各有差。四月丙辰,神武還晉陽。十一月,神武討山胡,破平之。俘獲一萬餘戶,分配諸州。



 三年正月甲午,開府儀同三司爾朱文暢、開府司馬任胄、都督鄭仲禮、中府主簿李世林、前開府參軍房子遠等謀賊神武。因十五日夜打蔟,懷刃而入。其黨薛季孝以告,並伏誅。丁未,神武請於并州置晉陽宮,以處配口。三月乙未,神武朝鄴。



 丙午,還晉陽。十月丁卯,神武上言,幽、安、定三州北接奚、蠕蠕,請於險要脩立城戍以防之。躬自臨履,莫不嚴固。乙未,神武請釋芒山俘桎梏,配以
 人間寡婦。



 四年八月癸巳,神武將西伐,自鄴會兵於晉陽。殿中將軍曹魏祖曰:「不可,今八月西方王,以死氣逆生氣,為客不利,主人則可。兵果行,傷大將。」神武不從。自東西魏構兵,鄴下每先有黃黑螘陣鬥。占者以為黃者東魏戎衣色,黑者西魏戎衣色,人間以此候勝負。是時黃螘盡死。九月,神武圍玉壁以挑西師,不敢應。



 西魏晉州刺史韋孝寬守玉壁。城中出鐵面。神武使兀盜射之,每中其目。用李業興孤虛術,萃其北。北,天險也。乃起土山,鑿十道。又於東面鑿二十一道,以攻之。



 城中無水,汲於汾。神武
 使移汾,一夜而畢。孝寬奪據土山。頓軍五旬,城不拔,死者七萬人,聚為一冢。有星墜於神武營,眾驢並鳴,士皆懾懼。神武有疾。十一月庚子,輿疾班師。庚戌,遣太原公洋鎮鄴。辛亥,徵世子澄至晉陽。有惡鳥集於亭樹,世子使斛律光射殺之。己卯,神武以無功,表解都督中外諸軍事。魏帝優詔許焉。是時,西魏言神武中弩。神武聞之,乃勉坐見諸貴。使斛律金敕勒歌,神武自和之,哀感流涕。



 侯景素輕世子,嘗謂司馬子如曰:「王在,吾不敢有異。王無,吾不能與鮮卑小兒共事。」子如掩其口。至是,世子為神武書,召景。景先與神武約,得書,書背微點,乃來。書
 至,無點,景不至。又聞神武疾,遂擁兵自固。神武謂世子曰:「我雖疾,爾面更有餘憂色,何也?」世子未對。又問曰:「豈非憂侯景叛邪?」



 曰:「然。」神武曰:「景專制河南十四年矣,常有飛揚跋扈志。顧我能養,豈為汝駕御也。今四方未定,勿遽發哀。庫狄乾鮮卑老公,斛律金敕勒老公,並性遒直,終不負汝。可朱渾道元、劉豐生遠來投我,必無異心。賀拔焉過兒樸實無罪過,潘相樂今本作道人,心和厚,汝兄弟當得其力。韓軌少戇,宜寬借之。彭相樂心腹難得,宜防護之。少堪敵侯景者,唯有慕容紹宗。我故不貴之,留以與汝,宜深加殊禮,委以經略。



 五年正月朔,日蝕。神武曰:「日蝕其為我邪?死亦何恨。」丙午,陳啟於魏帝。是日,崩於晉陽,時年五十二。祕不發喪。六月壬午,魏帝於東堂舉哀三日。



 制緦衰,詔凶禮依漢大將軍霍光、東平王蒼故事。贈假黃鉞、使持節、相國、都督中外諸軍事、齊王璽紱、輬車、黃屋左纛、前後羽葆鼓吹、輕車介士、兼備九錫殊禮。謚獻武王。八月甲申,葬於鄴西北漳水之西,魏帝臨送於紫陽。天保初,追崇為獻武帝。廟號太祖,陵曰義平。天統元年,改謚神武皇帝,廟號高祖。神武性深密高岸,終日儼然,人不能測。機權之際,變化若神。至於軍國大略,獨運懷抱。



 文武將吏,罕
 有預之。經馭軍眾,法令嚴肅,臨敵制勝,策出無方。聽斷昭察,不可欺犯,知人好士,全護勳舊。性周給,每有文教,常殷勤款悉。指事論心,不尚綺靡。擢人授任,在於得才。茍其所堪,乃至拔於廝養;有虛聲無實者,稀見任用。



 諸將出討,奉行方略,罔不克捷。違失指畫,多致奔亡。雅尚儉素,刀劍鞍勒無金玉之飾。少能劇飲,自當大任,不過三爵。居家如官。仁恕愛士。始范陽盧景裕以明經稱,魯郡韓毅以工書顯,咸以謀逆見禽,並蒙恩置之第館,教授諸子。其文武之士,盡節所事見執獲而不罪者甚多。故遐邇歸心,皆思效力。至南和梁國,北懷蠕蠕。吐谷渾、
 阿至羅咸所招納,獲其力用,規略遠矣。



 世宗文襄皇帝諱澄,字子惠,神武長子也。母曰婁太后。生而岐嶷,神武異之。



 魏中興元年,立為勃海王世子。就杜詢講學,敏悟過人,詢甚嘆服。二年,加侍中、開府儀同三司,尚孝靜帝妹馮翊長公主。時年十二,神情俊爽,便若成人。神武試問以時事得失,辨析無不中理。自是軍國籌策皆預之。



 天平元年,加使持節、尚書令、大行臺、並州刺史。三年,入輔朝政,加領軍左右、京畿大都督。時人雖聞器識,猶以少年期之。而機略嚴明,事無疑滯,於是朝野振肅。



 元象元年,攝吏部尚書。魏自崔亮以後,選人
 常以年勞為制。文襄乃釐改前式,銓擢唯在得人。又沙汰尚書郎,妙選人地以充之。至于才名之士,咸被薦擢。假有未居顯位者,皆致之門下,以為賓客。每山園游宴,必見招攜;執射賦詩,各盡其所長,以為娛適。



 興和二年,加大將軍,領中書監,仍攝吏部尚書。自正光已後,天下多事。在任群官,廉潔者寡。文襄乃奏吏部郎崔暹為御史中尉,糾劾權豪,無所縱捨。於是風俗更始,私枉路絕。乃榜於街衢,具論經國政術,仍開直言之路。有論事上書苦言切至者,皆優容之。



 武定四年十一月,神武西討,不豫,班師。文襄馳赴軍所,侍衛還晉陽。



 五年正月丙午,
 神武崩,祕不發喪。辛亥,司徒侯景據河南反,潁州刺史司馬世雲以城應之。景誘執豫州刺史高元成、襄州刺史李密、廣州刺史暴顯等。遣司空韓軌率眾討之。四月壬申,文襄朝于鄴。六月己巳,韓軌等自潁州班師。丁丑,文襄還晉陽,乃發喪,告喻文武,陳神武遺志。七月戊戌,魏帝詔以文襄為使持節、大丞相、都督中外諸軍、錄尚書事、大行臺、勃海王。文襄啟辭位,願停王爵。壬寅,魏帝詔太原公洋攝理軍國,遣中使敦喻。八月戊辰,文襄啟申神武遺令,請減國邑,分封將督各有差。辛未,朝于鄴,固辭丞相。魏帝詔曰:「既朝野攸憑,安危所繫,不得令遂
 本懷,須有權奪。可復前大將軍,餘如故。」壬辰,尚書祠部郎中元瑾、梁降人茍濟、長秋卿劉思逸及淮南王宣洪、華山王大器、濟北王徽等謀害文襄,事發伏誅。九月己亥,文襄請舊勛灼然未蒙齒錄者,悉求旌賞。朝士名行有聞,或以年耆疾滿告謝者,準其本秩。授以州郡,不得蒞事,聽蔭子孫。自天平元年以來,遇事亡官者,聽復本資。豪貴之家,不得占護山澤。其第宇車服婚姻送葬奢僭無限者,並令禁斷。從太昌元年以來,將帥有殊功異效者,其子弟年十歲以上,請聽依第出身。其兵士從征,身殞陣場者,蠲其家租課。若有藏器避世者,以禮招致,
 隨才擢敘。罷營構之官。在朝百司,怠惰不勤,有所曠廢者,免所居官。若清幹克濟,皎然可知者,即宜超敘,不拘常式。辛丑,文襄還晉陽。



 武定六年正月己未,文襄朝于鄴。二月己卯,梁遣使慰文襄,並請通和。文襄許其和而不答書。侯景之叛也,南兗州刺史石長宣頗相影響,諸州刺史、守、令、佐史多被詿誤。景破後,悉被禽獲,尚書咸處極刑,文襄並請減降。於是斬長宣,其餘並從寬宥。三月戊申,文襄請朝臣及牧、守、令、長各舉賢良及驍武膽略堪守邊城者,務在得才,不拘職素。其稱事六品、散官五品以上,朝廷所悉,不在舉限。



 其稱事七品、散官六品
 以下,并及州、郡、縣雜白身,不限在官、解職,並任舉之,隨才進擢。辛亥,文襄南臨黎陽,濟於武牢。自洛陽,從太行而反晉陽。於路遺書朝士,以相戒厲。於是朝野承風,莫不震肅。六月,文襄巡北邊城戍,振賜各有差。



 七月乙卯,文襄朝于鄴。八月庚寅,還晉陽。使大行臺慕容紹宗與太尉高岳、大都督劉豐討王思政於潁川。先是,文襄遣行臺尚書辛術率諸將略江淮之北。至是,凡所獲二十三州。



 七年四月甲辰,魏帝進文襄位相國,封齊王,綠綟綬。贊拜不名,入朝不趨,劍履上殿。食冀州之勃海、長樂、安德、武邑、瀛州之河間五郡,邑十五萬戶,使持節、都督
 中外諸軍事、錄尚書、大行臺並如故。丁未,文襄入朝。固讓,魏帝不許。五月戊寅,文襄帥師自鄴赴潁川。六月丙申克潁川,禽西魏大將軍王思政,以忠於所事,釋而待之。七月,文襄朝於鄴,請魏帝立皇太子,復辭爵位殊禮,未報。



 八月辛卯,遇盜而崩。初,梁將蘭欽子京見虜,文襄以配廚。欽求贖之,不許。京再訴,文襄使監廚蒼頭薛豐洛杖之,曰:「更訴,當殺汝。」京與其黨六人謀作亂。



 時文襄將受魏禪,與陳元康、崔季舒屏左右謀于北城東柏堂。太史啟言宰輔星甚微,變不一月。時京將進食,文襄卻之,謂人曰:「昨夜夢此奴斫我。」又曰:「急殺卻。」京聞之,置刀
 於盤下,冒言進食。文襄見之,怒曰:「我未索食,何遽來?」



 京揮刀曰:「將殺汝!」文襄自投,傷足,入床下。賊黨至,去床,因見弒,時年二十九。祕不發喪。明年正月辛酉,魏帝舉哀於太極東堂。詔贈物八萬段,凶事依漢大將軍霍光、東平王蒼故事。贈假黃鉞、使持節、相國、都督中外諸軍事、齊王璽紱,巉輬車、黃屋左纛、後部羽葆鼓吹、輕車介士,備九錫禮,謚曰文襄王。二月甲申,葬於義平陵之北。天保初,追尊曰文襄皇帝,廟號世宗,陵曰峻成。



 文襄美姿容,善言笑,談謔之際,從容弘雅。性聰警,多籌策,當朝作相,聽斷如流。愛士好賢,待之以禮,有神武之風焉。然少
 壯氣猛,嚴峻刑法。高慎西叛,侯景南翻,非直本懷狼戾,兼亦有懼威略。情欲奢淫,動乖制度。嘗於宮西造宅,墻院高廣,聽事宏壯,亞太極殿。神武入朝,責之,乃止。



 論曰:昔魏氏失馭,中原蕩析。齊神武爰從晉部,大號冀方。屢戰而翦兇徒,一麾以清京洛。尊主匡國,功濟天下。既而魏武帝規避權逼,歷數既盡,適所以速關、河之分焉。文襄嗣膺霸道,威略昭著。內除奸逆,外拓淮夷,擯斥貪殘,存情人物。而志在峻法,急於御下,於前王之德,有所未同。蓋天意人心,好生惡殺,雖吉兇報應,未皆影響。總而論之,積善多慶。然文襄之禍生所忽,蓋有由焉。



\end{pinyinscope}