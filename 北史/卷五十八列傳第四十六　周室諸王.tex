\article{卷五十八列傳第四十六 周室諸王}

\begin{pinyinscope}

 文帝
 十三
 王孝閔帝一王明帝二王武帝六王宣帝二王周文帝十三子:姚夫人生明帝。後宮生宋獻公震。文元皇后生孝閔皇帝。文宣叱奴皇后生武帝、衛剌王直。達步妃生齊煬王憲。王姬生趙僭王招。後宮生譙孝王儉、
 陳惑王純、越野王盛、代紘王達、冀康公通、滕聞王逌。



 宋獻公震,字彌俄突,幼而敏達。大統十六年,封武邑公,尚魏文帝女。其年薨。保定元年,追贈大司馬,封宋國公。



 無子,以明帝第三子實嗣。建德三年,進爵為王。大象中,為大前疑,尋為隋文帝害,國除。



 衛剌王直,字豆羅突。魏恭帝三年,封秦郡公。武成初,進封衛國公,歷雍州牧、大司空、襄州總管。直,武帝母弟也,性浮詭。以晉公護執政,遂貳於帝而暱護。及南討軍敗,慍於免黜,又請帝除護。帝宿有誅護意,遂與直謀之。及護誅,帝以齊王憲為大冢宰。直既乖本望,又請為大司
 馬,欲擅威權。帝知其意,謂曰:「汝兄弟長幼有序,何反居下列也?」以為大司徒。建德三年,進爵為王。



 初,帝以直第為東宮,更使直自擇所居。直歷觀府署,無稱意者,至廢陟屺佛寺,遂欲居之。齊王憲謂曰:「弟兒女成長,此寺偏小,詎是所宜?」直曰:「一身尚不自容,何論兒女!」憲怪而疑之。直嘗從帝校獵而亂行,帝怒,對眾撻之。



 自是,憤怨滋甚。及帝幸雲陽宮,直在京師反,攻肅章門,司武尉遲運閉門,不得入,退走。追至荊州獲之,免為庶人,囚諸宮中。尋有異志,及其子十人並誅之,國除。



 齊煬王憲,字毗賀突。性通敏,有度量。初封涪城縣公。少
 與武帝俱受詩傳,咸綜機要,得其指歸。文帝嘗賜諸子良馬,唯其所擇。憲獨取駁者。帝問之,對曰:「此馬色類既殊,或多駿逸。若從軍征伐,牧圉易分。」帝喜曰:「此兒智識不凡,當成重器。」後從上隴,經官馬牧,文帝每見駁馬,輒曰「此我兒馬也」,命取以賜之。魏恭帝元年,進封安城郡公。明帝即位,授大將軍。



 武成初,除益州總管,進封齊國公。初,平蜀之後,文帝以其形勝之地,不欲使宿將居之。諸子中欲有推擇,偏問武帝以下,誰欲此行,並未及對,而憲先請。



 文帝曰:「刺史當撫眾臨人,非爾所及。以年授者,當歸爾兄。」憲曰:「才用殊不關大小,試而無效,甘受面
 欺。」文帝以憲年尚幼,未之遣。明帝追遵先旨,故有此授。憲時年十六,善於撫綏,留心政術,辭訟輻湊,聽受不疲。蜀人悅之,共立碑頌德。



 保定中,徵拜雍州牧。及晉公護東伐,以尉遲迥為前鋒,圍洛陽。齊兵數萬,奄出軍後,諸軍恇駭,並各退散。唯憲與王雄、達奚武拒之,而雄為齊人所敗,三軍震懼。憲親自督勵,眾心乃安。時晉公護執政,雅相親委,賞罰之際,皆得預焉。



 天和三年,以憲為大司馬,行小塚宰,雍州牧如故。四年,齊將獨孤永業來寇,詔憲與柱國李穆出宜陽,築崇德等五城,絕其糧道。齊將斛律明月築壘洛南。五年,憲涉洛邀之,明月遁走。是
 歲,明月又於汾北築城,西至龍門。晉公護問計於憲,憲曰:「兄宜暫出同州為威容,憲請以精兵居前,隨機攻取。」六年,憲率眾出自龍門,齊將新蔡王康德潛軍宵遁。憲乃度河,攻其伏龍等四城,二日盡拔。又攻張壁,克之。斛律明月時在華容,弗能救,乃北攻姚襄城,陷之。汾州又見圍日久,憲遣柱國宇文盛運粟饋之。憲自入兩乳谷,襲克齊伯杜城。使柱國譚公會築石殿城以為汾州之援。齊平原王段孝先、蘭陵王高長恭引兵大至,大將軍韓歡為齊人所乘,遂退。憲身自督戰,齊眾稍卻。會日暮,乃各收軍。



 及晉公護誅,武帝召憲入,免冠拜謝。帝謂曰:「
 汝親則同氣,休戚共之,事不相涉,何煩致謝?」乃詔憲往護第,收兵符及諸簿籍等。尋以憲為大冢宰。時帝既誅宰臣,親覽朝政,方欲齊之以刑,爰及親親,亦為刻薄。憲既為護所任,自天和後,威勢漸隆。護欲有所陳,多令憲奏。其間或有可不,憲慮主相嫌隙,每曲而暢之。帝亦悉其此心,故得無患。然猶以威名過重,終不能平,雖遷授冢宰,實奪其權也。開府裴文舉,憲之侍讀,帝嘗御內殿引見,謂曰:「昔魏末不綱,太祖匡輔元氏;有周受命,晉公復執威權。積習生常,便謂法應須爾。豈有三十歲天子可為人所制乎?且近代以來,又有一弊,暫經隸屬,便即
 禮若君臣,此乃亂時權宜,非經國之術。爾雖陪侍齊公,不得即同臣主。且太祖十兒,寧可悉為天子?卿宜規以正道,無令兄弟自致嫌疑。」文舉再拜而出,歸以白憲。憲指心撫几曰:「吾心公寧不悉?但當盡忠竭節耳,知復何言!」



 建德三年,進爵為王。寧友劉休徵獻王箴一首,憲美之。休征後又以箴上之,帝方翦削諸弟,甚悅其文。憲嘗以兵書繁廣,自刊為要略五篇,至是表陳之。帝覽而稱善。



 其秋,帝於雲陽寢疾,衛王直於京師。帝召憲謂曰:「汝為前軍,吾亦續發。」



 直尋敗走。帝至京師,憲與趙王招俱入拜謝。帝曰:「管、蔡為戮,周公作輔,人心不同,有如
 其面。但愧兄弟親尋干戈,於我為不能耳。」初,直內忌憲,憲隱而容之,且以帝母弟,每加友敬。晉公護之誅也,直固請及憲。帝曰:「齊公心迹,吾自悉之,不得更有所疑。」及文宣皇后崩,直又密啟憲飲酒食肉與平昔不異。帝曰:「吾與齊王異生,俱非正嫡,特為吾意,今袒括是同。汝當愧之,何論得失。



 汝親太后之子,但須自助。」直乃止。



 四年,帝將東討,獨與內史王誼謀之,餘人莫知。後以諸弟才略,無出憲右,遂告之。憲即贊成其事。及大軍將出,憲表上金寶等一十六件以助軍資。詔不納,以憲表示公卿曰:「人臣當如此,朕貴其心耳,寧資此物。」乃詔寧為前軍,
 趣黎陽。帝親圍河陰,未剋。憲攻拔武濟,進圍洛口,拔其東西二城。以帝疾班師。是歲,初置上柱國,以憲為之。



 五年,大舉東討,憲復為前鋒,守雀鼠谷。帝親圍晉州,憲進克洪洞、永安二城,更圖進取。齊主聞晉州見圍,自來援之。時陳王純頓千里徑,大將軍永是公椿屯雞棲原,大將軍宇文盛守汾水關,並受憲節度。憲密謂椿曰:「捕者詭道,汝今為營,不須張幕,可伐柏為庵,示有處所。令兵去之後,賊猶致疑。」時齊主分軍萬人向千里徑,又令其眾出汾水關,自率大兵與椿對。宇文盛馳告急,憲自救之,齊人遽退。盛與柱國侯莫陳芮逐之,多有斬獲。俄
 而椿告齊眾稍逼,憲又救之。會椿被敕追還,率兵夜反。齊人果謂柏庵為帳幕,不疑軍退,翌日始悟。時帝已去晉州,留憲後拒。憲阻水為陣。齊領軍段暢至橋。憲隔水問暢姓名,暢曰:「領軍段暢也,公復為誰?」憲曰:「我虞候大都督耳。」暢曰:「觀公言語,不是凡人,何用隱名位。」憲乃曰:「我齊王也。」偏指陳王純已下,並以告之。暢鞭馬去,憲即命旋軍。齊人遽追之,戈甲甚銳。憲與開府宇文忻為殿拒之,斬其驍將賀蘭豹子、山褥環等,齊眾乃退。



 帝又命憲援晉州。齊主攻圍晉州,帝次于高顯,憲率所部先向晉州。明日諸軍總集,稍逼城下。齊人大陣於營南,帝召憲
 馳往觀之。憲反命曰:「請破之而後食。」



 帝悅。既而諸軍俱進,應時大潰,齊主遁走。齊人復據高壁及洛女,帝命憲攻洛女,破之。齊主已走鄴,留其安德王延宗據並州。帝進圍其城,憲攻其西面,剋之。延宗遁走,追而獲之。以功進封第二子安城公質為河間王,拜第三子賓為大將軍。仍詔憲趣鄴,進剋鄴城。



 憲善兵謀,長於撫御,摧鋒陷陣,為士卒先。齊人聞風,憚其勇略。齊任城王湝、廣寧王孝珩等守信都,復詔憲討之。仍令齊主手書招湝,湝不納。憲軍過趙州,湝令間諜二人覘,候騎執以白憲。憲乃集齊舊將,偏將示之曰:「吾所爭者大,不在汝等。」即放還,
 令充使,乃與湝書。憲至信都,湝陣於城南,登張耳冢望之。



 俄而湝所署領軍尉相願偽出略陣,遂降,湝殺其妻子。明日擒湝及孝珩等。



 先是稽胡劉沒鐸自稱皇帝,又詔憲督趙王招等平之。



 憲自以威名日重,潛思屏退。及帝欲親征北蕃,乃辭以疾。尋而帝崩,宣帝嗣位,以憲屬尊望重,深忌之。時尚未葬,諸王在內居服。司衛長孫覽總兵輔政,恐諸王有異志,奏令開府于智察其動靜。及山陵還,帝又命智就宅候憲,因是告憲有謀。帝遣小冢宰宇文孝伯謂憲曰:「今欲以叔為太師,九叔為太傅,十一叔為太保,何如?」憲辭以才輕。孝伯返命,復來曰:「詔王
 晚共諸王俱入。」既至殿門,憲獨被引進。帝先伏壯士於別室,至即執之。憲辭色不撓,固自陳說。帝使于智對憲。



 憲目光如炬,與智相質。或曰:「以王今日事勢,何用多言!」憲曰:「我位重屬尊,一旦至此,死生有命,寧復圖存?但老母在堂,恐留慈恨耳。」因擲笏於地,乃縊之。時年三十五。帝以于智為柱國,封齊國公。又殺上大將軍安邑公王興、上開府獨孤熊、開府豆盧紹等,皆以暱於憲也。帝既誅憲,無以為辭,故託興等與憲結謀,遂加戮焉。時人知其冤酷,咸云伴憲死也。



 憲所生達步干氏,蠕蠕人也。建德三年,上冊為齊國太妃。憲有至性,事母以孝聞。太妃舊
 患,屢經發動,憲衣不解帶,扶持左右。憲或東西從役,每心驚,母必有疾,乃馳使參問,果如所慮。六子,貴、質、賨、貢、乾禧、乾洽。



 貴字乾福,少聰敏,尤便騎射。始讀孝經,便謂人曰:「讀此一經,足為立身之本。」十歲,封安定郡公。文帝始封此郡,未嘗假人,至是封焉。年十一,從憲獵於監州,一圍中,手射野馬及鹿一十有五。建德二年,拜齊國世子。後出為豳州刺史。貴雖出自深宮,而留心庶政。性聰敏,過目輒記,嘗道逢二人,謂其左右曰:「此人是縣黨,何因輒行?」左右不識,貴便說其姓名,莫不嗟伏。白獸烽經為商人所燒,烽帥受貨,不言其罪。他日,此帥隨例來參,
 貴乃問云:「商人燒烽,何因私放?」烽帥愕然,遂即首伏。其明察如此。卒時年十七,武帝甚痛惜之。



 質字乾祐,以憲勛封河間郡王。賓字乾禮,中壩公。貢出後莒莊公,乾禧,安城公。乾洽,龍涸公。並與憲俱被誅。



 趙僭王招,字豆盧突。幼聰穎,博涉群書,好屬文,學庾信體,詞多輕艷。魏恭帝三年,封正平郡公。武城初,進封趙國公。歷益州總管、大司空、大司馬,進爵為王,除雍州牧。建德五年,從東伐,以功進位上柱國。又與齊王憲討平稽胡,斬賊帥劉沒鐸。宣政中,拜太師。大象元年,詔以洺州襄國郡邑萬戶為趙王國,招出就國。二年,宣帝不豫,
 徵招及陳、越、代、滕五王赴闕。比招等至而帝已崩。



 隋文帝輔政,加招等殊禮,入朝不趨,劍履上殿。



 隋文帝將遷周鼎,招密欲圖之,以匡社稷。乃要隋文帝至第,飲於寢室。招子員、貫及妃弟魯封、所親人史胄皆先在左右,佩刀而立。又藏兵刃於帷席間,後院亦伏壯士。隋文帝從者多在合外,惟楊弘、元胄胄弟威及陶徹坐戶側。招屢以佩刀割瓜啖隋文,隋文未之疑。元胄覺變,扣刀而入。招乃以大觴親飲胄酒,又命胄向廚取漿。胄不為之動。滕王逌後至,隋文降階迎,胄因得耳語曰:「公宜速出。」



 隋文共逌等就坐,須臾辭出。後事覺,陷以謀反,其年秋,誅招
 及其子德廣公員、永康王貫、越公乾銑、弟乾鏗等,國除。



 招所著文集十卷。



 譙孝王儉,字侯幼突。武成初,封譙國公。建德三年,進為王。從平鄴,拜大冢宰。薨,子乾惲嗣,為隋文帝所害,國除。



 陳惑王純,字堙智突。武成初,封陳國公。保定中,使突厥迎皇后,歷秦、陜二州總管。建德三年,進爵為王。從平齊,進位上柱國。歷並州總管、雍州牧、太傅。大象元年,詔以濟南郡邑萬戶為陳國,純出就國。二年,朝京師,並其子為隋文帝所害,國除。



 越野王盛,字立久突。武成初,封越國公。建德三年,進爵
 為王。從平齊,進位上柱國。歷相州總管、大冢宰。大象元年,遷大前疑、太保。其年,詔以豐州武當、安昌二郡,邑萬戶為越國,盛出就國。二年,朝京師,並其子為隋文帝所害,國除。



 代奕王達,字度斤突。性果決,善騎射。武成初,封代國公。建德初,進位柱國。出為荊州刺史,有政績,武帝手敕褒美之。所管禮州刺史蔡澤黷貨被訟。達以其勳庸,不可加戮,若曲法貸之,又非奉上之體,乃令所司精加案劾,密表奏之。



 事竟得釋,終亦不言。其處事周慎如此。雅好節儉,食無兼膳,侍姬不過數四,皆衣綈衣。又未嘗營產,
 國無儲積。左右嘗以為言。達曰:「君子憂道不憂貧,何煩於此。」三年,進為王。從平齊。齊淑妃馮氏尤為齊後主所幸,見獲,帝以達不邇聲色,特以馮氏賜之。宣帝即位,進上柱國。大象元年,拜大右弼。其年,詔以潞州上黨郡邑萬戶為代國,達出就國。二年,朝京師,及其子為隋文帝所害,國除。



 冀康公通,字屈率突。武成初,封冀國公。薨,子絢嗣。建德三年,進為王。



 大定中,亦為隋文帝所害。國除。



 滕聞王逌,字爾固突。少好經史,解屬文。武成初,封滕國公。建德三年,進爵為王。宣政元年,進位上柱國。大象元
 年,詔以荊州新野郡邑萬戶為滕國,逌出就國。三年,朝京師,為隋文帝所害,並其子,國除。



 逌所著文章頗行於世。



 孝閔帝一男:陸夫人生紀厲王康,字乾安。保定初,封紀國公。建德三年,進爵為王,出為利州總管。康驕侈無度,遂有異謀,司錄裴融諫,康殺之。五年,詔賜康死。子湜嗣,大定中,為隋文帝所害,國除。



 明帝三男:徐妃生畢剌王賢。後宮生豐王貞、宋王實。實出後宋獻公震。



 畢剌王賢,字乾陽。保定四年,封畢公。建德三年,進爵為
 王。歷荊州總管、大司空。大象初,進上柱國、雍州牧、太師。明年,宣帝崩。賢性強濟,有威略,慮隋文帝傾覆宗祐。言泄,並其子被害,國除。



 豐王貞,字乾雅。初封豐國公,建德三年,進爵為王。大象初,為大冢宰。大定中,並其子為隋文帝所害,國除。



 武帝七男:李皇后生宣帝、漢王贊。庫汗姬生秦王贄、曹王允。馮姬生道王充。



 薛世婦生蔡王兌。鄭姬生荊王元。



 漢王贊,字乾依。初封漢國公,建德三年,進爵為王。大象末,隋文帝輔政,欲順物情,乃進贊位上柱國,拜右大丞相。外示尊崇,實無所綜理。轉太師。尋及秦王贄、曹王允、
 道王充、蔡王兌荊王元並為隋文帝所害,國除。



 宣帝三子:朱皇后生靜皇帝。王姬生萊王衍。皇甫姬生郢王術。衍及術並大象二年封,並為隋文帝所害,國除。



 論曰:昔賢之議者,咸以周建五等,歷載八百;秦立郡縣,二世而亡。雖得失之迹可尋,是非之理互起,而因循莫變,復古未聞。良由著論者溺於貴遠,司契者難於易業,詳求適變之道,並未窮於至當也。嘗試論之:夫皇王迭興,為國之道匪一;聖賢間出,立德之指殊塗。斯豈故為相反哉,亦云為政而已矣。何則?五等之制,行於商、周之前;郡縣之設,始於秦、漢之後。論時則澆淳理隔,易地則
 用捨或殊。譬猶干戚日用,難以成垓下之業;稷嗣所述,不可施成周之朝。是知因時制宜者,為政之上務也;觀人立教才,經國之長策也。且夫裂封疆,建侯伯,擇賢能,署牧守,循名雖曰異軫,責實抑亦同歸。盛則與之共安,衰則與之共患。共安繫乎善惡,非禮義無以敦風共患寄以存亡,非甲兵不能靖亂。是以齊、晉帥禮,鼎業傾而復振;溫、陶釋位,王綱弛而更張。然則周之列國,非一姓也,晉之群臣,非一族也,豈齊、晉忠於列國,溫、陶賢於群臣哉?蓋位重者易以立功,權輕者難以盡節故也。由斯言之,建侯置守,乃古今之異術;兵權爵位,蓋安危之所
 階乎。周文之初定關右,日不暇給,既以人臣禮終,未遑蕃屏之事。晉蕩輔政,爰樹其黨,宗室長幼,並握兵權,雖海內謝隆平之風而國家有盤石之固矣。武皇克翦芒刺,思弘政術,懲專朝之為患,忘維城之遠圖,外崇寵任,內結猜阻。自是配天之基,潛有朽壤之墟矣。宣皇嗣位,凶暴是崇,芟刈先其本枝,削黜偏於公族。以齊王之奇姿傑出,足可牢籠於前載。處周公之地,居上將之重,肋冠俗,攻戰如神,敵國繫以存亡,鼎命由其輕重。屬道消之日,挾震主之威,斯人而嬰斯戮,君子是以知國祚之不永也。其餘雖地惟叔父,親則同生,假文能輔主,武
 能威敵,莫不謝卿士於當年,從侯服於郡國,號為千乘,位侔匹夫。是以權臣乘其機,謀士因其隙,遷龜鼎速於俯拾,殲王侯烈於燎原,悠悠邃古,未聞茲酷。豈非摧枯振朽,易為力乎?向使宣皇擇姬、劉之制,覽聖哲之術,分命賢戚,布於內外,料其輕重,間以親疏,首尾相持,遠近為用,使其位足以扶危,其權不能為亂,事業既定,僥幸自息,雖使臥赤子,朝委裘,社稷固以久安,憶兆可以無患矣。何後族之地而能窺其神器哉。



 昔張耳、陳餘,賓客廝役,所居皆取卿相,而齊王之文武僚吏,其後亦多臺牧,異代相符,可謂賢矣哉。



\end{pinyinscope}