\article{卷五十九列傳第四十七}

\begin{pinyinscope}

 寇洛趙貴從祖兄善李賢子詢崇孫敏弟遠穆穆子渾梁禦子睿寇洛,上谷昌平人也。累世為將吏。父延壽,魏和平中,以良家子鎮武川,因家焉。洛性明辯,不拘小節。賀拔岳西征,洛與岳鄉里,乃募從入關。以功封安鄉縣子。及岳為大行臺,以洛為右都督。侯莫陳悅既害岳,欲並其眾。時
 初喪元帥,洛於諸將中最為舊齒,素為眾信,乃收集將士,志在復讎。既至原州,眾推洛為盟主,統岳之眾,至平涼。周文帝至,以洛為右都督。從討侯莫陳悅,平之。拜涇州刺史。大統初,詔加開府,進爵京兆郡公,封洛母宋為襄城郡君。四年,鎮東雍州。



 五年,卒於鎮,贈太尉、尚書令,謚曰武。



 子和嗣。明帝二年,錄舊勛,以洛配享文帝廟庭,賜和姓若引氏,改封松陽郡公。



 趙貴,字元寶,天水南安人也。祖仁,以良家子鎮武川,因家焉。貴少有節概,爾朱榮以為別將,從討元顥有功,賜爵燕樂縣子。從賀拔岳平關中,累遷大都督。



 岳為侯莫
 陳悅所害,將吏奔敗,莫有守者。謂其黨曰:「吾聞仁義豈有常哉,行之則為君子,違之則為小人。朱伯厚、王脩感意氣微恩,尚能蹈履名節,況吾等荷賀拔公國士之遇,寧可自同眾人乎?」因涕泣噓唏,從之者五十人。乃詣悅詐降,悅信之。因請收葬岳,言辭慷慨,悅壯而許之。貴乃收岳屍還營,與寇洛等奔平涼,共圖拒悅。貴乃首議迎周文帝。周文至,以貴為大都督,領府司馬。悅平,行秦州事。



 後以預立魏文帝勳,進爵為公。梁簋定稱亂河右,以貴為隴西行臺討破之。



 從復弘農,戰沙苑,進爵中山郡公。河橋之戰,貴與怡峰為左軍,戰不利,先還。



 及高仲密
 以北豫州降,周文迎之,與東魏人戰於芒山。貴為左軍,失律,坐免官。



 尋復官爵。後拜柱國大將軍,賜姓乙弗氏。六官建,為太保、大宗伯,改封南陽郡公。周孝閔帝踐阼,遷大塚宰,進封楚國公,邑萬戶。



 初,貴與獨孤信等皆與文帝等夷。及晉公護攝政,貴自以元勛,每懷怏怏,與信謀殺護,為開府宇文盛告,被誅。



 善,字僧慶,貴之從祖兄也。少好學,美容儀,沉毅有遠量。爾朱天光討邢杲、萬俟醜奴,以為長史。普泰初,為大行臺尚書,封山北縣伯。天光拒齊神武於韓陵,敗,見殺。善請收葬其屍,齊神武義而許之。賀拔岳總關中,迎善,復
 以為長史。



 岳為侯莫陳悅所殺,善共諸將翊戴周文帝。魏孝武西遷,改封襄城縣伯。歷位尚書左右僕射,進爵為公。善性溫恭,有器局,雖位居端右,而愈自謙退。其職務克舉,則曰某官之力;有罪責,則曰善之咎也。時人稱其有公輔量。



 大統九年,從戰芒山,屬大軍不利,善為敵所獲,卒於東魏。建德初,周、齊通好,齊人乃歸其柩。其子詢表請贈謚。詔贈大將軍、大都督、四州諸軍事、岐州刺史,謚曰敬。



 李賢,字賢和,自云隴西成紀人,漢騎都尉陵之後也。陵沒匈奴,子孫因居北狄。後隨魏南遷,復歸水幵、隴。曾祖富,
 魏太武時以子都督討兩山屠各,歿於陣,贈寧西將軍、隴西郡守。大統末,以賢兄弟著勳,追贈司空公。



 賢幼有志節,不妄舉動。嘗出遊,逢一老人,鬢眉皓白,謂曰:「我年八十,觀士多矣,未有如卿。卿必為台牧,努力勉之。」九歲,從師受業,略觀大指而已。



 或譏其不精,答曰:「賢豈能領徒授業?至如忠孝之道,實銘於心。」問者慚服。



 十四遭父憂,撫訓諸弟,友愛甚篤。



 魏永安中,萬俊醜奴據岐、涇等州反,孝莊遣爾朱天光擊破之。天光令都督長孫邪利行原州事,以賢為主簿。累遷高平令。賀拔岳為侯莫陳悅所害,周文帝西征,賢與其弟遠、穆等密應侯莫陳崇。以功
 授都督,仍守原州。及大軍至秦州,悅棄城走。周文命兄子導追之,以賢為先鋒,至牽屯山及之。以功授假節、撫軍將軍、大都督。



 魏孝武西遷,周文令賢率騎迎衛,封上邽縣公。俄授左大都督,還鎮原州。大統二年,州人豆盧狼害都督大野樹兒等,據州城反。賢率敢死士一戰敗之,狼斬關遁走,賢追斬之。八年,授原州刺史。周文之奉魏太子西巡,至原州,遂幸賢第,讓齒而坐,行鄉飲酒禮。後帝復至原州,令賢乘路車,備儀服,以諸侯會遇禮相見。



 然後幸賢第,歡宴終日,凡是親族,頒賜有差。恭帝元年,進爵西河郡公。後以弟子植被誅,賢坐除名。保定二
 年,詔復賢官爵,仍授瓜州刺史。



 武帝及齊王憲之在襁褓,不利居宮中,周文令於賢家處之,六載乃還宮。因賜賢妻吳姓宇文氏,養為姪女,賜與甚厚。及武帝西巡原州,幸賢第,詔曰:「朕昔沖幼,爰寓此州。使持節、驃騎大將軍、開府儀同三司、大都督、瓜州諸軍事、瓜州刺史賢,斯土良家,勳德兼著,受委居朕,輔導積年。念其規弼,功勞甚茂。今巡撫屆此,不殊代邑,舉目依然,益增舊想。賢雖無屬籍,朕處之若親,凡厥昆季,乃至子姪等,可並預宴賜。」於是令中侍上士尉遲愷往瓜州,降璽書勞賢。賜衣一襲及被褥,並御所服十三環金帶一腰、中廄馬
 一疋、金裝鞍勒、雜綵五百段、銀錢一萬。賜賢弟申國公穆亦如之。子侄男女中外諸孫三十四人各賜衣一襲。拜賢甥庫狄樂為儀同。賢門生昔經侍奉者,二人授大都督,四人授帥都督,六人別將。奴已免賤者五人,授軍主;未免賤者十二人,酬替放之。



 四年,王師東討,西道空虛,慮羌、渾侵擾,乃授賢河州總管。河州舊非總管,至是創置。賢乃大營屯田,以省運漕,多設斥候,以備寇戎,於是羌、渾斂迹。五年,宕昌寇邊,乃於洮州置總管府以鎮遏之。遂廢河州總管,改授賢洮州總管。屬羌寇侵擾,賢頻破之,虜遂震懾,不敢犯塞。俄廢洮州總管,還於河州置
 總管府,復以賢為之。



 武帝思賢舊恩,徵拜大將軍。於京師薨,帝親入,哀動左右。贈使持節、柱國大將軍、大都督、十州諸軍事、原州刺史,謚曰桓。子端嗣。



 端位開府儀同三司,從平齊,戰沒,贈上大將軍,追封襄陽公,謚曰果。



 端弟吉,儀同三司。



 吉弟孝軌,開府儀同大將軍、升遷縣伯,後封奇章公。孝軌弟詢。



 詢,字孝詢,深沉有大略,頗涉書記。仕周,累遷司衛上士。武帝幸雲陽宮,委以留府事。衛王直作亂,焚肅章門,詢於內益火,故賊不得入。武帝善之。累遷英果中大夫,屢以軍功,加位大將軍,賜爵平高郡公。隋文帝為丞相,尉
 遲迥作亂,遣韋孝寬擊之,以詢為元帥長史,委以心膂。軍至永橋,諸將不一。詢密啟請重臣監護。文帝令高熲監軍。與熲同心,唯詢而已。及迥平,進位上柱國,改封隴西郡公。開皇初,歷位隰州總管,以疾徵還京師。卒,帝悼惜者久之,謚曰襄。子元方嗣。



 詢弟崇,字永隆,英果有籌算,膽力過人。周元年,以父勛,封回樂縣侯。時年尚小,拜爵日,親族相賀,宗獨泣下。賢問之,對曰:「無勳於國,幼少封侯,當報主恩,不得終於孝養,是以悲耳。」賢由此大奇之。起家州主簿,非其好也,辭不就職,求為將兵都督。隨宇文護伐齊,以功最,授儀
 同三司。歷位少侍伯大夫、少承御大夫,攝太子宮正。周武平齊,引參謀議,以勛加授開府,封襄陽縣公,尋改封廣宗縣公。



 隋文帝為丞相,加授上開府儀同大將軍、懷州刺史,進爵郡公。尉遲迥反,遣使招之。崇初欲相應,後知叔父穆以并州附文帝,慨然太息曰:「合家富貴數十人,遇國有難,竟不能扶傾繼絕,何面目處天地間乎!」韋孝寬亦疑之,與俱臥起。其兄詢時為元帥長史,每諷諭之。崇由是亦歸心焉。及迥平,授徐州總管,進位上柱國。



 開皇三年,除幽州總管。突厥犯塞,崇輒破之。奚、霄、契丹等讋嚇其威略,爭來內附。後突厥大為侵掠,崇率步騎三
 千拒之。轉戰十餘日,師人多死,遂保于沙城。突厥圍之,死亡略盡。突厥欲降之,謂曰:「降者封為特勤。」崇知不免,令其士卒曰:「吾喪師徒,罪當萬死,今效命以謝國家。看吾死,且可降賊,方便散走。還見至尊,道此意也。」乃挺刃突賊,復殺二人,沒於陣。主州諸軍事、豫州刺史,謚曰壯。子敏嗣。



 敏字樹生,文帝以其父死王事,養於宮中。及長,襲爵廣宗公,起家左千牛。



 美姿容,善騎射,工歌舞弦管。開皇初,周宣帝后樂平公主有女娥英,妙擇婚對,敕貴公子弟集弘聖宮者,日以百數。公主選取敏,禮儀如尚帝女。後
 將侍宴,公主謂敏曰:「我以天下與至尊,唯一女夫,當為汝求柱國。若授餘官,慎無謝。」及進見上,上親御琵琶,遣敏歌舞,大悅,謂公主曰:「敏何官?」對曰:「一白丁耳。」謂敏曰:「今授儀同。」敏不答。上曰:「不滿爾意耶?今授開府。」又不謝。上曰:「公主有大功於我,我何得向其女婿惜官,今授卿柱國。」敏乃拜而蹈舞。遂於坐發詔授柱國,以本官宿衛。



 後避煬帝諱,改封經城縣公。歷豳、金、華、岐數州刺史,多不蒞職,常留京師。往來宮內,侍從遊宴,賞賜超於功臣。大業初,轉衛尉卿。樂平公主將薨,遺言於煬帝「妾唯一女,不自憂死,深憐之。湯沐乞迴與敏。」帝從之,竟食五
 千戶。



 攝屯衛將軍。楊玄感反後,城闕大興,敏之策也。將作監。從征高麗,領新城道軍,加光祿大夫。十年,帝復征遼東,遣敏黎陽督運。



 時或言敏一名洪兒,帝疑「洪」字當讖,嘗面告之,冀其引決。敏由是大懼,數與金才、善衡等屏人私語。宇文述知而奏之,竟與渾同誅。其妻宇文氏尋亦賜鴆而終。



 賢弟遠。遠字萬歲,幼有器局,嘗與群兒為戰鬥戲,指麾便有軍陣之法。郡守見而異之,召使更戲。群兒散走,遠持杖叱之,復為向陣,意氣雄壯,殆甚於前。



 郡守曰:「此小兒必為將帥,非常人也。」



 及長,涉獵書傳。魏正光末,天下鼎沸,敕
 勒賊胡琮侵逼原州。遠昆季率勵鄉人,欲圖拒守,而眾情頗有異同。遠乃按劍喻以節義,因曰:「有異議者,請斬之。」



 眾懼,乃聽命,相與盟歃,深壁自守。無援,城隱,其徒多被害,唯遠兄弟並為人所匿,得免。遠乃使賢晦迹和光,潛身間行,入朝求援。魏朝嘉之,授武騎常侍,俄轉別將。及爾朱天光西伐,配遠精兵為鄉導。天光欽遠才望,除為長城郡守。後以應侯莫陳崇功。遷高平郡守。周文見面悅之,令居麾下。



 及魏孝武西遷,封安定縣伯。魏文帝嗣位之始,思享遐年,以遠字可嘉,令扶帝升殿。進爵為公,仍領左右。從征竇泰,復弘農,並有殊勳。授都督、
 原州刺史。



 周文謂遠曰:「孤有卿,若身之有臂。本州之榮,乃私事爾。」遂令遠兄賢代行州事。沙苑之役,遠功居最,進爵陽平郡公。尋除大丞相府司馬,參軍國機務。時河東初復,人情未安。周文以河東為國之要領,乃授河東郡守。遠敦獎風俗,勸課農桑,肅遏姦非,兼修守之備。曾未期月,百姓懷之。周文降書勞問。徵為侍中,遷太子少師。



 東魏北豫州刺史高仲密請舉州來附,周文以仲密所據遼遠,難為應接。諸將皆憚此行。遠曰:「北豫遠在賊境,高歡又屯兵河陽,常理而論,實難救援。但不入獸穴,不得獸子,若以奇兵出其不意,事或可濟。脫有利鈍,
 故是兵家之常。如其顧望不行,便無克定之日。」周文喜曰:「李萬歲所言,差強人意。」乃授行臺尚書,前驅東出。周文率大軍繼進。遠乃潛師而往,拔仲密以歸。仍從周文戰於芒山,時大軍不利,遠獨整所部為殿。



 尋授都督義州弘農等二十一郡諸軍事。遠善撫馭,有幹略,戰守之備,無不精銳。每厚撫境外之人,使為間諜,敵中動靜,必先知之。至有事泄被誅,亦不以為悔。嘗獵於莎柵,見石於叢薄中,以為伏兔,射之,鏃入寸餘,視之乃石。周文聞面異之,賜書曰:「昔李將軍親有此事,公今復爾,可謂世載其德矣。」東魏將段孝先趣宜陽,以送糧為名,實有窺
 窬之意。遠密知其計,遣兵襲破之。孝先遁走。



 周文賜所乘主金帶床帳衣被等,并彩二千匹,拜大將軍。頃之,除尚書左僕射,固辭。周文不許,遠不得已,方拜職。周文又以第十一子代王達令遠子之,其見親待如此。



 時周文嫡嗣未建,明居長,已有成德;孝閔處嫡,年尚幼沖。乃謂群公曰:「孤欲立子以嫡,恐大司馬有疑。」大司馬即獨孤信,明帝敬后父也。眾未有答。



 遠曰:「立子以嫡不以長,略陽公為嗣,公何疑焉?若以人為嫌,請即斬信。」便起拔劍。周文亦起曰:「何事至此!」信又自陳產,遠乃止。於是群公並從遠議。



 遠出外,拜謝信曰:「臨大事不得不爾。」信
 亦謝遠曰:「今日賴公決此大議。」



 六官建,授小司寇。周孝閔帝踐祚,進位柱國大將軍,復鎮弘農。



 遠子植,文帝時已為相府司錄,參掌朝政。及晉公護執權,密欲誅護,頗泄,護乃出植為梁州刺史。尋而廢帝,召遠及植還朝。遠恐有變,沉吟良久,乃曰:「大丈夫寧為忠鬼,安能作叛臣乎!」遂就徵,至京師。護以遠功名素重,猶欲全宥之,謂曰:「公兒遂有異謀,可早為之所。」乃以植付遠。遠素愛植,植又有口辯,云初無此謀。遠信之,詰朝將植謁護。護謂植已死,乃曰:「陽平公何意自來?」



 左右云:「植亦在門外。」護大怒曰:「陽平公不信我矣!」召入,命遠同坐,令帝與植相質
 於遠前。植辭窮,謂帝曰:「本為此謀,欲安社稷,利至尊耳。今日至此,何事云云。」遠聞之,自投於床,曰:「若爾,誠合萬死。」於是護乃害植,並逼遠自殺。建德元年,晉公護誅,贈本官,加太保,謚曰忠。隋開皇初,追贈上柱國,改謚曰懷。植及諸弟並加謚。



 植弟基,字仲和,幼有聲譽,美容儀,善談論,涉獵群書,尤工騎射。周文令尚義歸公主。以父勛,封建安縣公。累遷大都督,進爵清河郡公。及魏廢帝即位之後,猜隙彌深。時周文諸子年皆幼沖,章武公導、中山公護復東西作鎮,唯託意諸婿,以為心膂。基與義城公李暉、常山公于翼等俱為武衛將軍,分掌禁旅。魏
 帝深憚之,故密謀泄。魏恭帝即位,進爵敦煌郡公,尋進位驃騎大將軍、開府儀同三司,拜陽平國世子。六官建,授御正中夫。



 周孝閔帝踐阼,出為浙州刺史。尋為兄植,合坐死。以王婿,又為季父穆所請,得免。武成二年,除江州刺史。既被譴謫,常憂憤不得志。保定元年,卒於位。穆尤所鍾愛,每哭輒悲慟,謂所親曰:「好兒捨我去,門戶豈是欲興!」宣政元年,追贈使持節、上開府儀同大將軍、曹徐譙三州刺史、敦煌郡公,謚曰孝。子威嗣。



 威字安人,又改襲遠爵陽平郡公,加上開府。大象末,地至柱國,封公。



 賢弟穆,
 字顯慶,少明敏有度量。文帝入關,便給事左右,深被親遇。穆亦小心謹肅,未嘗懈怠。及侯莫陳悅害賀拔岳,周文自夏州赴難,而悅黨史歸據原州,猶為悅守。周文令侯莫陳崇襲之,穆時先在城中,與兄賢、遠應崇,遂禽歸。以功授都督。從迎魏孝武,封永平縣子。又領鄉兵。禽竇泰,復弘農,並有戰功。沙苑之捷,穆言:「歡今日已喪膽矣,請速逐之,則歡可禽也。」周文不聽。論前後功,進爵國公。



 芒山之戰,周文馬中流矢,驚逸墜地。敵人追及,左右皆散。穆下馬,以策擊周文背,因大罵曰:「籠陳軍士,爾曹主何在?爾獨住此!」敵人見其輕侮,不疑是貴人,遂捨而
 過。穆以馬授周文,遂俱逸。是日微穆,周文已不濟矣。既而與穆相對而泣,自是恩盼更隆。顧左右曰:「成我事者,其此人乎!」擢授武衛將軍、儀同三司,進封安武郡公。前後賞賜,不可勝計。周文歎其忠節,曰:「人所貴唯命,穆遂輕命濟孤,爵位玉帛,未足為報。」乃特賜鐵券,恕以十死。進驃騎大將軍、開府儀同三司、侍中。初,芒山之敗,穆授周文馬,後中廄此色者,悉以賜之。又賜穆嗣子惇安樂郡公,姊一人為郡君,自餘姊妹並為縣君,兄弟子姪及緦麻已上親并舅氏皆沾厚賜。其褒崇如此。



 從解玉壁圍,拜安定國中尉。歷同州刺史、太僕卿。從於謹平
 江陵,以功別封一子長城縣侯。尋進位大將軍,賜姓拓拔氏。又擊曲沔蠻破之。俄除原州刺史,拜世子惇為儀同三司,以賢子為平高郡守,遠子為平高縣令,並加鼓吹。穆自以叔姪一家三人皆牧宰鄉里,恩遇過隆,固辭不拜。周文不許。後人為雍州刺史,兼小塚宰。周孝閔帝踐阼,又封一子為升遷縣伯。穆請迴授賢子孝軌,許之。



 及兄子植謀害宇文護被誅,穆亦坐除名。先是穆知植非保家主,每勸遠除之,遠不能用。及遠臨刑,泣謂穆曰:「顯慶,吾不用汝言以至此,將奈何!」穆以此獲免,及其子弟亦免官。時植弟基當從坐戮,穆求以子惇、怡等代死,
 辭理酸切,聞者莫不動容。護矜之,遂特免基死。



 明帝即位,拜驃騎大將軍、開府儀同三司、大都督,復爵安武郡公,拜直州刺史。武成中,子弟免官爵者悉復之。累遷大司空。天和二年,進封申國公,舊爵回授一子。建德元年,遷太保,尋出為原州總管。四年,武帝東征,令穆別攻軹關及河北諸縣,並破之。後以帝疾班師,棄而不守。六年,進位上柱國,除并州總管。



 時東夏初平,人情尚擾,穆靖以鎮守,百姓懷之。大象元年,加邑至九千戶,遷大左輔,總管如舊。二年,詔加太傅,仍總管。



 及隋文作相,尉遲迥舉兵,遣使招穆,穆鎖其使,上其書。穆子士榮以穆所居
 天下精兵處,陰勸穆應之。穆弗聽,曰:「周德既衰,愚智共悉,天時若此,豈能違天?」乃遣使謁隋文帝,並上十三環金帶,蓋天子服也,以微申其意。時迥子誼為朔州刺史,亦執送京師。迥今其署行臺韓長業攻陷潞州,執刺史趙威,署城人郭子勝為刺史。穆遣兵討獲子勝。文帝嘉之,以穆勞同破鄴城第一勳,加三轉,聽分授其二子榮、才及賢子孝軌。榮及才並儀同大將軍,孝軌進開府儀同大將軍,又別封子雄為密國公。穆又密表勸進。文帝既受禪,詔曰:「公既舊德,且又父黨,敬惠來旨,便以今月十三日恭膺天命。」俄而穆來朝,文帝降座禮之。拜太師,
 贊拜不名,真食成安縣三千戶。穆子孫雖在襁褓,悉拜儀同,其一門執象笏者百餘人,貴盛當時無比。穆上表乞骸骨,詔曰:「公年既耆舊,筋力難煩,今勒所司敬蠲朝集。如有大事,須共謀謨,別遣侍臣,就第詢訪。」時太史奏當有移都事,帝以初受命,甚難之。穆乃上表極言宜移都之便。帝素嫌臺城制度迮小,又宮內多鬼妖。



 蘇威嘗勸遷,上不納,遇太史奏狀,意乃惑之。至是省穆表,帝曰:「天道聰明,已有徵應,太師人望,復抗此請,則可矣。」遂之。



 歲餘,下詔:「穆自今已後,雖有愆罪,但非謀逆,縱有百死,終不推問。」



 開皇六年薨,時年七十七,遺令以不得陪
 駕岱宗為恨。詔遣黃門侍郎監護喪事,贈十州諸軍事、冀州刺史,謚曰明。賜以石槨、前後部羽葆鼓吹、轀輬車,百僚送之郭外。詔太常卿牛弘齊哀冊文,祭以太牢。



 長子惇字士獻。周文帝令功臣長子並與略陽公遊處,惇於輩流中特被引接,每有遐方服玩珍奇,無不班賜。封安樂郡公,位驃騎大將軍、開府儀同三司、鳳州刺史。先穆卒。子筠,襲祖爵。



 惇弟怡,位儀同三司,贈渭州刺史。



 怡弟雅,少有識量。仕周,以軍功封西安縣男,位荊州總管。開皇初,進爵為公。



 雅弟恒,位監州刺史,封曲陽侯。



 恒弟榮,位合州刺史,長城縣公。



 榮弟直,位車騎將軍、歸政縣侯。



 直弟雄,位柱國、驃騎將軍、密國公。



 雄弟渾,仁壽初,忿筠心堯嗇,遣兄子善衡賊之。求盜不得,文帝大怒,盡追其親族。初,筠與從父弟瞿曇有隙,渾遂證瞿曇殺之,而善衡獲免。筠死,帝議立嗣。邳公蘇威奏筠不軌,請絕其封。帝不許,乃以渾嗣。



 渾字金才,姿貌瑰偉,美鬚髯。起家左侍上士。尉遲迥反於鄴,時穆在并州,隋文帝甚慮迥,遣渾乘驛詣穆遽令渾入京奉熨斗曰:「願執柄以慰天下也。」文帝大悅。又遣渾詣韋孝寬所而述穆意。會鄴平,以功授上儀同三司,封安武郡公。開皇中,晉王廣出籓,渾以驃騎將軍領
 親信,從往揚州。



 及筠死,渾規欲紹之,謂妻兄太子左衛率宇文述曰:「若得襲封,當以國賦之半,每歲相奉。」述因入白皇太子,奏文帝,竟詔渾襲申公以奉穆嗣。大業六年,追改穆封為郕公,渾仍襲焉。累加光祿大夫,遷右驍騎衛大將軍渾既紹父業,日增豪侈。二歲後不以奉物分述。述大恚,因醉謂其友人于象賢曰:「我竟為金才所賣,死且不忘。」渾聞之,由是結隙。及帝討遼東,有方士安伽陀謂帝曰:「李氏應為天子,宜盡誅天下李姓。」述知之,因構渾於帝曰:「臣與金才夙親,聞其數與李敏、善衡等日夜屏語,或終夕不寢。渾大臣也,家世隆盛,身捉禁兵,
 不宜然。」



 帝曰:「卿可覓其事。」述乃遣武賁郎將裴仁基表告渾反,即日遣述掩其家。遣左丞元文都、御史大夫裴蘊雜推之,數日,不得反狀。帝更遣述推。述入獄中召出敏妻宇文氏,謂曰:「夫人,帝甥也,何患無賢夫?李敏、金才名當妖讖,夫人當自求全。」因教言金才嘗告敏云:「汝應圖錄,當為天子。今主上好兵,勞擾百姓,此亦天亡隋時也。若復度遼,吾與汝必為大將軍,每軍二萬餘兵,固以五萬人矣。



 又發諸房子姪內外親婭並募從征,吾家子弟決為主帥,分領兵馬,散在諸軍。吾懷汝前發,襲取御營,子弟響赴,一日之間,天下定矣。」述口自傳授,令敏妻
 寫表,封云「上密」。述持入奏云:「已得金才反狀,並有敏妻密表。」帝覽之,泣曰:「吾宗社幾傾,賴親家公而獲全耳。」於是誅渾、敏等,自餘無少長皆徙嶺表。



 梁禦,字善通,其先安定人也。後因官北邊,遂家於武川,改姓紇豆陵氏。高祖俟力提,從魏太武征討,位揚武將軍、定陽侯。禦少好學,進趣詳雅,及長,更好弓馬。爾朱天光西討,知禦有志略,引為左右。共平關、隴,除益州刺史,第一領人酋長,封白水縣侯。從賀拔岳鎮長安。及岳被害,禦與諸將同謀翊戴周文帝。



 周文既平秦、隴,欲引兵東下。雍州刺史賈顯相見,
 因說顯,顯即出迎周文,禦遂入鎮雍州。大統元年,進爵信都縣公,授尚書右僕射。從周文復弘農,破沙苑,加侍中、開府儀同三司,進爵廣平郡公。出為東雍州刺史,為政舉大綱而已,人庶稱之。薨於州,臨終唯以國步未康為恨,言不及家。贈太尉、尚書令、雍州刺史,謚曰武昭。子睿。



 睿字恃德,少沉敏有行檢。周文帝時,以功臣子養宮中,復命與諸子遊。七歲,襲爵廣平郡公。累加儀同三司、本州大中正、開府,改封五龍郡公,渭州刺史。周閔帝受
 禪,徵為御伯。出為中州刺史,鎮新安以備齊。齊人來寇,睿輒挫之。帝甚嘉歎,拜大將軍。以禦佐命功,進爵蔣國公。入為司會。後從齊王憲拒齊將斛律明月於洛陽,每戰有功,遷小塚宰。歷敷州刺史、涼、安二州總管,俱有惠政,進位柱國。



 隋文帝總百揆,代王謙為益州總管。行至漢川西,謙反,攻始州,睿不得進。



 文帝命睿為行軍元帥,率行軍總管于義、張威、達奚長儒、梁昇、石孝義步騎二十萬討之。謙遣開府李三王守通谷,睿使張威擊破之。進至龍門,謙將趙儼、秦會擁眾十萬,據險為營,周亙三十里。睿令將士銜枚,出自間道,四面奮擊,力戰破之,遂鼓
 行而進。謙將敬豪守劍閣,梁巖拒平林,並懼而來降。謙又命高阿那瑰、達奚惎等以盛兵攻利州。聞睿將至,颭分兵據開遠。睿遣上開府拓拔宗趣、劍閣大將軍宇文瓊指巴西,大將軍趙達水軍入嘉陵。遣張威、王倫、賀若震、于義、韓相貴、阿那惠等分道攻惎,自午及申,破之。惎奔歸于謙。睿逼成都,謙令達奚惎、乙弗虔守城,親帥精兵五萬,背城結陳。睿擊敗之。謙將入城,惎、虔以城降。謙將麾下三十騎遁走,新都令王寶執之,睿斬謙于市,劍南悉平。進位上柱國,總管如故,賜物五千段、奴婢一千口、金二千兩、銀三千兩,邑千戶。



 睿時威振西州,夷獠歸
 附,唯南寧首帥爨震恃遠不賓。睿上疏曰:「南寧州,漢牂柯之地。近代已來,分置興古、雲南、建寧、朱提四郡,戶口殷眾,金寶富饒,二河有駿馬明珠,益、寧出鹽井犀角。晉泰始七年以益州曠遠,分置寧州。至偽梁,南寧州刺史徐文盛被湘東征赴荊州。屬東夏尚阻,未遑遠略,土人爨瓚遂竊據一方。



 國家遙授刺史,其子震相承至今。而震臣禮多虧,貢賦不入。如聞彼人苦其苛政,思被皇風幸因平蜀士眾,不煩重興師旅,狎獠既訖,即請略定南寧。」文帝深納之,然以天下初定,恐人心不安,故未之許。後竟遣史萬歲討平之,並因睿之策也。



 睿威惠兼著,人夷
 悅服,聲望逾重,文帝陰憚之。薛道衡從軍在蜀,說睿勸進,文帝大悅。及受禪,顧待彌隆。睿復上平陳策,帝善之,下詔曰:「昔公孫、隗囂,漢之賊也,光武與其通和,稱為皇帝。佗之於高祖,初猶不臣。孫皓之答晉文,書尚云『白』。或尋款服,或即滅亡。王者體大,義存遵養,雖陳國來朝,示盡蕃節,如公大略,誠須責罪,尚欲且緩其誅,宜如此意。淮海未滅,必興師旅,若命永襲,終當相屈,以身許國,無足致辭也。」睿乃止。睿時見突厥方強,恐為邊患,復陳鎮守之策十餘事。帝嘉歎久之,答以厚意。



 睿時自以周代舊臣,久居重鎮,內不自安,屢請入朝,於是徵還京師。
 及引見,上為之興,命睿升殿,握手極歡。睿退謂所親曰:「功遂身退,今其時也。」遂謝病,闔門自守,不交當時。帝賜以板輿,每有朝覲,必令三衛輿上殿。睿初平王謙之始,自以威名太盛,恐為時所忌,遂大受金賄以自穢。由是勳簿多不以實,詣朝堂稱屈者,前後百數人。上令有司案驗其事,主者多獲罪。睿懼,上表陳謝,請歸大理。上慰喻遣之。十五年,從至洛陽而卒,謚曰襄。



 子洋嗣,歷位嵩徐二州刺史、武賁郎將。大業六年,詔追改睿封為戴公,命以洋襲焉。



 論曰:賀拔岳變起倉卒,侯莫陳悅意在兼并,於時人有
 離心,士無固志。寇洛撫循散亂,抗禦仇讎,全師而還,敵人絕覬覦之望;度德而處,霸王建匡合之謀。



 趙貴居二闕之險,周室定二分之功。彼此一時,其功固不細也。



 李賢和兄弟屬亂離之際,居戎馬之間,志略從橫,忠勇奮發,頻摧勍敵,屢涉艱危。及逢時遇主,策名委質,荷生成之恩,蒙國士之遇,俱縻好爵,各著勳庸。



 遂得任兼文武,聲彰出內,位高望重,光國榮家,跗萼連暉,聊椒繁衍,冠冕之盛,當時莫與比焉。自周迄隋,鬱為西京盛族,雖金、張在漢,不之尚也。然而周文始崩,嗣君沖幼,內則功臣放命,外則強寇臨邊,晉公以猶子之親,膺負圖之託,遂
 能撫寧家國,開翦異端,革魏興周,遠安邇悅,功勤已著,過惡未彰。李植受遇先朝,宿參機務,慮威權之去已,懼將來之不容,生此厲階,成茲貝錦,乃以小謀大,由疏間親。主無昭帝之明,臣有上官之訴,嫌隙既兆,釁故因之,啟塚宰無君之心,成閔帝廢弒之禍,植之由也。李遠闕義方之訓,又無先見之明,以至誅夷,非為不幸。梁禦豫奉興王,參謀締構,驅馳畢力,夷險備嘗,雖遠志未申,亦云遇其時矣。



 穆及梁睿皆周室功臣,隋文王業初基,俱受腹心之寄,故穆首登師傅,睿終膺殊寵,觀其見機而動,抑亦人之先覺。然方魏朝之貞烈,有愧王凌,比晉室
 之忠臣,終慚徐廣。穆之子孫,特為隆盛,硃輪華轂,凡數十人,見忌當時,禍難遄及,得之非道,可不戒歟?



\end{pinyinscope}