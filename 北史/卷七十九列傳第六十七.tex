\article{卷七十九列傳第六十七}

\begin{pinyinscope}

 宇
 文述云定興趙行樞述子化及司馬德戡裴虔通王世充段達宇文述,字伯通,代郡武川人也。高祖侰與敦、曾祖長壽、祖孤,仕魏,並為沃野鎮軍主。父盛,仕周,位上柱國、大宗伯。



 述少驍銳,便弓馬。年十一時,有相者謂曰:「公子善自愛,後當位極人臣。」



 周武帝時,以父軍功,起家拜開府。述性謹密,周大塚宰宇文護甚愛之,以本官領護親信。及
 武帝親總萬機,召為左宮伯,累遷英果中大夫,賜爵博陵郡公,改封濮陽郡公。尉遲迥作亂,述以行軍總管從韋孝寬擊之,破迥將李雋軍於懷州,又與諸將破尉惇於永平橋。以功超拜上柱國,進爵褒國公。



 開皇初拜右衛大將軍。平陳之役,以行軍總管自六合而濟。時韓擒、賀若弼兩軍趣丹陽,述據石頭以為聲援。陳主既禽,而蕭瓛、蕭巖據東吳地。述領行軍總管元契、張默言等討之,落叢公燕榮以舟師自東海至,亦受述節度,於是吳會悉平。



 以功授子化及為開府,徙拜安州總管。時晉王廣鎮揚州,甚善於述,奏為壽州總管。



 王時陰有奪宗之
 志,請計於述。述曰:「皇太子失愛已久。大王才能蓋世,數經將領,主上之與內宮,咸所鐘愛,四海之望,實歸大王。然廢立國家大事,能移主上者,唯楊素耳。移素謀者,唯其弟約。述雅知約,請朝京師,與約共圖廢立。」晉王大悅,多齎金寶,資述入關。述數請約,盛陳器玩,與之酣暢,因共博戲,每陽不勝,輸所將金寶。約所得既多,稍以謝述。述因曰:「此晉王賜述,令與公為歡。」



 約大驚曰:「何為者?」述因為王申意。約然其說,退言於素,亦從之。於是晉王與述情好益密,命述子士及尚南陽公主,後賞賜不可勝計。及晉王為皇太子,以述為左衛率。舊令,率官第四
 品,以述素貴,遂進率品第三,其見重如此。



 煬帝嗣位,拜左衛大將軍,參掌武官選事。後改封許國公,尋加開府儀同三司,每冬正朝會,輒給鼓吹一部。從幸榆林,時鐵勒契弊歌稜攻敗吐谷渾。其部攜散,遂遣使請降,求救。帝令述以兵撫納降附。吐谷渾見述擁強兵,懼不敢降,遂西遁。



 述追至曼頭城,攻拔之。乘勝至赤水城,復拔之。其餘黨走屯丘尼川,進擊,大破之,獲其王公、尚書、將軍二百人。渾主南走雪山,其故地皆空。帝大悅。明年,從帝西巡至金山,登燕支,述每為斥候。時渾賊復寇張掖,述進擊走之。還至江都宮,敕述與蘇威常典選舉,參預朝
 政。述時貴重,委任與威等,其親愛則過之。帝所得遠方貢獻及四時口味,輒見班賜,中使相望於道。述善於供奉,俯仰折旋,容止便辟,宿衛咸取則焉。又有巧思,凡所裝飾,皆出人意表。數以奇服異物進宮掖,由是帝彌悅焉。言無不從,勢傾朝廷。左衛將軍張瑾與述連官,嘗有評議,偶不中意,述張目瞋之,瑾惶懼而走。文武百僚莫敢違忤。性貪鄙,知人有珍異物,必求取,富商大賈及隴右諸胡子弟,皆接以恩意,呼之為兒。由是競加饋遺,金寶累積。



 後庭曳羅綺者甚眾,家僮千餘人,皆控良馬,被服金玉。及征高麗,述為扶餘道軍將,臨發,帝謂曰:「禮,七
 十者行役以婦人從,公宜以家累自隨。古稱婦人不入軍,謂臨戰時耳。至軍壘間,無所傷也。項籍虞兮,即其故事。」述與九軍至鴨綠水,糧盡,議欲班師。諸將多異同,述又不測帝意。會乙支文德來詣其營,述先與於仲文俱奉密旨,令誘執文德。既而緩縱,文德逃歸,述內不自安,遂與諸將度水追之。時文德見述軍中多饑色,欲疲述眾,每鬥便北。述一日中七戰皆捷,既恃驟勝,又內逼群議,遂進,東濟薩水,去平壤城三十里,因山為營。文德復遣使偽降,請述曰:「若旋師者,當奉高元朝行在所。」述見士卒疲弊,不可復戰,又平壤險固,卒難致力,遂因其詐
 而還。眾半濟,賊擊後軍。於是大潰不可禁止,九軍敗績,一日一夜,還至鴨綠水,行四百五十里。初度遼,九軍三十萬五千人,及還至遼東城,唯二千七百人。帝怒,除其名。明年,帝又事遼東,復述官爵,待之如初。從至遼東,與將軍楊義臣率兵復臨鴨綠水。會楊玄感作亂,帝召述馳驛討玄感。時玄感逼東都,聞述軍至,西遁將圖關中。述與刑部尚書衛玄、右驍衛大將軍來護兒、武衛將軍屈突通等躡之。至閿鄉皇天原,與玄感相及,斬其首,傳行在所。復從東征,至懷遠而還。



 突厥之圍鴈門也,帝大懼,述請潰圍而出。來護兒及樊子蓋並固諫,帝乃止。



 及
 圍解,次太原,議者多勸帝還京師,帝有難色。述奏曰:「從官妻子多在東都,請便道向洛陽,自潼關入。」帝從之。尋至東都,又觀望帝意,勸幸江都宮。



 述於江都遇疾,及疾篤,帝令中使相望于第,謂述有何言。述曰:「願陛下一能降臨。」帝遣司宮魏氏謂曰:「公危篤,朕憚相煩動。必有言,可陳也。」述流涕曰:「臣子化及,早預籓邸,願陛下哀憐之。士及夙蒙天恩,亦堪驅策。臣死後,智及不可久留,願早除之,望不破門戶。」魏氏返命,隱其言,因詭對曰:「述唯憶陛下耳。」帝泫然曰:「述憶我耶?」將親臨之,宮人百僚諫乃止。及薨,帝為廢朝,贈司徒、尚書令、十郡太守,班劍四十
 人,轀輬車,前後部鼓吹,謚曰恭。



 詔黃門侍郎裴矩祭以太牢,鴻臚監護喪事。



 雲定興者,附會於述。初,定興女為皇太子勇昭訓,及勇廢,除名配少府。定興先得昭訓明珠絡帷,私賂於述,自是數共交游。定興每時節必有賂遺,并以音樂乾述。述素好著奇服,炫耀時人。定興為製馬韉,於後角上缺方三寸,以露白色,世輕薄者率仿學之,謂為許公缺勢。又遇天寒,定興曰:「入內宿衛,必當耳冷。」



 述曰:「然。」乃製夾頭巾,令深袹耳,人又學之,名為許公袹勢。述大悅曰:「雲兄所作,必能變俗。我聞作事可法,故不虛也。」後帝將事四夷,大造兵器,述薦之,因敕少府工匠
 並取其節度。述欲為之求官,謂之曰:「兄所製器仗並合上心,而不得官者,為長寧兄弟猶未死耳。」定興曰:「此無用物,何不勸上殺之?」



 述因奏曰:「房陵諸子,年並成立,今欲動兵征討,若將從駕,則守掌為難;若留一處,又恐不可。進退無用,請早處分。」因鳩殺長寧,又遣以下七弟分配嶺表,於路盡殺之。其年大閱,帝稱甲仗為佳,述奏並雲定興之功也。擢授少府丞。十一年,累遷屯衛大將軍。



 又有趙行樞者,本太常樂戶,家財億計。述謂為兒,受其賂遺,稱為驍勇,起家為折衝郎將。



 化及,述長子也。性兇險,不循法度,好乘肥挾彈,馳鶩道中,由是長安謂之輕
 薄公子。煬帝為太子時,常領千牛出入臥內。累遷至太子僕,以受納貨賄,再三免官。太子嬖暱之。俄而復職,又以其弟士及尚南陽公主。由此益驕,處公卿間,言辭不遜,多所凌轢。見人子女狗馬珍玩,必請託求之。常與屠販者游,以規其利。



 煬帝即位,拜太僕少卿,益恃舊恩,貪冒尤甚。煬帝幸榆林,化及與弟智及違禁與突厥交市。帝大怒,囚之數月。還京師,欲斬之而後入城,解衣辮髮訖,以主救之,乃釋,并智及並賜述為奴。述薨後,煬帝追憶之,起化及為右屯衛將軍,將作少監。



 時李密據洛口,煬帝懼,留淮左,不敢還都。從駕驍果多關中人,久客羈
 旅,見帝無西還意,謀欲叛歸。時武賁郎將司馬德戡總領驍果,屯於東城,風聞兵士欲叛,未審,遣校尉元武達陰問知情,因謀構逆。共所善武賁郎將元禮、直閣裴虔通互相扇惑曰:「聞陛下欲築宮丹陽,人人並謀逃去。我欲言之,恐先事見誅。今知而不言,後事發當族,將如之何?」虔通曰:「主上實爾。」德戡又謂兩人曰:「我聞關中陷沒,李孝常以華陰叛,陛下囚其二弟,將盡殺之。吾輩家屬在西安,得無此慮?」虔通等曰:「正恐旦暮及誅,計無所出。」德戡曰:「驍果若走,可與俱去。」虔通等曰:「誠如公言。」因遞相招誘。又轉告內史舍人元敏、鷹揚郎將孟景、符璽郎
 牛方裕,直長許弘仁、薛世良、城門郎唐奉義、醫正張愷等,日夜聚博,約為刎頸交,言無迴避,於坐中輒論叛計,並相然許。時李質在禁,令驍果守之,中外交通,所謀益急。又趙行樞先交智及;勳侍楊士覽者,宇文氏之甥。二人同以告智及。智及素狂勃,聞之喜,即共見德戡,期以三月十五日舉兵同叛,劫十二衛武馬,虜掠居人財物西歸。智及曰:「不然。今天實喪隋,英雄並起,因行大事,此帝王業也。」德戡然之。行樞、世良請以化及為主,約定,方告化及。化及性駑怯,初聞之,大懼,色動流汗,久之乃定。



 義寧二年三月一日,德戡欲告眾人,恐心未一,更譎詐
 以脅驍果,謂許弘仁、張愷曰:「君是良醫,國家所使,出言惑眾,眾必信。君可入備身府,遍告所識者,言陛下聞驍果欲叛,多醖毒酒,因享會,盡鳩殺之,獨與南人留此。群情必駭,因而舉事,無不諧矣。」其月五日,弘仁等宣布此言,驍果遞相告,謀反逾急。德戡等知計行,遂以十日總召故人,諭以所為。眾皆伏曰:「唯將軍命!」其夜,奉義主閉城門,門皆不下鑰,至夜三更。德戡於東城內集兵,得數萬人,舉火與城外相應。帝聞有聲,問是何事。虔通偽曰:「草坊被燒,外人救火,故喧囂耳。」中外隔絕,帝以為然。孟景、智及於城外得千餘人,劫候衛武賁馮普樂,共布兵
 捉郭下街巷。至五更,德戡授虔通兵,以換諸門衛士。虔通因自開門,領數百騎,至成象殿,殺將軍獨孤盛。武賁郎將元禮遂引兵進。宿衛者皆走。虔通進兵排左閤,馳入永巷問:「陛下安在?」有美人出房,指云:「在西閤。」從往執帝。帝謂虔通曰:「卿非我故人乎!何恨而反?」虔通曰:「臣不敢反,但將士思歸,奉陛下還京師耳。」帝曰:「即為汝歸。」虔通自勒兵守之。



 至旦,孟景以甲騎迎化及。化及未知事果,戰慄不能言,人有謁之,但低頭據案,答曰「罪過」。時士及在公主第,弗之知也。智及遣家僮莊桃樹就第殺之,桃樹不忍,執詣智及,久之乃見釋。化及至城門,德戡迎
 謁,引入朝堂,號為丞相。



 令將帝出江都門以示群賊,因復將入。遣令狐行達弒帝於宮中。又執朝臣不同己者數十人,及諸王外戚,無少長皆害之。唯留秦孝王子浩,立以為帝。



 十餘日,奪江都人舟楫,從水路西歸。至顯福宮,宿公麥孟才、折衝郎將沈光等謀擊化及,反為所害。化及於是入據六宮,其自奉一如煬帝故事。每帳中南面端坐,人有白事者,默然不對。下牙時,方收取啟狀。共奉義、方裕、世良、愷等參決之。行至徐州,水路不通,復奪人車牛,得二千兩,並載宮人珍寶。其戈甲戎器,悉令軍士負之。道遠疲極,三軍始怨。



 德戡失望,竊謂行樞曰:「君
 大誤我。當今撥亂,必藉英賢,化及庸暗,事將必敗,若何?」行樞曰:「廢之何難!」因共李孝本、宇文導師、尹正卿等謀,以後軍萬餘兵襲殺化及,立德戡為主。弘仁知之,密告化及,盡收德戡及支黨殺之。



 引兵向東郡,通守王軌以城降之。



 元文都推越王侗為主,拜李密為太尉,令擊化及。密壁清淇,與徐世勣以烽火相應。化及數戰不利,其將軍于弘達為密所禽,送於侗所,鑊烹之。化及糧盡,度永濟渠,與密決戰於童山。遂入汲郡求軍糧,又遣使拷掠東郡人吏,責米粟。王軌怨之,以城歸李密。化及大懼,自汲郡將圖以北諸州。其將陳智略率嶺南驍果萬餘
 人,張童兒率江東驍果數千人,皆叛歸李密。化及尚有眾二萬,北走魏縣。張愷與其將陳伯謀去之,事覺,為化及所殺。腹心稍盡,兵勢日蹙,兄弟更無他計,但相聚酣宴,奏女樂。醉後,尤智及曰:「我初不知,由汝為計,彊來立我。今所向無成,負弒主之名,天下所不納。滅族豈非由汝乎?」抱其兩子而泣。智及怒曰:「事捷之日,都不賜尤;及其將敗,乃欲歸罪。何不殺我以降建德!」兄弟數相鬥鬩,言無長幼,醒而復飲,以此為恒。



 自知必敗,乃歎曰:「人生故當死,豈不一日為帝乎!」於是鳩殺浩,僭皇帝位於魏縣,國號許,建元為天壽,置百官。攻元寶藏於魏州,反為
 所敗,乃東北趣聊城,將招攜海內諸賊。遣士及徇濟北,徵求餉餽。大唐遣淮安王神通安撫山東,神通圍之十餘日,不剋而退。竇建德悉眾攻之。先是,齊州賊帥王薄聞其多寶物,詐來投附。化及信之,與共居守。至是,薄引建德入城,禽化及,悉虜其眾。先執智及、元武達、孟景、楊士覽、許弘仁等,皆斬之。及以檻車載化及至大陸縣城下,數其弒逆,并二子承基、承趾皆斬之,傳首於突厥義城公主,梟之虜庭。士及自濟北西歸長安。



 智及幼頑凶,好與人群聚鬥雞,習放鷹狗。初以父功,賜爵濮陽郡公。蒸淫醜穢,無所不為。其妻長孫氏,妒而告述。述雖為隱,
 而大忿之,纖芥之愆,必加鞭棰。弟士及,恃尚主,又輕忽之。唯化及事事營護,父再三欲殺,輒救免之,由是頗相親暱。遂勸化及遣人入蕃,私為交易。事發,當誅,述獨證智及罪惡,而為化及請命,帝因兩釋之。述將死,抗表言其兇勃,必且破家。帝後思述,拜智及將作少監。其江都弒逆事,皆智及之謀也。化及為丞相,以為右僕射,領十二衛大將軍。



 及僭號,封齊王。竇建德獲而斬之,并其黨十餘人,皆暴死梟首。



 司馬德戡,扶風雍人。父元謙,仕周為都督。德戡幼孤,以屠豕自給。有桑門釋粲,通德戡母娥氏,遂撫教之,因解
 書計。開皇中,為侍官,漸遷至大都督。從楊素出討漢王諒,充內營左右。進止便僻,俊辯多姦計,素大善之。以勛授儀同三司。大業三年,為鷹揚郎將。從討遼左,進位正議大夫,遷武賁郎將。煬帝甚暱之。



 從至江都,領左右備身驍果萬人,營於城內。因隋末大亂,乃率驍果反,語在化及事中。既獲煬帝,與黨孟景等推化及為丞相。化及首封德戡為溫國公,加光祿大夫,仍統本兵。化及意甚忌之。後數日,化及署諸將,分配士卒,乃以德戡為禮部尚書,外示美遷,實奪其兵也。由是懷怨,所獲賞物皆賂於智及,智及為之言。行至徐州,捨舟登陸,令德戡將後
 軍。乃與趙行樞、李孝本、尹正卿、宇文導師等謀襲化及,遣人使于孟海公,結為外助。遷延未發,以待使報。許弘仁、張愷知之,以告化及。



 因遣其弟士及陽為游獵,至於後軍。德戡不知事露,出營參謁,因命執之,并其黨與。化及責之曰:「與公戮力共定海內,出於萬死。今始事成,願得同守富貴,公又何為反也?」德戡曰:「本殺昏主,苦其毒害。立足下而又甚之,逼於物情,不獲已也。」化及不對,命送至幕下,縊而殺之。



 裴虔通,河東人。初,煬帝為晉王,以親信從,稍遷至監門校尉。帝即位,擢舊左右,授宣惠尉。累從征役,至通議大
 夫。與司馬德戡同謀作亂,先開宮門,騎至成象殿,殺將軍獨孤盛,執帝於西閤。化及以虔通為光祿大夫、莒國公。化及引兵之北也,令鎮徐州。化及敗後,歸於大唐,即授徐州總管,轉辰州刺史,封長蛇男。尋以隋朝弒逆之罪,除名,徙於嶺表而死。



 王世充,字行滿,本西域胡人也。祖支頹褥,徙居新豐。頹褥死,其妻少寡,與儀同王粲野合,生子曰瓊,粲遂納之以為小妻。其父收幼孤,隨母嫁粲,粲愛而養焉,因姓王氏。官至懷、汴二州長史。



 世充捲髮豺聲,沉猜多詭詐,頗窺書傳,尤好兵法,曉龜策推步盈虛,然未嘗為人言也。
 開皇中,為左翊衛,後以軍功拜儀同,授兵部員外郎。善敷奏,明習法律,而舞弄文墨,高下在心。或有駮難之者,世充利口飾非,辭義鋒起,從雖知其否而莫能屈,稱為明辯。



 煬帝世,累遷至江都郡丞。時帝數幸江都,世充善候人主顏色,阿諛順旨,每入言事,帝善之。又以郡丞領江都宮監,乃彫飾池臺,陰奏遠方珍物,以媚於帝,由是益暱之。大業八年,隋始亂,世充內懷徼幸,卑身禮士,陰結豪俊,多收眾心。



 江淮間人素輕薄,又屬賊盜群起,人多犯法,有繫獄抵罪者,世充枉法出之,以樹私恩。及楊玄感反,吳人朱燮、晉陵人管崇起兵江南以應之,自稱
 將軍,擁眾十餘萬。帝遣將軍吐萬緒、魚俱羅討之,不能剋。世充募江都萬餘人,擊頻破之。每有剋捷,必歸功於下,所獲軍實,皆推與士卒,身無所取。由此人爭為用,功最居多。



 十年,齊郡賊帥孟讓自長白山寇掠諸郡,至盱眙,有眾十餘萬。世充以兵拒之,而羸師示弱,保都梁山為五柵,相持不戰。後因其懈馳,出兵奮擊,大破之,乘勝盡滅諸賊,讓以數十騎遁去,斬首萬人,六畜軍資,莫不盡獲。帝以世充有將帥才略,始遣領兵,討諸小盜,所向破之。然性多矯偽,詐為善,能自勤苦,以求聲譽。



 十一年,突厥圍帝於鴈門,世充盡發江都人往赴難。在軍中,垢
 面悲泣,曉夜不解甲,藉草而坐。帝聞之,以為愛己,益信任之。



 十二年,遷為江都通守。時厭次人格謙為盜數年,兵十餘萬,在豆子中。世充破斬之,威振群賊。又擊盧明月,破之於南陽。後還江都,帝大悅,自執杯酒以賜之。時世充又知帝好內,乃言江淮良家多有美女,願備後庭,無由自進。帝愈喜,因密令世充閱觀諸女,資質端麗合法相者,取正庫及應入京物以聘納之。所用不可勝計,帳上所司云敕別用,不顯其實。有合意者,則厚賞世充,或不中者,又以賚之。後令以船送東京,而道路賊起,使者苦役,於淮泗中沉船溺殺之者,前後十數。



 或有發
 露,世充為秘之,又遽簡閱以供進。是後益見親暱。遇李密攻陷興洛倉,進逼東都,官軍數敗,光祿大夫裴仁基以武牢降於密。帝惡之,大發兵,將討焉。特發中詔遣世充為將,軍於洛口以拒密。前後百餘戰,互有勝負。世充乃引軍度洛水,逼倉城。李密與戰。世充敗績,赴水溺死者萬餘人。時天寒,大雨雪,兵既度水,衣皆沾濕,在道凍死者又數萬人,比至河陽,纔以千數。世充自繫獄請罪,越王侗遣使赦之,召令還都。收合亡散,屯於含嘉城中,不敢復出。



 宇文化及殺帝於江都,世充與太府卿元文都、將軍皇甫無逸、右司郎盧楚奉侗為主。侗以世充為
 吏部尚書,封鄭國公。及侗用元文都、盧楚之謀,拜李密為太尉、尚書令,密遂稱臣,復以兵拒化及於黎陽,遣使獻捷。眾皆悅,世充獨謂其麾下諸將曰:「文都之輩,刀筆吏耳。吾觀其勢,必為李密所禽。且吾軍人馬每與密戰,殺其父兄子弟,前後已多,一旦為之下,吾屬無類矣。」出此言以激怒其眾。文都知而大懼,與楚等謀,將因世充入內,伏甲而殺之。期有日矣,將軍段達遣女婿張志以楚等謀告之。世充夜勒兵圍宮城,將軍費曜、田世闍等與戰於東太陽門外。曜軍敗,世充遂攻門而入。無逸以單騎遁走。獲楚,殺之。時宮門尚閉,世充遣人扣門言於
 侗曰:「元文都等欲執皇帝降於李密,段達知而以告臣。臣非敢反,誅反者耳。」文都聞變,入奉侗於乾陽殿,陳兵衛之。令將帥乘城以拒難,兵敗,侗命開門以納世充。世充悉遣人代宿衛者,明日入謁,頓首流涕而言曰:「文都等無狀,謀相屠害,事急為此,不敢背國。」侗與之盟。世充尋遣韋節等諷侗,命拜為尚書左僕射、總督內外諸軍事。又授其兄惲為內史令,入居禁中。未幾,李密破化及還,其勁兵良馬多戰死,士卒皆倦。世充欲乘其弊而擊之,恐人心不一,乃假託鬼神,言夢見周公,乃立祠於洛水之上,遣巫宣言周公欲令僕射急討李密,當有大功,
 不則兵皆疫死。世充兵多楚人,俗信妖妄,故出此言以惑之。眾皆請戰,世充簡練精勇得二萬餘人,馬千餘匹,營洛水南。密軍偃師北山上,時密新得志於化及,有輕世充之心,不設壁壘。世充遣二百餘騎,潛入北山,伏溪谷中,令軍秣馬蓐食。既而宵濟,人馬奔馳,比明而薄密。密出兵應之,陣未成列而兩軍合戰,其伏兵蔽山而上,潛登北原,乘高而下,壓密營。營中亂,無能拒者,即入縱火。密軍大驚而潰,降其將張童兒、陳智略。進下偃師。初,世充兄偉及子玄應隋化及至東郡,密得而囚之於城中。至是,盡獲之。又執密長史邴元真妻子、司馬鄭虔
 象之母及諸將子弟,皆撫慰之,各令潛呼其父兄。兵次洛口,元真、鄭虔象等舉倉城以應之。密以數十騎遁逸,世充收其眾而還。東盡於海,南至於江,悉來歸附。



 世充又令韋節諷侗,拜己為太尉,置署官屬,以尚書省為其府。尋自稱鄭王,遣其將高略帥師攻壽安,不利而旋。又帥師攻圍穀州,三日而退。明年,自稱相國,受九錫,備法物,是後不朝侗矣。有道士桓法嗣者,自言解圖讖,世充暱之。法嗣乃上《孔子閉房記》,畫作丈夫持一干以驅羊。法嗣云:「楊,隋姓也。乾一者,王字也。王居楊後,明相國代隋為帝也。」又取《莊子人間世》、《德充符》二篇上之,法嗣釋
 曰:「上篇言世,下篇言充,此則相國名矣。當德被人間,而應符命為天子也。」世充大悅曰:「此天命也。」再拜受之。即以法嗣為諫議大夫。世充又羅取雜鳥,書帛係其頸,自言符命而散之於空。或有彈射得鳥而來獻者,亦拜官爵。既而廢侗,陰殺之,僭即皇帝位,建元曰開明,國號鄭。



 大唐太宗帥師圍之。世充頻出兵,戰輒不利,諸城相繼降款。世充窘迫,遣使請救於竇建德,建德率兵援之。至武牢,太宗破之,禽建德以詣城下。世充將潰圍而出,諸將莫有應之者,於是出降。至長安,為仇家所殺。



 段達,武威姑臧人。父巖,周朔州刺史。達在周,年始三歲,
 襲爵襄坦縣公。



 及長,身長八尺,美鬚髯,便弓馬。隋文帝為丞相,以為大都督,領親信兵,常置左右。及踐祚,為左直齋,遷車騎將軍,督晉王府軍事。以擊高智慧功,授上儀同。



 又破汪文進等,加開府。仁壽初,為太子左衛副率。大業初,以籓邸之舊,拜左翊衛將軍。從征吐谷渾,進位金紫光祿大夫。帝征遼東,平原郝孝德、清河張金稱等並起為盜,帝令達擊之,數為金稱等所挫,諸賊輕之,號為段姥。後用鄃令楊善會謀,更與賊戰,方致剋捷。還京師,以公事坐免。明年,帝征遼東,使達留守涿郡。



 俄復拜左翊衛將軍。高陽魏刀兒聚眾,自號歷山飛,寇掠燕、趙。
 達率涿郡通守郭絢擊敗之。時盜賊既多,達不能因機決勝,唯持重自守,時人皆謂之為怯懦。



 十二年,帝幸江都宮,詔達與太府卿元文都等留守東都。李密縱兵侵掠城下,達與監門郎將龐玉、武牙郎將霍世舉禦之,以功遷左驍衛大將軍。王世充之敗也,密進據北芒,來薄上春門,達與判戶部尚書韋津拒之。達見賊,不陣而走,軍大潰,津沒於密。



 及帝崩於江都,達與文都等推越王侗為主,署開府儀同三司,兼納言,陳國公。



 元文都等之謀誅王世充,達預焉。既而陰告世充,達為之內應。及事發,迫越王送文都於世充,世充甚德於達。既破李密,諷
 越王禪讓。世充僭號,以達為司徒。及東都平,坐斬,妻子籍沒。



 論曰:宇文述便辟足恭,柔顏取悅。君所謂可,亦曰可焉,君所謂不,亦曰不焉。無所是非,不能輕重,默默茍容,偷安高位,甘素餐之責,受彼己之譏。此固君子所不為,亦丘明之深恥。化及以此下才,負恩累葉。時逢崩拆,不能竭命,乃因利乘便,先圖乾紀,率群不逞,職為亂階,擾本塞源,裂冠毀冕。釁深指鹿,事切食蹯,天地所不容,人神所同憤矣,世充頭筲小器,遭逢時幸,與蒙獎擢,禮越舊臣。而躬為戎首,親行鳩毒。竟而蛇豕醜類,繼踵誅夷,梟
 獍兇魁,相尋菹戮。



 垂炯戒於來葉,快忠義於當年,為人臣者,可無殷鑒哉!



\end{pinyinscope}