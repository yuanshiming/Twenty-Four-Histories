\article{卷五十五列傳第四十三}

\begin{pinyinscope}

 孫搴陳元康杜弼子臺卿房謨子恭懿張纂張亮趙起徐遠張曜王峻王紘敬顯俊平鑒唐邕白建元文遙趙彥深赫連子悅馮子琮子慈明郎基子茂孫搴,字彥舉,樂安人。世寒賤,少勵志勤學。自檢校御史
 再遷國子助教。太保崔光引修國史。歷行臺郎。後預崔祖螭反,逃於王元景家,遇赦乃出。孫騰以宗情,薦之齊神武,未被知也。會神武西征,登風陵,命中外府司馬李義深、相府城局李士略共作檄文,皆辭,請以搴代。神武乃引搴入帳,自為吹火,催促之。搴神色安然,援筆立就,其文甚美。神武大悅,即署相府主簿,專典文筆。又能通鮮卑語,兼宣傳號令,當煩劇之任,大見賞重。賜妻韋氏,既士人子女,又兼色貌,時人榮之。



 文襄初欲之鄴總知朝政,神武以其年少,未許。搴為致言,乃果行。恃此,自乞特進,文襄但加散騎常侍。時大括人為軍士,逃隱者,身
 及主人、三長、守、令罪以大辟,沒其家。於是所獲甚眾,搴之計也。



 搴學淺行薄,邢邵嘗謂曰:「須更讀書。」搴曰:「我精騎三千,足敵君嬴座數萬。」搴少時與溫子昇齊名,嘗謂子昇:「卿文何如我?」子昇謙曰:「不如卿。」



 搴要其為誓。子升笑曰:「但知劣於卿便是,何勞旦旦?」搴悵然曰:「卿不為誓,事可知矣!」搴常服棘刺丸,李諧調之曰:「卿應自足,何假外求?」坐者皆笑。



 司馬子如與高季式召搴飲酒,醉甚而卒。神武親臨之曰:「折我右臂。」贈吏部尚書、青州刺史。



 陳元康,字長猷,廣宗人也。父終德,魏濟陰內史,元康貴,贈度支尚書,謚曰貞。元康頗涉文史,機敏有乾用。魏正
 光中,從李崇北伐,以軍功賜爵臨清男。



 普泰中,除主書,累遷司徒高昂記室。初,司馬子如、高季式與孫搴劇飲,搴醉死,神武命求好替,子如舉魏收。他日,神武謂季式曰:「卿飲殺我孫主簿,魏收作文書,都不稱我意。司徒嘗道一人謹密,是誰?」委式以元康對,曰:「是能夜闇書快吏也。」召之,一見便授大丞相功曹,內掌機密。善陳事意,不為華藻。遷大行臺都官郎,封安平子。軍國多務,元康問無不知。神武臨行,留元康在後,馬上有所號令九十餘條,元康屈指數之,盡能記憶。神武甚親之,曰:「如此人,世間希有,我今得之,乃上天降佐也。」時趙彥深亦知機密,
 人謂之陳、趙,而元康勢居趙前。性又柔謹。神武之伐劉蠡升,天寒雪深,使人舉氈,元康於氈下作軍書,颯颯運筆,筆不及凍,俄頃數紙。及出,神武目之曰:「此何如孔子邪?」



 神武嘗怒文襄,親加毆蹋,極口肆罵。以告元康,元康俯伏泣下霑地曰:「王教世子過矣!」神武曰:「我性急,瞋阿惠,常如此。」元康大啼曰:「一度為甚,況常然邪!」神武自是為之懲忿。時或恚撻,輒曰:「勿使元康知。」又謂左右曰:「元康用心誠實,必與我兒相抱死。」高仲密之叛,神武知其由崔暹,將殺之。文襄匿暹,為之請,神武曰:「我為爾不殺,然須與苦手。」文襄乃出暹而謂元康曰:「暹若得杖,不須
 見我。」及暹見神武,將解衣受罰。元康趨入,止伍伯,因歷階升曰:「王方以天下付世子,世子有一崔暹不能免其杖,父子尚爾,況世間人邪?」



 神武意解:「不由元康,崔暹得一百。」乃捨之。



 文襄入輔,居鄴下,崔暹、崔季舒、崔昂等並被任用,張亮、張徽纂並為神武待遇,然皆出元康下。神武每與元康久語,文襄門外待接之。時人語曰:「三崔二張,不如一康。」左衙將軍郭瓊以罪死,子婦范陽盧道虔女也,沒官。神武啟以賜元康為妻。元康地寒,時以為殊賞。元康遂棄故妻李氏,識者非之。元康便辟善理人,而不能平心處物。溺於財利,受納金制,不可勝紀,責負
 交易,遍於州郡,為清論所譏。



 從神武於芒山,將戰,遺失陣圖,元康冒險求得之。西師既敗,神武會諸將,議進取策。或以為人馬疲瘦,不可遠追。元康曰:「兩雄交爭,歲月已久,今得大捷,便是天授,時不可失,必須乘勝追之。」神武曰:「若遇伏兵,孤何以濟?」



 元康曰:「前沙苑還軍,彼尚無伏,今者奔敗,何能遠謀?捨之必成後患。」神武不從。累遷大行臺左丞。及神武疾篤,謂文襄曰:「芒山之戰,不用元康言,方貽汝患,以此為恨,死不瞑目。事皆當與元康定也。」



 神武崩,密不發喪,唯元康知之。文襄嗣事,自晉陽將之鄴,令元康預作神武條教數十紙,留付段孝先、趙彥
 深,在後以次行之。別封昌國縣公,以從嘉名。



 侯景反,文襄逼於諸將,欲殺崔暹以謝之。元康諫曰:「今枉殺無辜,虧廢刑典,豈直上負天神,何以下安黎庶?晁錯前事,願公慎之。」文襄乃止。高岳討侯景未克,文襄欲遣潘相樂副之。元康曰:「相樂緩於機變,不如慕容紹宗。且先王有命,稱堪敵景。」時紹宗在遠,文襄欲召見之,恐其驚叛。元康曰:「紹宗知元康特蒙顧待,新使人來餉金,以致誠款。元康欲安其意,故受之而厚答其書,保無異也。」乃任紹宗,遂破景,賞元康金五十斤。



 王恩政入潁城,諸將攻之不能拔。元康進曰:「公自匡朝政,未有殊功,雖敗侯景,本
 非外賊。今潁城將陷,願公因而乘之,足以取威定業。」文襄令元康馳驛觀之,復命曰:「必可撥。」文襄乃親征潁川,益發眾軍,決既至而克之,賞元康金百鋌。



 初,魏朝授文襄相國、齊王,諸將皆勸恭膺朝命。元康以為未可。崔暹因間之,薦陸元規為大行臺郎,欲分元康權。元康既貪貨賄,文襄內漸嫌之,又欲用為中書令,以閑地處之,事未施行。



 屬將受魏禪,元康與楊愔、崔季舒並在坐,將大遷除朝士,共品藻之。文襄家倉頭蘭固成掌廚,與其弟阿改,謀害文襄。阿改時事文宣,常執刀從,期聞東齋叫,即加刃於文宣。時文宣別有所之,未還而難作。固成因
 進食,置刀盤下,而殺文襄。



 元康抱文襄。文襄曰:「可惜!可惜!」與賊爭刀,髻解,被刺,傷重腸出,猶手書辭母,口占祖孝徵陳權宜。至夜而終,時年四十三。時楊愔狼狽走出,遺一靴,崔季舒逃匿於廁,庫直紇奚舍樂捍賊死,散都督王師羅戰傷。監廚倉頭薛豐洛率宰人持薪以赴難,乃禽盜。固成一名京,事見齊本紀。密文襄凶問,故殯元康於宮中。



 託以出使南境,虛除中書令。明年,乃贈司空,謚曰文穆。元康卒後,母李氏哀感發病而終,贈廣宗郡君,謚曰貞昭。元康子善藏嗣。



 善藏溫雅有鑒裁,位給事黃門侍郎。隋開皇中,尚書郎。大業初,卒於彭城郡贊務。



 杜弼,字輔玄,中山曲陽人也。祖彥衡,淮南太守。父慈度,繁畤令。弼幼聰敏,家貧無書,年十二,寄郡學受業。同郡甄琛為定州刺史,簡試諸生,見而策問,應答如響,大歎異之,命其二子楷、寬與交。州牧任城王澄聞而召問,深相嗟賞,許以王佐之才。澄、琛還洛稱之,丞相高陽王等多相招命。但父祖官薄,不獲優敘。



 以軍功起家征虜府墨曹參軍,典管記。弼長於筆札,每為時輩所推。孝昌初,除太學博士。遷光州曲城令,為政清靜,遠近稱之。弼父在鄉,為賊所害,弼居喪六年。



 以常調,除侍御史,臺中彈奏,皆見信任。儀同竇泰西伐,詔弼監軍。及泰失利自殺,
 弼與其徒六人,走還陜州。刺史劉貴鎖送晉陽。神武責以不諫爭,賴房謨諫以免。



 累遷大行臺郎中,又引典掌機密,甚見信待。或有造次不及書教,直付空紙,即令宣讀。承間密勸受禪,神武舉杖擊走之。相府法曹辛子炎咨事云「取署」,子炎讀「署」為「樹」,神武怒其犯諱,杖之於前。弼進「孔子言『徵』不言『在』,子炎可恕。」神武罵曰:「眼看人嗔,乃復牽經引禮!」叱令出去。弼行十許步,呼還,子炎亦蒙宥。文襄在鄴聞之,謂楊愔曰:「王左右賴此人,天下蒙利,豈獨吾家也?」



 初,神武自晉陽東出,改爾朱氏貪政,使人入村,不敢飲社酒。及平京洛,貨賄漸行。弼以文武在
 位,罕有廉潔,言之神武。神武曰:「弼來,我語爾。天下濁亂,習俗已久,今督將家屬,多在關西,黑獺常相招誘,人情去留未定;江東復有一吳老翁蕭衍,專事衣冠禮樂,中原士大夫望之,以為正朔所在。我若急作法網,恐督將盡投黑獺,士子悉奔蕭衍,則何以為國?爾宜少待,吾不忘之。」及將有沙苑之役,弼又請先除內賊,卻討外寇,指諸勳貴掠奪百姓。神武不答,因令軍人皆張弓挾矢,舉刀按矛以夾道,使弼冒出其間,曰:「必無傷也。」弼戰怵流汗。神武然後喻之曰:「箭雖注不射,刀雖舉不擊,矛雖按不刺,爾猶頓喪魂膽。諸勳人觸鋒刃,百死一生,縱其貪
 鄙,所取處大。」弼頓顙謝曰:「愚人不識至理。」後破芒山軍,命為露布,弼即書絹,曾不起草。以功賜爵定陽縣男。



 奉使詣闕,魏帝見之九龍殿,曰:「聞卿精學,聊有所問。經中佛性法性,為一為異?」弼曰:「正是一理。」又問曰:「說者妄,皆言法性寬,佛性狹,如何?」



 弼曰:「在寬成寬,在狹成狹,若論性體,非狹非寬。」詔曰:「既言成寬成狹,何得非狹非寬?」弼曰:「若定是寬,則不能為狹;若定是狹,亦不能為寬。以非寬非狹,所成雖異,能成恒一。」上稱善,引入經庫,賜地持經一部,帛百疋。弼性好名理,探味玄宗,在軍恒帶經行。注孝子道德經二卷,表上之。遷廷尉卿。



 會梁貞陽侯蕭明等
 入寇彭城,大都督高岳、行臺慕容紹宗討之,詔弼為軍司,攝行臺左丞。臨發,文襄賜胡馬一疋,曰:「此廄中第二馬,孤恒自乘,聊以為贈。」



 又令陳政要可為鑒誡者,弼曰:「天下大務,莫過刑賞二端。賞一人使天下之人喜,罰一人使天下之人服,二事得衷,自然盡美。」文襄大悅曰:「言雖不多,於理甚要。」握手而別。破蕭明迴,破侯景於渦陽。後魏帝集名僧於顯陽殿講說佛理,敕弼升師子座,莫有能屈。帝歡曰:「此賢若生孔門,則何如也!」關中遣王思政據潁州,朝廷以弼行潁州,攝行臺左丞。及潁州平,文襄曰:「卿試論思政所以禽。」



 弼曰:「思政不察逆順之理,不
 識大小之形,不度強弱之勢,有此三蔽,宜其俘獲。」



 文襄曰:「古有逆取順守,大吳困於小越,弱燕能破強齊,卿之三義,何以自立?」



 弼曰:「王若順而不大,大而不強,強而不順,於義或偏,得如聖旨。今既兼備,鄙言可以還立。」



 文宣作相,位中書令,仍長史,進爵為侯。弼志在匡贊,知無不為。及受命,以預定策功,遷衛尉卿,別封長安縣伯。



 常與邢邵扈從東山,共論名理。邢以為人死還生,恐是為蛇畫足。弼曰:「物之未生,本亦無也。無而能有,不以為疑;因前生後,何獨致怪?」邢云:「聖人設教,本由勸獎,故懼以有來,望各遂其性。」弼曰:「聖人合德天地,齊信四時,言則為
 經,行則為法,而云以虛示物,以詭勸人,安得使北辰降光,龍宮韞檀。既如所論,福要可以鎔鑄性靈,弘獎風教,為益之大,莫極於斯。此即真教,何謂非實?」邢云:「季札言無不之,亦言散盡,若復聚而為物,不得言無不之也。」弼曰:「骨肉下歸於土,魂氣則無不之,此乃形墜魂游,往而大盡。由其尚有,故云無所不之。若也全無,之將焉適?」邢云:「神之在人,猶光之在燭,燭盡則光窮,人死則神滅。」弼曰:「燭則因質生光,質大光亦大;人則神不係形,形小神不小。



 故仲尼之智,必不短於長狄;孟德之雄,乃遠奇於崔琰。」其後,別與邢書,前後往復再三,邢理屈而止。文多
 不載。



 又以本官行鄭州事,未發,為家客告弼謀反,案察無實,久乃見原,因此絕朝見。復坐第二子廷尉監臺卿斷獄稽遲,與寺官俱為郎中封靜哲所訟,徙臨海鎮。時楚州人東方白額謀反,鎮為賊帥張綽、潘天命等所攻,弼率厲城人,終得全固。文宣嘉之,敕行海州事。後除膠州刺史。弼所在清靜廉潔,為吏人懷之。耽好玄理,注莊子惠施篇并易上下繫辭,名曰新注義苑,並行於世。



 性質直,在霸朝多所匡正。及文宣作相,致位僚首,初聞揖讓之議,猶有諫言。



 帝又嘗問弼:「治國當用何人?」對曰:「鮮卑車馬客,會須用中國人。」帝以為譏已。高德正居要,不能
 下之,乃至於眾前面折德正。德正深以為恨,數言其短。



 又令主書杜永珍密啟弼在長史日,受人屬,大營婚嫁,帝內銜之。弼恃舊,仍有公事陳請。十年夏,上因飲酒,積其愆失,遣使就州斬之。尋悔,驛追不及。子蕤及遠徙臨海鎮。次子臺卿,先徙東豫州。乾明初,並得還鄴。天統五年,追贈弼開府儀同三司、尚書右僕射。武平元年,又贈驃騎大將軍,謚曰文肅。



 蕤字子美,學業不如弟臺卿而幹局過之。武平中,位大理少卿,兼散騎常侍、聘陳使主、吏部郎中。隋開皇中,終於開州刺史。



 子公贍,仕隋,位安陽令。公贍子之松,大業中,起居舍人。



 臺卿字少山,好學博覽,解屬文。仕齊,位中書、黃門侍郎,修國史。既居清顯,忌害人物。趙彥深、和士開、高阿那肱等親信之。後兼尚書左丞,省中以其耳聾,多戲弄之。下辭不得理者,乃至大罵。臺卿見其口動,謂為自陳。令史又故不曉喻,訓對往往乖越,聽者以為嗤笑。及周武平齊,歸鄉里。以禮記、春秋講授子弟。隋開皇初,被徵入朝。臺卿採月令,觸類廣之,為書名玉燭寶典十二卷,至是奏之,賜帛二百疋。患耳,不堪吏職,請修國史,拜著作郎。後致仕,終於家。有集十五卷,撰齊記二十卷,並行於世。無子。



 房謨,字敬放,河南洛陽人也。其先代人,本姓屋引氏。少淳厚,雖無造次能,而沈深內敏。正光末,歷位昌平、代郡太守,所在著廉惠。及六鎮亂,謨率郡人入九崢山,結壘拒守。時外無救援,乃率所部奔中山。遇鮮于脩禮之亂,朝廷以謨得北邊人情,以為假燕州事。北轉至幽州南,為脩禮所執,仍陷葛榮。榮敗,爾朱榮啟授行冀州事。尋除太寧太守。榮死,其黨徵兵,謨不應,前後斬其三使。遣弟毓詣闕,孝莊以毓為都督,毓弟欽為行臺,並持節詣謨,同為經略。



 及京都淪覆,為賊黨建州刺史是蘭安定執繫州獄。蜀人聞謨被囚,並叛。安定於是給謨弱馬,令
 軍前慰勞。諸賊見謨,莫不遙拜。謨先所乘馬,安定別給將士,戰敗,蜀人得之,謂謨遇害,莫不悲泣。善養其馬,不聽乘騎,兒童婦女,竟投草粟,皆言此房公馬也。其結愛人心如此。爾朱世隆聞而嘉之,捨其罪,以為東北道行臺。及爾朱氏敗,濟州刺史侯景以謨先款附,推謨降首。謨以受眷爾朱,不宜先為反覆,不從其計。



 神武入洛,再遷潁川太守。魏孝武帝入關,神武以謨忠貞,遣其弟毓為大使,持節勞問。時軍國未寧,徵發煩速,至有數使同征一物,公私勞擾。謨請事遣一使,下自催勒,朝廷從之。徵為丞相右長史,以清直甚被賞遇。謨悉心盡力,知無
 不為。



 前後賜其奴婢,率多免放,神武後賜其生口,多黥面為房字而付之。神武討關右,以謨兼大行臺左丞,長史如故,總知府省務。天平三年,行定州事。請在左右拾遺補闕,固不肯行,神武責而罷之。



 未幾,出為兗州刺史。謨選用廉清,廣布恩信,僚屬守令,有犯必知,雖號細密,百姓安之。轉徐州刺史。始謨在兗州,彭城慕其政化,及為刺史,合境欣悅。



 謨為政如在瑕丘。先是,當州兵皆僚佐驅使,飢寒死病,動至千數。謨至,皆加檢勒,不令煩擾,以休假番代洗沐,督察主司,親自檢視。又使傭賃,令作衣服,終歲還家,無不溫飽,全濟甚多。時梁、魏和好,使人入
 其界者,咸稱歎之。神武與諸州刺史書,敘謨及廣平太守羊敦、廣宗太守竇瑗、平原太守許季良等清能,以為勸勵。謨曾啟神武,以天下未寧,宜降婚勳將,收將士心,深見納。任人所樂,朝廷從之。徵拜侍中,監國史。謨無他材學,每求退身,不許。尋兼吏部尚書,魏朝以河南數州,鄉俗絹濫,退絹一疋,徵錢三百,人庶苦之。謨乃表請錢絹兩受,任人所樂,朝廷從之。徵拜侍中,監國史。謨無他材學,每求退身,不許。尋兼吏部尚書,加衛大將軍。以子子遠罪,解官。久之,詔復本將軍,起為大丞相左長史。



 後除晉州刺史,加驃騎大將軍,又攝南汾州事。先時境接西魏,士人多受其官,為之防守。至是,酋長、鎮將及都督、守、令前後降附者三百餘
 人,謨撫接殷勤,人樂為用。爰及深險胡夷,咸來歸服。謨常以己祿物,充其餉齎,文襄嘉之,聽用公物。西魏懼,乃增置城戍。慕義者,自相糾合,擊破之。自是龍門已北,西魏戍皆平。文襄特賜粟千石,絹二百疋,班示天下。卒於州,州府相帥贈物及車牛,妻子遵其遺志,拒而不納。謨寡嗜慾,貞白自守;然內營家產,足為富贍,不假官俸,是以世稱清白。贈司空,謚曰文惠。



 謨與子結婚盧氏,謨卒後,盧氏將改適他姓。有平陽廉景孫者,少厲志節,以明經舉郡孝廉,為謨所重,至是訟之,臺府不為理。乃持繩詣神廟前北面大呼曰:「房謨清吏,忠事高祖,及其死也,
 妻子見陵。神而有知,當助申之。今引決,訴於地下。」便以繩自經於樹。衛士見之,救解送所司。朝廷哀其至誠,命女歸房族。



 謨前妻子子遠險薄,謨甚嫌之,不以為子列。時以謨為後妻盧氏所譖,神武亦以責謨。謨陳其惡。神武弗信,自收恤之,令與諸子同學,久乃令還。後與任胄等謀殺神武,事發,神武歎曰:「知子莫若父,信哉!」因上言房謨、鄭述祖、李道幡三家,理宜從法,竊以謨立身清白,履行忠謹;鄭仲禮嚴祖庶兒,晚始收拾;李世林生自外養,屬絕本宗。三人特乞罪止一房,魏帝許焉。及謨卒,子廣嗣。廣弟恭懿。



 恭懿字慎言,沈深有局量,達於從政。仕
 齊,歷平恩令、濟陰太守,並有能名。



 齊亡,不得調。後預尉遲迥亂,廢于家。隋開皇初,吏部尚書蘇威舉為新豐令,政為三輔最。上聞而嘉之,賜物四百段。以所得賜,分給窮乏。未幾,復賜米三百石,又振貧人。上聞,止之。時雍州諸縣令,每朔朝謁,上必呼恭懿至榻前,訪以化下之術。威又薦之,歷澤、德二州司馬。盧愷復奏其政美,上甚異之,復賜以帛。諸州朝集,稱為勸勵之首,以為「上天宗廟之所祐助,豈朕寡薄能致?朕即拜為刺史,卿等宜師之」。乃下詔褒美,因授海州刺史。



 未幾,國子博士何妥奏恭懿尉遲迥之黨,威、愷曲相舉薦。上大怒,恭懿竟放嶺南。未
 幾征還,至洪州卒。論者冤之。



 張纂,字徽纂,代郡平城人也。初事爾朱榮,又為爾朱兆長史,使於神武,遂被顧識。及相州城拔,參丞相軍事,封武安縣伯。累遷神武行臺右丞。從征玉壁,大軍將還山東,至晉州忽遇寒雨,士卒飢凍有死者。州以邊禁,不聽入城。時纂為別使,遇見,輒令開門內之,分寄人家,給其火食,多所全濟。神武聞而善之。纂性便僻,事神武二十餘歲,通傳教令,甚見親賞。文宣時,卒於護軍將軍。



 張亮,字伯德,西河隰城人也。初事爾朱兆,兆奔秀容,左右皆密通誠款;唯亮獨無啟疏。及兆敗,竄於窮山,令亮
 及倉頭陳山提斬己首以降,皆不忍。兆乃自縊於樹,亮因伏屍哭。神武嘉歎之,授丞相府參軍,漸見親待,委以書記之任。天平中,為文襄行臺郎中,典七兵事。雖為臺郎,常在神武左右。遷右丞。



 高仲密之叛,與大司馬斛律金守河陽。周文帝於上流放火船,欲燒河橋。亮乃備小艇百餘,皆載長鎖,鎖頭施釘,火船將至,即馳小船,以釘釘之,引鎖向岸,火船不得及橋。橋全,亮之計也。後自太中大夫拜幽州刺史。薛琡嘗夢亮於山上掛絲,以告亮,且占之曰:「山上絲,幽字也,君其為幽州乎?」數月而驗。累遷尚書右僕射、西南道行臺。



 亮性質直,勤力強濟,深為
 神武、文襄信委。然少風格,好財利,久在左右,不能廉潔。及歷數州,咸有黷貨之號。天保初,別封安定縣男,位中領軍。卒,贈司空。



 時霸府又有趙起、徐遠者,並見任委。



 起,廣平人,性沉謹。神武頻以為相府騎兵二局,典兵馬十餘載。至文宣即位,累遷大鴻臚卿。雖歷九卿、侍中,常以本官監兵馬,出內居腹心寄,與二張相亞。



 武平中,卒於師,贈都督、滄州刺史。



 遠,廣寧人,為丞相騎兵參軍事,深為神武所知。累遷東楚州刺史,政有恩惠。



 郭邑大火,城人亡產業,遠躬自赴救,對之流涕,仍為經營,皆得安立。卒於衛尉卿。起、遠前書並有傳,更無異迹,今附此云。



 張曜,字靈光,上谷昌平人也。少貞謹,韓軌為御史劾,州府僚佐及軌左右以贓挂網者百餘人,唯曜以清白免。天保初,賜爵都鄉男,累遷尚書右丞。文宣曾近出,令曜居守。帝夜還,曜不時開門,勒兵嚴備。帝駐驛門外久之,催迫甚急。曜以夜深,須火至面識,門乃可開。於是獨出見帝。帝笑曰:「卿欲效郅君章也?」



 乃使曜前開門,然後入。嗟賞之,賜以錦採。大寧初,遷秘書監。曜歷事累世,奉職恪勤,咸見親待,未嘗有過。每得祿賜,輒散之宗族。性節儉率素,車服飲食,取給而已。好讀春秋,月一遍,時人比之賈梁道。趙彥深嘗謂之曰:「君研尋左氏,豈求杜、服繆
 邪?」曜曰:「何為其然乎?左氏之書,備敘言事,惡者可以自戒,善者可以庶幾。故勵已溫尋,非欲詆訶古人得失也。」天統元年,奏事,暴疾,仆於御前。武成下坐臨視,呼不應。帝泣曰:「失我良臣也。」旬日卒,贈尚書右僕射,謚曰貞簡。



 王峻,字巒嵩,靈丘人也。明悟有幹略。歷事神武、文襄,為相府佐,賜爵北平男,除營州刺史。營州地接邊賊,數為人患。峻至州,遠設斥候,廣置疑兵,賊不敢發,合境獲安。先是,刺史陸士茂詐殺室韋八百餘人,因此朝貢遂絕。至是,峻要其行路,大破之。虜其酋帥,厚加恩禮,放遣之。室韋遂獻誠款,朝貢不絕,峻有力焉。蠕蠕主庵羅辰東
 徙,峻設伏大破之,於此遁走。歷位尚書。河清中,位南道行臺,坐違格私度禁物,并盜截軍糧,有司定處斬刑,家口配沒。詔決鞭一百,除名配甲坊,蠲其家口。武平初,卒於侍中,贈司空。



 王紘,字師羅,太安狄那人也。父基,頗讀書,有智略。初從葛榮,與周文帝相知。及周文據關中,神武遣基與長史侯景同往焉。周文留基不遣,後乃逃歸。歷南益、北豫二州刺吏,所歷皆好聚斂,然性和直,吏人不甚怨苦。後為奴所害,贈吏部尚書。



 紘善騎射,愛文學,性敏捷。年十三,見揚州樂太原郭元貞,撫其背曰:「讀何書?」曰:「誦孝經。」
 曰:「孝經云何?」曰:「在上不驕,為下不亂。」元貞曰:「吾豈驕乎?」紘曰:「君子防未萌,亦願留意。」元貞稱善。十五,隨父在北豫州,行臺侯景與人論掩衣法為當左右。尚書敬顯俊曰:「孔子云:『微管仲,吾其被髮左衽。』以此言之,右衽應是。」紘進曰:「國家龍飛朔野,雄步中原,五帝異儀,三王殊制,掩衣左右,何足是非?」景奇其早慧,賜以名馬。興和中,文襄召為庫直、奉朝請。文襄遇禍,紘冒刃捍禦。以忠節,進爵平春縣男。



 頗為文宣所知,為領左右都督。帝嘗與左右飲酒,曰:「快哉大樂!」紘曰:「亦有大苦。」帝曰:「何苦?」紘曰:「長夜荒飲,不悟國破,是謂大苦。」帝默然。後責紘曰:「爾與紇奚
 舍樂同事我兄,舍樂死,爾何不死?」紘曰:「君亡臣死,自是常節,但賊豎力薄,故臣不死。」帝使燕子獻反縛之,長廣王捉頭,帝手刃將下。紘呼曰:「楊遵彥、崔季舒逃難,位至僕射、尚書;冒危效命之士,翻見屠戮。曠古未有此事。」帝投刃於地曰:「王師羅不得殺。」遂舍之。



 後拜驃騎大將軍。武平初,加開府儀同三司。上言突厥與周男女來往,必相影響,南北寇邊,宜為之備。五年,陳人寇淮南,封輔相議討之。紘曰:「若復出頓江、淮,恐北狄西寇,乘弊而來。莫若薄賦省徭,息人養士,使朝廷協睦,遐邇歸心,征之以仁義,鼓之以道德,天下皆當肅清,豈直江南偽陳而已。」
 高阿那肱謂眾曰:「從王武衛者南席。」眾皆同焉。尋兼侍中,聘周。使還即正。未幾卒。



 紘好著述,作《鑒誡》二十四篇。



 敬顯俊,字孝英,陽平太平人也。少英俠,從神武信都義舉,歷位度支尚書。



 神武攻鄴,顯俊督造土山,以功封永安縣侯。出內多歷顯官,所在著名。河清中,卒於兗州刺史。



 子長瑜,武成時為廣陵太守,多所受納,刺史陸駿將表劾之以貨事。和士開以書屏風詐為長瑜獻,武成大悅,駿表尋至,遂不問焉。遷合州刺史,陷於陳,卒。



 子德亮,齊亡後,負屍歸。



 德亮,隋開皇中,卒於尚書郎。



 平鑒,字明達,燕郡薊人也。祖延,魏安平太守。父勝,安州
 刺史。鑒少聰敏,受學於徐遵明,受詩、禮於弘農楊文懿,通大義,不為章句。雅有豪俠氣。孝昌末,見天下將亂,乃之洛陽,與慕容儼以客騎馬為業,兼習弓矢。鑒性巧,夜則胡畫,以供衣食。俄奔爾朱榮,榮大奇之。以軍功累遷襄州刺史。神武起兵信都,鑒棄州自歸,即授本官。文襄輔政,封西平縣伯,遷懷州刺史。鑒奏請於州西故軹關道築城,以防西軍,從之。尋西魏將楊摽來攻。時新築之城,糧仗未集。素乏水,南門內有大井,隨汲即竭。鑒具衣冠,俯井而祝,至旦而井泉湧溢,有異於常,合城取足,揚示敵人。將士既觀非常,勇氣自立。楊摽敗,以功進開府
 儀同三司。累遷揚州刺史。其妻生男,鑒因喜酣醉,擅免境內囚,誤免關中細作二人。醒而知之,上表自劾。文宣特原其罪,賜犢百頭、羊二百口、酒百石,令作樂。河清二年,重拜懷州刺史。時和士開使求鑒愛妾阿劉,即送之。仍謂人曰:「老公失阿劉,與死何異?要自為身計,不得不然。」後卒於都官尚書,贈司空,謚曰文。



 子子敬嗣,輕險無賴,姦穢所至,禽獸不若。隋開皇中,為晉州行參軍,為並州總管秦王所殺。



 唐邕,字道和,太原晉陽人也。其先自晉昌徙焉。父靈芝,魏壽陽令,邕貴,贈司空公。邕少明敏,有材幹。初直神武
 外兵曹,以幹濟見知,擢為文襄大將軍督護。文襄崩,事出倉卒,文宣部分將校,鎮壓四方,夜中召邕支配,造次便了。帝甚重之。天保初,稍遷給事中,兼中書舍人,封廣漢鄉男。及從征奚虜,黃門侍郎袁猛舊典騎兵事,至是為割配遲留,鞭杖一百,仍令邕監騎兵事,以猛賜邕。文宣頻年出塞,邕必陪從,專掌兵機,承受敏速。自軍吏已上勞效由緒,無不諳練,占對如響。或御前簡閱,邕多不執文簿,唱官名未嘗謬誤。七年,於羊汾堤講武,令邕總為諸軍節度。事畢,仍監宴射之禮。親執其手,引至太后前,坐於丞相斛律金上。啟太后云:「邕一人當千。」仍別賜
 錢採。邕非唯強濟明辯,亦善揣上意,是以委任彌重。帝嘗白太后云:「邕手作文書,口且處分,耳又聽受,實是異人。」



 一日中六度賜物。又嘗解所服青鼠皮裘賜邕云:「朕意在與卿共弊。」除兼給事黃門、中書舍人。文宣嘗登并州童子佛寺望并州城,曰:「此何等城?」或曰:「金城湯池,天府之國。」帝云:「我謂唐邕是金城,此非也。」後謂邕云:「高德正妄說卿短,而薦主書郭敬,朕已殺之。卿劬勞既久,欲除卿作州,頻敕楊遵彥求堪代卿者,如卿實不可得,所以遂停。」文宣或切責侍臣云:「觀卿等,不中與唐邕作奴!」其愛遇如此。



 孝昭作相,署相府司馬。皇建元年,除給事
 黃門侍郎。太寧元年,除大司農卿。



 河清元年,突厥入寇,遣邕驛赴晉陽,纂集兵馬。在路聞虜將逼,邕基酌事宜,改敕,更促期會,由此兵士限前畢集。後拜侍中、并州大中正、護軍將軍。從武成幸晉陽,帝至武軍驛,因醉責虞候都督范洪,將殺之。邕諫,以為若非酒行戮,族誅人無所怨;假實有大罪,因酒殺人,恐招橫議。洪因得免死。邕又以軍人教習田獵,依令十二月,月別三圍,以為疲弊,請每月兩圍。又奏河陽、晉州,與周連境,請於河陽、懷州、永橋、義寧、烏籍各徙六州軍人并家,立軍府安置,以備機急之用。



 帝並從之。未幾,出為趙州刺史,侍中、護軍、大
 中正悉如故。謂曰:「朝臣未有帶侍中、護軍、中正臨州者,以卿舊勳,故有此舉。放卿百餘日休息,至秋間,當即召。」邕政頗嚴酷,然抑挫豪強,公事甚理。尋除中書監,仍侍中,遷尚書右僕射。



 武平初,坐斷事阿曲,為御史所劾,除名。久之,以舊恩,復除將軍、開府,累遷尚書令,封晉昌王。高思好構逆,令邕赴晉陽監勒諸軍。事平,錄尚書事。屬周師攻洛陽,右丞相高阿那肱赴援,邕配割不甚從允,那肱譖之,由是被疏。七年,車駕將幸晉陽,敕斛律孝卿總騎兵,事多自決。邕恃舊,一旦為孝卿所輕,鬱怏形於辭色。帝從平陽敗後,狼狽歸鄴,邕懼那肱譖醖,恨孝卿
 輕已,遂留晉陽,與莫多婁敬顯等樹安德王為帝。尋降周,邕依例授上開府儀同大將軍。再遷戶部,轉少司馬,封安福郡公,遷鳳州刺史。隋開皇初,卒。



 邕性識明敏,在齊一代,典執兵機。是以九州軍士,四方勇募,強弱多少,番代往還,器械精粗,糧儲虛實,精心勤事,莫不諳知。自太寧以來,奢侈糜費,比及武平之末,府藏漸虛,邕支度取捨,大有裨益。然既被任遇,意氣漸高,其未經府寺陳訴越覽辭牒,條數甚多,俱為憲臺及左丞彈劾,並御注放免。司空從事中郎封長業、太尉記室參軍平濤並為征官錢違限,邕各杖背三十。齊時宰相,未有撾撻朝士,
 至是,大駭物望。



 三子:長子君明,開府儀同三司,開皇初,卒於應州刺史。次子君徹,中書舍人,隋戎、順二州刺史,大業中,卒於武賁郎將。少子君德,以邕降周,伏法。



 齊朝因神武作相,丞相府外兵、騎兵曹,分掌兵馬。及受禪,諸司咸歸尚書,唯此二曹不廢,令唐邕、白建主之,謂之外兵省、騎兵省。後邕、建位望轉隆,各置省主,令中書舍人分判二省事,故世稱唐、白云。



 白建,字彥舉,太原陽邑人。初入大丞相府任兵曹,典文帳,明解書計,為同局所推。天保末,兼中書舍人。孝昭輔政,除大丞相騎兵參軍。河清二年,除員外散騎常侍,
 仍舍人。三年,突厥入境,代、忻二牧,悉是細馬,合數萬疋,在五臺山北合谷中避賊。賊退,敕建送馬定州,付人養飼。建以馬瘦,違敕以便宜從事。



 戎馬無損,建有力焉。武平末,歷位尚書、特進、侍中、中書令,封高昌郡公。父長命,贈開府儀同三司、都官尚書。建雖無他才伎,勤於在公,以溫柔自處。與唐邕俱以典執兵馬,致位卿相。諸子幼弱,俱為州郡主簿;男女婚嫁,皆得勝流。卒,贈司空。



 元文遙,字德遠,河南洛陽人也。魏昭成皇帝六世孫也。五世祖常山王遵。父唏,有孝行,父卒,廬於墓側而終。文遙貴,贈特進、開府儀同三司、中書監,謚曰孝。文遙敏慧
 夙成,濟陰王暉業每云:「此子王佐才也。」暉業常大會賓客,時有人將何遜集初入洛,諸賢皆贊賞之。河間邢邵試命文遙誦之,幾遍可得。文遙一覽誦,時年始十餘歲。濟陰王曰:「我家千里駒,今定如何?」邢云:「此殆古來未有。」起家員外散騎侍郎。遭父喪,服闋,除太尉東閣祭酒。以天下方亂,遂解官侍養,隱於林慮山。



 武定中,文襄徵為大將軍府功曹。齊受禪,於登壇所授中書舍人,宣傳文武號令。楊遵彥每云:「堪解穰侯印者,必在斯人。」後忽中旨幽執,竟不知所由。如此積年。文宣後自幸禁獄,執手愧謝,親解所著金帶及御服賜之,即日起為尚書
 祠部郎中。孝昭攝政,除大丞相府功參典機密。及踐阼,除中書侍郎,封永樂縣伯,參軍國大事。及帝大漸,與平秦王歸彥、趙郡王睿等同受顧託,迎立武成。武成即位,任遇轉隆,歷給事黃門侍郎、散騎常侍、侍中、中書監。天統二年,詔特賜姓高氏,籍屬宗正,子弟依例,歲時入廟朝祀。再遷尚書左僕射,進封寧都郡公,仍侍中。



 文遙歷事三主,明達世務,每入軒大集,多令宣敕,號令文武,聲韻高朗,發吐無滯。然探測上旨,時有委巷之言,故不為知音所重。齊因魏宰縣多用廝濫,至於士流,恥居百里。文遙以縣令為字人之功,遂請革選。於是密令搜揚貴
 游子弟,發敕用之。猶恐其披訴,總召集神武門,令趙郡王睿宣旨唱名,厚加慰喻。士人為縣,自此始也。既與趙彥深、和士開同被任遇,雖沼彥深清貞守道,又不為士開貪淫亂政,在於季孟之間。然性和厚,與物無競,故時論不在彥深之下。初,文遙自洛遷鄴,唯有地十餘頃,家貧,所資衣食。魏之將季,宗姓被侮,有人冒相侵奪,文遙即以與之。及貴,此人尚在,乃將家逃竄。文遙大驚,追加慰撫,還以與之,彼人愧而不受。彼此俱讓,遂為閑田。



 至後主嗣位,趙郡王睿、婁定遠等謀出和士開,文遙亦參其議。睿見殺,文遙由是出為西兗州刺史。詣士開別,士
 開曰:「處得言地,使元家兒作令僕,深負朝廷。」既言而悔,仍執手慰勉之。猶慮文遙自疑,用其子行恭為尚書郎,以慰其心。



 士開死,自東徐州刺史徵入朝,竟不用,卒。



 行恭美姿貌,有父風,兼俊才。位中書舍人,待詔文林館。齊亡,與陽休之等十八人同入關,稍遷司勛下大夫。隋開皇中,位尚書郎,坐事徙瓜州而卒。行恭少頗驕恣,文遙令與范陽盧思道交遊。文遙嘗謂思道云:「小兒比日微有所知,是大弟之力。然白擲劇飲,甚得師風。」思道答云:「六郎辭情俊邁,自是克荷堂構。



 而白擲劇飲,亦天性所得。」



 行恭弟行如,亦聰慧早成。武平末,著作佐郎。



 趙隱,字彥深,自云南陽宛人,漢太傅喜之後。高祖父難為齊州清河太守,有惠政,遂家焉。清河後改為平原,故為平原人也。隱避齊廟諱,改以字行。父奉伯,仕魏,位中書舍人,行洛陽縣令。彥深貴,贈司空。彥深幼孤貧,事母甚孝。年十歲,曾候司徒崔光。光謂賓客「古人觀眸子以知人,此人當必遠至。」性聰敏,善書計,安閑樂道,不雜交游,為雅論所歸服。昧爽,輒自掃門外,不使人見,率以為常。



 初為尚書令司馬子如賤客,供寫書。子如善其無誤,欲將入觀省舍。隱靴無氈,衣帽穿弊,子如給之。用為書令史,月餘,補正令史。神武在晉陽,索二史,子如舉彥
 深。後拜子如開府參軍,超拜水部郎。及文襄為尚書令攝選,沙汰諸曹郎,隱以地寒被出,為滄州別駕,辭不行。子如言於神武,徵補大丞相功曹參軍,專掌機密。文翰多出其手,稱為敏給。神武曾與對坐,遣造軍令,以手捫其額曰:「若天假卿年,必大有所至。」每謂司徒孫騰曰:「彥深小心恭慎,曠古絕倫。」



 及神武崩,秘喪事,文襄慮河南有變,仍自巡撫,乃委彥深後事,轉大行臺都官郎中。臨發,握手泣曰:「以母弟相託,幸得此心。」既而內外寧靜,彥深之力。



 及還發喪,深加褒美,乃披郡縣簿為選,封安國縣伯。從征潁川,時引水灌城,城雉將沒,西魏將王思
 政猶欲死戰。文襄令彥深單身入城告喻,即日降之,便手牽思政出城。文襄大悅。先是文襄謂彥深曰:「吾昨夜夢獵,遇一群豕,吾射,盡獲之。



 獨一大豕不可得,卿言當為吾取,須臾獲豕而進。」至是,文襄笑曰:「夢驗矣。」



 即解思政佩刀與彥深曰:「使卿常獲此利。」



 文宣嗣位,仍典機密,進爵為侯。天保初,累遷秘書監。以為忠謹,每郊廟,必令兼太僕,執御陪乘。轉大司農。帝或巡幸,即輔贊太子知後事。出為東南道行臺尚書、徐州刺史。為政尚恩信,為吏人所懷。多所降下,所營軍處,士庶追思,號趙行臺頓。文宣璽書勞勉,徵為侍中,仍掌機密。



 河清元年,進爵安樂
 公。累遷尚書左僕射、齊州大中正,監國史,遷尚書令,位特進,封宜陽王。武平二年,拜司空。為祖珽所間,出為西兗州刺史。四年,徵為司空,轉司徒。丁母憂,尋起為本官。七年六月,暴疾薨,時年七十。



 彥深歷事累朝,常參機近,溫柔謹慎,喜怒不形於色。自皇建以還,禮遇稍重,每有引見,或升御榻,常呼官號而不名也。凡諸選貢,先令銓定,提獎人物,皆行業為先,輕薄之徒,弗之齒也。孝昭既執朝權,群臣密多勸進,彥深獨不致言。孝昭嘗謂王晞云:「若言眾心皆謂天下有歸,何不見彥深有語?」晞以告,彥深不獲已,陳請。其為時重如此。常遜言恭己,未嘗以
 驕矜待物,所以或出或處,去而復還。



 母傅氏,雅有操識。彥深三歲,傅便孀居,家人欲以改適,自誓以死。彥深五歲,傅謂之曰:「家貧兒小,何以能濟?」彥深泣而言曰:「若天哀矜,兒大當仰報。」傅感其意,對之流涕。及彥深拜太常卿,還,不脫朝服,先入見母,跪陳幼小孤露,蒙訓得至於此。母子相泣久之,然後改服。後為宜陽國太妃。



 彥深有七子,仲將知名。沈敏有父風溫良恭儉,雖妻子亦未嘗怠慢,終日儼然。



 學涉群書,善草隸,雖與弟書,書字楷正。云:「草不可不解,若施之於人,即似相輕易;若當家卑幼,又恐其疑所在宜爾。是以必須隸筆。」彥深乞轉萬年縣
 子授之,位給事黃門侍郎、散騎常侍。隋開皇中,位吏部郎,終於安州刺史。



 齊朝宰相,善始令終唯彥深一人。然諷朝廷以子叔堅為中書侍郎,頗招物議。



 時馮子琮子慈明、祖珽子君信並相繼居中書,故時語云:「馮、祖及趙,穢我鳳池。」



 然叔堅身才最劣。



 赫連子悅,字士欣,僭夏赫連勃勃之後也。神武起兵時,為濟州別駕,勸刺史侯景赴神武。後除林慮太守。文襄往晉陽,由郡境,問所不便。悅云:「臨水、武安,去郡遙遠,山嶺重疊。若更屬魏郡,則地平路近。」文襄笑曰:「卿徒知便人,不覺損幹。」悅答曰:「所言者人所疾苦,不敢以私潤負
 公心。」文襄善之,乃敕依事施行。自是人屬近便,行路稱之。



 天保中,為揚州刺史。先是城門早閉晚開,廢於農作。子悅到,乃命以時開閉,人吏便之。累遷鄭州刺史,政為天下之最。入為都官尚書。鄭州人馬子韶、崔孝政等八百餘人,請立碑頌德,有詔許焉。加位開府,歷行北豫州事,兼吏部尚書。子悅在官,唯以清勤自守,既無學術,又闕風儀,人倫清鑒,去之彌遠,一旦居銓衡之首,大招物議。由是除太常卿,兼侍中,聘周使主,卒。



 子仲章,中書舍人。



 馮子琮,字子琮,長樂信都人,北燕主馮弘之後也。祖嗣
 興,相州刺史。父靈紹,尚書郎、太中大夫。子琮貴,贈開府儀同三司。子琮性識聰敏,為外祖滎陽鄭伯猷所異。初襲爵滎陽縣子。齊天保初,改為長安縣男。皇建初,為尚書駕部郎中,攝庫部。孝昭曾閱簿領,試令口陳。子琮諳對無有遺失。時梁丞相王琳歸國,孝昭詔子琮觀其形勢。琳即與赴鄴,甚見嘉賞。子琮妻,胡皇后姊也,故詔與胡長粲輔導太子。後轉太子中庶子。



 天統元年,武成禪位後主,謂子琮曰:「少君左右,宜得正人,以卿心存正直,今以後事相委。」再遷散騎常侍,奏門下事。尋兼并省祠部尚書。後與胡長粲有隙,武成深誡之曰:「脣亡齒寒,勿
 復如此。」武成在晉陽,既居舊殿,少帝未有別所,詔子琮監造大明宮。成,帝怪其不宏麗,子琮曰:「至尊幼承大業,欲令敦儉,以示萬邦。兼此北連天闕,不宜崇峻。」帝稱善。又詔子琮監議五禮,與趙郡王睿分爭異同,略無降下,大為識者所鄙。



 及武成崩,和士開秘喪三日。子琮問其故。士開引神武、文襄初崩,並秘不舉喪,至尊年少,恐王公貳,欲追集,然後與詳議。時趙郡王睿先預帷幄之謀,子琮素知士開忌睿及領軍婁定遠,恐其矯遺詔出睿外任,奪定遠禁衛權,因答支:「大行,神武之子,今上又是先皇傳位,君臣富貴,皆至尊父子之恩,但令一無改易,
 必無異望。世異事殊,不得與霸朝相比。且公不出宮門,已經數日,升遐之事,行路皆傳,久而不舉,恐有他變。」及發喪,元文遙以子琮太后妹夫,恐其獎成太后干政,說趙王睿及和士開出之。拜鄭州刺史。既非後主本意,賞賜甚厚。仍轉滄州別駕,封寧都縣伯。太后為齊安王納子琮長女為妃,子琮因請假赴鄴,遂授侍中、轉吏部尚書。其妻放縱,請謁公行,賄貨填積。守宰除授,先定錢帛,然後奏聞。



 其所通致,事無不允。子琮亦不禁制。又廣拓傍聆,增修宅宇,以夜繼晝,未曾休息。斛律光將兵度玉壁,至龍門。周有移書,別須籌議。詔子琮乘傳赴軍,與周
 將韋教寬面相要結。龍門等五城,因此內附。後主以為子琮之功,封昌黎郡公。遷尚書右僕射,仍攝選侍中如故。



 和士開居要日久,子琮舊所附託,中雖阻異,其後還相彌縫。士開弟士休與盧氏成婚,子琮檢校趨走,與士開府僚不異。時內外除授,多由士開奏擬,子琮既恃內戚,兼帶選曹,自擅權寵,頗生間隙。時陸媼勢震天下,太后與之結為姊妹,而和士開於太后有醜聲。子琮欲陰殺陸媼及士開,因廢帝而立琅邪王儼。以謀告儼,儼許之,乃矯詔殺士開。及儼見執,言子琮教己。太后怒,又使執子琮,遣右衛大將軍侯呂芬就內省以弓弦絞殺之。
 使內參以庫車載尸歸其家。諸子方握槊,聞庫車來,以為賜物,大喜,開視乃哭。



 子琮微有識鑒,頗慕存公。及位望轉隆,宿心頓改,擢引非類,公為深交,縱其子弟,不依倫次。又專營婚媾,歷選上門,例以官爵許之,旬月便驗。頓丘李克、范陽盧思道、隴西李胤伯、李子希、滎陽鄭庭堅並其女婿,皆至超遷。其矯縱如此。



 祖珽先與子琮有隙,於後具奏此事,諸子並坐此除名。太后以為言,又被擢用。子琮有五子,慈明取知名。



 慈明字無佚,在劉為中書舍人。隋開皇中,兼內史舍人。大業中,位尚書兵部郎,加朝請大夫。十三年,攝江都郡
 丞事。李密之逼東都,詔慈明追兵擊密,為密黨崔樞所執。密延與坐,論以舉兵之意。慈明曰:「慈明直道事人,有死而已,不義之言,非所敢對。」密厚禮之,冀其從己。慈明潛使奉表江都,及致書東都留守,論賊形勢。密知,又義而釋之。出至營門,為賊帥翠讓所嗔責。慈明勃然曰:「天子使我來,正欲除爾輩,不圖為賊黨所獲,我豈從汝求活邪?須殺但殺,何須罵詈!」



 讓益怒,亂刀斬之。梁郡通守楊汪上狀,煬帝歎惜之,贈銀青光祿大夫,拜其二子怦、惇俱為尚書承務郎。王世充推越王侗為主,重贈柱國、戶部尚書、黎郡公,謚曰壯武。



 長子忱,先在東都。王世充
 破李密,忱亦在軍中,遂遣奴負父屍柩詣東都,身不自送。未幾,又盛華燭納室,時論醜之。



 郎基,字世業,中山新市人也。祖智,魏魯郡太守,贈兗州刺史。父道恩,開府、陽平郡守。基身長八尺,美鬚髯,汎涉墳籍,尤長吏事。齊天保四年,除海西鎮將。遇東方白額稱亂淮南,州郡皆從逆。梁將吳明徹攻圍海西,基固守,乃至削木為箭,剪紙為羽。圍解還朝,僕射楊愔迎勞之曰:「卿本文吏,遂有武略,削木剪紙,皆無故事,班、墨之思,何以相過。」御史中丞畢義雲引為侍御史。趙州刺史尉粲,文宣外弟;揚州刺史郭元貞,楊愔妹夫。基不憚權威,
 並劾其贓罪。



 皇建初,除鄭州長史,帶潁川郡守。西界與周接境,因侯景背叛,其東西分隔,士人仍緣姻舊,私相交易。而禁格嚴重,犯者非一。基初蒞職,披檢格條,多是權時,不為久長。州郡因循,失於請讞,致密綱久放,得罪者眾。遂條件申臺省,仍以情量事科處,自非極刑,一皆決放。積年留滯,案狀膠加,數日之中,剖判咸盡。



 尋而臺省報下,並允基所陳。條綱既疏,獄訟清靜。基性清慎,無所營求,嘗語人云:「任官之所,木枕亦不須作,況重於此乎?」唯頗令人寫書。潘子義曾遺之書云:「在官寫書,亦是風流罪過。」基答云:「觀過知仁,斯亦可矣。」卒於官,贈驃騎
 大將軍、和州刺史,謚曰惠。柩將還,遠近赴送,莫不攀轅悲哭,哀不自勝。



 初,基任瀛州騎兵時,陳元康為司馬,畢義雲為屬,與基並有聲譽,為刺史元嶷所目:「三賢俱有當世才,後來皆當遠至。唯郎騎兵任真過甚,恐不足自達。」



 陳、畢後並貴顯,而基位止郡守。子茂。



 茂字蔚之,少敏慧,七歲誦騷、雅,日千餘言。十五,師事國子博士河間權會,受詩、易、三禮及玄象刑名之學。又就國子助教長樂張奉禮受三傳群言,至忘寢食。



 家人恐成病,常節其燭。及長,以博學稱,歷位保城令,有能名。周平齊,上柱國王誼薦之,授陳州戶曹。屬隋文帝為亳州
 總管,命掌書記。



 周武帝為象經,隋文從容謂茂曰:「人主之所為也,感天地,動鬼神,而象經多亂法,何以致久。」茂竊嘆曰:「此言豈常人所及!」陰自結納。隋文亦親禮之。



 後還家,為州主簿。及隋文為丞相,以書召之,言及疇昔,甚歡。授衛州司錄,有能名。尋除衛國令,時有繫囚二百,茂親自究審,數日釋免者百餘人。歷年辭訟,不詣州省。魏州刺史元暉謂曰:「長史言衛國人不敢申訴者,畏明府耳。」茂曰:「人猶水也,法令為隄防,隄防不固,必致奔突,茍無決溢,使君何患哉!」暉無以應。有部人張元預與從父弟思蘭不睦,丞尉請加嚴法。茂曰:「元預兄弟,本相憎嫉,
 又坐得罪,彌益其忿,非化人之意也。」乃遣縣中耆舊,更往敦諭,道路不絕。元預等各生感悔,詣縣頓首請罪。茂曉之以義,遂相親睦,稱為友悌。開皇中,累遷戶部侍郎。時尚書右僕射蘇威立條章,每歲責人間五品不遜。或答者乃云:「管內無五品家。」不相應領,類多如此。又為餘糧簿,擬有無相贍。茂以為繁紆不急,皆奏罷之。又奏身死王事者,子不退田;品官左貶不減地。皆發於茂。茂性明敏,剖決無滯,當時以吏乾見稱。



 煬帝即位,為尚書左丞,參掌選事。茂尤工政理,為世所稱。時工部尚書宇文愷、右翊衛大將軍于仲文競河東銀窟,茂奏劾:「愷位望
 已隆,祿賜優厚,拔葵去織,寂爾無聞,求利下交,曾無愧色;仲文大將,宿衛近臣,趨侍階庭,朝夕聞道,虞、芮之風,抑而不慕,分銖之利,知而必爭。何以貽範庶僚,示人軌物?」愷與仲文,竟坐得罪。茂與崔祖睿撰州郡圖經一百卷奏之,賜帛百段。



 時帝每巡幸,王綱已紊,茂既先朝舊臣,明習世事,然無謇諤之節,見帝忌刻,不敢措言,唯竊歎而已。以年老乞骸骨,不許。會帝征遼,以茂為晉陽宮留守。其常山贊務王文同與茂有隙,奏茂附下罔上。詔納言蘇威、御史大夫裴蘊雜推之。茂素與二人不平,因深文其罪,及弟司棣別駕楚之,皆除名徙且末郡。茂怡
 然任命,不以為憂,在途作《登隴賦》以自慰。後附表自陳,帝頗悟。十年,追還京兆,歲餘卒。子知年。



 論曰:孫搴入幕未久,倉卒致斃,神武以情寄之重,義切折肱,若不愛才子,何以成夫王業。元康以知能才幹,委質霸朝,綢繆帷幄,任寄為重,及難無茍免,忘生殉義,可謂得其地焉。杜弼識學甄明,發言讜正,禪代之際,先起異圖,王怒未終,卒蒙顯戮,直言多矣,能無及於此乎?房謨忠勤之操,始終若一。恭懿循良之風可謂世有人矣。張纂、張亮、張曜、王峻、王紘等並事霸朝,申其力用,皆有齊之良臣也。伯德之慟哭伏屍,靈光之拒關駐驆,有
 古人之風焉。顯俊明達,文武驅馳,盡其知力,不遑寧處。可謂德以稱位,能以稱官。道和爰從霸府,以終末路,四十餘載,典綜兵機,識用閑明,甚為朝臣所服。及于後主奔遁,莫知所之,首贊延宗,以從權變。既而晉陽傾覆,運極途窮,還鄴則義隔德昌,死事則情乖舊主,雖復全生握節,豈比背叛之流歟?夫縣宰之寄,綿歷古今,親人任功,莫尚於此。



 漢氏官人,尚書郎出宰百里;晉朝設法,不宰縣不得為郎。皆所以貴方城之職,重臨人之要。後魏令長,多選舊令史為之,故縉紳之流,恥居其位。爰逮有齊,此途未改。寧都公革斯流弊,弘之在人,固為美矣。司
 徒器度沈遠,有宰臣之量,始從文吏,終致臺輔,出內有常,夷險若一。而世人諭之胡廣,譏其不能廷爭。然古稱「見幾而作」,又曰「相時而動」,若時有開悟,或可希舜一功,而終遇奸回,便恐舟壑俱運,斯蓋趙公之志也。子悅牧宰流譽,子琮簿領見知,及居藻鏡,俱稱尸祿。馮溺於賄貨,於斯為甚。慈明赴蹈之義,蓋有銜須之節。郎基政績有聞,蔚之克荷堂構,美矣乎!



\end{pinyinscope}