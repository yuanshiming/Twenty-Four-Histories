\article{卷五十七列傳第四十五 周宗室}

\begin{pinyinscope}

 邵惠公顥
 子什肥導護叱羅協馮遷杞簡公連莒莊公洛生虞國公仲廣川公測弟深深子孝伯東平公神舉弟慶邵惠公顥,周文帝之長兄也。德皇帝娶樂浪王氏,是為德皇后。生顥,性至孝,居德皇后喪,哀毀過禮。德皇帝與衛可瑰戰,墜馬,顥與數騎奔救,乃免。顥遂戰歿。保定初,追贈大塚宰,封邵國公,謚曰惠。三子,什肥、導、護。



 什肥事母以孝聞。文帝入關,不能離母,遂留晉陽。文帝定秦、隴,什肥為齊神武所害。保定初,追贈大將軍、小塚宰,襲爵邵國公,謚曰景。子胄嗣。



 胄少孤,
 頗有幹略。景公之見害,以年幼下蠶室。保定初,詔以晉公護子會紹景公封。天和中,與齊通好,胄歸,襲爵邵國公。及隋文帝輔政,胄為滎州刺史,舉兵應尉遲迥,為清河公楊素所殺。國除。



 會字乾仁,胄至自齊,改封譚國公。後與護同誅。建德三年,追復封爵常武公。



 導字
 菩薩,少雄豪。初與諸父在葛榮中,榮敗,遷晉陽。與文帝隨賀拔岳入關,常從征伐。文帝討侯莫陳悅,導追斬之牽屯山,以功封饒陽縣伯。及魏文帝東征,留導為華州刺史。既而趙青雀、於伏德、慕
 容思慶等作亂,導禽伏德,斬思慶,屯渭橋會文帝軍。及事平,進爵章武郡公,加侍中。及高仲密以北豫州降,文帝東征,復以導為大都督,行華州刺史,甚得守扞之方。及大軍不利,東魏追至稠桑,知關中有備,乃退。侯景來附,詔徵隴右大都督獨孤信東下,令導代信為秦州刺史、大都督、十五州諸軍事。及齊氏稱帝,文帝討之,魏文帝遣齊王廓鎮隴右,徵導拜大將軍、大都督、二十三州諸軍事,屯咸陽。大軍
 還,乃旋舊鎮。



 導性寬明,善撫御,文帝每出征,導恒居守,深為吏人所附,朝廷重之。薨於上邽,魏帝遣侍中、漁陽王綱監護喪事,贈尚書令,謚曰孝。朝議以導撫和西戎,威恩顯著,欲令世鎮隴右,以彰厥德。乃葬上邽城西無疆原,華戎會葬者萬餘人,奠祭於路,悲號振野,皆曰「我君捨我乎」。大小相與負土成墳,高五十餘尺,周回八十餘步。為官司所止,然後泣辭而去。天和五年,重贈太師、柱國、豳國公。



 導五子,廣、亮、翼、椿、眾。亮、椿出後於杞。



 廣字乾歸,少方嚴,好文學。武成初,位大將軍、梁州總管,進封蔡國公,累遷秦州刺史、總管十三州諸軍事。性明
 察,善撫綏,人庶畏悅之。時晉公護諸子及廣弟杞公亮等侈靡踰制,廣獨率禮,又折節待士,朝野稱焉。曾侍於武帝所,食瓜美,持以奉進,帝悅之。廣以晉公護擅權,勸令挹損,護不能納。後除陜州總管,以病免。及孝公追封豳國公,召廣襲爵。初、廣母李氏以廣患,憂而成疾,遂歿。



 廣居喪加篤,乃以毀薨。世稱母為廣病,廣為母死,慈孝之道,極於一門。武帝素服親臨。其故吏儀同李克信等上表褒述,申其宿志,庶存儉約。詔曰:「昔河間才藻,追敘於中尉;東海謙約,見稱於身後。可斟酌前典,率由舊章,使易簀之言,得申遺志,黜殯之請,無虧令終。」於是贈本官,
 加太保、隴右十四州諸軍事、秦州刺史,謚曰文。葬於隴右,所司一遵儉約之典。子洽嗣,隋文輔政,被害,國除。



 翼字乾宜,封西陽郡公,早薨,曰昭。無子,以杞公亮子溫嗣,後坐亮反誅,國除。眾字乾道,少不慧,封天水郡公,為隋文所誅。



 護字薩保,幼方正有志度,特為德皇帝所愛。文帝之入關,以年小不從。普泰初,始自晉陽至平涼,時年十七。文帝諸子並幼,遂委以家務,內外無不嚴肅。文帝歎之,以為類已。及臨夏州,留護事賀拔岳。岳被害。文帝至平涼,以護為都督,從破侯莫陳悅。後以迎魏帝功,封水池縣
 伯。從文帝禽竇泰,復弘農,破沙苑,戰河橋,並有功。芒山之役,為敵人所圍,賴都督侯伏龍恩救,乃免。坐免官,尋復本位。大統十三年,進封中山公。十五年,遷大將軍。與于謹征江陵,進兵徑至江陵城下,以待大軍至,圍而剋之。師還,護又討平襄陽蠻帥向天保等萬餘落。初行六官,拜司空。



 文帝西巡,至牽屯山遇疾,召護至涇州,見文帝。帝曰:「吾形容若此,必不濟。諸子幼,天下事以屬汝。」護涕泣奉命。行至雲陽,文帝崩,護秘之,至長安乃發喪。時嗣子沖幼,強寇在近,人情不安。護綱紀內外,撫循文武,眾心乃定。



 先是,文帝常云「我得胡力」,當時莫曉其指,時
 人以「護」字當之。尋拜柱國。



 文帝山陵畢,護以天命有歸,遣諷魏帝以禪代事。孝閔踐阼,拜大司馬,封晉國公,邑萬戶。趙貴、獨孤信等將謀襲護,護因貴入朝,執之,黨與皆伏誅。拜大塚宰。



 時司會李植、軍司馬孫恒等密要宮伯乙弗鳳、張光洛、賀拔提、元進等為腹心,說帝,言護不守臣節,宜圖之。帝然之,數將武士於後園,為執縛勢。護微知之。



 出植為梁州,恒為潼州,欲遏其謀。後帝思植等,每欲召之。護諫曰:「天下至親,不過兄弟。若兄弟自構嫌隙,他人何易可親?但恐除臣後,姦回得逞其欲,非唯不處陛下,亦危社稷。」因泣涕,久之乃止。帝猶猜,鳳等益懼,
 密謀滋甚,遂剋日將誅護。光洛告護,護乃召柱國賀蘭祥、小司馬尉遲綱等以鳳謀告之。祥並勸廢帝。



 時綱總領禁兵,護乃遣綱入宮,召鳳等議事,以次執送護第。因罷散宿衛兵,遣祥逼帝,幽於舊邸。於是召公卿畢集護第。護曰:「先王勤勞王業三十餘年,寇賊未平,奄棄萬國。寡人地則猶子,親受顧命,以略陽公既居正嫡,與公等立而奉之,革魏興周,為四海主。自即位已來,荒淫無度,暱近群小,疏忌骨肉,大臣重將,咸欲誅夷。若此謀遂行,社稷必致傾覆。寡人若死,將何面目以見先王?今日寧負略陽公,豈可負社稷!寧都公年德兼茂,仁孝聖慈,今
 欲廢昏立明,公等以為何如?」



 群公咸曰:「此公之家事,敢不唯命是聽!」於是斬鳳等於門外,并誅植、恒。尋弒帝,迎明帝於岐州而立之。



 二年,拜太師,賜路車冕服,封子至為崇業郡公。初改雍州刺史為牧,以護為之,并賜金石之東。



 武成元年,護上表政,帝許之,軍國大事尚委於護。帝性聰睿,有識量,護深憚之。有李安者,本以鼎俎得寵於護,擢為膳部下大夫。至是,護令安因進食加毒,帝遂崩。護立武帝,百官總已以聽護。



 自文帝為丞相,立左右十二軍,總屬相府。文帝崩後,皆受護處分,凡所征廢,非護書不行。護第屯兵禁衛,盛於宮闕。事無巨細,皆
 先斷後聞。保定元年,以護國都督中外諸軍事,令一府總於天官。或有希護旨者,云周公德重,魯立文王之廟,以護功比周公,宜用此禮。於是詔於同州晉國第立德皇帝別廟,使護祭焉。三年,詔自今詔誥及百司文書並不得稱公名,以彰殊禮。護抗表固讓。初,文帝創業,即與突厥和親,謀為掎角,共圖高氏。是年,乃遣柱國楊忠與突厥東伐,破齊長城,至并州而還,期後年更舉,南北相應。齊主大懼。



 先是,護母閻與皇第四姑及諸戚屬並沒齊,皆被幽縶。護居宰相後,每遣間使尋求,莫知音息。至是,並許還朝,且請和好。四年,皇姑先至。齊主以護權重,
 乃留其母,以為後圖。仍令人為閻作書與護曰:吾念十九入汝家,今以八十矣。凡生汝輩三男二女,今日目下,不睹一人,興言及此,悲纏肌骨。賴皇齊恩恤,差安衰暮。又得與汝楊氏姑及汝叔母紇干、汝嫂劉及汝新婦等同居,頗以自適。但為微有耳疾,大語方聞,行動飲食,幸無多損。



 汝與吾別之時,年尚幼小,以前家事,或不委曲。昔在武川鎮,生汝兄弟,大才屬鼠,第二屬兔,汝身屬蛇。鮮于脩禮起日,吾合家大小先在博陵郡住,相將欲向左人城。至唐河北,被定州官軍打敗。汝祖及第二叔時俱戰亡。叔母賀拔及兒元寶、汝叔母紇干及兒菩提并
 吾與汝六人,同被禽捉入定州城。未幾間,將吾及汝送與元寶掌,賀拔、紇干各別分散。寶掌軍營在唐城內,經停三日。寶掌所掠得男夫女婦可六七千人,悉送向京。吾時與汝同被送限。至定州城南,夜宿同鄉人姬庫根家。蠕蠕奴望見鮮于脩禮營火。語吾云:「今走向本軍。」既至營,遂告吾輩在此。



 明旦日出,汝叔將兵邀截,吾及汝等還得向營。汝時年十二,共吾並乘馬隨軍,可不記此事由緣也?後吾共汝在壽陽任。時元寶、菩提及汝姑兒賀蘭盛洛,并汝身四人同學。博士姓成,為人嚴惡,汝等四人謀欲加害。吾共汝叔母聞知,各捉其兒打之。唯盛
 洛無母,獨不被打。後爾朱天柱亡歲,賀拔阿斗泥在關西,遣人迎家累。



 汝叔亦遣奴來富迎汝及盛洛等。汝時著緋綾袍、銀裝帶,盛洛著紫纖成纈通身袍,黃綾里,並乘騾同去。盛洛小於汝,三人並喚吾作阿摩敦。如此之事,當分明記之。



 今又寄汝小時所著錦袍表一領,至宜撿看,知吾含悲抱戚,多歷年祀。



 禽獸草木,母子相依,吾有何罪,與汝分隔,今復何福,還望見汝。世間所有,求皆可得,母子異國,何處可求!假汝貴極公王,富過山海;有一老母,八十之年,飄在千里,死亡旦夕,不得一朝暫見,不得一日同處,寒不得汝衣,飢不得汝食,汝雖窮榮極盛,
 光耀世間,汝何用為?於吾何益?吾今日之前,汝既不得申其供養,事往何論。今日以後,吾之殘命,唯繫於汝。戴天履地,中有鬼神,勿云冥昧,而可欺負。



 楊氏姑今雖炎暑,猶能先發。關、河阻遠,隔絕多年,書依常體,慮汝致惑,是以每存款質,兼亦載吾姓名,當識此理,勿以為怪。



 護性至孝,得書悲不自勝,左右莫能仰視。報書云:區宇分崩,遭遇災禍,違離膝下,三十五年。受形稟氣,皆知母子,誰知薩保,如此不孝!宿殃積戾,唯應賜鍾,豈悟綱羅,上嬰慈母。但立身立行,不負一物,明神有識,宜先哀憐。而子為公侯,母為俘隸,熱不見母熱,寒不見母寒,衣不知
 有無,食不知飢飽,泯如天地之外,無由暫聞。晝夜悲號,繼之以血,分懷冤酷,終此一生,死若有知,冀奉見於泉下耳。不謂齊朝解綱,惠以德音,摩敦、四姑,並許哀放。初聞此旨,魂爽飛越,號天叩地,不能自勝。四姑即蒙禮送,平安入境,以今月十八日於河東拜見。遙奉顏色,崩慟肝腸。但離絕多年,存亡阻隔,相見之始,口未忍言。唯敘齊朝寬弘,每存大德,云與摩敦雖處宮禁,常蒙優禮,今者來鄴,恩遇彌隆。重降矜哀,聽許摩敦垂敕,曲盡悲酷,備述家事。伏讀未周,五情屠割。書中所道,無一事敢忘。摩敦年尊,又加憂苦,常謂寢食貶損,或多遺漏。



 伏奉論
 述,次第分明。一則以悲,一則以喜。當鄉里破敗之日,薩保年以十餘歲,鄰曲舊事,猶自記憶;況家門禍難,親戚流離?奉辭時節,先後慈訓,刻肌刻骨,常纏心府。



 天長喪亂,四海橫流,太祖乘時,齊朝撫運,兩河三輔,各遇神機。源其事迹,非相負背。太祖升遐,未定薩保,薩保屬當猶子之長,親受顧命。雖身居重任,職當憂責,至於歲時稱慶,子孫在庭,顧視悲摧,心情斷絕,胡顏履戴,負愧神明。



 齊朝霈然之恩,既已霑洽,愛敬之至,施及傍人。草木有心,禽魚感澤,況在人倫,而不銘戴?有國有家,信義為本,伏度來期,已應有日。一得奉見慈顏,永畢生願。



 生死肉骨,豈過
 今恩,負山戴岳,未足勝荷。二國分隔,理無書信,主上以彼朝不絕母子之恩,亦賜許奉答。不期今日,得通家問,伏紙嗚咽,言不宣心。蒙寄薩保別時所留錦袍表,年歲雖久,宛然猶識,抱此悲泣,至於拜見,事歸忍死,知復何心!



 齊朝不即發遣,更令重與護書,要護重報。護復書,往返至於再三,而母竟不至。朝議以其失信,令有司移齊,移未送而母至。舉朝慶悅,大赦天下。護與母暌隔多年,一朝聚集,凡所資奉,窮極華盛。每四時伏臘,武帝率諸親戚,行家人禮,稱觴上壽,榮貴之極,振古未聞。



 是年,突厥復率眾赴期。護以齊氏初送國親,未欲即行,復慮失
 信蕃夷,不得已,遂請東征。九月,詔征二十四軍及左右廂散隸、秦隴巴蜀兵、諸蕃國眾二十萬人。十月,帝於廟庭授護斧鋮。出軍至潼關,乃遣柱國尉遲迥為前鋒,大將軍權景宣率山南兵出豫州,少師楊標出軹關。護連營漸進,屯軍弘農。迥圍洛陽,柱國齊王憲、鄭公達奚武等營芒山。護性無戎略,此行又非本心,故師出雖久,無所克獲。



 以無功,與諸將稽首請罪,帝弗之責。天和二年,護母薨,尋詔起令視事。五年,詔賜護軒懸之樂,六佾之舞。



 護性甚寬和,然暗於大體。自恃建立功,久當權軸,所任皆非其人。兼諸子貪殘,僚屬縱溢,莫不蠹政害人。
 以其暴慢,密與衛王直圖之。七年三月十八日,護自同州還,帝御文安殿見護訖,引入含仁殿,朝皇太后。先是,帝於禁中見護,常行家人禮。護謁太后,太后必賜之坐,帝每立侍。至是,護將入,帝謂曰:「太后春秋既尊,頗好酒,諸親朝謁,或廢引進。喜怒有時乖爽。比諫,未蒙垂納。兄今願更啟請。」因出懷中酒誥授護曰:「以此諫太后。」護入,如帝所誡,讀示太后。



 未訖,帝以玉珽自後擊之,踣地,又令宦者何泉以御刀斫之。泉懼,斫不能傷。時衛王直先匿於戶內,乃出斬之。



 初,帝欲圖護,王軌、宇文神舉、宇文孝伯頗預其謀。是日,軌等並在外,更無知者。殺護訖,乃
 召宮伯長孫覽等,即令收護子柱國譚國公會、大將軍莒國公至、崇業公靜、正平公乾嘉及乾基、乾蔚、乾祖、乾威等,并柱國侯伏侯龍恩、龍恩弟大將軍萬壽、大將軍劉勇、中外府司錄尹公正、袁傑、膳部下大夫李安等,於殿中殺之。齊王憲曰:「安出自阜隸,所典庖廚而已,未足加戮。」帝曰:「汝不知耳,世宗之崩,安所為也。」十九日,乃詔暴護等罪,大赦,改天和七年為建德元年。



 護世子訓為蒲州刺史,其夜遣柱國越公盛乘傅鎮蒲州,徵訓赴京師,至同州賜死。



 護長史叱羅協、司錄馮遷及所親任者皆除名。護子昌城公深使突厥,遣開府宇文德齊
 璽書就殺之。三年,詔復護及諸子先封,謚護曰蕩,並改葬之。



 叱羅協,代郡人,本名與武帝諱同,後改焉。少寒微,嘗為州小吏,以恭謹見知。竇泰為御史中尉,以協為書侍御史。泰向潼關,協為監軍。泰死,協見獲。文帝授大丞相東合祭酒,累遷相府屬、從事中郎。協歷事二京,詳練故事,又深自剋勵,文帝頗委任之。然猶以家屬在東,疑其戀本。及河橋戰敗,協隨軍還。文帝知協不貳,封冠軍縣男,進爵為侯。後為大將軍尉遲迥長史,率兵伐蜀,行潼州事。



 魏恭帝三年,文帝征協入朝,論蜀中事,乃賜姓宇文氏。



 晉公護既殺孫恒、李植等,欲委腹心於司會柳
 慶、司憲令狐整等,二人並辭,俱薦協。護遂征協入朝,引與同宿,深寄託之。協誓以軀命自效。護大悅,以為得協之晚。稍遷護府長史,進爵為公,常在護側。明帝知其材識庸淺,每按抑之,數謂曰:「汝何知也!」猶以護所親任,每含容之。及明帝崩,便授協司會中大夫、中外府長史。協形貌瘦小,舉措偏急,既以得志,每自矜高;及其所言,多乖事衷,當時莫不笑之。護以其忠己,每提獎之。協既受護重委,冀得婚連帝室,乃求復舊姓叱羅氏,許之。又進位柱國。護以協年老,許其致仕,而協貪榮,未肯告退。及護誅,除名。建德三年,以協宿齒,授儀同三司、賜爵南陽
 郡公。卒,子金剛嗣。



 馮遷字羽化,弘農人。少脩謹,有幹能,為護府司錄。性質直,小心畏慎,兼明練時事,善於斷決,每校閱文簿,孜孜不倦,以此甚為護委任。後授陜州刺史。



 遷本寒微,不為時輩所重。一旦刺舉本州,唯以廉恭接待鄉邑,人無怨者。復入為司錄,累遷小司空。自天和後,以年老,委任稍衰。及護誅,猶除名。卒於家。子恕,位儀同三司。



 杞簡公連,幼而謹厚,臨敵果毅。隨德皇帝遇定州軍於唐河,俱戰歿。保定初,追贈太傅、柱國大將軍、大司徒,封杞國公,謚曰簡。子元寶,為齊神武所害。保定初,追贈大
 將軍、小司徒,襲封杞國公,謚曰烈,以章武公導子亮嗣。



 亮字乾德,位梁州總管。及豳國公廣薨,以亮為秦州總管,廣所部悉以配焉。



 在州甚無政績。尋進柱國,從東伐,進上柱國。仍從平鄴,遷大司徒。大象初,以行軍總管與元帥鄭國公韋孝寬等伐陳。還至豫州,密謀襲孝寬營,將反逆,孝寬追斬之。子胲明坐亮誅,詔以亮弟椿為烈公後。



 椿字乾壽,位上柱國、大司徒。大定中,為隋文帝所害,并其五子。



 莒莊公洛生,少任俠,好施愛士,北州賢俊皆與之游,而才能多出其下。及葛榮破鮮于脩禮,以洛生為漁陽王,
 仍領德皇帝餘眾,時人皆呼為洛生王。洛生善撫將士,是以克獲常冠諸軍。爾朱榮定山東,時洛生在虜中,榮雅聞其名,心憚焉。



 尋為榮所害。保定初,追贈大將軍,封莒國公,謚曰莊。



 子菩薩,為齊神武所害。保定初,追贈大將軍、小宗伯,襲爵,謚曰穆,以晉公護子至嗣。至字乾附,後坐父護誅,詔以衛王直子賓為穆公後。賓字乾瑞,尋坐直誅,而齊王憲子廣都郡公貢襲。貢字乾貞,宣帝初,被誅,國除。



 虞國公仲,德皇帝從父兄也。卒於代。保定初,追贈太傅、柱國大將軍、大司徒,封虞國公。子興嗣。



 興生,屬兵亂,與
 仲相失,年幼莫知其戚屬遠近,與文帝兄弟,初不相識。沙苑之敗,預在行間,被虜,隨例散配諸軍。興性弘厚,有志度,雖流離世故,而風範可觀。保定二年,詔訪仲子孫,興始附屬籍。武帝以興帝戚近屬,尊禮之甚厚。



 位開府儀同三司、宗師,襲爵虞國公。薨,武帝親臨慟焉。詔大司空、申國公李穆監護喪事,贈柱國大將軍,謚曰靖。



 子洛嗣,位儀同三司。隋初為介國公,為隋室賓云。



 廣川公測,字澄鏡,文帝之族子也。高祖中山、曾祖豆頺、祖騏麟、父永,仕魏,位並顯達。測性沉密,少篤學詣,仕魏,位司徒右長史,尚宣武女陽平公主,拜駙馬都尉。及孝武
 疑齊神武,詔測詣文帝,密為之備。還,封廣川縣伯。尋從孝武西遷,進爵為公。文帝為丞相,以測為右長史,委以軍國,又令測詳定宗室昭穆遠近,附於屬籍。



 歷位侍中、開府儀同三司,行汾州事。政在簡惠,頗得人和。地接東魏,數相抄竊,或有獲其為寇者,多縛送之。測皆命解縛,置之賓館,然後引與相見,如客禮焉。仍宴設,放還其國,衛送出境。自是東魏人大慚,乃不為寇,兩界遂通慶弔,時論方之羊叔子。或有告測懷貳,文帝怒曰:「測為我安邊,何為間骨肉!」乃命斬之。仍許測便宜從事。轉行綏州事。每歲河冰合後,突厥即來寇掠。先是,常預遣居人入
 城堡以避之。測至,皆令安堵。乃於要路數百處並多積柴,仍遠斥候,知其動靜。是年十二月,突厥從連谷入寇,去界數十里,測命積柴處一時縱火。突厥謂大軍至,懼而遁走,委棄雜畜輜重不可勝數。自是不敢復至。測因請置戍以備之。



 後卒於太子少保,文帝親臨慟焉,仍令水池公監護喪事,謚曰靖。



 測性仁恕,好施與。在洛陽之日,曾被竊盜,所失物即其妻陽平公主之衣服也。



 州縣禽盜,并物俱獲。測恐此盜坐之以死,不認焉,遂遇赦免。盜既感恩,請為測左右、及測從孝武西遷,事極狼狽,盜人亦從測入關,並無異志。子該嗣,位除州刺史。測弟
 深。



 深字奴于,性鯁正,有器局。年數歲,便累石為營,折草作旌旗,布置行伍,皆有軍陣之勢。父永遇見之,喜曰:「汝自然知此,後必為名將。」孝武西遷,事起倉卒,人多逃散。深時為子都督,領宿衛兵,撫循所部,並得入關。以功賜爵長樂縣伯。大統中,累轉尚書直事郎中。



 及齊神武屯蒲阪,分遣其將竇泰趨潼關,高敖曹圍洛州。周文帝將襲泰,諸將咸難之。帝隱其事,陽若未有謀,獨問策於深。深曰:「竇氏,高歡驍將,歡每仗之禦侮。今大軍就蒲阪,則歡拒守,竇必援之,內外受敵,取敗道也。不如選輕銳潛出
 小關,竇性躁急,必來決戰,高歡持重,未即救之,則竇可禽也。虜竇,歡勢自沮,迴師禦之,可以制勝。」文帝喜曰:「是吾心也。」軍遂行,果獲泰,齊神武亦退。深又說文帝進取弘農,復剋之。文帝大悅,謂深曰:「君即吾家陳平也。」



 是冬,齊神武又率大眾至沙苑,諸將皆懼,惟深獨賀。文帝問其故,對曰:「歡撫河北,甚得眾心,雖乏智謀,人皆用命,以此自守,未易可圖。今懸師度河,非眾所欲,唯歡恥失竇氏,復諫而來,所謂忿兵,一戰可禽也。不賀何為?」文帝然之。尋大破齊軍,果如所策。俄進爵為侯。六官建,拜小吏部下大夫,遷中大夫。



 武成元年,遷豳州刺史,改封安化
 縣公。保定初,除京兆尹,入為司會中大夫。



 深少喪父,事兄甚謹。性多奇譎,好讀兵書,既居近侍,每進籌策。及在選曹,頗有時譽。性仁愛,從弟神舉、神慶幼孤,深撫訓之,義均同氣,世亦以此稱焉。



 卒於位,謚曰成康。子孝伯。



 孝伯,字胡王,其生與武帝同日,文帝甚愛之,養於第內。及長,又與武帝同學。武成元年,拜宗師上士,時年十六。性沉正謇諤,好直言。武帝即位,欲引置左右。時政在家臣,不得專制,乃託言少與同業受經,思相啟發。由是護弗之猜,得入為右侍上士,恒侍讀。及遭父憂,詔令服中襲爵。武帝嘗謂曰:「公於我,猶漢高與盧綰也。」賜以十三
 環金帶。自是恒侍左右,出入臥內,朝務皆得預焉。孝伯亦竭心盡力,無所回避。至於時政得失,外間細事,皆以奏聞。帝信委之,當時莫比。及將誅晉公護,密與衛王直圖之,惟孝伯及王軌、宇文神舉等頗得參預。護誅,授開府儀同三司,歷司會中大夫、左宮正。



 皇太子既無令德,孝伯言於帝曰:「皇太子德聲未聞,請妙選正人為其師友,調護聖質,不然,悔無所及。」帝斂容曰:「卿世載鯁正,竭誠所事,觀卿此言,有家風矣。」孝伯拜謝曰:「非言之難,受之難也,深願陛下思之。」帝曰:「正人豈復過君?」於是以尉遲運為右宮正,孝伯仍為左宮正、宗師中大夫。累遷右
 宮伯。嘗因侍坐,帝問:「我兒比進不?」答曰:「皇太子比懼天威,更無罪失。」



 及王軌因內宴捋帝鬚,言太子之不善。帝罷酒,責孝伯曰:「公常謂我云太子無過,今軌有此言,公為誑矣。」孝伯拜曰:「臣聞父子之際,人所難言,臣知陛下不能割情忍愛,遂爾結舌。」帝知其意,默然久之,乃曰:「朕已委公,公其勉之。」



 及大軍東討,拜內史下大夫,令掌留臺事。軍還,帝曰:「居守之重,無忝戰功。」於是加授大將軍,進爵廣陵郡公,并賜金帛女妓等。復為宗師。每車駕巡幸,常令居守。後帝北討,至雲陽宮寢疾,驛召孝伯赴行在所,執其手曰:「吾自量必無濟理,以後事付君。」是
 夜,授司衛上大夫,總宿衛兵馬,令馳驛入京鎮守。



 宣帝即位,授小塚宰。帝忌齊王憲,意欲除之,謂孝伯曰:「公能圖之,當以其官位相授。」孝伯叩頭曰:「齊王戚近功高,棟梁所寄。臣若順旨,則臣為不忠,陛下為不孝之子也。」帝因疏之,乃與于智、鄭譯等圖其事。令智告憲謀逆,遣孝伯召入,誅之。



 帝之西征也,在軍有過行,鄭譯時亦預焉。軍還,孝伯及王軌盡以白武帝。武帝怒,撻帝數十,乃除譯名。至是,帝追憾被杖,乃問譯:「我腳上杖痕誰所為也?」



 譯曰:「事由宇文孝伯及王軌。」譯又說軌捋帝鬚事,帝乃誅軌。尉遲運懼,私謂孝伯曰:「吾徒必不免禍,奈何?」孝伯曰:「
 有老母,地下有武帝,為臣為子,知欲何之!且委質事人,本徇名義,諫而不入,將焉逃死?足下若為身計,宜且遠之。」於是各行其志。運尋出為秦州總管。帝荒淫日甚,誅戮無度。孝伯頻諫不從,由是益疏。後稽胡反,令孝伯為行軍總管,從越王盛討平之。及軍還,帝將殺之,乃託以齊王事誚之曰:「公知齊王謀反,何以不言?」對曰:「臣知齊王忠於社稷,為群小媒蘗,加之以罪。臣以言必不用,所以不言。且先帝屬微臣輔陛下,今諫而不從,實負顧託。以此為罪,是所甘心。」帝慚,俯首不語。令賜死於家,時年三十六。



 及隋文帝踐極,以孝伯、王軌忠而獲罪,
 並令收葬,復其官爵。嘗謂高穎曰:「宇文孝伯實有周良臣,若此人在朝,我輩無措手處。」子歆嗣。



 東平公神舉,文帝之族子也。高祖普陵、曾祖求男,仕魏位並顯達。祖金殿,魏兗州刺史安喜縣侯。父顯和,少而襲爵,性矜嚴,頗涉經史,膂力絕人,彎弓數百斤,能左右馳射。孝武之在蕃,顯和早蒙眷遇。時屬多難,嘗問計於顯和。顯和具陳宜杜門晦迹,相時而動,帝深納焉。及即位,拜閣內都督,封城陽縣公,以恩舊遇之甚厚。顯和所居隘陋,乃撤殿省賜為寢室,其見重如此。及齊神武專政,帝每不自安,問顯和曰:「天下洶洶,將如之何?」對曰:「莫
 若擇善而從。」因誦詩云:「彼美人兮,西方之人兮。」帝曰:「是吾心也。」遂定入關策。以其母老,令預為計。對曰:「今日之事,忠孝不並。然臣不密則失身,安敢預為私計。」帝愴然改容曰:「卿,我之王陵也。」遷朱衣直合、合內大都督,改封長廣縣公。從孝武入關。至溱水,周文帝素聞其善射而未之見,俄而水傍有一小鳥,顯和射中之。



 文帝笑曰:「我知卿工矣。」進位車騎大將軍、儀同三司、散騎常侍。卒。建德三年,追贈驃騎大將軍、開府儀同三司。



 神舉早孤,有夙成之量。及長,神情倜儻,志略英贍,眉目疏朗,儀貌魁梧。



 明帝初,起家中侍上士。帝留意翰林,而神舉雅好篇什,
 每游幸,神舉恒從。襲爵長廣縣公。天和元年,累遷右宮伯中大夫,進爵清河郡公。建德三年,自京兆尹出為熊州刺史,齊人憚其威名。及帝東伐,從平并州,即授刺史。州既齊氏別都,多有姦猾,神舉示以威恩,遠近悅服。改封武德郡公,進柱國大將軍,又改封東平郡公。宣政元年,轉司武上大夫。及幽州人盧昌期等據范陽反,詔神舉討禽之。時齊黃門侍郎盧思道亦在反中,賊平,將解衣伏法,神舉乃釋而禮之,即令草露布。屬稽胡反,寇西河,神舉與越王盛討之。時突厥赴救,神舉以奇兵擊之,突厥敗走,稽胡款服。即授并州總管。



 神舉見待於武帝,
 處心腹之任,王軌、宇文孝伯等屢言皇太子之短,神舉亦頗預焉。及宣帝即位,荒淫無度,神舉懼及禍,懷不自安。初定范陽之後,威聲甚振,帝亦忌其名望,兼以宿憾,遂使人齊酖酒賜之,薨於馬邑,時年四十八。



 神舉美風儀,善辭令,博涉經史,性愛篇章,尤工奇射。臨戎對寇,勇而有謀,蒞職當官,每著聲績。兼好施愛士,以雄豪自居,故得任兼文武,聲彰外內。百僚無不仰其風則,先輩舊齒至於今稱之。



 子同嗣,位至儀同大將軍,神舉弟慶。



 慶字神慶,沈深有器局,少以敏見知。初受業東觀,頗涉經史。既而謂人曰:「書足記姓名而已,安能久事筆硯
 為腐儒業乎?」時文州賊亂,慶應募從征,以功授都督。衛王直鎮山南,引為左右。慶善射,有膽氣,好格猛獸,直甚壯之。稍遷車騎大將軍、儀同三司。及誅宇文護,慶有謀焉。時授驃騎大將軍,加開府。從武帝攻河陰,先登攀堞,與賊短兵接,中石乃墜,絕而後蘇。帝勞之曰:「卿勇可以賈人也。」復從武帝拔晉州,齊兵大至,慶與齊王憲輕騎覘之,卒與賊相遇,為賊所窘。憲挺身而遁。慶退據汾橋,眾賊爭進,慶射之,所中人馬必倒,賊乃稍卻。



 及拔高壁,剋并州,下信都,禽高湝,功並居最。進位大將軍,封汝南郡公。尋以行軍總管擊延安反胡,平之。歷延、寧二州總管。



 隋文帝為丞
 相,以行軍總管征江表,次白帝,以勞進上大將軍。帝與慶有舊,甚見親待,令督丞相軍事,委以心腹。尋加柱國。開皇初,拜左武衛將軍,進上柱國。數年,除涼州總管。歲餘徵還,不任以職。



 初,文帝龍潛時,嘗與慶言,謂曰:「天元質無積德,其相貌壽亦不長,加以法令繁苛,耽恣聲色,以吾觀之,殆將不久。又諸侯微弱,各令就國,曾無深根固本之計,羽翮既翦,何及遠?尉遲迥貴戚,早著聲望,國家有釁,必為亂階。然智量庸淺,子弟輕佻,貪而少惠,終致亡滅。司馬消難反覆之虜,亦非池內之物,變在俄頃。但輕薄無謀,未能為害,不過自竄江南耳。庸蜀險隘,
 易生艱阻,王謙愚蠢,素無籌略,但恐為人所誤,不足為虞。」未幾,上言皆驗。及此,慶恐上遺忘,不復收用,欲見舊蒙恩顧,具錄前言,為表奏之。上省表大悅,下詔曰:「朕言之驗,自是偶然;公乃不忘,彌表誠節。深感至意,嘉尚無已。」自是上每加優禮。



 卒於家。



 子靜亂,尚隋文女廣平公主,位儀同、安德縣公、熊州刺史。先慶卒。



 靜亂子協,位右翊衛將軍。宇文化及之亂,遇害。



 協弟晶,字婆羅門,大業中養于宮內,後為千牛左右。煬帝甚親暱之,每有游宴,必侍從。至於出入臥內,伺察六宮,往來不限門禁。時人號為宇文三郎。與宮人淫亂,至於妃嬪公主亦有醜聲。
 蕭后言於帝,晶聞,懼不敢見。協因奏晶壯,不可久在宮掖。帝不之罪,召入,待之如初。化及殺逆際,為亂兵所害。



 論曰:自古受命之君及守文之王,非獨異姓之輔,亦有骨肉之助焉。其茂親則有魯衛、梁楚,其疏屬則有凡蔣、荊燕,咸能飛聲騰實,不滅於百代之後。至若豳孝公之勳烈,加之以善政,蔡文公之純孝,飾之以儉約,峨峨焉足以轔轢於前載矣。



 有周受命之始,宇文護實預艱難。及文后崩殂,諸子沖幼,群公懷等夷之士,天下有去就之心,卒能變魏為周,捍危獲義者,護之力也。向使加之以禮讓,經之以忠貞,桐宮有悔過之期,未央終天年之數,
 同前史所載,焉足道哉?然護寡於學術,暱近群小,威福在已,征伐自出,有人臣無君之心,為人主不堪之事,終於妻子為戮,身首橫分,蓋其宜也。當隋氏之起,假天威而服海內,胄以葭莩之親,據一州而協義舉,可謂忠而能勇。功業不遂,悲夫!亮實庸才,圖非常於巨逆,古人稱不度德、不量力者,其斯之謂歟。宇文測兄弟驅馳於經綸之日,孝伯、神舉盡言於父子之間,觀其智勇忠概,並可追從於古人矣。



\end{pinyinscope}