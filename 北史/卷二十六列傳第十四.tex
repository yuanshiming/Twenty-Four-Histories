\article{卷二十六列傳第十四}

\begin{pinyinscope}

 宋隱從子愔愔孫弁弁孫欽道弁族弟翻弟子世良世軌翻弟世景許彥五世孫惇刁雍子遵曾孫沖、柔、辛紹先韋閬孫子粲杜銓宋隱,字處默,西河介休人也。曾祖奭,祖活,父恭。世仕慕容氏,位並通顯。



 慕容俊徙鄴,恭始家於廣平列人焉。隱性至孝,專精好學。仕慕容垂,位本州別駕。



 道武平中山,
 拜隱尚書吏部郎,積遷行臺右丞,領選。以老病乞骸骨,不許。尋以母喪歸列人,既葬被徵,乃棄妻子匿於長樂,數年而卒。臨終,謂其子經曰:「汝等茍能入順父兄,出悌鄉黨,仕郡幸而至功曹史,以忠清奉之,足矣。不勞遠詣臺閣,恐汝不能富貴,徒延門戶累耳。若忘吾言,是死若父也。使鬼有知,吾不歸食矣。」



 隱弟宣,字道茂,與范陽盧玄、勃海高允、博陵崔建、從子愔俱被徵,拜中書博士。後拜侍郎、行司徒校尉。卒,謚曰簡侯。宣子謨,字乾仁,襲爵,卒於遼西太守。子鸞襲爵,位東莞太守。



 鸞弟瓊,字普賢,以孝稱。母曾病,季秋月思瓜。瓊夢想見之,求而遂獲,時
 人異之。卒於家。



 愔歷中書博士、員外散騎常侍,使江南。爵列人子。卒於廣平太守。長子顯襲爵。顯無子,養弟子弁為後。



 弁字義和。父叔珍,娶趙郡李敷妹,因敷事而死。弁至京師,見尚書李沖,因言論移日。沖異之,退曰:「此人一日千里,王佐才也。」顯卒,弁襲爵。弁與李彪州里,迭相祗好。彪為秘書丞,請為著作佐郎。遷尚書殿中郎中。孝文曾因朝會次,歷訪政道。弁年少官微,自下而對,聲姿清亮,進止可觀。帝稱善者久之。因是大被知遇,賜名為弁,意取弁和獻玉,楚王不知寶之也。遷中書侍郎兼員外散騎
 常侍,使齊。齊司徒蕭子良、秘書丞王融等皆稱美之,以為志氣謇諤不逮李彪,而體韻和雅,舉止閑邃過之。轉散騎侍郎。時散騎位在中書之右。孝文曾論江左事,問弁在南興亡之數。弁以為蕭氏父子無大功於天下,既以逆取,不能順守,必不能貽厥孫謀,保有南海。若物憚其威,身免為幸。後車駕南征,以弁為司徒司馬、東道副將。軍人有盜馬靽者,斬而徇,於是三軍震懼,莫敢犯法。



 黃門郎崔光薦弁自代,帝不許,亦賞光知人。未幾,以弁兼司徒左長史。時大選內外群官,并定四海士族,弁專參銓量之任,事多稱旨。然好言人之陰短。高門大族意
 所不便者,弁因毀之,至於舊族淪滯而人非可忌者,又申達之。弁又為本州大中正,姓族多所降抑,頗為時人所怨。遷散騎常侍,尋遷右衛將軍、領黃門。弁屢自陳讓,帝曰:「吾為相知者,卿亦不可有辭。豈得專守一官,不助朕為政!且常侍者,黃門之粗冗;領軍者,三衛之假攝,不足空存推讓,以棄大委。」其被知遇如此。孝文北都之選,李沖多所參預,頗抑宋氏。弁恨沖而與李彪交結,雅相知重。及彪之抗沖,沖謂彪曰:「爾如狗耳!為人所嗾。」及沖劾彪,不至大罪,弁之力也。彪除名,弁大相嗟慨,密圖申復。



 孝文在汝南不豫,大漸,旬餘日不見侍臣,左右唯彭
 城王勰等數人而已。小瘳,乃引見門下及宗室長幼諸人。入者未能皆致悲泣,惟弁與司徒司馬張海歔欷流涕,由是益重之。車駕征馬圈,留弁以本官兼祠部尚書,攝七兵事。及行,執其手曰:「國之大事,在祀與戎,故令卿綰攝二曹。」弁頓首辭謝。弁劬勞王事,恩遇亞於李沖。帝每稱弁可為吏部尚書,及崩,遺詔以弁為之。與咸陽王禧等六人輔政,而弁先卒。年三十八。贈瀛州刺史,謚曰貞順。



 弁性好矜伐,自許膏腴。孝文以郭祚晉魏名門,從容謂弁曰:「卿固當推郭祚之門。」弁笑曰:「臣家未肯推祚。」帝曰:「卿自漢、魏以來,既無高官,又無俊秀,何得不推?」弁
 曰:「臣清素自立,要爾不推。」侍臣出後,帝謂彭城王勰曰:「弁人身自不惡,乃復欲以門戶自矜,殊為可怪。」



 長子維,字伯緒,襲父爵。為給事中。坐諂事高肇,出為益州龍驤府長史,辭疾不行。太尉、清河王懌輔政,以維名臣子,薦為通直郎,辟其弟紀行參軍。靈太后臨政,委任元叉,恃寵驕盈,懌每以公理裁斷。叉甚忿恨,思害懌,遂與維作計,以富貴許之。維見叉寵勢日隆,乃告司染都尉韓文殊父子謀逆立懌。懌被錄禁中。



 文殊父子懼而逃遁。鞫無反狀,以文殊亡走,懸處大辟。置懌於宮西別館,禁兵守之。維應反坐,叉言於太后,欲開將來告者之路,乃黜
 為燕州昌平郡守,紀為秦州大羌令。



 維及紀頗涉經史,而浮薄無行;懌尊親懿望,朝野瞻屬。維受懌眷賞而無狀構間,天下士人莫不怪忿而賤薄之。及叉殺懌,專斷朝政,以維兄弟前者告懌,征維為散騎侍郎,紀為太學博士、領侍御史。叉甚暱之。維超遷通直常侍,又除洛州刺史。紀超遷尚書郎。紀字仲烈。初,弁謂族弟世景,言「維疏險而紀識慧不足,終必敗吾業」。世景以為不爾。至是果然。聞者以為知子莫若父。尚書令李崇、左僕射郭祚、右僕射游肇每云:「伯緒凶疏,終敗宋氏,幸得殺身耳。」論者以為有征。



 後除營州刺史。靈太后反政,以叉黨除名,
 遂還鄉里。尋追其前誣告清河王事,於鄴賜死。



 子春卿早亡,弟紀以次子欽仁嗣。欽仁,武定末為太尉祭酒。紀,明帝末為北道行臺,卒晉陽。子欽道。



 欽道仕齊,歷位中山太守。長於撫接,然好察細事。其州府佐吏使人間者,先酬錢然後敢食。臨蒞處稱為嚴整。尋徵為黃門侍郎,又令在東宮教太子吏事。時鄭子默以文學見知,亦被親寵。欽道本文法吏,不甚諳識古今,凡有疑事,必詢子默。



 二人幸於兩宮,雖諸王貴臣莫敢不敬憚。欽道又遷秘書監,仍帶黃門侍郎。乾明初,遷侍中,與楊愔同誅。贈吏部尚書、趙州刺史。



 弁族弟穎,字文
 賢,位魏郡太守。納貨劉騰,騰言之,以為涼州刺史。穎前妻劉氏亡後十五年,穎夢見之。拜曰:「新婦今被處分為高崇妻,故來辭君。」泫然涕流。穎且見崇,言之。崇後數日而卒。



 穎族弟鴻貴,為定州平北府參軍。送戍兵於荊州,坐取兵絹四百匹,兵欲告之,乃斬兵十人。又疏凡不達見令,律有梟首罪,乃生斷兵手,以水澆之,然後斬決。



 尋坐伏法。時人哀兵之苦,笑鴻貴之愚。



 弁族弟翻。翻字飛烏,少有操行,世人以剛斷許之。孝莊時,除司徒左長史、河南尹。初,翻為河陰令,順陽公主家奴為劫,攝而不送。翻將兵圍主宅,執主婿馮穆,步驅向縣。時正炎
 暑,立之日中,流汗霑地。縣舊有大枷,時人號曰彌尾青。



 及翻為縣,主吏請焚之。翻曰:「置南墻下,以待豪右。」未幾,有內監楊小駒詣縣請事,辭色不遜,翻命取尾青以鎖之。小駒既免,入訴於宣武。宣武大怒,敕河南尹推之,翻具自陳狀。詔曰:「卿故違朝法,豈不欲作威以買名?」翻對曰:「造者非臣,買名者亦宜非臣。所以留者,非敢施於百姓,欲待凶暴之徒如駒者耳。」



 於是威振京師。



 及為洛陽,迄於河南尹,畏憚權勢,更相承接,故當世之名大致減捐。卒官,贈侍中、衛將軍、相州刺史。孝武初,重贈驃騎大將軍、儀同三司、尚書左僕射、雍州刺史,謚曰貞烈。翻弟
 毓:字道和,敦篤有志行。卒於太中大夫。子世良。



 世良字元友。年十五,便有膽氣。後隨伯父翻在南兗州,屢有戰功。行臺、臨淮王彧與語,奇之。魏朝以爾朱榮有不臣跡,帝將圖之,密令彧將兵赴洛。彧在梁郡,稱疾,假世良都督,令還南兗發兵以聽期。世良請簡見兵三千騎,五日必到洛陽,並陳三策,彧皆不能從。



 尋為殿中侍御史,詣河北括戶,大獲浮惰。還見汲郡城旁多骸骨,移書州郡,悉令收瘞。其夜甘雨滂沱。河內太守田估贓貨百萬,世良檢按之,未竟,遇赦而還。



 孝莊勞之曰:「知卿所括得丁,倍於本帳。若官人皆如此用心,便是更出一天
 下也。」



 其後遷殿中。世良奏殿中主齊會之事,請改付餘曹。帝曰:「卿意不欲親庖廚邪?



 宜付右兵,以為永式。」河州刺史梁景睿,枹罕羌首,恃遠不敬,其賀正使人,頻年稱疾。秦州刺史侯莫陳悅受其贈遺,常為送表。世良並奏科其罪。帝嘉之,謂長孫永業曰:「宋郎中實有家風,甚可重也。」後拜清河太守。世良才識閑明,尤善政術。在郡未幾,聲問甚高。陽平郡移掩劫盜三十餘人,世良訊其情狀,唯送十二人,餘皆放之。陽平太守魏明朗大怒云:「輒放吾賊!」及推問,送者皆實,放者皆非。明朗大服。郡東南有曲堤,成公一姓阻而居之,群盜多萃於此。人為之語
 曰:「寧度東吳會稽,不歷成公曲堤。」世良施八條之制,盜奔他境。人又謠曰:「曲堤雖險賊何益,但有宋公自屏跡。」齊天保初,大赦,郡無一囚,率群吏拜詔而已。



 獄內魯生,桃樹蓬蒿亦滿。每日牙門虛寂,無復訴訟者,謂之神門。其冬,醴泉出於界內。及代至,傾城祖道。有老人丁金剛者,泣而前謝曰:「老人年九十,記三十五政。府君非唯善政,清亦徹底。今失賢者,人何以濟?」莫不攀轅涕泣。後卒於東郡太守,贈信州刺史。世良強學,好屬文,撰《字略》五篇、《宋氏別錄》十卷。



 子伯宗,位侍御史。性清退好學,多所撰述。至齊亡,不徙職,遂不入仕。隋大業初,卒於家。世良
 弟世軌。



 世軌幼自脩整,好法律。天保初,歷三尚書三公、二千石、都官郎中,兼並州長史。執獄寬平,多所全濟。為都官郎中,有囚事枉,將送,垂致法。世軌遣騎追止之,切奏其狀,遂免。



 稍遷廷尉少卿。洛州人聚結欲劫河橋,吏捕案之,連諸元徒黨千七百人。崔昂為廷尉,以為反,數年不斷。及世軌為少卿,判其事為劫,唯殺魁首,餘從坐悉舍焉。大理正蘇珍之以平幹知名,寺中語曰:「決定嫌疑蘇珍之,視表見裏宋世軌。」



 時人以為寺中二絕。南臺囚到廷尉,世軌多雪之,仍移攝御史,將問其濫狀。中尉畢義雲
 不送,移往復不止。世軌遂上書極言義雲酷擅。文宣引見二人,親敕世軌曰:「我知臺欺寺久,卿能執理抗衡,但守此心,勿慮不富貴。」敕義雲曰:「卿比所為誠合死,以志在疾惡,故且一恕。」仍顧謂朝臣曰:「此二人並我骨鯁臣也。」



 及卒,廷尉、御史諸繫囚皆哭曰:「宋廷尉死,我等豈有生路!」贈光州刺史,謚曰平。無子,世良以第五子朝基嗣。



 翻弟世景。世景少自脩立,事親以孝聞。與弟道璵下帷讀誦,博覽群言,尤精經義。族兄弁甚重之。舉秀才上第。再遷彭城王勰開府法曹行參軍。勰愛其才學,雅相器敬。孝文甚
 嘉異之。遷司徒法曹行參軍。世景明刑理,著律令,裁決疑獄,剖判如流。轉尚書祠部郎。彭城王勰每稱曰:「宋世景精微,尚書僕射才也。」臺中疑事,右僕射游肇常以委之。世景既才長從政,加之夙勤不怠,兼領數曹,深著稱績。左僕射源懷引為行臺郎。巡察州鎮,十有餘所,黜陟賞罰,莫不咸允。遷七鎮,別置諸戍,明設亭候,以備不虞。懷大相委重,還,薦之宣武,以為不減李沖。



 帝曰:「朕亦聞之。」後為伏波將軍,行榮陽太守,鄭氏豪橫,號為難制。濟州刺史鄭尚弟遠慶,先為苑陵令,多所受納,百姓患之。而世景下車,召而誡之。遠慶行意自若,世景繩之以法。
 遠慶懼,棄官亡走。於是屬縣畏威,莫不改肅。終日坐於事,未嘗寢息。人間之事,巨細必知。發姦擿伏,有若神明。嘗有一吏,休滿還郡,食人雞豚。又有一幹,受人一帽,又食二雞。世景叱而告之,吏、乾叩頭伏罪。於是上下震悚,莫敢犯禁。坐弟道璵事除名。



 世景友于之性,過絕於人,及道璵死,哭之,酸感行路。歲餘,母喪,遂不勝哀而卒。世景曾撰《晉書》,竟未得就。



 遺腹子季儒,位太學博士。曾至譙、宋間,為文弔嵇康,甚有理致。後夜寢室壞,壓而殞,時人悼傷惜之。



 道璵少而敏俊,自太學博士轉京兆王愉法曹行參軍。坐愉反得罪。作詩及挽歌詞寄之朋親,
 以見冤痛。道璵又曾贈著作郎張始均詩,其末章云:「子深懷璧憂,餘有當門病。」道璵既不免難,始均亦遇世禍,時咸怪之。



 道璵從孫孝王,學涉,亦好緝綴文藻。形貌矬陋而好臧否人物,時論甚疾之。



 為北平王文學。求入文林館不遂,因非毀朝士,撰《朝士別錄》二十卷。會周武滅齊,改為《關東風俗傳》,更廣聞見,勒成三十卷以上之。言多妄謬,篇第冗雜,無著述體。周大象末。預尉迥事,誅死。



 許彥,字道謨,高陽新城人也。祖茂,仕慕容氏高陽太守。彥少孤貧,好讀書,從沙門法睿受《易》。太武徵令卜筮,頻驗,遂在左右,參與謀議。彥質厚慎密,與人言,不及內事,
 帝以此益親待之。賜爵武昌公,拜相州刺史。在州受納,多違法度,詔書切讓之,然以彥腹心近臣,弗之罪也。卒,謚宣公。子熙襲。熙卒,子安仁襲。安仁卒,子元康襲,降爵為侯。



 熙弟宗之,歷位殿中尚書、定州刺史,封潁川公。受敕討丁零。既平,宗之因循郡縣,求取不節。深澤人馬超毀謗宗之,宗之怒,毆殺超。超家人告狀,宗之上超謗訕朝政。文成聞之曰:「此必宗之懼罪誣超。」案驗果然,遂斬於都市。



 元康弟護,州主簿。子恂,字伯禮,頗有業尚,閨門雍睦,三世同居,吏部尚書李神俊常稱其家風。位司徒諮議參軍。脩起居注,拜太中大夫。卒,贈吏部尚書、冀州
 刺史。恂弟惇。



 惇字季良。清識敏速,達於從政。位司徒主簿,以明斷見知,時人號為「入鐵主簿」。稍遷陽平太守。時遷都於鄴,陽平為畿郡,軍國責辦,賦斂無準。又勳貴屬請,朝夕徵求。惇並御之以道,咸以無怨,政為天下第一。特加賞異,圖形於闕,詔頒天下。歷魏尹、齊梁二州刺史,政並有治聲。遷大司農。會王思政入據潁城,王師出討,惇常督軍,無乏絕。引洧水灌城,惇之策也。遷殿中尚書。惇美鬚,下垂至帶,省中號「長鬣公」。齊文宣嘗因酒酣,提惇鬚稱美;以刀截之,唯留一握。



 惇懼,因不復敢長,人又號「齊須公」。歷
 御史中丞、膠州刺史、司農大理二卿。



 再為度支尚書、太子少保、少師、光祿大夫、開府儀同三司、尚書右僕射、特進,賜爵萬年縣子,食邑下邳郡幹。惇年老,致仕於家。三年,卒。



 惇少純直,晚更浮動。齊朝體式,本州大中正以京官為之。乾明中,邢邵為中書監,德望甚高。惇與邵競中正。遂憑附宋欽道,出邵為刺史,朝議甚鄙薄之。雖久處朝行,歷官清顯,與邢邵、魏收、陽休之、崔勵、徐之才比肩同列,諸人或談說經史,或吟詠詩賦,更相嘲戲,欣笑滿堂,惇不好劇談,又無學術,或坐杜口,或隱几而睡,不為勝流所重。子文紀,武平末,度支郎中。



 文紀弟文經,勤學
 方雅,身無擇行,口無戲言。武平末,殿中侍御史。隋開皇初,侍御史、兼通直散騎常侍、聘陳使副、主爵侍郎。卒於相州長史。



 惇兄遜,字仲讓,有幹局。乾明中,平原太守。卒,贈信州刺史。遜子文高,司徒掾。



 刁雍,字淑和,勃海饒安人也。曾祖協,從晉元帝度江,居京口,位尚書令。



 父暢,晉右衛將軍。初,晉相劉裕微時,負社錢一萬,違時不還。暢兄逵執而徵焉。



 及誅桓玄,以嫌,先誅刁氏。雍與暢故吏遂奔姚興,為太子中庶子。



 及姚泓滅,與司馬休之等歸魏,請於南境自效。明元假雍建威將軍。雍遂於河、濟間招集流散,傳檄邊境。雍弟彌,時
 亦率眾入京口,親共討裕。裕頻遣兵破之。



 明元南幸鄴,雍朝於行宮。明元問曰:「縛劉裕者,於卿親疏?」雍曰:「伯父。」



 帝笑曰:「劉裕父子當應憚卿。」於是假雍鎮東將軍、青州刺史、東光侯,使別立義軍。又詔雍令隨機立效。雍於是招集譙、梁、彭、沛人五千餘家,置二十七營,遷鎮濟陰。遷徐州刺史,賜爵東安伯。後除薄骨律鎮將。雍以西土乏雨,表求鑿渠,溉公私田。又奉詔以高平、安定、統萬及薄骨律等四鎮,出車牛五千乘運屯穀五十萬斛付沃野,以供軍糧。道多深沙,車牛艱阻,求於牽屯山河水之次造船水運。又以所綰邊表,常懼不虞,造城儲穀,置兵備
 守。詔皆從之。詔即名此城為刁公城,以旌功焉。皇興中,雍與隴西王源賀及中書監高允等並以耆年特見優禮,錫雍几杖,劍履上殿,月致珍羞焉。



 雍性寬柔,好尚文典,手不釋書。明敏多智,凡所為詩、賦、論、頌並諸雜文百有餘篇。又汎施愛士,恬靜寡欲。篤信佛道,著《殺誡》二十餘篇以訓子孫。太和八年,卒,年九十五,謚曰簡。子遵。



 遵字奉國,襲爵。遵少不拘小節,長更脩改。太和中,例降為侯。嘗經篤疾,幾死,見有神明救之,言福門子當享長年。後卒於洛州刺史,謚曰惠侯。



 子楷,早卒。楷子沖。



 沖字文朗。十三而孤,孝慕過人。其祖母司空高允女,聰
 明婦人也。哀其早孤,撫養尤篤。沖免喪後,便志學他方,高氏泣涕留之,沖終不止。雖家世貴達,及從師於外,自同諸生。于時學制,諸生悉日直監廚。沖雖有僕隸,不令代己,身自炊爨。每師受之際,發志精專,不捨晝夜,殆忘寒暑。學通諸經,偏脩鄭說。陰陽、圖緯、算數、天文、風氣之書莫不關綜,當世服其精博。刺史郭祚聞其盛名,訪以疑義,沖應機解辯,無不祛其久惑。後太守范陽盧尚之、刺史河東裴桓並徵沖為功曹主簿。非所好也,受署而已,不關事務,唯以講學為心。四方學徒就其受業者,歲有數百。沖雖儒生,而執心壯烈,不畏強禦。延昌中,帝舅
 司徒高肇擅恣威權,沖乃抗表極言其事。辭旨懇直,文義忠憤,太傅、清河王懌覽而歎息。



 先是,沖曾祖雍作《行孝論》以誡子孫,稱古之葬者,衣之以薪,不封不樹。



 後世聖人,易之以棺槨。至秦以後,生則不能致養,死則厚葬過度。及於末世,至蘧蒢裹尸,惈而葬者。確而為論,並非折衷。既知二者之失,豈宜同之?當令所存者,棺厚不過三寸,高不過三尺。弗用繒採,斂以時服。轜車止用白布為幔,不加畫飾,名為清素車。又去挽歌、方相并明器雜物。及沖祖遵將卒,敕其子孫,令奉雍遺旨。河南尹丞張普惠謂為太儉,貽書於沖叔整。令與通學議之。沖乃致
 書國學諸儒,以論其事,學官竟不能答。



 神龜末,沖以嫡傳祖爵東安侯。京兆王繼為司空也,並以高選頻辟記室參軍。



 明帝將親釋奠,於是國子助教韓神固與諸儒詣國子祭酒崔光、吏部尚書甄琛,舉其才學,奏而徵焉。及卒,國子博士高涼及范陽盧道侃、盧景裕等復上狀陳沖業行,議奏謚曰安憲先生,祭以太牢。子欽,字志儒,早亡。



 楷弟整,字景智。少有大度,頗涉書史。太和十五年,為奉朝請。孝文都洛,親自臨選,除司空法曹參軍。累遷黃門郎。普泰初,假征東大將軍、滄冀瀛三州刺史、大都督。尋加車騎將軍、右光祿大夫。遂逢本鄉賊亂,奉母客
 於齊州。既而母卒。母即高允之女。崔光、崔亮皆經允接待,是以涼燠之際,光等每致拜焉。天平四年,卒於鄴,贈司空公,謚曰文獻。整解音律,輕財好施,交結名勝,聲酒自娛。



 然貪而好色,為議者所貶。子柔。



 柔字子溫。少好學,留心儀禮,性強記,至於氏族內外,皆所諳悉。居母喪以孝聞。初為魏宣武挽郎,解巾司空行參軍。齊天保初,累遷國子博士。中書令魏收撰魏史,啟柔等同其事。柔性專固,自是所聞,收常嫌憚。又參議律令。時議者以為五等爵邑承襲,無嫡子,立嫡孫;無嫡孫,立嫡子弟;園嫡子弟,立嫡孫弟。柔以為無嫡孫,應立嫡
 曾孫,不應立嫡子弟。議曰:案《禮》,立嫡以長,故謂長子為嫡子。嫡子死,以嫡子之子為嫡孫,死則曾、玄亦然。然則嫡子之名本為傳重。故《喪服》曰:「庶子不為長子三年,不繼祖與禰也。」《禮》:「公儀仲子之喪,檀弓曰:『我未之前聞也。』『仲子舍其孫而立其子,何也?』子服伯子曰:『仲子亦猶行古之道也。昔者文王舍伯邑考而立武王發,微子舍其孫腯而立其弟衍。』」鄭注曰:「仲子為親者諱耳,立子非也。文王之立武王,權也。微子嫡子死,立弟衍,殷禮也。」「子游問諸孔子,孔子曰:『不,立孫。』」注商以嫡子死,立嫡子之母弟;周以嫡子死,立嫡子之子為嫡孫。



 故《春秋公羊》之義,
 嫡子有孫而死,質家親親先立弟,文家尊尊先立孫。《喪服》云:「為父後者,為出母無服。」《小記》云:「祖父卒而後為祖母後者,三年。」



 為母無服者,不祭故也。為祖母三年者,大宗傳重故也。今議以嫡孫死而立嫡子母弟。嫡子母弟者,則為父後矣。嫡子母弟本非承嫡,以無嫡,故得為父後,則嫡孫之弟,理亦應得為父後,則是父卒然後為祖後者服斬。既得為祖服斬,而不得為傳重,未之聞也。若用商家親親之義,本不應舍嫡子而立嫡孫。若從周家尊尊之文,豈宜舍其孫而立其弟?或文或質,愚用或焉。《小記》云:「嫡婦為舅姑後者,則舅姑為之小功。」注云:「謂
 夫有廢疾、他故,若死無子,不受重者。小功,庶婦之服。凡父母於子,舅姑於婦,將不傳重於嫡,及將所傳重者非嫡,服之皆如眾子庶婦也。」言死無子者,謂絕世。無子,非謂無嫡子。如其子,焉得云無後?夫雖廢疾無子,婦猶以嫡為名。嫡名既在,而欲廢其子者如禮何?禮何有損益,革代相沿。必謂宗嫡可得而變者,則為後服斬亦宜有因而改。



 七年,卒。柔在史館未久,勒成之際,志在偏黨。《魏書》中與其內外通親者,並虛美過實,為時論所譏。



 整弟宣,字季達。以功封高城縣侯,歷位都官尚書、衛大將軍、滄州刺史。卒,贈太尉公,謚曰武。



 刁氏世有榮貴,而門風不甚
 脩潔,為時所鄙。



 雍族孫雙,字子山。高祖藪,晉齊郡太守。藪因晉亂,居青州之安樂。至雙始歸本鄉。雙少好學,兼涉文史,雅為中山王英所知賞。位西河太守。為政清簡,吏人安悅。及中山王熙起兵誅元叉,事敗,熙弟略投命於雙。雙藏護周年。時購略甚切,略懼,求送出境。雙曰:「會有一死,所難過耳。今遭知己,視死如歸,願不以為慮。」略復苦求南轉,雙乃遣從子昌送達江左。靈太后反政,知略因雙獲濟,徵拜光祿大夫。時略姊饒安主,刁宣妻也,頻訴靈太后,乞征略還。朝廷乃以徐州所獲俘江革、祖恆二人易之。以雙與略有舊,乃令至境迎接。明帝末,除
 西兗州刺史。時賊盜蜂起,州人張桃弓等招聚亡命,公行劫掠。雙至境,先遣使諭桃弓,陳示禍福,桃弓即隨使歸罪,雙捨而不問。後有盜發之處,令桃弓追捕,咸悉禽獲,於是州境清肅。孝莊初,行濟州刺史,以功封曲城鄉男。孝武初,遷驍騎大將軍、左光祿大夫。興和三年,卒,贈車騎大將軍、儀同三司、齊州刺史,謚曰清穆。



 辛紹先,隴西狄道人也。五世祖怡,晉幽州刺史。父深,仕西涼為驍騎將軍。



 及涼後主歆與沮渠蒙遜戰於蓼泉,軍敗,失馬,深以所乘授歆,而身死於難。以義烈見稱西土。涼州平,紹先內徙,家於晉陽。明敏有識量,與廣平游
 明根、范陽盧度世、同郡李承昭等甚相友。有至性,丁父憂,三年口不甘味,頭不櫛沐,髮遂落盡,故常著垂裙皁帽。自中書博士轉神部令。



 皇興中,薛安都以彭城歸魏。時朝廷欲綏安初附,以紹先為下邳太守。為政不甚皦察,舉其不綱而已,唯教人為產禦賊之備。及宋將陳顯達、蕭道成、蕭順之來寇,道成謂順之曰:「辛紹先未易侵也,宜共慎之。」於是不歷郡境,徑屯呂梁。



 卒於郡,贈並州刺史、晉陽侯,謚曰惠。



 子鳳達,耽道樂古,有長者之名。卒於京兆王子推國常侍。



 鳳達子祥,字萬福。舉司州秀才,再遷司空主簿。咸陽王禧妃,即祥妻之妹也。



 及禧構逆,
 親知多罹塵謗,詳獨蕭然不預。轉并州平北府司馬。有白壁還兵藥道顯,被誣為賊,官屬咸疑之。詳曰:「道顯面有悲色。察獄以色,其此之謂乎!」苦執申之。月餘,別獲真賊。後除郢州龍驤府長史,帶義陽太守。白早生之反也,梁遣來援,因此緣淮鎮戍,相繼降沒。唯祥堅城固守。梁又遣將胡武城、陶平虜,於州南金山之上,連營侵逼。祥出其不意,襲之。賊大崩,禽平虜,斬武城,以送京師。



 州境獲全。論功方有賞授,而刺史婁悅恥勳出其下,間之執政,事竟不行。胡賊劉龍駒作逆華州,除祥安定王燮征虜府長史,仍為別將,與討胡使薛和滅之。卒,贈南青州
 刺史。



 祥弟少雍,字季和,少聰穎,有孝行,尤為祖父紹先所愛。紹先性嗜羊肝,常呼少雍共食。及紹先卒,少雍終身不食肝。性仁厚,有禮義,門內之法,為時所重。



 稍遷司空、高陽王雍田曹參軍。少雍清正,不憚強禦,積年久訟,造次決之。請託路絕,時稱賢明。正始中,詔百官各舉所知,高陽王雍及吏部郎中李憲俱以少雍為舉首。卒於給事中。



 少雍妻王氏,有德義。少雍與從弟懷仁兄弟同居。懷仁等事之甚謹,閨門禮讓,人無間焉。士大夫以此稱美。子元桓,武定中,儀同府司馬。元桓弟遜士,太師開府功曹參軍。



 鳳達弟穆,字叔宗,舉茂才,東雍州別駕。初
 隨父在下邳,與彭城陳敬文友善。



 敬文弟敬武,少為沙門,從師遠學,經久不返。敬文病臨卒,以雜綾二十匹託穆與敬武。穆久不得見,經二十年,始於洛陽見敬武,以物還之,封題如故。世稱廉信。



 歷東荊州司馬,轉長史,帶義陽太守,領戍。雅有恤人之志。再轉汝陽太守。遇水澇人飢,上表請輕租賦。帝從之,遂敕汝陽一郡,聽以小絹為調。除平原相。徵為征虜將軍、太中大夫,未發,卒於郡。贈後將軍、幽州刺史。



 子子馥,字元穎,早有學行,累遷平原相。父子並為此郡,吏人懷安之。元顥入洛,子馥不從。莊帝反政,封三門縣男。天平中,除太尉府司馬。白山連
 接三齊,瑕丘數州之界,多有賊盜。子馥受使檢覆,因辨山谷要害宜立鎮戍之所。又諸州豪右。在山鼓鑄,姦黨多依之,又得密造兵仗。上表請破罷諸冶。朝廷善而從之。後卒於清河太守。子馥以《三傳》經同說異,遂總為一部,傳注並出,校比短長。會亡,未就。



 韋閬,字友觀,京兆杜陵人也。世為三輔冠族。祖楷,晉長樂清河二郡太守。



 父逵,慕容垂大長秋卿。閬少有器望,遇慕容氏政亂,避地薊城。太武初,徵拜咸陽太守,轉武都太守。卒郡。



 子範,試守華山郡,賜爵高平男。卒。



 範子俊,字穎超,早有學。少孤,事祖母以孝聞。性溫和廉讓,為州
 里所稱。



 太和中,襲爵。歷位都水使者。宣武崩,領軍於忠矯擅威刑,俊與左僕射郭祚昏嫁,故亦同時遇害。臨終,訴枉於尚書元欽,欽知而不敢申理。俊歎曰:「吾一生為善,未蒙善報;常不為惡,今為惡終,悠悠蒼天,抱直無訴!」時人咸怨傷焉。熙平元年,追贈洛州刺史,謚曰貞。子子粲。



 子粲字暉茂。齊王蕭寶夤為雍州刺史,引為府主簿,轉錄事參軍。及寶夤反,子粲與弟子爽執志不從,相率逃免。雍州平,賜爵長安子。普泰中,累遷中書侍郎。



 孝武帝入關,子粲歷行臺左丞、南汾州刺史。少弟道諧為鎮城
 都督。元象中,齊神武命將出討,子粲及道諧俱被獲,送於晉陽。子粲累遷南兗州刺史。齊天保初,封西僰縣男。後卒於豫州刺史,謚曰忠。



 子粲兄弟十三人,並有孝行,居父喪,毀瘠過禮。既葬,廬於墓側,負土成墳。



 弟榮亮最知名。



 榮亮字子昱。博學有文才,德行仁孝,為時所重。歷諫議大夫、衛大將軍。卒,贈河州刺史。子綱,字世紀,有操行,才學見稱,領袖本州,謂為中正。開皇中,位趙州長史。有子文宗、文英,並知名。



 閬從叔道福,父羆,為苻堅丞相王猛所器重,以女妻焉。仕堅為東海太守。堅滅,奔江左,仕宋為秦州刺史。道福有志略,仕宋位盱眙、南沛二郡
 太守,領鎮北府錄事參軍。與徐州刺史薛安都謀擁州內附,賜爵高密侯,因家彭城。卒,贈兗州刺史,謚曰簡。



 子欣宗,以歸國勳,別賜爵杜縣侯。歷位太中大夫、行幽州事。卒,贈南兗州刺史,謚曰簡。



 閬從子崇,字洪基。父肅,字道壽,隨劉義真度江,位豫州刺史。崇年十歲,父卒,母鄭氏攜以入魏,因寓居河、洛。少為舅兗州刺史鄭羲所器賞。位司徒從事中郎。孝文納其女為充華嬪,除南潁川太守。不好發擿細事,恆云:「何用小察,以傷大道?」吏人感之,郡中大安。帝聞而嘉賞,賜帛二百匹。遷洛,以崇為司州中正。尋除咸陽王禧開府從事中郎,復為河南邑中
 正。崇頻居衡品,以平直見稱。



 出為鄉郡太守,更滿應代,吏人詣闕乞留,復延三年。後卒。



 子猷之,釋褐奉朝請,轉給事中、步兵校尉,稍遷前、後將軍,太中大夫,卒。



 猷之弟休之,貞和自守,未嘗言行忤物。歷位給事中、河南邑中正、安西將軍、光祿大夫。卒。子道建、道儒。



 閬族弟珍,字靈智,孝文賜名焉。父子尚,字文叔。位樂安王良安西府從事中郎。卒,贈雍州刺史。



 珍少有志操,歷位尚書南部郎。孝文初,蠻首桓誕歸款,朝廷思安邊之略,以誕為東荊州刺史,令珍為使,與誕招慰蠻左。珍至桐柏山,窮淮源,宣揚恩澤,莫不懷附。淮源舊有祠堂,蠻俗恆用人祭之。
 珍乃曉告曰:「天地明靈,即人之父母,豈有父母,甘子肉味?自今宜悉以酒脯代用。」群蠻從約,自此而改。凡所招降七萬餘戶,置郡縣而還。以奉使稱旨,賜爵霸城子。後以軍功,進爵為侯。累遷顯武將軍、郢州刺史。所在有聲績,朝廷嘉之,遷龍驤將軍,賜驊騮二匹,帛五十匹,穀三百斛。珍乃召集州內孤貧者,謂曰:「天子謂我能撫綏卿等,故賜以穀帛,吾何敢獨當。」遂以所賜,悉分與之。



 尋轉荊州刺史。與尚書盧陽烏徵赭陽,為齊將垣歷生、蔡道恭所敗,免歸鄉里。



 臨別,謂陽烏曰:「主上聖明,志吞吳會。用兵機要,在於上流。若有事荊楚,恐老夫復不得停耳。」
 後車駕征鄧沔,復起珍為中軍大將軍、彭城王勰長史。鄧沔既平,試守魯陽郡。孝文復南伐,路經珍郡,加中壘將軍,正太守。珍從至清水,帝曰:「朕頃戎車再駕,卿恆翼務中軍。今日之舉,亦欲引卿同行,但三鴉險要,非卿無以守也。」因敕還。及孝文崩於行宮,秘匿而還,至珍郡,始發大諱。還,除中散大夫,尋加鎮遠將軍、太尉諮議參軍。卒,贈本將軍、青州刺史,謚曰懿。



 長子纘,字遵彥。年十三,補中書學生。聰敏明辯,為博士李彪所稱。再遷侍御中散。孝文每與德學沙門談論往復,纘掌綴錄,無所遺漏,頗見知賞。累遷長兼尚書左丞。壽春內附,尚書令王肅
 出鎮揚州,請纘行,為州長史。加平遠將軍,帶梁郡太守。肅薨,敕纘行州事。任城王澄代肅為州,復啟纘為長史。澄出征之後,梁將姜慶真乘虛攻襲,遂據外郭。雖尋克復,纘坐免官。卒。



 纘弟彧,字遵慶,亦有學識。解褐奉朝請,稍遷平遠將軍、東豫州刺史。綏懷蠻左,頗得其心。蠻酋田益宗子魯生、魯賢先叛父南入,數為寇掠。自彧至州,魯生等咸箋啟脩敬,不得為害。彧以蠻俗不識禮儀,乃立太學,選諸郡生徒於州總教。



 又於城北置崇武館以習武焉。州境清肅。罷還,遇大將軍、京兆王繼西征,請為長史。尋以本官兼尚書,為豳、夏行臺,以功封陰盤縣男。
 卒,贈撫軍將軍、雍州刺史,謚曰文。子彪襲。孝莊末,為藍田太守,因仕關西。



 彪弟融,以軍功賜爵長安伯。稍遷大司馬開府司馬。融娶司農卿趙郡李瑾女,疑其妻與章武王景哲姦通,乃刺殺之。懼,亦自殺。



 弟朏,字遵顯,少有志業。年十八,辟州主簿。時屬歲儉,朏以家粟造粥,以飼飢人,所活甚眾。解褐太學博士。稍遷右軍將軍,為荊、郢和糴大使。南郢州刺史田夷啟稱朏父珍往任荊州,恩洽夷夏,乞朏充南道別將,領荊州驍勇,共為腹背。



 詔從之。未幾,行南荊州事。遷東徐州刺史。梁遣其郢州刺史田粗憘率眾來寇,朏於石羊岡破斬之,以功封杜縣子。
 卒於侍中、雍州刺史,謚曰宣。



 長子鴻,字道衍,頗有乾用,累遷中書舍人。天平三年,坐漏泄,賜死於家。



 杜銓,字士衡,京兆人,晉征南將軍預五世孫也。祖胄,苻堅太尉長史。父嶷,慕容垂秘書監,仍僑居趙郡。銓學涉,有長者風,與盧玄、高允等同被徵為中書博士。



 初,密太后父豹喪在濮陽,太武欲令迎葬於鄴,謂司徒崔浩曰:「天下諸杜,何處望高?朕今方改葬外祖,意欲取杜中長老一人,以為宗正,令營護兇事。」浩曰:「京兆為美。中書博士杜銓,其家今在趙郡,是杜預後,於今為諸杜最。」密召見,銓器貌瑰雅,太武感悅,謂浩曰:「此真吾所欲也。」以為
 宗正,令與杜超子道生送豹喪柩,致葬鄴南。銓遂與超如親。超謂銓曰:「既是宗近,何緣僑居趙郡?」乃延引同屬魏郡。再遷中書侍郎,賜爵新豐侯。卒,贈相州刺史、魏縣侯,謚曰宣。子振,字季元。舉秀才,卒於中書博士。



 振子遇,字慶期,位尚書起部郎。竊官材瓦起立私宅,清論鄙之。卒於河東太守,贈都官尚書、豫州刺史,謚曰惠。銓族孫景,字宣明,學通經史。州府交辟,不就。



 景子裕,字慶延,雖官非貴仕,而文學相傳。仕齊,位止樂陵令。齊亡,退居教授,終于家。



 子正玄,字知禮,少傳家業,耽志經史。隋開皇十五年,舉秀才,試策高第。



 曹司以策過左僕射楊素,怒
 曰:「周孔更生,尚不得為秀才,刺史何忽妄舉此人?



 可附下考。」乃以策抵地,不視。時海內唯正玄一人應秀才,餘常貢者,隨例銓注訖,正玄獨不得進止。曹司以選期將盡,重以啟素。素志在試退正玄,乃手題使擬司馬相如《上林賦》、王褒《聖主得賢臣頌》、班固《燕然山銘》、張載《劍閣銘》、《白鸚鵡賦》,曰:「我不能為君住宿,可至未進令就。」正玄及時並了。素讀數遍,大驚曰:「誠好秀才!」命曹司錄奏。屬吏部選期已過,注色令還。期年重集,素謂曹司曰:「秀才杜正玄至。」又試《官人有奇器》闕並立成,文不加點。素大嗟之,命吏部優敘。曹司以擬長寧王記室參軍。時素情
 背曹官,及見,曰:「小王不盡其才也。」晉王廣方鎮揚州,妙選府僚,乃以正玄為晉王府參軍。後豫章王鎮揚州,又為豫章王記室。卒。



 正玄弟正藏,字為善,亦好學,善屬文。開皇十六年,舉秀才。時蘇威監選,試擬賈誼《過秦論》及《尚書湯誓》、《匠人箴》、《連理樹賦》、《几賦》、《弓銘》,應時並就,又無點竄。時射策甲第者合奏,曹司難為別奏,抑為乙科。



 正藏訴屈,威怒,改為丙第,授純州行參軍。遷梁郡下邑縣正。大業中,與劉炫同以學業該通,應詔被舉。時正藏弟正儀貢充進士,正倫為秀才,兄弟三人同時應命,當世嗟美之。著作郎王劭奏追脩史,司穀大夫薛道衡奏擬
 從事,並以見任且放還。



 九年,從駕征遼,為夫餘道行軍長史。還至涿郡,卒。



 正藏為文迅速,有如宿構。曾令數人並執紙筆,各題一文,正藏口授俱成,皆有文理,為當時所異。又為《文軌》二十卷,論為文體則,甚有條貫。後生寶而行之,多資以解褐,大行於世,謂之《杜家新書》云。



 論曰:宋隱操行貞白,遺略榮名;宣、愔並保退素,咸見征辟,可謂德門者矣。



 義和以才度見知,迹參顧命,拔萃出類,當有以哉!無子之歎,豈徒羊舌!宗祀不亡,蓋其幸也。翻剛鯁自立,猛而斷務。世良昆季,雅有家風。道謨卜筮取達,季良累於學淺。刁雍才識恢遠,著聲立事,禮遇優
 隆,世有人爵,堂構之義也。辛、韋不殞門風。杜銓所在為重。正玄難兄難弟,信為美哉!



\end{pinyinscope}