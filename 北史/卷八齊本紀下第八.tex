\article{卷八齊本紀下第八}

\begin{pinyinscope}

 世祖武成皇帝諱湛,神武皇帝第九子,孝昭皇帝之母弟也。儀表瑰傑,神武尤所鐘愛。神武方招懷荒遠,乃為帝娉蠕蠕太子庵羅辰女,號鄰和公主。帝時年八歲,冠服端嚴,神情閑遠,華戎嘆異。元象中,封長廣郡公。天保初,進爵為王,拜尚書令,尋兼司徒,遷太尉。乾明初,楊愔等密相疏忌,以帝為大司馬,領並州刺史。



 帝既與孝昭謀,誅諸執政,遷太傅、錄尚書事、領京畿大都督。皇建初,進位右丞相。孝昭幸晉陽,帝以懿親居守鄴,政事咸見委托。二年,孝昭崩,遺詔徵帝入統大位。及晉陽宮,發喪於崇德殿。皇太后令所司宣遺詔,左丞相斛律金率百僚敦勸,三奏乃許之。



 大寧元年冬十一月癸丑,皇帝即位於南宮。大赦,改皇建二年為大寧。乙卯,以司徒、平秦王歸彥為太傅;以尚書右僕射、趙郡王睿為尚書令;以太尉尉粲為太保;以尚書令段韶為大司馬;以豐州刺史婁睿為司空;以太傅、平陽王淹為太宰;以太保、彭城王浟為太師、錄尚書事;以冀州刺史、博陵王濟為太尉;以中書監、任城王湝為尚書左僕射;以並州刺史斛律光為右僕射。封
 孝昭皇帝太子百年為樂陵郡王。庚申,詔大使巡行天下,求政善惡,問人疾苦,擢進賢良。是歲,周武帝保定元年。



 河清元年春正月乙亥,車駕至自晉陽。辛巳,祀南郊。壬午,享太廟。丙戌,立妃胡氏為皇后,子緯為皇太子。戊子,大赦,內外百官,普加泛級;諸為父後者,賜爵一級。己亥,以前定州刺史、馮翊王潤為尚書左僕
 射。詔普斷屠殺,以順春令。



 二月丁未,以太宰、平陽王淹為青州刺
 史、
 太傅、領司徒;以領軍大將軍、宗師、平秦王歸彥為太宰、冀州刺史。乙卯,以兼尚書令、任城王湝為司徒。
 詔散騎常侍崔瞻聘於陳。夏四月辛丑,皇太后婁氏崩。乙巳,青州刺史上言。今月庚寅,河、濟清。以河、濟清,改大寧二年為河清,降罪人各有差。五月甲申,祔葬武明皇后於義平陵。己丑,以尚書右僕射斛律光為尚書令。秋七月,太宰、冀州刺史、平秦王歸彥據州反。詔大司馬段韶、司空婁睿討禽之。乙未,斬歸彥,並其三子及黨與二十人於都市。丁酉,以大司馬段韶為太傅;以司空婁睿為司徒;以太傅、平陽王淹為太宰;以尚書令斛律光為司空;以太子太傅、趙郡王睿為尚書令;中書監、河間王孝
 琬為尚書左僕射。癸亥,行幸晉陽。陳人來聘。冬十一月丁丑,詔兼散騎常侍封孝琰使於陳。十二月丙辰,車駕至自晉陽。是歲,殺太原王紹德。



 二年春正月乙亥,帝詔臨朝堂,策試秀、孝。以太子少傅魏收為兼尚書右僕射。



 己卯,兼右僕射魏收以阿縱除名。丁丑,以武明
 皇后
 配祭北郊。辛卯,帝臨都亭錄見囚,降在京罪人各有差。三月己丑,詔司空斛律光督五營軍士築戍於軹關。壬申,室韋國遣使朝貢。丙戌,以兼尚書右僕射趙彥深為左僕射。夏四月,并、汾、晉、東雍、南汾五州蟲旱傷稼,遣使振恤。戊午,陳人來聘。五月壬午,詔以城南雙堂之苑,迴造大總持寺。六月乙巳,齊州上言,
 濟河水口見八龍升天。乙卯,詔兼散騎常侍崔子武使於陳。庚申,司州牧、河南王孝瑜薨。秋八月辛丑,詔以三臺宮為大興聖寺。冬十二月癸巳,陳人來聘。己酉,周將楊忠帥突厥阿史那木可汗等一十餘萬人,自恆州分為三道,殺掠吏人。是時,大雨雪連月,南北千餘里,平地數尺。



 霜晝下,雨血於太原。戊午,帝至晉陽。己未,周軍逼並州,又遣大將達奚武帥眾數萬至東雍及晉州,與突厥相應。是歲,室韋、庫莫奚、靺鞨、契丹並遣使朝貢。



 三年春正月庚申朔,周軍至城下而陳。戰於城西,周軍及突厥大敗,人畜死者相枕,數百里不絕。詔平原王段
 韶追出塞而還。三月辛酉,以律令班下,大赦。己巳,盜殺太師、彭城王浟。庚辰,以司空斛律光為司徒,以侍中、武興王普為尚書左僕射。甲申,以尚書令、馮翊王潤為司空。夏四月辛卯,詔兼散騎常侍皇甫亮使於陳。五月甲子,帝至自晉陽。壬午,以尚書令、趙郡王睿為錄尚書事,以前司徒婁睿為太尉。甲申,以太傅段韶為太師。丁亥,以太尉、任城王湝為大將軍。壬辰,行幸晉陽。六月庚子,大雨,晝夜不息,至甲辰乃止。是月,晉陽訛言有鬼兵,百姓競擊銅鐵以捍之。殺樂陵王百年。歸宇文媼于周。秋九月乙丑,封皇子綽為南陽王、儼為東平王。是月,歸閻
 媼于周。陳人來聘。突厥寇幽州,入長城,虜掠而還。



 閏月乙未,詔遣十二使巡行水澇州,免其租調。乙巳,突厥寇幽州。周軍三道並出,使其將尉迥寇洛陽,楊摽入軹關,權景宣趣懸瓠。冬十一月甲午,迥等圍洛陽。戊戌,詔兼散騎常侍劉逖使於陳。甲辰,太尉婁睿大破周軍於軹關,禽楊摽。十二月乙卯,豫州刺史王士良以城降周將權景宣。丁巳,帝自晉陽南討。己未,太宰、平陽王淹薨。壬戌,太師段韶大破尉迥等,解洛陽圍。丁卯,帝至洛陽,免洛州經周軍處一年租賦;赦州城內死罪已下囚。己巳,以太師段韶為太宰,以司徒斛律光為太尉,并州刺史、
 蘭陵王長恭為尚書令。壬申,帝至武牢,經滑臺,次於黎陽。所經減降罪人。丙子,車駕至自洛陽。是歲,高麗、靺鞨、新羅並遣使朝貢。山東大水,飢死者不可勝計。詔發振給,事竟不行。



 四年春正月癸卯,以大將軍、任城王湝為大司馬。辛未,幸晉陽。二月甲寅,詔以新羅國王金真興為使持節、東夷校尉、樂浪郡公、新羅王。壬申,以年穀不登,禁酤酒。己卯,詔減百官食廩各有差。三月戊子,詔給西兗、梁、滄、趙州,司州之東郡、陽平、清河、武都。冀州之長樂、勃海遭水澇之處貧下戶粟各有差。家別斗升而已,又多不付。是
 月,彗星見。有物隕於殿廷。如赤漆鼓,帶小鈴。殿上石自起,兩兩相對。又有神見於後園萬壽堂前山穴中,其體壯大,不辨其面,兩齒絕白,長出於脣。帝直宿嬪御已下七百人咸見焉。帝又夢之。夏四月戊午,大將軍、東安王婁睿坐事免。乙亥,陳人來聘。太史奏,天文有變,其占當有易王。丙子,乃使太宰段韶兼太尉,持節奉皇帝璽綬,傳位於皇太子。大赦,改元為天統元年。



 百官進級,降罪,各有差。又詔皇太子妃斛律氏為皇后。於是群公上尊號為太上皇帝。軍國大事,咸以奏聞。始將傳政,使內參乘子尚乘驛送詔書於鄴。子尚出晉陽城,見人騎隨後,
 忽失之。尚未至鄴而其言已布矣。天統四年十二月辛未,太上皇帝崩於鄴宮乾壽堂,時年三十二。謚曰武成皇帝,廟號世祖。五年二月甲申,葬於永平陵。



 後主諱緯,字仁綱,武成皇帝之長子也。母曰胡皇后,夢於海上坐玉盆,日入裙下,遂有娠。天保七年五月五日,生帝於并州邸。帝少美容儀,武成特所愛寵,拜世子。及武成入纂大業,大寧二年正月丙戌,立為皇太子。河清四年,武成禪位於帝。



 天統元年夏四月丙子,皇帝即位於晉陽宮。大赦,改河清四年為天統。丁丑,以太保賀拔仁為太師;太尉侯莫
 陳相為太保;司空、馮翊王潤為司徒;錄尚書事、趙郡王睿為司空;尚書左僕射、河間王孝琬為尚書令。戊寅,以瀛州刺史尉粲為太尉;斛律光為大將軍;東安王婁睿為太尉;尚書右僕射趙彥深為左僕射。六月壬戌,彗星出文昌東北,其大如手,後稍長,乃至丈餘,百日乃滅。己巳,太上皇帝詔兼散騎常侍王季高使於陳。秋七月乙未,太上皇帝詔增置都水使者一人。冬十一月癸未,太上皇帝至自晉陽。己丑,太上皇帝詔改太祖獻武皇帝為神武皇帝,廟號高祖;獻明皇后為武明皇后。其文宣謚號,委有司議定。十二月庚戌,太上皇帝狩於北郊。



 壬
 子,狩於南郊。乙卯,狩於西郊。壬戌,太上皇帝幸晉陽。丁卯,帝至自晉陽。



 庚午,有司奏改高祖文宣皇帝為威宗景烈皇帝。是歲,高麗、契丹、靺鞨並遣使朝貢。河南大疫。



 二年春正月辛卯,祀圓丘。癸巳,祫祭於太廟。詔降罪人各有差。丙申,以吏部尚書尉瑾為尚書右僕射。庚子,行幸晉陽。二月庚戌,太上皇帝至自晉陽。壬子,陳人來聘。三月乙巳,太上皇帝詔以三臺施興聖寺。以旱故,降禁囚。夏四月,陳文帝殂。五月乙酉,以兼尚書左僕射、武興王普為尚書令。己亥,封太上皇帝子儼為東平王,仁弘為齊安王,仁固為北平王,仁英為高平王,仁光為淮南王。
 六月,太上皇帝詔兼散騎常侍韋道儒聘於陳。秋八月,太上皇帝幸晉陽。冬十月乙卯,以太保侯莫陳相為太傅;大司馬、任城王湝為太保;太尉婁睿為大司馬,徙馮翊王潤為太尉,開府儀同三司韓祖念為司徒。十一月,大雨雪。盜竊太廟御服。十二月乙丑,陳人來聘。是歲,殺河間王孝琬。突厥、靺鞨國並遣使朝貢。於周為天和元年。



 三年春正月壬辰,太上皇帝至自晉陽。乙未,大雪,平地三尺。戊戌,太上皇帝詔,京官執事散官三品已上,舉三人,五品已上,各舉二人;稱事七品已上,及殿中侍御史、
 尚書都、檢校御史、主書及門下錄事,各舉一人。鄴宮九龍殿災,延燒西廊。二月壬寅朔,帝加元服,大赦。九州職人,各進四級;內外百官,普進二級。夏四月癸丑,太上皇帝詔兼散騎常侍司馬幼之使於陳。五月甲午,太上皇帝詔以領軍大將軍、東平王儼為尚書令。乙未,大風,晝晦,發屋拔樹。六月己未,太上皇帝詔封皇子仁機為西河王,仁約為樂浪王,仁儉為潁川王,仁雅為安樂王,統為丹楊王,仁謙為東海王。閏六月辛巳,左丞相斛律金薨。壬午,太上皇帝詔尚書令、東平王儼錄尚書事。以尚書左僕射趙彥深為尚書令,并省尚書右僕射婁定遠
 為尚書左僕射,中書監徐之才為右僕射。秋八月辛未,太上皇帝詔以太保、任城王湝為太師,太尉、馮翊王潤為大司馬,太宰段韶為左丞相,太師賀拔仁為右丞相,太傅侯莫陳相為太宰,大司馬婁睿為太傅,大將軍斛律光為太保,司徒韓祖念為大將軍,司空、趙郡王睿為太尉,尚書令、東平王儼為司徒。九月己酉,太上皇帝詔諸寺署所綰雜保戶姓高者,天保之初,雖有優放,權假力用未免者,今可悉蠲雜戶,任屬郡縣,一准平人。丁巳,太上皇帝幸晉陽。是秋,山東大水,人饑,僵尸滿道。



 冬十月,突厥、大莫婁、室韋、百濟、靺鞨等國,各遣使朝貢。十
 一月丙午,以晉陽大明殿成故,大赦。文武百官進二級。免並州居城、太原一郡來年租。癸未,太上皇帝至自晉陽。十二月己巳,太上皇帝詔以故左丞相、趙郡王琛配饗神武廟廷。



 四年春正月壬子,詔以故清河王岳、河東王潘相樂十人並配饗神武廟廷。癸亥,太上皇帝詔兼散騎常侍鄭大護使於陳。三月乙巳,太上皇帝詔以司徙、東平王儼為大將軍,南陽王綽為司徒,開府儀同三司、廣寧王孝珩為尚書令。夏四月辛未,鄴宮昭陽殿災,及宣光、瑤華等殿。辛巳,太上皇帝幸晉陽。五月癸卯,以尚書右僕射
 胡長仁為左僕射,中書監和士開為右僕射。壬戌,太上皇帝至自晉陽。自正月不雨,至於是月。六月甲子朔,大雨。甲申,大風,拔木折樹。是月,彗星見于東井。



 秋九月丙申,周人來通和。太上皇帝詔侍中斛斯文略報聘于周。冬十月辛巳,以尚書令、廣寧王孝珩為錄尚書事,左僕射胡長仁為尚書令,右僕射和士開為左僕射,中書監唐邕為右僕射。十一月壬辰,太上皇帝詔兼散騎常侍李翥使於陳。是月,陳安成王頊廢其主伯宗而自立。十二月辛未,太上皇帝崩。丙子,大赦。九州職人普加一級,內外百官並加兩級。戊寅,上太上皇后尊號為皇太后。
 甲申,詔細作之務及所在百工悉罷之。又詔掖廷、晉陽、中山宮人等,及鄴下、并州太官官口二處,其年六十已上,及有癃患者,仰所司簡放。庚寅,詔天保七年已來,諸家緣坐配流者,所在令還。是歲,契丹、靺鞨國並遣使朝貢。



 五年春正月辛亥,詔以金鳳等三臺未入寺者,施大興聖寺。是月,殺定州刺史、博陵王濟。二月乙丑,詔應宮刑者,普免刑為官口。又詔禁網捕鷹鷂及畜養籠放之物。癸酉,大莫婁國遣使朝貢。乙丑,改東平王儼為瑯邪王。詔侍中叱列長文使於周。是月,殺太尉、趙郡王睿。三月
 丁酉,以司空徐顯秀為太尉,并省尚書令婁定遠為司空。是月,行幸晉陽。夏四月甲子,詔以并州尚書省為大基聖寺,晉祠為大崇皇寺。乙丑,車駕至自晉陽。秋七月己丑,詔降罪人各有差。戊申,詔使巡省河北諸州無雨處,境內偏旱者,優免租調。冬十月壬戌,詔禁造酒。十一月辛丑,詔以太保斛律光為太傅,大司馬、馮翊王潤為太保,大將軍、瑯邪王儼為大司馬。十二月庚午,以開府儀同三司、蘭陵王長恭為尚書令。庚辰,以中書監魏收為尚書右僕射。



 武平元年春正月乙酉朔,改元。太師、并州刺史、東安王
 婁睿薨。戊申,詔兼散騎常侍裴獻之聘于陳。二月癸亥,以百濟王餘昌為使持節、侍中、驃騎大將軍、帶方郡公,王如故。己巳,以太傅、咸陽王斛律光為右丞相,并州刺史、右丞相、安定王賀拔仁為錄尚書事,冀州刺史、任城王湝為太師。丙子,降死罪已下囚。閏月戊戌,錄尚書事、安定王賀拔仁薨。三月辛酉,以開府儀同三司徐之才為尚書左僕射。夏六月乙酉,以廣寧王孝珩為司空。甲辰,以皇子恆生故,大赦。內外百官,普進二級;九州職人,普進四級。己酉,詔以開府儀同三司唐邕為尚書右僕射。秋七月癸丑,封孝昭皇帝子彥基為城陽王,彥康為
 定陵王,彥忠為梁郡王。甲寅,以尚書令、蘭陵王長恭為錄尚書事,中領軍和士開為尚書令。癸亥,靺鞨遣使朝貢。



 癸酉,以華山王凝為太傅。八月辛卯,行幸晉陽。九月乙巳,立皇子恆為皇太子。



 冬十月辛巳,以司空、廣寧王孝珩為司徒,以上洛王思宗為司空,封蕭莊為梁王。



 戊子,曲降并州死罪已下囚。己丑,復改威宗景烈皇帝謚號為顯祖文宣皇帝。十二月丁亥,車駕至自晉陽。詔左丞相斛律光出晉州道,脩城戍。



 二年春正月丁巳,詔兼散騎常侍劉環俊使於陳。戊寅,以百濟王餘昌為使持節、都督、東青州刺史。二月壬寅,
 以錄尚書事、蘭陵王長恭為太尉,并省錄尚書事趙彥深為司空,尚書令和士開為錄尚書事,左僕射徐之才為尚書令,右僕射唐邕為左僕射,吏部尚書馮子琮為右僕射。夏四月壬午,以大司馬、瑯邪王儼為太保。甲午,陳遣使連和,謀伐周,朝議弗許。六月,段韶攻周汾州剋之,獲刺史楊敷。秋七月庚午,太保、瑯邪王儼矯詔殺錄尚書事和士開於南臺,即日誅領軍大將軍庫狄伏連、書侍御史王子宣等,尚書右僕射馮子琮賜死殿中。八月己亥,行幸晉陽。九月辛亥,以太師、任城王湝為太宰,馮翊王潤為太師。己未,左丞相、平原王段韶薨。戊午,曲
 降并州界內死罪已下,各有差。庚午,殺太保、瑯邪王儼。壬申,陳人來聘。冬十月,罷京畿府入領軍府。己亥,車駕至自晉陽。十一月庚戍,詔侍中赫連子悅使於周。丙寅,以徐州行臺、廣寧王孝珩為錄尚書事。庚午,以錄尚書事、廣寧王孝珩為司徒。癸酉,以右丞相斛律光為左丞相。



 三年春正月己巳,祀南郊。辛亥,追贈故瑯邪王儼為楚帝。二月己卯,以衛菩薩為太尉。辛巳,以并省吏部尚書高元海為尚書右僕射。庚寅,以左僕射唐邕為尚書令,侍中祖珽為左僕射。是月,敕撰《玄州苑御覽》,後改名《聖
 壽堂御覽》。



 三月辛酉,詔文武官五品已上,各舉一人。是月,周誅塚宰宇文護。夏四月,周人來聘。秋七月戊辰,誅左丞相、咸陽王斛律光,及其弟幽州行臺、荊山公豐樂。八月庚寅,廢皇后斛律氏為庶人。以太宰、任城王湝為右丞相,太師、馮翊王潤為太尉,蘭陵王長恭為大司馬,廣寧王孝珩為大將軍,安德王廷宗為司徒。使領軍封輔相聘于周。戊子,拜右昭儀胡氏為皇后。己丑,以司州牧、北平王仁堅為尚書令,特進許季良為左僕射,彭城王寶德為右僕射。癸巳,行幸晉陽。是月,《聖壽堂御覽》成,敕付史閣。後改為《修文殿御覽》。九月,陳人來聘。冬十月,
 降死罪已下囚。甲午,拜弘德夫人穆氏為左皇后,大赦。十二月辛丑,廢皇后胡氏為庶人。



 是歲,新羅、百濟、勿吉、突厥並遣使朝貢。於周為建德元年。



 四年春正月戊寅,以并省尚書令高阿那肱為錄尚書事。庚辰,詔兼散騎常侍崔象使於陳。是月,鄴都、并州並有狐媚,多截人髮。二月乙巳,拜左皇后穆氏為皇后。丙午,置文林館。乙卯,以尚書令、北平王仁堅為錄尚書事。丁巳,行幸晉陽。



 是月,周人來聘。三月辛未,盜入信州,殺刺史和士休,南兗州刺史鮮于世榮討之。



 庚辰,車駕至晉陽。夏四月戊午,以大司馬、蘭陵王長恭為太保,大將
 軍、定州刺史、南陽王綽為大司馬,大司馬、太尉衛菩薩為大將軍,司徒、安德王延宗為太尉,司空、武興王普為司徒,開府儀同三司、宜陽王趙彥深為司空。癸丑,祈皇祠。壇壝蕝之內忽有車軌之轍,案驗,傍無人跡,不知車所從來。乙卯,詔以為大慶,班告天下。己未,周人來聘。五月丙子,詔史官更撰《魏書》。癸巳,以領軍穆提婆為尚書左僕射,以侍中、中書監段孝言為右僕射。是月,開府儀同三司尉破胡、長孫洪略等與陳將吳明徹戰於呂梁南。大敗,破胡走以免,洪略戰歿。遂陷秦、涇二州。明徹進陷和、合二州。是月,殺太保、蘭陵王長恭。六月,明徹進軍
 圍壽陽。



 壬子,幸南苑,從官暍死者六十人。以錄尚書事高阿那肱為司徒。丙辰,詔開府王師羅使於周。秋九月,校獵于鄴東。冬十月,陳將吳明徹陷壽陽。辛丑,殺侍中崔季舒、張彫唐、散騎常侍劉逖、封孝琰、黃門侍郎裴澤、郭遵。癸卯,行幸晉陽。



 十二月戊寅,以司徒高阿那肱為右丞相。是歲,高麗、靺鞨並遣使朝貢,突厥使求婚。



 五年春正月乙丑,置左右娥英各一人。二月乙未,車駕至自晉陽。朔州行臺、南安王思好反。辛丑,行幸晉陽。尚書令唐邕等大破思好,投火死,焚其尸,并其妻李氏。丁未,車駕至自晉陽。甲寅,以尚書令唐邕為錄尚書事。夏
 五月,大旱,晉陽得死魃,長二尺,面頂各二目。帝聞之,使刻木為其形以獻。庚申,大赦。丁亥,陳人寇淮北。秋八月癸卯,行幸晉陽。甲辰,以高勱為尚書右僕射。是歲,殺南陽王綽。



 六年春三月乙亥,車駕至自晉陽。丁丑,烹妖賊鄭子饒於都市。是月,周人來聘。夏四月庚子,以中書監陽休之為尚書右僕射。癸卯,靺鞨遣使朝貢。秋七月甲戌,行幸晉陽。八月丁酉,冀、定、趙、幽、滄、瀛六州大水。是月,周師入洛川,屯芒山,攻逼洛城。縱火船焚浮橋,河橋絕。閏月己丑,遣右丞相高阿那肱自晉陽禦之,師次河陽,周師夜
 遁。庚辰,以司空趙彥深為司徒,斛律阿列羅為司空。辛巳,以軍國資用不足,稅關市、舟車、山澤、鹽鐵、店肆,輕重各有差,開酒禁。



 七年春正月壬辰,詔去秋已來,水潦,人飢不自立者,所在付大寺及諸富戶,濟其性命。甲寅,大赦。乙卯,車駕至自晉陽。二月辛酉,括雜戶女,年二十已下十四已上未嫁,悉集省。隱匿者,家長處死刑。二月丙寅,風從西北起,發屋拔樹,五日乃止。夏六月戊申朔,日有蝕之。庚申,司徒趙彥深薨。秋七月丁丑,大雨霖。



 是月,以水澇,遣使巡撫流亡人戶。八月丁卯,行幸晉陽。雉集於御坐,獲之,有
 司不敢以聞。詔營邯鄲宮。冬十月丙辰,帝大狩於祁連池。周師攻晉州。癸亥,帝還晉陽。甲子,出兵,大集晉祠。庚午,帝發晉陽。癸酉,帝列陣而行,上雞栖原,與周齊王憲相對,至夜不戰。周師斂陣而退。十一月,周武帝退還長安,留偏師守晉州,高阿那肱等圍晉州城。戊寅,帝至圍所。十二月戊申,周武帝來救晉州。庚戌,戰于城南,齊軍大敗。帝棄軍先還。癸丑,入晉陽,憂懼不知所之。甲寅,大赦。帝謂朝臣曰:「周師甚盛,若何?」群臣咸曰:「天命未改,一得一失,自古皆然。宜停百賦,安朝野,收遺兵,背城死戰,以存社稷。」帝意猶預,欲向北朔州。乃留安德王廷宗、廣
 寧王孝珩等守晉陽。若晉陽不守,即欲奔突厥。群臣皆曰不可,帝不從其言。開府儀同三司賀拔伏恩、封輔相、慕容鐘葵等宿衛近臣三十餘人,西奔周師。乙卯,詔募兵,遣安德王延宗為左廣,廣寧王孝珩為右廣。延宗入見帝,帝告欲向北朔州,延宗泣諫,不從。帝密遣王康德與中人齊紹等送皇太后、皇太子於北朔州。丙辰,帝幸城南軍營,勞將士,其夜欲遁,諸將不從。丁巳,大赦。改武平七年為隆化元年。其日,穆提婆降周。詔除安德王延宗為相國,委以備禦,延宗流涕受命。帝乃夜斬五龍門而出。欲走突厥,從官多散,領軍梅勝郎叩馬諫,乃迴之
 鄴。時唯高阿那肱等十餘騎,廣寧王孝珩、襄城王彥道續至,得數十人同行。戊午,延宗從眾議,即皇帝位於晉陽,改隆化為德昌元年。庚申,帝入鄴。



 辛酉,延宗與周師戰於晉陽,大敗,為周師所虜。



 帝遣募人,重加官賞,雖有此言,而竟不出物。廣寧王孝珩奏請出宮人及珍寶,班賜將士,帝不悅。斛律孝卿居中,受委帶甲以處分。請帝親勞,為帝撰辭,且曰:「宜慷慨流涕,感激人心。」帝既出臨眾,將令之,不復記所受言,遂大笑。左右亦群咍,將士莫不解體。於是自大丞相已下,太宰、大司馬、三師、大將軍、三公等官,並增員而授,或三或四,不可勝數。甲子,皇太
 后從北道至。引文武一品已上入朱華門。賜酒食及紙筆,問以禦周之方略。群臣各異議,帝莫知所從。又引高元海、宋士素、盧思道、李德林等欲議禪位皇太子。先是,望氣者言,當有革易,於是依天統故事,授位幼主。



 幼主名恆,帝之長子也。母曰穆皇后。武平元年六月,生於鄴。其年十月,立為皇太子。隆化二年春正月乙亥,即皇帝位,時年八歲。改元為承光元年,大赦。



 尊皇太后為太皇太后,帝為太上皇帝,后為太上皇后。於是黃門侍郎顏之推、中書侍郎薛道衡、侍中陳德信等勸太上皇帝往河外募兵,更為經略。若不濟,南投陳國。



 從之。丁丑,
 太皇太后、太上皇自鄴先趣濟州。周師漸逼。癸未,幼主又自鄴東走。



 己丑,周師至紫陽橋。癸巳,燒城西門,太上皇將百餘騎東走。乙亥,度河入濟州。



 其日,幼主禪位於大丞相、任城王湝,令侍中斛律孝卿送禪文及璽紱於瀛州。孝卿乃以之歸周。又為任城王詔,尊太上皇為無上皇。幼主為守國天王。留太皇太后濟州,遣高阿那肱留守。太上皇并皇后攜幼主走青州,韓長鸞、鄧顒等數十人從。太上皇既至青州,即為入陳之計。而高阿那肱召周軍,約生致齊主。而屢使人告,言賊軍在遠,已令人燒斷橋路。太上所以停緩。周軍奄至青州,太上窘急,將
 遜於陳,置金囊於鞍後。與長鸞、淑妃等十數騎至青州南鄧村,為周將尉暹綱所獲,送鄴。



 周武帝與抗賓主禮,并太后、幼主、諸王,俱送長安。封帝溫國公。至建德七年,誣與宜州刺史穆提婆謀反,及延宗等數十人,無少長咸賜死。神武子孫所存者一二而己。至大象末,陽休之、陳德信等啟大丞相隋公,請收葬。聽之,葬於長安北原洪瀆川。



 帝幼而令善;及長,頗學綴文,置文林館,引諸文士焉。而言語澀吶,無志度,不喜見朝士。自非寵私暱狎,未嘗交語。性懦不堪,人視者即有忿責。其奏事者,雖三公、令、錄莫得仰視。皆略陳大旨,驚走而出。每災異寇盜
 水旱,亦不自貶損;唯諸處設齋,以此為修德。雅信巫覡,解禱無方。初,瑯邪王舉兵,人告者誤云庫狄伏連反,帝曰:「此必仁威也。」又斛律光死後,諸武官舉高思好堪大將軍,帝曰:「思好喜反。」皆如所言,遂自以策無遺算,乃益驕縱。盛為無愁之曲,帝自彈胡琵琶而唱之,侍和之者以百數,人間謂之無愁天子。嘗出見群厲,盡殺之。或殺人,剝面皮而視之。任陸令萱、和士開、高阿那肱、穆提婆、韓長鸞等宰制天下;陳德信、鄧長顒、何洪珍參預機權。各引親黨,超居非次;官由財進,獄以賄成;其所以亂政害人。難以備載。諸官奴婢、閹人、商人、胡戶、雜戶、歌舞人、
 見鬼人濫得富貴者,將以萬數。庶姓封王者百數,不復可紀。開府千餘,儀同無數。領軍一時三十,連判文書,各作依字,不具姓名,莫知誰也。諸貴寵祖禰追贈,官歲一進,位極乃止。宮掖婢皆封郡君。宮女寶衣玉食者五百餘人。一裙直萬疋,鏡臺直千金。競為變巧,朝衣夕弊。承武成之奢麗,以為帝王當然。乃更增益宮苑,造偃武修文臺。其嬪嬙諸院中,起鏡殿、寶殿、瑇瑁殿,丹青彫刻,妙極當時。又於晉陽起十二院,壯麗逾於鄴下。所愛不恒,數毀而又復。夜則以火照作,寒則以湯為泥。百工困窮,無時休息。鑿晉陽西山為大佛像,一夜燃油萬盆,光照
 宮內。又為胡昭儀起大慈寺,未成,改為穆皇后大寶林寺。窮極工巧,運石填泉,勞費億計,人牛死者,不可勝紀。御馬則藉以氈罽,食物有十餘種。將合牝牡,則設青廬,具牢饌而親觀之。狗則飼以梁肉。馬及鷹犬,乃有儀同、郡君之號。故有赤彪儀同、逍遙郡君、陵霄郡君。高思好書所謂馱龍,逍遙著也。犬於馬上設褥以抱之。鬥雞亦號開府。犬馬雞鷹,多食縣幹。鷹之入養者,稍割犬肉以飼之,至數日乃死。又於華林園立貧窮村舍,帝自弊衣為乞食兒。又為窮兒之市,躬自交易。寫築西鄙諸城,黑衣為羌兵;鼓噪陵之,親率內參臨拒,或實彎弓射人。自
 晉陽東巡,單馬馳騖,衣解髮散而歸。又好不急之務,曾一夜索蠍,及旦,得三升。特愛非時之物,取求火急,皆須朝征夕辦。當勢者因之,貸一而責十焉。賦斂日重,徭役日煩;人力既殫,帑藏空竭。乃賜諸佞幸賣官,或得郡兩三,或得縣六七,各分州郡,下逮鄉官,亦多降中者。故有敕用州主簿,敕用郡功曹。於是州縣職司,多出富商大賈。



 競為貪縱,人不聊生。爰自鄴都及諸州郡,所在征稅,百端俱起。凡此諸役皆漸於武成,至帝而增廣焉。然未嘗有有帷薄淫穢,唯此事頗優於武成云。



 初,河清末,武成夢大蝟攻破鄴城,故索境內蝟膏以絕之。識者以後主
 名聲與蝟相協,亡齊徵也。又婦人皆剪剔以著假髻;而危邪之,狀如飛鳥,至於南面,則髻心正西。始自宮內為之。被於四遠。天意若曰:「元首翦落,危側,當走西也。」



 又為刀子者,刃皆狹細,名曰盡勢。遊童戲者,好以兩手持繩,拂地而卻上跳,且唱曰「高末」。高末之言,蓋高氏運祚之末也。然則亂亡之數,蓋有兆云。



 論曰:武成風度高爽,經算弘長。文武之官,俱盡謀力,有帝王之量矣。但愛狎庸豎,委以朝權;帷薄之間,淫侈過度。滅亡之兆,其在斯乎。玄象告變,傳位元子;名號雖殊,政猶己出;迹有虛飾,事非憲典;聰明臨下,何易可誣。又
 河南、河間、樂陵等諸王,或以時嫌,或以猜忌,皆無罪而殞。非所謂知命任天體大道之義也。後主以中庸之姿,懷易染之性。永言先訓,教匪義方。始自襁褓,至于傳位,隔以正人,閉其善道。養德所履,異乎春誦夏弦。過廷所聞,莫非不軌不物。輔之以中官奶媼,屬之以麗色淫聲;縱巘紲之娛,恣朋淫之好。語曰:「從惡若崩」,蓋言其易。武平在御,彌見淪胥;罕接朝士,不親政事;一日萬機,委諸凶族。內侍帷幄,外吐絲綸;威厲風霜,志迴天日;虐人害物,搏噬無厭;賣獄鬻官,谿壑難滿。重以名將貽禍,忠臣顯戮;始見浸溺之萌,俄觀土崩之勢。周武因機,遂混區
 夏,悲天!蓋桀紂罪人,其亡也忽焉,自然之理矣。



 鄭文貞公魏徵總而論之曰:神武以雄傑之姿,始基霸業;文襄以英明之略,伐叛柔遠。于時喪君有君,師出以律。河陰之役,摧宇文如反掌;渦陽之戰,掃侯景如拉枯。故能氣懾西鄰,威加南服。王室是賴,東夏宅心。文宣因累世之資,膺樂推之會,地居當璧,遂遷魏鼎。懷譎詭非常之才,運屈奇不測之智;網羅俊乂,明察臨下;文武名臣,盡其力用。親戎出塞,命將臨江。定單于於龍城,納長君於梁國。外內充實,疆場無警;胡騎息其南侵,秦人不敢東顧。既而荒淫敗德,罔念作狂;為善未能亡身,餘殃
 足以傳後。得以壽終,幸也;胤嗣不永,宜哉。孝昭地逼身危,逆取順守;外敷文教,內蘊雄圖;將以牢籠區域,奄有函夏。享齡不永,績用無成。若或天假之年,足使秦、吳旰食。武成即位,雅道陵遲。昭、襄之風,摧焉已墜。暨乎後主,外內崩離;眾潰於平陽,身禽于青土。天道深遠,或未易談;吉凶由人,抑可揚榷。觀夫有齊全盛,控帶遐阻:西包汾、晉,南極江、淮,東盡海隅,北漸沙漠。六國之地,我獲其五;九州之境,彼分其四。料甲兵之眾寡,校帑藏之虛實;折衝千里之將,帷幄六奇之士,比二方之優劣,無等級以寄言。然其太行、長城之固,自若也;江、淮、汾、晉之險,不
 移也;帑藏輸稅之富,未虧也;士庶甲兵之眾,不缺也。然而前王用之而有餘,後主守之而不足,其故何哉?前王之御時也,沐雨櫛風;拯其溺而救其焚,信必賞,過必罰,安而利之。既與共其存亡,故得同其生死。後主則不然。以人從欲,損物益己。雕墻峻宇,甘酒嗜音;廛肆遍於宮園,禽色荒於外內。俾晝作夜,罔水行舟;所欲必成,所求必得。既不軌不物,又暗於聽受;忠信弗聞,萋斐必入。視人如草芥,從惡如順流。佞閹處當軸之權,婢媼擅回天之力。賣官鬻獄,亂政淫刑;刳剒被於忠良,祿位加於犬馬。讒邪並進,法令多聞。持瓢者非止百人,搖樹者不唯
 一手。於是土崩瓦解,眾叛親離。



 顧瞻周道,咸有西歸之志。方更盛其宮觀,窮極荒淫;謂黔首之可誣,指白日以自保。驅倒戈之旅,抗前歌之師;五世崇基,一舉而滅。豈非鐫金石者難為功,摧枯朽者易為力歟。抑又聞之,「皇天無親,唯德是輔」。「天時不如地利,地利不如人和」。齊自河清之後,逮於武平之末,土木之工不息,嬪嬙之選無已。征稅盡,人力殫,物產無以給其求,江海不能贍其欲。所謂火既熾矣,更負薪以足之;數既窮矣,又為惡以促之。欲求大廈不燔,延期過歷,不亦難乎。由此言之,齊氏之敗亡,蓋亦由人,匪惟天道也。



\end{pinyinscope}