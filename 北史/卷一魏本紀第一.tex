\article{卷一魏本紀第一}

\begin{pinyinscope}

 魏之先,出自黃帝軒轅氏。黃帝子曰昌意,昌意之少子受封北國,有大鮮卑山,因以為號。其後世為君長,統幽都之北廣莫之野,畜牧遷徙,射獵為業,淳樸為俗,簡易為化,不為文字,刻木結繩而已。時事遠近,人相傳授,如史官之紀錄焉。黃帝以土德王。北俗謂土為托,謂后為跋,故以為氏。其裔始均,仕堯時,逐女魃於弱水,北人賴其勳,舜命為田祖。歷三代至秦、漢,獯鬻、獫狁、山戎、匈奴
 之屬,累代作害中州。而始均之裔不交南夏,是以載籍無聞。積六七十代,至成皇帝諱毛立,統國三十六,大姓九十九,威振北方。成帝崩,節皇帝貸立。節帝崩,莊皇帝觀立。莊帝崩,明皇帝樓立。明帝崩,安皇帝越立。安帝崩,宣皇帝推寅立。宣帝南遷大澤,方千餘里,厥土昏冥沮洳,謀更南徙,未行而崩。景皇帝利立。景帝崩,元皇帝俟立。元帝崩,和皇帝肆立。和帝崩,定皇帝機立。定帝崩,僖皇帝蓋立。



 僖帝崩,威皇帝儈立。威帝崩,獻皇帝鄰立。時有神人,言此土荒遐,宜徙建都邑。



 獻帝年老,乃以位授于聖武皇帝,命南移。山谷高深,九難八阻,於是欲止。有
 神獸,似馬,其聲類牛,導引歷年乃出,始居匈奴故地。其遷徙策略,多出宣、獻二帝,故時人並號曰推寅,蓋俗云鑽研之義。



 聖武皇帝諱詰汾,嘗田於山澤,欻見輜軿自天而下。既至,見美婦人自稱天女,受命相偶。旦日請還,期年周時復會于此,言終而別。及期,帝至先田處,果見天女,以所生男授帝,曰:「此君之子也,當世為帝王。」語訖而去。即始祖神元皇帝也。故時人諺曰:「詰汾皇帝無婦家,力微皇帝無舅家。」帝崩,神元皇帝立。



 神元皇帝諱力微。元年,歲在庚子,先是西部內侵,依於
 沒鹿迴部大人竇賓。



 神元有雄傑之度,後與賓攻西部,賓軍敗,失馬步走,神元使以所乘駿馬給之。賓歸,求馬主,帝隱而不言。賓後知,大驚,將分國之半奉帝,帝不受,乃進其愛女。



 賓猶思報恩,乃從帝所欲,徙所部北居長川。積數年,舊部人咸來歸附。及賓臨終,戒其二子,使謹奉神元。其子不從,乃陰謀逆。帝召殺之,盡并其眾。諸部大人悉服,控弦之士二十餘萬。三十九年,遷於定襄之盛樂。四月祭天,諸部君長皆來助祭,唯白部大人觀望不至。征而戮之,遠近肅然。帝乃告諸大人,為與魏和親計。



 四十二年,遣子文帝如魏,且觀風土。是歲,魏景元二
 年也。



 文帝諱沙漠汗,以國太子留洛陽。後文帝以神元春秋已高,求歸。晉武帝具禮護送。五十六年,文帝復如晉,其冬還國。晉征北將軍衛瓘以文帝雄異,恐為後患,請留不遣。復請以金錦賂國之大人,令致間隙。五十八年方遣帝。神元使諸部大人詣陰館迎帝。酒酣,帝仰視飛鳥,飛丸落之。時國俗無彈,眾大驚,相謂曰:「太子被服同南夏,兼奇術絕人。若繼國統,變易舊俗,吾等必不得志。」乃謀危害帝,並先馳還,曰:「太子引空弓而落飛鳥,似得晉人異法。」自帝在晉後,諸子愛寵,神元頗有所惑。及聞諸大人請,因曰:「當便除之。」於是諸大人馳詣塞南,矯
 害帝。



 其年,神元不豫。烏丸王庫賢親近任勢。先受衛瓘之貨,欲沮動諸部,因於庭中礪鉞斧,曰:「上恨汝曹讒殺太子,欲盡收諸大人長子殺之。」大人皆信,各各散走。神元尋崩,凡饗國五十八年,年一百四歲。道武即位,尊為始祖。子章皇帝悉鹿立,時諸部離叛。帝九年而崩。弟平皇帝綽立,七年而崩。文帝少子思皇帝立,思皇帝諱弗。政崇寬簡,百姓懷服。一年而崩。神元子昭帝祿官立。帝元年,分國為三部:一居上谷北,濡源西,東接宇文部,自統之;一居代郡之參合陂北,使文帝長子桓帝諱猗厓統之;一居定襄之盛樂故城,使桓帝弟穆
 帝猗盧統之。



 自神元以來,與晉和好。是歲,穆帝始出並州,遷雜胡北徙雲中、五原、朔方。



 又西度河,擊匈奴、烏丸諸部。自杏城以北八十里,迄長城原,夾道立碣,與晉分界。



 二年,葬文帝及皇后封氏。初,思帝欲改葬,未果而崩,至是述成前意焉。



 三年,桓帝度漠北巡,因西略諸國,凡積五歲,諸部降附者三十餘國。桓帝英傑魁岸,馬不能勝,常乘安車,駕大牛,牛角容一石。帝曾中蠱,嘔吐之地仍生榆,參合陂土無榆,故時人異之。



 十年,匈奴別種劉元海反晉於離石,自號漢王。並州刺史司馬騰來乞師,桓帝與帝大舉以助之,大破元海眾於西河、上黨。桓帝
 與騰盟於汾東而還,乃使輔相衛雄、段繁,於參合陂西累石為亭,樹碑以記行焉。



 十一年,晉假桓帝大單于金印紫綬。是歲,桓帝崩。桓帝統部凡十一年。後定襄侯衛操樹碑於大邗城,以頌功德。子普根代立。



 十三年,昭帝崩。穆帝遂總攝三部為一統。帝天姿英峙,勇略過人。



 元年,劉元海僭帝號,自稱大漢。



 三年,晉並州刺史劉琨遣子遵為質,乞師。帝使弟子平文皇帝助琨,破白部大人,次攻鐵弗劉武。晉懷帝進帝大單于,封代公。帝以封邑去國縣遠,從琨求句注陘北地。琨大喜,乃徙馬邑、陰館、樓煩、繁畤、崞五縣人於陘南,更立城邑,盡獻其地。東接
 代郡,西連西河、朔方數百百。帝乃徙十萬家以充之。



 六年,城盛樂以為北都,脩故平城以為南都。帝登平城西山,觀望地勢,乃更南百里,於水壘水之陽黃瓜堆築新平城,晉人謂之小平城。使子六脩鎮之,統領南部。



 八年,晉愍帝進帝為代王,置官屬,食代、常山二郡。先是國俗寬簡,至是明刑峻法,諸部人多以違命得罪。凡後期者皆舉部戮之,或有室家相攜,悉赴死所。



 人問何之,曰當就誅。其威嚴若此。



 九年,帝召六脩不至,怒,討之失利,遂崩。



 普根先守外境,聞難,來攻六脩滅之。普根立月餘薨。普根子始生,桓帝后立之,又薨,思帝子平文皇帝立。



 平文皇帝諱鬱律,姿質雄壯,甚有威略。元年,歲在丁丑。二年,劉武據朔方,來侵西部,帝大破之。西兼烏孫故地,東吞勿吉以西,控弦上馬將百萬。



 是歲,晉元帝即位於江南,劉曜僭帝位。帝聞晉愍帝為曜所害,顧謂大臣曰:「今中原無主,天其資我乎。」曜遣使請和,帝不納。



 三年,石勒自稱趙王,遣使乞和,請為兄弟,帝斬其使以絕之。五年,晉元帝遣使韓暢加崇爵服,帝絕之。講武,有平南夏志。桓帝后以帝得眾心,恐不利己子,害帝,遂崩,大人死者數十人。天興初,追尊曰太祖。



 桓帝中子惠帝賀傉立,以五年為元年。帝未親政事,太
 后臨朝。遣使與石勒通和,時人謂之女國使。四年,帝始臨朝,以諸部人情未悉款順,乃築城於東木根山,徙都之。五年,帝崩。



 弟煬帝紇那立,以五年為元年。三年,石勒遣石季龍寇邊部,帝禦之,不利,遷於大寧。



 時平文帝長子烈帝居於舅賀蘭部,帝遣使求之,賀蘭部帥藹頭擁護不遣。帝怒,召宇文部并力擊藹頭。宇文眾敗,帝還大寧。五年,帝出居於宇文部,賀蘭及諸部大人共立烈帝。



 烈皇帝諱翳槐,以五年為元年。石勒遣使求和,帝遣弟昭成帝如襄國,徙者五千餘家。七年,藹頭不脩臣職,召而戮之,國人復貳。於是煬帝自宇文部還入,諸部大人
 復奉之。



 煬帝以烈帝七年為後元年。時烈帝出居於鄴。三年,石季龍納烈帝於大寧。國人六千餘家部落叛,煬帝出居於慕容部。



 烈帝復立,以煬帝三年為後元年。城盛樂城,在故城東南十里。一年而崩。弟昭成皇帝立。



 昭成皇帝諱什翼犍,平文皇帝之次子也。生而奇偉,寬仁大度。身長八尺,隆準龍顏,立髮委地,臥則乳垂至席。烈帝臨崩,顧命迎帝,曰:「立此人則社稷乃安。」故帝弟孤自詣鄴奉迎,與帝俱還。



 建國元年十一月,帝即位於繁畤北,時年十九。



 二年春,始置百官,分掌眾職。東自水歲貊,西及破落那,莫不款附。五月,朝諸大人於參合陂,議定
 都水壘源川,連日不決,乃從太后計而止。娉慕容晃妹為皇后。



 三年春,移都雲中之盛樂宮。



 四年,築盛樂城於故城南八里。皇后慕容氏崩。十月,劉武寇西境,帝遣軍大破之。武死,子務桓立,始來歸順,帝以女妻之。



 七年二月,遣大人長孫秩迎后慕容氏於和龍。晃送女於境。七月,慕容晃遣使來聘,求交婚。帝許之,以烈帝女妻焉。



 十四年,帝以中州紛梗,將親率六軍,乘石氏之亂,廓定中原。諸大人諫,乃止。



 十八年,太后王氏崩。



 十九年正月,劉務桓死,其弟閼頭立,潛謀反。



 二十一年,閼頭部人多叛,懼而東走,度河半濟而冰陷。後眾盡歸其兄子悉勿祈。初,
 閼頭之叛,悉勿祈兄弟十二人在帝左右,盡遣之歸,欲其自相猜離。至是,悉勿祈奪其眾,閼頭窮而歸命,帝待之如初。



 二十二年春,帝東巡桑乾川。四月,悉勿祈死,弟衛辰立。



 二十三年六月,皇后慕容氏崩。七月,衛辰來會葬,因求婚,許之。



 二十五年,帝南巡君子津。



 二十八年正月,衛辰謀反,度河東。帝討之。衛辰懼,遁走。



 三十年十月,帝征衛辰。時河冰未成,帝乃以葦絙約澌。俄然冰合,乃散葦於上,冰草相結若浮橋,眾軍利涉。衛辰與宗族西走,收其部落而還。



 三十四年春,長孫斤謀反,伏誅。斤之反也,拔刃向御坐。太子寔格之,傷肋,五月薨。後追謚焉,
 是為獻明皇帝。七月,皇孫珪生,大赦。



 三十九年,苻堅遣其大司馬苻洛帥眾二十萬及其將朱彤、張蠔、鄧羌等諸道來寇,王師不利。帝時不豫,乃率國人避於陰山之北。高車雜種盡叛,四面寇抄,不得芻牧,復度漠南。堅軍稍退,乃還。十二月,至雲中。旬有二日,皇子寔君作亂。



 帝暴崩,時年五十七。道武即位,尊曰高祖。



 帝性寬厚。時國少繒帛,代人許謙盜絹二疋,守者以告,帝匿之,謂燕鳳曰:「吾不忍視謙之面,卿勿洩之。謙或慚而自殺,為財辱士,非也。」帝嘗擊西部叛賊,流矢中目。賊破後,諸大臣執射者,各持錐刀欲屠割之。帝曰:「各為其主,何罪也,釋之!」
 其仁恕若此。



 太祖道武皇帝諱珪,昭成皇帝之嫡孫,獻明帝之子也。母曰獻明賀皇后,初因遷徙,游於雲澤。寢夢日出室內,寤而見光自牖屬天,欻然有感。以建國三十四年七月七日生帝於參合陂北,其夜復有光明。昭成大悅,群臣稱慶,大赦,告于祖宗。



 保者以帝體重倍於常兒,竊獨奇怪。明年有榆生於藏胞之坎,後遂成林。帝弱而能言,目有光曜,廣顙大耳。六歲而昭成崩,苻堅遣將內侮,將遷帝長安,賴燕鳳乃免。堅軍既還,國眾離散。堅使劉庫仁、劉衛辰分攝國事。南部大人長孫嵩及元他等盡將故
 人眾南依庫仁,帝於是轉在獨孤部。



 元年,葬昭成皇帝於金陵,營梓宮木柿盡生成林。帝雖沖幼,而嶷然不群。劉庫仁常謂其子曰:「帝有高天下之志,必興復洪業。」



 七年十月,晉敗苻堅于淮南。慕容文等殺劉庫仁,弟眷代攝國部。



 八年,慕容弟沖僭立。姚萇自稱大單于、萬年秦王。慕容垂僭稱燕王。



 九年,劉庫仁子顯殺眷而代之,乃將謀逆。商人王霸知之,履帝足於眾中,帝乃馳還。是時,故大人梁盆子六眷為顯謀主,盡知其計,密使部人穆崇馳告。帝乃陰結舊臣長孫犍、元他等,因幸賀蘭部。其日,顯果使人殺帝,不及。語在《獻明太后傳》。是歲,乞伏
 國仁私署秦、河二州牧、大單于。姚萇殺苻堅,堅子丕僭即皇帝位於晉陽。



 登國元年春正月戊申,帝即代王位,郊天建元,大會於牛川。復以長孫嵩為南部大人,以叔孫普洛為北部大人。是月,慕容垂僭即皇帝位于中山,國號燕。二月,幸定襄之盛樂,息眾課農。慕容沖為其部下所殺。夏四月,改稱魏王。五月,姚萇僭即皇帝位於長安,國號大秦。秋八月,劉顯遣弟亢泥迎皇叔父窟咄於慕容永,以兵隨之,來逼南境。帝左右于桓等與諸部大人謀應之。事洩,誅造謀者五人,餘悉不問。帝慮內難,乃北踰陰山,幸賀蘭
 部,阻山為固。遣行人安同、長孫賀徵師於慕容垂。垂令其子賀驎率師隨同等。軍未至而寇逼。於是北部大人叔孫普洛等十三人及諸烏丸亡奔衛辰。帝自弩山幸牛川,屯於延水,南出代谷,會賀驎於高柳,大破窟咄,悉收其眾。冬十月,苻丕為晉將馮該所殺。慕容永僭即皇帝位於長子。十一月,苻登僭即皇帝位於隴東。十二月,慕容垂遣使奉帝西單于印綬,封上谷王。



 帝不納。



 二年夏五月,遣安同徵兵於慕容垂。垂遣子賀驎率眾來會。六月,帝親征劉顯,顯奔慕容永,盡收其部落。冬十二月,巡松漠,還幸牛川。



 三年夏五月癸亥,北征庫莫奚,大破之。六月,乞伏國仁死,其弟乾歸立,私署河南王。秋七月,庫莫奚部帥鳩集遺散,夜犯行宮,縱騎撲討,盡滅之。八月,使九原公儀於慕容垂。冬十月,垂遣使朝貢。



 四年春正月甲寅,襲高車諸部落。二月癸巳,遂至女水,討叱突鄰部。並大破之。是月,呂光自稱三河王。夏五月,使陳留公虔於慕容垂。冬十月,垂遣使朝貢。



 五年春三月甲申,西征,次鹿渾海,襲高車袁紇部,大破之。慕容垂遣子賀驎來會。夏四月丙寅,行幸意辛山,與賀驎討賀蘭、紇奚諸部落,大破之。秋八月,還幸牛川。使
 秦王觚於慕容垂。九月壬申,討叱奴部囊曲水,破之。冬十月,討高車豆陳部於狼山,破之。十二月,帝還次白漠。



 六年春正月,幸紐垤川。三月,遣九原公儀、陳留公虔等西討黜弗部,大破之。



 夏四月,祭天。秋七月壬申,講武於牛川。慕容垂止秦王觚而求名馬,帝絕之。乃遣使於慕容永,永使其大鴻臚慕容鈞奉表勸進尊號。九月,帝襲五原,屠之,收其積穀。還紐垤川,於棝陽塞北樹碑記功。冬十月戊戌,北征蠕蠕,追破之於大磧南商山下。十一月戊辰,還幸紐垤川。戊寅,衛辰遣子直力鞮寇南部。壬午,帝大破之於鐵歧山南,衛辰父子奔遁。十二月,滅之,
 衛辰少子屈丐亡奔薛于部。自河以南,諸部悉平。收衛辰子弟宗黨無少長五千餘人,盡殺之。是歲,起河南宮。



 七年春正月,幸木根山,遂次黑鹽池,饗群臣,北之美水。三月,還幸河南宮。



 秋七月,行幸漠南,仍築巡臺。冬十二月,慕容永遣使朝貢。



 八年春正月,南巡。二月,幸羖羊原,赴白樓。夏六月,北巡。秋七月,臨幸新壇。先是衛辰子屈丐奔薛干部,征之不送。八月,帝南征薛干部,屠其城。九月,還幸河南宮。



 九年春三月,北巡。使東平公元儀屯田於河北五原,至於棝陽塞外。夏五月,田於河東。秋七月,還幸河南宮。冬
 十月,蠕蠕社侖等率部落西走。是歲,姚萇子興僭立,殺苻登。慕容垂滅永。



 十年秋七月,慕容垂遣其子寶來寇五原。八月,帝親兵於河南。冬十月辛未,寶燒船夜遁。己卯,帝進軍濟河。乙酉夕,至參合陂。丙戌,大破之,禽其王公以下文武將吏數千人。於俘虜中擢其才識者賈彞、賈閏、晁崇等參謀議,憲章故實。



 十二月,還幸雲中之盛樂。



 皇始元年春正月,大蒐於定襄,因東幸善無北陂。三月,慕容垂寇桑乾川,陳留公虔死之。垂遂至平城西北,聞帝將至,乃築城自守。疾甚,遂遁,死於上谷。



 子寶秘喪,還
 至中山乃僭立。夏六月丁亥,皇太后賀氏崩。是月,葬獻明太后。呂光僭稱天王,國號涼。秋七月,左司馬許謙上書勸進尊號,於是改元,始建天子旌旗,出警入蹕。八月己亥,大舉討慕容寶。帝親勒六軍四十餘萬南出馬邑,踰句注,旌旗絡繹二千餘里,鼓行而前,人屋皆震。別詔將軍封真等從東道襲幽州,圍薊。



 九月戊午,次陽曲,乘西山,臨觀晉陽。寶並州牧、遼西王農棄城遁,並州平。初建臺省,置百官,封拜公、侯、將軍、刺史、太守。尚書郎以下悉用文人。帝初拓中原,留心慰納。諸士大夫詣軍門者,無少長皆引入,人得盡言,茍有微能,咸蒙敘用。己未,詔
 輔國將軍奚牧略地晉川,獲慕容寶、丹楊王買得等於平陶城。九月,晉孝武帝殂。冬十一月庚子朔,帝至真定。自常山以東,守宰或捐城奔竄,或稽顙軍門,唯中山、鄴、信都三城不下。別詔東平公儀攻鄴,冠軍將軍王建、左軍將軍李栗等攻信都,軍所行不得傷桑棗。戊午,進軍中山。己未,圍之。帝曰:「朕量寶不能出戰,必憑城自守,急攻則傷士,久守則費糧,不如先平鄴、信都,然後還取中山。」諸將稱善。丁卯,車駕幸魯口城。



 二年春正月壬戌,帝引騎圍信都。其夜,寶冀州刺史、宜都王慕容鳳踰城奔中山。癸亥,寶輔國將軍張驤、護軍
 將軍徐超舉城降。是月,鮮卑禿髮烏孤私署大單于、西平王。二月丁丑,帝軍于鉅鹿之柏肆塢,臨滹沱水。其夜,寶悉眾犯營,燎及行宮,兵人駭散。帝驚起,不及衣冠,跣出擊鼓。俄而,左右及中軍將士稍集。



 帝設奇陣,列烽營外,縱騎衝之。寶眾大敗,走還中山,獲其器械數十萬計。寶尚書閔亮、祕書監崔逞等降者相屬,賜拜職爵各有差。三月己酉,車駕次盧奴。寶求和,請送秦王觚,割常山以西奉魏,乞守中山以東。帝許之。已而寶背約。辛亥,車駕次中山,命將圍之。是夜,寶弟賀驎將妻子走西山。寶恐賀驎先據和龍,壬子夜,北遁。城內共立慕容普鄰為
 主。夏四月,帝以軍糧不繼,詔東平公儀罷鄴圍,徙屯巨鹿。五月庚子,帝以中山城內為普鄰所脅,乃招喻之。甲辰,曜兵揚威,以示城內,命諸軍罷圍南徙,以待其變。甲寅,以東平公儀為左丞相,封衛王。進襄城公題爵為王。秋七月,普鄰遣烏丸、張驤率五千餘人出城求食,寇靈壽。賀驎自丁零中入軍,因其眾,復入中山,殺普鄰而自立。八月丙寅朔,帝進軍九門。時大疫,人馬牛死者十五六,中山猶拒守,群下咸思北還。帝知之,謂曰:「斯固天命,將若之何!四海之人皆可與為國,在吾所以撫之耳,何恤乎無人!」群臣乃不敢言。



 九月,賀驎飢窮,率三萬餘
 人寇新市。甲子晦,帝進軍討之。太史令晁崇奏曰:「不吉。」帝曰:「何也?」對曰:「紂以甲子亡,兵家忌之。」帝曰:「周武不以甲子勝乎?」崇無以對。冬十月丙寅,帝進軍新市,賀驎退阻泒水,依漸洳澤以自固。甲戌,帝臨其營,戰於義臺塢,大破之。賀驎單馬走鄴,慕容德殺之。



 甲申,賀驎所署公卿尚書將吏士卒降者二萬餘人。其將張驤、李沈、慕容文等先來降,尋皆亡還,是日復獲之,皆赦而不問。獲其所傳皇帝璽綬、圖書、府庫珍寶。



 中山平。乙酉,襄城王題薨。



 天興元年春正月,慕容德走保滑臺,衛王儀剋鄴。庚子,
 行幸真定,遂幸鄴。



 百姓有老病不能自存者,詔郡縣振恤之。帝至鄴,巡登臺榭,遍覽宮城,將有定都之志,乃置行臺。遂還中山,所過存問百姓。詔大軍所經州郡皆復貲租一年,除山東人租賦之半。車駕將北還,發卒萬人通直道,自望都鐵關鑿恆嶺至代五百餘里。



 帝慮還後山東有變,乃於中山置行臺,詔衛王儀鎮之,使略陽公遵鎮勃海之合口。



 右軍將軍尹國先督租于冀州,聞帝將還,謀反,欲襲信都,安南將軍長孫嵩執送,斬之。辛酉,車駕發中山,至於望都堯山。徙山東六州人吏及徒何、高麗雜夷、三十六署百工伎巧十餘萬口以充京師。車
 駕次於恆山之陽。博陵、勃海、章武諸郡群盜並起,略陽公遵等討之。是月,慕容德自稱燕王,據廣固。二月,車駕至自中山。



 幸繁畤宮。更選屯衛。詔給內徙新戶耕牛,計口受田。三月,徵左丞相、衛王儀還京師,詔略陽公遵代鎮中山。夏四月壬戌,以歷陽公穆崇為太尉,鉅鹿公長孫嵩為司徒,進封略陽公遵為常山王,南安公順為毗陵王。祭天於西郊,旗幟有加焉。廣平太守、遼西公意列謀反,與郡人韓奇矯假讖圖,將襲鄴城。詔反者就郡賜死。是月,蘭汗殺慕容寶而自立為大單于、昌黎王。六月丙子,詔有司議定國號。群臣奏曰:「昔周、秦以前,帝王居
 所生之土,及王天下,即承為號。今國家啟基雲、代,應以代為號。」詔曰:「昔朕遠祖總御幽都,控制遐國,雖踐王位,未定九州。逮于朕躬,掃平中土,凶逆蕩除,遐邇率服,宜仍先號為魏。」秋七月,遷都平城,始營宮室,建宗廟,立社稷。慕容寶子盛殺蘭汗而自立為長樂王。八月,詔有司正封畿,制郊甸,端徑術,標道里,平五權,較五量,定五度。遣使循行郡國,舉奏守宰不法者,親覽察黜陟之。冬十月,起天文殿。十一月辛亥,尚書吏部郎中鄧彥海典官制,立爵品,定律呂,協音樂。儀曹郎中董謐撰郊廟、社稷、朝覲、饗宴之儀。三公郎中王德定律令,申科禁。太史令
 晁崇造渾儀,考天象。吏部尚書崔宏總裁之。閏月,左丞相衛王儀及王公卿士,詣闕上書曰:「臣等聞宸極居中,則列宿齊其晷。帝王順天,則群后仰其度。伏惟陛下德協二儀,道隆三五,仁風被于四海,盛化塞于天區,澤及昆蟲,恩霑行葦,謳歌所屬,八表歸心。而躬履謙虛,退身後己,宸儀未彰,袞服未御,非所以上允皇天之意,下副樂推之心。臣等謹昧死以聞。」



 帝三讓乃許之。十二月己丑,帝臨天文殿。太尉、司徒進璽綬,百官咸稱萬歲。大赦,改元,追尊成帝以下及后號謚,樂用《皇始之舞》。詔百司議定行次,尚書崔宏等奏從土德,服色尚黃,數用五,祖
 以未,臘以辰,犧牲用白,五郊立氣,宣贊時令,敬授人時,行夏之正。徙六州二十二郡守宰豪傑吏人二千家於代都。



 二年春正月甲子,初祀上帝于南郊,以始祖神元皇帝配,降壇視燎,成禮而反。



 乙丑,赦京師。始制三駕之法。庚午,北巡。分命諸將大襲高車,常山王遵三軍從東道出長川,高涼王樂真等七軍從西道出牛川,車駕親勒六軍從中道自駮髯水西北出。二月丁亥朔,諸軍同會,破高車雜種三十餘部。衛王儀督三將別從西北絕漠千餘里,破其遺迸七部。還次牛川,及薄山,並刻石紀功。以
 所獲高車眾起鹿苑於南臺陰,北距長城,東苞白登,屬之西山,廣輪數十里,鑿渠引武川水,注之苑中,疏為三溝,分流宮城內外。又穿鴻鴈池。三月己未,車駕至自北伐。甲子,初令《五經》群書各置博士,增國子太學生員三千人。是月,氐人李辯叛慕容德,求援於鄴。行臺尚書和跋以輕騎應之,剋滑臺,收德宮人府藏。秋七月,起天華殿。辛酉,大閱于鹿苑。八月,增啟京城十二門,作西武庫。除州郡人租賦之半。辛亥,詔禮官備撰眾儀,著于新令。范陽人盧溥聚眾海濱,稱幽州刺史,攻掠郡縣,殺幽州刺史封沓干。是月,禿髮烏孤死,其弟利鹿孤立,遣使朝
 貢。冬十月,太廟成,遷神元、平文、昭成、獻明皇帝神主於太廟。十二月,天華殿成。呂光立其子紹為天王,自稱太上皇,及死,庶子纂殺紹而僭立。



 三年春正月戊午,材官將軍和突破盧溥於遼西,獲之,及其子煥傳送京師,轘之。癸亥,祀北郊。分命諸官循行州郡,觀風俗,察舉不法。二月丁亥,詔有司祀日于東郊。始耕籍田。壬寅,皇子聰薨。三月戊午,立皇后慕容氏。是月,穿城南渠通於城內,作東西魚池。夏四月,姚興遣使朝貢。五月戊辰,詔謁者僕射張濟使於興。己巳,東巡,遂幸涿鹿,遣使者以太牢祀帝堯、帝舜廟。西幸馬邑,觀水
 壘源。六月庚辰朔,日有蝕之。秋七月,乞伏乾歸大為姚興所破。壬子,車駕還宮。



 起中天殿及雲母堂、金華室。時太史屢奏天文錯亂,帝親覽經占,多云宜改王易政,於是數革官號,欲以防塞凶狡,消弭災變。已而慮臣下疑惑,冬十二月丙申,下詔述成敗之理,鑒殷、周之失,革秦、漢之弊,以喻臣下。是歲,河右諸郡奉、涼武昭王李玄盛為秦涼二州牧、涼公,肇興霸業,年號庚子。



 四年春二月丁亥,命樂師入學習舞,釋菜于先聖、先師。丁酉,分命使者巡行州郡,聽察辭訟,糾劾不法。是月,呂光弟子隆弒呂纂而自立。三月,帝親漁,薦于寢廟。夏四
 月辛卯,罷鄴行臺。詔有司明揚隱逸。五月,起紫極殿、玄武樓、涼風觀、石池、鹿苑臺。六月,盧水胡沮渠蒙遜私署涼州牧、張掖公。秋七月,詔兗州刺史長孫肥南徇許昌、彭城。詔賜天下鎮戍將士布帛各有差。八月,段興殺慕容盛,叔父熙盡誅段氏,僭即皇帝位。冬十二月,集博士儒生比眾經文字,義類相從,凡四萬餘字,號曰《眾文經》。是歲,涼武昭王、沮渠蒙遜並遣使朝貢。



 五年春正月,帝聞姚興將寇邊,庚寅,大簡輿徒,詔并州諸軍積穀于平陽乾壁。



 三月,禿髮利鹿孤死。夏五月,姚興遣其弟義陽王平來侵平陽,攻陷乾壁。秋七月戊辰
 朔,車駕西討。八月乙巳,至乾壁,平固守,進軍圍之。姚興悉舉其眾來救。



 甲子,帝度蒙坑,逆擊興軍大破之。冬十月,平赴水而死,俘其餘眾三萬餘人,獲興尚書左僕射狄伯支以下四品將軍以上四十餘人。獲前亡臣王次多、靳勒,並斬以徇。興頻使請和,帝不許。群臣請進平蒲阪,帝慮蠕蠕為難,戊申,班師。十一月,車駕次晉陽。徵相州刺史庾岳為司空。十二月辛亥,至自西征。越勒莫弗率其部萬餘家內屬。



 六年春正月辛未,朔方尉遲部別帥率萬家內屬,入居雲中。夏四月癸巳朔,日有蝕之。五月,大簡輿徒,將略江
 淮。秋七月,鎮西大將軍、司隸校尉、毗陵王順有罪,以王還第。戊子,北巡,築離宮于豺山,縱士校獵,東北踰罽嶺,出參合、代谷。九月,行幸南平城,規度水壘南夏屋山,背黃瓜堆,將建新邑。辛未,車駕還宮。冬十月,起西昭陽殿。乙卯,立皇子嗣為齊王,加車騎大將軍,位相國。紹為清河王,加征南大將軍。熙為陽平王,曜為河南王。封故秦愍王子夔為豫章王,陳留桓王子悅為朱提王。丁巳,晉人來聘。十一月庚午,將軍伊謂大破高車。十二月,晉桓玄廢其主司馬德宗為平固王而自立,僭號楚。



 天賜元年春二月,晉劉裕起兵誅桓玄。三月,初限縣戶
 不滿百罷之。夏五月,置山東諸冶,發州郡徒謫造兵甲。秋九月,帝臨昭陽殿,分置眾職,引朝臣文武親自簡擢,量能敘用。制爵四等,曰王、公、侯、子,除伯、男之號。追錄舊臣,加封爵各有差。是秋,江南大亂,流人繦負奔淮北者行道相尋。冬十月辛巳,大赦,改元。築西宮。十一月,幸西宮,大選臣僚,令各辨宗黨,保舉才行,諸部子孫失業賜爵者二千餘人。



 二年春正月,晉主司馬德宗復位。夏四月,祀西郊,車旗盡黑。冬十月,慕容德死。



 三年春正月甲申,北巡,幸豺山宮,校獵,還至屋孤山。二
 月乙亥,幸代園山,建五石亭。三月庚子,車駕還宮。夏四月庚申,復幸豺山宮。占授著作郎王宜弟造《兵法孤虛立成圖》三百六十。時遂登定襄角史山,又幸馬城。甲戌,車駕還宮。



 六月,發八部五百里內男丁築水壘南宮,門闕高十餘丈。引溝穿池,廣苑囿。規立外城方二十里,分置市里,經途洞達。三十日罷。秋七月,太尉穆崇薨。八月甲辰,行幸豺山宮,遂至青牛山。丙辰,西登武要北原,觀九十九泉,造古亭,遂之石漠。



 九月甲戌朔,幸漠南鹽池。壬午,至漠中,觀天鹽池。度漠北,之吐鹽池。癸巳,南還長川。丙申,臨觀長陂。冬十月庚申,車駕還宮。



 四年春二月,封皇子脩為河間王,處文為長樂王,連為廣平王,黎為京兆王。



 夏五月,北巡,自參合陂東過蟠羊山,大雨,暴水流輜車數百乘,殺百餘人。遂東北踰石漠,至長川,幸濡源。常山王遵有罪賜死。六月,赫連屈丐自稱大單于、大夏天王。秋七月,西幸參合陂。築北宮垣,三旬而罷,乃還宮。慕容寶養子高雲殺慕容熙而自立,僭號天王。八月,誅司空庾岳。



 五年春正月,行幸豺山宮,遂如參合陂,觀漁于延水,至寧川。三月,姚興遣使朝貢。秋七月戊戌朔,日有蝕之。冬十月,禿髮傉檀僭即涼王位。



 六年夏,帝不豫。初,帝服寒食散,自太醫令陰羌死後,藥數動發,至此愈甚。



 而災變屢見,憂懣不安,或數日不食,或不寢達旦,歸咎群下,喜怒乖常。謂百僚左右不可信,慮如天文之占,或有肘腋之虞。追思既往成敗得失,終日竟夜獨語不止,若傍有鬼物對揚者。朝臣至前,追其舊惡,便見殺害。其餘或以顏色變動,或以喘息不調,或以行步乖節,或以言辭失措,帝以為懷惡在心,變見於外,乃手自毆擊。死者皆陳天安殿前。於是朝野人情各懷危懼,有司廢怠,莫相督攝,百工偷劫,盜賊公行,巷里之間,人為稀少。帝亦聞之,曰:「朕故縱之使然,幸過災年,
 當更清整之耳。」秋七月,慕容氏支屬百餘家謀欲外奔,發覺,伏誅死者三百餘人。



 八月,衛王儀謀叛,賜死。十月戊辰,清河王紹作亂,帝崩於天安殿,時年三十九。



 永興二年九月甲寅,上謚曰宣武皇帝,葬於盛樂金陵,廟號太祖,泰常五年改謚曰道武。



 太宗明元皇帝諱嗣,道武皇帝之長子也。母曰劉貴人,登國七年,生於雲中宮。



 道武晚有男,聞而大悅,乃大赦。帝明睿寬毅,非禮不動。天興六年,封齊王,拜相國。初,帝母既賜死,道武召帝告曰:「昔漢武將立其子而殺其母,不令婦人與國政,汝當繼統,故吾遠同漢武。」帝素純孝,
 哀不自勝。道武怒。帝還宮,哀不自止,道武知而又召帝。帝欲入,左右諫,請待和解而進,帝從之。及元紹之逆,帝還而誅之。



 永興元年冬十月壬午,皇帝即位,大赦改元,追尊皇妣為宣穆皇后。公卿大臣先罷歸第者,悉復登用之。詔南平公長孫嵩、北新侯安同對理人訟,簡賢任能。是月,馮跋弒其主高雲,僭號天王,國號北燕。閏十月丁亥,硃提王悅謀反,賜死。



 詔都兵將軍山陽侯奚斤巡諸州,問人疾苦。十二月戊戌,封衛王儀子良為南陽王,進陰平公列爵為王,改封高涼王樂真為平陽王。己亥,帝始居西
 宮,御天文殿。蠕蠕犯塞。是歲,乞伏乾歸自稱秦王。



 二年春正月甲寅朔,詔南平公長孫嵩等北征蠕蠕,因留屯漠南。夏五月,嵩等自大漠還,蠕蠕追圍之於牛川。壬申,帝北伐,蠕蠕聞而遁走。車駕還幸參合陂。



 六月,晉將劉裕滅慕容超。秋七月丁巳,立射臺於陂西,仍講武。乙丑,至自北伐。



 三年春二月戊戌,詔簡宮人非御及伎巧者,悉以賜鰥人。己亥,詔北新侯安同等持節巡行並、定二州及諸山居雜胡、丁零,問其疾苦,察舉守宰不法者。辛丑,簡宮人工伎之不急者出,賜人不能自存者。三月己未,詔侍臣
 常佩劍。夏五月丙寅,復出宮人賜鰥人。丁卯,車駕謁金陵於盛樂。己巳,昌黎王慕容伯兒謀反,伏誅。



 六月,姚興遣使朝貢。秋七月戊申,賜衛士酺三日。冬十一月丁未,大閱于東郊。



 四年春二月癸未,登獸圈,射猛獸。夏四月乙未,宴群臣於西宮,使各獻直言,勿有所諱。六月,乞伏乾歸為兄子公府所弒。閏月丙辰,大閱於東郊。秋七月己巳朔,東巡。置四廂大將,又放十二時,置十二小將。以山陽侯奚斤、元城公屈行左右丞相。己卯,大獼于石會山。戊子,臨去畿陂觀漁。庚寅,至于濡源,西巡,幸北部諸落。八月壬子,
 幸西宮,臨板殿,大饗群臣,命百姓大酺三日。乙卯,賜王公以下至宿衛將士布各有差。冬十一月己丑,賜宗室近屬南陽王良以下至於緦麻親布帛各有差。是月,沮渠蒙遜僭稱河西王。十二月丁巳,北巡,至長城而旋。



 五年春正月己巳,大閱畿內,男年十二以上悉集。己卯,幸西宮。頞拔大渠帥四十餘人詣闕奉貢,賜以繒帛錦罽各有差。乙酉,詔諸州,六十戶出戎馬一匹。庚寅,大閱于東郊,署將帥,以山陽侯奚斤為前軍,眾三萬;陽平王熙等十二將各一萬騎;帝臨白登,躬自校覽。二月庚戌,幸高柳川。癸丑,穿魚池於北苑。庚午,姚興遣使朝貢。己
 卯,詔使者巡行天下,招延俊彥,搜揚隱逸。夏四月乙卯,西巡。



 五月乙亥,行幸雲中舊宮之大室。丙子,大赦。六月,西幸五原,校獵於骨羅山,獲獸十萬。秋七月己巳,還幸薄山。帝登觀宣武游幸刻石頌德之處,乃於其旁起石壇而薦饗焉,賜從者大酺於山下。前軍奚斤等破越勒倍泥部落於跋那山西,徙二萬餘家而旋。丙戌,車駕自大室西南巡諸部落,遂南次定襄大洛城,東踰七嶺山,田於善無川。八月癸卯,車駕還宮。癸丑,奚斤等班師。甲寅,帝臨白登山,觀降人,數軍實。置新人於大寧,給農器,計口受田。冬十一月癸酉,大饗於西宮。姚興遣使朝貢,
 請進女,帝許之。



 神瑞元年春正月辛酉,以禎瑞頻集,大赦改元。辛巳,行幸繁畤。賜王公以下至于士卒百工布帛各有差。二月戊戌,車駕還宮。乙卯,起豐宮於平城東北。夏六月,乞伏熾盤滅禿髮傉檀。秋七月,晉將朱齡石滅蜀。八月戊子,詔馬邑侯元陋孫使於姚興。姚興遣使朝貢。九月丁巳朔,日有蝕之。冬十一月壬午,詔使者巡行諸州,校閱守宰資財,非自家所齎,悉簿為贓。守宰不如法,聽百姓詣闕告之。十二月丙戌朔,蠕蠕犯塞。丙申,車駕北伐。



 二年春正月丙辰,車駕至自北伐。二月丁亥,大饗於西
 宮。甲辰,立宣武廟於白登西。三月丁丑,詔以刺史守宰率多逋惰,今年貲調縣違者,謫出家財以充,不聽徵發於人。夏四月,晉人來聘。己卯,北巡。五月丁亥,次於參合,東幸大寧。



 丁未,田于四岬山。六月戊午,臨去畿陂觀漁。辛酉,次于濡源,立蜯臺。遂射白熊於頹牛山,獲之。丁卯,幸赤城,親見長老,問人疾苦,復租一年。南次石亭,幸上谷,問百年,訪賢俊,復田租之半。壬申,幸涿鹿,登嶠山,觀溫泉,使以太牢祠黃帝、唐堯廟。癸酉,幸廣寧,事如上谷。己卯,登廣寧之歷山,以太牢祠舜廟,帝親加禮焉。庚辰,幸代。秋七月癸未,車駕還宮,復所過田租之半。八月庚
 辰晦,日有蝕之。九月,京師人飢,聽就食山東。冬十月壬子,姚興使奉其西平公主至,帝以后禮納之。辛酉,行幸沮洳城。癸亥,車駕還宮。丙寅,詔以頻遇霜旱,年穀不登,命出布帛倉穀以振貧窮。



 泰常元年春二月丁未,姚興死。三月己丑,長樂王處文薨。夏四月壬子,大赦改元。庚申,河間王脩薨。五月甲申,彗星二見。六月丁巳,北巡。秋七月甲申,大獮于牛川,登釜山,臨殷繁水,南觀于九十九泉。戊戌,車駕還宮。辛亥晦,日有蝕之。九月,晉劉裕溯河伐姚泓,遣部將王仲德從陸道至梁城。兗州刺史尉建畏懦,棄守北渡,仲德遂
 入滑臺。詔將軍叔孫建等度河曜威,斬尉建於城下。冬十一月戊寅,起蓬臺于北苑。十二月,南陽王良薨。



 二年春正月甲戌朔,日有蝕之。二月丙午,詔使者巡行天下,觀風俗,問其所苦。是月,涼武昭王薨。五月,西巡至雲中,遂濟河,田於大漠。秋七月乙亥,車駕還宮。乙酉,起白臺於城南,高二十丈。是月,晉劉裕滅姚泓。冬十月癸丑,豫章王夔薨。十二月己酉,詔河東、河內購泓子弟播越人間者。



 三年春三月,晉人來聘。庚戌,幸西宮。以勃海、范陽郡去年水,復其租稅。



 夏四月己巳,徙冀、定、幽三州徒何於京
 師。五月壬子,東巡至濡源,及甘松。遣征東將軍長孫道生帥師襲馮跋,遂至龍城,徙其居人萬餘家而還。秋七月戊午,車駕至京師。八月,鴈門、河內大雨水,復其租稅。冬十月戊辰,築宮於西苑。十一月,赫連屈丐剋長安。十二月,晉安帝殂。



 四年春正月壬辰朔,車駕臨河,大蒐於犢渚。癸卯,還宮。三月,赫連屈丐僭即皇帝位。癸丑,築宮於蓬臺北。夏四月庚辰,享東廟,遠蕃助祭者數百國。辛巳,南巡,幸鴈門,賜所過無出今年租賦。五月庚寅朔,觀漁於水壘水。己亥,車駕還宮。秋八月辛未,東巡,遣使祠恆岳。甲申,車駕還
 宮,賜所過無出今年田租。九月甲寅,築宮於白登山。冬十一月丁亥朔,日有蝕之。十二月癸亥,西巡,至雲中,踰白道,北獵野馬於辱孤山。至於黃河,從君子津西度,大狩於薛林山。



 五年春正月丙戌朔,自薛林東還。至屋竇城,饗勞將士,大酺二日,班禽以賜之。己亥,車駕還宮。三月丙戌,南陽王意文薨。夏四月丙寅,起水壘南宮。五月乙酉,詔曰:「宣武皇帝體得一之玄遠,應自然之沖妙,大行大名,未盡盛美。今啟緯圖,始睹尊號,其更上尊謚曰道武皇帝,以章靈命之先啟,聖德之玄同。」庚戌,淮南侯司馬國璠、池陽
 侯司馬道賜等謀反,伏誅。六月丙寅,幸翳犢山。是月,晉恭帝禪位于宋。秋七月丁酉,西至五原。丁未,幸雲中大室,賜從者大酺。八月癸亥,車駕還宮。閏月甲午,陰平王烈薨。是歲,西涼亡。



 六年春二月己亥,詔天下戶二十輸戎馬一匹、大牛一頭。三月甲子,陽平王熙薨。乙亥,制六部人羊滿百口者調戎馬一匹。發京師六千餘人築苑,起自舊苑,東苞白登,周回四十餘里。夏六月乙酉,北巡,至于蟠羊山。秋七月乙卯,車駕還宮。



 癸酉,西巡。獵于祚山,親射猛獸,獲之。遂至于河。八月庚子,大獼于犢渚。九月庚戌,車駕還宮。
 壬申,宋人來聘。冬十月己亥,行幸代。十二月丙申,西巡於雲中。



 七年春正月甲辰朔,自雲中西幸屋竇城,賜從者大酺三日。二月丙戌,車駕還宮。三月乙丑,河南王曜薨。夏四月甲戌,封皇子燾為太平王,拜相國,加大將軍;丕為樂平王,加車騎大將軍;彌為安定王,加衛大將軍;範為樂安王,加中軍大將軍;健為永昌王,加撫軍大將軍;崇為建寧王,加輔國大將軍;俊為新興王,加鎮軍大將軍;獻懷長公主子嵇敬為長樂王,拜大司馬、大將軍。初,帝服寒食散,頻年發動,不堪萬機,五月,立太平王燾為皇太
 子,臨朝聽政。是月,宋武帝殂。秋九月,詔司空奚斤等帥師伐宋。乙巳,幸水壘南宮,遂如廣寧。己酉,詔皇太子率百國以法駕田於東苑,車乘服物皆以乘輿之副。辛亥,築平城外郭,周回三十二里。



 辛酉,幸嶠山,遣使者祠黃帝、唐堯廟。因東幸幽州,見耆年,問其所苦,賜以爵號。分遣使者巡行州郡,觀察風俗。冬十月甲戌,車駕還宮,復所過田租之半。奚斤等濟河,攻滑臺不拔,求濟師,帝怒不許。議親南征,為其聲援。壬辰,南巡,出自天門關,踰恒嶺,四方蕃附大人各帥所部從者五萬餘人。十一月,皇太子親統六軍鎮塞上,安定王彌與北新公安同居守。丙
 午,曲赦司州殊死以下。丙辰,次于中山,問人疾苦。十二月丙戌,行幸冀州,存問人俗。遣壽光侯叔孫達等率眾自平原東度,徇下青、兗諸郡。



 八年春正月丙辰,行幸鄴,存問人俗。司空奚斤既平兗、豫,還圍武牢,宋守將毛德祖距守不下。蠕蠕犯塞。二月戊辰,築長城於長川之南,起自赤城,西至五原,延袤二千餘里,備置戍衛。三月乙卯,濟自靈昌。夏四月丁卯,幸成皋,觀武牢。而城內乏水,縣綆汲河。帝令連艦,上施轒巉,絕其汲路;又穿地道,以奪其井。丁丑,幸洛陽,觀石經。閏月丁未,還幸河內,北登太行,幸高都。己未,武牢潰。士
 眾大疫,死者十二三。辛酉,幸晉陽,班賜王公以下至於廝役。五月丙寅,還次鴈門,皇太子率留臺王公迎於句注之北。庚寅,車駕至自南巡。六月己亥,太尉、宜都公穆觀薨。丙辰,北巡,至參合陂。秋七月,幸三會屋侯泉,詔皇太子率百官以從。八月,幸馬邑,觀于水壘源。九月乙亥,車駕還宮。冬十月癸卯,廣西宮,起外墻,周回二十里。是歲饑,詔所在開倉振給。十一月己巳,帝崩於西宮,時年三十二。遺詔以司空奚斤所獲軍實賜大臣自司徒長孫嵩以下,至于士卒各有差。



 十二月庚子,上謚曰明元皇帝,葬于雲中金陵,廟稱太宗。



 帝兼資文武,禮愛儒生,好
 覽史傳,以劉向所撰《新序》、《說苑》於經典正義多有所闕,乃撰《新集》三十篇,採諸經史,該洽古義云。



 論曰:自古帝王之興,誠有天命,亦賴累功積德,方契靈心。有魏奄宅幽方,代為君長。神元生自天女,桓、穆勤於晉室,冥符人事,夫豈徒然!



 昭成以雄傑之姿,苞君人之量,征伐四剋,威被遐荒。乃改都立號,恢隆大業,終百六十載,光宅區中,其原固有由矣。



 道武顯晦安危之中,屈申潛躍之際。驅率遺黎,奮其靈武。克翦方難,遂啟中原。垂拱人神,顯登皇極。雖冠履不暇,棲遑外土,而制作經謨,咸出長久,所謂大人利見,百姓與能,抑不世之神武
 也。而屯厄有期,禍生非慮,將人事不足,豈天實為之乎?



 明元承運之初,屬廓定之始,於時狼顧鴟峙,猶有窺覦。已加以天賜之末,內難尤甚。帝孝心睿略,權正兼運,纂業固基,內和外撫,終能周、鄭款服,聲教南被,祖功宗德,其義良已遠矣!



\end{pinyinscope}