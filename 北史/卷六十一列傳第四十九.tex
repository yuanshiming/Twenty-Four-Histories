\article{卷六十一列傳第四十九}

\begin{pinyinscope}

 王盟子勱
 從孫誼
 獨孤信子羅竇熾兄子榮定毅賀蘭祥叱列伏龜閻慶子毗史寧子雄祥權景宣王盟,字仵,明德皇后之兄也,其先樂浪人。六世祖波,前燕太宰。祖珍,魏黃門侍郎,贈并州刺史、樂浪公。父羆,伏波將軍,以良家子鎮武川,因家焉。魏正光中,破六韓拔
 陵攻陷諸鎮,盟亦為其所擁。拔陵平後,流寓中山,復以積射將軍從蕭寶夤西征。寶夤僭逆,盟遂逃匿人間。及爾硃天光入關,盟從之。隋賀拔岳禽萬俊志願奴,平秦隴,常先登力戰。及周文帝平侯莫陳悅,除盟原州刺史。孝武至長安,封魏昌縣公。大統三年,徵拜司空,轉司徒。迎文帝悼后於蠕蠕,加侍中,遷太尉。魏文帝東征,以留後大都督行雍州事,節度關中諸軍。趙青雀之亂,盟與開府李虎輔太子出頓渭北。事平,進長樂郡公,賜姓拓王氏。遷太保。九年,進位太傅,加開府儀同三司。盟姿度弘雅,仁而汎愛。雖居師傅,禮冠群后,而謙恭自處,未嘗以
 勢位驕人。魏文帝甚尊重之,及疾,數幸其第,親問所欲。十一年,薨,贈本官,謚曰孝定。



 子勱,字醜興,性忠果有材幹。年十七,從周文帝入關。及平秦隴,定關中,周文嘗謂曰:「為將坐見成敗者上也,被堅執銳者次也。」勱曰:「意欲兼被之。」



 周文大笑。尋拜散騎常侍,賜爵梁甫縣公。大統初,為千牛備身直長,領左右,出入臥內,小心謹厚。魏文帝常曰:「王勱可謂不二心臣也。」沙苑之役,勱以都督領禁兵,居左翼,當其前者死傷甚眾。勱亦被傷重,遂卒於行間。周文深悼焉。贈使持節、太尉、尚書令、十州諸軍事、雍州刺史,追封咸陽郡公,謚
 曰忠武。



 子弼襲爵,尚魏安樂公主,位大都督、通直散騎常侍。



 勱弟懋,字小興。盟之西征也,以懋尚幼,留在山東。永安中,始入關,與盟相見,遂從征伐。大統初,賜爵安平縣子。後進爵為公,累遷右衛將軍。于時疆場交兵,未申喪紀,服齊斬者並墨縗從事。及盟薨,懋上表辭位,乞終喪制,魏文帝不許。累遷開府儀同三司、侍中、左衛將軍、領軍將軍。懋溫和,小心敬慎,宿衛宮禁十有餘年,勤恪當官,未嘗有過。廢帝二年,除南岐州刺史,賜爵安寧郡公。



 後拜小司寇,卒於官。



 子悅嗣,位大將軍、同州刺史,改封濟南郡公。



 盟兄子顯,幼而敏悟,沉靜少言。初為周文
 帳內都督,累遷驃騎大將軍、開府儀同三司、光祿卿、鳳州刺史。賜爵洛邑縣公,進位大將軍,卒。子誼。



 誼字宜君,少有大志,便弓馬,博覽群言。周閔帝時,為左中侍上士。時大冢宰宇文護執政,帝拱默無所關預。有朝士於帝側微不恭,誼勃然而進,將擊之,其人惶懼請罪,乃止。自是朝臣無敢不肅。遷御正大夫。丁父艱,毀瘁過禮,廬於墓側,負土成墳。



 武帝即位,累遷內史大夫,封揚國公。從帝伐齊,至並州。帝既入城,反為齊人所敗,左右多死,誼率麾下驍雄赴之。齊平,自相州刺史徵為大內史。汾州稽胡亂,誼擊之。帝弟越王盛、譙王儉雖為總管,
 並受誼節度。賊平,封一子開國公。



 帝臨崩,謂皇太子曰:「王誼社稷臣,宜處以機密,不須遠任。」皇太子即位,為宣帝,憚誼剛正,出為襄州總管。



 及隋文帝為丞相,鄖州總管司馬消難舉兵反,帝以誼為行軍元帥討之,未至而消難奔陳。于時北至商、洛,南拒江、淮,東西二千餘里,巴蠻多叛,共推渠帥蘭洛州為主。洛州自號河南王以附消難,北連尉遲迥。誼分兵討之,旬月皆平。帝遣使勞問,冠蓋不絕,以第五女妻其子奉孝。尋拜大司徒。誼自以與帝有舊,亦歸心焉。及隋受禪,顧遇彌厚,帝親幸其第,與之極歡。



 太常卿蘇威議,以為戶口滋多,人田不贍,
 欲減功臣之地以給人。誼奏曰:「百官者,歷世勛賢,方蒙爵土,一旦削之,未見其可。」帝以為然,竟寢威議。



 帝將幸岐州,諫曰:「陛下初臨萬國,人情未洽,何用此行。」上戲之曰:「吾昔與公位望齊等,一朝屈節為臣,或當恥愧,是行也,振揚威武,欲以服公心耳。」



 誼笑而退。尋奉使突厥。帝嘉其稱旨,進郢國公。



 未幾,其子奉孝卒。踰年,誼上表言公主少,請除服。御史大夫楊素劾誼曰:「臣聞喪服有五,親疏異節;喪制有四,降殺殊文。王者之所常行,故曰不易之道也。而儀同王奉孝既尚蘭陵公主,以去年五月身喪,始經一周,而誼便請除釋。竊以雖曰王姬,終成
 下嫁之禮;公則主之,猶在移天之義。況復三年之喪,自上達下,及期釋服,在禮未詳。然夫婦之則,人倫攸始,喪紀之制,人道至大,茍不重之,取笑君子。故鑽燧改火,責以居喪之速;朝祥暮歌,譏以忘哀之早。然誼雖不自彊,爵位已重,欲為無禮,其可得乎?乃薄俗傷教,為父則不慈;輕禮易喪,致婦於無義。若縱而不正,恐傷風俗。」有詔不問。然恩禮稍薄,誼頗怨望。



 或告誼謀反,帝令案其事。主者奏誼有不遜之言,實無反狀。帝賜酒而釋之。



 時上柱國元諧亦頗失意,誼數與往來,言論醜惡。胡僧告之。公卿奏誼大逆不道,罪當死。帝見誼,愴然曰:「朕與公舊
 同學,甚相憐憫,將奈國法何。」於是詔曰:「誼有周之世,早預人倫,朕共遊庠序,遂相親好。然性懷險薄,巫覡盈門,鬼言怪語,稱神道。朕受命之初,深存戒約,口云改悔,心實不悛。乃說四天王神道,誼應受命,書有誼讖,天有誼星,桃、鹿二川,岐州之下,歲在辰巳,興帝王之業。



 密令卜問,伺殿省之災。又說其身是明王聖主。信用左道,所在詿誤。自言相表,當王不疑。此而赦之,將或為亂。禁暴除惡,宜伏國刑。」帝復令大理正趙綽謂誼曰:「時命如此,將若之何!」乃賜死於家,時年四十六。



 獨孤信,雲中人也,本名如願。魏初有四十六部,其先伏
 留屯者為部落大人,與魏俱起。祖俟尼,和平中,以良家子自雲中鎮武川,因家焉。父庫者,為領人酋長,少雄豪有節義,北州咸敬服之。信美容儀,善騎射。正光末,與賀拔度等同斬衛可瑰,由是知名。後為葛榮所獲。信既少年,自脩飾服章,軍中號為獨孤郎。及爾朱氏破葛榮,以信為別將。從征韓婁,信匹馬挑戰,禽賊漁陽王袁肆周。後以破元顥黨,賜爵受德縣侯,遷武衛將軍。賀拔勝出鎮荊州,乃表信為大都督。及勝弟岳為侯莫陳悅所害,勝乃令信入關,撫岳餘眾。屬周文帝已統岳兵,與信鄉里,少相友善,相見甚歡,因令信人洛請事。至雍州,大使
 元毗又遣信還荊州。尋徵入朝,魏孝武雅相委任。



 及孝武西遷,事起倉卒,信單騎及之於瀍澗。孝武嘆曰:「武衛遂能辭父母,捐妻子從我,世亂識忠良,豈虛言哉!」進爵浮陽郡公。時荊州雖隱東魏,人心猶戀本朝,乃以信為衛大將軍、都督三荊州諸軍事,兼尚書右僕射、東南道行臺、大都督、荊州刺史,以招懷之。既至,東魏刺史辛纂出戰,信縱兵擊纂,大敗之。都督楊忠等前驅斬纂,於是三荊遂定。



 東魏又遣其將高敖曹、侯景等奄至。信以眾寡不敵,遂率麾下奔梁。居三載,梁武帝方許信還北。信父母既在山東,梁武帝問信所往,答以事君無二。梁武
 義之,禮送甚厚。大統三年至長安,以虧損國威,上書謝罪。魏文帝付尚書議之。七兵尚書、陳郡王玄等議,以為既經恩降,請赦罪復職。詔轉驃騎大將軍,加侍中、開府。



 尋拜領軍將軍。仍從復弘農,破沙苑,改封河內郡公。俘虜中有信親屬,始得父凶問,乃發喪行服。尋起為大都督,與馮翊王元季海入洛陽,潁、豫、襄、廣、陳留之地並款附。四年,東魏將侯景等圍洛陽,信據金墉城,隨方拒守然有餘日。及周文帝至瀍東,景等退走。信與李遠為右軍,戰不處,東魏遂有洛陽。六年,侯景寇荊州,周文令信與李弼出武關,景退。即以信為大使,尉撫三荊。尋除隴
 右十一州大都督、秦州刺史。先是守宰闇弱,政令乖方,人有冤訟,歷年不能斷決。及信在州,事無擁滯。示以禮教,勸以耕桑,數年之中,公私富實,流人願附者數萬家。



 周文以其信著遐邇,故賜名為信。七年,岷州刺史赤水蕃王梁簋定舉兵反,詔信討之。簋定尋為其部下所殺,而簋定子弟仍收其餘眾。信乃勒兵向萬年,頓三交谷口。賊併力拒守。信因詭道趣稠松嶺。賊不虞信兵之至,望風奔潰。乘勝逐北,徑至城下,賊並出降。加授太子太保。



 芒山之戰,大軍不利。信與於謹帥散卒自後擊之,齊神武追騎驚擾,國因此得全。及涼州刺史宇文仲和
 據州不受代,周文令信率開府怡峰討之。仲和嬰城固守,信夜令諸將以衝梯攻其東北,信親率壯士襲其西南,達明克之。禽仲和,虜其六千戶送于長安。拜大司馬。十三年,大軍南討。時以蠕蠕為寇,令信移鎮河陽。十四年,進位柱國大將軍,錄前後功,增封,聽回授諸子。於是第二子善,封魏寧縣公;第三子穆,必要縣侯;第四子藏,義寧縣侯,邑各一千戶。第五子順,武成縣侯;第六子陀,建忠縣伯,邑各五百戶。信在隴右歲久,啟求還朝,周文不許。或有自東魏來者,又告其母凶問,信發喪行服。信陳哀苦,請終禮制,又不許。於是追贈信父庫者司空公,
 追封信母費連氏常山郡君。十六年,遷尚書令。六官建,拜大司馬。



 周孝閔帝踐阼,遷大宗伯,進封衛國公,邑萬戶。趙貴誅後,信以同謀坐免。



 居無幾,晉公護又欲殺之,以其名望素重,不欲顯其罪過,逼令自盡於家,時年五十五。



 信美風度,雅有奇謀大略。周文初啟霸業,唯有關中之地,以隴右形勝,故委信鎮之。既為百姓所懷,聲震鄰國。東魏將侯景之南奔梁也,魏收為檄梁文,矯稱信據隴右,不從宇文氏,乃云「無關西之憂」,欲以委梁人也。又信在秦州,嘗因獵日暮,馳馬入城,其帽微側,詰旦而吏人有戴帽者,咸慕信而側帽焉。其為鄰境及士庶所
 重如此。



 子羅,先在東魏,乃以次子善為嗣。及齊平,羅至而善卒,又以羅主嗣。信長女周明敬后,第四女元貞后,第七女隋文獻后。周、隋及皇家三代皆為外戚,自古以來,未之有也。隋文帝踐極,乃下詔褒贈信太師、上柱國、十州諸軍事、冀州刺史,封趙國公,邑一萬戶,謚曰恭,信母費連氏贈太尉趙恭公夫人。



 羅,字羅仁。父信隨魏孝武入關中,羅遂為高氏所囚。及信為宇文護誅,羅始見釋。寓居中山,孤貧無以自給。齊
 將獨孤永業以宗族故,哀之,為買田宅,遺以資畜。



 初,信入關後,復娶二妻。郭氏生子六人,善、穆、藏、順、陀、整;崔氏生隋獻皇后。及齊亡,隋文帝為定州總管,獻皇后遣人求羅,得之。相見悲不自勝,侍御者皆泣。於是厚遺車馬財物。未幾,周武帝以羅功臣子,久淪異域,徵拜楚安郡太守。以疾去官,歸京師。諸弟見羅少長貧賤,每輕侮,不以兄禮事之。然性長者,亦不與諸弟校競長短。后由是重之。



 文帝為丞相,拜羅儀同,常置左右。既受禪,詔追贈羅父。其諸弟以羅母沒齊,先無夫人號,不當承襲。上以問后,后曰:「羅誠嫡長,不可誣也。」於是襲爵趙國公。以其
 弟善為河內郡公,穆為金泉縣公,藏為武平縣公,陀為武喜縣公,整為千牛備身。擢拜羅為左領左右將軍,遷左衛將軍,前後賞賜不可勝計。出為涼州總管,進位上柱國,徵拜左武衛大將軍。煬帝嗣位,改封蜀國公。未幾卒官,謚曰恭。



 子纂嗣,位河陽都尉。



 纂弟武都,大業末,亦為河陽都尉。



 庶長子開遠。宇文化及之弒逆也,裴虔通率賊入成象殿,宿衛兵士皆從逆。開遠時為千牛,與獨孤盛力戰合下,為賊所執,賊義而捨之。



 善字伏陀。幼聰慧,善騎射,以父勳,封魏寧縣公。魏廢帝元年,又以父勛,授驃騎大將軍、開府儀同三司,加侍
 中,進爵長城郡公。周孝閔帝踐阼,除河州刺史。以父負釁,久廢於家。保定三年,乃授龍州刺史。天和六年,襲爵河內郡公。



 從帝東討,以功授上開府。尋除兗州刺史,政在簡惠,百姓安之。卒於州,贈持節、柱國、五州諸軍事、定州刺史。



 子覽嗣,位右候衛大將軍。大業末卒。



 陀字黎邪。仕周,胥附上士。坐父徙蜀十餘年,宇文護誅,始歸長安。隋文帝禪,拜上開府、領左右將軍,累轉延州刺史。



 陀性好左道,其外祖母高氏先事貓鬼,已殺其舅郭沙羅,因轉入其家。上微聞而不信。會獻皇后及楊素妻鄭氏俱有疾,召醫視之,皆曰:「此貓鬼疾。」上以陀,后
 之異母弟,陀妻,楊素之異母妹,由是意陀所為。陰令其兄左監門郎將穆以情喻之,上又避左右諷陀,陀言無有。上不說,左轉遷州刺史。出怨言,上令左僕射高熲、納言蘇威、大理正皇甫孝緒、大理丞楊遠等雜案之。陀婢徐阿尼言:本從陀母家來,常事貓鬼,每以子日夜祀之。言子者鼠也。其貓鬼每殺人者,所死家財物潛移於畜貓鬼家。陀嘗從家中索酒,其妻曰:「無錢可酤。」陀因謂阿尼曰:「可令貓鬼向越公家,使我足錢。」阿尼便咒之,居數日,貓鬼向素家。後上初從並州還,陀於園中謂阿尼曰:「可令貓鬼向皇后所,使多賜吾物。」阿尼復咒之,遂入宮
 中。



 楊遠乃於門下外省遣阿尼呼貓鬼,阿尼於是夜中置香粥一盆,以匙扣而呼曰:「貓女可來,無住宮中。」久之,阿尼色正青,若被牽拽者,云貓鬼已到。上以其事下公卿。奇章公牛弘曰:「妖由人興,殺其人,可以絕矣。」上令犢車載陀夫妻,將賜死於其家。陀弟司勛侍中整詣闕求哀,於是免陀死,除名,以其妻楊氏為尼。先是有人訟其母為人貓鬼所殺者,上以為妖妄,怒而遣之。及此,詔誅被訟行貓鬼家。



 陀未幾而卒,煬帝即位,追念舅氏,聽以禮葬。乃下詔贈正義大夫。帝意猶不已,復贈銀青光祿大夫。二子,延福、延壽。



 陀弟整,位幽州刺史。大業初,贈金
 紫光祿大夫、平鄉侯。



 竇熾,字光成,扶風平陵人,後漢大鴻臚章之後也。章子統,靈帝時為鴈門太守,避竇武之難,亡奔匈奴,遂為部落大人。後魏南徙,子孫因家代,賜姓紇豆陵氏。累世仕魏,皆至大官。父略,平遠將軍,以熾著勛,贈少保、住國大將軍、建昌公。熾性嚴明,有謀略,美鬚髯,身長八尺二寸。少從范陽祁忻受《毛詩》、《左氏春秋》,略通大義。善騎射,膂力過人。魏正光末,北鎮擾亂,乃隨略避地定州,投葛榮。榮欲官略,略不受。榮疑其有異志,遂留略於冀州,將熾及熾兄善隨軍。及爾朱榮破葛榮,熾乃將家隨榮於并
 州。時葛榮別帥韓婁等據薊城不下,以熾為都督,從驃騎將軍侯深討之。熾手斬婁,以功拜揚烈將軍。



 魏孝武即位,蠕蠕等諸蕃並遣使朝貢,帝臨軒宴之。有鴟飛鳴於殿前,帝素知熾善射,固欲矜示遠人,乃給熾御箭兩隻,命射之,鴟乃應弦而落,諸蕃人咸歎異焉。帝大悅。尋隨東南道行臺樊子鵠追爾朱仲遠,仲遠奔梁。時梁主又遣元樹入寇,據譙城。子鵠令熾擊破之,封行唐縣子,尋進爵上洛縣伯。時帝與齊神武構隙,以熾有威重,堪處爪牙任,拜閣內大都督,遷朱衣直閣,遂從帝西遷。仍與其兄善至城下,與武衛將軍高金龍戰於千秋門,敗之。因
 入宮城,取御馬四十匹并鞍勒,進之行所。帝大悅。賜熾及善駿馬各二匹,駑馬十匹。



 大統元年,別封真定縣公。從周文帝禽竇泰,復弘農,破沙苑,皆有功。河橋之戰,諸將退走,熾時獨從兩騎,為敵人追至芒山。熾乃下馬,背山抗之。俄而敵眾漸多,矢下如雨,熾騎士所執弓,並為敵人所射破。熾乃總收其箭以射之,所中人馬,應弦而倒。敵乃相謂曰:「得此三人,未足為功。」乃稍引退。熾因其怠,遂突圍得出。又從太保李弼討白額稽胡,破之。



 高仲密以北豫州來,熾從周文援之。至洛陽,會東魏人據芒山為陣,周文命留輜於瀍曲,率輕騎奮擊,中軍與
 右軍大破之,悉虜其步卒。熾獨追至石濟而還。大統十三年,進使持節、驃騎大將軍、開府儀同三司,加侍中。出為涇州刺史,蒞職數年,政號清靜。改封安武縣公。



 魏廢帝元年,除原州刺史。熾抑挫豪右,申理幽滯,在州十載,甚有政績。州城北有泉水,熾屢經游踐,嘗與僚吏宴於泉側,因酌水自飲,曰:「吾在此州,唯當飲水而已。」及去職後,人吏感其遺惠,每至此泉者,莫不懷之。恭帝元年,進爵廣武郡公。屬蠕蠕寇廣武,熾與柱國趙貴分路討之。蠕蠕引退,熾度河至麴伏川追及,大破之。武成二年,拜柱國大將軍。周明帝以熾前朝舊臣,勛望兼重,欲獨為
 造第。熾辭以天下未平,干戈未偃,不宜輒發徒役,周明不許。尋而帝崩,事方得寢。



 保定元年,進封鄧國公,邑一萬戶,別食資陽縣一千戶,收其租賦。天和五年,自大宗伯為宜州刺史。先是周文田於渭北,令熾與晉公護分射走兔,熾一日獲十七頭,護十一頭。護恥不及,因以為嫌。至是,熾又以周武年長,有勸護歸政之議,護惡之,故左遷焉。及護誅,徵拜太傅。



 熾既朝之元老,名望素隆,至於軍國大謀,常與參議。嘗有疾,周武帝幸其第問之,因賜金石之樂。其見禮如此。帝於大德殿將謀伐齊,熾年已衰老,乃扼腕曰:「臣雖朽邁,請執干櫓,首啟戎行。得一
 睹誅翦鯨鯢,廓清寰宇,省方觀俗,登岳告成,然後歸魂泉壤,無復餘恨。」帝壯其志節,遂以熾第二子武當公恭為左二軍總管。齊平之後,帝乃召熾歷觀相州宮殿。熾拜賀曰:「陛下真不負先帝矣。」帝大悅,進位上柱國。



 宣政元年,兼雍州牧。及周宣營建東京,以熾為京洛營作大監,宮苑制度,皆取決焉。大象初,改食樂陵縣,邑戶如舊。隋文帝入輔政,停洛陽宮作,熾請入朝。



 屬尉遲迥舉兵,熾乃移入金墉,與洛州刺史、平涼公元亨同心固守,仍權行洛陽鎮事。相州平,熾方入朝。屬文帝初為相國,百僚皆勸進,自以累世受恩,遂不肯署箋,時人綿高其節。
 及帝踐極,拜太傅,加殊禮,贊拜不名。開皇四年八月薨,時年七十八。贈八州諸軍事、冀州刺史,謚曰恭。



 熾事親孝,奉諸兄以悌順聞。及其望位隆重,而子孫皆處列位,遂為當時盛族。



 子茂嗣。茂有弟十三人,恭、威最知名。



 恭位至大將軍。從周武平齊,封贊國公,除西兗州總管,以罪賜死。



 熾兄善,以中軍大都督、南城公從魏孝武西遷,仕至太僕、衛尉卿、汾北華瀛三州刺史、驃騎大將軍、開府儀同三司、永富縣公,謚曰忠。子榮定嗣。



 榮定沉深有器局,容貌魁偉,美鬚髯,便弓馬。初為魏文帝千牛備身,周文帝見而奇之,授平東將軍,賜爵宜君
 縣子。後從周文與齊人戰於北芒,周師不利,榮定與汝南公宇文神慶帥精騎擊卻齊師。以功拜上儀同。尋復以軍功進位開府。襲爵永富縣公,除忠州刺史。從平齊,加上開府,拜前將軍、佽飛中大夫。



 其妻,則隋文帝長姊安成長公主也,文帝少與之情契甚厚。榮定亦知帝有人君之表,尤相推結。及帝作相,領左右宮伯,使鎮守天臺,總統露門內兩廂仗衛,常宿禁中。遇尉遲迥初平,朝廷頗以山東為意,拜榮定為洛州總管以鎮之。前後賜縑四千匹、西涼女樂一部。及受禪,來朝,賜馬三百匹、部曲八十戶遣之。坐事除名。



 公主曰:「天子姊乃作田舍兒
 妻!」上不得已,尋拜右武候大將軍。上數幸其第,恩錫甚厚,每令尚食局日供羊一口,珍味稱是。以佐命功,拜上柱國。



 歷位寧州刺史、右武候大將軍、秦州總管,賜吳樂一部。突厥沙缽略寇邊,為行軍元帥,率總管出涼州。與虜戰於高越原,兩軍相持,地無水,士卒渴甚,至刺馬血而飲,死者十二三。榮定仰天太息,俄而澍雨,軍復振。於是進擊,數挫其鋒,突厥憚之,請盟而去。賜縑萬匹,進爵安豐郡公,復封子憲為安康郡公,賜縑五千匹。歲餘,拜右武衛大將軍。帝欲以為三公,榮定上書固辭,陳畏懼之道,帝乃止。



 前後賞賜不可勝計。及卒,帝為之廢朝,令
 左衛大將軍元旻監護喪事,賻絹三千匹。



 上謂侍臣曰:「吾每欲致榮定於三事,其人固讓不可。今欲賜之,重違其志。」於是贈冀州刺史、陳國公,謚曰懿。子抗嗣。



 抗美容儀,性通率,長於巧思。父卒後,恩遇彌厚,所賜錢帛金寶亦以鉅萬。



 位定州刺史,檢校幽州總管。煬帝即位,漢王諒反,以為抗與通謀,由是除名,以其弟慶襲封陳公。



 慶亦有姿容,性和厚,頗工草隸。初封永富郡公,位河東太守、衛尉卿。大業末,為南郡太守,為盜賊所害。



 慶弟璡,亦工草隸,頗解鍾律。歷位潁川、南郡、扶風太守。



 熾兄子毅。毅字天武。父岳早卒,及毅著勛,追贈大將軍、冀州刺史。
 毅深沉有器度,事親以孝聞。魏孝武初,起家員外散騎侍郎。時齊神武擅朝,毅慨然有徇主之志。從孝武西遷,封奉高縣子。從禽竇泰,復弘農,戰沙苑,皆有功,進爵安武縣公。恭帝元年,進授驃騎大將軍、開府儀同三司、大都督,改封永安縣公。出為幽州刺史。周孝閔帝踐阼,進爵神武郡公。保定三年,拜大將軍。



 時與齊人爭衡,戎車歲動,並交結突厥以為外援。突厥已許納女於周,齊人亦甘言重幣,遣使求婚,狄人便欲有悔。朝廷乃令楊薦等累使結之,往返十餘,方復前好。至是雖期往逆,猶懼改圖。以毅地兼勳戚,素以威重,乃令為使。乃毅至,齊使
 亦在焉,突厥君臣,猶有貳志。毅抗言正色,以大義責之,累旬乃定,卒以皇后歸。朝議嘉之,別封成都縣公,進位柱國。歷同州刺史、蒲金二州總管,加上柱國,入為大司馬。隋開皇初,拜定州總管。累居籓鎮,咸得人和。二年,薨於州,贈襄、郢等六州刺史,謚曰肅。



 毅性溫和,每以謹慎自守,又尚周文帝第五女襄陽公主,特為朝廷所委信,雖任兼出內,未嘗有矜惰之容,時人以此稱焉。子賢嗣。



 賢字託賢,志業通敏,少知名。宣政元年,授使持節、儀同大將軍。開皇中,襲爵神武公,除遷州刺史。



 毅第二女即大唐太穆皇后。武德元年,詔贈毅司空、使持節、總管荊
 郢等十州諸軍事、荊州刺史、杞國公。又追贈賢子紹宣秦州刺史,并襲賢爵。紹宣無子,仍以紹宣兄子德藏嗣。



 賀蘭祥,字盛樂,其先與魏俱起,有乞伏者,為賀蘭莫何弗,因以為氏。後有以良家子鎮武川者,遂家焉。父初真,少知名,為鄉閭所重,尚文帝姊建安長公主。



 保定二年,追贈太傅、柱國、常山郡公。祥年十一而孤,居喪合禮。長於舅氏,特為周文帝所愛,雖在戎旅,常博延儒生,教以書傳。周文初入關,祥與晉公護俱在晉陽,後乃遣使迎致之。解褐奉朝請。少有膽氣,志在立功。尋擢補都督,恆居帳下。從平侯莫陳悅,又迎魏孝武,以前後功封撫夷
 縣伯。仍從擊潼關,獲東魏將薛長儒,又攻回洛拔之。還拜左右直長,進爵為公。



 大統九年,從周文與東魏戰於芒山,進位驃騎大將軍、開府儀同三司,加侍中。



 十四年,除都督、荊州刺史,進爵博陵郡公。先是祥嘗行荊州事,雖未期月,頗有惠政,至是重往,百姓安之。由是漢南流人襁負至者,日有千數,還近蠻夷莫不款附。祥隨機撫納,咸得其歡心。時盛夏亢陽,祥親巡境內,觀政得失,見有發掘古冢,暴露骸骨,乃謂守令曰:「此豈仁者為政邪!」命所在收葬之。即日澍雨,是歲大有年。境內多古墓,其俗好行發掘,至是遂息。祥雖周文密親,性甚清素。州境
 南接襄陽,西通岷蜀,物產所出,多諸珍異。既與梁通好,行李往來,公私贈遺,一無所受。梁雍州刺史、岳陽王蕭詧欽其風素,乃以竹屏風、絺綌之屬及經史贈之。



 祥難違其意,取而付諸所司。周文後聞之,並以賜祥。十六年,拜大將軍。周文以涇、渭溉灌之處,渠堰廢毀,乃令祥修造富平堰,開渠引水,東注於洛。功用既畢,人獲其利。魏廢帝二年,行華州事,後改華州為同州,仍以祥為刺史。尋拜尚書左僕射。六官建,授小司馬。周孝閔帝踐阼,進位柱國、大司馬。時晉公護執政,祥與護中表,少相親愛,軍國之事,護皆與祥參謀。及誅趙貴,廢閔帝,祥有力焉。



 武成初,吐谷渾侵掠州郡,詔祥與宇文貴總兵討之。祥乃遣其軍司檄吐谷渾,與渾廣定王、鍾留王等戰,破之,因拔其洮陽、洪和二城,以其地為洮州。撫安西土,振旅而還。進封涼國公。薨,贈太師、同岐等十三州諸軍事、同州刺史,謚曰景。



 有七子,敬、讓、璨、師、寬知名。



 敬少歷顯職,封化隆縣侯,後襲爵涼國公。位柱國、華州刺史。



 讓,大將軍、鄭州刺史、河東郡公。



 璨,開府儀同三司、宣陽郡公。建德五年,從於并州戰歿,贈上儀同大將軍,追封清都公。



 師,尚明帝女,位上儀同大將軍、幽州刺史、博陵郡公。



 寬,開府儀同大將軍、武始公。入隋,歷汴、鄭二州刺史,並著
 政績。



 祥弟隆,大將軍、襄樂縣公。隋文帝與祥有舊,開皇初,追贈上柱國。



 叱列伏龜,字摩頭陀,代郡西部人也。其先為部落大人,魏初入附,遂世為第一領人酋長,至龜五世。龜容貌瑰偉,腰帶十圍,進止詳雅,兼有武藝。嗣父業復為領人酋長。魏孝昌三年,以別將從長孫承業西征,累遷金紫光祿大夫。從還洛,授都督,遂為齊神武所寵任,加授大都督。沙苑之敗,隨例來降。周文帝以其豪門,解縛禮之,仍以邵惠公女妻之。大統四年,封長樂縣公。自此常從征討,亟有戰功。



 歷侍中、驃騎大將軍、開府儀同三司、恆州
 刺史。卒,子椿嗣。



 椿字千年。明帝時,位驃騎大將軍、開府儀同三司,改封永世縣公。天和初,除左宮伯,進位大將軍。



 閻慶,字仁度,河陰人也。曾祖善,仕魏歷龍驤將軍、雲州鎮將,因家雲州之盛樂郡。祖提,持節、車騎大將軍、敦煌鎮都大將。父進,有謀略,勇冠當時。正光中,拜龍驤將軍。屬衛可瑰作亂,攻圍盛樂,進率眾拒守,以功拜盛樂郡守。慶幼聰敏,重然諾,風儀端肅,望之儼然。隨父因守盛樂,頗有力焉,拜別將。後以軍功拜步兵校尉、中堅將軍。既而齊神武舉兵入洛,魏孝武西遷,慶謂所親曰:「高歡
 將有篡逆之謀,豈可茍安目前,受其控制也?」遂以大統三年自宜陽歸闕。



 稍遷後將軍,封安次縣子,以功進爵為伯。慶善於綏撫,士卒未休,未嘗先舍,故能盡其死力,屢獲勳勞。累遷散騎常侍、驃騎大將軍、開府儀同三司、雲州大中正,加侍中,賜姓大野氏。周孝閔帝踐阼,出為河州刺史,進爵石保縣公。州居河外,地接戎夷,慶留心撫納,頗稱簡惠。就拜大將軍,進爵太安郡公。入為小司空,歷雲、寧二州刺史。慶性寬和,不苛察,百姓悅之。天和六年,進位柱國。



 晉公護母,慶之姑也。護雖擅朝,而慶未嘗阿附。及護誅,武帝以此重之。詔慶第十二子毗尚帝
 女清都公主。慶雖位望隆重,婚連帝室,常以廉慎自守,時以此稱之。建德二年,抗表致事,優詔許焉。慶既衰老,恆嬰沉痼。宣帝以其先朝耆舊,特異恆倫,乃詔靜帝至第問疾。賜布千段,醫藥所須,令有司供給。大象二年,拜上柱國。



 隋文帝踐極,又令皇太子就第問疾,仍供醫藥之費。開皇二年薨,年七十七。



 贈司空、七州諸軍事、荊州刺史,謚曰成。長子常,先慶卒。次子毗嗣。



 毗,七歲襲爵石保縣公。及長,儀貌矜嚴,頗好經史,受漢書於蕭該,略通大旨。能篆書,草隸尤善,為當時之妙。周武帝見而悅之,命尚清都公主。宣帝即位,拜儀同三司。



 隋文帝受禪,以技藝侍東宮。數以琱麗之物取悅於皇太子,由是甚見親待,每稱之於上。尋拜車騎,宿衛東宮。上嘗遣高熲大閱於龍臺澤,諸軍部伍多不齊整,唯毗一軍,法制肅然。熲言之於上,特蒙賜制。俄兼太子宗衛率長史,尋加上儀同。



 太子服玩之物多毗所為。及太子廢,毗坐杖一百,與妻子俱配為官奴婢。二歲放免。



 煬帝嗣位,盛修軍器,以毗性巧,練習舊事,詔典其職。尋授朝請郎。毗立議,輦輅車輿,多所增損。擢拜起部郎。



 帝嘗大備法駕,嫌屬車太多,顧謂毗曰:「開皇之日,屬車十二乘,於事亦得。



 今八十一乘,以牛駕車,不足以益文物,朕欲
 減之,從何為可?」毗曰:「臣初定數,共宇文愷參詳故實,據漢胡伯始、蔡邕等議,屬車八十一乘。此起於秦,遂為後式。故張衡賦云『屬車九九』是也。次及法駕,三分減一,為三十六乘,此漢制也。又據宋孝建時,有司奏議,晉遷江左,唯設五乘,尚書令建平王宏曰:「八十一乘,義兼六國,三十六乘,無所準憑,江左五乘,儉不中禮。但帝王文物旂旒之數,爰及冕玉,皆用十二,今宜準此,設十二乘。』開皇平陳,因以為法。今憲章往古,大駕依秦,法駕依漢,小駕依宋,以為差等。」帝曰:「何用秦法!大駕宜三十六,法駕宜十二,小駕除之。」毗研精故事,皆此類也。



 長城之役,毗總
 其事。及帝有事恆岳,詔毗營立壇場。尋轉殿內丞,從幸張掖郡。高昌王朝於行所,詔毗持節迎勞,遂將護入東都。尋以母憂去職,未期,起令視事。將興遼東之役,自洛口開渠達涿郡以通漕,毗督其役。明年,兼領右翊衛長史,營建臨朔宮。及征遼東,以本官領武賁郎將,典宿衛。時軍圍遼東城,帝令毗詣城下宣諭,賊弓弩亂發,流矢中所乘馬,毗顏色不變,辭氣抑揚,卒事而去。遷殿內少監,又領將作少監。後復從帝征遼東。會楊玄感作逆,帝班師,從至高陽郡,卒。帝甚悼惜之,贈殿內監。



 史寧,字永和,建康表氏人也。曾祖豫,仕沮渠氏為臨松
 令。魏平涼州,祖灌隨例遷於撫寧鎮,因家焉。父遵,初為征虜府鎧曹參軍。杜洛周構逆,六鎮自相屠隱,遵遂率鄉里奔恆州。其後恆州為賊所敗,遵後歸洛陽,拜樓煩郡守。及寧著勛,贈散騎常侍、征西大將軍、涼州刺史,謚曰貞。寧少以軍功,累加持節、征東將軍、金光祿大夫。賀拔勝為荊州刺史,寧以本官為勝軍司,隨勝之部。會荊蠻騷動,三鵶路絕。寧先驅平之,因撫慰蠻左,翕然降附。尋除南郢州刺史。及勝為大行臺,表寧為大都督。攻梁下溠戍破之,封武平縣伯。又攻拔梁齊興鎮等九城。未及論功,屬孝武西遷,東魏遣侯景寇荊州,寧隨勝奔梁。
 梁武帝引寧至香蹬前,謂之曰:「觀卿風表,終是富貴,我當使卿衣錦還鄉。」寧答曰:「臣世荷魏恩,位為列將,天長喪亂,本朝傾覆,不能北面事逆賊,幸得息肩有道。儻如明詔,欣幸實多。」



 因涕泣橫流,梁武為之動容。在梁二年,勝乃與寧密圖歸計。寧曰:「朱異既為梁主所信任,請往見之。」勝然其言。寧乃見異,申以投分之言,微託思歸之意,辭氣雅至。異亦嗟挹,為奏梁主,果許勝等歸。



 大統二年,自梁歸,進爵為侯。久之,遷車騎將軍,行涇州事。時賊帥莫折後熾寇掠居人,寧率州兵與行原州事李賢討破之。轉東義州刺史。東魏亦以胡梨茍為東義州刺史。
 寧僅得入州,梨茍亦至,寧逆擊破之,斬其洛安郡守馮善道。州既鄰接疆場,百姓流移,寧留心撫慰,咸來復業。轉涼州刺史。寧未至而前刺史宇文仲和據州作亂,詔獨孤信與寧討之。寧先至涼州,為陳禍福,城中吏人皆相率降附。



 仲和仍據城不下,尋亦剋之。後遷驃騎大將軍、開府儀同三司,加侍中,進爵為公。



 十六年,宕昌叛羌獠甘作亂,逐其王彌定而自立,并連結傍乞鐵匆及鄭五醜等。



 詔寧率軍與宇文貴、豆盧寧等討之。寧別擊獠甘,而山路險阻,才通單騎,獠甘已分其黨立柵守險。寧進兵攻之,遂破其柵。獠甘將百騎走投生羌鞏廉玉。彌
 定遂得復位。寧以未獲獠甘,遂進軍大破之,生獲獠甘,徇而斬之。并執鞏廉玉送闕。所得軍實,悉分賞將士,寧無私焉。師還,召寧率所部鎮河陽。



 寧先在涼州,戎夷服其威惠,遷鎮之後,邊人並思慕之。魏廢帝元年,復除涼甘瓜三州諸軍事、涼州刺史。初蠕蠕與魏和親,後更離叛。尋為突厥所破,殺其主阿那環。部落逃逸者,仍奉環之子孫,抄掠河右。寧率兵邀擊,獲環子孫二人,并其種落酋長。自是每戰破之,前後降數萬人。進爵安政群公。二年,吐谷渾通使於齊,寧擊獲之,就拜大將軍。寧後遣使詣周文帝請事,周文即以所服冠履衣被及弓箭甲等
 賜寧,謂其使人曰:「為我謝涼州,孤解衣以衣公,推心以委公,善始令終,無損功名也。」



 時突厥木汗可汗假道涼州,將襲吐谷渾,周文令寧率騎隨之。軍至番禾,吐谷渾已覺,奔於南山。木汗將分兵追之,令俱會於青海。寧謂木汗曰:「樹敦、賀真二城是吐谷渾巢穴,今若拔其本根,餘種自然離散,此上也。」木汗從之,即分為兩軍,木汗從北道向賀真,寧趣樹敦。渾娑周王率眾逆戰,寧擊斬之。踰山履險,遂至樹敦。樹敦是渾之舊都,多諸珍藏。而渾主先已奔賀真,留其征南王及數千人固守。寧進兵攻之,偽退,渾人果開門逐之,因回兵奮擊,門未及闔,寧
 兵遂得入。



 生獲其征南王,俘虜男女財寶盡歸諸突厥。渾賀羅拔王依險為柵,欲塞寧路,寧攻破之。木汗亦破賀真,虜渾主妻子,大獲珍物。寧還軍於青海,與木汗會。木汗握寧手,歎其勇決,并遺所乘良馬,令寧於帳前乘之,木汗親自步送。突厥以寧所圖必破,皆畏憚之,咸曰:「此中國神智人也。」及將班師,木汗又遺寧奴婢一百口、馬五百匹、羊一萬口。寧乃還州,尋被徵入朝。屬周文帝崩,寧悲慟不已,乃請赴陵所盡哀,並告行師剋捷。



 周孝閔帝踐阼,拜小司徒,出為荊州刺史、荊襄淅郢等五十二州及江陵鎮防諸軍事。寧有謀畫,識兵權,臨敵指捴,
 皆如其策,甚得當時之譽。及在荊州,頗自奢縱,貪濁不修法度。嘗出,有人訴州佐屈法,寧還付被訟者治之。自是有事者不敢復言,聲名大損於西州。保定三年,卒於州,謚曰烈。子雄嗣。



 雄字世武。少勇敢,膂力過人,便弓馬,有算略。年十四,從寧於牽屯山奉迎周文帝。仍從校獵,弓無虛發,周文歎異之。尋尚周文女永富公主。除使持節、驃騎大將軍、開府儀同三司,累遷駕部中大夫、司馭中大夫。從柱國、枹罕公辛威鎮金城,遂卒於軍,時年二十四。雄弟祥。



 祥字世休,少有文武才幹。仕周,太子車右中士,襲爵武
 遂縣公。隋文帝踐阼,拜儀同,領交州事,進爵陽城郡公。在州頗有惠政。轉驃騎將軍。伐陳之役,從宜陽公王世積出九江道,破陳師,進拔江州。文帝大悅,下詔慰勉之。進位上開府。



 尋拜蘄州刺史,遷蘄州總管,徵拜左領軍將軍。復以行軍總管從晉王廣破突厥於靈武。遷右衛將軍。仁壽中,率兵屯弘化以備胡。煬帝時在東宮,遺祥書,論舊行兵時事,申以恩旨。祥為書陳謝。太子甚親遇之。



 及即帝位,漢王諒作亂,遣其將綦母良自滏口徇黎陽,塞白馬津,餘公理自太行下河內。帝以祥為行軍總管,軍於河陰,久不得濟。祥謂軍吏曰;』余公理輕而無謀,
 又新得志,謂其眾可恃,恃眾必驕。且河北人先不習兵,所謂擁市人而戰,不足圖也。」乃令軍中修攻具。公理使諜知之,果屯兵於河陽內城以備。祥於是艤船南岸,公理聚甲當之。祥乃簡精銳,於下流潛度。公理拒之,未成列,祥縱擊大破之。東趣黎陽,討綦良。綦良棄軍走,其眾大潰。進位上大將軍,賜縑彩七千段、女妓十人、良馬二十疋。轉太僕卿。帝嘗賜祥詩曰:「伯炯朝寄重,夏侯親遇深,貴耳唯聞古,賤目詎知今?早摽勁草質,久有背淮心,掃逆黎山外,振旅河之陰。



 功已書王府,留情太僕箴。」祥上表辭謝。帝手詔曰:「昔歲勞公,問罪河朔。賊爾日塞兩
 關之路,據倉阻河,公竭誠奮勇,一舉而剋。故聊示所懷,亦何謝也。」



 尋遷鴻臚卿,從征吐谷渾。祥出玉門道,擊虜破之。進位右光祿大夫,拜右驍衛大將軍。及征遼東,出蹋頓道,不利,由是除名。俄拜燕郡太守,被賊高開道所圍,城陷,開道甚禮之。會開道與羅藝通和,送祥於涿群,卒於塗。子義隆,永年令。



 祥弟雲,字世高,亦以父勛賜爵武平縣公。歷位司織下大夫、儀同大將軍、萊州刺史。



 雲弟威,字世儀,亦以父勛賜爵武當縣公。



 權景宣,字暉遠,天水顯親人也。父曇騰,魏隴西郡守,贈秦州刺史。景宣少聰悟,有氣俠,宗黨皆歎異之。年十七,
 魏行臺蕭寶夤見而奇之,表為輕車將軍。



 及寶夤敗,景宣歸鄉里。周文帝平隴右,擢為行臺郎中。孝武四遷,授鎮遠將軍、步兵校尉,加平西將軍、秦州大中正。大統初,轉祠部郎中。景宣曉兵權,有智略。



 從周文拔弘農,破沙苑,皆先登陷陣。轉外兵郎中。從開府于謹援洛陽,景宣督課糧儲,軍以周濟。



 時初復洛陽,將修繕宮室,景宣率徒三千,先出採運。會東魏兵至,司州牧元季海等以眾少拔還,屬城悉叛,道路擁塞。景宣將二十騎且戰且走,從騎略盡。景宣輕馬突圍,手斬數級,馳而獲免,因投人家自匿。景宣以久藏非計,乃偽作周文書,招募得五百
 餘人,保據宜陽,聲言大軍續至。東魏將段琛等率眾至九曲,憚景宣不敢進。景宣恐琛審其虛實,乃將腹心自隨,詐云迎軍,因得西遁。與儀同李延孫相會,攻拔孔城。洛陽以南,尋亦來附。周文即留景宣守張白塢,節度東南義軍。



 東魏將王元軌入洛,景宣與延孫等擊走之,以功授大行臺左丞。進屯宜陽,攻襄城,拔之,獲郡守王洪顯。周文嘉之,徵入朝。錄前後功,封顯親縣男,除南陽郡守。



 郡鄰敵境,舊制發人守防三十五處,多廢農桑,而姦宄猶作。景宣至,並除之,唯修起城樓,多備器械,寇盜斂迹,人得肄業焉。百姓稱之,立碑頌德。周文特賞粟帛,以
 旌其能。遷廣州刺史。



 侯景舉河南來附,景宣從僕射王思政經略應接。既而侯景南叛,恐東魏復有其地,以景宣為大都督、豫州刺史,鎮樂口。東魏亦遣張伯德為刺史。伯德令其將劉貴平率其戍卒及山蠻,屢來攻逼。景宣珍不滿千人,隨機奮擊,貴平乃退走。進授使持節、車騎大將軍、儀同三司。潁川陷後,周文以樂口等諸城道路阻絕,悉令拔還。襄州刺史杞秀以狼狽獲罪。景宣號令嚴明,戎旅整肅,所部全濟,獨被優賞。



 仍留鎮荊州,委以鵶南之事。



 初,梁岳陽王蕭詧將以襄陽歸朝,仍勒兵攻梁元帝於江陵。詧叛將杜岸乘虛襲之。景宣乃率騎
 三千助詧。詧因是乃送其妻王氏及子寮入質。景宣又與開府楊忠取梁將柳仲禮,拔安陸、隨郡。久之,隨州城人吳士英殺刺史黃道玉,因聚為寇。景宣以英小賊,可以計取之,若聲其罪,恐同惡者眾。迺與英書,偽稱道玉兇暴,歸功英等。英等果信之,遂相率而至。景宣執而戮之,獲其黨與。進攻應城,拔之,獲夏侯珍洽。於是應禮安隨並平。朝議以景宣威行南服,乃授并安肆郢新應六州諸軍事、并州刺史。尋進驃騎大將軍、開府儀同三司,加侍中,兼督江、北司二州諸軍事,進爵為伯。唐州蠻田魯嘉自號豫州伯,引致齊兵,大為人害。景宣又破之,獲魯
 嘉,以其地為郡。轉安州刺史。梁定州刺史李洪遠初款後叛,景宣惡其懷貳,密襲破之,虜其家口及部眾。洪遠脫身走免。自是酋帥懾服,無敢叛者。



 燕公于謹征江陵,景宣別破梁司空陸法和司馬羊亮於溳水。又遣別帥攻拔魯山。



 多造舟艋,益張旗幟,臨江欲度,以懼梁人。梁將王琳在湘州,景宣遺書喻以禍福,琳遂遣長史席壑因景宣請舉州款附。周孝閔帝踐阼,徵為司憲中大夫。。尋除基鄀硤平四州五防諸軍事、江陵防主,加大將軍。保定四年,晉公護東討,景宣別略河南。齊豫州刺史王士良、永州刺史世怡並以城降。景宣以開府謝徹守
 永州,開府郭彥守豫州,以士良、世怡及降卒一千人歸諸京師。尋而洛陽不守,乃棄二州,拔其將士而還。至昌州而羅陽蠻反,景宣回軍破之。還次霸上,晉公護親迎勞之。天和初,授荊州刺史,總管十七州諸軍事,進爵千金郡公。陳湘州刺史華皎舉州款附,表請援兵。敕景宣統水軍與皎俱下。景宣到夏口,陳人已至。而景宣以任遇隆重,遂驕傲縱恣,多自矜伐,兼納賄貨,指麾節度,朝出夕改。將士憤怒,莫肯用命。



 及水軍始交,一時奔北,戰艦器仗,略無孑遺。時衛公直總督諸軍,以景宣負敗,欲繩以軍法。朝廷不忍加罪,遣使就軍赦之。尋遇疾卒。贈
 河、渭、鄯三州刺史,謚曰恭。



 子如璋嗣,位至開府、膠州刺史。



 如璋弟仕玠,儀同大將軍、廣川縣侯。



 論曰:王盟始以親黨升朝,終而才能進達,勤宣運始,位列周行,實參迹於功臣,蓋弗由於恩澤。誼文武奇才,以剛正見忌,有隋受命,鬱為名臣,末路披猖,信有終之克鮮。獨孤信威申南服,化洽西州,信著遐方,光昭鄰國,雖不免其身,慶延于後,三代外戚,何其盛歟!竇熾儀表魁梧,器識雄遠,入參朝政,則嘉謀屢陳,出總籓條,則惠政斯洽。毅忠蕭奉上,溫恭接下,茂實彰於本朝,義聲播於殊俗。並以國華人望,論道當官,榮映一時,慶流來葉。及
 熾遲疑勸進,有送故之心,雖王公恨恨,何以加此。榮定以功懋賞,以勞定國,保其祿位,貽厥子孫,盛矣。



 賀蘭祥、叱列伏龜、閻慶等雖階緣戚屬,各以功名自終,而毗制造之功,亦足傳於後葉。史寧、權景宣並以將帥之才,受內外之寵,總戎薄伐,著克敵之功,布政蒞人,垂稱職之譽,若此者,豈非有國之良翰歟!然而史在末年,貨財虧其雅志,權亦晚節矜驕,喪其威聲,惜矣。楊諒干紀,祥獨克之,效亦足稱云爾。



\end{pinyinscope}