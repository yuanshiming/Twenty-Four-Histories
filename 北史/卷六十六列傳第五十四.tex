\article{卷六十六列傳第五十四}

\begin{pinyinscope}

 王傑王勇宇文虯耿豪高琳李和子徹伊婁穆達奚寔劉雄侯植李延孫韋祐陳欣魏玄泉李遷哲楊乾運扶猛陽雄席固任果
 王傑,金城直城人也,本名文達。父巢,魏榆中鎮將。傑少有壯志,每以功名自許。從孝武西遷,賜爵都昌縣子。周文奇其才,嘗謂諸將曰:「王文達萬人敵也,但恐勇決太過耳。」從復潼關,破沙苑,爭河橋,戰芒山,皆以勇敢聞。親待日隆,於是賜姓宇文氏,進爵為公。累遷侍中、驃騎大將軍、開府儀同三司。恭帝元年,從于謹圍江陵。時柵內有人,善用長槊,將士登者,多為所斃。謹令傑射之,應弦而倒。登者乃得入,遂拔之。謹曰:「濟我大事者在公此箭也。」周孝閔帝踐祚,進爵張掖郡公,為河州刺史。朝延以傑勛望俱重,故授以本州。後與隨公楊忠自漠北伐齊。
 又從齊公憲東禦齊將斛律明月。進位柱國。建德初,除涇州總管,頗為百姓所慕。宣帝即位,拜上柱國。薨。贈七州諸軍事、河州刺史,追封鄂國公,謚曰威。



 子孝遷,位開府儀同大將軍。



 王勇,代武川人也,本名胡仁。少雄健,有膽決。數從侯莫陳悅、賀拔岳征討,功居多,拜別將。周文為丞相,封包信縣子。從禽竇泰,復弘農,戰沙苑,氣蓋眾軍,所當必破。周文歎其勇敢,賞賜特隆,進爵為公。大軍不利,唯胡仁及王文達、耿令貴三人力戰,皆有殊功。軍還,拜上州刺史,以雍州、岐州、北雍州擬授胡仁等。然州頗有優劣,文令
 探籌取之。胡仁遂得雍州,文達得岐州,令貴得北雍州。



 仍賜胡仁名勇,令貴名豪,文達名傑,以彰其功。進侍中、驃騎大將軍、開府儀同三司。恭帝元年,從柱國趙貴徵蠕蠕,破之,進爵新陽郡公,賜姓庫汗氏。又論討蠕蠕功,別封永固縣伯。時有別封者,例聽迴授次子,勇獨請封兄子興,時人義之。



 尋進位大將軍。勇性雄猛,為當時驍將。矜功伐善,好論人之惡,時論亦以此鄙之。



 柱國侯莫陳崇勛高望重,與諸將同謁晉公護,聞勇數論人短,乃於眾中折辱之。勇慚恚,因疽發背卒。



 子昌嗣。官至大將軍。



 宇文虯,字樂仁,代武川人也。驍悍有膽略。少從征討,累有戰功,封南安侯。



 孝武西遷,以獨孤信為行臺,信引虯為帳內都督。隨信奔梁。大統三年歸闕,進爵為公。禽竇泰,復弘農,及沙苑、河橋之戰,皆有功。又從獨孤信討梁定,破之。累遷南秦州刺史、驃騎大將軍、開府儀同三司。虯每經行陣,必身先士卒,故上下同心,戰無不剋。後除金州刺史、大將軍。卒。



 耿豪,鉅鹿人也,本名令貴。其先家於武川。豪少粗獷,有武藝,好以氣陵人。



 賀拔岳西征,引為帳內。岳被害,歸周文,以武勇見知。豪亦自謂所事得主。從討侯莫陳悅及
 迎孝武,錄前後功,封平原子。沙苑之戰,豪殺傷甚多,血染甲裳盡赤。



 周文歎曰:「令貴武猛,所向無前,觀其甲裳,足以為驗,不須更論級數也。」進爵為公。從周文戰芒山,豪謂所部曰:「大丈夫除賊,須右手拔刀,左手把槊,直斫直刺,慎莫畏死。」遂大呼獨入,敵人鋒刃亂下,當時咸謂豪歿。俄然奮刀而還。



 戰數合,當豪前者死傷相繼。又謂左右曰:「吾豈樂殺人,但壯士除賊,不得不爾。



 若不能殺賊,又不為人所傷,何異逐坐人也!」周文嘉之。拜北雍州刺史,賜姓和稽氏。進位侍中、驃騎大將軍、開府儀同三司。豪性兇悍,言多不遜,周文惜其驍勇,每優容之。豪亦
 自謂意氣冠群,終無所屈。李穆、蔡祐初與豪同時開府,後並居豪之右。豪不能平,謂周文曰:「人間物議,謂豪勝李穆、蔡祐。」周文曰:「何以言之?」豪曰:「人言李穆、蔡祐是丞相髆髀,耿豪、王勇,丞相咽項,以在上,故為勝也。」豪之粗猛皆此類。卒,周文痛惜之。



 子雄嗣,位至大將軍。



 高琳,字季氏,其先高麗人也。仕於燕,又歸魏,賜姓羽真氏。琳母嘗祓禊泗濱,遇見一石,光彩朗潤,遂持以歸。是夜,夢人衣冠有若仙者,謂曰:「夫人向所將來石,是浮磬之精。若能寶持,必生令子。」母驚寤,舉身流汗。俄而有娠,及生,因名琳,字季氏。從孝武西遷,封鉅野縣子。河橋之
 役,琳勇冠諸軍。周文謂曰:「公即我之韓、白也。」復從戰芒山,除正平郡守。齊將東方老來寇,琳擊之。老中數創乃退,謂其左右曰:「吾經陣多矣,未見如此健兒。」後除鄜州刺史,加驃騎大將軍、開府儀同三司、侍中。周孝閔帝踐祚,進爵犍為郡公。武成二年,討平文州氏。師還,帝宴群公卿士,仍賦詩言志。琳詩末章云:「寄言竇車騎,為謝霍將軍。何以報天子?沙漠靜妖氛。」帝大悅曰:「獯、獫陸梁,未時款塞,卿言有驗,國之福也。」天和三年,為江陵副總管。時陳將吳明徹來寇,總管田弘與梁主蕭巋出保紀南城,唯琳與梁僕射王操固守江陵三城以抗之。晝夜拒
 戰,凡經十旬,明徹退走。巋表言其狀,帝乃優詔追琳入朝,親加勞問。六年,進位柱國。薨。



 贈本官。加五州諸軍事、冀州刺史,謚曰襄。



 子儒襲爵。位儀同大將軍。



 李和,本名慶和,朔方巖綠人也。父僧養,以累世雄豪,為夏州酋。和少敢勇有識度,狀貌魁偉,為州里所推。賀拔岳作鎮關中,引為帳內都督。後從周文,累遷侍中、驃騎大將軍、開府儀同三司、夏州刺史,賜姓宇文氏。周文嘗謂諸將曰:「宇文慶和累經任委,每稱吾意。」又賜名意焉。改封永豐縣公。保定二年,除司憲中大夫。尋改封德廣郡公,出為洛州刺史。和前在夏州,頗留遺惠,及有此授,
 商、洛父老莫不想望德音。和至州,以仁恕訓物,獄訟為之簡靜。進柱國大將軍。



 隋開皇元年,遷上柱國。和立身剛簡,老而逾勵,諸子趨事,若奉嚴君。以意是周文帝賜名,帝朝已革;慶和則父之所命,義不可違。至是,遂以和為名。二年,薨。



 贈本官,加司徒公,謚曰肅。子徹嗣。



 徹字廣達。性剛毅,有器幹。周武帝時,從皇太子西征吐谷渾,以功賜爵周昌縣男。從武帝平齊,錄前後功,再進爵。遷左武衛將軍。及隋晉王廣鎮並州,妙選府官,詔徹總晉王府軍事,進爵齊安郡公。時蜀王秀亦鎮益州,上謂侍臣曰:「安得文同王子相,武如李廣達者乎!」其見重
 如此。明年,突厥沙缽略可汗犯塞,上令衛王爽為元帥擊之,以徹為長史。遇虜於白道,行軍總管李充請襲之。諸將多以為疑,唯徹獎成其事,請同行,遂掩擊大破之。沙缽略棄所服金甲而遁。以功加上大將軍。沙缽略因此稱籓。改封安道郡公。開皇十年,進位柱國。及晉王為揚州總管,以徹為司馬,改封德廣郡公。尋徙封城陽郡公。其後突厥犯塞,徹復領行軍總管破之。及左僕射高熲得罪,以徹素與熲善,被疏忌。後出怨言,上聞,召入臥內賜宴,言及平生,因遇鳩卒。大業中,其妻元氏為孽子安遠誣以咒詛,伏誅。



 伊婁穆,字奴干,代人也。父靈,善騎射,為周文所知,嘗謂之曰:「若伊尹阿衡於殷,致主堯、舜。卿既姓伊,庶卿不替前緒。」於是賜名尹焉。歷衛將軍、隆州刺史、盧奴縣公。穆弱冠為周文帳內親信,以機辯見知。歷中書舍人、通直散騎常侍。嘗入白事,周文望見悅之,字之曰:「奴干作儀同面見我矣。」於是拜儀同三司,賜封安陽縣伯。周孝閔帝踐阼,進位驃騎大將軍。建德中,卒。



 達奚寔,字什伏代,河南洛陽人也。父顯相,武衛將軍。寔少修立,有幹局。



 從魏孝武西遷,封臨汾縣伯。從周文禽竇泰,復弘農,破沙苑,皆力戰有功。累遷相府從事中郎。
 寔性嚴重,深見器遇。六官建,行蕃部中大夫,加驃騎大將軍、開府儀同三司,進爵平陽縣公。周保定初,卒於刺史。謚曰恭。子豐嗣。



 劉雄,字猛雀,臨洮子城人也。少機辯,慷慨有大志,初為周文親信,後拜中大夫,兼中書舍人,賜姓宇文氏。周孝閔帝踐阼,加大都督。天和中,累遷驃騎大將軍、開府儀同三司,封周昌侯。歷位納言、內史中大夫、候正。武帝嘗從容謂曰:「古人云:『富貴不歸故鄉,猶衣錦夜遊。』」乃以雄為河州刺史。雄先已為本縣令,復有此授,鄉里榮之。及皇太子西征吐谷渾,雄自涼州從滕王道先入,功居多,
 加上開府儀同三司。從平并州,拜上大將軍,進爵趙郡公。平鄴城,進柱國。宣政元年,突厥寇幽州,雄戰歿。贈亳州總管。



 子昇嗣。以雄死王事,授儀同大將軍。



 侯植,字仁幹,其先上谷人也。高祖恕,為北地太守,子孫因家于北地之三水。



 植少倜儻,有大節,容貌奇偉,武藝絕倫。仕魏為義州刺史,甚有政績。後從孝武西遷,賜姓侯伏侯氏。從周文破沙苑,戰河橋,進大都督。涼州刺史宇文仲和據州作逆,植從開府獨孤信討禽之,封肥城縣公,賜姓賀屯氏。後從于謹平江陵,進驃騎大將軍、開府儀同三司,別封一子水開源縣伯。周孝文帝踐阼,進爵郡公。
 時帝幼沖,晉公護執政,植從兄龍恩為護所親。及護誅趙貴,諸宿將等多不自安。植謂從兄龍恩曰:「主上春秋既富。安危繫於數公,若多誅戮,自立威權,何止社稷有累卵之危,恐吾宗亦緣此敗。兄安得知而不言!」龍恩竟不能用。植又承間言於護曰:「公以骨肉之親,當社稷之寄,願推誠王室,擬迹伊、周,則率土幸甚。」護曰:「我誓以身報國,卿豈謂吾有他志邪?」又聞其先與龍恩言,乃陰忌之。植懼不免禍,遂以憂卒。贈大將軍、平州刺史,謚曰節。子定嗣。及護伏誅,龍恩及其弟萬壽並預其禍。武帝以植忠於朝廷,特免其子孫。



 李延孫,伊川人也。父長壽,性雄豪,少與蠻酋結託,侵掠闕南。魏孝昌中,朝議恐其為亂,乃以長壽為防蠻都督,給其鼓節。長壽盡其智力,防遏群蠻,伊川左右,寇盜為之稍息。永安之後,長壽徒侶日盛,魏帝藉其力用,因而撫之。累遷北華州刺史,賜爵清河郡公。及孝武西遷,長壽率勵義士拒東魏。後為廣州刺史。



 東魏遣行臺侯景攻之,城陷,遇害。追贈太尉。延孫亦雄武,有將率才略,少從長壽征討,以勇敢聞。賀拔勝為荊州刺史,表延孫為都督,肅清鵶路,頗有力焉。及長壽被害,延孫乃還,收集其父之眾。自孝武西遷後,朝士流亡。廣陵王欣、錄尚書
 長孫承業、穎川王斌之、安昌王子均及建寧、江夏、隴東諸王并百官等攜持妻子來投延孫者,即率眾衛送,並贈以珍玩,咸達關中。齊神武深患之,遣行臺慕容紹宗等數道攻擊,延孫大破之。乃授延孫京南行臺、節度河南諸軍事、廣州刺史。尋進車騎大將軍、儀同三司、大都督,賜爵華山郡公。延孫既蒙重委,每以克清伊、洛為已任,頻以寡擊眾,威振敵境。大統西年,為其長史楊伯蘭所害。贈司空。



 子人傑,有祖、父風。官至開府儀同三司,改封潁川郡公。



 韋祐,字法保,京兆山北人也,以字行。為州郡著姓,父義,
 上洛郡守。魏大統中,以法保著勳,追贈秦州刺史。法保少好遊俠,而質直少言,所與交遊,皆輕猾亡命。父沒,事母以孝聞。慕李長壽之為人,遂娶其女,因寓居闕南。正光末,王公避難者或依之,多得全濟,以此為貴遊所德。及孝武西遷,法保赴行在所,封固安縣男。及長壽被害,其子延孫收長壽餘眾,守禦東境。朝廷恐延孫兵少,乃除法保東洛州刺史,配兵數百,以援延孫。法保至潼關,弘農郡守韋孝寬謂曰:「恐子此役,難以吉還。」法保曰:「古人稱不入獸穴,不得獸子。安危之事,未可預量。」遂倍道兼行。與延孫兵接,乃并勢置柵於伏流。未幾,周文追法
 保與延孫還朝,賞勞甚厚。除河南尹。及延孫被害,法保乃率所部據延孫舊柵。嘗與東魏戰。



 流矢中頸,從口中出,久之乃蘇。大統九年,鎮九曲城。乃侯景以豫州附,法保率兵赴。景欲留之,法保疑其貳,乃固辭還所鎮。十五年,加驃騎大將軍、開府儀同三司,尋進爵為公。會東魏遣軍送糧饋宜陽,法保潛邀之,中流矢,卒於陣。謚曰莊。



 子初嗣。位開府儀同大將軍、閻韓防主。



 陳欣,字永怡,宜陽人也。少驍勇,有氣俠,姿貌魁岸,同類咸敬憚之。孝武西遷後,欣乃於辟惡山招集勇敢少年,寇掠東魏,仍密遣使歸附。授立義大都督,賜爵霸城縣
 男。累遷宜陽郡守。恭帝二年,進位驃騎大將軍、開府儀同三司,加侍中、宜陽邑大中正,賜姓尉遲氏。周文以欣著績累載,贈其祖昆及父興孫俱為儀同三司,位刺史。東魏洛州刺史獨孤永業,號有智謀,往來境上,欣與韓雄等恆令間諜覘其動靜,齊兵每至,輒破之,故永業深憚欣等,不敢為寇。周孝閔帝踐阼,進爵許昌縣公。後除熊州刺史,卒於州。欣與韓雄里閈姻婭,少相親暱,俱總兵境上三十餘載。每禦扞,二人相赴,常若影響。故數對勍敵,而常保功名。雖並有武力,至於挽強射中,欣不如雄;散財施惠,得士眾心,則雄不如欣。身死之日,將吏荷
 其恩德,莫不感慟。



 子萬敵嗣。朝廷以欣雅得士心,還令萬敵領其部曲。



 魏玄,字僧智,其先任城人也,後徙於新安。玄少慷慨,有膽略。孝武西遷,東魏北徙,人情各懷去就,玄每率鄉兵抗拒東魏。芒山之役,大軍不利,宜陽、洛州皆為東魏守,而玄母及弟並在宜陽。玄以為忠孝不兩立,乃率義徒還闕南鎮撫。



 周文手書勞之。除洛陽令,封廣宗縣子。周保定元年,累遷驃騎大將軍、開府儀同三司,鎮閻韓。遷熊州刺史,政存簡惠,百姓悅之。轉和州刺史、伏流防主,進爵為公。及齊將斛律明月率眾向宜陽,兵威甚盛,玄
 率眾禦之,每戰輒克。後以疾卒於位。



 泉,字思道,上洛豐陽人也。世雄商洛,自晉東度,常貢屬江東。曾祖景言,魏太延五年率鄉里歸化,仍引王師平商洛。拜建節將軍,假宜陽郡守,世襲本縣令,封丹水侯。父安志,復為建節將軍、宜陽郡守,領本縣令,降爵為伯。



 九歲喪父,哀毀類於成人。服闋襲爵,年十二,鄉人皇平、陳合等三百餘人詣州,請為縣令。州為申上。時吏部尚書郭祚以年少,請別選遣,終此一限,令代之。宣武詔依皇平等所請。巴俗事道,尤重老子之術。雖童幼,而好學恬靜,百姓安之。尋以母憂去職。縣中父
 老復表請起復本任。後除上洛郡守。及蕭寶夤反,遣兵趣青泥,圖取上洛,豪族泉、杜二姓密應之。與刺史董紹掩襲,二姓散走,寶夤亦退。遷淅州刺史,別封涇陽縣伯。永安中,大破梁將王玄真於順陽,除東雍州刺史,進爵為侯。部人楊羊皮,太保椿之從弟,恃椿,侵擾百姓。守宰多被其陵侮,皆畏而不敢言。收之,將加極法。楊氏慚懼,闔宗請恩。自此豪右無敢犯者。性又清約,纖毫不擾於人。在州五年,每於鄉里運米自給。梁魏興郡與洛州接壤,表請內屬。詔為行臺尚書以撫納之。大行臺賀拔岳以昔蒞東雍,為吏人所懷,乃表復為刺史。
 詔許之。蜀人張國俊聚黨剽劫,州郡不能制,收戮之,闔境清肅。



 及齊神武專政,孝武有西顧之心,欲委以山南之事,乃除洛州刺史。未幾,帝西遷。齊神武率眾至潼關,遣其子元禮禦之,神武不敢進。上洛人都督泉岳,其弟猛略與拒陽人杜窋等謀翻洛州以應東魏。知之,殺岳及猛略,傳首詣闕。



 大統元年,加開府儀同三司,兼尚書右僕射,進爵上洛郡公。志尚廉慎,每除一官,憂見顏色,寢食輒減。至是頻讓,魏帝手詔不許。三年,高敖曹圍逼州城,杜窋為其鄉導。拒守旬餘,矢盡援絕,城乃陷焉。謂敖曹曰:「泉力屈,志不服也。」及竇泰被
 禽,敖曹退走,遂執而東,以窋為刺史。臨發,密戒二子元禮、仲遵曰:「吾生平志願,不過令長,幸逢聖會,位亞台司。今爵祿既隆,年齒又暮,前途夷險,抑亦可知。汝等堪立功效,不得以我在東,遂虧臣節也,」



 乃揮涕而訣。聞者莫不憤歎。尋卒於鄴。



 元禮少有志氣,好弓馬,頗閑草隸,有士君子之風。賜爵臨洮縣伯,散騎常侍。



 及洛州陷,與俱被執而東。元禮於路逃歸。時杜窋雖為刺史,然巴人素輕杜而重泉。及元禮至,與仲遵相見,感父臨別之言,潛與豪右結託,遂率鄉人襲州城,斬窋,傳首長安。朝廷嘉之,代襲洛州刺史。從周文戰於沙苑,中流矢卒。
 子貞嗣。



 仲遵一名恭。少謹實,涉獵經史。年十三為郡主簿,十四為縣令。及長,有武藝。高敖曹攻洛州,與力戰拒守。矢盡,以棒杖扞之,為流矢中目,不堪復戰。



 及城陷,士卒歎曰:「若二郎不傷,豈至於此!」之東也,仲遵以被傷不行。



 後與元禮斬窋,以功封豐陽縣伯,東豫州刺史。及元禮戰沒,復以仲遵為洛州刺史。



 頗得譽。大統十三年,行荊州刺史事。梁司州刺史柳仲禮每為邊寇,周文令仲遵率鄉兵,從開府楊忠討之。梁隨郡守桓和拒守不降。忠謂諸將曰:「先取仲禮,則桓和不攻而自服也。」仲遵對曰:「若棄和深入,仲禮未即就禽,則首尾受敵,此危
 道也。」忠從之。仲遵以計由已出,乃先登城,遂禽和。從擊仲禮,又獲之。進驃騎大將軍、開府儀同三司、本州大中正,復行荊州刺史、十三州諸軍事。尋遭母憂,請終喪制,不許。大將軍王雄南征上津、魏興,仲遵從雄討平之。遂於上津置南洛州,以仲遵為刺史。仲遵留情撫接,百姓安之。



 初,蠻帥杜青和自稱巴州刺史,以州入附,朝遷因其所據而授之,仍隸東梁州都督。青和以仲遵善於撫御,請隸仲遵。朝議以山川非便,弗之許也。青和遂結安康酋帥黃眾寶等,舉兵共圍東梁州。復遣王雄討平之,改巴州為洵州,隸於仲遵。



 先是東梁刺史劉孟良在
 職貪婪,人多背叛。仲遵以廉簡處之,群蠻帥服。仲遵雖出自巴夷,而有方雅之操,歷官之處,皆以清白見稱。朝廷又以其父臨危抗節,乃令襲爵上洛郡公,舊封聽回授一子。尋出為都督、金州刺史。卒官。贈大將軍、三州刺史,謚曰莊。



 子恆嗣。位至開府儀同大將軍。



 李遷哲,字孝彥,安康人也。世為山南豪族,仕於江左。父元直,仕梁,歷東梁、衡二州刺史、散騎常侍、沌陽侯。遷哲少修立,有識度,慷慨善謀畫,起家文德主帥。其父為衡州,留遷哲本鄉,臨統部曲事。時年二十,撫馭群下,甚得其情。



 後襲爵沌陽侯,位都督、東梁州刺史。侯景篡逆,遷
 哲外禦邊寇,自守而已。大統十七年,周文遣達奚武、王雄等略地山南。遷哲軍敗,遂降於武。然猶意氣自若。



 武乃執送京師。周文責以不早歸國。答曰:「不能死節,實以此愧耳。」周文深嘉之,封沌陽縣伯。



 恭帝初,直州人樂熾、洋州人黃國等連結為亂。周文以遷哲信著山南,乃令與開府賀若敦同經略。熾等尋並平蕩,仍與敦南出徇地。遷哲先至巴州,入其封郭。



 梁巴州剌史牟安人開門請降。安人子宗徹等猶據巴城不下,遷哲攻剋之。軍次鹿城,城主遣使請降。遷哲謂其眾曰:「納降如受敵,吾觀其使,瞻視猶高,得無詐也?」



 遂不許之。梁人果於道左設
 伏以邀遷哲,遷哲進擊破之,遂屠其城。自此巴、濮之人,降款相繼。軍還,周文賜以所服紫袍玉帶及所乘馬,加授侍中、驃騎大將軍、開府儀同三司,除真州刺史,即本州也。仍給軍儀鼓節,令與田弘同討信州。時信州為蠻酋向五子王等所圍,弘遣遷哲赴援。此至,信州已陷。五子王等聞遷哲至,狼狽遁走。遷哲入據白帝,賀若敦等復至,遂共追五子王等,破之。及田弘旋軍,周文令遷哲留鎮白帝。信州先無倉儲,軍糧匱乏。遷哲乃收葛根造粉,兼米以給之,遷哲亦自取供食。時有異膳,即分賜兵士。有疾患者,又親加醫藥。以此軍中感之,人思效命。黔
 陽蠻田烏度、田烏唐,等每抄掠江中,為百姓患。遷哲隨機出討,殺討甚多,由是諸蠻畏威,各送糧餼。又遣子弟入質者千有餘家,遷哲乃於白帝城外築城以處之。并置四鎮,以靜峽路。自此寇抄頗息,軍糧贍給焉。周明帝初,授都督、信州刺史。二年,進爵西城縣公。武成元年,朝于京師。明帝甚禮之,賜甲第及莊田等。天和三年,進位大將軍。詔遷哲率金、上等諸州兵鎮襄陽。五年,陳將章昭達攻逼江陵,梁明帝告急於襄州,衛公直令遷哲往救焉。遷哲率其所部守江陵外城,自率騎出南門,又令步兵自北門出,兩軍首尾邀之,陳人多投水死。是夜,陳
 人又竊於城西堞以梯登城,登者已百數人。遷哲又率驍勇扞之,陳人復潰。俄而大風暴起,遷哲乘闇出兵擊其營,陳人大亂,殺傷甚眾。江陵總管陸騰復破之於西隄,陳人乃遁。建德二年,進爵安康郡公。三年,卒於襄州。贈金州總管,謚曰壯武。



 遷哲累葉雄豪,為鄉里所服。性復華侈,能厚自奉養。妾媵至有百數,男女六十九人。緣漢千餘里間,第宅相次,姬媵之有子者,分處其中,各有僮僕侍婢閽人守護。遷哲每鳴笳導從,往來其間,縱酒歡宴,盡生平之樂。子孫參見,或忘其年名者,披簿以審之。



 長子敬仁,先遷哲卒。第六子敬猷嗣,還統父兵,位儀
 同大將軍。



 遷哲弟顯,位上儀同大將軍。



 楊乾運,字玄邈,儻城興勢人也,少雄武,為鄉閭信服。為安康郡守。陷梁,仕歷潼、南梁二州刺史。及武陵王蕭紀稱尊號,以乾運威服巴、渝,乃拜梁州刺史,鎮潼州,封萬春縣公。時紀與其兄湘東王繹爭帝,乾運兄子略勸乾運歸附,乾運然之。會周文令乾運孫法洛至,略即夜送之,乾運送款,周文密賜乾運鐵券,授開府儀同三司、侍中、梁州刺史、安康郡公。及尉遲迥征蜀,遂降迥。迥因此進軍成都,數旬剋之。及至京師,禮遇隆渥。尋卒於長安。贈尚書右僕射。子端嗣。略亦以歸附功,位至開府儀同
 三司、大將軍,封上庸縣伯。



 乾運女婿樂廣,安州刺史,封安康縣公。



 扶猛,字宗略,上甲黃土人也。其種落號白獸蠻。猛仕梁,位南洛、北司二州刺史,封宕梁縣男。魏廢帝元年,以眾降。周文厚加撫納,復爵宕渠縣男,割二郡為羅州,以猛為刺史。令從開府賀若敦南討信州。敦令猛直道白帝,所由之路,人迹不通。猛乃梯山捫葛,備歷艱阻,遂入白帝。撫慰人夷。莫不悅附。以功進開府儀同三司。俄則信州蠻反,猛復從賀若敦平之,進爵臨江縣公。後從田弘破漢南諸蠻,進位大將軍。卒。



 陽雄,字元略,上洛邑陽人也。累葉豪族。父猛,從孝武西遷,以功封郃陽伯,位征東將軍、揚州刺史。雄起家奉朝請,以軍功封安平縣侯。得子孫相襲拜邑陽郡守。累遷平州刺史,進爵玉城縣公,加開府儀同三司、驃騎大將軍。歷京兆尹、戶部中大夫,進位大將軍,轉中外府長史,遷江陵總管,改封魯陽縣公。卒於鎮。追封郡公,謚曰懷。雄善附會,能自謀身,故任兼出內,保全爵祿。子長寬嗣。



 席固,字子堅,其先安定人也。高祖衡,因姚氏之亂,寓居襄陽,仕晉,為建威將軍,遂為襄陽著姓。固少有遠志。梁大同中,為齊興郡守。久居郡職,士多附之,遂有親兵千
 餘人。梁元帝時,遷興州刺史,軍人募從者至五千餘人。固欲自據一州,以觀時變。大統中,以地歸魏。時周文方南取江陵,西定蜀、漢,聞固至,甚禮遇之。就拜使持節、驃騎大將軍、開府儀同三司、大都督、侍中、豐州刺史,封新豐縣公。後轉湖州刺史,啟求入觀。及至,進爵靜安郡公。尋拜昌、歸、憲三州諸軍事、昌州刺史。固居家孝友,蒞官頗有聲績。卒於州。贈大將軍、五州刺史,謚曰肅,敕襄州賜其墓田。子雅嗣。



 雅字彥文。性方正,少以孝聞。位大將軍。



 雅弟英,上開府儀同大將軍。



 任果,字靜鸞,南安人也。本方隅豪族。父褒,仕梁,為沙州
 刺史、新巴縣公。



 果性勇決,志在立功。魏廢帝元年,率所部來附。周文嘉其遠至,待以優禮。果因面陳取蜀策,深被納之。乃授沙州刺史、南安縣公。從尉遲迥伐蜀。尋進授驃騎大將軍、開府儀同三司。及成都平,除始州刺史。周文以其方隅首領,早立忠節,進爵樂安郡公,賜以鐵券,聽相傳襲,并賜路車駟馬及儀衛等以光寵之。尋為刺客所害。



 論曰:王傑、王勇、宇文虯、耿豪、高琳、李和、伊婁穆、侯植等咸以果毅之姿,效節擾攘之際,各能屠堅覆銳,自致其功,高爵厚位,固其宜也。仲尼稱無求備於一人,信矣。夫
 文士懷溫恭之操,其弊也懦弱;武夫稟剛烈之資,其弊也敢悍。



 故有使酒不遜之禍,拔劍爭功之尤,大則莫全其生,小則僅而獲免。耿豪、王勇不其然乎!李延孫、韋祐、陳欣、魏玄等以勇略之姿,受扞城之委。灌瓜贈藥,雖有愧於昔賢;禦侮折衝,足方駕於前烈。用能觀兵伊、洛,保據崤、函,齊人阻西路之謀,周朝緩東貢之慮,皆其力也。泉長自山谷,素無月旦之譽,而臨難慷慨,無失人臣之節,豈非蹈仁義之徒歟!元禮、仲遵,聿遵其志,卒成功業,庶乎克負荷矣。李遷哲、楊乾運、席固之徒,屬方隅擾攘,咸知委質,遂享爵位,以保終始。



 觀遷哲之對周文,有
 尚義之氣。乾運受任武陵,乖事人之道。若乃校其優劣,固不可同年而語。陽雄任兼文武,聲著土內,抑亦志能之士也。



 舊史有代人宇文盛,字保興,以武毅顯,盛弟丘,字胡奴,盛子述,位柱國,並有傳。然事無足可紀。盛見子述傳首,丘略之云。



\end{pinyinscope}