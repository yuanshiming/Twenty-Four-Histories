\article{卷三十三列傳第二十一}

\begin{pinyinscope}

 李靈曾孫元忠
 渾弟子璨璨曾孫德饒公緒李順玄孫元操李孝伯兄孫謐謐弟子士謙李裔子子雄李義深弟幼廉李靈,字武符,趙郡平棘人也。父勰,字小同,恬靜好學,有聲趙、魏間。道武平中原,聞其已亡,哀惜之,贈宣威將軍、蘭陵太守。



 神蒨中,太武征天下才俊,靈至,拜中書博士。再遷淮陽太守。以學優,選授文成皇帝經,加中散、內博士,賜爵高邑子。文成踐阼,卒於洛州刺史,贈定州刺史、鉅鹿公,謚曰簡。



 子恢襲,以師傅子,拜
 長安鎮副將,進爵為侯,假鉅鹿公。後東平王道符謀反,遇害,贈定州刺史、鉅鹿公,謚曰貞。恢弟綜,事見於後。



 長子悅祖,襲爵高邑侯,例降為伯,卒。悅祖子瑾,字伯瓊,襲,位大司農卿。



 瑾淳謹好學,老而不倦。卒,贈司空。



 悅祖弟顯甫,豪俠知名,集諸李數千家於殷州西山,開李魚川方五六十里居之,顯甫為其宗主。以軍功賜爵平棘子,位河南太守,贈安州刺史,謚曰安。



 子元忠,少厲志操。粗覽書史及陰陽術數,有巧思,居喪以孝聞。襲爵平棘子,魏清河王懌為營明堂大都督,引為主簿。遭母憂去任,歸李魚川。嘗亡二馬,既獲盜,即以與之。在母喪,哭泣哀動旁人,而飲酒騎射不廢,曰:「禮豈為我?」初元忠以母多患,專心醫藥,遂善方技,性仁恕,無貴賤皆為救療。家素富,在鄉多有出貸求利,元忠焚契免責,鄉人甚敬之。孝莊時,盜賊蜂起,清河有
 五百人西戍;還經南趙郡,以路梗,共投元忠,奉絹千餘匹。元忠唯受一匹,殺五牛以食之,遣奴為導,曰:「若逢賊,但道李元忠遣。」如言,賊皆舍避。及葛榮起,元忠率宗黨作壘以自保,坐於大槲樹下,前後斬違命者凡三百人。賊至,
 元忠輒卻之。葛榮曰:「我自中山至此,連為趙李所破,則何以能成大
 事?」乃悉眾攻圍,執元忠以隨軍。賊平,就拜南趙郡太守。好酒,無政績。



 及莊帝幽崩,元忠棄官,潛圖義舉。會齊神武東出,元忠便乘露車
 載素箏濁酒以奉迎。神武聞其酒客,未即見之。元忠下車獨坐,酌酒擘脯食之,謂門者曰:「本言公招延俊傑,今聞國士到門,不能吐哺輟洗,其人可知。還吾刺,勿復通也。」



 門者以告,神武遽見之。引入,觴再行,元忠車上取箏鼓之,長歌慷慨。歌闋。謂神武曰:「天下形勢可見,明公猶欲事爾朱乎?」神武曰:「富貴皆由他,安敢不盡節。」元忠曰:「非英雄也。高乾邕兄弟
 曾來未?」是時,高乾邕已見,神武因紿曰:「從叔輩麤,何肯來?」元忠曰:「雖麤,並解事。」神武曰:「趙郡醉!」



 使人扶出,元忠不肯起。孫騰進曰:「此君天遣來,不可違也。」神武乃復留與言,元忠慷慨流涕,神武亦悲不自勝。元忠進從橫之策,深見嘉納。又謂神武曰:「殷州小,無糧仗,不足以注大事。冀州大籓,若向冀州,高乾邕兄弟必為明公主人。



 殷州便以賜委。冀、殷合,滄、瀛、幽、定自然弭從。唯劉誕黠胡,或當乖拒,然非明公之敵。」神武急握元忠手而謝焉。時殷州刺史爾朱羽生阻兵據州,元忠聚眾與大軍禽斬之。神武即令行殷州事。累遷太常卿、殷州大中正。後以
 從兄瑾年長,以中正讓之。



 魏孝武帝納神武女為后,詔元忠致娉於晉陽。每宴席論舊事,元忠曰:「昔日建義,轟轟大樂,比來寂寥無人問,更欲覓建義處。」神武撫掌笑曰:「此人逼我起兵。」賜白馬一匹。元忠戲曰:「若不與侍中,當更覓建義處。」神武曰:「建義處不慮無,止畏如此老翁不可遇耳。」元忠曰:「止為此翁難遇,所以不去。」



 因捋神武鬚大笑。神武悉其雅意,深重之。後神武奉送皇后,仍田於晉澤,元忠馬倒,良久乃蘇。神武親自撫視,封晉陽縣伯。後為光州刺史,時州境災儉,人皆菜色,元忠表求賑貸,被報聽用萬石。元忠以為少,遂出十五萬石賑之。事訖,
 表陳,朝廷嘉而不責。徵拜侍中。



 元忠雖處要任,初不以物務干懷,唯以聲酒自娛,大率常醉。家事大小,了不關心。園庭羅種果藥,親朋尋詣,必留連宴賞。每挾彈攜壺,遊遨里閈。每言寧無食,不可使我無酒;阮步兵吾師也,孔少府豈欺我哉。後自中書令復求為太常卿,以其有音樂而多美酒故。神武欲用為僕射,文襄言其放達常醉,不可委以臺閣。其子搔聞之,請節酒。元忠曰:「我言作僕射不勝飲酒樂;爾愛僕射時,宜勿飲酒。」



 每言於執事,云年漸遲暮,乞在閑冗,以養餘年,乃除驃騎大將軍、儀同三司。曾貢文襄王蒲桃一盤,文襄報以百縑,其見賞
 重如此。



 孫騰司馬子如嘗詣元忠,逢其方坐樹下,葛巾擁被,對壺獨酌。庭室蕪曠,使婢卷兩褥以質酒肉。呼妻出,衣不曳地。二公相視,歎息而去,大餉米絹,受而散之。俄復以本官領衛尉卿。卒,有米三石,酒數斛,書籍藥物,充滿篋架。未及賻至,金蟬質絹,乃得斂焉。贈司徒,謚曰敬惠。初,元忠將仕,夢手執炬入其父墓。



 中夜驚起,甚惡之。旦告其受業師,占云:「大吉,可謂光照先人也。」竟如其占。



 性甚工彈,彈桐葉常出一孔,擲棗栗而彈之,十中七八。嘗從文襄入謁魏帝,有梟鳴殿上,文襄命元忠彈之,問得幾丸而落,對曰:「一丸奉至尊威靈,一丸承大將軍
 意氣,兩丸足矣!」如其言而落之。子搔嗣。



 搔字德沈,少聰敏,有才藝。曾采諸聲,別造一器,號曰八絃,時人稱有思理。



 武定末,自丞相記室除河內太守。居數載,流人盡復。代至,將還都,父老號泣,追送二百餘里,生為立碑。終於儀曹郎。



 搔妹曰法行,幼好道,截指自誓不嫁,遂為尼。所居去鄴三百里,往來恆步,在路或不得食,飲水而已。逢屠牽牛,脫衣求贖,泣而隨之。雉兔馴狎,入其山居房室。齊亡後,遭時大檢,施糜粥於路。異母弟宗侃與族人孝衡爭地相毀,尼曰:「我有地,二家欲得者,任來取之,何為輕致忿訟?」宗侃等慚,遂讓為閑田。



 渾字季初,靈之曾孫也。祖綜,行河間郡,早卒。父遵,字良軌,有業尚,為魏冀州征東府司馬。京兆王愉冀州起逆,遇害。贈幽州刺史,謚曰簡。



 渾以父死王事,除給事中。後以四方多難,求為青州征東司馬,與河間邢邵、北海王昕俱奉老母攜妻子,同赴青、齊。未幾而爾朱榮入洛,衣冠殲盡,物論以為知幾。時河北流移人聚青土,眾踰二十萬,共劫河間邢杲為主,起自北海,襲東陽。



 青州刺史元世俊欲謀誅之,府人遂猜貳。渾乃與長史崔光韶具陳禍福,由是歃血而盟,上下還睦。普泰中,崔社客反於海岱,攻圍青州,詔渾為都官尚書、東北道行臺,赴援。社
 客諸城各自固保,渾以社客賊之根本,烏合易離,若銜枚夜襲,便可禽殄。如社客就禽,諸郡可傳檄而定。諸將尚遲疑,渾乃決行。果禽社客,斬首送洛陽,海隅清定。



 天平初,丁母憂,行喪冢側,殆將滅性。武定初,兼散騎常侍、聘梁使主。梁武謂曰:「伯陽之後,久而彌盛,趙李人物,今實居多。」使還,為東郡太守。以贓賄徵還。齊文襄王使武士提以入,置諸庭。渾抗言曰:「將軍今日猶自禮賢邪?」



 文襄笑而舍之。齊天保初,除太子少保。時太常邢邵為少師,吏部尚書楊愔為少傅,論者榮之。以參禪代儀注,賜爵涇陽縣男。文宣以魏《麟趾格》未精,詔渾與邢邵、崔甗、
 魏收、王昕、李伯倫等脩撰。嘗謂魏收曰:「彫蟲小技,我不如卿;國典朝章,卿不如我。」尋除海州刺史。後土人共圍州城,城中多石無井,常食海水,賊絕其路。城內先有一,夏旱涸竭,渾齋戒朝服而祈焉,一朝天雨,泉流涌溢。賊以為神,應時駭散。渾捕斬渠帥,傳首鄴都。渾妾郭,在州干政納貨,坐免,卒于鄴。



 子湛,字處元,涉獵文史,有家風。兼通直散騎常侍、聘陳使副,襲爵涇陽男。



 渾與弟繪、緯俱為聘使主,湛又為使副,是以趙郡人士,目為四使。



 繪字敬文。六歲便求入學,家人以偶年俗忌,不許,遂竊其姊筆牘用之。未踰晦朔,遂通《急就章》,內外以為非常
 兒。及長,儀貌端偉,神情朗俊。第五舅河間邢晏每與言,嘆其高遠,曰:「若披煙霧,如對珠玉,宅相之寄,良在此甥。」



 後敕撰五禮,繪與太原王乂同掌軍禮。魏靜帝於顯揚殿講《孝經》、《禮記》,繪與從弟褰、裴伯茂魏收、盧元明等俱為錄議,簡舉可觀。歷中書侍郎、丞相司馬。



 每霸朝文武總集,對揚王庭,常令繪先發言端,為群僚之首。音祠辯正,風儀都雅,聽者悚然,文襄益加敬異。又掌儀注。武定初,兼散騎常侍,為聘使主。梁武問高相今在何處?黑獺若為形容?高相作何經略?繪敷對明辯,梁武稱佳。與梁人汎言氏族,袁狎曰:「未若我本出自黃帝,姓在十四
 之限。」繪曰:「兄所出雖遠,當共車千秋分一字耳!」一坐皆笑。前後行人皆通啟求市,繪獨守清尚,梁人重其廉潔。



 使還,拜高陽內史。郡境舊有三猛獸,人常患之。繪欲脩檻,遂因鬥俱死於郡西。咸以為化感所致,皆勸申上。繪曰:「猛獸因鬥而斃,自是偶然,貪此為功,人將窺我。」竟不聽。高陽舊多陂澱,繪至後,澱水皆涸,乃置農正,專主勸課,墾田倍增,家給人足。瀛州三郡人俱詣州,請為繪立碑于郡街。神武東巡郡國,在瀛州城西駐馬久立,使郎中陳元康喻慰之。河間太守崔諶,恃其弟暹勢,從繪乞麋角鴿羽。繪答書曰:「鴿有六翮,飛則沖天;麋有四足,走
 便入海。下官膚體疏懶,手足遲鈍,不能近追飛走,遠事佞人。」時文襄使暹選司徒左長史,暹薦繪,既而不果,咸謂由此書。



 及文襄嗣業,普代山東諸郡,其特降書徵者,唯繪與清河太守辛術二人而已。



 至,補大將軍從事中郎,遷司馬。文襄以前司徒侯景進賢冠賜繪曰:「卿但直心事孤,當用卿為三公,莫學侯景叛也。」及文宣嗣事,仍為丞相司馬。天保初,除司徒右長史。繪質性方重,未嘗趣事權門,以此久而沈屈。卒,贈南青州刺史,謚曰景。子君道,有父風。



 繪弟緯,字乾經,少聰慧,有才學。與舅子河間邢昕少相倫輩,晚不逮之。位中散大夫。聘梁使主、侍
 中李神俊舉緯尚書南主客郎。緯前後接對凡十八人,頗為稱職。鄴下為之語曰:「學則渾、繪、緯,口則繪、緯、渾。」齊文襄攝選,以緯為司徒諮議參軍,謂曰:「自郎署至此,所謂不次,以卿人才,故有此舉耳。」梁謝蘭來聘,勞之。蘭問安平諸崔,緯曰:「子玉以還,彫龍絕矣。」崔暹聞之怒。



 緯詣門謝之,暹上馬不顧。緯語人曰:「雖失要人意,聘梁使不得舍我。」武定五年,兼散騎常侍,使梁。緯常逸遊放達,自號「隱君」,蕭然有絕塵之意。使還,除太子家令,卒。齊初,贈北徐州刺史,謚曰文。



 璨字世顯,靈弟趙郡太守均之子也。身長八尺五寸,容
 貌魁偉。受學於梁祚,位中書郎,雅為高允所知。天安初,宋徐州刺史薛安都舉彭城降,詔鎮南大將軍博陵公尉元、鎮東將軍城陽公孔伯恭等迎之,獻文復以璨參二府軍事。安都率文武出迎,元不加禮接,安都還城,遂不降。宋將張永、沈攸之等先屯下磕,元令璨與中書郎高閭入彭城說安都,即與俱載赴軍。元等入城,收管籥。其夜,永攻南門,不剋退還。璨勸元乘永,永失據,攻永米船,大破之,於是遂定淮北。加璨寧朔將軍,與張讜對為兗州刺史,安帖初附。以參定徐州功,賜爵始豐侯,卒,謚曰懿。子元茂襲爵。



 元茂以寬雅著稱,位司徒司馬、彭城鎮
 副將,人吏安之。卒,贈顯武將軍、徐州刺史,謚曰順。子秀之,字鳳起,襲爵,位尚書都官郎。秀之弟子雲,字鳳昇;子雲弟子羽,字鳳降;子羽弟子岳,字鳳歭。秀之等並早孤,事母孝謹,兄弟容貌並魁偉,風度審正,而皆早卒。鳳昇子道宗,位直閣將軍。道宗弟德林,司徒中兵參軍。



 元茂弟宣茂,太和初,拜中書博士,後兼定州大中正,受鄉人財貨,為御史所劾,除名。正始初,除太中大夫,遷光祿勳。與游肇往復,肇善之。卒於幽州刺史,遺令薄葬,贈齊州刺史,謚曰惠。



 子籍之,字修遠,性謹正,粗涉書史。位司徒諮議參軍、太中大夫。著《忠誥》一篇,文多不載。卒,贈定州
 刺史。子徹,仕齊,位尚書左丞。徹子純,隋開皇中為介州長史。



 純子德饒,字世文。少聰敏好學,有至性。弱冠仕隋為校書郎,仍直內史省,參掌文翰。轉監察御史,糾正不避權貴。大業三年,遷司隸從事。每巡四方,理冤枉,褒孝悌。雖位秩未通,德行為當時所重。凡與交結,皆海內髦彥。



 性至孝,父母寢疾,輒終日不食,十旬不解衣。及丁憂,水漿不入口五日;哀慟,歐血數升。及送葬,會仲冬積雪,行四十餘里,單縗徒跣,號踴幾絕。會葬者千餘人,莫不為之流涕。後甘露降於庭樹,有鳩巢其廬,納言楊達巡省河
 北,詣廬弔慰之,因改所居村名為孝敬村,里為和順里。後為金河縣長,未之官,屬群盜蜂起,賊帥格謙、孫宣雅等十餘頭聚眾於勃海,有敕許其歸首。謙等懼,不敢降,以德饒信行有聞,遣奏曰:「若德饒來者,即相率歸首。」帝遣德饒往勃海慰諸賊。



 至冠氏,會他賊攻陷縣城,見害。



 其弟德佋,性重然諾。大業末為離石郡司法書佐,太守楊子崇特禮之。及義兵起,子崇遇害,棄尸城下。德佋赴哭盡哀,收瘞之。至介休,詣義師請葬子崇。見許,因贈子崇官,令德佋為使者,往離石禮葬子崇。徹弟公緒。



 公緒字穆叔,性聰敏,博通經傳。魏末為冀州司馬,屬疾
 去官,絕迹贊皇山。



 齊天保初,以侍御史徵,不就。公緒沈冥樂道,又不閑時務,故誓心不仕。尤明天文,善圖緯之學,嘗謂子弟曰:「吾觀齊之分野,福德不多,國家祚終四七。」及齊亡歲,距天保之元二十八年矣。公緒雅好著書,撰《典言》十卷、《禮質疑》五卷、《喪服章句》一卷、《古今略記》二十卷、《玄子》五卷、《趙記》八卷、《趙語》十二卷,並行於世。公緒既善陰陽之術,有祕記,傳之子孫而不好焉,臨終取以投火。子少通,有學行。



 公緒弟概,字季節,少好學。然性倨傲,每對諸兄弟,露髻披服,略無少長之禮。為齊文襄大將軍府行參軍,進側集,題云「富春公主撰」。閑緩不任事,
 每被譏訶。除殿中侍御史,修國史。後為太子舍人,為副使聘于江南。江南多以僧寺停客,出入常袒露。還,坐事解。後卒於並州功曹參軍。撰《戰國春秋》及《音譜》並行於世。又自簡詩賦二十四首,謂之《達生丈人集》。其序曰:「達生丈人者,生於戰國之世,爵里姓名無聞焉爾,時人揆其行己,彊為之號。頗好屬文,成輒棄槁。常持論文云:古人有言,性情生於慾。又曰人之性靜,慾實汨之。然則性也者,所受於天,神識是也,故為形骸之主;情也者所受於性,嗜慾是也,故為形骸之役。



 由此言之,情性之辯,斷焉殊異。故其身泰,則均齊死生,塵垢名利,縱酒恣色,所
 以養情;否,則屏除愛著,擯落枝體,收神反聽,所以養識。是以遇榮樂而無染,遭厄窮而不悶,或出人間,或栖物表,逍遙寄託,莫知所終。」



 李順,字德正,鉅鹿公靈之從父弟也。父系,慕容垂散騎侍郎、東武城令。道武定中原,以為平棘令。卒,贈趙郡太守、平棘男。



 順博涉經史,有計策。神瑞中,拜中書博士,轉中書侍郎。從征蠕蠕,以籌略,賜爵平棘子。太武將討赫連昌,謂崔浩曰:「朕前北征,李順獻策數事,實合經略大謀。今欲使總前驅之事,何如?」浩曰:「順智足周務,實如聖旨。但臣與之婚姻,深知其行,然性果於去就,不可專委。」
 帝乃止。初,浩弟娶順妹,又以弟子娶順女,雖婚媾,而浩頗輕順,順又不伏,由是潛相猜忌,故浩毀之。至統萬,大破昌軍,順謀功居多。後徵統萬,昌出逆戰,順破其左軍。及剋統萬,帝賜諸將珍寶雜物,順固辭,唯取書數千卷,帝善之。遷給事黃門侍郎。又從擊赫連定於平涼。



 三秦平,進爵為侯,遷四部尚書,甚見寵待。



 沮渠蒙遜以河西內附,帝欲簡行人,崔浩曰:「宜令清德重臣,奉詔褒慰,尚書順即其人也。」帝曰:「順納言大臣,不宜方為此使,若蒙遜身執玉帛而朝於朕,復何以加之?」浩曰:「邢貞使吳,亦魏之太常,茍事是宜,無嫌於重。」帝從之,以順為太常,策
 拜蒙遜為太傅,涼王。使還,拜使持節、都督四州諸軍事、長安鎮都大將、寧西將軍、開府,進爵高平公。未幾,徵為四部尚書,加散騎常侍。延和初,使涼。蒙遜辭疾,箕坐隱几,無起動狀。順正色大言曰:「不謂此叟無禮,乃至於是!」握節而出。蒙遜使中兵校郎楊定歸追順曰:「太常云朝廷賜不拜之詔,是以敢自安耳;若曰爾拜爾跽,而不承命,乃小臣之罪矣。」順曰:「齊桓公九合諸侯,一匡天下,周公賜胙,命曰伯舅無拜,而桓公降而拜受。今朝廷未有不拜之詔,而便偃蹇自取,此乃速禍之道。」蒙遜拜伏盡禮。



 順還,帝問與蒙遜往復辭,及其政教得失。順曰:「蒙遜
 專威河右,三十許年,經涉艱難,粗識機變,雖不能貽厥孫謀,猶足以終其一世。但前歲表許十月送曇無懺,及臣往迎,便乖本意,不臣不信,於是而甚。以臣觀之,不復周矣。」帝曰:「若如卿言,則效在無遠,襲世之後,早晚當滅。」對曰:「臣略見其子,並非才俊。如聞敦煌太守牧犍,器性粗立,若繼蒙遜,必此人也。然比之於父,僉云不逮,殆天所用資聖明也。」帝曰:「朕方事于東,未暇營西,如卿所言,三五年間,不足為晚。」及蒙遜死問至,太武謂順曰:「卿言蒙遜死,驗矣;又言牧犍立,何其妙哉!朕剋涼州,亦當不遠。」於是賜絹千匹、廄馬一乘,寵待彌厚,政無巨細,無所
 不參。崔浩惡之。



 順凡使涼州十二回,太武稱其能。而蒙遜數與順游宴,頗有悖言,恐順泄之,以金寶納順懷中,故蒙遜罪釁得不聞。又西域沙門曇無懺有方術,在涼州,詔追之。



 順受蒙遜金,聽殺之。浩並知之,密言於帝。帝未之信。太延三年,順復使涼州,及還,帝問以將平河右計,順以人勞既久,不可頻動,帝從之。五年,議征涼州,順以涼州乏水草,不宜遠征。崔浩固以為宜徵,帝從浩議。及至姑臧,甚豐水草,帝與景穆書,頗嫌順。後謂浩曰:「卿昔所言,今果驗矣。」克涼州後,聞受蒙遜金而聽其殺曇無懺,益嫌之。猶以寵舊,未加其罪,尚詔順差次群臣,賜
 以爵位。



 順頗受納,品第不平。涼州人徐桀發其事,浩又毀之。帝大怒,刑順於城西。順死後數年,其從父弟孝伯為太武知重,居中用事。及浩誅,帝怒甚,謂孝伯曰:「卿從兄往雖誤國,朕意亦未至此。由浩,遂殺卿從兄。」皇興初,順子敷等貴寵,獻文追贈順侍中、鎮西大將軍、太尉公、高平王,謚曰宣王。妻邢氏曰孝妃。順四子。



 長子敷,字景文。真君二年,選入中書教學,以忠謹給侍東宮,以為中散。與、盧遐、度世等並以聰敏內參機密。敷性謙恭,加有文學,文成寵遇之。遷秘書下大夫,賜爵平棘子。後兼錄南部,遷散騎常侍、南部尚書、中書監,領內外祕書,
 襲爵高平公。朝政大議,事無不關。及宋徐州刺史薛安都、司州刺史常珍奇等以彭城、懸瓠降,于時朝議謂未必可信,敷乃固執必然。乃遣師接援,淮海寧輯。敷既見待二世,兄弟親戚在朝者十餘人。弟弈又有寵於文明太后。列其隱罪二十餘條,獻文大怒,皇興四年,誅敷兄弟,削順位號為庶人。敷從弟顯德、妹夫廣平宋叔珍等皆坐關亂公私,同時伏法。敷兄弟敦崇孝義,家門有禮,至於居喪法度,吉凶書記,皆合典則,為北州所稱美。既致斯禍,時人歎惜之。



 敷弟式,字景則,學業知名。位西兗州刺史、濮陽侯。式自以家據權要,心慮危禍,常敕
 津吏,臺有使者,必先啟然後度之。既而使人卒至,始云南過,既濟,突入執式赴都,與兄俱死。



 子憲,字仲軌,清粹善風儀,好學有器度。太和初,襲爵,又降為伯。拜祕書中散,雅為孝文知賞。後拜趙郡太守。趙脩與其州里,脩歸葬父母也,牧守以下畏之累跡,憲不為屈,時人高之。後以黨附高肇,為御史所劾。正光五年,行雍州刺史,尋除七兵尚書。孝昌中,除征東將軍、揚州刺史、淮南大都督。及梁平北大將軍元樹等來寇,憲力屈而降。因求還國。既至,敕付廷尉。憲女婿安樂王鑒據相州反,靈太后謂鑒心懷劫脅,遂詔賜憲死。永熙中,贈儀同三司、尚書令、
 定州刺史,謚曰文靖。子希遠,字景沖,早卒。希遠子祖悛,襲祖爵。



 希遠弟希宗,字景玄。性寬和,儀貌雅麗,有才學。位金紫光祿大夫。齊神武擢為中外府長史。文宣帝納其第二女為皇后。位上黨太守,卒。贈司空公、殷州刺史,謚曰文簡。



 希宗長子祖昇,儀容瑰麗,垂手過膝,文學足以自通。位齊州刺史。淫於從兵妻,見殺。



 祖昇弟祖勳,位給事黃門侍郎。齊文宣以其女為濟南王妃。除侍中,封丹楊郡王,尋改封公。濟南即位,除趙州刺史。濟南廢,還除金紫光祿大夫。大寧中,昭信后有寵於武成,除齊州刺史。贓賄狼籍,坐免官。復起為光州刺史。祖勳性貪慢,
 兼其妻崔氏驕豪干政,時論鄙之。女侍中陸媼母元氏,即祖勛妻姨,為此附會,又除西兗州刺史、殿中尚書。祖勛無才幹,自少及長,居官無可稱述。卒,贈尚書右僕射。武平中,將封后兄君璧等為王,還復祖勳王爵。其弟祖欽封竟陵王,位光祿卿。祖勳第三弟祖納,兄弟中最有識尚,以經史被知,卒於散騎常侍。



 希宗弟希仁,字景山,有學識。卒於侍中、太子詹事。子公統,仕齊,位員外郎。高歸彥之反,公統為之謀主。歸彥敗,伏法。其母崔氏當沒官,其弟宣寶行賕,改籍注老。事發,武成帝棓殺之,肝腦塗地。



 希仁弟騫,字希義,博涉經史,文藻富贍。位散騎常
 侍、殷州大中正、尚書左丞。以本官兼散騎常侍使梁。後坐事免,論者以為非罪。騫嘗贈親友盧元明、魏收詩云:「監河愛升水,蘇子惜餘明。益州達友趣,廷尉辯交情。」蓋失職之志云。



 後除給事黃門侍郎,卒。其文筆別有集錄。齊受禪,贈儀同三司,謚曰文惠。



 騫弟希禮,字景節,性敦厚,容止樞機,動遵禮度。起家著作佐郎,脩起居注。



 歷位太常少卿,兼廷尉少卿,行魏尹事,豫州刺史。仍居議曹,與邢邵等議定禮律。



 卒於信州刺史。



 子孝貞,字元操,好學善屬文。仕齊,釋褐司徒府參軍事。與弟孝基同見吏部郎中陸昂。昂戲之曰:「弟名孝基,兄
 其替矣!」孝貞對曰:「禮雖不肖,請附子臧。」昂握手曰:「士固不妄有名,吾賢必當遠至。」簡靜,不妄通接賓客。射策甲科,拜給事中。稍遷兼通直散騎常侍,副李翥使陳。



 孝貞從姊則昭信皇后,從兄祖勳女為廢帝濟南王妃,祖欽女一為後主娥英,一為瑯邪王儼妃,祖勛叔騫女為安德王延宗妃。諸房子女,多有才貌,又因昭信后,所以與帝室姻媾重疊。兄弟並以文學自達,恥為外戚家。于時黃門侍郎高乾和親要用事,求婚於孝貞,孝貞拒之。由是有隙,陰譖之,出為太尉府外兵參軍。後歷中書舍人。



 武平中,出為博陵太守,不得志。尋為司州別駕。後復兼
 散騎常侍,騁周使副。



 還,除給事黃門侍郎,待詔文林館,假儀同三司。以美於詞令,敕與中書侍郎李若、李德林別掌宣傳詔敕。周武帝平齊,授儀同三司、小典祀下大夫。宣帝即位,轉吏部下大夫。隋文帝為丞相,孝貞從韋孝寬討尉遲迥,以功授上儀同三司。開皇初,拜馮翊太守,為犯廟諱,於是稱字元操。



 後數歲,遷蒙州刺史,吏人安之。自此不復留意文筆。人問其故,慨然嘆曰:「五十之年,倏焉已過,鬢垂素髮,筋力已衰,宦意文情,一時盡矣,悲夫!」然每暇日,輒引賓客,絃歌對酒,終日為歡。後徵拜內史侍郎,與內史令李德林參典文翰。元操無幹劇之
 用,頗稱不理。上譴怒之,敕御史劾其事。由是出為金州刺史,卒官。所著文集三十卷行於世,子元玉。



 元操弟孝基,亦有才學,風詞甚美。以衛尉丞待詔文林館,位儀曹郎中。孝基弟孝俊,太子洗馬。孝俊弟孝威,字季重,涉學有器幹,兄弟之中,最為敦篤。位太尉外兵參軍,修起居注。仕隋,禮部侍郎、大理少卿。



 式弟弈,字景世,美容貌,有才藝。位都官尚書、安平侯,與兄敷同死。太和初,文明太后追念弈兄弟,及誅,存問憲等一二家,歲時賜以布帛。



 弈弟冏,字道度,少為中散,逃避得免。後歷位度支尚書。太和二十一年,孝文幸長安,冏以咸陽山河險固,
 秦、漢舊都,勸帝去洛陽都之。後孝文引見冏,笑謂曰:「昔婁敬一說,漢祖即日西駕。尚書今以西京說朕,使朕不廢東轅。當是獻可理殊,所以今古相反耳。」冏曰:「昔漢祖起於布衣,欲藉險以自固,婁敬之言,符於本旨。今陛下德洽四海,事同隆周,是以愚臣獻說,不能上動。」帝大悅。



 冏性鯁烈,敢直言,常面折孝文,彈駁公卿,無所迴避,百寮皆憚之。孝文常加優禮,每車駕巡幸,恆兼尚書右僕射。雖才學不及諸兄,然公彊當世,堪濟過之。



 卒。



 子祐,字長禧,篤穆友于,見稱於世。歷位給事中,累遷博陵太守,所在亦以清乾著。順弟脩基,陳留太守,卒。子探幽,高平
 太守。探幽兄子洪鸞,河間太守。



 李孝伯,高平公順從父弟也。父曾,少以鄭氏《禮》、《左氏春秋》教授為業。



 郡三辟功曹,並不就,曰:「功曹之職,雖曰鄉選高第,猶是郡吏耳;北面事人,亦何容易。」州辟主簿,到官月餘,乃歎曰:「梁敬叔云『州郡之職,徒勞人耳』。



 道之不行,身之憂也。」遂還家講授。道武時,為趙郡太守,令行禁止。並州丁零數為山東害,知曾能得百姓死力,憚不入境。賊於常山界得一死鹿,賊長謂趙郡地也,責之,還令送鹿故處。郡謠曰:「詐作趙郡鹿,猶勝常山粟。」其見憚如此。



 卒,贈荊州刺史、柏仁子,謚曰懿。



 孝伯少傳父業,博綜
 群言,美風儀,動有法度。從兄言之太武,徵為中散,謂曰「真卿家千里駒也」。遷祕書奏事中散,轉散騎侍郎、光祿大夫,賜爵魏昌子。



 以軍國機密,甚見親寵,謀謨切祕,時人莫能知。遷北部尚書。以頻從征伐規略之功,進爵壽光侯。



 真君末,宋文帝聞車駕南伐,遣其弟太尉、江夏王義恭率眾赴彭城。太武至彭城,登亞父冢以望城內,遣送其俘蒯應至小市門,宣詔勞問。義恭等問應士馬數,曰:「中軍四十餘萬。」宋徐州刺史武陵王駿遣人獻酒二器、甘蔗百挺,并請駱駝。



 帝明旦復登亞父冢,遣孝伯至小市門,駿亦使其長史張暢對。孝伯曰:「主上有詔詔
 太尉、安北,可暫出門,欲與相見。今遣賜駱駝及貂裘雜物。」暢曰:「有詔之言,何得稱之於此?」孝伯曰:「卿家太尉、安北是人臣不?縱為鄰國之君,何為不稱詔於鄰國之臣?又何至杜門絕橋?」暢曰:「二王以魏帝營壘未立,此精甲十萬,恐輕相陵踐,故且閉城。待彼休息兵士,然後共脩戰場,剋日交戲。」孝伯曰:「令行禁止,主將常事,何用廢橋杜門?復何以十萬誇大?我亦有良馬百萬,復可以此相矜。」既開門,暢屏人卻仗,出受賜物。孝伯曰:「詔以貂裘賜太尉,駱駝騾馬賜安北。」義恭獻皮褲褶一具,駿奉酒二器、甘蔗百挺。帝又遣賜義恭、駿等氈各一領,鹽各九種,
 並胡豉。孝伯曰:「有後詔:凡此諸鹽,各有所宜。白鹽食鹽,主上自所食;黑鹽療腹脹氣滿,末之六銖,以酒而服;胡鹽療目痛;戎鹽療諸瘡;赤鹽、駁鹽、臭鹽、馬齒鹽四種,並非食鹽。太尉、安北,何不遣人來至朕間,見朕小大,知朕老少,觀朕為人?」暢曰:「魏帝為人,久為往來所具,故不復遣信。」義恭獻蠟燭十挺,駿獻錦一匹。



 孝伯風容閑雅,應答如流,暢及左右甚相嗟歎。帝大喜,進爵宣城公。為使持節、散騎常侍、秦州刺史,卒。贈征南大將軍、定州刺史,謚曰文昭公。



 孝伯體度恢雅,明達政事,朝野貴賤,咸推重之。景穆曾啟太武,廣徵俊秀,帝曰:「朕有一孝伯,足理
 天下,何用多為?假復求訪,此人輩亦何可得?」其見貴如此。性方慎忠厚,每朝廷事有所不足,必手自書表,切言陳諫。或不從者,至於再三,削滅槁草,家人不見。公廷論議,常引綱紀。或有言事者,孝伯恣其所陳,假有是非,終不抑折。及見帝,言其所長,初不隱人姓名,以為已善。故衣冠之士,服其雅正。自崔浩誅後,軍國謀謨,咸出孝伯。太武寵眷,有亞於浩,亦以宰輔遇之。獻替補闕,其迹不見,時人莫得而知。卒之日,遠近哀傷焉。孝伯美名,聞於遐邇。李彪使江南,齊武帝謂曰:「北有李孝伯,於卿遠近?」其為遠人所知若此。



 其妻崔賾女,高明婦人,生一子元
 顯。崔氏卒後納翟氏,不以為妻,憎忌元顯。



 後遇劫,元顯見害,世云翟氏所為也。元顯志氣甚高,為時人所傷惜。翟氏二子,安人、安上,並有風度。安人襲爵壽光侯,司徒司馬。無子,爵除。安上鉅鹿太守,亦早卒。安人第豹子後追理先封,卒不得襲。



 孝伯兄祥,字元善。學傳家業,鄉黨宗之。位中書博士。時尚書韓元興率眾出青州,以祥為軍司。略地至陳、汝,淮北之人詣軍降者七千餘戶,遷之兗、豫之南,置淮陽郡以撫之。拜祥太守,流人歸者萬餘家,百姓安業。遷河間太守,有威恩之稱。徵拜中書侍郎,人有千餘上書,乞留數年,朝廷不許。卒官,追贈定州刺
 史、平棘子,謚曰憲。



 子安世,幼聰悟。興安二年,文成帝引見侍郎、博士子,簡其秀俊,欲以為中書學生。安世年十一,帝見其尚小,引問之。安世陳說父祖,甚有次第,即以為生。



 帝每幸國學,恆獨被引問。詔曰:「汝但守此至大,不慮不富貴。」天安初,拜中散,以謹慎,帝親愛之。累遷主客令。



 齊使劉纘朝貢,安世奉詔勞之。安世美容貌,善舉止,纘等自相謂曰:「不有君子,其能國乎!」纘等呼安世為典客。安世曰:「何以亡秦之官,稱於上國?」



 纘曰:「世異之號,凡有幾也?」安世曰:「周謂掌客,秦改典客,漢名鴻臚,今曰主客。君等不欲影響文、武,而殷勤亡秦。」纘又指方山曰:「此
 山去燕然遠近?」



 安世曰:「亦石頭之與番禺耳。」



 時每有江南使至,多出藏內珍物,令都下富室好容服者貨之,令使任情交易。



 使至金玉肆問價,纘曰:「北方金玉大賤,當是山川所出?」安世曰:「聖朝不貴金玉,所以同於瓦礫;又皇上德通神明,山不愛寶,故川無金,山無玉。」纘初將大市,得安世言,慚而罷。遷主客給事中。



 時人困饑流散,豪右多有占奪,安世乃上疏陳均量之制,孝文深納之。後均田之制,起於此矣。出為相州刺史,假趙郡公。敦農桑,斷淫祀。西門豹、史起有功於人者,為之脩飾廟堂。表薦廣平宋翻、陽平路恃慶,皆為朝廷善士。初,廣平人李波
 宗族強盛,殘掠不已,前刺史薛道親往討之,大為波敗,遂為逋逃之藪,公私成患。百姓語曰:「李波小妹字雍容,褰裙逐馬如卷蓬。左射右射必疊雙,婦女尚如此,男子那可逢!」安世設方略,誘波及諸子姪三十餘人,斬於鄴市,州內肅然。病卒于家。



 安世妻博陵崔氏,生一子枿。崔氏以妒悍見出,又尚滄水公主,生二子,謚、郁。



 枿字琚羅,涉歷史傳,頗有文才,氣尚豪爽,公彊當世。太師、高陽王雍表薦枿為友。時人多絕戶為沙門,枿上言:「三千之罪,莫大於不孝,不孝之大,無過於絕祀。安得輕縱背禮之情,而肆其向法之意;缺當世之禮,而求將來之益;棄
 堂堂之政,而從鬼教乎?」沙門都統僧暹等忿枿鬼教之言,以枿為謗毀佛法,泣訴靈太后。責之,枿自理曰:「鬼神之名皆是通靈達稱。佛非天非地,本出於人,名之為鬼,愚謂非謗。」靈太后雖以枿言為允,然不免暹等意,猶罰枿金一兩。



 轉尚書郎,隨蕭寶夤西征,以枿為統軍。枿德洽鄉閭,招募雄勇,其樂從者數百騎。枿傾家賑恤,率之西討。寶夤見枿至,拊其肩曰:「子遠來,吾事辦矣。」



 故其下每有戰功,軍中號曰李公騎。寶夤啟枿為左丞,仍為別將,軍機戎政,皆與參決。寶夤又啟為中書侍郎。還朝,除岐州刺史,坐辭不赴任,免官。建義初,河陰遇害。初贈尚
 書右僕射、殷州刺史,後又贈散騎常侍、驃騎大將軍、儀同三司、冀州刺史。



 俶儻有大志,好飲酒,篤於親知。每謂弟郁曰:「士大夫學問,稽博古今而罷,何用專經為老博士也?」與弟謐特相友愛。謐在鄉物故,枿慟哭絕氣,久而方蘇,不食數日,期年形骸毀悴,人倫哀歎之。



 謐字永和,少好學,周覽百氏。初師事小學博士孔璠,數年後,璠還就謐請業。



 同門生為之語曰:「青成藍,藍謝青,師何常,在明經。」謐以公子徵拜著作佐郎,辭以授弟郁,詔許之。州再舉秀才,公府二辟,並不就。唯以琴書為業,有絕世之心。覽《考工記》、《大戴禮盛德篇》,以明堂之制不
 同,遂著《明堂制度論》曰:余謂論事辯物,當取正於經典之真文;援證定疑,必有驗於周、孔之遺訓,然後可以稱準的矣。今禮文殘缺,聖言靡存,明堂之制,誰使正之?是以後人紛糾,競興異論,五九之說,各信其習。是非無準,得失相半,故歷代紛紜,靡所取正。



 乃使裴頠云:「今群儒紛糾,互相掎摭,就令其象可得而圖。其所以居用之禮莫能通也,為設虛器耳。況漢氏所作,四維之個,復不能令各處其辰。愚以為尊祖配天,其義明著,廟宇之制,理據未分,直可為殿屋以崇嚴父之祀。其餘雜碎,一皆除之。」



 斯豈不以群儒舛互,並乖其實,據義求衷,莫適可從
 哉?但恨典文殘滅,求之靡據而已矣,乃復遂去室牖諸制。施之於教,未知其所隆政,求之於情,未可喻其所以必須,惜哉言乎!仲尼有言曰:「賜也,爾愛其羊,我愛其禮。」餘以為隆政必須其禮,豈彼一羊哉?推此而論,則聖人之於禮,殷勤而重之;裴頠之於禮,任意而忽之,是則頠賢於仲尼矣!以斯觀之,裴氏子以不達失禮之旨也。余竊不自量,頗有鄙意,據理尋義,以求其真;貴合雅衷,不茍偏信。乃藉之以《禮傳》,考之以訓注;博採先賢之言,廣搜通儒之說;量其當否,參其同異,棄其所短,收其所長,推義察圖,以折厥衷,豈敢必善,聊亦合其言志矣。



 凡
 論明堂之制者雖眾,然校其大略,則二途而已。言五室者,則據《周禮考工》之記以為本,是康成之徒所執。言九室者則案《大戴盛德》之篇以為源,是伯喈之倫所持。此二書雖非聖言,然是先賢之中博見洽通者也。但各記所聞,未能全正,可謂既盡美矣,未盡善也。而先儒不能考其當否,便各是所習,卒相非毀,豈達士之確論哉?小戴氏傳禮事四十九篇,號曰《禮記》,雖未能全當,然多得其衷,方之前賢,亦無愧矣。而《月令》、《玉藻》、《明堂》三篇,頗有明堂之義,餘故採掇二家,參之《月令》。以為明堂五室,古今通則。其室居中者,謂之太室;太室之東者,謂之青陽;當
 太室之南者,謂之明堂;太室之西者,謂之總章;當太室之北者,謂之玄堂。四面之室,各有夾房,謂之左右個,三十六戶七十二牖矣。室個之形,今之殿前是其遺像耳。個者,即寢之房也。但明堂與寢,施用既殊,故房個之名,亦隨事而遷耳。今粗書其像,以見鄙意,案圖察義,略可驗矣。故檢之五室,則義明於《考工》;校之戶牖,則數協於《盛德》;考之施用,則事著於《月令》;求之閏也,合《周禮》與《玉藻》。既同夏、殷,又符周、秦,雖乖眾儒,儻或在斯矣。《考工記》曰:「周人明堂,度以九尺之筵。東西九筵,南北七筵,堂崇一筵。五室,凡室二筵。室中度以几,堂上度以筵。」余謂《記》
 得之於五室,而謬於堂之脩廣。何者?當以理推之,令愜古今之情也。夫明堂者,蓋所以告月朔,布時令,宗文王,祀五帝者也。然營構之範,自當因宜創制耳。故五室者,合於五帝各居一室之義。且四時之祀,皆據其方之正,又聽朔布令,咸得其月之辰,可謂施政及俱,二三但允。求之古義,竊為當矣。鄭康成漢末之通儒,後學所取正。釋五室之位,謂土居中,木火金水各居四維。然四維之室既乖其正,施令聽朔各失厥衷,左右之個棄而不顧。乃反文之以美說,飾之以巧辭,言水木用事交於東北,木火用事交於東南,火土用事交於西南,金水用事交
 於西北。既依五行,當從其用事之交,出何經典?可謂工於異端,言非而博,疑誤後學,非所望於先儒也。《禮記·玉藻》曰:「天子聽朔於南門之外,閏月則闔門左扉,立於其中。」鄭玄注曰:「天子之廟及路寢皆如明堂制。明堂在國之陽,每月就其時之堂而聽朔焉。卒事反宿路寢,亦如之。閏月非常月,聽其朔於明堂門下,還處路寢門,終月也。」而《考工記》「周人明堂」,玄注曰:「或舉王寢,或舉明堂,互言之以明其制同也。」



 其同制之言,皆出鄭注。然則明堂與寢,不得異矣。而《尚書·顧命篇》曰:「迎子釗南門之外,延入翼室。」此之翼室,即路寢矣。其下曰:「大貝賁鼓在西房,
 垂之竹矢在東房。」此則路寢有左右房,見於經史者也。《禮記·喪服·大記》曰:「君夫人卒於路寢。小斂,婦人髽,帶麻於房中。」鄭玄注曰:「此蓋諸侯禮。帶麻於房中,則西南。天子諸侯。」左右房見於注者也。論路寢則明其左右,言明堂則闕其左右個,同制之說還相矛楯,通儒之注,何其然乎?使九室之徒奮筆而爭鋒者,豈不由處室之不當哉?



 《記》云:東西九筵,南北七筵。五室,凡室二筵。置五室於斯堂,雖使班、倕構思,王爾營度,則不能令三室不居其南北也。然則三室之間,便居六筵之地,而室壁之外,裁有四尺五寸之堂焉。豈有天子布政施令之所,宗祀文
 王以配上帝之堂,周公負扆以朝諸侯之處,而室戶之外,僅餘四尺而已哉?假在儉約,為陋過矣。



 論其堂宇,則偏而非制;求之道理,則未愜人情,其不然一也。



 余恐為鄭學者,茍求必勝,競生異端,以相訾抑,云二筵者乃室之東西耳,南北則狹焉。餘故備論之曰:若東西二筵,則室戶之外為丈三尺五寸矣。南北戶外復如此,則三室之中南北裁各丈二耳。《記》云:「四旁兩夾窗。」若為三尺之戶,二尺窗,窗戶之間,裁盈一尺。繩樞甕牖之室,篳門圭窬之堂,尚不然矣。假令復欲小廣之,則四面之外,闊狹不齊,東西既深,南北更淺,屋宇之制,不為通矣。



 驗之
 眾塗,略無算焉。且凡室二筵,丈八地耳,然則戶牖之間,不踰二尺也。《禮記·明堂》:「天子負斧扆南向而立。」鄭玄注曰:「設斧於戶牖之間。」而鄭氏《禮圖》說扆制曰:「從廣八尺,畫斧文於其上,今之屏風也。」以八尺扆置二尺之間,此之叵通,不待智者,較然可見矣。且若二筵之室為四尺之戶,則戶之兩頰裁各七尺耳,全以置之,猶自不容,矧復戶牖之間哉?其不然二也。



 又復以世代驗之,即虞、夏尚朴,殷、周稍文,制造之差,每加崇飾。而夏后世室,堂修二七,周人之制,反更促狹,豈是夏禹卑宮之意,周監郁郁之美哉?以斯察之,其不然三也。



 又云「堂崇一筵」,便基
 高九尺,而壁戶之外裁四尺五寸,於營制之法自不相稱,其不然四也。



 又云「室中度以几,堂上度以筵」,而復云「凡室二筵」,而不以几,還自相違,其不然五也。



 以此驗之,《記》者之謬,抑可見矣。《盛德篇》云:明堂凡九室、三十六戶、七十二牖,上員下方,東西九仞,南北十筵,堂高三尺也。餘謂《盛德篇》得之於戶牖,失之於九室。何者?五室之制,傍有夾房,面各有戶,戶有兩牖,此乃因事立則,非拘異術。戶牖之數,固自然矣。九室者,論之五帝,事既不合,施之時令,又失其辰,左右之個,重置一隅,兩辰同處,參差出入,斯乃義無所據,未足稱也。



 且又堂之修廣,裁六十
 三尺耳,假使四尺五寸為外之基,其中五十四尺便是五室之地,計其一室之中,僅可一丈,置其戶牖,則於何容之哉?若必小而為之,以容其數,則令帝王側身出入,斯為怪矣!此匪直不合典制,抑亦可哂之甚也。餘謂其九室之言,誠亦有由。然竊以為戴氏聞三十六戶七十二牖,弗見其制,靡知所置,便謂一室有四戶之窗,計其戶牖之數,即以為九室耳,或未之思也。蔡伯喈,漢末之時學士,而見重於當時,即識其脩廣之不當,而必未思其九室之為謬。更脩而廣之,假其法象。可謂因偽飾辭,順非而澤,諒可歎矣。餘今省彼眾家,委心從善,庶探其
 衷,不為茍異。但是古非今,俗間之常情;愛遠惡近,世中之恆事。而千載之下,獨論古制,驚俗之談,固延多誚。脫有深賞君子者,覽而揣之,儻或存焉。



 謐不飲酒,好音律,愛樂山水。高尚之情,長而彌固,一遇其賞,悠爾忘歸,乃作《神士賦》。延昌四年卒,年三十二,遐邇悼惜之。其年,四門小學博士孔璠等學官四十五人上書曰:竊見故處士趙郡李謐,十歲喪父,哀號罷鄰人之相;幼事兄枿,恭順盡友于之誠。十三通《孝經》、《論語》、《毛詩》、《尚書》,歷數之術,尤盡其長。州閭鄉黨,有神童之號。年十八,詣學受業時博士即孔璠也。覽始要終,論端究緒,授者無不欣其言
 矣。於是鳩集諸經,廣校同異,比《三傳》事例,名《春秋叢林》十有二卷。為璠等判析隱伏,垂盈百條。滯無常滯,纖豪必舉;通不長通,有枉斯屈。不茍言以違經,弗飾辭而背理,辭氣磊落,觀者忘疲。每曰:「丈夫擁書萬卷,何假南面百城。」遂絕跡下帷,杜門卻掃,棄產營書,手自刪削,卷無重復者四千有餘矣。猶括次專家,搜比黨議,隆科達曙,盛暑通宵。雖仲舒不窺園,君伯之閉戶,高氏之遺漂,張生之忘食,方之斯人,未足為喻。



 謐嘗詣故太常卿劉芳,推問音義,語及中代興廢之由。芳乃歎曰:「君若遇高祖,侍中、太常非僕有也。」前河南尹、黃門侍郎甄琛,內贊近
 機,朝野傾目,于時親識有求官者,答云:「趙郡李謐,耽學守道,不悶于時,常欲致言,但未有次耳。諸君何為輕自媒衒?」謂其子曰:「昔鄭玄、盧植不遠數千里詣扶風馬融,今汝明師甚邇,何不就業也?」又謂朝士曰:「甄琛行不愧時,但未薦李謐,以此負朝廷耳。」又結宇依巖,憑崖鑿室,方欲訓彼青衿,宣揚墳典,冀西河之教重興,北海之風不墜。而祐善空聞,暴疾而卒。邦國銜殄悴之哀,儒生結摧梁之慕,況璠等或服議下風,或親承音旨,師儒之義,其可默乎?



 事奏,詔曰:「謐屢辭徵辟,志守沖素,儒隱之操,深可嘉美。可遠傍惠、康,近準玄晏。謚曰:貞靜處士,并表
 其門閭,以旌高節。」於是表其門曰文德,里曰孝義云。



 郁字永穆,好學沈靖,博通經史。為廣平王懷友,深見禮遇。時學士徐遵明教授山東,生徒甚盛。懷徵遵明在館,令郁問其《五經》義例十餘條,遵明所答數條而已。稍遷國子博士。自國學之建,諸博士率不講說,其朝夕教授,唯郁而已。謙虛寬雅,甚有儒者之風。再遷通直散騎常侍。建義中,以兄枿卒,遂撫育孤姪,歸於鄉里。永熙初,除散騎常侍、衛大將軍、左光祿大夫,兼都官尚書,尋領給事黃門侍郎。三年,於顯陽殿講《禮記》,詔郁執經。郁解說不窮,群難鋒起,無廢談笑。孝武及諸王凡預聽者,莫不
 嗟善。尋病卒,贈散騎常侍、驃騎大將軍、尚書左僕射、儀同三司、都督、定州刺史。



 謐子士謙,字子約,一名容郎,髫齔喪父,事母以孝聞。母曾歐吐,疑中毒,因跪嘗之。伯父枿深所嗟尚,每稱:「此兒吾家顏子也。」年十二,魏廣平王贊辟開府參軍事。後丁母憂,居喪骨立。有姊適宋氏,不勝哀而死。士謙服闋,捨宅為伽藍。脫身而出,詣學請業,研精不倦,遂博覽群籍,善天文術數。齊吏部尚書辛術召署員外郎,趙郡王睿舉德行,皆稱疾不就。和士開亦重其名,將諷朝廷,擢為國子祭酒,因辭得免。刺史高元海以禮再致之,稱為菩
 薩。隋有天下,畢志不仕。



 自以少孤,未嘗飲酒食肉,口無殺害之言。親賓至,輒陳樽俎,對之危坐,終日不倦。



 李氏宗黨豪盛,每春秋二社,必高會極宴,無不沈醉喧亂。嘗集士謙所,盛饌盈前,而先為設黍。謂群從曰:「孔子稱黍為五穀之長,荀卿亦云食先黍稷,古人所尚,寧可違乎!」少長肅然,無敢弛惰,退而相謂曰:「既見君子,方覺吾徒之不德也。」士謙聞而自責曰:「何乃為人疏,頓至於此!」



 家富於財,躬處節儉,每以振施為務。州里有喪事。不均,至相鬩訟。士謙聞而出財補其少者,令與多者相埒。兄弟愧懼,更相推讓,卒為善士。有牛犯其田者,士謙牽置涼
 處,飼之過於本主。望見盜刈禾黍者,默而避之。其家僮嘗執盜粟者,士謙慰喻之曰:「窮困所致,義無相責。」遽令放之。其奴嘗與鄉人董震因醉角力,震扼其喉,斃於手下。震懼請罪,士謙謂曰:「卿本無殺心,何為相謝?然可速去,無為吏拘。」性寬厚皆此類也。後出粟萬石以貸鄉人,屬年穀不登,債家無以償,皆來致謝。士謙曰:「吾家餘粟,本圖賑贍,豈求利哉!」於是悉召債家,為設酒食,對之燔契,曰:「債了矣,幸勿為念也。」各令罷去。明年大熟,債家爭來償,士謙拒之,一無所受。他年饑,多有死者,士謙罄家資為之糜粥,賴以全活者萬計;收埋骸骨,所見無遺;至
 春,又出田糧種子,分給貧乏。趙郡農人德之,撫其子孫曰:「此李參軍遺惠也。」仁心感物,群犬生子,交共相乳。凶年散穀至萬餘石,合諸藥以救疾癘,如此積三十年。或謂士謙:「子多陰德。」士謙曰:「夫言陰德,其猶耳鳴,己獨知之,人無知者。今吾所作,吾子皆知,何陰德之有?」



 士謙善談玄理,嘗有客坐,不信佛家應報義。士謙喻之曰:「積善餘慶,積惡餘殃,豈非休咎邪?佛經云『轉輪五道,無復窮已』,此則賈誼所言『千變萬化,未始有極,忽然為人』之謂也。佛道未來,而賢者已知其然矣。至若鮌為黃熊,杜宇為鶗鴂,褒君為龍,牛哀為猛獸,君子為鵠,小人為猿,彭
 生為豕,如意為犬,黃母為黿,宣武為鱉,鄧艾為牛,徐伯為魚,鈴下為烏,書生為蛇,羊祜前身李氏之子,此非佛家變受異形之謂邪?」客曰:「邢子才云『豈有松柏後身,化為樗櫟』,僕以為然。」士謙曰:「此不類之談也,變化皆由心作,木豈有心乎?」客又問三教優劣,士謙曰:「佛,日也;道,月也;儒,五星也。」客亦不能難而止。



 士謙平生時時為詠懷詩,輒毀其本,不示人。又嘗論刑罰,遺文不具。其略曰:「帝王制法,沿革不同,自可損益,無為頓改。今之贓重者死,是酷而不懲也。語曰:『人不畏死,不可以死恐之。』愚謂此罪,宜從肉刑,刖其一趾;再犯者,斷其左腕。流刑刖去右
 手三指;又犯者,下其腕。小盜宜黥。又犯,刖落其所用三指;又不悛,則下其腕。無不止也。無賴之人,竄之邊裔,職為亂階,適所以召戎矣,非求安之道也。博弈淫遊,盜之萌也,禁而不止,黥之則可。」有識者頗以為得政體。隋開皇八年,終於家。趙州士女聞之,莫不流淚曰:「我曹不死而令李參軍死乎!」會葬者萬餘人。李景伯等以士謙道著丘園,條其行狀,詣尚書省請先生之謚,事寢不行,遂相與樹碑於墓。其妻范陽盧氏,亦有婦德。及夫終,所有賻贈,一無所受。謂州里父老曰:「參軍平生好施,今雖殞歿,安可奪其志哉!」乃散粟五百石以賑窮乏,免奴婢
 六十人。



 案趙郡李氏,出自趙將武安君牧。當楚、漢之際,廣武君左車則其先也。左車十四世孫恢,字仲興,漢桓、靈間,高尚不仕,號有道大夫。恢生定,字文義,仕魏,位漁陽太守。有子四人,並仕晉。平字伯括,為樂平太守;機字仲括,位國子博士;隱字叔括,保字季括,位並尚書郎。兄弟皆以儒素著名,時謂之四括。



 機子楷,字雄方,位書侍御史,家于平棘南。有男子五人,輯、晃、棨、勁、睿。輯字護宗,晃字仲黃,棨字季黃,勁字少黃,睿字幼黃,並以友悌著美,為當世所宗,時所謂四黃者也。輯位高密郡守,二子,慎、敦。晃位鎮南府長史,一子,義。勁位書侍御史,四子,盛、
 敏、隆、喜。睿位高平太守,二子,勖、充。其後,慎、敦居柏仁,子孫甚微。義南徙故壘,世謂之南祖。勖兄弟居巷東,盛兄弟居巷西,世人指其所居,因以為目,蓋自此也。義字敬仲,位司空長史。生東宮舍人吉,字彥同。吉生尚書郎聰,字小時。聰生真,字。義深事列于後。勖字景賢,位頓丘太守。勖生趙郡太守頤,字彥祖。頤生勰、系、曾,各有令子,事並列于前。盛位中書郎。三子,纘、襲、閣。纘字緯業,位太尉祭酒。生四子,誕、休、重、苞。



 誕字紹元,假趙郡太守。生四子,建、追、磪、龜。龜字神龜,位州主簿。生二子,鳳林、秀林。



 李裔,字徽伯。父秀林,小名榼,性溫直。太和中,中書博
 士,為頓丘相,豪右畏之。景明初,試守博陵郡,抑彊扶弱,政以嚴威為名。以母憂去職。後為司徒司馬、定州大中正、太中大夫。卒,贈齊州刺史。裔出後伯父鳳林。孝昌中為定州鎮軍長史,帶博陵太守。於時逆賊杜洛周侵亂州界,裔潛引洛周,州遂陷沒。洛周特無綱紀,至于市令、驛帥咸以為王,呼曰市王、驛王,乃封裔定州王。洛周尋為葛榮所滅,裔仍事榮。爾朱榮禽葛榮,遂縶裔及高昂、薛脩義、李無為等於晉陽。



 從榮至洛,榮死乃免。天平初,以齊神武大丞相諮議參軍,參定策功,封固安縣伯,為候衛大將軍、陜州刺史。及周文帝攻剋州城,見害。東魏
 贈尚書令、司徒、定州刺史。子子旦襲。子旦弟子雄。



 子雄少慷慨有大志,陜州破,因隨周軍入長安。家世並以學業自通,子雄獨習騎射。其兄子旦讓之曰:「棄文尚武,非士大夫素業。」子雄曰:「自古誠臣貴仕,文武不備而能濟功業者鮮矣。既文且武,兄何病焉。」子旦無以應。仕周,累遷小賓部。後從達奚武與齊人戰於芒山,諸軍大破,子雄所領獨全。累遷涼州總管長史。



 從滕王逌破吐谷渾於青海,以功加上儀同。宣帝即位,行軍總管韋孝寬略定淮南,拜亳州刺史。隋文帝總百揆,徵為司會中大夫,以淮南功,加位上開府。及受禪,拜鴻臚卿,進爵高
 都郡公。



 及晉王廣出鎮并州,以子雄為河北行臺兵部尚書。上謂曰:「吾兒既少,卿兼文武之才,今者推誠相委,吾無北顧憂矣。」子雄頓首流涕,誓以效命。子雄當官正直,侃然有不可犯色,王甚敬憚,吏人稱焉。歲餘,卒官。子公挺嗣。



 裔從祖詵字令世,誕弟休之子也。休字紹則,散騎常侍。詵與族兄靈、族弟熙等俱被徵,事在高允《征士頌》。詵位中書侍郎、京兆太守。詵從祖弟善見,位趙郡太守。善見子顯進,位州主簿、濮陽太守。



 顯進子暎,字暉道,位相州中從事、步兵校尉,贈殷州刺史。暎子普濟,學涉有名,性和韻,位濟北太守,時人語曰「入粗入細李普濟」。
 武定中,位北海太守。



 暎弟育,字仲遠,位相州防城別將,以拒葛榮之勳,賜爵趙郡公。後除金紫光祿大夫,卒,贈都官尚書,謚曰貞。子愔襲,與從父兄普濟並應秀才舉,時人謂其所居為秀才村。



 愔位太子舍人。



 愔族叔肅,字彥邕,位員外常侍。初諂附侍中元暉。後以左道事侍中穆紹。常裸身被髮,畫復銜刀,於隱屏處為紹求福。故紹愛之,薦為黃門郎。性酒狂,從靈太后幸江陽王繼第,侍飲頗醉,言辭不遜,抗辱太傅、清河王懌。為有司彈劾,太后恕之。卒於夏州刺史。



 肅從弟皦,字景林,有學識,位廷尉少卿,贈齊州刺史,謚曰宣。子慎,武定中,位東平太守。



 皦從弟仲旋。司徒左長史、恆農太守。先是宮、牛二姓阻險為害,仲旋示以威惠,即並歸伏。累遷左光祿大夫。天平初,遷都於鄴,以仲FM為營構將,進號衛大將軍。出為兗州刺史,還除將作大匠,所歷並著聲績。卒,贈驃騎大將軍、儀同三司、青州刺史。子希良,侍御史。



 煥字仲文,小字醜瑰,中書侍郎盛弟隆之後也。隆字太彞,位阜城令。隆生幕縣令謀。謀生始平太守景,名犯太祖元皇帝諱。景生東郡太守伯應。伯應生煥。煥有乾用,與酈道元俱為李彪所知。恆州刺史穆泰據代都謀反,煥以書侍御史與任城王澄推究之。煥先驅至州,宣旨曉喻,乃執泰
 等。景明初,齊豫州刺史裴叔業以壽春歸附,煥以司空從事中郎為軍司馬,與楊大眼、奚康生等迎接,仍行揚州事,賜爵容城伯。及荊蠻擾動,敕煥兼通直散騎常侍慰勞之,降者萬餘家。除梁州刺史。



 時武興氐楊集起舉兵作逆,敕假煥平西將軍,督別將大破集起軍。又破秦州賊呂茍兒,及斬氐王楊定。還朝,遇患卒,贈幽州刺史,謚曰昭。



 子密,字希邕,少有節操。母患積年,名醫療之不愈,乃精習經方,洞閑針藥,母疾得除。由是以醫術知名。屬爾朱兆弒逆,與勃海高昂為報復計。後從神武,封容城縣侯,位襄州刺史。



 李義深,趙郡高邑人也。祖真,字令才,位中書侍郎。父紹,字嗣宗,殷州別駕。義深有當世才用,而心胸險峭,時人語曰:「劍戟森森李義深。」初以殷州別駕歸齊神武,再遷鴻臚少卿。見爾朱兆兵盛,叛歸之。兆平,神武恕其罪。遷齊州刺史,好利,多所受納。轉行梁州刺史,為陽夏太守段業告其在州聚斂,被禁止。



 卒於禁所。



 子騊駼,有才辯,位兼通直散騎常侍,聘陳。陳人稱之。後為壽陽道行臺左丞,與王琳同陷陳。周末逃歸。隋開皇中為永安郡太守、絳州長史,卒。



 子政藻,明敏有才幹。騊駼沒陳,政藻時為開府行參軍,判集書省事,便謝病解職,居處若在喪
 禮,人士稱之。開皇中,歷尚書工部員外郎,卒於宜州長史。



 騊駼弟文師,歷中書舍人,齊郡太守。



 義深弟同軌,體貌魁岸,腰帶十圍,學綜諸經,兼該釋氏,又好醫術。年二十,舉秀才,再遷著作郎,典儀注,脩國子博士。興和中,兼通直散騎常侍,使梁。梁武深耽釋學,遂集名僧於其愛敬、同泰二寺,講《涅般大品經》,引同軌豫席,兼遣其朝士議共觀聽,同軌論難久之,道俗咸以為善。盧景裕卒,齊神武引同軌在館教諸公子,甚嘉禮之。每旦入授,日暮始歸,緇素請業者,同軌夜為解說,四時恆爾,不以為倦。卒,時人傷惜之,神武亦嗟悼之。贈瀛州刺史,謚曰康。



 同
 軌弟幼舉,安德太守,以貪汙棄市。幼舉弟之良,有乾用,位金部郎中。



 之良弟幼廉,少寡欲,為兒童時,初不從人家有所求請。嘗故以金寶授之,終不取,彊付,輒擲之地。州牧以其蒙幼而廉,故以名焉。性聰敏,累遷齊文襄驃騎府長史。文襄薦為濟州儀同府長史,又遷瀛州長史。齊神武行經冀部,總合河北六州文籍,商榷戶口增損,親自部分,多在馬上征責文簿,指影取備,事非一緒。幼廉應機立成,恆先期會,為諸州準的。神武深加慰勉,乃責諸人曰:「碎卿等諸人,作得李長史一腳指不!」是時諸人並謝罪,幼
 廉獨前拜恩,觀者咸歎美之。神武還并州,以告文襄,文襄喜謂人曰:「吾是知人矣!」文襄嗣事,除霸府掾。時以并州王政所基,求好長史,舉者多不見納。後因大集,謂陳元康曰:「我教你好長史處,李幼廉即其人也。」遂命為并州長史。常在文襄第內,與隴西辛術等六人,號為館客。天保初,除太原郡太守。文宣嘗與語及楊愔,誤稱為楊公,以應對失宜,除濟陰郡守。累遷太僕大司農二卿、趙州大中正、大理卿,所在稱職。



 後主時,和士開權重,百僚盡傾,幼廉高揖而已,由是出為南青州刺史。主簿徐乾富而暴橫,歷政不能禁。幼廉初至,因其有犯,收繫之。乾
 密通疏,奉黃金百挺、妓婢二十人,幼廉不受,遂殺之。罷還鄴。祖孝徵執政。求紫石英於幼廉,以其南青州所出。幼廉辭無好者,固請,乃與二兩。孝徵有不平之言,或以告幼廉。



 幼廉抗聲曰:「李幼廉結髮從宦,誓不曲意求人。天生德於予,孝徵其如予何?假欲挫頓,不過遣向并州耳。」時已授并省都官尚書,辭而未報,遂發敕遣之。齊末官至三品已上,悉加儀同,獨不霑此例,語人曰:「我不作儀同,更覺為榮。」卒,贈吏部尚書。



 義深族弟神威,幼有風裁,家業《禮》學,又善音樂,撰集樂書近百卷,卒於尚書左丞。



 又有李翥,字彥鴻,世居柏仁,弱冠以文章知。仁齊,位
 東平太守。後待詔文林館,除通直散騎常侍,聘于梁。晚節頗以貪酒為累。貪無居宅,寄止佛寺中。嘗著巾帔,終日對酒,招致賓客,風調詳雅。翥從兄子朗,才辭翥之亞,兼有吏能,位中書舍人。



 論曰:古人云「燕、趙多奇士」,觀夫李靈兄弟,並有焉。靈則首應弓旌,道光師傅。順則器標楝幹,一時推重。孝伯風範鑒略,蓋亦過人。各能克廣門業,道風不殞,餘慶之美,豈非此之謂乎。至如元忠之倜儻從橫,功名自卒;季初之家風素業,昆季兼舉。有齊之日,雅道方振。憲之子弟,特盛衣纓,豈唯戚里是憑,固亦文雅所得。安世識具通
 雅,時幹之良。枿以豪俊達,鬱則儒博顯,謐之高逸,固可謂世有人焉。義深弟兄,人位兼美;子雄才官,不替門緒,茂矣。



\end{pinyinscope}