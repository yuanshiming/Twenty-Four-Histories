\article{卷三十四列傳第二十二}

\begin{pinyinscope}

 游雅從祖弟明
 根
 高
 閭趙逸兄子琰胡叟胡方回張湛段承根宗欽闞駰劉延明趙柔索敞宋繇孫遊道江式游雅,字伯度,小名黃頭,廣平任人也。太武時,與勃海高允等俱知名,徵拜中書博士。後使宋,授散騎侍郎,賜爵
 廣平子。稍遷太子少傅,領禁兵,進爵為侯。



 受詔與中書侍郎胡方回等改定律制。出為東雍州刺史,假梁郡公。在任廉白,甚有惠政。徵為秘書監,委以國史之任,竟無所成。雅性剛戇,好自矜誕,凌獵人物。



 高允重雅文學,而雅輕允才,允性柔寬,不以為恨。允將婚于邢氏,雅勸允娶其族,允不從。雅曰:「人貴河間邢,不勝廣平游;人自棄伯度,我自敬黃頭。」其貴己賤人,皆此類也。允著《征士頌》,殊重雅。雅因議論長短,忿儒者陳奇,遂陷奇至族。議者深責之。卒,贈相州刺史,謚曰宣侯。



 明根字志遠,雅從祖弟也。祖鱓,慕容熙樂浪太守。父幼,
 馮跋假廣平太守。



 明根幼年遭亂,為櫟陽王氏奴。主使牧羊,明根以漿壺倩人書字路邊,書地學之。長安鎮將竇瑾見之,呼問,知其姓名,乃告游雅。雅使人贖之,教書。年十六,辭雅歸鄉里,於白渠坎為窟,讀書積歲。雅稱薦之,太武擢為中書學生。性寡欲,綜習經史。文成踐阼,為都曹主書。帝以其敬慎,每嗟美之。假員外散騎常侍、安樂侯,使宋。宋孝武稱其長者,迎送禮加常使。獻文時,累遷東兗州刺史,封新泰侯,為政清平。孝文時為儀曹長,清約恭謹,號為稱職。歷儀曹尚書,加散騎常侍。



 遷大鴻臚卿、河南王幹師,尚書如故,隨例降侯為伯。又參定律
 令,屢進讜言。



 明根以年踰七十,表求致仕,優詔許之。引入陳謝,悲不自勝。帝言別殷勤,仍為流涕,賜青紗單衣、委貌冠、被褥、錦袍等物。其年,以司徒尉元為三老,明根為五更,行禮辟雍,賜步挽一乘,給上卿祿,供食之味,太官就第月送。以定律令。賜布帛等。歸本郡,又賜安車、兩馬、幄帳、被褥。車駕幸鄴,明根朝于行宮,優詔賜以穀帛,敕太官備送珍羞,為造甲第。國有大事,恆璽書訪之。舊疾發動,手詔問疾,太醫送藥。卒於家,宣武弔祭贈賵甚厚,贈光祿大夫,金章紫綬,謚靖侯。



 明根歷官內外五十餘年,處身以仁和,接物以禮讓,時論貴之。孝文初,明根
 與高閭以儒老學業,特被禮遇,公私出入,每相追隨;而閭以才筆,時侮明根。世號高、游焉。



 子肇襲,字伯始,孝文賜名焉。博綜經史。孝文初,為內秘書侍御中散,稍遷典命中大夫。車駕南伐,肇表諫,不納。尋遷太子中庶子。肇謙素敦重,文雅見任。



 以父老,求解官扶侍。孝文欲令祿養,出為本州南安王楨鎮北府長史,帶魏郡太守。



 王薨,復為高陽王雍鎮北府長史,太守如故。為政清簡,加以匡贊,歷佐二王,甚有聲績。以父憂解任。復授黃門侍郎,兼侍中,為畿內大使,黜陟善惡,賞罰分明。



 歷太府、廷尉卿,兼御史中尉,黃門如故。肇儒者,動存名教,直繩所舉,
 莫非傷風敗俗。持法仁平,斷獄務於矜恕。尚書令高肇,宣武之舅,百僚懾憚,以肇名與己同,欲令改易。肇以孝文所賜,執志不許,高肇甚銜之,宣武嘉其剛梗。



 盧昶之在朐山也,肇諫曰:「朐山蕞爾,僻在海濱,於我非急,於賊為利。如聞賊將屢以宿豫求易朐山,持此無用之地,復彼舊有之疆,兵役時解,其利為大。」



 帝將從之,尋而昶敗。遷侍中。梁軍主徐玄明斬其青、冀二州刺史張稷首,以郁州內附。朝議遣兵赴援,肇表以為不宜勞師爭海島之地,帝不納。及大將軍高肇伐蜀,肇又陳願俟後圖,又不納。明帝即位,遷中書令、相州刺史,有惠政。再遷尚書
 右僕射。肇於吏事斷決不速,主者諮呈反覆,至於再三,必窮其理,然後下筆。雖寵勢干請,終無迴撓,方正之操,時人服之。及元叉廢靈太后,將害太傅、清河王懌,乃集公卿會議其事。於時,君官莫不失色順旨,肇獨抗言,以為不可,終不下署。



 卒,謚文貞公。



 肇外寬柔,內剛直,耽好經傳,手不釋書。善《周易》、《毛詩》,尤精《三禮》。為《易集解》,撰《冠婚儀》、《白珪論》,詩賦表啟凡七十五篇。謙廉不競,曾撰《儒棋》,以表其志。清貧寡欲,資仰俸祿而已。為廷尉時,宣武嘗敕肇有所降恕,執而不從,曰:「陛下自能恕之,豈可令臣曲筆也。」其執意如此。及明帝初,近侍群官預在奉迎
 者,自侍中崔光以下並加封,封肇文安縣侯。肇獨曰:「子襲父位,今古之常,因此獲封,何以自處?」固辭不應。論者高之。



 子祥,字宗良,頗有才學。襲爵新泰伯,位國子博士,領尚書郎中。明帝以肇昔辭文安之封,復欲封祥,祥守其父志,卒不受。又追論肇前議清河,守正不屈,乃封祥高邑縣侯。卒,贈給事黃門侍郎、幽州刺史,謚曰文。



 高閭,字閻士,漁陽雍奴人也。五世祖原,晉安北將軍、上谷太守、關中侯,有碑在薊中。祖雅,少有令名,位州別駕。父洪,字季願,位陳留王從事中郎。閭貴,乃贈幽州刺史、固安貞子。



 閭早孤,少好學,博綜經史,下筆成章。少為車
 子,送租至平城,脩剌詣崔浩。



 浩與語奇之,使為謝中書監表。明日,浩歷租車過,駐馬呼閭,諸車子皆驚。閭本名驢,浩乃改為閭,而字焉,由是知名。和平末,為中書侍郎。文成崩,乙渾擅權,內外危懼,文明太后臨朝誅渾,引閭與中書令高允入禁中參決大政,賜爵安樂子。



 與鎮南大將軍尉元南赴徐州,以功進爵為侯。獻文即位,徙崇光宮,閭表上《至德頌》。高允以閭文章富逸,舉以自代,遂為獻文所知,參論政事。永明初,為中書令、給事中,委以機密。文明太后甚重閭,詔令書檄碑銘贊頌皆其文也。太和三年,出師討淮北,閭表諫,陳四疑,請時速返旆。文
 明太后曰:「六軍電發,有若摧朽,何慮四難也。」遷尚書、中書監。淮南王他奏求依舊斷祿,閭表以為若不班祿,則貪者肆其姦情,清者不能自保,詔從閭議。



 孝文又引見王公以下於皇信堂,令辯忠佞。閭曰:「佞者飾知以行事,忠者發心以附道,譬如玉石,皦然可知。」帝曰:「玉石同體而異名,忠佞異名而同理。



 求之於同,則得其所以異;尋之於異,則失其所以同。出處同異之間,交換忠佞之境,豈是皦然易明哉?或有託佞以成忠,或有假忠以飾佞,如楚之子綦,後事雖忠,初非佞也?」閭曰:「子綦諫楚,初雖隨述,終致忠言,此適欲幾諫,非為佞也。



 子綦若不設初
 權,後忠無由得顯。」帝善閭對。後上表曰:臣聞為國之道,其要有五:一曰文德,二曰武功,三曰法度,四曰防固,五曰刑賞。故遠人不服,則脩文德以來之;荒狡放命,則播武功以威之;人未知戰,則制法度以齊之;暴敵輕侵,則設防固以禦之;臨事制勝,則明賞罰以勸之。用能闢國寧方,征伐四剋。北狄悍愚,同於禽獸,所長者野戰,所短者攻城。若以狄之所短,奪其所長,則雖眾不能成患,雖來不能內逼。又狄散居野澤,隨逐水草,戰則與室家並至,奔則與畜牧俱逃。是以古人伐北方,攘其侵掠而已。歷代為邊患者,良以倏忽無常故也。六鎮勢分,倍眾不
 鬥,互相圍逼,難以制之。昔周命南仲,城彼朔方,趙靈、秦始,長城是築;漢之孝武,踵其前事。此四代之君,皆帝王之雄傑,所以同此役者,非智術之不長,兵眾之不足,乃防狄之要事,理宜然也。



 今故宜於六鎮之北築長城,以禦北虜,雖有暫勞之勤,乃有永逸之益。即於要害,往往開門,造小城於其側,因施卻敵,多置弓弩。狄來,有城可守,有兵可捍。



 既不攻城,野掠無獲,草盡則走,終始必懲。又宜發近州武勇四萬人,及京師二萬人,合六萬人,為武士;於苑內立征北大將軍府,選忠勇有志乾者以充其選。下置官屬,分為三軍:二萬人專習弓射,二萬人專
 習刀楯,二萬人專習騎槊。修立戰場,十日一習。採諸葛亮八陣之法,為平地禦敵之方,使其解兵革之宜,識旌旗之節。



 兵器精堅,必堪禦寇。使將有定兵,兵有常主,上下相信,晝夜如一。七月,發六郡兵萬人,各備戎作之具,敕臺北諸屯倉庫,隨近往來,俱送北鎮。至八月,征北部率所鎮,與六鎮之兵,直至磧南,揚威漠北。狄若來拒,與決戰;若其不來,然後散分其地,以築長城。計六鎮,東西不過千里,若一夫一月之功當二步之地,三百人三里,三千人三十里,三萬人三百里,則千里之地,強弱相兼,計十萬人一月必就。軍糧一月,不足為多,人懷永逸,勞
 而無怨。計築長城,其利有五:罷遊防之苦,其利一也;北部放牧,無抄掠之患,其利二也;登城觀敵,以逸待勞,其利三也;省境防之虞,息無時之備,其利四也;歲常遊運,永得不匱,其利五也。



 孝文詔曰:「比當與卿面論。」又詔閭為書問蠕蠕。時蠕蠕國有喪而書不敘凶事。帝曰:「卿職典文辭,不論彼之凶事,若知而不作,罪在灼然;若情思不至,應謝所任。」對曰:「昔蠕蠕主敦崇和親,其子屢犯邊境,如臣愚見,謂不宜弔。」



 帝曰:「敬其父則子悅,敬其君則臣悅,卿云不合弔慰,是何言歟?」閭遂免冠謝罪。帝曰:「蠕蠕使牟提,小心恭慎,同行疾其敦厚,恐其還北,必被謗
 誣。昔劉準使殷靈誕,每禁下人不為非禮事,及還,果被譖愬,以致極刑。今書可明牟提忠於其國,使蠕蠕主知之。」



 是年冬至,大饗群官,孝文親舞於太后前,群臣皆舞。帝乃長歌,仍率群臣再拜上壽。閭進曰:「臣聞大夫行孝,行合一家;諸侯行孝,聲著一國;天子行孝,德被四海。今陛下敦行孝道,臣等不勝慶踴,謹上千萬歲壽。」帝大悅。又議政於皇信堂,閭曰:「伏思太皇太后十八條之令,及仰尋聖朝所行,事周於百揆,願終成其事。」帝曰:「刑法者,王道所用。何者為法?何者為刑?施行之日,何先何後?」對曰:「刑制之會,軌物齊眾,謂之法;犯違制約,致之於憲,謂
 之刑。然則法必先施,刑必後著。」帝曰:「《論語》稱:冉子退朝,孔子曰:『何晏也?』曰:『有政。』子曰:『其事也,如其有政,雖不吾以,吾其與聞之。』何者為政?



 何者為事?」對曰:「政者,上之所行;事者,下之所綜。」後詔閭與太常採雅樂以營金石。又領廣陵王師,出除鎮南將軍、相州刺史。以參定律令之勤,賜布帛粟牛馬等。遷都洛陽,閭表諫,言遷有十損,必不獲已,請遷於鄴。帝頗嫌之。



 雍州刺史曹武據襄陽請降,車駕親幸懸瓠。閭表諫:洛陽草創,武既不遣質任,必非誠心,帝不納。武果虛詐,諸將皆無功而還。車駕還幸石濟,閭朝於行宮。帝謂曰:「朕往年之意,不欲決徵。但兵
 士已集,恐為幽王之失,不容中止,遂至淮南。而彼諸將並列州鎮,至無所獲,實由晚一月日故耳。」閭曰:「古攻戰法,倍則攻之,十則圍之。聖駕親征,誠應大捷,所以無大獲,良由兵少故也。今京邑甫爾,庶事造創,願陛下當從容伊、瀍,使德被四海。」帝曰:「願從容伊、瀍,實亦不少,但未獲耳。」閭曰:「司馬相如臨終,恨不封禪。今雖江介不賓,然中州地略以盡平,豈可聖明之辰,而闕盛禮?」帝曰:「荊揚未一,豈得如卿言也?」



 閭以江南非中國,且三代之境,亦不能遠。帝曰:「淮海惟揚州,荊及衡陽惟荊州,此非近中國乎?」



 及車駕至鄴,孝文頻幸其州館,下詔褒揚之。閭每
 請本州以自效,詔曰:「閭以懸車之年,方求衣錦。知進忘退,有塵謙德,可降號平北將軍。朝之老成,宜遂情願,徙授幽州刺史,令存勸兼行,恩法並舉。」閭以諸州罷從事,依府置參軍,於政體不便,表宜復舊。帝不悅。歲餘,表求致仕,優答不許。徵為太常卿,頻表陳遜,不聽。又車駕南討漢陽,閭上表諫求迴師,帝不納。漢陽平,賜閭璽書,閭上表陳謝。



 宣武踐阼,閭累表遜位,優詔授光祿大夫,金章紫綬;使吏部尚書邢巒就家拜授。及辭,引見東堂,賜以肴羞,訪之大政。以其先朝儒舊,告老求歸,帝為之流涕。優詔賜安車、几杖、輿馬、繪彩、衣服、布帛,事從豐厚。百
 寮餞之,猶群公之祖二疏也。閭進陟北芒,上《望闕表》以示戀慕之誠。卒於家,謚文貞。



 閭好為文章,集四十卷。其文亦高允之流,後稱二高,為當時所服。閭強果敢直諫,其在私室,言裁聞耳;及於朝廷廣眾之中,則談論鋒起,人莫能敵。孝文以其文雅之美,每優禮之。然貪褊矜慢。初在中書,好詈辱諸博士。學生百餘人,有所干求者,無不受其賄。及老為二州,乃更廉儉自謹,有良牧之譽。子元昌襲爵,位遼西、博陵二郡太守。閭弟悅,篤志好學,有美於閭,早卒。



 趙逸,字思群,天水人也。父昌,石勒黃門郎。逸好學夙成,
 仕姚興,歷中書侍郎。後為赫連屈丐所虜,拜著作郎。太武平統萬,見逸所著,曰:「此豎無道,安得為此言乎!作者誰也?速推之。」司徒崔浩進曰:「彼之謬述,亦子雲《美新》,固宜容之。」帝乃止。歷中書侍郎、赤城鎮將,頻表乞免,久乃見許。性好墳典,白首彌勤,年踰七十,手不釋卷,凡所著述,詩賦銘頌五十餘篇。



 逸兄溫,字思恭,博學有高名,為姚泓天水太守。劉裕滅泓,遂沒於氐。氐王楊難當稱籓,太武以溫為難當府司馬,卒于仇池令。



 溫子琰,字叔起。初,苻氏亂,琰為乳母攜奔壽春,年十四乃歸。孝心色養,飪熟之節,必親調之。皇
 興中,京師儉,婢簡粟糶之,琰遇見,切責,敕留輕比。



 嘗送子應冀州娉室,從者於路遇得一羊,行三十里而琰知之,令送於本處。又過路旁,主人設羊羹,琰訪知盜殺,卒辭不食。遣人買耜刃,得剩六耜,即命送還刃主。



 刃主高之,義而不受,琰命委之而去。初為兗州司馬,轉團城鎮副將。還京,為淮南王他府長史。時禁制甚嚴,不聽越關葬於舊兆,琰積四十餘年不得葬二親。及蒸嘗拜獻,未曾不嬰慕卒事。每於時節,不受子孫慶賀。年餘耳順,而孝思彌篤,慨歲月推移,遷窆無冀,乃絕鹽粟,斷諸肴味,食麥而已。年八十卒。遷都洛陽,子應等乃還鄉葬焉。應
 弟煦,字賓育,好音律,以善歌聞於世,位秦州刺史。



 胡叟,字倫許,安定臨涇人也,世為西夏著姓。叟少聰慧,年十三,辯疑釋理,鮮有屈焉。學不師受,拔讀群籍,再閱於目,皆誦焉。好屬文,既善典雅之詞,又工鄙欲之句。



 以姚氏將衰,遂入長安觀風化。隱匿名行,懼人見知。時京兆韋祖思少閱典墳,多蔑時彥,待叟不足。叟拂衣而出,祖思固留之曰:「當與君論天人之際,何遽返乎?」叟曰:「論天人者其亡久矣,與君相知,何夸言若是。」遂歸主人,賦韋、杜二族,一宿而成。時年十八矣。其述前載,無違舊美;敘中世,有協時事;而末及鄙黷。人皆奇其才,畏其筆。



 叟
 孤飄坎壈,未有仕路,遂入漢中。宋梁、秦二州刺史馮翊吉翰頗相禮接。授叟末佐,不稱其懷。未幾,翰遷益州,叟隨入蜀。時蜀沙門法成率僧數千人鑄丈六金像,宋文帝惡其聚眾,將加大辟。叟聞之,即赴丹楊,啟申其美,遂免。復還蜀,法成遺其珍物,價直千餘匹,叟一無所受。



 後入沮渠牧犍,牧犍遇之不重,叟乃為詩,示所知廣平程伯達。其略曰:「群犬吠新客,佞暗排疏賓;直途既已塞,曲路非所遵。望衛惋祝鮀,眄楚悼靈均。何用宣憂懷,託翰寄輔仁。」伯達見詩,謂曰:「涼州雖地居戎域,然自張氏以來,號有華風。今則憲章無虧,何祝鮀之有?」叟曰:「貴主奉
 正朔而弗淳,慕仁義而未允。吾之擇木,夙在大魏,與子暫違,非久闊也。」歲餘,牧犍破降。



 叟既先歸魏,朝廷以其識機,賜爵始復男。家於密雲,蓬室草筵,唯以酒自適。



 謂友人金城宗舒曰:「我此生活,似勝焦先,志意所棲,謝其高矣。」文成時,召叟及舒,並使作檄,檄宋、蠕蠕。舒文劣於叟。尋歸家,不事產業,常苦飢貧,然不以為恥。養子字螟蛉,以自給養。每至貴勝門,恆乘一牸牛,弊韋褲褶而已。作布囊。容三四斛,飲啖醉飽,盛餘肉餅以付螟蛉。見車馬榮華者,視之蔑如也。尚書李敷嘗遺以財,都無所取。初,叟一見高允曰:「吳、鄭之交,以糸寧縞為美談;吾之於子,
 以弦韋為幽贄。以此言之,彼可無愧也。」於允館見中書侍郎趙郡李璨,被服華靡;叟貧老衣褐,璨頗忽之。叟謂曰:「李子,今若相脫體上褲褶衣帽,君欲作何許也?」譏其惟假盛服。璨惕然失色。叟少孤,每言及父母,則淚下若孺子號。春秋當祭之前,則先求旨酒美膳,將其所知廣寧常順陽、馮翊田文宗、上谷侯法俊,提壺執俎,至郭外空靜處,設坐奠拜,盡孝思之敬。時燉煌汜潛家善釀酒,每節送一壺與叟。著作佐郎博陵許赤武、河東裴定宗等謂潛曰:「再三之惠,以為過厚,子惠於叟,何其恆也?」潛曰:「我恆給祭者,以其恆於孝思也。」論者以潛為君子矣。
 順陽等數子,稟叟獎示,頗涉文流。



 高閭曾造其家,遇叟短褐曳柴,從田歸舍,為閭設濁酒蔬食,皆手自辦。然案其館宇卑陋,園疇褊局,而飯菜精潔,醢醬調美。見其二妾,並年衰跛眇,衣布穿弊。閭見其貧,以衣物直十餘匹贈之,亦無辭免。閭作《宣命賦》,叟為之序。密雲左右皆祗仰其德,歲時奉以布麻穀麥,叟隨分散之,家無餘財。卒,無子。無家人營主凶事,胡始昌迎殯之于家,葬於墓次。即令弟繼之,襲其爵始復男、武威將軍。叟與始昌雖宗室,性氣殊詭,不相附,其存,往來乃簡;及亡,而收恤至厚。



 議者以為非必敦哀疏宗,或緣求利品秩也。



 胡方回,安定臨涇人也。父義周,姚泓黃門侍郎。方回仕赫連屈丐為中書侍郎。



 涉獵史籍,辭彩可觀,為屈丐《統萬城銘》、《蛇祠碑》諸文,頗行於世。太武破赫連昌,方回入魏,未為時知。後為北鎮司馬,為鎮脩表,有所稱薦,帝覽之嗟美。



 閭知方回,召為中書博士,賜爵臨涇子。遷侍郎,與太子少傅游雅等改定律制。司徒崔浩及當時朝賢,並愛重之。清貧守道,以壽終。



 張湛,字子然,一字仲玄,燉煌深泉人也。魏執金吾恭九葉孫,為河西著姓。



 祖質,仕涼,位金城太守。父顯,有遠量,武昭王據有西夏,引為功曹,甚器異之。



 嘗稱曰:「吾之臧
 子原也。」位酒泉太守。



 湛弱冠知名涼土,好學能屬文,沖素有大志。仕沮渠蒙遜,位兵部尚書。涼州平,拜寧遠將軍,賜爵南蒲男。司徒崔浩識而禮之。浩注《易》,敘曰:「敦煌張湛、金城宗欽、武威段承根三人皆儒者,並有俊才,見稱西州。每與餘論《易》,餘以《左氏傳》卦解之,遂相勸為解注,故為之解。」其見稱如此。



 湛至京師,家貧不立,操尚無虧。浩常給其衣食,薦為中書侍郎;湛知浩必敗,固辭。每贈浩詩頌,多箴規之言。浩亦欽敬其志,每常報答,極推崇之美。浩誅,湛懼,悉燒之,閉門卻掃,慶弔皆絕,以壽終。



 兄銑,字懷義,閑粹有才幹,仕沮渠蒙遜,位建昌令。性至
 孝,母憂,哀毀過人,服制雖除,而蔬糲弗改。崔浩禮之與湛等。卒於征西參軍。



 懷義孫通,字彥綽,博通經史,沈冥不預時事。頓丘李彪欽其學行,與之遊款。



 及彪用事,言於中書令李沖,沖召見,甚器重之。太和中,徵中書博士、中書侍郎,永平中,又徵汾州刺史,皆不赴,終於家。



 通四子,徹、麟、儉、鳳,皆傳家業,知名於世。徹子方明,位侍中、衛尉卿,封西平縣公。子敢之襲,位太中大夫、樂陵郡守。麟字嘉應,位廣平太守。儉字元慎,位涼州刺史。鳳字孔鸞,位國子博士、散騎常侍。著《五經異同評》十卷,為儒者所稱。



 段承根,武威姑臧人,自云漢太尉熲九世孫也。父暉,字長祚,身八尺餘。師事歐陽湯,湯甚器愛之。有一童子與暉同志,後二年,童子辭歸,從暉請馬。暉戲作木馬與童子。甚悅,謝暉曰:「吾太山府君子,奉敕遊學,今將歸,損子厚贈,無以報德。子後至常伯封侯,非報也,且以為好。」言終,乘馬騰虛而去。暉乃自知必將貴。仕乞伏熾盤為輔國大將軍、涼州刺史、御史大夫、西海侯。熾盤子慕末襲位,政亂,暉父子奔吐谷渾。慕容璝內附,暉與承根歸魏。



 太武至長安,人告暉欲南奔,云置金於馬韉中。帝密遣視之,果如告者言,斬之於市,暴尸數日。時有儒生京兆
 林白奴,欽暉德音,夜竊其尸,置之枯井。女為燉煌張氏婦,聞之,乃向長安收葬。



 承根好學機辯,有文思,而性行疏薄,有始無終。司徒崔浩見而奇之,與同郡陰仲達俱被浩引,以為俱涼土文華,才堪注述。言之太武,並請為著作郎,引與同事。世咸重承根文而薄其行。甚為燉煌公李寶所敬待。浩誅,承根與宗欽等俱死。



 宗欽字景若,金城人。少好學,有儒者風。仕沮渠蒙遜為中書郎、世子洗馬。



 上《東宮侍臣箴》。太武平涼州,入魏,賜爵臥樹男,拜著作郎。與高允書,贈詩,允答書并詩,甚相褒美。在河西撰《蒙遜記》十卷,無足可稱。



 闞駰,字玄陰,燉煌人也。祖倞,父玖,並有名於西土,玖位會稽令。駰博通經傳,聰敏過人,三史群言,經目則誦,時人謂之宿讀。注王朗《易傳》,撰《十三州志》。沮渠蒙遜甚重之,常侍左右,訪以政事損益。拜秘書、考課郎中,給文吏三十人,典校經籍,刊定諸子三千餘卷。牧犍待之彌重,拜大行臺,遷尚書。及姑臧平,樂安王丕鎮涼州,引為從事中郎。王薨,遷京師。家甚貧,不免饑寒。性能多食,一飯至三升乃飽。卒,無後。



 劉延明,燉煌人也。父寶,字子玉,以儒學稱。延明年十四,就博士郭瑀。瑀弟子五百餘人,通經業者八十餘人。瑀
 有女始笄,妙選良偶,有心於延明。遂別設一席,謂弟子曰:「吾有一女,欲覓一快女婿,誰坐此席者,吾當婚焉。」延明遂奮衣坐,神志湛然曰:「延明其人也。」瑀遂以女妻之。延明後隱居酒泉,不應州郡命,弟子受業者五百餘人。



 涼武昭王徵為儒林祭酒、從事中郎。昭王好尚文典,書史穿落者,親自補葺。



 延明時侍側,請代其事。王曰:「躬自執者,欲人重此典籍。吾與卿相遇,何異孔明之會玄德。」遷撫夷護軍,雖有政務,手不釋卷。昭王曰:「卿注記篇籍,以燭繼晝,白日且然,夜可休息。」延明曰:「朝聞道,夕死可矣,不知老之將至,孔聖稱焉。延明何人斯,敢不如此。」延
 明以三史文繁,著《略記》百三十篇、八十四卷,《燉煌實錄》二十卷,《方言》三卷,《靖恭堂銘》一卷,注《周易》、《韓子》、《人物志》、《黃石公三略》行於世。



 蒙遜平酒泉,拜秘書郎,專管注記。築陸沈觀於西苑,躬往禮焉,號玄處先生。



 學徒數百,月致羊酒。牧犍尊為國師,親自致拜;命官屬以下,皆北面受業。時同郡索敞、陰興為助教,並以文學見稱,每巾衣而入。太武平涼州,士庶東遷,夙聞其名,拜樂平王從事中郎。太武詔諸年七十已上,聽留本鄉,一子扶養。延明時老矣,在姑臧歲餘,思鄉而返,至涼州西四百里韮谷窟,疾卒。



 太和十四年,尚書李沖奏:延明河右碩儒,今子
 孫沈屈,未有祿潤;賢者子孫,宜蒙顯異。於是除其一子為郢州雲陽令。正光三年,太保崔光奏曰:「故樂平王從事中郎燉煌劉延明,著業涼城,遺文在茲。如或愆釁,當蒙數世之宥。況乃維祖逮孫,相去未遠,而令久淪皁隸,不獲收異,儒學之士,所為竊歎。乞敕尚書,推檢所屬,甄免碎役,敦化厲俗,於是乎在。」詔曰:「太保啟陳,深合勸善,其孫等三家,特可聽免。」河西人以為榮。



 趙柔,字元順,金城人也,少以德行才學,知名河右。沮渠牧犍時,為金部郎。



 太武平涼州,內徙京師。歷著作郎、河內太守,甚著信惠。柔嘗在路,得人所遺金珠一貫,價直
 數百縑,柔呼主還之。後有人遺柔鏵數百枚者,柔與子善明鬻之市。



 有人從柔買,柔索絹二十疋。有商人知其賤,與柔三十匹。善明欲取之,柔曰:「與人交易,一言便定,豈可以利動心?」遂與之。晉紳之流,聞而敬服。隴西王源賀采佛經幽旨作《祗洹精舍圖偈》六卷,柔為之注解,為當時俊僧所欽味。又憑立銘贊,頗行於世。子默,字沖明,武威太守。



 索敞,字巨振,燉煌人也。為劉延明助教,專心經籍,盡能傳延明業。涼州平,入魏,以儒學為中書博士。京師貴遊之子,皆敬憚威嚴,多所成益,前後顯達位至尚書、牧、守
 者數十人,皆受業於敞。敞以喪服散在眾篇,遂撰比為《喪服要記》。



 出補扶風太守,在位清貧,卒官。時舊同學生等為請謚,詔贈涼州刺史,謚曰獻。



 初,敞之在涼州,與鄉人陰世隆,文才相友。世隆至京師,被罪,徙和龍,屈上谷,困不前達,土人徐能抑掠為奴。敞因行至上谷,遇見世隆,對泣而別。敞為訴理,得免。世隆子孟貴,性至孝。每向田芸耨,早朝拜父,來亦如之,鄉人欽焉。



 宋繇,字體業,燉煌人也,世仕張氏。父僚,張玄靚武興太守。繇生而僚為張邕所誅。五歲喪母,事伯母張氏以孝聞。八歲而張氏卒,居喪過禮。喟然謂妹夫張彥曰:「門戶
 傾覆,歲荷在繇,不銜膽自厲,何以繼承先業。」遂隨彥至酒泉,追師就學,閉室讀書,晝夜不倦,博通經史。呂光時,舉秀才,除郎中。後奔段業,為中散騎常侍。以業無遠略,西奔涼武昭王。歷位通顯,家無餘財;雖兵革間,講誦不廢。每聞儒士在門,常倒屣出迎,引談經籍。尤明斷決,時事亦無滯也。沮渠蒙遜平酒泉,於繇室得書數千卷、鹽米數十斛而已。蒙遜歎曰:「孤不喜克李氏,欣得宋繇耳。」拜尚書吏部郎中,委以銓衡。蒙遜將死,以子牧犍託之。牧犍以為左丞,送其妹興平公主於京師。太武拜繇河西王右丞相,錫爵清水公。及平涼州,從牧犍至京師,卒,
 謚恭公。



 長子巖襲爵,改為西平侯。巖子蔭,中書議郎、樂安王範從事中郎,卒,贈咸陽太守。



 蔭子季預,性清嚴,居家如官,位勃海太守。子遊道。



 遊道弱冠隨父在郡,父亡,吏人贈遺無所受,事母以孝聞。與叔父別居。叔父為奴誣以構逆,遊道誘令返,雪而殺之。魏廣陽王深北伐,請為鎧曹,及為定州刺史,又以為府佐。廣陽為葛榮所殺,元徽誣其降賊,收錄妻子,遊道為訴得釋,與廣陽子迎喪返葬。中尉酈善長嘉其氣節,引為殿中侍御史。臺中語曰:「見惡能討,宋遊道。」



 孝莊即位,除左兵中軍。為尚書令臨淮王彧譴責,遊道乃執版長揖曰:「下官謝王瞋,
 不謝王理。」即日詣闕上書曰:「徐州刺史元孚頻有表,云偽梁廣發士卒,圖彭城,乞增羽林二千。以孚宗室重臣,告請應實,所以量奏給武官千人。孚今代下,以路阻自防,遂納在防羽林八百人;辭云疆境無事,乞將還家。臣忝局司,深知不可。尚書令臨淮王彧,即孚之兄子,遣省事謝遠,三日之中,八度逼迫,云宜依判許。臣不敢附下罔上,孤負聖明。但孚身在任,乞師相繼;及其代下,便請放還。進退為身,無憂國之意。所請不合,其罪下科。彧乃召臣於尚書都堂云:『卿一小郎,憂國之心,豈厚於我?』醜罵溢口,不顧朝章。右僕射臣世隆、吏部郎中臣薛琡已
 下百餘人,並皆聞見。臣實獻直言云:『忠臣奉國,事在其心,亦復何簡貴賤?比自北海入洛,王不能致身死難,方清宮以迎篡賊;鄭先護立義廣州,王復建旗往討。趣惡如流,伐善何速?」今得冠冕百寮,乃欲為私害政!』為臣此言,彧賜怒更甚。臣既不佞,干犯貴臣,乞解郎中。」帝召見遊道,嘉勞之。彧亦奏言:「臣忝冠百寮,遂使一郎攘袂高聲,肆言頓挫,乞解尚書令。」帝乃下敕,聽解臺郎。後除司州中從事。



 時將還鄴,會霖雨,行旅擁於河橋。遊道於幕下朝夕宴歌。行者曰:「何時節作此聲也?固大癡!」遊道應曰:「何時節而不作此聲也?亦大癡!」後齊神武自太原來
 朝,見之曰:「此人是遊道邪?常聞其名,今日始識其面。」遷遊道別駕,後日,神武之司州,饗朝士,舉觴屬遊道曰:「飲高歡手中酒者大丈夫,卿之為人,合飲此酒。」及還晉陽,百官辭於紫陌,神武執遊道手曰:「甚知朝貴中有憎忌卿者。但用心,莫懷畏慮,當使卿位與之相似。」於是啟以遊道為中尉。文襄執請,乃以吏部郎中崔暹為御史中尉,以游道為尚書左丞。文襄謂暹、遊道曰:「卿一人處南臺,一人處北省,當使天下肅然。」



 遊道入省,劾太師咸陽王但、太保孫騰、司徒高隆之、司空侯景、錄尚書元弼、尚書令司馬子如官貸金銀,催征酬價。雖非指事贓賄,終
 是不避權豪。又奏駮尚書違失數百條;省中豪吏王儒之徒,並鞭斥之;始依故事於尚書省立門名,以記出入早晚。令僕已下皆側目。



 魏安平王坐事亡,章武二王及諸王妃、太妃是其近親者,皆被徵責。都官郎中畢義云主其事,有奏而禁,有不奏輒禁者。遊道判下廷尉科罪。高隆之不同,於是反誣遊道厲色挫辱己,遂枉拷群令史證成之。與左僕射襄城王旭、尚書鄭述祖等上言曰:飾偽亂真,國法所必去;附下罔上,王政所不容。謹案:尚書左丞宋遊道,名望本闕,功績何紀?屬永安之始,朝士亡散,乏人之際,叨竊臺郎。躁行謟言,肆其奸詐,空識名
 義,不顧典文。人鄙其心,眾畏其口。出州入省,歷忝清資,而長惡不悛,曾無忌諱,毀譽由己,憎惡任情。比因安平王事,遂肆其褊心,因公報隙,與郎中畢義雲遞相糾舉。



 又在外兵郎中魏叔道牒云:「局內降人左澤等為京畿送省,令取保放出。」大將軍在省日,判聽。遊道發怒曰:「往日官府成何物官府?將此為例!」又云:「乘前旨格,成何物旨格?」依事請問,遊道並皆承引。案律:「對捍詔使,無人臣之禮大不敬者,死。」對捍使者尚得死坐,況遊道吐不臣之言,犯慢上之罪?口稱夷、齊,心懷盜跖,欺公賣法,受納苞苴,產隨官厚,財與位積。雖贓汙未露,而姦許如是,舉
 此一隅,餘詐可驗。今依禮據律,處遊道死罪。



 是時朝士皆忿為遊道不濟。而文襄聞其與隆之相抗之言,謂楊遵彥曰:「此真是鯁直大剛惡人。」遵彥曰:「譬之畜狗,本取其吠,今以數吠殺之,恐將來無復吠狗。」詔付廷尉,遊道坐除名。文襄使元景康謂曰:「卿早逐我向并州他經略,不忍殺卿。」遊道從至晉陽,以為大行臺吏部,又以為太原公開府諮議。及平陽公為中尉,遊道以諮議領書侍御史。尋以本官兼司徒左長史。



 及文襄疑黃門郎溫子昇知元瑾之謀,繫諸獄而餓之,食弊襦而死,棄屍路隅,遊道收而葬之。文襄謂曰:「吾近書與京師諸貴,論及朝
 士,云卿僻於朋黨,將為一病。今卿真是重舊節義人,此情不可奪。子昇吾本不殺之,卿葬之何所憚?天下人代卿怖者,是不知吾心也。」尋除御史中尉。東萊王道習參御史選,限外投狀,道習與遊道有舊,使令史受之。文襄怒,收遊道,辯而判之曰:「游道稟性獷悍,是非肆己,吹毛洗垢,創疵人物。往與郎中蘭景雲忿競,列事十條,及加推窮,便是虛妄。方共道習,陵侮朝典。法官而犯,特是難原,宜付省科。」游道被禁,獄吏欲為脫枷,游道不肯曰:「此令公命所著,不可輙脫。」文襄聞而免之。游道抗志不改。



 天保元年,以游道兼太府卿,乃於少府覆檢主司盜截,
 得鉅萬計。姦吏反誣奏之,下獄。尋得出,不歸家,徑之府理事。卒,遺令薄葬,不立碑表,不求贈謚。



 贈瓜州刺史。武平中,以子士素久典機密,重贈儀同三司,謚曰貞惠。



 游道剛直,疾惡如仇,見人犯罪,皆欲致之極法。彈糾見事,又好察陰私,問獄察情,捶撻嚴酷。兗州刺史李子貞在州貪暴,游道案之。文襄以子貞預建義勳,意將含忍。游道疑陳元康為其內助,密啟云:「子貞、元康交游,恐其別有請屬。」



 文襄怒,於尚書都堂集百寮,撲殺子貞。又兗州人為游道生立祠堂,像題曰「忠清君」。游道別劾吉寧等五人同死,有欣悅色。朝士甚鄙之。然重交游,存然諾之
 分。



 歷官嚴整,而時大納賄,分及親故之艱匱者,其男女孤弱,為嫁娶之,臨喪必哀,躬親營視。為司州綱紀,與牧昌樂、西河二王乖忤,及二王薨,每事經恤之。與頓丘李獎,一面便定死交。獎曰:「我年位已高,會用弟為佐史,令弟北面於我足矣。」



 游道曰:「不能。」既而獎為河南尹,辟游道為中正,便者相屬,以衣帢待之,握手歡謔。元顥入洛,獎受其命。出使徐州,都督元孚與城人趙紹兵殺之。游道為獎訟冤,得雪。又表為請贈,回己考一泛階以益之。又與劉廞結交,託廞弟粹於徐州殺趙紹。後劉廞伏法於洛陽,粹以徐州叛,官軍討平之,梟粹首於鄴市。孫騰
 使客告市司,得五百匹後,聽收。游道時為司州中從事,令家人作劉粹所視,於州陳訴,依律判許,而奏之。敕至,市司猶不許,游道杖市司,勒使速付。騰聞大怒,游道立理以抗之。既收粹尸,厚加贈遺。李獎二子構、訓居貧,游道後令其求三富人死事判免之,凡得錢百五十萬,盡以入構、訓。其使氣黨俠如此。時人語曰:「游道獼猴面,陸操科斗形,意識不關見,何謂醜者必無情。」



 構嘗因游道會客,因戲之曰:「賢從在門外,大好人,宜自迎接。」為通名,稱族弟游山。游道出見之,乃獼猴而衣帽也。將與構絕,構謝之,豁然如舊。游道死後,構為定州長史,游道第三
 子士遜為墨曹、博陵王管記,與典簽共誣奏構。構於禁所祭游道而訴焉。士遜晝臥如夢者,見游道怒己曰:「我與構恩義,汝豈不知?



 何共小人謀陷清直之士!」士遜驚跪曰:「不敢!不敢!」旬日而卒。



 游道每戒其子士素、士約、士慎等曰:「吾執法太剛,數遭屯蹇,性自如此,子孫不足以師之。諸子奉父言,柔和廉遜。



 士素沉密少言,有才識,稍遷中書舍人。趙彥深引入內省,參典機密。歷中書、黃門侍郎,遷儀同三司、散騎常侍,恆領黃門侍郎。自處機要,近二十年,周慎溫恭,甚為彥深所重。初,祖珽知朝政,出彥深為刺史。珽奏以士素為東郡守,中書侍郎李德林
 白珽留之,由是還除黃門侍郎,共典機密。士約亦重善士,官尚書左丞。



 江式,字法字,陳留濟陽人也。六世祖瓊,字孟琚,晉馮翊太守,善蟲篆詁訓。



 永嘉大亂,瓊棄官投張軌,子孫因居涼士,世傳家業。祖強,字文威,涼州平,內徙代京。上書三十餘法,各有體例,又獻經史諸子千餘卷,由是拜中書博士。卒,贈敦煌太守。父紹興,高允奏為秘書郎,掌國史二十餘年,以謹厚稱。卒於趙郡太守。式少專家學,數年中,常夢兩人時相教授;及寤,每有記識。初拜司徒長史兼行參軍,檢校御史,尋除符節令。以書文昭太后尊號
 謚冊,除奉朝請,仍符節令。



 篆體尤工,洛京宮殿諸門板題,皆式書也。延昌三年三月,式表曰:臣聞伏羲氏作而八卦形其畫,軒轅氏興而靈龜彰其彩。古史倉頡覽二象之爻,觀鳥獸之迹,別創文字,以代結繩,用書契以維事。宣之王迹,則百工以敘;載之方冊,則萬品以明。迄於三代,厥體頗異,雖依類取制,未能殊倉氏矣。故《周禮》:八歲入小學,保氏教國子以六書:一曰指事,二曰象形,三曰形聲,四曰會意,五曰轉注,六曰假借。蓋是史頡之遺法。及宣王太史史籀著《大篆》十五篇,與古文或同或異,時人即謂之籀書。孔子脩《六經》,左丘明述《春秋》,皆以古
 文,厥意可得而言。其後七國殊軌,文字乖別。暨秦兼天下,丞相李斯乃奏蠲罷不合秦文者。斯作《倉頡篇》,車府令趙高作《爰歷篇》,太史令胡母敬作《博學篇》,皆取史籀式,頗有省改,所謂小篆者也。於是秦燒經書,滌除舊典,官獄繁多,以趣約易,始用隸書,古文由此息矣。隸書者,始皇使下杜人程邈附於小篆所作也。世人以邈徒隸,即謂之隸書。故秦有八體:一曰大篆,二曰小篆,三曰符書,四曰蟲書,五曰摹印,六曰署書,七曰殳書,八曰隸書。



 漢興,有尉律學,復教以籀書,又習八體,試之課最,以為尚書史。書省字不正,輒舉劾焉。又有草書,莫知誰始,其
 形書雖無厥誼,亦是一時之變通也。孝宣時,召通《倉頡》讀者,獨張敞從受之。涼州刺史杜業、沛人爰禮講學,大夫秦近亦能言之。孝平時,徵禮等百餘人說文字於未央宮中,以禮為小學元士。黃門侍郎揚雄採以作《訓纂篇》。及亡新居攝,自以運應制作,大司馬甄豐校文字之部,頗改定古文。時有六書:一曰古文,孔子壁中書也;二曰奇字,即古文而異者;三曰篆書,云小篆也;四曰佐書,秦隸書也;五曰繆篆,所以摹印也;六曰鳥蟲,所以幡信也。壁中書者,魯恭王壞孔子宅而得《尚書》、《春秋》、《論語》、《孝經》也。又北平侯張倉獻《春秋左氏傳》,書體與孔氏相類,
 即前代之古文矣。後漢郎中扶風曹喜號曰工篆,小異斯法,而甚精巧,自是後學,皆其法也。又詔侍中賈逵脩理舊文,殊藝異術,王教一端,茍有可以加於國者,靡不悉集。逵即汝南許慎古學之師也。後慎嗟時人之好奇,歎俗儒之穿鑿,故撰《說文解字》十五篇,首一終亥,各有部屬,可謂類聚群分,雜而不越,文質彬彬,最可得而論也。左中郎將陳留蔡邕採李斯、曹喜之法,為古今雜形,詔於太學立石碑,刊載《五經》,題書楷法,多是邕書也。後開鴻都,書畫奇能,莫不雲集。時諸方獻篆,無出邕者。



 魏初,博士清河張揖著《埤倉》、《廣雅》、《古今字詁》。究諸《埤》、《廣》,綴
 拾遺漏,增長事類,抑亦於文為益者。然其《字詁》,方之許篇,古今體用,或得或失。陳留邯鄲淳亦與揖同,博開古藝,特善《倉》、《雅》。許氏字指、八體、六書,精究閑理,有名於揖。以書教諸皇子。又建《三字石經》於漢碑西,其文蔚煥,三體復宣。校之《說文》,篆、隸大同,而古字少異。又有京兆韋誕、河東衛覬二家,並號能篆。當時臺觀榜題,寶器之銘,悉是誕書。咸傳之子孫,世稱其妙。晉世義陽王典祠令任城呂忱表上《字林》六卷,尋其況趣,附託許慎《說文》,而按偶章句,隱別古籀奇惑之字,文得正隸,不差篆意也。忱弟靜別放故左校令李登《聲類》之法,作《韻集》五卷,使
 宮、商、角、徵、羽各為一篇,而文字與兄便是魯、衛,音讀楚、夏,時有不同。皇魏承百王之季,紹五運之緒。



 世易風移,文字改變,篆形謬錯,隸體失真。俗學鄙習,復加虛造。巧談辯士,以意為疑,炫惑於時,難以釐改。乃曰:追來為歸,巧言為辯,小免為,神嵒為蠶。



 如斯甚眾,皆不合孔氏古書、史籀《大篆》、許氏《說文》、《石經》三字也。凡所關古,莫不惆悵焉。嗟夫!文字者六籍之宗,王教之始,前人所以垂今,今人所以識古。



 臣六世祖瓊,家世陳留,往晉之初,與從父兄俱受學於衛覬,古篆之法,《倉》、《雅》、《方言》、《說文》之誼,當時並收善譽。而祖遇洛陽之亂,避地河西,數世傳習,
 斯業所以不墜也。世祖太延中,牧犍內附,臣亡祖文威杖策歸國,奉獻五世傳掌之書,古篆八體之法。時蒙褒錄,敘列於儒林,官班文省,家號世業。



 暨臣闇短,識學庸薄,漸漬家風,有忝無顯。是藉六世之資,奉遵祖考之訓,竊慕古人之軌,企踐儒門之轍。求撰集古來文字,以許慎《說文》為主,及孔氏《尚書》、《五經音注》、《籀篇》、《爾雅》、《三倉》、《凡將》、《方言》、《通俗文》、祖文宗《埤倉》、《廣雅》、《古今字詁》、《三字石經》、《字林》、《韻集》、諸賦文字有六書之誼者,以類編聯,文無復重,統為一部。其古籀、奇惑、俗隸諸體,咸使班於篆下,各有區別。詁訓假借之誼,僉隨文而解;音讀楚、夏之聲,並
 逐字而注。其所不知者,則闕如也。脫蒙遂許,冀省百氏之觀,而同文字之域。典書秘書所須之書,乞垂敕給;并學士五人嘗習文字者,助臣披覽;書生各五人,專令抄寫。侍中、黃門、國子祭酒一月一監,誣議疑隱,庶無紕繆。所撰名目,伏聽明旨。



 詔曰:「可如所請,併就太常,冀兼教八書史也。其有所須,依請給之。名目待書成重聞。」式於是撰集字書,號曰《古今文字》,凡四十卷,大體依許氏《說文》為本,上篆下隸。正光中,兼著作郎。卒官,贈巴州刺史。其書竟未能成。式兄子征虜將軍順和,亦工篆書。



 先是,太和中,兗州人沈法會能隸書。宣武之在東宮,敕法會
 侍書。後以隸迹見知於閭里者甚眾,未有如崔浩之妙。



 論曰:游雅才業,亦高允之亞,至於陷族陳奇,斯所以絕世而莫祀。明根雅道儒風,終受非常之遇。以太和之盛,有乞言之重,抑乃曠世一時。肇既聿脩,克隆堂構,正清梗概,顛沛不渝;辭爵主幼之年,抗節臣權之日,顧視群公,其風固已遠矣。高閭發言有章句,下筆富文詞,故能受遇累朝,見重明主,掛冠謝事,禮備懸輿。美矣!趙逸文雅自業,琰加之孝義,可謂世有人焉。胡叟顯晦之間,優遊無悶,亦一代之異人歟!胡方回、張湛、段承根、闞駰、劉延明、趙柔、索敞皆通涉經史,才志不群,價重西州,有聞
 東國,故流播之中,自拔泥滓。人之不可以無能,信也。宋繇處屈能申,終致顯達。游道剛直自立,任使為累。江式能世其業,亦足稱云。



\end{pinyinscope}