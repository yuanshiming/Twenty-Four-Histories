\article{卷十九列傳第七 文成五王 獻文六王 孝文六王}

\begin{pinyinscope}

 文成皇帝七男:孝元皇后生獻文皇帝;李夫人生安樂厲王長樂;曹夫人生廣川莊王略;沮渠夫人生齊郡順王簡;乙夫人生河間孝王若;悅夫人生安豐匡王猛;玄夫人生韓哀王安平,早薨,無傳。



 安樂王長樂,皇興四年,封建昌王,後改封安樂王。長樂性凝重,獻文器愛之。



 承明元年,拜太尉,出為定州刺史。
 頓辱衣冠,多不奉法,百姓詣闕訟之,孝文罰杖三十。貪暴彌甚,以罪徵詣京師。後謀不軌,事發,賜死於家,葬以王禮,謚曰厲。



 子詮,字搜賢,襲。宣武初,為涼州刺史。在州貪穢,政以賄成。後除定州刺史。及京兆王愉之反,詐言國變,在北州鎮咸疑朝廷有釁,遣使觀詮動靜。詮具以狀告,州鎮帖然。愉奔信都,詮以李平、高殖等四面攻燒,愉突門而出。尋除侍中,兼以首告之功,除尚書左僕射。薨,謚曰武康。



 子鑒,字長文,襲。後除相州刺史、北討大都督,討葛榮。仍兼尚書左僕射、北道行臺尚書令,與都督裴衍共攻信都。鑒既庸才,見天下多事,遂謀反,降附葛
 榮。都督源子邕與裴衍合圍鑒,斬首傳洛,詔改姓元氏。莊帝初,許復本族,又特復鑒王爵,贈司空。



 鑒弟斌之,字子爽,性險無行。及與鑒反,敗,遂奔葛榮。榮滅,得還。孝武帝時,封潁川郡王,委以腹心之任。帝入關,斌之奔梁。大統二年,還長安,位尚書令。薨,贈太尉,謚武襄。



 廣川王略,延興二年封,位中都大官。性明敏,鞫獄稱平。太和四年,薨。謚曰莊。



 子諧,字仲和,襲。十九年,薨。詔曰:「古者大臣之喪,有三臨之禮,此蓋三公已上。自漢已降,多無此禮。庶仰遵古典,哀感從情。雖以尊降伏,私痛寧爽。



 欲令親王有期親者為之三臨,大功親者為之再臨,小
 功緦麻為之一臨。廣川王於朕大功,必欲再臨者,欲於大斂日親臨盡哀,成服之後,緦衰而弔。既殯之緦麻,理在無疑。大斂之臨,當否如何?為須撫柩於始喪?為應盡哀於闔柩?」黃門侍郎崔光、宋弁、通直常侍劉芳、典命下大夫李元凱、中書侍郎高聰等議曰:「三臨之事,乃自古禮。爰及漢、魏,行之者稀;陛下方遵前軌。臣等以為若期親三臨,大功宜再。始喪之初,哀之至極,既以情降,宜從始喪。大斂之臨,伏如聖旨。」詔曰:「魏、晉已來,親臨多闕,至於戚臣,必於東堂哭之。頃大司馬安定王薨,朕既臨之後,受慰東堂。今日之事,應更哭不?」光等議曰:「東堂之哭,
 蓋以不臨之故。



 今陛下躬親撫視,群臣從駕,臣等議,以為不宜復哭。」詔曰:「若大司馬戚尊位重,必哭於東堂。而廣川既是諸王之子,又年位尚幼,卿等議之,朕無異焉。」諧將大斂,帝素委貌深衣哭之,入室哀慟,撫尸而出。



 有司奏:「廣川王妃薨於代京,未審以新尊從於卑舊,為宜卑舊來就新尊?」



 詔曰:「遷洛之人,自茲厥後,悉可歸骸芒嶺,皆不得就塋恒、代。其有夫先葬北,婦今喪在南,婦人從夫,宜還代葬。若欲移父就母,亦得任之。其有妻墳於恒、代,夫死於洛,不得以尊就卑。欲移母就父,宜亦從之。若異葬,亦從之。若不在葬限,身在代喪,葬之彼此,皆得
 任之。其戶屬恒、燕,身官京洛,去留之宜,亦從所擇。



 其屬諸州諸,各得任意。」詔贈諧武衛將軍,謚曰剛。及葬,帝親臨送之。子靈道襲。卒,謚悼王。



 齊郡王簡字叔亮,太和五年封,位中都大官。簡母,沮渠牧犍女也。簡性貌特類外祖。後為內都大官。孝文嘗與簡俱朝文明太后皇信堂,簡居帝之右,行家人禮。



 遷太保。孝文仁孝,以諸父零落,存者唯簡,每見,立以待之;俟坐,致敬問起居,停簡拜伏。簡性好酒,不能理公私之事。妻常氏,燕郡公喜女也,文明太后以賜簡。



 乾綜家事,頗節簡酒。乃至盜竊,求乞婢侍,卒不能禁。薨時,孝文不豫,
 詔曰:「叔父薨背,痛慕摧絕,不自勝任。但虛頓床枕,未堪奉赴,當力疾發哀。」謚曰靈王。宣武時,改謚曰順。



 子祐,字伯授。母常氏,孝文以納不以禮,不許其為妃。宣武以母從子貴,詔特拜為齊國太妃。祐位涇州刺史。薨,謚曰敬。



 河間王若字叔儒,未封而薨。追封河間,謚曰孝。詔京兆康王子太安為後。太安於若為從弟,非相後之義,廢之。以齊郡王子琛繼。



 琛字曇寶,幼敏慧,孝文愛之。宣武時,拜定州刺史。琛妃,宣武舅女,高皇后妹。琛憑恃內外,在州貪婪。及還朝,靈太后詔曰:「琛在定州,唯不將中山宮來,自餘無所不致,何可敘用!」由是廢于家。琛以明帝始
 學,獻金字《孝經》。



 又無方自達,乃與劉騰為養息,賂騰金寶巨萬計。騰為言,乃得兼都官尚書。出為秦州刺史,在州聚斂,百姓吁嗟。東益、南秦二州氐反,詔琛為行臺,仍充都督,還攝州事。既總軍省,求慾無厭。進討氐、羌,大被摧破。內恃劉騰,無所畏憚。



 為中尉彈糾,會赦,除名。尋復王爵。後討鮮于脩禮,敗,免官爵。後討汾晉胡、蜀,卒於軍,追復王爵。



 安豐王猛字季烈,太和五年封,加侍中。出為鎮都大將、營州刺史。猛寬仁雄毅,甚有威略,戎夷畏愛之。薨于州,贈太尉,謚曰匡。



 子延明襲。宣武時,授太中大夫。延昌初,
 歲大饑,延明乃減家財以拯賓客數十人,并贍其家。至明帝初,為豫州刺史,甚有政績。累遷給事黃門侍郎。延明既博極群書,兼有文藻,鳩集圖籍萬有餘卷。性清儉,不營產業。與中山王熙及弟臨淮王彧等並以文學令望,有名於世。雖風流造次不及熙、彧,而稽古淳篤過之。遷侍中,詔與侍中崔光撰定服制。後兼尚書右僕射。以延明博識多聞,敕監金石事。



 及元法僧反,詔為東道行臺、徐州大都督,節度諸軍事。與都督臨淮王彧、尚書李憲等討法僧。梁遣其豫章王綜鎮徐州。延明先牧徐方,甚得人譽;招懷舊土,遠近歸之。綜既降,延明因以軍乘
 之。復東南之境,至宿、豫而還。遷都督,徐州刺史。頻經師旅,人物彫弊。延明招攜新故,人悉安業,百姓咸附。



 莊帝時,兼大司馬。元顥入洛,延明受顥委寄。顥敗,奔梁,死於江南。莊帝末,喪還。孝武初,贈太保,王如故,謚曰文宣。



 所著詩賦讚頌銘誄三百餘篇。又撰《五經宗略》、《詩禮別義》;注《帝王世紀》及《列仙傳》。又以河間人信都芳工算圖。又集《器準》九篇,芳別為之注,皆行於世矣。



 孫長儒,孝靜時襲祖爵。



 獻文皇帝七男:思皇后生孝文皇帝;封昭儀生咸陽王禧;韓貴人生趙郡靈王乾、高陽文穆王雍;孟椒房生廣
 陵慧王羽;潘貴人生彭城武宣王勰;高椒房生北海王詳。



 咸陽王禧字思永,太和九年封,加侍中、驃騎大將軍、中都大官。文明太后令皇子皇孫於靜所別置學,選忠信博聞之士為之師傅,以匠成之。孝文以諸弟典三都職,謂禧曰:「弟等皆幼年任重,三都折獄,特宜用心。夫未能操刀而使割錦,非傷錦之尤,實授刀之責。」文明太后亦致誡勖。出為使持節、開府、冀州刺史,孝文餞於南郊。又以濟陽王鬱枉法賜死之事遣告禧,因以誡之。後禧朝京師,詔以廷尉卿李沖為禧師。



 時王國舍人應取八族
 及清脩之門,禧取任城王隸戶為之,深為帝責。帝以諸王婚多猥濫,於是為禧娉故潁川太守隴西李輔女;河南王乾娉故中散代郡穆明樂女;廣陵王羽娉驃騎諮議參軍榮陽鄭平城女;潁川王雍娉故中書博士范陽盧神寶女;始平王勰娉廷尉卿隴西李沖女;北海王詳娉吏部郎中榮陽鄭懿女。



 有司奏:「冀州人蘇僧瓘等三千人稱禧清明,有惠政,請世胙冀州。」詔曰:「畫野由君,理非下請。」入除司州牧。詔以禧元弟之重,食邑三千戶,自餘五王皆食邑二千。



 孝文引見朝臣,詔斷北語,一從正音,禧贊成其事。於是詔:「年三十已上,習性已久,容或不
 可卒革。三十已下,見在朝廷之人,語音不聽仍舊。若有故為,當降爵黜官。若仍舊俗,恐數世之後,伊洛之下,復成被髮之人。朕嘗與李沖論此,沖言:『四方之語,竟知誰是;帝者言之,即為正矣,何必改舊從新。』沖之此言,應合死罪。」乃謂沖曰:「卿實負社稷。」沖免冠陳謝。又責留京之官曰:「昨望見婦女之服,仍為夾領小袖,何為而違前詔?」禧對曰:「陛下聖過堯、舜,光化中原。舛違之罪,實合處刑。」孝文曰:「若朕言非,卿等當奮臂廷論,如何入則順旨,退有不從?昔舜語禹:『汝無面從,退有後言。』卿等之謂乎!」



 尋以禧長兼太尉公。後帝幸禧第,謂司空穆亮、僕射李沖
 曰:「元弟禧戚連皇極,且長兼太尉,以和飪鼎,朕恒恐君有空授之名,臣貽彼己之刺。今幸其宅,徒屈二賓,良以為愧。」帝篤於兄弟,以禧次長,禮遇優隆。然亦知其性貪,每加切誡,而終不改操。後加侍中,正太尉。



 及帝崩,禧受遺輔政。雖為宰輔之首,而潛受賄賂。姬妾數十,意尚未已,猶欲遠有簡娉,以恣其情。宣武頗惡之。景明二年春,召禧等入光極殿,詔曰:「恪比纏尪疾,實憑諸父。今便親攝百揆。且還府司,當別處分。」尋詔進位太保,領太尉。



 帝既覽政,禧意不安,遂與其妃兄兼給事黃門侍郎李伯尚謀反。帝時幸小平,禧在城西小宅。初欲勒兵直入
 金墉,眾懷沮異。禧心因緩,自旦達晡,計不能決。



 遂約不洩而散。直寢符承祖、薛魏孫與禧將害帝。是日,帝息於芒山,止浮圖陰下,少時睡臥,魏孫便欲赴廷。承祖私言於魏孫曰:「吾聞殺天子者身當癩。」魏孫且止。帝尋覺悟。俄有武興王楊集始出,便馳告。而禧意不疑,乃與臣妾向洪池別墅,遣其齋帥劉小茍奉啟,云檢行田牧。小茍至芒嶺,已逢軍人,怪小茍赤衣,將欲殺害。小茍言欲告反,乃緩之。



 禧是夜宿於洪池,不知事露。其夜,將士所在追禧,禧自洪池東南走,左右從禧者唯兼防閣尹龍武。禧憂迫,謂曰:「試作一謎,當思解之,以釋毒悶。」龍武欻憶
 舊謎云:「眠則同眠,起則同起,貪如犲狼,贓不入己。」都不有心於規刺也。



 禧亦不以為諷己,因解之曰:「此是眼也。」而龍武謂之是箸。渡洛水,至柏塢,顧謂龍武曰:「汝可勉心作與太尉公同死計。」龍武曰:「若與殿下同命,雖死猶生。」俄而禧被禽,送華林都亭,著千斤鎖格龍武,羽林掌衛之。時熱甚,禧渴悶垂死,敕斷水漿。侍中崔光令左右送酪漿升餘,禧一飲而盡。初,孝文觀台宿有逆謀氣,言於禧曰:「玄象變,汝終為逆謀,會無所成,但受惡而已。」至此,果如言。



 禧臨盡,畏迫喪志,乃與諸妹公主等訣,言及一二愛妾。公主哭且罵之,言:「坐多取此婢輩,貪逐財物,
 致今日之事,何復囑問此等!」禧愧而無言。遂賜死私第,絕其諸子屬籍。禧之諸女,微給資產、奴婢。自餘家財悉以賚高肇、趙脩二家,其餘賜內外百官,逮于流外,多百匹,下至十匹,其積聚若此。其宮人為之歌曰:「可憐咸陽王,奈何作事誤?金床玉几不能眠,夜蹋霜與露。洛水湛湛彌岸長,行人那得度!」其歌遂流至江表。北人之在南者,雖富貴,聞弦管奏之,莫不灑泣。



 禧八子。長子通,字曇和,竊入河內太守陸琇家。初與通情,既聞禧敗,乃殺之。



 通弟翼,字仲和,後會赦,詣闕上書,求葬父。不許,乃與二弟昌、曄奔梁。



 正光中,詔咸陽、京兆二王諸子並聽附屬
 籍。後復禧王爵,葬以王禮,詔曄弟坦襲。



 翼與昌,申屠氏出;曄,李妃所出也。翼容貌魁壯,風制可觀,梁武甚重之,封為咸陽王。翼讓其嫡弟曄,梁武不許。後為青、冀二州刺史,鎮郁州。翼謀舉州入國,為梁武所殺。



 翼弟樹,字秀和,一家獨立。美姿貌,善吐納,兼有將略。位宗正卿。後亦奔梁。梁武尤器之,封為魏郡王,後改封鄴王。數為將領,窺覦邊服。爾朱榮之害百官也,樹時為郢州刺史,請討榮。梁武資其士馬,侵擾境上。孝武初,御史中尉樊子鵠為行臺,率徐州刺史杜德、舍人李昭等討之。樹城守不下,子鵠使金紫光祿大夫張安期說之。樹請委城還南,
 子鵠許之,殺白馬為盟。樹恃誓,不為戰備。與杜德別,還南。德不許,送洛陽,置在景明寺。樹年十五奔南,未及富貴。每見嵩山雲向南,未嘗不引領歔欷。初發梁,睹其愛姝玉兒,以金指環與別,樹常著之。寄以還梁,表必還之意。朝廷知之,俄而賜死。未幾,杜德忽得狂病,云:「元樹打我不已。」至死,此驚不絕。舍人李昭尋奉使向秦州,至潼關驛,夜夢樹云:「我已訴天帝,待卿至隴,終不相放。」昭覺,惡之。及至隴口,為賀拔岳所殺。子鵠尋為達野拔所殺。



 孝靜時,其子貞自建業求隨聘使崔長謙赴鄴葬樹,梁武許之。詔贈樹太師、司徒、尚書令。貞既葬,還江南,位太
 子舍人。及侯景南奔,梁武以貞為咸陽王。送景,使為魏主。未幾,景反。



 曄字世茂,梁封為桑乾王,卒於南。



 坦一名穆,字延和。傲狠凶粗,因飲醉之際,於洛橋左右頓辱行人,為道路所患。從叔安豐王延明每切責之曰:「汝兇悖性與身而長。昔宋有東海王禕,志性凡劣,時人號曰驢王。我熟觀汝所作,亦恐不免驢號。」當時聞者號為「驢王」。禧誅後,坦兄翼、樹等五人相繼南奔,故坦得承襲。改封敷城王。永安初,復本封咸陽郡王。累遷侍中。莊帝從容謂曰:「王才非荀、蔡,中歲屢遷,當由少長朕家,故有超授。」初,禧死後,諸子貧乏,坦兄弟為彭城王勰所收養,故有
 此言。



 孝武初,其兄樹見禽。坦見樹既長且賢,慮其代己,密勸朝廷以法除之。樹知之,泣謂坦曰:「我往因家難,不能死亡,寄食江湖,受其爵命。今者之來,非由義至,求活而已,豈望榮華?汝何肆其猜忌,忘在原之義!腰背雖偉,善無可稱。」



 坦作色而去。樹死,竟不臨哭。



 後歷司徒、太尉、太傅,加侍中、太師、錄尚書事、宗師、司州牧。雖祿厚位尊,貪求滋甚,賣獄鬻官,不知紀極。為御史劾奏,免官,以王歸第。尋起為特進,出為冀州刺史。專復聚斂,每百姓納賦,除常別先責絹五匹,然後為受。性好畋漁,無日不出。秋冬獵雉兔,春夏捕魚蟹,鷹犬常數百頭。自言寧三日
 不食,不能一日不獵。入為太傅。



 齊天保初,準例降爵,封新豐縣公,除特進、開府儀同三司。坐子世寶與通直散騎侍郎彭貴平因酒醉誹謗,妄說圖讖,有司奏當死。詔並宥之。坦配北營州,死配所。



 趙郡王乾字思直,太和九年,封河南王,位大將軍。孝文篤愛諸弟,以幹總戎別道,誡之曰:「司空穆亮年器可師,散騎常侍盧陽烏才堪詢訪,汝其師之。」遷洛,改封趙郡王。除都督、冀州刺史。帝親餞於郊,誡曰:「刑獄之理,先哲所難。



 然既有邦國,得不自勵也!」詔以李憑為長史,唐茂為司馬,盧尚之為諮議參軍,以匡弼之。而憑等諫,乾殊
 不納。州表斬盜馬人,於律過重,而尚書以乾初臨,縱而不劾。詔曰:「尚書曲阿朕意,實傷皇度。乾闇於政理,律外重刑,並可推聞。」



 後轉特進、司州牧。車駕南討,詔乾都督中外諸軍事,給鼓吹一部,甲士三百人,出入殿門。



 乾貪淫不遵政典,御史中尉李彪將糾劾之,會遇幹於尚書下舍,屏左右誡之,而幹悠然不以為意。彪表彈之。詔幹與北海王詳俱隨太子詣行在所。及至,密使左右察其意色,無有憂悔,乃親數其過,杖之一百,免所居官,以王還第。薨,謚曰靈王。陪葬長陵。



 子謐襲封。幹妃穆氏表謐及謐母趙等悖禮愆常。詔曰:「妾於女君,猶婦人事姑舅;
 妾子於君母,禮加如子之恭。何得黷我風猷,可付宗正依禮正罪。」謐在母喪,聽聲飲戲,為御史中尉李平所彈。遇赦,復封。後為岐州刺史。



 謐性暴虐,明帝初,臺使元延到其州界,以驛邏無兵,攝帥檢核。隊主高保願列言:「所有之兵,王皆私役。」謐聞,大怒,鞭保願等五人各二百。數日間,謐召近州人夫,閉四門,內外嚴固,搜掩城人,楚掠備至。又無事而斬六人,合城兇懼。眾遂大呼,屯門。謐怖,登樓毀梯以自固。士人散走,城人分守四門。靈太后遣游擊將軍王竫馳驛喻之。城人既見IP至,開門謝罪。乃罷謐州,除大司農卿。



 遷幽州刺史。謐妃胡氏,靈太后從
 女也。未發,坐毆其妃,免官。後除都官尚書。



 車駕出拜圓丘,謐與妃乘赤馬犯鹵簿,為御史所彈,靈太后特不問。薨,高陽王雍,乾之母弟,啟論謐,贈假侍中、司州牧,謚貞景。



 謐兄諶,字興伯,性平和,位都官尚書。爾朱榮之入洛陽,啟莊帝欲遷都晉陽。



 帝以問諶,爭之以為不可。榮怒曰:「何關君而固執也!且河陰之役,君應之。」



 諶曰:「天下事天下論之,何以河陰之酷而恐元諶!宗室戚屬,位居常伯,生既無益,死復何損!正使今日碎首流腸,亦無所懼。」榮大怒,欲罪諶。其從弟世隆固諫,乃止。見者莫不震悚。諶顏色自若。後數日,帝與榮見宮闕壯麗,列樹成行,乃
 歎曰:「臣一昨愚志,有遷京之意,今見皇居壯觀,亦何用去河洛而就晉陽。臣熟思元尚書言,深不可奪。」是以遷都議因罷。永安元年,拜尚書左僕射,封魏郡王。諶本年長,應襲王封,為其父靈王愛其弟謐,以為世子。莊帝詔復諶封趙郡王。



 歷位司空、太保、太尉、錄尚書事。孝靜初,拜大司馬。薨,謚孝懿。諶無他才識,歷位雖重,時人忽之。



 謐弟譚,頗強立,少為宗室推敬,卒於秦州刺史。



 譚弟讞,貪暴無禮。位太中大夫,封平鄉男。河陰遇害。



 廣陵王羽字叔翻,太和九年封,加侍中,為外都大官。羽少聰慧,有斷獄之稱。



 後罷三都,以羽為大理,典決京師
 獄訟。遷特進、尚書右僕射,又為太子太保、錄尚書事。孝文將南討,遣羽持節安撫六鎮。發其突騎,夷夏寧悅。還領廷尉卿。及車駕發,羽與太尉元丕留守。帝友愛諸弟,及將別,不忍早分,詔羽從至雁門。及令羽歸,望其稱效,故賜如意以表心。



 十八年,羽表辭廷尉,不許。羽奏:「外考令文,每歲終,州鎮列牧守績狀。



 及至再考,隨其品第,以彰黜陟。雖外有成令,而內令未班。內外考察,理應同等。



 臣輒推準外考,以定京官績行。」詔曰:「論考之事,理在不輕,問績之方,應關朕聽。輒爾輕發,殊為躁也。今始維夏,且待至秋。」後孝文臨朝堂考群臣,顧謂羽曰:「上下二等,
 可為三品,中等但為一品。所以然者,上下是黜陟之科,故旌絲髮之美;中等守本,事可大通。」



 帝又謂羽曰:「汝功勤之績不聞於朝,阿黨之音頻干朕聽。今黜汝錄尚書、廷尉,但居特進、太保。」又謂尚書令陸睿曰:「叔翻在省之初,甚著善稱;自近以來,偏頗懈怠。豈不由卿等隨其邪偽之心?今奪卿尚書令祿一周。」謂左僕射元贊曰:「計叔翻之黜,卿應大辟。但以咎歸一人,不復相罪。今解卿少師之任,削祿一周。」詔吏部尚書澄曰:「觀叔父神志驕傲,可解少保。」又謂長兼尚書于果曰:「卿不能勤謹夙夜,數辭以疾。今解卿長兼,可光祿大夫、守尚書,削祿一周。」



 又
 謂守尚書尉羽曰:「卿恭勤。在集書,殊無憂存左史之事。今降為長兼常侍,亦削祿一周。」又謂守尚書盧陽烏曰:「卿在集書,雖非高功,為一省文學之士,常不以左史在意。今降卿長兼王師,守常侍、尚書如故,奪常侍祿一周。」謂左丞公孫良、右丞乞伏義受曰:「卿等不能正心直言,罪應大辟。但以事鐘叔翻,故不能別致貶。二丞可以白衣守本官。冠服、祿恤盡皆削奪。若三年有成,還復本任;如其無成,則永歸南畝。」謂散騎常侍元景曰:「卿等自任集書,合省逋墮,致使王言遺滯,起居不脩。今降為中大夫、守常侍,奪祿一周。」又謂諫議大夫李彥:「卿實不稱職,
 可去諫議,退為元士。」又謂中庶子游肇及中書舍人李平:「識學可觀,可為中第。」



 初,孝文引陸睿、元贊等前,曰:「朕為天子,何假中原?欲令卿等子孫博見多知。若永居恆北,遇不好文主,卿等子弟不免面墻也。」陸睿對曰:「實如明詔。



 金氏若不入仕漢朝,七葉知名,亦不可得也。」帝大悅。



 帝幸羽第,與諸弟言曰:「朕親受人訟,知廣陵之明了。」咸陽王禧曰:「臣年為廣陵兄,明為廣陵弟。」帝曰:我為汝兄,汝為羽昆,汝復何恨!」車駕南伐,除開府、青州刺史。詔羽曰:「海服之寄,故唯宗良。唯酒唯田,可不誡歟!」宣武即位,遷司州牧。及帝覽政,引入內,面授司徒。請為司空,乃
 許之。羽先淫員外郎馮俊興妻,夜私游。為俊興所擊,積日祕匿,薨於府。宣武親臨哀,贈司徒,謚曰慧。



 子恭襲,是為節閔帝。



 恭兄欣,字慶樂,性粗率,好鷹犬。孝莊初,封沛郡王,後封淮陽王。孝武時,加太師、開府,復封廣陵王,太傅、司州牧,尋除大司馬。孝武入關中,欣投託人使達長安,為太傅、錄尚書事。欣於中興宗室,禮遇最隆,自廣平諸王,悉居其下。



 又為大宗師,進大冢宰、中軍大都督。大統中,為柱國大將軍、太傅。文帝謂欣曰:「王三為太傅,再為太師,自古人臣,示聞此例。」欣遜謝而已。後拜司徒。恭帝初,遷大丞相。薨,謚曰容。欣好營產業,多所樹藝,京師
 名果皆出其園。所汲引及寮佐咸非長者,為世所鄙。



 高陽王雍字思穆,少俶儻不恒。孝文曰:』吾亦未能測此兒之深淺,然觀其任真率素,或年器晚成。」太和九年,封潁川王。或說雍待士以營聲譽,雍曰:「吾天子之子,位為諸王,用聲名何為?」改封高陽。後為相州刺史。帝誡曰:「為牧之道,亦易亦難。其身正,不令而行,故便是易;其身不正,雖令不從,故曰是難。」



 宣武初,遷冀州刺史。雍在二州,微有聲稱,入拜司州牧。帝時幸雍第,皆盡家人禮。遷司空,轉太尉,加侍中。尋除太保,領太尉、侍中如故。明帝初,詔雍入居太極西柏堂,諮決大政,給親信二十人。又詔
 雍為宗師,進太傅、侍中,領太尉公,別敕將作營國子學寺,給雍居之。領軍于忠擅權專恣,僕射郭祚勸雍出之,忠矯詔殺祚及尚書裴植,廢雍以王歸第。朝有大事,使黃門就諮訪之。忠尋復矯詔將殺雍,以問侍中崔光,拒之乃止。未幾,靈太后臨朝,出忠為冀州刺史。雍表暴忠罪,陳己不能匡正,請返私門。靈太后感忠保護之勳,不問其罪。除雍侍中、太師,領司州牧。



 雍表請王公已下賤妾悉不聽用織成錦繡、金玉珠璣,違者以違旨論;奴婢悉不得衣綾錦纈,止於縵繒而已;奴則布服,並不得以金銀為釵帶,犯者鞭一百。太后從之,而不能久也。詔雍
 乘步挽出入掖門,又以本官錄尚書事,朝晡侍講。明帝覽政,詔雍乘車出入大司馬門,進位丞相。又詔依齊郡順王簡太和故事,朝訖引坐,特優拜伏之禮。總攝內外,與元叉同決庶政。歲祿粟至四萬石,伎侍盈房,榮貴之盛,昆弟莫及。



 元妃盧氏薨後,更納博陵崔顯妹,欲以為妃。宣武初以崔顯世號東崔,地寒望劣,難之,久乃聽許。延昌已後,疏棄崔氏,別房幽禁,僅給衣食而已。未幾,崔暴薨,多云雍毆殺也。靈太后許賜其女伎,未及送之。雍遣其閹豎丁鵝,自至宮內,料簡四人,冒以還第。太后責其專擅,追停之。孝莊初,於河陰遇害。贈假黃鉞、相國,謚
 文穆。



 雍識懷短淺,又無學業,雖位居朝首,不為時情所推。自熙平以後,朝政褫落。



 及清河王懌之死,元叉專政,天下大責歸焉。



 嫡之泰,字昌,頗有時譽,位太常卿,與雍同時遇害。贈太尉公、高陽王,謚曰文。子斌襲。



 斌字善集,歷位侍中、尚書左僕射。斌美儀貌,性寬和,居官重慎,頗為齊文襄愛賞。齊天保初,準例降爵為高陽縣公,拜右光祿大夫。二年,從文宣討契丹還,至白狼河,以罪賜死。



 彭城王勰字彥和,少而歧嶷,姿性不群。太和九年,封始平王,加侍中。勰生而母潘氏卒,其年獻文崩。及有所知,啟求追服,文明太后不許。乃毀容憔悴,心喪三年,不參
 吉慶。孝文大奇之。敏而耽學,雅好屬文。長直禁內,參決軍國大政,萬機之事無不預焉。及車駕南伐,領宗子軍,宿衛左右。轉中書令,侍中如故,改封彭城王。



 帝升金墉城,顧見堂後桐竹,曰:「凰皇非梧桐不栖,非竹實不食。今梧竹並茂,詎能降凰乎?」勰曰:「凰皇應德而來,豈桐竹能降?」帝笑曰:「朕亦未望降之。」後宴侍臣於清徽堂。日晏,移於流化池芳林下。帝仰觀桐葉之茂,曰:「『其桐其椅,其實離離。愷悌君子,莫不令儀。』今林下諸賢,足敷歌詠。」遂令黃門侍郎崔光讀暮春群臣應制詩。至勰詩,帝乃為改一字,曰:「昔祁奚舉子,天下謂之至公。今見勰詩,始知中
 令之舉非私也。」勰曰:「臣露此拙,方見聖朝之私,賴蒙神筆賜刊,得有令譽。」帝曰:「雖琱琢一字,猶是玉之本體。」勰曰:『《詩》三百,一言可蔽。今陛下賜刊一字,足以價等連城。」勰表解侍中,詔曰:「蟬貂之美,待汝而光。人乏之秋,何容方退。」後從幸代都,次于上黨之銅鞮山,路傍有大松樹十數根。時帝進傘,遂住而賦詩,令示勰曰:「吾作詩雖不七步,亦不言遠。汝可作之,比至吾間,令就也。」時勰去帝十步,遂且行且作,未至帝所而就。詩曰:「問松林,松林經幾冬?山川何如昔?風雲與古同?」帝大笑曰:「汝此亦調責吾耳!」詔贈勰所生母潘氏為彭城國太妃。又除中書監,
 侍中如故。



 帝南討漢陽,假勰中軍大將軍,加鼓吹一部。勰以寵授頻煩,乃面陳曰:「臣聞兼親疏而兩,並異同而建。此既成文於昔,臣願誦之於後。陳思求而不允,愚臣不請而得。豈但今古云殊,遇否大異。」帝大笑,執勰手曰:「二曹才名相忌,吾與汝以道德相親,緣此而言,無慚前烈。」



 帝親講《喪服》於清徽堂,從容謂群臣曰:「彥和、季豫等年在沖蒙,早登纓紱,失過庭之訓,並未習《禮》。每欲令我一解《喪服》。自審義解浮疏,仰而不許。頃因酒醉坐,脫爾言從,故屈朝彥,遂親傅說。」御史中尉李彪對曰:「自古及今,未有天子講《禮》。臣得親承音旨,千載一時。」



 從征沔北,
 除使持節、都督南征諸軍事,正中軍大將軍、開府。勰於是親勒大眾。須臾有二大鳥從南來,一向行宮,一向幕府,各為人所獲。勰言於帝曰:「始有一鳥,望旗顛仆,臣謂大吉。」帝戲之曰:「鳥之畏威,豈獨中軍之略也?吾亦分其一耳!此乃大善,兵法咸說。」至明,便大破崔慧景、蕭衍。其夜大雨。帝曰:「昔聞國軍獲勝,每逢雲雨。今破新野、南陽,及摧此賊,果降時潤,誠哉斯言。」



 勰對曰:「水德之應,遠稱天心。」帝令勰為露布,辭曰:「臣聞露布者,布於四海,露之耳目。以臣小才,豈足大用。」帝曰:「汝亦為才達,但可為之。」及就,尤類帝文,有人見者,咸謂御筆。帝曰:「汝所為者,人
 謂吾製。非兄則弟,誰能辨之?」勰對曰:「子夏被嗤於先聖,臣又荷責於來今。」及至豫州,帝為家人書於勰曰:「每欲立一宗師,肅我元族。汝親則宸極,官乃中監;風標才器,實足軌範,宗制之重,捨汝誰寄?有不遵教典,隨事以聞。」



 帝不豫,勰內侍醫藥,外總軍國之務,遐邇肅然,人無異議。徐謇,當世上醫。



 先是,假歸洛陽;及召至,勰引之別所,泣涕執手,祈請懇至。左右見者莫不鳴咽。



 及引入,謇便欲進藥。勰以帝神力虛弱,唯令以食味消息。勰乃密為壇於汝水濱,依周公故事,告天地及獻文,為帝請命,乞以身代。帝瘳損,自懸瓠幸鄴,勰常侍坐輿輦,晝夜不離
 其側,飲食必先嘗之而後手自進御。車駕還京,會百僚於宣極堂,行飲至策勛之禮,以勰功為群將之最。尋以勰為司徒、太子太傅,侍中如故。



 俄而齊將陳顯達內寇,帝復親討之。詔勰持節、都督中外諸軍事,總攝六師。



 時帝不豫,勰辭侍疾無暇,更請一王總當軍要。帝曰:「吾慮不濟,安六軍保社稷者,捨汝而誰?」帝至馬圈,疾甚,謂勰曰:「今吾當成不濟。霍子孟以異姓受付,況親賢,不可不勉也!」勰泣曰:「士於布衣,猶為知己盡命,況臣託靈先皇,誠應竭股肱之力。但臣出入喉膂,每跨時要,此乃周旦遁逃,成王疑惑。臣非所以辭勤請逸,正欲仰成陛下日
 鏡之明,下令愚臣獲避退之福。」帝久之曰:「吾尋思汝言,理實難奪。」乃手詔宣武曰:「汝第六父勰,清規懋賞,與白雲俱潔;厭榮捨紱,以松竹為心。吾少與綢繆,提攜道趣,每請朝纓,恬真丘壑。吾以長兄之重,未忍離遠,何容仍屈素業,長嬰世網。吾百年之後,其聽勰辭蟬捨冕,遂其沖挹之性也。」



 帝崩于行宮,遏祕喪事,獨與右僕射、任城王澄及左右數人為計,奉遷於安車中。勰等出入如平常,視疾進膳,可決外奏。累日,達宛城,乃夜進安車於郡事;得加斂櫬,還載臥輿。六軍內外,莫有知者。遣中書舍人張儒奉詔徵宣武會駕。梓宮至魯陽,乃發喪行服。
 宣武即位,勰跪授遺敕數紙。咸陽王禧疑勰為變,停於魯陽郡外,久之乃入。謂勰曰:「汝非但辛勤,亦危險至極。」勰恨之。對曰:「兄識高年長,故知有夷險。彥和掘蛇騎武,不覺艱難。」禧曰:「汝恨吾後至耳。」



 自孝文不豫,勰常居中,親侍醫藥,夙夜不離左右,至于衣不解帶,亂首垢面。帝患久多忿,因之遷怒。勰每被誚詈,言至厲切;威責近侍,動將誅斬。勰承顏悉心,多所匡濟。及帝昇遐,齊將陳顯達奔遁始爾,慮凶問泄漏,致有逼迫。勰內雖悲慟,外示吉容,出入俯仰,神貌無異。及至魯陽,東宮官屬多疑勰有異志,竊懷防懼;而勰推誠盡禮,卒無纖介之過。勰上
 謚議:「協時肇享曰孝,五宗安之曰孝,道德博聞曰文,經緯天地曰文,上尊號為孝文皇帝,廟號高祖,陵曰長陵。」帝從之。



 既葬,帝固以勰為宰輔。勰頻口陳遺旨,請遂素懷。帝對勰悲慟,每不許之。



 頻表懇切,帝難違遺敕,遂其雅情。猶逼以外任,乃以勰為都督、定州刺史。勰仍陳讓,帝不許,乃述職。帝與勰書,極家人敬,請勰入京。景明初,齊豫州刺史裴叔業以壽春內屬,詔勰都督南征諸軍事,與尚書令王肅迎接壽春。復授司徒。又詔以本官領揚州刺史,進位大司馬,領司徒。齊將陳伯之屯於肥口,胡松又據梁城。



 勰部分將士,頻戰破之。淮南平,徵勰還
 朝。初,勰之定壽春,獲齊汝陰太守王果、豫州中從事庾稷等數人。勰傾衿禮之,常參坐席。果承間求還江外,勰衿而許之。



 果又謝曰:「果等今還,仰負慈澤,請聽仁駕振旅,反跡江外。」至此乃還。其為遠人所懷如此。



 勰至京師,頻表辭大司馬、領司徒及所增邑,乞還中山,有詔不許。乃除錄尚書,侍中、司徒如故,固辭不免。時咸陽王禧以驕矜,頗有不法,北海王詳陰言於帝;又言勰大得人情,不宜久在宰輔,勸帝遵遺敕。禧等又出領軍于烈為恒州,烈深以為忿。烈子忠常在左右,密令忠言於帝,宜早自覽政。時將礿祭,王公並齋於廟東坊。帝遣于烈將壯
 士六十人召禧、勰、詳等引見。帝謂勰曰:「頃來南北務殷,不容仰遂沖操。恪是何人,而敢久違先敕?今遂叔父高蹈之意。」詔乃為勰造宅,務從簡素,以遂其心。勰因是作《蠅賦》以喻懷。又以勰為太師,勰遂固辭。詔侍中敦喻,帝又為書於勰,崇家人之敬,勰不得已而應命。帝前後頻幸勰第。及京兆、廣平王暴虐不法,制宿衛隊主率羽林、武賁幽守諸王於其第,勰上表切諫,帝不納。



 時議定律令,勰與高陽王雍、八坐、朝士有才學者,五日一集,參論軌制應否之宜。



 凡所裁決,時彥歸仰。又加侍中。勰敦尚文史,撰自古帝王賢達至於魏世子孫,族從為三十卷,
 名曰《要略》。



 性仁孝。言於朝廷,以其舅潘僧固為長樂太守。京兆王愉構逆,僧固見逼。尚書令高肇性既兇愎,又肇兄女入為夫人。順皇后崩,帝欲以為后,勰固執以為不可。



 肇於是屢譖勰,因僧固之同愉逆,肇誣勰與愉通,南招蠻賊。勰國郎中令魏偃、前防閣高祖珍希肇提攜,構成事。肇初令侍中元暉以奏,暉不從。又令左衛元珍言之。



 帝訪暉,明勰無此。帝更以問肇,肇以魏偃、祖珍為證,乃信之。



 永平元年九月,召勰及高陽王雍、廣陽王嘉、清河王懌、廣平王懷及高肇等人。



 時勰妃方產,固辭不得已,意甚憂懼,與妃訣而登車。入東掖門,度一小橋,牛
 傷,人挽而入。宴於禁中,夜皆醉,各就別所消息。俄而元珍將武士齎毒酒至。勰曰:「一見至尊,死無恨也。」珍曰:「至尊何可復見!」武士以刀環築勰二下,勰大言稱冤。武士又以刀築勰,乃飲毒酒,武士就殺之。向晨,以褥裹屍,輿從屏門出,載屍歸第,云因飲而薨。勰妃李氏,司空沖之女也,號哭曰:「高肇枉理殺人,天道有靈,汝還當惡死。」及肇以罪見殺,還於此屋,論者知有報應焉。帝為舉哀於東堂。勰既有大功於國,無罪見害,行路士女皆流涕曰:「高肇小人,枉殺如此賢王!」在朝貴戚莫不喪氣。景明、報德寺僧鳴鐘欲飯,忽聞勰薨,二寺一千餘人皆嗟痛,為
 之不食,但飲水而齋。追贈假黃鉞、使持節、都督中外諸軍事、司徒公、太師。給鑒輅九旒,武賁班劍百人,前後部羽葆鼓吹,轀輬車。有司奏太常卿劉芳議勰謚,保大定功曰武,善問周達曰宣,宜謚武宣王。詔可。及莊帝即位,追號文穆皇帝,妃李氏為文穆皇后,遷神主於太廟,稱肅祖。節閔帝時,去其神主。嫡子劭,字子訥,襲封。



 劭善武藝,少有氣節。明帝初,梁將寇邊,劭表上粟九千斛、資絹六百匹、國吏二百人以充軍用。靈太后嘉其至意,不許。累遷青州刺史。孝昌末,靈太后失德,四方紛擾,劭遂有異志。為安豐王延明所啟,徵入為御史中尉。莊帝即位,
 尊為無上王。尋遇害河陰。追謚曰孝宣皇帝,妻李氏為文恭皇后。



 子韶,字世胄,好學,美容儀。初,爾朱榮將入洛,父劭恐,以韶寄所親榮陽太守鄭仲明。仲明尋為城人所殺。韶因亂,與乳母相失,遂與仲明兄子僧副避難。



 路中為賊逼,僧副恐不免,因令韶下馬。僧副謂客曰:「窮鳥投人,尚或矜愍,況諸王如何棄乎?」僧副舉刃逼之,客乃退。韶逢一老母姓程,哀之,隱於私家。居十餘日,莊帝訪而獲焉,襲封彭城王。齊神武後以孝武帝后配之,魏室奇寶多隨后入韶家。有二玉缽相盛,轉而不可出。馬腦榼容三升,玉縫之。皆稱西域鬼作也。



 歷位太尉、侍中、錄
 尚書事、司州牧、特進、太傅。



 齊天保元年,降爵為縣公。韶性行溫裕,以高氏婿,頗膺時寵。能自謙退,臨人有惠政,好儒學;禮致才彥,愛林泉,修第宅華而不侈。文宣常剃韶鬢鬚,加以粉黛,衣婦人服以自隨。曰:「以彭城為嬪御。」譏元氏微弱,比之婦女。



 十年,太史奏云:「今年當除舊布新。」文宣謂韶曰:「漢光武何故中興?」



 韶曰:「為誅諸劉不盡。」於是乃誅諸元以厭之。遂以五月誅元世哲、景武等二十五家,餘十九家並禁止之。韶幽於京畿地牢,絕食,啖衣袖而死。及七月,大誅元氏,自昭成已下並無遺焉。或父祖為王,或身常貴顯,或兄弟強壯,皆斬東市。其嬰兒
 投於空中,承之以槊。前後死者凡七百二十一人,悉投屍漳水。剖魚者多得爪甲,都下為之久不食魚。世哲從弟黃頭,使與諸囚自金凰臺各乘紙鴟以飛,黃頭獨能至紫陌乃墜,仍付御史獄,畢義雲餓殺之。



 北海王詳字季豫,美姿容,善舉止。太和九年封,加侍中。孝文自洛北巡,詳常與侍中彭城王勰並在輿輦,陪侍左右。至文成射銘所,帝停駕,詔諸弟及侍臣皆試射遠近。諸人皆去一二十步,唯詳箭及之。帝拊掌欣笑,遂詔勒銘,親自為制。



 車駕南伐,詳行中領軍,留守。孝文臨崩,顧命詳為司空輔政。



 宣武覽政,為中大將軍、錄尚書事。
 咸陽王禧之謀反,詳表求解任,制不許。



 除太尉、領司徒、侍中,錄尚書事如故。詳之拜命,其夜暴風震電,拔其廷中桐,樹大十圍,倒立本處。初,宣武之覽政,詳聞彭城王勰有震主之慮,而欲奪其司徒,大懼物議,故為大將軍,至是乃居之。天威如此,識者知其不終。



 既以季父崇寵,位望兼極,貪冒無厭,公私營販。又於東掖門外規占第宅,至有喪柩在室,請延至葬而不見許,輿櫬巷次,行路哀嗟。詳母高太妃頗助威虐,怨嗷然。妃宋王劉昶女,不見答禮。寵妾范氏,愛等伉儷。及死葬訖,猶毀隧視之。



 又烝於安定王燮妃高氏,即茹皓妻姊。詳既素附於皓,
 又緣淫好,往來綢密。詳雖貪侈,宣武禮敬尚隆。常別住華林圓西隅,與都亭宮館相接。帝每潛幸其所,肆飲終日,與高太妃相見,呼為阿母,伏而上酒,禮若家人。臨出,高每拜送,舉觴祝言:「願官家千萬年壽,歲一入妾母子舍也。」初,宣武之親政,詳與咸陽王禧、彭城王勰並被召入,共乘犢車,防衛嚴固。高時惶迫,以為必死,亦乘車傍路哭送至金墉。及詳得免,高云:「自今以後,不願富貴。但令母子相保,共汝掃市作活也。」至此,貴寵崇盛,不復言有禍敗之理。



 後為高肇所譖,云詳與皓等謀逆。時詳在南第。帝召中尉崔亮入禁,糾詳貪淫,及茹皓、劉胄、常季
 賢、陳掃靜等專恣之狀。夜即收禁南臺。又武賁百人,圍守詳第。夜中慮其驚懼奔越,遣左右郭翼開金墉門馳出喻之,示以中尉彈狀。詳母高見翼,頓首號泣,不能自勝。詳言:「審如中尉所糾,何憂也?人奉我珍異貨物,我實受之,果為取受,吾何憂乎?」至明,皓等皆賜死。引高陽王雍等五王入議詳罪。



 單車防守還華林館。母妻相與哭,入所居,小奴弱婢數人隨從。防援甚嚴。徙就太府寺,免為庶人。別營坊館於洛陽縣東北隅,如法禁衛,限以終身,名曰思善堂,將徙詳居之。會其家奴陰結黨輩,欲劫出,密抄名字,潛託侍婢通於詳。詳始得執省,而門防主
 司遙見,突入,就詳手中覽得,呈奏。帝密令害之。詳自至太府,令其母妻還居南宅,五日一來。此夜,母妻不來,死於奴婢手中。詔喪還南宅,諸王皇宗,悉令奔赴。賵物一依廣陵故事。詳之初禁,乃以淫高事告母。母大怒,詈之曰:「汝自有妻妾侍婢,少盛如花,何共高麗婢姦,令致此罪!我得高麗婢,當啖其肉。」乃杖詳背及兩腳百餘下。自行杖,力疲,乃使奴代。高氏素嚴,詳每有微罪,常加責罰,以絮裹杖。至是,去絮,皆至創膿。又杖其妃劉數十,云:「新婦大家女,門戶匹敵,何所畏而不檢校夫婿!」劉笑而受罰,卒無所言。詳貪淫之失,雖聞遠近,而死之日,罪無定
 名,遠近歎怪之。永平元年十月,詔追復王爵,謚曰平王。子顥襲。



 顥字子明,少慷慨,有壯氣。為徐州刺史,尋為御史彈劾,除名。後賊帥宿勤明遠、叱干騏驎等寇亂豳、華等州,乃復顥王爵,兼左僕射、西道行臺以討明遠。



 頻破賊,解豳、華之圍。後蕭寶夤等大敗於平涼,顥亦奔還京師。



 武泰初,為相州刺史,以禦葛榮。屬爾朱榮入洛,推莊帝,授顥太傅。顥以葛榮南侵,爾朱縱害,遂盤桓顧望,圖自安之策。事不諧,遂與子冠受奔梁。梁武以為魏主,假之兵將,令其北入。永安二年四月,於梁國城南登壇燔燎,年號孝基元年。莊帝詔濟陰王暉業於考城拒之,為
 顥所禽。莊帝北幸,顥遂入洛,改稱建武元年。



 顥以數千之眾,轉戰屢剋,據有都邑,號令自己。天下人情,想望風政。自謂天之所授,頗懷驕怠。宿昔賓客近習之徒,咸見寵待,干擾政事。又日夜縱酒,不恤軍國。所統南兵,陵竊市里,朝野失望。時又酷儉,公私不安。莊帝與爾朱榮還師討顥,顥自於河梁拒戰。冠受戰敗被禽。顥自轘轅出至臨潁,為臨潁縣卒所斬。



 初,顥入洛,其日暴風,欲入閶闔門,馬大驚不進,令人執轡乃入。有恒農楊曇華告人曰:「顥必無成,假服袞冕,不過六十日。」又諫議大夫元昭業曰:「昔更始自洛陽而西,初發,馬驚奔,觸北宮鐵柱,三
 馬皆死,而更始卒不成帝位。以古譬今,其兆一也。」至七月果敗。孝武初,贈太師、大司馬。



 顥弟頊,莊帝初,封東海王,位中書監。及顥入洛,成敗未分,便以意氣自得,為時人笑。顥敗,潛竄,為人執送,斬於都市。孝武初,贈太尉。



 孝文七男:林廢后生廢太子恂;文昭皇后生宣武皇帝、廣平武穆王懷;袁貴人生京兆王愉;羅夫人生清河文獻王懌、汝南王悅;鄭充華生皇子恌,未封,早夭。



 廢太子庶人恂,字元道。生而母死,文明太后撫視之,常置左右。年四歲,太后親為立名恂,字元道。於是大赦。太和十七年七月癸丑,立恂為皇太子。及冠恂於廟,孝文
 臨光極東堂,引恂入見,誡以冠義曰:「字汝元道,所寄不輕,汝當尋名求義,以順吾旨。」二十年,改字宣道。遷洛,詔恂詣代都,其進止儀體,帝皆為定。及恂入辭,帝曰:「今汝不應向代。但太師薨於恒壞,朕既居皇極之重,不容輕赴舅氏之喪,欲使汝展哀舅氏,拜汝母墓,一寫為子之情。山陵北海,汝至彼,太師事畢後日,宜一拜山陵。拜訖,汝族祖南安可一就問訊。在途當溫讀經籍,今日親見吾也。」後帝每歲徵幸,恂常留守,主執廟祀。



 恂不好書學,體貌肥大,深忌河、洛暑熱,意每追樂北方。中庶子高道悅數苦言致諫,恂甚銜之。孝文幸崧岳,恂留守金墉,謀
 欲召牧馬,輕騎奔代,手刃道悅於禁中。領軍元徽勒門防遏,夜得寧靜。帝聞之駭惋,外寢其事,仍至汴口而還。



 引恂數罪,與咸陽王禧等親杖恂。又令禧等更代百餘下,扶曳出外,不起者月餘。



 拘於城西別館。引見群臣於清徽堂,議廢之。司空、太子太傅穆亮,尚書僕射、少保李沖,並免冠稽首而謝。帝曰:「古人有言,大義滅親。此小兒今日不滅,乃是國家之大禍。脫待我無後,恐有永嘉之亂。」乃廢為庶人,置之河陽;服食所供,粗免飢寒而已。



 帝幸代,遂如長安,中尉李彪承閑密表,告恂復與左右謀逆。帝在長安,使中書侍郎邢巒與咸陽王禧奉詔齎椒
 酒詣河陽,賜恂死。時年十五餘。斂以粗棺常服,瘞於河陽城。二十二年冬,御史臺令史龍文觀坐法當死,告廷尉,稱恂前後被攝左右之日,有手書自理,不知狀。而中尉李彪、侍御史賈尚寢不為聞。賈坐繫廷尉。



 時彪免歸,帝在鄴,尚書表收彪赴洛,會赦,遂不窮其本末。賈尚出系,暴病數日死。



 初,帝將為恂娶司徒馮誕長女,以女幼,待年長,先為娉彭城劉長文、榮陽鄭懿女為左右孺子。時恂年十三四,帝嘗謂郭祚、崔光、宋弁曰:「人生須自放,不可終朝讀書。我欲使恂旦出省經傳,食後還內,晡時復出,日夕而罷。卿等以為何如?」光曰:「孔子稱血氣未定,
 戒之在色。太子尚以幼年涉學之日,不宜于正晝之時,捨書御內,又非所以安柔弱之體,固永年之命。」帝以光言為然,乃不令恂晝入內。無子。



 京兆王愉字宣德,太和二十一年封,拜都督、徐州刺史。以彭城王中宣府長史盧陽烏兼長史,州事巨細,委之陽烏。宣武初,為護軍將軍。帝留愛諸弟,愉等常出入宮掖,晨昏寢處,若家人焉。遷中書監。為納順皇后妹為妃,而不見禮答。愉在徐州納妾李氏,本姓楊,東郡人,夜聞其歌,悅之,遂被寵嬖。罷州還京,欲進貴之。託右中郎將趙郡李恃顯為之養父,就之禮迎,產子寶月。順皇后召
 李入宮,毀擊之。強令為尼於內,以子付妃養之。歲餘,后父于勁以后久無所誕,乃表勸廣嬪御。因令后歸李於愉,舊愛更甚。



 愉好文章,頗著詩賦。時引才人宋世景、李神俊、祖瑩、邢晏、王遵業、張始均等,共申宴喜。招四方儒學賓客嚴懷真等數十人,館而禮之。所得穀帛,率多散施。又崇信佛道,用度常至不接。與弟廣平王懷,頗相夸尚,競慕奢麗,貪縱不法。



 於是宣武攝愉禁中推案,杖愉五十,出為冀州刺史。



 始愉自以職求侍要,勢劣二弟,潛懷愧恨,頗見言色。又以幸妾屢被頓辱,內外離抑。及在州,謀逆。愉遂殺長史羊靈引及司馬李遵,稱得清河王
 密疏,云高肇謀為殺害主上。遂為壇於信都之南,柴燎告天,即皇帝位。赦天下,號建平元年,立李氏為皇后。宣武詔尚書李平討愉。愉出拒王師,頻敗,遂嬰城自守。愉知事窮,攜李及四子數十騎出門,諸軍追之,見執以送。詔征赴京師,申以家人之訓。愉每止宿亭傳,必攜李手,盡其私情。雖鎖縶之中,飲賞自若,略無愧懼之色。至野王,愉語人曰:「雖主上慈深,不忍殺我,吾亦何以面見至尊!」於是歔欷流涕,絕氣而死,年二十一。或云高肇令人殺之。斂以小棺,瘞。諸子至洛,皆赦之。後靈太后令愉之四子皆附屬籍,追封愉臨洮王。寶月乃改葬父母,追服
 三年。



 清河王懌字宣仁,幼而敏慧,美姿貌,孝文愛之。彭城王勰甚器異之,並曰:「此兒風神外偉,黃中內潤,若天假之年,繼二南矣。」博涉經史,兼綜群言,有文才,善談理。寬仁容裕,喜怒不形於色。太和二十一年封。宣武初,拜侍中,轉尚書僕射。懌才長從政,明於斷決,剖判眾務,甚有聲名。司空高肇以帝舅寵任,既擅威權,謀去良宗,屢譖懌及愉等。愉不勝其忿怒,遂舉逆冀州。因愉之逆,又構殺勰。懌恐不免。肇又錄囚徒以立私惠。懌因侍宴,酒酣,乃謂肇曰:「天子兄弟,詎有幾人,而炎炎不息?昔王莽頭禿,亦藉
 渭陽之資,遂篡漢室。今君曲形見矣,恐復終成亂階。」又言於宣武曰:「臣聞唯器與名,不可以假人。是故季氏旅泰山,宣尼以為深譏;仲叔軒懸,丘明以為至誡。諒以天尊地卑,君臣道別。宜杜漸防萌,無相僭越。至於減膳錄囚,人君之事,今乃司徒行之,詎是人臣之義?且陛下修政教,解獄訟,則時雨可降,玉燭知和。何使明君失之於上,姦臣竊之於下?



 長亂之基,於此在矣。」宣武笑而不應。



 孝明熙平初,遷太尉,侍中如故。詔懌裁門下之事,又典經義注。時有沙門惠憐者,自云咒水飲人,能差諸病。病人就之者,日有千數。靈太后詔給衣食。事力優重,使於城
 西之南,治療百姓病。懌表諫曰:「臣聞律深惑眾之科,禮絕妖淫之禁,皆所以大明居正,防遏姦邪。昔在漢末,有張角者,亦以此術,熒惑當時。論其所行,與今不異。遂能詃誘生人,致黃巾之禍。天下塗炭數十年間,角之由也。



 昔新垣姦,不登於明堂;五利僥,終嬰於顯戮。



 靈太后以懌孝明懿叔,德先具瞻,委以朝政,事擬周、霍。懌竭力匡輔,以天下為己任。領軍元叉,太后之妹夫也,恃寵驕盈。懌裁之以法,每抑黜之,為叉所疾。叉黨人通直郎宋準愛希又旨,告懌謀反。禁懌門下,訊問左右及朝貴,貴人分明,得雪,乃釋焉。懌以忠而獲謗,乃鳩集昔忠烈之士,
 為《顯忠錄》二十卷以見意焉。



 正光元年七月,叉與劉騰逼孝明於顯陽殿,閉靈太后於後宮,囚懌於門下省。



 懌罪伏,遂害之,時年三十四。朝野貴賤,知與不知,含悲喪氣,驚振遠近。夷人在京及歸,聞懌之喪,為之劈面者數百人。



 廣平王懷,闕自有魏諸王,召入華林別館,禁其出入。令四門博士董征授以經傳。孝武崩,乃得歸。



 汝南王悅,好讀佛經,覽書史;為性不倫,俶儻難測。悅妃閭氏,即東海公之女也。生一子,不見禮答。有崔延夏者,以左道與悅遊。合服仙藥松術之屬,時輕與出採之,宿
 於城外小人之所。遂斷酒肉粟稻,唯食麥飯。又絕房中,而更好男色。



 輕忿妃妾,至加捶撻,同之婢使。悅之出也,妃住於別第,靈太后敕檢問之。引入,窮悅事故。妃病杖床蓐,瘡尚未愈。太后因悅杖妃,乃下令禁斷。令諸親王及三蕃,其有正妃病患百日已上,皆遣奏聞。若有猶行捶撻,就削封位。



 及清河王懌為元叉所害,悅了無仇恨之意,乃以桑落酒候伺之,盡其私佞。叉大喜,以悅為侍中、太尉。臨拜日,就懌子亶求懌服玩之物。不時稱旨,乃召亶杖之百下。亶居廬未葬,形氣羸弱,暴加威撻,殆至不濟。仍呼阿兒,親自循撫。悅乃為大剉碓,置於州門,
 盜者便欲斬其手。時人懼其無常,能行異事,姦偷畏之而暫息。



 及爾朱榮舉兵向洛,悅遂奔梁。梁武厚相資待。莊帝崩,遂立為魏主,號年更興。節閔初,遣兵送悅,置於境上,以覬侵逼。及齊神武既誅爾朱,以悅孝文子,宜承大業,乃令人示意。悅既至,清狂如故,動為罪失,乃止。孝武初,除大司馬、開府。孝武以廣陵頗有德望,以悅屬尊地近,內懷畏忌,故前後害之。贈假黃鉞、太師、司州牧,大司馬、王如故。謚曰文宣。



 子穎,與父俱奔梁,遂卒於江左。



 皇子恌,年七歲,景明元年薨,就斂於華林棗間堂,葬於文昭皇后陵東。後以增廣文昭后墳塋,徙窆北崗。



 論曰:文成五王,安豐特標令望。延明學業該贍,加以雅談之美;及于永安,運迹冠戎。卒致奔亡,亦其命也。



 獻文諸子,俱漸太和之訓,而咸陽終於逆節,廣陵斃于桑中。人而無儀,各宜遄死。高陽器術缺然,終荷棟幹,至於橈敗,實尸其闕。武宣孝以為質,忠而樹行,及夫在安處危之操,送往事居之節,周旦匪佗之旨,霍光異姓之誠,事實兼之。竟而功高震主,德隆動俗。閑言一入,卒不全生。鳴呼!周成、漢昭未易遇也。北海義昧鶺鴒,奢淫自喪,雖禍發青蠅,亦自貽伊戚。顥取若拾遺,亡不旋踵,豈守之無術,其天將覆之。



 庶人險暴之性,自幼而長,終以廢黜,
 不得其終。斯乃硃、均之性,堯、舜不能訓也。京兆早有令問,晚致顛覆,習於所染,可不慎乎!清河器識才譽,以懿親作輔,時鐘屯詖,始遘墻茨之逼。運屬道消,晚扼兇權之手。悲哉!廣平早歲驕盈,汝南性致狂逸,揆其終始,俱不足論。而悅以天人所棄,卒嬰猜懼之毒,蓋地逼之尤也。



 魏自西遷之後,權移周室。而周文天縱寬仁,性罕猜忌;元氏戚屬,並見保全,內外任使,布於列職。孝閔踐祚,無替前緒,明武纘業,亦遵先志。雖天厭魏德,鼎命已遷,枝葉榮茂,足以愈於前代矣。



\end{pinyinscope}