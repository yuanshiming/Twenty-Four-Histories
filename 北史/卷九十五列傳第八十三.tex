\article{卷九十五列傳第八十三}

\begin{pinyinscope}

 蠻獠林邑赤土真臘婆利蠻之種類,蓋盤瓠之後。在江、淮之間,部落滋蔓,布於數州,東連壽春,西通巴、蜀,北接汝、潁,往往有焉。其於魏氏,不甚為患,至晉之末,稍以繁昌,漸為寇暴矣。自劉、石亂後,諸蠻無所忌憚,故其族漸得北遷,陸渾以南,滿於山谷,宛、洛蕭條,略為丘墟矣。



 道武既定中山,聲教被於河表。泰常八年,蠻王梅安率渠帥數千朝京師,求留質子,
 以表忠款。始光中,拜安侍子豹為安遠將軍、江州刺史、順陽公。興光中,蠻王文武龍請降,詔褒慰之,拜南雍州刺史、魯陽侯。



 延興中,大陽蠻首桓誕擁沔水以北,滍葉以南,八萬餘落,遣使內屬。孝文嘉之,拜誕征南將軍、東荊州刺史、襄陽王,聽自選郡縣。誕字天生,桓玄之子也。



 初,玄西奔至枚迥洲被殺,誕時年數歲,流竄大陽蠻中,遂習其俗。及長,多智謀,為群蠻所歸。誕既內屬,居朗陵。太和四年,王師南伐,誕請為前驅。乃授使持節、南征西道大都督,討義陽,不果而還。十年,移居潁陽。十六年,依例降王為公。



 十七年,加征南將軍、中道大都督,徵竟陵。
 遇遷洛,師停。是時,齊征虜將軍、直閣將軍蠻首田益宗率部曲四千餘戶內屬。襄陽首雷婆思等十一人率戶千餘內自徙,求居大和川,詔給廩食。後開南陽,令有沔北之地,蠻人安堵,不為寇賊。十八年,誕入朝,賞遇隆厚。卒,謚曰剛。子暉,字道進,位龍驤將軍、東荊州刺史,襲爵。



 景明初,大陽蠻首田育丘等二萬八千戶內附,詔置四郡十八縣。暉卒。贈冠軍將軍。



 三年,魯陽蠻魯北燕等聚眾攻逼,頻詔左衛將軍李崇討平之,徙萬餘家於河北諸州及六鎮。尋叛南走,所在追討,比及河,殺之皆盡。四年,東荊州蠻樊素安反,僭帝號。正始元年,素安弟秀安復
 反,李崇、楊大眼悉討平之。二年,梁沔東太守田清喜擁七郡三十一縣、戶萬九千,遣使內附,乞師討梁。其雍州以東,石城以西,五百餘里水陸援路,請率部曲斷之。四年,梁永寧太守文雲生六部,自漢東遣使歸附。



 永平初,東荊州表太守桓叔興前後招慰大陽蠻,歸附者一萬七百戶,請置郡十六、縣五十,詔前鎮東府長史酈道元檢行置之。叔興即暉弟也,延昌元年,拜南荊州刺史,居安昌,隸於東荊。三年,梁遣兵討江、沔,破掠諸蠻,百姓擾動。蠻自相督率二萬餘人,頻請統帥,蠻以為聲勢。叔興給一統并威儀,為之節度,蠻人遂安。其年,梁雍州刺史蕭
 藻遣其將蔡令孫等三將寇南荊之西南,沿襄、沔上下,破掠諸蠻,蠻首梁龍驤將軍樊石廉叛梁,來請援。遣叔興與石廉督集蠻夏二萬餘人擊走之,斬令孫等三將。藻又遣其新陽太守邵道林,於沔水之南石城東北立清水戍,為抄掠之基,叔興遣諸蠻擊破之。四年,叔興上表,請不隸東荊,許之。梁人每有寇抄,叔興必摧破之。



 正光中,叔興擁所部南叛。蠻首成龍強率戶數千內附,拜刺史;蠻帥田牛生率戶二千內徙揚州,拜為郡守。梁義州刺史邊城王文僧明、鐵騎將軍邊城太守田官德等率戶萬餘,舉州內屬。拜僧明平南將軍、西豫州刺史,封
 開封侯;官德龍驤將軍、義州刺史;自餘封授各有差。僧明、官德並入朝。蠻出山至邊城、建安者,八九千戶。義州尋為梁將裴邃所陷。梁定州刺史田超秀亦遣使求附,請援歷年,朝廷恐輕致邊役,未之許。會超秀死,其部曲相率內附,徙之。六鎮、秦、隴所在反叛,二荊、西郢蠻大擾動,斷三鵶路,殺都督,寇盜至於襄城、汝水,百姓多被其害。梁遣將圍廣陵,楚城諸蠻,並為前驅。自汝水以南,恣其暴掠,連年攻討,散而復合,其暴滋甚。



 又有冉氏、向氏、田氏者,陬落尤盛。餘則大者萬家,小者千戶,更相崇樹,僭稱王侯。屯據三峽,斷遏水路,荊蜀行人,至有假道者。周
 文略定伊、瀍,聲教南被,諸蠻畏威,靡然向風矣。大統五年,蔡陽蠻王魯超明內屬,授南雍州刺史,仍世襲焉。十一年,蠻酋梅勒特來貢其方物。尋而蠻帥田杜青及江、漢諸蠻擾動,大將軍楊忠擊破之。其後蠻帥杜青和自稱巴州刺史,入附,朝廷因其所稱而授之。



 杜青和後遂反,攻圍東梁州。其唐州蠻田魯嘉亦叛,自號豫州伯。王雄、權景宣等前後討平之。



 廢帝初,蠻首樊舍舉落內附,以為督淮北三州諸軍事、淮州刺史、淮安郡公。



 于謹等平江陵,諸蠻騷動,詔豆盧寧、蔡祐等討破之。恭帝二年,蠻酋宜人王田興彥、北荊州刺史梅季昌等相繼款
 附。以興彥、季昌並為開府儀同三司,加季昌洛州刺史,賜爵石臺縣公。其後,巴西人譙淹扇動君蠻以附梁,蠻帥向鎮侯、向白虎等應之;向五子王又攻陷信州;田烏度、田唐等抄斷江路;文子榮復據荊州之汶陽郡,自稱仁州刺史;并鄰州刺史蒲微亦舉兵逆命。詔田弘、賀若敦、潘招、李遷哲等討破之。周武成初,文州蠻叛,州軍討定之。尋而冉令賢、向五子王等又攻陷白帝,殺開府楊長華,遂相率作亂。前後遣開府元契、趙剛等總兵出討,雖頗翦其族類,而元惡未除。天和元年,詔開府陸騰督王亮、司馬裔等討之。騰水陸俱進,次于湯口,先遣喻
 之。而令賢方增浚城池,嚴設扞禦,遣其長子西黎、次子南王領其支屬,於江南險要之地,置立十城,遠結涔陽蠻為其聲援。令賢率其卒,固守水邏城。騰乃總集將帥謀進趣,咸欲先取水邏,然後經略江南。騰言於眾曰:「令賢內恃水邏金湯之險,外託涔輔車之援,兼復資糧充實,器械精新。以我懸軍,攻其嚴壘,脫一戰不剋,更成其氣。不如頓軍湯口,先取江南,翦其毛羽,然後遊軍水邏,此制勝之計也。」眾皆然之。乃遣開府王亮率眾渡江,旬日攻拔其八城,凶黨奔散,獲賊帥冉承公并生口三千人,降其部眾一千戶。遂簡募驍勇,數道分攻水邏。路
 經石壁城,險峻,四面壁立,故以名焉。唯有一小路,緣梯而上,蠻蜒以為峭絕,非兵眾所行。騰被甲先登,眾軍繼進,備經危阻,累日乃得舊路。且騰先任隆州總管,雅知其路蠻帥冉伯犁、冉安西與令賢有隙。騰乃招誘伯犁等,結為父子,又多遺錢帛。伯犁等悅,遂為鄉導。水邏側又有石勝城者,亦是險要,令賢使其兄龍真據之。



 勝又密告龍真云,若平水邏,使其代令賢處之。龍真大悅,遣其子詣騰。乃厚加禮接,賜以金帛。蠻貪利既深,仍請立效,乃謂騰曰:「欲翻所據城,恐人力寡少。」



 騰許以三百兵助之。既而遣二千人,銜枚夜進,龍真力不能禦,遂平石
 勝城。晨至水邏,蠻眾大潰,斬首萬餘級。令賢遁走,追而獲之。司馬裔又別下其二十餘城,獲蠻帥冉三公等。騰乃積其骸骨於水邏城側為京觀,後蠻蜒望見輒大哭,自此狼戾之心輟矣。



 時向五子王據石墨城,令其子寶勝據雙城。水邏平後,頻遺喻之,而五子王猶不從命。騰又遣王亮屯牢坪,司馬裔屯雙城以圖之。騰慮雙城孤峭,攻未可拔,賊若委城遁散,又難追討。乃令諸軍周迴立柵,遏其走路,賊乃大駭。於是縱兵擊破之,禽五子王於石墨,獲寶勝於雙城,悉斬諸向首領,生禽萬餘口。信州舊居白帝,騰更於劉備故宮城南,八陳之北,臨江岸築
 城,移置信州。又以巫縣、信陵、秭歸並築城置防,以為襟帶焉。



 天和六年,蠻渠冉祖裛、冉龍驤又反,詔大將軍趙訚討平之。自此群蠻懼息,不復為寇。



 獠者,蓋南蠻之別種,自漢中達于邛、笮,川洞這間,所在皆有。種類甚多,散居山谷,略無氏族之別。又無名字,所生男女,唯以長幼次第呼之。其丈夫稱阿謨、阿段,婦人阿夷、阿等之類,皆語之次第稱謂也。依樹積木,以居其上,名曰干闌,干闌大小,隨其家口之數。往往推一長者為王,亦不能遠相統攝。父死則子繼,若中國之貴族也。獠王各有鼓角一雙,使其子弟自吹擊之。好相殺害,多
 死,不敢遠行。能臥水底持刀刺魚,其口嚼食並鼻飲。死者,豎棺而埋之。性同禽獸,至於忿怒,父子不相避,唯手有兵刃者先殺之。若殺其父,走避外,求得一狗以謝,不復嫌恨。若報怨相攻擊,必殺而食之;平常劫掠,賣取豬狗而已。親戚比鄰,指授相賣。被賣者號哭不服,逃竄避之,乃將買人指捕,逐若亡叛,獲便縛之。但經被縛者,即服為賤隸,不敢稱良矣。亡失兒女,一哭便止,不復追思。唯執楯持矛,不識弓矢。用竹為簧,群聚鼓之,以為音節。能為細布,色至鮮凈。大狗一頭,賣一生口。其俗畏鬼神,尤尚淫祀。所殺之人美鬢髯者,乃剝其面皮,籠之於竹,
 及燥,號之曰鬼,鼓舞祀之,以求福利。至有賣其昆季妻孥盡者,乃自賣以供祭焉。



 鑄銅為器,大口寬腹,名曰銅爨,既薄且輕,易於熟食。



 建國中,李勢在蜀,諸獠始出巴西、渠川、廣漢、陽安、資中,攻破郡國,為益州大患。勢內外受敵,所以亡也。自桓溫破蜀之後,力不能制。又蜀人東流,山險之地多空,獠遂挾山傍谷。與夏人參居者,頗輸租賦;在深山者,仍不為編戶。



 梁、益二州歲伐獠,以裨潤公私,頗藉為利。



 正始中,夏侯道遷舉漢中內附,宣武遣尚書邢巒為梁、益二州刺史以鎮之,近夏人者安堵樂業,在山谷者不敢為寇。後以羊祉為梁州,傅豎眼為益
 州。祉性酷虐,不得物情。梁輔國將軍范季旭與獠王趙清荊率眾屯孝子谷,祉遣統軍魏胡擊走之。



 後梁寧朔將軍姜白復擁夷獠入屯南城,梁州人王法慶與之通謀,眾屯於固門川。祉遣征虜將軍討破之。豎眼施恩布信,大得獠和。後以元法僧代傅豎眼為益州,法僧在任貪殘,獠遂反叛,勾引梁兵,圍逼晉壽。朝廷憂之,以豎眼先得物情,復令乘傳往撫。獠聞豎眼至,莫不欣然,拜迎道路,於是而定。及元桓、元子真相繼為梁州,並無德績,諸獠苦之。其後,朝廷以梁、益二州控攝險遠,乃立巴州以統諸獠。



 後以巴酋嚴始欣為刺史。又立隆城鎮,所綰
 獠二十萬戶。彼謂北獠,歲輸租布,又與外人交通貿易。巴州生獠,並皆不順,其諸頭王,每於時節謁見刺史而已。孝昌初,諸獠以始欣貪暴,相率反叛,攻圍巴州。山南行臺魏子建勉喻,即時散罷。自是獠諸頭王,相率詣行臺者相繼,子建厚勞賚之。始欣見中國多事,又失彼心,慮獲罪譴,時梁南梁州刺史陰子春扇惑邊陲,始欣謀將南叛。始欣族子愷時為隆城鎮將,密知之,嚴設邏候,遂禽梁使人,并封始欣詔書、鐵券、刀劍、衣冠之屬,表送行臺。子建乃啟以豎眼久病,
 其子敬紹納始欣重賂,使得還州。始欣乃起眾攻愷,屠滅之,據城南叛。梁將蕭玩,率眾援接。時梁、益二州並遣將討之,攻陷巴州,執始欣,遂大破玩軍。及斬玩,以傅曇表為刺史。後元羅在梁州,為所陷,自此遂絕。



 及周文平梁、益之後,令在所撫慰,其與華人雜居者,亦頗從賦役。然天性暴亂,旋致擾動。每歲命隨近州鎮,出兵討之,獲其生口,以充賤隸,謂之為壓獠焉。



 後有商旅往來者,亦資以為貨,公卿達於人庶之家,有獠口者多矣。恭帝三年,陵州木籠獠反,詔開府陸騰討破之。周保定二年,鐵山獠又反,抄斷江路,陸騰又攻拔其三城。天和三年,梁
 州恒棱獠叛,總管長史趙文表討之。軍次巴州,文表欲率眾徑進。軍吏等曰:「此獠旅拒日久,部眾甚強,討之者四面攻之,以分其勢。今若大軍直進,不遣奇兵,恐人併力於我,未可制勝。」文表曰:「往者既不能制之,今須別為進趣。若四面遣兵,則獠降走路絕,理當相率以死拒戰;如從一道,則吾得示威恩,分遣人以理曉諭,為惡者討之,歸善者撫之,善惡既分,易為經略。事有變通,奈何欲遵前轍也?」文表遂以此意,遍令軍中。時有從軍熟獠,多與恆棱親識,即以實報之。恒棱獠相與聚議,猶豫之間,文表軍已至其界。獠中先有二路,一路稍平,一路極險。俄
 有生獠酋帥數人來見文表曰:「我恐官軍不識山川,請為鄉導。」文表謂之曰:「此路寬平,不須導引,卿但先去,好慰喻子弟也。」乃遣之。文表謂其眾曰:「向者獠帥,謂吾從寬路而行,必當設伏險要。若從險路,出其不慮,獠眾自離散矣。」於是勒兵從險道進,其有不通之處,即平之。乘高而望,果見其伏兵。獠既失計,爭攜妻子,退保險要。文表頓軍大蓬山下,示禍福,遂相率來降。文表皆撫慰之,仍徵其租稅,無敢動者。後除文表為蓬州刺史,又大得人和。



 建德初,李暉為蓬、梁州總管,諸獠亦望風從附。然其種滋蔓,保據嚴壑,依山走險,若履平地,雖屢加兵,弗
 可窮討。性又無知,殆同禽獸,諸夷之中,最難以道招懷者也。



 林邑,其先所出,事具《南史》。其國延袤數千里,土多香木、金寶,物產大抵與交趾同。以磚為城,蜃灰塗之,東向戶。尊官有二,其一曰西那婆帝,其二曰薩婆地歌。其屬官三等,其一曰倫多姓,次歌倫致帝,次乙地伽蘭。外官分為二百餘部,其長官曰弗羅,次曰可輪,如牧宰之差也。王戴金花冠,形如章甫,衣朝霞布,珠璣纓絡,足躡革履,時服錦袍。良家子侍衛者二百許人,皆執金裝。兵有弓、箭、刀、槊。以為竹為弩,傅毒於矢。樂有琴、笛、琵琶、五絃,頗與中
 國同。每擊鼓以警眾,吹蠡以即戎。其人深目高鼻,髮拳色黑。俗皆徒跣,以幅巾纏身,冬月衣袍。婦人椎髻。施椰葉席。每有婚媾,令媒者齎金銀釧、酒二壺、魚數頭至女家,於是擇日,夫家會親賓,歌舞相對,女家請一婆羅門送女至男家,婿盥手,因牽女授之。王死,七日而葬;有官者,三日;庶人,一日。皆以函盛屍,鼓舞導從,輿至水次,積薪焚之。收其餘骨,王則內金罌中,沉之於海;有官者,以銅罌,沉之海口;庶人以瓦,送之於江。男女皆截髮,哭至水次,盡哀而止,歸則不哭。每七日,燃香散花,復哭盡哀而止,百日、三年皆如之。人皆奉佛,文字同於天竺。



 隋文
 帝既平陳,乃遣使獻方物,後朝貢遂絕。時天下無事,群臣言林邑多奇寶者。仁壽末,上遣大將軍劉方為驩州道行軍總管,率欽州刺史寧長真、驩州刺史李暈、開府秦雄步騎萬餘,及犯罪者數千人擊之。其王梵志乘巨象而戰,方軍不利。



 方乃多掘小坑,草覆其上,因以兵挑之。方與戰偽北,梵志逐之,其象陷,軍遂亂,方大破之,遂棄城走。入其都,獲其廟主十八枚,皆鑄金為之,盡其國有十八世。



 方班師,梵志復其故地,遣使謝罪,於是朝貢不絕。



 赤土國,扶南之別種也。在南海中,水行百餘日而達。所
 都土色多赤,因以為號。東波羅刺國,西婆羅娑國,南訶羅旦國,北拒大海,地方數千里。其王姓瞿曇氏,多利富多塞,不知有國近遠。稱其父釋王位,出家為道,傳位於利富多塞,在位十六年矣。有三妻,並鄰國女也。居僧祗城,有門三重,相去各百許步。每門圖畫菩薩飛仙之象,懸金花鈴眊,婦人數十人,或奏樂,或捧金花。又飾四婦人,容飾如佛塔邊金剛力士之狀,夾門而立,門外者持兵仗,門內者執白拂。夾道垂素網,綴花。王宮諸屋,悉是重閣北戶。北面而坐三重之榻,衣朝霞布,冠金花冠,垂雜寶纓絡,四女子立侍左右,兵衛百餘人。王榻後作
 一木龕,以金銀五香木雜鈿之,龕後懸一金光焰;夾榻又樹二金鏡,鏡前並陳金甕,甕前各有金香爐;當前置一金伏牛,前樹一寶蓋,左右皆有寶扇。婆羅門等數百人,東西重行,相向而坐。其官:薩陀加邏一人,陀拏達叉一人,迦利密迦三人,共掌政事;俱羅末帝一人,掌刑法。



 每城置那邪迦一人,缽帝十人。



 其俗,皆穿耳翦髮,無跪拜之禮,以香油塗身。其俗敬佛,尤重婆羅門。婦人作髻於項後,男女通以朝霞朝雲雜色布為衣。豪富之室,恣意華靡,唯金鎖非王賜不得服用。每婚嫁,擇吉日,女家先期五日,作樂飲酒,父執女手以授婿,七日乃配。既娶,即
 分財別居,唯少子與父居。父母兄弟死,則剔髮素服,就水上構竹木為棚,棚內積薪,以屍置上,燒香建幡,吹蠡擊鼓以送,火焚薪,遂落於水。貴賤皆同、唯國王燒訖收灰,貯以金瓶,藏於廟屋。冬夏常溫,雨多霽少,種植無時。



 特宜稻、穄、白豆、黑麻,自餘物產,多同於交趾。以甘蔗作酒,雜以紫瓜根,酒色黃赤,味亦香美。亦以椰漿為酒。



 隋煬帝嗣位,募能通絕域者。大業三年,屯田主事常駿、虞部主事王君政等請使赤土。帝大悅,遣齎物五千段以賜赤土王。其年十月,駿等自南海郡乘舟,晝夜二旬,每日遇便風。至焦石山而過,東南詣陵伽缽拔多洲,西與
 林邑相對,上有神祠焉。又南行,至師子石。自是島嶼連接。又行二三日,西望見狼牙須國之山,於是南達雞籠島,至於赤土之界。



 其王遣婆羅門鳩摩羅,以舶三百艘來迎,吹蠡擊鼓樂隋使,進金鎖以纜船。月餘,至其都。王遣其子那邪迦請與駿等禮見。先遣人送金盤貯香花並鏡鑷,金合二枚貯香油,金瓶二枚貯香水,白疊布四條,以擬供使者盥洗。其日未時,那邪迦又將象二頭,持孔雀獸以迎使人,并致金盤、金花,以藉詔函,男女百人奏蠡鼓,婆羅門二人導路。至王宮,駿等奉詔書上閣,王以下皆坐,宣詔訖,引駿等坐,奏天竺樂,事畢,駿等還館。又
 遣婆羅門就館送食,以草葉為盤,其大方丈。因謂駿曰:「今是大國臣,非復赤土國矣。」後數日,請駿等入宴,儀衛導從如初見之禮。王前設兩床,床上並設草葉盤,方一丈五尺,上有黃、白、紫、赤四色之餅,牛、羊、魚、鱉、豬、蝳蝐之肉百餘品。延駿升床,從者於地席,各以金鍾置酒,女樂迭奏,禮遺甚厚。



 尋遣那邪迦隨貢方物,并獻金芙蓉冠、龍腦香,以鑄金為多羅葉,隱起成文以為表,金函封之,令婆羅門以香花奏蠡鼓而送之。既入海,見綠魚群飛水上。浮海十餘日,至林邑東南,並山而行。其海水色黃氣腥,舟行一日不絕,云是大魚糞也。



 循海北岸,達于交
 趾。駿以六年春與那邪迦於弘農謁帝。帝大悅,授駿等執戟都尉,那邪迦等官賞各有差。



 真臘國,在林邑西南,本扶南之屬國也,去日南郡舟行六十日而至。南接車渠國,西有朱江國。其王姓剎利氏,名質多斯那。自其祖漸已強盛,至質多斯那隧兼扶南而有之。死,子伊奢那先代立。居伊奢那城,郭下二萬餘家。城中有一大堂,是其王聽政所。總大城三十所,城有數千家,各有部帥,官名與林邑同。



 其王三日一聽朝,坐五香七寶床,上施寶帳,以文木為竿,象牙金鈿為壁,狀如小屋,懸金光焰,有同于赤土。前有金香,命二人侍側。王
 著朝霞古貝,瞞絡腰腹,下垂至脛,頭載金寶花冠,被真珠纓絡,足履革屣,耳懸金鐺。常服白疊,以象牙為屩。若露髮,則不加纓絡。臣下服制,大抵相類。有五大臣,一曰孤落支,二曰相高憑,三曰婆何多陵,四曰舍摩陵,五曰髯羅婁,及諸小臣。朝於王者,輒於階下三稽首,王呼上階,則跪,以兩手抱膊,繞王環坐。議政事訖,跪伏而去。



 階庭門閣,侍衛有千餘人,被甲持仗。其國與參半、朱江二國和親,數與林邑、陀桓二國戰爭。其人行止,皆持甲仗,若有征伐,因而用之。



 其俗,非王正妻子,不得為嗣。王初立日,所有兄弟,並刑殘之,或去一指,或劓其鼻,別處供
 給,不得仕進。人形小而色黑,婦人亦有白者。悉拳髮垂耳,性氣捷勁。居處器物,頗類赤土。以右手為凈,左手為穢。每旦澡洗,以楊枝凈齒,讀誦經咒,又澡灑乃食。食罷還用楊枝凈齒,又讀經咒。飲食多蘇酪、沙糖、秔粟、米餅。欲食之時,先取雜肉羹與餅相和,手擩而食。娶妻者唯送女人女,擇日遣媒人迎婦。男女二家,各八日不出,晝夜燃燈不息。男婚禮畢,即與父母分財別居。



 父母死,小兒未婚者,以餘財與之。若婚畢,財物入官。喪葬,兒女皆七日不食,剔髮而喪,僧尼、道士、親故皆來聚會,音樂送之。以五香木燒尸,收灰,以金銀瓶盛,送大水之內;貧者
 或用瓦,而以五彩色畫之。亦有不焚,送屍山中,任野獸食者。



 其國北多山阜,南有水澤。地氣尤熱,無霜雪,饒瘴癘毒蜇。宜粱、稻,少黍、粟。果菜與日南、九真相類。異者,有婆羅那娑樹,無花,葉似柿,實似冬瓜;庵羅樹,花、葉似棗,實似李;毗野樹,花似木瓜,葉似杏,實似楮;婆田羅樹,花、葉、實並似棗而小異;歌畢佗樹,花似林檎,葉似榆而厚大,實似李,其大如升。



 自餘多同九真。海有魚名建同,四足無鱗,鼻如象,吸水上噴,高五六十尺。有浮胡魚,形似且,嘴如鸚鵡,有八足。多大魚,半身出,望之如山。每五六月中,毒氣流行,即以白豬、白牛、羊於城西門外祠之。不然,
 五穀不登,畜多死,人疾疫。近都有陵伽缽婆山,上有神祠,每以兵二千人守衛之。城東神名婆多利,祭用人肉。其王年別殺人,以夜祠禱,亦有守衛者千人。其敬鬼如此。多奉佛法,尤信道士。佛及道士,並立像於其館。



 隋大業十二年,遣使貢獻,帝禮之甚厚,於後亦絕。



 婆利國,自交趾浮海,南過赤土、丹丹,乃至其國。國界,東西四月行,南北四十五日行。王姓剎利邪伽,名護濫那婆。官曰獨訶邪拿,次曰獨訶氏拿。國人善投輪,其大如鏡,中有竅,外鋒如鋸,遠以投人,無不中。其餘兵器,與中國略同,俗類真臘,物產同於林邑。其殺人及盜,截其手;
 姦者,鎖其足,期年而止。祭祀必以月晦,盤貯酒肴,浮之流水。每十一月必設大祭。海出珊瑚。有鳥名舍利,解人語。



 隋大業十二年,遣使朝貢,後遂絕。



 于時南荒有丹丹、盤盤二國,亦來貢方物,其風俗、物產,大抵相類云。



 論曰:《禮》云:「南方曰蠻,有不火食者矣。」然其種類非一,與華人錯居,其流曰蜒,曰獽,曰俚,曰獠,曰厓。居無君長,隨山洞而居。其俗,斷髮文身,好相攻討。自秦并三楚,漢平百越,地窮丹徼,景極日南,水陸可居,咸為郡縣。



 洎乎境分南北,割據各殊,蠻、獠之族,遞為去就。至於林邑、赤土、真臘、婆利則地隔江嶺,莫通中國。及隋氏受命,剋平九
 宇,煬帝纂業,威加八荒,甘心遠矣,志求珍異。故師出流求,兵加林邑,威振殊俗,過於秦、漢遠矣。雖有荒外之功,無救域中之敗。《傳》曰:「非聖人,外寧必有內憂。」誠哉斯言也。



 大業中,南荒朝貢者十餘國,其事跡湮滅,今可知者四國而已。



\end{pinyinscope}