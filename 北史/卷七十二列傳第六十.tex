\article{卷七十二列傳第六十}

\begin{pinyinscope}

 高熲牛弘李德林高熲,字昭玄,一
 名敏,自言勃海蓚人也。其先因官北邊,沒於遼左。曾祖皓,以太和中自遼東歸魏,官至衛尉卿。祖孝安,位兗州刺史。父賓,仕東魏,位諫議大夫。大統六年,避讒棄官奔西魏,獨孤信引賓為僚佐,賜姓獨孤氏。及信誅,妻子徙蜀。隋文獻皇后以賓父之故吏,每往來其家。賓敏於從政,果敢斷決。賜爵武陽縣伯,歷位齊公憲府長史、驃騎大將軍、開府儀同三司、襄州總管府司錄,卒於州。及熲貴,開皇中,贈禮部尚書、武陽公,謚曰簡。熲少明敏,有器局,略涉文史,尤善詞令。初,孩孺時,家有柳樹,高百許尺,亭亭如蓋。里中父老曰:「此家當出貴人。」年十七,周齊王憲引為記室。襲爵武陽縣伯,再遷內史下大夫。以平齊功,拜開府。



 隋文帝得政,素知熲強明,久習兵事,多計略,意欲引之入府。遣邗公楊惠諭意,熲承旨忻然,曰:「願受驅馳。縱公事不成,亦不辭滅族。」於是
 為府司錄。



 時長史鄭譯、司馬劉昉並以奢縱被疏,帝彌屬意於熲,委以心膂。尉遲迥起兵也,帝令韋孝寬伐之,軍至河陽,莫敢先進。帝以諸將不一,令崔仲方監
 之,仲方辭以父在山東。時熲見劉昉、鄭譯等並無去意,遂自請行,深合上旨。受命便發,遣人辭母云,忠孝不可兩兼,歔欷就路。至軍,為橋于沁水,賊於上流縱火筏,熲預為土狗以御之。既度,焚橋而戰,大破之。軍還,侍宴於臥內,帝撤禦帷以賜之。進位柱國,改封義寧縣公,遷丞相府司馬,任寄益隆。及帝受禪,
 拜尚書左僕射、納
 言,
 進封勃海郡公。朝臣莫與為比,帝每呼為獨孤而不名也。熲佯避權勢,上表遜位,讓於蘇威。帝欲成其美,聽解僕射。數日,帝曰:「蘇威高蹈前朝,熲能舉善。



 吾聞進賢受上賞,寧可
 令去官!」於是令熲復位。俄拜左衛大將軍,本官如故。突厥屢為邊患,詔熲鎮遏緣邊。及還,賜馬百疋,牛羊千計。領新都大監,制度多出於熲。熲每坐朝堂北槐樹下以聽事,其樹不依行列,有司將伐之。帝特命勿去,以示後人。其見重如此。又拜左領軍大將軍。餘官如故。母憂去職,二旬,起令視事。



 熲流涕辭讓,不許。



 開皇二年,長孫覽、元景山等伐陳,令熲節度諸軍。會陳宣帝殂,熲以禮不伐喪,奏請班師。蕭巖之叛,詔熲綏集江漢,甚得人和。帝嘗問熲以取陳之策,熲曰:「江北地寒,田收差晚,江南土熱,水田早熟。量彼收獲之際,微徵士馬,聲言掩襲。賊必
 屯兵禦守,足得廢其農時。彼既聚兵,我更解甲,再三若此,賊以為常。



 後更集兵,彼必不信,猶豫之頃,我乃濟師,登陸而戰,兵氣益倍。又江南土薄,舍多竹茅,所有儲積,皆非地窖。密遣行人,因風縱火,待彼修立,而更燒之。不出數年,自可財力俱盡。」帝用其策,由是陳人益弊。



 九年,晉王廣大舉伐陳,以熲為無帥長史,三軍皆取斷於熲。及陳平,晉王欲納陳主寵姬張麗華。熲曰:「武王滅殷,戮妲己。今平陳國,不宜取麗華。」乃命斬之。王甚不悅。及軍還,以功加上柱國,進爵齊國公,賜物九千段,定食千乘縣千五百戶。帝勞之曰:「公伐陳後,人云公反,朕已斬之。
 君臣道合,非青蠅所間也。」熲又遜位,優詔不許。



 是後右衛將軍龐晃及將軍盧賁等,前後短熲於帝。帝怒,皆被疏黜。因謂熲曰:「獨孤公猶鏡也,每被磨瑩,皎然益明。」未幾,尚書都事姜曄、楚州行參軍李君才並奏稱水旱不調,罪由高熲,請廢黜之。二人俱得罪而去,親禮逾密。帝幸并州,留熲居守。及還,賜縑五千匹,行宮一所為莊舍。其夫人賀拔氏寢疾,中使顧問不絕。帝親幸其第,賜錢百萬,絹萬匹,復賜以千里馬。嘗從容命熲與賀若弼言及平陳事,熲曰:「賀若弼先獻十策,後於蔣山苦戰破賊。臣文吏耳,焉敢與猛將論功!」



 帝大笑,時論嘉其有讓。尋
 以其子表仁尚太子勇女,前後嘗賜,不可勝計。



 時熒惑入太微,犯左執法。術者劉暉私於熲曰:「天文不利宰相,可修德以禳之。」熲不自安,以暉言奏之。上厚加賞慰。突厥犯塞,以熲為元帥擊破之。又出白道,進圖入磧,遣使請兵,近臣言熲欲反,帝未有所答,熲亦破賊而還。



 時太子勇失愛,帝潛有廢立志,謂熲曰:「晉王妃有神告之,言王必有天下。」



 熲跪曰:「長幼有序,不可廢。」遂止。獨孤皇后知熲不可奪,陰欲去之。初,熲夫人卒,後言於帝曰:「高僕射老矣,而喪夫人,陛下何以不為之娶?」帝以后言告熲,熲流涕謝曰:「臣今已老,退朝唯齋居讀佛經而已。雖陛
 下垂哀之深,至於納室,非臣所原。」帝乃止。至是,熲愛妾產男,帝聞極歡,后甚不悅,曰:「陛下當復信熲邪?始陛下欲為熲娶,熲心存愛妾,面欺陛下,今其詐已見。」帝由是疏熲。



 會議伐遼東,熲固諫不可。帝不從,以熲為元帥長史,從漢王征遼東,遇霖潦疾疫,不利而還。后言於帝曰:「熲初不欲行,陛上強之,妾固知其無功矣。」又帝以漢王年少,專委軍於熲。熲以任寄隆重,每懷至公,無自疑意。諒所言多不用,因甚銜之。及還,諒泣言於后曰:「免熲殺,幸矣!」帝聞,彌不平。俄而上柱國王積以罪誅,當推覆之際,乃有禁中事,云於熲處得之。帝欲成熲罪,聞此大驚。



 時上柱國賀若弼、吳州總管宇文幹、刑部尚書薛胄、戶部尚書斛律孝卿、兵部尚書柳述等明熲無罪,帝愈怒,皆以之屬吏。自是朝臣莫敢言。熲竟坐免,以公就第。



 未幾,帝幸秦王俊第,召熲侍宴。熲歔欷悲不自勝,獨孤皇后亦對之泣,左右皆流涕。帝謂曰:「朕不負公,公自負朕也。」因謂侍臣曰:「我於高熲勝兒子,雖或不見,常似目前。自其解落,瞑然忘之,如本無高熲。不可以身要君,自云第一也。」頃之,熲國令上熲陰事,稱:「其子表仁謂熲曰:「昔司馬仲達初託疾不朝,遂有天下。公今遇此,安知非福?」於是帝大怒,囚熲於內史省而鞫之。憲司復奏熲他事,
 云:「沙門真覺嘗謂熲曰:『明年國有大喪。』尼令暉復云:『十七、八年,皇帝有大厄。十九年不可過。」帝聞益怒,顧謂群臣曰:「帝王豈可力求。



 孔丘以大聖之才,作法垂於後代,寧不欲大位邪?天命不可耳。熲與子言,自比晉帝,此何心乎?」有司請斬之,帝曰:「去年殺虞慶則,今茲斬王積,如更誅熲,天下謂我何!」於是除熲名。初,熲為僕射,其母誡之曰:「汝富貴已極,但有斫頭耳,爾其慎之!」熲由是常恐禍變。及此,熲歡然無恨色,以為得免禍。



 煬帝即位,拜太常卿。時有詔收周、齊故樂人及天下散樂。熲奏:「此樂久廢。



 今若征之,恐無識之徒棄本逐末,遞相教習。」帝不悅。
 帝時侈靡,聲色滋甚,又起長城之役。熲甚病之,謂太常丞李懿曰:「周天元以好樂而亡,殷監不進,安可復爾!」時帝遇啟人可汗恩禮過厚,熲謂太府卿何稠曰:「此虜頗知中國虛實、山川險易,恐為後患。」復謂觀王雄曰:「近來朝廷殊無綱紀。」有人奏之,帝以為訕謗朝政,誅之,諸子徙邊。



 熲有文武大略,明達政務。及蒙任寄之後,竭誠盡節,進引貞良,發天下為己任。蘇威、楊素、賀若弼、韓禽等皆熲所薦,各盡其用,為一代名臣。自餘立功立事者,不可勝數。當朝執政將二十年,朝野推服,物無異議,時致升平,熲之力也。



 論者以為真宰相。及誅,天下入不傷惜,
 至今稱冤不已。所有奇策良謀及損益時政,熲皆削稿,代無知者。



 子盛道,位莒州刺史,徙柳城卒。道弟弘德,封應國公,晉王記室;次弟表仁,勃海郡公。徒蜀郡。



 牛弘,字里仁,安定鶉觚人也。其先嘗避難,改姓遼氏。祖熾,本郡中正。父元,魏侍中、工部尚書、臨涇公,復姓牛氏。弘在襁褓,有相者見之,謂其父曰:「此兒當貴,善愛養之。」及長,鬚貌甚偉,性寬裕,好學博聞。仕周,歷位中外府記室、內史上士、納言上士,專掌文翰,修起居注。後襲封臨涇公,轉內史下大夫、儀同三司。開皇初,授散騎常侍、秘書監。弘以典籍遺逸,上表請開獻書之路,曰:昔周德既
 衰,舊經紊棄。孔子以大聖之才,開素王之業,憲章祖述,制《禮》刊《詩》,正五始而修《春秋》,闡《十翼》而弘《易》道。及秦皇馭宇,吞滅諸侯,先王墳籍,掃地皆盡。此則書之一厄也。漢興,建藏書之策,置校書之官。至孝成之代,遣謁者陳農求遺書於天下,詔劉向父子讎校篇籍。漢之典文,於斯為盛。



 及王莽之末,並從焚燼。此則書之二厄也。光武嗣興,尤重經誥,未及下車,先求文雅。至肅宗親臨講肄,和帝數幸書林,其蘭臺、石室、鴻都、東觀,秘牒填委,更倍於前。及孝獻移都,吏人擾亂,圖書縑帛,皆取為帷囊。所收而西,載七十餘乘,屬西京大亂,一時燔蕩。此則書之
 三厄也。魏文代漢,更集經典,皆藏在秘書,內外三閣,遣秘書郎鄭默刪定舊文,論者美其朱紫有別。晉氏承之,文籍尤廣。晉秘書監荀勖定魏《內經》,更著《新簿》。屬劉、石馮陵,從而失墜。此則書之四厄也。永嘉之後,寇竊競興,其建國立家,雖傳名號,憲章禮樂,寂滅無聞。劉裕平姚,收其圖籍,《五經》子史,纔四千卷,皆赤軸青紙,文字古拙,並歸江左。



 宋秘書丞王儉依劉氏《七略》,撰為《七志》。梁人阮孝緒亦為《七錄》。總其書數,三萬餘卷。及侯景度江,破滅梁室,秘省經籍,雖從兵火,其文德殿內書史,宛然猶存。蕭繹據有江陵,遣將破平侯景,收文德之書及公私
 典籍重本七萬餘卷,悉送荊州。及周師入郢,繹悉焚之於外城,所收十纔一二。此則書之五厄也。



 後魏爰自幽方,遷宅伊洛,日不暇給,經籍闕如。周氏創基關右,戎車未息。



 保定之始,書止八千,後加收集,方盈萬卷。高氏據有山東,初亦採訪,驗其本目,殘闕猶多。及東夏初平,獲其經史,四部重雜,三萬餘卷。所益舊書,五千而已。



 今御出單本,合一萬五千餘卷,部帙之間,仍有殘缺。比梁之舊目,止有其半。至於陰陽《河洛》之篇,醫方圖譜之說,彌復為少。



 臣以經書自仲尼迄今,數遭五厄,興集之期,屬膺聖代。今秘藏見書,亦足披覽,但一時載籍,須令大備。
 不可王府所無,私家乃有。若猥發明詔,兼開購賞,則異典必致,觀閣斯積。



 上納之,於是下詔,獻書一卷,賚縑一疋。一二年間,篇籍稍備。進爵奇章公。



 三年,拜禮部尚書,奉敕修撰《五禮》,勒成百卷,行於當代。弘請依古制,修立明堂,上議曰:竊謂明堂者,所以通神靈,感天地,出教化,崇有德。黃帝曰合宮,堯曰五府,舜曰總章,布政興教,由來尚矣。《周官考工記》曰:「夏后氏代室,堂修二七,廣四修一。」鄭玄注云:「修十四步,其廣益以四分修之一,則廣十七步半也。」



 「殷人重屋,堂修七尋,四阿重屋。」鄭云:「其修七尋,廣九尋也。」「周人明堂,度九尺之筵,南北七筵。五室,凡
 室二筵。」鄭玄云:「此三者,或舉宗廟,或舉王寢,或舉明堂,互言之明其制同也。」馬融、王肅、干寶所注,與鄭亦異,今不具出。漢司徒馬宮議云「夏后氏代室,室顯於堂,故命以室。殷人重屋,屋顯於堂,故命以屋。周人明堂,堂大於夏室,故命以堂。夏后氏益其堂之廣百四十四尺,周人明堂,以為兩序間大夏后氏七十二尺。」若據鄭玄之說,則夏室大於周堂,如依馬宮之言,則周堂大於夏室。後王轉文,周大為是。但宮之所言,未詳其義。



 此皆去聖久遠,《禮》文殘缺,先儒解說,家異人殊。鄭注《玉藻》亦云:「宗廟路寢,與明堂同制。」《王制》曰:「寢不踰廟,明大小是同」。今依鄭
 注,每室及堂,止有一丈八尺,四壁之外,四尺有餘。若以宗廟論之,袷享之日,周人旅酬六尸,並后褷為七,先公昭穆二尸,先王昭穆二尸,合十一尸,三十六主,及君北面行事於二丈之堂,愚不及此。若以正寢論之,便須朝宴。據《燕禮》:「諸侯宴則賓及卿大夫脫屨升坐。」是知天子宴,則三公九卿並升堂。《燕義》又云:「席小卿次上卿。」言皆侍席。止於二筵之間,豈得行禮?若以明堂論之,總享之時,五帝各於其室。設青帝之位,須於木室內少北西面。太昊從食,坐於其西,近南北面。祖宗配享者,又於青帝南,稍退西面。丈八之室,神位有三,加以簠簋豆籩,牛羊之
 俎,四海九州美物咸設,復須席上升歌,出樽反坫,揖讓升降,亦以隘矣。



 據茲而說,近是不然。案劉向別錄及馬宮、祭邕等所見,當時有《古文明堂禮》、《王居明堂禮》、《明堂圖》、《明堂大圖》、《明堂陰陽》、《太山通義》、《魏文侯孝經傳》等,並說古明堂事。其書皆亡,莫得而正。今《明堂月令》者,鄭玄云是呂不韋著,《春秋十二紀》之首章,禮家鈔合為記。祭邕、王肅云周公作,《周書》有《月令》第五十三,即此也。各有證明,文多不載。束皙以為夏時書。



 劉瓛云:「不韋鳩集儒者,尋于聖王月令之事而記之。不韋安能獨為此記?」今案不得全稱周書,亦不可即為秦典,其內雜有虞、夏、殷
 之法,皆聖王仁恕之政也。



 蔡邕具為章名,又論之曰:「明堂所以宗祀其祖,以配上帝也。」夏后氏曰代室,殷人曰重屋,周人曰明堂。東曰青陽,南曰明堂,西曰總章,北曰玄堂,內曰太室。



 聖人南面而聽,響明而治,人君之位莫不正焉。故雖有五名,而主以明堂也。制度之數,各有所依。方一百四十四尺,坤之策也,屋圓楣徑二百一十六尺,乾之策也。



 太廟明堂方六丈,通天屋徑九丈,陰陽九六之變,且圓蓋方覆,九六之道也。八闥以象卦,九室以象州,十二宮應日辰。三十六戶,七十二牖,以四戶八牖乘九宮之數也。戶皆外設而不閉,示天下以不藏也。通天
 屋高八十一尺,黃鐘九九之實也。



 二十八柱布四方,四方七宿之象也。堂高三尺,以應三統,四向五色,各象其行。



 水闊二十四丈,象二十四氣,於外,以象四海。王者之大禮也。」觀其模範天地,則象陰陽,必據古文,義不虛出。今若直取《考工》,不參《月令》,青陽總章之號不得而稱,九月享帝之禮不得而用。漢代二京所建,與此說悉同。



 建安之後,海內大亂,魏氏三方未平,無聞興造。晉則侍中裴頠議「直為一殿,以崇嚴父之祀,其餘雜碎,一皆除之。」宋、齊已還,咸率茲禮,前王盛事,於是不行。後魏代都所造,也自李沖,三三相重,合為九屋。簷不覆基,房間通街,
 穿鑿處多,迄無可取。及遷洛陽,更加營構,五九紛競,遂至不成。宗祀之事,於焉靡託。



 今皇猷遐闡,化覃海外,方建大禮,垂之無窮。弘等不以庸虛,謬當議限。今檢明堂必須五室者何?《尚書帝命驗》曰:「帝者承天立五府,赤曰文祖,黃曰神鬥,白曰顯紀,黑曰玄矩,蒼曰靈府。」鄭玄注曰:「五府與周明堂同矣。」且三代相沿,多有損益,至於五室,確然不變。夫室以祭天,天實有五,若立九室,四無所用。布政視朔,自依其辰。鄭司農云:「十二月分在青陽等左右之位」,不云居室。鄭玄亦云「每月於其時之堂而聽政焉。」《禮圖》畫個,皆在堂偏,是以須為五室。明堂必須上
 圓下方者何?《孝經援神契》曰:「明堂者,上圓下方,八窗四達,布政之宮。」《禮記盛德篇》曰:「明堂四戶八牖,上圓下方。」是以須為圓方。明堂必須重屋者何?案《考工記》,夏言「九階,四旁兩夾窗,門堂三之二,室三之一。」殷、周不言者,明一同夏制。殷言「四阿重屋,」周承其後不言屋,制亦盡同可知也。其「殷人重屋」之下,本無五室之文。鄭注云:「五室者,亦據夏以知之。」明周不云重屋,因殷則有,灼然可見。《禮記明堂位》曰:「太廟,天子明堂。」言魯為周公之故,得用天子禮樂,魯之太廟,與周之明堂同。又曰:「復廟重簷,刮楹達響,天子之廟飾。」鄭注:「復廟,重屋也。」據廟既重屋,明
 堂亦不疑矣。《春秋》文公十三年,太室屋壞,《五行志》曰:「前堂曰太廟,中央曰太室,屋其上重者也。」服虔亦云「太室,太廟之上屋也。」《周書·作洛篇》曰:「乃立太廟宗宮路寢明堂,咸有四阿反坫,重亢重廊。」孔晁注云:「重亢,累棟;重廊,累屋也。」依《黃圖》所載,漢之宗廟皆為重屋。此去古猶近,遺法尚存,是以須為重屋。明堂必須為闢雍者何?《禮記盛德篇》云:「明堂者,明諸侯尊卑也。外水曰闢雍。」《明堂陰陽錄》曰:「明堂之制,周圜行水,左旋以象天,內有太室,以象紫宮。」此則明堂有水之明文也。然馬宮、王肅以為明堂、辟雍、太學同處,蔡邕、盧植亦以為明堂、靈臺、辟雍、太學
 同實異名。邕云:「明堂者,取其宗祀之清貌,則謂之清廟,取其正室,則曰太室,取其堂,則曰明堂,取其四門之學,則曰太學,取其周水圜如璧,則曰辟雍,其實一也。」其言別者,《五經通義》曰:「靈臺以望氣,明堂以布政,辟雍以養老教學。」三者不同。



 袁準、鄭玄亦以為別。歷代所疑,豈能輒定?今據《郊祀志》云:「欲為明堂,未曉其制。濟南人公玉帶上黃帝時《明堂圖》,一殿無壁,蓋之以茅,水圜宮垣,天子從之。」以此而言,其來則久。漢中元二年,起明堂、闢雍、靈臺於洛陽,並別處。然明堂並有璧水,李尤明堂銘曰「流水洋洋」是也。以此須有辟雍。



 今造明堂,須以禮經為
 本。形制依於周法,度數取於《月令》,遺闕之處,參以餘書,庶使該詳沿革之理。其五室九階,上圓下方,四阿重屋,四旁兩門,依《考工記》、《孝經》說。堂方一百四十四尺,屋圓楣徑二百一十六尺,太室方六丈,通天屋徑九丈,八闥二十八柱,堂高三尺,四向五色,依《周書月令》論。殿垣方在內,水周如外,水內徑三百步,依《太山》、《盛德記》、《觀禮經》。仰觀俯察,皆有則象,足以盡誠上帝,祗配祖宗,弘風布教,作範於後矣。



 上以時事草創,未邊制作,竟寢不行。



 六年,除太常卿。九年,詔定雅樂,又作樂府歌詞,撰定圓丘五帝凱樂,并議樂事。弘上議云:謹案禮,五聲六律,十二
 管還相為宮。《周禮》奏黃鐘,歌大呂,奏太蔟,歌應鐘,皆旋相為宮之義。蔡邕《明堂月令章句》曰:「孟春月則太蔟為宮,姑洗為商,蕤賓為角,南呂為徵,應鐘為羽,大呂為變宮,夷則為變徵。他月放此。」故先王之作律呂也,所以辨天地四方陰陽之聲。揚子雲曰:「聲生於律,律生於辰。」



 故律呂配五行,通八風,歷十二辰,行十二月,循環轉運,義無停止。譬如立春木王火相,立夏火王土相,季夏餘分,土王金相,立秋金王水相,立冬水王木相。遞相為宮者,謂當其王月,名之為宮。今若十一月不以黃鐘為宮,十三月不以太蔟為宮,便是春木不王,夏土不相。豈不陰
 陽失度,天地不通哉?劉歆《鍾律書》云:「春宮秋律,百卉必凋;秋宮春律,萬物必榮;夏宮冬律,雨雹必降;冬宮夏律,雷必發聲。」以斯而論,誠為不易。且律十二,今直為黃鐘一均,唯用七律,以外五律竟復何施?恐失聖人制作本意。故須依《禮》作還相為宮之法。



 上曰:「不須作旋相為宮,且作黃鐘一均也。」弘又論六十律不可行:謹案《續漢書律曆志》:「元帝遣韋玄成問京房於樂府。房對:『受學故小黃令焦延壽。六十律相生之法:以上生下,皆三生二;以下生上,皆三生四。陽下生陰,陰上生陽,終於中呂,十二律畢矣。中呂上生執始,執始下生去滅,上下相生,終於
 南事,六十律畢矣。十二律之變至於六十,猶八卦之變至於六十四也。冬至之聲,以黃鐘為宮,太蔟為商,姑洗為角,林鐘為徵,南呂為羽,應鐘為變宮,蕤賓為變徵。此聲氣之元,五音之正也。故各統一日。其餘以次運行,當日者各自為宮,而商徵以類從焉。』房又曰:『竹聲不可以度調,故作準以定數。準之狀如瑟,長一丈而十三弦,隱間九尺,以應黃鐘之律九寸。中央一弦,下畫分寸,以為六十律清濁之節。』執始之類,皆房自造。房云受法於焦延壽,未知延壽所承也。至元和元年,待詔候鐘律般肜上言:『官無曉六十律以準調音者。故待詔嚴嵩,具以準法
 教其子宣,願召宣補學官,主調樂器。』太史丞弘試宣十二律,其二中,其四不中,其六不知何律,宣遂罷。自此律家莫能為準施絃。熹平六年,東觀召典律者太子舍人張光問準意。光等不知,歸閱舊藏,乃得其器,形制如房書,猶不能定其絃緩急,故史官能辯清濁者遂絕。其可以相傳者,唯大榷常數及候氣而已。」據此而論,房法漢世已不能行。沈約《宋志》曰:「詳案古典及今音家,六十律無施於樂。」



 《禮》云「十二管還相為宮」,不言六十。《封禪書》云「大帝使素女鼓五十絃瑟而悲,破為二十五絃。」假令六十律為樂得成,亦所不用,取大樂必易,大禮必簡之意也。



 又議曰:案《周官》云「大司樂掌成均之法。」鄭眾注云:「均,調也。樂師主調其音。」



 《三禮義宗》稱「《周官》奏黃鐘者,用黃鐘為調,歌大呂者,用大呂為調。奏者謂堂下四縣,歌者謂堂上所歌。但以一祭之間,皆用二調。」是知據宮稱調,其義一也。明六律六呂迭相為宮,各自為調。今見行之樂,用黃鐘之宮,乃以林鐘為調,與古典有違。案晉內書監荀勖依典記,以五聲十二律還相為宮之法,制十二笛。黃鐘之笛,正聲應黃鐘,下徵應林鐘,以姑洗為清角。大呂之笛,正聲應大呂,下徵應夷則。以外諸均,例皆如是。然今所用林鐘,是勖下徵之調。不取其正,先用其下,於
 理未通,故須改之。



 上甚善其議,詔弘與姚察、許善心、何妥、虞世基等正定新樂。是後議置明堂,詔弘條上故事,議其得失。上甚敬重之。



 時楊素恃才矜貴,賤侮朝臣,唯見弘未嘗不改容自肅。素將擊突厥,詣太常與弘言別。弘送素至中門而止,素謂曰:「大將出征,故來敘別,何相送之近也?」



 弘遂揖而退。素笑曰:「奇章公可謂其智可及,其愚不可及也。」亦不以屑懷。尋授大將軍,拜吏部尚書。



 時帝又令弘與楊素、蘇威、薛道衡、許善心、虞世基、崔子發等并召諸儒,論新禮降殺輕重。弘所立議,眾咸推服之。及獻皇后崩,王公已不下能定其儀注。楊素謂弘曰:「
 公舊學時賢所仰。今日之事,決在於公。」弘了不辭讓,斯須之間,儀注悉備,皆有故實。素嘆曰:「衣冠禮樂盡在此矣,非吾所及也!」弘以三年之喪。祥禫具有降殺,期服十一月而練者,無所象法,以聞於帝。帝下詔除期練之禮,自弘始也。



 弘在吏部,先德行後文才,務在審慎。雖致緩滯,所有進用,並多稱職。吏部侍郎高孝基,鑒賞機晤,清慎絕倫,然爽俊有餘,迹似輕薄,時宰多以此疑之。唯弘深識其真,推心任委。隋之選舉,於斯為最,時論服弘識度之遠。



 煬帝之在東宮,數有詩書遺弘,弘亦有答。及嗣位,嘗賜弘詩曰:「晉家山吏部,魏代盧尚書,莫言先哲異,
 奇才並佐餘。學行敦時俗,道素乃沖虛,納言雲閣上,禮儀皇運初。彞倫欣有敘,垂拱事端居。」其同被賜詩者,至於文詞贊揚,無如弘美。大業二年,進位上大將軍。三年,改右光祿大夫。從拜恒岳,壇墠珪幣牲牢,並弘所定。還下太行山,煬帝嘗召弘入內帳,對皇后賜以同席飲食。其親重如此。弘謂其子曰:「吾受非常之遇,荷恩深重。汝等子孫,宜以誠敬自立,以答恩遇之隆。」六年,從幸江都,卒。帝傷惜之,賵贈甚厚。歸葬安定,贈開府儀同三司、光祿大夫、文安侯,謚曰憲。



 弘榮寵當世,而車服卑儉,事上盡禮,待下以仁,訥於言而敏於行。上嘗令宣敕,弘至階
 下,不能言,退還拜謝,云並忘之。上曰:「傳語小辯,故非宰臣任也。」



 愈稱其質真。大業之代,委遇彌隆。性寬厚,篤志幹學,雖職務繁雜,書不釋手。



 隋室舊臣,始終信任,悔吝不及,唯弘一人而已。弟弼,好酒而酗,嘗醉射殺弘駕車牛,弘還宅,其妻迎謂曰:「叔射殺牛。」弘聞,無所怪問,直答曰:「作脯。」



 坐定,其妻又曰:「叔忽射殺牛,大是異事。弘曰:「已知。」顏色自若,讀書不輟。其寬和如此。有文集十二卷行於世。



 長子方大,亦有學業,位內史舍人。



 次子方裕,兇險無仁心,在江都與裴虔通等謀殺逆,事見《司馬德戡傳》。



 李德林,字公輔,博陵安平人。祖壽,魏湖州戶曹從事。父
 敬族,歷太學博士、鎮遠將軍。魏靜帝時,命當世通人正定文籍,以為內校書,別在直閣省。德林幼聰敏,年數歲,誦左思《蜀都賦》,十餘日便度。高隆之見而歎異之,遍告朝士云:「若假其年,必為天下偉器。」鄴京人士多就宅觀之,月餘車馬不絕。年十五,誦《五經》及古今文集,日數千言。俄而該博墳典,陰陽緯候無不通涉。善屬文,詞核而理暢。魏收嘗對高隆之謂其父曰:「賢子文筆,終當繼溫子昇。」隆之大笑曰:「魏常侍殊己嫉賢,何不近比老彭,乃遠求溫子!」



 年十六,遭父艱,自駕靈輿,反葬故里。時嚴寒,單縗跌足,州里人物由是敬慕之。居貧感軻,母氏多疾,
 方留心典籍,無復宦情。其後母病稍愈,逼令仕進。



 齊任城王湝為定州刺史,重其才,召入州館,朝夕同遊,殆均師友。後舉秀才,尚書令楊遵彥考為上第,授殿中將軍。及長廣王作相,引為丞相府行參軍。未幾,王即帝位,累遷中書舍人,加通直散騎侍郎,別典機密。尋丁母艱,以至孝聞,朝廷嘉之。裁百日,奪情起復,固辭不起。魏收與陽休之論《齊書》起元事,百司會議。



 收與德林致書往復,詞多不載。後除中書侍郎,仍詔修國史,時齊帝留情文雅,召入文林館,與黃門侍郎顏之推同判文林館事。累遷儀同三司。



 周武帝平齊,遣使就宅宣旨云:「平齊之利,
 唯在於爾,宜入相見。」仍令從駕至長安,授內史上士,詔誥格式及用山東人物,一以委之。周武謂群臣曰:「我常日唯聞李德林與齊朝作書檄,我正謂其是天上人。豈言今日得其驅使,復為我作文書,極為大異。」神武公紇豆陵毅答曰:「臣聞明主聖王,得騏驎鳳皇為瑞,是聖德所感,非力能致之。瑞物雖來,不堪使用。如李德林來受驅策,亦是陛下聖德感致,有大才用,勝於騏驎鳳皇遠矣。」帝大笑曰:「誠如公言。」宣政末,授御正下大夫。後賜爵成安縣男。



 宣帝大漸,隋文帝初受顧命,令邗國公楊惠謂德林曰:「朝廷賜令總文武事,今欲與公共成,必不得
 辭。」德林答曰:「願以死奉公。」隋文大悅,即召與語。



 劉昉、鄭譯初矯詔召隋文受命輔少主,總知內外兵馬事。譯欲授隋文塚宰,譯自攝大司馬,昉為小冢宰。德林私啟:「宜作大丞相,假黃鉞,都督內外諸軍事。」遂以譯為相府長史。昉為相府司馬,二人由是不平。以德林為相府屬,加儀同大將軍。



 未幾而三方構亂,指授兵略,皆與之參詳。軍書羽檄,朝夕頓至,一日之中,動逾百數。或機速競發,口授數人,文意百端,不加治點。鄖公韋孝寬為東道元帥,師次永橋,沁水長,孝寬師未得度。長史李詢密啟:「諸大將受尉遲迥餉金。」隋文得啟,以為憂,議欲代之。德林
 曰:「臨敵代將,自古所難,樂毅所以辭燕,馬服以之敗趙也。公但以一腹心,明於智略,素為諸將所信伏者,速至軍所,觀其情偽。縱有異意,必不敢動。」隋文曰:「公不發此言,幾敗大事!」即令高熲馳驛往軍所,為諸將節度,竟成大功。凡厥謀謨,皆此類也。進授丞相府從事內郎。禪代之際,其相國總百揆、九錫殊禮詔策箋表璽書,皆德林之辭也。隋文癸祚之日,授內史令。初,將受禪,虞慶則等勸隋文盡滅宇文氏,德林固爭以為不可。隋文怒,由是品位不加,唯依班例,授上儀同,進爵為子。



 開皇元年,敕令與太尉於翼、高熲等同修律令。訖,奏聞,別賜駿馬及
 九環金帶。五年,敕令撰錄作相時文翰,勒成五卷,謂之《霸朝雜集》。隋文省讀訖,明旦謂德林曰:「自古帝王之興,必有異人輔佐。我昨讀《霸朝集》,方知感應之理。



 昨宵恨夜長,不得早見公面。」於是追贈其父定州刺史、安平縣公,謚曰孝。隋文後幸鄴,德林以疾不從。敕書追之,後御筆注云:「伐陳事意,宜自隨也。」時高熲入京,上語熲曰:「德林若患未堪行,宜自至宅,取其方略。」帝以之付晉王廣。



 初,大象末,文帝以逆人王謙宅賜之,尋又改賜崔謙,帝令德林自選一好宅並莊店作替。德林乃奏取逆人高阿那衛國縣市店八十區為替。九年,車駕幸晉陽,店人
 表訴,稱地是平人物,高氏強奪,於內造舍。上責德林。德林請勘逆人文簿及本換宅之意,上不聽,悉追店給所住者。由是嫌之。初,德林稱其父為太尉諮議,以取贈官,李元操等陰奏之曰:「德林父終於校書,妄稱諮議。」上甚銜之。至是,復庭議忤意,因數之曰:「公為內史,典朕機密,比不預計議者,以公不弘耳。朕方以孝理天下,故立五教以弘之。公言孝由天性,何須設教。然則孔子孫當說《孝經》也?又罔冒取店,妄加父官,朕實忿之而未能發。今當以一州相遣耳。」因出為湖州刺史。在州逢旱,課人掘井溉田,為考司所貶。歲餘,卒官,時年六十一。



 贈大將軍、廉州
 刺史,謚曰文。將葬,敕令羽林百人,並鼓吹一部,以給喪事,祭以太牢。



 德林美容儀,善談吐,器量沈深,時人未能測。齊任城王湝、趙彥深、魏收、陸仰大相欽重。德林少孤,未有字,魏收謂之曰:「識度天才,必至公輔,吾輒以此字卿。」從宦已後,即典機密,性慎密,嘗言古人不言溫樹,何足稱也。少以才學見知,及位望稍高,頗傷自任,爭競之徒,更相譖毀。所以運屬興王,功參佐命,十餘年間竟不徙級。所撰文集,勒成八十卷,遭亂亡失,見五十卷行於代。



 子百藥,博涉多才,詞藻清贍。大業末,位建安郡丞。



\end{pinyinscope}