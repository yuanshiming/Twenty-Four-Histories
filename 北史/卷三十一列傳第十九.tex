\article{卷三十一列傳第十九}

\begin{pinyinscope}

 高允從
 祖弟祐祐曾孫德正祐從子乾昂季式高允,字伯恭,勃海蓚人,漢太傅裒之後也。曾祖慶,慕容垂司空。祖父泰,吏部尚書。父韜,少以英朗知名,同郡封懿雅相推敬。亦仕慕容垂,為太尉從事中郎。道武平中山,以為丞相參軍,早卒。



 允少孤夙成,有奇度,清河崔宏見而異之,歎曰:「高子黃中內潤,文明外照,必為一代偉器,但吾恐不見耳。」年十餘歲,祖父泰喪,還本郡。允推財
 與二弟而為沙門,名法凈,未久而罷。性好文學,擔笈負書,千里就業。博通經史、天文、術數,尤好《春秋公羊》。曾作《塞上公詩》,有混欣戚、遺得喪之致。



 神蒨三年,太武舅陽平王杜超行征南大將軍,鎮鄴,以允為從事中郎,年四十餘矣。超以方春而諸州囚不決,表允與中郎呂熙等分詣諸州,共評獄事。熙等皆以貪穢得罪,唯允以清平獲賞。府解,還家教授,受業者千餘人。



 四年,與盧玄等俱被徵,拜中書博士,遷侍郎。與太原張偉並以本官領衛大將軍樂安王範徒事中郎。範,太武寵弟,西鎮長安,允甚有匡益,秦人稱之。尋被徵還。樂平王丕西討上邽,復
 以本官參丕軍事。以謀平涼州之勛,賜爵汶陽子。後奉詔領著作郎,與司徒崔浩述成國記。



 時浩集諸術士,考校漢元以來,日月薄蝕,五星行度,并譏前史之失,別為魏歷以示允。允曰:「善言遠者,必先驗於近。且漢元年冬十月,五星聚於東井,此乃歷術之淺事。今譏漢史而不覺此謬,恐後之譏今,猶今之譏古。」浩曰:「所謬云何?」允曰:「案《星傳》,金、水二星,常附日而行,冬十月,日旦在尾、箕,昏沒於申南,而東井方出於寅北,二星何因背日而行?是史官欲神其事,不復推之於理。」浩曰:「欲為變者,何所不可?君獨不疑三星之聚,而怪二星之來。」允曰:「此不可以
 空言爭,宜更審之。」時坐者咸怪,唯東宮少傅游雅曰:「高君長於歷,當不虛言也。」後歲餘,浩謂允曰:「先所論者,本不經心,及更考究,果如君語。以前三月聚於東井,非十月也。」又謂雅曰:「高允之術,陽源之射也。」



 眾乃歎服。允雖明於歷數,初不推步有所論說。惟游雅數以災異問允。允曰:「昔人有言,知之甚難。既知,復恐漏泄,不如不知也。天下妙理至多,何遽問此。」



 雅乃止。尋以本官為秦王翰傅。後敕以經授景穆,甚見禮待。又詔允與侍郎公孫質、李靈、胡方回共定律令。



 太武引允與論刑政,言甚稱旨。因問允「萬機何者為先」。時多禁封良田,又京師遊食眾。
 允因曰:「臣少也賤,所知唯田,請言農事。古人云:方一里則為田三頃七十畝,方百里則田三萬七千頃。若勸之,則畝益三升;不勸,則畝損三升。



 方百里損益之率,為粟二百二十二萬斛,況以天下之廣乎?若公私有儲,雖遇饑年,復何憂乎?」帝善之,遂除田禁,悉以授百姓。



 初,崔浩薦冀、定、相、幽、并五州士數十人,各起家為郡守。景穆謂誥曰:「先召之人,亦州郡選也,在職已久,勤勞未答。今可先補前召,外任郡縣;以新召者代為郎吏。又守令宰人,宜使更事者。」浩固爭而遣之。允聞之,謂東宮博士管恬曰:「崔其不免乎!茍逞其非而校勝於上,何以能濟?遼東
 公翟黑子有寵於太武,奉使并州,受布千疋。事發,黑子問允:「主上問我,首乎?諱乎?」允曰:「公幃幄寵臣,答詔宜實。」中書侍郎崔鑒、公孫質等咸言宜諱之。黑子以鑒等為親己,怒而絕允,而不以實對,終獲罪戮。



 時著作令史閔湛、卻性巧佞,為崔浩信待。見浩所注《詩》、《書》、《論語》及《易》,遂上疏言馬、鄭、王、賈不如浩之精微,請收藏境內諸書,班浩所注。并求敕浩注《禮》、《傳》。浩亦表薦湛有著述才。湛等又勸浩刊所撰國史於石,以彰直筆。允聞之,謂著作郎宗欽曰:「閔湛所營分寸之間,恐為崔門萬世之禍,吾徒無類矣。」未幾而難作。



 初,浩之被收,允直中書省。景穆
 使召允,留宿宮內。翌日,命驂乘至宮門,謂曰:「入當見至尊,吾自導卿,脫至尊有問,但依吾說。」既入見,景穆言允小心慎密,且微賤,制由於浩,請赦之。帝召允謂曰:「國書皆浩作不?」允曰:「《太祖記》,前著作郎鄧彥海所撰;《先帝記》及《今記》,臣與浩同作,然而臣多於浩。」帝大怒曰:「此甚於浩,安有生路?」景穆曰:「天威嚴重,允迷亂失次耳。臣向問,皆云浩作。」帝問:「如東宮言不?」允曰:「臣罪應滅族,不敢虛妄。殿下以臣侍講日久,哀臣乞命耳。實不問臣,不敢迷亂。」帝謂景穆曰:「直哉!此亦人情所難,而能臨死不移。且對君以實,貞臣也,寧失一有罪,宜宥之。」允竟得免。于是召
 浩前,使人詰,惶惑不能對。允事事申明,皆有條理。時帝怒甚,敕允為詔,自浩以下,僮吏以上,一百二十八人皆夷五族。允持疑不為,頻詔催切,允乞更一見,然後為詔。詔引前,允曰:「浩之所坐,若更有餘釁,非臣敢知。直以犯觸,罪不至死。」帝怒,命介士執允。景穆拜請,帝曰:「無此人忿朕,當有數千口死矣!」浩竟族滅,餘皆身死。宗欽臨刑歎曰:「高允其殆聖乎!」



 景穆後讓允,以不同己所導之言而令帝怒。允曰:「夫史籍,帝王之實錄,將來之炯誡,今之所以觀往,後之所以知今。是以言行舉動,莫不備載,故人君慎焉。



 然浩世受殊遇,榮曜當時,私欲沒其公廉,愛
 憎蔽其直理,此浩之責也。至於書朝廷起動之跡,言國家得失之事,此為史之本體,未為多違。然臣與浩實同其事,死生義無獨殊。誠荷殿下再造之慈,違心茍免,非臣之意。」景穆動容稱歎。允後與人言曰:「我不奉東宮導旨者,恐負翟黑子也。」



 景穆季年,頗親近左右,營立田園,以收其利。允諫曰:「殿下,國之儲貳,四海屬心,言行舉動,萬方所則。而營立私田,畜養雞犬,乃至販酤市厘,與人爭利,議聲流布,不可追掩。夫天下者,殿下之天下,富有四海,何求而不獲何欲而弗從?而與販夫販婦競此尺寸?願殿下少察過言,斥出佞邪,所在田園,分給貧下。



 如
 此,則休聲日至,謗議可除。」景穆不納。景穆之崩也,允久不進見,後見,升階歔欷,悲不能止。帝流淚,命允使出。左右莫知其故,相謂曰:「允無何悲泣,令至尊哀傷,何也?」帝聞之,召而謂曰:「汝不知高允悲乎?崔浩誅時,允亦應死。東宮苦請,是以得免。今無東宮,允見朕悲耳。」先是,敕允集天文災異,使事類相從,約而可觀。允依《洪範傳》、《天文志》,撮其事要,略其文辭,凡為八篇。帝覽而善之,曰:「高允之明災異,亦豈減崔浩乎?」及文成即位,允頗有謀焉,司徒陸麗等皆受重賞,允既不蒙褒異,又終身不言。其忠而不伐,皆此類也。



 給事中郭善明,性多機巧,欲逞其能,
 勸文成大起宮室。允諫曰:「臣聞太祖道武皇帝既定天下,始建都邑。其所營立,必因農隙。今建國已久,宮室已備,永安前殿,足以朝會萬國;西堂溫室,足以安御聖躬;紫樓臨望,可以周視遠近。若廣修壯麗為異觀者,宜漸致之,不可倉卒。計斫材軍士及諸雜役須二萬,丁夫充作,老小供餉,合四萬人,半年可訖。古人有言:『一夫不耕,或受其飢,一婦不織,或受其寒。』況數萬之眾,其所損費,亦已多矣!」帝納之。



 允以文成纂承平之業,而風俗仍舊,婚娶喪葬,不依古式,乃諫曰:前朝之世,屢發明詔,禁諸婚娶,不得作樂。及葬送之日,歌謠鼓舞,殺牲燒葬,一切
 禁絕。雖條旨久班,而不革變,將由居上者未能悛改,為下者習以成俗,教化陵遲,一至於此。《詩》云『爾之教矣,人胥效矣。』人君舉動,不可不慎。



 《禮》云:嫁女之家,三日不息火;娶妻之家,三日不舉樂。今諸王納室,皆樂部給伎以為嬉戲,而獨禁細人不得作樂,此一異也。



 古之婚者,皆采德義之門,妙簡貞閑之女,先之以媒娉,繼之以禮物,集僚友以重其別,親御輪以崇其敬。今諸王十五便賜妻別居。然所配者,或長少差舛,或罪入掖庭,而以作合宗王,妃嬪籓懿,失禮之甚,無復此過。今皇子娶妻,多出宮掖,令天下小人,必依禮限,此二異也。



 凡萬物之生,靡
 不有死,然葬者藏也,死者不可再見,故深藏之。昔堯葬穀林,農不易畝;舜葬蒼梧,市不改肆。秦始皇作為地市,下錮三泉,死不旋踵,尸焚墓掘。由此推之,堯舜之儉,始皇之奢,是非可見。今國家營葬,費損巨億,一旦焚之,以為灰燼。上為之而不輟,而禁下人之必止,此三異也。



 古者,祭必立尸,序其昭穆;使亡者有馮,致食饗之禮。今已葬之魂,人直求貌類者,事之如父母,宴好如夫妻,損敗風化,黷亂情禮,莫此之甚。上未禁之,下不改絕,此四異也。



 夫大饗者,所以定禮儀,訓萬國,故聖王重之。至乃爵盈而不飲,肴乾而不食,樂非雅聲則不奏,物非正色則不
 列。今之大會,內外相混,酒醉喧嘵,罔有儀式,又俳優鄙褻,汙辱視聽。朝廷積習以為美,而責風俗之清純,此五異也。



 今陛下當百王之末,踵晉亂之弊,而不矯然釐改,以厲頹俗,臣恐天下蒼生,永不聞見禮教矣。



 允如此非一,帝從容聽之。或有觸迕,帝所不忍聞者,命左右扶出。事有不便,允輒求見,帝知允意,逆屏左右以待之。禮敬甚重,晨入暮出,或積日居中,朝臣莫知所論。或有上事陳得失者,帝省而謂群臣曰:「君父一也,父有是非,子何為不作書於人中諫之,使人知惡,而於家內隱處也?豈不以父親,恐惡彰於外也。今國家善惡,不能面陳,而上
 表顯諫,以此,豈不彰君之短,明己之美。至如高允者,真忠臣矣。朕有是非,恒正言而論,至朕所不忍聞者,皆侃侃論說,無所避就。朕聞其過,而天下不知其諫,豈不忠乎。汝等在左右,不曾聞一正言,但伺朕喜以求官。汝等以弓刀侍朕,待立勞耳,皆至公、王,此人執筆匡我,不過著作郎。汝等不亦愧乎!」於是拜允中書令,著作如故。司徒陸麗曰:「高允雖蒙寵待,而家貧布衣,妻子不立。」帝怒曰:「何不先言?今見朕用之,方言其貧!」是日,幸允第,唯草屋數間,布被溫袍,廚中鹽菜而已。帝歎息曰:「古人之清貧,豈有此乎!」



 即賜帛五百疋,粟千斛,拜長子忱為長樂
 太守。允頻表固讓,帝不許。



 初與允同征游雅等,多至通官,封侯,及允部下吏百數十人,亦至刺史、二千石;而允為郎二十七年不徙官。時百官無祿,允恆使諸子樵採自給。初,尚書竇瑾坐事誅,瑾子遵亡在山澤,遵母焦沒入縣官。後焦以老得免,瑾之親故,莫有恤者。



 允愍焦年老,保護在家,積六年,遵始蒙赦。其篤行如此。轉太常卿,本官如故。



 允上《代都賦》,因以夫諷,亦《二京》之流也。時中書博士索敞與侍郎傅默、梁祚論名字貴賤,著議紛紜。允遂著《名字論》以釋其惑,甚有典證。復以本官領秘書監,解太常卿,進爵梁城侯。



 初,允與游雅及太原張偉同
 業相友。雅嘗論允曰:「夫喜怒者,有生所不能無也。而前史載卓公寬中,文饒洪量,褊心者或之弗信。餘與高子游處四十餘年,未見是非慍喜之色,不亦信哉。高子內文明而外柔弱,其言吶吶不能出口,餘常呼為『文子』。崔公謂餘云:『高生豐才博學,一代佳士,所乏者矯矯風節耳。』余亦然之。司徒之譴,起於纖微,及於詔責,崔公聲嘶股戰,不能一言。宗欽以下,伏地流汗,都無人色。高子敷陳事理,申釋是非,辭義清辯,音韻高亮。明主為之動容,聽者無不稱善。仁及寮友,保茲元吉,向之所謂矯矯者,更在斯乎!宗愛之任勢也,威振四海,嘗召百司於都坐,
 王公以下,望庭畢拜,高子獨升階長揖。由此觀之,汲長孺可臥見衛青,何抗禮之有!向之所謂風節者,得不謂此乎!知人故不易,人亦不易知。吾既失之於心內,崔亦漏之於形外。鐘期止聽於伯牙,夷吾見明於鮑叔,良有以也。」其為人物所推如此。



 文成重允,常不名之,恆呼為「令公」。令公之號,播於四遠矣。



 文成崩,獻文居諒闇,乙弗渾專擅朝命,謀危社稷。文明太后誅之,引允禁中,參決大政。又詔允曰:「朕稽之舊典,欲置學官於郡國。卿儒宗元老,宜與中祕二省,參議以聞。」允表:請制大郡立博士二人、助教四人、學生一百人;次郡立博士二人、助教二
 人、學生八十人;中郡立博士一人、助教二人、學生六十人;下郡立博士一人、助教一人、學生四十人。其博士取博關經典,履行忠清,堪為人師者,年限四十以上。助教亦與博士同,年限三十以上。若道業夙成,才任教授,不拘年齒。學生取郡中清望,人行脩謹,堪循名教者,先盡高門,次及中等。帝從之,郡國立學,自此始也。



 後允以老疾,頻上表乞骸骨。詔不許。於是乃著《告老詩》。又以昔歲同征,零落將盡,感逝懷人,作《征士頌》。蓋止於應命,其有命而不至,則闕焉。



 其著《頌》者:中書侍郎、固安侯范陽盧玄子真,郡功曹史博陵崔綽茂祖,河內太守、下樂侯
 廣寧燕崇玄略,上黨太守、高邑侯廣寧常陟公山,征南大將軍從事中郎勃海高毗子翼,征南大將軍從事中郎勃海李金道賜,河西太守、饒陽子博陵許堪祖根,中書郎、新豐侯京兆杜銓士衡,征西大將軍從事中郎京兆韋閬友規,京兆太守趙郡李詵令孫,太常博士、鉅鹿公趙郡李靈武符,中書郎中、即丘子趙郡李遐仲熙,營州刺史、建安公太原張偉仲業,輔國大將軍從事中郎范陽祖邁,征東大將軍從事中郎范陽祖侃士倫,東郡太守、蒲陰子中山劉策,濮陽太守、真定子常山許琛,行司隸校尉、中都侯西河宋宣道茂,中書郎燕郡劉遐彥
 鑒,中書郎、武恆子河間邢穎宗敬,滄水太守、浮陽侯勃海高濟叔仁,太平太守、原平子鴈門李熙士元,秘書監、梁郡公廣平游雅伯度,廷尉正、安平子博陵崔建興祖,廣平太守、列人侯西河宋愔,州主簿長樂潘符,郡功曹長樂杜熙,征東大將軍從事中郎中山張綱,中書郎上谷張誕叔術,秘書郎鴈門王道雅,祕書郎鴈門閔弼,衛大將軍從事中郎中山郎苗,大司馬從事中郎上谷侯辯,陳郡太守、高邑子趙郡呂季才,合三十四人。



 其詞曰:紫氣干天,群雄亂夏,王龔徂征,戎車屢駕。掃盪遊氛,克剪妖霸,四海從風,八垠漸化。政教無外,即寧且壹,偃武
 櫜兵,唯文是恤。帝乃虛求,搜賢採逸,巖隱投竿,異人並出。



 亹亹盧生,量遠思純,鑽道據德,遊藝依仁;旌弓既招,釋褐投巾,攝齋升堂,嘉謀日陳;自東徂南,躍馬馳輸,僭馮影附,劉以和親。茂祖煢單,夙離不造,克己勉躬,聿隆家道;敦心《六經》,遊思文藻,終辭寵命,以之自保。燕、常篤信,百行靡遺,仕不茍進,任理栖遲;居沖守約,好讓善推,思賢樂古,如渴如饑。子翼致遠,道賜悟深,相期以義,和若瑟琴;並參幕府,俱發德音,優遊卒歲,聊以寄心。祖根運會,克光厥猷,仰緣朝恩,俯因德友;功雖後建,爵實先受,班同舊臣,位並群后。士衡孤立,內省靡疚,言不崇華,
 交不遺舊;以產則貧,論道則富,所謂伊人,實邦之秀。卓矣友規,稟茲淑量,存彼大方,擯此細讓;神與理冥,形隨流浪,雖屈王侯,莫廢其尚。趙實名區,世多奇士,山岳所鍾,挺生三李;矯矯清風,抑抑容止,初九而潛,望雲而起。詵尹西都,靈惟作傅,載訓皇宮,載理雲霧;熙雖中夭,迹階郎署,餘塵可挹,終亦顯著。仲業深長,雅性清到,憲章古式,綢繆典誥;時逢險艱,常一其操,納眾以仁,訓下以孝;化洽龍川,人歸其教。邁則英賢,侃亦稱選,聞達邦家,名行素顯;志在兼濟,豈伊獨善,繩匠弗顧,功不獲展。劉、許履忠,竭力致躬,出則騁說,入獻其功;輶軒一舉,橈燕
 下崇,名彰魏世,享業亦隆。道茂夙成,弱冠播名,與朋以信,行物以誠;怡怡昆弟,穆穆家庭,發響九皋,翰飛紫冥。頻煩省闥,亦司于京,刑以之中,政以之平。猗歟彥鑒,思參文雅,率性任真,器成非假;靡矜於高,莫恥于下,乃謝朱門,歸迹林野。宗敬延譽,號為四俊,華藻雲飛,金聲夙振;中遇沈痾,賦詩以訊,忠顯于辭,理出於韻。高滄朗達,默識該通,領新悟異,發自心胸;質侔和璧,文照雕龍,燿姿天邑,衣錦舊邦。士元先覺,介焉不惑,振袂來庭,始賓王國;蹈方履正,好是繩墨,淑人君子,其儀不忒。孔稱游、夏,漢美卿、雲,越哉伯度,出類踰群;司言祕閣,作牧河、汾,
 移風易俗,理亂解紛。融彼滯義,渙此潛文,儒道以析,九流以分。



 崔、宋二賢,誕性英偉,擢穎閭閻,聞名象魏;謇謇儀形,邈邈風氣,達而不矜,素而能貴。潘符標尚,杜熙好和,清不潔流,渾不同波;絕悕龍津,止分常科,幽而逾顯,損而逾多。張綱柔謙,叔術正直,道雅洽聞,弼為兼識;拔萃衡門,俱漸鴻翼,發憤忘食,豈要斗食。率禮從仁,罔愆於式,失不繫心,得不形色。郎苗始舉,用均已試,智是周身,言足為志;性協於時,情敏於事,與今而同,與古而異。物以利移,人以酒昏,侯生潔己,唯義是敦;日縱醇醪,逾敬逾溫,其在私室,如涉公門。季才之性,柔而執競,屆彼
 南秦,申威致命;誘之以權,矯之以正,帝道用光,邊王內慶。群賢遭世,顯名有代。志竭其忠,才盡其概。體襲朱裳,腰紉雙佩,榮曜當時,風高千載;君臣相遇,理實難階。昔因朝命,與之克諧,披衿散想,解帶舒懷。此昕猶昨,存亡奄乖,靜言思之,衷心九摧。揮毫頌德,潛爾增哀。



 皇興中,詔允兼太常至兗州祭孔子廟。謂允曰:「此簡德而行,勿有辭也。」



 後允從獻文北伐,大捷而還,至武川鎮,上《北伐頌》,帝覽而善之。帝時有不豫,以孝文沖幼,欲立京兆王子推,集諸大臣,以次召問。允進跪上前,涕泣曰:「臣不敢多言以勞神聽。願陛下上思宗廟託附之重,追念周公
 抱成王之事。」帝於是傳位於孝文,賜允帛百疋,以標忠亮。又遷中書監,加散騎常侍。雖久典史事,然不能專勤屬述。時與校書郎劉模有所緝綴,大較依續崔浩故事,準《春秋》之體而時有刊正。自文成迄于獻文,軍國書檄,多允作也。末乃薦高閭以自代。以定議之勳,進爵咸陽公。尋授懷州刺史。



 允秋月巡境,問人疾苦。至邵縣,見邵公廟廢毀不立,乃歎曰:「邵公之德,闕而不祀,為善者何望!」乃表修葺之。允於時年將九十矣,勸人學業,風化頗行。



 然儒者優遊,不以斷決為事。後正光中,中書舍人河內常景追思允,率郡中故老,為允立祠於野王之南,樹
 碑紀德焉。



 太和二年,又以老乞還鄉,章十餘上,卒不聽許,遂以疾告歸。其年,詔以安車徵允,敕州郡發遣。至都,復拜鎮軍大將軍,領中祕書事。固辭,不許。扶引就內,改定皇誥;又被敕,論集往世酒之敗德,以為《酒訓》。孝文覽而悅之,常置左右,詔允乘車上殿,朝賀不拜。明年,詔允議定律令。雖年漸期頤,而志識無損,猶心存舊職,披考史書。又詔曰:「允年涉危境,而家貧養薄,可令樂部絲竹十人,五日一詣允,以娛其志。」特賜允蜀牛一頭、四望蜀車一乘、素几杖各一、蜀刀一口。又賜珍味,每春秋致之。尋詔朝晡給御膳,朔望致牛酒,衣服綿絹,每月送給。



 允
 皆分之親故。是時貴臣之門,並羅列顯官,而允子弟,皆無官爵,其廉退若此。



 遷尚書、散騎常侍。時延入,備几杖,詢以政事。



 十年,加光祿大夫,金章紫綬。朝之大議,皆諮訪焉。其年四月,有事西郊,詔御馬車迎允就郊所板殿觀矚。馬忽驚奔,車覆,傷眉三處。孝文、文明太后遣醫藥護療,存問相望。司駕將處重坐,允啟陳無恙,乞免其罪。先是,命中黃門蘇興壽扶侍允,曾雪中遇犬驚倒,扶者大懼,允慰勉之,不令聞徹。興壽稱共允接事三年,不嘗見其忿色。恂恂善誘,誨人不倦,晝夜手常執書,吟詠尋覽。篤親念故,虛己存納,雖處貴重,志同貧素。性好音樂,
 每至伶人弦歌鼓舞,常擊節稱善。又雅信佛道,時設齋講,好生惡殺。



 魏初法嚴,朝士多見杖罰。允歷事五帝,出入三省五十餘年,初無譴咎。始真君中,以獄訟留滯,始令中書以經義斷諸疑事。允據律評刑,三十餘載,內外稱平。



 允以獄者人命所係,常歎曰:「皋陶至德也,其後英、蓼先亡;劉、項之際,英布黥而王。經世雖久,猶有刑之餘釁。況凡人能無咎乎?」性簡至,不妄交遊。獻文之平青、齊,徙其族望於代。時諸士人,流移遠至,率皆飢寒。徙人之中,多允姻媾,皆徒步造門。允散財竭產,以相贍振,慰問周至,無不感其仁厚。又隨其才能,表奏申用。時議者皆
 以新附致異,允謂取材任能,無宜抑屈。



 先是,允被召在方山作頌,志氣猶不多損,談說舊事,了無所遺。十一年正月卒,年九十八。初,允每謂人曰:「吾在中書時有陰德,濟救人命,若陽報不差,吾壽應享百年矣。」先卒旬外,微有不適,猶不寢臥,呼醫請藥,出入行止,吟詠如常。孝文、文明太后聞而遣醫李修往脈視之,告以無恙。修入,密陳允榮衛有異,懼其不久。於是遣使備賜御膳珍羞,自酒米至于鹽醢,百有餘品,皆盡時味。及床帳衣服,茵被几杖,羅列於庭。王官往還,慰問相屬。允喜形於色,語人曰:「天恩以我篤老,大有所齎,得以贍客矣。」表謝而已,不
 有他慮。如是數日,夜中卒,家人莫覺。詔給絹一千疋、布二千疋、綿五百斤、錦五十疋、雜綵百疋、穀千斛,以周喪用。魏初以來,存亡蒙齎者莫及,朝廷榮之。將葬,贈侍中、司空公、冀州刺史,將軍、公如故。謚曰文,賜命服一襲。



 允所製詩賦詠頌箴論表贊誄、《左氏釋》、《公羊釋》、《毛詩拾遺》、《雜解》、《議何鄭膏肓事》凡百餘篇,別有集,行於世。允尤明算法,為《算術》三卷。



 子忱,字士和,位長安太守,為政寬惠,百姓安之。後例降爵為侯,卒,子貴賓襲。忱弟懷,字士仁,恬淡退靜,位太尉、東陽王丕諮議參軍。



 子綽,字僧裕。少孤,恭敏自立。身長八尺,腰帶十圍。沈雅有度量,博涉經
 史。稍遷洛陽令,為政強直,不避豪右,京邑憚之。延昌初,尚書右丞。後為御史中尉元匡奏高聰及綽朋附高肇,詔並原罪。歷豫、并二州刺史,卒,謚文簡。



 允弟推,字仲讓,早有名譽。太延中,以前後南使不稱,妙簡行人,游雅薦推應選。詔兼散騎常侍使宋,南人稱其才辯。卒於建業,贈臨邑子,謚曰恭。



 推弟燮,字季和,亦有文才。太武每詔征,辭疾不應,恒笑允屈折久官,栖泊京邑,常從容於家。州辟主簿,卒。孫市賓,永熙中,開府從事中郎。



 始神蒨中,允與從叔濟、族兄毗及同郡李金俱被徵。濟位滄水太守、浮陽子。



 卒,贈冀州刺史,謚曰宣。子矯襲。



 矯弟遵,字世
 禮。賤出,其兄矯等常欺侮之,及父亡,不令在喪位。遵遂馳赴平城,歸允。允為作計,乃為遵父舉哀,以遵為喪主,京邑無不弔集,朝貴咸識之。



 徐歸奔止。免喪後,為營宦路。遵感成益之恩,事允如諸父。涉歷文史,頗有筆札。



 隨都將長廣公侯窮奇等平定三齊。以功賜爵高昌男,補安定王相。撰太和、安昌二殿畫圖。後與中書令高閭增改律令,進中書侍郎。假中書令,詣長安,刊燕宣王廟碑,進爵安昌子。使濟、兗、徐三州,觀風理訟。進中都令。及新制衣冠,孝文恭薦宗廟,遵形貌莊潔,音氣雄暢,常兼太祝令;跪贊禮事,為俯仰之節,粗合儀矩,由是帝頗識待
 之。後與游明根、高閭、李沖等入議律令,親對御坐,時有陳奏。出為齊州刺史。建節歷本州,宗鄉改觀,而矯等彌妒毀之。



 遵性不廉清。在中書時,每假歸山東,必借備騾馬,將從百餘,屯逼人家,不得絲縑滿意,則詬詈不去。旬月之間,縑布千數,郡邑苦之。既蒞方岳,本意未弭,選召僚吏,多所取紅納。又其妻明氏,家在齊州,母弟舅甥,共相憑屬,爭取貨利。



 嚴暴,非理殺害甚多。貪酷之響,帝頗聞之。及車駕幸鄴,遵自州來朝。會有赦宥,遵臨還州,請辭。帝於行宮引見誚讓之。遵自陳無負。帝厲聲曰:「若無遷都赦,必無高遵矣!又卿非唯貪婪,又虐於刑法」。謂:「何如
 濟陰王,猶不免於法。卿何人,而為此行!自今宜自謹約。」還州,仍不悛革。齊州人孟僧振至洛訟遵,詔廷尉少卿鄧述窮鞫,皆如所訴。先,沙門道登過遵。遵以道登荷眷於孝文,多奉以貨,深託仗之。道登屢因言次,申啟救遵,帝不省納,遂詔述賜遵死。時遵子元榮詣洛訟冤,猶恃道登,不時還赴。道登知事決,方乃遣之。遵恨其妻,不與訣,別處沐浴,引椒而死。



 元榮學尚有文才,長於几案。位兼尚書右丞,為西道行臺,至高平鎮,遇城翻,被害。



 遵弟次文,雖無位宦,而貲產巨萬。遵每責其財,又結憾於遵,吉凶不相往反。



 時論責之。毗字子翼,鄉邑稱為長者,位
 征南從事中郎。



 初,允所引劉模者,長樂信都人,頗涉經籍。允撰修國記,選為校書郎,與其緝著。常令模帶持管籥,每日同入史閣,接膝對筵,屬述時事。允年已九十,手目稍衰,多遣模執筆而占授裁斷之,如此者五六歲。允所成篇卷,模預有功。太和中,除南潁川太守。



 王肅之歸闕,路經縣瓠,羈旅窮悴,時人莫識。模獨經給所須,弔待以禮,肅深感其意。及肅臨豫州,模猶在郡,徵報復之,由是為新蔡太守。在二郡積十年,寬猛相濟,頗有聲稱。遷陳留太守。時年七十餘矣,而飾老隱年,昧禁自效。遂家於南潁川,不復歸其舊鄉矣。



 祐字子集,允之從祖弟也。本名禧,以與咸陽王同名,孝文賜名焉。祖展,慕容寶黃門郎。道武平中山,徙京師。卒於三都大官。父讜,從太武滅赫連昌,以功賜爵南皮子。與崔浩共參著作,位中書侍郎、給事中、冀青二州中正。假散騎常侍、蓚縣侯,使高麗。卒,贈冀州刺史,假滄水公,謚曰康。祐兄祚襲爵,位東青州刺史。



 祐博涉書史,好文字雜說,性通放,不拘小節。自中書學生再遷中書侍郎,賜爵建康子。文成末,兗州東郡吏獲一異獸,送之京師,時無識者,詔以問祐。祐曰:「此是三吳所出,厥名鯪鯉。餘域率無,今我獲之,吳、楚之地,其有歸國乎?」



 又有人於靈
 丘得玉印一以獻,詔以示祐。祐曰:「印上有籀書二字,文曰『宋壽』,壽者命也,我獲其命,亦是歸我之徵。」獻文初,宋義陽王昶來奔,薛安都等以五州降附,時謂祐言有驗。



 孝文初,拜祕書令。後與丞李彪等奏曰:「《尚書》者,記言之體;《春秋》者,錄事之辭。尋覽前志,斯皆司勳之實錄也。惟聖朝創制上古,開基《長發》,自始祖以後,至於文成,其間世數久遠,是以史弗能傳。臣等疏漏,忝當史職,披覽國記,竊有志焉。愚謂自王業始基,庶事草創,皇始以降,光宅中土。宜依遷、固大體,令事類相從,紀傳區別,表志殊貫,如此修綴,事可備書。著作郎已下,請取有才用者,參
 造國書。如得其人,三年有成矣。」帝從之。



 孝文嘗問祐:「比水旱不調,何以止災而致豐稔?」祐曰:「堯湯之運,不能去陽九之會。陛下道同前聖,其如小旱何?但當旌賢佐政,則災消穰至矣。」又問止盜之方。祐曰:「茍訓之有方,寧不易息?當須宰守貞良,則盜賊止矣。」祐又上疏云:「今選舉不采職政之優劣,專簡年勞之多少,斯非盡才之謂。宜棄彼朽勞,唯才是舉。又勳舊之臣,年勤可錄而才非撫人者,則可加以爵賞,不宜委以方任。



 所謂王者可私人以財,不私人以官者也。」帝皆善之。加給事中、冀州大中正。時李彪專統著作,祐為令,時關豫而己。出為西兗州
 刺史,假東光侯,鎮滑臺。



 祐以郡國雖有太學,縣黨宜有黌序,乃縣立講學,黨立教學,村立小學。又令一家之中,自立一碓;五家之外,共造一井,以給行客,不聽婦人寄舂取水。又設禁賊之方,令五五相保,若盜發,則連其坐。初似煩碎,後風化大行,寇盜止息。



 轉宋王劉昶傅,以參定律令,賜帛粟馬等。昶以其舊官年耆,雅相祗重。拜光祿大夫,傅如故。昶薨,徵為宗正卿,而祐留連彭城,久不赴。僕射李沖奏祐無事稽命,處刑三歲,以贖論,免卿任。復為光祿,卒。太常謚曰煬侯。詔曰:「不遵上命曰靈,可謚為靈。」



 子和璧,字僧壽,有學尚,位中書博士,早卒。和璧子
 顥,字門賢,學涉有時譽。襲爵建康子,仕輔國將軍、朝散大夫,贈滄州刺史,謚曰惠。子德正襲。



 德正幼而敏慧,有風神儀表。初為齊文宣儀同開府參軍,尋知管記事,甚相親狎。累遷相府掾,神武委以腹心。徙給事黃門侍郎,方雅周慎,動見稱述。文襄嗣業,如晉陽。文宣在鄴居守,令德正參機密,彌見親重。文襄之崩,勳將等以纘戎事重,勸文宣早赴晉陽。文宣不決,夜中召楊愔、杜弼、崔季舒及德正等,策始定。



 以愔從,令德正居守。以為相府司馬,專知門下事。



 德正與文宣舊暱愛,言無不盡。散騎常侍徐之才館客宋景業,先為天文圖
 讖學,又陳山提家客楊子術有所援引,並因德正勸文宣行禪代事。德正又固請。文宣恐愔不決。自請赴鄴與愔言,乃定。還,未至而文宣便發晉陽。至平城都,召諸勳將入,告以禪讓事,諸將莫敢答者。時杜弼為長史,密啟文宣:恐關西因此自稱義兵,挾天子而東向,將何以待?之才云:今若先受魏禪,關西自應息心。縱欲屈強,止當逐我稱帝。弼無以答。文宣以眾意未協,又先得太后旨云:「汝父如龍,汝兄如猛獸,皆以帝王之重,不敢妄據,尚以人臣終。何欲行舜禹事?此正是高德正教汝。」



 又說者以為昔周武王再駕盟津,然始革命。於是乃旋晉陽。



 自
 是居常不悅。徐之才、宋景業等每言卜筮雜占陰陽緯候,必宜以五月應天命。



 德正亦敦勸不已,仍白文宣追魏收。收至,令撰禪讓詔冊、九錫、建臺及勸進文表。



 至五月初,文宣發晉陽。德正又錄在鄴諸事條進於文宣。文宣令陳山提馳驛齎事條并密書與楊愔。山提以五月至鄴,楊愔即召太常卿邢邵、七兵尚書崔甗、度支尚書陸操、太子詹事王昕、給事黃門侍郎陽休之、中書侍郎裴讓之等議撰儀注。六日,要魏太傅咸陽王坦、錄尚書事濟陰王暉業等總集,引入北宮,留于東齋,受禪後乃放還宅。文宣發至前亭,所乘馬忽倒,意甚惡之。至平城
 都,便不復肯進。德正與徐之才苦請曰:「山提先去,恐其漏泄,不果。」即命司馬子如、杜弼馳驛續入,觀察物情。七日,子如等至鄴,眾人以事勢已決,無敢異言。九日,文宣至城南頓所。時既未行詔敕,諸公文書唯云奉約束,德正及楊愔宣署而已。受禪日,堯難宗染赤雀以獻。帝尋知之,亦弗責也。是日,即除德正為侍中,又領宗正卿。尋遷吏部尚書,侍中如故,封藍田縣公。天保七年,遷尚書右僕射,兼侍中,食勃海郡幹。



 德正與尚書令楊愔,綱紀朝政,多有弘益。



 文宣末年,縱酒酣醉。德正屢進忠言,帝不悅。又謂左右云:「高德正恆以精神陵逼人。」德正甚憂
 懼,乃移疾,屏居佛寺,兼學坐禪,為退身之計。帝謂楊愔曰:「我大憂高德正,其疾何以?」愔知帝內忌之,由是答云:「陛下若用作冀州刺史,病即自差。」帝從之,德正見除書而起。帝大怒,謂曰:「聞爾病,我為爾針!」親以刀子刺之,血流霑地。又使曳下,斬去其趾。劉桃枝捉刀不敢下,帝起臨陛,切責桃枝,桃枝乃斬足之三指。帝怒不解,禁德正於門下省。其夜,開城門,以氈輿送還家。旦日,德正妻出寶物滿四床,欲以寄人。帝奄至其宅,見而怒曰:「我府藏猶無此物。」詰其所從得,皆諸元賂之也。遂曳出斬之,妻出拜謝,又斬之。并其子司徒東閣祭酒伯堅亦見害。



 後
 文宣謂群臣曰:「高德正常言,宜用漢除鮮卑,此即合死。又教我誅諸元,我今殺之,為諸元報仇也。」帝後悔,贈太保、冀州刺史,謚曰康。嫡孫王臣,襲爵藍田縣公,給事中、通直散騎侍郎。德正次子仲武,京畿司馬、平原郡守。



 顥弟雅,字興賢,有風度,位定州撫軍府長史。天平中,追贈冀州刺史。子德範,早有令問,位任城太守,卒。



 雅弟諒,字修賢,少好學,多識強記,居喪以孝聞。太和末,京兆王愉開府辟召,孝文妙簡僚佐,諒與隴西李仲尚、趙郡李鳳起等同時應選。正光中,加驍騎將軍,為徐州行臺。至彭城,屬元法僧反,逼諒同之,不從見害。贈滄州刺史。又詔
 以諒臨危授命,復贈使持節、平北將軍、幽州刺史,優授一子出身,謚曰忠侯。



 諒造《親表譜錄》四十餘卷,自五世以下,內外曲盡,覽者服其博記。



 祐從父弟翼,字次同,豪俠有風神。孝昌末,葛榮作亂,朝廷以翼山東豪右,即家拜勃海太守。翼率合境,徙居河、濟間。魏朝因置東冀州,以翼為刺史,封樂城縣侯。俄除定州刺史,以賊亂不行。及爾朱兆弒莊帝,翼保境自守,卒。中興初,贈使持節、侍中、太保、錄尚書、六州諸軍事、冀州刺史,謚曰文宣。子乾。



 乾字乾邕。性明悟俊偉,有智略,美音容,進止都雅。少時輕俠,長而修改,輕財重義,多所交結。起家拜員外散騎
 侍郎,稍遷員外散騎常侍。魏孝莊之居籓也,乾潛相託附。及爾朱榮入洛,乾東奔於翼。乾兄弟本有從橫志,見榮殺害人士,謂天下遂亂,乃率河北流人於河、濟間,受葛榮官爵。莊帝遣右僕射元羅巡撫三齊,乾兄弟相率出降。朝廷以乾為給事黃門侍郎,兼武衛將軍。爾朱榮以乾前罪,不應復居近要,莊帝聽乾解官歸鄉里。於是招納驍勇,以射獵自娛。及榮死,乃馳赴洛陽。莊帝見之大喜,以乾兼侍中,加撫軍將軍、金紫光祿大夫,鎮河北。又以弟昂為通直散騎常侍、平北將軍。令俱歸,招集鄉閭,為表裏形援。帝親送於河橋上,舉酒指水曰:「卿兄弟
 冀部豪傑,能令士卒致死。京城儻有變,可為朕河上一揚塵。」



 乾垂涕受詔,昂援劍起舞,誓以死繼之。



 及爾朱氏既弒害,遣其監軍孫白雞率百餘騎至冀州。託言括馬,其實欲因乾兄弟送馬收之。乾既宿有報復之心,而白雞忽至,知欲見圖。將先發,以告前河內太守封隆之。隆之父先為爾朱榮所殺,聞之喜曰:「國恥家怨,痛入骨髓,乘機而發,今正其時。謹聞命矣。」



 二月,乾與昂潛勒壯士,夜襲州城,執刺史元嶷,射白雞殺之。於葛榮殿為莊帝舉哀,素服,乾升壇誓眾,詞氣激揚,涕泗交集,將士莫不感憤。欲奉次同為王。



 次同曰:「和鄉里,我不及封皮。」乃推
 隆之為大都督,行州事。隆之欲逃,昂勃然作色,拔刀將斫隆之,隆之懼,乃受命。北受幽州刺史劉靈助節度,俄而靈助被爾硃氏禽。



 屬齊神武出山東,揚聲以討乾為辭,眾情惶懼。乾謂之曰:「高晉州雄材蓋世,不居人下。且爾朱弒主肆虐,正是英雄效節之時,今者之來,必有深計。勿憂,吾將諸君見之。」乃間行,與封隆之子子繪,俱迎於滏陽。因說神武曰:「爾朱氏酷逆,痛結人神,凡厥生靈,莫不思奮。明公威德素著,天下傾心,若兵以忠立,則屈強之徒不足為明公敵矣。鄙州雖小,戶口不減十萬,穀秸之稅,足濟軍資。願公熟詳其計。」神武大笑曰:「吾事諧
 矣!」遂與乾同帳而寢,呼乾為叔父。乾旦日受命而去。



 時神武雖內有遠圖,而外迹未見。爾朱羽生為殷州刺史,神武密遣李元忠於封龍山舉兵逼其城,令乾率眾偽往救之。乾遂輕騎入見羽生,偽為之計。羽生出勞軍,彭樂側從馬上禽斬之,遂平殷州。又共定策,推立中興主。拜侍中、司空公。是時,軍國草創,乾父喪,不得終制。及孝武立,天下初定,乾乃表請解職,行三年之禮。



 詔聽解侍中,司空如故,封長樂郡公。



 乾雖求退,不謂便見從許,既去內侍,朝政空關,居常怏怏。孝武將貳於神武,欲乘此撫之,於華林園宴罷,獨留乾,謂曰:「司空弈世忠良,今日
 復建殊效。相與雖則君臣,實義同兄弟,宜共立盟約。」勒逼之。乾曰:「臣以身許國,何敢有二?」乾雖有此對,然非其本心,事出倉卒,又不謂孝武便有異志,遂不固辭,亦不啟神武。帝以乾為誠己。



 時禁園養部曲稍至千人,驟令元士弼、王思政詣賀拔岳計,又以岳兄勝為荊州刺史。乾謂所親曰:「難將作矣,禍必及吾。」乃密以啟神武。神武召乾問之,乾因勸神武受禪。神武以袖掩其口曰:「勿復言。今啟叔復為侍中,門下之事,一以仰委。」及頻請而帝不答,乾懼變,啟神武,求為徐州。乃以乾為開府儀同三司、徐州刺史。將行,帝聞其與神武言,怒,使謂神武曰:「高
 乾與朕私盟,今復反覆。」



 神武聞其與帝盟,亦惡之,乃封其前後密啟以聞。帝對神武使詰乾。乾曰:「臣以身奉國,義盡忠貞。陛下既有異圖,更言臣反覆。以匹夫加諸,尚或難免,況人主推惡,何以逃命?所謂欲加之罪,其無辭乎!功大身危,自昔然也。若死而有知,差無負莊帝。」詔遂賜死於門下省,年三十七。臨死時,武衛將軍元整監刑,謂曰:「頗有書及家人乎?」乾曰:「吾諸弟分張,各在異處,今日之事,想無全者。兒子既小,未有所識,亦恐巢傾卵破,夫欲何言!」後神武討斛斯椿等,謂高昂曰:「若早用司空策,豈有今日之舉?」天平初,贈太師、錄尚書事、冀州刺史,
 謚曰文昭。以長子繼叔襲祖次同樂城縣侯,令第二子呂兒襲乾爵。



 乾弟慎,字仲密,頗涉文史,與兄弟志尚不同,偏為父所愛。歷位滄州刺史、東南道行臺尚書、光州刺史,加驃騎大將軍、儀同三司。時天下初定,聽慎以本鄉部曲數千自隨。為政嚴酷,又縱左右,吏人苦之。乾死,仲密棄州,將歸神武。武帝敕青州斷其歸路,慎間行至晉陽。神武以為大行臺左丞,轉尚書,當官無所迴避。



 累遷御史中尉,選用御史,多其親戚鄉閭,不稱朝望,文襄奏令改選焉。



 慎前妻,吏部郎中崔暹妹,為慎棄。暹時為文襄委任,乃為暹高嫁其妹,禮夕,親臨之。慎後妻趙郡
 李徽伯女也,艷且慧,兼善書記,工騎乘。慎之為滄州,甚重沙門顯公,夜常語,久不寢。李氏患之,構之於慎,遂被拉殺。文襄聞其美,挑之,不從,衣盡破裂。李以告慎,慎由是積憾,且謂暹構己,遂罕所糾劾,多行縱舍。



 神武嫌責之,彌不自安。出為北豫州刺史,遂據武牢降西魏。



 慎先入關,周文率眾東出,敗於芒山,慎妻子盡見禽。神武以其家勳,啟慎一房配沒而已。仲密妻逆口行中,文襄盛服見之,乃從焉。西魏以慎為侍中、司徒,遷太尉。慎弟昂。



 昂字敖曹。其母張氏,始生一男二歲,令婢為湯,將浴之。婢置而去,養猿繫解,以兒投鼎中,爓而死。張使積薪於
 村外,縛婢及猿焚殺之,揚其灰於漳水,然後哭之。



 昂性似其母,幼時便有壯氣。及長,俶儻,膽力過人,龍犀豹頸,姿體雄異。



 其父為求嚴師,令加捶撻。昂不遵師訓,專事馳騁,每言:「男兒當橫行天下,自取富貴,誰能端坐讀書,作老博士也?」其父曰:「此兒不滅吾族,當大吾門。」



 以其昂藏敖曹,故以名字之。



 少與兄乾數為劫掠,鄉閭畏之,無敢違忤。兄乾求博陵崔聖念女為婚,崔氏不許。昂與兄往劫之,置女村外,謂兄曰:「何不行禮?」於是野合而歸。乾及昂等並劫掠,父次同常繫獄中,唯遇赦乃出。次同語人曰:「吾四子皆五眼,我死後豈有人與我一鍬土邪?」及
 次同死,昂大起冢。對之曰:「老公!子生平畏不得一鍬土,今被壓,竟知為人不?」



 昂以建義初,兄弟共舉兵,既而奉魏莊帝旨散眾。仍除通直散騎侍郎,封武城縣伯。與兄乾俱為爾硃榮所黜,免歸鄉里。陰養壯士,又行抄掠。榮聞惡之,密令刺史元仲宗誘執昂,即送晉陽。及入洛,將昂自隨,禁於駝牛署。既而榮死,莊帝即引見勞勉之。時爾朱世隆還逼宮闕,帝親臨大夏門指麾處分。昂既免縲紲,被甲橫戈,與其從子長命,推鋒徑進,所向披靡。帝及觀者,莫不壯之,即除直閣將軍,賜帛千疋。昂以寇難尚繁,乃請還本鄉招集部曲,仍除通直散騎常侍,加北
 平將軍。



 及聞莊帝見害,京師不守,遂與父兄據信都起兵。爾朱世隆從叔殷州刺史羽生,率五千人掩至龍尾阪。昂將十餘騎,不擐甲而馳之。乾城守,繩下五百人追救,未及而昂已交兵,羽生敗走。昂馬槊絕世,左右無不一當百,時人比之項籍。神武至信都,開門奉迎。昂時在外略地,聞之,以乾為婦人,遺以布裙。神武使世子澄以子孫禮見之,昂乃與俱來。後廢帝立,除冀州刺史以終其身。仍為大都督,率眾從神武破爾朱兆於廣阿。又討四胡於韓陵。昂自領鄉人部曲王桃湯、東方老等三千人,神武將割鮮卑兵千餘人共相參合。對曰:「敖曹所將
 部曲,練習已久,不煩更配。」



 神武從之。及戰,神武軍小卻,兆等方乘之。昂與蔡俊以千騎自栗園出,橫擊,兆軍大敗。是日,微昂等,神武幾殆。太昌初,始之冀州。尋加侍中、開府,進爵為侯。及兄乾被殺,乃將十餘騎奔晉陽。神武向洛陽,令昂為前驅。武帝入關中,昂率五百騎倍道兼行,至崤、陜,不及而還。尋行豫州刺史。天平初,除侍中、司空公。昂以兄乾薨此位,固辭不拜,轉司徒公。好著小帽,世因稱司徒帽。



 神武以昂為西南道大都督,徑趣商、洛。昂度河祭河伯曰:「河伯,水中之神;高敖曹,地上之虎。行經君所,故相決醉。」時山道峻阻,巴寇守險,昂轉鬥而進,
 莫有當鋒。遂克上洛,獲西魏洛州刺史泉GC并將數十人,欲入藍田關。會竇泰失利,神武召昂。昂不忍棄眾,力戰全軍而還。時昂為流矢所中,創甚,顧左右曰:「吾死無恨,恨不見季式作刺史耳!」神武聞之,馳驛啟季式為濟州刺史。



 昂還,復為軍司、大都督,統七十六都督,與行臺侯景練兵於武牢。御史中尉劉貴時亦率眾在焉。昂與北豫州刺史鄭嚴祖握槊,貴召嚴祖,昂不時遣,枷其使。



 使者曰:「枷時易,脫時難。」昂使以刀就枷刎之,曰:「何難之有?」貴不敢校。



 明日,貴與昂坐,外白河役夫多溺死。貴曰:「頭錢價漢,隨之死。」昂怒,拔刀斫貴。貴走出還營,昂便鳴
 鼓會兵攻之。侯景與冀州刺史萬俟受洛解之乃止。時鮮卑共輕中華朝士,唯懌昂。神武每申令三軍,常為鮮卑言;昂若在列時,則為華言。



 昂嘗詣相府,欲直入,門者不聽,昂怒,引弓射之。神武知而不責。性好為詩,言甚陋鄙,神武每容之。元年,進封京兆郡公,與侯景等同攻獨孤信於金墉。與周文帝戰,敗於芒陰,死之。



 是役也,昂使奴京兆候西軍。京兆於傅婢強取昂佩刀以行,昂執殺之。京兆曰:「三度救公大急,何忍以小事賜殺?」其夜,夢京兆以血塗己。寤而怒,使折其二脛。時劉桃棒在勃海,亦夢京兆言訴得理,將公付賊。桃棒知昂必死,遽奔焉。昂
 心輕敵,建旗蓋以陵陣,西人盡銳攻之,一軍皆沒。昂輕騎東走河陽城,太守高永洛先與昂隙,閉門不受。昂仰呼求繩,又不得,拔刀穿闔,未徹,而追兵至。伏於橋下。追者見其從奴持金帶,問昂所在,奴示之。昂奮頭曰:「來,與爾開國公!」



 追者斬之以去。先是,昂夢為此奴所殺,以告盧武,將殺之。武諫乃止,果及難。



 時年四十八。桃棒會喪於路。神武聞之,如喪肝膽,杖永洛二百。西魏賞斬昂首者布絹萬段,歲歲稍與之,周亡猶未充。贈太師、大司馬、太尉公、錄尚書事、冀州刺史,謚曰忠武。西魏尋歸敖曹首,猶可識。



 先是,有鵲巢於庭中地上,家人怪之,及其首
 函至,置正當巢處。葬後,其妻張氏常見敖曹夜來旦去,有若生平。傍人莫見,唯犬隨而吠之,歲餘乃絕。其故吏東方老為南兗州刺史,追慕其恩,為立祠廟。靈像既成,頭上坼裂,改而更作,裂如初,見者咸稱神異。



 子突騎嗣,早卒。文襄復親簡昂諸子,以第三子道額嗣。皇建初,追封昂永昌王,以道額襲。武平末,開府儀同三司。入周,為儀同大將軍。隋開皇中,卒於黃州刺史。



 昂弟季式,字子通,亦有膽氣。太昌初,累遷尚食典御,尋加驃騎大將軍。天平中,為濟州刺史。季式兄弟貴盛,並有勳於時,自領部曲千餘人,馬八百疋,衣甲器仗皆備,
 故能追督境內賊盜,多致克捷。時濮陽人杜靈椿等,又陽平路叔文徒黨各為亂,季式並討平之。有客嘗謂季式曰:「濮陽、陽平乃是畿內,何忽遣私軍遠戰?」季式曰:「我與國家同安危,豈有見賊不討之理?若以此獲罪,吾亦無恨。」



 芒山之敗,所親部曲請季式奔梁。季式曰:「吾兄弟受國厚恩,與高王共定天下,一旦傾危而亡之,不義。」是役也,兄昂歿焉。興和中,行晉州事。解州,仍鎮永安。季式兄慎以武牢叛,遣信報季式。季式奔告神武,神武待之如初。武定中,除侍中,尋加冀州大中正、都督。以前後功,加儀同三司。天保初,封乘氏縣子。



 尋遷太常卿。仍為都
 督,隨司徒潘樂征江、淮間。為私使樂人於邊境交易,還京,坐被禁止。尋赦之。四年夏,發疽卒。贈侍中、開府儀同三司、冀州刺史,謚曰恭穆。



 季式豪率好酒,又恃舉家勛功,不拘檢節。與光州刺史李元忠生平遊款。在濟州夜飲,憶元忠,開城門,令左右乘驛馬持一壺酒往光州勸之。朝廷知而容之。兄慎叛後,少時解職。黃門郎司馬消難,左僕射子如之子,又是神武婿,勢盛當時。



 因退食暇,尋季式,酣歌留宿。旦日,重門並關,消難固請去。季式曰:「君以地勢脅我邪?」消難拜謝請出,終不見許。酒至,不肯飲。季式索車輪括消難頸,又更索一車輪自括頸,引滿
 相勸。消難不得已,笑而從之。方俱脫車輪,更留一宿。



 及消難出,方具言之。文襄輔政,白魏帝,賜消難美酒數石,珍羞十輿,并令朝士與季式親狎者,就季式宅宴集。其被優遇如此。



 自昂起兵,為羽翼者,有呼延族、劉貴珍、劉長秋、東方老、劉士榮、成五彪、韓願生、劉桃棒。隨其建義者,有李希光、劉叔宗、劉孟和等。名顯可知者,列之後云。



 東方老,安德鬲人,與昂為部曲。文宣受禪,封陽平縣伯,位南兗州刺史。後與蕭軌等度江,沒。



 李希光,勃海蓚人,初隨高乾起兵,後位儀同三司、揚州刺史。文宣責陳武帝廢蕭明,命儀同蕭軌率希光、東方老、裴英起、王敬寶
 步騎數萬,以天保七年三月度江,襲克石頭城。五將名位相侔,英起以侍中為軍司,蕭軌與希光並為都督。軍中抗禮,動必乖張。頓軍丹楊城下,遇霖雨五十餘日,故致敗。將卒俱死,軍士得還者十二三。



 劉叔宗名纂,樂陵平昌人,歸昂,位車騎將軍、左光祿大夫。



 劉孟和名協,浮陽饒安人,聚眾附昂兄弟,位終大丞相司馬,坐事死。其餘並不知所終云。



 神武初起兵,范陽盧曹亦以勇力稱,為爾朱氏守,據薊。神武厚禮召之,以昂相擬,曰:「宜來,與從叔為二曹。」曹慍曰:「將田舍兒比國士。」遂率其徒自薊入海島。得長人骨,以髑髏為馬皁;脛長丈六尺,以為二
 槊。送其一於神武,諸將莫能用,唯彭樂強舉之。未幾,曹遇疾,恫聲聞於外。巫言海神為崇,遂卒。其徒五百人皆服斬衰,葬畢潛散。曹身長九尺,鬢面甚雄,臂毛逆如豬鬣,力能拔樹。



 性弘毅方重,常從客雅服,北州敬仰之。嘗臥疾,猶申足以舉二人。蠕蠕寇范陽,曹登城射之,矢出三百步,投弓於外,群虜莫能彎,乃去之。時有沙門曇贊,號為神力,唯曹與之角焉。曇贊聞叫聲則勝。



 論曰:高允踐危禍之機,抗雷電之氣,處死夷然,忘身濟難,卒悟明主,保己全名。自非體鄰知命,鑒昭窮達,亦何能若此。宜光寵四世,終享百齡。有魏以來,斯人而已。僧
 裕藝用有聞,聿修之義。世禮貪而無道,能無及乎?子集學業優道,知名前世,儒俊之風,門舊不殞。德正受終之際,契協亂臣,雖鐘淫虐,而名亦茂矣!乾邕兄弟,不階尺土之資,奮臂河朔,自致勤王之舉,神武因之,以成霸業。



 但以非潁川元從,異豐沛故人,腹心之寄,有所未允。露其啟疏,假手天誅,枉濫之極,莫或過此。昂之膽力,氣冠萬夫,韓陵之下,風飛電擊。然則齊氏元功,一門而已。其餘托而義唱,亦足稱云。



\end{pinyinscope}