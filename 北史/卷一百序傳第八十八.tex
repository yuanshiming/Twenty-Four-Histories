\article{卷一百序傳第八十八}

\begin{pinyinscope}

 李
 氏之先,出自帝顓頊高陽氏。當唐堯之時,高陽氏有才子曰庭堅,為堯大理,以官命族,為理氏。歷夏、殷之季。其後理征字德靈,為翼隸中吳伯,以直道不容,得罪于紂。其妻契和氏,摧子利貞逃隱伊侯之墟,食木子而得全,遂改理為李氏。



 周時,裔孫日乾,聚于益壽氏女嬰敷。生子耳,字伯陽,為柱下史。



 子孫散居諸國,或在趙,或在秦。在魏者為段干大夫,段干木其後也。別孫心裡為魏
 文侯興富國之術焉。在趙者曰曇,以功封柏人,武安君牧其後也。在秦者名興族,為將軍。生子伯祐,建功北狄,封南鄭公。伯祐生二子,平燕、內德。子信為秦將,虜燕太子丹。信孫元曠,仕漢為侍中。元曠弟仲翔,位太尉。仲翔討叛羌於素昌,一名狄道。仲翔臨陣殞命,葬狄道川,因家焉。《史記李將軍傳》所云其先自槐里徙居成紀,實始此也。仲翔曾孫廣,仕漢,歷文、景、武三帝,位前將軍,立功沙漠。廣子當戶、椒、敢。當戶子陵,戰歿匈奴。椒。敢歷侍中、郎中令、關內侯。生子禹,位至侍中。並事具《史》、《漢》。禹生承公。承公生蜀郡太守先。



 先生長宗。長宗生博士況。況生
 孝廉本。本字上明,生巴郡太守次公。次公生臨淮太守軌。軌字逸文,生積弩將軍隆。隆字業緒,生雍。雍字俊熙,仁魏,歷尚書郎、濟北、東莞二郡太守。雍生柔。柔字德遠,晉舉秀才,為相國從事中郎、北地太守。



 柔生弇,字秀子,高亮果毅,有智局。晉末大亂,與從兄卓居相國晉王保下。



 卓位相國從事中郎,保政刑不脩,卓率宗族奔于張寔,弇亦隨焉。因仕於張氏,為驍騎左監。弇本名良,妻姓梁氏。張駿謂弇曰:「卿名良,妻又姓梁,令子孫何以目其舅氏?昔耿弇以弱年立功,啟中興之業,吾方賴卿,有同耿氏。」乃使名弇。



 歷天水太守、衛將軍,封安西亭侯。卒,年
 五十六,贈武衛將軍,建初中,追謚景公。子昶,字仲堅,幼有名譽,年十八而亡。建初中,追謚簡公。



 涼武昭王皓,字玄盛,小字長生,簡公昶之子也。遺腹而誕,祖母梁氏,親加撫育。幼好學,性沈敏寬和,美器度,通涉經史,尤長文義。及長,頗習武藝,誦孫、吳兵法。常與呂光太史令郭霡及其同母弟宋繇同宿。霡起謂繇曰:「君當位極人臣,李君必有國土之分。家有騧黃馬生白額駒。此其時也。」及呂光之末,段業自稱涼州牧,以昭王為效穀令。而敦煌護軍馮翊郭謙、沙州中從事敦煌索仙等以昭王溫毅有惠政,推為寧朔將軍、敦煌太守。昭王初難之。會宋繇
 仕於業,告歸,言於昭王曰:「兄忘郭霡言邪?白額駒今已生矣!」昭王乃從之。尋進號冠軍將軍,稱籓于業。業僭稱涼王,其右衛將軍索嗣構昭王于業,乃以嗣為敦煌太守,率騎而西,昭王命師擊走之。於是晉昌太守唐瑤移檄六郡,推昭王為大都督、大將軍、涼公,領秦涼二州牧、護羌校尉,依竇融故事。昭王乃赦境內,建元號庚子,追崇祖考,大開霸府,置左右長史、司馬、從事中郎,備置僚寀。廣闢土宇,屯玉門、陽關,大田積穀,為東討之資。立靖恭堂以議朝政,閱武事焉。圖讚自古聖帝、明王、忠臣、孝子、烈士、貞女,親為序頌,以明鑒誡之義。當時文武群公僚
 佐,亦皆圖讚所志。五年,改元為建初。遣舍人黃始、梁興間行歸表於晉。是歲,乃自敦煌徙都酒泉。又以表未報,復遣沙門法泉間行通表建鄴。于時百姓樂業,請勒銘酒泉,乃使儒林祭酒劉彥明為文,刻石頌德。又有白狼、白兔、白雀、白雉、白鳩等集于園間。群下以為白祥,金精所誕,皆應時邕而至;又有神光、甘露、連理、嘉禾眾瑞,請史官記其事。昭王從之。上巳日,宴于曲水,命群僚賦詩,昭王親為之序。



 於是寫諸葛亮訓誡以勖諸子焉。昭王以緯世之量,為群雄所奉,兵無血刃,遂啟霸業,乃修敦煌舊塞。薨,謚曰武昭王,廟號高祖,陵號建世,武昭王十
 子,譚、歆、讓、愔、恂、翻、豫、宏、眺、亮。世子譚早卒。



 後主諱歆,字士業,武詔王第二子也。武昭王薨,府僚奉為都督、大將軍、涼公,領涼州牧、護羌校尉,大赦境內,改元為嘉興。尊母尹氏為太后。在位四年,為沮渠蒙遜所敗,國亡。武昭王以魏道武皇帝天興二年立,後主以明元皇帝泰常五年而亡,據河右凡二世,二十一年。世子重耳奔於江左,遂仕于宋。後歸魏,位恆農太守,即皇室七廟之始也。



 後主弟讓,字士遜,雅量凝重,善於謀略,位寧朔將軍,領西羌校尉、輔國將軍、晉敦煌太守、新鄉侯,贈驃騎大將軍,謚曰穆。讓弟愔,字士正,位晉昌、敦煌太守。愔弟恂,字
 士如,有幹略,位酒泉、敦煌太守,遇家國之難而終。恂弟翻,字士舉,小字武疆,英雄秀出,有雄略,位車騎將軍,祈連、酒泉、晉昌郡太守。



 翻弟豫,字士寧,位西海太守。豫弟宏,字士贊,位前將軍、中華令。宏弟眺,字士遠,位左將軍。眺弟亮,字士融,位右將軍。



 寶字懷素,小字衍孫,晉昌太守翻之子也。沈雅有度量,驍勇善撫接。遇家難,為沮渠蒙遜囚於姑臧。歲餘,與舅唐契北奔伊吾,臣於蠕蠕。其遺眾之歸附者,稍至二千,寶傾身禮接,甚得其心,眾皆為之用,每希報雪。屬太武遣將討沮渠無諱於敦煌,無諱捐城遁走。寶自伊吾南歸敦煌,遂脩繕城府,規復
 先業,遣弟懷達,奉表歸誠,太武嘉其忠款,拜懷達散騎常侍、敦煌太守;別遣使授寶使持節、侍中、都督西垂諸軍事、鎮西大將軍、開府儀同三司、領護西戎校尉、沙州牧、敦煌公,仍鎮敦煌,四品已下,聽承制假授。真君五年,因入朝,遂留京師,拜外都大官。



 轉鎮南將軍、並州刺史,還除內部大官。文成初,代司馬文思鎮懷荒,改授鎮北將軍。太安五年薨,年五十三,詔賜命服一襲,贈以本官,謚曰宣。有六子,承、茂、輔、佐、公業、沖。公業早卒。



 承字伯業,少有謀略。初,寶欲歸款,僚庶多有異議。承時年十三,勸寶速定大計,於是遂決。寶仍令承隨表入賀。太武深相器
 異,禮遇甚優,賜爵姑臧侯。後遭父憂,居喪以孝聞。承應傳先封,以自有爵,乃以本封讓弟茂,時論多之。承方裕有鑒裁,為時所重。文成末,以散侯出為龍驤將軍、滎陽太守,為政嚴明,甚著聲稱。延興五年卒,時年四十五,贈使持節、大將軍、雍州刺史,謚曰穆。



 長子韶,字元伯,學涉有器量,與弟彥、虔、蕤並孝文賜名焉。韶雅為季父沖所知重。延興中,補中書學生,襲爵姑臧侯,除儀曹令。時脩改車服及羽儀制度,皆令韶典焉。遷給事黃門侍郎。後依例降侯為伯,兼大鴻臚卿,黃門如故。孝文將創遷都之計,詔引侍臣,訪以古事。韶對曰:「洛陽九鼎舊所,七百
 攸墓,地則土中,實均朝貢,惟王建國,莫尚於此。」帝稱善。遷太子右詹事,尋罷左右,仍為詹事、肆州大中正。出為安東將軍、兗州刺史。帝自鄴還洛,韶朝於路,帝言及庶人恂事曰:「卿若不出東宮,或未至此也。」宣武初,徵拜侍中,領七兵尚書,除撫軍將軍,并州刺史。以從弟伯尚同咸陽王禧之逆,免除官爵。久之,兼將作大匠,敕參定朝儀律令。及呂茍兒反於秦州,除撫軍將軍、西道都督,行秦州事,與右衛將軍元麗,率眾討之。事平,即真,璽書勞勉,復其先爵。時隴右新經師旅,百姓多不安業,韶善撫納,甚得夷夏之心。



 孝明初,自相州刺史入為殿中尚書,
 行雍州事,後除中軍大將軍、吏部尚書,加散騎常侍,出為冀州刺史。清簡愛人,甚收名譽,政績之美,聲冠當時。明帝嘉之,就加散騎常侍,遷車騎將軍,賜劍珮、貂蟬各一具,驊騮馬一匹,并衣服寢具。



 韶以年及懸車,抗表遜位,優旨不許。轉定州刺史,常侍如故。及赴中山,冀州父老皆送出西境,相聚而泣。二州境既連接,百姓素聞其德,州內大安。正光五年,卒於官,年七十二。詔贈帛七百匹,贈使持節、散騎常侍、車騎大將軍、司空公、雍州刺史,謚曰文恭。既葬之後,有冀州兵千餘人,戍於荊州,還徑韶墓,相率培塚,數日方還,其遺愛如此。永安中,以剋定
 秦、隴功,追封安城縣開國伯,邑四百戶。



 長子璵,字道璠,溫雅有識量。魏永平二年,釋褐太尉府行參軍,累遷尚書倉部郎中。後汝南王悅為司州牧,悅性質疏冗,情識不倫,朝廷以璵器望兼美,閑於政事,擢為悅府長史,兼知州務。甚得毗贊之方,因除司州別賀。遷光祿少卿。永安初,以本官兼度支尚書,襲封安城縣伯,又除司徒右長史,仍兼尚書。及遷都於鄴,留璵於後,監掌府藏。及撤運宮廟材木,以明乾見稱。加征南將軍、金紫光祿大夫,尋兼給事黃門侍郎,監典書事。出為東徐州刺史,為政清靜,人吏懷之。解州還,以老疾,不求仕進。齊受禪,追璵
 兼前將軍,導從於圜丘行禮。又攝護軍,陪神武神主入太廟。璵意不願策名兩朝,雖以宿德耆舊被徵,過事即絕朝請。文宣亦曾命璵預華林宴,顧訪舊事,甚重之。天保四年卒,年七十二。



 子詮,字世良,任城郡守,贈涇州刺史。



 子伯卿,太師府參軍事。伯卿子師上,聰敏好學,雅有詞致。外祖魏收無子,惟有一女,生師上,甚愛重之,童祇便自教屬文,有名於世。後與范陽盧公順俱為符璽郎,待詔文林館。與博陵崔君洽同志友善,從駕晉陽,寓居僧寺,朝士謂之康寺三少,為物論推許若此。隋煬帝居蕃,奏為王府記室,終於揚州。



 詮弟謐,字世安,位高陽郡
 守、司農卿、安州刺史。謐子千學,齊武平中尚神武女浮陽長公主,拜駙馬都尉、南青州刺史。



 謐弟誦,字世業,位假儀同三司、臨漳令。誦弟世韞,太子舍人、殿中郎。



 璵子孫繁衍,行人號其宅為李東徐村。



 璵弟瑾,字道瑜。美容貌,有才學,特為韶所鍾愛。清河王懌甚知賞之。懌為司徒,辟參軍事。轉著作郎,稍遷通直散騎侍郎,與給事黃門侍郎王遵業、尚書郎盧觀典脩儀注。王、盧即瑾之外兄。臨淮王彧謂瑾等三俊,共掌帝儀,可謂舅甥之國。及明帝崩,上謚策文,瑾所製也。莊帝初,於河陰遇害,年三十九,贈冠軍將軍、齊州刺史。



 子產之,字孫僑。容貌短陋,
 而撫訓諸弟,愛友篤至。其舅盧道將稱之曰:「此兒風調,足為李公家孫。」位北豫州司馬。子仲膺,字公祀。以學行稱,位太子洗馬。仕周,為東京少吏部上士。隋開皇中,卒於荊州總管司馬。



 產之弟茜之,字曼容,清通,好文學。齊天保初,歷太子洗馬,行陽翟郡守,為政清靜,吏人稱之。遷尚書考功郎中,遇文宣昏縱,見害,時人冤之。



 茜之弟壽之,位梁州中從事,性貞介,不負於人。



 壽之弟禮之,位司徒騎兵參軍。與妻鄭氏相重,妻先亡,遺言終不獨死。未幾,禮之腳上發腫,夢妻云,煮小麥漬之即差,如其言,反創而卒。



 禮之弟行之,字義通,小字師子。簡靜,善守門
 業,多識前言往行,而不以文學自名。居喪盡禮,與兄弟深相友愛。仕齊,歷位都水使者、齊郡太守,帶青州長史。任城王敬憚之,州人號曰李御史。仕周,為冬官府司寺下大夫。隋開皇初,封固始縣男,除唐州下溠郡太守,稱疾不行,卒。行之風素夷坦,為士友所稱。其舅子盧思道深所愛好,常贈詩云:「水衡稱逸人,潘、楊有世親,形骸預冠蓋,心思出囂塵。」時人以為實錄。及疾,內外多為求醫,行之曰:「居常待終,士之道也。



 貧既愈富,何知死不如生?」一皆抑絕。臨終,命家人薄葬,口授墓志以紀其志曰:「隴西李行之,以某年某月終於某所。年將六紀,官歷四朝,
 道協希夷,事忘可否。



 雖碩德高風,有傾先構;而立身行己,無愧夙心。以為氣變則生,生化曰死,蓋生者物之用,死者人之終,有何憂喜於其間哉!乃為銘曰:人生若寄,視死如歸。茫茫大夜,何是何非。」言終而絕,二子,夷、道。



 行之弟凝之,字惠堅。光州中從事,非其所好,僶俛而就,秩滿,徑還冀州棗強野舍。凝之明本草藥性,恆以服餌自持,雖年將耄及,而志力不衰。篤好古文,精心典禮,以之終老,未嘗懈倦。隋仁壽中卒。



 產之兄弟,並有器望。邢子才為禮之墓志云:「食有奇味,相待乃餐,衣無常主,易之而出。」時以為實錄。諸歸相親,皆如姊妹。茜之死,諸弟不
 避當時凶暴,行喪極哀。趙郡李榮來弔之,歎曰:「此家風範,海內所稱,今始見之,真吾師也。」



 欲與連類,即日自名勞之。



 瑾弟贊,字道璋。少有風尚,辟司徒參軍事。卒,贈漢陽郡太守。子脩年,開府參軍,早亡。



 韶弟彥,字次仲,有學業。孝文初,舉秀才,除中書博士,轉諫議大夫。後因考課,降為元士。尋行主客曹事,徙郊廟下大夫。時朝儀典章,咸未周備,彥留心考定,號為稱職。孝文南伐,彥諫曰:「臣以為蕞爾江、閩,未足親勞鑾駕。」頻表雖不見納,而以至誠見嘉。及六軍次於淮南,徵為廣陵王羽長史,加恢武將軍、西翼副將。軍還,除冀州趙郡王乾長史。轉青州廣
 陵王羽長史,帶齊郡太守。徵為龍驤將軍、司徒右長史,轉左長史、秦州大中正。出行揚州事,尋徵拜河南尹,還至汝陰,復敕行徐州事。尋徽拜平北將軍、平州刺史,遷平東將軍,徐州刺史。延昌二年夏,大霖雨,川瀆皆溢。彥相水陸形勢,隨便疏通,得無淹漬之害。朝廷嘉之,頻詔勞勉。入為河南尹,遷金紫光祿大夫、光祿勳,轉度支尚書。出為撫軍將軍、秦州刺史。時破六韓拔陵等反於北鎮,二夏、豳、涼,所在蜂起,而彥刑政甚嚴。正光五年六月,城人薛珍、劉慶、杜超等因四方離叛,突入州門害彥,推其黨莫折大提為帥。永安中,追贈侍中、驃騎大將軍、司徒
 公、雍州刺史,謚曰孝貞。



 子燮,字德諧,少有風望,位司徒主簿。卒,贈太常少卿。子士萬,有雅望,位高都太守。



 燮弟爽,字德明。弟充,字德廣。弱冠,太學博士。大將軍蕭寶夤西討,德廣為行臺郎,募眾而征,戰捷,乃手刃仇人,啖其肝肺。覺寶夤有異志,挺身歸闕,朝廷加爵,辭而不受。寶夤遂與萬俟醜奴同反,大行臺爾天光討之,請德廣為從事中郎。天光用其計,遂定秦、隴。以功除中散大夫。痛父非命,終身不食酒肉。妹夫盧元明嗟重之。



 子士英,有文才,王遵業以女妻之。



 次僧伽,脩整篤業,不應辟命。時鄭子默有名於世,僧伽曰:「行不適道,文勝其質,郭林
 宗所謂牆高基下,雖得必喪,此之徒也。」竟如其言。尚書袁叔德來候僧伽,先滅僕從,然後入門,曰:「見此賢,令吾羞對軒冕。」及卒,叔德為懷舊詩曰:「平生寡俗累,終身無世言。」其見重如此。僧伽弟法藏,內清介,位員外郎。



 德廣弟德顯,位散騎侍郎,贈東秦州刺史。



 德顯弟德明,敦重有器局,位高陽太守,贈光祿少卿、光州刺史。



 彥弟虔,字叔恭。太和初,為中書學生,遷秘書中散,轉冀州驃騎府長史、太子中舍人。宣武初,遷太尉從事中郎,出為清河太守。屬京兆王愉反,虔棄郡奔闕。



 宣武聞虔至,謂左右曰:「李虔在冀州日久,恩信著物,今拔難而來,眾情自解
 矣。」



 乃授虔別將,令軍前慰勞。事平,轉長樂太守。延昌初,冀州大乘賊起,令虔以本官為別將,與都督元遙討平之。遷後將軍、燕州刺史。還為光祿大夫,加平西將軍,兼大司農。出為散騎常侍、安東將軍、兗州刺史。追論平冀州之功,賜爵高平男。



 還京,除河南邑中正,遷領軍將軍、金紫光祿大夫。孝莊初,授特進、車騎大將軍、儀同三司、散騎常侍,又進號驃騎大將軍、開府儀同三司。永安三年薨,年七十四,贈侍中、驃騎大將軍、太尉公、都督冀定瀛三州諸軍事、冀州刺史,謚宣景。



 長子喚,字仁明,位尚書左外兵郎。莊帝初,於河陰遇害,年四十,贈度支尚書、
 安東將軍、青州刺史。子裒,章武郡守。裒弟匹,汲郡守。並以幹局見知。



 喚弟仁曜,位員外散騎侍郎、太尉錄事參軍。與兄喚同於河陰遇害,年三十八,贈散騎常侍、左將軍、兗州刺史。子捴,字道熾,學尚有風儀。魏武定中,司空長流參軍。齊天保末,為尚書郎,終於光州司馬。



 仁曜弟皓,字仁昭,位散騎侍郎。亦遇害河陰,贈征虜將軍、涼州刺史。子士元、士操、武定中,並儀同開府參軍事。



 皓弟曉,事列于後。



 虔弟蕤,字延賓,歷步兵校尉、東郡太守、司農少卿。卒,贈龍驤將軍、豫州刺史。



 子諺,字義興。有幹局,起家太學博士,領殿中侍御史,稍遷東郡太守。莊帝初,濟、
 廣二州刺史,加散騎常侍。節閔時,與第三弟通直散騎常侍義真、第七弟太常少卿義邕同為爾朱仲遠所害。義邕,莊帝居籓之日,以外親,甚見親暱。及即位,特蒙信任。爾朱榮之誅,義邕預其事,由是並及禍。節閔初,諺贈侍中、驃騎將軍、吏部尚書、冀州刺史,義真贈前將軍、齊州刺史,義邕贈安東將軍、青州刺史。諺次弟義順,司空屬。第四弟義遠,國子博士。莊帝初,並於河陰遇害,贈散騎常侍、征東將軍、雍州刺史。



 承弟茂,字仲宗。文成末,襲父爵鎮西將軍、敦煌公。孝文初,除長安鎮都將,轉西汾州刺史,將軍如故。入為光祿大夫,歷西兗州刺史,例降
 為侯。茂性謙慎,以弟沖寵盛,懼於盈滿,以疾求遜位。孝文不奪其志,聽食大夫祿,還私第。因居中山,自是優遊里舍,不入京師。卒年七十一,謚曰恭侯。



 子靜,字紹安,襲,位東平原太守。卒,子遐,字智遠,襲。遐有几案才,位河內太守。從孝莊南度河,於河陰遇亂兵所害。事寧,追贈散騎常侍、車騎大將軍、尚書右僕射、秦州刺史,封盧鄉伯。



 靜弟孚,字仲安。恭慎篤厚,歷汝南、中山二郡太守。孝莊初,以外親超授撫軍將軍、金紫光祿大夫,出為鎮東將軍、滄州刺史,加散騎常侍。



 孚弟季安,粗涉書史,位北海王顥撫軍長史。顥為關西都督,復引為長史,委以戎政。卒於
 軍,贈征虜將軍、涼州刺史。



 茂弟輔,字叔直。有器望,解褐中書博士,遷司徒議曹掾。太和中,孝文為咸陽王禧納其女為妃,除鎮遠將軍、潁川太守,帶長社戍。輔綏懷招集,甚得邊和。



 卒於郡,贈征虜將軍、秦州刺史,謚曰襄武侯。



 長子伯尚,少有重名,弱冠除秘書郎。孝文每云:「此李氏之千里駒。」稍遷通直散騎侍郎,敕撰《太和起居注》。宣武初,兼給事黃門侍郎,坐與咸陽王禧謀反誅。



 伯尚弟仲尚,儀貌甚美,少以文學知名。年二十,著《前漢功臣序贊》,及季父司空沖誄。高聰、邢巒見而歎曰:「後生可畏,非虛言也。」起家京兆王愉府參軍。坐兄事,賜死。



 仲尚弟季
 凱,沈敏有識量。坐兄事,與母弟俱徙邊,久之,會赦免。遂寓居晉陽,沈廢積年。後歷位并州安北府長史。孝明崩,爾朱榮陰圖義舉,季凱豫謀。及莊帝踐祚,徵拜給事黃門侍郎,封博平縣侯,加散騎常侍、秘書監、中軍將軍。後爾朱世隆以榮之死,謂季凱通知,於是見害。孝武初,追贈侍中、驃騎將軍、吏部尚書、定州刺史。



 季凱弟延慶,位陳留太守、金紫光祿大夫。延慶弟延度,衛將軍、安德太守。



 輔弟佐,字季翼,有文武才幹。孝文初,兼散騎常侍使高麗,以稱旨,還拜常山太守、真定縣子。遷懷州刺史,進爵山陽侯,加安南將軍、河內公,轉相州刺史,所在有稱
 績。後拜安遠將軍,敕與征南將軍城陽王鸞、安南將軍盧陽烏等攻赭陽,各不相節度。諸軍以敵強故班師,佐逆戰,為賊所敗,坐徙瀛州。車駕征宛、鄧,復起佐,假平遠將軍、統軍,以功封涇陽縣子。沔北既平,以佐為廣陽王嘉鎮南府長史,加輔國將軍,別鎮新野。及大軍凱旋,孝文執佐手曰:「沔北,洛陽南門,卿勉為朕善守。」孝文崩,遣敕以佐行荊州事。佐在州,威信大行,邊人悅附,前後歸者二萬許家。尋正刺史。宣武初,徵兼都官尚書。卒,年七十一,贈秦州刺史,謚曰莊。



 子遵襲。遵豪爽有父風,卒於司空司馬,贈洛州刺史。子果襲,位司空諮議參軍,坐通
 西魏見殺。



 遵弟柬,字休賢。郡辟功曹,以父憂去職,遂終身不食酒肉,因屏居鄉里。司空、任城王澄嘉其操尚,以為參軍事,累遷濟州刺史。卒,贈殿中尚書、相州刺史。



 柬弟挺,字神俊,小名提。少以才學知名,為太常劉芳所賞。歷位中書侍郎、太常少卿、荊州刺史。時梁將曹敬宗來寇,攻圍積時,又引水灌城,城不沒者數板,神俊循撫兵人,戮力固守。詔遣都督崔暹、別將王羆、裴衍等赴援,敬宗退走。時寇賊之後,城外有露骸,神俊令收葬之。徵拜大司農。孝明末,除鎮軍將軍,行相州事,時葛榮南逼,神俊憂懼,乃故墜馬傷足,仍停汲郡,有詔追還。莊帝即位,
 以神俊人望,拜散騎常侍、殿中尚書,追論固守荊州功,封千乘縣侯,轉中書監、吏部尚書。神俊意尚風流,情在推引人物,爾朱榮有所用人,神俊不從。見怒,懼,啟求解官,除右光祿大夫。尋屬爾朱兆入京,乘輿幽執,神俊遂逃人間。孝武初,歸闕,拜散騎常侍、驃騎大將軍、左光祿大夫、儀同三司。孝靜初,除驃騎大將軍、華州刺史,入為侍中,薨。年六十四,贈尚書左僕射、司徒公,雍州刺史。



 神俊風韻秀舉,博學多聞,朝廷舊章及人倫氏族,多所諳記。篤學好文雅,老而不輟。凡所交遊,皆一時名士,汲引後生,為其光價,四方才子,咸宗附之。滎陽鄭伯獻常云:「
 從舅為人物宗主。」在洛京時,瑯邪王誦亦美神俊,故名其子曰俊,庶其似之。梁武帝雅重其名,常云:「彼若遣李神俊來聘,我當今劉孝綽往。」



 其見重如此。頸多鼠乳。而性通率,不持檢度,至於少年之徒,皆與褻狎。北遷鄴,於路見狗,溫子升戲曰:「為是宋鵲?為是韓盧?」神俊曰:「為逐丞相東走?為共帝女南徂?」沙苑之敗,神俊策眇馬而走,曰:「丁掾力。」馬倒,曰:「丁掾誤我。」其不拘若此。既不能方重,識者以此為譏。喪二妻,又欲娶鄭嚴祖妹,神俊之從甥也。盧元明亦將為婚。遂至紛兢,二家鬩於嚴祖之門。鄭卒歸元明,神俊惆悵不已。時人以神俊為鳳德之衰。



 沖
 字思順,承少弟也,本名思沖,孝文改焉。少孤,為承訓養。承常言,此兒器重非恒,方為門戶所寄。沖雅有大量,隨兄至滎陽,時牧守子弟多侵亂人庶,輕有乞奪,沖與承長子韶獨清簡皎然,無所求取,時人美焉。獻文末,為中書學生,沖善交遊,不妄戲雜,流輩重之。孝文初,以例遷秘書中散,典禁中文字。以脩敕繁慧,漸見寵待,遷內祕書令,南部給事中。



 舊無三長,唯立宗主主督護,所以我隱冒,五十、三十家方為一戶。沖以三正所由來遠,於是創三長之制上之。文明太后覽而稱善,引見公卿議之,群臣多有不同。太后曰:「立三長則課有常準,賦有常分,包
 蔭之戶可出,僥倖之人可止,何為不可?」詞議雖有乖異,然惟以變法為難,更無異議,遂立三長,公私便之。



 遷中書令,加散騎常侍,給事中如故。尋轉南部尚書,賜爵順陽侯。沖為文明太后所幸,恩寵日盛,賞賜月必數千萬,進爵隴西公,密致珍寶服御以充其第,外人莫得而知。沖家素清貧,於是室富。而謙以自牧,積而能散,近自姻族,逮於鄉閭,莫不分及。虛己接物,垂念羈寒,衰舊淪屈由之躋敘者,亦以多矣,時以此稱之。初,沖兄佐與河南太守來崇同自涼州入國,素有微嫌,佐因構成崇罪,餓死獄中。後崇子護為南部郎,深慮為沖陷,常求退避,沖
 每慰撫之。護後坐贓罪,懼必不濟,沖具奏與護本末嫌隙,乞原恕之,遂得不坐。沖從甥陰始孫貧,來沖家,至如子姪。有人求官,因其納馬於沖,始孫輒受而不言。後假方便,借沖。此馬主見沖乘馬而不得官,後自陳首始末。沖聞大驚,執始孫,以狀款奏,始孫坐死。其處要自厲,不念愛惡,皆此類也。



 時循舊,王公重臣皆呼名,孝文帝謂沖為中書而不名之。文明太后崩後,孝文居喪,引見待接有加。及議律令,潤飾辭旨,刊定輕重,孝文雖自下筆,無不使訪焉。沖竭忠奉上,知無不盡,出入憂勤,形於顏色,雖舊臣戚輔,莫能逮之,俱服其明斷慎密而歸心焉。
 於是天下翕然。及殊方聽望,咸宗奇之。孝文亦深相仗信,親敬彌甚,君臣之間,情義莫二。及置百司,開建五等,以沖參定典式,封滎陽侯,拜廷尉卿,遷侍中、吏部尚書、咸陽王師。東宮建,拜太子少傅。孝文初依《周禮》置夫嬪之列,以沖女為夫人。及營明堂,詔沖領將作大匠,與司空、長樂公亮共監興繕。



 車駕南伐,加沖輔國大將軍,統眾翼從。自發都至洛陽,霖雨不霽,仍詔六軍發軫,孝文戎服執鞭,御馬而出,群臣稽顙於馬首之前。孝文曰:「今大軍將邁,公等更欲何云?」沖進,請曰:「發都淫雨,士馬困弊,矜喪反旆,於義為允。」



 孝文曰:「已至於此,何容停駕!」沖
 又進曰:「今者之舉,天下所不願,敢以死請。」孝文大怒曰:「方欲經營宇宙,而卿等儒生,屢疑大計,斧鉞有常,卿勿復言!」策馬將出。,大司馬安定王休、兼左僕射任城王澄等並殷勤泣諫,孝文乃喻群臣曰:「今者興動不小,勤而無成,何以示後?若不南鑾,即當移都於此。光宅土中,幾亦時矣,王公等以為何如?議之所決,不得旋踵,欲遷者左,不欲遷者右。」



 安定王休等相率如右。前南安王楨進曰:「愚者闇於成事,智者見於未行,見至德者不議於俗,成大功者不謀於眾。非常之事。廓神都以延王業,都中土以制帝京,周公啟之於前,陛下行之於後,固其宜也。
 請上安聖躬,下慰人望,光宅中原,輟彼南伐。此臣等之願,亦蒼生幸甚。群臣咸唱萬歲。孝文初謀南遷,恐眾心戀舊,乃示為大舉,因以脅定群情,外名南伐,其實遷也。舊人懷土,多所不願,內憚南征,無敢言者,於是定都洛陽。



 尋以沖為鎮南將軍,侍中、少傅如故。委以營構之任,改封陽平郡侯。車駕南征,以沖兼左僕射,留守洛陽,遷尚書左僕射,仍領少傅,改封清泉縣侯。及太子恂廢,沖罷少傅。孝文引見公卿於清徽常曰:「今徙極中天,創居嵩、洛,雖大構未成,要自條紀略舉。但南有未賓之豎,兼兇蠻密邇,朕取南之計決矣,所行之謀必定。頃來陰陽
 卜術之士咸勸,朕今征必剋。此既家國大事,宜其君臣各盡所見。」



 沖曰:「征戰之法,先之人事,然後卜筮。卜筮雖吉,猶恐人事未備。京師始遷,行業未定,加之征戰,以為未可。」帝曰:「僕射之言,非為不合朕意。然咫尺寇戎,無以自安,理須如此。若待人事備,復非天時,將若之何?如僕射之言,便終無徵理。」沖機敏有巧思,北京明堂、圓丘、太廟,及洛都初基,安處郊兆,新起宮寢,皆資於沖。勤志強力,孜孜無怠,且理文簿,兼營匠制,几案盈積,剞劂在前,初不勞厭也。然顯貴門族,榮益六姻,兄弟子姪,皆有官爵,一家歲祿,萬匹有餘。年纔四十,而鬢髮斑白,姿貌甚
 美,未有衰狀。



 李彪之入京也,孤微寡援而自立不群,以沖好士,傾心宗附。沖亦重其器學,禮而納焉,每言之入孝文,公私共相援益。及彪為中尉、尚書,為孝文知待,便謂非復藉沖,更相輕背,唯公坐斂袂而已,無復宗敬之意。沖頗銜之。後孝文南征,沖與吏部尚書、任城王澄並以彪倨傲無禮,遂禁止之,奏其罪狀。沖手自作表,家人不知,辭甚激切,因以自劾。孝文覽其表,嗟嘆久之。既而曰:「道固可謂隘也,僕射亦為滿矣!」沖時震怒,數責彪前後愆悖,瞋目大呼,投折几案,盡收御史,皆泥首面縛,大罵辱詈。沖素性溫柔,而一朝暴恚,遂發病荒悸,言語亂
 錯,猶扼腕叫詈,稱李彪小人。醫藥所不能療,或謂肝藏傷裂,旬餘日卒。時年四十九。



 孝文始聞沖病狀,謂右衛宋弁曰:「僕射執我樞衡,總釐朝務,使我無後顧之憂,一朝忽有此患,朕甚愴懷。」及聞沖卒,為舉哀於縣瓠,發聲悲泣,不能自勝。



 詔書褒述其美曰:「可謂國之賢也,朝之望也。」於是贈司空公,給東園祕器一具、衣一襲,贈錢三十萬、布五百匹、蠟二百斤。有司奏謚曰文穆。葬於覆舟山,近杜預冢,孝文之意也。後車駕自鄴還洛,經沖墓,左右以聞。孝文臥疾,望墳掩涕久之,遣太常致祭。及與留京百官相見,皆敘沖亡沒之故,言及流淚,其相痛惜
 如此。



 子延寔,字禧,性溫良,少為太子舍人。宣武初,襲父爵清泉縣侯。莊帝即位,以母舅之尊,超授侍中、太保,封濮陽郡王。延寔以太保犯祖諱,又以王爵非庶姓所宜,抗表固辭,徙封濮陽郡公,改授太傅。尋轉司徒公,出為使持節、侍中、太傅、錄尚書事、東道大行臺、都督、青州刺史。爾朱兆入京,乘與幽縶,延實以外戚見害於州館。孝武帝初,反葬洛陽,贈使持節、侍中、太師、太尉公、錄尚書事、都督、雍州刺史,謚曰孝懿。



 長子彧,字子文,尚莊帝姊豐亭公主,封東平郡公,位侍中、左光祿大夫、中書監、驃騎大將軍、開府儀同三司、廣州刺史。彧性膏俠,爾朱榮
 之死也,武毅之士,皆彧所進。孝靜初,陷法見害。尋詔復本爵。子道端襲。彧七子,並彭城王勰女豐亭公主所生,以道、德、仁、義、禮、智、信為名。第四子義雄,有識悟,勤學,手不釋書。仕齊,位瑯邪郡守。義雄弟禮成,最知名。



 禮成字孝諧,年七歲,與姑之子蘭陵太守滎陽鄭顥隨魏武帝入關。顥母每謂入曰:「此兒平生,未嘗回顧,當為重器。」及長,沈深有行檢,不妄通賓客。在魏,歷著作郎、太子洗馬、員外散騎常侍。周受禪,拜平東將軍、散騎常侍。于時貴公子皆竟習弓馬,被服多為軍容。禮成雖善騎射,而從容儒服,不失素望。後以軍功拜車騎大將軍、儀同三司,
 賜爵修陽侯,拜遷州刺史。時朝廷有所徵發,禮成度以蠻夷不可擾,擾必為亂,上表固諫,武帝從之。伐齊之役,從帝圍晉陽,齊將席毗羅精兵拒帝,禮成力戰擊退之。加開府,進封冠軍縣公,歷北徐州刺史、戶部中大夫。



 禮成妻竇氏早沒,知隋文帝有非常之表,遂聘帝妹為繼室。及帝為丞相,進位上大將軍,遷司武上大夫,委以心膂,及受禪,拜陜州刺史,進封絳郡公,賞賜優洽。累遷襄州總管、左衛大將軍。時突厥屢為寇患,緣邊要害,多委重臣,由是拜寧州刺史。以疾徵還京,終於家。子世師,位度支侍郎。



 禮成弟智源,有器重。仕齊,卒於高都郡守。



 智源
 弟信則,方雅廉慎。齊武平中,位南陽王大司馬屬。信則形短,中書侍郎頓丘李若戲之曰:「弟為府屬,可謂名以定體。」信則曰:「名以定體,豈過劣弱?」



 尋除尚書倉部郎中。入周,為東京司門下大夫。隋開皇中,卒於沔州刺史。



 彧弟彬,字子儒。其父延寔既別封,彬襲祖爵清泉縣侯。位中書侍郎,卒於左光祿大夫,贈驃騎大將軍、光祿勛、齊州刺史,謚曰獻。子桃杖襲。彬弟彰,位通直散騎侍郎,從父在青州,同時遇害。贈左將軍、瀛州刺史。



 延寔弟休纂,小字鍾羌,頗有父風。位終太子舍人,贈驃騎大將軍、尚書令、司徒公、雍州刺史,追封樂涫縣公,後進封高陽郡
 公。子昂襲。



 昂,魏末為廣平郡太守,齊天保中,卒於光祿卿。



 昂子道隆,有才識,明剖斷。仕齊,位並省尚書左丞。隋開皇中,為尚書比部侍郎。



 休纂弟延孝,位尚書屯田郎中。於河陰遇害,贈侍中、車騎大將軍、司空公、定州刺史,進封臨潁縣公。



 韶從弟仲遵,有器業,彭城王勰為定州,請為定州開府參軍,累遷營州刺史。



 時四方州鎮,逆叛相續,營州城內,咸有異心。仲遵單車赴州,及至,與大使盧同以恩信懷誘,率皆安帖。後明帝又詔盧同為行臺,北出慰勞,同疑人情難信,聚兵將往。城人劉安定、就德興等先有異志,謂欲圖己,逐仲遵害之。



 韶從祖抗,自
 涼州渡江左,仕宋,歷晉壽、安陸、東萊三郡太守。



 抗子思穆,字叔仁。有度量,善談論,工草隸,為當時所談。太和十七年,攜家累自漢中歸魏,位都水使者。及車駕南伐,以本官兼直閣將軍,從平南陽,以功賜爵樂平子。宣武踐祚,進爵為伯。累遷京兆內史,在郡八年,頗有政績。卒於營州刺史,贈安東將軍、華州刺史。有子十四人。嫡子斌襲,位散騎侍郎,早卒。



 斌兄獎,字道休,為莊帝所親,超贈思穆衛將軍、中書監、左光祿大夫,謚曰宣武。獎以戚里恩澤,賜爵廣平侯。歷中書侍郎、兼散騎常侍、聘梁使主、黃門郎、司徒左長史,行瀛州事。齊天保初,兼侍中、冀瀛
 滄三州大使,觀察風俗,還,拜魏尹。卒,贈濟州刺史、中書令。子瑰,位中書舍人黃門郎。



 韶族弟琰之,字景珍,小字墨蠡。少知名,號曰神童,從父沖雅所歎異。每曰:「興吾宗者,其此兒乎!」恆資給所須,愛同己子。弱冠舉秀才,不行。曾遊河內北山,便欲有隱遁意。會彭城王勰辟為行軍參軍,苦相敦引,沖又遣信喻之,久乃應召。尋為中尉李彪啟兼著作佐郎,修撰國史。稍遷國子博士,領尚書儀曹郎中,轉中書侍郎、司農少卿、黃門郎,修國史。遷國子祭酒,轉秘書監,兼七兵尚書,遷太常師。



 孝莊初,太尉元天穆北討葛榮,以琰之兼御史中尉,為北道軍司。還,除
 征東將軍,仍兼太常,出為衛將軍、荊州刺史,兼尚書左僕射、三荊二郢大行臺,尋加散騎常侍。琰之雖以儒素自業,而每語人,言吾家世將種,猶有關西風氣。及至州之後,大好射獵,以示威武。爾朱兆入洛,南陽太守趙修延以琰之莊帝外戚,誣琰之規奔梁國,襲州城,遂被囚執。修延仍自行州事。城內人斬修延,還推琰之釐州任。孝武初,徵兼侍中、車騎大將軍、左光祿大夫、儀同三司。永熙二年薨,朝廷悼惜之,贈侍中、驃騎大將軍、司徒公、雍州刺史,謚曰文簡。



 琰之少機警,善談論,經史百家,無不悉覽,朝廷疑事,多所訪質。每云:「崔博而不精,劉精而
 不博,我既精且博,學兼二子。」謂崔光、劉芳也。論者許其博,未許其精,當時議咸共宗之。又自誇文章,從姨兄常景笑而不許。每休閑之際,恆閉門讀書,不交人事。常謂人曰:「吾所以好讀書者,不求身後之名。但異見異聞,心之願也,是以孜孜搜討,欲罷不能。豈為聲名,疾勞世人也?此乃天性,非為力強。」前後再居史事,無所編緝。安豐王延明博聞多識,每有疑滯,常就琰之辨析,自以為不及也。



 二子綱、慧,並從孝武帝入關中。綱位宜州刺史,儀同三司。



 子充節,少慷慨,有英略。隋開皇中,頻以行軍總管擊突厥有功,位上柱國、武陽郡公、朔州總管。甚有威
 名,為虜所憚,後有人譖有謀反,徵還京師。上怒之,充節素剛,憂憤卒。子大亮。



 曉字仁略,太尉虔之子也。少而簡素,博涉經史,早有時譽,釋褐員外散騎侍郎,爾朱榮之立孝莊,曉兄弟四人,與百僚俱將迎焉。其夜,曉衣冠為鼠噬,不成行而免。其上三兄皆遇害。曉乃攜諸猶子,微服潛行,避難東郡。行至成皋,為滎陽令天水閻信所疑,辟易左右,謂曉曰:「觀君儀貌,豈是常倫?古人相知,未必在早,必有急難,須悉心以告。天下豈獨北海孫賓碩乎?」曉以能有長者之言,乃具告情實。信乃厚相資給以免。永安初,授輕車將軍、尚書左右主客郎,仍轉征虜將
 軍、中散大夫,又除前將軍、太中大夫。



 天平初,遷都於鄴,曉便寓居清河,依從母兄崔甗鄉宅。甗給良田三十頃,曉遂築室居焉。時豪右子弟,悉多驕恣,請託暴亂,州郡不能禁止。曉訓勖子弟,咸以學行見稱,時論以此多之。曉自河陰家禍之後,屬王途未夷,無復宦情,備在名級而已。及遷都之後,因退私門,外兄范陽盧叔彪勸令出仕,前後數四,確然不從。



 武定末,齊文襄嗣事,高選僚采,召曉及前開府長史房延祐,並為外兵郎。後徙平西將軍、太尉府咨議參軍事,除頓丘太守。天保中,頻歷廣武、東二郡太守,所在有惠政,為吏人所懷。卒於郡,年五十
 九,贈本官將軍、海州刺史。三子,伯山、仲舉、季遠。



 超字仲舉,以字行於世。性方雅善制,白析美鬚眉,高簡宏達,風調疏遠。博涉經史,不守章句業,至於吉凶禮制,親表咸取則焉。弱冠,仕齊為襄城王大司馬參軍事。時尚書左僕射元文遙以令長之徒,率多寒賤,奏請革選,妙盡高資。仲舉與范陽盧昌衡等八人,同見徵用。以仲舉為司州修武令。仲舉蒞以寬簡,吏人號曰寬明。于時昌衡為平恩令,百姓號曰恩明。故時稱盧、李恩寬之政。武平初,持節,使南定。州人並是蠻左,接帶邊嶂。仲舉具宣朝旨,邊服清謐,朝廷大嘉之,還,授晉州別駕。及周師圍晉州,
 外無救援,行臺左丞侯子欽內圖離貳,欲與仲舉謀,憚其嚴正,將言而止者數四。仲舉揣知其情,乃謂之曰:「城危累卵,伏賴於公,今之所言,想無他事,欲言而還中止也?」子欽曰:「告急官軍,永無消息,勢之危急,旦夕不謀,意欲不坐受夷戮,歸命有道,於公何如?」仲舉正色曰:「僕射高氏恩德未深,公於皇家沒齒非答。臣子之義,固有常道,何至今日,翻及此言!」



 子欽懼泄,夜投周軍。城尋破,周將梁士彥素聞仲舉名,引與言及時事。仲舉曰:「世居山東,受恩高氏,今國維不張,還勞師眾,不能死於臣道,豈敢干非其議。」



 士彥曰:「百里、左車,不無前事,想亦得之。」見
 逼不已,仲舉乃曰:「今者官軍遠來,方申弔伐,當先德澤,遠示威懷,明至聖之情,弘招納之略,令所至之所,歸誠有地,所謂王者之師,征而不戰也。」土彥深以為然,益相知重。初,城敗之後,公私蕩然,軍人簿帳,悉多亡毀,戶口倉儲,無所憑據。事無大小,士彥一委仲舉,推尋勾當,絲髮無遺,於軍用甚有助焉。



 鄴城平,仍將家隨例入關。仲舉以親故流離,情不願住,妻伯父京兆尹博陵崔宣猷留不許去。固辭,乃得還鄴。尋有詔,素望舊資,命州郡勒送,仲舉懼嚴命而至。補秋官賓部上士,深乖情願,乃取急言歸。



 隋開皇中,秦王俊鎮洛州,召補州主簿。友人蜀
 王府記室范陽盧士彥謂仲舉曰:「丈人往經徵辟,每致推辭,何為徒勞之任,忽爾降德?」仲舉笑曰:「屈伸之事,非子所知。」尋被敕追赴京,朝廷以仲舉婆娑州里,責黜左降為隆州錄事參軍。尋以疾歸,以琴書自娛,優遊賞逸,視人世蔑如也。會朝廷舉士,著作郎王劭又舉以應詔。以前致推遷為責,除冀州清江令,未幾,又以疾還。後以資例,授帥都督、洛陽令。彭城劉逸人謂仲舉曰:「君之才地,遠近所知,久病在家,恐貽時論,具為武職,差若自安。」仲舉曰:「吾性本疏惰,少無宦情,豈以垂老之年,求一階半級?所言武職,挂徐君墓樹耳。」竟不起。終於洛陽永康
 里宅。時年六十三,當世名賢,莫不傷惜之。二子,大師、行師。



 大師字君威,幼而爽悟,神情警發,標格嚴峻,人並敬憚之。身長七尺五寸,風儀甚偉。好學,無所不窺,善綴文。備知前代故事,若指諸掌;商較當世人物,皆得其精。弱冠,州將賀蘭寬召補主簿。寬當時位望,又與大師年事不侔,初見,言未及終,便改容加敬,曰:「名下故無虛士。今者非以相勞,自望坐嘯有託耳。」



 每於私室接遇,恒盡忘年之歡。俄而以資調補左翊衛率,尋除冀州司戶參軍。煬帝初,改州為郡,仍除信都司戶書佐。及大業暮年,王塗弛紊,居官者率多侵漁,皆致潤屋;大師獨守清戒,無
 所營求,家產益致窘迫。郡丞鞠孝稜益相嘆服,曰:「後於歲寒,此言於公得之。」十年,遷渤海郡主簿。及竇建德據有山東,被召為尚書禮部侍郎。武德三年,被遣使京師,因送同安公主,遂求和好。使畢,還至絳州,而建德違約,又助世充抗王師於武牢。高祖大怒,命所在拘留其使。世充、建德尋平,遂以譴徙配西會州。



 大師少時,嘗筮仕長安,遇日者姓史,因使占。時有從兄子同、妹夫鄭師萬、河東裴寂同以宿衛簡入文資。各使視即日官位,及將來所至。史生曰:「裴二及李,皆當依資敘用,然裴君終致台輔。鄭非直今歲虛歸,後歲亦當本資不敘。」指大師曰:「
 君才雖不滅趙元叔,恐賦命亦將同之。」言子同亦無遠到。時大師弟行師亦預賓貢,因問史生吉凶。生曰:「此郎雖非裴君之匹,亦至方伯。」既而大師及子同、裴寂並以資補州佐。師萬當年差舛,明年而齊資不敘。師萬任益州新都縣尉。



 及武德初,裴寂任尚書左僕射、魏國公。大師至是遷播,因獨笑曰:「史生之言,於茲驗矣。」行師貞觀中歷太常寺丞、都水使者、邛州刺史,皆如史生之占。



 大師既至會州,忽忽不樂,乃為《羈思賦》以見其事。侍中、觀公楊恭仁時鎮涼州,見賦異之,召至河西,深相禮重,日與遊處。



 大師少有著述之志,常以宋、齊、梁、陳、魏、齊、周、隋南
 北分隔,南書謂北為「索虜」,北書指南為「島夷」。又各以其本國周悉,書別國並不能備,亦往往失實。常欲改正,將擬《吳越春秋》,編年以備南北。至是無事,而恭仁家富於書籍,得恣意披覽。宋、齊、梁、魏四代有書,自餘竟無所得。居二年,恭仁入為吏部尚書,大師復還會州。武德九年,會赦,歸至京師。尚書右僕射封德彞、中書令房玄齡並與大師親通,勸留不去,曰:「時屬惟新,人思自效,方事屏退,恐失行藏之道。」大師曰:「昔唐堯在上,下有箕山之節,雖以不才,請慕其義。」於是俶裝東歸。家本多書,因編緝前所修書。貞觀二年五月,終於鄭州滎陽縣野舍,時年
 五十九。既所撰未畢,以為沒齒之恨焉。所製文筆詩賦,播遷及遭火,多致失落,存者十卷。子慶孫、正禮、利王、延壽、安世。



 延壽與敬播俱在中書侍郎顏師古、給事中孔穎達下刪削。既家有舊本,思欲追終先志,其齊、梁、陳五代舊事所未見,因於編緝之暇,晝夜抄錄之。至五年,以內憂去職。服闋從官蜀中,以所得者編次之。然尚多所闕,未得及終。十五年,任東宮典膳丞日,右庶子、彭陽公令狐德棻又啟延壽修晉書,因茲復得勘究宋、齊、魏三代之事所未得者。十七年,尚書右僕射褚遂良時以諫議大夫奉敕修《隋書》十志,復準敕召延壽撰錄,因此遍
 得披尋。時五代史既未出,延壽不敢使人抄錄,家素貧罄,又不辦雇人書寫。至於魏、齊、周、隋、宋、齊、梁、陳正史,並手自寫,本紀依司馬遷體,以次連綴之。又從此八代正史外,更勘雜史於正史所無者一千餘卷,皆以編入。其煩冗者,即削去之。始末修撰,凡十六載。始宋,凡八代,為《北史》、《南史》二書,合一百八十卷。其《南史》先寫訖,以呈監國史、國子祭酒令狐德棻,始末蒙讀了,乖失者亦為改正,許令聞奏。次以《北史》咨知,亦為詳正。因遍咨宰相,乃上表。表曰:臣延壽言:臣聞史官之立,其來已舊,執簡記言,必資良直。是以典謨載述,唐、虞之風尤著;《誥誓》斯陳,
 殷、周之烈彌顯。魯書有作,鹿門貽鑒於臧孫;晉乘無隱,桃園取譏於趙孟。斯蓋哲王經國,通賢垂範,懲誡之方,率由茲義。逮秦書既煬,周籍俱湮,子長創制,五三畢紀,條流且異,綱目咸張。自斯以後,皆所取則。雖左史筆削,無乏於時,微婉所傳,唯稱班、范。次有陳壽《國志》,亦曰名家。並已見重前修,無俟揚榷。



 泊紫氣南浮,黃旗東徙,時更五代,年且三百。元熙以前,則總歸諸晉,著述之士,家數雖多,泛而商略,未聞盡善。太宗文皇帝神資睿聖,天縱英靈,爰動沖襟,用紆玄覽,深嗟蕪穢,大存刊勒,既懸諸日星,方傳不朽。然北朝自魏以還,南朝從宋以降,運
 行迭變,時俗污隆,代有載筆,人多好事,考之篇目,史牒不少,互陳聞見,同異甚多。而小說短書,易為湮落,脫或殘滅,求勘無所。一則王道得喪,朝市貿遷,日失其真,晦明安取。二則至人高跡,達士弘規,因此無聞,可為傷歎。三則敗俗巨蠹,滔天桀惡,書法不記,孰為勸獎。



 臣輕生多幸,運奉千齡,從貞觀以來,屢叨史局,不揆愚固,私為修撰。起魏登國元年,盡隋義寧二年,凡三代二百四十四年,兼自東魏天平元年,盡齊隆化二年,又四十四年行事,總編為本紀十二卷、列傳八十八卷,謂之《北史》;又起宋永初元年,盡陳禎明三年,四代一百七十年,為
 本紀十卷、列傳七十卷,謂之《南史》。凡八代,合為二書,一百八十卷,以擬司馬遷《史記》。就此八代,而梁、陳、齊、周、隋五書,是貞觀中敕撰,以十志未奏,本猶未出。然其書及志,始末是臣所修。臣既夙懷慕尚,又備得尋聞,私為抄錄,一十六年,凡所獵略,千有餘卷。連綴改定,止資一手,故淹時序,迄今方就。唯鳩聚遣逸,以廣異聞,編次別代,共為部秩。除其冗長,捃其菁華。若文之所安,則因而不改,不敢茍以下愚,自申管見。雖則疏野,遠慚先哲,於披求所得,竊謂詳盡。其《南史》刊勒已定,《北史》勘校粗了。既撰自私門,不敢寢默,又未經聞奏,亦不敢流傳。輕用陳
 聞,伏深戰越。謹言。



\end{pinyinscope}