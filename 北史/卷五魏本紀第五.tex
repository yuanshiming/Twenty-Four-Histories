\article{卷五魏本紀第五}

\begin{pinyinscope}

 敬宗孝莊皇帝諱子攸,彭城王勰之第三子也。母曰李妃。明帝初,以勰有魯陽翼衛之勳,封帝武城縣公。幼侍明帝書於禁中;及長,風神秀慧,姿貌甚美,雅為明帝親待。孝昌二年八月,進封長樂王,歷位侍中、中軍將軍。以兄彭城王劭事,轉為衛將軍、左光祿大夫、中書監,實見出也。武泰元年二月,明帝崩。大都督爾朱榮謀廢立。以帝家有忠勛,且兼人望,陰與帝通,率眾來赴。帝與兄弟
 夜北度河,會榮於河陽。



 永安元年夏四月戊戌,帝南濟河,即皇帝位。以皇兄彭城王劭為無上王,皇弟霸城公子正為始平王。以爾朱榮為使持節、侍中、都督中外諸軍事、大將軍、尚書令、領軍將軍、領左右,封太原王。己亥,百僚相率,有司奉璽綬,備法駕,奉迎於河梁。西至陶渚,榮以兵權在己,遂有異志。乃害靈太后及幼主,次害無上王劭、始平王子正,又害丞相、高陽王雍已下王公卿士二千人,列騎衛帝,遷於便幕。榮尋悔,稽顙謝罪。



 辛丑,車駕入宮,御太極殿。大赦,改武泰為建義元年。壬寅,榮表請追謚無上王為皇
 帝。餘死河陰者,諸王、刺史贈三司,三品者令僕,五品者刺史,七品已下及庶人,郡、鎮。諸死者子孫,聽立後,授封爵。詔從之。癸卯,以前太尉、江陽王繼為太師、司州牧。相州刺史、北海王顥為太傅、開府,仍刺史。封光祿大夫、清泉縣侯李延實為陽平王,位太保,遷太傅;以並州刺史元天穆為太尉,封上黨王。



 以儀同三司楊椿為司徒。以儀同三司、頓丘郡公穆紹為司空,領尚書令,進爵為王。



 以雍州刺史長孫承業為開府儀同三司,進封馮翊王。以殿中尚書元諶為尚書右僕射,封魏郡王。以給事黃門侍郎元填為東海王。甲辰,以敷城王坦為咸陽王。
 以諫議大夫元貴平為東萊王。以直閣將軍元肅為魯郡王。以祕書郎中元曄為長廣王。以馮翊郡公源紹景復先爵隴西王。扶風郡公馮冏、東郡公陸子彰、北平公長孫悅並復先王爵。以北平王超還復為安定王。丁未,詔中外解嚴。庚戌,封大將軍爾朱榮次子義羅為梁郡王。詔蠕蠕王阿那瑰贊拜不名,上書不稱臣。是月,汝南王悅、北海王顥、臨淮王彧前後奔梁。五月丁巳朔,以右僕射元羅為東道大使,光祿勛元欣副之。循方黜陟,先行後聞。辛酉,大將軍爾朱榮還晉陽,帝餞於邙陰。六月癸卯,以高昌王世子光為平西將軍、瓜州刺史,襲爵泰
 臨縣伯、高昌王。帝以寇難未夷,避正殿,責躬撤膳。又班募格,收集忠勇。有直言正諫之士者,集華林園,面論時事。幽州平北府主簿河間邢杲率河北流移人萬餘戶,反於北海,自署漢王,年號天統。秋七月乙丑,加大將軍爾朱榮柱國大將軍、錄尚書事。壬子,光州人劉舉聚眾反於濮陽,自稱皇武大將。是月,高平鎮人萬俟醜奴僭稱大位。臨淮王彧自江南還朝。八月,太山太守羊侃據郡反。甲辰,詔大都督宗正珍孫討劉舉。平之。九月己巳,以齊州刺史元欣為沛郡王。壬申,柱國大將軍爾朱榮率騎七千討葛榮於滏口,破禽之。冀、定、滄、瀛、殷五州平。
 乙亥,以葛榮平,大赦,改元為永安。辛巳,以爾朱榮為大丞相,進榮子平昌郡公文殊、昌樂郡公文暢爵並為王。以司徒楊椿為太保,城陽王徽為司徒。冬十月丁亥,爾朱榮檻送葛榮於京師。帝臨閶闔門,榮稽顙謝罪,斬於都市。戊戌,江陽王繼薨。癸丑,復膠東縣侯李侃希祖爵南郡王。是月,大都督費穆大破梁軍。禽其將曹義宗,檻送京師。梁以北海王顥為魏主,年號孝基,入據南兗之銍城。十一月戊午,以無上王世子韶為彭城王;陳留王子寬為陳留王;寬弟剛為浮陽王;剛弟質為林慮王。癸亥,行臺于暉等大破羊侃於瑕丘。侃奔梁。戊寅,封前軍
 元凝為東安王。是歲,葛榮餘黨韓樓復據幽州反。



 二年春二月甲午,追尊皇考為文穆皇帝,廟號肅祖。皇妣為文穆皇后。夏四月癸未,遷文穆皇帝及文穆皇后神主于太廟,降畿內死罪已下刑。辛丑,上黨王天穆大破邢杲於濟南。杲降,送於京師,斬於都市。五月壬子朔,元顥剋梁國。乙丑,內外戒嚴。癸酉,元顥陷滎陽。甲戌夜,車駕北巡。乙亥,幸河內。丙子,元顥入洛。丁丑,進封城陽縣公元祉為平原王;安昌縣公元鷙為華山王。戊寅,太原王爾朱榮會車駕於長子,即日反旆。上黨王天穆北度,會車駕於河內。秋七月戊辰,都督爾朱兆、賀拔勝從
 硤石夜濟。破顥子冠受及安豐王延明軍。元顥敗走。庚午,車駕入居華林園,升大夏門大赦。壬申,以柱國大將軍、太原王爾朱榮為天柱大將軍。



 癸酉,臨潁縣卒江豐斬元顥,傳首京師。甲戌,以大將軍、上黨王天穆為太宰,以司徒、城陽王徽為大司馬、太尉。己卯,以南青州刺史元旭為襄城王;南兗州刺史元暹為汝陽王。閏月辛巳,帝始居宮內。辛卯,以兼吏部尚書楊津為司空。八月己未,以太傅李延實為司徒。丁卯,封瓜州刺史元太宗為東陽王。九月,大都督侯深破韓樓於薊,斬之。幽州平。冬十月己酉朔,日有蝕之。丁丑,以前司空、丹楊王蕭贊為司徒。十一月己卯,就德興自榮州遣使請降。丙午,以大
 司馬、太尉、城陽王徽為太保,以司徒、丹楊王蕭贊為太尉,以雍州刺史長孫承業為司徒。



 三年夏四月丁卯,雍州刺史爾朱天光討萬俟醜奴、蕭寶夤於安定。破禽之,囚送京師。甲戌,以關中平,大赦。斬醜奴於都市,賜寶夤死。六月戊午,嚈噠國獻師子一。是月,白馬龍泗胡王慶雲僭稱帝號於永洛城。秋七月丙子,爾朱天光平水洛城,禽慶雲。九月辛卯,天柱大將軍爾朱榮、上黨王天穆自晉陽來朝。戊戌,帝殺榮、天穆於明光殿,及榮子菩提。乃升閶闔門,大赦。遣武衛將軍奚毅、前燕州刺史侯深率眾鎮北中。是夜,左僕射爾朱世
 隆、榮妻鄉郡長公主率榮部曲,自西陽門出屯河陰。己亥,攻河橋,禽毅等,屠害之。據北中城,南逼京師。冬十月癸卯朔,封大鴻臚卿寶炬為南陽王,汝陽縣公脩為平陽王,新陽伯誕為昌樂王,瑯邪公昶為太原王。甲辰,徙封魏郡王諶為趙郡王,諶弟子趙郡王宣為平昌王。戊申,皇子生,大赦。乙卯,通直散騎常侍李苗以火船焚河橋,爾朱世隆退走。壬申,世隆停建興之高都,爾硃兆自晉陽來會之,共推長廣王曄為主。大赦所部,年號建明。



 徐州刺史爾朱仲遠反,率眾向京師。十一月乙亥,以司徒長孫承業為太尉,以臨淮王彧為司徒。丙子,進雍州
 刺史、廣宗郡公爾朱天光爵為王。十二月甲辰,爾朱兆、爾朱度律自富平津上,率騎涉度以襲京城。事出倉卒,禁衛不守。帝步出雲龍門。



 兆逼帝幸永寧寺,殺皇子。亂兵殺司徒臨淮王彧、左僕射范陽王誨。戊申,爾朱度律自鎮京師。甲寅,爾朱兆遷帝於晉陽。甲子,帝遇弒於城內三級佛寺,時年二十四。並害陳留王寬。中興二年,廢帝奉謚為武懷皇帝。及孝武立,又以廟諱故,改謚孝莊皇帝,廟號敬宗。葬靜陵。



 節閔皇帝諱恭,字修業,廣陵惠王羽之子也。母曰王氏。帝少有志度,事祖母、嫡母以孝聞。正始中,襲爵。位給事
 黃門侍郎。帝以元叉擅權,託稱喑病,絕言垂一紀。居於龍花佛寺,無所交通。永安末,有白莊帝,言帝不語,將有異圖。人間遊聲,又云常有天子氣。帝懼禍,遂逃匿上洛。尋見追躡,送京師,拘禁多日,以無狀獲免。及莊帝崩,爾朱世隆等以元曄疏遠,又非人望所推,以帝有過人之量,將謀廢立。恐實不語,乃令帝所親申意,兼迫脅。帝曰:「天何言哉。」世隆等大悅。及元曄至邙南,世隆等奉帝東郭外,行禪讓禮。太尉爾朱度律奉路車,進璽紱。



 服袞冕,百官侍衛,入自建春、雲龍門。



 普泰元年春二月己巳,皇帝即位於太極前殿,群臣拜
 賀。禮華,遂登閶闔門大赦。以魏為大魏。改建明二年為普泰元年。罷稅市及稅鹽之官。庚午,詔曰:「自秦之末競為皇帝,忘負乘之深殃,垂貪鄙於萬葉。予今稱帝,已為褒矣!可普告令知。」是月,鎮遠將軍清河崔祖螭聚青州七郡之眾圍東陽。幽州刺史劉靈助起兵於薊。河北大使高乾及其弟昂夜襲冀州。執刺史元嶷,共推前河內太守封隆之行州事。



 三月癸酉,封長廣王曄為東海王。以青州刺史、魯郡王肅為太師。沛郡王欣為太傅、司州牧,改封淮陽王。以徐州刺史彭城王爾朱仲遠、雍州刺史隴西王爾朱天光並為大將軍。以柱國大將軍。並州
 刺史、潁川王爾朱兆為天柱大將軍。封晉州刺史、平陽郡公高歡為勃海王。以特進、清河王亶為太傅。以尚書令、樂平王爾朱世隆為太保。以趙郡王諶為司空。丙申,定州刺史侯深破劉靈助於安國城。斬之,傳首京師。



 夏四月壬子,享太廟。癸亥,隴西王爾朱天光破宿勤明達。禽送京師,斬之。丙寅,以侍中爾朱彥伯為司徒。詔有司不得復稱偽梁。罷細作之條,無禁鄰國還往。五月丙子,爾朱仲遠使其都督魏僧勖等討崔祖螭於東陽,斬之。六月己亥朔,日有蝕之。



 庚申,勃海王高歡起兵信都,以誅爾朱氏為名。秋七月壬申,爾朱世隆等害前太保楊
 椿、前司空楊津。丙戌,司徒爾朱彥伯以旱遜位。九月,以彭城王爾朱仲遠為太宰。庚辰,以隴西王爾朱天光為大司馬。癸巳,追尊皇考為先帝,皇妣王氏為先太妃。封皇弟永業為高密王,皇子子恕為勃海王。冬十月壬寅,高歡推勃海太守元朗即皇帝位於信都。



 二年春閏二月,高歡敗爾朱天光等於韓陵。夏四月辛巳,高歡與廢帝至芒山。



 使魏蘭根慰喻洛邑,且觀帝之為人。蘭根忌帝雅德,還致毀謗,竟從崔陵議,廢帝於崇訓佛寺。而立平陽王修,是為孝武帝。帝既失位,乃賦詩曰:「朱門久可患,紫極非情玩。顛覆立可待,一年三易換。
 時運正如此,唯有修真觀。」五月丙申,帝遇弒,殂於門下外省,時年三十五。孝武帝詔百司赴會,葬用王禮。加九旒、鑾輅、黃屋、左纛,班劍百二十人。後西魏追謚節閔皇帝。



 廢帝諱朗字仲哲,章武王融第三子也。母曰程氏。帝少稱明悟。元曄建明二年正月戊子,為勃海太守。普泰元年十月,勃海王高歡奉帝以主號令。



 中興元年冬十月壬寅,皇帝即位於信都西。大赦,改普泰元年為中興。以勃海王高歡為丞相,都督中外諸軍事。以河北大使高乾為司空。辛亥,高歡大破爾朱兆於
 廣阿。十一月,梁將元樹入據譙城。



 二年春二月甲子,以勃海王高歡為大丞相、柱國大將軍、太師。及歡敗爾朱氏於韓陵,四月辛巳,帝於河陽遜位於別邸。五月,孝武封帝為安定郡王。十一月,殂於門下外省。時年二十。永熙二年,葬於鄴西南野馬崗。



 孝武皇帝諱修,字孝則,廣平武穆王懷之第三子也。母曰李氏。帝性沈厚,學涉,好武事,遍體有鱗文。年十八,封汝陽縣公。夢人有從諱謂己曰:「汝當大貴,得二十五年。」永安三年,封平陽王。普泰中,為侍中、尚書左僕射。



 中興二年,高歡既敗爾朱氏,廢帝自以疏遠,請遜大位。歡乃
 與百寮議。以孝文不可無後,時召汝南王悅於梁。至,將立之,宿昔而止。又諸王皆逃匿,帝在田舍。先是,嵩山道士潘彌望見洛陽城西有天子氣,候之乃帝也。於是造第密言之。



 居五旬而高歡使斛斯椿求帝。椿從帝所親王思政見帝。帝變色曰:「非賣我耶?」



 椿遂以白歡。歡遣四百騎奉迎帝入氈帳,陳誠,泣下霑襟。讓以寡德。歡再拜,帝亦拜。歡出,備服御,進湯沐。達夜嚴警。昧爽,文武執鞭以朝。使斛斯椿奉勸進表。椿入帷門,罄折延首而不敢前。帝令思政取表,曰:「視,便不得不稱朕矣。」



 於是假廢帝安定王詔策而禪位焉。即位于東郭之外。用代都舊制,
 以黑氈蒙七人,歡居其一。帝於氈上西向拜天訖,自東陽、雲龍門入。



 永熙元年夏四月戊子,皇帝御太極前殿,群臣朝賀。禮畢,升閶闔門大赦。改中興二年為太昌元年。壬辰,高歡還鄴。五月丙申,節閔帝殂。以太傅、淮陽王欣為太師,改封沛郡王。以司徒、趙郡王諶為太保。以司空、南陽王寶炬為太尉。以太保長孫承業為太傅。辛丑,復前司空高乾位。己酉,以儀同三司、清河王亶為司徒。乙卯,內外解嚴。六月癸亥朔,帝於華林園納訟。丁卯,南陽王寶炬坐事,降為驃騎大將軍,開府,以王歸第。己卯,臨顯陽殿納
 訟。丙戌,詔曰:「間者,凶權誕恣,法令變常,遂立夷貊輕賦,冀收天下之意。隨以箕斂之重,終納十倍之徵,掩目捕雀,何能過此!今歲租調,且兩收一丐,明年復舊。」秋七月庚子,以南陽王寶炬為太尉。乙卯,帝臨顯陽殿,親理冤獄。是月,東南道大行臺樊子鵠大破梁軍於譙城,禽其將元樹。八月丁卯,封西中郎將元寧為高平王。九月癸卯,進燕郡公賀拔允爵為王。癸丑,改封沛郡王欣為廣陵王;節閔子勃海王子恕為沛郡王。冬十月辛酉朔,日有蝕之。十一月丁酉,祀圓丘。甲辰,殺安定王朗及東海王曄。己酉,以汝南王悅為侍中、大司馬,開府。葬太后胡
 氏。十二月丁亥,殺大司馬、汝南王悅。大赦,改元為永興。以同明元時年號,尋改為永熙。是歲,蠕蠕、嚈噠、高麗、契丹、庫莫奚、高昌等國並遣使朝貢。



 二年春正月庚寅朔,朝饗群臣于太極前殿。丁酉,勃海王高歡大敗爾朱氏,山東平。罷諸行臺。丁巳,追尊皇考為武穆皇帝,太妃馮氏為武穆皇后,皇妣李氏曰皇太妃。二月,以咸陽王坦為司空。三月甲午,太師、魯郡王肅薨。丁巳,以太保、趙郡王諶為太尉。以太尉、南陽王寶炬為尚書令、太保,開府。是月,阿至羅十萬戶內附。詔復以勃海王高歡為大行臺,隨機裁處。夏四月己未朔,日有
 蝕之。秋七月壬辰,以太師、廣陵王欣為大司馬,以太尉、趙郡王諶為太師,並開府。庚戌,以前司徒、燕郡王賀拔允為太尉。冬十月癸未,以衛將軍、瓜州刺史、泰臨縣伯、高昌王麴子堅為儀同三司,進爵郡公。十二月丁巳,狩於嵩陽,士卒寒苦。己巳,遂幸溫湯。丁丑,還宮。



 三年春二月壬戌,大赦。壬午,封左衛將軍元斌之為潁川王。夏四月癸丑朔,日有蝕之。辛未,高平王寧坐事降爵為公。五月丙戌,置勛府庶子,箱別六百人;騎官,箱別二百人;閣內部曲,數千人。帝內圖高歡,乃以斛斯椿為領軍,使與王思政等統之,以為心膂。軍謀朝政,咸決於
 椿。分置督將及河南、關西諸刺史。辛卯,下詔戒嚴,楊聲伐梁,實謀北討。是夏,契丹、高麗、吐谷渾並遣使朝貢。秋七月己丑,帝親總六軍十餘萬,次河橋。高歡引軍東度。丙午,帝率南陽王寶炬、清河王亶、廣陽王湛、斛斯椿以五千騎宿於瀍西楊王別舍。沙門都維那惠臻負璽持千牛刀以從。有牛百頭,盡殺以食軍士。眾知帝將出,其夜亡者過半。清河、廣陽二王亦逃歸。略陽公宇文泰遣都督駱超、李賢和各領數百騎赴。駱超先至。甲戌,賢和會帝於崤中。己酉,高歡入洛,遣婁昭及河南尹元子思領左右侍官追帝,請迴駕。高昂率勁騎及帝於陜西。帝
 鞭馬長騖至湖城,饑渴甚,有王思村人以麥飯壺漿獻帝。帝甘之,復一村十年。是歲二月,熒惑入南斗,眾星北流,群鼠浮河向鄴。



 梁武跣而下殿,以禳星變。及聞帝之西,慚曰:「虜亦應天乎?」帝至稠桑,潼關大都督毛洪賓迎獻食。八月,宇文泰遣大都督趙貴、梁禦甲騎二千來赴,乃奉迎。



 帝過河謂禦曰:「此水東流而朕西上,若得重謁洛陽廟,是卿等功也。」帝及左右皆流涕。宇文泰迎帝於東陽,帝勞之,將士皆呼萬歲。遂入長安。以雍州公廨為宮,大赦。甲寅,高歡推司徒、清河王亶為大司馬,承制總萬機,居尚書省。歡追車駕至潼關。九月己酉,歡東還洛
 陽。帝親督眾攻潼關,斬其行臺華長瑜,又剋華州。



 其冬十月,高歡推清河王亶子善見為主,徙都鄴,是為東魏。魏於此始分為二。帝之在洛也,從妹不嫁者三:一曰平原公主明月,南陽王同產也;二曰安德公主,清河王懌女也;三曰蒺藜,亦封公主。帝內宴,命諸婦人詠詩。或詠鮑照樂府曰:「朱門九重門九閨,願逐明月入君懷。」帝既以明月入關。蒺藜自縊。宇文泰使元氏諸王取明月殺之。帝不悅,或時彎弓,或時推案,君臣由此不安平。閏十二月癸巳,潘彌奏言:「今日當慎有急兵。」其夜,帝在逍遙園宴阿至羅,顧侍臣曰:「此處仿佛華林園,使人聊增悽
 怨。」命取所乘波斯騮馬,使南陽王躍之。將攀鞍,蹶而死,帝惡之。日晏還宮,至後門,馬驚不前,鞭打入。謂潘彌曰:「今日幸無他不?」彌曰:「過夜半則大吉,」須臾,帝飲酒,遇鴆而崩,時年二十五。謚曰孝武。殯於草堂佛寺。十餘年乃葬雲陵。始宣武、孝明間謠曰:「狐非狐,貉非貉,焦梨狗子嚙斷索。」識者以為索謂本索髮,焦梨狗子指宇文泰,俗謂之黑獺也。



 文皇帝諱寶炬,孝文皇帝之孫,京兆王愉之子也。母曰楊氏。帝正始初坐父愉罪,兄弟皆幽宗正寺。及宣武崩,乃得雪。正光中,拜直閣將軍。時胡太后多嬖寵,帝與明
 帝謀誅之。事泄,免官。武泰中,封邵縣侯。永安三年,進封南陽王。孝武即位,拜太尉,加侍中。永熙二年,進位太保、開府、尚書令。三年,孝武與高歡構難,以帝為中軍四面大都督。及從入關,拜太宰、錄尚書事。孝武崩,丞相、略陽公宇文泰率群公卿士奉表勸進,三讓乃許焉。



 大統元年春正月戊申,皇帝即位於城西,大赦,改元。追尊皇考為文景皇帝,皇妣楊氏為皇后。己酉,進丞相、略陽公宇文泰都督中外諸軍、錄尚書事、大行臺,改封安定郡公。以尚書令斛斯椿為太保,廣平王贊為司徒。乙卯,立妃乙氏為皇后,立皇子欽為皇太子。甲子,以廣陵
 王欣為太傅,以儀同三司萬俟壽樂乾為司空。東魏將侯景攻陷荊州。二月,前南青州刺史大野拔斬兗州刺史樊子鵠,以州降東魏。



 夏五月,降罪人。加安定公宇文泰位柱國。秋七月,以開府儀同三司念賢為太尉,以司空萬俟壽樂乾為司徒;以開府儀同三司越勒肱為司空。梁州刺史元羅以州降梁。



 九月,有司詔煎御香澤,須錢萬貫。帝以軍旅在外,停之。冬十月,太師、上黨王長孫承業薨。十二月,以太尉念賢為太傅,以河州刺史梁景睿為太尉。



 二年春正月辛亥,祀南郊,改以神元皇帝配。東魏攻陷
 夏州。二月,儀同三司段敬討叛羌梁定平之。三月,以涼州刺史李叔仁為司徒,以司徒萬俟壽樂乾為太宰。夏五月,司空越勒肱薨。秦州刺史、建忠王萬俟普撥及其子太宰壽樂干率所部奔東魏。秋九月,以扶風王孚為司空,以太保斛斯椿為太傅。冬十一月,追改始祖神元皇帝為太祖,道武皇帝為烈祖。是歲,關中大饑,人相食,死者十七八。



 三年春二月,槐里獲神璽,大赦。夏四月,太傅斛斯椿薨。五月,以廣陵王欣為太宰,賀拔勝為太師。六月,以司空、扶風王孚為太保,以太尉梁景睿為太傅,以司徒、廣平
 王贊為太尉,以開府儀同三司王盟為司空。冬十月,安定公宇文泰大破東魏軍於沙苑,拜泰柱國大將軍。十二月,司徒李叔仁自涼州通使於東魏,建昌太守賀蘭植攻斬之。



 四年春正月辛酉,拜天於清暉室,終帝世遂為常。二月,東魏攻陷南汾、潁、豫、廣四州。廢皇后乙氏。三月,立蠕蠕女郁久閭氏為皇后,大赦。以司空王盟為司徒。秋七月,東魏將侯景等圍洛陽,帝與安定公宇文泰東伐。九月,車駕至自東伐。以撫軍將軍梁定為南洮州刺史,安西蕃。



 五年春二月,赦京城內。夏五月,以開府儀同三司李弼為司空。免妓樂雜役之徒,皆從編戶。秋七月,詔自今恆以朔望親閱京師見禁囚徒。以司空、扶風王孚為太尉。冬十月,於陽武門外縣鼓,置紙筆,以求得失。



 六年春正月庚戌,朝群臣。自西遷至此,禮樂始備。太尉、扶風王孚薨。二月,鑄五銖錢。降罪人。冬十一月,太師念賢薨。



 七年春二月,幽州刺史、順陽王仲景以罪賜死。三月,夏州刺史劉平謀反,大都督于謹討禽之。秋九月,詔班政事之法六條。冬十一月,叛羌梁定徒黨屯於赤水城,
 秦州刺史獨孤信擊平之。尚書奏班十二條制。十二月,御憑雲觀。引見諸王,敘家人之禮。手詔為宗誡十條以賜之。



 八年春三月,初置六軍。夏四月,鄯善王兄鄯朱那率眾內附。秋八月,以太尉王盟為太保。冬十月,詔皇太子鎮河東。十二月,行幸華州,起萬壽殿於沙苑北。



 九年春正月,降罪人。禁中外及從母兄弟姊妹為婚。閏月,車駕至自華州。二月,東魏北豫州刺史高仲密據武牢內附,以仲密為侍中、司徒,封勃海郡公。秋七月,大赦。以太保王盟為太傅,以太尉、廣平王贊為司空。冬十二
 月,以司空李弼為太尉。



 十年春正月甲子,詔公卿已下,每月上封事三條,極言得失。刺史二千石銅墨已上,有讜言嘉謀,勿有所諱。夏五月,太師賀拔勝薨。秋七月,更權衡度量。



 十一年夏五月,太傳王盟薨。詔諸鞫大辟獄,皆命三公覆審,然後加刑。冬,始築圓丘於城南。封皇子儉。



 十二年春二月,涼州刺史宇文仲和反,秦州刺史獨孤信討平之。三月,鑄五銖錢。夏五月,詔女年不滿十三以上,勿得以嫁。秋九月,東魏勃海王高歡攻玉壁,晉州刺史韋孝寬力戰禦之。冬十二月,歡燒營而退。



 十三年春正月,開白渠以溉田。二月,詔自今應宮刑者,直沒官,勿刑。亡奴婢應黥者,止科亡罪。以開府儀同三司若干惠為司空。東魏勃海王歡薨高。其司徒侯景據潁川率河南六州內附。授景太傅、河南大行臺、上谷郡公。三月,大赦。夏五月,以太傅侯景為大將軍,以開府儀同三司獨孤信為大司馬。晉王謹薨。秋七月,司空若干惠薨。大將軍侯景據豫州叛。封皇子寧為趙王。



 十四年春正月,赦潁、豫、廣、北、洛、東荊、襄等七州。以開府儀同三司趙貴為司空。皇孫生,大赦。夏五月,以安定公宇文泰為太師,廣陵王欣為太傅,太尉李弼為大宗伯,
 前太尉趙貴為大司寇,以司空于謹為大司空。



 十五年己巳五月,侯景殺梁武帝。初,詔諸代人太和中改姓者,並令復舊。六月,東魏勃海王高澄攻陷潁川。秋八月,盜殺東魏勃海王高澄。冬十二月,封梁雍州剌史、岳陽王蕭察為梁王。



 十六年夏四月,封皇子儒為燕王,公為吳王。五月,東魏靜帝遜位于齊。秋七月,安定公宇文泰東伐,至恆農。齊師不出,乃還。九月,大赦。



 十七年春三月庚戌,帝崩於乾安殿,時年四十五。夏四月庚辰,葬於永陵,上謚曰文皇帝。



 帝性強果,始為太尉
 時,侍中高隆之恃勃海王高歡之黨,驕狎公卿。因公會,帝勸酒不飲,怒而毆之。罵曰:「鎮兵,何敢爾也!」孝武以歡故,免帝太尉。歸第,命羽林守衛,月餘復位。及歡將改葬其父,朝廷追贈太師,百僚會弔者盡拜。



 帝獨不屈,曰:「安有生三公而拜贈太師耶!」及躋大位,權歸周室。嘗登逍遙觀望嵯峨山,因謂左右曰:「望此,令人有脫屣之意。若使朕年五十,便委政儲宮,尋山餌藥,不能一日萬機也。」既而大運未終,竟保天祿云。



 廢帝諱欽,文皇帝之長子也。母曰乙皇后。大統元年正月乙卯,立為皇太子。



 十七年三月,即皇帝位。是月,梁邵
 陵王蕭綸侵安陸,大將軍楊忠討禽之。



 元年冬十一月,梁湘東王蕭繹討侯景,禽之。遣其舍人魏彥來告,仍嗣位於江陵。



 二年秋八月,大將軍尉遲迥剋成都,劍南平。冬十一月,安定公宇文泰殺尚書元烈。



 三年春正月,安定公宇文泰廢帝而立齊王廓。帝自元烈之誅,有怨言。淮安王育、廣平王贊等並垂泣諫,帝不聽,故及於辱。



 恭皇帝諱廓,文皇之第四子也。大統十四年,封為齊王。廢帝三年正月,即皇帝位,改元。



 元年夏四月,蠕蠕乙旃達官寇廣武。五月,柱國李弼追擊之,斬首數千級,收輜重而還。冬十一月,魏師滅梁,戕梁元帝。梁太尉王僧辯奉元帝子方智為王,承制,居建業。



 二年秋七月,梁太尉王僧辯納貞陽侯蕭明於齊,奉以為主。梁王方智為太子。



 九月,梁司空陳霸先殺僧辯,廢蕭明,復奉方智為帝。是歲,梁廣州刺史王琳寇邊,大將軍豆盧寧帥師討之。



 三年春正月丁丑,初行《周禮》,建六官,以安定公宇文泰為太師、塚宰;以柱國李弼為大司徒;趙貴為太保、大宗
 伯;以尚書令獨孤信為大司馬;以于謹為大司寇;以侯莫陳崇為大司空。冬十月乙亥,安定公宇文泰薨。十二月庚子,帝遜位於周。周閔帝元年正月,封帝為宋公,尋殂。



 東魏孝靜皇帝諱善見,清河文宣王亶之世子也。母曰胡妃。永熙三年八月,拜開府儀同三司。孝武帝既入關,勃海王高歡乃與百僚會議,推帝以奉明帝之後,時年十一。



 天平元年冬十月丙寅,皇帝即位于城東北。大赦,改元。庚午,以太師、趙郡王諶為大司馬;以司空、咸陽王坦為
 太尉;以開府儀同三司高盛為司徒;以開府儀同三司高昂為司空。壬申,享太廟。丙子,車駕北遷于鄴。詔勃海王高歡留後部分。



 改司州為洛州。以尚書令元弼為儀同三司、洛州刺史,鎮洛陽。十一月兗州刺史樊子鵠、南青州刺史大野拔據瑕丘反。庚寅,車駕至鄴,居北城相州之廨。改相州刺史為司州牧,魏郡太守為魏尹。徙鄴舊人西徑百里,以居新遷人。分鄴置臨漳縣。



 以魏郡、林慮、廣平、陽丘、汲郡、黎陽、東濮陽、清河、廣宗等郡為皇畿。十二月丁卯,燕郡王賀拔允薨。庚午,詔內外戒嚴,百司悉依舊章,從容雅服,不得以務衫從事。丙子,進侍中封
 隆之等五人為大使,巡喻天下。丁丑,赦畿內。閏月,梁以元慶和為魏王,入據平瀨鄉。孝武崩于長安。初置四中郎將,於礓石橋置東中,蒲泉置西中,濟北置南中,洺水置北中。



 二年春正月乙亥,兼尚書右僕射、東南道行臺元晏討元慶和,破走之。二月壬午,以太尉、咸陽王坦為太傅,以司州牧、西河王心妻為太尉。己丑,前南青州刺史大野拔斬樊子鵠以降,兗州平。戊戌,梁司州刺史陳慶之寇豫州,刺史堯雄擊走之。三月辛酉,以司徒高盛為太尉,以司空高昂為司徒,濟陰王暉業為司空。勃海王高歡討
 平山胡劉蠡升。辛未,以旱故,詔京邑及諸州郡縣收瘞骸骨。是春,高麗、契丹並遣使朝貢。夏四月,前青州刺史侯梁反,攻掠青、齊。癸未,濟州刺史蔡俊討平之。壬辰,降京師見囚。夏五月,大旱。勒城門殿門及省府寺署坊門澆人,不簡王公,無限日,得雨乃止。六月,元慶和寇南頓,豫州刺史堯雄大破之。秋七月甲戌,封汝南王悅孫綽為瑯邪王。八月辛卯,司空、濟陰王暉業坐事免。甲午,發眾七萬六千人營新宮。九月丁巳,以開府儀同三司、襄城王旭為司空。冬十一月丁未,梁柳仲禮寇荊州,刺史王元擊破之。癸丑,祀圓丘。甲寅,閶闔門災。龍見並州人家
 井中。十二月壬午,車駕狩於鄴東。甲午,文武百官量事各給祿。是歲,西魏文帝大統元年也。



 三年春正月癸卯朔,饗群臣於前殿。戊申,詔百官舉士。舉不稱才者,兩免之。



 二月丁未,梁光州刺史郝樹以州內附。丁酉,加勃海王世子澄為尚書令、大行臺、大都督。三月甲寅,以開府儀同三司、華山王鷙為大司馬。丁卯,陽夏太守盧公纂據郡南叛,大都督元整破之。夏四月丁酉,昌樂王誕薨。五月癸卯,賜鰥寡孤獨貧窮者衣物各有差。丙辰,以錄尚書事、西河王心妻為司州牧。戊辰,太尉高盛薨。



 六月辛巳。趙郡王諶薨。秋七月庚子,大赦。梁
 夏州刺史田獨鞞、潁川防城都督劉鸞慶並以州內附。八月,並、肆、涿、建四州霜隕,大饑。九月壬寅,以定州刺史侯景兼尚書右僕射、南道行臺,節度諸軍南討。丙辰,平陽人路季禮聚眾反。辛酉,御史中尉竇泰討平之。。冬十一月戊申,詔遣使巡檢河北流移飢人。侯景攻剋梁楚州,獲刺史桓和。十二月,以並州刺史尉景為太保。辛未,遣使者板假老人官,百歲已下,各有差。壬申,大司馬、清河王亶薨。癸未,以太傅、咸陽王坦為太師。



 是歲,高麗、勿吉並遣使朝貢。



 四年春正月,以汝陽王暹為錄尚書事。夏四月辛未,遷
 七帝神主入新廟。大赦,內外百官普進一階。先是,滎陽人張儉等聚眾反於大騩山,通西魏。壬辰,武衛將軍高元咸討破之。六月己巳,幸華林園理訟。辛未,詔尚書掩骼埋胔,推錄囚徒。



 壬午,閶闔門災。秋七月甲辰,遣兼散騎常侍李楷聘於梁。八月,西魏剋陜州,刺史李徽伯死之。九月,侍中元子思與其弟子華謀西入,並賜死。閏月乙丑,衛將軍、右光祿大夫蔣天樂謀反,伏誅。禁京師酤酒。冬十月,以咸陽王坦為錄尚書事。壬辰,勃海王高歡西討,敗于沙苑。己酉,西魏行臺宮景壽、都督楊白駒寇洛州,大都督韓賢大破之。西魏又遣其大行臺元季海、
 大都督獨孤信逼洛州,刺史廣陽王湛棄城歸闕,季海、信遂據金墉。十一月丙子,以驃騎大將軍、儀同三司萬俟普為太尉。十二月甲寅,梁人來聘。河間人邢磨納、范陽人盧仲禮等各聚眾反。是歲,高麗、蠕蠕並遣使朝貢。



 元象元年春正月辛酉朔,日有蝕之。有巨象自至碭郡陂中,南兗州獲送于鄴。



 丁卯,大赦,改元。二月丙辰,遣兼散騎常侍鄭伯猷聘于梁。夏四月庚寅,曲赦畿內,開酒禁。六月壬辰,帝幸華林都堂,聽訟。是夏,山東大水,蝦蟆鳴於樹上。



 秋七月乙亥,高麗遣使朝貢。八月辛卯,大敗西魏于河陰。九月,大都督賀拔仁擊邢磨納、盧仲禮
 等破平之。冬十月,梁人來聘。十二月庚寅,遣陸操聘于梁。



 興和元年春正月辛酉,以尚書令孫騰為司徒。三月甲寅朔,封常山郡王第二子曜為陳郡王。五月甲戌,立皇后高氏。乙亥,大赦。是月,高麗遣使朝貢。六月乙酉,以尚書左僕射司馬子如為山東黜陟大使,尋為東北道行臺,差選勇士。庚寅,前潁州刺史奚思業為河南大使,簡發勇士。丁酉,梁人來聘。戊申,開府儀同三司、汝陽王暹薨。秋八月壬辰,遣兼散騎常侍王元景聘于梁。九月甲子,發畿內十萬人城鄴,四十日罷。辛未,曲赦畿內死
 罪已下,各有差。冬十一月癸亥,以新宮成,大赦,改元。八十已上賜綾帽及杖。七十旁無期親及有疾廢者,各賜粟帛。築城之夫,給復一年。



 二年春正月壬申,以太保尉景為太傅,以驃騎大將軍、開府儀同三司厙狄乾為太保。丁丑,庫御新宮,大赦。內外百官普進一階,營構主將別優一階。三月乙卯,梁人來聘。夏五月己酉,西魏行臺宮延和、陜州剌史宮元慶率戶內屬,置之河北馬場,振廩各有差。壬子,遣兼散騎常侍李象聘于梁。閏月丁丑朔,日有蝕之。己丑,封皇兄景植為宜陽王,皇弟威為清河王。謙為潁川王。六月壬
 子,大司馬、華山王鷙薨。冬十月丁未,梁人來聘。十二月乙卯,遣兼散騎常侍崔長謙聘於梁。是歲,高麗、蠕蠕、勿吉並遣使朝貢。



 三年春二月甲辰,阿至羅出吐拔那渾大率部來降。三月乙酉,梁州人公孫貴賓聚眾反,自號天王,陽夏鎮將討禽之。夏四月戊申,阿至羅國主副伏羅越君子去賓來降,封為高車王。六月乙丑,梁人來聘。秋七月己卯,宜陽王景植薨。八月甲子,遣兼散騎常侍李騫聘於梁。先是,詔群官於麟趾閣;議定新制,冬十月甲寅,班於天下。己巳,發夫五萬人築漳濱堰,三十五日罷。癸亥,車駕狩
 于西山。十一月戊寅,還宮。丙戌,以開府儀同三司、彭城王韶為太尉,以度支尚書胡僧敬為司空。



 是歲,蠕蠕、高麗、勿吉國並遣使朝貢。



 四年春正月丙辰,梁人來聘。夏四月丙寅,遣兼散騎常侍李繪聘于梁。乙酉,以侍中、廣陽王湛為太尉,以尚書右僕射高隆之為司徒,以太尉、彭城王韶為錄尚書事。丁亥,太傅尉景坐事,降為驃騎大將軍、開府儀同三司。辛卯,以太保庫狄乾為太傅,以領軍將軍婁昭為大司馬,封祖裔為尚書右僕射。六月丙申,復前侍中、樂良王忠爵。丁酉,復陳留王景皓、常山王紹宗、高密王永業爵。
 秋八月庚戌,以開府儀同三司、吏部尚書侯景為兼尚書僕射、河南行臺、隨機討防。冬十月甲寅,梁人來聘。十一月壬午,驃騎大將軍、開府儀同三司、青州刺史、西河王心妻薨。



 十二月辛亥,使兼散騎常侍陽斐使於梁。是歲,蠕蠕、高麗、吐谷渾並遣使朝貢。



 武定元年春正月壬戌朔,大赦,改元。己巳,車駕蒐于邯鄲之西山。癸酉,還宮。二月壬申,北豫州刺史高仲密據武牢西叛。三月丙午,帝親納訟。戊申,勃海王高歡大敗西魏師於邙山,追奔至恆農而還。豫、洛二州平。夏四月,封彭城王韶弟襲為武安王。五月壬辰,以剋復武牢,降
 天下死罪已下囚。乙未,以吏部尚書侯景為司空。六月乙亥,梁人來聘。戊寅,封前員外散騎侍郎元長春為南郡王。八月乙丑,以汾州刺史斛律金為大司馬。壬午,遣兼散騎常侍李渾聘于梁。冬十一月甲午,車駕狩于西山。乙巳,還宮。是歲,吐谷渾、高麗、蠕蠕並遣使朝貢。



 二年春二月丁卯,徐州人劉烏黑聚眾反,遣行臺慕容紹宗討平之。三月,梁人來聘。以旱故,宥死罪已下囚。丙午,以開府儀同三司孫騰為太保。壬子,以勃海王世子高澄為大將軍,領中書監。元弼為錄尚書事。以尚書左僕射司馬子如為尚書令。以太原公高洋為左僕射。夏
 五月甲午,遣散騎常侍魏季景聘于梁。丁酉,太尉、廣陽王湛薨。秋八月癸酉,尚書令司馬子如坐事免。九月甲申,以開府儀同三司、濟陰王暉業為太尉。太師、咸陽王坦坐事免,以王還第。冬十月丁巳,太保孫騰、大司馬高隆之各為括戶大使,凡獲逃戶六十餘萬。十一月,西河地陷,有火出。甲申,以司徒高隆之為尚書令,以前大司馬婁昭為司徒。庚子,祀圓丘。辛丑,梁人來聘。是歲,吐谷渾、地豆干、室韋、高麗、蠕蠕、勿吉等並遣使朝貢。



 三年春正月丙申,遣兼散騎常侍李獎聘于梁。二月庚申,吐谷渾國奉其從妹以備後庭,納為容華嬪。夏五月
 甲辰,大赦。秋七月庚子,梁人來聘。冬十月,遣中書舍人尉瑾聘于梁。十二月,以司空侯景為司徒,以中書令韓軌為司空。戊子,以太保孫騰為錄尚書事。是歲,高麗、吐谷渾、蠕蠕並遣使朝貢。



 四年夏五月壬寅,梁人來聘。六月庚子,以司徒侯景為河南大行臺,應機討防。



 秋七月壬寅,遣兼散騎常侍元廓聘于梁。八月,移洛陽漢魏石經于鄴。是歲,室韋、勿吉、地豆干、高麗、蠕蠕並遣使朝貢。



 五年春正月己亥朔,日有蝕之。丙午,勃海王高歡薨。辛亥,司徒侯景降于西魏以求救。西魏遣其將李弼、王思
 政赴之。思政等入據潁川,景乃出走豫州。乙丑,梁人來聘。二月,侯景復背西魏歸梁。夏四月壬申,大將軍高澄來朝。甲午,遣兼散騎常侍李緯聘於梁。五月丁酉朔,大赦。戊戌,以尚書右僕射、襄城王旭為太尉。



 甲辰,以太原公高洋為尚書令,領中書監。以青州刺史尉景為大司馬。以開府儀同三司庫狄乾為太師。以錄尚書事孫騰為太傅。以汾州刺史賀拔仁為太保。以司空韓軌為司徒。以領軍將軍可朱渾道元為司空。以司徒高隆之錄尚書事,以徐州刺史慕容紹宗為尚書左僕射,高陽王斌為右僕射。戊午,大司馬尉景薨。六月乙酉,帝為勃海
 王舉哀於東堂,服緦衰。秋九月辛丑,梁貞陽侯蕭明寇徐州,堰泗水於寒山,灌彭城,以應侯景。冬十一月乙酉,以尚書左僕射慕容紹宗為東南道行臺,與大都督高岳、潘相樂大破禽之,及其二子瑀道。十二月乙亥,蕭明至,帝御閶闔門,讓而宥之。岳等迴師討侯景。是歲,高麗、勿吉並遣使朝貢。



 六年春正月己亥,大都督高岳等於渦陽大破侯景。俘斬五萬餘人,其餘溺死於渦水,水為不流。景走淮南。二月己卯。梁遣使求和。許之。三月癸巳,以太尉、襄城王旭為大司馬,以開府儀同三司高岳為太尉。辛亥,以冬春
 亢旱,赦罪人各有差。夏四月甲子,吏部令史張永和、青州人崔闊等偽假人官,事覺糾檢,首者六萬餘人。甲戌,太尉高岳、司徒韓軌、大都督劉豐等討王思政於潁川,引洧水灌其城。



 九月乙酉,梁人來聘。冬十月戊申,侯景濟江,推梁臨賀王正德為主,以攻建業。



 是歲,高麗,室韋、蠕蠕、吐谷渾並遣使朝貢。



 七年春正月戊辰,梁北徐州刺史、中山侯蕭正表以鎮內附,封蘭陵郡公、吳郡王。三月丁卯,侯景剋建業。夏五月丙辰,侯景殺梁武帝。戊寅,勃海王高澄帥師赴潁川。六月,剋之,獲西魏大將軍王思政等。秋八月辛卯,立皇
 子長仁為太子。



 盜殺勃海王高澄。癸巳,大赦,內外百官並進二級。甲午,太原公高洋如晉陽。冬十月癸未,以開府儀同三司、咸陽王坦為太傅。甲午,以開府儀同三司潘相樂為司空。十二月甲辰,吳郡王蕭正表薨。己酉,以並州刺史彭樂為司徒。是歲,蠕蠕、地豆干、室韋、高麗、吐谷渾並遣使朝貢。



 八年春正月辛酉,帝為勃海王高澄舉哀於東堂。戊辰,詔太原公高洋嗣事,徙封齊郡王。甲戌,地豆干、契丹並遣使朝貢。二月庚寅,以尚書令高隆之為太保。



 三月庚申,進齊郡王高洋爵為齊王。夏四月乙巳,蠕蠕遣使朝
 貢。五月甲寅,詔齊王為相國,總百揆,備九錫之禮。以齊國太妃為王太后,王妃為王后。丙辰,遜帝位於齊。天保元年己未,封帝為中山王,邑一萬戶,上書不稱臣,答不稱詔,載天子旌旗,行魏正朔,乘五時副車。封王諸子為縣公,邑各一千戶。奉絹一萬疋,錢一萬貫。粟二萬石,奴婢三百人,水碾一具,田百頃,園一所,於中山國立魏宗廟。



 二年十二月己酉,中山王殂,時年二十八。三年二月,奉謚曰孝靜皇帝。葬於鄴西漳北。其後發之,陵崩,死者六十人。



 帝好文,美容儀。力能挾石師子以踰
 墻,射無不中。嘉辰宴會,多命群臣賦詩。



 從容沉雅,有孝文風。勃海王高澄嗣事,甚忌焉。以大將軍中兵參軍崔季舒為中書、黃門侍郎,令監察動靜,小大皆令季舒知。澄與季舒書曰:「癡人復何似?癡勢小差未?」帝嘗與獵於鄴東,馳逐如飛。監衛都督烏那羅受工伐從後呼帝曰:「天子莫走馬,大將軍怒!」澄嘗侍帝飲,大舉觴曰:「臣澄勸陛下。」帝不悅曰:「自古無不亡之國,朕亦何用此活!」澄怒曰:「朕,朕,狗腳朕!」澄使季舒毆帝三拳,奮衣而出。明日,澄,使季舒勞帝,帝亦謝焉。賜絹,季舒未敢受,以啟澄。



 澄使取一段。帝束百疋以與之,曰:「亦一段爾。」帝不堪憂辱,詠
 謝靈運詩曰:「韓亡子房奮,秦帝魯連恥。本自江海人,志義動君子。」常侍、侍講荀濟知帝意,乃與華山王大器、元瑾密謀於宮中。偽為山而作地道向北城。至千秋門,門者覺地下響動,以告澄。澄勒兵入營,曰:「陛下何意反耶?臣父子功存社稷,何負陛下耶?」及將殺諸妃嬪。帝正色曰:「王自欲反,何關於我?我尚不惜身,何況妃嬪!」



 澄下床叩頭,大啼,謝罪。於是酣飲,夜久乃出。居三日,幽帝於含章堂。大器、瑾等皆見烹於市。及將禪位於文宣,襄城王昶及司徒潘相樂、侍中張亮、黃門郎趙彥深等求入奏事。帝在昭陽殿見之。旭曰:「五行遞運,有始有終。齊王聖
 德欽明,萬姓歸仰。臣等昧死聞奏,願陛下則堯禪舜。」帝便斂容答曰:「此事推挹已久,謹當遜避。」又云:「若爾,須作詔書。」侍郎崔劼、裴讓之奏云:「詔已作訖。」



 即付楊愔進於帝,凡十條。書訖,曰:「將安朕何所?復若為而去?」楊愔對曰:「在北城,別有館宇,還備法駕,依常仗衛而去。」帝乃下御座,步就東廊。口詠范蔚宗《後漢書贊》云:「獻生不辰,身播國屯,終我四百,永作虞賓。」所司奏請發。帝曰:「古人念遺簪弊履,欲與六宮別,可乎?」高隆之曰:「今天下猶陛下之天下,況在後宮!」乃與夫人嬪以下訣,莫不欷歔掩涕。嬪趙國李氏誦陳思王詩云:「王其愛玉體,俱享黃髮期。」皇
 后已下皆哭。直長趙德以故犢車一乘,候於東上閣。帝上車,德超上車,持帝。帝肘之,曰:「朕畏天順人,授位相國,何物奴,敢逼人!」趙德尚不下。及出雲龍門,王公百僚衣冠拜辭。帝曰:「今日不減常道鄉公、漢獻帝。」眾皆悲愴,高隆之泣灑。遂入北城,下司馬子如南宅。及文宣行幸,常以帝自隨。帝后封太原公主,常為帝嘗食,以護視焉。竟遇鴆而崩。



 論曰:莊帝運接交喪,招納勤王。雖時事孔棘,而卒有四海。猾逆剪除,權強擅命,神慮獨斷,芒刺未除;而天未忘亂,禍不旋踵。自茲之後,魏室土崩。始則制屈強胡,終乃
 權歸霸政。主祭祀者不殊於寄坐,遇黜辱者有甚於弈棋。雖以節閔之明,孝武之長,祗以速是奔波。文帝以剛強之質,終以守雌自寶。靜、恭運終天祿,高蹈唐、虞,各得其時也。



\end{pinyinscope}