\article{卷二十八列傳第十六}

\begin{pinyinscope}

 陸俟源賀曾孫彪玄孫師師從叔雄劉尼薛提陸俟,代人也。曾祖乾,祖引,世領部落。父突,道武初帥部人從征伐,數有戰功,位離石鎮將、上黨太守、關內侯。俟少聰慧。明元踐祚,襲爵關內侯,位給事中,典選部、蘭臺事,當官無所撓。太武征赫連昌,詔俟督諸軍鎮以備蠕蠕。與西平公安頡攻剋武牢,賜爵建鄴公,拜冀州刺史。
 時考州郡,唯俟與河內太守丘陳為天下第一。轉武牢鎮大將。平涼休屠金崖、羌狄子玉等叛,復轉為安定鎮大將,追討崖等,皆獲之。遷懷荒鎮大將。未期,諸高車莫弗懼俟嚴急,請前鎮將郎孤。



 太武許之。征俟,至京朝見,言不過周年,孤身必敗,高車必叛。帝疑不實,切責之,以公歸第。明年,諸莫弗果殺孤以叛。帝聞之大驚,召俟問其故。俟曰:「夫高車之俗,上下無禮,無禮之人,難為其上。臣蒞以威嚴,節之憲網,欲漸加訓導,使知分限。而惡直醜正,實繁有徒,故訟臣無恩,稱孤之美。孤獲還鎮,欣其名譽,必加恩於百姓,譏臣為失,專欲以寬惠臨之,仁
 恕待之。無禮之人,易生陵傲,不過期年,無復上下。既無上下,然後收之以威,則人懷怨憝。怨憝既多,敗亂彰矣。」



 帝歎曰:「卿身乃短,慮何長也!」即日復除散騎常侍。



 帝征蠕蠕,破涼州,常隨駕別督輜重。又與高涼王那復渡河南略地。仍遷長安鎮大將。與高涼王那擊蓋吳於杏城,獲吳二叔。諸將欲送京師,俟獨不許,曰:「若不斬吳,恐長安之變未已。一身藏竄,非其親信,誰能獲之?若停十萬眾追一人,非上策也。不如私許吳叔,免其妻子,使自追吳。」諸將咸曰:「今獲其二叔,唯吳一人,何所復至?」俟曰:「諸君不見毒蛇乎?不斷其頭,猶能為害。況除腹心之疾,而
 曰必遺其類,可乎?」遂捨吳二叔,與之期。及期,吳叔不至,諸將皆咎俟。俟曰:「此未得其便耳,必不背也。」後數日,果斬吳以至,皆如其言。俟之明略獨決,皆此類也。遷內都大官。



 安定盧水劉超等叛,太武以俟威恩被關中,詔以本官加都督秦、雍諸軍,鎮長安。帝曰:「超等恃險,不順王命,朕若以重兵與卿,則超等必合為一;若以輕兵與卿,則不制矣。今使卿以方略定之。」於是俟單馬之鎮。既至,申揚威信,示以成敗,超猶無降意。俟乃率其帳下見超。超使人逆曰:「三百人以外,當以弓馬相待;三百人以內,當以酒食相供。」乃將二百騎詣超。超備甚嚴,遂縱酒,盡
 醉而還。後偽獵,詣超。與士卒約曰:「今會發機,當以醉為限。」俟乃詐醉,上馬大呼,斬超首。士卒應聲縱擊,遂平之。帝大悅,徵拜外都大官。



 文成踐祚,以子麗有定策勳,進爵東平王。薨,年六十七,謚成王。有子十二人。



 長子珝,多智,有父風。文成見而悅之,謂朝臣曰:「吾常歎其父智過其軀,是復踰於父矣!」少為內都下大夫。奉上接下,行止取與,每能逆曉人意。與其從事者無不愛之。興安初,賜爵聊城侯。出為相州刺史,假長廣公。為政清平,抑彊扶弱。州中有德宿老名望素重者,以友禮待之。詢之政事,責以方略,如此者十人,號曰十善。又簡取諸縣彊門百
 餘人以為假子。誘接殷勤,賜以衣服,令各歸家為耳目。於是發姦擿伏,事無不驗。百姓以為神明,無敢劫盜者。在州七年,家至貧約。



 徵為散騎常侍,百姓乞留珝者千餘人。獻文不許,謂群臣曰:「珝之善政,雖古人何以加之。」賜絹五百匹,奴婢十口。珝之代還也,吏人大斂布帛以遺之。珝皆不受,人亦不取,於是以此物起佛寺焉,因名長廣公寺。後襲父爵,改封建安王。



 時宋司州刺史常珍奇以懸瓠內附,新人猶懷去就。珝銜旨撫慰,諸有陷軍為奴婢者,珝皆免之。百姓欣悅,人情乃定。車駕討蠕蠕,詔珝為選部尚書,錄留臺事。



 及獻文將禪位於京兆王子
 推,任城王雲、隴西王源賀並固諫。珝抗言曰:「皇太子聖德承基,四海瞻望,不可橫議,干國之紀。臣請刎頸殿庭,有死無貳。」久之,帝乃解。詔曰:「珝直臣也,其能保吾子乎?」遂以珝為太保,與太尉源賀持節奉皇帝璽紱傳位于孝文。延興四年薨,贈以本官,謚曰貞王。珝有六子,琇、凱知名。



 琇字伯琳,珝第五子也。母赫連氏身長七尺九寸,甚有婦德。珝有以爵傳琇之意。琇年九歲,珝謂之曰:「汝祖東平王有十二子,我為嫡長,承襲家業。今已年老,屬汝幼沖,詎堪為陸氏宗首乎?」琇對曰:「茍非鬥力,何患童幼!」珝奇之,遂立琇為世子。珝薨,襲爵。琇沈毅少言,雅好
 讀書。以功臣子孫,為侍御長,累遷祠部尚書、司州大中正。會從兄睿事,免官。景明初,試守河內郡。咸陽王禧謀反,令子曇和等先據河內。琇聞禧反,斬曇和首。時以琇不先送曇和,禧敗始斬,責其通情,徵詣廷尉。少卿崔振窮罪狀,案琇大逆。陸宗大小,咸見收捕。會將赦,先斃於獄。琇弟凱仍上書訴冤,宣武詔復琇爵,子景詐襲。



 凱字智君,謹重好學。位太子庶子、給事黃門侍郎。凱在樞要十餘年,以忠厚見稱。後遇患,頻上書乞骸骨。除正平太守,在郡七年,號為良吏。



 初,孝文將議革變舊風,大臣並有難色。又每引劉芳、郭祚等,常與規謀,共論政事。而國
 戚謂遂疏己,怏怏有不平之色。帝乃令凱私喻之曰:「至尊但欲廣知前事,直當問其古式耳。終無寵彼而疏國戚舊人意。」乃稍解。及兄琇陷罪,凱亦被收,遇赦乃免。凱痛兄之死,哭無時節,目幾失明,訴冤不已。至正始初,宣武復琇官爵。凱大喜,置酒集諸親曰:「吾所以數年之中抱病忍死者,顧門計耳,今願已遂。」以其年卒,贈龍驤將軍、南青州刺史,謚曰惠。



 長子,字道暉,與弟恭之並有時譽。洛陽令賈禎見其兄弟,歎曰:「僕以老年,更睹雙璧。」又嘗兄弟共候黃門郎孫惠蔚。謂諸賓曰:「不意二陸,復在坐隅。



 吾德謝張公,無以延譽。」位尚書右戶、三公郎,
 坐事免。後除伏波將軍。卒,贈冠軍、恆州刺史。擬《急就篇》為《悟蒙章》,及《七誘》、《十醉》,章表數十篇。與恭之晚不和睦,為時所鄙。



 子元規位尚書郎。元規子撥,陰陽律曆,多所通解,位並州長流參軍。



 恭之字季順,有操尚,位東荊州刺史。贈吏部尚書,謚曰懿。恭之所著文章詩賦凡千餘篇。子曄,字仁崇,篤志文學,《齊律》序則仁崇之詞。位終通直散騎常侍。弟寬,字仁惠,太子中舍人,待詔文林館。寬兄弟並有才品,議者稱為三武。



 珝弟歸,位東宮舍人、駕部校尉。子珍,夏州刺史,贈太僕卿,謚曰靜。



 珍子旭,性雅淡,好《易》、緯候之學,撰《五星要決》及《兩儀真圖》,頗得
 其指要。太和中,徵拜中書博士,稍遷散騎常侍。知天下將亂,遂隱於太行山,屢征不起。卒後,贈並、汾、恆、肆四州刺史。子騰。



 騰字顯聖,少慷慨有大節。從爾朱榮平葛榮,以功賜爵清河縣伯。稍遷通直散騎常侍。及孝武西遷,時使青州,遂留鄴,為陽城郡守。



 大統九年,大軍東討陽城,被執。周文帝釋而與語,騰盛論東州人物,又敘述時事,辭理抑揚。周文嘆曰:「卿真不背本也!」即拜帳內大都督。未幾,除太子庶子,遷武衛將軍。騰既為周文所知,國欲立功,不願內職。



 及安康賊黃眾寶等作亂,攻圍東梁州。城中糧
 盡,詔騰率軍大破之。軍還,拜龍州刺史。使通江油路,直出南秦。周文謂曰:「此是卿取柱國之日。」即解所服金帶賜之。州人李廣嗣、李武等憑據巖險,歷政不能制。騰密令多造飛梯,夜襲破之,執廣嗣等於鼓下。其黨有任公忻,圍逼州城,請免廣嗣及武,即散兵請罪。騰謂將士曰:「吾不殺廣嗣等,可謂墮軍實而長寇仇。」即斬廣嗣及武,以首示之。



 於是出兵奮擊,盡獲之。進位驃騎大將軍、開府儀同三司,轉江州刺史,進爵上庸縣公。陵州木籠獠恃險,每行抄劫,詔騰討之。獠因山為城,攻之未可拔。騰遂於城下多設聲樂及諸雜伎,示無戰心。諸賊果棄其
 兵仗,或攜妻子臨城觀樂。騰知其無備,遂縱兵討擊,盡殺破之。



 周明帝初,陵、眉等八州夷夏並反,攻破郡縣,騰率兵討平之。及齊公憲作鎮於蜀,以騰為隆州刺史,令憲入蜀兵馬鎮防,皆委騰統攝。赴公招代憲,復請留之。



 遷隆州總管,領刺史。



 保定二年,資州石槃人反,殺郡守,據險自守,州軍不能制。騰率軍討擊,盡破斬之。而蠻子反,所在蜂起,山路險阻,難得掩襲。遂量山川形勢,隨便開道。



 蠻獠畏威,承風請服。所開之路,多得古銘,並是諸葛亮、桓溫舊道。是年,鐵山獠抄斷內江路,使驛不通。騰乃進軍討之,一日下其三城,招納降附者三萬戶。帝以
 騰母在齊,未令東討。適有其親屬自齊還朝者,晉公護奏令告騰云:「齊已誅公母兄。」蓋欲發其怒也。騰乃發哀泣血,志在復仇。四年,齊公憲與晉公護東征,請騰為副。趙公招時在蜀,復欲留之。晉公護與招書,於是令騰馳傳還朝,副憲東伐。



 天和初,信州蠻、蜑據江硤反叛,連結二千餘里,又詔騰討之。騰沿江南而下,軍至湯口,分道奮擊,所向摧破。乃築京觀,以旌武功。涪陵郡守蘭休祖又阻兵為亂,方二千餘里。復詔騰討之,巴蜀悉定,詔令樹碑紀功績焉。騰自在龍州至是,前後破平諸賊,凡賞得奴婢八百口,馬牛稱是。



 四年,遷江陵總管。陳遣其將
 章昭達圍江陵,衛王直聞有陳寇,遣大將軍趙訚、李遷哲等率步騎赴之,並受騰節度。時遷哲等守外城,陳將程文季、雷道勤夜來掩襲,遷哲等驚亂,不能抗禦。騰夜遣開門奮擊,大破之。陳人奔潰,道勤中流矢而斃。陳人決龍川寧朔堤,引水灌江陵城。騰親率將士,戰於西堤,破之,陳人乃遁。



 加位柱國,進爵上庸郡公。建德二年,徵拜大司空,尋出為涇州總管。宣政元年冬,薨於京師,贈太尉公,謚曰定。子玄嗣。



 玄字士鑒,入關時,年七歲。仕齊為奉朝請、成平縣令。齊平,武帝見玄,特加勞勉,即拜地官府都上士。大象末,為隋文帝相府內兵參軍。



 玄弟融,
 字士傾,最知名,少歷顯職。大象末,位至大將軍、定陵縣公。



 弟麗,少以忠謹,入侍左右,太武特親暱之。舉動審慎,初無愆失。賜爵章安子,稍遷南部尚書。太武崩,南安王餘立。既而為中常侍宗愛等所殺。百寮憂惶,莫知所立。麗首建大議,與殿中尚書長孫渴侯、尚書源賀、羽林中郎劉尼奉迎文成於苑中而立之。社稷獲安,麗之謀也。由是受心膂之任,在朝者無出其右。興安初,封平原王。麗頻讓,不聽,乃啟以讓父。文成曰:「朕為天下主,豈不能得二王封卿父子也?」以其父俟為東平王。麗尋遷侍中、撫軍大將軍、司徒公,復其子孫,賜妻妃號。麗以優寵既頻,
 固辭不受,帝益重之。領太子太傅。麗好學愛士,常以講習為業。甚孝,遭父憂,毀瘠過禮。



 和平六年,文成崩。先是,麗療疾於代郡溫泉,聞凶欲赴。左右止之曰:「宮車晏駕,王德望素重,姦臣若疾人譽,慮有不測之禍。」麗曰:「安有聞君父之喪,方慮禍難!」便馳赴。初,乙弗渾悖傲,每為不法,麗數諍之,由是見忌,害之。



 謚曰簡王,陪葬金陵。孝文追錄先朝功臣,以麗配饗廟庭。



 麗二妻,長曰杜氏,次張氏。長子定國,杜氏所生;次睿,張氏所出。



 定國在襁抱,文成幸其第,詔養宮內。至於游止,常與獻文同處。年六歲,為中庶子。及獻文踐祚,拜散騎常侍,賜封東郡王。定國
 以承父爵,辭,不許。又以父爵讓弟睿,乃聽之。俄遷侍中、儀曹尚書,轉殿中尚書。前後大駕征巡,擢為行臺,錄都曹事,超遷司空。定國恃恩,不循法度,延興五年,坐事免官爵為兵。大和初,復除侍中、鎮南將軍、秦益二州刺史,復王爵。八年,薨於州。贈以本官,謚曰莊王。



 子昕之,字慶始,風望端雅。襲爵,例降為公。尚獻文女常山公主,拜駙馬都尉,歷通直郎。景明中,以從叔琇罪,免官。尋以主婿,除通直散騎常侍。歷兗、青二州刺史,並有政績。轉安北將軍、相州刺史。卒,贈鎮東將軍、冀州刺史,謚曰惠。初,定國娶河東柳氏,生子安保。後娶范陽盧度
 世女,生昕之。二室俱為舊族,而嫡妾不分。定國亡後,兩子爭襲父爵。僕射李沖有寵於時,與度世子伯源婚親相好。沖遂左右助之,昕之由是承爵,尚主,職位赫弈。安保沈廢貧賤,不免飢寒。昕之容貌柔謹,孝文以其主婿,特垂暱眷。宣武時,年未四十,頻撫三籓,當世以此榮之。昕之卒後,母盧悼念,傷過而亡。公主奉姑有孝稱。神龜初,與穆氏瑯邪長公主並為女侍中。又性不妒忌,以昕之無子,為納妾媵,而皆育女。公主有三女,無男,以昕之從兄希道第四子子彰為後。



 子彰字明遠,本名士沈。年十六出後,事公主盡禮。丞相、高陽王雍常言曰:「常山妹
 雖無男,以子彰為兒,乃過自生矣。」正光中,襲爵東郡公,累遷給事黃門侍郎。子彰妻即咸陽王禧女。禧誅,養於彭城王第,莊帝親之,略同諸姊。建義初,爾朱榮欲循舊事,庶姓封王,由是封子彰濮陽郡王。尋而詔罷,仍復先爵。天平中,拜衛將軍、潁州刺史,以母憂去職。元象中,以本將軍除齊州刺史,又加驃騎將軍,行懷州事,轉北豫州刺史,仍除徐州刺史,將軍並如故。一年歷三州,當世榮之。還朝,除衛大將軍、右光祿大夫,行瀛州事。尋拜侍中,復行滄州事。進號驃騎大將軍。行冀州事。除侍讀,兼七兵尚書,行青州事。



 子彰初為州,以聚斂為事,晚節脩
 改,自行青、冀、滄、瀛,甚有時譽。加以虛己納物,人士敬愛之。除中書監。卒,贈開府儀同三司,謚曰文宣。子彰崇好道術,曾嬰重病藥中須桑螵蛸。子彰不忍害物,遂不服焉,其仁如此。教訓六子,雅有法度。子仰。



 仰字雲駒,少機悟,美風神。好學不倦,博覽群書,《五經》多通大義。善屬文,甚為河間邢邵所賞。劭又與子彰交游,嘗謂子彰曰:「吾以卿老蜯遂出明珠,意欲為群拜紀可乎?」由是名譽日高,雅為搢紳所推許。起家員外散騎侍郎,歷文襄大將軍主簿、中書舍人、兼中書侍郎,以本職兼太子洗馬。自梁、魏通和,歲有交聘,仰每兼官宴接。在
 席賦詩,仰必先成,雖未能盡工,以敏速見美。除中書侍郎,修國史。以父憂去職。居喪盡禮,哀毀骨立,詔以本官起。文襄時鎮鄴,嘉其至行,親詣門以慰勉之。



 仰母,魏上庸公主。初封藍田,高明婦人也。甚有志操。仰昆季六人,並主所出,故邢邵常謂人云:「藍田生玉,固不虛矣。」主教訓諸子,皆以義方。雖創巨痛深,出於天性;然動依禮度,亦母氏之訓焉。仰兄弟相率廬於墓側,負土成墳。



 朝廷所嗟尚,發詔褒揚,改其所居里為孝終里。服竟,當襲,不忍嗣侯。使迄未應受。



 齊天保初,常山王薦仰器幹,文宣面授給事黃門侍郎。遷吏部郎中。上洛王思宗為清都
 尹,辟為邑中正,食貝丘縣幹。遭母喪,哀慕毀悴,殆不勝喪,遂至沈篤,頓伏床枕,又成風疾。第五弟摶遇疾,臨終,謂其兄弟曰:「大兄尪病如此,性至慈愛,摶之死日,必不得使大兄知之,哭泣聲必不可聞徹,致有感動。」家人至於祖載,方始告之。仰聞而悲痛,一慟便絕。年四十八。



 仰自在朝行,篤慎周密。不說人短,不伐己長,言論清遠,有人倫鑒裁,朝野甚悲惜之。贈衛將軍、青州刺史,謚曰文。所著文章十四卷,行於世。齊之郊廟諸歌,多仰所制。



 子乂字旦,襲爵始平侯。乂聰敏博學,有文才,年十九舉司州秀才。歷祕書郎、南陽王文學、通直散騎侍郎,待詔文
 林館,兼散騎侍郎。迎陳使,還,兼中書舍人,加通直散騎常侍。乂於《五經》最精熟,館中謂之石經。人為之語曰:「《五經》無對,有陸乂。」



 仰第二弟駿,字雲驤。自中書舍人歷黃門侍郎、散騎常侍,卒於東廣州刺史。



 駿弟杳,字雲邁,亦歷中書舍人、黃門常侍,假儀同三司、秦州刺史。武平中,為寇所圍。經百餘日,就加開府儀同三司。城中多疫癘,死者過半,人無異心。遇疾卒。及城陷,陳將吳明徹以杳有善政,吏人所懷,啟陳主,還其屍,家累貲物無所犯。贈開府儀同三司、尚書僕射。子玄卿,位尚書膳部郎。



 杳弟騫,字雲儀,亦歷中書舍人、黃門常侍。武平末,吏部郎中。



 騫弟摶,字雲征,好學有行檢,卒於著作佐郎。



 摶弟彥師,字雲房,少以行檢稱。及長好學,解屬文。魏襄城王元旭引為參軍事,以父艱去職。哀毀殆不勝喪,與兄仰廬於墓次。鄉人重之,皆就墓側存問;晦朔之際,車馬不絕。中書令河間邢邵表薦之。未報,彭城王浟為司州牧,召補主簿。



 後歷中外府東閣祭酒。兄仰當襲父始平侯,以彥師昆弟中最幼,表讓封焉,彥師固辭而止。世稱友悌孝義,總萃一門。為中書舍人、通直散騎侍郎。每陳使至,必高選主客,彥師所接對者,前後六輩。歷中書、黃門侍郎。後以不阿宦者,遇讒,出為中山太守,有惠政。數年,徵為
 吏部郎中、散騎常侍,又拜銀青光祿大夫,假儀同三司,行鄭州刺史,尋除給事黃門侍郎。武平末,車駕如晉陽,北平王鎮鄴,委彥師留臺機密,以重慎見知。周武帝平齊,授彥師下大夫,轉少納言,賜爵臨水縣男。及隋文為丞相,彥師遇疾,請假還鄴。尉遲迥將為亂,彥師知之,遂將妻子潛歸長安。文帝嘉之,授內史下大夫,拜上儀同。及帝受禪,拜尚書左丞,進爵為子。



 彥師素多病,未幾,以務劇病動,乞解所職,有詔聽以本官就第。歲餘,轉吏部侍郎。隋承周制,官無清濁,彥師在職,凡所任人,頗甄別於士庶,論者美之。後復以病出為汾州刺史,卒官。



 睿字思弼,年十餘,襲爵撫軍大將軍、平原王。沈雅好學,折節下士。年未二十,時人便以宰輔許之。娶東徐州刺史博陵崔鑒女。時孝文尚未改北人姓,鑒謂所親云:「平原王才度不惡,但恨其姓名殊為重復。」睿婚,自東徐還經鄴,見李彪,甚敬悅之。仍與趣京,以為館客。後為北征都督,擊蠕蠕,大破之。遷侍中、都曹尚書。時蠕蠕又犯塞,詔睿討之,追至石磧,禽其帥赤阿突等數百人。還,加散騎常侍,遷尚書左僕射,領北部尚書。



 十六年,降五等之爵,以麗勳著前朝,封睿鉅鹿郡公。尋為使持節、鎮北大將軍、尚書令、衛將軍,討蠕蠕,大破之而還。以母憂解。孝
 文將有南伐之事,以本官起授征南將軍。睿固辭,請終情禮,敕有司敦喻不許。復除使持節、都督恆州刺史,行尚書令。時車駕南征,上表諫,帝不從。睿又表請車駕還代,親臨太師馮熙葬,坐削奪都督三州諸軍事。尋進號征北大將軍。以有順遷之表,加邑四百戶。時穆泰為定州刺史,以疾病,請恆州自效,乃以睿為定州刺史。未發,遂與泰等同謀構逆,賜死獄中。聽免孥戮,徙其妻子於遼西。



 睿長子希道,字洪度。有風貌,美鬚髯,歷覽經史,頗有文致。初拜中散,遷通直郎。坐父事,徙於遼西。於後得還,從征自效,以軍功賜爵淮陽男,拜諫議大夫。累遷前
 將軍、郢州刺史。希道善於馭邊,甚有威略。轉平西將軍、涇州刺史。



 卒官,贈撫軍將軍、定州刺史。



 希道有六子:士懋字元偉。天平中以其曾祖麗有翼戴之勳,詔特復鉅鹿郡公,令士懋襲。位營州刺史。士懋弟士宗,字仲彥,尚書左外兵郎中。士宗弟士述,字幼文,符璽郎中。建義初,並於河陰遇害。士述弟士沈,出繼叔昕之。士沈弟士廉,字季脩,建州平北府長史。永安末,爾朱世隆攻陷州城,見害。士廉弟士佩,字季偉,武定中,安東將軍、司州從事。



 希道弟希悅,尚書外兵郎中。麗季弟騏驎,侍御中散,轉侍御史。太和初,新平太守。子高貴,孝昌中,兗州鎮東府
 法曹參軍。



 高貴子操,字仲志,高簡有風格,早以學業知名,雅好文。操仕魏,兼散騎常侍聘梁。使還,為廷尉卿。齊文襄為世子,甚好色,崔季舒為掌媒焉。薛氏寘書妻元氏有色,迎入欲通之。元氏正辭,且哭。世子使季舒送付廷尉罪之。操曰:「廷尉守天子法,須知罪狀。」世子怒,召操,命刀環築之,更令科罪。操終不撓,乃口責之。後徙御史中丞。天保中,卒於殿中尚書。子孔璋,武平中,卒於高陽太守。



 高貴弟孟遠,位奉朝請。孟遠子概之,位司農卿。



 概之子爽,字開明。少聰敏,年九歲就學,日誦二千餘言。齊尚書僕射楊遵彥見而異之曰:「陸氏世有人焉。」仕齊,位
 中書侍郎。齊滅,周武帝聞其名,與陽休之、袁叔德等俱徵入關。諸人多將輜重,爽獨載數千卷書。至長安,授宣納上士。



 隋文帝受禪,頻遷太子洗馬,與左庶子宇文愷等撰《東宮典記》七十卷。朝廷以其博學有口辯,陳人至境,常令迎勞。卒官,贈上儀同、宣州刺史。



 子法言,敏學有家風,釋褐承奉郎。初,爽之為洗馬,常奏文帝云:「皇太子諸子未有嘉名,請依《春秋》之義,更立名字。」上從之。及太子廢,上追怒爽曰:「我孫製名,寧不自解?陸爽乃爾多事!扇惑於勇,亦由此人。其身雖故,子孫並宜屏黜,終身不齒。」法言竟坐除名。



 源賀,西平樂都人,私署河西王禿髮傉檀之子也。傉檀為乞伏熾盤所滅,賀自樂都奔魏。賀偉容貌,善風儀。太武素聞其名。及見,器其機辯,賜爵西平侯。謂曰:「卿與朕同源,因事分姓,今可為源氏。」從擊叛胡白龍,又討吐京胡,皆先登陷陣。以功進號平西將軍。太武征涼州,以為鄉導,問攻戰之計。賀曰:「姑臧外有四部鮮卑,各為之援,然皆臣祖父舊人。臣願軍前宣國威信,必相率請降。外援既服,然後攻其孤城,拔之如反掌耳。」帝曰:「善。」乃遣賀招慰,下三萬餘落。及圍姑臧,由是無外慮,故得專力攻之。涼州平,以功進爵西平公。又從征蠕蠕,擊五城吐京
 胡,討蓋吳諸賊,皆有功,拜散騎常侍。從駕臨江,為前鋒大將,善撫士卒,加有料敵制勝之謀。



 賀為人雄果,每遇強寇,輙自奮擊,帝深誡之。賀本名破羌,是役也,帝謂曰:「人之立名,宜保其實,何可濫也?」賜名賀焉。拜殿中尚書。南安王餘為宗愛所殺,賀部勒禁兵,靜遏外內,與南部尚書陸麗決議定策,翼戴文成。令麗與劉尼馳詣苑中奉迎,賀營中為內應。俄而麗抱文成,單騎而至。及即位,賀有力焉。以定策勛,進爵西平王。及班賜百寮,敕賀任意取之,辭以江南未賓,漢北不款,府庫不宜致匱。固使取之,唯取戎馬一疋。



 時斷獄多濫。賀上書曰:「案律,謀反
 之家,其子孫雖養他族,追還就戮所以絕罪人之類,彰大逆之辜。其為劫賊應誅者,兄弟子姪在遠道隔關津皆不坐。竊惟先朝制律之意,以不同謀,非絕類之罪,故特垂不死之詔。若年十三已下,家人首惡,計所不及。臣愚以為可原其命,沒入官。」帝納之。



 出為冀州刺史,改封隴西王。既受除,上書曰:「臣聞人之所寶,莫寶於生命;德之厚者,莫厚於宥死。然犯死之罪,難以盡恕。權其輕重,有可矜恤。今勍寇游魂於北,狡賊負險於南,其在疆場,猶須戍防。臣愚以為自非大逆、赤手殺人之罪,其坐贓及盜與過誤之愆應入死者,皆可原命,謫守邊境。是則
 已斷之體,更受生成之恩;徭役之家,漸蒙休息之惠。刑措之化,庶幾在茲。」帝喜納之,已後入死者,皆恕死徙邊。久之,帝謂群臣曰:「昔源賀勸朕,宥諸死刑,徙充北籓諸戍。自爾至今,一歲所活,殊為不少。濟命之理既多,邊戍之兵有益。茍人人如賀,朕臨天下,復何憂哉!」群臣咸曰:「非忠臣不能進此計,非聖明不能納此言。」



 賀之臨州,鞫獄以情,徭役簡省,清約寬裕,甚得人心。時武邑郡姦人石華告沙門道可與賀謀反,有司以聞。文成曰:「賀保無此。」乃精加訊檢,華果引誣。



 乃遣使慰勉之。帝顧左右曰:「賀忠誠,尚致誣謗,其不若是者,可無慎乎!」時考殿最,賀
 政為上第,賜衣馬器物,班宣天下。後徵拜太尉。蠕蠕寇邊,賀從駕討破之。及獻文將傳位于京兆王子推,時賀都督諸軍事屯漠南,乃馳傳征賀。賀至,正色固執不可。即詔持節奉皇帝璽綬以授孝文。是歲,河西叛,敕遣賀討之,多所降破。賀依古今兵法及先儒耆舊說,略採至要,為十二陳圖,上之,獻文覽而嘉焉。



 又都督三道諸軍屯漠南。



 時每歲秋冬,遣軍三道並出,以備北寇,至春中乃班師。賀以勞役京都,又非禦邊長計,乃上言,請募諸州鎮有武勇者三萬人,復其徭賦,厚加振恤,分為三部。



 二鎮之間築城,城置萬人,給強弩十二床,武衛三百乘。
 弩一床給牛六頭,武衛一乘給牛二頭。多造馬槍及諸器械,使武略大將二人以鎮撫之。冬則講武,春則種植,並戍並耕,則兵未勞而有盈蓄矣。又於白道南三處立倉,運近州鎮租粟以充之。足食足兵,以備不虞,於事為便。不可歲常舉眾。事寢不報。



 上書稱病乞骸骨,至于再三,乃許之。朝有大議,皆就詢訪,又給衣藥珍羞。



 太和元年二月,療疾於溫湯。孝文、文明太后遣使屢問消息,太醫視疾。患篤,還于京師。乃遺令諸子曰:「吾頃以老患辭事,不悟天慈降恩,爵逮於汝。汝其毋傲吝,毋荒怠,毋奢越,毋嫉妒。疑思問,言思審,行思恭,服思度。遏惡揚善,親
 賢遠佞,目觀必真,耳屬必正,忠勤以事君,清約以臨己。吾終之後,所葬,時服單櫝,足申孝心,皞靈明器,一無用也。」三年,薨,贈侍中、太尉、隴西王印綬,謚曰宣王。賜巉輬車及命服、溫明祕器,陪葬金陵。



 長子延,性謹厚,少好學,位侍御中散,賜爵廣武子。卒,贈涼州刺史,廣武侯,謚曰簡。子鱗襲。



 延弟思禮,後賜名懷,謙恭寬雅有大度。文成末,為侍御中散。父賀辭老,詔受父爵。後持節督諸軍屯於漠南,蠕蠕甚憚之。還,除殿中尚書,出為長安鎮將、雍州刺史。清儉有惠政,善撫恤,劫盜息止。復拜殿中尚書,加侍中,參都曹事。



 又督諸軍征蠕蠕,六道大將,咸受節度。
 遷尚書令,參議律令。後便降為公。除司州刺史。又從駕南征,加衛大將軍,領中軍事。以母憂去職,賜帛三百匹,穀一千石。車駕幸代,詔使者弔慰。



 景明二年,除尚書左僕射,加位特進。時詔以姦吏犯罪,每多逃遁,肆眚乃出,並皆釋然。自今犯罪,不問輕重,藏竄者,悉皆遠流。若永避不出,兄弟代徙。懷乃奏曰:「謹案條例,逃吏不在赦限。竊惟聖朝之恩,事異前宥,諸流徙在路,尚蒙旋返,況有未發,而仍遣邊戍?案守宰犯罪,逃走者眾,祿潤既優,尚有茲失,及蒙恩宥,卒然得還。今獨苦此等,恐非均一之法。」書奏,門下以成式既班,駁奏不許。懷重奏曰:「臣以為
 法貴經通,政尚簡要,刑憲之設,所以網羅罪人,茍理之所備,不在繁典。伏尋條例,勛品以下,罪發逃亡,遇恩不宥。雖欲抑絕姦途,匪為通式。謹按事條,侵官敗法,專據流外,豈九品已上,人皆貞白也?其諸州守宰,職任清流,至有貪濁,事發逃竄,而遇恩免罪;勳品已下,獨求斯例。如此,則寬縱上流,法切於下,育物有差,惠罰不等。又謀逆滔天,經恩尚免,吏犯微罪,獨不蒙赦,使大宥之經不通,開生之路致壅;進違古典,退乖今律。臣少踐天官,老荷樞要,每見訴訟,出入嗟苦,輒率愚見,以為宜停。」書奏,宣武納之。其年,除車騎大將軍、涼州大中正。



 懷又表
 曰:「昔世祖升遐,南安在位,出拜東廟為賊臣宗愛所賊。時高宗避難,龍潛苑中,宗愛異圖,神位未立。先臣賀與長孫渴侯、陸麗等奉迎高宗,纂徽寶命。



 麗以扶負聖躬,親所見識,蒙授撫軍、司徒公、平原王。興安二年,追論定策之勳,進先臣爵西平王。皇興季年,顯祖將傳大位於京兆王,先臣時都督諸將屯於武川,被徵詣京,特見顧問。先臣固執不可,顯祖久乃許之,遂命先臣持節授皇帝璽綬於高祖。至太和十六年,麗息睿狀祕書,稱其亡父與先臣援立高宗,朝廷追錄,封睿鉅鹿郡開國公。臣時丁艱草土,不容及例。至二十年,除臣雍州刺史。臨發
 奉辭,面奏先帝,申先臣舊勛。時蒙敕旨,但赴所臨,尋當別判。至二十一年,車駕幸雍,臣復陳聞。時蒙敕旨,徵還當授。自宮車晏駕,遂爾不申。竊惟先臣,遠則授立高宗,寶歷不墜;近則陳力顯祖,神器有歸。如斯之勛,超世之事也。麗以父功,而獲山河之賞;臣有家勳,不霑茅土之錫。得否相懸,請垂裁處。」詔曰:「宿老元臣,云如所訴,訪之史官,頗亦言此。可依授北馮翊郡開國公,食邑九百戶。」



 又詔為使持節,加侍中、行臺,巡行北邊六鎮,恆、北、朔三州,賑給貧乏,兼採風謠,考論殿最,事之得失,先決後聞。自京師遷洛,邊朔遙遠,加以連年旱儉,百姓困弊。懷銜
 命撫導,存恤有方,便宜運轉,有無通濟。時后父于勁勢傾朝野,勁兄子祚與懷宿昔通婚,時為沃野鎮將,頗有受納。將入鎮,祚郊迎道左,懷不與相聞,即劾祚免官。懷朔鎮將元尼須與懷少舊,亦貪穢狼籍。置酒請懷,曰:「命之長短,由卿之口,豈可不相寬貸?」懷曰:「今日之集,乃是源懷與故人飲酒之坐,非鞫獄之所也。明日公庭,始為使人檢鎮將罪狀之處。」尼須揮淚而已,無以對之。既而懷表劾尼須。其奉公不撓,皆此類也。時百姓為豪強陵壓,積年枉滯,一朝見申者,日有百數。所上事宜,便於北邊者,凡三十餘條,皆見嘉納。



 正始元年九月,有告蠕蠕
 率十二萬騎,六道並進,欲直趣沃野、懷朔,南寇恆、代。詔懷以本官加使持節、侍中,出據北蕃。指授規略,隨須徵發,諸所處分,皆以便宜從事。又詔懷子直寢徵隨懷北行。詔賜馬一匹、細鎧一具、御槊一枚。懷拜受既訖,乃於其庭,跨鞍執槊,躍馬大呼。顧謂賓客曰:「氣力雖衰,尚得如此。



 蠕蠕雖畏壯輕老,我亦未便可欺。今奉廟勝之規,總驍捍之眾,足以擒其酋帥,獻俘闕下耳。」時年六十一。懷至雲中,蠕蠕亡遁。旋至恆、代,乃案視諸鎮左右要害之地,可以築城置戍之處,皆量其高下,揣其厚薄,及儲糧積仗之宜,犬牙相救之勢,凡表五十八條,宣武並從
 之。卒,贈司徒公,謚曰惠。



 懷性寬簡,不好煩碎。恆語人曰:「為政貴當舉綱,何必須太子細也!如為屋,但外望高顯,楹棟平正,足矣。斧斤不平,非屋病也。」性不飲酒,而喜以飲人。



 好接賓客,雅善音律,雖在白首,至宴居之暇,常自操絲竹。



 子子邕,字靈和。少好文雅,篤志於學,推誠待士,士多歸之。累遷夏州刺史。



 時沃野鎮人破六韓拔陵首為反亂,統萬逆徒,寇害應接。子邕嬰城自守,城中糧盡,煮馬皮而食之。子邕善綏撫,無有離貳。以饑饉轉切,欲自出求糧,留子延伯據守。



 僚屬僉云,未若棄城俱去,更展規略。子邕泣請於眾曰:「吾世荷國恩,此是吾死地,更
 欲何求!」遂自率羸弱向東夏運糧。延伯與將士送出城,哭而拜辭,三軍莫不嗚咽。子邕為朔方胡帥曹阿各拔所邀,力屈被執。乃密遣人齎書間行與城中云:「大軍在近,汝其奉忠,勿移其操。」子邕雖被囚束,雅為胡人所敬,常以百姓禮事之。子邕為陳安危禍福之端,勸阿各拔令降。將從之,未果而死。拔弟桑生代總部眾,竟隨子邕降。時北海王顥為大行臺,子邕具陳諸賊可滅狀。顥給子邕兵,令其先出。時東夏合境反叛,所在屯結。子邕轉戰而前,九旬之中,凡數十戰,乃平東夏。徵稅租粟,運糧統萬,於是二夏漸寧。及蕭寶夤等為賊所敗,關右騷擾,
 時子邕新平黑城,遂率士馬並夏州募義人,鼓行南出。賊帥康維摩守鋸谷,斷絕棠橋。



 子邕與戰,大破之,禽維摩。又攻破賊帥契官斤於楊氏堡。出自西夏,至於東夏,轉戰千里。至是,朝廷始得委問。除兼行臺尚書。復破賊帥紇單步胡提於曲沃,明帝璽書勞勉之。子邕在白水郡破賊率宿勤明達子阿非軍,多所斬獲。除給事黃門侍郎,封樂平縣公。



 以葛榮久逼信都,詔子邕為北討都督。時相州刺史、安樂王鑒據鄴反,敕子邕與都督李神軌先討平之。改封陽平縣公。遂與裴衍發鄴,討葛榮。而信都城陷,除子邕冀州刺史,與裴衍俱進。子邕戰敗
 而歿,贈司空,謚曰莊穆。



 子邕弟子恭,字靈順,聰敏好學。稍遷尚書北主客郎,攝南主客事。時梁亡人許周自云梁給事黃門侍郎,朝士咸共信待。子恭奏以為真偽難辨,請下徐、楊二州密訪。周果以罪歸闕,詐假職位,如子恭所疑。河州羌卻鐵匆反,詔子恭為行臺討之。子恭示以威恩,兩旬間悉降。朝廷嘉之。正光元年,為行臺左丞,巡北邊。轉為起部郎中。明堂、辟雍並未建就,子恭上書,求加經綜。書奏,從之。稍遷豫州刺史。頻以軍功,加鎮南將軍,兼尚書行臺。元顥之入洛也,加子恭車騎將軍;子恭不敢拒之,而頻遣間使參莊帝動靜。未幾,顥敗,車駕
 還洛,錄前後征討功,封臨潁縣侯,侍中。



 爾朱榮之死也,世隆、度律斷據河橋,詔子恭為都督以討之。尋而太府卿李苗夜燒河橋,世隆退走,以子恭兼尚書僕射,為大行臺、大都督。節閔帝初,以預定策勳,封臨汝縣子。永熙中,入為吏部尚書。以子恭前在豫州戰功,追賞襄城縣男。



 又論子恭餘效,封新城縣子。子恭尋表請轉授第五子文盛,許之。



 天平初,除中書監。三年,拜魏尹,又為齊神武王軍司。卒,贈司空公,謚曰文獻。子彪。



 彪字文宗,學涉機警,少有名譽。魏永安中,以父功賜爵臨潁縣伯。天平四年,為涼州大中正。及齊文襄攝選,沙汰
 臺郎,以文宗為尚書祠部郎中。皇建二年,累遷涇州刺史。文宗以恩信待物,甚得邊境之和,為鄰人所欽服,前政被抄掠者,多被放遣。累遷秦州刺史,乘傳之府,特給後部鼓吹。時李貞聘陳,陳主云:「齊朝還遣源涇州來在瓜步,真可謂通和矣。」武平三年,授祕書監。



 陳將吳明徹寇淮南,歷陽、瓜步相尋失守。趙彥深於起居省密訪文宗討捍之計。



 文宗曰:「國家待遇淮南,失之同於蒿箭。以為宜以淮南委之王琳。琳於曇頊,不肯北面事之明矣。」彥深曰:「弟此良圖。但以口舌爭來十日,已是不見從。時事如此,安可盡言!」因相顧流涕。及齊平,與陽休之等十
 八人入京,授儀同大將軍、司成下大夫。隋開皇中,拜莒州刺史。遇病去官,卒。



 文宗以貴族子弟升朝列,才識敏贍,以幹局見知。然好游貴要之門,時論以為善附會。



 子師,字踐言。少知名,明辯有識悟,尤以吏事自許。仕齊為尚書左外兵郎中,又攝祠部。後屬孟夏,以龍見請雩。時高阿那肱為錄尚書事,謂為真龍出見,大驚喜。問龍所在,云:「作何顏色?」師整容云:「此是龍星初見,依禮當雩祭郊壇,非謂真龍別有所降。」阿那肱忿然作色曰:「漢兒多事,強知星宿!」祭事不行。



 師出,竊歎曰:「國家大事,在祀與戎,禮既廢也,其能久乎?齊亡無日矣!」尋周武帝平齊,授
 司賦上士。



 隋文帝受禪,累遷尚書左丞,以明乾著稱。時蜀王秀頗違法度,乃以師為益州總管司馬。俄而秀被徵,秀恐京師有變,將謝病。師數勸之,不可違命。秀乃作色曰:「此我家事,何預卿也?」師垂涕苦諫,秀乃從征。秀發後,州官屬多相連坐,師以此獲免。後加儀同三司。



 煬帝即位,拜大理少卿。帝在顯仁宮,敕宮外衛士,不得輒離所守。有一主帥,私令衛士出外,帝付大理。師據法奏徒。帝令斬之。師奏曰:「若陛下初便殺之,自可不關文墨。既付有司,義歸恆典。脫宿衛近侍者更有此犯,將何以加之?」帝乃止。



 師居職強明,有口辯,而無廉平之稱。卒於刑
 部侍郎。



 子恭弟纂,字靈秀,位太府少卿。遇害河陰,贈定州刺史。子雄。



 雄字世略,少寬厚,美姿容。初仕魏,歷位秘書郎。在周以伐齊功,封朔方公,歷冀、平二州刺史,檢校徐州總管。及尉遲迥作亂,時雄家累在相州,迥潛以書誘之。雄卒不顧。隋文帝遺書慰勉之。迥遣其將畢義緒據蘭陵,席毗陷昌慮下邑,雄遣眾悉平之。陳人見中原多故,遣其將陳紀、蕭摩訶、任蠻奴、周羅、樊毅等侵江北。自江陵,東距壽陽,人多應之,攻陷城鎮。雄與吳州總管于顗等擊走之。悉復故地。進位上大將軍,拜徐州總管,遷朔州總
 管。平陳之役,從秦王俊出信州道。



 陳平,以功進位上柱國,賜子崇爵端氏縣伯,褒為安化縣伯,復鎮朔方。後歲,上表乞骸骨,徵還京師,卒于家。



 子嵩嗣,大業中,為尚書虞部郎,討北海賊。力戰死之,贈正議大夫。



 劉尼,代人也。曾祖敦,有功於道武,為方面大人。父婁,為冠軍將軍。尼勇果善射,太武見而善之,拜羽林中郎,賜爵昌國子。宗愛既殺南安王餘於東廟,祕之,唯尼知狀。尼勸愛立文成。愛自以負罪於景穆,聞而驚曰:「君大癡人!皇孫若立,豈忘正平時事乎?」尼曰:「若爾,立誰?」愛曰:「待還宮,擢諸王子賢者而立之。」尼懼其有變,密以狀告殿
 中尚書源賀。時與尼俱典兵宿衛,仍共南部尚書陸麗謀,密奉皇孫。於是,賀與尚書長孫渴侯嚴兵守衛,尼與麗迎文成於苑中。



 麗抱文成於馬上,入於京城。尼馳還東廟,大呼曰:「宗愛殺南安王,大逆不道。



 皇孫已登大位。有詔,宿衛之士,皆可還宮。」眾咸唱萬歲。賀及渴侯登執宗愛、賈周等,勒兵而入,奉文成於宮門外,入登永安殿。以尼為內行長,封東安公。尋遷尚書右僕射,為定州刺史。在州清慎,然率多酒醉。文成末,為司徒。獻文即位,以尼有大功於先朝,特加尊重,賜別戶四十。皇興四年,車駕北征,帝親誓眾,而尼昏醉,兵陳不整。帝以其功重,特
 恕之,免官而已。延興四年薨,子社生襲。



 薛提,太原人,皇始中,補太學生,拜侍御史,累遷晉王丕衛兵將軍、冀州刺史,封太原公。有政績,徵拜侍中,攝都曹事。太武崩,祕不發喪,尚書左僕射蘭延、侍中和延等議。以皇孫幼沖,宜立長君,征秦王翰置之祕室。提曰:「皇孫有世嫡之重,人望所係,春秋雖少,令問聞於天下。廢所宜立而更別求,必有不可。」



 延等未決,中常侍宗愛知其謀,矯皇后令,徵提等入,殺之。文成即位,以提有謀立之誠,詔提弟浮子襲先爵太原公,有司奏降為侯。



 論曰:陸俟以智識見稱,珝乃不替風範,雅杖名節,自立
 功名;其傳芳銘典,豈徒然也?麗忠國奉主,鬱為梁棟。資忠履義,赴難如歸,世載克昌,名不虛得。



 睿、琇以沈雅顯達,何末亦披猖?子彰令終之美,家聲孔振。仰及彥師俱以孝為本,出處之譽,並可作範人倫。爽學業有聞,亦人譽也。源賀堂堂,非徒武節、觀其翼佐文成,廷抑禪讓,殆乎社稷之臣。懷幹略兼舉,出內馳譽,繼跡賢孝,不墮先業。



 子邕功立夏方,身亡冀野。彪著名齊朝。師、雄官成隋代,美矣。劉尼忠國,豈徒驍猛之用?薛提正議忠謀,見害奸閹,痛乎!



\end{pinyinscope}