\article{卷二十三列傳第十一}

\begin{pinyinscope}

 於
 栗磾孫勁六世孫謹謹子寔寔(子顗)仲文寔弟翼翼子璽翼弟義宣敏於栗磾,代人也。少習武藝,材力過人,能左右馳射。登國中,拜冠軍將軍,假新安子。與寧朔將軍公孫蘭,潛自太原,從韓信故道,開井陘關路,襲慕容寶於中山。道武後至,見道路修理,大悅,即賜其名馬。及趙魏平,帝置酒高會,謂栗磾曰:「卿,吾之黥、彭也!」進假新安公。道武田於白
 登山,見熊將數子,顧栗磾曰:「能搏之乎?」對曰:「若搏之不勝,豈不虛斃一壯士!自可驅致御前,坐而制之。」尋皆禽獲,帝顧而謝之。



 後為河內鎮將。劉裕之伐姚泓,慄磾慮北侵擾,築壘河上。裕憚之,遺栗磾書,假道西上。題書曰:「黑槊公麾下」。慄磾以狀表聞,明元因之授慄磾黑槊將軍。



 慄磾好持黑槊,裕望而異之,故有其號。遷豫州刺史,進爵新安侯。洛陽雖歷代所都,實為邊界,栗磾勞來安集,甚得百姓心。明元南幸盟津,謂慄磾曰:「河可橋乎?」栗磾曰:「杜預造橋,遺事可想。」乃編大船,構橋於野阪。六軍既濟,帝深歎美之。



 太武之徵赫連昌,敕慄磾與宋兵將
 軍周幾襲陜城,長驅至三輔。進爵為公。累遷外都大官,平刑折獄,甚有聲稱。卒,贈太尉。栗磾自少總戎,迄於白首,臨事善斷,所向無前。加以謙虛下士,刑罰不濫,太武甚悼惜之。



 子洛拔,有姿容,善應對。拜侍御中散。太武甚加愛寵,因賜名焉。轉監御曹令。景穆在東宮,厚加禮遇。洛拔恒畏避屏退,不敢逆自結納。頃之,襲爵。後為侍中、尚書令,百僚憚之。卒官。洛拔有六子。



 長子烈,善射,少言,有不可犯之色。少拜羽林中郎,累遷侍中、殿中尚書。



 于時孝文幼沖,文明太后稱制。烈與元丕、陸睿、李沖等各賜金策,許以有罪不死。



 進爵洛陽侯,轉衛尉卿。及遷都
 洛陽,人情戀本,多有異議。帝以問烈。曰:「陛下聖略深遠,非愚管所測。若隱心而言,樂遷之與戀舊,中半耳。」帝曰:「卿不唱異同,朕深感不言之益。」敕鎮代,留臺庶政,一相參委。車駕幸代,執烈手曰:「宗廟至重,翼衛不輕。卿當祗奉靈駕,時遷洛邑。」烈與高陽王雍奉神主於洛陽,遷光祿卿。



 十九年,大選百僚,烈子登引例求進。烈表引己素無教訓,請乞黜落。帝曰:「此乃有識之言,不謂烈能辨此!」乃引見登,詔曰:「朕今創禮新邑,明揚天下,卿父乃行謙讓之表,而有直士之風,故進卿為太子翊軍校尉。」又加烈散騎常侍,封聊城縣子。



 及穆泰、陸睿謀反舊京,帝幸
 代,泰等伏法。賜烈及李沖璽書,述敘金策之意。



 時代鄉舊族,同惡者多,唯烈一宗,無所染豫。帝益器重之。歎曰:「元儼決斷威恩,深自不惡,然盡忠猛決,不如烈也。爾日烈在代都,必即斬其五三元首。烈之節概,不謝金日磾。」詔除領軍將軍。以本官從征荊沔,加鼓吹一部。



 二十三年,齊將陳顯達入寇馬圈,帝輿疾討之。執烈手,以京邑為託。帝崩於行宮,彭城王勰秘諱而返。稱詔召宣武會駕魯陽。以烈留守之重,密報凶問。烈處分行留,神守無變。



 宣武即位,寵任如前。咸陽王禧為宰輔,權重當時。曾遣家僮傳言於烈,求舊羽林武賁執仗出入。烈不許。禧
 遣謂烈曰:「我是天子兒,天子叔,元輔之命,與詔何異?」烈厲色答曰:「向亦不道王非天子兒、叔。若是詔,應遣官人所由。若遣私奴索官家羽林,烈頭可得,羽林不可得也!」禧惡烈剛直,出之為恆州刺史。



 烈不願籓授,謂彭城王勰曰:「殿下忘先帝南陽之詔乎?而逼老夫乃至於此!」遂以疾辭。



 宣武以禧等專擅,潛謀廢之。景明二年正月,礿祭,三公致齋於廟。帝夜召烈子忠謂曰:「卿父明可早入。」及明,烈至。詔曰:「諸父慢怠,今欲使卿以兵召之,卿其行乎?」烈曰:「老臣歷奉累朝,頗以幹勇賜識。今日之事,所不敢辭。」



 乃將直閣以下六十餘人,宣旨召咸陽王禧、彭城
 王勰、北海王詳,衛送至帝前。諸公各稽首歸政。以烈為領軍,進爵為侯,自是長直禁中,機密大事皆所參焉。



 咸陽王禧之謀反,宣武從禽於野。左右分散,倉卒之際,莫知其計。乃敕烈子忠馳覘虛實。烈時留守,已處分有備。因忠奏曰:「臣雖朽邁,心力猶可。禧等猖狂,不足為慮。願緩蹕徐還,以安物望。」帝甚以為慰。車駕還宮,禧已逃,詔烈追執之。



 順后既立,以世父之重,彌見優禮。及卒,宣武舉哀於朝堂,給東園第一祕器,贈太尉,封鉅鹿郡公。子祚襲。



 祚弟忠,字思賢,本字千年。弱冠,拜侍御中散。文明太后臨朝,刑政頗峻,侍臣左右,多以微譴得罪。忠朴直
 少言,終無過誤。太和中,授武騎侍郎,因賜名登。累遷左中郎將,領直寢。元禧之亂,車駕在外,變起倉卒。忠曰:「臣父為領軍,計必無所慮。」帝遣忠馳觀之,烈嚴備,果如所量。忠還,宣武撫其背曰:「卿差彊人意。先帝賜卿名登,誠為美稱。朕嘉卿忠款,今改名忠,既表貞固之誠,亦以名實相副也。」以父憂去職。徙為司空長史。



 時太傅、錄尚書、北海王詳親尊權重,將作大匠王遇多隨詳所欲而給之。忠於詳前謂遇曰:「殿下國之周公,阿衡王室,何至阿諛附勢,損公惠私也?」遇既不寧,詳亦慚謝。以平元禧功,封魏郡公。及遷散騎常侍、兼武衛將軍,每以鯁氣正辭,
 為北海所忿。面責忠曰:「我憂在前見爾死,不憂爾見我死時也。」忠曰:「人生自有定分,若應死王手,避亦不免;不爾,王不能殺。」詳因忠表讓之際,密勸帝以忠為列卿,於是詔停其封,優進太府卿。



 正始二年,詔忠以本官使持節、兼侍中,為西道大使。刺史、鎮將贓罪顯暴者,以狀聞。守令以下,便行決斷。與尚書李崇分使二道。忠劾并州刺史高聰贓罪二百餘條,論以大辟。除華州刺史。遭繼母憂,不行。服闋,再遷衛尉卿、河南邑中正。



 忠與吏部尚書元暉、度支尚書元匡、河南尹元萇等推定代方姓族。高肇忌其為人,乃言於宣武,稱中山要鎮,作捍須才,乃
 出忠為定州刺史。既而帝悔,復授衛尉卿、領左衛將軍、恒州大中正,密遣使詣忠,慰勉之。延昌初,除都官尚書,領左衛、中正如故。又加散騎常侍。嘗因侍宴,賜之劍杖,舉酒屬忠曰:「卿世執貞節,故恒以禁衛相委。昔以卿行忠,賜名曰忠。今以卿才堪禦侮,以所御劍杖相錫。循名取義,意在不輕,出入恒以自防也。」遷侍中、領軍將軍。忠辭無學識,宣武曰:「學識有文章者不少,但心直不如卿。欲使卿劬勞於下,我當無憂於上。」



 及帝崩夜,忠與侍中崔光遣右衛將軍侯剛迎明帝於東宮而即位。忠與門下議,以帝沖年,未親機政,太尉高陽王雍屬尊望重,宜
 入居西柏堂,省決庶政;任城王澄明德茂親,可為尚書令,總攝百揆。奏中官,請即敕授。御史中尉王顯欲逞奸計,與中常侍、給事中孫蓮等厲色不聽,寢門下之奏。孫蓮等密欲矯太后令,以高肇錄尚書事,顯與高猛為侍中。忠即殿中收顯殺之。



 忠既居門下,又總禁衛,遂執朝政,權傾一時。初,太和中,軍國多事,孝文以用不足,百官祿四分減一。忠既擅朝,欲以惠澤自固,乃悉復所減之祿,職人進位一級。舊制:百姓絹布一匹之外,各輸綿麻八兩。忠悉以與之。乃白高陽王雍,自云宣武本許優轉。雍憚忠威權,便順意加忠車騎大將軍。忠自謂新故之
 際,有安社稷功,諷百僚令加己賞。太尉雍、清河王懌、廣平王懷難違其意,封忠常山郡公。



 忠又難於獨受,乃諷朝廷,同在門下者加封邑。尚書左僕射郭祚、尚書裴植以忠權勢日盛,勸雍出忠。忠聞之,逼有司誣奏其罪。祚有師傅舊恩,植擁地入國,忠並矯詔殺之。朝野憤忿,無不切齒。王公以下,畏之累跡。又欲殺高陽王雍,侍中崔光固執乃止,遂免雍太尉,以王還第。自此詔命生殺,皆出於忠。既尊靈太后為皇太后,居崇訓宮,忠為儀同三司、尚書令、崇訓衛尉,侍中、領軍如故。



 靈太后臨朝,解忠侍中、領軍、崇訓衛尉,止為儀同、尚書令、侍中。忠為令旬
 餘,靈太后引門下侍官,問忠在端右聲聽。咸曰不稱厥任,乃出為冀州刺史。太傅清河王等奏:「忠擅殺樞納,輒廢宰輔,朝野駭心,遠近怪愕。功過相除,悉不合賞,請悉追奪。」靈太后從之。



 熙平元年,御史中尉元匡奏:「忠以鴻勳盛德,受遇累朝,幸國大災,專擅朝命,無人臣之心。裴、郭受冤於既往,宰輔黜辱於明世。又自矯旨為儀同三司、尚書令、領崇訓衛尉。原其此意,便欲無上自處。既事在恩後,宜加顯戮。請遣御史一人、令史二人,就州行決。」靈太后令,以忠事經肆眚,遂不追罪。又詔以忠歷任禁要,誠節皎然,賜爵靈壽縣公。



 初,宣武崩後,高太后將害
 靈太后。劉騰以告侯剛,剛以告忠。忠請計於崔光。



 光曰:「宜置胡嬪於別所,嚴兵守衛。」忠從之,具以此意啟靈太后,太后意乃安。



 故太后深德騰等四人,並有寵授。



 忠以毀之者多,懼不免禍,願還京,欲自營救。靈太后不許。二年四月,除尚書右僕射,加侍中,將軍如故。



 神龜元年三月,復儀同三司。疾,未拜。見裴、郭為祟,自知必死,先表養亡弟第二子司徒掾永超為子,乞以為嫡。靈太后許之。薨,贈司空。有司奏太常少卿元端議:「案謚法,剛強理直曰武,怙威肆行曰鬼,宜謚武鬼公。」太常卿元修義議:「忠盡心奉上,翦除凶逆。依謚法,除偽寧直曰武,夙夜恭事
 曰敬,宜謚武敬公。」二卿不同。靈太后令依正卿議。



 忠性多阻忌,不交勝己,唯與直閣將軍章初環、千牛備身楊保元為斷金之交。



 李世哲求寵於忠,私以金帛貨初環、保元,二人談之,遂被賞愛,引為腹心。忠擅權昧進為崇訓之由,皆世哲計也。



 忠弟景,字百年。忠薨後,為武衛將軍。謀廢元叉,叉黜為懷荒鎮將。及蠕蠕主阿那瑰叛,鎮人請糧,景不給。鎮人遂執縛景及其妻,拘守別室,皆去其衣服,令景著皮裘,妻著故絳旗襖,毀辱如此。月餘,乃殺之。



 烈弟果,嚴毅直亮,有父兄風。歷朔、華、并、恒四州刺史,賜爵武城子。



 果弟勁。
 勁字鐘葵,頗有武略,位沃野鎮將,賜爵富昌子。宣武納其女為后,封勁太原郡公,妻劉氏為章武郡君。後為征北將軍、定州刺史。卒,贈司空,謚曰恭莊公。自栗磾至勁,累世貴盛,一皇后,四贈公,三領軍,二尚書令,三開國公。



 勁雖以后父,但以順后早崩,竟不居公輔。



 子暉,字宣明,后母弟也。少有氣幹。襲爵,位汾州刺史。暉善事人,為爾朱榮所親,以女妻其子長儒。歷侍中、河南尹。後兼尚書僕射、東南道行臺,與齊神武討平羊侃於兗州。元顥入洛,害之。



 勁弟天恩,位內行長、遼西太守。贈平東將軍、燕州刺史。天恩子仁生,位太中大夫。仁生子
 安定,平原郡太守、高平郡都將。安定子子提,隴西郡守、茂平縣伯。周保定二年,以子謹著勛,追贈太保、建平郡公。



 謹字思敬,小名巨引。沈深有識量,略窺經史,尤好《孫子》兵書。屏居未有仕進志。或有勸之者,謹曰:「州郡之職,昔人所鄙,台鼎之位,須待時來。」太宰元天穆見之,歎曰:「王佐材也。」及破六韓拔陵首亂北境,引蠕蠕為授,大行臺元纂討之。夙聞謹名,辟為鎧曹參軍事,從軍北伐。蠕蠕逃出塞,纂令謹追之,前後十七戰,盡降其眾。後率輕騎出塞覘賊,屬鐵勒數千騎奄至。謹以眾寡不敵,乃散其
 騎,使匿叢薄間。又遣人升山指麾,若分部軍眾。賊望見,雖疑有伏,恃眾不以為慮,乃進逼謹。以常乘駿馬一紫一騧,賊先所識,乃使二人各乘一馬,突陣而出。賊以為謹,爭逐之。乃率餘軍擊其追騎。賊走,因得入塞。



 正光四年,行臺、廣陽王元深北伐,引謹為長流參軍。特相禮接,使其世子佛陁拜焉。遂與廣陽破賊主斛律野穀祿等。謹請馳往喻之。謹兼解諸國語,乃單騎入賊,示以恩信,於是西部鐵勒酋長也列河等三萬餘戶並款附,相率南遷。廣陽與謹至析郭嶺迎接之。謹曰:「拔陵兵眾不少,聞也列河等款附,必來要擊。彼若先據險,則難與爭
 鋒。今以也烈河等鉺之,當競來抄掠,然後設伏而待,必指掌破之。」



 廣陽然其計。拔陵果來要擊,破也列河於嶺上,部眾皆沒。謹伏兵發,賊大敗,悉破收也列河之眾。



 孝昌元年,又隨廣陽王征鮮于修禮。軍次白斗牛邏。會章武王為修禮所害,遂停軍中山。侍中元晏宣言於靈太后曰:「廣陽盤桓不進,坐圖非望。又有于謹者,智略過人,為其謀主,恐非陛下純臣。」靈太后詔於尚書省門外立榜,募獲謹者,許以重賞。謹聞之,請詣闕披露腹心,廣陽許之。謹遂到榜下,曰:「吾知此人。」



 眾共詰之,謹曰;「我即是也。」有司以聞。靈后見之,大怒。謹備述廣陽忠款,兼陳停
 軍之狀。靈后遂捨之。後從爾朱天光與齊神武戰於韓陵山,天光敗,謹遂入關。



 周文帝臨夏州,以謹為防城大都督,兼夏州長史。及賀拔岳被害,周文赴平涼。



 謹言於周文曰:「關中秦漢舊都,古稱天府。今若據其要害,招集英雄,足觀時變。



 且天子在洛,逼迫群兇。請都關右,然後挾天子而令諸侯,千載一時也。」周文大悅。會有敕追謹為關內大都督,謹因進都關中策。魏帝西遷,仍從周文征潼關,破回洛城,授北雍州刺史,進爵藍田縣公。大統三年,大軍東伐,為前鋒,進拔弘農,禽東魏陜州刺史李徽伯。神武至沙苑,謹力戰,進爵常山郡公。又從戰河橋,
 拜大丞相府長史,兼大行臺尚書。再遷太子太保。芒山之戰,大軍不利,謹率麾下偽降,立於路左。神武乘勝逐北,不以為虞。謹自後擊之,敵人大駭。獨孤信又收兵於後奮擊,神武軍亂,以此大軍得全。十二年,拜尚書左僕射,領司農卿。及侯景款附,請兵為援,謹諫以為景情難測,周文不聽。尋兼大行臺尚書、大丞相長史,率兵鎮潼關,加授華州刺史,賜秬鬯一卣,珪瓚副焉。俄拜司空。恭帝元年,除雍州刺史。



 初,梁元帝於江陵嗣位,密與齊交通,將謀侵軼。其兄子岳陽王詧時為雍州刺史,以梁元帝殺其兄譽,逐結隙,據襄陽來附。乃命謹出討。周文餞
 於青泥谷。長孫儉曰:「為蕭繹計將如何?」謹曰:「曜兵漢沔,席卷度江,直據丹陽,是其上策。移郭內居人,退保子城,以待援至,是其中策。若難於移動,據守羅郭,是其下策。」儉曰:「裁繹出何策?」謹曰:「必用下。」儉曰:「何也?」對曰:「蕭氏保據江南,綿歷數紀。屬中原有故,未遑外略。又以我有齊氏之患,必謂力不能分。且繹懦而無謀,多疑少斷。愚人難與慮始,皆戀邑居,既惡遷移,當保羅郭。所以用下策。」謹令中山公護及大將軍楊忠等先據江津,斷其走路。梁人豎木柵於外城,廣輪六十里。尋而謹至,悉眾圍之。旬有六日,外城遂陷,梁主退保子城。翌日,率其太子以
 下,面縛出降。尋殺之。虜其男女十餘萬人,收其府庫珍寶。



 得宋渾天儀、梁日晷、銅表、魏相風烏、銅蟠螭趺、大玉徑西尺圍七尺及諸輿輦法物以獻,軍無私焉。立蕭詧為梁主,振旅而旋。周文親至其第,宴語極歡。賞謹奴婢一千口。及梁寶物,并金石絲竹樂一部,別封新野郡公。謹固辭,不許。又令司樂作《常山公平梁歌》十首,使工人歌之。



 謹自以久當權重,功名既立,願保優閑,乃上先所乘駿馬及所著鎧甲等。周文識其意,曰:「今巨猾未平,公豈得便爾獨善?」遂不受。六官建,拜大司寇。



 及周文崩,孝閔帝尚幼,中山公護雖受顧命而名位素下,群公各圖
 執政。護深憂之,密訪於謹。謹曰:「夙蒙丞相殊眷,今日必以死爭之。若對眾定策,公必不得讓。」明日,群公會議。謹曰:「昔帝室傾危,丞相志存匡救。今上天降禍,奄棄百寮。嗣子雖幼,而中山公親則猶子,兼受顧託,軍國大事,理須歸之。」辭色抗厲,眾皆悚動。護曰:「此是家事,護何敢有辭!」謹既周文等夷,護每申禮敬。



 至是,謹乃起而言曰:「公若統理軍國,謹等便有所依。」遂再拜。群公迫於謹,亦拜。眾議始定。



 孝閔踐阼,進封燕國公,邑萬戶,遷太傅、太宗伯,與李弼、侯莫陳崇等參議朝政。及賀蘭祥討吐谷渾,明帝令謹遙統其軍,授以方略。



 保定二年,謹以年老,乞
 骸骨,優詔不許。



 三年,以謹為三老,固辭,又不許。賜延年杖。武帝幸太學以食之。三老入門,皇帝迎拜屏間,三老答拜。有司設三老席於中楹,南向。太師、晉公護升階,設席施幾。三老升席,南面馮几而坐,師道自居。大司冠、楚國公寧升階,正舄。皇帝升,立於斧扆之前,西面。有司進饌,皇帝跪設醬豆,親自袒割。三老食訖。皇帝又親跪授爵以酳。有司撤訖。皇帝北面立訪道。三老乃起立於席。皇帝曰:「猥當天下重任,自惟不才,不知政術之要,公其誨之。」三老答曰:「木從繩則正,君從諫則聖。自古明王聖主,皆虛心納諫,以知得失,天下乃安。惟陛下念之。」又曰:「
 為國之本,在乎忠信。古人去食去兵,信不可失。國家興廢,莫不由之,願陛下守而勿失。」又曰:「為國之道,必須有法。法者,國之綱紀,不可不正。所正在於賞罰。若有功必賞,有罪必罰,則為善者日益,為惡者日止。若有功不賞,有罪不罰,則天下善惡不分,下人無所措其手足。」又曰:「言行者,立身之基,言出行隨,誠願陛下慎之。」三老言畢,皇帝再拜受之,三老答拜,禮成而出。



 及晉公護東伐,謹時有病。護以其宿將舊臣,猶請與同行,詢訪戎略。軍還,賜鐘磬一部。天和二年,又賜安車一乘。尋授雍州牧。三年,薨,年七十六。武帝親臨。詔譙王儉監護喪事,賜繒千
 段、粟夢千斛,贈本官,加使持節、太師、雍恒等二十州諸軍事、雍州刺史,謚曰文。及葬,王公以下,咸送郊外。配享於文帝廟庭。



 謹有智謀,善於事上。名位雖重,愈存謙挹,每朝參往來,不過從兩三騎而已。



 朝廷凡有軍國之務,多與謹決。謹亦竭其智能,故功臣中特見委信,始終若一,人無間言。每誡諸子,務存靜退。加以年齒遐長,禮遇隆重,子孫繁衍,皆至顯達,當時莫比。子寔嗣。



 寔字賓實,少和厚,以軍功封萬年縣子。大統十四年,累遷尚書。是歲,周文帝與魏太子西巡,寔時從行。周文刻石隴山上,錄功臣名位,以次鐫勒,預以寔為開府儀同
 三司,至十五年方授之。尋除渭州刺史,特給鼓吹一部,進爵為公。魏恭帝二年,羌東令姐率部落反,西連吐谷渾。大將軍豆盧寧討之,踰時不剋。又令寔往,遂破之。周文手書勞問,賜奴婢一百口,馬百匹。



 孝閔帝踐阼,授戶部中大夫,進爵延壽郡公。天和二年,延州蒲川賊郝三郎反,攻丹州。遣寔討平之,仍除延州刺史。五年,襲燕國公,進位柱國。以罪免。尋復本官,除涼州總管。大象二年,加上柱國,拜大左輔。隋開皇元年,薨,贈司空,謚曰安。子顗。



 顗字元武,身長八尺,美鬚眉。周大塚宰宇文護見而器
 之,以女妻之。以父勛,賜爵新野郡公。歷左右宮伯、郢州刺史。大象中,以水軍總管從韋孝寬經略淮南。



 尉遲迥之反,時總管趙文表與顗素不協,顗將圖之,因臥閣內,詐疾。文表獨至,顗殺之。因言文表與迥通謀,其麾下無敢動者。時隋文帝以迥未平,慮顗復生邊患,因宥免之,即拜吳州總管。以頻敗陳師,賜採數百段。及隋受禪,文表弟詣闕稱兄無罪。上令按其事,太傅竇熾等議顗當死。上以其門著勳績,特原之,貶為開府。



 後襲爵燕國公。尋拜澤州刺史。免,卒于家。子世虔。顗弟仲文。



 仲文字次武,少聰敏,髫齔就學,耽習不倦。父寔異之,曰:「
 此兒必興吾宗。」



 九歲,嘗於雲陽宮見周文帝。問曰:「聞兒好讀書,書有何事?」對曰:「資父事君,忠孝而已。」周文甚嗟嘆之。後就博士李詳受《周易》、《三禮》,略通大義。



 及長,倜儻有大志,氣調英拔。起家為趙王屬,安固太守。有任、杜兩家各失牛,後得一牛,兩家俱認,州郡久不決。益州長史韓伯俊曰:「于安固少年聰察,可令決之。」仲文曰:「此易解耳。」乃令二家各驅牛群至,乃放所認者,牛遂向任氏群中。又使人微傷其牛,任氏嗟惋,杜氏自若。仲文遂訶詰杜氏,服罪而去。始州刺史屈突尚,宇文護之黨也。先坐事下獄,無敢繩者。仲文至郡,窮之,遂竟其獄。



 蜀中語曰:「
 明斷無雙有于公,不避彊禦有次武。」徵為御正下大夫,封延壽郡公,以勛授儀同三司。



 宣帝時,為東郡太守。及尉遲迥作亂,使誘仲文,仲文拒之。迥遣儀同宇文威攻之。仲文迎擊,大破威,以功授開府。迥又遣其將宇文胄度石濟。宇文威、鄒紹自白馬,二道俱進,復攻仲文。郡人赫連僧伽、敬子哲率眾應迥。仲文自度不能支,棄妻子,潰圍而遁,達於京師。迥屠其三子一女。隋文帝引入臥內,為之下泣,賜採五百段,黃金二百兩。進位大將軍,領河南道行軍總管,給鼓吹。馳傳詣洛陽發兵,討迥將檀讓。



 時韋孝寬拒迥於永橋。仲文詣之,有所計議。總管宇
 文忻頗有自疑心,因謂仲文曰:「尉遲迥誠不足平,正恐事寧後,更有藏弓之慮。」仲文懼忻生變,謂曰:「丞相寬仁大度,明識有餘,仲文在京三日,頻見三善,非常人也。」忻曰:「三善何如?」仲文曰:「有陳萬敵新從賊中來,丞相即令其弟難敵召募鄉曲,從軍討賊。此大度一也。上士宋謙奉使勾檢,謙緣此別求他罪。丞相責之曰:『入綱者自可推求,何須別訪,以虧大體。』此不求人私二也。言及仲文妻子,未嘗不潸泫。



 此有仁心三也。」忻自是遂安。



 仲文軍至汴州東,頻破迥將。進攻梁郡,迥守將劉孝寬棄城走。初,仲文在蓼堤,諸將皆曰:「軍自遠來,疲弊不可決戰。」仲
 文令趣食列陳,既而破賊。諸將問其故,笑曰:「吾所部將士皆山東人,果於速進,不宜持久。乘勢擊之,所以制勝。」諸將皆曰:「非所及也。」進擊曹州,獲迥所署刺史李仲康及上儀同房勁。



 檀讓以餘眾屯成武,謂仲文未能卒至,方椎牛饗士。仲文選騎襲之,遂拔成武。迥將席毗羅,眾十萬,屯沛縣,將攻徐州。其妻子在金鄉。仲文遣人詐作毗羅使,謂金鄉城主徐善凈曰:檀讓明日午時到金鄉,將宣蜀公令,賞將士。」金鄉人謂為信然,皆喜。仲文簡精兵,偽建迥旗幟。善凈以為檀讓至,出城迎謁。仲文執之,遂取金鄉。諸將勸屠之,仲文曰:「當寬其妻子,其兵可自
 歸。如即屠之,彼皆絕矣。」



 眾皆稱善。於是毗羅恃眾來薄官軍,仲文背城結陣,設伏,兵發,俱拽柴鼓譟。毗羅軍潰,皆投洙水死,水為不流。獲檀讓,檻送京師,河南悉平。毗羅匿榮陽人家,執斬之,傳首闕下。勒石紀功,樹於泗上。入朝京師,文帝引入臥內,宴享極歡。



 賜雜採千段,妓女十人,拜柱國。屬文帝受禪,不行。



 未幾,其叔父太尉翼坐事下獄,仲文亦為吏所簿,於獄中上書曰:「曩者尉迥逆亂,所在景從。臣任處關河,地居衝要,嘗詹枕戈,誓以必死。迥時購臣,位大將軍,邑萬戶。臣不顧妻子,不愛身命。冒白刃,潰重圍,三男一女,相繼淪沒。



 披露肝膽,馳赴闕
 庭。蒙陛下授臣以高官,委臣以兵革。于時河南兇寇,狼顧鴟張。



 臣以羸兵八千,掃除氛祲。摧劉寬於梁郡,破檀讓於蓼堤;平曹州,復東郡,安成武,定永昌;解亳州圍,破徐州賊。席毗羅十萬之眾,一戰土崩。河南蟻聚之徒,應時戡定。當群兇問鼎之際,生靈乏主之辰,臣第二叔翼先在幽州,總馭燕、趙。



 南鄰群寇,北掃旄頭,內安外撫,得免罪戾。臣第五叔智建CM黑水,與王謙為鄰,式遏蠻陬,鎮綏蜀道。臣兄顗作牧淮南,坐制勍敵,乘機剿定,傳首京師。王謙竊據二州,叛換三蜀。臣第三叔義受脤廟庭,恭行天罰。自外父叔兄弟,皆當文武重寄。或銜命危難,
 或侍衛鉤陣,合門誠款,冀有可明。伏願垂泣辜之恩,降雲雨之施,則寒灰更然,枯骨還肉。」上覽表,并翼釋之。



 明年,拜行軍元帥,統十二州總管以擊胡。出服遠鎮,遇虜,破之。於是從金河出白道。遣總管辛明瑾、元滂、賀蘭志、呂楚、段諧等二萬人出盛樂道,趣那頡山。至護軍州北,與虜遇。可汗見仲文軍容整肅,不戰而退。仲文踰山追之。及還,上以尚書省文簿繁雜,吏多奸詐,令仲文勘錄省中事,所發擿甚多。上嘉其明斷,厚加勞賞。上每憂轉運不給,仲文請決渭水,開漕渠。上然之,使仲文總其事。及伐陳之役,拜行軍總管。高智慧等作亂江南,仲文復
 以行軍總管討之。時三軍乏食,米粟踴貴,仲文私糶軍糧,坐除名。明年,復官爵,率兵屯馬邑以備胡。晉王廣以仲文有將領才,每常屬意,至是奏之,乃令督晉王軍府事。後突厥犯塞,晉王為元帥,使仲文將前軍,大破賊而還。



 煬帝即位,遷左翊衙大將軍,參掌文武選事。從帝討吐谷渾,進位光祿大夫,甚見親重。遼東之役,仲文率軍指樂浪道。次烏骨城,仲文簡羸馬驢數千,置於軍後,既而率眾東過。高麗出兵掩襲輜重,仲文回擊,大破之。至鴨綠水,高麗將乙支文德詐降,來入其菅。仲文先奉密旨,若遇高元及文德者,必禽之。至是,文德來,仲文將執
 之。時尚書右丞劉士龍為慰撫使,固止之。仲文遂捨文德。尋悔,遣人紿文德曰:「更有言議,可復來也。」文德不從,遂濟。仲文選騎度水追之,每戰破賊。文德遺仲文詩曰:「神策究天文,妙算窮地理。戰勝功既高,知足願云止。」



 仲文答書諭之,文德燒柵而遁。時宇文述以糧盡欲還,仲文議以精銳追文德,可以有功。述固止之,仲文怒曰:「將軍杖十萬之眾,不能破小賊,何顏以見帝?且仲文此行也,固無功矣!」述因厲聲曰:「何以知無功?」仲文曰:「昔周亞夫之為將也,見天子,軍容不變。此決在一人,所以功成名遂。今者人各其心,何以赴敵?」



 初,帝以仲文有計畫,令
 諸軍諮稟節度,故有此言。由是述等不得已而從之。



 遂行,東至薩水。宇文述以兵餒退歸,師遂敗績。帝以屬吏,諸將皆委罪於仲文。



 帝大怒,釋諸將,獨繫仲文。仲文憂恚發病,困篤,方出之。卒於家,時年六十八。



 撰《漢書刊繁》三十卷、《略覽》三十卷。有子九人,欽明最知名。



 寔弟翼,字文若,美風儀,有識度。年十一,尚文帝女平原公主,拜員外散騎常侍,封安平縣公。大統十六年,進爵郡公,加大都督,領文帝帳下左右,禁中宿衛。遷武衛將軍。謹平江陵,所賜得軍實,分給諸子。翼一無所取,唯簡賞口內名望子弟有士風者,別待遇之。文帝聞之,賜奴
 婢二百口,翼固辭不受。尋授車騎大將軍、開府儀同三司。六官建,除左宮伯。



 孝閔帝踐阼,出為渭州刺史。翼兄寔先蒞此州,頗有惠政。翼又推誠布信,事存寬簡,夷夏感悅,比之大小馮君焉。時吐谷渾入寇河右,涼、鄯、河三州咸被攻圍,使來告急。秦州都督遣翼赴援,不從,寮屬咸以為言。翼曰:「攻取之術,非夷俗所長。此寇之來,不過鈔掠邊牧耳,安能頓兵城下,久事攻圍!掠而無獲,勢將自走。勞師以往,亦無所及。翼揣之已了,幸勿復言。」數日,問至,果如翼所策。賀蘭祥討吐谷渾,翼率州兵,先鋒深入,以功增邑。尋徵拜右宮伯。



 明帝雅愛文史,立麟趾學,
 在朝有藝業者,不限貴賤,皆聽預焉。乃至蕭捴、王褒等與卑鄙之徒同為學士。翼言於帝曰:「捴,梁之宗子;褒,梁之公卿,今與趨走同躋,恐非尚賢貴爵之義。」帝納之,詔翼定其班次,於是有等差矣。



 明帝崩,翼與晉公護同受遺詔,立武帝。保定元年,徙軍司馬。三年,改封常山郡公。天和初,遷司會中大夫。三年,皇后阿史那氏至自突厥,武帝行親迎之禮,命翼總司儀制。狄人雖蹲踞無節,然咸憚翼之禮法,莫敢違犯。遭父憂去職,居喪過禮,為時輩所稱。尋有詔起令視事。武帝又以翼有人倫之鑒,皇太子及諸王等相傅以下,並委翼選置。其所擢用,皆民
 譽也,時論僉謂得人。遷大將軍,總中外宿衛兵事。



 晉公護以帝委翼腹心,內懷猜忌,轉為小司徒,加拜柱國。雖外示崇重,實疏斥之。及誅護,帝召翼,遣往河東取護子中山公訓,仍代鎮蒲州。翼曰:「冢宰無君陵上,自取誅夷。元惡既除,餘孽宜殄。然皆陛下骨肉,猶謂疏不間親。陛下不使諸王,而使臣異姓,非直物有橫議,愚臣亦所未安。」帝然之,乃遣越王盛代翼。



 先是,與齊、陳二境,各修邊防,雖通聘好,而每歲交兵。然一彼一此,不能有所克獲。武帝既親萬機,將圖東討,詔邊城鎮並益儲峙,加戍卒。二國聞之,亦增修守御。翼諫曰:「疆埸相侵,互有勝敗,徒
 損兵儲,非策之上者。不若解邊嚴,減兵防,繼好息人,敬待來者。彼必喜於通和,懈而無備,然後出其不意,一舉而山東可圖。」帝納之。



 建德二年,出為安州總管。時大旱,溳水絕流。舊俗每逢亢旱,禱白兆山祈雨。



 帝先禁群祀,山廟已除。翼遣主簿祭之,即日澍雨。歲遂有年。百姓感之,聚會歌舞頌之。



 四年,武帝將東伐,朝臣未有知者。遣納言盧韞前後三乘驛詣翼問策。翼贊成之。及軍出,詔翼自宛,葉趣襄城,旬日下齊一十九城。所過秋毫無犯。所部都督輒入人村,即斬以徇。由是百姓欣悅,赴者如歸。屬帝有疾,班師,翼亦旋鎮。轉宜陽總管。以宜陽地非
 襟帶,請移鎮於陜。詔從之,仍除陜州刺史,總管如舊。其年,大軍復東討,翼自陜入,徑到洛陽。齊洛州刺史獨孤承業開門降,河南九州三十鎮,一時俱下。襄城人庶等喜復見翼,並壺漿道左。除河陽總管,仍徙豫州。陳將魯天念久圍光州,聞翼到汝南,望風退散。



 大象初,徵拜大司徒。詔翼巡長城,立亭鄣。西自雁門,東至碣石,創新改舊,咸得其要害。仍除幽州總管。先是,突厥屢為抄掠,居人失業。翼素有威武,兼明斥候,自是不敢犯塞,百姓安之。及尉遲迥據相州舉兵,以書招翼。翼執其使,并書送之。時隋文帝執政,賜翼雜繒一千五百段,并珍寶服玩
 等。進位上柱國,封任國公,增邑通前五千戶,別食任城縣一千戶,收其租賦。翼又遣子讓通表勸進,并請入朝,許之。



 隋開皇初,翼入朝,上降榻握手極歡。數日,拜太尉。或有告翼往在幽州,欲同尉遲迥。按驗,以無實見原。三年,薨於本位。加贈六州諸軍事、蒲州刺史,謚曰穆。翼性恭儉,與物無競,常以滿盈自戒,故能以功名終。子璽嗣。



 璽字伯符,少有器幹。仕周,位職方中大夫,封黎陽縣公。宣帝嗣位,轉右勳曹中大夫。尋領右忠義。隋文帝受禪,加上大將軍,進爵郡公。歷汴、邵二州刺史,所歷並有恩惠。後檢校江陵總管,邵州人張願等數十人詣闕上表,
 請留璽。上嘉歎良久,令還邵州,父老相賀。尋歷洛、熊二州刺史,亦粗有惠政。以疾還京師,卒於家,謚曰靜。有子志本。



 璽弟詮,位上儀同三司、吏部下大夫、常山公。詮弟讓,儀同三司。翼弟義。



 義字慈恭,少矜嚴,有操尚,篤志好學。大統末,以父功賜爵平昌縣伯。後改封廣都縣公。周閔帝踐阼,遷安武太守。專崇教化,不尚威刑。有郡人張善安、王叔兒爭財相訟,義曰:「太守德薄不勝所致。」於是以家財分與二人,喻而遣去。



 善安等各懷恥愧,移貫他州。於是風化大洽。進封建平郡公。明、武世,歷西兗、瓜、邵三州刺史。數從征伐,
 進位開府。



 宣帝即位,政刑日亂,義上疏諫帝。時鄭譯、劉昉以恩倖當權。謂義不利於己,先惡之於帝。帝覽表色動,謂侍臣曰:「于義謗訕朝廷也。」御正大夫顏之儀進曰:「古先哲王立謗訕之木,置敢諫之鼓,猶懼不聞過。于義之言,不可罪也。」帝乃解。



 及王謙構逆,隋文帝謀將於高熲,熲言義可為元帥。文帝將任之,劉昉曰:「梁睿任望素重,不可居義下。」乃以睿為元帥,義為行軍總管,將左軍,破謙將達奚惎於開遠。尋拜潼州總管,賜奴婢五百口,雜採三千段,超拜上柱國。歲餘,以疾免歸,卒於京師。贈豫州刺史,謚曰剛。子宣道、宣敏,並知名。



 宣道字元明,性謹密,不交非類。仕周,以父功,賜爵城安縣男,位小承御上士。隋文帝為丞相,引為外兵曹。及踐阼,遷內史舍人,進爵為子。父憂,水漿不入口者累日。歲餘,起令視事。免喪,拜車騎將軍,兼右衛長史,舍人如故。後遷太子左衛副率,進位上儀同。卒。



 子志寧,早知名。出繼叔父宣敏。



 宣敏字仲達,少沈密,有才思。年十一,詣周趙王招,命之賦詩。宣敏為詩,甚有幽貞之志。招大奇之,坐客莫不嗟賞。起家右侍上士,遷千牛備身。隋文帝踐阼,拜奉車都尉,奉使撫慰巴、蜀。及還,上疏曰:臣聞開磐石之宗,漢室
 於是惟永;建維城之固,周祚所以靈長。昔秦皇置牧守而罷諸侯,魏后暱謟邪而疏骨肉,遂使宗社移於他族,神器傳於異姓。此事之明,甚於觀火。然山川設險,非親勿居。且蜀土沃饒,人物殷阜,西通邛、僰,南屬荊、巫。周德之衰,茲土遂成戎首;炎政失御,此地便為禍先。是以明者防於無形,安者制其未亂,方可慶隆萬世,年逾七百。



 伏惟陛下日角龍顏,膺樂推之運;參天貳地,居揖讓之期。億兆宅心,百神受職。理須樹建籓屏,封植子孫,繼周、漢之宏圖,改秦、魏之覆軌。抑近習之權勢,崇公族之本枝。但三蜀、二齊,古稱天險,分王戚屬,今正其時。若使利
 建合宜,封樹得所,則巨猾息其非望,奸臣杜其邪謀。盛業洪基,同天地之長久;英聲茂實,齊日月之照臨。臣雖學謝多聞,然情深體國,輒申管見,戰灼惟深。



 帝省表嘉之,謂高熲曰:「於氏世有人焉。」竟納其言,遣蜀王秀鎮於蜀。



 宣敏常以盛滿之誡,昔賢所重,每懷靜退。著《述志賦》以見志焉。未幾,卒官,年二十九。



 義弟禮,上將軍、趙州刺史、安平郡公。



 禮弟智,初為開府。以受宣帝密旨,告齊王憲反,遂封齊國公。尋拜柱國,位大司空。智弟紹,上開府、綏州刺史、華陽郡公。紹弟弼,上儀同、平恩縣公。弼弟蘭,上儀同、襄陽縣開國公。蘭弟曠,上儀同。贈恒州刺史。



 論曰:魏氏平定中原,于栗磾有武功於三世。兼以虛己下物,罰不濫加,斯亦諸將所稀矣。洛拔任參內外,以功名自終。烈氣概沈遠。受任艱危之際,有柱石之質,殆禦侮之臣乎!忠以梗朴見親,乘非其據,遂擅威權,生殺自已。茍非女主之世,何以全其門族?不至誅滅,抑其幸也。謹負佐時之略,逢興運之期,為大廈之棟梁,擬巨川之舟楫。卒以耆年碩德,譽高望重。禮備上庠,功歌司樂。而常以滿盈為誡,覆折是憂,不有君子,何以能國。翼既功臣之子,地則姻親,荷累葉之恩,兼文武之寄,理同休戚,與存與亡。加以總戎馬之權,受扞城之託,智能足
 以衛難,勢力足以勤王。曾無釋位之心,但務隨時之義。弘名節以高貴,豈所望於斯人!仲文博涉書記,以英略自許,尉迥之亂,遂立功名。自茲厥後,屢當推穀。遼東之役,實喪師徒。斯乃大樹將顛,蓋非一繩之罪也。義運屬時來,宣其力用,崇基弗墜,析薪克荷,盛矣!



\end{pinyinscope}