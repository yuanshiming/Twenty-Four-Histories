\article{卷二十五列傳第十三}

\begin{pinyinscope}

 古弼張黎劉潔
 丘堆娥清伊乙瑰周幾豆代田車伊洛王洛兒車路頭盧魯元陳建來大幹宿石萬安國周觀尉撥陸真呂洛拔薛彪子子琡尉元慕容白曜和其奴茍頹宇文福古弼,代人也。少忠謹,善騎射。初為獵郎,門下奏事,以敏
 正稱。明元嘉其直而有用,賜名曰筆。後改名弼,言其有輔佐才也。令典西部,與劉潔等分綰機要,敷奏百揆。太武即位,以功拜立節將軍,賜爵靈壽侯。歷位侍中、吏部尚書,典南部奏事。後徵馮弘。弘將奔高麗,高麗救軍至,弘乃隨之,令婦人被甲居中,其精卒及高麗陳兵於外。弼部將高茍子擊賊軍。弼酒醉,拔刀止之,故弘得東奔。太武大怒,黜為廣夏門卒。尋復為侍中,與尚書李順使涼州。賜爵建興公,鎮長安,甚有威名。及議征涼州,弼與順咸言涼州乏水草,不宜行,帝不從。既剋姑臧,微嫌之,以其有將略,弗之責。



 宋將裴方明剋仇池,立楊玄庶子
 保熾。於是假弼節,督隴右諸軍討仇池,平之。



 未幾,諸氐復推楊文德為主,圍仇池。弼攻解其圍,文德走漢川。時東道將皮豹子聞仇池圍解,議欲還軍。弼使謂曰:「若其班師,寇眾復至,後舉為難。不出秋冬,南寇必來,以逸待勞,百勝之策也。」豹子乃止。太武聞之曰:「弼言長策也。制有南秦,弼謀多矣。」景穆總攝萬機,徵為東宮四輔,與宜都王穆壽並參政事。遷尚書令。弼雖事務殷湊,而讀書不輟。端謹慎密,口不言禁中事。功名等於張黎,而廉不及也。



 上谷人上書,言苑囿過度,人無田業,宜減大半,以賜貧者。弼入欲陳奏,遇帝與給事中劉樹棋,志不聽事。
 弼侍坐良久,不獲申聞。乃起,於帝前捽樹頭,掣下床,以手搏棋耳,以拳毆其背,曰:「朝廷不理,實爾之罪!」帝失容,放棋曰:「不聽奏事,過在朕,樹何罪?置之!」弼具狀以聞。帝奇弼公直,皆可其奏,以與百姓。弼曰:「為臣逞志於君前者,非無罪也。」乃詣公車,免冠徒跣,自劾請罪。帝召之,謂曰:「卿其冠履。吾聞築社之役,蹇蹶而築之,端冕而事之,神與之福。然則卿有何罪?自今以後,茍利社稷,益國便人者,雖復顛沛造次,卿則為之,無所顧也。」



 太武大閱,將校獵於河西,弼留守。詔以肥馬給騎人,弼命給弱者。太武大怒曰:「尖頭奴敢裁量朕也!朕還臺,先斬此奴!」弼頭
 尖,帝常名之曰:「筆頭」,時人呼為「筆公」。屬官懼誅。弼告之曰:「吾謂事君使田獵不過盤游,其罪小也。



 不備不虞,使戎冠恣逸,其罪大也。今北狄孔熾,南虜未滅,狡焉之志,窺伺邊境,是吾憂也。故選肥馬備軍實,為不虞之遠慮。茍使國家有利,吾寧避死乎?明主可以理幹,此自吾罪。」帝聞而歎曰:「有臣如此,國之寶也。」賜衣一襲,馬二疋,鹿十頭。後車駕田於山北,獲麋鹿數千頭,詔尚書發車牛五十乘運之。帝尋謂從者曰:「筆公必不與我,汝輩不如馬運之速。」遂還。行百餘里而弼表至,曰:「今秋穀懸黃,麻菽布野,豬鹿竊食,鳥鴈侵費,風波所耗,朝夕參倍。乞賜
 矜緩,使得收載。」帝謂左右曰:「筆公果如朕卜,可謂社稷之臣。」初,楊難當之來也,詔弼悉送其子弟於京師。楊玄少子文德,以黃金三十斤賂弼。弼受金留文德,而遇之無禮,文德亡入宋。太武以其正直,有戰功,弗加罪責。太武崩,吳王立,以弼為司徒。文成即位,與張黎並坐議不合旨,俱免。有怨謗之言,其家人告巫蠱,俱伏法。時人冤之。



 張黎,鴈門平原人也。善書計,道武知待之。明元器其忠亮,賜爵廣平公,管綜機要。太武以其功舊,任以輔弼,除大司農卿。軍國大議,黎常與焉。以征赫連定功,進號征
 北大將軍。與樂安王範、濟南公崔徽鎮長安。清約公平,甚著聲稱,代下之日,家無餘財。太武征涼州,蠕蠕吳提乘虛入寇,黎與司空長孫道生拒擊走之。景穆初總百揆,黎與崔浩等輔政,忠於奉上,非公事不言。詔賜浩、黎布帛各千疋,以褒舊勳。吳王餘立,以黎為太尉。後文成即位,與古弼俱誅。



 劉潔,長樂信都人也。昭成時,慕容氏獻女,潔祖父生為公主家臣,乃隨入魏。



 賜以妻妾,生子堤,位樂陵太守,封信都男。卒。潔襲堤爵。數從征討,進爵會稽公。後與永安侯魏勤及功勞將軍元屈等擊吐京叛胡,為其所執,送
 赫連屈丐。潔聲氣不撓,呼其字而與之言,神色自若。屈丐壯而釋之。後得還國,典東部事。明元寢疾,太武監國,潔與古弼等選侍東宮,對綜機要。



 太武即位,奇其有柱石用,委以大任。及議軍國,朝臣咸推其能。遷尚書令,改為鉅鹿公。車駕西伐,潔為前鋒。沮渠牧犍弟董來距戰於城南,潔信卜者之言,以日辰不協,擊鼓卻陣,故董來得入城。太武微嫌之。潔久在樞密,恃寵自專,帝心稍不平。時議伐蠕蠕,潔言不如廣農積穀。以待其來,群臣皆從其議。帝決行,乃從崔浩議。既出,與諸將期會鹿渾谷。而潔恨其計不用,欲沮諸將,乃矯詔更期,諸將不至。時
 虜眾大亂,景穆欲擊之,潔執不可。停鹿渾谷六日,諸將猶不集,賊已遠遁,追至石水,不及而還。師次漠中,糧盡,士卒多死。潔陰使人驚軍,勸帝棄軍輕還,帝不從。潔以軍行無功,奏歸罪於崔浩。帝曰:「諸將後期,及賊不擊,罪在諸將,豈在於浩?」又潔矯詔事遂發,輿駕至五原,收潔幽之。



 太武之徵也,潔私謂親人曰:「若軍出無功,車駕不返,即吾當立樂平王。」



 潔又使右丞張嵩求圖讖,問:「劉氏應王,繼國家後,我審有名姓不?」嵩對曰:「有姓而無名。」窮驗款引,搜嵩家,果得讖書。潔與南康公秋鄰及嵩等皆夷三族,死者百餘人。潔既居勢要,內外憚之,側目而視。籍
 其家,財產鉅萬。太武追忿,言則切齒。



 丘堆,代人也。美容儀。初以忠謹入侍。明元即位,拾遺左右,稍遷散騎常侍。



 太武監國臨朝,堆與太尉穆觀等為右弼。及即位,賜爵臨淮公,位太僕。與宗正娥清略地關右,而宜城王奚斤表留堆,合軍與赫連昌相拒。斤進擊赫連定,留堆守輜重。斤為定禽,堆聞而棄甲走長安。帝大怒,遣西平公安頡斬堆。



 娥清,代人也。少有將略,累著戰功,稍遷給事黃門侍郎。明元南巡,幸鄴,以清為中領軍將軍。與宋兵將軍周幾等度河,略地至湖陸,以功賜爵須昌侯。與幾等遂鎮枋
 頭。太武初,乃還京師,進為東平公。後從平統萬,遂與奚斤討赫連昌,至安定。及昌弟定西走,斤追之。清欲尋水往,斤不從,遂與斤俱為定禽。剋平涼,乃得還。後與古弼等東討馮弘,以不急戰,弘奔高麗。檻車徵,黜為門卒而卒於家。



 子延,賜爵南平公。



 伊珝,代人也。少勇健,走及奔馬,善射,力曳牛卻行。神初,擢為侍郎。



 轉三郎,賜爵汾陽子。太武將討涼州,議者咸以無水草諫,唯司徒崔浩勸行。群臣出後,珝曰:「涼州若無水草,何得為國?宜從浩言。」帝善之。及剋涼州,大會於姑臧。帝謂群臣曰:「崔公智計有餘,吾亦不復奇之。正
 奇珝弓馬士,所見能與崔同耳。」顧謂浩曰:「珝智力如此,終至公相。」浩曰:「何必讀書,然後為學。



 衛青、霍去病亦不讀書而致公輔。」帝欲以珝為尚書,封郡公。珝以尚書務殷,公爵至重,辭之;中、秘二省,多諸文士,請參其次。帝賢之,遂拜秘書監,賜爵河南公。拜司空。清約自守,為政舉大綱而已,不為苛碎。大安二年,領太子太保。



 三年,與司徒陸麗等並平尚書事。薨。子蘭襲爵,位庫部尚書。卒。



 子盆生,驍勇有膽氣,累有戰功,遂為名將。以勳賜爵平城子。為西道都督,戰歿。贈雍州刺史。



 乙瑰,代人也。其先世統部落。太武時,瑰父匹知遣瑰入
 貢,帝留之。瑰善騎射,手格猛獸。尚太武女上谷公主,除駙馬都尉,賜爵西平公。從駕南征,都督前鋒諸軍事,勇冠三軍。後進爵為王,又為西道都將。薨,年二十九,贈太尉公,謚曰恭。子乾歸襲爵。



 乾歸有氣幹,頗習書疏,尤好兵法。尚景穆女安樂公主,除駙馬都尉、侍中。



 獻文初,為秦州刺史,有惠政。孝文即位,為中道都將。卒,謚曰康。子海,字懷仁,位散騎侍郎。卒,謚曰孝。



 海子瑗,字雅珍,尚孝文女淮陽公主,除駙馬都尉,累遷西兗州刺史。天平元年,舉兵應樊子鵠,戰敗死。



 周幾,代人也。少以善射為獵郎。明元即位,為左部尚書,
 以軍功封交趾侯。



 太武以幾有智勇,遣鎮河南,威信著于外境。幾常嫌奚斤等綏撫關中失和,每至言論,形于聲色,斤等憚焉。進號宋兵將軍,率洛州刺史于慄磾以萬人襲陜城,卒於軍,軍人無不歎惜之。歸葬京師。謚曰桓。子步襲爵。



 豆代田,代人也。明元時,以善騎射為內細射。從攻武牢,詔代田登樓射賊,矢不虛發。以功遷內三郎。從討赫連昌,乘勝追賊,入其宮門。門閉,代田踰宮而出。太武壯之,拜勇武將軍。後從討平涼,破赫連定,得奚斤等,以定妻賜之。詔斤膝行授酒於代田。敕斤曰:「全爾身命者,代田
 功也。」以從討和龍戰功,封長廣公。卒於統萬鎮大將。贈長廣王,謚曰恭。子周求襲爵。



 車伊洛,焉耆胡也。世為東境部落帥,恆脩職貢。延和中,授平西將軍,封前部王。伊洛規欲歸闕,為沮渠無諱斷路,伊洛連戰破之。無諱卒。伊洛前後遣使招喻其子乾壽等,及其戶五百餘家,送之京師。又率部眾二千餘人伐高昌,討破焉耆東關七城。正平二年,伊洛朝京師,拜都官尚書,將軍、王如故。卒,謚康王,葬禮依盧魯元故事。子歇襲爵。



 王洛兒,京兆人也。明元在東宮,以善騎射給事帳下,謹
 愿未嘗有過。明元嘗獵于水壘南,冰陷沒馬。洛兒投水奉帝出,殆將凍死。帝解衣賜之,自是恩寵日隆。



 天賜末,帝避難居外,洛兒晨夜侍衛,恭勤發於至誠。元紹之逆,帝左右唯洛兒與車路頭。晝居山嶺,夜還洛兒家。洛兒鄰人李道潛相奉給,晨復還山。眾庶頗知,喜而相告。紹聞,收道斬之。洛兒猶冒難往返京都,通問於大臣,大臣遂出奉迎,百姓奔赴。明元還宮,社稷獲全,洛兒有功焉。明元即位,拜散騎常侍,賜爵新息公,加直意將軍。又追贈其父為列侯,賜僮隸五十戶。卒。贈太尉、建平王。賜溫明祕器,載以巉輬車,使殿中衛士為之導從,親臨哀慟者
 四焉。乃鴆其妻周氏,與合葬。子長城襲爵。



 車路頭,代人也。少以忠厚選給東宮,為帳下帥。天賜末,明元出於外,路頭隨侍竭力。及即位,封宣城公、忠意將軍。帝性明察,群臣多以職事遇譴,至有杖罰,故路頭優游不任事。性無害,每評獄處理,常獻寬恕之議,以此見重於朝,帝亦敬納之。卒,明元親臨哀慟,贈太保、宣城王,謚曰忠貞。喪禮一依安城王叔孫俊故事。陪葬金陵。子眷襲爵。



 盧魯元,昌黎徒河人也。曾祖副鳩,仕慕容氏,為尚書令、臨澤公。祖、父並至大官。魯元寬和有雅度。明元時,選為
 通直郎,以忠謹給侍東宮,太武親愛之。



 即位,以為中書侍郎,寵待彌渥。而魯元益加謹肅,帝愈親待之。內外大臣,莫不敬憚。性多容納,善與人交,好掩人過揚人美,由是公卿咸親附之。以工書有文才,累遷中書監,領秘書事。賜爵襄城公,贈其父為信都侯。從征赫連昌,太武親追擊,入其城門,魯元隨帝出入。是日微魯元,幾至危殆。後遷太保、錄尚書事。帝貴異之,臨幸其第,不出旬日。欲其居近,易往來,乃賜甲第於宮門南。衣食車馬皆乘輿之副。真君三年,駕幸陰山,魯元以疾不從。侍臣問疾,醫藥傳驛,相屬於路。



 及薨,帝甚悼惜之,還臨其喪,哭之哀
 慟。東西二宮,命大官日送奠。晨昏哭臨,訖則備奏鐘鼓伎樂。輿駕比葬三臨之。喪禮依安城王叔孫俊故事而賵送有加。贈襄城王,謚曰孝。葬於崞山,為建碑闕。自魏興,貴臣恩寵,無與為比。



 子統襲爵,以父任,侍東宮。太武以元舅陽平王杜超女南安長公主所生妻之。



 車駕親自臨送,太官設供具,賞賚千計。文成即位,典選部、主客二曹。卒,贈襄城王,謚曰景。無子。弟彌娥襲。卒,贈襄城王,謚曰恭。



 魯元少子內,給侍東宮。景穆深暱之,常與臥起,同衣食。父子有寵兩宮,勢傾天下。內性寬厚,有父風,而恭慎不及。正平初,宮臣伏誅。太武以魯元故,唯殺內而
 厚撫其兄弟。



 陳建,代人也。以善騎射擢為三郎,遷下大夫、內行長。太武討山胡白龍,輕之,單將騎數十,每自登山。白龍伏壯士,出不意;帝墜馬,幾至不測。建以身捍賊,奮擊,殺數人,被十餘瘡。帝壯之,賜別戶二十。文成初,出為幽州刺史,假秦郡公。帝以建貪暴懦弱,遣使就州罰杖五十。孝文初,徵為尚書右僕射,加侍中,進爵趙郡公。建與晉陽侯元仙德、長樂王穆亮、平原王陸睿密表啟南伐,帝嘉之。



 遷司徒,進爵魏郡王。帝與文明太后頻幸建第,賜建妻宴於後庭。薨,子念生襲。



 有罪,爵除。



 來大乾,代人也。父初真,從道武避難叱候山,參創業功。官至後將軍,武原侯,與在八議。大乾驍果善騎射。永興初,襲爵,位中散。至於朝賀之日,大乾常著御鎧,盤馬殿庭,朝臣莫不嗟嘆。遷內三郎、幢將,典宿衛禁旅。大乾用法嚴明,上下齊肅。嘗從明元校獵,見獸在高巖上,持槊直前刺之,應手而死,帝嘉其勇壯。



 太武踐阼,與襄城公盧魯元等七人俱為常侍,常持仗侍衛,晝夜不離左右。累從征伐,以戰功賜爵廬陵公,鎮雲中,兼統白道軍事。太武以其壯勇,數有戰功,兼悉北境險要,詔使巡撫六鎮,以防寇虜。經略布置,甚得事宜。後吐京胡反,以大乾
 為都將,討平之。在吐京卒。喪還,停於平城南。太武出游還,見而問之,左右以對,帝悼歎者良久。詔聽其喪入殯城內。贈司空,謚莊公。子丘頹襲爵,降為晉興侯。



 宿石,朔方人,赫連屈丐弟文陳之曾孫也。天興中,文陳父子歸魏,道武嘉之,以宗女妻焉,拜上將軍。祖若豆根,明元時賜姓宿氏,襲上將軍。父沓干,從太武征平涼有功,賜爵漢安男。後從討蠕蠕,戰沒。石年十三襲爵,擢為中散,遷內行令。從於苑中游獵,石走馬引前,道峻馬倒,殞絕,久之乃蘇。由是御馬得制。文成嘉之,賜以綿帛、駿馬,改爵義陽子。又常從獵,文成親欲射猛獸。石叩馬諫,
 引帝至高原上。後猛獸騰躍殺人。褒美其忠,許後有犯罪,宥而勿坐,賜駿馬一疋。



 尚上谷公主,拜駙馬都尉。位吏部尚書,進爵太山公,為北征中道都大將。卒,追贈太原王,謚康,葬禮依盧魯元故事。太和初,子倪襲爵。



 萬安國,代人也。世為酋帥。父振,尚高陽長公主,拜駙馬都尉,位長安鎮將,爵馮翊公。安國少明敏,以國甥復尚河南公主,拜駙馬都尉。獻文特親寵之,與同臥起。拜大司馬、大將軍,封安城王。安國先與神部長奚買奴不平,承明初,矯詔殺買奴於苑中。孝文聞之,大怒,遂賜死,年二十三。子翼襲王爵。有嵇根者,世為紇奚部帥。皇始初,
 率部歸魏,尚昭成女。生子拔,位尚書令。拔尚華陰公主,生子敬。元紹之逆也,主有功,超授敬大司馬,封長樂王。薨,子護襲,拜外都大官。根事迹遺落,故略附云。



 周觀,代人也,驍勇有膂力。太武以軍功賜爵金城公,位高平鎮將。善撫士卒,號有威名。後拜內都大官,出為秦州刺史。撫馭失和,部人薛永宗聚眾汾曲以叛。



 觀討永宗,為流矢所中。太武幸蒲阪,觀聞帝至,驚怖而起,瘡重遂卒。帝怒,絕其爵云。



 尉撥,代人也。父那,濮陽太守。撥為太學生,募從兗州刺史羅忸擊賊於陳、汝,有功,賜爵介休男。討和龍,擊吐谷
 渾,皆有軍功,進爵為子。累遷杏城鎮將,大得人和。文成以撥清平有惠績,賜以衣服。獻文即位,為北征都將。南攻懸瓠。



 進爵安城侯,位北豫州刺史。卒,謚敬侯。



 陸真,代人也。父洛侯,秦州刺史。真少善騎射。太武以真膂力過人,拜內三郎。真君中,從討蠕蠕,以功賜爵關內侯。後攻懸瓠,登樓臨射城中,弦不虛發。



 從太武至江,還攻盱眙,真功居多。文成即位,進爵都昌侯,位選部尚書。後拜長安平鎮將。時初置長蛇鎮,真率眾築城未訖,而氐豪仇傉檀等反叛。真擊平之,卒城長蛇而還。東平王道符反于長安,以真為長安鎮將,賜爵河南公。長安平兵
 人素伏其威信,及至,皆怗然安靜。在鎮數年,甚著威稱。卒,謚曰烈。



 子延,字契胡提,頗有氣幹。襲爵河南公,例降,改封汝陽侯。位懷朔鎮大將、太僕卿。受使綏慰秀容,為牧子所害呂洛拔,代人也。曾祖渴侯,昭成時率戶五千歸魏。父匹知,太武時為西部長,封榮陽公。洛拔以壯勇知名。文成末,為平原鎮都將。隨尉元攻宋將張永,大敗之,賜爵成武侯。卒。



 長子文祖,獻文以其勳臣子,補龍牧曹奏事中散。以牧產不滋,坐徙武川鎮。



 後文祖以舊語譯註皇誥,辭義通辯,為外都曹奏事中散。後坐事伏法。



 薛彪子,代人也。祖達頭,自姚萇時率部落歸魏。道武賜爵聊城侯,待以上客禮,賜妻鄭氏。卒,贈冀州刺史,謚曰悼。父野者,並、太二州刺史,封河東公,有聲稱。卒,謚曰簡。彪子姿貌壯偉,明斷有父風。為內行長,典奏諸曹事。當官正直,內外憚之。及文明太后臨朝,出為枋頭鎮將。素剛簡,為近臣所嫉;因小過,黜為鎮門士。及獻文南巡,次山陽,彪子拜訴於路,復除枋頭鎮將。累遷開府、徐州刺史。在州甚多惠政,百姓便之。沛郡太守邵安、下邳太守張攀,咸以贓汙,彪子案之於法。安等遣子弟上書,誣彪子南通賊虜。孝文曰:「此妄矣。」推案果虛。



 卒,謚曰文。子琡。



 琡字曇珍,形貌瑰偉。少以幹用為典客令。每引見,儀望甚美。宣武謂曰:「卿風度峻整,姿貌秀異,後當升進,何以處官?」琡答曰:「宗廟之禮,不敢不敬;朝廷之事,不敢不忠。自此之外,非庸臣所及。」正光中,行洛陽令,部內肅然。時以久旱,京師見囚悉召集於都亭,理問冤滯。洛陽獄唯有三人。孝明嘉之,賜縑百疋。琡本附元叉,叉廢,憂懼,由是政教廢弛,坐免官。李神軌有寵於靈太后,琡復事之。累遷吏部郎中。



 先是,吏部尚書崔亮奏立停年格,不簡人才,專問勞舊。琡乃上書曰:「臣聞錦縠雖輕,不委之以學割;瑚璉任重,豈寄之以弱力。若使選曹唯取年勞,不
 簡賢否,使義均行鴈。次若貫魚,勘簿呼名,一吏足矣。數人而用,何謂銓衡?今黎元之命繫於守長。若其得人,則蘇息有地;任非其器,為患更深。請郡縣之職,吏部先盡擇才,並學通古今曉達政職者,以應其選。不拘入職遠近,年勛多少,其積勞之中,有才堪牧人者,自在先用之限。其餘不堪者,既壯藉其力,豈容老而棄之。



 將佐丞尉去人稍遠,小小當否,未為多失,宜依次補敘,以酬其勞。」書奏,不報。



 後因引見,復陳之曰:「今四方初定,務在養人。臣請依漢氏更立四科,令三公宰貴各薦時賢,以補郡縣。明立條格,防其阿黨之端。庶令塗炭之餘,戴仰有地。」



 詔下公卿議之,事亦寢。



 元天穆討邢杲,以琡為行臺尚書。軍次東郡,時元顥已據贊阜城,邢杲又逼歷下,天穆議其所先。議者咸以杲盛,宜先經略。唯琡以杲為聚眾無名,雖強猶賊。



 元顥皇室暱親,來稱義舉。自河陰之役,人情駭怨,今有際會,易生感動。待顥事決,然後回師。天穆以群情所願,遂先討杲。杲降,軍還至定陶,天穆留琡行西兗州事。尋為元顥所陷。顥執琡自隨。爾朱榮破顥,天穆謂琡曰:「不用君言,乃至於此!」



 天平初,拜七兵尚書。齊神武引為丞相府長史,軍國之事,多所關知。叔亦推誠盡節,屢進忠讜。神武大舉西伐,將度蒲津。琡諫曰:「西賊
 連年饑饉,故冒死來入陜州。但宜置兵諸道,勿與野戰。比及來年麥秋,人應餓死,寶炬、黑獺自然歸降。願無渡河。」侯景亦曰:「今舉兵極大,萬不一捷,卒難收斂。不如分為二軍,相繼而進,前軍若勝,後軍合力;前軍若敗,後軍承之。」神武皆弗納,遂有沙苑之敗。



 後范陽盧仲禮反,琡與諸軍討平之。轉殷州刺史。為政嚴酷,吏人苦之。後歷位度支、殿中二尚書。天保元年,卒於兼尚書右僕射。臨終,敕其子斂以時服。踰月便葬,不聽干求贈官。自制喪車,不加彫飾,但用麻為旒蘇,繩網絡而已。明器等物,並不令置。



 琡久在省闥,明閑簿領,當官剖斷,敏速如流。然
 天性險忌,情義不篤。外若方格,內實浮動。受納貨賄,曲理舞法,深文刻薄,多所傷害。人士畏惡之。魏東平王元匡妾張氏,淫逸放恣。琡初與姦通,後納以為婦。惑其讒言,遂棄前妻于氏,不忍其子允。家人內忿,競相告列,深為世所譏鄙。贈開府儀同三司、尚書左僕射、青州刺史。謚曰威恭。子允嗣。



 尉元,字茍仁,代人也。世為豪宗。父目斤,勇略聞於當時,位中山太守。元以善射稱,為羽林中郎,以匪懈見知。稍遷駕部給事中,賜爵富城男。和平中,遷北部尚書,進爵太昌侯。



 天安元年,薛安都以徐州內附,獻文以元為持
 節、都督東道諸軍事,與城陽公孔伯恭赴之。宋兗州刺史畢眾敬遣東平太守章仇歸款,元並納之,遂長驅而進。



 宋遣將張永、沈攸之等屯于下蓋。安都出城見元。元依朝旨,授其徐州刺史,遣中書侍郎高閭、李璨等與安都俱還入城。別令孔伯恭撫安內外,然後元入彭城。元以永仍據險要,乃命安都與璨等同守。身率精銳,揚兵於外,分擊呂梁,絕其糧運。



 永遂捐城夜遁。於是遣高閭與張讜對為東徐州刺史;李璨與畢眾敬對為東兗州刺史。



 拜元開府、都督、徐州刺史、淮陽公。太和初,徵為內都大官。既而出為使持節、鎮西大將軍、開府、統萬鎮
 都將,甚得夷人之心。三年,進爵淮陽王,以舊老見禮,聽乘步挽,杖於朝。齊高帝既立,多遣間諜,扇動新人;不逞之徒,所在蜂起。以元威名夙振,使總率諸軍以討之。東南清晏,遠近帖然。入為侍中、都曹尚書,遷尚書令,進位司徒。



 十年,例降庶姓王爵,封山陽郡公。其年,頻表以老乞身,詔許之。元詣闕謝老,引見於庭;命升殿勞宴,賜玄冠、素服。又詔曰:「前司徒山陽郡公尉元、前大鴻臚卿新泰伯游明根,並元亨利貞,明允誠素,位顯台宿,歸老私第。可謂知始知卒,希世之賢也。公以八十之年,宜處三老之重;卿以七十之齡,可充五更之選。」



 於是養三老、五
 更於明堂,國老、庶老於階下。孝文再拜三老,親袒割牲,執爵而饋;於五更行肅拜之禮;賜國老、庶老衣服有差。既而元言曰:「自天地分判,五行施則,人之所崇,莫重於孝順。然五孝六順,天下之所先,願陛下重之,以化四方。臣既年衰,不究遠趣,心耳所及,敢不盡誠。」帝曰:「孝順之道,天地之經。



 今承三老明言,銘之于懷。」明根言曰:「夫至孝通靈,至順感幽,故《詩》云:『孝悌之至,通於神明,光于四海。』如此則孝順之道,無所不格。願陛下念之,以濟黎庶。臣年志朽弊,識見昧然,在於愚慮,不敢不盡。」帝曰:「五更助三老以言至範,敷展德音。當克己復禮,以行來授。」禮
 畢,乃賜步挽一乘。詔曰:「夫尊老尚更,列聖同致,欽年敬德,綿哲齊軌。朕雖道謝玄風,識昧睿則,然仰稟先誨,企遵猷旨。故推老以德,立更以元;父焉斯彰,兄焉斯顯矣。前司徒公元、前鴻臚卿明根,並以沖德懸車,懿量歸老,故尊老以三,事更以五。雖老、更非官,耄耋罔祿,然況事既高,宜加殊養。三老可給上公祿,五更可食元卿俸。供食之味,亦同其例。」十七年,元疾篤,帝親省疾。薨,謚景桓公,葬以殊禮,給羽葆鼓吹,假黃鉞,班劍四十人。



 子翊襲爵。遷洛,以山陽在畿內,改為博陵郡公。卒於恆州刺史,謚曰順。



 慕容白曜,慕容晃之玄孫也。父琚,歷官以廉清著稱,賜爵高都侯。終尚書左丞,謚曰簡。白曜少為中書吏,以敦直給事宮中。襲爵,稍遷北部尚書。文成崩,與乙渾共執朝政,遷尚書右僕射,進爵南鄉公。



 宋徐州刺史薛安都、兗州刺史畢眾敬並以城內附,詔鎮南大將軍尉元、鎮東將軍孔伯恭赴之。而宋東平太守申纂屯無鹽,並州刺史房崇吉屯斗城,遏絕王使。皇興初,加白曜使持節、都督軍事、征南大將軍,進爵上黨公。屯碻磝,為諸軍後繼。



 白曜攻纂於無鹽,拔其東郭。纂遁,遣兵追執之。迴攻斗城。肥城戍主聞軍至,棄城遁走,獲粟三十萬石。又下
 襲破麋溝、垣苗二戍,得粟十餘萬斛。由是軍糧充足。



 先是,淮陽公皮豹子再徵垣苗不剋,白曜一旬內頻拔四城,威震齊土。獻文下詔褒美之。斗城不降,白曜縱兵陵城,殺數百人,崇吉夜遁。白曜撫其人,百姓懷之。



 獲崇吉母妻,待之以禮。宋遣將吳喜公欲冠彭城,鎮南大將軍尉元請濟師,獻文詔白曜赴之。白曜到瑕丘,遇患,因停。會崇吉與從弟法壽盜宋盤陽城以贖母妻。白曜遣將軍長孫觀等率騎入自馬耳關赴之。觀至盤陽,諸縣悉降。白曜自瑕丘進攻歷城。二年,崔道固及兗州刺史梁鄒守將劉休賓並面縛而降。白曜皆釋之。送道固、休賓
 及其僚屬于京師。後乃徙二城人望於下館,朝廷置平齊郡懷寧、歸安二縣以居之。自餘悉為奴婢,分賜百官。白曜雖在軍旅,而接待人物,寬和有禮。所獲崇吉母妻、申纂婦女,皆別營安置,不令士卒喧雜。及進克東陽,擒沈文秀。凡獲倉粟八十五萬斛。始末三年,築圍攻擊,雖士卒死傷,無多怨叛。三齊欣然,安堵樂業。



 剋城之日,以沈文秀抗倨不為之拜,忿而撾撻,唯以此見譏。以功拜開府儀同三司、都督、青州刺史,進爵濟南王。初,乙渾專權,白曜頗所挾附,後緣此以為責。四年,見誅,云謀反叛,時論冤之。



 白曜少子真安,年十一,聞父被執,將自殺。
 家人止之曰:「輕重未可知。」



 真安曰:「王位高功重,若小罪,終不至此。我不忍見父之死。」遂自縊。太和中,著作佐郎成淹上表理白曜,孝文覽表嘉愍之。



 白曜弟子契,輕薄無檢。太和初,以名家子擢為中散,遷宰官中散。南安王楨有貪暴之響,遣中散閭文祖詣長安察之。文祖受楨金寶之賂,為楨隱而不言。事發,太后引見群臣,謂曰:「前論貪清,皆云剋脩。文祖時亦在中,後竟犯法。以此言之,人心信不可知。」孝文曰:「卿等自審不勝貪心者,聽辭位歸第。」契進曰:「小人之心無定,而帝王之法有常。以無恆之心奉有常之法,非所剋堪。乞垂退免。」



 帝曰:「契若知心
 不可常,即知貪之惡矣,何為求退?」遷宰官令,賜爵定陶男。



 後卒於都督、朔州刺史,謚曰剋。初,慕容氏破後,種族仍繁。天賜末,頗忌而誅之。時有免者,不敢復姓,皆以輿為氏。延昌末,詔復舊姓。而其子女先入掖庭者,猶號慕容,特多於他族。



 和其奴,代人也。少有操行,善射御。初為三郎。文成初,封平昌公,累遷尚書左僕射。又與河東王閭毗、太宰常英等並平尚書事。在官慎法,不受私請。遷司空,加侍中。文成崩,乙渾與林金閭擅殺尚書楊保年等。時殿中尚書元郁率殿中宿衛士欲加兵於渾。渾懼,歸咎於金閭,執
 以付郁。時其奴以金閭罪惡未分,出之為定州刺史。皇興元年,長安鎮將東平王道符反,詔其奴討之,未至而道符敗。軍還,薨,內外歎惜之。贈平昌王,謚曰宣。子受襲爵。



 茍頹,代人也。本姓若干。父洛拔,內行長。頹厚重寡言,少嚴毅清直,武力過人。擢為中散,小心謹敬。太武至江,賜爵建德男。累遷司衛監、洛州刺史。抑強扶弱,山蠻畏威,不敢為寇。太和中,歷位侍中、都曹尚書,進爵河南公。頹方正好直言,雖文明太后生殺不允,頹亦言至懇切。李惠、李之誅,頹並致諫。遷司空,進爵河東王。以舊老,聽
 乘步挽,杖於朝。大駕行幸三川,頹留守京師。沙門法秀謀反,頹率禁旅收掩畢獲,內外晏然。薨,謚僖王。長子愷襲爵河東王,例降為公。



 宇文福,其先南單于之遠屬也。世為擁部大人。祖活撥,仕慕容垂為唐郡內史、遼東公。道武之平慕容氏,活撥入魏,為第一客。福少驍果,有膂力。太和中,累遷都牧給事。及遷洛,敕福檢牧馬所。福規石濟以西,河內以東,拒黃河南北千里為牧地,今之馬場是也。及徙代移雜畜牧於其所,福善於將養,並無損耗。孝文嘉之。尋補司衛監。後以勛封襄樂縣男,歷位太僕卿、都官尚書、營州大
 中正、瀛州刺史。性忠清,在公嚴毅,以信御人,甚得聲譽。後除都督懷朔、沃野、武川三鎮諸軍事、懷朔鎮將。至鎮卒,謚曰貞惠。



 子延,字慶壽,體貌魁岸,眉目疏朗。位員外散騎侍郎。以父老,詔聽隨侍在瀛州。屬大乘妖黨突入州城,延率奴客逆戰,身被重瘡。賊縱火燒齋閣,福時在內。



 延突火入,抱福出外,支體灼爛,鬢髮盡焦。於是勒眾與賊苦戰,賊乃散走,以此見稱。累遷直寢。與萬俟鬼奴戰,沒。



 論曰:古弼軍謀經國,有柱石之量;張黎誠謹廉方,以勳舊見重。並纖介之間,一朝隕覆。宥及十世,乃徒言耳。劉
 潔咎之徒也;丘堆敗以亡身。娥清、伊珝俱以材力見用,而珝以謀猷取異,其殆優乎。乙瑰之驍猛,周幾之智勇,代田之騎射,其位遇豈徒然也。車伊洛宅心自遠,豈常戎乎。王洛兒、車路頭、盧魯元、陳建、來大乾、宿石,或誠發于衷,竭節危難;或忠存衛主,義足感人。茍非志烈,亦何能若此。宜其生受恩遇,歿盡哀榮。至如安國,以至覆亡,害盈之義也。周觀、尉撥、陸真、呂洛拔等,咸以勇毅自進,而觀竟致貶黜,異夫數子者矣。薛彪子世載強正,曇珍克盛家聲,美矣乎!魏之諸將,罕方面之績。尉元以寬雅之風,膺將帥之任,威名遠被,位極公老,自致乞言之地,
 無乃近代之一人歟!白曜出專薄伐,席卷三齊,考績圖勞,固不細矣。而功名難處,追猜嬰戮。宥賢議勤,未聞於斯日也。和其奴之貞正,茍頹之剛直,宇文福之氣幹,咸亦有用之士乎!



\end{pinyinscope}