\article{卷二十九列傳第十七}

\begin{pinyinscope}

 司馬休之司馬楚之曾孫裔司馬景之司馬叔璠司馬天助劉昶蕭寶夤兄子贊蕭正表蕭祗蕭退蕭泰蕭捴蕭圓肅蕭大圜司馬休之,字季豫,河內溫人,晉宣帝季弟譙王進之後也。晉度江之後,進子孫襲封譙王。至休之父恬,為鎮北將軍、青兗二州刺史。天興五年,休之為荊州刺史,被桓
 玄逼逐,遂奔慕容德。及玄誅,還建業,復為荊州刺史。



 休之頗得江漢人心。其子文思繼其兄尚之為譙王,謀圖劉裕。裕執送休之,令為其所。休之表廢文思,並與裕書陳謝。神瑞中,裕收休之子文寶、兄子文祖並殺之,乃討休之。休之與魯宗之及宗之子軌起兵討裕。兵敗,遂與子文思及宗之奔姚興。裕滅姚泓,休之與文思及晉河間王子道賜等數百人皆將妻子降長孫嵩。卒,贈征西大將軍、右光祿大夫、始平公,謚曰聲。



 文思與淮南公國璠、池陽子道賜不平,而偽親之。國璠性疏直,因醉欲外叛。



 文思告之,皆坐誅。以文思為廷尉,賜爵鬱林公。文思
 善於其職,聽斷,百姓不得匿其情。進爵譙王,位懷荒鎮將,薨。



 司馬楚之,字德秀,晉宣弟太常馗之八世孫也。父榮期,晉益州刺史,為其參軍楊承祖所殺。楚之時年十七,送父喪還丹楊。會劉裕誅夷司馬氏,叔父宣期、兄貞之並遇害。楚之乃逃,匿諸沙門中,濟江至汝、潁間。楚之少有英氣,能折節待士。及宋受禪,規欲報復。收眾據長社,歸之者常萬餘人。宋武深憚之,遣刺客沐謙圖害楚之。楚之待謙甚厚。謙夜詐疾,知楚之必來,欲因殺之。楚之聞謙病,果自齎湯藥往省之。謙感其意,出匕首於席下,以
 狀告,遂委身以事之。其推誠信物,得士心,皆此類也。



 明元末,山陽公奚斤略地河南,楚之遣使請降,授荊州刺史。奚斤既平河南,以楚之所率人戶,分置汝南、汝陽、南頓、新蔡四郡,以益豫州。太武初,楚之遣妻子內居於鄴。尋徵入朝,授安南大將軍,封瑯邪王,以拒宋師。賜前後部鼓吹。



 破宋將到彥之別軍於長社。又與冠軍安頡攻拔滑臺,禽宋將朱修之、李元德及東郡太守申謨,俘萬餘人。上疏求更進討,太武以兵久勞,不從,以散騎常侍徵還。宋將裴方明、胡崇之寇仇池。楚之與淮南公皮豹子等督關中諸軍擊走方明,禽崇之。



 仇池平而還。



 車駕
 征蠕蠕,楚之與濟陰公盧中山等督運以繼大軍。時鎮北將軍封沓亡入蠕蠕,說令擊楚之以絕糧運。蠕蠕乃遣覘楚之軍,截驢耳而去。有告失驢耳者,楚之曰:「必覘賊截之為驗耳,賊將至矣。」乃伐柳為城,灌水令凍,城立而賊至,不可攻逼,乃走散。太武聞而嘉之。尋拜假節、侍中、鎮西大將軍、開府儀同三司、雲中鎮大將、朔州刺史。



 在邊二十餘年,以清儉著聞。及薨,贈征南大將軍,領護西戎校尉、揚州刺史,謚貞王,陪葬金陵。長子寶胤,與楚之同入魏,拜中書博士、鴈門太守,卒。



 楚之後尚諸王女河內公主。生子金龍,字榮則,少有父風,後襲爵,拜侍中、
 鎮西大將軍、開府、雲中鎮大將、朔州刺史、吏部尚書。薨,贈司空公,謚康王。



 金龍初納太尉、隴西王源賀女。生子延宗,次纂,次悅。後娶沮渠氏,生子徽亮,即河西王沮渠牧犍女,太武妹武威公主所生也。有寵於文明太后,故以徽亮襲。



 例降為公,坐連穆泰罪,失爵,卒。



 悅字慶宗,歷位豫州刺史。時有汝南上蔡董毛奴者,齎錢五千。死於道路。郡縣人疑張堤為劫,又於堤家得錢五千,堤懼掠,自誣言殺。至州,悅觀色,疑其不實。引見毛奴兄靈之,謂曰:「殺人取錢,當時狼狽,應有所遺,得何物?」靈之曰:「唯得一刀削。」悅取視之,曰:「此非里巷所為也。」乃召州內刀匠
 示之。



 有郭門前曰:「此刀削,門手所作,去歲賣與郭人董及祖。」悅收及祖詰之,及祖款引。靈之又於及祖身上得毛奴所衣皂襦,及祖伏法。悅察獄,多此類也。



 俄與鎮南將軍元英攻克義陽,詔改梁司州為郢州,以悅為刺史。改為豫州刺史;論前勛,封漁陽子。永平元年,城人白早生謀為叛,遂斬悅首送梁。詔揚州移購悅首,贈青州刺史,謚曰莊子。子朏襲。



 朏尚宣武妹華陽公主,拜駙馬都尉、員外散騎常侍。卒,贈滄州刺史。子鴻,字慶雲,生粗武。襲爵,位都水使者,坐通西魏,賜死。子孝政襲。齊受禪,例降。



 朏弟裔。



 裔字遵胤,少孤,有志操。起家司徒府參軍事,後為員外散騎常侍。大統三年,大軍復弘農,乃於溫城送款歸西魏。六年,授北徐州刺史。八年,入朝。周文帝嘉之,特蒙賞勞。頃之,河內有四千餘家歸附,並裔之鄉舊,乃命領河內郡守,令安集流人。十五年,周文令山東立義諸將等能率眾入關者,並加重賞。裔領戶千室先至,周文欲以封裔。裔辭曰:「立義之士,遠歸皇化者,皆是其誠心內發,豈裔能率之乎?今以封裔,便是賣義士以求榮。」周文善而從之。授帥都督,拜其妻元為襄城郡公主。



 周孝閔帝踐祚,除巴州刺史,進使持節、驃騎大將軍、開府儀同三
 司,進爵瑯邪縣伯。四年,為御正中大夫,進爵為公。大軍東討,裔與少師楊守軹關,即授懷州刺史。天和初,隨上庸公陸騰討信州反蠻冉令賢等。裔自開州道入,先遣使宣示禍福,群蠻率服。歷信、潼二州刺史。六年,徵拜大將軍,除西寧州刺史,未及部,卒於京師。



 裔性清約,不事生產,所得俸祿,並散之親戚;身死之日,家無餘財。宅宇卑陋,喪庭無所,詔為起祠堂焉。贈本官,加泗州刺史,謚曰定。子侃嗣。



 侃字道遷,少果勇,未弱冠,便從戎旅。位樂安郡守,以軍功,加驃騎大將軍、開府儀同三司。遷兗州刺史,未之部,卒。贈本官,加豫州刺史,謚曰惠。子運嗣。



 金龍弟躍,字寶龍,尚趙郡公主,拜駙馬都尉。代兄為雲中鎮將,拜朔州刺史,假安北將軍、河內公。表求罷河西苑封,丐人墾殖。有司執奏,此苑麋鹿所聚,太官取給,若丐人,懼有所闕。躍固請,孝文從之。還為祠部尚書、大鴻臚卿、潁川王師,卒。



 楚之父子相繼鎮雲中,朔土服其威德。



 司馬氏桓玄、劉裕之際歸北者,又有司馬景之、叔璠、天助,位並崇顯。



 景之字洪略,晉汝南王亮之後。明元時歸闕,賜爵蒼梧公,加征南大將軍。清直有節操。卒,贈汝南王。子師子襲爵。



 景之兄準,字巨之,以泰常末歸魏。封新安公。除廣寧
 太守,改密陵侯。卒,子安國襲爵。



 叔璠,晉安平獻王孚之後。父曇之,晉河間王。桓玄、劉裕之際,叔璠與兄國璠奔慕容超。後投姚泓。泓滅,奔屈丐。統萬平,兄弟俱入魏。國璠賜爵淮南公,叔璠賜爵丹楊侯。



 天助,自云晉驃騎將軍元顯之子。歸闕,封東海公,歷青、兗二州刺史。



 劉昶,字休道,宋文帝子也。在宋封義陽王,位徐州刺史。及廢主子業立,疑昶有異志。昶和平六年,遂委母妻,攜妾吳氏,間行降魏。朝廷嘉重之,尚武邑公主,拜侍中、征
 南將軍、駙馬都尉,封丹楊王。歲餘,主薨,更尚建興長公主。皇興中,宋明帝使至,獻文詔昶與書,為兄弟式。宋明帝不答,責昶,以母為其國妾,宜如《春秋》荀鳷對楚稱外臣之禮。尋敕昶更為書。辭曰:「臣若改書,事為二敬,猶脩往文,彼所不納。請停今答。」朝廷從之。拜外都坐大官。公主復薨,更尚平陽長公主。



 昶好犬馬,愛武事。入魏歷紀,猶布衣皁冠,同凶素之服。然呵罵僮僕,音雜夷夏。雖在公坐,諸王每侮弄之。或戾手齧臂,至於痛傷,笑呼之聲,聞于御聽。



 孝文每優假之,不以怪問。至於陳奏本國事故,語及征役,則斂容涕泗,悲動左右。



 而天性褊躁,喜怒
 不恒。每至威忿,楚撲特苦;引待南士,禮多不足。緣此,人懷畏避。



 太和初,轉內都坐大官。及齊初,詔昶與諸將南伐。路經徐州,哭拜其母舊堂,哀感從者。乃遍循故居,處處隕涕,左右亦莫不酸鼻。及至軍所,將臨陣,四面拜諸將士,自陳家國滅亡,蒙朝廷慈覆。辭理切至,聲氣激揚,涕泗橫流,三軍咸為感歎。後昶恐水雨方降,表請還師,從之。



 又加儀同三司,領儀曹尚書。於時改革朝儀,詔昶與蔣少遊專主其事。昶條上舊式,略不遺忘。孝文臨宣文堂,引武興王楊集始入宴,詔昶曰:「集始,邊方之酋,不足以當諸侯之禮。但王者不遺小國之臣,故勞公卿於
 此。」又為中書監。開建五等,封昶齊郡公,加宋王之號。



 十七年,孝文臨經武殿,大議南伐。語及劉、蕭篡奪之事,昶每悲泣不已。帝亦為之流涕,禮之彌崇。



 十八年,除使持節、都督吳越楚彭城諸軍事、大將軍、開府,鎮徐州。昶頻表辭大將軍,詔不許。及發,帝親餞之,命百寮賦詩贈昶。又以其文集一部賜昶。帝因以所製文筆示之曰:「時契勝殘,事鐘文業。雖則不學,欲罷不能。脫思一見,故以相示,雖無足味,聊復為一笑耳。」其重昶如是。自昶背彭城,至是久矣,昔齋宇山池,並尚存立;昶更脩繕,還處其中。不能綏邊懷物,撫接義故,而閨門喧猥,內外奸雜,舊吏
 莫不慨歎。預營墓於彭城西南,與三公主同塋而異穴。發石累之,墳崩,壓殺十餘人。後復移改,公私費害。



 十九年,昶朝京師。孝文臨光極堂大選,曰:「國家昔在恆代,隨時制宜,非通世之長法。或言,唯能是寄,不必拘門。朕以為不然,何者?清濁同流,混齊一等,君子小人,名品無別,此殊為不可。我今八族以上,士人品第有九;九品之外,小人之官,復有七等。若茍有其人,可起家為三公。正恐賢才難得,不可止為一人,混我典制。故令班鏡九流,使千載之後,我得仿像唐、虞,卿等依希元、凱。」及論大將軍,帝曰:「劉昶即其人也。」後給班劍二十人。薨於彭城,孝文
 為之舉哀,給溫明祕器,贈假黃鉞、太傅,領揚州刺史。加以殊禮,備九錫,給前後部羽葆鼓吹,依晉瑯邪王伷故事,謚曰明。



 昶嫡子承緒,主所生也。少而尪疾,尚孝文妹彭城長公主,為駙馬都尉,先昶卒。



 承緒子暉,字重昌,為世子,襲封。尚宣武第二姊蘭陵長公主。主嚴妒,暉嘗私幸主侍婢。有身,主笞殺之;剖其孕子,節解,以草裝實婢腹,裸以示暉。暉遂忿憾,疏薄公主。公主姊因入聽講,言其故於靈太后。太后敕清河王懌窮其事。懌與高陽王雍、廣平王懷奏其不和狀,請離婚,削除封位。太后從之。公主在宮內周歲,雍等屢請聽復舊義。太后流涕送公
 主,誡令謹敕。正光初,暉又私淫張、陳二氏女。公主更不檢忌。主姑陳留公主共將扇獎,與暉復致忿諍。暉推主墜床,手腳毆蹈,主遂傷胎。暉懼罪逃逸。靈太后召清河王懌決其事。二家女髡笞會宮,兄弟皆坐鞭刑。徙配敦煌為兵。主因傷致薨,太后親臨慟哭,舉哀太極東堂。出葬城西,太后親送數里,盡哀而還。後執暉於河內溫縣,幽于司州,將加死刑。會赦,免。



 後復其官爵,遷征虜將軍、中散大夫,卒,家遂衰頓。



 蕭寶夤,字智亮,齊明帝第六子,廢主寶卷之母弟也。在齊封建安王。及和帝立,改封鄱陽王。梁武克建業,以兵
 守之,將加害焉。其家閹人顏文智與左右麻拱、黃神密計,穿墻夜出寶夤。具小船於江岸,脫本衣服,著烏布襦;腰繫千許錢,潛赴江畔;躡屩徒步,腳無全皮。防守者至明追之。寶夤假為釣者,隨流上下十餘里,追者不疑。待散,乃度西岸。遂委命投華文榮。文榮與其從天龍、惠連等三人,棄家,將寶夤遁匿山澗,賃驢乘之,晝伏宵行。景明二年,至壽春東城戍。戍主杜元倫推檢,知實蕭氏子,以禮延待。馳告揚州刺史、任城王澄。澄以車馬侍衛迎之。



 時年十六,徒步憔悴,見者以為掠賣生口也。澄待以客禮。乃請喪君斬衰之服,澄遣人曉示情禮,以喪兄之
 制,給其齊衰,寶夤從命。澄率官僚赴弔。寶夤居處有禮,不飲酒食肉;輟笑簡言,一同極哀之節。壽春多其故義,皆受慰唁。唯不見夏侯一族,以其同梁故也。改日造澄,澄深器重之。



 及至京師,宣武禮之甚重。伏訴闕下,請兵南伐,雖遇暴風大雨,終不暫移。



 是年,梁江州刺史陳伯之與其長史褚胄等自壽春歸降,請軍立效。帝謂伯之所陳,時不可失。以寶夤懇誠,除使持節、都督、東揚州刺史、鎮東將軍、丹楊郡公、齊王,配兵一萬,令據東城,待秋冬大舉。寶夤明當拜命,其夜慟哭。至晨,備禮策授,賜車馬什物,事從豐厚,猶不及劉昶之優隆也。又任其募天
 下壯勇,得數千人。



 以文智等三人為積弩將軍,文榮等三人為強弩將軍,並為軍主。寶夤雖少羈寓,而志性雅重。過期猶絕酒肉,慘悴形色,蔬食粗衣,未嘗嬉笑。及被命當南伐,貴要多相憑托,門庭賓客若市。而書記相尋,寶夤接對報復,不失其理。



 正始元年,寶夤行達汝陰,東城已陷,遂停壽春之棲賢寺。逢梁將姜慶真內侵,圍逼壽春。寶夤率眾力戰,破走之。寶夤勇冠諸軍,聞見者莫不壯之。還,改封梁郡公。及中山王英南伐,寶夤又表求徵。與英頻破梁軍,乘勝攻鐘離。淮水泛溢,寶夤與英狼狽引退,士卒死沒者十四五。有司奏處以極法。詔恕死,
 免官削爵還第。



 尋尚南陽長公主。公主有婦德,寶夤盡雍和之禮,雖好合而敬事不替。寶夤每入室,公主必立以待之,相遇如賓,自非太妃疾篤,未曾歸休。寶夤器性溫順,自處以禮,奉敬公主,內外庇穆。清河王懌親而重之。



 永平四年,盧昶克梁朐山戍,以瑯邪戍主傅文驥守之。梁師攻文驥,昶督眾軍救之。詔寶夤為使持節、假安南將軍,別將長驅往赴,受昶節度。寶夤受詔,泣涕橫流,哽咽良久。後昶軍敗,唯寶夤全師而還。



 延昌初,除瀛州刺史,復其齊王,遷冀州刺史。及大乘賊起,寶夤遣軍討之,頻為賊破。臺軍至,乃滅之。靈太后臨朝,還京師。



 梁將
 康絢於浮山堰淮以灌揚、徐。除寶夤使持節、都督東討軍事、鎮東將軍以討之,復封梁郡公。熙平初,梁堰既成,淮水將為揚、徐之患,寶夤乃於堰上流更鑿新渠,水乃小減。乃遣壯士千餘人夜度淮,燒其竹木營聚,破其三壘,火數日不滅。又分遣將破梁將垣孟孫、張僧副等於淮北。仍度淮南,焚梁徐州刺史張豹子等十一營。及還京師,為殿中尚書。寶夤之在淮堰,梁武寓書招誘之。寶夤表送其書,陳其忿毒之意。志存雪復,屢請居邊。神龜中,為都督、徐州刺史、車騎大將軍。



 乃起學館於清東,朔望引見土姓子弟,接以恩顏,與論經義。勤於聽理,吏人
 愛之。



 正光二年,徵為尚書左僕射。善於吏職,甚有聲名。四年,上表曰:竊惟文武之名,在人之極地;德行之稱,為生之最首。忠貞之美,立朝之譽;仁義之號,處身之端。自非職惟九官,任當四嶽,授曰爾諧,讓稱俞往,將何以克厭大名,允茲令問。自比以來,官罔高卑,人無貴賤,皆飾辭假說,用相褒舉。求者不能量其多少,與者不能核其是非,遂使冠履相貿,名實皆爽。謂之考功,事同汎陟,紛紛漫漫,焉可勝言!又在京之官,積年十考。其中,或所事之主,遷移數四;或所奉之君,身亡廢絕。雖當時文簿,記其殿最,日久月遙,散落都盡。累年之後,方求追訪,無不
 茍相悅附,共為脣齒;飾垢掩疵,妄加丹素,趣令得階而已,無所顧惜。賢達君子,未免斯患;中庸已降,夫復何論!官以求成,身以請立,上下相蒙,莫斯為甚。



 又勤恤人隱,咸歸守令,厥任非輕,所責實重。然及其考課,悉以六載為約,既而限滿代還,復經六年而敘。是則歲周十二,始得一階。於東西兩省,文武閑職,公府散佐,無事冗官,或數旬方應一直,或朔望止於暫朝。及其考日,更得四年為限。是則一紀之中,便登三級。彼以實勞劇任,而遷貴之路至難;此以散位虛名,而升陟之方甚易。何內外之相縣,令厚薄之若此!



 孟子曰:「仁義忠信,天爵也;公卿大
 夫,人爵也。古之人,脩其天爵而人爵從之。」故雖文質異時,污隆殊世,莫不寶茲名器,不以假人。是以賞罰之科,恆自持也。乃至周之藹藹,五叔無官;漢之察察,館陶徒請。誠以賞罰一差,則無以懲勸;至公暫替,則覬覦相欺。故至慎至惜,殷勤若此。況乎親非肺腑,才乖秀逸,或充單介之使,始無汗馬之勞;或說興利之規,終縣十一之潤。皆虛張無功,妄指贏益;坐獲數階之官,籍成通顯之貴。於是巧詐萌生,偽辯鋒出,役萬慮以求榮,開百方而逐利。抑之則其流已往,引之則有何紀極!



 夫琴瑟在於必和,更張求其適調。去者既不可追,來者猶或宜改。案《
 周官》:太宰之職,歲終,則令官府各正所司,受其會計,聽其事致而詔於王。三歲,則大計群吏之政而誅賞之。



 愚謂今可粗依其準。見居官者,每歲終,本曹皆明辨在官日月,具核才行能否,審其實用。而注其上下,游辭宕說,一無取焉。列上尚書,覆其合否。如有紕繆,即正而罰之,不得方復推詰委下,容其進退。既定其優劣,善惡交分,經奏之後,考功曹別書於黃紙、油帛。一通則本曹尚書與令僕印署,留於門下;一通則以侍中黃門印署,掌在尚書。嚴加緘密,不得開視。考績之日,然後對共裁量。其外內考格,裁非庸管,乞求博議,以為畫一。若殊謀異策,
 事關廢興,遐邇所談,物無異議者,自可臨時斟酌,匪拘恆例。至如援流引比之訴,貪榮求級之請。如不限以關鍵,肆其傍通,則蔓草難除,涓流遂積,穢我彞章,撓茲大典,謂宜明加禁斷,以全至化。



 詔付外博議,以為永式。竟無所改。



 時梁武弟子西豐侯正德來降,寶夤表曰:「正德既不親親,安能親人。脫包此凶醜,置之列位,百官是象,其何誅焉?臣釁結禍深,痛纏骨髓,日暮途遙,報復無日,豈區區於一豎哉!但才雖庸近,職居獻替,愚衷寸抱,敢不申陳。」正德既至京師,朝廷待之尤薄,歲餘,還叛。



 初,秦州城人薛伯珍、劉慶、杜遷等反,執刺史李彥,推莫折大
 提為首,自稱秦王。大提尋死,其第四子念生竊號天子,年曰天建。置官寮,以息阿胡為太子,其兄阿倪為西河王,弟天生為高陽王,伯珍為東郡王,安保為平陽王。天生率眾出隴東,遂寇雍州,屯於黑水。朝廷甚憂之,除寶夤開府、西道行臺,為大都督,西征。明帝幸明堂以餞之。寶夤與大都督崔延伯擊天生大破之,追奔至小隴。進討高平賊帥萬俟醜奴於安定,更有負捷。



 時有天水人呂伯度兄弟始共念生同逆,後與兄眾保於顯親聚眾討念生。戰敗,奔於胡琛。琛以伯度為大都督、秦王,資其士馬,還征秦州。大破念生將杜粲於成紀,又破其金城
 王莫折普賢於水洛城,遂至顯親。念生率眾身自拒戰,又大敗。伯度乃背胡琛,遣其兄子忻和率騎東引大軍。念生事迫,乃詐降於寶夤。朝廷嘉伯度立義之功,授涇州刺史、平秦郡公。而大都督元脩義、高聿停軍隴口,久不西進。



 念生復反,伯度為醜奴所殺。故賊勢更甚,寶夤不能制。



 孝昌二年,除寶夤侍中、驃騎大將軍、儀同三司、假大將軍、尚書令,給前後部鼓吹。寶夤初自黑水,終至平涼,與賊相對,年年攻擊,賊亦憚之。關中保全,寶夤之力。三年正月,除司空公。出師既久,兵將疲弊,是月大敗,還雍州。有司處寶夤死罪,詔恕為編戶。四月,除征西將
 軍、雍州刺史、開府、西討大都督,自關以西,皆受節度。九月,念生為其常山王杜粲所殺,合門皆盡。粲降寶夤。十月,除尚書令,復其舊封。



 時山東、關西,寇賊充斥,王師屢北,人情沮喪。寶夤自以出師累年,糜費尤廣,一旦覆敗,慮見猜責,內不自安。朝廷頗亦疑阻。及遣御史中尉酈道元為關中大使,寶夤謂密欲取己,將有異圖,問河東柳楷。楷曰:「大王齊明帝子,天下所屬,今日之舉,實允人望。且謠言:『鸞生十子九子段,一子不段關中亂。』武王有亂臣十人,亂者理也,大王當理關中,何所疑慮?」



 道元行達陰盤驛,寶夤密遣其將郭子恢等攻殺之,而詐收道
 元尸,表言白賊所害。遂反,僭舉大號,大赦其部內,稱隆緒元年,立百官。詔尚書僕射、行臺長孫承業討之。時北地毛鴻賓與其兄遐糾率鄉義,將討寶夤。寶夤遣其將侯終德往攻遐。



 終德還圖寶夤,軍至白門,寶夤始覺。與終德戰,敗,攜公主及其少子與部下百餘騎從後門出,遂奔萬俟醜奴。醜奴以寶夤為太傅。



 爾朱天光遣賀拔岳等破醜奴於安定,追禽醜奴及寶夤,並送京師。詔置閶闔門外都街中,京師士女聚觀,凡經三日。吏部尚書李神俊、黃門侍郎高道穆並與寶夤素舊,二人相與左右,言於莊帝,云其逆迹事在前朝,冀將救免。會應詔王
 道習時自外至,莊帝問道習在外所聞。道習曰:「唯聞陛下欲不殺蕭寶夤。人云李尚書、高黃門與寶夤周款,並居得言之地,必能全之。」道習因曰:「若謂寶夤逆在前朝,便將恕之;敗在長安,為醜奴太傅,並非陛下御歷之日?賊臣不翦,法欲安施?」



 帝然其言,乃於太僕駝牛署賜死。將刑,神俊攜酒就之敘故舊,因對之下泣。寶夤夷然自持,了不憂懼,唯稱推天委命,恨不終臣節。公主攜男女就寶夤訣別,慟哭極哀,寶夤亦色貌不改。



 寶夤三子皆公主所生,並凡劣。長子烈,復尚明帝妹建德公主,拜駙馬都尉,坐寶夤反,伏法。次子權與小子凱射戲,凱矢激,
 中之,死。凱妻,長孫承業女也,輕薄無禮,公主數加罪責。凱竊銜恨,妻復惑說之。天平中,凱遣奴害公主。乃轘凱於東市。妻梟首,家遂滅。寶夤兄子贊。



 贊字德文,本名綜。初,梁武滅齊,齊廢主東昏侯寶卷宮人吳氏始孕,匿不言;及生贊,梁武以為己子,封豫章王。及長,學涉有才思。其母告之以實。贊晝則談謔,夜則銜悲涕泣。有濟陰苗文寵、安定梁話,贊曲加禮接,割血自誓,布以心腹。



 寵、話感其情義,深相然諾。會元法僧以彭城叛入梁,梁武命贊都督江北諸軍事,鎮彭城。
 時明帝遣安豐王延明、臨淮王彧討之,贊與寵、話夜奔延明。



 孝昌元年秋,屆于洛陽。陛見後,就館舉哀,追服三載。寶夤時在關西,遣使觀察,問其形貌,斂眉悲感。朝廷賞賜豐渥,禮遇隆厚,授司空,封高平郡公、丹楊王。及寶夤反,贊怖,欲奔白鹿山,至河橋,為北中所執。朝議明其不相干預,仍蒙慰免。



 建義初,轉司徒,遷太尉,尚帝姊壽陽長公主,拜駙馬都尉。出為都督齊州刺史、驃騎大將軍、開府儀同三司。寶夤見禽,贊拜表請寶夤命。



 爾硃兆入洛,為
 城
 人趙洛周所逐。公主被錄送京,爾朱世隆欲相陵逼。公主守操被害。贊既棄州,為沙門,潛詣長白山。未幾,至陽平,病卒。



 贊機辯,文義頗有可觀,而輕薄俶儻,猶有父
 風。普泰初,迎其喪,以王禮與公主合葬嵩山。



 元象初,吳人盜其喪還江東,梁武猶以為子,祔葬蕭氏墓焉。贊,江南有子,在魏無後。



 蕭正表,字公儀,梁武帝弟臨川王宏之子也。在梁封山陰縣侯,位北徐州刺史,鎮鐘離。正表長七尺九寸,雖質貌豐美,而性理短暗。



 初,梁武未有子,以正表兄西豐侯正德為子。及自有子,正德歸本,私懷忿憾,以正光三年,背梁奔魏。魏朝以其人才庸劣,不禮焉。尋逃歸梁,梁武不之罪,封為臨賀王。



 侯景將濟江,知正德有恨,密與交通,許推為主,正德以船迎之。景度,攻揚州。正表聞正德
 為侯景所推,盤桓不赴援。景尋以正表為南兗州刺史,封南郡王。



 正表遂於歐陽立柵,斷梁援軍。南兗州刺史南康王蕭會理遣兵擊破之。正表走還鐘離,以武定七年,據州內屬,封蘭陵郡王。尋除侍中、太子太保、開府儀同三司。



 薨,贈司空公,謚曰昭烈。子廣壽。



 蕭祗,字敬式,梁武帝弟南平王偉之子也。少聰敏,美容儀。在梁封定襄縣侯,位東揚州刺史。于時江左承平,政寬人慢。祗獨蒞以嚴切,梁武悅之,遷北兗州刺史。太清二年,侯景圍建業,祗聞臺城失守,遂來奔,以武定七年至鄴。齊文襄令魏收、邢邵與相接對。歷位太子少傅,領
 平陽王師,封清河郡公。齊天保初,授右光祿大夫,領國子祭酒。時梁元帝平侯景,復與齊通好,文宣欲放祗等還南。俄而西魏克江陵,遂留鄴。卒,贈中書監、車騎大將軍、揚州刺史。



 子放,字希逸,隨祗至鄴。祗卒,放居喪以孝聞。所居廬室前,有二慈烏來集,各據一樹為巢,自午以前,馴庭飲啄;午後更不下樹。每臨時舒翅悲鳴,全似哀泣。



 家人則之,未嘗有闕。時以為至孝之感。服闋,襲爵。武平中,待詔文林館。



 放性好文詠,頗善丹青,因此在宮中披覽書史及近世詩賦,監畫工作屏風等雜物。見知,遂被眷待。累遷太子中庶子、散騎常侍。



 蕭退,梁武帝弟司空、鄱陽王恢之子也。退在梁封湘潭侯,位青州刺史。建業陷,與從兄祗俱入東魏。齊天保中,位金紫光祿大夫,卒。



 子慨,深沈有體表,好學,善草隸書,南士中稱為長者。歷著作佐郎,待詔文林館。卒於司徒從事中郎。



 蕭泰,字世怡,亦恢之子也。在梁封豐城侯,位譙州刺史。侯景襲而陷之,因被執,尋逃至江陵。梁元帝平侯景。以泰為兼太常卿、桂陽內史。未至郡,屬于謹平江陵,遂隨兄脩佐郢州。及脩卒,即以泰為刺史。湘州刺史王琳襲泰,泰以州輸琳。時陳武帝執政,徵為侍中,不就。乃奔齊,
 為永州刺史。保定四年,大將軍權景宣略地河南,泰遂歸西魏。以名犯周文帝諱,稱字焉。拜開府儀同三司,封義興郡公,授蔡州刺史。政存簡惠,深為吏人所安。卒官,子寶嗣。



 寶字季珍,美風儀,善談笑,未弱冠,名重一時。隋文帝輔政,引為丞相府典簽。開皇中,至吏部侍郎。後坐太子勇事誅,時人冤之。



 蕭捴,字智遐,梁武帝弟安成王秀之子也。性溫裕,有儀表,在梁封永豐縣侯。



 東魏遣李諧、盧元明使梁。梁武帝以捴辭令可觀,令兼中書侍郎,受幣於賓館。歷黃門侍郎,累遷巴西、梓潼二郡守。及侯景作亂,武陵王紀稱
 尊號。時宗室在蜀,唯捴一人,封捴秦郡王。紀率眾東下,以捴為尚書令、征西大將軍、都督、益州刺史,守成都。又令梁州刺史楊乾運守潼州。



 周文帝知蜀兵寡弱,遣大將軍尉遲迥總眾討之。迥入劍閣,長驅至成都。捴見兵不滿萬人,而倉庫空竭,於是率文武於益州城北,共迥升壇歃血立盟,以城歸魏。



 授侍中、開府儀同三司,封歸善縣公。周閔帝踐阼,進爵黃臺郡公。



 武成中,明帝令諸文儒於麟趾殿校定經史,仍撰《世譜》,捴亦豫焉。尋以母老,兼有疾疹,請在外著書,詔許之。保定元年,授禮部中大夫,又以歸款功,別賜食多陵縣五百戶,收其租賦。三
 年,出為上州刺史。為政以禮讓為本,嘗至元日,獄中囚繫,悉放歸家,聽三日然後赴獄。主者爭之,捴曰:「昔王長、虞延,見稱前史。吾雖寡德,竊懷景行。以之獲罪,彌所甘心。」諸囚荷恩,並依限而至,吏人稱其惠化。秩滿向還,部人季漆等三百餘人上表,乞留更兩載。詔雖不許,甚嘉美之。



 及捴入朝,屬置露門學。武帝以捴與唐瑾、元偉、王褒等四人,俱為文學博士。



 捴以母老,表請歸養私門,帝弗許。尋以母憂去職。歷少保、少傅,改封蔡陽郡公。卒,武帝舉哀於正武殿,贈使持節、大將軍、大都督、少傅、益州刺史,謚曰襄。



 捴善草隸,書名亞王褒,算數醫方,咸亦留
 意。所著詩賦雜文數萬言,頗行於世。



 子濟,字德成,少仁厚,頗好屬文。為東中郎將,從捴入朝。周孝閔帝踐阼,除中外府記室,後至薄陽郡守。



 蕭圓肅,字明恭,梁武帝之孫,武陵王紀之子也。風度淹雅,敏而好學。紀稱尊號,封宜都王,除侍中。紀下峽,令圓肅副蕭捴守成都。及尉遲迥至,與捴俱降。



 授開府儀同三司、侍中,封安化縣公。周明帝初,進棘城郡公,以歸款勳。別賜食思君縣五百戶,收其租賦。後拜咸陽郡守,甚有政績。尋改授太子少傅,作《少傅箴》。太子見而悅之,致書勞問。改授豐州刺史,尋進位上開府儀同大將軍,歷
 司宗中大夫、洛州刺史,進位大將軍。隋開皇初,授貝州刺史,以母老請歸就養,許之,卒於家。有文集十卷,又撰時人詩筆為《文海》四十卷、《廣堪》十卷、《淮海離亂志》四卷,行於世。



 蕭大圜,字仁顯,梁簡文帝第二十子也。幼而聰敏,年四歲,能誦《三都賦》及《孝經》、《論語》,七歲居母喪,便有成人性。梁大寶元年,封樂梁郡王,丹楊尹。屬侯景殺簡文,大圜潛遁獲免。景平,歸建業。時喪亂之後,無所依,乃寓居善覺佛寺。人有以告王僧辯,乃給船餼,得往江陵。梁元帝見之甚悅,賜以越衫胡帶,改封晉熙郡王,除瑯邪、彭城
 二郡太守。



 時大圜兄汝南王大封等猶未通謁。元帝性忌刻,甚恨望之,乃使大圜召之。大圜即日曉諭,兩兄相繼出謁,元帝乃安之。大圜恐讒忻醖生,乃屏絕人事;門客左右,不過三兩人。不妄遊狎,兄姊間,止箋疏而已。恆以讀《詩》、《禮》、《書》、《易》為事。元帝嘗自問《五經》要事數十條,大圜詞約指明,應答無滯。帝甚歎美之,因曰:「昔河間好學,爾既有之;臨淄好文,爾亦兼之。然有東平為善,彌高前載。」及於謹軍至,元帝乃令大封充使請和,大圜副焉,其實質也。出至軍所,信宿,元帝降。



 魏恭帝二年,大圜至長安,周文帝以客禮待之。保定二年,大封為晉陵縣公,大圜
 始寧縣公。尋加大圜車騎大將軍、儀同三司。俄而開麟趾殿,招集學士,大圜預焉。《梁武帝集》四十卷、《簡文集》九十卷各止一本,江陵平後,並藏秘閣。



 大圜入麟趾,方得見之,乃手寫二集,一年並畢,識者稱歎之。



 大圜深信因果,心安閑放,嘗云:拂衣褰裳,無吞舟之漏網;挂冠縣節,慮我志之未從。儻獲展禽之免,有美慈明之進。如蒙北叟之放,實勝濟南之徵。其故何哉?夫閭閻者有優遊之美,朝廷者有簪佩之累,蓋由來久矣。留侯追蹤於松子,陶朱成術於辛文,良有以焉。況乎智不逸群,行不高物,而欲辛苦一生,何其僻也。



 豈如知足知止,蕭然無累。北
 山之北,棄絕人間;南山之南,超踰世網。面脩原而帶流水,倚郊甸而枕平皋。築蝸舍於叢林,構環堵於幽薄。近瞻煙霧,遠睇風雲。藉纖草以蔭長松,結幽蘭而援芳桂。仰翔禽於百仞,俛泳鱗於千尋。果園在後,開窗以臨花卉;蔬圃居前,坐簷而看灌畎。二頃以供饘粥,十畝以給絲麻。侍兒五三,可充糸任織;家僮數四,足代耕耘。沽酷牧羊,協潘生之志;畜雞種黍,應莊叟之言。獲菽尋汜氏之書,露葵征尹君之錄。烹羔豚而介春酒,迎伏臘而候歲時。



 披良書,採至賾,歌纂纂,唱烏烏。可以娛神,可以散慮。有朋自遠,揚榷古今;田畯相過,劇談稼穡。斯亦足矣,樂
 不可支,永保性命,何畏憂責。



 豈若蹙足入絆,申頸就羈。遊帝王之門,趨宰衡之勢。不知飄塵之少選,寧覺年祀之斯須。萬物營營,靡存其意;天道昧昧,安可問哉?



 嗟乎!人生若浮,朝露寧俟。長繩繫景,實所願言。執燭夜遊,驚其迅邁。百年幾何,擎跽曲拳。四時如流,俯眉躡足。出處無成,語默奚當。非直丘明所恥,抑亦宣尼恥之。



 建德四年,除滕王逌友。逌嘗問大圜曰:「吾聞湘東王作《梁史》,有之乎?



 餘傳乃可抑揚,帝紀奚若?隱則非實,記則攘羊。」對曰:「言之妄也。如使有之,亦不足怪。昔漢明為《世祖紀》,章帝為《顯宗紀》,殷鑒不遠,足為成例。且君子之過,如日月
 之蝕,彰於四海,安得而隱之?如有不彰,亦安得不隱?蓋子為父隱,直在其中,諱國之惡,抑又禮也。」逌乃大笑。後大軍拔晉州,或問大圜:「師遂克不?」對曰:「高歡昔以晉州肇基偽迹,今本既拔矣,能無亡乎?所謂君以此始,必以此終。」居數月,齊氏果滅。聞者以為知言。



 隋開皇初,拜內史侍郎,卒於西河郡守。撰《梁舊事》三十卷、《寓記》三卷、《士喪儀注》五卷、《要決》兩卷,并文集二十卷。



 大封位開府儀同三司、陳州刺史。



 論曰:諸司馬以亂亡歸命,楚之最可稱乎!其餘碌碌,未足論也。而以往代遺緒,並當位遇,可謂幸矣。劉昶猜疑懼禍,蕭夤亡破之餘,並潛骸竄影,委命上國。



 俱稱曉了,
 盛當位遇。雖有枕戈之志,終無鞭墓之成。昶諸子狂疏,喪其家業;寶夤背恩忘義,梟鏡其心。蕭贊臨邊脫身,晚去仇賊,寵祿頓臻,顛狽旋至,信吉兇之相倚也。梁氏云季,子弟奔亡。王表動不由仁,胡顏之甚。祗、退、泰、捴、圓肅、大圜等雖羈旅異國,而終享榮名,非素有鎡基,懷文抱質,亦何能至於此也。



 方武陵擁眾東下,任捴以蕭何之事。君臣之道既篤,家國之情亦隆。金石不足比其心,河水不足明其誓。及魏安之至城下,旬日而智力俱竭,委金湯而不守,舉庸蜀而來王。若乃見機而作,誠有之矣;守節沒齒,則未可焉。



\end{pinyinscope}