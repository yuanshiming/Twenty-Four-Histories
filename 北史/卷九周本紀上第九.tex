\article{卷九周本紀上第九}

\begin{pinyinscope}

 周
 太祖文皇帝姓宇文氏,諱泰,字黑獺,代郡武川人也。其先出自炎帝。炎帝為黃帝所滅,子孫遁居朔野。其後有葛烏兔者,雄武多算略。鮮卑奉以為主,遂總十二部落,世為大人及其裔孫曰普回,因狩得玉璽三紐,文曰「皇帝璽」,普回以為天授,己獨異之。其俗謂天子曰「宇文」,故國號宇文,並以為氏。普回子莫那,自陰山南徙。始居遼西,是曰獻侯,為魏舅甥之國。自莫那九世至侯歸豆,
 為慕容晃所滅。其子陵仕燕,拜駙馬都尉,封玄菟公。及慕容寶敗,歸魏,拜都牧主,賜爵安定侯。



 天興初,魏遷豪傑於代都,陵隨例徙居武川,即為其郡縣人焉。陵生系,系生韜,韜生皇考肱,並以武略稱。肱任俠有氣幹。正光末,沃野鎮人破六韓拔陵作亂,其偽署王衛可瑰最盛。肱乃糾合鄉里,斬瑰,其眾乃散。後陷鮮于修禮,為定州軍所破,戰沒於陣。武成初,追謚曰德皇帝。



 帝,德皇帝之少子也。母曰王氏。初孕五月,夜夢抱子升天,纔不至而止。寤,以告德皇帝。德皇帝喜曰:「雖不至天,貴亦極矣。」帝生而有黑氣如蓋,下覆其身。及長,身長八尺,方顙廣額,
 美鬚髯,髮長委地,垂手過膝,背有黑子,宛轉若龍盤之形,面色紫光,人望而敬畏之。少有大度,不事家人生業。輕財好施,以交結賢士大夫為務。隨德皇帝在鮮于修禮軍。及葛榮殺修禮,帝時年十八。榮下任將帥,察其無成,謀與諸兄去之。計未行,會榮滅,因隨爾朱榮遷晉陽。榮忌帝兄弟雄傑,遂託以他罪誅帝第三兄洛生。帝以家冤自理,辭旨慷慨。榮感而免之,益加敬待。始以統軍從榮征討,後以別將從賀拔岳討北海王顥於洛陽。孝莊反正,以功封寧都子。後從岳入關,平萬俟醜奴,行原州事。時關、隴寇亂,帝撫以恩信,百姓皆喜,曰:「早遇宇文
 使君,吾等豈從逆亂。」帝嘗從數騎於野,忽聞簫鼓之音,以問從者,皆莫之聞,意獨異之。



 普泰二年,爾朱天光東拒齊神武,留弟顯壽鎮長安,召秦州刺史侯莫陳悅東下。



 岳知天光必敗,欲留悅共圖顯壽,計無所出。帝謂岳曰:「今天光尚近,悅未必貳心;若以此事告之,恐其驚懼。然悅雖為主將,不能制物,若先說其眾,必人有留心。進失爾朱之期,退恐人情變動;若乘此說悅,事無不遂。」岳大喜,即令帝入悅軍說之。悅遂與岳襲長安。帝輕騎為前鋒,追至華陰,禽顯壽。及岳為關西大行臺,以帝為左丞,領岳府司馬,事無巨細,皆委決焉。



 齊神武既除爾朱
 氏,遂專朝政。帝請往觀之,至并州。神武以帝非常人,曰:「此小兒眼目異。」將留之。帝詭陳忠款,具託左右,苦求復命,倍道而行。行一日而神武乃悔,發上驛千里,追帝至關,不及而反。帝還,謂岳曰:「高歡豈人臣邪,逆謀未發者,憚公兄弟耳。侯莫陳悅本實庸材,亦不為歡忌,但為之備,圖之不難。今費也頭控弦之騎,不下一萬;夏州刺史解拔彌俄突,勝兵三千餘人,及靈州刺史曹泥,並恃僻遠,常懷異望。河西流人紇豆陵伊利等,戶口富實,未奉朝風。



 今若移軍近隴,扼其要害,示之以威,懷之以德,即可收其士馬,以資吾軍。西輯氐、羌,北撫沙塞,還軍長安,
 匡輔魏室,此桓文之舉也。」岳大悅。復遣帝詣闕請事,密陳其狀。魏帝納之,加帝武衛將軍,還令報岳。岳遂引軍西次平涼。岳以夏州鄰接寇賊,欲求良刺史以鎮之,眾皆舉帝。岳曰:「宇文左丞,吾左右手,何可廢也。」沉吟累日,乃從眾議,表帝為夏州刺史。帝至州,伊利望風款附;而曹泥猶通使於齊神武。



 魏永熙三年正月,賀拔岳欲討曹泥,遣都督趙貴至夏州與帝謀。帝曰:「曹泥孤城阻遠,未足為憂。侯莫陳悅貪而無信,是宜先圖也。」岳不聽,遂與悅俱討泥。



 二月,至河曲,果為悅所害。眾散還平涼,唯大都督趙貴率部曲收岳屍還營。三軍未知所屬,諸將
 以都督寇洛年最長,推總兵事。洛素無雄略,威令不行,乃請避位。



 於是趙貴言於眾,稱帝英姿雄略。若告喪,必來赴難,因而奉之,大事濟矣。諸將皆稱善。乃令赫連達馳至夏州告帝。士吏咸泣,請留以觀其變。帝曰:「難得而易失者時也,不俟終日者機也;今不早赴,將恐眾心自離。」都督彌姐元進規應悅,密圖帝。事發,斬之。帝乃率帳下,輕騎馳赴平涼。時齊神武遣長史侯景招引岳眾。



 帝至安定,遇之於傳舍。吐哺上馬,謂曰:「賀拔公雖死,宇文泰尚存,卿何為也?」



 景失色曰:「我猶箭耳,隨人所射者也。」景於此還。帝至平涼,哭岳甚慟。將士悲且喜曰:「宇文公
 至,無所憂矣。」齊神武又使景與常侍張華原、義寧太守王基勞帝,帝不受命。與基有舊,將留之,并欲留景。並不屈,乃遣之。時斛斯椿在帝所,曰:「景,人傑也,何故放之?」帝亦悔,驛追之不及。基亦逃歸,言帝雄傑,請及其未定滅之。神武曰:「卿不見賀拔、侯莫陳乎,吾當以計拱手取之。」及沙苑之敗,神武乃始追悔。于時魏帝將圖神武。聞岳被害,遣武衛將軍元毗宣旨勞岳軍,追還洛陽。毗到平涼,會諸將已推帝。侯莫陳悅亦被敕追還。悅既附神武,不肯應召。帝曰:「悅枉害忠良,復不應詔命,此國之大賊。」乃令諸軍戒嚴,將討悅。及毗還,帝表於魏帝,辭以高歡
 至河東,侯莫陳悅在水洛,首尾受敵,乞少停緩。帝志在討悅,而未測朝旨;且眾未集,假為此辭。因與元毗及諸將,刑牲盟誓,同獎王室。



 初,賀拔岳營河曲,軍吏獨行,忽見一翁,謂曰:「賀拔雖據此眾,終無所成。



 當有一宇文家從東北來,後必大盛。」言訖不見。至是方驗。魏帝因詔帝為大都督,即統賀拔岳軍。帝乃與悅書,責以殺賀拔岳罪,又喻令歸朝。悅乃詐為詔書與秦州刺史萬俟普撥,令為己援。普撥疑之,封以呈帝,帝表奏之。魏帝因問帝安秦、隴計。帝請召悅,授以內官,及處以瓜、涼一籓。不然,則終致猜虞。三月,帝進軍至原州,眾軍悉集,諭以討悅
 意,士卒莫不懷憤。四月,引兵上隴,留兄子遵為都督,鎮原州。帝軍令嚴肅,秋毫無犯,百姓大悅。軍出木峽關,大雪,平地二尺。



 帝知悅怯而多猜,乃倍道兼行,出其不意。悅果疑其左右有異志,左右不自安,眾遂離貳。聞大軍且至,退保略陽,留一萬餘人據守永洛。帝至,圍之,城降。帝即輕騎數百趣略陽,以臨悅軍。其部將皆勸悅退保上邽。時南秦州刺史李弼亦在悅軍,間遣使請為內應。其夜,悅出軍,軍自驚潰,將卒或來降。帝縱兵奮擊,大破之。



 悅與其子弟及麾下數十騎遁走。帝乃命原州都督導追悅,至牽屯山斬之,傳首洛陽。



 帝至上邽,悅府庫財
 物山積,皆以賞士卒,毫釐無所取。左右竊以一銀甕歸,帝知而罪之,即剖賜將士,眾大悅。齊神武聞關隴剋捷,遣使於帝,深相倚結。帝拒而不納,封神武書以聞。時神武已有異志,故魏帝深仗於帝,仍令帝稍引軍而東。帝乃令大都督梁禦率步騎五千,將鎮河、渭合口,為圖河東計。魏帝進帝侍中、驃騎大將軍、開府儀同三司、關西大都督、略陽縣公,承制封拜,使持節如故。



 時魏帝方圖齊神武,又遣徵兵。帝乃令前秦州刺史駱超為大都督,率輕騎一千赴洛。魏帝進授帝兼尚書左僕射、關西大行臺,餘官如故。帝乃傳檄方鎮曰:蓋聞陰陽遞用,盛衰
 相襲。茍當百六,無聞三五。皇家創歷,陶鑄蒼生;保安四海,仁育萬物。運距孝昌,屯沴屢起,隴、冀騷動,燕、河狼顧。雖靈命重啟,蕩定有期,而乘釁之徒,因翼生羽。



 賊臣高歡,器識庸下;出自輿皁,罕聞禮義。直以一介鷹犬,效力戎行;靦冒恩私,遂階榮寵。不能竭誠盡節,專挾姦回,乃勸爾朱榮行茲篡逆。及榮以專政伏誅,世隆以凶黨外叛;歡苦相敦勉,令取京師。又勸吐萬兒復為弒虐,暫立建明,以令天下;假推普泰,欲竊威權。並歸廢斥,俱見酷害。於是稱兵河北,假討爾朱;亟通表奏,云取讒賊。既行廢黜,遂將篡弒。以人望未改,恐鼎鑊交及;乃求宗室,權
 允人心。天方與魏,必將有主;翊戴聖明,誠非歡力。而歡阻兵安忍,自以為功;廣布腹心,跨州連郡,端揆禁闥,莫非親黨;皆行貪虐,窫窳生靈。而舊將名臣,正人直士,橫生瘡磐,動挂網羅。故武衛將軍伊琳,清直武毅,禁旅攸屬;直閣將軍鮮于康仁,忠亮驍傑,爪牙斯在:歡收而戮之,曾無聞奏。司空高乾;是其黨與,每相影響,謀危社稷。但姦志未從,恐先泄漏,乃密白朝廷,使殺高乾,方哭對其弟,稱天子橫戮。孫騰、任祥,歡之心膂,並使入居樞近,伺國間隙,知歡逆謀將發,相繼歸逃。歡益加撫待,亦無陳白。然歡入洛之始,本有姦謀。令親人蔡俊作牧河、濟,
 厚相恩贍,為東道主人。故關西大都督清水公賀拔岳,勳德隆重,興亡攸寄。歡好亂樂禍,深相忌毒。乃與侯莫陳悅,陰圖陷害。幕府以受律專征,便即討戮。歡知逆狀已露,稍懷旅拒,遂遣蔡俊拒代;令竇泰佐之。又遣侯景等云向白馬,輔世珍等徑趣石濟,高隆之、及婁昭等屯據壺關,韓軌之徒擁眾蒲阪。於是上書天子,數論得失,訾毀乘輿,威侮朝廷。藉此微庸,冀茲大寶;溪壑可盈,禍心不測。或言徑赴荊、楚,開疆於外;或言分詣伊、洛,取彼讒人;或言欲來入關,與幕府決戰。今聖明御運,天下清夷;百僚師師,四隩來暨;人盡忠良,誰為君側?



 而歡威福
 自己,生是亂階;緝構南箕,指鹿為馬;包藏凶逆,伺我神器。是而可忍,孰不可容。幕府折衝宇宙,親當受脤;銳師百萬,彀騎千群;裹糧坐甲,唯敵是俟;義之所在,糜軀匪吝。頻有詔書,班告天下;稱歡逆亂,徵兵致伐。今便分命將帥,應機進討。或趣其要害,或襲其窟穴,電繞蛇擊,霧合星羅。而歡違負天地,毒被人鬼;乘此掃蕩,易同俯拾。歡若度河,稍逼宮廟,則分命諸將,直取并州。幕府躬自東轅,電赴伊、洛。若固其巢穴,未敢發動;亦命群帥,百道俱前,轘裂賊臣,以謝天下。其州鎮郡縣,率土黎人,或州鄉冠冕,或勛庸世濟,並宜舍逆歸順,立效軍門。封賞之
 科,已有別格;凡百君子,可不勉哉。



 帝謂諸軍曰:「高歡雖智不足而詐有餘,今聲言欲西,其意在入洛。吾欲令寇洛率馬步萬餘,自涇州東引;王羆率甲士一萬,先據華州。歡若西來,王羆足得抗拒;如其入洛,寇洛即襲汾、晉。吾便速駕,直赴京邑,使其進有內顧之憂,退有被躡之勢。一舉大定,此為上策。」眾咸稱善。七月,帝帥眾發自高平,前軍至于弘農。而齊神武稍逼京師,魏帝親總六軍屯河橋,令左衛元斌之、領軍斛斯椿鎮武牢。帝謂左右曰:「高歡數日行八九百里,曉兵者所忌,正須乘便擊之。而主上以萬乘之重,不能度河決戰,方緣津據守。且長
 河萬里,扞禦為難,一處得度,大事去矣。」即以大都督趙貴為別道行臺,自蒲阪濟,趣并州。遣大都督李賢將精騎一千赴洛陽。會斌之與斛斯椿爭權,鎮防不守,魏帝遂輕騎入關。帝備儀衛奉迎,謁見於陽驛,免冠流涕謝罪。乃奉魏帝都長安。披草萊,立朝廷,軍國之政,咸取決於帝。仍加授大將軍、雍州刺史,兼尚書令,進封略陽郡公。別置二尚書,隨機處分。解尚書僕射,餘如故。



 初,魏帝在洛陽,許以馮翊長公主配帝,未及結納而魏帝西遷。至是詔帝尚之,拜附馬都尉。八月,齊神武襲陷潼關,侵華陰。帝率諸軍屯霸上以待之。神武留其將薛瑾守關
 而退。帝乃進軍斬瑾,虜其卒七千。還長安,進位丞相。十一月,遣儀同李虎與李弼、趙貴等討曹泥於靈州,虎引河灌之。明年,泥降,遷其豪帥于咸陽。



 十二月,魏孝武帝崩,帝與群公定冊,尊立魏南陽王寶炬為嗣,是為文帝。



 大統元年正月己酉,魏帝進帝都督中外諸軍、錄尚書事、大行臺,改封安定郡王。帝固讓王及錄尚書。魏帝許之,乃改封安定郡公。東魏將同司馬子如寇潼關,帝軍霸上。子如乃回軍自蒲津寇華州,刺史王羆擊走之。三月,帝命有司為二十四條新制,奏行之。



 二年五月,秦州刺史、建忠王萬俟普撥率所部入東魏。
 帝輕騎追之,至河北千餘里,不及而還。



 三年正月,東魏寇龍門,屯軍蒲阪,造三道浮橋度河。又遣其將竇泰趣潼關,高昂圍洛州。帝出軍廣陽,召諸將謂曰:「賊掎吾三面,又造橋,示欲必度,是欲綴吾軍,使竇泰得西入耳。且歡起兵以來,泰每先驅,下多銳卒,屢勝而驕。今襲之必剋。剋泰,則歡不戰而走矣。」諸將咸曰:「賊在近,捨而襲遠;若差跌,悔何及也。」帝曰:「歡前再襲潼關,吾軍不過霸上。今者大來,謂吾但自守耳。又狃於得志,有輕我之心。乘此擊之,何往不剋。賊雖造橋,未能徑度,比五日中,吾取泰必矣。」庚戌,帝還長安,聲言欲向隴右。
 辛亥,謁魏帝而潛軍至小關。竇泰卒聞軍至,陳未成,帝擊之。盡俘其眾,斬泰,傳首長安。高昂聞之,焚輜重而走。齊神武亦撤橋而退。帝乃還。六月,帝請罷行臺,魏帝復申前命,授帝錄尚書事,固讓乃止。八月丁丑,帝率李弼、獨孤信、梁禦、趙貴、于謹、若干惠、怡峰、劉亮、王德、侯莫陳崇、李遠、達奚武等十二將東伐,至潼關。帝乃誓於師曰:「與爾有眾,奉天威,誅暴亂。惟爾眾士,整爾甲兵,戒爾戎事,無貪財以輕敵,無暴人以作威。用命則有賞,不用命則有戮,爾眾士其勉之。」乃遣于謹先徇地至盤豆,拔之。獲東魏將高叔禮,送于長安。戊子,至弘農,攻之,城潰。禽
 東魏陜州刺史李徽伯,虜其戰士八千。守將高千走度河,命賀拔勝追禽之,並送長安。於是宜陽、邵郡皆歸附。先是河南豪傑應東魏者,皆降。齊神武懼,率眾下蒲阪,將自后土濟。遣其將高昂以三萬人出河南。是歲,關中饑,帝館穀於弘農五十餘日。



 時軍士不滿萬人,聞神武將度,乃還。神武遂度河,逼華州。刺史王羆嚴守,乃涉洛,軍於許原西。帝至渭南,徵諸州兵,未會。將擊之,諸將以眾寡不敵,請且待歡更西以觀之。帝曰:「歡若至咸陽,人情轉騷擾。今及其新至,可擊之。」即造浮橋於渭,令軍士齎三日糧,輕騎度渭,輜重自渭南,夾渭而西。十月壬辰,
 至沙苑。距齊軍六十餘里,神武引軍來會。癸巳,侯騎告齊軍至,帝召諸將謀。李弼曰:「彼眾我寡,不可平地置陣。此東十里,有渭曲,可先據以待之。」遂進至渭,背水東西為陣。李弼為右拒,趙貴為左拒。命將士皆偃戈於葭蘆中,聞鼓聲而起。日晡,齊師至,望見軍少,競萃於左,軍亂不成列。兵將交,帝鳴鼓,士皆奮起。于謹等六軍與之合戰,李弼等率鐵騎橫擊之。絕其軍為二,遂大破之,斬六千餘級,臨陣降者二萬餘人。神武夜遁,追至河上,復大剋。前後虜其卒七萬,留其甲兵二萬,餘悉縱歸。收其輜重兵甲,獻俘長安。李穆曰:「高歡膽破矣,逐之可獲。」



 帝不
 聽,乃還軍渭南。時所徵諸州兵始至。乃於戰所,準當時兵,人種樹一株,栽柳七千根,以旌武功。魏帝進帝柱國大將軍,增邑并前五千戶。李弼等十二將,亦進爵增邑。以左僕射、馮翊王元季海為行臺,與開府獨孤信帥步騎二萬向洛陽。賀拔勝、李弼度河圍蒲阪。蒲阪鎮將高子信開門納勝軍,東魏將薛崇禮棄城走,勝等追獲之。帝進軍蒲阪,略定汾、絳。初,帝自弘農入關後,東魏將高昂圍弘農。聞其軍敗,退守洛陽。獨孤信至新安,昂復走度河,遂入洛陽。自梁、陳已西,將吏降者相屬。於是東魏將堯雄、趙育、是云寶出潁川,欲復降地。帝遣儀同宇文
 貴、梁遷等逆擊,大破之,趙育來降。東魏復遣任祥率河南兵與堯雄合,儀同怡峰與貴、遷等復擊破之。又遣都督韋孝寬取豫州。是云寶殺其東揚州刺史那椿,以州來降。



 四年三月,帝率諸將入朝,禮畢還華州。七月,東魏將侯景等圍獨孤信於洛陽,齊神武繼之。帝奉魏帝至穀城,臨陣斬東魏將莫多婁貸文,悉虜其眾,送弘農。遂進軍瀍東。景等夜解圍去。及旦,帝率輕騎追至河上。景等北據河橋,南屬芒山為陣,與諸軍戰。帝馬中流矢,驚逸,軍中擾亂。都督李穆下馬授帝,軍復振。於是大捷,斬其將
 高昂、李猛、宋顯等,虜其甲士一萬五千人,赴河死者萬數。是日,置陣既大,首尾懸遠,從旦至未,戰數十合,氛霧四塞,莫能相知。獨孤信、李遠居右,趙貴、怡峰居左,戰並不利。又未知魏帝及帝所在,皆棄其卒先歸。開府李虎、念賢等為後軍;遇信等退,即與俱還。由是班師,洛陽亦失守。大軍至弘農,守將皆已棄城西走。所虜降卒在弘農者,因相與閉門拒守。進攻拔之,誅其魁首數百人。大軍之東伐也,關中留守兵少,而前後所虜東魏士卒,皆散在百姓間,乃謀亂。及李虎等至長安,計無所出。乃與太尉王盟、僕射周惠達輔魏太子出次渭北。



 關中大震
 恐,百姓相剽劫。於是沙苑所俘軍人趙青雀、雍州人于伏德等遂反。青雀據長安子城,伏德保咸陽;與太守慕容思度各收降卒,以拒還師。長安城人皆相率拒青雀,每日接戰。魏帝留止閿鄉,令帝討之。長安父老見帝,且悲且喜曰:「不意今日,復得見公。」士女咸相賀。華州刺史宇文導襲咸陽,斬思度,禽伏德,南度渭,與帝會,攻破青雀。太傅梁景睿先以疾留長安,遂與青雀通謀。至是亦伏誅,關中乃定。魏帝還長安,帝復屯華州。十二月,是云寶襲洛陽,東魏將王元軌棄城走。都督趙剛襲廣州拔之。自襄、廣以西城鎮復西屬。



 五年冬,大閱於華陰。



 六年春,東魏將侯景出三鴉,將侵荊州。帝遣開府李弼、獨孤信各率騎出武關,景乃還。夏,蠕蠕度河至夏州,帝召諸軍屯沙苑以備之。



 七年十一月,帝奏行十二條制,恐百官不勉於職事,又下令申明之。



 八年十月,齊神武侵汾、絳,圍玉壁。帝出軍蒲阪,神武退;度汾追之,遂遁去。十二月,魏帝狩於華陰,大饗將士。帝帥諸將,朝於行在所。



 九年二月,東魏北豫州刺史高慎舉州來附,帝帥師迎
 之。三月,齊神武據芒山陣,不進者數日。帝留輜重於瀍曲,軍士銜枚,夜登芒山,未明擊之。神武單騎為賀拔勝所逐,僅免。帝率右軍若干惠,大破神武軍,悉虜其步卒。趙貴等五將軍居左,戰不利。神武復合戰,帝又不利,夜引還。入關,屯渭上。神武進至陜,開府達奚武等禦之,乃退。帝以芒山諸將失律,上表自貶,魏帝不許。於是廣募關、隴豪右,以增軍旅。十月,大閱於櫟陽,還屯華州。



 十年五月,帝朝京師。七月,魏帝以帝前後所上二十四條及十二條新制,方為中興永式;命尚書蘇綽更損益之,總為五卷,班於天下。於是搜簡賢才為牧、守、令,習新
 制而遣焉。數年間,百姓便之。十月,大閱於白水。



 十一年十月,大閱於白水,遂西狩岐陽。



 十二年春,涼州刺史宇文仲和據州反,瓜州人張保害刺史成慶以應之,帝遣開府獨孤信討之。東魏將侯景侵襄州,帝遣開府若干惠禦之,至穰,景遁去。五月,獨孤信平涼州,禽仲和,遷其百姓六千餘家於長安。瓜州都督令狐延起義誅張保,瓜州平。七月,帝大會諸軍於咸陽。



 十三年正月,東魏河南大行臺侯景舉河南六州來附,被圍於潁川。六月,帝遣開府李弼援之,東魏將韓軌等
 遁去。景遂徙鎮豫州。於是遣開府王思政據潁川,弼引軍還。七月,侯景密圖附梁,帝知其謀,悉追還前後所配景將士。景懼,遂叛。



 冬,帝奉魏帝西狩咸陽。



 十四年春,魏帝詔封帝長子覺為寧都郡公。初,帝以平元顥納孝莊帝功,封寧都縣子。至是,改以為郡,以封覺,用彰勤王之始也。五月,魏帝進帝位太師。帝奉魏太子巡撫西境,登隴,刻石紀事。遂至原州,歷北長城,大狩。東趣五原,至蒲州,聞魏帝不豫而還。及至,魏帝疾已愈,乃還華州。是歲,東魏將高岳圍王思政於潁川。



 十五年春,帝遣大將軍趙貴帥師援王思政。高岳堰洧
 水以灌城,潁川以北皆為陂澤,救兵不得至。六月,潁川陷。初,侯景圍建鄴,梁司州刺史柳仲禮赴臺城。



 梁竟陵郡守孫皓以郡內附,帝使大都督苻貴鎮之。及建鄴陷,仲禮還司州,來寇。



 皓以郡叛,帝大怒。十一月,遣開府楊忠攻剋隨州,進圍仲禮長史馬岫於安陸。



 十六年正月,仲禮來援安陸,楊忠逆擊於漴頭,大破之,禽仲禮。馬岫以城降。



 三月,魏帝封帝第二子震為武邑公。七月,帝東伐,拜章武公導為大將軍,總督留守諸軍,屯涇北,鎮關中。九月丁巳,軍出長安。連雨,自秋及冬,諸軍馬驢多死。



 遂於弘農北造橋濟河,自蒲阪還。於是河
 南自洛陽,河北自平陽以東,遂入齊。



 十七年三月,魏文帝崩,皇太子嗣位,帝以冢宰總百揆。十月,帝遣大將軍王雄出子午,伐上津、魏興,大將軍達奚武出散關,伐南鄭。



 廢帝元年春,王雄平上津、魏興,以其地置東梁州。四月,達奚武圍南鄭。月餘,梁州刺史宜豐侯蕭脩以州降武。八月,東梁州百姓圍州城,帝復遣王雄討之。



 二年正月,魏帝詔帝為左丞相、大行臺、都督中外諸軍事。二月,東梁州平,遷其豪帥於雍州。三月,帝遣大將軍、魏安公尉遲迥帥師伐梁武陵王蕭紀於蜀。四月,帝勒
 銳騎三萬,西踰隴,度金城河,至姑臧。吐谷渾震懼,遣使獻其方物。七月,帝至自姑臧。八月,尉遲迥剋成都,劍南平。十一月,尚書元烈謀亂,伏誅。



 三年正月,始作九命之典,以敘內外官爵。以第一品為九命,第九品為一命;改流外品為九秩,亦以九為上。又改置州、郡、縣,凡改州四十六,置州一,改郡一百六,改縣三百三十。魏帝有怨言,於是帝與公卿議,廢帝;立齊王廓。是為恭帝。



 恭帝元年四月,帝大饗群臣。魏史柳虯執簡書告於朝曰:「廢帝,文皇帝之嗣子,年七歲,文皇帝託於安定公曰:『
 是子也,才由于公;不才亦由于公,公宜勉之。』公既受茲重寄,居元輔之任,又納女為皇后;遂不能訓誨有成,致令廢黜,負文皇帝付屬之意,此咎非安定公而誰?」帝乃令太常盧辯作誥喻公卿曰:「嗚呼!



 我群后暨眾士,維文皇帝以襁褓之嗣託於予,訓之誨之,庶厥有成。而予罔能弗變厥心,庸暨乎廢墜我文皇帝之志。嗚呼!茲咎予其焉避?予實知之,矧爾眾人之心哉。惟予之顏,豈惟今厚,將恐來世,以予為口實。」乙亥,魏帝詔封帝子邕為輔城公,憲為安城公。七月,西狩至原州。梁元帝遣使請據舊圖以定疆界;又連結於齊,言辭悖慢。帝曰:「古人有言,
 天之所棄,誰能興之,其蕭繹之謂乎。」十月壬戌,遣柱國於謹、中山公護與大將軍楊忠、韋孝寬等步騎五萬討之。十一月癸未,師濟漢,中山公護與楊忠率銳騎先屯其城下。丙申,于謹至江陵,列營圍守。辛亥,剋其城,戕梁元帝,虜其百官士庶以歸,沒為奴婢者十餘萬,免者二百餘家。立蕭察為梁主,居江陵,為魏附庸。魏氏之初,統國三十六,大姓九十九,後多絕滅。



 至是,以諸將功高者為三十六國後;次者為九十九姓後;所統軍人,亦改從其姓。



 二年,梁廣州刺史王琳寇邊。十月,帝遣大將軍豆盧寧
 帥師討之。



 三年正月丁丑,初行《周禮》,建六官,魏帝進帝位太師、大冢宰。帝以漢、魏官繁,思革前弊。大統中,乃令蘇綽、盧辯依周制改創其事,尋亦置六卿官,然為撰次未成,眾務猶歸臺閣。至是始畢,乃命行之。四月,帝北巡。七月,度北河。



 魏帝封帝子直為秦郡公,招為正平公。九月,帝不豫,還至雲陽,命中山公護受遺輔嗣子。十月乙亥,帝薨於雲陽宮,還長安發喪,時年五十。十二月甲申,葬于成陵,謚文公。及孝閔帝受禪,追尊為文王,廟曰太祖。武成元年,追尊為文皇帝。



 帝知人善任使,從諫如順流。崇尚儒
 術,明達政事,恩信被物。能駕馭英豪,一見之者,咸思用命。沙苑所獲囚俘,釋而用之;及河橋之役,以充戰士,皆得其死力。諸將出征,授以方略,無不制勝。性好朴素,不尚虛飾,恆以反風俗復古始為心云。



 孝閔皇帝諱覺,字陀羅尼,文帝第三子也。母曰元皇后。大統八年,生於同州。



 七歲封略陽郡公。時善相者史元華見帝,退謂所親曰:「此公子有至貴相,但恨不壽耳。」



 魏恭帝三年三月,命為安定公世子。四月,拜大將軍。十月乙亥,文帝崩。丙子,世子嗣位為太師、大冢宰。十二月丁亥,魏帝詔以岐陽地封帝為周公。庚子,詔禪位于帝曰:「
 予聞皇天之命不于常,惟歸于德。故堯授舜,舜授禹,時宜也。



 天厭我魏邦,垂變以告,惟爾罔弗知。予雖不明,敢弗龔天命,格有德哉。今踵唐、虞舊典,禪位于周,庸布告爾焉。」使大宗伯趙貴持節奉冊書曰:「咨爾周公,帝王之位弗常,有德者受命,時乃天道。予式時庸,荒求于唐、虞之彞踵,曰我魏德之終舊矣。我邦小大罔弗知,今其可亢怫于天道而不歸有德歟。時用詢謀,僉曰:公昭考文公,格勛德于天地,丕濟黔黎。洎公,又躬宣重光。故玄象徵見于上,謳訟奔走于下,天之歷數,用實在焉,予安敢弗若。是以欽祗聖典,遜位於公。公其享茲天命,保有萬
 國,可不慎歟。」魏帝臨朝,遣戶部中大夫、濟北公元迪致皇帝璽綬。帝固辭,公卿百辟勸進,太史陳祥瑞,乃從之。是日,魏帝遜位于大司馬府。



 元年春正月,天王即位,柴燎告天,朝百官于路門。追尊皇考文公為文王,皇妣為文后,大赦。封魏帝為宋公。是日,槐里獻赤雀。百官奏議曰:「帝王之興,罔弗更正朔,明受之於天,革人視聽也。逮於尼甫,稽諸陰陽,云行夏之時,後王所不易。今魏歷告終,周室受命;以木承水,實當行錄;正用夏時,式遵聖道。惟文王誕玄氣之祥,有黑水之讖,服色宜尚烏。制曰:「可。」以大司徒、趙郡王李弼為太
 師;以大宗伯、南陽公趙貴為太傅、大冢宰;以大司馬、河內公獨孤信為太保;以大宗伯、中山公護為大司馬;以大將軍寧都公毓、高陽公達奚武、武陽公豆盧寧、小司冠陽平公李遠、小司馬博陵公賀蘭禪、小宗伯魏安公尉迥等並為柱國。



 壬寅,祀圓丘。詔曰:「予本自神農,其於二丘,宜作厥主。始祖獻侯,啟土遼海,配南北郊;文考德符五運,受天明命,祖于明堂,以配上帝。」癸卯,祀方丘。甲辰,遂祭太社。初除市門稅。乙巳,享太廟。丁未,會于乾安殿,班賞各有差。戊申,詔有司分命使者,巡察風俗,求人得失,禮餼高年,恤于鰥寡。辛亥,祀南郊。



 壬子,立王后元
 氏。辛酉,享太廟。癸亥,親耕籍田。二月癸酉朔,朝日于東郊。



 戊寅,祭太社。丁亥,柱國、楚國公趙貴謀反,伏誅。太保獨孤信罪免。甲午,以大司空、梁國公侯莫陳崇為太保;大司馬、晉國公護為大冢宰;柱國、博陵公賀蘭禪為大司馬;高陽公達奚武為大司寇;大將軍、化政公宇文貴為柱國。三月己酉,衛國公獨孤信賜死。癸亥,省六府士員三分之一。夏四月壬申,降死罪已下囚。壬午,謁成陵。丁亥,享太廟。五月己酉,帝將觀漁於昆明池,博士姜頃諫,乃止。



 秋七月壬寅,帝聽訟於右寢,多所哀宥。辛亥,享太廟。八月戊辰,祭太社。辛未,降死罪已下囚。甲午,詔二
 十四軍舉賢良。九月庚申,改太守為郡守。



 帝性剛果,忌晉公護之專。司會李植、軍司馬孫恆以先朝佐命,入侍左右,亦疾護權重,乃與宮伯乙鳳、賀拔提等潛請帝誅護,帝許之。又引宮伯張先洛。先洛以白護,護乃出植為梁州刺史,恆為潼州刺史。鳳等更奏帝,將召群臣入,因此誅護。先洛又白之。時小司馬尉綱總統宿衛兵。護乃召綱入殿中,詐呼鳳等論事,以次執送護弟,並誅之。綱乃罷禁兵,帝無左右,獨在內殿,令宮人執兵自守。護遣大司馬賀蘭祥逼帝遜位,貶為略陽公,遂幽於舊邸。月餘日,以弒崩,時年十六。



 植、恆等亦遇害。及武帝誅護後,
 乃詔曰:「故略陽公至德純粹,天姿秀傑。屬魏詐告終,寶命將改,謳歌允集,歷數攸歸。上協蒼靈之慶,下昭后祗之錫。而禍生肘腋,釁起蕭墻;白武噬驂,蒼鷹集殿;幽辱神器,弒酷乘輿;冤結生靈,毒流宇縣。今河海登清,氛沴消蕩;追尊之禮,宜崇徽號。」遣太師、蜀國公迥於南郊,上謚曰孝閔皇帝,陵曰靜陵。



 世宗明皇帝諱毓,小名統萬突。文皇帝之長子也。母曰姚夫人。永熙三年,文帝臨夏州,生於統萬城,因以名焉。大統十四年,封寧都郡公。魏恭帝三年,累遷大將軍,鎮隴右。孝閔踐阼,進位柱國,轉岐州刺史,有美政。及孝閔
 廢,晉公護遣迎帝於岐州。九月癸亥,至京師,止於舊邸。群臣上表勸進,備法駕奉迎,帝固讓,群臣固請,乃許之。



 元年秋九月,天王即位,大赦。乙丑,朝郡臣於延壽殿。冬十月癸酉,太師、趙國公李弼薨。己卯,以大將軍、昌平公尉綱為柱國。乙酉,祀圓丘。丙戌,祀方丘。甲午,祭太社。陽平公李遠賜死。辛未,梁敬帝遜位於陳。十一月庚子,享太廟。丁未,祀圓丘。十二月庚午,謁成陵。庚辰,以大將軍、輔城公邕為柱國。戊子,赦長安見囚。甲午,詔元氏子女自坐趙貴等事以來,所有沒入為官口者,悉免之。



 二年春正月乙未,以大冢宰、晉公護為太師。辛亥,親耕
 籍田。癸丑,立王后獨孤氏。丁巳,於雍州置十二郡。三月甲午,北豫州刺史司馬消難舉州來附。改雍州刺史為牧,京兆郡守為尹。庚申,詔三十六國、九十九姓,自魏南徙,皆稱河南人;今周室既都關中,宜改稱京兆人。夏四月己巳,以太師、晉公護為雍州牧。辛未,降死罪囚一等,五歲刑已下皆原之。甲戌,天王后獨孤氏崩。甲申,葬敬後。



 五月乙未,以大司空、梁國公侯莫陳崇為大宗伯。六月癸亥,嚈噠國遣使朝貢。己巳,板授高年刺史、守、令,恤鰥寡孤獨各有差。分長安為萬年縣,並居京城。壬申,遣使分行州郡,理囚徒,察風俗,掩骸埋胔。秋七月,順陽獻
 三足烏。八月甲子,群臣上表稱慶。於是大赦,文武普進級。九月辛卯,以大將軍楊忠、王雄並為柱國。甲辰,封少師元羅為韓國公,以紹魏後。丁未,行幸同州故宅,賦詩。冬十月辛酉,突厥遣使朝貢。癸亥,太廟成。乙亥,以功臣瑯邪貞獻公賀拔勝等十三人配享文帝廟庭。壬午,大赦。



 武成元年春正月己酉,太師、晉公護上表歸政。帝始親萬機,軍旅猶總於護。



 初改都督諸州軍事為總管。三月癸巳,陳六軍,帝親擐甲胄,迎太白於東方。吐谷渾寇邊。庚戌,遣大司馬、博陵公賀蘭祥率眾討之。夏五月戊子,
 詔有司造周歷。



 己亥,聽訟於正武殿。辛亥,以大宗伯、梁國公侯莫陳崇為大司徒;大司寇、高陽公達奚武為大宗伯;武陽公豆盧寧為大司寇;柱國、輔城公邕為大司空。乙卯,詔曰:「比屢有糾發官司赦前事者,有司自今勿推究。唯庫既倉廩,與海內所共。漢帝有云:『朕為天下守財耳。』若有侵盜公家財畜錢粟者,魏朝之事,年月既遠,一不須問;自周有天下以來,雖經赦宥,事迹可知者,有司宜即推窮。得實之日,免其罪,征備如法。」賀蘭祥攻拔洮陽、洪和二城,吐谷渾遁走。閏月,高昌遣使朝貢。六月戊子,大雨霖。詔公卿大夫士爰及牧守黎庶等,令各上
 封事,讜言極諫,無有所諱。其遭水者,有司可時巡檢,條列以聞。庚子,詔曰:「潁川從我,是曰元勳;無忘父城,實起王業。文考屬天地草昧,造化權輿,拯彼流亡,匡茲頹運。



 賴英賢盡力,文武同心,翼贊大功,克隆帝業。而被堅執銳,櫛風沐雨,永言疇昔,良用憮然。若功成名遂,建國割符,予唯休也。其有致死王事,妻子無歸者,朕甚傷之。凡從先王向夏州,發夏州從來,見在及薨亡者,並量賜錢帛,稱朕意焉。」



 是月,陳武帝殂。秋八月己亥,改天王稱皇帝,追尊文王為文皇帝。大赦,改元。



 癸丑,增御正四人,位上大夫。冬十月,齊文宣帝殂。



 二年春正月癸丑朔,大會群臣於紫極殿,始用百戲。三月辛酉,重陽閣成,會群臣公侯列將卿大夫及突厥使於芳林園,賜錢帛各有差。夏四月,帝因食糖追遇毒,庚子,大漸。詔曰:人生天地之間,稟五常之氣;天地有窮已,五常有推移,人安得長在。是以有生有死者,物理之必然。處必然之理,脩短之間,何足多恨。朕雖不德,性好典墳,披覽聖賢餘論,未嘗不以此自曉。今乃命也,夫復何言!諸公及在朝卿大夫士、軍中大小督將軍人等,並立勳效,積有年載;輔翼太祖,成我周家,令朕纘承大業,處萬乘之上。此上不負太祖,下不負朕躬。朕得啟手啟足,
 從先帝於地下,實無恨于心矣。所可恨者,朕享大位,可謂四年矣,不能使政化修理,黎庶豐足;九州未一,二方猶梗,顧此恨恨,目用不瞑。唯冀仁兄塚宰,洎朕先正先父公卿大臣等,協和為心,勉力相勸,勿忘太祖遺志,提挈後人。朕雖沒九泉,形骸不朽。今大位虛曠,社稷無主;朕兒幼少,未堪當國。魯國公邕,朕之介弟,寬仁大度,海內共聞,能弘我周家,必此子也。夫人貴有始終,公等事太祖,輔朕躬,可謂有始矣。



 若克念政道,顧其艱難,輔邕以主天下者,可謂有終矣。哀死事生,人臣大節,公等可思念此言,令萬代稱歎。朕稟生儉素,非能力行菲薄。每
 寢大布之被,服大帛之衣,凡是器用,皆無彫刻。身終之日,豈容違棄此好。喪事所須,務從儉約,斂以時服,勿使有金玉之飾。若以禮不可闕,皆令用瓦。小斂訖,七日哭。文武百官,各權辟麻苴,以素服從事。葬日,選擇不毛之地,因勢為墳,勿封勿樹。且厚葬傷生,聖人所誡。既服膺聖人之教,安敢違之。凡百官司,勿異朕意。四方州鎮使到,各令三日哭。哭訖,權辟凶服,還以素服從事,待大例除。非有呼召,各按部自守,不得輒奔赴闕庭。禮有通塞隨時之義,葬訖,內外悉除服從吉。三年之內,勿禁婚娶,一令如平常也。時事殷猥,病困心亂,止能及此。如事有
 不盡,準此以類為斷。



 死而可忍,古人有之,朕今忍死,盡此懷抱。



 其詔即帝口授也。辛丑,帝崩於延壽殿,時年二十七。謚曰明皇帝,廟號世宗。



 五月辛未,葬於昭陵。



 帝寬明仁厚,敦睦九族,有君人之量。幼而好學,博覽群書。善屬文,詞彩溫麗。及即位,集公卿已下有文學者八十餘人,於麟趾殿刊校經史。又捃採眾書,自義、農已來,訖於魏末,敘為《世譜》凡百卷。所著文章十卷。



 論曰:昔者水運將終,群凶放命。或權威震主,或釁逆滔天。咸謂大寶可以力致,神器可以求得,而卒誅夷繼及,亡不旋踵。是知天命有底,庸可慆乎。周文爰自潛躍,眾
 無一旋,驅馳戎馬之際,躡足行伍之間。時屬與能,運膺啟聖,鳩集義勇,糾合同盟。一舉而殄仇讎,再駕而匡帝室。於是內詢帷幄,外杖材雄;推至誠以待人,弘大順以訓物。高氏藉甲兵之眾,恃戎馬之強,屢入近畿,志圖吞噬。及英謀電發,神旆風馳。弘農建城濮之勛,沙苑有昆陽之捷;取威定霸,以弱為強。



 紹元宗之衰緒,創隆周之景命。南清江、漢,西舉巴、蜀,北控沙漠,東據伊、瀍。



 乃擯落魏、晉,憲章古昔;修六官之廢典,成一代之鴻規。德刑並用,勛賢兼敘。



 遠安邇悅,俗阜人和。億兆之望有歸,揖讓之期允集。功業若此,人臣以終,盛矣哉。非求雄略冠時,
 英姿不世;天與神授,緯武經文者,孰能與於此乎。昔漢獻蒙塵,曹公成夾輔之業;晉安播蕩,宋武建匡合之勛。校德論功,綽有餘裕。至於渚宮制勝,闔城孥戮;蠕蠕歸命,盡種誅夷。雖事出於權道,而用乖於德教,斯為過矣。孝閔承既安之業,膺樂推之運;明皇處代邸之尊,纂大宗之緒。始則權臣專命,終乃政出私門;俱懷芒刺之疑,用致幽弒之禍,惜哉。



\end{pinyinscope}