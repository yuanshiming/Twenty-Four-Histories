\article{卷五十一列傳第三十九 齊宗室諸王上}

\begin{pinyinscope}

 趙郡王琛子睿清河王岳子勱廣平公盛陽州公永樂襄樂王顯國上洛王思宗子元海弟思好平秦王歸彥兄子普長樂王靈山神武諸子趙郡王琛,字元寶,齊神武皇帝之弟也。少便弓馬,有志
 氣。封南趙郡公,累遷定州刺史、六州大都督,甚有聲譽。及斛斯椿等釁結,神武帥師入洛陽,以晉陽根本,召琛留總相府政事,天平中,除御史中尉。正色糾彈,無所回避,遠近肅然。



 尋亂神武後庭,因杖而斃。時年二十三。,太尉、尚書令,謚曰貞。天平三年,又贈假黃鉞、左丞相、太師、錄尚書事,進爵為王,配享神武廟廷。子睿嗣。



 睿小名須拔,幼孤,聰慧夙成,特為神武所愛,養於山公主也。令游娘母之,恩異諸子。魏興和中,襲爵南趙郡公。年至四歲,未嘗識母。其母魏華陽山主也。



 其從母姊鄭氏戲謂曰:「汝是我姨兒,何倒親游氏?」睿因訪問,遂失精神。神
 武疑其感疾,睿曰:「兒無患苦,但聞有所生,欲得暫見。」神武驚,命元夫人至,就宮見之。睿前跪拜,因抱頸大哭。神武甚悲傷,謂平秦王曰:「此兒至孝,吾子無及者。」遂為休務一日。睿讀《孝經》,至「資於事父」,輒流涕噓欷。十歲喪母,神武親送至領軍府。為發哀,舉聲殞絕,三日水漿不入口。神武與武明太后殷勤敦譬,方漸順旨。居喪長齋,骨立,杖而後起。神武令常山王與同臥起,日夜喻之,并敕左右,不許進水。雖絕清漱,午輒不肯食,由是神武食必呼與同案。神武崩,哭泣嘔血。及壯,將婚,貌有戚容。文襄謂曰:「我為爾娶鄭述祖女,何嫌而不樂?」對曰:「自痛孤遺,
 方從婚冠,彌用感切。」言未卒,鳴咽不自勝,文襄為之憫然。勱之勤學,常夜久方罷。文宣受禪,進爵為王。睿身長七尺,容儀其偉,閑習吏事,有知人之鑒。天保二年,出為定州刺史、六州大都督。時年十七,稱為良牧。六年,詔睿領兵監築長城。于時六月,睿途中屏蓋扇,親與軍人同勞苦。定州先常藏冰,長史宋欽道以睿冒熱,遣倍道送冰,正遇炎盛,咸謂一時之要。睿對之歎曰:「三軍皆飲溫水,吾何義獨進寒冰!」遂至銷液,竟不一嘗,兵人感悅。



 先是役罷,任其自歸,丁壯先返,羸弱多致僵殞。睿於是親帥營伍,強弱相持,賴全者十三四焉。八年,除都督、北朔
 州刺史。睿撫慰新遷,量置烽戍,備有條法,大為兵人所安。無水處禱而掘井,泉源湧出,至今號曰趙郡王泉。九年,濟南以太子監國,因立大都督府,與尚書省分理眾事,仍開府置佐史。文宣特崇其選,除睿侍中,攝大都督府長史。睿後因侍宴,帝從容謂常山王演等曰:「由來亦有如此長史不?」



 皇建初,兼并州事。孝昭帝臨崩,預受顧託,奉迎武成於鄴,拜尚書令。天統中,追贈父琛假黃鉞;母元氏贈趙郡王妃,謚曰貞昭,華陽長公主如故。有司備禮儀,就墓拜授。時隆冬盛寒,睿跣步號哭,面皆破裂,嘔血數升。及還,不堪參謝。



 帝親就第看問,拜司空、攝錄
 尚書事。



 河清三年,周師及突厥至並州。武成戎服,將以宮人避之,睿叩馬諫,乃止。



 帝親御戎,六軍進止,並令取睿節度,而使段孝先總焉。帝與宮為被緋甲,登故北城以望,軍營甚整。突厥咎周人曰:「爾言齊亂,故來伐之。今齊人眼中亦有鐵,何可當邪!」乃還。至陘嶺,凍滑,乃鋪氈以度。胡馬寒瘦,膝已下皆無毛,比至長城,死且盡。乃截槊杖之以歸。是役也,段孝先持重,不與賊戰,自晉陽失道,為虜所屠,無遺類焉。斛律光自三堆還,帝以遭大寇,抱其頭哭。任城王湝進曰:「何至此!」乃止。光面折孝先於帝前,曰:「段婆善為送女客。」於是以睿為能,加尚書令,封
 宣城郡公,拜太尉,監五禮。晚節頗以酒色為和士開所構。睿久典朝政,譽望日隆,漸被疏忌。乃撰古忠臣義士,號曰《要言》,以致其意。武成崩。



 葬後數日,睿與馮翊王潤、安德王延宗及元文遙奏後主云:「和士開不宜仍居內。」



 並入奏太后。因出士開為袞州刺史。太后欲留過百日,睿正色不許。太后令酌酒賜睿,睿正色曰:「今論國家大事,非為厄酒。」言訖便出。其夜,睿方寢,見一人長可丈五尺,臂丈餘,當門向床,以臂壓睿,良久遂失。甚惡之,起坐嘆曰:「大丈夫運命一朝至此!」旦欲入朝,妻子咸諫止之。睿曰:「社稷事重,吾當以死效之。吾寧死事先皇,不忍見
 朝廷顛沛。」至殿門,又有人曰:「願勿入。」睿曰:「吾上不負天,死亦無恨。」入見太后,太后復以為言,睿執之彌固。出至永巷,被送華林園,於雀離佛院令劉桃枝拉殺之,時年三十六。大霧三日,朝野冤惜之。



 其年,詔聽以王禮葬,竟無贈謚。子整信嗣,好學有行檢,位儀同三司,後終於長安。



 清河王岳,學洪略,神武從父弟也。父翻,字飛雀,以器度知名,卒於侍御中散。元象中,贈假黃鉞、大將軍、太傅、太尉、錄尚書事,謚孝宣公。岳幼孤貧,人未之知。長而敦直,姿貌嶷然,深沉有器量。初居洛邑,神武每使入洛,必止
 岳舍。岳母山氏嘗夜起,見神武室中無火而有光。移於別室,如前所見。怪之。詣卜者筮,遇《乾》之《大有》。占者曰:「吉,《易》稱『飛龍在天,大人造也』,貴不可言。」山氏歸報神武。神武後起兵於信都,山氏謂岳曰:「赤光之瑞,今當驗矣,汝可從之。」岳遂往信都,神武見之大悅。



 及戰於韓陵,神武將中軍,高昂將左軍,岳將右軍。中軍敗,岳舉麾大呼,橫衝賊陣,神武因大破賊。以功除衛將軍、左光祿大夫,封清河郡公。母山氏封郡君,授女侍中,入侍皇后。天平二年,除侍中、六州軍事都督,尋加開府。岳辟引時賢以為僚屬,論者美之。尋授使持節、六州大都督、冀州大中正。
 俄拜京畿大都督,其六州事悉隸京畿。時神武統務晉陽,岳與侍中孫騰等京師輔政。岳性至孝,母疾,衣不解帶。及遭喪去職,哀毀骨立。神武憂之,每日遣人勞勉。尋起復本位,歷冀、晉二州刺史、西南道大都督,有綏邊之稱。



 及神武崩,侯景叛,梁武乘間遣其貞陽侯明於寒山,擁泗水灌彭城,與景為掎角聲援。岳總諸軍南討,與行臺慕容紹宗擊破明,禽之。景仍於渦陽與左衛將軍劉豐等相持。岳又破之。以功除太尉。又統慕容紹宗、劉豐等攻王思政於長社。岳引洧水灌城。紹宗、劉豐為思政所獲。西魏出兵援思政,岳內外防禦,城不沒者三板。



 會
 文襄親臨,數日剋城,獲思政等。以功別封真定縣男。文襄以為己功,故賞典不弘。



 文襄崩,文宣出撫晉陽,令岳以本官兼尚書左僕射,留鎮鄴。天保初,進封清河郡王。五年,加太保。為西南道大行臺,統司徒潘相樂等救江陵。師次義陽,西魏克荊州。因略地,克郢州,獲梁郢州刺史陸法和,送鄴。詔岳旋師。岳自討寒山、長社,及出隨、陸,並有功,威名彌重。性華侈,尤悅酒色,歌姬舞女,陳鼎擊鐘,諸王皆莫及。初,高歸彥少孤,神武令岳撫養。輕其年幼,情禮甚薄,歸彥內銜之。



 及歸彥為領軍,岳謂其德己,更倚仗之。歸彥密構其短,奏岳造城南大宅,僭擬為
 永巷,但無闕耳。帝後夜行,見壯麗,意不平。仍屬帝召鄴下婦人薛氏入宮,而岳先嘗迎之,至宅,由其姊也。帝縣薛氏姊而鋸殺之,讓岳,以為姦人女。岳曰:「臣本欲取之,嫌其輕薄,非姦也。」帝益怒,使高歸彥就宅賜以鴆。岳曰:「臣無罪。」彥曰:「飲之!」飲而薨。朝野惜之,時年三十四。詔大鴻臚護喪事。贈太宰、太傅、假黃鉞、給轀輬車,謚曰昭武。敕以城南宅為莊嚴寺。



 初,岳與神武經綸天下,家有私兵戎器,儲甲千餘領。文襄末,岳表求納之,文襄推心相任,不許。文宣時,亦頻請納,又不許。將薨,遺表謝恩,并請上甲。



 葬畢,方許納焉。皇建中,配享文襄廟庭。後歸彥反,
 武成知其前譖,以歸彥良賤百口贈岳家。贈岳太師、太保,餘如故。子勱。



 勱字敬德,幼聰敏,美風儀,以仁孝聞。七歲襲爵清河王,十四為青州刺史。



 歷祠部尚書、開府儀同三司,改封安樂侯。性剛直,有才幹。斛律光雅敬之,每征伐則引為副。遷侍中、尚書右僕射。



 及後主為周師所敗,勱奉太后歸鄴。進宦官放縱,儀同茍子溢尤幸。勱將斬以徇,太后救之,乃得釋。劉文殊竊謂勱曰:「子溢之徒,言成禍福,何得如此!」



 勱攘袂曰:「今西軍日侵,朝貴多叛,正由此輩弄權。若今日殺之,明日就誅,無恨。」文殊甚愧之。勱勸後主,五
 品已下家略,悉置三臺上,脅之曰:「若戰不捷,則燒之。此輩必死戰,乃可捷也。」後主不從,遂棄鄴東遷。勱恒後殿,為周軍所得。武帝與語,大悅,因問齊亡所由,勱發言流涕,悲不自勝,帝為改容。授開府儀同三司。



 隋文帝為丞相,謂曰:「齊亡由任邪佞,公父子忠良,聞於鄰境,宜善自愛。」



 勱拜謝曰:「勱,亡齊末屬,不能扶危定傾,既蒙獲宥,已多優幸,況濫叨名級,致速官謗。」帝甚器之。再遷楚州刺史。城北有伍子胥廟,其俗敬鬼,祈者必以牛酒,至破產業。勱歎曰:「子胥賢者,豈宜損百姓乎!」告諭所部,自是遂止。百姓賴之。



 開皇七年,轉光州刺史。上表曰:「陳氏數年
 已來,荒悖滋甚,天厭亂德,妖實人興。或空裏時有大聲,或行路共傳鬼怪,或刳人肝以祠天狗,或自捨身以厭妖訛。人神怨憤,怪異薦發。臣以庸才,猥蒙朝寄,頻歷蕃守,與其鄰接。密邇仇仇,知其動靜。天討有罪,此即其時。若戎車雷動,戈船電邁,臣雖駑怯,請效鷹犬。」



 並上平陳五策,帝嘉之,答以優詔。及大舉伐陳,以勱為行軍總管,從宜陽公王世積下陳江州,以功拜上開府,賜物三千段。時隴右諸羌,數為寇亂。朝廷以勱有威名,拜洮州刺史。下車大崇威惠,人夷悅附,豪猾屏迹,路不拾遺,以善政稱。後吐谷渾來寇,勱時遇疾,不能拒戰,賊遂大掠而
 去。憲司奏勱亡戶口,坐免,卒于家。大唐褒顯前代名臣,追贈都督四州諸軍事、定州刺史。子士廉最知名。



 廣平公盛,神武從叔祖也。寬厚有長者風。神武起兵於信都,盛來赴,以為中軍大都督,封廣平郡公。歷位司徒、太尉。天平三年,薨於位,贈假黃鉞、太尉、太師、錄尚書事。無子,以兄子子瑗嗣。天保初,改封平昌王,卒於魏尹。



 陽州公永樂,神武從祖兄子也。太昌初,封陽州縣伯,進爵為公,累遷北豫州刺史。河橋之戰,司徒高昂失利奔退,永樂守洛陽南城。昂走趣城南,西軍追者將至,永樂不開門,昂遂為西軍所禽。神武大怒,杖之二百。後罷豫
 州,家產不立。



 神武問其故,對曰:「裴監為長史,辛公正為別駕,受王委寄,斗酒隻雞不敢入。」



 神武乃以永樂為濟州,仍以監、公正為長史、別駕。謂永樂曰:「爾勿大貪,小小義取莫復畏。」永樂至州,監、公正諫不見聽,以狀啟神武。神武封啟以示永樂,然後知二人清直,並擢用之。永樂卒於州,贈太師、太尉、錄尚書事,謚曰武昭。



 無子,從兄思宗以第二子孝緒為後,襲爵。天保初,改封修城郡王。



 永樂弟長弼,小名阿伽。性粗武,出入城市,好毆擊行路,時人皆呼為阿伽郎君。以宗室封廣武王。時有天恩道人,至凶暴,橫行閭肆,後入長弼黨,專以斗為事。文宣並收
 掩付獄,天恩等十餘人皆棄市,長弼鞭一百。尋為南營州刺史,在州無故自驚走。叛亡入突厥,竟不知死所。



 襄樂王顯國,神武從祖弟也。無才伎,直以宗室謹厚,天保元年,封襄樂郡王。



 位右衛將軍,卒。



 上洛王思宗,神武從子也。性寬和,頗有武幹。天保初,封上洛郡王。歷位司空、太傅,薨於官。



 子元海,累遷散騎常侍,願處山林,修行釋典,文宣許之。乃入林慮山,經二年,絕棄人事。志不能固,自啟求歸。徵復本任,便縱酒肆情,廣納姬侍。又除領軍將軍。器小志大,頗以智謀自許。皇建末,孝昭幸晉陽,武成居守,元海
 以散騎常侍留典機密。初,孝昭之誅楊愔等,謂武成云,事成,以汝為皇太弟。及踐位,乃使武成在鄴主兵,立子百年為皇太子,武成甚不平。



 先是,恒留濟南於鄴,除領軍厙狄伏連為幽州刺史,以斛律豐樂為領軍,以分武成之權。武成留伏連而不聽豐樂視事。乃與河陽王孝瑜偽獵,謀於野,暗乃歸。



 先是童謠云:「中興寺內白鳧翁,四方側聽聲雍雍,道人聞之夜打鐘。」時丞相府在北城中,即舊中興寺也;鳧翁謂雄雞,蓋指武成小字步落稽也;道人,濟南王小名也;打鐘,言將被擊也。既而太史奏言,北城有天子氣,昭帝以為濟南應之,乃使平秦王歸
 彥之鄴,迎濟南赴并州。武成先告元海,並問自安之計。元海曰:「皇太后萬福,至尊孝性非常,殿下不須別慮。」武成曰:「此豈我推誠之意邪?」元海乞還省一夜思之。武成即留元海後堂,元海達旦不眠,唯繞床徐步。夜漏未盡,武成遽出曰:「神算如何?」答云:「夜中得三策,恐不堪用耳。」因說梁孝王懼誅入關事,請乘數騎入晉陽,先見太后求哀,後見主上,請去兵權,以死為限,求不干朝政,必保太山之安,此上策也;若不然,當具表云威權大盛,恐取謗眾口,請青、齊二州刺史,沉靖自居,必不招物議,此中策也。更問下策,曰:「發言即恐族誅。」因逼之,答曰:「濟南世
 嫡,主上假太后令而奪之,今集文武,示以此敕,執豐樂,斬歸彥,尊濟南,號令天下,以順討逆,此萬世一時也。」武成大悅,狐疑,竟未能用。乃使鄭道謙卜之,皆曰:「不利舉事,靜則吉。」又召曹魏祖問之國事,對曰:「當有大凶。」又時有林慮令姓籓,知占候,密謂武成曰:「宮車當晏駕,殿下為天下王。」武成拘之於內以候之。又令巫覡卜之,多云不須舉兵,自有大慶。武成乃奉詔,令數百騎送濟南於晉陽。及孝昭崩,武成即位,除元海侍中、開府儀同三司、太子詹事。河清二年,元海為和士開譖,被馬鞭六十,責云:「爾在鄴城說我以弟反兄,幾許不義!以鄴城兵馬抗
 并州,幾許無智!不義無智,若為可使?」出為兗州刺史。



 元海後妻,陸太姬甥也,故尋被追任使。武平中,與祖珽共執朝政。元海多以太姬密語告珽。珽求領軍,元海不可,珽乃以其所告報太姬。姬怒,出元海為鄭州刺史。鄴城將敗,徵為尚書令。周建德七年,於鄴城謀逆,伏誅。



 元海好亂樂禍,然詐仁慈,不飲酒啖肉。文宣天保末年,敬信內法,乃至宗廟不血食,皆元海所為。及為右僕射,又說後主禁屠宰,斷酤酒。然本心非靖,故終致覆敗。



 思宗弟思好,本浩氏子也,思宗養以為弟,遇之甚薄。少以騎射事文襄。及文宣受命,為左衛大將軍。本名思孝,
 天保五年討蠕蠕,文宣悅其驍勇,謂曰:「爾擊賊如鶻入鴉群,宜思好事。」故改名焉。累遷尚書令、朔州道行臺、朔州刺史、開府、南安王。甚得邊朔人心。



 後主時,斫骨光弁奉使至州,思好迎之甚謹。光弁倨傲,思好因心銜恨。武平五年,遂舉兵反,與并州諸貴書曰:「主上少長深宮,未辨人之情偽,暱近凶狡,疏遠忠良。遂使刀鋸刑餘,貴溢軒階;商胡醜類,擅權帷幄。剝削生靈,劫掠朝市,暗於聽受,專行忍害。幽母深宮,無復人子之禮;二弟殘戮,頓絕孔懷之義。仍縱子立奪馬於東門,光弁制鷹於西市;駮龍得儀同之號,逍遙受郡君之名。犬馬班位,榮冠軒冕,
 人不堪役,思長亂階。趙郡王睿,實曰宗英,社稷惟寄。左相斛律明月,世為元輔,威著鄰國,並非有辜,奄見誅殄。孤既忝預皇枝,實蒙殊獎,今便擁率義兵,指除君側之害。幸悉此懷,無致疑惑。」行臺郎王行思之辭也。



 思好至陽曲,自號大丞相,置百官,以行臺左丞王尚之為長史。武衛趙海在晉陽掌兵,時倉卒,不暇奏,矯詔發兵拒之。軍士皆曰:「南安王來,我輩唯須唱萬歲奉迎耳。」帝聞變,使唐邕、莫多婁敬顯、劉桃枝、中領軍厙狄士文馳之晉陽,帝勒兵續進。思好軍敗,與行思投水而死。其麾下二千人,桃枝圍之,且殺且招,終不降,以至於盡。時帝在
 道,叱奴世安自晉陽遂露布,於城平都遇斛斯孝卿,孝卿誘使食,因馳詣行宮,叫已了。帝大懽,左右呼萬歲。良久,世安乃以狀自陳。



 帝曰:「告爾何物事?乃得坐食!」於是賞孝卿而免世安罪。暴思好屍七日,然後屠剝焚之,烹尚之於鄴市,令內參射其妃於宮內,仍火焚殺之。



 思好反前五旬,有人告其謀反。韓長鸞女適思好子,故奏言有人誣告諸貴,事相擾動,不殺無以息後,乃斬之。思好既誅,死者弟伏闕下訴求贈兄,長鸞不為通也。



 平秦王歸彥,字仁英,神武族弟也。父徽,魏末坐事當徙涼州。行至河、渭間,遇賊,以軍功得免流。因於河州積年,
 以解胡言為西域大使,得胡師子,以功行河東事,遂死焉。徽於神武,舊恩甚篤。及神武平京洛,迎徽喪,與穆同營葬。贈司徒,謚曰文宣。



 初,徽嘗過長安市,與婦人王氏私通而生歸彥,至是年已九歲,神武追見之,撫對悲喜。稍遷徐州刺史。歸彥少質朴,後更改節,放縱好聲色,朝夕酣歌。妻魏上黨王元天穆女也,貌不美而甚嬌妒。數忿爭,密啟文宣求離,事寢不報。天保元年,封平秦王,嫡妃康及所生母王氏,並為太妃。善事二母,以孝聞。徵為兼侍郎,稍被親寵。以討侯景功,別封長樂郡公,除領軍大將軍。領軍加大,自歸彥始也。



 文宣誅高德正,金寶財
 貨,悉以賜之。乾明初,拜司徒,仍總知禁衛。



 濟南自晉陽之鄴,楊愔宣敕,留從駕兵五千於西中,陰備非常。至鄴數日,歸彥乃知之,由是陰怨楊、燕等。楊、燕等欲去二王,問計於歸彥。歸彥詐喜,請共元海量之。元海亦口許心違,馳告長廣。長廣於是誅楊、燕等。孝昭將入雲龍門,都督成休寧列仗拒而不內,歸彥諭之,然後得入。進向柏閣、永巷亦知之。孝昭踐阼,以此彌見優重。每入,常在平原王段韶上。以為司空,兼尚書令。齊制,宮內唯天子紗帽,臣下皆戎帽。特賜歸彥紗帽以寵之。孝昭崩,歸彥從晉陽迎武成於鄴。



 及武成即位,進位太傅,領司徒,常聽
 將私部曲三人,帶刀入仗。從武成還都,諸貴戚等競要之。其所往處,一坐盡傾。歸彥既地居將相,志氣盈滿,發言陵侮,傍若無人。議者以威權震主,必為禍亂。上亦尋其前翻覆之迹,漸忌之。高元海、畢義雲、高乾和等咸數言其短。上幸歸彥家,召魏收對御作詔草,欲加右丞相。收曰:「至尊以右丞相登帝位,今為歸彥威名太盛,故出之,豈可復加此號?」乃拜太宰、冀州刺史。即乾和繕寫。晝日,仍敕門司不聽輒內。時歸彥在家縱酒,經宿不知,至明欲參。至門知之,大驚而退。及通名謝,敕令早發,別賜錢帛、鼓吹、醫藥,事事周備。又敕武職督將,悉送至清陽
 宮。拜而退,莫敢共語。唯與趙郡王睿久語,時無聞者。



 至州不自安,謀逆,欲待受調訖,班賜軍士。望車駕如晉陽,乘虛入鄴。為其郎中令呂思禮所告,詔平原王段韶襲之。歸彥舊於南境置私驛,聞軍將逼,報之,便嬰城拒守。先是,冀州長史宇文仲鸞、司馬李祖挹、別駕陳季璩、中從事房子弼、長樂郡守尉普興等疑歸彥有異,使連名密啟,歸彥追而獲之,遂收禁仲鸞等五人。



 仍並不從,皆殺之。軍已逼城,歸彥登城大叫云:「孝昭皇帝初崩,六軍百萬眾,悉由臣手,投身向鄴迎陛下。當時不反,今日豈有異心?正恨高元海、畢義雲、高乾和誑惑聖上,疾忌忠
 良。但為殺此三人,即臨城自刎。」其後城破,單騎北走。



 至交津,見獲,鎖送鄴。帝令趙郡王睿私問其故,歸彥曰:「使黃頷少兒牽挽我,何可不反?」曰:「誰邪?」歸彥曰:「元海、乾和,豈是朝廷老宿?如趙家老公時,又詎懷怨?」於是帝又使讓焉。對曰:「高元海受畢義雲宅,用作本州刺史,給後部鼓吹,臣為蕃王、太宰,仍不得鼓吹。正殺元海、義雲而已。」上令都督劉桃枝牽入,歸彥猶作前語,望活。帝命議其罪,皆云不可赦。乃載以露車,銜枚面縛,劉桃枝臨之以刃,擊鼓隨之,并子孫十五人,皆棄市。贈仁州刺史。



 魏時山崩,得石角二,藏在武庫。文宣入庫,賜從臣兵器,特以
 二石角與歸彥,謂曰:「爾事常山不得反,事長廣得反,反時,將此角嚇漢。」歸彥額骨三道,著幘不安。文宣見之怒,使以馬鞭擊其額,血被面曰:「爾反時,當以此骨嚇漢。」



 其言反,竟驗云。



 武興王普,字德廣,歸彥兄歸義之子也。性寬和,有度量。九歲歸彥自河州俱入洛,神武使與諸子同游處。天保初,封武興郡王。武平二年,累遷司空。六年,為豫州道行臺尚書令。後主奔鄴,就加太宰。周師逼,乃降。卒於長安,贈上開府、豫州刺史。



 長樂太守靈山,字景嵩,神族族弟也。從神武起兵信都,
 終長樂太守,贈大將軍、司空,謚曰文宣。子懿,卒於武平鎮將。無子,文宣以靈山從父兄齊州刺史建國子伏護為靈山後。



 伏護字臣援,粗有刀筆。天統初,累遷黃門侍郎。伏護歷事數朝,恒參機要,而性嗜酒,每多醉失。末路逾劇,乃至連日不食,專事酣酒,神識恍惚,遂以卒。



 贈袞州刺史。建國侯。孫乂襲。



 乂少謹,武平末,給事黃門侍郎。隋開皇中為太府少卿,坐事死。



 神武皇帝十五男:武明婁皇后生文襄皇帝、文宣皇帝、孝昭皇帝、襄城景王清、武成皇帝、博陵文簡王濟;王氏生永安簡平王浚;穆氏生平陽靖翼王淹;大爾朱氏生
 彭城景思王浟、華山王凝;韓氏生上黨剛肅王渙;小爾朱氏生任城王湝;游氏生高陽康穆王水是;鄭氏生馮翊王潤;馬氏生漢陽敬懷王洽。



 永安簡平王浚字定樂,神武第三子也。初,神武納浚母,當月而有孕。及產浚,疑非己類,不甚愛之。而浚早慧,後更被寵。年八歲,謂博士盧裕曰:「祭神如神在,為有神邪?無神邪?」對曰:「有。」浚曰:「有神,當云祭神神在,何煩如字?」景裕不能答。及長,嬉戲不節。曾以屬請受納,大見杖罰,拘禁府獄,既而見原。後稍折節,頗以讀書為務。元象中,封永安郡公。豪爽有氣力,善騎射,為文襄所愛。文宣性雌
 懦,每參文襄,有時洟出。浚恒責帝左右:「何因不為二兄拭鼻?」由是見銜。累遷中書監、兼侍中。出為青州刺史。雖頗好畋獵,聰明矜恕,上下畏悅之。保定初,進爵為王。



 文宣末年多酒,浚謂親近曰:「二兄舊來,不甚了了,自登阼已後,識解頓進。



 今因酒敗德,朝臣無敢諫者。大敵未滅,吾甚以為憂。欲乘驛至鄴面諫,不知用吾不?」人有知,密以白帝,又見銜。八年,來朝,從幸東山。帝裸裎為樂,雜以婦女,又作狐掉尾戲。浚進言,此非人主所宜。帝甚不悅。浚又於屏處召楊遵彥,譏其不諫。帝時不欲大臣與諸王交通,遵彥懼,以奏帝。大怒曰:「小人由來難忍!」



 遂罷酒
 還宮。浚尋還州,又上書切諫。詔令徵浚,浚懼禍,謝疾不朝。上怒,馳驛收浚,老幼泣送者數千人。至,盛以鐵籠,與上黨王渙俱置北城地牢下,飲食溲穢,共在一所。



 明年,帝親將左右,臨穴歌謳,令浚等和之。浚等惶怖且悲,不覺聲戰。帝為愴然,因泣,將赦之。長廣王湛先與浚不睦,進曰:「猛獸安可出穴?」帝默然。



 浚等聞之,呼長廣王小字曰:「步落稽,皇天見汝!」左右聞者,莫不悲傷。浚與渙皆有雄略,為諸王所傾服。帝恐為害,乃自刺渙,又使壯士劉桃枝就籠亂刺。槊每下,浚、渙輒以手拉折之,號哭呼天,於是薪火亂投籠,燒殺之,填以石土。後出,皮髮皆盡,屍
 色如炭,天下為之痛心。



 後帝以其妃陸氏配儀同劉郁捷,舊帝蒼頭也,以軍功見寵。時令郁捷害浚,故以配焉。後數日,帝以陸氏先無寵於浚,敕與離絕。乾明元年,贈太尉。無子,詔以彭城王浟第二子準字茂則嗣。



 平陽靖翼王淹,字子邃,神武第四子也。元象中,封平陽郡公,累遷尚書左僕射。天保初,進爵為王,歷位尚書、開府儀同三司、司空、太尉。皇建初,為太傅,與彭城、河間王並給仗身羽林百人。大寧元年,遷太宰。性沉謹,以寬厚稱。河清三年,薨於晉陽,或云以鴆終。還葬鄴,贈假黃鉞、太宰、錄尚書事。子德素嗣。



 彭城景思王浟,字子深,神武第五子也。元象二年,拜通直散騎常侍,封長樂郡公。博士韓毅教浟書,見浟筆迹未工,戲浟曰:「五郎書畫如此,忽為常侍開國,今日後,宜更用心!」浟正色答曰:「昔甘羅為秦相,未聞能書。凡人唯論才具何如,豈必勤勤筆迹。博士當今能者,何為不作三公?」時年蓋八歲矣。毅甚慚。



 武定六年,出為滄州刺史。為政嚴察,部內肅然。守令參佐,下及胥吏,行游往來,皆自齎糧食。浟纖介知人間事,有隰沃縣主簿張達,嘗詣州,夜投人舍,食雞羹,浟察知之。守令畢集,浟對眾曰:「食雞羹何不還他價直也?」達即伏罪,合境號為神明。又有
 一人從幽州來,驢馱鹿脯。至滄州界,腳痛行遲,偶會一人為伴,遂盜驢及脯去。明旦告州,浟乃令左右及府僚吏分市鹿膊,不限其價。其主見脯識之,推獲盜者。轉都督、定州刺史。時有人被盜黑牛,背上有白毛。長史韋道建謂中從事魏道勝曰:「使君在滄州日,禽姦如神。若捉得此賊,定神矣。」浟乃詐為上符,市牛皮,倍酬價直。使牛主認之,因獲其盜。建等歎服。又有老母姓王,孤獨,種菜三畝,數被偷。浟乃令人密往書菜葉為字,明日,市中看菜葉有字,獲賊。爾後境內無盜,政化為當時第一。



 天保初,封彭城王。四年,徵為侍中,人吏送別悲號。有老公數
 百人,相率具饌白浟曰:「自殿下至來五載,人不識吏,吏不欺人。百姓有識已來,始逢今化。



 殿下唯飲此鄉水,未食百姓食,聊獻疏薄。」浟重其意,為食一口。七年,轉司州牧,選從事皆取文才士明剖斷者,當時稱為美選。州舊案五百餘,水攸未期悉斷盡。



 別駕羊脩等恐犯權戚,乃詣閣諮陳。浟使告曰:「吾直道而行,何憚權戚?卿等當成人之美,反以權戚為言!」脩等慚悚而退。後加特進,兼司空、太尉,州牧如故。



 太妃薨,解任。尋詔復本官。俄拜司空,兼尚書令。濟南嗣位,除開府儀同三司、尚書令,領大宗正卿。皇建初,拜大司馬,兼尚書令,轉太保。武成入承大業。
 遷太師、錄尚書。



 浟明練世務,果於斷決,事無大小,咸悉以情。趙郡李公統預高歸彥之逆,其母崔氏,即御史中丞崔昂從父姊,兼右僕射魏收之內妹也。依令,年出六十,例免入官。崔增年陳訴,所司以昂、收故,崔遂獲免。浟摘發其事,昂等以罪除名。自後車駕巡幸,浟常留鄴。河清三年三月,群盜白子禮等數十人,謀劫浟為主。詐稱使者,徑向浟第。至內室,稱敕呼浟,牽上馬,臨以白刃,欲引向南殿。浟大呼不從,遂遇害,時年三十二。朝野痛惜焉。初浟未被劫前,其妃鄭氏夢人斬浟頭持去,惡之。數日而浟見殺。贈假黃鉞、太師、太尉、錄尚書事,給轀輬車。
 子寶德嗣。



 位開府,兼尚書左僕射。



 上黨剛肅王渙,字敬壽,神武第七子也。天姿雄傑,俶儻不群。雖在童幼,恒以將略自許。神武壯而愛之,曰:「此兒似我。」及長,力能扛鼎,材武絕倫。每謂左右曰:「人不可無學,但要不為博士耳。」故讀書頗知梗概,而不甚耽習。



 元象中,封平原郡公。文襄之遇賊,渙年尚幼,在西學。聞宮中言雚,驚曰:「大兄必遭難矣!」彎弓而出。武定末,除冀州刺史,在州有美政。天保初,封上黨王,歷中書令、尚書左僕射。與常山王演等築伐惡諸城。遂聚鄴下輕薄,陵犯郡縣,為法司所糾。文宣戮其左右數人,渙亦被譴。六年,率
 眾送梁王蕭明還江南,仍破東關,斬梁特進裴之橫等,威名甚盛。八年,錄尚書事。初,術士言亡高者黑衣,由是自神武後每出行不欲見桑門,為黑衣故也。是時文宣幸晉陽,以所忌問左右曰:「何物最黑?」對曰:「莫過漆。」帝以渙第七,為當之,乃使庫真都督破六韓伯昇之鄴徵渙。渙至紫陌橋,殺伯昇以逃,憑河而度,土人執以送帝。鐵籠盛之,與永安王浚同置地牢下。歲餘,與浚同見殺,時年二十六。以其妃李氏配馮文洛,是帝家舊奴,積勞位至刺史。帝令文洛等殺渙,故以其妻妻焉。至乾明元年,收二王餘骨葬之,贈司空,謚曰剛肅。有敕李氏還第,而
 文洛尚以故意,脩飾詣李。



 李盛列左右,引文洛立於階下,數之曰「遭難流離,以至大辱,志操寡薄,不能自盡。幸蒙恩詔,得反籓闈。汝是誰家孰奴?猶欲見侮!」於是杖之一百,流血灑地。



 渙無嫡子,庶長子寶嚴,以河清二年襲爵。位終金紫光祿大夫、開府儀同三司。



 襄城景王淯,神武第八子也。容貌甚美,弱年有器望。元象中,封章武郡公。



 天保初,封襄城郡王。二年春,薨。齊氏諸王選國臣府佐,多取富商群小,鷹犬少年。唯襄城、廣寧、蘭陵王等,頗引文藝清識之士,當時以此稱之。乾明元年二月,贈假黃鉞、太師、太尉、錄尚書事。無子,詔以常
 山王演第二子亮嗣。



 亮字彥道,性恭孝,美風儀,好文學。為徐州刺史,坐奪商人財物,免官。後主敗,奔鄴,亮從焉。遷兼太尉、太傅。周師入鄴,亮於啟夏門拒守,諸軍皆不戰而敗,周軍於諸城門皆入,亮軍方退走。亮入太廟行馬內,慟哭拜辭,然後為周軍所執。入關,依例授儀同,分配遠邊,卒於龍州。



 任城王湝,神武第十子也。少明慧,天保初封。自孝昭、武成時,車駕還鄴,嘗令湝鎮晉陽,總並省事。歷司徒、太尉、並省錄尚書。天統三年,拜太保,並州刺史,別封正平郡公。



 時有婦人臨汾水浣衣,有乘馬人換其新靴馳而
 去者。婦人持故靴詣州言之。湝召居城諸嫗,以靴示之,紿曰:「有乘馬人於路被賊劫害,遺此靴,焉得無親屬乎?」



 一嫗撫膺哭曰:「兒昨著此靴向妻家。」如其語,捕獲之,時稱明察。武平初,遷太師、司州牧。出為冀州刺史,加太宰,遷右丞相、都督、青州刺史。湝頻牧大蕃,雖不潔己,然寬恕,為吏人所懷。五年,青州人崔蔚波等夜襲州城。湝部分倉卒之際,咸得齊整,擊賊大破之。拜左丞相,轉瀛州刺史。及後主奔鄴,加湝大丞相。



 及安德王稱尊號於晉陽,使劉子昂脩啟於湝:「至尊出奔,宗廟既重,群公勸迫,權主號令。事寧終歸叔父。」湝曰:「我人臣,何容受此啟。」執
 子昂送鄴。帝至濟州,禪位於湝,竟不達。



 湝與廣寧王孝珩於冀州召募,得四萬餘人,拒周軍。周齊王憲來伐,先遣送書,並赦詔。湝並沉諸井。戰敗,湝、孝珩俱被禽。憲曰:「任城王,何苦至此!」湝曰:「下官神武帝子,兄弟十五人,幸而獨存。逢宗社顛覆,今日得死,無愧墳陵。」



 憲壯之,歸其妻子。將至鄴城,湝馬上大哭。自投于地,流血滿面。至長安,尋與後主同死。



 妃盧氏,賜斛斯徵。盧蓬首垢面,長齋不言笑,徵放之,乃為尼。隋開皇三年,表請文帝,葬湝及五子於長安北原。



 高陽康穆王湜,神武第十一子也。天保元年封。十年,稍
 遷尚書令。以滑稽便辟,有寵於文宣。在左右行杖,以撻諸王,太后深銜之。其妃父護軍長史張晏之,嘗要道拜湜,湜不禮焉。帝問其故,對曰:「無官職漢,何須禮!」帝於是擢拜晏之為徐州刺史。文宣崩,湜兼司徒,導引梓官。吹笛云:「至尊頗知臣不?」又擊胡鼓為樂。太后杖湜百餘,未幾薨。太后哭之哀,曰:「我恐其不成就,與杖,何期帶創死也!」乾明初,贈假黃鉞、太師、司徒、錄尚書事。子士義襲爵。



 博陵文簡王濟,神武第十二子也。天保元年封。濟嘗從文宣巡幸,在路忽憶太后,遂逃歸。帝怒,臨以白刃,因此驚怳。歷位太尉。河清初,出為定州刺史。天統五年,在州
 語人云:「計次第,亦應到我。」後主聞之,陰使人殺之。贈假黃鉞、太尉、錄尚書事。子智襲爵。



 華山王凝,神武第十三子也。天保元年,封新平郡王。九年,改封安定。十五年,封華山。歷位中書令、齊州刺史,就加太傅。薨於州,贈左丞相、太師、錄尚書。凝諸王中最為孱弱,妃王氏,太子洗馬王洽女也,與蒼頭姦,凝知而不能限禁。



 後事發,王氏賜死,詔杖凝一百,其愚如此。



 馮翊王潤,字子澤,神武第十四子也。幼時,神武稱曰:「此吾家千里駒也。」



 天保初封,歷位東北道行臺右僕射、都督、定州刺史。潤美姿儀。年十四五,母鄭妃與之同寢,有
 穢雜之聲。及長,廉慎方雅,習於吏職。至於摘發隱偽,姦吏無所匿其情。開府王回洛,與六州大都督獨孤枝侵竊官田,受納賄賂,潤按舉其事。二人表言:王出送臺使,登魏孝文舊壇,南望歎息,不測其意。武成使元文遙就州宣敕曰:「馮翊王少小謹慎,在州不為非法,朕信之熟矣。登高遠望,人之常情,鼠輩欲輕相間構,曲生眉目。」於是回洛決鞭二百,獨孤枝決杖一百。尋為尚書令,領太子少師,歷司徒、太尉、大司馬、司州牧、太保、河南道行臺、錄尚書,別封文成郡公,太師、太宰,復為定州刺史。薨,贈假黃鉞、左丞相。子茂德嗣。



 漢陽敬懷王洽,字敬延,神武第十五子也。天保元年封,五年薨,年十三。乾明元年,贈太保、司空。無子,以任城王第二子建德為後。



 論曰:趙郡王以跗萼之親,當顧命之重,安夫一德,固此貞必,踐畏途而不疑,履危機而莫懼,以其忠義,取斃凶匿。豈道光四海,不遇周成之明;將朝去三仁,終見殷墟之禍。不然,則邦國殄瘁,何若斯之速歟?清河屬經綸之期,青雲自致,出將入相,翊成鴻業。雖漢朝劉賈,魏室曹洪,俱未足諭其風烈,適足以彰文宣之失德焉。思好屬昏亂之機,歸彥因猜嫌之釁,咫尺鄴都,以速其禍,智小
 謀大,理則宜然。神武諸王,多有聲譽。永安以諫爭遇禍,固齊室之比干,彭城蒞人布政,乃與循良比跡,求之近古,未為易遇。上黨申威淮海,受辱牢阱,以英俠之氣,迫悲歌之思,欲食藜藿之羹,處茅茨之下,其可得乎!馮翊廉慎閑明,妄被讒匿,以武成陰忌之朝,而見免夫《角弓》之刺,已為幸矣。



\end{pinyinscope}