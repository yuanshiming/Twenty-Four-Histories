\article{卷九十列傳第七十八 藝術下}

\begin{pinyinscope}

 周澹李脩徐謇從孫之才王顯馬嗣明姚僧垣褚該許智藏萬寶常蔣少游何稠周澹,京兆鄠人也。多方術,尤善醫藥,遂為太醫令。明元嘗苦風頭眩,淡療得愈,由此位特進,賜爵成德侯。神瑞二年,京師饑,朝議遷都於鄴,淡與博士祭酒崔浩進計,
 言不可。明元曰:「唯此二人,與朕意同。」詔賜淡、浩妾各一人。



 卒,謚曰恭。



 李脩,字思祖,本陽平館陶人也。父亮,少學醫術,未能精究。太武時奔宋,又就沙門僧坦,略盡其術。針灸授藥,罔不有效。徐、兗間,多所救恤。亮大為事,以舍病人,死者則就而棺殯,親往弔視,其仁厚若此。累遷府參軍督護。本郡士門、宿官,咸相交暱,車馬金帛,酬賚無貲。脩兄元孫隨畢眾敬赴平陽,亦遵父業而不及,以功拜奉朝請。脩略與兄同,晚入代京,歷位中散令,以功賜爵下蔡子,遷給事中。太和中,常在禁內。文明太后時有不豫,脩侍
 針藥多效,賞賜累加,車服第宅,號為鮮麗。集諸學士及工書者百餘人,在東宮撰諸藥方百卷,皆行於世。



 先是咸陽公高允雖年且百歲,而氣力尚康,孝文、文明太后時令脩診視之。一旦,奏言允脈竭氣微,大命無逮,未幾果亡。後卒於太醫令,贈青州刺史。



 徐謇,字成伯,丹陽人也,家本東莞。與兄文伯等皆善醫藥。謇因至青州,慕容白曜平東陽,獲之,送京師。獻文欲驗其能,置病人於幕中,使謇隔而脈之,深得病形,兼知色候,遂被寵遇。為中散,稍遷內行長。文明太后時問經方,而不及李脩之見任用。謇合和藥劑攻療之驗,精妙
 於脩。而性秘忌。承奉不得其意,雖貴為王公,不為措療也。



 孝文遷洛,稍加眷待,體小不平,及所寵馮昭儀有病,皆令處療。又除中散大夫,轉侍御師。謇欲為孝文合金丹,致延年法,乃入居嵩高,採營其物,歷歲無所成,遂罷。二年,上幸縣瓠,有疾大漸,乃馳驛召謇,令水路赴行所,一日一夜行數百里。至,診省有大驗。九月,車駕次於汝濱,乃大為謇設太官珍膳。因集百官,特坐謇於上席,遍陳餚觴於前,命左右宣謇救攝危篤振濟之功,宜加酬齎。乃下詔褒美,以謇為大鴻臚卿、金卿縣伯,又賜錢絹、雜物、奴婢、牛馬,事出豐厚,皆經內呈。諸親王咸陽王禧
 等各有別齎,並至千匹。從行至鄴,上猶自發動,謇日夕左右。明年,從詣馬圈,上疾勢遂甚,蹙蹙不怡,每加切誚,又欲加之鞭捶,幸而獲免。帝崩後,謇隨梓宮還洛。



 謇常有將餌及吞服道,年垂八十,而鬢髮不白,力未多衰。正始元年,以老為光祿大夫。卒,贈安東將軍、齊州刺史,謚曰靖。子踐,字景昇,襲爵,位建興太守。



 文伯仕南齊,位東莞、太山、蘭陵三郡太守。



 子雄,員外散騎侍郎,醫術為江左所稱,事並見《南史》。



 雄子之才,幼而俊發,五歲誦《孝經》,八歲略通義旨。曾與從兄康造梁太子詹事汝南周捨宅,聽《老子》。捨為設食,乃戲之曰:「徐郎不用心思義,而但
 事食乎?」之才答曰:「蓋聞聖人虛其心而實其腹。」舍嗟賞之。年十三,召為太學生,粗通《禮》、《易》。彭城劉孝綽、河東裴子野、吳郡張嵊等每共論《周易》及《喪服》儀,酬應如響。咸共歎曰:「此神童也。」孝綽又云:「徐郎燕頷,有班定遠之相。」陳郡袁昂丹陽尹,辟為主簿,人務事宜,皆被顧訪。郡廨遭火,之才起望,夜中不著衣,披紅眠帕出房,映光為昂所見。功曹白請免職,昂重其才術,仍特原之。



 豫章王綜出鎮江都,復除豫章王國左常侍,又轉綜鎮北主簿。及綜入魏,三軍散走,之才退至呂梁,橋斷路絕,遂為魏統軍石茂孫所止。綜入魏旬月,位至司空。



 魏聽綜收斂
 僚屬,乃訪知之才在彭泗。啟魏帝,云之才大善醫術,兼有機辯。詔征之才。孝昌二年,至洛,敕居南館,禮遇甚優。謇子踐啟求之才還宅。之才藥石多效,又窺涉經史,發言辨捷,朝賢競相耍引,為之延譽。武帝時,封昌安縣侯。天平中,齊神武征赴晉陽,常在內館,禮遇稍厚。武定四年,自散騎常侍轉祕書監。



 文宣作相,普加黜陟,楊愔以其南士,不堪典掌功程,且多陪從,全廢曹務,轉授金紫光祿大夫,以魏收代。之才甚怏怏不平。



 之才少解天文,兼圖讖之學,共館客宋景業參校吉凶,知午年必有革易。因高德正啟之,文宣聞而大悅。時自婁太后及勳貴
 臣咸云:「關西既是勍敵,恐其有挾天子令諸侯之辭,不可先行禪代事。」之才獨云:「千人逐兔,一人得之,諸人咸息。須定大業,何容翻欲學人?」又援引證據,備有條目,帝從之。登阼後,彌見親密。之才非惟醫術自進,亦為首唱禪代,又戲謔滑稽,言無不至,於是大被狎暱。



 尋除侍中,封池陽縣伯。見文宣政令轉嚴,求出,除趙州刺史。竟不獲述職,猶為弄臣。皇建二年,除西兗州刺史,未之官。武明皇太后不豫,之才療之,應手便愈,孝昭賜彩帛千段、錦四百匹。之才既善醫術,雖有外授,頃即徵還。既博識多聞,由是於方術尤妙。



 大寧二年春,武明太后又病,之
 才弟之範為尚藥典御,敕令診候。內史皆令呼太后為石婆,蓋有欲忌,故改名以厭制之。之範出告之才曰:「童謠云:『周里跂求伽,豹祠嫁石婆,斬冢作媒人,唯得一量紫綖靴。』今太后忽改名,私所致怪。」



 之才曰:「跂求伽,胡言去已,豹祠嫁石婆,豈有好事?斬冢作媒人,但令合葬,自斬冢。唯得紫綖靴者,得至四月。何者?紫之為字,此下系,綖者熟,當在四月之中。」之範問靴是何義。之才曰:「靴者革旁化,寧是久物?」至四月一日,后果崩。有人患腳跟腫痛,諸醫莫能識。之才曰:「蛤精疾也,由乘船入海,垂腳水中。」疾者曰:「實曾如此。」之才為剖,得蛤子二,大如榆莢。又
 有以骨為刀子把者,五色斑斕。之才曰:「此人瘤也。」問得處,云:「於古冢見髑髏,額骨長數寸,試削視,有文理,故用之。」其明悟多通如此。



 天統四年,累遷尚書左僕射,俄除兗州刺史,特給鐃吹一部。之才醫術最高,偏被命召。武成酒色過度,怳忽不恆。曾病發,自云,初見空中有五色物,稍近,變成一美婦人,去地數丈,亭亭而立。食頃,變為觀世音。之才云:「此色欲多,大虛所致。」即處湯方,服一劑,便覺稍遠;又服,還變成五色物;數劑湯,疾竟愈。帝每發動,暫遣騎追之,針藥所加,應時必效,故頻有端執之舉。入秋,武成小定,更不發動。和士開欲依次轉進,以之才
 附籍兗州,即是本屬,遂奏附除刺史,以胡長仁為左僕射,士開為右僕射。及十月,帝又病動,語士開云:「浪用之才外任,使我辛苦。」其月八日,敕驛追之才。帝以十日崩,之才十一日方到。既無所及,復還赴州。在職無所侵暴,但不甚閑法理,頗亦疏慢,用捨自由。



 五年冬,後主征之才。尋左僕射闕,之才曰:「自可復禹之績。」武平元年,重除尚書左僕射。之才於和士開、陸令萱母子曲盡卑狎,二家若疾,救護百端。由是遷尚書令,封西陽郡王。祖珽執政,除之才侍中、太子太師。之才恨曰:「子野沙汰我。」珽目疾,故以師曠比之。



 之才聰辯強識,有兼人之敏。尤好劇
 談體語,公私言聚,多相嘲戲。鄭道育常戲之才為師公,之才曰:「既為汝師,又為汝公,在三之義,頓居其兩。」又嘲王昕姓云:「有言則言王,近犬便狂,加頸足而為馬,施角尾而成羊。」盧元明因戲之才云:「卿姓是未入人,名是子之誤,之當為之也。」即答云:「卿姓,在上為虐,在丘為虛,生男則為虜,配馬則為驢。」又常與朝士出游,遙望群犬競走,諸人試令目之。之才即應聲云:「為是宋鵲?為是韓盧?為逐李斯東走?為負帝女南徂?」李諧於廣坐因稱其父名曰:「卿嗜熊白生不?」之才曰:「平平耳。」又曰:「卿此言於理平不?」諧遽出避之,道逢其甥高德正。德正曰:「舅顏色何不
 悅?」



 諧告之故。德正徑造坐席,連索熊白。之才謂坐者曰:「個人諱底?」眾莫之應。



 之才曰:「生不為人所知,死不為人所諱,此何足問。」唐邕、白建方貴,時人言云:「并州赫赫唐與白。」之才茂之。元日,對邕為諸令史祝曰:「卿等位當作唐、白。」又以小史好嚼筆,故常執管就元文遙口曰:「借君齒。」其不遜如此。



 歷事諸帝,以戲狎得寵。武成生齻牙,問諸醫,尚藥典御鄧宣文以實對,武成怒而撻之。後以問之才,拜賀曰:「此是智牙,生智牙者,聰明長壽。」武成悅而賞之。為僕射時,語人曰:「我在江東,見徐勉作僕射,朝士莫不佞之。今我亦是徐僕射,無一人佞我,何由可活!」之
 才妻,魏廣陽王妹,之才從文襄求得為妻。



 和士開知之,乃淫其妻。之才遇見而避之,退曰:「妨少年戲笑。」其縱之如此。



 年八十,卒,贈司徒公、錄尚書事,謚曰文明。



 長子林,字少卿,太尉司馬。次子同卿,太子庶子。之才以其無學術,每歎曰:「終恐同《廣陵散》矣。」



 弟之範亦醫術見知,位太常卿,特聽襲之才爵西陽王。入周,授儀同大將軍。



 開皇中,卒。



 王顯,字世榮,陽平樂平人也。自言本東海郯人,王朗之後也。父安上,少與李亮同師,俱受醫藥,而不及亮。顯少歷本州從事,雖以醫術自通,而明敏有決斷才用。初文
 昭太后之懷宣武,夢為日所逐,化而為龍而繞后,后寤而驚悸,遂成心疾。文明太后敕徐謇及顯等為后診脈,謇云是微風入藏,宜進湯加針。顯言案三部脈,非有心疾,將是懷孕生男之象。果如顯言。久之,補待御師。



 宣武自幼有微疾,顯攝療有效,因稍蒙眄識。又罷六輔之初,顯為領軍于烈間通規策,頗有密功。累遷廷尉卿,仍在侍御,營進御藥,出入禁內。累遷御史中尉。



 顯前後居職,所在著稱。糾折庶獄,究其姦回,出內惜慎,憂國如家。及領憲臺,多所彈劾,百僚肅然。又以中尉屬官不悉稱職,諷求改革。詔委改選,務盡才能。



 而顯所舉,或有請屬,未
 皆得人,於是眾議喧嘩,聲望致損。後宣武詔顯撰藥方三十五卷,班布天下,以療諸疾。東宮建,以為太子詹事,委任甚厚。上每幸東宮,顯常近侍,出入禁中,仍奉醫藥。賞賜累加,為立館宇,寵振當時。以營療功,封衛國縣伯。



 及宣武崩,明帝踐阼,顯參奉璽策,隨從臨哭,微為憂懼。顯既蒙任遇,兼為法官,恃勢使威,為時所疾。朝宰託以侍療無效,執之禁中。詔削爵位,徙朔州。



 臨執呼冤,直閣伊盆生以刀鐶撞其腋下,傷中吐血,至右衛府,一宿死。子曄,尚書儀曹郎中,懼走,後被獲,拷掠百餘。宅沒於官。



 初,顯構會元景,就刑南臺。及顯之死,在右衛府,唯隔一
 巷,相去數十步。



 世以為有報應之驗。始顯布衣為諸生,有沙門相顯,後當富貴,誡其勿為吏,為吏必敗。由是宣武時,或欲令其兼攝吏部。每殷勤辭避。及宣武崩,帝夜即位,受璽策,於儀須兼太尉及吏部,倉卒,百官不具,以顯兼吏部行事。又顯未敗之前,有嫗卜相於市者,言人吉凶頗驗。時子曄已為郎,聞之,微服就嫗,問己終至何官。



 嫗言:「君今既有位矣,不復更進,當受父冤。」並如其語。



 馬嗣明,河內野王人也。少博綜經方,為人診脈,一年前知其生死。邢邵唯一子大寶,甚聰慧,年十七八患傷寒。嗣明為其診脈,退告楊愔云:「邢公子傷寒不療自差,然
 脈候不出一年便死。覺之少晚,不可復療。」數日後,楊、邢並侍宴內殿。文宣云:「邢子才兒大不惡,我欲乞其隨近一郡。」楊以年少,未合剖符。宴罷,奏云:「馬嗣明稱大寶脈惡,一年內恐死,若其出郡,醫藥難求。」遂寢。大寶未期而卒。楊愔患背腫,嗣明以練石塗之,便差,因此大為楊愔所重。作練石法:以麤黃色石如鵝鴨卵大,猛火燒令赤,內淳醋中,自有石屑落醋裏,頻燒至石盡,取石屑曝乾,搗下簁,和醋以塗腫上,無不愈。



 武平中,為通直散騎常侍,針灸孔穴,往往與《明堂》不同。嘗有一家,二奴俱患,身體遍青,漸虛嬴不能食。訪諸醫,無識者。嗣明為灸兩足
 趺上各三七壯,便愈。武平末,從駕往晉陽,至遼陽山中,數處見榜,云有人家女病,若能差之者,購錢十萬。又諸名醫多尋榜至是人家,問疾狀,俱不下手。唯嗣明為之療。問其病由,云曾以手持一麥穗,即見一赤物長二尺許,似蛇,入其手指中,因驚倒地,即覺手臂疼腫。月餘日,漸及半身,肢節俱腫,痛不可忍,呻吟晝夜不絕。嗣明即為處方,令馳馬往都市藥,示其節度,前後服十劑湯,一劑散。比嗣明明年從駕還,此女平復如故。嗣明藝術精妙,多如是。



 隋開皇中,卒於太子藥藏監。然性自矜大,輕諸醫人,自徐之才、崔叔鸞以還,俱為其所輕。



 姚僧垣,字法衛,吳興武康人,吳太常信之八世孫也。父菩提,梁高平令。嘗嬰疾疹歷年,乃留心醫藥。梁武帝召與討論方術,言多會意,由是頗禮之。僧垣幼通洽,居喪盡禮,年二十四,即傳家業。仕梁為太醫正,加文德主帥。梁武帝嘗因發熱,服大黃。僧垣曰:「大黃快藥,至尊年高,不宜輕用。」帝弗從,遂至危篤。



 太清元年,轉鎮西湘東王府中記室參軍。僧垣少好文史,為學者所稱。及梁簡文嗣位,僧垣兼中書舍人。梁元帝平侯景,召僧垣赴荊州,改受晉安王府諮議。梁元帝嘗有心腹病,諸醫皆請用平藥。僧垣曰:「脈洪實,宜用大黃。」元帝從之。進湯訖,果下
 宿食,因而疾愈。時初鑄錢,一當十,乃賜十萬貫,實百萬也。及魏軍剋荊州,僧垣猶侍梁元,不離左右,為軍人所止,方泣涕而去。尋而周文遣使馳驛徽僧垣。燕公于謹固留不遣,謂使人曰:「吾年衰暮,疾病嬰沉,今得此人,望與之偕老。」周文以謹勳德隆重,乃止。明年,隨謹至長安。



 武成元年,授小畿伯下大夫。金州刺史伊婁穆以疾還京,請僧垣省疾,乃云自腰至臍,似有三縛,兩腳緩縱,不復自持。僧垣即為處湯三劑,穆初服一劑,上縛即解;次服一劑,中縛復解;又服一劑,三縛悉除。而兩腳疼痺,猶自攣弱。更為合散一劑,稍得屈申。僧垣曰:「終待霜降,此
 患當愈。」及至九月,遂能起行。



 大將軍、襄樂公賀蘭隆先有氣疾,加以水腫,喘息奔急,坐臥不安。或有勸其服決命大散者,其家疑未能決,乃問僧垣。僧垣曰:「意謂此患,不與大散相當。」即為處方,勸急使服,便即氣通。更服一劑,諸患悉愈。大將軍、樂平公竇集暴感風疾,精神瞀亂,無所覺知。醫先視者,皆云已不可救。僧垣後至曰:「困矣,終當不死。」為合湯散,所患即療。大將軍、永世公叱伏列椿苦痢積時,而不損廢朝謁。



 燕公謹嘗問僧垣曰:「樂平、永世,俱有痼疾,意永世差輕。」對曰:「夫患有深淺,時有危殺,樂平雖困,終當保全;永世雖輕,必不免死。」謹曰:「當在
 何時?」



 對曰:「不出四月。」果如其言,謹歎異之。



 天和六年,遷遂伯中大夫。建德三年,文宣太后寢疾,醫巫雜說,各有同異。



 武帝引僧垣坐,問之。對曰:「臣準之常人,竊以憂懼。」帝泣曰:「公既決之矣,知復何言!」尋而太后崩。其後復因召見,乃授驃騎大將軍、開府儀同三司。敕停朝謁,若非別敕,不勞入見。四年,帝親戎東討,至河陰遇疾,口不能言;瞼垂覆目,不得視;一足短縮,又不得行。僧垣以為諸藏俱病,不可並療,軍中之要,莫過於語,乃處方進藥,帝遂得言。次又療目,目疾便愈。未及足,足疾亦瘳。比至華州,帝已痊復。即除華州刺史,仍詔隨駕入京,不令在鎮。
 宣政元年,表請致仕,優詔許之。是歲,帝幸雲陽,遂寢疾,乃召僧垣赴行在所。內史柳昂私問曰:「至尊脈候何如?」對曰:「天子上應天心,或當非愚所及。若凡庶如此,萬無一全。」



 尋而帝崩。



 宣帝初在東宮,常苦心痛,乃令僧垣療之,其疾即愈。及即位,恩禮彌隆。謂曰:「嘗聞先帝呼公為姚公,有之?」對曰:「臣曲荷殊私,實如聖旨。」帝曰:「此是尚齒之辭,非為貴爵之號。朕當為公建國開家,為子孫永業。」乃封長壽縣公。冊命之日,又賜以金帶及衣服等。大象二年,除太醫下大夫。帝尋有疾,至於大漸,僧垣宿直侍疾。帝謂隋公曰:「今日性命,唯委此人。」僧垣知帝必不全
 濟,乃對曰:「臣但恐庸短不逮,敢不盡心!」帝頷之。及靜帝嗣位,遷上開府儀同大將軍。



 隋開皇初,進爵北絳郡公。三年,卒,年八十五。遺誡衣帢入棺,朝服勿斂,靈上唯置香奩,每日設清水而已。贈本官,加荊、湖二州刺史。



 僧垣醫術高妙,為當時所推,前後效驗,不可勝紀。聲譽既盛,遠聞邊服,至於諸蕃外域,咸請託之。僧垣乃參校徵效者為《集驗方》十二卷,又撰《行記》三卷,行於世。



 長子察,《南史》有傳。



 次子最,字士會。博通經史,尤好著述。年十九,隨僧垣入關。明帝盛聚學徒,校書於麟趾殿,最亦預為學士。俄授齊王憲府水曹參軍,掌記室事,特為憲所禮接。



 最幼在江左,迄於入關,未習醫術。天和中,齊王憲奏遣最習之。憲又謂最曰:「博學高才,何如王褒、庾信?王庾名重兩國,吾視之蔑如,接待資給,非爾家比也。勿不存心。且天子有敕,彌須勉勵。」最於是始受家業,十許年中,略盡其妙。



 每有人告請,效驗甚多。



 隋文帝踐極,除太子門大夫。以父憂去官,哀毀骨立。既免喪,襲爵北絳郡公,復為太子門大夫。俄轉蜀王秀友。秀鎮益州,遷秀府司馬。及平陳,察至,最自以非嫡,讓封於察,隋文帝許之。秀後陰有異謀,隋文帝令公卿窮其事。開府慶整、郝瑋等並推過於秀。最獨曰:「凡有不法,皆最所為,王實不知也。」榜
 訊數百,卒無異辭,竟坐誅。論者義之。撰《梁後略》十卷,行於世。



 褚該,字孝通,河南陽翟人也。父義昌,梁鄱陽王中記室。該幼而謹厚,尤善醫術。仕梁,歷武陵王府參軍,隨府西上,後與蕭捴同歸周。自許奭死後,該稍為時人所重,賓客迎候,亞於姚僧垣。天和初,位縣伯下大夫,進授車騎大將軍、儀同三司。該性淹和,不自矜尚,但有請之者,皆為盡其藝術。時論稱其長者。後以疾卒。子則,亦傳其家業。



 許智藏,高陽人也。祖道幼,常以母疾,遂覽醫方,因而究
 極,時號名醫。誡諸子曰:「為人子者,嘗膳視藥,不知方術,豈謂孝乎。」由是,遂世相傳授。仕梁,位員外散騎侍郎。父景,武陵王諮議參軍。智藏少以醫術自達,仕陳,為散騎常侍。陳滅,隋文帝以為員外散騎侍郎,使詣揚州。會秦王俊有疾,上馳召之。俊夜夢其亡妃崔氏泣曰:「本來相迎,如聞許智藏將至。其人若到,當必相苦,為之奈何?」明夜,俊又夢崔氏曰:「妾得計矣,當入靈府中以避之。」及智藏至,為俊診脈曰:「疾已入心,即當發癇,不可救也。」果如言,俊數日而薨。上奇其妙,齎物百段。煬帝即位,智藏時致仕。帝每有苦,輒令中使就宅詢訪,或以輦迎入殿,扶
 登御床。智藏為方奏之,用無不效。卒於家,年八十。



 宗人許澄,亦以醫術顯。澄父奭,仕梁,為中軍長史,隨柳仲禮入長安,與姚僧垣齊名,拜上儀同三司。澄有學識,傳父業,尤盡其妙。歷位尚藥典御、諫議大夫,封賀川縣伯。父子俱以藝術名重於周隋二代,史失其事,故附云。



 萬寶常,不知何許人也。父大通,從梁將王琳歸齊,後謀還江南,事泄伏誅。



 由是寶常被配為樂戶,因妙達鐘律,遍工八音。與人方食,論及聲調。時無樂器,寶常因取前食器及雜物,以箸扣之,品其高下,宮商畢備,諧於絲竹,大為時人所賞。然歷周、隋,俱不得調。



 開皇初,沛國公鄭
 譯等定樂,初為黃鐘調。寶常雖為伶人,譯等每召與議,然言多不用。後譯樂成,奏之。上召寶常,問其可不。寶常曰:「此亡國之音,豈陛下所宜聞!」上不悅。寶常因極言樂聲哀怨淫放,非雅正之音,請以水尺為律,以調樂器,其聲率下鄭譯調二律。并撰《樂譜》六十四卷。且論八音旋相為宮法,改弦移柱之變,為八十四調,一百四十律,變化終於一千八百聲。時以《周禮》有旋宮之義,自漢已來,知音不能通,見寶常特創其事,皆哂之。至是,試令為之,應手成曲,無所疑滯,見者莫不嗟異。於是損益樂器,不可勝紀。其聲雅淡,不為時人所好。太常善聲者,多排毀
 之。又太子洗馬蘇夔以鐘律自命,尤忌寶常。夔父威方用事,凡言樂者附之而短寶常。數詣公卿怨望,蘇威因詰寶常所為,何所傳受。



 有一沙門謂寶常曰:「上雅好符瑞,有言徵祥者,上皆悅之。先生當言徒胡僧受學,云是佛家菩薩所傳音律,則上必悅。先生當言,所為可以行矣。」寶常遂如其言以答威。威怒曰:「胡僧所傳,乃四夷之樂,非中國宜行。」其事竟寢。寶常聽太常所奏樂,泫然泣曰:「樂聲淫厲而哀,天下不久將盡。」時四海全盛,聞言者皆謂不然。大業之末,其言卒驗。



 寶常貧而無子,其妻因其臥疾,遂竊其資物而逃,寶常竟餓死。將死,取其所著
 書焚之,曰:「何用此為?」見者於火中探得數卷,見行於世。



 開皇中,鄭譯、何妥、盧賁、蘇夔、蕭吉並討論墳籍,撰著樂書,皆為當時所用,至於天然識樂,不及寶常遠矣。安馬駒、曹妙達、王長通、郭令樂等能造曲,為一時之妙,又習鄭聲,而寶常所為,皆歸於雅。此輩雖公議不附寶常,然皆心服,謂以為神。時樂人王令言亦妙達音律。大業末,煬帝將幸江都,令言之子嘗於戶外彈胡琵琶,作翻調《安公子曲》,令言時臥室中,聞之驚起,曰:「變!變!」急呼其子曰:「此曲興自早晚?」其子曰:「頃來有之。」令言遂歔欷流涕,謂其子曰:「汝慎無從行,帝必不反。」子問其故,令言曰:「此
 曲宮聲往而不反。宮,君也,吾所以知之。」帝竟被弒於江都。



 蔣少游,樂安博昌人也。魏慕容白曜之平東陽,見俘,入於平城,充平齊戶。



 後配雲中為兵。性機巧,頗能畫刻,有文思,吟詠之際,時有短篇。遂留寄平城,以傭寫書為業,而名猶在鎮。後被召為中書寫書生,與高聰俱依高允。允並薦之,與聰俱補中書博士。自在中書,恆庇於李沖兄弟子姪之門。始北方不悉青州蔣族,或謂少游本非人士,又少游微,因工藝自達,是以公私人望,不至相重,唯高允、李沖,曲為體練。孝文、文明太后嘗因密宴謂百
 官曰:「本謂少游作師耳,高允老公乃言其人士。」然猶驟被引命,以規矩刻繢為務,因此大蒙恩賜,而位亦不遷陟也。



 及詔尚書李沖與馮誕、游明根、高閭等議定衣冠於禁中,少游巧思,令主其事。



 亦訪於劉昶。二意相乖,時致諍競,積六載乃成,始班賜百官。冠服之成,不游有效焉。後於平城將營太廟太極殿,遣少游乘傳詣洛,量準魏、晉基趾。後為散騎侍郎,副李彪使江南。孝文脩船乘,以其多有思力,除都水使者。遷兼將作大匠,仍領水池湖泛戲舟楫之具。及華林殿詔脩舊增新,改作金墉門樓,皆所措意,號為妍美。雖有文藻,而不得申其才用。恆
 以剞劂繩尺,碎據匆匆,徙倚園、湖、城、殿之側,識者為之歎慨。而乃坦爾為己任,不告疲恥。又兼太常少卿,都水如故。卒,贈龍驤將軍、青州刺史,謚曰質。有文集十卷餘。少游又為太極立模範,與董爾、王遇等參建之,皆未成而卒。



 初,文成時,郭善明甚機巧,北京宮殿,多其製作。孝文時,青州刺史侯文和亦以巧聞,為要舟,水中立射。滑稽多智,辭說無端,尤善淺俗委巷之語,至可玩笑。位樂陵、濟南二郡太守。宣武、明帝時,豫州人柳儉、殿中將軍關文備、郭安興並機巧。洛中製永寧寺九層佛圖,安興為匠也。



 始孝文時,有范寧兒者善圍棋,曾與李彪使齊。
 齊令江南上品王抗與寧兒,制勝而還。又有浮陽高光宗善樗蒲。趙國李幼序、洛陽丘何奴並工握槊。此蓋胡戲,近入中國。云胡王有弟一人遇罪,將殺之,弟從獄中為此戲以上之,意言孤則易死也。宣武以後,大盛於時。



 何稠,字桂林,國子祭酒妥之兄子也。父通,善琢玉。稠年十餘,遇江陵平,隨妥入長安。仕周,御飾下士。及隋文帝為丞相,召補參軍,並掌細作署。開皇中,累遷太府丞。稠博覽古圖,多識舊物。波斯嘗獻金線錦袍,組織殊麗。上命稠為之。



 稠錦成,踰所獻者,上甚悅。時中國久絕琉璃作,匠人無敢措意,稠以綠瓷為之,與真不異。尋加員外
 散騎侍郎。



 開皇末,桂州俚李光仕為亂,詔稠募討之。師次衡嶺,遣使招其渠帥,洞主莫崇解兵降款,桂州長史王文同鎖崇詣稠所。稠詐宣言曰:「州縣不能綏養,非崇之罪。」命釋之,引共坐,與從者四人,為設酒食遣之。大悅,歸洞不設備。稠至五更,掩及其洞,悉發俚兵以臨餘賊,象州逆帥杜條遼、羅州逆帥龐靖等相斷降款。



 分遣建州開府梁暱討叛夷羅壽,羅州刺史馮暄討賊帥李大檀,並平之。承制署首領為州縣官而還,眾皆悅服。有欽州刺史甯猛力帥眾迎軍。初,猛力欲圖為逆,至是惶懼,請身入朝。稠以其疾篤,示無猜貳,放還州,與約八九月
 詣京師相見。稠還奏狀,上意不懌。其年十月,猛力卒,上謂稠曰:「汝前不將猛力來,今竟死矣。」



 稠曰:「猛力共臣約,假令身死,當遣子入侍。越人性直,其子必來。」初,猛力臨終,誡其子長真曰:「我與大使期,不可失信於國士,汝葬我訖,即宜上路。」



 長真如言入朝。上大悅曰:「何稠著信蠻夷,乃至於此!」以勳授開府。



 仁壽初,文獻皇后崩,稠與宇文愷參典山陵制度。稠性少言,善候上旨,由是漸見親暱。上疾篤,謂稠曰:「汝既曾葬皇后,今我方死,亦宜好安置。囑此何益?



 但不能忘懷耳。魂而有知,當相見於地下。」上因攬太子頸曰:「何稠用心,我後事動靜當共平章。」



 大
 業初,煬帝將幸揚州,敕稠討閱圖籍,造輿服羽儀,送至江都。其日,拜太府少卿。稠於是營黃麾三萬六千人仗,及車輿輦輅、皇后鹵簿、百官儀服,依期而就,送於江都。所役工十萬餘人,用金銀錢物巨億計。帝使兵部侍郎胡雅、選部郎薛邁等勾覆,數年方竟,毫釐無舛。



 稠參會今古,多所改創。魏、晉以以來,皮弁有纓而無笄導。稠曰:「此古田獵服也,今服以入朝,宜變其制。」故弁施象牙簪導,自稠始也。又從省之服,初無佩綬。稠曰:「此乃晦朔小朝之服,安有人臣謁帝,而除去印綬,兼無佩玉之節乎?」乃加獸頭小綬及佩一隻。舊制,五輅於轅上起箱,天子與
 參乘同在箱內。稠曰:「君臣同所,過為相逼。」乃廣為盤輿,別構欄楯,侍臣立於其中。於內復起須彌平坐,天子獨居其上。自餘麾幢文物,增損極多。帝復令稠造戎車萬乘,鉤陳八百連。帝善之,以稠守太府卿,後兼領少府監。



 遼東之役,攝左屯衛將軍,領御營弩手三萬人。時工部尚書宇文愷造遼水橋不成,師未得濟,左屯衛大將軍麥鐵杖因而遇害。帝遣稠造橋,二日而就。初,稠制行殿及六合城,至是,帝於遼左與賊相對,夜中施之。其城,周回八里,城及女垣合高十仞,上布甲士,立仗建旗,四隅置闕,面列一觀,觀下三門,比明而畢。高麗望見,謂若神
 功。稍加至右光祿大夫。從幸江都,遇宇文化及亂,以為工部尚書。



 及敗,陷於竇建德,復為工部尚書、舒國公。建德敗,歸於大唐,授少府監,卒。



 又齊時有河間劉龍者,性強明,有巧思。齊後主令修三雀臺稱旨,因而歷職通顯。及隋文帝踐阼,大見親委,位右衛將軍,兼將作大匠。遷都之始,與高熲參掌制度,世號為能。



 大業中,有南郡公黃亙及弟兗,俱巧思絕人,煬帝每令其兄弟亙少府將作。於時改創多務,亙、兗每參典其事。凡有所為,何稠先令亙、兗立樣,當時工人莫有所損益。亙,位朝散大夫;兗,散騎侍郎。



 論曰:陰陽卜祝之事,聖哲之教存焉,雖不可以專,亦不可得而廢也。徇於是者不能無非,厚於利者必有其害。《詩》、《書》、《禮》、《樂》所失也淺,故先王重其德;方術伎巧所失也深,故往哲輕其藝。夫能通方術而不詭於俗;習伎巧而必蹈於禮者,幾於大雅君子。故昔之通賢,所以戒乎妄作。晁崇、張深、殷紹、王早、耿玄、劉靈助、李順興、檀特師、由吾道榮、顏惡頭、王春、信都芳、宋景業、許遵、吳遵世、趙輔和、皇甫玉、解法選、魏寧、綦母懷文、張子信、陸法和、蔣昇、強練、庾季才、盧太翼、耿詢、來和、蕭吉、楊伯醜、臨孝恭、劉祐、張胄玄等,皆魏來術藝之士也。觀其占候卜筮,推步
 盈虛,通幽洞微,近知鬼神之情狀,其間有不涉用於龜筴,而究人事之吉凶,如順興、檀特之徒,法和、強練之輩,將別稟數術,詎可以智識知?及江陵失守,前巧盡棄,還吳無路,入周不可,因歸事齊,厚蒙榮遇。雖竊之以叨濫,而守之以清虛,生靈所資,嗜欲咸遣,斯亦得道家之致矣。信都芳所明解者,乃是經國之用乎?周澹、李脩、徐謇、謇兄孫之才、王顯、馬嗣明、姚僧垣、褚該、許智藏方藥特妙,各一時之美也。而僧垣診候精審,名冠一代,其所全濟,固亦多焉。而弘茲義方,皆為令器,故能享眉壽,縻好爵。



 老聃云「天道無親,常與善人」,於是信矣!許氏之運針
 石,百載可稱。寶常聲律之奇,足以追蹤牙、曠,各一時之妙也。蔣、何以剞劂見知,沒其學思,藝成為下,其近是乎?



 周時,有樂茂雅以陰陽顯,史元華以相術稱,並所闕也。



\end{pinyinscope}