\article{卷八十八列傳第七十六 隱逸}

\begin{pinyinscope}

 眭
 夸馮亮鄭脩崔廓子賾徐則張文詡蓋兼濟獨善,顯晦之殊,其事不同,由來久矣。昔夷、齊獲全於周武,華矞不容於太公,何哉?求其心者,許以激貪之用;督其迹者,矯以教義之風。而肥遁不歸,代有其人矣。故《易》稱「遁世無悶」,「不事王侯」。《詩》云「皎皎白駒,在彼空
 谷」。《禮》云「儒有上不臣天子,下不事諸侯」。《語》曰「舉逸民,天下之人歸心焉」。雖出處殊途,語默異用,各言其志,皆君子之道也。



 洪崖兆其始,箕山扇其風,七人作乎周年,四皓光乎漢日。魏、晉以降,其流逾廣。其大者則輕天下,細萬物;其小者則安苦節,甘賤貧。或與世同塵,隨波瀾以俱逝;或違時矯俗,望江湖而獨往。狎玩魚鳥,左右琴書,拾遺粒而織落毛,飲石泉而庇松柏。放情宇宙之外,自足懷抱之中。然皆欣欣於獨善,鮮汲汲於兼濟。



 夷情得喪,忘懷累有。比夫邁德弘道,匡俗庇人,可得而小,不可得而忽也。而受命哲王,守文令主,莫不束帛交馳,蒲輪
 結轍,奔走巖谷,唯恐不逮者,何哉?以其道雖未弘,志不可奪,縱無舟楫之功,終有堅貞之操,足以立懦夫之志,息貪競之風。與茍得之徒,不可同年共日,所謂無用以為用,無為而無不為也。



 自叔世澆浮,淳風殆盡,錐刀之末,競入成群。而能冥心物表,介然離俗,望古獨適,求友千齡,亦異人矣!何必御霞乘雲而追日月,窮極天地,始為超遠哉!



 案《魏書》列眭夸、馮亮、李謐、鄭修為《逸士傳》。《隋書》列李士謙、崔廓、廓子賾、徐則、張文詡為《隱逸傳》。今以李謐、士謙附其家傳,其餘並編附篇,以備《隱逸傳》云。



 眭夸,一名旭,趙郡高邑人也。祖邁,晉東海王越軍謀掾,
 後沒石勒,為徐州刺史。父邃,字懷道,慕容寶中書令。夸少有大度,不拘小節,耽好書傳,未曾以世務經心。好飲酒,浩然物表。年三十,遭父喪,須鬢致白,每一悲哭,聞者為之流涕。高尚不仕,寄情丘壑。同郡李順願與之交,夸拒而不許。邦國少長莫不憚之。



 少與崔浩為莫逆之交。浩為司徒,奏徵為中郎,辭疾不赴。州郡逼遣,不得已,入京都,與浩相見。經留數日,唯飲酒談敘平生,不及世利。浩每欲論屈之,竟不能發言,其見敬憚如此。浩後遂投詔書於夸懷,亦不開口。夸曰:「桃簡,卿已為司徒,何足以此勞國士也?吾便將別。」桃簡,浩小名。浩慮夸即還,時乘
 一騾,更無兼騎,乃以夸騾內之廄中,冀相維縶。夸遂託鄉人輸租者,謬為御車,乃得出關。



 浩知而歎曰:「眭夸獨行士,本不應以小職辱之,又使其人杖策復路,吾當何辭以謝也!」時朝法甚峻,夸既私還,將有私歸之咎。浩仍相左右,始得無坐。經年,送夸本騾,兼遺以所乘馬,為書謝之。夸更不受其騾馬,亦不復書。及浩沒,為之素服,受鄉人弔唁,經一時乃止。歎曰:「崔公既死,誰能更容眭誇!」婦父巨鹿魏攀,當時名達之士,未嘗備婿之禮,情同朋好。或人謂夸曰:「吾聞有大才者必居貴仕,子何獨在桑榆乎?」遂著《知命論》以釋之。及卒,葬日赴會者如市。無子。



 馮亮,字靈通,南陽人,梁平北將軍蔡道恭之甥也。少博覽諸書,又篤好佛理。



 隨道恭至義陽,會中山王英平義陽,獲焉。英素聞其名,以禮待接。亮性清靜,後隱居嵩山,感英之德,以時展覲。英亡,亮奔赴,盡其哀慟。宣武嘗召以為羽林監,領中書舍人,將令侍講《十地》諸經,固辭不許。又欲使衣幘入見,苦求以幅巾就朝,遂不強逼。還山數年,與僧禮誦為業,蔬食飲水,有終焉之志。會逆人王敞事發,連山中沙門法。而亮被執赴尚書省,十餘日,詔特免雪。亮不敢還山,遂寓居景明寺,敕給衣食及其從者數人。後思其舊居,復還山室。亮既雅愛山水,又兼工
 思,結架巖林,甚得栖遊之適。頗以此聞,宣武給其工力,令與沙門統僧暹、河南尹甄深等同視嵩山形勝之處,遂造閑居佛寺。林泉既奇,營製又美,曲盡山居之妙。



 亮時出京師。延昌二年冬,因遇篤疾,宣武敕以馬輿送令還山,居嵩高道場寺,數日卒。詔贈帛二百匹,以供凶事。



 遺誡兄子綜,殮以衣蒨,左手持板,右手執《孝經》一卷,置尸盤石上,去人數里外,積十餘日,乃焚於山,灰燼處,起佛塔經藏。初、亮以盛冬喪,連日驟雪,窮山荒澗,鳥獸飢窘,僵尸山野,無所防護。時有壽春道人惠需,每旦往看其屍,拂去塵霰。禽蟲之迹,交橫左右,而初無侵毀。衣服
 如本,唯風蒨巾。又以亮識舊南方法師信大栗十枚,言期之將來十地果報,開亮手,以置把中。經宿,乃為蟲鳥盜食,皮殼在地,而亦不傷肌體。焚燎之日,有素霧蓊鬱,回繞其傍,自地屬天,彌朝不絕。山中道俗營助者百餘人,莫不異焉。



 鄭脩,北海人也。少隱於岐南凡谷中,依巖結宇,不交世俗,雅好經史,專意玄門。前後州將,每征不至。岐州刺史魏蘭根頻遣致命,脩不得已,暫出見蘭根,尋還山舍。蘭根申表薦脩,明帝詔付雍州刺史蕭寶夤訪實以聞。會寶夤作逆,事不行。



 崔廓,字士玄,博陵安平人也。父子元,齊燕州司馬。廓少孤貧,母賤,由是不為邦族所齒。初為里佐,屢逢屈辱,於是感激,逃入山中。遂博覽書籍,多所通涉,山東學者皆宗之。既還鄉,不應辟命。與趙郡李士謙為忘言友,時稱崔、李。



 士謙死,廓哭之慟,為之作傳,輸之秘府。士謙妻盧氏寡居,每家事,輒令人諮廓取定。廓嘗著論言刑名之理,其義甚精,文多不載。隋大業中,終於家。



 子賾,字祖浚,七歲能屬文。容貌短小,有口辯。開皇初,秦孝王薦之,射策高第。詔與諸儒定樂,授校書郎,轉協律郎。太常卿蘇威雅重之。母憂去職,性至孝,水漿不入口者五日。後徵
 為河南、豫章二王侍讀,每更日來往二王之第。及河南為晉王,轉記室參軍,自此去豫章。王重之不已,遺賾書曰:昔漢氏西京,梁王建國,平臺東苑,慕義如林,馬卿辭武騎之官,枚乘罷弘農之守。每覽史傳,嘗竊怪之:何乃脫略官榮,棲遲籓邸?以今望古,方知雅志。彼二子者,豈徒然哉!足下博聞強記,鉤深致遠,視漢臣之三篋,似陟蒙山;對梁相之五車,若吞雲夢。吾兄欽賢重士,敬愛忘疲,先築郭隗之宮,常置穆生之醴。今者重開土宇,更誓山河。地方七百,牢籠曲阜;城兼七十,包舉臨淄。大啟南陽,方開東閤。想得奉飛蓋,曳長裾,藉玳筵,躡珠履,歌山
 桂之偃蹇,賦池竹之檀欒。



 其崇貴也如彼,其風流也如此,幸甚幸甚,何樂如之!高視上京,有懷德祖;才謝天人,多慚子建。書不盡意,寧俟繁辭。



 賾答曰:一昨伏奉教書,榮貺非恆,心靈自失。若乃理高《象繫》,管輅思而不解;事富《山海》,郭璞注而未詳。至於五色相宣,八音繁會,鳳鳴不足喻,龍章莫之比。



 吳札之論《周頌》,詎盡揄揚;郢客之奏《陽春》,誰能赴節?伏惟令王殿下,稟潤天潢,承輝日觀,雅道邁於東平,文藝高於北海。漢則馬遷、蕭望,晉則裴楷、張華。雞樹騰聲,鵷池播美,望我清塵,悠然路絕。



 祖濬燕南贅客,河朔惰游,本無意於希顏,豈有心於慕藺。未
 嘗聚螢映雪,懸頭刺股。讀《論》唯取一篇,披《莊》不過盈尺。況復桑榆漸暮,藜藿屢空,舉燭無成,穿楊盡棄。但以燕求馬首,薛養雞鳴,謬齒鴻儀,虛班驥IZ。挾太山而超海,比報德而非難;堙崑崙以為池,匹酬恩而反易。



 忽屬周桐錫瑞,唐水承家,門有將相,樹宜桃李。真龍將下,誰好有名;濫吹先逃,何須別聽。但慈旨抑揚,損上益下,江海所以稱王,丘陵為之不逮。曹植儻豫聞高論,則不殞令名;楊脩若竊在下風,亦詎虧淳德。無任荷戴之至,謹奉啟以聞。



 豫章得書,齎米五十石,并衣服、錢帛。時晉邸文翰,多成其手。王入東宮,除太子齋帥,俄兼舍人。及元德
 太子薨,以疾歸於家。後征起居舍人。



 大業四年,從駕汾陽宮,次河陽鎮。藍田令王曇於藍田山得一玉人,長三四寸,著大領衣,冠幘。奏之。詔問群臣,莫有識者。賾答曰:「謹案:漢文帝已前,未有冠幘,即是文帝以來所製也。臣見魏大司農盧元明撰《嵩高山廟記》云:『有神人,以玉為形,像長數寸,或出或隱,出則令世延長。』伏惟陛下,應天順人,定鼎嵩、雒,岳神自見,臣敢稱慶。」因再拜,百官畢賀。天子大悅,賜縑二百匹。



 從駕往太山,詔問賾曰:「何處有羊腸阪?」賾答曰:「臣案《漢書地理志》,上黨壺關縣有羊腸阪。」帝曰:「不是。」又答曰:「臣案皇甫士安撰《地書》。云太原北
 九十里,有羊腸阪。」帝曰:「是也。」因謂牛弘曰:「崔祖濬所謂問一知二。」



 五年,受詔與諸儒撰《區宇圖志》二百五十卷,奏之。帝不善之,更令虞世基、許善心演為六百卷。以父憂去職,尋起令視事。遼東之役,授鷹揚長史。置遼東郡縣名,皆賾之議也。奉詔作《東征記》。九年,除越王長史。於時山東盜賊蜂起,帝令撫慰高陽、襄國,歸首者八百餘人。十二年,從駕江都。宇文化及之弒帝也,引為著作郎,稱疾不起。在路發疾,卒於彭城,年六十九。


賾與河南元善、河東柳
 \gezhu{
  巧言}
 、太原王劭、吳興姚察、瑯琊諸葛潁、信都劉焯、河間劉炫相善,每因休假,清談竟日。所著詞、賦、碑、志
 十餘萬言,撰《洽聞志》七卷,《八代四科志》三十卷。未及施行,江都傾覆,咸為煨燼。



 徐則,東海郯人也。幼沈靜,寡嗜欲,受業於周弘正,善三玄,精於論議,聲擅都邑。則歎曰:「名者實之賓,吾其為賓乎!」遂懷栖隱之操,杖策入縉雲山。



 後學者數百人苦請教授,則謝而遣之。不娶妻,常服巾褐。陳太建中,應召來憩於至真觀。期月,又辭入天台山。因絕粒養性,所資唯松水而已,雖隆冬冱寒,不服綿絮。太傅徐陵為之刊山立頌。



 初在縉雲山,太極真人徐君降之曰:「汝年出八十,當為王者師,然後得道也。」



 晉王廣鎮揚州,聞其名,手書
 召之曰:「夫道得眾妙,法體自然,包涵二儀,混成萬物,人能弘道,道不虛行。先生履德養空,宗玄齊物,深曉義理,頗味法門。悅性沖玄,恬神虛白,餐松餌術,棲息煙霞。望赤城而待風雲,游玉堂而駕龍鳳。雖復藏名台嶽,猶且騰實江、淮。藉甚嘉猷,有勞寤寐。欽承素道,久積虛襟,側席幽人,夢想巖穴。霜風已冷,海氣將寒,偃息茂林,道體休悆。昔商山四皓,輕舉漢庭;淮南八公,來儀籓邸。古今雖異,山谷不殊。市朝之隱,前賢已說。導凡述聖。非先生而誰?故遣使人,往彼延請,想無勞東帛,賁然來思,不待蒲輪,去彼空谷。希能屈己,佇望披雲。」則謂門人曰:「吾今
 年八十一,王來召我,徐君之旨,信而不征。」於是遂詣揚州。晉王將請受道法,則辭以時日不便。其後夕中,命待者取香火,如平常朝禮之儀,至於五更而死。支體柔弱如生,停留數旬,顏色不變。晉王下書曰:「天台真隱東海徐先生,虛確居宗,沖玄成德,齊物處外,檢行安身。草褐蒲衣,餐松餌,栖隱靈岳,五十餘年。卓矣仙才,飄然騰氣,千尋萬頃,莫測其涯。寡人欽承道風,久餐德素,頻遣使乎,遠此延屈,冀得虔受上法,式建良緣。至止甫爾,未淹旬日,厭塵羽化,反真靈府。身體柔軟,顏色不變,經方所謂屍解地仙者哉。誠復師禮未申,而心許有在,雖忘
 怛化,猶愴於懷。喪事所資,隨須供給。霓裳羽蓋,既且騰雲;空槨餘衣,詎藉墳壟?但杖舄在爾,可同俗法。宜遣使人,送還天台定葬。」


是時,自江都至天台,在道多見則徒步,云得放還。至其舊居,取經書道法,分遣弟子,仍令凈掃一房,曰:「若有客至,宜延之於此。」然後跨石梁而去,不知所之。須臾屍柩至,知其靈化,時年八十二。晉王聞而益異之,賵物千段,遣畫工圖其狀,令柳
 \gezhu{
  巧言}
 為之贊。



 時有建安宋玉泉、會稽孔道茂、丹陽王遠知等,亦行辟穀道,以松水自給,皆為煬帝所重。



 張文詡,河東人也。父琚,開皇中,為洹水令,以清正聞。文
 詡博覽群書,特精《三禮》。隋文帝方引天下名儒碩學之士,文詡時游太學,博士房暉遠等莫不推伏之。書侍御史皇甫誕,一時朝彥,恆執弟子之禮,以所乘馬就學邀屈。文詡遂每牽馬步進,意在不因人自致也。右僕射蘇威聞而召之,與語大悅,勸令從官,文詡固辭。仁壽末,學廢,文詡策杖而歸,灌園為業。州郡頻舉,皆不應命。事母以孝聞。每以德化人,鄉黨頗移風俗。嘗有人夜中竊刈其麥者,見而避之。盜因感悟,棄麥而謝。文詡慰諭之,自誓不言,固令持去。經數年,盜者向鄉人說之,始為遠近所悉。鄰家築牆,心有不直,文詡因毀舊堵以應之。文詡
 常有腰疾,會醫者自言善禁,文詡令禁之,遂為刀所傷,至於頓伏床枕。醫者叩頭請罪。文詡遽遣之,因為隱,謂妻子曰:「吾昨風眩,落坑所致。」其掩人短,皆此類也。州縣以其貧素,將加賑恤,輒辭不受。嘗閑居無事,從容歎曰:「老冉冉而將至,恐脩名之不立!」



 以如意擊几自樂,皆有處所,時人方之閔子騫、原憲焉。終於家,鄉人為立碑頌,號曰張先生。



 論曰:古之所謂隱逸者,非伏其身而不見也,非閉其言而不出也,非藏其智而不發也。蓋以恬淡為心,不皦不昧,安時處順,與物無私者也。眭夸忘懷纓冕,畢志丘園,
 或隱不違親,貞不絕俗;或不教而勸,虛往實歸,非有自然純德,其孰能至此?然文詡見傷無慍,徐則志在沈冥,不可親疏,莫能貴賤,皆可謂抱樸之士矣。



 崔廓感於屈辱,遂以肥遁見稱;祖浚文籍之美,足以克隆堂構。父子雖動靜殊方,其於成名一也,美哉!



\end{pinyinscope}