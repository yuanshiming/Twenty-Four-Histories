\article{卷十周本紀下第十}

\begin{pinyinscope}

 高祖武皇帝諱邕,字禰羅突,文帝第四子也。母曰叱奴太后。魏大統九年,生於同州,有神光照室。帝幼而孝敬,聰敏有器質。文帝異
 之曰:「成吾志者,此兒也。」年十二,封輔城郡公。孝閔帝踐阼,拜大將軍,出鎮同州。明帝即位,遷柱國,授蒲州刺史,入為大司空,行御正,進封魯國公,領宗師。甚見親愛,參議朝廷大事。性沉深,有遠識,非因問,終無所言。帝每嘆曰:「夫人不言,言必有中。」



 武成二年四月,帝崩,遣詔傳位於帝。帝固讓,百官勸進,乃從之。壬寅,即皇帝位,大赦。冬十二月,改作路門。是歲,齊孝昭帝廢其主殷而自立。



 保定元年春正月戊申,改元,文武百官各增四級。以大塚宰、晉公護為都督中外諸軍事,令五府總於天官。庚戌,祀圓丘。壬子,祀方丘。甲寅,祀感帝於南
 郊。



 乙卯,祭太社。己巳,享太廟。班文帝所述六官於廟庭。甲戌,板授高年官,各有差。乙亥,親耕籍田。丙子,大射於正武殿,賜百官各有差。二月己卯,遣大使巡察天下風俗。甲午,朝日於東郊。丙午,省輦輿,去百戲。三月丙寅,改八丁兵為十二丁兵,率歲一月役。夏四月丙子朔,日有蝕之。庚寅,以少傅、吳公尉綱為大司空。丁酉,白蘭遣使獻犀甲鐵鎧。五月丙午,封孝閔皇帝子康為紀國公,皇子贇為魯國公。晉公護獲玉斗以獻。六月乙酉,遣御正殷
 不害使於陳。秋七月戊申,以旱故,詔所在降死罪已
 下
 囚。更鑄錢,文曰布泉,以一當五,與五銖並行。九月甲辰,南寧州使獻滇馬及蜀鎧。冬十月甲戌朔,日有蝕之。十一月乙巳,陳人來聘。



 丁巳,狩於岐陽。是月,齊孝昭帝殂。十二月,車駕至自岐陽。是歲,突厥、吐谷渾、高昌、宕昌、龜茲等國並遣使朝貢。



 二年春正月壬寅,初於蒲州開河渠,同州開龍首渠,以廣溉灌。丁未,以陳主弟頊為柱國,送還江南。閏月己亥,大司馬、涼公賀蘭祥薨。二月癸丑,以久不雨,宥罪人。京城三十里內禁酒。梁主蕭察薨。夏四月甲辰,以旱故,禁屠宰。癸亥,詔曰:「諸柱國等勳德隆重,宜有優崇。各準別制,邑戶聽寄食他縣。」五月庚午,以南山眾瑞並集,免今年役及租賦之半。壬辰,以柱國、隨公楊忠為大司空。六月己亥,以柱國、蜀公尉迥為大司馬。分山南荊州、安州、襄州、江陵為四總管。秋九月戊辰朔,日有蝕之。陳人來聘。冬十月辛亥,帝御大武殿大射。戊午,講武於少陵原。
 十一月丁卯,以大將軍衛公直、趙公招並為柱國。



 三年春正月辛未,改光遷國為遷州。乙酉、太保、梁公侯莫陳崇賜死。二月庚子,初頒新律。辛酉,詔自今舉大事,行大政,非軍機急速,皆依月令,以順天心。



 三月乙丑朔,日有蝕之。丙子,宕昌國獻生猛獸二,詔放之南山。夏四月乙未,以柱國、鄭公達奚武為太保,大將軍韓果為柱國。己亥,帝御正武殿錄囚徒。癸卯,大雩。癸丑,有牛足生於背。戊午,幸太學,以太傅、燕公于謹為三老而問道焉。



 初禁天下報仇,犯者以殺人論。壬戌,詔百官及庶人上封事,極言得失。五月甲午朔,以旱故,避正寢,不受朝。甲
 戌,雨。秋七月戊辰,行幸原州。庚午,陳人來聘。丁丑,幸津門,問百年,賜以金帛,又賜高年板職,各有差。降死罪囚一等。



 八月丁未,改作路寢。九月甲子,自原州登隴山。丙戌,幸同州。戊子,詔柱國楊忠率騎一萬與突厥伐齊。己丑,初令世襲州、郡、縣者悉改為五等爵。州封伯,郡封子,縣封男。冬十月庚戌,陳人來聘。十二月辛卯,車駕至自同州。遣太保達奚武率騎三萬出平陽,以應楊忠。是月,有人生子,男而陰在背後,如尾,兩足指如獸爪。有犬生子,腰以後分為二身,兩尾六足。



 四年春正月庚申,楊忠破齊長城,至晉陽而還。二月庚
 寅朔,日有蝕之。三月庚辰,初令百官執笏。夏四月癸卯,以柱國、鄧公竇熾為大宗伯。五月壬戌,封明帝長子賢為畢公。癸酉,以大將軍、安武公李穆為柱國。丁亥,改禮部為司宗,大司禮為禮部,大司樂為樂部。六月庚寅,改御伯為納言。秋七月,焉耆國遣使獻名馬。八月丁亥朔,日有蝕之。詔柱國楊忠帥師與突厥東伐,至北河而還。戊子,以柱國、齊公憲為雍州牧,以許公宇文貴為大司徒。九月丁巳,以柱國、衛公直為大司空。陳人來聘。是月,以皇世母閻氏自齊至,大赦。閏月己亥,以大將軍韋孝寬、長孫儉並為柱國。冬十月癸亥,以大將軍陸通、宇文
 盛、蔡公廣並為柱國。甲子,詔大冢宰、晉公護伐齊,齊於太廟,庭授以斧鉞。於是護總大軍出潼關,大將軍權景宣帥山南諸軍出豫州,少師楊出軹關。丁卯,帝幸沙苑勞師。癸酉,還宮。十一月甲午,柱國尉遲迥圍洛陽,柱國、齊公憲營芒山,晉公護次陜州。十二月丙辰,齊豫州刺史王士良以州降。壬戌,齊師度河,晨至洛陽,諸軍驚散。尉迥帥麾下數十騎扞敵,得卻,至夜引還。柱國王雄力戰,死之。遂班師。楊於軹關戰沒。權景宜亦棄豫州而還。是歲,突厥、粟特等國並遣使朝貢。



 五年春正月甲申朔,以柱國王雄死王事故,廢朝。乙巳,
 以雄世子謙為柱國。



 二月辛酉,詔陳公純等逆皇后于突厥。丙寅,以柱國李穆為大司空,綏德公陸通為大司寇。壬申,行幸岐州。戊子,柱國豆盧寧薨。夏四月,齊武成帝禪位於其太子緯,自稱太上皇帝。五月己亥,左右武伯各置中大夫一人。六月庚申,彗星出三台,入文昌,犯上將,經紫宮入苑,漸長丈餘,百餘日乃滅。辛未,詔江陵人年六十五已上為官奴婢者,已令放免;其公私奴婢年七十以外者,所在官私宜贖為庶人。秋七月辛巳朔,日有蝕之。庚寅,行幸秦州,降死罪已下刑。辛丑,遣大使巡察天下。



 八月丙子,車駕至自秦州。冬十月辛亥,改函
 谷關城為通洛防。十一月丁未,陳人來聘。是歲,吐谷渾遣使朝貢。



 天和元年春正月己卯朔,日有蝕之。辛巳,考路寢,命群臣賦詩。京邑耆老亦會焉,頒賜各有差。癸未,大赦,改元,百官普加四級。己亥,親耕籍田。丁未,於宕昌國置宕州。遣小載師杜果使於陳。二月戊辰,詔三公已下,各舉所知。庚午,日鬥,光遂微,日中見烏。三月丙午,祀南郊。夏四月辛亥,雩。是月,陳文帝殂。



 五月庚辰,帝御正武殿,集群臣,親講《禮記》。吐谷渾龍涸王莫昌率戶內附,以其地為扶州。甲午,詔曰:「甲子、乙卯,《禮》云不樂。萇弘表昆吾之
 稔,杜蕢有揚觶之文。自世道喪亂,禮儀紊毀,此典茫然,已墜於地。宜依是日,有事停樂。



 庶知為君之難,為臣不易,貽之後昆,殷鑒斯在。」六月丙午,以大將軍辛威為柱國。秋七月戊寅,築武功、郿、斜谷、武都、留谷、津坑諸城,以置軍人。壬午,詔諸胄子入學,但束脩於師,不勞釋奠。釋奠者,學成之祭。自今即為恆式。八月己未,詔諸有三年之喪,或負土成墳,或寢苫骨立,一志一行,可稱揚者,本部官司,隨事上言。當加弔勉,以勵薄俗。九月乙亥,信州蠻反,詔開府陸騰討平之。



 冬十月甲子,初造《山雲舞》,以備六代樂。十一月丙戌,行幸武功等城。十二月庚申,還
 宮。



 二年春正月癸酉朔,日有蝕之。己亥,親耕籍田。三月癸酉,改武遊園為道會苑。丁亥,初立郊丘壇壝制度。夏四月乙巳,省併東南諸州。以大將軍、陳公純為柱國。六月辛亥,尊所生叱奴氏為皇太后。閏月庚午,地震。戊寅,陳湘州刺史華皎帥眾來附。壬辰,以大將軍、譙公儉為柱國。秋七月辛丑,梁州上言鳳凰集楓樹,群鳥列侍以萬數。甲辰,立路門學,置生七十二人。壬子,以太傅、燕公于謹為雍州牧。九月,衛公直等與陳將淳于量、吳明徹戰於沌口,王師敗績。元定以步騎數千先度,遂沒江南。冬
 十一月戊戌朔,日有蝕之。癸丑,太保、許國公宇文貴薨。



 是歲,突厥、吐谷渾、安息等國並遣使朝貢。



 三年春正月辛丑,祀南郊。三月癸卯,皇后阿史那氏至自突厥。甲辰,大赦。



 丁未,大會百寮及賓客於路寢。戊午,太傅、燕公于謹薨。夏四月辛巳,以太保達奚武為太傅,大司馬尉迥為太保,柱國、齊公憲為大司馬。五月庚戌,享太廟。六月甲戌,有星孛於東井。秋七月壬寅,柱國、隨公楊忠薨。八月乙丑,韓公元羅薨。



 齊人來聘,請和親,詔軍司馬陸逞報聘。癸酉,帝御大德殿,集百寮及沙門道士等,親講《禮記》。冬十月癸亥,享太廟。丁亥,上親帥六軍,
 講武於城南。京邑觀者,輿馬彌漫數十里,諸蕃使咸在焉。十一月壬辰朔,日有蝕之。壬子,遣開府崔彥穆使於齊。甲寅,陳安成王頊廢其主伯宗而自立。辛未,齊武成帝殂。



 四年春正月辛卯朔,以齊武成殂故,廢朝。遣司會李綸等會葬於齊。二月戊辰,帝御大德殿,集百僚道士沙門等討論釋老。夏四月己巳,齊人來聘。五月己丑,帝製《象經》成,集百寮講說。封魏廣平公子元謙為韓國公,以紹魏後。丁巳,柱國、吳公尉綱薨。六月,築原州及涇州東城。秋七月,突厥遣使獻馬。柱國、昌寧公長孫儉薨。



 五年春三月甲辰,初令宿衛官住關外者,將家累入京,不樂者,解宿衛。夏四月甲寅,以柱國宇文盛為大宗伯。省帥都督官。丙寅,遣大使巡察天下。六月庚子,以皇女生故,降宥罪人,並免逋租懸調。冬十月辛巳朔,日有蝕之。丁酉,太傅、鄭公達奚武薨。十一月丁卯,柱國、幽公廣薨。十二月癸巳,大將軍鄭恪帥師平越巂,置西寧州。是月,齊將斛律光侵邊,於汾北築城,自華谷至龍門。



 六年春正月己酉朔,以路門未成故,廟朝。丁卯,以大將軍王傑、譚公會、鴈門公田弘、魏公李暉等並為柱國。三月己酉,齊公憲自龍門度河,斛律光退保華谷,憲攻拔
 其新築五城。夏四月戊寅朔,日有蝕之。辛卯,信州蠻反,遣大將軍趙訚帥師討平之。庚子,以大將軍司馬消難、侯莫陳瓊、大安公閻慶、神武公竇毅、南陽公叱羅協、平高公侯伏侯龍恩並為柱國。五月癸亥,遣納言鄭詡使於陳。丙寅,以大將軍李昞、中山公訓、杞公亮、上庸公陸騰、安義公宇文丘、北平公寇紹、許公宇文善、犍為公高琳、鄭公達奚震、隴東公楊纂、常山公于翼並為柱國。六月乙未,以大將軍、太原公王秉為柱國。是月,齊將段孝先攻陷汾州。秋七月乙丑,以大將軍、越公盛為柱國。八月癸酉,省掖庭四夷樂、後宮羅綺工五百餘人。冬十月
 壬午,冀公通薨。乙未,遣右武伯谷會琨使於齊。壬寅,上親帥六軍講武於城南。十一月壬子,以大將軍梁公侯莫陳芮、大將軍李意並為柱國。丙辰,齊人來聘。丁巳,行幸散關。十二月己丑,還宮。是冬,牛疫死者十六七。



 建德元年春正月戊午,帝幸玄都觀,親御法座講說,公卿道俗論難,事畢還宮。



 降死罪及流罪一等,其五歲刑已下,並宥之。二月癸酉,遣大將軍、昌城公深使於突厥,司宗李際使於齊。乙酉,柱國、安義公宇文丘薨。三月癸卯朔,日有蝕之。



 齊人來聘。丙辰,誅大冢宰、晉公護及其子柱國、譚公會,並柱國侯伏侯龍恩及其弟大將軍萬
 壽,大將軍劉勇等。大赦,改元。罷中外府。癸亥,以太傅尉迥為太師,柱國竇熾為太傅,大司空李穆為太保,齊公憲為大冢宰,衛公直為大司徒,趙公招為大司空,柱國辛威為大司寇,綏德公陸通為大司馬。詔曰:「人勞不止則星動於天,作事不時則石言於國。頃興造無度,徵發不已;加以頻歲師旅,農畝廢業,去秋災蝗,年穀不登。自今正調以外,無妄徵發。」夏四月甲戌,以代公達、滕肥逌並為柱國。己卯,詔公卿已下,各舉所知。遣工部、代公達使於齊。丙戌,詔百官軍人上封事,極言得失。丁亥,詔斷四方非常貢獻。庚寅,追尊略陽公為孝閔皇帝。



 癸巳,立
 魯公贇為皇太子。大赦,百官各加封級。五月壬戌,以大旱,集百官於庭。



 詔之曰:「亢陽不雨,豈朕德薄,刑賞乖中歟?將公卿大臣,或非其人歟?宜盡直言,無有所隱。」公卿各引咎自責,其夜澍雨。六月庚子,改置宿衛官員。秋七月辛丑,陳人來聘。九月庚子朔,日有蝕之。庚申,扶風掘地得玉盃以獻。冬十月庚午,詔江陵所獲俘虜充官口者,悉免為百姓。辛未,遣小匠師楊勰使於陳。大司馬、綏德公陸通薨。十一月丙午,上親御六軍,講武於城南。庚戌,行幸羌橋,集京城東諸都督以上,頒賜各有差。乙卯,還宮。壬戌,以大司空、趙公招為大司馬。十二月壬申,行
 幸斜谷,集京城西諸都督以上,頒賜有差。丙戌,還宮。己丑,帝御正武殿,親錄囚徒,至夜而罷。庚寅,幸道會苑,以上善殿壯麗,遂焚之。



 二年春正月辛丑,祀南郊。乙巳,以柱國田弘為大司空,大將軍若干鳳為柱國。



 庚戌,復置帥都督官。乙卯,享太廟。閏月己巳,陳人來聘。二月甲寅,詔皇太子贇巡撫西土。壬戌,遣司會侯莫陳凱使於齊。省雍州內八郡,併入京兆、馮翊、扶風、咸陽等郡。三月己卯,皇太子於岐州獲白鹿二以獻。詔答曰:「在德不在瑞。」



 癸巳,省六府諸司中大夫以下官,府置四司,以下大夫為官之長,上士貳之。
 夏四月己亥,享太廟。丙辰,增改東宮官員。五月丁丑,以柱國侯莫陳瓊為大宗伯,滎陽公司馬消難為大司寇,上庸公陸騰為大司空。六月庚子,省六府員外諸官,皆為丞。壬子,皇孫衎生,文武官普加一級大階。大選諸軍將帥。丙辰,帝御路寢,集諸軍將,勖以戎事。庚申,詔諸軍旗旌皆畫以猛獸鷙鳥之象。秋七月己巳,享太廟。



 自春末不雨,至於是月。壬申,集百僚於大德殿,帝責躬罪己,問以時政得失。戊子,雨。八月丙午,改三夫人為三妃。關中大蝗。九月乙丑,陳人來聘。戊寅,詔曰:「頃者婚嫁,競為奢靡,有司宜加宣勒,使遵禮制。」冬十月癸卯,齊人來聘。



 甲辰,奏六代樂成,帝御崇信殿,集百官觀之。十一月辛巳,帝親帥六軍,講武於城東。癸未,集諸軍都督以上五十人於道會苑大射,帝親臨射堂,大備軍容。十二月癸巳,集群官及沙門道士等,帝升高座,辨釋三教先後。以儒教為先,道教次之,佛教為後。以大將軍赫連達為柱國。詔軍人之間,年多耆壽,可頒授老職,使榮沾邑里。戊午,聽訟於正武殿,自旦及夜,繼之以燭。



 三年春正月壬戌,朝群臣於路門。冊柱國齊公憲、衛公直、趙公招、譙公儉、陳公純、越公盛、代公達、滕公逌並進爵為王。己巳,享太廟。庚午,突厥遣使獻馬。癸酉,詔自今
 男年十五,女年十三以上,爰及鰥寡,所在以時嫁娶,務從節儉。



 乙亥,親耕籍田。丙子,初服短衣,享二十四軍督將以下,試以軍旅之法,縱酒盡歡。詔以往歲年穀不登,令公私道俗,凡有貯積粟麥者,皆準口聽留,已外盡糶。



 二月壬辰朔,日有蝕之。丁酉,紀公康、畢公賢、酆公貞、宋公實、漢公贊、秦公贄、曹公允並進爵為王。丙午,令六府各舉賢良清正之士。癸丑,柱國、許公宇文善有罪免。丙辰,大赦。三月癸酉,皇太后叱奴氏崩。帝居倚廬,朝夕共一溢米,群臣表請,累旬乃止。詔皇太子贇總庶政。夏四月乙卯,齊人來弔賵會葬。丁巳,有星孛於東井。五月庚
 申,葬文宣后於永固陵,帝袒跣至陵所。辛酉,詔曰:「齊斬之情,經籍彞訓,近代沿革,遂亡斯禮。伏奉遺令,既葬便除;攀慕几筵,情實未忍。三年之喪,達於天子,古今無易之道,王者之所常行。但時有未諧,不得全制;軍國務重,庶有聽朝。衰麻之節,苫廬之禮,率遵前典,以申罔極。百僚以下,宜遵遺令。」公卿上表固請俯就權制,過葬即吉。帝不許,引古答之。群臣乃止。



 於是遂申三年之制。五服之內,亦令依禮。初置太子諫議,員四人;文學,十人。



 皇子、皇弟友,員各二人;學士,六人。戊辰,詔故晉公護及諸子並追復先封,改葬加謚。丙子,初斷佛、道二教,經像悉毀,
 罷沙門、道士,並令還俗。並禁諸淫祀,非祀典所載者,盡除之。六月丁未,集諸軍將,教以戰陣之法。壬子,更鑄五行大布錢,以一當十,與布泉錢並行。戊午,詔曰:「至道弘深,混成無際;體包空有,理極幽玄。但歧路既分,源流逾遠;淳離朴散,形器斯乖。遂使三墨八儒,朱紫交競;九流七略,畢說相騰。道隱小成,其來舊矣;不有會歸,爭驅靡息。今可立通道觀。聖哲微言,先賢典訓,金科玉篆,秘賾玄文;可以濟養黎元,扶成教義者,並宜弘闡,一以貫之。俾夫玩培塿者識嵩岱之崇崛;守磧礫者悟渤澥之泓澄,不亦可乎。」秋七月庚申,行幸雲陽宮。乙酉,衛王直在
 京反,欲突入蕭章門。司武尉遲運等拒守,直敗,遁走。戊子,車駕至自雲陽宮。八月辛卯,禽直於荊州,免為庶人。冬十月丙申,詔御正楊尚希使於陳。庚子,詔蒲州人遭饑乏絕者,令向郿城以西及荊州管內就食。甲寅,行幸蒲州。乙卯,曲赦蒲州見囚大辟以下。丙辰,行幸同州。十一月戊午,于闐遣使獻名馬。己巳,大閱於同州城東。甲戌,車駕至自同州。十二月戊子,大會衛官及軍人以上,賜錢帛各有差。丙申,改諸軍軍人並名侍官。癸卯,集諸軍講武於臨皋澤。涼州比年地震,壞城郭,地裂涌泉出。



 四年春正月戊辰,初置營軍器監。壬申,布寬大之詔,多
 所蠲免。二月丙戌朔,日有蝕之。辛卯,改置宿衛官員。己酉,柱國、廣德公李意有罪免。三月丙辰,遣小司寇元偉使於齊。郡縣各省主簿一人。甲戌,以柱國、趙王招為雍州牧。夏四月甲午,柱國、燕公于實有罪免。丁酉,初令上書者並為表,於皇太子以下稱啟。秋七月己未,禁五行大布錢不得出入關,布泉錢聽入而不聽出。甲戌,陳人來聘。丙子,召大將軍以上於大德殿,帝親諭以伐齊之旨。言往以政出權宰,無所措懷,自親覽萬機,便圖東討。惡衣菲食,繕甲練兵,數年以來,戰備稍足。而偽主昏虐,恣行無道,伐暴除亂,斯實其時。群臣咸稱善。丁丑,下詔
 暴齊氏過惡。以柱國、陳王純為前一軍總管,滎陽公司馬消難為前二軍總管,鄭公達奚震為前三軍總管,越王盛為後一軍總管,周昌公侯莫陳瓊為後二軍總管,越王招為後三軍總管,齊王憲帥眾二萬趣黎陽,隨公楊堅、廣寧公侯莫陳迴師三萬自渭入河,柱國、梁公侯莫陳芮帥眾一萬守太行道。申國公李穆帥眾三萬守河陽道,常山公于翼帥眾二萬出陳、汝。壬午,上親帥六軍眾六萬,直指河陰。八月癸卯,入齊境,禁伐樹殘苗稼,犯者以軍法從事。丁未,上親帥諸軍,攻拔河陰大城。攻子城未剋,上有疾。九月辛酉夜,班師,水軍焚舟而退。
 齊王憲、于翼、李穆等所在剋捷,降拔三十餘城,皆棄而不守。唯以王藥城要害,令儀同三司韓正守之。正尋以城降齊。戊寅,至自東伐。冬十月戊子,初置上柱國、上大將軍官,改開府儀同三司為開府儀同大將軍,又置上開府、上儀同官。閏月,以柱國齊王憲、蜀公尉遲迥為上柱國。詔諸畿郡各舉賢良。十一月己亥,改置司內官員。十二月辛亥朔,日有蝕之。丙子,陳人來聘。



 是歲,岐、寧二州人饑,開倉振恤。



 五年春正月辛卯,行幸河東涑川,集關中河東諸軍校獵。甲午,還同州。丁酉,詔分遣大使,周省四方,察訟聽謠,問
 人恤隱。廢布泉錢。戊申,初令鑄錢者至絞,從者遠配。二月辛酉,遣皇太子贇巡撫西土,仍討吐谷渾。三月壬寅,車駕至自同州。文宣皇太后服再期。戊申,祥。夏六月戊申朔,日有蝕之。辛亥,享太廟。丙辰,利州總管、紀王康有罪,賜死。秋七月乙未,京師旱。八月戊申,皇太子入吐谷渾,至伏俟城而還。乙丑,陳人來聘。九月丁丑,大醮於正武殿,以祈東伐。冬十月,帝復諭群臣伐齊。以去歲屬有疹疾,遂不得剋平逋寇。于時出軍河外,直為撫背,未扼其喉。然晉州本高歡所起,統攝要重,今往攻之,彼必來援,嚴軍以待,擊之必剋。然後乘破竹之勢,鼓行而東,足
 以窮其窟穴。諸將多不願行。帝曰:「機者事之微,不可失矣。沮軍事者,以軍法裁之。」己酉,帝總戎東伐,以越王盛為右一軍總管,杞公亮為右二軍總管,隋公楊堅為右三軍總管;譙王儉為左一軍總管,大將軍竇泰為左二軍總管,廣化公丘崇為左三軍總管,齊王憲、陳王純為前軍。癸亥,帝至晉州,遣齊王憲帥精騎二萬守雀鼠谷,陳王純步騎二萬守千里徑,鄭公達奚震步騎一萬守統軍川,大將軍韓明步兵五千守齊子嶺,烏氏公尹升步騎五千守鼓鐘鎮,涼城公辛韶步騎五千守蒲津關,柱國趙王招步騎一萬自華谷攻汾州諸城,柱國宇文
 盛步兵一萬守汾水關,遣內史王誼監六軍攻晉州城。帝屯於汾曲。齊王憲攻洪洞、永安二城並拔之。是夜,虹見於晉州城上,首向南,尾入紫宮。帝每日自汾曲赴城下,親督戰。庚午,齊行臺左丞侯子欽出降。壬申,齊晉州刺史崔嵩夜密使送款,上開府王軌應之。未明登城,遂剋晉州。甲戌,以上開府梁士彥為晉州刺史以鎮之。十一月己卯,齊主自並州帥眾來援,帝以其兵新集,且避之,乃詔諸軍班師。齊主逐圍晉州。齊王憲屯諸軍於涑水為晉州聲援。河東地震。癸巳,至自東伐,獻俘于太廟。丙申,放齊諸城鎮降人還。丁酉,帝發京師。壬寅,度河,與
 諸軍合。十二月戊申,次晉州。庚戌,帝帥諸軍八萬,置陣東西二十餘里;乘常御馬,從數人巡陣。所至輒呼主帥姓名以慰勉之,將士感見知之恩,各思自厲。將戰,有司請換馬,帝曰:「朕獨乘良馬何所之?」齊主亦於塹北列陣。申後,齊人填塹南引,帝大喜,勒諸軍擊之,齊人便退。齊主與其麾下數十騎走還並州。齊眾大潰,軍資甲仗數百里間委棄山積。辛亥,帝幸晉州,仍率諸軍追齊主。諸將固請還師,帝曰:「縱敵患生。卿等若疑,朕將獨往。」諸將不敢言。甲寅,齊主遣其丞相高阿那肱守高壁,帝麾軍直進,那肱望風退散。丙辰,師次介休,齊將韓建業舉城
 降,以為上柱國,封郇國公。丁巳,大軍次並州。齊主留其從兄安德王延宗守並州,自將輕騎走鄴。是日,詔齊王公以下,示以逆順之道,於是齊將帥降者相繼。



 戊午,高延宗僭即偽位,改年曰德昌。己未,軍次并州。帝帥諸軍合戰,齊人退,帝遂北入城東門,諸軍繞城置陣。至夜,延宗帥其眾排陣而前,城中軍卻,人相蹂踐,大為延宗所敗。齊人欲閉門,以閫下積尸,扉不得闔,帝從數騎,崎嶇危險,僅得出門。至明,帥諸軍更戰,大破之,禽延宗,並州平。壬戌,詔曰:昔天厭水運,龍戰於野,兩京否隔,四紀於茲。朕垂拱巖廊,君臨宇縣;相邠人於海內,混楚弓於天
 下。一物失所,有若推溝。方欲德綏未服,義征不譓。偽主高緯,放命燕、齊,怠慢典刑,俶擾天紀。加以背惠怒鄰,棄信忘義。朕應天從物,伐罪弔人;一鼓而蕩平陽,再舉而摧強敵。偽署王公,相繼道左。高緯智窮數屈,逃竄草間。偽安德王高延宗,擾攘之間,遂竊名號,與偽齊昌王莫多婁敬顯等,收合餘燼,背城借一。王威既振,魚潰鳥離;破竹更難,建瓴非易。延宗眾散,衿甲軍門。根本既傾,枝葉自隕。幽青海岱,折簡而來;冀北河南,傳檄可定。八紘共貫,六合同風。方當偃伯靈臺,休牛桃塞,無疆之慶,非獨在予。漢皇約法,除其茍政,姬王輕典,刑彼新邦。思覃
 惠澤,被之率土;新集臣庶,皆從蕩滌。可大赦天下。高緯及王公以下,若釋然歸順,咸許自新。諸亡入偽朝,亦從寬宥。官榮次序,依例無失。齊制偽令,即宜削除。鄒、魯搢紳,幽、並騎士,一介可稱,並宜銓錄。



 丙寅,出齊宮中金銀寶器珠玉麗服及宮女二千人,班賜將士。以柱國趙王招、陳王純、越王盛、杞公亮、梁公侯莫陳芮、庸公王謙、北平公寇紹、鄭公達奚震並為上柱國,封齊王憲子安城郡公質為河間王。諸有功者封授各有差。癸酉,帝帥六軍趣鄴。



 六年春正月乙亥,齊主傳位於其太子恒,改年曰承光,
 自號太上皇。壬辰,帝至鄴。癸巳,帥諸軍圍之,齊人拒守,諸軍奮擊,大破之,遂平。齊主先送其母及妻子於青州,及城陷,帥數十騎走青州,遣大將軍尉勤追之。是戰也,於陣獲其齊昌王莫多婁敬顯,帝數之曰:「汝有死罪三:前從並州走鄴,棄母攜妻妾,是不孝;外為偽主戮力,內實通啟於朕,是不忠;送款之後,猶持兩端,是不信。如此用懷,不死何待。」遂斬之。是日,西方有聲如雷。甲午,帝入鄴城。詔去年大赦班宣未及之處,皆從赦例。己亥,詔曰:「晉州大陣至鄴,身殞戰場者,其子即授父本官。」



 尉勤禽齊主及其太子恒於青州。庚子,詔曰:「偽齊之末,姦佞擅
 權,濫罰淫刑,動挂羅網。偽右丞相咸陽王故斛律明月、偽待中特進開府故崔季舒等七人,或功高獲罪,或直言見誅。朕兵以義動,翦除凶暴,表閭封墓,事切下車。宜追贈謚,並加窆措。其見在子孫,各隨蔭敘錄。家口田宅沒官者,並還之。」辛丑,詔偽齊東山、南園及三臺,並毀撤。瓦木諸物凡入用者,盡賜百姓。山園之田,各還本主。



 二月丙午,論定諸軍勳,置酒於齊太極殿,會軍士以上,班賜有差。丁未,齊主至,帝降自阼階,見以賓主禮。齊任城王湝在冀州,擁兵未下,遣上柱國、齊王憲與柱國、隋公楊堅討平之。齊范陽王高紹義叛入突厥。齊諸行臺州
 鎮悉降,關東平。合州五十五,郡一百六十二,縣三百八十五,戶三百三十萬二千五百八十八,口二千萬六千八百八十六。乃於河陽及幽、青、南兗、豫、徐、北朔、定州置官管府。相、並二總管,各置官及六府官。癸丑,詔自偽武平三年以來,河南諸州人,偽齊破掠為奴婢者,不問公私,並放免之。其住在淮南者,亦即聽還;願住淮北者,可隨便安置。癃疾孤老不能自存者,所在矜恤。乙卯,車駕發自鄴。三月壬午,詔山東諸州各舉士。夏四月乙巳,至自東伐。列齊主於前,其王公等並從,車輿旌旗及器物以次陳於其後。大駕布六軍,備凱樂,獻俘於太廟。京邑觀者,
 皆稱萬歲。戊申,封齊主為溫國公。庚戌,大會群臣及諸蕃客於路寢。乙卯,廢蒲、陜、涇、寧四州總管。己巳,享太廟。詔分遣使人,巡方撫慰,觀風省俗。五月丁丑,以柱國、譙王儉為大塚宰。庚辰,以上柱國、杞公亮為大司徒,鄭公達奚震為大宗伯,梁公侯莫陳芮為大司馬,柱國、應公獨孤永業為大司寇,鄖公韋孝寬為大司空。辛巳,大醮於正武殿,以報功。己丑,祀方丘。詔曰:「往者,冢臣專任,制度有違,正殿別寢,事窮壯麗。非直彫牆峻宇,深戒前王;而締構弘敞,有踰清廟;不軌不物,何以示後。兼東夏初平,人未見德;率先海內,宜自朕始。其路寢、會義、崇信、含
 仁、雲和、思齊諸殿等,農隙之時,悉可毀撤。彫斲之物,並賜貧人。繕造之宜,務從卑朴。」戊戌,詔曰:「京師宮殿,已從撤毀;並、鄴二所,華侈過度,誠復作之非我,豈容因而弗革。諸堂殿壯麗,並宜除蕩;甍宇雜物,分賜窮人。三農之隙,別漸營構,止蔽風雨,務在卑狹。」庚子,陳人來聘。是月,青城門無故自崩。



 六月辛亥,御正武殿錄囚徒。甲子,東巡。丁卯,詔曰:「自今不得娶母同姓以為妻妾。」秋七月丙戌,行幸洛州。己丑,詔山東諸州,舉有才望者赴行在所,共論政事得失。八月壬寅,議權衡度量,頒於天下。其不依新式者,悉追停之。詔曰:「以刑止刑,以輕代重,罪不及
 嗣,皆有定科。雜役之徒,獨異常憲,一從罪配,百代不免。罰既無窮,刑何以措?凡諸雜戶,悉放為百姓。配雜之科,因之永削。」



 甲子,鄭州獻九尾狐,皮肉銷盡,骨體猶具。帝曰:「瑞應之來,必昭有德。若使五品時序,州海和平,家識孝慈,乃能致此。今無其時,恐非實錄。」乃令焚之。



 九月壬申,以柱國鄧公竇熾、申公李穆為上柱國。戊寅,初令庶人以上,非朝祭之服,唯得衣綢、綿綢、絲布、圓綾、紗、絹、綃、葛、布等九種。壬辰,詔東土諸州儒生,明一經以上,並舉送,州郡以禮發遣。冬十月戊申,行幸鄴宮。戊午,改葬德皇帝於冀州,帝服緦,哭於太極殿,百官素服哭。是月,誅
 溫公高緯。十一月壬申,封皇子充為道王,兌為蔡王。癸酉,陳將吳明徹侵呂梁,徐州總管梁士彥與戰不利,退守徐州。遣上大將軍、郯公王軌討之。是月,稽胡反,遣齊王憲討平之。



 詔自永熙三年七月以來,十月以前,東土人被鈔在化內為奴婢者;及平江陵日,良人沒為奴婢者,並免同人伍。詔曰:「正位於中,有聖通典,質文相革,損益不同。



 五帝則四星之象,三王制六宮之數。劉、曹已降,等列彌繁,選擇偏於生靈,命秩方於庶職,椒房丹地,有眾如雲,本由嗜欲之情,非關風化之義。朕運當澆季,思復古始,弘贊後庭,事從簡約。可置妃二人,世婦三人,御
 妻三人。自茲以外,宜悉減省。」己亥晦,日有蝕之。初行《刑書要制》。持杖群強盜一疋以上,不持杖群強盜五疋以上,監臨主掌自盜二十疋以上,小盜及詐請官物三十疋以上,正長隱五戶及十丁以上、隱地三頃以上,皆至死。《刑書》所不載者,自依律科。十二月,北營州刺史高寶寧據州反。庚申,行幸并州宮。移並州軍人四萬戶於關中。戊辰,廢並州宮及六府。是歲,吐谷渾、百濟並遣使朝貢。



 宣政元年春正月癸酉,吐谷渾偽趙王他婁屯來降。壬午,行幸鄴宮。辛卯,幸懷州。癸巳,幸洛州。詔於懷州置宮。
 二月甲辰,柱國、大塚宰、譙王儉薨。丁巳,車駕至自東巡。乙丑,以上柱國、越王盛為大冢宰,陳王純為雍州牧。三月戊辰,於蒲州置宮,廢同州及長春二宮。壬申,突厥遣使朝貢。甲戌,初服常冠,以皁紗為之,加簪而不施纓導,其制若今之折角巾也。上大將軍王軌破陳師於呂梁,禽其將吳明徹等,俘斬三萬餘人。丁亥,詔柱國豆盧寧征江南武陵、南平等郡所有士庶為人奴婢者,悉依江陵放免。壬辰,改元。夏四月壬子,初令遭父母喪者,聽終制。



 庚申,突厥入寇幽州。五月己丑,帝總戎北伐,遣柱國原公姬願、東平公宇文神舉等五道俱入。發關中公私
 馬驢悉從軍。癸巳,帝不豫,止于雲陽宮。丙申,詔停諸軍。六月丁酉,帝疾甚,還京。其夜崩於乘輿,時年三十六。遺詔曰:人肖形天地,稟質五常;修短之期,莫非命也。朕君臨宇縣,十有九年,未能使百姓安樂,刑措不用。未旦求衣,分宵忘寢。昔魏室將季,海內分崩;太祖扶危翼傾,肇開王業。燕、趙榛蕪,又竊名號。朕上述先志,下順人心,遂與王公將帥,共平東夏。雖復妖氛蕩定,而人勞未康,每一念如此,若臨冰谷。將欲包舉六合,混同文軌。今遘疾大漸,力氣稍微,有志不申,以此歎息。天下事重,萬機不易;王公以下,爰及庶寮,宜輔導太子,副朕遺意;令上不
 負太祖,下無失為臣。朕雖瞑目九泉,無所復恨。朕平生居處,每存菲薄;非直以訓子孫,亦乃本心所好。喪事資用,須使儉而合禮。墓而不墳,自古通典。隨吉即葬,葬訖公除。四方士庶,各三日哭。妃嬪以下無子者,悉放還家。



 謚曰武皇帝,廟稱高祖。己未,葬於孝陵。帝沉毅有智謀,初以晉公護專權,常自晦迹,人莫測其深淺。及誅護之後,始親萬機,剋己勵精,聽覽不怠。用法嚴整,多所罪殺。號令懇惻,唯屬意於政。群下畏服,莫不肅然。性既明察,少於恩惠;凡布懷立行,皆欲踰越古人。身衣布袍,寢布被,無金寶之飾。諸宮殿華綺者,皆撤毀之,改為土階數
 尺,不施櫨栱。其彫文刻鏤,錦繡纂組,一皆禁斷。後宮嬪御,不過十餘人。勞謙接下,自強不息。以海內未康,銳情教習,至於校兵閱武,步行山谷,履涉勤苦,皆人所不堪。平齊之役,見軍士有跣行者,帝親脫靴以賜之。



 每晏會將士,必自執盃勸酒,或手付賜物。至於征伐之處,躬在行陣。性又果決,能斷大事,故能得士卒死力,以弱制強。破齊之後,遂欲窮兵極武。平突厥、定江南,一二年間,必使天下一統,此其志也。



 宣皇帝諱贇,字乾伯,武帝長子也。母曰李太后。武成元年,生於同州。保定元年五月丙午,封魯國公。建德元年
 四月癸巳,武帝親告廟,冠於阼階,立為皇太子。二年,詔皇太子巡撫西土。文宣后崩,武帝諒闇,詔太子總朝政,五旬而罷。



 武帝每巡幸四方,太子常留監國。五年二月,又詔太子巡西土,因討吐谷渾。



 宣政元年六月丁酉,武帝崩,戊戌,太子即皇帝位。尊皇后曰皇太后。甲子,誅上柱國、齊王憲。閏月乙亥,詔山東流人新復業,及突厥侵掠家口破亡不能存濟者,給復一年。立妃楊氏為皇后。辛巳,以上柱國、趙王招為太師,陳王純為太傅,柱國、代王達、滕王逌、盧公尉遲運、薛公長孫覽並為上柱國。是月,幽州盧昌期據范陽反,詔柱國、東平公宇文神舉討
 平之。秋七月乙巳,享太廟。丙午,祀圓丘。



 戊申,祀方澤。庚戌,以小宗伯、岐公斛斯徵為大宗伯。壬戌,以南兗州總管、隋公楊堅為上柱國、大司馬。癸亥,尊所生李氏為帝太后。八月丙寅,夕月於西郊。



 長安、萬年二縣人居京城者,給復三年。壬申,幸同州。遣大使『巡察諸州。制九條,宣下州郡。其母族絕服外者,聽婚。以上柱國、薛公長孫覽為大司徒,柱國、楊公王誼為大司空。丙戌,以柱國、永昌公椿為大司寇。九月丁酉,以柱國宇文盛、張掖公王傑、枹罕公辛威、鄖國公韋孝寬並為上柱國。庚戌,封皇弟元為荊王。詔諸應拜者,皆以三拜成禮。冬十月癸酉,至
 自同州。戊子,百濟遣使朝貢。十一月己亥,講武於道會苑,帝親擐甲。是月,突厥犯邊,圍酒泉,殺掠吏士。十二月甲子,以柱國、畢王賢為大司空。己丑,以上柱國、河陽總管、滕王逌為行軍元帥,伐陳。免京師見徒,並令從軍。



 大象元年春正月乙丑,受朝於路門,帝服通天冠、絳紗袍,群臣皆服漢魏衣冠。



 大赦,改元為大成。初置四輔官,以大冢宰、越王盛為大前疑,蜀公尉遲迥為大右弼,申公李穆為大左輔,大司馬隋公楊堅為大後丞。癸卯,封皇子衍為魯王。甲辰,東巡。丙午,以柱國、常山公於翼為大司徒。辛亥,以柱國、許公宇文善為大宗伯。



 戊午,行幸
 洛陽。立魯王衍為皇太子。二月癸亥,詔曰:「河、洛之地,舊稱朝市,自魏氏失馭,城闕為墟。我太祖受命酆、鎬,有懷光宅;高祖往巡東夏,布政此宮。



 朕以眇身,祗承寶運,雖庶幾聿修之志,敢忘燕翼之心。一昨駐蹕金墉,備嘗游覽。



 百王制度,基址尚存。今若因循,為功易立。宜命邦事,修復舊都。奢儉取文質之間,功役依子來之義。北瞻河內,咫尺非遙,前詔經營,今宜停罷。」於是發山東諸州兵,增一月功為四十五日役,起洛陽宮。常役四萬人,以迄晏駕。並移相州六府於洛陽,稱東京六府。殺柱國、徐州總管、郯公王軌。停南討諸軍。以趙王招女為千金公主,
 嫁於突厥。乙亥,行幸鄴。丙子,初令總管、刺史行兵者加持節,餘悉罷之。辛巳,詔傳位於皇太子衍。大赦,改元,大成為大象。帝於是自稱天元皇帝,所居稱天臺,冕二十有四旒,車服旗鼓皆以二十四為節。內史、御正皆置上大夫。皇帝衍稱正陽宮。置納言、御正、諸衛等官,皆准天臺。尊皇太后為天元皇太后。癸未,日出、將入時,其中並有烏色,大如雞卵,經四日乃滅。戊子,以大前疑、越王盛為太傅,大右弼、蜀公尉迥為大前疑,代王達為大右弼。辛卯,詔徙鄴城石經於洛陽。又詔洛陽凡是元遷戶,並聽還洛州。此外欲往者,聽之。河陽、幽、相、豫、亳、青、徐七總
 管受東京六府處分。三月庚申,車駕至自東巡,大陳軍伍,親擐甲胄,入自青門。皇帝衍備法駕從,百官迎於青門外。是時驟雨,儀衛失容。



 辛酉,封趙王招第二子貫為永康縣王。夏四月壬戌朔,有司奏言日蝕,不視事。過時不蝕,乃臨軒。立妃朱氏為天元帝后。癸亥,以柱國、畢王賢為上柱國。己巳,享太廟。壬午,大醮於正武殿。五月辛亥,以洛州襄國郡為趙國,齊州濟南郡為陳國,豐州武當、安富二郡為越國,潞州上黨郡為代國,荊州新野郡為滕國,邑各一萬戶。令趙王招、陳王純、越王盛、代王達、滕王逌並之國。是月,遣使簡視京城及諸州士庶女,充
 選後宮。突厥寇並州。六月,咸陽有池水變為血。征山東諸州人修長城。秋七月庚寅,以大司空、畢王賢為雍州牧,大後丞、隋公楊堅為大前疑,柱國、滎陽公司馬消難為大後丞。丙申,納大後丞司馬消難女為正陽宮皇后。己酉,尊帝太后李氏為天皇太后。壬子,改天元帝后朱氏為天皇后,立妃元氏為天右皇后,妃陳氏為天左皇后。八月庚申,幸同州。壬申,還宮。甲戌,以天左皇后父大將軍陳山提、天右皇后父開府元晟並為上柱國。初,武帝作《刑書要制》,用法嚴重。



 及帝即位,恐物情未附,除之。至是,為《刑經聖制》,其法深刻,大醮於正武殿,告天而行
 焉。壬午,以上柱國、雍州牧畢王賢為太師,上柱國、郇公韓建業為大左輔。是月,所在蟻群鬥,各四五尺,死者十八九。九月己卯,以酆王貞為大塚宰。



 上柱國、鄖公韋孝寬為行軍元帥,率行軍總管杞公亮、郕公梁士彥伐陳。遣御正杜果使於陳。冬十月壬戌,幸道會苑,大醮,以高祖武皇帝配醮。初復佛象及天尊象,帝與二象俱南坐。大陳雜戲,令京城士庶縱觀。是月,相州人段德舉謀反,伏誅。



 十一月乙未夜,行幸同州。壬寅,還宮。丁巳,初鑄永通萬國錢,以一當千,與五行大布並行。是月,韋孝寬拔壽陽,杞國公亮拔黃城,梁士彥拔廣陵。陳人退走,於是
 江北盡平。十二月戊午,以災異屢見,帝御路寢,見百官。詔曰:「朕以寡德,君臨區宇。始於秋季,及此玄冬,幽憂殷勤,屢貽深戒。至有金入南斗,木犯軒轅;熒惑乾房,又與土合;流星照夜,東南而下。然則南斗主於爵祿,軒轅為於後宮,房曰明堂,布政所也。火、土則憂孽之兆,流星乃兵凶之驗。豈其宮人失序,女謁尚行,政事乖方,憂患將至,何其昭著,若斯之甚。將避正寢,齋居克念;惡衣減膳,去飾徹懸;披不諱之誠,開直言之路。欲使刑不濫及,賞弗踰等,選舉以才,宮闈修德。宜宣諸內外,庶盡弼諧;允葉人心,用消天譴。」於是舍仗衛,往天興宮。百官上表,勸
 復寢膳,許之。甲子,還宮,御正武殿。集百官及宮人、內外命婦,大列妓樂;又縱胡人乞寒,用水澆沃以為戲樂。乙丑,行幸洛陽。帝親御驛馬,日行三百里。四皇后及文武侍衛數百人,並乘驛以從。令四后方駕齊驅,或有先後,便加譴責。人馬頓仆,相屬於道。己卯,還宮。



 二年春正月丁亥,帝受朝于道會苑。癸巳,享太廟。乙巳,造二扆,畫日月象以置左右。戊申,雨雪。雪止又雨細黃土,移時乃息。乙卯,詔江右諸州新附人,給復二十年。初稅入市者,人一錢。二月丁巳,帝幸路門學,行釋奠禮。戊午,突厥遣使獻方物,且逆千金公主。乙丑,改制詔為天
 制,敕為天敕。尊天元皇太后為天元上皇太后,天皇太后李氏曰天元聖皇太后。癸未,立天元皇后楊氏為天元大皇后,天皇后朱氏為天大皇后,天右皇后元氏為天右大皇后,天左皇后陳氏為天左大皇后,正陽宮皇後直稱皇后。是月,洛陽有禿鶖鳥集新太極殿前,滎州有黑龍見,與赤龍鬥於汴水側,黑龍死。三月丁亥,賜百官及百姓大酺。詔進封孔子為鄒國公,邑數準舊,並立後承襲,別於京師置廟,以時祭享。戊子,行軍總管、杞公亮舉兵反,行軍元帥韋孝寬獲而殺之。辛卯,行幸同州。增候正,前驅式道,為三百六十重,自應門至赤岸澤,數
 十里間,幡旗相蔽,鼓樂俱作。又令武賁持鈒馬上,稱警蹕,以至同州。乙未,改同州宮為天成宮。庚子,車駕至自同州。詔天臺侍衛,皆著五色及紅紫綠衣,以雜色緣,名曰品色衣;有大事,與公服間服之。壬寅,詔內外命婦皆執笏,其拜宗廟及天臺,皆俯伏。甲辰,初置天中大皇后,立天左大皇后陳氏為天中大皇后,立妃尉遲氏為天左大皇后。夏四月己巳,享太廟。己卯,以旱故,降見囚死罪已下。壬午,幸仲山祈雨,至咸陽宮,雨降。甲申,還宮。令京城士女於衢巷作音樂以迎候。五月甲午,帝備法駕幸天興宮。乙未,帝不悆,還宮。



 詔揚州總管、隋公楊堅入
 侍疾。丁未,追趙、越、陳、代、滕五王入朝。己酉,大漸。御正下大夫劉昉與內史上大夫鄭譯矯制以隋公楊堅受遺輔政。是日,帝崩於天德殿,時年二十二。謚宣皇帝。七月丙申,葬定陵。



 帝之在東宮也,武帝慮其不堪承嗣,遇之甚嚴。朝見進止,與諸臣無異;雖隆寒盛署,亦不得休息。性嗜酒,武帝遂禁醪醴不許至東宮。帝每有過,輒加捶撲。



 嘗謂之曰:「古來太子被廢者幾人,餘兒豈不堪立邪!」於是遣東宮官屬錄帝言語動作,每月奏聞。帝懼威嚴,矯情修飾,以是惡不外聞。嗣位之初,方逞其欲。大行在殯,曾無戚容,即通亂先帝宮人。纔踰年,便恣聲樂,采擇
 天下子女,以充後宮。好自矜夸,飾非拒諫。禪位之後,彌復驕奢。耽酗於後宮,或旬日不出,公卿近臣請事者,皆附閹官奏之。所居宮殿,帷帳皆飾以金玉珠寶,光華炫耀,極麗窮奢。及營洛陽宮,雖未成畢,其規摹壯麗,踰於漢、魏遠矣。唯自尊崇,無所顧憚。



 國典朝儀,率情變改。後宮位號,莫難詳錄。每對臣下,自稱為天。以五色土塗所御天德殿,各隨方色。又於後宮,與皇后等列坐,用宗廟禮器罇彞珪瓚之屬。以次食焉。又令群臣朝天臺者,致齋三日,清身一日。車旗章服,倍於前王之數。既自比上帝,不欲令人同己。常自帶授及冠通天冠,加金附蟬,顧
 見侍臣武弁上有金蟬,及王公有綬者,並令去之。又不聽人有高者大者之稱,諸姓高者改為姜,九族稱高祖者為長祖,曾為次長祖。官稱名位,凡謂上及大者,改為長;有天者,亦改之。



 又令天下車皆渾成為輪,禁天下婦人皆不得施粉黛,唯宮人得乘有輻車,加粉黛焉。



 西陽公溫,杞公亮之子,即帝從祖兄子也。其妻尉遲氏有容色,因入朝,帝遂飲以酒,逼而淫之。亮聞之懼,謀反。纔誅溫,即追尉遲氏入宮,初為妃,尋立為皇后。



 每左右侍臣論議,唯欲興造革易,未嘗言及政事。其後遊戲無恒,出入不節;羽儀仗衛,晨出夜還;或幸天興宮,或遊道會苑,陪侍
 之官,皆不堪命。散樂雜戲,魚龍爛漫之伎,常在目前。好令京城少年為婦人服飾,入殿歌舞,與後宮觀之,以為喜樂。擯斥近臣,多所猜怨。又吝於財,略無賜與。恐群臣規諫,不得行己之志。



 常遣左右密伺察之,動止所為,莫不抄錄;小有乖違,輒加其罪。自公卿以下,皆被楚撻。其間誅戮黜免者,不可勝言。每捶人皆以百二十為度,名曰天杖。宮人內職亦如之。后妃嬪御,雖被寵嬖,亦多被杖背。於是內外恐懼,人不自安;皆求茍免,莫有固志;重足累息,以逮於終矣。



 靜皇帝諱衍,後改名闡,宣帝之長子也。母曰朱皇后。建
 德二年六月,生于東宮。大象元年正月癸卯,封魯王。戊午,立為皇太子。二月辛巳,宣帝於鄴宮傳位授帝,居正陽宮。



 二年五月乙未,宣帝寢疾,詔帝入宿路門學。己酉,宣帝崩,帝入居天臺,廢正陽宮。大赦,停洛陽宮作。庚戌,上天元上皇太后尊號為太皇太后,天元聖皇太后李氏為太帝太后,天元大皇后為皇太后,天大皇后朱氏為帝太后。其天中大皇后陳氏、天右大皇后元氏、天左大皇后尉遲氏並出俗為尼。以柱國、漢王贊為上柱國、右大丞相,上柱國、揚州總管、隋公楊堅為假黃鉞左大丞相,柱國、秦王贄為上柱國。帝居諒娼,百官總己以聽
 於左大丞相。壬子,以上柱國、鄖公韋孝寬為相州總管。罷入市稅錢。六月戊午,以柱國許公宇文善、神武公竇毅、修武公侯莫陳瓊、大安公閻慶並為上柱國。趙王招、陳王純、越王盛、代王達、滕王悄來朝。庚申,復佛、道二教。辛酉,以柱國巳公椿、燕公于實、郜公賀拔伏恩並為上柱國。甲子,相州總管尉遲迥舉兵不受代,詔發關中兵,即以韋孝寬為行軍元帥,討之。上柱國、畢王賢以謀執政,被誅。以上柱國、秦王贄為大塚宰,巳公椿為大司徒。



 己巳,詔南定、北光、衡、巴四州人為宇文亮抑為奴婢者,並免之。甲戌,有赤氣起西方,漸東行,遍天。庚辰,罷諸魚池
 及山澤公禁者,與百姓共之。以柱國、蔣公梁睿為益州總管。秋七月甲申,突厥送齊范陽王高紹義。庚寅,申州刺史李惠起兵。庚子,詔趙、陳、越、代、滕五王,入朝不趨,劍履上殿。滎州刺史、邵公宇文胄舉兵,遣大將軍楊素討之。青州總管尉遲綱舉兵。丁未,隋公楊堅都督內外諸軍事。己酉,鄖州總管司馬消難舉兵,以柱國、楊公王誼為行軍元帥討之。壬子,趙王招、越王盛以謀執政,被誅。癸丑,封皇弟潔為萊王,術為郢王。是月,豫州、襄州總管諸蠻,各帥種落反。八月庚申,益州總管王謙舉兵不受代,即以梁睿為行軍元帥討之。庚午,韋孝寬破尉迥於
 鄴,迥自殺,相州平。移相州於安陽,其鄴城及邑,毀廢之。丙子,以漢王贊為太師,以上柱國、並州總管、申公李穆為太傅,以宋王實為大前疑,以秦王贄為大右弼,以燕公于實為大左輔。己卯,以尉迥平,大赦。庚辰,司馬消難擁眾以魯山、甑山二鎮奔陳。遣大將軍元景山追擊之,鄖州平。沙州氐帥開府楊永安聚眾反,應王謙,遣大將軍達奚儒討之。楊素破宇文胄於滎陽,斬之。以上柱國、神武公竇毅為大司馬,以齊公于智為大司空。廢相、青、荊、金、晉、梁州六總管。九月丙戌,廢河陽總管為鎮,隸洛州。以小宗伯、竟陵公楊慧為大宗伯。壬辰,廢皇后司馬
 氏為庶人。戊戌,以柱國、楊公王誼為上柱國。



 庚戌,以柱國常山公于翼、化政公宇文忻並為上柱國。壬子,丞相去左右號,隋公楊堅為大丞相。冬十月甲寅,日有蝕之。壬戌,陳王純以怨執政,被誅。大丞相、隋公楊堅加大冢宰,五府總於天官。戊寅,梁睿破王謙,斬之。傳首京師,益州平十一月甲辰,達奚儒破楊永安,沙州平。丁未,上柱國、鄖公韋孝寬薨。十二月壬子,以柱國、蔣公梁睿為上柱國。丁巳,以柱國千阜公楊雄、普安公賀蘭謨、郕公梁士彥、上大將軍新寧公叱列長文、武鄉公崔弘度、大將軍中山公宇文恩、濮陽公宇文述、渭原公和乾子、任城公
 王景、漁陽公楊銳、上開府廣宗公李崇、隴西公李詢並為上柱國。庚申,以柱國、楚公豆盧績為上柱國。癸亥,詔曰:「太祖受命,龍德猶潛;三分天下,志扶魏室;多所改作,冀允上玄。文武群官,賜姓者眾,本殊國邑,實乖胙土。不歆非類,異骨肉而共蒸嘗。不愛其親,嗟行路而敘昭穆。且神徵革姓,本為歷數有歸;天命在人,推讓終而弗獲。故君臨區宇,累世於茲;不可仍遵謙挹之旨,久行權宜之制。諸改姓者,悉宜復舊。」甲子,大丞相、隋公楊堅進爵為王,以郡為隋國。己巳,以柱國、沛公鄭譯為上柱國。辛未,代王達、滕王悄並以謀執政,被誅。壬申,以大將軍、長
 寧公楊勇為上柱國、大司馬,以小冢宰、始平公元孝矩為大司寇。



 大定元年,春正月壬午,改元。丙戌,詔戎秩上開府以上,職事下大夫以上,外官刺史以上,各舉賢良。二月甲子,帝遜位於隋,居於別宮。隋氏奉帝為介國公,邑萬戶,車服禮樂,一如周制;上書不稱表,答表不稱詔。有其文,事竟不行。隋開皇元年五月壬申,帝崩,時年九歲。隋志也。謚曰靜皇帝。葬恭陵。



 論曰:自東西否隔,二國爭強;戎馬生郊,干戈日用;兵連禍結,力敵勢均;疆場之事,一彼一此。武皇纘業,未親萬
 機;慮遠謀深,以蒙養正。及英威電發,朝政惟新;內難既除,外略方始。乃苦心焦思,克己勵精;勞役為士卒之先,居處同疋夫之儉;修富國之政,務強兵之術;乘讎人之有釁,順天道而推亡。數年之間,大勛斯集。攄祖宗之宿憤,拯東夏之阽危。盛矣哉,有成功者也。若使翌日之瘳無爽,經營之志獲申;黷武窮兵,雖見譏於良史;雄圖遠略,足方駕於前王。而識嗣子之非才,顧宗祏之至重;滯愛同於晉武,則哲異於宋宣。但欲威之榎楚,期於懲肅,義方之教,豈若是乎。卒使昏虐君臨,姦回肆毒;迹宣后之行事,身歿已為幸矣。靜帝越自幼沖,紹茲衰統;內挾
 有劉之詐,戚籓無齊、代之強;隋氏因之,遂遷龜鼎。雖復民、峨投袂,翻成凌奪之威;漳、滏勤王,無救宗周之殞。嗚呼!



 以文皇之經啟鴻基,武皇之克隆景業,未逾二紀,不祀忽諸。斯蓋先帝之餘殃,非孺子之罪戾也。



\end{pinyinscope}