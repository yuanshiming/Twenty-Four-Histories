\article{卷五十列傳第三十八}

\begin{pinyinscope}

 辛雄族祖琛琛子術術族子德源楊機高道穆兄謙之綦俊山偉宇文忠之費穆孟威辛雄,字世賓,隴西狄道人也。父暢,汝南、鄉郡二郡太守。雄有孝性,居父憂,殆不可識。清河王懌為司空,辟為左曹。懌遷司徒,仍授左曹。雄用心平直,加以閑明政事,經其斷割,莫不悅服。懌每謂人曰:「必也無訟,辛雄有焉。」歷
 尚書駕部、三公郎。會沙汰郎官,唯雄與羊深等八人見留,餘悉罷遣。



 先是,御史中丞、東平王匡復欲輿棺諫諍,尚書令、任城王澄劾匡大不敬,詔恕死。雄奏理匡曰:「竊惟白衣元匡,歷奉三朝,每蒙寵遇,諤諤之性,簡自帝心。



 故高祖錫之以匡名,陛下任之以彈糾。當高肇之時,匡造棺致諫,主聖臣直,卒以無咎。假欲重造,先帝已容之於前,陛下亦宜寬之於後。」未幾,匡除平州刺史。



 右僕射元欽稱雄之美,左僕射蕭寶夤曰:「吾聞游僕射云:『得如雄者四五人共省事,足矣』今日之賞,何其晚哉!」



 初,廷尉少卿袁翻以犯罪之人,經恩競訴,枉直難明。遂奏曾染
 風聞者,不問曲直,推為獄成,悉不斷理。詔門下、尚書、廷尉議之。雄議曰:「《春秋》之義,不幸而失,寧僭不濫。僭則失罪人,濫乃害善人。今議者不忍罪姦吏,使出入縱情,令君子小人,薰蕕不別,豈所謂賞善罰惡,殷勤隱恤者也?古人唯患察獄之不精,未聞知冤而不理。」詔從雄議。自後每有疑議,雄與公卿駁難,事多見從。於是公能之名甚盛。又為《祿養論》,稱仲尼陳五孝,自天子至於庶人,無致仕之文。



 《禮記》:八十,一子不從政;九十,家不從政。鄭玄注云:「復除之。」然則止復庶人,非公卿士大夫之謂。以為宜聽祿養,不約其年。書奏,孝明納之。後除司空長史。時
 諸公皆慕其名,欲屈為佐,莫能得也。



 時諸方賊盛,而南寇侵境,山蠻作逆,孝明欲親討,以荊州為先。詔雄為行臺左丞,與臨淮王彧東趣葉城;別將裴衍,西通鴉路。衍稽留未進,議師已次汝濱。



 逢北溝求救,議以處分道別,不欲應之。雄曰:「王執麾閫外,唯利是從,見可而進,何必守道?」彧恐後有得失之責,要雄符下。雄以車駕將親伐,蠻夷必懷震動,乘彼離心,無往不破,遂符彧軍,令速赴擊。賊聞,果自走散。在軍上疏曰:「凡人所以臨堅陳而忘身,觸白刃而不憚者,一則求榮名,二則貪重賞,三則畏刑罰,四則避禍難。非此數事,雖聖王不能勸其臣,慈父
 不能厲其子。明主深知其情,故賞必行,罰必信,使親疏貴賤,勇怯賢愚,聞鐘鼓之聲,見旍旗之列,莫不奮激,競赴敵場。豈厭久生而樂早死也?利害縣於前,欲罷不能耳。自秦、隴逆節,將歷數年,蠻左亂常,稍已多載。凡在戎役,數十萬人,三方之師,敗多勝少,跡其所由,不明賞罰故也。陛下欲天下之早平,愍征夫之勤悴,乃降明詔,賞不移時。然兵將之勳,歷稔不決,亡軍之卒,晏然在家,致令節士無所勸慕,庸人無所畏懾。



 進而擊賊,死交而賞賒;退而逃散,身全而無罪,此其所以望敵奔沮,不肯進力者矣。為重發明詔,更量賞罰,則軍威必張,賊難可弭。
 臣聞必不得已,去食就信,以此推之,信不可斯須廢也。賞罰,陛下之所易,尚不能全而行之;攻敵,士之所難,欲其必死,寧可得也?」後為吏部郎中。



 及爾朱榮入洛,河陰之難,人情未安,雄潛竄不出。孝莊欲以雄為尚書,門下奏曰:「辛雄不出,存亡未知。」孝莊曰:「寧失亡而用之,可失存而不用也?」



 遂除度支尚書。後以本官兼侍中、關西尉勞大使。將發,請事五條:一言逋懸租調,宜悉不征;二言簡罷非時徭役,以紓人命;三言課調之際,使豐儉有殊,令州郡量檢,不得均一;四言兵起歷年,死亡者眾,或父或子,辛酸未歇,見存耆老,請假板職,悅生者之意,慰死
 者之魂;五言喪亂既久,禮儀罕習,如有閨門和穆,孝悌卓然者,宜旌其門閭。莊帝從之,因詔:人年七十者授縣,八十授郡,九十加四品將軍,百歲從三品將軍。



 永熙三年,兼吏部尚書。時近習專恣,雄懼其讒匿,不能守正,論者頗譏之。



 孝武南狩,雄兼左僕射,留守京師。永熙末,兼侍中。帝入關右,齊神武至洛,於永寧寺大集朝士,責雄及尚書崔孝芬、劉廞、楊機等曰:「為臣奉主,匡危救亂。



 若處不諫諍,出不陪隨,緩則耽寵,急便竄避,臣節安在?」乃誅之。



 二子,士璨、士貞,逃入關中。



 雄從父兄纂,字伯將,學涉文史,溫良雅正。初為袞州安東府主簿,與祕書丞同
 郡李伯尚有舊。伯尚與咸陽王禧同逆,逃竄投纂,事覺,坐免官。後為太尉騎兵參軍,每為府主清河王懌所賞。至定考,懌曰:「辛騎兵有學有才,宜為上第。」



 及梁將曹義宗攻新野,詔纂為荊州軍司。纂善撫將士,人多用命,賊甚憚之。會孝明崩諱至,咸以對敵,欲祕凶問。纂曰:「安危在人,豈關是也?」遂發喪號哭,三軍縞素,還入州城,申以盟約。尋為義宗所圍,相率固守。孝莊即位,除兼尚書,仍行臺。後大都督費穆擊義宗禽之,入城,因舉酒屬纂曰:「微辛行臺之在斯,吾亦無由建此功也。」



 永安二年,元顥乘勝至城下,為顥禽之。及孝莊還宮,纂謝不守之罪。帝曰:「
 於時朕亦北巡,東軍不守,豈卿之過。」轉滎陽太守。百姓姜洛生、康乞得者,舊是前太守鄭仲明左右,豪猾偷竊,境內患之。纂伺捕禽獲,梟於郡市,百姓欣然。



 纂僑屬洛陽,太昌中,乃為河南邑中正。



 永熙三年,除河內太守。齊神武赴洛,兵集城下,纂出城謁,神武慰勉之。因命前侍中司馬子如曰:「吾行途疲弊,宜代吾執河內手也。」尋為兼尚書、南道行臺、西荊州刺史。時蠻酋樊大能應西魏,纂攻之,不剋而敗,為西魏將獨孤信所害。



 贈司徒公。



 雄族祖琛。琛字僧貴。祖敬宗,父樹寶,並代郡太守。琛少孤,曾過友
 人,見其父母無恙,垂涕久之。釋褐奉朝請、滎陽郡丞。太守元麗性頗使酒,琛每諫之。



 麗後醉,輒令閉閣,曰:「勿使丞入也。」孝文南征,麗從輿駕,詔琛曰:「委卿郡事,如太守也。」景明中,為揚州征南府長史。刺史李崇,多事產業,琛每諫折,崇不從,遂相糾舉,詔並不問。後加龍驤將軍、南梁太守。崇因置酒謂琛曰:「長史後必為刺史,但不知得上佐何如人耳。」琛對曰:「若萬一叨忝,得一方正長史,朝夕聞過,是所願也。」崇有慚色。卒於官。



 琛寬雅有度量,涉獵經史,喜慍不形於色。當官奉法,所在有稱。



 長子悠,字元壽,早有器業,為侍御史,監揚州軍。賊平,錄勛書,時李
 崇猶為刺史,欲寄人名,悠不許。崇曰:「我昔逢其父,今復逢其子。」早卒。



 悠弟俊,字叔義,有文才。魏子建為山南行臺,以為郎中。有軍國機斷。還京,於滎陽為人所劫害。贈東秦州刺史。俊弟術。



 術字懷哲,少明敏,有識度,解褐司空胄曹參軍。與僕射高隆之共典營構鄴都宮室。術有思理,百工剋濟。再遷尚書右丞,出為清河太守,政有能名。追授并州長史,遭父憂去職。清河父老數百人,詣闕上書,請立碑頌德。齊文襄嗣事,與尚書左丞宋游道、中書侍郎李繪等並追詣晉陽,俱為上客。累遷散騎常侍。武定六年,侯景叛,除
 東南道行臺尚書,封江夏縣男。與高岳等破侯景,禽蕭明。遷東徐州刺史,為淮南經略使。齊天保元年,侯景徵江西租稅,術率諸軍度淮斷之,燒其稻數百萬石。還鎮下邳,人隨術北度淮者三千餘家。東徐州刺史郭志殺郡守,文宣聞之,敕術自今所統十餘州地,諸有犯法者,刺史先啟聽報;以下先斷,後表聞。齊代行臺兼總人事,自術始也。安州刺史、臨清太守、盱眙蘄城二鎮將犯法,術皆案奏殺之。睢州刺史及所部郡守,俱犯大辟,朝廷以其奴婢百口及貲財盡賜術。三辭不見許,術乃送詣所司,不復以聞。邢邵聞之,遺術書曰:「昔鐘離意云:孔子
 忍渴於盜泉,便以珠璣委地。足下今能如此,可謂異代一時。」及王僧辨破侯景,術招攜安撫,城鎮相繼款附,前後二十餘州。於是移鎮廣陵,獲傳國璽送鄴,文宣以璽告於太廟。此璽即秦所制,方四寸,上紐交盤龍,其文曰:「受命於天,既壽永昌。」



 二漢相傳,又歷魏、晉;晉懷帝敗,沒於劉聰;聰敗,沒於石氏;石氏敗,晉穆帝永和中,濮陽太守戴僧施得之,遣督護何融送於建業;歷宋、齊、梁;梁敗,侯景得之;景敗,侍中趙思賢以璽投景南袞州刺史郭元建,送於術,故術以進焉。尋徵為殿中尚書,領太常卿。仍與朝賢,議定律令。遷吏部尚書,食南袞州梁郡幹。遷
 鄴以後,大選之職,知名者數四,互有得失,未能盡美。文襄少年高朗,所弊也疏;袁叔德沈密謹厚,所傷者細;楊愔風流辨給,取士失於浮華;唯術性尚貞明,取士以才以器,循名責實,新舊參舉,管庫必擢,門閥不遺。考之前後銓衡,在術最為折衷,甚為當時所稱舉。天保末,文宣嘗令術選百員官,參選者二三千人,術題目士子,人無謗讟,其所旌擢,後亦皆致通顯。



 術清儉寡嗜欲,勤於所職,未嘗暫懈,臨軍以威嚴,牧人有惠政。少愛文史,晚更勤學,雖在戎旅,手不釋卷。及定淮南,凡諸貲物,一毫無犯。唯大收典籍,多是宋、齊、梁時佳本,鳩集萬餘卷,并顧、
 陸之徒名畫,二王已下法書,數亦不少。俱不上王府,唯入私門。及還朝,頗以饟遺貴要,物議以此少之。十年卒,年六十。皇建二年,贈開府儀同三司、中書監、青州刺史。



 子閣卿,尚書郎。閣卿弟衡卿,有識學,開府參軍事。隋大業初,卒於太常丞。



 術族子德源。德源字孝基,祖穆,魏平原太守。父子馥,尚書左丞。



 德源沈靜好學,十四解屬文,及長,博覽書記。美儀容,中書侍郎裴讓之特相愛好,兼有龍陽之重。齊尚書僕射楊遵彥、殿中尚書辛術皆一時名士,並虛襟禮敬,同舉薦之。後為兼員外散騎侍郎,聘梁使副。德源本貧素,因使,薄
 有資裝,遂餉執事,為父求贈,時論鄙之。中書侍郎劉逖上表薦德源:弱齡好古,晚節逾厲,枕藉《六經》,漁獵百氏;文章綺艷,體調清華。恭慎表於閨門,謙捴著於朋執;實後進之辭人,當今之雅器。由是除員外散騎侍郎。後兼通直散騎常侍,聘陳。及還,待詔文林館,位中書舍人。



 齊滅,仕周為宣納上士。因取急詣相州,會尉遲迥起逆,以為中郎,德源辭不獲免,遂亡去。隋受禪,不得調者久之。隱林慮山,鬱鬱不得志,著《幽居賦》以自寄。素與武陽太守盧思道友善,時相往來。魏州刺史崔彥武奏德源潛為交結,恐有姦計,由是謫令從軍討南寧。及還,祕書監
 牛弘以德源才學顯著,奏與著作郎王劭同修國史。德源每於務隙撰集,注《春秋三傳》三十卷,注《揚子法言》二十三卷。蜀王秀奏以為掾,轉諮議參軍,卒官。有集二十卷,又撰《政訓》、《內訓》各二十卷。有子素臣。



 德源從祖兄元植,齊天保中,司空司馬。學涉,有名聞於世。



 德源族叔珍之,少有氣俠,歷位北海太守,後行平州事,卒於州。贈驃騎大將軍、洛州刺史,謚曰恭。



 子愨,武定末,開府鎧曹參軍。



 楊機,字顯略,天水冀人也。祖伏恩,徙居洛陽,因家焉。機少有志節,為士流所稱。河南尹李平、元暉,並召署功曹。
 暉尤委以郡事。或謂暉曰:「弗躬弗親,庶人弗信,何得委事於機,高臥而已。」暉曰:「吾聞君子勞於求士,逸於任賢,吾既委得其才,何為不可?」由是聲名更著。時皇子國官多非其人,詔選清直之士,機見舉為京兆王愉國中尉,愉甚敬憚之。後為洛陽令,京輦伏其威風。訴訟者一經其前,後皆識其名姓,并記其事理。歷司州別駕、清河內史、河北太守,並有能名。



 永熙中,除度支尚書。機方直之心,久而彌厲,奉公正己,為時所稱。家貧無馬,多乘小犢車,時論許其清白。與辛雄等並為齊神武所誅。



 高恭之,字道穆,自云遼東人也。祖潛,獻文初,賜爵陽關
 男。詔以沮渠牧犍女賜潛為妻,封武威公主,拜駙馬都尉。父崇,字積善,少聰敏,以端謹稱。家資富厚,而崇志尚儉素。初,崇舅氏坐事誅,公主痛本生絕胤,遂以崇繼牧犍後,改姓沮渠。景明中,啟復本姓,襲爵,除洛陽令。為政清斷,吏人畏其威風,發手適不避強禦,縣內肅然。卒,贈滄州刺史,謚曰成。



 道穆以字行於世,學涉經史,所交皆名流俊士。幼孤,事兄如父。每謂人曰:「人生厲心立行,貴於見知,當使夕脫羊裘,朝佩珠玉。若時不我知,便須退迹江海,自求其志。」御史中尉元匡高選御史,道穆奏記求用於匡,匡遂引為御史。其所糾手適,不避權豪。正光中,出
 使相州。前刺史李世哲,即尚書令崇之子,多有非法,逼買人宅,廣興屋宇,皆置鴟尾,又於馬埒堠上為木人執節。道穆繩糾,悉毀去之,并表發其贓貨。爾朱榮討蠕蠕,道穆監其軍事,榮甚憚之。蕭寶夤西征,以為行臺郎中,委以軍機之事。後屬兄謙之被害,情不自安,遂託身於孝莊。孝莊時為侍中,深相保護。及即位,賜爵龍城侯,除太尉長史,領中書舍人。及元顥逼武牢,或勸帝赴關西者,帝以問道穆,道穆言關中殘荒,請車駕北度,循河東下。



 帝然之。其夜到河內郡北,帝命道穆燭下作詔書,布告遠近,於是四方知乘輿所在。



 尋除給事黃門侍郎、安
 喜縣公。於時爾朱榮欲迴師待秋,道穆謂曰:「大王擁百萬之眾,輔天子而令諸侯,此桓、文之舉也。今若還師,令顥重完守具,可謂養虺成蛇,悔無及矣。」榮深然之。及孝莊反政,因宴次謂爾朱榮曰:「前若不用高黃門計,社稷不安,可為朕勸其酒,令醉。」榮因陳其作監軍時,臨事能決,實可任用。



 尋除御史中尉,仍兼黃門。



 道穆外執直繩,內參機密,凡是益國利人之事,必以奏聞,諫爭盡言,無所顧憚。選用御史,皆當世名輩,李希宗、李繪、陽休之、陽斐、封君義、邢子明、蘇淑、宋世良等三十人。於時用錢稍薄,道穆表曰:「百姓之業,錢貨為本,救弊改鑄,王政所
 先。自頃以來,私鑄薄濫,官司糾繩,挂網非一。在市銅價,八十一文得銅一斤,私鑄薄錢,斤餘二百。既示之以深利,又隨之以重刑,得罪者雖多,姦鑄者彌眾。今錢徒有五銖之文,而無二銖之實,薄甚榆莢,上貫便破,置之水上,殆欲不沈。因循有漸,科防不切,朝廷失之,彼復何罪。昔漢文帝以五分錢小,改鑄四銖。至武帝復改三銖為半兩。此皆以大易小,以重代輕也。論今據古,宜改鑄大錢,文載年號,以記其始。則一斤所成,止七十六文。銅價至賤,五十有餘,其中人功,食料、錫炭、鉛鈔,縱復私營,不能自潤。直置無利,自應息心,況復嚴刑廣設也。以臣測
 之,必當錢貨永通,公私獲允。」後遂用楊侃計,鑄永安五銖錢。



 僕射爾朱世隆當朝權盛,因內見,衣冠失儀,道穆便即彈糾。帝姊壽陽公主行犯清路,執赤棒卒呵之不止,道穆令卒棒破其車。公主深恨,泣以訴帝。帝曰:「高中尉清直人,彼所行者公事,豈可私恨責之也?」道穆後見帝,帝曰:「一日家姊行路相犯,深以為愧。」道穆免冠謝,帝曰:「朕以愧卿,卿反謝朕!」尋敕監儀注。又詔:「祕書圖籍及典書緗素,多致零落,可令道穆總集帳目,并牒儒學之士,編比次第。」



 道穆又上疏曰:「高祖太和之初,置廷尉司直,論刑辟是非,雖事非古始,交濟時要。竊見御史出使,
 悉受風聞,雖時獲罪人,亦不無枉濫。何者?得堯之罰,不能不怨。守令為政,容有愛憎,姦猾之徒,恆思報惡,多有妄造無名,共相誣謗。



 御史一經檢究,恥於不成,杖木之下,以虛為實。無罪不能自雪者,豈可勝道哉!



 臣雖愚短,守不假器,繡衣所指,冀以清肅。若仍更踵前失,或傷善人,則尸祿之責,無所逃罪。如臣鄙見,請依太和故事,還置司直十人,名隸廷尉,秩以五品,選歷官有稱,心平性正者為之。御史若出糾劾,即移廷尉,令知人數。廷尉遣司直與御史俱發。所到州郡,分居別館。御史檢了,移付司直。司直覆問事訖,與御史俱還。中尉彈聞,廷尉科案,一
 如舊式。庶使獄成罪定,無復稽寬,為惡取敗,不得稱枉。若御史、司直糾劾失實,悉依所斷獄罪之。聽以所檢,迭相糾發。如二使阿曲,有不盡理,聽罪家詣門下通訴,別加案檢。如此,則肺石之傍,怨從可息;聚棘之下,受罪吞聲者矣。」詔從之,復置司直。



 及爾朱榮死,帝召道穆,付赦書,令宣於外,謂曰:「今當得精選御史矣。」



 先是,榮等常欲以其親黨為御史,故有此詔。及爾朱世隆等戰於大夏門北,道穆受詔督戰。又贊成太府卿李苗斷橋之計,世隆等於是北遁。加衛將軍、大都督,兼尚書右僕射、南道大行臺。時雖外託征蠻,而帝恐北軍不利,欲為南巡之
 計。未發,會爾硃兆入洛,道穆慮禍,託病去官。世隆以其忠於前朝,遂害之。太昌中,贈車騎大將軍、儀同三司、雍州刺史。子士鏡襲爵,為北豫州刺史。道穆兄謙之。



 謙之字道讓,少事後母以孝聞。專意經史,天文、算歷、圖緯之書,多所該涉。



 好文章,留心《老》、《易》。襲父爵。孝昌中,行河陰令。先是有人囊盛瓦礫,指作錢物,詐市人馬,因而逃去。詔令追捕,必得以聞。謙之乃偽枷一囚,立於馬市,宣言是前詐市馬賊,今欲刑之。密遣腹心,察市中私議者。有二人相見,忻然曰:「無復憂矣!」執送案問,悉獲其黨。并出前後盜處,失物之家,各得其本物,具以狀告。尋正
 河陰令。在縣二年,損益政體,多為故事。時道穆為御史,亦有能名,世美其父子兄弟並著當官之稱。



 舊制,二縣令得面陳得失。時佞幸之輩,惡其有所發聞,遂共奏罷。謙之乃上疏曰:「臣以無庸,謬宰神邑,實思奉法不撓,稱是官方。酬朝廷無貲之恩,盡人臣守器之節。但豪家支屬,戚里親媾,縲紲所及,舉目多是。皆有盜憎之色,咸起惡上之心。縣令輕弱,何能克濟?先帝昔發明詔,得使面陳所懷。臣亡父先臣崇之為洛陽令,常得入奏是非,所以朝貴斂手,無敢干政。近年已來,此制遂寢,致使神宰威輕,下情不達。今二聖遠遵堯、舜,憲章高祖,愚臣亦望
 策其駑蹇,少立功名。乞行新典,更明往制,庶姦豪知禁,頗自屏心。」詔付外量聞。



 謙之又上疏,以為:「自正光以來,邊城屢擾,命將出師,相繼於路。但諸將帥,或非其才,多遣親者,妄稱入募,唯遣奴客充數而已。對寇臨敵,略不彎弓。



 則是王爵虛加,征夫多闕,賊虜何可殄除,忠貞何以勸誡也?且近習侍臣,戚屬朝士,請託官曹,擅作威福。如有清貞奉法,不為回者,咸共譖毀,橫受罪罰。在朝顧望,誰肯申聞?蔽上擁下,虧風損政。使讒諂甘心,忠讜息義。且頻年以來,多有徵發,人不堪命,動致流離。茍保妻子,競逃王役,不復顧其桑井,憚此刑書。



 正由還有必困
 之理,歸無自安之路。若聽歸其本業,徭役微甄,則還者必眾,墾田增闢,數年之後,大獲課入。今不務以理還之,但欲嚴符切勒,恐數年之後,走者更多。故有國有家者,不患人不我歸,唯患政之不立;不恃敵不我攻,唯恃吾不可侮。此乃千載共遵,百王一致。伏願少垂覽察。」靈太后得其疏,以責左右近侍,諸寵要者由是疾之。乃啟太后,云謙之有學藝,除為國子博士。



 謙之與袁翻、常景、酈道元、溫子昇之徒,或申款舊。好施贍恤,言諾無虧。



 居家僮隸,對其兒不撻其父母,生三子便免其一世。無愆黥奴婢,常稱:「俱稟人體,如何殘害?」謙之以父舅氏沮渠蒙
 遜曾據涼土,國書漏闕,乃修《涼書》十卷,行於世。涼國盛事佛道,為論貶之,稱佛是九流之一家。當世名流,競以佛理來難,謙之還以佛義對之,竟不能屈。以時所行曆多未盡善,乃更改元修者撰,為一家之法。雖未行於世,識者歎其多能。時朝議鑄錢,以謙之為鑄錢都將長者史,乃上表求鑄三銖錢,曰:蓋錢貨之立,本以通有無,便交易,故錢之輕重,世代不同。太公為周置九府圜法。至景王時,更鑄大錢。秦兼海內,錢重半兩。漢興,以秦錢重,改鑄榆莢錢。



 至文帝五年,復為四銖。孝武時悉復銷壞,更鑄三銖。至無狩中,變為五銖。又造赤仄之錢,以一當五。
 王莽攝政,錢有六等:大錢重十二銖,次九銖,次七銖,次五銖,次三銖,次一銖。魏文帝罷五銖錢,至明帝復立。孫權江左鑄大錢,一當五百。權赤烏年,復鑄大錢,一當千。輕重大小,莫不隨時而變。竊以食貨之要,八政為首,聚財之貴,詒訓典文。是以昔之帝王,乘天地之饒,御海內之富,莫不腐紅粟於太倉,藏朽貫於泉府。儲畜既盈,人無困弊,可以寧謐四海,如身使臂者矣。



 昔漢之孝武,地廣財饒,外事四戎,遂虛國用。於是草茅之臣,出財助邊;興利之計,納稅廟堂。市列榷酒之官,邑有告緡之令。鹽鐵既興,錢弊屢改,少府遂豐,上林饒積。外闢百蠻,內不
 增賦者,皆計利之由也。今群妖未息,四郊多壘,徵稅既煩,千金日費,倉儲漸耗,財用將竭,誠楊氏獻稅之秋,桑兒言利之日。夫以西京之盛,錢猶屢改,並行大小,子母相權。況今寇難未除,州郡淪敗,人物凋零,軍國用少。別鑄小錢,可以富益,何損於政,何妨於人也?且政興不以錢大,政衰不以錢小,唯貴公私得所,政化無虧,既行之於古,亦宜效之於今矣。昔禹遭大水,以歷山之金鑄錢,救人之困;湯遭大旱,以莊山之金鑄錢,贖人之賣子者。今百姓窮悴,甚於曩日,欽明之主,豈得垂拱而觀之哉?臣今此鑄,以濟交乏,五銖之錢,任使並用,行之無損,國
 得其益。



 詔將從之。事未就,會卒。



 初,謙之弟道穆,正光中為御史,糾相州刺史李世哲事,大相挫辱,其家恒以為憾。至是世哲弟神軌為靈太后深所寵任,會謙之家僮訴良,神軌左右之,入諷尚書,判禁謙之於廷尉。時將赦,神軌乃啟靈太后,發詔於獄賜死。朝士莫不哀之。



 所著文章百餘篇,別有集錄。永安中,贈營州刺史,謚曰康。又除一子出身,以明冤屈。



 謙之弟謹之,字道修。父崇既還本姓,以謹之繼沮渠氏。



 綦俊,字剽顯,河南洛陽人也。其先居代。俊孝莊時仕,累遷為滄州刺史,甚為吏人畏悅。尋除太僕卿。及爾硃世
 隆等誅,齊神武召文武百司,下及士庶,議所立。莫有應者。俊避席曰:「廣陵王雖為爾朱扶戴,當今之聖主也。」神武將從之。



 時黃門崔甗議不同,高乾、魏蘭根等固執甗言,遂立孝武帝。及帝入關,神武深思俊言,常以為恨。尋除御史中尉,於路與僕射賈顯度相逢,顯度恃勳貴,排俊騶列倒,俊忿見於色,自入奏之。尋加散騎常侍、驃騎大將軍、左光祿大夫、儀同三司。



 俊佞巧,能候當塗,斛斯椿、賀拔勝皆與友善。性多詐,賀拔勝出鎮荊州,過俊別,因辭俊母。俊故見敗氈弊被,勝更遺之錢物。後兼吏部尚書,復為滄州刺史。徵還,兼中尉,章武縣伯。尋除殷州
 刺史,薨於州。贈司空公,謚曰文貞。



 子洪實,字巨正,位尚書左右郎、魏郡邑中正。嗜酒好色,無行檢,卒。



 山偉,字仲才,河南洛陽人也。其先居代。祖強,美容貌,身長八尺五寸,工騎射,彎弓五石,為奏事中散。從獻文獵方山,有兩狐起於御前,詔強射之,百步內,二狐俱獲。位內行長。父幼之,位金明太守。偉涉獵文史,孝明初,元匡為御史中尉,以偉兼侍御史。入臺五日,便遇正會,偉司神武門。其妻從叔為羽林隊主,撾直長於殿門,偉即劾奏。匡善之,俄然奏正,帖國子助教,遷員外郎、廷尉評。



 時天下無事,進仕路難,代遷之人,多不霑預。及六鎮、隴西
 二方起逆,領軍元叉欲用代來寒人為傳詔,以慰悅之,而牧守子孫投狀求者百餘人。叉因奏立勛附隊,令各依資出身。自是北人,悉被收敘。偉遂奏記,贊叉德美。叉素不識偉,訪侍中安豐王延明、黃門郎元順,順等因是稱薦之。叉令僕射元欽引偉兼尚書二千石郎,後正名士郎,修起居注。僕射元順領選,表薦為諫議大夫。



 爾硃榮之害朝士,偉時守直,故免禍。及孝莊入宮,仍除偉給事黃門侍郎。先是偉與儀曹郎袁升、屯田郎李延考、外兵郎李奐、三公郎王延業方駕而行,偉少居後。路逢一尼,望之歎曰:「此輩緣業,同日而死。」謂偉曰:「君方近天子,
 當作好官。」而昇等四人皆於河陰遇害,果如其言。



 俄領著作郎,節閔帝立,除祕書監,仍著作。初,爾朱兆入洛,官守奔散,國史典書高法顯密埋史書,故不遺落。偉自以為功,訴求爵賞。偉挾附世隆,遂封東阿縣伯,而法顯止獲男爵。偉尋進侍中。孝靜初,除衛大將軍,中書令,監起居。



 後以本官復領著作,卒官。贈驃騎大將軍、開府儀同三司、都督、幽州刺史,謚曰文貞公。



 國史自鄧彥海、崔深、崔浩、高允、李彪、崔光以還,諸人相繼撰錄。綦俊及偉等諂悅上黨王天穆及爾朱世隆,以為國書正應代人修緝,不宜委之餘人,是以綦、偉等更主大籍。守舊而已,初
 無述著,故自崔鴻死後,迄終偉身,二十許載,時事蕩然,萬不記一。後人執筆,無所憑據,史之遺闕,偉之由也。外示沈厚,內實矯競。與綦俊少甚相得,晚以名位之間,遂若水火。與宇文忠之之徒代人為黨,時賢畏惡之。而愛尚文史,老而彌篤。偉弟少亡,偉撫寡訓孤,同居二十餘載,恩義甚篤。不營產業,身亡之後,賣宅營葬,妻女不免飄泊,士友歎愍之。長子昂襲爵。



 宇文忠之,河南洛陽人也。其先南單于之遠屬,世據東部,後居代都。父侃,卒於書侍御史。忠之涉獵文史,頗有筆札,釋褐太學博士。天平初,除中書侍郎。



 裴伯茂與之
 同省,常侮忽之,以忠之色黑,呼為「黑宇」。後敕修國史。元象初,兼通直散騎常侍,副鄭伯猷,使梁。武定初,為尚書右丞,仍修史。未幾,以事除名。忠之好榮利。自為中書郎六七年矣,遇尚書省選右丞,預選者皆射策,忠之試焉。既獲丞職,大為忻滿,志氣囂然,有驕物之色。識者笑之。既失官爵,怏怏發疾,卒于君山。



 費穆,字朗興,代人也。祖於,位商賈二曹令、懷州刺史,賜爵松陽男。父萬襲爵,位梁州鎮將,贈冀州刺史。穆性剛烈,有壯氣,頗涉文史,好尚功名。宣武初,襲爵,稍遷涇州平西府長史。時刺史皇甫集,靈太后之元舅,恃外戚之
 親,多為非法。穆正色匡諫,集亦憚之。



 後蠕蠕主婆羅門自涼州歸降,其部眾因飢侵掠邊邑。詔穆銜旨宣慰,莫不款附。



 明年復叛,入寇涼州。除穆兼尚書右丞、西北道行臺,仍為別將,往討之。穆至涼州,蠕蠕遁走。穆謂其所部曰:「夷狄獸心,見敵便走,若不令其破膽,終恐疲於奔命。」乃簡練精騎,伏於山谷,使羸劣之眾為外營,以誘之。賊騎覘見,俄而競至,伏兵奔擊,大破之。及六鎮反叛,穆為別將,隸都督李崇北伐。都督崔暹失利,崇將議班師。以朔州是白道之衝,賊之咽喉,若不全,則并、肆危,選將鎮捍,僉議舉穆。崇乃請穆為朔州刺史。尋改雲州刺史。
 穆招集離散,頗得人心。北境州鎮皆沒,唯穆獨存。久之,援軍不至,穆乃棄城南走,投爾朱榮於秀容。既而詣闕請罪,詔原之。孝昌中,以都督討平二絳反蜀,拜散騎常侍。後妖賊李洪於陽城起逆,連結蠻左,詔穆兼武衛將軍擊破之。



 及爾朱榮向洛,靈太后徵穆,令屯小平。榮推奉孝莊,穆遂先降。榮素知穆,見之甚悅。穆潛說榮曰:「公士馬不出萬人,長驅向洛,前無橫陳者。政以推奉主上,順人心故。今以京師之眾,百官之盛,一知公之虛實,必有輕侮之心。若不大作討罰,更樹親黨,公還北之日,恐不得度太行而內難行矣。」榮心然之,於是有河陰之事。
 天下聞之,莫不切齒。榮入洛,穆為吏部尚書、魯縣侯,進封趙平郡公。



 為侍中、前鋒大都督,與大將軍元天穆討平邢杲。



 時元顥入京師,穆與天穆既平齊地,將擊顥。穆圍武牢,將拔,屬天穆北度,既無後繼,穆遂降顥。顥以河陰酷濫,事起於穆,引入詰讓,殺之。孝莊還宮,贈侍中、司徒公,謚曰武宣。



 孟威,字能重,河南洛陽人也。頗有氣尚,尤知北土風俗。歷東宮齊帥、羽林監。後以明解北人語,敕在著作,以備推訪。累遷沃野鎮將。前後頻使遠籓。粗能稱旨。普泰中,除大鴻臚卿,卒,贈司空公。子恂嗣。



 論曰:辛雄吏能歷職,琛以公方行己,懷哲體有清監,德源雅業無虧,並素門之所得也。楊機清斷在公。道穆兄弟有政事之用。綦俊遭逢受職。山偉位行頗爽。



 忠之雖文史足用,而雅道蔑聞。費穆出身效力,功名著矣,末路一言,禍延簪帶,其死也宜哉!孟威以方言陳力,其勤亦可稱矣。



\end{pinyinscope}