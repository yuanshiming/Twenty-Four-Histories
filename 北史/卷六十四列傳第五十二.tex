\article{卷六十四列傳第五十二}

\begin{pinyinscope}

 韋孝寬兄夐夐子世康韋瑱子師柳虯弟檜慶慶子機機子述機弟弘旦肅機從子謇之韋叔裕,字孝寬,京兆杜陵人也,少以字行。世為三輔著姓。祖直善,魏馮翊、扶風二郡守。父旭,武威郡守。建義初,為大行臺右丞,加輔國將軍、雍州大中正。



 永安二年,拜右將軍、南豳州刺史。時氐賊數為抄竊,旭隨機招撫,並即歸附。尋卒官,贈司空、冀州刺史,謚曰文惠。孝寬沈敏
 和正,涉獵經史。弱冠,屬蕭寶夤作亂關右,乃詣闕,請為軍前驅。朝廷嘉之,即拜統軍。隨馮翊公長孫承業西征,每戰有功。拜國子博士,行華山郡事。屬侍中楊侃為大都督,出鎮潼關,引孝寬為司馬。侃奇其才,以女妻之。永安中,授宣威將軍、給事中,尋賜爵山北縣男。普泰中,以都督從荊州刺史源子恭鎮穰城,以功除淅陽郡守。時獨孤信為新野郡守,同隸荊州,與孝寬情好款密,政術俱美,荊部吏人號為連璧。孝武初,以都督鎮城。



 周文帝自原州赴雍州,命孝寬隨軍。及剋潼關,即授弘農郡守。從拎竇泰,兼左丞,節度宜陽兵馬事。仍與獨孤信入洛,
 為陽城郡守,復與宇文貴、怡峰應接潁川義徒,破東魏將任祥、堯雄於潁川。孝寬又進平樂口,下豫州,獲刺史馮邕。又從戰於河橋。時大軍不利,邊境騷然,乃令孝寬以本將軍行宜陽郡事。尋遷南兗州刺史。是歲,東魏將段琛、堯傑復據宜陽,遣其陽州刺史牛道恆扇誘邊人。孝寬深患之,乃遣諜人訪獲道恆手迹,令善學書者偽作道恆與孝寬書,論歸款意,又為落燼燒迹,若火下書者,還令諜人送於琛營。琛得書,果疑道恆,其所欲經略,皆不見用。孝寬知其離阻,因出奇兵掩襲,禽道恆及琛等,崤澠遂清。大統五年,進爵為侯。八年,轉晉州刺史,尋移
 鎮玉壁,兼攝南汾州事。先是,山胡負險,屢為劫盜,孝寬示以威信,州境肅然。進授大都督。



 十二年,齊神武傾山東之眾,志圖西入,以玉壁衝要,先命攻之。連營數十里,至於城下。乃於城南起土山,欲乘之以入。當其山處,城上先有兩高樓。孝寬更縛木接之,令極高峻,多積戰具以禦之。齊神武使謂城中曰:「縱爾縛樓至天,我會穿城取爾。」遂於城南鑿地道,又於城北起土山,攻具,晝夜不息。孝寬復掘長塹,要其地道,仍簡戰士屯塹。城外每穿至塹,戰土即擒殺之。又於塹外積柴貯火,敵人有在地道內者,便下柴火,以皮排吹之。火氣一衝,咸即灼爛。城
 外又造攻車,車之所及,莫不摧毀,雖有排楯,莫之能抗。孝寬乃縫布為縵,隨其所向則張設之。



 布懸於空中,其車竟不能壞。城外又縛松於竿,灌油加火,規以燒布,并欲焚樓。



 孝寬復長作鐵鉤,利其鋒刃,火竿一來,以鉤遙豁之,松麻俱落。外又於城四面穿地,作二十一道,分為四路,於其中各施梁柱。作訖,以油灌柱,放火燒之,柱折,城並崩壞。孝寬又隨崩處,豎木柵以扞之,敵不得入。城外盡其攻擊之術,孝寬咸拒破之。神武無如之何,乃遣倉曹參軍祖孝徵謂曰:「未聞救兵,何不降也?」孝寬報云:「我城池嚴固,兵食有餘,攻者自勞,守者常逸,豈有旬朔
 之間,已須救援?適憂爾眾有不反之危。孝寬關西男子,必不為降將軍也。」俄而孝徵復謂城中人曰:「韋城主受彼榮祿,或復可爾,自外軍士,何事相隨入湯火中邪?」乃射募格於城中云:「能斬城主降者,拜太尉,封開國郡公,邑萬戶,賞帛萬匹。」孝寬手題書背,反射城外,云:「若有斬高歡者,一依此賞。」孝寬弟子遷,先在山東,又鎖至城下,臨以白刃云:「若不早降,便行大戮。」孝寬慷慨激揚,略無顧意。



 士卒莫不感勵,人有死難之心。神武苦戰六旬,傷及病死者十四五,智力俱困,因而發疾。其夜遁去。後因此忿恚,遂殂。魏文帝嘉孝寬功,令殿中尚書長孫紹遠、
 左丞王悅至玉壁勞問,授驃騎大將軍、開府儀同三司,進爵建忠郡公。



 廢帝二年,為雍州刺史。先是,路側一里置一土堠,經雨頹毀,每須修之。自孝寬臨州,乃勒部內,當堠處植槐樹代之。既免修復,行旅又得庇陰。周文後見,怪問知之,曰:「豈得一州獨爾,當令天下同之。」於是令諸州夾道一里種一樹,十里種三樹,百里種五樹焉。恭帝元年,以大將軍與燕公于謹伐江陵,平之,以功封穰縣公。還,拜尚書右僕射,賜姓宇文氏。三年,周文北巡,命孝寬還鎮玉壁。



 周孝閔帝踐祚,拜小司徒。明帝初,參麟趾殿學士,考校圖籍。保定初,以孝寬立勳玉壁,置勳州,
 仍授勛州刺史。齊人遣使至玉壁,求通互市。晉公護以其相持日久,絕無使命,一日忽來求交易,疑別有故。又以皇姑、皇世母先沒在彼,因其請和之際,或可致之。遂令司門下大夫尹公正至玉壁,共孝寬詳議。孝寬乃於郊盛設供帳,令公正接對使人,兼論皇家親屬在東之意。使者辭色甚悅。時又有汾州胡抄得關東人,孝寬復放東還,並致書一牘,具陳朝廷欲敦鄰好。遂以禮送皇姑及護母等。孝寬善於撫御,能得人心,所遣間諜入齊者,皆為盡力。亦有齊人得孝寬金貨,遙通書疏。故齊動靜,朝廷皆先知。時有主帥許盆,孝寬度以心膂,令守一
 城。盆乃以城東入。孝寬怒,遣諜取之,俄而斬首而還。其能致物情如此。



 汾州之北,離石以南,悉是生胡,抄掠居人,阻斷河路。孝寬深患之,而地入於齊,無方誅剪。欲當其要處,置一大城。乃於河西征役徒十萬,甲士百人,遣開府姚岳監築之。岳色懼,以兵少為難。孝寬曰:「計成此城,十日即畢。既去晉州四百餘里,一日創手,二日偽境始知,設令晉州徵兵,二日方集,謀議之間,自稽三日,計其軍行,二日不到。我之城隍,足得辦矣」。乃令築之。齊人果至南首,疑有大軍,乃停留不進。其夜,又令汾水以南,傍介山、稷山諸村,所在縱火。齊人謂是軍營,遂收兵自
 固。版築克就,卒如其言。



 四年,進位柱國。時晉公護將東討,孝寬遣長史辛道憲啟陳不可,護不納。既而大軍果不利。後孔城遂陷,宜陽被圍。孝寬乃謂其將帥曰:「宜陽一城之地,未能損益。然兩國爭之,勞師數載。彼多君子,寧乏謀猷?若棄崤東,來圖汾北,我之疆界,必見侵擾。今宜於華谷及長秋速築城,以杜賊志。脫其先我,圖之實難。」



 於是畫地形,具陳其狀。晉公護令長史叱羅協謂使人曰:「韋公子孫雖多,數不滿百。汾北築城,遣誰固守?」事遂不行。



 天和五年,進爵鄖國公,增邑通前一萬戶。是歲,齊人果解宜陽之圍,經略汾北,遂築城守之。其丞相斛律明
 月至汾東,請與孝寬相見。明月云:「宜陽小城,久勞戰爭。今既入彼,欲於汾北取償,幸勿怪也。」孝寬答曰:「宜陽彼之要衝,汾北我之所棄。我棄彼圖,取償安在?且若輔翼幼主,位重望隆,理宜調陰陽,撫百姓,焉用極武窮兵,構怨連禍!且滄、瀛大水,千里無煙,復欲使汾、晉之間,橫尸暴骨,茍貪尋常之地,塗炭疲弊之人,竊為君不取。」孝寬參軍曲巖頗知卜筮,謂孝寬曰:「來年東朝必大相殺戮。」孝寬因令巖作謠歌曰:「百升飛上天,明月照長安。」百升,斛也。又言:「高山不摧自崩,槲樹不扶自豎。」令諜人多賚此文,遺之於鄴。祖孝徵既聞,更潤色之,明月
 竟以此誅。



 建德之後,武帝志在平齊。孝寬乃上疏陳三策。



 其第一策曰:「臣在邊積年,頗見間隙,不因際會,難以成功。是以往歲出軍,徒有勞費,功績不立,由失機會。何者?長淮之南,舊為沃土,陳氏以破亡餘燼,猶能一舉平之。齊人歷年赴救,喪敗而反。內離外叛,計盡力窮。傳不云乎:『讎有釁焉,不可失也。』今大軍若出軹關,方軌而進,兼與陳氏共為掎角;並令廣州義旅出自三鵶;又募山南驍銳,沿河而下,復遣北山稽胡絕其並、晉之路。凡此諸軍,仍令各募關、河之外勁勇之士,厚其爵賞,使為前驅。岳動川移,雷駭電激,百道俱進,並趨虜庭。必當望旗
 奔潰,所向摧殄。一戎大定,實在此機。」



 其第二策曰:「若國家更為後圖,未即大舉,宜與陳人分其兵勢。三鵶以北,萬春以南,廣事屯田,預為貯積。募其驍悍,立為部伍。彼既東南有敵,戎馬相持,我出奇兵,破其疆埸。彼若興師赴援,我則堅壁清野,待其去遠,還復出師。常以邊外之軍,引其腹心之眾。我無宿舂之費,彼有奔命之勞。一二年中,必自離叛。



 且齊氏昏暴,政出多門,鬻獄賣官,唯利是視,荒淫酒色,忌害忠良。闔境熬然,不勝其弊。以此而觀,覆亡可待。然後乘間電掃,事等摧枯。」



 其第三策曰:「竊以大周土宇,跨據關、河,蓄席卷之威,持建瓴之勢。太祖
 受天明命,與物更新,是以二紀之中,大功克舉。南清江、漢,西龕巴、蜀,塞表無虞,河右底定。唯彼趙、魏,獨為榛梗者,正以有事三方,未遑東略。遂使漳、滏游魂,更存余晷。昔勾踐亡吳,尚期十載;武王取亂,猶煩再舉。今若更存遵養,且復相時,臣謂宜還崇鄰好,申其盟約,安人和眾,通商惠工,蓄銳養威,觀釁而動。斯則長策遠馭,坐自兼並也。」



 書奏,武帝遣小司寇淮南公元偉、開府伊婁謙等重幣聘齊。爾後遂大舉,再駕而定山東。卒如孝寬之策。



 孝寬每以年迫懸車,屢請致仕。帝以海內未平,優詔弗許。至是,復稱疾乞骸骨。帝曰:「往已面申本懷,何煩重請
 也。」



 五年,帝東伐,過幸玉壁。觀禦敵之所,深歎美之,移時乃去。孝寬自以習練齊人虛實,請為先驅。帝以玉壁要衝,非孝寬無以鎮之,乃不許。及趙王招率兵出稽胡,與大軍掎角,乃敕孝寬為行軍總管,圍守華谷以應接之。孝寬剋其四城。武帝平晉州,復令孝寬還舊鎮。及帝凱旋,復幸玉壁。從容謂孝寬曰:「世稱老人多智,善為軍謀。然朕唯共少年一舉平賊,公以為如何?」孝寬對曰:「臣今衰耄,唯有誠心而已。然昔在少壯,亦曾輸力先朝,以定關右。」帝大笑曰:「實如公言。」



 乃詔孝寬隨駕還京。拜大司空,出為延州總管,進位上柱國。



 大象元年,除徐、兗等十
 一州十五鎮諸軍事、徐州總管。又為行軍元帥,徇地淮南。乃分遣巳公宇文亮攻黃城,郕公梁士彥攻廣陵,孝寬率眾攻壽陽,並拔之。



 初,孝寬到淮南,所在皆密送誠款。然彼五門,尤為險要,陳人若開塘放水,即津濟路絕。孝寬遽令分兵據守之。陳刺史吳文立果遣決堰,已無及。於是陳人退走,江北悉平。軍還,至豫州,宇文亮舉兵反,立以數百騎襲孝寬營。時亮國官茹寬密白其狀,孝寬有備,亮不得入,遁走,孝寬追獲之。詔以平淮南之功,別封一子滑國公。



 及宣帝崩,隋文帝輔政。時尉遲迥先為相州總管,詔孝寬代之。又以小司徒叱列長叉為相
 州刺史,先令赴鄴。孝寬續進,至朝歌,迥遣其大都督賀蘭貴賚書候孝寬。孝寬留貴與語以察之,疑其有變,遂稱疾徐行。又使人至相州求醫藥,密以伺之。既到湯陰,逢長叉奔還。孝寬兄子魏郡守藝又棄郡南走。孝寬審知其狀,乃馳還。所經橋道,皆令毀撤,驛馬悉擁以自隨。又勒驛將曰:「蜀公將至,可多備肴酒及芻粟以待之。」迥果遣儀同梁子康將數百騎追孝寬,驛司供設豐厚,所經之處,皆輒停留,由是不及。



 時或勸孝寬,以為洛京虛弱,素無守備,河陽鎮防,悉是關東鮮卑,迥若先往據之,則為禍不小。乃入保河陽。河陽城內,舊有鮮卑八百人,
 家並在鄴,見孝寬輕來,謀欲應迥。孝寬知之,遂密造東京官司,詐稱遣行,分人詣洛受賜。既至洛陽,並留不遣。因此離解,其謀不成。



 六月,詔發關中兵,以孝寬為元帥東伐。七月,軍次河陽。迥所署儀同薛公禮等圍逼懷州,孝寬遣兵擊破之。進次懷縣永橋城之東南,其城既在要衝,雉堞牢固,迥已遣兵據之。諸將士以此城當路,請先攻取。孝寬曰:「城小而固,若攻而不拔,損我兵威。今破其大軍,此亦何能為也?」於是引軍次於武陟,大破迥子惇,惇輕騎奔鄴。軍次於鄴西門豹祠之南,迥自出戰,又破之。迥窮迫自殺。兵士在小城中者,盡坑之於游豫園。
 諸有未服,皆隨機討之。關東悉平。十月,凱還京師。十一月,薨,時年七十二。贈太傅、十二州諸軍事、雍州牧,謚曰襄。



 孝寬在邊多載,屢抗強敵。所有經略,布置之初,人莫之解;見其成事,方乃驚服。,雖在軍中,篤意文史,政事之餘,每自披閱。末年患眼,猶令學士讀而聽之。又早喪父母,事兄嫂甚謹,所得俸祿,不入私房。親族有孤遺者,必加振贍。



 朝野以此稱焉。長子諶,年十歲,魏文帝欲以女妻之。孝寬辭以兄子世康年長。帝嘉之,遂以妻世康。



 孝寬有六子,總、壽、霽、津知名。



 總字善會,聰敏好學。位驃騎大將軍、開府儀同三司、納言、京兆尹。武帝嘗戲總曰:「卿師尹
 帝鄉,故當不以富貴威福鄉里邪?」總乃正色對曰:「陛下擢臣非分,竊謂已鑒愚誠。今奉嚴旨,便似未照丹赤。豈可久忝此職,用疑聖慮。請解印綬,以避賢能。」帝大笑曰:「前言戲之耳。」五年,從武帝東征。總每率麾下,先驅陷敵,遂於並州戰歿,時年二十九。贈上大將軍,追封河南郡公,謚曰貞。六年,重贈柱國、五州刺史。



 子國成嗣,後襲孝寬爵鄖國公。隋文帝追錄孝寬舊勳,開皇初,詔國成食封三千戶,收其租賦。



 壽字世齡,以貴公子早有令譽。位京兆尹。武帝親征齊,委以後事。以父軍功,賜爵永安縣侯。隋文帝為丞相,以
 其父平尉遲迥,拜壽儀同三司,進封滑國公。文帝受禪,歷位恒、毛二州刺史,頗有能名。以疾徵還,卒于家。謚曰定。仁壽中,文帝為晉王昭納其女為妃。其子保巒嗣。



 壽弟霽,位太常少卿、安邑縣伯。



 霽弟津,位內史侍郎、戶部侍郎、判尚書事。



 孝寬兄夐。夐字敬遠,志尚夷簡,澹於榮利。弱冠,被召拜雍州中從事,非其好也,遂謝疾去。前後十見征辟,皆不應命。屬周文帝經綸王業,側席求賢,聞夐養高不仕,虛心敬悅,遣使辟之,備加禮命。雖情諭甚至,而竟不能屈。彌以重之,亦弗之奪也。所居之宅,枕帶林泉。夐對玩琴書,蕭然自
 逸,時人號為居士焉。至有慕其閑素者,或載酒從之,夐亦為之盡歡,接對忘倦。明帝即位,禮敬愈厚。乃為詩以貽之曰:「六爻貞遁世,三辰光少微。潁陽讓逾遠,滄州去不歸。香動秋蘭佩,風飄蓮葉衣。坐石窺仙洞,乘槎下釣磯。嶺松千仞直,巖泉百丈飛。聊登平樂觀,遙望首陽薇。詎能同四陷,來參餘萬機?」夐答帝詩,願時朝謁。帝大悅,敕有司日給河東酒一斗,號之曰逍遙公。時晉公護執政,廣營第宅。嘗召夐至宅,訪以政事。夐仰視其堂,徐而嘆曰:「酣酒嗜音,峻宇雕墻,有一於此,未或弗亡。」



 護不悅。有識者以為知言。陳遣其尚書周弘正來聘,素聞夐
 名,請與相見。朝廷許之。弘正乃造夐,談謔盡日,恨相遇之晚。後請夐至賓館,夐不時赴。弘正乃贈詩曰:「德星猶未動,真車詎肯來?」其為當時所欽挹如此。



 武帝嘗與夐夜宴,大賜之縑帛,令侍臣數人負以送出。夐唯取一匹,示承恩旨而已,帝以此益重之。孝寬為延州總管,夐至州,與孝寬相見。將還,孝寬以所乘馬及轡勒與夐。夐以其華飾,心弗欲之。笑謂孝寬曰:「昔人不棄遺簪墜履者,惡與之同出,不與同歸。吾之操行,雖不逮前烈,然捨舊錄新,亦非吾志也。」於是乃乘舊馬以歸。武帝又以佛、道、儒三教不同,詔夐辨其優劣。夐以三教雖殊,同歸於善,
 其跡似有深淺,其致理如無等級。乃著《三教序》奏之。帝覽而稱善。時宣帝在東宮,亦遺夐書,並令以帝所乘馬迎之,問以立身之道。夐對曰:「《傳》不云乎,儉為德之恭,侈為惡之大。欲不可縱,志不可滿。並聖人之訓也,願殿下察之。」



 夐子瓘,行隨州刺史,因疾物故。孝寬子總復於并州戰歿。一日之中,兇問俱至。家人相對悲慟,而夐神色自若,謂之曰:「死生命也,去來常事,亦何足悲!」



 援琴撫之如舊。夐又雅好名義,虛襟善誘,雖耕夫牧豎,有一介可稱者,皆接引之。



 特與族人處玄及安定梁曠為放逸之友。少愛文史,留情著述,手自抄錄數十萬言。



 晚年虛靜,
 唯以體道會真為務,舊所制述,咸削其槁,故文筆多並不存。



 建德中,夐以年老,預戒其子等曰:「昔士安以籧篨束體,王孫以布囊繞尸,二賢高達,非庸才能繼。吾死之日,可斂舊衣,勿更新造。使棺足周尸,牛車載柩,墳高四尺,壙深一丈。其餘煩雜,悉無用也。朝晡奠食,於事彌煩,吾不能頓絕汝輩之情,可朔望一奠而已。仍薦蔬素,勿設牲牢。親友欲以物弔祭者,並不得為受。



 吾常恐臨終恍惚,故以此言預戒汝輩。瞑目之日,勿違吾志也。」宣政元年二月,卒於家,時年七十七。武帝遣使弔祭,賻賵有加。其喪制葬禮,諸子等並遵其遺戒。



 子世康。



 世康幼而沈敏,有器度。年十歲,州辟主簿。在魏,弱冠為直寢,封漢安縣公,尚周文帝女襄樂公主,授儀同三司。仕周,歷位典祠下大夫,沔、硤二州刺史。從武帝平齊,授司州總管長史。時東夏初定,百姓未安,世康綏撫之,士庶胥悅。入為戶部中大夫,進位上開府,轉司會中大夫。尉遲迥之亂,隋文謂世康曰:「汾、絳舊是周、齊分界,因此亂階,恐生搖動,今以委公。」因授絳州刺史。以雅望鎮之,闔境清肅。



 世康性恬,素好古,不以得喪干懷。在州有止足之志,與子弟書曰:「吾生因緒餘,夙沾纓弁,驅馳不已,四紀於茲,亟登袞命,頻蒞方岳,志除三惑,心慎四知,
 以不貪而為實,處脂膏而莫潤。如斯之事,頗為時悉。今耄雖未及,壯年已謝。



 霜早梧楸,風先蒲柳。眼闇更劇,不見細書;足疾彌增,非可趨走。祿豈須多,防滿則退;年不待暮,有疾便辭。況襄春秋已高,溫清宜奉,晨昏有闕,罪在我躬。



 今世穆、世文,並從武役,吾與世沖,復嬰遠任,陟岵瞻望,此情彌切。桓山之悲,倍深常戀。意欲上聞,乞遵禮教,未訪汝等,故遣此及。興言遠慕,感咽難勝。」。



 諸弟報以事恐難遂,乃止。



 在任有惠政,奏課連最,擢為禮部尚書。世康寡嗜慾,不慕勢貴,未嘗以位望矜物。聞人之善,若己有之,亦不顯人過咎,以求名譽。進爵上庸郡公。轉
 吏部尚書,選用平允,請託不行。以母憂去職,固辭,乞終私制。上不許。開皇七年,將事江南,議重方鎮,拜襄州刺史。坐事免。未幾授安州總管,遷信州總管。十三年,復拜吏部尚書,前後十餘年間,多所進拔,朝廷稱為廉平。



 嘗因休暇,謂子弟曰:「吾聞功遂身退,古人常道。今年將耳順,志在懸車,汝輩以為云何?」子福嗣答曰:「大人澡身浴德,名立官成。盈滿之戒,先哲所重,欲追蹤二疏,伏奉尊命。」後因侍宴,世康再拜陳讓,願乞骸骨。上曰:「冀與公共理天下,今之所請,深乖本望。縱筋力衰謝,猶屈公臥臨一隅。」於是出拜荊州總管。時天下唯置荊、並、楊、益四大
 總管,並、楊、益三州並親王臨統,唯荊州委於世康,時論以此為美。世康為政簡靜,百姓愛悅。卒於州。上聞而痛惜,贈大將軍,謚曰文。



 世康性孝友,初以諸弟位並隆貴,獨季弟世約宦途不達,共推父時田宅盡以與之。世多其義。



 長子福子,位司隸別駕。



 次子福嗣,位內史舍人。後以罪黜。楊玄感之亂,從衛玄戰,敗於城北,為玄感所獲。令為文檄,詞甚不遜。尋背玄感還東都,帝銜之,車裂於高陽。



 少子福獎,通事舍人。在東都,與玄感戰沒。



 世康兄洸,字世穆。性剛毅,有器幹,少便弓馬。仕周,釋褐直寢上士。數從征伐,累遷開府,賜爵衛國縣公。隋文帝
 為丞相,從季父孝寬擊尉遲迥於相州,以功拜柱國,進襄陽郡公。時突厥寇邊,皇太子屯咸陽,令洸統兵出原州道。與虜相遇,擊破之。拜江陵總管,俄拜安州總管。伐陳之役,為行軍總管。及陳平,拜江州總管。略定九江,遂進圖嶺南。上與書慰勉之。洸至廣州,嶺表皆降之。上聞而大悅,許以便宜從事。洸所綏集二十四州,拜廣州總管。歲餘,番禺夷王仲宣反,以兵圍洸,洸拒之,中流矢卒。贈上柱國,賜綿絹萬段,謚曰敬。



 子協,字欽仁。好學有雅量,位秘書郎。其父在廣州有功,上命協齎詔書勞問,未至而父卒。上以其父死王事,拜協柱國,歷定、息、秦三州
 刺史,有能名。卒官。



 洸弟瓘,字世恭。御正下大夫,儀同三司、行隨州刺史。



 瓘弟藝,字世文。周武帝時,以軍功位上儀同,賜爵修武縣侯,授左旅下大失,出為魏郡太守。及隋文帝為丞相,尉遲迥陰圖不軌,朝廷遣藝季父孝寬馳往代迥。



 孝寬將至鄴,詐病止傳舍,從迥求藥,以密觀變。藝因投孝寬,即從孝寬擊迥。以功進位上大將軍,改封武威縣公,以修武縣侯別封一子。文帝受禪,進封魏興郡公,拜齊州刺史。為政通簡,士庶懷惠。遷營州總管。藝容貌瑰偉,每夷狄參謁,必整儀衛,盛服以見之,獨坐滿一榻。蕃人畏
 懼,莫敢仰視。而大修產業,與北夷貿易,家資巨萬。頗為清論所譏。卒官。謚曰懷。



 藝弟沖,字世沖。以名家子,在周釋褐衛公府禮曹參軍。從大將軍元定度江伐陳,為陳人所虜。周武帝以幣贖還之。帝復令沖以馬千匹使陳,贖開府賀拔華等五十人及元定之柩而還。沖有辭辯,奉使稱旨。累遷小御伯下大夫,加上儀同,拜汾州刺史。



 隋文帝踐阼,徵兼散騎常侍,進位開府,賜爵安固縣侯。歲餘,發南汾州胡千餘人北築長城,在途皆亡。上呼沖問計,沖曰:「皆由牧宰不稱所致,請以理綏靜,可不勞兵而定。」上因命沖綏懷叛者,月餘,並赴長城。上降書勞勉
 之。尋拜石州刺史,甚得諸胡歡心。以母憂去職。俄起為南寧州總管,持節撫慰,復遣柱國王長述以兵繼進。沖既至南寧,渠帥首領皆詣府參謁。上大悅,下詔褒揚之。其兄子伯仁隨沖在府,掠人之妻,士卒縱暴,邊人失望。上聞之,大怒,令蜀王秀按其事。



 益州長史元巖性方正,按沖無所寬貸。竟坐免官。其弟太子洗馬世約譖巖於皇太子。



 上謂太子曰:「古人云:『酤酒酸而不售者,為噬犬耳。』今何用世約乎!」世約遂除名。



 後令沖檢校括州事。時東陽賊帥陶子定、吳州賊帥羅慧方並聚眾為亂,沖率兵破之。改封義豐縣侯,檢校泉州事,遷營州總管。沖容
 貌都雅,寬厚得眾心,撫靺羯、契丹,皆能致其死力。奚、勣畏懼,朝貢相續。高麗嘗入寇,沖擊走之。及文帝為豫章王暕納沖女為妃,徵拜戶部尚書。卒官。少子挺知名。



 韋瑱,字世珍,京兆杜陵人也。世為三輔著姓。曾祖惠度,姚泓尚書郎。隨劉義真過江,仕宋為順陽太守、行南雍州事。後於襄陽歸魏,拜中書侍郎,贈洛州刺史。祖千雄,略陽郡守。父英,代郡守,贈兗州刺史。瑱幼聰敏,有夙成之量。起家太尉府法曹參軍,累遷諫議大夫。周文帝為丞相,封長安縣男。轉行臺左丞,遷南郢州刺史,復令為行臺左丞。瑱明察有幹局,再居左轄,時論榮之。從復弘
 農,戰沙苑,加衛大將軍、左光祿大夫。從戰河橋,進爵為子。大統八年,齊神武侵汾、絳,瑱從周文禦之。軍還,以本官鎮蒲津關,帶中水單城主。歷鴻臚卿。以望族兼領鄉兵,加帥都督,進散騎常侍。



 魏恭帝二年,賜姓宇文氏。三年,除瓜州刺史。州通西域,蕃夷往來,前後刺史多受賂遺,胡寇犯邊,又莫能禦。瑱雅性清儉,兼有武略,蕃夷贈遺,一無所受。



 胡人畏威,不敢為寇。公私安靜,夷夏懷之。周孝閔帝踐祚,進爵平齊縣伯。秩滿還京,吏人戀慕,老幼追送,留連十數日方得出境。明帝嘉之,授侍中、驃騎大將軍、開府儀同三司。卒,贈岐、宜二州刺史,謚曰惠。又追
 封為公,詔其子峻襲。



 峻位至車騎大將軍、儀同三司。峻子德政,隋大業中給事郎。峻弟師。



 師字公穎。少沈謹,有至性。初就學,始讀《孝經》,捨書而歎曰:「名教之極,其在茲乎!」少丁父母憂,居喪盡禮,州里稱其有孝行。及長,略涉經史,尤工騎射。周大冢宰宇文護引為中外府記室,轉賓曹參軍。師雅知諸蕃風俗及山川險易,其有夷狄朝貢,師必接對,論其國俗,如視諸掌。夷人驚服,無敢陷情。齊王憲為雍州牧,引為主簿,本官如故。及武帝親總萬機,轉少府大夫。及齊平,詔師安撫山東。徙為賓部大夫。隋文帝受禪,拜吏部侍郎,賜爵井
 陘侯。遷河北道行臺兵部尚書。奉詔為山東、河南十八州安撫大使。奏事稱旨,兼領晉王廣司馬。



 其族人世康為吏部尚書,與師素懷勝負。于時廣為雍州牧,盛存望第,以司空楊雄、尚書左僕射高熲並為州都,引師為主簿,而世康弟世約為法曹從事。世康恚恨不能食,又恥世約在師之下,召世約數之曰:「汝何故為從事!」遂杖之。



 後從上幸醴泉宮,上召師與左僕射高熲、上柱國韓擒等於臥內賜宴,令各敘舊事,以為笑樂。平陳之役,以本官領元帥掾。陳國府藏,悉委於師,秋毫無犯,稱為清白。後上為長寧王儼納其女為妃。除汴州刺史,甚有政
 名。卒官,謚曰定。



 師宗人暮,仕周,位內史大夫。隋文帝初,以定策功,累遷上柱國,封普安郡公。開皇初,卒於蒲州刺史。



 柳虯,字仲盤,河東解人也。五世祖恭,仕後趙為河東郡守。後以秦、趙喪亂,率人南徙,居汝、潁間,遂仕江表。祖緝,宋司州別駕、宋安郡守。父僧習,善隸書,敏於當世。與豫州刺史裴叔業據州歸魏,歷北地潁川二郡守、揚州大中正。虯年十三,便專精好學。時貴游子弟就學者,並車服華盛,唯虯不事容飾。遍受五經,略通大義,兼涉子史,雅好屬文。孝昌中,揚州刺史李憲舉虯秀才,兗州刺史馮
 俊引虯為府主簿。既而樊子鵠為吏部尚書,其兄義為揚州刺史,乃以虯為揚州中從事,加鎮遠將軍。非其好也,並棄官還洛陽。屬天下喪亂,乃退耕於陽城,有終焉之志。



 大統三年,馮翊王元季海、領軍獨孤信鎮洛陽。于時舊京荒廢,人物罕存,唯有虯在陽城,裴諏在潁川。信等乃俱徵之,以虯為行臺郎中,諏為北府屬,並掌文翰。時人為之語曰:「北府裴諏,南府柳虯。」時軍旅務殷,虯勵精從事,或通夜不寢。季海常云:「柳郎中判事,我不復重看。」四年入朝,周文帝欲官之,虯辭母老,乞侍醫藥。周文許焉。又為獨孤信開府從事中郎。信出鎮隴右,因為秦
 州刺史,以虯為二府司馬。雖處元僚,不綜府事,唯在信左右談論而已。因使見周文,被留為丞相府記室。追論歸朝功,封美陽縣男。



 虯以史官密書善惡,未足懲勸,乃上疏曰:「古者人君立史官,非但記事而已,蓋所為鑒誡也。動則左史書之,言則右史書之,彰善癉惡,以樹風聲。故南史抗節,表崔杼之罪;董狐書法,明趙盾之愆。是知執筆於朝,其來久矣。而漢、魏已還,密為記注,徒聞後世,無益當時。非所謂將順其美,匡救其惡者。且著述之人,密書縱能直筆,人莫知之。何止物生橫議,亦自異端互起。故班固致受金之名,陳壽有求米之論。著漢、魏者非
 一氏,造晉史者至數家。後代紛紜,莫知準的。伏惟陛下則天稽古,勞心庶政,開誹謗之路,納忠讜之言。諸史官記事者,請皆當朝顯言其狀,然後付之史閣。庶令是非明著,得失無陷,使聞善者日修,有過者知懼。」



 事遂施行。十四年,除秘書丞,領著作。舊丞不參史事,自虯為丞,始令監掌焉。



 遷中書侍郎,修起居注,仍領丞事。時人論文體者,有今古之異。虯又以為時有古今,非文有古今,乃為文質論。文多不載。廢帝初,遷秘書監,加車騎大將軍、儀同三司。



 虯脫略人間,不事小節,弊衣蔬食,未嘗改操。人或譏之。虯曰:「衣不過適體,食不過充飢,孜孜營求,徒
 勞思慮耳。」恭帝元年冬卒,時年五十四。贈兗州刺史,謚曰孝。有文章數十篇,行於世。子鴻漸嗣。虯弟檜。



 檜字季華。性剛簡,任氣少文,善騎射,果於斷決。年十八,起家奉朝請。居父喪,毀瘠骨立。服闋,除陽城郡丞、防城都督。大統四年,從周文戰於河橋,先登有功。授都督,鎮鄯州。八年,拜湟河郡守,仍典軍事。尋加平東將軍、太中大夫。吐谷渾入寇郡境,時檜兵少,人懷憂懼,檜撫而勉之,眾心乃安。因率數十人先擊之,渾人潰亂,餘眾乘之,遂大敗而走。以功封萬年縣子。時吐谷渾強盛,數侵疆埸,自檜鎮鄯州,屢戰必破之。數年之後,不敢為寇。十四
 年,遷河州別駕,轉帥都督。俄拜使持節、撫軍將軍、大都督。居三載,徵還京師。



 時檜兄虯為祕書丞,弟慶為尚書左丞。檜嘗謂兄弟曰:「兄則職典簡牘,褒貶人倫;弟則管轄九司,股肱朝廷。可謂榮寵矣。然而四方未靜,車書不一,檜唯當蒙矢石,履危難,以報國恩耳。」頃之,周文謂檜曰;』卿昔在鄯州,忠勇顯著。



 今西境肅清,無勞經略。九曲,國之東鄙,當勞君守之。」遂令檜鎮九曲。



 尋從大將軍王雄討上津、魏興,平之,即除魏興、華陽二郡守。安康人黃眾寶謀反,連結黨與,將圍州城,乃相謂曰:「常聞柳府君勇悍有餘,不可當。今既在外,方為吾徒腹心之疾也,不如
 先擊之。」遂圍檜郡。郡城卑下,士眾寡弱,又無守禦之備。連戰積十餘日,士卒僅有存者。於是力屈城陷,身被十餘創,遂為賊所獲。既而眾寶等進圍東梁州,乃縛檜置城下,欲令誘城中。檜乃大呼曰:「群賊烏合,糧食已罄,行即退散,各宜勉之!」眾寶大怒,乃臨檜以兵曰:「速更汝辭!



 不爾便就戮矣。」檜守節不變,遂害之,棄屍水中。城中人皆為之流涕。眾寶解圍之後,檜兄子止戈方收檜屍還長安。贈東梁州刺史。子斌嗣。



 斌字伯達。年十七,齊公憲召為記室。早卒。



 斌弟雄亮,字信誠。父檜在華陽見害,雄亮時年十四,哀毀過禮,陰有
 復讎之志。武帝時,眾寶率其部歸長安,帝待之甚厚。雄亮手斬眾寶於城中,請罪闕下。



 帝特原之。後累遷內史中大夫,賜爵汝陽縣子。隋文帝受禪,拜尚書考功侍郎,遷給事黃門侍郎。尚書省凡所奏事,多所駮正,深為公卿所憚。俄以本官檢校太子左庶子,進爵為伯。秦王俊鎮隴右,出為秦州總管府司馬,領山南道行臺左丞。卒。



 子贊嗣。



 檜弟鷟,好學善屬文,卒於魏臨淮王記室參軍事。



 子帶韋,字孝孫。深沈有度量,少好學,身長八尺三寸,美風儀,善占對。周文辟為參軍事。侯景作亂江南,周文令帶韋使江、郢二州,與梁邵陵、南平二王通好。行至安
 州,遇段寶等反,帶韋乃矯為周文書以安之,並即降附。及見邵陵,具申周文意。邵陵遣使隨帶韋報命。以奉使稱旨,授輔國將軍、中散大夫。



 後達奚武經略漢川,以帶韋為行臺左丞,從軍南討。時梁宜豐侯蕭修守南鄭,武攻之未拔,乃令帶韋入城,說修降之。廢帝元年,出為解縣令。加授驃騎將軍、左光祿大夫。轉汾陰令。發摘姦伏,百姓畏而懷之。周武成元年,授武藏下大夫。



 天和二年,封康城縣男。累遷兵部中大夫。雖頻改職,仍領武藏。五年,轉武藏中大夫。俄遷驃騎大將軍、開府儀同三司。凡居劇職十有餘年,處斷無滯,官曹清肅。



 時譙王儉為益
 州總管,漢王贊為益州刺史。武帝以帶韋為益州總管府長史,領益州別駕,輔弼二王,總知軍事。及大軍東討,徵為前軍總管齊王憲府長史。齊平,以功授上開府儀同大將軍,進爵為公。陳王純鎮並州,以帶韋為並州司會、并州總管府長史。卒官,謚曰愷。



 子祚嗣。少有名譽,位宣納上士。入隋,位司勳侍郎。



 鷟弟慶。慶字更興。幼聰敏有器量,博涉群書,不為章句,好飲酒,閑於占對。



 年十三,因暴書,父僧習試令慶於雜賦集中取賦一篇千餘言,誦之。慶立讀三遍,便誦之無所漏。時僧習為潁川郡守,地接都畿,人多豪右。將選鄉官,皆依
 貴勢,競來請託。選用既定,僧習謂諸子曰:「權貴請託,吾並不用。其使欲還,皆須有答。汝各以意為吾作書。」慶乃具書草。僧習讀,歎曰:「此兒有意氣,丈夫理當如是。」即依慶所草以報。起家奉朝請。



 慶出後第四叔,及遭父憂,議者不許為服重。慶泣曰:「禮緣人情,若於出後之家,更有苴斬之服,可奪此以從彼。今四叔薨背已久,情事不追。豈容奪禮,乖違天性!」時論不能抑,遂以苫塊終喪。既葬,乃與諸兄負土成墳。



 孝武將西遷,慶以散騎侍郎馳傳入關。慶至高平,見周文,共論時事。周文即請奉迎輿駕,仍令慶先還復命。時賀拔勝在荊州,帝屏左右謂慶
 曰:「朕欲往荊州,何如?」慶曰:「關中金城千里,天下之彊國也。荊州地無要害,寧足以固鴻基?」



 帝納之。及帝西遷,慶以母老不從。獨孤信之鎮洛陽,乃得入關。除相府東閣祭酒。



 大統十年,除尚書都兵郎中,并領記室。時北雍州獻白鹿,群臣欲賀。尚書蘇綽謂慶曰:「近代已來,文章華靡,逮於江左,彌復輕薄。洛陽後進,祖述未已。



 相公柄人軌物,君職典文房,宜製此表,以革前弊。」慶操筆立成,辭兼文質。綽讀而笑曰:「枳橘猶自可移,況才子也!」



 尋以本官領雍州別駕。廣陵王欣,魏之懿親。其甥孟氏,屢為兇橫。或有告其盜牛。慶捕得實,趣令就禁。孟氏殊無懼容,
 乃謂慶曰:「若加以桎梏,後獨何以脫之?」欣亦遣使辨其無罪。孟氏由此益驕。慶乃大集僚吏,盛言孟氏倚權侵虐之狀。言畢,令笞殺之。此後貴戚斂手。



 有賈人持金二十斤詣京師,寄人居止。每欲出行,常自執管鑰。無何,緘閉不異而並失之。謂主人所竊。郡縣訊問,主人自誣服。慶疑之,乃召問賈人曰:「卿鑰恆置何處?」對曰:「恆自帶之。」慶曰:「頗與人同宿乎?」曰:「無。」



 「與同飲乎?」曰:「日者曾與一沙門再度酣宴,醉而晝寢。」慶曰:「沙門乃真盜耳。」即遣捕沙門,乃懷金逃匿。後捕得,盡獲所失金。十二年,改三十六曹為十二部,以慶為計部郎中,別駕如故。



 又有胡家被
 劫,郡縣按察,莫知賊所,鄰近被囚者甚多。慶以賊是烏合,可以詐求之。乃作匿名書,多榜官門曰:「我等共劫胡家,徒侶混雜,終恐泄露。今欲首伏,懼不免誅。若聽先首免罪,便欲來告。」慶乃復施免罪之牒。居二日,廣陵王欣家奴面縛自告牒下,因此盡獲黨與。慶之守正明察,皆此類也。每歎曰:「或于公斷獄無私,闢高門以待封。儻斯言有驗,吾其庶幾乎。」封清河縣男,除尚書左丞,攝計部。



 周文嘗怒安定國臣王茂,將殺之,而非其罪。朝臣咸知,而莫敢諫。慶乃進爭之。周文逾怒曰:「卿若明其無罪,亦須坐之。」乃執慶于前。慶辭氣不撓,抗聲曰:「竊聞君有不
 達者為不明。臣有不爭者為不忠。慶謹竭愚誠,實不敢愛死,但懼公為不明之君耳。」周文乃悟而赦茂,已不及矣。周文默然。明日,謂慶曰:「吾不用卿言,遂令王茂冤死。可賜茂家錢帛,以旌吾過。」尋進爵為子。慶威儀端肅,樞機明辯。周文每發號令,常使慶宣之。天性抗直,無所回避。周文亦以此深委仗焉。恭帝初,進位驃騎大將軍、開府儀同三司、尚書右僕射,轉左僕射,領著作。六官建,拜司會中大夫。



 周孝閔帝踐祚,賜姓宇文氏,進爵平齊縣公。晉公護初執政,欲引為腹心。慶辭之,頗忤旨。又與楊寬有隙,及寬參知政事,慶遂見疏忌,出為萬州刺史。明
 帝尋悟,留為雍州別駕,領京兆尹。武成二年,除宜州刺史。慶自為郎,迄為司會,府庫倉儲,並其職也。及在宜州,寬為小冢宰,乃囚慶故吏,求其罪失。案驗積六十餘日,吏或有死於獄者,終無所言,唯得乘錦數匹。時人服其廉慎。又入為司會。



 先是,慶兄檜為魏興郡守,為賊黃眾寶所害。檜子三人皆幼弱,慶撫養甚篤。



 後眾寶歸朝,朝廷待以優禮。居數年,檜次子雄亮白日手刃眾寶於長安城中。晉公護聞而大怒,執慶諸子姪皆囚之,讓慶擅殺人。對曰:「慶聞父母之讎不同天,昆弟之讎不同國。明公以孝臨天下,何乃責於此乎?」護逾怒,慶辭色無屈,竟
 以俱免。卒。贈鄜、綏、丹三州刺史,謚曰景。子機嗣。



 機字匡時。偉容儀,有器局,頗涉經史。年十九,周武帝時為魯公,引為記室。



 及帝嗣位,累遷太子宮尹,封平齊縣公。宣帝時,為御正上大夫。機見帝失德,屢諫不聽,恐禍及己,託於鄭譯,求出,拜華州刺史。及隋文帝作相,徵還京師。時周代舊臣皆勸禪讓,機獨義形於色,無所陳請。俄拜衛州刺史。及踐祚,進爵建安郡公,徵為納言。機性寬簡,有雅望,當近侍,無所損益。又好飲酒,不親細務。



 數年,出為華州刺史,奉詔每月朝見。尋轉冀州刺史。後徵入朝,以其子述尚蘭陵公主,禮遇益隆。初,機在周,與族
 人文城公昂俱歷顯要,及此,昂、機並為外職。



 楊素時為納言,方用事,因上賜宴,素戲曰:「二柳俱摧,孤楊獨聳。」坐者歡笑,機竟無言。未幾還州。前後作守,俱稱寬惠。後以徵還,卒于家。贈大將軍、青州刺史,謚曰簡。子述嗣。



 述字業隆。性明敏,有幹略,頗涉文藝。以父蔭為太子親衛。後以尚主故,拜開府儀同三司、內史侍郎。上於諸婿中特見寵遇。歲餘,判兵部尚書事。父艱去職。



 未幾,起攝給事黃門侍郎事,襲爵建安郡公。



 仁壽中,判吏部尚書事。述雖職務修理,為當時所稱,然不達大體,暴於馭下,又怙寵驕豪,無所降屈。楊素時方貴重,朝臣莫不讋憚,
 述每陵侮之,數於上前面折素短。判事有不合,素意或令述改,輒謂將命者曰:「語僕射,道尚書不肯。」



 素由是銜之。俄而楊素被疏忌,不知省事。述任寄逾重,拜兵部尚書,參掌機密。



 述自以無功可紀,過叨匪服,抗表陳讓。上許之,命攝兵部尚書。



 上於仁壽宮寢疾,述與楊素、黃門侍郎元巖等侍疾宮中。時皇太子無禮於陳貴人,上知之,大怒,令述召房陵王。述與元巖出外作敕書。楊素見之,與皇太子謀,矯詔執述、巖屬吏。及煬帝嗣位,述坐除名。公主請與同徙,帝不聽。述在龍川數年,復徙寧越,遇瘴癘死。



 機弟弘,字匡道。少聰穎,工草隸,博涉群書,辭采雅贍。與弘農楊素為莫逆交。解巾中外府記室。建德初,除內史上士。歷小宮尹、御正上士。陳遣王偃人來聘,武帝令弘勞之。偃人謂弘曰:「來日至藍田,正逢滋水暴長,所賚國信,溺而從流。今所進,假之從吏。請勒下流人見為尋此物。」弘曰:「昔淳于之獻空籠,前史稱以為美。足下假物而進,詎是陳君命乎?」偃人慚不能對。武帝聞而嘉之,盡以偃人所進物賜弘,仍令報聘。占對敏捷,見稱於時。後卒於御正下大夫。贈晉州刺史。楊素誄之曰:「山陽王弼,風流長逝;潁川荀粲,零落無時。修竹夾池,永絕梁園之賦;
 長楊映沼,無復洛川之文。」其為士友所痛惜如此。有文集行於世。



 弘弟旦,字匡德。工騎射,頗涉書籍。仕周,位兵部下大夫。以行軍長史從梁睿討王謙,以功授儀同三司。開皇元年,加開府,封新城縣男,授掌設驃騎。歷羅、淅、魯三州刺史,並有能名。大業初,拜龍川太守。郡人居山洞,好相攻擊。旦為開設學校,大變其風。帝聞,下詔褒美之。徵為太常少卿,攝判黃門侍郎事。卒。



 子燮,官至河內郡掾。



 旦弟肅,字匡仁。少聰敏,閑於占對。仕周,位宣納上士。隋文帝作相,引為賓曹參軍。開皇初,授太子洗馬。陳使謝
 泉來聘,以才學見稱,詔肅宴接,時論稱其華辯。歷太子內舍人,遷太子僕。太子廢,坐除名。大業中,帝與段達語及庶人罪惡。達云:「柳肅在宮,大見疏斥。」帝問其故。答曰:「學士劉臻嘗進章仇太翼宮中,為巫蠱事。肅知而諫曰:『殿下位當儲貳,戒在不孝,無患見疑。劉臻書生,鼓搖脣舌,適足以相詿誤。願勿納之。』庶人不懌,他日,謂臻曰:『汝何漏泄,使柳肅知之,令面折我!』自是後,言皆不用。」帝曰:「肅橫除名。」乃召守禮部侍郎。坐事免。後守工部侍郎,大見親任,每幸遼東,常委於涿郡留守。卒官。



 機從子謇之,字公正。父蔡年,周順州刺史。謇之身長七
 尺五寸,儀容甚偉,風神爽亮,進止可觀。為童兒時,周齊王憲遇之於途,異而與語,大奇之,因奏為國子生。以明經擢第,拜宮師中士,轉守廟下士。武帝有事太廟,謇之讀祝文,音韻清雅,觀者屬目。帝善之,擢為宣納上士。開皇初,拜通事舍人,尋遷內史舍人。



 歷兵部、司勳二曹侍郎。朝廷以謇之雅望,善談謔,又飲酒至一石不亂,由是,每梁陳使至,輒令接對。遷光祿少卿。出入十餘年,每參掌敷奏。



 會吐谷渾來降,朝廷以宗女光化公主妻之,以謇之兼散騎常侍,送公主於西域。



 及突厥啟人可汗求和親,復令謇之送義成公主於突厥。前後使二國,得贈馬
 二千餘匹,雜物稱是,皆散之宗族,家無餘財。出為肅、息二州刺史,俱有惠政。煬帝踐祚,復拜光祿。大業初,啟人可汗自以內附,遂畜牧於定襄、馬邑間。帝使謇之諭令出塞。還,拜黃門侍郎。



 時元德太子初薨,朝野注望,以齊王當立。帝方重王府之選,拜為齊王長史。



 帝法服臨軒,命齊王立於西朝堂,遣吏部尚書牛弘、內史令楊約、左衛大將軍宇文述等從殿廷引謇之詣齊王所,西面立。弘宣敕謂齊王曰:「我出蕃之初,時年十二。



 先帝立我於西朝堂,乃令高熲、虞慶則、元旻等從內送王子相於我。誡我曰:『以汝未更世事,令子相作輔於汝,事無大小,皆
 可委之。無得暱近小人,疏遠子相。



 若從我言者,有益於社稷,成立汝名行;如不用此言,唯國及身,敗無日矣。』吾受敕,奉以周旋,不敢失墜。微子相之力,吾幾無今日矣。若與謇之從事,一如子相也。」又敕謇之曰:「今以卿作輔於齊,副朕所望。若齊王德業脩備,富貴自當鍾卿一門。若有不善,罪亦相及。」時齊王擅寵,喬令則之徒,深見暱狎,謇之知其非,不能匡正。及王得罪,謇之竟坐除名。及帝幸遼東,召檢校燕郡事。帝班師至燕郡,坐供頓不給,配戍嶺南,卒於洭口。子威明。



 論曰:高氏藉四胡之勢,跨有山東,周文承二將之餘,創
 基關右,似商、周之不敵,若漢、楚之爭雄。又連官渡之兵,未定鴻溝之約。雖弘農、沙苑,齊卒先奔;而河橋、北芒,周師橈敗。於是競圖進取,各務兵戈,齊謂兼並有餘,周則自守不足。韋孝寬奇材異度,緯武經文,居要害之地,受干城之託。東人怙恃其眾,悉力來攻,將欲釃酒未央,飲馬清渭。孝寬乃馮茲雉堞,抗彼仇讎,事甚析骸,勢危負戶,終能奮其智勇,應變無方,城守六旬,竟摧大敵。齊人既焚營宵遁,高氏遂憤恚而殂。雖即墨破燕,晉陽存趙,何以能尚?若使平陽不守,鄴城無眾人之師;玉壁啟關,函谷失封泥之固。斯豈一城之得喪,實亦二國之興亡
 者歟。韋夐陷不負人,貞不絕俗,怡神墳籍,養素丘園,哀樂無以動其心,名利不足乾其慮,確乎不拔,實近代之高人也。明帝比諸園、綺,豈徒然哉!世康風神雅量,一代稱偉,簪纓人物,見重京華。瑱素望高風,亦云美矣。柳虯兄弟,雅道是基,並能譽重搢紳,豈虛至也。慶束帶立朝,匪躬是蹈,蒞官從政,清白著美。至於畏避權寵,違忤宰臣,雖取詘於一時,實獲申於千載矣。機立身行已,本以寬雅流譽,至於登朝正色,可謂不違直道。雖陵穀遷貿,終以雅正自居,古所謂以道事人,斯之謂矣。述雖幹略見稱,終乃敗於驕寵,惜矣。



\end{pinyinscope}