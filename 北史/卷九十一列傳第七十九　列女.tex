\article{卷九十一列傳第七十九 列女}

\begin{pinyinscope}

 魏
 崔覽妻封氏封卓妻劉氏魏溥妻房氏胡長命妻張氏平原女子孫氏房愛親妻崔氏涇州貞女兒氏姚氏婦楊氏張洪祁妻劉氏董景起妻張氏陽尼妻高氏史映周妻耿氏
 任城國太妃孟氏茍金龍妻劉氏貞孝女宗河東姚氏女刁思遵妻魯氏西魏孫道溫妻趙氏孫神妻陳氏隋蘭陵公主南陽公主襄城王恪妃華陽王楷妃譙國夫人洗氏鄭善果母崔氏孝女王舜韓覬妻于氏陸讓母馮氏劉昶女鐘士雄母蔣氏孝婦覃氏元務光母盧氏
 裴倫妻柳氏趙元楷妻崔氏蓋婦人之德,雖在於溫柔,立節垂名,咸資於貞烈。溫柔仁之本也,貞烈義之資也。非溫柔無以成其仁,非貞烈無以顯其義。是以《詩書》所記,風俗所存,圖象丹青,流聲竹素。莫不守約以居正,殺身以成仁者也。若文伯、王陵之母,白公、杞殖之妻,魯之義姑,梁之高行,衛君靈王之妾,夏侯文寧之女,或抱信以會真,或蹈忠而踐義,不以存亡易心,不以盛衰改節,其佳名彰於既沒,徽音傳於不朽,不亦休乎!或有王公大人之妃,偶肆情於淫僻之俗,雖衣文衣,食珍膳,坐金屋,乘玉輦,不入彤管之書,不
 沾青史之筆,將草木以俱落,與麋鹿而同死者,可勝道哉!永言載思,實庶姬之恥也。



 魏隋二書,並有《列女傳》,齊周並無此篇。今又得武孫道溫妻趙氏、河北孫神妻陳氏,附魏、隋二傳,以備《列女篇》云。



 魏中書侍郎清河崔覽妻封氏者,勃海人,散騎常侍封愷女也。有才識,聰辯強記,多所究知。時李敷、公孫文叔雖已貴重,近世故事有所不達者,皆就而諮請焉。



 勃海封卓妻劉氏者,彭城人也。成婚一夕,卓官於京師,後以事伏法。劉氏在家,忽然夢想,知卓已死,哀泣,嫂喻之不止。經旬,凶問果至,遂憤歎而死。時人比之秦嘉妻。
 中書令高允念其義高而名不著,為之詩曰:兩儀正位,人倫肇甄。爰制夫婦,統業承先。雖曰異族,氣猶自然。生則同室,終契黃泉。其一封生令達,卓為時彥,內協黃中,外兼三變。誰能作配,克應其選,實有華宗,挺生淑媛。其二京野勢殊,山川乖互,乃奉王命,載馳在路。公務既弘,私義獲著,因媒致幣,遘止一幕。其三率我初冠,眷彼弱笄,形由禮比,情以趣諧。忻願難常,影跡易乖,悠悠言邁,戚戚長懷。其四時遇險迍,橫罹塵綱,伏質就刑,身分土壤。千里雖遐,應如影響,良嬪洞感,發於夢想。其五仰惟親命,俯尋嘉好,誰謂會淺,義深情到。畢志守窮,誓不二醮,何以驗之?



 殞身是效。其六人之處世,孰不厚生?必存於義,所重則輕。結憤鐘心,甘就幽冥,永捐堂宇,長辭母兄。其七芒芒中野,翳翳孤丘,葛蕾冥蒙,荊棘四周,理茍不昧,神必俱遊。異哉貞婦,曠世靡儔。其八鉅鹿魏溥妻房氏者,慕容垂貴鄉太守常山房湛女也。幼有烈操。年十六而溥遇疾,且卒,顧謂之曰:「死不足恨,但痛母老家貧,赤子蒙眇,抱怨於黃壚耳!」



 房垂泣而對曰:「幸承先人餘訓,出事君子,義在偕老,有志不從,蓋其命也。今夫人在堂,弱子衣強褓,顧當以身少相感,永深長往之恨。」俄而溥卒。及將大斂,房氏操刀割左耳,投之棺
 中,仍曰:「鬼神有知,相期泉壤。」流血滂然,助喪者哀懼。姑劉氏輟哭謂曰:「新婦何至於此?」對曰:「新婦少年,不幸早寡,實慮父母未量至情,覬持此自誓耳。」聞知者莫不感愴。



 於時,子緝生未十旬,鞠育於後房之內,未嘗出門。遂終身不聽絲竹,不預座席。緝年十二,房父母仍存,於是歸寧,父兄尚有異議。緝竊聞之,以啟其母。房命駕,紿云他行,因而遂歸。其家弗之知也。行數十里,方覺,兄弟來追,房哀嘆而不反。其執意如此。訓導一子,有母儀法度。緝所交遊,有名勝者,則身具酒饌;有不及己者,輒屏臥不飧,須其悔謝,乃食。善誘嚴訓,類皆如是。年六十五
 而終。



 緝子悅後為濟陰太守,吏民立碑頌德。金紫光祿大夫高閭為其文曰:「爰及處士,遘疾夙凋,伉儷秉志,識茂行高,殘形顯操,誓敦久要。」溥未仕而卒,故云處士焉。



 樂部郎胡長命妻張氏者,不知何許人也。事姑王氏甚謹。太安中,京師禁酒。



 張以姑老且患,私為醖之,為有司所糾。王氏詣曹自首,由己私釀。張氏曰:「姑老抱患,張主家事,姑不知釀。」主司不知所處。平原王陸麗以狀奏,文成義而赦之。



 平原鄃縣女子孫氏男玉者,夫為零陵縣人所殺。男玉追執仇人,欲自殺之。其弟止而不聽。男玉曰:「女人出適,
 以夫為天,當親自復雪,云何假人之手?」遂以杖毆殺之。有司處死,以聞。獻文詔曰:「男玉重節輕身,以義犯法,緣情定罪,理在可原,其特恕之。」



 清河房愛親妻崔氏者,同郡崔元孫之女也。性嚴明,有高節,歷覽書傳,多所聞知。親授子景伯、景光《九經》義,學行脩明,並當世名士。景伯為清河太守,每有疑獄,常先請焉。貝丘人列子不孝,吏欲案之,景伯為之悲傷,入白其母。母曰:「吾聞聞名不如見面,小人未見禮教,何足責哉!但呼其母來,吾與之同居,其子置汝左右,令其見汝事吾,或應自改。」景伯遂召其母,崔氏處之於榻,與之共
 食。景伯為之溫凊。其子侍立堂下,未及旬日,悔過求還。崔氏曰:「此雖顏慚,未知心愧,且可置之。」凡經二十餘日,其子叩頭流血,其母涕泣乞還,然後聽之,終以孝聞。其識度勵物如此。竟以壽終。



 涇州貞女兒氏者,許嫁彭老生為妻。聘幣既畢,未及成禮。兒氏率行貞淑,居貧,常自舂汲,以養父母。老生輒往逼之,女曰:「與君聘命雖畢,二門多故,未及相見,何由不稟父母,擅見陵辱!若茍行非禮,正可身死耳!」遂不肯從。老生怒而刺殺之,取其衣服。女尚能言,臨死謂老生曰:「生身何罪,與君相遇!我所以執節自固者,寧更有所邀,
 正欲奉給君耳。今反為君所殺,若魂靈有知,自當相報。」言終而絕。老生持女衣服珠纓,至其叔宅,以告。叔曰:「此是汝婦,奈何殺之,天不祐汝!」遂執送官。太和七年,有司劾以死罪。詔曰:「老生不仁,侵陵貞淑,原其強暴,便可戮之,而女守禮履節,沒身不改,雖處草莽,行合古跡。



 宜賜美名,以顯風操,其標墓旌善,號曰『貞女』」。



 姚氏婦楊氏者,閹人苻承祖姨也。家貧。及承祖為文明太后所寵貴,親姻皆求利潤,唯楊獨不欲。常謂其姊曰:「姊雖有一時之榮,不若妹有無憂之樂。」姊每遺其衣服,多不受。強與之,則云:「我夫家世貧,好衣美服則使人不
 安。」與之奴婢,云:「我家無食,不能供給。」終不肯受。常著破衣,自執勞事。時受其衣服,多不著,密埋之。設有著者,汙之而後服。承祖每見其寒悴,深恨其家,謂不供給之。乃啟其母曰:「今承祖一身,何所乏少,而使姨如是?」母具以語之。承祖乃遣人乘車往迎之,則厲志不起。遣人強輦於車上,則大哭言:「爾欲殺我也!」



 由是苻家內外,皆號為癡姨。及承祖敗,有司執其二姨至殿庭致法,以姚氏婦衣裳弊陋,特免其罪。其識機,雖呂嬃亦不如也。



 滎陽京縣人張洪祁妻劉氏者,年十七夫亡。遺腹生一子,三歲又沒。其舅姑年老,朝夕奉養,率禮無違。兄矜其
 少寡,欲奪嫁之,劉自誓不許,以終其身。



 陳留董景起妻張氏者,景起早亡,張時年十六,,痛夫少喪,哀傷過禮,蔬食長齋。又無兒息,獨守貞操,期以闔棺。鄉曲高之,終見標異。



 漁陽太守陽尼妻高氏者,勃海人也。學識有文翰,孝文敕令入侍後宮。幽后表啟,悉其辭也。



 滎陽史映周妻耿氏者,同郡耿氏女也。年十七,適於映周。太和二十三年,映周卒,耿氏恐父母奪其志,因葬映周,哀哭而殞。見者莫不悲嘆。屬大使觀風,以狀具上,詔標門閭。



 任城國太妃孟氏者,鉅鹿人,尚書、任城王澄之母也。澄為揚州之日,率眾出討。於後賊帥姜慶真陰結逆黨,襲陷羅城。長史韋纘倉卒,孟乃勒兵登陴,激厲文武,喻之逆順。於是咸有奮志,賊不能克,卒以全城。靈太后後敕有司樹碑旌美。



 梓潼太守茍金龍妻劉氏者,平原人也,廷尉少卿劉叔宗之姊也。宣武時,金龍為郡,帶關城戍主。梁人攻圍,會金龍疾病,不堪部分,劉遂厲城人脩理戰具,夜悉登城拒戰,百有餘日,兵士死傷過半。戍副高景陰圖叛逆,劉與城人斬景及其黨與數十人。自餘將士,分衣減食,勞
 逸必同,莫不畏而懷之。井在外城,尋為賊陷,城中絕水,渴死者多。劉乃集諸長幼,喻以忠節,遂相率告訴於天,俱時號叫,俄而澍雨。劉命出公私布絹及至衣服,懸之城內,絞而取水,所有雜器,悉儲之。於是人心益固。會益州刺史傅豎眼將至,梁人乃退。豎眼嘆異之,具狀奏聞。宣武嘉之。正光中,賞其子慶珍平昌縣子,又得二子出身。



 貞孝女宗者,趙郡柏人人,趙郡太守李叔胤之女,范陽盧元禮之妻也。性至孝,父卒,號慟幾絕者數四,賴母崔氏慰勉之,得全。三年之中,形骸銷瘠,非人不起。



 及歸夫
 氏,與母分隔,便飲食日損,涕泣不絕,日就羸篤。盧氏合家慰喻,不解。



 因遣歸寧還家,乃復故。如此者八九焉。及元禮卒,李追亡撫遺,事姑以孝謹著。



 母崔終於洛陽,凶問初到,舉聲慟絕,一宿乃蘇,水漿不入口者六日。其姑慮其不濟,親送奔喪,而氣力危殆,自范陽向都,八旬方達。攀櫬號踴,遂卒。有司以狀聞,詔追號貞孝女宗,易其里為孝德里,樹李、盧二門,以惇風俗。



 河東姚氏女者,字女勝。少喪父,無兄弟,母憐而守養。年六七歲,便有孝性,人言其父者,聞輒垂泣,鄰伍異之。正光中母死,勝年十五,哭泣不絕聲,水漿不入口者數日,
 不勝哀,遂死。太守崔遊申請為營墓立碑,自為制文,表其門閭,比之曹娥,改其里曰上虞里。墓在都城東六里,大道北,至今名為孝女塚。



 滎陽刁思遵妻者,魯氏女也。始笄為思遵所聘,未踰月而思遵亡。其家矜其少寡,許嫁已定。魯聞之,以死自誓。父母不達其志,遂經郡訴,稱刁氏吝護寡女,不使歸寧。魯乃與老姑徒步詣司徒府,自告情狀。普泰初,有司聞奏,節閔詔本司依式標榜。



 西魏武功縣孫道溫妻趙氏者,安平人也。萬俟醜奴之反,圍岐州,久之無援。



 趙乃謂城中婦女曰:「今州城方陷,
 義在同憂。」遂相率負土,晝夜培城,城竟免賊。大統六年,贈夫岐州刺史,贈趙安平縣君。



 河北孫神妻陳氏者,河北郡人也。神當遠戍,主吏配在夏州,意難其遠。有孤兄子,欲以自代。陳曰:「為國征戍,道路遼遠,何容身不肯行,以孤姪自代!天下物議,誰其相許?」神感其言,乃自行。在戍未幾,便喪。心彗柩至,陳望而哀慟,一哭而卒。文帝詔表其閭。



 隋蘭陵公主字阿五,文帝第五女也。美姿容,性婉順,帝於諸女中,特所鐘愛。



 初嫁儀同王奉孝。奉孝卒,適河東柳述,時年十八。諸姊並驕踞,主獨折節遵婦道,事舅姑
 甚謹,遇疾必親奉湯藥。帝聞之大悅,由是述漸見寵遇。初,晉王廣欲以主配其妃弟蕭瑒,文帝將許之,後遂適述,晉王因不悅。及述用事,彌惡之。文帝崩,述徙嶺表。煬帝令主與述離絕,將改嫁之。公主以死自誓,不復朝謁,表求免主號,與述同徙。帝大怒曰:「天下豈無男子,欲與述同徙邪?」主曰:「先帝以妾適柳家,今其有罪,妾當從坐。」帝不悅。主憂憤卒,時年三十二。臨終上表:生不得從夫死,乞葬柳氏。帝覽表愈怒,竟不哭,葬主於洪瀆川,資送甚薄。朝野傷之。



 南陽公主者,煬帝長女也。美風儀,有志節。十四嫁於許
 國公宇文述子士及,以謹厚聞。述病且卒,主親調飲食,手自奉上,世以此稱之。及宇文化及弒逆,公主隨至聊城,而化及為竇建德所敗,士及自濟北西歸大唐。時隋代衣冠引見建德,莫不惶懼失常,唯主神色自若。建德與語,主自陳國破家亡,不能報怨雪恥,淚上盈襟,聲辭不輟,情理切至。建德及觀聽者,莫不為之動容隕涕,咸敬異焉。及建德誅化及,時主有一子名禪師,年且十歲。建德遣武賁郎將於士證謂主曰:「宇文化及躬行弒逆,今將族滅其宗。公主之子,法當從坐,若不能割愛,亦聽留之。」



 主泣曰:「武賁既是隋室貴臣,此事何須見問?」建德
 竟殺之。公主尋請建德,剃髮為尼。及建德敗,將歸西京,復與士及遇於東都。主不與相見。士及就之,請復為夫妻。主拒曰:「我與君仇家,今恨不能手刃君者,以謀逆之際,君不預知耳。」



 固與告絕。士及固請,主怒曰:「必就死,可相見也!」士及知不可屈,乃拜辭而去。



 襄城王恪妃者,循州刺史柳旦女也。妃姿貌端麗,年十餘,以良家子合相,見聘為妃。未幾而恪被廢,妃修婦道,事之愈敬。煬帝嗣位,復徙邊,帝令使者殺之於道。恪與辭決,妃曰:「若王死,妾誓不獨生。」於是相對慟哭。恪死,棺斂訖,妃謂使者曰:「妾誓與楊氏同穴,若身死得不別埋,
 君之惠也。」遂撫棺號慟,自經而卒。見者莫不流涕。



 華陽王楷妃者,黃門侍郎、龍涸縣公河南元巖女也。巖明敏有器幹,煬帝嗣位,坐與柳述連事,除外徙南海。後會赦還長安,有人譖巖逃歸,收殺之。妃有姿色,性婉順,初以選為妃,未幾而楷被幽廢。妃事楷愈謹,每見楷有憂懼色,輒陳義理以慰諭之,楷甚敬焉。及江都之亂,楷遇害,宇文化及以妃賜其黨元武達。初以宗族禮之,置之別舍。後因醉而逼之,妃自誓不屈。武達怒,撻之百餘,詞色彌厲。



 元自毀其面,血淚俱下,武達釋之。妃謂其徒曰:「我不能早死致命,將見侵辱,我之罪也。」因不食而卒。



 譙國夫人洗氏者,高涼人也。世為南越首領,部落十餘萬家。夫人幼賢明,在父母家,撫循部眾,能行軍用師,壓服諸越。每勸宗族為善,由是信義結於本鄉。



 越人俗好相攻擊,夫人兄南梁州刺史挺恃其富強,侵掠傍郡,嶺表苦之。夫人多所規諫,由是怨隙止息,海南儋耳歸附者千餘洞。



 梁大同初,羅州刺史馮融聞夫人有志行,為其子高涼太守寶聘以為妻。融本北燕苗裔也。初,馮弘之南投,遣融大父業以三百人浮海歸宋,因留於新會。自業及融,三世為守牧,他鄉羈旅,號令不行。至夫人誡約本宗,使從百姓禮。每與夫寶,參決辭訟,首領有犯法
 者,雖是親族,無所縱捨。自此,政令有序,人莫敢違。後遇候景反,廣州都督蕭勃徵兵援臺,高州刺史李遷仕據大皋口,遣召寶。寶欲往,夫人疑其反,止之。數日,遷仕果反,遣主帥杜平虜率兵入灨石。寶以告,夫人曰:「平虜入灨,與官兵相拒,勢未得還,遷仕在州,無能為也。宜遣使詐之,云:『身未敢出,欲遣婦往參。』彼必無防慮。我將千餘人,步擔雜物,唱言輸賧,得至柵下,賊變可圖。」從之。遷仕果大喜,覘夫人眾皆提物,不設備。夫人擊之,大捷。因總兵與長城侯陳霸先會於灨石。還謂寶曰:「陳都督極得眾心,必能平賊,君厚資給之。」



 及寶卒,嶺表大亂,夫人懷
 集百越,數州晏然。陳永定二年,其子僕年九歲,遣帥諸首領朝于丹陽,拜陽春郡守。後廣州刺史歐陽紇謀反,召僕至南海,誘與為亂。僕遣使歸告夫人,夫人曰:「我為忠貞,經今兩代,不能惜汝負國。」遂發兵拒境,紇徒潰散。僕以夫人之功,封信都侯,加平越中郎將,轉石龍太守。詔使持節冊夫人為高涼郡太夫人,齎繡憲油絡駟馬安車一乘,給鼓吹一部,并麾幢旌節,一如刺史之儀。至德中,僕卒。



 後陳國亡,嶺南未有所附,數郡共奉夫人,號為聖母。隋文帝遣總管韋洸安撫嶺外,陳將徐璒以南康拒守,洸不敢進。初,夫人以扶南犀杖獻陳主,至此,晉
 王廣遣陳主遺夫人書,諭以國亡,命其歸化,并以犀杖及兵符為信。夫人見杖,驗知陳亡,集首領數千人,盡日慟哭。遣其孫魂,帥人迎洸。洸至廣州,嶺南悉定。表魂為儀同三司,冊夫人為宋康郡夫人。



 未幾,悉禺人王仲宣反,圍洸,進兵屯衡嶺。夫人遣其孫暄帥師援洸。時暄與逆黨陳佛智素相友,故遲留不進。夫人大怒,遣使執暄系州獄,又遣孫盎討佛智斬之。進兵至南海,與鹿愿軍會,共敗仲宣。夫人親被甲,乘介馬,張錦傘,領彀騎,衛詔使裴矩巡撫諸州。其蒼梧首領陳坦、罔州馮岑翁、梁化鄧馬頭、藤州李光略、羅州龐靖等皆來參謁。還令統其
 部落,嶺南悉定。帝拜盎為高州刺史,仍赦出暄,拜羅州刺史。追贈寶為廣州總管,封譙國。夫人幕府署長史已下官屬,給印章,聽發部落、六州兵馬,若有機急,便宜行事。降敕書褒美,賜物五千段。皇后以首飾及宴服一襲賜之。夫人並盛於金篋,并梁、陳賜物,各藏于一庫。每歲時大會,皆陳于庭,以示子孫曰:「汝等宜盡赤心向天子。我事三代主,唯用一好心。今賜物具存,此忠孝之報。」



 時番州總管趙訥貪虐,諸俚獠多有亡叛。夫人遣長史張融上封事,論安撫之宜,并言訥罪狀。上遣推訥,得其贓,竟致于法。敕委夫人招慰亡叛。夫人親載詔書,自稱使
 者,歷十餘州,宣述上意,諭諸俚獠,所至皆降。文帝賜夫人臨振縣湯沐邑一千五百戶,贈僕為崖州總管,平原郡公。仁壽初,卒,謚為誠敬夫人。



 鄭善果母崔氏者,清河人也。年十三,適滎陽鄭誠,生善果。周末,誠討尉遲迥,力戰死于陣。母年二十而寡,父彥穆欲奪其志,母抱善果曰:「婦人無再男子之義。且鄭君雖死,幸有此兒,棄兒為不慈,背死夫為無禮。寧當割耳剪髮,以明素心。違禮滅慈,非敢聞命。」



 善果以父死王事,年數歲,拜使持節、大將軍,襲爵開封縣公。開皇初,進封武德郡公。年十四,授沂州刺史。轉景州刺史,尋為魯郡
 太守。母性賢明,有節操,博涉書史,通曉政事。每善果出聽事,母輒坐胡床,於鄣後察之。聞其剖斷合理,歸則大悅,即賜之坐,相對談笑;若行事不允,或妄嗔怒,母乃還堂,蒙袂而泣,終日不食。善果伏於床前,不敢起。母方起謂之曰:「吾非怒汝,乃愧汝家耳。吾為汝家婦,獲奉灑掃,知汝先君忠勤之士也,守官清恪,未嘗問私,以身徇國,繼之以死。吾亦望汝,副其此心。汝既年小而孤,吾寡婦耳,有慈無威,使汝不知禮訓,何可負荷忠臣之業乎!汝自童子襲茅土,汝今位至方岳,豈汝身致之邪?不思此事,而妄加嗔怒,心緣驕樂,墮於公政。內則墜爾家風,或
 失亡官爵;外則虧天下法,以取罪戾。吾死日何面目見汝先人於地下乎!」



 母恒自紡績,每自夜分而寢。善果曰:「兒封侯開國,位居三品,秩俸倖足,母何自勤如此?」答曰:「吁!汝年已長,吾謂汝知天下理,今聞此言,公事何由濟乎?今秩俸乃天子報汝先人殉命也,當散贍六姻,為先君之惠,妻子奈何獨擅其利以為貴乎!又絲枲紡績,婦人之務,上自王后,下及大夫士妻,各有所製。若墮業者,是為驕逸。吾雖不知禮,其可自敗名乎!」



 自初寡便不御脂粉,常服大練。性又節儉,非祭祀賓客之事,酒肉不妄陳其前。



 靜室端居,未嘗輒出門閭。內外姻戚有吉凶事,
 但厚加贈遺,皆不詣其門。非自手作及莊園祿賜所得,雖親族禮遺,悉不許入門。善果歷任州郡,內自出饌於衙中食之。公廨所供,皆不許受,悉用修理公宇,及分僚佐。善果亦由此克己,號為清吏。



 煬帝遣御史大夫張衡勞之,考為天下最。征授光祿卿。其母卒後,善果為大理卿,漸驕恣,公清平允,遂不如疇昔焉。



 孝女王舜者,趙郡人也。父子春,與從兄長忻不協。齊亡之際,長忻與其妻同謀殺子春。舜時年七歲,有二妹,粲年五年,璠年二歲,並孤苦,寄食親戚。舜撫育二妹,恩義甚篤。而舜陰有復仇之心,長忻殊不為備。妹俱長,親戚
 欲嫁之,輒拒不從。乃密謂二妹曰:「我無兄弟,致使父仇不復,吾輩雖女子,何用生為!我欲共汝報復,汝竟何如?」二妹皆垂泣曰:「唯姊所命。」夜中,姊妹各持刀踰牆入,手殺長忻夫婦,以告父墓,因詣縣請罪。姊妹爭為謀首,州縣不能決。文帝聞而嘉歎,特原其罪。



 韓覬妻于氏者,河南人也,字茂德。父寔,周大左輔。于氏年十四,適於覬。



 雖生長膏腴,家門鼎貴,而動遵禮度,躬自儉約,宗黨敬之。年十八,覬從軍沒,于氏哀毀骨立,慟感行路。每朝夕奠祭,皆手自捧持。及免喪,其父以其幼少無子,欲嫁之。誓不許。遂以夫孽子世隆為嗣,身自撫
 育,愛同己生,訓導有方,卒能成立。自孀居以後,唯時或歸寧。至於親族之家,絕不來往。有尊就省謁者,送迎皆不出戶庭。蔬食布衣,不聽聲樂,以此終身。隋文帝聞而嘉歎,下詔褒美,表其門閭。長安中號為節婦門,終于家。



 陸讓母馮氏者,上黨人也。性仁愛,有母儀。讓即其孽子也,開皇末,為播州刺史。數有聚斂,贓貨狼籍,為司馬所奏。案覆得實,將就刑。馮氏蓬頭垢面,詣朝堂數讓罪。於是流涕鳴咽,親持盃粥,勸讓食。既而上表求哀,詞情甚切,上愍然為之改容。獻皇后甚奇其意,致請於上。書侍御史柳彧進曰:「馮氏母德之至,有感行路,如或戮之,何
 以為勸?」上於是集京城士庶於朱雀門,遣舍人宣詔曰:「馮氏這嫡母之德,足為世範,慈愛之道,義感人神,特宜矜免,用獎風俗。讓可減死除名。」復下詔褒美之,賜物五百段,集命婦與馮相識,以旌寵異。



 劉昶女者,河南長孫氏婦。昶在周尚公主,為上柱國、彭國公,位望甚顯。與隋文帝有舊,及受禪,甚見親禮。歷左武衛大將軍、慶州總管。



 其子居士為千牛備身,不遵法度,數得罪。上以昶故,每原之。居士轉恣,每大言曰:「男兒要當辮頭反縛,蘧蒢上作獠舞。」取公卿子弟膂力雄健者,輒將歸家,以車輪括其頸而棒之,殆死,能不屈者,稱
 為壯士,釋而與之交。黨與三百人,其趫捷者號為餓鶻隊,武力者號為蓬轉隊。韝鷹紲犬,連騎道中,毆擊路人,多所侵奪。長安市里,無貴賤見者辟易。至於公卿妃主,亦莫敢與校。其女則居士姊也,每垂泣誨之,居士不改,至破家產。昶年高,奉養甚薄。其女時寡居,哀昶如此,每歸寧于家,躬勤紡績,以致其肥鮮。



 有人告居士與其徒遊長安城,登故未央殿基,向南坐,前後列隊,意有不遜。



 每相約曰:「當作一死耳。」又時有人言居士遣使引突厥,令南寇,當於京師應之。



 上謂昶曰:「今日事當如何?」昶猶恃舊恩,不自引咎,直前曰:「黑白在于至尊。」



 上大怒,下昶
 獄,捕居士黨與。憲司又奏昶事母不孝。其女知昶必不免,不食者數日。每親調飲食,手自捧持,詣大理餉父。見獄卒,跪以進之,歔欷鳴咽,見者傷之,居士斬,昶賜死於家。詔百僚臨視。時其女絕而復蘇者數矣,公卿慰喻之。其女言父無罪,坐子及禍。詞情哀切,人皆不忍聞見。遂布衣蔬食,以終其身。上聞歎曰:「吾聞衰門之女,興門之男,固不虛也。」



 鐘士雄母蔣氏者,臨賀人也。士雄仕陳,為伏波將軍。陳主以士雄嶺南酋帥,慮其反覆,留蔣氏於都下。及晉王廣平江南,以士雄在嶺表,欲以恩義致之,遣蔣氏歸臨
 賀。既而同郡虞子茂、鐘文華等作亂攻城,遣召士雄,士雄將應之。蔣氏謂曰:「汝若背德忘義,我當自殺於汝前。」士雄遂止。蔣氏復為書與子茂等,諭以禍福。子茂不從,尋為官軍所敗。上聞蔣氏,甚異之,封安樂縣君。



 時伊州寡婦胡氏者,不知何許人妻,甚有志節,為邦族所重。江南之亂,諷諭宗黨,守節不從叛逆,封為密陵郡君。



 孝婦覃氏者,上郡鐘氏婦也。與夫相見未幾而夫死,時年十八,事後姑以孝聞。



 數年間,姑及伯叔皆相繼死。覃氏家貧,無以葬,躬自節儉,晝夜紡績,十年而葬八喪,為州里所敬。文帝聞而賜米百石,表其門閭。



 元務光母盧氏者,範陽人也。少好讀書,造次必以禮。盛年寡居,諸子幼弱,家貧不能就學,盧氏每親自教授,勖以義方。漢王諒反,遣將綦良往山東略地,良以務光為記室。及良敗,慈州刺史上官政簿籍務光家。見盧氏,逼之。盧氏以死自誓。政凶悍,怒甚,以燭燒其面。盧氏執志彌固,竟不屈節。



 裴倫妻柳氏者,河東人也,少有風訓。大業末,倫為渭源令,為賊薛舉所陷,倫遇害。柳氏時年四十,有二女及兒婦三人,皆有美色。柳氏謂曰:「我輩遭逢禍亂,汝父已死,我自念不能全汝。我門風有素,義不受辱于群賊。我將
 與汝等同死,如何?」女等垂泣曰:「唯母所命。」柳氏遂自投於井,其女及婦相繼而下,皆死井中。



 趙元楷妻崔氏者,清河人也,甚有禮度。隋末宇文化及之反,元楷隨至河北。



 將歸長安,至滏口遇盜,僅以身免。崔氏為賊所拘,請以為妻。崔氏曰:「我士大夫女,為僕射子妻,今日破亡,自可即死,終不為賊婦。」群賊毀裂其衣,縛於床簀之上,將陵之。崔氏懼為所辱,詐之曰:「今力已屈,當受處分。」賊遂釋之。



 妻因取賊刀倚樹而立曰:「欲殺我,任加刀鋸;若覓死,可來相逼。」賊大怒,亂射殺之。



 元楷後得殺妻者,支解以祭崔氏之柩。



 論曰:婦人主織紝中饋之事,其德以柔順為先,斯乃舉其中庸,未臻其極者也。



 至於明識遠圖,貞心峻節,志不可奪,唯義所高,考之圖史,亦何代而無之哉!魏隋所敘列女,凡三十四人。自王公妃主,下至庶人女妻,蓋有質邁寒松,心逾匪石,或忠壯誠懇,或文採可稱。雖子政集之於前,元凱編之於後,比其美節,亦何以尚茲。故知蘭玉芳貞,蓋乃稟其性矣。



\end{pinyinscope}