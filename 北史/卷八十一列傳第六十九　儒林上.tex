\article{卷八十一列傳第六十九 儒林上}

\begin{pinyinscope}

 梁越盧醜張偉梁祚平恆陳奇劉獻之張吾貴劉蘭孫惠蔚族曾孫靈暉馬子結石曜靈暉子萬壽徐遵明董徵李業興子崇祖李鉉馮偉張買奴劉軌思鮑季詳邢峙劉晝馬敬德子元
 熙張景仁權會張思伯張彫武郭遵儒者,其為教也大矣,其利物也博矣!以篤父子,以正君臣。開政化之本原,鑿生靈之耳目,百王損益,一以貫之。雖世或汙隆,而斯文不墜。自永嘉之後,宇內分崩,禮樂文章,掃地將盡。魏道武初定中原,雖日不暇給,始建都邑,便以經術為先。立太學,置《五經》博士生員千有餘人。天興二年春,增國子太學生員至三千人。豈不以天下可馬上取之,不可以馬上臨之?聖達經猷,蓋為遠矣。四年春,命樂師入學習舞,釋菜於先師。明元時,改國子為
 中書學,立教授博士。太武始光三年春,起太學於城東。後征盧玄、高允等,而令州郡各舉才學。於是人多砥尚,儒術轉興。獻文天安初,詔立鄉學,郡置博士二人,助教二人,學生六十人。後詔大郡立博士二人,助教四人,學生一百人;次郡立博士二人,助教二人,學生八十人;中郡立博士一人,助教二人,學生六十人;下郡立博士一人,助教一人,學生四十人。太和中,改中書學為國子學,建明堂、辟雍,尊三老五更,又開皇子之學。



 及遷都洛邑,詔立國子、太學、四門小學。孝文欽明稽古,篤好墳籍,坐輿據鞍,不忘講道。劉芳、李彪諸人以經書進,崔光、邢巒
 之徒以文史達。其餘涉獵典章,閑集詞翰,莫不縻以好爵,動貽賞眷。於是斯文鬱然,比隆周、漢。宣武時,復詔營國學。樹小學於四門,大選儒生以為小學博士,員四十人。雖黌宇未立,而經術彌顯。時天下承平,學業大盛,故燕、齊、趙、魏之間,橫經著錄,不可勝數。大者千餘人,小者猶數百。州舉茂異,郡貢孝廉,對揚王庭,每年逾眾。神龜中,將立國學,詔以三品以上,及五品清官之子以充生選。未及簡置,仍復停寢。正光三年,乃釋奠於國學,命祭酒崔光講《孝經》,始置國子生三十六人。暨孝昌之後,海內淆亂,四方校學,所存無幾。



 齊神武生於邊朔,長於戎
 馬,杖義建旗,掃清區縣。因魏氏喪亂,屬爾朱殘酷,文章咸盪,禮樂同奔,弦歌之音且絕,俎豆之容將盡。永熙中,孝武復釋奠於國學,又於顯陽殿詔祭酒劉欽講《孝經》,黃門李郁說《禮記》,中書舍人盧景宣講《大戴禮夏小正》篇,復置生七十二人。及永熙西遷,天平北徙,雖庠序之制,有所未遑,而儒雅之道,遽形心慮。時初遷都於鄴,國子置生三十六人。至興和、武定之間,儒業復盛矣。始天平中,范陽盧景裕同從兄仲禮於本郡起逆,齊神武免其罪,置之賓館,以經教授太原公以下。及景裕卒,又以趙郡李同軌繼之。二賢並大蒙恩遇,待以殊禮。同軌云
 亡,復徵中山張彫武、勃海李鉉、刁柔、中山石曜等遞為諸子師友。及天保、大寧、武平之朝,亦引進名儒,授皇太子、諸王經術。然爰自始基,暨於季世,唯濟南之在儲宮,性識聰敏,頗自砥礪,以成其美。自餘多驕恣傲狠,動違禮度,日就月將,無聞焉爾。鏤冰彫朽,迄用無成,蓋有由焉。夫帝王子孫,習性驕逸。況義方之情不篤,邪僻之路競開,自非得自生知,體包上智。而內縱聲色之娛,外多犬馬之好,安能入則篤行,出則友賢者也?徒有師傅之資,終無琢磨之實。貴游之輩,飾以明經,可謂稽山竹箭,加之括羽,俯拾青紫,斷可知焉。



 而齊氏司存,或失其守;
 師保疑丞,皆賞勛舊;國學博士,徒有虛名。唯國子一學,生徒數十人耳。胄子以通經進仕者,唯博陵崔子發、廣平宋游卿而已。自外莫見其人。幸朝章寬簡,政綱疏闊,游手浮惰,十室而九。故橫經受業之侶,遍於鄉邑;負笈從宦之徒,不遠千里。入閭里之內,乞食為資,憩桑梓之陰,動逾十數。燕、趙之俗,此眾尤甚焉。齊制,諸郡並立學,置博士、助教授經。學生俱久差逼充員,士流及豪富之家,皆不從調。備員既非所好,墳籍固不開懷。又多被州郡官人驅使,縱有游惰,亦不檢察。皆由上非所好之所致也。諸郡俱得察孝廉,其博士、助教及游學之徒通經者,
 推擇充舉。射策十條,通八以上,聽九品出身,其尤異者,亦蒙抽擢。



 周文受命,雅重經典。于時西都板蕩,戎馬生郊。先生之舊章,往聖之遺訓,掃地盡矣!於是求闕文於三古,得至理於千載,黜魏、晉之制度,復姬旦之茂典。



 盧景宣學通群藝,修五禮之缺;長孫紹遠才稱洽聞,正六樂之壞。由是朝章漸備,學者嚮風。明皇纂歷,敦尚學藝,內有崇文之觀,外重成均之職。握素懷鉛,重席解頤之士,間出於朝廷;員冠方領,執經負笈之生,著錄於京邑。濟濟焉,足以踰於向時矣。洎保定三年,帝乃下詔尊太保燕公為三老。帝於是服兗冕,乘碧輅,陳文物,備禮容,
 清蹕而臨太學,袒割以食之,奉觴以酳之。斯固一世之盛事也。其後命輶軒而致玉帛,征沈重於南荊。及定山東,降至尊而勞萬乘,待熊安生以殊禮。



 是以天下慕嚮,文教遠覃。衣儒者之服,挾先王之道,開黌舍,延學徒者比肩;勵從師之志,守專門炎業,辭親戚,甘勤苦者成市。雖通儒盛業,不逮魏、晉之臣,而風移俗變,抑亦近代之美也。



 自正朔不一,將三百年,師訓紛綸,無所取正。隋文膺期纂曆,平一寰宇,頓天鋼以掩之,賁旌帛以禮之,設好爵以縻之,於是四海九州,強學待問之士,靡不畢集焉。天子乃整萬乘,率百僚,遵問道之儀,觀釋奠之禮。博
 士罄縣河之辯,侍中竭重席之奧。考正亡逸,研核異同,積滯群疑,渙然冰釋。於是超擢奇俊,厚賞諸儒。京邑達乎四方,皆啟黌校。齊魯趙魏,學者尤多。負笈追師,不遠千里,講誦之聲,道路不絕。中州之盛,自漢魏以來,一時而已。及帝暮年,精華稍竭,不悅儒術,專尚刑名,執政之徒,咸非篤好。暨仁壽間,遂廢天下之學,唯存國子一所,弟子七十二人。煬帝即位,復開庠序,國子、郡縣之學,盛於開皇之初。徽辟儒生,遠近畢至。使相與講論得失於東都之下,納言定其差次,一以聞奏焉。于時,舊儒多已凋亡,惟信都劉士元、河間劉光伯拔萃出類,學通南北,
 博極今古,後生鑽仰。所製諸經義疏,縉紳咸師宗之。既而外事四夷,戎馬不息,師徒怠散,盜賊群起。禮義不足以防君子,刑罰不足以威小人,空有建學之名,而無弘道之實。其風漸墜,以至滅亡。方領矩步之徒,亦轉死溝壑,凡有經籍,因此湮沒於煨燼矣。



 遂使後進之士,不復聞《詩書》之言,皆懷攘竊之心,相與陷於不義。《傳》曰:「學者將殖,不學者將落。」然則盛衰是繫,興亡攸在,有國有家者,可不慎歟!



 漢世,鄭玄並為眾經注解,服虔、何休,各有所說。玄《易》、《詩》、《書》、《禮》、《論語》、《孝經》,虔《左氏春秋》,休《公羊傳》,大行於河北。王肅《易》,亦間行焉。晉世,杜預注《左氏》。預玄孫
 坦,坦弟驥,於宋朝並為青州刺史,傳其家業,故齊地多習之。



 自魏末,大儒徐遵明門下講鄭玄所注《周易》。遵明以傳盧景裕及清河崔瑾。



 景裕傳權會、郭茂。權會早入鄴都,郭茂恆在門下教授,其後能言《易》者,多出郭茂之門。河南及青齊之間,儒生多講王輔嗣所注,師訓蓋寡。



 齊時,儒士罕傳《尚書》之業,徐遵明兼通之。遵明受業於屯留王聰,傳授浮陽李周仁及勃海張文敬、李鉉、河間權會,並鄭康成所注,非古文也。下里諸生,略不見孔氏注解。武平末,劉光伯、劉士元始得費甝《義疏》,乃留意焉。



 其《詩》、《禮》、《春秋》,尤為當時所尚,諸生多兼通之。



 《三禮》並出
 遵明之門。徐傳業於李鉉、祖俊、田元鳳、馮傳、紀顯敬、呂黃龍、夏懷敬。李鉉又傳授刁柔、張買奴、鮑季詳、邢峙、劉晝、熊安生。安生又傳孫靈暉、郭仲堅、丁恃德。其後生能通《禮經》者,多是安生門人。諸生盡通《小戴禮》。於《周儀禮》兼通者,十二三焉。通《毛詩》者,多出於魏朝劉獻之。獻之傳李周仁。周仁傳董令度、程歸則。歸則傳劉敬和、張思伯、劉軌思。其後能言《詩》者,多出二劉之門。河北諸儒能通《春秋》者,並服子慎所注,亦出徐生之門。張買奴、馬敬德、邢峙、張思伯、張奉禮、張彫、劉晝、鮑長宣、王元則並得服氏之精微。又有衛覬、陳達、潘叔虔,雖不傳徐氏之門,
 亦為通解。又有姚文安、秦道靜,初亦學服氏,後兼更講杜元凱所注。其河外儒生,俱伏膺杜氏。其《公羊》、《穀梁》二傳,儒者多不厝懷。《論語》、《孝經》,諸學徒莫不通講。諸儒如權會、李欽、刁柔、熊安生、劉軌思、馬敬德之徒,多自出義疏。雖曰專門,亦皆相祖習也。



 大抵南北所為章句,好尚互有不同。江左,《周易》則王輔嗣,《尚書》則孔安國,《左傅》則杜元凱。河洛,《左傳》則服子慎,《尚書》、《周易》則鄭康成。《詩》則並主於毛公,《禮》則同遵於鄭氏。南人約簡,得其英華;北學深蕪,窮其枝葉。考其終始,要其會歸,其立身成名,殊方同致矣。



 自魏梁越已下,傳授講議者甚眾,今各依時
 代而次,以備《儒林》云爾。



 梁越,字玄覽,新興人也。博通經傳,性純和。魏初,為《禮經》博士。道武以其謹厚,遷上大夫,令授諸皇子經書。明元初,以師傅恩,賜爵祝阿侯,出為雁門太守。獲白雀以獻,拜光祿大夫,卒。



 盧醜,昌黎徒何人也。襄城王魯元之族也。太武監國,醜以博學入授經。後以師傅舊恩,賜爵濟陰公。位尚書,加散騎常侍,卒於河內太守。



 張偉,字仲業,太原中都人也。學通諸經。鄉里受業者,常數百人。儒謹汎納。



 雖有頑固,問至數十,偉告喻殷勤,曾
 無慍色。常依附經典,教以孝悌,門人感其仁化,事之如父。性清雅,非法不言。太武時,與高允等俱被辟命,授中書博士,累遷為中書侍郎,本國大中正。使酒泉慰勞沮渠無諱,又使宋,賜爵成皋子。出為營州刺史,進爵建安公。卒,贈并州刺史,謚曰康。



 梁祚,北地泥陽人也。父邵,皇始二年歸魏,位濟陽太守。至祚,居趙郡。祚篤志好學,歷習經典,尤善《公羊春秋》、鄭氏《易》,常以教授。有儒者風,而無當世之才。與幽州別駕平恒有舊,恆時請與論經史。辟祕書中散,稍遷秘書令,為李所排擯,退為中書博士。後出為統萬鎮司馬,徵
 為散令。撰并陳壽《三國志》,名曰《國統》。又作《代都賦》,頗行於世。清貧守素,不交勢貴,卒。子元吉,有父風。



 平恆,字繼叔,燕郡薊人也。祖視、父儒,並仕慕容為通宦。恆耽勤讀誦,多通博聞。自周以降,暨於魏世,帝王傳代之由,貴臣升降之緒,皆撰品第,商略是非,號曰《略注》,合百餘篇。安貧樂道,不以屢空改操。徵為中書博士。久之,出為幽州別駕。廉貞寡欲,不營資產,衣食至常不足,妻子不免饑寒。後遷祕書丞。



 時高允為監,河間邢祐、北平陽嘏、河東裴宗、廣平程駿、金城趙元順等為著作郎。



 允每稱博通經籍,無過恆也。



 恆三子,並不率父業,好酒自
 棄。恆常忿其世衰,植杖巡舍,側崗而哭。不為營事婚宦,任意官娶,曰:「此輩會是衰頓,何煩勞我!』故仕娉濁碎,不得及其門流。別構精廬,並置經籍於中,一奴自給,妻子莫得而往,酒食亦不與同。時有珍美,呼時老東安公刁雍等共飲啖之,家人無得嘗焉。太和十年,以恆為祕書令,而固請為郡,未受而卒。贈幽州刺史、都昌侯,謚曰康。



 陳奇,字脩奇,河北人也。少孤貧,而奉母至孝。齠祇聰識,有夙成之美。愛玩經典,常非馬融、鄭玄解經失旨。志在著述《五經》。始注《孝經》、《論語》,頗傳於世,為縉紳所稱。與河間邢祐同召赴京。時祕書省遊雅素聞其名,始頗好之,
 引入秘省,欲授以史職。後與奇論典誥,至《易訟卦》「天與水違行」,雅曰:「自葱嶺以西,水皆西流,推此而言,自葱嶺西,豈東向望天哉?」雅性護短,因以為嫌。嘗眾辱奇,或爾汝之,或指為小人。奇曰:「公身為君子,奇身且小人。」



 雅曰:「君言身且小人,君祖父是何人也?」奇曰:「祖,燕東部侯厘。」雅質奇曰:「侯釐何官也?」奇曰:「昔有雲師、火正、鳥師之名,以斯而言,世革則官異,時易則禮變。公為皇魏東宮內侍長,竟何職也?」先是,敕以奇付雅,令銓補祕書。雅既惡之,遂不復敘用焉。



 奇冗散數年,高允每嘉其遠致,稱奇通識,非凡學所及。允微勸雅曰:「君朝望具瞻,何為與野
 儒辯簡牘章句!」雅謂允有私於奇,曰:「君寧黨小人也?」乃取奇注《論語》、《孝經》,燒於庭內。奇曰:「公貴人,不乏樵薪,何乃燃奇《論語》!」雅愈怒,因告京師後生,不聽傳授。而奇無降志,亦評雅之失。雅製昭皇太后碑文,論后名字之美,比諭前魏之甄后。奇刺發其非,遂聞於上。詔下司徒檢對,雅有屈焉。



 有人為謗書,多怨時之言,頗稱奇不得志。雅乃諷在事云,此書言奇不遂,當是奇假人為之。如依律文,造謗書者,皆及孥戮。遂抵奇罪。時司徒、平原王陸麗知奇見枉,惜其才學,故得遷延經年,冀得寬宥。獄成,竟致大戮,遂及其家。奇於《易》尤長,在獄嘗自筮。卦未及
 成,乃攬破而歎曰:「吾不度來年冬季。」及奇受害,如其所占。奇初被召,夜夢星墜壓腳。明而告人曰:「星則好風,星則好雨,夢星壓腳,必無善徵。但時命峻切,不敢不赴耳。」



 奇外生常矯之,仕歷郡守。奇所注《論語》矯之傳掌,未能行於世。其義多異鄭玄,往往與司徒崔浩同。



 劉獻之,博陵饒陽人也。少而孤貧,雅好《詩》《傳》。曾受業於勃海程玄,後遂博觀眾籍。見名法之言,掩卷而笑曰:「若使楊、墨之流,不為此書,千載誰知其小也?」曾謂其所親曰:「觀屈原《離騷》之作,自是狂人,死其宜矣。孔子曰『無可無不可』,實獲我心。」時人有從獻之學者,獻之輒謂之曰:「
 人之立身,雖百行殊塗,準之四科,要以德行為首。子若能入孝出悌,忠信仁讓,不待出戶,天下自知。儻不能然,雖復下帷針股,躡屩從師,正可博聞多識,不過為土龍乞雨,眩惑將來。其於立身之道,有何益乎?孔門之徒,初亦未悟,見皋魚之歎,方乃歸而養親。嗟乎!先達何自覺之晚也?」由是四方學者,莫不高其行義,希造其門。



 獻之善《春秋》、《毛詩》。每講《左氏》,盡隱公八年便止,云:「義例已了,不復須解。」由是弟子不能究竟其說。後本郡逼舉孝廉,至京稱病而還。孝文幸中山,詔徵典內校書。獻之喟然歎曰:「吾不如莊周散木遠矣,一之謂甚,其可再乎!」固以疾辭。時中山張吾貴與獻之齊名,四海皆稱儒宗。吾貴每一講唱,門
 徒千數,其行業可稱者寡。獻之著錄,數百而已,皆通經之士。於是有識者辨其優劣。



 魏承喪亂之後,《五經》大義,雖有師說,而海內諸生,多有疑滯,咸決於獻之。六藝之文,雖不悉注,所標宗旨,頗異舊義。撰《三禮大義》四卷,《三傳略例》三卷,注《毛詩序義》一卷,行於世。並立《章句疏》二卷。注《涅槃經》,未就而卒。四子:放古、爰古、參古、修古。



 張吾貴,字吳子,中山人也。少聰慧口辯,身長八尺,容貌奇偉。年十八,本郡舉為太學博士。吾貴先未多學,乃從酈詮受《禮》,牛天祐受《易》。詮、祐粗為開發而已,吾貴覽讀一遍,便即別構戶牖,世人競歸之。曾在夏學,聚徒千數,
 而不講《傳》。生徒竊云:「張生之於《左氏》,似不能說。」吾貴聞之,謂曰:「我今夏講暫罷,後當說《傳》。君等來日,皆當持本。」生徒怪之而已。吾貴詣劉蘭,蘭遂為講《傳》。三旬之中,吾貴兼讀杜、服,隱括兩家,異同悉舉。諸生後集,便為講之,義例無窮,皆多新異,蘭仍伏聽。學者以此益奇之。而辯能飾非,好為詭說,由是業不久傳。而氣陵牧守,不屈王侯,竟不仕而終。



 劉蘭,武邑人也。年三十餘,始入小學書《急就篇》。家人覺其聰敏,遂令從師。受《春秋》、《詩》、《禮》於中山王保安。家貧,無以自資,且耕且學。三年之後,便白其兄,求講說。其兄笑
 而聽之,為立黌舍,聚徒二百。蘭讀《左氏》,五日一遍,兼能《五經》。先是,張吾貴以聰辯過人,其所解說,不本先儒之旨。



 唯蘭推《經》、《傳》之由,本注者之意,參以緯候及先儒舊事,甚為精悉。自後《經》義審博,皆由於蘭。蘭又明陰陽,博物多識,故為儒者所宗。



 瀛州刺史裴植,徵蘭講書於州南館。植為學主,故生徒甚盛,海內稱焉。又特為中山王英所重。英引在館,令授其子熙、誘、略等。蘭學徒前後數千,成業者眾。



 而排毀《公羊》,又非董仲舒,由是見譏於世。為國子助教。靜坐讀書,有人叩門,蘭命引入,葛巾單衣,入與蘭坐,謂曰:「君自是學士,何為每見毀辱?理義長短,
 竟在誰?而過無禮見陵也!今欲相召,當與君正之。」言終而出,蘭少時患死。



 孫惠蔚,武邑武遂人也。年十五,粗通《詩》、《書》及《孝經》、《論語》。



 十八,師董道季講《易》。十九,師程玄讀《禮經》及《春秋三傳》。周流儒肆,有名於冀方。太和初,郡舉孝廉,對策於中書省。時中書監高閭因相談薦,俄為中書博士,轉皇宗博士。閭被敕理定雅樂,惠蔚參其事。及樂成,閭上疏請集朝士於太樂,共研是非。祕書令李彪,自以才辯,立難於其前。閭命惠蔚與彪抗論,彪不能屈。黃門侍郎張彞,常與游處,每表疏論事,多參訪焉。十七年,孝文南征,上議
 告類之禮。及太師馮熙薨,惠蔚監其喪禮。上書,令熙未冠之子,皆服成人服。



 惠蔚與李彪以儒學相知,及彪位至尚書,惠蔚仍太廟令。孝文曾從容言曰:「道固既登龍門,而孫蔚猶沈涓澮,朕常以為負矣。」雖久滯小官,深體通塞,無孜孜之望,儒者以是尚焉。二十二年,侍讀東宮。先是,七廟以平文為太祖。孝文議定祖宗,以道武為太祖。祖宗雖定,然昭穆未改。及孝文崩,將祔神主於廟。侍中崔光兼太常卿,以太祖既改,昭穆以次而易。兼御史中尉、黃門侍郎邢巒,以為太祖雖改,昭穆仍不應易,乃立彈草,欲按奏光。光謂惠蔚曰:「此乃禮也,而執法欲見
 彈劾,思獲助於碩學。」惠蔚曰:「此深得禮變。」尋為書以與光,讚明其事。光以惠蔚書呈宰輔,乃召惠蔚與巒庭議得失。尚書令王肅又助巒,而巒理終屈,彈事遂寢。



 宣武即位之後,仍在左右,敷訓經典。自冗從僕射遷祕書丞、武邑郡中正。惠蔚既入東觀,見典籍未周。及閱舊典,先無定目,新故雜糅,首尾不全,有者累袠數十,無者曠年不寫。或篇第剝落,始末淪殘,或文壞字誤,謬爛相屬。卷目雖多,全定者少。請依前丞盧昶所撰甲乙新錄,欲裨殘補闕,損併有無,校練句讀,以為定本,次第均寫,永為常式。其省先無本者,廣加推尋,搜求令足。然經記浩博,
 諸子紛綸,部帙既多,章第紕繆,當非一二校書,歲月可了。求令四門博士及在京儒生四十人,在祕書省專精校考,參定字義。詔許之。



 後為黃門侍郎,代崔光為著作郎。才非文史,無所撰著。遷國子祭酒、祕書監,仍知史事。延昌三年,追賞講定之勞,封棗強縣男。明帝初,出為濟州刺史。還京,除光祿大夫。魏初已來,儒生寒宦,惠蔚最為顯達。先單名蔚,正始中,侍講禁內,夜論佛經,有愜帝旨,詔使加「惠」,號惠蔚法師焉。卒于官,贈瀛州刺史,謚曰戴。子伯禮襲封。



 伯禮善隸書,位國子博士。惠蔚族曾孫靈暉。



 靈暉少明敏,有器度。得惠蔚手錄章疏,研精尋問,更求師友,《三禮》、《三傳》,皆通宗旨。然始就鮑季詳、熊安生質問疑滯,其所發明,熊、鮑無以異也。舉冀州秀才,射策高第。仕齊,累至國子博士,授南陽王綽府諮議參軍。綽除定州刺史,仍隨綽之鎮。所為猖蹶,靈暉唯默默憂悴,不能諫止。綽表請靈暉為王師,以管記馬子結為諮議。朝廷以王師三品,奏啟不合。後主於啟下手詔云:「但用之。」儒者甚以為榮。綽除大將軍,靈暉以王師領大將軍司馬。綽誅,停廢。從綽死後,每至七日至百日,靈暉恆為綽請僧設齋行道。齊亡,卒。



 馬子結者,其先扶風人,世仕涼土,
 魏太和中入洛。父祖俱清官。子結及兄子廉、子尚三人,皆涉文學。陽休之牧西兗,子廉、子尚、子結與諸朝士各有贈詩。



 陽總為一篇酬答。詩云:「三馬皆白眉」者也。子結為南陽王綽管記,隨綽定州。



 綽每出游獵,必令子結走馬從禽。子結既儒緩,衣垂帽落,或叫或啼,令騎驅之,非墜馬不止。綽以為笑。由是漸見親狎,啟為諮議焉。



 石曜字白曜,中山安善人。亦以儒學進,居官清儉。武平中,為黎陽郡守。時丞相咸陽王世子斛律武都出為兗州刺史,性貪暴。先過衛縣,令丞以下,斂絹數千疋遺之。至黎陽,令左右諷動曜及縣官。曜手持一絹謂武都曰:「此是
 老石機杼,聊以奉贈。自此以外,並須出於吏人。吏人之物,一毫不敢輒犯。」武都亦知曜清素純儒,笑而不責。曜著《石子》十卷,言甚淺俗。位終譙州刺史。



 靈暉子萬壽,字仙期,一字遐年。聰識機警,博涉經史,善屬文,美譚笑。在齊,仕為陽休之開府行參軍。及隋文帝受禪,滕穆王引為文學。坐衣冠不整,配防江南。行軍總管宇文述,召典軍書。萬壽本自書生,從容文雅,一旦從軍,鬱鬱不得志。為五言詩贈京邑知友。詩至京,盛為當時吟誦,天下好事者,多書壁上而玩之。後歸鄉里,十餘年不得調。仁壽初,拜豫章王長史,非其好也。王轉封於齊,即為齊王
 文學。當時,諸王官屬,多被夷滅,由是彌不自安,因謝病免。久之,授大理司直,卒於官。有集十卷,行於世。



 徐遵明,字子判,華陰人也。幼孤,好學,年十七,隨鄉人毛靈和等詣山東求學。至上黨,乃師屯留王聰,受《毛詩》、《尚書》、《禮記》。一年,便辭聰游燕、趙,師事張吾貴。吾貴門徒甚盛。遵明伏膺數月,乃私謂友人曰:「張生名高而義無檢格,凡所講說,不愜吾心。請更從師。」遂與平原田猛略就范陽孫買德。



 受業一年,復欲去之。猛略謂遵明曰:「君年少從師,每不終業,如此用意,終恐無成。」遵明乃指其心曰:「吾今知真師所在矣,正在於此。」乃詣平原唐遷,居於
 蠶舍,讀《孝經》、《論語》、《毛詩》、《尚書》、《三禮》。不出門院,凡經六年,時彈箏吹笛,以自娛慰。又知陽平館陶趙世業家有《服氏春秋》,是晉世永嘉舊寫。遵明乃往讀之,復經數載。因手撰《春秋義章》,為三十卷。



 是後教授門徒,每臨講坐,先持執疏,然後敷講。學徒至今,浸以成俗。遵明講學於外,二十餘年,海內莫不宗仰。頗好聚斂,與劉獻之、張吾貴皆河北聚徒教授,懸納絲粟,留衣物以待之,名曰影質,有損儒者之風。遵明見鄭玄《論語序》云「書以八寸策」,誤作「八十宗」,因曲為之說。其僻也皆如此。獻之、吾貴又甚焉。遵明不好京輦,以兗州有舊,因徙屬焉。元顥入洛,任
 城太守李湛將舉義兵,遵明同其事。夜至人間,為亂兵所害。永熙二年,遵明弟子通直散騎侍郎李業興表求加策命,卒無贈謚。



 董徵,字文發,頓丘衛國人也。身長七尺二寸,好古學,尚雅素。年十七,師清河監伯陽受《論語》、《毛詩》、《春秋》、《周易》,河內高望崇受《周官》,後於博陵劉獻之遍受諸經。數年之中,大義精練,講授生徒。太和末,為四門小學博士。後宣武詔徵入IY華宮,令孫惠蔚問以《六經》。仍詔徵教授京兆、清河、廣平、汝南四王。後累遷安州刺史。徵因述職,路次過家,置酒高會,大享邑老。



 乃言曰:「腰龜返國,昔人稱
 榮,仗節還家,云胡不樂。」因誡二三子弟曰:「此之富貴,匪自天降,乃勤學所致耳。」時人榮之。入為司農少卿、光祿大夫,後以老解職。永熙二年,卒。孝武帝以徵昔授學業,故優贈儀同三司、尚書左僕射、相州刺史,謚曰文烈。子仲曜。



 李業興,上黨長子人也。祖虯、父玄紀,並以儒學舉孝廉。玄紀卒於金鄉令。



 業興少耿介志學,晚乃師事徐遵明於趙、魏之間。時有漁陽鮮于靈馥亦聚徒教授,而遵明聲譽未高,著錄尚寡。業興乃詣靈馥黌舍,類受業者。靈馥乃謂曰:「李生久逐羌博士,何所得也?」業興默爾不言。
 及靈馥說《左傳》,業興問其大義數條,靈馥不能對。於是振衣而起曰:「羌弟子正如此耳!」遂便徑還。自此,靈馥生徒傾學而就遵明。學徒大盛,業興之為也。



 後乃博涉百家,圖緯、風角、天文、占候,無不討練。尤長算歷。雖在貧賤,常自矜負,若禮待不足,縱於權貴,不為之屈。後為王遵業門客。舉孝廉,為校書郎。以世行趙匪曆,節氣後辰下算。延昌中,業興乃為《戊子元曆》上之。于時屯騎校尉張洪、盪寇將軍張龍詳等九家,各獻新曆。宣武詔令共為一曆。洪等後遂共推業興為主,成《戊子曆》,正光三年,奏行之。業興以殷曆甲寅,黃帝辛卯,徒有積元,術數亡缺。
 又修之,各為一卷,傳於世。建義初,敕典儀注。未幾,除著作郎。永安三年,以前造曆之勛,賜爵長子伯。後以孝武帝登極之初,豫行禮事,封屯留縣子,除通直散騎常侍。永熙三年二月,孝武帝釋奠,業興與魏季景、溫子昇、竇瑗為摘句。後入為侍讀。



 遷鄴之始,起部郎中辛術奏:「今皇居徙御,百度創始,營構一興,必宜中制。



 李業興碩學通儒,博聞多識,萬門千戶,所宜詢訪。今求就之披圖案記,考定是非,參古雜今,折中為制。」詔從之。於時尚書右僕射、營構大匠高隆之被詔繕修三署樂器、衣服及百戲之屬,乃奏請業興共事。



 天平四年,與兼散騎常侍李
 諧、兼吏部郎盧元明使梁。梁散騎常侍朱異問業興曰:「魏洛中委粟山是南郊邪?圓丘邪?」業興曰:「委粟是圓丘,非南郊。」異曰:「比聞郊、丘異所,是用鄭義。我此中用王義。」業興曰:「然。洛京郊丘之處,用鄭解。」異曰:「若然,女子逆降傍親,亦從鄭以不?」業興曰:「此之一事,亦不專從。若卿此間用王義,除禫應用二十五月,何以王儉《喪禮》,禫用二十七月也?」異遂不答。業興曰:「我昨見明堂,四柱方屋,都無五九之室,當是裴頠所制。明堂上圓下方,裴唯除室耳,今此上不圓,何也?」異曰:「圓方俗說,經典無文,何怪於方。」業興曰:「圓方之言,出處甚明,卿自不見。見卿錄梁主《
 孝經義》亦云『上圓下方』,卿言豈非自相矛盾?」異曰:「若然,圓方竟出何經?」業興曰:「出《孝經援神契》。」異曰:「緯候之書,何可信也!」業興曰:「卿若不信,《靈威仰》、《葉光紀》之類,經典亦無出者,卿復信不?」異不答。



 梁武問業興:「《詩·周南》,王者之風,繫之周公;《召南》,仁賢之風,系之召公。何名為繫?」業興對曰:「鄭注《儀禮》云:昔太王、王季居于岐陽,躬行《召南》之教以興王業。及文王行今《周南》之教以受命,作邑於酆。文王為諸侯之地所化之國,今既登九五之尊,不可復守諸侯之地,故分封二公,名為繫。」梁武又問:「《尚書》『正月上日,受終文祖』,此時何正?」業興對曰:「此夏正月。」梁武
 言:「何以得知?」業興曰:「案《尚書中候運衡篇》云『日月營始』,故知夏正。」又問:「堯時以前,何月為正?」業興對曰:「自堯以上,書典不載,實所不知。」梁武又云:「『寅賓出日』,是正月,『日中星鳥,以殷仲春』,即是二月。此出《堯典》,何得云堯時不知用何正?」業興對曰:「雖三正不同,言時節者,皆據夏時正月。《周禮》:『仲春二月,會男女之無夫家者。』雖自周書,月亦夏時。堯之日月,亦當如此。但所見不深,無以辯析明問。」梁武又曰:「《禮》:原壤母死,叩木而歌。孔子聖人,而與壤為友?」業興對曰:「孔即自解,言親者不失其親,故者不失其故。」又問:「壤何處人?」對曰:「《注》云:原壤,孔子幼之舊故。是
 魯人。」又問:「原壤不孝,有逆人倫,何以存故舊之小節,廢不孝之大罪?」對曰:「原壤所行,事自彰著,幼少之交,非是今始。既無大故,何容棄之?」又問:「孔子聖人,何以書原壤之事,垂法萬代?」業興對曰:「此是後人所錄,非孔子自制,猶合葬於防。如此之比,《禮記》之中,動有百數。」



 又問:「《易》有太極,極是有無?」業興對曰:「所傳太極是有。」還,兼散騎常侍,加中軍大將軍。



 業興家世農夫,雖學殖,而舊音不改。梁武問其宗門多少,答曰:「薩四十家。」



 使還,孫騰謂曰:「何意為吳兒所笑!」對曰:「業興猶被笑,試遣公去,當著被罵。」邢子才云:「爾婦疾,或問實耶?」業興曰:「爾大癡!但道此,
 人疑者半,信者半,誰檢看?」



 武定元年,除國子祭酒,仍侍讀。神武以業興明術數,軍行常問焉。業興曰某日某處勝,謂所親曰:「彼若告勝,自然賞吾;彼若凶敗,安能罪吾?」芒山之役,有風從西來入營。業興曰:「小人風來,當大勝。」神武曰:「若勝,以爾為本州刺史。」既而以為太原太守。五年,齊文襄引為中外府諮議參軍。後坐事禁止,業興乃造《九宮行棋曆》,以五百為章,四千四十為蔀,九百八十七為升分,還以己未為元,始終相維,不復移轉,與今曆法術不同。至於氣序交分,景度盈縮,不異也。文襄之征潁川,業興曰:「往必剋,剋後凶。」文襄既剋,欲以業興當凶
 而殺之。



 業興愛好墳籍,鳩集不已。手自補修,躬加題帖,其家所有,垂將萬卷。覽讀不息,多有異聞,諸儒服其深博。性豪俠,重意氣,人有急難,委命歸之,便能容匿。與其好合,傾身無吝;有乖忤,便即疵毀,乃至聲色,加以謗罵。性又躁隘,至於論難之際,無儒者之風。每語人云:「但道我好,雖知妄言,故勝道惡。」務進忌前,不顧後患,時人以此惡之。至於學術精微,當時莫及。業興二子,崇祖傳父業。



 崇祖字子述。文襄集朝士,命盧景裕講《易》。崇祖時年十一,論難往復,景裕憚之。業興助成其子,至於忿鬩。文襄
 色甚不平。姚文安難服虔《左傳解》七十七條,名曰《駮妄》。崇祖申明服氏,名曰《釋謬》。齊文宣營構三臺,材瓦工程,皆崇祖所算也。封屯留縣侯。遵祖,齊天保初難宗景曆甚精。崇祖為元子武卜葬地,醉而告之曰:「改葬後,當不異孝文。」武成,或告之,兄弟伏法。



 李鉉,字寶鼎,勃海南皮人也。九歲入學,書《急就篇》,月餘便通。家素貧,常春夏務農,冬乃入學。年十六,從浮陽李周仁受《毛詩》、《尚書》,章武劉子猛受《禮記》,常山房虯受《周官》、《儀禮》,漁陽鮮于靈馥受《左氏春秋》。



 鉉以鄉里無可師者,遂與州里楊元懿、河間宗惠振等結友,詣大儒徐遵
 明受業。居徐門下五年,常稱高第。年二十三,便自潛居討論是非。撰定《孝經》、《論語》、《毛詩》、《三禮義疏》及《三傳異同》、《周易義例》合三十餘卷。用心精苦,曾三秋冬不畜枕,每睡,假寐而已。年二十七,歸養二親,因教授鄉里。生徒恒數百人,燕趙間能言經者,多出其門。以鄉里寡文籍,來游京師,讀所未見書。舉秀才,除太學博士。及李同軌卒,齊神武令文襄在京妙簡碩學,以教諸子。文襄以鉉應旨,徵詣晉陽。時中山石曜、北平陽絢、北海王晞、清河崔瞻、廣平宋欽道及工書人韓毅同在東館,師友諸王。鉉以去聖久遠,文字多有乖謬,於講授之暇,遂覽《說文》、《倉》、《
 雅》,刪正六藝經注中謬字,名曰《字辨》。



 天保初,詔鉉與殿中尚書邢邵,中書令魏收等參議禮律,仍兼國子博士。時詔北平太守宋景業、西河太守綦母懷文等草定新歷,錄尚書、平原王高隆之令鉉與通直常侍房延祐、國子博士刁柔參考得失。尋正國子博士。廢帝之在東宮,文宣詔鉉以經入授,甚見優禮。卒,特贈廷尉少卿。及還葬,王人將送,儒者榮之。



 楊元懿、宗惠振官俱至國子博士。



 馮偉,字偉節,中山字喜人也。身長八尺,衣冠甚偉,見者肅然。少從李寶鼎學,李重其聰敏,恆別意試問之。多所
 通解,尤明《禮》、《傳》。後還鄉里,閉門不出,將三十年。不問生產,不交賓客,專精覃思,無所不通。齊趙郡王出鎮定州,以禮迎接,命書三至,縣令親至其門,猶辭疾不起。王將命駕致請,佐吏前後星馳報之,縣令又自為其整冠履,不得已而出。王下事迎之,止其拜伏,分階而上,留之賓館,甚見禮重。王將舉充秀才,固辭不就。歲餘請還。王知其不願拘束,以禮發遣,贈遺甚厚。一無所納,唯受時服而已。及還,不交人事,郡守縣令,每親至。歲時或置羊酒,亦辭不納。門徒束修,一毫不受。蠶而衣,耕而飯,簞食瓢飲,不改其樂。以壽終。



 張買奴,平原人也。經義該博,門徒千餘人,諸儒咸推重之。仕齊,歷太學博士、國子助教,卒。



 劉軌思,勃海人也。說《詩》甚精。少事同郡劉敬和,敬和事同郡程師則,故其鄉曲多為《詩》者。軌思仕齊,位國子博士。



 鮑季詳,勃海人也。甚明《禮》,兼通《左氏春秋》。少時,恆為李寶鼎都講。



 後亦自有徒眾,諸儒稱之。仕齊,卒於太學博士。



 從弟長暄,兼通《禮》、《傳》。為任城王湝丞相掾。恆在都教授貴游子弟。



 齊亡,卒於家。



 邢峙,字士峻,河間鄭人也。少學通《三禮》、《左氏春秋》。仕齊,
 初為四門博士,遷國子助教,以經入授皇太子。峙方正純厚,有儒者風。廚宰進太子食,菜有邪蒿,峙令去之,曰:「此菜有不正之名,非殿下宜食。」文宣聞而嘉之,賜以被褥縑纊,拜國子博士。皇建初,除清河太守,有惠政。年老歸,卒于家。



 劉晝,字孔昭,勃海阜城人也。少孤貧,愛學,伏膺無倦。常閉戶讀書,暑月唯著犢鼻褌。與儒者李寶鼎同鄉,甚相親愛。寶鼎授其《三禮》,又就馬敬德習《服氏春秋》,俱通大義。恨下里少墳籍,便杖策入都。知鄴令宋世良家有書五千卷,乃求為其子博士,恣意披覽,晝夜不息。還,舉秀
 才,策不第,乃恨不學屬文,方復緝綴辭藻。言甚古掘,制一首賦,以六合為名,自謂絕倫,乃歎儒者勞而寡功。



 曾以賦呈魏收而不拜。收忿之,謂曰:「賦名六合,已是太愚,文又愚於六合。君四體又甘於文。」晝不忿,又以示邢子才。子才曰:「君此賦,正似疥駱駝,伏而無嫵媚。」晝求秀才,十年不得,發憤撰《高才不遇傳》。冀州刺史酈伯偉見之,始舉晝,時年四十八。



 刺史隴西李璵,亦嘗以晝應詔。先告之,晝曰:「公自為國舉才,何勞語晝!」



 齊河南王孝瑜聞晝名,每召見,輒與促席對飲。後遇有密親,使且在齋坐,晝須臾徑去,追謝要之,終不復屈。孝昭即位,好受直言。
 晝聞之,喜曰:「董仲舒、公孫弘可以出矣。」乃步詣晉陽上書,言亦切直,而多非世要,終不見收采。編錄所上之書,為《帝道》。河清中,又著《金箱璧言》,蓋以指機政之不良。



 晝夜嘗夢貴人若吏部尚書者補交州興俊令,寤而密書記之。卒後旬餘,其家幼女鬼語,聲似晝,云「我被用為興俊縣令,得假暫來辭別」云。晝常自謂博物奇才,言好矜大。每言:「使我數十卷書行於後世,不易齊景之千駟也。」容止舒緩,舉動不倫,由是竟無仕,卒於家。



 馬敬德,河間人也。少好儒術,負笈隨徐遵明學《詩》、《禮》,略通大義,而不能精。遂留意於《春秋左氏》,沈思研求,晝夜
 不倦。教授於燕、趙間,生徒隨之者甚眾。乃詣州將,求秀才。將以其純儒,無意推薦。敬德請試方略,五條皆有文理,乃欣然舉送。至都,唯得中第。請試經業,問十條,並通。擢授國子助教,再遷國子博士。齊武成為後主擇師傅,趙彥深進之,入為侍講。其妻夜夢猛獸將來向之,敬德走超叢棘,妻伏地不敢動。敬德占曰:「吾當為大官。超棘,過幾卿也;爾伏地,夫人也。」後主既不好學,敬德侍講甚疏,時時以《春秋》入授。猶以師傅恩,拜國子祭酒、儀同三司、金紫光祿大夫、瀛州大中正。卒,其徒曰:「馬生勝孔子,孔子不得儀同。」尋贈開府、瀛州刺史。其後,侍書張景仁
 封王,趙彥深云:「何容侍書封王,侍講翻無封爵?」亦追封敬德廣漢郡王,令子元熙襲。



 元熙字長明,少傳父業,兼長文藻。以通直待詔文林館。武平中,皇太子將講《孝經》,有司請擇師。帝曰:「馬元熙,朕師之子,文學不惡。」於是以《孝經》入授皇太子。儒者榮其世載。性和厚,在內甚得名譽。隋開皇中,卒於秦王文學。



 張景仁,濟北人。幼孤,家貧,以學書為業,遂工草隸。選補內書生,與魏郡姚元標、潁川韓毅、同郡袁買奴、滎陽李超等齊名,文襄並引為賓客。天保八年,敕教太原王紹德書。後主在東宮,武成令侍書,遂被引擢。小心恭謹,後
 主愛之,呼為博士。登祚,累遷通直散騎常侍,在左右。與語,猶稱博士。胡人何洪珍有寵於後主,欲得通婚朝士。以景仁在內,官位稍高,遂為其兄子取景仁第二息瑜之女,因以表裹相援,恩遇日隆。景仁多疾,帝每遣徐之範等療之,給藥物珍羞,中使問疾,相望於道。是後,敕有司恆就宅送御食。車駕或有行幸,在道宿處,每送步障,為遮風寒。進位儀同三司,加開府,侍書如故。每旦須參,即在東宮停止。及立文林館,中人鄧長顒希旨,奏令總判館事。除侍中,封建安王。洪珍死後,長顒猶存舊款,更相彌縫,得無墜退。遂除中書監,卒。贈侍中、五州刺史、司
 空公。



 景仁為兒童時,在洛京,曾詣國學摹《石經》。許子華遇之學中,執景仁手曰:「張郎風骨,必當通貴,非但官爵遷達,乃與天子同筆硯,傳衣履。」子華卒二十餘年,景仁位開府,數賜衣冠、筆硯,如子華所言。出自寒微,本無識見,一旦開府、侍中、封王。其婦姓奇,莫知氏族所出,容制音辭,事事庸俚。既除王妃,與諸公主、郡君,同在朝謁之列,見者為其慚悚。



 景仁性本卑謙,及用胡人、巷伯之勢,坐致通顯,志操頗改,漸成驕傲。良馬輕裘,徒從擁冗;高門廣宇,當衢向術。諸子不思其本,自許貴游。自倉頡以來,八體取進,一人而已。



 權會,字正理,河間鄭人也。志尚沈雅,動遵禮則。少受鄭《易》,妙盡幽微;《詩》、《書》、《二禮》,文義該洽;兼明風角,妙識玄象。仕齊,初四門博士。



 僕射崔暹引為館客,甚敬重焉,命世子達挐盡師傅之禮。暹欲薦會與馬敬德等為諸王師。會性恬靜,不慕榮勢,恥於左宦,固辭。暹識其意,遂罷薦舉。尋追修國史,監知太史局事。後遷國子博士。會參掌雖繁,教授不闕。性甚儒綍,似不能言,及臨機答難,酬報如響,由是為諸儒所推。而貴游子弟慕其德義者,或就其宅,或寄宿鄰家,晝夜承間,受其學業。會欣然演說,未嘗懈怠。雖明風角玄象,至於私室,都不及言。學徒有請
 問者,終無所說。每云:「此學可知不可言,諸君並貴游子弟,不由此進,何煩問也。」唯有一子,亦不授此術。會曾遣家人遠行,久而不反。其行還將至,乃逢寒雪,寄息他舍。會方處學堂講說,忽有旋風吹雪入戶,會笑曰:「行人至,何意中停!」遂使追尋,果如其語。會每占筮,大小必中。但用爻辭彖象,以辨吉凶。《易》占之屬,都不經口。



 會本貧生,無僮僕,初任助教日,恆乘驢。其職事處多,非晚不歸。曾夜出城東門,會獨乘一驢。忽有二人,一人牽頭,一人隨後,有似相助。其迴動輕漂,有異生人。漸失路,不由本道。心甚怪之,遂誦《易經》上篇第一卷。不盡,前後二人,忽然
 離散。會亦不覺墮驢,迷悶,至明始覺。方知墮處乃是郭外,纔去家數里。



 有一子,字子襲,聰敏精勤,幼有成人之量。先亡。臨送者為其傷慟,會唯一哭而罷,時人尚其達命。武平末,自府還第,在路無故馬倒,遂不得語,因暴亡。注《易》一部,行於世。會生平畏馬,位望既至,不得不乘,果以此終。



 張思伯,河間樂城人也。善說《左氏傳》,為馬敬德之次。撰《刊例》十卷,行於時。亦為《毛詩》章句。以二經教授齊安王廓。位國子博士。



 又有長樂張奉禮,善《三傳》,與思伯齊名。位國子助教。



 張彫武,中山北平人也。家世寒微。其兄蘭武,仕尚書令史,微有資產。故護軍長史王元則時為書生,停其宅。彫武少美貌,為元則所愛悅,故偏被教。因好學,精力絕人,負卷從師,不遠千里。遍通《五經》,尤明《三傳》。弟子遠方就業者,以百數,諸儒服其強辯。齊神武召入霸府,令與諸子講說。乾明初,累遷平原太守,坐贓賄失官。武成即位,以舊恩,除通直散騎常侍。瑯邪王儼求博士,有司以彫武應選,時號得人。歷涇州刺史、散騎常侍。



 及帝侍講馬敬德卒,乃入授經書。帝甚重之,以為侍講,與侍書張景仁並被尊禮,同入華元殿,共讀《春秋》。加國子祭酒、假儀
 同三司,待詔文林館。以景仁宗室,自託於其親何洪珍,公私之事,彫武常為其指南。與張景仁號二張博士。時穆提婆、韓長鸞與洪珍同侍帷幄,知彫武為洪珍謀主,忌惡之。洪珍又奏彫武監國史。尋除侍中,加開府,奏度支事。大被委任,言多見從,特敕奏事不趨,呼為博士。



 彫武自以出於微賤,致位大臣,勵精在公,有匪躬之節。議論無所回避,左右縱恣之徒,必加禁約。數譏切寵要,獻替帷扆。帝亦深倚仗之,方委以朝政。彫武便以澂清為己任,意氣甚高。嘗在朝堂謂鄭子信曰:「向入省中,見賢家唐令處分,極無所以。若作數行兵帳,彫武不如邕;若
 致主堯、舜,身居稷、契,則邕不如我。」



 長鸞等陰圖之。及與侍中崔季舒、黃門侍郎郭遵諫幸晉陽,為長鸞所譖,誅。臨刑,帝使段孝言詰之。彫武曰:「臣起自諸生,光寵隆洽。今者之諫,臣實首謀,意善功惡,無所逃死。願陛下珍愛金玉,開發神明,數引賈誼之倫,語其政道,令聽覽之間,無所擁蔽,則臣雖死,猶生之年。」因歔欷流涕,俯而就戮。左右莫不憐而壯之。



 子德沖等徙北邊。南安王思好之反,德沖及弟德揭俱免。德沖聰敏好學,以帝師之子,早見旌擢,位中書舍人。其父之戮,德沖並在殿廷就執,目見冤酷,號哭,殞絕於地,久之乃蘇。



 郭遵者,鉅鹿人也。齊文宣為太原公時,為國常侍。帝家人有蓋豐洛者,典知家務,號曰蓋將。遵因其處分,曾抗拒,為高德正所貴。齊受禪,由是擢為主書,專令訪察。中書舍人硃謂為鉅鹿太守,遵為弟子求官,謂啟文宣,鞭之二百,付京畿。久之,除並省尚書都令史、建州別駕。會韓長鸞父永興為刺史,因此遂相參附。



 後擢為黃門侍郎,被誅。



 遵出自賤微,易為盈滿。宮門逢諸貴,輒呼姓字,語言布置,極為輕率。嘗於宮門牽韓長鸞,辭曰:「王在得言。主上縱放如此,曾不規諫,何名大臣?」長鸞嫌其率爾,便掣手而去,由是不加援,故及於禍。



\end{pinyinscope}