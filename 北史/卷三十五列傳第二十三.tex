\article{卷三十五列傳第二十三}

\begin{pinyinscope}

 王
 慧龍玄孫松年五世孫劭鄭羲孫述祖從曾孫道邕道邕子譯譯叔祖儼儼族孫偉王慧龍,太原晉陽人,晉尚書僕射愉之孫,散騎常侍郎緝之子也。幼聰慧,愉以為諸孫之龍,故名焉。初,宋武微時,愉不為之禮,及得志,愉合家見誅。慧龍年十四,為沙門僧彬所匿,因將過江。津人見其行意匆匆,疑為王氏子孫。彬稱為受業者,乃免。既濟,遂西上江陵,依叔祖忱
 故吏荊州前中從事習辟彊。時刺史魏詠之卒,辟彊與江陵令羅修、前別駕劉期公、土人王騰等謀舉兵,推慧龍為盟主,剋日襲州城。而宋武聞詠之卒,亦懼江陵有變,遣其弟道規為荊州,眾遂不果。羅修等將慧龍又與僧彬北詣襄陽。晉雍州刺史魯宗之資給慧龍,送度江,遂奔姚興。



 自言也如此。



 姚泓滅,慧龍歸魏。明元引見與言,慧龍請效力南討。言終,俯而流涕,天子為之動容。謂曰:「朕方混一車書,席卷吳會,卿情計如此,豈不能相資以眾乎?」



 然亦未之用。後拜洛城鎮將,鎮金墉。會明元崩,太武初即位,咸謂南人不宜委以師旅之任,遂停前授。



 初,崔浩弟恬聞慧龍王氏子,以女妻之。浩既婚姻,及見慧龍,曰:「信王家兒也。」王氏世皻鼻,江東謂之「皻王」。慧龍鼻漸大,浩曰:「真貴種矣!」數向諸公稱其美。司徒長孫嵩聞之不悅,言於太武,以其嗟服南人則有訕鄙國化之意。



 太武怒,召浩責之。浩免冠陳謝得釋。慧龍由是不調。久之,除樂安王範傅,領并、荊、揚三州大中正。慧龍抗表,願得南垂自效,崔浩固言之,乃授南蠻校尉、安南大將軍左長史。及宋荊州刺史謝晦起兵江陵,引慧龍為援。慧龍督司馬靈壽等一萬人,拔其思陵戍,進圍項城。晦敗,乃班師。後宋將王玄謨寇滑臺,詔假慧龍楚兵將軍,
 與安頡等同討之。相持五十餘日,諸將以賊盛,莫敢先,慧龍設奇兵大破之。



 太武賜以劍馬錢帛,授龍驤將軍,賜爵長社侯,拜滎陽太守,仍領長史。在任十年,農戰並修,大著聲績,招攜邊遠,歸附者萬餘家,號為善政。



 其後宋將到彥之、檀道濟頻頓淮、潁,大相侵掠;慧龍力戰,屢摧其鋒。彥之與友人蕭斌書曰:「魯軌頑鈍,馬楚粗狂,亡人之中,唯王慧龍及韓延之可為深憚。



 不意儒生懦夫,乃令老子訝之。」宋文縱反間,云慧龍自以功高而位不至,欲引寇入邊,因執安南大將軍司馬楚之以叛。太武聞曰:「此必不然,是齊人忌樂毅耳。」



 乃賜慧龍璽書曰:「義
 隆畏將軍如虎,欲相中害,朕自知之。風塵之言,想不足介意也。」宋文計既不行,復遣刺客呂玄伯購慧龍首二百戶男、絹一千匹。玄伯為反間來,屏人有所論。慧龍疑之,使人探其懷有尺刀。玄伯叩頭請死。慧龍曰:「各為其主也,吾不忍害此人。」左右皆言義隆賊心未巳,不殺玄伯,無以創將來。慧龍曰:「死生有命,彼亦安能害我。且吾方以仁義為干鹵,又何憂乎刺客。」遂捨之。時人服其寬恕。



 慧龍自以遭難流離,常懷憂悴,乃作《祭伍子胥文》以寄意焉。生一男一女,遂絕房室,布衣蔬食,不參吉事,舉動必以禮。太子少傅游雅言於朝曰:「慧龍,古之遺孝也。」
 撰帝王制度十八篇,號曰《國典》。真君元年,拜使持節、寧南將軍、武牢鎮都副將,未至鎮而卒。臨沒,謂功曹鄭曄曰:「吾羈旅南人,恩非舊結,蒙聖朝殊特之慈,得在疆場效命,誓願鞭屍吳市,戮墳江陰。不謂嬰此重疾,有心莫遂,非唯仰愧國靈,實亦俯慚后土。修短命也,夫復何言!身歿後,乞葬河內州縣之東鄉,依古墓而不墳,足藏髮齒而已。庶其魂而有知,猶希結草之報。」時制,南人入國者,皆葬桑乾。曄等申遺意,詔許之。贈安南將軍、荊州刺史,謚穆侯。



 吏人及將士共於墓所起佛寺,圖慧龍及僧彬像而贊之。呂玄伯感全宥之恩,留守墓側,終身不去。
 子寶興襲爵。



 寶興少孤,事母至孝。尚書盧遐妻,崔浩女也。初,寶興母及遐妻俱孕,浩謂曰:「汝等將來所生,皆我之自出,可指腹為親。」及昏,浩為撰儀,躬自監視,謂諸客曰:「此家禮事,宜盡其美。」及浩被誅,盧遐後妻寶興從母也,緣坐沒官。



 寶興亦逃避,未幾得出。盧遐妻時官賜度斤鎮高車滑骨,寶興盡賣貨產,自出塞贖之以歸。州辟中從事、別駕,舉秀才,皆不就。閉門不交人事。襲爵封長社侯、龍驤將軍。卒,子瓊襲爵。



 瓊字世珍,孝文賜名焉。太和九年,為典寺令六十年,降侯為伯。帝納其長女為嬪,拜前將軍、并州大中正。正始中,為光州刺史,有受納響,
 為中尉王顯所劾,終得雪免。神龜中,除左將軍、兗州刺史。去州歸京,多年沈滯。所居在司空劉騰宅西,騰雖勢傾朝野,初不候之。騰既權重,吞并鄰宅,增廣舊居,唯瓊終不肯與,以此久見屈抑。



 瓊女適范陽盧道亮,不聽歸其夫家。女卒,哀慟無已,瓊仍葬之別所,冢不即塞,常於壙內哭泣,久之乃掩,當時深怪之。加以聾疾,每見道俗,乞丐無已,造次見之,令人笑愕。道逢太保、廣平王懷,據鞍抗禮,自言馬瘦;懷即以誕馬并乘具與之。嘗詣尚書令李崇,騎馬至其黃閣,見崇子世哲,直問繼伯在否。崇趨出,瓊乃下。崇儉而好以紙帖衣領,瓊哂而掣去之。崇
 小子青肫嘗盛服,寵勢亦不足恨。



 領軍元叉使奴遺瓊馬,瓊并留奴。王誦聞之笑曰:「東海之風,於茲墜矣。」孝昌三年,除鎮東將軍、金紫光祿大夫、中書令。時瓊子遵業為黃門郎,故有此授。卒,贈征北將軍、中書監、并州刺史。自慧龍入國,三世一身,至瓊始有四子。



 長子遵業,風儀清秀,涉歷經史。位著作佐郎,與司徒左長史崔鴻同撰起居注。



 遷右軍將軍、兼散騎常侍,慰勞蠕蠕。乃詣代京,采拾遺文,以補起居所闕。與崔光、安豐王延明等參定服章。及光為孝明講《孝經》,遵業預講,延業錄義,並應詔作《釋奠侍宴詩》。時人語曰:「英英濟濟,王家兄弟。」轉司
 徒左長史、黃門郎,監典儀注。



 遵業有譽當時,與中書令陳郡袁翻、尚書瑯邪王誦並領黃門郎,號曰三哲。時政歸門下,世謂侍中、黃門為小宰相。而遵業從容恬素,若處丘園。嘗著穿角履,好事者多毀新履以學之。以胡太后臨朝,天下方亂,謀避地,自求徐州。太后曰:「王誦罷幽州始作黃門,卿何乃欲徐州也?更待一二年,當有好處分。」遵業兄弟並交游時俊,乃為當時所美。及爾朱榮入洛,兄弟在父喪中,以於莊帝有從姨兄弟之親,相率奉迎,俱見害河陰。議者惜其人才,而譏其躁競。贈并州刺史。著《三晉記》十卷。



 子松年,少知名,齊文襄臨并州,辟為
 主簿。累遷通直散騎常侍,副李緯使梁。



 使還,歷位尚書郎中。魏收撰《魏書》成,松年有謗言。文宣怒,禁止之,仍加杖罰。歲餘得免,除臨漳令。遷司馬、別駕、本州大中正。孝昭擢拜給事黃門侍郎。



 帝每賜坐,與論政事,甚善之。孝昭崩,松年馳驛至鄴都宣遺詔。發言涕泗,迄於宣罷,容色無改,辭吐諧韻,宣訖號慟,自絕於地,百官莫不感慟。還晉陽,兼侍中,護梓宮還鄴。諸舊臣避形迹,無敢盡哀,唯松年哭必流涕,朝士咸恐。武成雖忿松年戀舊情切,亦雅重之。以本官加散騎常侍,食高邑縣幹。參定律令,前後大獄多委焉。兼御史中丞。發晉陽之鄴,在道遇疾
 卒。贈吏部尚書,并州刺史,謚曰平。第二子劭最知名。



 劭字君懋,少沈默,好讀書。仕齊,累遷太子舍人,待詔文林館。時祖孝徵、魏收、陽休之等嘗論古事,有所遺忘,討閱不能得。問劭,劭具論所出,取書驗之,一無舛誤。自是大為時人所許,稱其博物。後遷中書舍人。齊滅入周,不得調。隋文帝受禪,授著作佐郎,以母憂去職。在家著《齊書》,時制禁私撰史,為內史侍郎李元操所奏。上怒,遣收其書,覽而悅之。於是起為員外散騎侍郎,修起居注。



 劭以上古有鑽燧改火之義,近代廢絕,於是上表請變火曰:「臣謹案《周官》:『四時變火,以救時疾。』明火不數變,時疾
 必興。聖人作法,豈徒然也?在晉時,有人以洛陽火度江者,世世事之,相續不滅,火色變青。昔師曠食飯,云是勞薪所爨,晉平公使視之,果然車輞。今溫酒及炙肉,用石炭、木炭火、竹火、草火、麻荄火,氣味各不同。以此推之,新火舊火,理應有異。伏願遠遵先聖,於五時取五木以變火。用功甚少,救益方大。縱使百姓習久,未能頓同,尚食內廚及東宮諸王食廚,不可不依古法。」上從之。劭又言上有龍顏戴乾之表,指示群臣。上大悅,賜物數百段,拜著作郎。上表言符命曰:昔周保定二年,歲在壬午,五月五日,青州黃河變清,十里鏡澈。齊氏以為己瑞,改元,年
 曰河清。是月,至尊以大興公始作隨州刺史。歷年二十,隋果大興。



 臣謹案《易·坤靈圖》曰:「聖人受命,瑞必先見於河。」河者最濁,未能清也。



 竊以靈貺休祥,理無虛發;河清啟聖,實屬大隋。午為鶉火,以明火德;仲夏火王,亦明火德。月五日五,合天地數,既得受命之辰,允當先見之兆。



 開皇初,邵州人楊令悊近河得青石圖一、紫石圖一,皆隱起成文,有至尊名,下云「八方天心」。永州又得石圖,剖為兩段,有楊樹之形,黃根青葉。汝水得神龜,腹下有文曰「天卜楊興」。安邑掘地得古鐵板,文曰「皇始天年,齎楊鐵券,王興」。同州得石龜,文曰「天子延千年,大吉」。臣以前
 之三石,不異《龍圖》。



 何以用石?石體久固,義與上名符合。龜腹七字何以著龜?龜亦久固,兼是神靈之物。孔子歎河不出圖,洛不出書。今於大隋聖世,圖書屢出。建德六年,亳州大周村有龍鬥,白者勝,黑者死。大象元年夏,熒陽汴水北有龍鬥。初見白氣屬天,自東方歷陽武而來。及至,白龍也,長十許丈。有黑龍乘雲而至,雲雨相薄,乍合乍離,自午至申,白龍昇天,黑龍墜地。謹案:龍,君象也。前鬥於亳州周村者,蓋象至尊以龍鬥之歲為亳州總管,遂代周有天下。後鬥於熒陽者,熒字三火,明火德之盛也。白龍從東方來,歷陽武者,蓋象至尊將登帝位,從
 東第入自崇陽門也。西北昇天者,當乾位天門。



 《坤靈圖》曰:「聖人殺龍,龍不可得而殺,皆感氣也。」又曰:「泰,姓商名宮,黃色,長八尺,六十世。河龍以正月辰見,白龍與五黑龍鬥,白龍陵,故泰人有命。」謹案此言,皆為大隋而發也。「聖人殺龍」者,前後龍死是也。「姓商」



 者,皇家於五姓為商也。「名宮」者,武元皇帝諱於五聲為宮。「黃色」者,隋色尚黃。「長八尺」者,武元皇帝身長八尺。「河龍以正月辰見」者,《泰》,正月卦,龍見之所於京師為辰地。「白龍與黑龍鬥」者,亳州、熒陽龍鬥是也。勝龍所以白者,楊姓納音為商,至尊又辛酉歲生,位皆在西方,西方白色也。死龍所以黑者,
 周色黑。所以稱五者,周閔、明、武、宣、靖凡五帝;越、陳、代、越、滕五王一時伏法,亦當五數。「白龍陵」者,陵猶勝也。鄭玄說「陵」當為「除」,凡鬥能去敵曰除。臣以「泰人有命」者,泰之為言,通也,大也,明其人道通德大,有天命也。《乾鑿度》曰:「泰表戴干。」鄭玄注云:「表者,人形體之彰識也。



 乾,盾也。泰人之表,戴干。」臣伏見至尊有戴乾之表,益知泰人之表,不爽毫釐。



 《坤靈圖》所云,字字皆驗。緯書又稱漢四百年,終如其言,則知六十世亦必然矣。



 昔宗周卜世三十,今則倍之。



 《稽覽圖》曰:「太平時,陰陽和合,風雨會同,海內不偏。地有阻險,故風有遲疾。雖太平之政猶有不能均,惟
 平均乃不鳴條,故欲風於亳。亳者陳留也。」



 謹案此言,蓋明至尊昔為陳留公世子,亳州總管,遂受天命,海內均同,不偏不黨,以成太平之風化也。在大統十六年,武元皇帝改封陳留公。是時,齊國有祕記云:「天王陳留入並州。」齊主高洋為是誅陳留王彭樂。其後,武元皇帝果將兵入并州。



 周武帝時,望氣者云「亳州有天子氣」,於是殺亳州刺史紇豆陵恭。至尊代為之。



 又陳留老子祠有枯柏,世傳云老子將度世,云:「待枯柏生東南枝,迴指,當有聖人出,吾道復行。」至齊,枯柏從下生枝,東南上指,夜有三童子相與歌曰:「老子廟前古枯樹,東南枝如傘,聖主
 從此去。」及至尊牧亳州,親至祠樹之下,自是柏枝回抱,其枯枝漸指西北,道教果行。考校眾事,太平主出於亳州陳留之地,皆如所言。《稽覽圖》又云:「政道得,則陰物變為陽物。」鄭玄注云:「蔥變為韭,亦是。」謹案自六年以來,遠近山石多變為玉。石為陰,玉為陽。又左衛園中,蔥皆變為韭。



 上覽之大悅,賜物五百段。未幾,劭復上書曰:《易·乾鑿度》曰:「《隨》,上六,拘係之,乃從維之,王用享于西山。



 《隨》者,二月卦。陽德施行,蕃決難解,萬物隨陽而出。故上六欲九五拘係之,維持之,明被陽化而欲陰隨從之也。」《易·稽覽圖》:「《坤》六月,有子女任政,一年傳為《復》。五月,貧之從東
 北來立,大起土邑;西北地動星墜,陽衛。



 《屯》十一月,神人從中山出,趙地動;北方三十日,千里馬數至。」謹案:凡此《易緯》所言,皆是大隋符命。《隨》者,二月之卦,明大隋以二月即皇帝位也。



 「陽德施行」者,明楊氏之德教施行於天下也。「蕃決難解」者,明當時蕃鄣皆通決,險難皆解散也。「萬物隨陽而出」者,明天地間萬物盡隨楊氏而出見也。「上六欲九五拘係之」者,五為王,六為宗廟,明宗廟神靈欲命登九五之位,帝王拘人以禮,係人以義也。「拘人以禮,係人以義」,此二旬,亦是《乾鑿度》之言。



 「維持之」者,明能以綱維持正天下也。」被陽化而欲陰隨從之」者,明諸陰
 類被服楊氏之風化,莫不隨從。陰,謂臣下也。「王用享於西山」者,蓋明至尊常以歲二月幸西山仁壽宮也。凡四稱「隨」,三稱「陽」,欲美隨楊,丁寧之至也。「《坤》六月」者,坤位在未,六月建未,言至尊以六月生也。「有子女任政」者,言樂平公主是皇帝子女,而為周后,任理內政也。「一年傳為《復》」者,《復》是《坤》之一世卦,陽氣初起,言周宣帝崩後一年,傳位與楊氏也。「五月,貧之從東北來立」,「貧之」當為「真人」,字之誤也。言周宣帝以五月崩,真人革命,當在此時。至尊謙讓而逆天意,故踰年乃立。昔為定州總管,在京師東北,本而言之,故曰「真人從東北來立」。「大起土邑」者,大
 起,即大興城邑也。「西北地動星墜」者,蓋天意去周授隋,故變動也。「陽衛」者,言楊氏得天衛助也。「《屯》,十一月,神人從中山出」者,此卦動而大亨作,故至尊以十一月被授亳州總管,將從中山而出也。「趙地動」者,中山為趙地,以神人將去,故變動也。



 「北方三十日」者,蓋至尊從北方將往亳州之時,停留三十日也。「千里馬」者,蓋至尊舊所乘騧騮馬也。《屯》卦,震下坎上,震於馬為作足,坎於馬為美脊,是故騧馬脊有肉鞍,行則先作弄四足也。「數至」者,言歷數至也。



 《河圖·帝通紀》曰:「形瑞出,變矩衡。赤應隨,葉靈皇。」《河圖·皇參持》曰:「皇辟出,承元訖。道無為,安率。被遂矩,
 戲作術。開皇色,握神日。



 投輔提,象不絕。立皇後,翼不格。道終始,德優劣。帝任政,河典出。葉輔嬉,爛可述。」謹案:凡此《河圖》所言,亦是大隋符命。「形瑞出,變矩衡」者,矩,法也;衡,北斗星名,所謂璇璣玉衡者也。大隋受命,形兆之瑞始出,天象則為之變動。北斗主天之法度,故曰矩衡。《易緯》:「伏戲,矩衡神。」鄭玄注,以為法玉衡之神。與此《河圖》矩衡義同。「赤應隨」者,言赤帝降精,感應而生隋也。



 故隋以火德為赤帝天子。「葉靈皇」者,葉,合也,言大隋德合上靈天皇大帝也。



 又年號開皇,與《靈寶經》之開皇年相合,故曰葉靈皇。「皇辟出」者,皇,大也;辟,君也。大君出,蓋謂至尊
 受命出為天子也。「承元訖」者,言承周天元終訖之運也。「道無為,安率」者,「安」下脫一字,言大道無為,安定,天下率從。



 「被遂矩,戲作術」者,矩,法也,昔遂皇握機矩,伏戲作八卦之術,言大隋被服彼二皇之法術也。「遂皇機矩」,語見《易緯》。「開皇色」者,言開皇年易服色也。「握神日」者,言握持群神,明照如日也。又開皇以來日漸長,亦其義也。



 「投輔提」者,言投授政事於輔佐,使之提挈也。「象不絕」者,法象不廢絕也。



 「立皇後,翼不格」者,格,至也,言本立太子以為皇家後嗣,而其輔翼之人不能至於善也。「道終始,德優劣」者,言前東宮道終而德劣,今皇太子道始而德優也。



 「
 帝任政,河典出」者,言皇帝親任政事,而邵州河濱得石圖也。「葉輔嬉,爛可述」者,葉,合也;嬉,興也。言群臣合心輔佐,以興政教,爛然可紀述也。所以於《皇參持》、《帝通紀》二篇,大陳符命者,明皇道帝德盡在於隋也。



 上大悅,以劭至誠,寵錫日隆。



 時有人於黃鳳泉浴,得二白石,頗有文理。遂附其文以為字,復言有諸物象,而上奏曰:「其大玉有日月、星辰、八卦、五岳及二麟、雙鳳、青龍、朱雀、騶虞、玄武,各當其方位。又有五行、十日、十二辰之各,凡二十七字。又有『天門、地戶、人門、鬼門閉』九字。又有卻非及二鳥。其鳥皆人面,則《抱朴子》所謂千秋萬歲者也。其小玉亦
 有五岳、卻非、虯、犀之象。二玉俱有仙人玉女乘雲控鶴之象。



 別有異狀諸神,不可盡識,蓋是風伯、雨師、山精、海若之類。又有天皇大帝、皇帝及四帝坐,鉤陳、北斗、三公、天將軍、土司空、老人、天倉、南河、北河、五星、二十八宿凡四十五官。諸字本無行伍,皆往往偶對。於大玉則有皇帝日名,並臨南面,與日字正鼎足。復有老人星,蓋明南面象日,而長壽也。皇后二字在西,上有月形,蓋明象月也。於次玉,則皇帝名與九千字次比,兩楊字與萬年字次比,隋與吉字正並,蓋明長久吉慶也。」劭復回互其字,作詩二百八十篇奏之。上以為誠,賜帛千匹。



 劭於是採
 人間歌謠,引圖書讖緯,依約符命,捃摭佛經,撰為《皇隋靈感誌》合三十卷,奏之。上令宣示天下。劭集諸州朝集使,洗手焚香,閉目讀之。曲折其聲,有如歌詠,經涉旬朔,遍而後罷。上益喜,賞賜優洽。



 及文獻皇后崩,劭復上言:「佛經說人應生天上及上品上生無量壽國之時,天佛放大光明,以香花妓樂來迎之。如來以明星出時入涅盤。伏惟大行皇后,聖德仁慈;福善禎符,備諸祕記,皆云是妙善菩薩。臣謹案:八月二十二日,仁壽宮內再雨金銀之花;二十三日,大寶殿後,夜有神光;二十四日卯時,永安宮北,有自然種種音樂,震滿虛空。至五更中,奄然
 如寐,便即升遐。與經文所說,事皆符驗。



 臣又以愚意思之,皇后遷化不在仁壽大興宮者,蓋避至尊常居正處也。在永安宮者,象京師永安門,平生所出入也。后升遐後二日,苑內夜有鐘聲二百餘處,此則生天之應,顯然也。」上覽之,且悲且喜。時蜀王秀以罪廢,上謂劭曰:「嗟乎!」吾有五子,三子不才。」劭進曰:「自古聖帝明王,皆不能移不肖之子。黃帝二十五子,同姓者二,餘各異德。堯十子,舜九子,皆不肖。夏有五觀,周有三監。」上然其言。後上夢欲上高山而不能得,崔彭捧腳,李盛扶肘,乃得上。因謂彭曰:「死生當與爾俱。」劭曰:「此夢大吉。上高山者,明高崇
 大安,永如山也。彭猶彭祖,李猶李老,二人扶侍,實為長壽之徵。」上聞之,喜見容色。其年,上崩,未幾,崔彭亦卒。



 煬帝嗣位,漢王諒作亂,帝不忍誅。劭上書曰:「臣聞黃帝滅炎,蓋雲母弟;周公誅管,信亦天倫;叔向戮叔魚,仲尼謂之遺直;石蠟殺石厚,丘明以為大義。



 此皆經籍明文,帝王常法。今陛下置此逆賊,度越前聖。謹案:賊諒毒被生靈者也。



 古者同德則同姓,德不同則異姓,故黃帝有二十五子,其得姓者十有四人,唯青陽、夷鼓與黃帝同為姬姓。諒既自絕,請改其氏。」劭以此求媚,帝依違不從。後遷祕書少監,卒於官。



 劭在著作,將二十年,專典國史,撰《
 隋書》八十卷。多錄口敕。又採迂怪不經之語,及委巷之言,以類相從,為其題目。詞義繁雜,無足稱者。遂使隋代文武名臣善惡之迹,堙滅無聞。初撰《齊志》為編年體二十卷,復為《齊書》,紀傳一百卷,及《平賊記》三卷,或文詞鄙野,或不軌不物,駭人視聽,大為有識嗤鄙。



 然其指摘經史謬誤,為《讀書記》三十卷,時人服其精博。爰自志學,暨于暮齒。



 篤好經史,遺略世事。用思既專,性頗恍忽,每至對食,閉目凝思,盤中之肉,輒為僕從所啖。劭弗之覺,唯責肉少,數罰廚人。廚人以情白劭,劭依前閉目,伺而獲之。廚人方免笞辱。其專固如此。



 遵業弟廣業,性沈雅,涉
 歷書傳,位太尉祭酒,遷屬。卒於太中大夫,贈徐州刺史。子乂,有儀望,以幹用見稱,卒於南鉅鹿太守。



 廣業弟延業,博學多聞,頗有才藻,位中書郎。河陰之役,遂亡骸骨。乂無子,贈齊州刺史。延業弟季和,位書侍御史、并州大中正,贈華州刺史。



 鄭羲,字幼麟,滎陽開封人,魏將作大匠渾之八世孫也。曾祖豁,慕容垂太常卿。父曄,不仕。娶長樂潘氏,生六子,粗有志氣,而羲第六,文學為優。弱冠舉秀才,尚書李孝伯以女妻之。文成末,拜中書博士。



 天安初,宋司州刺史常珍奇據汝南來降,獻文詔殿中尚書元石為都將赴
 之,遣羲參石軍事。到上蔡,珍奇率文武三百人來迎。既相見,議欲頓軍汝北,未即入城。



 羲謂石曰:「機事尚速,今珍奇雖來,意未可量。不如直入其城,奪其管籥,據有府庫。雖出珍奇非意,要以全制為勝。」石從羲言,遂策馬徑入其城。城中尚有珍奇親兵數百人,在珍奇宅內。石既克城,意益憍怠,置酒嬉戲,無警防之虞。羲勸嚴兵設備,以待非常。其夜,珍奇果使人燒府,欲因救火作難,以石有備,乃止。



 明旦,羲齎白武幡安慰郭邑,眾心乃定。明年,又引軍東討汝陰。宋汝陰太守張超城守不下,石攻之不克,議欲還軍長社,待秋擊之。羲曰:「今超驅市人,命不
 延月,宜安心守之。超食已盡,不降當走。而欲棄還長社,超必修城深塹,多積薪穀,將來恐難圖矣。」石不納,遂旋師長社。至冬,復往攻超,超果設備,無功而還。



 歷年,超死,楊文長代戍,食盡城潰,乃克之,竟如羲策。淮北平,遷中書侍郎。



 延興初,陽武人田智度年十五,妖惑動眾,擾亂京索。以羲河南人望,為州郡所信,遣乘傳慰喻。羲到,宣示禍福,眾皆散,智度尋見禽斬。以功賜爵泰昌男。



 孝文初,兼員外散騎常侍、寧朔將軍、陽武子,使於宋。



 中山王睿寵幸當世,並置王官,羲為其傅。是後歷年不轉,資產亦乏,因請假歸,遂盤桓不返。及李沖貴寵,與羲昏姻,乃
 就家徵為中書令。文明太后為父燕宣王立廟於長安,初成,以羲兼太常卿,假滎陽侯,具官屬,詣長安拜廟,建碑於廟門。還,以使功,仍賜侯爵。



 出為西兗州刺史,假南陽公。羲多所受納,政以賄成。性又嗇吝,人有禮餉者,不與杯酒臠肉,而西門受羊酒,東門沽賣之。以李沖之親,法官不之糾也。酸棗令鄭伯孫、鄄城令董騰、別駕賈懷德、中從事申靈度並在任廉貞,勤恤百姓,羲皆申表稱薦,時論多之。文明太后為孝文納其女為嬪,徵為祕書監。太和十六年卒,尚書奏謚曰宣。詔曰:「蓋棺定謚,先典成式;激揚清濁,政道明範。羲雖宿有文業,而政闕廉清。
 尚書何乃情遺至公,愆違明典?依謚法,博聞多見曰文,不勤成名曰靈,可贈以本官,加謚文靈。」



 長子懿,字景伯,涉歷經史。位太子中庶子,襲爵滎陽伯。懿閑雅有政事才,為孝文所器遇,拜長兼給事黃門侍郎、司徒左長史。宣武初,以從弟思和同咸陽王禧逆,與弟通直常侍道昭俱坐緦親出禁。拜太常少卿,出為齊州刺史。懿好勸課,善斷決,雖不清潔,義然後取,百姓猶思之。卒,贈兗州刺史,謚曰穆。子恭業襲爵,武定三年,坐與房子遠謀害齊神武,伏誅。



 懿弟道昭,字僖伯,少好學,綜覽群言。兼中書侍郎,從征沔北。孝文饗侍臣於縣瓠方丈竹堂,道昭
 與兄懿俱侍坐。樂作酒酣,孝文歌曰:「白日光天兮無不曜,江左一隅獨未照。」彭城王勰續曰:「願從聖明兮登衡、會,萬國馳誠混日外。」



 鄭懿歌曰:「雲雷大振兮天門闢,率土來賓一正歷。」邢巒歌曰:「舜舞干戚兮天下歸,文德遠被莫不思。」道昭歌曰:「皇風一鼓兮九地匝,戴日依天清六合。」



 孝文又歌曰:「遵彼汝墳兮昔化貞,未若今日道風明。」宋弁歌曰:「文王政教兮暉江召,寧如大化光四表。」孝文謂道昭曰:「自比遷豫雖猥,與諸才俊不廢詠綴,未若今日。」遂命邢巒總集敘記。「當爾之年,卿頻丁艱私,每眷文席,常用慨然」。



 尋正除中書郎,累遷國子祭酒。廣平王
 懷為司州牧,以道昭與宗正卿元匡為州都督。道昭上表曰:臣聞唐、虞啟運,以文德為本;殷、周創業,以道藝為先。然則禮樂者,為國之基,不可斯須廢也。伏惟大魏,定鼎伊、瀍,惟新寶歷。九服感至德之和,四垠懷擊壤之慶。而蠢爾閩吳,阻化江湫;先帝爰震武怒,戎車不息。



 而停鑾駐蹕,留心典墳。命故御史中尉臣李彪,與吏部尚書任城王臣澄等,妙選英儒,以崇學校。澄等依旨,置四門博士四十人。其國子博士、太學博士及國子助教,宿已簡置。伏尋先旨,意在速就;但軍國多事,未遑營立。自爾迄今,垂將一紀,學官彫落,四術寢廢。遂使碩儒耆德,卷
 經而不談;俗學後生,遺本而逐末。進競之風,實由於此矣。伏惟陛下,欽明文思,玄鑒洞遠,垂心經素。優柔墳籍,屢發中旨,敦營學館,房宇既修,生徒未立。臣往年刪定律令,謬預議筵。謹依準前修,尋訪舊事,參定學令,事訖封呈。請早敕施行,使選授有依,生徒可準。」詔褒美之,而尚未允遂。道昭又表曰:「臣自往年以來,頻請學令,並置生員,前後累上,未蒙一報。當以臣識淺濫官,無能有所感悟者也。館宇既修,生房粗構,博士見員,足可講習。雖新令未班,請依舊權置國子學生,漸開訓業,使播教有章,儒風不墜。



 至若孔廟既成,釋奠告始,揖讓之容,請俟
 令出。」不報。遷秘書監,滎陽邑中正,出歷光、青二州刺史,復入為祕書監。卒,謚曰文恭。



 道昭好為詩賦,凡數十篇。其在二州,政務寬厚,不任威刑,為吏人所愛。



 子嚴祖,頗有風儀,粗觀文史,輕躁薄行,不修士業。孝武時,御史中尉綦俊劾嚴祖與宋氏從姊姦通,人士咸恥言之,而嚴祖聊無愧色。孝靜初,除驃騎將軍、左光祿大夫、鴻臚卿,出為北豫州刺史,還除鴻臚卿。卒,贈司空公。



 庶子仲禮,少輕險,有膂力。齊神武嬖寵其姊火車,以親戚被暱,擢為帳內都督。掌神武弓矢,出入隨從。與任胄俱好酒,不憂公事,神武責之。胄懼,潛通西魏,為人糾告,懼,遂謀逆。
 事發,火車欲乞哀,神武避不見。賴武明皇后及文襄爭為言,故仲禮死而不及其家。嚴祖更無子,弟敬祖以子紹元嗣。紹元小字安都,位太尉諮議、趙郡太守,卒。



 子子翻,字靈雀。少有器識,學涉,好文章。齊武平末,位司徒記室參軍。尋遇齊亡,歷周、隋,遂不仕,隱居滎陽三窟山。傲誕不自羈束,或有所之造,乘驢衣韉,破弊而往。遠近欽其高名,皆謂有異狀,觀者如堵。及見,形乃短陋,不副所聞。然風神俊發,無貴賤並敬服之。納言楊素聞其名,因使過滎陽,迎與相見,言談彌日,深加禮重。及歸,言之朝廷,累徵不至。終於家。



 子翻二弟子騰、天壽,俱仕隋。子騰
 位蔣州司馬,天壽開府參軍,並以雅素稱。



 嚴祖弟敬祖,起家著作郎。鄭儼之敗也,為鄉人所害。



 子元禮,字文規。少好學,愛文藻,有名望。齊文襄引為館客,歷兼中書舍人、南主客郎中、太尉諮議參軍、長廣樂陵二郡守,待詔文林館,太子中舍人。崔昂後妻,元禮姊也,魏收又昂之妹夫。昂嘗持元禮數篇詩示盧思道,乃曰:「看元禮比來詩詠,亦曾不減魏收。」思道答云:「未覺元禮賢於魏收,且知妹夫疏於婦弟。」



 元禮,大象中卒於始州別駕。



 敬祖弟述祖,字恭文。少聰敏,好屬文,有風檢,為先達所稱譽。歷位司徒左長史、尚書、侍中、太常卿、丞相右長史。
 齊天保中,歷太子少保、左光祿大夫、儀同三司、兗州刺史。時穆子容為巡省使,嘆曰:「古人有言,聞伯夷之風,貪夫廉,懦夫有立志,今於鄭兗州見之矣。」遷光州刺史。



 初,述祖父為兗州,於鄭城南小山起齋亭,刻石為記。述祖時年九歲。及為刺史,往尋舊迹,得一破石,有銘云:「中岳先生鄭道昭之白雲堂。」述祖對之嗚咽,悲動群寮。有人入市盜布,其父怒曰:「何負吾君?」執之以歸首。述祖特原之,自是境內無盜。百姓歌曰:「大鄭公,小鄭公,相去五十載,風教猶尚同。」



 述祖能鼓琴,自造《龍吟十弄》,云嘗夢人彈琴,寤而寫得。當時以為絕妙。



 所在好為山池,松竹交
 植,盛肴饌以待賓客,將迎不倦。少時在鄉,單馬出行,忽有騎者數百,見述祖皆下馬,曰「公在此」,行列而拜。述祖顧問從人,皆不見,心甚異之。未幾被徵,終歷顯位。及病篤,乃自言之。且曰:「吾老矣,一生富貴足矣,以清白之名遺子孫,死無所恨。」前後行瀛、殷、冀、滄、趙、定六州事,正除懷、兗、光三州刺史,又重行殷、懷、趙三州刺史,所在皆有惠政。天統元年卒,年八十一,贈開府、中書監、北豫州刺史,謚曰平簡公。



 述祖女為趙郡王睿妃,述祖常坐受王拜,命坐,王乃坐。妃薨後,王更娶鄭道蔭女,王坐受道蔭拜。王命坐,乃敢坐。王謂道蔭曰:「鄭尚書風德如此,又貴
 重宿舊,君不得並之。」



 述祖子元德,多藝術,官瑯邪太守。述祖弟遵祖,祕書郎,贈光州刺史。遵祖弟順祖,卒於太常丞。



 自靈太后豫政,淫風稍行;及元叉擅權,公為奸穢,自此素族名家,遂多亂雜。



 法官不加糾正,婚宦無貶,於時有識,咸以歎息矣。



 羲長兄白驎,次小白,次洞林,次叔夜,次連山,並恃豪門,多行無禮,鄉黨之內,疾之若仇。小白位中書博士。子胤伯,有當世器幹,孝文納其女為嬪,位東徐州刺史,卒於鴻臚少卿,謚曰簡。子希俊,未官而卒。子道育,武定中,開封太守。



 希俊弟幼儒,好學修謹,丞相、高陽王雍以女妻之。位司州別駕,有當官稱。



 卒,贈散
 騎常侍、兗州刺史,謚曰肅。幼儒亡後,妻淫蕩兇悖,肆行無禮。幼儒時望甚優,其從兄伯猷每謂所親曰:「從弟人才,足為令德,不幸得如此婦。今死復重死,可為悲歎。」



 幼儒子敬道、敬德,俱仕西魏。敬道並巴、開、新三州刺史。敬道子正則仕周,復州刺史。



 胤伯弟平城,廣陵王羽納其女為妃,位東平原太守。性猜狂使酒,為政貪殘。



 卒,贈南青州刺史。



 長子伯猷,博學有文才,早知名。舉司州秀才,歷太學博士,領殿中御史。與當時名勝,咸申遊款。明帝釋奠,詔伯猷錄義。後為尚書外兵郎中,典起居注,以軍功賜爵陽武子。節閔帝初,以舅氏超授征東將軍、金紫
 光祿大夫,領國子祭酒。



 轉護軍將軍,賜爵武城子。



 元象初,以本官兼散騎常侍使梁。前後使人,梁武令其侯王於馬射之日宴對申禮。伯猷之行,梁武令其領軍將軍臧盾與之接。議者以此貶之。使還,除南青州刺史。在官貪婪,妻安豐王元延明女,專為聚斂,貨賄公行,潤及親戚。戶口逃散,邑落空虛。乃誣陷良善,云欲反叛,籍其資財,盡以入己,誅其丈夫,婦女配沒。



 百姓冤苦,聲聞四方。為御史糾劾,死罪數十條。遇赦免,因以頓廢。齊文襄作相,每誡厲朝士,常以伯猷及崔叔仁為喻。武定七年,除太常卿。卒,贈驃騎大將軍、中書監、兗州刺史。子蘊,太子
 舍人、陽夏太守。伯猷弟仲衡,武定中,儀同開府中郎。



 仲衡弟輯之,司徒諮議。齊大寧中,以軍功賜爵成皋男,位金紫光祿大夫,東濟北太守、肥城戍主。卒,贈度支尚書、北豫州刺史。



 輯之弟懷孝,司徒諮議。齊大寧中,仁州刺史。



 洞林子敬叔,滎陽邑中正、濮陽太守,坐貪穢除名。子籍,字承宗,徐州平東府長史。



 籍弟瓊,字祖珍,有強乾稱,位范陽太守,頗有聲,卒。孝昌中,弟儼寵要,重贈青州刺史。瓊兄弟雍睦,其諸娣姒亦咸相親愛,閨門之內,有無相通,為時人所稱美。子道邕。



 道邕字孝穆。幼謹厚,以清約自居,年未弱冠,涉歷經史。
 父叔四人並早歿,昆服季之中,道邕居長,撫訓諸弟,有如同生,閨庭之中,怡怡如也。魏孝昌初,解褐太尉行參軍,累以戰功進至左光祿大夫、太師咸陽王長史。及孝武西遷,從入關,除司徒左長史,領臨洮王友,賜爵永寧縣侯。



 大統中,行岐州刺史,在任未幾,有能名。王羆時為雍州刺史,欽其善政,貽書盛相稱述。先是,所部百姓,久遭離亂,逃散殆盡。道邕下車之日,戶止三千,留情綏撫,遠近咸至,數年之內,有四萬家。歲考績為天下最,周文帝賜書歎美之。



 徵拜京兆尹。及梁岳陽王蕭察稱蕃,乃假道邕散騎常侍,持節拜察為梁王。使還,稱旨,進儀同三
 司,加散騎常侍。



 時周文東討,除大丞相府右長史,封金鄉縣男。軍次潼關,命道邕與左長史孫儉、司馬楊寬、尚書蘇亮、諮議劉孟良等分掌眾務。仍令道邕引接關東歸附人士,并品藻才行而任用之,撫納銓敘,咸得其宜。後拜中書令,賜姓宇文氏,尋以疾免。



 周孝閔帝踐阼,加驃騎大將軍、開府儀同三司,進爵為子。歷御伯中大夫、御正、宜、華、虞、陜四州刺史。頻歷數州,皆有政績。入為少司空,卒。贈本官,加鄭、梁、北豫三州刺史,謚曰貞。



 子詡嗣,歷位納言,為聘陳使。後至開府儀同大將軍、邵州刺史。詡弟譯於隋文帝有翊贊功,開皇初,又追贈道邕大將
 軍、徐兗等六州刺史,改謚曰文。



 譯字正義。幼聰敏,涉獵群書,工騎射,尤善音樂,有名於世。譯從祖文寬,尚周文帝元后妹魏平陽公主,無子,周文命譯後之。由是譯少為周文所親,恆令與諸子遊集。年十餘歲,嘗詣府司錄李長宗。長宗於眾中戲之,譯斂容謂曰:「明公位望不輕,瞻仰斯屬,輒相玩狎,無乃喪德也。」長宗甚異之。文寬後誕二子,譯復歸本生。



 周明帝時,詔令事輔城公,是為武帝。及帝即位,為左侍上士,與儀同劉昉,恆侍帝側。譯時喪妻,帝令譯尚梁安固公主。及帝親總萬機,以為御正下大夫,頗被顧遇。東宮建,轉太
 子宮尹下大夫,特被太子親待。時太子多失德,內史中大夫烏丸軌每勸帝廢太子立秦王,由是太子恆不自安。建德二年,為聘齊使副。後詔太子西征吐谷渾,太子陰謂譯曰:「秦王,上愛子也;烏丸軌,上信臣也,今吾此行,得無扶蘇之事乎?」譯曰:「願殿下勉著仁孝,無失子道而已。」太子然之。既破賊,譯以功最,賜爵開國子。後坐褻狎皇太子,烏丸軌、宇文孝伯等以聞。帝大怒,除譯名。宮臣親幸者咸被譴。太子復召譯,戲狎如初。因曰:「殿下何時可得據天下?」太子悅而益暱之。例復官,仍拜吏部下大夫。



 及武帝崩,宣帝嗣位,超拜開府儀同大將軍、內史中
 大夫,封歸昌縣公。既以恩舊,任遇甚重,委以朝政。遷內史上大夫,進封沛國公。上大夫之官,自譯始也。



 以其子善願為歸昌公,元琮為永安縣男。又監國史。譯頗專權,時帝幸東京,譯擅取官材,自營私第,坐除名。劉昉數言於帝,帝復召之,顧待如初,詔領內史事。



 初,隋文帝與譯有同學之舊,譯又素知隋文相表有奇,傾心相結。至是,隋文為宣帝所忌,情不自安,嘗在永巷,私於譯曰:「久願出籓,公所悉也,敢布心腹,少留意焉。」譯曰:「以公德望,天下歸心,欲求多福,豈敢忘也?謹即言之。」



 時將遣譯南征。譯曰:「若定江東,自非懿戚重臣,無以鎮撫。可令隋公行,
 且為壽陽總管,以督軍事。」帝從之,乃下詔,以隋文為揚州總管,譯發兵俱會壽陽以代陳。行有日矣,帝不悆,譯遂與御正下大夫劉昉謀,引隋文入受顧託。既而譯宣詔,文武百官,皆受隋文節度。時御正中大夫顏之儀與宦者謀,引大將軍宇文仲輔政。仲已至御坐,譯知之,遽率開府楊惠及劉昉、皇甫績、柳裘俱入。仲與之儀見譯等,愕然,逡巡欲出。隋文因執之。於是矯詔,復以譯為內史上大夫。明日,隋文為丞相,拜譯柱國、府長史,行內史上大夫事。及隋文為大冢宰,總百揆,以譯兼領天官都府司會,總六府事。出入臥內,言無不從,賞賜玉帛,不可
 勝計,每出入以甲士從。拜其子元璹為儀同。時尉遲迥、王謙、司馬消難等作亂,隋文逾加親禮,進上柱國,恕以十死。



 譯性輕險,不親職務,而贓貨狼籍。隋文陰疏之,然以其有定冊功,不忍廢放,陰敕官屬不得白事於譯。譯猶坐事,無所關預,懼,頓首求解職。隋文寬喻之,接以恩禮。及帝受禪,譯以上柱國歸第。賞賜豐厚,進子元璹成皋郡公,元珣永安男,追贈其父及亡兄二人並為刺史。



 譯自以被疏,陰呼道士章醮,以祈福助。其婢奏譯厭蠱左道。帝謂譯曰:「我不負公,此何意也?」譯無以對。譯又與母別居,為憲司所劾,由是除名。下詔云:「譯嘉謀良策,
 寂爾無聞;鬻獄賣官,沸騰盈耳。若留之於世,在人為不道之臣;戮之於朝,入地為不孝之鬼。有累幽顯,無以置之。宜賜以《孝經》,令其熟讀,仍遣與母共居。」



 未幾,詔譯參撰律令。復授開府、隆州刺史。請還療疾,有詔徵之,見於醴泉宮,賜宴甚歡。因謂譯曰:「貶退已久,情相矜愍。」於是顧謂侍臣曰:「鄭譯與朕同生共死,間關危難,興言念此,何日忘之。」譯因奉觴上壽。帝令內史李德林立作詔書,復爵沛國公,位上柱國。高熲戲謂譯曰:「筆幹。」答曰:「出為方岳,杖策言歸,不得一錢,何以潤筆!」上大笑。未幾,詔譯參議樂事。譯以周代七聲廢缺,自大隋受命,禮樂宜新。
 更修七始之義,名曰《樂府聲調》,凡八篇,奏之。



 帝嘉美焉。俄拜岐州刺史。歲餘,復奉詔定樂於太常。帝勞譯曰:「律、令,則公定之;音樂,則公正之。禮、樂、律、令,公居其三,良足美也。」尋還岐州。開皇十一年卒,年五十二,謚曰達。子元璹嗣。煬帝初立,五等悉除,以譯佐命元功,詔追改封譯莘公,以元璹襲。



 元璹歷位右光祿大夫、右衛將軍。大業末,為文城太守,以城歸國。



 瓊弟儼。儼字季然,容貌壯麗。初為司徒胡國珍行參軍,因為靈太后所幸,時人未知之。後太后廢,蕭寶夤西征,以儼為友。及太后反政,儼請使還朝,復見寵待。拜諫議大夫、中書舍人,領尚食典御,
 晝夜禁中,寵愛尤甚。儼每休沐,太后常遣閹童隨侍,儼見其妻,唯得言家事而已。



 與徐紇俱為舍人,儼以紇有智數,仗為謀主。紇以儼寵幸既盛,傾身承接。共相表裏,勢傾內外。城陽王徽亦與之合,當時政令,歸於儼等。遷散騎常侍、車騎將軍,舍人、常侍如故。明帝崩,事出倉卒,天下咸言儼計。爾朱榮舉兵向洛陽,以儼、紇為辭。榮逼京師,儼走歸鄉里。儼從兄仲明欲據郡起眾,尋為其部下所殺,與仲明俱傳首洛陽。子文寬從武帝入關西。



 敬叔弟子恭,燕郡太守。孝昌中,因儼勢,除衛尉少卿,遷衛將軍、左光祿大夫。卒後,贈尚書右僕射,謚曰貞。



 叔夜子
 伯夏,位東萊太守。卒,贈青州刺史。伯夏弟謹,字仲恭,瑯邪太守。



 連山性嚴暴,撾撻僮僕,酷過人理。父子一時為奴所害,斷首投馬槽下,乘馬北逃。其第二子思明,驍勇善騎射,被髮率村義馳追之。及河,奴乘馬投水。思明止將從,自射之,一發而中,落馬墮流,禽至家,臠殺之。



 思明,弟思和,並以武力自效。思明位直閣將軍,坐弟思和同元禧逆,徙邊。



 會赦,免。卒後,贈濟州刺史。



 子先護,少有武幹。莊帝居籓也,先護得自結託。及爾朱榮稱兵向洛,靈太后令先護與鄭季明等守河梁。先護聞莊帝即位於河北,遂開門納榮。以功封平昌縣侯,廣州刺史。元顥入
 洛,莊帝北巡,先護據州起義兵,不受命。莊帝還京,進爵郡公。



 歷東雍、豫二州刺史,兼尚書右僕射。及爾朱榮死,徐州刺史爾朱仲遠擁兵向洛。



 詔先護與都督賀拔勝、行臺楊昱同討之。聞京師不守,先護部眾逃散,因奔梁。尋歸,為仲遠所害。孝武初,贈使持節、都督、四州刺史。子偉。



 偉字子直,少倜儻有大志,每以功名自許,善騎射,膽力過人。爾朱氏滅後,自梁歸魏。及武帝西遷,偉亦歸鄉里,不求仕進。大統三年,河內公獨孤信既復洛陽,偉乃與宗人榮業,糾合州里舉兵於陳留,信宿間,眾有萬人。遂拔梁州,禽東魏刺史鹿永及鎮城守將令狐德,并獲
 陳留郡守趙季和。乃率眾西附。因是,梁、陳間相次降款。偉弛入關西,周文帝與語,歎美之,拜北徐州刺史,封武陽縣伯。從戰河橋及解玉壁圍,偉常先鋒陷陣。侯景歸款,周文命偉率所部應接。及景叛,偉亦全軍而還。除滎陽郡守,進爵襄城郡公,侍中、驃騎大將軍、開府儀同三司。魏恭帝二年,進位大將軍、江陵防主、都督十五州諸軍事。



 偉性粗獷,不遵法度,睚眥之間,便行殺戮。朝廷以其有立義之效,每優容之。



 及在江陵,乃專戮副防主杞賓王,坐除名。保定元年,詔復官爵。天和六年,為華州刺史。偉前後蒞職,皆以威猛為政,吏人莫敢犯禁,盜賊亦
 為之休止。雖非仁政,然頗以此見稱。卒於州,贈本官,加少傅、都督、司州刺史,謚曰肅。



 偉性吃,少時嘗逐鹿於野。失之,遇牧豎,問焉。牧豎答之,其言亦吃。偉怒,謂其效己,遂射殺之。其忍暴如此。子大士嗣。



 述祖族子雛,有識尚,操行清整,仕至膠州刺史。初,齊文宣為皇太子納其女為良娣,雛時為尚書郎,趙郡李祖升兄弟微相敬憚。楊愔奏授雛趙郡太守,祖昇兄弟具服至雛門,投刺拜謁。文宣聞之喜,笑曰:「足得殺李家兒矣。」



 論曰:王慧龍拔難自歸,間關夷險,撫人督眾,見憚嚴敵。世珍實有令子,克播家聲。松年之送終戀舊,有古人風
 矣。劭爰自幼童,訖于白首;好學不倦,究極群書,晉紳洽聞之士,無不推其博物。雅好著述,久在史官,既撰《齊書》,兼修隋典。好詭怪之說,尚委曲之談;文詞鄙穢,體統煩雜,直愧南、董,才無遷、固,徒煩翰墨,不足觀采。經營符瑞,雜以妖訛。為河朔清流而乾沒榮利,得不以道而頹其家聲。惜矣!



 鄭羲機識明悟,為時所許。懿兄弟風尚,俱有可觀,故能並當榮遇,共濟其美。



 述祖德業,足嗣家聲。嚴祖、仲禮,大虧門素。幼儒令問促年。伯猷以賄敗德。道邕撫寧離散,仁惠克舉。譯實受顧託,適足為敗。及帝行明德,義非簡在;鹽梅之寄,固不攸歸。言追昔款,內懷觖望,
 恥居吳、耿之末,羞與絳、灌為伍。事君盡禮,既闕於夙心;不愛其親,遽彰於物議。格之名教,君子所深尤也。儼名編《恩倖》,取辱前載。偉翻然豹變,蓋知機之士乎。



\end{pinyinscope}