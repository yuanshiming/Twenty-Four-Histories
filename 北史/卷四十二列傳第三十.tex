\article{卷四十二列傳第三十}

\begin{pinyinscope}

 王肅劉芳孫逖芳從子懋常爽孫景王肅,字恭懿,瑯邪臨沂人也。父奐,齊雍州刺史,《南史》有傳。肅少聰辯,涉獵經史,頗有大志。仕齊,位秘書丞。父奐及兄弟並為齊武帝所殺。太和十七年,肅自建鄴來奔。孝文幸鄴,聞其至,虛衿待之,引見問故。肅辭義敏切,辯而有禮,帝甚哀惻之。遂語及為國之道。肅所陳說,深會旨,帝促席移景,不覺坐之疲也。



 肅因言蕭氏危亡之兆,可以乘機,帝於是圖南之規轉銳。器重禮遇,日有加焉;親貴舊臣莫之間也,或屏左右,談說至夜分不罷。肅亦盡忠輸誠,無所隱避,自謂君臣之際,
 猶孔明之遇玄德也。尋除輔國、大將軍長史,賜爵開陽伯。肅固辭伯爵,許之。



 詔肅討齊義陽,聽招募壯勇以為爪牙,其募士有功,賞加等。其從肅行者,六品已下聽先擬用,以後聞;若投化人,聽五品已下先即優授。肅至義陽,頻破賊軍,除持節、都督、豫州刺史、揚州大中正。肅善撫接,甚有聲稱。尋徵入朝,帝手詔曰:「不見君子,中心如醉,
 一日三歲,我勞如何。飾館華林,拂席相待,卿欲以何日發汝墳也?」又詔曰:「蕭丁荼虣世,志等伍胥,窮逾再期,蔬縕不改。
 有
 司依禮喻解,為裁練禪之制。」



 二十年七月,帝以久旱不雨輟膳,百寮詣闕。帝在崇虛樓,遣舍人問肅。對曰:「伏承陛下輟膳,已經三日,群臣不敢自寧。臣聞堯水湯旱,自定之數,須聖人以濟,未聞由聖以至災,是以國儲九年,以禦九年之變。昨四郊之外已蒙滂澍,唯京城之內微為少澤。蒸庶未闕一食,陛下輟膳三日,臣庶惶惶,無復情地。」帝遣答曰:「雖不食數朝,猶然無感,朕誠心未至之所致也。朕志確然,死而後已。」



 是夜,澍雨大降。以破齊將裴叔業功,進號鎮南將軍,加都督四州諸軍事,封汝陽縣
 子。肅頻表固讓,不許,詔加鼓吹一部。



 初,齊之收肅父奐也,奐司馬黃瑤起攻奐殺之。二十二年平漢陽,瑤起為輔國將軍,特詔以付肅,使紓泄哀情。



 孝文崩,遺詔以肅為尚書令,與咸陽王禧等同為宰輔,徵會駕魯陽。肅至,遂與禧參同謀謨。自魯陽至京洛,行途喪紀,委肅參量,憂勤經綜,有過舊戚。禧兄弟並敬暱之,上下稱為和輯。唯任城王澄以其起自羈遠,一旦在己之上,每謂人曰:「朝廷以王肅加我上,尚可;從叔廣陵,宗室尊宿,歷任內外,云何一朝令肅居其右也?」肅聞,恒降避之。尋為澄所奏劾,稱肅謀叛,事尋申釋。詔肅尚陳留長公主,本劉昶
 子婦彭城公主也,賜錢二十萬、帛三千疋。肅奏:「考以顯能,陟由績著升明退闇,於是乎在。自百寮曠察,四稔于茲,請依舊例,考檢能否。」從之。



 裴叔業以壽春內附,拜肅使持節、都督江西諸軍事,與彭城王勰率步騎十萬以赴之。齊豫州刺史蕭懿屯小峴,交州刺史李叔獻屯合肥,將圖壽春。肅進師討擊,大破之,禽叔獻,走蕭懿。還京師,宣武臨東堂,引見勞之,進位開府儀同三司,封昌國縣侯。尋為散騎常侍、都督淮南諸軍事、揚州刺史。肅頻在邊,悉心撫接,遠近歸懷,附者若市,咸得其心。清身好施,簡絕聲色,終始廉約,家無餘財。然性微輕恌,頗以功
 名自許,護疵稱伐,少所推下,孝文每以此為言。



 景明二年,薨於壽春,年三十八。宣武為舉哀,給東園祕器、朝服一襲、錢三十萬、帛一千疋、布五百疋、蠟三百斤,并問其卜遷遠近,專遣侍御史一人監護喪事。又詔曰:「杜預之歿,窆於首陽,司空李沖,覆舟是託,顧瞻斯所,亦二代之九原也。故揚州刺史肅,忠義結於二世,英惠符於李、杜。平生本意,願終京陵,既有宿心,宜遂先志。其令葬於沖、預兩墳之間,使之神游相得也。」贈侍中、司空公。有司奏以肅貞心大度,宜謚匡公,詔謚宣簡。明帝初,詔為肅建碑銘。



 自晉氏喪亂,禮樂崩亡,孝文雖釐革制度,變更風
 俗,其間朴略,未能淳也。



 肅明練舊事,虛心受委,朝儀國曲,咸自肅出。子紹襲。



 紹字三歸,位中書侍郎。卒,贈徐州刺史。子遷襲,齊受禪,爵隨例降。



 紹弟理,孝靜初得還朝,位著作佐郎。紹,肅前妻謝生也。肅臨薨,謝始攜女及紹至壽春。宣武納其女為夫人,明帝又納紹女為嬪。



 肅弟康,字文政,涉獵書史,微有兄風。宣武初,攜兄子誦、翊、衍等入魏,拜中書侍郎。卒幽州刺史,贈征虜將軍、徐州刺史。



 誦字國章,肅長兄融之子。學涉有文才,神氣清俊,風流甚美。歷位散騎常侍、光祿大夫、右將軍、幽州刺史、長兼祕書監、給事黃門侍郎。明帝崩,靈太后之立幼主也,
 於時大赦。誦宣讀詔書,言制抑揚,風神竦秀,百寮傾屬,莫不歎美。孝莊初,於河陰遇害,贈尚書左僕射、司空公,謚曰文宣。子孝康,尚書郎中。孝康弟俊賦,性清雅,頗有文才,齊文襄王中外府祭酒。



 誦弟衍,字文舒,名行器藝亞於誦。位光祿大夫、廷尉卿、揚州刺史、大中正、度支七兵二尚書、太常卿。出為散騎常侍、西兗州刺史。為爾朱仲遠所禽,以其名望,不害。令騎牛從軍,久乃見釋遠洛。孝靜初,位侍中。卒,敕給東園秘器,贈尚書令、司徒公,謚曰文獻。衍篤於交舊。有故人竺IQ,於西兗州為仲遠所害,其妻子飢寒,衍置於家,累年贍恤,世人稱其敦厚。



 翊
 字士游,肅次兄深子也。風神秀立,好學有文才。位中書侍郎。頗銳於榮利,結婚於元叉。為濟州刺史,清靜有政績。入為散騎常侍、金紫光祿大夫,領國子祭酒。卒,贈司空公、徐州刺史。子琛,武定中,儀同、開府記室參軍。



 劉芳,字伯支,彭城叢亭里人,漢楚元王交之後也。六世祖訥,晉司隸校尉。



 祖該,宋青、徐二州刺史。父邕,宋兗州長史。芳出後宋東平太守遜之。邕同劉義宣之事,身死彭城。芳隨伯母房逃竄清州,會赦免。舅元慶,為宋青州刺史沈文秀建威府司馬,為文秀所殺。芳母子入梁鄒城。慕容白曜南討青、齊,梁鄒降,芳北徙為平齊人,時年
 十六。



 南部尚書李敷妻,司徒崔浩之弟女,芳祖母,浩之姑也。芳至京師,詣敷門。



 崔恥芳流播,拒不見之。芳雖處窮窘之中,而業尚貞固。聰敏過人,篤志墳典,晝則人庸書以自資給,夜則誦經不寢。至有易衣人併日之弊,而淡然自守,不急急於榮利,不戚戚於貧賤,乃著《窮通論》以自慰。常為諸僧人庸寫經論,筆迹稱善,卷直一縑,歲中能入百餘疋。如此數年,賴以頗振。由是與德學大僧多有還往。時有南方沙門慧度以事被責,未幾暴亡,芳因緣聞知。文明太后召入禁中,鞭之一百。



 時中官李豐主其始末,知芳篤學有志行,言之於太后。微愧於心。會齊使劉
 纘至,芳之始族兄也,擢芳兼主客郎,與纘相接。拜中書博士。後與崔光、宋弁、刑產等俱為中書侍郎。俄而詔芳與產入授皇太子經,遷太子庶子,兼員外散騎常侍。



 從駕洛陽,自在路及旋京師,恒侍坐講讀。芳才思深敏,特精經義,博聞強記,兼覽《蒼雅》,尤長音訓,辯析無疑。於是禮遇日隆,賞賚豐渥。俄兼通直常侍,從駕南巡,撰述行事,尋而除正。



 王肅之來奔也,孝文雅相器重,朝野屬目。芳未及相見。嘗宴群臣於華林,肅語次云:「古者唯婦人有笄,男子則無笄。」芳曰:「推經《禮》正文,古者男子婦人俱有笄。」肅曰:「《喪服》稱男子免而婦人髽,男子冠而婦人笄,
 如此則男子不應有笄。」芳曰:「此專謂凶事也。《禮》:初遭喪,男子免,時則婦人髽;男子冠,時則婦人笄。言俱時變,男子婦人免髽、冠笄之不同也。又冠尊,故奪其笄,且互言也。非謂男子無笄。又《禮內則》稱:『子事父母,雞初鳴,櫛纚笄總。』以茲而言,男子有笄明矣。」高祖稱善者久之。肅亦以芳言為然,曰:「此非劉石經也?」昔漢世造三字石經於太學,學者文字不正,多往質焉。芳音義明辯,疑者皆往詢訪,故時人號為劉石經。酒闌,芳與肅俱出。肅執芳手曰:「吾少來留意《三禮》,在南諸儒,亟共討論,皆謂此義,如吾向言。今聞往釋,頓祛平生之惑。」



 芳理義精贍,類皆如
 是。



 孝文遷洛,路由朝歌,見殷比干墓,愴然悼懷,為文以弔之。芳為注解,表上之。詔曰:「覽卿注,殊為富博。但文非屈、宋,理慚張、賈。既有雅致,便可付之集書。」詔以芳經學精洽,超遷國子祭酒。以母憂去官。



 帝徵宛、鄧,起為輔國將軍、太尉長史,從太尉、咸陽王禧攻南陽。齊將裴叔業入寇徐州,疆場之人,頗懷去就。帝憂之,以芳為散騎常侍、國子祭酒、徐州大中正,行徐州事。後兼侍中,從征馬圈。孝文崩於行宮,及宣武即位,芳手加兗冕。



 孝文襲斂,暨乎啟祖、山陵、練祭,始末喪事,皆芳撰定。咸陽王禧等奉申遺旨,令芳入授宣武經。及南徐州刺史沈陵外叛,
 徐州大水,遣芳撫慰振恤之。尋正侍中,祭酒、中正並如故。芳表曰:夫為國家者罔不崇儒尊道,學校為先。唐虞以往,典籍無據;隆周以降,任居武門。蔡氏《勸學篇》云:「周之師氏居武門左。」今之祭酒則周師氏。《洛陽記》:「國子學宮與天子宮對。太學在開陽門外。」案《學記》云:「古之王者,建國親人,教學為先。」鄭氏注:「內則設師保以教,使國子學焉;外則有太學庠序之官。」



 由斯而言,國學在內,太學在外,明矣。臣謂今既徙縣崧瀍,皇居伊洛,宮闕府寺,僉復故址,至於國學,豈宜舛錯?校量舊事,應在宮門之左。至如太學,基所見存,仍舊營構。



 又云太初太和二十年,
 發敕立四門博士,於四門置學。臣案:自周已上,學唯以二,或尚東,或尚西,或貴在國,或貴在郊。爰暨周室,學蓋有六:師氏居內,太學在國,四小在效。《禮記云:「周人養庶老於虞庠,虞庠在國之四郊。」《禮》又云:「天子設四學,當入學而太子齒。」注云:「四學,周四郊之虞庠也。」



 《大戴·保傅篇》云:「帝入東學,尚親而貴仁;帝入南學,尚齒而貴信;帝入西學,尚賢而貴德;帝入北學,尚貴而尊爵;帝入太學,承師而問道。」周之五學,於此彌彰。案鄭注《學記》,周則六學,所以然者,注云:「內則設師保以教,使國子學焉;外則有太學庠序之官。」此其證也。漢、魏已降,無復四郊。謹尋先
 旨,宜在四門。案王肅注云:「天子四郊有學,去都五十里。」考之鄭氏,不云遠近。



 今太學故坊,基址寬曠。四郊別置,相去遼闊,檢督難周。計太學坊并作四門,猶為太曠。以臣愚量,同處無嫌。且今時制置,多循中代,未審四學應從古不?求集儒禮官議其定所。



 從之。遷中書令,祭酒如故。出除青州刺史。為政儒緩,不能禁止姦盜;然廉清寡欲,無撓公私。還朝,議定律令。芳斟酌古今,為大議之主,其中損益,多芳意也。宣武以朝儀多闕,其一切諸議悉委芳脩正,於是朝廷吉凶大事,皆就諮訪焉。



 轉太常卿。



 芳以所置五郊及日月之位,去城里數於《禮》有違;又靈
 星、周公之祀,不應隸太常,乃上疏曰:臣聞國之大事,莫先郊祀;郊祀之本,實在審位。臣學謝全經,業乖通古,豈可輕薦瞽言,妄陳管說!竊見所置壇祠,遠近之宜,考之典制,或未允衷,既曰職司,請陳膚淺。



 《孟春令》云:「其數八。」又云:「迎春於東郊。」盧植云:「東郊,八里郊也。」賈逵云:「東郊,木帝太昊,八里。」許慎云:「東郊,八里郊也。」鄭玄《孟春令》注云:「王居明堂。《禮》曰:「王出十五里迎歲。」蓋殷禮也。周禮,近郊五十里。」鄭玄別注云:「東郊去都城八里。」高誘云:「迎春氣於東方,八里郊也。」王肅云:「東郊八里,因木數也。」此皆同謂春郊八里之明據也。



 《孟夏令》云:「其數七。」又云:「迎夏
 於南郊。」盧植云:「南郊,七里郊。」



 賈逵云:「南郊,火帝,七里。許慎云:「南郊,七里郊也。」鄭玄云:「南郊去都城七里。」高誘云:「南郊,七里之郊也。」王肅云:「南郊七里,因火數也。」



 此又南郊七里之審據也。《中央令》云:「其數五。」盧植云:「中郊,五里之郊也,」賈逵云:「中兆黃帝之位,并南郊之季,故云兆五帝於四郊也。」鄭玄云:「中郊,西南未地,去都城五里。」此又中郊五里之審據也。《孟秋令》云:「其數九。」又云:「以迎秋於西郊。」盧植云:「西郊,九里。」賈逵云:「西郊,金帝少昊,九里。」許慎云:「西郊,九里郊也。」鄭玄云:「西郊去都城九里。」



 高誘云:「西郊,九里之郊也。」王肅云:「西郊九里,因金數也。」此又
 西郊九里之審據也。《孟冬令》云:「其數六。」又云:「迎冬於北郊。」盧植云:「北郊,六里郊也。」賈逵云:「北郊,水帝顓頊,六里,」許慎云:「北郊,六里郊也。」鄭玄云:「北郊去都城六里。」高誘云:「北郊,六里之郊也。」王肅云:「北郊六里,因水數也。」此又北郊六里之審據也。宋氏《含文嘉》注云:「《周禮》:王畿內千里,二十分其一,以為近郊。近郊五十里,倍之為遠郊。迎王氣蓋於近郊。漢不設王畿,則以其方數為郊處。故東郊八里,南郊七里,西郊九里,北郊六里,中郊在西南未地五里。」《祭祀志》云:「建武二年正月,初制郊兆於雒陽城南七里,依採元始中故事,北郊在雒陽城北四里。」此又
 漢世南、北郊之明據也。今地祗準此。至如三十里郊,進乖鄭玄所引殷、周二代之據,退違漢、魏所行故事。凡邑外曰郊。今計四郊各以郭門為限,里數依上。



 《禮》:朝拜日月皆於東西門外。今日月之位,去城東西,路各三十,竊又未審。《禮》又云:「祭日於壇,祭月於坎。」今計造如上。《禮儀志》云:「立高禖祠于城南。」不云里數,故今用舊。靈星本非禮事,兆自漢初,專為祈田,恒隸郡縣。《郊祀志》云:「高祖五年,制詔御史,其令天下立靈星祠,牲用太牢,縣邑令、長侍祠。」晉《祠令》云:「郡、縣、國祠社稷、先農,縣又祠靈星。」此靈星在天下諸縣之明據也。周公廟所以別在洛陽者,蓋
 緣姬旦創成洛邑,故傳世洛陽,崇祠不絕,以彰厥庸。夷、齊廟者,亦世為洛陽界內神祠。今並移太常,恐乖其本。正下此類甚眾,皆當部郡縣脩理,公私施之禱請。竊惟太常所司,郊廟神祇自有常限,無宜臨時斟酌以意,若遂爾妄營,則不免淫祀。二祠在太常,在洛陽,於國一也,然貴在審本。



 臣以庸蔽,謬忝今職,考括墳籍,博採群議,既無異端,謂粗可依據。今玄冬務隙,野罄人閑,遷易郊壇,二三為便。



 詔曰:「所上乃有明據,但先朝置立已久,且可從舊。」



 先是,孝文於代都,詔中書監高閭、太常少卿陸琇并公孫崇等十餘人,脩理金石及八音之器。後崇為
 太樂令,乃上請尚書僕射高肇,更共營理。宣武詔芳共主之。



 芳表以禮樂事大,不容輒決,自非博延公卿,廣集儒彥,討論得失,研窮是非,無以垂之萬葉,為不朽之式。被報聽許,數旬之間,頻煩三議。于是朝士頗以崇專綜既久,不應乖謬,各默然無發論者。芳乃探引經誥,搜括舊文,共相難質,皆有明據,以為盈縮有差,不合典式。崇雖示相酬答,而不會問意,卒無以自通。尚書依事述奏,仍詔委芳別更考制。於是學者彌歸宗焉。芳以社稷無樹,又上疏曰:依《合朔儀》注:日有變,以朱絲為繩,以繞係社樹三匝。而今無樹。又《周禮大司徒》職云:「設其社稷之
 壝而樹之田主,各以其社所宜木。」鄭玄注云:「所宜木,謂若松、柏、栗也。」此其一證也。又《小司徒·封人》職云:「掌設王之社壝,為畿封而樹之。」鄭玄注云:「不言稷者,王主於社;稷,社之細也。」



 此其二證也。又《論語》曰:「哀公問社於宰我。宰我對曰:夏后氏以松,殷人以柏,周人以栗。」是乃土地之所宜也。此其三證也。又《白武通》:社、稷所以有樹,何也?尊而識之也。使人望見既敬之,又所以表功也。」案此正解所以有樹之義,了不論有之與無也。此其四證也。此云「社、稷所以有樹何」,然則稷亦有樹明矣。又《五經通義》云:「天子太社、王社,諸侯國社、侯社,制度奈何?曰,社皆
 有垣無屋,樹其中以木。有木者,土主生萬物,萬物莫善於木,故樹木也。」



 此其五證也,此最其丁寧備解有樹之意也。又《五經要義》云:「社必樹之以木。



 《周禮·司徒》職曰:班社而樹之,各以土地所生。《尚書·逸篇》曰:太社惟松,東社惟柏,南社惟梓,西社惟慄,北社惟槐。」此其六證也。此又太社及四方皆有樹別之明據也。又見諸家《禮圖》,社稷圖皆畫為樹,唯誡社、誡稷無樹。此其七證也。



 雖辨有樹之據,猶未正所植之木。案《論語》稱「夏后氏以松,殷人以柏,周人以慄」,便是世代不同。而《尚書·逸篇》則云「太社惟松」,如此,便以一代之中而立社各異也。愚以為宜植以
 松。何以言之?《逸書》云「太社惟松」,今者植松,不慮失禮。惟稷無成證。稷乃社之細,蓋亦不離松也。



 宣武從之。



 芳沈雅方正,概尚甚高,《經》、《傳》多通,孝文尤器敬之,動相顧訪。太子恂之在東宮,孝文欲為納芳女,芳辭以年貌非宜,帝歎其謙慎。帝更敕芳舉其宗女,芳乃稱其族子長文之女,孝文乃為恂娉之,與鄭懿女對為左右孺子焉。



 崔光於芳有中表之敬,每事詢仰。芳撰鄭玄所注《周官·儀禮音》、干寶所注《周官音》、王肅所注《尚書音》、何休所注《公羊音》、范寧所注《穀梁音》、韋昭所注《國語音》、范曄《後漢書音》各一卷,《辯類》三卷,《徐州人地錄》二十卷,《急就篇續注
 音義證》三卷,《毛詩箋音義證》十卷,《禮記義證》十卷,《周官·儀禮義證》各五卷。崔光表求以中書監讓芳,宣武不許。卒,贈鎮東將軍、徐州刺史,謚文貞侯。



 長子懌,字祖欣。雅有父風,頗好文翰。歷徐州別駕、兗州左軍府長史、司空諮議參軍,屢為行臺出使,所歷皆有當官之稱。轉通直散騎常侍、徐州大中正,行郢州事,尋遷安南將軍、大司農卿。卒,贈徐州刺史,謚曰簡。無子,弟廞以第三子峻為後。



 廞字景興,好學強立。善事當世,高肇之盛及清河王懌為宰輔,廞皆與其子姪交游。靈太后臨朝,又與太后兄子往還相好。太后令廞以詩武授弟元吉。稍遷光祿
 大夫。孝武帝初,除散騎常侍,遷驃騎大將軍、國子祭酒。孝武於顯陽殿講《孝經》,廞為執經,雖酬答論難未能精盡,而風采音制,足有可觀。尋兼都官尚書,又兼殿中尚書。及孝武入關,齊神武至洛,責廞誅之。



 子騭,字子昇。少有風氣,頗涉文史。位徐州開府從事中郎。父廞之死,騭率勒鄉部赴兗州,與刺史樊子鵠抗禦王師。每戰,流涕突陣。城陷,禽送晉陽。齊神武矜而赦之。文襄為儀同開府,以騭為屬本州大中正,轉中書舍人。時與梁和通,騭前後受敕對其使一十六人。為司徒左長史,卒,贈南青州刺史。廞弟彧,位金紫光祿大夫。彧子逖。



 逖字子長,少聰敏。好弋獵騎射,以行樂為事;愛交游,善戲謔。齊文襄以為永安公浚開府行參軍。逖遠離家鄉,倦於羈旅,發憤自勵,專精讀書。晉陽都會之所,霸朝人士攸集,咸務於宴集。逖在游宴之中,卷不離手,遇有文籍所未見者,則終日諷誦,或通夜不歸。其好學如此。亦留心文藻,頗工詩詠。



 齊天保初,行定陶縣令,坐奸事免,十餘年不得調。其姊為任氏婦,沒入宮,敕以賜魏收。收所提攜,後為開府參軍。及文宣崩,文士並作挽歌,楊遵彥擇之,員外郎盧思道用八首,逖用二首,餘人多者不過三四。中書郎李愔戲逖曰:「盧八問訊劉二。」逖銜之。乾
 明元年,兼員外散騎常侍,使送梁主蕭莊。還,兼三公郎中。



 武成時,和士開寵要,逖附之。正授中書侍郎,入典機密。時李愔獻賦,言天保中被讒。逖摘其文,奏曰:「誹謗先朝,大不敬。」武成怒,大加鞭朴。逖喜復前憾,曰:「高搥兩下,執鞭一百,何如呼劉二時。」尋兼散騎常侍,聘陳使主。



 逖欲獨擅文藻,不願與文士同行。時黃門侍郎王松年妹夫盧士游,性沈密,逖求以為副。又逖姊魏家者,收時已放出,逖因次欲嫁之士游,不許。逖恐事露,亦不逼焉。遷給事黃門侍郎,修國史。加散騎常侍,除假儀同三司,聘周使副。二國始通,禮儀未定,逖與周朝議論往復,斟酌
 古今,事多合禮,兼文辭可觀,甚行名譽。使還,拜儀同三司。



 及武成崩,和士開欲改元,議者各異。逖請為「武平」,私謂士開曰:「武平反為明輔,逖作此以為公。」士開悅而從之。時士開為眾口所排,婁定遠同輔政,逖遂回附之,使得西貨,悉以餉定遠。定遠外任,逖不自安,又陰結斛律明月、胡長仁以自固。士開知之,未甚信,忽於明月門巷逢之,彌以為實。初,逖名宦未達時,欲事祖珽。珽未原,謂人曰:「我言彭城楚子,應有氣俠,唯將崔季舒詩示人,殊乖氣望。」逖乃為弟娶珽女,遂成密好。珽之將訴趙彥深、和士開也,先與逖謀,逖乃告二人。故二人得為之計。珽
 被黜,令弟出其妻。及是,逖解士開所嫌。尋出為仁州刺史。珽乃要行臺尚書盧潛陷逖,許潛重遷。潛曰:「如此事,吾不為也。」



 更戒逖而護之。後被徵還,待詔文林館,重除散騎常侍,奏門下事。未幾與崔季舒等同戮,時年四十九。所制文筆三十卷。子逸人,開府行參軍。仕隋,終於洛陽令。



 芳懋從子懋。



 懋字仲華,祖泰之,父承伯,仕宋並有名位。懋聰敏好學,博綜經史;善草隸書,識奇字。宣武初入朝,位尚書外兵郎中。芳甚重之,凡所撰朝廷軌儀,皆與參量。尚書博議,懋與殿中郎袁翻常為議主。達於從政,臺中疑事,咸所
 訪決。尚書李平與結莫逆交。遷步兵校尉,領郎中,兼東宮中舍人。轉員外常侍、鎮遠將軍,領考功郎中,立考課之科,明黜陟之法,甚有條貫。



 孝昭初,大軍攻硤石,懋為李平行臺郎中。城拔,懋頗有功。太傅、清河王懌愛其風雅,常目而送之曰:「劉生堂堂,搢紳領袖,若天假之年,必為魏朝宰輔。」



 詔懋與諸才學之士撰成儀令。懌為宰相積年,禮懋尤重,令諸子師之。遷太尉司馬。



 熙平二年冬,暴病卒。家甚清貧,亡之日,徙四壁而已。太傅懌及當時才俊莫不痛惜之。贈持節、前將軍、南泰州刺史,謚曰宣簡。懋詩誄賦頌及文筆見稱於時,又撰諸器物造作之
 始十五卷,名曰《物祖》。



 常爽,字仕明,河內溫人,魏太常卿林六世孫也。祖珍,苻堅南安太守,因世亂,遂居涼州。父坦,乞伏世鎮遠將軍、大夏鎮將、顯美侯。



 爽少而聰敏,嚴正有志概,雖家人僮隸未嘗見其寬誕之容。篤志好學,博聞強識,明習緯候、《五經》、百家,多所研綜。州郡禮命,皆不就。武成西征涼土,爽與兄士國歸款軍門。武成嘉之,賜士國爵五品,顯美男;爽為六品,拜宣威將軍。



 是時,戎車屢駕,征伐為事,貴游子弟未遑學術。爽置館溫水之右,教授門徒七百餘人,京師學業,翕然復興。爽立訓甚有勸罰之科,弟子事之,
 若嚴君焉。尚書左僕射元贊、平原太守司馬真安、著作郎程靈虯皆是爽教所就。崔浩、高允並稱爽之嚴教,獎勵有方。允曰:「文翁柔勝,先生剛克,立教雖殊,成人一也。」其為通識歎服如此。因教授之暇,述《六經略注》,以廣制作,甚有條貫。其序曰:《傳》稱立天之道,曰陰與陽;立地之道,曰柔與剛;立人之道,曰仁與義。



 然則仁義者,人之性也;經典者,身之文也。皆以陶鑄神情,啟悟耳目,未有不由學而能成其器,不由習而能利其業。是故季路勇士也,服道以成忠烈之概;寧越庸夫也,講藝以全高尚之節。蓋所由者習也,所因者本也;本立而道生,身文而
 德備焉。



 昔者先王之訓天下也,莫不導以《詩》、《書》,教以《禮》、《樂》,移其風俗,和其人民。故恭儉莊敬而不煩者,教深於《禮》也;廣博易良而不奢者,教深於《樂》也;溫柔敦厚而不愚者,教深於《詩》也;疏通知遠而不誣者,教深於《書》也;潔靜精微而不賊者,教深於《易》也;屬辭比事而不亂者,教深於《春秋》也。夫《樂》以和神,《詩》以正言,《禮》以明體,《書》以廣聽,《春秋》以斷事。五者,蓋五常之道,相須而備。《易》為之源,故曰《易》不可見,則乾坤其幾乎息矣。由是言之,《六經》者,先王之遺烈,聖人之盛事也,安可不游心寓目習性文身哉!頃因暇日,屬意藝林,略撰所聞,討論其本,名曰《六
 經略注》,以訓門徒焉!



 其《略注》行於世。



 爽不事王侯,獨守閑靜,講肄經典二十餘年,時號為「儒林先生」。年六十三,卒於家。子文通,歷官至鎮西司馬、南天水太守、西翼校尉。文通子景。



 景字永昌,少聰敏,初讀《論語》、《毛詩》,一受便覽。及長,有才思,雅好文章。廷尉公孫良舉為協律博士,孝文親得其名,既而用之為門下錄事。正始初,招尚書、門下於金墉中書外省考論律令,敕景參議。宣武季舅護軍將軍高顯卒,其兄右僕射肇託景及尚書邢巒、并州刺史高聰、通直郎徐紇各作碑銘,並以呈御。帝悉付侍中崔光簡
 之,光奏景名位乃處諸人之下,文出諸人之上,遂以景文刊石。



 肇尚平陽公主,未幾主薨,肇欲使公主家令居廬制服,已付學官議正施行。尚書又以訪景,景以婦人無專國之理,家令不得有純臣之義,乃執議曰:喪紀之本,實稱物以立情;輕重所因,亦緣情以制禮。雖理關盛衰,事經今古,而制作之本,降殺之宜,其實一焉。是故臣之為君,所以資敬而崇重;為君母妻,所以從服而制義。然而諸侯大夫之君者,謂其有地土、有吏屬,無服文者,言其非世爵也。今王姬降適,雖加爵命,事非君邑,理異列土。何者?諸王開國,備立臣吏,生有趨奉之勤,死盡
 致喪之禮。而公主家令,唯有一人,其丞已下,命之屬官,既無接事之儀,實闕為臣之體。原夫公主之貴,所以立家令者,蓋以主之內事,脫須關外,理無自達,必也因人。然則家令唯通內外之職及典主家之事耳,無關君臣之理,名義之分也。由是推之,家令不得為純臣,公主不可為正君,明矣。



 且女人之為君,男子之為臣,古禮所不載,先朝所未議。而四門博士裴道廣、孫榮乂等以公主為之君,以家令為之臣,制服以斬,乖繆彌甚。又張虛景、吾難羈等不推君臣之分,不尋致服之情,猶同其議,準母制齊,求之名實,理未為允。竊謂公主之爵,既非食採
 之君;家令之官,又無純臣之式。若附如母,則情議罔施;若準小君,則從服無據。案如經《禮》,事無成文,即之愚見,謂不應服。



 朝廷從之。



 景淹滯門下積歲,不至顯官,以蜀司馬相如、王褒、嚴君平、揚子雲等四賢,皆有高才而無重位,乃託意以贊之。景在樞密十有餘年,為侍中崔光、盧昶、游肇、元暉尤所知賞。累遷積射將軍、給事中。延昌初,東宮建,兼太子屯騎校尉,錄事皆如故。受敕撰門下詔書凡四十卷。尚書元萇出為安西將軍、雍州刺史,請景為司馬。以景階次不及,除錄事參軍、襄威將軍,帶長安令,甚有惠政,人吏稱之。



 先是,太常劉芳與景等撰朝
 令,未及班行。別典儀注,多所草創,未成。芳卒,景纂成其事。及宣武崩,召景赴京,還脩儀注。拜謁者僕射,加寧遠將軍,又以本官兼中書舍人。後授步兵校尉,仍舍人。又敕撰太和之後朝儀已施行者,凡五十餘卷。時靈太后詔依漢世陰、鄧二后故事,親奉廟祀,與帝交獻。景乃據正以定儀注,朝廷是之。正光初,除龍驤將軍、中散大夫,舍人如故。時明帝行講學之禮於國子寺,司徒崔光執經,敕景與董紹、張徹、馮元興、王延業、鄭伯猷等俱為錄義。事畢,又行釋奠之禮,並詔百官作釋奠詩,以景作為美。



 是年九月,蠕蠕主阿那瑰歸闕,朝廷疑其位次。高陽
 王雍訪景。曰:「昔咸寧中,南單于來朝,晉世處之王公、特進之下。今日為班,宜在蕃王、儀同三司之間。」



 雍從之。朝廷典章,疑而不決,則時訪景而行。



 初,平齊之後,光祿大夫高聰徙於北京,中書監高允為之聘妻,給其資宅。聰後為允立碑,每云「吾以此文報德足矣。」豫州刺史常綽以未盡其美。景尚允才器,先為《遺德頌》,司徒崔光聞而觀之,尋味良久,乃云:「高光祿平日每矜其文,自許報允之德,今見常生此頌,高氏不得獨擅其美也。」侍中崔光、安豐王延明受詔議定服章,敕景參脩其事。尋進號冠軍將軍。阿那瑰之還國也,境上遷延,仍陳窘乏。遣尚書
 左丞元孚奉詔振恤,阿那瑰執孚過柔玄,奔于漠北。遣尚書令李崇、御史中尉兼右僕射元纂追討不及。乃令景出塞,經絺山,臨瀚海,宣敕勒眾而返。



 景經涉山水,悵然懷古,乃擬劉琨《扶風歌》十二首。進號征虜將軍。



 孝昌初,給事黃門侍郎,尋除左將軍、太府少卿,仍舍人。固辭少卿不拜,改授散騎常侍,將軍如故。徐州刺史元法僧叛入梁,梁武遣其豫章王蕭綜入據彭城。



 時安豐王延明為大都督、大行臺,率臨淮王彧等眾軍討之。既而蕭綜降附,徐州清復,遣景兼尚書,持節馳與行臺都督觀機部分。景經洛納,乃作銘焉。是時尚書令蕭寶夤、都督
 崔延伯、都督北海王顥、都督車騎將軍元恒芝等並各出討,詔景詣軍宣旨勞問。還,以本將軍授徐州刺史。b杜洛周反於燕州,仍以景兼尚書為行臺,與幽州都督、平北將軍元譚以禦之。景表求勒幽州諸縣悉入古城,山路有通賊之處,權發兵夫,隨宜置戍,以為防遏。又以頃來差兵,不盡強壯,今之三長,綿是豪門多丁為之,今求權發為兵。明帝皆從之。進號平北將軍。別敕譚西至軍都關,北從盧龍塞,據此二險,以杜賊出入之路。又詔景山中險路之處,悉令捍塞。景遣府錄事參軍裴智成發范陽三長之兵以守白閏,都督元譚據居庸下口。俄而安
 州石離、冗城、斛鹽三戍兵反,結洛周,有眾二萬餘落,自松岍赴賊。譚勒別將崔仲哲等截軍都關以待之。仲哲戰沒,洛周又自外應之,腹背受敵,譚遂大敗,諸軍夜散。詔以景所部別將李琚為都督,代譚徵下口,降景為後將軍,解州任。仍詔景為幽、安、玄四州行臺。



 賊既南出,鈔略薊城,景命統軍梁仲禮率兵士邀擊。破之,獲賊將禦夷鎮軍主孫念恒。都督李琚為賊所攻薊城之北,軍敗而死。率屬城人禦之,賊不敢逼。洛周還據上谷。授景平北將軍、光祿大夫,行臺如故。洛周遣其都督王曹紇真、馬叱斤等率眾薊南,以掠人穀,乃遇連雨,賊眾疲勞。景
 與都督於榮、制史王延年置兵慄國,邀其走路,大敗之,斬曹紇真。洛周率眾南趨范陽,景與延年及榮破之。又遣別將重破之於州西彪眼泉,禽斬之及溺死者甚眾。後洛周南圍范陽,城人翻降,執刺史延年及景,送於洛周。尋為葛榮所吞,景又入榮。榮破,景得還朝。



 永安初,詔復本官,兼黃門侍郎,又攝著作,固辭不就。二年,除中軍將軍,正黃門。先是參議《正光壬子曆》,至是賜爵高陽子。元顥內逼,莊帝北巡,景與侍中、大司馬、安豐王延明在禁中召諸親賓,乃安慰京師。顥入洛,景乃居本位。



 莊帝還宮,解黃門。普泰初,除車騎將軍、右光祿大夫、秘書監。
 以預詔命之勤,封濮陽縣子,後以例追。永熙二年,監議事。



 景自少及老,恒居事任,清儉自守,不營產業。至於衣食,取濟而已。耽好經史,愛玩文詞,若遇新異之書,殷勤求訪,或復質買,不問價之貴賤,必以得為期。



 友人刁整每謂曰:「卿清德自居,不事家業,雖儉約可尚,將何以自濟也?吾恐摯太常方餒於柏谷耳。」遂與衛將軍羊深矜其所乏,乃率刁雙、司馬彥邕、李諧、畢祖彥、結義顯等各出錢千文而為買馬焉。天平初遷鄴,是時詔下三日,戶四十萬狼狽就道,收百官馬,尚書丞、郎已下非陪從者,盡乘驢。齊神武以景清貧,特給車牛四乘,妻孥方得達
 鄴。後除儀同三司,仍本將軍。武定六年,以老疾去官,詔特給右光祿事力終其身。八年薨。



 景善與人交,終始若一。其游處者皆服其深遠之度,未曾見其矜吝之心。好飲酒,淡於榮利,自得懷抱,不事權門。性和厚恭慎。每讀書見韋弦之事、深薄之危,乃圖古昔可以鑒戒,指事為象,贊而述之曰:《周雅》云:「謂天蓋高,不敢不局;謂地蓋厚,不敢不蹐。」有朝隱大夫鑒戒斯文,乃惕焉而懼曰:夫道喪則性傾,利重則身輕。是故乘和體遜,式銘方冊;防微慎獨,載象丹青。信哉辭人之賦,文晦而理明。仰瞻高天,聽卑視諦;俯測厚地,岳峻川渟。誰共戴之,不私不畏;誰
 其踐之,不陷不墜。故善惡是征,物罔同異。論亢匪久,人咸敬忌。嗟乎!唯地厚矣,尚亦兢兢。。浩浩名位,孰識其親。



 搏之弗得,聆之無聞。故有戒於顯而急於微。好爵是冒,聲奢是基。身陷於祿利,言溺於是非。或求欲而未厭,或知足而不辭。是故位高而勢逾迫,正立而邪逾欺。



 安有位朽而危不萃,邪榮而正不彫。故悔多於地厚,禍甚於天高。夫悔未結,誰肯曲躬。夫禍未加,誰肯累足。固機發而後思圖,車覆而後改躅。改之無及,故狡兔失穴;思之在後,故逆鱗易觸。君子則不然。體舒則懷卷,視溺則思濟。原夫人闕之度,邈於無階之天,勢位之危,深於不測
 之地。餌厚而躬不競,爵降而心不係。



 守善於已成,懼愆於未敗。雖盈而戒沖,通而慮滯。以知命為遐齡,以樂天為大惠。



 以戢智而從時,以懷愚而游世。曲躬焉,累足焉,茍行之晝已決矣,猶夜則思其計;誦之口亦明矣,故心必賞其契。故能不同不誘,而弭謗於群小;無毀無譽,而貽信於上帝。託身與金石俱固,立名與天壤相弊。囂競無侵,優游獨逝。夫如是,綺閣金門,可安其宅;錦衣玉食,可頤其形。柳下三黜,不慍其色;子文三陟,不喜其情。



 而惑者見居高可以持勢,欲乘高以據榮。見直道可以脩己,欲專道以邀聲。夫去聲然後聲可立,豈矜道之所宣。
 慮危然後安可固,豈假道之所全。是以君子鑒恃道不可以流聲,故去聲而懷道。鑒專道不可以守勢,故去勢以崇道。何者?履道雖高,不得無亢;求聲雖道,不得無悔。然則聲奢繁則實儉彫,功業進則身迹退。如此則精靈遂越,驕侈自親。情與道絕,事與勢鄰。方欲役思以持勢,乘勢以求津。



 故利慾誘其性,禍難嬰其身。利慾交則幽顯以之變,禍難構則智術無所陳。若然者,雖縻爵帝局,焉得而寧之?雖結珮皇庭,焉得而榮之?故身道未究,而崇邪之徑已形。成功未立,而脩正之術已生。福祿交蹇於人事,屯難頓萃於時情。忠介剖心於白日,耿節沉骨
 於幽靈。因斯愚智之所機,倚伏之所系,全亡之所依,其在遜順而已哉。嗚呼鑒之!嗚呼鑒之!



 景所著述數百篇見行於世。刪正晉司空張華《博物志》及撰《儒林》、《列女傳》各數十篇云。長子昶,少學識,有文才,早卒。昶弟彪之,永安中,司空行參軍。



 論曰:古人云:才未半古,功已過之。王肅流寓之士,見知一面,榮任赫然,寄同舊列,雖器業自致,抑亦逢時之所致焉。劉芳矯然特立,沈深好古,博通洽識,為世儒宗。懋才流識學,見重於世,不虛然也。常爽以儒素著稱,景以文義見宗,美乎。



\end{pinyinscope}