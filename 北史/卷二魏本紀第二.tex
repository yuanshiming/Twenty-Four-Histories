\article{卷二魏本紀第二}

\begin{pinyinscope}

 世祖太武皇帝諱燾,明元皇帝之長子也。母曰杜貴嬪。天賜五年,生於東宮。



 體貌瑰異,道武奇之,曰:「成吾業者必此兒也。」泰常七年四月,封太平王。五月,立為皇太子。及明元帝疾,命帝總攝百揆。帝聰明大度,意豁如也。



 八年十
 一月己
 巳,明元帝崩,壬申,太子即皇帝位,大赦天下。十二月,追尊皇妣為密皇太后。進司徒長孫嵩爵為北平王;司空奚斤為宜城王;藍田公長孫翰為平陽王。
 其餘普增爵位各有差。於是除禁錮,釋嫌疑,開倉庫,振窮乏。河南流人相率內屬者甚眾。



 始光元年春正月丙寅,安定王彌薨。夏四月甲辰,東巡,幸大寧。六月,宋徐羨之弒其主義符。秋七月,車駕還宮。八月,蠕蠕六萬騎入雲中,殺略人吏,攻陷盛樂。帝帥輕騎討之,虜乃退走。九月,大簡輿徒於東郊,將北討。冬十二月,遣平陽王長孫翰等討蠕蠕,車騎次祚山,蠕蠕北遁,諸軍追之,大獲而還。



 二年春正月己卯,車駕至自北伐。三月丙辰,尊保母竇氏曰保太后。丁巳,以北平王長孫嵩為太尉,平陽王長
 孫翰為司徒,宜城王奚斤為司空。庚申,營故東宮為萬壽宮,起永安、安樂二殿、臨望觀、九華堂。初造新字千餘。夏四月詔龍驤將軍步堆使宋。五月,詔天下十家發大牛一頭運粟塞上。秋八月,赫連屈丐死。九月,永安、安樂二殿成,丁卯,大饗以落之。冬十月癸卯,車駕北伐,東西五道並出。



 平陽王長孫翰等絕漠追寇,蠕蠕北走。



 三年春正月壬申,車駕至自北伐。乞伏熾盤遣使朝貢,請討赫連昌。二月,起太學於城東,祀孔子,以顏回配。夏五月辛卯,進中山公纂爵為王,復南安公素先爵常山王。六月,幸雲中舊宮,謁陵廟,西至五原,田於陰山,東至
 和兜山。秋七月,築馬射臺於長川,帝親登臺走馬。王公諸國君長馳射中者,賜金錦繒絮各有差。



 八月,車駕還宮。宋人來聘。帝以赫連屈丐死,諸子相攻,冬十月丁巳,車駕西伐,幸雲中,臨君子津。會天暴寒,數日冰合。十一月戊寅,率輕騎襲赫連昌。壬午,徙萬餘家而還。至祚山,班虜獲以賜將士各有差。十二月,詔奚斤西據長安。秦、隴氐羌皆叛昌詣斤降。武都王楊玄及沮渠蒙遜等使使內附。



 四年春正月乙酉,車駕至自西伐,賜留臺文武各有差。從人在道多死,到者裁十六七。己亥,行幸幽州。赫連昌
 遣其弟定向長安。帝聞之,遣就陰山伐木造攻具。



 二月,車駕還宮。三月丙午,詔執金吾桓貸造橋於君子津。丁丑,廣平王連薨。夏四月丁未,詔員外散騎常侍步堆使於宋。五月,車駕西討赫連昌,次拔鄰山。築城舍輜重,以輕騎三萬先行。戊戌,至黑水。帝親祈天,告祖宗之靈而誓眾。六月癸卯朔,日有蝕之。甲辰,大破赫連昌,昌奔上邽。乙巳,車駕入城,虜昌群弟及其母妹妻妾宮人萬數,府庫珍寶車旗器物不可勝計。辛酉,班師。留常山王素、執金吾桓貸鎮統萬。秋七月己卯,築壇於祚嶺,戲馬馳射,賜中者金帛繒絮各有差。蠕蠕寇雲中,聞破赫連昌,
 懼而逃。八月壬子,車駕至自西伐,飲至策勛,告宗廟,班軍實以賜留臺百僚各有差。冬十一月,以氐王楊玄為假征南大將軍、都督、梁州刺史、南秦王。十二月,行幸中山,守宰貪污免者十數人。癸卯,車駕還宮,復所過田租之半。



 神蒨元年春正月,以天下守令多非法,精選忠良悉代之。辛未,京兆王黎薨。



 二月,改元。司空奚斤進軍安定。監軍侍御史安頡出戰,禽昌。其餘眾立昌弟定為主,走還平涼。三月辛巳,侍中古弼送赫連昌至于京師。司空奚斤追赫連定於平涼馬髦嶺,為定所禽。將軍丘堆先在
 安定,聞斤敗,東走長安。帝大怒,詔頡令斬之。



 夏四月,赫連定遣使朝貢。壬子,西巡。戊午,田於河西,大赦。南秦王楊玄遣使朝貢。五月,乞伏熾盤死。秋八月,東幸廣寧,臨觀溫泉。以太牢祭黃帝、堯、舜廟。九月,車駕還宮。冬十一月乙未朔,日有蝕之。是月,行幸河西,大校獵。十二月甲申,車駕還宮。



 二年夏四月,宋人來聘。庚寅,車駕北伐。五月丁未,次於沙漠,舍輜重,輕騎兼冀馬至慄水。蠕蠕震怖,焚廬舍,絕跡西走。冬十月,振旅凱旋于京師,告于宗廟。列置新人於漠南,東至濡源,西暨五原、陰山,竟三千里。十一月,西
 巡,田于河西,至祚山而還。



 三年春正月庚子,車駕還宮。壬寅,大赦。癸卯,行幸廣寧,臨溫泉,作《溫泉歌》。二月丁卯,司徒、平陽王長孫翰薨。戊辰,車駕還宮。三月壬寅,進會稽公赫連昌為秦王。夏四月甲子,行幸雲中。敕勒萬餘落叛走,詔尚書封鐵追滅之。



 五月戊午,論討敕勒功,大明賞罰。秋七月己亥,詔諸征鎮將軍、王公杖節邊遠者,聽開府辟召,其次增置吏員。庚子,詔大鴻臚卿杜超假節都督冀、定、相三州諸軍事、行征南大將軍、太宰,進爵為王,鎮鄴,為諸軍節度。八月,宋將到彥之自清水入河,溯流西行。丙寅,彥之遣將
 度河攻治阪,冠軍將軍安頡督諸軍擊破之。九月癸卯,立密皇太后廟于鄴。甲辰,行幸統萬,遂征平涼。是月,馮跋死。冬十月乙卯,冠軍將軍安頡濟河攻洛陽,丙子,拔之。辛巳,安頡平武牢。十一月乙酉,車駕至平涼。己亥,行幸安定。庚子,帝自安西還臨平涼,遂掘塹圍守之。行幸紐城,安慰初附,赦秦、隴之人,賜復七年。辛丑,安頡帥諸軍攻滑臺。沮渠蒙遜遣使朝貢。壬寅,封壽光侯叔孫建為丹楊王。十二月丁卯,赫連定弟社於度洛孤面縛出降,平涼,收其珍寶。定長安、臨晉、武功守將皆奔走,關中平。壬申,車駕還東,留巴東公延普等鎮安定。



 四年春正月壬午,車駕次木根山,大饗群臣。丙申,宋將檀道濟、王仲德從清水救滑臺。丹楊王叔孫建、汝陰公長孫道生拒之,道濟等不敢進。是月,赫連定滅乞伏慕末。二月辛酉,安頡、司馬楚之平滑臺。癸酉,車駕還宮,飲至策勛,告于宗廟,賜留臺百官各有差。戰士賜復十年。定州人飢,詔開倉以振之。宋將檀道濟、王仲德東走。三月庚戌,冠軍將軍安頡獻宋俘萬餘人,甲兵三萬。夏六月,赫連定北襲沮渠蒙遜,為吐谷渾慕璝所執。閏月乙未,蠕蠕國遣使朝貢。詔散騎侍郎周紹使于宋。秋七月己酉,行幸河西。起承華宮。八月乙酉,沮渠蒙遜遣子安
 周入侍。



 吐谷渾慕璝遣使奉表,請送赫連定。己丑,以慕璝為大將軍,封西秦王。九月癸丑,車駕還宮。庚申,加太尉長孫嵩柱國大將軍,以左光祿大夫崔浩為司徒,征西大將軍長孫道生為司空。癸亥,詔兼太常李順持節拜河西王沮渠蒙遜為假節、加侍中、都督涼州持節及西域羌戎諸軍事、行征西大將軍、太傅、涼州牧、涼王。壬申,詔曰:「范陽盧玄、博陵崔綽、趙郡李靈、河間邢穎、勃海高允、廣平游雅、太原張偉等皆賢俊之胄,冠冕州邦,有羽儀之用。《易》曰:『我有好爵,吾與爾縻之。』如玄之比,隱跡衡門,不曜名譽者,盡敕州郡以禮發遣。」遂徵玄等。州郡
 所遣至者數百人,皆差次敘用。冬十月戊寅,詔司徒崔浩改定律令。行幸漠南。十一月丙辰,北部敕勒莫弗庫若于率其部數萬騎驅鹿獸數百萬詣行在所。帝因而大狩,以賜從者,勒石漠南,以記功德。宜城王奚斤坐事降爵為公。十二月,車駕還宮。



 延和元年春正月丙午,尊保太后為皇太后,立皇后赫連氏,以皇子晃為皇太子,謁于太廟,大赦改元。三月丁未,追贈夫人賀氏為皇后。壬申,西秦王吐谷渾慕璝送赫連定於京師。夏五月,宋人來聘。六月庚寅,車駕伐和龍。詔尚書左僕射安原等屯于漠南,以備蠕蠕。辛卯,詔
 兼散騎常侍鄧穎使於宋。秋七月己巳,車駕至和龍,穿塹以守之。是月,築東宮。九月乙卯,車駕西還。徙營丘、成周、遼東、樂浪、帶方、玄菟六郡人三萬家于幽州,開倉以振之。冬十月,吐谷渾慕璝遣使朝貢。



 十一月己巳,車駕至自和龍。十二月己丑,馮弘子長樂公崇及其母弟朗、朗弟邈以遼西內屬。先是,辟召賢良而州郡多逼遣之,詔以禮申喻,任其進退。



 二年春二月庚午,詔兼鴻臚卿李繼持節假馮崇車騎大將軍、遼西王,承制,聽置尚書已下。壬午,詔兼散騎常侍宋宣使於宋。夏四月,沮渠蒙遜死,以其子牧犍為車
 騎將軍,改封西河王。六月,遣永昌王健、尚書左僕射安原督諸軍討和龍。辛巳,詔樂安王範發秦、雍兵一萬築小城於長安城內。秋八月,遼西王馮崇上表求說降其父,帝不聽。九月,宋人來聘,并獻馴象一。戊午,詔兼大鴻臚卿崔賾持節拜征虜將軍楊難當為征南大將軍、儀同三司,封南秦王。冬十二月己巳,大赦天下。



 辛未,幸陰山北。詔兼散騎常侍盧玄使於宋。



 三年春正月乙未,車駕次于女水,大饗群臣。戊戌,馮弘遣使求和,帝不許。



 丙辰,南秦王楊難當克漢中,送雍州流人七千家于長安。二月戊寅,詔以頻年屢征,有事西
 北,運輸之役,百姓勤勞,令郡縣括貧富以為三級,富者租賦如常,中者復二年,下窮者復三年。辛卯,車駕還宮。三月甲寅,行幸河西。閏月甲戌,秦王赫連昌叛走。丙子,河西候將格殺之。驗其謀反,群弟皆伏誅。己卯,車駕還宮。進彭城公粟爵為王。秋七月辛巳,東宮成,備置屯衛,三分西宮之一,壬午,行幸美稷,遂至隰城。命諸軍討山胡白龍于河西。九月戊子,剋之。斬白龍及其將帥,屠其城。冬十一月,車駕還宮。十二月甲辰,行幸雲中。



 太延元年春正月乙未朔,日有蝕之。壬午,降死罪刑已下各一等。癸未,出道武、明元宮人,令得嫁。甲申,大赦改
 元。二月庚子,蠕蠕、焉耆、車師各遣使朝貢。詔長安及平涼人徙在京師其孤老不能自存者,聽還鄉里。丁未,車駕還宮。夏五月庚申,進宜都公穆壽為宜都王,汝陰公長孫道生為上黨王,宜城公奚斤為恆農王,廣陵公婁伏連為廣陵王。遣使者二十輩使西域。甲戌,行幸雲中。六月甲午,詔曰:「去春小旱,東作不茂,憂勤剋己,祈請靈祐。豈朕精誠有感,何報應之速。



 雲雨震灑,流澤霑渥。有鄙婦人持方寸玉印詣潞縣侯孫家,既而亡去,莫知所在。



 印有三字,為龍鳥之形,要妙奇巧,不類人迹,文曰『旱疫平』。推尋其理,蓋神靈之報應也。比者以來,禎瑞仍臻,
 甘露流液,降於殿內;嘉瓜合蒂,生於中山;野木連理,殖於魏郡;在先后載誕之鄉,白燕集于盛樂舊都,玄鳥隨之,蓋有千數;嘉禾頻歲合秀於恆農;白兔並見於勃海,白雉三隻又集於平陽太祖之廟。天降嘉貺,將何德以酬之?其令天下大酺五日,禮報百神,守宰祭界內名山大川,上答天意。」



 丙午,高麗、鄯善國並遣使朝貢。秋七月,田於棝陽。己卯,樂平王丕等五將東伐,至和龍,徙男女六千口而還。八月丙戌,行幸河西。粟特國遣使朝貢。九月,車駕還宮。冬十月癸卯,尚書左僕射安原謀反,伏誅。甲辰,行幸定州,次於新城宮。



 十一月己巳,校獵於廣川。
 丙子,行幸鄴,祀密太后廟。諸所過親問高年,褒禮賢俊。十二月癸卯,遣使者以太牢祀北岳。



 二年春正月甲寅,車駕還宮。二月戊子,馮弘遣使朝貢,求送侍子,帝不許。



 壬辰,遣使者十餘輩詣高麗、東夷諸國,詔喻之。三月丙辰,宋人來聘。辛未,遣平東將軍娥清、安西將軍古弼討馮弘。弘求救於高麗,高麗遣其大將葛蔓盧迎之。



 夏四月甲寅,皇子小兒、苗兒並薨。五月乙卯,馮弘奔高麗。戊午,詔散騎常侍封撥使高麗,徵送馮弘。丁卯,行幸河西。赫連定之西也,楊難當竊據上邽,秋七月庚戌,命樂平王丕等討之。詔散騎常侍游雅使於
 宋。八月丁亥,遣使六輩使西域。



 帝校獵于河西,詔廣平公張黎發定州七郡一萬二千人通莎泉道。甲辰,高車國遣使朝貢。九月庚戌,樂平王丕等至,略陽公難當奉詔攝上邽守。高麗不送馮弘,帝將伐之,納樂平王丕計而止。冬十一月己酉,幸棝陽。驅野馬於雲中,置野馬苑。閏月壬子,車駕還宮。乙丑,改封潁川王提為武昌王。河西王沮渠牧犍遣使朝貢。是歲,吐谷渾慕璝死。



 三年春正月癸未,中山王纂薨。戊子,太尉、北平王長孫嵩薨。乙巳,丹楊王叔孫建薨。二月乙卯,行幸幽州,存恤孤老,問人疾苦。還幸上谷,遂至代,所過復田租之半。三
 月己卯,車駕還宮。丁酉,宋人來聘。夏五月己丑,詔天下吏人得舉告守令不如法者。丙申,行幸雲中。秋七月戊子,使永昌王健、上黨王長孫道生討山胡白龍餘黨於西河,滅之。八月甲辰,行幸河西。九月甲申,車駕還宮。丁酉,遣使者拜西秦王慕璝弟慕利延為鎮西大將軍、儀同三司,改封西平王。冬十月癸卯,行幸雲中。十一月壬申,車駕還宮。是歲,河西王沮渠牧犍世子封壇來朝,高麗、契丹、龜茲、悅般、焉耆、車師、粟特、疏勒、烏孫、渴盤陁、鄯善、破洛那、者舌等國各遣使朝貢。



 四年春三月庚辰,鄯善王弟素延耆來朝。癸未,罷沙門
 年五十以下。江陽王根薨。是月,高麗殺馮弘。夏五月戊寅,赦。秋七月壬申,車駕北伐。冬十一月丁卯朔,日有蝕之。十二月,車駕至自北伐。上洛巴、泉蕇等相帥內附。詔兼散騎常侍高雅使于宋。



 五年春正月庚寅,行幸定州。三月辛未,車駕還宮。庚寅,以故南秦王世子楊保宗為征南大將軍、秦州牧、武都王,鎮上邽。夏五月癸未,遮逸國獻汗血馬。六月甲辰,車駕西討沮渠牧犍。侍中、宜都王穆壽輔皇太子決留臺事,大將軍長樂王嵇敬、輔國大將軍建寧王崇二萬人屯漠南,以備蠕蠕。秋七月己巳,車駕至上都屬國城,大
 饗群臣,講武馬射。壬午,留輜重,分部諸軍。八月丙申,車駕至姑臧,牧犍兄子祖踰城來降。乃分軍圍之。九月丙戌,牧犍與左右文武五千人面縛軍門,帝解其縛,待以籓臣之禮。收其城內戶口二十餘萬,倉庫珍寶不可稱計。進張掖公禿髮保周爵為王,與龍驤將軍穆羆、安遠將軍源賀分略諸郡。牧犍弟張掖太守宜得西奔酒泉太守無諱,後奔晉昌;樂都太守安周南奔吐谷渾。戊子,蠕蠕犯塞,遂至七介山,京都大駭。皇太子命上黨王長孫道生等拒之。冬十月辛酉,車駕還宮。徙涼州人三萬餘家於京師。留樂平王丕、征西將軍賀多羅鎮涼州。癸
 亥,遣張掖王禿髮保周喻諸部鮮卑,保周因率諸部叛於張掖。十一月乙巳,宋人來聘,並獻馴象一。



 十二月壬午,車駕至自西伐,飲至策勳,告于宗廟。楊難當寇上邽,鎮將元勿頭討走之。是歲,鄯善、龜茲、疏勒、焉耆、高麗、粟特、渴盤陀、破洛那、悉居半等國並遣使朝貢。



 太平真君元年春正月己酉,沮渠無諱國酒泉。辛亥,分遣侍臣巡行州郡,觀察風俗,問人疾苦。二月己巳,詔假通直常侍邢穎使於宋。發長安人五千浚昆明池。



 三月,酒泉陷。夏四月戊午朔,日有蝕之。庚辰,沮渠無諱寇張掖。禿髮保周屯刪丹。六月丁丑,皇孫濬生,大赦改元。秋
 七月,行幸陰山。己丑,永昌王健大破禿髮保周,走之。丙申,保太后竇氏崩于行宮。癸丑,保周自殺,傳首京師。八月甲申,沮渠無諱降。九月壬寅,車駕還宮。是歲,州鎮十五饑,詔開倉振恤之。以河南主曜子羯兒為河間王,後改封略陽王。



 二年春正月癸卯,拜沮渠無諱為征西大將軍、涼州牧、酒泉王。三月辛卯,葬惠太后於崞山。庚戌,新興王俊、略陽王羯兒有罪,黜為公。辛亥,封蠕蠕郁久閭乞歸為朔方王,沮渠萬年為張掖王。夏四月丁巳,宋人來聘。秋八月辛亥,詔散騎侍郎張偉使于宋。九月戊戌,永昌王健
 薨。冬十一月庚子,鎮南大將軍奚眷平酒泉。



 十二月丙子,宋人來聘。



 三年春正月甲申,帝至道壇,親受符籙,備法駕,旗幟盡青。三月壬寅,北平王長孫頹有罪,削爵為侯。夏四月,酒泉王沮渠無諱走渡流沙,據鄯善。涼武昭王孫李寶據敦煌,遣使內附。五月,行幸陰山北。六月丙戌,楊難當朝於行宮。先是,起殿於陰山北,殿成而難當至,因曰廣德焉。秋八月甲戌晦,日有蝕之。冬十月己卯,封皇子伏羅為晉王,翰為秦王,譚為燕王,建為楚王,餘為吳王。十二月辛巳,太保、襄城公盧魯元薨。丁酉,車駕還宮,李寶遣
 使朝貢,以寶為鎮西大將軍、開府儀同三司、沙州牧、敦煌公。



 四年春正月庚午,行幸中山。二月丙子,次于恆山之陽,詔有司刊石勒銘。是月,剋仇池。三月庚申,車駕還宮。夏四月,武都王楊保宗謀反,諸將禽送京師。



 氐、羌復推保宗弟文德為主,圍仇池。六月庚寅,詔復人貲賦三年,其田租歲輸如常,牧守不得妄有徵發。癸巳,大閱于西郊。九月辛丑,行幸漠南。甲辰,捨輜重,以輕騎襲蠕蠕,分軍為四道。冬十一月甲子,車駕還至朔方。詔曰:「夫陰陽有往復,四時有代謝,授子任賢,蓋古今不易之令典也。其
 令皇太子副理萬機,總統百揆。諸功臣勤勞日久,皆當以爵歸第,隨時朝請,饗宴朕前,論道陳謨而己,不宜復煩以劇職。更舉賢俊,以備百官,明為科制,以稱朕心。」十二月辛卯,車駕至自北伐。



 五年春正月壬寅,皇太子始總百揆。侍中中書監宜都王穆壽、司徒東都公崔浩、侍中廣平公張黎、侍中建興公古弼輔太子以決庶政。諸上書者皆稱臣,上疏儀與表同。戊申,詔自王公已下至於庶人,私養沙門、巫及金銀工巧之人在其家者,皆遣詣官曹,限今年二月十五日。過期不出,巫、沙門身死,主人門誅。庚戌,詔自三公已
 下至於卿士,其子息皆詣太學,其百工伎巧騶卒子息當習其父兄所業,不聽私立學校,違者師身死,主人門誅。二月辛未,中山王辰等八人以北伐後期,斬于都南。癸酉,樂平王丕薨。庚辰,行幸廬。三月戊辰,大會于那南。遣使者四輩使西域。甲辰,車駕還宮。夏四月乙亥,太宰、陽平王杜超為帳下所殺。五月丁酉,行幸陰山北。六月,西平王吐谷渾慕利延殺其兄子緯代,立緯弟,叱力延等來奔,乞師。以叱力延為歸義王。秋八月乙丑,田于河西。壬午,詔員外散騎常侍高濟使於宋。九月,帝自河西至于馬邑,觀于崞川。己亥,車駕還宮。丁未,行幸漠南。冬
 十月癸未,晉王伏羅大破慕利延。慕利延走奔白蘭,其部一萬三千內附。十一月,宋人來聘。十二月丙戌,車駕還宮。



 六年春正月辛亥,行幸定州,引見長老,存問之。詔兼員外散騎常侍宋愔使于宋。二月,遂西幸上黨,觀連理樹於玄氏。至吐京,討徙叛胡,出配郡縣。三月庚申,車駕還宮。詔諸有疑獄皆付中書,以經義量決。夏六月戊子朔,日有蝕之。壬辰,北巡。秋八月壬辰,散騎常侍成周公萬度歸以輕騎至鄯善,執其王真達,與詣京師。帝大悅,厚待之。車駕幸陰山北,次于廣德宮。詔發天下兵,三取一,各
 當戒嚴,以須後命。徙諸種雜人五千餘家於北邊。令人北徙畜牧至廣漠,以餌蠕蠕。



 壬寅,征西大將軍、高涼王那等討吐谷渾慕利延。軍到蔓頭城,慕利延驅其部落西度流沙,那急追,故西秦王莫璝世子被囊逆軍拒戰,那擊破之。中山公杜豐追度三危,至雪山,禽被囊及慕利延兄子什歸、熾盤子成龍,送於京師。慕利延遂西入于闐國。九月,盧水胡蓋吳聚眾反于杏城。冬十一月,高涼王那振旅還京師。庚申,遼東王竇漏頭薨。河東蜀薛永宗聚黨入汾曲。西通蓋吳,受其位號。蓋吳自號天台王,署百官。辛未,車駕還宮。選六州兵勇猛者,使永昌
 王仁、高涼王那分領為二道,南略淮、泗以北。徙青、徐之人以實河北。癸未,西巡。



 七年春正月戊辰,車駕次東雍,禽薛永宗,斬之。其男女無少長皆赴水死。辛未,南幸汾陰。蓋吳退走北地。二月丙戌,幸長安,存問父老。丁亥,幸昆明池,遂田于岐山之陽。所過誅與蓋吳通謀反害守將者。三月,詔諸州坑沙門,毀諸佛像,徙長安城內工巧二千家於京師。夏四月甲申,車駕至自長安。戊子,毀鄴城五層佛圖,於泥像中得玉璽二,其文皆曰:「受命於天,既壽永昌」。其一刻其旁曰「魏所受漢傳國璽」。五月,蓋吳復聚杏城,自號秦地王。
 丙戌,發司、幽、定、冀四州十萬人築畿上塞圍,起上谷,西至于河,廣袤皆千里。六月癸未朔,日有蝕之。



 秋八月,蓋吳為其下人所殺,傳首京師。復略陽公羯兒王爵。



 八年春正月癸未,行幸中山。三月,河西王沮渠牧犍謀反,伏誅。夏五月,車駕還宮。六月,西征諸將扶風公處真等八將坐盜沒軍資,所在虜掠,贓各千萬計,並斬之。秋八月,樂安王範薨。冬十一月,侍中、中書監、宜都王穆壽薨。十二月,晉王伏羅薨。



 九年春正月,宋人來聘。二月癸卯,行幸定州。山東人饑,詔開倉振之。罷塞圍作。遂西幸上黨。詔於壺關東北大
 王山累石為三封,又斬其鳳凰山南足以斷之。



 三月,車駕還宮。夏五月甲戌,以交趾公韓拔為假征西將軍、領護西戎校尉、鄯善王,鎮鄯善,賦役其人,比之郡縣。六月辛酉,行幸廣德宮。丁卯,悅般國遣使求與王師俱討蠕蠕。帝許之。秋八月,詔中外諸軍戒嚴。九月乙酉,練兵於西郊。丙戌,幸陰山。是月,成周公萬度歸千里驛上:大破焉耆國,其王鳩尸卑那奔龜茲。



 冬十月辛丑,恆農王奚斤薨。癸卯,以婚姻奢靡,喪葬過度,詔有司更為科限。癸亥,大赦。十二月,詔成周公萬度歸自焉耆西討龜茲。皇太子朝于行宮。遂從北討。



 至受降城,不見蠕蠕,因積糧
 城內,留守而還。北平王長孫敦坐事降爵為公。



 十年春正月戊辰朔,帝在漠南,大饗百寮。甲戌,蠕蠕吐賀真懼,遠遁。三月,蒐于河西。庚寅,車駕還宮。夏四月丙申朔,日有蝕之。九月,閱武於磧上,遂北伐。冬十月庚子,皇太子及群官奉迎於行宮。十二月戊申,車駕至自北伐。己酉,以平昌公託真為中山王。



 十一年春正月乙丑,行幸洛陽。所過郡國,皆親對高年,存恤孤寡。二月甲午,大蒐於梁山。皇子真薨。是月,大修宮室,皇太子居于北宮。車駕遂征懸瓠。夏四月癸卯,車駕還宮,賜從者及留臺郎吏已上生口各有差。六月己
 亥,誅司徒崔浩。



 辛丑,北巡陰山。秋七月,宋將王玄謨攻滑臺。八月癸亥,田於河田。癸未,練兵於西郊。九月辛卯,車駕南伐。癸巳,皇太子北伐,屯於漠南。吳王餘留守京都。



 庚子,曲赦定、冀、相三州死罪已下。冬十月乙丑,車駕濟河,玄謨棄軍而走,乃命諸將分道並進。車駕自中道。十一月辛卯,至鄒山。使使者以太牢祀孔子。是月,頞盾國獻師子一。十二月丁卯,車駕至淮。詔刈雚葦作筏數萬而濟,淮南皆降。癸未,車駕臨江,起行宮於瓜步山。諸軍同日皆臨江,所過城邑,莫不望塵奔潰,其降附者不可勝數。甲申,宋文帝使獻百牢,貢其方物,又請進女於
 皇孫,以求和好。



 帝以師婚非禮,許和而不許婚,使散騎侍即夏侯野報之。帝詔皇孫為書,致馬通問焉。



 正平元年春正月丙戌朔,大會郡臣於江上,文武受爵者二百餘人。丁亥,車駕北旋。二月癸未,次於魯口。皇太子朝於行宮。三月己亥,車駕至自南伐,飲至策勛,告於宗廟,以降人五萬餘家分置近畿,賜留臺文武所獲軍資生口各有差。夏五月壬寅,大赦。六月壬戌,改元。車師國王遣子入侍。詔以刑綱太密,犯者更眾,命有司其案律令,務求厥中,自餘有不便於人者,依比增損。詔太子少傅游雅、中書侍郎胡方回等改定律制。略陽王羯兒、
 高涼王那有罪賜死。戊辰,皇太子薨。壬申,葬景穆太子於金陵。秋七月丁亥,行幸陰山。省諸曹吏員三分之一。九月癸巳,車駕還宮。冬十月庚申,行幸陰山。宋人來聘。詔殿中將軍郎法祐使於宋。己巳,司空、上黨王長孫道生薨。十二月丁丑,車駕還宮。封皇孫濬為高陽王,尋以皇孫世嫡,不宜在籓,乃止。改封秦王翰為東平王,燕王譚為臨淮王,楚王建為廣陽王,吳王餘為南安王。



 二年春正月庚辰朔,南來降人五千餘家於中山謀叛,州軍討平之。冀州刺史、張掖王沮渠萬年與降人通謀,賜死。三月甲寅,中常侍宗愛構逆,帝崩於永安宮,時年
 四十五。秘不發喪。愛又矯皇后令,殺東平王翰,迎周安王餘立。大赦,改元為永平。尊謚曰太武皇帝,葬於雲中金陵,廟號世祖。帝生不逮密太后,及有所識,言則悲慟,哀感傍人,明元聞而嘉歎。及明元不豫,衣不釋帶。性清儉率素,服御飲膳,取給而已,不好珍麗,食不二味。所幸昭儀、貴人,衣無兼彩。群臣白帝,更峻京邑城隍以從《周易》設險之義,又陳蕭何壯麗之說。帝曰:「古人有言,在德不在險。屈丐蒸土築城,而朕滅之,豈在城也?今天下未平,方須人力,土功之事,朕所未為。蕭何之對,非雅言也。」每以財者軍國之本,無所輕費。至於賞賜,皆是勳績之
 家,親戚愛寵,未嘗橫有所及。臨敵,常與士卒同在矢石間。左右死傷者相繼,而帝神色自若。是以人思效命,所向無前。命將出師,指授節度,從命者無不制勝,違爽者率多敗失。性又知人。拔士於卒伍之中,唯其才效所長。不論本末,兼甚嚴斷,明於刑賞。功者賞不遺賤,罪者刑不避親,雖寵愛之,終不虧法。



 常曰:「法者,朕與天下共之,何敢輕也。」故大臣犯法,無所寬假。雅長聽察,瞬息之間,下無以措其姦隱。然果於誅戮,後多悔之。司徒崔浩死後,帝北伐,時宣城公李孝伯疾篤,傳者以為卒,帝聞而悼之,謂左右曰:「李宣城可惜。」又曰:「朕向失言,崔司徒可
 惜,李宣城可哀。」褒貶雅意,皆此類也。



 景穆皇帝諱晃,太武皇帝之長子也。母曰賀夫人。延和元年正月丙午,立為皇太子,時年五歲。明慧強識,聞則不忘。及長,好讀經史,皆通大義。太武甚奇之。



 及西征涼州,皇太子監國。初,太武之伐河西,李順等咸言姑臧無水草,不可行師。



 太子有疑色。及車駕至姑臧,乃詔太子曰:「姑臧城東西門外涌泉,合於城北,其大如河,澤草茂盛,可供大軍數年。人之多言,亦可惡也。」太子謂宮臣曰:「為人臣不實若此,豈是忠乎!吾初聞有疑,但帝決行耳。幾誤人大事,言者復何面目見帝也。」



 真君四年,從征蠕
 蠕,至鹿渾谷,與賊遇。虜惶怖擾亂。太子言於太武曰:「宜速進擊,掩其不備。」尚書令劉潔固諫,以為塵盛賊多,須軍大集。太子曰:「此由賊恇擾,何有營上而有此塵?」太武疑之,遂不急擊,蠕蠕遠遁。既而獲虜候騎,乃云不覺官軍卒至,上下惶懼。北走經六七日,知無追者,乃徐行。帝深恨之。自是太子所言軍國大事,多見納用,遂知萬機。及監國,命有司使百姓有牛家以人牛相貿。又禁飲酒雜戲棄本沽販者,於是墾田大增。



 正平元年六月戊辰,薨於東宮,時年二十四。庚午,命持節兼太尉張黎、兼司空竇瑾奉策即柩謚景穆太子。文成即位,追尊為景穆
 皇帝,廟號恭宗。



 高宗文成皇帝諱濬,景穆皇帝之長子也。母曰閭氏。真君元年六月,生於東宮。



 帝少聰達,太武常置左右,號世嫡皇孫。年五歲,太武北巡,帝從在後,逢虜帥桎一奴,將加罰。帝謂曰:「奴今遭我,汝宜釋之。」帥奉命解縛。太武聞之曰:「此兒雖小,欲以天子自處。」意奇之。及長,風格異常,每參決大政可否。



 正平二年三月,中常侍宗愛弒逆,立南安王餘。十月丙午朔,又賊餘。於是殿中尚書長孫渴侯與尚書陸麗奉迎世嫡皇孫。



 興安元年冬十月戊申,皇帝即位於永安前殿。大赦,改
 元,正平二年為興安。



 以驃騎大將軍元壽樂為太宰、都督中外諸軍、錄尚書事。以尚書長孫渴侯為尚書令、儀同三司。十一月丙子,二人爭權,並賜死。癸未,廣陽王建、臨淮王譚薨。甲申,皇妣閭氏薨。進平南將軍、宋子侯周忸爵為樂陵王,南部尚書、常安子陸麗為平原王,文武各加位一等。壬寅,追尊皇考景穆太子為景穆皇帝,妣閭氏為恭皇后,尊保母常氏為保太后。十二月戊申,祔葬恭皇后於金陵。乙卯,初復佛法。丁巳,以樂陵王周忸為太尉,平原王陸麗為司徒,鎮西將軍杜元寶為司空。保達、沙獵等國各遣使朝貢。戊寅,進建業公陸俟爵為
 東平王,進廣平公杜遣爵為王。癸亥,詔以營州蝗,開倉振恤。甲子,太尉、樂陵王周忸有罪賜死。進濮陽公閭若文爵為王。



 二年春正月辛巳,進司空杜元寶爵為京兆王。廣平王杜遺薨。進尚書僕射、東安公劉尼爵為王。封建寧王崇子麗為濟南王。癸未,詔與百姓雜調十五。丙戌,進尚書、西平公源賀爵為王。二月己未,司空、京兆王杜元寶謀反,伏誅。建寧王崇、崇子濟南王麗為元寶所引,各賜死。乙丑,發京師五千人穿天泉池。是月,宋太子劭殺文帝。三月,尊保太后為皇太后。進安豐公閭武皮爵為河間
 王。夏五月,宋孝武帝殺太子劭而自立。閏月乙亥,太皇太后赫連氏崩。秋七月辛亥,行幸陰山。濮陽王閭若文、永昌王仁謀反。乙卯,仁賜死,若文伏誅。己巳,車駕還宮。是月,築馬射臺於南郊。八月戊戌,詔曰:「朕即位以來,風雨順序,邊方無事,眾瑞兼呈。又於苑內獲方寸玉印,其文曰『子孫長壽』。群公卿士咸曰休哉,豈朕一人,克臻斯應,實由天地祖宗降祐之所致也。思與兆庶,共茲嘉慶。其令百姓大酺三日,降殊死已下囚。」九月壬子,閱武於南郊。冬十一月辛酉,行幸信都、中山,觀察風俗。十二月甲午,車駕還宮。復北平公長孫敦王爵。是歲,疏勒、渴盤
 陀、庫莫奚、契丹、罽賓等國各遣使朝貢。



 興光元年春正月乙丑,以侍中、河南公伊珝為司空。二月甲午,帝至道壇,登受圖籙。禮華,曲赦京師。夏六月,行幸陰山。秋七月丙申朔,日有蝕之。庚子,皇子弘生。辛丑,大赦改元。八月甲戌,趙王深薨。乙亥,車駕還宮。乙丑,皇叔武頭、龍頭薨。九月,庫莫奚國獻名馬,有一角,狀如麟。閉都門,大索三日,獲姦人亡命數百人。冬十一月戊戌,行幸中山,遂幸信都。十二月丙子,還幸靈丘,至溫泉宮。庚辰,車駕還宮。出於、叱萬單等國各遣使朝貢。



 太安元年春正月辛酉,奉太武、景穆神主于太廟。樂平
 王拔有罪,賜死。二月癸未,武昌王提薨。三月己亥,以太武、景穆神主入太廟,改元,曲赦京師死囚已下。夏六月壬戌,詔名皇子弘,曲赦。癸酉,詔尚書穆真等二十人巡行州郡,觀察風俗,大明賞罰。冬十月庚午,以遼西公常英為太宰,進爵為王。是歲,遮逸、波斯、疏勒等國各遣使朝貢。



 二年春正月乙卯,立皇后馮氏。二月丁巳,立皇子弘為皇太子,大赦。夏六月,羽林中郎于判、元提等謀逆,誅。秋八月,田于河西。平西將軍、漁陽公尉眷北擊伊吾,剋其城,大獲而還。九月辛巳,進河東公閭毗、零陵公閭紇爵,
 並為王。冬十月甲申,車駕還宮。甲午,曲赦京師。十一月,改封西平王源賀隴西王。嚈噠、普嵐等國各遣使朝貢。



 三年春正月,征漁陽公尉眷拜太尉,進爵為王,錄尚書事。夏五月,封皇弟新成為陽平王。六月癸卯,行幸陰山。秋八月,田於陰山之北。己亥,還宮。冬十月,將東巡,詔太宰常英起行宮於遼西黃山。十二月,州鎮五蝗,百姓饑,使開倉振給之。是歲,粟特、于闐等五十餘國並遣使朝貢。



 四年春正月丙午朔,初設酒禁。乙卯,行幸廣寧溫泉宮,遂東巡。庚午,至遼西黃山宮。遊宴數日,親對高年,勞問
 疾苦。二月丙子,登碣石山,觀滄海,大饗群臣於山上,班賞進爵各有差。改碣石山為樂遊山,築壇記行於海濱。戊寅,南幸信都,田於廣川。三月丁未,觀馬射於中山。所過郡國賜復一年。丙辰,車駕還宮。



 起太華殿。乙丑,東平王陸俟薨。夏五月壬戌,詔曰:「比年以來,雜調減省,而所在州郡咸有逋懸。非在職之官綏導失所,貪穢過度,誰使之然?自今常調不充,人不安業,宰人之徒,加以死罪。」六月丙申,田於松山。秋七月庚午,行幸河西。



 九月丁巳,還宮。辛亥,太華殿成。丙寅,饗群臣,大赦。冬十月甲戌,北巡,至陰山。有故冢毀廢,詔曰:「昔姬文葬枯骨,天下歸仁。
 自今有穿墳壟者,斬之。」



 辛卯,次於車輪山,累石記行。十一月,車駕渡漠,蠕蠕絕跡遠遁。十二月,中山王託真薨。



 五年春二月己酉,司空、河南公伊珝薨。三月庚寅,曲赦京師死罪已下。夏四月己巳,封皇弟子推為京兆王。五月,居常國遣使朝貢。六月戊申,行幸陰山。秋八月庚戌,遂幸雲中。壬戌,還宮。九月戊辰,儀同三司、敦煌公李寶薨。冬十二月戊申,詔以六鎮、雲中、高平、二雍、秦州遍遇災旱,年穀不收,開倉廩振乏。



 有徙流者,喻還桑梓。



 和平元年春正月甲子朔,大赦改元。庚午,詔散騎侍郎馮闡使於宋。夏四月戊戌,皇太后常氏崩於壽安宮。五
 月癸酉,葬昭太后於廣寧鳴雞山。六月甲午,詔征西大將軍、陽平王新成等討吐谷渾什寅。崔浩之誅也,史官遂廢,至是復置。秋七月,西征諸軍至西平,什寅走保南山。九月庚申朔,日有蝕之。是月,諸軍濟河,追什寅。遇瘴氣,多病疫,乃引還。庚午,車駕還宮。冬十月,居常王獻馴象三。



 十一月,詔散騎侍郎盧度世使於宋。



 二年春正月乙酉,詔曰:「刺史牧人,為萬里之表。自頃每因發調,逼人假貸,大商富賈,要射時利,上下通同,分以潤屋。為政之弊,莫過於此,其一切禁絕。



 犯者,十疋以上皆死。布告天下,咸令知禁。」二月,行幸中山,遂幸信都。三
 月,宋人來聘。車駕所過,皆親對高年,問疾苦。詔年八十,一子不從役。靈丘南有山高四百餘丈,乃詔群臣仰射山峰,無能踰者。帝彎弧發矢,出三十餘丈,過山南二百二十步。遂刊石勒銘。是月,發並、肆州五千餘人脩河西獵道。辛巳,車駕還宮。



 夏四月乙未,河東王閭毗薨。五月癸未,詔南部尚書黃盧頭、李敷業考課諸州。秋七月戊寅,封皇弟小新成為濟陰王,天賜為汝陰王,萬壽為樂良王,洛侯為廣平王。



 八月,波斯國遣使朝貢。冬十月,詔假員外散騎常侍游明根使于宋。廣平王洛侯薨。



 三年春正月壬午,以東郡公乙渾為太原王。癸未,樂良
 王萬壽薨。二月壬子朔,日有蝕之。癸酉,田於崞山,遂觀漁于旋鴻池。三月甲申,宋人來聘。高麗、蓰王、契嚙、思厭、於師、疏勒、石那、悉居半、渴盤陀等國並遣使朝貢。夏六月庚申,行幸陰山。秋七月壬寅,幸河西。九月壬辰,常山王素薨。冬十月,詔員外散騎常侍游明根使于宋。十一月壬寅,車駕還宮。十二月乙卯,制戰陣之法十有餘條,因大儺曜兵,有飛龍騰蛇魚麗之變,以示威武。戊午,零陵王閭拔薨。



 四年春三月乙未,賜京師人年七十以上太官廚食,以終其年。皇子胡仁薨,追封樂陵王。夏四月癸亥,上幸西
 苑,親射猛獸三頭。五月壬辰,侍中、漁陽王尉眷薨。壬寅,行幸陰山。秋七月壬午,詔曰:「朕每歲閑月,命群臣講武。所幸之處,必立宮壇。糜費之功,勞損非一,宜仍舊費,何必改作也。」八月丙寅,遂田于河西。九月辛巳,車駕還宮。冬十月,以定、相二州隕霜傷稼,免其田租。詔員外散騎常侍游明根使於宋。十二月辛丑,詔以喪葬嫁娶,大禮未備,命有司為之條格,使貴賤有章,上下咸序,著之于今。壬寅,詔曰:「婚姻者,人道之始。比者以來,貴族之門多不率法,或貪利財賂,或因緣私好,在於茍合,無所擇選。塵穢清化,虧損人倫,將何以宣示典謨,垂之來裔。今制
 皇族師傅王公侯伯及士庶之家,不得與百工伎巧卑姓為婚,犯者加罪。」



 五年春正月丁亥,封皇弟雲為任城王。二月,詔以州鎮十四去歲蟲水,開倉振恤。夏四月癸卯,進封頓丘公李峻為王。閏月戊子,帝以旱故,減膳責身。是夜,澍雨大降。五月,宋孝武帝殂。六月丁亥,行幸陰山。秋七月壬寅,行幸河西。九月辛丑,車駕還宮。冬十月,琅邪侯司馬楚之薨。十二月,南秦王楊難當薨。吐呼羅國遣使朝貢。



 六年春正月丙申,大赦。二月丁丑,行幸樓煩宮。高麗、蓰王、對曼等國各遣使朝貢。三月戊戌,相州刺史、西平郡
 王吐谷渾權薨。乙巳,車駕還宮。夏四月,破洛那國獻汗血馬,普嵐國獻寶劍。五月癸卯,帝崩于太華殿,時年二十六。六月丙寅,奉尊謚曰文成皇帝,廟號高宗。八月,葬雲中之金陵。



 顯祖獻文皇帝諱弘,文成皇帝之長子也,母曰李貴人。興光元年七月生於陰山之北。太安二年二月,立為皇太子。



 和平六年五月甲辰,即皇帝位,大赦。尊皇后曰皇太后。車騎大將軍乙渾矯詔殺尚書楊保年、平陽公賈愛仁、南陽公張天度于禁中。戊申,司徒公、平原王陸麗自湯泉入朝,又殺之。己酉,以渾為太尉公,以錄尚書事、
 東安王劉尼為司徒公,以尚書左僕射和其奴為司空公。六月,封繁陽侯李嶷為丹楊王,征東大將軍馮熙為昌黎王。秋七月癸巳,以太尉乙渾為丞相,位居諸王上,事無大小皆決焉。九月庚子,曲赦京師。丙午,詔曰:「先朝以州牧親人,宜置良佐,故敕有司班九條之制,使前政選吏以待後人。然牧司舉非其人,愆于典度。今制刺史守宰到官之日,仰自舉人望忠信,以為選官,不論前政,共相平置。若簡任失所,以罔上論。」是月,宋義陽王劉昶自彭城來奔。冬十月,征陽平王新成、京兆王子推、濟陰王小新成、汝陰王天賜、任城王雲入朝。十一月,宋湘東
 王彧殺其主子業而自立。



 天安元年春正月己丑朔,大赦,改元。二月庚申,丞相、太原王乙渾謀反,伏誅。乙亥,以侍中元孔雀為濮陽王,侍中陸定國為東郡王。三月庚子,以隴西王源賀為太尉公。辛丑,京宗文成皇帝神主祐於太廟。辛亥,帝幸道壇,親受符籙。曲赦京師。秋九月己酉,初立鄉學,郡置博士二人,助教二人,學生六十人。冬十二月,皇弟安平王薨。是歲,州鎮十一旱,人飢,開倉振恤。



 皇興元年春正月癸巳,鎮南大將軍尉元大破宋將張永、沈攸之於呂梁東。宋人來聘。庚子,東平王道符謀反
 於長安,其司馬段太陽斬之,傳首京師。道符兄弟皆伏誅。閏月,以頓丘王李峻為太宰。二月,濟陰王小新成薨。宋東平太守申纂戍無鹽,遏絕王使,詔征南大將軍慕容白曜督諸軍往討。三月甲寅,剋之。秋八月丁酉,幸武州山石窟寺。戊申,皇子宏生。大赦,改元。九月己巳,進馮翊公李白為梁郡王。冬十月己亥朔,日有蝕之。癸卯,田於那男池。濮陽王孔雀坐怠慢降為公。



 二年春二月癸未,田于西山,親射武豹。三月,慕容白曜進圍東陽。戊午,宋人來聘。夏四月丙子朔,日有蝕之。辛丑,進南郡公李惠爵為王。五月乙卯,田于崞山,遂幸繁
 畤。辛酉,車駕還宮。六月庚辰,以河南避地,曲赦京師殊死已下。



 以昌黎王馮熙為太傅。秋九月辛亥,封皇叔楨為南安王,長壽為城陽王,太洛為章武王,休為安定王。冬十月癸酉朔,日有蝕之。辛丑,田于冷泉。十一月,州鎮二十七水旱,詔開倉振恤。十二月甲午,詔曰:「頃張永敢拒王威,暴骨原隰。天下之人一也,其永軍殘廢之士,聽還江南。露骸草莽者,敕州縣收瘞之。」



 三年春正月乙丑,東陽潰,虜沈文秀。戊辰,司空、平昌公和其奴薨。二月己卯,進上黨公慕容白曜爵為濟南王。夏四月壬辰,宋人來聘。丙申,名皇子宏,大赦。丁酉,田于
 崞山。五月,徙青、齊人於京師。六月辛未,立皇子宏為皇太子。



 冬十月丁酉朔,日有蝕之。是月,太宰、頓丘王李峻薨。十一月,進襄城公韓頹爵為王。



 四年春正月,州鎮大飢,詔開倉振恤。二月,以東郡王陸定國為司空公。詔征西大將軍、上黨王長孫觀討吐谷渾什寅。廣陽王石侯薨。三月丙戌,詔天下人病者,所在官司遣醫就家診視,所須藥任醫所量給之。夏四月辛丑,大赦。戊申。長孫觀軍至曼頭山,大破什寅。五月,封皇弟長樂為建昌王。六月,宋人來聘。秋八月,蠕蠕犯塞。九月丙寅,車駕北伐,諸將俱會于女水,大破虜軍。司徒、東
 安公劉尼坐事免。壬申,車駕至自北伐,飲至策勛,告於宗廟。冬十月,誅濟南王慕容白曜、高平公李敷。十一月,詔弛山澤禁。十二月甲辰,幸鹿野苑、石窟寺。陽平王新成薨。



 五年春二月乙亥,詔假員外散騎常侍邢祐使于宋。夏四月,北平王長孫敦薨。



 六月丁未,行幸河西。秋七月丙寅,遂至陰山。八月丁亥,車駕還宮。帝幼而神武,聰睿機悟,有濟人之規。仁孝純至,禮敬師友。及即位,雅薄時務,常有遺世之心,欲禪位于叔父京兆王子推,群臣固請,乃止。丙午,使太保建安王陸珝、太尉源賀奉皇帝璽綬,
 冊命皇太子升帝位。於是群公奏上尊號太上皇帝。己酉,太上皇帝徙御崇光宮,採椽不斲,土階而已。國之大事咸以聞。承明元年,文明太后有憾,帝崩於永安殿,年二十三。上尊謚曰獻文皇帝,廟號顯祖。葬雲中金陵。



 論曰:太武聰明雄斷,威靈傑立。藉二世之資,奮征伐之氣,遂戎軒四出,周旋夷險。平秦、隴,掃統萬,翦遼海,蕩河源。南夷荷擔,北蠕絕迹,廓定四表,混一華戎。其為武功也大矣。遂使有魏之業,光邁百王。豈非神睿經綸,事當命世。



 至於初則東儲不終,末乃釁成所忽,固本貽防,殆弗思乎。



 景穆明德令聞,夙世殂夭,其戾園之悼歟。



 文成
 屬太武之後,內頗虛耗,既而國釁時艱,朝野楚楚。帝與時消息,靜以鎮之。養威布德,懷緝中外,自非機悟深裕,矜濟為心,亦何能若此?可謂有君人之度矣。



 獻文聰睿夙成,兼資雄斷,故能更清漠野,大啟南服。而早有厭世之心,終致宮闈之變,將天意也。



\end{pinyinscope}