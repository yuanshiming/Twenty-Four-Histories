\article{卷九十六列傳第八十四}

\begin{pinyinscope}

 氐吐谷渾宕昌鄧至白蘭黨項附國稽胡氐者,西夷之別種,號曰白馬。三代之際,蓋自有君長,而世一朝見,故《詩》稱「自彼氐、羌,莫敢不來王」也。秦、漢以來,世居岐、隴以南,漢川以西,自立豪帥。漢武帝遣中郎將郭昌、衛廣滅之,以其地為武都郡。
 自汧、渭抵於巴、蜀,種類實繁,或謂之白氏,或謂之故氐,各有侯王,受中國封拜。



 漢建安中,有楊騰者,為部落大帥。騰勇健多計略,始徙居仇池,方百頃,因以為號。四面斗絕,高七里餘,蟠道三十六回,其上有豐水泉,煮土成鹽。騰後有名千萬者,魏拜為百頃氐王。



 千萬孫名飛龍,漸強盛,晉武帝假平西將軍。無子,養外甥令狐茂搜為子。惠帝元康中,茂搜自號輔
 國
 將
 軍、右賢王,群氐推以為王。關中人士流移者,多依之。



 愍帝以為驃騎將軍、左賢王。茂搜死,子難敵統位,與弟堅頭分部曲。難敵自號左賢王,屯下辨;堅頭號右賢王,屯河池。難敵死,子毅立。自號使持節、龍驤將軍、左賢王、下辨公;以堅頭子盤為使持節、冠軍將軍、右賢王、河池
 公。臣晉,晉以毅為征南將軍。



 三年,毅族兄初襲殺毅,并有其眾,自立為仇池公。臣於石季龍,後稱蕃於晉。



 永和十年,改初為天水公。十一年,毅小弟宋奴使姑子梁三王因侍直手刃殺初,初子國率左右誅三王及宋奴,復自立為仇池公。桓溫表國為秦州刺史,國子安為武都太守。



 十二年,國從叔俊復殺國自立。國子安叛苻生,殺俊,復稱蕃於晉。死,子世自立為仇池公。晉太和三年,以世為秦州刺史,弟統為武都太守。世死,統廢世子纂自立。統一名德。纂聚黨襲殺統,自立為仇池公,遣使詣簡文帝。以纂為秦州刺史。晉咸安元年,苻堅遣楊安伐纂,剋
 之,徙其人于關中,空百頃於地。



 宋奴之死,二子佛奴、佛狗逃奔苻堅,堅以女妻佛奴子定,拜為尚書、領軍。



 苻堅之敗,關右擾亂,定盡力於堅。堅死,乃率眾奔隴右,徙居歷城,去仇池百二十里,置食儲於百頃。招夷夏得千餘家,自稱龍驤將軍、仇池公,稱蕃於晉。孝武即以其自號假之,後以為秦州刺史。登國四年,遂有秦州之地,號隴西王。後為乞佛乾歸所殺,無子。



 佛狗子盛,先為監國守仇池,乃統事,自號征西將軍、秦州刺史、仇池公。謚定為武王。分諸氐、羌為二十部護軍,各為鎮戍,不置郡縣。遂有漢中之地,仍稱蕃于晉。天興初,遣使朝貢,詔以盛為
 征南大將軍、仇池王。隔礙姚興,不得歲通貢使。盛以兄子撫為平南將軍、梁州刺史,守漢中。宋永初中,宋武帝封盛為武都王。盛死,私謚曰惠文王。子玄統位。



 玄字黃眉,號征西大將軍、開府儀同三司、秦州刺史、武都王。雖蕃於宋,仍奉晉義熙之號。後始用宋元嘉正朔。初,盛謂玄曰:「吾年已老,當終為晉臣,汝善事宋帝。」故玄奉焉。玄善於待士,為流舊所懷。始光四年,太武遣大鴻臚公孫軌拜玄為征南大將軍、督梁州刺史、南秦王。玄上表請比內蕃,許之。玄死,私謚孝昭王。子保宗統位。



 初,玄臨終謂弟難當曰:「今境候未寧,方須撫慰,保宗沖昧,吾授卿
 國事,其無墜先勳。」難當固辭,請立保宗以輔之。保宗既立,難當妻姚氏謂難當曰:「國險,宜立長君,反事孺子,非久計。」難當從之,廢保宗而自立,稱蕃於宋。



 難當拜保宗為鎮南將軍,鎮石昌;以次子順為鎮東將軍、秦州刺史,守上邽。保宗謀襲難當,事泄,被繫。先是,四方流人以仇池豐實,多往依附。流人有許穆之、郝惔之二人投難當,並改姓為司馬,穆之自云名飛龍,惔之自云名康之,云是晉室近戚。康之尋為人所殺。時宋梁州刺史甄法護刑政不理,宋文帝遣刺史蕭思話代任。



 難當以思話未至,遣將舉兵襲梁州,破白馬,遂有漢中之地。尋而思話使
 其司馬蕭承之先驅進討,所向剋捷,遂平梁州。因又附宋。難當後釋保宗,遣鎮董亭。保宗與兄保顯歸京師,太武拜保宗征南大將軍、秦州牧、武都王,尚公主;保顯為鎮西將軍、晉壽公。後遣大鴻臚崔頤拜難當為征南大將軍、儀同三司、領護西羌校尉、秦梁二州牧、南秦王。



 難當後自立為大秦王,號年曰建義,立妻為王后,世子為太子,置百官具擬天朝。然猶貢獻于宋不絕。尋而其國大旱,多災異,降大秦王復為武都王。太延初,難當立鎮上邽。太武遣車騎大將軍、樂平王丕等督河西、高平諸軍取上邽,又詔喻難當,奉詔攝守。尋而傾國南寇,規有蜀
 土,襲宋益州,攻涪城,又伐巴西,獲雍州流人七千餘家,還于仇池。宋文帝怒,遣將裴方明等伐之。難當為方明所敗,棄仇池,與千餘騎奔上邽。太武遣中山王辰迎之赴行宮。方明既剋仇池,以保宗弟保熾守之,河間公齊擊走之。



 先是,詔保宗鎮上邽,又詔鎮駱谷,復其本國。保宗弟文德先逃氐中,乃說保宗令叛。事泄,齊執保宗送京師,詔難當殺之。氐、羌立文德,屯於濁水。文德自號征西將軍、秦河梁三州牧、仇池公,求援于宋,封文德為武都王,遣偏將房亮之等助之。齊逆擊,禽亮之。文德奔守葭蘆,武都、陰平氐多歸之。詔淮陽公皮豹子等率諸軍
 討文德,走漢中,收其妻子、僚屬、資糧。及保宗妻公主送京師,賜死。



 初,公主勸保宗反,人問曰:「背父母之邦若何?」公主曰:「禮,婦人外成,因夫而榮。事立,據守一方,我亦一國之母,豈比小縣之主?」以此得罪。



 文成時,拜難當營州刺史,還為外都大官。卒,謚曰忠。子和,隨父歸魏,別賜爵仇池公。子德子襲難當爵,早卒。子小眼襲,例降為公,拜天水太守,卒。子大眼,別有傳。小眼子公熙襲爵。正光中,尚書右丞張普惠為行臺,送租於南秦、東益,普惠啟公熙俱行。至南秦,以氐反不得進,遣公熙先慰氐。東益州刺史魏子建以公熙險薄,密令訪察,公熙果有潛謀,將
 為叛亂。子建仍報普惠,令其攝錄。



 普惠急追公熙,公熙竟不肯赴,東出漢中。普惠表列其事,公熙大行賄賂,終得免罪。後為假節、別將,與都督元志同守岐州,為秦賊莫折天生所虜,死於秦州。



 文德後自漢中入統汧、隴,遂有陰平、武興之地。後為宋荊州刺史劉義宣所殺。



 保宗之執也,子元和奔宋,以為武都、白水太守。元和據城歸順,文成嘉之,拜征南大將軍、武都王,內徙京師。



 元和從叔僧嗣復自稱武都王於葭蘆。僧嗣死,從弟文度自立為武興王,遣使歸順。獻文授文度武興鎮將,既而復叛。孝文初,征西將軍皮歡喜攻葭蘆破之,斬文度首。



 文度弟
 弘,小名鼠,名犯獻文廟諱,以小名稱。鼠自為武興王,遣使奉表謝罪,貢其方物,孝文納之。鼠遣子狗奴入侍,拜鼠都督、南秦州刺史、征西將軍、西戎校尉、武都王。鼠死,從子後起統位,孝文復以鼠爵授之。鼠子集始為白水太守。



 後起死,以集始為征西將軍、武都王。集始復朝於京師,拜都督、南秦州刺史、安南大將軍、領護南蠻校尉、漢中郡侯、武興王,賜以車旗、戎馬、錦彩、繒纊。尋還武興,進號鎮南將軍,加督寧、湘五州諸軍事。後仇池鎮將楊靈珍襲破武興,集始遂入齊。景明初,集始來降,還授爵位,歸守武興。死,子紹先立,拜都督、南秦州刺史、征虜將
 軍、漢中郡公、武興王,贈集始車騎大將軍、開府儀同三司、謚安王。



 紹先年幼,委事二叔集起、集義。夏侯道遷以漢中歸順也,梁白馬戍主尹天保率眾圍之。道遷求援於集起、集義,二人貪保邊蕃,不欲救之。唯集始弟集朗心願立功。率眾破天保,全漢川,朗之力也。集義見梁、益既定,恐武興不得久為外籓,遂扇動諸氐,推紹先僭稱大號,集起、集義並稱王,外引梁為援。安西將軍邢巒遣建武將軍傅豎眼攻武興剋之,執紹先,送於京師,遂滅其國,以為武興鎮,復改鎮為東益州。



 前後鎮將唐法樂、刺史杜纂、邢豹以威惠失衷,氐豪仇石柱等相率反叛,
 朝廷以西南為憂。正光中,詔魏子建為刺史,以恩信招撫,風化大行,遠近款附,如內地焉。後唐永代子建為州,未幾,氐人悉反。永棄城東走,自此復為氐地。



 魏末,天下亂,紹先奔還武興,復自立為王。周文定秦、隴,紹先稱籓,送妻子為質。大統元年,紹先請其女妻,周文奏魏帝許之。紹先死,子辟邪立。



 四年,南岐州氐苻壽反,攻陷武都,自號太白王,詔大都督侯莫陳順與渭州刺史長孫澄討降之。九年,清水氐酋李鼠仁據地作亂,氐帥梁道顯叛,攻南由,周文遣典簽趙昶慰諭之,鼠仁等相繼歸附。十一年,於武興置東益州,以辟邪為刺史。



 十五年,安夷
 氐復叛。趙昶時為郡守,收首逆者二十餘人斬之,乃定。於是以昶行南秦州事。氐帥蓋鬧等作亂,鬧據北谷,其黨西結宕昌羌獠甘,共推蓋鬧為主。昶分道遣使,宣示禍福,然後出兵討之。擒蓋鬧,散其餘黨。興州叛氐復侵逼南岐州,刺史叱羅協遣使告急,昶赴救,又大破之。先是,氐酋楊法深據陰平自稱王,亦盛之苗裔也。魏孝昌中,舉眾內附,自是職貢不絕。廢帝元年,以深為黎州刺史。二年,楊辟邪據州反,群氐復與同逆。詔叱羅協與趙昶討平之。周文乃以大將軍宇文貴為大都督、興州刺史。貴威名先著,群氐頗畏服之。來歲,楊法深從尉遲迥
 平蜀,軍迴,法深尋與其宗人楊崇集、楊陳侳各擁其眾,遞相攻討。趙昶時督成、武、沙三州諸軍事,遣使和解之。法深等從命,乃分其部落,更置州郡以處之。



 恭帝末,武興氐反,圍利州,鳳州固道氐魏天王等亦聚眾響應,大將軍豆盧寧等討平之。周明帝時,興州人段吒及下辯、柏樹二縣人反,相率破蘭皋戍。氐酋姜多復率廚中氐屬攻陷落叢郡以應之。趙昶討平二縣,並斬段吒。而陰平、葭蘆氐復往往屯聚,與廚中相應。昶乃簡精騎,出其不意,徑入廚中,至大竹坪,連破七柵,誅其渠帥,二郡並降。及昶還,廚中生氐,復為寇掠。昶又遣儀同劉崇義、宇文琦
 入廚中討之,於是群氐並平。



 及王謙舉兵,沙州氐帥開府楊永安又據州應謙,大將軍達奚儒討平之。



 吐谷渾,本遼東鮮卑徒河涉歸子也。涉歸一名弈洛韓,有二子,庶長曰吐谷渾,少曰若洛廆。涉歸死,若洛廆代統部落,是為慕容氏。涉歸之在也,分戶七百以給吐谷渾,與若洛廆二部。馬鬥相傷,若洛廆怒,遣入謂吐谷渾曰:「先公處分,與兄異部,何不相遠,而馬鬥相傷?」吐谷渾曰:「馬食草飲水,春氣發動,所以斗,鬥在馬,而怒及人,乖別甚易,今當去汝萬里外!」若洛廆悔,遣舊老及長史七那樓謝之。吐谷渾曰:「我乃祖以來,樹德遼右,先公之世,
 卜筮之言云:『有二子,當享福祚,並流子孫。』我是卑庶,理無並大。今以馬致怒,殆天所啟。諸君試驅馬令東,馬若還東,我當隨去。」即令從騎擁馬令迴,數百步,欻然悲鳴,突走而西,聲若頹山,如是者十餘輩,一迴一迷。樓力屈,乃跪曰:「可汗,此非復人事!」



 渾謂其部落曰:「我兄弟子孫並應昌盛,廆當傳子及曾玄孫,其間可百餘年;我乃玄孫間始當顯耳。」於是遂西附陰山,後假道上隴。若洛廆追思吐谷渾,作《阿于歌》徒河以兄為阿于也。子孫僭號,以此歌為輦後鼓吹大曲。



 吐谷渾遂從上隴,止於枹罕。自枹罕暨甘松,南界昂城、隴涸,從洮水西南極白蘭,數
 千里中,逐水草,廬帳而居,以肉酪為糧。西北諸雜種謂之阿柴虜。



 吐谷渾死,有子六十人。長子吐延,身長七尺八寸,勇力過人,性刻暴。為昂城羌酋姜聰所刺,劍猶在體,呼子葉延語其大將絕拔泥曰:「吾氣絕,棺斂訖,便速去保白蘭。地既險遠,又土俗懦弱,易控禦。葉延小兒,欲授餘人,恐倉卒終不能相制。今以葉延付汝,竭股肱之力以輔之,孺子得立,吾無恨也。」抽劍而死。



 有子十二人。



 葉延少而勇果,年十歲,縛草為人,號曰姜聰,每旦輒射之,射中則嗥叫泣涕。



 其母曰:「仇賊諸將已屠膾之,汝年小,何煩朝朝自苦!」葉延嗚咽若不自勝,答母曰:「誠知無益,
 然罔極之心,不勝其痛。」性至孝,母病,母三日不食,葉延亦不食。頗視書傳,自謂曾祖弈洛韓始封昌黎公,吾為公孫之子,案《禮》,公孫之子得以王父字為氏,遂以吐谷渾為氏焉。



 葉延死,子碎奚立。性淳謹,三弟專權,碎奚不能制,諸大將共誅之。奚憂哀不復攝事,遂立子視連為世子,委之事。號曰莫賀郎,華言父也。奚遂以憂死。視連立,以父憂思,不遊娛酣宴。十五年死,弟視羆立。死,子樹洛干等並幼,弟烏紇提立,而妻樹洛干母,生二子慕璝、慕利延。烏紇提一名大孩,死,樹洛干立,自號車騎將軍。是歲,晉義熙初也。樹洛干死,弟阿豺立,自號驃騎將軍、沙
 州刺史。部內有黃沙,周迴數百里,不生草木,因號沙州。阿豺兼並氐、羌,地方數千里,號為強國。昇西強山,觀墊江源,問於群僚曰:「此水東流,更有何名?由何郡國入何水也?」其長史曾和曰:「此水經仇池,過晉壽,出宕渠始號墊江,至巴郡入江,度廣陵入於海。」阿豺曰:「水尚知歸,吾雖塞表小國,而獨無所歸乎!」



 遣使通宋,獻其方物。宋少帝封為澆河公。未及拜受,宋文帝元嘉三年,又加除命。



 又將遣使朝貢,會暴病,臨死召諸子弟告之曰:「先公車騎捨其子虔,以大業屬吾,豈敢忘先公之舉而私於緯代!其以慕璝繼事。」阿豺有子二十人,緯代長子也。阿豺
 又謂曰:「汝等各奉吾一隻箭,將玩之地下。」俄而命母弟慕利延曰:「汝取一隻箭折之。」慕利延折之。曰:「汝取十九隻箭折之。」慕延不能折。阿豺曰:「汝曹知不?單者易折,眾則難摧,戮力一心,然後社稷可固。」言終而死。慕璝立。



 先是,阿豺時,宋命竟未至而死。慕璝又奉表通宋,宋文帝又授隴西公。慕璝招集秦、涼亡業之人,及羌戎雜夷眾至五六百落,南通蜀、漢,北交涼州、赫連,部眾轉盛。太武時,慕璝始遣其侍郎謝大寧奉表歸魏。尋討禽赫連定,送之京師。



 太武嘉之,遣使者策拜慕璝為大將軍、西秦王。



 慕璝表曰:「臣誠庸弱,敢竭精款,俘擒僭逆,獻捷王府,
 爵秩雖崇,而土不增廓,車旗既飾,而財不周賞,願垂鑒察,亮基單款。臣頃接寇逆,疆境之人,為賊所抄,流轉東下,今皇化混一,求還鄉土。乞佛曰連、窟略寒、張華等三人家弱在此,分乖可愍,願並敕遣,使恩洽遐荒,存亡感戴。」



 太武詔公卿朝堂會,議答施行。太尉長孫嵩及議郎、博士二百七十九人議曰:「前者有司所處,以為秦王荒外之君,本非政教所及,來則受之,去則不禁。



 皇威遠被,西秦王慕義畏威,稱臣納貢,求受爵號。議者以為古者要荒之君,雖人土眾廣,而爵不擬華夏。陛下加寵王官,乃越常分,容飾車旗,班同上國。至於繒絮多少,舊典所
 無,皆當臨時以制豐寡。自漢、魏以來,撫綏遐荒,頗有故事。呂后遺單于御車二乘、馬二駟,單于答馬千匹。其後匈奴和親,敵國,遺繒絮不過數百;呼韓邪稱臣,身自入朝,始乃至萬匹。今西秦王若以土無桑蠶,便當上請,不得言財不周賞也。周室衰微,齊侯小白一匡天下,有賜胙之命,無益土之賞。晉侯重耳破楚城濮,唯受南陽之田,為朝宿之邑。西秦所致,唯定而已。塞外之人,因時乘便,侵入秦、涼,未有經略拓境之勳,爵登上國,統秦、涼、河、沙四州之地,而云土不增廓。比聖朝於弱周,而自同於五霸,無厭之情,其可極乎!西秦王忠款於朝廷,原其本
 情,必不至此。或左右不敕,因致斯累。



 檢西秦流人,賊時所抄,悉在蒲阪。今既稱籓,四海咸泰,天下一家,可敕秦州送詣京師,隨後遣還。所請乞佛三人,昔為賓國之使,來在王庭,國破家遷,即為臣妾,可勿聽許。



 制曰:「公卿議之,未為失體。西秦王所書金城、枹罕、隴西之地,彼自取之,朕即與之,便是裂土,何須復廓?西秦款至,綿絹隨使疏數增益之,非一匹而已。」



 自是,慕璝貢獻頗簡。又通於宋,宋文封為隴西王。



 太延二年,慕璝死,弟慕利延立。詔遣使者策謚慕璝曰惠王。後拜慕利延鎮西大將軍、儀同三司,改封西平王;以慕璝子元緒為撫軍將軍。時慕
 利延又通宋,宋封為河南王。太武征涼州,慕利延懼,遂率其部人,西遁沙漠。太武以利延兄有禽赫連定之功,遣使宣喻之,乃還。後慕利延遣使表謝,書奏,乃下詔褒獎之。



 慕利延兄子緯代懼慕利延害已,與使者謀欲自歸,慕利延覺而殺之。緯代弟叱力延等八人逃歸京師,請兵討慕利延。太武拜叱力延歸義王,詔晉王伏羅率諸將討之。軍至大母橋,慕利延兄子拾寅走河西,伏羅遣將追擊之,斬首五千餘級。慕利延走白蘭,慕利延從弟伏念、長史䳕鳩黎、部大崇娥等率眾一萬三千落歸降。後復遣征西將軍、高涼王那等討之於白蘭。慕利
 延遂入于闐國,殺其王,死者數萬人。



 南征罽賓。遣使通宋求援,獻烏丸帽、女國金酒器、胡王金釧等物,宋文帝賜以牽車。七年,遂還舊土。



 慕利延死,樹洛干子拾寅立。始邑於伏羅川,其居止出入,竊擬王者。拾寅奉修貢職,受魏正朔;又受宋封爵,號河南王。太武遣使拜為鎮西大將軍、沙州刺史、西平王。後拾寅自恃險遠,頗不恭命。通使於宋,獻善馬、四角羊,宋明帝加之官號。



 文成時,定陽侯曹安表拾寅今保白蘭,多有金銀、牛馬,若擊之,可以大獲。



 議者咸以先帝忿拾寅兄弟不睦,使晉王伏羅、高涼王那再徵之,竟無多剋,拾寅雖復遠遁,軍亦疲勞。
 今在白蘭,不犯王塞,不為人患,非國家之所急也。若遣使招慰,必求為臣妾,可不勞而定也。王者之於四荒,羈縻而已,何必屠其國,有其地。



 安曰:「臣昔為澆河戍將,與之相近,明其意勢。若分軍出其左右,拾寅必走保南山,不過十日,牛馬草盡,人無所食,眾必潰叛,可一舉而定也。」從之。詔陽平王新成、建安王穆六頭等出南道,南郡公李惠、給事中公孫拔及安出北道以討之。



 拾寅走南山,諸軍濟河追之。時軍多病,諸將議賊已遠遁,軍容已振,今驅疲病之卒,要難冀之功,不亦過乎?眾以為然,乃引還,獲駝馬二十餘萬。



 獻文復詔上黨王長孫觀等率
 州郡兵討拾寅。軍至曼頭山,拾寅來逆戰,觀等縱兵擊敗之,拾寅宵遁。於是思悔復蕃職,遣別駕康盤龍奉表朝貢。獻文幽之,不報其使。拾寅部落大饑,屢寇澆河。詔平西將軍、廣川公皮歡喜率敦煌、涼州、枹罕、高平諸軍為前鋒,司空、上黨王長孫觀為大都督以討之。觀等軍入拾寅境,芻其秋稼。拾寅窘怖,遣子詣軍,表求改過,觀等以聞。獻文以重勞將士,乃下詔切責之,徵其任子。拾寅遣子斤入侍,獻文尋遣斤還。拾寅後復擾掠邊人,遣其將良利守洮陽,枹罕所統也。枹罕鎮將、西郡公楊鍾葵貽拾寅書以責之。拾寅表曰:「奉詔,聽臣還舊土,故遣良利
 守洮陽。若不追前恩,求令洮陽貢其土物。」辭旨懇切,獻文許之,自是歲修職貢。



 太和五年,拾寅死,子度易侯立。遣其侍郎時真貢方物,提上表稱嗣事。後度易侯伐宕昌,詔讓之,賜錦彩一百二十匹,喻令悛改;所掠宕昌口累,部送時還。



 易侯並奉詔。死。



 子伏連籌立。孝文欲令入朝,表稱疾病,輒修洮陽、泥和城而置戍焉。文明太后崩,使人告凶,伏連籌拜命不恭。有司請伐之,孝文不許。群臣以其受詔不敬,不宜納所獻。帝曰:「拜受失禮,乃可加以詰責。所獻土毛,乃是臣之常道。杜棄所獻,便是絕之,縱欲改悔,其路無由矣。」詔曰:「朕在哀疚之中,未存征討。



 而去春枹罕表取其洮陽、泥和二戍,時以此既邊將之常,即便聽許。及偏師致討,二戍望風請降,執訊二千餘人,又得婦女九百口。子婦可悉還之。」伏連籌乃遣世子賀魯頭朝于京師。禮錫有加,拜伏連籌使持節、都督西垂諸軍事、征西將軍、領護西戎中郎將、西海郡開國公、吐谷渾王,麾旗章綬之飾,皆備給之。



 後遣兼員外散騎常侍張禮使於伏連籌。謂禮曰:「昔與宕昌通和,恒見稱大王,己則自有興動,殊違臣節。當發之日,宰輔以為君若返迷知罪,則克保蕃業;脫守愚
 不改,則禍難將至。」伏連籌遂默然。及孝文崩,遣使赴哀,盡其誠敬。



 伏連籌內修職貢,外並戎狄,塞表之中,號為強富。準擬天朝,樹置官司,稱制諸國,以自誇大。宣武初,詔責之曰:「梁州表送卿報宕昌書。梁彌邕與卿並為邊附,語其國則鄰籓,論其位則同列,而稱書為表,名報為旨。有司以國常刑,殷勤請討。朕慮險遠多虞,輕相構惑,故先宣此意,善自三思。」伏連籌上表自申,辭誠懇至。終宣武世至於正光,嫠牛、蜀馬及西南之珍,無歲不至。後秦州城人莫折念生反,河西路絕。涼州城人萬於菩提等東應念生,囚刺史宋穎。穎密遣求援於伏連籌,伏連籌
 親率大眾救之,遂獲保全。自爾以後,關徼不通,貢獻遂絕。



 伏連籌死,子夸呂立,始自號為可汗。居伏俟城,在青海西十五里。雖有城郭而不居,恒處穹廬,隨水草畜牧。其地,東西三千里,南北千餘里。官有王、公、僕射、尚書及郎中、將軍之號。夸呂椎髻毦珠,以皂為帽,坐金獅子床。號其妻為母尊,衣織成裙,披錦大袍,辮髮於後,首戴金花冠。



 其俗:丈夫衣服略同於華夏,多以羅冪為冠,亦以繒為帽;婦人皆貫珠貝,束髮,以多為貴。兵器有弓、刀甲、鞘。國無常賦,須則稅富室商人以充用焉。其刑罰:殺人及盜馬,死;餘則徵物以贖罪,亦量事決杖。刑人必以
 氈蒙頭,持石從高擊之。父兄死,妻後母及嫂等,與突厥俗同。至於婚,貧不能備財者,輒盜女去。



 死者亦皆埋殯,其服制,葬訖則除之。性貪婪,忍於殺害。好射獵,以肉酪為糧。



 亦知種田,有大麥、粟、豆。然其北界氣候多寒,唯得蕪青、大麥,故其俗貧多富少。青海周回千餘里,海內有小山。每冬冰合後,以良牝馬置此山,至來春收之,馬皆有孕,所生得駒,號為龍種,必多駿異。吐谷渾嘗得波斯草馬,放入海,因生驄駒,能日行千里,世傳青海驄者也。土出犛牛、馬、騾,多鸚鵡,饒銅、鐵、朱砂。地兼鄯善、且末。



 興和中,齊神武作相,招懷荒遠,蠕蠕既附於國,夸呂遣使
 致敬。神武喻以大義,徵其朝貢,夸呂乃遣使人趙吐骨真假道蠕蠕,頻來東魏。又薦其從妹,靜帝納以為嬪。遣員外散騎常侍傅靈檦使於其國。夸呂又請婚,乃以濟南王匡孫女為廣樂公主以妻之。此後朝貢不絕。



 西魏大統初,周文遣儀同潘濬喻以逆順之理,於是夸呂再遣使獻能舞馬及羊、牛等。然寇抄不已,緣邊多被其害。廢帝二年,周文勒大兵至姑臧,夸呂震懼,使貢方物。是歲,夸呂又通使於齊。涼州刺史史寧覘知其還,襲之於州西赤泉,獲其僕射乞伏觸狀、將軍翟潘密,商胡二百四十人,駝騾六百頭,雜彩絲絹以萬計。恭帝三年,史
 寧又與突厥木桿可汗襲擊夸呂,破之,虜其妻子,獲珍物及雜畜。武成初,夸呂復寇涼州,刺史是云寶戰沒。賀蘭祥、宇文貴率兵討之,夸呂遣其廣定王、鍾留王拒戰。祥等破之,廣定等遁走。又拔其洮陽、洪和二城,置洮州而還。保定中,夸呂前後三輩遣使獻方物。天和初,其龍涸王莫昌率來降,以其地為扶州。二年五月,復遣使來獻。建德五年,其國大亂,武帝詔皇太子征之。軍至伏俟城,夸呂遁走,虜其餘眾而還。明年,又再遣使奉獻。宣政初,其趙王他婁屯來降。自是,朝獻遂絕。



 及隋開皇初,侵弘州,地曠人梗,廢之。遣上柱國元諧率步騎數萬擊之。
 賊悉發國中,自曼頭至樹敦,甲騎不絕。其所署河西總管定城王鐘利房及其太子可博汗前後來拒戰,諧頻破之。夸呂大懼,率親兵遠遁,其名王十三人召率部落而降。上以其高寧王移茲裒素得眾心,拜大將軍,封河南王,以統降眾。自餘官賞各有差。



 未幾,復來寇邊,州刺史皮子信拒戰死之。汶州總管梁遠以銳卒擊之,乃奔退。俄而入寇廓州,州兵擊走之。



 夸呂在位百年,屢因喜怒廢殺太子。其後太子懼殺,遂謀執夸呂而降,請兵於邊吏。秦州總管河間王計應之,上不許。太子謀泄,為其父所殺。復立少子嵬王訶為太子。疊州刺史杜祭請因
 其釁討之,上又不許。六年,嵬王訶復懼父誅,謀歸國,請兵迎接。上謂其使者曰:「溥天之下,皆是朕臣妾,各為善事,即朕稱心。嵬王既有好意,欲來投服,唯教嵬王為臣子法,不可遠遣兵馬,助為惡事。」嵬王乃止。



 八年,其名王拓拔木彌請以千餘家歸化。上曰:「叛天背父,何可收納!又其本意,正自避死,若今違拒,又復不仁。若有音信,宜遣慰撫,任其自拔,不須出兵馬應接。其妹夫及甥欲來,亦任其意,不勞勸誘也。」是歲,河南王移茲裒死,文帝令其弟樹歸襲統其眾。平陳之後,夸呂大懼,逃遁險遠,不敢為寇。



 十一年,夸呂卒,子世伏使其兄子無素奉表稱
 籓,并獻方物,請以女備後庭。



 上謂無素曰:「若依來請,他國便當相學,一許一塞,是謂不平。若並許之,又非好法。」竟不許。十一年,遣刑部尚書宇文弼撫慰之。十六年,以光化公主妻世伏,上表稱公主為天后,上不許。



 明年,其國大亂,國人殺世伏,立其弟伏允為主。使陳廢立事,并謝專命罪,且請依俗尚主,上從之。自是朝貢歲至,而常訪國家消息,上甚惡之。煬帝即位,伏允遣子順來朝。時鐵勒犯塞,帝遣將軍馮孝慈出敦煌禦之,戰不利。鐵勒遣使謝罪請降,帝遣黃門侍郎裴矩慰撫之,諷令擊吐谷渾以自效。鐵勒即勒兵襲破吐谷渾,伏允東走,保西
 平境。帝復令觀德王雄出澆河,許公宇文述出西平掩之,大破其眾。



 伏允遁逃於山谷間,其故地皆空。自西平臨羌城以西,且末以東,祁連以南,雪山以北,東西四千里,南北二千里皆為隋有。置郡、縣、鎮、戍,發天下輕罪徙居之。



 於是留順不之遣。伏允無以自資,率其徒數千騎,客於黨項。帝立順為主,送出玉門,令統餘眾,以其大寶王泥洛周為輔。至西平,其部下殺洛周,順不果入而還。



 大業末,天下亂,伏允復其故地,屢寇河右,郡縣不能制。



 吐谷渾北有乙弗勿敵國,國有屈海,海周迴千餘里。眾有萬落,風俗與吐谷渾同。然不識五穀,唯食魚及蘇子。
 蘇子狀若中國枸巳子,或赤或黑。



 有契翰一部,風俗亦同,特多狼。



 白蘭山西北,又有可蘭國,風俗亦同。目不識五色,耳不聞五聲,是夷蠻戎狄之中醜類也。土無所出,直大養群畜,而戶落亦可萬餘人。頑弱不知鬥戰,忽見異人,舉國便走。性如野獸,體輕工走,逐不可得。



 白蘭西南二千五百里,隔大嶺,又度四十里海,有女王國。人庶萬餘落,風俗土著,宜桑麻,熟五穀,以女為王,故因號焉。譯使不至,其傳亦然。



 宕昌羌者,其先蓋三苗之胤。周時與庸、蜀、微、盧等八國從武王滅商。漢有先零、燒當等,世為邊患。其地東接中
 華,西通西域,南北數千里。姓別自為部落,酋帥皆有地分,不相統攝,宕昌即其一也。俗皆土著,居有屋宇。其屋,織犛牛尾及羖羊毛覆之。國無法令,又無徭賦。唯戰伐之時,乃相屯聚;不然,則各事生業,不相往來。皆衣裘褐,牧養犛牛、羊、豕以供其食。父子、伯叔、兄弟死者,即以繼母、世叔母及嫂、弟婦等為妻。俗無文字,但候草木榮落,記其歲時。三年一相聚,殺牛、羊以祭天。



 有梁勤者,世為酋帥,得羌豪心,乃自稱王焉。勤孫彌忽,太武初,遣子彌黃奉表求內附。太武嘉之,遣使拜彌忽為宕昌王,賜彌黃爵甘松侯。彌忽死,孫彪子立。其地自仇池以西,東西
 千里;席水以南,南北八百里。地多山阜,人二萬餘落。



 世修職貢,頗為吐谷渾所斷絕。彪子死,彌治立。彪子弟羊子先奔吐谷渾,遣兵送羊子,欲奪彌治位。彌治遣使請救,獻文詔武都鎮將宇文生救之,羊子退走。彌治死,子彌機立,遣其司馬利柱奉表貢方物。楊文度之叛,圍武都,彌機遣其二兄率眾救武都,破走文度。孝文時,遣使子橋表貢朱沙、雄黃、白石膽各一百斤。自此後,歲以為常,朝貢相繼。後孝文遣鴻臚劉歸、謁者張察拜彌機征南大將軍、西戎校尉、梁益二州牧、河南公、宕昌王。以助
 之。



 鄧至者,白水羌也,世為羌豪,因地名號,自稱鄧至。其地自亭街以東,平武以西,汶嶺以北,宕昌以南,士風習俗,亦與宕昌同。其王像舒治遣使內附,高祖拜龍驤將軍、鄧至王,遣貢不絕。周文命章武公導率兵送之。



 鄧至之西有赫羊國。初,其部內有一羊,形甚大,色至鮮赤,故因為國名。



 又有東亭衛、大赤水、寒宕、石河、薄陵、下習山、倉驤、覃水等諸羌國,風俗粗獷,與鄧至國不同焉。亦時遣貢使,朝廷納之,皆假之以雜號將軍,子、男、渠帥之名。



 白蘭者,羌之別種也。其地東北接吐谷渾,西北利摸徒,南界那鄂。風俗物產,與宕昌略同。周保定元年,遣使獻
 犀甲、鐵鎧。



 黨項羌者,三苗之後也。其種有宕昌、白狼,皆自稱獼猴種。東接臨洮、西平,西拒葉護,南北數千里,處山谷間。每姓別為部落,大者五千餘騎,小者千餘騎。



 織犛牛尾及醿惣毛為屋,服裘褐,披氈為上飾。俗尚武力,無法令,各為生業,有戰陣則屯聚,無徭役,不相往來。養犛牛、羊、豬以供食,不知稼穡。其俗淫穢蒸報,於諸夷中為甚。無文字,但候草木以記歲時。三年一聚會,殺牛羊以祭天。



 人年八十以上死者,以為令終,親戚不哭;少死者,則云夭枉,共悲哭之。有琵琶、橫吹,擊缶為節。



 魏、周之際,數來擾
 邊。隋文帝為丞相時,中原多故,因此大為寇掠。蔣公梁睿既平王謙,請因還師討之。開皇元年,有千餘家歸化。五年,拓拔寧叢等各率眾詣旭州內附,授大將軍,其部下各有等差。十六年,復寇會州,詔發隴西兵討之,大破其眾,人相率降,遣子弟入謝罪。帝謂曰:「還語爾父兄,人生須有定居,養老長幼。乃乍還乍走,不羞鄉里邪!」自是朝貢不絕。



 附國者,蜀郡西北二千餘里,即漢之西南夷也。有嘉良夷,即其東部,所居種姓自相率領,土俗與附國同,言語少殊。不統一,其人並無姓氏。



 附國王字宜繒。其國南北八
 百里,東西千五百里。無城柵,近川谷,傍山險。



 俗好復讎,故壘石為巢,以避其患。其巢高至十餘丈,下至五六丈,每級以木隔之,基方三四步,巢上方二三步,狀似浮圖。於下級開小門,從內上通,夜必關閉,以防賊盜。國有重罪者,罰牛。人皆輕捷,便擊劍。漆皮為牟甲,弓長六尺,竹為箭。



 妻其群母及嫂,兒弟死,父兄亦納其妻。好歌舞,鼓簧,吹長角。有死者,無服制,置屍高床之上,沐浴衣服,被以牟甲,覆以獸皮。子孫不哭,帶甲舞劍而呼云:「我父為鬼所取,我欲報冤殺鬼。」自餘親戚,哭三聲而止。婦人哭,必兩手掩面。



 死家殺牛,親屬以豬酒相遺,共飲敢而瘞之。
 死後一年,方始大葬,必集親賓,殺馬動至數十匹。立木為祖父神而事之。其俗以皮為帽,形圓如缽,或戴冪狖。衣多毼皮裘,全剝牛腳皮為靴。項系鐵鎖,手貫鐵釧。王與酋帥,金為首飾,胸前懸一金花,徑三寸。其土高,氣候涼,多風少雨,宜小麥、青稞。山出金、銀、銅、多白雉。水有嘉魚,長四尺而鱗細。



 大業四年,其王遣使素福等八人入朝。明年,又遣其弟子宜林率嘉良夷六十人朝貢。欲獻良馬,以路險不通。請開山道,修職貢物,煬帝以勞人不許。



 嘉良有水闊六七十丈,附國有水闊百餘丈,並南流。用皮為舟而濟。



 附國南有薄緣夷,風俗亦同。西有女國。其
 東北連山綿亙數千里,接於黨項。



 往往有羌,大小左封、昔衛、葛延、白狗、向人、望族、林臺、舂桑、利豆、迷桑、婢藥、大硤、白蘭、北利摸徒、那鄂、當迷、渠步、桑悟、千碉,並在深山窮谷,無大君長。其風俗略同於黨項,或役屬吐谷渾,或附附國。大業中,朝貢。緣西南邊置諸道總管以管之。



 稽胡,一曰步落稽,蓋匈奴別種,劉元海五部之苗裔也。或云山戎赤狄之後。



 自離石以西,安定以東,方七八百里,居山谷間,種落繁熾。其俗土著,亦知種田,地少桑蠶,多衣麻布。其丈夫衣服及死亡殯葬,與中夏略同;婦人則多貫蜃貝以為耳頸飾。與華人錯居。其渠帥頗識文
 字,言語類夷狄,因譯乃通。蹲踞無禮,貪而忍害。俗好淫穢,女尤甚,將嫁之夕,方與淫者敘離,夫氏聞之,以多為貴。既嫁,頗亦防閑,有犯姦者,隨事懲罰。又兄弟死者,皆納其妻。雖分統郡縣,列於編戶,然輕其徭賦,有異華人。山谷阻深者,又未盡役屬,而兇悍恃險,數為寇。



 魏孝昌中,有劉蠡升者,居雲陽谷,自稱天子,立年號,署百官。屬魏氏亂,力不能討。蠡升遂分遣部眾抄掠,汾、晉之間,略無寧歲。神武遷鄴後,始密圖之,乃偽許以女妻蠡升太子。蠡升遂遣子詣鄴,齊神武厚禮之,緩以婚期。蠡升既恃和親,不為之備。魏大統元年三月,齊神武襲之,蠡升
 率輕騎出外征兵,為其北部王所殺,送於神武。其眾復立蠡升第三子南海王為主,神武滅之,獲其偽主及弟西海王並皇后、夫人、王公以下四百餘人,歸於鄴。



 居河西者,多恃險不賓。時周文方與神武爭衡,未遑經略,乃遣黃門侍郎楊就安撫之。五年,黑水部眾先叛。七年,別帥夏州刺史劉平伏又據上郡反。自是北山諸部,連歲寇暴。周文前後遣于謹、侯莫陳崇、李弼等相繼討平之。



 武成初,延州稽胡郝阿保、狼皮率其種人,附於齊氏。阿保自署丞相,狼皮自署柱國,并與其別部劉桑德共為影響。柱國豆盧寧督諸軍擊破之。二年,狼皮等餘黨
 復叛,詔大將軍韓果討破之。



 保定中,離石生胡數寇汾北,勛州刺史韋孝寬於險要築城,置兵糧,以遏其路。



 及楊忠與突厥伐齊,稽胡等便懷旅拒,不供糧餼。忠乃詐其酋帥,云與突厥迴兵討之,酋帥等懼,乃相率供饋焉。其後丹州、綏州等部內諸胡,與蒲川別帥郝三郎等又頻年逆命,復詔達奚震、辛威、於寔等前後窮討,散其種落。天和二年,延州總管宇文盛率眾城銀川,稽胡白郁久同、喬是羅等欲邀襲,盛並討斬之。又破其別帥喬三勿同等。五年,開府劉雄出綏州,巡檢北邊川路。稽胡帥白郎、喬素勿同等度河逆戰,雄復破之。



 建德五年,武
 帝敗齊帥於晉州,乘勝逐北,齊人所棄甲仗,未暇收斂,稽胡乘間竊出,並盜而有之。乃立蠡升孫沒鐸為主,號聖武皇帝,年曰石平。六年,武帝定東夏,將討之,議欲窮其巢穴。齊王憲以為種類既多,又山谷阻絕,王師一舉,未可盡除,且當翦其魁帥,餘加慰撫。帝然之,乃以憲為行軍元帥,督行軍總管趙王招、譙王儉、滕王逌等討之。憲軍次馬邑,乃分道俱進。沒鐸遣其黨天柱守河東,又遣其大帥穆支據河西,規欲分守險要,掎角憲軍。憲命譙王儉擊破之,斬獲千餘級。趙王招又擒沒鐸,眾盡降。宣政元年,汾胡帥劉受羅千復反,越王盛督諸軍討禽
 之。自是寇盜頗息。



 論曰:氐、羌、吐谷渾等曰殊俗,別處邊陲,考之前代,屢經叛服,窺覘首鼠,蓋其本性。夫無德則叛,有道則伏,先王所述荒服也。



\end{pinyinscope}