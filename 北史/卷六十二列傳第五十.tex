\article{卷六十二列傳第五十}

\begin{pinyinscope}

 王羆孫長述王思政尉遲迥弟綱綱子運王軌樂運王羆,字熊羆,京兆霸城人,漢河南尹遵之後,世為州郡著姓。羆質直木彊,處物平當,州閭敬憚之。魏太和中,除殿中將軍,稍遷雍州別駕,清廉疾惡,勵精公事。刺史崔亮有知人之鑒,見羆雅相欽挹。亮後轉定州,啟羆為長史。執政者恐羆不稱,不許。及梁人寇硤石,亮為都督南
 討,復啟羆為長史,帶銳軍。朝廷以亮頻舉羆,故當可用。及剋硤石,羆功居多。先是南岐、東益氐羌反叛,乃拜羆冠軍將軍,鎮梁州,討平諸賊。還,授西河內史,辭不拜。時人謂曰;』西河大邦,奉祿優厚,何為致辭?」羆曰:「京洛材木,盡出西河,朝貴營第宅者,皆有求假。



 如其私辦,則力所不堪,若科發人間,又違犯憲法。以此致辭耳。」



 後以軍功封定陽子,除荊州刺史。梁復遣曹義宗圍荊州,堰水灌城,不沒者數版。時既內外多虞,未遑救援,乃遺羆鐵券,雲城全當授本州刺史。城中糧盡,羆乃煮粥與將士均分食之。每出戰,常不擐甲胄,大呼告天曰:「荊州城,孝文
 皇帝所置。天若不祐國家,使箭中王羆額;不爾,王羆須破賊。」屢經戰陣,亦不被傷。



 彌歷三年,義宗方退。進封霸城縣公。元顥入洛,以羆為左軍大都督。顥敗,莊帝以羆受顥官,故不得本州,更除岐州刺史。



 時南秦數叛,以羆行南秦州事。羆至州,召其魁帥為腹心,擊捕反者略盡。乃謂魁帥等曰:「汝黨皆死盡,何用活為!」乃以次斬之。自是南秦無復反者。又詔羆行秦州事。尋遷涇州刺史。未及之部,屬周文帝徵兵為勤王之舉,羆請前驅效命,遂為大都督,鎮華州。孝武西遷,進車騎大將軍、儀同三司,別封萬年縣伯,乃除華州刺史。齊神武率軍進潼關,人懷
 危懼,羆勸勱交士,眾心乃安。神武退,拜驃騎大將軍,加侍中、開府。嘗修州城未畢,梯在城外。神武遣韓軌、司馬子如從河東宵濟襲羆,羆不覺。比曉,軌眾已乘梯入城。羆尚臥未起,聞閣外洶洶有聲,便袒身露髻徒跣,持一白棒,大呼而出,謂曰:「老羆當道臥,貉子那得過!」敵見,驚退。逐至東門,左右稍集,合戰破之。軌遂投城遁走。文帝聞而壯之。時關中大饑,征稅人間穀食,以供軍費。或隱匿者,令遞相告,多被篣捶,以是人有逃散。



 唯羆信著於人,莫有隱者,得粟不少諸州,而無怨讟。沙苑之役,神武士馬甚盛。



 文帝以華州衝要,遣使勞羆,令加守備。及神
 武至城下,謂羆曰;「何不早降?」



 羆乃大呼曰:「此城是王羆家,死生在此,欲死者來!」神武不敢攻。



 後移鎮河東,以前後功進爵扶風郡公。河橋之戰,王師不利,趙青雀據長安城,所在莫有固志。羆乃大開州門,召城中戰士謂曰:「如聞天子敗績,不知吉凶,諸人相驚,咸有異望。王羆受委於此,以死報恩。諸人若有異圖,可來見殺。必恐城陷沒者,亦任出城。如有忠誠,能與王羆同心,可共固守。」軍人見其誠信,皆無異心。及軍還,徵拜雍州刺史。是蠕蠕度河南寇,候騎已至豳州。朝廷慮其深入,乃徵發士馬,屯守京城,塹諸街巷,以備侵軼。右僕射周惠達召羆議
 之。羆不應命,臥而不起,謂其使曰:「若蠕蠕至渭北者,王羆率鄉里自破之,不煩國家兵。何為天子城中,遂作如此驚動!由周家小兒恇怯致此。」羆輕侮權貴,守正不回,皆此類也。未幾,還鎮河東。



 羆性儉率,不事邊幅。嘗有臺使至,羆為設食,使乃裂去薄餅緣。羆曰:「耕種收獲,其功已深,舂爨造成,用力不少,爾之選擇,當是未飢。」命左右撤去之。使者愕然大慚。又客與羆食瓜,客削瓜皮,侵肉稍厚,羆意嫌之。及瓜皮落地,乃引手就地取而食之。客甚愧色。性又嚴急,嘗有吏挾私陳事者,羆不暇命捶撲,乃手自取鞾履,持以擊之。每至享會,自秤量酒肉,分給
 將士。時人尚其均平,嗤其鄙碎。羆舉動率情,不為巧詐,凡所經處,雖無當時功迹,咸去乃見思。卒于官,贈太尉、都督、相冀等十州刺史,謚曰忠。



 羆安於貧素,不營生業,後雖貴顯,鄉里舊宅,不改衡門,身死之日,家甚貧罄,當時伏其清潔。



 子慶遠,弱冠以功臣子拜直閣將軍,先羆卒。孫述。



 述字長述。少孤為祖羆所養。聰敏有識度。年八歲,周文帝見而奇之曰:「王公有此孫,足為不朽。」解褐員外散騎侍郎,封長安縣伯。羆薨,居喪過禮,有詔褒之。免喪,襲封扶風郡公。除中書舍人,修起居注,改封龍門郡公。周受
 禪,拜賓部下大夫。累遷廣州刺史,甚有威惠。朝議嘉之,就拜大將軍。後歷襄、仁二州總管,並有能名。隋文帝為丞相,授信州總管,位上大將軍。王謙作亂,遣使致書於長述。因執其使,上書,又陳取謙策。上大悅,前後賜金五百兩,授行軍總管,討謙。以功進位柱國。開皇初,獻平陳計,修營戰艦,為上流之師。上善其能,頻加賞勞。後數歲,以行軍總管擊南寧,未至而卒。上甚傷惜之。贈上柱國、冀州刺史,謚曰莊。



 子謨嗣。謨弟軌,大業末郡守。少子文楷,起部郎。



 王思政,太原祁人,漢司徒允之後也。自魏太尉凌誅後,
 冠冕遂絕。父祐,州主簿。思政容貌魁梧,有籌策,解褐員外散騎侍郎。屬萬俟醜奴、宿勤明達等擾亂關右,北海王顥討之,聞思政壯健,啟與隨軍,所有謀議,並與參詳。時孝武在籓,素聞其名,乃引為賓客,遇之甚厚。及登大位,委以心膂。預定策功,封祁縣侯,為武衛將軍。俄而齊神武潛有異圖,帝以思政可任大事,拜使持節、中軍大將軍、大都督,總宿衛兵。思政乃言於帝曰:「洛陽四面受敵,非用武之地。關中有崤函之固,且士馬精彊。宇文夏州糾合同盟,願立功效若聞車駕西幸,必當奔走奉迎。



 藉天府之資,因已成之業,二年修復舊京,何慮不克。」帝
 深然之。及神武兵至河北,帝乃西遷。進爵太原郡公,拜光祿卿、并州刺史,加散騎常侍、大都督。



 大統之後,思政雖被任委,自以非相府之舊,每不自安。周文帝曾在同州,與群公宴集,出錦罽及雜綾絹數千段,令諸將摴蒲取之。物盡,周文又解所服金帶,令諸人遍擲,曰:「先得盧者即與之。」群公擲將遍,莫有得者。次至思政,乃斂容跪而誓曰:「王思政羈旅歸朝,蒙宰相國士之遇,方願盡心效命,上報知已。若此誠有實,令宰相賜知者,願擲即為盧;若內懷不盡,神靈亦當明之,使不作也,便當殺身以謝所奉。」辭氣慷慨,一座盡驚。即拔所佩刀,橫於膝上,攬
 摴蒲,拊髀擲之。比周文止之,已擲為盧矣。徐乃拜而受帶。自此朝寄更深。



 及河橋之戰,思政下馬,用長槊左右橫擊,一擊踣數人。時陷陣既深,從者死盡,思政被重創悶絕。會日暮,敵亦收軍。思政久經軍旅,戰唯著破衣弊甲,敵人疑非將帥,故得免。有帳下督雷五安於戰處哭求思政,會已蘇,遂相得。乃割衣裹創,扶思政上馬,夜久方得還軍。仍鎮弘農,除侍中、東道行臺。思政以玉壁地險要,請築城。即自營度,移鎮之。管汾晉并三州諸軍事、并州刺史、行臺如故,仍鎮玉壁。八年,東魏復來寇,卒不能克。以全城功,授驃騎大將軍、開府儀同三司。



 高仲
 密以北豫州來附,周文親接援之,乃驛召思政,將鎮成皋。未至而班師,復命思政鎮弘農。思政入弘農,令開城門,解衣而臥,慰勉將士,示不足畏。數日後,東魏將劉豐生率數千騎至城下,憚之,不敢進,乃引軍還。於是修城郭,起樓櫓,營田農,積芻秣,凡可以守禦者皆具焉。弘農之有備,自思政始也。



 十二年,加特進,兼尚書左僕射、行臺、都督、荊州刺史。境內卑濕,城塹多壞。思政乃命都督藺小歡督工匠繕修之。掘得黃金三十斤,夜中密送。至旦,思政召佐史,以金示之曰:「人臣不宜有私。」悉封金送上。周文嘉之,賜錢二十萬。



 思政之去玉壁也,周文命舉
 代人,思政乃進所部都督韋孝寬。其後東魏來寇,孝寬卒能全城,時論稱其知人。



 十三年,侯景叛東魏,請援乞師。當時未即應接。思政以為若不因機進取,後悔無及,即率荊州步騎萬餘,從魯關向陽翟。周文聞思政已發,乃遣太尉李弼赴潁川。東魏將高岳等聞大軍至,收軍而遁。思政入守潁川。景引兵向豫州,外稱略地,乃密遣送款於梁。先是,周文遣帥都督賀蘭願德助景扞禦,景既有異圖,因厚撫願德等,冀為己用。思政知景詭詐,乃密追願德。思政分布諸軍,據景七州十二鎮。



 周文乃以所授景使持節、太傅、大將軍,兼尚書令、河南大行臺、河
 南諸軍事,回授思政,思政並讓不受。頻使敦喻,唯受河南諸軍事。



 十四年,拜大將軍。九月,東魏太尉高岳、行臺慕容紹宗、儀同劉豐生等率步騎十萬來攻潁川,殺傷甚眾。岳又築土山以臨城中,飛梯火車,盡攻擊之法。思政亦作火,因迅風便投之土山。又射以火箭,燒其攻具。仍募勇士,縋而出戰,據其兩土山,置樓堞以助防守。齊文襄更益兵,堰洧水以灌城。時雖有怪獸,每衝壞其堰。然城被灌已久,多亦崩頹。岳悉眾苦攻。思政身當矢石,與士卒同勞苦。岳乃更修堰,作鐵龍雜獸,用厭水神。堰成,水大至。城中泉涌溢,懸釜而炊,糧力俱竭。慕容
 紹宗、劉豐生及其將慕容永珍意以為閑,共乘樓船以望城內,令善射人俯射城中。俄而大風暴起,船乃飄至城下。城上人以長鉤牽船,弓弩亂發。紹宗竊急,透水而死。豐生浮向土山,復中矢而斃。禽永珍,并獲船中器械。思政謂永珍曰:「僕之破亡,在於晷漏。誠知殺卿無益,然人臣之節,守之以死。」乃流涕斬之。



 並收紹宗等尸,以禮埋瘞。



 岳既失紹宗等,志氣沮喪,不敢逼城。齊文襄聞之,乃率步騎十萬來攻。思政知不濟,率左右據土山,因仰天大哭,左右皆號慟。思政西向再拜,便欲自剄。先是,文襄告城中人曰:「有能生致王大將軍者,封侯重賞。若大
 將軍身有損傷,親近左右皆從大戮。」都督駱訓固止之,不得引決。齊文襄遣其通直散騎常侍趙彥深,就土山遺以白羽扇而說之,牽手以下。引見文襄,辭氣慷慨,涕淚交流,無撓屈之容。文襄以其忠於所事,起而禮之,接遇甚厚。其督將分禁諸州地牢。數年盡死。



 思政初入潁川,士卒八千人。被圍既久,城中無鹽,腫死者十六七,及城陷之日,存者纔三千人。雖外無救援,遂無叛者。思政常以勤王為務,不營資產。嘗被賜園地,思政出征後,家人種桑果雜樹。及還,見而怒曰:「匈奴未滅,去病辭家,況大賊未平,欲事產業,豈所謂憂公忘私邪!」命左右拔而
 棄之。故身陷之後,家無蓄積。及齊文宣受東魏禪,以思政為都官尚書、儀同三司。卒,贈以本官,加兗州刺史。



 初,思政在荊州,自武關以南延袤一千五百里,置三十餘城,並當衝要之地。



 凡所舉薦,咸得其才。



 子康,沈毅有度量,後為周文親信。思政陷後,詔以因水城陷,非戰之罪,增邑三千五百戶,以康襲爵太原公,除驃騎大將軍、侍中、開府儀同三司。康弟揆,先封中都縣侯,增邑通前一千五百戶,進爵為公。揆弟邗,封西安縣侯。邗弟恭,忠誠縣伯。恭弟細,顯親縣伯。康姊封齊郡君。康兄元遜亦陷於潁川,封其子景晉陽縣侯。康抗表固讓,不許。十六年,
 王師東討,加康使持節、大都督,以思政所部兵皆配之。魏廢帝二年,隨尉遲迥征蜀,鎮天水郡。尋賜姓拓王氏。為鄜州刺史。



 武成末,除匠師中大夫,轉載師。保定二年,歷安、襄二州總管,位柱國。入隋,終於汴州刺史。



 尉遲迥,字薄居羅,代人也。其先,魏之別種,號尉遲部,因而氏焉。父俟兜,性弘裕有鑒識,尚周文帝姊昌樂大長公主,生迥及綱。迥年七歲,綱年六歲,俟兜病且卒,呼二子,撫其首曰:「汝等並有貴相,但恨吾不見耳,各勉之。」武成初,追贈柱國大將軍、太傅、長樂郡公,謚曰定。迥少聰敏,美容儀。及長,有大志,好施愛士。尚魏文帝女金明公主,
 拜駙馬都尉,封西都侯。大統十一年,拜侍中、驃騎大將軍、開府儀同三司,進爵魏安郡公。十五年,遷尚書左僕射,兼領軍將軍。



 迥通敏有幹能,雖任兼文武,頗允時望,周文以此深委仗焉。十六年,拜大將軍。



 侯景之渡江也,梁元帝時鎮江陵,請修鄰好。其弟武陵王紀在蜀稱帝,率眾東下,將攻之。梁元帝大懼,移書請救。周文曰:「蜀可圖矣!取蜀制梁,在茲一舉。」



 乃與群公會議,諸將多有異同。唯迥以為紀既盡銳東下,蜀必空虛,王師臨之,必有征無戰。周文以為然,謂曰:「伐蜀之事,一以委汝。」於是令迥督開府元珍、乙弗亞、侯呂陵始、叱奴興、綦連雄、宇文
 升等六軍甲士取晉壽,開平林舊道。迥前軍臨劍閣,紀安州刺史樂廣以州先降。紀梁州刺史楊乾運時鎮潼水,先已遣使詣闕,密送誠款,然恐其下不從,猶據潼水別營拒守。迥遣元珍、侯呂陵始等襲之,乾運還保潼川。珍等遂圍之,乾運降。迥至潼川,大餉將士,度涪江,至青溪,登南原,勒兵講武,修繕約束,閱器械,自開府以下賞金帛各有差。時夏中連雨,山路險峻,將士疲病者十二三,迥親自勞問,加以湯藥,引之而西。紀益州刺史蕭捴嬰城自守,進軍圍之。初,紀至巴郡,遣前南梁州刺史史欣景、幽州刺史趙拔扈等為捴外援。迥分遣元珍、乙弗
 亞等擊破之。拔扈等遁走,欣景遂降。捴被圍五旬,頻戰為迥所破。遣使乞降,許之。



 捴乃與紀子宜都王圓肅率其文武詣軍門請見,迥以禮接之。其吏人等各令復業,唯收僮隸及儲積以賞將士。號令嚴肅,軍無私焉。詔以迥為大都督、益潼等十二州諸軍事、益州刺史。三年,加督六州,通前十八州諸軍事。以平蜀功,封一子安固郡公。自劍閣以南得承制封拜及黜陟。迥乃明賞罰,布恩威,綏輯新邦,經略未附,人夷懷而歸之。



 性至孝,色養不怠,身雖在外,所得四時甘脆,必先薦奉,然後敢嘗。大長公主年高多病,迥往在京師,每退朝參候起居,憂悴形
 於容色。大長公主每為之和顏進食,以寧迥心。周文知其至性,徵迥入朝,以慰其母意。遣大鴻臚郊勞,仍賜迥袞冕之服。蜀人思之,為立碑頌德。六官初建,拜小宗伯。



 周孝閔帝踐阼,進位柱國大將軍,以迥有平蜀功,同霍去病冠軍之義,改封寧蜀公。遷大司馬。尋以本官鎮隴右。武成元年,進封蜀國公,邑萬戶,除秦州總管、秦渭等十四州諸軍事、隴右大都督。保定二年,拜大司馬。及晉公護東伐,迥帥師攻洛陽。齊王憲等軍於芒山,齊眾度河,諸軍驚散。迥率麾下反行卻敵,於是諸將遂得全師而還。遷太保、太傅。建德初,拜太師,尋加上柱國。宣帝即
 位,以迥為大右弼,轉大前疑,出為相州總管。宣帝崩,隋文帝輔政,以迥位望宿重,懼為異圖,乃令迥子魏安郡公惇齎詔書以會葬征迥。尋以鄖國公韋孝寬代迥為總管。迥以隋文帝當權,將圖篡奪,遂謀舉兵,留惇而不受代。隋文帝又令候正破六韓裒詣迥喻旨,密與總管府長史晉昶等書,令為之備。迥聞之,殺昶,集文武士庶等登城北樓而令之。於是眾咸從命,莫不感激。乃自稱大總管,承制署官司。于時趙王招已入朝,留少子在國,迥又奉以號令。迥弟子大將軍、成平郡公勤時為青州總管,初得迥書表送之,尋亦從迥。迥所管相、衛、黎、毛、洺、
 貝、趙、冀、瀛、滄,勤所統青、齊、膠、光、莒諸州皆從之,眾數十萬。滎州刺史邵國公宇文胄、申州刺史李惠、東楚州刺史費也利進國、東潼州刺史曹孝達各據州以應迥。徐州總管司錄席毗與前東平郡守畢義緒據兗州及徐州之蘭陵郡,亦以應迥。永橋鎮將紇豆陵惠以城降迥。迥又北結高寶寧以通突厥;南連陳人,許割江淮之地。



 隋文帝於是徵兵討迥,即以韋孝寬為元帥,陰羅雲監諸軍,郕國公梁士彥、樂安公元諧、化政公宇文忻、濮陽公宇文述、武鄉公崔弘度、清河公楊素、隴西公李詢、延壽公于仲文等皆為行軍總管。迥遣所署大將軍石愻
 攻建州,刺史宇文弁以州降愻。迥又遣西道行臺韓長業攻陷潞州,執刺史趙威,署城人郭子勝為刺史。上儀同赫連士猷攻晉州,即據小鄉城。紇豆陵惠襲陷定州之鉅鹿郡,遂圍恆州。上大將軍宇文威攻汴州,上開府莒州刺史烏丸尼、開府尉遲俊率膠、光、青、齊、莒、兗之眾圍沂州。大將軍檀讓攻陷曹、亳二州,屯兵梁郡。大將軍、東南道行臺席毗眾號八萬,軍於籓城,攻陷昌慮、下邑、豐縣。李惠自申州攻永州,焚之而還。宇文胄軍於洛口。開府梁子康攻懷州。



 魏安公惇率眾十萬人入武德,軍於沁東。孝寬等諸軍隔水,相持不進。隋文帝又遣高熲
 馳驛督戰。惇布兵二十餘里,麾軍小卻,欲待孝寬軍半度而擊之。孝寬因其卻,乃鳴鼓齊進,惇遂大敗。孝寬乘勝進至鄴,迥與其子惇、祐等又悉其卒十三萬,陣於城南。迥別統萬人,皆綠巾錦襖,號曰黃龍兵。勤率眾五萬自青州赴迥,以三千騎先到。迥舊集軍旅,雖老,猶被甲臨陣。其麾下兵皆關中人,為之力戰。



 孝寬等軍失利而卻。鄴中士女觀者如堵。高熲與李詢乃整陣先犯觀者,因其擾而乘之。迥眾大敗,遂入鄴城。迥走保北城,孝寬縱兵圍之。李詢、賀婁子乾以其屬先登。迥上樓,射殺數人,乃自殺。勤、惇、祐等東走青州,未至,開府郭衍追及之,
 並為衍所獲。隋文帝以勤初有誠款,特釋之。李惠先是自縛歸罪,隋文帝復其官爵。



 迥末年衰耄,惑於後妻王氏,而諸子多不睦。及起兵,以開府、小御正崔達拏為長史,自餘委任,亦多用齊人。達拏文士,無籌略,舉措多失綱紀,不能匡救。



 迥自起兵至於敗,凡經六十八日焉。



 子寬,大將軍、長樂郡公,先迥卒。寬兄誼,開府、資中郡公。寬弟順,以迥平蜀功,授開府、安固郡公。後以女為宣帝皇后,拜上柱國,封胙國公。順弟惇,軍正下大夫、魏安郡公。惇弟祐耆,西都郡公。皆被誅,而誼等諸子以年幼,並獲全。



 武德中,迥從孫庫部員外郎耆福上表請改葬。朝議以
 迥忠於周室,有詔許焉,仍贈絹百匹。迥弟綱。



 綱字婆羅,少孤,與兄迥依託舅氏。周文帝西討關隴,迥、綱與母昌樂大長公主留于晉陽。後方入關。從周文征伐,常陪侍帷幄,出入臥內。以軍功封廣宗縣伯。



 綱驍果有膂力,善騎射,周文甚寵之,委以心膂。河橋之戰,周文馬中流矢,因而驚奔。綱與李穆等左右力戰,眾皆披靡,文帝方得乘馬。大統十四年,進爵平昌郡公。廢帝二年,拜大將軍,兼領軍。及魏帝有異謀,言頗漏泄。周文以綱職典禁旅,使密為之備。俄而廢帝立齊王,仍以綱為中領軍,總宿衛事。



 綱兄迥伐蜀,從周文送之於城西,見一
 走兔,周文命綱射之。誓曰:「若獲此兔,必當破蜀。」俄而綱獲兔而返。周文喜曰:「事平,當賞汝佳口。」及克蜀,賜綱侍婢二人。又嘗從周文北狩雲陽,見五鹿俱走,綱獲其三。每從遊宴,周文以珍異之物令諸功臣射而取之,綱所獲輒多。



 周孝閔帝踐阼,綱以親戚掌禁兵,除小司馬。又與晉公護廢帝。明帝即位,進位柱國大將軍。武成元年,進封吳國公,邑萬戶,除涇州總管。歷位少傅、大司空、陜州總管。晉公護東討,乃配綱甲士,留鎮京師。大軍還,綱復歸。天和二年,以綱政績可紀,賜帛及錢穀等,增邑,以褒賞之。陳公純等以皇后阿史那氏自突厥將入塞,詔
 徵綱與大將軍王傑率眾迎衛於境首。三年,追論河橋功,封一子縣公。四年,薨于京師。贈太保,謚曰武。



 第二子安以嫡嗣。大象末,位柱國。入隋,歷鴻臚卿、左衛大將軍。安兄運。



 運少彊濟,志在立功。魏大統十六年,以父勳封安喜縣侯。周明帝立,以預定勛,進爵周城縣公。歷位隴州刺史,再遷左武伯中大夫,尋加軍司馬。運既職兼文武,甚見委任。進爵廣業郡公,轉右司衛。時宣帝在東宮,親狎諂佞,數有罪失。



 武帝於朝臣內選忠諒鯁正者以匡弼之,於是以運為右宮正。



 建德三年,帝幸雲陽宮,又令運
 以本官兼司武,與長孫覽輔皇太子居守。俄而衛刺王直作亂,率其黨襲肅章門。覽懼,走行在所。運時偶在門中,直兵奄至,不暇命左右,乃手自闔門。直黨與運爭門,斫傷運指,僅而得閉。直既不得入,乃縱火。運恐火盡,直黨得進,乃取宮中材木及床等以益火,更以膏油灌之,火轉熾。



 久之,直不得進,乃退。運率留守兵因其退以擊之,直大敗而走。是夜微運,宮中已不守矣。武帝嘉之,授大將軍,賜以直田宅、妓樂、金帛、車馬、什物等不可勝數。



 四年,出為同州刺史,同州、蒲津、潼關等六防諸軍事。帝將伐齊,召運參議,東夏底定,頗有力焉。五年,拜柱國,進
 爵盧國公。轉司武上大夫,總宿衛軍事。



 帝崩於雲陽宮,秘未發喪,運總侍衛兵還京師。



 宣帝即位,授上柱國。運之為宮正也,數進諫於帝。帝不納,反疏忌之。時運又與王軌、宇文孝伯等皆為武帝親待。軌屢言帝失於武帝,帝謂運預其事,愈更銜之。及軌被誅,運懼及於禍,尋而得出秦州總管。至州,猶懼不免,遂以憂薨於州。



 贈大後丞、七州諸軍事、秦州刺史,謚曰忠。子靖嗣。



 運弟勤,大象末,青州總管,起兵應伯迥。



 勤弟敬,尚明帝女河南公主,位儀同三司。



 王軌,太原祁人也,小名沙門。漢司徒允之後,世為州郡
 冠族。累葉仕魏,賜姓烏丸氏。父光,少雄武,有將帥才略。頻有戰功,周文帝遇之甚厚。位至驃騎大將軍、開府儀同三司、平原縣公。軌性質直,起家事輔城公。及武帝即位,累遷內史下大夫,遂處腹心之任。帝將誅晉公護,軌贊成其謀。建德初,轉內史中大夫,加授開府儀同三司,又拜上開府儀同大將軍,封上黃縣公,軍國之政,皆參預焉。



 從平并、鄴,以功進位上大將軍,進爵郯國公。



 及陳將吳明徹入寇呂梁,徐州總管梁士彥頻與戰不利,乃退保州城。明徹遂堰清水以灌之,列船艦於城下,以圖攻取。詔以軌為行軍總管,率諸軍赴救。軌潛於清水入淮
 口,多豎大木,以鐵鎖貫車輪,橫截水流,以斷其船路,方欲密決其堰以斃之。明徹知之,乃破堰遽退,冀乘決水以得入淮。比至清口,川流已闊,水勢亦衰,船並礙於車輪,不復得過。軌因率兵圍而蹙之。唯有騎將蕭摩訶以二十騎先走,得免。明徹及將士三萬餘人并器械輜重並就俘獲。陳之銳卒,於是殲焉。進位柱國,仍拜徐州總管。軌性嚴重,善謀略,兼有呂梁之捷,威振敵境。陳人甚憚之。



 宣帝之征吐谷渾也,武帝令軌與宇文孝伯並從,軍中進趣,皆委軌等,宣帝仰成而已。時宮尹鄭譯、王端並得幸於宣帝。宣帝軍中頗有失德,譯等皆預焉。軍還,
 軌等言之於武帝。武帝大怒,乃撻宣帝,除譯等名,仍加捶楚。宣帝因此大銜之。



 軌又嘗與小內史賀若弼言及此事,且言皇太子必不克負荷。弼深以為然,勸軌陳之。



 軌後因侍坐,乃白武帝言:「皇太子多涼德,恐不了陛下家事。愚臣暗短,不足以論是非。陛下恒以賀若弼有文武奇才,識度宏遠,而弼比再對臣,深以此事為慮。」



 武帝召弼問之。弼曰:「皇太子養德春宮,未聞有過。未審陛下何從得聞此言?」



 既退,軌誚弼曰:「平生言論,無所不道,今者乃爾翻覆!」弼曰:「此公之過也。



 皇太子國之儲副,豈易為言,事有差跌,便至滅門之禍。本謂公密臧否,何得
 遂至昌言?」軌默然久之,乃曰:「吾專心國家,遂不存私計。向者對眾,良實非宜。」



 其後軌因內宴上壽,又捋武帝鬚曰:「可愛好老公,但恨後嗣弱耳」!武帝深以為然。但漢王次長,又不才,此外諸子並幼,故不能用其說。



 及宣帝即位,追鄭譯等復為近侍。軌自知必及於禍,謂所親曰:「吾昔在先朝,實申社稷至計。今日之事,斷可知矣。此州控帶淮南,鄰接彊寇,欲為身計,易同反掌。但忠義之節,不可虧違。況荷先帝厚恩,每思以死自效,豈以獲罪於嗣主,便欲背德於先帝?止可於此待死,義不為他計。冀千載之後,知吾此心。」



 大象元年,帝使內史杜虔信就徐州
 殺軌。御正中大夫顏之儀切諫,帝不納,遂誅之。軌立朝忠恕,兼有大功,忽以無罪被戮,天下知與不知皆傷惜。



 時京兆郡丞樂運亦以直言數諫於帝。樂運,字承業,南陽淯陽人,晉尚書令廣之八世孫。祖文素,齊南郡守。父均,梁義陽郡守。運少好學,涉獵經史。年十五而江陵滅,隨例遷長安。其親屬等多被籍沒,運積年為人傭保,皆贖免之。事母及寡嫂甚謹,由是以孝聞。梁故都官郎瑯邪王澄美之,次其行事為孝義傳。性方直,未嘗求媚於人。臨淄公唐瑾薦之,自柱國府記室為露門學士。前後犯顏屢諫武帝,多被納用。建德二年,除
 萬年縣丞。抑挫豪右,號稱強直。武帝嘉之,特許通籍,事有不便於時者,令巨細奏聞。



 武帝嘗幸同州,召運赴行在所。既至,謂曰:「卿言太子如何人?」運曰:「中人也。」時齊王憲以下並在帝側,帝顧謂憲等曰:「百官佞我,皆云太子聰明睿智,唯運云中人,方驗運之忠直耳。」於是因問運中人之狀。運對曰:「班固以齊桓公為中人,管仲相之則霸,豎貂輔之則亂。可與為善,亦可與為惡也。」帝曰:「我知之矣。」遂妙選宮官以匡弼之。乃超拜運京兆郡丞。太子聞之,意甚不悅。



 及武帝崩,宣帝嗣位,葬訖,詔天下公除,帝及六宮,便議即吉。運上疏曰:「三年之喪,自天子達於
 庶人。先王制禮,安可誣之。禮:天子七月而葬,以候天下畢至。今葬期既促,事訖便除,文軌之內,奔赴未盡;鄰境遠聞,使猶未至。若以喪服受弔,不可既吉更凶;如以玄冠對使,未知此出何禮?進退無據,愚臣竊所未安。」書奏,帝不納。



 自是德政不修,數行赦宥。運又上疏曰:「臣謹按周官曰:『國君之過市,刑人赦。』此謂市者交利之所,君子無故不遊觀焉,則施惠以悅之也。尚書曰:『眚災肆赦。」此謂過誤為害,罪雖大,當緩赦之。謹尋經典,未有罪無輕重,溥天大赦之文。故管仲曰:『有赦者,奔馬之委轡;不赦者,痤疽之礪石。』又曰:『惠者,人之仇讎;法者,人之父母。』吳
 漢遺言,猶云『唯願無赦。』王符著論,亦云:『赦者非明世之所宜有。』大尊豈可數施非常之惠,以肆姦宄之惡乎。」帝亦不納,而昏暴滋甚。運乃輿櫬詣朝堂,陳帝八失:一曰:內史御正,職在弼諧,皆須參議,共理天下。大尊比來小大之事,多獨斷之。堯、舜至聖,尚資輔弼,況大尊未為聖主,而可專恣已心?凡諸刑罰爵賞,爰及軍國大事,請參諸宰輔,與眾共之。



 二曰:內作色荒,古人重誡。大尊初臨四海,德惠未洽,先搜天下美女,用實後宮,又詔儀同以上女,不許輒嫁。貴賤同怨,聲溢朝野。請姬媵非幸御者,放還本族。欲嫁之女,勿更禁之。



 三曰:天子未明求衣,日
 旰忘食,猶恐萬機不理,天下擁滯。大尊比來一入後宮,數日不出。所須聞奏,多附內豎。傳言失實,是非可懼。事由宦者,亡國之徵。



 請準高祖,居外聽政。



 四曰:變故易常,乃為政之大忌;淫刑酷罰,非致安之弘規。若罰無定刑,則天下皆懼;政無常法,則人無適從。豈有削嚴刑之詔未及半祀,便即遣改,更嚴前制?政令不定,乃至於此!今宿衛之官,有一夜不直者,罪至削除;因而逃亡者,遂便籍沒。此則大逆之罪,與杖十同科。雖為法愈嚴,恐人情愈散。一人心散,尚或不可止,若天下皆散,將如之何?請遵輕典,並依大律,則億兆之人,手足有所措矣。



 五曰:高
 祖折雕為朴,本欲傳之萬世。大尊朝夕趨庭,親承聖旨。豈有崩未踰年,而遽窮奢麗,成父之志,義豈然乎?請興造之制,務從卑儉,雕文刻鏤,一切勿營。



 六曰:都下之人,徭賦稍重。必是軍國之要,不敢憚勞。豈容朝夕徵求,唯供魚龍爛漫;士庶從役,只為俳優角抵?紛紛不已,財力俱竭,業業相顧,無復聊生。



 凡無益之事,請並停罷。



 七曰:近見有詔,上書字誤者即科其罪。假有忠讜之人,欲陳時事,尺有所短,文字非工,不密失身,義無假手,脫有舛謬,便迫嚴科。嬰徑尺之鱗,其事非易;下不諱之詔,猶懼未來。更加刑戮,能無鉗口?大尊縱不能採誹謗之言,無
 宜杜獻替之路。請停此詔,則天下幸甚。



 八曰:或桑穀生朝,殷王因之獲福,今玄象垂戒,此亦興周之祥。大尊雖滅膳撤縣,未盡銷譴之理。誠願諮諏善道,修布德政,解兆庶之慍,引萬方之罪。則天變可除,鼎業方固。大尊若不革茲八事,臣見周廟不血食矣。



 帝大怒,將戮之。內史元嚴諫,因而獲免。翌日,帝頗感悟,召運謂之曰:「朕昨夜思卿所奏,實是忠臣。先皇聖明,卿數有規諫;朕既昏暗,卿復能如此!」



 乃賜御食以賞之。朝之公卿,初見帝甚怒,莫不為運寒心。後見獲賞,又皆相賀,以為幸免獸口。



 內史鄭譯常以私事請託,運不之許,因此銜之。及隋文帝
 為丞相,譯為長史,遂左遷運為廣州滍陽令。開皇五年,轉毛州高唐令。頻歷二縣,並有聲績。運常願處一諫官,從容諷議,而性訐直,為人所排抵,遂不被任用。乃發憤錄夏、殷以來諫爭事,集而部之,凡六百三十九條,合四十一卷,名曰諫苑。奏上之。隋文帝覽而嘉焉。



 論曰:王羆剛峭有餘,弘雅未之聞也。情安儉率,志在公平。既而奮節危城,抗辭勍敵,梁人為之退舍,高氏不敢加兵。以此見稱,信非虛矣。至述不隕門風,亦足稱也。王思政驅馳有事之秋,慷慨功名之際。及乎策名霸府,作鎮潁川,設縈帶之險,修守禦之術,以一城之眾,抗傾國
 之師,率疲駘之兵,當勁勇之卒,猶能亟摧大敵,屢建奇功。忠節冠於本朝,義聲動於鄰聽。運窮事蹙,城陷身囚,壯志高風亦足奮於百世矣。尉遲迥地則舅甥,職惟台袞,沐恩累葉,荷眷一時,居形勝之地,受籓維之託,顛而不扶,憂責斯在?及主威云謝,鼎業將遷,九服移心,三靈改卜,遂能志存赴蹈,投袂稱兵。忠君之勤未宣,違天之禍便及。校其心,翟義、葛誕之儔歟。綱、運積宣王室,勤勞出內。觀其自致榮寵,豈唯恩澤而已乎?夫士之成名,其途不一,蓋有不待爵祿而貴,不因學藝而重者何?亦云忠孝而已。若乃竭力以奉其親者,人子之行也;致身以
 事其君者,人臣之節也。斯固彌綸三極,囊括百代。當宣帝之在東朝,兇德方兆,王軌志惟無諱,極議於骨肉之間,竟遇淫刑,以至夷滅。若斯人者,人或以為其不忠,則天下莫之信也。觀樂運之所以行已之節,其有古之遺直之風乎。



\end{pinyinscope}