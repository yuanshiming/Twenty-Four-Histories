\article{卷十五列傳第三 魏諸宗室}

\begin{pinyinscope}

 上
 谷公紇羅,神元皇帝之曾孫也。初從道武皇帝自
 獨孤
 如賀蘭部,與弟建勸賀蘭訥推道武為主。及道武即帝位,以援立功,與建同日賜爵為公。卒。



 子題,賜爵襄城公,後進爵為王。擊慕容麟於義臺,中流矢薨。帝以太醫令陰光為視療不盡術,伏法。子悉襲,降爵為襄城公,卒,贈襄城王。神元後又有建德公嬰文、真定侯陸,並仕太武,特獲封爵。



 武陵侯因、長樂王壽樂,並章帝之後也。因從道武平中原,以功封曲逆侯。太武時,改爵武陵。壽樂位選部尚書、南安王,改封長樂王。文成即位,壽樂有援立功,拜太宰、大都督中外諸軍、錄尚書事。矜功,與尚書令長孫渴侯爭權,並伏法。



 望都公頹,昭帝之後也。隨道武平中原,賜爵望都侯。太武以頹美儀容,進止可觀,使迎左昭儀於蠕蠕,進爵為公。卒。



 曲陽侯素延、順陽公郁、宜都王目辰,並桓帝之後也。



 素延以小統從道武征討諸部,初定并州,為刺史。道武之
 驚於柏肆也,並州守將封竇真為逆,素延斬之。時道武意欲撫悅新附,悔參合之誅,而素延殺戮過多,坐免官。中山平,拜幽州刺史,豪奢放逸,左遷上谷太守。後賜爵曲陽侯。時道武留心黃、老,欲以純風化俗;雖乘輿服御,皆去雕飾。素延奢侈過度,帝深銜之,積其過,因徵,坐賜死。



 郁少忠正亢直。文成時,位殿中尚書,賜爵順陽公。文成崩,乙渾專權,郁從順德門入,欲誅渾。渾窘怖,遂奉獻文臨朝。後復謀殺渾,為渾所誅。獻文錄郁忠正,追贈順陽王,謚曰簡。



 目辰,文成即位,歷侍中、尚書左僕射,封南平公。乙渾謀亂,目辰、順陽公謀殺之。事發,目辰逃免。獻
 文傳位,有定策勳。孝文即位,進爵宜都王,除雍州刺史,鎮長安。有罪,伏法,爵除。



 六修,穆帝長子也。少兇悖。穆帝五年,遣六修與輔相衛雄、范班及姬淡等救劉琨,帝躬統大兵為後繼。劉粲懼,突圍而走,殺傷甚眾。帝因大獵壽陽山,陳閱皮肉,山為變赤。穆帝少子比延有寵,欲以為後;六修出居新平城,而黜其母。六修有驊騮駿馬,日行五百里,穆帝欲取以給比延。後六修來朝,穆帝又命拜比延,六修不從。穆帝乃坐比延於己所乘步輦,使人導從出遊。六修望見,以為穆帝,謁伏路左;及至,乃是比延,慚怒而去。穆帝怒,伐
 之。帝軍不利,六修殺比延。帝改服微行人間,有賤婦人識帝,遂暴崩。桓帝子普根先守于外,聞難來赴,滅之。



 吉陽男比干、江夏公呂,並道武族弟也。比干以司衛監討白澗丁零有功,賜爵吉陽男。後為南道都將,戰沒。呂以軍功封江夏公,位外都大官,大見尊重。卒,贈江夏王,陪葬金陵。



 高涼王孤,平文皇帝之第四子也。多才藝,有志略。烈帝之前元年,國有內難,昭成如襄國。後烈帝臨崩,顧命迎立昭成。及崩,群臣咸以新有大故,昭成來未可果,宜立長君。次弟屈剛猛多變,不如孤之寬和柔順。於是大人
 梁蓋等殺屈,共推孤。不肯,乃自詣鄴奉迎,請身留為質,石季龍義而從之。昭成即王位,乃分國半部以與之。薨。



 子斤,失職懷怒,構寔君為逆,死於長安。道武時,以孤勳高,追封高涼王,謚曰神武。斤子真樂,頻有戰功,後襲祖封。明元初,改封平陽王。薨。



 子禮,襲本爵高涼王。薨,謚懿王。



 子那,襲爵,拜中都大官,驍猛善攻戰。正平初,坐事伏法。獻文即位,追那功,命子紇紹封。薨。



 子大曹,性愿直。孝文時,諸王非道武子孫者,例降爵為公。以大曹先世讓國功重,高祖真樂勛著前朝,改封太原郡公。卒,無子,國除。宣武又以大曹從兄子洪威紹。恭謙好學,為潁川太
 守,有政績。孝靜初,在潁川聚眾應西魏,齊神武遣將討平之。



 禮弟陵,太武賜爵襄邑男,進爵為子。卒。



 子瑰,位柔玄鎮司馬。瑰子鷙,字孔雀,孝文末,以軍功賜爵晉陽男。武泰元年,爾朱榮至河陰,殺戮朝士,時鷙與榮共登高塚,俯而觀之。自此後,與榮合。



 永安初,封華山王。莊帝既殺爾朱榮,從子兆為亂。帝欲率諸軍親討,而鷙與兆陰通,乃勸帝曰:「黃河萬仞,寧可卒度?」帝遂自安。及兆入殿,鷙又約止衛兵。



 帝見逼,京邑破,皆由鷙之謀。孝靜初,入為大司馬,加侍中。鷙容貌魁壯,腰帶十圍,有武藝。木訥少言,性方厚,每息直省闥,雖暑月不解衣冠。曾於侍中
 高岳之席,咸陽王坦恃力使酒,眾皆下之。坦謂鷙曰:「孔雀老武官,何因得王?」鷙答曰:「斬反人元示喜首,是以得之。」眾皆失色,鷙怡然如故。興和三年,薨,贈假黃鉞、尚書令、司徒公。



 子大器,襲爵。後與元瑾謀害齊文襄,見害。孤孫度,道武初,賜爵松滋侯,位比部尚書。卒。子乙斤,襲爵襄陽侯。獻文崇舊齒,拜外都大官,甚優重。卒。



 子平,字楚國,襲世爵松滋侯,以軍功賜艾陵男。卒。



 子萇,孝文時,襲爵松滋侯,例降侯,賜艾陵伯。萇性剛毅,雖有吉慶事,未嘗開口而笑。孝文遷都,萇以代尹留鎮,除懷朔鎮都大將。因別,賜萇酒,雖拜飲而顏色不泰。帝曰:「聞公一生不笑,
 今方隔山,當為朕笑。」竟不能得。帝曰:「五行之氣,偏有所不入;六合之間,亦何事不有!」左右見者,無不把腕大笑。



 宣武時,為北中郎將,帶河內太守。萇以河橋船絙路狹,不便行旅,又秋水泛漲,年常破壞,乃為船路。遂廣募空車從京出者,率令輸石一雙,累以為岸。橋闊,來往便利。近橋諸郡,無復勞擾,公私賴之。歷位度支尚書、侍中、雍州刺史。卒,謚曰成。萇中年以後,官位微達,乃自尊倨,閨門無禮,昆季不穆,性又貪虐,論者鄙之。



 萇子子華,字伏榮,襲爵。孝莊初,除齊州刺史。先是,州境數經反逆,邢杲之亂,人不自保。而子華撫集豪右,委之管籥,眾皆感悅,境
 內帖然。而性甚褊急,當其急也,口不擇言,手自捶擊。長史鄭子湛,子華親友也。見侮罵,遂即去之。



 子華雖自悔厲,終不能改。在官不為矯潔之行,凡有饋贈者,辭多受少,故人不厭其取。鞫獄訊囚,務加仁恕,齊人樹碑頌德。後除濟州刺史。爾朱兆之入洛也,齊州城人趙洛周逐刺史,丹楊王蕭贊表濟南太守房士達攝行州事。洛周子元顯先隨子華在濟州,邀路改表,請子華復為濟州刺史。子華母房氏曾就親人飲食,夜還,大吐,人以為中毒,母甚憂懼。子華遂掬吐盡啖之,其母乃安。尋以母憂還都。孝靜初,除南兗州刺史。弟子思通使關西,朝廷使
 右衛將軍郭瓊收之。子思謂瓊僕曰:「速可見殺,何為久執國士?」子華謂子思曰:「由汝粗疏,令我如此!」頭叩床,涕泣不自勝。子思以手捋鬚,顧謂子華曰:「君惡體氣。」尋與子思俱賜死於門下外省。



 子思,字眾念,性剛暴,恒以忠烈自許。元天穆當朝權,以親從薦為御史中尉。



 先是,兼尚書僕射元順奏,以尚書百揆之本,至於公事,不應為送御史。至子思,奏曰:案《御史令》文:「中尉督司百寮,書侍御史糾察禁內。」又云「中尉出行,車輻前驅,除道一里,王公百辟避路。」時經四帝,前後中尉二十許人,奉以周旋,未曾暫廢,府寺臺省並從此令。唯肅宗之世為臨洮
 舉哀,故兼尚書左僕射臣順不肯與名,又不送簿。故中尉臣酈道元舉而奏之,而順復啟云:「尚書百揆之本,令僕納言之貴,不宜下隸中尉,送名御史。」尋亦蒙敕,聽如其奏。從此迄今,使無準一。臣初上臺,具見其事,意欲申請決議,但以權兼斯,未宜便爾。日復一日,遂歷炎涼。



 去月朔旦,臺移尚書,索應朝名帳,而省稽留不送。尋復移催並主吏,忽為尚書郎中裴獻伯後注云:「案舊事,御史中尉逢臺郎於復道,中尉下車執板,郎中車上舉手禮之,以此而言,明非敵體。」臣既見此,深為怪愕,旅省二三,未解所以。



 正謂都省別被新式,改易高祖舊命,即遣移
 問,事何所依。又獲尚書郎中王元旭報:「出蔡氏《漢官》,似非穿鑿。」始知裴、王亦規壞典謨,兩人心欲自矯。



 臣案《漢書宣秉傳》云,詔徵秉為御史中丞,與司隸校尉、尚書令俱會殿廷,並專席而坐,京師號之為三獨坐。又尋《魏書崔琰傳》、晉文陽《傅嘏傳》,皆云既為中丞,百寮震悚。以此而言,則中丞不揖省郎,蓋已久矣。憲臺不屬都坐,亦非今日。又尋《職令》云:「朝會失時,即加彈糾。」則百官簿帳應送上臺,灼然明矣。又皇太子以下違犯憲則,皆得糾察,則令僕朝名宜付御史,又亦彰矣。不付名至,否臧何驗?臣順專執,未為平通;先朝曲遂,豈是正法!謹案尚書郎
 中臣裴獻伯、王元旭等望班士流,早參清宦,輕弄短札,斐然若斯,茍執異端,忽焉至此。



 此而不綱,將隳朝令。請以見事免獻伯等所居官,付法科處。尚書納言之本,令僕百揆之要,同彼浮虛,助茲乖失,宜明首從,節級其罪。



 詔曰:「國異政,不可據之古事。付司檢高祖舊格,推處得失以聞。」尋從子思奏,仍為元天穆所忿,遂停。元顥之敗,封安定縣子。孝靜時,位侍中而死。



 萇弟珍,字金雀,襲爵艾陵男。宣武時,曲事高肇,遂為帝寵暱。彭城王勰之死,珍率壯士害之。後卒於尚書左僕射。



 平弟長生,位游擊將軍,卒。孝莊時,以子天穆貴盛,贈司空。天穆性和厚,美
 形貌,射有能名。六鎮之亂,尚書令李崇、廣陽王深北討,天穆以太尉使勞諸軍。



 路出秀容,見爾朱榮,深相結託,約為兄弟。未幾,改授別將,赴秀容,為榮腹心,除并州刺史。及榮赴洛,天穆參其始謀。莊帝踐阼,除太尉,封上黨王,征赴京師。



 後增封,通前三萬戶。尋監國史,錄尚書事,開府,世襲並州刺史。



 初,杜洛周、鮮于修禮為寇,瀛、冀諸州人多避亂南向。幽州前北平府主簿河間邢杲擁率部曲,屯據鄚城,以拒洛周、葛榮,垂將三載。及廣陽王深等敗後,杲南度,居青州北海界。靈太后詔流人所在皆置,命屬郡縣,選豪右為守令以撫鎮之。



 時青州刺史元
 世俊表置新安郡,以杲為太守,未報。會臺申休簡授郡縣,以杲從子子瑤資廕居前,乃授河間太守。杲深恥恨,於是遂反。所在流人,先為土人陵忽,聞杲起逆,率來從之,旬朔之間,眾踰十萬。先是,河南人常笑河北人好食榆葉,故齊人號之為「沓榆賊」。杲東掠光州,盡海而還,又破都督李叔仁軍。詔天穆與齊神武討,大破之。杲乃請降,傳送京師斬之。



 時元顥乘虛陷滎陽。天穆聞莊帝北巡,自畢公壘北度,會車駕於河內。爾朱榮以天時炎熱,欲還師。天穆苦執不可,榮乃從之。莊帝還宮,加太宰、羽葆鼓吹,增邑通前七萬戶。



 天穆以疏屬,本無德望,憑藉
 爾朱,爵位隆極當時。熏灼朝野,王公已下每旦盈門;受納財貨,珍寶充積。而寬柔容物,不甚見忌於時。莊帝以其榮黨,外示優寵,詔天穆乘車馬出入大司馬門。天穆與榮相倚,榮黨以兄禮事之。世隆等雖榮子侄,位遇已重,天穆曾言其失,榮即加杖,其相親任如此。莊帝內畏惡之,與榮同時見殺。節閔初,贈丞相、柱國大將軍、雍州刺史、假黃鉞,謚曰武昭。子儼襲,美才貌,位都官尚書。及齊受禪,聞敕召,假病,遂怖而卒。



 西河公敦,平文帝之曾孫也。道武初,從征,名冠諸將。後從征中山,所向無前。明元時,拜中都大官。太武時,進爵
 西河公,寵遇彌篤。卒,子撥襲。



 司徒石,平文帝之玄孫也。有膽略。從太武南討,至瓜步山。位尚書令、雍州刺史,歷北部侍郎、華州刺史。



 武衛將軍謂,烈帝之第四子也。寬雅有將略。常從道武征討,有功,除武衛將軍。子烏真,膂力絕人,隨道武征伐,屢有戰功,官至鉅鹿太守。子興都,聰敏剛毅。文成時,為河間太守,賜爵樂城子。為政嚴猛,百姓憚之。獻文初,以子丕貴重,進爵樂城侯。謝老歸家,帝益禮之,賜几杖服物,致膳於第。其妻婁氏,為東陽王太妃。卒,追贈定州刺史、河間公,謚曰宣。



 子提,襲公侯爵。提弟丕,太武時從駕
 臨江,賜爵興平子。獻文即位,累遷侍中。丞相乙渾謀反,丕以奏聞,詔收渾誅之。遷尚書令,改封東陽公。孝文時,封東陽王,拜侍中、司徒公。丕子超生,車駕親幸其第。以執心不二,詔賜丕入八議,傳示子孫,犯至百,聽責數恕之。放其同籍丁口雜使役調,求受復除。若有姦邪人方便讒毀者,即加斬戮。尋遷太尉、錄尚書事。



 時淮南王佗、淮陽王尉元、河東王茍頹並以舊老見禮。每有大事,引入禁中,乘步挽杖於朝,進退相隨。丕、佗、元三人皆容貌壯偉,腰帶十圍,大耳秀眉,鬚鬢斑白,百寮觀瞻,莫不祗聳。唯茍頹小為短劣,姿望亦不逮之。孝文、文明太
 后重年敬舊,存問周渥。丕聲氣高郎,博記國事;饗宴之際,恆居坐端;必抗音大言,敘列既往成敗,帝后敬納焉。然謟事要人,驕侮輕賤,每見王睿、苻承祖,常傾身下之。時文明太后為王睿造宅,故亦為造甲第。第成,帝、后幸之,率百官文武饗宴焉。使尚書令王睿宣昭,賜丕金印一紐。太后親造勸戒歌辭以賜群官,丕上疏贊謝。太后令曰:「臣哉鄰哉!鄰哉臣哉!君則亡逸於上,臣則履冰於下。若能如此,太平豈難致乎!」及丕妻段氏卒,謚曰恭妃,又特賜丕金券。後例降王爵,封平陽郡公。求致仕,詔不許。



 及車駕南伐,丕與廣陵王羽留守京師,並加使持節。
 詔丕、羽曰:「留守非賢莫可。太尉年尊德重,位總阿衡。羽,朕之懿弟,溫柔明斷。故使二人留守京邑,授以二節,賞罰在手。其祗允成憲,以稱朕心。」丕對曰:「謹以死奉詔。」羽對曰:「太尉宜專節度,臣但可副貳而已。」帝曰:「老者之智,少者之決,汝何得辭也?」及帝還代,丕請作歌,詔許之。歌訖,帝曰:「公傾朕還車,故親歌述志。



 今經構已有次第,故暫還舊京,願後時亦同茲適。」乃詔丕等以移都之事,使各陳志。燕州刺史穆羆進曰:「今四方未平,謂可不移。臣聞黃帝都涿鹿,古昔聖王不必悉居中原。」帝曰:「黃帝以天下未定,故居於涿鹿。既定,亦遷於河南。」廣陵王羽曰:「臣思奉神規,光崇丕業,請決之卜筮。」帝曰:「昔
 軒轅請卜兆,龜焦,乃問天老,謂為善,遂從其言,終致昌吉。然則至人之量未然,審于龜矣。」



 帝又詔群臣曰:「昔平文皇帝棄背,昭成營居盛樂。道武神武應天,遷居平城。朕幸屬勝殘之運,故移宅中原。北人比及十年,使其徐移。朕自多積倉儲,不令窘乏。」



 前懷州刺史青龍、前秦州刺史呂受恩等仍守愚固,帝皆撫而答之,辭屈,退。



 帝又將北巡,丕遷太傅、錄尚書事,頻表固讓。詔斷表啟,就家拜授。及車駕發代,丕留守。詔在代之事,一委太傅;賜上所乘車馬,往來府省。



 丕雅愛本風,不達新式。至於變俗遷洛,改官制服,禁絕舊言,皆所不願。帝亦不逼之,但誘示大理,令其不
 生同異。至於衣冕已行,朱服列位,而丕猶常服,列在坐隅。晚乃稍加弁帶,而不能修飾容儀。帝以丕年衰體重,亦不彊責。及罷降非道武子孫及異姓王者,雖駁於公爵,而利享封邑,亦不快。



 帝南征,丕表乞少留,思更圖後舉。會司徒馮誕薨,詔六軍反旆。丕又以熙薨于代都,表求鑾駕親臨。詔曰:「今洛邑肇構,跂望成勞。開闢暨今,豈有以天子之重遠赴舅國之喪?朕縱欲為孝,其如大孝何!縱欲為義,其如大義何!天下至重,君臣道懸,豈宜茍相誘引,陷君不德。令僕已下,可付法官貶之。」又詔以丕為都督、領並州刺史。後詔以平陽畿甸,改封新興公。



 初,
 李沖文德望所屬,既當時貴要,有杖情,遂與子超娶沖兄女,即伯尚妹也。



 丕前妻子隆,同產數人,皆與別居;後得宮人,所生同宅共產。父子情因此偏。丕父子大意不樂遷洛。帝之發平城,太子恂留於舊京。及將還洛,隆與穆泰等密謀留恂,因舉兵據陘北。丕時以老居並州,雖不預始計,而隆、超咸以告丕。丕外慮不成,口乃致難,心頗然之。及帝幸平城,推穆泰等首謀,隆兄弟並是黨。丕亦隨駕至平城,每於測問,令丕坐觀。與元業等兄弟並以謀逆,有司奏處孥戮。詔以丕應連坐,但以先許不死之詔,躬非染逆之身,聽免死,仍為太原百姓,其後妻二子
 聽隨。隆、超母弟及餘庶兄弟皆徙敦煌。丕時年垂八十,猶自平城力載隨駕至洛,留洛陽。帝每遣左右慰勉之,乃還晉陽。



 孝文崩,丕自並來赴,宣武引見之,以丕舊老,禮有加焉。尋敕留洛陽。後宴于華林都亭,特令二子扶侍坐起。丕仕歷六世,垂七十年,位極公輔,而還為庶人,然猶心戀京邑,不能自絕人事。詔以丕為三老。景明四年,薨,年八十二。詔贈左光祿大夫、冀州刺史,謚曰平。長子隆,先以反誅。隆弟乙升、超,亦同誅。超弟俊、邕,並以軍功,俊封新安縣男,邕封涇縣男。



 淮陵侯大頭,烈帝之曾孫也。善騎射,擢為內三郎。文成
 初,封淮陵。性謹密,帝甚重之,位寧北將軍。卒,贈高平公,謚曰烈。



 河間公齊,烈帝之玄孫也。少雄傑魁岸。太武征赫連昌。太武馬蹶,賊逼帝。



 齊以身蔽捍,決死力戰,賊乃退,帝得上馬。是日微齊,帝幾至危殆。帝以微服入其城,齊固諫不許,乃與數人從帝入。城內既覺,諸門悉閉。帝及齊等因入其宮中,得婦人裙,繫之槊上。帝乘而上,因此得拔,於齊有力焉。賜爵浮陽侯。從征和龍,以功拜尚書,進爵為公。後與新興王俊討禿髮保周,坐事免官爵。宋將裴方明陷仇池,太武復授齊前將軍,與建興公古弼討之。
 遂剋仇池,威振羌、氐。復賜爵河間公,與武都王楊保宗對鎮駱谷。時保宗弟文德說保宗閉險自固,有期矣。秦州主簿邊因知之,密告齊。晨詣保宗,呼曰:「古弼至,欲宣詔。」保宗出,齊叱左右扶保宗上馬,馳驛送臺。諸氐遂推文德為主,求授於宋。宋遣將房亮之、苻昭、啖龍等率眾助文德。齊擊斬殺龍,禽亮之,氐遂平。以功拜內都大官。卒,謚敬王。



 長子陵襲爵。陵性抗直,天安初,為乙渾所害。陵弟蘭,以忠謹見寵。孝文初,賜爵建陽子,卒於武川鎮將。



 子志,字猛略,少清辯彊幹,歷覽書傳,頗有文才。為洛陽令,不避彊禦,與御史中尉李彪爭路,俱入見。面陳得失。彪
 言:「御史中尉辟承華蓋,駐論道劍鼓,安有洛陽令與臣抗衡?」志言:「神鄉縣主,普天之下,誰不編戶?豈有俯同眾官,趨避中尉?」孝文曰:「洛陽,我之豐、沛,自應分路揚鑣,自今以後,可分路而行。」及出,與彪折尺量道,各取其半。帝謂邢巒曰:「此兒竟可,所謂王孫公子,不鏤自彫。」巒曰:「露竹霜條,故多勁節;非鸞則鳳,其在本枝也。」



 員外郎馮俊,昭儀之弟,恃勢恣撾所部里正。志令主吏收繫,處刑除官。由此忤旨,左遷太尉主簿。俄為從事中郎。車駕南征,帝微服觀戰所。有箭欲犯帝,志以身鄣之,帝便得免。矢中志目,因此一目喪明。以志行恒州事。宣武時,除荊州
 刺史。還朝,御史中尉王顯奏志於在州日抑買良人為婢,兼乘請供朝,會赦免。明帝初,兼廷尉卿。後除揚州刺史,賜爵建忠伯。志在州,威名雖減李崇,亦為荊楚所憚。尋為雍州刺史。晚年耽好聲伎。在揚州日,侍側將百人,器服珍麗,冠於一時。及在雍州,逾尚華侈,聚斂無極,聲名遂損。及莫折念生反,詔志為西征都督討之。念生遣其弟天生屯龍口,與志相持。為賊所乘,遂棄大眾奔還岐州。賊遂攻城,州刺史裴芬之疑城人與賊潛通,將盡出之,志不聽。城人果開門引賊,金巢志及芬之送念生,見害。節閔初,贈尚書僕射、太保。



 扶風公處真,烈帝之後也。少以壯烈聞,位殿中尚書,賜爵扶風公,委以大政,甚見尊禮。吐京胡曹僕渾等叛,招引朔方胡為援,處真與高涼王那等討滅之。性貪婪,在軍烈暴,坐事伏法。



 文安公泥,魏之疏族也。性忠直,有智畫。道武厚遇之,賜爵文安公,拜安東將軍。卒。子屈襲爵。明元時,居門下,出納詔命。性明敏,善奏事,每合上旨。



 賜爵元城侯,加功勞將軍。與南平公長孫嵩、白馬侯崔密等並決獄訟。明元東巡,命屈行右丞相,山陽侯奚斤行左丞相,命掌軍國,甚有聲譽。後吐京胡與離石胡出以兵等叛,置立將校,
 外引赫連屈丐。屈督會稽劉潔、永安侯魏勤捍之。勤沒於陣,潔墜馬,胡執送屈丐,唯屈眾猶存。明元以屈沒失二將,欲斬之。時並州刺史元六頭荒淫怠事,乃赦屈,令攝州事。屈嗜酒,頗廢政事。帝積其前後失,檻車徵還,斬於市。



 子磨渾,少為明元所知。元紹之逆也,明元潛隱於外,磨渾與叔孫詐云明元所在。紹使帳下二人隨磨渾往,規為逆。磨渾既得出,便縛帳下,詣明元斬之。帝得磨渾,大喜,因為羽翼。以勳,賜爵長沙公,拜尚書,出為定州刺史。卒。



 昭成皇帝九子:庶長曰寔君,次曰獻明帝,次曰秦王翰,
 次曰閼婆,次曰壽鳩,次曰紇根,次曰地乾,次曰力真,次曰窟咄。



 寔君性愚,多不仁。昭成季年,苻堅遣其行唐公苻洛等來寇南境,昭成遣劉庫仁逆戰於石子嶺。昭成時不勝,不能親勒眾軍,乃率諸部避難陰山,度漠北。高車四面寇抄,復度漠南。苻洛軍退,乃還雲中。



 初,昭成以弟孤讓國,乃以半部授孤。孤子斤失職懷怨,欲伺隙為亂。獻明皇帝及秦明王翰皆先終,道武年甫五歲,慕容后子閼婆等雖長,而國統未定。斤因是說實君曰:「帝將立慕容所生,欲先殺汝,是以頃來諸子戎服,夜以兵仗繞廬舍,
 伺便將發。」時苻洛等軍猶在君子津,夜常警備,諸皇子挾仗徬徨廬舍,寔君以斤言為信,乃盡殺諸皇子,昭成亦暴崩。其夜,諸皇子婦及宮人奔告洛軍。堅將李柔、張蠔勒兵內逼,部眾離散。苻堅聞之,召燕鳳問其故,以狀對。堅曰:「天下之惡一也!」乃執寔君及斤,轘之長安。



 寔君孫勿期,位定州刺史,賜爵林慮侯。卒。子六狀,真定侯。



 秦王翰,少有高氣。年十五,便請征伐。昭成壯之,使領騎二千。長統兵,號令嚴信,多有剋捷。建國十五年,卒。道武即位,追贈秦王,謚曰明。子儀,長七尺五寸,容貌甚偉,美髯,有算略。少能舞劍,騎射絕人。道武幸賀蘭部,侍從出入。
 登國初,賜爵九原公。從破諸部,有謀戰功。及帝將圖慕容垂,遣儀觀釁。垂問儀道武不自來之意。儀曰:「先人以來,世據北土,子孫相承,不失其舊。乃祖受晉正朔,爵稱代王,東與燕世為兄弟。儀之奉命。理謂非失。」垂壯其對,因戲曰:「吾威加四海,卿主不自見吾,云何非失?」儀曰:「燕若不修文德,欲以兵威自強,此乃本朝將帥之事,非儀所知也。」及還,報曰:「垂死乃可圖,今則未可。」帝作色問之,儀曰:「垂年已暮,其子寶弱而無威,謀不能決。慕容德自負才氣,非弱主之臣,釁將內起,是可計之。」帝以為然。後改封平原公。



 道武征衛辰,儀出別道,獲衛辰尸,傳首
 行宮。帝大喜,徙封東平公。命督屯田於河北,自五原至棝陽塞外,分農稼,大得人心。慕容寶之寇五原,儀躡據朔方,要其還路。及並州平,儀功多,遷尚書令。從圍中山。慕容德敗也,帝以普驎妻周氏賜儀,並其僮僕財物。尋遷都督中外諸軍事、左丞相,進封衛王。中山平,復遣儀討鄴,平之。道武將還代都,置中山行臺,詔儀守尚書令以鎮之,遠近懷附。尋徵儀以丞相入輔。又從征高車,儀別從西北破其別部。又從討姚平有功,賜以絹布綿牛馬羊等。



 儀膂力過人,弓力將十石,陳留公虔槊大稱異。時人云:「衛王弓,桓王槊。」



 太武初育也,道武喜,夜召儀
 入,曰:「卿聞夜喚,乃不怪懼乎?」儀曰:「怪則有之,懼實無也。」帝告以太武生,賜儀御馬、御帶、縑錦等。



 先是,上谷侯岌、張袞、代郡許謙等有名於時,初來入軍。聞儀待士,先就儀,儀並禮之,共談當世之務。謙等三人曰:「平原公有大才,不世之略,吾等宜附其尾。」道武以儀器望,待之尤重,數幸其弟,如家人禮。儀矜功恃寵,遂與宜都公穆崇伏甲謀亂。崇子逐留在伏士中,道武召之,將有所使。逐留聞召,恐發,踰牆告狀,帝祕而恕之。天賜六年,天文多變,占者云:「當有逆臣,伏尸流血。」帝惡之,頗殺公卿,欲以厭當天災。儀內不自安,單騎遁走。帝使人追執之,遂賜死,
 葬以百姓禮。



 儀十五子。纂,五歲,道武命養於宮中,恩與諸皇子同。太武陵阼,除定州刺史,封中山公,進爵為王,賜步挽几以優異之。纂好酒愛佞,政以賄成。太武殺其親嬖人。後悔過修謹,拜內大將軍。居官清約簡慎,更稱廉平。纂於宗屬最長,宗室有事,咸就諮焉。薨,謚曰簡。



 纂弟良,性忠篤。明元追錄儀功,封南陽王以紹儀後。



 良弟幹,善弓馬,以騎從明元於白登之東北,有雙鴟飛鳴於上,帝命左右射之,莫能中。鴟游飛稍高,乾以二箭下雙鴟。帝賜之御馬、弓矢、金帶一,以旌其能。



 軍中於是號乾為射鴟都將。從太武南巡,進爵新蔡公。文成即位,拜都
 官尚書。卒,謚曰昭。



 子禎,膽氣過人。太武時,為司衛監。從征蠕蠕,忽遇賊別部,多少不敵。禎乃就山解鞍放馬,以示有伏,賊果疑而避之。孝文初,賜爵沛郡公,後拜南豫州刺史。大胡山蠻時鈔掠,前後守牧多羈縻而已。禎乃召新蔡、襄城蠻首,使之觀射。



 先選左右能射者二十餘人,禎自發數箭皆中,然後命左右以次而射。先出一囚犯死罪者,使參射限,命不中,禎即責而斬之。蠻魁等伏伎畏威,相視股心慄。又預教左右取死囚十人,皆著蠻衣,云是鈔賊。禎乃臨坐,偽舉目瞻天,微有風動,禎謂蠻曰:「風氣少暴,似有鈔賊入境,不過十人,當在西南五十里
 許。」即命騎追掩,果縛送十人。禎告諸蠻曰:「爾鄉里作賊如此,合死以不?」蠻等皆叩頭曰:「合萬死。」禎即斬之。」因慰喻遣還,自是境無暴掠。淮南人相率投附者三千餘家,置之城東汝水之側,名曰歸義坊。初,豫州城豪胡丘生數與外交通,及禎為刺史,丘生嘗有犯懷恨,圖為不軌,詐以婚進城人,告云:「刺史欲遷城中大家,送之向代。」共謀翻城。城人石道起以事密告禎,速掩丘生,并諸預謀者。禎曰:「吾不負人,人何以叛?但丘生誑誤。若即收掩,眾必大懼,吾靜以待之,不久自當悔服。」



 語未訖而城中三百人自縛詣州門,陳丘生譎誑之罪。而丘生單騎逃走,
 禎恕而不問。



 後徵為都牧尚書。卒。贈侍中、儀同三司,謚簡公。有八子。



 第五子瑞。初,瑞母尹氏有娠致傷,後晝寢,夢一老翁具衣冠告之曰:「吾賜汝一子,汝勿憂也。」寤而私喜,又問筮者,筮者曰:「大吉。」未幾而生瑞。禎以為協夢,故名瑞,字天賜。位太中大夫。卒,贈太常卿。



 儀弟烈,剛武有智略。元紹之逆,百僚莫敢有聲,唯烈行出外,詐附紹,募執明元,紹信之,自延秋門出。遂迎立明元。以功進爵陰平王。薨,謚曰熹。子求襲。



 弟道子,位下大夫。道子子洛,位羽林幢將。洛子乞,中散大夫。乞子晏,孝靜初,累遷吏部尚書,平心不撓,時論稱之。出為瀛州刺史,在任未幾,
 百姓欣賴。蔣天樂之逆,見引,詔錄送定州賜死。晏好集圖籍,家書多秘閣,諸有假借,咸不逆其意,亦以此見稱。



 烈弟觚,勇烈有膽氣。少與兄儀從道武,侍衛左右。使於慕容垂,垂末年政在群下,遂止觚以求賂,道武絕之。觚率左右馳還,為垂子寶所執,垂待之更厚,因留心學業,誦讀經書數十萬言,垂國人咸稱重之。道武之討中山,慕容普驎遂害觚以固眾心。帝聞之哀慟。及平中山,發普驎塚,斬其尸,收議害觚者傅高霸、程同等,皆夷五族,以大刃剉殺之。乃葬觚,追謚秦愍王,封子夔為豫章王以紹觚。



 常山王遵,壽鳩之子也。少而壯勇,不拘小節。道武初,有佐命勳,賜爵略陽公。慕容寶之敗也,別率騎七百,邀其歸路,由是有參合之捷。及平中山,拜尚書左僕射,加侍中,領勃海之合口。及博陵、勃海群盜起,遵討平之,遷州牧,封常山王。遵好酒色。天賜四年,坐醉亂,失禮於太原公主。賜死,葬以百姓禮。



 子素,明元從母所生,特見親寵。太武初,復襲爵。休屠郁原等叛,素討之,斬渠率,徙千餘家於涿鹿之陽,立平原郡以處之。及平統萬,以素有威懷之略,拜假節、征西大將軍以鎮之。後拜內都大官。文成即位,務崇寬政,罷諸雜調。有司奏國用不足,固請復
 之,唯素曰:「臣聞百姓不足,君孰與足?」帝善而從之。素,宗屬之懿,又年老,帝每引入,訪以政事,固辭疾歸第。雅性方正,居官五十載,終始若一,時論賢之。薨,謚曰康,陪葬金陵,配饗廟廷。



 長子可悉陵,年十七,從太武獵,逐一猛獸,陵遂空手搏之以獻。帝曰:「汝才力絕人,當為國立功立事,勿如此也!」即拜內行阿干。又從平涼州,沮渠茂虔令一驍將與陵相擊,兩槊皆折,陵抽箭射之墜馬。陵恐其救至,未及拔劍,以刀子戾其勁,使身首異處。帝壯之,即日拜都幢將,封暨陽子。卒於中軍都將。



 弟陪斤襲爵,坐事國除。陪斤子昭,小字阿倪,尚書張彝引兼殿中郎。
 孝文將為齊郡王蘭舉哀,而昭乃作宮懸。帝大,詔曰:「阿倪愚騃,誰引為郎?」於是黜彞白衣守尚書,昭遂停廢。宣武時,昭從弟暉親寵用事,稍遷左丞。宣武崩,於忠執政,昭為黃門郎,又曲事之。忠專權擅威,枉陷忠賢,多昭所指導也。靈太后臨朝,為尚書、河南尹,聾而狠戾,理務峭急,所在患之。尋出為雍州刺史,在州貪虐,大為人害。後入為尚書,諂事劉騰。進號征西將軍。卒,贈尚書左僕射。納貨元叉,所以贈禮優越。



 子玄,字彥道,以節儉知名。孝壯時,為洛陽令。及節閔即位,玄上表乞葬莊帝,時議善之。後除尚書左丞。孝武帝即位,以孫騰為左僕射。騰
 即齊神武心膂,仗入省,玄依法舉劾,當時咸為玄懼。孝武重其強正,封臨淄縣子。及從入關,封陳郡王,位儀同三司,加開府。薨,謚曰平。



 昭弟紹,字醜倫,少聰慧。遷尚書右丞。紹斷決不避強禦。宣武詔令檢趙修獄,以修佞幸,因此遂加杖罰,令其致死。帝責紹不重聞,紹曰:「修姦佞甚於董賢,臣若不因釁除之,恐陛下復被哀帝之名。」以其言正,遂不罪焉。及出,廣平王懷拜紹,賀曰:「阿翁乃皇家之正直,雖朱雲、汲黯何以仰過!」紹曰:「但恨戮之稍晚,以為愧耳。」卒於涼州刺史。



 陪斤弟忠,字仙德,以忠謹聞。孝文時,累遷右僕射,賜爵城陽公,加侍中、鎮西將軍,有
 翼贊之勤,百寮咸敬之。太和四年,病篤辭退」養疾於高柳,輿駕親送都門之外,群寮侍臣執別者莫不涕泣。及卒,皆悼惜之。謚曰宣,命有司為立碑銘。



 子盛,字始興,襲爵,位謁者僕射。卒。子懋,字伯邕,襲爵,降為侯。從駕入關,封北平王。薨,贈尚書左僕射,謚曰貞慧。子陟,字景升,開府儀同三司。



 弟順,字敬叔,從孝武入關,封濮陽王,位侍中。及武帝崩,祕未發喪,諸人多舉廣平王為嗣。順於別室垂涕謂周文曰:「廣平雖親,年德並茂,不宜居大寶。」



 周文深然之,因宣國諱,上南陽王尊號。以順為中尉,行雍州事,又加開府儀同三司、秦州刺史。順善射。初,孝武在
 洛,於華林園戲射,以銀酒卮容二升許,懸於百步外,命善射者十餘人共射,中者即以賜之。順發矢即中,帝大悅,並賞金帛。



 順仍於箭孔處鑄一銀童,足蹈金蓮,手持剷炙,遂勒背上,序其射工。



 子偉,字子猷,有清才。大統十六年,封南安郡王。及尉遲迥伐蜀,以偉為司錄,書檄文言,皆偉所為。六官建,拜師氏下大夫,改淮南縣公。周明帝初,拜師氏中大夫,受詔於騏麟殿刊正經籍。建德中,累遷小司寇,為使主,報聘於齊。是秋,武帝親戎東討,偉遂為齊所留。齊平,偉方見釋,加授上開府。後除襄州刺史,位大將軍。偉性溫柔,好虛靜,篤學愛文。初自鄴還,庾
 信贈其詩曰:「梁亡垂棘反,齊平寶鼎歸。」為辭人所重如此。後疾卒。



 盛弟壽興,少聰慧好學。宣武初,為徐州刺史。在官貪虐,失於人心。其從兄侍中暉深害其能,因譖之於帝,詔尚書崔亮馳驛檢核。亮發日,受暉旨,遂鞭撻三寡婦,令其自誣,稱壽興壓己為婢。壽興終恐不免,乃令其外弟中兵參軍薛修義將車十乘,運小麥,經其禁之旁。壽興因踰牆出,修義以大木函盛壽興,其上加麥,載之而出,遂至河東,匿修義家。逢赦乃出,見帝,自陳為暉所譖,帝亦更無所責。



 初,壽興為中庶子時,王顯在東宮。賤,因公事,壽興杖之四十。及顯有寵,為御史中尉,奏壽
 興在家每有怨言,誹謗朝廷。因帝極飲,無所覺悟,遂奏其事,命帝注可,直付壽興賜死。帝書半不成字,當時見者亦知非本心,但懼暉等威,不敢申拔。及行刑日,顯自往看之。壽興命筆自作墓志銘曰:「洛陽男子,姓元名景,有道無時,其年不永。」餘文多不載。顧謂其子曰:「我棺中可著百張紙,筆兩枚,吾欲訟顯於地下。若高祖之靈有知,百日內必取顯。如遂無知,亦何足戀!」及宣武崩,顯尋被殺。壽興之死,時論亦以為前任中尉彈高闕讒諷所致。靈太后臨朝,三公郎中崔鴻上疏理壽興,詔書追雪,贈豫州刺史,謚曰莊。



 子最,字幹,從孝武入關,封樂平
 王,位侍中,兼尚書左僕射,加特進。



 壽興弟益生,少亡。



 子毗,字休弼。武帝之在籓邸,少親之,及即位,出必陪乘,入於臥內。及帝與齊神武有隙,時議者各有異同。或勸天子入夷,或言與齊神武決戰,或云奔梁。



 唯毗數人以關中帝王桑梓,殷勤叩頭請西入。策功論賞,毗與領軍斛斯椿等十三人為首,封魏郡王。時王者邑止一千戶,唯毗邑一千五百。齊神武宣告關東云:「將天子西入,事起元毗,雖百赦不在原限。」薨,謚曰景。子綽。



 忠弟德,封河間公,卒於鎮南將軍,贈曹州刺史。德子悝,潁川太守,卒於光州刺史,謚曰恭。



 子嶷,宇子仲。孝武初,授兗州刺
 史。于時城人王奉伯等相扇謀逆,棄城出走。



 懸門發,斷嶷要而出。詔齊州刺史尉景、本州刺史蔡俊各部在州士往討之。嶷返復任。封濮陽縣伯。孝靜時,轉尚書令,攝選部。嶷雖居重任,隨時而已。薨於瀛州刺史,贈司徒公,謚曰靖懿。



 悝弟暉,字景襲。少沉敏,頗涉文史。宣武即位,為給事黃門侍郎。初,孝文遷洛,舊貴皆難移,時欲和眾情,遂許冬則居南,夏便居北。宣武頗惑左右之言,外人遂有還北之問。至乃榜賣田宅,不安其居。暉乃請間言事,具奏所聞,曰:「先皇移都,以百姓戀土,故發冬夏二居之詔,權寧物意耳。乃是當時之言,實非先皇深意。且比來遷
 人,安居歲久,公私計立,無復還情。伏願陛下終高祖既定之業,勿信邪臣不然之說。」帝納之。再遷侍中,領右衛將軍。雖無補益,深被親寵。



 凡在禁中要密之事,暉別奉旨,藏之於櫃。唯暉入乃開,其餘侍中、黃門莫有知者。



 侍中盧昶亦蒙恩眄,故時人號曰「餓彪將軍,飢鷹侍中。」遷吏部尚書。納貨用官,皆有定價,大郡二千匹,次郡一千匹,下郡五百匹,其餘官職各有差,天下號曰市曹。出為冀州刺史。下州之日,連車載物,發信都至湯陰間,首尾相屬,道路不斷。



 其車少脂角,即於道上所逢之牛,生截取角,以充其用。暉檢括丁戶,聽其歸首,出調絹五萬匹。
 然聚斂無極,百姓患之。明帝初,徵拜尚書左僕射,詔攝吏部選事。



 後昭暉與任城王澄、京兆王愉、東平王匡共決門下大事。暉又上書論政要:其一曰:御史之職,務使得賢。必得其人,不拘階秩,久於其事,責其成功。其二曰:安人寧邊,觀時而動。頃來邊將亡遠大之略,貪萬一之功,楚、梁之好未聞,而蠶婦之怨屢結,斯乃庸人所為,銳於姦利之所致也。平吳之計,自有良圖,不在於一城一戍也。又河北數州,國之基本;飢荒多年,戶口流散。方今境上,兵復徵發,即如此日,何易舉動?愚謂數年以來,唯宜靜邊,以息召役,安人勸農,惠此中夏。請嚴敕邊將,自
 今在戍賊求內附者,不聽輒遣援接,皆須表聞。違者雖有功,請以違詔書論。三曰:國之資儲,唯籍河北。饑饉積年,戶口逃散,生長姦詐,因生隱藏。



 出縮老小,妄注死失,收人租調,割入於己。人困於下,官損於上。自非更立權制,善加檢括,損耗之來,方在未已。請求其議,明宣條格。帝納之。暉雅好文學,招集儒士崔鴻等撰錄百家要事,以類相從,名為《科錄》,凡二百七十卷,上起伏羲,迄於晉,凡十四代。暉疾篤,表上之。卒,賜東園秘器,贈使持節、都督中外諸軍事、司空公,謚曰文憲。將葬,給羽葆班劍鼓吹二十人,羽林百二十人。



 子弼,字宗輔,性和厚,美容儀。
 以莊帝舅子婿,特封廣川縣子。天平初,累遷尚書令。弼妹為孝武所納,以親情見委,禮遇特隆。歷中書監、錄尚書事,位特進、宗師。齊受禪,除左光祿大夫。天保三年,卒。十年,諸子與諸元同誅死。



 弼弟子士將,有巧思。至齊武成時,位將作大匠。



 德弟贊,頗有名譽,好陳軍國事宜。初置司州,以贊為刺史,賜爵上谷侯。孝文戒贊化畿甸,可宣孝道,必令風教洽和,文禮大備。自今有不孝不悌者,比其門,以刻其柱。又詔曰:「司州刺史,官尊位重,職總京畿,選屬懿親,以允具瞻之望。但諸王年少,未閑政體,故以授贊,庶能助暉道化。今司州始立,郡縣初置,公卿
 已下皆有本屬,可人率子弟,用相展敬。」於是賜名曰「贊」。詔贊乘步挽入殿門,加太子少師,遷左僕射。孝文將謀遷洛,諸公多異同,唯贊贊成大策。帝每歲南伐,執手寄以後事。卒,贈衛將軍,僕射如故。後以留守贊輔之功,進封晉陽縣伯。



 贊弟淑,字買仁。彎弓三百斤,善騎射。孝文時,為河東太守。河東俗多商賈,罕事農桑,人至有年三十不識耒耜。淑下車勸課,躬往教示,二年間,家給人足,為之謠曰:「泰州河東,杼柚代舂。元公至止,田疇始理。」卒於平城鎮將,謚曰靜。有七子。



 季海字元泉,兄弟中最有名譽,位洛州刺史。季海妻,司空李沖之女,莊帝從母也,
 賜爵唐郡君。政在爾朱,禍難方始,勸季海為外官以避纖介。及孝莊之難,季海果以在籓得免。從孝武入關,封馮翊王,位中書令、雍州刺史,遷司空。病薨,謚曰穆。



 子亨,字德良,一名孝才。遇周、齊分隔,時年數歲,與母李氏在洛陽。齊神武以亨父在關中,禁固之。其母遂稱凍餒,得就食湯陰,託大豪李長壽,攜亨及孤姪數人,得至長安。周文以功臣子,甚禮之。大統末,襲爵馮翊王,累遷勳州刺史,改封平涼王。周受禪,例降為公。隋文帝受禪,自洛州刺史徵拜太常卿。尋出為衛州刺史,在職八年,風化大洽。以老病乞骸骨,吏人詣闕上表請留,上嗟嘆者久
 之。



 其年,亨以篤疾,重請還京,上令使者致醫藥,問動靜,相望於道。卒於家,謚曰宣。



 陳留王虔,紇根之子也。登國初,賜爵陳留公。與衛王儀破黜弗部,從攻衛辰。



 慕容寶來寇,虔絕其左翼,寶敗。垂恚憤來桑乾,虔勇而輕敵,於陳戰沒。虔姿貌魁傑,武力絕倫,每以矛細短,大作之,猶患其輕,復綴鈴於刃下。其弓力倍加常人。以其殊異,代京武庫常存而志之。虔常以槊刺人,遂貫而高舉。又嘗以一手頓槊於地,馳馬偽退,敵人爭取,引不能出。虔引弓射之,一箭殺二三人,搖槊之徒,亡魂而散,徐乃令人取槊而去。每從征討,及為
 偏將,常先登陷陣,勇冠當時,敵無眾寡,莫敢抗其前者。及薨。舉國悲歎,為之流涕,道武追惜傷慟者數焉。追謚陳留桓王,配饗廟廷,封其子悅為朱提王。悅外和內狠。道武常以桓王死王事,特加親寵,為左將軍,襲封,後為宗師。悅恃寵驕矜,每謂所親王洛生之徒言曰:「一旦宮車晏駕,吾止避衛公。除此,誰在吾前!」衛王儀美髯,為內外所重,悅故云。初,姚興之贖狄伯支,悅送之,路由雁門。悅因背誘姦豪,以取其意。後遇事譴逃亡,投鴈門,規收豪傑,欲為不軌,為土人執送。帝恕而不罪。明元即位,引悅入侍,仍懷姦計,說帝云:「京師雜人不可保信,宜誅其
 非類者。」又云:「鴈門人多詐,並可誅之。」欲以雪其私忿,帝不從。悅內自疑懼,懷刃入侍,謀為大逆。叔孫俊疑之。竊視其懷有刃,執而賜死。



 弟崇,太武詔令襲桓王爵。崇性沉厚。初,衛王死後,道武欲敦宗親之義,詔引諸王子弟入宴。常山王素等三十餘人咸謂與衛王相坐,疑懼。皆出逃遁,將奔蠕蠕,唯崇獨至。道武見之,甚悅,厚加禮賜,遂寵敬之,素等於是亦安。久之,拜并州刺史,有政績。從征蠕蠕,別督諸軍出大澤,越涿耶山,威懾漠北。薨,謚曰景王。



 子建襲,降爵為公,位鎮北將軍,懷荒鎮大將。卒。建子琛,位恒、肆二州刺史。琛子翌,尚書左僕射。翌子
 暉。



 暉字叔平,鬚眉如畫,進止可觀。好涉獵書記,少得美名於京下。周文禮之,命與諸子遊處,每同硯席,情契甚厚。再遷武伯下大夫。時突厥屢為寇患,朝廷將結和親,令暉買錦彩十萬,使突厥。暉說以利害,可汗大悅,遣其名王隨獻方物。



 俄拜儀同三司。周武帝之娉突厥后,令暉致禮。授開府,轉司憲大夫。及平關東,使暉安集河北,封義寧子。隋文帝總百揆,加上開府,進爵為公。開皇初,拜都官尚書,兼領太僕。奉詔決杜陽水灌三畤原,溉舄鹵之地數千頃,人賴其利。再遷兵部尚書,監漕渠之役。未幾,坐事免。頃之,拜魏州刺史,頗有惠政。後以疾去職,卒
 于京師。帝嗟悼久之,敕鴻臚監護喪事,謚曰元。子肅嗣,位光祿少卿。肅弟仁,器性明敏,位日南郡丞。



 建弟嫡子祚,字龍壽。宣武校藝,每於歲暮,詔令教習講武。初,建以子罪失爵,祚欲求本封。有司奏聽祚襲公,其王爵不輕,共求更議,詔從之。卒于河州刺史。節閔時,贈侍中、尚書僕射。



 虔兄顗,性嚴重少言。道武常敬之,雅有謀策。從平中山,以功賜爵蒲城侯。



 特見寵厚,給鼓吹羽儀,禮同岳牧。蒞政以威信著稱,居官七年,乃以元易乾代顗為郡。時易乾子萬言得寵於道武,易乾恃其子,輕忽於顗,不告其狀,輕騎卒至,排顗墜床,而據其坐。顗不知代己,謂
 以罪見捕。既而知之,恥其侮慢,謂易乾曰:「我更滿被代,常也。汝無禮見辱,豈可容哉!」遂搏而殺之。以狀具聞,道武壯之。萬言累以訴請,乃詔顗輸贖。顗乃自請罪,道武赦之,復免其贖。病卒。



 子崙,太武時襲父爵,以功除統萬鎮將。後從永昌王仁南征,別出汝陰。濟淮,宋將劉康祖屯於慰武亭以邀軍路,師人患之。崙曰:「今大風既勁,若令推草車,方軌並進,乘風縱煙火,以精兵自後乘之,破之必矣。」從之,斬康祖,傳首行宮。



 文成即位,除秦州刺史,進爵隴西公。卒,謚曰定公。子琛襲爵。



 毗陵王順,地乾之子也。性疏狠。登國初,賜爵南安公。及
 道武討中山,留順守京師。柏肆之敗,軍人有亡歸者,言大軍奔散,不知帝所在。順聞之,欲自立,納莫題諫,乃止。時賀力眷等聚眾作亂於陰館,順討之不剋。乃從留官自白登南入繁畤故城,阻水壘水為固,以寧人心。道武善之,進封為王,位司隸校尉。道武好黃、老,數召諸王及朝臣親為說之,在坐莫不祗肅,唯順獨坐寐,不顧而唾。帝怒廢之。以王薨於家。



 遼西公意烈,力真之子也。先沒於慕容垂。道武征中山,棄妻子迎於井陘。及平中原,有戰獲勳,賜爵遼西公,除廣平太守。時和跋為鄴行臺,意烈性雄耿,自以帝屬,恥
 居跋下,遂陰結徒黨,將襲鄴。發覺,賜死。子拔幹,博知古今。父雖有罪,道武以拔干宗親,委之心腹。有計略,屢效忠勤。明元踐阼,除勃海太守,吏人樂之。賜爵武遂子,轉平原鎮將,得將士心。卒,謚曰靈公。子受洛襲,進爵武邑公。卒。子叱奴,武川鎮將。



 叱奴子洪超,頗有學涉,大乘賊亂之後,詔洪超持節兼黃門侍郎,綏慰冀部。



 還,上言冀土寬廣,界去州六七百里,負海險遠,宜分置一州,鎮遏海曲。朝議從之,後遂立滄州。卒於北軍將、光祿大夫。



 意烈弟勃,善射御,以勛賜爵彭城公。卒,謚曰闕。陪葬金陵。長子粟襲。太武時,督諸軍屯漠南。蠕蠕表聞。粟亮直,
 善馭眾,撫恤將士,必與之同勞逸。征和龍,以功進封為王。薨,陪葬金陵。



 粟弟渾,少善弓馬,太武嘉之。會有諸方使,命渾射獸三頭,發皆中,時舉坐咸以為善。及為宰官尚書,頗以驕縱為失,坐事免,徙長社,為人所害。



 子庫汗,為羽林中郎將。從北巡,有兔起乘輿前,命庫汗射之,應弦而斃。太武悅,賜一金兔,以旌其能。文成起景穆廟,賜爵陽豐侯。獻文即位,復造文成廟,拜殿中給事,進爵為公。庫汗明於斷決,每奉使察行州鎮,折獄以情,所歷皆稱之。



 秦州父老詣闕乞庫汗為刺史者,前後千餘人。朝廷許之,未及遣,遇病卒。子古辰襲。



 窟咄,昭成崩後,苻洛以其年長,逼徙長安。苻堅禮之,教以書學。因亂,隨慕容永東遷。永以為新興太守。劉顯之敗,遣弟亢掞等迎窟咄。遂逼南界,於是諸部騷動。道武左右于桓等謀應之,同謀人單馬乾以告帝。帝慮駭人心,沉吟未發。



 後三日,桓以謀白其舅穆崇。又告之,帝乃誅桓等五人,餘莫題等七姓悉原不問。



 帝慮內難,乃北踰陰山,幸賀蘭部,遣安同及長孫漫徵兵於慕容垂。賀曼亡奔窟咄,安同間行,遂達中山。慕容垂遣子賀驎步騎六千以隨之。安同與垂使人蘭紇俱還,達牛川,窟咄兄子意烈捍之。安同乃隱藏於商賈囊中。至暮,乃入空
 井得免,仍奔賀驎。軍既不至,而稍前逼賀染干。賀染干陰懷異端,乃為窟咄來侵北部。人皆驚駭,莫有固志。於是北部大人叔孫普洛節及諸烏丸亡奔衛辰。賀驎聞之,遽遣安同、朱譚等來。既知賀驎軍近,眾乃少定。道武自弩山幸牛川,窟咄進屯高柳。道武復使安同詣賀驎,因剋會期。安同還,帝踰參合,出代北,與賀驎會於高柳。窟咄窮迫,望旗奔走,遂為衛辰殺之。帝悉收其眾,賀驎執帝別歸中山。



 論曰:魏氏始自幽都,肇基帝業。上谷公等分枝若木,疏派天潢。或績預經綸,大開土宇;或迹同凶悖,自致殲夷。
 其禍福之來,唯人所召。至如神武之不事黃屋,高揖萬乘,義感鄰國;祚隆帝統,太伯、延陵未足多也。高涼讓國之胤,子那猛壯之風,或大位未加,或功不贖罪;褒德圖勞,其義為闕。松滋氣幹相承,聲迹俱顯;天穆得不以道,任過其量。持盈必悔,殺身為幸。武衛父子兼將,丕略始見器重,終以姦棄,不足觀矣。河間、扶風,武烈宣著,宗子之可稱乎!衛王英風猛概,折衝見重,謀之不臧,卒以自喪。秦王體度恢偉,陳留膽氣絕倫,亡身強寇,志力不展,惜哉!常山勇冠戚屬,與魏升降,亦以優乎!陰平忠烈,蒲陰器宇,榮寵兼萃,蓋有由焉。毗陵疏狠,遼西狷介,全身
 保位,固亦難矣。苻堅之轘寔君,衛辰之誅窟咄,逆子賊臣,蓋亦天下之惡一焉。



\end{pinyinscope}