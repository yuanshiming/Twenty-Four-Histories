\article{卷九十七列傳第八十五 西域}

\begin{pinyinscope}

 《夏書》稱:「西戎即序。」班固云:「就而序之,非盛威武致其貢物也。」



 漢氏初開西域,有三十六
 國。其後,分立五十五王,置校尉、都護以撫之。王莽篡位,西域遂絕。至於後漢班超所通者五十餘國,西至西海,東西萬里,皆來朝貢。



 復置都護、校尉,以相統攝。其後或絕或通,漢朝以為勞弊中國,其官時置時廢。



 暨魏、晉之後,互相吞滅,不可復詳記焉。



 道武初,經營中原,未暇及於四表。既而西戎之貢不至,有司奏依漢氏故事,請通西域,可以振威德
 於
 荒外,又可致奇貨於天府。帝曰:「漢氏不保境安人,乃遠開西域,使海內虛耗,何利之有?今若通之,前弊復加百姓矣!」遂不從。歷明元世,竟不招納。



 太延中,魏德益以遠聞,西域龜茲、疏勒、烏孫、悅般、渴槃陀、鄯善、焉耆、車師、粟特諸國王始遣使來獻。太武以西域漢世雖通,有求則卑辭而來,無欲則驕慢王命,此其自知絕遠,大兵不可至故也。若報使往來,終無所益,欲不遣使。有司奏:「九國不憚遐險,遠貢方物,當與其進,安可豫抑後來?」乃從之。於
 是始遣行人王恩生、許綱等西使。恩生出流沙,為蠕蠕所執,竟不果達。又遣散騎侍郎董琬、高明等多齎錦帛,出鄯善,招撫九國,厚賜之。初,琬等受詔:便道之國,可往赴之。琬過九國,北行至烏孫國。其王得魏賜,拜受甚悅。謂琬等曰:「傳聞破洛那、者舌皆思魏德,欲稱臣致貢,但患其路無由耳。今使君等既到此,可往二國,副其慕仰之誠。」琬於是自向破洛那,遣明使者舌。烏孫王為發導譯,達二國,琬等宣詔慰賜之。已而琬、明東還,烏孫、破洛那之屬遣使與琬俱來貢獻者,十有六國。自後相繼而來,不
 間於城,國使亦數十輩矣。



 初,太武每遣使西域,常詔河西王沮渠牧犍,令護送。至姑臧,牧犍恆發使導路,出於流沙。後使者自西域還至武威,牧犍左右謂使者曰:「我君承蠕蠕吳提妄說,云:『去歲魏天子自來伐我,士馬疫死,大敗而還,我擒其長弟樂平王丕。』我君大喜,宣言國中。又聞吳提遣使告西域諸國:『魏已削弱,今天下唯我為強。



 若更有魏使,勿復恭奉。』西域諸國,亦有貳。」且牧犍事主,稍以慢墮。使還,具以狀聞。太武遂議討牧犍。涼州既平,鄯善國以為辱亡齒寒,自然之道也。今武威為魏所滅。次及我矣。若通其使人,知我國事,取亡必近;不如絕之,可以支久。



 乃斷塞行路,西域貢獻,歷年不入。後平
 鄯善,行人復通。



 始,琬等使還京師,具言凡所經見及傳聞傍國,云:西域自漢武時五十餘國,後稍相并,至太延中為十六國。分其地為四域:自蔥嶺以東,流沙以西為一域;蔥嶺以西,海曲以東為一域;者舌以南,月氏以北為一域;兩海之間,水澤以南為一域。內諸小渠長,蓋以百數。其出西域,本有二道,後更為四:出自玉門,度流沙,西行二千里至鄯善,為一道;自玉門度流沙,北行二千二百里至車師,為一道;從莎車西行一百里至蔥嶺,蔥嶺西一千三百里至伽倍,為一道;自莎車西南五百里,蔥嶺西南一千三百里至波路,為一道焉。自琬所不傳
 而更有朝貢者,紀其名,不能具國俗也。



 東西魏時,中國方擾,及於齊、周,不聞有事西域,故二代書並不立記錄。



 隋開皇、仁壽之間,尚未云經略。煬帝時,乃遣侍御史韋節、司隸從事杜行滿使於西籓諸國,至罽賓得瑪瑙盃,王舍城得佛經,史國得十舞女、師子皮、火鼠毛而還。帝復令聞喜公裴矩於武威、張掖間往來以引致之。其有君長者四十四國,矩因其使者入朝,啖以厚利,令其轉相諷諭。大業中,相率而來朝者四十餘國,帝因置西戎校尉以應接之。尋屬中國大亂,朝貢遂絕。然事亡失,書所存錄者二十國焉。



 魏時所來者,在隋亦有不至,今總
 而編次,以備前書之《西域傳》云。至於道路遠近,物產風俗,詳諸前史,或有不同。斯皆錄其當時,蓋以備其遺闕爾。



 鄯善國,都扞泥城,古樓蘭國也。去代七千六百里。所都城方一里。地多沙鹵,少水草,北即白龍堆路。至太延初,始遣其弟素延耆入侍。及太武平涼州,沮渠牧犍弟無諱走保敦煌。無諱後謀渡流沙,遣其弟安周擊鄯善,王比龍恐懼欲降。會魏使者自天竺、罽賓還,俱會鄯善,勸比龍拒之,遂與連戰。安周不能剋,退保東城。



 後比龍懼,率眾西奔且末,其世子乃應安周。



 鄯善人頗剽劫之,令
 不得通,太武詔散騎常侍、成周公萬度歸乘傳發涼州兵討之。度歸到敦煌,留輜重,以輕騎五千度流沙,至其境。時鄯善人眾布野,度歸敕吏卒不得有所侵掠。邊守感之,皆望旗稽服。其王真達面縛出降,度歸釋其縛,留軍屯守,與真達詣京都。太武大悅,厚待之。是歲,拜交趾公韓拔為假節、征西將軍、領護西戎校尉、鄯善王以鎮之,賦役其人,比之郡縣。



 且末國,都且末城,在鄯善四,去代八千三百二十里。真君三年,鄯善王比龍避沮渠安周之難,率國人之半奔且末。後役屬鄯善。且末西北有流沙數百里,夏日有熱
 風,為行旅之患。風之所至,唯老駝預知之,即嗔而聚立,埋其口鼻於沙中。



 人每以為候,亦即將氈擁蔽鼻口。其風迅駃,斯須過盡,若不防者,必至危斃。



 大統八年,其兄鄯善米率眾內附。



 于闐國,在且末西北,蔥嶺之北二百餘里。東去鄯善千五百里,南去女國三千里,去朱俱波千里,北去龜茲千四百里,去代九千八百里。其地方亙千里,連山相次,所都城方八九里。部內有大城五,小城數十。于闐城東三十里有首拔河,中出玉石。土宜五穀并桑、麻。山多美玉。有好馬、駝、騾。其刑法,殺人者死,餘罪各隨輕重懲罰之。
 自外風俗物產,與龜茲略同。俗重佛法,寺塔、僧尼甚眾。王尤信尚,每設齋日,必親自灑掃饋食焉。城南五十里有贊摩寺,即昔羅漢比丘盧旃為其王造覆盆浮圖之所。石上有辟支佛跣處,雙跡猶存。于闐西五百里有比摩寺,云是老子化胡成佛之所。俗無禮義,多盜賊淫縱。自高昌以西諸國人等,深目高鼻,唯此一國,貌不甚胡,頗類華夏。城東二十里有大水北流,號樹枝水,即黃河也,一名計式水。城西十五里亦有大水名達利水,與樹枝水會,俱北流。



 真君中,太武詔高涼王那擊吐谷渾慕利延,慕利延懼,驅其部落渡流沙。那進軍急追之,慕利
 延遂西入于闐,殺其王,死者甚眾。獻文末,蠕蠕寇于闐。於闐患之,遣使素目伽上表曰:「西方諸國,今皆已屬蠕蠕。奴世奉大國,至今無異。今蠕蠕軍馬到城下,奴聚兵自固,故遣使奉獻,遙望救援。」帝詔公卿議之。公卿奏曰:「于闐去京師幾萬里,蠕蠕之性,唯習野掠,不能攻城。若為害,當時已旋矣,雖欲遣師,勢無所及。」帝以公卿議示其使者,亦以為然。於是詔之曰:「朕承天理物,欲令萬方各安其所,應敕諸軍,以拯汝難。但去汝遐阻,政復遣援,不救當時之急,是以停師不行,汝宜知之。朕今練甲養卒,一二歲間,當躬率猛將,為汝除患。汝其謹敬候,以待
 大舉。」先是,朝廷遣使者韓羊皮使波斯,波斯王遣使獻馴象及珍物。經于闐,於闐中於王秋仁輒留之,假言慮有寇不達。羊皮言狀,帝怒,又遣羊皮奉詔責讓之。自後每使朝貢。



 周建德三年,其王遣使獻名馬。



 隋大業中,頻使朝貢。其王姓王,字早示門。練錦帽,金鼠冠,妻戴金花。其王髮不令人見,俗言若見王髮,其年必儉云。



 蒲山國,故皮山國也。居皮城,在于闐南,去代一萬二千里。其國西南三里有凍凌山。後役屬于闐。



 悉居半國,故西夜國也,一名子合。其王號子。治呼犍。在于闐西,去代萬二千九百七十里。太延初,遣使來獻,自
 後貢使不絕。



 權於摩國,故烏秅國也。其王居烏秅城。在悉居半西南,去代一萬二千九百七十里。



 渠莎國,居故莎車城,在子合西北,去代一萬二千九百八十里。



 車師國,一名前部,其王居交河城。去代萬五十里。其地北接蠕蠕,本通使交易。太武初,始遣使朝獻,詔行人王恩生、許綱等出使。恩生等始度流沙,為蠕蠕所執。恩生見蠕蠕吳提,持魏節不為之屈。後太武切讓吳提,吳提懼,乃遣恩生等歸。許綱到敦煌病死,朝廷壯其節,賜謚
 曰貞。



 初,沮渠無諱兄弟之渡流沙也,鳩集遺人,破車師國。真君十一年,車師王車夷落遣使琢進薛直上書曰:「臣亡父僻處塞外,仰慕天子威德,遣使奉獻,不空於歲。天子降念,賜遣甚厚。及臣繼立,亦不闕常貢,天子垂矜,亦不異前世。敢緣至恩,輒陳私懇。臣國自無諱所攻擊,經今八歲,人民飢荒,無以存活。賊今攻臣甚急,臣不能自全,遂捨國東奔,三分免一。即日已到焉耆東界,思歸天闕,幸垂賑救。」於是下詔撫慰之,開焉耆倉給之。正平初,遣子入侍,自後每使朝貢不絕。



 高昌者,車師前王之故地,漢之前部地也。東西二百里,
 南北五百里,四面多大山。或云:昔漢武遣兵西討,師旅頓弊,其中尤困者因住焉。地勢高敞,人庶昌盛,因名高昌。亦云:其地有漢時高昌壘,故以為國號。東去長安四千九百里。漢西域長史及戊巳校尉並居於此。晉以其地為高昌郡。張軌、呂光、沮渠蒙遜據河西,皆置太守以統之。敦煌十三日行。



 國有八城,皆有華人。地多石磧,氣候溫暖,厥土良沃,穀麥一歲再熟,宜蠶,多五果,又饒漆。有草名羊刺,其上生蜜。而味甚佳。引水溉田。出赤鹽,其味甚美,復有白鹽,其形如玉,高昌人取以為枕,貢之中國。多蒲桃酒。俗事天神,兼信佛法。國中羊、馬,牧在隱
 僻處以避寇,非貴人不知其處。北有赤石山,山北七十里有貪汗山,夏有積雪。此山北,鐵勒界也。



 太武時有闞爽者,自為高昌太守。太延中,遣散騎侍郎王恩生等使高昌,為蠕蠕所執。真君中,爽為沮渠無諱所襲,奪據之。無諱死,弟安周代立。和平元年,為蠕蠕所并。蠕蠕以闞伯周為高昌王,其稱王自此始也。



 太和初,伯周死,子義成立。歲餘,為從兄首歸所殺,自立為高昌王。五年,高車王阿至羅殺首歸兄弟,以敦煌人張孟明為王。後為國人所殺,立馬儒為王,以鞏顧禮、麴嘉為左右長史。二十一年,遣司馬王體玄奉表朝貢,請師逆接,求舉國內徙。孝文
 納之,遣明威將軍韓安保率騎千餘赴之,割伊吾五百里,以儒居之。至羊榛水,儒遣嘉、禮率步騎一千五百迎安保。去高昌四百里而安保不至。禮等還高昌,安保亦還伊吾。安保遣使韓興安等十二人使高昌,儒復遣顧禮將其世子義舒迎安保。至白棘城,去高昌百六十里。而高昌舊人情戀本土,不願東遷,相與殺儒而立麴嘉為王。



 嘉字靈鳳,金城榆中人。既立,又臣於蠕蠕那蓋。顧禮與義舒隨安保至洛陽。



 及蠕蠕主伏圖為高車所殺,嘉又臣高車。初,前部胡人悉為高車所徙,入於焉耆,又為嚈噠所破滅,國人分散,眾不自立,請主於嘉。嘉遣第
 二子為焉耆王以主之。



 永平元年,嘉遣兄子私署左衛將軍、田地太守孝亮朝京師,仍求內徙,乞軍迎援。



 於是遣龍驤將軍孟威發涼州兵三千人迎之,至伊吾,失期而反。於後十餘遣使獻珠像、白黑貂裘、名馬、鹽枕等,款誠備至。唯賜優旨,卒不重迎。三年,嘉遣使朝貢,宣武又遣孟威使詔勞之。延昌中,以嘉為持節、平西將軍、瓜州刺史、泰臨縣開國伯,私署王如故。熙平初,遣使朝獻。詔曰:「卿地隔關山,境接荒漠,頻請朝援,徙國內遷。雖來誠可嘉,即於理未帖。何者?彼之庶,是漢、魏遺黎,自晉氏不綱,因難播越,成家立國,世積已久。惡徙重遷,人懷戀舊。
 今若動之,恐異同之變,爰在肘腋,不得便如來表也。」神龜元年冬,孝亮復表求援內徙,朝廷不許。正光元年,明帝遺假員外將軍趙義等使於嘉。嘉朝貢不絕,又遣使奉表,自以邊遐,不習典誥,求借《五經》、諸史,并請國子助教劉燮以為博士,明帝許之。



 嘉死,贈鎮西將軍、涼州刺史。



 子堅立。於後關中賊亂,使命遂絕。普泰初,堅遣使朝貢,除平西將軍、瓜州刺史,泰臨縣伯,王如故。又加衛將軍。至永熙中,特除儀同三司,進為郡公。後遂隔絕。至大統十四年,詔以其世子玄嘉為王。恭帝二年,又以其田地公茂嗣位。



 武成元年,其王遣使獻方物。保定初,又遣
 使來貢。



 其國,周時,城有一十六。後至隋時,城有十八。其都城周回一千八百四十步,於坐室畫魯哀公問政於孔子之像。官有令尹一人,比中夏相國;次有公二人,皆王子也,一為交河公,一為田地公;次有左右衛;次有八長史,曰吏部、祠部、庫部、倉部、主客、禮部、戶部、兵部等長史也;次有五將軍,曰建武、威遠、陵江、殿中、伏波等將軍也;次有八司馬,長史之副也;次有侍郎、校郎、主簿、從事,階位相次,分掌諸事。次有省事,專掌導引。其大事決之於王,小事則世子及二公隨狀斷決。評章錄記,事訖即除,籍書之外,無久掌文案。官人雖有列位,並無曹府,唯有每
 早集於牙門,評議眾事。諸城各有戶曹、水曹、田曹。城遣司馬、侍郎相監檢校,名為令。服飾,丈夫從胡法,婦人裙襦,頭上作髻。其風俗政令,與華夏略同,兵器有弓、刀、箭、楯、甲、槊。文字亦同華夏,兼用胡書。有《毛詩》、《論語》、《孝經》,置學官弟子,以相教授。雖習讀之,而皆為胡語。賦稅則計田輸銀錢,無者輸麻布。其刑法、風俗、昏姻、喪葬與華夏小異而大同。自敦煌向其國,多沙磧,茫然無有蹊徑,欲往者,尋其人畜骸骨而去。路中或聞歌哭聲,行人尋之,多致亡失,蓋魑魅魍魎也。故商客往來,多取伊吾路。



 開皇十年,突厥破其四城,有二千人來歸中國。



 堅死,子伯
 雅立。其大母本突厥可汗女,其父死,突厥令依其俗。伯雅不從者久之。突厥逼之,不得已而從。煬帝即位,引致諸蕃。



 大業四年,遣使貢獻,帝待其使甚厚。明年伯雅來朝,因從擊高麗。還,尚宗室女華容公主。八年冬,歸蕃,下令國中曰:「先者,以國處邊荒境,被髮左衽。



 今大隋統御,宇宙平一。孤既沐浴和風,庶均大化。其庶人以上,皆宜解辮削衽。」



 帝聞而善之,下詔曰:「光祿大夫、弁國公、高昌王伯雅,本自諸華,世祚西壤,昔因多難,翦為胡服。自我皇隋,平一宇宙,伯雅踰沙忘阻,奉貢來庭,削衽曳裾,變夷從夏,可賜衣冠,仍班製造之式。」然伯雅先臣鐵勒,恆
 遣重臣在高昌國,有商胡往來者則稅之,送于鐵勒。雖有此令取悅中華,然竟畏鐵勒,不敢改也。自是歲令貢方物。



 且彌國,都天山東于大谷,在車師北,去代一萬五百七十里。本役屬車師。



 焉耆國,在車師南都員渠城,白山南七十里,漢時舊國也,去代一萬二百里。



 其王姓龍,名鳩尸畢那,即前涼張軌所討龍熙之胤。所都城方二里。國內凡有九城。



 國小人貧,無綱紀法令。兵有弓、刀、甲、槊。婚姻略同華夏。死亡者,皆焚而後葬,其服制滿七日則除之。丈夫並翦髮以
 為首飾。文字與婆羅門同。俗事天神,並崇信佛法也。尤重二月八日、四月八日。是日也,其國咸依釋教,齋戒行道焉。氣候寒,土田良沃,穀有稻、粟、菽、麥,畜有駝、馬。養蠶,不以為絲,唯充綿纊。



 俗尚蒲桃酒,兼愛音樂。南去海十餘里,有魚鹽蒲葦之饒。東去高昌九百里,西去龜茲九百里,皆沙磧。東南去瓜州二千二百里。



 恃地多險,頗剽劫中國使。太武怒之,詔成周公萬度歸討之,約齎輕糧,取食路次。度歸入焉耆東界,擊其邊守左回、尉犁二城,拔之,進軍圍員渠。鳩尸畢那以四五萬人出城,守險以距。度歸募壯勇,短兵直往衝,鳩尸畢那眾大潰,盡虜之,
 單騎走入山中。度歸進屠其城,四鄙諸戎皆降服。焉耆為國,斗絕一隅,不亂日久,獲其珍奇異玩,殊方譎詭難識之物,橐駝、馬、牛、雜畜巨萬。時太武幸陰山北宮,度歸破焉耆露板至,帝省訖,賜司徒崔浩書曰:「萬度歸以五千騎,經萬餘里,拔焉耆三城,獲其珍奇異物及諸委積不可勝數。自古帝王,雖云即序西戎,有如指注,不能控引也。朕今手把而有之,如何?」浩上書稱美。遂命度歸鎮撫其人。初,鳩尸畢那走山中,猶覬城不拔,得還其國。既見盡為度歸所剋,乃奔龜茲。龜茲以其婿,厚待之。



 周保定四年,其王遣使獻名焉。



 隋大業中,其王龍突騎支遣
 使貢方物。是時,其國勝兵千餘人而已。



 龜茲國,在尉犁西北,白山之南一百七十里,都延城,漢時舊國也,去代一萬二百八十里。其王姓白,即後涼呂光所立白震之後。其王頭繫彩帶,垂之於後,坐金師子床。所居城方五六里。其刑法,殺人者死,劫賊則斷其一臂,並刖一足。賦稅,準地征租,無田者則稅銀。風俗、婚姻、喪葬、物產與焉耆略同,唯氣候少溫為異。又出細氈、燒銅、鐵、鉛、麖皮、氍毹、鐃沙、鹽綠、雌黃、胡粉、安息香、良馬、犎牛等。東有輪臺,即漢貳師將軍李廣利所屠者。其南三百里,有大河東流,號計戍水,即黃河也。東去焉耆九百里,
 南去于闐一千四百里,西去疏勒一千五百里,北去突厥牙六百餘里,東南去瓜州三千一百里。其東關城戍,寇竊非一,太武詔萬度歸率騎一千以擊之。龜茲遣烏羯目提等領兵三千距戰,度歸擊走之,斬二百餘級,大獲駝馬而還。俗性多淫,置女市,收男子錢以入官。土多孔雀,群飛山谷間,人取而食之,孳乳如雞鶩,其王家恆有千餘隻云。其國西北大山中有如膏者,流出成川,行數里入地,狀如食弟糊,甚臭。服之,髮齒已落者,能令更生,癘人服之,皆愈。自後每使朝貢。



 周保定元年,其王遣使來獻。



 隋大業中,其王白蘇尼巫遣使朝,貢方物。是時,其國勝
 兵可數千人。



 姑默國,居南城,在龜茲西,去代一萬五百里。役屬龜茲。



 溫宿國,居溫宿城,在姑默西北,去代一萬五百五十里。役屬龜茲。



 尉頭國,居尉頭城,在溫宿北,去代一萬六百五十里。役屬龜茲。



 烏孫國,居赤谷城,在龜茲西北,去代一萬八十里。其國數為蠕蠕所侵,西徙蔥嶺山中。無城郭,隨畜牧逐水草。



 太延三年,遣使者董琬等使其國,後每使朝貢。



 疏勒國,在姑默西,白山南百餘里,漢時舊國也。去代一
 萬一千二百五十里。



 文成末,其王遣使送釋迦牟尼佛袈裟一,長二丈餘。帝以審是佛衣,應有靈異,遂燒之以驗虛實,置於猛火之上,經日不然,觀者莫不悚駭,心形俱肅。其王戴金師子冠。土多稻、粟、麻、麥、銅、鐵、錫、雌黃,每歲常供送於突厥。其都城方五里。國內有大城十二,小城數十。人手足皆六指,產子非六指者即不育。勝兵者二千人。南有黃河,西帶蔥嶺,東去龜茲千五百里,西去鏺汗國千里,南去朱俱波八九百里,東北至突厥牙千餘里,東南去瓜州四千六百里。



 悅般國,在烏孫西北,去代一萬九百三十里。其先,匈奴
 北單于之部落也。為漢車騎將軍竇憲所逐,北單于度金微山西走康居,其羸弱不能去者,住龜茲北。地方數千里,眾可二十餘萬,涼州人猶謂之單于王。其風俗言語與高車同,而其人清潔於胡。俗翦髮齊眉,以食弟糊塗之,昱昱然光澤。日三澡漱,然後飲食。其國南界有火山,山傍石皆燋鎔,流地數十里乃凝堅,人取以為藥,即石流黃也。



 與蠕蠕結好,其王嘗將數千人入蠕蠕國,欲與大檀相見。入其界百餘里,見其部人不浣衣,不絆髮,不洗手,婦人口舐器物。王謂其從臣曰:「汝曹誑我,將我入此狗國中。」乃馳還。大檀遣騎追之,不及。自是相仇讎,數
 相征討。



 真君九年,遣使朝獻。并送幻人,稱能割人喉脈令斷,擊人頭令骨陷,皆血出或數升或盈斗,以草藥內其口中,令嚼咽之,須臾血止,養瘡一月復常,又無痕瘢。



 世疑其虛,乃取死罪囚試之,皆驗。云中國諸名山皆有此草,乃使人受其術而厚遇之。又言:其國有大術者,蠕蠕來抄掠,術人能作霖雨、盲風、大雪及行潦,蠕蠕凍死漂亡者十二三。是歲,再遣使朝貢,求與官軍東西齊契討蠕蠕。太武嘉其意,命中外諸軍戒嚴,以淮南王佗為前鋒,襲蠕蠕。仍詔有司,以其鼓舞之節,施於樂府。自後每使朝貢。



 者至拔國,都者至拔城,在疏勒西,去代一萬一千六百二十里。其國東有潘賀那山,出美鐵及師子。



 迷密國,都迷密城,在者至拔西,去代一萬二千一百里。正平元年,遣使獻一峰黑橐駝。其國東有山名郁悉滿山,出金、玉,亦多鐵。



 悉萬斤國,都悉萬斤城,在迷密西,去代一萬二千七百二十里。其國南有山名伽色那山,出師子。每使朝貢。



 忸密國,都忸密城,在悉萬斤西,去代二萬二千八百二十八里。



 破洛那國,故大宛國也。都貴山城,在疏勒西北,去代萬四
 千四百五十里。



 太和三年,遣使獻汗血馬,自此每使朝貢。



 粟特國,在蔥嶺之西,古之奄蔡,一名溫那沙,居於大澤,在康居西北,去代一萬六千里。先是,匈奴殺其王而有其國,至王忽倪,已三世矣。其國商人先多詣涼土販貨,及魏克姑臧,悉見虜。文成初,粟特王遣使請贖之,詔聽焉。自後無使朝獻。



 周保定四年,其王遣使貢方物。



 波斯國,都宿利城,在忸密西,古條支國也。去代二萬四千二百二十八里。城方十里,戶十餘萬,河經其城中南流。土地平正,出金、銀、金俞石、珊瑚、琥珀、車渠、馬腦,多大真珠、
 頗梨、琉璃、水精、瑟瑟、金剛、火齊、鑌鐵、銅、錫、朱砂、水銀、綾、錦、疊、毼、氍毹、毾、赤麞皮、及薰六、鬱金、蘇合、青木等香,胡椒、蓽撥、石蜜、千年棗、香附子、訶梨勒、無食子、鹽綠、雌黃等物。



 氣候暑熱,家自藏冰。地多沙磧,引水溉灌。其五穀及鳥獸等與中夏略同,唯無稻及黍、稷。土出名馬、大驢及駝,往往有一日能行七百里者,富室至有數千頭。又出白象、師子、大鳥卵。有鳥形如橐駝,有兩翼,飛而不能高,食草與肉,亦能敢火。其王姓波氏名斯,坐金羊床,戴金花冠,衣錦袍、織成帔,飾以真珠寶物。



 其俗:丈夫翦髮,戴白皮帽,貫頭衫,兩箱近下開之,亦有巾帔,緣以織
 成;婦女服大衫,披大帔,其髮前為髻,後披之,飾以金銀花,仍貫五色珠,絡之於膊。王於其國內別有小牙十餘所,猶中國之離宮也。每年四月出遊處之,十月仍還。王即位以後,擇諸子內賢者,密書其名,封之於庫,諸子及大臣莫之知也。王死,眾乃共發書視之,其封內有名者,即立以為王。餘子出各就邊任,兄弟更不相見也。國人號王曰醫贊,妃曰防步率,王之諸子曰殺野。大官有摸胡壇,掌國內獄訟;泥忽汗,掌庫藏、關禁;地卑,掌文書及眾務。次有遏羅訶地,掌王之內事;薛波勃,掌四方兵馬,其下皆有屬官,分統其事。兵有甲、槊、圓排、劍、弩、弓、箭。戰
 兼乘象,百人隨之。



 其刑法:重罪懸諸竿上,射殺之;次則系獄,新王立,乃釋之;輕罪則劓、刖若髡,或翦半鬢及繫牌於項,以為恥辱;犯強盜,繫之終身;姦貴人妻者,男子流,婦人割其耳鼻。賦稅,則準地輸銀錢。俗事火神天神。文字與胡書異。多以姊妹為妻妾,自餘婚合,亦不擇尊卑,諸夷之中最為醜穢矣。百姓女年十歲以上有姿貌者,王收養之,有功勛人,即以分賜。死者,多棄屍於山,一月著服。城外有人別居,唯知喪葬之事,號為不凈人。若入城市,搖鈴自別。以六月為歲首,尤重七月七日、十二月一日。其日,人庶以上,各相命召,設會作樂,以極懽娛。
 又每年正月二十日,各祭其先死者。



 神龜中,其國遣使上書貢物,云:「大國天子,天之所生,願日出處常為漢中天子。波斯國王居和多千萬敬拜。」朝廷嘉納之。自此,每使朝獻。恭帝二年,其王又遣使獻方物。



 隋煬帝時,遣雲騎尉李昱使通波斯。尋使隨昱貢方物。



 伏盧尼國,都伏盧尼城,在波斯國北,去代二萬七千三百二十里。累石為城,東有大河南流,中有鳥,其形似人,亦有如橐駝、馬者,皆有翼,常居水中,出水便死。城北有云尼山,出銀、珊瑚、琥珀,多師子。



 色知顯國,都色知顯城,在悉萬斤西北,去代一萬二千
 九百四十里。土平,多五果。



 伽色尼國,都伽色尼城,在悉萬斤南,去代一萬二千九百里。土出赤鹽,多五果。



 薄知國,都薄知城,在伽色尼國南,去代一萬三千三百二十里。多五果。



 牟知國,都牟知城,在忸密西南,去代二萬二千九百二十里。土平,禽獸草木類中國。



 阿弗太汗國,都阿弗太汗城,在忸密西,去代二萬三千七百二十里。土平,多五果。



 呼似密國,都呼似密城,在阿弗太汗西,去代二萬四千
 七百里。土平,出銀、琥珀,有師子,多五果。



 諾色波羅國,都波羅城,在忸密南,去代二萬三千四百二十八里。土平,宜稻、麥,多五果。



 早伽至國,都早伽至成,在忸密西,去代二萬三千七百二十八里。土平,少田殖,取稻、麥於鄰國,有五果。



 伽不單國,都伽不單城,在悉萬斤西北,去代一萬二千七百八十里。土平,宜稻、麥,有五果。



 者舌國,故康居國,在破洛那西北,去代一萬五千四百五十里。太延三年,遣使朝貢,不絕。



 伽倍國,故休密翕侯,都和墨城,在莎車西,去代一萬三
 千里。人居山谷間。



 折薛莫孫國,故雙靡翕侯,都雙靡城,在伽倍西,去代一萬三千五百里。居山谷間。



 鉗敦國,故貴霜翕侯,都護澡城,在折薛莫孫西,去代一萬三千五百六十里,居山谷間。



 弗敵沙國,故肹頓翕侯,都薄茅城,在鉗敦西,去代一萬三千六百六十里。居山谷間。



 閻浮謁國,故高附翕侯,都高附城,在弗敵沙南,去代一萬三千七百六十里。



 居山谷間。



 大月氏國,都剩鹽氏城,在弗敵沙西,去代一萬四千五
 百里。北與蠕蠕接,數為所侵,遂西徙都薄羅城,去弗敵沙二千一百里。其王寄多羅勇武,遂興師越大山,南侵北天竺。自乾陀羅以北五國,盡役屬之。太武時,其國人商販京師,自云能鑄石為五色琉璃。於是採礦山中,於京師鑄之,既成,光澤乃美於西方來者。乃詔為行殿,容百餘人,光色映徹,觀者見之,莫不驚駭,以為神明所作。自此,國中琉璃遂賤,人不復珍之。



 安息國,在蔥嶺西,都蔚搜城。北與康居,西與波斯相接,在大月氏西北,去代二萬一千五百里。



 周天和二年,其王遣使朝獻。



 條支國,在安息西,去代二萬九千四百里。



 大秦國,一名黎軒,都安都城,從條支西渡海曲一萬里,去代三萬九千四百里。



 其海滂出,猶渤海也,而東西與渤海相望,蓋自然之理。地方六千里,居兩海之間。



 其地平正,人居星布。其王都城分為五城,各方五里,周六十里。王居中城,城置八臣,以主四方。而王城亦置八臣,分主四城。若謀國事及四方有不決者,則四城之臣,集議王所,王自聽之,然後施行。王三年一出觀風化。人有冤枉詣王訴訟者,當方之臣,小則讓責,大則黜退,令其舉賢人以代之。其人端正長大,衣服、車旗,擬儀中國,故外域
 謂之大秦。其土宜五穀、桑、麻,人務蠶、田。多璆琳、琅玕、神龜、白馬朱鬣、明珠、夜光璧。東南通交趾,又水道通益州永昌郡。多出異物。



 大秦西海水之西有河,河西南流。河西有南北山,山西有赤水,西有白玉山,玉山西有西王母山,玉為堂室云。從安息西界循海曲,亦至大秦,迴萬餘里。於彼國觀日月星辰,無異中國,而前史云:條支西行百里,日入處,失之遠矣。



 阿鉤羌國,在莎車西南,去代一萬三千里。國西有縣度山,其間四百里,中往往有棧道,下臨不測之深,人行以繩索相持而度,因以名之。土有五穀、諸果。



 市用錢為貨。
 居止立宮室。有兵器,土出金珠。



 波路國,在阿鉤羌西北,去代一萬三千九百里。其地濕熱,有蜀馬。土平,物產國俗與阿鉤羌同類焉。



 小月氏國,都富樓沙城,其王本大月氏王寄多羅子也。寄多羅為匈奴所逐,西徙。後令其子守此城,因號小月氏焉。在波路西南,去代一萬六千六百里。先居西平、張掖之間,被服頗與羌同。其俗以金銀錢為貨,隨畜牧移徙,亦類匈奴。其城東十里有佛塔,周三百五十步,高八十丈。自佛塔初建計至武定八年,八百四十二年,年謂百丈佛圖也。



 罽賓國,都善見城,在波路西南,去代一萬四千二百里。居在四山中,其地東西八百里,南北三百里。地平,溫和,有苜蓿、雜草、奇木、檀、槐、梓、竹。種五穀。糞園。田地下濕,生稻。冬食生菜。其人工巧,雕文刻鏤,織罽。有金、銀、銅、錫,以為器物。市用錢。他畜與諸國同。每使朝獻。



 吐呼羅國,去代一萬二千里。東至范陽國,西至悉萬斤國,中間相去二千里;南至連山,不知名,北至波斯國,中間相去一萬里。薄提城周匝六十里,城南有西流大水,名漢樓河。土宜五穀,有好馬、駝、騾。其王曾遣使朝貢。



 副貨國,去代一萬七千里。東至阿富使且國,西至沒誰
 國,中間相去一千里;南有連山,不知名,北至奇沙國,相去一干五百里。國中有副貨城,周匝七十里。



 宜五穀、蒲桃,唯有馬、駝、騾。國王有黃金殿,殿下有金駝七頭,各高三尺。其王遣使朝貢。



 南天竺國,去代三萬一千五百里。有伏醜城,周匝十里。城中出摩尼珠、珊瑚。



 城東三百里有拔賴城,城中出黃金、白真檀、石蜜、蒲桃。土宜五穀。



 宣武時,其國王婆羅化遣使獻駿馬、金、銀。自此,每使朝貢。



 疊伏羅國,去代三萬一千里。國中有勿悉城,城北有鹽奇水,西流。有白象。



 並有阿末黎木,皮中織作布。土宜五
 穀。



 宣武時,其國王伏陀末多遣使獻方物。自是,每使朝貢。



 拔豆國,去代五萬一千里。東至多勿當國,西至旃那國,中間相去七百五十里;南至罽陵伽國,北至弗那伏且國,中間相去九百里。國中出金、銀、雜寶、白象、水牛、犛牛、蒲桃、五果,土宜五穀。



 嚈噠國,大月氏之種類也,亦曰高車之別種。其原出於塞北。自金山而南,在于闐之西,都烏滸水南二百餘里,去長安一萬一百里。其王都拔底延城,蓋王舍城也。其城方十里餘,多寺塔,皆飾以金。風俗與突厥略同。其俗,
 兄弟共一妻,夫無兄弟者,妻戴一角帽,若有兄弟者,依其多少之數更加帽角焉。衣服類加以纓絡,頭皆翦髮。其語與蠕蠕、高車及諸胡不同。眾可有十萬,無城邑,依隨水草,以氈為屋,夏遷涼土,冬逐煖處,分其諸妻,各在別所,相去或二百、三百里。其王巡歷而行,每月一處。冬寒之時,三月不徙。王位不必傳子,子弟堪者,死便受之。



 其國無車,有輿,多駝、馬。用刑嚴急,偷盜無多少,皆腰斬,盜一責十。死者,富家累石為藏,貧者掘地而埋,隨身諸物,皆置塚內。其人凶悍,能鬥戰,西域康居、于闐、沙勒、安息及諸小國三十許,皆役屬之,號為大國。與蠕蠕婚姻。



 自
 太安以後,每遣使朝貢,正光末,遣貢師子一,至高平,遇萬俟醜奴反,因留之。醜奴平,送京師。永熙以後,朝獻遂絕。



 至大統十二年,遣使獻其方物。廢帝二年、周明帝二年,並遣使來獻。後為突厥所破,部落分散,職貢遂絕。至隋大業中,又遣使朝貢方物。



 其國去漕國千五百里,東去瓜州六千五百里。



 初,熙平中,明帝遣剩伏子統宋雲、沙門法力等使西域,訪求佛經,時有沙門慧生者,亦與俱行。正光中,還。慧生所經諸國,不能知其本末及山川里數,獸舉其略云。



 朱居國,在於闐西。其人山居,有麥,多林果。咸事佛,語與
 于闐相類,役屬嚈噠。



 渴槃陀國,在蔥嶺東,朱駒波西。河經其國東北流,有高山,夏積霜雪。亦事佛道,附於嚈噠。



 缽和國,在渴槃陀西。其土尤寒,人畜同居,穴地而處。又有大雪山,望若銀峰。其人唯食餅面,飲麥酒,服氈裘。有二道,一道西行向嚈噠,一道西南趣烏萇。



 亦為嚈噠所統。



 波知國,在缽和西南。土狹人貧,依託山谷,其王不能總攝。有三池,傳云大池有龍,次者有龍婦,小者有龍子,行人經之,設祭乃得過,不祭,多遇風雪之困。



 賒彌國,在波知之南。山居,不信佛法,專事諸神。亦附嚈噠。



 東有缽盧勒國,路險,緣鐵鎖而度,不下見底。熙平中,宋雲等竟不能達。



 烏萇國,在賒彌南。北有蔥嶺,南至天竺。婆羅門胡為其上族。婆羅門多解天文吉凶之數,其王動則訪決焉。土多林果,引水灌田,豐稻、麥。事佛,多諸寺塔,極華麗。人有爭訴,服之以藥,曲者發狂,直者無恙。為法不殺,犯死罪唯徙於靈山。西南有檀特山,山上立寺,以驢數頭運食山下,無人控御,自知往來也。



 乾陀國,在烏萇西。本名業波,為嚈噠所破,因改焉。其王
 本是敕勒,臨國已二世矣。好征戰,與罽賓斗,三年不罷,人怨苦之。有鬥象七百頭,十人乘一象,皆執兵仗,象鼻縛刀以戰。所都城東南七里有佛塔,高七十丈,周三百步,即所謂雀離佛圖也。



 康國者,康居之後也,遷徙無常,不恒故地,自漢以來,相承不絕。其王本姓溫,月氏人也,舊居祁連山北昭武城,因被匈奴所破,西踰蔥嶺,遂有國。枝庶各分王,故康國左右諸國並以昭武為姓,示不忘本也。王字世夫畢,為人寬厚,甚得眾心。其妻,突厥達度可汗女也。都於薩寶水上阿祿迪城。多人居,大臣三人,共掌國事。其王素冠
 七寶花,衣綾、羅、錦、繡、白疊。其妻有髮,幪以皁巾。丈夫翦髮,錦袍。名為彊國,西域諸國多歸之。米國、史國、曹國、何國、安國、小安國、那色波國、烏那曷國、穆國皆歸附之。有胡律,置於祅祠,將決罰,則取而斷之。重者族,次罪者死,賊盜截其足。人皆深目、高鼻、多髯。善商賈,諸夷交易,多湊其國。有大小鼓、琵琶、五絃、箜篌。婚姻喪制與突厥同。國立祖廟,以六月祭之,諸國皆助祭。奉佛,為胡書。氣候溫,宜五穀,勤脩園蔬,樹木滋茂。出馬、駝、驢、犎牛、黃金、硇沙、香、阿薩那香、瑟瑟、麞皮、氍、錦、疊。多蒲桃酒,富家或致千石,連年不敗。



 大業中,始遣使貢方物,後遂絕焉。



 安國,漢時安息國也。王姓昭武氏,與康國王同族,字設力;妻,康國王女也。



 都在那密水南,城有五重,環以流水,宮殿皆平頭。王坐金駝座,高七八尺,每聽政,與妻相對,大臣三人,評理國事。風俗同於康居,唯妻其姊妹及母子遞相禽獸,此為異也。



 隋煬帝即位,遣司隸從事杜行滿使西域,至其國,得五色鹽而返。



 國西百餘里有畢國,可千餘家。其國無君長,安國統之。大業五年,遣使貢獻。



 石國,居於藥殺水,都城方十餘里。其王姓石名涅。國城東南立屋,置座於中。



 正月六日,以王父母燒餘之骨,金甕盛置床上,巡繞而行,散以花香雜果,王率臣下設祭
 焉。禮終,王與夫人出就別帳,臣下以次列坐,享宴而罷。有粟、麥,多良馬。其俗善戰。曾貳於突厥,射匱可汗滅之,令特勤甸職攝其國事。南去鏺汗六百里,東南去瓜州六千里。



 甸職以隋大業五年遣使朝貢,後不復至。



 女國,在蔥嶺南。其國世以女為王,姓蘇毗,字末羯,在位二十年,女王夫號曰金聚,不知政事。國內丈夫,唯以征伐為務。山下為城,方五六里,人有萬家。



 王居九層之樓,侍女數百人,五日一聽朝,復有小女王共知國政。其俗婦人輕丈夫,而性不妒忌。男女皆以彩色塗面,而一日中或數度變改之。人皆被髮,以皮為鞋。



 課稅無常。氣候
 多寒,以射獵為業。出金俞石、朱砂、麝香、BX牛、駿馬、蜀馬。



 尤多鹽,恒將鹽向天竺興販,其利數倍。亦數與天竺、黨項戰爭。其女王死,國中厚斂金錢,求死者族中之賢女二人,一為女王,次為小王。貴人死剝皮,以金屑和骨肉置瓶中,埋之。經一年,又以其皮納鐵器埋之。俗事阿脩羅神,又有樹神,歲初以人祭,或用獼猴。祭畢,入山祝之,有一鳥如雌雉,來集掌上,破腹其視之,有眾粟則年豐,沙石則有災,謂之鳥卜。



 隋開皇六年,遣使朝貢,後遂絕。



 鏺汗國,都蔥嶺之西五百餘里,古渠搜國也。王姓昭武,字阿利柒。都城方四里,勝兵數千人。王坐金羊床,妻戴
 金花。俗多朱砂、金、鐵。東去疏勒千里,西去蘇對沙那國五百里,西北去石國五百里,東北去突厥可汗二千餘里,東去瓜州五千五百里。



 隋大業中,遣使貢方物。



 吐火羅國,都蔥嶺西五百里,與挹怛雜居。都城方二里,勝兵者十萬人,皆善戰。其俗奉佛。兄弟同一妻,迭寢焉,每一人入房,戶外掛其衣以為志,生子屬其長兄。其山穴中有神馬,每歲牧馬於穴所,必產名駒。南去漕國千七百里,東去瓜州五千八百里。



 大業中,遣使朝貢。



 米國,都那密水西,舊康居之地。無王,其城主姓昭武,康國王之支庶,字閉拙。都城方二里,勝兵數百人。西北去
 蘇對沙那國五百里,西南去史國二百里,東去瓜州六千四百里。



 大業中,頻貢方物。



 史國,都獨莫水南十里,舊康居之地也。其王姓照昭武,字狄遮,亦康國王之支庶也。都城方二里,勝兵千餘人。俗同康國。北去康國二百四十里,南吐火羅五百里,西去那色波國二百里,東北去米國二百里,東去瓜州六千五百里。



 大業中,遣使貢方物。



 曹國,都那密水南數里,舊是康居之地也。國無主,康國王令子烏建領之。都城方三里,勝兵千餘人。國中有得悉神,自西海以東諸國並敬事之,其神有金人,破羅闊
 丈有五尺,高下相稱,每日以駝五頭、馬十匹、羊一百口祭之,常有數千人,食之不盡。東南去康國百里,西去何國百五十里,東去瓜州六千六百里。



 大業中,遣使貢方物。



 何國,都那密水南數里,舊是康居地也。其王姓昭武,亦康國王之族類,字敦。



 都城方二里,勝兵者千人。其王坐金羊座。東去曹國百五十里,西去小安國三百里,東去瓜州六千七百五十里。



 大業中,遣使貢方物。



 烏那遏國,都烏滸水西,舊安息之地也。王姓昭武,亦康國王種類,字佛食。



 都城方二里,勝兵數百人。王坐金羊
 座。東北去安國四百里,西北去穆國二百餘里,東去瓜州七千五百里。



 大業中,遣使貢方物。



 穆國,都烏滸河之西,亦安息之故地,與烏那遏為鄰。其王姓昭武,亦康國王之種類也,字阿濫密。都城方三里,勝兵二千人。東北去安國五百里,東去烏那遏二百餘里,西去波斯國四千餘里,東去瓜州七千七百里。



 大業中,遣使貢方物。



 漕國,在蔥嶺之北,漢時罽賓國也。其王姓昭武,字順達,康國王之宗族也。



 都城方四里,勝兵者萬餘人。國法嚴,殺人及賊盜皆死。其俗重淫祠,蔥嶺山有順天神者,儀
 制極華,金銀鍱為屋,以銀為地,祠者日有千餘人。祠前有一魚脊骨,有孔,中通馬騎出入。國王戴金牛頭冠,坐金馬座。多稻、粟、豆、麥,饒象、馬、犎牛、金、銀、鑌鐵、氍、朱沙、青黛、安息青木等香、石蜜、黑監、阿魏、沒藥、白附子。去北帆延七百里,東去劫國六百里,東北去瓜州六千六百里。



 大業中,遣使貢方物。



 論曰:自古開遠夷,通絕域,必因宏放之主,皆起好事之臣。張騫鑿空於前,班超投筆於後,或結之以重寶,或懾之以利劍,投軀萬死之地,必要一旦之功,皆由主尚來遠之名,臣徇輕生之節。是知上之所好,不必效焉。西域
 雖通於魏氏,于時中原始平,天子方以混一為心,未遑及此。其信使往來,得羈縻勿絕之道。及隋煬帝規摹宏侈,掩吞秦、漢,裴矩方進《西域圖記》以蕩其心,故萬乘親出玉門關,置伊吾、且末鎮,而關右暨於流沙,騷然無聊生矣。若使北狄無虞,東夷告捷,必將脩輪臺之戍,築烏壘之城,求大秦之明珠,致條支之鳥卵,往來轉輸,將何以堪其弊哉!古者哲王之制也,方五千里,務安諸夏,不事要荒。豈威不能加,德不能被?蓋不以四夷勞中國,不以無用害有用也。是以秦戍五嶺,漢事三邊,或道殣相望,或戶口減半。隋室恃其強盛,亦狼狽於青海。此皆一
 人失其道,故億兆罹其苦。



 載思即敘之義,固辭都護之請,返其千里之馬,不求白狼之貢,則七戎九夷,候風重譯,雖無遼東之捷,豈及江都之禍乎!案西域開於往漢,年世積久,雖離並多端,見聞殊說,此所以前書後史,踳駁不同,豈其好異,地遠故也。人之所知,未若其所不知,信矣。但可取其梗概,夫何是非其間哉?



\end{pinyinscope}