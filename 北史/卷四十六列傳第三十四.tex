\article{卷四十六列傳第三十四}

\begin{pinyinscope}

 孫紹張普惠成淹範紹劉桃符鹿悆張耀劉道斌董紹馮元興孫紹,字世慶,昌黎人也。少好學,通涉經史。初為校書
 郎,稍遷給事中,後為門下錄事。好言得失,與常景共修律令。延昌中,紹表曰:臣聞建國有計,雖危必安;施化能和,雖寡必盛;政乖人理,雖合必離;作用失機,雖成必敗。此乃古今同然,百王之定法也。今二虢京門,了無嚴防;南北二中,復闕固守;長安、鄴城,股肱之寄;穰城、上黨,腹
 背所馮。四軍、五校之軌,領、護分事之式,徵兵儲粟之要,舟車水陸之資,山河要害之權,緩急去來之用,持平赴救之方,節用應時之法,特宜修置,以固堂堂之基。持盈之體,何得而忽?



 且法開清濁,而清濁不平;申滯理望,而卑寒亦免。士庶同悲,兵徒懷怨。中正賣望
 於
 下
 里,主案舞筆於上臺,真偽混淆,知而不糾,得者不欣,失者倍怨。使門齊身等而涇、渭奄殊,類應同役而苦樂縣異,士人居職,不以為榮;兵士役苦,必不忘亂。故有競棄本生,飄藏
 他土。或詭名託養,散沒人間;或亡命山藪,漁獵為命;或投杖強豪,寄命衣食。又應遷之戶,逐樂諸州;應留之徒,避寒歸暖。職人子弟,隨榮浮游,南北東西,卜居莫定。關禁不修,任意取適,如此之徒,不可勝數。爪牙不復為用,百工爭棄其業。混一之計,事實闕如;考課之方,責辦無日;流浪之徒,決須精校。今強敵窺時,邊黎伺隙,內人不平,久戍懷怨。戰國之勢,竊謂危矣。必造禍源者,北邊鎮戍之人也。若夫一統之年,持平用之者,大道之計也;亂離之期,縱橫作之者,行權之勢也。故道不可久,須文質以換情;權不可恆,隨污隆以牧物。文質應世,道形自安;
 污隆獲衷,權勢亦濟。然則王者計法之趣,化物之規,圓方務得其境,人物不失其地。又先帝時,律、令並議,律尋施行,令獨不出,十餘年矣。臣以令之為體,即帝王之身,分處百揆之儀,安置九服之節,乃是有為之樞機,世法之大本也。然修令之人,亦皆博古,依古撰置,大體可觀,比之前令,精粗有在。但主議之家,大用古制。若令依古,高祖之法,復須升降,誰敢措意有是非哉?以是爭故,久廢不理。然律、令相須,不可偏用,今律班令止,於事甚滯。若令不班,是無典法,臣下執事,何依而行?臣等修律,非無勤止,署下之日,臣乃無名,是謂農夫盡力,他食其秋,
 功名之所,實懷於悒。



 正光初,兼中書侍郎。紹性抗直,每上封事,常至懇切,不憚犯忤。但天性疏脫,言乍高下,時人輕之,不見採覽。紹兄世元善彈箏,早卒。紹後聞箏聲,便涕泗鳴咽,捨之而去。後為太府少卿,曾因朝見,靈太后謂曰:「卿年稍老矣。」紹曰:「臣年雖老,臣卿乃少。」太后笑之。遷右將軍、太中大夫。



 紹曾與百僚赴朝,東掖未開,守門候旦。紹於眾中引吏部郎中辛雄於眾外,竊謂曰:「此中諸人,尋當死盡,唯吾與卿,猶享富貴。」未幾,有河陰之難。紹善推祿命,事驗甚多,知者異之。



 永安中,拜太府卿,以前參議《正光壬子歷》,賜爵新昌子。後卒於右光祿大
 夫,贈尚書左僕射,謚曰宣。子伯元襲爵。



 張普惠,字洪賑,常山九門人也。身長八尺,容貌魁偉,精於《三禮》,兼善《春秋》、百家之說。太和十九年,為主書,帶制局監,頗為孝文所知。轉尚書都令史。任城王澄重其學業,為其聲價。澄為雍州刺史,啟普惠為府錄事參軍,尋行馮翊郡事。



 澄功衰在身,欲七月七日集文武北園馬射。普惠奏記於澄曰:竊聞三殺九親,別疏暱之敘;五服六術,等衰麻之心。皆因事飾情,不易之道者也。然則莫大之痛,深於終身之外;書策之哀,除於喪紀之內。外者不可無節,故斷之以三年;內者不可遂除,故敦之以日
 月。況《禮》,大練之日,鼓素琴,蓋推以即吉也;小功以上,非虞祔練除不沐浴,此拘之以制也。曾子問曰:「相識有喪服,可以與於祭乎?」孔子曰:「緦不祭,又何助於人。」祭既不與,疑無宴食之道。又曰:「廢喪服,可以與於饋奠之事乎?」子曰:「脫衰與奠,非禮也。」



 注云:「謂其忘哀疾。」愚謂除喪之始,不與饋奠,小功之內,其可觀射乎?《雜記》云:「大功以下,既葬適人,人食之。其黨也食之,非其黨不食。」食猶擇人,於馬射為或非宜。伏見明教,立射會之限,將以二七令辰,集城中文武肄武藝於北園,行揖讓於中否。時非大閱之秋,景涉妨農之節,國家縞禫甫除,殿下功衰仍襲,
 釋而為樂,以訓百姓,便是易先王之典教,忘哀戚之情,恐非所以昭令德、視子孫者也。案射儀,射者以禮樂為本,忘而從事,不可謂禮;鐘鼓弗設,不可謂樂。捨此二事,何用射為!又七日之戲,令制無之,班勞所施,慮違事體,府庫空虛,宜待新調。乞至九月,備飾盡行,然後奏《狸首》之章,宣矍相之命,聲軒縣,建雲鉦,神人忻暢於斯時也。



 澄意納其言,託辭自罷,乃答曰:「今雖非公制,而此州承前已有斯式。且纂文習武,人之常藝。豈可於常藝之間,要須令制乎?《禮》,兄弟內除,明哀已殺;小功,客至主不絕樂。聽樂則可,觀武豈傷?直自事緣須罷,先以令停,方獲
 此請,深具來意。」



 澄轉揚州,啟普惠以羽林監領鎮南大將軍開府主簿。普惠既為澄知,歷佐二籓,甚有聲譽。還朝,仍羽林監。



 澄遭太妃憂,臣僚為立碑頌,題碑欲云「康王元妃之碑」。澄訪於普惠,普惠答曰:「謹尋朝典,但有王妃,而無元字。魯夫人孟子稱元妃者,欲下與繼室聲子相對。今烈懿太妃作配先王,更無聲子、仲子之嫌,竊謂不假元字以別名位。且以氏配姓,愚以為在生之稱,故《春秋》『夫人姜氏至自齊』;既葬,以謚配姓,故經書『葬我小君文姜』,又曰『來歸夫人成風之襚』,皆以謚配姓。古者婦人從夫謚,今烈懿太妃德冠一世,故特蒙褒錫,乃萬代
 之高事,豈容於定名之重,而不稱『烈懿』乎。」澄從之。



 後為步兵校尉,以本官領河南尹丞。宣武崩,坐與甄楷等飲酒游從,免官。故事,免官者,三載之後,降一階而敘,若才優擢授,不拘此限。熙平中,吏部尚書李韶奏普惠有文學。依才優之例,敕除寧遠將軍、司空倉曹參軍。朝議以不降階為榮。時任城王澄為司空,表議書記多出普惠。



 廣陵王恭、北海王顥疑為所生祖母服期與三年,詔群僚會議。普惠議曰:謹案:二王祖母皆受命先朝,為二國太妃,可謂受命於天子,為始封之母矣。



 《喪服》「慈母如母」,在三年章,傳曰:「貴父命也。」鄭注云:「大夫之妾子,父在為
 母大功,則士之妾子為母期。父卒,則皆得伸。」此大夫命其妾子,以為母所慈,猶曰貴父命,為之三年;況天子命其子為列國王,命其所生母為國太妃,反自同公子為母練冠之與大功乎。《傳》曰:「始封之君,不臣諸父昆弟。」則當服其親服。若魯、衛列國,相為服期,判無疑矣。何以明之?《喪服》:「君為姑姊妹女子子嫁於國君者。」《傳》曰:「何以大功?尊同也。尊同,則得服其親服。



 諸侯之子稱公子,公子不得禰先君。」然則兄弟一體,位列諸侯,自以尊同,得相為服,不可還準公子,遠厭天王。故降有四品,君、大夫以尊降,公子、大夫之子以厭降,名例不同,何可亂也。《禮》,大
 夫之妾子,以父命慈己,申其三年。太妃既受命先帝,光昭一國,二王胙土茅社,顯錫大邦,舍尊同之高據,附不禰之公子,雖許、蔡失位,亦不是過。《服問》曰:「有從輕而重,公子之妻,為其皇姑。」



 公子雖厭,妻尚獲申,況廣陵、北海,論封君則封君之子,語妃則命妃之孫,承妃纂重,遠別先皇,更以先后之正統,厭其所生之祖嫡,方之皇姑,不以遙乎?今既許其申服,而復限之以期,比之慈母,不亦爽歟?《經》曰:「為君之祖父母、父母、妻、長子」,《傳》曰:「何以期?父母長子君服斬。妻則小君。父卒,然後為祖後者,服斬。」今祖乃獻文皇帝,諸侯不得祖之。母為太妃,蓋二王三年
 之證。



 議者近背正經,以附非類,差之毫毛,所失或遠。且天子尊則配天,莫非臣妾,何為命之為國母,而不聽子服其親乎?《記》曰:「從服者,所從亡則已。」又曰:「不為君母之黨服,則為其母之黨服。今所從既亡,不以親服服其所生,則屬從之服,於何所施?若以諸王入為公卿,便同大夫者,則當今之議皆不須以國為言也。



 今之諸王,自同列國,雖不之國,別置臣僚,玉食一方,得不以諸侯言之?



 敢據《周禮》,輒同三年。



 當時議者,亦有同異。國子博士李郁於議罷之後,書難普惠,普惠據《禮》還答,鄭重三反,郁議遂屈。轉諫議大夫,澄謂普惠曰:「不喜君得諫議,唯喜
 諫議得君。」



 時靈太后父司徒胡國珍薨,贈相國、太上秦公。普惠以前世后父無太上之號,詣闕上疏,陳其不可。左右畏懼,莫敢為通。會聞胡家穿壙下墳有盤石,乃密表曰:「竊見故侍中、司徒胡公,懷道含靈,實誕聖后,近樞剋惟允之寄,居槐體論道之明。故以功餘九錫,褒假鸞纛,深聖上之加隆,極慈后之至愛,憲章天下,不亦可乎!而太上之號,竊謂未衷。何者?《禮記》曰:『天無二日,土無二王,嘗禘郊社,尊無二上。』竊謂高祖受禪於獻文皇帝,故仰尊為太上皇,此因上上而生名也。



 皇太后稱令以繫敕下,蓋取三從之道,遠同文母,列於十亂,則司徒為太
 上,恐乖繫敕之意。《易》曰『困於上者,必反於下。』比剋吉定兆,而以淺改卜,群心悲惋,亦或天地神靈所以垂至戒,啟聖情。伏願停司徒逼同之號,從卑下不踰之稱,則天下幸甚。」



 太后覽表,親至國珍宅,召集五品已上博議其事。任城王澄、太傅清河王懌、侍中崔光、御史中尉元匡、尚書崔亮並同有難,普惠並以理正之,無所屈。廷尉少卿袁翻曰:「《周官》:上公九命,上大夫四命,命數雖殊,同名為上,何必上者皆是極尊。」普惠厲聲呵翻曰:「禮有下卿、上士,何止大夫與公。但今所行,以太加上,二名雙舉,不得非極。雕蟲小藝,微或相許,至於此處,豈卿所及!」翻甚
 有慚色,默不復言。議者咸以太后當朝,志相黨順,遂奏曰:「張普惠辭雖不屈,然非臣等所同。渙汗已流,請依前詔。」太后復遣元叉、賈璨宣令謂普惠曰:「朕之所行,孝子之志;卿之所陳,忠臣之道。群公已有成議,卿不得苦奪朕懷。後有所見,勿得難言。」



 初,普惠被召,傳詔馳驊騮馬來,甚迅速,佇立催去。普惠諸子憂怖涕泗。普惠謂曰:「我當休明之朝,掌諫議之任,若不言所難言,諫所難諫,便是唯唯,曠官尸祿。人生有死,死得其所,夫復何恨。然朝廷有道,汝輩勿憂。」及議罷,旨勞還宅,親故賀其幸甚。



 時中山杜弼遺書普惠曰:「明侯深儒碩學,身負大才,執此
 公方,來居諫職,謇謇如也,諤諤如也。一昨承在胡司徒弟,當庭面諍,雖問難鋒至,而應對響出。



 宋城之帶始縈,魯門之柝裁警,終使群后逡巡,庶僚拱默。雖不見用於一時,固已傳美於百代。聞風快然,敬裁此白。」普惠美其此書,每為口實。



 普惠以天下人調,幅度長廣,尚書計奏,復徵綿麻,恐人不堪命。上疏曰:「伏聞尚書奏復綿麻之調,遵先皇之軌。夙宵惟度,欣戰交集。仰惟高祖廢大斗,去長尺,改重秤,所以愛萬姓,從薄賦。知軍國須綿麻之用,故云幅度之間,億兆應有綿麻之利,故絹上稅綿八兩,布上稅麻十五斤。萬姓得廢大斗,去長尺,改重秤,荷
 輕賦之饒,不適於綿麻而已。故歌舞以供其賦,奔走以役其勤。夫信行於上,則億兆樂輸於下。自茲已降,漸漸長闊,百姓嗟怨,聞於朝野。伏惟皇太后未臨朝之前,陛下居諒闇之日,宰輔不尋其本,知天下之怨綿麻,不察其幅廣、度長、秤重、斗大,革其所弊,存其可存,而特放綿麻之調,以悅天下之心。此謂悅之不以道,愚臣所以未悅者也。」



 普惠又表乞朝直之日,時聽奉見。自此之後,月一陛見。又以孝明不親視朝,過崇佛法,郊廟之事,多委有司,上疏曰:「伏惟陛下重暉纂統,欽明文思,天地屬心,百神佇望。伏願躬致郊廟之虔,親紆朔望之澤,釋奠成
 均,竭心千畝,明發不寐,潔誠禋祼,孝弟可以通神明,德教可以光四海。然後精進三寶,信心如來。



 道由化深,故諸漏可盡;法隨禮積,故彼岸可登。量撤僧寺不急之華,還復百官久折之秩。已興之構,務從簡成;將來之造,權令停息。但仍舊貫,亦何必改作。庶節用愛人,法俗俱賴。」尋別敕付外,議釋奠之禮。



 時史官剋日蝕,豫敕罷朝。普惠以逆廢非禮,上疏陳之。又表論時政得失:一曰審法度,平斗尺,租調務輕,賦役務省。二曰聽輿言,察怨訟,先皇舊事有不便於政者,請悉追改。三曰進忠謇,退不肖,任賢勿貳,去邪勿疑。四曰興滅國,繼絕世,勳親之胤,所
 宜收敘。書奏,孝明、靈太后引普惠於宣光殿,隨事難詰。延對移時,太后曰:「小小細務,一一翻動,更成煩擾。」普惠曰:「聖上之養庶物,若慈母之養赤子。今赤子幾臨危壑,將赴水火,以煩勞而不救,豈赤子所望於慈母!」



 太后曰:「天下蒼生,寧有如此苦事?」普惠曰:「天下之親懿,莫重於太師彭城王,然遂不免枉死。微細之苦,何可得無?」太后曰:「彭城之苦,吾已封其三子,何足復言。」普惠曰:「聖后封彭城之三子,天下莫不忻至德,知慈母之在上。臣所以重陳者,凡如此枉,乞垂聖察。」太后曰:「卿云興滅繼絕,意復誰是?」普惠曰:「昔淮南逆終,漢文封其四子,蓋骨肉之
 不可棄,親親故也。竊見咸陽、京兆,乃皇子皇孫,一德之虧,自貽悔戾;沈淪幽壤,緬焉弗收,豈是興滅繼絕之意?」



 太后曰:「卿言有理,當命公卿博議。」



 及任城王澄薨,普惠荷其恩待,朔望奔赴,至於禫除,雖寒暑風雨,無不必至。



 初,澄嘉賞普惠,臨薨啟為尚書右丞。靈太后既深悼澄,覽啟從之。詔行之後,尚書諸郎以普惠地寒,不應便居管轄,相與為約,並欲不放上省,紛紜多日乃息。



 正光二年,詔遣楊鈞送蠕蠕主阿那瑰還國。普惠謂遣之將貽後患,上疏極言其不可。表奏不從。魏子建為益州刺史,有贓罪,普惠被使驗之,事遂得釋,故子建父子甚德
 之。時梁西豐侯正德詐稱降款,朝廷頗事當迎。普惠請付揚州,移還蕭氏,不從。俄而正德果逃還。後除光祿大夫,右丞如故。



 先是,仇池武興郡氐數反,西垂郡戍,租連久絕。詔普惠以本官為持節、西道行臺,給秦、岐、涇、華、雍、豳、東秦七州兵武三萬人,任其召發;送南秦、東益二州兵租,分付諸戍。其所部將統,聽於關西牧守之中隨機召遣。軍資板印之屬,悉以自隨。事訖還朝,賜絹布一百段。時詔訪冤屈,普惠上疏,多所陳論。出除東豫州刺史。淮南九戍十三郡,猶因梁前弊,別郡異縣之人錯雜居止。普惠乃依次括比,省減郡縣,上表陳狀,詔許之。宰守
 因此,綰攝有方,奸盜不起,人以為便。



 普惠不營財業,好有進舉,敦於故舊。冀州人侯堅固少時與其游學,早終。其子長瑜,普惠每於四時請祿,無不減贍,給其衣食。及為豫州,啟長瑜解褐,攜其合門拯給之。在州卒,謚曰宣恭。



 成淹,字季文,上谷居庸人也。好文學,有氣尚。仕宋為員外郎,領軍主,援東陽、歷城。皇興中,降慕容白曜,赴闕,授兼著作佐郎。時獻文於仲冬月欲巡漠北,朝臣以寒甚固諫,並不納。淹上《接輿釋游論》,帝覽之,詔尚書李曰:「卿諸人不如成淹論,通釋人意。」乃敕停行。



 太和中,文明
 太后崩,齊遣其散騎常侍裴昭明、散騎侍郎謝竣等來弔,欲以朝服行事。主客不許,昭明等執志不移。孝文敕尚書李沖選一學識者更與論執。沖奏遣淹。昭明言:「不聽朝服行禮,義出何典?」淹言:「玄冠不弔,童孺共聞。昔季孫將行,請遭喪之禮,千載之下,猶共稱之。卿方謂義出何典,何其異哉!」昭明言:「齊高帝崩,魏遣李彪通弔,初不素服,齊朝亦不為疑。」淹言:「彪通弔之日,朝命以弔服自隨。彼不遵高宗追遠之慕,乃踰月即吉。齊之君臣,皆已鳴玉盈庭,彪行人,何容獨以衰服間衣冠之中?我皇處諒暗以來,百官聽於冢宰,卿豈得以此方彼也?」昭明乃
 搖膝而言曰:「三皇不同禮,亦安知得失所歸。」淹言:「若如來談,卿以虞舜、高宗為非也?」昭明相顧笑曰:「非孝者,宣尼有成責,行人亦弗敢言。使人唯齎褲褶,不可以弔,幸借衣颻,以申國命。今為魏朝所逼,還南日,必得罪本朝。」淹言:「彼有君子也,卿將命折中,還南日,應有高賞。



 若無君子也,但令有光國之譽,雖非理得罪,亦復何嫌。南史、董狐,自當直筆。」



 既而敕送衣颻給昭明等,明旦引入,皆令文武盡哀。後正佐郎。



 其後,齊遣其散騎常侍庾蓽、散騎侍郎何憲、主書邢宗慶等來聘,孝文敕淹接於外館。宗慶語淹言:「南北連和既久,而比棄信絕好,為利而動,
 豈是大國善鄰之義?」淹言:「夫為王者不拘小節,豈得眷眷守尾生之信!且齊先主歷事宋朝,當應便爾欺奪?」宗慶、庾蓽及從者皆相顧失色。何憲知淹昔從南入,以手掩目曰:「卿何不作于禁而作魯肅?」淹言:「我捨逆效順,欲追蹤陳、韓,何于禁之有!」



 憲亦不對。



 王肅之至,鑾輿行幸。肅多扈從,敕淹將引,若有古跡,皆使知之。行到朝歌,肅問:「此是何城?」淹言:「紂都朝歌城。」肅言:「故應有殷之頑人。」淹言:「昔武王滅紂,悉居河洛,中因劉、石亂華,仍隨司馬東度。」肅知淹寓青州,乃笑謂曰:「青州何必無其餘種。」淹以肅本隸徐州:「若言青州,本非其地,徐州間今日重來,
 非所知也。」肅遂伏馬上掩口笑,顧謂侍御史張思寧曰:「向聊因戲言,遂致辭溺。」思寧馳馬以聞,孝文大悅,謂彭城王勰曰:「淹此段足為制勝。」



 輿駕至洛,肅因侍宴,帝戲肅曰:「近者行次朝歌,聞成淹共卿殊有往復,卿試重敘之。」肅言:「臣於朝歌失言,一之已甚,豈宜再說。」遂大笑。肅又言淹才詞,宜應敘進。帝言:「若因此進淹,恐辱卿轉甚。」肅言:「臣屈己達人,正可顯臣之美。」帝曰:「卿為人所屈,欲求屈己之名,復於卿大優。」肅言:「淹既蒙進,臣得屈己申人,此所謂陛下惠而不費。」遂酣笑而止。賜淹龍廄上馬一匹,并鞍勒宛具,朝服一襲。轉謁者僕射。



 時遷都,帝以
 淹家貧,敕給事力,送至洛陽,使與家累相隨。及車駕濟淮,敕征淹。淹於路左請見,曰:「敵不可小,願聖明保萬全之策。伏聞發洛已來,諸有諫者,解官奪職,恐非聖明納下之義。」帝優而容之。



 帝幸徐州,敕淹與閭龍駒專主舟楫,將汎泗入河,溯流還洛。軍次磝碻,淹以黃河浚急,慮有傾危,乃上疏陳諫。帝敕淹曰:「朕以恒、代無運漕之路,故京邑人貧。今移都伊、洛,欲通運四方。黃河急浚,人皆難涉,我因此行乘流,所以開百姓之心。知卿誠至而不得相納。」賜驊騮馬一匹,衣冠一襲。除羽林監、主客令。



 于時宮殿初構,運材日有萬計。伊、洛流澌,苦於厲涉。淹遂
 啟求敕都水造浮航。帝賞納之,意欲榮淹於眾。朔旦受朝,百官在位,乃賜帛百匹,知左右二都水事。景明三年,出除平陽太守。還朝,病卒,贈光州刺史,謚曰定。



 子宵,字景鸞,好為文詠,坦率多鄙俗,與河東姜質等朋游相好,詩賦間起,知音之士所共嗤笑。卒於書侍御史。



 范紹,字始孫,燉皇龍勒人也。少聰敏。年十二,父命就學,師事崔光。以父憂廢業。母又誡之曰:「汝父卒日,令汝遠就崔生,希有成立。今已過期,宜遵成命。」紹還赴學。太和初,充太學生,轉算生,頗涉經史。孝文選為門下通事令史,遷錄事,掌奏文案。帝善之,又為侍中李沖、黃門崔光
 所知。帝曾謂近臣曰:「崔光從容,范紹之力。」後朝廷有南討計,發河北數州田兵,通緣淮戍兵合五萬餘人,廣開屯田。八座奏紹為西道六州營田大使,加步兵校尉。紹勤於勸課,頻歲大獲。又詔與都督、中山王英論攻鐘離。紹觀其城隍,恐不可陷,勸令班師,英不從。紹還,具以狀奏聞。俄而英敗。後歷位并州刺史、太常卿。莊帝初,遇害河陰。



 劉桃符,中山盧奴人也。生不識父,九歲喪母。性恭謹,好學。舉孝廉,射策甲科。歷碎職,累遷中書舍人。以勤明見知,久不遷職。宣武謂曰:「揚子雲為黃門,頓歷三世。卿居
 此任始十年,不足辭也。」東豫州刺史田益宗居邊貪穢,宣武頻詔桃符慰喻之。桃符還,具稱益宗老耄,而諸子非理處物。宣武後欲代之。恐其背叛,拜桃符東豫州刺史,與後將軍李世哲領眾襲益宗。語在《益宗傳》。桃符善恤蠻左,為人吏所懷。久之,徵還。病卒,贈洛州刺史。



 鹿悆,字永吉,濟陰乘氏人也。祖壽興,沮渠氏庫部郎。父生,再為濟南太守,有政績。獻文嘉其能,特徵赴季秋馬射,賜以驄馬,加以青服,彰其廉潔。時三齊始附,人懷茍且,蒲博終朝,頗廢農業。生立制斷之,聞者嗟善。後卒於淮陽太守,追贈兗州刺史。悆好兵書、陰陽、釋氏之學,彭
 城王勰召為館客。嘗詣徐州,馬疲,附船而至大梁。夜睡,者上岸,竊禾四束飼馬。船行數里,悆覺,即停船至取禾處,以縑三丈置禾束下而反。



 初為真定公子直國中尉,恒勸以忠廉之節。嘗賦五言詩曰:「嶧山萬丈樹,雕鏤作琵琶,由此材高遠,糸玄響藹中華。」又曰:「援琴起何調?幽蘭與白雪,絲管韻未成,莫使糸玄響絕!」子直少有令問,悆欲其善終,故以諷焉。後隨子直鎮梁州,州有兵糧和糴,和糴者靡不潤屋,悆獨不取。子直強之,終不從。



 孝莊為御史中尉,悆兼殿中侍御史,監臨淮王彧軍。時梁遣其豫章王綜據徐州,綜密信通彧,云欲歸款。眾議謂不然,
 悆遂請行,曰:「綜若誠心,與之盟約;如其詐也,豈惜一人命乎!」時徐州始陷,邊方騷擾,綜部將成景俊、胡龍牙並總強兵,內外嚴固。悆遂單馬間出,徑趣彭城。未至之間,為綜軍主程兵潤所止。問其來狀。悆曰:「我為臨淮王所使。」兵潤遣人白龍牙等。綜既有誠心,聞悆被執,語景俊等曰:「我每疑元略規欲叛城,將驗虛實,且遣左右為元略使,入魏軍中喚彼一人。其使果至,可令人詐作略身,在一深室,託為患狀,呼使戶外,令人傳語。」



 時略始被梁武追還。綜又遣腹心人梁話迎悆,密語意狀,令善酬答。引悆詣龍牙所。



 龍牙語悆曰:「元中山甚欲相見,故令喚
 卿。」又曰:「安豐、臨淮,將少弱卒,規復此城,容可得乎?」悆曰:「彭城,魏之東鄙,勢在必爭,可否在天,非人所測。」龍牙曰:「當如卿言。」復詣景俊住所,停悆外門,久而未入。時夜已久,有綜軍主姜桃來與悆言,謂曰:「元法僧魏之微子,拔城歸梁,梁主待物有道。」



 乃上指曰:「今歲星在斗,吳之分野,君何不歸梁國?」悆答曰:「法僧,莒僕之流,而梁納之,無乃有愧於季孫也!今月建鶉首,斗牛受破,歲星木也,逆而剋之,吳國敗喪不久。且衣錦夜游,有識不許。」言未盡,乃引入見景俊。景俊良久謂曰:「卿不為刺客也?」答曰:「今者為使,欲反命本朝,相刺之事,更卜後圖。」為設食,悆強
 飲多食,向敵數人,微自夸矜。諸人相謂曰:「壯哉!」乃引向元略所,一人引入戶,指床令坐。一人別在室中出,謂悆曰:「中山王有教:『我昔有以向南,且遣相喚,欲問卿事。晚來患動,不獲相見。』遂辭而退。須臾天曉,綜軍主范勖、景俊司馬楊票等,競問北朝士馬多少,悆陳士馬之盛。尋而與梁話盟契訖。



 未旬,綜降詔封悆定陶縣子,除員勻散騎常侍。永安中,為右將軍、給事黃門侍郎,進爵為侯。雖任居通顯,志在謙退,迎送親賓,加於疇昔。而自無屋宅,常假賃居止,布衣糲食,寒暑不變。孝莊嘉其清潔,時復賜以錢帛。



 及東徐城人呂文欣殺刺史元大賓,南
 引梁人,詔悆以使持節、散騎常侍、安東將軍為六州大使,與行臺樊子鵠討破之。悆又購斬文欣。還,拜金紫光祿大夫,兼尚書右僕射、東南道三徐行臺。與都督賀拔勝等拒爾朱仲遠,軍敗還京。



 天平中,除梁州刺史。時滎陽人鄭榮業反,圍州城。城降,滎業送悆於關西。



 張耀,字景世,自云南陽西鄂人也。仕魏,累遷步兵校尉。永寧寺塔大興,經營務廣。靈太后曾幸作所,凡有顧問,耀敷陳指畫,無所遺闕,太后善之。後為別將,以軍功封長平男。歷岐、東荊州刺史。



 天平初,遷鄴草創,右僕射高隆之、吏部尚書元世俊奏曰:「南京宮殿,毀撤送都,連筏
 竟河,首尾大至,自非賢明一人,專委受納,則恐材木耗損,有關經構。



 耀清直素著,有稱一時,臣等輒舉為大將。」詔從之。耀勤於其事,尋轉營構左都將。興和初,加衛大將軍。宮殿成,除東徐州刺史。卒於州,贈司空公,謚曰懿。



 劉道斌,武邑灌津人也。有器幹,腰帶十圍,鬚髯甚美。初拜校書郎,轉主書,頗為孝文所知。從征南陽,還,加積射將軍、給事中。帝謂黃門郎邢巒曰:「道斌是行,便異儕流矣。」宣武即位,遷謁者僕射。後歷恒農太守、岐州刺史,所在有清貞稱。卒於州,謚曰康。道斌在恒農,修立學館,建孔子廟堂,圖畫形像。去郡後,故吏追思之,復立道斌形
 於孔像之西而拜謁焉。



 董紹,字興遠,新蔡鮦陽人也。少好學,頗有文義。起家四門博士,累遷兼中書舍人,為宣武所賞。豫州城人白早生以城南叛,詔紹慰勞,為賊鎖禁送江東。梁領軍呂僧珍暫與紹言,便相器重。梁武聞之,使勞紹云:「忠臣孝子不可無之,今當聽卿還國。」紹曰:「老母在洛,無復方寸,既奉恩貸,實若更生。」乃引見之,謂曰:「戰爭多年,人物塗炭,是以不恥先言,欲與魏朝通好,卿宜備申此意。若欲通好,今以宿豫還彼,彼當以漢中見歸。」及紹還,雖陳說和計,朝廷不許。後除洛州刺史。紹好行小惠,頗得人情。蕭
 寶夤反於長安,紹上書求擊之,云:「臣當出瞎巴三千,生啖蜀子。」孝明謂黃門徐紇曰:「此巴真瞎也?」紇答:「此紹之壯辭,云巴人勁勇,見敵無所畏,非實瞎也。」帝大笑,敕紹速行。以拒寶夤功,賞新蔡縣男。爾朱天光為關右大行臺,啟為大行臺從事,兼吏部尚書。天光敗,賀拔岳復請紹為其開府諮議參軍。岳後攜紹於高平牧馬,紹悲而賦詩曰:「走馬山之阿,馬渴飲黃河。寧謂胡關下,復聞楚客歌!」岳死,周文帝亦重之。及孝武西遷,除御史中丞,非其好也。鬱鬱不得志,或行戲街衢,或與少年游聚,不自拘持,頗類失性。孝武崩,周文與百官推奉文帝,上表勸
 進,令呂思禮、薛憕作表,前後再奏,帝尚執謙沖不許。周文曰:「為文能動至尊,唯董公耳!」乃命紹為第三表,操筆便成。表奏,周文曰:「開進人意,不當如此也?」及登祚,方任用之,而紹議論朝廷,賜死。孫嗣。



 馮元興,字子盛,東魏郡肥鄉人也。少有操尚。舉秀才,中尉王顯召為檢校御史,遷殿中御史。司徒江陽王繼召為記室參軍,遂為元叉所知。叉執朝政,引為尚書殿中郎,領中書舍人,仍御史,預聞時事。卑身克己,人無恨焉。家素貧約,食客恆數十人,同其飢飽,時人嘆尚之。太保崔光臨薨,薦元興為侍讀,尚書賈思伯為侍講,授孝明《
 杜氏春秋》。元興常為擿句,儒者榮之。叉既賜死,元興亦被廢。



 乃為《浮萍詩》以自喻曰:「有草生碧池,無根水上蕩,脆弱惡風波,危微苦驚浪。」



 普泰初,為光祿大夫,領中書舍人。太昌初,卒於家,贈齊州刺史。元興世寒,因元叉之勢,託其交道,相用為州主簿,論者以為非倫。



 時有濟郡曹昂,有學識,舉秀才。永安中,除太學博士,兼尚書郎。常徒步上省,以示清貧,忽遇盜,大失綾縑,時人鄙其矯詐。



 論曰:孫紹關左之士,又能指論時務。張普惠明達典故,強直從官,侃然不撓,其有王臣之風矣。成淹、范紹、劉桃符、鹿悆、張耀、劉道斌、董紹、馮元興等,身遭際會,俱得效
 其所能,茍曰非才,亦何能致於此也。



\end{pinyinscope}