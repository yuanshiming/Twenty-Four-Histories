\article{卷四十四列傳第三十二}

\begin{pinyinscope}

 崔光子劼弟子鴻崔亮從弟光韶叔祖道固崔光,清河人,本名孝伯,字長仁,孝文賜名焉。祖曠,從慕容德南度河,居青州之時水。慕容氏滅,仕宋為樂陵太守。於河南立冀州,置郡縣,即為東清河鄃人。縣分易,更為南平原貝丘人也。父靈延,宋長廣太守,與宋冀州刺史崔道固共拒魏軍。慕容白曜之平三齊,光年十七,隨
 父徙代。家貧好學,晝耕夜誦,傭書以養父母。



 太和六年,拜中書博士、著作郎,與秘書丞李彪參撰國書,再遷給事黃門侍郎。



 甚為孝文所知待,常曰:「孝伯才浩浩如黃河東注,固今日之文宗也。」以參贊遷都謀,賜爵朝陽子。拜散騎常侍,著作如故,兼太子少傅。又以本官兼侍中、使持節為陜西大使,巡方省察。所經述敘古事,因賦詩三十八篇。還,仍兼侍中。以謀謨之功,進爵為伯。光少有大度,喜怒不見於色。有毀惡之者,必善言以報,雖見誣謗,終不自申曲直。皇興初,有同郡二人並被掠為奴婢,後詣光求哀,光乃以二口贖免。孝文聞而嘉之。雖處機
 近,未曾留心文案,唯從容論議,參贊大政而已。



 孝文每對群臣曰:「以崔光之高才大量,若無意外咎譴,二十年後當作司空。」其見重如是。



 宣武即位,正除侍中。初,光與李彪共撰國書,太和之末,彪解著作,專以史事任光。彪尋以罪廢。宣武居諒闇,彪上表求成《魏書》,詔許之,彪遂以白衣於秘書省著述。光雖領史官,以彪意在專功,表解侍中、著作以讓彪。宣武不許。遷太常卿,領齊州大中正。



 正始元年夏,有典事史元顯獻四足四翼雞,詔散騎侍郎趙邕以問光。光表曰:臣謹案《漢書·五行志》宣帝黃龍元年,未央殿路軨中雌雞化為雄,毛變而不鳴不將
 無距。元帝初元中,丞相府史家雌雞伏子,漸化為雄,冠距鳴將。永光中,有獻雄雞生角。劉向以為雞者小畜,主司時起居,小臣執事為政之象也,言小臣將乘君之威,以害政事,猶石顯也。竟寧元年,石顯伏辜,此共效也。靈帝光和元年,南宮寺雌雞欲化為雄,一身皆似雄,但頭冠上未變,詔以問議郎蔡邕。邕對曰:「貌之不恭,則有雞禍。臣竊推之,頭為元首,人君之象也。今雞一身已變,未至於頭,而上知之,是將有其事而不遂成之象也。若政無所改,頭冠或成,為患滋大。」



 是後張角作亂,稱黃巾賊,遂破壞四方,疲於賦役,人多叛者。上不改政,遂至天下
 大亂。今之雞狀不同,其應頗相類矣。向、邕並博達之士,考物驗事,信而有證,誠可畏也。臣以邕言推之,翅足眾多,亦群下相扇助之象。雛而未大,腳羽差小,亦其勢尚微,易制御也。



 臣聞災異之見,皆所以示吉凶。明君睹之而懼,乃能招福,闇主視之彌慢,所用致禍。《詩》、《書》、《春秋》、秦、漢之事多矣,此皆陛下所觀者。今或有自賤而貴,關預政事,殆亦前代君房之匹。比者南境死亡千計,白骨橫野,存有酷恨之痛,歿為怨傷之魂。義陽屯師,盛夏未反;荊蠻狡猾,征人淹次。東州轉輸,多往無還,百姓困窮,絞縊以殞。北方霜降,蠶婦輟事。群生憔悴,莫甚於今。此亦
 賈誼哭歎,谷永切諫之時。司寇行戮,君為之不舉,陛下為人父母,所宜矜恤。



 國重戎戰,用兵猶火,內外怨弊,易以亂離。陛下縱欲忽天下,豈不仰念太祖取之艱難,先帝經營劬勞也?誠願陛下留聰明之鑒,警天地之意,禮處左右,節其貴越。



 往者鄧通、董賢之盛,愛之正所以害之。又躬饗如罕,宴宗或闕,時應親享郊廟,延敬諸父。檢訪四方,務加休息,爰發慈旨,撫振貧瘼。簡費山池,減撤聲飲,晝存政道,夜以安身。博採芻蕘,進賢黜佞,則兆庶幸甚,妖弭慶進,禎祥集矣。



 帝覽之大悅。後數日而茹皓等並以罪失伏法,於是禮光逾重。



 二年八月,光表曰:「去
 二十八日,有物出于太朽之西序,敕以示臣。臣案其形,即《莊子》所謂『蒸成菌』者也。又云『朝菌不終晦朔』。雍門周所稱「磨蕭斧而伐朝菌」,指言蒸氣鬱長,非有根種,柔脆之質,彫殞速易,不延旬月,無擬蕭斧。又多生墟落穢濕之地,罕起殿堂高華之所。今極宇崇麗,壇築工密,翼朽弗加,沾濡不及,而茲菌欻構,厥狀扶疏,誠足異也。夫野木生朝,野鳥入廟,古人以為敗亡之象。然懼災脩德,咸致休慶,所謂家利而怪先,國興而妖豫。是故桑穀拱庭,太戊以昌;雊雉集鼎,武丁用熙。自比鴟鵲巢於廟殿,梟鵩鳴於宮寢,菌生賓階軒坐之正,準諸往記,信可為誡。
 且東南未靜,兵革不息,郊甸之內,大旱跨時,人勞物悴,莫此之甚。承天子育者所宜矜恤。伏願陛下追殷二宗感變之意,側躬聳誠,惟新聖道,節夜飲之忻,強朝御之膳,養方富之年,保金玉之性,則魏祚可以永隆,皇壽等於山嶽。」



 四年,除中書舍人。永平元年秋,將誅元愉妾李氏,群官無敢言者。敕光為詔,光逡巡不作,奏曰:「伏聞當刑元愉妾李,加之屠割。妖惑扇亂,誠合此罪。但外人竊云,李今懷妊,例待分產。且臣尋諸舊典,兼推近事,戮至刳胎,謂之虐刑,桀、紂之主,乃行斯事。君舉必書,義無隱諱,酷而乖法,何以示後?陛下春秋已長,未有儲體,皇子
 襁褓,至有夭失。臣之愚識,知無不言,乞停李獄,以俟育孕。」



 帝納之。



 延昌元年,遷中書監,侍中如故。二年,宣武幸東宮,召光與黃門甄琛、廣陽王深等並賜坐,詔光曰:「卿是朕西臺大臣,當令為太子師傅。」光起拜固辭,詔不許。即令明帝出焉,從者十餘人,敕以光為傅之意,令明帝拜光。光又拜辭,不當受太子拜,復不蒙許。明帝遂南面再拜。詹事王顯啟請從太子拜,於是宮臣畢拜。



 光北面立,不敢答拜,唯西面拜謝而出。於是賜光繡采一百匹,琛、深各有差。尋授太子少傅,遷右光祿大夫,侍中、監如故。



 四年正月,宣武夜崩,光與侍中、領軍將軍于忠迎明
 帝於東宮,安撫內外,光有力焉。帝崩後二日,廣平王懷扶疾入臨,以母弟之親,徑至太極西廡,哀慟禁內。



 呼侍中、黃門、領軍二衛,云身欲上殿哭大行,又須入見主上。諸人皆愕然相視,無敢抗對者。光獨攘衰振杖,引漢光武初崩,太尉趙熹橫劍當階,推下親王故事,辭色甚厲。聞者莫不稱善,壯光理義有據。懷聲淚俱止,云:「侍中以古事裁我,我不敢不服。」於是遂還,頻遣左右致謝。



 初,永平四年,以黃門郎孫惠蔚代光領著作。惠蔚首尾五歲,無所厝懷。至是,尚書令、任城王澄表光宜還史任。於是詔光還領著作,遷特進。以奉迎明帝功,封博平縣公,領
 國子祭酒,詔乘步挽於雲龍門出入。尋遷車騎大將軍、儀同三司。靈太后臨朝後,光累表遜位。於忠擅權,光依附之。及忠稍被疏黜,光并送章綬冠服茅土,表至十餘上,靈太后優答不許。有司奏追于忠及光封邑。熙平元年二月,太師、高陽王雍等奏舉光授明帝經。初,光有德於靈太后。四月,更封光平恩縣侯,以朝陽伯轉授第三子勖。其月,敕賜羊車一乘。



 時靈太后臨朝,每於後園親執弓矢,光乃表上中古婦人文章,因以致諫。是秋,靈太后頻幸王公第宅,光表諫曰:「《禮記》云:諸侯非問疾弔喪,入諸臣之家,謂之君臣為謔。不言王后夫人,明無適臣
 家之義。夫人父母在,有時歸寧;親沒,使卿大夫聘。《春秋》紀陳、宋、齊之女並為周王后,無適本國之事。是制深於士大夫。許嫁唁兄,又義不得,衛女思歸,以禮自抑,《載馳》、《竹竿》所為作也。



 漢上官皇后將廢昌邑,霍光外祖也,親為宰輔,后猶御武帷以接群臣,示男女之別,國之大節。伯姬待姆,安就炎燎;樊姜候命,忍赴洪流。《傳》皆綴集,以垂來訓。



 昨軒駕頻出,幸馮翊君、任城王第。雖漸中秋,餘熱尚蒸。衡蓋往還,聖躬煩倦。



 左右僕侍,眾過千百,扶衛跋涉,袍鉀在身。昔人稱陛下甚樂,臣等至苦,或其事也。但帝族方衍,勛貴增遷,祗請遂多,將成彞式。陛下遵酌
 前王,貽厥後矩,天下為公,億兆己任。專薦郊廟,止決大政,輔神養和,簡息游幸,則率土屬賴,令生仰悅矣。」



 神龜元年,光表曰:「尋石經之作,起自炎劉,昔來雖屢經戎亂,猶未大崩侵。



 如聞往者刺史臨州,多構圖寺,官私顯隱,漸加肅撤。由是經石彌減,文字增缺。



 今求遣國子博士一人堪任幹事者,專主周視,驅禁田牧,制其踐穢,料閱碑牒所失次第,量厥補綴。」詔曰:「此乃學者之根原,不朽之永格,便可一依公表。」光乃令國子博士李郁與助教韓神固、劉燮等勘校石經,其殘缺,計料石功,并字多少,欲補脩之。後靈太后廢,遂寢。



 二年八月,靈太后幸永寧
 寺,躬登九層佛圖。光表諫曰:「伏見親升上級,佇蹕表剎之下,祗心圖構,誠為福善,聖躬玉趾,非所踐陟。臣庶恇惶,竊謂未可。」



 九月,靈太后幸嵩山佛寺,光上表諫,不從。



 正光元年冬,賜光几杖衣服。二年春,明帝親釋奠國學,光執經南面,百寮陪列。司徒、京兆王繼頻上表以位讓光。四月,以光為司徒,侍中、國子祭酒、領著作如故。光表固辭,歷年終不肯受。



 八月,獲禿鶖於宮內,詔以示光。光表曰:「此即《詩》所謂『有鶖在梁』。



 解云「禿鶖也」,貪惡之鳥,野澤所育,不應入於殿廷。昔魏氏黃初中,有鵜鶘集於靈芝池,文帝下詔,以曹恭公遠君子,近小人,博求賢俊,
 太尉華歆由此遜位而讓管寧者也。臣聞野物入舍,古人以為不善。是以張臶惡𪀼,賈誼忌鵩。鵜鶘暫集而去,前王猶為至誠,況今親入宮禁,為人所獲,方被畜養,晏然不以為懼。準諸往義,信有殊矣。饕餮之禽,必資魚肉,菽麥稻粱,時或飡啄,一食之費,容過斤鎰。今春夏陽旱,穀糴稍貴,窮窘之家,時有菜色。陛下為人父母,撫之如傷,豈可棄人養鳥,留意於醜形惡聲哉!衛侯好鶴,曹伯愛雁,身死國滅,可為寒心。



 願遠師殷宗,近法魏祖,脩德進賢,消災集慶,放無用之物,委之川澤,取樂琴書,頤養神性。」明帝覽表大悅,即棄之池澤。



 冬,詔光與安豐王延
 明議定服章。三年六月,詔光乘步挽至東西上閣。九月,進位太保,光又固辭。光年耆多務,病疾稍增。而自強不已,常在著作,疾篤不歸。



 四年十月,帝親臨光疾,詔斷賓客,中使相望,為止聲樂,罷諸游眺,拜長子勵為齊州刺史。十一月,疾甚,敕子姪等曰:「吾荷先帝厚恩,位至於此,史功不成,歿有遺恨。汝等速可送我還宅。」氣力雖微,神明不亂,至第而薨,年七十三。明帝聞而悲泣,中使相尋,詔給東園溫明祕器、朝服一具、衣一襲、錢六十萬、布一千匹、蠟匹百斤,大鴻臚監護喪事。車駕親臨,撫屍慟哭,御輦還宮,流涕於路,為減常膳,言則追傷,每至光坐講
 讀之處,未曾不改容悽悼。贈太傅,領尚書令、驃騎大將軍、開府、冀州刺史,侍中如故。又敕加後部鼓吹、班劍,依太保廣陽王故事,謚文宣。明帝祖喪建春門外,望轜哀感,儒者榮之。



 初,光太和中依宮商角徵羽本音而為五韻詩,以贈李彪。彪為十二次詩以報光。



 光又為百三郡國詩以答之。國別為卷,為百三卷焉。



 光寬和慈善,不忤於物,進退沈浮,自得而已。常慕胡廣、黃瓊為人,故為氣概者所不重。始領軍于忠,以光舊德,事之。元叉於光亦深宗敬。及郭祚、裴植見殺,清河王懌遇禍,光隨時俯仰,竟不匡救,於是天下譏之。自從貴達,罕所申薦,曾啟其
 女婿彭城劉敬徽,云敬徽為荊州五隴戍主,女隨夫行,常慮寇抄,南北分張,乞為徐州長兼別駕,暫集京師。明帝許之。時人比之張禹。光初為黃門則讓宋弁;為中書監讓汝南王悅;為太常讓劉芳;為少傅讓元暉、穆紹、甄琛;為國子祭酒讓清河王懌、任城王澄;為車騎、儀同讓江陽王繼,又讓靈太后父胡國珍,皆顧望時情,議者以為矯飾。



 崇信佛法,禮拜讀誦,老而逾甚。終日怡怡,未曾恚忿。曾於門下省晝坐讀經,有鴿飛集膝前,遂入於懷。緣臂上肩,久之乃去。道俗贊詠詩頌者數十人。每為沙門、朝貴請講《維摩》、《十地經》,聽者常數百人。即為二經義
 疏三十餘卷,識者知其疏略。凡所為詩賦銘贊誄頌表啟數百篇,五十餘卷,別有集。



 光子勵,字彥德。器學才德,最有父風。舉秀才,中軍彭城王參軍、祕書郎中,以父光為著作,固辭不拜。後除中書侍郎。領軍將軍元叉為明堂大將,以勵為長史。



 與從兄鴻俱有名於世。父光疾甚,拜征虜將軍、齊州刺史。侍父疾,衣不解帶;及薨,孝明每加存慰。光葬本鄉,詔遣主書張文伯宣弔。孝昌元年,除太尉長史,襲父爵。建義初,遇害河陰。贈侍中、衛將軍、青州刺史。勵弟劼。



 劼字彥玄,少清虛寡欲,好學有家風。魏末,累遷中書侍
 郎。興和三年,兼通直散騎常侍,使于梁。天保初,以議禪代,除給事黃門侍郎,加國子祭酒,直內省,典機密。清儉勤慎,甚為齊文宣所知。拜南青州刺史,有政績。入為祕書監、齊州大中正,遷并省度支尚書,俄授京省。尋轉五兵尚書,監國史。臺閣之中,見稱簡正。武成之將禪後主,先以問劼,劼諫以為不可。由是忤意,出為南兗州刺史。代還,重為度支尚書、儀同三司,食文登縣幹。尋除中書令,加開府,待詔文林館,監脩撰新書。卒,贈齊州刺史、尚書左僕射,謚文貞。



 初,和士開擅朝,曲求物譽,諸公因此頗為子弟干祿。世門之胄。多處京官,而劼二子拱、捴並
 為外任。弟廓之從容謂劼曰:「拱幸得不凡,何不在省府中清華之所,而並出外籓?」劼曰:「立身來,恥以言自達。今若進兒,與身何異!」卒無所求。聞者莫不歎服。劼常恨魏收書,欲更作編年紀,而才思竟不能就。



 光弟敬友,本州從事。頗有受納,御史案之。乃與守者俱逃。後除梁郡太守,會遭所生憂,不拜。敬友精心佛道,晝夜誦經,免喪之後,遂菜食終身。恭寬接下,脩身厲節。自景明已降,頻歲不登,飢寒請丐者,皆取足而去。又置逆旅於肅然山南大路之北,設食以供行者。卒于家。弟子鴻。



 鴻字彥鸞,少好讀書,博綜經史,稍遷尚書都兵郎中。詔
 太師、彭城王勰以下公卿朝士儒學才明者三十人,議定律令於尚書上省,鴻與光俱在其中,時論榮之。



 後為三公郎中,加員外散騎常侍。



 延昌二年,將大考百寮,鴻以考令於體例不通,乃建議曰:「竊惟昔者為官求才,使人以器,黜陟幽明,揚清激濁。故績效能官,才必稱位者,朝昇夕進,豈拘一階半級者哉。二漢以降,太和以前,茍必官須此人,人稱此職,或超騰昇陟,數歲而至公卿,或長兼、試守稱允當遷進者,披卷則人人而是,舉目則朝貴皆然。故能時收多士之譽,國號豐賢之美。竊見景明以來考格,三年成一考,一考轉一階。



 貴賤內外,萬有餘
 人,自非犯罪,不問賢愚,莫不上中,才與不肖,比肩同轉。雖有善政如黃、龔,儒學如王、鄭,才史如班、馬,文章如張、蔡,得一分一寸,必為常流所攀,選曹亦抑為一概,不曾甄別。琴瑟不調,改而更張,雖明旨已行,猶宜消息。」武帝不從。



 三年,鴻以父憂解任,甘露降其廬前樹。十一月,宣武以本官征鴻。四年,復有甘露降其京兆宅之庭樹。後遷中散大夫、高陽王友,仍領郎中。正光元年,加前將軍,修孝文、宣武《起居注》。



 光撰魏史,徒有卷目,初未考正,闕略尤多,每云:「此史會非我世所成,但須記錄時事,以待後人。」臨薨,言鴻於孝明。五年,詔鴻以本官修緝國史。孝
 昌初,拜給事黃門侍郎,尋加散騎常侍、齊州大中正。鴻在史甫爾,未有所就。尋卒,贈鎮東將軍、度支尚書、青州刺史。



 鴻弱冠便有著述志。見晉、魏前史,皆成一家,無所措意。以劉元海、石勒、慕容俊、苻健、慕容垂、姚萇、慕容德、赫連屈孑、張軌、李雄、呂光、乞伏國仁、禿髮烏孤、李皓、沮渠蒙遜、馮跋等並因世故,跨僭一方,各有國書,未有統一,鴻乃撰為《十六國春秋》,勒成百卷,因其舊記,時有增損褒貶焉。鴻二世仕江左,故不錄僭晉、劉、蕭之書,又恐識者責之,未敢出行於外。宣武聞其撰錄,遣散騎常侍趙邕詔鴻曰:「聞卿撰定諸史,甚有條貫,便可隨成者送
 至,朕當於機事之暇覽之。」鴻以其書有與國初相涉,言多失體,且既訖,不奏聞。鴻後典起居,乃妄載其表曰:臣聞帝王之興也,雖誕應圖籙,然必有驅除,蓋所以翦彼厭政,成此樂推。故戰國紛紜,年過十紀,而漢祖夷殄群豪,開四百之業。歷文、景之懷柔蠻夏,世宗之奮揚威武,始得涼、朔同文,䍧、越一軌。於是談、遷感漢德之盛,痛諸史放絕,乃鈐括舊書,著成《太史》,所謂緝茲人事,光彼天時之義也。



 昔晉惠不競,華戎亂起,三帝受制於姦臣,二皇晏駕於非所,五都蕭條,鞠為煨燼。趙、燕既為長蛇,遼海緬成殊域,中原無主,八十餘年。遺晉僻遠,勢略孤微,
 人殘兵革,靡所歸控。皇魏龍潛幽、代,內脩德政,外抗諸偽,并、冀之人,懷寶之士,襁負而至者日月相尋。太祖道武皇帝以神武之姿,接金行之運,應天順人,龍飛受命。太宗必世重光,業隆玄默。世祖雄才力睿略,闡曜威靈,農戰兼脩,掃清氛穢。歲垂四紀,而寰宇一同,百姓始得陶然蘇息,欣於堯、舜之代。



 自晉永寧以後,雖所在稱兵,競自尊樹,而能建邦命氏,成為戰國者,十有六家。善惡興滅之形,用兵乖會之道,亦足以垂之將來,昭明勸戒。但諸史殘缺,體例全虧,編錄紛謬,繁略失所,宜審正同異,定為一書。誠知敏謝允南,才非承祚,然《國志》、《史考》之美,
 竊亦輒所庶幾。始自景明之初,搜集諸國舊史,屬遷京甫爾,率多分散,求諸公私,驅馳數歲。及臣家貧祿微,唯任孤力,至於書寫所資,每不周接。暨正始元年,寫乃向備。謹於吏案之暇,草構此書,區分時事,各系本錄。稽以長歷,考諸舊志,刪正差謬,定為實錄,商較大略,著《春秋》百篇。



 至三年之末,草成九十五卷。唯常琚所撰李雄父子據蜀時書,尋訪不獲,所以未及善成。輟筆私求,七載于今。此書本江南撰錄,恐中國所無,非臣私力所能終得。



 其起兵僭號,事之始末,乃亦頗有,但不得此書,懼簡略不成。久思陳奏,乞敕緣邊求採,但愚賤無因,不敢輕
 輒。散騎常侍、太常少卿、荊州大中正趙邕忽宣明旨,敕臣送呈,不悟九皋微志,乃得上聞。奉敕欣惶,慶懼兼至。今謹以所訖者附臣邕呈奏。



 臣又別作《序例》一卷、《年志》一卷,仰表皇朝統括大義,俯明愚臣著錄微體。徒竊慕古人立言美意,文致疏鄙,無一可觀,簡御之日,伏深慚悸。



 鴻意如此。自正光以前,不敢顯行其書。自後以其伯光貴重當朝,知時人未能發明其事,乃頗傳讀。然鴻經綜既廣,多有違謬。至道武天興二年,姚興改號鴻始,而鴻以為改在元年;明元永興二年,慕容超禽於廣固,鴻又以為在元年;太常二年,姚泓敗於長安,而鴻亦以為
 滅在元年。如此之失,多不考正。



 子子元,秘書郎。後永安中,乃奏其父書,稱:「臣亡考散騎常侍、黃門侍郎、前將軍、齊州大中正鴻,正始之末,任屬記言,撰緝餘暇,乃刊著趙、燕、秦、夏、西涼、乞伏、西蜀等遺載,為之贊序,褒貶評論。先朝之日,草構悉了,唯有李雄蜀書,搜索未獲,闕茲一國,遲留未成。去正光三年,購訪始得,討論適訖,而先臣棄世。凡十六國,名為《春秋》,一百二卷,近代之事,最為備悉。未曾奏上,弗敢宣流。今繕寫一本,敢以仰呈,乞藏秘閣,以廣異家。」子元後謀反,事發逃竄,會赦免,尋為其叔鵾所殺。



 光從祖弟長文,字景翰。少亦徙於代都,聰敏有
 學識。永安中,累遷平州刺史,以老還家,專讀佛經,不關世事。卒,贈齊州刺史,謚曰貞。子懋,字德林,徐州征東府長史。



 長文從弟庠,字文序,有乾用。為東郡太守,元顥寇逼郡界,庠拒不從命,棄郡走還鄉里。孝莊還宮,賜爵平原伯,拜潁川太守,頗有政績。永熙初,除東徐州刺史。二年,為城人王早、蘭寶等所害。後贈驃騎將軍、吏部尚書、齊州刺史。子罕襲爵,齊受禪,例降。



 光族弟榮先,字隆祖。涉歷經史,州辟主簿。子鐸,有文才,位中散大夫。鐸弟覲,羽林監。



 崔亮,字敬儒,清河東武城人,魏中尉琰之後也。高祖瓊,
 為慕容垂車騎屬。



 曾祖輯,南徙青州,因仕宋為太山太守。祖脩之,清河太守。父元孫,尚書郎。青州刺史沈文秀之叛,宋明帝使元孫討之,為文秀所害。亮母房攜亮依其叔祖冀州刺史道固於歷城,及慕容白曜平三齊,內徙桑乾為平齊人。時年十歲,常依季父幼孫。



 居貧,傭書自業。



 時隴西李沖當朝任事,亮族兄光往依之,謂亮曰:「安能久事筆硯而不往託李氏也?彼家饒書,因可得學。」亮曰:「弟妹饑寒,豈容獨飽?自可觀書於市,安能看人眉睫乎!」光言之於沖,沖召亮與語,因謂曰:「比見卿先人《相命論》,使人胸中無復怵迫之念。今遂亡本,卿能記之不?」
 亮即為誦之,涕淚交零,聲韻不異。沖甚奇之,迎為館客。沖謂其兄子彥曰:「大崔生寬和篤雅,汝宜友之;小崔生峭整清徹,汝宜敬之,二人終將大至。」沖薦之為中書博士,轉議郎,尋遷尚書二千石。孝文在洛,欲創革舊制,選置百官,謂群臣曰:「與朕舉一吏部郎,必使才望兼允者,給卿三日假。」又一日,孝文曰:「朕已得之,不煩卿輩也。」驛徵亮兼吏部郎。俄為太子中舍人,遷中書侍郎,兼尚書左丞。亮雖歷顯任,其妻不免親事舂簸,孝文聞之,嘉其清貧,詔帶野王令。



 孝明親政,遷給事黃門侍郎,仍兼吏部郎,領青州大中正。亮自參選事,垂將十年,廉慎明決,
 為尚書郭祚所委,每云:「非崔郎中選事不辦。」尋除散騎常侍,仍為黃門。遷度支尚書,領御史中尉。白遷都之後,經略四方,又營洛邑,費用甚廣。亮在度支,別立條格,歲省億計。又議脩汴、蔡三渠以通邊運,公私賴焉。



 侍中、廣平王懷以母弟之親,左右不遵憲法,敕亮推究。宣武禁懷不通賓客者久之。後因宴集,懷侍親使忿,欲陵突亮。亮乃正色責之,即起於宣武前脫冠請罪,遂拜辭欲出。宣武曰:「廣平粗疏,向來又醉,卿之所悉,何乃如此也!遂詔亮復坐,令懷謝焉。亮外雖方正,內亦承候時情。宣傳左右郭神安頗被宣武識遇,以弟託亮,亮引為御史。及
 神安敗後,因集禁中,宣武令兼侍中盧昶宣旨責亮曰:「在法官,何故受左右囑請!」亮拜謝而已,無以上對。轉都官尚書,又轉七兵,領廷尉卿,加散騎常侍。徐州刺史元昞撫御失和,詔亮馳驛安撫。亮至,劾昞處以大辟,勞賚綏慰,百姓帖然。



 除安西將軍、雍州刺史。城北渭水淺不通船,行人艱阻。亮謂寮佐曰:「昔杜預乃造河梁,況此有異長河,且魏、晉之日,亦自有橋。吾今決欲營之。」咸曰:「水淺,不可為浮橋;汎長無恒,又不可施柱。恐難成立。」亮曰:「昔秦居咸陽,橫橋度渭,以像閣道,此即以柱為橋。今唯慮長柱不可得耳。」會天大雨,山水暴至,浮出長木數百
 根,籍此為用,橋遂成立。百姓利之,至今猶名崔公橋。亮性公清,敏于斷決,所在並號稱職,三輔服其德政。宣武嘉之,詔賜衣馬被褥。後納其女為九嬪,徵為太常卿,攝吏部事。



 孝明初,出為定州刺史。梁左游擊將軍趙祖悅率眾據硤石,詔亮假鎮南將軍,齊王蕭寶夤鎮東將軍,章下王融安南將軍,並使持節,督諸軍以討之。靈太后勞遣亮等,賜戎服雜物。亮至硤石,祖悅出城逆戰,大破之。祖悅復於城外置二柵,欲拒軍,亮焚擊破之。亮與李崇為水陸之期,日日進攻,而崇不至。及李平至,崇乃進軍,共平硤石。



 靈太后賜亮璽書曰:「硤石既平,大勢全舉,
 淮堰孤危,自將奔遁。若仍敢游魂,此當易以立計。禽翦蟻徒,應在旦夕。將軍推轂所馮,親對其事,處分經略,宜共協齊,必令得掃盪之理,盡彼遺燼也。隨便守禦,及分度掠截,扼其咽喉,防塞走路,期之全獲,無令漏逸。若畏威降首者,自加蠲宥,以仁為本,任之雅算。」



 以功進號鎮北將軍。



 李平部分諸軍,將水陸兼進,以討堰賊。亮違平節度,以疾請還,隨表而發。



 平表亮輒還京,失乘勝之機,闕水陸之會,今處亮死,上議。靈太后令曰:「亮去留自擅,違我經略,雖有小捷,豈免大咎。但吾攝御萬機,庶茲惡殺,可特聽以功補過。」及平至,亮與爭功禁中,形於聲
 色。



 尋除殿中尚書,遷吏部尚書。時羽林新害張彞之後,靈太后令武官得依資入選。



 官員既少,應選者多,前尚書李韶循常擢人,百姓大為怨。亮乃奏為格制,不問士之賢愚,專以停解日月為斷,雖復官須此人,停日後者終不得。庸才下品,年月久者灼然先用。沈滯者皆稱其能。亮外甥司空諮義劉景安書規亮曰:「殷、周以鄉塾貢士,兩漢由州郡薦才,魏、晉因循,又置中正。諦觀在昔,莫不審舉,雖未盡美,足應十收六七。而朝廷貢才,止求其文,不取其理。察孝廉唯論章句,不及治道;立中正不考人才行業,空辨氏姓高下。至於取士之途不溥,沙汰
 之理未精。而舅屬當銓衡,宜須改張易調。如何反為停年格以限之,天下士子誰復脩厲名行哉?」亮答書曰:汝所言乃有深致。吾乘時徼幸,得為吏部尚書。當其壯也,尚不如人,況今朽老,而居帝難之任。常思同升舉直,以報明主之恩;盡忠竭力,不為貽厥之累。昨為此格,有由而然。今已為汝所怪,千載之後,誰知我哉!可靜念吾言,當為汝論之。



 吾兼正六為吏部郎,三為尚書,銓衡所宜,頗知之矣。但古今不同,時宜須異。



 何者?昔有中正品其才第,上之尚書,尚書據狀,量人授職,此乃與天下群賢共爵人也。吾謂當爾之時,無遺才、無濫舉矣,而當猶云
 十收六七。況今日之選,專歸尚書,以一人之鑒,照察天下,劉毅所云一吏部、兩郎中而欲究鏡人物,何異以管窺天而求其博哉!今勳人甚多,又羽林入選。武夫崛起,不解書計,唯可彍弩前驅,指蹤捕噬而已。忽令垂組乘軒,求其烹鮮之效,未曾操刀,而使專割。又武人至多,官員至少,不可周溥。設令十人共一官,猶無官可授,況一人望一官,何由可不怨哉?吾近面執,不宜使武人入選,請賜其爵,厚其祿。既不見從,是以權立此格,限以停年耳。



 昔子產鑄刑書以救敝,叔向譏之以正法,何異汝以古禮難權宜哉?仲尼云:「德我者《春秋》,罪我者亦《春秋》。」吾
 之此指,其由是也。但令當來君子,知吾意焉。



 後甄琛、元脩義、城陽王徽相繼為吏部尚書,利其便己,踵而行之。自是賢愚同貫,涇、渭無別。魏之失才,從亮始也。



 歷侍中、太常卿、左光祿大夫、尚書右僕射。時劉騰擅權,亮託妻劉氏,傾身事之。故頻年之中,名位隆赫。有識者譏之。轉尚書僕射,加散騎常侍。疽發於背,明帝遣舍人問疾,亮上表乞解僕射,詔不許。尋卒。詔給東園祕器,贈車騎大將軍、儀同三司,謚曰貞烈。



 亮在雍州,讀《杜預傳》,見其為八磨,嘉其有濟時用,遂教人為碾。及為僕射,奏於張方橋東堰穀水,造磑磨數十區,其利十倍,國用便之。亮有
 三子,士安、士和、士泰,並強士,善於當世。



 士安歷尚書北部郎,卒於諫議大夫,贈左將軍、光州刺史。無子,弟士和以子乾亨繼。乾亨,武定中,尚書都兵郎中。



 士和初為司空主簿。蕭寶夤之在關中,高選寮佐,以為都督府長史。時莫折念生遣使詐降,寶夤表士和兼度支尚書為隴右行臺,令入秦撫慰,為念生所害。



 士泰歷給事中、司空從事中郎、諫議大夫、司空司馬。明帝末,荊蠻侵斥,以士泰為龍驤將軍、征蠻別將。事平,以功賜爵五等男。建義初,遇害於河陰,贈都督、青州刺史,謚曰文肅。了肇師襲爵。



 肇師少時疏放,長遂變節,更成謹厚。涉獵經史,頗有
 文思。天平初,以通直散騎侍郎為尉勞青州使,至齊州界,為土賊崔迦葉等拘,欲逼與同事。肇師執志不動,喻以禍福,賊遂捨之。仍巡慰青部而還。肇師以從弟乾亨同居,事伯母甚謹。



 齊文襄嘗言肇師合誅,左右問其故,曰:「崔鴻《十六國春秋》述諸僭偽而不及江東。」左右曰:「肇師與鴻別族。」乃止。天保初,以參定渾代禮儀,封襄城縣男,仍兼中書侍郎,卒。始鄴下有薛生者,能相人,言趙彥琛當大貴。肇師因問己,答曰:「公門望雖高,爵位不及趙。」終如其言。



 亮弟敬默,奉朝請,卒於征虜長史,贈南陽太守。子思韶。從亮征硤石,以軍功賜爵武城子,為冀州別
 駕。



 敬默弟敬遠,以其賤出,殊不經紀,論者譏焉。



 光韶,亮從父弟也。父幼孫,太原太守。光韶事親以孝悌。初除奉朝請,光韶與弟光伯孿生,操業相侔,特相友愛,遂經吏部尚書李沖,讓官於光伯,辭色懇至。



 沖為奏聞,孝文嘉而許之。太和二十年,以光韶為司空行參軍,復請讓從叔和,曰:「臣誠微賤,未登讓品,屬逢皇朝,恥無讓德。」和亦謙退,辭而不當。孝文善之,遂以和為廣陵王國常侍。尋敕光韶秘書郎,掌校華林御書。累遷青州中從事。後為司空騎兵參軍,又兼司徒戶曹。出為濟州輔國府司馬,刺史高植甚知之,政事多委訪焉。遷青州平東
 府長史。府解,敕知州事。光韶清直明斷,吏人畏愛之。入為司空從事中郎,以母老解官歸養,賦詩展意,朝士屬和者數十人。久之,徵為司徒諮議,固辭不拜。



 光韶性嚴,聲韻抗烈,與人平談,常若震厲。至於兄弟議論,外聞謂為忿怒,然孔懷雍睦,人少逮之。孝莊初,河間邢杲率河北流人十餘萬眾攻逼州郡,刺史元俊憂不自安。州人乞光韶為長史以鎮之。時陽平路回寓居齊土,與杲潛相影響,引賊入郭,光韶臨機處分,在難確然。賊退之後,刺史表光韶忠毅,朝廷嘉之,發使慰勞。尋為東道軍司。及元顥入洛,自河以南,莫不風靡。刺史廣陵王欣集文
 武以議所從,在坐之人,莫不失色。光韶獨抗言曰:「元顥受制梁國,稱兵本朝,亂臣賊子,曠代少疇。何但大王家事,所宜切齒。等荷朝眷,未敢仰從。」長史崔景茂、前瀛州刺史張烈、前郢州刺史房叔祖、徵士張僧皓咸云:「軍司議是。」欣乃斬顥使。



 尋徵輔國將軍,再遷廷尉卿。祕書監祖瑩以贓罪被劾。光韶必欲致之重法,太尉城陽王徽、尚書令臨淮王彧、吏部尚書李神俊、侍中李彧並勢望當時,皆為瑩求寬。光韶正色曰:「朝賢執事,於舜之功,未聞其一,如何反為罪人言乎。」其執意不回如此。永安據亂,遂還鄉里。



 光韶博學強辯,尤好理論,至於人倫名教,
 得失之間,榷而論之,不以一毫假物。家足於財,而性儉吝,衣馬敝瘦,食味麤薄。始光韶在都,同里人王蔓於夜遇盜,害其二子。孝莊詔黃門高道穆,令加檢捕,一坊之內,家別搜索。至光韶宅,綾絹錢布匱篋充積。議者譏其矯嗇。其家資產,皆光伯所營。光伯亡,悉焚其契。



 河間邢子才曾貸錢數萬,後送還之。光韶曰:「此亡弟相貸,僕不知也。」竟不納。



 刺史元弼前妻,是光韶之繼室兄女。弼貪婪不法,光韶以親情亟相非責,弼銜之。時恥翔反於州界,弼誣光韶子通與賊連結,囚其合家,考掠非理。而光韶與之辨爭,詞色不屈。會樊子鵠為東道大使,知其見
 枉,理出之。時人勸令詣樊陳謝,光韶曰:「羊舌大夫已有成事,何勞往也!」子鵠亦歎尚之。後刺史侯深代下,疑懼,謀為不軌。夜劫光韶,以兵脅之,責以謀略。光韶曰:「凡起兵須有名義,使君今日舉動,直是作賊耳,知復何計!」深雖恨之,敬而不敢害。尋除征東將軍、金紫光祿大夫,不起。



 光韶以世道屯邅,朝廷屢變,閉門卻掃,吉兇斷絕。誡子孫曰:「吾自謂立身無慚古烈,但以祿命有限,無容希世取進。在官以來,不冒一級,官雖不達,經為九卿。且吾平生素業,足以遺汝,官閥亦何足言也。吾既運薄,便經三娶,而汝之兄弟各不同生。合葬非古,吾百年之後,不
 須合也。然贈謚之及,出自君恩,豈容子孫自求之也?勿須求贈。若違吾志,如有神靈,不享汝祀。吾兄弟自幼及老,衣服飲食未嘗一片不同,至於兒女官婚,榮利之事,未嘗不先以推弟。弟頃橫禍,權作松櫬,亦可為吾作松棺,使吾見之。」卒,年七十一。孝靜初,侍中賈思申啟,稱述光韶,詔贈散騎常侍、驃騎將軍、青州刺史。



 光韶弟光伯。為青州別駕,後以族弟休臨州,申牒求解。尚書奏:「案《禮》:始封之君,不臣諸父、昆弟;封君之子,臣昆弟,不臣諸父;封君之孫,得盡臣。



 計始封之君,即是世繼之祖,尚不得臣,況今刺史既非世繼,而得行臣吏之節,執笏稱名
 者乎?檢光伯請解,率禮不愆,謂宜許遂。」靈太后令從之。尋除北海太守,有司以其更滿,依例奏代。明帝詔曰:「光伯自蒞海沂,清風遠著,兼其兄光韶復能辭榮侍養,兄弟忠孝,宜有甄錄,可更申三年,以廣風化。」後歷太傅諮議參軍。



 節閔帝時,崔祖螭、張僧皓起逆,攻東陽,旬日間,眾十餘萬。刺史、東萊王貴平欲令光伯出城慰勞。兄光韶爭之曰:「以下官觀之,非可慰喻止也。」貴平逼之,不得已,光伯遂出城。未及曉喻,為飛天矢所中,卒,贈青州刺史。子滔,武定末殷州別駕。脩之弟道固。



 道固字季堅,其母卑賤,嫡母兄攸之、目連等輕侮之。父
 輯謂攸之曰:「此兒姿識,或能興人門戶,汝等何以輕之?」攸之等遇之彌薄。輯乃資給道固,令其南仕。時宋孝武為徐、兗二州刺史,以道固為從事。道固美形貌,善舉止,習武事,孝武嘉之。會青州刺史新除,過彭城,孝武謂曰:「崔道固人身如此,豈可為寒士?



 而世人以其偏庶侮之,可為歎息。」刺史至州,辟為主簿。後為宋諸王參軍,被遣青州募人,長史以下並詣道固。道固諸兄等逼其所生自致酒炙於客前。道固驚起接取,謂客曰:「家無人力,老親自執劬勞。」諸客皆知其兄所作,咸拜其母。母謂道固曰:「我賤,不足以報貴賓,汝宜答拜。」諸客皆歎美道固母
 子,賤其諸兄。



 後為冀州刺史,鎮歷城。



 宋明帝立,徐州刺史薛安都與道固等立廢帝子業弟子勛,敗乃歸魏。獻文帝以為南冀州刺史、清河公。宋明帝遣說道固,以為徐州刺史,復歸宋。



 皇興初,獻文詔征南大將軍慕容白曜討道固,道固面縛請罪。白曜送赴都,詔恕其死。乃徙齊土望共道固守城者數百家於桑乾,立平齊郡於平城西北北新城,以道固為太守,賜爵臨淄子。尋徙居京城西南二百餘里舊除館之西。延興中卒,子景徽襲爵。



 初,道固之在客邸,與薛安都、畢眾敬鄰館,時以公集相見。本既同由武達,頗結寮舊。時安都志已衰朽,於道固
 疏略,而眾敬每盡殷勤。道固謂劉休賓、房法壽曰:「古人云:「非我族類,其心必異」,安都視人,殊自蕭索,畢固依依也。」



 景徽字文睿,卒於平州刺史,謚曰定。子休纂襲爵。



 道固兄曰連子僧祐。僧深坐兄僧祐與沙門法秀謀反,徙薄骨律鎮。後位南青州刺史。元妻房氏生子伯驎、伯驥。後薄房氏,納平原杜氏,與俱徙。生四子,伯鳳、祖龍、祖螭、祖虯。僧深得還之後,絕房氏,遂與杜氏及四子寓青州。伯驎、伯驥與母房居冀州,雖往來父間,而心存母氏,孝慈之道,頓阻一門。僧深卒,伯驎奔赴,不敢入家,寄哭寺門。祖龍剛躁,與兄伯驎訟嫡庶,並以刀劍自衛,苦怨讎
 焉。



 祖螭小字社客,普泰初反,爾朱仲遠討斬之。祖虯,少好學,不馳競。



 僧深從弟和,位平昌太守。家巨富而性吝,埋錢數百斛,其母李春思堇,惜錢不買。子軌,字啟則,盜錢百萬,背和亡走。後至儀同、開府鎧曹參軍,坐貪偽,賜死晉陽。



 論曰:崔光風素虛遠,學業深長,孝文歸其才博,許其大至,明主固知臣也。



 歷事三朝,師訓少主,不出宮省,坐致台傅,斯亦近世之所希有。但顧懷大雅,託迹中庸,其於容身之譏,斯乃胡廣所不免也。鴻博綜古今,立言為事,亦才志之士乎。崔亮既明達從事,動有名迹,於斷年之
 選,失之逾遠,救弊未聞,終為國蠹,無茍而已,其若是乎。光韶居雅仗正,有國士之風矣。



\end{pinyinscope}