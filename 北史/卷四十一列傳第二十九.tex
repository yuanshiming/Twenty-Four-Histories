\article{卷四十一列傳第二十九}

\begin{pinyinscope}

 楊播子侃播弟椿椿子昱椿弟津津子遁逸謐謐弟愔燕子獻鄭頤楊敷子素孫玄感素弟約約從叔異敷叔父寬寬子文恩紀楊播,字延慶,弘農華陰人也。高祖結,仕慕容氏,位中山相。曾祖珍,道武時歸國,位上谷太守。祖真,河內、清河二郡太守。父懿,延興末為廣平太守,有稱績。孝文南巡,吏人頌之,徵為選部給事中,有公平譽。除安南將軍、洛州刺史,未之任,卒。贈本官,加弘農公,謚曰簡。



 播本字元休,
 孝文賜改焉。母王氏,文明太后之外姑。播少脩飭,奉養盡禮。



 擢為中散,累遷衛尉少卿。與陽平王頤等出漠北擊蠕蠕,大致克獲。遷武衛將軍,復征蠕蠕,至居然山而還。及車駕南討,假前將軍,從至鐘離。師回,詔播為圓陣禦之。相拒再宿,軍人食盡,賊圍更急。播乃領精騎三百,歷其船大呼曰:「我今欲度,能戰者出。」遂擁而濟,賊莫敢動。賜爵華陰子。後從駕討破崔慧景、蕭愆於鄧城,進號平東將軍。時車駕耀威城沔水,上巳設宴,帝與中軍彭城王勰賭射,左衛元遙在勰朋內,而播居帝曹。遙射侯正中,籌限已滿。帝曰:「左衛籌足,右衛不得不解。」對曰:「仰
 恃聖恩,庶幾必爭」,於是箭正中。帝笑曰:「雖養由之妙,何復過是。」遂舉卮以賜播曰:「古人酒以養病,朕今賞卿之能,可謂古今殊也。」除太府卿,進爵為伯。



 後為華州刺史。至州,借人田,為御史王基所劾,除官爵,卒于家。子侃等停柩不葬,披訴積年。至熙平中,乃贈鎮西將軍、雍州刺史,并復其爵,謚曰壯。



 侃字士業,頗愛琴書,尤好計畫。時播一門,貴滿朝廷,子侄早通,而侃獨不交遊,公卿罕有識者。親朋勸其出仕,侃曰:「茍有良田,何憂晚歲,但恨無才具耳。」年三十一,襲爵華陰伯。



 揚州刺史長孫承業請為錄事參軍。梁豫州
 刺史裴邃規相掩襲,密購壽春人李瓜花、袁建等令為內應。邃已纂勒兵士,慮壽春疑覺,遂謬移云:「魏始於馬頭置戍,如聞復欲脩白捺舊城。若爾,便稍相侵逼。此亦須營歐陽,設交境之備。今板卒已集,唯聽信還。」佐寮咸欲以實答之,云無脩白捺意。而侃曰:「白捺小城,本非形勝,邃集兵遣移,虛構是言,得無有別圖也?」承業乃云:「錄事可造移報。」



 移曰:「彼之纂兵,想別有意,何為妄構白捺?他人有心,予忖度之,勿謂秦無人也。」邃得移,謂已覺,便散兵。瓜花等以期契不會,便相告發,伏辜者十數家。



 邃後竟襲襲壽春,入羅城而退,遂列營於黎漿、梁城,日夕鈔
 掠。承業乃奏侃為統軍。



 後雍州刺史蕭寶夤據州反,隨業討之,除侃為承業行臺左丞。軍次恆農,侃白承業曰:「今賊守潼關,全據形勝。須北取蒲阪,飛棹西岸,置兵死地,人有鬥心,華州之圍,可不戰而解;潼關之賊,必望風潰散。諸處既平,長安自克。愚計可錄,請為明公前驅。」承業從之,令其子子產等領騎與侃於恆農北度,便據石錐壁。乃班告曰:「今且停軍於此,以待步卒,兼觀人情向背。若送降名者,各自還村,侯臺軍舉三烽火,各亦應之,以明降款。其無應烽,即是不降之村,理須殄戮。」人遂傳相告報。實未降者,亦詐舉烽,一宿之間,火光遍數百里
 內。圍城之寇,不測所以,各自散歸。長安平,侃頗有力焉。建義初,除岐州刺史。屬元顥內逼,詔行北中郎將。



 孝莊徙河北,執侃手曰:「朕停卿蕃寄,移任此者,正為今日。但卿尊卑百口,若隨朕行,所累處大。卿可還洛,寄之後圖。」侃曰:「寧可以臣微族,頓廢君臣之義。」固求陪從。除度支尚書,兼給事黃門侍郎,敷西縣公。及車駕南還,顥令梁將陳慶之守北中城,自據南岸。有夏州義士為顥守河中渚,乃密信通款,求破橋立效。爾朱榮赴之。及橋破,應接不果,皆為顥屠。榮將為還計,欲更圖後舉。侃曰:「若今即還,人情失望,未若召發人材,唯多縛筏,間以舟楫,沿
 河廣布。令數百里中,皆為度勢,顥知防何處?一旦得度,必立大功。」榮大笑從之。於是爾朱兆等於馬渚諸楊南度,顥便南走。車駕入都,侃解尚書,正黃門。以濟河功,進爵濟北郡公,復除其長子師仲為祕書郎。



 時所用錢,人多私鑄,稍就薄小,乃至風飄水浮,米斗幾直一千。侃奏聽人與官並鑄五銖,使人樂為,而俗弊得改。莊帝從之。後除侍中,加衛將軍、右光祿大夫。



 莊帝將圖爾朱榮,侃與內弟李晞、城陽王徽、侍中李彧等咸預其謀。爾朱兆入洛,侃時休沐,遂竄歸華陰。普泰初,天光在關西,遣侃子婦父韋義遠招慰之,立盟許恕其罪。侃從兄昱恐為
 家禍,令侃出應,假其食言,不過一人身沒,冀全百口。



 侃赴之,為天光所害。太昌初,贈車騎將軍、儀同三司、幽州刺史。子純陀襲。



 播弟椿。椿字延壽,本字仲考,孝文賜改焉。性寬謹。為內給事,與兄播並侍禁闈。後為中部法曹,折訟公正,孝文嘉之。及文明太后崩,孝文五日不食。椿諫曰:「聖人之禮,毀不滅性,從陛下欲自賢於萬代,其若宗廟何!」帝感其言,乃一進粥。轉授宮輿曹少卿,加給事中,出為豫州刺史,再遷梁州刺史。



 初,武興王楊集始降於齊,自漢中而北,規復舊土。椿貽書集始,開以利害。



 集始執書對使者曰:「楊使
 君此書,除我心腹疾。」遂來降。尋以母老解還。後兼太僕卿。



 秦州羌呂茍兒、涇州屠各陳瞻等反,詔椿為別將,隸安西將軍元麗討之。賊守峽自固。或謀伏兵斷其出入,待糧盡攻之。或云斬山木,從火焚之。椿曰:「並非計也。賊深竄,正避死耳。今宜勒三軍勿更侵掠,賊必謂見險不前,心輕我軍,然後掩其不備,可一舉而平。」乃緩師。賊果出掠,仍以軍中驢馬餌之。銜枚夜襲,斬瞻傳首。入正太僕卿。



 初,獻文世有蠕蠕萬餘戶降附,居於高平、薄骨律二鎮。太和末叛走,唯有一千餘家。太中大夫王通、高平鎮將郎育等求徙置淮北,防其後叛。詔椿徙焉。椿上書,
 以為裔不謀夏,夷不亂華,是以先朝居之荒服之間,正欲悅近來遠。今新附者眾,若舊者見徙,新者必不安,愚謂不可。時八坐不從,遂於濟州緣河居之。及冀州元愉之難,果悉浮河赴賊,所在鈔掠,如椿所策。後除朔州刺史。在州為廷尉奏椿前為太僕卿,招引百姓,盜種牧田三百四十頃,依律處刑五歲。尚書邢巒據正始別格,奏罪應除名,注籍盜門,同籍合門不仕。宣武以親律既班,不宜雜用舊制,詔依斷,以贖論。后除定州刺史。



 自道武平中山,多置軍府,以相威攝。凡有八軍,軍各配兵五千,食祿主帥軍各四十六人。自中原稍定,八軍之兵漸割
 南戍,一軍兵纔千餘,然主帥如故,費祿不少。椿表罷四軍,減其主帥百八十四人。椿在州,因修黑山道餘功,伐木私造佛寺,役兵,為御史所劾,除名。



 後累遷為雍州刺史,進號車騎大將軍、儀同三司。尋以本官加侍中,兼尚書右僕射,為行臺,節度關西諸將。遇暴疾,頻啟乞解,詔許之,以蕭寶夤代為刺史、行臺。



 椿還鄉里,遇子昱將還京師,使陳寶夤賞罰云為,不依常憲,恐有異心。昱還,面啟明帝及靈太後,並不納。及寶夤邀害御只中尉酈道元,猶上表自理,稱為椿父子所謗。



 建義元年,為司徒。永安初,進位太保,加侍中,給後部鼓吹。元顥入洛,椿子昱
 為顥禽。又椿弟順、順子仲宣、兄子侃、弟子遁並從駕河內,為顥嫌疑。以椿家世顯重,恐失人望,未及加罪。時人助其憂,或勸椿攜家避禍。椿曰:「吾內外百口,何處逃竄?正當坐任運耳。」



 莊帝還宮,椿上書頻請歸老,詔聽服侍中服,賜朝服一襲、八尺床帳、几、杖,不朝,乘安車,駕駟馬,給扶,傳詔二人,仰所在郡縣四時以禮存問安否。椿奉辭於華林園,帝下御座,執手流淚曰:「公先帝舊臣,實為元老。但高尚其志,決意不留,既難相違,深用悽切。」椿亦歔欷,欲拜,帝親執不聽。賜以絹布,給羽林衛送。群公百寮餞於城西張方橋,行路觀者莫不稱歎。椿臨行,誡子
 孫曰:我家入魏之始,即為上客。自爾至今,二千石方伯不絕,祿恤甚多。於親姻知故吉凶之際,必厚加贈襚;來往賓寮,必以酒肉飲食,故六姻朋友無憾焉。國家初,丈夫好服彩色。吾雖不記上谷翁時事,然記清河翁時服飾。恆見翁著布衣韋帶,常自約敕諸父曰:「汝等後世若富貴於今日者,慎勿積金一斤、綵帛百匹已上,用為富也。」不聽興生求利,又不聽與勢家作婚姻。至吾兄弟,不能遵奉。今汝等服乘漸華好,吾是以知恭儉之德,漸不如上也。又吾兄弟,若在家,必同盤而食;若有近行,不至,必待其還。亦有過中不食,忍飢相待。吾兄弟八人,今存
 者有三,是故不忍別食也。又願畢吾兄弟,不異居異財。汝等眼見,非為虛假。如聞汝等兄弟,時有別齋獨食者。此又不如吾等一世也。吾今日不為貧賤,然居住舍宅,不作壯麗華飾者,正慮汝等後世不賢,不能保守之,將為勢家所奪。



 北都時,朝法嚴急。太和初,吾兄弟三人並居內職:兄在高祖左右,吾與津在文明太后左右。于時口敕,責諸內官,十日仰密得一事,不列便大嗔嫌。諸人多有依敕密列者,亦有太后、高祖中間傳言構間者。吾兄弟自相誡曰:「今忝二聖近臣,居母之間難,宜深慎之。又列人事,亦何容易,縱被嗔責,勿輕言。」十餘年中,不嘗
 言一人罪過。時大被嫌責,答曰:「臣等非不聞人語,正恐不審,仰誤聖聽,以是不敢言。」於後終以不言。蒙賞及二聖間言語,終不敢輒爾傳通。太和二十一年,吾從濟州來朝,在清徽堂豫宴。高祖謂諸貴曰:「北京之日,太后嚴明,吾每得杖。左右因此有是非言。和朕母子者,唯楊播兄弟。」遂舉爵賜兄及我酒。汝等脫若萬一蒙明主知遇,宜深慎言語,不可輕論人惡也。吾自惟文武才藝、門望姻援不勝他人。一旦位登侍中、尚書,四歷九卿,十為刺史,光祿大夫、儀同、開府、司徒、太保,津今復為司空者,正由忠謹慎口,不嘗論人之過,無貴無賤,待之以禮,以是
 故至此耳。聞汝等學時俗人,乃有坐待客者,有驅馳勢門者,有輕論人惡者;及見貴勝則敬重之,見貧賤則慢易之,此人行之大失,立身之大病也。汝家仕皇魏以來,高祖以下乃有七郡太守、三十二州刺史,內外顯職,時流少比。汝等若能存禮節,不為奢淫驕慢,假不勝人,足免尤誚,足成名家。吾今年始七十五,自惟氣力,尚堪朝覲天子,所以孜孜求退者,正欲使汝等知天下滿足之議,為一門法耳,非是茍求千載之名。汝等能記吾言,吾百年後終無恨矣。



 椿還華陰。踰年,為爾朱天光所害,時人莫不怨痛之。太昌初,贈太師、丞相、都督、冀州刺史。子
 昱。



 昱字元略,起家廣平王懷左常侍。懷好武事,數遊獵,昱每規諫。正始中,以京兆、廣平二王國臣多縱恣,詔御史中尉崔亮窮案之,伏法都市者三十餘人,不死者悉除名,唯昱與博陵崔楷以忠諫免。後除太學博士、員外散騎侍郎。



 初,尚書令王肅除揚州刺史,出頓洛陽東亭。酣後,廣陽王嘉、北海王詳等與播論議競理,播不為屈。北海王顧昱曰:「尊伯性剛不伏理,大不如尊使君也。」



 昱對曰:「昱父道隆則從其隆,道洿則從其洿;伯父剛則不吐,柔亦不茹。」坐歎其能言。肅曰:「非此郎,何得申二父之
 美。」



 延昌三年,以本官帶詹事丞。時明帝在懷抱中,至於出入,左右、乳母而已,不令宮寮聞知。昱諫曰:「陛下不以臣等凡淺,備位宮臣,太子動止,宜令翼從。



 自比以來,輕爾出入,進無二傅導引之美,退闕群寮陪侍之式。非所謂示人軌儀,著君臣之義。陛下若召太子,必降手敕,令臣下咸知,為後世法。」於是詔自今若非手敕,勿令兒輒出,宮臣在直者,從至萬歲門。轉太尉掾,兼中書舍人。



 靈太后嘗謂昱曰:「親姻在外,不稱人心,卿有所聞,慎勿諱隱。」昱奏揚州刺史李崇五車載貨,恆州刺史楊鈞造銀食器十具,並餉領軍元叉。靈太后令召叉夫妻,泣而責
 之。叉深恨昱。昱第六叔舒妻,武昌王和之妹,和即叉之從祖父。舒早喪,有一男六女,及終喪,元氏請別居。昱父椿集親姻泣謂曰:「我弟不幸早終,今男未婚,女未嫁,何便求別居?」不聽。遂懷憾。神龜二年,瀛州人劉宣明謀反,事覺逃竄。叉使和及元氏誣告昱藏宣明,云昱父椿、叔津並送甲仗三百具,謀圖不逞。叉又構成其事。乃遣夜圍昱宅收之,並無所獲。太后問狀。昱具對元氏構釁之端,言至哀切。太后乃解昱縛,和及元氏並處死刑。而叉相左右,和直免官,元氏卒亦不坐。及叉之廢太后也,乃出昱為濟陰內史。中山王熙起兵於鄴,叉遣黃門盧同
 詣鄴刑熙,并窮黨與。同希叉旨,就郡鎖昱赴鄴,囚訊百日乃還任。



 孝昌初,除中書侍郎,遷給事黃門侍郎。後賊圍豳州,詔昱兼侍中,持節催西北道大都督、北海王顥,仍隨軍監察。豳州圍解。雍州蜀賊張映龍、姜神達知州內虛,謀欲攻掩。刺史元脩義懼而請援,一日一夜,書移九通。都督李叔仁遲疑不赴。



 昱曰:「若長安不守,大軍自然瓦散,此軍雖往,有何益也。」遂與叔仁等俱進,於陣斬神達,諸賊迸散。詔以昱受旨催督,而顥軍稽緩,遂免昱官。尋除涇州刺史。



 未幾,昱父椿為雍州,徵昱除吏部郎中。及蕭寶夤等敗於關中,以昱兼七兵尚書、持節、假撫
 軍、都督,防守雍州。昱遇賊失利而返。後除鎮東將軍、假車騎將軍、東南道都督,又加散騎常侍。於後太山守羊侃據郡南叛,侃兄深時為徐州行臺,府州咸欲禁深。昱曰:「昔叔向不以鮒也見廢,奈何以侃罪深,宜呼朝旨。」不許群議。



 還朝未幾,元顥侵逼大梁,除昱南道大都督,鎮滎陽。顥禽濟陰王暉業,乘虛徑進,城陷。昱與弟息五人在門樓上。顥至,執昱下,責曰:「卿今死甘心不?」



 答曰:「分不望生,向所以不下樓,正慮亂兵耳。但恨八十老父無人供養,乞小弟一命,便是死不朽也。」顥將陳慶之、胡光等伏顥帳前曰:「陛下度江三千里,無遺鏃費。昨日殺傷五
 百餘人,求乞楊昱以快意。」景曰:「我在江東聞梁主言,初下都,袁昂為吳郡不降,稱其忠節。奈何殺昱。」於是斬昱下統帥三十七人,皆令蜀兵刳腹取心食之。



 孝莊還,復前官。爾朱榮之死,昱為東道行臺拒爾朱仲遠。會爾朱兆入洛,昱還京師。後歸鄉里,亦為天光所害。太昌初,贈司空公、定州刺史。



 子孝邕,員外郎,奔免。匿蠻中,潛結渠率,謀報爾朱氏。微服入洛,為爾硃世隆所殺。椿弟穎,字惠哲,本州別駕。



 穎弟順,字延和,寬裕謹厚。豫立莊帝功,封三門縣伯,位冀州刺史。罷州還,遇害。太昌初,贈太尉公、錄尚書事、相州刺史。子辯,字僧達,位東雍州刺史。



 辯
 弟仲宣,有風度才學。位正平太守,爵恆農伯,在郡有能名。還京,兄弟與父同遇害。太昌初,辯贈儀同三司、恆州刺史;仲宣贈尚書右僕射、青州刺史。



 仲宣子玄就,幼而俊拔。收捕時,年九歲,牽挽兵人曰:「欲害諸尊,乞先就死。」兵以刀斫斷其臂,猶請死不止,遂先殺之。永熙初,贈汝陰太守。



 順弟津。津字羅漢,本字延祚,孝文賜改焉。少端謹,以器度見稱。年十一,除侍御中散。時孝文幼沖,文明太后臨朝,津曾入侍左右,忽咳逆失聲,遂吐血數升,藏之衣袖。太后聞聲,閱而不見,問其故,具以實言,遂以敬慎見知。賜縑百
 匹,遷符璽郎中。津以身在禁密,不外交遊,至宗族姻表罕相參侯。司徒馮誕與津少結交友,而津見其貴寵,每恆退避,及相招命,多辭疾不往。誕以為恨,而津逾遠焉。人或謂之曰:「司徒,君之少舊,何自外也?」津曰:「為勢家所厚,復何容易!但全吾今日,亦足矣。」轉振威將軍,領監曹奏事令。孝文南征,以津為都督、征南府長史。後遷長水校尉,仍直閣。



 景明中,宣武遊於北芒,津時陪從。太尉、咸陽王禧謀反,帝馳入華林。時直閣中有同禧謀,皆在從限。及禧平,帝顧謂朝臣曰:「直閣半為逆黨,非至忠者安能不豫此謀。」因拜津左中郎將,遷驍騎將軍,仍直閣。



 出除岐州刺史,津巨細躬親,孜孜不倦。有武功人齎絹三匹,去城十里,為賊所劫。時有使者馳驛而至,被劫人因以告之。使者到州,以狀白津。津乃下教,云有人著某色衣,乘某色馬,在城東十里被殺,不知姓名。若有家人,可速收視。有一老母行哭而出,云是己子。於是遣騎追收,並絹俱獲。自是闔境畏服。至於守令寮佐有濁貨者,未曾公言其罪,常以私書切責之。於是官屬感厲,莫有犯法者。以母憂去職。



 延昌末,起為華州刺史,與兄播前後牧本州,當世榮之。先是,受調絹度尺特長,在事因緣,共相進退,百姓苦之。津乃令依公尺度其輸物,尤好者
 賜以杯酒而出;其所輸少劣者,為受之,但無酒以示其恥。於是競相勸厲,官調更勝。



 孝昌中,北鎮擾亂,侵逼舊京,乃加津安北將軍,北道大都督,尋轉左衛,加撫軍將軍。津始受命,出據靈丘。而賊帥鮮于脩禮起於博陵,定州危急,遂回師南赴。始至城下,榮壘未立,而州軍新敗。津以賊既乘勝,士眾榮疲,柵壘未安,不可擬敵,欲移軍入城,更圖後舉。刺史元固稱賊既逼城,不可示弱,乃閉門不內。



 津揮刃欲斬門者,軍乃得入。賊果夜至,見柵空而去。其後,賊攻州城東面,已入羅城。刺史閉小城東門,城中騷擾。津開門出戰,賊退,人心少安。尋除定州刺史,
 又兼吏部尚書、北道行臺。初,津兄椿得罪此州,由鉅鹿人趙略投書所致。及津至,略舉家逃走。津乃下教慰喻,令其還業。於是闔州愧服,遠近稱之。時賊帥鮮于脩禮、杜洛周賤掠州境,孤城獨立,在兩寇之間。津脩理戰具,更營雉堞。又於城中去城十步,掘地至泉,廣作地道,潛兵涌出,置爐鑄鐵,持以灌賊。賊遂相告曰:「不畏利槊堅城,唯畏楊公鐵星。」津與賊帥元洪業等書喻之,並授鐵券,許之爵位,令圖賊帥毛普賢。洪業等感寤,復書云欲殺普賢。又云:「賊欲圍城,正為取北人,城中所有人,必須盡殺。」津以城內北人,雖是惡黨,然掌握中物,未忍便
 殺,但收內子城,防禁而已。將吏無不感其仁恕。朝廷初送鐵券二十枚,委津分給。



 津隨賊中首領,間行送之;脩禮、普賢頗亦由此而死。



 既而杜洛周圍州城,津盡力捍守。詔加衛將軍,將士有功者任津科賞,兵人給復八年。葛榮以司徒說津。津大怒,斬其使以絕之。自受攻圍,經歷三稔,朝廷不能拯赴。乃遣長子遁突圍出。詣蠕蠕主阿那瑰,令其討賊。遁日夜泣訴,阿那瑰遣其從祖吐豆發率精騎南出。前鋒已達廣昌,賊防塞益口,蠕蠕遂還。津長史李裔引賊入,津苦戰不敵,遂見拘執。洛周脫津衣服,置地牢下數日,將烹之。諸賊還相諫止,遂得免害。
 津曾與裔相見,對諸賊帥以大義責之,辭淚俱發,裔大慚。典守者以告洛周,弗之責。及葛榮併洛周,復為榮所拘。榮破,始得還洛。



 永安二年,兼吏部尚書。元顥內逼,莊帝將親出討,以津為中軍大都督,兼領軍將軍。未行,顥入。及顥敗,津乃入宿殿中,掃灑宮掖,遣第二子逸封閉府庫,各令防守。及帝入也,津迎於北芒,流涕謝罪。帝深嘉慰之。尋以津為司空,加侍中。爾朱榮死,使津以本官為兼尚書令、北道大行臺、都督、並州刺史,委以討胡經略。津馳至鄴,將從滏口而入。遇爾朱兆等已克洛,相州刺史李神等議欲與津舉城通款,津不從。以子逸既為
 光州刺史,兄子昱時為東道行臺,鳩率部曲,在於梁、沛,津規欲東轉,更為方略。乃率輕騎望於濟州度河。而爾朱仲遠已陷東郡,所圖不果,遂還京師。普泰元年,亦遇害於洛。太昌初,贈大將軍,太傅、都督、雍州刺史,謚曰孝穆。將葬本鄉,詔大鴻臚持節監護喪事。長子遁。



 遁字山才。其家貴顯,諸子弱冠,咸縻王爵。而遁性靜退,年近三十,方為鎮西府主簿。累遷尚書左丞、金紫光祿大夫,亦被害於洛。太昌初,贈車騎大將軍、儀同三司、幽州刺史,謚曰恭定。



 遁弟逸,字遵道,有當世才。起家員外散騎侍郎,以功賜爵華陰男。建義初,莊帝猶在河陽,逸
 獨往謁。帝特除給事黃門侍郎,領中書舍人。及朝士濫禍,帝益憂怖,詔逸晝夜陪侍,常寢御床前。帝曾夜中謂逸曰:「昨來舉目唯見異人,賴卿差以自慰。」再遷南秦州刺史,加散騎常侍,時年二十九,時方伯之少,未有先之者。仍以路阻不行,改光州刺史。時災儉連歲,逸欲以倉粟振給,而所司懼罪不敢。



 逸曰:「國以人為本,人以食為命,假令以此獲戾,吾所甘心。」遂出粟,然後申表。右僕射元羅以下,謂公儲難闕,並執不許。尚書令、臨淮王彧以為宜貸二萬,詔聽貸二萬。逸既出粟之後,其老小殘疾不能自存活者,又於州門造粥飼之,將死而得濟者以
 萬數。帝聞而善之。逸為政愛人,尤憎豪猾,廣設耳目,善惡畢聞。其兵出使下邑,皆自持糧,人或為設食者,雖在暗室,終不敢進,咸言楊使君有千里眼,那可欺之。在州政績尤美。



 及其家禍,爾朱仲遠遣使於州害之。吏人如喪親戚,城邑村落營齋供,一月之中,所在不絕。太昌初,贈都督、豫郢二州刺史,謚曰貞。



 逸弟謐,字遵和。歷員外散騎常侍,以功賜爵恆農伯,鎮軍將軍、金紫光祿大夫、衛將軍。在晉陽,為爾朱兆所害。太昌初,贈驃騎將軍、兗州刺史。謐弟愔,事列于後。



 津弟,字延季。弘厚,頗有文學。位武衛將軍,加散騎常侍、安南將軍。莊帝初,遇害河
 陰,曾儀同三司、雍州刺史。



 播家世純厚,為並敦議讓,昆季相事,有如父子。播性剛毅,椿、津恭謙,兄弟旦則聚於堂,終日相對,未曾入內。有一美味,不集不食。堂間,往往幃慢隔障,為寢息之所,時就休偃,還共談笑。椿年老,曾他處醉歸,津扶侍還室,仍假寢閣前,承候安否。椿、津年過六十,並登台鼎;而津常旦暮參問,子姪羅列階下,椿不命坐,津不敢坐。椿每近出,或日斜不至,津不先飯;椿還,然後共食。



 食則津親授匙箸,味皆先嘗,椿命食,然後食。津為司空,於時府主皆自引寮佐。



 人有就津求官者,津曰:「此事須家兄裁之,何為見問。」初,津為肆州,椿在
 京宅,每有四時嘉味,輒因使次附之,若或未寄,不先入口。椿每得所寄,輒對之下泣。兄弟並皆有孫,唯椿有曾孫,年十五六矣。椿常欲為之早娶,望見玄孫。自昱已下,率多學尚,時人莫不欽焉。一家之內,男女百口,緦服同爨,庭無間言。魏世以來,唯有盧陽烏兄弟及播昆季,當世莫逮焉。



 爾朱世隆等將害椿家,誣其為逆,奏請收之。節閔不許;世隆復苦執,不得已,乃下詔。世隆遂遣步騎夜圍其宅,天光亦同日收椿於華陰,東西兩處,無少長皆遇禍,籍沒其家。節閔惋悵久之。



 愔字遵彥,小名秦王。兒童時,口若不能言;而風度深敏,
 出入門閭,未嘗戲弄。六歲學史書,十一受《詩》、《易》,好《左氏春秋》。幼喪母,曾詣舅源子恭。子恭與之飲,問讀何書。曰:「誦《詩》」。子恭曰:「誦至《渭陽》未邪?」



 愔便號泣感噎。子恭亦對之歔欷,遂為之罷酒。子恭後謂津曰:「常謂秦王不甚察慧,從今已後,更欲刮目視之。」



 愔一門四世同居,家甚隆盛,昆季就學者三十餘人。學庭前有柰樹,實落地,群兒咸爭之。愔頹然獨坐。其季父適入學館,見之,大用嗟異。顧謂賓客曰:「此兒恬裕,有我家風。」宅內有茂竹,遂為愔於林邊別葺一室,命獨處其中,常銅盤具盛饌以飯之。因以督厲諸子曰:「汝輩但如遵彥謹慎,自得竹林別
 室、銅盤重肉之食。」愔從父兄黃門侍郎昱特相器重,曾謂人曰:「此兒駒齒未落,已是我家龍文;更十歲後,當求之千里外。」昱嘗與十餘人賦詩,愔一覽便誦,無所遺失。



 及長,能清言,美音制,風神俊悟,容止可觀,人士見之,莫不敬異;有識者多以遠大許之。



 正光中,隨父之並州。性既恬默,又好山水,遂入晉陽西縣甕山讀書。孝昌初,津為定州刺史,愔亦隨父之職。以軍功除羽林監,賜爵魏昌男,不拜。及中山為杜洛周陷,全家被囚縶。未幾,洛周滅,又沒葛榮。榮欲以女妻之,又逼以偽職。愔乃託疾,密含牛血數合,於眾中吐之,仍陽喑不語。榮以為信然,乃
 止。永安初,還洛,拜通直散騎侍郎,年十八。



 元顥入洛時,愔從父兄侃為北中郎將,鎮河梁。愔適至侃處,便屬乘輿失守,夜至河。侃雖奉迎車駕北度,而潛南奔。愔固諫止之,遂相與扈從達建州。除通直散騎常侍。愔以世故未夷,志在潛退,乃謝病。與友人中直侍郎河間邢邵隱於嵩山。



 及莊帝誅爾朱榮,其從兄侃參贊帷幄。朝廷以其父津為並州刺史、北道大行臺,愔隨之任。有邯鄲人楊寬者,求義從出籓,愔請津納之。俄而孝莊幽崩,愔時適欲還都,行達邯鄲,過楊寬家,為寬所執。至相州,見刺史劉誕,以愔名家盛德,甚相哀念,付長史慕容白澤禁
 止焉。遣隊主鞏榮貴防禁送都,至安陽亭,愔謂榮貴曰:「僕百世忠臣,輸誠魏室,家亡國破,一至於此。雖曰囚虜,復何面目見君父之讎!



 得自縊於一繩,傳首而去,君之惠也。」榮貴深相矜感,遂與俱逃。愔乃投高昂兄弟。



 既潛竄累載,屬齊神武至信都,遂投刺轅門。便蒙引見,贊揚興運,陳訴家禍,言辭哀壯,涕泗橫集。神武為之改容,即署行臺郎中。南攻鄴,歷楊寬村,寬於馬前叩頭請罪。愔謂曰:「人不識恩義,蓋亦常理。我不恨卿,無假驚怖。」時鄴未下,神武命愔作祭天文,燎畢而城陷。由是轉大行臺右丞。于時霸圖草創,軍國務廣,文檄教令皆自愔及崔
 甗出。



 遭罹家難,常以喪禮自居,所食唯鹽米而已,哀毀骨立。神武愍之,常相開慰。



 及韓陵之戰,愔每陣先登。朋僚咸共怪歎曰:「楊氏儒生,今遂為武士,仁者必勇,定非虛論。」頃之,表請解職還葬,一門之內,贈太師、太傅、丞相、大將軍者二人;太尉、錄尚書及尚書令者三人;僕射、尚書者五人;刺史、太守者二十餘人。



 追榮之盛,古今未之有也。及喪柩進發,吉凶儀衛亙二十餘里,會葬者將萬人。是日,隆冬盛寒,風雪嚴厚,愔跣步號哭,見者無不哀之。尋征赴晉陽,仍居本職。



 愔從兄幼卿為岐州刺史,以直言忤旨見誅。愔聞之悲懼,因哀感發疾,後取急就雁
 門溫湯療疾。郭季素害其能,因致書恐之曰:「高王欲送卿於帝所。」仍勸其逃亡。愔遂棄衣冠於水濱,若見沈者。變易名姓,自稱劉士安。入嵩山,與沙門曇謨徵等屏居削跡。又潛之光州,因東入田橫島,以講誦為業,海隅之士謂之劉先生。



 太守王元景陰佑之。



 神武知愔存,遣愔從兄寶猗齎書慰喻;仍遣光州刺史奚思業令搜訪,以禮發遣。



 神武見之悅,除太原公開府司馬,轉長史,復授大行臺右丞,封華陰縣侯,遷給事黃門侍郎,妻以庶女。又兼散騎常侍,為聘梁使主。至碻磝,州內有愔家舊佛寺。



 精廬禮拜,見太傅容像,悲感慟哭,嘔血數升,遂發病
 不成行,輿疾還鄴。久之,以本官兼尚書吏部郎中。武定末,以望實之美,超拜吏部尚書,加侍中、衛將軍,侍學典選如故。



 天保初,以本官領太子少傅,別封陽夏縣男。又詔監太史,遷尚書右僕射。尚太原長公主,即魏孝靜后也。會有雉集其舍,又拜開府儀同三司、尚書右僕射,改封華山郡公。九年,徙尚書令,又拜特進、驃騎大將軍。十年,封開封王。文宣之崩,百寮莫有下淚,愔悲不自勝。濟南嗣業,任遇益隆,朝章國命,一人而已。推誠體道,時無異議。乾明元年二月,為孝昭帝所誅,時年五十。天統末,追贈司空公。



 愔貴公子,早著聲譽,風表鑒裁,為朝野所
 稱。家門遇禍,唯有二弟一妹及兄孫女數人。撫養孤幼,慈旨溫顏,咸出仁厚。重分義,輕貨財,前後賜與,多散之親族。群從弟姪十數人,並待而舉火。頻遭迍厄,冒履艱危,一飡之惠,酬答必重;性命之仇,捨而不問。典選二十餘年,獎擢人倫,以為已任。然取士多以言貌,時致謗言,以為愔之用人,似貧士市瓜,取其大者。愔聞,不以為意。其聰記彊識,半面不忘。每有所召,或單稱姓,或單稱名,無有誤者。後有選人魯漫漢,自言猥賤,獨不見識。愔曰:「卿前在元子思坊騎禿尾草驢,經見我不下,以方麴鄣面,我何不識卿?」漫漢驚服。又調之曰:「名以定體,漫漢果
 自不虛。」又令吏唱人名,誤以盧士深為士琛。士深自言,愔曰:「盧郎潤朗,所以比玉。」



 自尚公主後,衣紫羅袍、金鏤大帶。遇李庶,頗以為恥,謂曰:「我此衣服,都是內裁,既見子將,不能無愧。」



 及居端揆,經綜機衡,千端萬緒,神無滯用。自天保五年已後,一人喪德,維持匡救,實有賴焉。每天子臨軒,公卿拜授,施號發令,宣揚詔冊,愔辭氣溫辯,神儀秀發,百寮觀聽,莫不悚動。自居大位,門絕私交。輕貨財,重仁義,前後賞賜,積累巨萬,散之九族;架篋之中,唯有書數千卷。太保、平原王隆之與愔鄰宅,愔嘗見其門外有富胡數人,謂左右曰:「我門前幸無此物。」性周密畏
 慎,恆若不足,每聞後命,愀然變色。



 文宣大漸,以常山、長廣二王位地親逼,深以後事為念。愔與尚書左僕射平秦王歸彥、侍中燕子獻、黃門侍郎鄭子默受遺詔輔政,並以二王威望先重,咸有猜忌之心。初在晉陽,以大行在殯,天子諒訚,議令常山王在東館,欲奏之事皆先諮決,二旬而止。仍欲以常山王隨梓宮之鄴,留長廣鎮晉陽。執政復生疑貳,兩王又俱從至于鄴。子獻立計。欲處太皇太后於北宮,政歸皇太后。又自天保八年已來,爵賞多濫,至是,愔先自表解其開封王,諸叨竊榮恩者皆從黜免。由是嬖寵失職之徒盡歸心二叔。高歸彥初雖
 同德,後尋反動,以疏忌之跡,盡告兩王。可硃渾天和又每云:「若不誅二王,少主無自安之理。」宋欽道面奏帝,稱二叔威權既重,宜速去之。帝不許曰:「可與令公共詳其事。」愔等議出二王為刺史,以帝仁慈,恐不可所奏,乃通啟皇太后,具述安危。有宮人李昌儀者,北豫州刺史高仲密之妻,坐仲密事入宮。太后與昌儀宗情,甚相暱愛。太后以啟示之,昌儀密白太皇太后。愔等又議不可令二王俱出,乃奏以長廣王為大司馬、并州刺史,常山王為太師、錄尚書事。及二王拜職,於尚書省大會百寮,愔等並將同赴。子默止之云:「事不可量,不可輕脫。」愔云:「吾等
 至誠體國,豈有常山拜職,有不赴之理?何為忽有此慮?」



 長廣旦伏家僮數十人於錄尚書後室,仍與席上勳貴數人相知,並與諸勳胄約:行酒至愔等,我各勸雙盃,彼必致辭,我一曰「捉酒」,二曰「捉酒」,三曰「何不捉,」



 爾輩即捉。及宴如之。愔大言曰:「諸王反逆,欲殺忠良邪!尊天子,削諸侯,赤心奉國,未應及此。」常山王欲緩之,長廣王曰:「不可。」於是愔及天和、欽道皆被拳杖亂毆擊,頭面血流,各十人持之。使薛孤延、康買執子默於尚藥局。子默曰:「不用智者言,以至於此,豈非命也!」



 二叔率高歸彥、賀拔仁、斛律金擁愔等唐突入雲龍門。見都督叱利騷,招之不
 進,使騎殺之。開府成休寧拒門,歸彥喻之,乃得入。送愔等於御前。長廣王及歸彥在朱華門外。太皇太后臨昭陽殿,太后及帝側立。常山王以磚叩頭,進而言曰:「臣與陛下骨肉相連。楊遵彥等欲擅朝權,威福自己,自王公以還,皆重足屏氣,共相脣齒,以成亂階。若不早圖,必為宗社之害。臣與湛等為國事重,賀拔仁、斛律金等惜獻皇帝業,共執遵彥等,領入宮,未敢刑戮。專輒之失,罪合萬死。」帝時默然。領軍劉桃枝之徒陛衛,叩刀仰視,帝不睨之。太皇太后令卻仗不肯,又厲聲曰:「奴輩即今頭落!」乃卻。因問楊郎何在,賀拔仁曰:「一目已出。」太皇太后愴
 然曰:「楊郎何所能,留使不好邪?」乃讓帝曰:「此等懷逆,欲殺我二兒,次及我耳。何縱之?」帝猶不能言。太皇太后怒且悲,王公皆泣。太皇太后曰:「豈可使我母子受漢老嫗斟酌。」太后拜謝。常山王叩頭不止。太皇太后謂帝:「何不安慰爾叔?」帝乃曰:「天子亦不敢與叔惜,豈敢惜此漢輩!但願乞兒性命,兒自下殿去,此等任叔父處分。」遂皆斬之。長廣王以子默昔讒己,作詔書,故先拔其舌,截其手。



 太皇太后臨愔喪,哭曰:「楊郎忠而獲罪。」以御金為之一眼,親內之,曰:「以表我意。」常山亦悔殺之。先是童謠曰:「白羊頭毣禿,羖䍽頭生角。」又曰:「羊羊吃野草,不吃野草
 遠我道,不遠打爾腦。」又曰:「阿鷿姑,禍也;道人姑夫,死也。」羊為愔也,「角」文為用刀,「道人」謂廢帝小名,太原公主嘗作尼,故曰「阿鷿姑」,愔、子獻、天和皆尚帝姑,故曰「道人姑夫」云。



 於是乃以天子之命,下詔罪之;罪止一身,家口不問。尋復簿錄五家,王晞固諫,乃各沒一房,孩幼盡死,兄弟皆除名。



 遵彥死,仍以中書令趙彥深代總機務。鴻臚少卿陽休之私謂人曰:「將涉千里,殺騏驥而榮蹇驢,可悲之甚!」愔所著詩賦表奏書論甚多,誅後散失,門生鳩集所得者萬餘言。



 燕子獻字季則,廣漢下洛人。少時相者謂曰:「使役在胡、
 代,富貴在齊、趙。」



 後遇周文於關中創業,用為典簽,將命使於蠕蠕。子獻欲驗相者之言,來歸。神武見之大悅。神武舊養韓長鸞姑為女,是為陽翟公主,遂以嫁之,甚被待遇。文宣時,官至侍中。濟南即位,委任彌重,除尚書右僕射。子獻素多力,頭少髮,當狼狽之際,排眾走出省門,斛律光逐而禽之。子獻歎曰:「丈夫為計遲,遂至此!」天統五年,追贈司空。天和事見兄元傳。



 鄭頤字子默,彭城人。高祖據,魏彭城太守,自滎陽徙焉。頤聰敏,頗涉文義,而邪險不良。初為太原公東閣祭酒。天保世,稍遷中書侍郎。與宋欽道特相友愛,欽道每師
 事之。楊愔始輕宋、鄭,不為之禮。俄而自結人主,稍不可制。欽道舊與濟南款狎,共相引致,無所不言。乾明初,拜散騎常侍,兼中書侍郎。二人權將楊愔相埒。愔見害之時,邢子才流涕曰:「楊令君雖其人,死日恨不得一佳伴。」頤後與愔同詔追贈殿中尚書、廣州刺史。頤弟抗,字子信,頗有文學。武平末,兼左右郎中,待詔文林館。



 楊敷,字文衍,播族孫也。高祖暉,洛州刺史,贈恆農公,謚曰簡。曾祖恩,河間太守。祖鈞,博學彊識,頗有乾用。位七兵尚書、北道行臺、恆州刺史、懷朔鎮將,贈侍中、司空公,進封臨貞縣伯,謚曰恭。父暄,字宣和。性通朗,彊識有學。
 位諫議大夫,以別將從廣陽王深征葛榮,遇害。贈殿中尚書、華州刺史。



 敷少有志操,重然諾,人景慕之。魏建義初,襲祖鈞爵臨貞縣伯。稍遷廷尉少卿,斷獄以平允稱。周孝閔踐阼,進爵為侯。天和中,為汾州刺史,進爵為公。齊將段孝先率眾來寇,城陷見禽。齊人方任用之,敷不為屈,遂以憂憤卒於鄴。子素。



 素子處道,少落拓有大志,不拘小節。世人多未之知,唯從祖寬深異之,每謂子孫曰:「處道逸群絕倫,非常之器,非汝曹所逮。」後與安定牛弘同志好學,研精不倦,多所通涉。善屬文,工草隸書,頗留意風角。美鬚髯,有英傑之
 表。



 周大冢宰宇文護引為中外記室,轉禮曹,加大都督。周武帝親總萬機,素以其父守節陷齊,未蒙朝命,上表申理,至於再三。帝大怒,命左右斬之。素又言曰:「臣事無道天子,死其分也。」帝悟其言,贈敷使持節、大將軍、譙、廣、復三州刺史,謚曰忠壯。拜素車騎大將軍、儀同三司,漸見禮遇。常令為詔,下筆立成,詞義兼美。帝嘉之,謂曰:「善相自勉,勿憂不富貴。」素應聲曰:「臣但恐富貴來逼臣,臣無心圖富貴。」



 及平齊之役,素請率麾下先驅,帝從之。賜以竹策曰:「朕方欲大相驅策,故用此物賜卿。」從齊王憲與齊人戰於河陰,以功封清河縣子,授司城大夫。復從
 憲拔晉州,屯兵雞棲原。齊主以大軍至,憲懼,宵遁。為齊兵躡,眾多敗散。素與驍將十餘人盡力苦戰,憲僅而獲免。齊平,加上開府,改封成安縣公。尋從王軌破陳將吳明徹於呂梁,行東楚州事。封弟慎為義安侯。陳將樊毅築城泗口,素擊走之,夷毅所築城。宣帝即位,襲父爵臨貞縣公,以弟約為安成公。尋從韋孝寬徇淮南。



 及隋文帝為丞相,素深自結納,帝甚器之,以為汴州刺史。至洛陽,會尉遲迥作亂。滎州刺史宇文胄據武牢應迥,素不得進。帝拜素大將軍,擊胄破之。遷徐州總管,位柱國,封清河郡公,以弟岳為臨貞公。及隋受禪,加上柱國,拜御
 史大夫。



 其妻鄭氏性妒悍,素忿之曰:「我若作天子,卿定不堪為皇后。」鄭氏奏之,由是坐免。



 上方圖江表。先是,素數進取陳計。未幾,拜信州總管,賜錢百萬、錦千段、馬二百匹遣之。素居永安,造大艦,名曰五牙,上起樓五層,高百餘尺,左右前後置六檣竿,並高百五十尺,容戰士八百人,旗幟加於上。次曰黃龍,置兵百餘人。



 自餘平乘、舴艋等各有差。及大舉攻伐,以素為行軍元帥,引舟師趣三硤。至流頭灘,陳將戚欣以青龍百餘艘屯兵守狼尾灘,以遏軍路。共地險峭,諸將患之。素曰:「負勝在此一舉,若晝日下船,彼則見我,灘流迅激,制不由人,則吾失其
 便。」



 乃夜掩之。素親率黃龍十艘,銜枚而下。遣開府王長襲從南岸擊欣別柵。令大將軍劉仁恩趣白沙北岸。比明而至,擊之,欣敗。虜其眾,勞而遣之,秋毫不犯,陳人大悅。素率水軍東下,舟艦被江,旌甲曜日。素坐平乘大船。容貌雄偉,陳人望之,懼曰:「清河公即江神也。」



 陳南康內史呂仲肅屯岐亭,正據江峽,於北岸纜巖綴鐵鎖三條,橫截上流,以遏戰船。素與仁恩登陸俱發,先攻其柵;仲肅軍夜潰,素徐去其鎖。仲肅復據荊州之延洲。素遣巴蜒卒數千,乘五牙四艘,以檣竿碎賊十餘艦,遂大破之,仲肅僅以身免。陳主遣其信州刺史顧覺鎮安蜀城,荊
 州刺史陳紀鎮公安,皆懼而走。巴陵以東,無敢守者。湘州刺史岳陽王陳叔慎請降。素下至漢口,與秦孝王會,乃還。拜荊州總管,進爵郢國公,真食長壽縣千戶;以其子玄感為儀同三司,玄獎為清河郡公;賜物萬段,粟萬石,加之金寶;又賜陳主妹、女妓十四人。素言於上曰:「里名勝母,曾子不入,逆人王誼前封郢,臣不願與同。」於是改封越國公。尋拜納言,轉內史令。



 俄而江南人李稜等為亂,以素為行軍總管討之。帝命平定日,男子悉斬,女婦賞征人,在陣免者從賤。賊朱莫問自稱南徐州刺史,以盛兵據京口。素舟師入自楊子津,進擊破之。晉陵顧
 世興自稱太守,與其都督鮑遷等復來拒戰。素逆擊破之,執遷,虜三千餘人。進擊無錫賊帥葉皓,又平之。吳郡沈玄懀、沈傑等以兵圍蘇州,刺史皇甫績頻戰不利,素率眾援之。玄懀勢迫,走投南沙賊帥陸孟孫。素擊孟孫於松江,大破之,禽孟孫、玄懀。黝、歙賊帥沈雪、沈能據柵自固,又攻拔之。



 江浙賊高智慧自號東揚州刺史,吳州總管五原公元契鎮會稽,以其兵盛而降之。



 智慧盡屠其眾,契自殺。智慧有船艦千餘艘,屯據要害,兵甚勁。素擊之,自旦至申,苦戰破之。智慧逃入海。躡之,從餘姚汎海趣永嘉。智慧來拒戰,素擊走,賊帥汪文進自稱天子,
 據東陽,署其徒蔡道人為司空,守樂安。素進討。悉平之。又破永嘉賊帥沈孝徹。於是步道向天台,指臨海郡。遂捕遺逸,前後百餘戰,智慧遁守閩越。上以素久勞於外,詔令馳傳入朝,加子玄感上開府,賜彩八千段。素以餘寇未殄,恐為後患,又自請行。詔以素為元帥,復乘傳至會稽。



 先是,泉州人王國慶,南安豪族也,殺刺史劉弘,據州為亂。自以海路艱阻,非北人所習,不設備伍。素泛海奄至,國慶遑遽,棄州走。素分遣諸將,水陸追捕。



 時南海先有五六百家,居水為亡命,號曰遊艇子;智慧、國慶欲往依之。素乃密令人說國慶,令斬智慧以自效。因慶乃
 斬智慧於泉州。自余支黨悉降,江南大定。上遣左領軍將軍獨孤陀至浚儀迎勞,比到京師,問者日至。拜素子玄獎儀同,賜黃金四十斤,加銀瓶,實以金錢,縑三千段、馬二百匹、羊三千口、田百頃、宅一區。



 代蘇威為尚書右僕射,與高熲專掌朝政。素性疏而辯,高下在心,朝貴之內,頗推高熲,敬牛弘,厚接薛道衡,視蘇威蔑如也。自餘朝臣,多被陵轢。其才藝風調,優於高熲。至於推誠體國,處物平當,有宰相識度,不如熲遠矣。



 尋令素監營仁壽宮,素遂夷山堙谷,督役嚴急,作者多死,宮側時聞鬼哭。及宮成,上令高熲前視,奏稱頗傷綺麗,大損人丁。帝不
 悅。素懼,即於北門啟獨孤皇后曰:「帝王法有離宮別館,今天下太平,造一宮何足損費。」后以此理諭上,上乃解。於是賜錢百萬、綿絹三千段。



 開皇十八年,突厥達頭可汗犯塞,以素為靈州道行軍總管。出塞討之,賜物二千段、黃金百斤。先是諸將與虜戰,每慮胡騎奔突,皆戎車步騎相參,與鹿角為方陣,騎在內。素曰:「此乃自固之道。」於是悉除舊法,令諸軍為騎陣。達頭聞之,大喜,以為天賜,下馬仰天而拜,率精騎十餘萬至。素奮擊,大破。達頭被重創而遁,眾號哭而去。優詔賜縑二萬匹及萬釘寶帶,加子玄感位大將軍,玄獎、玄縱、積善並上儀同。



 素多
 權略,乘機赴敵,應變無方。然大抵馭戎嚴整,有犯令者,立斬無所寬貸。



 每將臨寇,輒求人過失而斬之,多者百餘人,少不下數十,流血盈前,言笑自若。



 及對陣,先令一二百人赴敵,陷陣則已,如不能陷而還,無問多少,悉斬之。又令二百人復進,還如向法。將士股慄,有必死心,由是戰無不勝,稱為名將。素時貴倖,言無不從。其從素征代者,微功必錄。至於他將,雖大功,多為文吏所譴卻。



 故素雖嚴忍,士亦以此願從。



 二十年,晉王廣為靈、朔道行軍元帥,素為長史,王卑躬交素。及為太子,素之謀也。仁壽初,代高熲為尚書左僕射,賜良馬十匹、牝馬二百匹、
 奴婢百口。其年,以素為行軍元帥,出雲中擊突厥,連破之。突厥走,追至夜及之。將復戰,恐賊越逸,令其騎稍後,於是親將兩騎並降突厥二人與虜並行,不之覺也。侯其頓舍未定,趣後騎掩擊,大破之。自是突厥遠遁,磧南無復虜庭。以功進子玄感位柱國,玄縱為淮南郡公,賞物二萬段。



 及獻皇后崩,山陵制度多出於素。上善之,下詔曰:「君為元首,臣則股肱,共理百姓,義同一體。上柱國、尚書左僕射、仁壽宮大監、越國公素,志度恢弘,機鑒明遠,懷佐時之略,包經國之才。王業初基,霸圖肇建,策名委質,受脤出師,禽翦凶魁,克平虢、鄭。頻承廟算,揚旌江
 表;每稟戎律,長驅塞垣。南指而吳越肅清,北臨而獯獫摧服。自居端揆,參贊機衡,當朝正色,直言無隱。論文則詞藻從橫,語武則權奇間出,既文且武,唯朕所命。任使之處,夙夜無怠。獻皇后奄離六宮,遠日云及,塋兆安厝,委素經紀。然葬事依禮,唯卜泉石,至如吉凶,不由於此。素義存奉上,情深體國,欲使幽明俱泰,永保無窮。以為陰陽之書,聖人所作,禍福之理,特須審慎。乃遍歷川原,親自占擇,志圖元吉,孜孜不已。遂得神皋福壤,營建山陵。論素此心,事極誠孝,豈與平戎定寇,比其功業,若不加褒賞,何以申茲勸勵。可別封一子義康郡公、邑萬戶,子
 子孫孫承襲不絕,餘如故。」並賜田三十頃、絹萬匹、米萬石;金缽一,實以金;銀缽一,實以珠;並綾錦五百段。



 時素貴寵日隆。其弟約、從父文思、弟紀及族父異並尚書、列卿,諸子無汗馬勞,位柱國、刺史。家僮數千,後庭妓妾曳綺羅者以千數;第宅華侈,制擬宮禁。



 有鮑亨者善屬文,殷胄者工草隸,並江南士人,因高智慧沒為奴。親戚故吏,布列清顯。其盛近古未聞。煬帝初為太子,忌蜀王秀,與素謀之,構成其罪,後竟廢黜。



 朝臣有違忤者,雖至誠體國如賀若弼、史萬歲、李綱、柳彧等,素皆陰中之。若有附會及親戚,雖無才用,必加進擢。朝廷靡然,莫不畏附。
 唯兵部尚書柳述,以帝婿之重,數於上前面折素。大理卿梁毗,抗表言素作威作福。上漸疏忌之,後因出敕曰:「僕射,國之宰輔,不可躬親細務,但三五日一度向省評論大事。」外示優崇,實奪之權,終仁壽之末,不復通判省事。上賜王公已下射,素箭為第一,上手以外國所獻金精盤價直巨萬以賜之。四年,從幸仁壽宮,宴賜重疊。



 及上不豫,素與兵部尚書柳述、黃門侍郎元巖等入侍疾。時皇太子入居大寶殿,慮上有不諱,須豫防擬,乃手自為書,封出問素。素條錄事狀,以報太子。宮人潛送於上,上覽而大恚。所寵陳貴人又言太子無禮。上遂發怒,欲
 召庶人勇。太子謀之素,素矯詔追東宮兵士帖上臺宿衛,門禁出入,並取宇文述、郭衍節度。又令張衡侍疾。上以此日崩,由是頗有異論。



 會漢王諒反,遣茹茹天保往東蒲州,燒斷河橋,又遣王子並力拒守。素將輕騎五千襲之。潛於渭口宵濟,比明擊之。天保敗,子懼,以城降。有詔徵還。初素將行,計日破賊,皆如所量。帝於是以素為并州道行軍總管、河北道安撫大使,討諒。



 時晉、絳、呂三州並為諒城守,素各以二千人縻之而去。諒遣趙子開擁眾十餘萬,築絕徑路,屯據高壁,布陣五十里。素令諸將以兵臨之,自以奇兵深入霍山,緣崖谷而進,直
 指其營,一戰破之,諒所署介州刺史梁脩羅屯介休,聞素至,懼,棄城而走。進至清源,去並州三十里。率其將王世宗、趙子開、蕭摩訶等來拒戰,又擊破之,禽蕭摩訶。退保并州,素進兵圍之。諒窮而降,餘黨悉平。帝遣素弟脩武公約齎手詔勞,素上表陳謝。其月,還京師。從駕幸洛陽,以素領營東京大監。以平諒功,拜其子萬石、仁行、侄玄挺皆儀同三司,賚物五萬段、羅綺千匹、諒之妓妾二十人。大業元年,遷尚書令賜東京甲第一區、物二千段,尋拜太師,餘官如故。



 前後賞錫不可勝計。明年,拜司徒,改封楚公,真食二千五百戶。其年病薨,謚曰景武。
 贈光祿大夫、太尉公、弘農河東絳郡臨汾文城河內汲郡長平上黨河十郡太守,給轀輬車、班劍三十人、前後部羽葆鼓吹、粟麥五千石、物五千段,鴻臚監護喪事。



 帝又下詔立碑,以彰盛美。素嘗以五言詩七百字贈番州刺史薛道衡,詞氣穎拔,風韻秀上,為一時盛作。未幾而卒,道衡歎曰:「人之將死,其言也善,若是乎!」



 《集》十卷。



 素雖有建立策及平楊諒功,然特為帝猜忌,外示殊禮,內情甚薄。太史言楚分野有大喪,因改封素於楚。寢疾之日,帝每令名醫診侯,賜以上藥;然密問醫人,恒恐不死。素又自知名位已極,不肯服藥,變不將慎。每語弟約曰:「
 我豈須更活邪?」



 素貪財貨,營求產業,東西京居宅侈麗,朝毀夕復,營繕無已。爰及諸方都會之處,邸店水磑田宅以千百數。時議以此鄙之。子玄感。



 玄感少時晚成,人多謂之癡。唯素每謂所親曰:「此兒不癡也。」及長,美鬚髯,儀貌雄俊,好讀書,便騎射。弱冠,以父軍功位柱國,與其父俱為第二品,朝會則齊列。後文帝命玄感降一等,玄感拜謝曰:「不意陛下寵臣之甚,許以公庭獲展私敬。」初拜郢州刺史,到官潛布耳目,察長吏能不,纖介必知,吏人敬服,皆稱其能。後轉宋州刺史,父憂去職。歲餘,拜鴻臚卿,襲爵楚公,遷禮部尚書。性雖驕
 居,而愛重文學,四海知名之士多趨其門。



 後見朝綱漸紊,帝又猜忌日甚,內不自安,遂與諸弟潛謀廢帝立秦王浩。及從征吐谷渾,還至達斗拔谷,時從官狼狽,玄感欲襲擊行宮。其叔慎曰:「士心尚一,國未有釁,不可圖也。」玄感乃止。時帝好征伐,玄感欲立威名,陰求將領,以告兵部尚書段文振。振以白帝,帝嘉之,謂群臣曰:「將門有將,故不虛也。」於是賚物千段,禮遇益隆,頗預朝政。



 帝征遼東,令玄感黎陽督運。遂與武賁郎將王仲伯、汲郡贊治趙懷義等謀,不時進發。帝遣使者逼促,玄感揚言曰:「水路多盜,不可前後而發。」其弟武賁郎將玄縱、鷹
 揚郎將萬石並從幸遼東,玄感潛遣人召之。時來護兒以舟師自東萊,將入海趣平壤城,軍未發。玄感無以動眾,乃遣家奴偽為使,從東方來,謬稱護失軍期而反。玄感遂入黎陽縣,閉城大募勇夫。於是取颿布為牟甲,署置官屬皆準開皇之舊。移書傍郡以討護為名,令發兵會於倉所。以東光縣尉元務本為黎州刺史,趙懷義為衛州刺史,河內郡主簿唐禕為懷州刺史,有眾且一萬,將襲洛陽。唐禕至河內,馳往東都告之。越王侗、戶部尚書樊子蓋等勒兵備禦。脩武縣人相率守臨清關,玄感不得濟,遂於汲郡南度河。從亂如市,數日,屯兵上春門,
 眾至十餘萬。子蓋令河南贊務裴弘策拒之,弘策戰敗,父老競致牛酒。玄感屯兵尚書省,每有誓眾曰:「我身為上柱國,家累巨萬金,至富貴,無所求也。今者不顧破家滅族者,為天下解倒懸之急,救黎元之命耳。」眾皆悅,詣轅門請自效者日數千。及與樊子蓋書曰:夫建忠立義,事有多途,見機而作,蓋非一揆。昔伊尹放太甲於桐宮,霍光廢劉賀於昌邑,此並公度內,不能一二披陳。高祖文皇帝誕膺天命,造茲區宇,在璇璣以齊七政,握金鏡以馭六龍,無為而至化流,垂拱而天下乂。今上纂承寶歷,宜固洪基,乃自絕于天,殄人敗德。頻年肆眚,盜賤於
 是滋多;所在脩營,人力為之凋盡。荒淫酒色,子女必被其侵;耽玩鷹犬,禽獸皆離其毒。朋黨相扇,貸賄公行,納邪佞之言,杜正直之口。加以轉輸不息,徭役無期;士卒填溝壑,骸骨蔽原野;黃河之北則千里無煙,江、淮之間則鞠為茂草。



 玄感世荷國恩,位居上將。先公奉遺詔曰:「好子孫為我輔弼之,惡子孫為我屏黜之。」所以上稟先旨,下順人心,廢此淫昏,更立明哲。今四海同心,九有咸應,士卒用命,如赴私仇,人庶相趨,義形公道。天意人事,較然可知。公獨害孤城,勢何支久?願以黔黎在念,社稷為心,勿拘小禮,自貽伊戚。誰謂國家,一旦至此!執筆潸
 然,言無所具。



 遂進逼東都城。刑部尚書衛玄率眾自關中來援東都,以步騎二萬度瀍、澗挑戰。



 玄感偽北,玄逐之,伏兵發,前軍盡沒。後數日,玄復與玄感戰。兵始合,玄感詐令人大呼曰:「官軍已得玄感矣。」玄軍稍怠,玄感與數千騎乘之,大潰,擁八千人而去。玄感驍勇多力,每戰,親運長矛,身先士卒,喑鳴叱吒,所當莫不震懾,論者方之項羽。又善撫馭,士樂致死。由是戰無不捷。玄軍日蹙,糧又盡,乃悉眾決戰,陣於北邙,一日間戰十餘合。玄感弟玄挺中流矢而斃,玄感稍卻。樊子蓋復遣兵攻尚書省,又殺數百人。



 帝遣武賁郎將陳稜攻元務本於黎陽。
 武衛將軍屈突通屯河陽,左翊衛大將軍宇文述發兵繼進,右驍衛大將軍來護兒復來趙援。玄感與前戶部尚書李子雄計曰:「屈突通曉兵事,若度河則勝負難決,不如分兵拒之。不能濟,則樊、衛失援。」



 玄感然之,將拒通。子蓋知其謀,數擊其營,玄感不果進。通遂濟河,軍於破陵。



 玄感為兩軍,西拒衛玄,東拒屈突通。子蓋復出兵大戰,玄感軍頻北。復與子雄計,子雄勸之直入關中,開永豐倉振貧乏,三輔可指麾而定。據有府庫,東面而爭天下,此亦霸王之業。



 會華陰諸楊請為鄉導,玄感遂釋洛陽,西圖關中。宣言已破東都,取關西。宇文述等諸軍躡
 之。至弘農宮,父老遮說玄感曰:「宮城空虛,又多積粟,攻之易下。



 進可絕敵人之食,退可割宜陽之地。」玄感以為然,留攻三日,城不下,追兵遂至。



 玄感西至閿鄉,上槃豆,布陣亙五十里,與官軍且戰且行,一日三敗。復陣於董杜原,諸軍大敗之。玄感獨與十餘騎竄林木間,將奔上洛。追騎至,玄感叱之,皆懼而返走。至葭蘆戍,窘迫,獨與弟積善步行,謂積善曰:「事敗矣,我不能受人戮辱,汝可殺我。」積善殺之,因自刺不死,為追兵所執,與玄感首俱送行在所,磔其屍於東都市,三日,復臠而焚之。餘黨悉平。



 其弟玄獎為義陽太守,將歸玄感,為郡丞周旋玉所
 殺。玄縱弟萬石,自帝所逃歸,至高陽,止傳舍,監事許華與郡兵執之,斬於涿郡。萬石弟仁行,官至朝議大夫,斬於長安。並具梟磔。公卿請改玄感姓為梟氏,詔可之。



 玄感之亂,有趙元淑者預謀,誅。又有劉元進,亦舉兵應之。



 元淑,博陵人。父世模,初從高寶,後以眾歸周,授上開府,寓居京兆之雲陽。



 隋文帝踐阼,恒典宿衛。後從晉王伐陳,力戰而死。朝廷以其身死王事,以元淑襲父本官,賜物三千段。元淑性疏誕,不事產業,家徒壁立。後授驃騎將軍,將之官,無以自給。時長安富人宗連家累千金,仕周為三原令,有季女,慧而有色。連每求賢夫,聞元淑,請
 與相見。連有風儀,美談笑,元淑亦慕之。及至其家,服玩居處,擬於將相,酒酣,奏女樂,元淑所未見也。及出,連又致殷勤。元淑再三來,宴樂更侈於前。因問所須,盡買與之,元淑致謝,連復拜求以女妻之。元淑感而納焉,遂為富人。



 從楊素平楊諒,以功進位柱國,歷德州刺史、潁川太守,並有威惠。入為司農卿。玄感有異志,遂與結交。遼東之役,領將軍、典宿衛,加光祿大夫,封葛國公。



 明年,帝復征高麗,以元淑鎮監渝。及玄感作亂,其弟玄縱自駕所逃歸,路經臨渝。



 元淑出其小妻魏氏見玄縱,對宴極歡,因與通謀,并受玄縱賂遺。及玄感敗,人有告其事者,
 帝以屬吏,元淑及魏氏俱斬於涿郡,籍沒其家。



 元進,餘杭人。少好任俠,為州里所宗,兩手各長尺餘,臂垂過膝。屬遼東之役,百姓騷動,元進自以相表非常,遂聚亡命。會玄感起於黎陽,元進應之。旬月,眾至數萬,將度江而玄感敗。吳郡朱燮、晉陵管崇亦舉兵,有眾七萬,共迎元進,奉以為主。據吳郡,稱為天子,以燮、崇俱為僕射,署百官。帝令將軍吐萬緒、光祿大夫魚俱羅討焉。為緒所敗,硃燮戰死。俄而緒、俱羅並得罪。江都郡丞王世充發兵擊之。有大流星墜於江都,未及地而南逝,磨拂竹木皆有聲,至吳郡而落于地。



 元進惡之,令掘地入二丈得
 一石,徑丈餘。數日,失石所在。世充度江,元進遣兵人各持茅,因風縱火。世充大懼,將棄營。遇反風火轉,元進眾懼燒而退,世充大破之。元進及崇俱為世充所殺。世充坑其眾於黃亭澗,死者三萬人。其後董道沖、沈法興、李子通等並乘此而起。素母弟約。



 約字惠伯。童兒時嘗登樹,墜地為查傷,由是竟為宦者。性如沈靜,內多譎詐,好學彊記。素友愛之,凡有所為,先籌於約而行。在周末,以素軍功賜爵安成縣公,拜上儀同三司。文帝受禪,歷位長秋卿、鄜州刺史、宗正、大理三少卿。



 時皇太子無寵,晉王廣規奪宗,以素幸於上而雅
 信約,乃用張衡計,遣宇文述大以金寶賂約,因通王意,說之曰:「夫守正履道,固人臣之常致;反經合義,亦達者之令圖。自古賢人君子,莫不與時消息,以避禍患。公兄弟功名蓋世,用事有年,朝臣為足下家所屈辱者,可勝數哉?又儲宮以所欲不行,每切齒於執政。公雖自結於人主,而欲危公者亦多矣。主上一旦棄群臣,公亦何以取庇?今皇太子失愛於皇后,主上素有廢黜之心,此公所知也。今若請立晉王,在賢兄之口耳。誠能因此時建大功,王必鎮銘於骨髓,斯則去累卵之危,成太山之安也。」約然之,又白素。素本凶險,聞之大喜,乃撫掌曰:「吾智
 慧殊不及此,賴汝起餘。」約知其計行,復謂素曰:「今皇后之言,上無不用,宜因機會,早自結託,則匪惟長保榮祿,傳祚子孫。又晉王傾身禮士,聲名日盛;躬履節儉,有主上之風。以約料之,必能安天下。兄若遲疑,一旦有變,令太子用事,恐禍至無日。」素遂行其策,太子果廢。



 及晉王入東宮,引約為左庶子,封脩武公,進位大將軍。及帝崩,遣約入京,易留守者,縊殺庶人勇,然後陳兵發凶問。煬帝聞之曰:「令兄之弟,果堪大任。」



 即位數日,拜內史令。約有學術,兼達時務,帝甚任之。後加右光祿大夫。



 及帝在東都,令約詣京師享廟,行至華陰,見其先墓。遂枉道拜
 哭,為憲司所劾,坐免官。尋拜浙陽太守。其兄子玄感時為禮部尚書,與約恩義甚篤,既愴分離,形於顏色。帝謂曰:「公比憂瘁,得非為叔也?」玄感再拜流涕曰:「誠如聖旨。」



 帝亦思約廢立功,由是徵入朝。未幾卒,以素子玄挺後之。



 穆字紹叔,暄弟也。仕魏,華州別駕。孝武末,弟寬請以澄城縣伯讓穆,詔許之。終于並州刺史,贈開府儀同三司、華州刺史。



 穆弟儉,字景則。偉容儀,有才行。位北雍州刺史,政尚寬惠,夷夏安之。後從破齊神武於沙苑,封夏陽縣侯,位開府儀同三司、華州刺史。卒,謚靜。



 子異,字文殊。美風儀,有器局。髫齔就學,日誦千言,見者奇之。九歲丁父憂,哀毀過禮,殆將滅性。及免喪之後,絕慶弔,閉戶讀書。數年之間,博涉書記。



 周閔帝時,為寧都郡太守,甚有能名,賜爵樂昌縣子,後數以軍功進爵為侯。隋文帝作相,行濟州事。及踐阼,拜宗正少卿,加上開府。蜀王秀之鎮益州也,朝廷盛選綱紀,以異方直,拜益州總管長史,尋遷西南道行臺兵部尚書。後歷宗正卿、刑部尚書,出為吳州總管,甚有能名。時晉王廣鎮揚州,詔令異每歲一與王相見,評論得失,規諫疑闕。卒於官。子虔遜。



 寬子蒙仁,儉弟也。少有大志,每與諸兒童遊處,必擇高大之物坐之,見者咸異焉。及長,頗解屬文,尤尚武藝。弱冠,除奉朝請。父鈞出鎮恆州,請隨從展效,乃授高闕戍主。既而蠕蠕亂,共主阿那瑰奔魏,魏帝詔鈞衛送,寬亦從行。時北邊賊起,攻圍鎮城。鈞卒,城人等推寬守禦。尋而城陷,寬乃北走蠕蠕,後討六鎮賊破,寬始得還朝。



 廣陽王深與寬素相暱,深犯法得罪,寬被逮捕。孝莊為侍中,與寬有舊,藏之於宅,遇赦得免。除宗正丞。北海王顥少相器重,時為大行臺北征葛榮,欲啟寬為左丞。寬辭以孝莊厚恩未報,義不見利而動。顥未之許,顥妹婿
 李神軌謂顥曰:「匹夫猶不可奪志,況義士乎。」乃止。



 孝莊踐阼,累遷洛陽令,以都督從太宰、上黨王元穆討平邢杲。師未還。屬元顥入洛,莊帝出居河內。天穆懼,集諸將謀之。寬勸天穆徑取成皋,會兵伊、洛。



 天穆然之,乃趣成皋,令寬與爾朱兆為後拒。尋以眾議不同,乃回赴石濟。寬夜行失道,遂後期,諸將咸言寬少與北海周旋,今不來矣。天穆答曰:「楊寬非輕去就者也,吾當為諸君明之。」言訖,候騎白寬至。天穆撫髀而笑曰:「吾固知其必來。」



 遽出帳迎,握其手曰:「是所望也」與天穆俱謁孝莊於太行。仍為都督,從平河內,進圍北中。時梁陳慶之為顥勒
 兵守北門,天穆駐馬圍外,遣寬至城下說慶之,不答,久之乃曰:「賢兄撫軍在,頗欲相見不?」寬答:「僕兄既力屈凶威,迹淪逆黨,人臣之理,何煩相見。」天穆聞之,自此彌敬。



 孝莊反正,除太府卿、華州大中正,封澄城縣伯。爾朱榮被誅,其從弟世澄等出據河橋,還逼京師,進寬使持節、大都督,隨機捍禦。世隆謂寬曰:「豈忘大宰相知之深也?」寬答曰:「太宰見愛以禮,人臣之交耳,今日之事,事君之節。」



 及爾朱兆陷洛陽,囚執孝莊帝,寬還洛不可,遂自成皋奔梁。至建鄴,聞莊帝弒崩,寬發喪盡禮,梁武義之。尋而禮送還。孝武初,除給事黃門侍郎。



 孝武與齊神武有
 隙,遂召募驍勇,廣增宿衛,以寬為閣內大都督,專總禁旅。



 從孝武入關,兼吏部尚書,錄從駕勳,進爵華山郡公。大統初,遷太子太傅。五年,除驃騎大將軍、開府儀同三司、都督、東雍州刺史,即本州也。廢帝初,為尚書左僕射、將作大監,坐事免。周明帝初,拜大將軍,從駕蘭祥討吐谷渾,破之,別封宜陽縣公。除小冢宰,轉御正中大夫。武成二年,詔寬與麟趾殿學士參定經籍。



 寬性通敏,有器幹。頻牧數州,號稱清簡。歷居臺閣,有當官之譽。然與柳機不協,案成其罪,時論頗以此譏之。保定元年,除總管梁興等十九州諸軍事、梁州刺史。薨於州,贈華、陜、虞、上、
 潞五州刺史,謚曰元。子文恩。



 文恩字溫才。在周,年十一,拜車騎大將軍、儀同三司、散騎常侍。尋以父功,封新豐縣子。天和初,行武都太守。十姓獠反,文恩討平之。復行翼州事。黨項羌叛,文恩又討平之。進擊資中、武康、隆山等生獠及東山獠,並破之。從陳王攻齊河陰城,又從武帝攻拔晉州,授上儀同三司,改封承寧縣公。壽陽劉叔仁作亂,從清河公宇文神舉討之,戰於專井,在陣禽叔仁。又別從王誼破賊於鯉魚柵。後累以軍功遷果毅左旅下大夫。



 隋文帝為丞相,從韋孝寬拒尉遲迥於武陟,與行軍總管宇文述擊走其
 將李俊,遂解懷州圍。破尉遲惇,平鄴城,皆有功,進授上大將軍,改封洛川縣公,尋拜隆州刺史。開皇元年,進爵正平郡公。後為魏州刺史,甚有惠政,及去職,吏人思之,為立碑頌德。轉冀州刺史。



 煬帝嗣位,徵為戶部尚書,轉納言,改授右光祿大夫。從幸江都宮,以足疾,不堪趨奏,復授戶部尚書,位右光祿大夫。卒官,謚曰定。初文恩當襲父爵,自以非嫡,遂讓弟紀,當世多之。



 紀字溫範,少剛正,有器局。在周,襲爵華山郡公。累遷安州總管長史,將兵迎陳降將王瑗於齊安,與陳將周法尚遇,擊走之,以功進開府。入為虞部下大夫。



 文帝為丞相,改封汾陰縣
 公。從梁睿討王謙,以功進授上大將軍。歷資州刺史、宗正少卿,坐事除名。後尋復其爵位,拜熊州刺史,改封上明郡公。除宗正卿,兼給事黃門侍郎,判禮部尚書事。遷荊州總管。卒,謚曰恭。



 論曰:楊播兄弟俱以忠毅謙謹,荷內外之任;公卿牧守,榮赫累朝,所謂門生故吏遍於天下。而言色恂恂,出於誠至;恭德慎行,為世師範,漢之陳紀,門法所不過焉。後魏以來,一門而已。諸子秀立,青紫盈庭,積善之慶,蓋有憑也。及逆胡擅朝,淫刑肆毒,以斯族而遇斯禍,何報施之反哉。愔雅道風流,早同標致,公望人物所推。夫處亂
 虐之世,當機衡之重,朝有善政,是也。及寄天下之命,託六尺之孤,旬朔未幾,身亡君辱。進不能送往事居,觀幾衛主;退不能保身全名,辭寵招福。朝廷之釁,既已仗義斷恩;猜忌之塗,無容推心受亂。是知變通之術,非所長也。處道少而輕俠,俶儻不羈;兼文武之資,包英奇之略,志懷遠大,以功名自許。屬隋文帝將清六合,委以腹心之寄。掃妖氛於牛斗,江海恬波;摧驍猛於龍庭,匈奴遠遁。若其夷凶靜亂,功臣莫居其右;覽其奇策高文,足為一時之傑。然以智詐自立,不由仁義之道,阿諛時主,高下其心。營構離宮,陷君於奢侈;謀廢冢嫡,致國於傾危。
 終使宗廟丘墟,市朝霜露,究其禍敗之源,實乃素之由也。玄感宰相之子,荷恩二世,君之失德,當竭腹心。未議致身,先圖問鼎,假稱伊、霍之事,將肆莽、卓之心,人神同疾,敗不旋踵。昆弟就菹醢之誅,先人受焚如之酷,不亦甚乎。約外示溫柔,內懷狡算,為蛇畫足,終傾國本,俾無遺育,不亦宜哉。



 寬閑關夷險,竟以功名自卒。文恩能以爵讓,其殆仁乎。



\end{pinyinscope}