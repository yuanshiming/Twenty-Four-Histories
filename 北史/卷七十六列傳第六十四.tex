\article{卷七十六列傳第六十四}

\begin{pinyinscope}

 段文振來護兒樊子蓋周羅周法尚衛玄劉權李景薛世雄段文振,北海期原人也。祖壽,魏滄州刺史。父威,周洮、
 河、
 甘、渭四州刺史。文振少有膂力,膽智過人,明達世務。初為周冢宰宇文護親信,護知其有器局幹用,擢授中外府兵曹。後從周武帝攻齊海昌王尉相貴於晉州,其亞
 將侯子欽、崔景嵩為內應,文振杖槊與崔仲方等數十人先登城。文振隨景嵩至相貴所,拔佩刀劫之,相貴不敢動,城遂下。及攻並州,陷東門而入,齊安德王延宗懼而出降。錄前後勳,將拜柱國,以讒毀獲譴,因授上儀同,賜爵襄國縣公。進平鄴都,又賜綺羅二千段,後從滕王逌擊稽胡,破之。又以天官都上士從韋孝寬經略淮南。俄而尉遲迥作亂,時文振老母妻子俱在鄴城,迥遣人誘之,文振不顧。隋文帝引為丞相掾。



 司馬消難之奔陳,文帝令文振安集淮南,還除衛尉少卿,兼內史侍郎。尋以行軍長史從達奚震討平叛蠻,加上開府,遷鴻臚卿。
 衛王爽北征突厥,以文振為長史,坐勛簿不實免官。後為石、河二州刺史,甚有威惠。遷蘭州總管,改封龍岡縣公。突厥犯塞,以行軍總管擊破之,遂北至居延塞。



 開皇九年,大舉伐陳,為元帥秦王司馬,別領行軍總管,及平江南,授揚州總管司馬,轉并州總管司馬,以母憂去職。後拜雲州總管,遷太僕卿。十九年,突厥犯塞,以行軍總管破達頭可汗於沃野。文振先與王世積有舊。初,文振北征,世積遺以駝。比還,世積以罪誅,文振坐與交關,功遂不錄。後平越巂叛蠻,賜奴婢二百口。仁壽初,嘉州獠反,文振以行軍總管討之。引軍山谷間,為賊所襲,遂
 大敗。



 文振復收散兵,竟破之。文振性素剛直,無所降下。初,軍次益州,謁蜀王秀,貌頗不恭,秀甚銜之。及此,奏文振師徒喪亂。右僕射蘇威與文振有隙,因譖之,坐是除名。及秀廢黜,文振上表自申,帝慰諭之,授大將軍,拜靈州總管。煬帝即位,徵為兵部尚書,待遇甚重。從征吐谷渾,文振督兵屯雪山,連營三百餘里,東接楊義臣,西連張壽,合圍渾主於覆袁川以功進位右光祿大夫。帝幸江都,以文振行江都郡事。



 文振見文帝時容納突厥啟人,居于塞內,妻以公主,賞賜重疊,及大業初,恩澤彌厚,恐為國患。乃上表請以時喻遣,令出塞外,然後明設烽
 候,緣邊鎮防,務令嚴重,此乃萬世之長策。時兵部侍郎斛斯政專掌兵事,文振知政險薄,不可委以機要,屢言於帝。帝並弗納。及遼東之役,授左候衛大將軍。出南蘇道。在軍疾篤,上表以為「遼東小醜,未服嚴刑。但夷狄多詐,深須防擬,口陳降款,心懷背叛,詭伏多端,勿得便受。水潦方降,不可淹遲,唯願嚴勒諸軍,星馳速發,則平壤孤城,勢可拔也。若傾其本根,餘城自剋。如不時定,脫遇秋霖,深為艱弊,兵糧又竭,強敵在前,靺鞨出後,遲疑不決,非上策也」。卒於師。帝省表,悲歎久之,贈光祿大夫、尚書右僕射、北平公,謚曰襄。



 長子詮,位武牙郎將。次子綸,少
 以俠氣聞。



 文振弟文操,大業中,為武賁郎將,性甚剛嚴。帝令督秘書省學士。時學士頗存儒雅,文操輒鞭撻之,前後或至千數,時議者鄙之。



 來護兒,字崇善,本南陽新野人,漢中郎將歙十八世孫也。曾祖成,魏新野縣侯,後歸梁,徙居廣陵,因家焉。位終六合令。祖嶷,步兵校尉、秦郡太守、長寧縣侯。父法敏,仕陳終於海陵令。護兒未識而孤,養於世母吳氏。吳氏提攜鞠養,甚有慈訓,幼而卓犖,初讀《詩》,至「擊鼓其鏜,踴躍用兵」,「羔裘豹飾,孔武有力」。因捨書歎曰:「大丈夫在世當如是,會為國滅賊以取功名,安能區區專事筆硯也!」群
 輩驚其言而壯其志。及長,雄略秀出,志氣英進。涉獵書史,不為章句學。



 始,侯景之亂,護兒世父為鄉人陶武子所害,吳氏每流涕為護兒言之。武子宗族數百家,厚自封植。護兒每思復怨,因其有婚禮,乃結客數人,直入其家,引武子斬之,賓客皆懾不敢動。乃以其頭祭伯父墓,因潛伏歲餘。會周師定淮南,乃歸鄉里。所住白土村,地居疆埸,數見軍旅,護兒常慨然有立功名之志。及開皇初,宇文忻、宇若弼等鎮廣陵,並深相禮重。除大都督,領本鄉兵。破陳將曾永,以功授儀同三司,平陳之役,護兒有功焉,進位上開府,賞物一千段。



 十一年,高智慧據江南
 反,以子總管統兵隋楊素討之。賊據浙江岸為營,周亙百餘里,船艦被江,鼓噪而進。護兒言於素曰:「吳人累銳,利在舟楫。必死之賊,難與爭鋒。公且嚴陣以待之,勿與接刃,請假奇兵數千,潛度江,掩破其壁,使退無所歸,進不得戰,此韓信破趙之策也。」素以為然。護兒乃以輕舸數百,直登江岸,襲破其營,因縱火,煙焰張天。賊顧火而懼,素因是動,一鼓破之。智慧將逃於海,護兒追至閩中,餘黨皆平。進位大將軍。除泉州刺史,封襄陽縣公,食邑一千戶,賜物二千段、奴婢百人。護兒招懷初附,威惠兼舉。璽書勞問,前後相屬。



 時智慧餘黨盛道延阻兵為亂,
 護兒又討平之。遷建州總管。又與蒲山公李寬討平黟、歙逆黨汪文進,進位柱國,封永寧郡公。文帝嘉其功,使畫工圖其像以進。十八年,詔追入朝,賜以宮女、寶刀、駿馬、錦彩等物,仍留長子楷為千牛備身,使護兒還職。



 仁壽初,遷瀛州刺史,以善政聞,頻見勞勉。煬帝嗣位,被追入朝,百姓攀戀,累日不能出境,詣闕上書致請者,前後數百人。帝謂曰:「昔國步未康,卿為名將,今天下無事,又為良二千石,可謂兼美矣。」仍除右驍衛大將軍。尋遷左。又改上柱國為光祿大夫,徙右翊衛大將軍,進封榮國公,恩禮隆密,朝臣無比。大業六年,車駕幸江都,謂護兒
 曰:「衣錦晝遊,古人所重,卿今是也。」乃賜物二千段,并牛酒,令謁先人墓,宴鄉里父老。仍令三品已上並集其宅,酣飲盡日,朝野榮之。



 遼東之役,以護兒為平壤道行軍總管,兼檢校東萊郡太守,率樓船指滄海。入自浿水,去平壤六十里,高麗主高元掃境內兵以拒之,列陣數十里。諸將咸懼,護兒笑謂副將周法尚及軍吏曰:「吾本謂其堅城清野以待王師,今來送死,當殄之而朝食。」高元弟建驍勇絕倫,率敢死數百人來致師。護兒命武賁郎將費青奴及第六子左千牛整馳斬其首,乃從兵追奔,直至城下,俘斬不可勝計,因破其郛,營於城外,以待諸
 軍。高麗晝閉城門,不敢出。會宇文述等眾軍皆敗,乃旋軍。以功賜物五千段,以第五子弘為杜城府鷹揚郎將,以先封襄陽公賜其子整。明年,又出滄海道,師次東萊,會楊玄感反,進攻洛陽,護兒聞之,召裨將周法尚等議旋軍討逆。



 法尚等咸以無敕,不宜擅還,再三固執不從。護兒厲聲曰:「洛陽被圍,心腹之疾。



 高麗逆命,猶疥癬耳。公家之事,知無不為,專擅在吾,當不關諸人也。有沮議者,軍法從事。」即日迴軍。令子弘及整馳驛奏聞。帝見弘等甚悅,曰:「汝父擅赴國難,乃誠臣也。」授弘通議大夫,整公路府鷹揚即將,乃降璽書於護兒曰:「公旋師之時,是
 朕敕公之日,君臣授弘意合,遠同符契。梟此元惡,期在不遙,勒名太常,非公而誰也!」於是護兒與宇文述破玄感於閿鄉,斬平之。還,加開府儀同三司,賜物五千段、黃金千兩、奴婢百人,贈父法敏東陽郡太守、永寧縣公。



 十一年,又率師渡海,破高麗奢卑等二城。高麗舉國來戰,護兒大破之。將趣平壤,高元震懼,使執叛臣斛斯政詣遼東城下請降。帝許之,詔護兒旋軍。護兒集眾軍謂曰:「三度出兵,未能平賊。此還也,不可重來。今高麗困弊,野無青草,以我眾戰,不日剋之。吾欲進兵,徑圍平壤,取其偽主,獻捷而歸也。」於是拜表請行,不肯奉詔。長史崔君肅固
 爭之,以為不可。護兒曰:「賊勢破矣。吾在閫外,事合專決,寧征得高元,還而獲譴,捨此成功,所不能矣。」君肅告眾曰:「若從元帥,違拒詔書,必當奏聞。」諸將懼,乃同勸還師,方始奉詔。及帝於鴈門為突厥所圍,將選精騎潰圍而出,護兒及樊子蓋並固諫,乃止。



 十二年,駕幸江都,護兒諫曰:「自皇家受命,將四十年,薄賦輕徭,戶口滋殖。陛下以高麗逆命,稍興軍旅,百姓無知,易為咨怨,在外群盜,往往聚結,車駕遊幸,深恐非宜。伏願駐駕洛陽,與時休息,出師命將,掃清群醜,上稟聖算,指日剋除。陛下今幸江都,是臣衣錦之地,臣荷恩深重,不敢專為身謀。」帝聞
 之,厲色而起,數日不得見。後怒解,方被引入,謂曰:「公意乃爾,朕復何望!」護兒因不敢言。尋代宇文述為左翊衛大將軍。及宇文化及構逆,深忌之。是日旦將朝,見執。護兒曰:「陛下今何在?」左右曰:「今被執矣。」護兒嘆曰:「吾備位大臣,荷國重任,不能肅清凶逆,遂令王室至此,抱恨泉壤,知復何言!」乃遇害。



 護兒重然諾,敦交契,廉於財利,不事產業。至於行軍用兵,特多謀算,每覽兵法,曰:「此亦豈異人意也!」善撫士卒,部分嚴明,故咸得其死力。



 子十二人,楷通議大夫,弘金紫光祿大夫,整左光祿大夫。整尤驍勇,善撫御,討擊群盜,所向皆捷。諸賊歌曰:「長白山頭
 百戰場,十十五五把長鎗。不畏官軍千萬眾,只怕榮公第六郎。」至是,並遇禍,子姪死者十人,唯少子恆、濟二人免。



 樊子蓋,字華宗,廬江人也。祖道則,梁越州刺史。父儒,侯景之亂奔齊,位仁州刺史。子蓋仕齊,位東海北陳二郡太守、員外散騎常侍,封富陽侯。周武帝平齊,授儀同三司、郢州刺史。隋文帝受禪,以儀同領鄉兵,後除樅陽太守。平陳之役,以功加上開府,改封上蔡縣伯,歷辰、嵩、齊三州刺史,轉循州總管,許以便宜從事。十八年,入朝,奏嶺南地圖,賜以良馬雜物,加統四州,令還任所,遣光祿少
 卿柳謇之餞於灞上。



 煬帝即位,轉涼州刺史,改授銀青光祿大夫、武威太守,以善政聞。大業三年,入朝,加金紫光祿大夫。五年,車駕西巡,將入吐谷渾。子蓋以彼多瘴氣,獻青木香,以禦霧露。及帝還,謂曰:「人道公清,定如此不?「子蓋謝曰:「臣安敢清,止是小心不敢納賄耳。」於是賜之口味百餘斛,加右光祿大夫。子蓋曰:「願奉丹陛。」帝曰:「公侍朕則一人而已,委以西方,則萬人之敵,宜識此心。」六年,帝避暑隴川宮,又云欲幸河西。子蓋傾望鑾輿,願巡郡境。帝知之,下詔慰勉之。



 是歲,朝於江都宮,帝謂曰:「富貴不還故鄉,真衣繡夜行耳。」因敕廬江郡設三千人
 會,賜米麥六千石,使謁墳墓,宴故老,當時榮之。還除戶部尚書。時處羅可汗及高昌王款塞,復以子蓋檢校武威太守,應接二蕃。遼東之役,攝左武衛將軍,出長岑道。後以宿衛不行。加左光祿大夫。其年,帝還東都,使子蓋涿郡留守。



 九年,駕復幸遼東,命子蓋東都留守。屬楊玄感作逆,逼城,子蓋遣河南贊務裴弘策逆擊之,反為所敗,遂斬弘策以徇。國子祭酒楊汪小不恭,子蓋又將斬之。



 汪拜謝,頓首流血,久乃釋免。於是三軍莫不戰慄,將吏無敢仰視。玄感每盡銳攻城,子蓋徐設備禦,至輒摧破。會來護等救至,玄感乃解去。子蓋凡所誅殺萬人。



 又
 檢校河南內史。車駕至高陽,追詣行在所,帝勞之,以比蕭何、寇恂,加光祿大夫,封建安侯,賜女樂五十人。謂曰:「朕遣越王留守東都,示以皇枝盤石,社稷大事,終以委公。特宜持重,戈甲五百人而後出,此勇夫重閉之義。無賴不軌者,便誅鉏之,凡可施行,無勞形迹。今為公別造玉麟符,以代銅獸。」又指越、代二王曰:「今以二孫委公與衛文昇耳。宜選貞良宿德有方幅者教習之。」於是賜以良田、甲第。



 十年,駕還東都,帝謂子蓋曰:「玄感之反,神明故以彰公赤心耳。析珪進爵,宜有令謨。」是日進爵為濟公,言其功濟天下,特為立名,無此郡國也。後與蘇威、宇
 文述陪宴積翠池,帝親以金盃屬子蓋酒,曰:「良算嘉謀,俟公後動,即以此盃賜公,用為永年之瑞。」



 十一年,從駕至鴈門,為突厥所圍。帝欲選精騎潰圍出,子蓋及來護諫,因垂泣:「願暫停遼東之役,以慰眾望。聖躬親出慰撫,厚為勳格,人心自奮,不足為憂。」帝從之,後援兵至,虜乃去。納言蘇威追論勳格太重,宜在斟酌。子蓋執奏不宜失信。帝曰:「公欲收物情邪?」子蓋默然不敢對。



 從駕還東都。時絳郡賊敬槃陀、柴保昌等阻兵數萬,汾、晉苦之,詔子蓋進討。



 時人物殷阜,子蓋善惡無所分別,汾水北村塢盡焚之。百姓大駭,相率為盜。其歸首者,無少長悉坑
 之。擁數萬眾,經年不能破賊,詔徵還,又將兵擊宜陽賊,以疾停,卒於東京。上悲傷者久之,顧黃門侍郎裴矩曰:「子蓋臨終何語?」矩曰:「子蓋病篤,深恨雁門之恥。」帝聞之歎息,令百官就弔,贈開府儀同三司,謚曰景。會葬萬餘人。武威人吏聞其死,莫不嗟痛,立碑頌德。



 子蓋無他權略,在軍持重,未嘗負敗,蒞官明察,下莫敢欺。嚴酷少恩,果於殺戮,臨終之日,見斷頭鬼前後重沓,為之厲云。



 周羅,字公布,九江尋陽人也。父法皓,仕梁,至南康內史、臨蒸縣侯。羅年十五,善騎射,好鷹狗,任俠放蕩,收聚亡命,陰習兵書。從祖景彥誡之曰:「吾世恭謹,汝獨放
 縱,若不喪身,必將滅吾族。」羅終不改。仕陳,為句容令。



 後從大都督吳明徹與齊師戰於江陽,為流矢中左目。齊師之圍明徹於宿預也,諸軍相顧,莫有鬥心。羅躍馬突進,莫不披靡。太僕卿蕭摩訶副之,斬首不可勝計。



 進師徐州,與周將梁士彥戰於彭城,摩訶臨陣墮馬,羅進救之於重圍之內,勇冠三軍。明徹之敗,羅全眾而歸。後以軍功除右軍將軍。封始安縣伯,總檢校揚州中外諸軍事。賜金銀三千兩,盡散之將士,分賞驍雄。陳宣帝深歎美之。出為晉陵太守,進爵為侯。後除使持節、都督豫章十郡諸軍事、豫章內史。獄訟庭決,不關吏手,
 人懷其惠,立碑頌德。



 至德中,除持節、都督南州諸軍事。江州司馬吳世興密奏羅甚得人心,擁眾嶺表,意在難測。陳主惑焉。蕭摩訶、魯廣達等保明之。外有知者,或勸其反,羅拒絕之。還除太子左術率,信任愈重,時參宴席。陳主曰:「周左率武將,詩每前成,文士何為後也?」都官尚書孔範曰:「周羅執筆製詩,還如上馬入陣,不在人後。」自是益見親禮。及隋伐陳,羅都督巴峽緣江諸軍事以拒秦王俊。及陳主被禽,上江猶不下,晉王廣遣陳主手書命之。羅與諸將大臨三日,放兵士散,然後乃降。文帝慰喻之,許以富貴。羅垂泣對曰:「本朝淪亡,
 臣無節可紀。陛下所賜,獲全為幸,富貴榮祿,非臣所望。」帝甚器之。賀若弼謂曰:「聞公郢、漢捉兵,即知揚州可得。王師利涉,果如所量。」羅答曰:「若得與公周旋,勝負未可知也。」其年秋,拜上儀同三司,鼓吹送之于宅。先是,陳裨將羊翔歸降,使為鄉導,位至開府,班在羅上。韓禽於朝堂戲之曰:「不知機變,位在羊翔下。」



 羅答曰:「昔在江南,久承令問,謂公天下節士。今日所言,殊匪人臣之論。」



 禽有愧色。歷幽、涇二州刺史,並有能名。



 開皇十八年,征遼東,徵為水軍總管。自東萊汎海趣平壤城,遭風,船多漂沒,入功而旋。十九年,突厥達頭可汗犯塞,從楊素
 致討,羅先登,大破之。進大將軍。仁壽元年,入為東宮右虞候率,賜爵義寧郡公。轉右衛率。煬帝即位,授右武候大將軍,副楊素計平漢王諒,進授上大將軍。及陳主卒,羅請一臨哭,帝許之。



 衰絰送至墓,葬還,釋服而後入朝。帝甚嘉尚之,世論稱其有禮。時諒餘黨據絳、晉等三州未下,詔羅行晉、絳、呂三州諸軍事,進兵圍之。中流矢,卒。送柩還京,行數里,無故輿馬自止,策之不動,不飄風旋繞焉。絳州長史郭雅稽首祝曰:「公恨小寇未平邪?尋既除殄,無為戀恨。」是時風靜馬行,見者莫不悲歎。其年七月,子仲隱夢羅曰:「我明日當戰。」其靈坐所有
 弓箭刀劍無故自動,若人帶持之狀。絳州城陷,是其日也。贈柱國、右翊衛大將軍,謚曰壯。子仲安,位上開府。



 周法尚,字德邁,汝南安成人也。祖靈起,梁廬、桂二州刺史。父炅,定州刺史、平北將軍。



 法尚少果勁,有風概,好讀兵書。其父卒後,監定州事,督父本兵。數有戰功,為散騎常侍,領齊昌郡事,封山陰縣侯。既而其兄武昌縣公法僧代為定州刺史。法尚與長沙王叔堅不相能,叔堅言其將反。陳宣帝執禁法僧,發兵欲取法尚。其下將吏皆勸之歸北,法尚未決。長史殷文則曰:「樂毅所以辭燕,良不獲已也。」法尚遂歸周,拜開府、順州刺史,封歸義縣公,
 賜良馬五匹、女妓六人、彩物五百段,加以金帶。陳將樊猛濟江討之,法尚遣部曲督朗韓朗詐為背己奔陳,偽告猛曰:「法尚部兵不願降北,若得軍來,必無鬥者。」猛引師急進。法尚設奇兵,大敗之,猛僅以身免。



 隋文帝為丞相,司馬消難作亂,陰遣上開府段珣攻圍之。外無救援,法尚棄城走。消難虜其母弟及家累三百人歸陳。及文帝受禪,拜巴州刺史,破三鵶叛蠻,復從柱國王誼擊走陳寇。遷衡州總管,改封譙郡公。後上幸洛陽,召之,賜金鈿酒鐘一雙、彩五百段、良馬十五匹、奴婢三百口,給鼓吹一部。法尚固辭。上曰:「公有大功於國,特給鼓吹者,欲公
 卿知朕之寵公也。」轉黃州總管,使經略江南。及伐陳之役,以行軍總管隸秦孝王。轉鄂州刺史,遷永州總管,安集嶺南,仍給黃州兵三千五百人為帳內。前後賞賜甚厚。轉柱州總管,仍嶺南道安撫大使。後數年入朝,以本官宿衛。未幾,桂州人李光仕反,令法尚與上柱國王世積討之。法尚發嶺南兵,世積徵嶺北軍。俱會尹州。世積所部多遇瘴,不能進,頓于衡州。法尚獨討之,捕得其弟光略、光度,追斬光仕,平之。仕壽中,遂州獠叛,復以行軍總管討平之。巂州烏蠻反,詔法尚便道討擊破之。軍還,檢校潞州事。



 煬帝嗣位,轉雲州刺史,遷定襄太守,進金
 紫光祿大夫。時帝幸榆林,法尚朝于行宮。內史令元壽言於帝曰:「漢武出塞,旌旗千里。今御營外,請分為二十四軍,日別遣一軍發,相云三十里,旗幟相望,鉦鼓相聞,首尾連注,千里不絕。」



 法尚曰:「兵亙千里,動間山谷,卒有不虞,四分五裂,腹心有事,首尾未知。雖有故事,此取敗道也。」帝不懌曰:「卿以為如何?」法尚曰:「請為方陣,四面外拒,六宮及百官家口並住其間。若有變,當頭分抗,車為壁壘,重設鉤陳,此與據城何異?臣謂牢固萬全策也。」帝曰:「善。」因拜左武衛將軍。明年,黔安夷向思多反,殺將軍鹿愿,圍太守蕭造。法尚與將軍李景分路討之,法尚破
 思多于清江。及還,從討吐谷渾,別出松州道,逐捕亡散,至于青海。出為敦煌太守,遷會寧太守。



 遼東之役,以舟師指朝鮮道。會楊玄感反,與宇文述、來護等破之。以功進授右光祿大夫。時齊郡人王薄、孟讓等為盜,保長白山,法尚頻擊破之。明年,復臨滄海,在軍遇疾卒。贈武衛大將軍,謚曰僖。有子六人,紹範最知名。



 衛玄,字文升,河南洛陽人也。祖悅,魏司農卿。父,侍中、左武衛大將軍。



 玄少有器識,周武帝在籓,引為記室。遷給事上士,襲爵興勢公。武帝親總萬機,拜益州總管長史,賜以萬釘寶帶。稍遷開府儀同三司、太府中大夫,攝
 內史事,仍領京兆尹,稱為強濟。隋文帝作相,檢校熊州事。及受禪,遷淮州總管,進封同軌郡公,坐事免。未幾,拜嵐州刺史。會起長城之役,詔玄監督之。後為衛尉少卿。



 仁壽初,山獠作逆,以玄為資州刺史,以鎮撫之。玄既到官,時獠攻圍大牢鎮,玄單騎造其營,謂群獠曰:「我是刺史,銜天子詔安養汝等,汝等勿驚」諸賊莫敢動。



 於是說以利害,渠帥感悅,解兵歸附者十餘萬口。文帝大悅,賜縑二千匹,除遂州總管,仍令劍南安撫。



 煬帝即位,復徵為衛尉卿,夷獠攀戀,數百里不絕。及與之決,並揮涕而去。



 遷工部尚書。後拜魏郡太守,尚書如故。未幾,拜右
 衛大將軍,檢校左候衛事。轉刑部尚書。遼東之役,檢校右禦衛大將軍,帥師出增地道。時諸軍多不利,玄獨全眾而還。拜金紫光祿大夫。



 九年,駕幸遼東,使玄與代王侑留守京師,拜為京兆內史,尚書如故,許以便宜從事,敕代王待以師傅禮。會楊玄感圍東都,玄率步騎七萬援之。至華陰,掘楊素塚,焚其骸骨,夷其塋域,示士卒以必死。既出潼關,議者恐崤函有伏兵,請於陜縣沿流東下,直趨河陽,以攻其背。玄曰:「此計非豎子所及也。」乃鼓行而進。



 既度函谷,卒如所量。乃遣武賁郎將張峻為疑軍於南道,玄以大兵直趨城北。玄感逆拒之,且戰且行,
 屯軍金谷。於軍中掃地而祭文帝曰:「若社稷靈長,宜令醜徒冰碎;如或大運去矣,幸使老臣先死。」詞氣激揚,三軍莫不涕咽。時眾寡不敵,民賊頻戰不利,死傷太半。玄苦戰,賊稍卻,進屯北芒。會宇文述、來護等援兵至,玄感西遁。玄遣通議大夫斛斯萬善、監門直閣龐玉前鋒追之,及于閿鄉,與宇文述等合擊破之。車駕至高陽,徵詣行在所。帝勞之曰:「社稷臣也。使朕無西顧之憂。」



 進右光祿大夫,賜以良田、甲第,資物鉅萬,還鎮京師,帝謂曰:「關右之任,一委於公。公安,社稷乃安;公危,社稷亦危。出入須有兵衛,坐臥恒宜自牢也。今特給千兵,以充侍從。」
 與樊子蓋俱賜以玉麟符,以代銅獸。



 十一年,詔玄撫關中。時盜賊蜂起,百姓饑饉,玄竟不能救恤。而官方壞亂,貨賄公行。自以年老,上表乞骸骨,帝遣內史舍人封德彞馳喻之曰:「京師國本,宗廟園陵所在,籍公臥以鎮之。」玄乃止。義師入關,自知不能守,尤懼稱疾,不知政事。城陷,歸於家。義寧中,卒。



 子孝則,位通事舍人、兵部承務郎。卒。



 劉權,字世略,彭城豐人也。祖軌,齊羅州刺史。權少有俠氣,重然諾,藏亡匿死,吏不敢過門,。後更折節好學,動循法度。仕齊,位行臺郎中。齊亡,周武帝以為假淮州刺史。
 開皇中,以車騎將軍領鄉兵。後從晉王廣平陳,進授開府儀同三司。宋國公賀若弼甚禮之。十二年,拜蘇州刺史,賜爵宋城縣公。時江南初平,權撫以恩信,甚得人和。煬帝嗣位,拜衛尉卿,進位銀青光祿大夫。大業五年,從征吐谷渾,權出伊吾道,逐賊至青海,乘勝至伏俟城。帝復令權過曼頭、赤水,置河源郡、積石鎮,大開屯田,留鎮西境。在邊五年,諸羌懷附,貢賦歲入,吐谷渾餘燼遠遁,道路無壅。徵拜司農卿,加金紫光祿大夫。尋為南海太守。行至鄱陽,會群盜起,不得進,詔權召募討之。權率兵遇賊,不戰,先乘單舸詣賊營,說以利害。群賊感悅,一時
 降附。帝聞而嘉之。及至南海,甚有異政。數歲,遇盜賊群起,群豪多願推權為首,權竟固守以拒之。子世徹又密遣齎人書詣權,稱四方擾亂,諷令舉兵。權召集佐僚,對斬其使,竟無異圖,守之以死。卒官。



 世徹倜儻不羈,頗為時人所許。大業末,群雄並起,世徹所至處輒見忌,多拘禁之。後竟為兗州賊帥徐圓朗所殺。



 權從叔烈,字子將,美容儀,有器局,位鷹揚郎將。有子德威,知名於世。



 李景,字道興,天水休官人也。父超,周應、戎二州刺史。景容貌奇偉,膂力過人,美鬚髯,驍勇善射。平齊之役,頗有功,授儀同三司。後以平尉遲迥,進位開府,賜爵平寇縣
 公。隋開皇九年,以行軍總管從世積伐陳,以功進上開府。及高智慧等反,復以行軍總管從楊素擊之,還授鄜州刺史。



 十七年,遼東之役,為馬軍總管。及還,配事漢王。文帝奇其壯武,使袒而觀之,曰:「卿相表當位極人臣。」尋從史萬歲擊突厥於大斤山,別路邀賊,大破之。



 後與上明公楊紀送義城公主於突厥,至恆安,遇突厥來寇。時代州總管韓洪為虜所敗,景率所領數百人力戰三日,殺虜甚眾。改授韓州刺史。以事王故,不之官。



 仁壽中,檢校代州總管。漢王諒作亂,景發兵拒之。諒頻遣劉嵩、喬鐘葵等攻之,景率士卒殊死戰,屢挫賊鋒。司馬馮孝
 慈、司法參軍呂玉並驍勇善戰,儀同三司侯莫陳乂多謀畫,工拒守之術。景推誠此三人,無所關預,唯在閤持重,時出撫循而已。及朔州總管楊義臣援兵至,合擊大破之。先是,府內井中甃上生花如蓮,并有龍見,時變為鐵馬甲士。又有神人長數丈見城下,跡長四尺五寸。景問巫者,巫者曰:「此不祥之物,來食血耳。」景大怒,推出之。旬日而兵至,死者數萬。



 景尋被徵,進柱國,拜右武衛大將軍。賜女樂一部,加以珍物。



 景智略非所長,而忠直為時所許,帝甚信之。又擊破叛蠻向思多。明年,擊吐谷渾於青海,破之,進位光祿大夫。五年,車駕西巡,至天水,景
 獻食於帝。帝曰:「公,主人也。」賜坐齊王暕上。至隴川宮,帝將大獵,景與左武衛大將軍郭衍俱有難色,為人奏。帝大怒,令Ξ之,竟以坐免。歲餘,復位,與宇文述等參掌選舉。



 明年,攻高麗武列城,破之,賜爵苑丘侯。八年,出渾彌道。九年,復出遼東。及旋,使景殿,高麗追兵大至,景擊走之。進爵滑國公。楊玄感之反,朝臣子弟多預焉,景獨無關涉。帝曰:「公誠直天然,我梁棟也。」賜以美女。帝每呼李大將軍而不名,見重如此。



 十二年,帝令景營遼東戰具於北平,賜御馬一匹,名師子吉。于時盜賊蜂起,景遂召募,以備不虞。武賁郎將羅藝與景有隙,誣景將反。帝遣
 其子慰諭曰:「縱人言公窺天闕,據京都,吾無疑也。」後為高開道所圍,獨守孤城,士卒患腳腫死者十六七,景撫循之,一無離叛。遼東軍資多在其所,粟帛山積,景無所私焉。及帝崩於江都,遼西太守鄧皓救之,遂歸柳城。將還幽州,遇賊見害。契丹、靺鞨素感其恩,聞之莫不流涕;幽、燕人士,于今傷惜之。子世謨。



 薛世雄,字世英,本河東汾陰人也。其先寓居敦煌。父回,字道弘,仕周,位涇州刺史。開皇初,封舞陰郡公,領漕渠監。世雄兒童時與群輩戲,輒畫地為城郭,令諸兒為攻守勢,不從令者輒撻之,諸兒畏憚,莫不齊整。其父見而
 奇之,謂人曰:「此兒當興吾家。」年十七,從周武帝平齊,以功拜帥都督。隋開皇中,累遷右親衛車騎將軍。



 煬帝嗣位,為右監門郎將。從征吐谷渾,進位通議大夫。世雄性廉慎,行軍破敵之處,秋毫無犯。帝由是嘉之。帝嘗謂群臣曰:「欲舉好人,諸君識否?」咸曰:「不測聖心。」帝曰:「我欲舉薛世雄。」群臣皆稱善。於是超拜右翊衛將軍。歲餘,為玉門道行軍大將軍,與突厥啟人可汗邊兵擊伊吾。師次玉門,啟人背約,兵不至。世雄孤軍度磧。伊吾懼,請降。世雄遂於漢舊伊吾城東築城,號新伊吾,留銀青光祿大夫王威鎮之而還。進位正議大夫。



 遼東之役,為沃沮道
 軍將,與宇文述同敗績於平壤。還次白石山,為賊所圍百餘重,四面矢下如雨。世雄以羸師為方陣,選勁騎二百縱擊,破之而還。所亡失多,竟坐免。明年,帝復征遼東,拜右候衛將軍。兵指蹋頓道。軍至烏骨城,會楊玄感反,班師。帝至柳城,以世雄為東北道大使,行燕郡太守,鎮懷遠。



 十年,復從帝至遼東,遷左禦衛大將軍。仍領涿郡留守。未幾,李密逼東都,詔世雄率幽、薊精兵將擊之。次河間,營於城南,竇建德率精銳數百,夜來襲之。



 大敗。世雄與左右數十騎遁入河間城,慚恚發病。歸涿郡,卒。



 子萬述、萬淑、萬鈞、萬徹、萬備,並以驍武知名。



 論曰:段文振有周之日,早以武毅見知,隋氏之初,又以幹力受委,任兼文武,稱為諒直。其高位厚秩,非虛致也。來護幼懷倜儻,猛概抑揚,晚致勤王,驅馳畢力。樓船制勝,掃勍敵如拾遺;閿鄉討亂,翦兇魁如摧朽。位班上將,顯居大國,道消遘難,忠至不渝,惜矣!子蓋雅有幹局,質性方嚴,見義而勇,臨機能斷,保全邦邑,勤亦懋哉!羅忠亮之性,所在稱重,送往之節,義感人臣,死而有知,乃結草之義。法尚征伐四夷,亦足嘉焉。文昇東都解圍,頗亦宣力,西京居守,政以賄成,鄙哉,鄙哉,夫何足數!劉權淮楚舊族,雄名早著,時逢擾攘,任等尉佗,遂能拒子邪
 言,足驗誠臣之節。李、薛並以驍武之用,當於有事之秋,致茲富貴,可謂自取。時迍遭躓,良有命乎!



\end{pinyinscope}