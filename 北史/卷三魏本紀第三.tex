\article{卷三魏本紀第三}

\begin{pinyinscope}

 高祖孝文皇帝諱宏,獻文皇帝之太子也。母曰李夫人。皇興元年八月戊申,生於平城紫宮,神光照室,天地氛氳,和氣充塞。帝潔白有異姿,襁褓岐嶷,長而弘裕仁孝,綽然有人君之表。獻文尤愛異之。三年六月辛未,立為皇太子。五年,受禪。延興
 元年
 秋八月丙午,皇帝即位於太華前殿,改皇興五年為延興。丁未,宋人來聘。九月壬戌,詔在位及人庶進直言。壬午,青州高陽人封辨聚黨
 自號齊王,州軍討平之。冬十月丁亥,沃野、統萬二鎮敕勒叛,詔太尉、隴西王源賀追擊至枹罕,滅之。徙其遺迸於冀、定、相三州為戶。十二月乙酉,封駙馬都尉穆亮為趙郡王。



 壬辰,詔求舜後,獲東萊人媯茍之。復其家畢世,以彰盛德之不朽。復前濮陽王孔雀本封。辛丑,徙趙郡王穆亮為長樂王。癸卯,日有蝕之。



 二年春正月,大陽蠻酋桓誕率戶內屬,拜征南將軍,封襄陽王。曲赦京師及河西,南至秦、涇,西至枹罕,北至涼州及諸鎮。詔假員外散騎常侍邢祐使於宋。二月丁巳,詔曰:「頃者,淮徐未賓,尼父廟隔非所,致令祠典寢頓,禮
 章殄滅,遂使女巫妖覡淫進非禮。自今有祭孔廟,制用酒脯而已,不聽婦女雜合,以祈非望之福。犯者以違制論。其公家有事,自如常禮。」蠕蠕犯塞,太上皇帝次於北郊,詔諸將討之,悉皆遁走。北部敕勒叛,奔蠕蠕。太上皇帝追至石磧,不及而還。三月戊辰,以散騎常侍、駙馬都尉萬安國為大司馬、大將軍,封安城王。庚午,親耕籍田。連川敕勒謀叛,徙配青、徐、齊、兗四州為營戶。夏四月庚子,詔工商雜伎,盡聽赴農。諸州課人益種菜果。辛亥,宋人來聘。癸酉,詔沙門不得去寺,行者以公文。是月,宋明帝殂。五月丁巳,詔軍警給璽印傳符,次給馬印。六月,安州遭
 水雹,詔丐租振恤。丙申,詔:「今年貢舉,尤為猥濫。自今所遣,皆可門盡州郡之高,才極鄉閭之選。」戊午,行幸陰山。秋七月壬寅,詔州郡縣各遣二人才堪專對者,赴九月講武,當親問風俗。八月,百濟遣使請兵伐高麗。九月辛巳,車駕還宮。戊申,統萬鎮將、河間王閭武皮坐貪殘賜死。己酉,詔以州鎮十一水旱,丐其田租,開倉振恤。又詔流迸之人,皆令還本,違者徙邊。冬十月,蠕蠕犯塞,及五原。十一月,太上皇帝親討之,將度漠。蠕蠕聞之,北走數千里。丁亥,封皇叔略為廣川王。壬辰,分遣使者巡省風俗,問人疾苦。帝每月一朝崇光宮。十二月庚戌,詔曰:「頃
 者以來,官以勞升,未久而代。牧守無恤人之心,競為聚斂,送故迎新,相屬於路,非所以固人志、隆政道也。自今牧守溫良仁儉、克己奉公者,可久於其任。歲積有成,遷位一級;其貪殘非道,侵削黎庶者,雖在官甫爾,必加黜罰。著之於令,以為彞準。」詔以代郡事同豐沛,代人先配邊戍者免之。是歲,高麗、地豆干、庫莫奚、高昌等國並遣使朝貢。



 三年春正月庚辰,詔員外散騎常侍崔演使於宋。丁亥,改崇光宮為寧光宮。二月戊午,太上皇帝至自北討,飲至策勛,告于宗廟。甲戌,詔縣令能靜一縣劫盜者,兼理
 二縣,即食其祿;能靜三縣者,三年遷為郡守。二千石能靜二郡上至三郡,亦如之,三年遷為刺史。三月壬午,詔諸倉屯穀麥充積者,出賜貧人。夏四月戊申,詔假司空、上黨王長孫觀等討吐谷渾拾寅。壬子,詔以孔子二十八世孫魯郡孔乘為崇聖大夫,給十戶以供灑掃。六月甲子,詔曰:「往年縣召秀才二人,問守宰善惡,而賞者未幾,罪者眾多,肆法傷生,情所未忍。諸為人所列者,特原其罪,盡可代之。」秋七月,詔河南六州人,戶收絹一匹、綿一斤、租三十石。丁亥,行幸陰山。



 八月庚申,帝從太上皇帝幸河西。拾寅謝罪請降,許之。九月辛巳,車駕還宮。丁
 亥,宋人來聘。己亥,詔曰:「今京師及天下囚未判,在獄致死,無近親者,給衣衾棺櫝葬之,不得暴露。」辛丑,詔遣十使,循行州郡,撿括戶口。冬十月,太上皇帝將南討,詔州郡之人,十丁取一,充行;戶租五十石,以備軍糧。十一月戊寅,詔以河南州郡牧守多不奉法,致新邦百姓莫能上達。遣使者觀風察獄,黜陟幽明,搜揚振恤。癸巳,太上皇帝南巡至懷州,所過問人疾苦,賜高年孝悌力田布帛。十二月癸卯朔,日有蝕之。庚戌,詔關外苑囿,聽人樵採。是歲,高麗、契丹、庫莫奚、悉萬斤等國並遣使朝貢。州鎮十一水旱,丐人田租,開倉振恤。相州人餓死者二千
 八百四十五人。妖人劉舉自稱天子,齊州刺史、武昌王平原捕斬之。



 四年春正月癸酉朔,日有蝕之。丁丑,太尉、隴西王源賀以病辭位。二月甲辰,太上皇帝至自南巡。辛未,禁寒食。三月丁亥,詔員外散騎常侍許赤武使於宋。夏四月丁卯詔:自今非謀反大逆,干紀外奔,罪止其身而已。秋七月己卯,曲赦仇池。



 八月戊申,大閱於北郊。九月,以宋亂故,詔將軍元蘭等伐蜀漢。冬十月庚子,宋人來聘。十一月,分遣侍臣循河南七州,觀察風俗,撫慰初附。是歲,粟特、敕勒、吐谷渾、高麗、曹利、闊悉、契丹、庫莫奚、地豆干等
 國並遣使朝貢。州鎮十三大饑,丐人田租,開倉振之。



 五年春二月癸丑,詔定考課,明黜陟。夏四月,詔禁畜鷹鷂,開相告之制。五月丙午,詔員外散騎常侍許赤武使於宋。丁未,幸武州山。辛酉,幸車輪山。六月庚午,禁殺牛馬。壬申,曲赦京師死罪,遣備蠕蠕。秋九月癸卯,洛州人賈伯奴稱恒農王,豫州人田智度稱上洛王,郡討平之。冬十月,太上皇帝大閱於北郊。十二月丙寅,改封建昌王長樂為安樂王。己丑,城陽王長壽薨。庚寅,宋人來聘。是歲,高麗、吐谷渾、龜茲、契丹、庫莫奚、地豆干、蠕蠕等國並遣使朝貢。



 承明元年春二月,司空、東郡王陸定國坐事免官爵為兵。夏五月,冀州人宋伏龍聚眾自稱南平王。郡縣捕斬之。六月甲子,詔中外戒嚴。分京師見兵為三等,第一軍出,遣第一兵,二等亦如之。辛未,太上皇帝崩。壬申,大赦,改元。大司馬、大將軍、安城王萬安國坐法賜死。戊寅,以征西大將軍、安樂王長樂為太尉;尚書左僕射、南平公目辰為司徒,進封宜都王;以南部尚書李為司空。尊皇太后為太皇太后,臨朝稱制。秋七月甲辰,追尊皇妣李貴人為思皇后。濮陽王孔雀有罪賜死。



 八月甲子,詔群公卿士,有便人益國者,具狀以聞。甲戌,以長安二蠶
 多死,丐人歲賦之半。九月丁亥,曲赦京師。冬十月丁巳,起七寶永安行殿。乙丑,進假東陽王丕爵為王。己未,詔群官卿士下及吏人,各聽上書,直言極諫,勿有所隱。諸有益政利人可以正風俗者,有司以聞。辛未,幸建明佛寺,大宥罪人。進濟南公羅拔為王。是歲,蠕蠕、高麗、庫莫奚、波斯、契丹、宕昌、悉萬斤等國並遣使朝貢。



 太和元年春正月乙酉,改元。辛亥,起太和、安昌二殿。己酉,秦州略陽人王元壽聚眾,自號衝天王。雲中饑,開倉振恤。二月辛未,秦益二州刺史、武都公尉洛侯討破王元壽。三月庚子,以雍州刺史、東陽王歪為司徒。丙午,詔
 曰:「去年牛疫,死傷大半。今東作既興,人須肄業,其敕在所督課田農,有牛者加勤於常歲,無牛者倍庸於餘年。一夫制田四十畝,中男二十畝,無令人有餘力,地有遺利。」



 夏四月,樂安王良薨。詔復前東郡王陸定國官爵。五月,車駕祈雨於武州山,俄而澍雨大洽。秋七月壬辰,京兆王子推薨。庚子,定三等死刑。己酉,起朱明、思賢門。是月,宋人殺其主昱。八月壬子,大赦。丙子,詔曰:「工商皂隸,各有厥分,而有司縱濫,或染清流。自今戶內有工役者,唯止本部丞,已下準次而授。若階藉元勛以勞定國者,不從此制。」戊寅,宋人來聘。九月乙酉,詔群臣定律令於
 太華殿。庚子,起永樂遊觀殿於北苑,穿神泉池。冬十月辛亥朔,日有蝕之。癸酉,宴京邑耆老年七十已上於太華殿,賜以衣服。詔七十已上一子不從役。宋葭蘆戍主楊文度遣弟鼠襲陷仇池。十一月丁亥,懷州人伊祁茍自稱堯後,應王,聚眾於重山。



 洛州刺史馮熙討平之。閏月庚午,詔員外散騎常侍李長仁使於宋。十二月壬寅,征西將軍皮喜攻陷葭蘆,斬楊文度,傳首京師。丁未,州郡八水旱蝗,人飢,詔開倉振恤。是歲,高麗、契丹、庫莫奚、蠕蠕、車多羅、西天竺、舍衛、疊伏羅、栗楊婆、員闊等國並遣使朝貢。



 二年春正月丁巳,封昌黎王馮熙第二子始興為北平王。二月丁亥,行幸代之湯泉,所過問人疾苦,以宮女賜貧人無妻者。癸卯,車駕還宮。乙酉晦,日有蝕之。



 三月丙子,以河南公梁彌機為宕昌王。夏四月己丑,宋人來聘。京師旱。甲辰,祈天災於北苑,親自禮焉,減膳避正殿。丙午,澍雨大洽,曲赦京師。五月,詔曰:「乃者人漸奢尚,婚葬越軌。又皇族貴戚及士庶之家,不惟氏族高下,與非類婚偶。



 先帝親發明詔,為之科禁。而百姓習常,仍不肅改。朕念憲章舊典,永為定準,犯者以違制論。」六月庚子,皇叔若薨。秋八月,分遣使者,考察守宰,問人疾苦。



 丙戌,詔
 罷諸州禽獸之貢。九月己巳朔,日有蝕之。丙辰,曲赦京師。冬十月壬辰,詔員外散騎常侍鄭發使於宋。十二月癸巳,誅南郡王李惠。是歲,龜茲國獻名駝龍馬珍寶甚眾。吐谷渾、蠕蠕、勿吉等國並遣使朝貢。州鎮二十餘水旱,人飢,詔開倉振恤。



 三年春正月癸丑,坤德六合殿成。庚申,詔罷行察官。二月辛巳,帝、太皇太后幸代郡湯泉,問人疾苦。鰥寡貧者妻以宮女。己亥,還宮。三月癸卯朔,日有蝕之。甲辰,曲赦京師。夏四月壬申,宋人來聘。癸未,樂良王樂平薨。甲午,宋順帝禪位于齊。庚子,進淮陽公尉元爵為王。宜都王
 目辰有罪賜死。五月丁巳,帝祈雨於北苑,閉陽門,是日澍雨大洽。六月辛未,以雍州人飢,開倉振恤。起文石室靈泉殿於方山。秋七月壬寅,詔免宮人年老及病者。八月壬申,詔群臣進直言。乙亥,幸方山,起思遠佛寺。丁丑,還宮。九月壬子,以司徒、東陽王丕為太尉;趙郡公陳建為司徒,進爵魏郡王;河南公茍頹為司空,進爵河東王。進太原公王睿中山王,隴東公張祐新平王。乙未,定州刺史、安樂王長樂有罪賜死。庚申,隴西王源賀薨。冬十月己巳朔,大赦。十一月癸卯,賜京師貧窮高年疾患不能自存衣服布帛各有差。癸丑,進假梁郡公元嘉爵為
 假王,督二將出淮陰;隴西公元琛三將出廣陵;河東公薛豹子三將出廣固,至壽春。是歲,吐谷渾、高麗、蠕蠕、地豆干、契丹、庫莫奚、龜茲、粟特、州逸、河龔、疊伏羅、員闊、悉萬斤等國各遣使朝貢。



 四年春正月癸卯,乾象六合殿成。乙卯,廣川王略薨。丁巳,罷畜鷹鷂之所,以其地為報德佛寺。戊午,襄城王韓頹有罪,削爵徙邊。二月癸巳,以旱故,詔天下祀山川群神及能興雲雨者,修飾祠堂,薦以牲璧。人有疾苦,所在存問。夏四月乙卯,幸廷尉、籍坊二獄,引見諸囚。詔隨輕重決遣,以赴耕耘。甲申,賜天下貧人一戶之內無雜財
 穀帛者廩一年。六月丁卯,以澍雨大洽,曲赦京師。秋七月辛亥,行幸火山。壬子,詔會京師耆老,賜錦彩衣服几杖稻米蜜面,復家人不徭役。閏月丁亥,幸獸圈,親錄囚徒,輕者皆免之。壬辰,頓丘王李鐘葵有罪賜死。八月乙酉,詔諸州置冰室。九月乙亥,思義殿成。壬午,東明觀成。戊子,詔曰:「隆寒雪降,可遣侍臣詣廷尉獄及囚所,察飢寒者給以衣食,桎梏者代以輕鎖。」是歲,郡鎮十八水旱,人飢,詔開倉振恤。蠕蠕、悉萬斤等國並遣使朝貢。



 五年春正月乙卯,南巡。丁亥,至中山,親見高年,問人疾苦。二月辛卯朔,大赦。賜孝悌力田孤貧不能自存者,穀
 帛各有差。免宮人之老者,還其親。丁酉,至信都,存問如中山。癸卯,還中山。己酉,講武於唐水之陽。庚戌,車駕還宮。



 沙門法秀謀反,伏誅。假梁郡王嘉大破齊,俘獲三萬餘口,送京師。三月辛酉朔,幸肆州。癸亥,講武于雲水之陽。所經考察守宰,黜陟之。己巳,車駕還宮。詔曰:「法秀妖詐亂常,妄說符瑞。蘭臺御史張求等一百餘人招結奴隸,謀為大逆。有司科以族誅,誠合刑憲。但矜愚重命,猶所不忍。其五族者降止同祖,三族止一門,門誅止身。」夏四月己亥,行幸方山。建永固石室,於山立碑焉。銘太皇太后終制於金冊。又起鑒玄殿。甲寅,以旱故,詔所在掩
 骸骨,祈衣壽神祇。任城王云薨。



 五月庚申,以農月時耍,詔天下勿使有留獄。六月甲辰,中山王睿薨。戊午,封皇叔簡為齊郡王,猛為安豐王。秋七月庚申朔,日有蝕之。甲子,齊人來聘。九月庚午,閱武於南郊,大饗群臣。齊使車僧朗以班在宋使殷靈誕後,辭不就席。宋降人解奉君刃僧朗於會中。詔誅奉君等。乙亥,封昌黎王馮熙世子誕為南平王。冬十二月癸巳,州鎮十二饑,詔開倉振恤。是歲,鄧至、蠕蠕等國並使朝貢。



 六年春正月甲戌,大赦。二月辛卯,詔以靈丘郡土既褊脊,又諸州路衝,復其人租十五年。癸巳,白蘭王吐谷渾
 翼世以誣罔伏誅。乙未,詔曰:「蕭道成逆亂江淮,戎旗頻舉。七州之人既有徵運之勞,深乖輕徭之義,其復常調三年。」癸丑,賜王公已下清勤著稱者,穀帛有差。三月庚辰,幸獸圈。詔曰:「武狼猛暴,食肉殘生,從今勿復捕貢。」辛巳,幸武州山石窟寺,賜貧老衣服。是月,齊高祖殂。



 夏四月甲辰,賜畿內鰥寡孤獨不能自存者,粟帛各有差秋七月,發州郡五萬人修靈丘道。八月癸未朔,分遣大使巡行天下遭水之處,丐其租賦,貧儉不自存者,賜以粟帛。庚子,罷山澤禁。九月辛酉,以氐楊後起為武都王。是歲,地豆干、吐谷渾等國並遣使朝貢。



 七年春正月庚申,詔曰:「朕每思知百姓疾苦以增修寬政,故具問守宰苛虐之狀於州郡使者。今秀孝計掾對多不實,甚乖朕虛求之意。宜案以大辟,明罔上必誅。



 然情猶未忍。可恕罪聽歸,申下天下,使知後犯無恕。」丁卯,詔青、齊、光、東徐四州戶,運倉粟一十萬石送瑕丘、琅邪,復租算一年。三月甲戌,以冀、定二州饑,詔郡縣為粥於路以食之,又弛關津之禁。夏四月庚子,幸崞山,賜所過鰥寡不能自存者衣服粟帛。壬寅,車駕還宮。閏月癸丑,皇子生,大赦。六月,定州上言,為粥所活九十四萬七千餘口。秋七月甲申,詔假員外散騎常侍李彪使於齊。改
 封濟南王羅拔為趙郡王。九月壬寅,詔求讜言。冀州上言,為粥所活七十五萬一千七百餘口。冬十月戊午,皇信堂成。十一月辛丑,齊人來聘。十二月乙巳朔,日有蝕之。



 癸丑,詔曰:「夏、殷不嫌一族之婚,周世始絕同姓之娶。斯皆教隨時設,政因事改者也。皇運初基,日不暇給,古風遺樸,未遑釐改。自今悉禁絕之,有犯者以不道論。」庚午,開林慮山禁,與人共之。州鎮十三饑,詔開倉振恤。



 八年春正月,詔隴西公琛、尚書陸睿為東西二道大使,褒善罰惡。夏五月己卯,詔振賜河南七州戍兵。甲申,詔員外散騎常侍李彪使於齊。六月丁卯,詔曰:「置官班祿,
 行之尚矣。自中原喪亂,茲制中絕。先朝因循,未遑釐改。朕顧憲章舊典,始班俸祿,罷諸商人,以簡人事。戶增調三匹、穀二斛九斗,以為官司之祿。均預調為二匹之賦,即兼商用。雖有一時之煩,終克永逸之益。祿行之後,贓滿一匹者死。變法改度,宜為更始,其大赦天下,與之惟新。」戊辰,武州水壞人居。秋八月甲辰,詔以班制俸祿,更興刑書,寬猛未允,人或異議。制百辟卿士工商吏人各上便宜,勿有所隱。九月甲午,齊人來聘。戊戌,詔俸制十月為首,每季一請。於是內外百官,受祿有差。冬十一月乙未,詔員外散騎常侍李彪使于齊。十二月,州鎮十五
 水旱,人飢,詔使者開倉振恤。是歲,蠕蠕、高麗等國各遣使朝貢。



 九年春正月戊寅,詔禁圖讖秘緯及名《孔子閉房記》,留者以大辟論。又諸巫覡假稱神鬼,妄說吉凶,及委巷諸非墳典所裁者,嚴加禁斷。癸未,大饗群臣于太華殿,班賜皇誥。二月己亥,制皇子封王者、皇孫皇曾孫紹封者、皇女封者,歲祿各有差。封廣陽王建第二子嘉為廣陽王。乙巳,詔百辟卿士工商吏人各上書極諫,靡有所隱。三月丙申,封皇弟禧為咸陽王,乾為河南王,羽為廣陵王,雍為潁川王,勰為始平王,詳為北海王。夏五月,齊人
 來聘。秋七月丙午朔,新作諸門。癸未,遣使拜宕昌王梁彌機兄子彌承為宕昌王。八月庚申,詔曰:「數州災水,饑饉薦臻,致有賣鬻男女者。天譴在予一人,百姓橫罹艱毒。今自太和六年已來,買定、冀、幽、相四州饑人良口者,盡還所親。雖娉為妻妾,遇之非理,情不樂者,亦離之。」



 冬十月丁未,詔使者循行州郡,與牧守均給天下之田,還受以生死為斷。勸課農桑,興富人之本。辛酉,司徒、魏郡王陳建薨。詔員外散騎常侍李彪使於齊。十二月乙卯,以侍中、淮南王他為司徒。是歲,京師及州鎮十三水旱傷稼。宕昌、高麗、吐谷渾等國並遣使朝貢。



 十年春正月癸亥朔,帝始服袞冕,朝饗萬國。二月甲戌,初立黨、里、鄰三長,定人戶籍。三月庚戌,齊人來聘。夏四月辛酉朔,始制五等公服。甲子,帝初法服御輦祀西郊。六月乙卯,名皇字曰恂,大赦。秋八月乙亥,給尚書五等品爵已上朱衣玉佩大小組綬。九月辛卯,詔起明堂辟雍。冬十月癸酉,有司議依故事配始祖於南郊。十一月,議定州郡縣官依口給俸。十二月乙酉,汝南、潁川饑,詔丐人田租,開倉振恤。是歲,蠕蠕、高麗、吐谷渾、勿吉等國並遣使朝貢。



 十一年春正月丁亥朔,詔定樂章,非雅者除之。二月甲
 子,肆州之鴈門及代郡人飢,詔開倉振恤。夏五月癸巳,南平王渾薨。甲午,詔復七廟子孫及外戚緦服已上,賦役無所與。六月辛巳,秦州人饑,詔開倉振恤。秋七月己丑,詔今年穀不登,聽人出關就食。遣使者造籍,分遣去留,所在開倉振恤。八月壬申,蠕蠕犯塞,遣平原王陸睿討之。庚辰,大議北伐。辛巳,罷山北苑,以其地賜貧人。冬十月辛未,詔罷起部無益之作,出宮人不執機杼者。甲戌,詔曰:「鄉飲之禮廢,則長幼之序亂。孟冬十月,人閑歲隙,宜於此時,導以德義。可下諸州,黨、里之內,推賢而長者,教其里人父慈、子孝、兄友、弟順、夫和、妻柔。不率長教
 者,具以名聞。」



 十一月丁未,詔罷尚方錦繡綾羅之工,百姓欲造,任之無禁。其御府衣服金銀珠玉綾紬錦、太官雜器、太僕乘具、內庫弓矢,出其大半,班齎百官及京師人庶,下至工商皁隸,逮於六鎮戍士,各有差。戊申,詔今寒氣勁切,杖棰難任。自今月至來年孟夏,不聽栲問罪人。又歲饑,輕囚宜速決了,無令薄罪久留獄犴。十二月,詔秘書丞李彪、著作郎崔光改析國記,依紀傳體。是歲大饑,詔所在開倉振恤。吐谷渾、高麗、悉萬斤等國並遣使朝貢。



 十二年春正月辛巳朔,初建五牛旌旗。乙未,詔鎮戍流
 徙之人,年滿七十,孤單窮獨,無成人子孫,旁無期親者,具狀以聞。二月辛亥朔,日有蝕之。三月丁亥,中散梁眾保等謀反,伏誅。夏四月甲子,大赦。己巳,齊將陳顯達攻陷灃陽,長樂王穆亮率騎討之。五月丁酉,詔六鎮、雲中、河西及關內郡,各修水田,通渠溉灌。



 壬寅,增置彝器於太廟。秋九月甲午,詔曰:「日蝕修德,月蝕修刑。乃者癸巳夜,月蝕盡,公卿已下,宜慎刑罰,以答天意。」丁酉,起宣文堂、經武殿。癸卯,淮南王他薨。冬閏十月甲子,帝觀築圓丘于南郊。十一月,雍、豫二州人飢,詔開倉振恤。梁州刺史、臨淮王提坐貪縱,配北鎮。是歲,高麗、宕昌、吐谷渾、勿
 吉、武興等國並遣使朝貢。



 十三年春正月辛亥,祀圓丘,初備大駕。乙丑,兗州人王伯恭聚眾勞山,自稱齊王,東萊鎮將孔伯孫討斬之。戊辰,齊人寇邊,淮南太守王僧俊擊走之。二月庚子,引群臣訪政道得失損益之宜。三月,夏州刺史章武王彬以貪財削封。夏四月丁丑,詔曰:「升樓散物,以齎百姓,至使人馬騰踐,多有毀傷。今可斷之。以本所費之物賜窮老貧獨者。」州鎮十五大饑,詔所在開倉振恤。五月庚戌,祀方澤。六月,汝陰王天賜、南安王楨並坐贓賄,免為庶人。秋七月,立孔子廟於京師。八月乙亥,詔兼員外散騎常
 侍邢產使于齊。九月,出宮人賜北鎮人貧鰥者。冬十一月己未,安豐王猛薨。十二月丙子,司空、河東王茍頹薨。甲午,齊人來聘。己亥,以尚書令尉元為司徒,左僕射穆亮為司空。是歲,高麗、吐谷渾、陰平、中赤、武興、宕昌等國各遣使朝貢。



 十四年春正月己巳朔,日有蝕之。三月戊寅,初詔定起居注制。詔遣侍臣巡行州郡,問人疾苦。夏四月,地豆干頻犯塞。甲戌,征西大將軍、陽平王熙擊走之。



 甲午,詔兼員外散騎常侍邢產使于頤齊。五月己酉,庫莫奚犯塞,安州都將樓龍兒擊走之。沙門司馬御惠自言聖王,謀破
 平原郡,禽獲伏誅。秋七月甲辰,詔罷都牧雜制。八月,詔議國之行次。九月癸丑,太皇太后馮氏崩。詔聽籓鎮曾經內侍者,前後奔赴。冬十月戊辰,詔將親侍龍輿,奉訣陵隧,諸常從之具,悉可停之。其武衛之官,防侍如法。癸酉,葬文明太皇太后於永固陵。甲戌,車駕謁永固陵。群臣固請公除,帝不許。己卯,車駕謁永固陵。庚辰,帝居廬,引見群僚於太和殿。太尉、東陽王丕等據權制固請。帝引古禮往復,群臣乃止。京兆王太興有罪,免官削爵。詔曰:「公卿屢依金冊遺旨,中代權制,式請過葬即吉。朕思遵遠古,終三年之制。依禮,既虞卒哭。此月二十一日授
 服,以葛易麻。既以衰服在上,公卿不得獨釋於下,故於朕之授服,變從練禮已下復為節降。斟酌古今,以制厥衷。且取遺旨速除之一端,粗申臣子罔極之巨痛。」癸未,詔曰:「朕遠遵古式,欲終三年之禮。百辟群臣,據金冊顧命,將奪朕心,從先朝之制。朕仰惟金冊,俯自推省,取諸二衷,不許眾議。以衰服過期,終四節之慕。又奉遵聖訓,聿脩誥旨,不敢暗默自居,以曠機政。庶不愆遺令之意,差展哀慕之情。並下州鎮,長至、三元,絕告慶之禮。」甲申,車駕謁永固陵。十一月甲寅,詔內外職人先朝班次及諸方雜客,冬至之日,盡聽入臨。三品已下衰服者,至夕
 復臨。其餘唯旦臨而已。其拜哭之節,一依別儀。丁巳,齊人來聘。十二月壬午,詔依準丘井之式,遣使與州郡宣行條制。



 隱口漏丁,即聽附實。若朋附豪勢,陵抑孤獨,罪有常刑。是歲,吐谷渾、宕昌、武興、陰平、高麗等國並遣使朝貢。



 十五年春正月丁巳,帝始聽政於皇信東室。初分置左右史官。癸亥晦,日有蝕之。二月己丑,齊人來聘。三月甲辰,車駕謁永固陵。夏四月癸亥,帝始進蔬食。



 乙丑,謁永固陵。自正月不雨至於癸酉,有司奏祈百神。詔曰:「何宜四氣未周,便行禮事,唯當考躬責己,以待天譴。」甲戌,詔
 員外散騎常侍李彪使于齊。己卯,經始明堂,改營太廟。五月己亥,議改律令。於東明觀折疑獄。乙卯,枹罕鎮將長孫百年攻吐谷渾所置洮陽、泥和二戍剋之,俘獲三萬餘人。詔悉免歸。丙辰,詔造五輅。六月丁未,濟陰王鬱以貪殘賜死。秋七月乙丑,謁永固陵。規建壽陵。己卯,詔議祖宗,以道武皇帝為太祖。乙酉,車駕巡省京邑,聽訟而還。八月壬辰,議養老。又議肆類上帝、禋于六宗禮,帝親臨決。詔郡國有時物可以薦宗廟者,貢之。



 戊戌,移道壇於桑乾之陰,改曰崇虛寺。己亥,詔諸州舉秀才,先盡才學。乙巳,親定禘祫禮。丁巳,議律令事,仍省雜祀。九月
 辛巳,齊人來聘。冬十月庚寅,車駕謁永固陵。是月,明堂太廟成。十一月丁卯,遷七廟神主於新廟。乙亥,大定官品。戊寅,考諸牧守。詔假通直散騎常侍李彪聘於齊。丙戌,初罷小歲賀。丁亥,詔二千石考上上者,假四品將軍,賜乘黃馬一匹;上中者,五品將軍;上下者,衣一襲。十二月壬辰,遷社於內城之西。癸巳,班賜刺史已下衣冠。以安定王休為太傅,齊郡王簡為太保。帝為高麗王璉舉哀於城東行宮。己酉,車駕迎春於東郊。辛卯,詔簡選樂官。是歲,吐谷渾、悉萬斤、高麗、鄧至、宕昌等國並遣使朝貢。



 十六年春正月戊午朔,朝饗群臣於太華殿。帝始為王公興縣而不樂。己未,宗祀顯祖獻文皇帝於明堂,以配上帝。遂升臺以觀雲物,降居青陽左個,布政事。每朔依以為常。辛酉,始以太祖配南郊。壬戌,詔定行次,以水承金。甲子,詔罷袒裸。乙丑,制諸遠屬非太祖子孫及異姓為王者,皆降為公,公為侯,侯為伯,子男仍舊。皆除將軍之號。戊辰,帝臨思義殿,策問秀、孝。丙子,始以孟月祭廟。二月戊子,帝移御永樂宮。庚寅,壞太華殿,經始太極殿。辛卯,罷寒食享。壬辰,幸北部曹,歷觀諸省。巡省京邑,聽理冤訟。甲千,車駕初朝日於東郊,遂以為常。



 丁酉,詔祀
 唐堯於平陽,虞舜於廣寧,夏禹於安邑,周文於洛陽。丁未,改謚宣尼曰文聖尼父,告謚孔廟。三月丁卯,巡省京邑。癸酉,省西郊郊天雜事。乙亥,車駕初迎氣於南郊,自此為常。辛巳,以高麗王璉孫雲為其國王。齊人來聘。夏四月丁亥朔,頒新律令,大赦。甲寅,幸皇宗學,親問博士經義。五月癸未,詔群臣於皇信堂更定律條,流徒限制,帝親臨決之。秋七月壬戌,詔曰:「自今選舉,每以季月,本曹與吏部銓簡。」甲戌,詔兼員外散騎常侍宋弁使於齊。八月庚寅,車駕初祀夕月於西郊,遂以為常。乙未,詔陽平王頤、左僕射陸睿督十二將北討蠕蠕。



 丙午,宕昌王
 梁彌承來朝。司徒尉元以老遜位。己酉,以尉元為三老,游明根為五更。又養國老、庶老,將行大射之禮。雨-,不克成。癸丑,詔曰:「國家雖宗文以懷九服,修武以寧八荒。然於習武之方,猶為未盡。將於馬射之前,先行講武之式。



 可敕有司豫修場埒。其列陣之儀,五戎之數,別俟後敕。」九月甲寅朔,大序昭穆於明常,祀文明太皇太后於玄堂。辛未,帝以文明太皇太后再周忌日,哭於陵左,絕膳三日,哭不輟聲。辛巳,武興王楊集始來朝。冬十月己亥,以太傅、安定王休為大司馬,特進馮誕為司徒。甲辰,詔以功臣配饗太廟。庚戌,太極殿成,饗群臣。



 十一月乙卯,
 依古六寢,權制三室,以安昌殿為內寢,皇信堂為中寢,四下為外寢。



 十二月,賜京邑老人鳩杖。齊人來聘。是歲,高麗、鄧至、契丹、齒、吐谷渾等國並遣使朝貢。



 十七年春正月壬子朔,饗百寮於太極殿。乙丑,詔大賜諸蕃君長車、旗、衣、馬、錦彩、繒纊,多者一千,少者三百,各以命數為差。詔兼員外散騎常侍邢巒使於齊。丙子,以吐谷渾伏連籌為其國王。庚辰,蠲大司馬安定王休、太保齊郡王簡朔望之朝。二月乙酉,詔賜議律令之官各有差。己酉,車駕始籍田於都南。三月戊辰,改作後宮。夏四月戊戌,立皇后馮氏。是月,齊直閣將軍蠻酋田益宗
 率部落內屬。五月壬戌,宴四廟子孫於宣文堂,帝親與之齒,行家人禮。甲子,帝臨朝堂,引見公卿以下。決疑政,錄囚徒。丁丑,以旱徹膳。襄陽蠻酋雷婆思等率其部內徙,居於太和川。六月庚辰朔,日有蝕之。丙戌,帝將南伐,詔造河橋。乙未,講武。



 乙巳,詔曰:「比百秩雖陳,事典未敘。自八元樹位,躬加省覽,作職員令二十一卷。事迫戎期,未善周悉,須待軍回,更論所闕。權可付外施行。」立皇子恂為皇太子。秋七月癸丑,以皇太子立,詔賜人為父後者爵一級,為公士。曾為吏屬者爵二級,為上造。鰥寡孤獨不能自存者,人粟五斛。戊午,中外戒嚴。是月,齊武帝
 殂。八月乙酉,三老山陽郡公尉元薨。丙戌,車駕類於上帝,遂臨尉元喪。丁亥,帝辭永固陵。己丑,發京師南伐,步騎三十餘萬。太尉丕奏請以宮人從,詔以臨戎不語內事,不許。壬寅,車駕至肆州。人年七十已上,賜爵一級。路見眇跛,停駕親問,賜衣食,復終事。戊申,幸並州,親見高年,問疾苦。九月壬子,詔兼員外散騎常侍高聰聘於齊。丁巳,詔車駕所經,傷人秋稼者,畝給穀五斛。戊辰,濟河。



 詔洛、懷、並、肆所過四州,賜高年爵,恤鰥寡孤獨各有差。孝悌廉義文武應求者,皆以名聞。又詔廝養戶不得與庶士婚,有文武之才積勞應進者,同庶族例,聽之。



 庚午,
 幸洛陽,周巡故宮基跡。帝顧謂侍臣曰:「晉德不修,荒毀至此!」遂詠《黍離詩》,為之流涕。壬申,觀河橋。幸太學,觀石經。丙子,六軍發軫。丁丑,帝戎服執鞭,御馬而出。群臣稽顙於馬前,請停南伐,帝乃止。仍議遷都計。冬十月戊寅朔,幸金墉城。詔徵司空穆亮與尚書李沖、將作大匠董爵經始洛京。己卯,幸河南城。乙酉,幸豫州。癸巳,次於石濟。乙未,解嚴。設壇於滑臺宮。詔京師及諸州從戎者,賜爵一級。應募者,加二級。主將加三級。癸卯,幸鄴城。乙巳,詔安定王休率從官迎家口於代,車駕送於漳水上。初,帝之南伐,起宮殿於鄴西。



 十一月癸亥,宮成,徙御焉。十
 二月戊寅,巡省六軍。乙未,詔隱恤軍士,死亡疾病,務令優給。是歲,勿吉、吐谷渾、宕昌、陰平、契丹、庫莫奚、高麗、鄧至等國並遣使朝貢。



 十八年春正月丁未朔,朝群臣於鄴宮澄鸞殿。癸亥,南巡。詔相、兗、豫三州賜高年爵,恤鰥寡孤老各有差。孝悌廉義文武應求者,皆以名聞。戊辰,經殷比干墓,祭以太牢。乙亥,幸洛陽西宮。二月己丑,行幸河陰,規建方澤之所。丙申,徙封河南王乾為趙郡王,潁川王雍為高陽王。壬寅,北巡。癸卯,齊人來聘。甲辰,詔喻天下以遷都意。閏月癸亥,次勾注陘南。皇太子朝於蒲地。壬申,至平城宮。



 癸酉,臨朝堂,部分遷留。甲戌,謁永固陵。三月庚辰,罷西郊祭天。壬辰,帝臨太極殿,喻在代群臣遷移之略。夏五月甲戌朔,日有蝕之。乙亥,詔罷五月五日、七月七日饗。六月己巳,詔兼員外散騎常侍盧昶使於齊。秋七月乙亥,以宋王劉昶為大將軍。壬辰,北巡。戊戌,謁金陵。辛丑,幸朔州。是月,齊蕭鸞殺其主昭業。



 八月亥亥,皇太子朝於行宮。甲辰,行幸陰山,觀雲川。丁未,幸閱武臺,臨觀講武。因幸懷朔、武川、撫冥、柔玄等四鎮。乙丑,南還。所過皆親見高年,問人疾苦,貧窘孤老者,賜以粟帛。丙寅,詔六鎮及禦夷城人年老孤貧廢疾者,賜粟宥罪各有差。戊
 辰,車駕次旋鴻池。庚午,謁永固陵。辛未,還平城宮。九月壬申朔,詔曰:「三載考績,自古通經,三考黜陟,以彰能否。朕今三載一考,考即黜陟。



 欲令愚滯無妨於賢者,才能不擁於下位。各令當曹,考其優劣為三等。六品已下,尚書重問;五品已上,朕將親與公卿論其善惡。上上者遷之,下下者黜之,中中者守其本任。」壬午,帝臨朝堂,親加黜陟。壬辰,陰平王楊炅來朝。冬十月甲辰,以太尉、東陽王丕為太傅。戊申,親告太廟,奉迎神主。辛亥,車駕發平城宮。壬戌,次於中山之唐湖。乙丑,分遣侍臣,巡問疾苦。己巳,幸信都。庚午,詔曰:「比聞緣邊之蠻,多有竊掠,致有
 父子乖離,室家分絕。可詔荊、郢、東荊三州,勒諸蠻人,勿有侵暴。」是月,齊蕭鸞殺其主昭文而自立。十一月辛未朔,詔冀、定二州,賜高年爵,恤鰥寡孤老各有差。孝義廉貞文武應求者,具以名聞。丁丑,幸鄴。甲申,經比干墓,親為弔文,樹碑刊之。己丑,車駕至洛陽。十二月辛丑朔,分命諸將南征。壬寅,革衣服之制。癸卯,詔中外戒嚴。戊申,復代遷戶租賦三歲。



 己酉,詔王、公、伯、子、男開國食邑者:王食半;公三分食一;侯、伯四分食一;子、男五分食一。辛亥,車駕南伐。丁卯,詔郢、豫二州賜高年爵,恤孤寡鰥老各有差。緣路之丁,復田租一歲。孝悌廉貞文武應求者,
 具以名聞。戊辰,車駕至懸瓠。己巳,詔壽陽、鐘離、馬頭之師所獲男女口皆放還南。是歲,高麗國遣使朝貢。



 十九年春正月辛未朔,朝饗群臣於懸瓠。癸酉,詔禁淮北人不得侵掠,犯者以大辟論。壬午,講武於汝水西,大齎六軍。平南將軍王肅、左將軍元麗並大破齊軍。



 己亥,車駕濟淮。二月甲辰,幸八公山。路中雨甚,詔去蓋。見軍士病者,親隱恤之。戊申,車駕巡淮南,東人皆安堵,租運屬路。丙辰,幸鐘離。戊午,軍士禽齊人三千。帝曰:「在君為君,其人何罪?」於是免歸。辛酉,發鐘離,將臨江水。



 司徒馮誕薨。壬戌,詔班師。丁卯,遣使臨江,數齊主罪惡。三月戊
 子,太師馮熙薨。夏四月丁未,曲赦徐、豫二州,其運轉之士,復租三年。辛亥,詔賜高年爵,恤孤寡老疾各有差。德著丘園者,具以名聞。齊人降者,給復十五年。癸丑,幸小沛。使以太牢祭漢高祖廟。己未,幸瑕丘。使以太牢祠岱嶽。詔宿衛武官增位一級。



 庚申,幸魯城。親祠孔子廟。辛酉,詔拜孔氏四人,顏氏二人為官。詔兗州刺史舉部內士人堪軍國及守宰政績者,具以名聞。詔賜兗州人爵及粟帛如徐州。又詔選諸孔宗子一人封崇聖侯,邑一百戶,以奉孔子祀。命兗州為孔子起園栢,修飾墳隴,更建碑銘,褒揚聖德。戊辰,行幸碻磝。太和廟成。五月己巳,
 城陽王鸞赭陽失利,降為定襄縣王。廣川王諧薨。庚午,遷文成皇后馮氏神主於太和廟。甲戌,行幸滑臺。丙子,次于石濟。庚辰,皇太子朝於平桃城。癸未,車駕至自南伐。甲申,滅閑官祿以裨軍國之用。乙酉,行飲至禮,班賜各有差。甲午,皇太子冠於廟。六月己亥,詔不得以北俗之語,言於朝廷。違者,免所居官。辛丑,詔復軍士從駕渡淮者租賦三年。癸卯,詔皇太子赴平城宮。壬子,詔濟州、東郡、滎陽及河南諸縣車駕所經者,賜高年爵,恤孤寡老疾各有差。。孝悌廉義文武應求者,具以名聞。癸丑,求天下遺書。祕閣所無,有裨時用者,加以厚賞。乙卯,曲赦
 梁州,復人田租三歲。丙辰,詔遷洛人,死葬河南,不得還北。於是代人南遷者,悉為河南洛陽人。



 戊午,詔改長尺大斗,依《周禮》制度,班之天下。秋八月,幸西宮。路見壞塚露棺,駐輦埋之。乙巳,詔選天下勇士十五萬人為羽林、武賁,以充宿衛。丁巳,詔諸從兵從征被傷者,皆聽還本。金墉宮成。甲子,引群臣歷宴殿堂。九月,六宮及文武盡遷洛陽。丙戌,行幸鄴。丁亥,詔諸墓舊銘記見存昭然為時人所知者,三公及位從公者,去墓三十步。尚書令僕、九列,十五步。黃門、五校,十步:各不聽墾殖。壬辰,遣黃門郎以太牢祭比干墓。乙未,車駕還宮。冬十月甲辰,曲赦
 相州,賜高年爵,恤孤老痼疾各有差。丙辰,車駕至自鄴。辛酉,詔州郡舉士。壬戌,詔諸州牧考屬官為三等之科以聞,將加親覽,以定升降。詔徐、兗、光、南、青、荊、洛七州嚴纂戎備,應須赴集。十一月,行幸委粟山。議定圓丘。甲申,祀圓丘。丙戌,大赦。十二月乙未朔,引見群臣光極堂,宣下品令,為大選之始。辛酉,以咸陽王禧為長兼太尉,復前南安王楨本爵。甲子,引見群臣光極堂,班賜冠服。是歲,高麗、鄧至、吐谷渾等國各遣使朝貢。



 二十年春正月丁卯,詔改姓元氏。壬辰,封始平王勰為彭城王,復封定襄王鸞為城陽王。二月辛丑,幸華林,聽
 訟於都亭。壬寅,詔自非金革,皆聽終三年喪。



 丙午,詔畿內七十己上,暮春赴京師,將行養老禮。庚戌,幸華林,聽訟於都亭。



 癸丑,詔介山之邑,聽為寒食,自餘禁斷。三月丙寅,宴群臣及國老、庶老於華林園。詔國老黃耇以上,假中散大夫、郡守。耋年以上,假給事中、縣令。庶老直假郡縣。各賜鳩杖衣裳。丁丑,詔諸州中正各舉其鄉人望,年五十已上,守素衡門者,授以令長。夏五月丙子,詔敦勸農功,令畿內嚴加課督。墮業者申以楚撻,力田者具以名聞。丙戌,初營方澤於河陰。遣使以太牢祭漢光武及明、章三帝陵。又詔漢、魏、晉諸帝陵各禁方百步不得
 樵蘇踐藉。丁亥,祀方澤。秋七月,廢皇后馮氏。戊寅,帝以久旱,咸秩群神。自癸未不食至于乙酉。是夜,澍雨大洽。八月壬辰朔,幸華林園,親錄囚徒,咸降本罪二等決遣之。丁巳,南安王楨薨。幸華林園聽訟。



 九月戊辰,車駕閱武于小平津。癸酉,還宮。丁亥,將通洛水入穀,帝親臨觀。庚寅晦,日有蝕之。冬十月戊戌,以代遷之士,皆為羽林、武賁。司州之人,十二夫調一吏,為四年更卒,歲開番假,以供公私方役。己酉,曲赦京師。十一月乙酉,復封前汝陰王天賜孫景和為汝陰王,前京兆王太興為西河王。十二月甲子,以西北州郡旱儉,遣侍臣巡察,開倉振恤。
 乙丑,開鹽池禁。丙寅,廢皇太子恂為庶人。



 戊辰,置常平倉。樂陵王思譽知恒州刺史穆泰謀反不告,削爵為庶人。



 二十一年春正月丙申,立皇子恪為皇太子。賜天下為父後者爵一級。己亥,遣侍臣巡方省察,問人疾苦,黜陟守宰。乙巳,北巡。二月壬戌,次於太原。親見高年,問所不便。乙丑,詔并州士人年六十以上,假以郡守。先是,定州人王金鉤訛言自稱應王。丙寅,州郡捕斬之。癸酉,車駕至平城。甲戌,謁永固陵。乙未,南巡。甲寅,詔汾州賜高年爵各有差。丙辰,次平陽。使以太牢祭唐堯。夏四月庚申,
 幸龍門。使以太牢祭夏禹。癸亥,幸蒲阪。使以太牢祭虞舜。修堯、舜、夏禹廟。



 辛未,幸長安。壬申,武興王楊集始來朝。乙亥,親見高年,問所疾苦。丙子,遣侍臣分省縣邑,振賜穀帛。戊寅,幸未央殿、阿房宮,遂幸昆明池。癸未,宋王劉昶薨。丙戌,使以太牢祀漢帝諸陵。五月丁亥朔,衛大國遣使朝貢。己丑,車駕東旋,泛渭入河。庚寅,詔雍州士人百年以上,假華郡太守。九十以上,假荒郡。八十以上,假華縣。七十以上,假荒縣。庶老以年各減一等,七十已上,賜爵三級。



 其營船夫,賜爵一級。孤寡鰥貧,各賜穀帛。其孝友德義文武才幹,悉仰貢舉。壬辰,使以太牢祭周
 文王於酆,祭周武王於鎬。癸卯,遣使祭華岳。六月庚申,車駕至自長安。壬戌,詔冀、定、瀛、相、濟五州,發卒士二十萬,將以南討。癸亥,司空穆亮遜位。秋七月甲午,立昭儀馮氏為皇后。甲寅,帝親為群臣講《喪服》於清徽堂。八月丙辰,詔中外戒嚴。壬戌,立皇子愉為京兆王,懌為清河王,懷為廣平王。戊辰,講武於華林園。庚辰,車駕南討。九月丙申,詔司州洛陽人年七十以上無子孫,六十以上無期親,貧不自存者,給以衣食。及不滿六十而有廢痼之疾,無大功親,窮困無以自療者,皆於別坊,遣醫救護,給太醫師四人,豫請藥物療之。



 辛丑,帝留諸將攻赭陽,
 引師南討。丁未,車駕發南陽,留太尉咸陽王禧、前將軍元英攻之。己酉,車駕至新野。冬十月丁巳,四面進攻不剋,詔左右軍築長圍以守之。乙亥,追廢貞皇后林氏為庶人。十一月丁酉,大破齊軍於沔北。於是人皆復業。



 九十以上,假以郡守。六十五以上,假以縣令。十二月丁卯,詔流、徙之囚,皆勿決遣,登城之際,令其先鋒自效。庚午,車駕臨沔,遂東還。戊寅,還新野。己卯,親行營壘,恤六軍。以齊郡王子琛紹河間王若後。高昌國遣使朝貢。



 二十二年春正月癸未朔,饗群臣於新野行宮。丁亥,拔新野,斬其太守劉忌於宛。庚午,至自新野。辛未,詔以穰
 人首歸大順始終若一者,給復三十年,標其所居曰歸義鄉。次降者,給復十五年。三月壬午朔,大破齊將崔慧景、蕭衍軍於鄧城。



 庚寅,行幸樊城,觀兵襄沔,耀武而還。曲赦二荊、魯陽。辛亥,行幸懸瓠。夏四月,趙郡王乾薨。秋七月壬午,詔后之私府損半。六宮嬪御、五服男女恒恤恒供,亦令減半。在戎之親,三分省一:以供賞。是月,齊明帝殂。八月辛亥,皇太子自京師來朝。壬戌,高麗國遣使朝貢。九月己亥,帝以禮不伐喪,詔反IM。丙午,車駕發懸瓠。冬十月乙酉朔,曲赦二豫州殊死已下,復人田租一歲。十一月辛巳,幸鄴。



 二十三年春正月戊寅朔,朝饗群臣於鄴。先是,帝不豫,至是有瘳。庚辰,群臣上壽,大饗於澄鸞殿。壬午,幸西門豹祠,遂歷漳水而還。戊戌,車駕至自鄴。



 癸卯,行飲至策勛之禮。甲辰,大赦。太保、齊郡王簡薨。二月辛亥,以長兼太尉、咸陽王禧為太尉。癸亥,以中軍大將軍、彭城王勰為司徒。復樂陵王思譽本封。癸酉,齊將陳顯達攻陷馬圈戍。三月庚辰,車駕南伐。癸未,次梁城。丙戌,帝不豫。



 丁酉,車駕至馬圈。戊戌,頻戰破之。己亥,收其戎資億計。諸將追奔漢水,斬獲及赴水死者十八九。庚子,帝疾甚,車駕北次穀塘原。甲辰,詔賜皇后馮氏死。詔司徒勰徵太
 子於魯陽踐阼。以北海王詳為司空,王肅為尚書令,廣陽王嘉為左僕射,尚書宋弁為吏部尚書。與太尉咸陽王禧、右僕射任城王澄等六人輔政。夏四月丙午朔,帝崩于穀塘原之行宮,時年三十三。祕諱至魯陽發喪,還京師。上謚曰孝文皇帝,廟曰高祖。五月丙申,葬長陵。



 帝幼有至性。年四歲時,獻文患癰,帝親自吮膿。五歲受禪,悲泣不自勝。獻文問其故,對曰:「代親之感,內切於心。」獻文甚歎異之。文明太后以帝聰聖,後或不利馮氏,將謀廢帝。乃於寒月,單衣閉室,絕食三朝,召咸陽王禧將立之。



 元丕、穆泰、李沖固諫乃止。帝初不有憾,唯深德丕等。
 撫念諸弟,始終曾無纖介。



 惇睦九族,禮敬俱深。雖於大臣,持法不縱。然性寬慈,進食者曾以熱羹覆帝手,又曾於食中得蟲穢物,並笑而恕之。宦者先有譖帝於太后,太后杖帝數十,帝默受,不自申明。太后崩後,亦不以介意。



 聽覽政事,從善如流。哀矜百姓,恆思所以濟益。天地、五郊、宗廟、二分之禮,帝必躬親,不以寒暑為倦。尚書奏案,多自尋省;百官大小,無不留心。務於周洽,每言:凡為人君,患於不均,不能推誠遇物;茍能均誠,胡越之人,亦可親如兄弟。常從容謂史官曰:「直書時事,無諱國惡。人君威福自己,史復不書,將何所懼!」南北征巡,有司奏請
 脩道,帝曰:「粗修橋梁,通輿馬便止,不須去草剷令平也。」凡所修造,不得已而為之,不為不急之事,重損人力。巡幸淮南,如在內地。軍事須伐人樹者,必留絹以酬其直。人苗稼無所傷踐。諸有禁忌禳厭之方非典籍所載者,一皆除罷。雅好讀書,手不釋卷。《五經》之義,覽之便講。學不師受,探其精奧;史傳百家,無不該涉。善談莊、老,尤精釋義。才藻富贍,好為文章;詩賦銘頌,在興而作。有大文筆,馬上口授;及其成也,不改一字。自太和十年已後,詔冊皆帝文也。自餘文章,百有餘篇。愛奇好士,情如飢渴。待納朝賢,隨才輕重。常寄以布素之意。,悠然玄邁,不以
 世務嬰心。又少善射,有膂力:年十餘,能以指彈碎羊膊骨;射禽獸,莫不隨行所至而斃之。至十五,便不復殺生,射獵之事悉止。性儉素,常服浣濯之衣,鞍勒鐵木而已。帝之雅志,皆此類也。



 論曰:有魏始基代朔,廓平南夏;闢土經世,咸以威武為業。文教之事,所未遑也。孝文纂承洪緒,早著睿聖之風。時以文明攝事,優游恭己;玄覽獨得,著自不言;神契所標,固以符於冥化。及躬總大政,一日萬機,十許年間,曾不暇給;殊塗同歸,百慮一致。夫生靈所難行,人倫之高跡,雖尊居黃屋,盡蹈之矣。若乃欽明稽古,協御天人,帝
 王制作,朝野軌度。斟酌用舍,煥乎其有文章。海內黔黎,咸受耳目之賜。加以雄才大略,愛奇好士,視下如傷,役己利物,亦無得而稱之。



 其經緯天地,豈虛謚也!



\end{pinyinscope}