\article{卷六十三列傳第五十一}

\begin{pinyinscope}

 周惠達馮景蘇
 綽子威從兄
 亮周惠達,字懷文,章武文安人也。父信,歷樂鄉、平舒、成平三縣令,皆以廉能稱。惠達幼有節操,好讀書,美容貌。魏齊王蕭寶夤為瀛州刺史,召惠達及河間馮景同在閣下,甚禮之。及寶夤還明,惠達隨入洛陽。寶夤西征,惠達復隨入關。



 寶夤除雍州刺史,今惠達使洛陽。未還,而寶夤謀反聞於京師。有司以惠達是其行人,將執之。惠達
 乃私馳還。至潼關,遇大使楊侃。侃謂曰:「何為故入獸口?」



 惠達曰:「蕭王必為左右所誤,今往,庶其改圖。」及至,寶夤反形已露,不可彌縫。遂用惠達為光祿勳、中書舍人。寶夤既敗,唯惠達等數人從之。寶夤語惠達曰:「人生富貴,左右咸言盡節,及遭厄難,乃知歲寒也。」



 賀拔岳為關中大行,惠達為岳府屬。岳為侯莫陳悅所害,惠達遁入漢陽之麥積崖。悅平,歸於周文帝。文帝復以為府司馬,便委任焉。周文帝為大將軍、大行臺,以惠達為行臺尚書、大將軍府司馬,封文安縣子。周文出鎮華州,留惠達知後事。



 時既承喪亂,庶事多闕。惠達營造戎仗,儲積倉
 糧,簡閱士馬,以濟軍國之務,甚為朝廷所稱。後拜中書令,進爵為公。大統四年,兼尚書右僕射。其年,周文與魏文帝東討,令惠達輔魏太子居守,總留臺事。及芒山失律,人情駭動。趙青雀據長安子城反,惠達奉太子出渭橋北以禦之。軍還,青雀等誅。拜吏部尚書。久之,復為右僕射。自關右草創,禮樂缺然。惠達與禮官損益舊章,是以儀軌稍備。魏文帝因朝奏樂,顧謂惠達曰:「此卿功也。」惠達雖居顯職,性廉退,善下人,盡心勤公,愛拔良士,以此皆敬而附之。薨,子題嗣。隋開皇初,以惠達著績前代,追封蕭國公。



 馮景,字長明,河間武垣人也。父傑,為伏與令。景少與周惠達友,俱以客從蕭寶夤。寶夤後為尚書右僕射,引景領尚書都令史。正光中,寶夤為關西大行臺,景又為行臺都令史。及寶夤敗還長安,或議歸罪闕下,或言留州立功。景曰:「擁兵不還,此罪將大。」寶夤不從,遂反。及寶夤平,景方得還洛。朝廷聞景有諫言,故不罪之。後事賀拔岳為行臺郎。岳使景詣齊神武,察其行事。神武聞岳使至,甚有喜色,問曰:「賀拔公詎憶吾邪?」即與景歃血,託岳為兄弟。景還,以狀報岳。



 岳曰:「此姦有餘,而實不足。自古王臣無私盟者也,吾料之熟矣。」岳北合費也頭,東引紇
 豆陵伊利,西總侯莫陳悅、河州刺史梁景睿及酋渠為盟誓,共會平涼,移軍東下。懼有專任之嫌,使景啟孝武帝。帝甚悅。又為岳大都督府從事中郎。後侯莫陳悅平,周文使景於京師告捷。帝有西遷意,因問關中事勢。景勸帝西遷。後以迎孝武功,封高陽縣伯,除散騎常侍、行臺尚書。大統初,詔行涇州事,卒於官。



 蘇綽,字令綽,武功人,魏侍中則之九世孫也。累世二千石。父協,武功郡守。



 綽少好學,博覽群書,尤善算術。從兄讓為汾州刺史,周帝餞于都門外。臨別,謂曰:「卿家子弟之中,誰可任用者?」讓因薦綽。周文乃召為行臺郎中。在
 官歲餘,未見知。然諸曹疑事,皆詢於綽而後定。所行公文,綽又為之條式。臺中咸稱其能。



 周文與僕射周惠達論事,惠達不能對,請出外議之。乃召綽,告以其事,綽即為量定。惠達入呈,周文稱善,謂曰:「誰與卿為此議者?」惠達以綽對,因稱其有王佐才。周文曰:「吾亦聞之久矣。」尋除著作佐郎。



 屬周文與公卿往昆明池觀漁,行至城西漢故倉地,顧問左右,莫有知者。或曰:「蘇綽博物多通,請問之。」周文乃召綽問,具以狀對。周文大悅,因問天地造化之始,歷代興亡之迹。綽既有口辯,應對如流。周文益嘉之,乃與綽並馬徐行至池,竟不設網罟而還。遂留綽
 至夜,問以政道,臥而聽之。綽於是指陳帝王之道,兼述申、韓之要。周文乃起,整衣危坐,不覺膝之前席。語遂達曙不厭。詰朝,謂周惠達曰:「蘇綽真奇士,吾方任之以政。」即拜大行臺左丞,參典機密。自是寵遇日隆。綽始制文案程式,朱出墨入,及計帳、戶籍之法。



 大統三年,齊神武三道入寇,諸將咸欲分兵禦之,獨綽意與周文同。遂併力拒竇泰,擒之於潼關。封美陽縣伯。十一年,授大行臺度支尚書,領著作,兼司農卿。



 周文方欲革易時政,務弘強國富人之道,故綽得盡其智能,贊成其事。減官員,置二長,並置屯田以資軍國。又為六條詔書,奏施行之。



 其
 一,先脩心,曰:凡今之方伯守令,皆受命天朝,出臨下國,論其尊貴,並古之諸侯也。是以前代帝王,每稱共理天下者唯良宰守耳。明知百僚卿尹雖各有所司,然其理人之本,莫若守宰之最重也。凡理人之體,當先理已心,心者一身之主,百行之本。心不清靜,則思慮妄生。思慮妄生,則見理不明。見理不明,則是非謬亂。是非既亂,則一身不能自理,安能理人也?是以理人之要,在於清心而已。夫所謂清心者,非不貪貨財之謂,乃欲使心氣清和,志意端靜。心和志靜,則邪僻之慮無因而作。邪僻不作,則凡所思念無不皆得至公之理。率至公之理以臨
 其人,則彼下人孰不從化?



 是以稱理人之本,先在理心。



 其次又在理身。凡人君之身者,乃百姓之表,一國之的也。表不正,不可求直影;的不明,不可責射中。今君身不能自理,而望理百姓,是猶曲表而求直影也;君行不能自脩,而欲百姓脩行者,是猶無的而責射中也。故為人君者,必心如清水,形如白玉,躬行仁義,躬行孝悌,躬行忠信,躬行禮讓,躬行廉平,躬行儉約,然後繼之以無倦,加之以明察。行此八者以訓其人。是以其人畏而愛之,則而象之,不待家教日見而自興行矣。



 其二,敦教化,曰:天地之性,唯人為貴。明其有中和之心,仁恕之行,異於
 木石,不同禽獸,故貴之耳。然性無常守,隨化而遷。化於敦朴者則質直,化於澆偽者則浮薄。浮薄者則衰弊之風,質直者則淳和之俗。衰弊則禍亂交興,淳和則天下自治。自古安危興亡,無不皆由所化也。



 然世道彫喪,已數百年。大亂滋甚,且二十載。人不見德,唯兵革是聞;上無教化,唯刑罰是用。而中興始爾,大難未弭,加之以師旅,因之以饑饉,凡百草創,率多權宜。致使禮讓弗興,風俗未反。比年稍登稔,徭賦差輕,衣食不切,則教化可脩矣。凡諸牧守令長,各宜洗心革意,上承朝旨,下宣教化矣。



 夫化者,貴能扇之以淳風,浸之以太和,被之以道德,示之以
 朴素。使百姓亹亹,日遷於善,邪偽之心,嗜慾之性,潛以消化,而不知其所以然,此之謂化也。



 然後教之以孝悌,使人慈愛;教之以仁順,使人和睦;教之以禮義,使人敬讓。慈愛則不遺其親,和睦則無怨於人,敬讓則不競於物。三者既備,則王道成矣。此之謂教也。先王之所以移風易俗,還淳反素,垂拱而臨天下以至於太平者,莫不由此。



 此之謂要道也。



 其三,盡地利,曰:人生天地之間,衣食為命。食不足則饑,衣不足則寒。饑寒切體,而欲使人興行禮讓者,此猶逆阪走丸,勢不可得也。是以古之聖王知其若此,先足其衣食,然後教化隨之。夫衣食所以
 足者,由於地利盡。地利所以盡者,由於勸課有方。主此教者,在乎牧守令長而已。人者冥也,智不自周,必待勸教然後得盡其力。諸州郡縣,每至歲首,必戎敕部人,無問少長,但能操持農器者,皆令就田,墾發以時,勿失其所。及布種既訖,嘉苗須理,麥秋在野,蠶停於室,若此之時,皆宜少長悉力,男女併功,若揚湯救火,寇盜之將至,然後可使農夫不失其業,蠶婦得就其功。



 若游手怠惰,早歸晚出,好逸惡勞,不勤事業者,則正長牒名郡縣,守令隨事加罰,罪一勸百。此則明宰之教也。



 夫百畝之田,必春耕之,夏種之,秋收之,然後冬食之。此三時者,農之
 要月也。若失其一時,則穀不可得而食。故先王之戒曰:「一夫不耕,天下必有受其饑者;一婦不織,天下必有受其寒者。」若此三時,不務省事,而令人廢農者,是則絕人之命,驅以就死然。單劣之戶,及無牛之家,勸令有無相通,使得兼濟。三農之隙,及陰雨之暇,又當教人種桑植果,藝其蔬菜,修其園圃,畜育雞豚,以備生生之資,以供養老之具。



 夫為政不欲過辭,碎則人煩;勸課亦不容太簡,簡則人怠。善為政者,必消息時宜而適煩簡之中。故詩曰:「不剛不柔,布政優優,百祿是求。」如不能爾,則必陷於刑辟矣。



 其四,擢賢良,曰:天生蒸黎,不能自化,故必立
 君以理之。人君不能獨理,故必置臣以佐之。上自帝王,下及列國,置臣得賢則安,失賢則亂,此乃自然之理,百王不能易也。



 今刺史縣令,悉有僚吏,皆佐助之人也。刺史府官則命於天朝;其州吏以下,並牧守自置。自昔以來,州郡大夫,但取門資,多不擇賢良;末曹小吏,唯試刀筆,並不問志行。夫門資者,乃先世之爵祿,無妨子孫之愚瞽;刀筆者,乃身外之末材,不廢性行之澆偽。若門資之中而得賢良,是則策騏驥而取千里也;若門資之中而得愚瞽,是則土牛木馬,形似而用非,不可以涉道也。若刀筆之中而得志行,是則金相玉質,內外俱美,實為
 人寶也;若刀筆之中而得澆偽,是則飾畫朽木,悅目一時,不可以充榱椽之用也。今之選舉者,當不限資陰,唯在得人。茍得其人,自可起廝養而為卿相,則伊尹、傅說是也,而況兄州郡之職乎?茍非其人,則丹朱、商均雖帝王之胤,不能守百里之封,而況於公卿之胄乎?由此而言,官人之道可見矣。



 凡所求材藝者,為其可以理人。若有材藝而以正直為本者,必以材而為理也;若有材藝而以姦偽為本者,將因其官而亂也,何致化之可得乎?是故將求材藝,必先擇志行,善者則舉之,其志行不善則去之。



 而今擇人者,多云邦國無賢,莫知所舉。此乃未之
 思也,非適理之論。所以然者,古人有言;明主聿興,不降佐於昊天;大人基命,不擢才於后土。常引一世之人,理一世之務。故殷、周不待稷、契之臣,魏、晉無假蕭、曹之佐。仲尼曰:「十室之邑,必有忠信如丘者焉。」豈有萬家之都,而云無士?但求之不勤,擇之不審,或授之不得其所,任之不盡其材,故云無耳。古人云:「千人之秀曰英,萬人之英曰俊。」今之智效一官,行聞一邦者,豈非近英俊之士也?但能勤而審之,去虛取實,各得州郡之最而用之,則人無多少,皆足化矣。孰云無賢!



 夫良玉未剖,與瓦石相類;名驥未馳,與駑馬相雜。及其剖而瑩之,馳而試之,玉
 石駑驥,然後始分。彼賢士之未用也,混於凡品,竟何以異。要任之以事業,責之以成務,方與彼庸流較然不同。或呂望之屠釣,百里奚之飯牛,寧生之扣角,管夷吾之三敗,當此之時,悠悠之徒,豈謂其賢?及升王朝,登霸國,積數十年,功成事立,始識其奇士也。於是後世稱之,不容於口。彼環瑋之才,不世之傑,尚不能以未遇之時,自異於凡品,況降此者哉!若必待太公而後用,是千載無太公;必待夷吾而後任,是百世無夷吾。所以然者,士必從微而至著,功必積小以至大,豈有未任而已成,不用而先達也?若識此理,則賢可求,士可擇。得賢而任之,得
 士而使之,則天下之理,何向而不可成也?



 然善官人者,必先省其官。官省,則善人易充。善人易充,則事無不理。官煩,則必雜不善之人。雜不善之人,則政必有得失。故語曰:「官省則事省,事省則人清;官煩則事煩,事煩則人濁。」清濁之由,在於官之煩省。案今吏員,其數不少。



 昔人殷事廣,尚能克濟,況今戶口減耗?依員而置,猶以為少。如聞在下州郡,尚有兼假,擾亂細人,甚為無理。諸如此輩,悉宜罷黜,無得習常。



 非直州郡之官,宜須善人,爰至黨族閭里正長之職,皆當審擇,各得一鄉之選,以相監統。夫正長者,理人之基。基不傾者上必安。



 凡求賢之路,
 自非一途。然所以得之審者,必由任而試之,考而察之。起於居家,至於鄉黨,訪其所以,觀其所由,則人道明矣,賢與不肖別矣。率此以求,則庶無愆悔矣。



 其五,恤獄訟,曰:人受陰陽之氣以生,有情有性。性則為善,情則為惡。善惡既分,賞罰隨焉。



 賞罰得中,則惡止而善勸;賞罰不中,則人無所措手足,則怨叛之心生。是以先王重之,特加戒慎者,欲使察獄之官,精心悉意,推究根源。先之以五聽,參之以證驗。妙睹情狀,窮鑒陷伏,使姦無所容,罪人必得。然後隨事加刑,輕重皆當,舍過矜愚,得情勿喜。又能消息情理,斟酌禮律,無不曲盡人心,而遠明大教,
 使獲罪者如歸。此則善之上者也。然宰守非一,不可人人皆有通識,推理求情,時或難盡。唯當率至公之心,去阿枉之志,務求曲直,念盡平當。聽察之理,必窮所見,然後拷訊以法,不苛不暴,有疑則從輕,未審不妄罰,隨事斷理,獄無停滯。此亦其次。若乃不仁恕而肆其殘暴,同人木石,專用捶楚。巧詐者,雖事彰而獲免;辭弱者,乃無罪而被罰。有如此者,斯則下矣,非共理所寄。今之宰守,當勤於中科,而慕其上善。如在下條,則刑所不赦。



 又當深思遠大,念存德教。先王之制曰:與殺無辜,寧赦有罪;與其害善,寧其利淫。明必不得中,寧濫捨有罪,不謬害
 善人也。今之從政者則不然,深文巧劾,寧致善人於法,不免有罪於刑。所以然者,非皆好殺人也,但云為吏寧酷,可免後患。此則情存自便,不念至公,奉法如此,皆姦人也。夫人者,天地之貴物,一死不可復生。然楚毒之下,以痛自誣,不被申理,遂陷刑戮者,將恐往往而有。是以自古已來,設五聽三宥之法,著明慎庶獄之典,此皆愛人甚也。凡伐木殺草,田獵不順,尚違時令而虧帝道;況刑罰不中,濫害善人,寧不傷天心,犯和氣!和氣損而欲陰陽調適,四時順育,萬物阜安,蒼生悅樂者,不可得也。故語曰,一夫吁嗟,王道為之傾覆,正謂此也。凡百宰守,
 可無慎乎!



 若深姦巨猾,傷化敗俗,悖亂人倫,不忠不孝,故為背道,殺一利百,以清王化,重刑可也。識此二途,則刑政盡矣。



 其六,均賦役,曰:聖人之大寶曰位。何以位,曰仁。何以聚人,曰財。明先王必以財聚人,以仁也。今寇逆未平,軍國費廣,雖未遑減省,以恤人瘼,然宜令平均,使下無怨。平均者,不舍豪強而征貧弱,不縱姦巧而困愚拙,此之謂均也。故聖人曰:「蓋均無貧。」



 然財貨之生,其功不易。紡紝織績,起於有漸,非旬日之間,所可造次。必須勸課,使預營理。絹鄉先事
 織紝,麻土早脩紡績。先時而備,至時而輸,故王賦獲供,下人無困。如其不預勸戒,臨時迫切,復恐稽緩,以為己過,捶撲交至,取辦目前。富商大賈,緣茲射利,有者從之貴買,無者與之舉息。輸稅之人,於是弊矣。



 租稅之時,雖有大式,至於斟酌貧富,差次先後,皆事起於正長,而繫之於守令。若斟酌得所,則政和而人悅;若檢理無方,則吏奸而人怨。又差發徭役,多不存意,致令貧弱者或重徭而遠戍,富強者或輕使而近防。守令用懷如此,不存恤人之心,皆王政之罪人也。



 周文甚重之,常置諸坐右。又令百司習誦之,其牧守令長非通六條及計帳者,不
 得居官。



 自有晉之季,文章競為浮華,遂以成俗。周文欲革其弊,因魏帝祭廟,群臣畢至,乃命綽為大誥,奏行之。其詞曰:惟中興十有一年仲夏,庶邦百辟,咸會於王庭。柱國泰洎群公列將罔不來朝。



 時迺大稽百憲,敷於庶邦,用綏我王度。皇帝若曰:「或堯命羲和,允釐百工。舜命九官,庶績咸熙。武丁命說,克號高宗。時惟休哉,朕其欽若。格爾有位,胥暨我太祖之庭,朕將丕命女以厥官。」



 六月丁巳,皇帝朝格於太廟,凡厥具僚,罔不在位。



 皇帝若曰:「咨我元輔、群公、列將、百辟、卿士、庶尹、御事,朕惟夤敷祖宗之靈命,稽于先王之典訓,以大誥於爾在位。昔我太
 祖神皇,肇膺明命,以創我皇基。烈祖、景宗,廓開四表,底定武功。暨乎文祖,誕敷文德。龔惟武考,不霣其舊。自時厥後,陵夷之弊,用興大難于彼東土,則我黎庶,咸墜塗炭。惟台一人,纘戎下武,夙夜祗畏,若涉大川,罔識攸濟。是用稽於帝典,揆於王度,拯我人瘼。



 惟彼哲王,示我通訓,曰天生黎蒸,罔克自乂,上帝降鑒叡聖,植元后以乂之。時惟元后弗克獨乂,博求明德,命百辟群吏以佐之。肆天之命辟,辟之命這,惟以恤人,弗惟逸豫。辟惟元首,庶黎惟趾,股肱惟弼。上下一體,各勤攸司,茲用克臻於皇極。故其彞訓曰:『后克艱厥后,臣克艱厥臣,政乃乂。』
 今台一人,膺天之嘏,既陟元后。股肱百辟,乂服我國家之命,罔不咸守厥職。嗟!后弗艱厥后,臣弗艱厥臣,政於何弗斁?嗚呼艱哉!凡爾在位,其敬聽命。」



 皇帝若曰:「柱國,惟四海之不造,載繇二紀。天未絕我太祖、烈祖之命,用錫我以元輔。國家將附,公惟棟梁。皇之弗極,公惟作相。百揆愆度,公惟大錄。



 公其允文允武,克明克乂,迪七德,敷九功,龕暴除亂,下綏我蒼生,傍施於九正,若伊之在商,周之有呂,說之相丁,用保我無疆之祚。」



 皇帝若曰:「群公、太宰、太尉、司徒、司空。惟公作朕鼎足,以弼乎朕躬。



 宰惟天官,克諧六職。尉惟司武,武在止戈。徒惟司眾,敬敷五教。空
 惟司土,利用厚生。惟時三事,若三階之在天;惟茲四輔,若四時之成歲。天工人其代諸。」



 皇帝若曰:「列將,汝惟鷹揚,作朕爪牙。寇賊姦宄,蠻夷猾夏,汝徂征。綏之以惠,董之以威,刑期無刑,萬邦咸寧。俾八表之內,莫違朕命,時汝功。」



 皇帝若曰:「庶邦列辟,汝惟守土,作人父母。人惟不勝其饑,故先王重農;不勝其寒,故先王貴女工。人之不率於孝慈,則骨肉之恩薄;弗惇於禮讓,則爭奪之萌生。惟茲六物,實為教本。嗚呼!為上在寬,寬則人怠,齊之以禮,不剛不柔,稽極於道。」



 皇帝若曰:「卿士、庶尹、凡百御事,王省惟歲,卿士惟月,庶尹惟日,御事惟時。歲月日時,罔
 易其度,百憲咸貞,庶績其凝。嗚呼!惟若王官,陶均萬國,若天之有斗,斟元氣,酌陰陽,弗失其和,蒼生永賴;悖其序,萬物以傷。時惟艱哉!」



 皇帝若曰:「惟天地之道,一陰一陽;禮俗之變,一文一質。爰自三五,以迄于茲,匪惟相革,惟其救弊;匪惟相襲,惟其可久。惟我有魏,承乎周之末流,接秦、漢遺弊,襲魏、晉之華誕,五代澆風,因而未革,將以穆俗興化,庸可暨乎!



 嗟我公輔、庶僚、列辟,朕惟否德,其一朕心力,祗慎厥艱,克遵前王之丕顯休烈,弗敢怠荒。咨爾在位,亦協于朕心,惇德允元,惟厥艱是務。克捐厥華,即厥實,背厥偽,崇厥誠。勿愆勿忘,一乎三代之彞
 典,歸於道德仁義,用保我祖宗之丕命。



 荷天之休,克綏我萬方,永康我黎庶。戒之哉,朕言不再。」



 柱國泰洎庶僚百辟拜手稽首曰:「『亶聰明,作元后,元后作人父母』。惟三五之王,率繇此道,用臻於刑措。自時厥後,歷千載未聞。惟帝念功,將反叔世,逖致於雍熙,庸錫降丕命于我群臣。博哉王言,非言之難,行之實難。臣聞『靡不有初,鮮克有終』。商書曰;『終始惟一,德乃日新。』惟帝敬厥始,慎厥終,以躋日新之德,則我群臣,敢不夙夜對揚休哉!惟茲大誼,未光於四表,以邁種德,俾九域幽遐,咸昭奉元后之明訓,率遷於道,永膺無疆之休。」



 帝曰:「欽哉。」



 自是之後,文
 筆皆依此體。



 綽性儉素,不事產業,家無餘財。以海內未平,常以天下為已任。博求賢俊,共弘政道,凡所薦達,皆至大官。周文亦推心委任,而無間言焉。或出游,常預署空紙以授綽,若須有處分,則隨事施行。及還,啟知而已。綽常謂為國之道,當愛人如慈父,訓人如嚴師。每與公卿議論,自晝達夜,事無巨細,若指諸掌。積思勞倦,遂成氣疾。十二年,卒于位,時年四十九。



 周文痛惜之,哀動左右。及將葬,乃謂公卿等曰:「蘇尚書平生謙退,敦尚儉約。吾欲全其素志,便恐悠悠之徒,有所未達;如其厚加贈謚,又乖宿昔相知之道。



 進退惟谷,孤有疑焉。」尚書令史
 麻瑤越次而進曰:「或晏子,齊之賢大夫,一狐裘三十年。及其死也,遣車一乘。齊侯不奪其志。綽既操履清白,廉挹自居,愚謂宜從儉約,以彰其美。」周文稱善,因薦瑤於朝廷。及綽歸葬武功,唯載,以布車一乘。周文與群公,皆步送出同州郭外。周文親於車後酹酒而言曰:「尚書平生為事,妻子兄弟不知者,吾皆知之。惟爾知吾心,吾知爾意。方欲共定天下,不幸遂捨吾去,奈何!」因舉聲慟哭,不覺卮墜於手。至葬日,又遣使祭以太牢,周文自為其文。



 綽又著佛性論、七經論,並行於世。周明帝二年,以綽配享文帝廟廷。子威嗣。



 威字無畏。少有至性,五歲喪父,哀毀有若成人。周文帝時,襲爵美陽縣公,仕郡功曹。大冢宰宇文護見而禮之,以其女新興公主妻焉。威見護專權,恐禍及已,逃入山。為叔父所逼,卒不獲免。然每居山寺,以諷讀為娛。未幾,授持節、車騎大將軍、儀同三司,改封懷道縣公。武帝親總萬機,拜稍伯下大夫。前後所授,並辭疾不拜。



 有從父妹適河南元世雄。世雄先與突厥有隙,突厥入朝,請世雄及其妻子,將甘心焉。周遂遣之。威以夷人昧利,遂標賣田宅,罄資產贖世雄。論者義之。宣帝嗣位,就拜開府。



 隋文帝為丞相,高熲屢言其賢,亦素重其名,召入臥內,
 與語大悅。居月餘,威聞禪代之議,遁歸田里。高熲請追之。帝曰:「此不欲預吾事,且置之。」及受禪,徵拜太子少保,追贈其父邳國公,以威襲焉。俄兼納言,威上表陳讓,優詔不許。



 帝嘗與文獻皇后對觴,召威及高熲、楊素、廣平王雄四人,謂曰:「太史言朕祚運盡於三年,朕憂懣,故舉此酒耳。今欲營南山險處,與公等固之,以觀時變,將如何?」威進曰:「周文脩德,旋地動之災;宋景一言,退法星三舍。願陛下恢崇德度,享天之休。若棄德恃險,同舟之人,誰非敵國!縱南山之岨,安足固哉?」



 帝善其言,屬之以酒。



 初,威父綽在魏,以國用不足,為征稅法,頗稱為重。既而
 嘆曰:「所為者正如張弓,非平世法也。後之君子,誰能馳乎?」威聞其言,每以為己任。至是,奏減賦役,務從輕典,帝悉從之。漸見親重,與高熲參掌朝政。威見宮中以銀為幔鉤,因盛陳節儉之美,諭帝。帝為改容,雕飾舊物,悉命除毀。帝嘗怒一人,將殺之。



 威入閣進諫,不納。帝怒甚,將自出斬之。威當前不去,帝避之而出。威又遮止帝,帝拂衣入。良久,乃召威謝曰:「公能若是,吾無憂矣。」於是賜馬二匹、錢十餘萬。歲餘,尋復兼大理卿、京兆尹、御史大夫,本官悉如故。持書侍御史梁毗劾威兼領五職,安繁戀劇,無舉賢自代心。帝曰:「蘇威朝夕孜孜,志存遠大,舉賢
 有闕,何遽迫之。」顧謂威曰:「用之則行,捨之則藏,唯我與爾有是夫!」因謂朝臣曰:「蘇威不遇我,無以措其言;我不得蘇威,何以行其道?楊素才辯無雙,至若斟酌古今,助我宣化,非威匹也。蘇威若逢亂世,商山四皓,豈易屈哉!」其見重如此。



 未幾,拜刑部尚書,解少保、御史大夫官。後京兆尹廢,檢校雍州別駕。時高熲與威同心,協贊政刑,大小無不籌之,故革運數年,天下稱平。俄轉戶部尚書,納言如故。屬山東諸州人饑,帝令威振恤之。遷吏部尚書,兼領國子祭酒。隋承戰爭之後,憲章踳駁。帝令朝臣釐改舊法,為一代通典,律令格式多威所定。世以為能。
 九年,拜尚書右僕射。其年,以母憂去職,柴毀骨立。敕勉諭殷勤,未幾,起令視事。固辭,優詔不許。明年,帝幸并州,命與高熲同總留事。俄追詣行在所,使決人訟。



 尋令持節巡撫江南,得以便宜從事。過會稽,踰五嶺而還。江表自晉已來,刑法疏緩,代族貴賤,不相陵越。平陳之後,牧人者盡改變之,無長幼悉使誦五教。



 威加以煩鄙之辭,百姓嗟怨。使還,奏言江表依內州責戶籍。上以江表初平,召戶部尚書張嬰,責以政急。時江南州縣又論言欲徙之入關,遠近驚駭。饒州吳世華起兵為亂,生臠縣令,啖其肉。於是舊陳率土畢反,執長吏,抽其腸而殺之,曰:「
 更使儂誦五教邪!」尋詔內史令楊素討平之。時突厥都藍可汗屢為患,復令威至可汗所。



 威子夔以公子盛名,引致賓客,四海士大夫多歸之。時議樂,夔與國子博士何妥各有所持。於是夔、妥各為一議,使百僚署其所同。朝廷多附威,同夔者十八九。



 妥恚曰:「吾席間函丈四十餘年,反為昨暮兒之所屈也!」遂奏威與禮部尚書盧愷、吏部侍郎薛道衡、尚書右丞王弘、考功侍郎李同和等為朋黨,省中呼王弘為世子,李同和為叔,言二人如威子弟。復言威以曲道任其從父弟徹、肅等罔冒為官。又國子學請黎陽人王孝逸為書學博士,威屬盧愷,以為
 其府參軍。上令蜀王秀、上柱國虞慶則等雜按之,事皆驗。帝以宋書謝晦傳中朋黨事令威讀之。威懼,免冠頓首。



 帝曰:「謝已晚矣!」於是免威官爵,以開府就第。知名之士,坐威得罪者百餘人。



 未幾,帝曰:「蘇威德行者,但為人誤耳。」命之通籍。



 歲餘,復爵邳公,拜納言。從祠太山,坐不敬免。俄而復位。帝謂群臣曰:「世人言蘇威詐清,家累金玉,此妄言也。然其性狠戾,不切世要,求名太甚,從己則悅,違之必怒,此其大病耳。」仁壽初,復拜尚書右僕射。帝幸仁壽宮,以威總留事。及帝還,御史奏威職事多不理。帝怒,詰責威。威謝,帝亦止。



 煬帝嗣位,上將大起長城之
 役,威諫止之。高熲、賀若弼之誅也,威坐相連免官。歲餘,拜魯郡太守,修羽儀。召拜太常卿。從征吐谷渾,進拜右光祿大夫。歲餘,復為納言,與左翊衛大將軍宇文述、黃門侍郎裴矩、御史大夫裴蘊、內史侍郎虞世基參掌朝政,時人稱為五貴。及征遼東,以本官領右武衛大將軍,進位光祿大夫,賜爵房陵侯,尋進封房公。以年老乞骸骨,不許。復以本官參掌選事。明年,從征遼東,領右禦衛大將軍。



 楊玄感之反,帝引威於帳中,懼見於色,謂曰:「此小兒聰明,得不為患邪?」



 威曰:「粗疏非聰明者,必無慮,但恐浸成亂階耳。」威見勞役不已,百姓思亂,以此微欲諷
 帝。帝竟不悟。



 從還,至涿郡,詔威安撫關中,以其孫尚輦直長儇副。威子鴻臚少卿夔先為關中簡黜大使。一家三人,俱使關右,三輔榮之。歲餘,帝手詔曰:「玉以潔潤,丹紫莫能渝其質;松表歲寒,霜尋莫能凋其采。可謂溫仁勁直,性之然乎。房公威,先后舊臣,朝之宿齒,棟梁社稷,弼諧朕躬,守文奉法,卑身率禮。昔漢之三傑,輔惠帝者蕭何;周之十亂,佐成王者邵奭。國之寶器,其在得賢。參璟台階,具瞻斯允。雖事藉論道,終期獻替,銓衡時和,朝寄為重。可開府儀同三司,餘並如故。」



 威當時尊重,朝臣莫與為比。



 後從幸鴈門。為突厥所圍,朝廷危懼。帝欲
 輕騎潰圍而出。威諫曰:「城守則我有餘力,輕騎則彼之所長。陛下萬乘主,何宜輕脫!」帝乃止。突厥俄亦解圍去。



 車駕次太原,威以盜賊不止,勸帝還京師,深根固本,為社稷計。帝初從之,竟用宇文述等議,遂往東都。天下大亂,威知帝不可匡正,甚患之。屬帝問盜賊事。宇文述曰:「盜賊信少,不足為虞。」威不能詭對,以身隱殿柱。帝呼問之。威曰:「臣非職司,不知多少,但患其漸近。」帝曰:「何謂也?」威曰:「他日賊據長白山,今者近在滎陽、汜水。」帝不悅而罷。屬五月五日,百僚上饋,多以珍玩,威獻尚書一部,微以諷帝。帝彌不平。後復問伐遼東事,威對顧赦群盜,遣
 討高麗,帝益怒。御史大夫輩蘊希旨,令御史張行本,奏威昔在高陽典選,濫授人官,怯畏突厥,請還京師。帝令案其事,乃下詔曰:「威立性朋黨,好異端,懷挾詭道,徼幸名利,詆訶律令,謗訕臺省。昔歲薄伐,奉述先志,凡預切問,各盡胸臆,而威不以開懷,遂無對命。啟沃之道,其若是乎!」於是除名。後月餘,人有奏威與突厥陰圖不軌。大理簿責威。威自陳精誠不能上感,瑕爨屢彰,罪當萬死。帝憫而釋之。其年,從幸江都宮。帝將復用威,裴蘊、虞世基奏言昏耄贏疾,帝乃止。



 宇文化及弒逆,以威為光祿大夫、開府儀同三司。化及敗,歸於李密。密敗,歸東都,越
 王侗以為上柱國、邳公。王世充僭號,署太師。威自以隋室舊臣,遭逢喪亂,所經之處,皆與時消息,以求容免。



 及太宗平世充,坐於東都閶闔門內,威請謁見,稱老病不能拜起。上遣人數之曰:「公隋朝宰輔,政亂不能匡救,遂令品物塗炭,君弒國亡。見李密、世充皆拜伏舞蹈。今既老病,無勞相見。」尋入長安,至朝堂請見,高祖又不許。終於家,時年八十二。



 威行己清儉,以廉慎見稱。然每至公議,惡人異已,雖或小事,必固爭之。時人以為無大臣之體。所修格令章程,並行於當世,頗傷煩碎,論者以為非簡久之法。



 及大業末年,尤多征役,至於論功行賞,威每承
 望風旨,輒寢其事。時群盜蜂起,郡縣有奏聞者,又訶詰使人,令減賊數,故出師攻討,多不剋捷。由是遂致敗亂,為物議所譏。子夔。



 夔字伯尼。聰敏有口辯,然性輕險無行。八歲誦詩,兼解騎射。年十三,從父至尚書省。與安德王雄射,賭得駿馬而歸。十四詣學,與諸儒議論,詞致可觀。見者皆稱善。及長,博覽群言,尤以鍾律自命。初名哲,字知人,父威由是改之,頗為有識所哂,起家太子通事舍人。楊素見而奇之,每戲威曰:「楊素無兒,蘇夔無父。」後與鄭譯、何舀議樂,得罪,議寢不行。著樂志十五篇以見其志。數載,遷太子
 舍人,以罪免居數年。仁壽三年,詔天下舉達禮樂源者。晉王昭時為雍州牧,舉夔。與諸州所舉五十餘人謁見。帝望夔,謂侍臣曰:「唯此一人,稱吾所舉。」



 於是拜晉王友。



 煬帝嗣位,歷太子洗馬、司朝謁者。以父免職,夔亦去官。後歷尚書職方郎、燕王司馬。遼東之役,以功拜朝散大夫。時帝方勤遠略,蠻夷來朝。帝謂宇文述、虞世基曰:「四夷率服,觀禮華夏,鴻臚之職,須歸令望。寧有多才藝美容儀,可接賓客者為之乎?」威以夔對。即日拜鴻臚少卿。其年,高昌王麴伯雅來朝,朝廷妻以公主。夔有雅望,令主婚。



 其後延安、弘化等數郡盜賊屯結,詔夔巡關中。及
 突厥圍鴈門,夔於鎮城東南為弩樓、車箱、獸圈,一夕而就。帝見善之。以功進位通議大夫。坐父事,除名。



 後會丁母憂,不勝哀,卒,時年四十九。



 綽弟椿,字令欽。性廉慎,沈勇有決斷。魏正光中,關右賊亂,椿應募討之,授盪寇將軍。以功累遷中散大夫,賜爵美陽子。大統初,拜鎮東將軍、金紫光祿大夫,賜姓賀蘭氏。後除帥都督,行弘農郡事。椿當官強濟,特為周文帝所知。



 十四年,置當州鄉師,自非鄉望允當眾心者不得預焉。乃令驛追椿,領鄉兵。



 其年,破槃頭氏有功,除散騎常侍,加大都督。十六年,征隨郡。軍還,除武功郡守。既為
 本邑,以清儉自居,小大之政,必盡忠恕。進爵為侯,位驃騎大將軍、開府儀同三司、大都督。卒。子植嗣。



 亮字景順,綽從兄也。祖稚,字天佑,位中書侍郎、玉門郡守。父祐,泰山郡守。



 亮少通敏,博學好屬文,善章奏,與弟湛等皆著名西土,一家舉二秀才。亮初舉秀才,至洛陽,過河內常景。景深器之,而謂人曰:「秦中才學可以抗山東者,將此人乎」!魏齊王蕭寶夤引為參軍。寶夤遷大將軍,仍為之掾。寶夤雅相知重,凡有文檄謀議,皆以委之。尋行武功郡事,甚著聲績。寶夤作亂,以亮為黃門侍郎。



 亮善處人間,與物無忤。及寶夤敗,從之者多遇禍,唯亮獲
 全。及長孫承業、爾朱天光等西討,並以亮為郎中,專典文翰。賀拔岳為關西行臺,引亮為左丞,典機密。



 魏孝武西遷,遷吏部郎中。大統二年,拜給事黃門侍郎,領中書舍人。魏文帝子宜都王式為秦州刺史,以亮為司馬。帝謂亮曰:「黃門侍郎豈可為秦州司馬?直以朕愛子出籓,故以心腹相委,勿以為恨。」臨辭,賜以御馬。八年,封臨涇縣子,除中書監,領著作,修國史。亮有機辯,善談笑。周文帝甚重之,有所籌議,率多會旨。記人之善,忘人之過,薦達後進,常如弗及,故當世敬慕。歷祕書監、大行臺尚書,出為岐州刺史。朝廷以其作牧本州,特給路車、鼓吹,先
 還其宅,并給騎士三千,列習儀,游鄉黨,經過故人,歡飲旬日,然後入州。世以為榮。十七年,徵拜侍中,卒於位。贈本官。



 亮少與從弟綽俱知名,然綽文章稍不逮亮,至於經畫進趣,亮又減之。故世稱二蘇焉。亮自大統以來,無歲不轉官,一年或至三遷。僉曰才至,不怪其速也。所著文筆數十篇,頗行於世。子師嗣,以亮名重於時,起家黃門侍郎。



 亮弟湛,字景俊。少有志行,與亮俱著名西土。年二十餘,舉秀才,除奉朝請,領侍御史,加員外散騎侍郎。蕭寶夤西討,以湛為行臺郎中,深見委任。及寶夤將謀叛逆,湛時臥疾於家。寶夤乃令湛從母弟天水姜儉謂湛曰:「吾
 不能坐受死亡,今便為身計,不復作魏臣也。與卿死生榮辱,方當共之,故以相報。」湛聞之,舉聲大哭。儉遽止之曰:「何得便爾?」湛曰:「闔門百口,即時屠滅,云何不哭!」



 哭數十聲,徐謂儉曰:「為我白齊王,王本以窮而歸人,賴朝廷假王羽翼,遂得榮寵至此。既屬國步多虞,不能竭誠報德,豈可乘人間隙,便有問鼎之心乎!今魏德雖衰,天命未改,王之恩義,未洽於人,破亡之期,必不旋踵。蘇湛終不能以積世忠貞之基,一旦為王族滅也。」寶夤復令儉謂湛曰:「此是救命之計,不得不爾。」



 湛復曰:「凡舉大事,當得天下奇士。今但共長安博徒小兒輩為此計,豈有辦
 哉?



 湛不忍見荊棘生王戶庭也。願賜骸骨還舊里,庶歸全地下,無愧先人。」寶夤素重之,知必不為已用,遂聽還武功。寶夤後果敗。



 孝莊帝即位,徵拜尚書郎。帝嘗謂之曰:「聞卿答蕭寶夤,甚有美辭,可為我說之。」湛頓首謝曰:「臣自惟言辭不如伍被遠矣,然始終不易,竊謂過之。但臣與寶夤周旋契闊,言得盡心,而不能令其守節,此臣之罪也。」孝莊大悅,加散騎侍郎。尋遷中書。孝武初,以疾還鄉里,終於家。贈散騎常侍、鎮西將軍、雍州刺史。



 湛弟讓,字景恕。幼聰敏,好學,頗有人倫鑒。初為本州主簿,稍遷別駕、武都郡守、鎮遠將軍、金紫光祿大夫。及周
 文帝為丞相,引為府屬,甚見親待。出為衛將軍、南汾州刺史,有善政。尋卒官。贈車騎大將軍、儀同三司、涇州刺史。



 論曰:周惠達見禮寶夤,遂契闊於戎寇,不以夷險易志,斯固篤終之士也。周文提劍而起,百度草創,施約法之制於競逐之辰,修太平之禮於鼎峙之日,終能斲雕為朴,變奢從儉,風化既被,而下肅上尊,疆埸屢動,而內安外附,斯蓋蘇綽之力也。邳公周道云季,方事幽貞,隋室龍興,首應旌命。綢繆任遇,窮極寵榮,久處機衡,多所損益,罄竭心力,知無不為。然志尚清儉,體非弘廣,好同惡
 異,有乖直道,不存易簡,未為通德。歷事二帝,三十餘年,雖廢黜當時,終稱遺老。君邪而不能正言,國亡而情均眾庶,予違汝弼,徒聞其語,疾風勁草,未見其人。禮命闕於興王,抑亦此之由也。夔志識沈敏,方雅可稱,若天假之年,足以不虧堂構矣。



\end{pinyinscope}