\article{卷八十列傳第六十八 外戚}

\begin{pinyinscope}

 賀訥姚黃眉杜超賀迷閭毗馮熙李惠高肇胡國珍從曾孫長粲楊騰乙弗繪趙猛胡長仁隋文帝外家呂氏夫左賢右戚,尚德尊功,有國者所以御天下也。殷肇王基,不藉莘氏為佐;周成王業,未聞姒姓為輔。然歷觀累
 代外戚之家,乘母后之權以取高位厚秩者,多矣!



 而鮮能有克終之美,必罹顛覆之患,何哉?皆由乎居上不以至公任物,在下徒用私寵要榮。繭犢引大車,升質任厚棟,無德而尊,不知紀極,忽於滿盈之戒,罔念高危之咎。故鬼瞰其室,憂必及之,所以殺身傾族相繼於西京也。夫誠著艱難,功宣社稷,不以謙沖自牧,未免顛蹶之禍;而況道不足以濟時,仁不足以利物,自矜於己,以富貴驕人者乎!



 魏道武初,賀訥有部眾之業,翼成皇祚,其餘或以勞勤,或緣恩澤。齊氏后妃之族,多自保全。胡長仁以譖訴貽禍,斛律光以地勢被戮,俱非女謁盛衰之所
 致也。



 婁昭自以佐命之功,崇其名器,且霸業權輿,時方同德,陵暴之釁,因茲而起。其靖德、昭訓二門,並良家遺孽,守死無暇,固不足涉言。又子非繼世,權難妄假。



 昭信非惟素門履道,訖構廢辱,威望之地,自致無由。有周御歷,后門初無與政,既而末跡竊權,竟移鼎璽,斯乃西漢覆車之轍,魏文所以深誡。隋文潛躍之初,獻后便相推轂;煬帝大橫方兆,蕭妃密勿經綸。是以恩禮綢繆,始終不易。然外內親戚,莫預朝權,昆弟在位,亦無殊寵。至於居擅玉堂,家稱金穴,暉光戚里,熏灼四方,將三司以比儀,命五侯而同拜者,終始一代,寂無聞焉。考之前王,可
 謂矯其弊矣。故雖時經擾攘,無有陷於不義,市朝遷貿,而皆得以保全。比夫憑藉寵私,階緣恩澤,乘其非據,旋就顛隕者,豈可同日而言哉!此所謂愛之以禮者也。



 案外戚,《魏書》有賀訥、劉羅辰、姚黃眉、杜超、賀迷、閭毗、馮熙、李峻、李惠、高肇、于勁、胡國珍、李延實,《齊書》有趙猛、婁睿、爾朱文暢、鄭仲禮、李祖升、元蠻、胡長仁,《周書》不立此篇,《隋書》有獨孤羅、蕭巋。今以劉羅辰、李峻、于勁、李延實、婁睿、爾朱文暢、鄭仲禮、李祖昇、元蠻、獨孤羅、蕭巋各附其家傳,其餘並入此篇。又檢楊騰、乙弗繪附之魏末,以備《外戚傳》。



 賀訥,代人,魏道武皇帝之舅,獻明后之兄也。其先世為君長。祖紇,尚平文女。父野干,尚昭成女遼西公主。昭成崩,諸部乖亂,獻明后與道武及衛、秦二王依訥。會苻堅使劉庫仁分攝國事,道武還居獨孤。訥總攝東部為大人,遷居大寧,行其恩信,眾多歸之,侔於庫仁。苻堅假訥鷹揚將軍。後劉顯謀逆,道武輕騎歸訥,訥驚拜曰:「官家復國,當念老臣。」帝笑答曰「誠如舅言,要不亡也。」訥中弟染干粗暴,忌帝,常圖為逆。每為皇姑遼西公主擁護,故染干不得肆共禍心。諸部大人請訥兄弟,求舉道武為主,染干不從。遂與諸大人勸進,道武登代王位于牛川。



 及帝討吐突鄰部,訥兄弟遂懷異圖,率諸部救之。帝擊之,大潰,訥西遁。衛辰遣子直力鞮征訥,告急請降。道武簡精騎二十萬救之,遂徙訥部落及諸弟,處之東界。



 訥又通於慕容垂,垂以訥為歸善王。染干謀殺訥而代立,訥遂與染干相攻。垂遣子麟討之,敗染干於牛都,破訥於赤城。道武遣師救訥,麟乃引退。訥從道武平中原,拜安遠將軍。其後離散諸部,分土定居,不聽遷徙。其君長大人,皆同編戶。訥以元舅,甚見尊重,然無統領,以壽終於家。



 訥弟盧,亦從平中原,以功賜爵遼西公。帝遣盧會衛王儀伐鄴,而盧自以帝之季舅,不肯受儀節度。帝遣
 使切責之,盧遂忿恨,與儀司馬丁建構成其嫌,彌加猜忌。會道武敕儀去鄴,盧亦引歸。道武以盧為廣川太守。盧性雄豪,恥居冀州刺史王輔下,襲殺輔,奔慕容德。德以為并州刺史、廣寧王。廣固敗,盧亦沒。



 訥從父弟悅。初,道武居賀蘭部下,人情未甚附,唯悅舉部隨從。又密為帝祈禱天神,請成大業,出於誠至。帝嘉之,甚見寵待。後平中原,以功賜爵鉅鹿侯,進爵北新,卒。



 子泥襲爵,後降為肥如侯。道武崩,京師草草,泥出,舉烽於安陽城北,賀蘭部人皆往赴之。明元即位,乃罷。詔泥與元渾等八人拾遺左右。與北新侯安同持節行并、定二州,劾奏並州
 刺史元六頭等,皆伏罪,州郡肅然。後從太武征赫連昌,以功進爵為瑯邪公,軍國大議,每參豫焉。又征蠕蠕,為別道將,坐逐賊不進,詐增虜,當斬。贖為庶人。久之,拜光祿勛,為外都大官,復本爵。卒官,子醜建襲。



 姚黃眉,姚興之子,明元昭哀皇后之弟也。姚泓滅,黃眉間來歸魏。明元厚禮待之,賜爵隴西公。尚陽翟公主,拜附馬都尉,隸戶二百。太武即位,遷內都大官,後拜太常卿,卒。贈雍州刺史、隴西王,謚曰獻,陪葬金陵。黃眉寬和溫厚,希言得失,太武悼惜之,故贈禮有加。



 杜超,字祖仁,魏郡鄴人,密皇后之兄也。少有節操。泰常
 中,為相州別駕。



 始光中,太武思念舅氏,以超為陽平公,尚南安長公主,拜附馬都尉,位大鴻臚卿,車駕幸其第,賞賜巨萬。神三年,以超行征南大將軍、太宰,進爵為王,鎮鄴。



 追加超父豹鎮東大將軍、陽平景王,母曰鉅鹿惠君。真君五年,超為帳下所害,太武臨其喪,哀慟者久之。謚曰威王。長子道生賜爵城陽侯,後為秦州刺史,進爵河東公。道生弟鳳凰襲爵,加侍中、特進。太武追思超不已,欲以鳳凰為定州刺史。



 鳳凰不願違離闕庭,乃止。鳳凰弟道俊賜爵發干侯,鎮枋頭,除兗州刺史。



 超既薨,復授超從弟遺侍中、安南將軍、開府、相州刺史,入為內
 都大官,進爵廣平王。遺性忠厚,頻歷州郡,所在著稱。薨,贈太傅,謚曰宣王。



 長子元寶,位司空。元寶弟胤寶,司隸校尉。元寶又進爵京兆王。及歸而父遺喪。明當入謝,元寶欲以表聞。文成未知遺薨,怪其遲,召之。元寶將入,時人止之曰:「宜以家憂自辭。」元寶欲見其寵,不從,遂冒哀而入。未幾,以謀反伏誅,親從皆斬,唯元寶子世衝逃免。時朝議欲追削超爵位,中書令高允上表理之。後兗州故吏汲宗等,以道俊遺惠在人,前從坐爵受誅,委骸土壤,求得收葬。書奏,詔義而聽之。贈散騎常侍、安南將軍、南康公,謚曰昭。世衝襲遺公爵。



 賀迷,代人,太武敬哀皇后之從父也。皇后生景穆。初,后少孤,父兄近親唯迷,故蒙賜爵長鄉子。卒,贈光祿大夫、五原公。



 閭毗,代人,蠕蠕主大檀之親屬,太武時自其國來降。毗即恭皇后之兄也。后生文成。文成太安二年,以毗為平北將軍,賜爵河東公。弟紇為寧北將軍,賜爵零陵公。其年,並加侍中,進爵為王。毗,征東將軍,評尚書事;紇,征西將軍、中都大官。自餘子弟賜爵為王者二人,公五人,侯六人,子三人,同時受拜,所以隆崇舅氏。和平二年,追謚后祖父延襄康公,父辰定襄懿王。毗薨,贈太尉,追贈毗妻
 河東王妃。子惠襲。紇薨,贈司空。子豆,後賜名莊。太和中,初立三長,以莊為定戶籍大使,甚有時譽。十六年,例降爵。後為七兵尚書,卒。



 紇弟染,位外都大官、冀州刺史、江夏公,卒。



 先是,文成以乳母常氏有保護功,既即位,尊為保太后,後尊為皇太后。興安二年,太后前兄英,字世華,自肥如令超為散騎常侍、鎮軍大將軍,賜爵遼西公;弟喜,鎮軍大將軍、祠曹尚書、帶方公;三妹皆封縣君;妹夫王暏為平州刺史,遼東公。追贈英祖父苻堅扶風太守亥為鎮西將軍、遼西蘭公;勃海太守澄為侍中、征東大將軍、太宰、遼西獻王;英母許氏博陵郡君。遣兼太常盧
 度世持節改葬獻王於遼西,樹碑立廟,置守塚百家。太安初,英為侍中、征東大將軍、太宰,進爵為王;喜左光祿大夫,改封燕郡;從兄泰為安東將軍、朝鮮侯;子伯夫,散騎常侍、選部尚書;次子員,金部尚書;喜子振,太子庶子。三年,英領太師,評尚書事,內都大官、伏寶泰等州刺史。五年,詔以太后母宋氏為遼西王太妃。和平元年,喜為洛州刺史。



 初,英事宋不能謹,而暏奉宋甚至,就食於和龍。無車牛,宋疲不進,者負宋於笈。至是,宋於英等薄,不如暏之篤。謂太后曰:「何不王暏而黜英?」太后曰:「英為長兄,門戶主也,家內小小不順,何足追計。暏雖盡力,故
 是他姓,奈何在英上。本州郡公,亦足報耳。」天安中,英為平州刺史,為幽州刺史,伯夫進爵范陽公。英濁貨,徙敦煌。諸常自興公及至是,皆以親疏受爵賜田宅,時為隆盛。後伯夫為洛州刺史,以臟汙欺妄,徵斬於京師。承明元年,徽英復官。薨,謚遼西平王。始英之徵也,夢日墜其所居黃山下水中,村人以車牛挽致不能出,英獨抱載而歸。聞者異之。



 後員與伯夫子禽可共為飛書,誣謗朝政。事發,有司執憲,刑及五族。孝文以明太后故,罪止一門。年老,赦免歸家,恕其孫一人扶養之,經奴婢田宅。其家僮入者百人,金錦布帛數萬計,賜尚書已下宿
 衛已上。其女婿及親從在朝,皆免官歸本鄉。十一年,孝文、文明太后以文昭太后故,悉出其家前後沒入婦女,以喜子振試守正平郡,卒。



 馮熙,字晉國,長樂信都人,文明太后之兄也。祖弘,北燕王。太武平遼海,熙父朗內徙,官至秦雍二州刺史、遼西郡公,坐事誅。文明太后臨朝,追贈假黃鉞、太宰、燕宣王,立廟長安。



 熙生於長安,為姚氏魏母所養。以叔父樂陵公邈因戰入蠕蠕,魏母攜熙逃避至氐羌中撫育。年十二,好弓馬,有勇幹,氐羌皆歸附之。魏母惡其如此,將還長安,始就博士學問。從師受《孝經》、《論語》,好陰陽兵法事。
 及長,游華陰、河東二郡間。性汎愛,不拘小節,人無士庶,來則納之。



 熙姑先入掖庭,為太武左昭儀。妹為文成帝後,即文明太后也。使人外訪,知熙所在,征赴京師,拜冠軍將軍,賜爵肥如侯,尚景穆女博陵長公主,拜駙馬都尉。



 出為定州刺史,進爵昌黎王。獻文即位,為太傅,累拜內都大官。孝文即位,文明太后臨朝,帝乃承旨以熙為侍中、太師、中書監,領祕書事。熙以頻履師傅,又中宮之寵,為群情所駭,心不自安,乞轉外任。文明太后亦以為然,除都督、洛州刺史,侍中、太師如故。



 洛陽雖經破亂,而舊《三字石經》宛然猶在。至熙與常伯夫相繼為州,廢毀
 分用,大至頹落。熙為政不能仁厚,而信佛法。自出家財在諸州鎮建佛圖精舍,合七十二處。寫十六部一切經,延致名德沙門,日與講論,精勤不倦,所費亦不貲。而營塔寺多在高山秀阜,傷殺人牛。有沙門勸止之,熙曰:「成就後,人唯見佛圖,焉知殺人牛也。」其北芒寺碑文,中書侍郎賈元壽詞。孝文頻登北芒寺,親讀碑文,稱為佳作。熙為州,因取人子女為奴婢,有容色者幸之為妾,有子女數十人,號為貪縱。



 後授內都大官,太師如故。熙事魏母孝謹,如事所生。魏母卒,乃散髮徒跣,水漿不入口三日。詔不聽服,熙表求依趙氏之孤。帝以熙情難奪,聽服
 齊衰期。後以例降,改封京兆郡公。



 帝納其女為后,曰:「《白武通》云:王所不臣,數有三焉。妻之父母,抑言其一。此所謂供承宗廟,不欲奪私心。然吾季著於《春秋》,無臣證於往牒,既許通體之一,用開至尊之敬。比長秋配極,陰政既敷,未聞有司,陳奏斯式。可詔太師,輟臣從禮。」又勒集書造儀付外。孝文前後納熙三女,二為后,一為左昭儀。



 由是馮氏寵貴益隆,賞賜累巨萬。帝每詔熙上書不臣,入朝不拜,熙上書如舊。



 熙於後遇疾,綿寢四載。詔遣監問,道路相望,車駕亦數幸焉。將遷洛,帝親與熙別,見其困篤,歔欷流涕。密敕宕昌公主遇曰:「太師萬一,即可監
 護喪事。」



 十九年,薨於代。車駕在淮南,留臺表聞,還至徐州,乃舉哀,為制緦服。詔有司預辨凶儀,并開魏京之墓,令公主之柩,俱向伊洛。凡所營送,皆公家為備。又敕代給彩帛,前後六千匹,以供凶用。皇后詣代都赴哭,太子恂亦赴代哭弔。將葬,贈假黃鉞、侍中、都督十州諸軍事、大司馬、太尉、冀州刺史,加黃屋、左纛,備九錫、前後部羽葆鼓吹,皆依晉太宰、安平獻王故事。有司奏謚,詔曰:「可以威彊恢遠曰武,奉謚於公。」柩至七里澗,帝服縗往迎,叩靈悲慟而拜焉。葬日,送臨墓所,親作誌銘。



 主生二子,誕、脩。



 誕字思正,脩字寶業,皆姿質妍麗。年纔十餘,文明
 太后俱引入禁中,申以教誡。然不能習讀經史,兄弟並無學術,徒整飾容儀,寬雅恭謹而已。誕與孝文同歲,幼侍書學,仍蒙親待,尚帝妹樂安長公主,拜駙馬都尉、侍中、征西大將軍、南平王。脩侍中、鎮北大將軍、尚書、東平公。又除誕儀曹尚書,知殿中事。及罷庶姓王,誕為侍中、都督中外諸軍事、中軍將軍、特進,改封長樂郡公。誕拜官,孝文立於庭,遙受其拜,既訖還室。脩降為侯。



 誕、脩雖並長宮禁,而性趣乖別。誕性淳篤,脩乃浮競。誕亦未能誨督其過,然時言於太后。孝文嚴責之,至於楚捶。由是陰懷毒恨,遂結左右有憾於誕者,求藥,欲因食害誕。事
 覺,帝自詰之,具得情狀。誕引過謝,乞全脩命。帝以誕父老,又重其意,不致於法,撻之百餘,黜為平城百姓。脩妻,司空穆亮女也,求離婚,請免官。帝引管、蔡事,皆不許。



 帝寵誕,仍作同輿而載,同案而食,同席坐臥。彭城王勰、北海王詳雖直禁中,然親近不及。十六年,以誕為司徒。帝既愛誕,除官日,親為制三讓表,并啟。將拜,又為其章謝。尋加車騎大將軍、太子太師。十八年,帝謂其無師傅獎導風,誕深自誨責。從駕南伐,十九年,至鐘離。誕遇疾,不能侍從,帝日省問,醫藥備加。



 帝銳意臨江,乃命六軍發鐘離南轅,與誕泣訣。左右皆入,無不掩涕。時誕已惙然,
 彊坐視帝,悲而淚不能下,言「夢太后來呼臣「。帝嗚咽,執手而出,遂行。是日,去鐘離五十里許,昏時,告誕薨問,帝哀不自勝。時崔慧景、裴叔業軍在中淮,去所次不過百里,帝乃輕駕西還,從者數千人,夜至誕薨所。拊屍哀慟,若喪至戚,達旦聲淚不絕。從者亦迭舉音。帝以所服衣幍充襚,親自臨視,徹樂去膳,宣敕六軍,止臨江之駕。帝親北度,慟哭極哀。喪至洛陽,車駕猶在鐘離。詔留守賜賻物布帛五千匹、穀五千斛,以供葬事。贈假黃鉞、使持節、大司馬,領司徒、侍中、都督,太師、駙馬、公如故。加以殊禮,備錫九命,依晉大司馬、齊王攸故事。有司奏謚,詔曰:「
 案謚法,主善行德曰元,柔剋有光曰懿。昔貞惠兼美,受三謚之榮;忠武雙徽,錫兩號之茂。式準前訓,宜契具瞻。既自少綢繆,知之惟朕,案行定名,謚曰元懿。」帝又親為作碑文及挽歌詞,皆窮美盡哀,事過其厚。車駕還京,遂親至誕墓,停車而哭。使彭城王勰詔群官脫朱衣,服單衣介幘而哭司徒,貴者示以朋友,微者示如僚佐。公主貞厚有禮度,產二男。



 長子穆,字孝和,襲熙爵,避皇子愉封,改封扶風郡公。尚孝文女順陽長公主,拜駙馬都尉,歷員外通直散騎常侍。穆與叔輔興不和。輔興亡,贈相州刺史,祖載在庭,而穆方高車良馬,恭受職命,言宴滿
 堂,忻笑自若,為御史中尉、東平王匡所劾。後位金紫光祿大夫,遇害河陰,贈司空、雍州刺史。子冏,字景昭,襲爵昌黎王。尋以庶姓罷王,仍襲扶風郡公。子峭,字子漢,齊受禪,例降。



 穆弟顥,襲父誕長樂郡公。



 脩弟聿,字寶興,廢後同產兄也。位黃門郎、信都伯。後坐妹廢,免為長樂百姓。宣武時,卒於河南尹。



 聿同產弟風,幼養於宮,文明太后特加愛念。數歲賜爵至北平王,拜太子中庶子,出入禁闥,寵侔二兄。孝文親政後,恩寵稍衰,降爵為侯。幽后立,乃復敘用。



 后死,亦冗散。卒,贈青州刺史。



 崔光之兼黃門也,與聿俱直。光每謂之曰:「君家富貴大盛,終必衰敗。」
 聿云:「我家何負四海,乃咒我也!」光云:「以古推之,不可不慎。」時熙為太保,誕司徒、太子太傅,脩侍中、尚書,聿黃門,廢后在位,禮愛未弛。是後歲餘,脩以罪棄,熙、誕喪亡,后廢,聿退。時人以為盛必衰也。



 李惠,中山人,思皇后之父也。父蓋,少知名,歷位殿中都官二尚書、左將軍、南郡公。初,太武妹武威長公主,故涼王沮渠牧犍之妻。太武平涼州,頗以公主通密計之助,故寵遇差隆,詔蓋尚焉。蓋妻與氏以是出。後蓋加侍中、駙馬都尉、殿中都官尚書、右僕射。卒官,贈征南大將軍、定州刺史、中山王,謚曰莊。



 惠弱冠襲父爵,妻襄城王韓
 頹女,生二女,長即后也。惠歷位散騎常侍、侍中,征西大將軍、秦益二州刺史,進爵為王。轉雍州刺史、征南大將軍,加長安鎮大將。



 惠長於思察。雍州事,有燕爭巢,斗已累日,惠令人掩獲,試命綱紀斷之,並辭。惠乃使卒以弱竹彈兩燕,既而一去一留,惠笑謂吏屬曰:「此留者自計為巢功重,彼去者既經楚痛,理無固心。」群下伏其深察。人有負鹽負薪者,同釋重簷息樹陰。二人將行,爭一羊皮,各言藉背之物。惠遣爭者出,顧州綱紀曰:「此羊皮可拷知主乎?」群下咸無答者。惠令人置羊皮席上,以杖擊之,見少鹽屑,「得其實矣。」使爭者視之,負薪者乃伏
 而就罪。凡所察究,多如此類,由是吏人莫敢欺犯。後為開府儀同三司、青州刺史,王如故。歷政有美績。



 惠素為文明太后所忌。誣惠將南叛,誅之。惠二弟初、樂與惠諸子同戮。後妻梁氏亦死青州,盡沒其家財。惠本無釁故,天下冤惜焉。



 惠從弟鳳為定州刺史、安縣王長樂主簿。後長樂以罪賜死,時卜筮者河間邢瓚辭引鳳,云長樂不軌,鳳為謀主,伏誅。唯鳳弟道念與鳳子及兄弟之子皆逃免,後遇赦乃出。太和十二年,孝文將爵舅氏,詔訪存者。而惠諸從以再離孥戮,難於應命。唯道念敢先詣闕,乃申后妹及鳳兄弟子女之存者。於是賜鳳子屯爵
 柏人侯,安祖浮陽侯,興祖安喜侯,道念真定侯,從弟寄生高邑子,皆加將軍。十五年,安祖昆弟四人,以外戚蒙見。詔謂曰:「卿之先世,內外有犯,得罪於時。然官必用才,以親非興邦之選。外氏之寵,超於末葉。從今已後,自非奇才,不得復外戚謬班抽舉。既無殊能,今且可還。」後例降爵,宜祖等改侯為伯,並去軍號。



 帝奉馮氏過厚,於李氏過薄,舅家了無敘用,朝野人士,所以竊議。太常高閭,顯言于禁中。及宣武寵隆外家,並居顯位。乃惟孝文舅氏,存已不霑恩澤。景明末,特詔興祖為中山太守。正始初,詔追崇惠為使持節、驃騎將軍、開府儀同三司、定州
 刺史、中山公。太常考行,上言:案謚法,武而不遂曰壯,謚曰壯公。



 興祖自中山遷燕州刺史,卒。以兄安祖子侃晞為後,襲先封南郡王。後以庶姓罷王,改為博陵郡公。侃晞為莊帝所親幸,拜散騎常侍、嘗食典御。帝之圖爾朱榮,侃晞與魯安等持刃於禁內殺榮。及莊帝蒙塵,侃晞奔梁。



 高肇,字首文,文昭皇太后之兄也。自云本勃海蓚人。五世祖顧,晉永嘉中,避亂入高麗。父颺,字法脩。孝文初,與弟乘信及其鄉人韓內、冀富等入魏,拜厲威將軍、河間子;乘信明威將軍,俱待以客禮。遂納颺女,是為文昭皇
 后,生宣武。



 颺卒,景明初,宣武追思舅氏,徵肇兄弟等。錄尚書事、北海王詳等奏,颺宜贈左光祿大夫,賜爵勃海公,謚曰敬。其妻蓋氏,宜追封清河郡君。詔可。又詔颺嫡孫猛襲勃海公爵,封肇平原郡公,肇弟顯澄城郡公,三人同日受封。始宣武未與舅氏相接,將拜爵,乃賜衣幘,引見肇、顯於華林都亭。皆甚惶懼,舉動失儀,數日之間,富貴赫奕。是年,咸陽王禧誅,財物珍寶、奴婢、田宅多入高氏。未幾,肇為尚書右僕射、冀州大中正,尚宣武姑高平公主,遷尚書令。肇出自夷土,時望輕之。



 及在位居要,留心百揆,孜孜無倦,世咸謂之為能。宣武初,六輔專政,
 後以咸陽王禧無事構逆,由是委肇。肇既無親族,頗結朋黨,附之者旬月超升,背之者陷以大罪。以北海王詳位居其上,構殺之。又說宣武防衛諸王,殆同囚禁。時順皇后暴崩,世議言肇為之。皇子昌薨,僉謂王顯失於醫療,承肇意旨。及京兆王愉出為冀州刺史,畏肇恣擅,遂至不軌。肇又譖殺彭城王勰。由是朝野側目,咸畏惡之。因此專權,與奪任己。又嘗與清河王懌於雲門外廡下,忽忿諍,大至紛紜。太尉、高陽王雍和止之。高后既立,逾見寵信。肇既當衡軸,每事任己,本無學識,動違禮度。好改先朝舊制,減削封秩,抑黜勛人,由是怨聲盈路矣。



 延
 昌初,遷司徒。雖貴登台鼎,猶以去要怏怏,眾咸嗤笑之。父兄封贈雖久,竟不改瘞。三年,乃詔令還葬。肇不自臨赴,唯遣其兄子猛改服詣代,遷葬於鄉。



 時人以肇無識,哂而不責也。及大舉征蜀,以肇為大將軍、都督諸軍,為之節度。



 與都督甄琛等二十餘人,俱面辭宣武於東堂,親奉規略。是日,肇所乘駿馬,停於神獸門外,無故驚倒,轉臥渠中,鞍具瓦解,眾咸怪異。肇出,惡焉。



 四年,宣武崩,赦,罷征軍。明帝與肇及征南將軍元遙等書,稱諱言以告凶問。



 肇承變,非唯仰慕,亦憂身禍,朝夕悲泣,至于羸悴。將至,宿瀍澗驛亭,家人夜迎省之,皆不相視。直至闕
 下,縗服號哭,昇太極殿,盡哀。太尉高陽王先居西柏堂,專決庶事,與領軍于忠,密欲除之。潛備壯士直寢邢豹、伊盆生等十餘人於舍人省下。肇哭梓宮訖,於百官前引入西廓,清河王懌、任城王澄及諸王等皆竊言目之。肇入省,壯士扼而拉殺之,下詔暴其罪惡,稱為自盡。自餘親黨,悉無追問。



 削除職爵,葬以士禮。逮昏,乃於廁門出其尸歸家。初肇西征,行至函谷,車軸中折,從者皆以為不獲吉還也。靈太后臨朝,令特贈營州刺史。永熙二年,孝武帝贈使持節、侍中、中外諸軍事、太師、大丞相、太尉公、錄尚書事、冀州刺史。



 肇子植,自中書侍郎為濟州
 刺史,率州軍討破元愉別將有功,當蒙封賞。不受。



 云:「家荷重恩,為國致效,是其常節,何足以膺進陟之報?」懇惻發於至誠。歷青、相、朔、恆四州刺史,卒。植頻蒞五州,皆清能著稱,當時號為良刺史。贈安北將軍、冀州刺史。



 肇長兄琨,早卒,襲颺封勃海郡公,贈都督五州諸軍事、鎮東大將軍、冀州刺史。詔其子猛嗣。



 猛字豹兒,尚長樂公主,即宣武同母妹也。拜駙馬都尉,歷位中書令,出為雍州刺史,有能名。入為殿中尚書,卒。贈司空、冀州刺史。孝武帝時,復贈太師、大丞相、錄尚書事。公主無子,猛先在外有男,不敢令主知,臨終方言之,年歲三十矣。乃召為喪
 主。尋卒,無後。



 琨弟偃,字仲游。太和十年,卒。正始中,贈安東將軍、都督、青州刺史,謚曰莊侯。景明四年,宣武納其女為貴嬪,及于順皇后崩,永平元年,立為皇后。二年,八坐奏封后母王氏為武邑郡君。



 偃弟壽,早卒。壽弟即肇也。肇弟顯,侍中、高麗國大中正,早卒。



 胡國珍,字世玉,安定臨涇人也。祖略,姚興勃海公姚逵平北府諮議參軍。父深,赫連屈丐給事黃門侍郎。太武剋統萬,深以降款之功,賜爵武始侯。後拜河州刺史。



 國珍少好學,雅尚清儉。太和十五年襲爵,例降為伯。女以選入掖庭,生明帝,即靈太后也。孝明帝踐祚,以國珍為
 光祿大夫。靈太后臨朝,加侍中,封安定郡公。



 追崇國珍妻皇甫氏為京兆郡君,置守塚十戶。尚書令、任城王澄奏,安定公宜出入禁中,參諮大務。詔屈公入決萬機。尋進位中書監、儀同三司,侍中如故。賜絹,歲八百疋,妻梁四百匹,男女姊妹各有差。國珍與太師高陽王雍、太傅清河王懌、太保廣平王懷入居門下,同釐庶政。詔依漢車千秋、晉安平王故事,給步挽一乘,自掖門至于宣光殿,得以出入,并備幾杖。後與侍中崔光,俱授帝經,侍直禁中。



 國珍上表陳刑政之宜,詔皆施行。



 延和初,加國珍使持節、都督、雍州刺史,驃騎大將軍開府。靈太后以國
 珍年老,不欲令其在外,且欲示以方面之榮,竟不行。遷司徒公,侍中如故。就宅拜之。



 靈太后、明帝率百僚幸其第,宴會極歡。又追京兆郡君為秦太上君。太上君景明三年薨於洛陽,於此十六年矣。太后以太上君墳瘞卑局,更增廣,為起塋域門闕碑表。



 侍中崔光等奏:「按漢高祖母始謚曰昭靈夫人,後為昭靈后;薄太后母曰靈文夫人,皆置園邑三百家,長丞奉守。今秦太上君未有尊謚,陵寢孤立,即秦君名,宜上終稱,兼設掃衛,以慰情典。請上尊謚曰孝穆,權置園邑三十戶,立長丞奉守。」太后從之。封國珍繼室梁氏為趙平郡君。元叉妻拜為女侍
 中,封新平郡君,又徙封馮翊君。國珍子祥妻長安縣公主,即清河王懌女也。



 國珍年雖篤老,而雅敬佛法,時事潔齊,自禮拜。至於出入侍從,猶能跨馬據鞍。神龜元年四月七日,步從所建佛像,發第至閶闔門四五里。八日,又立觀像,晚乃肯坐。勞熱增甚,因遂寢疾。靈太后親侍藥膳,十二日薨,年八十。給東園溫明祕器,五時朝服各一具,衣一襲,贈布五千匹,錢一百萬,蠟千斤。大鴻臚持節監護喪事。太后還宮,成服於九龍殿,遂居九龍寢室。明帝服小功服,舉哀於太極東堂。又詔,自始薨至七七,皆為設千僧齋,齋令七人出家,百日設萬人齋,二七人
 出家。先是巫覡言將有凶,勸令為厭勝法,國珍拒而不從,云吉凶有定分,唯脩德以禳之。臨死,與太后訣,云「母子善臨天下」,殷勤至於再三。又及其子祥云,「我唯有一子,死後勿如比來威抑之」。靈太后以其好戲,時加威訓,國珍故以為言。



 始國珍欲就祖、父,西葬舊鄉。後緣前世諸胡多在洛葬,有終洛之心。崔光嘗對太后前問國珍:「國公萬年後,為在此安厝?為歸長安?」國珍言:「當陪葬天子山陵。」及病危,太后請以後事,竟言還安定。語遂惛忽。太后問清河王懌與崔光等,議去留。懌等皆以病亂,請從先言。太后猶記崔光昔與國珍言,遂營墓於洛陽。太
 后雖外從眾議,而深追臨終之語,云:「我公之遠慕二親,亦吾之思父母也。」



 追崇假黃鉞、使持節、侍中、相國、都督中外諸軍事、太師,領太尉公、司州牧,號太上秦公,加九錫,葬以殊禮。給九旒鑾輅,武賁班劍百人,前後部羽葆鼓吹,巉輬車,謚曰文宣公。賜物三千段,粟一千五百石。又詔贈國珍祖父、父,兄下逮從子,皆有封職。持節就安定監護喪事。靈太后迎太上君神柩還第,與國珍俱葬,贈襚一與國珍同。及國珍神主入廟,詔太常權給以軒縣之樂,六佾之舞。



 初,國珍無男,養兄真子僧洗為後。後納趙平君,生子祥,字元吉,襲封。故事,世襲例皆減邑,唯
 祥獨得全封。趙平君薨,給東園祕器,明帝服小功服,舉哀於東堂,靈太后服齊衰期。葬於太上君墓左,不得祔合。祥歷位殿中尚書、中書監、侍中,改封平涼郡公。薨,贈開府儀同三司、雍州刺史,謚曰孝景。



 僧洗字湛輝,封爰德縣公,位中書監、侍中,改封濮陽郡公。僧洗自永安後廢棄,不預朝政。天平四年,薨。詔給東園秘器,贈太師、太尉公、錄尚書事、雍州刺史,謚曰孝。



 真長子寧,字惠歸,襲國珍先爵,改為臨涇伯,後進為公。歷岐涇二州刺史,卒,謚曰孝穆。女為清河王亶妃,生孝靜皇帝。武定初,贈太師、太尉公、錄尚書事,謚曰孝昭。



 子虔,字僧敬。元叉之廢
 靈太后,虔時為千牛備身,與備身張車渠等謀叉。事發,叉殺車渠等,虔坐遠徙。靈太后反政,徵為吏部郎中。太后好以家人禮與親族宴戲,虔常致諫,由是,後宴謔多不預焉。出為涇州刺史,封安陽縣侯。興和三年,以帝元舅,超遷司空公。薨,贈太傅、太尉公、尚書僕射、徐州刺史,謚曰宣。葬日,百官會葬,乘輿送於郭外。子長粲。



 長粲仕齊,累遷章武太守,為政清靜,頗得人和。除兼并省尚書左丞,當官正色,無所回避。尚書左僕射趙彥深密勿樞要,中書舍人裴澤便蕃左右,以殿門受拜,皆彈糾之。彥深等頗有恨言,長粲不以介意。後主踐祚,長粲被敕,與
 黃門馮子琮出入禁中,專典敷奏。武成還鄴,後主在晉陽,長粲仍受委留後。後主從武成還鄴,仍敕在京省判度支尚書,監議五禮。武成崩,與領軍婁定遠、錄尚書趙彥深、左僕射和士開、高文遙、領軍綦連猛、高阿那肱、右僕射唐邕,同知朝政,時人號為八貴。於後定遠、文遙並出,唐邕專典外兵,綦連猛、高阿那肱別總武任,長粲常在左右,兼宣詔令。從幸晉陽。後主既富於春秋,庶事皆相歸委。長粲盡心毗奉,甚得名譽。又正為侍中。丁母憂,給假馳驛奔喪。尋有詔,起復前任。隴東王長仁心欲入處機要之地,為執政不許。長仁疑長粲通謀,大以為恨,
 言於太后,發其陰私,請出為州。太后為言於後主,不獲已,從焉。除趙州刺史。及辭,眷戀流涕,後主亦憫然慰勉之。至州,存心政事,為人吏所懷。因沐髮,手不得舉,失瘖,卒於州。



 後主聞而傷悼,在朝文武嗟嘆,咸惜之。贈司空公、尚書左僕射、瀛州刺史,謚文貞公。



 長粲性溫雅,在官清潔。但始居要密,便為子叔泉取清河王崔德儉女為妻。在晉陽處分,用妻弟王逖與德儉對為司徒主簿,時論以此譏之。又性好內。有一侍婢,其妻王驕妒,手刺殺之,為此忿恨,數年不相見。親表為之語曰:「自我不見,于今三年。」後納妾李氏,仍與王氏別宅,亦無朝拜之禮。嫠
 婦公孫氏也,已殺三夫,長粲不信,彊取之,令與李氏同住,未期而亡。子仲操,位陳留太守。次叔泉,通直散騎侍郎。



 先是,望氣者上言,太白食昴,法當大赦。和士開奏聞,詔降罪人以應之。尚書左僕射徐之才諳練往事,語士開曰:「天垂象,見吉凶,有成災者,有不成災者。



 案昴,趙分,或云趙地有災。古者,王侯各在封邑,故分野有災,當其君長。今吾等虛名,竟不之國。刺史專令一境,善惡所歸,比來多以刺史為驗。」未幾而長粲死焉。



 寧弟盛,字歸興,位左衛將軍,賜爵江陽男。歷幽、瀛二州刺史,為政清靜,人吏愛之。轉冀州刺史,卒,賜司徒公、錄尚書事、定州刺
 史,追封陽平郡公,謚曰懿穆。明帝後納其女為皇太后。



 太后舅皇甫集,字元會,一字文都,安定朝那人。封涇陽縣公,位儀同三司、雍州刺史、右衛大將軍,贈侍中、司空公,謚曰靜。



 集弟度,字文亮,封安縣公,累遷尚書左僕射,領左衛將軍。度頑蔽,每與人言,自稱僕射,時人方之毛嘉。正光初,元叉出之為都督、瀛州刺史。度不願出,頻表固辭,乃除右光祿大夫。孝昌元年,為司空、領軍將軍,加侍中。元叉之見出也,恐朝夕誅滅,度與妻陳氏,多納其貨,為之左右。度無子,養兄集子子熙為子。



 子熙嫂趙郡太守裴佗女。佗還京師,度問佗外何消息,佗曰:「行路所聞,
 唯道明公多取元叉金帛,遠近無不慨歎。公宜戮此罪人,以謝天下。」陳氏聞而惡之。又攝吏部事,遷司徒,兼尚書令,不拜。尋轉太尉,孜孜營利,老而彌甚。遷授之際,皆自請乞。靈太后知其無用。以舅氏,難違之。然所歷官,最為貪蠹。爾朱榮入洛,西奔兄子華州刺史邕,尋與邕為人所殺。



 楊騰,弘農人,文帝之舅也。父貴,瑯邪郡守,封華陰男。騰妹為京兆王愉妃,故騰得處貴游。景明初,襲爵。後為襄城太守,甚有聲稱。文帝即位,位開府儀同三司,出鎮河東。薨,贈司空、雍州刺史,謚曰貞襄。子盛。



 乙弗繪,河南洛陽人,文帝皇后之兄也。文帝即位,位開府儀同三司、侍中、中書監、魏昌縣公。又為吏部尚書。



 趙猛,太安狄那人也。姊為齊文穆皇帝繼室,生趙郡公琛。猛性方直,頗有器幹。齊神武舉義,以預義勳,封信都縣伯。累遷南營州刺史。卒,贈司空公。



 胡長仁,字孝隆,安定臨涇人,齊武成皇后長兄也。父延之,魏中書令、兗州刺史。大寧中,贈司空公。



 長仁以內戚,歷位尚書左僕射、尚書令。及武成崩,預參朝政,封隴東郡王。



 左丞酈孝裕、郎中陸仁惠、盧元亮厚相結託。長仁每上省,孝裕必方駕而來。省務既繁,簿案堆積,令史欲
 諮都坐者,日有百數。孝裕屏人私話,朝退亦相隨。仁惠、元亮又伺閑而往,停斷公事,人號為三佞。長仁私遊仄密,處處追尋。孝裕勸其求進,和士開深疾之,於是奏除孝裕為章武郡守,元亮為淮南郡守,仁惠為幽州長史。



 孝裕又說長仁曰:「王陽臥疾,和士開必來,因而殺之。入見太后,不過百日失官,便代其處。」士開知其謀,更徙孝裕為北營州建德郡守。長仁每干執事,求為領軍。



 將相文武以主上富於春秋,母后家不可專政,故抑而不許。以本官攝選。長仁性好威福,意猶未盡。先是尚書胡長粲奏事內省,長仁疑粲間己,苦請太后出之。



 天統五年,
 從駕自並還鄴,夜發滏口,帝以夜漏尚早,停於路傍。長仁後來,謂是從行諸貴,遂遣門客程牙馳騎呼問。帝遣中尚食陳德信問是何人,牙不答而走。



 帝命左右追射之。既而捉獲,因令壯士撲之,決馬鞭二百,牙一宿便死。士開因此,遂令德信列長仁倚親驕豪無畏憚。由是,除齊州刺史。及辭於昭陽,列仗引見,長仁不敢發語,唯泣涕橫流。到任,啟求暫歸,所司不為奏。怨憤,謀令冀州人李揩墻刺和士開,其弟長咸告之。士開密與祖孝徵議之,孝徵引漢文帝殺薄昭為故事,於是敕遣張固、劉桃枝馳驛詣齊州,責長仁謀害宰輔,遂賜死。



 先是,太白食
 昴,占者曰:「昴為趙分,不利胡王。」長仁未幾死。長仁性好歌舞,飲酒至數斗不亂。自至齊州,每進酒後,必長歎欷歔,流涕不自勝,左右莫不怪之。



 尋而後主納長仁女為后,重加贈。長仁子君璧,襲爵隴東王。君璧弟君璋,及長仁弟長雍等,前後七人並賜爵,合門貴盛。后廢後,稍稍黜退焉。



 隋文帝外家呂氏,其族蓋微。平齊後求訪,不知所在。開皇初,濟南郡上言,有男子呂永吉,自稱有姑字苦桃,嫁為楊諱妻。勘驗,知是舅子。始追贈外祖雙周為上柱國、太尉、八州諸軍事、青州刺史,封齊郡公,謚曰敬。外祖母
 姚氏為齊敬公夫人。詔並改葬,於齊州立廟,置守冢十家,以永吉襲爵,留在京師。及大業中,授上黨郡太守。性識庸劣,職務不理。後去官,不知所終。



 從父道貴,性尤頑騃,言詞鄙陋。初自鄉里徵入長安,上見之悲泣。道貴略無戚容,但連呼帝名云:「種未定不可偷,大似苦桃姊。」後數犯忌諱,動致違忤。



 上甚恥之,乃命高熲厚加供給,不許接對朝士。拜上儀同三司,出為濟南太守,令即之任,斷其入朝。道貴還至本郡,高自崇重,每與人言,自稱皇舅。數將儀衛,出入閭里,從故人游宴,庶僚咸苦之。後郡廢,終於家,子孫無聞焉。



 論曰:三五哲王,防深慮遠;舅甥之國,罕執鈞衡;母後之家,無聞傾敗。爰及漢晉,顛覆繼軌,皆由乎進不以禮,故其斃亦速。自魏至隋,時移四代,得失之跡,斯文可睹。茍不傾宗,終致亡國,周隋之際,可為鑒焉。若使開皇創業,不取懲於已往,獨孤權侔呂、霍,必敗於仁壽之前;蕭氏勢均梁、竇,豈全於大業之後。



 今或不隕舊基,或更隆先構,豈非處之以道,遠權之所致乎?



\end{pinyinscope}