\article{卷三十六列傳第二十四}

\begin{pinyinscope}

 薛辯五
 世孫端端子胄端從子浚端從祖弟湖湖子聰聰子孝通孝通子道衡聰弟子善善弟慎薛寘薛憕薛辯,字允白,河東汾陰人也。曾祖興,晉尚書右僕射、冀州刺史、安邑公,謚曰:莊。祖濤襲爵,位梁州刺史,謚曰忠惠。京都傾覆,皆以義烈著聞。父強,字威明,幼有大志,懷軍國籌略。與北海王猛,同志友善。及桓溫入關中,猛以巾褐謁之。溫曰:「江東無卿比也,秦國定多奇士,如生輩尚有幾人?吾欲與之俱南。」



 猛曰:「公求可與撥亂濟時者,友人薛威明其人也。」溫曰:「聞之久矣。」方致朝命。強聞之,自商山來謁,與猛皆署軍謀祭酒。強察溫有大志而無成功,乃勸猛止。俄而溫敗。及苻堅立,猛見委任。其平陽公融為書,
 將以車馬聘強。猛以為不可屈,乃止。及堅如河東伐張平,自與數百騎馳至強壘下,求與相見。強使主簿責之。因慷慨宣言曰:「此城終無生降之臣,但有死節之將
 耳。」堅諸將請攻之,堅曰:「須吾平晉,自當面縛。舍之以勸事君者。」後堅伐晉,軍敗,強遂總宗室強兵,威振河輔,破慕容永於陳川。姚興聞而憚之,遣使重加禮命,徵拜右光祿大夫、七兵尚書,封馮翊郡公,轉左戶尚書。年九十八,卒。贈輔國大將軍、司徙公,謚曰宣。



 辯幼而俊爽,俶儻多大略,由是豪傑多歸慕之。強卒,復襲統其營。仕姚興,歷太子中庶子、河
 北
 太守。辯知姚氏運衰,遂棄歸家保鄉邑。及晉將劉裕平姚泓,即署相國掾。尋除平陽太守,委以北道鎮捍。及長安失守。辯遂歸魏。仍立功於河際,位平西將軍、東雍州刺史,賜爵汾陰侯。其年詣闕,明元深加器重,明年方得旋鎮。帝謂之曰:「朕委卿西
 蕃,志在關右,卿宜克終良算,與朕為長安主人。」



 辯既還任,務農教戰。恆以數千之眾,摧抗赫連氏。帝甚褒獎之。又除并州刺史,徵授大羽真。泰常七年,卒於官。帝以所圖未遂,深悼惜之。贈并、雍二州刺史。



 子謹,字法順。容貌魁偉,高才博學。隨劉裕度江,位府記室參軍。辯將歸魏,密報謹,謹遂亦來奔。授河東太守,後襲爵汾陰侯。始光三年,與宜都王奚斤共討赫連昌,禽其東平公乙兜,剋蒲阪。遂以新舊百姓并為一郡,除平西將軍,復為太守。神蒨三年,除使持節、秦州刺史。山胡白龍憑險作逆,太武詔南陽公奚眷與謹並為都將,討平之,封涪陵郡公。
 太延初,征吐沒骨,平之。謹自郡遷州,威恩兼被,風化大行。時兵荒之後,儒雅道息,謹命立庠序,教以詩書。三農之暇,悉令受業,躬巡邑里,親加考試,河汾之地,儒道更興。真君元年,徵授內都坐大官,輔政。深見賞重,每訪以政道,車駕臨幸者前後數四。後從駕北討,與中山王辰等後期,見殺。尋贈鎮西將軍、秦雍二州刺史,謚曰元公。



 長子初古拔,一曰車轂拔,本名洪祚,太武賜名焉。沈毅有器識。弱冠,司徙崔浩見而奇之。真君中,蓋吳擾動關右,薛永宗屯據河側,太武親討之。詔拔糾合宗鄉,壁於河際,斷二寇往來之路。事平,除中散,賜爵永康侯。太武
 南討,以拔為都將,從駕臨江而還。又共陸真討反氐仇傉檀、強免生,平之。皇興三年,除散騎常侍,尚文成女西河長公主,拜駙馬都尉。其年,拔族叔徐州刺史安都據城歸順,敕拔詣彭城勞迎,除南豫州刺史。延興二年,除鎮西大將軍、開府儀同,進爵平陽公。三年,拔與南兗州刺史游明根、南平太守許含等,以善政徵詣京師。獻文親自勞勉,復令還州。太和六年,改爵河東公。卒,贈左光祿大夫,謚曰康。



 長子胤,字寧宗。少有父風。弱冠,拜中散。襲爵鎮西大將軍、河東公,除懸瓠鎮將。尋授持節、義陽道都將。後除立忠將軍、河北太守。郡帶山河,俗多盜賊。



 有韓、馬兩姓各二千餘家,恃強憑險,最為狡害,劫掠道路,侵暴鄉閭。胤至郡,即收其姦魁二十餘人,一時戮之。於是群盜懾氣,郡中清肅。卒於郡,謚曰敬。



 子裔,字豫孫,襲爵。性豪爽,盛營園宅,賓客聲伎,以恣嬉游。卒於洛州刺史。子孝紳襲爵,位太中大夫。孝紳立行險薄,坐事為河南尹元世俊所劾,死。後贈華州刺史。



 拔弟洪隆,字菩提,驎位河東太守。長子驎駒,好讀書,舉秀才,除中書博士。



 齊使至,詔驎駒兼主客郎以接之。卒,贈河東太守,謚曰宣。始拔尚西河主,有賜田在馮翊,驎駒徙居之。遂家於馮翊之夏陽。



 長子慶之,字慶集。頗有學業,閑解几案,位
 廷尉丞。廷尉寺鄰北城,曾夏日寺傍得一狐,慶之與廷尉正博陵崔纂,或以城狐狡害,宜速殺之;或以長育之月,宜待秋分。二卿裴延俊、袁翻,互有同異。雖曰戲謔,詞義可觀,事傳於世。後兼左丞,為并、肆行臺,賜爵龍丘子,行滄州刺史。為葛榮攻圍,城陷。尋患,卒,贈華州刺史。



 慶之弟英集,性通率。隨舅李崇在揚州,以軍功累至書侍御史、通直散騎常侍,卒。英集子端。



 端字仁直,本名沙陀。有志操,遭父憂,居喪合禮。與弟裕勵精篤學,不交人事。年十七,司空高乾邕辟為參軍。賜爵平陰男。端以天下擾亂,遂棄官歸鄉里。



 魏孝武西遷,
 周文令大都督薛崇禮據龍門,引端同行。崇禮尋失守,降東魏。東魏遣行臺薛脩義督乙干貴西度,據楊氏壁。與宗親及家僮等先在壁中,脩義乃令其兵逼端等東度。方欲濟河,會日暮,端密與宗室及家僮等叛之。脩義亦遣騎追,端且戰且馳,遂入石城柵,得免。柵中先有百家,端與并力固守。貴等數來慰喻,知端無降意,遂拔還河東。東魏又遣其將賀蘭懿、南汾州刺史薛琰達守楊氏壁。端率其屬,并招喻村人,多設奇兵以臨之。懿等疑有大軍,便東遁,赴船溺死者數千人。



 端收其器械,復還楊氏壁。周文遣南汾州刺史蘇景恕鎮之。降書勞問,徵
 端赴闕,以為大丞相府戶曹參軍。從禽竇泰,復弘農,戰沙苑,並有功,進爵為伯。後改封交城縣伯,累遷吏部郎中。



 端性強直,每有奏請,不避權貴。周文嘉之,故賜名端,欲令名質相副。自居選曹,先盡賢能,雖貴游子弟,才劣行薄者,未嘗升擢之。每啟周文云:「設官分職,本康時務,茍非其人,不如曠職。」周文深然之。大統十六年,軍東討,柱國李弼為別道元帥,妙簡英寮,數日不定。周文謂弼曰:「為公思得一長史,無過薛端。」弼對曰:「真才也」乃遣之。轉尚書右丞,仍掌選事。



 梁主蕭察曾獻馬瑙鐘,周文帝執之顧丞郎曰:「能擲摴蒱頭得盧者,便與鐘。」



 已經數人
 不得。頃至端,乃執摴[HT]蒱頭而言曰:「非為此鐘可貴,但思露其誠耳。」



 便擲之,五子皆黑。文帝大悅,即以賜之。



 魏帝廢,近臣有勸文帝踐極,文帝召端告之。端以為三方未一,遽正名號,示天下以不廣。請待龕翦僭偽,然後俯順樂推。文帝撫端背曰:「成我者卿也。卿心既與我同,身豈與我異。」遂脫所著冠帶袍褲並以賜之。進授吏部尚書,賜姓宇文氏。端久處選曹,雅有人倫之鑒,其所擢用,咸得其才。六官建,拜軍司馬,加侍中、驃騎大將軍、開府儀同三司,進爵為侯。



 周孝閔帝踐阼,再遷戶部中大夫,進爵為公。晉公護將廢帝,召群臣議之。端頗具同異,護不
 悅,出為蔡州刺史。為政寬惠,人吏愛之。轉基州刺史。基州地接梁、陳,事藉鎮撫,總管史寧遣司馬梁榮催令赴任。蔡州父老訴榮,請留端者千餘人。至基州未幾,卒。遺誡薄葬,府州贈遺,勿有所受。贈本官,加大將軍,進封文城郡公,謚曰質。子胄嗣。



 胄字紹玄,少聰明,每覽異書,便曉其義。常歎訓注者不會聖人深旨,輒以意辯之,諸儒莫不稱善。性慷慨,志立功名。周明帝時,襲爵文城郡公。累遷上儀同,尋拜司金大夫,後加開府。



 隋文帝受禪,三遷為兗州刺史。到官,繫囚數百。胄剖斷旬日便了,囹圄空虛。



 有陳州人向道力
 偽作高平郡守,將之官。胄遇諸塗,察其有異,將留詰之。司馬王君馥固諫,乃聽詣郡。既而悔之,即遣主簿追道力。有部人徐俱羅嘗任海陵郡守,先是已為道力偽代之。比至秩滿,公私不悟。俱羅遂語君馥曰:「向道力經賜代為郡,使君豈容疑之。」君馥以俱羅所陳,又固請胄。胄呵,君馥乃止。遂收之,道力懼而引偽。其發姦擿伏,皆此類也。時人謂為神明,先是,兗州城東沂、泗二水合而南流,泛濫大澤中。胄遂積石堰之,決令西注,陂澤盡為良田。又通轉運,利盡淮海,百姓賴之,號為薛公豐兗渠。



 胄以天下太平,遂遣博士登泰山觀古跡,撰封禪圖及儀
 上之。帝謙讓不許。轉郢州刺史,有惠政。徵拜衛尉卿,轉大理卿,持法寬平,名為稱職。遷刑部尚書。



 時左僕射高熲稍被疏忌,及王世積誅,熲事與相連,上因此欲成熲罪。胄明雪之,正議其獄。由是忤旨,械繫之,久而得免。檢校相州事,甚有能名。



 漢王諒作亂并州,遣其將綦良東略地,攻逼慈州。刺史上官政請援於胄,胄畏諒兵鋒,不敢拒。良又引兵攻胄,胄欲以計卻之,遣親人魯世範說良曰:「天下事未可知。胄為人臣,去就須得其所,何遽相攻也?」良乃釋去,進圍黎陽。及良為史祥所攻,棄軍歸胄。朝廷以胄懷貳心,鎖詣大理。相州吏人素懷其恩,詣闕
 理胄者百餘人。胄竟坐除名,配防嶺南,道卒。子筠、獻知名。



 端弟裕,字仁友。少以孝悌聞於州里。弱冠,丞相參軍事。時京兆韋夐志安放逸,不干世務。裕慕其恬靜,數載酒肴侯之,談宴終日。夐遂以從孫女妻之。裕嘗謂親友曰:「大丈夫當聖明之運,而無灼然文武之用為世所知,雖復棲遑,徙為勞苦耳。至如韋居士,退不丘壑,進不市朝,怡然守道,榮辱弗及,何其樂也。」



 裕曾宿宴於夐之廬,後庭有井,裕夜出戶,若有人欲牽其手,裕便卻行,遂落井。同坐共出之,因勸裕酒曰:「向慮卿不測夐,幸得無他,宜盡此爵。」裕曰:「墜井蓋小小耳,方當逾於此也。」人問其
 故,裕曰:「近夢,恐有兩楹之憂。」



 尋卒,文章之士誄之者數人。周文傷惜之,追贈洛州刺史。



 胄從祖弟濬,字道賾。父琰,周渭南太守。濬少孤,養母以孝聞。幼好學,有志行。周天和中,襲爵虞城侯,位新豐令。隨開皇中,歷尚書虞部、考功侍郎。帝聞濬事母孝,以其母老,賜輿服几杖、四時珍味,當世榮之。後其母疾病,濬貌甚憂瘁,親故弗之識。暨丁母艱,詔鴻臚監護喪事,歸葬夏陽。時隆冬極寒,濬衰絰徙跣,冒犯霜雪,自京及鄉,五百餘里,足凍墮指,創血流離,朝野為之傷痛。州里賵助,一無所受。尋起令視事,上見其毀瘠過甚,為之改容,
 顧群臣曰:「吾見薛濬哀毀,不覺悲感傷懷。」嗟異久之。浚竟不勝喪,病且卒。其弟謨時為晉王府兵曹參軍事,在揚州。濬遺書於謨曰:吾以不造,幼丁艱酷,窮游約處,屢絕簞瓢。晚生早孤,不聞《詩禮》。賴奉先人貽厥之訓,獲稟母氏聖善之規。負笈裹糧,不憚艱遠,從師就業,欲罷不能。



 砥行礪心,困而彌篤,用膺教義,爰至長成。自釋耒登朝,于茲二十三年矣。雖官非聞達,而祿喜逮親,庶保期頤,得終色養。何圖精誠無感,禍酷薦臻;兄弟俱被奪情,苦廬靡申哀訴。是用叩心泣血,隕氣摧魂者也。既而創鉅釁深,不勝荼毒,啟手啟足,幸及全歸。使夫死而有知,
 得從先人於地下矣,豈非至願哉?但念爾伶俜孤宦,遠在邊服,顧此悢悢,如何可言!適已有書,冀得與汝面訣,忍死待汝,已歷一旬。汝既未來,便成今古,緬然永別,為恨何言!勉之哉!勉之哉!



 書成而絕。有司以聞,文帝為之屑涕,降使齎冊書弔祭。濬性清儉,死日家無遺財。



 濬初為兒時,與宗中兒戲澗濱,見一黃蛇,有角及足。召群童共視,了無見者。



 以為不祥,歸大憂悴。母問之,以實對。時有胡僧詣宅乞食,母以告之。僧曰:「此兒之吉應。且此兒早有名位,然壽不過六七耳。」言終而出,忽然不見。後終於四十二,六七之言驗矣。子乾福,武安郡司倉書佐。



 洪隆弟湖,字破胡。少有節操,篤志於學;專精講習,不干時務;與物無競,好以德義服人。或有兄弟忿鬩,鄰里爭訟者,恐湖聞之,皆內自改悔。鄉閭化其風教,咸以敬讓為先。三召州都,再辟主簿,州將傾心致禮,並不獲己而應之。為本州中從事、別駕、除河東太守。兄弟並為本郡,當世榮之。復受詔為仇池都將。後罷郡,終於家。有八子,長子聰知名。



 聰字延智。方正有理識,善自標致,不妄游處。雖在闇室,終日矜莊,見者莫不懍然加敬。博覽墳籍,精力過人,至於前言往行,多所究悉。詞辯占對,尤是所長。遭父憂,廬
 於墓側,哭泣之聲,酸感行路。友于篤睦,而家教甚嚴;諸弟雖昏宦,恆不免杖罰,對之肅如也。未弱冠,州辟主簿。



 太和十五年,釋褐著作佐郎。于時,孝文留心氏族,正定官品。士大夫解巾,優者不過奉朝請。聰起家便佐著作,時論美之。後遷書侍卸史,凡所彈劾,不避強禦;孝文或欲寬貸者,聰輒爭之。帝每云:「朕見薛聰,不能不憚,何況諸人也?」



 自是貴戚斂手。累遷直閣將軍,兼給事黃門侍郎、散騎常侍,直閣如故。



 聰深為孝文所知,外以德器遇之,內以心膂為寄。親衛禁兵,委總管領。故終太和之世,恆帶直閣將軍。群臣罷朝之後,聰恆陪侍帷幄,言兼晝
 夜。時政得失,預以謀謨;動輒匡諫,事多聽允。而重厚沈密,外莫窺其際。帝欲進以名位,輒苦讓不受。帝亦雅相體悉,謂之曰:「卿天爵自高,固非人爵之所榮也。」又除羽林監。



 帝曾與朝臣論海內姓地人物,戲謂聰曰:「世人謂卿諸薛是蜀人,定是蜀人不?」



 聰對曰:「臣遠祖廣德,世仕漢朝,時人呼為漢。臣九世祖永,隨劉備入蜀,時人呼為蜀。臣今事陛下,是虜非蜀也。」帝撫掌笑曰:「卿幸可自明非蜀,何乃遂復苦朕。」聰因投戟而出。帝曰:「薛監醉耳。」其見知如此。



 二十三年,從駕南征,兼御史中尉。及宣武即位,除都督、齊州刺史,政存簡靜。卒於州,吏人追思,留其
 所坐榻以存遺愛。贈征虜將軍、華州刺史,謚曰簡懿侯。魏前二年,重贈車騎大將軍、儀同三司、延州刺史。子孝通最知名。



 孝通字士達。博學有俊才。蕭寶夤征關中,引參驃騎大將軍府事,禮遇甚隆。



 及寶夤將有異志,孝通悟其萌,託以拜掃求歸,乃見許。同寮咸怪,止之;但笑而不答,遽還鄉里。寶夤後果逆命。



 北海王元顥入洛,宗人薛永宗、脩義等又聚徙作亂,欲以應之。孝通與所親計曰:「北海乘虛遠入,吳兵不能久住,事必無成。今若與永宗等舉,滅族道也。」



 乃率其近親,與河東太守元襲嬰城固守。及
 寶夤平定,元顥退走,預其事者咸罹禍,唯同孝通者皆免。事寧,入洛,除員外散騎侍郎。爾朱天光鎮關右,表為關西大行臺郎中,深見任遇。關中平定,預有其力,以功賜爵汾陰侯。莊帝既幽崩,元曄地又疏遠,更議主社稷。孝通以廣陵王恭,高祖猶子,又在茂親,夙有令望。不言多載,理必陽瘖。奉以為主,天人允葉。世隆等並以為疑。孝通密贊天光察之。廣陵王曰:「天何言哉?」於是定冊,即節閔帝也。以首創大議,拜銀青光祿大夫、散騎常侍,兼中書舍人,封藍田縣子。孝通求以官贈亡兄景懋,又言己有侯爵,請轉授兄息子舒。節閔覽啟傷感,以侯爵既
 重,不容轉授,乃下詔褒美。特贈景懋撫軍、北雍州刺史。孝通尋遷中書郎,深為節閔所知重。



 普泰二年正月乙酉,中書舍人元翽獻酒肴,帝因與元翌及孝通等宴,兼奏絃管,命翽吹笛;帝亦親以和之。因使元翌等嘲,以酒為韻。孝通曰:「既逢堯舜君,願上萬年壽。」帝曰:「平生好玄默,慚為萬國首。」帝曰:「卿所謂壽,豈容徙然!」



 便命酌酒賜孝通,仍命更嘲,不得中絕。孝通即豎忠為韻。帝曰:「卿不忘忠臣之心。」翽曰:「聖主臨萬機,享世永無窮。」孝通曰:「豈唯被草木,方亦及昆蟲。」



 翌曰:「朝賢既濟濟,野苗又芃芃。」帝曰:「君臣體魚水,書軌一華戎。」孝通曰:「微臣信慶渥,何
 以答華嵩?」于時,孝通內典機密,外參朝政,軍國動靜,預以謀謨。加以汲引人物,知名之士,多見推薦。



 外兄裴伯茂性豪俊,多所輕忽。唯欽賞孝通,每有著述,共參同異。孝通以裴宏放過甚,每謂之曰:「兄以阮籍、嵇康何如管仲、樂毅?」蓋自許經綸,抑裴傲也。裴笑而不答,宏放自若。



 屬齊神武起兵河朔,攻陷相州刺史劉誕。爾朱天光自關中討之。孝通以關中險固,秦漢舊都,須預謀鎮遏,以為後計。縱河北失利,猶足據之。節閔深以為然,問誰可任者。孝通與賀拔岳同事天光,又與周文帝有舊;二人並先在關右,因並推薦之。乃超授岳岐、華、秦、雍諸軍事,關
 西大行臺,雍州牧。周文帝為左丞,孝通為右丞。齎詔書馳驛入關授岳等,同鎮長安。岳深相器重,待以師友之禮。與周文帝結為兄弟,情寄特隆。後天光敗於韓陵,節閔遂不得入關,為齊神武幽廢。孝武帝即位後,神武方得志,徵賀拔岳為冀州刺史。岳懼,欲單馬入朝。孝通乃謂岳曰:「高王以數千鮮卑破爾朱百萬之眾,其鋒誠亦難敵。然公兩兄太師、領軍,宿在其上。侯深、樊子鵠、賈知、斛斯椿、大野胡也杖、吒呂延慶之徒,於爾朱之世,皆其夷等。韓陵之役,此輩前後降附,皆由事勢危逼,非其本心。在於高王,曹操之孔融,馬懿之葛誕。今或在京師,或
 據州鎮,除之又失人望,留之腹心之疾。雖令孫騰在闕下,婁昭處鉤陳,必不能如建安之時,明矣。以今觀之,隙難未已。吐萬仁雖復退逸,猶在并州,高王之計,先須平殄。今方綏撫群雄,安置內外,何能去其巢穴,與公事關中地也?且六郡良家之子,三輔禮義之人,踰幽、并之驍騎,勝汝、潁之奇士,皆係仰於公,效其智力。據華山以為城雉,因黃河而為池塹;退守不失封泥,進兵同於建水。乃欲束手受制於人,不亦鄙乎?」言未卒,岳執孝通手曰:「君言是也。」乃遜辭為啟,而不就徵。



 太昌元年,孝通因使入朝,仍被留京師,重除中書侍郎。永熙三年三月,出為
 常山太守,仍以經節閔任遇故也。及孝武西遷,或稱孝通與周文友密,及樹置賀拔岳鎮關中之計,遂見拘執,將赴晉陽。及引見,咸為之憂。孝通神氣從容,辭理切正,齊神武更相欽歎,即日原免。然猶致疑忌,不加位秩,但引為坐客,時訪文典大事而已。齊神武讓劍履上殿表,猶使為文。曾與諸人同詣晉祠,皆屈膝盡禮。孝通獨捧手不拜,顧而言曰:「此乃諸侯之國,去吾何遠,恭而非禮,將為神笑。」



 拜者漸焉。興和二年,卒於鄴。魏前二年,周文帝追軫舊好,奏贈車騎將軍、儀同三司、青州刺史。齊神武武平初,又贈鄭州刺史。文集八十卷,行於時。



 子
 道衡,字玄卿。六歲而孤,專精好學。年十歲,講《左傳》,見子產相鄭之功,作《國僑贊》,頗有詞致,見者奇之。其後才名益著。齊司州牧、彭城王浟引為兵曹從事。尚書左僕射楊愔見而嗟賞,授奉朝請。吏部尚書隴西辛術與語,歎曰:「鄭公業不亡矣!」河東裴讞目之曰:「鼎遷河朔,吾謂『關西孔子』,罕遇其人,今復遇薛君矣!」



 武成即位,兼散騎常侍,接對周、陳二使。武平初,詔與諸儒脩定五禮,除尚書左外兵郎。陳使傅縡聘齊,以道衡兼主客郎接對之。縡贈詩五十韻,道衡和之,南北稱美。魏收曰:「傅縡所謂以蚓投魚耳。」待詔文林館,與范陽盧思道、安平李德林
 齊名友善。復以本官直中書省,尋拜中書侍郎,仍參太子侍讀。齊後主之世,漸見親用,與侍中斛律孝卿參預政事。道衡具陳備周之策,孝卿不能用。



 及齊亡,周武帝引為御史二命士。後歸鄉里,自州主簿入為司祿上士,隋文作相,從元帥梁睿擊王謙,攝陵州刺史。大定中,授儀同,守邛州刺史。文帝受禪,坐事除名。



 河間王弘北征突厥,召典軍書。還,除內史舍人。其年,兼散騎常侍,聘陳使主。道衡因奏曰:「陛下比隆三代,平一九州,豈容區區之陳,久在天網之外?臣今奉使,請責以稱蕃。」帝曰:「朕且含養,致之度外,勿以言辭相折。」江東雅好篇什,陳主尤
 愛彫蟲,道衡每有所作,南人無不吟誦焉。



 及八年伐陳,拜淮南道行臺尚書吏部郎,兼掌文翰。王師臨江,高熲夜幕中,謂曰:「今段定克江東以不?君試言之。」道衡答曰:「凡論大事成敗,先須以至理斷之。《禹貢》所載九州,本是王者封域。郭璞有云:『江東偏王三百年,還與中國合。』今數將滿矣。以運數而言,其必剋一也。有德者昌,無德者亡,自古興滅,皆由此道。主上躬履恭儉,憂勞庶政。叔寶峻宇彫牆,酣酒荒色。其必剋二也。



 為國之體,在於任寄。彼之公卿,備員而已。拔小人施文慶,委以政事;尚書令江總唯事詩酒,本非經略之才;蕭摩訶、任蠻奴是其
 大將,一夫之用耳。其必剋三也。



 我有道而大,彼無德而小。量其甲士,不過十萬,西自巫峽,東極滄海,分之則勢懸而力弱;聚之則守此而失彼。其必剋四也。席卷之勢,其在不疑。」熲忻然曰:「君言成敗,理甚分明。本以才學相期,不意籌略乃爾。」還除吏部侍郎。



 後坐抽擢人物,有言其黨蘇威,任人有意故,除名,配防嶺表。晉王廣時在揚州,陰令人諷道衡,遣從揚州路,將奏留之。道衡不樂王府,用漢王諒之計,遂出江陵道而去。尋詔徵還,直內史省。晉王由是銜之。然愛其才,猶頗見禮。後數歲,授內史侍郎,加上儀同三司。道衡每構文,必隱坐空齋,蹋壁而
 臥,聞戶外有人便怒,其沈思如此。帝每曰:「道衡作文書稱我意。」然誡之以迂誕。後帝謂楊素、牛弘曰:道衡老矣,驅使勤勞,宜使朱門陳戟。」於是進上開府,賜物百段。道衡辭以無功。帝曰:「爾久勞階陛,國家大事,皆爾宣行,豈非爾功也?」



 道衡久當樞要,才名益顯。太子、諸王爭與交好,高熲、楊素雅相推重,聲名籍甚,無競一時。仁壽中,楊素專掌朝政。道衡既與素善,上不欲道衡久知機密,因出檢校襄州總管。道衡一旦見出,不勝悲戀,言之哽咽。帝愴然改容曰:「爾光陰晚暮,侍奉誠勞,朕欲令爾將攝。今爾之去,朕如斷一臂。」於是賚物三百段,九環金帶并
 時服一襲,馬十匹,慰勉遣之。在任清簡,吏人懷其惠。



 煬帝嗣位,轉潘州刺史。歲餘,上表求致仕。帝謂內史侍郎虞世基曰:「道衡將至,當以秘書監待之。」道衡既至,上《高祖文皇帝頌》。帝覽之不悅。顧謂蘇威曰:「道衡致美先朝,此魚藻之義也。」於是拜司隸大夫,將置之罪。道衡不悟,司隸刺史房彥謙素與相善,知必及禍,勸之杜絕賓客,卑辭下氣,而道衡不能用。



 會議新令,久不能決,道衡謂朝士曰:「向使高熲不死,令當久行。」有人奏之。



 帝怒曰:「汝憶熲乎?」付執法者推之。道衡自以非大過,促憲司早解。奏日,冀帝赦之,敕家人具饌以備客來侯者。及奏,帝令
 自盡。道衡殊不意,未能引訣。憲司重奏,縊而殺之。妻子徙且末。時年七十。天下冤之。有集七十卷,行於世。



 有子五人,收最知名,出後族父孺。



 孺清貞孤介,不交流俗。涉歷經史,有才思,雖不為大文,所有詩詠,大致清遠。開皇中,為侍御史、揚州總管司功參軍。每以方直自處,府寮多不便之。卒於襄城郡掾。所蒞官皆有能名。道衡偏相友愛,收初生,即與孺為後。養於孺宅,至於成長,殆不識本生。太常丞胡仲操曾在朝堂就孺借刀子割爪甲。孺以仲操非雅士,竟不與之。其不肯妄交,清介獨行,皆此類也。



 道衡兄溫,字尼卿。沈敏有器局,博覽墳典,尤善隸
 書。仕周為上黃郡守。周平齊,徙燕郡太守,以簡惠稱。宣政元年,賜爵齊安縣子。卒於郡。子邁嗣。



 邁字弘仁,性寡言,長於詞辯。開皇初,襲爵齊安子,改封鐘山。歷位太子舍人。大業中,為刑部、選部二侍郎。



 道衡從父弟道實,位禮部侍郎、離石郡太守,知名於世。從子德音,有俊才,起家游騎尉。佐魏淡脩《魏史》,史成,遷著作佐郎。及越王侗稱制東都,王世充之僭號,軍書羽檄,皆出其手。世充平,以罪誅。其文筆多行於世。



 聰弟和,南青州刺史。和子善。



 善字仲良。少為司空府參軍。再遷鹽池都將。孝武西遷,魏改河東為秦州,以善為別駕。善家素富,僮僕數百人。
 兄元信,杖氣豪侈,每食方丈,坐客恆滿,絃歌不絕;而善獨恭己率素,愛樂閑靜。



 大統三年,齊神武敗於沙苑,留善族兄崇禮守河東。周文帝遣李弼圍之,崇禮固守不下。善密說崇禮,猶持疑不決。會善從弟馥妹夫高子信為防城都督,守城南面,遣馥來詣善,云「意欲應接西軍,但恐力所不制。」善即令弟濟將門生數十人,與信、馥等斬關引弼軍入。時預謀者並賞五等爵。善以背逆歸順,臣子常情,豈容闔門大小俱叨封邑,遂與弟慎並固辭不受。周文嘉之,以善為汾陰令。善幹用強明,一郡稱最。太守王羆美之,令善兼督六縣事。尋為行臺郎中。



 時欲
 廣置屯田以供軍費,乃除司農少卿,領同州夏陽縣二十屯監。又於夏陽諸山置鐵冶,復令善為監,每月役八千人,營造軍器。善自督課,兼加慰撫,甲兵精利而皆忘其苦焉。遷大丞相府從事中郎。追論屯田功,賜爵龍門縣子。遷黃門侍郎,除河東郡守,進驃騎大將軍、開府儀同三司,賜姓宇文氏。六官建,拜工部中大夫,進爵博平縣公。再遷戶部中大夫。



 時晉公護執政,儀同齊軌語善云:「兵馬萬機,須歸天子,何因猶在權門,」



 善白之,護乃殺軌。以善忠於己,引為中外府司馬,遷司會中大夫,副總六府事。



 加授京兆尹,仍行司會。出為隆州刺史,兼益州總
 管府長史。徵拜武威少府。卒,贈三州刺史。帝以善告齊軌事,謚曰繆公。子褒嗣,官至高陽郡守。



 善弟慎,字伯護。好學,能屬文,善草書。與同郡裴叔逸、裴諏之、柳虯、范陽盧柔、隴西李璨並友善。起家丞相府墨曹參軍。周文於行臺省置學,取丞郎及府佐德行明敏者充生。悉令旦理公務,晚就講習,先《六經》,後子史。又於諸生中簡德行淳懿者侍讀書。慎與李璨及隴西李伯良、辛韶、武功蘇衡、譙郡夏侯裕、安定梁曠、梁禮、河南長孫璋、河東裴舉、薛同、滎陽鄭朝等十二人,並應其選。又以慎為學師,以知諸生課業。周文雅好談論,并簡名僧
 深識玄宗者一百人,於第內講說。又命慎等十二人兼學佛義,使內外俱通。由是四方競為大乘學。在學數年,復以慎為宜都公侍讀。累遷禮部郎中。六官建,拜膳部下大夫。慎兄善又任工部,並居清顯,時人榮之。



 周孝閔帝踐阼,除御正下大夫,封淮南縣子。歷師氏、御伯中大夫。保定初,出為湖州刺史。界既雜蠻夷,恆以劫掠為務。慎乃集諸豪帥,具宣朝旨,仍令首領每月一參,或須言事者,不限時節。慎每見,必殷勤勸誡,及賜酒食。一年之間,翕然從化。諸蠻乃相謂曰:「今日始知刺史真人父母也。」莫不欣悅。自是襁負而至者千餘戶。蠻俗,婚娶之後,
 父母雖在,即與別居。慎謂守令曰:「牧守令長是化人者也,豈有其子娶妻,便與父母離析?非唯萌俗之失,亦是牧守之罪。」慎乃親自誘導,示以孝慈。并遣守令,各喻所部。有數戶蠻,別居數年,遂還侍養,及行得果膳,歸奉父母。慎以其從善之速,具以狀聞,有詔蠲其賦役。於是風化大行,有同華俗。尋為蕃部中大夫。以疾去職,卒於家。有文集,頗為世所傳。



 薛寘,河東汾陰人也。祖遵顏,魏河東郡守、安邑侯。父乂,清河、廣平二郡守。寘幼覽篇籍,好屬文,起家奉朝請。從魏孝武西遷,封郃陽縣子。廢帝元年,領著作佐郎,脩國
 史。尋拜中書侍郎,脩起居注。遷中書令。燕公于謹征江陵,以寘為司錄,軍中謀略,寘並參之。江陵平,進爵為伯。朝廷方改物創制,欲行《周禮》,乃令寘與小宗伯盧辯斟酌古今,共詳定之。六官建,授內史下大夫。



 周孝閔帝踐阼,進爵為侯,轉御正中大夫。時前中書監盧柔,學業優深,文藻華贍,而寘與之方駕,故世號曰盧、薛焉。久之,進位驃騎大將軍、開府儀同三司,出為淅州刺史。卒於位,吏人哀惜之。贈虞州刺史,謚曰理。所著文筆二十餘卷,行於世。又撰《西京記》三卷,引據該洽,世稱其博聞焉。寘性至孝,雖年齒已衰,職務繁廣,至於溫清之禮,朝夕無
 違。當時以此稱之。子明嗣。大象末,儀同大將軍、清水郡守。


薛憕,字景猷,河東汾陰人也。曾祖弘敞,逢赫連之亂,率宗人避地襄陽。憕早喪父,家貧。躬耕以養祖母,有暇則覽文籍。疏宕不拘,時人未之奇也。江表取人,多以世族。憕世無貴仕,解褐不過侍郎。既羈旅,不被擢用。常歎曰:「豈能五十年戴幘,死一校尉,低頭傾首,俯仰而向人也!」常鬱鬱不得志,每在人間,輒陵架勝達,負才使氣,未嘗趨世祿之門。左中郎將京兆韋潛度謂曰:「君門地非下,身材不劣,何不
 \gezhu{
  敝衣}
 裾數參吏部?」憕曰:「『世胄躡高位,英俊
 沈下寮』,古人以為歎息,竊所未能也。」潛度告人曰:「此年少實慷慨,但不遭時耳。」



 孝昌中,杖策還洛陽。先是憕從祖真度與族祖安都擁徐、兗歸魏,其子懷俊見憕,甚相親善。屬爾朱榮廢立,怔遂還河東,止懷俊家。不交人物,終日讀書,手自抄略,將二百卷。唯郡守元襲時相要屈,與之抗禮。懷俊每謂曰:「汝還鄉里,不營產業,不肯取妻,豈復欲南乎?」憕亦不介意。普泰中,拜給事中,加伏波將軍。



 及齊神武起兵,憕乃東游陳、梁間,謂族人孝通曰:「高歡阻兵陵上,喪亂方始。關中形勝之地,必有霸王據之。」乃與孝通俱游長安。侯莫陳悅聞之,召為行臺郎,除鎮
 遠將軍、步兵校尉。及悅害賀拔岳,軍人咸相慶慰。憕獨謂所親曰:「悅才略本寡,輒害良將,敗亡之事,其則不遠。吾屬今即為人所虜,何慶之有乎?」



 長高以憕言為然,並有憂色。尋而周文平悅,引憕為記室參軍。武帝西遷,授征虜將軍、中散大夫,封夏陽縣男。文帝即位,拜中書侍郎,加安東將軍,進爵為伯。



 大統四年,宣光、清徽殿初成,憕為之頌。文帝又造二欹器:一為二仙人共持一缽,同處一盤,缽蓋有山,山有香氣,一仙人又持金瓶以臨器上,傾水灌山,則出於瓶而注乎器,煙氣通發山中,謂之仙人欹器。一為二荷同處一盤,相去盈尺,中有蓮,下垂器上,以水
 注荷,則出於蓮而盈乎器,為鳧鴈蟾以飾之,謂之水芝欹器。二盤各處一床,缽圓而床方,中有人,三才之象也。皆置清徽殿前。形似觥而方,滿而平,溢則傾。憕各為頌。



 大統初,儀制多闕。周文令憕與盧辯、檀翥等參定之。以流離世故,不聽音樂,雖幽室獨處,常有戚容。後坐事死。子舒嗣,官至禮部下大夫、儀同大將軍、聘陳使副。



 論曰:薛辯有魏之初,功業早樹,門膺人爵,無替榮名。端以謙直見知;胄以公平自命。濬之孝悌,素緒之所得也。道衡雅道弈葉,世擅文宗,令望攸歸,豈徒然矣。而運逢季叔,卒蹈誅戮,痛乎!仲良任惟繁劇,弘益流譽;而陷齊
 諂護,以要權寵,易名為繆,斯豈虛哉!寘、憕並學稱該博,文擅雕龍,或揮翰鳳池,或著書麟閣,咸居祿位,各逞琳瑯。擬彼徐、陳,慚後生之可畏;論其任遇,實當時之良選也。



\end{pinyinscope}