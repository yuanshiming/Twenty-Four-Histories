\article{卷九十九列傳第八十七 突厥 鐵勒}

\begin{pinyinscope}

 突
 厥者,其先居西海之右,獨為部落,蓋匈奴之別種也。姓阿史那氏。後為鄰國所破,盡滅其族。有一兒,年且十歲,兵人見其小,不忍殺之,乃刖足斷其臂,棄草澤中。有牝狼以肉餌之,及長,與狼交合,遂有孕焉。彼王聞此兒尚在,重遣殺之。使者見在狼側,并欲殺狼。於時若有神物,投狼於西海之東,落高昌國西北山。山有洞穴,穴內
 有平壤茂草,周迥數百里,四面俱山。狼匿其中,遂生十男。



 十男長,外託妻孕,其後各為一姓,阿史那即其一也,最賢,遂為君長。故牙門建狼頭纛,示不忘本也。漸至數百家,經數世,有阿賢設者,率部落出於穴中,臣於蠕蠕。至大葉護,種類漸強。當魏之末,有伊利可汗,以兵擊鐵勒,大敗之,降五萬餘家。遂求婚於蠕蠕主。阿那瑰大怒,遣使罵之。伊利斬其使,率眾襲蠕蠕,破之。卒,弟阿逸可汗立,又破蠕蠕。病且卒,捨其子攝圖,立其弟俟叔稱為木桿可汗。



 或云突厥本平涼雜胡,姓阿史那氏。魏太武皇帝滅沮渠氏,阿史那以五百家奔蠕蠕。世居金山之
 陽,為蠕蠕鐵工。金山形似兜鍪,俗號兜鍪為突厥,因以為號。



 又曰突厥之先,出於索國,在匈奴之北。其部落大人曰阿謗步,兄弟七十人,其一曰伊質泥師都,狼所生也。阿謗卻等性並愚癡,國遂被滅。泥師都既別感異氣,能征占風雨。取二妻,云是夏神、冬神之女。一孕而生四男:其一變為白鴻;其一國於阿輔水、劍水之間,號為契骨;其一國於處折水;其一居跋斯處折施山,即其大兒也。山上仍有阿謗步種類,並多寒露,大兒為出火溫養之,咸得全濟。遂共奉大兒為主,號為突厥,即納都六設也。都六有十妻,所生子皆以母族姓,阿史那是其小
 妻之子也。都六死,十母子內欲擇立一人,乃相率於大樹下,共為約曰:「向樹跳躍,能最高者,即推立之。」阿史那子年幼而跳最高,諸子遂奉以為主,號阿賢設。此說雖殊,終狼種也。



 其後曰土門,部落稍盛,始至塞上市繒絮,願通中國。西魏大統十一年,周文帝遣酒泉胡安諾槃陀使焉。其國皆相慶曰:「今大國使至,我國將興也。」十二年,土門遂遣使獻方物。時鐵勒將伐蠕蠕,土門率所部邀擊破之,盡降其眾五萬餘落。



 恃其強盛,乃求婚於蠕蠕主。阿那瑰大怒,使人詈辱之曰:「爾是我鍛奴,何敢發是言也!」土門亦怒,殺其使者,遂與之絕,而求婚於魏。周
 文帝許之,十七年六月,以魏長樂公主妻之。是歲,魏文帝崩,土門遣使來弔,贈馬二百疋。廢帝元年正月,土門發兵擊蠕蠕,大破之於懷荒北。阿那瑰自殺,其子庵羅辰奔齊,餘眾復立阿那瑰叔父鄧叔子為主。土門遂自號伊利可汗,猶古之單于也;號其妻為可賀敦,亦猶古之閼氏也。亦與齊通使往來。



 土門死,子科羅立。科羅號乙息記可汗,又破叔子於沃野北賴山。且死,捨其子攝圖,立其弟俟斤,是為木桿可汗。



 俟斤一名燕都,狀貌奇異,面廣尺餘,其色赤甚,眼若琉璃,剛暴,勇而多知,務於征伐。乃率兵擊鄧叔子,破之。叔子以其餘燼奔西魏。俟
 斤又西破嚈噠,東走契丹,北并契骨,威服塞外諸國。其地,東自遼海以西,至西海,萬里;南自沙漠以北,至北海,五六千里:皆屬焉。抗衡中國,後與魏伐齊,至并州。



 其俗:被髮左衽,穹廬氈帳,隨逐水草遷徙,以畜牧射獵為事,食肉飲酪,身衣裘褐。賤老貴壯,寡廉恥,無禮義,猶古之匈奴。其主初立,近侍重臣等輿之以氈,隨日轉九回,每回臣下皆拜,拜訖乃扶令乘馬,以帛絞其頸,使纔不至絕,然後釋而急問之曰:「你能作幾年可汗?」其主既神情瞀亂,不能詳定多少。臣下等隨其所言,以驗脩短之數。大官有葉護,次設,次特勤,次俟利發,次吐屯發,及餘小官,凡
 二十八等,皆世為之。兵器有角弓、鳴鏑、甲、槊、刀、劍、佩飾則兼有伏突。旗纛之上,施金狼頭。待衛之士謂之附離,夏言亦狼也。蓋本狼生,志不忘舊。善騎射,性殘忍。無文字,其徵發兵馬及諸稅雜畜,刻木為數,並一金鏃箭,蠟封印之,以為信契。候月將滿,轉為寇抄。其刑法:反叛、殺人、及姦人之婦、盜馬絆者,皆死;淫者,割勢而腰斬之;奸人女者,重責財物,即以其女妻之;鬥傷人者,隨輕重輸物,傷目者償以女,無女則輸婦財,折支體者輸馬;盜馬及雜物者,各十餘倍征之。死者,停屍於帳,子孫及親屬男女各殺羊、馬,陳於帳前祭之,繞帳走馬七匝,詣帳門
 以刀剺面且哭,血淚俱流,如此者七度乃止。擇日,取亡者所乘馬及經服用之物,并屍俱焚之,收其餘灰,待時而葬。春夏死者,候草木黃落;秋冬死者,候華茂,然後坎而痤之。葬日,親屬設祭及走馬、剺面如初死之儀。表為塋,立屋,中圖畫死者形儀,及其生時所戰陣狀,嘗殺一人,則立一石,有致千百者。又以祭之羊、馬頭,盡懸之於標上。是日也,男女咸盛服飾,會於葬所,男有悅愛於女者,歸即遣人聘問,其父母多不違也。父、兄、伯、叔死,子、弟及姪等妻其後母、世叔母、嫂,唯尊者不得下淫。移徙無常,而各有地分。可汁恒處於都斤山,牙帳東開,蓋敬日
 之所出也。每歲率諸貴人,祭其先窟。又以五月中旬,集他人水拜祭天神。於都斤西五百里有高山迥出,上無草樹、謂為勃登凝梨,夏言地神也。其書字類胡,而不知年曆,唯以草青為記。男子好樗蒲,女子踏鞠,飲馬酪取醉,歌呼相對。敬鬼神,信巫覡,重兵死,恥病終,大抵與匈奴同俗。



 俟斤部眾既盛,乃遣使請誅鄧叔子等,周文帝許之,收叔子已下三千人,付其使者,殺之於青門外。三年,俟斤襲擊吐谷渾破之。周明帝二年,俟斤遣使來獻。



 保定元年,又遣三輩,貢其方物。時與齊人交爭,戎車歲動,故連結之,以為外援。



 初,恭帝時,俟斤許進女於周文帝,契
 未定而周文崩。尋而俟斤又以他女許武帝,未及結納,齊人亦遣求婚,俟斤貪其幣厚,將悔之。至是,武帝詔遣涼州刺史楊薦、武伯王慶等往結之。慶等至,諭以信義,俟斤遂絕齊使而定婚焉。仍請舉國東伐,於是詔隨公楊忠率眾一萬與突厥伐齊。忠軍度陘嶺,侯斤率騎十萬來會。明年正月,攻齊主於晉陽,不剋,俟斤遂縱兵大掠而還。忠還,言於武帝曰:「突厥甲兵惡,賞罰輕,首領多而無法令,何謂難制馭?由比者使人妄道其強盛,欲令國家厚其使者,身往重取其報。朝廷受其虛言,將士望風畏懾。但虜態詐健,而實易與耳。今以臣觀之,前後使人
 皆可斬也。」武帝不納。是歲,俟斤復遣使來獻,更請東伐。



 詔楊忠率兵出沃野,晉公護趣洛陽以應之。會護戰不利,俟斤引還。五年,詔陳公純、大司徒宇文貴、神武公竇毅、南安公楊薦往逆女。天和二年,俊斤又遣使來獻。



 陳公純等至,俟斤復貳於齊。會有雷風變,乃許純等以后歸。四年,又遣使貢獻。



 俟斤死,復捨其子大邏便而立其弟,是為他缽可汗。他缽以攝圖為爾伏可汗,統其東面;又以其弟褥但可汗為步離可汗,居西方。自俟斤以來,其國富強,有凌轢中夏之志。朝廷既與之和親,歲給繒絮、錦彩十萬段。突厥在京師者,又待以優禮,衣錦食肉,
 常以千數。齊人懼其寇掠,亦傾府藏以給之。他缽彌復驕傲,仍令其徒屬曰:「但使我在南兩個兒孝順,何憂無物邪?」齊有沙門惠琳,掠入突厥中,因謂他缽曰:「齊國強富,皆為有佛法。」遂說以因緣果報之理。他缽聞而信之,建一伽藍,遣使聘齊,求《凈名》、《涅槃》、《華嚴》等經,並《十誦律》。他缽亦躬自齋戒,繞塔行道,恨不生內地。建德二年,他缽遣使獻馬。及齊滅,齊定州刺史、范陽王高紹義自馬邑奔之。他缽立紹義為齊帝,召集所部,云為之復仇。



 宣政元年四月,他缽遂入寇幽州。柱國劉雄拒戰,兵敗死之。武帝親總六軍,將北伐,會帝崩,乃班師。是冬,他缽
 復寇邊,圍酒泉,大掠而去。大象元年,他缽復請和親,帝策趙王招女為千金公主以嫁之,并遣執紹義送闕。他缽不許,仍寇并州。



 二年,始遣使奉獻,且迎公主為親,而紹義尚留不遣。帝又令賀若誼往諭之,始送紹義。



 他缽病且卒,謂其子庵邏曰:「吾聞親莫過於父子。吾兄不親其子,委位於我,我死,汝當避大邏便。」及卒,國中將立大邏便,以其母賤,眾不服。庵邏實貴,突厥素重之。攝圖最後至,謂國中曰:「若立庵邏者,我當率兄弟以事之;如立大邏便,我必守境,利刃長矛以相待。」攝圖長而且雄,國人莫敢拒,竟立庵邏為嗣。



 大邏便不得立,心不服庵邏,每
 遣人詈辱之。庵邏不能制,因以國讓攝圖。國中相與議曰:「四可汗子,攝圖最賢。」因迎立之,號伊利俱盧設莫何始波羅可汗,一號沙缽略,居都斤山。庵邏降居獨洛水,稱第三可汗。大邏便乃謂沙缽略曰:「我與爾俱可汗子,各承父後,爾今極尊,我獨無位,何也?」沙缽略患之,以為阿波可汗,還領所部。



 沙缽略勇而得眾,北夷皆歸附之。隋文帝受禪,待之甚薄,北夷大怨。會營州刺史高寶寧作亂,沙缽略與之合軍,攻陷臨渝鎮。上敕緣邊修保鄣,峻長城,以備之。沙缽略妻,周千金公主,傷宗祀絕滅,由是悉眾來寇,控弦士四十萬。上令柱國馮昱屯乙弗泊,
 蘭州總管叱李崇屯幽州,達奚長儒據周槃,皆為虜敗。於是縱兵自木硤、石門兩道來寇,武威、天水、安定、金城、上郡、弘化、延安六畜咸盡。



 天子震怒,下詔曰:往者周、齊抗衡,分割諸夏,突厥之虜,俱通二國。周人東慮,恐齊好之深;齊氏西虞,懼周交之厚。各謂慮意輕重,國遂安危。非徒並有大敵之憂,思減一邊之防。竭生靈之力,供其來往,傾府庫之財,棄於沙漠。華夏之地,實為勞擾。朕受天明命,子育萬方,愍臣下之勞,除既往之弊。回入賊之物,加賜將士;息在路之人,務於耕織。凶醜愚闇,未知深旨,將大定之日,比戰國之時,乘昔世之驕,結今時之恨。
 近者,盡其巢窟,俱犯北邊,而遠鎮偏師,逢而摧翦,未及南上,遽已奔北。



 且彼渠師,其數凡五,昆季爭長,父叔相猜,世行暴虐,家法殘忍。東夷諸國,盡挾私讎;西戎群長,皆有宿怨。突厥之北,契骨之徒,切齒磨牙,常伺其後。達頭前攻酒泉,于闐、波斯、揖怛三國,一時即叛;沙缽略近趣周槃,其部內薄孤、東紇羅尋亦翻動。往年利稽察大為高麗、靺鞨所破,沙毗設又為紇支可汗所殺。與其為鄰,皆願誅剿,部落之下,盡異純人。千種萬類,仇敵怨偶,泣血拊心,銜悲積恨。圓首方足,皆人類也,有一於此,更切朕懷。彼地咎徵妖作,將年一紀。乃獸為人語,人作神
 言,云其國亡,訖而不見。每冬雷震,觸地火生。種類資給,唯藉水草,去歲四時,竟無雨雪,川枯蝗暴,卉木燒盡,饑疫死亡,人畜相半。舊居之地,赤土無依,遷徙漠南,偷存晷刻。斯蓋上天所忿,驅就齊斧,幽明合契,今也其時。



 故選將練兵,贏糧聚甲,義士奮發,壯夫肆憤,願取名王之首,思撻單于之背。



 此則王恢所說,其猶射癰,何敵能當,何遠不剋。但皇王舊迹,北止幽都,荒遐之表,文軌所棄,得其地不可而居,得其人不忍皆殺。無勞兵革,遠規溟海。普告海內,知朕意焉。



 於是河間王弘、上柱國豆盧績、竇榮定、左僕射高熲、右僕射虞慶則並為元帥,出塞擊
 之。沙缽略率阿波、貪汗二可汗來拒戰,皆敗走。時虜飢不能得食,粉骨為糧,又多災疫,死者極眾。



 既而沙缽略以阿波驍悍,忌之,因其先歸,襲擊其部,大破之,殺阿波母。阿波還無所歸,西奔達頭可汗。達頭者,名玷厥,沙缽略之從父也,舊為西面可汗。



 既而大怒,遣阿波率兵而東,其部落歸之者將十萬騎,遂與沙缽略相攻。又有貪汗可汗,素睦於阿波,沙缽略奪其眾而廢之,貪汗亡奔達頭。沙缽略從弟地勤察,別統部落,與沙缽略有隙,復以眾叛歸阿波。連兵不已,各遣使詣闕,請和求援,上皆不許。



 會千金公主上書,請為一子之例,文帝遣開府徐平和
 使於沙缽略。晉王廣時鎮并州,請因其釁乘之,上不許。沙缽略遣使致書曰:「辰年九月十日,從天生大突厥天下賢聖天子伊利俱盧設莫何始波羅可汗致書大隋皇帝:使人開府徐平和至,辱告言語,具聞也。皇帝是婦父,即是翁,此是女夫,即是兒例,兩境雖殊,情義是一。今重疊親舊,子子孫孫,乃至萬世不斷。上天為證,終不違負。此國所有羊、馬,都是皇帝畜生;彼有繒彩,都是此物。彼此不異也。」文帝報書曰:「大隋天子貽書大突厥伊利俱盧設莫何沙缽略可汗:得書,知大有好心向此也。既是沙缽略婦翁,今日看沙缽略共兒子不異。既以親舊
 厚意,常使之外,今特別遣大臣虞慶則往彼看女,復看沙缽略也。」沙缽略陳兵列其寶物,坐見慶則,稱病不能起,且曰:「我伯父以來,不向人拜。」慶則責而喻之。千金公主私謂慶則曰:「可汗豺狼性,過與爭,將齧人。」長孫晟說諭之,攝圖屈,乃頓顙受璽書,以戴於首。既而大慚,其群下因相聚慟哭。慶則又遣稱臣,沙缽略謂其屬曰:「何名為臣?」報曰:「隋國臣,猶此稱奴。」沙缽略曰:「得作大隋天子奴,虞僕射之力也。」贈慶則馬千匹,并以從妹妻之。



 時沙缽略既為達頭所困,又東畏契丹,遣使告急,請將部落度漠南,寄居白道川內。有詔許之。晉王廣以兵援之,給
 以衣食,賜以車服、鼓吹。沙缽略因西擊阿波,破擒之。而阿拔國部落乘虛掠其妻子。官軍為擊阿拔,敗之,所獲悉與沙缽略。



 沙缽略大喜,乃立約,以磧為界。因上表曰:「大突厥伊利俱盧設始波羅莫何可汗臣攝圖言:大使、尚書右僕射虞慶則至,伏奉詔書,兼宣慈旨,仰惟恩信之著,愈久愈明,徒知負荷,不能答謝。突厥自天置以來,五十餘載,保有沙漠,自王蕃隅,地過萬里,士馬億數,恒力兼戎夷,抗禮華夏,在於戎狄,莫與為大。頃者,氣候清和,風雪順序,意以華夏其有大聖興焉。伏惟大隋皇帝真皇帝也,豈敢阻兵恃險,偷竊名號?今便感慕淳風,歸心
 有道。雖復南瞻魏闕,山川悠遠,北面之禮不敢廢。



 當令侍子入朝,神馬歲貢,朝夕恭承,惟命是親。謹遣第七兒臣窟合真等奉表以聞。」



 文帝下詔曰:「沙缽略往雖與和,猶是二國,今作君臣,便成一體。已敕有司,肅告郊廟,宜傳播天下,咸使知聞。」自是詔答諸事,並不稱其名以異之。其妻可賀敦周千金公主,賜姓楊氏,編之屬籍,改封大義公主。策拜窟合真為柱國,封安國公,宴於內殿,引見皇后,賞勞甚厚。沙缽略大悅。於是,歲時貢獻不絕。



 七年正月,沙缽略遣其子入貢方物。因請獵於恒、代之間,詔許之,仍遣使人,賜其酒食。沙缽略率部落再拜受賜。
 沙缽略一日手殺鹿十八頭,齎尾舌以獻。還至紫河鎮,其牙帳為火所燒,沙缽略惡之,月餘而卒。上為之廢朝三日,遣太常弔祭焉,贈物五千段。



 初,攝圖以其子雍虞閭性懦,遣令立其弟葉護處羅侯。雍虞閭遣使迎處羅侯,將立之,處羅侯曰:「我突厥自木桿可汗來,多以弟代兄,以庶奪嫡,失先祖之法,不相敬畏。汝當嗣位,我不憚拜汝也。」雍虞閭又遣使謂處羅侯曰:「叔與我父,共根連體,我是枝葉,寧有我作主,令根本反同枝葉?願叔勿疑。」相讓者五六,處羅侯竟立,是為葉護。遣使上表言狀,上賜之鼓吹、幡旗。處羅侯長頤僂背,眉目疏朗,勇而有謀。以
 隋所賜旗鼓,西征阿波,敵人以為得隋兵所助,多來降附,遂擒阿波。既而上書,請阿波死生之命。上下其議,左僕射高熲進曰:「骨肉相殘,教之蠹也,宜存養以示寬大。」上曰:「善。」熲因奉觴進曰:「自軒轅以來,獯粥多為邊患。今遠窮北海,皆為臣妾,此之盛事,振古未聞。臣敢再拜上壽。」



 後處羅侯又西征,中流矢卒,其眾奉雍虞閭為主,是為頡伽施多那都藍可汗。



 雍虞閭遣使詣闕,賜物三千段,每歲遣使朝貢。時有流人楊欽,亡入突厥中,謬云彭國公劉昶與宇文氏謀反,令大義公主發兵擾邊。都藍執欽以聞,并貢勃布、魚膠。



 其弟欽羽設部落強盛,都藍
 忌而擊之,斬首於陣。其年,遣其母弟褥但特勤獻於闐玉杖,上拜褥但為柱國、康國公。明年,突厥部落大人相率遣使貢馬萬匹,羊二萬口,駝、牛各五百頭。尋遣請緣邊置市,與中國貿易,詔許之。



 平陳後,上以陳叔寶屏風賜大義公主,主心恆不平,因書屏風為詩,敘陳亡以自寄曰:「盛衰等朝暮,世道若浮萍,榮華實難守,池臺終自平。富貴今安在?空事寫丹青。盃酒恒無樂,絃歌詎有聲?余本皇家子,飄流入虜庭,一朝睹成敗,懷抱忽縱橫。古來共如此,非我獨申名。唯有《昭君曲》,偏傷遠嫁情。」上聞惡之,禮賜益薄。公主復與西突厥泥利可汗連結,上恐
 其為變,將圖之。會主與所從胡私通,因發其事,下詔廢之。恐都藍不從,遣奇章公牛弘將美妓四人以啖之。時沙缽略子曰染干,號突利可汗,居北方,遣使求婚。上令裴矩謂曰:「當殺大義公主方許婚。」突利以為然,復譖之。都藍因發怒,遂殺公主於帳。



 都藍因與突利可汗有隙,數相征伐,上和解之,各引兵去。十七年,突利遣使來逆女,上舍之太常,教習六禮,妻以宗女安義公主。上欲離間北狄,故特厚其禮,遣牛弘、蘇威、斛律孝卿相繼為使。突厥前後遣使入朝,三百七十輩。突利本居北方,以尚主故,南徙度斤舊鎮,錫賚優厚。雍虞閭怒曰:「我大可汗也,
 反不如染干!」於是朝貢遂絕,數為邊患。



 十八年,詔蜀王秀出靈州道擊之。明年,又遣漢王諒為元帥,左僕射高熲率將軍王察、上柱國趙仲卿並出朔州道,右僕射楊素率柱國李徹、韓僧壽出靈州道,上柱國燕榮出幽州,以擊之。雍虞閭與玷厥舉兵攻染干,盡殺其兄弟子女,遂渡河入蔚州。染干夜以五騎與隋使長孫晟歸朝。上令染干與雍虞閭使者因頭特勤相辯詰,染干辭直,上乃厚待之。雍虞閭弟都速六棄其妻子,與突利歸朝。上嘉之,敕染干與都速六樗蒱,稍稍輸以寶物,用歸其心。六月,高熲、楊素擊玷厥,大破之。拜染干為意利珍豆啟人
 可汗,華言意智健也。啟人上表謝恩。上於朔州築大利城以居之。時安義公主以卒,上以宗女義城公主妻之。部落歸者甚眾。雍虞閭又擊之,上復令入塞。雍虞閭侵掠不已,遂遷於河南,在夏、勝二州間,發徒掘塹數百里,東西距河,盡為啟人畜牧地。



 於是遣越國公楊素出靈州,行軍總管韓僧壽出慶州,太平公史萬歲出燕州,大將軍姚辯出河州,以擊都藍。師未出塞,而都藍為其麾下所殺,達頭自立為步伽可汗,其國大亂。遣太平公史萬歲出朔州以擊之,遇達頭於大斤山,虜不戰而遁。尋遣其子侯利伐徒磧東攻啟人,上又發兵助啟人守要
 路,侯利伐退走入磧。啟人上表陳謝曰:「大隋聖人莫緣可汗憐養,百姓蒙恩,赤心歸服,或南入長城,或住白道。



 染干如枯木重起枝葉,枯骨重生皮肉,千世萬世,長與大隋典羊、馬也。」



 仁壽元年,代州總管韓洪為虜敗於恆安,詔楊素為雲州道行軍元帥,率啟人北征。斛薛等諸姓初附於啟人,至是而叛。素軍河北,逢突厥阿勿思力俟斤等南渡,掠啟人男女雜畜而去,素率上大將軍梁默追之,大破俟斤,悉得人畜以歸啟人。素又遣柱國張定和、領軍大將軍劉昇別路邀擊,並多斬獲而還。兵既渡河,賊復掠啟人部落,素率驃騎范貴於窟結谷東南復
 破之。



 是歲,泥利可汗及葉護俱被鐵勒所敗,步迦尋亦大亂。奚、雪五部內徙,步伽奔吐谷渾,啟人遂有其眾,遣使朝貢。



 大業三年,煬帝幸榆林,啟人及義城公主來朝行宮,前後獻馬三千匹。帝大悅,賜帛萬三千段。啟人及義城公主上表曰:「已前聖人先帝莫緣可汗存日,憐臣,賜臣安義公主,臣種末為聖人先帝憐養。臣兄弟妒惡,相共殺臣。臣當時無處去,向上看只見天,下看只見地,實憶聖人先帝言語,投命去來。聖人先帝見臣,大憐臣死命,養活勝於往前,遣臣作大可汗坐著也。突厥百姓死者以外,還聚集作百姓也。



 至尊今還如聖人先帝於
 天下四方坐也,還養活臣及突厥百姓,實無少短。至尊憐臣時,乞依大國,服飾法用一同華夏。帝下其議,公卿請依所奏,帝以為不可。乃詔曰:「君子教人,不求變俗,何必化諸削衽,縻以長纓?」乃璽書答啟人,以為磧北未靜,猶復征戰,但使存心孝順,何必改衣服也。帝法駕御千人大帳,享啟人及其部落酋長三千五百人,賜物二千段,其下各有差。復下詔褒寵之,賜路車、乘馬、鼓吹、幡旗,贊拜不名,位在諸侯王上。帝親巡雲中,溯金河而東,北幸啟人所居。



 啟人奉觴上壽,跪伏甚恭。帝大悅,賦詩曰:「鹿塞鴻旗駐,龍庭翠輦回,氈帳望風舉,穹廬向日開。呼
 韓頓顙至,屠耆接踵來,索辮擎膻肉,韋韌獻酒盃。何如漢天子,空上單于臺?」帝賜啟人及主金甕各一,及衣服、被褥、錦彩;特勤以下各有差。



 先是,高麗私通使啟人所,啟人不敢隱境外之交,是日,持高麗使見。敕令牛弘宣旨謂曰:「朕以啟人誠長奉國,故親至其所。明年當往涿郡。爾回日,語高麗主,宜早來朝。」使人甚懼。啟人乃扈從入塞至定襄,詔令歸蕃。明年,朝於東都,禮賜益厚。是歲,疾終,上為廢朝三日。



 其子吐吉立,是為始畢可汗。表續尚公主,詔從其俗。十一年,來朝於東都。



 其年,車駕避暑汾陽宮。八月,始畢率其種落入寇,圍帝於雁門。援兵方
 至,始畢引去。由是朝貢遂絕。明年,復寇馬邑,唐公擊走之。隋末亂離,中國人歸之者無數,遂大強盛。迎蕭后置於定襄。薛舉、竇建德、王世充、劉武周、梁師都、李軌、高開道之徒,雖僭尊號,皆稱臣,受其可汗之號,使者往來,相望於道。



 西突厥者,木桿可汗之子大邏便也。與沙缽略有隙,因分為二,漸以強盛。東拒都斤,西至龜茲,鐵勒、伊吾及西域諸胡悉附之。大邏便為處邏侯所執,其國立鞅素特勤之子,是為泥利可汗。卒,子達漫立,號泥撅處羅可汗。其母向氏,本中國人,生達漫而泥利卒,向氏又嫁其弟婆實特勤。開皇末,婆實共向氏入朝,遇達頭之
 亂,遂留京師,每舍之鴻臚寺。處羅可汗居無恒處,終多在烏孫故地。復立二小可汗,分統所部,一在石國北,以制諸胡國;一居龜茲北,其地名應娑。官有俟發、閻洪達,以評議國事,自餘與東國同。每五月、八月,聚祭神,歲使重臣向其先世所居之窟致祭焉。



 當大業初,處羅可汗撫御無道,其國多叛,與鐵勒屢相攻,大為鐵勒所敗。時黃門侍郎裴矩在敦煌引致西域,聞其國亂,復知處羅思其母氏,因奏之。煬帝遣司朝謁者崔君肅齎書慰諭之。處羅甚踞,受詔不肯起。君肅謂處羅曰:「突厥本一國也,中分為二,自相仇敵,每歲交兵,積十年而莫能相滅
 者,明知啟人與處羅國其勢敵耳。今啟人舉其部落,兵且百萬,入臣天子,甚有丹誠者,何也?但以切恨可汗而不能獨制,故卑事天子以借漢兵,連二大國,欲滅可汗耳。百官兆庶咸請許之,天子弗違,師出有日矣。顧可汗母向氏,本中國人,歸在京師,處于賓館,聞天子之詔,懼可汗之滅,旦夕守闕,哭甚悲哀,是以天子憐焉,為其輟策。向夫人又匍匐謝罪,因請發使以召可汗,令入內屬,乞加恩禮,同於啟人。天子從之,遣使到此。可汗若稱籓拜詔,國乃永安,而母得延壽;不然者,則向夫人為誑天子,必當取戮而傳首虜庭。發大隋之兵,資北蕃之眾,左提右挈,
 以擊可汗,死亡則無日矣!



 奈何惜兩拜之禮,剿慈母之命,吝一句稱臣,喪匈奴之國也?」處羅聞之,瞿然而起,流涕再拜,跪受詔書。



 君肅又說處羅曰:「啟人內附,先帝嘉之,賞賜極厚,故致兵強國富。今可汗後附,與之爭寵,須深結於天子,自表至誠。既以遠道,未得朝覲,宜立一功,以明臣節。」處羅曰:「如何?」君肅曰:「吐谷渾者,啟人少子莫賀咄設之母家也。



 今天子又以義城公主妻於啟人,畏天子之威,而與之絕。吐谷渾亦因憾漢,職貢不修。可汗若請誅之,天子必許。漢擊其內,可汗攻其外,破之必矣。然後自入朝,道路無阻,因見老母,不亦可乎?」處羅大喜,
 遂遣使朝貢。



 帝將四狩,六年,遣侍御史韋節召處羅,令與車駕會於大斗拔谷。其國人不從,處羅謝使者,辭以他故。帝大怒,無如之何。適會其酋長射匱遣使來求婚,裴矩奏曰:「處羅不朝,恃強大耳。臣請以計弱之,分裂其國,即易制也。射匱者,都六之子,達頭之孫,世為可汗,君臨西面。今聞其失職,附隸於處羅,故遣使來以結援。願厚禮其使,拜為大可汗,則突厥勢分,兩從我矣。」帝曰:「公言是也。」



 因遣裴矩,朝夕至館,微諷喻之。帝於仁風殿召其使者,言處羅不順之意,稱射匱有好心,吾將立為大可汗,令發兵誅處羅,然後當為婚也。取桃竹白羽箭一枚以
 賜射匱,因謂之曰:「此事宜速,使疾如箭也。」使者返,路經處羅。愛其箭,將留之,使者譎而得免。射匱聞而大喜,興兵襲之,處羅大敗,棄妻子,將左右數千騎東走。在路又被劫掠,遁於高昌,車保時羅漫山。高昌王麴伯雅上狀,帝遣裴矩將向氏親要左右,馳至玉門關晉昌城。矩遣向氏使詣處羅所,論朝廷弘養之義,丁寧曉喻之。遂入朝,然每有怏怏之色。



 以七年冬,處羅朝於臨朔宮。帝享之,處羅稽首謝曰:「臣總西面諸蕃,不得早來朝拜,今參見遲晚,罪責極深。臣心裏悚懼,不能盡道。」帝曰:「往者與突厥遞相侵擾,不得安居。今四海既清,與一家無異,朕皆
 欲存養,使遂性靈。譬如上天,止有一個日照臨,莫不寧帖;若有兩箇、三個日,萬物何以得安?比者,亦知處羅總攝事繁,不得早來相見。今日見處羅,懷抱豁然歡喜。處羅亦當豁然,不煩在意。」明年元會,處羅上壽曰:「自天以下,地以上,日月所照,唯有聖人可汗。今是大日,願聖人可汗千歲、萬歲,常如今日也。」詔留其羸弱萬餘口,令其弟達度闕設牧畜會寧郡。處羅從征高麗,號為曷薩那可汗,賞賜甚厚。



 十年正月,以信義公主嫁焉,賜錦彩,袍千具、彩萬匹。帝將復其故地,以遼東之役,故未遑也。每從行幸。江都之亂,隨化及至河北。化及將敗,奔歸京師,
 為北蕃突厥所害。



 鐵勒之先,匈奴之苗裔也。種類最多,自西海之東依山據谷,往往不絕。獨洛河北,有僕骨、同羅、韋紇、拔也古、覆羅,並號俟斤,蒙陳、吐如紇、斯結、渾、斛薛等諸姓,勝兵可二萬。伊吾以西,焉耆之北,傍白山,則有契弊、薄落職、乙咥、蘇婆、那曷、烏護、紇骨、也咥、於尼護等,勝兵可二萬。金山西南,有薛延阤、咥勒兒、十槃、達契等,一萬餘兵。康國北,傍阿得水,則有訶咥、曷截、撥忽、比干、具海、曷北悉、何嵯蘇、拔也末、謁達等,有三萬許兵。得嶷海東西,有蘇路羯、三素咽、篾促、薩忽等諸姓,八千餘。拂菻東,則有恩屈、阿
 蘭、北褥、九離、伏嗢昏等,近二萬人。北海南,則都波等。雖姓氏各別,總謂為鐵勒。並無君長,分屬東西兩突厥。居無恒所,隨水草流移。人性凶忍,善於騎射,貪婪尤甚,以寇抄為生。近西邊者,頗為藝植,多牛而少馬。



 自突厥有國,東西征討,皆資其用,以制北荒。開皇末,晉王廣北征,納啟人,破步迦可汗,鐵勒於是分散。大業元年,突厥處羅可汗擊鐵勒諸部,厚稅斂其物,又猜忌薛延陀等,恐為變,遂集其魁帥數百人,盡誅之。由是一時反叛,拒處羅。



 遂立俟利發、俟斤契弊歌楞為易勿真莫何可汗,居貪汗山;復立薛延陀內俟斤子也咥為小可汗。處羅既敗,莫
 何可汗始大。莫何勇毅絕倫,甚得眾心,為鄰國所憚,伊吾、高昌、焉耆諸國悉附之。



 其俗大抵與突厥同。唯丈夫婚畢,便就妻家,待產乳男女,然後歸舍;死者埋殯之:此其異也。大業三年,遣使貢方物,自是不絕云。



 論曰:四夷之為中國患也,久矣,北狄尤甚焉。種落實繁,迭雄邊塞,年代遐邈,非一時也。五帝之世,則有獯鬻焉;其在三代,則獫狁焉;逮乎兩漢,則匈奴焉;當塗、典午,則烏丸、鮮卑焉;後魏及周,則蠕蠕、突厥。此其酋豪相繼,互為君長者也。皆以畜牧為業,侵抄為資,倏來忽往,雲飛鳥集。智謀之士,議和親於廟堂之上;折衝之臣,論奮擊
 於塞垣之下。然事無恒規,權無定勢,親疏因其強弱,服叛在其盛衰,衰則款塞頓顙,盛則率兵寇掠。屈伸異態。彊弱相反。正朔所不及,冠帶所不加。唯利是視,不顧盟誓,至於莫相救護,驕黠憑陵。和親結約之謀,行師用兵之事,前史論之備矣,故不詳而究焉。



 及蠕蠕衰微,突厥始大,至於木桿,遂雄朔野。東極東胡舊境,西盡烏孫之地,彎弓數十萬,列處於代陰,南向以臨周、齊。二國莫之能抗,爭請盟好,求結和親。



 乃與周合從,終亡齊國。隋文遷鼎,厥徒孔熾,負其眾力,將蹈秦郊。內自相圖,遂以乖亂,達頭可汗遠遁,啟人願保塞下。於是推亡固存,返其舊
 地,追討餘燼,部眾遂彊,卒於仁壽,不侵不叛。暨乎始畢,未虧臣禮。煬帝撫之非道,始有鴈門之圍,俄屬群盜並與,於此浸以雄盛。豪傑雖建名號,莫不請好息人。於是分置官司,總統中國,子女玉帛,相繼於道,使者之車,往來結轍。自古蕃夷驕僭,未有若斯之甚也。



 及聖哲應期,掃除氛祲。暗於時變,猶懷抵拒,率其群醜,屢隳亭鄣,殘敗我雲、代,搖蕩我太原,肆掠於涇陽,飲馬於渭汭。太宗文皇帝奇謀內運,神機密動,遂使百世不羈之虜,一舉而滅。瀚海龍庭之地,盡為九州;幽都窮髮之鄉,隸於編戶。實帝皇所不及,書契所未聞。由此言之,雖天道有盛
 衰,亦人事之工拙也。加以為而弗恃,有而弗居,類天地之含容,同陰陽之化育,斯乃大道之行也,固無得而稱焉。



\end{pinyinscope}