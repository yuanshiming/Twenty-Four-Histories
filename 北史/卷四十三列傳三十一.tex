\article{卷四十三列傳三十一}

\begin{pinyinscope}

 郭祚張彞孫晏之曾孫乾威邢巒弟子昕族孫臧邵李崇從弟平平子獎諧郭祚,字季祐,太原晉陽人,魏車騎將軍淮弟亮之後也。祖逸,本州別駕,前後以二女妻司徒崔浩,一女妻浩弟上黨太守恬。太武時,浩親寵用事,拜逸徐州刺史,假榆次侯,贈光祿大夫。父洪之,坐浩事誅。祚亡竄得免。



 少孤貧,姿貌不偉,鄉人莫之識。有女巫相祚後當富貴。祚涉
 歷經史,習崔浩之書,尺牘文章見稱於世。弱冠為州主簿,刺史孫小委之書記。又太原太守王希彥,逸妻之姪也,共相賙恤,乃振。孝文初,舉秀才,對策上第,拜中書博士。轉中書侍郎,遷尚書左丞,長兼給事黃門侍郎。祚清勤在公,夙夜匪懈,帝甚賞之。從南征,及還,正黃門。車駕幸長安,行經渭橋,過淮廟,問祚曰:「是卿祖宗所承邪?」



 祚曰:「是臣七世伯祖。」帝曰:「先賢後哲,頓在一門。」祚對曰:「昔臣先人以通儒英博,唯事魏文。微臣虛薄,遭奉聖明,自惟幸甚。」因敕以太牢祭淮廟,令祚自撰祭文。以贊遷洛之規,賜爵東光子。孝文曾幸華林園,因觀故景陽山。
 祚曰:「山以仁靜,水以智流,願陛下修之。」帝曰:「魏明以奢失於前,朕何為襲之於後?」祚曰:「高山仰止。」帝曰:「得非景行之謂?」遷散騎常侍,仍領黃門。



 是時,孝文銳意典禮,兼銓鏡九流,又遷都草創,征討不息;內外規略,號為多事。祚與黃門宋弁參謀帷幄,隨其才用,各有委寄。祚承稟注疏,特成勤劇。嘗以立馮昭儀,百官夕飲清徽後園。孝文舉觴賜祚及崔光曰:「郭祚憂勤庶事,獨不欺我。崔光溫良博物,朝之儒秀。不勸此兩人,當勸誰也!」其見知若此。初,孝文以李彪為散騎常侍,祚因入見,帝謂祚曰:「朕昨誤授一人官。」祚對曰:「豈容聖詔一行,而有差異!」帝沈
 吟曰:「此自應有讓,因讓,朕欲別授一官。」須臾,彪有啟云:「伯石辭卿,子產所惡,臣欲之已久,不敢辭讓。」帝歎謂祚曰:「卿之忠諫,李彪正辭,使朕遲回,不能復決。」遂不換李彪官也。



 乘輿南討,祚以兼侍中從,拜尚書,進爵為伯。孝文崩,咸陽王禧等奏祚兼吏部尚書。尋除長兼吏部尚書、并州大中正。宣武詔以姦吏逃刑,縣配遠戍,若永避不出,兄弟代之。祚奏曰:「若以姦吏逃竄,徙其兄弟,罪人妻子,復應徙之,此則一人之罪,禍傾二室。愚謂罪人既逃,止徙妻子,走者之身,縣名永配,於眚不免,姦途自塞。」詔從之。尋正吏部。祚持身潔清,重惜官位。至於銓授,假
 令得人,必徘徊久之,然後下筆,下筆即云:「此人便以貴矣。」由是事頗稽滯,當時每招怨讟。然所拔用者,皆量才稱職,時又以此歸之。



 出為使持節、鎮北將軍、瀛州刺史。及太極殿成,祚朝於京師,轉鎮東將軍、青州刺史。祚逢歲不稔,闔境饑弊,矜傷愛下,多所振恤,雖斷決淹留,號為煩緩,然士女懷其德澤。入為侍中、金紫光祿大夫、并州大中正。遷尚書右僕射。



 時議定新令,詔祚與侍中、黃門參議刊正。故事,令、僕、中丞騶唱而入宮門,至於馬道。及祚為僕射,以為非盡敬之宜,言於帝,納之。下詔御在太極,騶唱至止車門;御在朝堂,至司馬門。騶唱不入宮,
 自此始也。詔祚本官領太子少師。祚曾從幸東宮,明帝幼弱,祚持一黃出奉之。時應詔左右趙桃弓與御史中尉王顯迭相脣齒,深為帝所信,祚私事之,時人謗祚者,號為桃弓僕射、黃少師。



 祚奏曰:「謹案前後考格,雖班天下,如臣愚短,猶有未悟。今須定職人遷轉由狀,超越階級者即須量折。景明初考格,五年者得一階半。正始中,故尚書、中山王英奏考格,被旨:「但可正滿三周為限,不得計殘年之勤。」又去年中,以前二制不同,奏請裁決。旨云:「黜陟之體,知依舊來恆斷。」今未審舊來之旨,為從景明之斷?為從正始為限?景明考法,東西省文武閑
 官悉為三等,考同任事。而前尚書盧昶奏,上等之人三年轉半階。今之考格,復分為九等,前後不同,參差無準。」詔曰:「考在上中者,得泛以前,有六年以上遷一階,三年以上遷半階,殘年悉除。考在上下者,得汎以前,六年以上遷半階,不滿者除。其得泛以後,考在上下者,三年遷一階。散官從盧昶所奏。」



 祚又奏言:「考察令:公清獨著、德績超倫而無負殿者為上上,一殿為上中,二殿為上下,累計八殿,品降至九。未審今諸曹府寺,凡考,在事公清,然才非獨著;績行稱務,而德非超倫;幹能粗可,而守平堪任;或人用小劣,處官濟事并全無負殿之徒:為依何
 第?景明三年以來,至今十有一載,準限而判,三應升退。今既通考,未審為十年之中,通其殿最,積以為第?隨前後年斷,各自除其善惡而為升降?且負注之章,數成殿為差,此條以寡愆為最,多戾為殿。未審取何行是寡愆?



 何坐為多戾?結累品次,復有幾等?諸文案失衷應杖十者為一負,罪依律次,過隨負記。十年之中,三經肆眚,赦前之罪,不問輕重,皆蒙宥免。或為御史所彈,案驗未周,遇赦復任者,未審記殿得除以不?」詔曰:「獨著、超倫及才備、寡咎,皆謂文武兼上上之極言耳。自此以降,猶有八等,隨才為次,令文已具。其積負累殿及守平得濟,皆含
 在其中,何容別疑也?所云通考者,據總多年之言。至於黜陟之體,自依舊來年斷,何足復請。其罰贖已決之殿,固非免限。遇赦免罪,準其殿者除之。



 尋加散騎常侍。時詔營明堂、國學,祚奏曰:「今雲羅西舉,開納岷、蜀;戎旗東指,鎮靖淮、荊;漢、沔之間,復須防捍。徵兵發眾,所在殷廣。邊郊多壘,烽驛未息,不可於師旅之際,興板築之功。且獻歲云暨,東作將始。臣愚量謂宜待豐靖之年,因子來之力,可不時而就。」從之。



 宣武末年,每引祚入東宮,密受賞賚,多至百餘萬,雜以錦繡。又特賜以劍杖,恩寵甚深。遷左僕射。先是,梁將康絢遏淮,將灌揚、徐。祚表曰:「蕭衍
 狂狡,擅斷川瀆,役苦人勞,危亡已兆。宜敕揚州選一猛將,遣當州之兵,令赴浮山,表裹夾攻。」朝議從之。除使持節、散騎常侍、都督、雍州刺史、征西將軍。



 太和以前,朝法尤峻,貴臣蹉跌,便致誅夷。李沖之用事也,欽祚識幹,薦為左丞,又兼黃門,意便滿足。每以孤門,往經崔氏之禍,常慮危亡,苦自陳挹,辭色懇然,發於誠至。沖謂之曰:「人生有運,非可避也。但當明白當官,何所顧畏。」



 自是積十數年,位秩隆重,而進趣之心,更復不息。又以東宮師傅之資,列辭尚書,志在封侯之賞,儀同之位。尚書令、任城王澄為之奏聞。及為征西、雍州,雖喜外撫,尚以府號不
 優,心望加大。執政者頗怪之。



 於時領軍于忠恃寵驕恣,崔光之徒,曲躬承接。祚心惡之,乃遣子太尉從事中郎景尚說高陽王雍,令出忠為州。忠聞而大怒,矯詔殺祚。祚達於政事。凡所經履,咸為稱職,每有斷決,多為故事。名器既重,時望亦深,一朝非罪見害,遠近莫不惋惜。靈太后臨朝,遣使弔慰,追復伯爵。正光中,贈使持節、車騎將軍、儀同三司、雍州刺史,謚文貞公。初,孝文之置中正,從容謂祚曰:「並州中正,卿家故應推王瓊也。」祚退謂寮友曰:「瓊真偽今自未辨,我家何為減之?然主上直信李沖吹噓之說耳。」祚死後三歲而于忠死,見祚為祟。



 祚子
 景尚,字思和。涉歷書傳,曉星歷占候,言事頗驗。初為彭城王中軍府參軍,遷員外郎、司徒主簿、太尉從事中郎。公強當世,善事權寵,世號曰郭尖。位中書侍郎,未拜而卒。景尚弟慶禮,位通直郎。慶禮子元貞,武定末,定州驃騎府長史。



 張彞,字慶賓,清河東武城人也。曾祖幸,慕容超東牟太守。歸魏,賜爵平陸侯,位青州刺史。祖準之襲,又為東青州刺史。父靈真,早卒。



 彞性公強有風氣,歷覽經史,襲祖侯爵。與盧陽烏、李安人等結為親友,往來朝會,常相追隨。陽烏為主客令,安人與彞並散令。彞少而豪放,出入
 殿庭,步眄高上,無所顧忌。文明太后雅尚恭謹,因會次見其如此,遂召集百寮督責之,令其修悔,而猶無悛改。善於督察,每有所巡檢,彞常充其選,清慎嚴猛,所至人皆畏伏,儔燈亦以此高之。遷主客令,例降侯為伯,轉太中大夫,仍行主客曹事,尋為黃門。後從駕南征,母憂解任。彝居喪過禮,送葬自平城達家,千里步從,不乘車馬,顏貌瘦瘠,當世稱之。孝文幸冀州,遣使弔慰,詔以驍騎將軍起之,還復本位。



 以參定遷都之勳,進爵為侯。轉太常少卿,遷散騎常侍,兼侍中,持節巡察陜東河南十二州,甚有聲稱。使還,以從征之勤,遷尚書。坐舉元昭為兼
 郎中,黜為守尚書。宣武初,除正尚書,兼侍中,尋正侍中。



 宣武親政,罷六輔。彞與兼尚書邢巒聞處分非常,懼,出京奔走。為御史中尉甄琛所彈,云「非武非兕,率彼曠野。」詔書切責之。尋除安西將軍、秦州刺史。



 彞務尚典式,考訪故事,及臨隴右,彌加討習。於是出入直衛,方伯羽儀,赫然可觀。羌、夏畏伏,憚其威整;一方肅靜,號為良牧。其年冬,太極初就,彞與郭祚等俱以勤舊被徵。及還州,進號撫軍將軍。彞表解州任,詔不許。



 彞敷政隴右,多所制立,宣布新風,革其舊俗,人庶愛仰之。為國造佛寺,名曰興皇,諸有罪咎者,隨其輕重,謫為土木之功,無復鞭杖
 之罰。時陳留公主寡居,彞意願尚主,主亦許之。僕射高肇亦望尚主,主意不可。肇怒,譖彞擅立刑法,勞役百姓,詔遣直後萬貳興馳驛檢察。貳興,肇所親愛,必欲致彞深罪。彝清身奉法,求其愆過,遂無所得。見代還洛,猶停廢數年。



 因得偏風,手腳不便;然志性不移,善自將攝,稍能朝拜。久之,除光祿大夫,加金章紫綬。彞愛好知己,輕忽下流,非其意者,視之蔑爾。雖疹疾家庭,而志氣彌高。上《歷帝圖》五卷,起元庖犧,終於晉末,凡十六代,一百二十八帝,歷三千二百七十年,雜事五百八十九。宣武善之。



 明帝初,侍中崔光表:「彞及李韶,朝列之中,唯此二人,
 出身官次,本在臣右,器能幹世,又並為多。而近來參差,便成替後。計其階途,雖應遷陟,然恐班秩,猶未賜等。昔衛之公叔,引下同舉;晉之士丐,推長伯游。古人所高,當時見許。敢緣斯義,乞降臣位一階,授彼汎級。」詔加征西將軍、冀州大中正。



 雖年向六十,加之風疹,而自強人事,孜孜無怠。公私法集,衣冠從事,延請道俗,修營齋講。好善欽賢,愛獎人物,南北親舊,莫不多之。大起第宅,微號華侈。頗侮其疏宗舊戚,不甚存紀,時有怨憾焉。榮宦之間,未能止足,屢表在秦州豫有開援漢中之勛,希加賞報,積年不已,朝廷患之。



 第二子仲瑀上封事,求銓別選
 格,排抑武人,不使預在清品。由是眾口喧喧,謗讟盈路,立榜大巷,克期會集,屠害其家。彞殊無畏避之意,父子安然。神龜二年二月,羽林武賁將幾千人,相率至尚書省詬罵,求其長子尚書郎始均不獲,以瓦石擊打公門。上下懾懼,莫敢討抑。遂持火虜掠道中薪蒿,以杖石為兵器,直造其第,曳彞堂下,捶撻極意,唱呼焚其屋宇。始均、仲瑀當時踰北垣而走。始均回救其父,拜伏群小,以請父命。羽林等就加毆擊,生投之於煙火中,及得尸骸,不復可識,唯以髻中小釵為驗。仲瑀走免。彞僅有餘命,沙門寺與其比鄰,輿致於寺。



 遠近聞見,莫不惋駭。乃卒。
 官為收掩羽林凶強者八人斬之。不能窮誅群豎,即為大赦,以安眾心,有識者知國紀之將墜矣。



 喪還所焚宅,與始均東西分斂於小屋。仲瑀遂以創重,避居滎陽。至五月得漸瘳,始奔父喪,詔賜以布帛。靈太后以其累朝大臣,特垂矜惻,數月猶追言泣下,謂諸侍臣曰:「吾為張彞飲食不御,乃至首髮微有虧落。」悲痛之若此。



 初,彞曾祖幸所招引河東人為州,裁千餘家。後相依合,旋罷入冀州。積三十年,析別有數萬戶。故孝文比校天下人戶,最為大州。彝為黃門,每侍坐,以為言。



 孝文謂之曰:「終當以卿為刺史,酬先世誠效。」彞追孝文往旨,累乞本州,朝
 議未許。彞亡後,靈太后云:「彝屢乞冀州,吾欲用之,有人違我此意。若從其請,或不至是,悔之無及。」乃贈使持節、衛將軍、冀州刺史,謚文侯。



 始均字子衡,端潔好學,才幹有美於父。改陳壽《魏書》為編年之體,廣益異聞為三十卷。又著《冠帶錄》及諸詩賦數十篇,並亡失。初,大乘賊起於冀、瀛之間,遣都督元遙討平之,多所殺戮,積尸數萬。始均以郎中為行臺,忿軍士以首級為功,令檢集人首數千,一時焚爇,至於灰燼,用息僥倖,見者莫不傷心。及始均之死也,始末在煙炭之間,有焦爛之痛,論者或亦推咎焉。贈樂陵太守,謚曰孝。



 子皓之,襲祖爵。武定中,開
 府主簿,齊受禪,爵例降。皓之弟晏之。



 晏之字熙德幼孤,有至性,為母鄭氏教誨,動依禮典。從爾朱榮平元顥,賜爵武城子。累遷尚書二千石郎中。高岳征潁川,復以為都督中兵參軍,兼記室。晏之文士,兼有武幹。每與岳帷帳之謀,又嘗以短兵接刃,親獲首級,深為岳所嗟賞。



 齊天保初,文宣為高陽王納晏之女為妃,令赴晉陽成禮。晏之後園陪宴,坐客皆賦詩。晏之詩云:「天下有道,主明臣直;雖休勿休,永貽世則。」文宣笑曰:「得卿箴諷,深以慰懷。」後行北徐州事,尋即真,為吏人所愛。御史崔子武督察州郡,至北徐,無所案劾,唯得百姓
 制《清德頌》數篇,乃歎曰:「本求罪狀,遂聞頌聲。」



 遷兗州刺史,未拜,卒。贈齊州刺史、太常卿。子乾威。



 乾威字元敬,性聰敏。涉獵群書,其世父皓謂人曰:「吾家千里駒也。」仕齊,位太常丞。仕周,為宣納中士。隋開皇中,累遷晉王屬。王甚美其才,與河內張衡俱見禮重,晉邸稱為二張焉。及王為太子,遷員外散騎侍郎、太子內舍人。煬帝即位,授內史舍人、儀同三司,又以籓邸之舊,加開府。尋拜謁者大夫,從幸江都,以本官攝江都贊務,稱為幹理。乾威嘗在塗,見一遺囊,恐其主求失,因令左右負之而行。後數日,物主來認,悉以付之。淮南太守楊
 綝嘗與十餘人同來謁見,帝問乾威曰:「其首立者為誰?」乾威下殿就視而答曰:「淮南太守楊綝。」帝謂乾威曰:「卿為謁者大夫,而乃不識參見人,何也?」乾威對曰:「臣非不識楊綝,但慮不審,所以不敢輕對。石建數馬足,蓋慎之至。」其廉慎皆此類也。帝甚嘉之。



 于時帝數巡幸,百姓疲弊,乾威因上封事以諫,帝不悅,自此見疏。未幾卒官。有子爽。仕至蘭陵令。



 乾威弟乾雄,亦有才器。秦孝王俊為秦州總管,選為法曹參軍。王嘗親案囚徒,乾雄誤不持狀,口對百餘人,皆盡事情,同輩莫不歎能。後歷壽春、陽城二縣令,俱有政績。



 邢巒,字洪賓,河間鄚人,魏太常貞之後也。族五世祖嘏,石勒頻征不至。嘏無子,巒高祖蓋自旁宗入後。蓋孫穎,字宗敬,以才學知名。太武時,與范陽盧玄等同征。後拜中書侍郎,改通直常侍、平城子使宋。還,以病歸鄉。久之,帝曰:「往憶邢穎長者,有學義,宜侍講東宮,今安在?」司徒崔浩曰:「穎臥病在家。」



 帝遣太醫馳驛就療。卒,贈定州刺史,謚曰康,子修年,即巒父也,位州主簿。



 巒少好學,負帙尋師,守貧厲節,遂博覽書傳,有文才幹略。美鬚髯,姿貌甚偉。累遷兼員外散騎常侍,使齊。還,再遷中書侍郎,甚見顧遇,常參坐席。孝文因行藥至司空府南,見巒宅,謂
 巒曰:「朝行藥至此,見卿宅乃住。東望德館,情有依然。」巒對曰:「陛下移構中京,方建無窮之業。臣意在與魏升降,寧容不務永年之宅。」帝謂司空穆亮、僕射李沖曰:「巒之此言,其意不小。」有司奏策秀、孝,詔曰:「秀、孝殊問,經、權異策,邢巒才清,可令策秀。」後兼黃門郎,從征流北。



 巒在新野,後至。帝曰:「伯玉天迷其心,鬼惑其慮,守危邦,固逆主。至此以來,雖未禽滅,城隍已崩,想在不遠。所以緩攻者,正待中書為露布耳。」尋除正黃門,兼御史中尉、瀛州大中正,遷散騎常侍,兼尚書。



 宣武時,巒奏曰:「先皇深觀古今,去諸奢侈,服御尚質,不貴彫鏤,所珍在素,不務奇綵,
 至乃紙絹為帳扆,銅鐵為轡勒,訓朝廷以節儉,示百姓以憂矜。逮景明之初,承升平之業,四疆清晏,遠近來同。於是蕃貢繼路,商估交入,諸所獻貿,倍多於常。雖加以節約,猶歲損萬計,珍貨常有餘,國用恆不足。若不裁其分限,便恐無以支歲。自今非為要須者,請皆不受。」帝從之。尋正尚書。



 梁、溱二州行事夏侯道遷以漢中內附,詔加巒使持節、都督征梁、漢諸軍事,進退征攝,得以便宜從事。巒至漢中,遣兵討之,賊畢款附,乘勝追奔至關城之下。



 詔拜巒使持節、梁、秦二州刺史。於是開地定境,東西七百,南北千里,獲郡十四。



 二部護軍及諸縣戍,遂
 逼涪城。



 巒表曰:「揚州、成都,相去萬里。陸途既絕,唯資水路。水軍西上,非周年不達。外無軍援,一可圖也。益州頃經劉季連反叛,鄧元起攻圍,倉庫空竭,無復固守之意,二可圖也。蕭深藻是裙屐少年,未洽政務;今之所任並非宿將重名,皆是左右少年而已,三可圖也。蜀之所恃,惟阻劍閣。今既克南安,已奪其險,據彼界內,三分已一。從南安向涪,方軌任意,前軍累破,後眾喪魂,四可圖也。深藻是蕭衍兄子,骨肉至親,若其逃亡,當無死理。脫軍走涪城,深藻何肯城中坐而受困?五可圖也。臣聞乘機而動,武之善經,未有捨干戚而康時,不征伐而統一。臣
 以不才,屬當戎寄,上憑國威,頻有薄捷,瞻望涪、益,旦夕可屠,正以兵少糧匱,未宜前出。今若不取,後圖便難。輒率愚管,必將殄克。如其無功,分受憲坐。若朝廷未欲經略,臣便為無事,乞歸侍養,微展烏鳥。」



 巒又表曰:「昔鄧艾、鐘會率十八萬眾,傾中國資給,裁得平蜀。所以然者,鬥實力也。況臣才絕古人,何宜請二萬之眾而希平蜀?所以敢者,正以據得要險,士庶慕義。此往則易,彼來則難,任力而行,理有可克。今王足前進,已逼涪城。



 脫得涪城,則益州便是成禽之物。臣誠知征戎危事,未易可為,自軍度劍閣以來,須髮中白。所以勉強者,既到此地而自
 退不守,恐孤先皇之恩遇,負陛下之爵祿。



 是以孜孜,頻有陳請。」宣武不從。又王足於涪城輒還,遂不定蜀。



 巒既克巴西,遣軍主李仲遷守之。仲遷得梁將張法養女,有美色,甚惑之。散費兵儲,專心酒色,公事諮承,無能見者。巒忿之切齒。仲遷懼,謀叛。城人斬其首以降梁將譙希遠,巴西遂沒。武興氐楊集起等反,巒遣統軍傅豎眼討平之。巒之初至漢中,從容風雅,接豪右以禮,撫從庶以惠。歲餘之後,頗因其去就,誅滅百姓,籍為奴婢者二百餘口,兼商販聚斂,清論鄙之。征授度支尚書。



 時梁人侵軼徐、兗,朝廷乃以巒為使持節、都督東討諸軍事、安東
 將軍,尚書如故。宣武勞遺巒於東堂曰:「知將軍旋京未久,膝下難違;然東南之寄,非將軍莫可。自古忠臣亦非無孝也。」巒曰:「願陛下勿以東南為慮。帝曰:「漢祖有云:『金吾擊郾,吾無憂矣。』今將軍董戎,朕何慮哉!」巒至,乃分遣將帥致討,兗州悉平;進圍宿豫,平之。帝賜巒璽書慰勉之。



 及梁城賊走,中山王英乘勝攻鐘離,又詔巒率眾會之。巒以為鐘離天險,朝貴所具,若有內應,則所不知,如其無也,必無克狀。且俗語云「耕則問田奴,絹則問織婢」,臣既謂難,何容強遣。巒既累表求還,帝許之。英果敗退,時人伏其識略。



 初,侍中盧昶與巒不平,昶與元暉俱為宣
 武所寵,御史中尉崔亮,昶之黨也,昶、暉令亮糾巒,事成,許言於宣武,以亮為侍中。亮奏巒在漢中掠良人為婢。巒懼,乃以漢中所得巴西太守龐景仁女化生等二十餘口與暉。化生等數人,奇色也。



 暉大悅,乃背昶為巒言,云巒新有大功,已經赦宥,不宜方為此獄,帝納之。高肇以巒有克敵效而為昶等所排,助巒申釋,故得不坐。



 豫州城人白早生殺刺史司馬悅,以城南入梁,遣其將齊茍仁率眾入據縣瓠。詔巒持節率羽林精騎討之。封平舒縣伯,賞宿豫之功也。宣武臨東堂勞遣巒曰:「早生走也?守也?何時平?」巒曰:「今王師若臨,士人必翻然歸順,圍
 之窮城,奔走路絕,不度此年,必傳首京師。願陛下不足為慮。」帝笑曰:「卿言何其壯哉!



 知卿親老,頻勞於外,然忠孝不俱,不得辭也。」於是巒率騎八百,倍道兼行。五日於鮑口,擊賊大將胡孝智,乘勝至縣瓠,因即度汝。既而大兵繼至,遂長圍圍之。



 詔巒使持節、假鎮南將軍,都督南討諸軍事。中山王英南討三關,亦次縣瓠,以後軍未至,前寇稍多,憚不敢進。乃與巒分兵,將掎角攻之。梁將齊茍仁等二十一人開門出降,即斬早生等同惡數十人,豫州平。巒振旅還京師,宣武臨東堂勞之。巒曰:「此陛下聖略威靈,英等將士之力,臣何功之有?」帝笑曰:「卿匪
 直一月三捷,所足稱奇。乃存士伯,讓功而弗處。」



 巒自宿豫大捷及平縣瓠,志行修正,不復以財賄為懷,戎資軍實,絲毫無犯。



 遷殿中尚書,加撫軍將軍,卒於官。巒才兼文武,朝野瞻望,上下悼惜之。贈車騎大將軍、瀛州刺史。初,帝欲贈冀州,黃門甄琛以巒前曾劾己,乃云:「瀛州巒之本郡,人情所俗。」乃從之。及琛為詔,乃云優贈車騎將軍、瀛州刺史,議者笑琛淺薄。謚曰文定。子遜。



 遜字子言,貌雖陋短,頗有風氣。襲爵後,遷國子博士、本州中正。因謁靈太后,自陳功名之子,久抱沈屈:「臣父屢為大將,而臣身無軍國階級。臣父唯為忠臣,不為慈父」。
 靈太后慨然,以遜為長兼吏部郎中。後位大司農卿,與少卿元慶哲至相糾訟。遜銳於財利,議者鄙之。卒,贈光祿勳、幽州刺史。子祖征,開府祭酒。父喪未終,謀反,伏法。祖征弟祖效,貌寢,有風尚。仕齊,卒於尚書郎。祖效弟祖俊,開府行參軍。開皇中,位尚書都官郎中。巒弟偉,尚書郎中。偉子昕。



 昕字子明,幼孤,見愛於祖母李氏。好學,早有才情,解褐盪寇將軍,累遷太尉記室參軍。吏部尚收李神俊奏昕修起居注。太昌初,除中書侍郎,加平東將軍、光祿大夫。時言冒竊官級,為中尉所劾,免官,乃為《述躬賦》。未幾,受
 詔與祕書監常景典儀注事。武帝行釋奠禮,昕與校書郎裴伯茂等俱為錄義。永熙末,昕入為侍讀,與溫子昇、魏收參掌文詔。遷鄴,乃歸河間。



 天平初,與侍中從叔子才、魏季景、魏收同征赴都,尋還鄉里,既而復徵。時梁使兼散騎常侍劉孝儀等來聘,詔昕兼正員郎,迎於境上。司徒孫騰引為中郎。尋除通直常侍,加中軍將軍。既有才藻,兼長几案。自孝昌之後,天下多務,世人競以吏工取達,文學大衰。司州中從事宋游道以公斷見知,時與昕嘲謔,昕謂之曰:「世事同知文學外。」游道有慚色。興和中,以本官副李象使於梁。昕好忤物,人謂之牛。是行也,
 談者謂之牛象鬥於江南。齊文襄王攝選,擬昕為司徒右長史,未奏,遇疾卒,士友悲之。贈車騎將軍、都官尚書、冀州刺史,謚曰文。所著文章自有集錄。



 偉弟晏,字幼平。美風儀,博涉經史,善談釋老,雅好文詠。位滄州刺史,為政清靜,吏人安之。卒,贈尚書左僕射、瀛州刺史,謚曰文貞。晏篤於義讓,初為南兗州,例得一子解褐,乃啟其孤弟子子慎為朝請。子慎年甫十二,而其子已弱冠矣。後為滄州,復啟孤兄子昕為府主簿,而其子並未從宦,世人以此多之。



 子亢,字子高,頗有文學。位兼通直散騎常侍。使於梁,時年二十八。後為中外府屬,坐事死於晉陽。



 巒叔祖祐,字宗祐。少有學尚,知名於時。假員外散騎常侍,使於宋。以將命之勤,除建威將軍、平原太守、賜爵城平男。政清刑肅,百姓安之。卒于官。



 子產,字神寶。好學善屬文,少時作《孤蓬賦》,為進所稱。舉秀才,除著作佐郎。假常侍、鄚縣子,使於齊。產仍世將命,時人美之。歷中書侍郎、太子中庶子,卒,朝廷嗟惜焉。贈平州刺史、樂城子,謚曰定。



 祐從子虯,字神彪。著作郎敏之子也。少為《三禮》鄭氏學,明經有文思。舉秀才上第,為中書議郎、尚書殿中郎。孝文因公事與語,問朝觀宴饗禮,虯以經對,大合上旨。帝崩,尚書令王肅多用新儀,虯往往折以《五經》正禮。
 為尚書左丞,多所糾正,臺閣肅然。時鴈門人有害母者,八坐奏轅之而瀦其室,宥其二子。虯駮奏云:「君親無將,將而必誅。謀逆者戳及期親,害親者令不及子,既逆甚梟鏡,禽獸之不若,而使禋祀不絕,遺育承傳,非所以勸忠孝之道,存三綱之義。若聖教含容,不加孥戮,使父子罪不相及,惡止於其身者,則宜投之四裔,敕所在不聽配匹。《盤庚》言無令易種新邑,漢法五月食梟羹,皆欲絕其類也。」奏入,宣武從之。



 後為光祿少卿。母在鄉遇患,請假歸。遇秋水暴長,河梁破絕,虯得一小船而度。船漏滿不沒,時人異之。母喪,哀毀過禮,為時所稱。卒,贈幽州刺
 史,謚曰威。虯善與人交,清河崔亮、頓丘李平並與親善。所作碑頌雜筆三十餘篇。長子臧。



 臧字子良,幼孤,早立操尚,博學有藻思。年二十一,神龜中舉秀才,考上第,為太學博士。正光中,議立明堂,臧為裴頠一室之議。事雖不行,當時稱其理博。



 出為本州中從事,雅為鄉情所附。永安初,徵為金部郎中,以疾不赴。轉除東牟太守。時天下多事,在職少能廉白,臧獨清慎奉法,吏人愛之。隴西李延寔,莊帝之舅,以太傅出除青州,啟臧為屬。領樂安內史,有惠政。後除濮陽太守,尋加安東將軍。



 臧和雅信厚,有長者之風,為時人所愛敬。為
 特進甄琛行狀,世稱其工。與裴敬憲、盧觀兄弟並結友,曾共讀《回文集》,臧獨先通之。撰古來文章并敘作者氏族,號曰《文譜》,未就,病卒,時賢悼惜之。其文筆凡百餘篇。贈鎮北將軍、定州刺史,謚曰文。



 子恕,涉學有識悟。齊武平末,尚書屯田郎。隋開皇中,尚書侍郎。卒於沂州長史。



 臧弟邵,字子才,小字吉。少時有避,遂不行名。年五歲,魏吏部郎清河崔亮見而奇之曰:「此子後當大成,位望通顯。」十歲便能屬文,雅有才思,聰明強記,日誦萬餘言。族兄巒有人倫鑒,謂子弟曰:「宗室中有此兒,非常人也。」少在洛
 陽,會天下無事,與時名勝,專以山水游宴為娛,不暇勤業。嘗霖雨,乃讀《漢書》,五日略能遍之,後因飲謔倦,方廣尋經史,五行俱下,一覽便無所遺。文章典麗,既贍且速。年未二十,名動衣冠。嘗與右北平陽固、河東裴伯茂、從兄罘、河南陸道暉等至北海王昕舍宿飲,相與賦詩,凡數十首,皆在主人奴處。旦日奴行,諸人求詩不得,邵皆為誦之。諸人有不認詩者,奴還得本,不誤一字。諸人方之王粲。



 吏部尚書隴西李神俊大相欽重,引為忘年之交。



 釋巾為魏宣武挽郎。除奉朝請,遷著作佐郎,深為領軍元叉所禮。叉新除遷尚書令,神俊與陳郡袁翻在席,
 叉令邵作謝表,須臾便就,以示諸賓。神俊曰:「邢邵此表,足使袁公變色。」孝昌初,與黃門侍郎李琰之對典朝議。



 自孝明之後,文雅大盛。邵彫蟲之美,獨步當時,每一文初出,京師為之紙貴,讀誦俄遍遠近。于時袁翻與范陽祖瑩位望通顯,文筆之美,見稱先達;以邵藻思華贍,深共嫉之。每洛中貴人拜職,多憑邵為謝章表。嘗有一貴勝初授官,大事賓食,翻與邵俱在坐,翻意主人託其為讓表。遂命邵作之,翻甚不悅。每告人云:「邢家小兒常客作章表,自買黃紙,寫而送之。」邵恐為翻所害,乃辭以疾。屬尚書令元羅出鎮青州,啟為府司馬,遂在青土,終日
 酣賞,盡山泉之致。



 永安初,累遷中書侍郎。所作詔文體宏麗。及爾朱兆入洛,京師擾亂。邵與弘農楊愔避地嵩高山。普泰中,兼給事黃門侍郎,尋為散騎常侍。太昌初,敕令恆直內省,給御史,令覆案尚書門下事,凡除大官,先問其可不,然後施行。除衛將軍、國子祭酒。以親老還鄉,詔所在特給兵力五人,並令歲一入朝,以備顧問。丁母憂,哀毀過禮。後楊愔與魏元叉及邵請置學,奏曰:二黌兩學,盛自虞、殷。所以宗配上帝,以著莫大之嚴;宣布下土,以彰則天之軌。養黃髮以詢哲言,育青衿而敷典教。用能享國長久,風徽萬祀者也。爰暨亡秦,改革其道,
 坑儒滅學,以蔽黔黎。故九服分崩,祚終二代。炎漢勃興,更修儒術。故西京有六學之義,東都有三本之盛。逮自魏、晉,撥亂相因,兵革之中,學校不絕,仰惟高祖孝文皇帝,稟聖自天,道鏡今古,列教序於鄉黨,敦詩書於郡國。



 但經始事殷,戎軒屢駕,未遑多就,弓劍弗追。世宗統曆,聿遵先緒,永平之中,大興板築。續以水旱,戎馬生郊,雖逮為山,還停一簣。而明堂禮樂之本,乃鬱荊棘之林;膠序德義之基,空盈牧豎之跡。城隍嚴固之重,闕磚石之工;墉構顯望之要,少樓榭之飾。加以風雨稍侵,漸致虧墜,非所謂追隆堂構,儀刑萬國者也。伏聞朝議以高祖
 大造區夏,道侔姬文,擬祀明堂,式配上帝。今若基宇不修,仍同丘畎,即使高皇神享,闕於國陽,宗事之典,有聲無實。此臣子所以匪寧,億兆所以佇望也。



 臣又聞官方授能,所以任事,既任事矣,酬之以祿。如此則上無曠官之議,下絕尸素之謗。今國子雖有學官之名,而無教授之實,何異免絲燕麥,南箕北哉。



 昔劉向有言寔:者宜興辟雍、陳禮樂以風天下。夫禮樂所以養人,刑法所以殺人。而有司勤勤,請定刑法,至於禮樂,則曰未敢。是敢於殺人,不敢於養人也。



 臣以為當今四海清平,九服寧晏,經國要重,理應先營,脫復稽延,則劉向之言徵矣。但
 事不兩興,須有進退。以臣愚量,宜罷尚方彫靡之作,頗省永寧土木之功,並減瑤光材瓦之力,兼分石窟鐫琢之勞,及諸事役非世急者,三時農隙,修比數條。



 使辟雍之禮,蔚爾而復興;諷誦之音,煥然而更作。美榭高墉,嚴壯於外;槐宮棘寺,顯麗於中。更明古今,重遵鄉飲,敦進郡學,精課經業。如此則元、凱可得之於上序,游、夏可致之於下國,豈不休歟。



 靈太后令曰:「配饗大禮,為國之本,比以戎馬在郊,未遑修繕,今四表晏寧,當束有司,別議經始。」累遷尚書令,加侍中。



 于時與梁和,妙簡聘使,邵與魏收及從子子明被徵入朝。當時文人,皆邵之下,但以
 不持威儀,名高難副,朝廷不令出境。南人曾問賓司:「邢子才故應是北間第一才士,何為不作聘使?」答云:「子才文辭實無所愧,但官位已高,恐非復行限。」



 南人曰:「鄭伯猷,護軍猶得將命,國子祭酒何為不可?」邵既不行,復請還故郡。



 武帝在京輔政,征之,在第為賓客。除給事黃門侍郎,與溫子昇對為侍讀。宣武富於春秋,初總朝政,崔暹每勸禮接名賢,詢訪得失,以邵宿有名望,故請徵焉。



 宣武甚親重之。多別引見。邵舊鄙崔暹無學術,言論之際,遂云暹無所知解。宣武還以邵言告暹,並道「此漢不可親近。」暹頗銜之。邵奏魏帝,發敕用妻兄李伯倫為司
 徒祭酒。詔書已出,暹即啟宣武,執其專擅,伯倫官事便寢。邵由是被疏。



 其後除驃騎、西兗州刺史。在州有善政,桴鼓不鳴,吏人姦伏,守令長短,無不知之。定陶縣去州五十里,縣令妻日暮取人斗酒束脯,邵逼夜攝令,未明而去,責其取受,舉州不識其所以。在任都不營生產,唯南兗糴粟,就濟陽食之。邵繕修觀宇,頗為壯麗;皆為之名題,有清風觀、明月樓,而不擾公私,唯使兵力。吏民為立生祠,並勒碑頌德。及代,吏人父老及媼嫗皆遠相攀追,號泣不絕。至都,除中書令。



 舊格制:生兩男者,賞羊五口,不然則絹十匹。僕射崔暹奏絕之。邵云:「此格不宜輒
 斷。句踐以區區之越,賞法:生三男者給乳母。況以天下之大而絕此條!



 舜藏金於山,不以為乏,今藏之於民,復何所損。」又準舊皆訊囚取占,然後送付廷尉。邵以為不可,乃立議曰:「設官分職,各有司存,丞相不問鬥人,虞官弓招不進。豈使尸祝兼刀匕之役,家長侵雞犬之功。」詔並從之。



 自除太常卿兼中書監,攝國子祭酒。是時朝臣多守一職,帶領二官甚少。邵頓居三職,並是文學之首,當世榮之。幸晉陽,路中頻有甘露之瑞,朝臣皆作《甘露頌》,尚書符令邵為之序。及文宣崩,凶禮多見訊訪,敕撰哀策。後授特進,卒。



 邵率情簡素,內行修謹,兄弟親姻之
 間,稱為雍睦。博覽墳籍,無不通曉。晚年尤以《五經》章句為意,窮其指要。吉凶禮儀,公私諮稟,質疑去惑,為世指南。



 每公卿會議,事關典故,邵援筆立成,證引該洽。帝命朝章,取定俄頃,詞致宏遠,獨步當時。與濟陰溫子昇為文士之冠,世論謂之溫、刑。鉅鹿魏收雖天才艷發,而年事在二人之後,故子升死後,方稱邢魏焉。雖望實兼重,不以才位傲物。脫略簡易,不修威儀,車服器用,充事而已。有齋不居,坐臥恆在一小屋。果餌之屬,或置之梁上,賓至,下而共啖。天姿質素,特安異同,士無賢愚,皆能傾接,對客或解衣覓虱,且與劇談。有書甚多,而不甚讎校。
 見人校書,笑曰:「何愚之甚!天下書至死讀不可遍,焉能始復校此。日思誤書,更是一適。」妻弟李季節,才學之士,謂子才曰:「世間人多不聰明,思誤書何由能得?」子才曰:「若思不能得,便不勞讀書。」與婦甚疏,未嘗內宿。自云嘗晝入內閣,為狗所吠,言畢便撫掌大笑。性好談賞,又不能閑獨,公事歸休,恆須賓客自伴。



 事寡嫂甚謹,養孤子恕慈愛特深。在兗州,有都信云恕疾,便憂之廢寢食,顏色貶損。及卒,人士為之傷心,痛悼雖甚,竟不再哭,賓客弔慰,抆淚而已。其高情達識,開遣滯累,東門吳以還,所未有也。有集三十卷,見行於世。邵世息大寶,有文情。孽
 子大德、大道,略不識字焉。



 李崇,字繼長,小名繼伯,頓丘人也。文成元皇后第二兄誕之子。年十四,召拜主文中散,襲爵陳留公,鎮西大將軍。孝文初,為荊州刺史,鎮上洛,敕發秦、陜二州兵送崇至理。崇辭曰:「邊人失和,本怨刺史,奉詔代之,但須一宣詔旨而已。不勞發兵自防,使人懷懼。」孝文從之。乃輕將數十騎馳到上洛,宣詔綏慰,人即帖然。邊戍掠得齊人者,悉令還之。南人感德,仍送荊州口二百許人。兩境交和,無復烽燧之警。在州四年,甚有稱績。召還京師,賞賜隆厚。



 除兗州刺史。兗土舊多劫盜,崇乃村置一樓,樓懸
 一鼓,盜發之處,雙槌亂擊,四面諸村,聞鼓皆守要路。俄頃之間,聲布百里,其中險要,悉有伏人,盜竊始發,便爾禽送。諸州置樓縣鼓,自崇始也。後例降為侯,改授安東將軍。車駕南征,詔崇副驃騎大將軍、咸陽王禧都督左翼諸軍事。徐州降人郭陸聚黨作逆,人多應之。



 崇遣高平卜冀州詐稱犯罪,逃亡歸陸,陸納之,以為謀主。數月,冀州斬陸送之,賊徒潰散。入為河南尹。



 後車駕南討漢陽,崇行梁州刺史。氐楊靈珍遣弟婆羅與子雙領步騎萬餘,襲破武興,與齊相結。詔崇為使持節、都督隴右諸軍事,率眾討之。崇槎山分進,出其不意,表裹以襲,群氐
 皆棄靈珍散歸,靈珍眾減太半。崇進據赤土。靈珍又遣從弟率五千人屯龍門,躬率精勇一萬據鷲硤。龍門之北數十里中,伐樹塞路。鷲硤之口,積大木,聚礌石,臨崖下之,以拒官軍。崇乃命統軍慕容拒率眾五千,從他路夜襲龍門,破之。崇自攻靈珍。靈珍連戰敗走,俘其妻子。崇多設疑兵,襲克武興。齊梁州刺史陰廣宗遣參軍鄭猷、王思考率眾援靈珍。崇大破之,並斬婆羅首,殺千餘人,俘獲猷等。靈珍走奔漢中。孝文在南陽,覽表大悅曰:「使朕無西顧之憂者,李崇功也。」拜梁州刺史,手詔曰:「便可善思經略,去其可除,安其可育,公私所患,悉令芟
 夷。」及錄珍偷據白水,崇擊破之,靈珍遠遁。



 宣武初,徵為右衛將軍,兼七兵尚書,轉左衛將軍、相州大中正。魯陽蠻柳北喜、魯北燕等聚眾反叛,諸蠻悉應之,圍逼湖陽。游擊將軍李暉光鎮北城,盡力捍禦。賊勢甚盛,詔以崇為使持節、都督征蠻諸軍事以討之。蠻眾數萬,屯據形要,以拒官軍。崇累戰破之,斬北燕等,徙萬餘戶於幽、並諸州。宣武追賞平氐之功,封魏昌縣伯。



 東荊州蠻樊安聚眾於龍山,僭稱大號。梁武遣兵應之。諸將擊不利,乃以崇為鎮南將軍、都督征蠻諸軍事,率步騎討之。崇分遣諸將,攻擊賊壘,連戰克捷,生禽樊安,進討西荊,諸蠻
 悉降。尋兼侍中、東道大使,黜陟能否,著賞罰之稱。出除散騎常侍、征南將軍、揚州刺史。詔曰:「應敵制變,算非一途,救左擊右,疾雷均勢。今朐山蟻寇,久結未殄,賊愆狡詐,或生詭劫,宜遣銳兵,備其不意。崇可都督淮南諸軍事,坐敦威重,遙運聲算。」



 延昌初,加侍中、車騎將軍、都督江西諸軍事。先是,壽春縣人茍泰有子三歲,遇賊亡失,數年不知所在,後見在同縣趙奉伯家。泰以狀告,各言己子,並有鄰證,郡縣不能斷。崇令二父與兒各在別處,禁經數旬,然後告之曰:「君兒遇患,向已暴死,可出奔哀也。」茍泰聞即號啕,悲不自勝;奉伯咨嗟而已,殊無痛意。
 崇察知之,乃以兒還泰,詰奉伯詐狀。奉伯款引,云先亡一子,故妄認之。



 又定州流人解慶賓兄弟,坐事俱徙揚州。弟思安背役亡歸。慶賓懼後役追責,規絕名貫,乃認城外死尸,詐稱其弟為人所殺,迎歸殯葬。頗類思安,見者莫辨。



 又有女巫陽氏自云見鬼,說思安被害之苦,飢渴之意。慶賓又誣疑同軍兵蘇顯甫、李蓋等所殺,經州訟之。二人不勝楚毒,各自款引。獄將決竟,崇疑而停之。密遣二人非州內所識者,偽從外來,詣慶賓告曰:「僕住在北州,比有一人見過寄宿。



 夜中共語,疑其有異,便即詰問,乃云是流兵背役,姓解字思安。時欲送官,苦見求
 及,稱『有兄慶賓,今往揚州相國城內,嫂姓徐。君脫矜慜,為往告報,見申委曲,家兄聞此,必重相報。今但見質,若往不獲,送官何晚?』是故相造,指申此意。君欲見雇幾何?當放賢弟。若其不信,可見隨看之。」慶賓悵然失色,求其少停。此人具以報崇,攝慶賓問之,伏引。更問蓋等,乃云自誣。數日之間,思安亦為人縛送。崇召女巫視之,鞭笞一百。崇斷獄精審,皆此類也。



 時有泉水湧於八公山頂,壽春城中有魚數從地湧出,野鴨群飛入城,與鵲爭巢。



 五月,大霖雨十有三日,大水入城,屋宇皆沒。崇與兵泊於城上,水增未已,乘船附於女墻,城不沒者二版而已。
 州府勸崇棄州保北山。崇曰:「吾受國重恩,忝守籓岳,淮南萬里,繫于吾身,一旦動腳,百姓瓦解,揚州之地,恐非國物。昔王尊慷慨,義感黃河,吾豈愛一軀,取愧千載。但憐茲士庶,無辜同死,可桴筏隨高,人規自脫。吾必守死此城。」時州人裴絢等受梁假豫州刺史,因乘大水,謀欲為亂,崇皆擊滅之。又以洪水為災,請罪解任。詔曰:「夏雨泛濫,斯非人力,何得以此辭解。今水涸路通,公私復業,便可繕甲積糧,修復城雉,勞恤士庶,務盡綏懷之略也。」崇又表解州,不聽。是時,非崇則淮南不守矣。



 崇沈深有將略,寬厚善御眾。在州凡十年,常養壯士數千人,寇賊
 侵邊,所向摧破,號曰:「臥彪」,賊甚憚之。梁武惡其久在淮南,屢設反間,無所不至。宣武雅相委重,梁無以措謀。乃授崇車騎大將軍、開府儀同三司、萬戶郡公,諸子皆為縣侯,欲以構崇。崇表言其狀。宣武屢賜璽書慰勉之。賞賜珍異,歲至五三,親待無與為比。梁武每歎息,服宣武之能任崇也。



 孝明踐阼,褒賜衣馬。及梁遣其游擊將軍趙祖悅襲據西硤石,更築外城,逼徙緣淮之人於城內。又遣二將昌義之、王神念率水軍溯淮而上,規取壽春,田道龍寇邊城,路長平寇五門,胡興茂寇開、霍。揚州諸戍,皆被寇逼。崇分遣諸將,與之相持;密裝船艦二百餘
 艘,教之水戰,以待臺軍。梁霍州司馬田休等寇建安,崇遣統軍李神擊走之。又命邊城戍主邵申賢要其走路,破之於濡水,俘斬三千餘人。靈太后璽書勞勉。許昌縣令兼糸寧麻戍主陳平王南引梁軍,以戍歸之。崇自秋請援,表至下餘,詔遣鎮南將軍崔亮救俠石,鎮東將軍蕭寶夤於梁堰上流決淮東注。朝廷以諸將不相赴,乃以尚書李平兼右僕射持節節度之。崇遣李神乘鬥艦百餘艘,沿淮與李平、崔亮合攻硤石。李神水軍剋其東北外城。祖悅力屈,乃降。朝廷嘉之,進號驃騎將軍、儀同三司,刺史、都督如故。



 梁淮堰未破,水勢日增。崇乃於硤石
 戍間編舟為橋。北更立船樓十,各高三丈;十步置一籬,至兩岸,蕃版裝治,四箱解合,賊至舉用,不戰解下。又於樓船之北,連覆大船,東西竟水,防賊火筏。又於八公山之東南,更起一城,以備大水,州人號曰魏昌城。崇累表解州,前後十餘上,孝明乃以元志代之。尋除中書監、驃騎大將軍,儀同如故。出為使持節、侍中、都督四州諸軍事、定州刺史。徵拜尚書左僕射,遷尚書令,加侍中。



 崇在官和厚,明於決斷,然性好財賄,敗肆聚斂。孝明、靈太后嘗幸左藏,王公嬪主從者百餘人,皆令任力負布絹,即以賜之。多者過二百匹,少者百餘。唯長樂公兩手持絹
 二十匹而出,示不異眾而已,世稱其廉儉。崇與章武王融以所負多,顛仆於地,崇乃傷腰,融至損腳。時人為之語曰:「陳留、章武,傷腰折股,貪人敗類,穢我明主。」



 蠕蠕主阿那瑰犯塞,詔崇以本官都督北討諸軍事以討之。崇辭於顯陽殿,戎服武飾,志氣奮揚,時年六十九,幹力如少。孝明目而壯之,朝臣莫不稱善。遂出塞三千餘里,不及賊而還。崇請改六鎮為州,兵編戶,太后不許。



 後北鎮人破落汗拔陵反,所在響應。征北將軍、臨淮王彧大敗於五原,安北將軍李叔仁尋敗於白道,賊眾日甚。詔引丞相、令、僕、尚書、侍中、黃門於顯陽殿,曰:「賊勢侵淫,寇連
 恆,朔,金陵在彼,夙夜憂惶。諸人宜陳良策。」吏部尚書元修義以為須得重貴,鎮壓壓恆、朔,總彼師旅,備衛金湯。詔曰:「去歲阿那瑰叛逆,遣李崇北征,崇遂長驅塞北,返旆榆關,此一時之盛。朕以李崇國戚望重,器識英斷,意欲還遣崇行,總督三軍,揚旌恆、朔,諸人謂可爾不?」僕射蕭寶夤等曰:「陛下此遣,實合群望。」於是詔崇以本官加使持節、開府、北討大都督,撫軍將軍崔暹、鎮軍廣陽王深皆受崇節度。又詔崇子光祿大夫神軌假平北將軍,隨崇北討。崇至五原,崔暹大敗于白道之北,賊遂並力攻崇。崇與廣陽王深力戰,累破賊眾。相持至冬,乃引
 還平城。深表崇長史祖瑩詐增功級,盜沒軍資。崇坐免官爵,徵還,以後事付深。



 後徐州刺史元法僧以彭城南叛,時除安樂王鑒為徐州刺史以討之。為法僧所敗,單馬之。乃詔復崇官爵,為徐州大都督、節度諸軍事。會崇疾篤,乃以衛將軍、安豐王延明代之。改除開府、相州刺史,侍中、將軍、儀同並如故。



 孝昌元年,薨於位。贈侍中、驃騎大將軍、司徒公、雍州刺史,謚曰武康,後重贈太尉公,餘如故。



 長子世哲,性輕率,供奉豪侈。少經征伐,頗有將用,為三關別將,討群蠻大破之。還,拜鴻臚少卿。性傾巧,善事人,亦以貨賂自達。高肇、劉騰之勢也,皆與親
 善,故世號為李錐。為相州刺史,斥逐百姓,遷徙佛寺,逼買其地,部內患之。



 崇北征之後,徵兼太常卿。御史高道穆毀發其宅,表其罪過。後除涇州刺史,賜爵衛國子。卒,贈吏部尚書、冀州刺史。



 世哲弟神軌,小名青肫,受父爵陳留侯。累出征伐,頗有將領之氣。孝昌中,靈太后淫縱,分遣腹心媼姬出外,陰求悅人。神軌為使者所薦,寵遇勢傾朝野,時云見幸帷幄,與鄭儼為雙。頻遷征東將軍、武衛將軍、給事黃門侍郎,常領中書舍人。時相州刺史、安樂王鑒據州反,詔神軌與都督源子邕等討平之。後於河陰遇害。



 建義初,贈侍中、司空公、相州刺史,謚曰烈。
 崇從弟平。



 平字雲定,少有大度;及長,涉獵群書,好《禮》、《易》,頗在文才。太和初,拜通直散騎侍郎,孝文禮之甚重。頻經大憂,居喪以孝稱。後以例降,襲爵彭城公。累遷太子庶子。平請自效一郡,帝曰:「卿復欲以吏事自試也?」拜長樂太守,政務清靜,吏人懷之。征行河南尹,豪右權戚憚之。宣武即位,除黃門郎,遷司徒左長史,行尹如故。尋正尹,長史如故。



 車騎將幸鄴,平上表諫,以為:「嵩都創構,洛邑俶營,雖年跨十稔,根基未就。代人至洛,始欲向盡,資產罄於遷移,牛畜斃於輦運;陵太行之險,越長津之難,辛勤備經,
 劣達京闕;富者猶損太半,貧者可以意知。兼歷歲從戎,不遑啟處。



 自景明以來,差得休息。事農者未積一年之儲,築室者裁有數間之屋,莫不肆力伊、瀍,人急其務。實宜安靜新人,勸其稼穡,令國有九載之糧,家有水旱之備。若乘之以羈紲,則所廢多矣。」不從。



 詔以本官行相州事。帝至鄴,親幸平第,見其諸子。尋正刺史。平勸課農桑,修飾太學,簡試通儒以充博士,選五郡聰敏者以教之。圖孔子及七十二弟子於講堂,親為立贊。前來臺使,頗好侵漁。平乃畫「履武尾,踐薄冰」於客館,注頌其下,以示誡焉。徵拜度支尚書,領御史中尉。



 冀州刺史、京兆王愉
 反於信都,以平為持節、都督北討諸軍事、行冀州以討之。



 宣武臨式乾殿勞遣平,因曰:何圖今日,言及斯事!」歔欷流涕。平對曰:「愉天迷其心,構此梟悖。陛下不以臣不武,委以總督之任。如其稽顙軍門,則送之大理。



 若不悛待戮,則鳴鼓釁鉦,非陛下之事。」平進次經縣,諸軍大集。夜有蠻兵數千斫平前壘,矢及平帳,平堅臥不動,俄而乃定。遂至冀州城南十六里,大破逆眾,逐北至城門,遂圍城。愉與百餘騎突門走,平遣統軍叔孫頭追之。去信都八十里,禽愉。冀州平,以本官領相州大中正。



 平先為尚書令高肇、侍御史王顯所恨,後顯代平為中尉,平加
 散騎常侍。顯劾平在冀州隱截官口,肇又扶成其狀,奏除平名。延昌初,詔復官爵,除定、冀二州刺史。前來良賤之訟,多有積年不決;平奏不問真偽,一以景明年前為限,於是諍訟止息。武川鎮人飢,鎮將任款請貸未許,擅開倉振恤,有司繩以費散之條,免其官爵。平奏款意在濟人,心無不善,帝原之。遷中書令,尚書如故。孝明初,轉吏部尚書。



 平高明強濟,所在有聲,但以性急為累。尚書令、任城王澄奏理平定冀之勳,靈太后乃封武邑郡公,賜縑二千五百匹。



 先是,梁遣其趙祖悅逼壽春,鎮南崔亮攻之。未剋,又與李崇乖貳。詔平以本官使持節、鎮
 軍大將軍,兼尚書左僕射為行臺,節度諸軍,東西州將,一以稟之,如有乖異,以軍法從事。詔平長子獎以通直郎從。於是率步騎二千赴壽春,嚴勒崇、亮,令水陸兼備,剋期齊舉。崇、亮憚之,無敢乖互。頻日交戰,破賊軍。安南將軍崔延伯立橋於下蔡,以拒賊之援,賊將王神念、昌義之等不得進救。祖悅守死窮城,平乃部分攻之,斬祖悅,送首於洛。以功遷尚書右僕射,加散騎常侍。平還京師,靈太后見於宣光殿,賜以金裝刀仗一口。



 時南徐州表云:梁堰淮水,日為患。詔公卿議之。平以為不假兵力,終自毀壞。



 及淮堰破,太后大悅,引群臣入宴,敕平前,孝
 明手賜縑布百段。卒,遺令薄葬。



 詔給東園秘器、朝服一具、衣一襲、帛七百匹。靈太后為舉哀於東堂。贈侍中、驃騎大將軍、儀同三司、冀州刺史,謚文烈公。平自在度支,至於端副,夙夜在公,孜孜匪懈,凡處機密十有餘年,有獻替之稱。所制文筆別有集錄。長子獎襲。



 獎字遵穆,容貌魁偉,有當世才度。位中書侍郎、吏部郎中。以本官兼尚書,出為相州刺史。初,元叉擅朝,獎為其親待,頻居顯職。靈太后反政,削除官爵。



 孝莊初,為散騎常侍、河南尹。獎前後所歷,皆以明濟著稱。元顥入洛,顥以獎兼尚書右僕射,慰勞徐州。羽林及城人不承顥旨,
 害獎,傳首洛陽。孝武帝初,獎故吏宋游道上書理獎,詔贈冀州刺史。子構襲。



 構字祖基,少以方正見稱,襲爵武邑郡公。齊天保初,降爵為縣侯,位終太府卿,贈吏部尚書。構早有名譽,歷官清顯,常以雅道自居,甚為名流所重。子丕,有父風,位尚書祠部郎中。丕弟克,通直散騎常侍。獎弟諧。



 諧字虔和,幼有風采。趙郡李搔嘗過元叉門下,見之,歸謂其父元忠曰:「領軍門下見一神人。」元忠曰:「必李諧也。」問之果然。襲父先爵彭城侯。文辯為時所稱,歷位位中書侍郎。



 天平末,魏欲與梁和好,朝議將以崔甗為使主。甗
 曰:「文採與識,甗不推李諧;口頰々,諧乃大勝。」於是以諧兼常侍、盧元明兼吏部郎、李業興兼通直常侍聘焉。梁武使朱異覘客,異言諧、元明之美。諧等見,及出,梁武目送之,謂左右曰:「朕今日遇勍敵,卿輩常言北間都無人物,此等何處來?」謂異曰:「過卿所談。」是時鄴下言風流者,以諧及隴西李神俊、范陽盧元明、北海王元景、弘農楊遵彥、清河崔贍為首。初通梁國,妙簡行人,神俊位已高,故諧等五人繼踵,而遵彥遇疾道還,竟不行。既南北通好,務以俊乂相矜,銜命接客,必盡一時之選,無才地者不得與焉。梁使每入,鄴下為之傾動,貴勝子弟盛飾
 聚觀,禮贈優渥,館門成市。宴日,齊文襄使左右覘之,賓司一言制勝,文襄為之拊掌。魏使至梁,亦如梁使至魏,梁武親與談說,甚相愛重。諧使還後遷秘書監,卒於大司農。



 諧為人短小,六指,因癭而舉頤,因跛而緩步,因謇而徐言,人言李諧善用三短。文集十餘卷。



 諧長子岳,字祖仁,官中散大夫。性純至,居期慘,未曾聽婢過前;追思二親,言則流涕。



 岳弟庶,方雅好學,甚有家風。歷位尚書郎、司徒掾,以清辯知名。常攝賓司,接對梁客,梁客徐陵深歎美焉。庶生而天閹,崔諶調之曰:「教弟種鬚,以錐遍刺作孔,插以馬尾。」庶曰:「先以此方回施貴族,藝眉有效,
 然後樹鬚。,」世傳諶門有惡疾,以呼沱為墓田,故庶言及之。邢子才在傍大笑。除臨漳令。



 《魏書》之出,庶與盧斐、王松年等訟共不平。魏收書王慧龍自云太原人,又書王瓊不善事;以盧同附《盧玄傳》;李平為陳留人,云其家貧賤。故斐等喧訟,語楊愔云:「魏收合誅。愔黨助魏,故遂白齊文宣,庶等並髡頭鞭杖二百,庶死於臨漳獄中。庶兄岳痛之,終身不歷臨漳縣門。



 庶妻,元羅女也。庶亡後,岳使妻伴之寢宿。積五年,元氏更適趙起。嘗夢庶謂己曰:「我薄福,託劉氏為女,明旦當出,彼家甚貧,恐不能見養。夫妻舊恩,故來相見告,君宜乞取我。劉家在七帝坊十
 字銜南,東入窮巷是也。」元氏不應,庶曰:「君似懼趙公意,我自說之。」於是起亦夢焉。起寤問妻,言之符合。遂持錢帛躬往求劉氏,如所夢得之,養女長而嫁焉。



 庶弟蔚,少清秀,有襟期倫理,涉觀史傳,兼屬文詞。昆季並尚風流,長裾廣袖,從容甚美,然頗涉疏放。唯蔚能自持公幹理,甚有時譽。坐兄庶事徙平州。後還,位尚書左中兵郎中,仍聘陳使副。江南以其父曾經將命,甚重焉。還,坐將人度江私市,除名。後卒於祕書丞,士友悼惜之。



 蔚弟若,聰敏,頗傳家業,風采詞令,有聲鄴下。坐兄庶事徙臨海。乾明初,追還,後兼散騎常侍。大被親狎,加儀同三司。若性
 滑稽,善諷誦,數奉旨詠詩,並使說外間世事可笑樂者。凡所話談,每多會旨。嘗在省中,趨而前卻,對答學奏事之象,和士開聞而奏之。帝每狎弄之。武成以斛律金舊老,每朝,賜羊車上殿。



 金曾使人奉啟,若為舍人,誤奏云在闕下,詔命出羊車。若重思,知金不至,竊言:「羊車、鹿車何所迎?」帝聞,亦笑而不責。又帝於後園講武,令若為吳將,皇后皆出,引若當前,觀其進止俯仰。事罷,遣使謝之,厚加賞賜。韓長鸞等忌惡之,密構其短,坐免官。未幾,詔復本官。隋開皇中,卒於秦王府諮議。



 諧弟邕,字修穆,幼而俊爽,有逸才。位高陽王雍友。幾所交游,皆倍年俊秀。



 卒,
 贈洛州刺史,謚曰文。



 論曰:郭祚才幹敏實,有世務之長。孝文經綸之始,獨在勤勞之地。居官任事,可稱述焉。張彝風力謇謇,有王臣之氣,銜命擁旄,風聲克舉。俱魏氏器能之臣乎!



 遭隨有命,二子俱逢世亂,悲哉!晏之、乾威,可謂亡焉不絕。邢巒以文武才策,當軍國之任,內參機揆,外寄折衝,其緯世之器歟!子才少有盛名,鼓動京洛,文宗學府,獨秀當年,舉必任真,情無飾智,疏通簡易,罕見其人,足為一代之模楷也。及明崔甗之謗言,執侯景之姦使,昔人稱孟軻為勇,於文簡公見之。唯嘗短崔暹,頗為累德。阮籍未嘗
 品藻人物,斯亦良有以焉。李崇風質英重,毅然秀立,任當將相,望高朝野。平以高明幹略,效智於時,出入當官,功名克著,贊務之材也。



 諧風流文辯,蓋人望乎!



\end{pinyinscope}