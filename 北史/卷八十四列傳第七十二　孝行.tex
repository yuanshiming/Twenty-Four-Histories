\article{卷八十四列傳第七十二 孝行}

\begin{pinyinscope}

 長孫慮乞伏保孫益德董洛生楊引閻元明吳悉達王續生李顯達倉跋張升王崇郭文恭荊可秦族皇甫遐張元王頒弟頍楊慶田翼紐因劉仕俊翟普林華秋
 徐孝肅《孝經》云:「夫孝,天之經也,地之義也,人之行也。」《論語》云:「君子務本,本立而道生,孝悌也者,其為仁之本歟!」《呂覽》云:「夫孝,三皇五帝之本務,萬事之納紀也。執一術而百善至,百邪去,天下順者,其唯孝乎!」然則孝之為德至矣,其為道遠矣,其化人深矣。故聖帝明王行之於四海,則與天地合其德,與日月齊其明;諸侯卿大夫行之於國家,則永保其宗社,長守其祿位;匹夫匹婦行之於閭閻,則播徽烈於當年,揚休名於千載。是以堯、舜、湯、武居帝王之位,垂至德以敦其風;孔、墨、荀、孟稟聖賢之資,弘正道
 以勵其俗。觀其所由,在此而已矣。



 然而淳源既往,澆風愈扇,禮義不樹,廉讓莫脩。若乃綰銀黃,列鐘鼎,立於朝廷之間,非一族也;積龜貝,實倉廩,居於閭巷之內,非一家也。其於愛敬之道,則有未能備焉。哀思之節,罕有得其中焉。斯乃詩人所以思素冠,孔門有以責衣錦也。



 且生盡色養之方,終極哀思之地,厥迹多緒,其心一焉。若乃誠達泉魚,感通鳥獸,事匪常倫,斯蓋希矣。至如溫床、扇席,灌樹、負土,茍或加人,咸疾俗。



 斯固仁人君子所以興歎,哲后賢宰所宜屬心。如令明教化以救其弊,優爵賞以勸其心,存懇誠以誘其進,積歲月以求其終,則
 今之所謂少者,可以為多矣;古之所謂難者,可以為易矣。



 長孫慮等闕稽古之學,無俊偉之才。或任其自然,情無矯飾;或篤於天性,勤其四體。並竭股肱之力,咸盡愛敬之心,自足膝下之歡,忘懷軒冕之貴。不言而化,人神通感。雖或位登台輔,爵列王侯,祿積萬鐘,馬跡千駟,死之日曾不得與斯人之徒隸齒。孝之大也,不其然乎。



 案《魏書》列趙琰、長孫慮、乞伏保、孫益德、董洛生、楊引、閻元明、吳悉達、王續生、李顯達、倉跋、張昇、王崇、郭文恭為《孝感傳》,《周書》列李棠、柳檜、杜叔毗、荊可、秦族、皇甫遐、張元為《孝義傳》,《隋書》列陸彥師、田德懋、薛濬、王頒、田翼、楊慶、
 郭世俊、紐因、劉仕俊、郎方貴、翟普林、李德饒、華秋、徐孝肅為《孝義傳》。今趙琰、李棠、柳檜、杜叔毗、陵彥師、李德饒入別傳及其家傳,其餘並從此編緝,以備《孝行傳》云。



 長孫慮,代人也。母因飲酒,其父真呵叱之,誤以杖擊,便即致死。真為縣囚執,處以重坐。慮列辭尚書云:「父母忿爭,本無餘惡,直以謬誤,一朝橫禍。今母喪未殯,父命旦夕,慮兄弟五人並沖幼。慮身居長,今年十五,有一女弟,向始四歲。更相鞠養,不能保全,父若就刑,交墜溝壑。乞以身代老父命,使嬰弱眾孤,得蒙存立。」尚書奏云:「慮於父為孝子,於弟為仁兄,尋情究狀,特可矜感。」



 孝文帝詔
 特恕其父死罪,以從遠流。



 乞伏保,高車部人也。父居,獻文時為散騎常侍,領牧曹尚書,賜爵寧國侯。



 以忠謹慎密,常在左右,出內詔命。賜宮人河南宗氏,亡後,賜以宮人申氏,宋太子左率申坦兄女也。歲餘,居卒。申撫養伏保,性嚴肅,捶罵切至,而伏保奉事孝謹,初無恨色。襲父侯爵,例降為伯。稍遷左中郎將。每請祿賜,在外公私尺丈所用,無不白知。出為鄯善鎮將。申年踰八十,伏保手製馬車,親自扶接,申欣然隨之。申亡,伏保解官,奉喪還洛。復為長兼南中郎將,卒。



 孫益德,樂安人也。其母為人所害。益德童幼,為母復仇,
 還家哭於殯,以待縣官。孝文、文明太后以其幼而孝決,又不逃罪,特免之。



 董洛生,代人也。居父喪過禮,詔遣祕書中散溫紹伯奉璽書慰之,令自抑割,以全孝道。又詔其宗親,使相喻獎,勿令有滅性之譏。



 楊引,鄉郡襄垣人也。三歲喪父,為叔所養。母年九十二終,引年七十五,哀毀過禮。三年服畢,恨不識父,追服斬衰,食粥麤服,誓終身命。經十三年,哀慕不改,為郡縣鄉閭三百餘人上狀稱美。有司奏宜旌賞,復其一門,樹其純孝。詔別敕集書標揚引至行,又可假以散員之名。



 閻元明,河東安邑人也。少而至孝,行著鄉閭。太和五年,除北隨郡太守。元明以違離親養,興言悲慕。母亦慈念,泣淚喪明。悲號上訴,許歸奉養。一見其母,母目便開。刺史呂壽恩列狀上聞,詔下州郡,表為孝門,復其租調兵役,令終母年。



 母亡服終,心喪積載,每忌日,悲動傍鄰。昆弟雍和,尊卑諧穆,安貧樂道,白首同歸。



 又猗氏縣人令狐仕,兄弟四人,早喪父,泣慕十載,奉養其母,孝著鄉邑。而力田積粟,博施不已。



 又河東郡人楊風等七百五十人,列稱樂戶皇甫奴兄弟,雖沉屈兵伍,而操尚彌高,奉養繼親,甚著恭孝之稱。



 又東郡小黃縣人董吐渾、兄養,
 事親至孝,三世同居,閨門有禮。景明初,畿內大使王凝奏請標異,詔從之。



 吳悉達,河東聞喜人也。兄弟三人,年並幼小,父母為人所殺。四時號慕,悲感鄉鄰。及長報仇,避地永安。昆弟同居四十餘載,閨門和睦,讓逸競勞。雖於儉年,糊饘不繼,賓客經過,必傾所有。每守宰殯喪,私辦車牛,送終葬所。鄰人孤貧窘困者,莫不解衣輟糧,以相賑恤。鄉閭五百餘人詣州稱頌焉。刺史以悉達兄弟行著鄉里,板贈悉達父勃海太守。悉達後欲改葬,亡失墳墓,推尋弗獲。號哭之聲,晝夜不止,叫訴神祇。忽於悉達足下地陷,得父
 銘記,因遷葬曾祖已下三世九喪。



 傾盡資業,不假於人,哀感毀悴,有過初喪。有司奏聞,標閭復役,以彰孝義。



 時有齊州人崔承宗,其父於宋世仕漢中,母喪因殯彼。後青、徐歸魏,遂為隔絕。承宗性至孝,萬里投險,偷路負喪還京師。黃門侍郎孫惠蔚聞之,曰:「吾於斯人,見廉范之情矣。」於是弔贈盡禮,如舊相識。



 王續生,滎陽京縣人也。遭繼母憂,居喪,杖而後起。乃終禮制,鬢髮盡落。



 有司奏聞,宣武詔標旌門閭,甄其徭役。



 李顯達,潁川陽翟人也。父喪,水漿不入口七日,鬢髮墮落,形體枯悴。六年廬於墓側,哭不絕聲,殆於滅性。州牧
 高陽王雍以狀奏,靈太后詔表其門閭。



 倉跋,滎陽京縣人也。喪母,水漿不入口五日,吐血數升,居憂毀瘠,見稱州裏。有司奏聞,孝武帝詔標門閭。



 張昇,滎陽京縣人也。喪父,飲水絕鹽,哀毀過度,形骸枯悴,骨立而已,髮落殆盡。聲聞鄉里,盜賊不侵其閭。州表以聞,標其門閭。



 王崇,字乾邕,陽夏雍人也。兄弟並以孝稱,身勤稼穡,以養二親。仕梁州鎮南府主簿。母亡,杖而後起,鬢髮墮落。未及葬,權殯宅西。崇廬於殯所,晝夜哭泣,鳩鴿群至。有一小鳥,素質黑眸,形大於雀,棲於崇廬,朝夕不去。母
 喪闋,復丁父憂,哀毀過禮。是年夏,風雹,所經處,禽獸暴死,草木摧折。至崇田畔,風雹便止,禾麥十頃,竟無損落。及過崇地,風雹如初。咸稱至行所感。崇雖除服,仍居墓側。於其室前,生草一根,莖葉甚茂,人莫能識。至冬中,復有鳥巢崇屋,乳養三子,毛羽成長,馴而不驚。守令聞之,親自臨視。州以聞奏,標其門閭。



 郭文恭,太原平遙人也。仕為太平縣令。年踰七十,父母喪亡。文恭孝慕罔極,乃居祖父墓次,晨夕拜跪。跣足負土,培祖父二墓,寒暑竭力,積年不已。見者莫不哀歎。尚書聞奏,標其門閭。



 荊可,河東猗氏人也。性質朴,容止有異於人。能苦身勤力,供養其母,隨時甘旨,終無匱乏。母喪,水漿不入口三日,悲號擗踴,絕而後蘇者數四。葬母之後,遂廬於墓側,晝夜悲哭,負土成墳,蓬髮不櫛,菜食飲水而已。然可家舊墓,塋域極大,榛蕪至深,去家十餘里。而可獨宿其中,與禽獸雜處,哀感遠近,邑里稱之。



 大統中,可鄉人以可孝行足以勸勵風俗,乃上言焉。周文令州縣表異之。及服終之後,猶若居喪。大冢宰、晉公護聞可孝行,特引見焉。與可言論,時有會於護意。



 而護亦至孝,其母閻氏,沒於敵境,不測存亡。每見可,自傷久乖膝下,而重可至性。
 可卒後,護猶思其純孝,收可妻子於京城,恒給其衣食。



 秦族,上郡洛川人也。祖白、父雚,並有至性,聞於閭里。魏太和中,板白潁州刺史。大統中,板雚酈城郡守。族性至孝,事親竭力。及父喪,哀毀過禮,每一慟哭,酸感行路。既以母在,恒抑割哀情,以慰其母意。四時珍羞,未嘗匱乏。與弟榮先,復相友愛,閨門之中,怡怡如也。尋而其母又沒,哭泣無時,唯飲水食菜而已。終喪之後,猶蔬食,不入房室二十許年。鄉里咸歎異之。其邑人王元達等七十餘人上其狀,有詔表其門閭。



 榮先亦至孝,遭父喪,哀慕不已,遂以毀卒。邑里化其孝行。周文嘉之,乃下詔褒美
 其行,贈滄州刺史,以旌厥異。



 皇甫遐,字永賢,河東汾陰人也。累世寒微,而鄉里稱其和睦。遐性純至,少喪父,事母以孝聞。後遭母喪,乃廬於墓側,負土為墳。復於墓南作一禪窟,陰雨則穿窟,晴霽則營墓。曉夕勤力,未嘗暫停。積以歲年,墳高數丈,周回五十餘步,禪窟重臺兩匝,總成十有二室,中間行道,可容百人。遐食粥枕塊,櫛風沐雨,形容枯悴,家人不識。當其營墓之初,乃有鴟鳥各一,徘徊悲鳴,不離墓側,若助遐者,經月餘日乃去。遠近聞其至孝,競以米面遺之,遐皆受而不食,悉以營佛齋焉。



 郡縣表上其狀,有詔旌異
 之。



 張元,字孝始,河北芮城人也。祖成,假平陽郡守。父延俊,仕州郡,累為功曹主簿。並以純至為鄉里所推。元性謙謹,有孝行,微涉經史,然精釋典。年六歲,其祖以其夏中熱,欲將元就井浴。元固不肯從,謂其貪戲,乃以杖擊其頭曰:「汝何為不肯浴?」元對曰:「衣以蓋形,為覆其褻。元不能褻露其體於白日之下。」



 祖異而捨之。



 南鄰有二杏樹,杏熟多落元園中。諸小兒競取而食之。元所得者,送還其主。



 樹陌有狗子為人所棄者,元即收而養之。其叔父怒曰:「何用此為!」將欲更棄之。



 元對曰:「有生之類,莫不重
 其性命。若天生天殺,自然之理。今為人所棄而死,非其道也。若見而不收養,無仁心也。是以收而養之。」叔父感其言,遂許焉。未幾,乃有狗母銜一死兔置元前而去。



 及元年十六,其祖喪明三年。元恒憂泣,晝夜讀佛經,禮拜以祈福佑。後讀《藥師經》,見「盲者得視」之言。遂請七僧,然七燈,七日七夜轉《藥師經》行道。每言:「天人師乎!元為孫不孝,使祖喪明。今以燈光普施法界,願祖目見明,元求代闇。」如此經七日,其夜夢見一老翁,以金鑱療其祖目,於夢中喜躍,遂即驚覺。乃遍告家人。三日,祖目果明。其後,祖臥疾再周,元恒隨祖所食多少,衣冠不解,旦夕扶
 侍。及祖沒,號踴絕而後蘇。隨其父,水漿不入口三日。鄉里咸歎異之。縣博士楊軌等二百餘人上其狀,有詔表其門閭。



 王頒,字景彥,太原祁人也。父僧辯,《南史》有傳。頒少俶儻,有文武幹局。



 僧辯平侯景,留頒荊州。遇梁元帝為周師所陷,頒因入關。聞其父為陳武帝所殺,號慟而絕,食頃乃蘇,哭不絕聲,毀瘠骨立。至服闋,常布衣蔬食,藉槁而臥。周明帝嘉之,召授左侍上士。累遷漢中太守,尋拜儀同三司。



 隋開皇初,以平蠻功,加開府,封蛇丘縣公。獻取陳之策,上覽而異之,召見,言畢歔欷,上為之改容。及大
 舉伐陳,頒自請行。率兵數百人,從韓擒虎先鋒夜濟,力戰被傷。恐不堪復鬥,悲感嗚咽。夜中睡,夢有人授藥,比寤而瘡不痛。時人以為孝感。



 及陳滅,頒密召父在時士卒,得千餘人,對之涕泣。其間壯士或問曰:「郎君仇恥已雪,而悲哀不止者,將不為霸先早死,不得手刃之邪?請發其丘隴,鬥櫬焚骨,亦可申孝心矣。」頒頓桑陳謝,額盡流血,答曰:「其為墳塋甚大,恐一宵發掘,不及其屍,更至明朝,事乃彰露。」諸人請具鍬鍤。於是夜發其陵,剖棺,見陳武帝鬚皆不落,其本皆出自骨中。頒遂焚骨取灰,投水飲之。既而自縛歸罪。晉王表其狀。文帝曰:「朕以義平
 陳。王頒所為,亦孝義之道,何忍罪之?」舍而不問。有司錄其戰功,將加柱國,賜物五千段。頒固辭曰:「臣緣國威靈,得雪怨恥,本心徇私,非是為國。所加官賞,終不敢當。」帝從之。拜代州刺史,甚有惠政。



 卒於齊州刺史。



 弟頍,字景文。年數歲而江陵亡,同諸兄入關。少好游俠,年二十,尚不知書,為其兄顒所責怒。於是感激,始讀《孝經》、《論語》,晝夜不倦,遂讀《左傳》、《禮》、《易》、《詩》、《書》,乃歎曰:「書無不可讀者。」勤學累載,遂遍通《五經》,究其旨趣,大為儒者所稱。解綴文,善談話。年三十,周武帝引為露門學士,每有議決,多頍所為。性識甄明,精力不倦,好讀諸子,遍記異
 書,以博物稱。又曉兵法,益有從橫之志,每歎不逢時,常以將相自許。



 開皇五年,授著作佐郎,尋令於國子講授。會帝親臨釋奠。國子祭酒元善講《孝經》,頍與相論難,詞義鋒起,善往往見屈。帝大奇之,超授國子博士。後坐事解職,配防嶺南。



 數載,授漢王諒府諮議參軍,王甚禮之。時諒見房陵及秦、蜀二王相次廢黜,潛有異志。頍陰勸諒繕甲兵。及文帝崩,諒遂舉兵反,多頍之計也。頍後數進奇策,諒不能用。楊素至蒿澤,將戰。頍謂其子曰:「氣候殊不佳,兵必敗。汝可隨從我。」



 既而兵敗,頍將歸突厥。至山中,徑路斷絕,知必不免。謂其子曰:「吾之計謀,不減楊
 素,但為言不見從,遂至於此。不能坐受禽執,以成豎子之名也。吾死後,汝慎勿過親故!」於是自殺,瘞之石窟中。其子數日不得食,遂過其故人,竟為所禽。楊素求頍屍得之,斬首,梟於太原。所撰《五經大義》三十卷,有集二十卷,並因兵亂,無復存焉。



 楊慶,字伯悅,河間人也。祖玄、父剛,並以至孝知名。慶美容止,性辯慧。



 年十六,齊國子博士徐遵明見而異之。及長,頗涉書記。年二十五,郡察孝廉,以侍養不赴。母有疾,不解襟帶者七旬。及居母憂,哀毀骨立,負土成墳。齊文宣表其門閭,賜帛及綿粟各有差。隋文帝受禪,屢加褒
 賞,擢授儀同三司,板平陽太守。



 卒於家。



 田翼,不知何許人也。養母以孝聞。其後母臥疾歲餘,翼親易燥濕,母食則食,母不食則不食。隋開皇中,母患暴痢。翼謂中毒藥,遂親嘗穢惡。母終,翼一慟而絕。妻亦不勝哀而死。鄉人厚共葬之。



 紐因,字孝政,河東安邑人也。性至孝。周武成中,父母喪,廬於墓側,負土成墳。廬前生麻一株,高丈許,圍之合拱,枝葉鬱茂,冬夏恒青。有鳥棲上,因舉聲哭,鳥即悲鳴。時人異之。周武帝表其閭,擢授甘棠令。隋開皇初卒。



 子士雄,少質直孝友。喪父,復廬於墓側,負土成墳。其庭前有
 一槐樹,先甚鬱茂,及士雄居喪,樹遂枯死。服闋還宅,死槐復榮。隋文帝聞之,歎其父子至孝,下詔褒揚,號其居為累德里。



 劉仕俊,彭城人也。性至孝。丁母喪,絕而復蘇者數矣,勺飲不入口者七日。



 廬於墓側,負土成墳,列植松柏,虎狼馴擾,為之取食。隋文帝受禪,表其門閭。



 翟普林,楚丘人也。事親以孝聞。州郡辟皆不就,躬耕色養。鄉閭謂為楚丘先生。後父母疾,親易澡濕,不解衣者七旬。大業初,父母俱終,哀毀殆將滅性。廬於墓側,負土成墳。盛冬不衣繒絮,唯著單縗而已。家有鳥犬,隨其在
 墓,若普林哀臨,犬亦悲號,見者嗟異。有二鵲巢其廬前柏樹,入廬馴狎,無所驚懼。司隸巡察,奏其孝感,擢授孝陽令。



 華秋,汲郡臨河人也。幼喪父,事母以孝聞。家貧,傭賃為養。其母患疾,秋容貌毀悴,鬢鬚盡改。母終,遂絕櫛沐,髮盡禿落。廬於墓側,負土成墳。有人欲助之者,秋輒拜而止之。隋大業初,調狐皮,郡縣大獵。有一兔,逐之,奔入秋廬中,匿秋膝下。獵人至廬所,異而免之。自爾,此兔常宿廬中,馴其左右。郡縣嘉其孝感,具以狀聞。降使勞問,而表其門閭。後群盜起,常往來廬之左右,咸相誡曰:「勿犯
 孝子鄉。」賴秋全者甚眾。



 徐孝肅,汲郡人也。宗族數十家,多以豪侈相尚,唯孝肅儉約。事親以孝聞。



 雖在幼小,宗黨間每有爭訟,皆至孝肅所平論,短者無不引咎而退。孝肅早孤,不識父。及長,問其母父狀,因畫工圖其形,構廟置之而定省焉,朔望享祭。養母至孝,數十年家人未見其忿恚色。母老疾,孝肅視易燥濕,憂悴數年,見者莫不悲悼。



 母終,孝肅茹蔬飲水,盛冬單縗,毀瘠骨立。祖父母、父母墓,皆負土成墳。廬于墓所四十餘載,被髮徒跣,遂以終身。



 其弟德備終,子處默,又廬於墓側。弈世稱孝焉。



 論曰:塞天地而橫四海者,唯孝而已矣。然則孝始愛敬之方,終極哀思之道,厥亦多緒,其心一焉。若上智稟自然之質,中庸有企及之義,及其成名,其美一也。



 長孫慮等或出公卿之緒,藉禮教之資;或出茆笪之下,非獎勸所得。並因心乘理,不逾禮教,感通所致,貫之神明。乃有負土成墳,致毀滅性,雖乖先王之典制,亦觀過而知仁矣。



\end{pinyinscope}