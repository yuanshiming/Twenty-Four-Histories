\article{卷四十七列傳第三十五}

\begin{pinyinscope}

 袁
 翻弟躍躍子聿脩陽尼從孫固固子休之固從兄藻藻子斐固從弟元景賈思伯祖瑩子珽袁翻,字景翔,陳郡項人也。父宣,為宋青州刺史沈文秀府主簿,隨文秀入魏。



 而大將軍劉昶言是其外祖淑近親,令與其府諮議參軍袁濟為宗。宣時孤寒,甚相依附。及翻兄弟官顯,與濟子洸、演遂各陵競,洸等乃經公府,以相排斥。翻少入東觀,為徐紇所薦,李彪引兼著作佐
 郎,參史事。後拜尚書殿中郎。正始初,詔尚書門下於金墉中書外省考論律令,翻與門下錄事常景、孫紹、廷尉監張彪、律博士侯堅固、書侍御史高綽、前將軍邢苗、奉車都尉程靈虯、羽林監王元龜、尚書郎祖瑩、宋世景、員外郎李琰之、太樂令公孫崇等並在議限。又詔太師彭城王勰、司州牧高陽王雍、中書監京兆王愉、青州刺史劉芳、左衛將軍元麗、兼將作大匠李韶、國子祭酒鄭道昭、廷尉少卿王顯等入豫其事。後除豫州中正。



 是時,修明堂辟雍,翻議曰:謹按明堂之義,今古諸儒論之備矣。蓋唐、虞以上,事難該悉;夏、殷以降,校可知之。按《周官考
 工》所記,皆記其時事,具論夏、殷名制,豈其糸比繆?是知明堂五室,三代同焉,配帝像行,義則明矣。及《淮南》、《呂氏》與《月令》同文,雖布政班時,有堂個之別,然推其體,則無九室之證。



 既而正義殘隱,妄說斐然。明堂九室,著自《戴禮》,探緒求源,罔知所出,而漢氏因之,自欲為一代之法。故鄭玄云:「周人明堂五室,是帝一室也,合於五行之數。《周禮》依數,以為之室。」本制著存,是周五室也。於今不同,是漢異周也。漢為九室,略可知矣。但就其此制,猶有懵焉。何者?張衡《東京賦》云:「乃營三宮,布教班常,復廟重屋,八達九房。」此乃明堂之文也。而薛綜注云:「房,室也。謂堂後
 有九室。」堂後有九室之制,非巨異乎。裴頠又云:「漢氏作四維之個,不能令各據其辰,就使其像可圖,莫能通其居用之禮,此為設虛器也。」



 甚知漢世徒欲削滅周典,捐棄舊章,改物創制,故不復拘於載籍。且鄭玄之詁訓《三禮》及釋《五經》異義,並盡思窮神,不墜周公之舊法也。伯喈損益漢制,章句繁雜,既違古背新,又不能易玄之妙矣。魏、晉書紀,亦有明堂祀五帝之文,而不記其經始之制,雙無坦然可準。觀夫今之基趾,猶或仿佛,高卑廣狹,頗與《戴禮》不同,何得以意抑心,便謂九室可明?且三雍異所,復乖盧、蔡之義,進退無據,何用經通?晉朝亦以鉆
 鑿難明,故有一屋之論,並非經典正義,皆以意妄作,茲為不典。學家常談,不足以範時軌世。皇代既乘乾統歷,得一御宸,自宜稽古則天,憲章文武,追蹤周孔,述而不作。豈容虛追子氏放篇之浮說,徒損經紀雅誥之遺訓,而欲以支離橫義,指畫妄圖,儀刑宇宙而貽來葉者也?



 又北京制置,求皆允怗,繕修草創,以意良多。事移化變,存者無幾,理茍宜革,何必仍舊。且遷都之始,日不遑給,先朝規度,每事循古,是以數年之中,悛換非一,良以永法為難,數改為易。何為宮室府庫多因故跡,而明堂辟雍獨遵此制?



 建立之辰,復未可知矣。既猥班訪逮,輒輕
 率瞽言,明堂五室,請同周制,郊建三雍,求依故所,庶有會經誥,無失典刑。



 後議選邊戍事,翻議曰:臣聞兩漢警於西北,魏、晉備在東南。是以鎮邊守塞,必寄威重;伐叛柔服,實賴溫良。故田叔、魏尚,聲高於沙漠;當陽、鉅平,績流於江漢。紀籍用為美談,今古以為盛德。自皇上以睿明纂御,風清化遠,威厲秋霜,惠沾春露,故能使淮海輸誠,華陽即序,連城革面,比屋歸仁。縣車劍閣,豈伊曩載;鼓噪金陵,復在茲日。然荊、揚之牧,宜盡一時才望;梁、郢之君,尤須當今秀異。



 自比緣邊州郡,官至便登,疆場統戍,階當即用。或逢穢德凡人,或遇貪家惡子,不識字人
 溫恤之方,唯知重役殘忍之法。廣開戍邏,多置帥領,或用其左右姻親,或受人貨財請屬,皆無防寇禦賊之心,唯有通商聚斂之意。其勇力之兵,驅合抄掠,若遇強敵,即為奴虜;如有執獲,奪為己富。其羸弱老小之輩,微解金鐵之工,少閑草木之作,無不搜營窮壘,苦役百端。自餘或伐木高山,或芸草平陸,販貨往還,相望道路。此等祿既不多,資亦有限,皆收其實絹,給其虛粟。窮其力,薄其衣,用其工,節其食,綿冬歷夏,加之疾苦,死於溝瀆者常十七八焉。是以吳、楚間伺,審此虛實,皆云糧匱兵疲,易可乘擾,故驅率犬羊,屢犯疆場。頻年已來,甲胄生蟣,
 十萬在郊,千金日費。為弊之深,一至於此!皆由邊任不得其人,故延若斯之患。賈生所以痛哭,良有以也。



 夫潔其流者清其源,理其末者正其本,既失之在始,庸可止乎。愚謂自今已後,荊、揚、徐、豫、梁、益諸蕃及所統郡縣府佐統軍至于戍主,皆令朝臣王公已下各舉所知,必選其才,不拘階級。若能駕御有方,清高獨著,威足臨戎,信能懷遠,撫循將士,得其忻心,不營私潤,專修公利者,則就加爵賞,使久於其任,以時褒賚,厲其忠款。所舉之人,亦垂優異,獎其得士,嘉其誠節。若不能一心奉公,才非捍禦,貪婪日富,經略無聞,人不見德,兵厭其勞者,即加
 顯戮,用章其罪。所舉之人,隨事免降,責其謬薦,罰其偽薄。如此則舉人不得挾其私,受任不得孤其舉。善惡既審,沮勸亦明。庶邊患永消,譏議攸息矣。



 遭母憂去職。熙平初,除廷尉少卿,頗有不平之論,為靈太后所責。出為陽平太守,甚不自得,遂作思歸賦。



 神龜末,遷涼州刺史。時蠕蠕主阿那瑰、後主婆羅門並以國亂來降,朝廷問安置之計。翻表曰:今蠕蠕內為高車所討滅,外憑大國之威靈,兩主投身,一期而至,百姓歸誠,萬里相屬。然夷不亂華,前鑒無遠,覆車在於劉、石,毀轍固不可尋。今蠕蠕雖主奔於上,人散於下,而餘黨實繁,部落猶眾,高車
 亦未能一時并兼,盡令率附。又高車士馬雖眾,主甚愚弱,上不制下,下不奉上,唯以掠盜為資,陵奪為業。而河西捍禦強敵,唯涼州、敦煌而已。涼州土廣人稀,糧仗素闕,敦煌、酒泉,空虛尤基。若蠕蠕無復豎立,令高車獨擅北垂,則西顧之憂,匪旦伊夕。



 愚謂蠕蠕二主,並宜存之。居阿那瑰於東偏,處婆羅門於西裔,分其降人,各有攸屬。那瑰住所,非所經見,其中事勢,不可輒陳。婆羅門請修西海故城以安處之。西海郡本屬涼州,今在酒泉,直抵張掖西北千二百里,去高車所住金山一千餘里。正是北虜往來之衝要,漢家行軍之舊道,土地沃衍,大宜
 耕殖。非但今處婆羅門,於事為便,即可永為重戍,鎮防西北。雖外為署蠕蠕之聲,內實防高車之策。



 一二年後,足食足兵,斯固安邊保塞之長計也。若婆羅門能自克厲,使餘燼歸心,收離聚散,復興其國者,乃漸令北轉,徙度流沙,即是我之外籓,高車之勍敵,西北之虞,可無過慮。如其奸回反覆,孤恩背德者,此不過為逋逃之寇,於我何損?



 今不早圖,戎心一啟,脫先據西河,奪我險要,則酒泉、張掖,自然孤危,長河已西,終非國有。不圖厥始,而求憂其終,噬臍之恨,悔將何及。



 愚見如允,乞遣大使往涼州敦煌及於西海,躬行山谷要害之所,親閱亭障遠
 近之宜,商量士馬,校糸柬糧仗,部分見定,處置得所。入春,西海之間,即令播種,至秋,收一年之食,使不復勞轉輸之功也。且西徼北垂,即是大磧,野獸所聚,千百為群,正是蠕蠕射獵之處。殖田以自供,籍獸以自給,彼此相資,足以自固。今之豫度,似如小損,歲終大計,其利實多。高車豺狼之心,何可專信?假令稱臣致款,正可外加優納,而復內備彌深,所謂先人有奪人之心者也。



 時朝議是之。還,拜吏部郎中。遷齊州刺史,無多政績。孝昌中,除安南將軍、中書令,領給事黃門侍郎,與徐紇俱在門下,並掌文翰。翻既才學名重,又善附會,亦為靈太后所信待。
 是時蠻賊充斥,六軍將親討之,翻乃上表諫止。後蕭寶夤大敗於關西,翻上表,請為西軍死亡將士舉哀,存而還者,并加賑賚。後拜度支尚書,尋轉都官。翻上表,願以安南、尚書換一金紫。時天下多事,翻雖外請閑秩,而內有求進之心,識者怪之。於是加撫軍將軍。明帝、靈太后曾燕華林園,舉觴謂群臣曰:「袁尚書朕之杜預,欲以此杯敬屬元凱,今為盡之。」侍坐者莫不羨仰。



 翻名位俱重,當時賢達咸推與之。然獨善其身,無所獎拔,排抑後進,論者鄙之。建義初,遇害河陰。所著文筆百餘篇,行於世。贈使持節、侍中、車騎將軍、儀同三司、青州刺史。嫡子寶
 首,武定中,司徒記室參軍事。翻弟躍。



 躍字景騰,博學俊才,性不矯俗,篤交友。翻每謂人曰:「躍可謂我家千里駒也。」歷位尚書都兵郎中,加員外散騎常侍。將立明堂,躍乃上議,當時稱其博洽。



 蠕蠕主阿那環亡破來奔,朝廷矜之,送復其國。既而每使朝貢,辭旨頗不盡禮。躍為朝臣書與環,陳以禍福,言辭甚美。後遷車騎將軍太傅清河王懌文學,雅為懌所愛賞。懌之文表,多出於躍。卒,贈冠軍將軍、吏部郎中。所制文集行於世。無子,兄翻以子聿脩繼。



 聿脩,字叔德。七歲遭喪,居處禮若成人。九歲,州辟主簿。
 性深沈,有鑒識,清靖寡欲,與物無競。姨丈人尚書崔休深所知賞。年十八,領本州中正,兼尚書度支郎中。齊天保初,除太子庶子,以本官行博陵太守,大有聲績,遠近稱之。累遷司徒左長史,領兼御史中丞。司徒錄事參軍盧思道私貸庫錢三十萬,娉太原王乂女為妻,而王氏以先納陸孔文禮娉為定。聿脩為首僚,又國之司憲,知而不劾,免中丞。尋遷秘書監。



 天統中,詔與趙郡王睿等議定三禮。出為信州刺史,即其本鄉也。時久無例,莫不榮之。為政清靖,不言而化,自長史以下,爰逮鰥寡孤幼,皆得其歡心。武平初,御史普出,過諸州悉有舉劾,唯不
 到信州。及還都,人庶道俗,追列滿道,或將酒脯,涕泣留連,競欲遠送。時既盛暑,恐其勞敝,往往為之駐馬,隨舉一酌,示領其意,辭謝令去。還後,州人鄭播宗等七百餘人請為立碑,斂縑布數百匹,託中書侍郎李德林為文,以記功德。敕許之。尋除都官尚書。聿脩少年平和溫潤,素流之中,最為規檢,以名家子歷任清華,時望多相器待,許其風鑒。在郎署之日,時趙彥深為水部郎中,同在一院,因成交友。彥深後重被沙汰停私,門生藜藿,聿脩猶以故情音問來往。彥深任用,銘戢甚深,雖人才無愧,蓋亦由彥深接引。為吏部尚書以後,自以物望得之。



 初,
 馮子琮以僕射攝選,婚姻相尋。聿脩常非笑之,語人云:「馮公營婚,日不暇給。」及自居選曹,亦不能免,時論以為地勢然也。素品孤官,頗有怨響。然在官廉謹,當時少匹。魏、齊世,臺郎多不免交通餉饋。初,聿脩為尚書郎十年,未曾受升酒之遺。尚書邢邵與聿脩舊款,每省中語戲,常呼聿脩為清郎。大寧初,聿脩以太常少卿出使巡省,仍令考校官人得失。經袞州,時邢邵為刺史,別後,送白紬為信。聿脩不受,與邢邵書云:「今日仰過,有異常行,瓜田李下,古人所慎,願得此心,不貽厚責。」邵亦欣然領解,報書云:「老夫忽忽,意不及此,敬承來旨,吾無間然。弟昔
 為清郎,今日復作清卿矣。」及在吏部,屬政衰道喪,若違忤要勢,禍不旋踵,雖以清白自守,猶不免請謁之累。



 入周,位儀同大將軍、吏部下大夫、東京司宗中大夫。隋開皇初,加上儀同,遷東京都官尚書。東京廢,入朝,除都官尚書。二年,出為熊州刺史,卒。子知禮,大業初卒於太子內舍人。



 躍弟颺,卒於豫州冠軍府司馬。颺弟升,位正員郎。颺死後,昇通其妻。翻恚,為之發病,昇終不止,時人鄙穢之。亦於河陰見害。贈左將軍、齊州刺史。



 陽尼,字景文,北平無終人也。累世仕於慕容氏。尼少好學,博通群籍,與上谷侯天護、頓丘李彪同志齊名。幽州
 刺史胡泥表薦之,徵拜祕書著作郎。及改中書學為國子。時中書監高閭、侍中李沖等以尼碩學,舉為國子祭酒。後兼幽州中正。



 孝文臨軒,令諸州中正各舉所知,尼與齊州大中正房千秋各舉其子。帝曰:「昔有一祁,名垂往史,今有二奚,當聞來牒。」出為幽州平北府長史,帶漁陽太守,未拜,坐為中正時受鄉人貨免官。每自傷曰:「吾昔未仕,不曾羨人,今日失官,與本何異?然非吾宿志,命也如何!」既而還家,有書數千卷。所造《字釋》數十篇,未就而卒。其從孫太學博士承慶撰為《字統》二十卷,行於世。承慶從弟固。



 固字敬安,性倜儻,不拘小節,少任俠,好劍客,弗事生產。年二十六,始折節好學,博覽篇籍,有文才。太和中,從大將軍、宋王劉昶徵義陽,板府法曹行參軍。昶性嚴暴,三軍戰慄,無敢言者。固啟諫,並面陳事宜。昶大怒,欲斬之,使監當攻道。固在軍勇決,意志閑雅,了無懼色,昶甚奇之。軍還,言之孝文。年三十餘,始辟大將軍府參軍事,累遷書侍御史,多所劾奏。



 宣武廣訪得失,固上讜言表曰:「當今之務,宜早正東儲,立師傅以保護,立官司以防衛,以係蒼生之心。攬權衡,親宗室,強幹弱枝,以立萬世之計。舉賢良,黜不肖,使野無遺才,朝無素餐。孜孜萬機,躬
 勤庶政,使人無謗讟之響。省徭役,薄賦斂,修學宮,遵舊章,貴農桑,賤工賈,絕談虛窮微之論,簡桑門無用之費,以救飢寒之苦。然後備器械,修甲兵,習水戰,滅吳會,撰封禪之禮,襲軒、唐之軌,豈不茂哉!」



 初,帝委任群下,不甚親覽,好桑門之法。尚書令高肇以外戚權寵,專決朝事。



 又咸陽王禧等並有釁,故宗室大臣相見疏薄,而王畿人庶,勞弊益甚。固乃作《南北二都賦》,稱恒代田漁聲樂侈靡之事,節以中京禮儀之式,因以諷諫。



 宣武末,中尉王顯起宅既成,集僚屬饗宴。酒酣,問固曰:「此宅何如?」固曰:「晏嬰湫隘,流稱于今,豐屋生災,著於《周易》。此蓋同傳
 舍耳,唯有德能卒,願公勉之。」顯嘿然。他日又謂固曰:「吾作太府卿,府庫充實,卿以為何如?」



 固對曰:「公收百官之祿四分之一,州郡贓贖悉入京藏,以此充府,未足為多。且有聚斂之臣,寧有盜臣,豈不戒歟!」顯大不悅,以此銜固。以有人間固於顯,因奏固剩請米麥,免固官。遂闔門自守,著《演賾賦》以明幽微通塞之事。又作《刺讒疾嬖幸詩》二首曰:巧佞巧佞,讒言興兮。營營習習,似青蠅兮。以白為黑,在汝口兮。汝非蝮蠆,毒何厚兮。巧巧佞佞,一何工矣。司閒司忿,言必從矣。朋黨噂沓,自相同矣。



 浸潤之譖,傾人墉矣。成人之美,君子責焉。攻人之惡,君子恥焉。
 汝何人斯,譖毀日繁?子實無罪,何騁汝言?番番緝緝,讒言側入,君子好讒,如或弗及。天疾讒說,汝其至矣,無妄之禍,行將及矣。泛泛遊鳧,弗制弗拘,行藏之徒,或智或愚。維餘小子,未明茲理,毀與行俱,言與釁起。我其懲矣,我其悔矣,豈求人兮,忠恕在己。



 彼諂諛兮,人之蠹兮。刺促昔粟,罔顧恥辱,以求媚兮。邪干側入,如恐弗及,以自容兮。志行褊小,好習不道。朝挾其車,夕承其輿,或騎或徒,載奔載趨。或言或笑,曲事親要。正路不由,邪徑是蹈。不識大猷,不知話言,其朋其黨,其徒實繁。有詭其行,有佞其音,籧篨戚施,邪媚是欽,既詭且妒,以通其心。是信
 是任,敗其以多,不始不慎,末如之何。習習宰嚭,營營無極。梁丘寡智,王鮒淺識,伊戾息夫,異世同力,江充趙高,甘言似直,豎刁上官,擅生羽翼。乃如之人,僭爽其德,豈徒喪邦,又亦覆國。嗟爾中下,其親其暱。不謂其非,不覺其失,好之有年,寵之有日。我思古人,心焉苦疾。凡百君子,宜其慎矣,覆車其鑒,近可信矣。言既備矣,事既至矣,反是不思,維塵及矣。



 明帝即位,除尚書考功郎中。奏諸秀孝考中第者聽敘,自固始。大軍征硤石,敕為僕射李平行臺七兵郎。平奇固勇敢,軍中大事,悉與謀之。又命固節度水軍。



 固設奇計,先期乘賊,獲其外城。後太傅、清河
 王懌舉固,除步兵校尉,領汝南王悅郎中令。時悅年少,行多不法,固上疏諫悅,悅甚敬憚之。懌大悅,以為舉得其人。除洛陽令,在縣甚有威風。丁母憂,號慕毀疾,杖而能起,練禫之後,酒肉不進。時固年踰五十,而喪過於哀,鄉黨親族咸歎服焉。清河王懌領太尉,辟固從事中郎,屬懌被害,不奏。懌之遇害,元叉執政,朝野震悚,懌諸子及門生僚吏,莫不慮禍,隱避不出。固以嘗被辟命,遂獨詣喪所,盡哀慟哭,良久乃還。僕射游肇聞而嘆曰:「雖欒布、王脩,何以尚也?君子哉若人!」及汝南王悅為太尉,選舉多非其人,又輕肆撾撻。固以前為元卿,雖離國,猶上
 疏切諫,事在《悅傳》。後悅辟固為從事中郎,不就。京兆王繼為司徒,高選官僚,辟固從事中郎。府解,除前軍將軍,又典科揚州勛賞。初,硤石之役,固有先登之功,而朝賞未及,至是,與尚書令李崇訟勳,更相表。崇雖貴盛,固據理不撓,談者稱焉。卒,贈輔國將軍、太常少卿,謚曰文。



 固剛直雅正,不畏強禦,居官清潔,家無餘財,終沒之日,室徒四壁,無以供喪,親故為其棺斂。初,固著《終制》一篇,務從儉約。臨終,又敕諸子一遵先制。



 五子,長子休之。



 休之字子烈,俊爽有風概,好學,愛文藻,時人為之語曰:「能賦能詩陽休之。」



 初為州主簿。孝昌中,杜洛周陷薊城,
 休之與宗室南奔章武,轉至青州。葛榮寇亂,河北流人,多水奏青州。休之知將有變,請其族叔伯彥等潛歸京師避之,多不能從。



 休之垂涕別去。俄而葛榮邢杲作亂,伯彥等咸為土人所殺,諸陽死者數十人,唯休之兄弟免。



 莊帝立,累遷太尉記室參軍。李神俊監起居注,啟休之,與河東裴伯茂、范陽盧元伯、河間邢子才俱入撰次。普泰中,為太保長孫承業府屬。尋敕與魏收、李同軌等修國史。後行臺賀拔勝經略樊沔,請為南道軍司。俄而魏武帝入關,勝令休之奉表詣長安參謁。時齊神武亦啟除休之太常少卿。尋屬勝南奔,仍隨勝至江南。休之聞
 神武推奉靜帝,乃白勝啟梁武求還,文襄以為大行臺郎中。神武幸汾陽之天池,池邊得一石,上有隱起字,文曰「六王三川。問休之曰:「此文字何義?」對曰:「『六』者,大王字。河、洛、伊為三川,大王若受天命,終應統有關右。」神武曰:「世人常道我欲反,今若聞此,更致紛紜,慎莫妄言也。」元象初,錄荊州軍功,封新泰縣伯。



 武定二年,除中書侍郎。先是,中書專主綸誥,魏宣武已來,事移門下。至是發詔依舊,任遇甚顯。時魏收為散騎常侍,領兼侍郎,與休之參掌詔命,世論以為中興。有人士戲嘲休之云:「有觸籓之羝羊,乘連錢之驄馬,從晉陽而向鄴,懷屬書而盈把。」
 左丞盧斐以其文書請謁,啟神武禁止,會赦不問。歷尚食典御、太子中庶子、給事黃門侍郎、中軍將軍、幽州大中正,兼侍中,持節奉璽書詣并州,敦喻文宣為相國、齊王。時將受魏禪,發晉陽至平陽郡,為人心未一,且還并州,恐漏泄,仍斷行人。休之性疏放,使還,遂說其事,鄴中悉知。後高德正以聞,文宣忿之而未發。齊受禪,除散騎常侍,監修起居注。頃之,坐詔書脫誤,左遷驍騎將軍,積其前事也。文宣郊天,百僚咸從,休之衣兩襠甲,手持白棓。時魏收為中書令,嘲之曰:「義真服未?」休之曰:「我昔為常伯,首戴蟬冕;今處驍游,身被衫甲。允文允武,何必減
 卿。」談笑晏然,議者服其夷曠。以禪讓之際,參定禮儀,別封始平縣男。後除中山太守。先是,韋道建、宋欽道代為定州長史,帶中山太守,並立制,監臨之官出行,不得過百姓飲食。有者,即數錢酬之。休之常以為非。及至郡,復相因循。或問其故,休之曰:「吾昔非之者,為其失仁義;今日行之者,自欲避嫌疑。豈是夙心,直是處世難耳。」在郡三年,再致甘露之瑞。



 文宣崩,徵休之至晉陽,經紀喪禮,與魏收俱至。尚書令楊遵彥與休之等款狎,相遇中書省,言及喪事,收掩淚失聲,休之嚬眉而已。他日遵彥謂曰:「昨聞諱,魏少傅悲不自勝,卿何容都不流涕?」休之曰:「
 天保之世,魏侯時遇甚深,鄙夫以眾人見待,佞哀詐泣,實非本懷。」



 皇建初,兼度支尚書。昭帝留心政道,訪以政術,休之答以明賞罰,慎官方,禁淫侈,恤人患,為政教之先。帝深納之。大寧中,歷都官、七兵、祠部三尚書。



 河清三年,出為西袞州刺史。天統初,徵為光祿卿,監國史。尋除吏部尚書。休之多識故事,諳悉氏族,凡所選用,莫不才地俱允。前國子助教熊安生,當時碩儒,因喪解職,久而不見調,休之引為國子博士,儒者以此歸之。簡率不樂煩職,典選稍久,非其所好,每謂人曰:「此官實自清華,但煩劇,妨吾賞適,直是樊籠矣。」



 武成崩後,頻乞就閒。武平
 初,除中書監、尚書右僕射。三年,加位特進,與朝士撰《聖壽堂御覽》。六年,正除尚書左僕射,領中書監。



 休之早得才名,為人物所傾服,外如疏放,內實謹厚。少年頗以峻急為累,晚節以通美見稱。重衿期,好遊賞。太常卿盧元明,人地華重,罕所交接,非一時名士,不得與之游。休之始為行臺郎,便坦然投分,文酒會同,相得甚款,鄉曲人士莫不企羨焉。太子中庶子平原明少遐,風流名士也,梁亡奔鄴,昔因通聘,與休之同游。及少遐卒,其妻窮敝,休之經紀振恤,恩分甚厚。尚書僕射崔暹為文襄所親任,勢傾朝列,休之未嘗請謁。暹子達拏幼而聰敏,年十
 餘已作五言詩。時梁國通和,聘使在館,暹持達拏數首詩示諸朝士有才學者,又欲示梁客。餘人畏暹,皆隨宜應對,休之獨正言:「郎子聰明,方成偉器。但小兒文藻,恐未可以示遠人。」



 其方直如此。元景每云:「當今直諫,陽子烈其有焉。」



 晚節,說祖珽撰《御覽》,書成加特進,令其子辟強預修《御覽》書。及珽黜,便布言於朝廷,云先有隙。及鄧長顒、顏之推奏立文林館,之推本意不欲令耆舊貴人居之,便相附會,與少年朝請、參軍之徒,同入待詔。時論貶焉。魏收監史之日,立《神武本紀》,取平西胡之歲為齊元。收在齊州,恐史官改奪其志,上表論之。



 及收還朝,敕
 集朝賢議其事,休之立議從天保為限斷。魏收存日,猶兩議未決。收死,便諷動內外,發詔從其議。後領中書監,謂人云:「我已三為中書監,用此何為!」隆化還鄴,舉朝多有遷授,封休之燕郡王。乃謂所親曰:「我非蠻奴,何忽此授?」凡此諸事,為識者所譏。好學不倦,博綜經史,文章雖不華靡,亦為典正。



 魏收在日,深為收所輕,魏殂後,以先達見推。位望雖高,虛懷接物,為搢紳所愛重。



 周武帝平齊,與吏部尚書袁聿脩、衛尉卿李祖欽、度支尚書元脩伯、大理卿司馬幼之、司農卿崔達拏、秘書監源宗、散騎常侍兼中書侍郎李若、散騎常侍兼給事黃門侍郎李
 孝貞、給事黃門侍郎盧思道、給事黃門侍郎顏之推、通直散騎常侍兼中書侍郎李德林、通直散騎常侍兼中書舍人陸乂、中書侍郎薛道衡、中書舍人元行恭、辛德源、王邵、陸開明十八人同徵,令隨駕後赴長安。尋除開府儀同,依例封臨澤縣男。歷納言中大夫、太子少保,進位上開府,除和州刺史。隨開皇二年罷任,終於洛陽。所著文集四十卷,又撰《幽州人物志》,並行於世。



 初,休之在洛,將仕,夜夢見黃河北驛道上行,從東向西。道南有一塚,極高大。休之步登冢頭,見一銅柱,趺為連花形。休之從西北登一柱礎上,以手捉一柱,柱遂右轉。休之咒曰:「
 柱轉三匝,吾至三公」,柱遂三匝而止。休之尋寤,意如在鄴城東南者,其夢竟驗云。



 子辟彊,字君大,性疏脫,又無藝,休之亦引入文林館,為時人所嗤鄙。武平末,為尚書水部郎中。



 休之弟綝之,天平中入關。次俊之,位兼通直常侍,聘陳副,尚書郎。當文襄時,多作六言歌辭,淫蕩而拙,世俗流傳,名為《陽五伴侶》,寫而賣之,在市不絕。俊之嘗過市,取而改之,言其字誤。賣書者曰:「陽五古之賢人,作此《伴侶》,君何所知,輕敢議論!」俊之大喜。後待詔文林館,自言:「有文集十卷,家兄亦不知吾是才士也。」固從兄藻。



 藻字景德,少孤,有雅志,涉獵經史。位中書博士,詔兼禮官,拜燕宣王廟於長安。還,賜爵魏昌男。累遷瀛州安東府長史,以年老歸家,為賊杜洛周所囚,發病卒。永熙中,贈幽州刺史。子裴。



 斐字叔鸞,魏孝莊時,於西袞州督護流人有功,賜爵方城伯。歷廣平王開府中郎,修起居注。除起部郎中,兼通直散騎常侍,聘梁。梁尚書羊侃,魏之叛人也,與斐舊故,欲召斐至宅,三致書,斐不答。梁人曰:「羊來已久,經貴朝遷革,李、盧亦詣宅相見,卿何致難?」斐曰:「柳下惠則可,吾不可。」梁武帝又親謂斐曰:「侃極願相見,今二國和好,安
 得復論彼此。」斐終辭焉。還,除廷尉少卿。石齊河溢,橋壞,斐移津於白馬,中河起石潬,兩岸造關城,累年乃就。東郡太守陸士佩以黎陽關河形勝,欲因山壑以為公家苑囿。斐書答以國步始康,人勞未息,誠宜輕徭薄賦,勤恤人隱,不從。天保中,除都水使者。詔斐監築長城。累遷殿中尚書,以本官監瀛州事,拜儀同三司。卒,贈中書監、北豫州刺史,謚曰簡。子師孝,中書舍人固從弟昭。



 昭字元景,學涉史傳,尤閑案牘。為齊文襄府墨曹參軍,甚見親委,與陳元康、崔暹等參謀機密。及崔甗為崔暹所告,元景劾成其獄,賴邢子才證白以免,時以元景為
 告而順旨。初,文襄擇日將受魏禪,令元景等定儀注,草詔冊,並授官,未畢而文襄殂,罷府。天保初,除給事黃門侍郎。後以風氣彌留,不堪近侍,出除青州高陽內史,卒於郡。文集十卷。



 子靜立,性淳孝,操履清方,美詞令,善尺牘。仕齊,位三公郎中。隋開皇初,州主簿。



 賈思伯,字仕休,齊郡益都人也。其先自武威徙焉。世父元壽,中書侍郎,有學行,見稱於時。思伯自奉朝請累遷中書侍郎,頗為孝文所知。任城王澄之圍鐘離也,以思伯持節為其軍司。及澄失利,思伯為後殿。澄以其儒者,謂之必死。及至,大喜曰:「仁者必有勇,常謂虛談,今於軍
 司見之矣!」思伯託以失道,不伐其功,時論稱其長者。累遷南青州刺史。初,思伯與弟思同師事北海陰鳳,業竟,無資酬之,鳳遂質其衣物。時人為之語曰:「陰生讀書不免癡,不識雙鳳脫人衣。」及思伯之部送縑百匹遺鳳,因具車馬迎之,鳳慚不往。時人稱歎焉。昭帝時,拜涼州刺史,思伯以邊遠不願,辭以男女未婚。靈太后不許,因舍人徐紇言乞得停。後除廷尉卿,自以儒素為業,不好法律,希言事。俄轉衛尉卿。



 時議建明堂,多有同異。思伯上議曰:案《周禮》,夏后氏世室,殷重屋,周明堂,皆五室。鄭注云:「此三者或舉宗廟,或舉王寢,或舉明堂,互言之以明
 其制同也。」若然,則夏、殷之世已有明堂矣。唐、虞以前,其事未聞。戴德《禮記》云:「明堂凡九室十二堂。」蔡邕云:「明堂者,天子太廟,饗功、養老、教學、選士皆於其中,九室十二堂。」案戴德撰《記》,世所不行。且九室十二堂,其於規制,恐難得厥衷。《周禮》:營國,左祖右社,明堂在國之陽。則非天子太廟明矣。然則《禮記月令》四堂及太室皆謂之廟者,當以天子暫配享五帝故耳。又《王制》云:「周人養國老於東膠。」鄭注云:「東膠即辟雍,在王宮之東。」又《詩·大雅》云:「邕邕在宮,肅肅在廟。」



 鄭注云:「宮謂辟雍宮也,所以助王,養老則尚和,助祭則尚敬。」又不在明堂之驗矣。案《孟子》云
 齊宣王謂孟子曰:「吾欲毀明堂。」若明堂是廟,則不應有毀之問。且蔡邕論明堂之制云:「堂方百四十尺,象坤之策;屋圓徑二百一十六尺,象乾之策;方六丈,徑九丈,象陰陽九六之數;九室以象九州;屋高八十一尺,象黃鐘九九之數;二十八柱以象宿;外廣二十四丈以象氣。」案此皆以天地陰陽氣數為法,而室獨象九州,何也?若立五室以象五行,豈不快也?如此,蔡邕之論,非為通典;九室之言,或未可從。



 竊尋《考工記》雖是補闕之書,相承已久,諸儒注述,無言非者,方之後作,不亦優乎。其《孝經援神契》、《五經要義》、舊《禮圖》皆作五室,及徐、劉之論,謂同《考
 工》者多矣。朝廷若獨絕今古,自為一代制作者,則所願也。若猶祖述舊章,規摹前事,不應捨殷、周成法,襲近代妄作。且損益之極,極於三王,後來疑議,難可準信。鄭玄云:「周人明堂五室,是帝各有一室也,合於五行之數,《周禮》依數以為之室。施行于今,雖有不同,時說然矣。」尋鄭此論,非為無當。



 案《月令》亦無九室之文,原其制置,不乖五室。其青陽右個即明堂左個,明堂右個即總章左個,總章右個即玄堂左個,玄堂右個即青陽左個。如此,則室猶是五,而布政十二。五室之理,謂為可按。其方圓高廣自依時量。戴氏九室之言,蔡子廟學之議,子乾靈臺
 之說,裴逸一屋之論,及諸家紛紜,並無取焉。



 學者善其義。後為都官尚書。時崔光疾甚,表薦思伯侍講,中書舍人馮元興為侍讀。思伯遂入授明帝杜氏《春秋》。思伯少雖明經,從官廢業,至是更延儒生,夜講晝授。性謙和,傾身禮士,雖在街途,停車下馬,接誘恂恂,曾無倦色。客有謂曰:「公今貴重,寧能不驕?」思伯曰:「衰至便驕,何常之有?」當世以為雅言。思伯與元興同事,大相友暱,元興時為元叉所寵,論者譏其趨勢云。卒,贈青州刺史,又贈尚書左僕射,謚曰文貞。



 子彥始,武定中淮陽太守。



 思伯弟思同,字仕明,少勵志行,雅好經史,與兄思伯,年少時俱為
 鄉里所重。



 累遷襄州刺史,雖無明察之譽,百姓安之。元顥之亂,思同與廣州刺史鄭光護並不降。莊帝還宮,封營陵縣男。後與國子祭酒韓子熙並為侍講,授靜帝杜氏《春秋》。



 加散騎常侍,兼七兵尚書,尋拜侍中。卒,贈尚書右僕射、司徒公,謚曰文獻。



 初,思同為青州別駕,清河崔光韶先為中從事,自恃資地,恥居其下,聞思同還鄉,遂便去職,州里人物為思同恨之。及光韶亡,遺誡子姪不聽求贈。思同遂表訟光韶操業,特蒙贈謚,論者歎尚焉。



 思同之侍講也,國子博士遼西衛冀隆精服氏學,上書難杜氏《春秋》六十三事,思同復駁冀隆乖錯者一十餘
 條,互相是非,積成十卷。詔下國學,集諸儒考之,事未竟而思同卒。後魏郡姚文安、樂陵秦道靜復述思同意。冀隆亦尋物故,浮陽劉休和又持冀隆說。竟未能裁正。



 祖瑩,字元珍,范陽遒人也。曾祖敏,仕慕容垂為平原太守。道武定中山,賜爵安固子,拜尚書左丞。卒,贈并州刺史。祖嶷,字元達,以從征平原功進爵,為侯,位馮翊太守,贈幽州刺史。父季真,多識前言往行,位中書侍郎、鉅鹿太守。



 瑩年八歲能誦詩書,十二為中書學生,耽書。父母恐其成疾,禁之不能止。常密於灰中藏火,驅逐僮僕,父母寢睡之後,燃火讀書,以衣被蔽塞窗戶,恐漏光明,為
 家人所覺。由是聲譽甚盛,內外親屬呼為聖小兒。尤好屬文,中書監高允每歎曰:「此子才器,非諸生所及,終當遠至。」時中書博士張天龍講《尚書》,選為都講。



 生徒悉集。瑩夜讀勞倦,不覺天曉,催講既切,遂誤持同房生趙郡李孝怡《曲禮》卷上座。博士嚴毅,不敢復還,乃置《禮》於前,誦《尚書》三篇,不遺一字。孝文聞之,召入,令誦《五經》章句並陳大義。帝戲盧昶曰:「昔流共工於幽州,北裔之地那得忽有此子?」昶對曰:「當是才為世生。」以才名拜太學博士。徵署司徒彭城王勰法曹行參軍。帝顧謂勰曰:「蕭賾以王元長為子良法曹,今為汝用祖瑩,豈非倫匹也?」敕
 令掌勰書記。瑩與陳郡袁翻齊名秀出,時人為之語曰:「京師楚楚袁與祖,洛中翩翩祖與袁。」再遷尚書三公郎中。尚書令王肅曾於省中詠《悲平城詩》云:「悲平城,驅馬入雲中。陰山常晦雪,荒松無罷風。」彭城王勰甚嗟其美,欲使肅更詠,乃失語云:「公可更為誦《悲彭城詩》。」肅因戲勰云:「何意呼《悲平城》為《悲彭城》也?」勰有慚色。瑩在座,即云:「悲彭城,王公自未見。」肅云:「可為誦之。」瑩應聲云:「悲彭城,楚歌四面起。屍積石梁亭,血流睢水裏。」蕭甚嗟賞之。勰亦大悅,退謂瑩曰:「卿定是神口,今日若不得卿,幾為吳子所屈。」



 為冀州鎮東府長史,以貨賄事發,除名。後侍
 中崔光舉為國子博士,仍領尚書左戶郎。李崇為都督北討,引瑩為長史,坐截沒軍資除名。未幾,為散騎侍郎。孝昌中,於廣平王第掘得古玉印,敕召瑩與黃門侍郎李琰之辨之。瑩云:「此是于闐國王晉太康中所獻。」乃以墨塗字觀之,果如瑩言,時人稱為博物。累遷國子祭酒,領給事黃門侍郎、幽州大中正,監起居事,又監議事。



 元顥入洛,以瑩為殿中尚書。莊帝還宮,坐為顥作詔罪狀爾朱榮,免官。後除秘書監,中正如故。以參義律歷,賜爵容城縣子。坐事繫於廷尉。會爾朱兆入,焚燒樂署,鐘石管弦略無存者。敕瑩與錄尚書事長孫承業、侍中元孚
 典造金石雅樂,三載乃就。遷車騎大將軍。及孝武登阼,瑩以太常行禮,封文安縣子。天平初,將遷鄴,齊神武困召瑩議之,以功進爵為伯。卒,贈尚書左僕射、司徒公。



 瑩以文學見重,常語人云:「文章須自出機杼成一家風骨,何能共人同生活也。」



 蓋譏世人好竊他文以為己用。而瑩之筆札亦無乏天才,但不能均調,玉石兼有,其製裁之體減於袁、常焉。性爽俠,有節氣,士有窮厄,以命歸之,必見存拯,時亦以此多之。其文集行於世。子珽襲。



 珽字孝徵,神情機警,詞藻遒逸,少馳令譽,為當世所推。起家祕書郎,對策高第,為尚書儀曹郎中,典儀注。嘗為
 冀州刺史萬俟受洛製《清德頌》,其文典麗,由是齊神武聞之。時文宣為并州刺史,署珽開府倉曹參軍。神武口授珽三十六事,出而疏之,一無遺失,大為僚類所賞。時神武送魏蘭陵公主出塞嫁蠕蠕,魏收賦《出塞》及《公主遠嫁詩》二首,珽皆和之,大為時人傳詠。



 珽性疏率,不能廉慎守道。倉曹雖云州局,及受山東課輸,由此大有受納,豐於財產。又自解彈琵琶,能為新曲,招城市年少,歌舞為娛,游集諸倡家,與陳元康、穆子容、任胄、元士亮等為聲色之游。諸人嘗就珽宿,出山東大文綾并連珠孔雀羅等百餘匹,令諸嫗擲摴蒱賭之,以為戲樂。參軍元
 景獻,故尚書令元世俊子也,其妻司馬慶雲女,是魏孝靜帝故博陵長公主所生。珽忽迎景獻妻赴席,與諸人遞寢,亦以貨物所致。其豪縱淫逸如此。常云:「丈夫一生不負身。」



 已文宣罷州,珽例應隨府,規為倉局之間,致請於陳元康。元康為白,由是還任倉曹。珽又委體附參軍事,攝典簽陸子先,為畫計,請糧之際,令子先宣教出倉粟十車。為僚官捉送。神武親問之,珽自言不署,歸罪子先,神武信而釋之。珽出而言曰:「此丞相天緣明鑒,然實孝徵所為。」性不羈,放縱。曾至膠州刺史司馬世雲家飲酒,遂藏銅疊二面,廚人請搜諸客,果於珽懷中得之。見
 者以為深恥。所乘老馬,常稱騮駒。又與寡婦王氏奸通,每人前相聞往復。裴讓之與珽早狎,於眾中嘲珽曰:「卿那得如此詭異,老馬年十歲,猶號騮駒,奸耳順,尚稱娘子。」于時喧然傳之。後為神武中外府功曹。神武宴僚屬,於坐失金叵羅,竇太令飲酒者皆脫帽,於珽髻上得之,神武不能罪也。後為秘書丞,領舍人,事文襄。州客至,請賣《華林遍略》。文襄多集書人,一日一夜寫畢,退其本曰:「不須也。」珽以《遍略》數帙質錢摴蒱,文襄杖之四十。又與令史李雙、倉督成祖等作晉州啟,請粟三千石,代功曹參軍趙彥深宣神武教,給城局參軍。事過典簽高景略,
 景略疑其不實,密以問彥深。彥深答都無此事,遂被推檢。珽即引伏。神武大怒,決鞭二百,配甲坊,加鉗刓,其穀倍徵。未及科,會并州定國寺成,神武謂陳元康、溫子升曰:「昔作芒山寺碑文,時稱妙絕,今定國寺碑當使誰作詞也?」元康因薦珽才學并解鮮卑語。乃給筆札,就禁所具草,二日內成,其文甚麗。神武以其工而且速,特恕不問,然猶免官,散參相府。



 文襄嗣事,以為功曹參軍。及文襄遇害,元康被傷創重,倩珽作書,屬家累事,并云「祖喜邊有少許物,宜早索取。」珽乃不通此書,喚祖喜私問,得金二十五挺,唯與祖喜二挺,餘盡自入,又盜元康家書
 數千卷。祖喜懷恨,遂告元康二弟叔諶、季璩等。叔諶以語楊愔,愔嚬眉答曰:「恐不益亡者。」因此得停。



 文宣作相,珽擬補令史十餘人,皆有受納,而諮取教判,并盜官《遍略》一部。



 時又除珽祕書丞,兼中書舍人。還鄴後,其事皆發。文宣付從事中郎王士闕推檢,并書與平陽公淹,令錄珽付禁,勿令越逃。淹遣田曹參軍孫子寬往喚。珽受命,便爾私逃。黃門郎高德正副留臺事,謀云:「珽自知有犯,驚竄是常。但宣一命向祕書,稱奉并州約束,須《五經》三部,仰丞親檢校催遣。如此,則珽意安,夜當還宅,然後掩取。」珽果如德正圖,遂還宅,薄晚,就家掩之,縛珽送廷
 尉。據犯枉法處絞刑,文宣以珽伏事先世,諷所司,命特寬其罰,遂奏免死除名。天保元年,復被召從駕,依除免例,參於晉陽。



 珽天性聰明,事無難學,凡諸伎藝,莫不措懷。文章之外,又善音律,解四夷語及陰陽占候。醫藥之術,尤是所長。帝雖嫌其數犯刑憲,而愛其才技,令直中書省掌詔誥。珽通密狀,列中書侍郎陸元規,敕令裴英推問,元規以應對忤旨,被配甲坊。除珽尚藥丞,尋選典御。又奏造胡桃油,復為割藏免官。文宣每見之,常呼為賊。文宣崩,普選勞舊,除為章武太守。會楊愔等誅,不之官。授著作郎。數上密啟,為孝昭所忿,敕中書、門下二省
 斷珽奏事。



 珽善為胡桃油以塗畫,為進之長廣王,因言:「殿下有非常骨法,孝徵夢殿下乘龍上天。」王謂曰:「若然,當使兄大富貴。」及即位,是為武成皇帝,擢拜中書侍郎。帝於後園使珽彈琵琶,和士開胡舞,各賞物百段。士開忌之,出為安德太守,轉齊郡太守。以母老乞還侍養,詔許之。會南使入聘,為申勞使。尋為太常少卿、散騎常侍、假儀同三司,掌詔誥。



 初,珽於乾明、皇建之時,知武成陰有大志,遂深自結納,曲相祗奉。武成於天保頻被責,心常銜之。珽至是希旨,上書請追尊太祖獻武皇帝為神武,高祖文宣皇帝改為威宗景烈皇帝,以悅武成。武成
 從之。



 時皇后愛少子東平王儼,願以為嗣,武成以後主體正居長,難於移易。珽私於士開曰:「君之寵幸,振古無二。宮車一日晚駕,欲何以克終?」士開因求策焉。



 珽曰:「宜說主上云:襄、宣、昭帝子俱不得立,今宜命皇太子早踐大位,以定君臣。若事成,中宮少主皆德君,此萬全計也。君且微說,令主上相解,珽當自外表論之。」士開許諾。因有慧星出,太史奏云除舊布新之徵,珽於是上書,言:「陛下雖為天子,未是極貴。案《春秋元命苞》云:『乙酉之歲,除舊革政。』今年太歲乙酉,宜傳位東宮,令君臣之分早定。且以上應天道。」並上魏獻文禪子故事。



 帝從之。由是拜
 祕書監,加儀同三司,大被親寵。



 既見重二宮,遂志於宰相。先與黃門侍郎劉逖友善,乃疏侍中尚書令趙彥深、侍中左僕射元文遙、侍中和士開罪狀,令逖奏之。逖懼,不敢通,其事頗泄。彥深等先詣帝自陳。帝大怒,執珽詰曰:「何故毀我士開?」珽因厲聲曰:「臣由士開得進,本無心毀之。陛下今既問臣,臣不敢不以實對。士開、文遙、彥深等專弄威權,控制朝廷,與吏部尚書尉瑾內外交通,共為表裏,賣官鬻獄,政以賄成,天下歌謠。若為有識所知,安可聞於四裔?陛下不以為意,臣恐大齊之業隳矣!」帝曰:「爾乃誹謗我。」珽曰:「不敢誹謗,陛下取人女。」帝曰:「我以
 其儉餓,故收養之。」珽曰:「何不開倉振給,乃買取將入後宮乎?」帝益怒,以刀鐶築口,鞭杖亂下,將撲殺之。大呼曰:「不殺臣,陛下得名;殺臣,臣得名。若欲得名,莫殺臣,為陛下合金丹。」遂少獲寬放。珽又曰:「陛下有一范增不能用,知如何!」



 帝又怒曰:「爾自作范增,以我為項羽邪?」珽曰:「項羽人身亦何由可及,但天命不至耳。項羽布衣,率烏合眾,五年而成霸王業。陛下藉父兄資財得至此,臣以謂項羽未易可輕。臣何止方於范增?縱擬張良,亦不能及。張良身傅太子,猶因四皓,方定漢嗣。臣位非輔弼,疏外之人,竭力盡忠,勸陛下禪位,使陛下尊為太上,子居宸
 扆,於己及子,俱保休祚。蕞爾張良,何足可數!」帝愈怒,令以土塞其口,珽且吐且言,無所屈撓。乃鞭二百,配甲坊。尋徙於光州。刺史李祖勛遇之甚厚。



 別駕張奉禮希大臣意,上言珽雖為流囚,常與刺史對坐。敕報曰:「牢掌。」奉禮曰:「牢者,地牢也。」乃為深坑,置諸內,苦加防禁,桎梏不離其身,家人親戚不得臨視,夜中以蕪菁子燭熏眼,因此失明。



 武成崩,後主憶之,就除海州刺史。是時陸令萱外干朝政,其子穆提婆愛幸。



 珽乃遺陸媼弟悉達書曰:「趙彥深心腹陰沈,欲行伊、霍事,儀同姊弟豈得平安!



 何不早用智士邪?」和士開亦以珽能決大事,欲以為謀主,
 故棄除舊怨,虛心待之。



 與陸媼言於帝曰:「襄、宣、昭三帝,其子皆不得立,令至尊獨在帝位者,實由祖孝徵。又有大功,宜重報之。孝徵心行雖薄,奇略出人,緩急真可馮仗。且其雙盲,必無反意。請喚取,問其謀計。」帝從之。入為銀青光祿大夫、秘書監,加開府儀同三司。



 和士開死後,仍說陸媼出彥深,以珽為侍中。在晉陽通密啟,請誅瑯邪王。其計既行,漸被任遇。又靈太后之被幽也,珽欲以陸媼為太后,撰魏帝皇太后故事,為太姬言之。謂人曰:「太姬雖云婦人,實是雄傑,女媧已來無有也。」太姬亦稱珽為「國師」、「國寶」。由是拜尚書左僕射,監國史,加特進,入文林館,總監撰書;封燕郡公,食太原郡干,給兵七十人。所住宅在義井坊,旁拓鄰居,大事修築。
 陸媼自往案行,勢傾朝野。



 斛律光甚惡之,遙見竊罵云:「多事乞索小人,欲作何計數!」嘗謂諸將云:「邊境消息,處分兵馬,趙令恆與吾等參論之。盲人掌機密來,全不共我輩語,止恐誤他國家事。」又珽頗聞其言,因其女皇后無寵,以謠言聞上,曰「百升飛上天,明月照長安」。令其妻兄鄭道蓋奏之。帝問珽,珽證實。又說謠云:「高山崩,槲樹舉,盲老公背上下大斧,多事老母不得語。」珽並云:「盲老公是臣」,自云與國同憂戚,勸上行,語「其多事老母,似道女侍中陸氏」。帝以問韓長鸞、穆提婆,並令高元海、段士良密議之,眾人未從。因光府參軍封士讓啟告光反,遂滅其族。



 珽又附
 陸媼,求為領軍,後主許之。詔須覆述,取侍中斛律孝卿署名。孝卿密告高元海,元海語侯呂芬、穆提婆云:「孝徵漢兒,兩眼又不見物,豈合作領軍也?」



 明旦面奏,具陳珽不合之狀,並書珽與廣寧王孝珩交結,無大臣體。珽亦求面見,帝令引入。珽自分疏,並云:「與元海素嫌,必是元海譖臣。」帝弱顏,不能諱,曰:「然。



 珽列元海共司農卿尹子華、太府少卿李叔元、平準令張叔略等結朋樹黨。遂除子華仁州刺史,叔元襄城郡守,叔略南營州錄事參軍。陸媼又唱和之,復除元海鄭州刺史。



 珽自是專主機衡,總知騎兵、外兵事。內外親戚,皆得顯位。後主亦令中要數人扶侍出
 入,著紗帽直至永巷,出萬春門向聖壽堂,每同御榻,論決政事,委任之重,群臣莫比。自和士開執事以來,政體隳壞,珽推崇高望,官人稱職,內外稱美。



 復欲增損政務,沙汰人物。始奏罷京畿府並於領軍,事連百姓,皆歸郡縣;宿衛都督等號位從舊官名,文武服章並依故事。又欲黜諸閹豎及群小輩,推誠延士,為致安之方。



 陸媼、穆提婆議頗同異。珽乃諷御史中丞麗伯律,令劾主書王
 子沖納賂,知其事連提婆,欲使贓罪相及,望因此坐,並及陸媼。猶恐後主溺於近習,欲因後黨為援,請以皇后兄胡君瑜為侍中、中領軍,又徵君瑜兄梁州刺史君璧,欲以為御史中丞。陸媼聞而懷怒,百方排毀,即出君瑜為金紫光祿大夫,解中領軍,君璧還鎮梁州。皇后之廢,頗亦由此。王子沖釋而不問。珽日以益疏,又諸宦者更共譖毀之,無所不至。後主問諸太姬,憫嘿不對。三問,乃下床拜曰:「老婢合死,本見和士開道孝徵多才博學,言為善人,故舉之。此來看之,極是罪過,人實難容,老婢合死。」後主令韓鳳檢案,得其詐出敕受賜
 十餘事,以前與其重誓不殺,遂解珽侍中、僕射,出為北徐州刺史。



 珽求見分疏,韓長鸞積嫌於珽,遣人推出柏閣。珽固求面見,坐不肯行。長鸞乃令軍士牽曳而出,立珽於朝堂,大加誚責。上道後,復令追還,解其開府儀同、郡公,直為刺史。



 至州,會有陳寇,百姓多反。珽不閉城門,守陴者皆令下城靜坐,街巷禁斷人行,雞犬不聽鳴吠。賊無所聞見,莫測所以。或疑人走城空,不設警備。至夜,珽忽令大叫,鼓噪聒天。賊眾大驚,登時走散。後復結陳向城,珽乘馬自出,令錄事參軍王君植率兵馬,仍親臨戰。賊先聞其盲,謂為不能拒抗,忽見親在戎行,彎弧縱鏑,相與驚怪,畏之而罷。時提婆憾之不已,欲令城陷沒賊,雖知危急,不遣救援。珽且守且戰十餘日,賊竟奔走,城卒保全。卒於州。



 子君信,涉獵書史,多諳雜藝。位兼通直散騎常侍,聘陳使副,中書郎。珽出,亦見廢免。



 君信弟君彥,容貌短小,言辭澀訥,少有才學。隋大業中,位至東平郡書佐。



 郡陷翟讓,因為李密所得。密甚禮之,署為記室,軍書羽檄,皆成其手。及密敗,為王世充所殺。



 珽弟孝隱,亦有文學,早
 知名。詞章雖不逮兄,機警有
 口辨,兼解音律。魏末為兼散騎常侍,迎梁使。時徐君房、庾信來聘,名譽甚高,魏朝聞而重之。接對者多取一時之秀,盧元景之徒,並降階攝職,更遞司賓。孝隱少處其中,物議稱美。



 孝隱從父弟茂,頗有辭情,然好酒性率,不為時所重。大寧中,以經學為本鄉所薦,除給事,以疾辭,仍不復仕。珽受任寄,故令呼茂,茂不獲已,暫來就之。



 珽欲為奏官,茂乃逃去。



 珽族弟崇儒,涉學有辭,少以幹局知名。武平末,位司州別駕、通直常侍。入周,為容昌郡太守。隋開皇初,終宕州長史。



 論曰:袁翻弟兄,可為一時才秀;聿修行業,亦乃不殞家風。景文學義見稱,敬安正情自立,休之加以藻思,可謂載德者焉。思伯經明行修,乃惟門素。
 祖瑩幹能藝用,實曰時良;孝徵俊才雖多,適足敗國。叔鸞器懷清峻,元景才幹知名,並匡佐齊初,
 一時推重,美矣哉!



\end{pinyinscope}