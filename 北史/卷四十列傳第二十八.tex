\article{卷四十列傳第二十八}

\begin{pinyinscope}

 韓麒麟程駿李彪孫昶高道悅甄琛高聰韓麒麟,昌黎棘城人。自云漢大司馬增之後也。父瑚,秀容、平原二郡太守。



 麒麟幼而好學,美姿容,善騎射。景穆監國,為東曹主書。文成即位,賜爵漁陽男。



 父亡,在喪有禮。後參征南慕容白曜軍事。進攻升城,師人多傷。及城潰,白曜將坑之。麒麟諫曰:「今方圖進趣,宜示寬厚,勍敵
 在前,而便坑其眾,恐三齊未易圖也。」白曜從之,皆令復業,齊人大悅。後白曜表麒麟與房法壽對為冀州刺史。



 白曜攻東陽,麒麟上義租六十萬斛,并攻戰器械,於是軍須無乏。及白曜被誅,麒麟停滯多年。



 孝文時,拜齊州刺史,假魏昌侯。在官寡於刑罰,從事劉普慶說麒麟曰:「明公仗節方夏,無所斬戮,何以示威?」麒麟曰:「人不犯法,何所戮乎?若必須斬斷以立威名,當以卿應之。」普慶慚懼而退。麒麟以親附之人,未階臺官,士人沈抑,乃表請守宰有闕,宜推用豪望,增置吏員,廣延賢哲,則華族蒙榮,良才獲敘,懷德安土,庶或在茲。朝議從之。



 太和十一
 年,京都大饑,麒麟表陳時務曰:古先哲王,經國立政,積儲九稔,謂之太平。故躬藉千畝,以率百姓。用能衣食滋茂,禮教興行。逮於中代,亦崇斯業,入粟者與斬敵同爵,力田者與孝悌均賞。



 實百王之常軌,為政之所先。今京師人庶,不田者多;游食之口,三分居二。蓋一夫不耕,或受其飢,況於今者,動以萬計?故頃年山東遭水,而人有餒終,今秋京都遇旱,穀價踴貴,實由農人不勸,素無儲積故也。



 伏惟陛下天縱欽明,道高三五,上垂覆載之澤,下有凍餒之人,皆由有司不為其制,長吏不恤其本。自承平日久,豐穰積年,競相矜夸,浸成侈俗。故令耕者日
 少,田者日荒。穀帛罄於府庫,寶貨盈於市里,衣食匱於室,麗服溢於路。飢寒之本,實在於斯。愚謂凡珍玩之物,皆宜禁斷。吉凶之禮,備為格式,令貴賤有別,人歸朴素。制天下男女,計口受田。宰司四時巡行,臺使歲一案檢,勤相勸課,嚴加賞罰。數年之中,必有盈贍,雖遇凶災,免於流亡矣。



 往年校比戶貫,租賦輕少。臣所統齊州,租粟纔可給俸,略無入倉。雖於人為利,而不可長久。脫有戎役,或遭天災,恐供給之方,無所取濟。請減絹布,增益穀租,年豐多積,歲儉出振。所謂私人之穀,寄積於官;官有宿積,則人無荒年矣。



 卒官,遺敕其子,殯以素棺,事從儉
 約。



 麒麟立性恭慎,恒置律令於坐傍。臨終之日,唯有俸絹數十疋,其清貧如此。



 贈散騎常侍、燕郡公,謚曰康。長子興宗,字茂先。好學有文才,位秘書中散。卒,贈漁陽太守。



 子子熙,字元雍。少自修整,頗有學識,為清河王懌郎中令。初,子熙父以爵讓弟顯宗,不受;子熙成父素懷,卒亦不襲。及顯宗卒,子熙別蒙賜爵,乃以先爵讓弟仲穆。兄弟友愛如此。母亡,居喪有禮。子熙為懌所眷遇,遂闕位,待其畢喪後,復引用。及元叉害懌,久不得葬。子熙為之憂悴,屏居田野。每言王若不得復封,以禮遷葬,誓以終身不仕。後靈太后反政,以叉為尚書令,解其領軍。子
 熙與懌中大夫劉定興、學官令傅靈、賓客張子慎伏闕上書,理懌之冤,極言元叉、劉騰誣誷。書奏,靈太后義之,乃引子熙為中書舍人。後遂剖騰棺,賜叉死。尋修國史。建義初,兼黃門,尋為正。



 子熙清白自守,不交人事。又少孤,為叔顯宗所撫養。及顯宗卒,顯宗子伯華又幼,子熙愛友等於同生。長猶共居,車馬資財,隨其費用,未嘗見於言色。又上書求析階與伯華,於是除伯華東太原太守。及伯華在郡,為刺史元弼所辱。子熙乃泣訴朝廷。明帝詔遣案檢,弼遂大見詰讓。



 爾朱榮之禽葛榮,送至京師。莊帝欲面數之,子熙以為榮既元凶,自知必死,恐
 或不遜,無宜見之。爾朱榮聞而大怒,請罪子熙。莊帝恕而不責。及邢杲起逆,詔子熙慰勞。杲詐降,子熙信之。遷至樂陵,杲復反,子熙還。坐付廷尉,論以大辟,恕死免官。孝武初,領著作,以奉冊勳,封歷城縣子。天平初,為侍讀,除國子祭酒。子熙儉素安貧,常好退靜。遷鄴之始,百司並給兵力,時以祭酒閑務,止給二人。或有令其陳請者,子熙曰:「朝廷自不與祭酒兵,何關韓子熙事。」論者高之。元象中,加衛大將軍。



 先是,子熙與弟娉王氏為妻,姑之女也,生二子。子熙尚未婚,後遂與寡嫗李氏姦合而生三子。王、李不穆,迭相告言。子熙因此慚恨,遂以發疾。卒,
 遺戒不求贈謚,其子不能遵奉,遂至干謁。武定初,贈驃騎大將軍、儀同三司、幽州刺史。



 興宗弟顯宗,字茂親。剛直,能面折廷諍,亦有才學。沙門法撫,三齊稱其聰悟。嘗與顯宗校試,抄百餘人名,各讀一遍,隨即覆呼,法撫猶有一二舛謬,顯宗了無誤錯。法撫歎曰:「貧道生平以來,唯服郎耳。」



 太和初,舉秀才,對策甲科,除著作佐郎。後兼中書侍郎。既定遷都,顯宗上書:一曰:竊聞輿駕今夏若不巡三齊,當幸中山。竊以為非計也。何者?當今徭役宜早息,洛京宜速成。省費則徭役可簡,并功則洛京易就。願早還北京,以省諸州供帳之費,則南州免雜徭之煩,
 北都息分析之歎;洛京可以時就,遷者僉爾如歸。



 二曰:自古聖帝必以儉約為美,亂主必以奢侈貽患。仰惟先朝,皆卑宮室而致力於經略,故能基宇開廣,業祚隆泰。今洛陽基趾,魏明所營,取譏前代。伏惟陛下損之又損之。頃來北都富室,競以第宅相尚,今因遷徙,宜申禁約,令貴賤有檢,無得踰制。端廣衢路,通利溝洫,使寺署有別,士庶異居,永垂百世不刊之範。



 三曰:竊聞輿駕還洛陽,輕將數千騎,臣甚為陛下不取也。夫千金之子,猶坐不垂堂,況萬乘之尊,富有四海乎。清道而行,尚恐銜橛之失,況履涉山河而不加三思哉。



 四曰:竊惟陛下耳聽
 法音,目玩墳典,口對百辟,心慮萬機,晷昃而食,夜分而寢。加以孝思之至,與時而深;文章之業,日成篇卷。雖睿明所用,未足為煩,然非所以嗇神養性,熙無疆之祚。莊周有言:「形有待而智無涯,以有待之形,役無涯之智,殆矣。」此愚臣所不安也。



 孝文頗納之。顯宗又上言:前代取士,必先正名,故有賢良方正之稱。今州郡貢察,徒有秀、孝之名,而無秀、孝之實。而朝廷但檢其門望,不復彈坐。如此則可令別貢門望以敘士人,何假冒秀、孝之名也?夫門望者,是其父祖之遺烈,亦何益於皇家。益於時者,賢才而已。茍有其才,雖屠釣奴虜之賤,聖皇不恥以為
 臣;茍非其才,雖三后之胤,自墜於皁隸矣。議者或云:今世等無奇才,不若取士於門。此亦失矣。豈可以世無周、邵,便廢宰相而不置哉。但當校其有寸長銖重者,即先敘之,則賢才無遺矣。



 又曰:夫帝皇所以居尊以御下者,威也;兆庶所以徙惡以從善者,法也。是以有國有家,必以刑法為政,生人之命,於是而在。有罪必罰,罰必當辜,則雖以捶撻薄刑,而人莫敢犯。有制不行,人得僥倖,則雖參夷之誅,不足以肅。自太和以來,未多坐盜棄市,而遠近肅清。由此言之,止姦在於防檢,不在嚴刑。今州郡牧守,邀當時之名,行一切之法;臺閣百官,亦咸以深酷
 為無私,以仁恕為容盜。迭相敦厲,遂成風俗。陛下居九重之內,視人如赤子;百司分萬務之要,遇下如仇讎。



 是則堯、舜止一人,而桀、紂以千百,和氣不至,蓋由於此。宜敕示百官,以惠元元之命。



 又曰:昔周王為犬戎所逐,東遷河洛,鎬京猶稱宗周,以存本也。光武雖曰中興,實自草創,西京尚置京尹,亦不廢舊。今陛下光隆先業,遷宅中土,稽古復禮,於斯為盛。按《春秋》之義,有宗廟謂之都,無謂之邑,此不刊之典也。況北代,宗廟在焉,山陵託焉,王業所基,聖躬所載,其為神鄉福地,實亦遠矣。今便同之郡國,臣竊不安。愚謂代京宜建畿置尹,一如故事。崇
 本重舊,以光萬葉。



 又曰:「伏見洛京之制,居人以官位相從,不依族類。然官位非常,有朝榮而夕悴,則衣冠淪於廝豎之邑,臧獲顯於膏腴之里,物之顛倒,或至於斯。古之聖王,必令四人異居者,欲其業定而志專。業定則不偽,志專則不淫,故耳目所習,不督而就;父兄之教,不肅而成。仰惟太祖道武皇帝,創基撥亂,日不暇給,然猶分別士庶,不令雜居,伎作屠沽,各有攸處。但不設科禁,買賣任情,販貴易賤,錯居渾雜。假令一處彈箏吹笛,緩舞長歌;一處嚴師苦訓,誦《詩》講《禮》,宣令童齔,任意所從,其走赴舞堂者萬數,往就學館者無一。此則伎作不可雜
 居,士人不宜異處之明驗也。故孔父云里仁之美,孟母弘三徙之訓。賢聖明誨,若此之重。今令伎作之家習士人風禮,則百年難成;令士人兒童效伎作容態,則一朝可得。以士人同處,則禮教易興;伎作雜居,則風俗難改。朝廷每選舉人士,則校其一婚一官,以為升降,何其密也。至於伎作官塗,得與膏梁華望接閈連甍,何其略也。今稽古建極,光宅中區,凡所徙居,皆是公地。分別伎作,在於一言,有何為疑,而虧盛美?



 又曰:自南偽相承,竊有淮北,欲擅中華之稱,且以招誘邊人,故僑置中州郡縣。自皇風南被,仍而不改,凡有重名,其數甚眾,非所以疆
 域物士,必也正名之謂也。愚以為可依地理舊名,一皆釐革,小者並合,大者分置。及中州郡縣,昔以戶少并省,今人口既多,亦可復舊。君人者,以天下為家,不得有所私也。故倉庫儲貯,以俟水旱之災,供軍國之用。至於有功德者,然後加賜。爰及末代,乃寵之所隆,賜賚無限。自比以來,亦為太過。在朝諸貴,受祿不輕,土本被綺羅,僕妾厭梁肉,而復厚賚屢加,動以千計。若分賜鰥寡,贍濟實多。如不悛革,豈「周急不繼富」之謂也?



 又曰:諸宿衛內直者,宜令武官習弓矢,文官諷書傳。無令繕其蒲博之具,以成褻狎之容,徙損朝儀,無益事實。如此之類,一宜
 禁止。



 帝善之。



 孝文曾謂顯宗及程靈虯曰:「著作之任,國書是司。卿等之文,朕自委悉;中省之品,卿等所聞。若欲取況古人,班、馬之徒,固自遼闊。若求之當世,文學之能,卿等應推崔孝伯。」又謂顯宗曰:「校卿才能,可居中第。」謂程靈虯曰:「卿與顯宗,復有差降,可居下上。」顯宗曰:「臣才第短淺,比於崔光,實為隆渥。然臣竊謂陛下貴古而賤今。昔揚雄著《太玄經》,當時不免覆甕之譚,二百年外,則越諸子。今臣所撰,雖未足光述帝載,然萬祀之後,仰觀祖宗巍巍之功,上睹陛下明明之德,亦何謝欽明於《唐典》,慎徽於《虞書》。」帝曰:「假使朕無愧於虞舜,卿復何如堯
 臣?」顯宗曰:「陛下齊蹤堯、舜,公卿寧非二八之儔。」



 帝曰:「卿為著作,僅名奉職,未是良史也。」顯宗曰:「臣仰遭明時,直筆無懼,又不受金,安眠美食,此優於遷、固也。」帝哂之。後與員外郎崔逸等參定朝儀。



 帝曾詔諸官曰:「近代已來,高卑出身,恆有常分。朕意所為可,復以為不可,宜校量之。」李沖曰:「未審上古已來,置官列位,為欲為膏梁兒地,為欲益政贊時?」帝曰:「俱欲為人。」沖曰:「若欲為人,陛下今日何為專崇門品,不有拔才之詔?」帝曰:「茍有殊人之技,不患不知。然君子之門,假使無當世之用者,要自德行純篤,朕是以用之。」沖曰:「傅巖、呂望,豈可以門見舉?」帝曰:「
 如此濟世者希,曠代有一兩耳。」沖謂諸卿士曰:「適欲請救諸賢。」秘書令李彪曰:「師旅寡少,未足為援,意有所懷,敢不盡言於聖日。陛下若專以地望,不審魯之三卿,孰若四科?」帝曰:「猶如向解。」顯宗進曰:「陛下光宅洛邑,百禮惟新,國之興否,指此一選。且以國事論之,不審中祕監、令之子,必為秘書郎,頃來為監、令者,子皆可為不?」帝曰:「卿何不論當世膏腴為監、令者?」顯宗曰:「陛下以物不可類,不應以貴承貴,以賤襲賤。」帝曰:「若有高明卓爾,才具俊出者,朕亦不拘此例。」後為本州中正。



 二十一年,車駕南征,以顯宗為右軍府長史、統軍。次赭陽,齊戍主成公
 期遣其軍主胡松、高法援等並引蠻賊,來擊軍營。顯宗拒戰,斬法援首。顯宗至新野,帝曰:「何不作露布也?」顯宗曰:「臣頃見鎮南將軍王肅獲賊二三,驢馬數匹,皆為露布。臣在東觀,私每哂之。近雖仰憑威靈,得摧醜虜,兵寡力弱,禽斬不多。



 脫復高曳長縑,虛張功捷,尤而效之,其罪彌甚。所以斂毫卷帛,解上而已。」帝笑曰:「如卿此勳,誠合茅社,須赭陽平定,檢審相酬。」新野平,以顯宗為鎮南廣陽王嘉諮議參軍。顯宗上表,頗自矜伐,訴前征勛。詔曰:「顯宗進退無檢,虧我清風,付尚書推列以聞。」兼尚書張彞奏免顯宗官。詔以白衣守諮議,展其後效。



 顯宗既
 失意,遇信向洛,乃為五言詩贈御史中尉李彪,以申憤結。二十三年卒。顯宗撰馮氏《燕志》、《孝友傳》各十卷。景明初,追赭陽勳,賜爵章武男。子伯華襲。



 程逡,字驎駒,本廣平曲安人也。六世祖良,晉都水使者,坐事流涼州。祖父肇,呂光人部尚書。駿少孤貧,居喪以孝稱。師事劉延明,性機敏好學,晝夜無倦。



 延明謂門人曰:「舉一隅而以三隅反者,此子亞之也。」駿白延明曰:「今名教之儒,咸謂老莊其言虛誕,不切實要,不可以經世。駿為不然。夫老子著抱一之言,莊生申性本之旨,若斯者,可謂至順矣。人若乖一,則煩偽生;爽性,則沖真喪。」



 延
 明曰:「卿年尚幼,言若老成,美哉。」由是聲譽益播。沮渠牧犍擢為東宮侍講。



 太延五年,涼州平,遷于京師。為司徒崔浩所知。文成踐阼,為著作郎。皇興中,除高密太守。尚書李敷奏駿實史才,方申直筆,請留之。書奏,從之。獻文屢引駿與論《易》、《老》義,顧謂群臣曰:「朕與此人言,意甚開暢。」問駿年,對曰:「六十一。」帝曰:「昔太公老而遭文王,卿今遇朕,豈非早也。」駿曰:「臣雖才謝呂望,陛下尊過西伯。覬天假餘年,竭《六韜》之效。」



 延興末,高麗王璉求納女於掖庭,假駿散騎常侍,賜爵安豐男,持節如高麗迎女。駿至平壤城。或勸璉曰:「魏昔與燕婚,既而伐之,由行人具其
 夷險故也。今若送女,恐不異於馮氏。」璉遂謬言女喪。駿與璉往復經年,責璉以義方。璉不勝其忿,遂斷駿從者酒食,欲逼辱之,憚而不敢害。會獻文崩,乃還。拜秘書令。



 初,遷神主于太廟,有司奏:舊事,廟中執事官例皆賜爵,今宜依舊。詔百寮評議,群臣咸以為宜依舊事。駿獨以為不可,表曰:「臣聞名器為帝王所貴,山河為區夏之重,是以漢祖有約,非功不侯。未見預事於宗廟,而獲賞於疆土。雖復帝王制作,弗相沿襲。然一時恩澤,豈足為長世之軌乎。」書奏,從之。文明太后謂群臣曰:「言事,固當正直而準古典;安可依附暫時舊事乎!」賜駿衣一襲,帛二
 百匹。又詔曰:「駿歷官清慎,言事每愜。門無挾貨之賓,室有懷道之士。可賜帛六百匹,旌其儉德。」駿悉散之親舊。



 性介直,不競時榮。太和九年正月病篤,遺命曰:「吾存尚儉薄,豈可沒為奢厚哉。昔王孫裸葬,有感而然;士安籧篨,頗亦矯厲。可斂以時服,明器從古。」



 初駿病甚,孝文、文明太后遣使者更問其疾,敕侍御師徐謇診視,賜以湯藥。臨終,詔以小子公稱為中散,從子靈虯為著作佐郎。及卒,孝文、文明太后傷惜之。賜東園秘器、朝服一稱、帛三百匹,贈兗州刺史、曲安侯,謚曰憲。所作文章,自有集錄。



 李彪,字道固,頓丘衛國人也,孝文賜名焉。家寒微,少孤貧,有大志,好學不倦。初受業於長樂監伯陽,伯陽稱美之。晚與漁陽高悅、北平陽尼等將隱名山,不果而罷。悅兄閭博學高才,家富典籍,彪遂於悅家手抄口誦,不暇寢食。既而還鄉里。平原王陸睿年將弱冠,雅有志業。娶東徐州刺史博陵崔鑒女,路由冀、相,聞彪名而詣之。修師友之禮,稱之州郡遂。遂舉孝廉,至京師,館而受業焉。高閭稱之朝貴,李沖禮之其厚,彪深宗附之。



 孝文初,為中書教學博士。後假散騎常侍、衛國子,使於齊。遷秘書丞,參著作事。自成帝已來,至於太和,崔浩、高允著述國書,
 編年序錄為《春秋》體,遺落時事。彪與祕書令高祐始奏從遷、固體,創為紀、傳、表、志之目焉。



 彪又表上封事七條,曰:古先哲王之為制也,自天子以至公卿,下及抱關擊柝,其宮室車服,各有差品。



 小不得僭大,賤不得踰貴。夫然,故上下序而人志定。今時浮華相競,情無常守;大為消功之物,巨制費力之事,豈不謬哉。夫消功者,錦繡彫文是也;費力者,廣宅高宇,壯制麗飾是也。其妨男業害女工者,可勝言哉!漢文時,賈誼上疏,云今之王政可為長太息者六,此即是其一也。夫上之所好,下必從之。故越王好勇而士多輕死;楚王好瘠而國有飢人。今二聖
 躬行儉素,詔令殷勤,而百姓之奢猶未革者,豈楚、越之人易變如彼,大魏之士難化如此?此蓋朝制不宣,人未見德使之然耳。



 臣愚以為第宅車服,自百官以下至於庶人,宜為其等制。使貴不逼賤,卑不僭高,不可以稱其侈意,用違經典。



 其二曰:《易》稱:「主器者莫若長子。」《傳》曰:「太子奉塚嫡之粢盛。」然則祭無主則宗廟無所饗,冢嫡廢則神器無所傳。聖賢知其如此,故垂誥以為長世之法。



 昔姬王得斯道也,故恢崇儒術以訓世嫡。世嫡於是乎習成懿德,用大協於黎蒸。是以世統黎元,載祀八百。逮嬴氏之君於秦也,弗以義方教厥塚子,塚子於是習成
 兇德,肆虐以臨黔首。是以饗年不永,二世而亡。亡之與興,道在於師傅。故《禮》云:「塚子生,因舉以禮,使士負之,有司齊肅端冕,見于南郊。」明塚嫡之重,見乎天也。「過闕則下,過廟則趨,」明孝敬之道也。然古之太子,自為赤子而教固以行矣。此則遠世之鏡也。高宗文成皇帝慨少時師不勤教,嘗謂群臣曰:「朕始學之日,年尚幼沖,情未能專。既臨萬機,不遑溫習。今而思之,豈非唯予之咎,抑亦師傅之不勤。」尚書李免冠而謝。此則近日之可鑒也。伏惟太皇太后翼贊高宗,訓成顯祖,使巍巍之功,邈乎前王。陛下幼蒙鞠誨,聖敬日躋,及儲宮誕育,復親撫誥,日
 省月課,實勞神慮。今誠宜準古立師傅,以詔導太子。詔導正則太子正,太子正則皇家慶,皇家慶則人事幸甚矣。



 其三曰:《記》云:國無三年之儲,謂國非其國。光武以一畝不實,罪及牧守。聖人之憂世重穀,殷勤如彼;明君之恤人勸農,相切若此。頃年山東饑,去歲京師儉,內外人庶,出入就豐。既廢營產,疲困乃加,又於國體,實有虛損。若先多積穀,安而給之,豈有驅督老弱,糊口千里之外。以今況古,誠可懼也。臣以為宜析州郡常調九分之二,京都度支歲用之餘,各立官司。年豐糴積於倉,時儉則加私之二,糶之於人。如此,人必事田以買官絹,又務貯
 財以取官粟。年登則常積,歲凶則直給。



 又別立農官,取州郡戶十分之一以為屯人。相水陸之宜,料頃畝之數,以贓贖雜物餘財市牛科給,令其肆力。一夫之田,歲責六十斛,甄其正課並征戍雜役。行此二事,數年之中,則穀積而人足,雖災不害。



 臣又聞前代明王皆務懷遠人,禮賢引滯。故漢高過趙,求樂毅之胄;晉武廓定,旌吳、蜀之彥。臣謂宜於河表七州人中,擢其門才,引令赴闕,依中州官比,隨能序之。一可以廣聖朝均新舊之義,二可以懷江、漢歸有道之情。



 其四曰:漢制,舊斷獄報重盡季冬,至孝章時改盡十月,以育三微。後歲旱,論者以不十
 月斷獄,陰氣微,陽氣泄,以故致旱,事下公卿。尚書陳寵曰:「冬至陽氣始萌,故十一月有射干芸荔之應,周以為春。十二月陽氣上通,雉雊雞乳,殷以為春。十三月陽氣已至,蟄蟲皆震,夏以為春。三微成著,以通三統。三統之月斷獄流血,是不稽天意也。」章帝善其言,卒以十月斷。今京都及四方斷獄報重,常竟季冬,不推三正以育三微。寬宥之情,每過於昔,遵之典憲,猶或闕然。今豈所謂助陽發生,垂奉微之仁也?誠宜遠稽周典,近採漢制,天下斷獄起自初秋,盡於孟冬。不於三統之春,行斬絞之刑。如此則道協幽顯,仁垂後昆矣。



 其五曰:古者大臣有
 坐不廉而廢者,不謂之不廉,乃曰簠簋不飾。此君之所以禮貴臣,不明言其過也。臣有大譴,則白冠氂纓盤水加劍,造室而請死,此臣之所以知罪而不敢逃刑也。聖朝賓遇大臣,禮崇古典,自太和降,有負罪當陷大辟者,多得歸第自盡。遣之日,深垂隱愍,言發淒淚,百官莫不見,四海莫不聞,誠足以感將死之心,慰戚屬之情。然恩發於衷,未著永制,此愚臣所以敢陳末見。



 昔漢文時,人有告丞相勃謀反者,逮繫長安獄,頓辱之與皂隸同。賈誼乃上書,極陳君臣之義,不宜如是。夫貴臣者,天子為其改容而體貌之,吏人為共俯伏而敬貴之。其有罪過,
 廢之可也,賜之死可也;若束縛之,輸之司寇,搒笞之,小吏詈罵之,殆非所以令眾庶見也。及將刑也,臣則北面再拜,跪而自裁。天子曰:「子大夫自有過耳,吾遇子有禮矣。上不使人抑而刑之也。」孝文深納其言。是後大臣有罪,皆自殺不受刑。至孝武時,稍復下獄。良由孝文行之當時,不為永制故耳。



 今天下有道,庶人不議之時,安可陳瞽言於朝?且恐萬世之後,繼體之主有若漢武之事。焉得行恩當時,不著長世之制乎。



 其六曰:《孝經》稱父子之道天性,蓋明一體而同氣,可共而不右離者也。及其有罪不相及者,乃君上之厚恩也。而無情之人,父兄繫
 獄,子弟無慘惕之容;子弟即刑,父兄無愧恧之色。宴安榮位,游從自若,軍馬仍華,衣冠猶飾。寧是同體共氣,分憂均戚之理也?臣愚以為父兄有犯,宜令子弟素服肉袒,詣闕請罪;子弟有坐,宜令父兄露板引咎,乞解所司。若職任必要,不宜許者,慰勉留之。如此,足以敦厲凡薄,使人知有所恥矣。



 其七曰:《禮》云:臣有大喪,君三年不呼其門。此聖人緣情制禮,以終孝子之情也。



 周季陵夷,喪禮稍亡,是以要糸至即戎,素冠作刺。逮乎虐秦,殆皆泯矣。漢初,軍旅屢興,未能遵古。至宣帝時,人當從軍屯者,遭大父母、父母死,未滿三月,皆弗徭役。其朝臣喪制,未有
 定聞。至後漢元初中,大臣有重憂,始得去官終服。



 暨魏武、孫、劉之世,日尋干戈,前世禮制,復廢不行。晉時鴻臚鄭默喪親,固請終服,武帝感其孝誠,遂著令以為常。



 聖魏之初,撥亂反正,未遑建終喪之制。今四方無虞,百姓安逸,誠是孝慈道洽,禮教興行之日也。然愚臣所懷,竊有未盡。伏見朝臣丁大憂者,假滿赴職,衣錦乘軒,從效廟之祀;鳴玉垂緌,同節慶之宴。傷人子之道,虧天地之經。愚謂如有遭父母喪者,皆得終服。若無其人有曠官者,則優旨慰喻,起令視事。但綜理所司,出納敷奏而已,國之吉慶,一令無預。其軍戎之警,墨縗從役,雖愆於禮,
 事所宜行也。



 帝覽而善之,尋皆施行。彪稍見禮遇。詔曰:「彪雖宿非清第,代闕華資,然識性嚴聰,學博墳籍,剛辯之才,頗堪時用。兼優吏職,載宣朝美,若不賞庸敘績,將何以勸獎勤能。特遷祕書令。以參議律令之勤,賜帛五百匹,馬一匹、牛二頭。」



 其年,加員外散騎常侍,使於齊。



 齊遣其主客郎劉繪接對,並設宴樂。彪辭樂。及坐,彪曰:「向辭樂者,卿或未相體。我皇孝性自天,追慕罔極,故有今者喪除之議。去三月晦,朝臣始除縗裳,猶以素服從事。裴、謝在北,固應具此。今辭樂,想卿無怪。」繪答言:「請問魏朝喪禮竟何所依?」彪曰:「高宗三年,孝文踰月。今聖上追
 鞠育之深恩,感慈訓之厚德,報於殷、漢之間,可謂得禮之變。」繪復問:「若欲遵古,何不終三年?」



 彪曰:「萬機不可久曠,故割至慕,俯從群議。服變不異三年,而限同一期,可謂失禮?」繪言:「汰哉叔氏,專以禮許人。」彪曰:「聖朝自為曠代之制,何關許人。」繪言:「百官總己聽於塚宰,萬機何慮於曠?」彪曰:「五帝之臣,臣不若君,故君親攬其事。三王君臣智等,故共理機務。主上親攬,蓋遠軌軒、唐。」彪將還,齊主親謂彪曰:「卿前使還日,賦阮詩云:『但願長閑暇,後歲復來游。』果如今日。卿此還也,復有來理否?」彪答:「請重賦阮詩曰:『宴衍清都中,一去永矣哉。』」齊主惘然曰:「清都可
 爾,一去何事!觀卿此言,似成長闊。朕當以殊禮相送。」遂親至瑯邪城,登山臨水,命群臣賦詩以送別。其見重如此。彪前後六度銜命,南人奇其謇博。後為御史中尉,領著作郎。



 彪既為孝文所寵,性又剛直,遂多劾糾,遠近畏之。豪右屏氣。帝常呼為李生,從容謂群臣曰:「吾之有李生,猶漢之有汲黯。」後除散騎常侍,領御史中尉,解著作事。帝宴群臣於流化池,謂僕射李沖曰:「崔光之博,李彪之直,是我國得賢之基。」



 車駕南伐,彪兼度支尚書,與僕射李沖、任城王澄等參理留臺事。彪素性剛豪,與沖等意議乖異,遂形於聲色,殊無降下之心。沖積其前後罪
 過,乃於尚書省禁止彪,上表曰:「案臣彪昔於凡品,特以才拔,等望清華,司文東觀,綢繆恩眷,繩直憲臺,左加金璫,右珥蟬冕。東省。宜感恩厲節,忠以報德。而竊名忝職,身為違傲,矜勢高亢,公行僭逸。坐與禁省,冒取官材,輒駕乘黃,無所憚懼。肆志傲然,愚聾視聽。此而可忍,誰不可懷。臣今請以見事免彪所居職,付廷尉獄。」沖又表曰:臣與彪相識以來,垂二十二載。彪始南使之時,見其色厲辭辯,臣之愚識,謂是拔萃之一人。及彪官位升達,參與言宴,聞彪平章古今,商略人物。興言於侍筵之次,啟論於眾英之中;賞忠識正,發言懇惻,惟直是語,辭無隱避。臣
 雖下愚,輒亦欽其正直。及其始居司直,執志徑行,其所彈劾,應弦而倒。赫赫之威,振於下國;肅肅之稱,著自京師;天下改目,貪暴僉手。然時有私於臣云其威暴者,臣以直繩之官,人所忌疾,風謗之際,易生音謠,心不承信。



 往年以河陽事,曾與彪在領軍府共太尉、司空及領軍諸卿等集閱廷尉所問囚徒。



 時有人訴枉者,二公及臣少欲聽採。語理未盡,彪便振怒,東坐攘袂揮赫,口稱賊奴,叱吒左右。高聲大呼曰:「南臺中取我木手去,搭奴肋折!」雖有此言,終竟不取。即言:「南臺所問,唯恐枉活,終無枉死。」時諸人以所枉至重,有首實者多,又心難彪,遂各
 默爾。因緣此事,臣遂心疑有濫,知其威虐。猶謂益多損少,故不以申徹,實失為臣知無不聞之義。及去年大駕南行以來,彪兼尚書,日夕共事,始乃知其言與行舛,是己非人,專恣無忌,尊身忽物。臣與任城卑躬曲己,其所欲者無不屈從。依事求實,悉有成驗。如臣列得實,宜亟投彪於有北,以除姦矯之亂政;如臣列無證,宜放臣於四裔,以息青蠅之白黑。



 帝在懸瓠,覽表歎愕曰:「何意留京如此也!」有司處彪大辟;帝恕之,除名而已。



 彪尋歸本鄉。帝北幸鄴,彪野服稱草茅臣,拜迎鄴南。帝曰:「朕以卿為已死。」



 彪對曰:「子在,回何敢死。」帝悅,因謂曰:「朕期卿每
 以貞松為志,歲寒為心,卿應報國,盡心為用,近見彈文,殊乖所以。卿罹此譴,為朕與卿?為宰事?為卿自取?」彪曰:「臣愆由己至,罪自身招,實非陛下橫與臣罪,又非宰事無辜濫臣。



 臣罪既如此,宜伏東皋之下,不應遠點屬車之清塵。但伏承聖躬不豫,臣肝膽塗地,是以敢至,非謝罪而來。」帝曰:「朕欲用卿,憶李僕射不得。」帝尋納宋弁之言,將復採用。會留臺表至,言彪與御史賈尚往窮庶人恂事,理有誣抑,奏請收彪。彪自言事枉,帝明彪無此,遣左右慰勉之。聽以牛車散載,送之洛陽。會赦得免。



 宣武踐阼,彪自託於王肅,又與郭祚、崔光、劉芳、甄琛、邢巒等
 詩書往來,迭相稱重。因論求復舊職,修史官之事,肅等許為左右。彪乃表曰:惟我皇魏之奄有中華也,歲越百齡,年幾十紀,史官敘錄,未充其盛。加以東觀中圮,冊勳有闕,美隨日落,善因月稀。故諺曰:「一日不書,百事荒蕪。」至于太和之十一年,先帝,先后召名儒博達之士,以充麟閣之選。于時忘臣眾短,采臣片志,令臣出納,授臣丞職,猥屬斯事,無所與讓。高祖時詔臣曰:「平爾雅志,正爾筆端,書而不法,後世何觀。」臣奉以周旋,不敢失墜。



 伏惟孝文皇帝承天地之寶,崇祖宗之業,景功未就,奄焉崩殂,凡百黎萌,若無天地。賴遇陛下體明睿之真,應保合
 之量,恢大明以燭物,履靜恭以和邦。天清其氣,地樂其靜,可謂重明疊聖,元首康哉。《記》曰:「善迹者欲人繼其行,善歌者欲人繼其聲。」故《傳》曰:「文王基之,周公成之。」然先皇之茂勛聖達,今王之懿美洞鑒,準之前代,其德靡悔也。時哉時哉,可不光昭哉!合德二儀者,先皇之陶鈞也。齊明日月者,先皇之洞照也。慮周四時者,先皇之茂功也。合契鬼神者,先皇之玄燭也。遷都改邑者,先皇之達也。變是協和者,先皇之鑒也。思同書軌者,先皇之遠也。守在四夷者,先皇之略也。海外有截者,先皇之威也。禮由岐陽者,先皇之義也。張樂岱郊者,先皇之仁也。鑾幸
 幽漠者,先皇之智也。燮伐南荊者,先皇之禮也。升中告成者,先皇之肅也。親虔宗社者,先皇之敬也。兗實無闕者,先皇之德也。開物成務者,先皇之貞也。觀乎人文者,先皇之蘊也。革弊創新者,先皇之志也。孝慈道洽者,先皇之衷也。先皇有大功二十,加以謙尊而光,為而弗有者,可謂四三皇而六五帝矣。誠宜功書於竹素,聲播於金石。



 臣竊謂史官之達者,大則與日月齊其明,小則與四時並其茂,故能聲流無窮,義昭來裔。是以金石可滅,而風流不泯者,其唯載籍乎。諺曰:「相門有相,將門有將。」斯不唯其性,蓋言習之所得也。竊謂天文之官,太史之
 職,如有其人,宜其世矣。是以談、遷世事而功立,彪、固世事而名成,此乃前鑒之軌轍,後鏡之蓍龜也。然前代史官之不終業者,皆陵遲之世,不能容善。是以平子去史而成賦,伯喈違閣而就志。近僭晉之世,有佐郎王隱,為著作虞預所毀,亡官在家。晝則樵薪供爨,夜則觀文屬綴,集成《晉書》,存一代之事。司馬紹敕尚書唯給筆札而已。



 國之大籍,成於私家,末世之弊,乃至如此。此史官之不遇時也。今大魏之史,職則身貴,祿則親榮,優哉游哉,式穀令爾休矣!而典謨弗恢者,其有以也。而故著作漁陽傅毗、北平陽尼、河間邢產、廣平宋弁、昌黎韓顯宗並
 以文才見舉,注述是同,並登年不永,弗終茂績。前著作程靈虯同時應舉,共掌此務,今徙他職,官非所司。唯著作崔光一人,雖不移任,然侍官兩兼,故載述致闕。



 臣聞載籍之興,由於大業;雅頌垂薦,起於德美。昔史談誡其子遷曰:「當世有美而不書,汝之罪也。」是以久而見美。孔明在蜀,不以史官留意,是以久而受譏。《書》稱「無曠庶官,」《詩》有「職思其憂」,臣雖今非所司,然昔忝斯任,故不以草茅自疏,敢言及於此。語曰:「患為之者不必知,知之者不得為。」臣誠不知,強欲為之耳。竊尋先朝賜臣名彪者,遠則擬《漢史》之叔皮,近則準《晉史》之紹統,推名求義,欲罷
 不能。今求都下乞一靜處,綜理國籍,以終前志。官給事力,以充所須。雖不能光啟大錄,庶不為飽食終日耳。近則期月可就,遠則三年有成,正本蘊之麟閣,副貳藏之名山。



 時司空北海王詳、尚書令王肅許之。肅以其無祿,頗相賑餉。遂在祕書省,同王隱故事,白衣修史。



 宣武親政,崔光表曰:「臣昔為彪所致,與之同業積年,其志力貞強,考述無倦。頃來契闊,多所廢離,近蒙收起,還綜厥事。老而彌厲,史才日新。若克復舊職,專功不殆,必能昭明《春秋》,闡成皇籍。既先帝厚委,宿歷高班,纖負微愆,應從滌洗。愚謂宜申以常伯,正綰著作。」宣武不許。詔彪兼通
 直散騎常侍、行汾州事,非彪好也,固請不行。卒於洛陽。



 始彪為中尉,號為嚴酷。以姦款難得,乃為木手擊其脅腋,氣絕而復屬者時有焉。又慰喻汾州叛胡,得其兇渠,皆鞭面殺之。及彪病,體上往往瘡潰,痛毒備極。



 贈汾州刺史,謚曰剛憲。彪在秘書歲餘,史業竟未及就,然區分書體,皆彪之功。



 述《春秋三傳》,合成十卷。其餘著詩頌賦誄章表別有集。



 彪雖與宋弁結管、鮑交,弁為大中正,與孝文私議,猶以寒地處之,殊不欲微相優假。彪亦知之,不以為恨。弁卒,彪痛之無已,為之哀誄,備盡辛酸。郭祚為吏部,彪為子志求官,祚乃以舊第處之。彪以位經常
 伯,又兼尚書,謂祚應以貴游拔之,深用忿怨,形於言色。時論以此非祚。祚每曰:「爾與義和至友,豈能饒爾而怨我乎。」任城王澄與彪先亦不穆,及為雍州,彪詣澄,為志求其府寮。澄釋然為啟,得為列曹行參軍,時稱澄之美。



 志字鴻道,博學有才幹,年十餘,便能屬文。彪奇之,謂崔鴻曰:「子宜與鴻道為二鴻於洛陽。」鴻遂與交款往來。



 彪有女,幼而聰令。彪每奇之,教之書學,讀誦經傳。嘗竊謂所親曰:「此當興我家,卿曹容得其力。」彪亡後,宣武聞其名,召為婕好。在宮常教帝妹書,誦授經史。始彪奇志及婕妤,特加器愛。公私坐集,必自稱詠,由是為孝文所貴。
 及彪亡後,婕妤果入掖廷,後宮咸師宗之。宣武崩後,為比丘尼,通習經義,法座講說,諸僧歎重之。



 志歷官所在著績。桓叔興外叛,南荊荒毀,領軍元叉舉其才任撫導,抉為南荊州刺史。建義初,叛入梁。



 志弟游,有才行。隨兄志在南荊州,屬爾朱之亂,與志俱奔江左。子昶。



 昶小名那。性峻急,不雜交游。幼年已解屬文,有聲洛下。時洛陽初置明堂,昶年十數歲,為《明堂賦》,雖優洽未足,才制可觀。見者咸曰有家風也。初謁周文,周文深奇之,厚加資給,令入太學。周文每見學生,必問才行於昶。昶神情清悟,應對明辯,周文每稱歎之。綏德公陸通盛選
 僚採,請以昶為司馬,周文許之。



 昶雖年少,通特加接待,公私之事,咸取決焉。又兼二千石郎中,典儀注。累遷都官郎中、相州大中正。昶雖處郎官,周文恆欲以書記委之。於是以為丞相府記室參軍、著作郎、修國史,轉大行臺郎中、中書侍郎,又轉黃門侍郎,對臨黃縣伯。嘗謂曰:「卿祖昔在中朝,為御史中尉;卿操尚貞固,理應不墜家風。但孤以中尉彈劾之官,愛憎所在,故未即授卿耳。然此職久曠,無以易卿。」乃奏昶為御史中尉,賜姓宇文氏。



 六官建,拜內史下大夫,進爵為侯。明帝初,行御伯中大夫。武成元年,除中外府司錄。保定初,進驃騎大將軍、開
 府儀同三司,轉御正中大夫。時以近侍清要,盛選國華,乃以昶及安昌公元則、中都公陸逞、臨淄公唐瑾等並為納言。尋進爵為公。五年,出為昌州刺史。在州遇疾,求入朝,詔許之。未至京,卒,贈相、瀛二州刺史。



 昶,周文世已當樞要。兵馬處分,專以委之;詔冊文筆,皆昶所作也。及晉公護執政,委任如舊。昶常曰:「文章之事,不足流於後世,經邦致化,庶及古人。」



 故所作文筆,了無槁草,唯留心政事而已。又以父在江南,身寓關右,自少及終,不飲酒聽樂。時論以此稱焉。子丹嗣。



 高道悅,字文欣,遼東新昌人也。曾祖策,馮跋散騎常侍、
 新昌侯。祖育,馮弘建德令。太武東討,率部歸命,授建忠將軍、齊郡建德二郡太守,賜爵肥如子。



 父玄起,武邑太守,遂居勃海蓨縣。



 道悅少為中書學生、侍御主文中散。後為諫議大夫,正色當官,不憚強禦。車駕南征,徵兵秦、雍,大期秋季閱集洛陽。道悅以使者書侍御史薛聰、侍御史主文中散元志等稽違期會,奏舉其罪。又奏兼左僕射、吏部尚書、任城王澄,位總朝右,任屬戎機,兵使會否,曾不檢奏。尚書左丞公孫良職綰樞轄,蒙冒莫舉。請以見事免澄、良等所居官。時道悅兄觀為外兵郎中,澄奏道悅有黨兄之負,孝文詔責。然以事經恩宥,遂寢而
 不論。詔曰:「道悅資性忠篤,稟操貞亮。居法樹平肅之規,處諫著必犯之節。王公憚其風鯁,朕實嘉其一至,謇諤之誠,何愧黯、鮑也。其以為主爵下大夫,諫議如故。」



 車駕幸鄴,又兼御史中尉,留守洛京。時宮闕初基,廟庫未構,車駕將水路幸鄴。已詔都水回營構之材,以造舟楫。道悅表諫,以為闕居宇之功,作游嬉之用,損耗殊倍。又深薄之危,古今共慎。於是帝遂從陸路。轉道悅太子中庶子,正色立朝,嚴然難犯,宮官上下,咸畏憚之。



 太和二十年秋,車駕幸中岳,詔太子恂入居金墉。而恂潛謀還代,忿道悅前後規諫,遂於禁中殺之。帝甚加悲惜,贈散騎
 常侍、營州刺史,并遣王人慰其妻子,又詔使者監護喪事。葬於舊塋,謚曰貞侯。宣武又追錄忠概,拜長子顯族給事中。



 顯族亦以忠厚見稱,卒於右軍將軍。



 顯族弟敬猷,有風度。蕭寶夤西征,引為驃騎司馬。及寶夤謀逆,敬猷與行臺郎中封偉伯等潛圖義舉,謀洩見殺。贈滄州刺史,聽一子出身。道悅長兄嵩,字昆侖,魏郡太守。



 嵩弟雙,清河太守。坐黷貨,將刑於市,遇赦免。時北海王詳為錄尚書事,雙多納金寶,除司空長史。後為涼州刺史,專肆貪暴,以罪免。後貨高肇,復起為幽州刺史。以貪穢被劾,罪未判,遇赦復任。未幾而卒。



 雙弟觀,尚書左外兵郎
 中、城陽王鸞司馬。南征赭陽,先驅而歿,謚曰閔。



 甄琛,字思伯,中山毋極人,漢太保邯之後也。父凝,州主簿。琛少敏悟。閨門之內,兄弟戲狎,不以禮法自居。學覽經史,稱有刀筆。而形貌短陋,鮮風儀。



 舉秀才,入都積歲,頗以奕棋棄日,至乃通夜不止。手下倉頭,常令執燭,或時睡頓,大加其杖,如此非一。奴後不勝楚痛,乃曰:「郎君辭父母仕宦,若為讀書執燭,不敢辭罪,乃以圍棋,日夜不息,豈是向京之意?而賜加杖罰,不亦非理!」



 琛悵然慚感。遂從許赤彪假書研習,聞見日優。太和初,拜中書博士,遷諫議大夫,時有所陳,亦為孝文知賞。宣武踐阼,以
 琛為中散大夫,兼御史中尉。琛表曰:《月令》稱山林藪澤,有能取蔬食禽獸者,皆野虞教導之。其迭相侵奪者,罪之無赦。此明導人而弗禁,通有無以相濟也。《周禮》雖有川澤之禁,正所以防其殘盡,必令取之有時。斯所謂鄣護在公,更所以為人守之耳。今者天為黔首生鹽,國為黔首鄣護。假獲其利,猶是富專口齗,不及四體也。且天下夫婦,歲貢粟帛,四海之有,備奉一人;軍國之資,取給百姓,天子亦何患乎貧,而茍禁一池?臣每觀上古愛人之迹,時讀中葉驟稅之書,未嘗不歎彼遠大,惜此近狹。今偽弊相承,仍崇關廛之稅。大魏宏博,唯受穀帛之輸。
 是使遠方聞者,莫不歌德。語稱出內之吝,有司之福;施惠之難,人君之禍。夫以府藏之物,猶以不施而為災,況府外之利,而可吝之於黔首?願弛鹽禁,使沛然遠及。依《周體》置川衡之法,使之監導而已。



 詔付八坐議可否以聞。彭城王勰、兼尚書邢巒等奏:琛之所列,但恐坐談則理高,行之則事闕,是用遲回,未謂為可。竊惟大道既往,恩惠生焉,下奉上施,卑高理睦。恆恐財不賙國,澤不厚人,故多方以達其情,立法以行其志。至乃取貨山澤,輕在人之貢;立稅關市,裨十一之儲。收此與彼,非利己也;回彼就此,非為身也。所謂集天地之產,惠天地之人,藉
 造物之富,賑造物之貧。禁此泉池,不專太官之御;僉此匹帛,豈為後宮之資。既潤不在己,彼我理一,積而散之,將焉所吝。然自行以來,典司多怠,出入之間,事不如法。



 此乃用之者無方,非興之者有謬。至使朝廷識者,聽營其間。今而罷之,懼失前旨。



 宜依前式。



 詔曰:「司鹽之稅,乃自古通典,然興制利人,亦世或不同。甄琛之表,實所謂助政毗俗者也。可從其前計,尚書嚴為禁豪強之制也。」



 詔琛參八坐議事,尋正中尉。遷侍中,領中尉。琛俛眉畏避,不能繩糾貴游,凡所劾者,率多下吏。於時趙脩寵貴,琛傾身事之。琛父凝為中散大夫,弟僧林為本州別駕,
 皆託脩申達。至脩姦詐事露,明當收考,今日乃舉其罪。及監決脩鞭,猶相隱惻,然告人曰:「趙脩小人,背如土牛,殊耐鞭杖。」有識以此非之。脩死之明日,琛與黃門郎李憑以朋黨被召詣尚書。兼尚書元英、邢巒窮其阿附之狀。琛曾拜官,諸賓悉集,巒乃晚至。琛謂巒:「何處放蛆來,今晚始顧?」雖以言戲,巒變色銜忿。及此,大相推窮。司徒、錄尚書事、北海王詳等奏曰:謹案侍中、領御史中尉甄琛,身居直法,糾擿是司。風邪響黷,猶宜劾糾,況趙脩侵公害私,朝野切齒?而琛嘗不陳奏,方更往來,中外影響,致其談譽。令布衣之父,超登正四之官;七品之弟,越陟
 三階之祿。虧先皇之選典,塵聖明之官人。



 又與黃門郎李憑,相為表裏。憑兄叨封,知而不言。及脩釁彰,方加彈奏。生則附其形勢,死則就地排之。竊天之功,以為己力,仰欺朝廷,俯罔百司。其為鄙詐,於茲甚矣。謹依律科從,請以職除。其父中散,實為叨越,雖皇族帝孫,未有此例。



 既得不以倫,請下收奪。李憑朋附趙脩,是親是仗,緇點皇風,塵鄙正化,此而不糾,將何以肅整阿諛,獎厲忠概?請免所居官以肅風軌。



 奏可。琛遂免歸本郡。左右相連死黜者二十餘人。



 始琛以父母老,常求解官扶侍,故孝文授以本州長史。及貴達,不復請歸,至是乃還。供養數
 年,遭母憂。母鉅鹿曹氏,有孝性。夫氏去家,路踰百里,每得魚肉菜果珍美口實者,必令僮僕走奉其母,乃後食焉。琛母服未闋,復喪父。琛於塋兆內手種松柏,隆冬負掘水土。鄉老哀之,咸助加力。十餘年中,墳成木茂。與弟僧林誓以同居沒齒,專事產業,躬親農圃,時以鷹犬馳逐自娛。朝廷有大事,猶上表陳情。



 久之,復除散騎常侍,領給事黃門侍郎、定州大中正,大見親寵。委以門下庶事,出參尚書,入廁帷幄。孝文時,琛兼主客郎,迎送齊使彭城劉纘。琛欽其器貌,常歎詠之。纘子昕為朐山戍主。昕死,家屬入洛。有女年未二十,琛乃納昕女為妻。



 婚日,
 詔給廚費。琛所好悅,宣武時調戲之。遷河南尹,黃門、中正如故。琛表曰:國家居代,患多盜竊。世祖太武皇帝親自發憤,廣置主司,里宰皆以下代令長及五等散男有經略者乃得為之。又多置吏士,為其羽翼。崇而重之,始得禁止。今遷都已來,天下轉廣;四遠赴會,事過代都。寇盜公行,劫害不絕。此由諸坊混雜,釐比不精,主司闇弱,不堪檢察故也。今擇尹既非南金,里尉鉛刀而割,欲望清肅都邑,不可得也。里正乃流外四品,職輕任碎,多是下才。人懷茍且,不能督察,故使盜得容姦,百賦失理。邊外小縣,所領不過百戶,而令長皆以將軍居之。京邑諸
 坊,大者或千戶、五百戶,其中皆王公卿尹,貴勢姻戚,豪猾僕隸,蔭養姦徒,高門邃宇,不可干問。比之邊縣,難易不同。今難彼易此,實為未愜。



 王者立法,隨時從宜;先朝立品,不必即定。施而觀之,不便則改。今閑官靜任,猶聽長兼,況煩劇要務,不得簡能下領。請取武官中八品將軍以下乾用貞濟者,以本官俸恤領里尉之任,各食其祿。高者領六部尉,中者領經途尉,下者領里正。



 不爾,請少高里尉之品,選下品中應遷者,進而為之。則督責有所,輦轂可清。



 詔曰:「里正可進至勳品、經途從九品、六部尉正九品諸職中簡取,何必須武人也。」琛又奏以羽林
 為游軍,於諸坊巷司察盜賊。於是京邑清靜,後皆踵焉。



 轉太子少保,黃門如故。及高肇死,琛以黨不宜復參朝政,出為營州刺史,遷涼州刺史。猶以高氏之暱,不欲處之於內。久之,為吏部尚書。未幾,除定州刺史。



 固辭曰:「陛下在東宮,崔光為少傅,臣為少保,今光為車騎大將軍、儀同三司、開國公。故僕射游肇時為侍中,與臣官階相似;肇在省為僕射,死贈車騎將軍、儀同三司、冀州刺史。臣今適為征北將軍、定州刺史。生師保不如死游肇。」詔書慰遣之。琛既至鄉,衣錦晝游,大為稱滿;政體嚴細,甚無聲譽。



 崔光辭司徒之授也,琛與光書,外相抑揚,內實
 附會。光亦揣其意,復書以悅之。徵為車騎將軍、特進,又拜侍中。以其衰老,詔賜御府杖,朝直杖以出入。卒,詔給東園祕器,贈司徒公、尚書左僕射,加後部鼓吹。太常議謚文穆,吏部郎袁翻奏曰:案禮,謚者行之迹也;號者功之表也;車服者位之章也。是以大行受大名,細行受細名。行生於己,名生於人。故闔棺然後定謚,皆累其生時美惡,所以為將來勸戒;身雖死,使名常存也。凡薨亡者,屬所即言大鴻臚,移本郡大中正。條其行迹功過,承中正移,言公府,下太常部博士評議,為謚列上。謚不應法者,博士坐如選舉不以實論。若行狀失實,中正坐如博
 士。自古帝王,莫不殷勤重慎,以為褒貶之實也。今之行狀,皆出自其家,任其臣子自言君父之行,無復是非之事。臣子之欲光揚君父,但苦迹之不高,行之不美,是以極辭肆意,無復限量。觀其狀也,則周、孔聯鏕,伊顏接衽。論其謚也,雖窮文盡武,無或加焉。然今之博士與古不同,唯知依其行狀,又先問其家人之意;臣子所求,便為議上。都不復斟酌與奪,商量是非。致號謚之加,與泛階莫異;專以極美為稱,無復貶降之名。禮官之失,一至於此。案甄司徒行狀,至德與聖人齊蹤,鴻名共大賢比跡,文穆之謚,何足加焉。但比來贈謚,於例普重,如甄之流,
 無不復謚。謂宜依謚法,慈惠愛人曰孝,宜謚曰孝穆公。



 自今以後,明勒太常、司徒,有行狀如此,言辭流宕,無復節限者,悉請裁量,不聽為受。仍踵前來之失者,皆付法司科罪。



 詔從之。琛祖載,明帝親送,降車就輿,弔服哭之,遣舍人慰其諸子。



 琛性輕簡,好嘲謔,故少風望。然明解有幹具,在官清白。自孝文、宣武,咸相知待。明帝以師傅之義而加禮焉。所著文章,鄙碎無大體,時有理詣。《磔四聲》、《姓族廢興》、《會通緇素三論》及《家晦》二十篇,《篤學文》一卷,頗行於世。



 琛長子侃,字道正,位秘書郎。性險薄,多與盜劫交通。隨琛在京,以酒色夜宿洛水亭舍,毆擊主人。
 為司州所劾,淹在州獄。琛大以慚慨。廣平王懷為牧,與琛先不協,欲具案窮推。琛託左右以聞,宣武敕懷寬放。懷固執之,久乃特旨出侃。



 自此沈廢,卒家。



 侃弟楷,字德方。粗有文學,頗更吏事。琛啟除秘書郎。宣武崩,未葬,楷與河南尹丞張普惠等飲戲,免官。後稍遷尚書儀曹郎。有當官之稱。明帝末,丁憂在鄉,定州刺史廣陽王深召楷兼長史,委以州任。尋屬鮮于脩禮、毛普賢等率北鎮流人反於州西北之左人城,屠村掠野,引向州城。州城內先有燕、恆、雲三州避難戶,脩禮等聲云,欲將此輩共為舉動。楷見人情不安,慮有變起,乃走收三州人中粗
 暴者殺之,以威外賊。及刺史元冏、大都督揚津等至,楷乃還家。後脩禮等忿楷屠害北人,遂掘其父墓,載棺巡城,示相報復。孝莊時,徵為中書侍郎。後齊文襄取為儀同府諮議參軍。卒,贈驃騎將軍、秘書監、滄州刺史。



 琛從父弟密,字叔雍。清謹少嗜慾,頗涉書史。疾世俗貪競,乾沒榮寵,曾為《風賦》以見意。後參中山王英軍事。英鐘離敗退,鄉人蘇良沒於賊中,密盡私財以贖之。良歸,傾資報密。密一皆不受,曰:「濟君之日,本不求貨,豈相贖之意。」



 及葛榮侵擾河北,詔密為相州行臺,援守鄴城。莊帝以密全鄴勳,賞安市縣子。孝靜初,為衛尉卿,在官有平直
 之譽。出為北徐州刺史,卒官。贈驃騎將軍、儀同三司、瀛州刺史,謚曰靖。



 琛同郡張纂,字伯業。祖珍,字文表,慕容寶度支尚書。道武平中山,入魏,卒於涼州刺史,謚曰穆。纂頗涉經史,雅有氣尚,交結勝流。為樂陵太守,在郡多所受納。聞御史至,棄郡逃走,於是除名,乃卒。天平初,贈定州刺史。纂叔感,字崇仁,有器業,不應州郡之命。



 子宣軌,少孤,事母以孝聞。累遷相州撫軍府司馬。宣軌性通率,輕財好施。



 屬葛榮圍城,與刺史李神有固守效,以功賜爵中山公。後坐事死鄴。纂從弟元賓,位奉朝請。及外生高昂貴達,啟贈瀛州刺史。



 高聰,字僧智,本勃海人也。曾祖軌,隨慕容德徙青州,因居北海之劇縣。父法昂,少隨其車騎將軍王玄謨征伐,以功至員外郎,早卒。聰生而喪母,祖母王撫育之。大軍攻剋東陽,聰徙平城,與蔣少游為雲中兵戶,窘困無所不為。族祖允視之若孫,大加賙給。聰涉獵經史,頗有文才。允嘉之,數稱其美,言之朝廷,由是與少游同拜中書博士。轉侍郎,為高陽王雍傅,稍為孝文知賞。太和十七年,兼員外散騎常侍,使於齊。後兼太子左率。



 聰微習弓馬,乃以將略自許。孝文銳意南討,專訪王肅以軍事。聰託肅,願以偏裨自效。肅言於帝,故假聰輔國將
 軍,受肅節度,同援渦陽。聰躁怯少威重,及與賊交,望風退敗。孝文恕死,徙平州。行屆瀛州,刺史王質獲白兔,將獻,託聰為表。帝見表,顧王肅曰:「在下那得有此才,令朕不知。」肅曰:「比高聰北徙,或其所製。」帝悟曰:「必應然也。」



 宣武初,聰復竊還京師,說高肇廢六輔。宣武親政,除給事黃門侍郎,後加散騎常侍。及幸鄴還,於河內懷界,帝射矢一里五十餘步。侍中高顯等奏,盛事奇迹必宜表述,請勒銘射宮,永彰聖藝。遂刊銘射所,聰為之詞。趙脩嬖境,聰深朋附。



 及詔追贈脩父,聰為碑文,出入同載,觀視碑石。聰每見脩,迎送盡禮。聰又為脩作表,陳當時便宜,
 教其自安之術,由是迭相親狎。脩死,甄琛、李憑皆被黜落,聰深用危慮,而先以疏宗之情,曲事高肇,竟獲自免,肇之力也。脩之任勢,聰傾身事之;及死,言必毀惡。茹皓之寵,聰又媚附,每相招命,稱皓才識非脩之儔。



 乃因皓啟請田宅,皆被遂許。及皓見罪戮,聰以為死之晚也。其薄於情義皆如此。



 侍中高顯為護軍,聰代兼其任。顯與兄肇疑聰間構而求之。聰居兼十餘旬,出入機要,言即真,無遠慮,藉貴因權,耽於聲色,賄納之音,聞於遐邇。中尉崔亮知肇微恨,遂面陳聰罪,出為並州刺史。聰善於去就,知肇嫌之,側身承奉,肇遂待之如舊。聰在并州數
 歲,多不率法,又與太原太守王椿有隙,再為大使御史舉奏。



 肇每以宗私相援,事得寢緩。宣武末,拜散騎常侍、平北將軍。



 明帝踐阼,以其素附高肇,出為幽州刺史。尋以高肇之黨,與王世義、高綽、李憲、崔楷、蘭氛之為中尉元匡所彈,靈太后並特原之。聰遂廢于家,斷絕人事,唯脩營園果,世稱高聰梨,以為珍異。又唯以聲色自娛。後拜光祿大夫,卒。靈太后聞其亡,嗟惋良久。贈青州刺史,謚曰獻。



 聰有妓十餘人,有子無子皆注籍為妾,以悅其情。及病,欲不適他人,並令燒指吞炭,出家為尼。聰所作文筆二十卷。長子雲,字彥鴻,位輔國將軍、中散大夫。



 河
 陰遇害,贈兗州刺史。



 論曰:韓麒麟由才器識用,遂見紀於齊士。顯宗以文學自立,而時務屢陳;至於實錄之功,未之聞也。子熙清尚自守,榮過其器。程駿才業見知,蓋當時之長策。



 李彪生自微族,見擢明世,輶軒驟指,聲駭江南,執筆立言,遂為良史。逮於直繩在手,厲氣明目,持堅無術,末路蹉跎。行百里者半於九十,彪之謂也。高道悅謇直之風,見憚於世,醜正貽禍,有可悲乎!甄琛以學尚刀筆,早樹聲名;受遇三朝,終至崇重。高聰才尚見知,名位顯著。而異軌同奔,咸經於危覆之轍,惜乎!



\end{pinyinscope}