\article{卷九十二列傳第八十 恩幸}

\begin{pinyinscope}

 王睿王
 仲興寇猛趙脩茹皓趙邕侯剛徐紇宗愛仇洛齊段霸王琚趙默孫小張宗之劇鵬張祐抱嶷王遇苻承祖王質李堅秦松白整劉騰賈粲
 楊範成軌王溫孟欒平季封津劉思逸張景嵩毛暢郭秀和士開穆提婆高阿那肱韓鳳齊諸宦者夫令色巧言,矯情飾貌,邀眄睞之利,射咳唾之私,乃茍進之常道也。況乃親由褻狎,恩生趨走,便僻俯仰,當寵擅權。斯乃夏桀、殷紂所以喪兩代,石顯、張讓所以翦二京焉。



 魏世王睿幸於太和之初,鄭儼寵於孝昌之季,宗愛之弒帝害王,劉騰之廢后戮相,此蓋其甚者爾。其間盜宮賣爵,污辱宮闈者多矣,亦何可枚舉哉?斯乃王者
 所宜深誡。而齊末又有甚焉。乃自書契以降,未之有也。若乃心利錐刀,居台鼎之任;智昏菽麥,當機衡之重。亦有西域醜胡,龜茲雜伎,封王開府,接武比肩。非直獨守幸臣,且復多干朝政。賜予之費,帑藏以虛;杼柚之資,剝掠將盡。齊運短促,固其宜哉!神武、文襄,情存庶政,文武任寄,多貞幹之臣,唯郭秀小人,有累明德。天保五年之後,雖罔念作狂,所幸有通州刺史梁伯和、陸芃兒之徒,唯左右驅馳,內外褻狎,其朝廷之事,一不與聞,故不入此傳。大寧之後,姦佞浸繁,盛業鴻基,以之顛覆,生靈厄夫左衽,非不幸也!



 《魏書》有《恩幸傳》及《閹官傳》,《齊書》有《佞
 幸傳》。今用比次,以為《恩幸》篇云。舊書鄭儼在《恩幸》中,今從例附其家傳,其餘並編於此。其宦者之徒,尤是亡齊之一物,醜聲穢跡,千端萬緒,其事闕而不書,乃略存姓名,附之此傳之末。其帝家諸奴及胡人樂工叨竊貴幸者,亦附出焉。



 王睿,字洛誠,自云太原晉陽人也。六世祖橫,張軌參軍。晉亂,子孫因居於武威姑臧。父橋,字法生,解天文卜筮。涼州平,入京。家貧,以術自給,歷位終於侍御中散。天安初,卒,贈平遠將軍、涼州刺史、顯美侯,謚曰敬。睿少傳父業,而姿貌偉麗,景穆之在東宮,見而奇之。興安初,擢為
 太卜中散,稍遷為令,領太史。承明元年,文明太后臨朝,睿因緣見幸。超遷給事中。俄為散騎常侍、侍令,領太史。承明元年,文明太后臨朝,睿因緣見幸,超遷給事中。俄為散騎常侍、侍中、吏部尚書,賜爵太原公。於是內參機密,外豫政事,愛寵日隆,朝士懾憚焉。



 太和二年,孝文及文明太后率百僚與諸方客臨獸圈,有猛獸逸,登門閣道,幾至御坐。左右侍衛皆驚靡,睿獨執戟禦之,猛獸乃退。故親任轉重。三年春,詔睿與東陽王丕同入八議,永受復除。四年,遷尚書令,進爵中山王,加鎮東大將軍,置王官二十二人,中書侍郎鄭羲為傅,郎中令以下,皆當時名士。又拜睿妻丁氏為妃。



 及沙門法秀謀逆事發,多所牽引。睿曰:「與殺不辜,寧
 赦有罪,宜梟斬首惡,餘從原赦,不亦善乎!」考文從之,得免者千餘人。



 睿出入帷幄,太后密賜珍玩繒綵,人莫能知。率常以夜帷載閹官防致,前後鉅萬,不可勝數。加以田園、奴婢、牛馬雜畜,並盡良美。大臣及左右因是以受賚賜,外示不私,所費又以萬計。及疾病、孝文、太后每親視疾,侍官省問,相望於道。



 及疾篤,上疏陳刑政之宜。尋薨,孝文、文明太后親臨哀慟。賜溫明秘器,宕昌公王遇監護喪事。贈衛大將軍、太宰、并州牧,謚曰宣王。內侍長董醜奴營墳墓。將葬於城東,孝文登城樓以望之。京都文士為作哀詩及誄者百餘人。乃立睿祀於都南二十
 里大道右,起廟,以時祭薦,并立碑銘,置守祀五家。又詔褒揚睿,圖其捍猛獸狀於諸殿,令高允為之贊。京邑士女,諂稱睿美,造新聲而絃歌之,名曰《中山王》。詔班樂府,合樂奏之。



 初,睿女妻李沖兄子蕤,次女以適趙國李恢子華。女之將行,先入宮中,其禮略如公主、王女之儀。太后親御太華殿,寢其女於帳中,睿與張祐侍坐。睿所親及兩李家丈夫、婦人列於東西廊。及女子登車,太后送過中路。時人竊謂天子、太后嫁女。睿之葬也,假親姻義舊衰絰縞冠送喪者千餘人,皆舉聲慟泣,以要榮利,時謂之義孝。



 睿既貴,乃言家本太原晉陽,遂移屬焉。故其
 兄弟封爵,多以并州郡縣。薨後,重贈睿父橋侍中、征西將軍、左光祿大夫、儀同三司、武威王,謚曰定。追策睿母賈氏為妃,立碑於墓左。父子並葬城東,相去里餘。遷洛後,更徙葬太原晉陽故地。



 子襲,字元孫。睿薨,孝文詔襲代領都曹,為尚書令,領吏部曹。後襲王爵,例降為公。太后崩後,襲禮遇稍薄,不復關與時事。後出為并州刺史。輿駕詣洛,路幸其州,人庶多為立銘,置於大路,虛相稱美。或云襲所教也,尚書奏免其官,詔唯降號二等。卒,贈豫州刺史,謚曰質。



 襲弟椿,字元壽。正始中,拜太原太守,坐事免。椿僮僕千餘,園宅華廣,聲伎自適,無乏於時。或
 有權椿仕者,椿笑而不答。雅有巧思,凡所營製,可為後法。



 由是正光中元叉將營明堂、辟雍,俗徵為將作大匠,椿聞而固辭。孝昌中,爾朱榮以汾州胡逆,表椿慰勞汾胡。汾胡與椿比州,服其聲望,所至降下。事寧,授太原太守。以預立莊帝功,封遼陽縣子,尋轉封真定縣。永熙中,除瀛州刺史。時有風雹之變,詔書廣訪讜言,椿乃上疏言政事之宜。椿性嚴察,下不容姦,所在吏人畏之重足。天平末,更滿還鄉。初,椿於宅構起事,極為高壯。時人忽云:「此乃太原王宅,豈是王太原宅?」椿往為本郡,世皆呼為王太原。未幾,爾朱榮居椿之宅,榮封太原王焉。到
 於齊神武之居晉陽,霸朝所在,人士輻湊。椿禮敬親知,多所拯接。後以老病辭疾,客居趙郡之西鯉魚祠山。卒,贈尚書左僕射、太尉公、冀州刺史,謚曰文恭。及葬,齊神武親自吊送。



 椿妻巨鹿魏悅次女,明達有遠操,多識往行前言。隨夫在華州,兄子建在洛遇患,聞而馳赴,膚容虧損,親類歎尚之。爾朱榮妻鄉郡長公主深所禮敬。永安中,詔以為南和縣君。內足於財,不以華飾為意。撫兄子收,情同己子。存拯親類,所在周給。椿名位終始,魏有力焉。卒,贈巨鹿郡君。椿無子,以兄孫叔明為後。



 王仲興,趙郡南欒人也。父天德,起自細微,至殿中尚書。
 仲興幼而端謹,以父任,早給事左右,累遷越騎校尉。孝文在馬圈,自不豫、大漸迄於崩,仲興頗預侍護。宣武即位,轉左中郎將。及帝親政,與趙修並見寵任,遷光祿大夫,領武衛將軍。雖與脩並,而畏慎自退,不若脩倨傲無禮。咸陽王禧之出奔也,當時上下微為震駭,帝遣仲興先馳入金墉安慰。後與領軍于勁參機要,因自回馬圈侍疾及入金墉功,遂封上黨郡開國公。自拜武衛及受封日,車駕每臨饗其宅。宣武游幸,仲興常侍,不離左右,外事得徑以聞,百僚亦聳體而承望焉。兄可久,以仲興故,自散爵為征虜府長史,帶彭城太守。仲興世居趙郡,
 自以寒微,云舊出京兆霸城,故為雍州大中正。尚書後以仲興賞報過優,北海王詳嘗以面啟,奏請降減,事久不決。



 可久在徐州,恃仲興寵勢,輕侮司馬梁郡太守李長壽,乃令僮僕邀毆長壽,遂折其脅。州以表聞,北海王詳因百僚朝集,厲色大言曰:「徐州名籓,先帝所重,朝廷云何簡用上佐,遂至此紛紜,以徹荒外,豈不為國醜辱!」仲興是後漸疏。宣武乃下詔奪其封邑。後卒於并州刺史。



 宣武時,又有上谷寇猛,少以姿乾充武賁,稍遷至武衛將軍。出入禁中,無所拘忌。自以上谷寇氏,得補燕州大中正,而不能甄別士庶也。卒,贈燕州刺史。



 趙脩,字景業,趙郡房子人也。父謐,陽武令。脩本給事東宮,為白衣左右,頗有膂力。宣武踐阼,愛遇日隆。然天性暗塞,不親書疏。宣武親政,旬月間頻有轉授。每受除設宴,帝幸其宅,諸王公百僚悉從,帝親見其母。



 脩能劇飲,至於逼勸觴爵,雖北海王詳、廣陽王嘉等皆亦不免,必致困亂。每適郊廟,脩常驂陪,出入華林,恒乘馬至禁內。咸陽王禧誅,其家財貨多賜高肇及脩。脩之葬父,百官自王公已下,無不弔祭,酒犢祭奠之具,填塞門街,。於京師為制碑銘、石獸、石柱,皆發人車牛,傳致本縣,財用之費,悉自公家。凶吉車乘將百兩,道路供給,皆出於官。時將
 馬射,宣武留脩過之,帝如射宮,又驂乘,輅車旒竿觸東門折。脩恐不逮葬日,驛赴窆期。左右求從及特遣者數十人,脩道路嬉戲,殆無戚容,或與賓客姦掠婦女裸觀,從者噂沓喧嘩,詬詈無節,莫不畏而惡之。是年,又為脩廣增宅舍,多所并兼,洞門高堂,房廡周博,崇麗擬於諸王。其四面鄰居,賂入其地者侯天盛兄弟,越次出補長史大郡。



 脩起自賤伍,暴致富貴,奢傲無禮,物情所疾,困其在外,左右或諷糾其罪。



 自其葬父還也,舊寵小薄。初,王顯附脩,後因忿鬩,密伺其過,列脩葬父時,路中淫亂不軌。又云與長安人趙僧樹謀匿玉印事。高肇、甄琛等
 構成其罪,乃密以聞。始琛及李憑等曲事脩,無所不至,懼相連及,乃爭共糾手適。遂有詔按其罪惡,鞭之一百,徒敦煌為兵。其家宅作徒,即仰停罷,所親在內者,悉令出禁。是日,脩詣領軍于勁第,與之樗蒱。籌未及畢,羽林數人,相續而至,稱詔呼之。脩驚起,隨出。路中執引脩馬詣領軍府。琛與顯監決其罪,先具問事有力者五人,更迭鞭之,占令必死。旨決百靴,其實三百。脩素肥壯,腰腹博碩,堪忍楚毒,了不轉動。鞭訖,即召驛馬,促之令發。出城西門,不自勝舉,縛置鞍中,急驅馳之,其母妻追隨,不得與語,行八十里乃死。



 初,於后之入,脩之力也。脩死後,領
 軍于勁猶追感舊意,經恤其家。自餘朝士昔相宗承者,悉棄絕之,以示己之疏遠焉。



 茹皓,字禽奇,舊吳人也。父謙之,本名要,隨宋巴陵王休若為將,至彭城,遂寓居淮陽上黨。皓年十五六,為縣金曹吏。南徐州刺史沈陵見而善之,自隨入洛,舉充孝文白衣左右。宣武踐阼,皓侍直禁中,稍被寵接。宣武嘗拜山陵,路中欲引與同車,黃門侍郎元匡切諫乃止。乃帝親政,皓眷賚日隆。時趙脩亦被幸,妒之,求出皓。皓亦慮見危禍,不樂內官,遂超授濮陽太守,其父因皓,訟理舊勳,先除兗州陽平太守,賜以子爵。父子剖符名邦,郡境相
 接。皓忻於去內,不以疏外為戚。



 及趙脩等敗,竟獲全。雖起微細,為守乃清簡寡事。後授左中郎將,領直閣,寵待如前。皓既宦達,自云本出雁門,雁門人諂附者,乃因薦皓於司徒,請為肆州大中正,詔特依許。遷驍騎將軍,領華林諸作。皓性微工巧,多所興立,為山於天泉池西,採掘北芒及南山佳石,徙竹汝、潁,羅蒔其間。經構樓觀,列於上下,樹草栽木,頗有野致。帝心悅之,以時臨幸。



 皓貴寵日昇,關豫政事,太傅、北海王詳以下,咸祗憚之。皓娶僕射高肇從妹,於帝為從母,迎納之日,詳親詣之,禮以馬物,皓又為弟聘安豐王延明妹,延明恥非舊流,不許。
 詳勸之云:「欲覓官職,如何不與茹皓婚姻也?」延明乃從焉。皓頗敏慧,折節下人,潛自經營,陰有納受,貨產盈積,起宅宮西,朝貴弗及。時帝雖親萬務,皓率常居內,留宿不還,傳可門下奏事。未幾,轉光祿少卿。意殊不已,方欲陳馬圈從先帝勞,更希榮舉。



 初,脩、皓之寵,北海王詳皆附之。又直閣劉胄本為詳薦,常感恩。高肇素嫉諸王,常規陷害,既知詳與皓等交關相暱,乃構之,云皓等將有異謀。宣武乃召中尉崔亮,令奏皓、胄、常季賢、陳掃靜四人擅勢納賄及私亂諸事。即日執皓等,皆詣南臺,翌日,奏處殺之。皓妻被髮出堂,哭而迎皓。皓徑入哭別,食椒
 而死。



 胄字元孫,後位直閣將軍。



 季賢起於主馬,宣武初好騎乘,因是獲寵。位司藥丞,仍主廄閑。



 掃靜、徐義恭,並彭城舊營人。掃靜能為宣武典櫛梳,義恭善執衣服,並以巧便,旦夕居中,愛幸相侔,官敘不異。二人皆承皓,皓亦接眷。而掃靜偏為親密,與皓常在左右,略不歸休。皓敗,掃靜亦死於家。義恭小心謹慎,皓等死後,彌見幸信。宣武不豫,義恭晝夜扶抱,崩於懷中。義恭諂附元叉,叉有淫宴,多在其宅。



 位終左光祿大夫。



 趙邕,字令和,自云南陽人也。潔白美髭眉。司空李沖之貴寵也。邕以少年端謹,出入其家,頗給桉磨奔走之役。
 沖令與諸子游處,人有束帶謁沖者,時託之以自通。太和中,給事左右,至殿中監。宣武即位及親政,猶居本任。微與趙脩結為宗援,然亦不甚相附也。邕父怡,以邕寵,召拜太常少卿,尋為荊州大中正,出為荊州刺史。怡乃致其母喪,葬於宛城之南,趙氏舊墟。後拜金紫光祿大夫,卒,贈相州刺史。宣武每出入郊廟,脩恒以常侍兼侍中陪乘,而邕兼奉車都尉,執轡同載。



 時人竊論,號為二趙。以趙出南陽,徙屬荊州。邕轉給事中,南陽中正。以父為荊州大中正,罷。宣武崩,邕兼給事黃門。後為幽州刺史,貪與范陽盧氏為婚,女父早亡,其叔許之,而母不從。
 母北平陽氏,攜女至家藏避,規免。邕乃考掠陽叔,遂至於死。陽氏訴冤,邕坐處死。會赦,免。孝昌初,卒。



 侯剛,字乾之,河南洛陽人也。其先代人,本出寒微。少以善於鼎俎,得進膳出入,積官至嘗食典御。宣武以其質直,賜名剛焉。稍遷左中郎將,領刀劍左右,後領太子中庶子。宣武崩,剛與侍中崔光迎明帝於東宮,尋除衛尉卿,封武陽縣侯。



 俄為侍中、撫軍將軍、恒州大中正,進爵為公。熙平中,侍中游肇出為相州,剛言於靈太后曰:「昔高氏擅權,游肇抗衡不屈,而出牧一籓,未盡其美。宜還引入,以輔聖主。」太后善之。



 剛寵任既隆,江陽王繼、尚書
 長孫承業皆以女妻其子。司空、任城王澄以其起由膳宰,頗竊侮之云:「此近為我舉食。」然公坐對集,敬遇不虧。後剛坐掠殺試射羽林,為御史中尉元匡所彈,處剛大辟。尚書令、任城王澄為之言於靈太后,令削封三百戶,解嘗食典御。剛於是頗為失意。剛自太和進食,遂為典御,歷兩都、三帝、二太后,將三十年,至此始解。御史中尉元匡之廢也,剛為太傅、清河王懌所舉,除車騎將軍,領御史中尉。及領軍元叉執政,剛長子,叉之妹夫,乃引剛為侍中、左衛將軍,還領嘗食典御,以為枝援。復令御史中尉。剛啟軍旅稍興,國用不足,求以己邑俸粟,賑給征
 人,比至軍下。明帝許之。



 孝昌元年,除領軍。初,元叉之解領軍,靈太后以叉腹心尚多,恐難卒制,故權以剛代之,示安其意。尋出為冀州刺史。剛在道,詔暴其朋黨元叉,逼脅內外,降為征虜將軍,餘悉削黜。終於家。永安中,贈司徒公。剛以上谷先有侯氏,於是始家焉。



 徐紇字,武伯,樂安博昌人也。家世寒微。紇少好學,頗以文詞見稱。宣武初,自主書除中書舍人。諂附趙脩,脩誅,坐徙枹罕。雖在徒役,志氣不撓。故事,捉逃役流兵五人者,聽免,紇以此得還。久之,復除中書舍人。太傅、清河王懌以文翰待之。及元叉害懌,出為鴈門太守,稱母老解
 郡。尋飾貌事叉,大得叉意。



 靈太后反政,以紇曾為懌所顧待,復自母憂中起為中書舍人。曲事鄭儼,是以特被信任,俄遷給事黃門侍郎,仍領舍人,總攝中書、門下事,軍國詔命,莫不由之。時有急速,令數吏執筆,或行或臥,人別占之,造次俱成,不失事理,雖無雅才,咸得濟用。時黃門侍郎太原王遵業、瑯邪王誦,並稱文學,亦不免為紇執筆,承其指授。紇機辯有智數,當公斷決,終日不以為勞。長直禁中,略無休息。時復與沙門講論,或分宵達曙,而心力無怠,道俗歎服之。然性浮動,慕權利,外似謇正,內實諂諛。時豪勝己,必相陵駕;書生貧士,矯意禮之。
 其詭態若此,有識鄙焉。紇既處腹心,參斷機密,勢傾一時,遠近填湊。與鄭儼、李神軌寵任相亞,時稱徐、鄭焉。然無經國大體,好行小數,說靈太后以鐵券間爾硃榮左右。榮知,深以為憾,啟求誅之。榮將入洛,既剋河梁,紇矯詔夜開殿中,取驊騮御馬十餘疋,東走兗州。羊侃時為太山太守,紇往投之,說侃令舉兵。侃從之,遂聚兵反,共紇圍兗州。孝莊初,遣侍中于暉為行臺,與齊神武討之。紇慮不免,說侃請乞師於梁,侃信之,遂奔梁。文筆駮論十卷,多有遺落,時或存於世焉。



 宗愛不知其所由來,以罪為閹人,歷碎職至中常侍。正
 平元年元正,太武大會於江上,班賞群臣,以愛為秦郡公。景穆之監國也,每事精察,愛天性險暴,行多非法,景穆每銜之。給事中侯道盛、侍郎任平城等任事東宮,微為權勢,太武頗聞之。二人與愛並不睦,愛懼道盛等案其事,遂構告其罪,詔斬道盛等於都街。時太武震怒,景穆遂以憂薨。



 是後,太武追悼不已,愛懼誅,遂謀逆。二年春,太武暴崩,愛所為也。尚書左僕射蘭延、侍中吳興公和疋、侍中太原公薛提等秘不發喪。延、疋二人議,以文成沖幼、欲立長君,徵秦王翰,置之秘室。提以文成有世嫡之重,不可廢所宜立而更求君。延等猶豫未決。愛知
 其謀。始愛負罪於東宮,而與吳王餘素協,乃密迎餘,自中宮便門入,矯皇后令徵延等。延等以愛素賤,弗之疑,皆隨之入。愛先使閹豎三十人持仗於宮內,及延等入,以次收縛,斬於殿堂。執秦王翰,殺之於永巷,而立餘。餘以愛為大司馬、大將軍、太師、都督中外諸軍事,領中秘書,封馮翊王。



 分既立餘,位居元輔,錄三省,兼總戎禁,坐召公卿,權恣日甚,內外憚之。



 群情咸以為愛必有趙高、閻樂之禍,餘疑之,遂謀奪其權。愛憤怒,使小黃門賈周等夜殺餘。文成立,誅愛、周等,皆具五刑,夷三族。



 仇洛齊,中山人也,本姓侯氏。外祖父仇款,始出馮翊重
 泉,款仕石季龍末,徙鄴南枋頭。仕慕容為烏丸護軍、長水校尉。生二子,長曰嵩,小曰騰。嵩仕慕容垂,遷居中山,位殿中侍御史。嵩有二子,長曰廣,小曰盆。嵩妹子洛齊,生而非男,嵩養為子,因為仇姓。初,嵩長女有姿色,充冉閔婦。閔破,入慕容俊,又轉賜盧豚,生子魯元。魯元有寵於太武,而知外祖嵩已死,唯有三舅,每言於帝。



 帝為訪其舅。時東方罕有仕者,廣、盆皆不樂入平城。洛齊獨請行曰:「我養子,兼人道不全,當為兄弟試禍福也。」乃乘驢赴京。魯元候知將至,結從者百餘騎,迎於桑乾河,見而下拜,從者亦同致敬。入言於太武。太武問其才用所
 宜,魯元曰:「臣舅不幸,生為閹人,唯合與陛下守宮闈耳。」而不言其養子。帝矜焉,引見敘用,賜爵文安子,稍遷給事黃門侍郎。



 魏初,禁網疏闊,人戶隱匿,漏脫者多。東州既平,綾羅戶人樂葵,因是請採漏戶,供為綸綿,自後逃戶占為紬綾羅縠者非一。於是雜營戶帥遍於天下,不屬守宰,發賦輕易,人多私附,戶口錯亂,不可撿括。洛齊奏議罷之,一屬郡縣。從征平涼,以功超遷散騎常侍。又加中書令,進爵零陵公,拜侍中、冀州刺史、內都大官。卒,謚曰康。養子儼,襲爵。



 太武時,又有段霸,以謹敏見知。歷中常侍、殿中尚書、定州刺史。



 王琚,高平人也。自云本太原人,高祖始,晉豫州刺史。琚以秦常中被刑,入宮禁。小心守節,久乃見敘用,稍遷禮部尚書,賜爵廣平公。孝文以琚歷奉前朝,志存公正,授散騎常侍。後歷位冀州刺史,假廣平王,進爵高平王。孝文、文明太后東巡冀州,親幸其家。還京,以其年老,拜散騎常侍,養老於家,前後賜以車馬、衣物,不可稱計。又降爵為公。扶老自平城從遷洛邑。常飲牛乳,色如處子。卒年九十,贈冀州刺史,謚靖公。



 趙默,字文靜,初名海,本涼州隸戶。自云,其先河內溫人也,五世祖術,晉末為西夷校尉,因居酒泉安彌縣。海生
 而涼州平,沒入而為閹人,因改名默。有容貌,恭謹小心,賜爵睢陽侯,累遷選部尚書。能自謹勵,當官任舉,頗得其人,加侍中,進爵河內公。獻文將傳位京兆王子推,訪諸群臣,百官唯唯,莫敢先言,唯源賀等辭義正直,不肯奉詔。獻文怒,變色,復以問默。默對曰:「臣以死奉戴皇太子。」獻文默然良久,遂傳位孝文。孝文立,得幸兩宮,祿賜優厚。時尚書李亦有寵於獻文,與默對綰選部。奏中書侍郎崔鑒為東徐州,北部主書郎公孫處顯為荊州,選部監公孫蘧為幽州,皆曰有能,實有私焉。默疾其虧亂選體,遂爭於殿庭曰:「以功授官,因爵與祿,國之常
 典。中書侍郎、尚書主書郎、諸曹監,勳能俱立,不過列郡。今皆以為州,臣實為惑。」於是默與遂為深隙。竟列默為監藏。因黜為門士。默廢寢忘食,規報前怨。踰年,還入為侍御、散騎常侍、侍中、尚書左僕射,復兼選部如昔。及將獲罪,默因拘成以誅之,然後食甘寢安,志於職事。出為儀同三司、定州刺史,進爵為王。克己清儉,事濟公私。後薨於冀州刺史,追贈司空,謚曰康。



 孫小,字茂翹,咸陽石安人也。父瓚,姚泓安定護軍,為赫連屈丐所殺,小沒入宮刑。會魏平統萬,遂徙平城。內侍東宮,以聰識有智略稱。未幾,轉四臺中散。



 太武幸瓜步,
 慮有北寇之虞,賜爵泥陽子,除留臺將軍。車駕還都,乃請父瓚贈謚,求更改葬。詔贈秦州刺史、石安縣子,謚曰戴。小後拜并州刺史,進爵中都侯。州內四郡百餘人,詣闕頌其政化。後遷冀州刺史,聲稱微少於前。然所在清約,當時牧伯,無能及也。性頗忍酷,所養子息,驅逐鞭撻,視如仇讎。小之為並州,以郭祚為主簿。重祚文才,兼任以書記,時人多之。



 張宗之,字益宗,河南鞏人也。家世寒微。父孟舒,晉將劉裕西征,板假洛陽令。初、緱氏宗文邕謀反,脅孟舒等事晉。孟舒敗,走免。宗之被執入京,腐刑。



 以忠厚謹慎,擢為
 侍御中散,賜爵鞏縣侯。歷儀曹、庫部二曹尚書,領中祕書,進爵彭城公,後例降為侯。卒於冀州刺史,贈懷州刺史,謚曰敬。



 始宗之納南來殷孝祖妻蕭氏,宋儀同三司思話弟思度女也,多悉婦人儀飾故事。



 太和中,初制六宮服章,蕭被命在內,豫見訪採,數蒙賜賚云。



 劇鵬,高陽人也。粗覽經史,閑曉吏事。與王質等俱充宦官,性通率,不以閽閹為恥。孝文遷洛,常為宮官任事。幽后之惑薛菩薩也,鵬密諫止之,不從,遂發憤卒。



 張祐,字安福,安定石唐人也。父成,扶風太守,太武末,坐事誅。祐充腐刑,積勞至曹監、中給事。文明太后臨朝,中
 官用事,祐寵幸冠諸閹,官特遷、尚書,進爵隴東公,仍綰內藏曹。未幾監都曹,加侍中,與王睿等俱入八議。太后嘉其忠誠,為造甲第。宅成,孝文、太后親率文武往宴會焉。拜尚書左僕射,進爵新平王,受職於太華庭,備威儀於宮城南,觀者以為榮。孝文、太后親幸其宅,饗會百官。



 祐性恭密,出入機禁二十餘年,未嘗有過。由是特被恩寵,歲月賞賜,家累巨萬。



 與王質等十七人,俱賜金券,許以不死。薨,孝文親臨之,詔鴻臚典護喪事。贈司空,謚曰恭。葬日,車駕親送近郊。



 祐養子顯明,後名慶,少歷內職,有姿貌,江陽王繼以女妻之。襲爵,降為隴東公,又降為
 侯。



 抱嶷,字道德,安定石唐人也,居於直谷。自言其先姓巳,漢靈帝時,巳匡為安定太守。董卓時,懼誅易氏,即家焉。無得而知也。幼時,隴東人張乾王反,家染其逆。及乾王敗,父睹生逃免。嶷獨與母沒入內宮,受刑,遂為宦人。小心慎密,累遷中常侍、中曹侍御尚書,賜爵安定公。自總納言,職當機近,諸所奏議,必致抗直。孝文、文明太后嘉之,以為殿中侍御尚書。太后既寵之,乃徵其父睹生,拜太中大夫。將還,見於皇信堂,孝文執手曰:「老人歸途,幾日可達?好慎行路!」



 其見幸如此。睹生卒,贈秦州刺史,謚
 曰靖。賜黃金八十斤,繒彩及絹八百疋,以供喪用。并別使勞尉。加嶷大長秋卿。嶷老疾,乞外祿,乃出為涇州刺史,特加右光祿大夫。將之州,孝文餞於西郊樂陽殿,以御白羽扇賜之。十九年,以刺史從駕南征,以老舊,每見勞問,數道稱嶷之正直。命乘馬出入行禁之間,與司徒馮誕同例。軍回,還州。自以故老前官,為政多守往法,不能遵用新制。侮慢土族,簡於禮接。天性酷薄,雖弟姪甥婿,略無存潤。卒於州。



 先以從弟老壽為後,又養太師馮熙子次興。嶷死後,二人爭立。嶷妻張氏,致訟經年,得以熙子為後。老壽亦仍陳訴,終獲紹爵,次興還於本族。老
 壽凡薄,酒色肆情。御史中尉王顯奏言:「前洛州刺史陰平子石榮、積射將軍抱老壽,恣蕩非軌,易室而姦,臊聲布於朝野,醜音被於行路,男女三人,莫知誰子。人理所未聞,鳥獸之不若。請以見事免官,付廷尉正罪。」詔可之。老壽死後,其舊奴婢尚六七百人。老壽及石榮祖父皆造碑銘,就鄉建立,言西方直谷出二貴人。



 石榮自被劾後,遂廢頓。子長宣,位南兗州刺史,與侯景反,伏法。



 王遇,字慶時,本名他惡,馮翊李潤鎮羌也。與雷、黨、不蒙俱為羌中強族。



 自云其先姓王,後改為鉗耳氏,宣武時,改為王焉。自晉已來,恒為渠長。遇坐事腐刑,累遷吏部
 尚書,爵宕昌公。出為華州刺史,加散騎常侍。幽后之前廢也,遇頗言其過。及後進幸,孝文對李沖等申后無咎,而稱遇謗議之罪,遂免遇官,奪其爵。宣武初,為光祿大夫,復舊爵。馮氏為尼也,公私罕相供恤,遇自以嘗更奉接,往來祗謁,不替舊敬。



 遇性工巧,強於部分。北都方山、靈泉道俗居宇,及文明太后陵廟,洛京東郊馬射壇殿,修廣文昭太后墓園,及東西兩堂,內外諸門制度,皆遇監作。雖年在耆老,朝夕不倦。又長於人事,留意酒食之間。每逢僚舊,觴膳精豐。然競於榮利,趨求勢門。趙修之寵也,遇深附會,受敕為之造宅,增於本旨,笞擊作人,莫
 不嗟怨。卒於官。初遇之疾,太傅北海王與太妃俱往臨問,視其危惙,為之泣下。其善奉諸貴,致相悲悼如此。贈雍州刺史。



 苻承祖,略陽氐人也。因事為閹人,為文明太后所寵,賜爵略陽公。歷吏部尚書,加侍中,知都曹事。初,太后以承祖居腹之心任,許以不死之詔。後承祖坐贓應死,孝文原之,命削職禁錮在家,授悖義將軍、佞濁子。月餘遂死。



 王質,字紹奴,高陽易人也。其家坐事,幼下蠶室。頗解書學,為中曹吏、內典監。稍遷祕書中散,賜爵永昌子,領監御。遷為侍御給事。又領選部、監御二曹事,進爵魏昌侯。
 轉選部尚書。出為瀛州刺史,風化粗行,人庶畏服之;而刑政峻刻,號為威酷。孝文頗念其忠勤宿舊,每行留大故、馮司徒亡、廢馮后、陸睿、穆泰等事,皆賜質以璽書手筆,莫不委至,同之戚貴。質皆寶掌。入為大長秋卿,卒。



 李堅,字次壽,高陽易人也。文成初,坐事為閹人,稍遷中給事中,賜爵魏昌伯。小心謹慎,常在左右,雖不及王遇、王質等,而亦見任用。宣武初,自太僕卿出為瀛州刺史。本州之榮,同於王質。所在受納,家產巨萬。卒於光祿大夫,贈相州刺史。



 太和末,又有秦松、白整,位並長秋卿。



 劉騰,字青龍,本平原城人也,徙屬南兗州之譙郡。幼時坐
 事受刑,補小黃門,轉中黃門。孝文之在縣瓠,問其中事,騰具言幽后私隱,與陳留公主所告符協,由是進冗從僕射,仍中黃門。後與茹皓使徐、兗,采召人女。還,遷中給事。



 靈太后臨朝,以與於忠保護勳,除崇訓太僕,加侍中,封長樂縣公。拜其妻魏氏為巨鹿郡君,每引入內,受賞賚亞於諸主外戚。所養二子,為郡守、尚書郎。騰曾疾篤,靈太后慮或不救,遷衛將軍、儀同三司。後疾瘳。騰之拜命,孝明當為臨軒,會日,大風寒甚,乃遣使持節授之。騰幼充宮役,手不解書,裁知署名而已,而姦謀有餘,善射人意。靈太后臨朝,特蒙進寵,多所乾託,內外碎密,棲棲
 不倦。



 洛北永橋、太上公、太上君及城東三寺,皆主脩營。



 吏部嘗望騰意,奏其弟為郡,帶戍。人資乖越,清河王懌抑而不奏。騰以為恨,遂與領軍元叉害懌,廢靈太后於宣光殿。宮門晝夜長閉,內外斷絕。騰自執管籥,明帝亦不得見,裁聽傳食而已。太后服膳俱廢,不免饑寒。又使中常侍賈粲假言侍明帝書,密令防察。叉以騰為司空,表裏擅權,共相樹置。叉為外禦,騰為內防,迭直禁闥,共裁刑賞。騰遂與崔光同受詔,乘步挽出入殿門。四年之中,生殺之威,決於叉、騰之手。八坐九卿,旦造騰宅,參其顏色,然後方赴省府;亦有歷日不能見者。公私屬請,唯
 在財貨,舟車之利,水陸無遺,山澤之饒,所在固護,剝削六鎮,交通底市,歲入利息以巨萬計。又頗役嬪御,時有徵求,婦女器物,公然受納,逼奪鄰居,廣開室宇,天下咸苦之。薨於位,中官為義息衰絰者四十餘人。騰之立宅也,奉車都尉周恃為之筮,不吉,深諫止之。騰怒而不用。恃告人曰:「必困於三月、四月之交。」至是果死。事甫成,陳屍其下。追贈太尉、冀州刺史。葬,閹官為義服,杖絰衰縞者以百數。朝貴皆從,軒蓋填塞,相屬郊野。魏初以來,權閹存亡之盛,莫及焉。



 靈太后反政,追奪爵位,發其塚,散露骸骨,沒入財產。後騰所養一子叛入梁,太后大怒,
 悉徙騰餘養於北裔,尋遣密使追殺之於汲郡。



 賈粲,字季宣,酒泉人也。太和中,坐事腐刑。頗涉書記。與元叉、劉騰等同其謀謨,進光祿勳卿。專侍明帝,與叉、騰等伺帝動靜。右衛奚康生之謀殺叉也,靈太后、明帝同升於宣光殿,左右侍臣,俱立西階下。康生既被囚執,粲紿太后曰:「侍官懷恐不安,陛下宜親安慰。」太后信之,適下殿,粲便扶明帝出東序,前御顯陽,還閉太后於宣光殿。粲既叉黨,威福亦震於京邑。自云本出武威,魏太尉文和之後,遂移家屬焉。時武威太守韋景承粲意,以其兄緒為功曹。緒時年向七十。



 未幾,又以緒為西平太守。
 靈太后反政,欲誅粲,以叉、騰黨與不一,恐驚動內外,乃止。出粲為濟州刺史。未幾,遣武衛將軍刁宣馳驛殺之。



 楊範,字法僧,長樂廣宗人也。文成時,坐事宮刑,為王琚所養,恩若父子。



 累遷為中尹。靈太后臨朝,為中常侍、崇訓太僕,領中嘗藥典御,賜爵華陰子,出為華州刺史。中官內侍貴者,靈太后皆許其方岳,以範年長,拜跪為難,故遂其請。



 父子納貨,為御史所糾,遂廢於家。後為崇訓太僕、華州大中正,卒。



 成軌,字洪義,上谷居庸人也。少以罪刑,入事宮掖。以謹厚稱,為中謁者僕射。孝文意有所欲,軌候容色,時有奏
 發,輒合帝心。從駕南征,專進御食。時孝文不豫,常居禁中,晝夜無懈。延昌末,遷中常侍、嘗食典御、光祿大夫,統京染都將。孝昌二年,以勤舊封始平縣伯。明帝所幸潘嬪以軌為假父,頗為中官之所敬憚。後進爵為侯,卒於衛將軍,贈雍州刺史,謚曰孝惠。



 王溫,字桃湯,趙郡欒城人也。父冀,高邑令,坐事誅,溫與兄繼叔俱充宦者,稍遷中嘗食典御、中給事,加左中郎將。宣武之崩,群官迎明帝於東宮,溫於臥中起明帝,與保母扶抱明帝,入踐帝位。高陽王雍既居冢宰,慮中人朋黨,出為巨鹿太守。靈太后臨朝,徵為中常侍,賜爵欒
 城伯。累遷左光祿大夫、光祿勳卿、侍中,進封欒城縣侯。溫自陳本陽平武陽人,改封武陽縣侯。建義初,於河陰遇害。



 孟欒,字龍兒,不知何許人也。坐事為閹人。靈太后臨朝,為左中郎將、給事中。素被病,面常黯黑。於九龍殿下暴疾,歸家,甚夜亡。欒初出,靈太后聞之曰:「欒必不濟,我為之憂。」乃奏其死,為之下淚曰:「其事我如此,不見我一日忻樂時也。」賜帛三百疋、黃絹一十疋,以供喪用。七日,靈太后為設二百僧齋。



 平季,字幼穆,燕國薊人也。坐事腐刑。累遷新興太守。明
 帝崩,與爾朱榮等議立莊帝。莊帝即位,超拜肆州刺史。尋除中侍中。以參謀勳,封元城縣侯。永熙中,加驃騎大將軍,卒。



 封津,字醜漢,勃海蓚人也。父令德,娶常寶女。寶伏誅,令德以連坐伏法。



 津受刑,給事宮掖。累遷奉車都尉、中給事中。靈太后令津侍明帝書,遷常山太守。



 津少長宮闈,給事左右,善候時情,號為機悟。天平初,除開府儀同三司、懷州刺史。元象初,復為中侍中、大長秋卿,仍開府儀同。薨,贈司徒、冀州刺史,謚曰孝惠。



 劉思逸,平原人也。以罪,少充腐刑。初為小史,累遷中侍
 中。武定中,與元瑾等謀反,伏誅。



 又有張景嵩、毛暢者,咸以閽寺在明帝左右。靈太后亦密仗之通傳意計於明帝。



 元叉之出,景嵩、暢頗有力焉。靈太后反政,以妹故,未即戮叉。時內外喧喧,元叉還欲入知政事。暢等恐禍及己,乃啟明帝,欲詔右衛將軍楊津密往殺叉。詔書已成,未及出外,叉妻知之,告太后:「景嵩、暢與清河王息欲廢太后。」太后信之,責暢。暢出詔草以呈太后。太后讀之,知無廢己狀,意小解。然叉妻構之不已,出暢為頓丘太守,景嵩為魯郡太守。尋令捕殺暢。景嵩,孝靜時位至中侍中,坐事死。



 郭秀,范陽涿人也。事齊神武,稍遷行臺右丞,封壽陽伯。親寵日隆,多受賂遺,進退人物。張伯德、祁仲彥、張華原之徒,皆深相附會。秀疾,神武親視之,問所欲官,乃啟為七兵尚書,除書未至而卒。家無成人子弟,神武自至其宅,親使錄知其家資粟帛多少,然後去。贈儀同三司、恒州刺史。命其子孝義與太原公以下同學讀書。初,秀忌嫉楊愔,誑脅令其逃亡。秀死後,愔還,神武追忿秀,即日斥遣孝義,終身不齒。



 和士開,字彥通,清都臨漳人也。其先西域商胡,本姓素和氏。父安,恭敏善事人,稍遷中書舍人。魏靜帝嘗夜與
 朝賢講集,命安看斗柄所指。安曰:「臣不識北斗。」齊神武聞之,以為淳直,由是啟除給事黃門侍郎,位儀州刺史。士開貴,贈司空公、尚書左僕射、冀州刺史,謚文貞公。



 士開幼而聰慧,選為國子學生,解悟捷疾,為同業所尚。天保初,武成封長廣王,辟士開開府行參軍。武成好握槊,士開善此戲,由是遂有斯舉。加以傾巧便僻,又能彈胡琵琶,因致親寵。嘗謂王曰:「殿下非天人也,是天帝也。」王曰:「卿非世人也,是世神也。」其深相愛重如此。文宣知其輕薄,不欲令王與小人相親善,責其戲狎過度,徙之馬城。乾明元年,孝昭誅楊愔等,敕追還,長廣王請之也。



 武
 成即位,累遷給事黃門侍郎。侍中高元海、黃門郎高乾和及御史中丞畢義雲等疾之,將言其事。士開乃奏元海等交結朋黨,欲擅威福。乾和因被疏斥,義雲反納貨於士開,除兗州刺史。士開初封定州真定縣子,尋進為伯。天統元年,加儀同三司,尋除侍中,加開府。及遭母劉氏憂,帝聞而悲惋,遣武衛將軍侯呂芬詣宅,晝夜扶侍,並節哀止哭。又遣侍中韓寶業齎手敕慰諭云:「朕之與卿,本同心腹,今懷抱痛割,與卿無異。當深思至理,以自開慰。」成服後,呂芬等始還。其日,遣韓寶業以犢車迎士開入內,帝親握手,下泣曉諭,然後遣還。駕幸晉陽,給假,
 聽過七日續發,其見重如此。并諸弟四人,並起復本官。四年,再遷尚書右僕射。



 帝先患氣疾,因飲酒輒大發動,士開每諫不從。後屬帝氣疾發,又欲飲酒,士開淚下噓欷而不能言。帝曰:「卿此是不言之諫。」因不飲酒。及冬,公主出降段氏,帝幸平原王第,始飲酒焉。又除尚書左僕射,仍兼侍中。武成外朝視事,或在內宴賞,須臾之間,不得不與士開相見。或累月不歸,一日數入;或放還之後,俄頃即追,未至之間,連騎催喚。姦諂日至,寵愛彌隆,前後賞賜,不可勝紀。言辭容止,極諸鄙褻,以夜繼晝,無復君臣之禮。至說武成云:「自古帝王,盡為灰土,堯舜、桀紂,
 竟復何異?陛下宜及少壯,恣意作樂,從橫行之,即是一日快活敵千年。國事分付大臣,何慮不辦?無為自勤約也。」帝大悅,於是委趙彥深掌官爵,元文遙掌財用,唐邕掌外兵,白建掌騎兵,馮子琮、胡長粲掌東宮。帝三四日乃一坐朝,書數字而已,略無言,須臾罷入。及帝寢疾於乾壽殿,士開入侍醫藥。帝謂士開有伊、霍之才,殷勤屬以後事,臨崩握其手曰:「勿負我也。」仍絕於士開之手。



 後主以武成顧託,深委任之。又先得幸於胡太后,是以彌見親密。趙郡王睿與婁定遠、元文遙等謀出士開,仍引任城、馮翊二王及段韶、安吐根共為計策。屬太后觴朝貴於前殿,睿面陳士開罪失云:「士開,先帝弄臣,城狐社鼠,受納貨
 賄,穢亂宮掖。臣等義無杜口,冒以死陳。」太后曰:「先帝在時,王等何意不道?今日欲欺孤寡邪!但飲酒,勿多言。」睿詞色愈厲。安吐根繼進曰:「臣本商胡,得在諸貴行末,既受厚恩,豈敢惜死?不出士開,朝野不定。」太后曰:「別日論之,王等且散。」睿等或投冠於地,或拂衣而起,言詞咆哱,無所不至。明日,睿等復於雲龍門令文遙入奏,三反,太后不聽。段韶呼胡長粲傳言於太后。曰:「梓宮在殯,事太匆速,猶欲王等更思量。」趙郡王等遂並拜謝。長粲復命,太后謂曰:「成妹母子家計者,兄之力也。」厚賜睿等而罷之。



 太后及後主召問士開,士開曰:「先帝群臣中,待臣最
 重。陛下諒陰始爾,大臣皆有覬覦,今若出臣,正是翦陛下羽翼。宜謂睿等,云文遙與臣同是任用,豈得一去一留,並可以為州。且依舊出納,待過山陵,然後發遣。睿等謂臣真出,心必喜之。」後主及太后告睿等,如其言,以士開為兗州刺史,文遙為西兗州刺史。山陵畢,睿等促士開就路。士開載美女珠簾及諸寶玩以詣婁定遠,謝曰:「諸貴欲殺士開,蒙王特賜性命,用作方伯。今欲奉別,且送二女子、一珠簾。」定遠大喜,謂士開曰:「欲還入不?」士開曰:「在內久,常不自安,不願更人。」定遠信之,送至門。士開曰:「今日遠出,願一辭觀二宮。」定遠許之。由是得見後主
 及太后,進說曰:「先帝一旦登遐,臣愧不能自死。觀朝貴意勢,欲以陛下為乾明。臣出之後,必有大變,復何面目見先帝於地下!」因慟哭。後主及太后皆泣,問計將安出。



 士開曰:「臣已得入,復何所慮?正須數行詔書耳。」於是詔定遠為青州刺史;責趙郡王睿以不臣,召入殺之;復除士開侍中、尚書左僕射。定遠歸士開所遺,加以餘珍賂之。武平元年,封淮陽王,尋除尚書令,還錄尚書事,食定州常山郡幹。



 武成時,恆令士開與太后握槊,又出入臥內,遂與太后為亂。及武成崩後,彌自放恣。瑯邪王儼惡之,與領軍大將軍厙狄伏連、侍中馮子琮、書侍御史王
 子宜、武衛大將軍高舍洛等謀誅之。伏連發京畿軍士帖神武千秋門外,並私約束,不聽士開入殿。士開雖為領軍,恆性好內,多早下,縱當直,必須還宅,晚始來。門禁宿衛,略不在意。及旦,士開依式早參,厙狄伏連把士開手曰:「今有一大好事。」



 王子宜便授一函云:「有敕,令王向臺。」遣軍士防送,禁治書侍御事。儼遣都督馮永洛就臺斬之。先是鄴下童謠云:「和士開,當入臺。」士開謂入上臺,至是果驗。儼令御史李幼業、羊立正將令史就宅簿錄家口,自領兵士縱殿西北角出。斛律明月說後主親自曉告軍士,軍士果散。即斬伏連及王子宜,並支解,棄
 屍殿西街。



 自餘皆辮頭反縛,付趙彥深於涼風堂推問,死者十餘人。帝哀悼,不視事數日。後追憶不已,詔起復其子道盛通直散騎常侍,又敕其弟士休入內省,參典機密。詔贈士開假黃鉞、右丞相、太宰、司徒公,錄尚書事,謚曰文定。



 士開稟性庸鄙,不窺書傳,發言吐論,唯以諂媚自資。自河清、天統以後,威權轉盛,富商大賈,朝夕填門,聚斂貨財,不知紀極。雖公府屬掾,郡縣守長,不拘階次,啟牒即成。朝士不知廉恥者,多相附會,甚者為其假子,與市道小人丁鄒、嚴興等同在昆季行列。又有一人士,曾參士開疾患,遇醫人云,王傷寒極重,應服黃龍湯,
 士開有難色。是人云:「此物甚易,王不須疑惑,請為王先嘗之。」一舉便盡。士開深感此心,為之強服,遂得汗病愈。其勢傾朝廷如此。雖以左道事之者,不隔賢愚,無不進擢;而正理違忤者,亦頗能含容之。士開見人將加刑戮,多所營救,既得免罪,即令諷論,責其珍寶,謂之贖命物。雖有全濟,皆非直道。



 安吐根,安息胡人,曾祖入魏,家於酒泉。吐根魏末充使蠕蠕,因留塞北。天平初,蠕蠕主使至晉陽,吐根密啟本蕃情狀,神武得為之備。蠕蠕果遣兵入掠,無獲而反。神武以其忠款,厚加賞賚。其後與蠕蠕和親,結成婚媾,皆吐根為行人也。



 吐根性和善,頗有
 計策,頻使入朝,為神武親待。在其本蕃,為人所譖,奔投神武。



 文襄嗣事,以為假節、涼州刺史、率義侯,稍遷儀同三司,食永昌郡幹。皇建中,加開府。齊亡年,卒。



 穆提婆,本性駱,漢陽人也。父超,以謀叛伏法,提婆母陸令萱配入掖庭,提婆為奴。後主在襁褓中,令其鞠養,謂之乾阿妳,呼姊姊,遂為胡太后暱愛。令萱姦巧多機辯,取媚百端,宮掖之中,獨擅威福,封為郡君。和士開,高阿那肱皆為郡君義子。天統初,奏引提婆入侍後主,朝夕左右,大被親狎,無所不為。武平元年,稍遷儀同三司,又加開府,尋授武衛大將軍、秦州大中正。二年,除侍中,轉食
 樂陵郡幹,寵遇彌隆。遂至尚書左右僕射、領軍大將軍、錄尚書,封城陽郡王。



 贈其父司徒公、尚書左僕射、城陽王。令萱又佞媚穆昭儀,養之為女,是以提婆改姓穆。及穆氏定位,號視第一品,班在長公主之上。



 自武平三年之後,令萱母子勢傾內外,賣官鬻獄,取斂無厭,每一賜與,動傾府藏。令萱則自太后以下,皆受其指麾;提婆則唐邕之徒,皆重跡屏氣。提婆嘗有罪,太姬於帝前駕之曰:「奴斷我兒!」兒謂帝,奴謂提婆也。



 斛律皇后之廢也,太后欲以胡昭儀正位後宮,力不能遂,乃卑辭厚禮,以求令萱。令萱亦以胡氏寵幸方睦,不得已而白後主立之。
 然意在穆昭儀,每私謂後主曰:「豈有男為皇太子,而身為婢妾?」又恐胡后不可以正義離間,乃外求左道行厭蠱之術,旬朔之間,胡氏遂即精神恍惚,言笑無恆,後主遂漸相畏惡。令萱一旦忽以皇后服御衣被穆昭儀,又先別造寶帳,爰及枕席器玩,莫匪珍奇,坐昭儀於帳中,謂後主云:「有一聖女出,將大家看之。」及見,昭儀更相媚悅。令萱云:「如此人不作皇后,遣何物人作皇后?」於是立穆氏為右皇后,以胡氏為左皇后,尋復黜胡,以穆為正嫡。引祖珽為宰相,殺胡長仁,皆令萱所為也。自外殺生與奪,不可盡言。



 提婆雖庸品廝濫,而性乃和善,不甚害
 物。耽聲色,極奢侈,晚朝早退,全不以公事關懷。未嘗毒害,士人亦由此稱之。晉州軍敗,後主還鄴,提婆奔投周軍,令萱自殺,子孫小大皆棄市,籍沒其家。周武帝以提婆為柱國、宜州刺史。未幾,云將據宜州起兵,與後主相應,誅死。後主及齊氏諸王,並因此非命。



 高阿那肱,善無人也。父市貴,從神武以軍功封常山郡公,位晉州刺史,贈太尉公。及阿那肱貴寵,贈成皋王。



 阿那肱初為庫直,每從征討,以功封直城縣男。天保初,除庫直都督。四年,從破契丹及蠕蠕,以蹻捷見知。大寧初,除假儀同三司、武衛將軍。那肱工於騎射,便僻善事人,
 每宴射之次,大為武成愛重。又諂悅和士開,尤相褻狎。士開每見為之言,由是彌見親待。河清中,除儀同三司,食汾州定陽、仵城二郡幹。以破突厥,封宜君縣伯。天統初,加開府,除侍中、驃騎大將軍、領軍,別封昌國縣侯。後主即位,除并省右僕射。武平元年,封淮陰郡王,仍遷并省尚書左僕射,又除并省尚書令、領軍大將軍、并州刺史。



 那肱才技庸劣,不涉文史,識用尤在士開下。而姦巧計數,亦不逮士開。既為武成所幸,多令在東宮侍衛,後主所以大寵遇之。士開死後,後主謂其識度足繼士開,遂致位宰輔。武平四年,令其錄尚書事,又總知外兵及
 內省機密。頓不如和士開、駱提婆母子賣獄鬻官,韓長鸞憎疾良善;而那肱少言辭,不妄喜怒,亦不察人陰私,虛相讒構。遂至司徒公、右丞相,其錄尚書、刺史並如故。及周師逼平陽,後主於天池校獵,晉州頻遣馳奏,從旦至午,驛馬三至。那肱云:「大家正作樂,邊境小小兵馬,自是常事,何急奏聞?」向暮,更有使至,云平陽城已陷賊,方乃奏知。明即欲引軍,淑妃又請更合圍,所以彌致遲緩。及軍赴晉州,命那肱率前軍先進,仍總節度諸軍。



 後主至平陽城下,謂那肱曰:「戰是邪?不戰是邪?」那肱曰:「兵雖多,堪戰者不過十萬,病傷及繞城火頭,三分除一。昔攻玉壁,
 援軍來,即退。今日將士豈勝神武皇帝時?不如勿戰,守高梁橋。」安吐根曰:「一把子賊,馬上刺取擲汾河中。」帝未決,諸內參曰:「彼亦天子,我亦天子,彼尚能縣軍遠來,我何為守塹示弱?」帝曰:「此言是也。」於是橋塹進軍,使內參讓阿那肱曰:「爾富貴足,惜性命邪!」



 後主從穆提婆觀戰,東偏頗有退者,提婆怖曰:「大家去!大家去!」帝與淑妃奔高梁。開府奚長樂諫曰:「半進半退,戰家常體。今眾全整,未有傷敗,陛下舍此安之?御馬一動,人情驚亂,願速還安慰之。」武衛張常山自後至,亦曰:「軍尋收訖,甚整頓,圍城兵亦不動,至尊宜回。不信臣言,乞將內參往視。」帝將
 從之,提婆引帝肘曰:「此言何可信!」帝遂北馳。有軍士雷相,告稱:「阿那肱遣臣招引西軍,行到文侯城,恐事不果,故還聞奏。」後主召侍中斛律孝卿,令其檢校。孝卿固執云:「此人自欲投賊,行至文侯城,迷不得去,畏死妄語耳。」



 事遂寢。還至晉陽,那肱腹心人馬子平告那肱謀反,又以為虛妄,斬子平。乃顛沛還鄴,侍衛逃散,唯那肱及閹寺等數十騎從行。復除大丞相。



 後主走度河,令那肱以數千人投濟州關,仍遣覘候周軍進止,日夕馳報。那肱每奏云:「周軍未至,且在青州集兵馬,未須南行。」及周軍且至關首,所部兵馬皆散,那肱遂降。時人皆云,那肱表
 款周武,必仰生致齊主,故不速報兵至,使後主被禽。那肱至長安,授大將軍,封郡公,尋出為隆州刺史。大象末,在蜀從王謙起兵,誅死。



 初,天保中,文宣自晉陽還鄴,愚僧禿師於路中大叫,呼文宣姓名云:「阿那瑰終破你國。」時蠕蠕主阿那瑰在塞北強盛,帝尤忌之,所以每歲討擊。後亡齊者遂屬高阿那肱云。雖作「肱」字,世人皆稱為「瑰」音。斯固亡秦者胡,蓋縣定於窈冥也。



 韓鳳,字長鸞,昌黎人也,父永興,開府、青州刺史、高密郡公。鳳少聰察,有膂力,善騎射,稍遷烏賀真、大賢真正都督。後主居東宮,年尚幼,武成簡都督三十人,送令侍衛,
 鳳在其數。後主親就眾中牽鳳手曰:「都督,看兒來。」因此被識,數喚共戲。襲爵高密郡公,位開府儀同三司。武平二年,和士開為厙狄伏連等矯害,敕咸陽王斛律明月、宜陽王趙彥深在涼風堂推問支黨。其事祕密,皆令鳳口傳,然後宣詔敕號令文武。禁掖防守,悉以委之。除侍中、領軍,總知內省機密。



 祖珽曾與鳳於後主前論事,珽語鳳云:「彊弓長槊,容相推謝;軍國謀算,何由得爭?」鳳答云:「各出意見,豈在文武優劣!」後主將誅斛律明月,鳳固執不從。祖珽因有讒言,既誅明月,數日後主不興語,後尋復舊。仍封舊國昌黎郡王,又加特進。及祖珽除北徐
 州刺史,即令赴任。既辭之後,遲留不行。其省事徐孝遠密告祖珽誅斛律明月後,矯稱敕賜其珍寶財物,亦有不云敕而徑回取者。敕令領軍將軍侯呂芬追珽還,引入侍中省鎖禁,其事首尾,並鳳約敕責之。



 進位領軍大將軍,餘悉如故。息寶行尚公主,在晉陽賜甲第一區。其公主生男滿月,駕幸鳳宅,宴會盡日。每旦早參,先被敕喚顧訪,出後方引奏事官。若不視事,內省急速者,皆附奏聞。軍國要密,無不經手。東西巡幸,及山水游戲射獵,獨在御傍。與高阿那肱、穆提婆共處衡軸,號曰三貴。損國害政,日月滋甚。



 壽陽陷沒,鳳與穆提婆聞告敗,握槊
 不輟曰:「他家物,從他去。」後帝使於黎陽臨河築城戍,曰:「急時且守此作龜茲國子。更可憐人生如寄,唯當行樂,何用愁為?」君臣應和若此。鳳恆帶刀走馬,未曾安行,瞋目張拳,有啖人之勢。每吒曰:「恨不得剉漢狗飼馬!」又曰:「刀止可刈賊漢頭,不可刈草。」其弟萬歲,及其二子寶行、寶信,並開府儀同,萬歲又拜侍中,亦處機要。寶信尚公主,駕復幸其宅,親戚咸蒙官賞。



 鳳母鮮於,段孝言之從母子姊也,為此偏相參附,奏遣監造晉陽宮。陳德信馳驛檢行,見孝言役官夫匠自營宅,即語云:「僕射為至尊起臺殿未訖,何用先自營造?」鳳及穆提婆亦遣孝言分工匠為己造宅。德信還,具奏聞。及幸晉陽,鳳又以官馬與他人乘騎,
 上因此發忿,與提婆並除名。亦不露其罪。仍毀其宅,公主離婚,復被遣尚鄴吏部門參。及後主晉陽走還,被敕喚入內,尋詔復王爵及開府、領軍大將軍,常在左右。仍從後主走度河,到青州,并為周軍所獲。



 鳳被寵要之中,尤嫉人士,朝夕宴私,唯相譖訴。崔季舒等冤酷,皆鳳所為也,每一賜與,動至千萬。恩遇日甚,彌自驕恣,意色嚴厲,未嘗與人相承接。朝士諮事,莫敢仰視,動致呵叱,輒詈云:「狗漢大不可耐!唯須殺卻!」若見武職,雖廝養末品,亦容下之。仕隋,位終於隴州刺史。



 宦者韓寶業、盧勒叉、齊紹、秦子徵並神武舊左右,唯閣
 內驅使,不被恩遇。



 歷天保、皇建之朝,亦不至寵幸,但漸有職任。寶業至長秋卿,勒叉等或為中常侍。



 武成時有曹文摽、夏侯通、伊長游、魯恃伯、郭沙彌、鄧長顒及寶業輩,亦有至儀同食乾者。唯長顒武平中任參宰相,干預朝權。如寶業及勒叉、齊紹、子徵後並封王,俱自收斂,不過侵暴。又有陳德信亦參時宰,與長顒並開府封王,俱為侍中、左右光祿大夫,領侍中。又有潘師子、崔孝禮、劉萬通、研胥光弁、劉通遠、王弘遠、王子立、王玄昌、高伯華、左君才、能純陀、宮鍾馗、趙野叉、徐世凝、茍子溢、斛子慎、宋元寶、康德汪,並於後主之朝,肆其姦佞。敗政虐人,古
 今未有。



 多授開府,罕止儀同,亦有加光祿大夫,金章紫綬者。多帶中侍中、中常侍,此二職乃至數十人。恆出入門禁,往來園苑,趨侍左右,通宵累日。承候顏色,競進諂諛,發言動意,多會深旨。一戲之賞,動逾巨萬,丘山之積,貪吝無厭。猶以波斯狗為儀同、郡休,分其幹祿,神獸門外,有朝貴憩息之所,時人號為解卸。諸閹或在內多日,暫放歸休,所乘之馬,牽至神獸門階,然後升騎。飛鞭競走,十數為群,馬塵必坌諸貴,爰至唐、趙、韓、駱,皆隱趨避,不敢為言。齊、盧、陳、鄧之徒,亦意屬尚書、卿尹,宰相既不為致言,時主亦無此命。唯以工巧矜功,用長顒為
 太府卿焉。



 神武時有倉頭陳山提、蓋豐樂,俱以驅馳便僻,頗蒙恩遇。魏末,山提通州刺史,豐樂嘗食典御。又有劉郁斤、趙道德、劉桃枝、梅勝郎、辛洛周、高舍洛、郭黑面、李銅鍉、王恩洛,並為神武驅使。天保、大寧之朝,漸以貴盛。至武平時,山提等皆以開府封王。其不及武平者則追贈王爵。雖賜與無貲,顧眄深重,乃至陵忽宰輔,然皆不得干預朝政。



 武平時有胡小兒,俱是康阿馱、穆叔兒等富家子弟,簡選黠慧者數十人以為左右,恩眄出處,殆與閹官相埒。亦有至開府儀同者。其曹僧奴、僧奴子妙達,以能彈胡琵琶,甚被寵遇,俱開府封王。又有何海
 及子洪珍,開府封王,尤為親要。洪珍侮弄權勢,鬻獄賣官。其何朱弱、史醜多之徒十數人,咸以能舞工歌及善音樂者,亦至儀同開府。



 閹官猶以宮掖驅馳,便蕃左右,漸因暱狎,以至大官。倉頭始自家人,情寄深密,及於後主,則是先朝舊人,以勤舊之勞,致此叨竊。至於胡小兒等,眼鼻深險,一無可用,非理愛好,排突朝貴,尤為人士之所疾惡。



 其以音樂至大官者:沈過兒,官至開府儀同;王長通,年十四五便假節、通州刺史。



 時又有開府薛榮宗,常自云能使鬼。及周兵之逼,言於後主曰:「臣已發遣斛律明月將大兵在前去。」帝信之。經古冢,榮宗謂舍人
 元行恭:「是誰塚?」行恭戲之曰:「林宗塚。」復問:「林宗是誰?」行恭曰:「郭元貞父。」榮宗前奏曰:「臣向見郭林宗從塚出,著大帽、吉莫靴,棰馬鞭,問臣:『我阿貞來不?』」是時群妄,多皆類此。



 論曰:古諺有之,「人之多幸,國之不幸。」然則寵私為害,自古忌之。大則傾國亡身,小則傷賢害政,率由斯也,所宜誡焉。《詩》曰:「殷鑒不遠,近在夏后之世。」觀夫魏氏以降,亦後來之殷鑒矣。為國家者,可無鑒之哉?



\end{pinyinscope}