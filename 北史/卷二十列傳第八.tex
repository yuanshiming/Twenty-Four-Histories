\article{卷二十列傳第八}

\begin{pinyinscope}

 衛
 操莫含劉庫仁弟子羅辰羅辰曾孫仁之尉古真從玄孫瑾穆崇奚斤叔孫建安同庾業延王建羅結樓伏連曾孫寶閭大肥奚牧和跋莫題賀狄乾李慄奚
 眷衛操,字德元,代人也。少通俠,有才略。晉征北將軍衛瓘以操為牙門將。當魏神元時,頗自結附。及神元崩後,與從子雄及其宗室鄉親姬淡等來歸,說桓、穆二帝招納晉人。桓帝以為輔相,任以國事。及劉、石之亂,桓帝匡助晉氏。操稍遷至右將軍,封定襄侯。



 桓帝崩後,操立碑於大邗城南,以頌功德,云「魏,軒轅之苗裔」。言桓、穆二帝「統國御眾,威禁大行,國無姦盜,路有頌聲。威武所向,下無交兵。招喻六狄,咸來歸誠。奉承晉皇,扞禦邊疆。王室多難,天網弛綱。豪心遠濟,靡離其殃。



 歲翦逆命,姦盜豺狼。永安元年,歲次甲子。姦黨猶逆,東西狼歭。敢逼天王,兵
 甲屢起。怙眾肆暴,虐用將士。鄴、洛構隙,棄親求疏。乃招異類,屠各、匈奴。



 交刃千里,長蛇塞塗。晉道應天,言展良謨。使持節、平北將軍、並州刺史、護匈奴中郎將、東嬴公司馬騰,才神絕世,規略超遠。欲求外救,朝臣莫應。簡賢選士,命茲良使。遣參軍壺倫、牙門中行嘉、義陽亭侯衛謨、協義亭侯衛健等,馳奉檄書,至晉陽城」。



 又稱桓、穆二帝,「心存宸極。輔相二衛,對揚毗翼。操展文謀,雄奮武烈。



 承命會議,諮論奮發。翼衛內外,鎮靜四方。志在竭力,奉戴天王。忠恕用暉,外動亦攘。功濟方州,勳烈光延。升平之日,納貢充籓。馮瞻鑾蓋,步趾三川。有德無祿,大命不
 延。年三十九,以永興三年六月二十四日寢疾薨殂。背棄華殿,雲中名都。國失惠主,哀感欷歔。悲痛煩冤,載呼載號。遠近親軌,奔赴梓廬。仰訴造化,痛延悲夫!」時晉光熙元年也。



 皇興初,雍州別駕雁門段榮於大邗掘得此碑,文雖非麗,事宜載焉,故略附於傳。操以穆帝三年卒。始操所與宗室鄉親入國者,衛勤安樂亭侯,衛崇、衛清並都亭侯,衛沈、段繁並信義將軍、都鄉侯,王發建武將軍、都亭侯,范班折衝將軍、廣武亭侯,賈慶建武將軍、上洛亭侯,賈循都亭侯,李壹關中侯,郭乳關內侯,皆為桓帝所表授也。六修之難,存者多隨劉琨任子遵南奔。



 衛
 雄、姬淡、莫含等名皆見碑。雄字世遠,淡字世雅,並勇健多計,桓帝並以為將,常隨征伐。雄稍遷至左將軍、雲中侯。淡亦以勇績著名,桓帝末,至信義將軍、樓煩侯。穆帝初,並見委任,衛操卒後,俱為左右輔相。六修之逆,國內大亂,雄、淡並為群情所附,乃與劉遵率烏丸、晉人數萬而叛。劉琨聞之,大悅,如平城撫納之,欲因以滅石勒。後為勒將孔長所滅。



 莫含,雁門繁畦人也。劉琨為并州,辟含從事。含居近塞下,常交通國中。穆帝愛其才器。及為代王,備置官屬,求含於琨,琨喻遣之。乃入參國官,常參軍國大謀。卒於左
 將軍、關中侯。其故宅在桑乾川南,世稱莫含壁,含音訛,或謂之莫回城云。



 子顯,昭成世為左常侍。



 顯子題,道武初,為大將,以功賜爵東宛侯。常與李栗侍宴,栗坐不敬獲罪,題亦被黜為濟陽太守。後道武欲廣宮室,規度平城四方數十里,將模鄴、洛、長安之制,運材數百萬根。以題機巧,徵令監之。召入,與論興造之制,題久侍頗怠,賜死。



 題弟雲,好學善射。道武時,常典選曹,賜爵安德侯。遷執金吾,參軍國謀議。



 太武克赫連昌,詔雲與常山王素留鎮統萬,進爵安定公。雲撫慰新舊,皆得其所。



 卒,謚敬公。



 劉庫仁字沒根,獨孤部人,劉武之宗也。少豪俠,有智略。母平文皇帝之女。



 昭成皇帝復以宗女妻之,為南部大人。建國三十九年,照成暴崩,道武未立,苻堅以庫仁為陵江將軍、關內侯。令與衛辰分國眾統之,河西屬衛辰,河東屬庫仁。於是獻明皇后攜道武及衛、秦二王自賀蘭部來居焉。庫仁盡忠奉事,不以興廢易節。



 苻堅處衛辰在庫仁下,衛辰怒,叛,攻庫仁。庫仁伐衛辰,破之。苻堅賜庫仁妻公孫氏,厚其資送。



 慕容垂圍苻丕於鄴,又遣將平規攻堅幽州刺史王永于薊。庫仁遣妻兄公孫希助永擊規,大破之。庫仁復將大舉以救丕。發雁門、上谷、
 代郡兵,次於繁畤。先是,慕容文等當徙長安,遁依庫仁部,常思東歸。是役也,文等夜率三郡人,攻殺庫仁,乘其駿馬,奔慕容垂。公孫希聞亂走丁零。



 庫仁弟眷,繼攝國事。眷第三子羅辰,機警有智謀,謂眷曰:「從兄顯,忍人也,願早圖之。」眷不以為意。後庫仁子顯果殺眷而代立。顯既殺眷,又謀逆。及道武即位,討顯於馬邑,追至彌澤,大破之。後奔慕容驎,驎徙之中山。



 羅辰即宣穆皇后兄也。顯既殺眷,羅辰遂奔道武。顯恃強,每謀逆,羅辰輒先聞奏。拜南部大人。從平中原,以功賜爵永安公。以軍功除征東將軍、定州刺史。



 卒,謚曰敬。



 子殊暉襲爵,位并州刺
 史,卒。子求引,位武衛將軍。卒,謚曰貞。子爾頭,位魏昌、癭陶二縣令,贈鉅鹿太守。



 子仁之,字山靜,少有操尚,粗涉書史。歷位衛將軍、西兗州刺史,在州有當時之譽。武定二年卒,贈衛大將軍、吏部尚書、青州刺史,謚曰敬。



 仁之外示長者,內多矯詐。其對賓客,破床弊席,粗飯冷菜,衣服故惡,乃過逼下。善候當塗,能為詭激。每於稠人廣眾中,或撾一姦吏,或縱一孤貧,大言自眩,淺識皆稱其美。公能之譽,動過其實。性又酷虐,在晉陽曾營城雉,仁之統監作役。以小稽緩,遂杖前殷州刺史裴瑗、并州刺史王綽。齊神武大加譴責。性好文字。吏書失體,便加鞭撻;
 言韻微訛,亦見捶楚,吏人以此苦之。而愛好文史,敬重人流。與齋帥馮元興交款。元興死後積年,仁之營視其家,常出隆厚,時人以此尚之。



 仁之伯乞歸,真君中,除中散大夫。性寬和,與物無競,未嘗言人善惡。曾遇患晝寢,有奴偷竊,乞歸詐睡不見,亦不泄之。此奴走入蠕蠕,方笑言之,亦無嗔色。獻文末,除主客尚書。孝文初,位東雍州刺史,賜爵永安侯。卒。



 子嵩,字阿龍,好周人之急。與王仲興自平城被追赴洛,家貧不能自達,嵩事事資遣。宣武時,仲興寵幸,乃奏除給事。請疏黃河,以通船漕,授龍門都將。歷年功不就,坐流。元曄僭立,授大鴻臚卿。子桃
 湯,位終奉朝請。



 尉古真,代人也。道武之在賀蘭部,賀染干遣侯引乙突等將肆逆。古真知之,密以馳告。染干疑古真泄其謀,乃執拷之,以兩車軸押其頭,傷其一目。不服,乃免之。後從平中原,以功賜爵束州侯。明元初,為鴻飛將軍,鎮大洛。卒於定州刺史。子億萬襲。



 古真弟諾,以忠謹著稱。從道武圍中山,先登,傷一目。道武歎曰:「諾兄弟並毀目以建功效,誠可嘉也!」賜安樂子。從平姚平,還,拜國部大人。太武時,改邑遼西公。卒,第八子歡襲。



 諾長子眷,忠謹有父風。明元時,執事左右,為太官令。時侍臣受斤亡入蠕蠕,
 詔眷追之。遂至虜庭,禽之大檀前。由是以驍烈聞。太武即位,命眷與散騎侍郎劉庫仁等八人分典四部,綰奏機要,加陳兵將軍。文成時,拜侍中、太尉,封漁陽王,與太宰常英等錄尚書事。文成北巡狩,以寒雪方降,議還。眷曰:「今去都不遠而旋,虜必疑我有內難。方寒雪,宜更進前。」帝遂度漠而還。帝以眷元老,賜杖履上殿。薨,謚曰莊。子多侯襲爵。



 多侯少有武幹。獻文時,假節、領護羌戎校尉、敦煌鎮將。至,求輕騎五千,西入于闐,兼平諸國,因敵取資,平定為效。弗許。孝文初,又求北取伊吾,斷蠕蠕通西域路。帝善其計,以東作方興,難之。為妻元所害。



 多侯
 弟子慶賓,善騎射,有將略,稍遷太中大夫。明帝時,朝議送蠕蠕主阿那瑰還國。慶賓上表固爭,不從。後蠕蠕遂執行臺元孚。慶賓後拜肆州刺史。時爾朱榮兵威漸盛,曾經肆州,慶賓惡之,據城不納。榮襲之,拘還秀容,呼為假父。後以憂還都。尋起為光祿大夫、都督,鎮汝陰。還朝,卒,贈司空。子瑾。



 瑾少而敏悟,好學慕善。以國姓門資,稍遷直後。司馬子如執政,瑾娶其甥皮氏為妻,由此除中書舍人。後除吏部郎中。齊文襄崩,文宣命瑾在鄴北宮,共高德正典機密。天保中,累遷七兵尚書侍郎。孝昭輔政,除吏部尚書。武成踐祚,趙彥深本子如賓寮,元文遙、和
 士開並帝鄉故舊,共相薦達,任遇彌重。又吏部銓衡所歸,事多秘密,由是朝之機事,頗亦預聞。後為尚書右僕射,卒。武成方在三臺饗宴,文遙奏聞,遂命撤樂罷飲。



 瑾外雖通顯,內關風訓;閨門穢雜,為世所鄙。有女在室,忽從奔誘,瑾遂以適婦姪皮逸人。瑾又通寡嫂元氏。瑾嘗譏吏部郎中頓丘李構云:「郎不稽古。」構對令史云:「我實不稽古,未知通嫂得作稽古不?」瑾聞大慚。然亦能折節下士,意在引接名流,但不之別也。有賈彥始者,儀望雖是儒生,稱堪充聘陳使。司徒戶曹祖崇儒,文辯俱不足,言將為當世莫及。好學吳人搖脣振足,為人所哂。見人
 好笑,時論比之寒蟬。又少威儀。子德載,以蒲鞭責之,便自投井。瑾自臨井上,呼云:「兒出!聞者皆笑。及位任重,便大躁急,省內郎中將論事者,逆即瞋罵。既居大選,彌自驕狠。皮子賤恃其親通,多所談薦,大有受納。瑾死後,其弟靜忿而發之。子賤坐決鞭二百,配北營州。



 初,瑾為聘梁使,梁人陳昭善相,謂瑾曰:「二十年後當為宰相。」瑾出,私謂人曰:「此公宰相後,不過三年,當死。」昭後為陳使主,兼散騎常侍,至齊。



 瑾時兼右僕射,鳴騶鐃吹。昭復謂人曰:「二年當死。」果如言焉。德載位通直散騎侍郎。



 眷弟地乾,機悟有才藝,馳馬立射五的,時人莫能及。太武時,位
 庫部尚書,加散騎常侍,領侍輦郎。奉上忠謹,尤善嘲笑。太武見其效人舉措,忻悅不能自勝。



 甚見親愛,參軍國大謀。時征平原,試衝車以攻冢。地乾為索所摐,折脅而卒。帝親往哭慟,贈中領軍、燕郡公,謚曰惠。



 子長壽,位右曹殿中尚書,賜爵會稽公,卒於涇州刺史。



 古真族玄孫聿,字成興,性耿介。明帝時,為武衛將軍。時領軍元叉執權,百寮莫不加敬,聿獨長揖不拜。尋出為水京州刺史。水京州緋色,天下之最,叉送白綾二千匹令染,聿拒不受。叉諷御史劾之,驛徵至京。覆,無狀。還任,卒。



 穆崇,代人也,其先代效節於神元、桓、穆之時。崇少以盜
 竊為事。道武之居獨孤部,崇往來奉給,時人無及者。後劉顯之逆,平文皇帝外孫梁眷知之,密遣崇告道武。眷謂崇曰:「顯若知之,雖刀劍刳割勿泄也。」因以寵妻及所乘良馬付崇曰:「事覺,吾當以此自明。」崇來告難,道武馳如賀蘭部。顯果疑眷泄,將囚之。



 崇乃唱言:「梁眷不顧恩義,將顯為逆。今我掠得其妻、馬,足以雪忿。」顯聞信之。窟咄之難,崇外甥于植等與崇謀執道武以應之。崇夜告道武,道武誅植等。北踰陰山,復幸賀蘭部。



 道武為魏王,崇從平中原,位侍中、豫州刺史、太尉、宜都公。天賜三年,薨。



 先是,衛王儀謀逆,崇預焉。道武惜其功,祕之。及有司
 奏謚,帝親覽謚法,述義不剋曰丁,曰:「此當矣。」乃謚丁公。



 初,道武避窟咄難,遣崇還察人心。崇留馬與從者,微服入其營。會有火光,為舂妾所識,賊皆驚起。崇求從者不得,因匿坑中,徐乃竊馬奔走。宿於大澤,有白狼向崇號,崇覺悟,馳隨狼奔,遂免難。道武異之,命崇立祀,子孫世奉焉。太和中,追錄功臣,以崇配饗。



 崇長子逐留,以功賜爵零陵侯。後以罪廢。



 子乙,以功賜爵富城公。卒於侍中,謚曰靜。



 子真,尚長城公主,拜駙馬都尉。後敕離婚,納文明后姊。位南部尚書、侍中、卒,謚曰宣。孝文追思崇勳,令著作郎韓顯宗與真撰定碑文,建於白登山。



 真子泰,
 本名石洛,孝文賜名焉。以功臣子孫,尚章武長公主,拜駙馬都尉,典羽獵四曹事。後為尚書右僕射、馮翊侯,出為定州刺史。初,文明太后幽孝文於別室,將謀黜廢,泰切諫乃止。孝文德之,故寵待隆至。自陳久病,乞為恒州,許之。



 泰不願遷都,潛圖叛,乃與定州刺史陸睿及安樂侯元隆等,謀推朔州刺史陽平王頤為主。頤密表其事,帝乃遣任城王澄發并、肆兵討之。澄先遣書侍御史李煥單騎入代,出其不意。泰等驚駭,計無所出。煥曉喻逆徒,示以禍福,於是凶黨離心,莫為之用。泰自度必敗,乃率麾下攻煥郭門,不克。走出,為人禽送。孝文幸代,泰等伏
 誅。



 子士儒,字叔賢,徙涼州。後得還,為太尉參軍事。



 子子容,少好學,無所不覽。求天下書,逢即寫錄,所得萬餘卷。魏末,為兼通直散騎常侍聘梁。齊受禪,卒於司農卿。



 逐留弟觀,字闥拔,襲崇爵。少以文藝知名。明元中,位為左衛將軍,綰門下、中書,出納詔命,及訪舊事,未嘗有遺漏。尚宜陽公主,拜駙馬都尉,位太尉。



 太武監國,觀為右弼,出則統攝朝政,入則應對左右,事無巨細,皆關決焉。



 終日怡怡,無慍色。勞謙善誘,不以富貴驕人。泰常八年,暴疾薨,年三十五。明元親臨其喪,悲動左右。賜以通身隱起金飾棺,喪禮一依安城王叔孫俊故事。贈宜都王,謚
 曰文成。太武即位,每與群臣談宴,未嘗不歎息殷勤,以為自道武以來,佐命勳臣文武兼濟無及之者。



 子壽襲爵,尚樂陵公主,拜駙馬都尉。明敏有父風。太武愛重之,擢為下大夫。



 敷奏機辯,有聲內外。遷侍中、中書監、領南部尚書,進爵宜都王,加征東大將軍。



 壽辭曰:「臣祖崇,先皇之世,屢逢艱危。幸天贊梁眷,誠心先告,故得效功前朝,流福於後。昔陳平受賞,歸功無知。今眷元勳未錄,臣獨奕世受榮,豈惟仰愧古賢,抑亦有虧國典。」太武嘉之,乃求眷孫,賜爵郡公。



 輿駕征涼州,命壽輔景穆,總錄機要,內外聽焉。次雲中,將濟河,帝別御靜室,召壽及司徒
 崔浩、尚書李順,謂壽曰:「蠕蠕吳提與牧犍連和,今聞朕征涼州,必來犯塞。若伏兵漠南,殄之為易。牧田訖,可分伏要害,以待虜至;引使深入,然後擊之。若違朕指授,為虜侵害,朕還斬卿。崔浩、李順為證,非虛言也。」壽信卜筮言,謂賊不來,竟不設備。吳提果至,京邑大駭。壽不知所為,欲築西郭門,請景穆避保南山,惠保太后不聽,乃止。遣司空長孫道生等擊之。太武還,以無大損傷,故不追咎。



 景穆監國,壽與崔浩等輔政。人皆敬浩,壽獨陵之。又自恃位任,以人莫己及。



 謂其子師曰:「但令吾兒及我,亦足勝人,不須苦教之。」遇諸父兄弟有如僕隸,夫妻並坐
 共食,而令諸父餕餘。為時人鄙笑。薨,贈太尉,謚曰文宣。



 子平國襲爵,尚城陽長公主,拜駙馬都尉、侍中、中書監,為太子四輔。卒。



 子伏乾襲,尚濟北公主,拜駙馬都尉。卒,謚曰康。無子。



 伏乾弟羆襲爵,尚新平長公主,拜駙馬都尉、武牢鎮將。頻以下法致罪,孝文以其勳德之舊,讓而赦之。轉吐京鎮將,深自剋勵。後改吐京鎮為汾州,仍以羆為刺史。前吐京太守劉升,在郡甚有威惠,限滿還都,胡人八百餘人詣羆請之。前定陽令吳平仁亦有恩信,戶增數倍。羆以吏人懷之,並為表請,孝文皆從焉。羆既頻薦升等,所部守令,咸自砥礪,威化大行。州人李軌、郭
 及祖七百餘人詣闕稱羆恩德。孝文以羆政和人悅,增秩延限。後徵為光祿勳,隨例降王為魏郡公。累遷侍中、中書監。穆泰之反,罷與潛通,赦後事發,削封為編戶。卒於家。宣武時,追贈鎮北將軍、恒州刺史。



 羆弟亮,字幼輔,早有風度。獻文時,起家侍御中散。尚中山長公主,拜駙馬都尉,封趙郡王。加侍中,徙封長樂王。



 孝文時,除征南大將軍、領護西戎校尉、仇池鎮將。宕昌王梁彌機死,子彌博立。為吐谷渾所逼,來奔仇池。亮以彌博兇悖,氐羌所棄;彌機兄子彌承,戎人歸樂,表請納之。孝文從焉。於是擊走吐谷渾,立彌承而還。氐豪楊卜自延興以來,從
 軍二十一戰,前來鎮將,抑而不聞。亮表卜為廣業太守,豪右咸悅,境內大安。



 徵為侍中、尚書左僕射。于時復置司州。孝文曰:「司州始立,未有寮吏,須立中正,以定選舉。然中正之任,必須德望兼資。世祖時,崔浩為冀州中正,長孫嵩為司州中正,可謂得人。公卿等宜審推舉。」尚書陸睿舉亮為司州大中正。後拜司空,參議律令。例降爵為公。



 時文明太后崩,已過期月,孝文毀瘠猶甚。亮表請上承金冊遺訓,下稱億兆之心。時襲輕服,數御常膳;修崇郊祠,垂惠咸秩。詔曰:「茍孝悌之至,無所不通。



 今飄風亢旱,時雨不降,實由誠慕未濃,幽顯無感也。」尋領太子
 太傅。時將建太極殿,帝引見群臣於太華殿,曰:「將營殿宇,今欲徙居永樂,以避囂埃。土木雖復無心,毀之能不悽愴!今故臨對卿等,與之取別。此殿乃高宗所制,爰歷顯祖,逮朕沖年,受位於此。但事來奪情,將有改制。仰惟疇昔,唯深悲感。」亮稽首請稽之卜筮。又以去歲役作,為功甚多,太廟、明堂,一年便就。若仍歲頻興,恐人力彫弊。且材幹新伐,願待餘年。帝曰:「朕遠覽前王,無不興造。故有周創業,經建靈臺;洪漢受命,未央是作。草創之初,猶尚若此;況朕承累聖之運,屬太平之基,欲及此時,以就大功。人生定分,修短命也;蓍蔡雖智,其如命何!當委之
 分,豈假卜筮。」移御永樂宮。



 後帝臨朝堂,嘗謂亮曰:「三代之禮,日出視朝。自漢、魏以降,禮儀漸殺。



 《晉令》有朔望集公卿於朝堂而論政事,亦無天子親臨之文。今因卿等日中之集,中前,卿等自論政事;中後,與卿等共議可否。」遂命讀奏案,帝親決之。



 及遷都,加武衛大將軍,以本官董攝中軍事。帝南伐,以亮錄尚書事,留鎮洛陽。後帝自小平津水凡舟幸石濟。亮諫曰:「漢帝欲乘舟渡渭,廣德將以首血污車輪,帝乃感而就橋。渭之小水,猶尚若斯,況洪河有不測之慮。」帝曰:「司空言是也。」及羆預穆泰反事覺,亮上表自劾,帝優詔還令攝事。亮固請,久乃許之。



 後徙
 封頓丘郡公,以紹崇爵。



 宣武即位,拜尚書令、司空公。薨,宣武親臨小斂,贈太尉,謚曰匡。



 子紹,字永業,尚瑯邪長公主,拜駙馬都尉。歷位秘書監、侍中、衛將軍、太常卿、中書令、七兵殿中二尚書。遭所生憂,免,居喪以孝聞。又歷衛大將軍、中書監、侍中,領本邑中正。



 紹無他才能,而資性方重,罕接賓客,稀造人門。領軍元叉當權薰灼,曾往紹宅,紹迎送下階而已。時人歎尚之。及靈太后欲黜叉,猶豫未決,紹贊成之。以功加特進、侍中。元順與紹同直,嘗因醉入寢所。紹擁被而起,正色讓順曰:「老身二十年侍中,與卿先君亟連職事,縱卿後進,何宜相排突也!」遂
 謝事還家,詔喻乃起。除侍中,託疾未起,故免河陰之害。



 莊帝立,爾朱榮徵之。紹以為必死,哭辭家廟。及見榮,捧手不拜。榮亦矯意禮之,顧謂人曰:「穆紹不虛作大家兒。」車駕入宮,尋授尚書令、司空,進爵為王,給班劍四十人,仍加侍中。時河南尹李獎往詣紹。獎以紹郡人,謂必致敬。紹又恃封邑是獎國主,匡坐待之,不為動膝。獎憚其位望,致拜而還。議者兩譏焉。



 未幾,降王,復本爵。



 普泰元年,除驃騎大將軍、開府、青州刺史,加都督。未行而薨,贈大將軍、尚書令、太保,謚曰文獻。



 子長嵩,字子岳,襲爵,位光祿少卿。



 平國弟正國,尚長樂公主,拜駙馬都尉。



 正國
 子平城,早卒。孝文時,始平公主薨於宮,追贈平城駙馬都尉,與公主冥婚。



 壽弟多侯,封長寧子,位司衛監。文成崩,乙渾專權,召司徒陸麗。麗時在溫湯療疾,多侯謂曰:「渾有無君心。大王,眾所望也,去必危。宜徐歸而圖之。」



 麗不從,遂為渾害。多侯亦見殺。



 觀弟翰,平原鎮將、西海王。薨。



 子龍兒襲爵,降為公。卒。



 子弼,有風格,善自位置,涉獵經史,與長孫承業、陸希道等齊名。然而矜己陵物,頗以此損焉。孝文定氏族,欲以弼為國子助教,弼辭以為屈。帝曰:「朕欲敦勵胄子,屈卿先之。白玉投泥,豈能相污!」弼曰:「既遇明時,恥沈泥滓。」



 會司州牧咸陽王禧入。帝曰:「朕
 與卿作州督,舉一主簿。」即命弼謁之。因為帝所知。宣武初,為廣平王懷國郎中令,數有匡諫之益。除中書舍人,卒於華州刺史,謚曰懿。



 翰弟顗,有才力。以侍御郎從太武征赫連昌。勇冠一時,賜爵泥陽子,拜司衛監。從太武由崞山,有虎突出,顗搏而獲之。帝歎曰:「《詩》云:『有力如虎』,顗乃過之!」後從征白龍,討蠕蠕,以功進爵建安公。後拜殿中尚書,出鎮涼州。



 還,加散騎常侍、領太倉尚書。文成時,為征西大將軍,督諸軍西征吐谷渾。坐擊賊不進,免官爵,徙邊。文成以顗著勳前朝,徵為內都大官。卒,贈征西大將軍、建安王,謚曰康。子寄生襲。



 崇宗人醜善,道武
 初,率部歸附,與崇同心戮力,捍禦左右。拜天部大人,居東蕃。



 子莫提,從平中原,位相州刺史、假陵陽侯。其子孫位亦通顯。



 奚斤,代人也,世典馬牧。父簞。有寵於昭成皇帝。時國有良馬曰騶駠,一夜忽逸。後知南部大人劉庫仁所盜,養於窟室。簞聞而馳往取馬,庫仁以國甥恃寵,慚而逆擊簞,簞捽其髮落,傷其一乳。及符堅使庫仁與衛辰分領國部,簞懼,遂奔衛辰。及道武滅衛辰,簞晚乃得歸,故名位後於舊臣。



 斤機辯有識度。登國初,與長孫肥等俱統禁兵。後以為侍郎,親近左右。從征慕容寶於參合。皇始
 初,拜越騎校尉,典宿衛禁旅。車駕還京師,博陵、勃海、章武諸郡群盜並起,斤與略陽公元遵等討平之。從征,破高車諸部。又破庫狄、宥連部,徙其別部諸落於塞南。又進擊侯莫陳部,至大峨谷,置戍而還。遷都水使者,出為晉兵將軍、幽州刺史,賜爵山陽侯。



 明元即位,為鄭兵將軍。詔以斤世忠孝,贈其父簞長寧子。明元幸雲中,斤留守京師。昌黎王慕容伯兒謀反,斤召入天安殿東廡下,誅之。詔與南平公長孫嵩等俱坐朝堂,錄決囚徒。明元大閱于東郊。講武,以斤行左丞相,大蒐於石會山。車駕西巡,詔斤先驅,討越勒部於鹿那山,大破之。又詔斤與
 長孫嵩等八人坐止車門左,聽理萬機。拜天部大人,進爵為公。命斤出入乘軺軒,備威儀導從。



 太武之為皇太子,臨朝聽政,以斤為左輔。宋廢主義符立,其國內離阻。乃遣斤收河南地,假斤節,都督前鋒諸軍事、司空、晉兵大將軍、行揚州刺史。率吳兵將軍公孫表等南征。用表計攻滑臺,不拔,求濟師。帝怒其不先略地,切責之。乃親南巡,次中山,斤自滑臺趣洛陽,長驅至武牢,遂平兗、豫諸郡。還圍武牢。及武牢潰,斤置守宰以撫之。自魏初大將行兵,唯長孫嵩拒宋武,斤征河南,獨給漏刻及十二牙旗。



 太武即位,進爵宜城王,仍為司空。太武征赫連昌,
 遣斤率義兵將軍封禮等襲蒲阪。斤又西據長城,秦、雍氐羌皆來歸附。斤與赫連定相持,累戰破定。定聞昌敗,走上邽。斤追至雍,不及而還。詔斤班師,斤請因其危平之,乃進討安定。昌退保平涼,斤屯軍安定,以糧竭馬死,深壘自固。監軍侍御史安頡擊昌,禽之。昌眾復立昌弟定為主,守平涼。斤恥以元帥而禽昌之功更不在己,乃舍輜重,追定於平涼。定眾將出,會一小將有罪,亡入賊,具告其實。定知斤軍無糧乏水,乃邀斤前後。斤眾大潰,斤及將娥清、劉拔為定所禽。後太武剋平涼,斤等得歸。免為宰人,使負酒食從駕還京師以辱之。尋拜安東將
 軍,降爵為公。太延初,為衛尉,改為恆農王。後為萬騎大將軍。太武議伐涼州,斤等三十餘人議以為不可,帝不從。



 涼州平,以戰功賜僮隸七十戶。又以斤元老,賜安車,平決獄訟,諮訪朝政。



 斤聰辯彊識,善於談論,遠說先朝故事,雖未皆是,時有所得,聽者歎美之。



 每議大政,多見從用,朝廷稱焉。真君九年,薨,時年八十九。太武親臨哀慟,謚曰昭王。斤有數十婦,子男二十餘人。



 長子他觀襲爵。太武曰:「斤西征之敗,國有常刑。以其佐命先朝,故復其爵秩,將收孟明之效。今斤終其天年,君臣之分全矣。」於是降他觀爵為公。傳國至孫緒,無子,國除。太和中,孝
 文追錄先朝功臣,以斤配饗廟庭。宣武繼絕世,以緒弟子監紹其後。



 叔孫建,代人也。父骨,為昭成母王太后所養,與皇子同列。建少以智通著稱。



 道武之幸賀蘭部,常從左右。登國初,為外朝大人,與安同等十三人迭典庶事,參軍國之謀。隨秦王觚使慕容垂,歷六載乃還。累遷中領軍,賜爵安平公,出為并州刺史。後以公事免,守鄴城園。



 明元即位,念前功,以為正直將軍、相州刺史。飢胡劉武等聚黨叛。明元假建前號、安平公,督公孫表等以討武。斬首萬餘級,餘眾奔走,投沚水死,水為不流。



 晉將劉裕伐姚泓,
 令其部將王仲德為前鋒,將逼滑臺。兗州刺史尉建率所部棄城濟河。仲德遂入滑臺,乃宣言曰:「晉本意欲以布帛七萬匹假道於魏,不謂魏之守將便爾棄城。」明元聞之,詔建度河曜威,斬尉建,投其屍於河。呼仲德軍人與語,詰其侵境之狀。



 尋遷廣阿鎮將,威名甚著。久之,除使持節、都督前鋒諸軍事、楚兵將軍、徐州刺史。率眾自平原濟河,徇下青、兗諸郡。遂東入青州,圍宋刺史竺夔於東陽城。



 宋遣將檀道濟、王仲德救夔,建不克而還。以功賜爵壽光公。與汝陰公長孫道生濟河而南,仲德等自清入濟,東走青州。太武以建威名南震,為宋所憚,除
 平原鎮大將,封丹楊王,加征南大將軍。



 先是,簡幽、易以南戍兵集于河上,一道討洛陽,一道攻滑臺。宋將檀道濟、王仲德救滑臺。建與汝陰公道生拒擊之。建分軍挾戰,縱輕騎邀其前後,焚燒穀草以絕其糧道。道濟兵飢,叛者相繼。由是安頡等得拔滑臺。



 建沈敏多智,東西征伐,常為謀主,容貌清整,號曰嚴明。又雅尚人倫,禮賢愛士。在平原十餘年,綏懷內外,甚得邊稱。魏初名將,鮮有及之。南方憚其威略,青、兗輟不為寇。太延三年,薨,時年七十三,謚曰襄,賜葬金陵。



 長子俊,字醜歸,少聰敏。年十五,內侍左右,性謹密,初無過行。以便弓馬,轉為獵郎。道
 武崩,清河王紹閉宮門,明元在外。紹拘逼俊以為己援。外雖從紹,內實忠款,仍與元磨渾等說紹歸明元。時明元左右唯車路頭、王洛兒等,及得俊等,大悅,以為爪牙。及即位,稍遷衛將軍,賜爵安成公。及朱提王悅懷刃入禁,欲行大逆。俊覺悅舉動有異,乃於悅懷中得兩刃匕首,遂執悅殺之。明元以俊前後功重,軍國大計一以委之。群官上事,先由俊銓校,然後奏聞。



 性平正柔和,未嘗有喜怒色。忠篤愛厚,不諂上抑下。每奉詔宣外,必告示殷勤。是以上下嘉嘆。泰常元年,卒,時年二十八。明元親臨哀慟,朝野無不追惜。



 贈司空、安成王,謚孝元,賜溫
 明秘器,載以巉輬車,衛士導從,陪葬金陵。子蒲襲爵。後有大功及寵幸貴臣薨,賵賻送終禮皆依俊故事,無得踰之者。



 初,俊卒,明元命其妻桓氏曰:「夫生既共榮,沒宜同穴,能殉葬者,可任意。」



 桓氏乃縊,遂合葬焉。



 俊既為安城公,俊弟鄰襲父爵,降為丹楊公,位尚書令、涼州鎮大將。與鎮副將奚牧,並以貴戚子弟,競貪財貨,遂相糾,坐誅。



 安同,遼東胡人也。其先祖曰世高,漢時以安息王侍子入洛。歷魏至晉,避亂遼東,遂家焉。父屈,仕慕容。為苻堅所滅,屈友人公孫眷妹沒入苻氏宮,出賜劉庫仁
 為妻,庫仁貴寵之。同隨眷商販,見道武有濟世才,遂留奉侍。性端嚴明惠,好長者之言。登國初,道武徵兵於慕容垂,同頻使稱旨。為外朝大人,與和跋等出入禁中,迭典庶事。



 從征姚平於柴壁,姚興悉眾救平,同進計曰:「汾東有蒙坑,東西三百餘里,徑路不通。姚興來,必從汾西,乘高臨下,直至柴壁,如此則寇內外勢接。宜截汾為南北浮橋,乘西岸築圍。西圍既固,賊至無所施其智力矣。」從之。興果視平屠滅而不能救。以謀功,賜爵北新侯。



 明元即位,命同與南平公長孫嵩並理人訟。又詔同與肥如侯賀護持節循察并、定二州及諸山居雜胡、丁零。宣
 詔撫慰,問其疾苦,糾舉守宰不法,郡國肅然。同東出井陘。至鉅鹿,發眾欲修大嶺山,通天門關。又築城於宋子,以鎮靜郡縣。護嫉同得眾心,使人告同築城聚眾,欲圖大事。



 太武監國,臨朝聽政,以同為左輔。及即位,進爵高陽公,冀青二州刺史。同長子屈,明元時,典太倉事,盜官粳米數石,欲以養親。同大怒,求戮屈,自劾不能訓子。帝嘉而恕之,遂詔長給米。同在官明察,長於校閱,家法修整,為世所稱。



 及在冀州,年老,頗殖財貨,大興寺塔,為百姓所苦。卒,贈高陽王,謚曰恭惠。



 屈弟原,雅性矜嚴,沈勇多智略。明元時,為獵郎,出監雲中軍事。時赫連屈丐犯
 河西,原以數十騎擊之,殺十餘人。帝以原輕敵,違節度,加罪。然知原驍勇,遂任以為將,鎮雲中。蠕蠕犯塞,原輒破之,以功賜爵武原侯,加魯兵將軍。



 太武即位,拜駕部尚書。車駕征蠕蠕大檀,分為五道。遷尚書左僕射,進爵河間公。原在朝無所比周,然恃寵驕恣,多所排抑。為子求襄城公盧魯元女,魯元不許。原告其罪狀,事相連逮,歷時不決。原懼不勝,遂謀逆,事泄,伏誅。原兄弟外節儉而內實積聚,及誅後,籍其財至數萬。



 弟頡,辯慧多策略,最有父風。明元初,為內侍長,令察舉百寮,糾刺姦慝,無所迴避。嘗告其父陰事,帝以為忠,特親寵之。



 宜城王奚
 斤自長安追赫連昌至安定,頡為監軍侍御史。斤以馬多疫死,士眾乏糧,乃築壘自固。遣太僕丘堆等督租於人間,為昌所敗。昌遂驕矜,日來侵掠。頡曰:「等死,當戰死!寧可坐受囚乎?」斤猶以馬死為辭。頡乃陰與尉眷等謀,選騎焉。昌來攻壘,頡出應之,昌馬蹶而墜。頡禽昌送京師,賜爵西平公,代堆統攝諸軍。



 赫連定將復入長安,詔頡鎮蒲阪以拒之。宋將到彥之寇河南以援定,列守南岸,至於衡關。太武西征定,以頡為冠軍將軍,督諸軍擊彥之。遂濟河,攻洛陽,拔之。



 進攻武牢,武牢潰。又與瑯邪王司馬楚之平滑臺,禽宋將朱修之、李元德及東郡太守
 申謨。乃振旅還京師,進爵為王。卒,謚曰襄。頡為將善綏士眾,及卒,宋士卒降者無不歎惜。



 庾業延,代人也,後賜名岳。其父及兄和辰世典畜牧,稍轉中部大人。昭成崩,苻氏內侮,事難之間,收斂畜產,富擬國君。劉顯謀逆,道武外幸,和辰奉獻明太后歸道武,又得其資用。以和辰為內侍長。和辰分別公私舊畜,頗不會旨,道武由是恨之。岳獨恭慎修謹,善處危難之間,道武嘉之。與王建等俱為外朝大人,參預軍國。從平中原,拜安遠將軍。



 官軍之警於柏肆也,賀蘭部帥附力眷、紇突鄰部帥匿物尼、紇奚部帥叱奴根等聞之,反於陰
 館。南安公元順討之,不剋,詔岳。討破離石叛胡帥呼延鐵、西河叛胡帥張崇等。以功賜爵西昌公,遷鄴行臺。岳為將有謀略,士眾服其智勇,名冠諸將。及罷鄴行臺,以所統六郡置相州,即拜岳為刺史。秉法平當,百姓稱之。鄴舊有圓池,時果初熟,丞吏送之,岳不受,曰:「果未進御,吾何得先食!」其謹如此。遷司空。



 岳兄子路,有罪,諸父兄弟悉誅,特赦岳父子。候官告岳衣服鮮麗,行止風采擬儀人君。遇道武不豫,多所猜忌,遂誅之。時人咸冤惜焉。



 岳葬在代西善無界。後太武征赫連氏,經其墓宅,愴然改容,遂下詔為立廟,令一川之人,四時致祭。求其子孫
 任為帥者,得其子陵。從征有功,聽襲爵。



 王建,廣寧人也。祖姑為平文后,生昭成皇帝。伯祖豐,以帝舅貴重。豐子支,尚昭成女,甚見親待。建少尚公主。登國初,為外朝大人。與和跋等十三人迭典庶事,參與計謀。道武遣使慕容垂,建辭色高亢,垂壯之。還為左大夫。建兄迴,時為大夫,諸子多不慎法,建具以狀聞,回父子伏誅。其訐直如此。



 從征伐諸國,破二十餘部。又從征衛辰,破之。為中部大人。破慕容寶於參合,帝乘勝將席卷南夏。於是簡擇俘眾,有才能者留之;其餘欲悉給衣糧遣歸,令中州之人咸知恩德。建以為寶覆敗於此,國內
 空虛,獲而歸之,縱敵生患,不如殺之。



 帝曰:「若從建言,非伐罪弔人之義。」諸將咸以建言為然。建又固執,乃坑之。



 帝既而悔焉。



 並州既平,車駕出井陘,次常山。諸郡皆降,唯中山、鄴、信都三城不下。乃遣衛王儀南攻鄴,建攻信都等城。建等攻城六十餘日,不能剋,士卒多傷。帝自中山幸信都,降之。車駕幸鉅鹿,破寶眾於柏肆塢,遂圍中山。寶棄城走和龍,城內無主,將夜入乘勝據守其門。建貪而無謀,意在虜獲。恐士卒肆掠,盜亂府庫,請候天明,帝乃止。是夜,徒何人共立慕容普驎為主,遂閉門固守。帝乃悉眾攻之,使人登巢車臨城,招其眾。皆曰:「但恐如
 參合之眾,故求全月日命耳。」帝聞之,顧視建而唾其面。



 中山平,賜建爵濮陽公。遷太僕,徙真定公,加散騎常侍、冀青二州刺史。卒,陪葬金陵。



 羅結,代人也。其先世領部落,為魏附臣。劉顯之逆,結從道武幸賀蘭部。後賜爵屈蛇侯。太武初,累遷侍中、外都大官,總三十六曹事。年一百七歲,精爽不衰。太武以其忠愨,甚信待之。監典後宮,出入臥內,因除長秋卿。年一百一十,詔聽歸老。賜大寧東川為私第別業。並為築城,即號曰羅侯城。朝廷每有大事,驛馬詢問焉。年一百二十,卒,謚曰貞。



 子斤,從太武討赫連昌,力戰有功,歷位四
 部尚書。從平涼州,以功賜爵帶方公,除長安鎮都大將。會蠕蠕侵境,除柔玄鎮都大將。卒,謚曰靜,陪葬金陵。



 子敢襲爵,位庫部尚書。卒,子伊利襲。



 婁伏連,代人也。代為酋帥。伏連忠厚有器量,年十三,襲父位,領部落。道武初,從破賀蘭部,又平中山。及征姚平於柴壁,以功賜爵安邑侯。明元時,為晉兵將軍、并州刺史。太武即位,封廣陵公,再遷光祿勳,進爵為王。後鎮統萬。薨,謚恭王。



 子真襲,降爵為公。



 真弟大拔,封鉅鹿子。



 大拔孫寶,字道成,性淳朴,好讀書。明帝時,仕至朔州刺史。時邊事屢興,人多流散。及寶至,稍安集之。殘壞舊宅,皆
 命葺構;人歸繼路,歲考為天下最。



 後隨大都督源子邕討擊葛榮。王師敗績。寶囚於榮軍,變姓名,匿於戎伍,以免害。久之。賊中有朔州人識寶者,謂寶曰:「使君寧自苦至此?」遂將詣榮。笑曰:「婁公,吾方圖事,何相見之晚!」因顧謂人曰:「此公行善,天道報之,得免亂兵,即其驗也。」寶遇逃者,密啟賊形勢,規為內應。天子感其壯志,召寶第二子景賢,授員外散騎常侍郎。葛榮滅,寶始得還。



 永安中,除假員外散騎常侍,使蠕蠕。先是,蠕蠕稱籓上表,後以中州不競,書為敵國之儀。寶責之。蠕蠕主大驚,自知惡,謝曰:「此作書人誤。」遂更稱籓。



 孝武帝立,敕寶與行臺長
 孫子彥鎮恆農。後從入關,封廣寧縣伯。大統元年,詔領著作郎,監修國史事,別封平城縣子。後授國子祭酒、侍中,進儀同三司,兼太子太傅,攝東宮詹事。寶為人清簡少言,頗諳舊事。位歷師傅,守靖謙恭,以此為人所敬。後行涇州事,卒於州。



 閭大肥,蠕蠕人也。道武時歸魏,尚華陽公主,賜爵其思子。與弟並為上賓,入八議。明元即位,為內都大官,進爵為侯。宜城王奚斤之攻武牢,大肥與娥清領十二軍出中道。太武初,復與奚斤出雲中白道討大檀,破之。後從討赫連昌,以功授榮陽公。公主薨,復尚護澤公主。太武
 將拜大肥為王,遇疾卒。



 奚牧,代人也。重厚有智謀,道武寵遇之,稱曰仲兄。初,劉顯害帝,梁眷知之,潛使牧與穆崇至七個山以告。帝錄先帝舊臣,又以牧告顯功,使敷奏政事,參與計謀。從征慕容寶,以功拜并州刺史,賜爵任城公。州與姚興接界,興頗寇邊。



 牧乃與興書,稱頓首,均禮抗之,責興侵邊不直之意。興以與國和通,恨之,有言於道武,道武戮之。



 和跋,代人也。世領部落,為魏附臣。至跋,以才辯知名。道武擢為外朝大人,參軍國大謀,雅有智算,賜爵日南公。從平中原,以功進為尚書,鎮鄴。以破慕容德軍,改封定
 陵公。與常山王遵討賀蘭部別帥木易千,破之。出為平原太守。道武寵跋於諸將。群臣皆敦尚恭儉,而跋好修虛譽,炫曜於時。性尤奢淫,帝戒之不革。



 後車駕北狩豺山,收跋,刑之路側。妻劉氏自殺以從。初將刑跋,道武命其諸弟毗等視訣。跋謂毗曰:「水壘北地瘠,可居水南,就耕良田,廣為產業,各相勸勵。」



 令之背己,曰:「汝曹何忍視吾之死!」毗等解其微意,詐稱使者,奔長安。道武誅其家。



 後太武幸豺山校獵,忽暴霧四塞,怪問之。群下僉言跋世居此,祠塚猶存,或者能致斯變。帝遣建興公古弼祭以三牲,霧即除。後太武蒐狩之日,每先遣祭之。



 莫題,代人也。多智,有才用。初為幢將,領禁兵。道武之征慕容寶,寶夜犯營,軍人驚駭。遂有亡還京師者,言官軍敗於柏肆。京師不安,南安公元順因欲攝國事。題曰:「大事不可輕爾,不然,禍將及矣!」順乃止。後封高邑公。窟咄寇南鄙,題時貳於帝,遺箭於窟咄,謂之曰:「三歲犢豈勝重載!」言窟咄長而帝少也。帝既銜之,後有告題居處倨傲,擬則人主。帝乃使人示之箭,告之曰:「三歲犢能勝重載不?」題奉昭,父子對泣。詰朝,乃刑之。



 賀狄干,代人也。家本小族,世忠厚,為將以平當稱。稍遷北部大人。登國初,與長孫嵩為對。明於聽察,為人愛敬。
 道武遣狄乾致馬千匹,結婚於姚萇。會萇死,興立,因止狄乾而絕婚。興弟平寇平陽,道武討平之,禽其將狄伯支、唐小方等四十餘人。後興以駿馬千匹贖伯支,而遣狄干還,帝許之。乾在長安,因習讀書史,通《論語》、《尚書》諸經,舉止風流,有似儒者。初,帝普封功臣,狄乾雖為姚興所留,遙賜狄干爵襄武侯,加秦兵將軍。及狄干至,帝見其言語衣服類中國,以為慕而習之,故忿焉,既而殺之。



 李栗,鴈門人也。昭成時,父祖入北。栗少辯捷,有才能兼將略。初隨道武幸賀蘭部,愛其藝能。時王業草創,爪牙心腹,多任親近,唯栗一介遠寄,兼非戚舊。



 數有戰功,拜
 左軍將軍。栗性簡慢,矜寵,不率禮度。每在道武前舒放倨傲,不自祗肅,笑唾任情。道武即其宿過誅之。於是威嚴始厲,制勒群下盡卑謙之禮,自栗始也。



 奚眷,代人也。少有將略。道武世,有戰功。明元時,為武牢鎮將,為寇所憚。



 太武時,賜爵南陽公。及征蠕蠕,眷以都曹尚書督偏將出別道。詔會鹿渾海,眷與中山王辰等諸大將俱後期,斬于都南,爵除。



 論曰:帝王之興,雖則天命,經綸所說,咸藉股肱。神元、桓、穆之際,王迹未顯,操、含託身馳驟之秋,自立功名之地,可謂志識之士矣。而劉庫仁兄弟忠以為心,盛衰不二。純
 節所存,其意蓋遠,而並貽非命,惜乎!尉真兄弟忠勇奮發,義以忘生。眷威略著時,增隆家業。穆崇夙奉龍顏,早著誠款,遂膺寵眷,位極台司。至乃身豫逆謀,卒蒙全護,從享於廟,抑亦尚功。世載公卿,弈弈青紫,盛矣!



 奚斤世稱忠孝,征伐有剋。平涼之役,師殲身虜,雖敗崤之責已赦,封尸之效靡立,而恩禮隆渥,沒祀廟廷。叔孫建少展誠勤,終著庸伐,臨邊有術,威震夷楚。俊委節明元,義彰顛沛,察朱提之變,有日磾之風,加以柔而能正,見美朝野。安同異類之人,智識入用,任等時俊,當有由哉!頡禽赫連昌,摧宋氏眾,遂為名將,未易輕也。庾業延見紀危
 難之中,受事草創之際,智勇既申,功名尤舉,而不免傾覆,蓋亦其命。王建位遇既高,訐以求直,參合之役,不其罪歟!羅結枝附葉從,子孫榮祿。婁伏連、閭大肥並徵代著績,策名前代。奚牧、和跋、莫題、賀狄干、李慄、奚眷有忠勤征伐之效,不能以功名自卑,俱至誅夷,亦各其命也。



\end{pinyinscope}