\article{卷四十五列傳第三十三}

\begin{pinyinscope}

 裴
 叔業夏侯道遷李元護席法友王世弼江悅之淳于誕沈文秀張讜李苗劉藻傅永傅豎眼張烈李叔彪路恃慶房亮曹世表潘永基朱元旭裴叔業,河東聞喜人。魏冀州刺史徽之後也。五世祖苞,晉秦州刺史。祖邕,自河東居于襄陽。父順宗,兄叔寶,仕
 宋、齊,並有名位。叔業少有氣幹,頗以將略自許。宋元徽末,歷官為羽林監、齊高帝驃騎行參軍。齊受命,累遷為寧蠻長史、廣平太守。叔業早與齊明帝同事。明帝輔政,以為心腹,使領軍奄襲諸蕃鎮,盡心用命。及即位,以為給事黃門侍郎,封武昌縣伯。孝文南次鐘離,拜叔業為徐州刺史,以水軍入淮。帝令郎中裴聿往與之語,叔業盛飾左右服玩以夸之。聿曰:「伯父儀服誠為美麗,但恨不晝游耳。」



 齊帝崩,廢帝即位。誅大臣,都下屢有變發。叔業登壽春城,北望肥水,謂部下曰:「卿等欲富貴乎?我言富貴亦可辦耳。」未幾,見徙南袞州刺史。會陳顯達圍建
 鄴,叔業遣司馬李元護應之,及顯達敗而還。叔業慮內難未已,不願為南兗州。



 齊廢主嬖臣茹法珍、王咺之等疑其有異,去來者並云叔業北入。叔業兄子植、IR、瑜、粲等棄母奔壽陽。法珍等以其既在疆場,且欲羈縻之,白齊主,遣中書舍人裴穆慰誘之,許不須回換。叔業雖得停,而憂懼不已。時梁武帝為雍州刺史,叔業遣親人馬文範以自安之計訪之梁武帝,曰:「雍州若能堅據襄陽,輒當戮力自保。若不爾,回面向北,不失河南公。」梁武報曰:「唯應送家還都以安慰之,自然無患。



 若意外相逼,當勒馬二萬人,直出橫江,以斷其後,則天下事一舉可定。
 若欲北向,彼必遣人相代,以河北一地相處,河南公寧復可得?如此,則南歸望絕矣。」叔業沉疑未決,遣信詣豫州刺史薛真度,訪入北之宜。真度答書,盛陳朝廷風化。叔業乃遣子芬之及兄女夫韋伯昕奉表內附。



 景明元年正月,宣武詔授叔業持節、散騎常侍、都督、豫州刺史、征南將軍,封蘭陵郡公。又賜叔業璽書,遣彭城王勰、尚書令王肅赴接。軍未度淮,叔業病卒,李元護、席法友等推叔業兄子植監州事。詔贈叔業驃騎大將軍、開府儀同三司,謚忠武公,給東園溫明祕器。



 子蒨之,字文德,仕齊,隨郡王左常侍,先卒。



 子譚紹封。譚粗險好殺,所乘
 牛馬為小驚逸,手自殺之。然孝事諸叔,盡於子道,國祿歲入,每以分贍,世以此稱之。位輔國將軍、中散大夫。卒,贈南豫州刺史,謚曰敬。



 子測,字伯源,襲。歷通直散騎侍郎,天平中,走於關中。



 蒨之弟芬之,字文馥,長者好施,篤愛諸弟。仕齊,位羽林監。入魏,以父勛封上蔡伯。為東秦州刺史,在州有清靜稱。後徙封山茌縣,遷岐州刺史。為隴賊所圍,城陷,賊以送上邽,為莫折念生所害,贈青州刺史。



 芬之弟藹之,字幼重,性輕率,好琴書。其內弟柳諧善鼓琴,藹之師而微不及也。位汝陽太守。



 叔業長兄子彥先,少有志尚。叔業以壽春入魏,彥先封雍丘縣子,位
 勃海相。



 卒,謚曰惠恭。



 彥先子約,字元儉,性頗剛鯁,後襲爵。冀州大乘賊起,敕為別將,行勃海郡事,城陷見害。



 長子英起,武定末洛州刺史。英起弟威起,卒於齊王府中兵參軍,贈鴻臚少卿。



 彥先弟絢,揚州中從事。時揚州霖雨,水入城,刺史李崇居城上,繫船憑焉。



 絢率城南人數千家水凡舟南走高原。謂崇還北,遂與別駕鄭祖起等送子十四人於梁。



 崇勒水軍討之,眾潰見獲,投水而死。



 植字文遠,叔業兄叔寶子也。少而好學,覽綜經史,尤長釋典,善談理義。隨叔業在壽春。叔業卒,席法友、柳玄達等共舉植監州。祕叔業喪問,教命處分,皆出於植。於是開
 門納魏軍。詔以植為袞州刺史、崇義縣侯,入為大鴻臚卿。後以長子昕南叛,有司處之大辟,詔特恕其罪,以表勛誠。尋除授揚州大中正,出為瀛州刺史,再遷度支尚書,加金紫光祿大夫。



 植性非柱石,所為無恒。袞州之還也,表請解官,隱於嵩山,宣武不許,深以為怪。然公私集論,自言人門不後王肅,怪朝廷處之不高。及為尚書,志意頗滿,欲以政事為己任,謂人曰:「非我須尚書,尚書亦須我。」辭氣激揚,見於言色。



 及入參議論,時對眾官,面有譏毀。又表毀征南將軍田益宗,言華夷異類,不應在百世衣冠之上。率多侵侮,皆此類也。侍中于忠、黃門元昭
 覽之切齒,寢而不奏。



 韋伯昕告植欲謀廢黜。尚書又奏,羊祉告植姑子皇甫仲達,云受植旨,遂詐稱被詔,率合部曲,欲圖領軍於忠。時忠專權,既構成其禍,又矯詔殺之,朝野稱冤。臨終,神志自若,遺令子弟,命盡之後,剪落鬚髮,被以法服,以沙門禮葬於嵩高之陰。



 初,植與僕射郭祚、都水使者韋俊等同時見害。後祚、俊事雪加贈,而植追復封爵而已。植故吏勃海刁沖上疏訟之,於是贈尚書僕射、揚州刺史,乃改葬。



 植母,夏侯道遷姊也。性甚剛峻,於諸子皆如嚴君。長成後,非衣幍不見,小有罪過,必束帶伏門,經五三日乃引見之,督以嚴訓。唯少子
 衍得以常服見之,旦夕溫凊。植在瀛州也,其母年踰七十,以身為婢,自施三寶,布衣麻菲,手執箕帚於沙門寺掃灑。植弟瑜、粲、衍並亦奴僕之服,泣涕而從,有感道俗。諸子各以布帛數百贖免其母,於是出家為比丘。入嵩高積歲,乃還家。植既長嫡,母又年老,其在州數歲,以妻子自隨。雖自州送祿奉母及贍諸弟,而各別資財,同居異爨,一門數灶,蓋亦染江南之俗也。論者譏焉。



 植弟颺,壯果有謀略。在齊,以軍功位驍騎將軍。入魏,為南司州刺史,封義陽縣伯。詔命未至,為賊所殺,進爵為侯。宣武以颺勳效未立而卒,其子烱不得襲封。明帝初,烱行貨
 於執事,乃封城平縣伯。



 烱字休光,小字黃頭,頗有文學,善事權門。領軍元叉納其金帛,除鎮遠將軍、散騎常侍、揚州大中正,進爵為侯,改封高城。尋兼尚書右丞,出為東郡太守,為城人所害。贈散騎常侍、青州刺史,謚曰簡。



 颺弟瑜,字文琬,封下密縣子,試守滎陽郡,坐虐暴殺人免官。後徙封灌津子,卒於勃海太守,贈豫州刺史,謚曰定。



 瑜弟粲,字文亮,封舒縣子。沉重善風儀,頗以驕豪為失。歷正平、恒農二郡太守。高陽王雍曾以事屬粲,粲不從,雍甚為恨。後因九日馬射,敕畿內太守皆赴京師,雍時為州牧,粲脩謁,雍含怒待之。粲神情閒邁,舉止抑揚,
 雍目而不覺解顏。及坐定,謂粲曰:「可更為一行。」粲便下席為行,從容而出。坐事免。後宣武聞粲善自標置,欲觀其風度,令傳詔就家急召之,須臾間,使者相屬,合家恇懼,不測所以,粲更恬然,神色不變。帝歎異之。時僕射高肇以外戚之貴,勢傾一時,朝士見者,咸望塵拜謁。粲候肇,唯長揖而已。及還,家人尤責之,粲曰:「何可自同凡俗也。」又曾詣清河王懌,下車始進,便屬暴雨,粲容步舒雅,不以霑、濡改節。懌乃令人持蓋覆之,歎謂左右曰:「何代無奇人!」性好釋學,親昇講座,雖持義未精,而風韻可重。但不涉經史,終為知音所輕。



 後為揚州大中正、中書令。
 明帝釋奠,以為侍講,轉金紫光祿大夫。元顥入洛,以粲為西袞州刺史,尋為濮陽太守崔巨倫所逐,棄州入嵩高山。節閔帝初,復為中書令。後正月晦,帝出臨洛濱,粲起御前,再拜上壽酒。帝曰:「昔北海入朝,暫竊神器,爾日卿戒之以酒;今欲我飲,何異於往情?」粲曰:「北海志在沈湎,故諫其所失,陛下齊聖溫克,臣敢獻微誠。」帝曰:「甚愧來譽。」仍為命酌。孝武初,出為驃騎大將軍、膠州刺史。屬時亢旱,土人勸令禱於海神。粲憚違眾人,乃為祈請,直據胡床,舉盃曰:「僕白君。」左右云:「前後例皆拜謁。」粲曰:「五岳視三公,四瀆視諸侯,安有方伯致禮海神。」卒不肯拜。
 時青州叛賊耿翔寇亂三齊,粲唯高譚虛論,不事防禦之術。翔乘其無備,掩襲州城。左右白言賊至,粲云:「豈有此理!」左右又言「已入州門!」粲乃徐云:「耿王可引上聽事,自餘部眾,且付城人。」不達時變如此。尋為翔害,送首於梁。



 子含,字文若,員外散騎侍郎。



 粲弟衍,字文舒,學識優於諸兄,才亦過之。事親以孝聞,兼有將略。仕齊,位陰平太守。歸魏,授通直郎,衍堅辭朝命,上表請隱嵩高。詔從之。宣武末稍以出山,干祿執事。從歷建興、河內二郡太守。歷二郡,廉貞寡欲,善撫百姓,人吏追思之。孝昌初,梁將曹敬宗寇荊州。詔衍為別將,與恒農太守王羆救荊
 州。衍大破之,荊州圍解。除北道都督,鎮鄴西之武城,封安陽縣子。時相州刺史安樂王鑒潛圖叛逆,衍覺其有異,密表陳之。尋而鑒所部別將嵇宗馳驛告變,乃詔衍與都督源子邕、李神軌等討鑒,平之。除相州刺史、北道大都督,進封臨汝縣公。詔衍與子邕北討葛榮,軍敗見害。贈車騎大將軍、司空、相州刺史。子嵩襲。



 叔業之歸魏,又有尹挺、柳玄達、韋伯昕、皇甫光、梁祐、崔高容、閻慶胤、柳僧習並預其功。



 尹挺,天水冀人,仕齊,位陳郡太守。與叔業參謀歸誠,歷南司州刺史。



 柳玄達,河東解人,頗涉經史,仕齊,諸王參軍。與叔業姻婭周旋。叔業獻款,玄達
 贊成其計。入魏,除司徒諮議參軍,封南頓縣子。卒,改封夏陽縣,子絳襲。



 絳弟遠,字季雲,性粗放無拘檢,時人或謂之柳癲。好彈琴耽酒,時有文詠。孝武初,除儀同、開府參軍事。情琴酒之間,每出行返,家人或問消息,答云:「無所聞,縱聞亦不解。」後客遊卒。玄達弟玄瑜,位陰平太守,卒。子諧,頗有文學,善鼓琴,以新聲手勢,京師士子翕然從學。除著作佐郎,於河陰遇害。



 韋伯昕,京兆杜陵人,學尚有壯氣。自以才智優於裴植,常輕之,植嫉之如仇。



 即彥先之妹夫也。叔業以其有大志,故遣子芬之為質。入魏,封零陵縣男,歷南陽太守,坐事免。後拜員外散騎
 常侍,加中壘將軍。告裴植謀為廢黜,植坐死。後百餘日,伯昕亦病卒。臨亡,見植為祟,口云:「裴尚書死,不獨見由,何以見怒?」



 皇甫光,安定人,美鬚髯,善言笑。入魏,卒於勃海太守。兄椿齡,從薛安都於彭城內附,除岐州刺史。椿齡子璋,鄉郡相。璋弟枿,位吏部郎,性貪婪,多所受納,鬻賣吏官,皆有定價。後以丞相、高陽王雍之婿,為豫州刺史。為政殘暴,百姓患之。卒於安南將軍、光祿大夫、贈尚書左僕射。子長卿,太尉司馬。



 梁祐,北地人,叔業從姑子也。好學,便弓馬,隨叔業征伐,身被五十餘創。



 景明初,賜爵山桑子。出為北地太守,清身率下,甚有聲稱。歷大中大
 夫。從容風雅,好為談詠,常與朝廷名賢,泛舟洛水,以詩酒自娛。遷光祿大夫,端然養志,不歷權門,卒於京兆內史。



 崔高容,清河人,博學善文辭,美風彩。景明初,位散騎侍郎,出為揚州開府掾,帶陳留太守,卒官。



 閻慶胤,天水人,博識洽聞,善於談論,聽其言說,不覺忘疲。卒於敷城太守。



 柳僧習見其子虯傳。



 夏侯道遷,譙國人也。少有志操。年十七,父母為結婚韋氏,道遷云:「欲懷四方之志,不願取婦。」家人咸謂戲言。及婚,求覓不知所在。訪問,乃云逃入益州。後隨裴叔業於壽春,為南譙太守。二家雖為姻好,親情不協,遂單騎歸
 魏,拜驍騎將軍,隨王肅至壽春。肅薨,道遷棄戍南叛。



 會梁以莊丘黑為征虜將軍、梁秦二州刺史,鎮南鄭。黑請道遷為長史,帶漢中郡。會黑死,而道遷陰圖歸順。先是,仇池鎮將楊靈珍反叛南奔,梁以靈珍為征虜將軍,假武都王,助戍漢中。道遷乃擊靈珍,斬其父子,送首於京師。江悅之等推道遷為梁、秦二州刺史。道遷遣表歸闕,詔璽書慰勉,授持節、散騎常侍、平南將軍、豫州刺史,封豐縣侯,遣尚書邢巒指授節度。道遷表受平南、常侍,而辭豫州、豐縣侯,引裴叔業公爵為例。宣武不許。



 道遷自南鄭來朝京師,引見於太極東堂,免冠徒跣謝曰:「比在
 壽春,遭韋纘之酷,申控無所,致此猖狂。是段之來,希酬昔遇。」宣武曰:「卿建為山之功,一簣之玷,何足謝也。」道遷以賞報為微,逡巡不拜,尋改封濮陽縣侯。歲餘,頻表解州,宣武許之。除南袞州大中正,不拜。



 道遷雖學不深洽,而歷覽書史,閑習尺牘。好言宴,務口實,京師珍羞,罔不畢有。於京城西水次市地,大起園池,殖列蔬果,延致秀彥,時往遊適。妓妾十餘,常自娛樂,國秩歲入三千餘匹,專供酒饌,不營家產。每誦孔融語曰:「坐上客恒滿,樽中酒不空,餘非吾事也。」識者多之。歷華、瀛二州刺史,為政清嚴,善禁盜賊。卒,贈雍州刺史,謚明侯。初,道遷以拔漢
 中歸誠本由王潁興之計,求分邑戶五百封之,宣武不許。靈太后臨朝,道遷重求分封,太后大奇之,議欲更以三百戶封潁興,會卒,遂寢。道遷不聘正室,唯有庶子數人。



 長子夬、,字元廷,歷鎮遠將軍、南袞州大中正。夬性好酒,居喪不戚,醇醪肥鮮,不離於口,沽買飲啖,多所費用。父時田園,貨賣略盡,人間債猶數千餘匹,穀食至常不足,弟妹不免飢寒。



 初,道遷知夬好酒,不欲傳授國封。夬未亡前,忽夢見征虜將軍房世寶至其家聽事,與其父坐,屏人密言。夬心驚懼,謂人曰:「世寶為官,少間必擊我也。」



 尋有人至,云:「官呼郎」,隨召即去,遣左右杖之二百,不勝
 楚痛,大叫。良久乃悟,流汗徹於寢具。至明,前京城太守趙卓詣之,見其衣濕,謂夬曰:「卿昨夜當大飲,溺衣如此。」夬乃具陳所夢。先是旬餘,秘書監鄭道昭暴病卒,夬聞,謂卓曰:「人生何常,唯當縱飲。」於是昏酣遂甚。夢後,二日不能言,針之,乃得語,而猶虛劣。俄而心悶而死。洗浴者視其尸體,大有杖處,青赤隱起,二百下許。



 贈鉅鹿太守。



 初,夬與南人辛諶、庾遵、江文遙等終日遊聚。酣飲之際,恒相謂曰:「人生局促,何殊朝露,坐上相看,先後間耳。脫有先亡者,於良辰美景,靈前飲宴,儻或有知,庶共歆饗。」及夬亡後,三月上巳,諸人相率至夬靈前,仍共酌飲。時
 日晚天陰,室中微暗,咸見夬在坐,衣服形容,不異平昔,時執盃酒,似若獻酬,但無語耳。夬家客雍僧明心有畏恐,披簾欲出,便即僵仆,狀若被毆。夬從兄欣宗云:「今是節日,諸人憶弟疇昔之言,故來共飲。僧明何罪,而被嗔責?」僧明便悟。



 而欣宗鬼語如夬平生,并怒家人,皆得其罪,又發陰私竊盜,咸有次緒。



 夬妻,裴植之女也,與道遷諸妾不睦,訟鬩徹于公庭。子籍,年十餘歲,襲祖封已數年,而夬弟翽等言其眇目癇疾,不任承繼,自以與夬同庶,己應紹襲。尚書奏籍承封。



 道遷兄子,位咸陽太守。



 道遷之謀,又襄陽羅道珍、北海王安世、潁川辛諶、漢
 中姜永等皆參其勳末。



 道珍為齊州東平原相,有能名。安世,苻堅丞相王猛玄孫也。歷涉書傳,位北華州刺史。諶,魏衛尉辛毗後也,有文學,位濮陽、上黨二郡太守。永善彈琴,有文學,位漢中太守。永弟漾,亦善士,性至孝。時潁川庾道者,亦與道遷俱入國,雖不參勛謀,亦為奇士。歷覽史傳,善草隸書,輕財重義。仕梁,右中郎將。及至洛陽,環堵弊廬,多與俊秀交舊,積二十餘歲,殊無宦情。後為饒安縣令,罷,卒。



 李元護,遼東襄平人,晉司徒胤之八世孫也。胤子順、璠及孫沉、志皆有名宦。



 沉孫根,仕慕容寶,為中書監。根子
 後智等隨慕容德南渡河,居青州,數世無名,三齊豪門多輕之。元護以魏平齊後,隨父懷慶南奔。身長八尺,美須髯,少有武力。



 仕齊,位馬頭太守,雖以將用自達,然亦頗覽文史,習於簡牘。後為裴叔業司馬,帶汝陰太守。叔業歸順,元護贊同其謀。叔業疾病,元護督率上下以俟援軍。壽春剋定,元護頗有力焉。景明初,以元護為齊州刺史、廣饒縣伯。尋以州人柳世明圖為不軌,元護誅戮所加,微為濫酷。州內饑儉,表請賑貸,蠲其賦役。但多有部曲,時為侵擾,城邑苦之,故不得為良刺史也。三年卒。病前月餘,京師無故傳其凶問。



 又城外送客亭柱有人
 書曰「李齊州死」,綱佐餞別者見而拭之,後復如此。元護妾妓十餘,聲色自縱。情慾既甚,支骨稍消,鬚長二尺,一時落盡。贈青州刺史。元護為齊州,經拜舊墓,巡省故宅,饗賜村老,莫不欣暢。及將亡,謂左右曰:「吾嘗以方伯簿伍至青州,士女屬目。若喪過東陽,不可不好設儀衛,哭泣盡哀,令觀者改容也。」家人遵其誡。



 子會襲,正始中降爵為子。會頑騃好酒,其妻南陽太守清河房伯玉女也,甚有姿色,會不答之。房乃通其弟機,因會醉,殺之。子景宣襲。機與房遂如夫婦,積十餘年,房氏色衰,乃更婚娶。



 元護弟靜,性貪忍,兄亡未斂,便剝妓服玩及餘物。歷齊
 郡內史。



 席法友,安定人也,祖、父南奔。法友仕齊,以膂力自效,任安豐新蔡二郡太守、建安戍主。後與裴叔業同謀歸魏,拜豫州刺史、苞信縣伯。叔業卒後,法友與裴植追成業志,淮南剋定,法友有力焉。歷華、并二州刺史。後為別將出淮南,欲解朐山之圍。法友始渡淮而朐山敗沒,遂停十年。恬靜自安,不競世利。宣武末,除濟州刺史,廉和著稱。又徙封乘氏。後卒於光祿大夫,贈秦州刺史,謚襄侯。



 子景通襲,善事元叉,兼賂叉父繼。繼為司空,引景通為掾。卒,贈衛尉少卿。



 子郾襲,走關西。



 王世弼,京兆霸城人也。姚泓之滅,其祖、父南遷。世弼身長七尺八寸,魁岸有壯氣,善草隸書,好愛墳典。仕齊為軍主,助戍壽春,遂與裴叔業同謀歸誠。除南徐州刺史,封慎縣伯。後除東秦州刺史,政任於刑,為人所怨,有受納之響,為御史中尉李平所彈,會赦免。後為河北太守,有清稱。再遷中山內史,加平北將軍。



 直閣元羅,領軍元叉弟也,曾過中山,謂曰:「二州刺史,翻復為郡,當恨恨耳。」



 世弼曰:「儀同之號,起自鄧騭,平北為郡,始在下官。」卒,贈豫州刺史,謚曰康。



 長子會,汝陽太守。次子由,字茂道,好學有文才,尤善草隸書,性方厚,有名士風,又工摹書,為
 時人所服。位東萊太守,罷郡寓居潁川。天平初,元洪威構逆,大軍攻討,為亂兵所害。名流悼惜之。



 江悅之,字彥和,濟陽考城人也。七世祖統,晉散騎常侍,避劉、石之亂,南渡。祖興之,父範之,並為宋武所誅。悅之少孤,仕宋,歷諸王參軍。好兵書,有將略,善待士,有部曲數百人。仕齊,為後軍將軍,部曲稱眾,千有餘人。梁初,以討滅劉季連功,進號冠軍將軍。武興氐攻破白馬,進圖南鄭,悅之大破氐眾,還復白馬。梁、秦二州刺史莊丘黑死,夏侯道遷與悅之及龐樹,軍主李忻榮、張元亮、士孫天與等謀以梁州內附。梁華陽太守尹天寶率眾向州
 城,遂圍南鄭。悅之晝夜督戰,會武興軍至,天寶敗。道遷克全勳款,悅之實有力焉。與道遷俱至洛陽。尋卒,贈梁州刺史,追封安平縣子,謚曰莊。



 悅之二子,文遙、文遠。文遙少有大度,輕財好士,士多歸之。道遷之圖楊靈珍,文遙奮劍請行,遂手斬靈珍。襲父封,拜咸陽太守。勤於禮接,終日坐聽事。



 至者見之,假以恩顏,屏人密問,於是人所疾苦,大盜姓名,奸猾吏長,無不知悉。



 郡中震肅,奸劫息止,政為雍州諸郡之最。後為安州刺史,善於綏納,甚得物情。



 時杜洛周、葛榮等相繼叛逆,幽、燕已南悉沒,唯文遙介在群賊之外,孤城獨守,鳩集荒餘,且耕且戰,
 百姓皆樂為用。卒官,長史許思祖等以文遙有遺愛,復推其子果行州事。既攝州事,乃遣使奉表。莊帝嘉之,除果通直散騎侍郎,行安州事。



 既而賊勢轉盛,救援不接,果乃攜諸弟并率城人東奔高麗。天平中,詔高麗送果等。



 元象中,乃得還朝。文遠善騎射,勇於攻戰,以軍功位中散大夫、龍驤將軍。



 淳于誕,字靈遠,其先太山博人也,後世居蜀漢,或家安固之桓陵縣。父興宗,齊南安太守。誕年十二,隨父向揚州。父於路為群盜所害,誕雖幼而哀感奮發,傾資結客,旬朔之內,遂得復仇。州里之間,無不稱嘆。景明中,自漢
 中歸魏,陳伐蜀計,宣武嘉納之。延昌末,王旅大舉,除驃騎將軍、都督、別部司馬,領鄉導統軍。誕不願先受榮爵,乃固讓實官,止參戎號。及奉辭之日,詔若剋成都,即以益州許之。師次晉壽,蜀人大震。屬宣武晏駕,不果而還。後以客例,起家羽林監。



 正光中,秦、隴反叛,詔誕為西南道軍司馬,與行臺魏子建共參經略。時梁益州刺史蕭深猷遣將樊文熾蕭世澄等率眾數萬圍小劍戍。子建遣誕勒兵馳赴,大敗之,禽世澄等十一人,文熾先走獲免。孝昌初,子建以誕行華陽郡,帶白馬戍。後卒於東梁州刺史,贈益州刺史,謚曰莊。



 沈文秀,字仲遠,吳興武康人也。伯父慶之,《南史》有傳。文秀仕宋,位青州刺史。和平六年,宋明帝殺其主子業,文秀與諸州推立子業弟子勛。子勛敗,皇興初,文秀與崔道固俱以州降魏。宋遣其弟文景來諭之,文秀復歸宋,為刺史如故。



 後慕容白曜長驅至東陽,文秀始欲降,以軍人虜掠,遂有悔心,乃嬰城固守。白曜既下歷城,乃并力攻討,自夏至春始剋。文秀取所持節,衣冠儼然,坐於齋內。亂兵入曰:「文秀何在!」文秀歷聲曰:「身是!」執而裸送于白曜。左右令拜,文秀曰:「各二國大臣,無相拜禮。」白曜忿之,因至撾撻。後還其衣,為之設饌,與長史房天樂、司馬
 沈嵩等鎖送京師,面縛數罪,宥死,待為下客,給以粗衣蔬食。



 獻文重其節義,稍亦嘉禮之,拜外都下大夫。太和三年,遷外都大官。孝文嘉其忠於其國,賜絹彩二百匹。後為南征都將,臨發,賜以戎服。除懷州刺史,假吳郡公。



 守清貧而政寬,不能禁止盜賊。大興水田,於公私頗有利益。卒官。



 子保沖,後為徐州冠軍長史,坐據連口退敗,有司處之死刑。孝文詔:「保沖,文秀之子,可特原命,配洛陽作部終身。」宣武時,卒於下邳太守。



 房天樂者,清河人,滑稽多智。文秀板為長史,督齊郡,州府事一以委之。卒於京師。弟子嘉慶,漁陽太守。



 張讜,字處言,清河東武城人也。六世祖弘,晉長秋卿。父華,慕容超左僕射。



 讜仕宋,位東徐州刺史。及平徐、兗,讜乃歸順於尉元,亦表授東徐州刺史。遣中書侍郎高閭與讜對為刺史。後至京師,禮遇亞於薛、畢,賜爵平陸侯。讜性開通,篤於接恤,青、齊之士,雖疏族末姻,咸相敬視。李敷、李等寵要勢家,亦推懷陳款,無所顧避。畢眾敬等皆敬重之,高允之徒亦相器待。卒,贈青州刺史,謚康侯。



 子敬伯,求致父喪,出葬冀州清河舊墓,久不被許,停柩在家積五六年。第四子敬叔,先在徐州,初聞父喪,不欲奔赴,而規南叛,為徐州所勒送。至乃自理,後得襲父
 爵。敬伯自以隨父歸國功,賜爵昌安侯,出為樂陵太守。敬叔,武邑太守。



 父喪得葬舊墓,還屬清河。



 初,讜兄弟十人。兄忠,字處順,在南為合鄉令。歸降,賜爵新昌侯。卒於新興太守,贈冀州刺史。讜妻皇甫氏被掠,賜中官為婢,皇甫遂詐癡,不能梳沐。後讜為宋冀州長史,因貨千餘匹,購求皇甫。文成怪其納財之多,引見之,時皇甫年垂六十矣。文成曰:「南人奇好,能重室家之義。此老母復何所任,乃能如此致費也。」皇甫氏歸,讜令諸妾境上奉迎。數年卒。後十年而讜入魏。



 讜兄子安世,正始中,自梁漢同夏侯道遷歸款,為客積年,出為東河間太守。



 卒。



 李苗,字子宣,梓潼涪人也。父膺,梁太僕卿。苗出後叔父畎。畎為梁州刺史,大著威名。王足之伐蜀,梁武命畎拒足於涪,許其益州。及足退,梁武遂改授。畎怒,將有異圖,事發被害。苗年十五,有報雪志。延昌中歸魏,仍陳圖蜀計。大將軍高肇西伐,詔假苗龍驤將軍鄉導。次晉壽,宣武宴駕,班師。後以客例,除員外散騎侍郎。苗有文武才幹,以大功不就,家恥未雪,常懷慷慨。乃上書陳平定江南之計,其文理甚切於時。明帝幼沖,無遠略之意,竟不能納。



 正光末,三秦反叛,侵及三輔。時承平既久,人不習戰。苗以隴兵強悍,且群聚無資,乃上書以為:「食少兵精,
 利於速戰;糧多卒眾,事宜持久。今隴賊猖狂,非有素蓄,雖據兩城,本無德義,其勢在於疾攻,日有降納,遲則人情離阻,坐受崩潰。夫飆至風起,逆者求萬一之功;高壁深壘,王師有全制之策。今且宜勒大將,深溝高壘,堅守勿戰。別命偏師,精卒數千,出麥積崖以襲其後,則水幵、岐之下,群妖自散。」於是詔苗為統軍,與別將淳于誕出梁、益,隸行臺魏子建。子建以苗為郎中,仍領統軍,深見知待。



 孝昌中,兼尚書左丞,為西北道行臺,與大都督宗正珍孫討汾、絳蜀賊,平之。


及殺
 \gezhu{
  人小}
 朱榮,榮從弟世隆擁部曲還逼都邑。孝莊幸大夏門,集群臣博議,百僚計無所出。
 苗獨奮衣起曰:「今朝廷有不測之危,正是忠臣烈士效節之時,請以一旅之眾,為陛下徑斷河梁。」莊帝壯而許焉。苗乃募人於馬渚上流,以師夜下。去橋數里,放火燒船,俄然橋絕,賊沒水死者甚眾。官軍不至,賊乃涉水與苗死鬥,眾寡不敵,苗浮河而沒。帝聞,哀傷久之。贈都督、梁州刺史、車騎大將軍、儀同三司、河陽縣侯,謚忠烈。



 苗少有節操,志尚功名。每讀《蜀書》,見魏延請出長安,諸葛不許,歎息謂亮無奇計。及覽《周瑜傳》,未曾不嗟咨絕倒。太保城陽王徽、司徒臨淮王彧並重之。二王頗或不穆,苗每諫責。徽寵勢隆極,猜忌彌甚,苗謂人曰:「城陽蜂目
 豺聲,今轉彰矣!」解鼓琴,善屬文詠,工尺牘之敏,當世罕及。死之日,朝野悲壯之。及帝幽崩,世隆入洛,主者追苗贈封,以白世隆。世隆曰:「吾爾時群議,更三日便欲大縱兵士,燒燔都邑,任其採掠。賴苗,京師獲全。天下之善一也,不宜追之。」子曇襲爵。



 劉藻,字彥先,廣平易陽人也。六世祖遐,從晉元帝南渡。父宗之,宋廬江太守。藻涉獵群籍,美談笑,善與人交,飲酒至一石不亂。太安中,與姊夫李嶷俱來歸魏,賜爵易陽子。擢拜南部主書,號為稱職。



 時北地諸羌,恃險作亂,前後宰守不能制。朝廷患之,以藻為北地太守。藻推誠
 布信,諸羌咸來歸款,朝廷嘉之。雍州人王叔保等三百人表乞藻為騃奴戍主,詔曰:「選曹已用人,藻有惠政,自宜他敘。」在任八年,遷離城鎮將。太和中改鎮為岐州,以藻為岐州刺史。轉秦州刺史。秦人恃險,率多粗暴,或拒課輸,或害吏長,自前守宰,皆遙領,不入郡縣。藻開示恩信,誅戮豪橫,羌、氐憚之,守宰於是始得居其舊所。遇車駕南伐,以藻為東道都督。秦人紛擾,詔藻還州,人情乃定。



 仍與安南元英征漢中,破賊軍,長驅至南鄭,垂平梁州,奉詔還軍,乃不果克。



 後車駕南伐,以藻為征虜將軍,督統軍高聰等四軍為東道別將,辭於洛水之南。



 孝文
 曰:「與卿石頭相見。」藻對曰:「臣雖才非古人,庶亦不留賊虜而遺陛下。



 輒當釃曲阿之酒以待百官。」帝大笑曰:「今未至曲阿,且以河東數石賜卿。」後與高聰等戰敗,俱徙平州。景明初,宣武追錄舊功,拜藻為太尉司馬。卒。



 子紹珍,無他才用,善附會,好飲酒。結託劉騰,啟為其國郎中令,襲子爵。



 永安中,歷河北、黎陽二郡太守,所在無政績。天平中,坐子洪業入於關中,率眾侵擾,伏法。



 傅永,字脩期,清河人也。幼隨叔父洪仲與張幸自青州入魏,尋復南奔。有氣幹,拳勇過人,能手執鞍橋,倒立馳騁。年二十餘,有友人與之書而不能答,請洪仲,洪仲深
 讓之而不為報。永乃發憤讀書,涉獵經史,兼有才幹。為崔道固城局參軍,與道固俱降,入為平齊百姓。父母並老,飢寒十數年,賴其強於人事,戮力傭丐,得以存立。晚為奉禮郎,詣長安拜文明太后父燕宣王廟,賜爵貝丘男,除中書博士。王肅之為豫州,又以永為王肅平南長史。咸陽王禧慮肅難信,言於孝文。曰:「已選傅脩期為其長史,雖威儀不足,而文武有餘矣。」肅以永宿士,禮之甚厚;永亦以肅為帝眷遇,盡心事之,情義至穆。



 齊將魯康祖、趙公政侵豫州之太倉口,肅令永擊之。永量吳、楚兵好以斫營為事,又賊若夜來,必於渡淮之所,以火記其
 淺處。永既設伏,仍密令人以瓠盛火,渡南岸,當深處置之,教云:「若有火起,即亦燃之。」其夜,康祖、公政等果親率領來斫營。東西二伏夾擊之,康祖等奔趨淮水。火既競起,不能記其本濟,遂望永所置火爭渡。水深溺死,斬首者數千級,生禽公政。康祖人馬墜淮,曉而獲其尸,斬首并公政送京師。



 時裴叔業率王茂先、李定等東侵楚王戍,肅復令永將伏兵,擊其後軍,破之,獲叔業傘扇鼓幕甲仗萬餘。兩月之中,遂獻再捷。帝嘉之,遣謁者就豫州,策拜永安遠將軍、鎮南府長史、汝南太守、貝丘縣男。帝每歎曰:「上馬能擊賊,下馬作露布,唯傅修期耳。」



 裴叔業
 又圍渦陽,時帝在豫州,遣永為統軍,與高聰、劉藻、成道益、任莫問等救之。永曰:「深溝固壘,然後圖之。」聰等不從,一戰而敗。聰等棄甲奔懸瓠,永獨收散卒徐還。賊追至,又設伏擊之,挫其銳。藻徙邊,永免官爵而已。不經旬,詔永為汝陰鎮將,帶汝陰太守。



 景明初,裴叔業將以壽春歸魏,密通於永。及將迎納,詔永為統軍,與楊大眼、奚康生等諸軍俱入壽春。同日而永在後,故康生、大眼二人並賞列士,永唯清河男。



 齊將陳伯之逼壽春,沿淮為寇。時司徒彭城王勰、廣陵侯元衍同鎮壽春,以九江初附,人情未洽,兼臺援不至,深以為憂。詔遣永為統軍,領汝
 陰三千人先援之。永至,勰令永引軍入城。永曰:「若如教旨,便共殿下同被圍守,豈是救援之意?」



 遂孤軍城外,與勰并勢以擊伯之,頻有剋捷。



 中山王英之徵義陽,永為寧朔將軍、統軍,當長圍遏其南門。齊將馬仙琕連營稍進,規解城圍。永乃分兵付長史賈思祖,令守營壘,自將馬步千人,南逆仙琕。



 賊俯射永,洞其左股,永出箭復入,遂大破之。仙琕燒營卷甲而遁。英曰:「公傷矣!且還營。」永曰:「昔漢祖捫足,不欲人知。下官雖微,國家一帥,奈何使虜有傷將之名!」遂與諸軍追之,極夜而返。時年七十餘矣,三軍莫不壯之。



 義陽既平,英使司馬陸希道為露布,意
 謂不可,令永改之。永亦不增文采,直與之改,陳列軍儀,處置形要,而英深賞之。還京,除太中大夫。



 後除恒農太守,非心所樂。時英東征鐘離,表請永,求以為將,朝廷不聽。永每言曰:「馬援、充國,竟何人哉?吾獨白首見拘此郡!」然於御人非其所長,故在任無多聲稱。後為南袞州刺史。年踰八十,猶能馳射,盤馬奮槊,常諱言老,每自稱六十九。還京,拜光祿大夫。卒,贈齊州刺史。



 永嘗登北芒,於平坦處奮矛躍馬,盤旋瞻望,有終焉之志。遠慕杜預,近好李沖、王肅,欲葬附墓。遂買左右地數頃,遺敕子叔偉:「此吾之永宅也。」永妻賈氏留本鄉,永至代都,娶妾馮氏,
 生叔偉及數女。賈後歸平城,無男,唯一女。馮恃子,事賈無禮,叔偉亦奉賈不順,賈常忿之。馮先永卒,叔偉稱父命欲葬北芒,賈疑叔偉將以馮合葬,遂求歸葬永於所封貝丘縣。事經司徒,司徒胡國珍感其所慕,許叔偉葬焉。賈乃邀訴靈太后,太后從賈意,乃葬於東清河。又永昔營宅兆,葬父母於舊鄉,賈於此強徙之,與永同處,永宗親不能抑。葬已數十年矣,棺為桑棗根所繞束,去地尺餘,甚為周固。以斧斫,出之於坎,時人咸怪。



 叔偉膂力過人,彎弓三百斤,左右馳射,能立馬上與人角騁,見者以為得永武而不得永文。



 傅豎眼,本清河人也。七世祖伷。伷子遘,石季龍太常。祖父融,南徙度河,家于磐陽,為鄉閭所重。性豪俠,有三子,靈慶、靈根、靈越,並有才力。融以自負,謂足為一時之雄。嘗謂人曰:「吾昨夢夜,有一駿馬,無堪乘者,人曰『何由得人乘』,有一人曰『唯傅靈慶堪乘此馬』;又有弓一張,亦無人堪引,人曰『唯有傅靈根可彎此弓』;又有數紙文書,人皆讀不能解,人曰『唯有傅靈越能解此文。』」



 融謂其三子文武才幹足以駕馭當世,常從容謂鄉人曰:「汝聞之不?鬲蟲之子有三靈,此圖讖文也。」好事者然之,故豪勇士多相歸附。



 宋將蕭斌、王玄謨寇磝碻。時融始死,玄謨強引
 靈慶為軍主。將攻城,攻車為城內所燒,靈慶懼軍法,詐云傷重,令左右輿還營,遂與壯士數十騎遁還。斌、玄謨命追之。左右諫曰:「靈慶兄弟並有雄才,兼其部曲多是壯勇,如彭超、尸生之徒,皆一當數十人,援不虛發,不可逼也。」玄謨乃止。靈慶至家,遂與二弟匿山澤間。時靈慶從叔乾愛為斌法曹參軍,斌遣乾愛誘呼之,以腰刀為信,密令壯健者隨之。而乾愛不知斌之欲圖靈慶。既至,斌所遣壯士執靈慶殺之。靈慶將死,與母崔氏訣,言:「法曹殺人,不可忘也。」



 靈根、靈越奔河北。靈越至京師,因說齊人慕化,青州可平。文成大悅,拜靈越青州刺史、貝丘
 子,鎮羊蘭城;靈根為臨齊副將,鎮明潛壘。靈越北入之後,母崔氏遇赦免。宋恐靈越在邊擾三齊,乃以靈越叔父琰為冀州中從事,乾愛為樂陵太守。樂陵與羊蘭隔河相對,命琰遣其門生與靈越婢詐為夫婦,投化以招之。靈越與母分離思積,遂與靈根南走。靈越與羊蘭奮兵相擊,乾愛出,遣船迎之,得免。靈根差期,不得俱渡,臨齊人知,剉斬殺之。乾愛出郡迎靈越,問靈根愆期狀,靈越殊不應答。乾愛不以為惡,敕左右出匣中烏皮褲褶,令靈越代所常服。靈越言「不須」。乾愛云:「汝可著體上衣服見垣公也?」時垣護之為刺史。靈越奮聲言:「垣公!垣公!
 著此當見南方國主,豈垣公也!」竟不肯著。及至丹陽,宋孝武見而禮之,拜袞州司馬,而乾愛亦遷青、冀司馬,帶魏郡。後二人俱還建鄴。靈越意恒欲為兄復仇,而乾愛初不疑防。知乾愛嗜雞肉葵菜食,乃為作之,下以毒藥,乾愛飯還而卒。後數年,靈越為太原太守,戍升城。後舉兵同孝武子子勛,子勛以靈越為前軍將軍。子勛敗,靈越軍眾散亡,為明帝將王廣之軍人所擒,厲聲曰:「我傅靈越也,汝得賊何不即殺!」廣之生送詣宋輔國司馬劉勔,勔躬自慰勞。靈越曰:「人生歸於死,實無面求活。」勔壯其意,送詣建康。宋明帝欲加原宥,靈越辭對如一,乃殺之。



 豎眼即靈越子也,沉毅壯烈,少有父風。入魏,鎮南王肅見而異之,且奇其父節,傾身禮敬,表為參軍。以軍功累遷益州刺史。高肇伐蜀,假豎眼征虜將軍、持節,領步兵三萬,先討巴北,所至剋捷。豎眼性既清素,不營產業,衣食之外,俸祿粟帛皆以饗賜夷首,振恤士卒。撫蜀人以恩信為本,保境安人,不以小利侵竊。



 有掠蜀人入境者,皆移送還本。檢勒部下,守宰肅然。遠近雜夷相率款謁,仰其德化,思為魏人矣。宣武甚嘉之。



 明帝初,屢請解州,乃以元法僧代之,益州人追隨戀泣者數百里。梁將趙祖悅逼壽春,鎮南將軍崔亮討之,以豎眼為持節、鎮南
 軍司。



 法僧既至,大失人和。梁遣其衡州刺史張齊因人心怨入寇,進圍州城。朝廷以西南為憂,乃驛徵豎眼於淮南,以為益州刺史。尋加散騎常侍、西征都督,率步騎三千以討齊。給銅印千餘,須有假職者,聽六品已下板之。豎眼既出梁州,梁軍所在拒塞,豎眼三日中轉戰二百餘里,甲不去身,頻致九捷。蜀人聞豎眼復為刺史,人人喜悅,迎於路者日有百數。豎眼至州,白水已東,人皆寧業。張齊仍阻白水屯,寇葭萌。豎眼分遣諸將水陸討之,大破其軍。齊被重創,奔而退,小劍大劍賊亦捐城西走,益州平。靈太后璽書慰勞,賜驊騮馬一匹,寶劍一口。



 後轉岐州刺史,仍轉梁州刺史。梁州人既得豎眼為牧,人咸自賀。而豎眼至州遇患,不堪綜理。其子敬紹險暴不仁,聚貨耽色,甚為人害,遠近怨望。尋假鎮南將軍,都督梁、西益、巴三州諸軍事。梁遣其北梁州長史錫休儒等十軍率眾三萬人寇直城,豎眼遣敬紹總眾赴擊,大破之。敬紹頗覽書傳,微有膽力,而奢淫倜儻,輕為殘害,又見天下多事,陰懷異圖,欲杜絕四方,擅據南鄭。令其妾兄唐崑崙扇攪於外,聚眾圍城,敬紹謀為內應。賊圍既合,事泄,在城兵執敬紹;白豎眼而殺之。豎眼恚,發疾卒。永安中,贈吏部尚書、齊州刺史。孝武帝初,贈司空
 公、相州刺史。



 長子敬和,次敬仲,並好酒薄行,傾側勢家。敬和,孝莊時以其父有遺惠於益州,復為益州刺史。至州,聚斂無已,好酒嗜色,遠近失望。仍為梁將樊文熾攻圍,城降,送於江南。後以齊神武威德日廣,令敬和還北,以申和通之意。除北徐州刺史,復以耽酒為土賊掩襲,棄城走。遂廢棄,卒於家。



 張烈,字徽之,清河東武城人也,孝文帝賜名曰烈,仍以本名為字焉。高祖悕,為慕容俊尚書右僕射。曾祖恂,散騎常侍,隨慕容德南度,因居齊郡之臨淄縣。



 烈少孤貧,涉獵經史,有氣概,時青州有崔徽伯、房徽叔、與烈並有
 令譽,時人號「三徽」。孝文時,入官代都,歷侍御、主文中散。遷洛,為太子步兵校尉。



 齊將陳顯達謀將入寇,時順陽太守王青石,世官江南,荊州刺史、廣陽王嘉慮其有異,表請代之。詔侍臣各舉所知,互有申薦者。帝曰:「太子步兵張烈,每論軍國事,時有會人意處,朕欲用之如何?」彭城王勰稱贊之,遂除順陽太守。烈到郡二日,便為齊將崔慧景攻,圍之七十餘日,烈撫厲將士,甚得軍人之和。會車駕南討,慧景遁走。帝親勞之曰:「卿果能不負所寄。」烈謝曰:「不遇鑾輿親駕,臣不免困於犬羊。自是陛上不負臣,非臣能不負陛下。」帝善其對。



 宣武即位,追錄先勳,
 封清河縣子。尋以母老歸養,積十餘年。頻遇凶儉,烈為粥以食飢人,蒙濟者甚眾,鄉黨以此稱之。



 明帝即位,為司空長史。先是元叉父江陽王繼曾為青州刺史,及叉當權,烈託故義之懷,遂相諂附。歷給事黃門侍郎、光祿大夫。靈太后反政,以叉黨出為青州刺史。時議者以烈家產畜殖,家僮甚多,慮其有異,恐不宜出為本州,改瀛州刺史。



 為政清靜,吏人安之。後因辭老還鄉,兄弟同居怡然,為親類所慕。卒於家。



 烈先為家誡千餘言,並自敘志行及所歷之官。臨終,敕子姪不聽求贈,但勒家誡立碣而已。其子質奉行焉。



 質博學有才藝,位諫議大夫。



 烈
 弟僧皓,字山容,歷涉群書,工於談說,有名於當世。以諫議大夫、國子博士、散騎侍郎徵,並不起,世號徵君焉。好營產業,孜孜不已,藏鏹巨萬,他資稱是。兄弟自供儉約,車馬瘦弊,身服布裳,而婢妾紈綺。僧皓尤好蒲弈,戲不擇人,是以獲譏於世。節閔帝時,崔祖螭舉兵攻東陽城,僧皓與同事,事敗,死於獄中。



 李叔彪,勃海蓚人也。從祖金,神蒨中,與高允俱徵,位征南從事中郎。叔彪好學博聞,有識度,為鄉閭所稱。太和中,拜中書博士,與清河崔亮、河間邢巒並相親友。三遷國子博士、本國中正,攝樂陵中正。性清直,甚有公平之
 稱。歷中書侍郎。太尉、高陽王雍以其器操重之。尋除假節,行華州事,為吏人所稱。卒,贈南青州刺史,謚曰穆。



 叔彪子述,字道興,有學識,州舉秀才,拜太常博士。使詣長安冊祭燕宣王廟。



 還,除儀曹郎,賜爵蓚縣男。稍遷興平太守。卒。



 子象,字孟則,清簡有風概,博涉群書。襲爵,稍遷中書侍郎、光祿大夫,兼散騎常侍,使梁。卒,贈驃騎大將軍、儀同三司,冀州刺史。象從容風素,有名於時,喪妻無子,終竟不娶,論者非之。



 路恃慶,字伯瑞,陽平清泉人也。祖綽,陽平太守。恃慶有乾用,與廣平宋翻俱知名,為鄉閭所稱。太和中,除奉朝
 請,恃慶以從兄文舉有才望,因推讓之,孝文遂並拜焉。累遷定州河間王琛長史。琛貪暴肆意,恃慶每進苦言。卒,贈左將軍、安州刺史,謚曰襄。子祖璧,給事中。



 恃慶弟仲信、思令,並有令名官位。



 房亮,字景高,清河人也。父法延,譙郡太守。亮好學有節操。太和中,舉秀才,為奉朝請。後兼員外常侍,使高麗。高麗王託疾不拜。以亮辱命,坐白衣守郎中。歷濟北、平原二郡太守,以清嚴稱。後為東荊州刺史,亮留心撫納,夷夏安之。



 時邊州刺史例得一子出身,亮不言其子而啟弟子起為奉朝請,議者稱之。卒於光祿大夫,贈撫軍將
 軍、齊州刺史。



 弟詮、悅等,並歷位清顯。



 曹世表,字景昇,魏大司馬休九世孫也。祖謨,父慶,並有學問。世表性雅正,工尺牘,涉獵群書。為司徒記室,與武威賈思伯、范陽盧同、隴西辛雄並相友善。



 侍中崔光,鄉里貴達,每稱美之。延昌中,除清河太守,臨官省約,百姓安之。孝昌中,為尚書左丞,出行東豫州刺史,遷東南道行臺。卒,贈齊州刺史。



 潘永基,字紹業,長樂廣宗人也。父靈乾,中書侍郎。永基性通率,輕財好施。



 為長樂太守。時葛榮攻信都,永基與刺史元孚同心防捍。力窮城陷,榮欲害孚,永基請以身
 代孚死。永安二年,除潁川太守,遷東徐州刺史。永熙中,為車騎將軍、左光祿大夫,尋加衛大將軍。復除東徐州刺史,前後在州,為吏人所愛。卒,贈尚書右僕射、司徒公、冀州刺史。



 子子義、子智。子義學涉有父風,仕隨,至尚書右丞。



 朱元旭,字君升,本樂陵人也。頗涉子史,開解幾案。稍遷尚書度支郎中。神龜末,以郎選不精,大加沙汰。元旭與隴西辛雄、范陽祖瑩、太山羊深、西平源子恭並以才用見留。尋兼尚書右丞,仍郎中、本州中正。時關西都督蕭寶夤啟云所統十萬,食唯一月。明帝大怒,詔問所由,錄、
 令已下皆推罪元旭。入見御坐前,屈指校計,寶夤兵糧乃踰一年,事乃得釋。後遷衛將軍、左光祿大夫。天平中,復拜尚書左丞。既無風操,俛仰隨俗,性多機數,自容而已。於時朝廷分汲郡河內二界挾河之地立義州,置關西歸款戶,除元旭義州刺史,卒官。



 論曰:壽春形勝,南鄭要險,乃建鄴之肩髀,成都之喉嗌。裴叔業、夏侯道遷體運知機,翻然鵲起,舉地而來,功誠兩茂,其以大啟茅賦,兼列旄,固其宜矣。植不恒其德,器小志大,斯所以顛覆也。衍才行將略,不遂其終,惜哉!李、席、王、江雖復因人成事,亦為果決之士。淳于誕好立
 功名,有志竟不遂也。文秀不回,有死節之氣,非直身蒙嘉禮,遂乃子免刑戮,在我欲其罵人,忠義可不勉也?



 張讜觀機委質,篤恤流離,亦仁智矣。李苗以文武幹局,沉毅過人,臨難慨然,奮斯大節,蹈忠履義,沒而後已,仁必有勇,其斯人之謂乎!劉藻、傅永,豎眼文武器幹,知名於時。豎眼加以撫邊導俗,風化尤美,方之二子,固已優乎,抑又魏世良牧。張烈早有氣尚,名輩見知,趣舍沉浮,俱至顯達,雅道正路,其殆病諸。李、路器尚所及,俱可觀者。象風彩詞涉,亦當年之俊乂。房亮、曹世表、潘永基、硃元旭拔萃從官,咸享名器,各有由也。



\end{pinyinscope}