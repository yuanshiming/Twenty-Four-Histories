\article{卷八十九列傳第七十七 藝術上}

\begin{pinyinscope}

 晁
 崇張深殷紹王早耿玄劉靈助沙門靈遠李順興檀特師由吾道榮張遠游顏惡頭王春信都芳宋景業許遵麴紹吳遵世趙輔和皇甫玉解法選魏寧綦母懷文張子信陸法和
 蔣昇強練庾季才子質盧太翼耿詢來和蕭吉楊伯醜臨孝恭劉祐張胄玄夫陰陽所以正時日,順氣序者也;卜筮所以決嫌疑,定猶豫者也;醫巫所以禦妖邪,養性命者也;音律所以和人神,節哀樂者也;相術所以辨貴賤,明分理者也;技巧所以利器用,濟艱難者也。此皆聖人無心,因人設教,救恤災患,禁止淫邪,自三五哲王,其所由來久矣。昔之言陰陽者,則有箕子、裨灶、梓慎、子韋;曉音律者,則師曠、師摯、伯牙、杜夔;敘卜筮,則史扁、史蘇、嚴君平、司馬季主;論
 相術,則內史叔服、姑布子卿、唐舉、許負;語醫巫則文摯、扁鵲、季咸、華佗;其巧思,則奚仲、墨翟、張平子、馬德衡。凡此諸君,莫不探靈入妙,理洞精微。



 或弘道以濟時,或隱身以利物,深不可測,固無得而稱矣。近古涉乎斯術者,鮮有存夫貞一,多肆其淫僻,厚誣天道。或變亂陰陽,曲成君欲;或假託神怪,熒惑人心。遂令時俗妖訛,不獲返其真性,身罹災毒,莫得壽終而死。藝成而下,意在茲乎!



 歷觀經史百家之言,無不存夫藝術。或敘其玄妙,或記其迂誕,非徒用廣異聞,將以明乎勸戒。是以後來作者,咸相祖述。



 自魏至隋,年移四代,至於遊心藝術,亦為多
 矣。在魏,則敘晁崇、張深、殷紹、王早、耿玄、劉靈助、江式、周澹、李脩、徐謇、王顯、崔彧、蔣少遊,以為《術藝傳》;在齊,則有由吾道榮、王春、信都芳、宋景業、許遵、吳遵世、趙輔和、皇甫玉、解法選、魏寧、綦母懷文、張子信、馬嗣明為《方伎傳》;在周,則有冀俊、蔣升、姚僧垣、黎景熙、趙文深、褚該、強練,以為《藝術傳》;在隋,則有庾季才、盧太翼、耿詢、韋鼎、來和、蕭吉、張胄玄、許智藏、萬寶常為《藝術傳》。今檢江式、崔彧、冀俊、黎景熙、趙文深各編別傳。又檢得沙門靈遠、李順興、檀特師、顏惡頭,並以陸法和、徐之才、何稠附此篇,以備《藝術傳》。前代著述,皆混而書之。但道茍不同,則其流
 異。今各因其事,以類區分。先載天文數術,次載醫方伎巧云。



 晁崇,字子業,遼東襄平人也。善天文術數,為慕容垂太史郎。從慕容寶敗於參合,為道武所獲。從平中原,拜太史令。詔崇造渾儀,遷中書侍郎,令如故。天興五年,月暈左角,崇奏,占為角蟲將死。帝既剋姚平於柴壁,以崇言之徵,遂命諸軍焚車而反。牛果大疫,輿駕所乘巨犗數百頭,亦同日斃於路側,自餘首尾相繼。



 是歲天下牛死者十七八,麋鹿亦多死。



 崇弟懿,明辯而才不及崇。以善北人語,為黃門侍郎。懿好矜容儀,被服僭度,言音類帝,
 左右每聞其聲,莫不驚悚。帝知而惡之。後其家奴告崇、懿叛,招引姚興。及興寇平陽,帝以奴言為實,執崇兄弟,並賜死。



 張深,不知何許人也。明占候。自云,嘗事苻堅,堅欲征晉,深勸不行,堅不從,果敗。又仕姚興為靈臺令,姚泓滅,入赫連昌。昌復以深及徐辯對為太史令。



 統萬平,深、辯俱見獲,以深為太史令。神二年,將討蠕蠕,深、辯皆謂不宜行,與崔浩爭於太武前。深專守常占,而不能鉤深賾遠,故不及浩。後為驃騎軍謀祭酒,著《觀象賦》,其言星文甚備,文多不載。



 又明元時,有容城令徐路,善占候,坐繫
 冀州獄。別駕崔隆宗就禁慰問之,路曰:「昨夜驛馬星流,計赦須臾應至。」隆宗先信之,遂遣人出城候焉,俄而赦至。



 又道武、明元時,太史令王亮、蘇垣,太武時,破和龍得馮弘太史令閔盛,孝文時,太史趙樊生,並知天文。後太史令趙勝、趙翼、趙洪慶、胡世榮、胡法通等二族,世業天文。又永安中,詔以恆州人高崇祖善天文,每占吉凶有驗,特除中散大夫。



 永熙中,詔通直散騎常侍孫僧化與太史胡世榮、太史令張寵、趙洪慶及中書舍人孫子良等在門下外省,校比天文書,集甘、石二家星經,及漢、魏以來二十三家經占,集五十五卷。後集諸家撮要,前後
 所上雜占,以類相從,日月、五星、二十八宿、中外官及圖,合為七十五卷。



 僧化,東莞人也。識星分,案文占以言災異,時有所中。普泰中,爾朱兆惡其多言,遂繫於廷尉,免官。永熙中,孝武帝召僧化與中散大夫孫安都共撰兵法,未就而帝入關,遂罷。元象中,死於晉陽。



 殷紹,長樂人也。達《九章》、《七曜》。太武時,為算生博士,給事東宮西曹。太安四年,上《四序堪輿》,表言:「以姚氏之時,行學伊川,遇遊遁大儒成公興,從求《九章》要術。興字廣明,自云膠東人也,山居隱跡,希在人間。興將臣到陽翟九崖巖沙門釋曇影間,興即北還。臣獨留住,依止影所,求
 請《九章》。



 影復將臣向長廣東山,就道人法穆。法穆時共影為臣開述《九章》數家雜要。復以先師和公所注黃帝《四序經》文三十六卷,合有三百二十四章,專說天地陰陽之本。



 其第一,孟序,九卷八十一章,說陰陽配合之原;第二,仲序,九卷八十一章,解四時氣王,休殺吉凶;第三,叔序,九卷八十一章,明日月辰宿,交會相生為表裏;第四,季序,九卷八十一章,具釋六甲,刑禍福德。以此經文,傳授於臣。山神禁嚴,不得齎出。尋究經年,粗舉綱要。山居險難,無以自供,不堪窘迫,心生懈怠。



 以甲寅之年,日維鶉火,感物懷歸。自爾至今,二十五載。臣前在東宮,以
 狀奏聞,奉被景穆皇帝聖詔,敕臣撰錄,集其要最。仰奉明旨,謹審先所見《四序經》文,抄撮要略,當世所須吉凶舉動,集成一卷。上至天子,下及庶人,貴賤等級,尊卑差別,吉凶所用,罔不畢備。未及內呈,先帝晏駕。依先撰錄,謹以上聞。」共《四序堪輿》遂大行於世。



 其從子玖,亦以學術著名。



 王早,勃海南皮人也。明陰陽、九宮及兵法,善風角。明元時,喪亂之後,有人詣早,求問勝術。早為設法,令各無咎,由是州里稱之。時有東莞鄭氏,執得仇人趙氏,剋明晨會宗族,當就墓所刑之。趙氏求救於早。早為占候,並授
 以一符曰:「君今且還,選取七人,令一人為行主者佩此符,於雞鳴時,伏在仇家宅東南二里。



 平旦,當有十人相隨向西北,行中有二人乘黑牛,一黑牛最在前,一黑牛應第七。



 但捉取第七者將還,事必無他。」趙氏從之,果如其言。乃是鄭氏男五父也,諸子並為其族所宗敬,故和解二家,趙氏竟免。



 後早與客清晨立於門內,遇有卒風振樹,早語客曰:「依法當有千里外急使。



 日中時,有兩匹馬,一白一赤,從西南來,至即取我,逼我不聽與妻子別。」語訖便入,召家人鄰里辭別,仍沐浴帶書囊,日中出門候使。如期,果有馬一白一赤,從州而至,即促早上馬,遂
 詣行宮。時太武圍涼州未拔,故許彥薦之。早,彥師也。



 及至,詔問何時當剋此城。早對曰:「陛下但移據西北角,三日內必剋。」帝從之,如期而剋。輿駕還都,久不雨,帝問早。早曰:「今日申時必大雨。」比至未,猶無片雲,帝召早詰之。早曰:「願更少時。」至申時,雲四合,遂大雨滂沱。早苦以疾辭,乞歸鄉里。詔許之,遂終於家。或言許彥以其術勝,恐終紡己,譎令歸之耳。



 耿玄,鉅鹿宋子人也。善卜占。有客叩門,玄在室已知其姓字,並所齎持及來問之意。其所卜筮,十中八九。別有《林占》,時或傳之。而性不和俗,時有王公欲求其筮者,玄
 則拒而不許。每云:「今既貴矣,何所求而復卜也?欲望意外乎?」



 代京法禁嚴切,王公聞之,莫不驚悚而退。故玄多見憎忿,不為貴勝所親。官止鉅鹿太守。



 劉靈助,燕郡人也。師事范陽劉弁,而粗疏無賴。或時負販,或復劫盜,賣術於市。後事爾朱榮,榮信卜筮,靈助所占屢中,遂被親待,為榮府功曹參軍。建義初,榮於河陰害王公卿士。時奉車都尉盧道虔兄弟,亦相率朝行宮,靈助以其州里,衛護之。由是朝士與諸盧相隨免害者數十人。榮入京師,超拜光祿大夫,封長子縣公。從上黨王元天穆討邢杲。



 元顥入洛,天穆度河,會爾朱榮於太
 行。及將攻河內,令靈助筮之。靈助曰:「未時必剋。」時已向中,士眾疲怠,靈助曰:「時將至矣!」榮鼓之,即便剋陷。



 及至北中,榮攻城不獲。以時盛暑,議欲且還,以待秋涼。莊帝詔靈助筮之。靈助曰:「必破,十八九間。」果如言。車駕還宮,進爵燕郡公,贈其父僧安為幽州刺史。尋兼尚書左僕射,慰勞幽州流人。北還,與都督侯深等討葛榮餘黨韓婁,滅之於薊。仍釐州務,又為幽、並、營、安四州行臺。及爾朱榮死,莊帝幽崩,靈助本寒微,一朝至此,自謂方術堪能動眾,又以爾朱有誅滅之兆,遂自號燕王、大行臺,為莊帝舉義兵。馴養大鳥,稱為己瑞,妄說圖讖,言劉氏當
 王。又云:「欲知避世入鳥村。」遂刻氈為人象,書桃木為符書,作詭道厭祝法,人多信之。時西河人紇豆陵步籓,舉兵逼晉陽,爾朱兆頻戰不利。故靈助唱言:「爾朱自然當滅,不須我兵。」由是幽、瀛、滄、冀人悉從之。從之者,夜舉火為號;不舉火者,諸村共屠之。普泰元年,率眾至博陵之安國城,與叱列延慶、侯深,爾朱羽生等戰。戰敗被禽,斬於定州,傳首洛陽,支分其體。



 初,靈助每云:「三月末,我必入定州,爾朱亦必滅。」及將戰,靈助自筮,卦不吉,以手折蓍棄之地,云:「此何知!」尋見禽。果以三月入定州。而齊神武以明年閏三月,滅兆等於韓陵山。永熙二年,贈尚書
 左僕射、開府儀同三司、幽州刺史,謚曰恭。



 時又有沙門靈遠者,不知何許人,有道術。嘗言爾朱榮成敗,預知其時。又言代魏者齊。葛榮聞之,故自號齊。及齊神武至信都,靈遠與勃海李嵩來謁。神武待靈遠以殊禮,問其天文人事。對曰:「齊當興,東海出天子。今王據勃海,是齊地。



 又太白與月並,宜速用兵,遲則不吉。」靈遠後罷道,姓荊字次德。求之,不知所在。



 李順興,京兆杜陵人也。年十餘,乍愚乍智,時莫識之。其言未來事,時有中者。盛冬單布衣,跣行冰上及入洗浴,略不患寒。家嘗為齋,方食,器用不周。順興言:「昆明池中
 有大荷葉,可取盛餅食。」其所居去池十數里,日不移影,順興負荷葉而歸,腳猶泥,舉坐驚異。後稍出城市,常冠道士冠,人有憶者,不過數日,輒至其家。號為李練。好飲酒,但不至醉。貴賤並敬之。得人所施,輒散乞貧人。



 蕭寶夤反,召順興問曰:「朕王可幾年?」對曰:「為天子自有百年者,十年者,一年者,百日者,事由可知。」及寶夤敗,裁百日也。有侯終德者,寶夤之黨。



 寶夤敗後,收集反者。順興稱其必敗,德乃棒殺順興,置城隍中。頃之,起活如初。



 後賀拔岳北征,順興與魏收書,上為毛鴻賓等九人姓名者悉放貴還。順興從後提一河東酒缸,以繩繫之,於城巷
 牽行。俄而蒲阪降。又無何,至太傅梁覽家庭中臥,以布衫倒覆身上。後覽於趙崔反,通使東魏,事泄被誅,覽以衣倒覆,果如順興之形。周文嘗至溫泉,順興求乞溫泉東間驪山下二畝地,周文曰:「李練用此何為?」



 對曰:「有用。」未幾,至溫湯遇患,卒於其地。



 初,大統十三年,順興謂周文曰:「可於沙苑北作一老君象,面向北,作笑狀。」



 周文曰:「何為?」答曰:「令笑破蠕蠕。」時甚惑,未解其意。及蠕蠕國滅,周文憶語,遂作順興象於老君側。



 檀特師者,名惠豐,身為比丘,不知何處人。飲酒啖肉,語默無常,逆論來事,後皆如言。居於涼州,宇文仲和為刺
 史,請之至州內,歷觀廄庫。乃云:「何意畜他官馬官物!」仲和怒,不聽住涼州。未幾,仲和拒不受代,朝廷令獨孤信禽之,仲和身死,資財沒官。周文遣書召之,檀特發至岐州,會齊神武來寇玉壁,檀特曰:「狗豈能至龍門也?」神武果不至龍門而還。侯景未叛東魏之前,忽捉一杖,杖頭刻為獼猴。令其面常向西,日夜弄之。又索一角弓,牽挽之。俄而景啟降,尋復背叛,人皆以為驗。



 至大統十七年春初,忽著一布帽,周文左右驚問之。檀特曰:「汝亦著,王亦著也。」至三月而魏文帝崩。復取一白絹帽著之,左右復問之。檀特云:「汝亦著,王亦著也。」未幾,丞相夫人薨。後
 又著白絹帽,左右復問之。云:「汝不著,王亦著也。」尋而丞相第二兒武邑公薨。其事驗多如此也。俄而疾死。



 由吾道榮,瑯琊沐陽人也。少為道士,入長白山、太山,又遊燕、趙間。聞晉陽有人大明法術,乃尋之。是人為人家傭力,無名者,久求訪始得。其人道家,符水禁咒、陰陽歷數、天文藥性,無不通解。以道榮好尚,乃悉授之。歲餘,是人謂榮云:「我本恆岳仙人,有少罪過,為天官所謫。今限滿將歸,卿宜送吾至汾水。」



 及至汾河,遇水暴長,橋壞,船渡艱難。是人乃臨水禹步,以一符投水中,流便絕。



 俄頃,水積將至天。是人徐自沙石上渡。唯道榮見其如是,傍
 人咸云:「水如此長,此人遂能浮過。」共驚異之。如此法,道榮所不得也。



 道榮仍歸本郡,隱於瑯邪山中,辟穀餌松茯苓,求長生之祕。又善洞視,蕭軌等之敗於江南,其日,道榮言之如目見。其後鄉人從役得歸者,勘問敗時形勢,與道榮所說符同。尋為文宣追往晉陽,道榮恆野宿,不入逆旅。至遼陽山中,夜初馬驚,有猛獸去馬止十餘步,所追人及防援者並驚怖將走。道榮徐以杖畫地成火坑,猛獸遽走。道榮至晉陽,文宣見之甚悅。後歸鄉里。隋開皇初,備禮徵辟,授上儀同三司、諫議大夫、沐陽縣公。從晉王平陳還,苦辭歸。至鄉卒,年八十五。



 又有張
 遠遊者,文宣時,令與諸術士合九轉金丹。及成,帝置之玉匣云:「我貪人間作樂,不能飛上天,待臨死時取服。」



 顏惡頭,章武郡人也。妙於《易》筮。遊州市觀卜,有婦人負囊粟來卜,歷七人,皆不中而強索其粟,惡頭尤之。卜者曰:「君若能中,何不為卜?」惡頭因筮之,曰:「登高臨下水泂泂,唯聞人聲不見形。」婦人曰:「妊身已七月矣,向井上汲水,忽聞胎聲,故卜。」惡頭曰:「吉,十月三十日有一男子。」詣卜者乃驚服曰:「是顏生邪?」相與具羊酒謝焉。有人以三月十三日詣惡頭求卜,遇《兌》之《履》。惡頭占曰:「君卜父,父已亡,當上天,聞哭聲,忽復蘇,而有言。」



 其人曰:「父臥疾三
 年矣,昨日雞鳴時氣盡,舉家大哭。父忽驚寤云:『我死,有三尺人來迎,欲升天,聞哭聲,遂墜地。』」惡頭曰:「更三日,當永去。」果如言。人問其故,惡頭曰:「《兌》上天下土,是今日庚辛本宮火,故知卜父。今三月,土入墓,又見宗廟爻發,故知死。變見生氣,故知蘇。《兌》為口,主音聲,故知哭。《兌》變為《乾》,《乾》天也,故升天。《兌》為言,故父言。故知有言。未化入戍為土,三月土墓,戍又是本宮鬼墓,未後三日至戍,故知三日復死。」



 惡頭又語人曰:「長樂王某年某月某日當為天子。」有人姓張,聞其言,數以寶物獻之,豫乞東益州刺史。及期,果為天子,擢張用之。惡頭自言厄在彭城。後遊
 東都,逢彭城王爾朱仲遠將伐齊神武於鄴,召惡頭令筮。惡頭野生,不知避忌,高聲言:「大惡。」仲遠怒其沮眾,斬之。



 王春,河東安邑人也。少精《易》占,明陰陽風角,齊神武引為館客。韓陵之戰,四面受敵,從寅至午,三合三離,將士皆懼。神武將退軍,春叩馬諫曰:「比至未時,必當大捷。」遽縛其子詣軍門為質,若不勝請斬之。賊果大敗。後從征討,恒令占卜,其言多中。位東徐州刺史,賜爵安夷縣公。卒,贈秦州刺史。



 信都芳,字玉琳,河間人也。少明算術,兼有巧思,每精心
 研究,或墜坑坎。



 常語人云:「算歷玄妙,機巧精微,我每一沈思,不聞雷霆之聲也。」其用心如此。



 後為安豐王延明召入賓館。有江南人祖恆者,先於邊境被獲,在延明家,舊明算歷,而不為王所待。芳諫王禮遇之。恆後還,留諸法授芳,由是彌復精密。延明家有群書,欲抄集《五經》算事為《五經宗》,及古今樂事為《樂書》,又聚渾天、欹器、地動、銅烏、漏刻、候風諸巧事,并圖畫為《器準》,並令芳算之。會延明南奔,芳乃自撰注。



 後隱於并州樂平之東山,太守慕容保樂聞而召之,芳不得已而見焉。於是保樂弟紹宗薦之於齊神武,為館客,授中外府田曹參軍。芳性清
 儉質樸,不與物和。紹宗給其羸馬,不肯乘騎;夜遣婢侍以試之,芳忿呼毆擊,不聽近己。狷介自守,無求於物。後亦注重差、勾股,復撰《史宗》。



 芳精專不已,又多所窺涉。丞相倉曹祖珽謂芳曰:「律管吹灰,術甚微妙,絕來既久,吾思所不至,卿試思之。」芳留意十數日,便報珽云:「吾得之矣,然終須河內葭莩灰。」祖對試之,無驗。後得河內灰,用術,應節便飛,餘灰即不動也。



 為時所重,竟不行用,故此法遂絕。



 又著《樂書》、《遁甲經》、《四術周髀宗》。其序曰:「漢成帝時,學者問蓋天,楊雄曰:『蓋哉,未幾也。』問渾天,曰:『落下閎為之,鮮於妄人度之,耿中丞象之,幾乎,莫之息矣。』此言
 蓋差而渾密也。蓋器測影而造,用之日久,不同於祖,故云『未幾也』。渾器量天而作,乾坤大象,隱見難變,故云『幾乎』。



 是時,太史令尹咸窮研晷蓋,易古周法,雄乃見之,以為難也。自昔周公定影王城,至漢朝,蓋器一改焉。渾天覆觀,以《靈憲》為文;蓋天仰觀,以《周髀》為法,覆仰難殊,大歸是一。古之人制者,所表天效玄象。芳以渾算精微,術機萬首,故約本為之省要,凡述二篇,合六法,名《四術周髀宗》。」



 又上黨李業興撰新歷,自以為長於趙匪又、何承天、祖沖之三家,芳難業興五闕。又私撰曆書,名曰《靈憲曆》,算月頻大頻小,食必以朔,證據甚甄明。每云:「何承天亦
 為此法,而不能精。《靈憲》若成,必當百代無異議者。」書未成而卒。



 宋景業,廣宗人也。明《周易》,為陰陽緯候之學,兼明歷數。魏武定初,任北平太守。齊文宣作相,在晉陽。景業因高德政上言:「《易稽覽圖》曰:『《鼎》,五月,聖人君,天與延年齒,東北水中,庶人王,高得之。』謹案:東北水,謂勃海也。高得之,明高氏得天下也。」時魏武定八年三月也。高德政、徐之才並勸文宣應天受禪,乃之鄴。至平城都,諸大臣沮計,將還。賀拔仁等又云:「宋景業誤王,宜斬之以謝天下。」帝曰:「宋景業當為帝王師,何可殺也?」還至并州,文宣令景
 業筮,遇《乾》之《鼎》。景業曰:「乾,君也,天也。《易》曰:『時乘六龍,以御天。』《鼎》,五月卦也,宜以仲夏吉辰,順天受禪。」或曰:「陰陽書,五月不可入官。犯之,卒於其位。」景業曰:「此乃大吉,王為天子,無復下期,豈得不終於其位?」帝大悅。天保初,封長城縣子,受詔撰《天保歷》,李廣為之序。



 許遵,高陽新城人也。明《易》善筮,兼曉天文、風角、占相、逆刺,其驗若神。齊神武引為館客。自言祿命不富貴,不橫死,是以任性疏誕,多所犯忤,神武常容借之。芒陰之役,遵謂李業興曰:「賊為水陳,我為火陳,水勝火,我必敗。」



 果如其言。清河王岳以遵為開府記室。岳後將救江陵,遵
 曰:「此行必致後凶,宜辭疾勿去。」岳曰:「勢不免去,正當與君同行。」遵曰:「遵好與生人相隨,不欲與死人同路。」岳強給其馬以行。至都,尋喪。三臺初成,文宣宴會尚書以上,三日不出。許遵妻季氏憂之,以問遵。遵曰:「明日當得三百匹絹。」季氏曰:「若然,當奉三束。遵曰:「不滿十匹。」既而皆如言。文宣無道日盛,遵語人曰:「多折算來,吾筮此狂夫何時得死。」於是布算滿床,大言曰:「不出冬初,我乃不見。」文宣以十月崩,遵果以九月死。



 子暉,亦學術數。遵謂曰:「汝聰明不及我,不勞多學。」唯授以婦人產法,豫言男女及產日,無不中。武成時,以此數獲賞焉。



 又有滎陽麴紹
 者,亦善占。侯景欲試之,使與郭生俱卜二伏牛何者先起。」卜得火兆,郭生曰:「赤牛先起。紹曰:「青牛先起。」景問其故,郭生曰:「火色赤,故知赤牛先起。」紹曰:「火將然,煙先起。煙上色青,故知青牛起。」既而如紹言。



 吳遵世,字季緒,勃海人也。少學《易》。入恆山,忽見一老翁,授之開心符。



 遵世跪,水吞之,遂明占卜。後出遊京洛,以卜筮知名。魏孝武帝之將即位,使之筮,遇《否》之《萃》,曰:「先否後喜。」帝曰:「喜在何時?」遵世曰:「剛決柔,則春末夏初也。」又筮,遇《明夷》之《賁》,曰:「初登於天,後入於地。



 若能敬始慎終,不失法度,無憂入地矣。」終如其言。後齊文襄引為大
 將軍府墨曹參軍。從遊東山,有雲起,恐雨廢射,戲使筮。遇《剝》,李業興云:「坤上艮下,《剝》。艮為山,山出雲,故知有雨。」遵世云:「坤為地,土制水,故知無雨。」



 文襄使崔暹書之云:「遵世若著,賞絹十匹;不著,罰杖十。業興若著,無賞;不著,罰杖十。」業興曰:「同是著,何獨無賞?」文襄曰:「遵世著,會我意,故賞也。」須臾雲散,二人各受賞罰。皇建中,武成以丞相在鄴下居守,自致猜疑,甚懷憂懼。謀起兵,每宿輒令遵世筮。遵世云:「自有大慶。」由是不決。俄而趙郡王等奉太后令,以遺詔追武成。更令筮之。遵世云:「比已作十餘卦,其占自然有天下之徵。」及即位,除中書舍人,固辭老
 疾,授中散大夫。和士開封王,妻元氏無子,以側室長孫為妃,令遵世筮。遵世云:「此卦偶與占同。」乃出其占書云:「元氏無子,長孫為妃。」士開喜於妙中,於是起叫而舞。遵世著《易林雜占》百餘卷。後預尉遲迥亂,死焉。



 趙輔和,清都臨漳人也。少以明《易》善筮為齊神武館客。神武崩於晉陽,葬有日矣,文襄令文宣與吳遵世等擇地,頻卜不吉。又至一所,筮遇《革》,咸云凶。



 輔和少年,最在眾人後,進云:「《革卦》於天下人皆凶,唯王家用之大吉。《革彖辭》云『湯武革命,應天順人。』」文宣遽登車,顧云:「以此地為定。」即義平陵也。有人父為刺史,得書云疾。是人詣館,
 別託相知者筮。遇《泰》,筮者云:「此卦甚吉。」是人出後,輔和謂筮者云:「《泰》,乾下坤上,則父入土矣,豈得言吉。」果凶問至。有人父疾,託輔和筮,遇《乾》之《晉》,慰諭令去。後告人云:「《乾》之遊魂。乾為天,為父,父變為魂,而升於天,能無死乎?」亦如其言。大寧、武平中,筮後宮誕男女及時日,多中,遂至通直常侍。入周,亦為儀同。隋開皇中,卒。



 皇甫玉,不知何許人也,善相人。齊文襄之自潁川歸,文宣從後。玉於傍縱觀,謂人曰:「大將軍不作物。」指文宣曰:「會道北垂鼻洟者。」及文宣即位,試玉相術,故以帛巾襪其眼,使歷摸諸人。至文宣曰:「此最大達官。」於任城王曰:「
 當至丞相。」於常山、長廣二王,並曰:「亦貴。」至石動桶曰:「此弄癡人。」



 至二供膳曰:「正得好飲食而已。」玉嘗為高歸彥相曰:「位極人臣,但莫反。」



 歸彥曰:「我何為反?」玉曰:「公有反骨。」孝昭賜趙郡王十死不問,王喜曰:「皇甫玉相臣,云當惡死,今復何慮?」帝以玉輒為諸王相,心不平之。玉謂其妻曰:「殿上者不過二年。」妻以告舍人斛斯洪慶妻,洪慶以啟帝。怒曰:「向婦女小兒評論萬乘主!」敕召玉。玉每照鏡,自言兵死。及被召,謂妻曰:「我今去,不迴,若過日午時,當得活。」既至正中,遂斬之。



 文襄時,有吳士,雙盲,妙於聲。文襄歷試之,聞劉桃枝聲曰:「有所繫屬,然當大富貴。王
 侯將相,多死其手。譬如鷹犬,為人所使。」聞趙道德聲曰:「亦繫屬人,富貴翕赫,不及前人。」聞侯呂芬聲,與道德相似。聞太原公聲曰:「當為人主。」聞文襄聲,不動。崔暹私掏之,乃謬言:「亦國主也。」文襄以為我家群奴猶極貴,況吾身也。



 又時有御史賈子儒,亦能相人。崔暹嘗將子儒私視文襄,子儒曰:「人有七尺之形,不如一尺之面;一之面,不如一寸之眼。大將軍臉薄眄速,非帝王相也。」



 竟如言。



 齊代善相者,有館客趙瓊。其婦叔寄弓,弓已轉在人處,盡知之。時人疑其別有假託,不然,則姑布子卿不如也。



 初,魏正始前,有沙門學相,遊懷朔,舉目見人,皆有富
 貴之表。以為必無此理,燔其書。而後皆如言,乃知相法不虛也。



 解法選,河內人也。少明相術,又受《易》於權會,筮亦頗工。陳郡袁叔德以太子庶子出行博陵太守,不願之官,以親老言於執政楊愔。愔語云:「既非正除,尋當遣代。」叔德意欲留尊累在京,令法選占。云:「不踰三年,得代,終不還也。」



 勸其盡家而行。又為叔德相云:「公邑邑,終為吏部尚書,鑒照人物。」後皆如言。



 又頻為和士開相中,士開牒為開府行參軍。



 魏寧,鉅鹿人也。以善推祿命,徵為館客。武成以己生年
 月,託為異人,問之。



 寧曰:「極富貴,今年入墓。」武成驚曰:「是我!」寧變辭曰:「若帝王,自有法。」



 又有陽子術語人曰:「謠言:盧十六,雉十四,犍子拍頭三十二。且四八天之大數,太上之祚,恐不過此。」既而武成崩,年三十二。



 綦母懷文,不知何許人也,以道術事齊神武。武定初,齊軍戰芒山,時齊軍旗幟盡赤,西軍盡黑,懷文曰:「赤,火色;黑。水色。水能滅火,不宜以赤對黑。



 土勝水,宜改為黃。」神武遂改為赭黃,所謂河陽幡者也。



 懷文造宿鐵刀,其法,燒生鐵精以重柔鋌,數宿則成剛。以柔鐵為刀脊,浴以五牲之溺,淬以五牲之脂,斬甲過三十札。今襄國冶家
 所鑄宿柔鋌,是其遺法,作刀猶甚快利,但不能頓截三十札也。懷文又云:「廣平郡南幹子城,是干將鑄劍處,其土可瑩刀。」每云:「昔在晉陽為監館,館中有一蠕蠕客,同館胡沙門指語懷文云:『此人別有異算術。』仍指庭中一棗樹云:『令其布算子,即知其實數。』乃試之,并辨若干純赤,若干赤白相半。於是剝數之,唯少一子。算者曰:『必不少,但更撼之。』果落一實。」懷文位信州刺史。



 又有孫正言謂人曰:「我昔聞曹普演有言:『高王諸兒,阿保當為天子,至高德之承之,當滅。』阿保,謂天保也;德之,謂德昌也;滅年號承光,即承之矣。」



 張子信,河內人也。頗涉文學,少以醫術知名。恆隱白鹿山,時出遊京邑,甚為魏收、崔季舒所重。大寧中,徵為尚藥典御。武平初,又以太中大夫征之,聽其所志,還山。又善《易》筮及風角之術。武衛奚永洛與子信對坐,有鵲鳴庭樹,鬥而墮焉。子信曰:「不善。向夕,當有風從西南來,歷此樹,拂堂角,則有口舌事。



 今夜有人喚,必不可往,雖敕亦以病辭。」子信去後,果有風如其言。是夜,瑯邪王五使切召永洛,且云:「敕喚。」永洛欲起,其妻苦留之,稱墜馬腰折,不堪動。



 詰朝而難作。子信,齊亡卒。



 陸法和,不知何許人也。隱於江陵百里洲,衣食居處,一
 與戒行沙門同。耆老自幼見之,容色常定,人莫能測也。或謂出自嵩高,遍游遐邇。既入荊州汶陽郡高要縣之紫石山,無故舍所居山,俄有蠻賊文道期之亂,時人以為預見萌兆。



 及侯景始告降於梁,法和謂南郡硃元英曰:「貧道共檀越擊侯景去。」元英曰:「侯景為國立效,師云擊之何也?」法和曰:「正自如此。」及景度江,法和時在青溪山,元英往問曰:「景今圍城,其事云何?」法和曰:「凡人取果,宜待熟時。」



 固問之,曰:「亦克,亦不克。」景遣將任約擊梁湘東王於江陵,法和乃詣湘東乞征約。召諸蠻弟子八百人在江津,二日便發。湘東遣胡僧祐領千餘人與同行。
 法和
 登艦,大笑曰:「無量兵馬。」江陵多神祠,人俗恒所祈禱。自法和軍出,無復一驗,人以為神皆從行故也。至赤沙湖,與約相對。法和乘輕船,不介胄,沿流而下,去約軍一里乃還。謂將士曰:「聊觀彼龍睡不動,吾軍之龍,甚自踴躍,即攻之。



 若得彼明日,當不損客主一人而破賊,然有惡處。」遂縱火船,而逆風不便,法和執白羽扇麾風,風即返。約眾皆見梁兵步於水上,於是大潰,皆投水。約逃竄不知所之,法和曰:「明日午時當得。」及期而未得,人問之,法和曰:「吾前於此洲水乾時建一剎,語檀越等:此雖為剎,實是賊標。今何不向標下求賊也?」如其言,果於水
 中見約抱剎,仰頭裁出鼻,遂禽之。約言:「求就師目前死。」法和曰:「檀越有相,必不兵死。且於王有緣,決無他慮。王於後當得檀越力耳。」湘東果釋用為郡守。及魏圍江陵,約以兵赴救,力戰焉。



 法和既平約,往進見王僧辯於巴陵,謂曰:「貧道已卻侯景一臂,其更何能為?



 檀越宜即逐取。」乃請還。謂湘東王曰:「侯景自然平矣,無足可慮。蜀賊將至,法和請守巫峽待之。」乃縱諸軍而往,親運石以填江。三日,水遂不流,橫之以鐵鎖。武陵王紀果遣蜀兵來度,峽口勢蹙,進退不可,王琳與法和經略,一戰而殄之。



 軍次白帝,謂人曰:「諸葛孔明可謂為名將,吾自見之。此城
 旁有其埋弩箭鏃一斛許。」因插表令掘之,如其言。又嘗至襄陽城北大樹下,畫地方二尺,令弟子掘之。得一龜,長尺半,以杖叩之曰:「汝欲出,不能得,已數百歲。不逢我者,豈見天日乎?」為授《三歸》,龜乃入草。初,八疊山多惡疾人,法和為采藥療之,不過三服,皆差,即求為弟子。山中多毒蟲猛獸,法和授其禁戒,不復噬蜇。所泊江湖,必於峰側結表,云此處放生。漁者皆無所得。才或少獲,輒有大風雷,船人懼而放之,風雨乃定。晚雖將兵,猶禁諸軍漁捕,有竊違者,中夜猛獸必來欲噬之,或亡其船纜。有小弟子戲截蛇頭,來詣法和。法和曰:「汝何意殺!」因指以
 示之,弟子乃見蛇頭齚褲襠而不落。法和使懺悔,為蛇作功德。又有人以牛試刀,一下而頭斷,來詣法和。法和曰:「有一斷頭牛,就卿徵命殊急,若不為作功德,一月內報至。」其人弗信,少日果死。法和又為人置宅圖墓以避禍求福。嘗謂人曰:「勿繫馬於碓。」其人行過鄉曲,門側有碓,因繫馬於其柱。入門中,憶法和戒,走出將解之,馬已斃矣。



 梁元帝以法和為都督、郢州刺史,封江乘縣公。法和不稱臣,其啟文朱印名上,自稱居士,後稱司徒。梁元帝謂其僕射王褒曰:「我未嘗有意用陸為三公,而自稱,何也?」褒曰:「彼既以道術自命,容是先知。」梁元帝以法和
 功業稍重,遂就加司徒,都督、刺史如故。部曲數千人,通呼為弟子。唯以道術為化,不以法獄加人。



 又列肆之所,不立市丞,牧佐之法,無人領受。但以空檻籥在道間,上開一孔以受錢。賈客店人,隨貨多少,計其估限,自委檻中。所掌之司,夕方開取,條其孔目,輸之於庫。又法和平常言若不出口,時有所論,則雄辯無敵,然猶帶蠻音。善為攻戰具。



 在江夏,大聚兵艦,欲襲襄陽而入武關,梁元帝使止之。法和曰:「法和是求佛之人,尚不希釋梵天王坐處,豈規王位?但於空王佛所與主上有香火因緣,見主上應有報至,故救援耳。今既被疑,是業定不可改也。」
 於是設供食,具大食追薄餅。及魏舉兵,法和自郢入漢口,將赴江陵,梁元帝使人逆之曰:「此自能破賊,師但鎮郢州,不須動也。」法和乃還州,堊其城門,著粗白布衫,褲布邪巾,大繩束腰,坐葦席,終日乃脫之。及聞梁元敗滅,復取前凶服著之,哭泣受弔。梁人入魏,果見食追餅焉。法和始於百里洲造壽王寺。既架佛殿,更截梁柱,曰:「後四十許年,佛法當遭雷雹,此寺幽僻,可以免難。」及魏平荊州,宮室焚燼,總管欲發取壽王佛殿,嫌其材短,乃停。後周氏滅佛法,此寺隔在陳境,故不及難。



 天保六年春,清河王岳進軍臨江,法和舉州入齊。文宣以法和為大都督、
 十州諸軍事、太尉公、西南大都督、五州諸軍事、荊州刺史,安湘郡公宋蒞為郢州刺史,官爵如故。蒞弟簉為散騎常侍、儀同三司、湘州刺史、義興縣公。梁將侯瑱來逼江夏,齊軍棄城而退,法和與宋蒞兄弟入朝。文宣聞其有奇術,虛心相見之。備三公鹵簿,於城南十二里供帳以待之。法和遙見鄴城,下馬禹步。辛術謂曰:「公既萬里歸誠,主上虛心相待,何作此術?」法和手持香爐,步從路車至於館。明日引見,給通憲油絡網車,仗身百人。詣闕通名,不稱官爵,不稱臣,但云荊山居士。文宣宴法和及其徒屬於昭陽殿,賜法和錢百萬、物萬段、甲第一區、田
 一百頃、奴婢二百人,生資什物稱是;宋蒞千段;其餘儀同、刺史以下各有差。法和所得奴婢,盡免之,曰:「各隨緣去。」錢帛散施,一日便盡。以官所賜宅營佛寺,自居一房,與凡人無異。三年間再為太尉,世猶謂之居士。無疾,而告弟子死期。至時,燒香禮佛,坐繩床而終。浴訖將殮,屍小縮止三尺許。文宣令開棺而視之,空棺而已。



 法和書其所居屋壁而塗之,及剝落,有文曰:「十年天子為尚可,百日天子急如火,周年天子遞代坐。」又曰:「一母生三天,兩天共五年。」說者以為婁太后生三天子,自孝昭即位至武成傳位後主,共五年焉。



 法和在荊郢,有少姬,年可
 二十餘,自稱越姥,身披法服,不肯嫁娶。恆隨法和東西,或與其私通,十有餘年。今者賜棄,別更他淫。有司考驗,並實。越姥因爾改適,生子數人。



 蔣昇,字鳳起,楚國平河人也。少好天文玄象之學,周文雅信待之。大統三年,東魏竇泰頓軍潼關,周文出師馬牧澤。時西南有黃紫氣抱日,從未至酉。周文謂昇曰:「此何祥也?」昇曰:「西南未地,主土。土王四季,秦分。今大軍既出,喜氣下臨,必有大慶。」於是與泰戰,禽之。自後遂降河東,剋弘農,破沙苑,由此愈被親禮。九年,高仲密以北豫州來附,周文欲遣兵援之。昇曰:「春王在東,熒惑又在井
 鬼分,行軍非便。」周文不從。軍至芒山,不利而還。太師賀拔勝怒曰:「蔣昇罪合萬死!」周文曰:「蔣升固諫曰:『師出不利。』此敗也,孤自取之。」



 恭帝元年,以前後功,授車騎大將軍、儀同三司,封高城縣子。後除大中大夫,以年老請致事。詔許之,加定州刺史,卒於家。



 強練,不知何許人也,亦不知其名字。先是李順興語默不恆,好言未然之事,當時號為李練,世人以強類之,故亦呼為練焉。容貌長壯,有異於人,神情敞怳,莫之能測。意欲有所說,逢人輒言;若值其不欲言,縱苦加祈請,不相酬答。初聞其言,略不可解,事過後,往往有驗。恆寄住
 諸佛寺,好行人家,兼歷造王公邸第。



 所至,人皆敬信之。晉公護未誅前,練曾手持一瓠,到護第門外抵破曰:「瓠破子苦。」時柱國、平高公侯伏龍恩深被任委,強練至龍恩宅,呼其妻元氏及其妾媵並婢僕等,並令連席而坐。諸人以逼夫人,苦辭不肯。強練曰:「汝等一例人耳,何有貴賤。」遂逼就坐。未幾而護誅,諸子並死;龍恩亦伏法,仍籍沒其家。建德中,每夜上街衢邊樹,大哭釋迦牟尼佛,或至申旦。如此者累月,聲甚哀苦。俄而廢佛、道二教。大象末,又以一無底囊,歷長安市肆告乞,市人爭以米麥遺之。強練張囊受之,隨即漏之於地。人或問之,強練曰:「
 但欲使諸人見盛空耳。」至隋開皇初,果移都於龍首山,城遂空廢。後莫知其所終。



 又有蜀郡衛元嵩者,亦好言將來事,蓋江左寶誌之流。天和中,遂著詩,預論周隋廢興及皇家受命,并有徵驗。尤不信釋教,嘗上疏極論之。



 庾季才,字叔弈,新野人也。八世祖滔,隨晉元帝過江,官至散騎常侍,封遂昌侯,因家於南郡江陵縣。祖詵,《南史》有傳。父曼倩,光祿卿。季才幼穎悟,八歲誦《尚書》,十二通《易》,好占玄象,居喪以孝聞。梁湘東王繹引授外兵參軍。西臺建,累遷中書郎,領太史,封宣昌縣伯。季才固辭太史,梁元帝曰:「漢司馬遷歷世居掌,魏高堂隆猶領此職,
 卿何憚焉!」帝亦頗明星歷,謂曰:「朕猶慮禍起蕭墻。」季才曰:「秦將入郢,陛下宜留重臣,作鎮荊陜,還都以避其患。」



 帝初然之,後與吏部尚書宗懍等議,乃止。



 俄而江陵覆滅。周文帝一見,深加優禮,令參掌太史,曰:「卿宜盡誠事孤,當以富貴相答。」初,荊覆亡,衣冠士人,多沒為賤。季才散所賜物,購求親故。



 周文問:「何能若此?」季才曰:「郢都覆敗,君信有罪,縉紳何咎,皆為賤隸?



 誠竊哀之,故贖購耳。」周文乃悟曰:「微君,遂失天下之望。」因出令,免梁浮為奴婢者數千口。武定二年,與王褒、庾信同補麟趾學士,累遷稍伯大夫。後宇文護執政,問以天道徵祥,對曰:「頃上
 台有變,不利宰輔,公宜歸政天子,請老私門。」護沈吟久之,曰:「吾本意如此,但辭未獲免。」自是漸疏。及護夷滅,閱其書記,有假託符命,妄造異端者,皆誅。唯得季才兩紙,盛言緯候,宜免政歸權。



 帝謂少宗伯斛斯徵曰:「季才甚得人臣之禮。」因賜粟帛,遷太史中大夫。詔撰《靈臺秘苑》,封臨潁縣伯。宣帝嗣位,加驃騎大將軍、開府儀同三司。



 及隋文帝為丞相,嘗夜召問天時人事,季才曰:「天道精微,難可悉察。竊以人事卜之,符兆已定,季才縱言不可,公得為箕、潁事乎?」帝默然久之曰:「吾今譬騎武,誠不得下矣。」因賜以彩帛曰:「愧公此意。」大定元年正月,季才上
 言:「今月戊戌平旦,青氣如樓闕,見國城上。俄而變紫,逆風西行。《氣經》云:『天不能無雲而雨,皇王不能無氣而立。』今王氣已見,須即應之。二月,日出卯入酉,居天之正位,謂之二八之門。日者人君之象,人君正位,宜用二月。其月十三日甲子,甲為六甲之始,子為十二辰之初。甲數九,子數又九,九為天數。其日即是驚蟄,陽氣壯發之時。昔周武王以二月甲子定天下,享年八百;漢高帝以二月甲午即帝位,享年四百。故知甲子、甲午為得天數。今月甲子,宜應天受命。」上從之。



 開皇元年,授通直散騎常侍。帝將遷都,夜與高熲、蘇威二人定議。季才旦奏:「臣仰
 觀玄象,俯察圖記,龜兆允襲,必有遷都。且漢營此城,經今將八百歲,水皆鹹鹵,不甚宜人,願為遷徒計。」帝愕然,謂熲等曰:「是何神也!」遂發詔施行。賜季才絹布及進爵為公。謂曰:「朕自今已後,信有天道。」於是令季才與其子質撰《垂象》、《地形》等志。謂曰:「天道秘奧,推測多途,執見不同,不欲令外人干預此事,故令公父子共為之。」及書成奏之,賜米帛甚優。九年,出為均州刺史。時議以季才術藝精通,有詔還委舊任。以年老,頻求去職,優旨每不許。



 會張胄玄歷行,及袁充言日景長,上以問季才,因言充謬。上大怒,由是免職,給半祿歸第。所有祥異,常令人就
 家訪焉。仁壽三年,卒。


季才局量寬弘,術業優博,篤於信義,志好賓遊。常吉日良辰,與瑯邪王褒、彭城劉玨、河東裴政及宗人信等為文酒之會。次有劉臻、明克讓、柳
 \gezhu{
  巧言}
 之徒,雖後進,亦申遊款。撰《靈臺秘苑》一百二十卷,《垂象志》一百四十二卷,《地形志》八十七卷,並行於世。



 子質,字行脩。早有志尚,八歲誦梁元帝《玄覽》、《言志》等十賦,拜童子郎。仕隋,累遷隴州司馬。大業初,授太史令。操履貞懿,立言忠鯁,每有災異,必指事面陳。煬帝多忌刻,齊王暕亦被猜嫌。質子儉時為齊王屬,帝謂質曰:「汝不能一心事我,乃使兒事齊王。」由是出為合水令。八年,帝
 親伐遼東,徵至臨渝,問東伐剋不。對曰:「伐之可剋,不願陛下親行。」帝作色曰:「朕今總兵至此,豈可未見賊而自退!」質曰:「願安駕住此,命將授規,事宜在速,緩必無功。」



 帝不悅曰:「汝既難行,可住此也。」及師還,授太史令。九年,復征高麗,又問:「今段何如?」對猶執前見。帝怒曰:「我自行尚不能剋,遣人豈有成功?」帝遂行。既而楊玄感反,斛斯政奔高麗,帝大懼,遽歸。謂質曰:「卿前不許我行,當為此耳。今玄感成乎?」質曰:「今天下一家,未易可動。」帝曰:「熒惑入斗,如何?」對曰:「斗,楚分,玄感之封。今火色衰謝,終必無成。」十年,帝自西京將往東都。質諫宜鎮撫關內,使百姓歸
 農,三五年,令四海少豐,然後巡省。帝不悅。質辭疾不從,帝聞之怒,遣馳傳鎖質詣行在所。至東都下獄,竟死獄中。



 子儉,亦傳父業,兼有學識。仕歷襄武令、元德太子學士、齊王屬。義寧初,為太史令。



 盧太翼,字協昭,河間人也。本姓章仇氏。七歲詣學,日誦數千言,州里號曰神童。及長,博綜群書,尤善占候、算歷之術。隱於白鹿山,徙居林慮山茱萸澗。



 受業者自遠而至。初無所拒,後憚其煩,逃於五臺山。地多藥物,與弟子數人,廬於巖下,以為神仙可致。隋太子勇聞而召之。太翼知太子必不為嗣,謂所親曰:「吾拘逼而來,不知所稅
 駕也。」及太子廢,坐法當死。文帝惜其才,配為官奴,久乃釋。其後目盲,以手摸書而知其字。仁壽末,帝將避暑仁壽宮,太翼固諫曰:「恐是行鑾輿不反。」帝大怒,繫之長安獄,期還斬之。帝至宮寢疾,臨崩,命皇太子釋之。及煬帝即位,漢王諒反,帝問之。答曰:「何所能為!」未幾,諒果敗。



 帝從容言天下氏族,謂太翼曰:「卿姓章仇,四岳之胄,與盧同源。」於是賜姓盧氏。大業九年,從駕至遼東。太翼言黎陽有兵氣,後數日而楊玄感反書聞。帝甚異之,數加賞賜。太翼所言天文之事,不可稱數,關諸祕密,時莫能聞。後數歲,卒於雒陽。



 耿詢,字敦信,丹楊人也。滑稽辯給,伎巧絕人。陳後主時,以客從東衡州刺史王勇於嶺南。勇卒,詢不歸。會群俚反叛,推詢為主,柱國王世積討禽之。罪當誅,自言有巧思,世積釋之,以為家奴。久之,見其故人高智寶以玄象直太史,詢從之受天文算術。詢創意造渾天儀,不假人力,以水轉之,施於暗室中,使智寶外候天時,動合符契。世積知而奏之,文帝配詢為官奴,給太史局。後賜蜀王秀,從往益州,秀甚信之。及秀廢,復當誅。何稠言耿詢之巧,思若有神,上於是特原其罪。詢作馬上刻漏,世稱其妙。煬帝即位,進欹器。帝善之,免其奴。歲餘,授右尚方署
 監事。七年,車駕東征,詢上言曰:「遼東不可討,師必無功。」帝大怒,命左右斬之。何稠苦諫得免。及平壤之敗,帝以詢言為中,以詢守太史丞。宇文化及弒逆之後,從至黎陽,謂其妻曰:「近觀人事,遠察天文,宇文必敗,李氏當王,吾知所歸矣。」謀欲去之,為化及所殺。著《鳥情占》一卷,行於世。



 來和,字弘順,京兆長安人也。少好相術,所言多驗。周大冢宰宇文護引之左右,累遷畿伯下大夫,封洹水縣男。隋文帝微時,詣和。曰:「公當王有四海。」



 及為丞相,拜儀同。既受禪,進爵為子。開皇末,和上表自陳龍潛所言曰:「昔
 陛下在周,與永富公竇榮定語,臣曰:『我聞有行聲,即識其人。』臣當時即言:『公眼如曙星,無所不照,當王有天下,願忍誅殺。』建德四年五月,周武帝在雲陽宮謂臣曰:『諸公皆汝所識,隋公相祿何如?』臣報武帝曰:『隋公止是守節人,可鎮一方,若為將領,陣無不破。』臣即於宮東南奏聞,陛下謂臣:『此語不忘。』明年,烏丸軌言於武帝曰:『隋公非人臣。』帝尋以問臣。臣知帝有疑,臣詭報曰:『是節臣,更無異相。』於時王誼、梁彥光等知臣此語。大象二年五月,至尊從永巷東門入,臣在永巷門東,北面立,陛下問臣曰:『我得無災鄣不?』臣奏陛下曰:『公骨法氣色相應,天命
 已有付屬。』未幾,遂總百揆。」上覽之大悅,進位開府。



 和同郡韓則嘗詣和相,和謂之:「後四五當得大官。」人初不知所謂。則至開皇十五年五月終。人問其故,和曰:「十五年為三五,加以五月為四五。大官,槨也。」



 和言多此類。著《相經》三十卷。



 道士張賓、焦子順、應門人董子華等,此三人當文帝龍潛時,並私謂帝曰:「公當為天子,善自愛。」及踐位,以賓為華州刺史,子順為開府,子華為上儀同。



 蕭吉,字文休,梁武帝兄長沙宣武王懿之孫也。博學多通,尤精陰陽、算術。



 江陵覆亡,歸於魏,為儀同。周宣帝時,吉以朝政日亂,上書切諫,帝不納。及隋受禪,進上儀同,
 以本官太常,考定古今陰陽書。



 吉性孤峭,不與公卿相浮沈,又與楊素不協,由是擯落,鬱鬱不得志。見上好徵祥之說,欲乾沒自進,遂矯其跡為悅媚焉。開皇十四年,上書曰:「今年歲在甲寅,十一月朔旦,以辛酉為冬至。來年乙卯,正月朔旦,以庚申為元日。冬至之日,即在朔旦。《樂汁圖征》云:『天元十二月朔旦冬至,聖王受享祚。』今聖主在位,居天元之首,而朔旦冬至,此慶一也。辛酉之日,即至尊本命。辛德在丙,此十一月建丙子,酉德在寅,正月建寅,為本命與月合德,而居元朔之首,此慶二也。庚申之日,即是行年。乙德在庚,卯德在申,來年乙卯,是行
 年與歲合德,而在元旦之朝,此慶三也。《陰陽書》云:『年命與歲月合德者,必有福慶。』《洪範傳》云:『歲之朝,月之朝,日之朝,主王者。』經書並謂三長,應之者,延年福吉。



 況乃甲寅,蔀首;十一月,陽之始;朔旦冬至,是聖王上元。正月,是正陽之月,歲之首,月之先;朔旦是歲之元,月之朝,日之先,嘉辰之會。而本命為九元之先,行年為三長之首,並與歲月合德。所以《靈寶經》云:『角音龍精,其祚曰強。』來歲年命,納音俱角,歷之與經,如合符契。又甲寅、乙卯,天地合也。甲寅之年,以辛酉冬至;來年乙卯,以甲子夏至。冬至陽始,郊天之日,即是至尊本命,此慶四也。夏至陰始,
 祀地之辰,即是皇后本命,此慶五也。至尊德並乾之覆育,皇后仁同地之載養,所以二儀元氣,並會本辰。」上覽之悅,賜物五百段。



 房陵王時為太子,言東宮多鬼魅,鼠妖數見。上令吉詣東宮禳邪氣。於宣慈殿設神坐,有回風從艮地鬼門來,掃太子坐。吉以桃湯葦火驅逐之,風出宮門而止。



 謝土於未地,設壇為四門,置五帝坐。於時寒,有蝦蟆從西南來,入人門,升赤帝坐,還從人門而出,行數步,忽然不見。上大異之,賞賜優洽。又上言:太子當不安位。時上陰欲廢立,得其言,是之。由此,每被顧問。及獻皇后崩,上令吉卜擇葬所。吉歷筮山原,至一處,云:「卜
 年二千,卜世二百。」具圖而奏之。上曰:「吉凶由人,不在於地。高緯父葬,豈不卜乎?國尋滅亡。正如我家墓田,若云不吉,朕不當為天子;若云不凶,我弟不當戰沒。」然竟從吉言。表曰:「去月十六日,皇后山陵西北,雞未鳴前,有黑雲方圓五六百步,從地屬天;東南又有旌旗、車馬、帳幕,布滿七八里,并有人往來檢校,部伍甚整。日出乃滅。同見者十餘人。



 謹案《葬書》云『氣王與姓相生,大吉,今黑氣當冬王,與姓相生,是大吉利,子孫無疆之候也。」上大悅。其後上將親臨發殯,吉復奏曰:「至尊本命辛酉,今歲斗魁及天岡臨卯酉,謹案《陰陽書》,不得臨喪。」上不納。退而
 告族人蕭平仲曰:「皇太子遣宇文左率深謝餘云:『公前稱我當為太子,竟有驗,終不忘也。今卜山陵,務令我早立。我立之後,當以富貴相報。』吾記之曰:『後四載,太子御天下。』今山陵氣應,上又臨喪,兆益見矣。且太子得政,隋其亡乎!當有真人出矣。吾前紿云『卜年二千』者,是三十字也;『卜世二百者』,取世二運也。吾言信矣,汝其志之。」



 及煬帝嗣位,拜太府少卿,加位開府。嘗行經華陰,見楊素冢上有白氣屬天,密言於帝。帝問其故,吉曰:「其候,素家當有兵禍,滅門之象。改葬者,庶可免乎!」帝後從容謂楊玄感曰:「公宜早改葬。」玄感亦微知其故,以為吉祥,託以
 遼東未滅,不遑私門之事。未幾而玄感以反族滅,帝彌信之。



 後歲餘卒官。著《金海》三十卷,《相經要錄》一卷,《宅經》八卷,《葬經》六卷,《樂譜》二十卷,及《帝王養生方》二卷,《相手版要決》一卷,《太一立成》一卷,並行於時。



 楊伯醜,馮翊武鄉人也。好讀《易》,隱於華山。隋開皇初,徵入朝,見公卿不為禮,無貴賤皆汝之,人不能測也。文帝召與語,竟無所答。賜衣服,至朝堂捨之而去。於是被髮陽狂,游行市里,形體垢穢,未嘗櫛沐。時有張永樂者,賣卜京師,伯醜每從之遊。永樂為卦有不能決者,伯醜輒為分析爻象,尋幽入微,永樂嗟服,自以為非所及也。伯
 醜亦開肆賣卜。有人嘗失子就伯醜筮者。卦成,伯醜曰:「汝子在懷遠坊南門東,道北壁上有青裙女子抱之,可往取也。」如言,果得。或有金數兩,夫妻共藏之,於後失金,其夫意妻有異志,將逐之。其妻稱冤,以詣伯醜。伯醜為之筮:「金在矣。」悉呼其家人,指一人曰:「可就取。」果得之。又將軍許知常問吉凶,伯醜曰:「汝勿東北行。必不得已,當速還。不然者,楊素斬汝頭。」未幾,上令知常事漢王諒。俄而上崩,諒舉兵反,知常逃歸京師。知常先與楊素有隙,及素平并州,先訪知常,將斬之,賴此獲免。又有人失馬來詣伯醜卜者,時伯醜為皇太子所召,在途遇之,立
 為作卦。卦成,曰:「我不遑為卿說,且向西市東壁門南第三店,為我買魚作鱠,當得馬矣。」其人如教,須臾,有一人牽所失馬而至,遂禽之。崖州嘗獻徑寸珠,其使者陰易之,上心疑焉,召伯醜令筮。



 伯醜曰:「有物出自水中,質圓而色光,是大珠也。今為人所隱。」且言隱者姓名、容狀。上如言簿責之,果得本珠,上奇之,賜帛二十匹。國子祭酒何妥嘗詣之論《易》。聞妥之言,悠爾而笑曰:「何用鄭玄、王弼之言乎?」久之,微有辯答,所說辭義,皆異先儒之旨,而思理玄妙。故論者以為天然獨得,非常人所及也。竟以壽終。



 臨孝恭,京兆人也。明天文、算術,隋文帝甚親遇之。每言災祥之事,未嘗不中。上因令考定陰陽書,官至上儀同。著《欹器圖》三卷,《地動銅儀經》一卷,《九宮五墓》一卷,《遁甲錄》十卷,《元辰經》十卷,《元辰厄》百九卷,《百怪書》十八卷,《祿命書》二十卷,《九宮龜經》一百一十卷,《太一式經》三十卷,《孔子馬頭易卜書》一卷,並行於世。



 劉祐,滎陽人也。隋開皇初,為大都督,封索盧縣公。其所占候,合如符契,文帝甚親之。初與張賓、劉暉、馬顯定歷。後奉詔撰兵書十卷,名曰《金韜》,上善之。復著《陰策》二十卷,《觀臺飛候》六卷,《玄象要記》五卷,《律歷術文》一卷,《婚姻
 志》三卷,《產乳志》二卷,《式經》四卷,《四時立成法》一卷,《安歷志》十二卷,《歸正易》十卷,並行於世。



 張胄玄,勃海蓚人也。博學多通,尤精術數。冀州刺史趙煚薦之,隋文帝征授雲騎尉,直太史,參議律歷事。時輩多出其下,由是太史令劉暉等甚忌之。然暉言多不中,胄玄所推步甚精密。上異之,令楊素與術士數人,立議六十一事,皆舊法久難通者,令暉與胄玄等辯析之。暉杜口一無所答,胄玄通者五十四焉。由是擢拜員外散騎侍郎,兼太史令,賜物千段。暉及黨與八人,皆斥逐之。改定新歷,言前歷差一日。內史通事顏慜楚上言曰:「漢
 時落下閎改《顓頊歷》,作《太初歷》,云:『後當差一日,八百年當有聖者定之。』計今相去七百一十年,術者舉其成數,聖者之謂,其在今乎!」上大悅,漸見親用。



 胄玄所謂歷法,與古不同者三事:其一,宋祖沖之於歲周之末,創設差分,冬至漸移,不循舊軌,每四十六年,卻差一度。至梁虞廣刂歷法,嫌沖之所差太多,因以一百八十六年,冬至移一度。胄玄以此二術,年限縣隔,追檢古注,所失極多。



 遂折中兩家,以為度法,冬至所宿,歲別漸移,八十三年,卻行一度。則上合堯時,日永星火;次符漢歷,宿起牛初。明其前後,並皆密當。其二,周馬顯造《丙寅元歷》,有陰陽轉
 法,加減章分,進退蝕餘,乃推定日,創開此數。當時術者,多不能曉。張賓因而用之,莫能考正。胄玄以為加時先後,逐氣參差,就月為斷,於理未可。乃因二十四氣,列其盈縮所出。實由日行遲,則月逐日易及,令合朔加時早;日行速,則月逐日少遲,令合朔加時晚。檢前代加時早晚,以為損益之率。日行,自秋分已後至春分,其勢速,計一百八十二日而行一百八十度;自春分已後至秋分,日行遲,計一百八十二日而行一百七十六度。每氣之下,即其率也。其三,自古諸歷,朔望逢交,不問內外,入限便蝕。張賓立法,創有外限,應蝕不蝕,猶未能明。



 胄玄以
 日行黃道,歲一周天;月行月道,二十七日有餘一周天。月道交絡黃道,每行黃道內十三日有奇而出,又行道外十三日有奇而入,終而復始。月經黃道,謂之交。朔望去交前後各五度以下,即為當蝕。若月行內道,則在黃道之北,蝕多有驗;月行外道,在黃道之南也,雖遇正人,無由掩映,蝕多不驗。遂因前法,別立定限,隨交遠近,逐氣求差,損益蝕分,事皆明著。



 其超古獨異者有七事:其一,古歷五星行度,皆守恆率,見伏盈縮,悉無格準。



 胄玄候之,各得真率,合見之數,與古不同。其差多者,至加減三十許日。即如熒惑,平見在雨水氣,即均加二十九日;
 見在小雪氣,則均減二十五日。加減平見,以為定見。諸星各有盈縮之數,皆如此例,但差數不同。特其積候所知,時人不能原其旨。其二,辰星舊率,一終再見,凡諸古歷,皆以為然。應見不見,人未能測。



 胄玄積候,知辰星一終之中,有時一見。及同類感召,相隨而出。即如辰星,平晨見在雨水者,應見即不見;若平晨見在啟蟄者,去日十八度外,三十六度內。晨有木火土金一星者,亦相隨見。其三,古歷步術,行有定限,自見已後,依率而推,進退之期,莫知多少。胄玄積候,知五星遲速留退真數,皆與古法不同,多者差八十餘日,留回所在,亦差八十餘度。
 即如熒惑,前疾初見在立冬初,則二百五十日行一百七十七度;定見夏至初,則一百七十日行九十二度。追步天驗,今古皆密。



 其四,古歷食分,依平即用,推驗多少,實數罕符。胄玄積候,知月從木火土金四星行,有向背。月向四星即速,背之則遲。皆十五度外及循本率。遂於交分,限其多少。其五,古歷加時,朔望同術。胄玄積候,知日蝕所在,隨方改變,傍正高下,每處不同。交有淺深,遲速亦異,約時立差,皆會天象。其六,古歷交分即為蝕數,去交十四度者,食一分;去交十三度,食二分;去交十度,食三分;每近一度,食益一分;當交即蝕既。其應多少,自
 古諸歷,未悉其原。胄玄積候,知當交之中,月掩日不能畢盡,故其蝕反少;去交五六時,月在日內,掩日便盡,故其蝕及既。



 自此以後,更遠者,其蝕又少。交之前後,在冬至,皆爾。若近夏至,其率又差。



 胄玄所立蝕分,最為詳密。其七,古歷二分,晝夜皆等。胄玄積候,知其有差。春、秋二分,晝多夜漏半刻。皆由日行遲疾盈縮使其然也。凡此,胄玄獨得於心,論者服其精密。大業中,卒於官。



\end{pinyinscope}