\article{卷七十八列傳第六十六}

\begin{pinyinscope}

 張定和張奫麥鐵杖沈光權武王仁恭吐萬緒董純魚俱羅王辯陳稜趙才張定和,字處謐,京兆萬年人也。家少貧賤,有志節。初為侍官,隋開皇九年平陳,定和當從征,無以自給。其妻有嫁時衣服,定和求鬻之,妻不與,定和遂行。



 以功拜儀同,賜帛千匹,遂棄其妻。後數以軍功,加上開府、驃騎將軍。
 從上柱國李充征突厥,先登陷陣,所勞問之。進位柱國,封武安縣侯,嘗物二千段,良馬二匹,金百兩。煬帝嗣位,歷宜州刺史、河內太守,頗有惠政。遷左屯衛大將軍。從帝征吐谷渾,至覆袁川。時吐谷渾主與數騎遁,其名王詐為渾主,保車我真山,帝命定和擊之。既與賊遇,輕其眾少,呼之令降,賊不肯下。定和不被甲,挺身登山,中流矢而斃。其亞將柳武達擊賊,悉斬之。帝為之流涕,贈光祿大夫。時舊爵例除,於是復封和武安侯,謚曰壯武。子世立嗣,尋拜光祿大夫。



 張奫,字文懿,清河東武城人也。本名犯廟諱。七代祖沈,石季龍末,自廣陵六合度江家焉。仕至桂陽太守。孫朏,晉佐著作郎。坐外祖楊佺期除名,徙于南譙,因寓居之。奫好讀兵書,長於騎射,尤便刀楯。父雙,自清河太守免,歸周。時鄉人郭子冀密引陳寇,雙欲率子弟擊之,猶豫未決。奫贊成其謀,竟破賊,由是以勇決知名。起家州主簿。及隋文帝作相,授丞相府大都督,領鄉兵。賀若弼之鎮江都也,特敕奫從,因為間諜。平陳之役,頗有力焉。進位開府儀同三司,封文安縣子。



 歲餘,奫率水軍破逆賊笮子游於京口、薛子建於和州。徵入,拜大將軍。文帝命
 升御坐宴之,謂曰:「卿可為朕兒,朕為卿父。今日聚集,示無外也。」後賜綠沈甲、獸文具裝,綺羅千匹。尋從楊素征江表,別破高智慧於會稽,吳世華於臨海。進位上大將軍。歷撫、濟二州刺史,俱有能名。開皇十八年,為行軍總管,從漢王諒征遼東。諒軍多物故,奫眾獨全,帝善之。仁壽中,卒於潭州總管,謚曰莊。子孝廉。



 麥鐵杖,始興人也。貧賤,少勇驍,有膂力,日行五百里,走及奔馬。性疏誕使酒,好交遊,重信義,每以漁獵為事,不修生業。陳大建中,結聚為群盜,廣州刺史歐陽頠俘之以獻,沒為官戶,配執御傘。每罷朝後,行百餘里,夜至南
 徐州,窬城而入,行光火劫盜。旦還,及牙時,仍又執傘。如此者十餘度,物主識之,州以狀奏。朝士見鐵杖每旦恒在,弗之信。後南徐州數告變,尚書蔡徵曰:「此可驗矣。」於仗下時,購以百金,求人送詔書與南徐州刺史。鐵杖出應募,齎敕而往,明旦反奏事。帝曰:「信然,為盜明矣。」惜其勇捷,誡而釋之。陳亡後,徙居清流縣。遇江東反,楊素遣鐵杖頭戴草束,夜浮度江,覘賊中消息,具知還報。後復更往,為賊所禽,逆帥李稜縛送高智慧。行至庱亭,衛者憩食,哀其餒,解手以給其餐。鐵杖取賊刀亂斬衛者,殺之皆盡,悉割其鼻,懷之以歸。素大奇之。後
 敘戰勛,不及鐵杖,遇素馳驛歸於京師,鐵杖步追之,每夜則同宿。素見而悟,特奏授儀同三司。以不識書,放還鄉里。成陽公李徹稱其驍武,開皇十六年,徵至京師,除車騎將軍。仍從楊素北征突厥,加上開府。



 煬帝即位,漢王諒反,從楊素擊之,每戰先登。進位柱國。除萊州刺史,無蒞政名。轉汝南太守,稍習法令,群盜屏迹。後因朝集,考功郎竇威嘲之曰:「麥是何姓?」鐵杖應聲曰:「麥豆不殊,何忽相怪?」威赧然無以應,時人以為敏捷。



 尋除右屯衛大將軍。帝待之愈密。



 鐵杖自以荷恩深重,每懷竭命之志。及遼東之役,請為前鋒,顧謂醫者吳景賢曰:「大丈夫
 性命自有所在,豈能艾炷灸頞,瓜釭褲鼻,療黃不差,而臥死兒女手中乎!」將度遼,呼其三子曰:「阿奴!當備淺色黃衫。吾荷國恩,今是死日。我得被殺,爾當富貴。唯誠與孝,爾其勉之。」及濟,橋未成,去東岸尚數丈,賊大至。杖跳上岸,與賊戰,死。武賁郎將錢士雄、孟金叉亦死之,左右更無及者。帝為之流涕,購得其屍,贈光祿大夫、宿國公,謚曰武烈。子孟才嗣,授光祿大夫。



 孟才二弟仲才、季才,俱拜正議大夫。賵贈鉅萬,賜轀輬車,給前後部羽葆鼓吹。



 命平壤道敗將宇文述等百餘人皆為執紼,王公以下送至郊外。士雄贈左光祿大夫、右屯衛將軍、武強
 侯,謚曰剛。子傑嗣。金叉贈右光祿大夫,子善誼襲官。



 孟才,字智稜,果烈有父風,帝以其死節將子,恩錫殊厚,拜武賁郎將。及江都之難,慨然有復仇志。與武牙郎將錢傑素交友,二人相謂曰:「事等世荷國恩,門著誠節。今賊臣殺逆,社稷淪亡,無節可紀,何面目視息世間哉!」乃流涕扼腕,相與謀於顯福宮邀擊宇文化及。事臨發,陳籓之子謙知而告之,與其黨沈光俱為化及所害,忠義之士哀焉。



 光字總持,吳興人也。父居道,仕陳為吏部侍郎。陳滅,徙家長安。皇太子勇引署學士。後為漢王諒府掾,諒敗,除名。



 光少驍捷,善戲馬,為天下之最。略綜書記,微有
 詞藻,常慕立功名,不拘小節。家貧,父兄並以傭書為事,光獨跅,交通輕俠,為京師惡少年所附。人多贍遺,得以養親,每致甘食美服,未嘗困匱。初建禪定寺,其中幡竿高十餘丈,適值繩絕,非人力所及。光謂僧曰:「當相為上繩。」與諸僧驚喜。光因取索口銜,拍竿而上,直至龍頭。擊繩畢,手足皆放,透空而下,以掌拓地,倒行十餘步。觀者駭悅,莫不嗟異,時人號為「肉飛仙」。



 大業中,煬帝徵天下驍果之士伐遼東,光預焉。同類數萬人,皆出其下。光將詣行在所,賓客送至灞上百餘騎。光酹酒誓曰:「是行若不建功立名,當死於高麗,不復與諸君相見。」及從帝攻
 遼東,以衝梯擊城,竿長十五丈,光升其端,臨城與賊戰,知兵接敵,殺傷十數人。賊競擊而墜,未及地,適遇竿有垂絙,光接而復上。



 帝望見,壯而異之,馳召與語,大悅,即日拜朝散大夫,賜寶刀良馬。恒置左右,親顧漸密。未幾,以為折衝郎將,賞遇優重。帝每推食解衣賜之,同輩莫比。



 光自以荷恩深重,思懷竭節。及江都之難,潛構義勇,將為帝復仇。先是,帝寵暱官奴,名為給使,宇文化及以光驍男,方任之,使總統,營於禁內。時麥孟才、錢傑等陰圖化及,因謂光曰:「我等荷國厚恩,不能死難,又俯首事讎,受其驅率,何用生為!吾必欲殺之,死無所恨。公義士
 也,肯從我乎?」光泣下霑衿曰:「是所望於將軍也。僕領給使數百人,並荷先帝恩,今在化及內營。以此復仇,如鷹鸇之逐烏雀。」孟才為將軍。領江淮眾數千人,期以營將發時,晨起襲化及。光語泄,陳謙告其事。化及大懼曰:「此麥鐵杖子也,及沈光者,並勇決不可當,須避其鋒。」



 是夜即與腹心走出營外,留人告司馬德戡等,遣領兵馬,逮捕孟才。光聞營內喧聲,知事發,不及被甲。即襲化及營,空無所獲。逢舍人元敏,數而斬之。德戡兵至,四面圍合。光大呼潰圍,給使齊奮,斬首數十級,賊皆披靡。德戡輒復遣騎,翼而射之。光身無介胄,遇害,時年二十八。麾下
 百人皆鬥死,一無降者。壯士聞之,莫不為之隕涕。



 權武,字武弄,天水人也。祖超,魏秦州刺史。父襲慶,仕周,為開府。時武元皇帝之為周將也,與齊師戰於并州。襲慶時從,被圍百餘重,力戰矢盡,短兵接戰,殺傷甚眾,刀槊皆折,脫胄擲地,向賊大罵曰:「何不來斫頭!」賊遂殺之。



 武以忠臣子,起家拜開府,襲爵齊郡公。武少果勁,勇力絕人,能重甲上馬。嘗倒投於井,未及泉,復躍而出,其拳捷如此。頻以軍功增邑。周宣帝時,拜勁捷左旅上大夫,進位上開府。隋文帝為丞相,引置左右。平陳之役,以行軍總管從晉王出六合,還拜豫州刺史。以創業之舊,進
 位大將軍,檢校潭州總管。其年,桂州人李世賢作亂,武以行軍總管與武候大將軍虞慶則擊平之。慶則以罪誅,功竟不錄,復還于州。多造金帶,遣嶺南酋領,其人復答以寶物,武皆納之,由是致富。後武晚生一子,與親客宴集,酒酣,遂擅赦所部獄囚。武常以南越邊遠,政從其俗,務適便宜,不依律令,而每言當今法急,官不可為。上令有案之,皆驗,令斬之。武於獄中上書,言父為武元皇帝戰死於馬前,以求哀,由是除名。仁壽中,復拜大將軍。



 封邑如舊。未幾,授太子右衛率。煬帝即位,拜右武衛將軍,坐事免。後為右屯衛大將軍。坐事除名。卒於家。子
 弘。



 王仁恭,字元實,天水上邽人也。祖建,周鳳州刺史。父猛,鄯州刺史。仁恭少剛毅修謹,工騎射。秦孝王引為記室,後為車騎將軍。從楊素擊突厥於靈武,以功拜上開府。以驃騎將軍典蜀王軍事。蜀王以罪廢,官屬多罹其患。上以仁恭素質直,置而不問。後從楊素討平漢王諒,以功進位大將軍。歷呂、衛二州刺史。尋改為汲郡太守,有能名。上徵入朝,慰勉之,褒賜甚厚。遷信都太守。汲郡吏民扣馬號哭於道,數日不得出境。遼東之役,以仁恭為軍將。及班師,仁恭為殿,遇賊,敗之。進左光祿大夫,明年,
 復以軍將指扶餘道,帝謂曰:「往者諸軍多不利,公獨以一軍破賊。古人云,敗軍之將不可以言勇,諸將其可任乎?今委公為前軍。」



 前後賞賚甚重。仁恭遂進軍。至新城,破其軍,因圍之。帝聞之大悅,遣賜以珍物,進光祿大夫。會楊玄感反,其兄子武賁郎將仲伯預焉,由是坐免。尋而突厥為寇,詔仁恭以本官領馬邑太守。其年,始畢可汗來寇馬邑,復令二將勒兵南過。時郡兵不滿三千,仁恭簡精銳逆擊,破之,并斬二將。後突厥復入定襄,仁恭復大破之。



 時天下大亂,道路隔絕,仁恭頗改舊節,受納貨賄,又不敢輒開倉賑恤百姓。其麾下校尉劉武周與
 仁恭侍婢姦通,恐其事泄,遂害之。武周於是開倉賑給,郡內皆從之,自稱天子,置百官,轉攻傍郡。



 吐萬緒,字長緒,代郡鮮卑人也。父通,周郢州刺史。緒少有武略,在周,襲爵元壽縣公,累遷大將軍、小司武。隋文帝受禪,拜襄州總管,封穀城郡公。轉青州總管,頗有政名。徙朔州總管,甚為北狄所憚。後帝有吞陳志,轉為徐州總管,令修戰具。及大舉濟江,緒以行軍總管與四河紇豆陵洪景屯兵江北。及陳平,拜夏州總管。晉王廣為太子,引為右虞候率。及帝即位,恐漢王諒為變,拜緒晉、絳二州刺史。未出關,諒已舉兵,詔緒從楊素擊破之,拜
 左武候將軍。大業初,轉光祿卿。賀若弼遇讒,引緒為證,緒明其無罪,由是免官。後守東平太守。帝幸江都,路經其境,迎謁道傍。帝命升龍舟,緒因頓首謝往事。帝大悅,拜金紫光祿大夫,太守如故。及遼東之役,請為先鋒,拜左屯衛大將軍。指蓋馬道。及還,留鎮懷遠,進位左光祿大夫。時劉元進作亂,攻潤州,帝徵緒討之。緒擊破元進,解潤州圍。



 賊窮蹙請降,元進及其偽僕射朱燮僅以身免,於陣斬其偽僕射管崇及其將軍陸顗等五千餘人。進解會稽圍。元進復據建安,帝令進討之。緒以士卒疲弊,請息甲待來春。帝不悅,密求緒罪,有司奏緒怯懦違
 詔,除名配防建安。尋徵詣行在所,緒鬱鬱不得志,還至永嘉,發疾而卒。



 董純,字德厚,隴西成紀人。祖和,魏太子左衛率。父昇,周柱國。純少有膂力,便弓馬。仕周,位司御上士、典馭下大夫。從武帝平齊,拜儀同,進為大興縣侯。隋文帝受禪,進爵漢曲縣公。後以軍功,進位上開府。開皇末,以勞舊拜左衛將軍,改封順政縣公。後從楊素平漢王諒,以功拜柱國,進爵郡公,再遷左驍衛將軍。齊王暕之得罪,純坐與交通,帝譴之。純曰:「此數詣齊王者,以先帝、先后往在仁壽宮,置元德太子及齊王於膝上,謂臣曰:『汝好看此
 二兒,勿忘吾言。』臣誠不敢忘先帝言。時陛下亦侍先帝側。」帝改容曰:「誠有斯旨。」於是捨之。



 數日,出為汶山太守。歲餘,突厥寇邊,轉榆林太守。會彭城賊帥張大彪、宗世模等保懸薄山,帝令純討破之,斬萬餘級,築為京觀。又破賊魏麒麟於單父。及帝重征遼東,復以純為彭城留守。東海賊彭孝才轉入沂水,保伍不及山,純擊之,禽孝才於陣,車裂之。時盜賊日益,純雖剋捷,而所在蜂起。有譖純怯懦不能平賊,帝遣鎖詣東都。有司見帝怒甚,希旨致純死罪,竟誅。



 魚俱羅,馮翊下邽人。身長八尺,膂力絕人,聲氣雄壯,言
 聞數百步。為大都督,從晉王廣平陳,以功拜開府。及沈玄懀、高智慧等作亂江南,楊素以俱羅壯勇,請與同行。有功,加上開府,封高唐縣公,拜疊州總管。以母憂去職。還至扶風,會楊素將出靈州道擊突厥,逢之,送與俱行。及遇賊,俱羅與數騎奔擊,瞋目大呼,所當皆披靡。以功進位柱國,拜豐州總管。突厥入境,輒禽斬之,自是屏迹,不敢畜牧於塞下。



 初,煬帝在籓,俱羅弟贊以左右從,累遷大都督。及帝嗣位,拜車騎將軍。贊凶暴,令左右炙肉,遇不中意,以簽刺瞎其眼,溫酒不適口者,立斷其舌。帝以籓邸之舊,不忍加誅,謂近臣曰:「弟既如此,兄亦可知。」因
 召俱羅責之,出贊於獄,令自為計。贊至家,飲藥而死。帝恐俱羅不安,慮生邊患,轉安州刺史,遷趙郡太守。後因朝集至東都,與將軍梁伯隱有舊,數相往來。又從郡多將雜物以貢獻,帝不受,因遺權貴。御史劾俱羅以郡將交通內臣,帝大怒,與伯隱俱坐除名。未幾,越巂飛山蠻反,詔俱羅白衣領將,并率蜀郡都尉段鐘葵討平之。大業九年,重征高麗,以俱羅為碣石道軍將。及還,江南劉元進作亂,詔俱羅將兵向會稽諸郡逐捕之。



 時百姓思亂,從盜如市,俱羅擊賊帥朱燮、管崇等,戰無不捷。然賊勢浸盛,敗而復聚。俱羅度賊非歲月可平,諸子並在京、
 洛,又見天下漸亂,終恐道路隔絕。于時東都饑饉,穀食踴貴,俱羅遣家僮將船米至東都糶之,益市財貨,潛迎諸子。朝廷微知之,恐有異志,案驗不得其罪。帝復令大理司直梁敬真就鎖將詣東都,俱羅相表異人,目有重瞳,陰為帝之所忌。敬真希旨,奏俱羅師徒敗衄,斬東都市,家口籍沒。



 王辯,字警略,馮翊蒲城人也。祖訓,以行商致富。魏世,出粟助給軍糧,為假清河太守。辯少習兵書,尤善騎射,慷慨有大志。在周,以軍功授帥都督。仁壽中,累遷車騎將軍。後從楊素討平漢王諒,賜爵武寧縣男。累以軍功,加
 至通議大夫,尋遷武賁郎將。及山東盜賊起,帝引辯升御榻,問以方略。辯論取賊勢,帝稱善曰:「誠如此,賊不足憂。」於是發從行步騎三千,擊敗之,賜黃金二百兩。勃海賊帥高士達自號東海公,眾以萬數。令辯擊之,屢挫其銳。帝在江都宮,聞而召之,及見,禮賜甚厚,復令往信都經略士達,復戰破之,優詔褒顯。時賊帥郝孝德、孫宣雅、時季康、竇建德、魏刀兒等往往屯聚,大者十數萬,小者數千,寇掠河北。



 辯擊之,所向皆捷。及翟讓寇徐、豫,辯頻擊走之。讓尋與李密屯據洛口倉,辯與王世充討密,阻洛水相持經年。辯攻敗密。乘勝將入城,世充不知,恐將
 士勞倦,鳴角收兵,翻為密徒所乘,官軍大潰,不可救止。辯至洛水,橋已壞。遂涉水至中流,為溺人所引墜馬,竟溺死。三軍莫不痛惜之。



 時有河南斛斯萬善,驍勇果毅,與辯齊名。從衛玄討楊玄感,萬善與數騎追及之,玄感窘迫自殺。由是知名,拜武賁郎將。突厥始畢之圍鴈門,萬善奮擊之,所向皆破。由是突厥莫敢逼城,十許日竟退,萬善力也。後頻討群盜,累功至將軍。



 又有將軍鹿願、范貴、馮孝慈,俱為將帥,數從征伐,並有名於世。事皆亡失,故史官闕云。



 陳稜,字長威,廬江襄安人也。祖碩,以漁釣自給。父峴,少
 驍勇,事章大寶為帳內部曲。告大寶反,授譙州刺史。陳滅,廢於家。高智慧、汪文進反,廬江豪傑亦舉兵相應。以峴舊將,共推為主。峴欲拒之,稜謂峴曰:「眾亂既作,拒之禍且及己,不如偽從,別為後計。」峴然之。後潛使稜至柱國李徹所,請為內應。徹上其事,拜上大將軍、宣州刺史,封譙郡公,詔徹應接之。徹軍未至,謀泄,為其黨所殺,稜以獲免。上以其父之故,拜開府,尋領鄉兵。



 大業三年,拜武賁郎將。後與朝請大夫張鎮周自義安汎海擊流求國,月餘而至。



 流求人初見船艦,以為商旅,往往詣軍貿易。稜率眾登岸,遣鎮周為先鋒。其主歡斯渴刺兜遣
 兵拒戰,鎮周頻破之。稜進至低沒檀洞,其小王歡斯老模拒戰,稜敗之,斬老模。其日霧雨晦冥,將士皆懼,稜刑白馬以祭海神,既而開霽。分為五軍,趣其都邑,乘勝逐北,至其柵,破之,斬渴刺兜,獲其子島槌,虜男女數千而歸。帝大悅。加稜右光祿大夫,鎮周金紫光祿大夫。



 遼東之役,以宿衛遷左光祿大夫。明年,帝復征遼東,稜為東萊留守。楊玄感反,稜擊平黎陽,斬玄感所署刺史元務本。尋奉詔於江南營戰艦。至彭城,賊帥孟讓據都梁宮,阻淮為固。稜潛於下流而濟,至江都,襲破讓。以功進位光祿大夫,賜爵信安侯。



 後帝幸江都宮,俄而李子通據
 海陵,左才相掠淮北,杜伏威屯六合,帝遣夌擊之,往見剋捷,超拜右禦衛將軍。復度清江,擊宣城賊。俄而帝以弒崩,宇文化及引軍北上,召稜守江都。稜集眾縞素,為煬帝發喪,備儀衛,改葬於吳公臺下,衰杖送喪,慟感行路,論者深義之。稜後為李子通所陷,奔杜伏威,伏威忌而害之。



 趙才,字孝才,張掖酒泉人也。祖隗,魏銀表光祿大夫、樂浪太守。父壽,周順政太守。才少驍武,便弓馬,性粗悍,無威儀。仕周,為輿正上士。隋文帝受禪,以軍功至上儀同。後配事晉王,為右虞候率。煬帝即位,轉左備身驃騎、右
 驍衛將軍。帝以才籓邸舊臣,漸見親待。才亦恪勤匪懈,所在有聲。轉右候衛將軍。從征吐谷渾,以為行軍總管,率衛尉卿劉權、兵部侍郎明雅等出合河道,破賊,以功進金紫光祿大夫。及遼東之役,再出碣石道。再遷右候衛大將軍。時帝每事巡幸,才恒為斥候,肅遏姦非,無所回避。在途遇公卿妻子有違禁者,才輒醜言大罵,多所援及。時人雖患其不遜,然才守正,無如之何。



 十二年,帝將幸江都,才見四海土崩,諫請還京師,安兆庶。帝大怒,以才屬吏,旬日乃出之。遂幸江都,待遇逾暱。時江都糧盡,內史侍郎虞世基、祕書監袁充等多勸帝幸丹陽。才
 極陳入京策,世基極言度江便。帝無言,才與世基相忿而出。



 宇文化及殺逆之際,才時在苑北,化及遣驍果席德方執之,謂曰:「今日之事,祇得如此。」才默然不對。化及忿才無言,將殺之,三日乃釋,以本官從事,鬱鬱不得志。才嘗對化及宴,請勸其同謀逆者十八人楊士覽等酒,化及許之。才執盃曰:「十八人止可一度作,勿復餘處更為。諸人默然不對。行止聊城,遇疾。俄而化及為竇建德所破,才復見虜。心彌不平,數日而卒。



 仁壽、大業間有蘭興洛、賀蘭蕃,俱為武候將軍,剛嚴正直,不避強禦,咸以稱職知名。



 論曰:虎嘯風生,龍騰雲起,英賢奪發,亦各因時。張定和、張、麥鐵杖皆一時壯士,而困於貧賤。當其鬱抑未遇,亦安知有鴻鵠志哉!終能振拔汙泥,申其力用,符馬革之願,快生平之必,得丈夫之節矣。孟才、錢傑、沈光等感懷恩舊,臨難亡身,雖功無所成,其志有可稱矣。權武素無行檢,不拘刑憲,終取黜辱,不亦宜哉!仁恭武毅見知,文以取達,初在汲郡,清能可紀,後居馬邑,貪吝而亡。



 鮮克有終,斯言乃驗。吐萬緒、董純以萑蒲不翦,遽嬰罪戮。大業之季,盜可盡乎?



 俱羅欲加之罪,非其咎釁。王辯殞身勍敵,志在勤王。陳稜縞素發喪,哀感行路,義之所動,
 固已深乎!趙才雖人而無儀,志在強直,拒世基之諂,可謂不茍同矣。



\end{pinyinscope}