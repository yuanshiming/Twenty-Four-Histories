\article{卷六十八列傳第五十六}

\begin{pinyinscope}

 豆盧寧子勣孫毓楊紹子雄王雅子世積韓雄子禽賀若敦子弼弟誼豆盧寧,字永安,昌黎徒何人。其先本姓慕容氏,燕北地王精之後也。高祖勝,以燕。皇始初歸魏,授長樂郡守,賜姓豆盧氏。或云北人謂歸義為「豆盧」,因氏焉,又云避難改焉,未詳孰是。父萇,魏柔玄鎮將,有威重,見稱於時。武成中,以寧勳,追贈柱國大將軍、少保、涪陵郡公。寧少驍果,
 有志氣,身長八尺,美姿容,善騎射。魏永安中,以別將隨爾朱天光入關。以破萬俟醜奴功,賜爵靈壽縣男。



 嘗與梁定遇於平涼川,相與肄射,乃相去百步懸莎草以射之,七發五中。



 定服其能,贈遺甚厚,天光敗,從侯莫陳悅。及周文討悅,寧與李弼來歸。



 孝武西遷,以奉迎勳,封河陽縣伯,後進爵為公。從禽竇泰,復弘農,破沙苑,除衛大將軍、兼大都督。大統七年,從于謹破稽胡帥劉平伏於上郡。及梁定反,以寧為軍司,監隴右諸軍事。賊平,進位侍中、使持節、驃騎大將軍、開府儀同三司。九年,從周文迎高仲密,與東魏戰於芒山。遷左衛將軍,進
 爵范陽郡公。十六年,拜大將軍。羌帥傍乞鐵匆及鄭五醜等反叛,寧討平之。恭帝二年,改封武陽郡公,遷尚書右僕射。周孝閔帝踐祚,授柱國大將軍。武成初,出為同州刺史。遷大司寇,進封楚國公,邑萬戶,別食鹽亭縣一千戶,收其租賦。保定四年,授岐州刺史。屬大兵東討,寧輿疾從軍。薨於同州。贈太保、十州諸軍事、同州刺史,謚曰昭。



 初,寧未有子,養弟永恩子勣。及生子贊,親屬皆請贊為嗣。寧曰:「兄弟之子猶子也,吾何擇焉。」遂以勣嗣。時以此多之。及寧薨,勣襲爵。



 勣字定東。生時,周文親幸寧家稱慶,時遇新破齊軍,周
 文因字曰定東。勣聰悟,有器局。初以勳臣子封義安縣侯。周閔帝受禪,授稍伯下大夫、開府儀同三司,改封丹陽郡公。明帝時,為左武伯中大夫。勣自以經業未通,請解職遊露門學。帝嘉之,敕以本官就學。齊王憲納勣妹為妃,恩禮愈厚。武帝嗣位,渭源燒當羌因饑作亂,以勣有才略,拜渭州刺史。甚有惠政,華夷悅服,大致祥瑞。烏鼠山俗呼為高武隴,其下渭水所出。其山絕壁千尋,由來乏水,諸羌苦之。勣馬足所踐,忽飛泉涌出。有白烏翔止前,乳子而後去,有白狼見於襄武,人為之謠曰:「我有丹陽,山出玉漿。濟我人夷,神烏來翔。」百姓因號其泉
 曰玉漿泉。後丁父艱,毀瘁過禮。襲爵楚國公。大象二年,累遷利州總管,尋拜柱國。隋文帝為丞相,益州總管王謙作亂,勣嬰城固守。謙將達奚惎等攻之,起土山,鑿城為七十餘穴,堰江以灌之。勣時戰士不過二千,晝夜相拒。經四旬,梁睿軍且至,賊解去,授上柱國,賜一子爵中山縣公。開皇中,為夏州總管。帝以其家貴盛,勛效克彰,後為漢王諒納其女為妃,恩遇彌厚。七年,追守利州功,詔食始州臨津縣邑千戶。十年,以疾徵還京師,詔諸王並至勣第,中使顧問,道路不絕。卒,謚曰襄。



 子賢嗣。位顯州刺史、大理少卿、武賁郎將。次子毓。



 毓字道生,少英果,有氣節。漢王諒出鎮並州,毓以妃兄為王府主簿。以征突厥功,授儀同三司。及煬帝即位,諒納諮議王頍謀作亂。毓苦諫不從,因謂其弟懿曰:「吾匹馬歸朝,自得免禍。此乃身計,非為國也。今且偽從,以思後計。」毓兄顯州刺史賢言於帝曰:「臣弟毓素懷志節,必不從亂,但逼兇威,不能克遂。臣請從軍,與毓為表裏,諒不足圖也。」帝許之。賢密遣家人齎敕書至毓所,與之計。



 諒將往介州,令毓與總管屬朱濤留守。毓與濤議拒之,濤拂衣不從,毓追斬之。時諒司馬皇甫誕以諫被囚,毓出之,與協計,及開府,盤石侯宿勤武等閉城拒諒。部分
 未定,有人告諒。諒攻之,城陷,見害,時年二十八。諒平,贈大將軍,封正義縣公,謚曰愍。



 子願師嗣。拜儀同三司。大業初,行新令,五等並除。未幾,帝復下詔改封雍丘侯,復以願師襲。



 讚以寧勳,建德初,賜爵華陰縣侯。累遷開府儀同大將軍,進爵武陽郡公。



 永恩少有識度,與寧俱歸周文。以迎孝武功,封新興伯。屢從征討,皆有功,進位驃騎大將軍、開府儀同三司。周孝閔帝踐祚,授鄯州刺史,改封沃野縣公。保定元年,入為司會中大夫。寧封楚國公,請以先封武陽郡三千戶益沃野之封,詔許焉。卒于官。贈少保,
 謚曰敬。子通嗣。



 通字平東,一名會,弘厚有器局。在周,以父功賜爵臨貞縣侯,改封沃野縣公。



 位開府、北徐州刺史。開皇初,進爵南陳郡公,尚隋文帝妹昌樂縣長公主。歷定相二州刺史、夏洪二州總管,並以寬惠稱。卒官。謚曰安。子寬嗣。



 楊紹,字子安,弘農華陰人也,祖興,魏新平郡守。父國,中散大夫。紹少慷慨有志略,屢從征伐,力戰有功。普泰初,封平鄉縣男。大統元年,進爵冠軍縣公。



 四年,為鄜城郡守。紹性恕直,兼有威惠,百姓安之。累遷驃騎大將軍、開府儀同三司、鄜州刺史,賜姓叱呂引氏。周孝閔帝踐祚,
 進爵儻城郡公,位大將軍。卒,贈成、文等八州刺史。謚曰信。子雄嗣。



 雄初名惠,美姿容,有器度,雍容閑雅,進止可觀。周武帝時,為太子司旅下大夫。帝幸雲陽宮,衛王直作亂,襲肅章門,雄逆拒破之。封武陽郡公,遷右衛上大夫。大象中,進爵邘國公。隋文帝為丞相,雍州牧、畢王賢構作難,雄時為別駕,知其謀,以告文帝。賢伏誅,以功授柱國、雍州牧,仍領相府虞候。周宣帝葬,備諸王有變,令雄率六千騎送至陵所。進位上柱國。



 文帝受禪,除左衛將軍,兼宗正卿。遷右衛大將軍,參預朝政。封廣平王,以邘公別封
 一子。雄請封弟士貴,朝廷許之。或奏高熲朋黨者,帝言之於朝,雄深明其虛,帝亦以為然。雄時貴寵,冠絕一時,與高熲、虞慶則、蘇威稱為「四貴」。



 雄寬容下士,朝野顧屬。帝陰忌之,不欲其典兵馬,乃改授司空,外示優崇,而內實奪其權也。雄乃閉門不通賓客。尋改封清漳王。仁壽初,帝以清漳不允聲望,命職方進地圖,指安德郡示群臣曰:「此號足為名德相稱」。乃改封安德王。



 大業初,授太子太傅。元德太子薨,檢校鄭州刺史。遷懷州刺史、京兆尹。帝親征吐谷渾,詔雄總管澆河道諸軍。及還,改封觀王。遼東之役,檢校左翊衛大將軍,出遼東道。次瀘河鎮,
 遘疾死,帝為之廢朝,詔鴻臚監護喪事。有司請謚曰懿,帝曰:「王道高雅俗,德冠生靈。」乃謚曰德。贈司徒、襄國等十郡太守。



 子恭仁,位吏部侍郎。



 恭仁弟綝,性和厚,頗有文學。歷義州刺史、淮南郡太守。及父薨,起為司隸大夫。遼東之役,楊玄感反,其弟玄縱自帝所逃赴其兄,路逢綝,綝避人偶語久之。



 司隸刺史劉休文奏之,時恭仁將兵於外,帝寢其事。綝憂,發病而卒。



 雄弟達,字士達,有學行,仕周,位儀同、內史下大夫,封遂寧縣男。文帝受禪,拜給事黃門侍郎,進爵為子。遷兼吏部侍郎,加開府。轉內史侍郎、鄯鄭趙三州刺史,俱有能
 名。平陳後,帝差品天下牧宰,達為第一,擢拜工部尚書,加上開府。達為人弘厚,有局度,楊素每曰:「有君子貌兼君子心者,唯楊達耳。」獻皇后及文帝山陵制度,達並參預焉。煬帝嗣位,轉納言,領營東都副監。遼東之役,領右武衛將軍。進位左光祿大夫。卒於師。贈吏部尚書、始安侯,謚曰恭。



 王雅,字度容,闡熙新紵人也。少沈毅,木訥寡言。有膽勇,善騎射。周文聞其名,召入軍,以功賜爵居庸縣子。從禽竇泰於潼關。沙苑之戰,雅謂所部曰:「彼軍殆有百萬,今我不滿萬人,常理論之,實難與敵。但相公神武,以順討
 逆,豈計眾寡?大丈夫不以此時破賊,何用生為!」乃擐甲出戰,所向披靡,周文壯之。



 又從戰芒山。時大軍失利,諸將皆退,雅獨拒之。敵人見其無繼,步騎競進。雅左右奮擊,斬九級,敵眾稍退,雅乃還。周文歎曰:「王雅舉身悉是膽也!」進爵為伯。累遷驃騎大將軍、開府儀同三司。明帝初除汾州刺史。勵精為政,人庶悅附,自遠至者七百餘家。卒於夏州刺史。子世積嗣。



 世積容貌魁岸,腰帶十圍,風神爽拔,有人傑之表。在周,以功拜上儀同,封長子縣公。隋文帝受禪,進封宜陽郡公。高熲美其才能,甚善之。嘗謂穎曰:「吾輩俱周臣子,社
 稷淪沒,若何?」熲深拒之。未幾,授蘄州總管,平陳之役,以舟師自蘄水趣九江。以功進位柱國、荊州總管。後桂州人李光仕作亂,世積以行軍總管討平之,進位上柱國,甚見隆重。



 世積見帝性忌刻,功臣多獲罪,由是縱酒,不與執政言及時事。上以為有酒疾,舍之宮內,令醫者療之。世積詭稱疾愈,始得就第。及征遼東,世積與漢王並為行軍元帥。至柳城,遇疾而還。拜涼州總管,令騎士七百人送之官。



 未幾,其親信安定皇甫孝諧有罪,吏捕之,亡抵世積,不納,由是有憾。孝諧竟配防桂州,事總管令狐熙,熙又不禮焉。甚困窮,因徼幸上變,稱:「世積嘗令道
 人相其貴不,道人云:『當為國主。』謂其妻曰:『夫人當為皇后。』又將之涼州,其所親謂世積曰:『河西天下精兵處,可圖大事。』世積曰:『涼州土曠人稀,非用武國。』」由是被徵,案其事。有司奏:「左衛大將軍元旻、右衛大將軍元胄、左僕射高熲,並與世積交通,受其名馬之贈。」世積竟坐誅,旻胄等免官,拜孝諧為上大將軍。



 韓雄,字木蘭,河南東垣人也。祖景,孝文時為赭陽郡守。雄少敢勇,膂力絕人,工騎射,有將率材略。及孝武西遷,雄便慷慨有立功之志。大統初,遂與其屬六十餘人於洛西舉兵,數日間,眾至千人,與河南行臺楊琚共為掎
 角。每抄掠東魏,所向剋獲。東魏洛州刺史韓賢以狀聞,鄴乃遣其軍司慕容紹宗與賢合勢討雄。戰數十合,雄眾略盡,兄及妻子皆為賢所獲,將以為戮。乃遣人告雄曰:「若雄至,皆免之。」雄乃詣賢軍。即隨賢還洛。潛引賢黨,謀欲襲之。事洩,遁免。謁周文於弘農,封武陽縣侯,遣還鄉里,更圖進取。雄乃招集義眾,從獨孤信入洛陽。芒山之役,周文命雄邀齊神武於隘道。神武怒,命三軍拜并力取雄,雄突圍得免。除東徐州刺史。東魏雍州刺史郭叔略接境,頗為邊患。雄密圖之,輕將十騎,夜入其境,伏於道側,遣都督韓仕於略城服東魏人衣服,詐若自河陽叛投關西
 者,略出馳之。



 雄自後射之,再發咸中,遂斬略首。除河南尹,進爵為公。尋進驃騎大將軍、開府儀同三司、侍中、河南邑中正。周孝閔帝踐祚,進爵新義郡公,賜姓宇文氏。明帝二年,除都督、中州刺史。雄久在邊,具知敵人虛實,每率眾深入,不避艱難。前後經四十五戰,雖時有勝負,而雄志氣益壯,東魏深憚之。卒于鎮。贈大將軍、五州諸軍事。謚曰威。子禽嗣。



 禽字子通,少慷慨,以膽略稱。容貌魁岸,有雄傑之表。性又好書,經史百家皆略知大旨。周文見而異之,令與諸子遊集。以軍功稍遷儀同三司,襲爵新義郡公。



 武帝伐
 齊,禽說下獨孤永業於金墉城。及平范陽,加上儀同、永州刺史。隋文帝作相,遷和州刺史。陳將甄慶、任蠻奴、蕭摩訶等共為聲援,頻寇江北,前後入界。



 禽屢挫其鋒,陳人奪氣。



 開皇初,文帝潛有吞江南志,拜禽廬州總管,委以平陳之任,甚為敵人所憚。



 及大舉伐陳,以禽為先鋒。禽領五百人宵濟,襲採石,守者皆醉,遂取之。進攻姑熟,半日而拔。次於新林。江南父老素聞其威信,來謁軍門,晝夜不絕,其將樊巡、魯世真、田瑞等相繼降。晉王遣行軍總管杜彥與禽合軍。陳叔寶遣領軍蔡徵守朱雀航,聞禽將至,眾懼而潰。任蠻奴為賀若弼所敗,棄軍降禽。
 禽以精騎直入朱雀門。



 陳人欲戰,蠻奴捴之曰:「老夫尚降,諸君何事!」眾皆散走。遂平金陵,執陳主叔寶。時賀若弼亦有功,乃下詔晉王曰:「此二公者,朕本委之,悉如朕意。以名臣之功,成太平之業,天下盛事,何用過此!」又下優詔於禽、弼曰:「申國威於萬里,宣朝化於一隅,使東南之人俱出湯火,數百年賊旬日廓清,專是公之功也。



 高名塞於宇宙,盛業光於天壤。逖聽前古,罕聞其匹。班師凱入,誠知非遠,相思之甚,寸陰若歲。」及至京,弼與禽爭功於上前,弼曰:「臣在蔣山死戰,破其銳卒,禽其驍將,震揚威武,遂平陳國。禽略不交陣,豈臣之比!」禽曰:「本奉明
 旨,令臣與弼同取偽都。弼乃敢先期,逢賊遂戰,致將士傷死甚多。臣以輕騎五百,兵不血刃,直取金陵,降任蠻奴,執陳叔寶,據其府庫,傾其巢穴。弼至夕方扣北掖門,臣啟關而納之。斯乃救罪不暇,安得與臣為比!」上曰:』二將俱合上勳。」



 於是進位上柱國,賜物八千段。有司劾禽縱士卒淫汙陳宮。坐此不得國公及真食邑。



 大軍之始出也,上敕有司曰:「亡國物,我一不以入府,可於苑內築五垛,當悉賜文武百官大射以取之。」及是,上御玄堂,大陣陳之奴婢貨賄,會王公文武官七品已上,武職領兵都督已上,及諸考使以射之。



 先是,江東謠曰:「黃斑青驄
 馬,發自壽陽涘,來時冬氣末,去日春風始。」



 皆不知所謂。禽本名禽武,平陳之際,又乘青驄馬,往返時節與歌相應,至是方悟。



 後突厥來朝,上謂曰:「汝聞江南有陳國天子乎?」對曰:「聞之。」上命左右引突厥詣禽前,曰:「此是執得陳國天子者。」禽厲然顧之,突厥惶恐不敢仰視。其威容如此。別封壽光縣公,真食千戶。以行軍總管屯金城,禦備胡寇,即拜涼州總管。



 俄征還京,恩禮殊厚。無何,其鄰母見禽門下儀衛甚盛,有同王者,母異而問之。其中人曰:「我來迎王。」忽不見。又有人疾篤,忽驚走至禽家曰:「我欲謁王。左右問何王,曰:「閻羅王。」禽子弟欲撻之,禽止之
 曰:「生為上柱國,死作閻羅王,亦足矣。」因寢疾卒。子世諤嗣。



 世諤倜儻驍捷,有父風。楊玄感亂,引為將,每戰先登。玄感敗,為吏所拘。



 時帝在高陽,送詣行在所。世諤日令守者市酒肴以酣暢,揚言曰:「吾死在朝夕,不醉何為!」漸以酒進守者,守者狎之,遂飲令醉,因得逃奔山賊,不知所終。



 禽母弟僧壽,字玄慶,亦以勇烈知名。周武帝時,為侍伯中旅下大夫。隋文帝得政,從韋孝寬平尉遲迥。以功授大將軍。封昌樂縣公。開皇初,拜安州刺史。時禽為廬州總管,朝廷不欲其兄弟同在淮南,轉熊、蔚二州刺史,
 進爵廣陵郡公。尋以行軍總管擊破突厥於雞頭山。後坐事免。數歲,復拜蔚州刺史。突厥甚憚之。後檢校靈州總管事。從楊素破突厥,進位上柱國,改封江都郡公。



 煬帝即位,封新蔡郡公,自是不復任用。大業五年,從幸太原。時有京兆人達奚通妾王氏,能清歌,朝臣多相命觀之,僧壽亦預焉。坐除名。尋命復位,卒於京師。子孝基。



 僧壽弟洪,字叔明,少驍勇,善騎射,膂力過人。仕周,以軍功拜大都督。隋文為丞相,從韋孝寬破尉遲迥,加上開府,封甘棠縣侯。及帝受禪,進爵為公。開皇九年,平陳之後,授行軍總管。及陳平,晉王廣大獵於蔣山,有猛獸
 在圍中,眾皆懼,洪馳馬射之,應弦而倒。陳氏諸將列觀,皆歎伏焉。王大喜,賜縑百匹。尋以功加柱國,拜蔣州刺史,轉廉州。



 時突厥屢為邊患,朝廷以洪驍勇,令檢校朔州總管事。尋拜代州總管。仁壽元年,突厥達頭可汗犯塞,洪率蔚州刺史劉隆、大將軍李藥王拒之。遇虜於恆安,眾寡不敵,洪四面搏戰,身被重創,將士沮氣。虜悉眾圍之,矢下如雨。洪偽與虜和,圍少懈。洪率所領潰圍而出。死者太半,殺虜亦倍。洪及藥王除名,隆竟坐死。煬帝北巡,至恒安,見白骨被野,以問侍臣,曰:「往韓洪與虜戰處也。」帝憫然傷之,收葬骸骨,命五郡沙門為設齋供,
 拜洪隴西太守。



 未幾,朱崖人王萬昌作亂,詔洪平之。以功加金紫光祿大夫,領郡如故。俄而萬昌弟仲通復叛,又詔洪平之。還師未幾,旋遇疾卒。



 賀若敦,河南洛陽人也。其先居漠北,世為部落大人。曾祖貸,魏獻文時入國,為都官尚書,封安富縣公。祖伏連,仕魏,位雲州刺史。父統,勇健不好文學,以祖蔭為秘書郎。永安初,從太宰元天穆討邢杲,以功封當亭子。齊神武初起,以統為潁州長史。執刺史田迅,以州降,拜兗州刺史,賜爵當亭縣公。歷位北雍、恒二州刺史。卒,贈司空公,謚曰哀。敦少有氣幹。統之將執田迅也,慮事不果,又
 以累弱既多,難以自拔,沈吟者久之,敦年十七,進策贊成其謀。統流涕從之,遂定謀歸西。時群盜蜂起,大龜山賊張世顯潛來襲統,敦挺身赴戰,手斬七八人,賊乃走。統大悅,謂左右僚屬曰:「我少從軍旅,戰陣非一,如此兒年時膽略,未見其人。非唯成我門戶,亦當為國名將。」



 明年,從河內公獨孤信於洛陽被圍,敦彎三石弓,箭不虛發。信乃言於周文,引至麾下,授都督,封安陵縣伯。嘗從校獵甘泉宮,時圍人不齊,獸多越逸。周文大怒,人皆股戰。圍內唯有一鹿,俄亦突圍而走。敦躍馬馳之,鹿上東山。敦棄馬步逐,至山半,便乃掣之而下。周文大悅,諸將
 因得免責。累遷太子庶子。廢帝二年,拜右衛將軍。俄加驃騎大將軍、開府儀同三司,進爵廣鄉縣公。時岷蜀初開,人情尚梗。巴西人譙淹據南梁州,與梁西江州刺史王開業共為表裏,扇動群蠻。周文令敦討平之,進爵武都郡公,拜典祀中大夫。尋為金州都督。蠻帥向白彪、向五子王等聚眾為寇,圍逼信州。詔敦與開府田弘赴救,未至而城已陷。乃進軍追討,遂平信州。是歲,荊州蠻帥文子榮自號仁州刺史,復令敦與開府潘招討禽子榮,並虜其眾。



 武成元年,入為軍司馬。陳將侯瑱、侯安都等圍逼湘州,遏絕糧援,乃令敦度江赴救。敦連戰破瑱,乘
 勝遂次湘州。俄而秋水汛溢,江路遂斷。糧援既絕,恐瑱等知其糧少,乃於營內多為土聚,覆之以米,召側近村人,陽有所訪問,隨即遣之。



 瑱等聞之,良以為實。敦又增修營壘,造廬舍,示以持久。湘、羅之間遂廢農業。



 瑱等無如之何。初,土人亟乘輕船,載米粟及籠雞鴨以餉瑱軍。敦患之,乃偽為土人,裝船伏甲士於中。瑱軍人望見,謂餉船之至,逆來爭取,敦甲士遂禽之。又敦軍數有叛人乘馬投瑱,瑱輒納之。敦又別取一馬,牽以趣船,令船中逆以鞭鞭之。



 如是者再三,馬便畏船不上。後伏兵於江岸,使人乘畏船馬以招瑱軍。詐云投附。



 瑱便遣兵迎接,
 競來牽馬。馬既畏船不上,伏兵發,盡殺之。此後實有饋餉及亡奔瑱者,猶謂敦之詐,並不敢受。相持歲餘,瑱不能制,求借船送敦度江。敦慮其或詐,謂曰:「舍我百里,當為汝去。」瑱等遂留船,於是將兵去津路百里。敦覘之非詐,勒眾而還。在軍病死者十五六。晉公護以敦失地無功,除其名。



 保定五年,累遷中州刺史,鎮函谷。敦恃功負氣,顧其流輩皆為大將軍。敦獨未得,兼以湘州之役,全軍而反,翻被除名,每出怨言。晉公護怒,徵還,逼令自殺。臨刑,呼子弼謂曰:「吾必欲平江南,然心不果,汝當成吾志。吾以舌死,汝不可不思。」因引錐刺弼舌出血,誡以
 慎口。建德初,追贈大將軍。謚曰烈。



 弼字輔伯。少有大志,驍勇便弓馬,解屬文,博涉書記,有重名。周齊王憲聞而敬之,引為記室。封當亭縣公,遷小內史。與韋孝寬伐陳,攻拔數十城,弼計居多。拜壽州刺史,改封襄邑縣公。隋文帝為丞相,尉遲迥作亂,帝恐弼為變,遣長孫平馳驛代之。



 及帝受禪,陰有平江南志,訪可任者,高熲薦弼有文武才幹,於是拜吳州總管,委以平陳事,弼忻然以為己任。與壽州總管源雄並為重鎮。弼遺雄詩曰:「交河驃騎幕,合浦伏波營,勿使騏驎上,無我二人名。」獻取陳十策,上稱善,賜以寶刀。



 開皇九年,大
 舉伐陳,以弼為行車總管。將渡江,酹酒祝曰:「弼親承廟略,遠振國威,若使福善禍淫,大軍利涉;如事有乖違,得葬江魚腹中,死且不恨。」



 先是,弼請緣江防人每交代際,必集歷陽。於是大列旗幟,營幕被野,陳人以為大兵至,悉發國中士馬。既知防人交代,其眾復散。後以為常,不復設備。及此,弼以大軍濟江,陳人弗覺。襲陳南徐州,拔之,執其刺史黃恪。軍令嚴肅,秋毫不犯,有軍士於人間酤酒者,弼立斬之。進屯蔣山之白土岡,陳將魯廣達、周智安、任蠻奴、田瑞、孔範、蕭摩訶等以勁兵拒戰。田瑞先犯,擊走之。魯廣達等相繼遞進,弼軍屢卻。弼揣知其驕,
 士卒且惰,於是督萬將士,殊死戰,遂大破之。麾下士開府員明禽麾訶至,弼命左右牽斬之。摩訶色自若,弼釋而禮之。從北掖門入。時韓禽已執陳叔寶。弼至,呼叔寶視之。叔寶惶懼流汗,股慄再拜。弼謂曰:「小國之君當大國卿,拜,禮也。入朝不失作歸命侯,無勞恐懼。」



 既而弼恚恨不獲叔寶,於是與禽相訽,挺刃而出。令蔡徵為叔寶作降箋,命乘騾車歸己,事不果。上聞弼有功,大悅,下詔褒揚之。晉王以弼先期決戰,違軍命,於是以弼屬吏。上驛召之,及見。迎勞曰:「剋定三吳,公之功也。」命登御坐,賜物八千段,加位上柱國。進爵宋國公,真食襄邑三千
 戶,加寶劍、寶帶、金甕、金盤各一,並雉尾扇、曲蓋,雜彩二千段,女樂二部,又賜陳叔寶妹為妾。拜右領軍大將軍。



 平陳後六年,弼撰其畫策上之,謂為《御授平陳七策》。上弗省,曰:「公欲發揚我名,我不求名,公宜自載家傳。」七策:「其一,請廣陵頓兵一萬,番代往來。陳人初見設備,後以為常,及大兵南伐,不復疑也。其二,使兵緣江時獵,人馬喧噪。及兵臨江,陳人以為獵也。其三,以老馬多買陳船而匿之,買弊船五六十艘於瀆內。陳人覘以為內國無船。其四,積葦獲於揚子津,其高蔽艦。及大兵將度,乃卒通瀆於江。其五,塗戰船以黃,與枯荻同色,故陳人不預
 覺之。其六,先取京口倉儲,速據白土岡,置兵死地,故一戰而剋。其七,臣奉敕,兵以義舉。及平京口,俘五千餘人,便悉給糧勞遣,付其敕書,命別道宣喻。是以大兵度江,莫不草偃,十七日之間,南至林邑,東至滄海,西至象林,皆悉平定。」



 轉右武候大將軍。弼時貴盛,位望隆重,其兄隆為武都郡公,弟柬萬榮郡公,並刺史、列將。弼家珍玩不可勝計,婢妾曳綺羅者數百,時人榮之。



 弼自謂功名出朝臣之右,每以宰相自許。既而楊素為右僕射,弼仍為將軍,甚不平,形於言色,由是免官,弼怨望愈甚。後數載,下弼獄,上謂曰:「我以高熲、楊素為宰相,汝每昌言此
 二人唯堪啖飯耳,是何意也?」弼曰:「熲,臣之故人,素,臣之舅子,臣並知其為人,誠有此語。」公卿奏弼怨望,罪當死,上曰:「臣下守法不移,公可自求活理。」弼曰:「臣恃至尊威靈,將八千兵度江,即禽陳叔寶,竊以此望活。」上曰:「此已格外酬賞,何用追論!」弼曰:「平陳之日,諸公議不許臣行。推心為國,已蒙格外重賞,今還格外望活。」既而上低徊者數日,惜其功,特令除名。歲餘,復其爵位。上亦忌之,不復任使,然每宴賜,遇之甚厚。



 十九年,上幸仁壽宮,宴王公,詔弼為五言詩,詞意憤怨,帝覽而容之。明年春,弼又有罪,在禁所,詠詩自若。上數之曰:「人有性善行惡者,公
 之為惡,及與行俱。有三太猛:嫉妒心太猛,自是非人心太猛,無上心太猛,昔在周朝,已教他兒子反,此心終不能改邪?」他日,上謂侍臣曰:「初欲平陳時,弼謂高熲曰:『陳叔寶可平。不作高鳥盡,良弓藏邪?』熲云:『必不然。』平陳後,便索內史,又索僕射。我語熲曰:『功臣正宜授勳官,不可豫朝政。』弼後語熲:『皇太子於己,出口入耳,無所不盡。公終久何必不得弼力,何脈脈邪!』意圖鎮廣陵,又求荊州總管,並是作亂處,意終不改也。」



 後突厥入朝,上賜之射,突厥一發中的。上曰:「非弼無能當此。」乃命弼。



 弼再拜祝曰:「臣若赤誠奉國,當一發破的;如不然,發不中也。」弼射
 一發而中。



 上大悅,顧謂突厥曰:「此人天賜我也!」



 煬帝之在東宮,嘗謂曰:「楊素、韓禽、史萬歲三人,俱良將也,優劣如何?」



 弼曰:「楊素是猛將,非謀將;韓禽是鬥將,非領將;史萬歲是騎將,非大將。」



 太子曰:「然則大將誰也」?弼拜曰:「唯殿下所擇。」弼意自許為大將。及煬帝嗣位,尤被疏忌。大業三年,從駕北巡至榆林。時為大帳,下可坐數千人,召突厥啟人可汗饗之。弼以為太侈,與高熲、宇文幹等私議得失,為人所告,竟坐誅,時年六十四。妻子為官奴婢,群從徙邊。



 子懷亮,慷慨有父風。以柱國世子,拜儀同三司。坐弼為奴,俄亦誅死。



 敦弟誼。誼性剛果,有幹略。周文
 據關中,引之左右,累遷儀同三司、略陽公府長史。周閔帝受禪,封霸城縣子,加開府,歷原、信二州總管。及兄敦以讒毀伏誅,坐免官。從武帝平齊,拜洛州刺史,進封建威縣侯。開皇中,位左武候將軍、海陵郡公。後以突厥為邊患,誼素有威名,拜靈州刺史,進位柱國。誼時年老,猶能重鎧上馬,甚為北夷所憚。數載,上表乞骸骨,卒於家。子舉襲爵。



 論曰:周文帝屬禍亂之辰,以征伐而定海內,大則連兵百萬,繫之以存亡,小則轉戰邊亭,不闋於旬月。是以兵無少長,士無賢愚,莫不投筆要功,橫戈請奮。



 豆盧寧、楊
 紹、王雅、韓雄等,或攀翼雲漢,底績屯夷,雖運移年代,而名成終始,美矣哉!豆盧勣譽宣分竹,毓節見臨危,可謂載德象賢也。觀德王位登台兗,慶流後嗣,保茲寵祿,實仁厚之所致乎!王世積俊才雖多,適足為害者矣。賀若敦志略慷慨,深入敵境,勍寇絕其糧道,江淮阻其歸途。臨危而策出無方,事迫而雄心彌厲,故能利涉死地,全師以反。而茂勳莫紀,嚴刑已及,天下是以知宇文護之不能終其位也。自南北分隔,將三百年。隋文爰應千齡,將一函夏。賀若弼慷慨,申必取之長策,韓禽奮發,賈餘勇以爭先。隋氏自此一戎,威加四海。稽諸天道,或時有
 廢興;考之人謀,實二臣之力。其俶儻英略,賀弼居多,武毅威雄,韓禽稱重。



 方於晉之王、杜,勛庸綽有餘地。然賀弼功成名立,矜伐不已,竟顛殞於非命,亦不密以失身。若念父臨終之言,必不及於斯禍。韓禽累葉將家,威聲動俗,敵國既破,名遂身全,幸也。廣陵、甘棠,咸有武藝,驍雄膽略,並為當時所推,赳赳干城,難兄難弟矣。



\end{pinyinscope}