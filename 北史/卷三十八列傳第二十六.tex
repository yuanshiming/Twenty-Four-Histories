\article{卷三十八列傳第二十六}

\begin{pinyinscope}

 裴駿從孫敬憲莊伯從弟安祖裴延俊裴佗子讓之孫矩皇甫和裴果裴寬裴俠子祥肅裴文舉裴仁基裴駿,字神駒,小名皮,河東聞喜人也。父雙碩,位恆農太守、安邑子,贈東雍州刺史、聞喜侯。駿幼而聰慧。親表稱為神駒,因以為字。弱冠,通涉經史,方檢有禮度,鄉里宗敬焉。蓋吳作亂於關中,汾陰人薛永宗聚眾應之,來襲
 聞喜。縣令憂惶,計無所出。駿在家聞之,便率歷鄉豪奔赴之。賊退,刺史以狀聞。會太武親討蓋吳,引見駿。駿陳敘事宜,帝大悅,謂崔浩曰:「裴駿有當世才,其忠義可嘉。」補中書博士。浩亦深器駿,目為三河領袖。轉中書侍郎。宋使明僧皓來聘,以駿有才學,假給事中、散騎常侍,於境上勞接。卒,贈秦州刺史、聞喜侯,謚曰康。



 子修,字元寄。清辯好學,歷位祕書中散、主客令。累遷中大夫,兼祠部曹事,職主禮樂,每有疑議,修斟酌故實,咸有條貫。卒,謚曰恭伯,宣武時追贈東秦州刺史。修早孤,居喪以孝聞。二弟三妹,並在幼弱,撫養訓誨,甚有義方。次弟務早喪,
 修哀傷之,感於行路。愛育孤姪,同於己子,及將異居,奴婢田宅悉推與之,時人以此稱焉。



 子詢,字敬叔。美儀貌,多藝能,音律博弈,咸所閑解。位平昌太守。時太原長公主寡居,與詢私姦,明帝仍詔詢尚焉。尋以主婿,特除散騎常侍。時本邑中正闕,司徒召詢為之。詢族叔昞,自陳情願此官,詢遂讓焉。時論善之。尋監起居事,遷秘書監,出為郢州刺史。詢以凡司戍主蠻酋田朴特,地居要險,眾踰數萬,足為邊捍,遂表朴特為西郢州刺史。朝議許之。梁將李國興寇邊,朴特與部曲為表裏聲援,郢州獲全,朴特頗有力焉。徵為七兵尚書。武泰中,以本官兼侍
 中為關中大使。



 未及發,於河陰遇害。贈司空公,謚曰貞烈。無子。



 修弟宣,字叔令。通辯博物,早有聲譽。少孤,事母兄以孝友稱。司空李沖有人倫鑒,見而重之。孝文初,徵為尚書主客郎,累遷太尉長史。宣上言:自遷都以來,凡戰陣之處及軍罷兵還之道,所有骸骼無人覆藏者,請悉令州郡戍邏檢行埋掩。



 并符出兵之鄉,其家有死於戎役者,皆使招魂復魄,祔祭先靈,復其年租調。身被傷痍者,免其兵役。朝廷從之。出為益州刺史。宣至州綏撫,甚得戎羌之心。後晉壽更置益州,改宣所蒞為南秦州。



 宣家世以儒學為業,常慕廉退,每歎曰:「以賈誼之才,漢
 文之世,而不歷公卿,將非運也?」乃謂親賓曰:「吾本無當世之志,直隨牒至此,祿厚養親,效不光國,可以言歸矣。」因奉表求解。宣武不許,乃作《懷田賦》以敘心焉。宣素明陰陽之書,自始患便剋亡日,果如其言。贈豫州刺史,謚曰定,尋改為穆。子敬憲嗣。



 敬憲字孝虞,少有志行,學博才清,撫訓諸弟,專以讀誦為業。淡於榮利,風氣俊遠。郡征功曹不就,諸府辟命,先進其弟,世人歡美之。司州牧、高陽王雍舉秀才,射策高第,除太學博士。性和雅,未嘗失色於人。工隸草,解音律,五言之作,獨擅於時,名聲甚重,後進咸共宗慕之。中山
 將之部,朝賢送於河梁,賦詩言別,皆以敬憲為最。其文不能贍逸,而有清麗之美。少有氣病,年三十三卒,人物甚悼之。敬憲世有仁義於鄉里,孝昌中,蜀賊陳雙熾所過殘暴,至敬憲宅,輒相約束,不得焚燒,為物所伏如此。永興三年,贈中書侍郎,謚曰文。



 敬憲弟莊伯,字孝夏。亦有文才,器度閑雅,喜慍不形於色;博識多聞,善以約言辯物。司空、任城王澄辟為行參軍,甚加知賞。年二十一,上《神龜頌》,時人異之。文筆與敬憲相亞。臨淮王彧北討,引為記室參軍,委以章奏之事。及聞敬憲寢疾,求假不許,遂徑自還,亦矜而不問。扶侍
 兄病,晝夜不離於側,形容憔悴。



 因葬敬憲於鄉,遇病卒,年二十八。兄弟才學知名,同年俱喪,世共嗟惜之。永安三年,贈通直散騎侍郎,謚曰獻。兄弟並無子,所著詞藻,莫為集錄。



 莊伯弟獻伯,廷尉卿、濟州刺史,少以學尚風流,有名京洛。為政嚴酷,不得吏人之和,但以清白流譽。卒於殿中尚書。



 駿從弟安祖,少聰慧,年八九歲,就師講《詩》,至《鹿鳴篇》,語諸兄云:「鹿得食相呼,而況人乎。」自此未曾獨食。弱冠,州辟主簿。人有兄弟爭財,詣州相訟。安祖召其兄弟,以禮義責讓之。此人兄弟,明日相率謝罪。州內欽服之。



 後有
 人勸其仕進,安祖曰:「高尚之事,非敢庶幾,但京師遼遠,實憚於棲屑耳。」



 於是閑居養志,不出城邑。曾天熱,舍於樹下。有鷙鳥逐雉,雉急投之,遂觸樹而死。安祖愍之,乃取置陰地,徐徐護視,良久得蘇,喜而放之。後夜忽夢一丈夫,衣冠甚偉,著繡衣曲領,向安祖再拜。安祖怪問之,此人云:「感君前日見放,故來謝德。」聞者異焉。



 後孝文幸長安,至河東,存訪故老,安祖朝於蒲阪。帝與語甚悅,仍拜安邑令,以老病固辭,詔給一時俸以供湯藥焉。年八十三,卒於家。



 裴延俊,字平子,河東聞喜人也,魏冀州刺史徽之八世
 孫也。曾祖奣,諮議參軍、并州別駕。祖雙彪,河東太守,贈雍州刺史,謚曰順。父山松,州主簿,行平陽郡事,以平蜀賊丁蟲功,贈東雍州刺史。延俊少孤,事後母以孝聞。涉獵墳史,頗有才筆。舉秀才,射策高第,除著作佐郎,累遷太子洗馬,又領本邑中正。及太子恂廢,以宮官例免。宣武即位,為中書侍郎。時帝專心釋典,不事墳籍,延俊上疏致諫。後除司州別駕。及詔立明堂,群官博議,延俊獨著一堂之論。太傅、清河王懌時典眾議,讀而笑曰:「子故欲遠符僕射也。」明帝時,累遷幽州刺史。范陽郡有舊督亢渠,徑五十里;漁陽、燕郡有故戾陵諸堨,廣袤三十里,
 皆廢毀多時,莫能修復。時水旱不調,延俊乃表求營造。遂躬自履行,相度形勢,隨力分督,未幾而就,溉田百萬餘畝,為利十倍,百姓賴之。又命主簿酈惲修起學校,禮教大行,人歌謠之。在州五年,考績為天下最。拜太常卿,歷七兵殿中二尚書、散騎常侍、中書令、御史中尉,又以本官兼侍中、吏部尚書。延俊在臺閣,守職而已,不能有所裁斷直繩也。莊帝初,於河陰遇害。贈儀同三司、都督、雍州刺史。子元直、敬猷,並有學尚,與父同時遇害。元直贈光州刺史。敬猷妻丞相、高陽王雍外孫,超贈尚書僕射。延俊從叔愛醜、桃弓並見稱於鄉里。



 子夙,字買興。沈
 雅有器識,儀望甚偉,孝文見而異之。吏部尚書、任城王澄有知人鑒,每歎美夙,以遠大許之。位河北太守,以忠恕接下,百姓感而懷之。卒於郡。三子,範、昇之、鑒。



 鑒字道徽,性強正,有學涉,卒於廷尉卿。鑒居官清苦,時論稱之。贈東雍州刺史。



 子澤,頗有文學。齊孝昭初,為齋帥,奏舍人。孝昭崩,魏收議為恭烈皇帝,澤正色抗論曰:「魏收死後,亦不肯為恭烈之謚,何容以擬大行。且比皇太后不豫,先帝飧寢失常,聖躬貶損,今者易名,必須加孝。」遂改為孝昭。因此忤旨,出為廣州司馬。尋歷位中書侍郎,兼給事黃門侍郎,以漏泄免。後為散騎侍郎,尋為誹毀大
 臣趙彥深等,兼詠石榴詩,微以託意,有人以奏武成,武成決杖六十,髡頭除名。後主即位,為清河郡守。與祖珽有舊,珽奏除尚書左丞,又引為兼黃門。執政疾其祖珽之黨,與崔季舒等同見誅。



 澤本勁直,無所回避,及被出追還,折節和光。然好戲笑,無規檢,故頻敗。



 妻鉅鹿魏氏,恩好甚隆,不能暫相離,澤每從駕,其妻不宿。亦至性強立,時人以為健婦夫半。



 延俊從祖弟良,字元賓,稍遷尚書考功郎中。時汾州吐京胡薛羽等作逆,以良兼尚書左丞,為西北道行臺。時有五城郡山胡馮宜都、賀悅回成等,以妖妄惑眾,假稱帝號,服素衣,持白傘白幡,率諸
 逆眾,於雲臺郊抗王師。良大破之。又山胡劉蠡升,自云聖術,胡人信人,咸相影附,旬日之間,逆徒還振。以良為汾州刺史,加輔國將軍,行臺如故。良以城人饑窘,夜率眾奔西河。汾州之居西河,自良始也。



 孝靜初,為衛大將軍、太府卿,卒於官。贈吏部尚書,謚曰貞,又重贈侍中、尚書僕射。



 子叔祉,粗涉文學,居官甚著聲績,位終司空右長史。



 良從父兄子慶孫,字紹遠。少孤,性倜儻,重然諾。正光末,汾州吐京群胡薛悉公、馬牒騰並自立為王,眾至數萬。詔慶孫為募人別將,招率鄉豪以討之。慶孫每摧其鋒,進軍深入,至雲臺郊;大戰郊西,賊眾大潰。征赴都,
 除直後。於是賊復鳩集,北連蠡升,南通絳蜀,兇徒轉盛。以慶孫為別將,從軹關入討,深入二百餘里,至陽胡城。朝廷以此地被山帶河,衿要之所,明帝末,遂立邵郡,因以慶孫為太守。慶孫務安緝之,咸來歸業。爾朱榮之死也,世隆擁眾北度,詔慶孫為大都督,與行臺源子恭率眾追擊。慶孫與世隆密通,事洩,追還河內斬之。



 慶孫任俠有氣,鄉曲壯士及好事者多相依附,撫養咸有恩紀。在郡日,逢歲饑凶,四方遊客恆有百餘,慶孫自以家糧贍之。性雖麤武,愛好文流,與諸才學之士咸相交結。輕財重義,坐客恆滿,是以為時所稱。



 延俊從祖弟仲規,少
 好經史,頗有志節。咸陽王禧為司州牧,辟為主簿,仍表行建興郡事。車駕自代還洛,次於郡境。仲規備供帳,朝於路側。詔仲規曰:「畿郡望重,卿何能自致此也?」仲規曰:「陛下棄彼玄壤,來宅紫縣,臣方躍馬吳、會,冀功銘帝籍,豈一郡而已。」孝文笑曰:「冀卿必副此言。」駕還,見咸陽王曰:「昨得汝主簿為南道主人,六軍豐贍。元弟之寄,殊副所望。」除司徒主簿。



 仲規父在鄉疾病,棄官奔赴,以違制免。久之,中山王英徵義陽,引為統軍,奏復本資。於陣戰沒。贈河東太守,謚曰貞。無子,弟叔義以第二子伯茂後之。



 伯茂少有風望,學涉群書,文藻富贍,釋褐奉朝請。大
 將軍、京兆王繼西討,引為鎧曹參軍。南征絳蜀陳雙熾,為行臺長孫承業行臺郎中。承業還京師,留伯茂仍知行臺事。以平薛鳳賢等,賞平陽伯。再遷散騎常侍,典起居注。太昌初,為中書侍郎。永熙中,孝武帝兄子廣平王贊盛選賓寮,以伯茂為文學。後加中軍大將軍。



 伯茂好飲酒,頗涉疏傲。久不徙官,曾為《豁情賦》。天平初遷鄴,又為《遷都賦》。二年,因內宴,伯茂侮慢殿中尚書、章武王景哲。景哲遂申啟,稱伯茂棄其本列,與監同行,以梨擊案,傍汙冠服,禁庭之內,令人挈衣。詔付所司,後竟無坐。



 伯茂既出後其伯仲規,與兄景融別居。景融貧窘,伯茂了
 無賑恤,殆同行路,世以此貶薄之。卒,年三十九,知舊歎惜焉。



 伯茂末年,劇飲不已,乃至傷性,多有愆失。未亡前數日,忽云吾得密信,將被收掩,乃與婦乘車西逃避。後因顧指壁中,言有官人追逐,其妻方知其病。卒後,殯於家園。友人常景、李渾、王元景、盧元明、魏季景、李騫等十許人於墓傍置酒設祭,哀哭涕泣,一飲一酹,曰:「裴中書魂而有靈,知吾曹也。」乃各賦詩一篇。



 李騫以魏收亦與之友,寄以示收。收時在晉陽,乃同其作,論敘伯茂,其十字云:「臨風想玄度,對酒思公榮。」時人以伯茂性侮傲,謂收詩頗得事實。贈散騎常侍、衛將軍、度支尚書,雍州刺
 史,重贈吏部尚書,謚曰文。伯茂曾撰晉書,竟未能成。



 無子,兄景融以第二子孝才繼。齊武平末,位中書舍人。



 叔義亦有學行,累遷太山太守,為政清靜,吏人安之。遷司徒從事中郎。卒,贈東秦州刺史,謚曰宣。



 子景融,字孔明,篤學好屬文。舉秀才,射策高第,除太學博士,稍遷諫議大夫,領著作。元象中,儀同高岳以為錄事參軍。弟景龍、景顏被劾廷尉獄,景融入選。吏部擬郡,為御史中尉崔暹所彈,云其貪榮昧進,遂坐免官。病卒。景融卑退廉謹,無競於時,雖才不稱學,而緝綴無倦,文詞汎濫,理會處寡。所作文章,別有集錄。



 景顏頗有學尚,孝靜初,為司空
 長史,在官貪穢,為中尉崔暹所劾,遇病死獄中。



 延俊族兄聿,字外興,以操尚貞立,被孝文所知。為北中府長史。時帝以聿與中書侍郎崔亮清貧,欲以幹錄優之,乃以亮帶野王縣事,聿帶溫縣。時人榮之。卒於平秦郡太守,贈洛州刺史。子子袖入關西。



 延俊族人瑗,字珍寶,太和中析屬河北郡。少孤貧,清苦自立。為汝南王悅郎中令。孝靜初,卒於雍州刺史。



 延俊從父兄宣明,位華州刺史,有惠政,謚曰簡。二子景鸞、景鴻,並有逸才,河東呼景鸞為驥子,景鴻為龍文。景鸞位華州刺史。子文端,齊行臺郎。四子,願、安志、弘、振。景鴻,齊和夷郡守。子叔卿,博涉有
 孝行,時人號曰「裴曾子」。



 隋貝丘令。子神舉、神符,而神舉最知名。



 裴佗,字元化,河東聞喜人也。六世祖詵,仕晉位太常卿。因晉亂,避地涼州。



 苻堅平河西,東歸,因居解縣。世以文學顯,五舉秀才,再舉孝廉,時人美之。父景惠,州別駕。佗容貌魁偉,聵然有器望。舉秀才,以高第除中書博士。累遷趙郡太守,為政有方,威惠甚著,狡吏姦人,莫不改貫,所得俸祿,分恤貧窮。轉前將軍、荊州刺史,郡人戀仰,傾境餞送。蠻酋田盤石、田敬宗等部落萬餘家,恃眾阻險,不賓王命,前後牧守,未能降款。佗至州,單使宣慰,示以
 禍福,田敬宗聞風歸附。於是合境清晏,襁負至者千餘家。後加中軍將軍,以老乞還。卒,遺令不聽請贈,不受賵襚,諸子皆遵行之。



 佗性剛直,不好與俗人交游,其投分者必當時名勝。清白任真,不事家產,宅不過三十步,又無田園,暑不張蓋,寒不衣裘,其貞儉若此。子讓之。



 讓之字士禮,年十六喪父,殆不勝哀。其母辛氏泣撫之曰:「棄我滅性,得為孝子乎!」由是自勉。辛氏高明婦人,又閑禮度;夫喪,諸子多幼弱,廣延師友,或親自教授,內外親屬有吉凶禮制,多取則焉。



 讓之少好學,有文情,清明俊辯,早得聲譽。魏天平中,舉秀才,對策高第。



 累遷屯田、
 主客郎中,省中語曰「能賦詩,裴讓之」。為太原公開府記室。與楊愔友善,相遇則清談竟日。愔每云:「此人風流警拔,裴文季為不亡矣。」梁使至,常令讓之攝主客郎。



 第二弟諏之奔關右,兄弟五人皆拘繫。齊神武問云:「諏之何在?」答曰:「昔吳、蜀二國,諸葛兄弟各得盡心,況讓之老母在此,君臣分定,失忠與孝,愚夫不為。伏願明公以誠信待物。若以不收處物,物亦安能自信?以此定霸,猶卻行而求道耳。」神武善其言,兄弟俱釋。



 歷文襄大將軍主簿,兼中書舍人。後兼散騎常侍聘梁。文襄嘗入朝,讓之導引,容儀醖籍,文襄目之曰:「士禮,佳舍人也。」遷長兼中書
 侍郎,領舍人。齊受禪,靜帝遜居別宮,與諸臣別,讓之流涕歔欷。以參掌儀注,封寧都縣男。帝欲以為黃門侍郎,或言其體重不堪趨侍,乃除清河太守。至郡未幾,楊愔謂讓之諸弟曰:「我與賢兄交款,企聞善政,適有人從清河來,云姦吏斂迹,盜賊清靖。期月之期,翻更非速。」



 清河有二豪吏田轉貴、孫舍興,久吏奸猾,多有侵削,因事遂脅人取財,計贓依律不至死,讓之以其亂法,殺之。時清河王岳為司州牧,遣部從事案之。侍中高德政舊與讓之不協,密奏言:「當陛下受禪之時,讓之眷戀魏朝,嗚咽流涕,比為內官,情非所願。」既而楊愔請救之,云罪不合
 死。文宣大怒,謂愔曰:「欲得與裴讓之同冢邪!」於是無敢言者,事奏,竟賜死於家。



 讓之次弟諏之,字士正。少好儒學,釋褐太學博士。嘗從常景借書百卷,十許日便返。景疑其不能讀,每卷策問,應答無遺。景歎曰:「應奉五行俱下,禰衡一覽便記,今復見之於裴生矣。」楊愔闔門改葬,托諏之頓作十餘墓誌,文皆可觀。



 讓之、諏之及皇甫和、和弟亮,並知名於洛下。時人語曰:「諏勝於讓,和不如亮。」



 司空高乾致書曰:「相屈為戶曹參軍。」諏之復書不受署。沛王開大司馬府,辟為記室。遷鄴後,諏之留在河南。西魏領軍獨孤信入據金墉,以諏之為開府屬,號曰「洛陽
 遺彥」。信敗,諏之居南山,洛州刺史王元軌召為中從事。西師忽至,尋退,遂隨西師入關。周文帝以為大行臺倉曹郎中。卒,贈徐州刺史。



 次讞之,字士平。七歲便勤學,早知名。累遷司徒主簿。楊愔每稱歎曰:「河東士族,京官不少,唯此家兄弟,全無鄉音。」讞之雖年少,不妄交游,唯與隴西辛術、趙郡李繪、頓丘李構、清河崔贍為忘年友。昭帝梓宮將還鄴,轉儀曹郎。尤悉歷代故事,儀注、喪禮皆能裁正。為許昌太守,客旅過郡,皆出私財供給,人間無所預。代下日,為吏人所懷。仕周,卒伊川太守。



 次謀之,字士令。少有風格,邢邵每云「我裴四」。武成為開府,辟為參
 軍,掌書記。



 次訥之,字士言。純謹有局量。弱冠為平原公開府墨曹,掌書記,從至并州。



 其母在鄴,忽得心痛,訥之是日不勝思慕,心亦驚痛,乃請急而還。當時以為孝感。



 文宣踐阼,幸晉陽。皇太子監國;留訥之與杜臺卿並為齋帥,領東宮管記。轉太子舍人,奏中書舍人事。衛尉杜弼被其家客誣云「有怨言,誹訕時政」。并稱訥之與弼交好,亦知之。坐免官。卒,天統中追贈平州刺史。



 長子曰樊,出後讓之。次子矩,最知名。



 矩字弘大,襁褓而孤;及長,好學,頗愛文藻,有智數。世父讓之謂曰:「觀汝神識,足成才士,欲求宦達,當資幹世之
 務。」矩由是始留情世事。仕齊,為高平王文學。齊亡,不得調。隋文帝為定州總管,補記室,甚親敬之。以母憂去職。



 及帝作相,遣使馳召之,參相府記室事。受禪,遷給事郎,奏舍人事。伐陳之役,領元帥記室。既破丹陽,晉王廣令矩與高熲收陳圖籍。



 明年,奉詔巡撫嶺南。未行而高智慧、汪文進等作亂,吳、越道閉。上難遣矩行,矩請速進,上許之。行至南康,得兵數千人。時俚帥王仲宣逼廣州,遣其部將周師舉圖東衡州,矩與大將軍鹿愿赴之。賊立九柵,屯大庾嶺,共為聲援。矩進擊破之。賊懼,釋東衡州,據原長嶺,又擊敗之。遂斬師舉,進軍自南海拔廣州,仲
 宣懼而潰散。矩所綏集者二十餘州,又承制署渠帥為刺史縣令。及還,上大悅,命升殿勞苦之,謂高熲、楊素曰:「韋洸將二萬兵,不能早度嶺,每患其兵少。裴矩以三千弊卒徑至南海,有臣若此,朕亦何憂。」以功拜開府,賜爵聞喜縣公,賚物二千段。除戶部侍郎,遷內吏侍郎。



 時突厥強盛,都藍可汗妻大義公主即宇文氏女,由是數為邊患。後因公主與從胡私通,長孫晟先發其事,矩請出使說都藍,顯戳宇文。上從之,竟如其言。公主見殺後,都藍與突利可汗構難,屢犯亭鄣。詔太平公史萬歲為行軍總管,出定襄道,以矩為行軍長史,破達頭可汗於塞
 外。萬歲被誅,功竟不錄。上以啟人可汗初附,令矩撫慰之。還,為尚書左丞。其年,文獻皇后崩,太常舊無儀注,矩與牛弘、李百藥等據齊禮參定。轉吏部侍郎,名為稱職。煬帝即位,營建東都,矩職修府省,九旬功就。



 時西域諸蕃多至張掖與中國交市,帝令矩掌其事。矩知帝方勤遠略,諸胡至者,矩誘令言其國俗山川險易,撰西域圖記三卷,入朝奏之。其序曰:臣聞禹定九州,導河不踰積石。秦兼六國,設防止於臨洮。故知西胡雜種,僻居遐裔,禮教之所不及,書典之所罕傳。自漢氏興基,開拓河右,始稱名號者有四十六國。其後分立,乃五十五王。仍置
 校尉、都護,以存招撫。然叛服不恆,屢經征戰。後漢之世,頻廢此官;雖大宛以來,略知戶數,而諸國山川,未有名目。至如姓氏、風土、服章、物產,全無纂錄,世所弗聞。復以春秋遞謝,年代久遠,兼并誅討,互有興亡。或地是故邦,改從今號;或人非舊類,同襲昔名。兼復部人交錯,封疆移改,戎狄音殊,事難窮驗。于闐之北,蔥嶺以東,考於前史,三十餘國。



 其後更相屠滅,僅有十存,自餘淪沒,掃地俱盡,空有丘墟,不可記識。



 皇上應天育物,無隔華夷;率土黔黎,莫不慕化。風行所及,日入以來,職貢皆通,無遠不至。臣既因撫納,監知關市,尋討書籍,訪採胡人。或有
 所疑,即詳眾口,依其本國服飾儀形,王及庶人各顯容止,即丹青摸寫為《西圖域記》,共成三卷,合三十五國。仍別造地圖,窮其要害,從西頃以去,北海之南,縱橫所互,將二萬里。諒由富商大賈,周游經涉,故諸國之事,罔不偏知。復有幽荒遠地,卒訪難曉,不可憑虛,是以致闕。而二漢相踵,西域為傳,戶人數十,即稱國王,徒有名號,有乖其實。今者所編,皆餘千戶,利盡西海,多產珍異。見山居之屬,非有國名及部落小者,多亦不載。



 發自燉煌,至于西海,凡為三道,各有襟帶。北道從伊吾經蒲類海、鐵勒部、突厥可汗庭,度北流河水、至拂菻國,達于西海。其
 中道從高昌、焉耆、龜茲、疏勒,度蔥嶺,又經鏺汗、蘇勒沙那國、康國、曹國、何國、大小安國、穆國,至波斯,達于西海。其南道從鄯善、于闐、朱俱波、喝盤陀,度蔥嶺,又經護密、吐火羅、挹騑、忛延、漕國,至北婆羅門,達于西海。其三道諸國,亦各自有路,南北交通。其東安國、南婆羅門國等,並隨其所往,諸處得達。故知伊吾、高昌、鄯善並西域之門戶也,總湊燉煌,是其咽喉之地。



 以國家威德,將士驍雄,汎濛汜而揚旌,越崑崙而躍馬,易如反掌,何往不至。



 但突厥、吐谷渾分領羌胡之國,為其擁遏,故朝貢不通。今並因商人,密送誠款,引領翹首,願為臣妾。聖情含養,
 澤及普天,服而撫之,務在安輯。故皇華遣使,弗動兵車,諸蕃既從,突厥可滅。混一戎夏,其在茲乎。不有所記,無以表威化之遠也。



 帝大悅,賜物五百段,每日引矩至御坐,親問西方之事。矩盛言胡中多諸寶物,吐谷渾易可並吞。帝由是甘心,將通西域,西夷經略,咸以委之。



 後遷黃門侍郎,復令往張掖,引致西蕃,至者十餘國。大業三年,帝有事於恒嶽,咸來助祭。帝將巡河右,復令矩往敦煌,矩遣使說高昌王麴伯雅及伊吾吐屯設等,啖以厚利,導之使入朝。及帝西巡,次燕支山。高昌王、伊吾設等及西蕃胡二十七國謁於道左,皆令佩金玉,被錦罽,焚
 香奏樂,歌舞喧噪。復令張掖、武威士女盛飾縱觀,填咽周互數十里,以示中國之盛。帝見而大悅。竟破吐谷渾,拓地數千里。並遣兵戍之,每歲委輸巨億萬計。諸蕃懼懾,朝貢相續。帝謂矩有綏懷略,進位銀青光祿大夫。



 其年冬,帝至東都。矩以蠻夷朝貢者多,諷帝令都下大戲,征四方奇伎異藝陳於端門街,衣錦綺、珥金翠者以十萬數。又勒百官及百姓士女列坐棚閣而縱觀焉,皆被服鮮麗,終月而罷。又令交市店肆皆設帷帳,盛酒食,遣掌蕃率蠻夷與人貿易,所至處悉令邀延就坐,醉飽而散。蠻夷嗟歎,謂中國為神仙。帝稱矩至誠,謂宇文述、牛
 弘曰:「裴矩凡所陳奏,皆朕之成算,朕未發,矩輒以聞。自非奉國,孰能若是。」



 帝遣將軍薛世雄城伊吾,令矩共往經略。矩諷諭西域諸國曰:「天子為蕃人交易懸遠,所以城耳。」咸以為然,不復來競。及還,賜錢四十萬。矩又白狀,令反間射匱,潛攻處羅。後處羅為射匱所迫,竟隨使者入朝。帝大悅,賜矩貂裘及西域珍器。



 從帝巡塞北,幸啟人帳。時高麗遣使先通於突厥,啟人不敢隱,引之見帝。矩因奏曰:「高麗地本孤竹國,周代以之封箕子,漢世分為三郡,晉氏亦統遼東。今乃不臣,列為外域,故先帝欲征之久矣。但以楊諒不肖,師出無功。當陛下時,安得不
 事,使此冠帶之境仍為蠻貊之鄉乎?今其使朝於突厥,親見啟人合國從化,必懼皇靈之遠暢,慮後服之先亡,脅令入朝,當可致也。」帝曰:「如何?」矩曰:「請面詔其使,放還本國,遣語其王,令速朝覲。不然者,當率突厥,即日誅之。」



 帝納焉。高元不用命,始建征遼之策。



 王師臨遼,以本官領武賁郎將。明年,復從至遼東。兵部侍郎斛斯政亡入高麗,帝令矩兼掌兵事。以前後度遼功,進位右光祿大夫。



 時皇綱不振,人皆變節,左翊衛大將軍宇文述、內史侍郎虞世基等用事,文武多以賄聞。唯矩守常,無贓穢之響,以是為世所稱。後以楊玄感初平,帝令矩安集隴
 右,因之會寧,存問曷薩那部落,遣闕達度設寇吐谷渾,頻有虜獲,部落致富。



 還而奏狀,帝大賞之。後從至懷遠鎮,詔護北蕃軍事。



 矩以始畢可汗部眾漸盛,獻策分其勢。將以宗女嫁其弟叱吉設,拜為南面可汗。



 叱吉不敢受,始畢聞而漸怨。矩又曰:「突厥本淳,易可離間,由其內多有眾胡,盡皆桀黠,教導之耳。臣聞史蜀胡悉尤多奸計,幸於始畢,請誘殺之。」帝曰:「善。」矩因遣人告胡悉曰:「天子大出珍物,今在馬邑,欲共蕃內多作交關,若前來者,即得好物。」胡悉信之,不告始畢,率其部落,盡驅六畜爭進,冀先互市。



 矩伏兵馬邑,誘而斬之。詔報始畢曰:「史蜀
 胡悉忽領部落,走來至此,云背可汗,請我容納。今已斬之,故令往報。」始畢亦知其狀,由是不朝。



 十一年,帝北巡狩,始畢率騎數十萬圍帝於鴈門,詔矩與虞世基宿朝堂以待顧問。及圍解,從至東都。屬射匱可汗遣其猶子率西蕃諸胡朝貢,詔矩宴接之。



 尋從幸江都宮。時四方盜賊蜂起,郡縣上奏者不可勝計。矩言之,帝怒,遣矩詣京師接蕃客。以疾不行。及義兵入關,帝遣虞世基就宅問矩方略。矩曰:「太原有變,京畿不靜,遙為處分,恐失事機,唯願鑾輿早還。」俄而驍衛大將軍屈突通敗問至,矩以聞,帝失色。矩素勤謹,未嘗忤物,又見天下方亂,恐為
 身禍,其待遇人,多過其所望,故雖廝役,皆得其嘆心。



 時從駕驍果數有逃散。帝憂之,以問矩。矩曰:「今車駕留此,已經二年。驍果之徒,盡無家口,人無匹合,則不能久安。臣請聽兵士於此納室。」帝大書曰:「公定多智,此奇計也。」因令矩檢校為將士等娶妻。矩召江都境內寡婦及未嫁女皆集宮監。又召諸將帥及兵等恣其所取。因聽自首,先有奸通婦女及尼、女官等,並即配之。由是驍果等悅,咸相謂曰:「裴公之惠也。」



 宇文化及反。矩晨起將朝,至坊門,遇逆黨數人,控矩馬詣孟景所。賊皆曰:「不關裴黃門。」既而化及從百餘騎至,矩迎拜,化及慰諭之。令矩參
 定儀注,推秦王子浩為帝。以矩為侍內,隨化及至河北。化及僭帝號,以矩為尚書右僕射,加光祿大夫,封蔡國公,為河北道宣撫大使。



 及宇文氏敗,為竇建德所獲。以矩隋代舊臣,遇之甚厚。復以為吏部尚書,轉尚書右僕射。建德起自群盜,未有節文,矩為之制定朝儀,旬月之間,憲章頗擬於王者。建德大悅。及建德敗時,矩與其將曹旦等於洛州留守。旦長史李公淹及大唐使人魏徵等說旦及齊善行,令矩歸順。旦等從之,乃令矩與徵、公淹領旦及八璽,舉山東之地歸降。授左庶子,轉詹事、戶部尚書,卒。



 讓之第六弟謁之,字士敬。少有志節,好直言。
 文宣末年昏縱,朝臣罕有言者。



 謁之上書正諫,言甚切直。文宣將殺之,白刃臨頸,謁之辭色不變。帝曰:「癡漢何敢如此!」楊愔曰:「望陛下放以取後世名。」帝投刀歎曰:「小子望我殺爾以取後世名,我終不成爾名。」遣人送出。齊亡,卒於壺關令。



 皇甫和者,字長諧,安定朝那人。其先因官,寓居漢中。祖澄,南齊秦、梁二州刺史。



 父征,字子玄,梁安定、略陽二郡守。魏正始二年,隨其妻父夏侯道遷入魏。



 道遷別上勛書,欲以徽為元謀。徽曰:「創謀之始,本不關預,雖貪榮賞,內媿於心。」遂拒而不許。梁州刺史羊靈祐重其敦實,表
 為征虜府司馬,卒。



 和十一而孤。母夏侯氏才明有禮則,親授以經書。及長,深沈有雅量,尤明禮義,宗親吉凶,多相諮訪。卒於濟陰太守。子聿道,以幹局知名,位廣平令。隋大業初,比部郎。



 和弟亮,字君翼。九歲喪父,哀毀有若成人。齊神武起義,為大行臺郎中。亮率性任真,不樂劇職,除司徒東閣祭酒。思還鄉里,啟乞梁州褒中,即本郡也。後降梁。以母兄在北,求還。梁武不奪也。至鄴,無復宦情,遂入白鹿山,恣泉石之賞,縱酒賦詩,超然自樂。復為尚書殿中郎,攝儀曹事。以參撰禪代儀注,封榆中男。亮疏慢自任,無幹務才,每有禮儀大事,常令餘司攝焉。



 性
 質朴純厚,終無片言矯飾。屬有敕下司,各列勤惰。亮三日不上省,文宣親詰其故。亮曰:「一日雨,一日醉,一日病酒。」文宣以其恕實,優容之,杖脛三十而已。所居宅洿下,標榜賣之。將買者或問其故,亮每答云:「為宅中水淹不洩,雨即流入床下。」由此宅終不售。其淳實如此。



 以兼散騎常侍,聘陳使主,以不稱免官。後除任城太守,病不之官,卒於鄴。



 贈驃騎大將軍、安州刺史。



 裴果,字戎昭,河東聞喜人也。祖思賢,魏青州刺史。父遵,齊州刺史。果少慷慨有志略。魏太昌中,為陽平郡丞。周文帝曾使并州,與果遇。果知非常人,密託附焉。永安末,
 盜賊蜂起,果從軍征討。乘黃驄馬,衣青袍,每先登陷陣,時人號為「黃驄年少。」永熙中,授河北郡守。



 及齊神武敗於沙苑,果乃率其宗黨歸闕。周文嘉之,賜田宅奴婢牛馬什物等。



 從戰河橋,解玉壁圍;摧鋒奮擊,所向披靡。大統九年,又從戰芒山。於周文前挺身陷陣,禽東魏都督賀婁焉邏蘭。勇冠當時,眾人莫不歎服。以此周文愈親待之。



 補帳內都督,遷帥都督、平東將軍。後從開府楊忠平隨、安陸,以功加大都督,除正平郡守。正平,果本郡也,以威猛為政,百姓畏之,資賊亦為之屏息。遷司農卿。



 又從大將軍尉遲迥伐蜀,果率所部為前軍。開劍閣,破季
 慶堡,降楊乾運,皆有功。



 廢帝三年,授龍州刺史,封冠軍縣侯。俄而州人張遁、李拓驅率百姓,圍逼州城;時糧仗皆闕,兵士又寡。果設方略以拒之,賊便退走。於是出兵追擊,累戰破之;旬日之間,州境清晏。轉陵州刺史。



 周孝閔帝踐阼,除隆州刺史,加持節、驃騎大將軍、開府儀同三司,進爵為公。



 歷眉、復二州刺史。果性嚴猛,能斷決。抑挫豪右,申理屈滯,歷牧數州,號為稱職。卒於位。贈本官,加絳、晉、建州刺史,謚曰質。子孝仁嗣。



 孝仁幼聰敏,涉獵經史,有譽於時。起家舍人上士,累遷長寧鎮將,扞禦齊人,甚有威邊之略。歷建、譙、亳三州刺史。



 裴寬,字長寬,河東聞喜人也。祖德歡,魏中書侍郎、河內郡守。父靜慮,銀青光祿大夫,贈汾州刺史。寬儀貌瑰偉,博涉群書,弱冠為州里所稱。親歿,撫諸弟以篤友聞,滎陽鄭孝穆嘗謂其從弟文直曰:「裴長寬兄弟,天倫篤睦,人之師表,吾愛之重之,汝可與之游處。」年十三,以選為魏孝明帝挽郎,釋褐員外散騎侍郎。



 及孝武西遷,寬謂其諸弟曰:「君臣逆順,大義昭然。今天子西幸,理無東面以虧臣節。」乃將家屬避難於大石嶺。獨孤信鎮洛陽,始出見焉。時汾州刺史韋子粲降於東魏,子粲兄弟在關中者咸已從坐。其季弟子爽先在洛,窘急乃投寬,寬開
 懷納之。遇有大赦,或傳子爽合免,因爾遂出,子爽卒以伏法。獨孤信知而責之,寬曰:「窮來見歸,義無執送,今日獲罪,是所甘心。」以經赦宥,遂得不坐。



 大統五年,授都督、同軌防長史,加征虜將軍。十三年,從防主韋法保向潁川,解侯景圍。景密謀南叛,偽親狎於法保。寬謂法保曰:「侯景狡猾,必不肯入關,雖託款於公,恐未可信。若伏兵以斬之,亦一時之功也。如曰不然,便須深加嚴警,不得信其誑誘,自貽後悔。」法保納之。然不能圖景,但自固而已。



 十四年,與東魏將彭樂、樂恂戰於新城,因傷被禽。至河陰,見齊文襄。寬舉止詳雅,善於占對,文襄甚賞異之;
 解鎖付館,厚加禮遇。寬乃裁所臥氈,夜縋而出,因得遁還,見於周文帝。帝顧謂諸公曰:「被堅執銳,或有其人;疾風勁草,歲寒方驗。裴長寬為高澄如此厚遇,乃能冒死歸我,雖古之竹帛所載,何以加之。」



 乃手書署寬名下,授持節、帥都督,封夏陽縣男,即除孔城城主。



 十六年,遷河南郡守,仍鎮孔城。廢帝元年,進使持節、車騎大將軍、儀同三司、散騎常侍。周孝閔帝踐阼,進爵為子。寬在孔城十三年,與齊洛州刺史獨孤永業相對。永業有計謀,多譎詐。或聲言春發,秋乃出兵;或掩蔽消息,倏忽而至。



 寬每揣知其情,出兵邀擊,無不剋之。



 天和三年,除溫州刺
 史。初,陳氏與周通和,每修聘好。自華皎附後,乃圖寇掠。沔州既接敵境,於是以寬為沔州刺史。陳將程靈洗攻之,力屈城陷。陳人乃執寬至揚州,尋被送嶺外,經數載,復還建鄴,遂卒於江左。子義宣後從御正杜果使於陳,始得將寬柩還。隋開皇元年,文帝詔贈襄、郢二州刺史。義宣,位司金二命士、合江令。



 寬弟漢,字仲霄。操尚弘雅,聰敏好學,嘗見人作百字詩,一覽便誦。魏孝武初,解褐員外散騎侍郎。大統五年,除大丞相府士曹行參軍,轉墨曹。漢善尺牘,尤便簿領,理識明贍,斷割如流。相府為之語曰「日下粲爛有裴漢。」武成中,為司車路下大夫,與
 工部郭彥、太府高賓等參議格令。每較量時事,必有條理。天和五年,加車騎大將軍、儀同三司。



 漢少有宿疾,恆帶虛羸,劇職煩官,非其好也。時晉公護擅權,搢紳等多諂附之以圖仕進。漢直道自守,故八年不徙職。性不飲酒,而雅好賓游。每良辰美景,必招引時彥,宴賞留連,間以篇什,當時人物,以此重之。自寬沒後,遂斷絕游從,不聽琴瑟;歲時伏臘,哀慟而已。撫養兄弟子,情甚篤至。借人異書,必躬自錄本,至于疾診彌年,亦未嘗釋卷。卒,贈晉州刺史。



 子鏡人,少聰敏,涉獵經史。為大將軍、譚公會記室參軍,累遷春官府都上士。



 仕隋,位兵曹郎。漢弟尼,
 字景尼,性弘雅,有器局,位御正下大夫。卒,贈隨州刺史。子之隱,趙王招府記室參軍。之隱弟師人,好學有識度,見稱於時。起家秦王贄府記室參軍,仍兼侍讀。



 寬族弟鴻,少恭謹,有幹略。歷官內外。周天和初,拜郢州刺史,轉襄州總管府長史,賜爵高邑縣侯。從衛公直南征,軍敗遂沒,尋卒於陳。朝廷哀之,贈豐、資、遂三州刺史。



 裴俠,字嵩和,河東解人也。祖思齊,舉秀才,拜議郎。父欣,西河郡守,贈晉州刺史。俠年七歲,猶不能言。後於洛城見群烏蔽天從西來,舉手指之而言。遂志識聰慧,有異常童。年十三,遭父憂,哀毀有若成人。將擇葬地而行,空
 中有人曰:「童子何悲,葬於桑東,封公侯。」俠懼,以告其母。母曰:「神也,吾聞鬼神福善,爾家未嘗有惡,當以吉祥告汝耳。」時俠宅側有大桑林,因葬焉。州辟主簿,舉秀才。



 魏正光中,解巾奉朝請,稍遷義陽郡守。元顥入洛,使執其使人,焚其赦書。



 孝莊嘉之,授東郡太守,帶防城別將。及孝武與齊神武有隙,徵兵,俠率所部赴洛陽。武衛將軍王思政謂曰:「當今權臣擅命,王室日卑,若何?」俠曰:「宇文泰為三軍所推,居百二之地,所謂己操戈矛,寧肯授人以柄,雖欲撫之,恐是『據於蒺藜』也」。思政曰:「奈何?」俠曰:「圖歡有立至之憂,西巡有將來之慮。且至關右,日慎一日,
 徐思其宜耳。」思政然之,乃進俠於帝,授左中郎將。及帝西遷,俠將行而妻子猶在東郡。滎陽鄭偉謂俠曰:「天下方亂,未知烏之所集,何如東就妻子,徐擇木焉。」俠曰:「既食人祿,寧以妻子易圖也?」遂從入關。賜爵清河縣伯,除丞相府士曹參軍。



 大統三年,領鄉兵從戰沙苑,先鋒陷陣。俠本名協,至是周文帝嘉其勇決,乃曰:「仁者必勇。」因命名俠焉。以功進爵為侯。王思政鎮玉壁,以俠為長史。齊神武以書招思政,思政令俠草報書甚壯烈。周文善之曰:「雖魯仲連無以加也。」



 除河北郡守。



 俠躬履儉素,愛人如子,所食唯菽麥鹽菜而已,吏人莫不懷之。此郡舊
 制,有漁獵夫三十人以供郡守。俠曰:「以口腹役人,吾所不為也。」乃悉罷之。又有丁三十人,供郡守役,俠亦不以入私,並收庸為市官馬。歲時既積,馬遂成群。去職之日,一無所取。人歌曰:「肥鮮不食,丁庸不取;裴公貞惠,為世規矩。」俠嘗與諸牧守俱謁周文,周文命俠別立,謂諸牧守曰:「裴俠清慎奉公,為天下之最。」



 令眾中有如俠者,可與之俱立。眾皆默然,無敢應者。周文乃厚賜俠,朝野服焉,號為「獨立使君」。



 又撰九世伯祖《貞侯潛傳》,述裴氏清公,欲使後生奉而行之。宗室中知名者,咸付一通。從弟伯鳳、世彥時並為丞相府佐,笑曰:「人生仕進,須身名並
 裕,清苦若此,竟欲何為?」俠曰:「夫清者蒞職之本,儉者持身之基。況我大宗,世濟其美,故能存見稱於朝廷,沒流芳於典策。今吾幸以凡庸,濫蒙殊遇,固其窮困,非慕名也。志在自修,懼辱先也,翻被嗤笑,知復何言!」伯鳳等慚而退。



 再遷郢州刺史,加儀同三司。梁竟陵守孫皓、酂城守張建並以郡來附。俠見之,密謂人曰:「皓目動言肆,輕於去就者也;建神情審定,當無異心。」乃馳啟其狀。



 周文曰:「裴俠有鑒,深得之矣。」遣大都督苻貴鎮竟陵,而酂城竟不遣監統。及柳仲禮軍至,皓還以郢叛,卒如俠言。尋轉大將軍、拓州刺史,徵拜雍州別駕。



 周孝閔帝踐作,除
 司邑下大夫,加驃騎大將軍、開府儀同三司,進爵為公。遷戶部中大夫。時有姦吏主守倉儲,積年隱沒至千萬者。及俠在官,勵精發擿,數旬之內,姦盜略盡。轉工部中大夫。有大司空掌錢物典李貴乃於府中悲泣,或問其故,對曰:「所掌官物,多有費用,裴公清嚴有名,懼遭罪責,所以泣耳。」俠聞之,許其自首。貴自言隱費錢五百萬。



 俠嘗遇疾沈頓,士友憂之。忽聞五鼓,便即驚起,顧左右曰:「可向府耶。」



 所苦因此而瘳。晉公護聞之曰:「裴俠危篤若此而不廢憂公,因聞鼓聲,疾病遂愈,此豈非天祐其勤恪也?」又司空許國公宇文貴、小司空北海公申徵並來
 侯俠疾。所居第屋,不免霜露。貴等還,言之於帝。帝矜其貧苦,乃為起宅,并賜良田十頃,奴隸耕耒糧粟莫不備足。搢紳咸以為榮。卒於位,贈太子少師、蒲州刺史,謚曰貞。



 河北郡前功曹張回及吏人等感俠遺愛,乃作頌紀其清德焉。



 子祥,性忠謹,有理劇才。少為城都令,清不及俠,斷決過之。後除長安令,為權貴所憚。遷司倉下大夫。俠之終也,以毀卒。祥弟肅。



 肅字神封,貞亮有才藝。少與安定梁毗同志友善。天和中,舉秀才。累遷御正下大夫,以行軍長史從韋孝寬征淮南。屬隋文帝為丞相,肅聞而歎曰:「武帝以雄才定六
 合,墳土未乾而一朝遷革,豈天道歟!」文帝聞之,甚不悅,由是廢于家。



 開皇五年,授膳部侍郎。歷朔州總管長史、貝州長史,俱有能名。



 仁壽中,肅見皇太子勇、蜀王秀、左僕射高熲俱廢黜,遣使上書,言:「高熲天挺良才,元勳佐命,願錄其大功,忘其小過。二庶人得罪已久,寧無革心,願各封小國,觀其所為。若得遷善,漸更增益;如或不悛,貶削非晚。」書奏,上謂楊素曰:「肅憂我家事如此,亦至誠也。」於是徵肅入朝。皇太子聞之,謂左庶子張衡曰:「使勇自新,欲何為也?」衡曰:「觀肅意欲令如吳太伯、漢東海王耳。」



 太子甚不悅。肅至京,見上於含章殿。上謂曰:「貴為天
 子,富有四海,後宮寵幸,不過數人,自勇以下,並皆同母,非為愛憎,輕事廢立。」因言勇不可復收之意。



 既已,罷遣之。未幾,上崩。煬帝嗣位,不得調者久之,肅亦杜門不出。後執政者以嶺表遐遠,希旨授肅永平郡丞,甚得夷人心。歲餘卒,夷獠思之,為立廟於鄣江之浦。有子尚賢。



 裴文舉,字道裕,河東聞喜人也。祖秀業,魏天水郡守,贈平州刺史。父邃,性方嚴,為州里所推挹。大統三年,東魏來寇,邃乃糾合鄉人,分據險要以自固。



 及李弼略地東境,邃為之鄉導,多所降下。周文帝嘉之,特賞衣物,封澄城縣子。



 卒於正平郡守,贈儀同三司、定州刺史。



 文舉少
 忠謹,涉獵經史。大統十年,起家奉朝請。時周文帝諸子年幼,盛簡賓友。文舉以選與諸公子游,雅相欽敬,未嘗戲狎。遷著作郎、中外府參軍。恭帝二年,賜姓賀蘭氏。周孝閔帝踐阼,襲爵澄城縣子。



 齊公憲初開幕府,以文舉為司錄。及憲出鎮劍南,復以文舉為總管府中郎。武成二年,就加使持節、車騎大將軍、儀同三司。蜀土沃饒,商販百倍,或有勸文舉以利者,文舉答之曰:「利之為貴,莫若安身,身安則道隆。非貨之謂,是以不為,非惡財也。」憲矜其貧窶,每欲資給之。文舉恆自謙遜,辭多受少。



 保定三年,遷絳州刺史。邃之任正平也,以廉約自守。每行春
 省俗,單車而已。



 及文舉臨州,一遵其法,百姓美而化之。總管韋孝寬特相欽重,每與談論,不覺膝前於席。天和初,進驃騎大將軍、開府儀同三司,尋為孝寬柱國府司馬。六年,入為司憲中大夫,進爵為伯,轉軍司馬。



 文舉少喪父,其兄又在山東,唯與弟璣幼相訓養,友愛甚篤。璣又早亡,文舉撫視遺孤,逾於己子,時人以此稱之。初,文舉叔父季和為曲沃令,終於聞喜川;而叔母韋氏卒於正平縣,屬東西分隔,韋氏墳隴,遂在齊境。及文舉在本州,每加賞募。齊人感其孝義,潛相要結,以韋柩西歸,竟得合葬。六年,除南青州刺史。



 宣政元年,卒於位。子胄嗣,
 位至大都督。子神,安邑通守。有子知禮。



 裴仁基,字德本,河東人也。祖伯鳳,周汾州刺史。父定,上儀同。仁基少驍武,便弓馬。平陳之役,以親衛從征,先登陷陣,拜儀同,賜物千段。以本官領漢王諒府親信。諒反,仁基苦諫見囚。諒敗,超拜護軍。後改授武賁郎將,從將軍李景討叛蠻向思多於黔安,以功進銀青光祿大夫。擊破吐谷渾,加授金紫光祿大夫。



 斬獲寇掠靺鞨,拜左光祿大夫。從征高麗,進位光祿大夫。



 李密據洛口,帝令仁基為河南道討捕大使,據武牢拒密。仁基見強寇在前,士卒勞弊,所得軍資,即用分賞。臨軍御史蕭懷靜止
 之,眾咸怒懷靜。懷靜又陰持仁基長短,欲有奏劾。仁基懼,殺懷靜,以其眾歸密。密以為河東郡公。其子行儼,驍勇善戰。密復以為絳郡公,甚相委暱。



 王世充以東都食盡,悉眾詣偃師,求決戰。密與諸將計。仁基曰:「世充盡銳而至,洛下必虛。可分兵守其要路,令不得東;簡精兵三萬,傍河西出,以逼東都。



 世充卻還,我且按甲。世充重出,我又逼之。如此,則我有餘力,彼勞奔命。兵法所謂彼出我歸,彼歸我出,數戰以疲之,多方以誤之者也。」密曰:「公知其一,不知其二。東都兵馬有三不可當:器械精一也,決計而來二也,食盡求斗三也。我按兵蓄力以觀其弊,
 彼求鬥不得,欲走無路。不過十日,世充之首可懸於麾下。」



 單雄信等諸將輕世充,皆請戰。仁基苦爭不得。密難違諸將言,戰遂大敗。仁基為世充所虜。世充以仁基父子並驍勇,深禮之,以兄女妻行儼。及僭尊號,署仁基為禮部尚書,行儼為左輔大將軍。行儼每戰,所當皆披靡,號萬人敵。世充憚其威名,頗加猜防。仁基知之,甚不自安,遂與世充所署尚書左丞宇文儒童、尚食直長陳謙、秘書丞崔德本等謀。令陳謙於上食之際,持匕首劫世充,行儼以兵應之。事定,然後輔越王侗。事臨發,將軍張童兒告之,俱為世充所殺。



 論曰:裴駿雅業有資,器行仍世,所以布於列位,不替其美。延俊器能位望,有可稱乎。伯茂才名,亦時之良也。元化以文學傳業,而又修史著美。讓之弟兄,修身厲行,觀夫出處之跡,良足稱乎。矩學涉經史,頗有幹局。至於恪勤匪懈,夙夜在公,求之古人,殆未之有。與聞政事,多歷歲年,雖處危亂之中,未虧廉謹之節。然與時消息,承望風旨,使高昌入朝,伊吾獻地;聚糧且末,師出玉門,關右騷然,頗亦矩之由矣。果及長寬,早知去就。而寬淪迹異域,蓋乃命乎。嵩和廉約居身,忠勤奉上,人懷其惠,吏畏其威,雖古之良吏,何以加此。肅歷官周、隋,志存鯁正。竟而
 忠誠慷慨,犯忤龍鱗,固知嫠婦憂宗周之亡,處女悲太子之少,非徒語也。文舉之在絳州,世載清德,辭多受少,有廉讓之風焉。仁基以武略見知,自升顯級,竟而蹈履非所,身名隳壞,時也。



\end{pinyinscope}