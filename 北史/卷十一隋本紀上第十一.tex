\article{卷十一隋本紀上第十一}

\begin{pinyinscope}

 隋高
 祖文皇帝姓楊氏,諱堅,小名那羅延。本弘農華陰
 人,漢太尉震之十四世孫也。震八世孫,燕北平太守鉉。鉉子元壽,魏初為武川鎮司馬,因家於神武樹頹焉。元壽生太原太守惠嘏,嘏生平原太守烈,烈生寧遠將軍禎,禎生皇考忠。初,禎屬魏末喪亂,避地中山,結義徒以討鮮於修禮,遂死之。周保定中,皇考著勛,追贈柱國大將軍、少保、興城郡公。



 皇考美須髯,身長七尺八寸,狀貌瑰偉,武藝絕倫;識量深重,有將率之略。



 年十八,客游泰山,會梁兵陷郡國,沒江南。及北海王元顥入洛,乃與俱歸。顥敗,爾硃度律召為帳下統軍。後從獨孤信,屢有軍功。又與信從魏孝武西遷。東魏荊州刺史辛纂據穰城,皇考
 從信討之,與都督康洛兒、元長生乘城而入。彎弓大呼,斬纂以徇,城中懾服。居半歲,以東魏之逼,與信俱歸。周文帝召居帳下。嘗從周文狩於龍門,皇考獨當一猛獸,左挾其腰,右拔其舌,周文壯之。北臺謂猛獸為掩贍,因以字之。從禽竇泰,破沙苑陣,封襄武縣公。河橋之役,皇考與壯士五人力戰守橋,敵人不敢進。又與李遠破黑水稽胡,並與怡峰解玉壁圍,以功歷云、洛二州刺史。芒山之戰,先登陷陣,除大都督。及侯景度江,梁氏喪敗。周文將經略,乃授
 皇考都督荊等十五州諸軍事,鎮穰城。梁雍州刺史、岳陽王蕭察,雖曰稱籓,而尚懷貳心。皇考自樊城觀兵漢濱,易旗遞進。實二千騎,察登樓望之,以為三萬,懼而服焉。又攻梁隨郡,克之,獲其守桓和。所過城戍,望風請服。進圍安陸。梁同州刺史柳仲禮恐安陸不守,馳歸赴援。諸將恐仲禮至則安陸難下,請急攻之。皇考曰:「仲禮已在近路,吾以奇兵襲之,一舉必克,則安陸不攻自拔,諸城可傳檄而定。」於是選騎二千,銜枚夜進。遇仲禮於漴頭,禽之,悉俘其眾。安陸、竟陵並降。梁元帝大懼,送子方略為質,並送載書,請魏以石城為限,梁以安陸為界。皇考乃旋師。進爵陳留郡公,位大將軍。十七年,梁元帝逼其兄邵陵王綸。綸送質於齊,欲來寇。梁元帝密報周文。遣
 皇考討之。禽綸,數其罪,殺之。初,皇考禽柳仲禮,遇之甚厚。仲禮至京,反譖皇考,言在軍大取金寶。周
 文以皇考功重,不問。



 然皇考悔不殺仲禮,故至此殺綸。皇考間歲再舉,盡定漢東地,甚得新附心。魏恭帝賜姓普六茹氏,行同州事。及於謹伐江陵,皇考為前軍,屯江津,遏其走路。梁人束刃於象鼻以戰,皇考射之,二象反走。江陵平,周文立蕭察為梁主,令皇考鎮穰城。周孝閔踐阼,入為小宗伯。及司馬消難請降,皇考與柱國達奚
 武援之。入齊境五百里,前後遣三使報消難,皆不反命。及去北豫州三十里,武疑有變,欲還。



 皇考曰:「有進死,無退生。」獨以千騎,夜趣城下。候門開而入,乃馳遣召武。



 時齊鎮城伏敬遠勒甲士三千據東陴,舉烽嚴警。武憚之,不欲保城,乃多取財寶,以消難先歸。皇考以三千騎殿,到洛南,皆解鞍而臥。齊眾來追,至於洛北。皇考謂將士曰:「但飽食,今在死地,賊必不敢
 度水。」食畢,齊兵陽若度水,皇考馳將擊之;齊兵不敢逼,遂徐引而還。武嘆曰:「達奚武自言是天下健兒,今
 日服矣。」



 進位柱國大將軍。武成元年,進封隋國公,邑萬戶,別食竟陵縣一千戶,收其租賦。



 保定二年,為大司空。時朝議與突厥伐齊。公卿咸以齊兵強國富,斛律明月不易可當,兵非十萬眾不可。皇考獨曰:「萬騎足矣,明月豎子,亦何能為!」三年,乃以皇考為元帥,大將軍楊纂、李穆、王傑、爾硃敏及開府元壽、田弘、慕容近等皆隸焉。又令達奚武帥步騎三萬自南道進,期會晉陽。皇考乃留敏據什賁,游兵河上。



 皇考出武川,過故宅,祭先人,饗將士,度卷二十餘城。齊人守陘嶺之隘,皇考縱奇兵大破之,留楊纂屯靈丘為後拒。突厥木桿可汗控地頭可汗、步離可汗等,以十萬騎來會。四年正月朔,攻晉陽。時大雪風寒,齊人乃悉其精銳,鼓噪而出。突厥引上西山,不肯戰,眾失色。皇考乃率七百人步戰,死者十四五。以武後期,乃班師。齊人亦不敢逼。突厥乃縱兵大掠。自晉陽至平城,七百餘里,人畜無遺。周武帝拜皇考為太傅,晉公護以其不附己,以
 為涇州總管。是歲,大軍又東伐,晉公護出洛陽,令皇考出沃野,以應接突厥。時軍糧少,諸將憂之。皇考曰:「當獲以濟事耳。」乃招誘稽胡首領,咸令在坐,使王傑盛軍容鳴鼓而出。皇考陽怪問之。傑曰:「大塚宰已至洛陽,天子聞銀、夏間胡擾動,故使傑就攻除之。」又令突厥使者馳告曰:「可汗更入並州,留兵馬十萬在長城下,故令問公,若有稽胡不服,欲來共破之。」坐者皆懼。皇考慰喻遣之,於是歸命,饋輸填積。屬晉公護先退,皇考亦罷兵而還
 鎮。又以政績稱,詔賜錢三十萬,布五百匹,穀二千斛。以疾還京,周武及晉公護屢臨視焉。薨,贈太保、都督同朔等十三州軍事、同州刺史,本官如故。謚曰桓公。開皇元年,追尊為武元皇帝,廟號太祖。



 帝,武元皇帝之長子也。皇妣曰呂氏。以周大統七年六月癸丑夜,生帝於馮翊波若寺。有紫氣充庭。時有尼來自河東,謂皇妣曰:「此兒所從來甚異,不可於俗間處之。」乃將帝舍於別館,躬自撫養。皇妣抱帝,忽見頭上出角,遍體起鱗,墜帝於
 地。尼自外見,曰:「已驚我兒,致令晚得天下。」帝龍頷,額上有五柱入頂,目光外射;有文在手曰「王」字,長上短下,沈深嚴重。初入太學,雖至親暱,不敢狎也。年十四,京兆尹薛善闢為功曹。十五,以皇考勛,授散騎常侍、車騎大將軍、儀同三司,封成紀縣公。十六,遷驃騎大將軍,加開府。周文帝見而
 嘆曰:「此兒風骨,非世間人。」明帝即位,授右小宗伯,進封大興郡公。明帝嘗遣善相者來和視帝。和詭對曰:「不過柱國。」既而私謂帝曰:「公當為天下君,必大誅殺而後定。」



 周武帝即位,遷左小宮伯,出為隨州刺史,進位大將軍。後徵還,遇皇妣寢疾三年,晝夜不離左右,以純孝稱。宇文護執政,尤忌帝,屢將害焉。賴大將軍侯伏侯壽等救護以免。後襲爵隋國公。周武既為皇太子娉帝長女為妃,益加禮重。齊王憲言於周武曰:「普六茹堅相貌,臣每見之,不覺
 自
 失。恐非人下,請早除之。」



 周武曰:「此止可為將耳。」內史王軌驟諫曰:「皇太子非社稷主,普六茹堅有反相。」周武不悅曰:「必天命,將若之何?」帝甚懼,深自晦匿。後從周武平齊,進柱國。又與齊王憲破齊任城王湝於冀
 州,除定州總管。先是州城門久閉不行,齊人白:「文宣時,或請開之,文宣不許,曰:『當有聖人啟之。』」及帝至而開之,莫不驚異。遷亳州總管。



 周宣帝即位,以后父,徵拜上柱國、大司馬。大象初,遷太後丞、右司武,俄轉大前疑。周宣每巡幸,恒委以居守。時周宣為《刑經聖制》,其法深刻,帝以法令滋章,非興化之道,切諫,不納。帝位望益隆,周宣頗以為忌。時周宣四幸女並為皇后,爭寵相毀。周宣每謂后曰:「必族滅爾家。」因召帝,命左右曰:「若色動,即殺之。」帝容色自若,遂免。



 大象二年五月,以帝為揚州總管,將發,暴足疾而止。乙未,周宣不悆。時靜帝幼沖,前內史上
 大夫鄭譯、御正大夫劉昉以帝皇后之父,眾望所集,遂矯詔引帝入侍疾。因受遺輔政,都督內外諸軍事。帝恐周氏諸王在籓生變,稱趙王招將嫁女於突厥為詞以征之。己酉,周宣崩。庚戌,靜帝詔假黃鉞、左大丞相,百官總己而聽焉。以正陽宮為丞相府,以鄭譯為長史,劉悆昉為司馬,具置僚佐。周宣時刑政峻酷者,悉更以寬大之制,天下歸心矣。六月,趙王招、陳王純、趙王盛、代王達、滕王悄並至長安。相州總管尉遲迥自以宿將,至是不能平,遂舉兵。趙、魏之士響應,旬日間,眾至十餘萬。宇文胄以滎州,石悆以建州,席毗以沛郡,毗弟叉羅以兗州,皆
 應。迥遣子質於陳,以求援。帝命上柱國、鄖公韋孝寬討之。雍州牧、畢王賢及趙、陳等五王謀作亂,帝執賢斬之,而掩趙王等罪。因詔五王劍履上殿。入朝不趨,以安之。時五王陰謀滋甚,帝以酒肴造趙王,觀其指。趙王伏甲於臥內,帝賴元胄以免,於是誅趙、越二王。八月庚午,韋孝寬破尉遲迥,斬之,傳首闕下,餘黨悉平。初,迥之亂,鄖州總管司馬消難據州應迥,淮南州縣多從之。襄州總管王誼討之,消難奔陳。荊、郢群蠻乘釁而起,命亳州總管賀若誼討平之。先是,上柱國王謙為益州總管,亦擁眾邑、蜀,以匡復為辭。帝以東夏、山南為事,未遑致討,謙
 遂屯劍口,陷始州。至是,乃命上柱國梁睿討平之,傳首闕下。隳劍閣之險,以絕好亂之萌焉。九月壬子,周帝進帝大丞相。十月,周帝詔追贈皇曾祖烈為柱國、太保、都督十州諸軍事、徐州刺史、隋國公,謚曰康。皇祖禎為柱國、都督十三州諸軍事、同州刺史、隋國公,謚曰獻。皇考忠為上柱國、太師、大冢宰、都督十三州諸軍事、雍州牧。壬戌,誅陳王純。周帝進帝大塚宰,五府總於天官。十一月辛未,誅代王達、滕王悄。十二月甲子,周帝授帝相國,總百揆,去都督內外諸軍事、大冢宰之號,進爵為王。以隋州之崇業,鄖州之安陸、城陽,溫州之宜人,應州之平
 靖、上明,順州之淮南,士州之永川,昌州之廣昌、安昌,申州之義陽、淮安,息州之新蔡、建安,豫州之汝南、臨潁、廣寧、初安,蔡州之蔡陽,郢州之漢東二十郡為隋國。劍履上殿,入朝不趨,贊拜不名,備九錫之禮。加璽紱、遠游冠,相國印綠綟綬,位在諸侯王上。隋國置丞相以下,一依舊式。帝再讓,乃受王爵,十郡而已。周帝詔進皇祖、皇考爵並為王,夫人為王妃。



 大定元年二月壬子,下令曰:「以前賜姓,皆復其舊。」甲寅,帝受九錫之禮。



 丙辰,周帝又詔帝冕十有二旒,建天子旌旗;出警入蹕,乘金根車,駕六馬,備五時副車,置旄頭
 雲罕;樂舞八佾,設鐘諄宮縣;王妃為王后,世子為太子。前後三讓,乃受。俄而下詔,依唐虞、漢魏故事。帝三讓,不許。乃遣太傅、上柱國、巳國公椿奉冊曰:咨爾相國隋王,粵若上古之初。爰啟清濁,降符授聖,為天下君;事上帝而理兆庶,和百靈而利萬物;非以區宇之富,未以宸極為尊。大庭、軒轅以前,驪連、赫胥之日,咸以無為無欲,不將不迎。遐哉,其詳不可聞已。厥有載籍,遺文可觀,聖莫逾於堯,美未過於舜。堯得太尉,已作運衡之篇;舜遇司空,便敘菁華之竭。



 褰裳脫屣,二宮設饗,百官歸禹,若帝之初。斯蓋上則天時,不敢不授;下祗天命,不敢不受。湯
 代於夏,武革於殷,干戈揖讓,雖復異揆,應天順人,其道靡異。自漢迄晉,有魏至周,天曆逐獄訟之歸,神鼎隨謳歌之去。道高者稱帝,祿盡者不王;與夫文祖神宗,無以別也。周德將盡,禍難頻興。宗戚姦回,咸將竊發,顧瞻宮闕,將圖宗社。籓維連率,逆亂相尋,搖蕩三方,不合如礪。蛇行鳥攫,投足無所。王受天明命,睿德在躬。救頹運之艱,匡墜地之業;援大川之溺,救燎原之火;除群凶於城社,廓妖氣於遠服。至德合於造化,神用洽於天壤;八極九野,萬方四裔,圓首方足,莫不樂推。往歲長星夜掃,經天晝見。八風比夏后之作,五緯同漢帝之聚,除舊之徵,
 昭然在上。近者赤雀降祉,玄龜效靈,鐘石變音,蛟魚出穴,有新之貺,煥焉在下。九區歸往,百靈協贊,人神屬望,我不獨知。仰祗皇靈,俯順人願,敬以帝位,禪於爾躬。天祚告窮,天祿永終。於戲!王其允執厥和,儀刑典訓;升圓丘而敬蒼昊,御皇極而撫黔黎,副率土之心,恢無疆之祚,可不盛歟!



 遣大宗伯、大將軍、金城公趙蟹奉皇帝璽紱,百官勸進,帝乃受焉。開皇元年春二月甲子,自相府常服入宮,備禮即皇帝位於臨光殿。設壇於南郊,遣兼太傅、上柱國、鄧公竇熾柴燎告天。是日,告廟。大赦,改元。京師慶雲見。改周官,依
 漢、魏之舊。制:以相國司馬高譫為尚書左僕射兼納言,相國司錄虞慶則為內史監兼史部尚書,相國內郎李德林為內史令,上開府韋世康為禮部尚書,上開府元暉為都官尚書,開府、戶部尚書元巖為兵部尚書,上儀同、司宗長孫毗為工部尚書,上儀同、司會楊尚希為度支尚書,雍州牧楊惠為左衛大將軍。乙丑,追尊皇考為武元皇帝,廟號太祖;皇妣呂氏為元明皇后。改周氏左社右廟制為右社左廟。遣八使巡省風俗。丙寅,修廟社。立王后獨孤氏為皇后,王太子勇為皇太子。丁卯,以大將軍趙蟹為尚書右僕射,以上開府伊婁彥恭為右武
 候大將軍。己巳,以五千戶封周帝介國公為隋室賓;旌旗車服禮樂,一如其舊;上書不為表,答表不稱詔。周氏諸王,盡降為公。辛未,以皇弟同安郡公爽為雍州牧。乙亥,封皇弟邵國公慧為滕王,同安公爽為衛王,皇子雁門公廣為晉王,俊為秦王,秀為越王,諒為漢王。並州總管李穆為太師,上柱國竇熾為太傅,幽州總管于翼為太尉,觀國公田仁恭為太子太師,武德郡公柳敏為太子太保。丁丑,以晉王廣為並州總管,封陳留郡公智積為蔡王,興城郡公靜為道王。戊寅,改東京府為尚書省,發官牛五千頭,分賜貧人。三月,宣仁門槐樹連理,眾枝內
 附。壬午,白狼國獻方物。丁亥,詔犬馬器玩口味,不得獻上。戊子,弛山澤禁。己丑,移盩啡連理樹植于宮庭。戊戌,以太子少保蘇威兼納言、吏部尚書。庚子,詔前代品爵,悉依舊定。丁未,梁蕭巋使其太宰蕭巖來賀。



 夏四月辛巳,大赦。戊戌,太常散樂並免為編戶。禁雜樂百戲。辛丑,陳人來聘于周,至而上已受禪,致之介國。是月,發稽胡修恐長城,二旬而罷。五月戊午,封邗國公楊雄為廣平王,永康郡公楊弘為河間王。辛未,介公薨,上舉哀於朝堂,謚曰周靜帝。六月癸未,詔以初受命,赤雀降祥,推五德相生,為火色。其郊及社、廟,依服冕之儀;而朝會之服、
 旗幟、犧牲盡尚赤,戎服尚黃。秋七月乙卯,上始服黃,百僚畢賀。八月壬午,廢東京官。甲午,遣樂安公元諧擊吐谷渾於青海,破而降之。九月戊申,遣使振給戰亡者家。庚午,陳將周羅攻陷胡墅,蕭摩訶寇江北。辛未,以越王秀為益州總管,改封蜀王。壬申,以薛公長孫覽、宋安公元景山並為行軍元帥。伐陳,仍令尚書左僕射高譫節度諸軍。是月,行五銖錢。冬十月乙酉,百濟王扶餘昌遣使來賀。授昌上開府儀同三司、帶方郡公。戊子,行新律。壬辰,行幸岐州。十一月乙卯,以永富郡公竇榮定為右武候大將軍。遣兼散騎侍郎鄭捴使於陳。己巳,有流
 星如墜牆,光照于地。十二月甲申,以禮部尚書韋世康為吏部尚書。庚子,至自岐州。壬寅,高麗王高陽遣使朝貢,授陽大將軍、遼東郡公。



 太子太保柳敏卒。是歲,靺鞨、突厥阿波可汗、沙缽略可汗並遣使朝貢。



 二年春正月庚申,陳宣帝殂。辛酉,置河北道行臺尚書省於並州,以晉王廣為尚書令。置河南道行臺尚書省於洛州,以秦王俊為尚書令。置西南道行臺尚書省於益州,以蜀王秀為尚書令。戊辰,陳人遣使請和,求歸胡墅。甲戌,詔舉賢良。二月己丑,詔以陳有喪,命高熲等班師。庚寅,加晉王廣左武衛大將軍,秦王俊右武衛大將
 軍。庚子,京師雨土。三月,初命入宮殿門通籍。戊申,開渠引柱陽水於三畤原。夏四月丁丑,以寧州刺史竇榮定為左武候大將軍。庚寅,大將軍韓僧壽破突厥於雞頭山,上柱國李充破突厥於河北山。五月戊申,以上開府長孫平為度支尚書。



 己酉,以旱故,上親省囚徒,其日大雨。己未,高寶寧寇平州,突厥入長城。庚申,以豫州刺史皇甫績為都官尚書。甲子,改傳國璽曰受命璽。丁卯,制人年六十以上免課。六月壬午,以太府卿蘇孝慈為兵部尚書。甲申,使使弔於陳。乙酉,上柱國李充破突厥于馬邑。丙申詔曰:朕祗奉上玄,群臨萬國。屬生靈之弊,處
 前代之宮,以為作之者勞,居之者逸。



 改創之事,心未遑也。而王公大臣,陳謀獻策,咸云:羲、農以降,至於姬、劉,有當世而屢遷,無革命而不徙。曹、馬之後,時見因循,乃末世之宴安,非往聖之宏義。此城從漢,彫殘日久,屢為戰場,舊經喪亂。今之宮室,近代權宜,又非謀筮從龜,瞻星揆日;不足建皇王之邑,合大眾所聚。論變通之數,具幽顯之情。同心固請,詞情深切。然則京師百官之府,四海歸向,非朕一人之所獨有。茍利於物,其可違乎。且殷之五遷,恐人盡怨。是則以吉凶之土,制長短之命,謀新去故,如農望秋:雖則劬勞,其究安宅。今區宇寧一,陰陽順
 序,安安以遷,勿懷胥怨。龍首山川原秀麗,卉物滋阜,卜食相土,宜建都邑。定鼎之基永固,無窮之業在斯。



 公私府宅,規模遠近,營構資須,隨事修茸。



 仍詔左僕射高熲、將作大匠劉龍、鉅鹿郡公賀婁子幹、太府少卿高龍叉等創造新都。秋七月癸巳,詔新置都處墳墓,令悉遷葬設祭,仍給人功;無主者,命官為殯葬。甲午,行新令。冬十月,以撤毀故,徙居東宮。給內外官人祿。癸酉,皇太子勇屯兵咸陽,以備胡虜。庚寅,上疾愈,享百僚於觀德殿,賜錢帛,皆任自取,盡力以出。辛卯,以營新都副監賀婁子乾為工部尚書。十一月丙午,初命為方陣戰法,及制軍
 營圖樣,下諸軍府,以擬征突厥。十二月辛未,上講武于後園。甲戌,上柱國竇毅卒。丙子,名新都曰大興城。乙酉,遣彭城公虞慶則屯弘化以備胡。突厥寇周槃,行軍總管達奚長儒為虜所敗。丙戌,賜國子生經明者束帛。丁亥,親錄囚徒。是歲,高麗、百濟並遣使朝貢。



 三年春正月庚子,將遷新都,大赦。禁大刀長。始令人以二十一成丁,歲役功不過二十日,不役者收庸。廢遠近酒坊,罷鹽井禁。二月己巳朔,日有蝕之。癸酉,陳人來聘。突厥犯邊。癸未,以左武衛大將軍李禮成為右武衛大將軍。三月丁未,上柱國、鮮虞縣公謝慶恩卒。丙辰,以
 雨故,常服入新都。京師承明里醴泉出。



 丁巳,詔購遺書於天下。癸亥,城榆關。夏四月己巳,衛王爽大破突厥於白道山,停築原陽、雲內、紫河等鎮而還。上柱國、建平郡公于義卒。庚午,吐谷渾寇臨洮,洮州刺史皮子信死之。壬申,以尚書右僕射趙煚兼內史令。丁丑,以滕王瓚為雍州牧。庚辰,行軍總管陰壽大破高寶寧于黃龍。甲申,以旱故,上親祀雨師。丙戌,詔天下勸學行禮。己丑,陳郢州城主張子譏遣使請降,上以和好不納。辛卯,遣兼散騎常侍薛舒聘於陳。癸巳,上親雩。五月癸卯,太尉、任城公于翼薨。行軍總管李晃破突厥於摩那渡口。乙巳,梁
 太子蕭琮來賀遷都。辛酉,親祀方澤。壬戌,行軍元帥竇榮定破突厥及吐谷渾於涼州。赦黃龍死罪以下。六月庚午,封衛王爽子集為遂安郡王。戊寅,突厥遣使求和。庚辰,行軍總管梁遠破吐谷渾於爾汗山,斬其名王。秋七月壬戌,詔曰:「往者山東河表,經此妖亂,孤城遠守,多不自全。濟陰太守杜猷身陷賊徒,命懸冠手;郡省事范臺玫傾產營護,免其戮辱。眷言誠節,實有可嘉。宜超恒賞,用明沮勸。臺玫可大都督,假湘州刺史。」丁卯,日有蝕之。



 八月壬午,遣尚書右僕射高熲出寧州道,吏部尚書虞慶則出原州道,並為行軍元帥以擊胡。戊子,親祀太
 社。九月壬子,幸城東觀穀稼。癸丑,大赦。冬十月甲戌,廢河南道行臺省。十一月,發使巡省風俗。庚辰,陳人來聘。陳主知帝貌異世人,使副使袁彥圖像而去。甲午,罷天下諸郡。十二月乙卯,遣兼散騎常侍唐令則使於陳。戊午,以刑部尚書蘇威為戶部尚書。是歲,高麗、突厥、靺鞨並遣使朝貢。



 四年正月甲子朔,日有蝕之。祀太廟。辛未,祀南郊。壬申,梁主蕭巋來朝。



 甲戌,大射於北苑,十日而罷。壬午,齊州水。辛卯,渝州獲獸,似麋,一角同蹄。



 壬辰,班新歷。二月乙已,上餞梁主于霸上。庚戌,行幸隴州。突厥可汗阿史
 那玷厥率其屬來降。夏四月己亥,敕總管、刺史,父母及子年十五以上,不得將之官。



 庚子,以吏部尚書虞慶則為尚書右僕射,瀛州刺史楊尚希為兵部尚書,毛州刺史劉仁恩為刑部尚書。五月癸酉,契丹主莫賀弗遣使請降,拜大將軍。六月庚子,降囚徒。壬子,開通濟渠,自渭達河,以通運漕。甲寅,制官人非戰功不授上柱國以下戎官。以雍、同、華、岐、宜五州旱,命無出今年租調。戊午,秦王俊來朝。秋七月丙寅,陳人來聘。八月甲午,遣十使巡省天下。戊戌,衛王爽來朝。壬寅,上柱國、太傅、鄧公竇熾薨。乙卯,陳將夏侯苗請降,上以通和不納。九月己巳,上
 親錄囚徒。庚午,契丹內附。甲戌,以關中饑,行幸洛陽。冬十一月壬戌,遣兼散騎常侍薛道衡使於陳。甲戌,改周十二月為臘蠟。是歲靺鞨及女國並遣使朝貢。



 五年春正月戊辰,詔行新禮。壬申,詔罷江陵總管。其後,梁主請依舊,許之。



 三月戊午,以尚書左僕射高熲為左領軍大將軍,以上柱國宇文忻為右領軍大將軍。



 夏四月甲午,契丹遣使朝貢。壬寅,上柱國王誼謀反,誅。乙巳,詔征山東大儒馬榮伯等。戊申,車駕至自洛陽。五月甲申,初置義倉。梁主蕭巋殂。遣上大將軍元契使于突厥阿波可汗。秋七月庚申,陳人來聘。壬午,突厥沙缽略可
 汗上表稱臣。



 八月甲辰,河南諸州水,遣戶部尚書蘇威振給之。戊申,有流星數百,四散而下。



 九月乙丑,改鮑陂曰杜陂,霸水曰滋水。丙子,遣兼散騎常侍李若使於陳。冬十一月丁卯,晉王廣來朝。十二月丁未,降囚徒。



 六年春正月甲子,黨項羌內附。庚午,班歷於突厥。壬申,使戶部尚書蘇威巡省山東。二月乙酉,山南荊浙七州水,遣前工部尚書長孫毗振恤之。丙戌,制刺史上佐,每歲暮,更入朝上考課。丁亥,發丁男十一萬修築長城,二旬而罷。庚子,大赦。三月己未,洛陽男子高德上書,請帝為太上皇,傳位皇太子。帝曰:「朕承天命,撫育蒼生,日旰
 孜孜,猶恐不逮。豈學近代帝王,事不師古,傳位於子,自求逸樂哉。」癸亥,突厥沙缽略可汗遣使朝貢。夏四月己亥,陳人來聘。秋七月辛亥,河南諸州水。乙丑,京師雨毛如馬尾,長者二尺餘,短者有六七寸。八月辛卯關內七州旱,蠲其賦稅。遣散騎常侍裴世豪使于陳。戊申,上柱國、太師、申公李穆薨。閏月丁卯,皇太子鎮洛陽。辛未,晉王廣、秦王俊並來朝。丙子,上柱國郕公梁士彥、上柱國巳公宇文忻、柱國舒公劉昉謀反,伏誅。上柱國、許公宇文善有罪,除名。九月辛巳,帝素服御射殿,詔百寮射梁士彥三家資物。丙戌,上柱國、宋安公元景山卒。辛丑,詔
 振恤大象以來死事之家。冬十月己酉,以河北道行臺尚書令、並州總管、晉王廣為雍州牧,餘官如故。以兵部尚書楊尚希為禮部尚書。癸丑,置山南道行臺尚書省於襄州,以秦王俊為尚書令。



 七年春正月癸巳,祀太廟。乙未,制諸州歲貢三人。二月丁巳,祀朝日於東郊。



 己巳,陳人來聘。壬申,幸醴泉宮。是月,發丁男十萬修築長城,二旬而罷。夏四月庚戌,於揚州開山陽瀆,以通運漕。突厥沙缽略可汗卒。癸亥,頒青龍符於東方總管、刺史;西方以白武;南方以朱雀;北方以玄武。甲戌,遣兼散騎常侍楊周使于陳。以戶部尚書
 蘇威為吏部尚書。五月乙亥朔,日有蝕之。己卯,隕石於武安、滏陽間,十餘里。秋七月己丑,衛王爽薨。八月庚申,梁主蕭琮來朝。九月乙酉,梁安平王蕭巖掠於其國以奔陳。辛卯,廢梁國,曲赦江陵。以梁主蕭琮為柱國,封莒國公。冬十月庚申,行幸同州。以先帝所居故,曲降囚徒。癸亥,幸蒲州。丙寅,宴父老,上極歡,曰:「此間人物,衣服鮮麗,容止閑雅。良由仕宦之鄉,陶染成俗也。」十一月甲午,幸馮翊,祭故社。父老對詔失旨,上大怒,免其縣官而去。



 戊戌,車駕至自馮翊。



 八年春正月乙亥,陳人來聘。二月辛酉,陳人寇硤州。三
 月辛未,上柱國、隴西公李詢卒。甲戌,遣兼散騎常侍程尚賢使于陳。戊寅,詔大舉伐陳。秋八月丁未,河北諸州饑,遣吏部尚書蘇威振恤之。九月癸巳,嘉州言龍見。冬十月己未,置淮南行臺省於壽春,以晉王廣為尚書令。辛酉,陳人來聘,拘留不遣。甲子,有星孛於牽牛。享太廟,授律,令晉王廣、秦王俊、清河公楊素並為行軍元帥以伐陳。於是晉王出六合,秦王出襄陽,清河公楊素出信州,荊州刺史劉仁恩出江陵,宜陽公王世積出蘄春,新義公韓擒出廬江,襄邑公賀若弼出吳州,落叢公燕榮出東海,合總管九十。兵五十一萬八千,皆受晉王節度。
 東接滄海,西拒巴蜀,旌旃舟楫,橫亙數千里。仍曲赦陳國。十一月丁卯,車駕餞師。詔購陳叔寶,位上柱國、萬戶公。



 乙亥,行幸定城,陳師誓眾。丙子,幸河東。十二月,車駕至自河東。



 九年春正月癸酉,以尚書左僕射虞慶則為右衛大將軍。丙子,賀若弼敗陳師於蔣山,獲其將蕭摩訶;韓擒進師入建鄴,獲陳主叔寶,陳國平。合州四十,郡一百,縣四百,戶五十萬,口二百萬。癸巳,遣使持節巡撫之。二月乙未,廢淮南尚書省。



 丙申,制五百家為鄉,正一人;百家為里,長一人。夏四月己亥,幸驪山,親勞旋師。乙巳,三軍凱
 入,獻俘於太廟。以晉王廣為太尉。庚戌,帝御廣陽門,宴將士,頒賜各有差。辛亥,大赦。以陳都官尚書孔範、散騎常侍王差、王儀、御史中丞沈觀等邪佞於其主,以致亡滅,皆投之邊裔。陳人普給復十年。軍人畢世免徭役。



 擢陳之文武眾才而用之。宮奴數千,可歸者歸之,其餘盡以分賜將士及王公貴臣。



 其資物,皆於五垛賜王公以下大射。毀所得秦漢三大鐘,越二大鼓。又設亡陳女樂,謂公卿等曰:「此聲似啼,朕聞之甚不喜,故與公等一聽亡國之音,俱為永鑒焉。」



 辛酉,以吏部侍郎宇文弼為刑部尚書,宗正卿楊異為工部尚書。壬戌,詔曰:「今率土大
 同,含生遂性。兵可立威,不可不戢;刑可助化,不可專行。禁衛九重之餘,鎮守四方之外;戎旅軍器,皆宜停罷。武力之子,俱可學文。人間甲仗,悉皆除毀。」



 閏月丁丑,頒木魚符於總管、刺史,雌一雄三。己卯,以吏部尚書蘇威為尚書右僕射。六月乙丑,以荊州總管楊素為納言。丁卯,以吏部侍郎盧愷為禮部尚書。時群臣咸請封禪,詔不許,曰:「豈可命一將軍除一小國,以薄德而封名山,用虛言而干上帝邪。」八月壬戌,以廣平王雄為司空。冬十一月壬辰,考使定州刺史豆盧通等上表請封禪,上不許。庚子,以右衛大將軍虞慶則為右武候大將軍,右領軍
 將軍李安為右領軍大將軍。甲寅,降囚徒。十二月甲子,詔太常卿牛弘、通直散騎常侍許善心、秘書丞姚察、通直郎虞世基等議定樂。



 十年春正月乙未,以皇孫昭為河南王,楷為華陽王。二月庚申,行幸並州。夏五月乙未,詔曰:「魏未喪亂,宇縣瓜分,役軍歲動,未遑休息。兵士軍人,權置坊府,南征北伐,居處無定;家無完堵,地罕苞桑;恒為流寓之人,竟無鄉里之號,朕甚愍之。凡是軍人,可悉屬州縣,墾田籍帳,一同編戶。軍府統領,宜依舊式。」



 罷山東、河南及北方緣邊之地新置軍府。六月辛酉,制人年五十,免役折庸。秋七
 月癸卯,以納言楊素為內史令。庚戌,上親錄囚徒。辛亥,高麗遼東郡公高陽卒。



 八月壬申,遣柱國韋洸、上開府王景並持節巡撫嶺南,百越皆服。九月丁酉,至自並州。冬十月甲子,頒木魚符於京官五品以上。十一月辛卯,幸國學,頒賜各有差。



 辛丑,祀南郊。是月,婺州人汪文進、會稽人高智慧、蘇州人沈玄懀皆舉兵反,自稱天子。樂安蔡道人、饒州吳世華、永嘉沈孝徹、泉州王國慶、餘杭楊寶英、交恥李春等,皆自稱大都督。詔內史令楊素討平之。是歲,吐谷渾、契丹並遣使朝貢。



 十一年春正月丁酉,以平陳所得古器,多為妖變,悉命
 毀之。丙午,皇太子妃元氏薨,上舉哀於東宮文思殿。二月戊午,以大將軍蘇孝慈為工部尚書。丙子,以臨潁令劉曠政績尤異,擢為莒州刺史。辛巳晦,日有蝕之。夏五月乙巳,以右衛將軍元旻為左衛大將軍。秋八月壬申,滕王瓚薨。乙亥,上柱國沛國公鄭譯卒。是歲,高麗、靺鞨並遣使朝貢。突厥獻七寶碗。



 十二年春二月己巳,以蜀王秀為內史令,兼右領軍大將軍;以漢王諒為雍州牧、右衛大將軍。秋七月乙巳,尚書右僕射邳公蘇威、禮部尚書容城侯盧愷並坐事除名。



 壬申晦,日有蝕之。八月甲戌,制天下死罪,諸州不得
 便決,皆令大理覆之。癸巳,制宿衛者不得輙離所守。丁酉,上柱國、楚公豆盧績卒。戊戌,上親錄囚徒。冬十月丁丑,以遂安王集為衛王。壬午,祀太廟。至太祖神主前,帝流涕鳴咽,不自勝。



 十一月辛亥,祀南郊。己未,上柱國、新義公韓擒卒。甲子,百僚大射於武德殿。



 十二月乙酉,以內史令楊素為尚書右僕射。是歲,突厥、吐谷渾、靺鞨並遣使朝貢。



 十三年春正月乙巳,上柱國、郇公韓建業卒。壬子,祀感帝。己未,以信州總管韋世康為吏部尚書。壬戌,行幸岐州。二月丙子,詔營仁壽宮。丁亥,至自岐州。



 己卯,立皇孫
 暕為豫章王。戊子,晉州刺史南陽郡公賈悉達、隰州總管撫寧郡公韓延等以賄伏誅。己丑,制坐事去官者,配防一年。丁酉,制私家不得陷藏緯候圖讖。



 夏五月癸亥,詔禁人間撰集國史,臧否人物。秋七月戊辰晦,日有蝕之。九月丙辰,降囚徒。庚申,封邵公楊綸為滕王。冬十一月乙卯,上柱國、華陽公梁彥光卒。是歲,契丹、燧、室韋、靺鞨並遣使朝貢。



 十四年夏四月乙丑,詔曰:「比命有司,總令研究,正樂雅聲,詳定已訖,宜即施用,見行者停。人間音樂,流僻日久;棄其舊體,競造繁聲;流宕不歸,遂以成俗。宜加禁約,務
 存其本。」五月辛酉,京師地震。關內諸州旱。六月丁卯,詔省、府、州、縣皆給廨田,不得興生,與人爭利。秋七月乙未,以邳公蘇威為納言。



 八月辛未,關中大旱,人饑。行幸洛陽,並命百姓山東就食。冬閏十月甲寅,詔曰:「梁、齊、陳往皆創業一方,綿歷年代。既宗祀廢絕,祭奠無主;興言矜念,良以愴然。莒國公蕭琮及高仁英、陳叔寶等,宜令以時世脩祭祀,所須器物,有司給之。



 乙卯,制外官九品以上,父母及子年十五不得從之官。十一月壬戌,制州縣佐史,三年一代,不得重任。癸未,有星孛于角、亢。十二月乙未,東巡狩。



 十五年春正月壬戌,車駕次齊州,親問疾苦。丙寅,旅王符山。庚午,以歲旱,祀太山以謝愆咎,大赦。二月丙辰,禁私家畜兵器,關中、緣邊不在其例。禁河以東無得乘馬。丁巳,上柱國、蔣公梁睿卒。三月己未,車駕至自東巡。望祭五嶽海瀆。丁亥,幸仁壽宮。夏四月己丑朔,大赦。甲辰,以趙州刺史楊達為工部尚書。



 五月丁亥,制京官五品以上佩銅魚符。六月戊子,詔鑿砥柱。庚寅,相州刺史豆盧通貢綾文布,命焚之于朝堂。辛丑,詔名山未在祀典者,悉命祀之。秋七月甲戌,遣邳公蘇威巡省江南。戊寅,至自仁壽宮。辛巳,制九品以上官,以理去官者,並聽執
 笏。冬十二月戊子,敕盜邊糧一升以上,皆斬,籍沒其家。己丑,詔文武官以四考更代。是歲,吐谷渾、林邑等國並遣使朝貢。



 十六年春二月丁亥,封皇孫裕為平原王,筠為安成王,嶷為安平王,恪為襄城王,該為高陽王,韶為建安王,蟹為潁川王。夏六月甲午,制工商不得進仕。並州大蝗。辛丑,詔九品以上妻、五品以上妾,夫亡不得改嫁。秋八月庚戌,詔決死罪者,三奏而後行刑。冬十月己丑,幸長春宮。十一月壬子,至自長春宮。



 十七年春二月癸未,太平公史萬歲伐西寧,剋之。庚寅,
 行幸仁壽宮。庚子,上柱國王世積討桂州賊李光仕,平之。三月丙辰,詔諸司屬官有犯,聽於律令外斟酌決杖。辛酉,上親錄囚徒。癸亥,上柱國、彭城公劉昶以罪狀誅。庚午,遣御史柳彧、皇甫誕巡省河南北。夏四月戊寅,頒新歷。五月庚申,宴百僚於玉女泉,班賜各有差。己巳,蜀王秀來朝。閏月己卯,群鹿入殿門。馴擾侍衛之內。秋七月丁丑,桂州人李世賢反,遣右武候大將軍虞慶則討平之。丁亥,並州總管、秦王俊坐事免,以王就第。九月甲申,車駕至自仁壽宮。庚寅,上謂侍臣曰:「廟庭設樂,本以迎神。齋祭之日,觸目多感,當此之際,何可為心?在路奏
 樂,禮為未允。公卿宜更詳之。」冬十月丁未,頒銅武符於驃騎、車騎府。戊申,道王靜薨。庚午,詔曰:「五帝異樂,三王殊禮,皆隨事而有損益,因情而立節文。仰惟祭享宗廟,瞻敬如在,罔極之感,情深茲日。而禮畢升路,鼓吹發音,還入宮門,金石振響;斯則哀樂同日,心事相違,情所不安,理實未允。宜改茲往式,用弘禮教。自今享廟日,不須備鼓吹,殿庭勿設樂縣。」辛未,京下大索。十二月壬子,上柱國、右武候大將軍、魯公虞慶則以罪伏誅。是歲,高麗、突厥並遣使朝貢。



 十八年春正月辛丑,詔曰:「吳、越之人,往承弊俗;所在之
 處,私造大船,因相聚結,致有侵害。江南諸州,人間有船長三丈以上,悉括入官。」二月甲辰,幸仁壽宮。乙巳,以漢王諒為行軍元帥,水陸三十萬,伐高麗。夏五月辛亥,詔畜貓鬼蠱毒厭魅野道之家,投于四裔。六月丙寅,詔黜高麗王高元官爵。秋八月丙子,詔京官五品以上、總管、刺史舉志行修謹、清平幹濟之士。九月己丑,漢王諒師遇疾疫而旋,死者十二三。庚寅,敕舍客無公驗者,坐及刺史、縣令。辛卯,車駕至自仁壽宮。冬十一月甲戌,帝親錄囚徒。癸未,祀南郊。十二月庚子,上柱國、夏州總管、東萊公王景以罪伏誅。是歲,自京師至仁壽宮,置行宮十
 所。巳、宋、陳、亳、曹、戴、潁等州水,詔並免庸調。



 十九年春正月癸酉,大赦。戊寅,大射于武德殿。二月己亥,晉王廣來朝。甲寅,幸仁壽宮。夏四月丁酉,突厥利可汗內附。達頭可汗犯塞,行軍總管史萬歲擊破之。六月丁酉,以豫章王暕為內史令。秋八月癸卯,上柱國、尚書左僕射、齊公高熲坐事免。辛亥,上柱國、皖城公張威卒。甲寅,上柱國、城陽公李徹卒。九月乙丑,以太常卿牛弘為吏部尚書。冬十月甲午,以突厥利可汗為啟人可汗,築大利城,處其部落。十一月,有司言元年已來,日漸長。十二月乙未,突厥都藍可汗為部下所殺,國大亂。星隕
 於勃海。二十年春正月辛酉朔,突厥、高麗、契丹並遣使朝貢。二月丁丑,無雲而雷。三月辛卯,熙州人李英林反,遣行軍總管張衡討之。



 夏四月壬戌,突厥犯塞,以晉王廣為行軍元帥,擊破之。乙亥,天有聲如寫水,自南而北。六月丁丑,秦王俊薨。秋九月丁未,車駕至自仁壽宮。冬十月乙丑,廢皇太子勇及其諸子。並為庶人。殺柱國、太平公史萬歲。己巳,殺左衛大將軍、五原公元旻。十一月戊子,以晉王廣為皇太子。天下地震。京城大風雪。十二月戊午,詔東宮官屬於皇太子不得稱臣。辛巳,詔毀壞偷盜佛
 及天尊像、嶽鎮海瀆神形者,以不道論。沙門壞佛像,道士壞天尊像,以惡逆論。



 仁壽元年春正月乙酉朔,大赦,改元。以尚書右僕射楊素為左僕射,以納言蘇威為右僕射。丁酉,徙河南王昭為晉王。突厥寇恒安,遣柱國韓洪擊之,敗焉。以晉王昭為內史令。辛丑,詔曰:「投生殉節,自古稱難,殞身王事,禮加二等。而世俗之徒,不達大義,致命戎旅,不入兆域。興言念此,每深愍歎。且入廟祭祀,並不廢闕,何止墳塋,獨在其外?自今戰亡之徒,宜入墓域。」二月乙卯朔,日有蝕之。夏五月己丑,突厥男女九萬餘口來降。壬辰,驟雨震
 雷,大風拔木,宜君湫水,移於始平。六月乙卯,遣十六使巡省風俗。乙丑,廢太學及州縣學,唯留國子一學,取正三品以上子七十二人充生。頒舍利於諸州秋七月戊戌,改國子為太學。



 十一月己丑,祀南郊。十二月,楊素擊突厥,大破之。



 二年春三月己亥,幸仁壽宮。夏四月庚戌,岐、雍二州地震。秋七月丙戌,詔內外官各舉所知。八月己巳,皇后獨孤氏崩。九月丙戌,車駕至自仁壽宮。壬辰,河南北諸州大水,遣工部尚書楊達振恤之。乙未,上柱國、袁州總管、金水公周搖卒。隴西地震。冬十月壬子,典赦益州管內。
 癸丑,以工部尚書楊達為納言。閏月甲申,詔尚書左僕射楊素與諸術者刊定陰陽舛謬。己丑,詔楊素、右僕射蘇威、吏部尚書牛弘、內史侍郎薛道衡、秘書丞許善心、內史舍人虞世基、著作郎王劭等修定五禮。壬寅,葬獻皇后於太陵。十二月癸巳,益州總管、蜀王秀有罪,廢為庶人。



 交州人李佛子舉兵反,遣行軍總管劉方討平之。



 三年春二月戊子,以大將軍、蔡陽郡公姚辯為左武候大將軍。夏五月癸卯,詔曰:「六月十三日是朕生日,其日令海內為武元皇帝、元明皇后斷屠。」六月甲午,詔曰:《禮》云:親以期斷。蓋以四時之變易,萬物之更始,故聖人象
 之。其有三年,加隆爾也。但家無二尊,母為厭降,是以父在喪母,還服于期者,服之正也。豈容期內而更小祥!然三年之喪而有小祥者,《禮》云:「期祭,禮也;期而除喪,道也。」以是之故,雖未再期,而天地一變,不可不祭,不可不除。故有練焉,以存喪祭之本。然期喪有練,於理未安。雖云十一月而練,乃無所法象,非期非時,豈可除祭?而儒者徒擬三年之喪,立練禫之節,可謂茍存其變,而失其本;欲漸於奪,乃薄於喪。致使子則冠練去絰,黃裏縓緣;絰則布葛在躬,粗服未改。豈非絰哀尚存,子情已奪;親疏失倫,輕重顛倒。乃不順人情,豈聖人之意也?故非先聖
 之禮,廢於人邪!三年之喪,尚有不行之者;至於祥練之節,安能不墜者乎!



 《禮》云:父母之喪,無貴賤一也。而大夫士之喪父母,乃貴賤異服。然則禮壞樂崩,由來漸矣。所以晏平仲之斬粗縗,其老謂之非禮,滕文公之服三2年,其臣咸所不欲。蓋由王道既衰,諸侯異政,將踰越於法度,惡禮制之害己。乃滅去篇籍,自制其宜。遂至骨肉之恩,輕重從俗,無易之道,降殺任情。夫禮不從天降,不從地出,乃人心而已者,謂情緣於恩也。故恩厚者其禮隆,情輕者其禮殺。聖人以是稱情立文,別親疏貴賤之節。自臣子道消,上下失序,莫大之恩,逐情而薄;莫重之化,
 與時而殺。此乃服不稱喪,容不稱服,非所謂聖人緣恩表情制禮之義也。然喪與其易也,寧在於戚,則禮之本也。禮有其餘,未若於哀,則情之實也。今十一月而練者,非禮之本,非情之實。由是言之,父在喪母,不宜有練。但依《禮》十三月而祥,中月而禫,庶以合聖人之意,達人子之心。



 秋七月丁卯,詔州縣搜揚賢哲,皆取明知古今,通識安危,究政教之本,達禮樂之源。不限多少,不得不舉。徵召將送,必須以禮。八月壬申,上柱國、檢校幽州總管、落叢公燕榮以罪伏誅。九月壬戌,置常平官。甲子,以營州總管韋沖為戶部尚書。十二月癸酉,河南諸州水,遣納
 言楊達振恤之。



 四年春正月丙辰,大赦。甲子,幸仁壽宮。夏四月乙卯,上不豫。六月庚午,大赦。有星入月中,數日而退,長人見於鴈門。秋七月乙未,日青無光,八日乃復。



 甲辰,帝疾甚,臥於仁壽宮,與百僚辭訣,上握手歔欷。丁未,崩于大寶殿,時年六十四。詔曰:嗟乎!自昔晉室播遷,天下喪亂,四海不一。以至周、齊,戰爭相尋,年將三百。故割疆土者非一所,稱帝王者非一人。書軌不同,生靈塗炭。上天降監,受命于朕,用登大位,豈關人力?故得撥亂反正,偃武修文;天下大同,聲教遠被,此又是天意欲寧區夏。所以昧旦
 臨朝,不敢逸豫;;一日萬機,留心親覽;晦明寒暑,不憚劬勞;匪曰朕躬,蓋為百姓故也。王公卿士,每日闕庭,刺史以下,歲時朝集。



 何嘗不罄竭心府,誡敕殷勤。義乃君臣,情兼父子,庶藉百寮之智,萬國歡心。欲令率土之人,永得安樂。不謂遘疾彌留,至於大漸。此乃人生常分,何足言及。但四海百姓,衣食不豐;教化政刑,猶未盡洽。興言念此,唯以留恨。朕今踰六十,不復稱夭;但筋力精神,一時勞竭。如此之事,本非為身,止欲安養百姓,所以致此。人生子孫,誰不念愛?既為天下,事須割情。勇及秀等,並懷悖惡;既無臣子之心,所以黜廢。古人有云:「知臣莫若
 君,知子莫若父。」令勇、秀得志,共理家國,亦當戮辱遍於公卿,酷毒流於人庶。今惡子孫已為百姓黜屏,好子孫足堪負荷大業。此雖朕家事,理不容隱,前對文武侍衛,具已論述。皇太子廣,地居上嗣,仁孝著聞。以其行業,堪成朕志。但念內外群官,同心戮力,以此共安天下。朕雖瞑目,何所復恨?國家大事,不可限以常禮;既葬公除,行之自昔,今宜遵用,不勞改定。凶禮所須,纔令周事,務從節儉,不得勞人。諸州總管、刺史以下,宜率其職,不須奔赴。自古哲王,因人作法,前帝後帝,沿革隨時。律令格式有不便於事者,宜依前修改,務當政要。嗚呼!敬之哉,無
 墜朕命。



 乙卯,發喪。河間楊柳四株,無故黃落,既而花葉復生。八月丁卯,梓宮至自仁壽宮。丙子,殯於大興前殿。十月乙卯,葬於太陵,同墳而異穴。士庶赴葬者,皆聽入視陵內。帝性嚴重有威容,外質木而內明敏,有大略。初得政之始,群情不附。諸子幼弱,內有六王之謀,外致三方之亂,握強兵、居重鎮者,皆周之舊臣。



 上推以赤心,各盡其用。不踰期月,剋定三邊;未及十年,平一四海。薄賦斂,輕刑罰;內修制度,外撫戎夷。每旦聽朝,日仄忘倦。居處服玩,務存節儉,令行禁止,上下化之。開皇、仁壽之間,丈夫不衣綾綺而無金玉之飾。常服率多布帛,裝帶不
 過以銅鐵骨角而已。雖嗇於財,至於賞賜有功,亦無所愛惜。每乘輿四出,路逢上表者,駐馬親自臨問。或潛遣行人,采聽風俗;吏政得失,人間疾苦,無不留意。嘗遇關中饑,遣左右視百姓所食。有得豆屑雜糠而奏之者,上流涕以示群臣;深自咎責,為之損膳而不御酒肉者,殆將一期。及東拜太山,關中戶口就食洛陽者,道路相屬。帝敕斥候,不得輒有驅逼,男女參廁於仗衛之間。遇逢扶老攜幼者,輒引馬避之,慰勉而去。至艱險之處,見負擔者,遽令左右扶助之。其有將士戰歿,必加優賞,仍令使者,就家勞問。自強不息,朝夕孜孜。人庶殷繁,帑藏充
 實。雖未能臻於至道,亦足稱近代之良主。然雅性沈猜,素無學術,好為小數。言神燭聖杖,堪能療病。又信王劭解石文以為己瑞焉。不達大體如是。故忠臣義士,莫得盡心竭辭。其草創元勳,及有功諸將,誅夷獲罪,罕有存者。又不悅詩書,楊素由之希旨,遂奏除學校。唯婦言是用,廢黜諸子。逮于暮年,持法尤峻,喜怒失常,果於殺戮。嘗令左右送西域朝貢使出關,其人所經之處,受牧宰小物,饋鸚鵡、麖皮、馬鞭之屬,聞而大怒。又詣武庫,見署中蕪穢不理,於是執武庫令及諸受遺者,出開遠門外,親自臨決,死者數十人。又往往潛令賂遺令史,府史
 受者必死,無所寬貸,議者以此少之。



 論曰:隋文帝樹基立本,積德累仁。徒以外戚之尊,受托孤之任,與能之議,未為所許。是以周室舊臣,咸懷憤惋。既而王謙固三蜀之阻,不踰期月;尉遲迥舉全齊之眾,一戰而亡。斯乃非止人謀,抑亦天之所贊。乘茲機運,遂遷周鼎。於時蠻夷猾夏,荊、揚未一,劬勞日仄,經營四方。樓船南邁,則金陵失險;驃騎北指,則單于款塞。《職方》所載,並入疆理;《禹貢》所圖,咸受正朔。雖晉武之克平吳會,漢宣之推亡固存,比義論功,不能尚也。七德既敷,九歌已洽,尉候無警,遐邇肅清。於是躬節儉,平徭賦,倉廩實,
 法令行。君子咸樂其生,小人各安其業,強不陵弱,眾不暴寡;人物殷阜,朝野歡娛。自開皇二十年間,天下無事;區宇之內,晏如也。考之前王,足以參蹤盛烈。而素無術業,不能盡下。無寬仁之度,有刻薄之資。暨乎暮年,此風愈扇。又雅好瑞符,暗於大道。建彼維城,權侔京室,皆同帝制,靡所適從。聽姑歸之言,惑邪臣之說,溺寵廢嫡,託付失所。滅父子之道,開昆弟之隙;縱其尋斧,翦伐本根。墳土未幹,子孫繼踵為戮;松檟纔列,天下已非隋有。惜哉!迹其衰怠之源,稽其亂亡之兆,起自文皇,成於煬帝;所由來遠矣,非一朝一夕,其不祀忽諸,未為不幸也。



\end{pinyinscope}