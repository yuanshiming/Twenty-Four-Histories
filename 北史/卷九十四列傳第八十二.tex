\article{卷九十四列傳第八十二}

\begin{pinyinscope}

 高麗百濟新羅勿吉奚契丹室韋豆莫婁地豆干烏洛侯流求倭蓋天地之所覆載至大,日月之所照臨至廣。萬物之內,生靈寡而禽獸多;兩儀之間,中土局而殊俗曠。人寓形天地,稟氣陰陽,愚智本於自然,剛柔繫於水土。



 故霜露所會,風氣所通,九川為紀,五岳作鎮,此之謂諸夏,生其
 地者,則仁義所出;昧谷嵎夷,孤竹北戶,限以丹徼紫塞,隔以滄海交河,此之謂荒裔,感其氣者,則凶德行稟。若夫九夷、八狄,種落繁熾,七戎、六蠻,充牣邊鄙,雖風土殊俗,嗜慾不同,至於貪而無厭,狠而好亂,強則旅拒,弱則稽服,其揆一也。



 秦皇鞭笞天下,黷武於遐方;漢武士馬強盛,肆志於遠略。匈奴已卻,其國乃虛;天馬既來,其人亦困。是知鴈海龍堆,天所以絕夷夏也;炎方朔漠,地所以限內外也。況乎時非秦、漢,志甚嬴、劉,逆天道以求其功,殫人力而從所欲,顛墜之釁,固不旋踵。是以先王設教,內諸夏而外夷狄;往哲垂範,美樹德而鄙廣地。



 雖禹
 迹之東漸西被,不過海及流沙;《王制》之自北徂南,裁猶穴居交趾。豈非道貫三古,義高百代者乎!自魏至隋,市朝屢革,其四夷朝享,亦各因時。今各編次,備《四夷傳》云。



 高句麗,其先出夫餘。王嘗得河伯女,因閉於室內,為日所照,引身避之,日影又逐,既而有孕,生一卵,大如五升。夫餘王棄之與犬,犬不食;與豕,豕不食;棄於路,牛馬避之;棄於野,眾鳥以毛茹之。王剖之不能破,遂還其母。母以物裹置暖處,有一男破而出。及長,字之曰朱蒙。其俗言「朱蒙」者,善射也。夫餘人以朱蒙非人所生,請除之。王不聽,命之養馬。朱蒙私試,知有善惡,駿者減食令瘦,駑
 者善養令肥。夫餘王以肥者自乘,以瘦者給朱蒙。後狩於田,以朱蒙善射,給之一矢。硃蒙雖一矢,殪獸甚多。夫餘之臣,又謀殺之,其母以告朱蒙,硃蒙乃與焉違等二人東南走。中道遇一大水,欲濟無梁。夫餘人追之甚急,硃蒙告水曰:「我是日子,河伯外孫,今追兵垂及,如何得濟?」於是魚鱉為之成橋,朱蒙得度。



 魚鱉乃解,追騎不度。硃蒙遂至普述水,遇見三人,一著麻衣,一著衲衣,一著水藻衣,與朱蒙至紇升骨城,遂居焉。號曰高句麗,因以高為氏。其在夫餘妻懷孕,朱蒙逃後,生子始閭諧。及長,知朱蒙為國王,即與母亡歸之。名曰閭達,委之國事。



 朱
 蒙死,子如栗立。如栗死,子莫來立,乃并夫餘。



 漢武帝元封四年,滅朝鮮,置玄菟郡,以高句麗為縣以屬之。漢時賜衣幘朝服鼓吹,常從玄菟郡受之。後稍驕,不復詣郡,但於東界築小城受之,遂名此城為幟溝漊。「溝漊婁」者,句麗「城」名也。王莽初,發高句麗兵以伐胡,而不欲行,莽強迫遣之,皆出塞為寇盜。州郡歸咎於句麗侯騶,嚴尤誘而斬之。莽大悅,更名高句麗,高句麗侯。光武建武八年,高句麗遣使朝貢。



 至殤、安之間,莫來裔孫宮,建寇遼東。玄菟太守蔡風討之,不能禁。



 宮死,子伯固立。順、和之間,復數犯遼東,寇抄。靈帝建寧二年,玄菟太守耿臨
 討之,斬首虜數百級,伯固乃降,屬遼東。公孫度之雄海東也,伯固與之通好。



 伯固死,子伊夷摸立。伊夷摸自伯固時,已數寇遼東,又受亡胡五百餘戶。建安中,公孫康出軍擊之,破其國,焚燒邑落,降胡亦叛。伊夷摸更作新國。其後伊夷摸復擊玄菟,玄菟與遼東合擊,大破之。



 伊夷摸死,子位宮立。始位宮曾祖宮,生而目開能視,國人惡之。及長凶虐,國以殘破。及位宮亦生而視人,高麗呼相似為「位「,以為似其曾祖宮,故名位宮。



 位宮亦有勇力,便鞍馬,善射獵。魏景初二年,遣太傅、司馬宣王率眾討公孫文懿,位宮遣主簿、大加將數千人助軍。正始三年,
 位宮寇遼西安平。五年,幽州刺史毋丘儉將萬人出玄菟,討位宮,大戰於沸流。敗走,儉追至赬峴,懸車束馬登丸都山,屠其所都。位宮單將妻息遠竄。六年,儉復討之,位宮輕將諸加奔沃沮。儉使將軍王頎追之,絕沃沮千餘里,到肅慎南,刻石紀功。又刊丸都山、銘不耐城而還。其後,復通中夏。



 晉永嘉之亂,鮮卑慕容廆據昌黎大棘城,元帝授平州刺史。位宮玄孫乙弗利頻寇遼東,廆不能制。



 弗利死,子釗代立。魏建國四年,慕容廆子晃伐之,入自南陜,戰于木底,大破釗軍。追至丸都。釗單馬奔竄,晃掘釗父墓,掠其母妻、珍寶、男女五萬餘口,焚其室,
 毀丸都城而還。釗後為百濟所殺。



 及晉孝武太元十年,句麗攻遼東、玄菟郡。後燕慕容垂遣其弟農伐句麗,復二郡。垂子寶以句麗王安為平州牧,封遼東、帶方二國王,始置長史、司馬、參軍官。



 後略有遼東郡。



 太武時,釗曾孫璉始遣使者詣安東,奉表貢方物,并請國諱。太武嘉其誠款,詔下帝系名諱於其國。使員外散騎侍郎李敖拜璉為都督遼海諸軍事、征東將軍、領東夷中郎將、遼東郡公、高句麗王。敖至其所,居平壤城,訪其方事,云:去遼東南一千餘里,東至柵城,南至小海,北至舊夫餘,人戶參倍於前魏時。後貢使相尋。



 歲致黃金二百斤、白銀四
 百斤。時馮弘率眾奔之,太武遣散騎常侍封撥詔璉,令送弘。璉上書稱當與弘俱奉王化,竟不遣。太武怒,將往討之。樂平王丕等議等後舉,太武乃止。而弘亦壽為璉所殺。



 後文明太后以獻文六宮未備,敕璉令薦其女。璉奉表云:女已出,求以弟女應旨。朝廷許焉,乃遣安樂王真、尚書李敷等至境送幣。璉惑其左右之說,云朝廷昔與馮氏婚姻,未幾而滅其國。殷鑒不遠,宜以方便辭之。璉遂上書,妄稱女死。朝廷疑其矯拒,又遣假散騎常侍程駿切責之,若女審死,聽更選宗淑。璉云:「若天子恕其前愆,謹當奉詔。」會獻文崩,乃止。至孝文時,璉貢獻倍前,
 其報賜亦稍加焉。時光州於海中得璉遣詣齊使餘奴等,送闕。孝文詔責曰:「道成親殺其君,竊號江左,朕方欲興滅國於舊邦,繼絕世於劉氏。而卿越境外鄉,交通篡賊,豈是籓臣守節之義?今不以一過掩舊款,即送還籓。其感恕思愆,祗承明憲,輯寧所部,動靜以聞。」



 太和十五年,璉死,年百餘歲。孝文舉哀於東郊,遣謁者僕射李安上策贈車騎大將軍、太傅、遼東郡公、高句麗王,謚曰康。又遣大鴻臚拜璉孫雲使持節、都督遼海諸軍事、征東將軍、領護東夷中郎將、遼東郡公、高句麗王。賜衣冠服物車旗之飾。又詔雲遣世子入朝,令及郊丘之禮。雲上
 書辭疾,遣其從叔升于隨使詣闕嚴責之,自此,歲常貢獻。正始中,宣武於東堂引見其使芮悉弗,進曰:「高麗係誠天極,累葉純誠,地產土毛,無愆王貢。但黃金出夫餘,珂則涉羅所產。今夫餘為勿吉所逐,涉羅為百濟所并。國王臣雲惟繼絕之義,悉遷于境內。二品所以不登王府,實兩賊之為。」宣武曰:「高麗世荷上將,專制海外,九夷黠虜,實得征之。



 昔方貢之愆,責在連率。宜宣朕旨於卿主,務盡威懷之略,使二邑還復舊墟,土毛無失常貢也。」



 神龜中,雲死,靈太后為舉哀於東堂。遣使策贈車騎大將軍、領護東夷校尉、遼東郡公、高麗王。又拜其世子安
 為鎮東將軍、領護東夷校尉、遼東郡公、高麗王。



 正光初,光州又於海中執得梁所授安寧東將軍衣冠劍珮,及使人江法盛等,送京師。



 安死,子延立。孝武帝初,詔加延使持節、散騎常侍、車騎大將軍、領護東夷校尉、遼東郡公、高句麗王。天平中,詔加延侍中、驃騎大將軍,餘悉如故。



 延死,子成立。訖於武定已來,其貢使無歲不至。大統十二年,遣使至西魏朝貢。及齊受東魏禪之歲,遣使朝貢於齊。齊文宣加成使持節、侍中、驃騎大將軍,領東夷校尉、遼東郡公、高麗王如故。天保三年,文宣至營州,使博陵崔柳使于高麗,求魏末流人。敕柳曰:「若不從者,以
 便宜從事。」及至,不見許。柳張目叱之,拳擊成墜於床下,成左右雀息不敢動,乃謝服,柳以五千戶反命。



 成死,子湯立。乾明元年,齊廢帝以湯為使持節、領東夷校尉、遼東郡公、高麗王。周建德六年,湯遣使至周,武帝以湯為上開府儀同大將軍、遼東郡公、遼東王。隋文帝受禪,湯遣使詣闕,進授大將軍,改封高麗王。自是,歲遣使朝貢不絕。



 其國,東至新羅,西度遼,二千里;南接百濟,北鄰靺鞨,一千餘里。人皆士著,隨山谷而居,衣布帛及皮。土田薄瘠,蠶農不足以自供,故其人節飲食。其王好修宮室,都平壤城,亦曰長安城,東西六里,隨山屈曲,南臨浿水。城
 內唯積倉儲器備,寇賊至日,方入固守。王別為宅於其側,不常居之。其外復有國內城及漢城,亦別都也。其國中呼為三京。復有遼東、玄菟等數十城,皆置官司以統攝。與新羅每相侵奪,戰爭不息。



 官有大對盧、太大兄、大兄、小兄、竟侯奢、鳥拙、太大使者、大使者、小使者、褥奢、翳屬、仙人,凡十二等,分掌內外事。其大對盧則以強弱相陵奪而自為之,不由王署置。復有內評、五部褥薩。人皆頭著折風,形如弁,士人加插二鳥羽。



 貴者,其冠曰蘇骨,多用紫羅為之,飾以金銀。服大袖衫、大口褲、素皮帶、黃革履。婦人裙襦加襈。書有《五經》、《三史》、《三國志》、《晉陽秋》。兵
 器與中國略同。及春秋校獵,王親臨之。稅,布五疋、穀五石;游人則三年一稅,十人共細布一疋。租,戶一石,次七斗,下五斗。其刑法,叛及謀逆者,縛之柱,爇而斬之,籍沒其家;盜則償十倍,若貧不能償者樂及公私債負,皆聽評其子女為奴婢以償之。用刑既峻,罕有犯者。樂有五弦、琴、箏、篳篥、橫吹、簫、鼓之屬,吹蘆以和曲。每年初,聚戲浿水上,王乘腰輦、列羽儀觀之。事畢,王以衣入水,分為左右二部,以水石相濺擲,喧呼馳逐,再三而止。俗潔凈自喜,尚容止,以趨走為敬。拜則曳一腳,立多反拱,行必插手。性多詭伏,言辭鄙穢,不簡親疏。父子同川而浴,共
 室而寢。好歌舞,常以十月祭天,其公會衣服,皆錦繡金銀以為飾。



 好蹲踞,食用俎機。出三尺馬,云本朱蒙所乘馬種,即果下也。風俗尚淫,不以為愧,俗多游女,夫無常人,夜則男女群聚而戲,無有貴賤之節。有婚嫁,取男女相悅即為之。男家送豬酒而已,無財聘之禮;或有受財者,人共恥之,以為賣婢。死者,殯在屋內,經三年,擇吉日而葬。居父母及夫喪,服皆三年,兄弟三月。初終哭泣,葬則鼓舞作樂以送之。埋訖,取死者生時服玩車馬置墓側,會葬者爭取而去。



 信佛法,敬鬼神,多淫祠。有神廟二所:一曰夫餘神,刻木作婦人像;一曰高登神,云是其始
 祖夫餘神之子。並置官司,遣人守護,蓋河伯女、朱蒙云。



 及隋平陳後,湯大懼,陳兵積穀,為守拒之策。開皇十七年,上賜璽書,責以每遣使人,歲常朝貢,雖稱籓附,誠節未盡。驅逼靺鞨,禁固契丹。昔年潛行貨利,招動群小,私將弩手,巡竄下國,豈非意欲不臧,故為竊盜?坐使空館,嚴加防守;又數遣馬騎,殺害邊人。恒自猜疑,密覘消息,殷勤曉示,許其自新。湯得書惶恐,將表陳謝。會病卒。



 子元嗣。文帝使拜元為上開府儀同三司,襲爵遼東公,賜服一襲。元奉表謝恩,并賀祥瑞,因請封王。文帝優冊為王。明年,率靺鞨萬餘騎寇遼西,營州總管韋世沖擊走
 之。帝大怒,命漢王諒為元帥,總水陸討之,下詔黜其爵位。時餽運不繼,六軍乏食,師出臨渝關,復遇疾疫,王師不振。及次遼水,元亦惶懼,遣使謝罪,上表稱「遼東糞土臣元」云云。上於是罷兵,待之如初。元亦歲遣朝貢。



 煬帝嗣位,天下全盛,高昌王、突厥啟人可汗並親詣闕貢獻,於是徵元入朝。



 元懼,蕃禮頗闕。大業七年,帝將討元罪,車駕度遼水,止營於遼東地,分道出師,各頓兵於其城下。高麗出戰多不利,皆嬰城固守。帝令諸軍攻之,又敕諸將,高麗若降,即宜撫納,不得縱兵入。城將陷,賊輒言降,諸將奉旨,不敢赴機。先馳奏,比報,賊守禦亦備,復出拒戰。
 如此者三,帝不悟。由是食盡師老,轉輸不繼,諸軍多敗績,於是班師。是行也,唯於遼水西拔賊武厲邏,置遼東郡及通定鎮而還。



 九年,帝復親征,敕諸軍以便宜從事。諸將分道攻城,賊勢日蹙。會楊玄感作亂,帝大懼,即日六軍並還。兵部侍郎斛斯政亡入高麗,高麗具知事實,盡銳來追,殿軍多敗。十年,又發天下兵,會盜賊蜂起,所在阻絕,軍多失期。至遼水,高麗亦困弊,遣使乞降,因送斛斯政贖罪。帝許之,頓懷遠鎮受其降,仍以俘囚軍實歸。



 至京師,以高麗使親告太廟,因拘留之。仍徵元入朝,元竟不至。帝更圖後舉,會天下喪亂,遂不復行。



 百濟之國,蓋馬韓之屬也,出自索離國。其王出行,其侍兒於後妊娠,王還,欲殺之。侍兒曰:「前見天上有氣如大雞子來降,感,故有娠。」王捨之。後生男,王置之豕牢,豕以口氣噓之,不死;後徙於馬闌,亦如之。王以為神,命養之,名曰東明。及長,善射,王忌其猛,復欲殺之。東明乃奔走,南至淹滯水,以弓擊水,魚鱉皆為橋,東明乘之得度,至夫餘而王焉。東明之後有仇台,篤於仁信,始立國於帶方故地。漢遼東太守公孫度以女妻之,遂為東夷強國。初以百家濟,因號百濟。



 其國東極新羅,北接高句麗,西南俱限大海,處小海南,東西四百五十里,南北九百餘里。其都
 曰居拔城,亦曰固麻城。其外更有五方:中方曰古沙城,東方曰得安城,南方曰久知下城,西方曰刀先城,北方曰熊津城。王姓餘氏,號「於羅瑕」,百姓呼為「鞬吉支」,夏言並王也。王妻號「於陸」,夏言妃也。官有十六品:左平五人,一品;達率三十人,二品;恩率,三品;德率,四品;杅率,五品;奈率,六品。已上冠飾銀華。將德,七品,紫帶。施德,八品,皂帶。固德,九品,赤帶。



 季德,十品,青帶。對德,十一品;文督,十二品,皆黃帶。武督,十三品;佐軍,十四品;振武,十五品;剋虞,十六品,皆白帶。自恩率以下,官無常員。各有部司,分掌眾務。內官有前內部、穀內部、內掠部、外掠部、馬部、刀
 部、功德部、藥部、木部、法陪、後宮部。外官有司軍部、司徒部、司空部、司寇部、點口部、客部、外舍部、綢部、日官部、市部,長吏三年一交代。都下有萬家,分為五部,曰上部、前部、中部、下部、後部,部有五巷,士庶居焉。部統兵五百人。五方各有方領一人,以達率為之,方佐貳之。方有十郡,郡有將三人,以德率為之。統兵一千二百人以下,七百人以上。城之內外人庶及餘小城,咸分隸焉。



 其人雜有新羅、高麗、倭等,亦有中國人。其飲食衣服,與高麗略同。若朝拜祭祀,其冠兩廂加翅,戎事則不。拜謁之禮,以兩手據地為禮。婦人不加粉黛,女辮髮垂後,已出嫁,則分為兩
 道,盤於頭上。主似袍而袖微大。兵有弓箭刀槊。俗重騎射,兼愛墳史,而秀異者頗解屬文,能吏事。又知醫藥、蓍龜,與相術、陰陽五行法。有僧尼,多寺塔,而無道士。有鼓角、箜篌、箏竽、篪笛之樂,投壺、樗蒲、弄珠、握槊等雜戲。尤尚奕棋。行宋《元嘉歷》,以建寅月為歲首。賦稅以布、絹、絲、麻及米等,量歲豐儉,差等輸之。其刑罰,反叛、退軍及殺人者,斬;盜者,流,其贓兩倍征之;婦犯姦,沒入夫家為婢。婚娶之禮,略同華俗。父母及夫死者,三年居服,餘親則葬訖除之。土田濕,氣候溫暖,人皆山居。有巨栗,其五穀、雜果、菜蔬及酒醴肴饌之屬,多同於內地。唯無駝、騾、驢、
 羊、鵝、鴨等。



 國中大姓有八族,沙氏、燕氏、氏、解氏、真氏、國氏、木氏、苗氏。其王每以四仲月祭天及五帝之神。立其始祖仇台之廟於國城,歲四祠之。國西南,人島居者十五所,皆有城邑。



 魏延興二年,其王餘慶始遣其冠軍將軍駙馬都尉弗斯侯、長史餘禮、龍驤將軍、帶方太守司馬張茂等上表自通,云:「臣與高麗,源出夫餘,先世之時,篤崇舊款。



 其祖釗,輕廢鄰好,陵踐臣境。臣祖須,整旅電邁,梟斬釗首。自爾以來,莫敢南顧。自馮氏數終,餘燼奔竄,醜類漸盛,遂見陵逼,構怨連禍,三十餘載。若天慈曲矜,遠及無外,速遣一將,來救臣國。當奉送鄙女,執掃後
 宮,并遣子弟,收圉外廄,尺壤疋夫,不敢自有。去庚辰年後,臣西界海中,見尸十餘,并得衣器鞍勒。



 看之,非高麗之物。後聞乃是王人來降臣國,長蛇隔路,以阻于海。今上所得鞍一,以為實矯。」



 獻文以其僻遠,冒險入獻,禮遇優厚,遣使者邵安與其使俱還。詔曰:「得表聞之無恙。卿與高麗不睦,至被陵犯,茍能順義,守之以仁,亦何憂於寇讎也。前所遣使,浮海以撫荒外之國,從來積年,往而不反,存亡達否,未能審悉。卿所送鞍,比校舊乘,非中國之物。不可以疑似之事,以生必然之過。經略權要,已具別旨。」又詔曰:「高麗稱籓先朝,供職日久,於彼雖有自昔
 之釁,於國未有犯令之愆。卿使命始通,便求致伐,尋討事會,理亦未周。所獻錦布海物,雖不悉達,明卿至心。今賜雜物如別。」又詔璉護送安等。至高麗,璉稱昔與餘慶有讎,不令東過。安等於是皆還,乃下詔切責之。五年,使安等從東萊浮海,賜餘慶璽書,褒其誠節。安等至海濱,遇風飄蕩,竟不達而還。



 自晉、宋、齊、梁據江左,亦遣使稱籓,兼受拜封。亦與魏不絕。



 及齊受東魏禪,其王隆亦通使焉。淹死,子餘昌亦通使命於齊。武平元年,齊後主以餘昌為使持節、侍中、車騎大將軍,帶方郡公、百濟王如故。二年,又人餘昌為持節、都督東青州諸軍事、東青
 州刺史。



 周建德六年,齊滅,餘昌始遣使通周。宣政元年,又遣使為獻。



 隋開皇初,餘昌又遣使貢方物,拜上開府、帶方郡公、百濟王。平陳之歲,戰船漂至海東耽牟羅國。其船得還,經於百濟,昌資送之甚厚,并遣使奉表賀平陳。



 文帝善之,下詔曰:「彼國懸隔,來往至難,自今以後,不須年別入貢。」使者舞蹈而去。十八年,餘昌使其長史王辯那來獻方物。屬興遼東之役,遣奉表,請為軍導。帝下詔,厚其使而遣之。高麗頗知其事,兵侵其境。餘昌死,子餘璋立。大業三年,餘璋遣使燕文進朝貢。其年,又遣使王孝鄰入獻,請討高麗。煬帝許之,命覘高麗動靜。然餘璋
 內與高麗通和,挾詐以窺中國。七年,帝親征高麗,餘璋使其臣國智牟來請軍期。帝大悅,厚加賞賜,遣尚書起部郎席律詣百濟,與相知。明年,六軍度遼,餘璋亦嚴兵於境,聲言助軍,實持兩端。尋與新羅有隙,每相戰爭。十年,復遣使朝貢。後天下亂,使命遂絕。



 其南,海行三月有耽牟羅國,南北千餘里,東西數百里,土多麞鹿,附庸於百濟。西行三日,至貊國千餘里云。



 新羅者,其先本辰韓種也。地在高麗東南,居漢時樂浪地。辰韓亦曰秦韓。相傳言秦世亡人避役來適,馬韓割其東界居之,以秦人,故名之曰秦韓。其言語名物,有似
 中國人,名國為邦,弓為弧,賊為寇,行酒為行觴,相呼皆為徒,不與馬韓同。



 又辰韓王常用馬韓人作之,世世相傳,辰韓不得自立王,明其流移之人故也。恒為馬韓所制。辰韓之始,有六國,稍分為十二,新羅則其一也。或稱魏將毋丘儉討高麗破之,奔沃沮,其後復歸故國,有留者,遂為新羅,亦曰斯盧。其人雜有華夏、高麗、百濟之屬,兼有沃沮、不耐、韓、滅之地。其王本百濟人,自海逃入新羅,遂王其國。初附庸于百濟,百濟征高麗,不堪戎役,後相率歸之,遂致強盛。因襲百濟,附庸於迦羅國焉。傳世三十,至真平。以隋開皇十四年,遣使貢方物。文帝拜真
 平上開府、樂浪郡公、新羅王。



 其官有十七等:一曰伊罰干,貴如相國,次伊尺干,次迎干,次破彌干,次大阿尺干,次阿尺干,次乙吉干,次沙咄干,次及伏乾,次大奈摩干,次奈摩,次大舍,次小舍,次吉士,次大烏,次小烏,次造位。外有郡縣。其文字、甲兵,同於中國。選人壯健者悉入軍,烽、戍、邏俱屯營部伍。風俗、刑政、衣服略與高麗、百濟同。每月旦相賀,王設宴會,班賚群官。其日,拜日月神主。八月十五日設樂,令官人射,賞以馬、布。其有大事,則聚官詳議定之。服色尚畫素,婦人辮髮繞頸,以雜彩及珠為飾。婚嫁禮唯酒食而已,輕重隨貧富。新婦之夕,女先
 拜舅姑,次即拜大兄、夫。死有棺斂,葬送起墳陵。王及父母妻子喪,居服一年。田甚良沃,水陸兼種。其五穀、果菜、鳥獸、物產,略與華同。



 大業以來,歲遣朝貢。新羅地多山險,雖與百濟構隙,百濟亦不能圖之也。



 勿吉國在高句麗北,一曰靺鞨。邑落各自有長,不相總一。其人勁悍,於東夷最強,言語獨異。常輕豆莫婁等國,諸國亦患之。去洛陽五千里。自和龍北二百餘里有善玉山,山北行十三日至祁黎山,又北行七日至洛環水,水廣里餘,又北行十五日至太岳魯水,又東北行十八日到其國。國有大水,闊三里餘,名速末水。其部類凡有
 七種:其一號粟末部,與高麗接,勝兵數千,多驍武,每寇高麗;其二伯咄部,在粟末北,勝兵七千;其三安車骨部,在伯咄東北;其四拂涅部,在伯咄東;其五號室部,在拂涅東;其六黑水部,在安車骨西北,其七白山部,在粟末東南。



 勝兵並不過三千,而黑水部尤為勁健。自拂涅以東,矢皆石鏃,即古肅慎氏也。東夷中為強國。所居多依山水。渠帥曰大莫弗瞞咄。國南有從太山者,華言太皇,俗甚敬畏之,人不得山上溲汙,行經山者,以物盛去。上有熊羆豹狼,皆不害人,人亦不敢殺。地卑濕,築土如堤,鑿穴以居,開口向上,以梯出入。其國無牛,有馬,車則步推,相
 與偶耕。土多粟、麥、穄,菜則有葵。水氣堿,生鹽於木皮之上,亦有鹽池。其畜多獵,無羊。嚼米為酒,飲之亦醉。婚嫁,婦人服布裙,男子衣豬皮裘,頭插武豹尾。俗以溺洗手面,於諸夷最為不潔。初婚之夕,男就女家,執女乳而罷。妒,其妻外淫,人有告其夫,夫輒殺妻而後悔,必殺告者。由是姦淫事終不發。人皆善射。以射獵為業。角弓長三尺,箭長尺二寸,常以七八月造毒藥,傅矢以射禽獸,中者立死。煮毒藥氣亦能殺人。其父母春夏死,立埋之,塚上作屋,令不雨濕;若秋冬死,以其尸捕貂,貂食其肉,多得之。



 延興中,遣乙力支朝獻。太和初,又貢馬五百匹。乙
 力支稱:初發其國,乘船溯難河西上,至太濔河,沈船於水。南出陸行,度洛孤水,從契丹西界達和龍。自云其國先破高句麗十落,密共百濟謀,從水道并力取高麗,遣乙力支奉使大國,謀其可否。詔敕:「三國同是籓附,宜共和順,勿相侵擾。」乙力支乃還。從其來道,取得本船,汎達其國。九年,復遣使侯尼支朝。明年,復入貢。其傍有大莫盧國、覆鐘國、莫多回國、庫婁國、素和國、具弗伏國、匹黎爾國、拔大何國、郁羽陵國、庫伏真國、魯婁國、羽真侯國,前後各遣使朝獻。太和十二年,勿吉復遣使貢楛矢、方物於京師。十七年,又遣使人婆非等五百餘人朝貢。景明
 四年,復遣使侯力歸朝貢。自此迄于正光,貢使相尋。爾後中國紛擾,頗或不至。興和二年六月,遣石文云等貢方物。以至於齊,朝貢不絕。



 隋開皇初,相率遣使貢獻。文帝詔其使曰:「朕聞彼土人勇,今來實副朕懷。



 視爾等如子,爾宜敬朕如父。」對曰:「臣等僻處一方,聞內國有聖人,故來朝拜。



 既親奉聖顏,願長為奴僕。」其國西北與契丹接,每相劫掠,後因其使來,文帝誡之,使勿相攻擊。使者謝罪。文帝因厚勞之,令宴飲於前,使者與其徒皆起舞,曲折多戰鬥容。上顧謂侍臣曰:「天地間乃有此物,常作用兵意。」然其國與隋懸隔,唯粟末、白山為近。煬帝初,與
 高麗戰,頻敗其眾。渠帥突地稽率其部降,拜右光祿大夫,居之柳城。與邊人來往,悅中國風俗,請被冠帶,帝嘉之,賜以錦綺而褒寵之。及遼東之役,突地稽率其徒以從,每有戰功,賞賜甚厚。十三年,從幸江都,尋放還柳城。李密遣兵邀之,僅而得免。至高陽,沒於王須拔。未幾,遁歸羅藝。



 奚,本曰庫莫奚,其先東部胡宇文之別種也。初為慕容晃所破,遺落者竄匿松漠之間。俗甚不潔凈,而善射獵,好為寇抄。登國三年,道武親自出討,至弱水南大破之,獲其馬、牛、羊、豕十餘萬。帝曰:「此群狄諸種,不識德義,鼠
 竊狗盜,何足為患?今中州大亂,吾先平之,然後張其威懷,則無所不服矣。」既而車駕南遷,十數年間,諸種與庫莫奚亦皆滋盛。及開遼海,置戍和龍,諸夷震懼,各獻方物。文成、獻文之世,庫莫奚歲致名馬、文皮。孝文初,遣使朝貢。太和四年,輒入塞內,辭以畏地豆干抄掠,詔書切責之。二十二年,入寇安州,時營、燕、幽三州兵數千人擊走之。後復款附,每求入塞交易。宣武詔曰:「庫莫奚去太和二十一年以前,與安、營二州邊人參居,交易往來,並無欺貳。至二十二年叛逆以來,遂爾遠竄。今雖款附,猶在塞表,每請入塞,與百姓交易。若抑而不許,乖其歸向之
 心;信而不慮,或有萬一之驚。交市之日,州遣士監之。」自此已後,歲常朝獻,至武定已來不絕。齊受魏禪,歲時來朝。



 其後種類漸多,分為五部:一曰辱紇主,二曰莫賀弗,三曰契個,四曰木昆,五曰室得。每部俟斤一人為其帥。隨逐水草,頗同突厥。有阿會氏,五部中最盛,諸部皆歸之。每與契丹相攻擊,虜獲財畜,因遣使貢方物。



 契丹國,在庫莫奚東,與庫莫奚異種同類。並為慕容晃所破,俱竄於松漠之間。



 登國中,魏大破之,遂逃迸,與庫莫奚分住。經數十年,稍滋蔓,有部落,於和龍之北數百里為寇盜。真君以來,歲貢名馬。獻文時,使莫弗紇何辰
 來獻,得班饗於諸國之末。歸而相謂,言國家之美,心皆忻慕,於是東北群狄聞之,莫不思服。悉萬丹部、何大何部、伏弗郁部、羽陵部、日連部、匹潔部、黎部、吐六干部等各以其名馬、文皮獻天府。遂求為常,皆得交市於和龍、密雲之間,貢獻不絕。太和三年,高句麗竊與蠕蠕謀,欲取地豆干以分之。契丹舊怨其侵軼,其莫賀弗勿幹率其部落,車三千乘、眾萬餘口,驅徙雜畜求內附,止於白狼水東。自此歲常朝貢。後告飢,孝文聽其入關市糴。及宣武、孝明時,恒遣使貢方物。熙平中,契丹使人初真等三十人還,靈太后以其俗嫁娶之際以青為上服,人
 給青兩匹,賞其誠款之心,餘依舊式朝貢。及齊受東魏禪,常不斷絕。



 天保四年九月,契丹犯塞,文宣帝親戎北討。至平州,遂西趣長塹。詔司徒潘相樂帥精騎五千,自東道趣青山;復詔安德王韓軌帥精騎四千東趣,斷契丹走路。



 帝親踰山嶺,奮擊大破之,虜十餘萬口、雜畜數十萬頭。相樂又於青山大破契丹別部。所虜生口,皆分置諸州。其後復為突厥所逼,又以萬家寄於高麗。



 其俗與靺鞨同,好為寇盜。父母死而悲哭者,以為不壯。但以其屍置於山樹之上,經三年後,乃收其骨而焚之。因酌酒而祝曰:「冬月時,向陽食。若我射獵時,使我多得豬、鹿。」
 其無禮頑囂,於諸夷最甚。



 隋開皇四年,率莫賀弗來謁。五年,悉其眾款塞,文帝納之,聽居其故地。責讓之,其國遣使詣闕,頓顙謝罪。其後,契丹別部出伏等背高麗,率眾內附。文帝見來,憐之。上方與突厥和好,重失遠人之心,悉令給糧還本部,敕突厥撫納之。



 固辭不去。部落漸眾,遂北徙,逐水草,當遼西正北二百里,依託紇臣水而居,東西亙三百里,分為十部。兵多者三千,少者千餘。逐寒暑,隨水草畜牧。有征伐,則曾帥相與議之,興兵動眾,合如符契。突厥沙缽略可汗遣吐屯潘垤統之,契丹殺吐屯而遁。大業七年,遣使朝,貢方物。



 室韋國,在勿吉北千里,去洛陽六千里。「室」或為「失」,蓋契丹之類,其南者為契丹,在北者號為失韋。路出和龍千餘里,入契丹國,又北行十日至啜水,又北行三日有善水,又北行三日有犢了山,其山高大,周回三百里。又北行三百餘里,有大水名屈利,又北行三日至刃水,又北行五日到其國。有大水從北而來,廣四里餘,名奈水。國土下濕,語與庫莫奚、契丹、豆莫婁國同。頗有粟、麥及穄。



 夏則城居,冬逐水草,多略貂皮。丈夫索髮,用角弓,其箭尤長。女婦束髮作叉手髻。其國少竊盜,盜一征三;殺人者責馬三百匹。男女悉衣白鹿皮襦褲。有曲,釀酒。俗愛赤
 珠,為婦人飾,穿掛於頸,以多為貴。女不得此,乃至不嫁。父母死,男女眾哭三年,尸則置於林樹之上。



 武定二年四月,始遣使張烏豆伐等獻其方物。迄武定末,貢使相尋。及齊受東魏禪,亦歲時朝聘。



 其後分為五部,不相總一,所謂南室韋、北室韋、缽室韋、深末怛室韋、大室韋,並無君長。人貧弱,突厥以三吐屯總領之。



 南室韋在契丹北三千里,土地卑濕,至夏則移向北。貸勃、欠對二山多草木,饒禽獸,又多蚊蚋,人皆巢居,以避其患。漸分為二十五部,每部有餘莫弗瞞咄,猶酋長也。死則子弟代之,嗣絕則擇賢豪而立之。其俗,丈夫皆被髮,婦女盤髮,衣
 服與契丹同。乘牛車,以蘧蒢為屋,如突厥氈車之狀。度水則束薪為伐,或有以皮為舟者。馬則織草為韉,結繩為轡。匡寢則屈木為室,以蘧蒢覆上,移則載行。



 以豬皮為席,編木為藉,婦女皆抱膝坐。氣候多寒,田收甚薄。無羊,少馬,多豬、牛。與靺鞨同俗,婚嫁之法,二家相許竟,輒盜婦將去,然後送牛馬為聘,更將婦歸家,待有孕,乃相許隨還舍。婦人不再嫁,以為死人之妻,難以共居。部落共為大棚,,人死則置其上。居喪三年,年唯四哭。其國無鐵,取給於高麗。多貂。



 南室韋北行十一日至北室韋,分為九部落,繞吐紇山而居。其部落渠帥號乞引莫賀咄。
 每部有莫何弗三人以貳之。氣候最寒,雪深沒馬。冬則入山居土穴,土畜多凍死。饒麞鹿,射獵為務,食肉衣皮,鑿冰沒水中而網取魚鱉。地多積雪,懼陷坑阱,騎木而行,人答即止。皆捕貂為業,冠以狐貂,衣以魚皮。



 又北行千里至缽室韋,依胡布山而住,人眾多北室韋,不知為幾部落。用樺皮蓋屋,其餘同北室韋。



 從缽室韋西南四日行,至深末怛室韋,因水為號也。冬月穴居,以避太陰之氣。



 又西北數千里至大室韋,徑路險阻,言語不通。尤多貂及青鼠。



 北室韋時遣使貢獻,餘無至者。



 豆莫婁國,在勿吉北千里,舊北夫餘也。在室韋之東,東至
 於海,方二千餘里。



 其人土著,有居室倉庫。多山陵廣澤,於東夷之域,最為平敞。地宜五穀,不生五果。其人長大,性強勇謹厚,不冠抄。其君長皆六畜名官,邑落有豪帥。飲食亦用俎豆。有麻布,衣製類高麗而帽大。其國大人,以金銀飾之。用刑嚴急,殺人者死,沒其家人為奴婢。俗淫,尤惡妒者,殺之尸於國南山上,至腐,女家始得輸牛馬,乃與之。或言濊貊之地也。



 地豆干國,在室韋西千餘里。多牛、羊,出名馬,皮為衣服,無五穀,唯食肉酪。延興二年八月,遣使朝貢,至於太和六年,貢使不絕。十四年,頻來犯塞,孝文詔征西大將軍
 陽平王頤擊走之。自後時朝京師,迄武定末,貢使不絕。及齊受禪,亦來朝貢。



 烏洛侯國,在地豆干北,去代都四千五百餘里。其地下濕,多霧氣而寒。入冬則穿地為室,夏則隨原阜畜牧。多豕,有穀、麥。無大君長,部落莫弗,皆世為之。



 其俗,繩髮皮服,以珠為飾。人尚勇,不為姦竊,故慢藏野積而無寇盜。好射獵。



 樂有箜篌,木槽革面而施九弦。其國西北有完水,東北流合於難水,其小水,皆注於難,東入海。又西北二十日行,有于巳尼大水,所謂北海也。



 太武真君四年來朝,稱其國西北有魏先帝舊墟石室,南北九十步,東西
 四十步,高七十尺,室有神靈,人多祈請。太武遣中書侍郎李敞告祭焉,刊祝文於石室之壁而還。



 流求國,居海島,當建安郡東。水行五日而至。土多山洞。其王姓歡斯氏,名渴刺兜,不知其由來有國世數也。彼土人呼之為可老羊,妻曰多拔茶。所居曰波羅檀洞,塹柵三重,環以流水,樹棘為籓。王所居舍,其大一十六間,雕刻禽獸。多鬥鏤樹,似橘而葉密,條纖如髮之下垂。國有四五帥,統諸洞,洞有小王。往往有村,村有鳥了帥,並以善戰者為之,自相樹立,主一村之事。男女皆白糸寧繩纏髮,從項後盤繞至額。其男子用鳥羽為冠,裝以珠貝,
 飾以赤毛,形製不同。婦人以羅紋白布為帽,其形方正。織鬥鏤皮并雜毛以為衣,製裁不一。綴毛垂螺為飾,雜色相間,下垂小貝,其聲如珮。綴璫施釧,懸珠於頸。織藤為笠,飾以毛羽。有刀槊、弓箭、劍鈹之屬。其處少鐵,刀皆薄小,多以骨角輔助之。編糸寧為甲,或用熊豹皮。王乘木獸,令左右輿之,而導從不過十數人。小王乘機,鏤為獸形。國人好相攻擊,人皆驍健善走,難死耐創。諸洞各為部隊,不相救助。兩軍相當,勇者三五人出前跳噪,交言相罵,因相擊射。如其不勝,一軍皆走,遣人致謝,即共和解。



 收取鬥死者聚食之,仍以髑髏將向王所,王則賜之
 以冠,便為隊帥。



 無賦斂,有事則均稅。用刑亦無常準,皆臨事科決。犯罪皆斷於鳥了帥,不伏則上請於王,王令臣下共議定之。獄無枷鎖,唯用繩縛。決死刑以鐵錐大如筋,長尺餘,鑽頂殺之,輕罪用杖。俗無文字,望月虧盈,以紀時節,草木榮枯,以為年歲。人深目長鼻,類於胡,亦有小慧。無君臣上下之節,拜伏之禮。父子同床而寢。



 男子拔去髭須,身上有毛處皆除去。婦人以黑黥手為蟲蛇之文。嫁娶以酒、珠貝為聘,或男女相悅,便相匹偶。婦人產乳,必食子衣,產後以火自灸,令汗出,五日便平復。以木槽中暴海水為鹽,木汁為酢,米面為酒,其味甚薄。
 食皆用手。遇得異味,先進尊者。凡有宴會,執酒者必待呼名而後飲,上王酒者,亦呼王名後銜盃共飲,頗同突厥。歌呼蹋蹄,一人唱,眾皆和,音頗哀怨。扶女子上膊,搖手而舞。



 其死者氣將絕,輦至庭前,親賓哭泣相弔。浴其屍,以布帛縛纏之,裹以葦席,襯土而殯,上不起墳。子為父者,數月不食肉。其南境風俗少異,人有死者,邑里共食之。有熊、豺、狼,尤多豬、雞、無羊、牛、驢、馬。厥田良沃,先以火燒,而引水灌,持一鍤,以石為刃,長尺餘,闊數寸,而墾之。宜稻、粱、禾、黍、麻、豆、赤豆、胡黑豆等。木有楓、栝、樟、松、楩、楠、枌、梓。竹、藤、果、藥,同於江表。風土氣候,與嶺南相類。俗
 事山海之神,祭以肴酒。戰鬥殺人,便將所殺人祭其神。或依茂樹起小屋,或懸髑髏於樹上,以箭射之,或累石繫幡,以為神主。王之所居,壁下多聚髑髏以為佳。人間門戶上,必安獸頭骨角。



 隋大業元年,海師何蠻等,每春秋二時,天清風靜,東望依稀,似有煙霧之氣,亦不知幾千里。三年,煬帝令羽騎尉朱寬入海求訪異俗,何蠻言之,遂與蠻俱往。



 同到流求國,言不通,掠一人而反。明年,復令寬慰撫之,不從。寬取其布甲而歸。


時倭國使來朝見之,曰:「此夷邪夕國人所用。」帝遣武賁郎將陳稜、朝請大夫張鎮州率兵自義安浮海至高華嶼,又東行二日到
 \gezhu{
  句
  黽}
 鼊嶼,又一日,便至流求。流求不從,稜擊走之。進至其都,焚其官室,虜其男女數千人,載軍實而還。自爾遂絕。



 倭國,在百濟、新羅東南,水陸三千里,於大海中依山島而居。魏時,譯通中國三十餘國,皆稱子,夷人不知里數,但計以日。其國境,東西五月行,南北三月行,各至於海。其地勢,東高西下。居於邪摩堆,則《魏志》所謂邪馬臺者也。又云:去樂浪郡境及帶方郡並一萬二千里,在會稽東,與儋耳相近。俗皆文身,自云太伯之後。計從帶方至倭國,循海水行,歷朝鮮國,乍南乍東,七千餘里,始度一海。又南千餘里,度一海,闊千餘里,名瀚海,至一支國。又
 度一海千餘里,名末盧國。又東南陸行五百里,至伊都國。又東南百里,至奴國。又東行百里,至不彌國。又南水行二十日,至投馬國。又南水行十日,陸行一月,至邪馬臺國,即倭王所都。



 漢光武時,遣使入朝,自稱大夫。安帝時,又遣朝貢,謂之倭奴國。靈帝光和中,其國亂,遞相攻伐,歷年無主。有女子名卑彌呼,能以鬼道惑眾,國人共立為王。無夫,有二男子,給王飲食,通傳言語。其王有宮室、樓觀、城柵,皆持兵守衛,為法甚嚴。魏景初三年,公孫文懿誅後,卑彌呼始遣使朝貢。魏主假金印紫綬。



 正始中,卑彌呼死,更立男王。國中不服,更相誅殺,復立卑彌
 呼宗女臺與為王。



 其後復立男王,並受中國爵命。江左歷晉、宋、齊、梁,朝聘不絕。



 及陳平,至開皇二十年,倭王姓阿每,字多利思比孤,號阿輩雞彌,遣使詣闕。



 上令所司訪其風俗,使者言倭王以天為兄,以日為弟,天明時出聽政,跏趺坐,日出便停理務,云委我弟。文帝曰:「此大無義理。」於是訓令改之。王妻號雞彌,後宮有女六七百人,名太子為利歌彌多弗利。無城郭,內官有十二等:一曰大德,次小德,次大仁,次小仁,次大義,次小義,次大禮,次小禮,次大智,次小智,次大信,次小信,員無定數。有軍尼一百二十人,猶中國牧宰。八十戶置一伊尼翼,如今里
 長也。十伊尼翼屬一軍尼。其服飾,男子衣裙襦,其袖微小;履如屨形,漆其上,繫之腳。人庶多跣足,不得用金銀為飾。故時,衣橫幅,結束相連而無縫,頭亦無冠,但垂發於兩耳上。至隋,其王始制冠,以錦彩為之,以金銀鏤花為飾。



 婦人束髮於後,亦衣裙襦,裳皆有襈。扦竹聚以為梳。編草為薦,雜皮為表,緣以文皮。有弓、矢、刀、槊、弩、、斧,漆皮為甲,骨為矢鏑。雖有兵,無征戰。其王朝會,必陳設儀仗,奏其國樂。戶可十萬。俗,殺人、強盜及姦,皆死;盜者計贓酬物,無財者,沒身為奴;自餘輕重,或流或杖。每訊冤獄,不承引者,以木壓膝;或張強弓,以弦鋸其項。或置小石
 於沸湯中,令所競者探之,云理曲者即手爛;或置蛇甕中,令取之,云曲者即螫手。人頗恬靜,罕爭訟,少盜賊。樂有五弦、琴、笛。男女皆黥臂,點面,文身。沒水捕魚。無文字,唯刻木結繩。敬佛法,於百濟求得佛經,始有文字,知卜筮,尤信巫覡。每至正月一日,必射戲飲酒,其餘節,略與華同。好棋博、握槊、樗蒱之戲。氣候溫暖,草木冬青。土地膏腴,水多陸少。



 以小環掛鸕鶿項,令入水搏魚,日得百餘頭。俗無盤俎,藉以槲葉,食用手餔之。



 性質直,有雅風。女多男少,婚嫁不取同姓,男女相悅者即為婚。婦入夫家,必先跨火,乃與夫相見。婦人不淫妒。死者斂以棺
 郭,親賓就屍歌舞,妻子兄弟以白布制服。貴人三年殯,庶人卜日而痤。及葬,置屍船上,陸地牽之,或以小輿。有阿蘇山,其石無故火起接天者,俗以為異,因行祭禱。有如意寶珠,其色青,大如雞卵,夜則有光,云魚眼睛也。新羅、百濟皆以倭為大國,多珍物,並仰之,恆通使往來。



 大業三年,其王多利思比孤遣朝貢,使者曰:「聞海西菩薩天子重興佛法,故遣朝拜,兼沙門數十人來學佛法。」國書曰:「日出處天子致書日沒處天子,無恙。」



 云云。帝覽不悅,謂鴻臚卿曰:「蠻夷書有無禮者,勿復以聞。」明年,上遣文林郎裴世清使倭國,度百濟,行至竹島,南望耽羅國,
 經都斯麻國,迥在大海中。又東至一支國,又至竹斯國。又東至秦王國,其人同於華夏,以為夷洲,疑不能明也。



 又經十餘國,達於海岸。自竹斯國以東,皆附庸於倭。倭王遣小德何輩臺從數百人,設儀仗,鳴鼓角來迎。後十日,又遣大禮哥多毗從二百餘騎,郊勞。既至彼都,其王與世清。來貢方物。此後遂絕。



 論曰:廣谷大川異制,人生其間異俗,嗜欲不同,言語不通,聖人因時設教,所以達其志而通其俗也。九夷所居,與中夏懸隔,然天性柔順,無橫暴之風,雖綿邈山海,而易以道御。夏、殷之世,時或來王。暨箕子避地朝鮮,始有
 八條之禁,疏而不漏,簡而可入,化之所感,千載不絕。今遼東諸國,或衣服參冠冕之容,或飲食有俎豆之器,好尚經術,愛樂文史,游學於京都者,往來繼路,或沒世不歸,非先哲之遺風,其孰能致於斯也?故孔子曰:「言忠信,行篤敬,雖蠻貊之邦行矣。」



 誠哉斯言。其俗之可採者,豈楛矢之貢而已乎?自魏迄隋,年移四代,時方爭競,未遑外略。洎開皇之末,方征遼左,天時不利,師遂無功。二代承基,志苞宇宙,頻踐三韓之地,屢發千鈞之弩。小國懼亡,敢同困獸,兵不載捷,四海騷然,遂以土崩,喪身滅國。兵志有之曰:「務廣德者昌,務廣地者亡。」然遼東之地,不
 列於郡縣久矣,諸國朝正奉貢,無闕於歲時。二代震而矜之,以為人莫己若,不能懷以文德,遽動干戈,內恃富強,外思廣地,以驕取怨,以怒興師,若此而不亡,自古未聞也。然四夷之戒,安可不深念哉!其豆莫婁、地豆干、烏洛侯,歷齊周及隋,朝貢遂絕,其事故莫顯云。



\end{pinyinscope}