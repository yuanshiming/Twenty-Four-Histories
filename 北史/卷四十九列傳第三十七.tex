\article{卷四十九列傳第三十七}

\begin{pinyinscope}

 硃
 瑞叱列延慶斛斯椿子徵孫政賈顯度弟智樊子鵠侯深賀拔允弟勝勝弟岳侯莫陳悅念賢梁覽雷紹毛遐弟鴻賓乙弗朗硃瑞,字元龍,代郡桑乾人也。祖就,沛縣令。父惠,行太原太守。瑞貴達,並贈刺史。瑞長厚質直,敬愛人士,爾朱榮引為大行臺郎中,甚見親任,以為黃門侍郎,仍中書舍
 人。榮恐朝廷事意有所不知,故居之門下,為腹心之寄。封陽邑縣公。及元顥內逼,從車駕於河陽,除侍中、兼吏部尚書,改封北海郡公。莊帝還洛,改封樂陵郡公,仍侍中。瑞雖為爾朱榮所委,而善處朝廷間。帝亦賞遇之,嘗謂侍臣曰:「為人臣當須忠實,至如朱元龍者,朕待之亦不異餘人。」瑞以青州樂陵有朱氏,意欲歸之,故求為青州中正。又以滄州樂陵亦有朱氏,而心好河北,遂乞三從內並屬滄州樂陵郡。詔許之,仍轉滄州大中正。爾朱榮死,瑞與世隆俱北走。以莊帝待之素厚,且見世隆等並無雄才,終當敗喪,於路乃還,帝大悅。時爾朱天光擁
 眾關右,帝招納之,乃以瑞兼尚書左僕射,為西道大行臺,以慰勞焉。既達長安,會爾朱兆入洛,復還京師。都督斛斯椿先與瑞有隙,數譖之於世隆,世隆遂誅之。



 太昌初,贈開府儀同三司、青州刺史,謚曰恭穆。



 叱列延慶,代西部人也,世為酋帥。延慶娶爾朱世隆姊,故被爾朱榮親遇。普泰初,世隆得志,特見委重,兼尚書左僕射、山東行臺、北海郡公。時幽州刺史劉靈助以莊帝幽崩,遂舉兵唱義,世隆白節閔帝,以延慶與大都督侯深於定州討之。



 深以靈助善占,百姓信惑,未易可圖,欲還師入據關拒險,以待其變。延慶以靈助庸人,彼皆
 恃其妖術,坐看符厭,寧肯戮力致死。宜詭言西歸,可襲而禽。深從之,乃出頓城西,聲云將還,詰朝造靈助壘,遂破禽之。及韓陵戰敗,延慶與爾朱仲遠走度石濟。仲遠南竄,延慶北降齊神武,仍從並州。後赴洛,孝武帝以為中軍大都督。孝武之西,齊神武誅之。



 斛斯椿,字法壽,廣牧富昌人也。其先世為莫弗大人。父足,一名敦,明帝時為左牧令。時河西賊起,牧人不安,椿乃將家投爾朱榮。征伐有功,稍遷中散大夫,署外兵事。椿性佞巧,甚得榮心,軍之密謀,頗亦關預。莊帝初,改封陽曲縣公,除榮大將軍府司馬。後為東徐州刺史。及榮
 死,椿甚憂懼。時梁以汝南王悅為魏主,資其士馬,次於境上。椿遂棄州歸悅。悅授尚書左僕射、司空公,封靈丘郡公,又為大行臺前驅都督。會爾朱兆入洛,悅知不逮,南旋。椿復背悅歸兆。以參立節閔謀,拜侍中、驃騎大將軍、儀同三司,封城陽郡公。尋加開府。時椿父足先在秀容,忽有傳其死問,椿請減己階以贈之。尋知其父猶存,詔復官,仍除其父為車騎將軍、揚州刺史。



 椿以爾朱兆擅權,懼禍,乃與賀拔勝俱說世隆以正道。世隆不悅,欲害椿,賴爾朱天光救,得免。及世隆、度律與兆自相疑,椿與賀拔勝和之,兆執椿、勝還營。



 椿又陳以正理,兆謝而
 遣之。椿謂勝曰:「天下皆怨毒爾朱,吾等附之,亡無日矣,不如圖之。」勝曰:「天光與兆,各據一方,今俱禽為難。」椿曰:「易致耳。」



 乃說世隆追天光等赴洛,討齊神武。及韓陵之敗,椿謂都督賈顯智等曰:「若不先執爾朱,我等死無類矣。」遂與顯智等夜於桑下盟約,倍道兼行。椿入北中城,收爾朱部曲,盡殺之。令弟元壽與張歡、長孫承業、顯智等襲世隆、彥伯兄弟,並斬於閶闔門外。椿入洛,縣世隆兄弟首於其門樹。椿父出見,謂曰:「汝與爾朱約為兄弟,今何忍縣其頭於家門?寧不愧負天地!」椿乃傳世隆等首,並囚度律、天光,送於齊神武。



 及神武入洛,椿謂賀拔
 勝曰:「今天下事在吾與君,若不先制人,將為人所制。



 高歡初至,圖之不難。」勝曰:「彼有心於人,害之不詳。比數夜與歡同宿,具序往昔之懷,兼荷兄恩意甚多,何苦憚之!」椿乃止。孝武帝立,拜椿侍中、儀同開府、城陽郡公。父足亦加開府,子悅太中大夫,同日受拜。當時榮之。



 椿自以數反,意常不安,遂密勸孝武帝置閣內都督部曲,又增武直人數百,直閣已下員別數百,皆選天下輕剽以充之。又說帝數出游幸,號令部曲,別為行陣,椿自約勒指麾其間。從此以後,軍謀朝政,一決於椿。又勸帝徵兵,詭稱南討,將以伐齊神武。帝從之。以椿為前驅大都督。椿
 因奏請率精騎二千,夜度河,掩其勞弊。帝始然之,黃門侍郎楊寬說帝曰:「高歡以臣伐君,何所不至?今假兵於人,恐生他變。今度河,萬一有功,是滅一高歡,復生一高歡矣。」帝遂敕椿停行。椿歎曰:「頃熒惑入南斗,今上信左右間構,不用吾計,豈天道乎!」



 帝勒兵河橋,命椿自洛而東,至武牢。帝以賈顯智背叛,東師失律,將幸關中。



 乃遣使命椿因從入關。拜尚書令,侍中如故,封常山郡公。歷位司徒、太保,仍尚書令。時寇難未息,內外戒嚴,唯椿得列威儀,鳴騶清路。遷太傅,薨,年四十三。



 帝親臨弔,百僚赴哭。詔賜東園秘器,遣尚書、梁郡王景略監護喪事。贈
 大將軍、錄尚書、三十州諸軍事、侍中、恒州刺史、常山郡王,謚曰文宣,祭以太牢。又詔改大將軍,贈大司馬,給轀輬車。及葬,車駕臨於渭陽,止紼慟哭。



 帝嘗給椿店數區,耕牛三十頭。椿以國難未平,不可與百姓爭利,辭店受牛,日烹一頭,以饗軍士。及死,家無餘資。有四子:悅、恢、徵、演。演為齊神武所殺,三子入關。



 徵字士亮,博涉群書,尤精三禮,兼解音律。有至性。居父喪,朝夕共一溢米。



 少以父勳賜爵城陽郡公。大統末,起家通直散騎常侍,稍遷兼太常少卿。



 自魏孝武遷西,雅樂廢缺,徵博採遺逸,稽諸典故,創新改舊,方始備焉。又
 樂有錞于者,近代絕此器,或有自蜀得之,皆莫之識。徵見之曰:「此錞于也。」



 眾弗信之,徵遂依干寶《周禮注》,以芒筒捋之。其聲極清,眾乃歎服。徵仍取以合樂焉。六官建,拜司樂下大夫,遷司樂中大夫,進位驃騎大將軍、開府儀同三司,轉內史下大夫。天和三年,周武帝以徵經有師法,詔令授皇諸子。宣帝時為魯公,與諸皇子等咸服青衿,行束修之禮,受業於徵。仍並呼徵為夫子,儒者榮之。六年,除司宗中大夫,行內史,仍攝樂部。進封岐國公,尋轉小宗伯。除太子太傅,仍小宗伯。宣帝嗣位,遷上大將軍、大宗伯。時武帝初崩,梓宮在殯,帝意欲速葬,令朝
 臣議之。徵與內史宇文孝伯等,固請依《禮》七月,帝竟不許。



 帝之為太子也,宮尹鄭譯坐不能以正道調護,被謫除名。而帝雅親愛譯。至是,拜譯內史中大夫,甚委任之。譯乃獻新樂,十二月各一笙,每笙用十六管。帝令與徵議之。徵駮而奏之曰:「《禮》云,十二律轉相生,聲五具在十六焉,六律十二管,還相為宮。然詳一笙十六管,總一百九十二管,既無相生之理,又無還宮之義。



 臣恐鄭聲亂樂,未合於古。夫音樂之起,本於人心,天之應人,有如影響。為善者,天報之以福;為惡者,天譴之以殃。故舜彈五弦之琴,歌《南風》之詩,而天下化。



 紂為朝歌、北里之音,而
 社稷滅。是知樂也者,和情性,移風俗,動天地,感鬼神,禍福所基,盛衰攸繫,安可不慎哉!案譯之所為,不師古始。若以月奏一笙,則鐘鼓諸色,各須一十有二。雅樂之備,已充廟廷,今若益之,於何陳列?方須更闢階墀,增修廊宇,非急之務,寧可勞人?如謂笙管之外,不須加造,則樂之損益,豈繫於笙?進退無據,竊謂不可。」帝頗納之,且令停譯所獻。



 及武帝山陵回,帝欲作樂,復令議其可不。徵曰:「《孝經》云『聞樂不樂。』聞尚不樂,其況作乎!」鄭譯曰:「既云聞樂,明即非無,止可不樂,何容不奏。」



 帝遂依譯議,譯因此銜之。帝後肆行非度,昏慮日甚。徵以荷武帝重恩,嘗
 備位師傅,乃上疏極諫,指陳帝失。不納。譯因譖之,遂下徵於獄。徵懼不免,獄卒張元平哀之,乃以佩刀穿墻,送之出。元平被捶拷百數,而無所言。徵既出,匿於人家,後遇赦得免,然猶坐除名。



 隋文帝踐極,例復官爵,除太子太傅,仍詔徵脩撰樂書。開皇四年薨,年五十六。初,隋文帝為大司馬,有外姻喪,徵就第弔之。久而不出。徵怒,遂弗之待。



 比出候,徵已去矣。隋文帝以此常恨之。至是,詔所司謚之曰闇。子該嗣。徽所撰《樂典》十卷。



 兄恢,散騎常侍,新蔡郡公。子政嗣。



 政明悟有器幹,隋開皇中,以軍功授儀同,甚為楊素所
 禮。大業中,位尚書兵曹郎,漸見委遇。玄感兄弟,俱與之交。遼東之役,兵部尚書段文振卒,侍郎明雅復以罪廢,帝彌屬意於政。尋遷兵部侍郎。稱為幹理。玄感之反,政與通謀,及玄縱等亡歸,亦政之計。及帝窮玄縱黨與,政亡奔高麗。明年,帝復東征,高麗請和,遂送政。鎖至京師以告廟,左翊衛大將軍宇文述請變常法行刑,帝許之。以出金光門,縛之於柱,公卿百僚,並親擊射。臠其肉,多有啖者,然後烹焚,揚其骨灰。



 椿弟元壽,性剛毅諒直,武力過人,彎弓兩石,左右馳射。歷位吏部尚書,封桑乾縣伯。孝武踐阼,進爵為公,除豫州刺史。及車駕西巡,為部
 下所殺。贈司空公,謚曰景莊。



 賈顯度,中山無極人也。父道監,沃野鎮長史。顯度形貌偉壯,有志氣。初為別將,防守薄骨律鎮。正光末,北鎮擾亂,顯度乃率鎮人浮河而下,達秀容,為爾朱榮所留。隨榮破葛榮,封石艾縣公,累遷南袞州刺史。爾朱榮之死,顯度奔梁。



 普泰初,還朝。後隨爾朱度律等敗於韓陵,與斛斯椿及弟智等先據河橋,誅爾朱氏。



 孝武帝初,除尚書左僕射,尋加驃騎大將軍、開府儀同三司、定州大中正。永熙三年,為雍州刺史、西道大行臺。親故祖餞於張方橋,顯度執酒曰:「顯智性輕躁,好去就,覆敗吾家,其此
 人也!」武帝入關後,顯智果同於齊神武。孝武帝怒,乃賜顯度死。



 智字顯智,少有膽決,以軍功累遷金紫光祿大夫,封義陽縣伯。及爾朱仲遠為徐州刺史,智隸仲遠赴彭城。爾朱榮死,仲遠舉兵向洛,智不從之。莊帝聞而善之。



 普泰初,還洛。仲遠忿其乖背,議欲殺之。智兄顯度先為世隆所厚,世隆為解喻得全。後進爵為公。隨度律等敗於韓陵。智與顯度、斛斯椿謀誅爾朱氏,顯度據守北中城,令智等入京,禽世隆兄弟。



 孝武帝初,除開府儀同三司、滄州刺史。在州貪縱,甚為人害。孝武徵還京師,加侍中,除
 濟州刺史,率眾達東郡,仍停不進。於長壽津為相州刺史竇泰所破。天平初,赴晉陽。智去就多端,後坐事死。



 樊子鵠,代郡平城人也。其先荊州蠻酋,徙代。父興,平城鎮長史、歸義侯。



 普泰中,子鵠貴,乃贈荊州刺史。子鵠逢北鎮擾亂,南至并州,爾朱榮引為都督府倉曹參軍。使詣京師,靈太后問榮兵勢,子鵠應對稱旨。太后嘉之,除直齊,封南和縣子,令還赴榮。建義初,拜晉州刺史,封永安縣伯。永安二年,以招納叛蜀,進封中都縣公,又兼尚書行臺,政有威信。尋徵授都官尚書、西荊州大中正。後兼右僕射,為行臺。進封西陽郡公,尚書如故,假驃騎將
 軍,率所部為都督。時爾朱榮在晉陽,京師之事,子鵠頗預委寄,故在臺閣,征官不解。後出為殷州刺史。屬歲旱儉,子鵠恐人流亡,乃勒有粟家分濟貧者,并遣人牛易力,多種二麥,州內以此獲安。



 爾朱榮死,世隆等遣書招子鵠,子鵠不從。以母在晉陽,啟求移鎮河南。莊帝嘉之,除都督、豫州刺史。行達汲郡,聞爾朱兆入洛,乃度河見仲遠。仲遠遣鎮汲郡。兆徵子鵠赴洛,既見,責以乖異之意,奪其部眾,將還晉陽。元曄以為侍中、御史中尉、中軍大都督。太昌初,兼尚書左僕射、東南道大行臺,總大都督杜德等追討爾朱仲遠。仲遠奔梁,收其兵馬。時梁遣
 元樹入寇,陷據譙城,詔子鵠與德討之。樹大敗,奔入城門,遂圍之。樹請歸南,以地還魏,許之。及樹眾半出,子鵠擊破之,禽樹及梁譙州刺史朱文開。班師,遷吏部尚書,轉尚書右僕射。尋加驃騎大將軍、開府,典選。後除兗州刺史。子鵠先遣腹心,緣歷人間,採察得失。及至境,太山太守彭穆參候失儀,子鵠責讓穆,并數其罪狀,穆皆引伏,於是州內震悚。



 及孝武帝入關,子鵠據城為應,南青州刺史大野拔率眾就子鵠。天平初,齊神武遣儀同三司婁昭等討之。城久不拔,昭以水灌城。而大野拔因與相見,令左右斬子鵠以降。



 侯深,神武尖山人也。機警有膽略。孝明末年,六鎮飢亂,深隨杜洛周南寇。



 後與妻兄念賢,背洛周歸爾朱榮。路中遇寇,身披苫褐。榮賜其衣帽,厚待之,以為中軍副都督。莊帝即位,封厭次縣子。從榮討葛榮於滏口,戰功尤多。除燕州刺史。時葛榮別帥韓樓、郝長等屯據薊城,榮令深討樓,配眾甚少。或以為言,榮曰:「深臨機設變,是其所長,若總大眾,未必能用。」止給騎七百。深遂廣張軍聲,率數百騎深入樓境。去薊百餘里,遇賊帥陳周馬步萬餘,大破之,虜其卒五千餘人。



 尋還其馬仗,縱令入城。左右諫,深曰:「我兵少,不可力戰,事須為計以離隙之。」



 深度
 其已至,遂率騎夜進,昧旦叩其城門。韓樓果疑降卒為內應,遂遁走。追禽之。



 以功賜爵為侯,尋為平州刺史,仍鎮范陽。



 及爾朱榮死,太守盧文偉誘深出獵,閉門拒之。深率部曲屯於郡南,為榮舉哀,勒兵南向。莊帝使東萊王貴平為大使,慰勞燕、薊。乃詐降,貴平信之,遂執貴平自隨。進至中山,行臺僕射魏蘭根邀擊之,為深所敗。元曄立,授深儀同三司、定州刺史、左軍大都督、漁陽郡公。節閔帝立,仍加開府。後隨爾朱兆拒齊神武於廣阿,兆敗走。深後從神武破爾朱氏於韓陵。永熙初,除齊州刺史。孝武帝末,深與袞州刺史樊子鵠、青州刺史東萊王
 貴平使信往來,以相連結。又遣使通誠於神武。



 及孝武入關,復懷顧望。汝陽五暹既除齊州刺史,深不時迎納。城人劉桃符等潛引暹入,據西城。深爭門不克,率騎出奔,妻兒部曲,為暹所虜。行達廣里,會承制以深行青州事,齊神武又遺書深曰:「卿勿以部曲輕少,難於東邁。齊人澆薄,齊州人尚能迎汝陽王,青州人豈不能開門待卿也?」深乃復還,暹始歸其部曲。而貴平自以斛斯椿黨,亦不受代。深襲高陽郡,克之,置部曲家累於城中,親率輕騎,夜趣青州,城人執貴平出降。深自惟反覆,慮不獲安,遂斬貴平,傳首于鄴,明不同於斛斯椿。及子鵠平,詔
 以封延之為青州刺史。深既不獲州任,情又恐懼。行達廣川,遂劫光州庫軍反。遣騎詣平原,執前膠州刺史賈璐,夜襲青州南郭,劫前廷尉卿崔光韶以惑人情,攻掠郡縣。其部下督帥叛拒之,遂奔梁。達南青州境,為賣漿者斬之,傳首于鄴,家口配沒。



 賀拔允,字可泥,神武尖山人也。其先與魏氏同出陰山,有如回者,魏初為大莫弗。祖爾頭,驍勇絕倫,以良家鎮武川,因家焉。獻文時以功賜爵龍城縣男,為本鎮軍主。父度拔,性果毅,襲爵,亦為本鎮軍主。正光末,沃野人破六韓拔陵反,懷朔鎮將楊鈞聞度拔名,召補統軍,配以
 一旅。其賊偽署王衛可環徒黨尤盛,既攻沒武川,又陷懷朔,度拔父子並為賊所虜。度拔乃與周德皇帝合謀,率州里豪傑珍、念賢、乙弗庫根、尉遲檀等,招義勇,襲殺可環。朝廷嘉之。未及封賞,度拔與鐵勒戰沒。孝昌中,追贈度拔肆州刺史。允便弓馬,頗有膽略。初度拔之死,允兄弟俱奔恒州刺史廣陽王深。深敗,歸爾朱榮。允父子兄弟並以武藝稱,榮素聞其名,待之甚厚。建義初,封壽陽縣侯。永安中,進爵為公。魏長廣王立,除開府儀同三司,封燕郡王,兼侍中,使蠕蠕。還至晉陽,屬神武將出山東,允素知神武非常人,早自結託;神武以其北土之
 望,尤親禮之。遂與允出信都,參定大策。中興初,轉司徒,領尚書令。神武入洛,進爵為王,轉太尉,加侍中。魏孝武既忌神武,以允弟岳據關中,有重兵,深相委託,潛使來往,當是咸慮允為變。及岳死,孝武又委岳兄勝心腹之寄。神武重舊,尤全護之。天平元年,因與神武獵,或告允引弓擬神武,乃置於樓上餓殺之,年四十八。神武親臨哭之,贈太保。



 允三子:世文、世樂、難陀。興和末,齊神武並召與諸子同學。武定中,敕居定州,賜田宅。允弟勝。



 勝字破胡,少有志操,善左右馳射,北邊莫不推其膽略。衛可瑰之圍懷朔,勝時亦為軍主,從父度拔鎮守。既被
 圍,經年而外援不至,勝乃慷慨白鎮將楊鈞,請告急於大軍。鈞許之,乃募勇敢少年,得十餘騎,夜潰圍出。賊追及之,勝曰:「我賀拔破胡也。」賊不敢逼。至朔州,白臨淮王彧以懷朔被圍之急。彧以勝辭義懇至,許以出師,還令報命。乃復攻圍而入,賊追之,射殺數人。至城下,大呼曰:「賀拔破胡與官軍至矣!」城中納之。鈞復遣勝出覘武川。武川已陷。勝乃馳還報懷朔。懷朔亦潰,勝父子遂為賊所虜。



 尋而襲殺可瑰,眾令勝馳告朔州,未反而度拔已卒。刺史費穆奇勝才略,厚禮留之,委以兵事。時廣陽王深在五原,為破六韓賊所圍,召勝為軍主。以功拜統軍。



 又隸僕射元纂鎮恒州。時有鮮于河胡擁朔州流人南下為寇,恒州城人應之。勝與兄允弟岳相失,勝南投肆州,允、岳投爾朱榮。榮與肆州刺史尉慶賓構隙,引岳攻肆州,陷。榮得勝,大悅曰:「吾得卿兄弟,天下不足定。」勝兄弟三人,遂委質事榮。



 時杜洛周據幽、定,葛榮據冀、瀛。榮謂勝曰:「並陘險要,我之東門,欲屈君鎮之,如何?」勝曰:「是所願也。」榮乃表勝鎮井陘,以所乘大馬並銀鞍遺之。



 及榮入洛,以預定策立孝莊帝功,封易陽縣伯。後元天穆北征葛榮,大破之。時杜洛周餘燼韓樓在薊城結聚,以勝為大都督,鎮中山,樓讋勝威名,竟不敢南寇。元顥入
 洛陽,榮徵勝,使與爾朱兆自硤石度,大破顥軍,禽其子冠受,遂前驅入洛。



 進爵真定縣公。及榮死,勝與田怡等奔赴榮第,時宮殿之門未加嚴防,怡等議即攻門。勝止之曰:「天子既行大事,必當更有奇謀,吾眾旅不多,何輕爾!」怡乃止。



 及世隆夜走,勝隨至河橋。勝以為臣無仇君之義,遂勒所部還都。莊帝大悅。仲遠逼東郡,詔以本官假驃騎大將軍、東征都督,率騎一千,會鄭先護討之,為先護所疑,置之營外,人馬未得休息。俄而仲遠兵至,與戰不利,降之。復與爾朱氏同謀立節閔帝,以功拜右衛將軍。及爾朱氏將討齊神武,勝時從爾朱度律。度律與
 兆不平。勝以臨敵構隙,取敗之道,乃與斛斯椿詣兆營和之,反為兆所執。度律大懼,引軍還。兆將斬勝,數之曰:「爾殺可環,罪一也;天柱薨後,不與世隆等俱來而東征仲遠,罪二也。我欲殺爾久矣!」勝曰:「可環作逆,勝父子誅之,其功不小,反以為罪。天柱被戮,以君誅臣,勝寧負王,不負朝廷。今日之事,生死在王。但去賊密邇,內構嫌隙,自古迄今,未有不破亡者。勝不憚死,恐王失策。」兆乃捨之。勝既免,行百餘里,方追及度律。齊神武既克相州,兵威漸盛,於是兆及天光、仲遠、度律等眾十餘萬陣於韓陵。兆率鐵騎陷陣,出齊神武後,將乘其背而擊之。



 度律
 惡兆之驍悍,懼其陵己,勒兵不進。勝以其攜貳,遂以麾下降齊神武。度律軍以此先退,遂大敗。



 太昌初,以勝為領軍將軍,尋除侍中。孝武帝將圖齊神武,以勝弟岳擁眾關西,欲廣其勢援,乃拜都督、荊州刺史、驃騎大將軍、開府儀同三司、南道大行臺、尚書左僕射。勝多所克捷,沔北盪為丘墟。梁武帝敕其子雍州刺史續曰:「賀拔勝北間驍將,爾宜慎之,勿與爭鋒。」續遂城守不敢出。尋進位尚書令,進爵瑯邪郡公。



 及齊神武與孝武帝有隙,詔勝引兵赴洛,至廣州,猶豫未進,而帝已入關。勝還軍南陽,遣右丞楊休之奉表入關,又令府長史元穎行州事,
 勝自率所部,將西赴關中。進至浙陽,詔授勝太保、錄尚書事。聞齊神武已平潼關,禽毛鴻賓,勝乃還荊州。州人鄧誕執元穎,引齊師。時齊神武已遣行臺侯景、大都督高敖曹赴之,勝敗,中流矢,奔梁。在南三年,梁武帝遇之甚厚。勝乞師北討齊神武,既不果,乃求還。梁武帝許之,親餞於南苑。勝自是之後,每執弓矢,見鳥獸南向者,皆不射之,以申懷德之意。既至長安,詣闕謝罪。魏帝握勝手,噓欷久之,曰:「初平西徙,永嘉南度,漢、晉皆爾。事乃關天,非公之咎也。」乃授太師。從周文帝禽竇泰於小關。攻弘農。下河北,禽郡守孫晏。摧破東魏軍於沙苑,奔追至
 河上。仍與李弼別攻河東,略定汾、絳。河橋之役,勝大破東魏軍,周文令勝收其降卒而還。



 及齊神武率眾攻玉壁,勝以前軍大都督從周文。見齊武旗鼓,識之,乃募敢勇三千人,配勝以犯其軍。勝適與神武遇,連叱而字之曰:「賀六渾,賀拔破胡必殺汝也!」



 時勝持槊追神武數里,刃垂及之,神武汗流,氣殆盡。會勝馬為流矢所中,死。比副騎至,神武已逸去。勝歎曰:「今日之事,吾不執弓矢者,天也!」



 是歲,勝諸子在東者,皆為神武所害。勝憤恨,因動氣疾,大統十年,薨于位。



 臨終,手書與周文曰:「勝萬里杖策,歸身闕庭,冀望與公掃除逋寇。不幸殞斃,微志不申。
 若死而有知,猶望魂飛賊庭,以報恩遇耳。」周文覽書,流涕久之。勝長於喪亂之中,尤工武藝,走馬射飛鳥,十中其五六。周文每云:「諸將對敵,神色皆動,唯賀拔公臨陣如平常,真大勇也。」自居重任,始愛墳籍,乃招引文儒,討論義理。性又通率,重義輕財,身死之日,唯有隨身兵仗及書千卷而已。



 初,勝至關中,自以年位素重,見周文不拜。尋而自悔,周文亦有望焉。後從宴昆明池,時有雙鳧游池中,周文授弓矢於勝曰:「不見公射久矣,請以為歡。」



 勝射之,一發俱中。因拜曰:「使勝得奉神武,以討不庭,皆如此也。」周文悅,因是恩禮日重,勝亦盡誠推奉焉。贈太
 宰、錄尚書事,謚曰貞獻。明帝二年,以勝配饗文帝廟庭。



 無子,以弟岳子仲華嗣。位開府儀同三司,襲爵瑯邪公。大象末,位江陵總管。



 勝弟岳。



 岳字阿斗泥,少有大志,愛施好士。初為太學生。及長,能左右馳射,驍果絕人。不讀兵書,而暗與之合,識者咸異之。與父兄赴援懷朔,賊王衛可環在城西三百餘步,岳乘城射之,箭中環臂,賊大駭。後廣陽王深以為帳內軍主,與兄勝俱鎮恒州。州陷,投爾朱榮,榮以為都督。每帳下與計事,多與榮意合。榮與元天穆謀入匡朝廷,問計於岳。岳曰:「夫非常之事,必俟非常之人。將軍士馬精強,
 位望隆重,若首舉義旗,伐叛匡救,何往不克,何向不摧!古人云:『朝謀不及夕,言發不俟駕。』此之謂矣。」榮與天穆相顧良久,曰:「卿此言,真丈夫之論也。」



 未幾,孝明帝暴崩,榮疑有故,乃舉兵赴洛。配岳甲卒二千為先驅。至河陰,榮既殺朝士,因欲稱帝,疑未能決。岳乃從容致諫,榮尋亦自悟,乃尊立孝莊。以定策功,賜爵樊城鄉男。從榮破葛榮,平元顥,累遷左光祿大夫、武衛將軍。時萬俟醜奴僭稱大號,關中騷動,榮將遣岳討之。私謂其兄勝曰:「醜奴足為勍敵,若岳往無功,罪責立至;假令克定,恐讒愬生焉。」乃請爾朱氏一人為元帥,岳副貳之。榮大悅,乃以
 天光為使持節、大都督、雍州刺史,以岳為左廂大都督,又以征西將軍侯莫陳悅為右廂大都督,並為天光之副,以討之。時赤水蜀賊兵斷路,天光眾不滿二千。及軍次潼關,天光有難色,岳乃進破之於渭北,軍容大振。



 時醜奴自圍岐州,遣其大行臺尉遲菩薩、僕射萬俟行醜同向武功,南度渭水,攻圍趨柵。天光遣岳率千騎赴援。菩薩攻柵已克,率步騎二萬至渭北。岳以輕騎數十,與菩薩隔水交言。岳稱揚國威,菩薩乃自驕,令省事傳語。省事恃水,應答不遜。岳怒,舉弓射之,應弦而倒。時已逼暮,於是各還。岳於渭南傍水,分精兵數十為一處,隨地
 形勢置之。明日,將百餘騎,隔水與賊相見,且並東行。岳漸前進,先所置騎,隨岳而集,騎既漸增,賊不復測其多少。行二十許里,至水淺可濟處,岳便馳馬東出,似欲奔遁。賊謂岳走,乃棄步兵,南度渭水,輕騎追岳。岳東行十餘里,依橫岡設伏兵以待之,身先士卒,急擊之,賊便退走。岳號令所部,賊下馬者皆不聽殺。賊顧見之,便悉投馬。俄虜三千人。馬亦無遺,遂禽菩薩。仍度渭北,降步卒萬餘。醜奴尋棄岐州,北走安定。天光方自雍至,與岳合勢。宣言今氣候已熱,非征討之時,待至秋涼,更圖進取。醜奴聞之,遂以為實,分遣諸軍散營農於岐州北百里
 網川。使太尉侯伏侯元進據險立柵。岳知其勢分,密與天光嚴備。昧旦,攻圍元進柵,拔之,即禽元進,自餘諸柵悉降。又輕騎追醜奴,及之於平涼之長坑,一戰禽之。高平城中又執蕭寶夤以歸。



 賊行臺萬俟道洛退保牽屯,岳攻之。道洛敗入隴,投略陽賊帥王慶雲。以道洛驍果絕倫,得之甚喜,以為將。天光又與岳度隴,至慶雲所居永洛城。慶雲、道洛頻出城拒戰,並禽之,餘眾皆悉坑之。三秦、河、渭、瓜、涼、鄯州咸來歸款。賊帥夏州人宿勤明達降復叛,岳又討禽之。天光雖為元帥,而岳功效居多,進封樊城縣伯。尋詔岳都督、涇州刺史,進爵為公。天光入
 洛,使岳行雍州事。普泰初,除都督、岐州刺史,進清水郡公,尋加侍中,給後部鼓吹。進位開府儀同三司兼尚書左僕射、隴右行臺,仍停高平。後以隴中猶有土人不順,岳助侯莫陳悅,所在討平之。二年,加都督、雍州刺史。天光將拒齊神武,遣問計於岳。岳曰:「莫若且鎮關中,以固根本。」天光不從,後果敗。岳率軍下隴赴雍,禽天光弟顯壽以應齊神武。



 及孝武即位,加關中大行臺。永熙二年,孝武密令岳圖齊神武,遂刺心血,持以寄岳。岳懼,乃自詣北境,安置邊防,率眾趨平涼西界,布營數十里,託以牧馬於原州,為自安之計。先是,費也頭萬俟受洛干、鐵
 勒斛律沙門、解拔彌俄突、紇豆陵伊利等擁眾自守,至是皆款附。秦、南秦、河、渭四州刺史又會平涼,受岳節度。唯靈州刺史曹泥不應召,通使於齊神武。神武乃遣左丞翟嵩使至關中,間岳及侯莫陣悅。三年,岳召悅會於高平,將討曹泥,令悅前驅,而悅受神武指,密圖岳。



 岳弗之知,而先又輕悅,悅乃誘岳入營,共論兵事。悅詐云腹痛,起而徐行,令其婿元洪景斬岳於幕中。朝野莫不痛惜之。贈侍中、太傅、錄尚書事、都督關中二十州諸軍事、大將軍、雍州刺史,謚曰武莊。翟嵩復命于神武,神武下床鳴其頰曰:「除吾病者,卿也,何日忘之!」後岳部下收岳
 尸,葬於雍州北石安原,葬以王禮。



 子緯嗣,拜開府儀同三司。周保定中,錄岳舊德,進爵霍國公,尚周文帝女。



 侯莫陳悅,代人也。父婆羅門為駝牛都尉,故悅長於河西。好田獵,便騎射,會牧子作亂,遂歸爾朱榮。榮引為府長流參軍。莊帝初,除金紫光祿大夫,封柏人縣侯。爾朱天光之討關西,榮以悅為天光右廂大都督。西伐克獲,皆與天光、賀拔岳略同。除鄯州刺史。爾朱榮死後,亦隨天光下隴。元曄立,進爵為公,改封白水郡公。普泰中,除秦州刺史。天光之東出,將抗齊神武,悅與岳下隴以應神武,至雍州,會爾朱覆敗。永熙初,加開府儀同三司、都
 督隴右諸軍事,仍兼秦州刺史。



 三年,岳召悅共討曹泥,悅誘岳斬之。岳左右奔散,悅遣人安慰,眾皆畏服。悅心猶豫,不即撫納,乃還入隴,止永洛城。岳所部聚於平涼,規還圖悅。周文帝時為夏州刺史,眾遣奉迎。周文至,遂總岳部眾并家口入高平城,以自安固。乃勒眾入隴征悅。悅聞之,棄城南據山水之險。悅先召南秦州刺史李景和。其夜景和遣人詣周文,密許翻降。至暮,景和乃勒其所部,使上驢駝,云:「儀同有教,欲還秦州,守以拒賊。」復給帳下云:「儀同欲還秦州,汝等何不裝辦?」眾謂言實,以次相驚,皆散趣秦州。景和先馳至城,據門以慰輯之。悅
 部眾離散,猜畏傍人,不聽左右近己。與其二弟井兒及謀殺岳者八九人,棄軍迸走,數日之中,盤回往來,不知所趣。左右勸向靈州,而悅不決。言下隴後恐為人見,乃放馬山中,令從者悉步,自乘一騾,欲往靈州。中路追騎將及,縊死野中。弟息部下,悉見禽殺。唯先謀殺岳者悅中兵參軍豆盧光,走至靈州,後奔晉陽。悅自殺岳後,精神恍惚,不復如常。



 恒言:「我睡即夢岳語我『兄欲何處去?』隨逐我不相置。」因此彌不自安,而致敗滅。



 念賢,字蓋盧,金城枹罕人也。父求就,以大家子戍武川鎮,仍家焉。賢美容質,頗涉經史。為兒童時,在學中讀書,
 有善相者過學,諸生競詣之。賢獨不往,笑謂諸生曰:「男兒死生富貴,皆在天也,何遽相乎!」少遭父憂,居喪有孝稱。



 後以破衛可環功,除別將,又以軍功封屯留縣伯。從爾朱榮入洛,兼尚書右僕射、東道行臺,進爵平恩縣公。永熙中,孝武以賢為中軍北向大都督,進爵安定郡公,加侍中、開府儀同三司。大統初,拜太尉,為秦州刺史,加太傅,給後部鼓吹。三年,轉太師、都督、河州刺史、大將軍。久之還朝,兼錄尚書事。後與廣陵王欣、扶風王季等同為正直侍中。時行殿初成,未有題目,帝詔近侍各名之,對者非一,莫允帝心。賢乃為「圓極」,帝笑曰:「正與朕意同。」
 即名之。河橋之役,賢不力戰,乃先還,自是名頗減。五年,除都督、秦州刺史,薨於州。謚曰昭定。賢於諸公,皆為父黨,自周文以下,咸拜敬之。



 子華,性和厚,有長者風。官至開府儀同三司、合州刺史。



 梁覽,字景睿,金城人也。其先出自安定,避難走西羌,世為部落酋帥。曾祖穆,以枹罕城歸吐谷渾,後又歸魏,封臨洮公。祖顥,為尚書,封南安公。父釗,河華二州刺史,封新陽縣伯。覽家世豪富,貲累千金。孝昌初,秦州莫折念生、胡琛等反,散財招募,有二千人,鎮河州。從大軍平賊,歷涼、河二州刺史,封安德縣侯。覽既為本州刺史,盛脩
 甲仗,人馬精銳。吐谷渾憚不敢出,皆曰:「梁公在,未可行也。」永安中,詔大鴻臚瑯邪王皓就策授世為河州刺史。永熙中,改封郡公。



 大統二年,加太尉。其年,覽從弟GC定反,欲圖覽,覽與數戰未能平,王師至,始破之。四年,遷太傅。及河橋之役,王師敗,時病留長安,趙青雀反北城,覽為之謀主。事平,乃見殺。



 子鸛雀,位儀同三司、大都督,後坐事免,死。



 雷紹,字道宗,武川鎮人也。九歲而孤。有膂力,善騎射。年十八,給事鎮府。



 嘗使洛陽。見京都禮義之美,還謂同僚曰:「徒知邊備尚武,以圖富貴;不謂文學,身之寶也。生世
 不學,其猶穴處,何所見焉?」遂逃歸,辭母求師。經年,通《孝經》、《論語》。嘗讀書至人行莫大於孝,乃投卷嘆曰:「吾離違侍養,非人子之道。」即還鄉里,躬耕奉養。遭母憂,哀毀骨立,由是知名。鎮將召補鎮佐。後隨賀拔岳征討,為岳長史。岳有大事,常訪而後行。及齊神武起兵,岳恥居其下。紹乃勸岳迎孝武西都長安,以順討逆。岳曰:「吾本意也。」後岳信諸將言,欲保關中,坐觀成敗。紹知計不用,請為邊州,建功效。岳曰:「君有毗佐之力,當總大州。」遂以紹為京兆太守。清平理物,甚得人和。在郡踰年,岳被害。初,紹見岳數與侯莫陳悅宴語,嘗謂岳曰:「公其慎之!」岳不從,
 果及於難。紹乃棄郡,馳赴岳軍,與寇洛等迎周文帝。悅平,以功授大都督、涼州刺史。紹請留所領兵以助東討,請單騎赴州。刺史李叔仁擁州逆命,紹遂歸。永熙三年,以紹為渭州刺史,進爵昌國伯。初,紹為岳長史,周文為岳左丞,及居相,常以恩舊接之。卒於州。



 紹性好施,祿賜皆分贍親故,及死日,無以送終。兼敬信佛道,遣敕其子曰:「吾本鄉葬法,必殺犬馬,於亡者無益。汝宜斷之,斂以時服,事從約儉。」還葬長安,天子素服臨弔,贈太尉,賜東園祕器。子渙。



 毛遐,字鴻遠,北地三原人也。世為酋帥。曾祖天愛,太武
 時,至定州刺史、始昌子。傳至遐,四世不絕。正光中,蕭寶夤為大都督,討關中諸賊,咸陽太守韋遂時為都督,以遐為都督府長史。寶夤敗還長安,三輔騷擾。遐因辭遂還北地,與弟鴻賓聚鄉曲豪傑,遂東西略地,氐、羌多赴之,共推鴻賓為盟主。既而賊帥宿勤買奴自號京兆王於北地,遐詐降之,而與鴻賓攻其壁。賊自相斫射,縱兵追擊,七柵皆平。後寶夤構逆謀,遐知之,乃寄書與鴻賓,索馬迎接,復於馬祗柵建旗鼓以拒寶夤,攻其將盧祖遷,禽之。寶夤以是日拜南郊,竊號。禮未畢而告敗,寶夤懼,口乾色變,不遑部伍,人皆亂還。詔授遐南幽州刺史,
 進爵為伯。遐又攻破其將侯終德。寶夤知內外勢異,輕將十數騎走巴中。冬,萬俟醜奴陷秦州,詔以遐兼尚書,二州行臺。孝武帝入關,敕周文帝置二尚書,分掌機事,遐與周惠達始為之。稍遷驃騎大將軍、儀同三司,卒。



 遐少任俠,有智謀。世為豪右,貲產巨億,士流貧乏者,多被賑贍。故中書郎檀翥、尚書郎公孫範等,常依託之。至於自供衣食,粗弊而已。死之日,鄉黨赴葬,咸共痛惜。



 鴻賓大鼻眼,多鬢鬚,黑而且肥,狀貌頗異,氐、羌見者皆畏之。加膽略騎射,俶儻不拘小節,昆季之中,尤輕財好施。遐雖云早立,而名出其下。及賊起,鄉里推為盟主,常
 與遐一守一戰。後拜岐州刺史、散騎常侍、開國縣侯。遐笑謂鴻賓曰:「擊賊之功,吾不居汝後,至於受賞,汝在吾前,當以德濟物,不及汝故。」



 明帝以鴻賓兄弟所定處多,乃改北地郡為北雍州,鴻賓為刺史。詔曰:「此以晝錦榮卿也。」改三原縣為建中郡,以旌其兄弟。後爾朱天光自關中還洛,夷夏心所忌者,皆將自隨。鴻賓亦領鄉中壯武二千人以從。洛中素聞其名,衣冠貧冗者,競與之交。尋拜西袞州刺史。羈寓倦游之輩,四座常滿,鴻賓資給衣食,與己悉同。私物不足,頗有公費。轉南青州刺史。未幾,徵還,為有司所糾,鴻賓遂逃匿人間。



 月餘,特詔原之。



 及孝武帝與齊神武有隙,令鴻賓鎮潼關,為西道之寄。車駕西幸,漿糗乏絕,侍官三二日間,唯飲澗水。鴻賓奉獻酒食,迎於稠桑,文武從者,始解飢渴。武帝把其手曰:「寒松勁草,所望於卿也。事平之日,寧忘主人。」仍留守潼關。後神武來寇,見禽至并州,憂恚卒。



 鴻賓弟鴻顯,位散騎常侍,封縣侯。遐乳母所產也,一字七寶。遐養之為弟,因姓毛氏。勁悍多力,後隨諸兄戰鬥,多先鋒陷陣。大統四年,為廣州刺史,與駱超鎮東陽,陷東魏。卒。子野叉。



 乙弗朗,字通照,其先東部人也。世為部落大人,與魏徙代,後因家上樂焉。



 朗少有俠氣,在鄉里以善騎射稱。孝
 莊末,北邊擾亂,避地居並、肆間。爾朱榮見而重之,甚相接待,以功封連勺子。後隸賀拔岳,從爾朱天光西討,為岳左廂都督。



 孝武帝之御齊神武,授朗閣內大都督。及帝西入,詔朗為軍司,先驅靖路。至長安,封長安縣公。卒於岐州刺史。



 初,朗患積冷,周文賜三石東生散,令朗法服之,使人問疾,朝夕相繼,見重如此。臨終惟云:「恨不見河、洛清平,重反京縣」,以此為恨,三舉手搥床,而便氣盡。贈太尉。



 子鳳,位宮伯、開府儀同三司。與周閔帝謀宇文護,見殺。



 論曰:朱瑞以向義受戮,延慶以違順遇禍,各其命焉。斛
 斯椿屢踐危機,終獲貞吉,豈人謀之所致也?徵洽聞強記,以夔、襄任己,終使《咸》、《英》不墜,《韶》、《濩》惟新。加以盡心所事,無忘直道,抗辭正色,顛沛不渝,蓋有周之忠烈乎?賈顯智、樊子鵠、侯深等並驅馳風塵之際,但自陷夷戮。觀其遺跡,雖獲罪於霸政,求之有魏,得失未可知也。賀拔允昆季以勇略之資,當馳競之日,並邀時投隙,展效立功。始則委質爾朱,中乃結款高氏,太昌之後,即帝圖高。察其所由,固非守節之士。及勝垂翅江左,憂魏室之危亡;奮翼關西,感梁朝之顧遇,有長者之風矣。終能保榮持寵,良有以焉。岳以二千羸兵,抗三秦勍敵,奮其智勇,
 克翦兇渠,雜種畏威,遐方慕義,斯亦一時之盛矣。卒以勛高速禍,無備嬰戮,惜哉!昔陳涉首事不終,有漢因而創業;賀拔功成夙殞,周文籍以開基。不有所廢,君何以興?信乎其然矣。侯莫陳悅肆行殘慝,死不旋踵,觀其亡滅,蓋自取之。念賢有始有卒,取敬群公。梁覽終以取禍,鮮克之義。雷紹馳騖雲雷之秋,毛遐兄弟致力經綸之日,乙弗朗展轉擾攘之中,卒獲歸順,美矣!



\end{pinyinscope}