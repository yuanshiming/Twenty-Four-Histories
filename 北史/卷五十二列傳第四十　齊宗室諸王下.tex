\article{卷五十二列傳第四十 齊宗室諸王下}

\begin{pinyinscope}

 文襄諸子文宣諸子孝昭諸子武成諸子後主諸子文襄六男:文敬元皇后生
 河
 間王孝琬;宋氏生河南王孝瑜;王氏生廣寧王孝珩;蘭陵王長恭不得母氏姓;陳氏生安德王延宗;燕氏生漁陽王紹信。



 河南康獻王孝瑜,字正德,文襄長子也。初封河南郡公,齊受禪,進爵為王。



 歷位中書令、司州牧。初,孝瑜養於神武宮中,與武成同年相愛。將誅楊愔等,孝瑜預其謀。及武成即位,禮遇特隆。帝在晉陽手敕之曰:「吾飲汾清二盃,勸汝於鄴酌兩盃。」其親愛如此。



 孝瑜容貌魁偉,精彩雄毅,謙慎寬厚,兼愛文學,讀書敏速,十行俱下,覆棋不失一道。初,文襄於鄴東起山池游觀,時俗眩之,孝瑜遂於第作水堂龍舟,植幡槊於舟上,數集諸弟,宴射為樂。武成幸其第,見而悅之,故盛興後園之玩。於是貴賤慕斅,處處營造。



 武成嘗使和士開與胡后對坐握槊,孝瑜
 諫曰:「皇后天下之母,不可與臣下接手。」帝深納之。後又言趙郡王父死非命,不可而親。由是睿及士開皆側目。士開密告其奢僭,睿又言山東唯聞河南王,不聞有陛下。帝由是忌之。爾朱御女名摩女,本事太后,孝瑜先與之通,後因太子婚夜,孝瑜竊與之言。武成大怒,頓飲其酒三十七盃。體至肥大,腰帶十圍,使婁子彥載以出,鴆之於車。至西華門,煩熱躁悶,投水而絕。贈太尉、錄尚書事。子弘節嗣。



 孝瑜母,魏吏部尚書宋弁孫也。本魏潁川王斌之妃,為文襄所納,生孝瑜。孝瑜還第,為太妃。孝瑜妃盧正山女,武成胡后之內姊也。孝瑜薨後,宋太妃為
 盧妃所譖訴,武成殺之。



 廣寧王孝珩,文襄第二子也。歷位司州牧、尚書令、司空、司徒、錄尚書、大將軍、大司馬。孝珩愛賞人物,學涉經史,好綴文,有技藝。嘗於事壁自畫一蒼鷹,見者皆以為真。又作朝士圖,亦當時之妙絕。



 後主自晉州敗,奔鄴,詔王公議於含光殿。孝珩以大敵既深,事藉機變,宜使任城王領幽州道兵入土門,揚聲趣并州;獨孤永業領洛州道兵趣潼關,揚聲取長安。



 臣請領京畿兵出滏口,鼓行逆戰。敵聞南北有兵,自然潰散。又請出宮人寶物賞將士,帝不能用。



 承光即位,以孝珩為太宰,與呼延族、莫
 多婁敬顯、尉相願同謀,其正月五日,孝珩於千秋門斬高阿那肱。相願在內,以禁兵應之,族與敬顯自游豫園勒兵出。既而阿那肱從別宅取便路入宮,事不果。乃求出拒西軍,謂阿那肱、韓長鸞、陳德信等云:「朝廷不賜遣擊賊,豈不畏孝珩反邪?破宇文邕遂至長安,反時何與國家事?



 以今日之急,猶作如此猜!」高、韓恐其變,出孝珩為滄州刺史。至州,以五千人會任城王於信都,共為匡復計。周齊王憲來伐,兵弱不能敵。怒曰:「由高阿那肱小人,吾道窮矣!」齊叛臣乞扶令和以槊刺孝珩墜馬,奴白澤以身扦之,孝珩猶傷數處,遂見虜。



 齊王憲問孝珩齊亡
 所由,孝珩自陳國難,辭淚俱下,俯仰有節。憲為之改容,親為洗瘡傅藥,禮遇甚厚。孝珩獨歎曰:「李穆叔言齊氏二十八年,今果然矣!自神武皇帝以外,吾諸父兄弟無一人得至四十者,命也。嗣君無獨見之明,宰相非柱石之寄,恨不得握兵符,受廟算,展我心力耳。」至長安,依例受開府、縣侯。



 後周武帝在雲陽宴齊君臣,自彈胡琵琶,命孝珩吹笛。辭曰:「亡國之音,不足聽也。」固命之,舉笛裁至口,淚下嗚咽,武帝乃止。其年十月疾甚,啟歸葬山東,從之。尋卒,還葬鄴。



 河間王孝琬,文襄第三子也。天保元年封。天統中,累遷
 尚書令。初,突厥與周師入太原,武成將避之而東,孝琬叩馬諫,請委趙郡王部分之,必整齊。帝從其言。孝琬免胄將出,帝使追還之。周軍退,拜并州刺史。孝琬以文襄世嫡,驕矜自負。河南王之死,諸王在宮內,莫敢舉聲,唯孝琬大哭而出。又怨執政,為草人而射之。和士開與祖珽譖之云:「草人擬聖躬也。又前突厥至州,孝琬脫兜鍪抵地云:『豈是老嫗,須著此!』此言屬大家也。」初魏世謠言:「河南種穀河北生,白楊樹頭金雞嗚。」珽以說曰:「河南河北,河間也;金雞嗚,孝琬將建金雞而大赦。」



 帝頗惑之。



 時孝琬得佛牙,置於第內,夜有神光。照玄都法順請以奏,
 不從。帝聞,使搜之,得填庫槊幡數百。帝聞,以為反狀。訊其諸姬,有陳氏者,無寵,誣對曰:「孝琬畫作陛下形哭之。」然實是文襄像,孝琬時時對之泣。帝怒,使武衛赫連輔玄倒鞭撾之。孝琬呼阿叔。帝怒曰:「誰是爾叔?敢喚我作叔!」孝琬曰:「神武皇帝嫡孫,文襄皇帝嫡子,魏孝靜皇帝外甥,何為不得喚作叔也?」帝愈怒,折其兩脛而死。瘞諸西山,帝崩後乃改葬。



 子正禮嗣。幼聰穎,能誦《左氏春秋》。齊亡,遷綿州卒。



 蘭陵武王長恭,一名孝瓘,文襄第四子也。累遷并州刺史。突厥入晉陽,長恭盡力擊之。芒山之敗,長恭為中軍,
 率五百騎再入周軍,遂至金墉之下,被圍甚急。



 城上人弗識,長恭免胄示之面,乃下弩手救之,於是大捷。武士共歌謠之,為《蘭陵王入陣曲》是也。歷司州牧、青瀛二州,頗受財貨。後為太尉。與段韶討柏谷,又攻定陽。韶病,長恭總其眾。前後以戰功,別封鉅鹿、長樂、樂平、高陽等郡公。



 芒山之捷,後主謂長恭曰:「入陣太深,失利悔無所及。」對曰:「家事親切,不覺遂然。」帝嫌其稱家事,遂忌之。及在定陽,其屬尉相願謂曰:「王既受朝寄,何得如此貪殘?」長恭未答。相願曰:「豈不由芒山大捷,恐以威武見忌,欲自穢乎?」長恭曰:「然。」相願曰:「朝廷若忌王,於此犯便當行罰,
 求福反以速禍。」



 長恭泣下,前膝請以安身之術。相願曰:「王前既有勛,今復告捷,威聲大重,宜屬疾在家,勿預時事。」長恭然其言,未能退。及江淮寇擾,恐復為將,歎曰:「我去年面腫,今何不發?」自是有疾不療。武平四年五月,帝使徐之範飲以毒藥。



 長恭謂妃鄭氏曰:「我忠以事上,何辜於天而遭鴆也?」妃曰:「何不求見天顏?」



 長恭曰:「天顏何由可見!」遂飲藥而薨。贈太尉。



 長恭貌柔心壯,音容兼美。為將,躬勤細事。每得甘美,雖一瓜數果必與將士共之。初在瀛州,行參軍陽士深表列其贓,免官。及討定陽,士深在軍,恐禍及。



 長恭聞之曰:「吾本無此意。」乃求小失,杖
 深二十,以安之。嘗入朝而出,僕從盡散,唯有一人。長恭獨還,無所譴罰。武成賞其功,命賈護為買妾二十人,唯受其一。有千金責券,臨死悉燔之。



 安德王延宗,文襄第五子也。母陳氏,廣陽王妓也。延宗幼為文宣所養。年十二,猶騎置腹上,令溺己臍中。抱之曰:「可憐,止有此一個。」問欲作何王,對曰:「欲作衝天王。」文宣問楊愔,愔曰:「天下無此郡名,願使安於德。」於是封安德焉。為定州刺史。於樓上大便,使人在下,張口承之。以蒸豬糝和人糞以飼左右,有難色者鞭之。孝昭帝聞之,使趙道德就州杖之一百。道德以延宗受杖不謹,又加
 三十。又以囚試刀,驗其利鈍。驕縱多不法。武成使撻之,殺其暱近九人,從是深自改悔。



 蘭陵王芒山凱捷,自陳兵勢,諸兄弟咸壯之。延宗獨曰:「四兄非大丈夫,何不乘勝徑入?使延宗當此勢,關西豈得復存!」及蘭陵死,妃鄭氏以頸珠施佛,廣寧王使贖之,延宗手書以諫,而淚滿紙。河間死,延宗哭之,淚赤。又為草人以像武成,鞭而訊之曰:「何故殺我兄!」奴告之,武成覆臥延宗於地,馬鞭撾之二百,幾死。後歷司徒、太尉。



 及平陽之役,後主自禦之,命延宗率右軍,先戰城下,禽周開府宗挺。及大戰,延宗以麾下再入,周軍莫不披靡。諸軍敗,延宗獨全軍。後主
 將奔晉陽,延宗言:「大家但在營莫動,以兵馬付臣,臣能破之。」帝不納。及至并州,又聞周軍已入鸑鼠谷。乃以延宗為相國、并州刺史,總山西兵事。謂曰:「并州阿兄取,兒今去也。」延宗曰:「陛下為社稷莫動,臣為陛下出死力戰。」駱提婆曰:「至尊計已成,王不得輒沮。」後主竟奔鄴。



 在并將卒咸請曰:「王若不作天子,諸人實不能與王出死力。」延宗不得已,即皇帝位。下詔曰:「武平孱弱,政由宦豎,釁結蕭墻,盜起疆場。斬關夜遁,莫知所之,則我高祖之業,將墜於地。王公卿士,猥見推逼,今便祗承寶位,可大赦天下。」改武平七年為德昌元年,以晉昌王唐邕為宰輔,
 齊昌王莫多婁敬顯、沐陽王和阿於子、右衛大將軍段暢、武衛將軍相里僧伽、開府韓骨胡、侯莫陳洛州為爪牙。眾聞之,不召而至者前後相屬。延宗容貌充壯,坐則仰,偃則伏,人皆笑之。



 及是,赫然奮發,氣力絕異,馳騁行陣,勁捷若飛。傾府藏及後宮美女以賜將士,籍沒內參千餘家。後主謂近臣曰:「我寧使周得并州,不欲安德得之!」左右曰:「理然。」延宗見士卒,皆親執手陳辭,自稱名,流涕嗚噎。眾皆爭為死,童兒女子亦乘屋攘袂,投磚石以禦周軍。特進、開府那盧安生守太谷,以萬兵叛。周軍圍晉陽,望之如黑雲四合。延宗命莫多婁敬顯、韓骨胡拒
 城南;和阿于子、段暢拒城東;延宗親當周齊王於城北,奮大槊往來督戰,所向無前。尚書令史沮山亦肥大多力,捉長刀步從,殺傷甚多。武衛蘭芙蓉、綦連延長皆死於陣。和阿於子、段暢以千騎投周軍,周軍攻東門,際昏遂入。進兵焚佛寺門屋,飛焰照天地。延宗與敬顯自門入,夾擊之,周軍大亂,爭門相填。齊人後斫刺,死者一千餘人。周武帝左右略盡,自拔無路。承御上士張壽輒牽馬頭,賀拔佛恩以鞭拂其後,以崎嶇僅得出。



 齊人奮擊,幾中馬。城東厄曲,佛恩及降者皮子信為之導,僅免。時四更也。延宗謂周武帝崩於亂兵,使於積屍中求長鬣
 者,不得。時齊人既勝,入坊飲酒,盡醉臥,延宗不復能整。周武帝出城,飢甚,欲為遁逸計。齊王憲及柱國王誼諫,以為去必不免。延宗叛將段暢亦盛言城內空虛。周武帝乃駐馬,鳴角收兵,俄傾復振。詰旦,還攻東門,克之。又入南門。延宗戰,力屈,走至城北,於人家見禽。周武帝自投下馬,執其手。延宗辭曰:「死人手何敢迫至尊!」帝曰:「兩國天子,有何怨惡?



 直為百姓來耳!勿怖,終不相害。」使復衣帽,禮之。



 先是,高都郡有山焉,絕壁臨水,忽有墨書云:「齊亡延宗。」洗視,逾明。



 帝使人就寫,使者改亡為上。至是應焉。延宗敗前,在鄴聽事,以十二月十三日晡時受敕
 守并州,明日建尊號。不間日而被圍,經宿,至食時而敗。年號德昌。好事者言其得二日云。既而周武帝問取鄴計,辭曰:「亡國大夫不可以圖存,此非臣所及。」強問之,乃曰:「若任城王援鄴,臣不能知;若今主自守,陛下兵不血刃。」



 及至長安,周武與齊君臣飲酒,令後主起舞。延宗悲不自持,屢欲仰藥自裁,侍婢苦執諫而止。未幾,周武誣後主及延宗等,云遙應穆提婆反,使並賜死。皆自陳無之,延宗攘袂,泣而不言。以椒塞口而死。明年,李妃收殯之。



 後主之傳位於太子也,孫正言竊謂人曰:「我昔武定中為廣州士曹,聞襄城人曹普演有言:高王諸兒,阿保
 當為天子,至高德之承之,當滅。阿保謂天保,德之謂德昌也,承之謂後主年號承元,其言竟信云。」



 漁陽王紹信,文襄第六子也。歷特進、開府、中領軍、護軍、青州刺史。行過漁陽,與大富人鐘長命同床坐,太守鄭道蓋來謁,長命欲起,紹信不聽曰:「此何物小人,主人公為起!」乃與長命結為義兄弟,妃與長命妻為姊妹,責其闔家長幼,皆有贈賄,鐘氏因此遂貧。齊滅,死於長安。



 文宣五男,李后生廢帝及太原王紹德;馮世婦生范陽王紹義;裴嬪生西河王紹仁;顏嬪生隴西王紹廉。



 太原王紹德,文宣第二子也。天保末,為開府儀同三司。
 武成因怒李后,罵紹德曰:「爾父打我時,竟不來救。」以刀環築殺之,親以土埋之游豫園。



 武平元年,詔以范陽王子辨才為後,襲太原王。



 范陽王紹義,文宣第三子也。初封廣陽,徙封范陽。歷位侍中、清都尹。好與群小同飲,擅致內參打殺博士任方榮。武成嘗杖之二百,送付昭信后,后又杖一百。



 及後主奔鄴,以紹義為尚書令、定州刺史。周武帝克并州,以封輔相為北朔州總管。



 此地齊之重鎮,諸勇士多聚焉。前長史趙穆、司馬王當萬等謀執輔相,迎任城王於瀛州。事不果,迎紹義。紹義至馬邑。輔相及其屬韓阿各奴等
 數十人,皆齊叛臣,自肆州以北城戍二百八十餘,盡從輔相,及紹義至,皆反焉。紹義與靈州刺史袁洪猛引兵南出,欲取并州。至新興而肆州已為周守,前隊二儀同以所部降周。周兵擊顯州,執刺史陸瓊,又攻陷諸城。紹義還保北朔。周將宇文神舉軍逼馬邑,紹義遣杜明達拒之,兵大敗。紹義曰:「有死而已,不能降人。」遂奔突厥。眾三千家,令之曰:「欲還者任意。」於是哭拜別者大半。



 突厥他缽可汗謂文宣為英雄天子,以紹義重踝似之,甚見愛重。凡齊人在北者,悉隸紹義。高寶寧在營州,表上尊號,紹義遂即皇帝位,稱武平元年,以趙穆為天水王。他
 缽聞寶寧得平州,亦招諸部,各舉兵南向,云共立范陽王作齊帝,為其報仇。周武帝大集兵於雲陽,將親北伐,遇疾暴崩。紹義聞之,以為天贊己。盧昌期據范陽,亦表迎紹義。俄而周將宇文神舉攻滅昌期。其日,紹義適至幽州,聞周總管出兵于外,欲乘虛取薊城。列天子旌旗,登燕昭王冢,乘高望遠,部分兵眾。神舉遣大將軍宇文恩將四千人馳救幽州,半為齊軍所殺。紹義聞范陽城陷,素服舉哀,回軍入突厥。周人購之於他缽,又使賀若誼往說之。他缽又不忍,遂偽與紹義獵於南境,使誼執之,流于蜀。紹義妃,勃海封孝琬女,自突厥逃歸。紹義在
 蜀,遺妃書云:「夷狄無信,送吾於此。」竟死蜀中。



 西河王紹仁,文宣第四子也。天保末,為開府儀同三司。尋薨。



 隴西王紹廉,文宣第五子也。初封長樂,後改焉。性粗暴,嘗拔刀逐紹義,紹義走入廄,閉門拒之。紹義初為清都尹,未及理事。紹廉先往,喚囚悉出,率意決遣之。能飲酒,一舉數升,終以此薨。



 孝昭七男:元皇后生樂陵王百年;桑氏生襄城王亮,出後襄城景王;諸姬生汝南王彥理、始平王彥德、城陽王彥基、定陽王彥康、汝陽王彥忠。



 樂陵王百年,孝昭第二子也。孝昭初即位,在晉陽,群臣請建中宮及太子,帝謙未許。都下百僚又請,乃稱太后令,立為皇太子。帝臨崩,遺詔傳位於武成,并有手書。其末曰:「百年無罪,汝可以樂處置之,勿學前人。」大寧中,封樂陵王。



 河清三年五月,白虹圍日再重,又橫貫而不達;赤星見,帝以盆水承星影而蓋之,一夜盆自破。欲以百年厭之。會博陵人賈德胄教百年書,百年嘗作數敕字,德胄封以奏。帝發怒,使召百年。百年被召,自知不免,割帶玦,留與妃斛律氏。見帝於玄都苑涼風堂。使百年書敕字,驗與德胄所奏相似。遣左右亂捶擊之,又令人
 曳百年繞堂,且走且打,所過處,血皆遍地。氣息將盡,曰:「乞命,願與阿叔作奴。」遂斬之,棄諸池,池水盡赤,於後園親看埋之。



 妃把玦哀號,不肯食,月餘亦死。玦猶在手,拳不可開,時年十四。其父光自擘之,乃開。



 後主時,改九院為二十七院,掘得小屍,緋袍金帶,一髻一解,一足有靴。諸內參竅言,百年太子也。或以為太原王紹德。詔以襄城王子白澤襲爵樂陵王。齊亡入關,徙蜀死。



 汝南王彥理,武平初封王,位開府、清都尹。齊亡入關,隨例授儀同大將軍、封縣子。女入太子宮,故得不死。隋開皇初,卒於并州刺史。



 始平王彥德、城陽王彥基、定陽王彥康、汝陽王彥忠與汝南王同受封,並加儀同三司,後事闕。



 武成十三男:胡皇后生後主及瑯邪王儼;李夫人生南陽王綽;後宮生齊安王廓、北平王貞、高平王仁英、淮南王仁光、西河王仁機、樂平王仁邕、潁川王仁儉、安樂王仁雅、丹楊王仁直、東海王仁謙。



 南陽王綽,字仁通,武成長子也。以五月五日辰時生,至午時,後主乃生。武成以綽母李夫人非正嫡,故貶為第二。初名融,字君明,出後漢陽王。河清三年,改封南陽,別為漢陽置後。



 綽始十餘歲,留守晉陽。愛波斯狗。尉破胡
 諫之,欻然斫殺數狗,狼藉在地。



 破胡驚走,不敢復言。後為司徒、冀州刺史。好裸人,畫為獸狀,縱犬噬而食之。



 左轉定州,汲井水為後池,在樓上彈人。好微行,游獵無度,恣情強暴,云學文宣伯為人。有婦人抱兒在路,走避入草,綽奪其兒飼波斯狗。婦人號哭,綽怒,又縱狗使食,狗不食,塗以兒血,乃食焉。後主聞之,詔鎖綽赴行在所。至而宥之,問在州何者最樂。對曰:「多取歇,將蛆混看,極樂。」後主即夜索歇一斗,比曉,得二三升,置諸浴斛,使人裸臥浴斛中,號叫宛轉。帝與綽臨觀,喜噱不已。



 謂綽曰:「如此樂事,何不早馳驛奏聞?」綽由是大為後主寵,拜大將
 軍,朝夕同戲。



 韓長鸞間之,除齊州刺史。將發,長鸞令綽親信誣告其反,奏云:「此犯國法,不可赦。」後主不忍顯戮,使寵胡何猥薩後園與綽相撲,扼殺之。瘞於興聖佛寺,經四百餘日乃大斂,顏色毛髮皆如生。俗云五月五日生者,腦不壞。



 綽兄弟皆呼父為兄兄,嫡母為家家,乳母為姊姊,婦為妹妹。



 齊亡,妃鄭氏為周武帝所幸,請葬綽,敕所司葬於永平陵北。



 瑯邪王儼字仁威,武成第三子也。初封東平王,拜開府、侍中、中書監、京畿大都督、領軍大將軍,領御史中丞。遷大司徒、尚書令、大將軍、錄尚書事、大司馬。



 魏氏舊制,中
 丞出,千步清道,與皇太子分路行,王公皆遙住車,去牛頓軛於地,以待中丞過。其或遲違,則赤棒棒之。自都鄴後,此儀浸絕。武成欲雄寵儼,乃使一依舊制。儼初從北宮出,將上中丞,凡京畿步騎,領軍之官屬,中丞之威儀,司徒之鹵簿,莫不畢備。帝與胡后在華林園東門外,張幕隔青紗步障觀之。遣中貴驟馬趣仗,不得入,自言奉敕,赤棒應聲碎其鞍,馬驚人墜。帝大笑,以為善。更敕令駐車,傳語良久,觀者傾京邑。



 儼恆在宮中,坐含章殿以視事,諸父皆拜焉。帝幸并州,儼恆居守。每送駕,或半路,或至晉陽乃還。王師羅嘗從駕,後至,武成欲罪之。辭曰:「
 臣與第三子別,留連不覺晚。」武成憶儼,為之下泣,舍師羅不問。儼器服玩飾皆與後主同,所須悉官給。於南宮嘗見新冰綠李,還,怒曰:「尊兄已有,我何意無?」從是,後主先得新奇,屬官及工匠必獲罪。太上、胡后猶以為不足。儼嘗患喉,使醫下針,張目不瞬。又言於帝曰:「阿兄軟,何能率左右!」帝每稱曰:「此黠兒也,當有所成。」以後主為劣,有廢立之意。武成崩,改封瑯邪。儼以和士開、駱提婆等奢恣,盛修第宅,意甚不平。嘗謂曰:「君等所營宅,早晚當就,何太遲也?」二人相謂曰:「瑯邪王眼光弈弈,數步射人,向者暫對,不覺汗出。天子門奏事,尚不然。」由是忌之。



 武
 平二年,出儼居北宮,五日一朝,不復得無時見太后。四月,詔除太保,餘官悉解,猶帶中丞,且京畿。以北城有武庫,欲移儼於外,然後奪其兵權。書侍御史王子宜與儼左右開府高舍洛、中常侍劉辟強說儼曰:「殿下被疏,正由士開間構,何可出北宮,入百姓叢中也?」儼謂侍中馮子琮曰:「士開罪重,兒欲殺之。」子琮心欲廢帝而立儼,因贊成其事。儼乃令子宜表彈士開罪,請付禁推。子琮雜以他文書奏之,後主不審省而可之。儼誑領軍厙狄伏連曰:「奉敕,令領軍收士開。」



 伏連以諮子琮,且請覆奏。子琮曰:「瑯邪王受敕,何須重奏。」伏連信之,伏五十人於神
 獸門外,詰旦,執士開送御史。儼使馮永洛就臺斬之。



 儼徒本意唯殺士開。及是,因逼儼曰:「事既然,不可中止。」儼遂率京畿軍士三千餘人,屯千秋門外。帝使劉桃枝將禁兵八十人召儼。桃枝遙拜,儼命反縛,將斬之,禁兵散走。帝又使馮子琮召儼。儼辭曰:「士開昔來實合萬死,謀廢至尊,剃家家頭使作阿尼,故擁兵馬,欲坐著孫鳳珍宅上。臣為是,矯詔誅之。尊兄若欲殺臣,不敢逃罪;若放臣,願遣姊姊來迎臣,臣即入見。」姊姊即陸令萱也,儼欲誘出殺之。令萱執刀帝後,聞之戰慄。又使韓長鸞召儼。儼將入,劉辟強牽衣諫曰:「若不斬提婆母子,殿下無由
 得入。」廣寧、安德二王適從西來,欲助成其事,曰:「何不入?」辟強曰:「人少。」安德王顧眾而言曰:「孝昭殺楊遵彥,止八十人,今乃數千,何言人少?」後主泣啟太后曰:「有緣,更見家家,無緣,永別。」乃急召斛律光,儼亦召之。光聞殺士開,撫掌大笑曰:「龍子作事,固自不似凡人。」



 入見後主於永巷。帝率宿衛者步騎四百,授甲將出。光曰:「小兒輩弄兵,與交手,即亂。鄙諺云:『奴見大家心死。』至尊宜自至千秋門,瑯邪必不敢動。」皮景和亦以為然,後主從之。光步道,使人走出曰:「大家來。」儼徒駭散。帝駐馬橋上,遙呼之,儼猶立不進。光就謂曰:「天子弟殺一漢,何苦?」執其手,強引
 以前。



 請帝曰:「瑯邪王年少,腸肥腦滿,輕為舉措,長大自不復然,願寬其罪。」帝拔儼帶刀環,亂築辮頭,良久乃釋之。收伏連及高舍洛、王子宜、劉辟強、都督翟顯貴於後園,帝親射之而後斬,皆支解,暴之都街下。文武職吏,盡欲殺之。光以皆勳貴子弟,恐人心不安,趙彥深亦云「《春秋》責帥」,於是罪之各有差。儼之未獲罪也。鄴北城有白馬佛塔,是石季龍為澄公所作。儼將修之,巫曰:「若動此浮圖,此城失主。」不從,破至第二級,得白蛇,長數丈,回旋失之。數旬而敗。



 自是,太后處儼於宮內,食必自嘗之。陸令萱說帝曰:「人稱瑯邪王聰明雄勇,當今無敵,觀其根
 表,殆非人臣。自專殺以來,常懷恐懼,宜早為計。」何洪珍與和士開素善,亦請殺之。未決,以食輦密迎祖班問之。班稱周公殺管叔,季友鴆慶父,帝納其言。以儼之晉陽,使右衛大將軍趙元侃誘執儼。元侃曰:「臣昔事先帝日,見先帝愛王,今寧就死,不能行。」帝出元侃為豫州刺史。九月下旬,帝啟太后曰:「明旦欲與仁威出獵,須早早還。」是夜四更,帝召儼,儼疑之。陸令萱曰:「兄兄喚,兒何不去?」儼出至永巷,劉桃枝反接其手。儼呼曰:「乞見家家、尊兄!」桃枝以袖塞其口,反袍蒙頭負出,至大明宮,鼻血滿面,立殺之,時年十四。



 不脫靴,裹以席,埋於室內。帝使啟太后,
 臨哭十餘聲,便擁入殿。明年三月,葬於鄴西,贈謚曰楚恭哀帝,以慰太后。



 有遺腹四男,生數月,皆幽死。以平陽王淹孫世俊嗣。儼妃李祖欽女也,進為楚帝后,居宣則宮,齊亡,乃嫁焉。



 齊安王廓,字仁弘,武成第四子也。性長者,無過行,位特進、開府儀同三司、定州刺史。



 北平王貞,字仁堅,武成第五子也。沉審寬恕,帝常曰:「此兒得我鳳毛。」



 位司州牧、京畿大都督、兼尚書令、錄尚書事。帝行幸,總留臺事。積年,後主以貞長大,漸忌之。阿那肱承旨,令馮士幹劾,繫貞於獄,奪其留後權。



 高平王仁英,武成第六子也。舉止軒昂,精神無檢格。位定州刺史。



 淮南王仁光,武成第七子也。性躁又暴,位清都尹。次西河王仁機,生而無骨,不自支持。次樂平王仁邕;次潁川王仁儉;次安樂王仁雅,從小有暗疾;次丹楊王仁直;次東海王仁謙,皆養於北宮。



 瑯邪王死後,諸王守禁彌切。武平末年,仁邕已下,始得出外,供給儉薄,取充而已。尋後主窮蹙,以廓為光州,貞為青州,仁英為冀州,仁儉為膠州,仁直為濟州刺史。自廓已下,多與後主死於長安。仁英以清狂,仁雅以喑疾,獲免,俱徙蜀。隋開皇中,追仁
 英,詔與蕭琮、陳叔寶修其本宗祭祀。未幾而卒。



 後主五男:穆皇后生幼主;諸姬生東平王恪,次善德,次質德,次質錢;胡太后以恪嗣瑯邪王,尋夭折。



 齊滅,周武帝以任城已下大小三十王歸長安,皆有封爵。其後不從戮者,散配西土,皆死邊。



 論曰:文襄諸子,咸有風骨。雖文雅之道,有謝間、平,然武藝英姿,多堪禦侮。縱咸陽賜劍,殲覆有徵,若使蘭陵獲全,未可量也。而終見誅翦,以至土崩,可為太息者矣。安德以時艱主暗,晦迹韜光;及平陽之陣,奮其忠勇,蓋以臨難見危,義深家國。德昌大舉,事迫群情,理至淪亡,無
 所歸命。廣寧請出後宮,竟不獲遂,非孝珩辭致,有謝李同,自是後主心識,去平原已遠。存亡事異,安可同年而說。武成殘忍奸穢,事極人倫;太原跡異猜嫌,情非釁逆,禍起昭信,遂及淫刑。



 嗟乎!欲求長世,未之有也。以孝昭德音,庶可慶流後嗣,百年之酷,蓋濟南之濫觴。其云「莫效前人」之言,可為傷嘆。各愛其子,豈其然乎?瑯邪雖無師傅之資,而早聞氣尚,士開淫亂,多歷歲年,一朝剿絕,慶集朝野,以之受斃,深可痛焉。



 然專戮之釁,未之或免。贈帝謚恭,矯枉過直。觀過知仁,不亦異於是乎!



\end{pinyinscope}