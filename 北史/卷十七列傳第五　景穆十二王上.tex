\article{卷十七列傳第五 景穆十二王上}

\begin{pinyinscope}

 景穆皇帝十四男:恭皇后生文成皇帝;袁椒房生陽平幽王新成;尉椒房生京兆康王子推、濟陰王小新成;陽椒房生汝陰靈王天賜;樂良厲王萬壽、廣平殤王洛侯母並闕。孟椒房生任城康王雲;劉椒房生南安惠王楨、城陽康王長壽。慕容椒房生章武敬王太洛;尉椒房生樂陵康王胡兒;孟椒房生安定靖王休。趙王深早薨,無
 傳,母闕。魏舊太子後庭未有位號,文成即位,景穆宮人有子者,並號為椒房。



 陽平王新成,太安三年封,後為內都大官。薨,謚曰幽。



 長子安壽襲爵,孝文後賜名頤。累遷懷朔鎮大將。都督三道諸軍事北討,詔征赴京,勖以戰伐之事。對曰:「當仰杖廟算,使呼韓同渭橋之禮。」帝歎曰:「壯哉王言,朕所望也。」未發,遭母憂,詔遣侍臣以金革敦喻,既殯而發。與陸睿集三道諸將議軍途所詣。於是中道出黑山,東道趣士盧河,西道向侯延河。軍過大磧,大破蠕蠕。頤入朝,詔曰:「王之前言,果不虛也。」後除朔州刺史。及恆州刺史穆泰
 謀反,遣使推頤為主,頤密以狀聞。泰等伏誅,帝甚嘉之。宣武景明元年,薨於青州刺史,謚曰莊王。傳國至孫宗胤。明帝時,坐殺叔父賜死,爵除。



 頤弟衍,字安樂,賜爵廣陵侯,位梁州刺史。表請假王,以崇威重。詔曰:「可謂無厭求也,所請不合。」轉徐州刺史。至州病重,帝敕徐成伯乘傳療疾。差,成伯還。帝曰:「卿定名醫。」賚絹三千疋。成伯辭,請受一千。帝曰:「《詩》云:『人之云亡,邦國殄瘁。』以是而言,豈惟三千疋乎?」其為帝所重如此。後所生母雷氏卒,表請解州。詔曰:「先君餘尊之所厭,《禮》之明文。季末陵遲,斯典或廢。侯既親王之子,宜從餘尊之義,便可大功。」後卒於雍
 州刺史,謚曰康侯。



 衍性清慎,所在廉潔,又不營產業,歷牧四州,皆有稱績,亡日無斂屍具。



 子暢,字叔暢,從孝武帝入關,拜鴻臚,封博陵王。大統三年東討,沒於陣。



 子敏,嗜酒多費,家為之貧。其婿柱國乙弗貴、大將軍大利稽祐家貲皆千萬,每營給之。敏隨即散盡,而帝不之責。貴、祐後遂絕之。位儀同三司,改封南武縣公。



 暢弟融,字叔融,貌甚短陋,驍武過人。莊帝謀殺爾朱榮,以融為直閣將軍。



 及爾朱兆入洛,融逃人間。後從孝武入關,封魏興王,位侍郎、殿中尚書。



 衍弟欽,字思若,位中書監、尚書右僕射、儀同三司。欽色
 尤黑,故時人號為黑面僕射。欽淫從兄麗妻崔氏,為御史中尉封回劾奏,遇赦免。尋除司州牧。欽少好學,早有令譽。時人語曰:「皇宗略略,壽安、思若。」及晚年貴重,不能有所匡益,論者輕之。欽曾託青州人高僧壽為子求師,師至,未幾逃去。欽以讓僧壽。



 僧壽性滑稽,反謂欽曰:「凡人絕粒七日乃死,始經五朝,便爾逃遁,去食就信,實有所闕。」欽乃大慚,於是待客稍厚。後除司空公,封鉅平縣公。於河陰遇害,贈假黃鉞、太師、太尉公。



 子子孝,字季業,早有令譽。年八歲,司徒崔光見而異之,曰:「後生領袖,必此人也。」孝武帝入關,不及從駕。後赴長安,封義陽王。



 子
 孝美容儀,善笑謔,好酒愛士,縉紳歸之,賓客常滿,終日無倦。性又寬慈,敦穆親族。乃置學館於私第,集群從子弟,書夜講讀。并給衣食,與諸子同。後歷尚書令、柱國大將軍。子孝以國運漸移,深自貶晦,日夜縱酒。後例降為公,復姓拓拔氏。未幾,卒,子贇襲。



 京兆王子推,太安五年封,位侍中、征南大將軍、長安鎮大將。子推性沈雅,善於綏接,秦、雍之人服其威惠。入為中都大官,察獄有稱。獻文將禪位於子推,以大臣固諫,乃傳孝文。孝文即位,拜侍中、本將軍、開府儀同三司、青州刺史。



 未至,道薨。



 子太興襲,拜長安鎮大將。以黷貨削
 除官爵。後除秘書監,還復前爵,改封西河。轉守衛尉卿。初,太興遇患,請諸沙門行道。所有資財,一時布施,乞求病愈,名曰散生齋。及齋後,僧皆四散,有一沙門方云乞齋餘食。太興戲之曰;「齋食既盡,唯有酒肉。」沙門曰:「亦能食之。」因出酒一斗,羊腳一隻。食盡,猶言不飽。及辭出後,酒肉俱在。出門追之,無所見。太興遂佛前乞願:「向者之師,當非俗人。若此病得差,即捨王爵入道。」未幾便愈,遂請為沙門。表十餘上,乃見許。時孝文南討在軍,詔皇太子於四月八日為之下髮,施帛二千疋。既為沙門,名僧懿,居嵩山。太和二十二年終。



 子昴,字伯暉,襲,薨。



 昴子心
 妻,字魏慶,襲。孝靜時,累遷太尉、錄尚書事、司州牧、青州刺史。



 薨於州,贈假黃鉞、太傅、司徒公,謚曰文。悰寬和有度量,美容貌,風望儼然。



 得喪之間,不見於色。性清儉,不營產業,身死之日,家無餘財。



 昴弟仲景,性嚴峭。孝莊時,兼御史中尉,京師肅然。每向臺,恆駕赤牛,時人號「赤牛中尉」。太昌初,為河南尹,奉法無私。時吏部尚書樊子鵠部下縱橫,又為盜竊。仲景密加收捕,悉獲之,咸即行決。於是豪貴寒心。孝武帝將入關,授仲景中軍大都督,留京師。齊神武欲至洛陽,仲景遂棄妻子,追駕至長安。仍除尚書右僕射,封順陽王。



 仲景既失妻子,乃娶故爾朱天
 光妻也列氏。本倡女,有美色,仲景甚重之。經數年,前妻叔袁紇氏自洛陽間行至。也列遂徙居異宅。久之,有奸。事露,詔仲景殺之。仲景寵情愈至,謬殺一婢,蒙其屍而厚葬以代焉。列徙於密處,人莫知其詐。



 仲景三子濟、鐘、奉,叔袁紇氏生也,皆以宗室,早歷清官。仲景以列尚在,恐妻子漏之,乃謀殺袁紇。紇先覺,復欲陰害列。列謂從奴曰:「若袁紇殺我,必投我廁中;我告丞相,冀或不死。若不理首愆,猶埋我好地,爾為我告之。」奴遂告周文帝。周文依奏,詔笞仲景一百,免右僕射,以王歸第。也列以自告而逐之。仲景猶私不已。又有告者,詔重笞一百,付宗
 正,官爵盡除。仲景仍通焉。後周文帝以其歷任有令名,且杖策追駕,乃奏復官爵。也列、袁紇於是同居。大統五年,除幽州刺史。仲景多內亂,後就州賜死。



 仲景弟暹,字叔照。孝莊初,除南兗州刺史。在州猛暴,多所殺害。元顥入洛,暹據州不屈。莊帝還宮,封汝陽王,累遷秦州刺史。先秦州城人屢為反覆,暹盡誅之,存者十一二。普泰元年,除涼州刺史,貪暴無極。欲規府人及商胡富人財物,詐一臺符,誣諸豪等,云欲加賞。一時屠戮,所有資財生口,悉沒自入。孝靜時,位侍中、錄尚書事。薨,贈太師、錄尚書。子沖襲。無子,國絕。



 太興弟遙,字太原,有器望。以左衛
 將軍從孝文南征,賜爵饒陽男。宣武初,遭所生母憂,表請解任。詔以餘尊所厭,不許。明帝初,累遷左光祿大夫,仍領護軍。



 時冀州沙門法慶既為妖幻,遂說勃海人李歸伯。歸伯合家從之,招率鄉人,推法慶為主。法慶以歸伯為十住菩薩、平魔軍司、定漢王,自號大乘。殺一人者為一住菩薩,殺十人者為十住菩薩。又合狂藥,令人服之,父子兄弟不相知識,唯以殺害為事。刺史蕭寶夤遣兼長史崔伯驎討之,敗於煮棗城,伯驎戰沒。凶眾遂盛,所在屠滅寺舍,斬戮僧尼,焚燒經像,云:「新佛出世,除去眾魔。」詔以遙為使持節、都督北征諸軍事,討破之。禽法
 慶,并其妻尼惠暉等。斬法慶,傳首京師;後禽歸伯,戮於都市。



 初,遙大功昆弟皆是景穆之孫,至明帝而本服絕,故除遙等屬籍。遙表曰:竊聞聖人所以南面而聽天下,其不可得變革者,則親也尊也。四世而緦服窮,五世而袒免,六世而親屬竭矣。去茲以往,猶繫之以姓而弗別,綴之以食而弗殊。



 又《律》云議親者,非唯當世之屬親,歷謂先帝之五世。謹尋斯旨,將以廣帝宗,重盤石。先皇所以變茲事條,為此別制者,太和之季,方有意於吳、蜀;經始之費,慮深在初;割減之起,暫出當時也。且臨淮王提分屬籍之始,高祖賜帛三千疋,所以重分離。樂良王長命亦賜縑
 二千疋,所以存慈眷。此皆先朝殷勤克念,不得已而然者也。



 古人有言,「百足之蟲,至死不僵」者,以其輔己者眾。臣誠不欲妄親太階,茍求潤屋,但傷大宗一分,則天子屬籍不過十數人而已。在漢諸王之子,不限多少,皆列土而封,謂之曰侯;至於魏、晉,莫不廣胙河山,稱之曰公者,蓋惡其大宗之不固,骨肉之恩疏矣。



 臣去皇上雖是五世之遠,於先帝便是天子之孫。高祖所以國秩祿賦,復給衣食,后族唯給其賦,不與衣食者,欲以別外內,限異同也。今諸廟之感,在心未忘;行道之悲,倏然已及。其諸封者,身亡之日,三年服終,然後改奪。今朝廷猶在遏
 密之中,便議此事,實用未安。



 詔付尚書博議以聞。尚書令任城王澄、尚書左僕射元暉奏同遙表,靈太后不從。



 卒,謚曰宣公。



 遙弟恆,字景安,粗涉書史。恆以《春秋》之義,為名不以山川,表求改名芝。



 歷位太常卿、中書監、侍中。後於河陰遇害,贈太傅、司徒公,謚曰宣穆公。



 濟陰王小新成,和平二年封,頗有武略。庫莫奚侵擾,詔新成討之。新成乃多為毒酒。賊逼,便棄營而去。賊至,兢飲。遂簡輕騎縱擊,俘馘甚多。後位外都大官。薨,贈大將軍,謚曰惠公。



 子鬱,字伏生,襲。位開府,為徐州刺史。以黷貨賜死,國除。



 長子弼,字邕明,剛正有文學,位中散大夫。
 以世嫡,應襲先爵。為季父尚書僕射麗因于氏親寵,遂奪弼王爵,橫授同母兄子誕。於是弼絕棄人事,託疾還私第。



 宣武徵為侍中,弼上表固讓。入嵩山,以穴為室,布衣蔬食。卒。建義元年,子暉業訴復王爵。永安三年,追贈尚書令、司徒公,謚曰「文獻」。初,弼嘗夢人謂之曰:「君身不得傳世封,其紹先爵者,君長子紹遠也。」弼覺,即語暉業,終如其言。



 暉業少險薄,多與寇盜交通。長乃變節,涉子史,亦頗屬文,而慷慨有志節。



 歷位司空、太尉,加特進,領中書監,錄尚書事。齊文襄嘗問之曰:「比何所披覽?」



 對曰:「數尋伊、霍之傳,不讀曹、馬之書。」暉業以時運漸謝,不復
 圖全。唯事飲啖,一日三羊,三日一犢。又嘗賦詩云:「昔居王道泰,濟濟富群英。今逢世路阻,狐兔鬱縱橫。」齊初,降封美陽縣公,開府儀同三司、特進。



 暉業之在晉陽也,無所交通,居常閑暇,乃撰魏籓王家世,號為《辨宗錄》四十卷,行於世。位望隆重,又以性氣不倫,每被猜忌。



 天保二年,從駕至晉陽,於宮門外罵元韶曰:「爾不及一老嫗,背負璽與人,何不打碎之!我出此言,知即死,然爾亦詎得幾時!」文宣聞而殺之,并斬臨淮公孝友。孝友臨刑,驚惶失措,暉業神色自若。仍鑿冰沈其屍。



 暉業弟昭業,頗有學尚,位諫議大夫。莊帝將幸洛南,昭業立於閶闔門外,
 叩馬諫,帝避之而過。後勞勉之。位給事黃門侍郎、衛將軍、右光祿大夫。卒,謚曰文侯。



 鬱弟偃,位太中大夫。



 子誕,字曇首。初,誕伯父鬱以貪汙賜死,爵除。詔以誕,偃正妃子,立為嫡孫,特聽紹封。累遷齊州刺史。在州貪暴,大為人患。牛馬騾驢,無不逼奪,家之奴隸,悉迫取良人為婦。有沙門為誕採藥,還見誕,問外消息,對曰:「唯聞王貪,願王早代。」誕曰:「齊州七萬家,吾至來,一家未得三十錢,何得言貪?」後為御史中尉元纂所糾,會赦免。薨,謚靜王。



 子撫,字伯懿,襲。莊帝初,為從兄暉業訴奪王爵。



 偃弟麗,字寶掌,位兼宗正卿、右衛將軍。遷光祿勛,宗正、右衛如故。
 時秦州屠各王法智推州主簿呂茍兒為主,號建明元年,置立百官,攻逼州郡。涇州人陳瞻亦聚眾自稱王,號聖明元年。以麗為使持節、都督,與楊椿討之。茍兒率眾十餘萬,屯孤山,別據諸險,圍逼州城。麗出擊,大破之,使進軍水洛。賊徒逆戰,麗夜擊走之。行秦州事李韶破茍兒于孤山,乘勝追掩,獲其父母妻子。諸城之圍,亦悉奔散。茍兒率其王公三十餘人詣麗請罪。麗因平賊之勢,枉掠良善七百餘人。宣武嘉其功,詔有司不聽追檢。



 拜雍州刺史,為政嚴酷,吏人患之。其妻崔氏誕一男,麗遂出州獄囚,死及徒、流案未申臺者,一時放免。遷冀州刺
 史,入為尚書左僕射。帝問曰:「聞公在州殺戮無理,枉濫非一,又大殺道人。」對曰:「臣在冀州可殺道人二百許人,亦復何多?」帝曰:「一物不得其所,若納諸隍,況殺道人二百,而言不多!」麗脫冠謝,賜坐。卒,謚曰威。



 子顯和,少有節操,歷司徒記室參軍。司徒崔光每見之,曰:「元參軍風流清秀,容止閑雅,乃宰相之器。」除徐州安東府長史。刺史元法僧叛,顯和與戰被禽。



 執手命與連坐。顯和曰:「顯和與阿翁同源別派,皆是盤石之宗,一朝以地外叛,若遇董狐,能無慚德?」遂不肯坐。法僧猶欲慰喻。顯和曰:「乃可死作惡鬼,不能生為叛臣!」及將殺之,神色自若。建義初,
 贈秦州刺史。



 汝陰王天賜,和平三年封,後為內都大官。孝文初,殿中尚書胡莫寒簡西部敕勒豪富兼丁者,為殿中武士,而大納財貨。眾怒,殺莫寒及高平假鎮將奚陵。於是諸部敕勒悉叛。詔天賜與給事中羅雲討之。前鋒敕勒詐降,雲信之。副將元伏曰:「敕勒色動,恐有變,今不設備,將為所圖。」雲不從。敕勒襲殺雲,天賜僅得自全。累遷懷朔鎮大將。坐貪殘,恕死,削除官爵。卒,孝文哭於思政觀,贈本爵,葬從王禮,謚曰靈王。



 子逞,字萬安,卒於齊州刺史,謚曰威。



 逞子慶和,東豫州刺史,為梁將所攻,舉城降之。梁
 武以為北道總督、魏王。



 至項城,朝廷出師討之,望風退走。梁武責之曰:「言同百舌,膽若鼷鼠。」遂徙合浦。



 逞弟汎,字普安,自元士稍遷營州刺史。性貪殘,人不堪命,相率逐之,汎走平州。後除光祿大夫、宗正卿,封東燕縣男。於河陰遇害。



 汎弟修義,字壽安,頗有文才。自元士稍遷齊州刺史。修義以齊州頻喪刺史,累表固辭。詔不許,聽隨便立解宇。修義乃移東城。為政寬和。遷秦州刺史。明帝初,表陳庶人禧、庶人愉等,請宥前愆,賜葬陵域。靈太后詔曰:「收葬之恩,事由上旨,籓岳何得越職乾陳!」



 在州多受納。累遷吏部尚書。及在銓衡,唯事貨賄,授官大小,皆
 有定價。時中散大夫高居者,有旨先敘。上黨郡缺,居遂求之。修義私已許人,仰居不與。居大言不遜,修義命左右牽曳之。居對大眾呼天唱賊。人問居曰:「白日公庭,安得有賊?」居指修義曰:「此坐上者,違天子明詔,物多者得官,京師白劫,此非大賊乎?」修義失色。居行罵而出,後欲邀車駕論修義罪狀,左僕射蕭寶夤喻之乃止。



 二秦反,假修義兼尚書右僕射、西道行臺、行秦州事,為諸軍節度。修義性好酒,每飲連日,遂遇風病,神明昏喪,雖至長安,竟無部分之益。元志敗沒,賊東至黑水,更遣蕭寶夤討之,以修義為雍州刺史。卒於州,贈司空,謚曰文。



 子均,
 位給事黃門侍郎。後入西魏,封安昌王,位開府儀同三司。薨,贈司空,謚曰平。



 子則,字孝規,襲爵,位義州刺史。仕周為小塚宰、江陵總管。



 子文都,性梗直,仕周為右侍上士。隋開皇初,授內史舍人。煬帝即位,累遷御史大夫,坐事免。未幾,授太府卿,甚有當時譽。大業十三年,帝幸江都宮,詔文都與段達、皇甫無逸、韋津等同為東都留守。帝崩,文都與達、津等共推越王侗為帝。侗署文都為內史令、開府儀同三司、光祿大夫、左驍衛大將軍、攝右翊衛將軍、魯國公。



 既而宇文化及立秦王浩為帝,擁兵至彭城,所在響震。文都諷侗遣使通於李密。



 密乃請降,因
 授官爵,禮其使甚厚。王世充不悅,文都知之,陰有誅世充計。侗以文都領御史大夫,世充固執而止。盧楚說文都誅之,文都遂懷奏入殿。有人以告世充,世充馳還含嘉城。至夜難作,攻東太陽門而入,拜於紫微觀下,曰:「請斬文都,歸罪司寇。」侗見兵勢盛,遣其所署將軍黃桃樹執文都以出。文都顧謂侗曰:「臣今朝亡,陛下亦當夕及。」侗慟哭遣之,左右莫不憫默。出至興教門,世充令左右亂斬之,諸子並見害。



 則弟矩,字孝矩,西魏時,襲祖爵始平縣公,拜南豐州刺史。時見元氏將危,陰謂昆季曰:「宇文之心,路人所見。顛而不扶,焉用宗子!」為兄則所遏,乃
 止。



 後周文為兄子晉公護娶其妹為妻,情好甚密。及護誅,坐徙蜀。後拜司憲大夫。隋文帝重其門地,娶其女為房陵王妃。及為丞相,拜少冢宰,位柱國,賜爵洵陽郡公。



 及房陵立為皇太子,立其女為皇太子妃,親禮彌厚,拜壽州總管。時陳將任蠻奴等屢寇江北,復以孝矩領行軍總官,屯兵江上。後以年老,上表乞骸骨。輕涇州刺史。



 卒官,謚曰簡。子無竭嗣。



 矩次弟雅,字孝方,有文武幹用。開皇中,歷左領左右將軍、集沁二州刺史,封順陽郡公。



 雅弟褒,字孝整,少有成人量。年十歲而孤,為諸兄所愛養。善事諸兄。諸兄議欲別居,褒泣諫,不從。家素富,多
 金寶,褒一無所受,脫身而出。仕周,位開府、北平縣公、趙州刺史。從韋孝寬平尉遲迥,以功拜柱國,進封河間郡公。



 隋開皇中,拜原州總管。有商人為賊劫,其人疑同宿者而執之。褒察其色冤而辭正,遂捨之。商人詣闕訟褒受金縱賊。隋文帝遣窮之,使者簿責褒何故利金而捨盜。褒引咎無異辭。使者與褒俱詣京師,遂坐免官。其盜尋發他所。上曰:「何至自誣?」褒曰:「臣受委一州,不能息盜,臣罪一也;百姓為人所謗,不付法司,懸即放免,臣罪二也;無顧形迹,至今為物所疑,臣罪三也。臣有三罪,何所逃責!



 臣又不言受賂,使者復將有所窮究,然則縲紲
 橫及良善,重臣之罪,是以自誣。」



 上歎異之,稱為長者。



 煬帝即位,拜齊郡太守。及遼東之役,郡官督事者前後相屬。有西曹掾當行,詐疾,褒杖之。掾大言曰:「我將詣行在所,欲有所告。」褒大怒,因杖百餘,數日死。坐免官,卒於家。



 樂良王萬壽,和平三年封,拜征東大將軍,鎮和龍。性貪暴,徵還,道憂薨,謚曰厲王。子康王樂平襲。薨。子長命襲。坐殺人賜死,國除。



 子忠,明帝時,復前爵,位太常少卿。孝武帝汎舟天泉池,命宗室諸王陪宴。



 忠愚而無智,性好衣服,遂著紅羅襦,繡作領,碧綢褲,錦為緣。帝謂曰:「朝廷衣冠,應有常式,何為著百戲衣?」忠曰:「臣少來所愛,情存
 綺羅,歌衣舞服,是臣所願。」帝曰:「人之無良乃至此乎?」



 廣平王洛侯,和平二年封。薨,謚曰殤。無子,後以陽平幽王第五子匡後之。



 匡字建扶,性耿介,有氣節。孝文器之。謂曰:「叔父必能儀形社稷,匡輔朕躬,今可改名為匡,以成克終之美。」宣武即位,累遷給事黃門侍郎。茹皓始有寵,百寮微憚之。帝曾於山陵還,詔匡陪乘,又使皓登車。皓褰裳將上,匡諫,帝推之令下,皓恨匡失色。當時壯其忠謇。宣武親政,除肆州刺史。匡既忤皓,懼為所害,廉慎自修,甚有聲績。遷恆州刺史。徵為大宗正卿、河南邑中正。



 匡奏親王及始籓、二籓王妻,悉有妃號。而三籓以下,
 皆謂之妻。上不得同為妃名,而下不及五品以上有命婦之號,竊以為疑。詔曰:「夫貴於朝,妻榮於室,婦女無定,升從其夫。三籓既啟王封,妃名亦宜同等。妻者齊也,理與己齊,可從妃例。」自是三籓王妻,名號始定。從除度支尚書。匡表引樂陵、章武之例,求紹洛侯封。詔付尚書議。尚書奏聽襲封,以明興絕之義。



 時宣武委政於高肇,宗室傾憚,唯匡與肇抗衡。先自造棺,置於聽事,意欲輿棺詣闕,論肇罪惡,自殺切諫。肇聞而惡之。後因與太常卿劉芳議爭權量,遂與肇聲色。御史中尉王顯奏匡曰:自金行失御,群偽競興,禮壞樂崩,彞倫攸斁。高祖孝文皇
 帝以睿聖統天,克復舊典。乃命故中書監高閭,廣旌儒林,推尋樂府,以黍裁寸,將均周、漢舊章。



 屬雲構中遷,尚未云就。高祖睿思玄深,參考經、記,以一黍之大,用成分體,準之為尺,宣布施行。



 暨正始中,故太樂令公孫崇輒自立意,以黍十二為寸,別造尺度,定律刊鐘。



 皆向成訖,表求觀試。時敕太常卿臣芳,以崇造既成,請集朝英,議其得否。芳疑崇尺度與先朝不同,察其作者,於經史復異,推造鮮據,非所宜行。時尚書令臣肇、清河王懌等,以崇造乖謬,與《周禮》不同,遂奏臣芳依《周禮》更造,成訖量校,從其善者。而芳以先朝尺度,事合古典,乃依前詔書,
 以黍刊寸,並呈朝廷,用裁金石。於時議者多云芳是。唯黃門侍郎臣孫惠蔚與崇扶同。二途參差,頻經考議。



 而尚書令臣肇以芳造。崇物故之後,而惠蔚亦造一尺,仍雲扶。以比崇尺,自相乖背。量省二三,謂芳一尺為得。而尚書臣匡表云,劉、孫二尺,長短相傾,稽考兩律,所容殊異,言取中黍,校彼二家,云並參差,折中無所,自立一途,請求議判。



 當時議者,或是於匡,兩途舛駮,未即時定。肇又云:「權斛斗尺,班行已久,今者所論,豈踰先旨,宜仰依先朝故尺為定。」



 自爾以後,而匡與肇厲言都坐,聲色相加,高下失其常倫,尊競無復彞序。匡更表列,據己十是,
 云芳十非。又云:「肇前被敕旨,共芳營督,規立鐘石之名,希播制作之譽。乃憑樞衡之尊,藉舅氏之勢,與奪任心,臧否自己,阿黨劉芳,遏絕臣事。望勢雷同者,接以恩言;依經案古者,即被怒責。雖未指鹿化馬,移天徙日,實使蘊藉之士,聳氣坐端;懷道之夫,結舌筵次。」又言「芳昔與崇競,恆言自作,今共臣論,忽稱先朝。豈不前謂可行,輒欲自取;後知錯謬,便推先朝。殊非大臣之體,深失為下之義。復考校勢臣之前,量度偏頗之手,臣必刖足內朝,抱璞人外。」囂言肆意,彰於朝野。



 然匡職當出納,獻替所在,斗尺權度,正是所司。若己有所見,能練臧否,宜應首
 唱義端,早辨諸惑,何故默心隨從,不關一言,見芳成事,方出此語?計芳才學,與匡殊懸,所見淺深,不應相匹。今乃始發,恐此由心,借智於人,規成虛譽。



 況匡表云:「所據銅權,形如古誌,明是漢作,非莽別造。」及案權銘,「黃帝始祖,德布於虞;虞帝始祖,德布於新」。若莽佐漢時事,寧有銘偽新之號哉?又尋莽傳,云莽居攝,即變漢制度。考校二證,非漢權明矣。復云「芳之所造,又短先朝之尺。臣既比之,權然相合。更云「芳尺與千金堰不同。」臣復量比,因見其異,二三浮濫,難可據準。又云「共構虛端,妄為疑似,託以先朝,云非己製」。臣案此欺詐,乃在於匡,不在於芳。
 何以言之?



 芳先被敕,專造鐘律,管籥優劣,是其所裁,權斛尺度,本非其事。比前門下索芳尺度,而芳牒報云「依先朝所班新尺,復應下黍,更不增損,為造鐘律,調正分寸而已」。檢匡造時,在牒後一歲,芳於爾日,匡未共爭,已有此牒,豈為詐也?



 計崇造寸,積黍十二,群情共知。而芳造寸,唯止十黍,亦俱見。先朝詔書,以黍成寸,首尾歷然,寧有輒欲自取之理?肇任居端右,百寮是望,言行動靜,必副具瞻。若恃權阿黨,詐託先詔,將指鹿化馬,徙日移天,即是魏之趙高,何以宰物?



 肇若無此,匡既誣毀宰相,訕謗時政,阻惑朝聽,不敬至甚。請以肇、匡並禁尚書,推
 窮其原,付廷尉定罪。



 詔曰「可」。有司奏匡誣肇,處匡死刑。宣武恕死,降為光祿大夫。又兼宗正卿。出為兗州刺史。匡臨發,帝引見於東堂,勞勉之。匡猶以尺度金石之事,國之大經,前雖為南臺所彈,然猶許更議。若議之日,願聽臣暫赴。帝曰:「劉芳學高一時,深明典故。其所據者,與先朝尺乃寸過一黍,何得復云先朝之意也?兗州既所執不經,後議之日,何待赴都也。」



 明帝初,入為御史中尉。匡嚴於彈糾,始奏于忠,次彈高聰等免官,靈太后並不許。違其糾惡之心,又慮匡辭解,欲獎安之,進號安南將軍,後加鎮東將軍。



 匡屢請更權衡不已,於是詔曰:「謹權
 審度,自昔令典,定章革曆,往代良規。



 匡宗室賢亮,留心既久,可令更集儒貴,以時驗決。必務權衡得衷,令寸籥不舛。」



 又詔曰:「故廣平殤王洛侯體自恭宗,茂年薨殞,國除祀廢,不祀忽諸。匡親同若子,私繼歲久,宜樹維城,永茲盤石,可特襲王爵,封東平郡王。」匡所制尺度訖,請集朝士議定是非,詔付門下、尚書、三府、九列議定以聞。太師、高陽王雍等議,以為「晉中書監荀勖所造之尺,與高祖所定,毫釐略同。侍中崔光得古象尺,於時亦準議令施用。仰惟孝文皇帝德邁前王,睿明下燭,不刊之式,事難變改。臣等參論,請停匡議,永遵先皇之制。」詔從之。



 匡
 每有奏請,尚書令、任城王澄時致執奪。匡剛隘,內遂不平。先所造棺,猶在僧寺,乃復修事,將與澄相攻。澄頗知之,後將赴省,與匡逢遇,騶卒相撾,朝野駭愕。澄因是奏匡罪狀三十餘條,廷尉處以死刑。詔付八議,特加原宥,削爵除官。三公郎中辛雄奏理之。後特除平州刺史,徙青州刺史。尋為關右都督、兼尚書行臺。遇疾,還京。孝昌初,卒,謚曰文貞。後追復本爵,改封濟南王。



 第四子獻襲,薨。子祖育襲。武定初,墜馬薨。子勒叉襲。齊受禪,爵例降。



\end{pinyinscope}