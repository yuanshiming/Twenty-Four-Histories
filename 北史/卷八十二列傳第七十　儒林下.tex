\article{卷八十二列傳第七十 儒林下}

\begin{pinyinscope}

 沈重樊深熊安生樂遜黎景熙冀俊趙文深辛彥之何妥蕭該包愷房暉遠馬光劉焯劉炫褚暉顧彪魯世達張沖王孝籍沈重,字子厚,吳興武康人也。性聰悟,弱歲而孤,居喪合禮。及長,專心儒學,從師不遠千里。遂博覽群書,尤明《詩》
 及《左氏春秋》。梁武帝欲高置學官,以崇儒教。中大通四年,乃革選,以重補國子助教。後除《五經》博士。梁元帝之在籓也,甚歎異之。及即位,乃遣主書何武迎重西上。



 魏平江陵,重乃留事梁主蕭察,累遷都官尚書,領羽林監。察又令重於合歡殿講《周禮》。武帝以重經明行脩,乃遣宣納上士柳裘致書禮聘,又敕襄州總管衛公直敦喻遣之,在途供給,務從優厚。保定末,至於京師,詔令討論《五經》,并校定鐘律。天和中,復於紫極殿講三教義,朝士、儒生、桑門、道士至者二千餘人。



 重辭義優洽,樞機明辯,凡所解釋,咸為諸儒所推。六年,授驃騎大將軍、開府儀
 同三司、露門博士,仍於露門館為皇太子講《論語》。建德末,表請還梁,武帝優詔不許。重固請,乃許。為遣小司門上士楊汪送之。梁主蕭巋拜重散騎常侍、太常卿。大象二年,來朝京師。開皇三年卒,年八十四。隋文帝遣舍人蕭子寶祭以少牢,贈使持節、上開府儀同三司、許州刺史。重學業該博,為當世儒宗。至於陰陽圖緯、道經、釋典,無不通涉。著《周禮義》三十一卷、《儀禮義》三十五卷、《禮記義》三十卷、《毛詩義》二十八卷、《喪服經義》五卷、《周禮音》一卷、《儀禮音》一卷、《禮記音》二卷、《毛詩音》二卷。



 樊深,字文深,河樂猗氏人也。事繼母甚謹,弱冠好學,負
 書從師於河西,講習《五經》,晝夜不倦。魏永安中,隨軍征討,以功累遷中散大夫。嘗讀書,見吾丘子,遂歸侍養。



 孝武西遷,樊王二姓舉義,為魏所誅。深父保周、叔父歡周並被害。深因避難,墜崖傷足,絕食再宿。於後遇得一簞餅,欲食之,然念繼母老痺,或免虜掠,乃弗食。夜中匍匐尋覓,母得見,因以饋母。還復遁去,改易姓名,遊學於汾晉間。習天文及算歷之術。後為人所告,囚送河東。屬東魏將韓軌長史張曜重其儒學,延深至家,因是便得逃隱。周文平河東,贈保周南郢州刺史,歡周儀同三司。深歸葬其父,負土成墳。



 尋而于謹引為府參軍事,令在館
 授教子孫。周文置學東館,教諸將子弟,以深為博士。深經學通贍,每解書,多引漢魏以來諸家義而說之。故後生聽其言者,不能曉悟,背而譏之曰:「樊生講書,多門戶,不可解。」然儒者推其博物。性好學,老而不怠。朝暮還往,常據鞍讀書,至馬驚墜地,損折支體,終亦不改。後除國子博士,賜姓萬紐于氏。天平二年,遷縣伯中大夫,加開府儀同三司。建德元年,表乞骸骨,詔許之。朝廷有疑議,常召問焉。後以疾卒。



 深既專經,又讀諸史及《倉》、《雅》、篆、籀、陰陽、卜筮之書。學雖博贍,訥於辭辯,故不為當時所稱。撰《孝經》《喪服問疑》各一卷。又撰《七經異同》三卷。子義綱。



 熊安生,字植之,長樂阜城人也。少好學,勵精不倦。從陳達受《三傳》,從房虯受《周禮》,事徐遵明,服膺歷年,後受《禮》於李寶鼎,遂博通《五經》。



 然專以《三禮》教授,弟子自遠方至者千餘人。乃討論圖緯,捃摭異聞。先儒所未悟者,皆發明之。齊河清中,陽休之特奏為國子博士。時西朝既行《周禮》,公卿以下,多習其業,有宿疑碩滯者數十條,皆莫能詳辨。天和三年,周齊通好,兵部尹公正使焉。與齊人語及《周禮》,齊人不能對。乃令安生至賓館,與公正言。公正有口辯,安生語所未至者,便撮機要而驟問之。安生曰:「《禮》義弘深,自有條貫,必欲升堂睹奧,寧可汨其先
 後?但能留意,當為次第陳之。」公正於是問所疑,安生皆為一一演說,咸究其根本。公正嗟服。還,具言之於武帝,帝大欽重之。



 及入鄴,安生遽令掃門。家人怪而問之,安生曰:「周帝重道尊儒,必將見我矣。」俄而帝幸其第,詔不聽拜,親執其手,引與同坐,謂曰:「朕未能去兵,以此為愧。」安生曰:「黃帝尚有阪泉之戰,況陛下龔行天罰乎!」帝又曰:「齊氏賦役繁興,竭人財力,朕救焚拯溺,思革其弊,欲以府庫及三臺雜物散之百姓,公以為何如?」安生曰:「昔武王克商,散鹿臺之財,發巨橋之粟,陛下此詔,異代同美。」帝又曰:「朕何如武王?」安生曰:「武王伐紂,懸首白旗;陛
 下平齊,兵不血刃,愚謂聖略為優。」帝大悅,賜帛三百匹、米三百石、宅一區,並賜象笏及九鐶金帶,自餘什物稱是。又詔所司給安車駟馬,令隨駕入朝,并敕所在供給。



 至京,敕令於大乘佛寺,參議五禮。宣政元年,拜露門博士、下大夫,時年八十餘。



 尋致仕,卒於家。



 安生既學為儒宗,嘗受其業,擅名於後者,有馬榮伯、張黑奴、竇士榮、孔籠、劉焯、劉炫等,皆其門人焉。所撰《周禮義疏》二十卷,《禮記義疏》三十卷、《孝經義》一卷,並行於世。安生與同郡宗道暉、張暉、紀顯敬、徐遵明等為祖師。



 道暉好著高翅帽、大屐,州將初臨,輒服以謁見,仰頭舉肘,拜於屐上,自言
 學士比三公。後齊任城王湝鞭之,道暉徐呼安偉,安偉出,謂人曰:「我受鞭,不漢體。」



 復躡屐而去。冀州人為之語曰「顯公鐘,宋公鼓,宗道暉屐,李洛姬肚」,謂之四大。顯公,沙門也,宋公,安德太守也;洛姬,婦人也。



 安生在山東時,歲歲遊講,從之者傾郡縣。或誑之曰:「某村古塚,是晉河南將軍熊光,去七十二世。舊有碑,為村人埋匿。」安生掘地求之,不得,連年訟焉。冀州長史鄭大言雚判之曰:「七十二世,乃是羲皇上人;河南將軍,晉無此號。



 訴非理記。」安生率其族向塚而號。將通名,見徐之才、和士開二人相對,以徐之才諱「雄」,和士開諱「安」,乃稱「觸觸生」,群公哂之。



 樂遜,字遵賢,河東猗氏人也。幼有成人之操,從徐遵明於趙、魏間,受《孝經》、《喪服》、《論語》、《詩》、《書》、《禮》、《易》、《左氏春秋》大義。尋而山東寇亂,學者散逸,遜於擾擾之中,猶志道不倦。大統七年,除子都督。



 九年,太尉李弼請遜教授諸子。既而周文盛選賢良,授以守令。相府戶曹柳敏、行臺郎中盧光、河東郡丞辛粲相繼舉遜,稱有牧人之才。弼請留不遣。魏廢帝二年,周文召遜教授諸子。在館六年,與諸儒分授經業,講《孝經》、《論語》、《毛詩》及服虔所注《春秋左氏傳》。周閔帝踐阼,以遜有理務材,除秋官府上士,轉小師氏下大夫。自譙王儉以下,並束脩行弟子之禮。遜以經
 術教授,甚有訓導之方。及衛公直鎮蒲州,遜為直主簿。



 武成元年六月,以霖雨經時,詔百官上封事。遜陳時宜十四條,其五條切於政要。其一,崇教方。其二,省造作。其三,明選舉。其四,重戰伐。其五,禁奢侈。



 保定二年,以訓導有方,頻加賞賜,遷遂伯中大夫。五年,詔魯公贇、畢公賢等,俱以束脩之禮,同受業焉。



 天和元年,岐州刺史陳公純舉遜以賢良。五年,遜以年在懸車,上表致仕,優詔不許。於是賜以粟帛及錢等,授湖州刺史,封安邑縣子。人多蠻左,未習儒風。



 遜勸勵生徒,加以課試,數年之間,化洽州境。蠻俗生子,長大多與父母異居。遜每加勸導,多
 革前弊。在任數載,頻被褒錫。秩滿還朝,拜皇太子諫議,復在露門教授皇子。大象初,進爵崇業郡公,又為露門博士。二年,進位開府儀同大將軍,出為汾陰郡守。遜以老病固辭,詔許之,乃改授東揚州刺史。仍賜安車、衣服及奴婢等,又於本郡賜田十頃,儒者以為榮。隋開皇元年,卒於家,年八十二。贈本官,加蒲、陜二州刺史。



 遜性柔謹,寡交遊,立身以忠信為本。不自矜尚。每在眾言論,未嘗為人之先,學者以此稱之。所著《孝經》、《論語》、《毛詩》、《左氏春秋序論》十餘篇。



 又著《春秋序義》,通賈、服說,發杜氏違,辭理並可觀。



 初,周又有黎景熙,以古學顯。



 黎景熙,字季明,河間鄭人,少以孝行聞於世。曾祖嶷,魏太武時,以軍功賜爵容城縣男,後為燕郡守。祖鎮、父瓊,並襲爵。季明少好讀書,性強記默識,而無應對之能。其從祖廣,太武時尚書郎,善古學。常從吏部尚書清河崔宏受字義,又從司徒崔浩學楷篆,自是家傳其法。季明亦傳習之,頗與許氏有異。又好玄象,頗知術數,而落魄不事生業。有書千餘卷。雖窮居獨處,不以飢寒易操。與范陽盧道源為莫逆交。永安中,道源勸令入仕,始為威烈將軍。孝武西遷,季明乃寓居伊洛。侯景徇地河外,召季明從軍,稍遷黎陽郡守。季明從至懸瓠,察景終不足
 恃,遂去之。客於潁川。時王思政鎮潁川,累使召季明,留於內館。月餘,周文又徵之,遂入關。乃令季明正定古今文字於東閣。大統末,拜著作佐郎。於時倫輩,皆位兼常伯,車服華盛,唯季明獨以貧素居之,而無愧色。又勤於所職,著述不怠。然性尤專固,不合於時,是以一為史官,遂十年不調。武成末,遷外史下大夫。



 保定三年,盛營宮室。春夏大旱,詔公卿百僚,極言得失。季明上封事曰:臣聞成湯遭旱,以六事自陳。宣王太甚,而珪璧斯竭。豈非遠慮元元,俯哀黎庶。今農要之月,時雨猶愆,率土之心,有懷渴仰。陛下垂情萬類,子愛群生,覲禮百神,猶未豐
 洽。豈或作事不節,有違時令,舉措失中,當邀斯旱。



 《春秋》,君舉必書,動為典禮。水旱陰陽,莫不應行而至。孔子曰:「言行,君子之所以動天地,可不慎乎!」《春秋》莊公三十一年冬,不雨,《五行傳》以為是歲一年而三築臺,奢侈不恤人也。僖公二十一年夏,大旱,《五行傳》以為時作南門,勞人興役。漢惠帝二年夏,大旱,五年夏,大旱,江河水少,谿澗水絕,《五行傳》以為先是發十四萬六千人城長安。漢武帝元狩三年夏,大旱,《五行傳》以為是歲發天下故吏,穿昆明池。然則土木之功,動人興役,天輒應之以異。典籍作誡,倘或可思,上天譴告,改之則善。今若息人省役,
 以答天譴,庶靈澤時降,嘉穀有時,則年登可覬,子來非晚。《詩》云:「人亦勞止,迄可小康,惠此中國,以綏四方。」或恐極陽生陰,秋多雨水,年復不登,人將無覬。如又薦飢,為慮更甚。



 時豪富之家,競為奢麗。季明又上書曰:臣聞寬大所以兼覆,慈愛所以懷眾。故天地稱其高厚者,萬物得其容養焉;四時著其寒暑者,庶類資其忠信焉。是以帝王者,寬大象天地,忠信則四時。招搖東指,天下識其春;人君布德,率土懷其惠。伏惟陛下,資乾御宇,品物咸亨,時乘六龍,自強不息,好問受規,天下幸甚。



 自古至道之君,亦皆廣延博訪,詢採皞蕘,置鼓樹木,以求其過。頃
 者亢旱踰時,人懷望歲,陛下爰發明詔,廣求六瘼,同禹、湯之罪己,高宋景之守正,澍雨應時,年穀斯稔。剋己節用,慕質去華,此則尚矣。然而朱紫仍耀於衢路,綺縠猶侈於豪富,短褐未充於細人,糟糠未厭於編戶。此則勸導之理,有所未周故也。今雖導之以禮,齊之以刑,風俗固難以一矣。昔漢文帝集上書之囊,以作帷帳;惜十家之產,不造露臺。後宮所幸,衣不曳地,方之今日富室之飾,嘗不如婢隸之服。



 然而以身率下,國富刑清,廟稱太宗,良有以也。臣聞聖人久於其道而天下化成。



 今承魏氏衰亂之後,貞信未興。宜先尊五美,屏四惡,革浮華之
 俗,抑流競之風,察鴻都之小藝,焚雉頭之異服,無益之貨勿重於時,虧德之器勿陳於側,則人知德矣。



 臣又聞之,為政之要,在於選舉。若差之毫釐,則有千里之失;後來居上,則致積薪之譏。是以古之善為政者,貫魚以次,任必以能。爵人於朝,不以私愛。簡才以授其官,量能以任其用。官得其才,任當其用,六轡既調,坐致千里。虞舜選眾,不仁者遠,則庶事康哉,人知其化矣。



 帝覽而嘉之。



 時外史廨宇屢移,未有定所。季明又上言曰:「外史之職,漢之東觀,帝王所寶,此焉攸在。自魏及周,公館不立,臣雖愚瞽,猶知其非。是以去年十一月中,敢冒奏陳,特降
 中旨,即遣脩營。荏苒一周,未知功力。臣職思其憂,敢不重請。」



 帝納焉,於是廨宇方立。天和二年,進車騎大將軍、儀同三司。後以疾卒。



 又周文初,屬天下分崩,時學術之士蓋寡,故曲學末伎,咸見引納。至若冀俊、趙文深之徒,雖才愧昔人,而名著於世,並見收用。



 冀俊,字僧俊,太原陽邑人也。性沈謹,善隸書,特工模寫。初為賀拔岳墨曹參軍。岳被害,周文引為記室。時周文志平侯莫陳悅,乃令俊偽為魏帝敕書與費也頭,令將兵助周文討悅。俊尋舊敕模寫,及代舍人、主書等署,與真無異。周文大悅。費也頭見敕,不以為疑,遂遣兵受周
 文節度。大統初,封長安縣男,從征弘農,戰於沙苑,進爵為子。累遷襄樂郡守。尋徵還,教明帝及宋獻公等隸書。時俗入書學者亦行束脩之禮,謂之謝章。俊以書字所興,起自蒼頡,若同常俗,未為合禮,遂啟周文,釋奠蒼頡及先聖、先師。除黃門侍郎、本州大中正。累遷湖州刺史。靜退,每以清約自處。前後所歷,頗有聲稱。尋加驃騎大將軍、開府儀同三司。後進爵為昌樂侯,卒。



 趙文深,字德本,南陽宛人也。父遐,以醫術仕魏,為尚藥典御。文深少學楷隸。年十一,獻書於魏帝。後立義歸朝,除大丞相府法曹參軍。雅有鐘、王之則,筆勢可觀。當時
 碑榜,唯文深、冀俊而已。大統十二年,追論立義功,封白石縣男。



 文帝以隸書紕繆,命文深與黎季明、沈遐等依《說文》及《字林》,刊定六體,成一萬餘言,行於世。及平江陵之後,王褒入關,貴遊等翕然並學褒書。文深之書,遂被遐棄。文深慚恨,形於言色。後知好尚難及,亦改習褒書。然竟無所成,轉被譏議,謂之學步邯鄲焉。至於碑榜,餘人猶莫之逮。王褒亦每推先之。宮殿樓閣,皆其迹也。遷縣伯下大夫。明帝令至江陵書影覆寺碑,漢南人士,亦以為工。梁主蕭察觀而美之,賞遺甚厚。天和元年,露寢等初成,文深以題榜之功,除趙興郡守。



 文深雖居外任,
 每須題榜,輒復追之。後以疾卒。



 辛彥之,隴西狄道人也。祖世敘,魏涼州刺史。父靈補,周渭州刺史。彥之九歲而孤,不交非類。博涉經史,與天水牛弘同志好學。後入關,遂家京兆。周文見而器之,引為中外府禮曹,賜以衣馬珠玉。時國家草創,朝貴多出武人,修定儀注,唯彥之而已。尋拜中書侍郎。及周閔帝受禪,彥之與小宗伯盧辯,專掌儀制。歷典祀、太祝、樂部、御正四曹大夫,開府儀同三司,封五原郡公。宣帝即位,拜小宗伯。時帝立五皇后,彥之切諫,由是忤旨,免官。



 隋文帝受禪,除太常少卿,改封任城郡公,進位開府。歷國子
 祭酒、禮部尚書。



 與秘書監牛弘撰新禮。帝嘗令彥之與沈重論議,重不能抗,避席而謝曰:「辛君所謂金城湯池,無可攻之勢。」帝大悅。後除隨州刺史。時州牧多貢珍玩,惟彥之所貢,並供祭之類。上謂朝臣曰:「人安得無學!彥之所貢,稽古之力也。」遷潞州刺史,前後俱有惠政。彥之又崇信佛道,於城內立浮圖二所,並十五層。開皇十一年,州人張元暴死,數日乃蘇。云遊天上,見新構一堂,制極崇麗。元問其故,云潞州刺史辛彥之有功德,造此堂以待之。彥之聞而不悅。其年卒,謚曰宣。



 彥之撰《墳典》一部、《六官》一部、《祝文》一部、《禮耍》一部、《新禮》一部、《五經異義》
 一部,並行於世。子孝舒、仲龕,並早有令譽。



 何妥,字棲風,西城人也。父細腳胡,通商入蜀,遂家郫縣。事梁武陵王紀,主知金帛,因致巨富,號為西州大賈。妥少機警,八歲遊國子學,助教顧良戲之曰:「汝姓何,是荷葉之荷?為河水之河?」妥應聲答曰:「先生姓顧,是眷顧之顧?



 為新故之故?」眾咸異之。十七,以伎巧事湘東王。後知其聰明,召為誦書左右。



 時蘭陵蕭翽,亦有俊才,住青楊巷,妥住白楊頭。時人為之語曰:「世有兩俊,白楊何妥,青楊蕭翽。」其見美如此。



 江陵平,入周,仕為太學博士。宣帝初立五后,問儒者辛彥之。對曰:「后與天子匹體齊尊,不
 宜有五。」妥駮曰:「帝嚳四妃,舜又二妃,亦何常數?」由是封襄城縣男。文帝受禪,除國子博士,加通直散騎常侍,進爵為公。



 妥姓勁急,有口才,好是非人物。納言蘇威嘗言於上曰:「臣先人每誡臣云:唯讀《孝經》一卷,足可立身經國,何用多為?」上亦然之。妥進曰:「蘇威所學,非止《孝經》。厥父若信有此言,威不從訓,是其不孝;若無此言,面欺陛下,是其不誠。不誠不孝,何以事君?且夫子又云:『不讀《詩》無以言,不讀《禮》無以立。』豈容蘇綽教子,獨反聖人之訓乎?」威時兼領五職,上甚親重之。妥因奏威不可信任。又以掌天文律度,皆不稱職,妥上八事以諫。



 其一事曰:臣
 聞知人則哲,惟帝難之。孔子曰:舉直錯枉則人服,舉枉錯直則人不服。由此言之,政之安危,必慎所舉。故進賢受上賞,蔽賢蒙顯戮。察今之舉人,良異于此。無論諂直,莫擇賢愚。心欲崇高,則起家喉舌之任;意須抑屈,必白首郎署之官。人不之服,實由於此。臣聞爵人於朝,與士共之;刑人於市,與眾棄之。伏見留心獄訟,愛人如子,每應決獄,無不詢訪群公,刑之不濫,君之明也。



 刑既如此。爵亦宜然。若有懋功,簡在帝心者,便可擢用。自斯以降,若選重官,必參以眾議,勿信一人之舉,則上不偏私,下無怨望。



 其二事曰:孔子云:是察阿黨,則罪無掩蔽。又曰:「
 君子周而不比,小人比而不周。」所謂比者,即阿黨也。謂心之所愛,既已光華榮顯,猶加提挈;心之所惡,既已沈滯屈辱,薄言必怒。提挈既成,必相掩蔽,則欺上之心生矣;屈辱既加,則有怨恨,謗讟之言出矣。伏願廣加訪察,勿使朋黨路開,威恩自任。有國之患,莫大於此。



 其三事曰:臣聞舜舉十六族,所謂八元八凱也。計其賢明,理優今日。猶復擇才授任,不相侵濫。故得四門雍穆,庶績咸熙。今官員極多,用人甚少,一人身上,乃兼數職。為是國無人也?為是人不善也?今萬乘大國,髦彥不少,縱有明哲,無由自達。東方朔言曰:「尊之則為將,卑之則為虜。」斯
 言信矣。今當官之人,不度德量力,既無呂望、傅說之能,自負傅巖、渭水之氣。不慮憂深責重,唯畏總領不多。安斯寵任,輕彼權軸。顛沛致蹶,實此之由。《易》曰:「鼎折足,覆公餗,其形渥,凶。」言不勝其任也。臣聞窮力舉重,不能為用。伏願更任賢良,分才參掌,使各行其力,則庶事康哉。



 其四事曰:臣聞《禮》云:析言破律,亂名改作,執左道以亂政者殺。孔子曰:仍舊貫,何必改作。伏見比年以來,改作者多矣。如範威刻漏,十載不成;趙翊尺秤,七年方決;公孫濟迂誕,醫方費逾巨萬;徐道慶迴互子午,糜耗飲食;常明破律,多歷歲時;王渥亂名,曾無紀極;張山居未知
 星位,前已蹂藉太常;曹魏祖不識北辰,今復蘭轢太史。莫不用其短見,便自夸毗,邀射名譽,厚相誣罔。請今日已後,有如此者,若其言不驗,必加重罰。庶令有所畏忌,不敢輕奏狂簡。



 其餘文多不載。時蘇威權兼數職,先嘗隱武功,故妥言「自負傅巖、渭水之氣」,以此激上。書奏,威大銜之。二年,威定考文學,妥更相訶詆。威勃然曰:「無何妥,不慮無博士!」妥應聲曰:「無蘇威,亦何憂無執事!」於是與威有隙。



 其後,上令妥考定鐘律。妥又上表曰:臣聞明則有禮樂,幽則有鬼神。然則動天地,感鬼神,莫近於禮樂。又云:樂至則無怨,禮至則不爭。揖讓而臨天下者,禮
 樂之謂也。臣聞樂有二:一曰姦聲,二曰正聲。夫姦聲感人而逆氣應之,正聲感人而順氣應之。順氣成象,故樂行而倫清,耳目聰明,血氣和平,移風易俗,天下皆寧。孔子曰:「放鄭聲,遠佞人。」



 故鄭、衛、宋、趙之聲出,內則發疾,外則傷人。是以宮亂則荒,其君驕;商亂則破,其官壞;角亂則憂,其人怨;徵亂則哀,其事勤;羽亂則危,其財匱。五者皆亂,則國亡無日矣。



 魏文侯問子夏曰:「吾端冕而聽古樂,則欲寐;聽鄭衛之音而不倦,何也?」



 子夏對曰:「夫古樂者,始奏以文,復亂以武。修身及家,平均天下。鄭衛之音者,姦聲以亂,溺而不止,優雜子女,不知父子。今君所問
 者,樂也,所愛者,音也。



 夫樂之與音,相近而不同。為人君者,謹審其好惡。」案聖人之作樂也,非止茍悅耳目而已矣。欲使在宗廟之內,君臣同聽之,則莫不和敬;在鄉里之內,長幼同聽之,則莫不和順;在閨門之內,父子同聽之,則莫不和親。此先王立樂之方也。故知聲而不知音者,禽獸是也;知音而不知樂者,眾庶是也。故黃鐘、大呂,弦歌干戚,童子皆能舞之。能知樂者,其惟君子。不知聲者不可與言音,不知音者不可與言樂,知樂則幾於道矣。紂為無道,太師抱樂器以奔周。晉君德薄,師曠固惜清徵。



 上古之時,未有音樂,鼓腹擊壤,樂在其間。《易》曰:「先
 王作樂崇德,殷薦之上帝,以配祖考。」至于黃帝作《咸池》,顓頊作《六莖》,帝嚳作《五英》,堯作《大章》,舜作《大韶》,禹作《大夏》,湯作《大濩》,武王作《大武》。



 從夏以來,年代久遠,唯有名字,其聲不可得聞。自殷至周,備于《詩·頌》。故自聖賢已下,多習樂者,至如伏羲減瑟,文王足琴,仲尼擊磬,子路鼓瑟,漢高擊築,元帝吹簫。



 漢祖之初,叔孫通因秦樂人,制宗廟之樂。迎神於廟門,奏《嘉至之樂》,猶古降神之樂也。皇帝入廟門,奏《永至之樂》,以為行步之節,猶古《采薺肆夏》也。乾豆上薦,奏《登歌之樂》,猶古清廟之歌也。登歌再終,奏《休成之樂》,美神饗也。皇帝就東廂坐定,奏《永安之
 樂》,美禮成也。其《休成》、《永至》二曲,叔孫通所制也。漢高祖廟,奏《武德》、《文始》、《五行之舞》。當春秋時,陳公子完奔齊,陳是舜後,故齊有《韶》樂。孔子在齊聞韶,三月不知肉味是也。秦始皇滅齊,《韶》樂傳於秦。漢高祖滅秦,《韶》樂傳於漢。漢高祖改名《文始》,以示不相襲也。《五行舞》者,本周《大武》樂也,始皇改曰《五行》。



 及于孝文,復作《四時之舞》,以示天下安和,四時順也。孝景采《武德舞》以為《昭德》,孝宣又采《昭德》以為《盛德》。雖變其名,大抵皆因秦舊事。至於晉、魏,皆用古樂。魏之三祖,並制樂辭。自永嘉播越,五都傾蕩,樂聲南度,以是大備江東。宋、齊已來,至于梁代,所行樂
 事,猶皆傳古。三雍四始,實稱大盛。及侯景篡逆,樂師分散,其四舞三調,悉度偽齊。齊氏雖知傳受,得曲而不用之於宗廟朝廷也。



 臣少好音律,留意管弦,年雖耆老,頗皆記憶。及東土克定,樂人悉反,問其逗留,果云是梁人所教。今三調四舞,並皆有手,雖不能精熟,亦頗具雅聲。若令教習傳授,庶得流傳古樂。然後取其會歸,撮其指要,因循損益,更制嘉名,歌盛德於當今,傳雅正於來葉,豈不美歟。謹具錄三調四舞曲名,又製歌辭如別。其有聲曲流宕,不可以陳於殿庭者,亦悉附之於後。



 書奏,別敕太常,取妥節度。於是作清、平、瑟三調聲,又作八佾《鞸》、《
 鐸》、《巾》、《拂》四舞。先是太常所傳宗廟雅樂,歷數十年,唯作大呂,廢黃鐘。妥又深乖古意,乃奏請用黃鐘。詔下公卿議,從之。俄而子蔚為秘書郎。有罪當刑,上哀之,減死論。是後恩禮漸薄。六年,出為龍州刺史。時有負笈遊學者,妥皆為講說教授之。又為《刺史箴》,勒于州門外。在職三年,以疾請還,詔許之。



 復知學事。



 時上方使蘇夔在太常參議鐘律,夔有所建議,朝士多從之。妥獨不同,每言夔之短。帝下其議,群臣多排妥。妥復上封事,指陳得失,大抵論時政損益,并指斥當世朋黨。於是蘇威及吏部尚書盧愷、侍郎薛道衡等皆坐得罪。除伊州刺史,不行。



 尋為國子祭酒,卒官。謚曰肅。



 撰《周易講疏》三卷、《孝經義疏》二卷、《莊子義疏》四卷。與沈重等撰《三十六科鬼神感應等大義》九卷、《封禪書》一卷、《樂要》一卷、文集十卷,並行於世。



 于時學士之自江南來者,蕭該、包愷並知名。



 蕭該,蘭陵人。梁鄱陽王恢之孫,少封攸侯。荊州平,與何妥同至長安。性篤學,《詩》、《書》、《春秋》、《禮記》並通大義,尤精《漢書》,甚為貴遊所禮。開皇初,賜爵山陰縣公,拜國子博士。奉詔與妥正定經史。然各執所見,遞相是非,久而不能就。上譴而罷之。該後撰《漢書》及《文選音義》,咸為當時所貴。



 包愷,字和樂,東海人。其兄愉,明《五經》,愷悉傳其業。及從王仲通受《史記》、《漢書》,尤稱精究。大業中,為國子助教。于是《漢書》學者以蕭、包二人為宗,遠近聚徒教授者數千人。卒,門人起墳立碣焉。



 房暉遠,字崇儒,恒山真定人也。世傳儒學。暉遠幼有志行,明《三禮》、《春秋三傳》、《詩》、《書》、《周易》,兼善圖緯。恒以教授為務,遠方負笈而從者,動以千計。齊南陽王綽為定州刺史,聞其名,召為博士。周武帝平齊,搜訪儒俊,暉遠首應辟命,授小學下士。隋文帝受禪,遷太常博士。太常卿牛弘每稱為《五經》庫。吏部尚書韋世康薦之,遷太學博士。
 尋與沛公鄭譯修正樂章。後復為太常博士,未幾擢為國子博士。會上令國子生通一經者,並悉薦舉,將擢用之。



 既策問訖,博士不能時定臧否。祭酒元善怪問之,暉遠曰:「江南、河北,義例不同,博士不能遍涉。學生皆持其所短,稱己所長;博士各各自疑,所以久而不決也。」



 祭酒因令暉遠考定之,暉遠攬筆便下,初無疑滯。或有不服者,暉遠問其所傳義疏,輒為始末誦之,然後出其所短。自是無敢飾非者。所試四五百人,數日便決。諸儒莫不推其通博,皆自以為不能測也。尋奉詔預修令式。文帝嘗謂群臣曰:「自古天子有女樂乎?」楊素以下,莫知所出,
 遂言無女樂。暉遠曰:「臣聞『窈窕淑女,鐘鼓樂之』,此即王者房中之樂,著於《雅》《頌》,不得言無。」帝大悅。仁壽中,卒官,朝廷嗟惜焉,賵賻甚厚,贈員外散騎常侍。



 馬光,字榮伯,武安人也。少好學,從師數十年,晝夜不息,圖書讖緯,莫不畢覽。尤明《三禮》,為儒者所宗。



 隋開皇初,征山東義學之士,光與張仲讓、孔籠、竇仕榮、張買奴、劉祖仁等俱至,並授太學博士,時人號為六儒。然皆鄙野無儀範,朝廷不之貴也。仕榮尋病死。仲讓未幾告歸鄉里,著書十卷,自云:「此書若奏,必為宰相。」又數言玄象事。州縣列上,竟坐誅。孔籠、張買奴、劉祖仁未幾亦被譴亡。
 唯光獨存。



 嘗因釋奠,帝親幸國子學,王公已下畢集,光升坐講《禮》,啟發章門。已而諸儒生以次論難者十餘,皆當時碩學。光剖析疑滯,雖辭非俊辯,而《禮》義弘贍。



 論者莫測其淺深,咸共推服。上嘉而勞焉。山東《三禮》學者,自熊安生後,唯宗光一人。初教授瀛、博間,門徒千數,至是多負笈從入長安。後數年,丁母憂歸鄉里,以疾卒于家。



 劉焯,字士元,信都昌亭人也。犀額龜背,望高視遠,聰敏沉深,弱不好弄。



 少與河間劉炫結盟為友,同受《詩》於同郡劉軌思,受《左傳》於廣平郭懋,嘗問《禮》於阜城熊安生,皆不卒業而去。武強交津橋劉智海家,素多墳籍,焯就
 之讀書,向經十載,雖衣食不繼,晏如也。遂以儒學知名,為州博士。



 隋開皇中,刺史趙煚引為從事。舉秀才,射策甲科。與著作郎王劭同修國史,兼參議律歷。仍直門下省,以待顧問。俄除員外將軍。後與諸儒於秘書省考定群言。



 因假還鄉里,縣令韋之業引為功曹。尋復入京,與左僕射楊素、吏部尚書牛弘、國子祭酒蘇威、元善、博士蕭該、何妥、太學博士房暉遠、崔崇德、晉王文學崔賾等,於國子共論古今滯義,前賢所不通者。每升坐,論難鋒起,皆不能屈。楊素等莫不服其精博。六年,運洛陽《石經》至京師,文字磨滅,莫能知者。奉敕與劉炫二人論義,深
 挫諸儒,咸懷妒恨。遂為飛章所謗,除名。



 於是優游鄉里,專以教授著述為務,孜孜不倦。賈、馬、王、鄭所傳章句,多所是非。《九章算術》、《周髀》、《七曜歷書》十餘部,推步日月之經,量度山海之術,莫不核其根本,窮其秘奧。著《稽極》十卷,《歷書》十卷,《五經述議》,並行於世。劉炫聰明博學,名亞於焯,故時人稱二劉焉。天下名儒後進,質疑受業,不遠千里而至者,不可勝數。論者以為數百年已來,博學通儒無能出其右者。然懷抱不曠,又嗇於財。不行束脩者,未嘗有所教誨,時人以此少之。



 廢主子勇聞而召之,未及進謁,詔令事蜀王。非共好也,久之不至。王聞而大怒,
 遣人枷送於蜀,配之軍防。其後典校書籍。王以罪廢,焯又與諸儒脩定禮、律,除雲騎尉。煬帝即位,遷太學博士,俄以品卑去職。數年,復被徵以待顧問。因上所著《歷書》,與太史令張胄玄多不同,被駁不用。卒,劉炫為之請謚,朝廷不許。



 劉炫,字光伯,河間景城人也。少以聰敏見稱。與信都劉焯閉戶讀書,十年不出。炫眸子精明,視日不眩,強記默識,莫與為儔。左畫圓,右畫方,口誦,目數,耳聽,五事同舉,無所遺失。周武帝平齊,瀛州刺史宇文亢召為戶曹從事。後刺史李繪署禮曹從事,以吏乾知名。



 隋開皇中,奉
 敕與著作郎王劭同修國史,俄直門下省,以待顧問。又詔諸術者修天文律歷,兼於內史省考定群言。內史令博陵李德林甚禮之。炫雖遍直三省,竟不得官,為縣司責其賦役。炫自陳於內史,內史送詣吏部。尚書韋世康問其所能,炫自為狀曰:「《周禮》、《禮記》、《毛詩》、《尚書》、《公羊》、《左傳》、《孝經》、《論語》,孔、鄭、王、何、服、杜等注,凡十三家,雖義有精粗,並堪講授;《周易》、《儀禮》、《穀梁》用功差少;史子文集,嘉言故事,咸誦於心;天文、律歷,窮核微妙。至於公私文翰,未嘗假手。」吏部竟不詳試。然在朝知名之士十餘人,保明炫所陳不謬,於是除殿內將軍。時牛弘奏購求天下遺逸
 之書,炫遂偽造書百餘卷,題為《連山易》、《魯史記》等,錄上送官,取賞而去。後有人訟之,經赦免死,坐除名。歸于家,以教授為務。廢太子勇聞而召之。既至京師,敕令事蜀王秀,遷延不往。秀大怒,枷送益州。既而配為帳內,每使執仗為門衛。



 俄而釋之,典校書史。炫因擬屈原《卜居》為《筮塗》以自寄。及秀廢,與諸儒修定五禮,授旅騎尉。



 吏部尚書牛弘建議以為《禮》:諸侯絕傍期,大夫降一等。今之上柱國雖不同古諸侯,比大夫可也,官在第二品,宜降傍親一等。議者多以為然。炫駁之曰:「古之仕者,宗一人而已,庶子不得進,由是先王重嫡。其宗子有分祿之義,
 族人與宗子雖疏遠,猶服衰三月,良由受其恩也。令之仕者,位以才升,不限嫡庶,與古既異,何降之有。令之貴者,多忽近親,若或降之,人道之疏,自此始矣。」遂寢其事。



 開皇二十年,廢國子、四門及州縣學,唯置太學,博士二人,學生七十二人。



 炫上表言學校不宜廢,情理甚切,帝不納。時國家殷盛,皆以遼東為意。炫以為遼東不可伐,作《撫夷論》以諷焉。當時莫有悟者。及大業之季,三征不剋,炫言方驗。



 煬帝即位,牛弘引炫修律令。始文帝時,以刀筆吏類多小人,年久長姦,勢使然也;又以風俗陵遲,婦人無節。於是立格:州縣佐吏,三年而代之;九品妻,無
 得再醮。炫著論以為不可,弘竟從之。諸郡置學官及流外給稟,皆發於炫。弘嘗問炫:「案《周禮》,士多而府史少,今令史百倍於前,判官減則不濟。其故何也?」



 炫曰:「古人委任責成,歲終考其殿最,案不重校,文不繁悉,府史之任,掌要目而已。今之文簿,恒慮勘覆鍛煉,若其不密,萬里追證百年舊案。故諺云:『老吏抱案死。』今古不同,若此之相懸也。事煩政弊,職此之由。」弘又問:「魏、齊之時,令史從容而已,今則不遑寧舍。其事何由?」炫曰:「齊氏立州,不過數十;三府行臺,遞相統領,文書行下,不過十條。今州三百。其繁一也。往者,州唯置綱紀,郡置守、丞,縣唯令而已,
 其所具僚,則長官自辟,受詔赴任,每州不過數十。今則不然,大小之官,悉由吏部,纖介之迹,皆屬考功。其繁二也。省官不如省事,省事不如清心,官事不省而望從容,其可得乎!」弘甚善其言而不能用。



 納言楊達舉炫博學有文章,射策高第,除太學博士。歲餘,以品卑去任。還至長平,奉敕追詣行在所。或言其無行,帝遂罷之。歸于河間。時盜賊峰起,穀食踴貴,經籍道息,教授不行。炫與妻子,相去百里,聲聞斷絕,鬱鬱不得志,乃自為贊曰:通人司馬相如、揚子雲、馬季長、鄭康成等皆自敘徽美,傳芳來葉。餘豈敢仰均先進,貽笑後昆?徒以日迫桑榆,大命
 將近,故友飄零,門徒雨散,溘死朝露,魂埋朔野。親故莫照其心,後人不見其迹。殆及餘喘,薄言胸臆,貽及行邁,傳之州里,使夫將來俊哲,知余鄙志耳。



 餘從綰髮以來,迄於白首,嬰孩為慈親所恕,捶撻未嘗加;從學為明師所矜,榎楚弗之及。暨乎敦敘邦族,交結等夷,重物輕身,先人後己。昔在幼弱,樂參長者;爰及耆艾,數接後生。學則服而不厭,誨則勞而不倦。幽情寡適,心事多違。



 內省生平,顧循終始,其大幸有四,深恨有一。



 性本愚蔽,家業貧窶,為父兄所饒,廁縉紳之末。遂得博覽典誥,窺涉今古,小善著於丘園,虛名聞於邦國。其幸一也。



 隱顯人間,
 沈浮世俗,數忝徒勞之職,久執城旦之書。名不挂於白簡,事不染於丹筆。立身立行,慚恧實多,啟手啟足,庶几可免。其幸二也。



 以此庸虛,屢動宸眷;以此卑賤,每升天府。齊鑣驥騄,比翼鵷鴻,整紬素於鳳池,記言動於麟閣。參謁宰輔,造請群公,厚禮殊恩,增榮改價。其幸三也。



 晝漏方盡,大耋已嗟,退反初服,歸骸故里。玩文史以怡神,閱魚鳥以散慮。



 觀省野物,登臨園沼,緩步代車,無事為貴。其幸四也。



 仰休明之盛世,慨道教之陵遲,蹈先儒之逸軌,傷群言之蕪穢。馳騁墳典,釐改僻謬,修撰始畢,事業適成。天違人願,途不我與,世路未夷,學校盡廢,道不
 備於當時,業不傳於身後。銜恨泉壤,實在茲乎!其深恨一也。



 時在郡城,糧餉斷絕。其門人多隨賊盜。哀炫窮乏,詣城下索炫,郡官乃出炫與之。炫為賊所將,過下城堡。未幾,賊為官軍所破,炫飢餓無所依,復投縣官。



 縣官意炫與賊相知,恐為後變,遂閉門不納。時夜冰寒,因此凍餒而死。其後門人謚曰宣德先生。



 炫性躁競,頗好俳諧,多自矜伐,好輕侮當世,為執政所醜,由是宦途不遂。



 著《論語述議》十卷、《春秋攻昧》十卷、《五經正名》十二卷、《孝經述議》五卷、《春秋述議》四十卷、《尚書述議》二十卷、《毛詩述議》四十卷,注《詩序》一卷、《算術》一卷,並所著文集,並行於
 世。



 時儒學之士,又有褚暉、顧彪、魯世達、張沖、王孝籍並知名。



 褚暉,字高明,吳郡人。以《三禮》學稱於江南。煬帝時,徵天下儒術之士,悉集內史省,相次講論。暉辯博,無能屈者,由是擢為太學博士。撰疏一百卷。



 顧彪,字仲文,餘杭人。明《尚書》、《春秋》。煬帝時,為秘書學士。撰《古文尚書義疏》二十卷,行於世。



 魯世達,餘杭人。煬帝時,為國子助教。撰《毛詩章句義疏》四十二卷,行於世。



 張沖,字叔玄,吳郡人。仕陳,為左中郎將,非其好也。乃覃
 思經典,撰《春秋義略》,異於杜氏七十餘事,《喪服義》三卷、《孝經義》三卷、《論語義》十卷、《前漢音義》十二卷。官至漢王侍讀。



 王孝籍,平原人。少好學,博覽群言,遍習《五經》,頗有文翰。與河間劉炫,同志友善。開皇中,召入秘書,助王劭修國史。劭不之禮。在省多年,不免輸稅,鬱鬱不得志,奏記於吏部尚書牛弘曰:竊以毒螫絪膚,則申旦不寐;飢寒切體,亦卒歲無聊。何則?痛苦難以安,貧窮易為戚。況懷抱之內,冰火鑠脂膏,腠理之間,風霜侵骨髓。安可齰舌緘脣,吞聲飲氣,惡呻吟之響,忍酸辛之酷哉!伏惟明尚書
 公,動哀矜之色,開寬裕之懷,咳唾足以活涸鱗,吹噓可用飛窮羽。芬椒蘭之氣,暖布帛之詞,許小人之請,聞大君之聽。雖復山川綿遠,鬼神在茲,信而有征,言無不履。猶恐拯溺遲於援手,救跌緩於扶足,待越人之舟楫,求魯燕之雲梯,則必懸於喬樹之枝,沒於深泉之底。



 夫以一介貧人,七年直省,課役不免,慶賞不霑。賣貢禹之田,供釋之之費;有弱子之累,乏強兄之產。加以慈母在堂,光陰遲暮,寒暑違闕,關山超遠。齧臂為期,前途逾邈;倚閭之望,朝夕傾對。謝相如之病,無官可以免;發梅福之狂,非仙所能避。愁疾甚乎厲鬼,人生異夫金石。營魂且
 散,恐筮予無徵;齎恨入冥,則虛緣恩顧。此乃王稽所以致言,應侯為之不樂也。潛鬢髮之內,居眉睫之間,子野未曾聞,離朱所未見。久淪東觀,留滯南史,終無薦引,永同埋殯。三世不移,雖由寂寞;十年不調,實乏知己。



 夫不世出者,聖明之君也;不萬一者,誠賢之臣也。以夫不世出而逢不萬一,小人所以為明尚書幸也。坐人物之源,運銓衡之柄,反被狐白,不好緇衣,此小人為明尚書不取也。昔荊玉未剖,刖卞和之足;百里未用,碎禽息之首。居得言之地,有能用之資,憎耳目之明,無首足之戚,憚而不為,孰知其解!夫官或不稱其能,士或未申其屈,一
 夫竊議,語流天下,勞不見圖,安能無望!倘病未及死,狂還克念,汗窮愁之簡,屬離憂之詞。託志於前脩,通心於來哲,使千載之下,哀其不遇,追咎執事,有玷清塵。則不肖之軀,死生為累,小人之罪,方且未刊。願少加憐愍,留心無忽。



 弘亦知其學業,而竟不得調。後歸鄉里,以教授為業,終于家。注《尚書》及《詩》,遭亂零落。



 論曰:古語云:「容體不足觀,勇力不足恃,族姓不足道,先祖不足稱,然而顯聞四方,流聲後胤者,其惟學乎?」信哉斯言也!梁越之徒,篤志不倦,自求諸己,遂能聞道下風,稱珍席上。或聚徒千百,或服冕乘軒,咸稽古之力也。然
 遠惟漢、魏,碩學多清通;逮乎近古,巨儒多鄙俗。文武不墜,弘之在人,豈獨愚蔽於當今,而皆明哲於往昔?在乎用與不用,知與不知耳。然曩之弼諧庶績,必舉德於鴻儒;近代左右邦家,咸取士於刀筆。縱有學優入室,勤踰刺股,名高海內,擢第甲科,若命偶時來,未有望於青紫;或數將運舛,必見棄於草澤。然則古之學者,祿在其中;今之學者,困於貧賤。明達之人,志識之士,安肯滯於所習,以求貧賤者哉!此所以儒罕通人,學多鄙俗者也。至若劉焯,德冠縉紳,數窮天象,既精且博,洞究幽微,鉤深致遠,源流不測。數百年來,斯一人而已。劉炫學實通儒,
 才堪成務,九流七略,無不該覽。雖探賾索隱,不逮於焯;裁成義說,文雅過之。並時不我與,餒棄溝壑。斯乃子夏所謂,「死生有命,富貴在天」。天之所與者聰明,所不與者貴仕,上聖且猶不免,焯、炫其如命何!孝籍徒離騷其文,尚何救也!



\end{pinyinscope}