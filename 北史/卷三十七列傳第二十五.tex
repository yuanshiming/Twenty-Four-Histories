\article{卷三十七列傳第二十五}

\begin{pinyinscope}

 韓茂皮豹子封敕文呂羅漢孔伯恭田益宗孟表奚康生楊大眼崔延伯李叔仁韓茂,字元興,安定安武人也。父耆字黃耇,永興中,自赫連屈丐來降,位常山太守,假安武侯,仍居常山之九門。卒,贈涇州刺史,謚曰成。茂年十七,膂力過人,尤善騎射。明元曾親征丁零翟猛,茂為中軍執幢。時大風,諸軍旌
 旗皆偃仆,茂於馬上持幢,初不傾倒。帝異而問之,謂左右曰:「記之。」尋徵詣行在所,以為武賁郎將。後從太武討赫連昌,大破之,以功賜爵蒲陰子,遷侍輦郎。又從破統萬,平平涼,當茂所衝,莫不應弦而殪。拜內侍長,進爵九門侯。後從征蠕蠕,頻戰大捷。與樂平王丕等伐和龍,茂為前鋒都將,戰功居多。遷司衛監,錄前後功,拜散騎常侍、殿中尚書,進爵安定公。從破薛永宗、蓋吳,轉都官尚書。從車駕南征,拜徐州刺史。還,拜侍中、尚書左僕射。文成踐阼,拜尚書令,加侍中、征南大將軍。茂沈毅篤實,雖無文學,每議論合理,為將善於撫眾,勇冠當世,為朝廷
 所稱。太安二年,領太子少師。卒,贈涇州刺史,安定王,謚曰桓。



 長子備,字延德,賜爵行唐侯,歷太子庶子、寧西將軍,典遊獵曹,加散騎常侍。襲爵安定公、征南大將軍。卒,贈雍州刺史。謚曰簡。



 備弟均,字天德,少善射,有將略。初為中散,賜爵范陽子,遷金部尚書,加散騎常侍。兄備卒,無子,均襲爵安定公、征南大將軍。歷定、青、冀三州刺史,甚有譽。廣阿澤在定、冀、相三州界,土曠人稀,多有寇盜,乃置鎮以靜之。以均在冀州,劫盜止息,除大將軍、廣阿鎮大將,加都督三州諸軍事。均清身率下,禁斷奸邪,於是趙郡屠各、西山丁零聚黨山澤以劫害為業者,均皆
 誘慰追捕,遠近震跼。先是,河外未賓,人多去就,故權立東青州,為招懷之本。新附人咸受優復,然舊人奸逃者,多往投焉。均表陳非便,朝議罷之。後均所統,劫盜頗起,獻文詔書讓之。又以五州人戶殷多,編籍不實,詔均檢括,出十餘萬戶。復授定州刺史,百姓安之。卒。謚康公。



 皮豹子,漁陽人也。少有武略。泰常中,為中散。太武時,為散騎常侍,賜爵新安侯,又拜選部尚書。後除開府儀同三司,進爵淮陽公,鎮長安,坐盜官財,徙於統萬。真君三年,宋將裴方明等侵南秦王楊難當,遂陷仇池。太武征豹子,復其爵位,尋拜使持節、仇池鎮將,督並中諸軍與
 建興公古弼等分命諸將,十道並進。



 四年正月,豹子進擊樂鄉,大破之。宋使其秦州刺史胡崇之鎮仇池,至漢中,聞官軍已西,懼不敢進。豹子與司馬楚之至濁水,擊禽崇之,盡虜其眾。仇池平。未幾,諸氐復推楊文德為主以圍仇池,古弼討平之。



 時豹子次下辨,聞圍解,欲還。弼使謂豹子曰:「賊恥其負敗,必求報復,不如陳兵以待之。」豹子以為然。尋除都督秦、雍、荊、梁、益五州諸軍事,進號征西大將軍、開府,仇池鎮將、持節、公如故。宋復遣楊文德、姜道盛寇濁水,別遣將青陽顯伯守斧山,以拒豹子。濁水城兵射殺道盛。豹子至斧山,斬顯伯,悉俘其眾。



 初,
 南秦王楊難當歸命,詔送楊氏子弟詣京師。文德以行賂得留,出奔漢中。



 宋以文德為武都王,守葭蘆城,招誘氐羌。於是武都陰平五部氐人叛應文德,詔豹子討之。文德阻兵固險,以拒豹子。文德將楊高來降,文德棄城南走,收其妻子寮屬及故武都王保宗妻公主送京師。宋白水太守郭啟玄率眾救文德,豹子大破之。啟玄、文德走還漢中。



 興安二年,宋遣蕭道成等入漢中,別令楊文德、楊頭等率氐、羌圍武都。豹子分兵將救之,聞宋人增兵益將,表狀求助。詔高平鎮將茍莫干率突騎二千以赴之,道成等乃退。征豹子為尚書,出為內都大官。宋
 遣其將殷孝祖修兩當城於清東,以逼南境。天水公封敕文擊之,不剋。詔豹子與給事中周丘等助擊之。宋瑕丘鎮遣步卒五千助戍兩當,豹子大破之。追至城下,其免者千餘人而已。既而班師。先是,河西諸胡亡匿避命,豹子討之;不捷而還,又坐免官。尋以前後戰功復擢為內都大官。卒,文成追惜之,贈淮陽王,謚曰襄。子道明襲。



 道明第八弟懷喜,文成以其名臣子,擢為侍御中散,遷侍御長。孝文初,吐谷渾拾夤部落飢窘,侵掠澆河。詔假平西將軍、廣川公,與上黨王長孫觀討拾夤。又以其父豹子昔鎮仇池,有威信,拜使持節、侍中、都督秦、雍、荊、梁、
 益五州諸軍事、本將軍、開府、仇池鎮將,假公如故。懷喜至,申布恩惠。夷人大悅,酋帥率戶歸附。置廣業、固道二郡以居之。徵為南部尚書,賜爵南康侯。



 太和元年,宋葭蘆戍主楊文度遣弟鼠據仇池,詔懷喜討鼠。鼠棄城南走。進次濁水,遂軍於覆津。文度將強大黑固守津道,懷喜部分將士,擊大黑走之。追奔,攻拔葭蘆城,斬文度,傳首京師。詔慰勉之。又詔於駱谷築城,懷喜表求待來年築城。詔責之曰:「若不時築,築而不成,成而不固,以軍法繩之。」南天水人柳旃據險不順,懷喜討滅之。後為豫州刺史,詔讓其在州寬怠,以飲酒廢事,威不禁下,遣使就
 州,決以杖罰。卒,謚曰恭公。子承宗襲。



 封敕文,代人也,本姓是賁。祖豆,位開府、冀青二州刺史、關內侯。父灊,侍御長,贈定州刺史、章武侯,謚曰隱。敕文始光初為中散,稍遷西部尚書,出為使持節、開府、領護西夷校尉、秦益二州刺史,賜爵天水公,鎮上邽。詔敕文征吐谷渾慕利延兄子拾歸於枹罕。眾少不制,詔廣川公乙烏頭等二軍與敕文會隴右。軍次武始,拾歸夜遁,敕文引軍入枹罕,虜拾歸妻子及其人戶,分徙千家於上邽,留烏頭守桴罕。



 金城邊冏,天水梁會謀反,據上邽東城南城,攻逼西城。敕文先已設備,賊乃退。冏、會復攻
 城,氐、羌一萬屯南嶺,休官、屠各及雜戶二萬餘人屯北嶺,為冏等形援。敕文設奇兵大破之,斬冏。眾復推梁會為主。安豐公閭根率軍助敕文,敕文又表求助,未及報。梁會欲謀逃遁。先是敕文掘重塹於東城之外,幾斷賊走路。



 夜半,會乃飛梯騰塹而走。敕文先嚴兵於塹外,拒鬥,從夜至旦。敕文謀於眾曰:「困獸猶鬥,而況於人。」乃以白武幡宣告賊眾,若能歸降,原其生命,應時降者六百餘人。會知人心沮壞,於是分遁。敕文縱騎騰躡,死者太半。



 略陽王元達因梁會之亂,聚黨攻城,招引休官、屠各之眾,推天水休官王官興為秦地王。敕文與臨淮公莫
 真討破之。天安元年卒,長子萬護讓爵於弟翰。于時讓者唯萬護及元氏侯趙辟惡子元伯讓其弟次興,朝廷義而許之。



 呂羅漢,本東平壽張人也,其先石勒時徙居幽州。祖顯,字子明。少好學,性廉直,鄉人有忿爭者皆就質焉。慕容垂以為河間太守。皇始初,以郡降,道武賜爵魏昌男。拜鉅鹿太守。清身奉公,妻子不免飢寒,百姓頌之曰:「時惟府君,克清克明,緝我荒土,人胥樂生,願壽無疆,以享長齡。」卒官。父溫,字晞陽。善書,好施,有文武才略。位上黨太守,有能名。卒,贈豫州刺史、野王侯,謚曰敬。



 羅漢仁厚篤
 慎,弱冠以武幹知名。父溫之為秦州司馬,羅漢隨侍。隴右氐楊難當寇上邽,鎮將元意頭知羅漢善射,共登西城樓令射。難當隊將及兵二十三人應弦而殪。賊眾轉盛,羅漢曰:「今不出戰,示敵以弱。」意頭善之,即簡千餘人,令羅漢出戰,眾皆披靡。難當大驚,會太武賜難當璽書,責其跋扈,難當還仇池。意頭具以狀聞,徵為羽林郎。上邽休官呂豐、屠各、王飛鹿等據險為逆,詔羅漢討禽之。後從征懸瓠,以功遷羽林中郎、幢將,賜爵烏程子。及南安王餘立,羅漢猶典宿衛,文成之立,羅漢有力焉。加龍驤將軍,仍幢將,進爵野王侯,拜司衛監。遷散騎常侍、殿
 中尚書,進爵山陽公。



 後為鎮西將軍、秦、益二州刺史。時仇池氐、羌反,逼駱谷,鎮將吳保元走登百頃,請援於羅漢族。羅漢帥步騎隨長孫觀,掩擊氐、羌大破之,賊眾退散。詔書慰勉之。涇州人張羌郎聚眾千人,州軍討之。不能制,羅漢擊禽之。仇池氐、羌叛逆。其賊帥蛩廉、苻忻等皆受宋官爵鐵券。略陽公伏阿奴為都將,與羅漢赴討,所在破之,禽廉、忻等。秦、益阻遠,南連仇池,西接赤水,諸羌恃險,數為叛逆,自羅漢蒞州,撫以威惠,西戎懷德,土境怗然。孝文下詔褒美之。徵拜內都大官,聽察多得其情。卒官,謚莊公。長子興祖襲爵山陽公,後例降為侯。



 孔伯恭,魏郡鄴人也。父昭,位侍中、幽州刺中、魯郡公。卒,謚曰康。伯恭以父任拜給事中,後賜爵濟陽男,進彭城公。獻文初,宋徐州刺史薛安都以彭城內附。宋遣將張永、沈攸之等擊安都。安都請援,獻文進伯恭號鎮東將軍,副尚書尉元救之。永與攸之棄船而走。伯恭以書喻下邳、宿豫城內。時攸之、吳喜公等率眾來援下邳,屯軍焦墟曲,去下邳五十餘里。伯恭密造火車攻其營,水陸俱進。攸之等既聞將戰,引軍退保樊階城。宋寧朔將軍陳顯達領眾溯清而上,以迎攸之;屯于睢、清合口。伯恭率眾度水,大破顯達。攸之聞顯達軍敗,順流退下。伯恭
 從清西與攸之合戰,大破之,吳喜公輕騎遁走。乘勝追奔八十餘里,軍資器械虜獲萬計。



 進攻宿豫,宋戍將魯僧遵棄城夜遁。又遣將孔大恆等南討淮陽,宋太守崔武仲焚城南走,遂據淮陽。皇興二年,以伯恭為散騎常侍、彭城鎮將、都督徐南北兗州諸軍事,假東海公。卒,贈鎮東大將軍、東海王,謚曰桓。



 伯恭弟伯遜,襲父爵魯郡公,位東萊鎮將、東徐州刺史。坐事免官,卒于家。



 田益宗,光城蠻也。身長八尺,雄果有將略,貌狀舉止,有異常蠻。世為四山蠻帥,受制於齊。太和十七年,遣使張超奉表歸魏。十九年,拜員外散騎常侍、都督、南司州刺
 史、光城縣伯,食蠻邑一千戶,所統守宰,任其銓置。後以益宗既度淮北,不可仍為司州,乃於新蔡立東豫州,以益宗為刺史。尋改封安昌縣伯。



 景明初,梁師寇三關,益宗遣光城太守楊興之進至陰山關。南據長風城,逆擊大破之。梁建寧太守黃天賜築城赤亭,復遣其將黃公賞屯於漴城,與長風相持。益宗命安蠻太守梅景秀與興之掎角擊討。破之,獲其二城。上表陳攻取之術。宣武納之,遣鎮南將軍元英攻義陽。益宗遣其息魯生斷梁人糧運,破梁戍主趙文興,倉米運舟,焚燒蕩盡。時樂口已南,郢、豫二州諸縣皆沒於梁,唯有義陽而
 已。梁招益宗以車騎大將軍、開府儀同三司、五千戶郡公。當時安危在益宗去就,而益宗守節不移,郢、豫克平,益宗力也。



 益宗年稍衰老,聚斂無厭。兵人患其侵擾,諸子及孫,競規賄貨。部內苦之,咸言欲叛。宣武深亦慮焉,乃遣中書舍人劉桃符宣旨慰喻,庶以安之。桃符還,啟益宗侵掠之狀。詔之曰:「聞卿息魯生在淮南貪暴,橫殺梅伏生,為爾不已,損卿誠效,可會魯生與使赴闕,當加任使。」魯生久未至。延昌中,詔以益宗為使持節、鎮東將軍、濟州刺史,常侍如故。帝慮其不受代,遣後將軍李世哲與桃符率從襲之,奄入廣陵。益宗子魯生、魯賢等奔
 於關南,招引梁兵,光城已南,皆為梁所保。世哲擊破之,復置郡戍,以益宗還。授征南將軍、金紫光祿大夫,加散騎常侍,改封曲陽縣伯。益宗生長邊地,不願內榮,雖位秩崇重,猶以為恨,表陳桃符讒毀之狀。



 詔曰:「既經大宥,不容方更為獄。」熙平初,益宗又表乞東豫,以招二子。靈太后令答不許。卒,贈征東大將軍、郢州刺史,謚曰莊。少子纂襲,位中散大夫,卒,贈東豫州刺史。



 益宗長子隨興,位弋陽、東汝南二郡太守。益宗兄興祖,位江州刺史。



 孟表,字武達,濟北蛇丘人也。自云本屬北地,號索里諸孟。青、徐內屬後,表因事南度,仕齊為馬頭太守。太和十
 八年,表據郡歸魏,除南兗州刺史,領馬頭太守,賜爵譙縣侯,鎮渦陽。後齊遣其豫州刺史裴叔業攻圍六十餘日,城中食盡,唯以朽革及草木皮葉為糧。表撫循將士,戮力固守。會鎮南將軍王肅救之,叔業乃退。初,有一南人,自云姓邊字叔珍,攜妻息從壽春投表,未及送闕,會叔業圍城。



 表後察叔珍言色頗有異,即推核,乃是叔業姑兒,規為內應。所攜妻子,並亦假妄,於北門外斬之,人情乃安。孝文嘉其誠,封汶陽縣伯,歷濟州刺史、散騎常侍、光祿大夫、齊州刺史。卒,贈兗州刺史,謚曰恭。



 奚康生,河南陽翟人也。本姓達奚,其先居代,世為部落
 大人。祖真,柔玄鎮將、內外三都大官,賜爵長進侯。卒,贈幽州刺史,謚曰簡。康生少驍武,彎弓十石,矢異常箭,為當時所服。太和初,蠕蠕頻寇,康生為前驅軍主,壯氣有聞,由是為宗子隊主。從駕征鐘離,駕旋濟淮,五將未度,齊將據渚斷津路。孝文募破中渚賊者,以為直閣將軍。康生應募,縛筏積柴,因風放火,燒其船艦,依煙直過,飛刀亂斫,投河溺死者甚眾。乃假康生直閣將軍。後以勳除太子三校、西臺直後。



 吐京胡反,自號辛支王,康生為軍主,從章武王彬討之。分為五軍,四軍俱敗,康生軍獨全。率精騎一千追胡至車突谷,詐為墜馬,胡皆謂死,爭
 欲取之。康生騰騎奮矛,殺傷數十人,射殺辛支。



 齊置義陽,招誘邊人,康生復為統軍,從王肅討之。齊將張伏護自昇樓城樓,言辭不遜。肅令康生射之,望樓射窗,扉開即入,應箭而斃。彼人見箭,皆以為狂弩。齊將裴叔業率眾圍渦陽,欲解義陽之急,詔遣高聰、元衍等援之,並敗退。帝乃遣康生馳往,一戰大破之。及壽春來降,遣康生領羽林千人,給龍廄馬兩匹,馳赴之。破走其將桓和、陳伯之。以功除征虜將軍,封安武縣男。出為南青州刺史。



 後梁郁州遣軍主徐濟寇邊,康生破禽之。時梁聞康生能引強弓,故特作大弓兩張,長八尺,把中圍尺有二寸,箭
 麤殆如今之長笛,送與康生。康生便集文武,用之平射,猶有餘力。觀者以為絕倫。弓即表送,置之武庫。後梁遣都督臨川王蕭宏勒甲十萬規寇徐州,詔授康生武衛將軍,一戰敗之。還京,召見宴會,賞帛千匹,賜驊騮御胡馬一匹。出為華州刺史,頗有聲績。轉涇州刺史,以輒用官炭瓦,為御史所劾,削除官爵。尋復之。梁直閣將軍徐玄明戍郁州,殺其刺史張稷,以城內附,詔康生迎接;賜細御銀纏槊一張,并棗柰果。面敕曰:「果者果如朕心,棗者早遂朕意。」未發間,郁州刺史復叛。及大舉征蜀,假康生安西將軍,邪趣綿竹。至隴右,宣武崩,班師。



 後除相州
 刺史,在州以天旱令人鞭石季龍畫像,復就西門豹祠祈雨,不獲,令吏取豹舌。未幾,二兒暴喪,身亦遇疾,巫以為季龍、豹之祟。徵拜光祿勛,領右衛將軍,與元叉同謀廢靈太后。遷河南尹,仍右衛、領左右。與子難娶左衛將軍侯剛女,即元叉妹夫也。叉以其通姻,深相委託,三人多宿禁內,或迭出入。叉以康生子難為千牛備身。



 康生性粗武,言氣高下。叉稍憚之,見于顏色,康生亦微懼不安。正光二年二月,明帝朝靈太后于西林園。文武侍坐,酒酣迭舞。次至康生,乃為力士舞,及於折旋。每顧視太后,舉手蹈足,嗔目頷首,為殺縛之勢。太后解其意而不
 敢言。日暮,太后欲攜帝宿宣光殿。侯剛曰:「至尊已朝訖,嬪御在南,何勞留宿?」康生曰:「至尊陛下兒,隨陛下將東西,更復訪問誰!」群臣莫敢應。靈太后自起援帝臂,下堂而去;康生大呼唱萬歲於後,近侍皆唱萬歲。明帝引前入閣,左右競相排,閣不得閉。康生奪其子難千牛刀,斫直後元思輔,乃得定。



 明帝既上殿,康生時有酒勢,將出處分;遂為叉所執,鎖於門下。至曉,叉不出,令侍中、黃門、僕射、尚書等十餘人就康生所,訊其事。處康生斬刑,難處絞刑。叉與剛並在內矯詔決之。康生如奏,難恕死從流。難哭拜辭父。康生忻子免死,慷慨了不悲泣。語其子
 云:「我不反,死,汝何為哭也?」有司驅逼,奔走赴市,時已昏闇,行刑人注刀數下,不死;於地刻截。咸言稟叉意旨,過至苦痛。嘗食典御奚混與康生同執刀入內,亦就市絞刑。



 康生久為將,及臨州,多所殺戮。而乃信向佛道,每捨居宅立寺塔,凡歷四州,皆有建置。死時年五十四。子難年十八,以侯剛婿,得停百日,竟徙安州。後尚書盧同為行臺,叉令殺之。康生於南山立佛圖三層,先死,忽夢崩壞。沙門有為解云:「檀越當不吉利,無人供養佛圖,故崩耳。」康生稱然,竟及於禍。靈太后反政,贈都督冀瀛滄三州諸軍事、驃騎大將軍、司空、冀州刺史,謚曰武貞,又追
 封壽張縣侯。子剛襲。



 楊大眼,武都氐難當之孫也。少驍捷,跳走如飛。然庶孽,不為宗親顧待,不免飢寒。太和中,起家奉朝請。時將南伐,尚書李沖典選征官,大眼往求焉,沖弗許。大眼曰:「尚書不見知,聽下官出一技。」便出長繩三丈許,繫髻而走,繩直如矢,馬馳不及。見者無不驚歎。沖因曰:「千載以來,未有逸材若此者也。」遂用為軍主。大眼顧謂同寮曰:「吾之今日,所謂蛟龍得水之秋。自此一舉,不復與諸君齊列矣。」未幾,遷統軍,從車駕征宛、葉、穰、鄧、九江、鐘離之間,所經戰陣,莫不勇冠六軍。



 宣武初,裴叔業以壽春內附,
 大眼與奚康生等率眾先入,以功封安成縣子。除直閣將軍,出為東荊州刺史。時蠻酋樊秀安等反,詔大眼為別將,隸都督李崇討平之,大眼功尤多。妻潘氏,善騎射,自詣軍省大眼。至攻戰遊獵之際,潘亦戎裝,齊鑣並驅。及至還營,同坐幕下,對諸寮佐,言笑自得。大眼時指謂諸人曰:「此潘將軍也。」



 梁武遣其將張惠紹總率眾軍,竊據宿豫。又假大眼平東將軍為別將,與都督邢巒討破之。遂與中山王英同圍鐘離。大眼軍城東,守淮橋東西道。屬水汎長,大眼所綰統軍劉神符、公孫祉兩軍夜中爭橋奔退,大眼不能禁,相尋而走。坐徙營州為兵。



 永平
 中,追其前勛,起為試守中山內史。時高肇征蜀,宣武慮梁人侵軼,乃徵大眼為太尉長史、持節、假平南將軍、東征別將,隸都督元遙,遏禦淮、肥。大眼至京師,時人思其雄勇,喜於更用,臺省門巷,觀者如市。後梁將康絢於浮山遏淮,規浸壽春。明帝加大眼光祿大夫,率諸軍鎮荊山,復其封邑。後與蕭寶夤俱征淮堰,不能克,遂於堰上流鑿渠決水而還。加平東將軍。



 大眼撫循士卒,呼為兒子,及見傷痍,為之流泣。自為將帥,恆身先兵士,當其鋒者,莫不摧拉。南賊所遣督將,皆懷畏懼。時傳言淮、泗、荊、沔之間童兒啼者,恐之云「楊大眼至」,無不即止。王肅弟
 康之初歸國也,謂大眼曰:「在南聞君之名,以為眼如車輪。及見,乃不異於人。」大眼曰:「旗鼓相望,瞋眸奮發,足使君目不能視,何必大如車輪。」當世推其驍果,以為關、張弗之過也。然征淮堰之役,喜怒無常,捶撻過度,軍士頗憾焉。識者以為性移所致。又為荊州刺史,常縛槁為人,衣以青布而射之。召諸蠻渠,指示之曰:「卿等若作賊,吾政如此相殺也。」又北淯郡嘗有虎害,大眼搏而獲之,斬其頭縣於穰市。自是荊蠻相謂曰:「楊公惡人,常作我蠻形以射之。又深山之虎,尚所不免。」遂不敢復為寇盜。在州二年,卒。大眼雖不學,恆遣人讀書而坐聽之,悉皆記
 識。令作露布,皆口授之,而竟不多識字也。



 有三子,長甑生,次領軍,次征南,皆潘氏所生,咸有父風。初,大眼徙營州,潘在洛陽,頗有失行。及為中山,大眼側生女夫趙延寶告之於大眼。大眼怒,幽潘而殺之。後娶繼室元氏。大眼之死也,甑生等問印綬所在。時元始懷孕,自指其腹謂甑生等曰:「開國當我兒襲之,汝等婢子,勿有所望。」甑生等深以為恨。及大眼喪將還京,出於城東七里,營車而宿。夜二更,甑生等開大眼棺,延寶怪而問焉,征南射殺之。元怖,走入水,征南又彎弓將射之。甑生曰:「天下豈有害母之人。」



 乃止。遂取大眼屍,令人馬上抱之,左右扶
 挾以叛。荊人畏甑生等驍武,不敢苦追,遂奔梁。



 崔延伯,博陵人也。祖壽,於彭城陷入江南。延伯少以武壯聞,仕齊為緣淮遊軍,帶濠口戍主。太和中入魏。常為統帥,膽氣絕人,兼有謀略,積勞稍進。除征虜將軍、荊州刺史,賜爵定陵男。荊州土險,蠻左為寇,每有聚結,延伯輒自討之,莫不摧殄。由是穰土帖然,無敢為患。永平中,轉幽州刺史。



 梁遣左遊擊將軍趙祖悅率眾偷據硤石,詔延伯為別將,與都督崔亮討之。亮令延伯守下蔡。延伯與別將伊瓫生挾淮為營。延伯遂取車輪,去輞,削銳其輻,兩兩接對,揉竹為糸亙,貫連相屬,並十餘道。橫水為
 橋,兩頭施大鹿盧,出沒任情,不可燒斫,既斷祖悅走路,又令舟舸不通。由是梁軍不能赴救,祖悅合軍咸見俘虜。



 於軍拜征南將軍、光祿大夫。



 延伯與楊大眼等至自淮陽,靈太后幸西林園引見,謂曰:「卿等志尚雄猛,皆國之名將。比平硤石、公私慶快,此乃卿等之功也。但淮堰仍在,宜須預謀,故引卿等,親共量算,各出一圖,以為後計。」大眼對曰:「臣輒謂水陸二道一時俱下,往無不剋。」延伯曰:「既對聖顏,答旨宜實。水南水北,各有溝瀆,陸地之計,如何可前。愚臣短見,願聖心思水兵之勤,若給復一年,專習水戰,脫有不虞,召便可用。」靈太后曰:「卿之所言,
 深是宜要,當敕如請。」二年,除并州刺史。



 在州貪污,聞於遠近。還為金紫光祿大夫,出為鎮南將軍,行岐州刺史,假征西將軍。賜驊騮馬一匹。正光五年秋,以往在揚州,建淮橋之勛,封當利縣男,改封新豐子。時莫折念生兄天生下隴東寇,征西將軍元志為天生所禽,賊眾甚盛,進屯黑水。詔延伯為使持節、征西將軍、西道都督。行臺蕭寶夤與延伯結壘馬嵬,南北相去百餘步。延伯曰:「今當仰為明公參賊勇怯。」延伯選精兵數千,下度黑水,列陣而進,以向賊營。寶夤率騎於水東尋原西北,以示後繼。於時賊眾大盛,水西一里,營營連接。延伯徑至賊壘,
 揚威脅之,徐而還退。賊以延伯眾少,開營競追,眾過十倍,臨水逼蹙。寶夤親觀之,懼有虧損。延伯不與其戰,身自殿後,抽眾東度,轉運如神。須臾濟盡,徐乃自度。賊徒奪氣,相率還營。寶夤大悅,謂宮屬曰:「崔公,古之關、張也,今年何患不制賊。」延伯馳見寶夤曰:「此賊非老奴敵,公但坐看。」後日,延伯勒眾而出,寶夤為後拒。天生悉眾來戰,延伯身先士卒,陷其前鋒。於是驍銳競進,大破之,俘斬十餘萬,追奔及於小隴。秦賊勁強,諸將所憚,初議遣將,咸云非延伯無以定之,果能克敵。詔授左衛將軍,餘如故。



 於時萬俟醜奴、宿勤明達等寇掠涇州。先是盧祖
 遷、伊瓫生數將,皆以元志前行之始,同時發雍,從六陌道將取高平。志敗,仍停涇部。延伯既破秦賊,乃與寶夤率眾會於安定。甲卒十二萬,鐵馬八千匹,軍威甚盛。時醜奴置營涇州西北七十里當原城。時或輕騎暫來挑戰,大兵未交,便示奔北。延伯矜功負勝,遂唱議先驅。



 伐木別造大排,內為鎖柱,教習強兵;負而趨走,號為排城。戰士在外,輜重居中,自涇州緣原北上。眾軍將出討賊。未戰之間,有賊數百騎詐持文書,云是降簿,乞緩師。寶夤、延伯謂其事實,逡巡未鬥。俄而宿勤明達率眾自東北而至,乞降之賊從西競下,諸軍前後受敵。延伯上馬
 突陣,賊勢摧挫,便爾逐北,徑造其營。賊本輕騎,延伯軍兼步卒,兵力疲怠,賊乃乘間得入排城。延伯軍大敗,死傷者將有二萬。寶夤斂軍退保涇州。延伯修繕器械,購募驍勇,復從涇州西進,去賊彭坑谷柵七里結營。延伯恥前挫辱,不報寶夤,獨出襲賊,大破之。俄頃間平其數柵。賊皆逃迸。見兵人采掠,散亂不整,還來衝突,遂大奔敗。延伯中流矢,為賊所害,士卒死者萬餘人。



 延伯善將撫,能得眾心,與康生、大眼為諸將之冠。延伯末路,功名尤重。時大寇未平而延伯死,朝野歎懼焉。贈使持節、車騎大將軍、儀同三司、定州刺史,謚曰武烈。



 李叔仁,隴西人也。驍健有武力,前後數從征討,以功賜爵獲城鄉男。梁豫州刺史王超宗內侵,叔仁時為兼統軍,隸揚州刺史薛真度。真度遣叔仁討超宗,大破之。以功累遷洛州刺史,假撫軍將軍。後以軍功封陳郡公,又除光祿大夫、朔州刺史。齊州廣川人劉執、清河太守邵懷,聚眾反,自署大行臺。詔叔仁為都督,討平之。除鎮西將軍、金紫光祿大夫,轉車騎大將軍、儀同三司。邢果反於青州,叔仁為大都督,出討於淮,失利而還。永安三年,坐事除名,尋復官爵。節閔帝初,加散騎常侍、開府。後除涼州刺史。遣使密通款於東魏,事覺見殺。叔仁所用之
 槊,長大異於常槊,時人壯之。



 論曰:韓茂、皮豹子、封敕文、呂羅漢、孔伯恭之為將也,皆以沈勇篤實,仁厚扶眾,功成事立,不徒然矣。與夫茍要一戰之利,僥幸暫勝之名,豈同年而語也。



 田益宗蠻夷荒帥,翻然效款,終於懷金曳紫,不其美歟。孟表之致名位,不徒然也。



 夫人主聞鞞鼓之響,則思將帥之臣,何則?夷難平暴,折沖禦侮,為國之所系也。



 奚康生等俱以熊虎之姿,奮征伐之氣,亦一時之驍猛,壯士之功名乎。



\end{pinyinscope}