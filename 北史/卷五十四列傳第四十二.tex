\article{卷五十四列傳第四十二}

\begin{pinyinscope}

 孫
 騰高隆之司馬子如子消難裴藻兄子膺之竇泰尉景婁昭兄子睿厙狄干孫士文韓軌段榮子韶孝言斛律金子光羨孫騰,字龍雀,咸陽石安人也。祖通,仕沮渠氏,為中書舍人。沮渠氏滅,因徙居北邊。及騰貴,魏朝贈司徒。父機,贈太尉。騰少質直,明解吏事。魏正光中,北方擾,歸爾朱榮。尋為齊神武都督長史。神武為晉州,又引為長史,封石
 安縣伯。



 及起兵於信都,常以誠款預謀策。累遷郡公,入為侍中,尋兼尚書左僕射。時魏京兆王愉女平原公主寡,騰願尚之,而公主欲侍中封隆之。騰妒隆之,遂相間構。神武啟免騰官,俄而復之。與斛其椿同掌機密,隆之見忌慮禍,奔晉陽。神武入討椿,留騰行并州事。入為尚書左僕射,內外之事,騰咸知之。兼司空,除侍中,兼尚書令。時西魏攻南袞州,詔騰率諸將討之。騰性怯無威略,失利而還。又除司徒,餘官如故。初北境亂,騰亡一女。及貴,推訪不得,疑其為人婢。及為司徒,奴婢訴良者皆免之,願免千人,冀得其女。神武知之大怒,解司徒。尋為尚
 書左僕射、太保,仍侍中,遷太傅。



 初,博陵崔孝芬取貧家子賈氏為養女。孝芬死,其妻元更適鄭伯猷,攜賈於鄭氏。賈有色,騰納之為妾。其妻袁死,騰以賈有子,正以為妻,詔封丹陽郡君。復請以袁氏爵回授其女。其違禮肆情,多此類也。



 騰早依神武,神武深信待之,置於魏朝,寄以心腹。遂志氣驕盈,與奪自己。



 納賄不知紀極,官贈非財不行。肴藏銀器,盜為家物,親狎小人,專為聚斂。與高岳、高隆之、司馬子如,號四貴。非法專恣,騰為甚焉。神武、文襄,屢加誚讓,終不悛改,朝野深非笑之。武定六年薨,贈太師、開府、錄尚書事,謚曰文。天保初,以騰佐命,詔祭
 告其墓。皇建中,配饗神武廟庭。



 子鳳珍嗣,性庸暗,卒於儀同三司。



 高隆之,字延興,洛陽人也。為閹人徐成養子。少時,以賃升為事。或曰父乾為姑婿高氏所養,因從其姓。隆之後有參定功。神武命為弟,仍云勃海蓚人。幹贈司徒公。隆之身長八尺,美鬚髯,深沉有志氣。初,行臺于暉引為郎中,與神武深相結託。後從起兵於山東,累遷並州刺史,入為尚書右僕射。時初給人田,權貴皆占良美,貧弱咸受脊薄,隆之啟神武,更均平之。又領營構大將,以十萬夫徹洛陽宮殿,運於鄴,構營之制,皆委隆之。增築南城,
 周二十五里。以漳水近帝城。起長堤以防汎溢。又鑿渠引漳水,周流城郭,造水碾磑並有利於時。



 魏自孝昌之後,天下多難。刺史、太守皆為當部都督,雖無兵事,皆立佐僚,所在頗為煩擾。隆之請非實邊要,見兵馬者,悉斷之。又朝貴多假常侍以取貂蟬之飾,隆之自表解侍中,並陳諸假侍中服者,請亦罷之。詔皆如表。自軍國多事,冒名竊官者,不可勝數,隆之奏請檢括,旬日獲五萬餘人。而群小言雚囂,隆之懼而止。詔監起居事,進位司徒。武定中,除尚書令,遷太保。文襄作宰,風俗肅清。



 隆之時有受納,文襄於尚書省大加責讓。齊受禪,進爵為王。尋以
 本官錄尚書事,領大宗正卿,監國史。隆之性好小巧,至於公家羽儀,百戲服制,時有改易,不循典故。時論非之。於射堋土上立三人像,為壯勇之勢。文宣曾至東山,因射,謂隆之曰:「堋上可作猛獸,以存古義,何為終日射人?」隆之無以對。



 先是,文襄委任崔暹、崔季舒等。及文襄崩,隆之啟文宣,並欲害之,不見許。



 文宣以隆之舊齒,委以政事。隆之子淫於楊遵彥前妻,帝妹也,故遵彥讒毀日至。



 崔季舒等仍以前隙,譖云:「隆之每見訴訟者,輒加哀矜之意,以示非己能裁。」



 文宣以其受任既久,知有冤狀,便宜申浟,何過要名,非大臣義。天保五年,禁止尚書省。
 隆之曾與元昶宴,語昶曰:「與王交遊,當死生不相背。」人有密言之者。



 又帝未登庸日,隆之意常侮帝。帝將受禪,大臣咸言未可,隆之又在其中。帝深銜之。因此大怒,罵曰:「徐家老公!」令壯士築百餘拳,放出。渴,將飲水,人止之,隆之曰:「今日何在!」遂飲之。因從駕,死於路中。贈太尉、太保、陽夏王,竟不得謚。



 隆之雖不學涉,而欽尚文雅,搢紳名流,必存禮接。寡姊為尼,事之如母。訓督諸子,必先文義。世以此稱之。



 文宣末年,多猜害,追忿隆之,執其子司徒中兵慧登等二十人於前。慧登言乞命,帝曰:「不得已。」以鞭扣鞍,一時頭絕,並投之漳水。發隆之塚,出屍,其貌
 不敗。斬骸骨焚之,棄於漳流。天下冤之。隆之嗣遂絕。乾明中,詔其兄子子遠為隆之後,襲爵陽夏王,還其財產。



 隆之見信神武,性陰毒,儀同三司崔孝芬以結婚姻不果,太僕卿任集同知營構,頗相乖異;瀛州刺史元晏請託不遂。並構成其罪,誅害之,終至家門殄滅。論者謂有報應焉。



 司馬子如,字遵業,自云河內溫人也,徙居雲中,因家焉。子如初為懷朔鎮省事,與齊神武相結託,分義甚深。孝昌中,北州淪陷,子如南奔肆州,為爾硃榮所禮,封平遙子,稍遷大行臺郎。榮死,隨榮妻子與爾朱世隆等走出
 京城。節閔帝立,以前後功,進爵陽平郡公。神武入洛,以為大行臺尚書,朝夕左右,參知軍國。天平初,除尚書左僕射、開府,與高岳、孫騰、高隆之等共知朝政,甚見信重。神武鎮晉陽,子如時往謁見。及還,神武、武明后俱有齎遺,率以為常。



 子如性既豪爽,兼恃恩舊,簿領之務,與奪任情,公然受納。興和中,以北道行臺巡檢諸州守令已下,至定州,斬深澤令;至冀州,斬東光令,皆稽留時刻,致之極刑。進退少不合旨者,便令武士頓曳,白刃臨頸。士庶惶懼,不知所為。轉尚書令。及文襄輔政,以賄為御史中尉崔暹劾,在獄一宿而髮皆白。辭曰:「司馬子如本從
 夏州策一杖投相王,王給露車一乘,觠牸牛犢。犢在道死,唯觠角存。



 此外,皆人上取得。」神武書敕文襄曰:「馬令是吾故舊,汝宜寬之。」文襄駐馬行街,以出子如,脫其鎖。子如懼曰:「非作事邪?」於是,除削官爵。神武後見之,哀其憔悴,以膝承其首,親為擇虱,賜酒百瓶,羊五百口,粳米五百石。子如曰:「無事尚被囚幾死,若受此,豈有生路邪?」未幾,起行冀州事,能自改厲,甚有聲譽。詔復官爵,別封野王縣男。齊受禪,以翼贊功,別封須昌縣公。尋除司空。



 子如性滑稽,不事檢裁,言戲穢褻,識者非之。而事姊有禮,撫諸兄子慈篤,當時名士,並加欽愛,復以此稱之。然
 素無鯁正,不能以平道處物。文襄時,中尉崔暹、黃門郎崔季舒俱被任用。文襄崩,暹等赴晉陽,子如以糾劾之釁,乃啟文宣,言其罪,勸帝誅之。後子如以馬度關,為有司所奏。文宣讓之曰:「崔暹、季舒事朕先世,有何大罪,卿令我殺之!」因此免官。久之,猶以先帝之舊,拜太尉。尋以疾薨。贈太師、太尉,謚曰文明。長子消難嗣。



 消難字道融。幼聰慧,微涉經史,有風神,好自矯飾,以求名譽。子如既當朝貴盛,消難亦愛賓客,邢子才、王元景、魏收、陸仰、崔瞻等皆遊其門。稍遷光祿卿,出為北豫州刺史。



 文宣末年,昏虐滋甚,消難常有自全之謀,曲意撫
 納,頗為百姓所附。不能廉潔,為御史所劾。又尚公主,而情好不睦,公主愬之。屬文宣在并州,驛召上黨王煥,煥懼害,斬使者東奔,鄴中大擾,後竟獲於濟州。煥之初走,朝士疑赴成皋,云:「若與司馬北豫連謀,必為國患。」此言達於文宣,頗見疑。消難懼,密令所親人河東裴藻間行入關,請降。



 入周,封滎陽郡公,累遷大司寇。從武帝東伐,還除梁州總管。大象初,遷大後丞,女為靜帝后。尋出為雲阜州總管。及隋文帝輔政,消難乃與蜀公尉遲回合勢舉兵,使其子永質於陳,以求援。隋文帝命襄州總管王誼討之,消難奔陳。位司空,隨郡公。



 初,隋武元帝之迎消
 難,結為兄弟,情好甚篤,隋文每以叔禮事之。及平陳,消難至,特免死配為樂戶,二旬而免。猶以舊恩,特被引見。尋卒於家。



 消難性貪淫,輕於去就,故世言反覆者,皆以方之。其妻高,齊神武女也,在鄴極加禮敬,入關便相棄薄。及赴雲阜州,留妻及三子在京。妻言於文帝曰:「滎陽公攜寵自隨,必不顧妻子,願防慮之。」及消難入陳,高母子因此獲免。子譚,即高氏所生,以消難勳,拜儀同大將軍,坐消難除名。



 裴藻字文芳。少機辨,有不羈之志,為子如太傅主簿。消難鎮北豫,又以為中兵參軍。入周,封聞喜縣男,除晉州
 刺史。



 子如兄纂。纂長子世雲,輕險無行。累遷潁州刺史,肆行姦穢。將見推,懼,遂從侯景。文襄猶以子如恩舊,免其諸弟死罪,徙北邊。世雲以侯景敗於渦陽,復有異志,為景所殺。世雲弟膺之。



 膺之字仲慶。美鬚髯,有風貌,好學,厚自封植,神氣甚高。歷中書、黃門侍郎。天平中,叔父子如執鈞當軸。膺之既宰相猶子,兼自有名望,所與遊集,盡一時名流。與邢子才、王景等,並為莫逆之交。及兄世雲陷於逆亂,期親皆應誅。膺之及諸弟並有人才,為朝廷所惜,文襄特減死徙近鎮。文宣嗣業,得還。齊受禪,子如別封須昌縣公,迴授膺之。子如撫愛甚慈,膺之昆
 季,事之如父。性方古,不會俗舊。與楊愔同為黃門郎。至愔為尚書令,抗禮如初。愔嘗有從姊慘,尚書卿尹皆跪弔,膺之執手而出。曾路逢愔,威儀道引,乃於樹下側避之。愔於車望見,令呼謂曰:「兄何意避弟?」膺之曰:「我自避赤棒,本不避卿。」愔甚重之。然以其疏簡傲物,竟天保間,淪滯不齒。乾明中,除衛尉少卿,遷國子祭酒。河清末,拜金紫光祿大夫。患泄痢,積年不起。武平中,就家拜儀同三司。班台之貴,近世專以賞勳勤,膺之雖為猥雜,名器猶重。初,司徒趙彥深起自孤微,為子如管記,膺之甚相忽略,不為之禮。及彥深為宰相,朝士輻水奏,膺之自念,故
 被延請,永不至門,每與相見,捧袂而已。太常卿段孝言,左丞相孝先之弟也,位望甚隆,嘗詣其弟幼之,舉座傾敬。膺之時牽疾,在外齋馮几而坐,不為動容。直言:「我患痢久,太常不得致怪。」黃門郎陸杳,貴遊後進,膺之嘗與棋。杳忽後至,寒溫而已,棋遂輟。園宅閑素,門無雜客,性不飲酒,而不愛重賓遊。病久,不復堪讀書,或以奕棋永日。名士有素懷者,時相尋候。無雜言,唯論經史。好讀《太玄經》,又注揚雄《蜀都賦》。每云:「我欲與揚子雲周旋。」患痢十七年,竟不愈。齊亡歲,以痢疾終。



 膺之弟子瑞,為御史中丞,正色舉察,為朝廷所許。以疾去職,就拜祠部尚書。



 卒,
 贈儀同三司、瀛州刺史,謚曰文節。子瑞妻,陸令萱妹。及令萱得寵於後主,重贈子瑞開府儀同三司、中書監、溫縣伯。諸子亦並居顯職:同遊,給事黃門侍郎;同回,太常少卿;同憲,通直常侍。同遊終為佳吏,隨開皇中,為尚書戶部侍郎,卒於遂州刺史。



 子瑞弟幼之,清貞有行。武平末,為大理卿。開皇中,卒於眉州刺史。



 竇泰,字世寧,太安捍殊人也。本出清河觀津胄。祖羅,魏統萬鎮將,因居北邊。父樂,魏末破六韓拔陵為亂,與鎮將楊鈞固守,遇害。泰貴,追贈司徒。初,泰母夢風雷暴起,若有雨狀。出庭觀之,見電光奪目,駛雨霑灑。寤而驚汗,
 遂有娠。期而不產,大懼。有巫曰:「度河湔裙,產子必易。」便向水所。忽見一人曰:「當生貴子,可徙而南。」泰母從之,俄而生泰。及長,善騎射,有勇略。泰父兄戰歿於鎮,泰身負骸骨歸爾朱榮。以從討邢杲功,賜爵廣阿子。神武之為晉州,請泰為鎮城都督,參謀軍事。累遷侍中、京畿大都督,尋領御史中尉。泰以勛戚居臺,雖無多糾舉,而百僚畏懼。天平三年,神武西討,令泰自潼關入。四年,泰至小關,為周文帝所襲,眾盡沒,泰自殺。初,泰將發鄴,鄴有惠化尼,謠云:「竇行臺,去不迴。」未行之前夜,三更,忽有朱衣冠幘數千人入臺,云收竇中尉。宿直兵吏皆驚。其人入
 數屋。俄頃而去。旦視關鍵不異,方知非人,皆知其必敗。贈大司馬、太尉、錄尚書事,謚曰武貞。



 泰妻,武明婁后妹也。泰雖以親見待,而功名自建。齊受禪,祭告其墓。皇建初,配享神武廟庭。子孝敬嗣,位儀同三司。



 尉景,字士真,善無人也。泰、漢置尉堠官,其先有居此職者,因以氏焉。景性溫厚,頗有俠氣。魏孝昌中,北鎮反,景與神武入杜洛周中,仍共歸爾朱榮。以軍功,封博野縣伯。後從神武起兵信都。韓陵之戰,唯景所統失利。神武入洛,留景鎮鄴。尋進封為公。景妻常山君,神武之姊也。以勳戚,每有軍事,與厙狄干常被委重。而不能忘懷財
 利,神武每嫌責之。轉冀州刺史,又大納賄,發夫獵,死者三百人。厙狄干與景在神武坐,請作御史中尉。神武曰:「何意下求卑官?」乾曰:「欲捉尉景。」神武大笑,令優者石董桶戲之。董桶剝景衣曰:「公剝百姓,董桶何為不剝公?」神武誡景曰:「可以無貪也。」景曰:「與爾計生活孰多,我止人上取,爾割天子調。」神武笑不答。改封長樂郡公,歷位太保、太傅。坐匿亡人,見禁止。使崔暹謂文襄曰:「語阿惠,兒富貴,欲殺我邪?」神武聞之泣,詣闕曰:「臣非尉景無以至今日。」三請,帝乃許之。於是黜為驃騎大將軍、開府儀同三司。



 神武造景,景恚,臥不動,叫曰:「殺我時趣邪?」常山君謂
 神武曰:「老人去死近,何忍煎迫至此!」又曰:「我為爾汲水,胝生。」因出其掌。神武撫景,為之屈膝。先是,景有果下馬,文襄求之,景不與,曰:「土相扶為墻,人相扶為王。



 一馬亦不得畜而索也?」神武對景及常山君責文襄而杖之。常山君泣救之,景曰:「小兒慣去,放使作心腹,何須乾啼濕哭,不聽打邪?」尋授青州刺史,操行頗改,百姓安之。征授大司馬,遇疾,薨於州。贈太師、尚書令。齊受禪,以景元勳,詔祭告其墓。皇建初,配享神武廟庭,追封長樂王。



 子粲,少歷顯職,性粗武。天保初,封厙狄乾等為王,粲以父不預王爵,大恚恨,十餘日閉門不朝。帝怪,遣使就宅問之。
 隔門謂使人曰:「天子不封粲父作王,粲不如死。」使云:「須開門受敕。」粲遂彎弓隔門射。使者以狀聞之,文宣使段韶諭旨。粲見韶,唯撫膺大哭,不答一言。文宣親詣其宅慰之,方復朝請。尋追封景長樂王,粲襲爵。位司徒、太傅,薨。



 子世辨嗣。周師將入鄴,令世辨率千餘騎覘候。出滏口,登高阜西望,遙見群鳥飛起,謂是西軍旗幟,即馳還。比至紫陌橋,不敢顧。隋開皇中,卒於浙州刺史。



 婁昭,字菩薩,代郡平城人也,武明皇后之母弟也。祖父提,雄傑有識度,家僮千數,牛馬以谷量。性好周給,士多歸附之。魏太武時,以功封真定侯。父內干,有武力,未仕
 而卒。昭貴,魏朝贈司徒。齊受禪,追封太原王。昭方雅正直,有大度深謀,腰帶八尺,弓馬冠時。神武少親重之,昭亦早識人雄,曲盡禮敬。數隨神武獵,每致請,不宜乘危歷險。神武將出信都,昭贊成大策,即以為中軍大都督。



 從破爾朱兆於廣阿,封安喜縣伯,改濟北公,又徙濮陽郡公,授領軍將軍。魏孝武將貳於神武,昭以疾辭還晉陽。後從神武入洛。兗州刺史樊子鵠反,以昭為東道大都督討之。子鵠既死,諸將勸昭盡捕誅其黨。昭曰:「此州無狀,橫被殘賊,其賊是怨,其人何罪?」遂皆捨焉。後轉大司馬,仍領軍。遷司徒,出為定州刺史。昭好酒,晚得偏風,
 雖愈,猶不能處劇務。在州,事委僚屬,昭舉其大綱而已。薨於州,贈假黃鉞、太師、太尉,謚曰武。齊受禪,詔祭告其墓,封太原王。皇建初,配享神武廟庭。



 長子仲達嗣,改封濮陽王。



 次子定遠,少歷顯職。外戚中,偏為武成愛狎,別封臨淮郡王。武成大漸,與趙郡王等同受顧命,位司空。趙郡王之奏黜和士開,定遠與其謀。遂納士開賄賂,成趙郡之禍,其貪鄙如此。尋除瀛州刺史。初,定遠弟季略,穆提婆求其伎妾,定遠不許。因高思好作亂,提婆令臨淮國郎中金造遠陰與思好通。後主令開府段暢率三千騎掩之,令侍御史趙秀通至州,以贓貨事劾定遠。定
 遠疑有變,遂縊而死。



 昭兄子睿。睿字佛仁。父拔,魏南部尚書。睿幼孤,被叔父昭所養。為神武帳內都督,封掖縣子。累遷光州刺史。在任貪縱,深為文襄所責。後改封九門縣公。



 齊受禪,除領軍將軍,別封安定侯。睿無他器幹,以外戚貴幸,縱情財色。為瀛州刺史,聚斂無厭。皇建初,封東安王。大寧元年,進位司空。平高歸彥於冀州,還拜司徒。河清三年,濫殺人,為尚書左丞宋仲羨彈奏,經赦乃免。尋為太尉,以軍功進大司馬。武成至河陽,仍遣總偏師赴縣瓠。睿在豫境,留停百餘日,專行非法。



 詔免官,以王還第。尋除太尉,薨,贈大司
 馬。



 子子產嗣,位開府儀同三司。



 厙狄干,善無人也。曾祖越豆眷,魏道武時,以功割善無之西臘汗山地方百里以處之。後率部落北遷,因家朔方。乾鯁直少言,有武藝。魏正光初,除掃逆黨,授將軍,宿衛於內。以家在寒鄉,不宜毒暑,冬得入京師,夏歸鄉里。孝昌元年,北邊擾亂,奔雲中,為刺史費穆送於爾朱榮。以軍主隨榮入洛。後從神武起兵,破四胡於韓陵,封廣平縣公,尋進郡公。河陰之役,諸將大捷,唯干兵退。神武以其舊功,竟不責黜。尋轉太保、太傅。及高仲密以武牢叛,神武討之,以乾為大都督,前驅。乾上道不過家,見侯
 景,不遑食,景使騎追饋之。時周文自將兵至洛陽,軍容甚盛。諸將未欲南度,乾決計濟河,神武大兵繼至,遂大破之。還為定州刺史。



 不閑吏事,事多煩擾,然清約自居,不為吏人所患。遷太師。天平初,以乾元勳佐命,封章武郡王,轉太宰。乾尚神武妹樂陵長公主,以親地見待。自預勤王,常總大眾,威望之重,為諸將所伏。而最為嚴猛。曾詣京師,魏譙王元孝友於公門言戲過常,無能面折者,干正色責之,孝友大慚,時人稱善。薨,贈假黃鉞、太宰,給轀輬車,謚曰景烈。



 乾不知書,署名為乾字,逆上畫之,時人謂之穿錘。又有武將王周者,署名先為吉,而後成
 其外。二人至孫,始並知書。乾,皇建初配享神武廟庭。



 子伏敬,位儀同三司卒,子士文嗣。



 士文性孤直,雖鄰里至親,莫與通狎。在齊,襲封章武郡王,位領軍將軍。周武帝平齊,山東衣冠多來迎,唯士文閉門自守。帝奇之,授開府儀同三司、隨州刺史。隋文受禪,加上開府,封湖陂縣子,尋拜貝州刺史。性清苦,不受公料,家無餘財。其子嘗啖官廚餅,士文枷之於獄累日,杖之二百,步送還京。僮隸無敢出門。



 所買鹽菜,必於外境。凡有出入,皆封署其門,親故絕迹,慶弔不通。法令嚴肅,吏人股戰,道不拾遺。有細過,必深文陷害。嘗入朝,遇上賜公卿入左藏,任取
 多少。人皆極重,士文獨口銜絹一匹,兩手各持一匹。上問其故,士文曰:「臣口手俱足,餘無所須。」上異之,別賞遺之。



 士文至州,發摘姦諂,長吏尺布斗粟之贓,無所寬貸。得千人,奏之,悉配防嶺南。親戚相送,哭聲遍於州境。至嶺南遇瘴厲,死者十八九。於是父母妻子,唯哭士文。士文聞之,令人捕搦,捶楚盈前而哭孝彌甚。司馬京兆韋焜、清河令河東趙達,二人並苛刻,唯長史有惠政。時人語曰:「刺史羅殺政,司馬蝮蛇瞋,長史含笑判,清河生吃人。」上聞,歎曰:「士文暴過毒獸!」竟坐免。未幾,為雍州長史。謂人曰:「我向法深,不能窺候貴要,無乃必死此官!」及下
 車,執法嚴正,不避貴戚,賓客莫敢至門,人多怨望。



 士文從妹為齊氏嬪,有色,齊滅後,賜薛公長孫覽。覽妻鄭氏妒,譖之文獻后,令覽離絕。士文恥之,不與相見。後應州刺史唐君明居母憂,娉以為妻。由是君明、士文並為御史劾。士文性剛,在獄數日,憤恚而死。家無餘財,有三子,朝夕不繼,親賓無贍之者。



 韓軌,字伯年,太安狄那人也。少有志操,深沈,喜怒不形於色。神武鎮晉州,引為鎮城都督。及起兵於信都,軌贊成大策。從破爾朱兆於廣阿,又從韓陵陣,封平昌縣侯。仍督中軍,從破爾朱兆於赤谼嶺。再遷秦州刺史,甚得
 邊和。神武巡秦州,欲以軌還,仍賜城人戶別絹布兩疋,州人田昭等七千戶皆辭不受,唯乞留軌。



 神武嘉歎,乃留焉。頻以軍功,進封安德郡公,遷瀛州刺史。在州聚斂,為御史糾劾,削除官爵。未幾,復其安德郡公。歷位中書令、司徒。齊受禪,封安德郡王。



 軌妹為神武所納,生上黨王渙,復以勛庸,歷登台鉉,常以謙恭自處,不以富貴驕人。後拜大司馬,從文宣征蠕蠕,在軍暴疾,薨。贈假黃鉞、太宰、太師,謚曰肅武。皇建初,配享文襄廟庭。



 子晉明嗣。天統中,改封東萊王。晉明有俠氣,諸勳貴子孫中,最留心學問。



 好酒誕縱。招引賓客,一席之費,動至萬錢,猶恨
 儉率。朝廷欲處之貴要地,必以疾辭,告人云:「廢人飲美酒,對名勝。安能作刀筆吏,披反故紙乎?」武平末,除尚書左僕射,百餘日,便謝病解官。



 段榮,字子茂,姑臧武威人也。祖信,仕沮渠氏。後入魏,以豪族徙北邊,仍家於五原郡。父連,安北府司馬。榮少好歷術,專意星象。正光中,謂人曰:「吾今觀玄象,察人事,不及十年,當有亂矣。亂起此地,天下因此橫流,無可避也。」



 未幾如言。榮初之杜洛周,因奔爾朱榮。及神武起兵,榮贊成之。神武南討鄴,留榮鎮信都,仍授定州刺史。時攻鄴未克,榮轉輸無闕。神武入洛,論功封姑臧縣侯,轉授
 瀛州刺史。榮妻,武明皇后長姊也,榮恐神武招私親之議,固推諸將,竟不之州。尋歷相、濟、秦三州,所在百姓愛之。神武將圖關右,榮稱未可,及渭曲敗,神武曰:「不用段榮言,以至於此。」尋除山東大行臺,領本州流人大都督,甚得物情。卒,贈太尉,謚曰昭景。皇建初,配享神武廟庭。二年,重贈大司馬、尚書令、武威王。長子韶嗣。



 韶字孝先,少工騎射,有將領才略。以武明皇后甥,神武益器愛之,常置左右,以為心腹,領親信都督。



 神武拒爾朱兆於廣阿,憚兆兵眾。韶曰:「所謂眾者,得眾人之死;所謂彊者,得天下之心。爾硃裂冠毀冕,拔本塞原。芒山之
 會,搢紳何罪?殺主立君,不脫旬朔。天下從亂,士室而九。王躬昭德義,誅君側之惡,何往而不克哉!」神武曰:「吾雖以順討逆,恐無天命。」韶曰:「聞小能敵大,小道大淫,皇天無親,唯德是輔。今爾朱外賊天下,內失善人,智者不為謀,勇者不為斗。不肖失職,賢者取之,復何疑也!」遂與挑戰,敗之。頻以軍功,封下洛縣男,後回賜父爵姑臧縣侯。



 芒山之役,為賀拔勝所窘,韶從傍馳馬反射,斃其馬,追騎不敢進,遂免。賜鞍下馬並金,進爵為公。及征玉壁,攻城未下,神武不豫。謂大司馬斛律金、司徒韓軌、左衛將軍劉豐等曰:「吾每謂孝先論兵,殊有英略,若比來用其
 謀,可無今日之勞矣。吾患危篤,欲委孝先以鄴下事,若何?」金等咸曰:「知臣莫若君,實無出孝先者。」仍令韶從文宣鎮鄴,召文襄赴軍顧命。文襄以孝先為託,令軍旅大事,並與籌之。及神武崩,侯景反,文襄還鄴,留韶守晉陽,委以軍事。加驃騎大將軍、開府儀同三司。文宣受禪,除尚書右僕射,遷冀州刺史。



 天保四年,梁將東方白額潛至宿豫,詔韶討之。既至,會梁將嚴超達等軍逼涇州,陳霸先將攻廣陵,尹令思謀襲盱眙,三軍咸懼。韶謂諸將曰:「自梁氏喪亂,國無定主,人懷去就。霸先外託同德,內有離心,吾揣之熟矣。」乃留儀同三司敬顯俊等圍宿豫,
 自倍道赴涇州。塗出盱眙,令思不虞大軍卒至,望旗而奔。進破超達軍。迴赴廣陵,霸先遁走。旋師宿豫,遣辨士喻白額。白額開門請盟。盟訖,度白額終不為用,斬之,並其諸弟,並傳首京師。封平原郡王,歷司空、司徒、大將軍、尚書令、太子太師。以繼母憂,去職。尋起為大司馬,仍為尚書令,遷錄尚書事、並州刺史。後與東安王婁睿平高歸彥,遷太傅,仍蒞並州。為政不存小察,甚得人和。周文遣將率羌夷與突厥合眾逼晉陽,武成自鄴倍道赴之。時大雪,諸將或欲逆之,韶曰:「不如陣以待之。彼勞我逸,破之必矣。」遂大破之。進位太師。



 周塚宰宇文護母閻氏,
 先配中山宮,護聞尚存,乃因邊境移書,請還其母,並通鄰好。韶以為護外託為相,其實王也。為母請和,不通一介之使,據移送書,恐示以弱。且外許之,待通和往復,放之未晚。不聽,遂遣使以禮將送。護得母,仍遣將尉遲迥等襲洛陽。詔蘭陵王長恭、大將軍斛律光擊之。軍次芒山下,逗留未進。



 武成召韶,欲赴洛陽圍,但以突厥為慮。韶曰:「北虜侵邊,事等疥癬;西羌窺逼,是膏肓之病。」帝仍令韶督精騎一千發晉陽,五日便濟河。遇周軍於大和谷,與諸將陣以待之。韶為左軍,蘭陵王為中軍,斛律光為右軍。上山逆戰,韶且卻引,待其力弊,下馬擊之,周人
 大潰。洛城圍亦即奔遁。除太宰,封靈武縣公。天統三年,除左丞相。四年,別封永昌郡公。食滄州幹。武平二年,出晉州道,到定隴,築威敵、平寇二城而還。二月,周師來寇,遣韶與右丞相斛律光、太尉蘭陵王長恭往。



 行達西境,有柏谷城者,敵之絕險,諸將莫肯攻圍。韶曰:「汾北河東,勢為國家之有,若不去柏谷,事同痼疾。計彼會兵在南道。今斷其要路,救不能來。城勢雖高,其中甚狹,火弩射之,一旦可盡。」遂攻之,城潰。仍城華谷,置戍而還。封廣平郡公。是月,周又遣將攻邊,斛律光先率軍禦之,韶亦請行。五月,到服秦城。



 西人於姚襄城南更起城鎮,韶抽壯
 士從北襲之,使人潛度河告姚襄城中,內外相應,進戰大破之。諸將咸欲攻其新城,韶曰:「此城一面陰河,三面地險,不可攻。不如更作一城,壅其要道。破服秦,併力圖之。」從之。六月,徙圍定陽。七月,屠其外城。時韶病在軍中,謂蘭陵王曰:「此城三面重澗,並無走路,唯慮東面一處耳。賊若突圍,必從此出。」長恭乃設伏。其夜,果如策,伏兵擊之,大潰。韶竟以病薨。賜溫明祕器、轀輬車。軍校之士,陣送至平恩墓所,發卒起塚。贈假黃鉞、相國、太尉、錄尚書事,謚忠武。



 韶出總軍旅,入參幃幄,功既居高,重以婚媾之故,望傾朝野。而長於計略,善於御眾,得將士之心。
 又雅性溫慎,有宰相之風。教訓子弟,閨門雍肅,事後母以孝聞。齊代勳貴家,罕有及者。然僻於好色,雖居要重,微服間行。魏黃門郎元瑀妻皇甫氏,緣瑀謀逆,沒官。韶美之,上啟固請,文襄賜之。別宅處之,禮同正嫡。尤嗇於財,親戚故舊,略無施與。其子深尚公主,并省丞郎在家佐事十餘日,事畢辭還,人唯賜一杯酒。



 元妃所生三子:懿、深、亮,皆宦達。



 懿字德猷,尚潁川長公主,拜駙馬都尉,襲封平原王。位行臺右僕射,兼殿中尚書,卒。子寶鼎,尚中山長公主。隋開皇中,開府儀同三司。大業初,卒於饒州刺史。



 深字德深,美容貌,寬謹有父風。天保中,受父封
 姑臧縣公。尚東安公主,位侍中。韶病篤,詔封深濟北王,以慰其意。入周,拜大將軍、郡公,坐事死。



 亮字德堪。隋大業初,位汴州刺史。卒於汝南郡守。



 韶弟孝言,少警發,有風儀。齊受禪,其兄韶以別封霸城縣侯授之。歷中書黃門侍郎。典機密。又歷祕書監、度支尚書、清都尹。



 孝言本以勳戚致位通顯,驕奢無憚。曾夜過其客宋孝王家,呼坊人防援,不時赴,遂拷殺之。又與諸淫婦密遊。其夫覺,又拷掠而殞。時苑內須果木,課人間及僧寺備輸,孝言悉分向其私宅種植。又殿內及園中須石,差車從漳河運載,復分車迴取。事發,出為海州
 刺史。累遷吏部尚書。祖珽執政,將廢趙彥深,引孝言為助,加侍中。孝言待物不平,抽擢非賄則舊。有將作丞崔成於眾中抗言:「尚書,天下尚書,豈獨段家尚書也!」孝言元辭以對,唯厲色遣下。尋除中書監,加特進。又託韓長鸞共構祖珽之短。及珽出後,孝言除尚書右僕射,仍掌選。恣情用捨,請謁大行。敕浚京城北隍,孝言監作。儀同三司崔士順、將作大匠元士將、太府少卿酈孝裕、尚書左戶郎中薛叔昭、司州中從事崔龍子、清都尹丞李道隆、鄴縣令尉長卿、臨漳令崔象、成安令高子徹等,並在孝言部下典作。日別置酒高會,諸人膝行跪伏,稱觴上
 壽,或自陳屈滯,更請轉官。孝言意色揚揚,以為己任,皆隨事報答,許有加授。富商大賈,多被銓擢,所進用人士,咸是險縱之流。尋遷左僕射,特進、侍中如故。孝言富貴豪侈,尤好女色。後取婁定遠妾董氏,大耽愛之。為此內外不和,更相糾列。又於晉陽監作,坐事除名,徙光州。隆化主敗後,有敕追還。



 孝言雖黷貨無厭,恣情酒色,然舉止風流。招致名士。美景良辰,未嘗虛棄;賦詩奏伎,以盡歡洽。雖草萊之士,粗關文藝,多引入賓館,與同興賞。其貧躓者,亦時乞遺。時論復以此多之。齊亡入周,位上開府。



 斛律金,字阿六敦,朔州敕勒部人也。高祖倍侯利,魏道武時內附,位大羽真,賜爵孟都公。祖幡地斤,殿中尚書。父那瑰,光祿大夫。贈司空。金性敦直,善騎射,行兵用匈奴法,望塵知馬步多少,嗅地知軍度遠近。初為軍主,與懷朔鎮將楊鈞送蠕蠕主阿那環。環見金獵射,歎其工。及破六韓拔陵構逆,金擁眾屬焉,署金為王。金度陵終敗,乃統所部叛陵,詣雲州。魏除為第二領人酋長,秋朝京師,春還部落,號曰鴈臣。仍稍引南出黃瓜堆,為杜洛周所破。與兄平二人脫身歸爾朱榮,為別將。孝莊立,賜爵阜城男。位金紫光祿大夫。神武密懷匡復,金贊成大
 謀。太昌初,為汾州刺史,進爵為侯。從神武破紇豆陵於河西。沙苑之役,神武以地阨少卻,軍為西師所乘,遂亂。張華原以簿帳歷營點兵,莫有應者。神武將集兵便戰,金曰:「眾散將離,其勢不可復用,宜急向河東。」神武據鞍未動,金以鞭拂馬,神武乃還。於是大崩,喪甲士八萬。侯景斂。西魏力人持大棒守河橋,衣甲厚,射之不入。賀拔仁候其轉面,射,一發斃之。是役也,無金先請還,幾至危矣。及高仲密西叛,周文攻洛陽,從神武破之。還,除大司馬,改封石城郡公。



 金性質直,不識文字。本名敦,苦其難署,改名為金,從其便易,猶以為難。



 司馬子如教為金字,
 作屋況之,其字乃就。神武重其古質,每誡文襄曰:「爾所使多漢,有讒此人者,勿信之。」及文襄嗣事,為肆州刺史。文宣受禪,封咸陽郡王。



 天保三年,就除太師。四年,解州,以太師還晉陽。車駕幸其第,六宮及諸王盡從,置酒極夜方罷。帝欣甚,詔金第二子豐樂為武衛大將軍,賜帛五千匹。謂曰:「公元勳佐命,父子忠誠,朕當結以婚姻,永為籓衛。」仍詔金孫武都尚義寧公主。成禮之日,帝從皇太后幸金宅,皇后、太子、諸王皆從。其見待如此。後蠕蠕為突厥破散,慮其犯塞,詔金屯兵白道以備之。多所俘獲,並表陳虜可取狀。文宣乃與金共討之。進位右丞相,
 食齊州幹。遷左丞相。帝晚年敗德,嘗持槊走馬以擬金胸者三,金立不動,於是賜物千段。



 孝昭踐阼,納其孫女為皇太子妃。詔金朝見,聽乘步挽車至階。武成即位,禮遇彌重,又納其孫女為太子妃。金曾遣人獻食,中書舍人李若誤奏,云金自來。武成出昭陽殿,敕侍中高文遙將羊車引之。若知事誤,更不敢出映廊下。文遙還覆奏,帝罵若云:「空頭漢,合殺!」亦不加罪。



 金長子光,大將軍。次子羨及孫武都,並開府儀同三司,出鎮方岳。其餘子孫,皆封侯貴達。一門一皇后,二太子妃,三公主,尊寵當時莫比。金嘗謂光曰:「我雖不讀書,聞古來外戚梁冀等,無
 不傾滅。女若有寵,諸貴人妒;女若無寵,天子嫌之。我家直以立勳抱忠致富貴,豈藉女也?」辭不獲免,常以為憂。天統三年薨,年八十,贈假黃鉞、相國、太尉公,贈錢百萬。謚曰武。子光嗣。



 光字明月,馬面彪身,神爽雄傑,少言笑,工騎射。初為侯景部下,彭樂謂高敖曹曰:「斛律家小兒,不可三度將行,後奪人名。」以庫直事文襄。從出野,見鴈雙飛來,文襄使光馳射之,以二矢俱落焉。後從金西征,周文帝長史莫孝暉在行間,光年十七,馳馬射中之,因禽於陣。神武即擢授都督,封永樂子。雙嘗從文襄於洹橋校獵。雲表見
 一大鳥,射之正中其頸,形如車輪,旋轉而下,乃雕也。丞相屬邢子高歎曰:「此射雕手也。」當時號落雕都督。齊受禪,別封西安縣子。皇建元年,進爵鉅鹿郡公。時樂陵王百年為皇太子,求妃。孝昭以光世載醇謹,納其長女為太子妃。歷位太子太保、尚書令、司空、司徒。河清三年,周大司馬尉遲迥、齊公憲、庸公王雄等眾十萬攻洛陽。光率騎五萬馳往,戰於芒山,迥等大敗。光親射雄殺之,迥、憲僅而獲免。仍築京觀。武成幸洛陽策勳,遷太尉。



 初,文宣時,周人常懼齊兵之西度,恒以冬月,守河椎冰。及帝即位,朝政漸紊,齊人椎冰,懼周兵之逼。光憂曰:「國家常
 有吞關、隴之志,今日至此,而唯玩聲色!」先是,武成納光第二女為太子妃,天統元年,拜皇后,光轉大將軍。三年六月,父喪去官。其月,詔起光及弟羨,並復位。秋,除太保,襲爵咸陽王,遷太傅。十二月,周軍圍洛陽,壅絕糧道。武平元年正月,詔光率步騎三萬禦之,鋒刃纏交,周將宇文桀眾大潰,直到宜陽。軍還,擊周齊王憲等眾大潰。詔加右丞相、並州刺史。其年冬,光又率步騎五萬於玉壁築華谷、龍門二城,與憲相持,憲不敢動。二年,率眾築平隴等鎮戍十三所。周柱國枹罕公普屯威、柱國韋孝寬等來逼平隴,光與戰於汾水,大破之。周遣其柱國紇干
 廣略圍宜陽,光率步騎五萬赴之,戰于城下。取周建安等四戍,捕千餘人而還。軍未至鄴,敕令便放兵散。光以功勳者未得慰勞,若散,恩澤不施。乃密表,請使宣旨,軍仍且進。朝廷發使遲留,軍還將至紫陌,光駐營待使。帝聞光軍營已逼,心甚惡之,急令舍人追光入見,然後宣勞散兵。拜左丞相,別封清河郡公。



 光嘗在朝堂,垂簾而坐。祖珽不知,乘馬過其前。光怒,謂人曰:「此人乃敢爾!」後珽在內省,言聲高慢,光過聞之,又怒。珽知光忿,賂其從奴搕頭。曰:「自公用事,相王每夜抱膝歎曰:『盲人用權,國必破矣」珽省事褚士達夢人倚戶授其詩曰:「九升八合
 粟,角斗定非真,堰卻津中水,將留何處人。」以告珽。珽占之曰:「角斗,斛字;津卻水,何留人,合成律字;非真者,解斛律於我不實。」



 士達又言所夢狀,乃其父形也。珽由是懼。又穆提婆求娶光庶女,不許。帝賜提婆晉陽之田,光言於朝曰:「此田,神武以來,常種禾飼馬,以擬寇難。今賜,無乃闕軍務也?」帝又以鄴清風園賜提婆租賃之。於是官無菜,賒買於人,負錢三百萬,其人訴焉。光曰:「此菜園賜提婆,是一家足;若不賜提婆,便百官足」。由是祖、穆積怨。周將韋孝寬懼光,乃作謠言,令間諜漏之於鄴曰:「百斗飛上天,明月照長安。」又曰:「高山不推自崩,槲樹不扶自
 豎。」珽讀之曰:「盲老公背上下大斧,饒舌老母不得語。」令小兒歌之於路。提婆聞,以告其母。令萱以饒舌為斥己,盲老公謂祖珽也,遂協謀,以謠言啟帝曰:「斛律累世大將,明月聲震關西,豐樂威行突厥,女為皇后,男尚公主,謠言可畏」帝以問韓長鸞。鸞以為不可,事寢。



 光又嘗謂人曰:「今軍人皆無褌褲,後宮內參,一賜數萬匹,府藏稍空,此是何理?」



 受賜者聞之,皆曰:「天子自賜我,關相王何事?」珽又通啟求見,帝使以庫車載入,珽因請間,唯何洪珍在側。帝曰:「前得公啟,即欲施行,長鸞以為無此理,未可。」珽未對。洪珍進曰:「若本無意,則可;既有此意,不決行,
 萬一事泄,如何!」帝然洪珍言,而猶預未決。珽令武都妾兄顏玄,告光謀為不軌;又令曹魏祖奏,言上將星盛,不誅,恐有災禍。先是天狗西流,占曰秦地。案秦即咸陽也。



 自太廟及光宅,並見血。先是三日,鼠常晝見光寢室,常投食與之,一朝三鼠俱死。



 又床下有二物如黑豬,從地出走,其穴膩滑。大蛇屢見。屋脊有聲,如彈丸落。又大門橫木自焚。搗衣石自移。既而丞相府佐封士讓密啟云:「光前西討還,敕令便放兵散,光令軍逼帝京,將為不軌,不果而止。家藏弩甲,奴僮千數,每使豐樂、武都處,陰謀往來。若不早圖,恐事不可測。」帝謂何洪珍曰:「人心亦大
 聖,我前疑其欲反,果然。」帝性怯,恐即有變,令洪珍馳召祖珽告之。又恐追光不從命。



 珽因請賜其一駿馬,令明日乘至東山遊觀,須其來謝,因執之。帝如其言。光將上馬,頭眩。及至,引入涼風堂,劉桃枝自後撲之,不倒。光曰:「桃枝常作如此事,我不負國家。」桃枝與力士三人,以弓絃肙其頸,遂拉殺之,年五十八。血流於地,刬之迹終不滅。於是下詔稱其反,族滅之。



 使二千石郎邢祖信掌簿籍其家。珽於都省問所得物,祖信曰:「得弓十五張,宴射箭一百,貝刀七口,賜槊二張。」珽又厲聲曰:「更得何物?」曰:「得棗子枝二十束,擬奴僕與人鬥者,不問曲直,即以杖
 之一百。」珽大慚,乃下聲曰:「朝廷已加重刑,郎中何可分雪?」及出,人尤其抗直。祖信慨然曰:「好宰相尚死,我何惜餘生!」祖信少年時,父遜為李庶所卿,因詣庶,謂庶曰:「暫來見卿,還辭卿去。」庶父諧杖庶而謝焉。



 光居家嚴肅,見子弟若君臣。雖極貴盛,性節儉,簡聲色,不營財利,杜絕饋餉。門無賓客,罕與朝士交言,不肯預政事。每會議,常獨後言,言輒合理。將有表疏,令人執筆,口占之,務從省實。行兵用匈奴卜法,吉凶無不中。軍營未定,終不入幕,或竟日不坐。身不脫介胄,常為士卒先。有罪者,唯大杖撾背,未嘗妄殺。眾皆爭為之死。宜陽之役,謂周人曰:「歸
 我七年人,不然取爾十倍。」周人即歸之。在西境築定誇諸城,馬上以鞭指畫,所取地皆如其言,拓地五百里而未嘗伐功。板築之役,鞭撻人士,頗稱其嚴。自結髮從戎,未嘗失律,深為鄰敵懾憚。



 罪既不彰,一旦屠滅,朝野惜之。周武帝聞光死,赦其境內。後入鄴,追贈上柱國、崇國公。指詔書曰:「此人若在,朕豈得至鄴?」



 長子武都,位特進、開府儀同三司、梁、袞二州刺史,所在唯事聚斂。光死,遣使於州斬之。



 小子鐘,年甫數歲,獲免。周朝襲封崇國公。隋開皇中,卒於車騎將軍。



 羨字豐樂,少機警,善騎射。河清三年,為都督、幽州刺史。
 其年,突厥十餘萬寇州境,羨總諸將禦之。突厥望見軍容齊整,遂不敢戰,遣使求款附。天統元年五月,突厥可汗遣使請朝貢,自是歲時不絕,羨有力焉。詔加行臺僕射。羨以虜屢犯邊塞,自庫推戍東拒於海。二千餘里,其間凡有險要,或斬山築城,斷谷起障,並置立戍邏五十餘所。又導高梁水,北合易京,東會於潞,因以灌田,公私獲利。



 在州養馬二千匹,部曲三千,以備邊,突厥謂之南面可汗。四年,遣行臺尚書令,別封高城縣侯。



 羨歷事數帝,以謹直稱,雖極榮寵,不自矜尚。以合門貴盛,深以為憂。武平元年,乃上書推讓,乞解所職。詔不許。其年秋,進
 爵荊山郡王。羨慮禍,使人騎快騾迎至鄴,無日不得音問。後二日鄴使不至,家人乞養憂之。又夢著枷鎖,勸豐樂速奔突厥,羨不從。占其夢曰:「枷者加官,鎖者鎖鎖吉利。」及光誅,敕中領軍賀拔伏恩等十餘人馳驛捕之,遣領軍大將軍鮮于桃枝、洛州行臺僕射獨孤永業便發定州騎卒續進。伏恩等既至,門者白羨曰:「使人衷甲馬汗,宜閉城門。」羨曰:「敕使豈可疑拒!」出迎之,遂見執,死於長史事。謂其妻曰:「啟太后,臣兄弟死自當知。」臨刑歎曰:「富貴如此,女為皇后,公主滿家,常使三百兵,何得不敗?」並害五子,年十五已下者宥之。羨未誅前,忽令其在
 州諸子五六人,鎖頸乘驢出城,合家泣送之至閣,日晚而歸。吏人莫不驚異。行燕郡守馬嗣明,道術之士也,為羨所欽,竊問之,答云:「須有衣襄厭。」數日而有此變。



 羨及光並工騎射。少時獵,父金命子孫會射而觀之,泣曰:「明月、豐樂用弓不及我,諸孫又不及明月、豐樂,世衰矣。」每日令出田,還即效所獲。光獲少,必麗龜達腋;羨獲雖多,非要害之所。光恆蒙賞,羨或被捶。人問其故,云:「明月必背上著箭,豐樂隨處即下手,數雖多,去兄遠矣。」聞者服其言。



 金兄平,少便弓馬。神武起,以都督從。皇建初,封定陽郡公。後為青州刺史。



 卒,贈太尉。



 論曰:齊神武以晉陽戎馬之地,霸圖攸屬,練兵訓旅,遙制朝權,鄴都機務,情寄深遠。孫騰、高隆之、司馬子如等俱不能清貞守道,以康亂為懷,而厚斂貨財,填彼溪壑。昔蕭何之鎮關中,荀彧之居許下,不亦異於是乎!賴文襄入輔,責以驕縱,厚遇崔暹,奮其霜簡;不然則君子屬厭,豈易聞焉。子如徒以少相親重,情深暱狎,義非草昧,恩結寵私,勛德莫聞,坐致台輔。消難去齊歸周,義非殉國,向背不已;晚又奔陳,一之謂甚,胡可而再。膺之風素可重,幼之清簡自立,有足稱者。竇泰、尉景、婁昭、厙狄干、韓軌等,並以外戚近親,屬雲雷之舉,位非寵進,功籍勢
 成,附翼攀鱗,鬱為佐命之首;定遠以常人之才,而因趙郡忠正,將以志除朝蠹,謀逐佞臣,而信納姦凶,反受其亂。遂使庸豎肆毒,賢戚見誅,敗政害時,莫大於此。鄙語曰:「利以昏智」,況定遠非智者乎。段榮以姻戚之重,遇時來之會,功伐之地,亦足稱焉。韶光輔七君,克隆門業,每出當閫外,或任處留臺。以猜忌之朝,終其眉壽;屬亭候多警,為有齊上將,豈其然乎!當以志謝矜功,名不渝實,不以威權御物,不以智數要時,欲求覆餗,其可得也。《禮》云「率性之謂道」,此其效歟!斛律金以神武撥亂之始,翼成王業,忠款之至,成此大功,故能終享遐年,位高百闢。
 視其盈滿之戒,動之微也,纔及後嗣,遂至誅夷。既處威權之重,蓋符道家所忌。光以上將之子,有沈毅姿,戰將兵權,暗同韜略,臨敵制勝,變化無方。自關、河分隔,年將四紀,以高氏霸王之期,屬宇文草創之日,出軍薄伐,屢挫兵威。而大寧已還,東鄰浸弱,關西前收巴蜀,又殄江陵,葉建瓴而用武,成并吞之壯志。光每臨戎誓眾,式遏邊鄙,戰則前無完陣,攻則罕有全城;齊氏必致拘原之師,秦人無復啟關之策。而世亂讒勝,詐以震主之威;主暗時艱,自毀籓籬之固。昔李牧之為趙將也,北翦胡冠,西卻秦軍,郭開譖之,牧死趙滅。其議誅光者,豈秦之反
 間歟?何同術而同亡也!內令諸將解體,外為強鄰滅仇。嗚呼!後之君子,可為深戒者歟!



\end{pinyinscope}