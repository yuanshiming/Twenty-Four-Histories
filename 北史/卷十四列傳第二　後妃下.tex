\article{卷十四列傳第二 後妃下}

\begin{pinyinscope}

 齊
 武明皇后婁
 氏蠕蠕公主鬱久閭氏彭城太妃爾硃氏小爾硃氏上黨太妃韓氏馮翊太妃鄭氏高陽太妃游氏馮娘李娘文襄敬皇后元氏瑯邪公主文宣皇后李氏段昭儀王嬪薛嬪孝昭皇后元氏武成皇後胡氏弘德李夫人後主皇后斛律氏後主
 皇後胡氏後主皇后穆氏馮淑妃周文皇后元氏文宣皇后叱奴氏孝閔皇后元氏明敬皇后獨孤氏武成皇后阿史那氏
 武皇后
 李氏宣皇后
 楊氏宣皇后朱氏宣皇后陳氏宣皇后元氏宣皇后尉遲氏靜皇后司馬氏隋文獻皇后獨孤氏宣華夫人陳氏容華夫人蔡氏煬愍皇后蕭氏
 齊武明皇后婁氏,諱昭君,贈司徒內乾之女也。少明悟,強族多娉之,並不肯行。及見神武城上執役,驚曰:「此真吾夫也。」乃使婢通意,又數致私財,使以娉己,父母不得已而許焉。神武既有澄清之志,傾產以結英豪,密謀祕策,后恆參預。及拜勃海王妃,閫闈之事悉決焉。



 后高明嚴斷,雅遵儉約,往來外舍,侍從不過十人。性寬厚,不妒忌,神武姬侍咸加恩待。神武嘗將西討出師,后夜孿生一男一女,左右以危急,請追告神武。



 后弗聽,曰:「王出統大兵,何得以我故輕離軍幕?死生命也,來復何為。」神武聞之,嗟嘆良久。沙苑敗後,侯景屢言請精騎二萬,必能
 取之。神武悅,以告于后。



 后曰:「若如其言,豈有還理?得獺失景,亦有何利。」乃止。神武逼於蠕蠕,欲娶其女而未決。后曰:「國家大計,願不疑也。」及蠕蠕公主至,后避正室處之,神武愧而拜謝焉。曰:「彼將有覺,願絕勿顧。」慈愛諸子,不異己出,躬自紡績,人賜一袍一褲。手縫戎服,以帥左右。弟昭以功名自達,其餘親屬,未嘗為請爵位,每言有材當用,義不以私亂公。



 文襄嗣位,進為太妃。文宣將受魏禪,后固執不許,帝所以中止。天保初,尊為皇太后,宮曰宣訓。濟南即位,尊為太皇太后。尚書令楊愔等受遺詔輔政,疏忌諸王。太皇太后密與孝昭及諸大將定策
 誅之,下令廢立。孝昭即位,復為皇太后。



 孝昭崩,太后又下詔立武成帝。大寧二年春,太后寢疾,衣忽自舉,用巫媼言,改姓石氏。四月辛丑,崩於北宮,時年六十二。五月甲申,合葬義平陵。



 太后凡孕六男二女,皆感夢。孕文襄則夢一斷龍;孕文宣則夢大龍,首尾屬天地,張口動目,勢狀驚人;孕孝昭則夢蠕龍於地;孕武成則夢龍浴於海;孕魏二后,並夢月入懷;孕襄城、博陵二王,夢鼠入衣下。后未崩,有童謠曰:「九龍母死不作孝。」及后崩,武成不改服,緋袍如故。未幾,登三臺,置酒作樂;宮女進白袍,帝怒,投諸臺下。和士開請止樂,帝大怒,撾之。帝於昆季,次
 實九,蓋其徵驗也。



 蠕蠕公主者,蠕蠕主郁久閭阿那瑰女也。蠕蠕強盛,與西魏通和,欲連兵東伐。



 神武病之,令杜弼使蠕蠕,為世子求婚。阿那瑰曰:「高王自娶則可。」神武猶豫,尉景與武明皇后及文襄並勸請,乃從之。武定三年,使慕容儼往娉之,號曰蠕蠕公主。八月,神武迎於下館,阿那瑰使其弟禿突佳來送女,且報聘,仍戒曰:「待見外孫,然後返國。」公主性嚴毅,一生不肯華言。神武嘗有病,不得往公主所,禿突佳怨恚,神武自射堂輿疾就公主。其見將護如此。神武崩,文襄從蠕蠕國法,蒸公主,產一女焉。



 彭城太妃爾朱氏,榮之女,魏孝莊后也。神武納為別室,敬重踰於婁妃,見必束帶,自稱下官。神武迎蠕蠕公主還,爾朱氏迎於木井北,與蠕蠕公主前後別行,不相見。公主引角弓仰射翔鴟,應弦而落;妃引長弓斜射飛烏,亦一發而中。神武喜曰:「我此二婦,並堪擊賊。」後為尼,神武為起佛寺。天保初,為太妃。及文宣狂酒,將無禮於太妃,太妃不從,遂遇禍。



 小爾朱者,兆之女也。初為建明皇后。神武納之,生任城王。未幾,與趙郡公琛私通,徙於靈州。後適范陽盧景璋。



 上黨太妃韓氏,軌之妹也。神武微時欲娉之,軌母不許。
 及神武貴,韓氏夫已死,乃納之。



 馮翊太妃鄭氏,名大車,嚴祖妹也。初為魏廣平王妃。遷鄴後,神武納之,寵冠後庭,生馮翊王潤。神武之征劉蠡升,文襄蒸於大車。神武還,一婢告之,二婢為證。神武杖文襄一百而幽之,武明后亦見隔絕。時彭城爾朱太妃有寵,生王子浟,神武將有廢立意。文襄求救於司馬子如。子如來朝,偽為不知者,請武明后。神武告其故。子如曰:「消難亦姦子如妾,如此事,正可覆蓋。妃是王結髮婦,常以父母家財奉王,王在懷朔被杖,背無完皮,妃晝夜供給看瘡。後避葛賊,同走並州。



 貧困,然馬屎,自作靴,恩
 義何可忘?夫婦相宜,女配至尊,男承大業,又婁領軍勳,何宜搖動?一女子如草芥,況婢言不必信。」神武因使子如鞫之。子如見文襄,尤之曰:「男兒何意畏威自誣?」因教二婢反辭,脅告者自縊,乃啟神武曰:「果虛言。」神武大悅,召后及文襄。武明后遙見神武,一步一叩頭。文襄且拜且進,父子夫妻相泣,乃如初。神武乃置酒曰:「全我父子者,司馬子如。」賜之黃金百三十斤,文襄贈良馬五十匹。



 高陽太妃游氏,父京之,為相州長史。神武剋鄴,欲納之;京之不許,遂牽曳取之。京之尋死。游氏於諸太妃中最有德訓,諸王、公主婚嫁,常令主之。



 馮娘者,子昂妹也,初
 為魏任城王妃,適爾朱世隆。神武納之,生浮陽公主。



 李娘者,延實從妹也。初為魏城陽王妃。又王娘生永安王浚,穆娘生平陽王淹。並早卒,不為太妃。



 文襄敬皇后元氏,魏孝靜帝之姊也。孝武帝時,封馮翊公主,而歸於文襄。容德兼美,曲盡和敬。初生河間王孝琬,時文襄為世子,三日而孝靜幸世子第,贈錦彩及布帛萬匹。世子辭,求通受諸貴禮遺,於是十屋皆滿。次生兩公主。文宣受禪,尊為文襄皇后,居靜德宮。及天保六年,文宣漸致昏狂,乃移居於高陽之宅而取其府庫,曰:「吾兄昔姦我婦,我今須報。」乃淫於后。其高氏女婦,無親
 疏皆使左右亂交之於前。以葛為絙,令魏安德主騎上,使人推引之。又命胡人苦辱之。帝又自呈露,以示群下。武平中,后崩,祔葬義平陵。



 瑯邪公主名玉儀,魏高陽王斌庶生妹也。初不見齒,為孫騰妓,騰又放棄。文襄遇諸途,悅而納之,遂被殊寵,奏魏帝封焉。文襄謂崔季舒曰:「爾由來為我求色,不如我自得一絕異者。崔暹必當造直諫,我亦有以待之。」及暹諮事,文襄不復假以顏色。居三日,暹懷刺,墜之於前。文襄問:「何用此為?」暹悚然曰:「未得通公主。」文襄大悅,把暹臂入見焉。季舒語人曰:「崔暹常忿吾佞,在大將軍前,每言叔父合殺。及自作體佞,
 乃體過於吾。」玉儀同產姊靜儀,先適黃門郎崔括,文襄亦幸之,皆封公主。括父子由是超授,賞賜甚厚焉。



 文宣皇后李氏諱祖娥,趙郡李希宗女也。容德甚美。初為太其原公夫人。及帝將建中宮。高隆之、高德正言漢婦人不可為天下母,宜更擇美配。楊愔固請依漢、魏故事,不改元妃。而德正猶固請廢后而立段昭儀,欲以結勛貴之援。帝竟不從而立后焉。帝好捶撻嬪御,乃至有殺戮者,唯后獨家禮敬。天保十年,改為可賀敦皇后。孝昭即位,降居昭信宮,號昭信皇后。武成踐阼,逼后淫亂云:「若不許我,當殺爾兒。」后懼,從之。後有娠,太原王紹德至
 閣,不得見。慍曰:「兒豈不知邪?姊姊腹大,故不見兒。」后聞之大慚,由是生女不舉。帝橫刀詬曰:「爾殺我女,我何不殺爾兒?」對后前築殺紹德。后大哭,帝愈怒,裸后亂撾撻之,號天不已。盛以絹囊,流血淋漉,投諸渠水,良久乃蘇,犢車載送妙勝尼寺。后性愛佛法,因此為尼。齊亡,入關,隋時得還趙郡。



 段昭儀,韶妹也。婚夕,韶妻元氏為俗弄女婿法戲文宣,文宣銜之。後因發怒,謂韶曰:「我會殺爾婦!」元氏懼,匿婁太后家,終文宣世不敢出。昭儀才色兼美,禮遇殆同正嫡。後主時,改適錄尚書唐邕。



 王嬪者,瑯邪人也。嬪姊先適崔修,文宣並幸之。數數降其夫家,超
 用修為尚書郎。



 薛嬪者,本倡家女也。年十四五時,為清河王岳所好。其父求內宮中,大被嬖寵。其姊亦俱進御。文宣後知先與岳通,又為其父乞司徒公。帝大怒,先鋸殺其姊。



 薛嬪當時有娠,過產亦從戮。



 孝昭皇后元氏,開府元蠻女也。初為常山王妃,天保末,賜姓步六孤。孝昭即位,立為皇后。帝崩,從梓宮之鄴。始度汾橋,武成聞后有奇藥,追索之不得,使閹人就車頓辱。降居順成宮。武成既殺樂陵王,元被閟隔,不得與家相知。宮闈內忽有飛語,帝令檢推,得后父兄書信,元蠻由是坐免官。后以齊亡,入周氏宮中。



 隋文帝作相,放還
 山東。



 武成皇后胡氏,安定胡延之女。其母範陽盧道約女。初懷孕,有胡僧詣門曰:「此宅瓠蘆中有月。」既而生后。天保初,選為長廣王妃。產後主日,有鴞鳴於產帳上。武成崩,尊為皇太后。陸媼及和士開密謀殺趙郡王睿,出婁定遠、高文遙為刺史。和、陸諂事太后,無所不至。初,武成時,后與諸閹人褻狎。武成寵幸和士開。每與后握槊,因此與后姦通。自武成崩後,數出詣佛寺,又與沙門曇獻通。布金錢於獻度下,又挂寶裝胡床於獻屋壁,武成平生之所御也。乃置百僧於內殿,託以聽講,日夜與曇獻寢
 處。以獻為昭玄統。僧徒遙指太后以弄曇獻,乃至謂之為太上者。帝聞太后不謹,而未之信。後朝太后,見二少尼,悅而召之,乃男子也。於是曇獻事亦發,皆伏法。并殺元山王三郡君,皆太后之所暱也。帝自晉陽奉太后還鄴,至紫陌,卒遇大風。兼舍人魏僧伽明風角。奏言:「即時當有暴逆事。」帝詐云鄴中有急,彎弓纏弰,馳入南城。令鄧長顒幽太后北宮。仍有敕,內外諸親一不得與太后相見。久之,帝迎復太后。太后初聞使者至,大驚,慮有不測。每太后設食,帝亦不敢嘗。周使元偉來聘,作《述行賦》,敘鄭莊公剋段而遷姜氏。文雖不工,當時深以為愧。齊
 亡,入周,恣行姦穢。開皇中殂。



 弘德夫人李氏,趙郡李叔讓女也。初為魏靜帝嬪,武成納焉。生南陽王仁盛,為太妃。姊為南安王思妃,坐夫反,以燒死。太妃聞之,發狂而薨。



 文宣王嬪及中人盧勒叉妹,武成並以為嬪。武成崩後,胡后令二嬪自殺。二嬪悲哭,後主為之惻愴。私遺衣物,令出外避焉。盧養淮南王,後為太妃。



 又有馬嬪,亦得幸,為后所妒,自縊死。



 彭榮、任祥並有女,因坐父兄事,皆入宮,為文宣所幸。武成以彭為夫人,養齊安王,任生丹楊王,並為太妃。



 後主皇后斛律氏,左丞相光之女也。初為皇太子妃,後
 主受禪,立為皇后。武平三年正月,生女。帝欲悅光,詐稱生男,為之大赦。光誅,后廢在別宮,後令為尼。齊滅,嫁為開府元仁妻。



 後主皇后胡氏,隴東王長仁女也。胡太后失母儀之道,深以為愧,欲求悅後主,故飾后於宮中。令帝見之。帝果悅,立為弘德夫人,進左昭儀,大被寵愛。斛律后廢,陸媼欲以穆夫人代之,太后不許。祖孝徵請立胡昭儀,遂登為皇后。陸媼既非勸立,又意在穆夫人,其後於太后前作色而言曰:「何物親姪女,作如此語言!」



 太后問有何言。曰:「不可道。」固問之,乃曰:「語大家云,太后行多非法,不可
 以訓。」太后大怒,喚后出,立剃其髮,送令還家。帝思之,每致詩以通意。後與斛律廢后俱召入內。數日而鄴不守,后亦改嫁云。



 後主皇后穆氏,名邪利,本斛律后從婢也。母名輕霄,本穆子倫婢也,轉入侍中宋欽道家,姦私而生后,莫知氏族,或云后即欽道女子也。小字黃花,後字舍利。



 欽道婦妒,輕霄面黥為宋字。欽道伏誅,黃花因此故宮。有幸於後主,宮內稱為「舍利大監」。女侍中陸太姬知其寵,養以為女,薦為弘德夫人。武平元年六月,生皇子恆。于時後主未有儲嗣,陸陰結待,以臣撫之任不可無主。時皇后
 斛律氏,丞相光之女也,慮其懷恨,先令母養之立為皇太子。陸以國姓之重,穆、陸相對,又奏賜姓穆氏。胡庶人之廢也,陸有助焉。故遂立為皇后,大赦。初,有折衝將軍元正烈,於鄴城東水中得璽以獻,文曰「天皇后璽」,蓋石氏所作。詔書頒告,以為穆后之瑞焉。武成為胡后造真珠裙褲,所費不可稱計,被火燒。後主既立穆皇后,復為營之。屬周武遭太后喪,詔侍中薛孤、康買等為弔使,又遣商胡齎錦彩三萬疋與弔使同往。欲市真珠,為皇后造七寶車。周人不與交易,然而竟造焉。先是,童謠曰:「黃花勢欲落,清觴滿盃酌。」言黃花不久也。後主自立穆后
 以後,昏飲無度,故云「清觴滿杯酌」。陸息駱提婆,詔改姓為穆;陸,太姬。皆以皇后故也。



 后既以陸為母,提婆為家,更不採輕霄。輕霄後自療面,欲求見,為太姬陸媼使禁掌之,竟不得見。



 馮淑妃名小憐,大穆后從婢也。穆后愛衰,以五月五日進之,號曰「續命」。



 慧黠能彈琵琶,工歌舞。後主惑之,坐則同席,出則並馬,願得生死一處。命淑妃處隆基堂,淑妃惡曹昭儀所常居也,悉令反換其地。周師之取平陽,帝獵於三堆,晉州亟告急。帝將還,淑妃請更殺一圍,帝從其言。識者以為後主名緯,殺圍言非吉徵。及帝至晉州,
 城已欲沒矣。作地道攻之,城陷十餘步,將士乘勢欲入。帝敕且止,召淑妃共觀之。淑妃妝點,不獲時至。周人以木拒塞,城遂不下。舊俗相傳,晉州城西石上有聖人跡,淑妃欲往觀之。帝恐弩矢及橋,故抽攻城木造遠橋,監作舍人以不速成受罰。帝與淑妃度橋,橋壞,至夜乃還。稱妃有功勳,將立為左皇后,即令使馳取禕翟等皇后服御。仍與之並騎觀戰,東偏少卻,淑妃怖曰:「軍敗矣!」



 帝遂以淑妃奔還。至洪洞戍,淑妃方以粉鏡自玩,後聲亂唱賊至,於是復走。內參自晉陽以皇后衣至,帝為按轡,命淑妃著之,然後去。帝奔鄴,太后後至,帝不出迎;淑妃
 將至,鑿城北門出十里迎之。復以淑妃奔青州。後主至長安,請周武帝乞淑妃,帝曰:「朕視天下如脫屣,一老嫗豈與公惜也!」仍以賜之。



 及帝遇害,以淑妃賜代王達,甚嬖之。淑妃彈琵琶,因弦斷,作詩曰:「雖蒙今日寵,猶憶昔時憐。欲知心斷絕,應看膠上弦。」達妃為淑妃所譖,幾致於死。



 隋文帝將賜達妃兄李詢,令著布裙配舂。詢母逼令自殺。



 後主以李祖欽女為左昭儀,進為左娥英。裴氏為右娥英。娥英者,兼取舜妃娥皇、女英名,陽休之所制。



 樂人曹僧奴進二女,大者忤旨,剝面皮;少者彈琵琶,為昭儀。以僧奴為日南王。僧奴死後,又貴其兄弟妙達等
 二人,同日皆為郡王。為昭儀別起隆基堂,極為綺麗。陸媼誣以左道,遂殺之。



 又有董昭儀、毛夫人、彭夫人、王夫人、小王夫人、二李夫人,皆嬖寵之。毛能彈箏,本和士開薦入。帝所幸彭夫人,亦音妓進;死於晉陽,造佛寺,與總持相埒。一李是隸戶女,以五弦進。一李即孝貞之女也。小王生一,男,諸閹人在傍,皆蒙賜給。毛兄思安,超登武衛。董父賢義,為作軍主,由昭儀亦超登開府。自餘姻屬,多至大官。



 周文皇后元氏,魏孝武之妹也。初封平原公主,適開府張歡。歡性貪殘,遇后無禮。帝殺歡,改封后為馮翊公主,
 以配周文帝。生孝閔帝。魏大統十七年,薨。



 恭帝三年十二月,合葬成陵。孝閔踐阼,追尊為王后。武成初,又追尊為皇后。



 文宣皇后叱奴氏,代人也。周文帝為丞相,納為姬,生武帝。天和二年六月,尊為皇太后。建德三年三月,崩。五月,葬永固陵。



 孝閔皇后元氏,名胡摩,魏文帝第五女也。初封晉安公主。帝之為略陽公也,尚焉。及踐阼,立為王后;帝被廢,后出俗為尼。建德初,武帝誅晉公護,上帝尊號,以后為孝閔皇后,居崇義宮。隋革命,后出居里第。大業十二年,殂。



 明敬皇后獨孤氏,太保、衛公信之長女也。帝之在籓,納為夫人。二年正月,立為王后。四月,崩,葬昭陵。武成初,追崇為皇后。明帝崩,與后合葬焉。



 武成皇后阿史那氏,突厥木桿可汗俟斤之女也。突厥滅蠕蠕後,盡有塞表之地,志陵中夏。周文方與齊人爭衡。結以為援。俟斤初欲以女配帝,既而悔之。武帝即位,前後累遣使焉。保定五年二月,詔陳公純。許公於文貴、神武公竇毅、南安公楊薦等,備皇后文物及行殿,并六宮以下一百二十人,至俟斤牙所迎后。俟斤又許齊婚,將有異志,純等累請,不得反命。會雷風大起,飄壞其穹
 廬,俟斤大懼,以為天譴,乃禮送后,純等奉之以歸。天和三年三月至,武帝接以親迎之禮。后有姿貌,善容止,帝深敬禮焉。宣帝即位,尊后為皇太后。大象元年二月,改為天元皇太后。二年二月,又尊曰天元上皇太后。宣帝崩,靜帝尊為太皇太后。隋開皇二年,殂,年三十二。隋文詔有司備禮,祔葬后於孝陵。



 武皇后李氏,名娥姿,楚人也。于謹平江陵,后家被籍沒。至長安,周文以后賜武帝。後得親幸,生宣帝。宣政元年七月,尊為帝太后。大象元年二月,改為天元帝太后。七月,又尊為天皇太后。二年二月,尊為天元聖皇太后。宣
 帝崩,靜帝尊為大帝太后。隋開皇元年三月,出俗為尼,改名常悲。八年,殂,以尼禮葬于京城南。



 宣皇后楊氏名麗華,隋文帝之長女也。帝在東宮,武帝為帝納后為皇太子妃。



 宣政元年閏六月,並為皇后。帝後自稱天元皇帝,號后為天元皇后。尋又立天皇后及左右皇后,與為四皇后。二年二月,詔取象四星,於是后及三皇后並加大焉。冊授后為天元大皇后,又立天中大皇后,與后為五皇后焉。后性柔婉,不妒忌,四皇后及嬪御等咸愛而仰之。帝後昏暴滋甚,喜怒乖度。嘗譴后,欲加之罪,后進止詳閑,辭色不撓。帝大怒,遂賜后死,逼
 令自引決。后母獨孤氏聞之,詣閣陳謝,叩頭流血,然後得免。帝崩,靜帝尊后為皇太后,居弘聖宮。初,宣帝不豫,詔隋文帝入禁中侍疾。及大漸,劉昉、鄭譯等因矯詔以隋文帝受遺輔政。后初雖不預謀,然以嗣主幼沖,恐權在他族,不利於己,聞昉、譯已行此詔,心甚悅。後知隋文有異圖,意頗不平。及行禪代,憤惋愈甚。隋文內甚愧之。開皇初,封后為樂平公主。



 後又議奪其志,后誓不許,乃止。大業五年,從煬帝幸張掖,殂於河西。詔還京,所司備禮,祔葬后於定陵。



 宣帝后朱氏,名滿月,吳人也。其家坐事,沒入東宮。宣帝
 之為太子,后被選掌衣服,帝召幸之,遂生靜帝。大象元年四月,立為天元帝后。七月,改為天皇后。



 二年二月,又改為天大皇后。后本非良家子,又年長於帝十餘歲,疏賤無寵。以靜帝故,特尊崇之,班亞楊皇后焉。宣帝崩,靜帝尊后為帝太后。隋開皇元年二月,出俗為尼,改名法靜。六年,殂,以尼禮葬於京城西。



 宣帝后陳氏,名月儀,自云潁川人,大將軍山提之第八女也。大象元年六月,以選入宮,拜為德妃。月餘日,立為天左皇后。二年二月,改為天左大皇后。三月,又詔以坤儀比德,土數惟五,四大皇后外,增置天中大皇后一人。
 於是以后為天中大皇后。帝崩,后出俗為尼,改名華光。后永徽初終。



 父山提,本爾朱兆之隸。仕齊,位特進、開府、東兗州刺史、謝陽王。武帝平齊,拜大將軍,封淅陽公。大象元年,以后父超授上柱國,進鄅國公,除大宗伯。



 宣帝皇后元氏,名樂尚,河南洛陽人,開府晟之第二女也。年十五,被選入宮,拜貴妃。大象元年七月,立為天右皇后。二年二月,改為天右大皇后。帝崩,后出家為尼,改名華勝。初,后與陳皇后同時被選入宮,俱拜為妃;及升后,又同日受冊。帝寵遇二后,禮數均等,年齒復同,特相親愛。及為尼後,李、朱及尉遲后並相繼殞歿,而二后貞
 觀中尚存。



 后父晟,少以元氏宗室,拜開府。大象元年七月,以后父進位上柱國,封翼國公。



 宣帝皇后尉遲氏名繁熾,蜀公迥之孫女也。有美色。初適杞公亮子西陽公溫,以宗婦例入朝,帝逼幸之。及亮謀逆,帝誅溫,追后入宮,拜長貴妃。大象二年三月,立為天左大皇后。帝崩,后出俗為尼,改名華道。隋開皇十五年,殂。



 靜帝司馬皇后名令姬,柱國、滎陽公消難之女也。大象元年二月,宣帝傳位於帝,七月為帝納后為皇后。二年九月,隋文帝以后父奔陳,廢后為庶人。後嫁為隋司州
 刺史李丹妻,貞觀初猶存。



 隋文獻皇后獨孤氏,諱伽羅,河南洛陽人,周大司馬、衛公信之女也。信見文帝有奇表,故以后妻焉。時年十四。帝與后相得,誓無異生之子。后姊為周明帝后,長女為周宣帝后;貴戚之盛,莫與為比,而后每謙卑自守。及周宣帝崩,隋文居禁中,總百揆。后使李圓通謂文帝曰:「騎獸之勢,必不得下,勉之!」及帝受禪,立為皇后。



 突厥嘗與中國交市,有明珠一篋,價直八百萬;幽州總管陰壽白后市之。后曰:「當今戎狄屢寇,將士罷勞,未若以八百萬分賞有功者。」百寮聞而畢賀。文帝甚寵憚之。帝每臨朝,
 后輒與上方輦而進,至閣乃止。使宮官伺帝,政有所失,隨則匡諫,多所弘益。候帝退朝而同反宴寢,相顧欣然。后早失二親,常懷感慕,見公卿有父母者,每為致禮焉。有司奏曰:「《周禮》,百官之妻,命於王后。憲章在昔,請依古制。」后曰:「以婦人與政,或從此漸,不可開其源也。」不許。后每謂諸公主曰:「周家公主類無婦德,失禮於舅姑,離薄人骨肉,此不順事,爾等當誡之。」后姑子都督崔長仁犯法當斬,文帝以后故免之。后曰:「國家之事,焉可顧私!」長仁竟坐死。異母弟陀以貓鬼巫蠱咒詛於后,坐當死。后三日不食,為之請命曰:「陀若蠹政害民者,不敢言。今坐
 為妾身,請其命。」陀於是減死一等。



 后雅性儉約。帝常合止利藥,須胡粉一兩,宮內不用,求之竟不得。又欲賜柱國劉嵩妻織成衣領,宮內亦無。上以后不好華麗,時齊七寶車及鏡臺絕巧麗,使毀車而以鏡臺賜后。后雅好讀書,識達今古,凡言事皆與上意合,宮中稱為二聖。嘗夢周阿史那後,言受罪辛苦,求營功德。明日言之,上為立寺追福焉。后兄女,夫死於并州,后嫂以女有娠,請不赴葬。后曰:「婦人事夫,何容不往!其姑在,宜自諮之。」姑不許,女遂行。



 后頗仁愛,每聞大理決囚,未嘗不流涕。然性尤妒忌,後宮莫敢進御。尉遲迥女孫有美色,先在宮中,
 帝於仁壽宮見而悅之,因得幸。后伺帝聽朝,陰殺之。上大怒,單騎從苑中出,不由徑路,入山谷間三十餘里。高熲、楊素等追及,扣馬諫。



 帝太息曰:「吾貴為天子,不得自由!」高熲曰:「陛下豈以一婦人而輕天下?」



 帝意少解,駐馬良久,夜方還宮。后候上於閣內,及帝至,流涕拜謝。熲、素等和解之,上置酒極歡。后自此意頗折。



 初,后以高熲是父之家客,甚見親禮。至是,聞熲謂己為一婦人,因以銜恨。



 又以熲夫人死,其妾生男,益不善之,漸加譖毀。帝亦每事唯后言是用。后見諸王及朝士有妾孕者,必勸帝斥之。時皇太子多內寵,妃元氏暴薨,後意太子愛妾雲
 氏害之。由是諷帝,黜高熲,竟廢太子立晉王廣,皆后之謀也。



 仁壽二年八月甲子,日暈四重。己巳。太白犯軒轅。其夜,后崩於永安宮,時年五十九,葬於太陵。其後宣華夫人陳氏、容華夫人蔡氏俱有寵,帝頗惑之,由是發疾。及危篤,謂侍者曰「使皇后在,吾不及此」云。



 宣華夫人陳氏,陳宣帝女也。性聰慧,姿貌無雙。及陳滅,配掖庭,後選入宮為嬪。時獨孤皇后性妒,後宮罕得進御,唯陳氏有寵。煬帝之在籓也,陰有奪宗之計,規為內助,每致禮焉。進金蛇、金駝等物,以取媚於陳氏。皇太子廢立之際,頗有力焉。及文獻皇后崩,進位為貴人。專房
 擅寵,主斷內事,六宮莫與為比。及帝大漸,遺詔拜為宣華夫人。初,帝寢疾於仁壽宮,夫人與皇太子同侍疾。平旦更衣,為太子所逼,夫人拒之得免。歸於上所,上怪其神色有異,問之。夫人泣以實對。帝恚曰:「畜生何堪付大事,獨孤誠誤我!」意謂獻皇后也。因呼兵部尚書柳述、黃門侍郎元巖曰:「呼我兒!」述等呼太子。帝曰:「勇也。」述、巖出閣為敕書訖,示左僕射楊素。素以白太子,太子遣張衡入寢殿,遂令夫人及後宮同侍疾者並就別室。俄聞上崩,而未發喪也。夫人與諸後宮相顧曰:「事變矣!」皆色動股粟。晡後,太子遣使者齎金合,帖紙於際,親署封字,以賜
 夫人。夫人見,怕懼,以為鴆毒,不敢發。使者促之,乃發,見合中有同心結數枚。諸宮人相謂曰:「得免死矣!」陳氏恚而卻坐,不肯致謝。諸宮人共逼之,乃拜使者。其夜,太子蒸焉。



 煬帝即位,出居仙都宮。尋召入,歲餘而終,時年二十九。帝深悼之,為製《神傷賦》。



 容華夫人蔡氏,丹陽人也。陳滅。以選入宮,為世婦。容儀婉嬈,帝甚悅之。



 以文獻后故,希得進幸。后崩後,漸見寵遇,拜為貴人,參斷宮掖,亞於陳氏。帝寢疾,加號容華夫人。帝崩後,亦為煬帝所蒸。



 煬帝愍皇后蕭氏,梁明帝巋之女也。江南風俗,二月生
 子者不舉。后以二月生,由是季父岌收養之。未歲,岌夫妻俱死,轉養舅張軻家。軻甚貧窶,后躬親勞苦。



 煬帝為晉王,文帝為選妃於梁,卜諸女皆不吉。巋乃迎后於舅氏,令使者占之,曰:「吉。」遂冊為妃。



 后性婉順,有智識,好學解屬文,頗知占候,文帝大善之。煬帝甚寵敬焉。及帝嗣位,立為皇后。帝每游幸,未嘗不隨從。時后見帝失德,心知不可,不敢措言,因為《述志賦》以自寄焉。其詞曰:承積善之餘慶,備箕帚於皇庭。恐脩名之不立,將負累於先靈。乃夙夜而匪懈,實夤懼於玄冥。雖自強而不息,亮愚蒙之多滯。思竭節於天衢,才追心而弗逮。實庸薄之多
 幸,荷隆寵之嘉惠。賴天高而地厚,屬王道之升平。均二儀之覆載,與日月而齊明。乃春生而夏長,等品物而同榮。願立志於恭儉,私自兢於誡盈。孰有念於知足,茍無希於濫名。惟至德之弘深,情弗邇於聲色。感懷舊之餘恩,求故劍於宸極。叨不世之殊眄,謬非才而奉職。何寵祿之踰分,撫胸襟而未識。雖沐浴於恩光,內慚惶而累息。顧微躬之寡昧,思令淑之良難。實不遑於啟處,將有情而自安!



 若臨深而履薄,心戰粟其如寒。夫居高而必危,每處滿而防溢。知恣夸之非道,乃攝生於沖謐。嗟寵辱之易驚,尚無為而抱一。履謙光而守志,且願守乎容
 膝。珠簾玉箔之奇,金屋瑤臺之美;雖時俗之崇麗,蓋哲人之所鄙。愧絺綌之不工,豈絲竹而喧耳。知道德之可尊,明善惡之由己。蕩囂煩之俗慮,乃伏膺於經史。綜箴誡以訓心,觀女圖而作軌。遵古賢之令範,冀福祿之能綏。時循躬而三省,覺今是而昨非。嗤黃、老之損思,信為善之可歸。慕周姒之遺風,美虞妃之聖則。仰先哲之高才,慕至人之休德。質非薄而難蹤,心恬愉而去惑。乃平生之耿介,實禮義之所遵。



 雖生知之不敏,庶積行以成仁。懼達人之蓋寡,謂何求而自陳。誠素志之難寫,同絕筆於獲麟。



 及帝幸江都,臣下離貳,有宮人白后曰:「外聞
 人人欲反。」后曰:「任汝奏之。」宮人言於帝,帝大怒曰:「非汝宜言!」乃斬之。後宮人復白后曰:「宿衛者往往偶語謀反。」后曰:「天下事一朝至此,勢去已然,無可救也。何用言,徒令帝憂煩耳!」自是無復言者。



 及宇文化及之亂,隨軍至聊城。化及敗,沒於竇建德。建德妻曹氏妒悍,煬帝妃嬪美人並使出家,并后置於武強縣。是時突厥處羅可汗方盛,其可賀敦即隋義城公主也,遣使迎后。建德不敢留,遂攜其孫正道及諸女入於虜庭。大唐貞觀四年,破突厥,皆以禮致之。歸于京師,賜宅於興道里。二十一年,殂。詔以皇后禮於揚州合葬於煬帝陵,謚曰愍。



 論曰:男女正位,人倫大綱。三代已還,逮於漢、晉,何嘗不敗於嬌詖而興於聖淑。至如后稷稟靈巨迹,神元生自天女,克昌來葉,異世同符。魏諸后婦人之識,無足論者。文明邪險,幸不墜國。靈后淫恣,卒亡天下。傾城之誡,其在茲乎。乙后迫於畏逼,有足傷矣。昔鉤弋年少子幼,漢武所以行權,魏世遂為常制,子貴而其母必死。矯枉之義,不亦過乎!孝文終革其失,良有以也。



 神武肇興齊業,武明追蹤周亂。溫公之敗邦家,馮妃比跡褒后。然則汙隆之義,蓋有係焉。其餘作孽為眚,外平內蠹,鑒之近代,於齊為甚。



 周氏粵自文皇,逮乎武帝,年踰二紀,世歷四
 君。業非草昧之辰,事殊權宜之日。乃棄同即異,以夷亂華。汨婚姻之彞序,求豺狼之外利。既而報者倦矣,施者無厭。向之所謂和親,未幾已成仇敵。奇正之道,有異於斯。於時武皇雖受制於人,未親庶政,而謀士韞奇,直臣鉗口,過矣哉!而歷觀前載,以外戚而居宰輔者多矣;而傾漢室者王族,喪周家者楊氏,何滅亡之禍,若合契焉。



 隋文取鑒於已遠,大革前失,故母後之家不罹禍故。獨孤權無呂、霍,獲全仁壽之前;蕭氏勢異梁、竇,不傾大業之後。至或不隕舊基,或更隆克構,豈非處之以道,其所致然乎?



\end{pinyinscope}