\article{卷六十七列傳第五十五}

\begin{pinyinscope}

 崔彥穆
 楊纂段永令狐整子熙唐永子瑾柳敏子昂王士良崔彥穆,字彥穆,清河東武城人,魏司空安陽侯林之九世孫也。曾祖顗,後魏平東府諮議參軍。祖蔚,遭從兄司徒浩之難,南奔江左。仕宋,為給事黃門侍郎、汝南義陽二郡守。延興初,復歸於魏,拜潁川郡守,因家焉。後終於郢州刺史。父幼,位終永昌郡守。隋開皇初,以獻皇后外
 曾祖,追贈上開府儀同三司、新州刺史。



 彥穆幼明悟。神彩卓然。魏吏部尚書隴西李神人雋,有知人之鑒,見而嘆曰:「王佐才也。」永安末,除司徒府參軍事,再遷大司馬從事中郎。孝武西遷,彥穆時不得從。大統三年,乃與兄彥珍於成皋舉義,因攻拔滎陽,禽東魏郡守蘇淑。仍與鄉郡王元洪威攻潁川,斬其刺史李景遺。即拜滎陽郡守,尋賜爵千乘縣侯。十四年,授散騎常侍、司農卿。時軍國草創,眾務殷繁,周文乃引彥穆入幕府,兼掌文翰。及于謹伐江陵,彥穆以本官從平之。周明帝初,進驃騎大將軍、開府儀同三司。



 俄拜安州刺史,總管十二州諸軍事。
 入為御正大夫。陳氏請敦鄰好。詔彥穆使焉。



 彥穆風韻閑曠,器度方雅,善玄言,解談謔,甚為江表所稱。轉戶部中大夫,進爵為公。天和三年,聘齊還,除金州刺史,總管七州諸軍事,進位大將軍。尋徵拜小司徒。



 及宣帝崩,隋文帝輔政,三方起兵,以彥穆為行軍總管,與襄州總管王誼討司馬消難。軍次荊州,總管獨孤永業有異志,遂收而戮之。及事平,隋文帝徵王誼入朝,即以彥穆為襄州刺史,總管六州諸軍事,加授上大將軍,進爵東郡公。頃之,永業家自理得雪,彥穆坐除名。尋復官爵。開皇元年卒。子君綽嗣。



 君綽性夷簡,博覽經史,有父風。大象末,
 丞相府賓曹參軍。



 君綽弟君肅,解巾道王侍讀,大象末,潁川郡守。



 楊纂,廣寧人也,父安仁,魏朔州鎮將。纂少慷慨有志略,勇力兼人。年二十,從齊神武起兵於信都,以軍功。稍遷武州刺史。自以賞薄,志懷怨憤,每歎曰:「大丈夫富貴何必故鄉!若以妻子經懷,豈不沮人雄志!」大統初,乃間行入關。



 周文執纂手曰:「人所貴者忠義也,所懼者危亡也,其能不憚危亡,蹈茲忠義者,今方見之於卿耳。」即授征南將軍、大都督,封永興縣侯。從周文解洛陽圍,經河橋、芒山之戰,纂每先登,軍中咸推其敢勇,累遷驃騎大將軍、
 開府儀同三司,加侍中,進爵為公,賜姓莫胡盧氏。俄授岐州刺史。周孝閔帝踐阼,進爵宋熙郡公。



 保定元年,位大將軍,改封隴東郡公,除隴州刺史。從隋公楊忠東伐,至并州而還。



 天和六年,進授柱國大將軍,轉華州刺史。纂性質樸,又不識文字,前後蒞職,但推誠信而已。吏人以其忠恕,頗亦懷之。尋卒於州。



 子睿,位至上柱國、漁陽郡公。



 段永,字永賓,其先遼西石城人,晉幽州刺史疋磾之後也,曾祖愄,仕魏黃龍鎮將,因徙高陸之河陽焉。永幼有志操,閭里稱之。魏正光末,北鎮擾亂,遂攜老幼,避地中
 山。後赴洛陽,拜平東將軍,封沃陽縣伯。青州人崔社客舉兵反,永討平之。進爵為侯,除左光祿大夫。時有賊魁元伯生,西自崤、潼,東至鞏、洛,屠陷城壁,所在為患。孝武遣京畿大都督疋婁昭討之,昭請以五千人行。永進曰:「此賊既無城柵,唯以寇抄為資,取之在速,不在眾也。若星馳電發,出其不虞,精騎五百足矣。」帝然其計,於是命永代昭,以五百騎倍道兼進,遂破平之。及帝西遷,永時不及從。大統初,乃結宗人,潛謀歸款。密與都督趙業等襲斬西中郎將慕容顯和,傳首京師。以功別封昌平縣子,徐州刺史。從禽竇泰,復弘農,破沙苑,並有戰功,進爵
 為公。河橋之役,永力戰先登,授南汾州刺史。累遷驃騎大將軍、開府儀同三司,賜姓爾綿氏。廢帝元年,授恆州刺史。于時朝貴多其部人,謁永之日,冠蓋盈路,當時榮之。周孝閔帝踐阼,進爵廣城郡公。歷文、瓜二州刺史,戶部中大夫。保定四年,拜大將軍。永歷任內外,所在頗有聲稱,輕財好士,朝野以此重焉。天和四年,授小司寇。尋為右二軍總管,率兵北道講武。遇疾,卒於賀葛城。喪遠,武帝親臨,贈使持節、柱國大將軍、同華等五州刺史,謚曰基。



 子岌嗣。位至儀同三司、兵部下大夫。



 令狐整,字延保,敦煌人也,本名延。世為西士冠冕。曾祖
 嗣,祖紹安,官至郡守,咸為良二千石。父虯,早以名德著聞,仕歷瓜州司馬、敦煌郡守、郢州刺史,封長城縣子。魏大統末,卒於家。周文帝傷悼之,遣使者監護喪事,又敕鄉人為營墳壟。贈龍驤將軍、瓜州刺史。整幼聰敏,沈深有識量,學藝騎射並為河右所推。



 刺史魏東陽王元榮辟整為主簿,加盪寇將軍。整進趨詳雅,對揚辯暢,謁見之際,州府傾目。榮器整德望,嘗謂僚屬曰:「令狐延保,西州令望,方成重器,豈州郡之職所可縶維?但一日千里,必基武步,寡人當委以庶務,畫諾而已。」頃之,孝武西遷,河右擾亂。榮仗整防扞,州境獲寧。及鄧彥竊據瓜州,拒
 不受代,整與開府張穆等密應使者申徽,執彥送京師。周文嘉其忠節,表為都督。尋而城人張保又殺刺史成慶,與涼州刺史宇文仲和構逆,規據河西。晉昌人呂興復害郡守郭肆,以郡應保。初,保等將圖為亂,慮整守義不從,既殺成慶,因欲及整。然人之望,復恐其下叛之,遂不敢害。雖外加禮敬,內甚忌整。整亦偽若親附,而密欲圖之。陰令所親說保曰「郡與仲和結為脣齒,令東軍漸逼涼州,彼勢孤危,恐不能敵。若或摧衄,則禍及此土。宜分遣銳師,星言救援。二州合勢,則東軍可圖。然後保境息人,計之上者。」保然之,而未知所任。整又令說保曰:「
 歷觀成敗,在於任使,所擇不善,旋致傾危。令狐延保兼資文武,才堪統御,若使為將,蔑不濟矣。」保納其計,且以整父兄等並在城中,弗之疑也,遂令整行。整至玉門郡,召集豪傑,說保罪逆,馳還襲之。先定晉昌,斬呂興,進軍擊保。州人素服整威名,並棄保來附。保遂奔吐谷渾。眾議推整為刺史。整曰:「本以張保肆逆,殺害無辜,闔州之人,俱陷不義。今者同心,務在除凶,若共相推薦,復恐效尤致禍。」於是乃波斯使主張道義行州事。具以狀聞。詔以申徽為刺史。徵整赴闕,授壽昌郡守,封襄武縣男。周文謂整曰:「卿早建殊勳,今官位未足酬賞,方當與卿共平
 天下,同取富貴。」遂立為瓜州義首。整以國難未寧,常願舉宗效力,遂率鄉親二千餘人入朝,隨軍征討。整善於撫馭,躬同豐約,是以士眾並忘羈旅,盡其力用。周文嘗從容謂整曰:「卿遠祖立忠而來,可謂積善餘慶,世濟其美者也。」整遠祖漢建威將軍邁,不為王莽屈,其子稱避地河右,故周文稱之云。累遷驃騎大將軍、開府儀同三司,加侍中。周文又謂整曰:「卿勳同婁、項,義等骨肉,立身敦雅,可以範人。」遂賜姓宇文氏,并賜名整焉。宗人二百餘戶,並列屬籍。



 周孝閔帝踐阼,拜司憲中大夫,處法平允,為當時所稱。進爵彭城縣公。初,梁興州刺史席固以
 州來附,周文以固為豐州刺史。固蒞職既久,猶忌梁法,凡所施為,多虧政典。朝議密欲代之,而難其選。令整權鎮豐州,委以代固之略。整廣布威恩,傾身撫接,數月之間,化洽州府。於是除整豐州刺史,以固為湖州。豐州舊不居民中,賦役參集,勞逸不均。整請移居武當,詔可其奏。獎勵撫導,遷者如歸,旬月之間,城府周備。固之遷也,其部曲多願留為整左右,整諭以朝制,弗之許焉,莫不流涕而去。及整秩滿代至,人吏戀之,老幼送整,遠近畢集,數日停留,方得出界。其得人心如此。拜御正中大夫,出為中華郡守,轉同州司會,遷始州刺史。



 整雅識情偽,
 尤明政術,恭謹廉慎,常懼盈滿,故歷居內外,所在見稱。進位大將軍。晉公護之初執政也,欲委整以腹心。整辭不敢當,頗忤其意,護以此疏之。及護誅,附會者咸伏法,而整獨保全。時人稱其先覺。卒。贈本官,加四州諸軍事、鄜州刺史,謚曰襄。子熙嗣。



 熙字長熙。性嚴重,有雅量,雖在私室,終日儼然。不妄通賓客,凡所交結,必一時名士。博覽群書,尤明三禮,善騎射,頗知音律。起家以通經為吏部上士,轉夏官府都上士,俱有能名。以母憂去職,殆不勝喪。其父戒之曰:「大孝在於安親,義不絕嗣。吾今見存,汝又隻立,何得過爾毀
 頓,貽吾憂也?」熙自是稍加饘粥。服闋,除少駕部。復丁父憂,非杖不起。人有聞其哭聲,莫不為之下泣。河陰之役,詔令墨衰從事,授職方下大夫,襲彭城縣公。及武帝平齊,以留守功,進位儀同。歷司勛、吏部二曹中大夫,甚有當時譽。隋文帝受禪之際,熙以本官行納言事。尋除司徒左長史,加上儀同,進爵河南郡公。時吐谷渾寇邊,以行軍長史從元帥元諧討之,以功進上開府。後拜滄州刺史,在職數年,風教大洽,稱為良二千石。



 開皇四年,上幸洛陽。熙來朝,吏人恐其遷,悲泣於道。及還,百姓出境迎謁,歡叫盈路。在州獲白烏、白麞、嘉麥,甘露降於庭前
 柳樹。八年,徙為河北道行臺度支尚書。吏人追思,相與立碑頌德。及行臺廢,累遷鴻臚卿。後以本官兼吏部尚書,往判五曹尚書事,民為明幹。上甚任之,用上祠太山,還次汴州,惡其殷盛,多有姦俠,以熙為汴州刺史。下車,禁游食,抑工商,人有向術開門者杜之,船客停於郭外,星居者勒為聚落,僑人逐令歸本,其有滯獄,並決遣之,令行禁止。上聞而嘉之,顧侍臣曰:「鄴都,天下難臨處,敕相州刺史豆處通,令習熙法。」其年來朝,考績為天下之最。賜帛三百疋,頒告天下。



 以嶺南夷數起亂,徵拜桂州總管、十七州諸軍事,許以便宜從事,刺史已下官,得承
 制補授,給帳內五百人。賜帛五百疋,發傳送其家累,改封武康郡公。熙至部,大弘恩信。其溪洞渠帥更相謂曰:「前總管皆以兵威相脅,今者乃以手教相諭,我輩其可違乎!」於是相率歸附。先是州縣生梗,長吏多不得之官,寄政於總管府。



 熙悉遣之,為建城邑,開設學校,人夷感化焉。



 時有寧猛力者,與陳後主同日生,自言貌有貴相,在陳世已據南海。平陳後,文帝因而撫之,即拜安州刺史。然驕倨恃險,未常參謁。熙手書諭之,申以交友之分。其母有疾,熙復遺以藥。猛力感之,詣府請謁,不敢為非。熙以州縣多有同名,於是奏改安州為欽州,黃州為峰州,
 利州為智州,德州為歡州,東寧州為融州。上皆從之。在職數年,上表以年老疾患,請解所任。優詔不許,賜以醫藥。



 熙奉詔令交州梁帥李佛子入朝,佛子欲為亂,請至仲冬上道。熙意在羈縻,遂從之。有人詣闕,訟熙受佛子賂而捨之。上聞。佛子反問至,上大怒,以為信然,遣使鎖熙詣闕。熙性素剛,鬱鬱不得志,行至永州,憂憤病卒。上怒不解,沒其家財。及行軍總管劉方禽佛子送京師,言熙實無贓。上悟,乃召其四子聽仕。少子德棻最知名。



 整弟休,幼聰敏,有文武材用。與整同起兵逐張保,授帥都督。後為中外府樂曹參軍。時諸功臣多為本州刺史。晉
 公護謂整曰:「以公勛望,應得本州,但朝廷藉公委任,無容遠出。然公一門之內,須有衣錦之榮。」乃以休為敦煌郡守。在郡十餘年,甚有政績。卒於合州刺史。



 唐永,北海平壽人也。本居晉昌之憤安縣,晉亂,徙於丹楊。祖揣,始遠魏,官至北海太守,因家焉。父倫,青州刺史。永身長八尺,少耿介,有將帥才,讀《班超傳》,慨然有萬里之志。正光中,為北地太守,當郡別將。俄而賊將宿勤明達、車金雀等寇郡境,永擊破之,境內稍安。永善馭下,士人競為之用。熙陣常著帛裙襦,把角如意以指麾處分,辭色自若。在北地四年,與賊數十戰,未常敗北。



 時人語
 曰:「莫陸梁,恐爾逢唐將。」永所營處,至今猶稱唐公壘也。行臺蕭寶夤表永為南幽州刺史,夷人送故者,莫不垂淚,當路遮留,隨數日,始得出境。大統元年,拜東雍州刺史,尋加衛將軍,封平壽伯。卒,贈司空公。永性清廉,家無蓄積,妻子不免飢寒,世以此稱之。



 子陵,少習武藝,頗閑吏職,位大都督、應州刺史、車騎大將軍、儀同三司。



 陵子悟,美風儀,博涉經史,文詠可觀。周大象中,頗被宣帝任遇,位至內史下大夫、漢陽公。隋文帝得政,廢於家而卒。陵弟瑾。



 瑾字附璘。性溫恭,有器量,博涉經史,雅好屬文。身長八
 尺二寸,容貌甚偉。



 年十七,周文聞其名,乃貽永書曰:「聞公有二子,曰陵、曰瑾,陵從橫多武略,瑾雍容富文雅,可並遣入朝,孤欲委以文武之任。」因召拜尚書員外郎、相府記室參軍事。軍書羽檄,瑾多掌之。從破沙苑,戰河橋,並有功,封姑藏縣子。累遷尚書右丞、吏部郎中。于時魏室播遷,庶務草創,朝章國典,瑾並參之。遷戶部尚書,進位驃騎大將軍、開府儀同三司,賜姓宇文氏。



 時燕公于謹,勳高望重,朝野所屬。白周文,言瑾學行兼修,願與之同姓,結為兄弟,庶子孫承其餘論,有益義方。周文歎異者久之,更賜瑾姓萬紐于氏。謹乃深相結納,敦長幼之
 序;瑾亦庭羅子孫,行弟姪之敬。其為朝望所宗如此。進爵臨淄縣伯。轉吏部尚書,銓綜衡流,雅有人倫之鑒。以父憂去職,尋起令視事。時六尚書皆一時之秀,周文自謂得人,號為六俊,然瑾尤見器重。于謹南伐江陵,以瑾為元帥府長史,軍中謀略,多出瑾焉。江陵既平,衣冠仕伍,並沒為僕隸。瑾察其才行有片善者,輒議免之,賴瑾獲濟者甚眾。時論多焉。及軍還,諸將多因虜掠,大獲財物。瑾一無所取,唯得書兩車,載之以歸。或白周文曰:「唐瑾大有輜重,悉是梁朝珍玩。」周文初不信之,然欲明其虛實,密遣使檢閱之,唯見墳籍而已。



 乃歎曰:「孤知此人
 來二十許年,明其不以利干義。向若不令檢視,恐常人有投杼之疑。孤所以益明之耳。凡受人委任當如此也。」論平江陵功,進爵為公。



 六官建,授禮部中大夫。出為蔡州刺史,歷拓州、硤州,所在皆有德化,人吏稱之。轉荊州總管府長史。入為吏部中大夫,歷御正、納言、內史中大夫。曾未十旬,遂遷四職,搢紳咸以為榮。久之,除司宗中大夫,兼內史。尋卒于位。贈小宗伯,謚曰方。



 瑾性方重,有風格,退朝休假,恒著衣冠以對妻子,遇迅雷風烈,雖閑夜晏寢,必起,冠帶端笏危坐。又好施與,家無餘財,所得祿賜,常散之宗族,其尤貧乏者,又割膏腴田宅以振之。
 所留遺子孫者,並墝埆之地。朝野以此稱之。撰《新儀》十篇,所著賦、頌、碑、誄二十餘萬言。孫大智嗣。



 瑾次子令則,性好篇章,兼解音律,文多輕艷,為時人所傳。天和初,以齊馭下大夫使於陳。大象中,官至樂部下大夫。仕隋,位太子左庶子。皇太子勇廢,被誅。



 柳敏,字白澤,河東解縣人,晉太常純之七世孫也。父懿,魏車騎大將軍、儀同三司、汾州刺史。敏九歲而孤,事母以孝聞。性好學,涉獵經史,陰陽卜筮之術,靡不習焉。年未弱冠,起家員外散騎侍郎。累遷河東郡丞。朝議以敏之本邑,故有此授。敏雖統御鄉里,而處物平允,甚得時
 譽。及周文剋復河東,見而器異之,乃謂之曰:「今日不喜得河東,喜得卿也。」即拜丞相府參軍事。俄轉戶曹參軍,兼記室。每有四方賓客,恆令接之,爰及吉凶禮儀,亦令監綜。又與蘇綽等修撰新制,為朝廷政典。遷禮部郎中,封武城縣子,加帥都督,領本鄉兵。俄進大都督。遭母憂,居喪,旬日之間,鬢髮半白。尋起為吏部郎中,毀瘠過禮,杖而後起。周文見而歎異之,特加稟賜。及尉遲迥伐蜀,以敏為行軍司馬,軍中籌略,並以委之。益州平,進驃騎大將軍、開府儀同三司,加侍中,遷尚書,賜姓宇文氏。六官建,拜禮部中大夫。



 周孝閔帝踐阼,進爵為公。又除河
 東郡守,尋復徵拜禮部。出為郢州刺史,甚得物情。及將還朝,夷夏士人,感其惠政,並齎酒肴及物產候之於路。敏乃從他道而還。復拜禮部,後改禮部為司宗,仍以敏為之。敏操履方正,性又恭勤,每日將朝,必夙興待旦。又久處臺閣,明練故事,近儀或乖先典者,皆案據舊章,刊正取中。遷小宗伯,監修國史。轉小司馬,又監修律令。進位大將軍,出為鄜州刺史,以疾不之部。武帝平齊,進爵武德郡公。敏自建德以後,寢疾積年,武帝及宣帝並親幸其第問疾焉。開皇元年,進位上大將軍、太子太保。其年卒。贈五州諸軍事、晉州刺史。臨終戒其子等,喪事所
 須,務從簡約。其子等並涕泣奉行。少子昂。



 昂字千里。幼聰穎有器識,幹局過人。周武帝時,為內史中大夫、開府儀同三司,賜爵文城郡公,當途用事,百僚皆出其下。昂竭誠獻替,知無不為,謙虛自處,未嘗驕物,時論以此重之。武帝崩,受遺輔政,稍被宣帝疏,然不離本職。隋文帝為丞相,深自結納。文帝以為大宗伯。拜日,遂得偏風,不能視事。文帝受禪,疾愈,加上開府,拜潞州刺史。昂見天下無事,上表請勸學行禮。上覽而善之,優詔答昂,自是天下州縣皆置博士習禮焉。昂在州,甚有惠政,卒官。



 子調,歷秘書郎、侍御史。左僕射楊素嘗於朝
 堂見調,因獨言日:「柳條通體弱,獨搖不須風」調斂版正色曰:「調信無取,公不當以為侍御;信有可取,不應發此言。公當具瞻之地,樞機何可輕發!」素甚奇之。煬帝嗣位,累遷尚書左司郎中。時王綱不振,朝士多贓貨,唯調清素守常,為時所美,然乾用非其所長。



 王士良,字君明,其先太原晉陽人也。後因晉亂,避地涼州。魏太武平沮渠氏,曾祖景仁歸魏,為敦煌鎮將。祖公禮,平城鎮司馬,因家於代。父延,蘭陵郡守。



 士良少修謹,不妄交游。孝莊末,爾朱仲遠啟為府參軍事。歷大行臺郎中、諫議大夫,封石門縣男。後與紇豆陵步籓交戰,軍
 敗,為籓所禽,遂居河右。偽行臺紇豆陵伊利欽其才,擢授右丞,妻以孫女。士良既為姻好,便得盡言,遂曉以禍福,伊利等即歸附。朝廷嘉之。太昌初,進爵晉陽縣子,尋進爵瑯邪縣侯,授太中大夫、右將軍。出為殷州車騎府司馬。東魏徙鄴之後,置京畿府,專典兵馬。時齊文襄為大都督,以士良為司馬,領外兵參軍。尋遷長史,加安西將軍,徙封符壘縣侯。武定初,除行臺右中兵郎中,又轉大將軍府屬、從事中郎,仍攝外兵事。王思政鎮穎川,齊文襄率眾攻之。授士良大行臺左丞,加鎮西將軍,進爵為公,令輔其弟演於并州居守。



 齊文宣即位,入為給事
 黃門侍郎,領中書舍人,仍總知并州兵馬事,加征西將軍,別封新豐縣子。俄除驃騎將軍、尚書吏部郎中。文宣自晉陽赴鄴宮,復以士良為尚書左丞,統留後事。仍遷御史中丞。轉七兵尚書。未幾,入為侍中,轉殿中尚書。頃之,復為侍中、吏部尚書。士良少孤,事繼母梁氏以孝聞。及卒,居喪合禮。



 文宣尋起令視事,士良屢表陳誠,再三不許,方應命。文宣見其毀瘠,乃許之。因此臥疾歷年,文宣每自臨視。疾愈,除滄州刺史。乾明初,徵還鄴,授儀同三司。



 孝昭即位,遣三道使搜揚人物。士良與尚書令趙郡王高睿、太常卿崔昂分行郡國,但有一介之善者,無
 不以聞。齊武成初,除太子少傅、少師,復除侍中,轉太常卿,尋加開府儀同三司。出為豫州道行臺、豫州刺史。



 保定四年,晉公護東伐,權景宣以山南兵圍豫州,士良舉城降。授大將軍、小司徒,賜爵廣昌郡公。尋除荊州總管,行荊州刺史,復入為小司徒。俄除鄜州史,轉荊州刺史。士良去鄉既久,忽臨本州,耆老故人,猶有存者,遠近咸以為榮。加授上大將軍,以老病乞骸骨,優詔許之。開皇元年卒,時年八十二。



 子德衡,大象末,儀同大將軍。



 論曰:昔陽貨外叛,庶其竊邑,而《春秋》譏之;韓信背項,陳平歸漢,而史遷美之。蓋以運屬既安,君道已著,則徇利
 忘德者罪也;時逢擾攘,臣禮未備,則轉禍為福者可也。崔彥穆、楊纂、段永等昔在山東,沈渝下位,並以羈旅之士,邅回於燕雀之伍,終佩龜組,可謂見機者乎?令狐整乾用確然,雅望重於河右,處州里則勳著方隅,升朝廷則績宣出內,而畏避權寵,克保終吉,不然,何以自致顯名而取高位也?熙歷職流譽,風政克舉,雖古之循吏,亦何以加茲,而毫釐為爽,丘山成過,唯命也夫!唐永良能之名,所在著美,清白之譽,顯於累職,所謂幹能之士也。瑾、敏並挺巳梓之林,蘊瑚璉之器,博觀載籍,多識舊章,固乃國之名臣,時之領袖,周無君子,斯焉取斯。王士良
 之仕於齊,職居卿牧,而失忠與義,臨難茍免,其背叛之徒歟!



\end{pinyinscope}