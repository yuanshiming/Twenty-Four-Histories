\article{卷四魏本紀第四}

\begin{pinyinscope}

 世宗宣武皇帝諱恪,孝文皇帝第二子也。母曰高夫人。初,夢為日所逐,避於床下。日化為龍,繞己數匝,寤而驚悸,遂有娠。太和七年閏四月,生帝於平城宮。



 二十一年正月丙申,立為皇太子。



 二十三年四月丙午,孝文帝崩。丁巳,太子即皇帝位。諒闇,委政宰輔。五月,高麗國遣使朝貢。六月乙卯,分遣侍臣,巡行州郡,問人疾苦,考察守令,黜陟幽明,褒禮名賢。戊辰,追尊皇妣曰文昭皇后。秋
 八月戊申,遵遺詔,孝文皇帝三夫人已下,悉免歸家。癸丑,增宮臣位一級。冬十月癸未,鄧至國王象舒彭來朝。丙戌,謁長陵。丁酉,享太廟。十一月,幽州人王惠定聚眾反,自稱明法皇帝。刺史李肅捕斬之。是歲,州鎮十八水饑,分遣使者,開倉振恤。



 景明元年春正月辛丑朔,日有蝕之。壬寅,謁長陵。乙巳,大赦,改元。丁未,齊豫州刺史裴叔業以壽春內屬。二月戊戌,復以彭城王勰為司徒。齊將胡松、李居士軍屯宛,陳伯之水軍逼壽春。夏四月丙申,司徒彭城王勰、車騎將軍王肅大破之。



 己亥,皇弟恌薨。五月甲寅,北鎮饑,遣
 兼侍中楊播巡撫振恤。六月丙子,以司徒、彭城王勰為大司馬。秋七月己亥朔,日有蝕之。齊將陳伯之寇淮南。八月乙酉,彭城王勰破伯之於肥口。九月,齊州人柳世明聚眾反。冬十月丁卯朔,謁長陵。庚寅,齊、兗二州討世明平之。丁亥,改授彭城王勰司徒、錄尚書事。十一月丁巳,陽平王頤薨。是歲,州鎮十七大饑,分遣使者,開倉振恤。高麗、吐谷渾等國並遣使朝貢。



 二年春正月丙申朔,謁長陵。庚戌,帝始親政。遵遺詔。聽司徒、彭城王勰以老歸第。進太尉、咸陽王禧位太保,以司空、北海王詳為大將軍、錄尚書事。丁巳,引見群臣於
 太極前殿,告以覽政之意。壬戌,以太保、咸陽王禧領太尉,以大將軍、廣陵王羽為司空。分遣大使。黜陟幽明。二月庚午,進宿衛之官位一級。甲戌,大赦。三月乙未朔,詔以比年連有軍旅,正調之外,皆蠲罷。壬戌,青、齊、徐、兗四州大饑,人死者萬餘口。是月,齊雍州刺史蕭衍奉其南康王寶融為主,東赴建鄴。



 夏五月壬子,廣陵王羽薨。壬戌,太保、咸陽王禧謀反,賜死。六月丁亥,考諸州刺史,加以黜陟。秋七月癸巳朔,日有蝕之。乙巳,蠕蠕犯塞。辛酉,大赦。九月丁酉,發畿內夫五萬五千人築京師三百二十坊,四旬罷。己亥,立皇后于氏。乙卯,免壽春營戶,並隸
 揚州。冬十一月丙申,以驃騎大將軍穆亮為司空。丁酉,以大將軍、北海王詳為太傅,領司徒。壬寅,改築圓丘於伊水之陽。乙卯,仍有事焉。十二月,齊直後張齊殺其主蕭寶卷以降蕭衍。是歲,高麗、吐谷渾等國並遣使朝貢。



 三年春二月戊寅,以旱故,詔州郡掩骸骨。三月,齊建安王寶夤來奔。夏四月,詔撫軍將軍李崇討魯陽反蠻。齊主蕭寶融遜位於梁。閏四月丁巳,司空穆亮薨。秋七月丁巳朔,日有蝕之。八月丁卯,以前太傅、平陽公元丕為三老。九月丁巳,行幸鄴。丁卯,詔使者弔比干墓。戊寅,閱武於鄴南。冬十月庚子,帝躬御弧矢射,遠及一里五十
 步,群臣勒銘於射所。甲辰,車駕還宮。十二月壬寅,以太極前殿初成,饗群臣,賜布帛有差。是歲,河州大饑,死者二千餘口。西域二十七國並遣使朝貢。



 四年春正月乙亥,親耕籍田。三月己巳,皇后先蠶於北郊。四月癸未朔,以蕭寶夤為東揚州刺史,封丹楊郡公、齊王。庚寅,南天竺國獻辟支佛牙。戊戌,為旱故,命鞫冤獄。己亥,減膳徹懸。辛丑,澍雨大洽。五月甲戌,行梁州事楊椿大破反氐。六月壬午朔,封皇弟悅為汝南王。秋七月乙卯,三老平陽公元丕薨。庚午,詔復收鹽池利。辛未,以彭城王勰為太師。八月,勿吉國貢楛矢。冬十一月己
 未,封武興國世子楊紹先為武興王。



 正始元年春正月丙寅,大赦,改元。夏五月丁未朔,太傅、北海王詳以罪廢為庶人。六月,以旱故,徹樂減膳。癸巳,詔有司修案舊典,祗行六事。甲午,帝以旱故,親薦享於太廟。戊戌,詔立周旦、夷、齊廟於首陽山。庚子,以旱故,公卿以下,引咎責躬。又錄京師見囚,殊死以下皆減一等;鞭杖之坐,悉原之。秋七月丙子,假鎮南將軍李崇大破諸蠻。八月丙子,假鎮南將軍元英破梁將馬仙琕於義陽。



 詔洛陽令有大事,聽面敷奏。乙酉,元英攻拔義陽。辛卯,英又大破梁軍,仍清三關。丁酉,封英為中山王。九月,
 詔諸州蠲停徭役,不得橫有徵發。蠕蠕犯塞,詔左僕射源懷討之。冬十月乙未,詔斷群臣白衣募吏。十一月戊午,詔有司依漢、魏舊章,營繕國學。十二月丙子,以苑牧公田分賜代遷戶。己卯,詔群臣議定律令。



 閏月癸卯朔,行梁州事夏侯道遷據漢中來降。乙丑,以高陽王雍為司空。是歲,高麗遣使來朝貢。



 二年春正月丙子,封宕昌世子梁彌博為宕昌王。二月,梁州氐、蜀反,絕漢中運路,州刺史邢巒頻大破之。夏四月己未,城陽王鸞薨。乙丑,詔曰:「中正所銓,但為門第,吏部彞倫,仍不才舉。八坐可審議往代擢賢之禮,必令才
 學並申,資望兼致。」邢巒遣統軍王足西伐,頻破梁諸軍,遂入劍閣。秋七月戊子,王足擊破梁軍,因逼涪城。八月壬寅,詔中山王英南討襄沔。冬十一月戊辰朔,武興王楊紹先叔父集起謀反,詔光祿大夫楊椿討之。王足圍涪城,益州諸郡戍降者十二三,送編籍者五萬餘戶。既而足引軍退。是歲,鄧至國遣使朝貢。



 三年春正月丁卯朔,皇子昌生,大赦。壬申,梁、秦二州刺史邢巒連破氐賊,剋武興。秦州人王智等聚眾,自號王公,尋推秦州主簿呂茍兒為主,年號建明。己卯,楊集起兄弟相率降。二月丙辰,詔求讜言。戊午,詔右衛將軍元
 麗等討呂茍兒。



 三月己巳,以戎旅興,詔停諸作。己卯,樂良王長命坐殺人,賜死。庚寅,平南將軍、曲江縣公陳伯之自梁城南奔。夏四月丁未,詔罷鹽池禁。五月丙寅,詔以時澤未降,春稼已旱,或有孤老餒疾無人贍救,因以致死,暴露溝塹者,令洛陽部尉,依法棺埋。秋七月庚辰,元麗大破秦賊,降呂茍兒及其王公三十餘人,秦、涇二州平。戊子,中山王英大破梁徐州刺史王伯敖於陰陵。己丑,詔發定、冀、瀛、相、並、肆六州卒十萬,以濟南軍。八月壬寅,安東將軍邢巒破梁將桓和於孤山。諸將所在剋捷,兗州平。壬戌,曲赦涇、秦、岐、涼、河五州。九月癸酉,邢巒
 大破梁軍於淮南,遂攻鐘離。冬十一月甲子,帝為京兆王愉、清河王懌、廣平王懷、汝南王悅講《孝經》於式乾殿。是歲,高麗、蠕蠕國並遣使朝貢。



 四年夏四月戊戌,鐘離大水,中山王英敗績而還。六月己丑朔,詔有司準前式,置國子,立太學,樹小學於四門。秋八月己亥,中山王英、齊王蕭寶夤坐鐘離敗,除名。辛丑,敦煌人饑,詔開倉振恤。九月己未,詔以徙正宮極,庸績未酬,以司空、高陽王雍為太尉,尚書令、廣陽王嘉為司空,百官悉進位一級。庚申,夏州長史曹明謀反,伏誅。甲子,開斜谷舊道。丙戌,司州人饑,詔開倉振恤。閏月甲
 午,禁大司馬門不得車馬出入。冬十月丁卯,皇后于氏崩。自碣石至於劍閣,東西七千里,置二十二郡尉。是歲,西域、東夷四十餘國並遣使朝貢。



 永平元年春三月戊子,皇子昌薨。丙午,以去年旱儉,遣使者所在振恤。夏五月辛卯,帝以旱故,減膳徹懸。六月壬申,詔依洛陽舊圖,修聽訟觀。秋七月甲午,立夫人高氏為皇后。八月壬子朔,日有蝕之。癸亥,冀州刺史、京兆王愉據州反。



 丁卯,大赦,改元。九月丙戌,復前中山王英本封。戊戌,殺太師、彭城王勰。癸卯,假鎮北將軍李平剋信都,冀州平。冬十月,豫州彭城人白早生殺刺史司馬
 悅,據城南叛。十二月己未,尚書邢巒剋懸瓠,斬早生,禽梁將齊茍兒等。是歲,北狄、東夷、西域十八國並遣使朝貢。高昌國王曲嘉表求內徙。



 二年春正月,涇州沙門劉慧汪聚眾反,詔華州刺史奚康生討之。夏四月己酉,武川鎮饑,詔開倉振恤。甲子,詔緣邊州鎮,自今一不聽寇盜境外,犯者罪同境內。



 五月辛丑,帝以旱故,減膳徹懸,禁斷屠殺。甲辰,幸華林都亭錄囚徒,死罪以下,降一等。六月辛亥,詔曰:「江海方同,車書宜一,諸州軌轍,南北不等。今可申敕四方,遠近無二。」秋八月丙午朔,日有蝕之。戊申,以鄧至國世子像覽蹄
 為其國王。九月辛巳,封故北海王子顥為北海王。壬午,詔定諸門闥名。冬十月癸丑,以司空、廣陽王喜為司徒。庚午,郢州獻七寶床,詔不納。冬十一月甲申,詔禁屠殺含孕,以為永制。己丑,帝於式乾殿為諸僧、朝臣講《維摩詰經》。冬十二月,詔五等諸侯,其同姓者出身:公,正六下;侯,從六上;伯,從六下;子,正七上;男,正七下。異族出身:公,從七上;侯,從七下;伯,正八上;子,正八下;男,從八上。清修出身:公,從八下;侯,正九上;伯,正九下;子,從九上;男,從九下。是歲,西域、東夷二十四國並遣使朝貢。



 三年春二月壬子,秦州沙門劉光秀謀反,州郡捕斬之。
 癸亥,秦州隴西羌殺鎮將趙俊反,州軍討平之。三月丙戌,皇子詡生,大赦。夏四月,平陽郡之禽昌、襄陵二縣大疫,自正月至此月,死者二千七百三十人。五月丁亥,冀、定二州旱儉,詔開倉振恤。六月甲寅,詔重求遺書於天下。冬十月辛卯,中山王英薨。丙申,詔太常立館,使京畿內外疾病之徒,咸令居處。嚴敕醫署分師救療,考其能否而行賞罰。又令有司集諸醫工,惟簡精要,取三十卷以班九服。十二月辛巳,江陽王繼坐事除名。甲申,詔於青州立孝文皇帝廟。殿中侍御史王敞謀反,伏誅。是歲,西域、東夷、北狄十六國並遣使朝貢。



 四年春正月丁巳,汾州劉龍駒聚眾反,詔諫議大夫薛和討之。二月壬午,青、齊、徐、兗四州人饑甚,遣使振恤。三月壬戌,司徒、廣陽王嘉薨。夏四月,梁遣其鎮北將軍張稷及馬仙琕寇朐山。詔徐州刺史盧昶率眾赴之。五月己亥,遷代京銅龍置天泉池西。丙辰,詔禁天文學。冬十一月,朐山城陷,盧昶大敗而還。十二月壬戌朔,日有蝕之。是歲,西域、東夷、北狄二十九國並遣使朝貢。



 延昌元年春正月乙巳,以頻年水旱,百姓饑弊,分遣使者,開倉振恤。丙辰,以尚書令高肇為司徒,清河王懌為司空。三月甲午,州郡十一大水,詔開倉振恤。



 以京師穀
 貴,出倉粟八十萬石以振恤貧者。己未,安樂王詮薨。夏四月,詔以旱故,斷食粟之畜。丁卯,詔曰:「遷京嵩縣,年將二紀,博士端然虛祿。靖言念之,有兼愧慨。可嚴敕有司,國子學,孟冬使成。太學、四門,明年暮春令就。」戊辰,以旱故,詔尚書與群司鞫理獄訟。辛未,詔饑人就穀六鎮。丁丑,帝以旱故,減膳徹懸。癸未,詔曰:「肆州地震陷裂,死傷甚多。亡者不可復追,生病宜加療救。



 可遣太醫、折傷醫并給所須藥就療。」乙酉,大赦,改元。詔立理訴殿、申訟車,以盡冤窮之理。五月丙午,詔天下有粟之家,供年之外,悉貸饑人。自二月不雨至於是月。己未晦,日有蝕之。六
 月壬申,澍雨大洽。戊寅,通河南牝馬之禁。庚辰,詔出太倉粟五十萬石,以振京師及州郡饑人。冬十月乙亥,立皇子詡為皇太子。十一月丙申,詔以東宮建,賜天下為父後者爵一級。孝子順孫廉夫節婦旌表門閭,量給粟帛。十二月己巳,詔守宰為御史彈赦免者,及考在中第,皆代之。是歲,西域、東夷十國並遣使朝貢。



 二年春正月戊戌,帝御申訟車,親理冤訟。二月丙辰朔,振恤京師貧人。甲戌,以六鎮大饑,開倉拯贍。己卯,進太尉、高陽王雍位太保。閏月辛丑,以苑牧地賜代遷人無田者。是春,人饑,死者數萬口。夏四月庚子,以絹十五萬
 疋振河南郡人。



 五月甲寅朔,日有蝕之。是月,壽春大水。遣平東將軍奚康生等步騎數千赴之。六月乙酉,青州人饑,詔使者開倉振恤。甲午,曲赦揚州。辛亥,帝御申訟車,親理冤訟。是夏,十三郡大水。秋八月辛卯,詔以水旱饑儉,百姓多陷罪辜,降死以下刑。九月丙辰,以貴族豪門,崇習奢侈,詔尚書嚴立限級,節其流宕。冬十月,詔以恆、肆地震,人多死傷,重丐一年租賦。十二月丙戌,丐洛陽、河陰二縣租賦。



 乙巳,詔以恆、肆地震,人多離災,其有課丁沒盡,老幼單立,家無受復者,各賜廩粟,以接來稔。是歲,東夷、西域十餘國並遣使朝貢。



 三年春二月乙未,詔曰:「肆州秀容郡敷城縣、鴈門郡原平縣並自去年四月以來,山鳴地震,于今不已。告譴彰咎,朕甚懼焉。可恤瘼寬刑,以答災譴。」夏四月,青州人饑。辛巳,開倉振恤。乙巳,上御申訟車,親理冤訟。秋八月甲申,帝臨朝堂,考百司而加黜陟。冬十一月辛亥,詔司徒高肇為大將軍、平蜀大都督,步騎十五萬,西伐益州。丁巳,幽州沙門劉僧紹聚眾反,自號凈居國明法王。州郡捕斬之。十二月庚寅,詔立明堂。是歲,東夷、西域八國並遣使朝貢。



 四年春正月甲寅,帝不豫。丁巳,崩于式乾殿,時年三十
 三。二月甲戌朔,上尊謚曰宣武皇帝,廟號世宗。甲午,葬景陵。帝幼有大度,喜怒不形於色,雅性儉素。初,孝文欲觀諸子志尚,大陳寶物,任其所取。京兆王愉等皆競取珍玩,帝唯取骨如意而已。孝文大奇之。及庶人恂失德,孝文謂彭城王勰曰:「吾國疑此兒有非常志相,今果然矣!」乃見立為儲貳。雅愛經史,尤長釋氏之義,每至講論,連夜忘疲。善風儀,美容貌,臨朝深默,端嚴若神,有人君之量矣。



 肅宗孝明皇帝諱詡,宣武皇帝之第二子也。母曰胡充華。永平三年三月丙戌,生於宣光殿之東北,有光照於
 庭中。延昌元年十月乙亥,立為皇太子。



 四年正月丁巳,宣武帝崩。是夜,太子即皇帝位。戊午,大赦。己未,徵下西討東防諸軍。庚申,詔太保、高陽王雍入居西栢堂決庶政,以任城王澄為尚書令,百官總己以聽二王。二月庚辰,尊皇后高氏為皇太后。辛巳,司徒高肇至京師,以罪賜死。癸未,進太保、高陽王雍位太傅,領太尉。以司空、清河王懌為司徒,以驃騎大將軍、廣平王懷為司空。乙亥,尊胡充華為皇太妃。三月甲辰朔,皇太后出俗為尼,徙御金墉城。丙辰,詔進宮臣位一級。乙丑,進文武群官位一級。夏六月,沙門法慶聚眾反於冀州,殺阜城令,自稱
 大乘。秋八月乙亥,領軍於忠矯詔殺左僕射郭祚、尚書裴植,免太傅、高陽王雍官,以王還第。丙子,尊皇太妃為皇太后。



 戊子,帝朝太后於宣光殿。大赦。己丑,進司徒、清河王懌為太傅,領太尉。以司空、廣平王懷為太保,領司徒。任城王澄為司空。庚寅,以車騎大將軍于忠為尚書令,特進崔光為車騎大將軍,並儀同三司。壬辰,復江陽王繼本國,復濟南王彧先封為臨淮王。群臣奏請皇太后臨朝稱制。九月乙巳,皇太后親覽萬機。甲寅,征西大將軍元遙破斬法慶,傳首京師。安定王燮薨。冬十二月辛丑,以高陽王雍為太師。



 己酉,鎮南將軍崔亮破梁將
 趙祖悅軍,遂圍硤石。丁卯,帝、皇太后謁景陵。是歲,東夷、西域、北狄十八國並遣使朝貢。



 熙平元年春正月戊辰朔,大赦,改元。荊沔都督元志大破梁軍。以吏部尚書李平為行臺,節度討硤石諸軍。二月乙巳,鎮東將軍蕭寶夤大破梁將於淮北。癸亥,初聽秀才對策,第中上已上敘之。乙丑,鎮南崔亮、鎮軍李平等剋硤石,斬趙祖悅,傳首京師,盡俘其眾。三月戊辰朔,日有蝕之。夏四月戊戌,以瀛州人饑,開倉振恤。五月丁卯朔,以炎旱,命釐察獄訟,權停作役。庚午,詔放華林野獸於山澤。



 秋七月庚午,重申殺牛禁。八月丙午,詔古帝
 諸陵四面各五十步,勿聽耕稼。九月丁丑,淮堰破,梁緣淮城戍村落十餘萬口,皆漂入海。是歲,吐谷渾、宕昌、鄧至、高昌、陰平等國並遣使朝貢。



 二年春正月,大乘餘賊,復相聚攻瀛州,刺史宇文福討平之。甲戌,大赦。庚寅,詔遣大使巡行四方,問疾苦,恤孤寡,黜陟幽明。二月丁未,封御史中尉元匡為東平王。三月丁亥,太保、領司徒、廣平王懷薨。夏四月丁酉,詔京尹所統年高者,板賜郡各有差。戊申,以開府儀同三司胡國珍為司徒。乙卯,皇太后幸伊闕石窟寺,即日還宮。改封安定王超為北平王。五月庚辰,重申天文禁,犯者以
 大辟論。



 秋七月乙亥,儀同三司、汝南王悅坐殺人免官,以王還第。己巳,享太廟。八月戊戌,宴道武以來宗室年十五以上於顯陽殿,申家人禮。己亥,詔庶族子弟年未十五,不聽入仕。庚子,詔咸陽、京兆二王子女,還附屬籍。丁未,詔太師、高陽王雍入居門下,參決尚書奏事。冬十月,以幽、冀、滄、瀛、光五州饑,遣使巡撫,開倉振恤。是歲,東夷、西域、氐、羌等十一國並遣使朝貢。



 神龜元年春正月甲子,詔以氐酋楊定為陰平王。壬申,詔給京畿及諸州老人板郡縣各有差,及賜鰥寡孤獨粟帛。庚辰,詔以雜役戶或冒入清流,所在職人,皆令五
 人相保。無人任保者,奪官還役。乙酉,秦州羌反。幽州大饑,死者三千七百九十人。詔刺史開倉振恤。二月己酉,詔以神龜表瑞,大赦,改元。東益州氐反。三月,南秦州氐反。夏四月丁酉,司徒胡國珍薨。甲辰,改封江陽王繼為京兆王。六月,自正月不雨,是月辛卯,澍雨乃降。秋七月,河州人卻鐵匆聚眾反,自稱水池王。閏月甲辰,開恒州銀山禁。八月癸丑朔,詔京師見囚殊死以下,悉減一等。甲子,卻鐵匆詣行臺源子恭降。九月戊申,皇太后高氏崩于瑤光寺。冬十月丁卯,以尼禮葬高太后於芒山。十二月辛未,詔曰:「人生有終,下歸兆域。京邑隱振,口盈億
 萬,貴賤攸憑,未有定所。今制乾脯山以西,擬為九原。」是歲,東夷、西域、北狄十一國並遣使朝貢。



 二年春正月辛巳朔,日有蝕之。丁亥,詔曰:「皇太后捴挹自居,稱號弗備。



 宜遵舊典,稱詔宇內,以副黎蒸元元之望。」是月,改葬文昭皇太后高氏。二月乙丑,齊郡王祐薨。庚午,羽林千餘人焚征西將軍張彞第,毆傷彞,燒殺其子均。乙亥,大赦。丁丑,詔求直言。壬寅,詔以旱故,命依舊雩祈,察理冤獄,掩胔埋骼,振窮恤寡。三月甲辰,澍雨大洽。夏五月戊戌,以司空、任城王澄為司徒,京兆王繼為司空。秋八月乙未,御史中尉、東平王匡坐事削除官爵。
 九月庚寅,皇太后幸嵩高山。癸巳,還宮。冬十二月癸丑,司徒、任城王澄薨。庚申,大赦。詔除淫祀,焚諸雜神。是歲,吐谷渾、宕昌、嚈噠等國並遣使朝貢。



 正光元年春正月乙亥朔,日有蝕之。夏四月丙辰,詔尚書長孫承業巡撫北蕃,觀察風俗。五月辛巳,以炎旱故,詔八坐鞫見囚,申枉濫。秋七月丙子,侍中元叉、中常侍劉騰奉帝幸前殿,矯皇太后詔,歸政遜位。乃幽皇太后北宮,殺太傅、清河王懌,總勒禁旅,決事殿中。辛卯,帝加元服,大赦,改元。內外百官進位一等。



 八月甲寅,相州刺史、中山王熙舉兵欲誅騰。不果,見殺。九月壬辰,蠕蠕主
 阿那瑰來奔。戊戌,以太師、高陽王雍為丞相。冬十月乙卯,以儀同三司、汝南王悅為太尉。十一月己亥,封阿那瑰為朔方郡公、蠕蠕王。十二月壬子,詔送蠕蠕王阿那瑰歸北。辛酉,以司空、京兆王繼為司徒。



 二年春正月,南秦州氐反。二月,車駕幸國子學,講《孝經》。三月庚午,幸國子學,祠孔子,以顏回配。甲午,右衛將軍奚康生於禁中將殺元叉,不果,為叉所害。以儀同三司劉騰為司空。夏四月庚子,進司徒、京兆王繼位太保。壬寅,以儀同三司崔光為司徒。五月丁酉朔,日有蝕之。秋七月己丑,以旱故,詔有司修案舊典,祗行六事。八月己
 巳,蠕蠕後主郁久閭侯匿代來奔懷朔鎮。十二月甲戌,詔司徒崔光、安豐王延明等議定服章。庚辰,以東益、南秦州氐反,詔河間王琛討之,失利。是歲,烏萇、居密、波斯、高昌、勿吉、伏羅、高車等國並遣使朝貢。



 三年春正月辛亥,耕籍田。夏四月庚辰,以高車國主覆羅伊匐為鎮西將軍、西海郡公、高車國王。五月壬辰朔,日有蝕之。六月己巳,以旱故,詔分遣有司馳祈岳瀆及諸山川百神能興雲雨者。命理冤獄,止土功,減膳徹懸,禁止屠殺。冬十一月己丑朔,日有蝕之。己巳,祀圓丘。丙午,詔班歷,大赦。十二月癸酉,以太保、京兆王繼為太傅,
 司徒崔光為太保。是歲,波斯、不溪、龜茲、吐谷渾並遣使朝貢。



 四年春二月壬申,追封故咸陽王禧為敷城王,京兆王愉為臨洮王,清河王懌為范陽王,以禮加葬。丁丑,河間王琛、章武王融並以貪污削爵除名。己卯,蠕蠕主阿那瑰率眾犯塞。遣尚書左丞元孚為北道行臺,持節喻之。蠕蠕後主郁久閭侯匿代來朝。司空劉騰薨。夏四月,阿那瑰執元孚北遁。秋八月癸未,追復故范陽王懌為河間王。九月丁酉,詔太尉、汝南王悅入居門下,與丞相、高陽王雍參決尚書奏事。



 冬十一月癸未朔,日有蝕之。丙
 申,趙郡王謐薨。丁酉,太保崔光薨。十二月,以太尉、汝南王悅為太保。徐州刺史、北海王顥坐貪汙,削爵除官。是歲,宕昌、庫莫奚國並遣使朝貢。



 五年春正月辛丑,祀南郊。三月,沃野鎮人破六韓拔陵反,聚眾殺鎮將,號真王元年。夏四月,高平酋長胡琛反,自稱高平王,攻鎮以應拔陵。別將盧祖遷擊破之。五月,都督北征諸軍事、臨淮王彧攻討,敗於五原,削除官爵。壬申,詔尚書令李崇為大都督,率廣陽王淵等北討。六月,秦州城人莫折大提據城反,自稱秦王,殺刺史李彥。大提尋死,子念生代立,僭稱天子,年號天建,置立百官。
 丁酉,大赦。秋七月戊午,復河間王琛、臨淮王彧本封。是月,涼州幢帥于菩提、呼延雄執刺史宋潁,據州反。念生遣其兄高陽王天生下隴東寇。八月甲午,雍州刺史元志西討,大敗於隴東,退守岐州。丙申,詔諸州鎮軍元非犯配者,悉免為編戶。改鎮為州,依舊立稱。九月壬申,詔尚書左僕射、齊王蕭寶夤為西道行臺、大都督。復撫軍、北海王顥官爵,為都督。並率諸將西討。乙亥,帝幸明堂,餞寶夤等。吐谷渾主伏連籌遣兵討涼州。于菩提走,追斬之。城人趙天安復推宋潁為刺史。冬十月,營州城人劉安定、就德興據城反,執刺史李仲遵。城人王惡兒斬
 安定以降。德興東走,自號燕王。十二月,詔太傅、京兆王繼為太師、大將軍,率諸將西討。汾州正平、平陽胡叛逆,詔復征東將軍、章武王融封爵,為大都督,率眾討之。莫折念生遣兵攻涼州,城人趙天安復執刺史以應之。是歲,嚈噠、契丹、地豆干、庫莫奚等國並遣使朝貢。



 孝昌元年春正月庚申,徐州刺史元法僧據城反,自稱宋王,年號天啟。遣其子景仲歸梁。梁遣其將豫章王綜入守彭城。法僧擁其僚屬南入。詔臨淮王彧、尚書李憲為都督,安豐王延明為東道行臺,俱討徐州。癸亥,蕭寶夤及征西將軍崔延伯大破賊於黑水。天生退走入隴,
 涇、岐及隴悉平。以太師、大將軍、京兆王繼為太尉。



 二月,詔追復故樂良王長命爵,以其子忠紹之。戊戌,大赦。三月甲戌,詔五品以上,各薦所知。夏四月辛卯,皇太后復臨朝攝政,引群臣面陳得失。壬辰,征西將軍、都督崔延伯大敗於涇川,戰歿。六月癸未,大赦,改元。蠕蠕主阿那瑰大破拔陵。是月,諸將逼彭城,蕭綜夜潛出降,梁諸將奔退,眾軍追躡,免者十一二。秋八月癸酉,詔斷遠近貢獻珍麗,違者免官。柔玄鎮人杜洛周反於上谷,年號真王。



 九月乙卯,詔減天下諸調之半。壬戌,詔五品以上,各舉所知。辛未,曲赦南北秦州。冬十月,蠕蠕遣使朝貢。十
 一月辛亥,詔父母年八十以上者,皆聽居官。時四方多事,諸蠻復反。十二月,山胡劉蠡升反,自稱天子。



 二年春正月庚戌,封廣平王懷長子誨為范陽王。壬子,以太保、汝南王悅領太尉。是月,五原鮮於修禮反於定州,年號魯興。二月庚申,帝及皇太后臨大夏門,親覽冤訟。三月庚子,追復中山王熙本爵,以其子叔仁紹之。夏四月,大赦。戊申,北討都督河間王琛、長孫承業失利奔還,詔並免官爵。五月丁未,車駕將北討,內外戒嚴。前給事黃門侍郎元略自梁還朝,封義陽王。以丞相、高陽王雍為大司馬。



 六月己巳,曲赦齊州。絳蜀陳雙熾聚眾反,
 自號始建王。曲赦平陽、建興、正平三郡。詔假鎮西將軍、都督長孫承業討雙熾,平之。丙子,改封義陽王略為東平王。



 戊寅,詔復京兆王繼本封江陽王。戊子,詔曰:「自運屬艱棘,歷載於茲。朕威德不能遐被,經略無以及遠,何以茍安黃屋,無愧黔黎!今便避居正殿,蔬食素服。



 當親自招募,收集忠勇。其有直言正諫之士,敢決徇義之夫,二十五日,悉集華林東門。人別引見,共論得失。」秋八月丙子,進封廣川縣公元為常山王。戊子,進武城縣公子攸為長樂王。癸巳,賊帥元洪業斬鮮於修禮請降,為賊黨葛榮所殺。九月辛亥,葛榮敗都督廣陽王深、章武
 王融於博野白牛邏。融歿於陣。榮自稱天子,國號齊,年稱廣安。冬十一月戊戌,杜洛周攻陷幽州,執刺史王延年及行臺常景。



 丙午,稅京師田租,畝五升。借賃公田者,畝一斗。閏月,稅市,人出入者,各一錢,店舍為五等。梁將元樹逼壽春,揚州刺史李憲力屈而降。初留州縣及長史、司馬、戍主副質子於京師。詔:「頃舊京淪覆,中原喪亂,宗室子女屬籍在七廟內為雜戶濫門拘辱者,悉聽離絕。」是歲,疊伏羅、庫莫奚國並遣使朝貢。



 三年春正月甲戌,以司空皇甫度為司徒,儀同三司蕭寶夤為司空。辛巳,葛榮陷殷州,刺史崔楷固節死之。甲
 申,詔峻鑄錢之制。蕭寶夤大敗于涇州,北海王顥尋亦敗走。曲赦關西及正平、平陽、建興。戊子,以司徒皇甫度為太尉。己丑,以四方未平,詔內外戒嚴,將親征。二月丁酉,詔開輸賞格。輸粟入瀛、定、岐、雍四州者,官斗二百斛賞一階;入二華州者,五百石賞一階。不限多少,粟畢授官。



 虜賊據潼關。三月甲子,詔將西討,中外戒嚴。虜賊走,復潼關。秋七月,相州刺史、安樂王鑒據州反。己丑,大赦。八月,都督源子邕、李神軌、裴衍攻鄴。丁未,斬鑒,相州平。九月己未,東豫州刺史元慶和以城南叛。秦州城人杜粲殺莫折念生,自行州事。冬十月戊申,曲赦恆農已西
 河北、正平、平陽、邵郡及關西諸州。甲寅,雍州刺史蕭寶夤據州反,自號齊,年稱隆緒。十一月己丑,葛榮攻陷冀州,執刺史元孚,逐出居人,凍死者十六七。十二月戊申,都督源子邕、裴衍與榮戰,敗於陽平東北,並歿。是月,杜粲為駱超所殺。超遣使歸罪。是歲,蠕蠕遣使朝貢。



 武泰元年春正月乙丑,生皇女,秘言皇子。丙寅,大赦,改元。丁丑,雍州人侯終德相率攻蕭寶夤。寶夤度渭走,雍州平。二月癸丑,帝崩於顯陽殿,時年十九。



 甲寅,皇子即位,大赦。皇太后詔曰:「皇家握歷受圖,年將二百,祖宗累聖,社稷載安。高祖以文思先天,世宗以下武繼世。大行
 在御,重以寬仁奉養,率由溫明恭順。實望穹靈降祐,麟趾眾繁。自潘充華有孕椒宮,冀誕儲兩,而熊羆無兆,唯虺遂彰。于時直以國步未康,假稱統胤。欲以底定物情,係仰宸極。何圖一旦弓劍莫追!皇曾孫故臨洮王寶暉世子釗,體自高祖,天表卓異。大行平日養愛特深,義齊若子,事符當璧,允膺大寶。即日踐祚。可班宣遠邇,咸使知之。」乙卯,幼主即位。儀同三司、大都督爾朱榮抗表請入奔赴,勒兵而南。是月,杜洛周為葛榮所并。三月甲申,上尊謚曰孝明皇帝。乙酉,葬於定陵,廟號肅宗。四月戊戌,爾朱榮濟河。庚子,皇太后、幼主崩。



 論曰:宣武承聖考德業,天下想望風化,垂拱無為,邊徼稽服。而寬以攝下,從容不斷,太和之風替矣。比之漢世,元、成、安、順之儔歟。宣武之後,繼以元成,孝明沖齡統業,靈后婦人專制,任用非人,賞罰乖舛。於是釁起宇內,禍延邦畿。卒於享國不長,抑亦淪胥之始也。



\end{pinyinscope}