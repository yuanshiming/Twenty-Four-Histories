\article{卷四十八列傳第三十六}

\begin{pinyinscope}

 爾硃榮子文暢文略從子兆從弟彥伯彥伯子敞彥伯弟仲遠世隆榮從父弟度律榮從祖兄子天光爾朱榮,字天寶,北秀容人也。世為部落酋帥,其先居爾硃川,因為氏焉。高祖羽健,魏登國初為領人酋長,率契胡武士從平晉陽,定中山,拜散騎常侍。以居秀容川,詔割方三百里封之,長為世業。道武初,以南秀容川原沃衍,欲令居之。



 羽健曰:「家世奉國,給侍左右,北秀容既在
 內,差近京師,豈以沃脊,更遷遠地?」帝許之。所居處曾有狗舐地,因而穿之得甘泉,因名狗舐泉。曾祖鬱德、祖代勤,繼為酋長。代勤,太武敬哀皇后舅也。既以外親,兼數征伐有功,給復百年,除立義將軍。曾圍山而獵,部人射虎,誤中其髀。代勤仍令拔箭,竟不推問,曰:「此既過誤,何忍加罪。」部內咸感其意。位肆州刺史,封梁郡公,以老致仕,歲賜帛百疋以為常。卒,謚曰莊。孝莊初,追贈太師、司徒公、錄尚書事。父新興,太和中繼為酋長。曾行馬群,見一白蛇,頭有兩角,咒之,求畜牧蕃息。自是牛羊駝馬,日覺滋盛,色別為群,谷量之。朝廷每有征討,輒獻私馬,
 兼備資糧,助裨軍用。孝文嘉之。及遷洛,特聽冬朝京師,夏歸部落。每入朝,諸公王朝貴,競以珍玩遺之,新興亦報以名馬。位散騎常侍、平北將軍、秀容第一領人酋長。新興每春秋二時,恒與妻子閱畜牧於川澤,射獵自娛。明帝時,以年老,啟求傳爵於榮。



 卒,謚曰簡。孝莊初,贈太師、相國、西河郡王。



 榮潔白,美容貌,幼而神機明決。及長,好射獵,每設圍誓眾,便為軍陣之法,號令嚴肅,眾莫敢犯。秀容界有池三所,在高山上,清深不測,相傳曰祁連池,魏言天池也。父新興曾與榮游池上,忽聞簫鼓音,謂榮曰:「古老相傳,聞此聲,皆至公輔。吾年老暮,當為汝耳。」
 榮襲爵,後除直寢、游擊將軍。正光中,四方兵起,遂散畜牧,招合義勇。以討賊功,進封博陵郡公,其梁郡前爵聽賜第二子。時榮率眾至肆州,刺史尉慶賓閉城不納。榮怒,攻拔之,乃署其從叔羽生為刺史,執慶賓還秀容。自是兵威漸盛,朝廷亦不能罪責。及葛榮吞杜洛周,榮恐其南逼鄴城,表求東援相州,帝不許。榮以山東賊盛,慮其西逸,乃遣兵固守滏口以防之。於是北捍馬邑,東塞井陘。尋屬明帝崩,事出倉卒,榮乃與元天穆等密議,入匡朝廷。



 抗表云:「今海內草草,異口一言,皆云大行皇帝鴆毒致禍,舉潘嬪之女以誑百姓,奉未言之兒而臨四
 海。求以徐紇、鄭儼之徒,付之司敗。更召宗親,推其明德。」



 於是將赴京師。靈太后甚懼,詔以李神軌為大都督,將於太行杜防。榮抗表之始,遣從子天光、親信奚毅及倉頭王相入洛,與從弟世隆密議廢立。天光乃見莊帝,具論榮心,帝許之。天光等還北,榮發晉陽,猶疑所立,乃以銅鑄孝文及咸陽王禧等五王子孫像,成者當奉為主。唯莊帝獨就。師次河內,重遣王相密迎莊帝與帝兄彭城王邵、弟始平王子正。武泰元年四月,莊帝自高渚度,至榮軍,將士咸稱萬歲。



 及莊帝即位,詔以榮為使持節、都督中外諸軍事、大將軍、開府、尚書令、領軍將軍、領左
 右、太原王。及度河,太后乃下髮入道,內外百官皆向河橋迎駕。榮惑武衛將軍費穆之言,謂天下乘機可取,乃譎朝士共為盟誓,將向河陰西北三里。



 至南北長堤,悉命下馬西度,即遣胡騎四面圍之。妄言丞相高陽王欲反,殺百官王公卿士二千餘人,皆斂手就戮。又命二三十人拔刀走行宮。莊帝及彭城王、霸城王俱出帳。榮先遣并州人郭羅察共西部高車叱列殺鬼在帝左右,相與為應。及見事起,假言防衛,抱帝入帳,餘人即害彭城、霸城二王。乃令四五十人遷帝於河橋,沉靈太后及少主於河。時又有朝士百餘人後至,仍於堤東被圍。遂臨
 以白刃,唱云:「能為禪文者出,當原其命。」時有隴西李神俊、頓丘李諧、太原溫子昇並當世辭人,皆在圍中,恥是從命,俯伏不應。有御史趙元則者,恐不免死,出作禪文。榮令人誡軍士,言元氏既滅,爾朱氏興。其眾咸稱萬歲。榮遂鑄金為己像,數四不成。



 時榮所信幽州人劉靈助善卜占,言今時人事未可。榮乃曰:「若我作不吉,當迎天穆立之。」靈助曰:「天穆亦不吉,唯長樂王有王兆耳。」榮亦精神恍惚,不自支持,遂便愧悔。至四更中,乃迎莊帝,望馬首叩頭請死。其士馬三千餘騎,既濫殺朝士,乃不敢入京,即欲向北為移都之計。持疑經日,始奉駕向洛陽
 宮。及上北芒,視城闕,復懷畏懼,不肯更前。武衛將軍汎禮苦執不聽,復前入城,不朝戍。北來之人,皆乘馬入殿。諸貴死散,無復次序。莊帝左右,唯有故舊數人。榮猶執移都之議,上亦無以拒焉。又在明光殿重謝河橋之事,誓言無復二心。莊帝自起止之,因復為榮誓,言無疑心。榮喜,因求酒一遍。及醉熟,帝欲誅之,左右苦諫乃止。



 即以床輦向中常侍省。榮夜半方寤,遂達旦不眠。自此不復禁中宿矣。



 榮女先為明帝嬪,欲上立為后,帝疑未決。給事黃門侍郎祖瑩曰:「昔文公在秦,懷嬴入侍。事有反經合義,陛下獨何疑焉?」上遂從之。榮意甚悅。于時,人間
 猶或云榮欲遷都晉陽,或云欲肆兵大掠,迭相驚恐,人情駭震。京邑士子,十不一存,率皆逃竄,無敢出者,直衛空虛,官守廢曠。榮聞之,上書謝愆。無上王請追尊帝號;諸王、刺史,乞贈三司;其位班三品,請贈令僕;五品之官,各贈方伯;六品已下及白身,贈以鎮郡。諸死者無後,聽繼,即授封爵。均其高下,節級別科,使恩洽存亡,有慰生死。詔從所表。又啟帝,遣使巡城勞問。於是人情遂安,朝士逃亡者,亦稍來歸闕。榮又奏請番直,朔望之日,引見三公、令、僕、尚書、九卿及司州牧、河南尹、洛陽河陰執事之官,參論國政,以為常式。



 五月,榮還晉陽,乃令元天穆
 向京,為侍中、太尉公、錄尚書事、京畿大都督,兼領軍將軍,封上黨王,樹置腹心在列職,舉止所為,皆由其意。七月,詔加榮柱國大將軍。



 時葛榮向京師,眾號百萬。州刺史李神俊閉門自守。榮率精騎七千,馬皆有副,倍道兼行,東出滏口。而與葛榮眾寡非敵。葛榮聞之,喜見於色,乃令其眾辦長繩,至便縛取。自鄴以北,列陣數十里,箕張而進。榮潛軍山谷為奇兵,分督將已上三人為一處,處有數百騎,令所在揚塵鼓噪,使賊不測多少。又以人馬逼戰,刀不如棒。密勒軍士,馬上各齎袖棒一枚,至戰時,慮廢騰逐,不聽斬級,使以棒,棒之而已。乃分命壯
 勇,所當衝突。號令嚴明,將士同奮。身自陷陣,出於賊後,表裏合擊,大破之。於陣禽葛榮,餘眾悉降。榮恐其疑懼,乃普令各從所樂,親屬相隨,任所居止。於是群情喜悅,登即四散,數十萬眾,一朝散盡。待出百里之外,乃始分道押領,隨便安置,咸得其宜。獲其渠帥,量才授用,新附者咸安。時人服其處分機速。乃檻車送葛榮赴闕。詔加榮大丞相、都督河北畿外諸軍事。初,榮將討葛榮,軍次襄垣,遂大獵,有雙兔起於馬前,榮彎弓誓之曰:「中則禽葛榮,不中則否。」既而並應弦而殪,三軍咸悅。及後,命立碑於其所,號雙兔碑。又將戰,夜夢一人從葛榮索千牛
 刀,葛榮初不肯與,此人自稱己是道武皇帝,葛榮乃奉刀,此人手持授榮。寤而喜。自知必勝。又詔以冀州之長樂、相州之南趙、定州之博陵、滄州之浮陽、平州之遼西、燕州之上谷、幽州之漁陽七郡,各萬戶,通前滿十萬。



 為太原國邑,又加位太師。



 建義初,北海王元顥南奔梁,梁立為魏主,資以兵將。時邢杲以三齊應顥。朝廷以顥孤弱,永安二年春,詔元天穆先平齊地,然後徵顥。顥乘虛徑進,榮陽、武牢並不守,車駕出居河北。榮聞之,馳傳朝行宮於上黨之長子,輿駕於是南趣。榮為前驅,旬日之間,兵馬大集。天穆克平邢杲,亦度河以會。車駕幸河內。
 榮與顥相持於河上,無船不得即度。議欲還北,更圖後舉。黃門郎楊侃、高道穆等並固執以為不可。屬馬渚諸楊云有小船數艘,求為鄉導。榮乃令都督爾朱兆等率精騎夜濟。



 顥奔。車駕度河,入居華林園。詔加榮天柱大將軍,增封通前二十萬戶,加前後部羽葆鼓吹。



 榮尋還晉陽,遙制朝廷,親戚腹心,皆補要職,百僚朝廷動靜,莫不以申。至於除授,皆須榮許,然後得用。莊帝雖受制權臣,而勤政事,朝夕省納,孜孜不已。



 數自理冤獄,親覽辭訟。又選司多濫,與吏部尚書李神俊議正綱紀。而榮乃大相嫌責。曾關補定州曲陽縣令,神俊以階縣不奏,別
 更擬人。榮大怒,即遣其所補者往奪其任。榮使入京,雖復微蔑,朝貴見之,莫不傾靡。及至闕下,未得通奏,恃榮威勢,至乃忿怒。神俊遂上表遜位。榮欲用世隆攝選,上亦不違。榮曾啟北人為河內諸州,欲為掎角勢,上不即從。天穆入見論事,上猶未許。天穆曰:「天柱既有大功,為國宰相,若請普代天下官屬,恐陛下亦不得違。如何啟數人為州,便停不用?」帝正色曰:「天柱若不為人臣,朕亦須代;如其猶存臣節,無代天下百官理。」



 榮聞,大怒曰:「天子由誰得立?今乃不用我!」語皇后復嫌內妃嬪甚有妒恨之事。



 帝遣世隆語以大理,后曰:「天子由我家置立,今
 便如此。我父本日即自作,今亦復決?」世隆曰:「兄止自不為,若本自作,臣今亦得封王。」帝既外迫強臣,內逼皇后,恒怏怏不以萬乘為貴。



 先是,葛榮枝黨韓婁仍據幽、平二州,榮遣都督侯深討斬之。時萬俟醜奴、蕭寶夤擁眾豳、涇,榮遣其從子天光為雍州刺史,令率都督賀拔岳、侯莫陳悅等入關討之。天光至雍州,以眾少未進。榮大怒,遣其騎兵參軍劉貴馳驛詣軍,加天光杖罰。天光等大懼,乃進討,連破之,禽醜奴、寶夤,並檻車送闕。天光又禽王慶雲、萬俟道樂,關中悉平。於是天下大難便盡。莊帝恒不慮外寇,唯恐榮為逆。常時諸方未定,欲使與之
 相持。及告捷之日,乃不甚喜,謂尚書令、臨淮王彧曰:「即今天下,便是無賊?」臨淮見帝色不悅,曰:「臣恐賊平以後,方勞聖慮。」帝畏餘人怪,還以他語解之,曰:「其實撫寧荒餘,彌成不易。」



 榮好射獵,不捨寒暑,法禁嚴重。若一鹿出,乃有數人殞命。曾有一人,見猛獸便走,謂曰:「欲求活邪!」遂即斬之。自此獵如登戰場。曾見一猛獸在窮谷中,乃令餘人重衣空手搏之,不令復損。於是數人被殺,遂禽得之。持此而樂焉。列圍而進,雖阻險不得迴避,其下甚苦之。



 太宰元天穆從容言榮勳業,宜調政養人。榮便攘肘謂天穆曰:「太后女主,不能自正,推奉天子者,此是人
 臣常節。葛容之徒,本是奴才,乘時作亂,譬如奴走,禽獲便休。頃來受國大寵,未能混一海內,何宜今日便言勛也?如聞朝士猶自寬縱,今秋欲共兄戒勒士馬,校獵嵩原,令貪汙朝貴,入圍搏虎。仍出魯陽,歷三荊,悉擁生蠻,北填六鎮。迴軍之際,因平汾胡。明年簡練精騎,分出江、淮,蕭衍若降,乞萬戶侯;如其不降,徑度數千騎,便往縛取。待六合寧一,八表無塵,然後共兄奉天子巡四方,觀風俗,布政教,如此乃可稱勳耳。今若止獵,兵士懈怠,安可復用也?」



 及見四方無事,乃遣人奏曰:「參軍許周勸臣取九錫,臣惡其此言,已發遣令去。」榮時望得殊禮,故以意
 諷朝廷。帝實不欲與之,因稱其忠。榮見帝年長明悟,為眾所歸,欲移自近,皆使由己。每因醉云,入將天子,拜謁金陵後,還復恒朔。



 而侍中硃元龍輒從尚書索太和中遷京故事,於是復有移都消息。



 榮乃暫來向京,言看皇后娩難。帝懲河陰之事,終恐難保,乃與城陽王徽、侍中楊侃、李彧、尚書右僕射元羅謀,皆勸帝刺殺之。唯膠東侯李侃晞、濟陰王暉業言榮若來,必有備,恐不可圖。又欲殺其黨與,發兵拒之。帝疑未定,而京師人懷憂懼,中書侍郎邢子才之徒,已避之東出。榮乃遍與朝士書,相任留。中書舍人溫子昇以書呈帝,帝恒望其不來,及見
 書,以榮必來,色甚不悅。武衛將軍奚毅,建義初往來通命,帝每期之甚重,然以為榮通親,不敢與之言情。毅曰:「若必有變,臣寧死陛下難,不能事契胡。」帝曰:「朕保天柱無異心,亦不忘卿忠款。」



 三年八月,榮將四五千騎,發并州向京。時人皆言其反,復道天子必應圖之。



 九月初,榮至京。有人告云,帝欲圖之。榮即具奏。帝曰:「外人亦言王欲害我,豈可信之?」於是榮不自疑,每入謁帝,從人不過數十,皆不持兵仗。帝欲止,城陽王曰:「縱不反,亦何可耐?況何可保耶?」又北人語訛,語「爾朱」為「人主」。



 上又聞其在北言,我姓人主。先是,長星出中台,掃大角,恒州人高榮
 祖頗明天文,榮問之曰:「是何祥也?」答曰:「除舊布新象也。昔長星掃大角,秦以之亡。」



 榮聞之悅。又榮下行臺郎中李顯和曾曰:「天柱至,那無九錫,安須王自索也?亦是天子不見機!」都督郭羅察曰:「今年真可作禪文,何但九錫。」參軍褚光曰:「人言并州城上有紫氣,何慮天柱不應。」榮下人皆陵侮帝左右,無所忌憚,其事皆上聞。奚毅又見,求聞。帝即下明光殿與語。帝又疑其為榮,不告以情。及知毅赤誠,乃召城陽王徽及楊侃、李彧,告以毅語。榮小女嫁與帝兄子陳留王,小字伽邪。榮嘗指之曰:「我終當得此女婿力。」徽又云:「榮慮陛下終為此患,脫有東宮,必
 貪立孩幼。若皇后不生太子,則立陳留以安天下。」并言榮指陳留語狀。帝既有圖榮意,夜夢手持一刀自害,落十指節,都不覺痛。惡之,以告城陽王徽及楊侃。徽解夢曰:「蝮蛇螫手,壯士解腕。割指節與解腕何異?去患乃是吉祥。」聞者皆言善。



 九月十五日,天穆到京,駕迎之。榮與天穆並從入西林園燕射。榮乃奏曰:「近來侍官皆不習武,陛下宜將五百騎出獵,因省辭訟。」先是奚毅言榮因獵挾天子移都,至是,其言相符。至十八日,召中書舍人溫子昇告以殺榮狀,并問以殺董卓事。子昇具通本,上曰:「王允若即赦涼州人,必不應至此。」良久,語子昇曰:「朕
 之情理,卿所具知,死猶須為,況必不死!寧與高貴鄉公同日死,不與常道鄉公同日生。」上謂殺榮、天穆,即赦其黨,便應不動。應詔王道習曰:「爾朱世隆、司馬子如、朱元龍比來偏被委付,具知天下虛實,謂不宜留。」城陽王及楊侃曰:「若世隆不全,仲遠、天光豈有來理?」帝亦謂然,無復殺意。城陽曰:「榮數征伐,腰間有刀,或能狠戾傷人。臨事,願陛下出。」乃伏侃等十餘人於明光殿東。



 其日,榮與天穆並入,坐食未訖,起出。侃等從東階上殿,見榮、天穆出至中庭,事不果。十九日是帝忌日。二十日榮忌日。二十一日,暫入,即向陳留王家,飲酒極醉。遂言病動,頻日
 不入。上謀頗泄,世隆等以告榮。榮輕帝,不謂能反。預帝謀者皆懼。二十五日旦,榮、天穆同入,其日大欲革易。上在明光殿東序中西面坐,榮與天穆並御床西北小床上南坐,城陽入,始一拜。榮見光祿卿魯安等持刀從東戶入,即馳向御坐,帝拔千牛刀,手斬之,時年三十八。得其手板上有數牒啟,皆左右去留人名,非其腹心,悉在出限。帝曰:「豎子!若過今日,便不可制。」時又天穆與榮子菩提亦就戮,於是內外喜叫,聲滿京城。既而大赦。



 榮雖威名大振,而舉止輕脫,止以馳射為伎藝,每入朝見,更無所為,唯戲上下馬。於西林園宴射,恒請皇后出觀,并
 召王公妃主,共在一堂。每見天子射中,輒自起舞叫,將相卿士,悉皆盤旋,乃至妃主婦人,亦不免隨之舉袂。及酒酣耳熱,必自匡坐,唱虜歌,為《樹梨普梨》之曲。見臨淮王彧從容閑雅,愛尚風素,固令為敕勒舞。日暮罷歸,便與左右連手蹋地,唱《迴波樂》而出。性甚嚴暴,慍喜無恒,弓箭刀槊,不離於手,每有瞋嫌,即行忍害,左右恒有死憂。曾欲出獵,有人訴之,披陳不已,發怒,即射殺之。曾見沙彌重騎一馬,榮即令相觸,力窮不復能動,遂使傍人以頭相擊,死而後已。



 節閔帝初,世隆等得志,乃詔贈假黃鉞、相國、錄尚書、都督中外諸軍事、晉王,加九錫,給九
 旒鑾輅,武賁班劍三百人,轀輬車,準晉太宰、安平獻王故事,謚曰武。又詔百官議榮配饗,司直劉季明曰:「晉王若配永安,則不能終臣節。以此論之,無所配。」世隆作色曰:「卿合配?」季明曰:「下官預在議限,據理而言,不合上心,誅翦唯命。」眾為之危,季明自若。世隆意不已,乃配享孝文廟庭。



 菩提位太常卿、開府儀同三司、侍中、特進。死時年十四。節閔帝初,加贈司徒,謚曰惠。



 菩提弟叉羅,武衛將軍、梁郡王。尋卒,贈司空公。



 叉羅弟文殊,封平昌郡王。孝靜初,轉襲榮爵太原王。薨於晉陽,時年九歲。



 文殊弟文暢,初封昌樂郡公。以榮破葛賊之勳,進爵為王。其姊
 魏孝莊皇后。



 及韓陵之敗,齊神武納之,待其家甚厚。文暢由是拜開府儀同三司、肆州刺史。家富於財,招致賓客,窮極豪侈。與丞相司馬任胄、主簿李世林、都督鄭仲禮、房子遠等相狎,外示盃酒交,而潛謀害齊神武。自魏氏舊俗,以正月十五日夜為打蔟戲,能中者即時賞帛。胄令仲禮藏刀於褲中,因神武臨觀,謀竊發,事捷,共奉文暢。



 為任氏家客薛季孝所告。以姊寵,止坐文暢一房。文暢死時年十八。



 弟文略,以兄叉羅卒無後,襲叉羅爵梁郡王。文暢事當從坐,靜帝使人往晉陽,欲拉殺之。神武特加寬貸,奏免之。文略聰明俊爽,多所通習。齊文襄
 嘗令章永興馬上彈琵琶,奏十餘曲,試使文略寫之,遂得八。文襄戲之曰:「聰明人多不老壽,梁郡其慎之!」文略對曰:「命之修短,皆在明公。」文襄愴然曰:「此不足慮。」



 初,神武遣令恕文略十死,恃此益橫,多所陵忽。齊天保末,嘗邀平秦、武興、汝南諸王至宅,供設奢麗,各有贈賄。諸王共假聚寶物以要之,文略弊衣而往,從奴五十人,皆駿馬侯服。其豪縱不遜如此。平秦王有七百里馬,文略敵以好婢,賭取之。明日,平秦王使人致請,文略殺馬及婢,以二銀器盛婢頭馬肉而遺之。平秦王訴之於文宣,繫於京畿獄。文略彈琵琶,吹橫笛,謠詠倦極,便臥唱挽歌。
 居數月,奪防者弓矢以射人,曰:「不然,天子不憶我。」有司奏,遂伏法。文略嘗大遺魏收金,請為父作佳傳,收論榮比韋、彭、伊、霍,蓋由是也。



 兆字萬仁,榮從子也。少善騎射,趫捷過人,數從榮游獵,至窮巖絕澗,人所不能升降者,兆必先之。手格猛獸,無所疑避。榮以此特加賞愛,任為爪牙。榮曾送臺使,見二鹿,授兆二箭,令取供今食。遂構火以待之。俄而兆獲其一,榮欲誇使人,責兆不盡取,杖之五十。榮之入洛,兆兼前鋒都督。孝莊即位,封潁川郡公。



 後從上黨王天穆平邢杲。又與賀拔勝擊斬元顥子冠受,禽之。進破安豐王
 延明,顥乃退走。莊帝還宮,論功除車騎大將軍、儀同三司、汾州刺史。



 爾朱榮死,兆自汾州據晉陽。元曄立,授兆大將軍,進爵為王。兆與世隆等定謀攻洛。兆遂輕兵倍道,掩襲京邑。先是,河邊人夢神謂己曰:「爾朱家欲度河,用爾作水壘波津令,為之縮水脈。」月餘,夢者死。及兆至,有行人自言知水淺處,以草往往表插而導焉,忽失其所在。兆遂策馬涉度。是日暴風鼓怒,黃塵張天,騎叩宮門,宿衛乃覺。彎弓欲射,袍撥弦,矢不得發,一時散走。莊帝步出雲龍門外,為兆騎所擊,幽於永寧佛寺。兆撲殺皇子,汙辱妃嬪,縱兵虜掠。停洛旬餘,先令衛送莊帝於晉
 陽,兆後於河梁監閱財貨。



 初,兆將入洛,遣使招齊神武,欲與同舉。神武時為晉州刺史,謂長史孫騰曰:「臣而伐君,其逆已甚。我今不往,恐彼致恨,卿可往申吾意,但云山蜀未平,不可委去。」騰乃詣兆,具申意。兆不悅,曰:「還白高兄弟,有吉夢,今行必克。



 吾比夢吾亡父登一高堆,堆傍地悉耕熟,唯有馬蘭草株,往往猶在,吾父顧我,令下拔之。吾手所至,無不盡出。以此而言,往必有利。」騰還,具報之。神武曰:「兆等猖狂,舉兵犯順,吾勢不可反事爾朱也。今天子列兵河上,兆進不能度,必退還。吾乘山東下,出其不意,此徒可一舉而禽。」俄而兆克京師,孝莊幽縶,
 都督尉景從兆南行,以書報神武。神武大驚,召騰,令馳驛詣兆,示以謁賀,密觀天子所在,當於路邀迎,唱大義於天下。騰遇帝於中路,神武時率騎東轉,聞帝已度,於是西還。仍與兆書,具陳禍福,不宜害天子,受惡名於海內。兆怒不納,而帝遂遇弒。



 初,榮既死,莊帝詔河西人紇豆陵步蕃等,令襲秀容。兆入洛後,步蕃兵勢甚盛,南逼晉陽。兆所以不暇留洛,迴師禦之。頻為步蕃所敗,於是部勒士馬,謀出山東,令人頻徵神武。神武晉州僚屬,並勸不行。神武揣其勢迫,必無他慮,決策赴之。兆乃分三州六鎮之人,令神武統領。神武既分兵別營,乃引兵南
 出,避步蕃之銳。步蕃至樂平郡,神武與兆還討,破斬之。及節閔帝立,授兆使持節、侍中、都督中外諸軍事、柱國大將軍,兼錄尚書事、大行臺。又以兆為天柱大將軍,兆以是榮所終之官,固辭不拜。尋加都督十州諸軍事,世襲并州刺史。



 神武之克殷州也,兆與仲遠、度律約拒之。仲遠、度律次陽平,兆屯廣阿,眾號十萬。神武廣縱反間,於是兩不相信,各致猜疑。仲遠等頻使斛斯椿賀拔勝往喻之。兆輕騎三百,來就仲遠,同坐幕下。兆性粗獷,意色不平,手舞馬鞭,長嘯凝望,深疑仲遠等有變,遂趨出馳還。仲遠遣椿、勝等追而曉譬,兆遂拘縛將還,經日放
 遣。仲遠等於是奔退。神武乃進擊,兆軍大敗。兆與仲遠、度律遂相疑阻,久而不和。世隆請節閔納兆女為皇后,兆乃大喜。世隆謀抗神武,乃降辭厚禮,喻兆赴洛。兆與天光、度律更自信約,然後大會韓陵山。戰敗,復奔晉陽。其年秋,神武自鄴進討之,兆遂大掠并州,走於秀容。神武又追擊,度赤洪嶺,破之。兆竄於窮山,殺所乘馬,自縊於樹。神武收葬之。



 兆勇於戰鬥,而無將領之能。榮雖奇其膽決,然每云:「兆不過將三千騎,多則亂矣。」



 兆弟智彪,節閔帝封為安定王。與兆俱走,神武禽之。後死於晉陽。



 彥伯,榮從弟也。祖侯真,文成時并、安二州刺史、始昌侯。
 父買珍,宣武時武衛將軍、華州刺史。



 彥伯性和厚,永安中,為榮府長史。節閔帝潛嘿於龍花佛寺,彥伯敦喻往來,尤有勤款。帝既立,爾朱兆以己不豫謀,大為忿恚,將攻世隆。詔令華山王鷙慰兆,兆猶不釋。世隆復令彥伯自往喻之,兆乃止。及還,帝宴彥伯於顯陽殿。時侍中源子恭、黃門郎竇瑗並侍坐。彥伯曰:「源侍中比為都督,與臣相持於河內。當爾之時,旗鼓相望,眇如天隔。寧期同事陛下,為今日之忻也?」子恭曰:「蒯通有言,犬吠非其主。他日之事永安,猶今日之事陛下耳。」帝曰:「源侍中可謂有射鉤之心也。」遂令二人極醉而罷。後封博陵郡王,位
 司徒公。于時炎旱,有勸彥伯解司徒者,乃上表遜位,詔許之。俄除儀同三司、侍中,餘如故。彥伯於兄弟之中,差無過患。天光等敗於韓陵,彥伯欲領兵屯河橋,世隆不從。及張勸等掩襲世隆,彥伯時在禁直。長孫承業等啟陳,神武義功既振,將除爾朱。節閔令舍人郭崇報彥伯知,彥伯狼狽出走,為人所執。尋與世隆同斬於閶闔門外,縣首於斛斯椿門樹,傳於神武。先是洛中謠曰:「三月末,四月初,揚灰簸土覓真珠。」又曰:「頭去項,腳根齊,驅上樹,不須梯。」至是並驗。子敞。



 敞字乾羅。彥伯之誅,敞小,隨母養於宮中。年十二,敞自
 竇走至大街,見童兒群戲,敞解所著綺羅金翠服,易衣而遁。追騎至,不識敞,便執綺衣兒。比究問知非,會日已暮,由是免。遂入一村,見長孫氏媼,踞胡床坐,敞再拜求哀,長孫氏愍之,藏於復壁之中。購之愈急,追且至,長孫氏資而遣之。遂詐為道士,變姓名,隱嵩高山。略涉經史。數年間,人頗異之。嘗獨坐巖石下,泫然歎曰:「吾豈終此乎!伍子胥獨何人也?」乃奔長安。周文帝見而禮之,拜行臺郎中、靈壽縣伯。



 保定中,遷開府儀同三司,進爵為公。後為膠州刺史。迎長孫氏至其第,置于家,厚資給之。隋文帝受禪,改封邊城郡公。黔安蠻叛,命敞討平之。師旋,
 拜金州總管,政號嚴明,吏人懼之。後以年老乞骸骨,賜二馬輅車歸河內,卒于家。子最嗣。



 仲遠,彥伯弟也。明帝末年,爾朱榮兵威稍盛,諸有啟謁,率多見從。而仲遠摹寫榮書,又刻榮印,與尚書令吏,通為奸詐。造榮啟表,請人為官,大得財貨,以資酒色。落魄無行業。及孝莊即位,封清河公、徐州刺史,兼尚書左僕射、三徐大行臺。尋進督三徐諸軍事。仲遠上言:「竊見比來行臺采募者,皆得權立中正,在軍定第,斟酌授官。今求兼置,權濟軍要。若立第亦爽,關京之日,任有司裁奪」。



 詔從之。於是隨情補授,肆意聚斂。



 爾朱榮死,仲遠勒其
 部眾,來向京師。節閔立,進爵彭城王,加大將軍,又兼尚書令,鎮大梁。仲遠遣使請準朝式,在軍鳴騶。節閔帝覽啟,笑而許之。其肆情如此。復進督東道諸軍事、本將軍、袞州刺史,餘如故。仲遠天性貪暴,心如峻壑。



 大宗富族,誣之以反,沒其家口,簿籍財物,皆以入己。丈夫死者,投之河流,如此者不可勝數。諸將婦有美色者,莫不被其淫亂。自滎陽以東,輸稅悉入其軍,不送京師。時天光控關右,仲遠在大梁,兆據并州,世隆居京邑,各自專恣,權強莫此。所在並以貪虐為事,於是四方解體。又加太宰,解大行臺。仲遠專恣尤劇,方之彥伯、世隆,最為無禮。東
 南牧守,下至人俗,比之豺狼,特為患苦。後移屯東郡,率眾與度律等拒齊神武。爾硃兆領騎數千自晉陽來會。軍次陽平,神武縱以間說,仲遠等迭相猜貳,狼狽遁走。中興二年,復與天光等於韓陵戰敗,南走。尋乃奔梁,死於江南。



 世隆,字榮宗,仲遠弟也。明帝末,兼直閣,加前將軍。爾朱榮表請入朝,靈太后惡之,令世隆詣晉陽慰喻榮。榮因欲留之,世隆曰:「朝廷疑兄,故令世隆來。



 今遂住,便有內備,非計之善。」榮乃遣入。榮舉兵南出,世隆遂走,會榮於上黨。



 建義初,除給事黃門侍郎。莊帝之立,世隆預其謀,
 封樂平郡公。元顥逼大梁,詔為前將軍、都督,鎮武牢。顥既克滎陽,世隆懼而遁還,莊帝倉卒北巡。及車駕還宮,除尚書左僕射,攝選。



 莊帝之將圖爾朱榮,每屏人言。世隆懼變,乃為匿名書,自榜其門曰:「天子與侍中楊侃、黃門高道穆等為計,欲殺天柱。」還復自以此書與榮妻北鄉郡公主,并以呈榮,勸其不入。榮毀書唾地曰:「世隆無膽,誰敢生心!」世隆又勸其速發。



 榮曰:「何忽忽?」皆不見從。



 榮死,世隆奉榮妻,燒西陽門夜走。北次河橋,殺武衛將軍奚毅,率眾還戰大夏門外。及李苗燒絕河梁,世隆乃北遁。攻建州克之,盡殺人以肆其忿。至長子,與度律等
 共推長廣王曄為主。曄小名盆子,聞者皆以為事類赤眉。曄以世隆為尚書令,封樂平郡王,加太傅,行司州牧,會兆於河陽。兆既平京邑,讓世隆曰:「叔父在朝多時,耳目應廣,如何令天柱受禍?」按劍嗔目,詞色甚厲。世隆遜辭拜謝,然後得已,而深恨之。



 時仲遠亦自滑臺入京。世隆與兄弟密謀,慮元曄母干豫朝政,伺其母衛氏出行,遣數十騎如劫賊,於京巷殺之。公私驚愕,莫識所由。尋縣榜,以千萬錢募賊。百姓知之,莫不喪氣。尋又以曄疏遠,欲推立節閔帝。而度律意在南陽王,乃曰:「廣陵不言,何以主天下?」後知能語,遂行廢立。



 初,世隆之為僕射,尚
 書文簿,在家省閱。性聰解,又畏榮,深自剋勉,留心几案,傍接賓客,遂有解了之名。榮死之後,無所顧憚。及為令,常使尚書郎宋游道、邢昕在其宅聽事,東西別座,受納訴訟,稱命施行。既總朝政,生殺自由,公行淫泆,信任群小,隨情與奪。又兄弟群從,各擁強兵,割剝四海,極其貪虐。姦諂蛆酷,多見信用;溫良名士,罕豫腹心。於是天下之人,莫不厭毒。世隆尋讓太傅。節閔特置儀同三師之官,位次上公之下,以世隆為之。贈其父買珍相國、錄尚書事、大司馬。



 及齊神武起義兵,仲遠、度律等愚贛恃強,不以為慮,而世隆獨深憂恐。及天光等敗於韓陵,世隆
 請赦天下,節閔不許。斛斯椿既據河橋,盡殺世隆黨附,令行臺長孫承業詣闕奏狀,掩執世隆及兄彥伯,俱斬之。



 初,世隆曾與吏部尚書元世俊握槊,忽聞局上詨然有聲,一局子盡倒立,世隆甚惡之。又曾晝寢,其妻奚氏忽見一人持世隆首去。奚氏驚,就視,而世隆寢如故。



 既覺,謂妻曰:「向夢人斷我頭持去,意殊不適。」又此年正月晦日,令、僕並不上省,西門不開。忽有河內太守田帖家奴,告省門亭長云:「今旦為令王借車牛一乘,終日於洛濱游觀。至晚,王還省,將車出東掖門,始覺車上無褥,請為記識。」



 亭長以令僕不上,西門不開,無跡入者。此奴固
 陳不已,公文列訴。尚書都令史謝遠疑,謂妄有假借,白世隆,付曹推驗。時都官郎中穆子容究之。奴言,初來時,至司空府西,欲向省。令王嫌遲,遣催車。車入,到省西門,王嫌牛小,繫於關下槐樹,更將一青牛駕車。令王著白紗、高頂帽,短小、黑色,儐從皆裙襦褲褶,握板,不似常時服章。遂遣一吏將奴送入省中事東閣內,東廂第一屋中。其屋先常閉。奴云,入此屋中有板床,床上無席,大有塵土,兼有甕米。奴拂床坐,兼畫地戲,甕中米亦握看之。子容與謝遠看之,閉極久,全無開跡。及入,狀皆符同。具以此對世隆。世隆悵然,意以為惡。未幾見誅。



 世隆弟世承,莊帝時位侍中,領御史中尉。人才猥劣,備員而已。及元顥內逼,世承守轘轅,為顥所禽。顥讓而臠之。莊帝還宮,贈司徒。



 世承弟弼,字輔伯,節閔帝時,封河間郡公。尋為青州刺史。韓陵之敗,欲奔梁,數日,與左右割臂為約。弼帳下都督馮紹隆為弼信待,乃說弼曰:「今方同契闊,宜當心瀝血,示眾以信。」弼從之。大集部下,弼乃踞胡床,令紹隆持刀披心。



 紹隆因推刃殺之,傳首京師。



 度律,榮從父弟也,鄙朴少言。莊帝初,封樂鄉縣伯。榮死,與世隆赴晉陽。



 元曄之立,以度律為太尉公、四面大都
 督,封常山王。與爾朱兆入洛。兆遷晉陽,留度律鎮京師。節閔帝時,為使持節、侍中、大將軍、太尉公,兼尚書令、東北道行臺,與仲遠出拒義旗。齊神武間之,與爾朱兆遂相疑貳,自敗而還。度律雖在軍戎,聚斂無厭,所經為百姓患毒。其母山氏聞度律敗,遂恚憤發病。及至,母責之曰:「汝荷國恩,無狀而反,我何忍見他屠戮汝也!」言終而卒,時人怪異之。後韓陵之敗,斛斯椿先據河橋,遂西走水壘波津,為人執送。椿囚之,送齊神武,斬之都市。



 天光,榮從祖兄子也。少勇決,榮特親愛之,常預軍戎謀。孝昌末,榮據并、肆,仍以天光為都將,總統肆州兵馬。明
 帝崩,榮向京師,委以後事。建義初,為肆州刺史,封長安縣公。榮將討葛榮,留天光在州,鎮其根本。謂曰:「我身不得至處,非汝無以稱我心。」永安中,與元天穆東破邢杲。元顥入洛,天光與天穆會榮於河內。榮發後,并、肆不安,詔天光兼尚書僕射,為并、肆等九州行臺,仍行并州事。天光至并州,部分約勒,所在寧輯。顥破,還京師,改封廣宗郡公。



 初,高平鎮城人赫貴連恩等為逆,共推敕勤酋長胡琛為主,號高平王。遙臣沃野鎮賊帥破六韓忉夤。琛入據高平城,遣其大將萬俟醜奴來寇涇州。琛後與莫折念生交通,侮僈忉夤。遣使人費律如至高平,誘斬
 琛,為醜奴所并,與蕭寶夤相拒於安定。寶夤敗還。建義元年夏,醜奴擊寶夤於靈州,禽之,遂僭大號。時獲西北貢師子,因稱神獸元年,置百官。



 朝廷憂之,乃除天光使持節、都督、雍州刺史,率大都督武衛將軍賀拔岳、大都督侯莫陳悅等討醜奴。天光初行,唯有軍士千人。時東雍赤水蜀賊斷路,天光入關擊破之,簡取壯健。至雍,又稅人馬,合得萬疋。以軍人寡少,停留未進。榮遣責之,杖天光百下。榮復遣軍士二千人赴天光。天光令賀拔岳率千騎先驅,至岐州,禽其行臺尉遲菩薩。醜奴棄岐州走還安定。天光發雍至岐,與岳合勢,破醜奴,獲蕭寶夤。
 於是涇、豳、二夏,北至靈州,及賊黨結聚之類,並降。唯賊行臺萬俟道洛不下,率眾西依牽屯山,據險自守。榮責天光不獲道洛,復遣使杖之百,詔削爵為侯。天光與岳、悅等復向牽屯討之,道洛戰敗,投略陽賊帥王慶雲。慶雲以道洛驍果絕倫,得之甚喜,便謂大事可圖,乃自稱皇帝,以道洛為大將軍。天光乃入隴,至慶雲所居永洛城,破其東城。賊遂併趣西城。城中無水,眾聚熱渴。有人走降,言慶雲、道洛欲突出。天光恐失賊帥,乃遣謂慶雲,可以早降,若水決,當聽諸人今夜共議。又謂曰:「相知須水,今為小退。」賊眾安悅,無復走心。天光密使軍人多作
 木槍,各長七尺,至昏,布立人馬,為防衛之勢,又伏人槍中。其夜,慶雲、道洛果突出,至槍,馬各傷倒。伏兵便起,同時禽獲。賊窮,乞降而已。天光、岳、悅等議悉坑之,死者萬七千人,分其家口。於是三秦、河、渭、瓜、梁、鄯善咸來款順。詔復天光前官爵。



 岳聞榮死,還涇州以待,天光亦下隴,與岳圖入洛之策。既而莊帝進天光爵為廣宗王,元曄又以為隴西王。及聞爾朱兆已入京,天光乃輕騎向都,見世隆等,尋便還雍。世隆等議廢元曄,更舉親賢,遣告天光。天光與定策,立節閔帝。又加開府儀同三司、尚書令、關西大行臺。天光北出夏州,遣將討宿勤明達,禽之,
 送洛。



 時費也頭帥紇豆陵伊利、萬俟受洛于等據有河西,未有所附。天光以齊神武起兵信都,內懷憂恐,不暇他事。伊利等,但微遣備之而已。又除大司馬。



 時神武軍既振,爾朱兆、仲遠等並經敗退。世隆累使徵天光,天光不從。後令斛斯椿苦要天光云:「非王無以能定,豈可坐看宗家之滅?」天光不得已,東下,與仲遠等敗於韓陵。斛斯椿等先還,於河橋拒之,天光不得度,西北走,被執,與度律並送於神武。神武送於洛,斬於都市。



 爾硃專恣,分裂天下,各據一方,賞罰自出,而天光有定關西之功,差不酷暴,比之兆與仲遠,為不同矣。



 論曰:魏自宣武之後,政道頗虧。及明皇幼沖,女主南面。始則于忠專恣,繼以元叉權重,居官者肆其聚斂,乘勢者極其陵暴,於是四海囂然,已有群飛之漸。



 逮於靈后反政,宣淫於朝,傾覆之徵,於此至矣。爾朱榮緣將帥之列,藉部眾之威,屬天下暴虐,人神怨憤。遂有匡頹拯弊之志,援主逐惡之功。及夫禽葛榮,誅元顥,戮邢杲,揃韓婁,醜奴、寶夤,咸梟馬市,然則榮之功烈,亦已茂矣。而始則希覬非望,睥睨宸極,終乃靈后、少帝,沈流不反。河陰之下,衣冠塗地,其所以得罪人神者焉。至於末跡凶忍,地逼亦已除矣。而朝無謀難之宰,國乏折衝之將,遂使
 餘孽相糾,還成嚴敵。隆實指蹤,兆為戎首,山河失險,莊帝幽崩。宗屬分方,作威跋扈,廢帝立主,回天倒日;揃剝黎獻,割裂神州,刑賞任心,征伐自己。天下之命,縣於數胡,喪亂弘多,遂至於此。豈非天將去之,始以共定;終於惡稔,以至殄滅。抑亦魏紓其難,齊以驅除矣。



\end{pinyinscope}