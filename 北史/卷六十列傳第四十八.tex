\article{卷六十列傳第四十八}

\begin{pinyinscope}

 李弼曾孫密宇文貴子忻愷侯莫陳崇子穎崇兄順王雄子謙李弼,字景和,隴西成紀人。六世祖振,慕容垂黃門郎。父永,魏太中大夫,贈涼州刺史。弼少有大志,膂力過人。屬魏亂,謂所親曰:「大太夫生世,會須履鋒刃,平寇難,以取功名,安能碌碌依階以求仕。」初為別將,從爾朱天光西討,破赤水蜀,以功封石門縣伯。又與賀拔岳討萬俊醜
 奴、萬俊道洛、王慶雲,皆破之。



 賊咸畏之曰:「莫當李將軍前也。」



 及天光赴洛,弼隸侯莫陳悅,征討屢有剋捷。及悅害賀拔岳,周文帝自平涼討悅。弼諫悅,令解兵謝之。悅惶惑,計無所出。弼知悅必敗。周文帝至,悅乃棄秦州南出,據險以自固。是日,弼密通於周文,許背悅。至夜,弼乃勒所部,云悅欲向秦州,命皆裝束。弼妻,悅之姨也,時為悅所親委,眾咸信之,人皆散走。弼慰輯之,遂擁以歸周文。悅由此敗。周文謂曰:「公與吾同心,天下不足平也。」



 大統初,進位驃騎大將軍、開府儀同三司。從平竇泰,斬獲居多。周文以所乘騅馬及泰所著牟甲賜弼。又從平弘
 農。與齊神武戰於沙苑,弼軍為敵所乘。弼將其麾下九十騎橫截之,賊分為二,因大破之。以功進爵趙郡公。四年,從周文東討洛陽,弼為前驅。東魏將莫多婁貸文率眾至穀城,弼倍道而前進,遣軍士鼓噪,曳柴揚塵。貸文以為大軍至,遂走。弼追斬貸文,傳首大軍。翌日,又從周文與齊神武戰河橋,身被七創,遂為所獲,陽隕絕於地,睨其傍有馬,因躍上得免。歷位司空、太保、柱國大將軍。廢帝元年,賜姓徒何氏。六官建,拜太傅、大司徒。及晉公護執政,朝之大事,皆與于謹及弼等參議。周孝閔帝踐阼,除太師,進封趙國公,邑萬戶,前後賞賜巨萬。



 弼每征討,
 朝受命,夕便引路,略不問私事,亦未嘗宿於家。兼性沈雅,有深識,故能以功名終。薨於位,明帝即日舉哀,比葬,三臨其喪。發卒穿冢,給大路、龍旂,陳軍至墓。謚曰武。尋追封魏國公,配食文帝廟庭。



 子曜居長,以次子暉尚文帝女義安長公主,故遂以為嗣。



 初賜爵義城郡公,嘗臥疾期年,文帝憂之,賜錢一千萬,供其藥石之費。魏恭帝二年,加驃騎大將軍、開府儀同三司。出為岐州刺史。從文帝西巡,率公卿子弟別為一軍。後襲趙國公,改襲魏國公。天和六年,進位柱國。建德元年,出為梁州總管。時渠、蓬二州生獠積年侵暴,至州綏撫,並來歸附。璽書勞
 之。



 暉弟衍,字拔豆,少專武藝,慨慷有志略。仕周,為義州刺史,封真鄉公。王謙作亂,以行軍總管從梁睿擊平之,進上大將軍。隋開皇元年,以行軍總管討平叛蠻,進位柱國。後拜安州總管,以疾還京,卒。子仲威嗣。



 衍弟綸,最知名,有文武才用。以功臣子少居顯職,位至司會中大夫、開府儀同三司,封河陽郡公。為聘齊使主,卒。



 子長雅嗣,尚隋文帝女襄國公主,位內史侍郎、河州刺史、檢校秦州總管。



 綸弟晏,開府儀同三司、趙郡公,從平齊,歿并州。子憬,以晏死王事,即襲其官爵。



 曜既不得嗣,朝廷以弼功重,封曜邢國公,位開府。



 子寬,幹略過人,自周及隋,
 數經將領,位柱國、蒲山郡公,號為名將。



 弼弟摽,字雲傑,長不盈五尺,性果決,有膽氣。魏永安元年,以兼別將從爾朱榮破元顥。榮誅,隨爾朱兆入洛。及魏孝武西遷,摽從都督元斌之興齊神武戰,敗,遂與斌之奔梁。後得逃發,進封晉陽縣子。尋為周文帝帳內都督,從復弘農,破沙苑。



 摽時跨馬運矛,衝堅陷隱陣,隱身鞍甲之中。敵人見之,皆曰「避此小兒」。



 不知摽之形貌,正自如此。周文初亦聞摽驍悍,未見其能,至是方嗟歎之。謂曰:「但問膽決如何,何必要須八尺之軀也。」以功進爵為公。武成初,從豆盧寧征稽胡,進爵汝南郡公。出為總管延綏丹三州諸軍
 事、延州刺史,卒官。



 無子,以弼子椿嗣,位開府儀同大將軍、右宮伯,改封河東郡公。



 密字法主,蒲山公寬之子也。才兼文武,志氣雄遠,少襲爵蒲山公。養客禮賢,無所愛吝。與楊玄感為刎頸交。後更折節耽學,尤好兵書,謂皆在口。師事國子助教包愷,受史記、漢書。愷門徒皆出其下。大業初,授親衛大都督,以疾歸。



 及玄感有逆謀,召密,令與弟玄挺赴黎陽,以為謀主。密進三計曰:「今天子遠在遼外,公長驅入薊,直扼其喉,前有高麗,退無歸路,不戰而禽,此計上也。



 又關中四塞,衛文昇不足為意,今率眾務早入西,萬全之勢,此
 計中也。若隨近先向東都,以引歲月,此計之下也。」玄感曰:「公下計乃上策矣。今百官家口並在東都,若不取之,安能動物?且經城不拔,何以示威?」密計不行。玄感既至東都,自謂功在朝夕。及獲韋福嗣,既非同謀,設籌皆持兩端。玄感後使作檄文,固辭不肯。密揣知其情,請斬之。玄感不從。密退謂所親曰:「楚公好反而不欲勝,吾屬今為虜矣。」後玄感將西入,福嗣晚亡歸東都。時李雄勸玄感速稱尊號,玄感以問密,密以為不可。玄感笑而止。及宇文述、來護等軍且至,玄感謂密計將安出。密曰:「元弘嗣統強兵於隴右,今可揚言其反,遣使迎公,因此入關,
 可得紿眾。」



 玄感遂用密謀號令。西至陜縣,圍弘農不拔,西至閿鄉,追兵至,玄感敗。密間行入關,與玄感從叔詢相隨,匿馮翊詢妻家。尋為鄰人告,被捕,與其黨俱送帝所。



 在途,與其眾謀逃。其徒多金,密令出示使者曰:「吾等死日,此金留付公,幸用相瘞,其餘即皆報德。」使者利金,遂相許。及出關,密每夜宴飲。行次邯鄲,夜宿村中,密等七人皆穿墻而遁。與王仲伯亡抵平原賊帥郝孝德,孝德不甚禮之。備遭饑饉,削樹皮而食之。仲伯潛歸天水。密詣淮陽,舍於村中,變姓名稱劉智遠,聚徒教授。經數月,鬱鬱不得志,為五言詩,詩成,泣下數行。時人有怪之,
 以告太守趙他,下縣捕之。密亡抵其妹夫雍丘令丘君明。君明從子懷義後告之,密得遁去,君明竟坐死。



 密投東郡賊帥翟讓,乃因王伯當以策干讓。遣說諸小賊,所至輒降,讓始敬焉,召與計事。密以兵眾無糧,勸讓直趣滎陽,休兵館穀,然後爭利。讓從之,乃掠下滎陽。太守郇王慶及通守張須陀以兵討讓。讓數為須陀敗,將遠避之。密勸讓列陣以待,密以奇兵掩擊,大破之,斬須陀於陣。讓於是令密建牙,別統所部。復說讓以廓清天下為事,令掩據興洛倉,發粟以振窮乏。於是興讓以義寧元年春出陽城,北踰方山,自羅口襲興洛倉,破之,開倉振
 百姓。越王侗遣武賁郎將劉長恭討密。



 密一戰破之,長恭僅以身免。讓於是推密為主。密城洛口周回四十城以居之。讓上密號為魏公,設壇場即位,稱元年。以房彥藻為左長史,邴元真為右長史,楊德方為左司馬,鄭德韜為右司馬。拜讓為司徒,封東郡公。長白山賊孟讓掠東郡,燒豐都市而歸。密攻下鞏縣,獲縣長柴孝和,拜為護軍。武賁郎將裴仁基以武牢歸密,密因遣仁基與孟讓襲破回洛倉,據之。俄而德韜、德方俱死,筆以鄭頲為左司馬,鄭虔象為右司馬。



 柴孝和說密令裴仁基守回洛,翟讓據洛口,身率精銳,西襲長安,不然他人我先。密曰:「此誠上策,然我之所部並山東人,既見未下
 洛陽,恐不肯西入。」孝和請間行觀隙,乃與數十騎至陜縣,賊歸之者萬餘人。密時兵鋒甚銳,每入苑與官軍連戰。會密為流矢所中,臥於營內,東都出兵擊之,密眾大潰,棄回洛倉歸洛口。



 孝和之眾聞密敗,各分散而去,孝和輕騎歸密。煬帝遣王世充率江淮勁卒五萬討密,敗之。孝和溺洛水,密甚傷之。



 世充營於洛西,與密相拒百餘日。武陽郡丞元寶藏、黎陽賊帥李文相、洹水賊帥張昇、清河賊帥趙君德、平原賊帥郝孝德並歸密,共襲破黎陽倉,據之。周法明舉江、黃之地以附密。齊郡賊帥徐圓朗、任城大俠徐師仁、淮陽太守趙他等前後款附
 以千百數。



 翟讓所部王儒信勸讓為太宰,總眾務以奪密權。兄寬復謂讓曰:「天子止可自作,安得與人?汝若不作,我當為之。」密聞,惡之。會讓拒世充,軍退數百步,密與單雄信等赴之,世充敗走。讓欲乘勝破其營,會日暮,固止之。明日,讓與數百人至密所,欲為宴樂。其所將左右各就食,諸門並設備,讓不覺。密引讓入坐,令讓射。引滿將發,密遣壯士蔡建自後斬之。遂殺其兄寬及儒信等,從者亦有死焉。



 讓部將徐世勛為亂兵所斫,中重創,密止之,僅得免。雄信等皆叩頭求哀,密並釋而慰之。於是詣讓營,遣王伯當、邴元真、單雄信等告以殺讓意,令世
 勛、雄信、伯當分統其眾。



 世充夜襲倉城,密拒破之,斬武賁郎將費青奴。世充復營洛北,於洛水構浮橋,悉眾擊密。密拒之,不利而退。世充因薄其城下,密擊之,大潰,爭橋,橋陷,溺水者數萬人。武郎將楊威、王辨、霍世舉、劉長恭梁德重、董智通等皆沒於陣。世充僅而獲免,不敢還東都,遂走河陽。其夜大雪,餘眾死亡殆盡。密乃修金墉故城居之,眾三十餘萬,攻上春門。留守韋津出戰,被執。其黨勸密即尊號,密不許。



 及義師圍東都,密出軍爭之,交綏而退。



 俄而宇文化及弒逆,自江都北指黎陽,密拒之。會越王侗稱尊號,遣使授密太尉、尚書令、東南道
 大行臺、行軍元帥、魏國公,令先平化及,然後入朝輔政。化及至黎陽,徐世勛守倉城不下。密共化及隔水語,密數之曰:「卿本匈奴阜隸破野頭耳,父與兄弟皆受隋恩,豈容躬行殺虐?今若速來歸義,尚可全後嗣。」化及默然,俯仰良久,乃嗔目大言曰:「共你論相殺事,何須作書傳雅語!」密謂從者曰:「化及庸懦如此,忽欲圖帝王,吾當折杖驅之。」知其糧且盡,在偽與之和。化及大喜,恣其兵食,冀密饋之。會密下有人獲罪,亡投之,具言密情。化及大怒,又食盡,與密戰於童山下。自辰達酉,密中流矢,頓於汲縣。化及掠汲郡,北趣魏縣,以輜重留於東郡,遣其
 刑部尚書王軌守之。軌以郡降,密以軌為滑州總管。



 密引兵而西,遣記室參軍李儉朝於東都,執弒帝人于弘達以獻越王侗。侗以儉為司農少卿,使召密入朝。密至溫縣,聞世充已殺元文都、盧楚等,乃歸金墉城。



 世充既擅要權,乃厚賜將士。時密兵少衣,世充乏食,乃請交易。邴元真等各求私利,遽勸密,密許焉。初東都絕糧,人歸密者日有數百,至此得食,降人益少。密悔而止。密雖據倉,無府庫,兵數戰不賞,又厚撫初附兵,於是眾心漸怨。時邴元真守洛口倉,性貪鄙。宇文溫每謂密曰:「不殺元真,公難未已。」密不答。而元真知之,謀叛。楊慶聞而告密,密
 因疑焉。會世充悉眾來戰,密留王伯當守金墉,自就偃師,北阻芒山以待之。世充令數百騎度御河,密遣裴行儼等逆之。會日暮,行儼、孫長樂、程咬金等驍將十數人皆重創,密甚惡之。世充夜潛,詰朝而陣,密方覺之。狼狽出戰,敗績,馳向洛口。世充夜圍偃師,守將鄭頲為其部下,翻城而降世充。密將入洛口倉城,元真已遣人引世充。密陰知之,不發其事,欲待世充兵半度洛水,然後擊之。密候騎不時覺,比將出戰,世充軍悉已濟。密引騎而遁,元真以城降世充。



 密眾漸離,將如黎陽。人或曰:「殺翟讓之際,徐世勛幾死,其心安可保?」



 密乃止。時王伯當
 棄金墉城,保河陽,密自武牢濟,歸之。謂曰:「久苦諸君,我今日自刎以謝眾。」眾皆泣,莫能仰視。密復曰:「諸君幸不相棄,當共歸關中。



 密身雖愧無功,諸君必保富貴。」其府掾柳燮曰:「明公與長安宗族,有疇昔之遇,雖不陪起義,然阻東都,斷隋歸路,使唐國不戰而得京師,此公之功也。」眾咸曰:「然。」密遂歸朝,封邢國公,拜光祿卿。尋奉使出關安撫,至熊州而逃叛,見殺。



 宇文貴,字永貴,其先昌黎大棘人也,徙居夏州。父莫豆干,保定中,以貴勳追贈柱國大將軍、少傅、夏州刺史、安平郡公。貴母初孕貴,夢老人抱一子授之曰:「賜爾是子,
 俾壽且貴。」及生,形類所夢,故以永貴字之。貴少從師受學,嘗輟書歎曰:「男兒當提劍汗馬以取公侯,何能為博士也!」魏正光末,破六韓拔陵圍夏州,刺史源子邕嬰城固守,以貴為統軍。後從爾朱榮禽葛榮於滏口,加別將。又從元天穆平邢杲,轉都督。元顥入洛,貴率鄉兵從爾硃榮有功,封革融縣侯。除郢州刺史,入為武衛將軍、關內大都督。從魏孝武西遷,進爵化政郡公。貴善騎射,有將帥才。周文帝又以宗室,甚親委之。



 大統初,與獨孤信入洛陽。東魏潁州長史賀若統據潁川來降,東魏遣將堯雄、趙育、是云寶率眾二萬攻潁川。貴自洛陽率步騎
 二千救之,軍次陽翟。雄等去潁川四十里,東魏行臺任祥又率眾四萬,將與雄合。諸將咸以彼眾我寡,不可爭鋒。貴曰:「若賀若一陷,吾輩坐此何為?」遂入潁川。雄等稍進,貴率千人背城為陣,與雄合戰。貴馬中流矢,乃短兵步斗,雄大敗輕走,趙育於是降。任祥聞雄敗,遂不敢進。貴乘勝逼祥,敗之。是云寶亦降。師還。魏文帝在天游園,以金卮置侯上,令公卿射中者即賜之。貴一發而中。帝笑曰:「由基之妙,正當爾耳。」進侍中、驃騎大將軍、開府儀同三司。十六年,遷中外府左長史,進位大將軍。



 宕昌王梁彌定為宗人獠甘所逐,來奔。又有羌酋傍乞鐵忽,因
 梁簋定反後,據有渠株川,擁隸數千家,與渭州人鄭五醜同反。周文令貴與豆盧寧、史寧討之,貴等禽斬鐵忽及五醜,史寧又別擊獠甘破之。乃納彌定,并於渠株川置岷州。朝廷重功,遂於粟阪立碑,以紀其績。



 廢帝三年,詔貴代尉遲迥鎮蜀。時隆州人開府李光易反於鹽亭,攻圍隆州;而隆州人李拓亦聚眾反,開府張道應之。貴乃命開府叱奴與牧隆州,又令開府成亞擊拓及道降之,並送京師。除益州刺史,未就拜小司徒。先是蜀人多劫盜,貴乃召任俠傑健者署為游軍二十四部,令其督捕,由是頗息。



 周孝閔帝踐阼,進位柱國,拜御正中大。武成初,
 與賀蘭祥討吐谷渾。軍還,進封許國公、邑萬戶,舊爵回封一子。遷大司空,行小冢宰,歷大司徒,遷太保。



 貴好音樂,耽奕棋,留連不倦。然好施愛士,時人頗以此稱之。保定末,使突厥,迎皇后。天和二年,還至張掖,薨。贈太傅,謚曰穆。



 子善嗣。善紘厚有武藝。大象末位上柱國,封許國公。隋文帝受禪,遇之甚厚,拜其子穎上儀同。及善弟忻誅,並廢於家。善未幾卒。



 穎,大業中,位司農少卿,後沒李密。善弟忻。



 忻字仲樂,幼而敏慧,為童兒時,與群輩戲,輒為部伍,進止行列,無不用命者。年十二,能左右馳射,驍捷若飛。恒
 謂所親曰:「自古名將,唯以韓、白、衛、霍為美談,吾察其行事,未足多尚,使與僕並時,不令豎子獨擅高名。」年十八,從周齊王憲討突厥,以功拜儀同三司,賜爵興固縣公。韋孝寬以忻驍勇,請與鎮玉壁,以戰功加開府,進爵化政郡公。從武帝攻拔晉州。齊後主親總兵,六軍憚之,欲旋。忻諫曰:「以陛下之聖武,乘敵人之荒縱,何往而不克?若齊人更得令主,君臣協力,未易平也。」帝從之,乃戰,遂大剋。及帝攻陷并州,先勝後敗。帝為賊所窘,挺身而遁。諸將多勸帝還,忻勃然曰:「破城士卒輕敵,微有不利,何足為懷?今破竹形已成,奈何棄之而去!」帝納其言,明日
 復戰,拔晉陽。齊平,進位大將軍。尋與烏丸軌破陳將吳明徹於呂梁,進位柱國,除豫州總管。



 隋文帝龍潛時,與忻情好甚協,及為丞相,恩顧彌隆。尉遲迥作亂,以忻為行軍總管,隋韋孝寬擊之。時兵頓河陽,帝令高熲馳驛監軍,與穎密謀進取者,唯忻而已。迥遣子惇盛兵武陟,忻擊走之。進臨相州,迥遣精甲三千伏野馬岡,忻以五百騎襲之,斬獲略盡。進至草橋,迥又拒守,忻以奇兵破之,直趨鄴下。迥背城結陣,大戰,官軍不利。時鄴城士庶觀戰者數萬人,忻謂左右曰:「事急矣,吾當以權道破之。」於是時觀者走之,轉相騰籍,聲如雷霆。忻乃傳呼曰:「賊
 敗矣」。



 眾復振,齊力急擊之,迥軍大敗。及平鄴,以功遷上柱國。文帝謂「尉遲迥傾山東之眾,連百萬之師,公舉無遺算,策無全陣,誠天下英傑也。」進封英國公。



 自是每參帷幄,出入臥內,禪代之際,忻有力焉。後拜右領軍大將軍,寵顧彌重。忻解兵法,馭戎齊整,當時六軍有一善事,雖非忻建,在下輒相謂曰:「此必英公法也。」其見推服如此。後改封祀國公。上嘗欲令忻擊突厥,高潁曰:「忻有異志,不可委以大兵。」乃止。忻既佐命功臣,頻經將領,甚有威名,上由是微忌之,以譴去官。與梁士彥暱狎,數相往來。士彥時亦怨望,陰圖不軌。忻謂士彥曰:「帝王豈有
 常乎?相扶即是。公於蒲州起事,我必從征,兩陣相當,然後中連結,天下可圖也。」謀泄伏誅,家口籍沒。忻弟愷。



 愷字安樂,在周以功臣子,年三歲賜爵雙泉伯,七歲進封安平公。愷少有器局,諸兄並以弓馬自達,愷獨好學。博覽書記,解文,多伎藝,為名公子。累遷御正中大夫、儀同三司。隋文帝為丞相,加上開府,近師中大夫。及踐阼,誅宇文氏,愷亦將見殺,以與周本別,又兄忻有功,故見赦。後拜營宗廟副監、太子左庶子。廟成,別封甑山縣公。及遷都,上以愷有巧思,詔領營新都副監。高熲雖總大綱,凡所規畫,皆出於愷。及決渭水達河以通運漕,詔愷總督
 其事。後拜萊州刺史,甚有能名。坐兄忻誅,除名於家,久不得調。



 會朝廷以魯班故道,久絕不行,令愷修之。既而上建仁壽宮,右僕射楊素言愷有巧思,於是檢校將作大匠。歲餘,拜仁壽宮監,授儀同三司,尋為將作少監。文獻皇后崩,愷與楊素營山陵。上善之,復爵安平郡公。煬帝即位,遷都洛陽,以愷為營東都副監,尋遷將作大匠。愷揣帝心在宏侈,於是東都制度,窮極壯麗。帝大悅,進位開府,拜工部尚書。及長城之役,詔愷規度之。時帝北巡,欲誇戎狄,令愷為大帳,其下坐數千人。帝大悅,賜物千段。又造觀風行殿,上容衛者數百人,離合為之,下施
 輪軸,推移倏忽,有若神功。戎狄見之,莫不驚駭。帝彌悅,前後賞賜不可勝紀。



 是時將復古制明堂,議者皆不能決。愷博考群籍,為明堂圖樣奏之。又以「張衡渾象用三分為一度,裴秀輿地以一寸為千里,臣之此圖以一分為一尺,推而演之」。



 又引於時議者,或以綺井為重屋,或以圓楣為隆棟,將為臆說,事不經見。今錄其疑難,為之通釋,皆出證據,以相發明。為議曰:臣愷謹按《淮南子》曰:「昔者神農之御天下也,甘雨以時,五穀蕃植,春生夏長,秋收冬藏,月省時考,終歲獻貢,以時嘗穀,祀於明堂。明堂之制,有蓋而無四方,風雨不能襲,燥濕不能傷,遷延
 而入之。」臣愷以為上古朴略,創立典刑。



 《尚書帝命驗》曰:「帝者承天,立五府以尊天重象,赤曰文祖,黃曰神斗,白曰顯紀,黑曰玄矩,蒼曰靈府。」注云:「唐虞之天府,夏之世室,殷之重屋,周之明堂,皆同矣。」《尸子》曰:「有虞氏曰總章。」《周官考工記》曰:「夏后氏世室,堂脩二七,博四脩一。」注云:「脩,南北之深也。夏度以步,合堂脩十四步,其博益以四分脩之一,則堂博十七步半也。」臣愷案:三王之世,夏最為古,從質尚文,理應漸就寬大,何因夏室乃大殷堂?相形為論,理恐不爾。《記》云:「堂脩二七,博四脩一。」若夏度以步,則應脩七步。注云:「今堂脩十四步。」



 乃是增益《記》文。殷、
 周二堂,獨無加字,便是義類例不同。山東《禮》本輒加二七之字,何得殷無加尋之文,周闕增筵之義?研窮其趣,或是不然。讎校古書,並無「二」字。此乃桑間俗儒,信情加減。《黃圖》議云:「夏后氏益其堂之大百四十四尺,周人明堂以為兩杼間。」馬宮之言,止論堂之一面。據此為準,則三代堂基並方,得為上圖之制。諸書所說,並為下方,鄭注《周官》,獨為此義,非直與古違異,亦乃乖背《禮》文。尋文求理,深恐未愜。



 《尸子》曰:「殷人陽館。」《考工記》曰:「殷人重屋,堂脩七尋,堂崇三尺,四阿重屋。」注云:「其脩七尋,五丈六尺。放夏周,則其博九尋,七丈二尺。」



 又曰:「周人明堂,度和
 尺之筵,東西九筵,南北七筵,堂崇一筵,五室,凡室二筵。」《禮記明堂位》曰:「天子之廟,復廟重簷。」鄭注云:「復廟,重屋也。」



 注《玉藻》云:「天子廟及路寢,皆如明堂制。」《禮圖》云:「於內室之上,起通天之觀,觀八十一尺,得宮之數,其聲濁,君之象也。」《大戴禮》曰:「明堂者,古有之。凡九室,室有四戶八牖,以茅蓋,上圓下方。外水曰璧雍。赤綴戶,白綴牖。堂高三尺,東西九仞,南北七筵。其宮方三百步。」「凡人疾、六畜疫、五穀災。生於天道不順。天道不順,生於明堂不飾。故有天災則飾明堂。」《周書》曰:「明堂方百一十二尺,高四尺,階博六尺三寸,室居內,方百尺,室內方六十尺,高八尺,
 博四尺。」《作洛》曰:「明堂、太廟、路寢咸有四阿,重亢重廊。」



 孔氏注云:「重亢累棟,重廓累屋。」



 《禮圖》曰:「秦明堂,九室十二階,各有所居。」《呂氏春秋》「有十二堂。」



 與《月令》同。並不論尺丈。臣愷案:十二階雖不與《禮》合,一月一階,非無理思。



 《黃圖》曰:「堂方百四十四尺,坤之策也,方象地;屋圓,楣徑二百一十六尺,乾之策也,圓象天。室九宮,法九州;太室方六丈,法陰之變數;十二堂,法十二月;三十六戶,法極陰之變數;七十二牖,法五行所得日數;八達象八風,法八卦;通天臺徑九尺,法乾以九覆六;高八十一尺,法黃鍾九九之數;二十八柱,象二十八宿;堂高三尺,土階三
 等,法三統;堂四向五色,法四時五行;殿門去殿七十二步,法五行所行。門堂長四丈,取太室三之二。垣高無蔽目之照,牖六尺,其外之。殿垣方,在水內,法地陰也;水四周於外,象四海,圓法陽也;水闊二十四丈,應二十四氣;水內徑三丈,應《覲禮經》。武帝元封二年,立明堂汶上,無室,其外略依此制。《泰山通義》今亡,不可得而辨也。



 元始四年八月,起明堂、璧雍長安城南門,制度如儀。一殿,垣四面,門八觀,水外周堤,壤高。四方和會,築作三旬。五年正月六日辛未,始郊太祖高皇帝以配天。二十二日丁亥,宗祀孝文皇帝於明堂以配上帝。及先賢百辟卿
 士有益者,於是秩而祭之。親扶三老五更,袒而割牲,跪而進之。因班時令,宣恩澤。諸侯宗室、四夷君長、匈奴西國侍子,悉奉貢助祭。



 《禮圖》曰:「建武三十年作有堂,堂上圓下方。圓法天,方法地。十二堂法日辰,九室法九州,八窗象八風,八九七十二,法一時之王。室有二戶,二九十八戶,法土王十八日。內堂正壇高三尺,土階三等。」胡伯始注《漢官》云:「古清廟蓋以茅,今蓋以瓦,瓦下藉茅,以薦古制。」《東京賦》曰:「乃營三宮,布政頒常。復廟重屋,八達九房。造舟清池,惟水泱泱。」《薛綜》注云:「復重廟覆,謂屋平覆重棟也。」《續漢書祭祀志》曰:「明帝永平二年,祀五帝於明
 堂。五帝坐各處其方,黃帝在未,皆如南郊之位。光武位在青帝之南,少退,西面,各一犢,奏樂如南郊。」臣愷案《詩》云:「《我將》,祀文王於明堂也。我將我享,維羊維牛。」據此,則備大牢之祭。今云一犢,恐與古殊。自晉以前,未有鴟尾,其門墻璧水,一依本圖。



 晉《起居注》裴頠議曰:「尊祖配天,其義明著,廟宇之制,理據未分。直可為一殿以崇嚴祀,其餘雜碎,一皆除之。」臣愷案:「天垂象,聖人則之。」辟雍之星,既有圖狀,晉室方構,不合天文。既闕重樓,又無璧水,空堂乖五室之義,直殿違九階之文。非古欺天,一何過甚!



 後魏於北臺城南,造圓墻,在璧水外,門在水內迥立,
 不與墻相連。其堂上九室,三三相重,不依古制。室間通巷,違舛處多。其室皆用鑿累,極成褊陋。《後魏樂志》曰:「孝昌二年立明堂,議者或言九室,或言五室,詔斷從五室。後元叉執政,復改為九室。遭亂不成。」



 宋《起居注》曰:「孝武大明五年立明堂,其墻宇規範,擬同太廟,唯十二間,以應期數。依漢《汶上圖儀》,設五帝位,太祖文皇帝對饗。鼎俎簠簋,一依廟禮。」



 梁武即位之後,移宋時太極殿以為明堂,無室,十二間。《禮疑議》云:「祭用純,漆俎瓦樽,文於郊,質於廟,止一獻,用清酒。」平陳之後,臣得目觀,遂量步數,記其尺丈。猶見焚燒殘柱,毀破之餘,入地一丈,儼然如
 舊。柱下以樟木為跗,長丈餘,闊四尺許,兩兩相並,凡安數重。宮城處所,乃在郭內。雖湫隘卑陋,未合規摹,但祖宗之靈,得崇嚴祀。



 周齊二代,闕而不脩,大饗之典,於焉靡托。



 自古《明堂圖》唯有二本。一是宗周,劉熙、阮諶、劉昌宗等作,三圖略同。



 一是後漢建武三十年作,《禮圖》有本,不詳撰人。臣遠尋《經傳》,傍求子史,研究眾說,總撰今圖。其樣以木為之,下為方堂,堂有五室,上為圓觀,觀有四門。



 帝可其奏。會遼東之役,事不果行。



 以度遼之功,進位金紫光祿大夫。其年卒官,帝甚惜之,謚曰康。撰《東都圖記》二十卷、《明堂圖議》二卷、《釋疑》一卷,見行於世。



 長子儒
 童,游騎尉。少子溫,起部承務郎。



 侯莫陳崇,字尚樂,代武川人也。其先魏之別部,居庫斛真水。祖元,以良家子鎮武川,因家焉。父興,殿中將軍、羽林監,後以崇著勛,追贈柱國、太保、清河郡公。



 崇少驍勇,善馳射,謹愨少言。年十五,隨賀拔岳與爾朱榮征葛榮。後從岳入關,破赤水蜀。又從岳力戰,破萬俟醜奴。崇與輕騎逐北,至涇州長坑及之。賊未成列,崇單騎入賊中,於馬上生禽醜奴,遂大破之。封臨涇縣侯。及岳為侯莫陳悅所害,崇與諸將同謀迎周文帝。文帝至軍,原州刺史史歸猶為悅守。周文遣崇襲歸,直到城下,即據城
 門。時李遠兄弟在城內,先知崇來,中外鼓澡,伏兵悉起,遂禽歸斬之。以崇行原州事,仍從平悅,別封廣武縣伯。累遷儀同三司,改封彭城郡公。



 從禽竇泰,復弘農,破沙苑,戰河又另討平稽胡,累戰皆有功,進位柱國大將軍。



 六官建,拜大司空。周孝閔踐阼,進封梁國公,加太保。歷大宗伯、大司徒。



 保定三年,從武帝幸原州。時帝夜還京師,竊怪其故。崇謂所親人常升曰:「吾比日聞卜筮者言,晉公今年不利,車駕今忽夜還,不過是晉公死耳。」於是皆傳之。或有發其事者,帝集諸公卿於大德殿責崇,崇惶懼謝罪。其夜,護遣使將兵就崇宅,逼令自殺。葬禮如
 常儀,謚曰躁。護誅,改謚曰莊閔。



 子芮嗣,位柱國。從武帝東伐,率眾守太行道。并州平,授上柱國。仍從平鄴,拜大司馬。隋大業初,以譴,詔流配嶺南。芮弟穎。



 穎字遵道,少有器量,風神警發,為時輩所推。魏大統末,以父軍功,賜爵廣平侯,累遷開府儀同三司。周武帝時,從滕王逌擊龍泉、文城叛胡。穎與柱國豆盧勣分路而進,穎懸軍五百餘里,破其三柵。先是稽胡叛亂,輒略邊人為奴婢。至是,詔胡有厭匿良人者誅,籍沒其妻子。有人言為胡村所隱匿者,勛將誅之。穎曰:「將在外,君命有所不行。諸胡固非悉反,但相迫脅為亂。今慰撫,自可不
 戰而定;如即誅之,轉相驚恐,為難不細。未若召其渠帥,以隱匿才付之,令自歸首,則群胡可安。」勣從之,諸胡爭降附,北土以安。遷司武,加振威中大夫。



 隋文帝受禪,加上開府,進爵昇平郡公。平陳之役,以行軍總管從秦王俊出魯山道,與行軍總管段文振度江,安集歸附。再遷瀛州刺史,甚有惠政。後坐與秦王俊交通,免官。百姓送者莫不流涕,因相與立碑,頌穎清德。後拜邢州刺史。仁壽中,吏部尚書牛弘持節巡撫山東,以穎為第一,上優詔褒揚。時朝廷以嶺南刺史縣令多貪鄙,蠻夷怨叛,妙簡清吏。於是征穎入朝。上與言及平生,以為歡笑,即日
 進位大將軍,拜桂州總管、十七州諸軍事。及至官,大崇恩信,人夷悅服。



 煬帝即位,穎兄梁國公芮坐事徙邊,圾廷恐穎不自安,徵還京師。後拜恒山太。



 其年,嶺南、閩越多不附,帝以穎前在桂州有惠政,為南方所信伏,拜南海太守。



 卒官,謚曰定。子虔會最知名。



 崇兄順,少豪俠有志度。初事爾朱榮為統軍。普泰元年,封木縣子。後從魏孝武入關。順與周文帝同里乾,素相友善,且崇先在關中,周文見之甚歡,進爵彭城郡公。及梁簋定圍逼河州,以順為大都督,與趙貴討破之,即行河州事。



 大統四年,魏文帝東討,順與太尉王盟、僕射周
 惠達等留鎮長安。時趙青雀反,盟及惠達奉魏太子出次渭北。順於渭橋與賊戰,頻破之。魏文帝還,執順手曰:「渭橋之戰,卿有殊力。」便解所服金鏤玉梁帶賜之。南岐州氐苻安壽。遂率部落一千家款附。時順弟崇又封彭城郡公,遂改封順河間郡公。六年,加驃騎大將軍、開府儀同三司,行西夏州事,改封平原郡公。周孝閔帝踐阼,拜少師,進位柱國。



 其年薨。



 崇弟瓊,歷位荊州總管、上柱國,封脩武郡公。



 瓊弟凱,以軍功賜爵下蔡縣男。崇以平原州功,賜爵靈武縣侯,詔聽轉授凱。



 孝閔踐阼,進位開府儀同三司,進爵為公。天和中,為司會中大夫。建德
 二年,為聘齊使主。



 王雄,字胡布頭,太原人也。父崘,以雄著勳,追贈柱國大將軍、少傅、安康郡公。雄儀貌魁梧,少有謀略。魏末,從賀拔岳入關,除金紫光祿大夫。孝武西遷,封臨貞縣伯。大統中,進爵武威郡公,累遷大將軍,行同州事。恭帝元年,賜姓可頻氏。周孝閔帝踐阼,授少傅,進位柱國大將軍。武成初,進封庸國公,邑萬戶。



 出為涇州總管。



 保定四年,從晉公護東征,至芒山,與齊將斛律明月戰。退走,左右皆散,矢又盡,唯餘一奴一矢在焉。雄案稍不及明月者丈餘,曰:「惜爾,不得殺,但生將爾見天子。」明月反射雄
 中額,抱馬走至營,薨。贈使持節、太保、同華等二十州諸軍事、同州刺史,謚曰忠。子謙。



 謙字敕萬,性恭謹,無他才能,以父功封安樂縣伯。保定二年,父雄封庸國公,以武威郡公回封謙,安樂伯回封第三弟震。雄死,朝議以謙父殞行陣,特加殊寵,授柱國大將軍,襲爵庸國公。建德五年,武帝東征,謙力戰,進位上柱國。



 六年,授益州總管十八州諸軍事。及宣帝崩,隋文帝輔政,以梁睿為益州總管。



 時謙使司錄賀若昂奉表詣闕。昂還,具陳京師事。謙以父子受國恩,將圖匡復,遂舉兵,署置官司。總管長史乙弗虔、益州刺史達奚惎
 勸謙憑險觀變。隆州刺史高阿那肱為謙畫三策曰:「公親率精銳,直指散關,蜀人知公有勤王之節,必當各思效命,此上策也;出兵梁、漢,以顧望天下,此中策也;坐守劍南,發兵自衛,此下策也。」謙參用其中下之策。



 梁睿未至大劍,謙先遣兵鎮始州。隋文帝即以睿為行軍元帥,便發利、鳳、文、秦、成諸州兵討之。謙所署柱國達奚惎、高阿那肱、大將軍乙弗虔、楊安、任峻、侯翕、景孱等眾號十萬,盡銳攻利州,總管、楚國公豆盧勣拒戰將四旬。惎等諸軍聞睿將至,眾遂潰。謙所署大將軍苻子英攻巴州,又為刺史呂珍所破。睿乘其弊,縱兵深入。棋、虔密遣使
 詣睿,請為內應以贖罪。謙不知惎、虔之反己也,並令守成都。謙先無籌略,且所任用多非其才,及聞睿兵奄至,惶懼計無所出,乃自率眾逆戰,又以惎、虔之子為左右軍。行數十里,左右軍皆叛,謙奔新都,縣令王寶執而斬之,傅首京師。惎、虔以成都降。隋文帝以惎、虔首謀,令殺之於蜀市。餘眾並散。阿那肱尋亦被誅。



 論曰:李弼懷佐時之略,逢興運之期,締構艱難,綢繆顧遇,方面宣其庸績,帷幄盡其謀猷,非唯攀附成名,抑亦材謀自取。密遭風雲之會,奮其鱗翼,思封函谷,將割鴻溝,期月之間,眾數十萬。威行萬里,聲動四方。雖事屈興
 王,運乖天眷,而雄名克振,何其壯歟!然志性輕狡,終致顛覆,固其宜也。宇文貴負將帥之材,蘊剛銳之氣,遭逢喪亂,險阻備嘗,自致高位,亦云美矣。忻武藝之風,名高一代。及晚節遇禍,雖鳥盡弓藏,然亦器盈斯概,夷戮非為不幸。愷學藝兼該,思理通贍,規矩之妙,參從班、爾,當時缺席,咸取則焉。其起仁壽宮,營建洛邑,要求時幸,窮侈極麗,使文皇失德,煬帝亡身,危亂之原,抑亦由此。至於考覽書傳,定《明堂圖》,雖意過其通,有足觀者。侯莫陳崇以勇悍之氣,逢戰爭之秋,輕騎啟高平之扉,疋馬和長坑之俊。以宏材遠略,附鳳攀龍,茂績元勛,位居上袞,
 而識慚明哲,遂以凶終,惜哉!王雄身參佐命,謙寵列山河,及投袂勤王,志匡社稷,雖忠君之效未宣,與夫懷祿圖存者異也。



 初,魏孝莊帝以爾朱榮有翊戴之功,拜榮柱國大將軍,位在丞相上。榮敗後,此官遂廢。大統三年,魏文帝復以周文帝建中興之業,始命為之。其後功參佐命,望實俱重者亦居此職。自大統十六年已前,任者凡有八人。周文帝位總百揆,都督中外軍事。魏廣陵王欣,元氏懿戚,從容禁闥而已。此外六人,各督二大將軍,分掌禁旅,當爪牙禦侮之寄。當時榮盛,莫與為比。故今之稱門閥者,咸推八柱國家。



 今并十二大將軍錄之於左:
 使持節、太尉、柱國大將軍、大都督、尚書左僕射、隴右行臺、少師、隴西郡開國公虎。



 使持節、太傅、柱國大將軍、大宗師、大司徒、廣陵王元欣。



 使持節、柱國大將軍、大都督、大宗伯、趙郡開國公李弼。



 使持節、柱國大將軍、大都督、大司馬、河內郡開國公獨孤信。



 使持節、柱國大將軍、大都督、大司寇、南陽郡開國公趙貴。



 使持節、柱國大將軍、大都督、大司空、常山郡開國公于謹。



 使持節、柱國大將軍、大都督、少傅、彭城郡開國公侯莫陳崇。



 與周文帝為八柱國。



 使持節、大將軍、大都督、少保、廣平王元贊。



 使持節、大將軍、大都督、淮安王元育。



 使持節、大將軍、大都
 督、齊王元廓。



 使持節、大將軍、大都督、平原郡開國公侯莫陳順。



 使持節、大將軍、大都督、七州諸軍事、秦州刺史、章武郡開國公宇文導。



 使持節、大將軍、大都督、雍州諸軍事、雍州刺史、高陽郡開國公達奚武。



 使持節、大將軍、大都督、陽平郡開國公李遠。



 使持節、大將軍、大都督、范陽郡開國公豆盧寧。



 使持節、大將軍、大都督、化政郡開國公宇文貴。



 使持節、大將軍、大都督、荊州諸軍事、荊州刺史、博陵郡開國公賀蘭祥。



 使持節、大將軍、大都督、陳留君開國公楊忠。



 使持節、大將軍、大都督、岐州諸軍事、岐州刺史、武威群開國公王雄。



 是為十二大將軍。每大
 將軍督二開府,凡為二十四員,分團統領,是二十四軍。



 每一團,儀同二人。自相督率,不編戶貫。都十二大將軍。十五日上,則門欄陛戟,警晝巡夜;十五日下,則教旗習戰。無他賦役。每兵唯辦弓刀一具,月簡閱之。甲槊戈弩,並資官給。



 自大統十六年以前,十二大將軍外,念賢及王思政亦拜大將軍。然賢作牧隴右,思政出鎮河南,並不在領兵之限。此後功臣位至柱國及大將軍者眾矣,不限此秩,無所統御。六柱國、十二大將軍之後,有以位次嗣掌其事者,而德望素在諸公之下,並不得預於此例。



\end{pinyinscope}