\article{卷三十二列傳第二十}

\begin{pinyinscope}

 崔鑒兄孫伯謙
 崔辯孫士謙士謙子彭士謙弟說說子弘度崔挺子孝芬孫宣猷曾孫仲方仲方從叔昂挺從子季舒挺族孫暹崔鑒,字神具,博陵安平人也。六世祖贊,魏尚書僕射。五世祖洪,晉吏部尚書。曾祖懿,字世茂,仕燕,位祕書監。祖遭,字景遇,位鉅鹿令。父綽,少孤,學行修明,有名於世。與範陽盧玄、勃海高允、趙郡李靈等俱被徵,尋以母老固
 辭。



 後為郡功曹,卒。鑒頗有文學,自中書博士轉侍郎,賜爵桐廬縣子。出為東徐州刺史。鑒欲安新附,人有年老者,表求假以守令,詔從之。又於州內銅冶為農具,兵人獲利。卒,贈青州刺史、安平侯,謚曰康。子合,字貴和,少有時譽,襲爵桐廬子,位終常山太守。



 合弟秉,少有志氣,陽平王順之為定州,秉為衛軍府錄事,帶毋極令。時甄琛為長史,曾因公事,言競之間,以拳擊琛墜床。琛以本縣長,笑而不論。其豪率若此。彭城王勰行壽春,秉從行,招致壯俠,以為部下。勰目之,謂左右曰:「吾當寄膽氣於此人。」累遷廣平內史,大納財貨,為清論所鄙。後為燕州刺
 史,為杜洛周攻圍,堅守歷年。朝廷遣都督元譚赴救,譚敗,秉奔定州,坐免官。太昌中,除驃騎大將軍、儀同三司。頻以老病求解,永熙三年,去職。薨,贈尚書令、司徒公,謚曰靖穆。



 長子忻,字伯悅,有世乾。以鄭儼之甥,累遷兼尚書左丞。莊帝初,遇害河陰。



 追贈殿中尚書、冀州刺史。



 忻弟仲哲,早喪所生,為祖母宋氏所養。六歲,宋亡,啼慕不止,見者悲之。



 性恢達,常以將略自許。以軍功賜爵安平縣男。及父康於燕被圍,泣訴朝廷,遂除別將,與都督元譚赴援,戰歿。子長瑜,位至開府中兵參軍。



 長瑜子子樞,學涉好文詞,經辯有才幹。仕齊,位考功郎中,參議五禮,
 待詔文林館。兼散騎常侍,聘周。使還,除通直散騎常侍,兼知度支。子樞明解世務,所居稱職。因度支有受納風聞,為御史劾,遇赦免。仕周,位至上士。預尉遲迥事,被誅。



 子樞次弟子端,亦有才幹,而文藝為優。歷殿中侍御史,卒於通直散騎侍郎。



 子端弟子博,武平末,為河陽道行臺郎。隋開皇末,卒於泗州刺史。



 子博弟子發,有文才,武平末,秘書郎,修起居注。仕隋為秦王文學,卒於國子博士。



 長瑜弟叔瓚,頗有學識,性好直言。其妻即齊昭信皇后姊也,文宣擢為魏尹丞。



 屬蝗蟲為災,帝以問叔瓚。對曰:「案《漢書五行志》:『土功不時,蝗蟲作厲。』當今外築長城,
 內興三臺,故致此災。」帝大怒,令左右毆之,又擢其髮,以溷汁沃其頭,曳以出,由是廢頓久之。後卒於陽平太守,贈本州刺史。



 仲弟叔彥,位撫軍。



 叔彥弟季通,位司農少卿。季通子德立,好學,愛屬文,預撰《御覽》,位濟州別駕。



 季通弟季良,風望閑雅,位太學博士,以征討功,賜爵蒲陰縣子,累遷太尉長史。及康東還鄉,季良亦去職歸養。後位中軍將軍、光祿大夫,先康卒於家,贈尚書右僕射,謚曰簡。康弟習,字貴禮,有世用,卒於河東太守,贈并州刺史。



 鑒兄標,字洛祖,行博陵太守。標子文業,中書郎、鉅鹿太守。文業子伯謙。



 伯謙字士遜,貧居養母。齊神武召補相府兼功曹,稱之曰:「崔伯謙清直奉公,真良佐也。」轉七兵、殿中、左戶三曹郎中。弟仲讓為北豫州司馬,與高慎同叛。



 坐免官。後歷瀛州別駕、京畿司馬。文襄將之晉陽,勞之曰:「卿聘足瀛部,已著康歌。督府務總,是用相授。」臨別,又馬上執手曰:「執子之手,與子偕老,卿宜深體此情。」族弟暹當時寵要,伯謙與之舊寮同門,非吉凶未嘗造請,以雅道自居。



 天保初,除濟北太守,恩信大行,富者禁其奢侈,貧者勸課周給。縣公田多沃壤,伯謙咸易之以給人。又改鞭,用熟皮為之,不忍見血,示恥而已。朝貴行過郡境,問人太守
 政何似?對曰:「府君恩化,古者所無。」誦人為歌曰:「崔府君,能臨政。退田易鞭布威德,人無爭。」客曰:「既稱恩化,何因復威?」對曰:「長吏憚其威嚴,人庶蒙其恩惠,故兼言之。」以相府舊寮,例有加授,征赴鄴。



 百姓號泣遮道,數日不得前。



 以弟仲讓在關中,不復居內任,除南鉅鹿太守。下車導以禮讓,豪族皆改心整肅。事無巨細,必自親覽。在縣有貧弱未理者,皆曰「我自告白鬚公,不慮不決」。



 在郡七年,獄無停囚。每有大使巡察,恒處上第。徵拜銀青光祿大夫。



 伯謙少時讀經、史,晚年好《老》、《莊》,容止儼然無慍色,親賓至,則置酒相娛。清言不及俗事,士大夫以為儀表。
 卒,贈南充州刺史,謚曰懿。伯謙弟仲讓,仕西魏,位至鴻臚少卿。



 崔辯,字神通,鑒之從祖弟也。祖琨,字景龍,行本郡太守。父經,贈兗州刺史。辯學涉經史,風儀整峻,獻文徵拜中書博士、武邑太守。政事之餘,專以勸學。



 卒,贈安南將軍、定州刺史,謚曰恭。



 長子景俊,鯁正有高風,好古博涉,以經明行修,徵拜中書博士。歷侍御史、主文中散。孝文賜名為逸。後為員外散騎侍郎,與著作郎韓興宗參定朝儀。雅為孝文所知重,遷國子博士。每有公事,逸常被詔獨進,博士特命自逸始。轉通直散騎常侍、廷尉少卿,卒。



 子巨倫,字孝宗,幼孤。及長,歷涉經史,有文學武藝。叔楷為殷州,巨倫仍為長史、北道別將。在州陷賊,斂恤存亡,為賊所義。葛榮聞其才名,欲用為黃門郎,巨倫心惡之。至五月五日,會集官寮,令巨倫贈詩。巨倫乃曰:「五月五日時,天氣已大熱,狗便呀欲死,牛復喘吐舌。」以此自晦,獲免。結死士,夜中南走,逢賊,俱恐不濟。巨倫曰:「寧南死一寸,豈北生一尺!」便欺賊曰:「吾受敕而行。」賊爇火觀敕,火未然。巨倫手刃賊十餘人,賊乃四潰,得馬數匹。夜陰失道,唯看佛塔戶而行。到洛陽,持節別將北討。初,楷喪之始,巨倫收殯倉卒,事不周固;至是遂偷路改殯,并竊
 家口以歸。尋授國子博士。



 莊帝即位,除東濮陽太守。時河北紛梗,人避賊,多入郡界。歲儉飢乏,巨倫傾資贍恤,務相全濟。時類高之。元顥入洛,據郡不從,莊帝還宮,封漁陽縣男。



 後除光祿大夫。卒,子子武襲。



 初,巨倫有姊,明慧有才行。因患眇一目,內外親族,莫有求者。其家議欲下嫁之。巨倫姑,趙國李叔胤之妻,聞而悲感曰:「吾兄盛德,不幸早世,豈令此女,屈事卑族!」乃為子翼納之。時人歎其義識。



 逸弟模,字叔軌。身長八尺,圍亦如之。出後其叔,雅有志度。蕭寶夤討關、隴,引為西征別將,屢有戰功,封槐里縣伯。後行岐州事,擊賊,歿於陣。永熙中,贈驃騎
 大將軍、儀同三司、都督、相州刺史。模弟楷。



 楷字季則,為廣平王懷文學。正始中,以王國官非其人,多被戮,唯楷與楊昱以數諫諍獲免。後為太子中舍人、左中郎將。以黨附高肇,為中尉所劾。事在《高聰傳》。楷性嚴烈,能摧挫豪強,時人語曰:「莫鋋鬱買反彳解孤楷反,付崔楷。」時冀、定數州頻遭水害,楷上疏導之便宜,事遂施行。



 孝昌初,置殷州,以楷為刺史,加後將軍。楷將之州,人咸勸單身述職。楷曰:「單身赴任,朝廷謂吾有進退之計,將士又誰肯固志?」遂闔家赴州。賊勢已逼,或勸減小弱以避之,乃遣第四女、第三男夜出。既而曰:「一朝送免兒女,將謂吾心不固。」
 遂命追還。及賊來攻,楷率力拒抗,莫不爭奪,咸稱崔公尚不惜百口,吾等何愛一身?力竭城陷,楷執節不屈,賊遂害之。楷兄弟父子並死王事,朝野傷歎焉。贈侍中、鎮軍將軍、定州刺史。永熙中,又特贈驃騎大將軍、儀同三司、都督、冀州刺史。



 長子士元,沈雅有學尚。州陷,戰沒,贈平州刺史。



 子育王,少以器乾稱,仕齊至起部郎。子文豹,字蔚,少有文才,本州大中正。



 士元弟士謙。



 士謙,孝昌初解褐著作佐郎。後賀拔勝出鎮荊州,以士謙為行臺左丞。孝武西遷,士謙勸勝倍道兼行,謁帝關右,勝不能用。州人劉誕引侯景軍奄至,勝與戰,敗績,遂
 奔梁。士謙與俱行。及至梁,每乞師赴援。梁武雖不為出軍,而嘉勝等志節,並許其還國。乃令士謙先,且通鄰好。周文素聞其名,甚禮之,賜爵千乘縣男。



 及勝至,拜太師長史,以功進爵為子,拜尚書右丞。從周文解洛陽圍,經河橋戰,加定州大中正、瀛州刺史。又破柳仲禮於隨郡,討李遷哲於魏興,並有功,進驃騎大將軍、開府儀同三司、直州刺史。賜姓宇文氏。恭帝初,轉利州刺史。



 士謙性明悟,深曉政術,吏人畏而愛之。周保定二年,遷總管、安州刺史,加大將軍,進爵武康郡公。天和中,授江陵總管、荊州刺史。州既統攝遐長,俗兼夷夏,又南接陳境,東鄰
 齊寇。士謙外禦強敵,內撫軍人,風化大行,號稱良牧。每年考績,常為天下之最,屢有詔褒美焉。士謙隨賀拔勝之在荊州也,雖被親遇,而名位未顯;及踐其位,朝野以為榮。卒於州,闔境痛惜之,立祠堂,四時祭饗。子曠嗣。



 士謙性至孝,與弟說特相友愛,雖復年位並高,資產皆無私焉。居家嚴肅,曠及說子弘度並奉其遺訓云。



 曠少溫雅。大象末,位開府儀同大將軍、浙州刺史。曠弟彭。



 彭字子彭,少孤,事母以孝聞。性剛毅,有武略,工騎射,善《周官》、《尚書》,並略通大義。仕周,累遷門正上士。隋文帝為相,周陳王純鎮齊州。帝恐其為變,遣彭以兩騎征純入
 朝。彭未至齊州三十里,因詐病止傳舍,遣人召純。純疑有變,多將從騎至彭所。彭請間,因顧騎士執而鎖之。乃大言曰:「陳王有罪,詔徵入朝,左右不得輒動。」左右愕然而去。至,拜上儀同。及踐祚,遷監門郎將,兼領右衛長史,賜爵安陽縣男。再遷驃騎將軍,恆典宿衛。性謹密,在省闥二十餘年,當上,在仗危坐終日,未嘗有墮容。上每謂曰:「卿當上日,我寢處自安。」



 又嘗曰:「卿弓馬固以絕人,頗知學不?」彭曰:「臣少愛《周禮》、《尚書》,休沐之暇,不敢廢也。」上曰:「試為我言之。」彭因說君臣戒慎之義,上稱善。



 觀者以為知言。後加上開府,遷備身將軍。



 上嘗宴達頭可汗使
 者於武德殿,有鴿鳴於梁上。命彭射之,中,上大悅,賜錢一萬。及使者反,可汗復遣使請崔將軍一與相見。上曰:「此必善射聞於虜庭。」



 遂遣之。及至,可汗召善射者數十人,因擲肉於野,以集飛鳶,遣其善射者射之,多不中。彭連發數矢,皆應弦而落。突厥莫不歎服。仁壽末,進爵安陽縣公。煬帝即位,遷左領軍大將軍。時漢王諒初平,令彭鎮遏山東,復領慈州事。卒,贈大將軍,謚曰肅。子寶德嗣。



 士謙弟說。說本名士約。少有氣概,膂力過人,尤工騎射。賀拔勝牧荊州,以為假節、冠軍將軍、防城都督。又隨奔梁。復自梁
 歸西魏。授武衛將軍、都督,封安昌縣子。從周文復弘農,戰沙苑,皆有功,進爵為侯,除京兆郡守。累遷都官尚書、定州大中正,改封安固縣侯;賜姓宇文,并賜名說焉。進驃騎大將軍、開府儀同三司,加侍中,進爵萬年縣公。再遷總管、涼州刺史。說蒞政強毅,百姓畏之。



 後除使持節、能和中三州、崇德等十三防諸軍事,加授大將軍,改封安平縣公。建德四年,卒,贈廓、延等五州刺史,謚曰壯。子弘度。



 弘度字摩訶衍。膂力絕人,儀貌魁岸,鬚面甚偉,性嚴酷。年十七,周大冢宰宇文護引為親信,累轉大都督。時護
 子中山公訓為蒲州刺史,令弘度從焉。嘗與訓登樓,至上層,去地四五丈,俯臨之。訓曰:「可畏也!」弘度曰:「此何足畏?」



 欻擲下,至地無所損,訓大奇之。後以戰功授儀同。從平齊,進上開府、鄴縣公。



 尋從汝南公宇文神舉破盧昌期於范陽,鄖公韋孝寬經略淮南。以前後勳進位上大將軍。襲父爵安平縣公。



 及尉遲迥反,弘度以行軍總管從韋孝寬討之,所當無不披靡。弘度妹先適迥子為妻。及破鄴城,迥窘迫升樓,弘度直上龍尾追之。迥將射弘度,弘度脫兜鍪謂曰:「今日各圖國事,不得顧私。事既如此,早為身計,何所待也?」迥擲弓於地,罵大丞相極口,自
 殺。弘度顧弟弘昇,使取迥頭。進位上柱國。時行軍總管例封國公,以弘度不時殺迥,縱致惡言,由是降爵一等為武鄉郡公。



 開皇初,以行軍總管拒突厥於原州。還,拜華州刺史。納妹為秦孝王妃。尋遷襄州總管。



 弘度素貴,御下嚴急,所在令行禁止,盜賊屏跡。梁主蕭琮來朝被止,以弘度為江陵總管,鎮荊州。陳人憚之,不敢窺境。以行軍總管從秦孝王平陳,賜物五千段。高智慧等作亂,復以行軍總管隸楊素。弘度與素品同,而年長於素,素每屈下之,一旦隸素,意甚不平。素亦優容之。及還,以行軍總管檢校原州事,以備胡。



 無虜而還。上甚禮之,復以
 其弟弘昇女為河南王妃。仁壽中,檢校太府卿。



 自以一門二妃,無所降下。每誡其寮吏曰:「人當誠恕,無得欺誑。」皆曰:「諾。」後嘗食鱉,侍者八九人。弘度問之曰:「鱉美乎?」人懼之,皆曰:「美。」



 弘度大罵曰:「傭奴!」何敢誑我?汝初未食鱉,安知其美?」俱杖之八十。官屬百工見之,莫不汗流,無敢欺隱。時有屈突蓋為武候車騎,亦嚴刻。長安為之語曰:「寧飲三斗醋,不見崔弘度;寧炙三斗艾,不逢屈突蓋。」然弘度居家,子弟班白,動行捶楚,閨門整肅,為當世所稱。未幾秦王妃以罪誅,河南王妃復被廢,弘度憂恚,謝病於家。諸弟乃與之別居,彌不得志。煬帝即位,河南王為
 太子。帝將復立崔妃,遣中使就第宣旨。使者詣弘昇家,弘度不之知。使者反,帝曰:「弘度有何言?」使者曰:「弘度稱疾不起。」帝默然,其事竟寢。弘度憂憤,未幾卒。



 弘昇字上客,在周為右侍上士。從平尉遲迥,以功拜上儀同。尋加上開府,封黃臺縣侯。隋文受禪,進爵為公,授驃騎將軍。歷慈鄭二州刺史、襄州總管。以戚屬故,待遇隆重。及河南王妃罪廢,弘升亦免官。煬帝即位,歷冀州刺史、信都太守,位金紫光祿大夫,轉涿郡太守。遼東之役,檢校左武衛大將軍事,指平壤。與宇文述等同敗,奔還,發病卒。



 崔挺,字雙根,辯之從父弟也。父鬱,位濮陽太守。挺幼孤,
 居喪盡禮,少敦學。五代同居,後頻年饑,家始分析。挺與弟振推讓田宅舊資,惟守墓田而已。家徒壁立,兄弟怡然,手不釋卷。鄉人有贍遺,挺辭而後受,仍亦散之。舉秀才,射策高第。拜中書博士,轉侍郎。以工書,受敕於長安書文明太后父燕宣王碑,賜爵秦昌子。轉登聞令,遷典屬國下大夫。以參議律令,賜帛、穀、馬、牛等。尚書李沖甚重之。孝文以挺女為嬪。宋王劉昶南鎮彭城,詔挺為長史,以疾辭免,乃以王肅為長史,其被遇如此。後拜昭武將軍、光州刺史,風化大行。



 及車駕幸兗州,召挺赴行在所,問以臨邊之略,因及文章。帝甚悅,謂曰:「別卿以來,倏
 焉二載。吾所綴文,以成一集,今當給卿副本。」顧謂侍臣曰:「擁旄者皆如此,何憂哉!」復還州。及散騎常侍張彞巡行風俗,謂曰:「彞受使巡方,採察謠訟,入境觀政,實愧清使之名。」州舊掖城西北數里有斧山,峰嶺高峻。北臨滄海,南望岱岳。挺於頂上欲營觀宇,故老曰:「此嶺上,秋夏之際,常有暴雨。相傳云是龍道,恐此觀不可久立。」挺曰:「人龍相去,何遠之有?虯龍倏忽,豈一路乎?」遂營之。數年間,果無風雨之異。挺既代,即為風雨所毀,遂莫能立。眾以為善化所感。時以犯罪配邊者多有逃越,遂立重制,一人犯罪逋亡,闔門充役。挺上書,以為《周書》父子罪不
 相及,以一人犯罪,延及闔門,豈不哀哉!辭甚雅切,帝納之。



 先是,州內少鐵,器用皆求之他境。挺表復鐵官,公私有賴。孝文將辨天下氏族,仍亦訪定,乃遙授挺本州大中正。掖縣有人年踰九十,板輿造州。自稱少曾充使林邑,得一美玉,方尺四寸,甚有光采,藏之海島,垂六十歲,忻逢明政,今願奉之。挺曰:「吾雖德謝古人,未能以玉為寶。」遣船隨取,光潤果然,迄不肯受,乃表送都。景明初,見代,老幼泣涕追隨,縑帛送贈,悉不納。



 散騎常侍趙修得幸宣武,挺雖同州壤,未嘗詣門。北海王詳為司徒、錄尚書事,以挺為司馬,固辭不免。世人皆嘆其屈,而挺處之
 夷然。詳攝選,眾人競稱考第,以求遷敘,挺終無言。詳曰:「崔光州考級並未加授,宜投一牒,當為申請。蘧伯玉恥獨為君子,亦何故默然?」挺曰:「階級是聖朝大例,考課亦國之恆典,至於自炫求進,竊以羞之。」詳大相稱歎。其為司馬,詳未曾呼名,常稱州號,以示優禮。卒,贈輔國將軍、幽州刺史,謚曰景。光州故吏聞凶問,莫不悲感,共鑄八尺銅象,於城東廣固寺赴八關齋,追奉冥福。



 初,崔光貧賤,挺贍遺衣食,常親敬焉。又識邢巒、宋弁於童幼,世稱其知人。



 歷官三十餘年,家資不益,食不重味,室無綺羅,閨門之內,雍雍如也。欲諸子恭敬廉讓,因以孝為字。及
 葬,親故多有贈賵,諸子推挺素志,一無所受。有子六人,長子孝芬。



 孝芬字恭梓。早有才識,博學好文章。孝文召見,甚嗟賞之。李彪謂挺曰:「比見賢子謁帝,旨喻殊優,今當為絕群耳。」挺曰:「卿自欲善處人父子之間,然斯言吾不敢聞也。」後襲父爵,累遷司空屬、定州大中正。長於剖判,甚有能名,府主任城王澄雅重之。澄奏地制八條,孝芬所參定也。遷廷尉少卿。孝昌初,梁將裴邃等寇淮南,詔行臺酈道元、都督河間王榮討之。敕孝芬持節催令赴接,賊退而還。遷荊州刺史,兼尚書、南道行臺,領軍司,率諸將以
 援神俊,因代焉。孝芬遂從恆農道南入,敵便奔散,人還安堵。明帝嘉勞之。



 後以元叉之黨,與盧同、李獎等並除名,徵還。又除孝芬為廷尉。章武王融以贓貨被劾,孝芬案以重法。及融為都督,北討鮮于修禮,時孝芬弟孝演率宗從在博陵,為賊攻陷,遇害。融密啟云孝演入賊為逆,遂見收捕。全家投梁,遇赦乃還。



 後梁將成景俊逼彭城,孝芬兼尚書右丞,為徐州行臺。孝芬將發,入辭。靈太后謂曰:「卿女今事我兒,與卿是親。曾何相負,而內頭元叉車內,稱此嫗須了卻!」



 孝芬曰:「臣蒙國厚恩,義無斯語;假有斯語,誰能得聞?若有此聞,即此人於元叉親密,過
 臣遠矣。乞對之,足辨虛實。」太后乃有愧色。孝芬既至,景俊等力屈退走。以孝芬兼尚書,為徐、兗二州行臺。



 建義初,太山太守羊侃據郡反,引南賊圍兗州行臺。除孝芬散騎常侍、鎮東將軍、金紫光祿大夫,仍兼尚書、東道行臺,與大都督刁宣往救援。與行臺于侃時相接。至便圍之,侃突圍奔梁。永安中,授西兗州刺史;孝芬倦外役,固辭不行,仍為太常卿。太昌初,兼殿中尚書,後加儀同三司,兼吏部尚書。孝武帝入關,齊神武至洛,與尚書辛雄、劉廞等並被誅。沒其家口,天平中,乃免之。



 孝芬博聞口辯,善談論,愛好後進,終日忻然。商榷古今,間以嘲謔,聽
 者忘疲。文筆數十篇。有子八人。



 長子勉,字宣祖,頗涉史傳。普泰中,兼尚書右丞。勉善附會,世論以浮競譏之。為尚書令爾朱世隆所親待,而尚書郎魏季景尤為世隆所知,勉與季景內頗不睦。



 季景於世隆求右丞,奪勉所兼;世隆啟用季景,勉遂帳怏自失。太昌初,除散騎常侍、征東將軍、金紫光祿大夫、定州大中正,敕左右廂出入。其家被收之際,逃免。



 後見齊神武,勞撫之。天平初,遣勉送勳貴妻子赴定州,因得還。屬母李氏喪亡,勉哀號過性,遇病卒。無子,弟宣度以子龍子為後。勉弟猷。



 猷字宣猷。少好學,風度閑雅。性鯁正,有軍國籌略。普泰
 初,累遷司徒從事中郎。既遭家難,遂間行入關。及謁魏孝武,哀動左右。帝為之改容,目送曰:「忠孝之道,萃此一門。」即以本官奏門下事。大統初兼給事黃門郎、平原縣伯。



 二年,正黃門。行軍禽竇泰,復弘農,破沙苑,猷常以本官從軍典文翰。五年,除司徒左長史,加驃騎將軍。時太廟初成,四時祭祀猶設俳優角抵之戲;其郊廟祭官,多有假兼。猷上疏諫;書奏,並納焉。遷京兆尹。時婚姻禮嫁聚會之辰,多舉音樂。



 又廛里富室,衣服奢淫,乃有織成文繡者。猷請禁斷,事並施行。與盧辯等創修六官。十二年,除浙州刺史。十四年,侯景據河南歸款,遣行臺王思
 政赴之。周文與思政書曰:「崔宣猷智略明贍,有應變之才。若有所疑,宜與量其可不。」思政初頓兵襄城,後於潁川為行臺,并致書於猷。猷書曰:「襄城控帶京洛,實當今之要地,如有動靜,易相應接。潁川既鄰寇境,又無山川之固,賊若潛來,徑至城下。



 莫若頓兵襄城,為行臺所。潁川置州,遣郭賢守。則表裏膠固,人心易安,縱有不虞,豈能為患。」使人見周文,其以啟聞。周文令依猷策。思政重啟,求與朝廷立約,賊若水攻,乞一周為斷;陸攻,請三歲為期。限內有事,不煩赴援。過此以往,惟朝廷所裁。乃許之。及潁川沒,周文深追悔焉。以疾去職,屬大軍東征,周
 文賜以馬,隨軍與之籌略。十七年,進侍中、驃騎大將軍、開府儀同三司、本州大中正,賜姓宇文氏。



 恭帝元年,周文欲開梁、漢舊路,乃命猷督儀同劉道通等五人開通車路,鑿山堙谷五百餘里,至于梁州。即以猷為都督、梁州刺史。及周文崩,始、利、沙、興等諸州阻兵為逆,信、合、開、楚四州亦叛,唯梁州境內,人無二心。利州刺史崔士謙請援,猷遣兵六千赴之。信州糧盡,猷為送米四千斛。於是二鎮獲全。猷第二女,帝養為己女,封富平公主。



 周明帝即位,徵拜御正中大夫。時依《周禮》稱天王,又不建年號。猷以為世有澆淳,故帝王因以沿革。今天子稱王,不
 足以威天下。請遵秦漢,稱皇帝,建年號。朝議從之。除司會中大夫,御正如故。明帝崩,遺詔立武帝。晉公護謂猷曰:「今奉遵遺旨。君以為何如?」對曰:「殷道尊尊,周道親親,今朝廷既遵《周禮》,無容輙違此義。」雖不行,時稱其守正。



 及陳將蔡皎來附,晉公護議欲南伐,公卿莫敢言。猷獨進曰:「前歲東征,死傷過半,比雖加撫循,而創痍未復。近者長星為災,乃上玄所以垂鑒誡也,豈可窮兵極武,而重其譴負哉?」議不從。後水軍果敗,而裨將元定等遂沒江南。建德六年,拜少司徒,加上開府儀同大將軍。隋文帝受禪,以猷前代舊齒,授大將軍,進爵汲郡公。



 開皇四
 年,卒,謚曰明。子仲方嗣。



 仲方字不齊。少好讀書,有文武才略。年十五,周文帝見而異之,令與諸子同就學。隋文帝亦在其中。由是與帝少相款密。後以明經為晉公宇文護參軍,轉記室,遷司正大夫,與斛斯征、柳敏等同修禮律。後以軍功授平東將軍、銀青光祿大夫,賜爵石城縣男。時武帝陰有滅齊志,仲方獻二十策,帝大奇之。復與少內史赴芬刪定格式。尋從帝攻下晉州,又令仲方說下翼城等四城,授儀同,進爵范陽縣侯。後以行軍長史從郯國公王軌禽陳將吳明徹於呂梁,仲方策居多。宣帝嗣位,為少內史。



 會
 帝崩,隋文帝為丞相,與仲方相見,握手極歡,仲方亦歸心焉。其夜上便宜十八事,帝並嘉納之。又勸帝應天受命,從之。及受禪,上召仲方與高熲議正朔服色事。仲方曰:「晉為金行,後魏為水,周為木,皇家以火承木德之統。又聖躬載誕之初,有赤光之瑞。車服旗牲,並宜用赤。」又勸上除六官,依漢魏之舊。並從之。進位上開府,授司農少卿,進爵固安縣公。令發丁三萬於朔方、靈武築長城,東至黃河,西拒綏州,南至勃出嶺,綿歷七百里。明年,復令仲方發丁十萬,於朔方已東,緣邊險要,築數十城,以遏胡寇。丁父艱,去職。未期,起為虢州刺史。



 上書論取陳
 之策曰:臣謹案:晉太康元年,歲在庚子,晉武帝平吳。至今開皇六年,歲次丙午,合三百七載。《春秋寶乾圖》云:「王者三百年一蠲法。」今三百之期,可謂至矣。



 陳氏草竊,起於丙子,至今丙午,又子午為衝,陰陽之忌。昔史趙有言曰:「陳,顓頊之族,為水,故歲在鶉火以滅。」又云:「周武王克商,封胡公滿於陳。」至魯昭九年,陳災,裨灶曰:「歲五及鶉火而後陳亡,楚克之。」楚,祝融後也,為火正,故復滅陳。陳承舜後,舜承顓頊。太歲左行,歲星右轉;鶉火之歲,陳族再亡,戊午之年,媯虞運盡。語跡雖殊,考事無別。皇朝五運相承感火德。而國號為隋,隋與楚同分,楚是火正。午
 為鶉火,未為鶉首,申為實沈,酉為大梁。既當周、秦、晉、趙之分,若當此分發兵,將得歲之助。以今量古,陳滅不疑。臣謂午、未、申、酉並其數極。蓋聞天時不如地利,地利不如人和。況主聖臣良,兵強國富,陳既主昏於上,人讟於下,險無百二之固,眾非九國之師,獨此島夷,而稽天討!伏度朝廷,自有宏謨,皞蕘所見,冀申螢爝。今唯須武昌以下,蘄、和、徐、方、吳、海等州,更帖精兵,密營渡計;益、信、襄、荊、基、郢等州,速造舟楫,多張形勢,為水戰之具。蜀、漢二江,是其上流,水路衝要,必爭之所。賊雖於流頭、荊門、延洲、公安、巴陵、隱磯、夏口、盆城置船,然終聚漢口、峽口,以
 水戰火決。



 若賊必以上流有軍,令精兵赴援者,下流諸將,即須擇便橫度。如擁眾自衛,上江水軍,鼓行以前。雖恃九江五湖之險,非德無以為固;徒有三吳百越之兵,無恩不能自立。



 上覽,大悅。轉基州刺史,徵入朝。仲方因陳經略,上善之,賜以御袍褲并羅雜綵五百段,進位開府。及大舉伐陳,以仲方為行軍總管,與秦王會。及陳平,坐事免。未幾,復位。



 後數載,授會州總管。時諸羌猶未賓附,詔仲方擊之,與賊三十餘戰,紫祖、四鄰、望方、涉題、乾碉、小鐵圍山、白男、弱水等赭都諸賊悉平。賜奴婢一百二十口、黃金三十斤。遷代州總管。總被徵入朝。會文帝
 崩,漢王餘黨據呂州不下。



 煬帝遣周羅攻之,中流矢卒。及令仲方代總其眾,拔之,進位大將軍。歷戶部、禮部尚書,坐事免。尋為國子祭酒,轉太常卿。朝廷以其衰老,出拜上郡太守。以母憂去職,歲餘起為信都太守。後乞骸骨,優詔許之,卒於家。子燾,位定陶令。



 宣猷弟宣度,位齊王開府司馬、恆農太守。宣度弟宣軌,頗有才學,位尚書考功郎中,與弟宣質、宣靜、宣略並早卒。



 孝芬弟孝偉,趙郡太守。郡經葛榮離亂後,人皆賣鬻兒女。夏椹大熟,孝偉勸戶人多收之,郡內乃安。教其人種殖,招撫遺散,先恩後威,一周之後,流戶大至。



 興立學校,親加勸厲,百
 姓賴之。卒郡,贈瀛州刺史,謚曰簡。朝議謂為未申,復贈安北將軍、定州刺史。一子昂。



 昂字懷遠,七歲而孤,事母以孝聞。伯父吏部尚書孝芬嘗謂親友曰:「此兒終當遠至,是吾家千里駒也。」



 昂性端直,頗綜文詞。天平二年,文襄引為記室參軍,季以腹心之任。及輔國政,召為開府長史,并攝京畿長史事。時勳將親族賓客,多行不軌,孫騰、司馬子如之門尤劇。昂受文襄密旨,以法繩之,未幾間,內外齊肅。尋遷司徒右長史。時左府有陽平人吳賓為妄認繼嗣事,披訴經久。長史王昕、郎中鄭憑、掾盧斐、屬王敬寶等窮其獄,始末積
 年,鞫掠不獲實。司徒婁昭付昂推問,即日詰根緒,獲其真狀。昭歎曰:「左府都官數人,不如右府一長史。」昕、憑甚以為愧。



 武定中,文襄普令內外極言得失。昂上書曰:「屯田之設,其來尚矣。曹魏破蜀,業以興師。馬晉平吳,兵因取給。朝廷頃以懷、洛兩邑,鄰接邊境,薄屯豐稔,糧儲已贍。準此而論,龜鏡非遠。其幽、安二州,控帶奚賊、蠕蠕;徐、揚、兗、豫,連接吳越強鄰。實藉轉輸之資,常勞私糴之費。諸道別遣使營之,每考其勤惰,則人加勸勵,倉廩充實,供軍濟國,實謂在茲。其次,法獄之重,人命所懸。頃者官司糾察,多不審練,乃聞緣淺入深,未有雪大為小,咸以
 畏避嫌疑,共相殘刻。



 至如錢絹粟麥,其狀難分,徑指為贓,罪從此定。乞勒群司,務存獲實。如此則有息將來,必無枉濫。」文襄納之。後除尚書左丞,其年兼度支尚書。左丞之兼尚書,近代未有,朝野榮之。度支水漕陸運,昂設轉輸相入之差,付給新陳之法,有利於人,遂為常式。右僕射崔暹奏請海沂煮鹽,有利軍國。文襄以問昂。昂曰:「亦既官煮,須斷人灶,官力雖多,不及人廣。請準關市,薄為灶稅,私館官給,彼此有宜。」朝廷從之。



 武定六年,甘露降宮闕,文武同賀。魏帝問右僕射崔暹、尚書楊愔、崔甗、邢邵、散騎常侍魏收、御史中丞陸操、國子祭酒李澤曰:「
 可各言德績感致所由。」



 次至昂,昂曰:「吉凶兩門,不由符瑞,故桑雉之戒,實啟中興;小鳥孕大,未聞福感。所願陛下,雖休勿休,允答天意。」帝為斂容。後攝都官尚書,上勸田事七條。尋兼太府卿。齊受禪,改散騎常侍,兼大司農卿。二寺所掌,世號繁劇,昂校理有術,下無姦偽。又奏上橫市妄費事三十四條。其年,與太子少師邢邵議定國初禮式,仍封華陽縣男。又詔刪定律令,損益禮樂,令尚書右僕射琡等四十三人在領軍府議定。帝尋幸晉陽。將發,敕遞相遵率;不者,命昂以聞。昂部分科條,校正今古,手所增損,十有七八。轉廷尉卿。



 昂號深文,世論不以
 平恕相許。又與尚書盧斐,別典京畿詔獄,並有殘刻之聲。



 至於推繩大事,理可明言是非,不至冤酷。有濮陽子沈子遐,齎侯景鐵券,告徐州都督府長史畢義緒期舉兵應景;又衛尉卿杜弼門生郝子寬,告弼誹謗,并與元子雄謀逆。帝盛怒,付昂窮鞫。昂皆執正雪免,告者引妄獲罪。天保三年,除度支尚書。



 時有肴藏小吏,因內臣投書告事,又別有飛書告事者,並付昂窮檢。昂言笑間,咸得情,告者辭窮,並引嫌狀。於是飛書遂絕。轉都官尚書,仍兼都官事,食濟州北郡幹。



 文宣幸東山,謂曰:「舊人多出為州,當用卿為令僕,勿望刺史。卿六十外,當與卿本
 州。中間,州不可得也。」後九卿以上陪集東宮,帝指昂及尉瑾、司馬子瑞謂皇太子曰:「此是國家名臣,汝宜記之。」未幾,復侍宴金鳳臺,歷數諸人,咸有罪負,至昂,曰:「崔昂直臣,魏收才士,婦兄妹夫,俱省罪過。」十年,除兼右僕射,數日,即拜為真,未幾,還為兼。楊愔少時與昂不平,文宣崩後,遂免昂右僕射,除儀同三司、光祿勳。皇建元年,轉太常卿。河清元年,兼御史中丞,太常如故。



 昂從甥李公統坐高歸彥事誅。依律,婦人年六十以上免配宮。時公統母年始五十餘而稱六十,公統舅宣寶求吏以免其姊。昂弗知,錄尚書、彭城王浟發其事,竟坐除名。三年,復
 為五兵尚書,遷祠部。天統元年,卒,贈趙州刺史。



 昂有風調才識,奮立堅正剛直之名。然好揣上情,感激時主,或陳便宜蠲省,或列陰私罪失。深為文宣所知賞,朝之大事,多以委之。情尚嚴猛,每行鞭撻,雖苦楚萬端,對之自若。前則崔暹、季舒為之親援,後乃高德正是其中表,常有挾恃,意色矜高。以此不為名流歸服。有五子。第三子液,字君洽,頗習文藻,有學涉,風儀器局為時論所許。以奉朝請待詔文林館。隋開皇中,為中書侍郎。



 孝偉弟孝演,字則伯,出繼伯父。性通率,美鬚髯,姿貌魁傑,少無宦情,沈浮鄉里。位瀛州安西府外兵參軍,因罷歸。及鮮于
 修禮起逆,遇害。無子,弟孝直以子士游為後。



 孝直字叔廣,身長八尺,眉目疏朗,早有志尚。稍遷直閣將軍、通直散騎常侍。



 爾朱兆入洛,孝直以天下未寧,去職歸鄉里。太昌中,除衛將軍、右光祿大夫,辭不赴。卒於家,誡諸子曰:「吾才疏效薄,於國無功。若朝廷復加贈謚,宜循吾意,不得祗受。若致干求,則非吾意。」子士順,位太府卿。



 孝直弟孝政,字季讓。十歲挺亡,號哭不絕,見者為之悲慘。志尚貞立,博學經史,雅好辭賦。喪紀特所留情,衣服制度,手能執造。位太尉汝南王悅行參軍。



 孝芬兄弟孝義慈厚,弟孝演、孝政先亡,孝芬等哭泣哀慟,絕肉蔬食,容貌
 毀瘠,見者傷之。孝偉等奉孝芬盡恭順之禮,坐食進退,孝芬不命則不敢也。雞鳴而起,且溫顏色,一錢尺帛,不入私房,吉凶有須,聚對分給。諸婦亦相親愛,有無共之。始挺兄弟同居,孝芬叔振既亡後,孝芬等承奉叔母李氏,若事所生。旦夕溫清,出入啟覲,家事巨細,一以諮決。每兄弟出行,有獲財物,尺寸以上,皆入李之庫;四時分齎,李氏自裁之,如此二十餘歲。撫從弟宣伯、子朗,如同氣焉。挺弟振。



 振字延根。少有學行,居家孝,為宗族所稱。為祕書中散,在內謹敕,為孝文所知。孝文南討,自高陽內史徵兼尚
 書左丞,留京。振既才幹被擢,當世以為榮。



 遷太子庶子。景明初,除長兼廷尉少卿。振有公斷,以明察稱。河內太守陸琇與咸陽王禧同謀為逆,禧敗事發,振窮案之。時琇內外親黨及當朝貴要咸為言之,振研核切至,終無縱緩,遂斃之於獄。其奉法如此。除肆州刺史,在任有政績。卒於河東太守,贈南兗州刺史,謚曰定。



 振歷官四十餘載,考課恆為稱職,議者善之。子子朗,美容貌,涉獵經史,少溫厚,有風尚。位侍御史,加平東將軍,卒。挺從父子瑜,字仲璉,少孤,有學業,位鴻臚少卿,封高邑男,贈瀛州刺史。子孟舒,字長才,襲父爵,位廣平太守。卒,贈殷州刺
 史、鎮東將軍,謚曰康。孟舒弟仲舒,位鄴縣令。仲舒弟季舒,最知名。



 季舒字叔正。少孤,性明敏,涉獵經史,長於尺牘,有當世才具。年十七,為州主簿。為大將軍、趙郡公琛所器重,言之齊神武。神武親簡丞郎,補季舒大行臺都官郎中。文襄輔政,轉大將軍中兵參軍,甚見親寵。以魏帝左右,須置腹心,擢拜中書侍郎。文襄為中書監,移門下機事,總歸中書。又季舒善音樂,故內伎亦回隸焉。內伎屬中書,自季舒始也。文襄每進書魏帝,有所諫請,或文詞繁雜,季舒輒修飾通之,得申勸戒而已。靜帝報答霸朝,恆與
 季舒論之,云崔中書是我妳母。



 轉給事黃門侍郎,領主衣都統。雖迹在魏朝,而歸心霸府,密謀大計,皆得預聞。



 於是賓客輻湊,傾身接禮,甚得名譽,勢傾崔暹。暹嘗於朝堂屏人拜之曰:「暹若得僕射,皆叔父之恩。」其權重如此。



 時勳貴多不法,文襄無所縱舍,外議以季舒及崔暹等所為,甚被怨嫉。及文襄遇難,文宣將赴晉陽,黃門郎陽休之勸季舒從,曰:「一日不朝,其間容刀。」季舒性愛聲色,心在閑放,遂不請行,欲恣其行樂。司馬子如緣宿憾,及尚食典御陳山提等列其過狀。由是季舒及暹各鞭二百,徙北邊。天保初,文宣知其無罪,追為將作大匠。再
 遷侍中,俄兼尚書左僕射、儀同三司,大被恩遇。乾明初,楊愔以文宣遺旨,停其僕射。遭母喪解任。起服,除光祿勳,兼中兵尚書。出為齊州刺史。



 坐遣人度淮平市,亦有贓賄事,為御史所劾,會赦不問。武成居籓,曾病,文宣令季舒療病,備盡心力。大寧初,追還,引入慰勉。累遷度支尚書、開府儀同三司。



 營昭陽殿,敕令監造,以判事式。為胡長仁密言其短,出為西兗州刺史。為進典簽於吏部,被責免官。又以詣廣寧王宅,決韋鞭數十。及武成崩,不得預於哭泣。久之,除膠州刺史,遷侍中、開府,食新安、河陰二郡幹。加左光祿大夫,待詔文林館,監撰《御覽》。加特
 進,監國史。



 季舒素好圖籍,暮年轉更精勤,兼推薦人士,獎勸文學,議聲翕然,遠近稱美。



 祖珽受委,奏季舒總監內作。珽被出,韓長鸞以為珽黨,亦欲出之。屬車駕將適晉陽,季舒與張雕議,以為壽春被圍,大軍出拒,言使往還,須稟節度。兼道路小人,或相驚恐,云大駕向并州,畏避南寇,若不啟諫,必動人情。遂與從駕文官,連名進諫。時貴臣趙彥深、唐邕、段孝言等初亦同心,臨時疑貳,季舒與爭,未決。長鸞遂奏云:「漢兒文官,連名總署,聲云諫止向并州,其實未必不反,宜加誅戮。」



 帝即召已署表官人集含章殿,以季舒、張雕、劉逖、封孝琰、裴澤、郭遵等為
 首,並斬之殿庭。長鸞令棄其屍於漳水。自外同署,將加鞭撻,趙彥深執諫獲免。季舒等家屬男女徙北邊,妻女及子婦配奚官,小男下蠶室,沒入貲產。



 季舒本好醫術,天保中於徙所無事,更銳意研精,遂為名手,多所全濟。雖位望轉高,未曾懈怠;縱貧賤廝養,亦為之療護。



 庶子長君,尚書右外兵郎中。次鏡玄,著作佐郎。並流於長城。未幾,季舒等六人妻,以年老放出。後南安王思好更稱朝廷罪惡,以季舒等見害為詞,悉召六人兄弟子姪隨軍趣晉陽。事敗,長君等並從戮。六人之妻,又追入官。周武帝滅齊,詔斛律光與季舒等六人同被優贈,季舒贈
 開府儀同大將軍、定州刺史。



 挺從祖弟敬邕,性長者,為左中郎將,以軍功賜爵臨淄男,位營州刺史。庫莫奚國有馬數百疋,因風入境,敬邕悉令送還,於是夷人感附。卒於太中大夫,贈濟州刺史,謚曰恭。



 敬邕從弟接,字願賓。容貌魁偉,放邁自高,不拘檢。為中書博士、樂陵內史。



 雅為任城王澄所禮待,及澄為本部,接了無人王敬,王忻然容下之。後為樂陵太守,還鄉卒。



 挺族子纂,字叔則。博學有文才,既不為時知,乃著《無談子論》。尋為廷尉正,每有大獄,多所據明,有當官之譽。時太原王靜自廷尉監遷少卿,纂恥居其下,乃與靜書,辭氣抑揚,無上下禮。
 入啟求解位。後為洛陽令,卒,贈司徒左長史。



 纂兄穆,字子和,雅有度量,州辟主簿,卒。穆子暹。



 暹字季倫。少為書生,避地勃海,依高乾,以妹妻其弟慎。慎後臨滄、光二州,啟暹為工史,委以職事。趙郡公琛鎮定州,辟為開府諮議,隨琛往晉陽。神武與語悅之,以兼丞相長史。神武舉兵將入洛,留暹佐琛,凡百後事,一以屬暹,握手殷勤,至于三四。琛後以罪被責,暹亦黜免。尉景為并州,起暹為別駕。文襄代景,轉暹為開府諮議,仍行別駕事。從文襄鎮撫鄴都,加散騎常侍,遷左丞、吏部郎,領定州大中正,主議《麟趾格》。暹親遇日隆,好薦人士,
 言邢邵宜親重。言論之際,邵遂毀暹。文襄不悅,謂暹曰:「卿說子才長,子才專言卿短,此癡人也。」



 暹曰:「子才言暹短,暹說子才長,皆是實事,不為癡也。」高慎之叛,偽與暹隙,神武後知之,欲發其事而殺暹,文襄苦救得止。遷御史中尉,選畢義雲、盧潛、宋欽道、李愔、崔贍、杜蕤、嵇曄、酈伯偉、崔子武、李廣皆為御史,世稱其知人。



 文襄欲假暹威勢,諸公在坐朝,令暹後通名,因待以殊禮。暹乃高視徐步,兩人擎裾而入,文襄分庭對揖。暹不讓席而坐,觴再行,便辭退。文襄曰:「下官薄有蔬食,公少留。」暹曰:「適受敕,在臺檢校。」遂不待食而去,文襄降送之。旬日後,文襄
 與諸公出之東山,遇暹在道,前驅為赤棒所擊,文襄回馬避之。



 暹前後表彈尚書令司馬子如,及尚書元羨、殷州刺史慕容獻,又彈太師司州牧咸陽王恆、并州刺史可硃渾道元、冀州刺史韓軌,罪與鄴下諸貴,極言褒美,且誡屬之。先是僧尼猥濫,暹奏設科條篇,沙門法上為昭玄都以檢約之。



 神武如鄴,群官迎於紫陌。神武握暹手勞之曰:「小兒任重才輕,非中尉何有今日?榮華富貴,直是中尉自取,高歡父子無以相報。」賜暹馬,使騎之以從,且行且語。暹下拜,馬驚走,神武親為擁之而授轡。魏帝宴華林園,謂神武曰:「自頃所在百司,多有貪暴。朝廷
 中有用心公平,直言彈劾,不避親戚者,王可勸酒。」



 神武降階跪言:「唯御史中尉崔暹一人,謹奉明旨,敢以酒勸,并臣所射賜物千段,乞以回賜。」帝又褒美之。於是文襄亦催暹酒,神武親為之抃。文襄退,謂暹曰:「我尚畏羨,何況餘人!」神武將還晉陽,又以所乘馬加彩物賜暹。由是威名日盛,內外莫不畏服。



 神武崩,未發喪,文襄以暹為度支尚書,監國史,兼右僕射,委以心腹之寄,仍為魏帝侍讀。暹憂國如家,以天下為己任。文襄盛寵王昭儀,欲立為正室。暹諫曰:「天命未改,魏室尚存,公主無罪,不容棄辱。」文襄意不悅,苦請乃從之。



 文襄車服過度,誅戮變
 常,言談進止,或有虧失。暹每厲色極言,文襄亦為之止。



 臨淮王孝友被文襄狎愛,數歌舞戲謔於前,顧見暹,輒斂容而止。有獄囚數百,文襄盡欲誅之,每催文帳,暹故緩之,不以時進,文襄意釋,竟免。司州別駕司馬仲粲、中從事陸士佩並被文襄毆擊,付獄將餓殺,暹送食藥,為致言而釋之。



 自出身從官,常日晏乃歸。侵曉則與兄弟跪問母之起居,暮則嘗食視寢,然後至外齋,對親賓論事,或與沙門辯玄理,夜久乃還寢。一生不問家產,魏、梁通和,要貴皆遣人隨聘使交易,暹唯寄求佛經。梁武帝聞之,繕寫,以幡花寶蓋贊唄送至館焉。



 然好大言,調戲
 無節。嘗密令沙門明藏著《佛論》而署己名,傳諸江表。子達拏,年十三,令儒者權會教其解《周易》兩字,乃集朝貴名流,命達拏高坐開講。



 同郡眭仲讓陽屈服之,暹用仲讓為司徒中郎。鄴下為之語曰:「講義兩行得中郎。」



 仲讓官至右丞。此皆暹之短也。



 文宣初嗣霸業,司馬子如、韓軌等挾舊怨,言暹罪重。高隆之亦言宜寬政網,去糾察法官,黜崔暹,則得遠近人意,文宣從之。及踐阼,譖毀者猶不息,帝令都督陳山提、舍人獨孤永業搜暹家。甚貧匱,得神武、文襄與暹書千餘紙,多論軍國大事。帝嗟賞之。仍不免眾口,流暹於馬城,晝則負土供役,夜則置諸
 地牢。歲餘,奴告暹謀反,鎖赴晉陽,窮驗無實。



 先是,文襄疑文宣佯愚,慮其有後變,將陰圖之,以問暹。暹曰:「嘗與二郎俱在行位,試以手板拍其背而不瞋,乃將犀手板換暹竹者,自揩拭而玩視之,以是知其實癡。不足慮也。」帝既鎖暹,責其往昔打背。暹自陳所對文襄之言,明己功以贖死。帝悟曰:「我免禍,乃暹之力。」釋而勞之,使行太原郡事,遷太常卿。



 謂群臣曰:「崔暹清正,天下無雙,卿等不及也。」初,文襄欲以最小妹嫁與暹子達拏,會崩,遂寢。至是,宴於宣光殿,群臣多在焉,文宣謂暹曰:「賢子達拏甚有才學,亡兄長女樂安公主,魏帝外甥,勝朕諸妹,思
 成大兄宿志,故欲作婚姻。」



 乃以主降達拏。暹尋遷中書監,兼并省右僕射。是時法網已嚴,官司難於剖決,繫獄者千餘人。暹初上省,便大錄囚,旬月間,斷雪略盡。文襄時欲封暹,神武亦欲封之,暹並固辭。文宣數出游,多至暹宅,以暹女為皇太子妃,李后不可,乃止。



 天保八年,遷尚書右僕射、儀同三司。時調絹以七丈為匹,暹言之,乃依舊焉。帝謂左右曰:「崔暹諫我餘酒過多,然我飲酒何所廢?」常山王私謂暹曰:「至尊威嚴多醉,太后尚不能致言,吾兄弟杜口。僕射獨犯顏,內外深相感愧。」十年,卒,帝撫靈哭之,贈開府儀同三司、尚書左僕射、定州刺史,謚
 曰貞節。



 達拏溫良廉謹,有識學。位儀同三司、司農卿,周御府大夫。大象中使鄴,屬尉遲迥起兵,以為總管司馬。迥平,伏誅。初,文宣嘗問樂安公主:「達拏於汝何似?」答云:「甚相敬,唯阿家憎兒。」文宣令宮人召達拏母入而殺之,投漳水。



 齊滅,達拏殺主以復仇。暹兄謀開。



 纂從祖弟游,字延叔,少有風概。為東郡太守。郡有鹽戶,常供州郡為兵,子孫見丁從役。矜其勞苦,乃為表聞,請聽更代,郡內感之。太學舊在城內,游移置城南閑敞處,親自說經,當時學者莫不勸勉,號為良守。正光中,除南秦州刺史。



 先是,州人楊松柏、洛德兄弟數為反叛,游深加招慰,兄弟俱
 至。松柏既郡之豪帥,感恩獎喻,郡賊咸來歸款,且以過在前政,不復自疑,游乃因宴會,一時俱斬。於是外人以其不信,合境皆反。正光五年,秦州城人殺刺史李彥為逆。數日後,游知必不安,謀欲出外,尋為城人韓祖香等所攻。游事窘登樓,慷慨悲歎,乃推下小女而殺之,義不為群小所辱,為祖香等害。永安中,贈散騎常侍、鎮北將軍、定州刺史。子伏護。



 論曰:崔鑒以文業應利用之秋,世家有業,餘慶不已,人位繼軌,亦為盛哉!



 辯器業著聞,位不遠到;逸德優官薄,仍世恨之。模雄壯之烈,楷忠貞之操,殺身成義,臨難如
 歸,非大丈夫亦何能若此矣!士謙昆弟非唯武毅見重,忠公之稱,亦足嘉云。挺兄弟風操高亮,懷文抱質,歷事著聞,見重朝野,繼世承家,門族並著,市朝可變,人焉不絕。至若宣猷之立入朝贊務,則嘉謀屢陳,出撫宣條,則威恩具舉。仲方之兼資文武,雅長謀算,伐陳之策,信為深遠。弈世載德,夫豈徒然?昂智足立功,能足幹事,霸朝委遇,良有以焉。而謝彼仁心,安茲苛政,晚途遭躓,理其宜也。季舒蹈龍逢之節,季倫受分庭之遇,雖遭逢異日,得喪不同,考其遺跡,而榮名一也,蓋所謂彼有人焉。



\end{pinyinscope}