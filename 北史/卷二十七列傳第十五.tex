\article{卷二十七列傳第十五}

\begin{pinyinscope}

 屈遵張蒲谷渾曾孫楷公孫表張濟李先賈彞竇瑾李韓延之袁式毛修之嚴棱硃修之唐和寇贊孫俊酈範子道元韓秀堯暄孫雄柳崇屈遵,字子度,昌黎徒何人也。博學多才藝。慕容垂以為博陸令。道武南伐,博陵太守申永南奔河外,高陽太守
 崔宏東走海濱。屬城長吏,率多逃竄,遵獨歸道武。道武素聞其名,拜中書令。中原既平,賜爵下蔡子。卒。



 子須襲爵。除長樂太守,進爵信都侯。卒,贈昌黎公,謚曰恭。



 須長子恆,字長生,沈粹有局量。歷位尚書右僕射,加侍中。以破平涼功,賜爵濟北公。太武委以大政,車駕出征,常居中留鎮。與襄城公盧魯元俱賜甲第。真君四年,墜馬卒。時帝幸陰山,景穆遣使乘傳奏狀。帝甚悼惜之,謂使人曰:「汝等殺朕良臣,何用乘馬?」遂令步歸。贈征西大將軍,謚曰成公。



 子道賜襲爵。道賜善騎射,機辯有辭氣,太武甚器之。位尚書右僕射,加侍中。



 卒,謚曰哀公。



 子拔襲爵。
 帝追思其父祖,年十四,以為南部大人。時太武南伐,禽守將胡盛之以付拔。酒醉不覺,盛之逃。太武令斬之。將伏鑕,帝愴然曰:「若鬼有知,長生問其子孫,朕將何以應?」乃赦拔。後獻文以其功臣子,拜營州刺史。



 張蒲,字玄則,河內修武人也。本名謨。父攀,仕慕容垂,位兵部尚書,以清方稱。蒲少有父風,仕慕容寶為尚書左丞。道武定中山,寶官司敘用,多降品秩。



 帝既素聞蒲名,仍拜尚書左丞。明元即位,為內都大官,賜爵泰昌子。參決庶獄,私謁不行。後改為壽張子。太武即位,以蒲清貧,妻子衣食不給,乃以為相州刺史。



 扶弱抑彊,進善黜惡,
 風化大行。卒於官,吏人痛惜之。蒲在謀臣之列,屢出為將,朝廷論之,常以為稱首。贈平東將軍、廣平公,謚曰文恭。子昭襲。以軍功進爵修武侯,位幽州刺史,以善政見稱。



 谷渾,字元沖,昌黎人也。父袞,彎弓三百斤,勇冠一時。仕慕容垂,位廣武將軍。渾少有父風,任俠好氣,晚乃折節受經業,被服類儒者。道武時,以善隸書為內侍左右。太武時,累遷侍中、儀曹尚書,賜爵濮陽公。渾正直有操行,性不茍合。然愛重舊故,不以富貴驕人,時人以此稱之。在官廉直,為太武所器重。以渾子孫年十五以上,悉補
 中書學生。卒,謚曰文宣。



 子闡,字崇基,襲爵。位外都大官。卒,謚曰簡公。子洪,字元孫,位尚書,賜爵滎陽公。性貪奢,僕妾衣服錦綺。時獻文舅李峻等初至,官給衣服,洪輒截沒。



 為有司所糾,並窮其前後贓罪,伏法。子穎,位太府少卿。卒,贈營州刺史,謚曰貞。子士恢,字紹達,位鴻臚少卿,封元城縣侯。太后嬖幸鄭儼,懼紹達間構於帝,因言次,以紹達為州。紹達耽寵,不願出。太后誣其罪,殺之。



 渾曾孫楷。楷有幹局,稍遷奉車都尉。眇一目,性甚嚴忍,前後奉使皆以酷暴為名,時人號曰「瞎武」。累遷城門校尉,卒。



 公孫表,字玄元,燕郡廣陽人也。為慕容沖尚書郎。慕容垂破長子,從入中山。



 慕容寶走,乃歸,為博士。初,道武以慕容垂諸子分據勢要,權柄推移,遂至亡滅,表詣闕上《韓非書》二十卷。道武稱善。明元初,賜爵固安子。河西飢胡劉武反於上黨,詔表討之。為胡所敗,帝深銜之。泰常七年,宋武帝殂。時議取河南侵地,以奚斤為都督,以表為吳兵將軍、廣州刺史。表既剋滑臺,遂圍武牢。車駕次汲郡。



 始昌子蘇坦、太史令王亮奏表置軍武牢東,不得形便之地,故令賊不時滅。明元雅好術數,又積前忿,及攻武牢,士卒多傷,乃使人夜就帳中縊殺之。以賊未退,
 秘而不宣。



 初,表與勃海封愷友善,後為子求愷從女,愷不許,表甚銜之。及封氏為司馬國璠所逮,帝以舊族,欲原之。表證其罪,乃誅封氏。表外和內忌,時人以此薄之。



 表本與王亮同營署,及其出也,輕侮亮,故及於死。



 第二子軌,字元慶。明元時,為中書郎。出從征討,補諸軍司馬。太武平赫連昌,引諸將帥入其府藏,各令任意取金玉。諸將取之盈懷,軌獨不取。帝把手親探金賜之,謂曰:「卿臨財廉,朕所以增賜者,欲顯廉於眾人。」後兼大鴻臚,持節拜立氐楊玄為南秦王。及境,玄不郊迎;軌數玄無蕃臣禮。玄懼,詣郊受命。使還稱旨,拜尚書,賜爵燕郡公,出
 為武牢鎮將。初,太武將北征,發驢以運糧,使軌部調雍州。軌令驢主皆加絹一匹,乃與受之。百姓語曰:「驢無彊弱,輔脊自壯。」



 眾共嗤之。坐徵還。卒。



 軌既死,帝謂崔浩曰:「吾過上黨,父老皆曰:公孫軌為將,受貨縱賊,使至今餘姦不除,軌之罪也。其初來,單馬執鞭;及去,從車百兩。載物而南,丁零渠帥,乘山罵軌。軌怒,取罵軌者之母,以矛刺其陰而死之,曰:『何以生此逆子!』從下倒劈,分磔四支於山樹上。是忍行不忍之事。軌幸而早死,至今在者,吾必族誅之。」



 軌終得娶封氏,生子睿,字叔文。位儀曹長,賜爵陽平公。時獻文於苑內立殿,敕中秘群官制名。睿奏
 曰:「臣聞至尊至貴,莫崇於帝王;天人挹損,莫大於謙光。



 臣愚以為宜曰崇光。」奏可。卒於南部尚書,謚曰宣。



 睿妻,崔浩弟女也。生子良,字遵伯,聰明好學。為尚書左丞,為孝文所知遇。



 良弟衡,字道津。良推爵讓之,仕至司直。良以別功,賜爵昌平子。子崇基襲。



 軌弟質,字元直,有經義,為中書學生,稍遷博士。太武征涼州,留宜都王穆壽輔景穆。時蠕蠕乘虛犯塞,京師震恐。壽雅信任質,為謀主。質性好卜筮;卜筮者咸云必不來,故不設備。由質,幾敗國。後屢進讜言,超遷尚書。卒,贈廣陽侯,謚曰恭。



 第二子邃,字文慶,位南部尚書,封襄平伯,出為青州刺史。以邃
 在公遺迹可紀,下詔褒述。卒官。孝文在鄴宮,為之舉哀。時百度唯新,青州佐吏疑為邃服,詔曰:「專古也,理與今違;專今也,太乖曩義。當斟酌兩途,商量得失,人吏之情亦不可茍順也。主簿云,近代相承服斬,過葬便,可如故。自餘無服,大成寥落。



 可準諸境內之人,為齊衰三月。」子同始襲爵,卒於給事中。



 邃、睿為從父兄弟。睿才器小優,又封氏之男,崔氏之婿。邃母鴈門李氏,地望懸隔。鉅鹿太守祖季真多識北方人物,每云:「士大夫當須好婚親。二公孫同堂兄弟耳,吉凶會集,便有士庶之異。」



 張濟,字士度,西河人也。父千秋,慕容永驍騎將軍。永滅,
 來奔。道武善之,拜建節將軍,賜爵成紀侯。濟涉獵書傳,清辯善儀容。道武愛之,與公孫表等俱為行人,拜散騎侍郎,襲爵。先是,晉雍州刺史楊佺期乞師於常山王遵以禦姚興。帝遣濟為遵從事,即報之。濟自襄陽還,帝問濟江南事。濟曰:「司馬昌明死,子德宗代立,君弱臣彊,全無綱紀。佺期問臣:『魏初伐中山,幾十萬眾?』臣答:『四十餘萬。』佺期曰:『魏被甲戎馬,可有幾匹?』臣答:『中軍精騎十餘萬,外軍無數。』佺期曰:『以此討羌,豈不滅也!』又曰:『魏定中山,徙幾戶於北?』臣答:『七萬餘家。』佺期曰:『都何城?』臣答:『都平城。』佺期曰:『有此大眾,何用城為!』又曰:『魏帝欲為久都
 平城?將移也?』臣答:『非所知也。』佺期聞朝廷不都山東,貌有喜色,曰:『洛城救援,仰恃於魏,若獲保全,當必厚報。如為羌所乘,寧使魏取。』」道武嘉其辭,厚賞其使,許救洛陽。後以累使稱旨,拜勝兵將軍。卒,子多羅襲爵,坐事除。



 李先,字容仁,中山盧奴人。少好學,善占相術。慕容永迎為謀主,勸永據長子城。仕永,位祕書監。永滅,徙中山。皇始初,先於井陘歸。道武問先曰:「卿何國人?祖父及身悉歷何官?」先曰:「臣本趙郡平棘人。大父重,晉平陽太守、大將軍右司馬。父懋,石季龍樂安太守、左中郎將。臣,苻丕左主客郎,慕容永祕書監、高密侯。」車駕還代,以先為尚
 書右中兵郎。再遷博士、定州大中正。帝問先:「何者最善,可以益人神智?」先曰:「唯有經書,三皇、五帝政化之典,可以補王者神智。」又問:「朕欲集天下書籍,如何?」對曰:「主之所好,集亦不難。」帝於是班制天下,經籍稍集。



 道武討姚興於柴壁也,問計於先。對曰:「兵以正合,戰以奇勝。聞姚興欲屯兵天渡,利其糧道。及其到前,遣奇兵先邀天渡,柴壁左右嚴設伏兵,備其表裏,興欲進不得,住又乏糧。夫高者為敵所棲,深者為敵所囚,兵法所忌。而興居之,可不戰而取。」帝從其計,興果敗歸。明元即位,問左右:「舊臣中誰為先帝所親信?」新息公王洛兒曰:「有李先者,為
 先帝所知。」俄而召先,讀韓子《連珠論》二十二篇,《太公兵法》十一事。詔有司曰:「先所知者,皆軍國大事,自今常宿於內。」賜先絹彩及御馬一匹,拜安東將軍、壽春侯,賜隸戶二十二。卒於內都大官,年九十五。詔賜金縷命服一襲,贈定州刺史、中山公,謚曰文懿。子國襲爵。



 國子鳳,中書博士。鳳子預,字元凱。太和初,歷祕書令、齊郡王友、征西大將軍長史,帶馮翊太守。府解,罷郡,遂居長安。羨古人飧玉法,乃採訪藍田,躬往攻掘,得若環璧雜器形者,大小百餘。頗有粗黑者,亦篋盛以還。至而觀之,皆光潤可玩。預乃椎七十枚為屑食之,餘多惠人。後預及聞者
 更求玉於故處,皆無所見。馮翊公源懷弟得其玉,琢為器佩,皆鮮明可寶。預服經年,云有效驗。而世事寢食,皆不禁節,又加好酒損志。及疾篤,謂妻子曰:「吾酒色不絕,自致於死,非藥過也。然吾尸體必當有異,勿速殯,令後人知飧服之妙。」時七月中旬,長安毒熱,預停屍四宿,而體色不變。其妻常氏,以玉珠二枚琀之,口閉。常謂曰:「君自云飧玉有神驗,何不受琀?」言訖,齒啟納珠。因噓其口,都無穢氣。舉斂於棺,堅直不傾委。死時有遺玉屑數升,囊盛納諸棺中。



 先少子皎。天興中,密問先曰:「子孫永為魏臣,將復事他姓邪?」先曰:「國家政化長遠,不可紀極。」皎
 為寇謙之弟子,遂服氣絕粒數十年,隱於恆山。



 年九十餘,顏如少童。一旦,沐浴冠帶,家人異之,俄而坐卒。道士咸稱其得尸解仙道。



 皎孫義徽。太和中,以儒學博通,有才華,補清河王懌府記室。箋書表疏,文不加點,清典贍速,當世稱之。又為懌撰《輿地圖》及《顯忠錄》。性好《老莊》,甚嗤釋教。靈太后臨朝,屬有沙門惠憐以咒水飲人,云能愈疾,百姓奔湊,日以千數。義徽白懌,稱其妖妄。因令義徽草奏以諫,太后納其言。元叉惡懌,徙義徽都水使者。俄而懌被害,因棄官隱於大房山。



 少子蘭,以純孝著聞,不受辟召。孝昌中,旌表門閭。



 正光中,文宣王亶嗣位,思
 義徽雅正惇篤,薦其孫景儒,位至奉車都尉。自皇始至齊受禪,百五十歲。先之所言,有明徵焉。



 景儒子昭徽,博涉稽古,脫略不羈,時人稱其為播郎。因以字行於燕、趙焉。



 善談論,有宏辯,屬文任氣,不拘常則。志好隱逸,慕葛洪之為人。尋師訪道,不遠千里。遇高尚則傾蓋如舊,見庸識雖王公蔑如。初為道士,中年應詔舉,為高唐尉。大業中,將妻子隱於嵩山,號黃冠子。有文集十卷,為學者所稱。



 賈彞,字彥倫,本武威姑臧人也。六世祖敷,魏幽州刺史、廣川都亭侯,子孫因家焉。父為苻堅鉅鹿太守,坐訕謗
 系獄。彞年十歲,詣長安訟父獲申。遠近歎之,僉曰:「此子英英,賈誼之後,莫之與京。」弱冠,為慕容垂遼西王農記室參軍。



 道武先聞其名,常遣使者求彞於垂,垂彌增器敬。垂遣其太子寶來寇,大敗於參合,執彞及其從兄代郡太守潤等。道武即位,拜尚書左丞,參預國政。天賜末,彞請詣溫陽療疾,為叛胡所掠,送於姚興。積數年遁歸,又為赫連屈丐所執,拜祕書監,卒。太武平赫連昌,子秀迎其尸柩,葬於代南。



 秀位中庶子,賜爵陽都男,本州大中正。獻文即位,進爵陽都子。時丞相乙渾妻庶姓,而求公主之號,屢言於秀,秀默然。後因公事,就第見渾。渾夫
 妻同坐,厲色曰:「爾管攝職事,無所不從。我請公主,不應,何意?」秀慷慨大言對曰:「公主之稱,王姬之號,尊寵之極,非庶族所宜。秀寧就死於今朝,不取笑於後日。」



 渾左右莫不失色,為之震懼,秀神色自若。渾夫妻默然含忿。他日,乃書太醫給事楊惠富臂,作「老奴官慳」字,令以示秀。渾每欲伺隙陷之。會渾伏誅,遂免難。



 時秀與中書令勃海高允俱以儒舊重於時。皆選擬方岳,以詢訪被留,各聽長子出為郡。秀固讓不受,許之。自始及終,歷奉五帝。雖不至大官,常當機要。廉清儉約,不營資產。年七十三,遇疾,詔給醫藥,賜几杖。時朝廷舉動及大事不決,每遣尚
 書、高平公李敷就第訪決。卒,贈冀州刺史、武邑公,謚曰簡。



 子俊,字異鄰。襲爵,位荊州刺史,依例降爵為伯。先是,上洛置荊州,後改為洛州,在重山,人不知學,俊表置學官。在州五載,清靖寡事,為吏人所安。卒,贈兗州刺史。子叔休襲爵。



 潤曾孫禎,字叔願,學涉經史,居喪以孝聞。太和中,以中書博士副中書侍郎高聰使江左。還,以母老患,輒在家定省,坐免官。後為司徒諮議參軍、通直散騎常侍,加冠軍將軍。卒,贈齊州刺史。



 禎兄子景俊,亦以學識知名,為京兆王愉府外兵參軍。愉起逆於冀州,將授其官;不受,死之。贈河東太守,謚曰貞。



 景俊弟景輿,清
 峻鯁正,為州主簿,遂棲遲不仕。後葛榮陷冀州,稱疾不拜。



 景輿每捫膝而言曰:「吾不負汝。」以不拜榮也。



 竇瑾,字道瑜,頓丘衛國人,自云漢司空融之後也。高祖成,頓丘太守,因家焉。瑾少以文學知名,自中書博士為中書侍郎,賜爵繁陽子。參軍國謀,屢有功,進爵衛國侯,轉四部尚書。初定三秦,人猶去就,拜長安鎮將、毗陵公。在鎮八年,甚著威惠。徵為殿中都官尚書。太武親待之,賞賜甚厚。從征蓋吳,吳平,留瑾鎮長安。還京復為殿中、都官,典左右執法。太武歎曰:「國之良輔,毗陵公之謂矣。」



 出為冀州刺史,清約沖素,著稱當時。還為內都大官。興
 光初,瑾女婿鬱林公司馬彌陀以選尚臨涇公主,瑾教彌陀辭。託有誹謗咒詛之言,與彌陀同誅,唯少子遵逃匿得免。



 遵善楷篆,北京諸碑及臺殿樓觀宮門題署多遵書。位濮陽太守,多所受納。其子僧演姦通人婦,為部人賈邈告,坐免。後以善書拜庫部令,卒官。



 李,字元盛,小名真奴,范陽人也。曾祖產,產子績,二世知名於慕容氏。



 父崇,馮跋吏部尚書、石城太守。車駕至和龍,崇率十餘郡歸降,太武甚禮之,呼曰李公。為北幽州刺史、固安侯。卒,謚曰襄侯。母賤,為諸兄所輕。崇曰:「此子之生,相者言貴,吾每觀,或未可知。」遂使入都為中
 書學生。太武幸中書學,見而異之,指謂從者曰:「此小兒終效用於朕之子孫。」因識眄之。帝舅陽平王杜超有女,將許貴戚,帝曰:「李後必官達,益人門戶,可以妻之。」遂勸成婚。南人李哲常言必當貴達。杜超之死也,帝親哭三日。以超女婿,得在喪位出入。帝指謂左右曰:「觀此人舉動,豈不異於眾也?必為朕家幹事臣。」聰敏機辯,彊記明察。初,李靈為文成博士,詔崔浩選中書學生器業優者為助教。浩舉其弟子箱子與盧度世、李敷三人應之。給事高讜子祐、尚書段霸兒姪等以為浩阿黨其親戚,言於景穆。以浩為不平,聞之於太武。太武意在
 ,曰:「云何不取幽州刺史李崇老翁兒?」浩對曰:「前亦言合選,但以其先行在外,故不取之。」帝曰:「可待還,箱子等罷之。」遂除中書助教、博士,入授文成經。



 文成即位,以舊恩親寵,遷儀曹尚書,領中祕書,賜爵扶風公。贈其母孫氏為容城君。帝顧群臣曰:「朕始學之歲,情未能專;既總萬機,溫習靡暇。是故儒道實有闕焉。豈惟予咎,抑亦師傅之不勤。所以爵賞仍隆,蓋不遺舊也。」免冠拜謝。出為相州刺史。為政清簡,百姓稱之。上疏求於州郡各立學官,使士望之流,衣冠之胄,就而受業。其經藝通明者,上王府。書奏,獻文從之。以政為諸州之最,
 加賜衣服。自是遂有驕矜自得之志,受納人財物,商胡珍寶。兵人告言。



 尚書李敷與少長相好,每左右之。或有勸以奏聞,敷不許。獻文聞罪狀,檻車徵,拷劾抵罪。敷兄弟將見疏斥,有司諷以中旨嫌敷兄弟之意,令告列敷等隱罪,可得自全。深所不欲,且弗之知也,乃謂其女婿裴攸曰:「吾與李敷,族世雖遠,情如一家。在事既有此勸,昨來引簪自刺,以帶自絞,而不能致絕。且亦不知其事。」攸曰:「何為為他死?敷兄弟事釁可知。有馮闌者,先為敷殺,其家切恨之。但呼闌弟問之,足可知委。」從其言。又趙郡范標具列敷兄弟事狀,有司以聞,敷
 坐得罪。詔列貪冒應死,以糾李敷兄弟,故免。百鞭髡刑,配為廝役。



 之廢也,平壽侯張讜見,與語,奇之。謂人曰:「此佳士也,終不久屈。」



 未幾而復為太倉尚書,攝南部事。用范標陳策計,令千里之外,戶別轉運,詣倉輸之。使所在委滯,停延歲月。百姓競以貨賂,各求在前,於是遠近大為困弊。道路群議曰:「畜聚斂之人,未若盜臣。」弟左軍將軍璞謂曰:「范標善能降人以色,假人以辭,未聞德義之言,但有勢利之說。聽其言也甘,察其行也賊,所謂諂諛讒慝,貪冒姦佞。不早絕之,後悔無及。」不從,彌信之,腹心事皆以告標。



 既寵於獻文,參決軍國
 大議,兼典選舉,權傾內外,百寮莫不曲節以事之。標以無功起家拜盧奴令。



 獻文崩,遷司空,進爵范陽公,出為侍中、鎮南大將軍、開府儀同三司、徐州刺史。范標知文明太后之忿,又知內外疾之,太和元年,希旨告外叛。文明太后徵至京師,言其叛狀。曰:「無之。」引標證。言:「爾妄云知我,吾又何言!雖然,爾不顧餘之厚德,而忍為此,不仁甚矣。」標曰:「公德於標,何若李敷之德於公?公昔忍於敷,標今敢不忍公乎?」慨然曰:「吾不用璞言,自貽伊戚,萬悔於心,何嗟及矣!」遂見誅。



 璞字季直,性惇厚,多識人物。賜爵宜陽侯,太常卿。



 韓延之,字顯宗,南陽堵陽人,魏司徒暨之後也。仕晉,位建威將軍、荊州從事,轉平西府錄事參軍。晉將劉裕伐司馬休之,未至江陵,密與延之書招之。延之報書,辭甚激厲,曰:「劉裕足下:海內之人,誰不見足下此心,而復欲欺誑國士!」



 其不屈如此。事見《南史宋本紀》。延之以裕父名翹,字顯宗,於是己字顯宗,名子為翹,蓋示不臣劉氏也。後奔姚興。泰常二年,與司馬文思等俱入魏。明元以延之為武牢鎮將,賜爵魯陽侯。



 初,延之曾來往柏谷塢,省魯宗之墓,有終焉之志。因謂之孫云:「河洛三代所都,朝廷必有居此者。我死,不勞向北代葬也,即可就此。」子
 從其言,遂葬宗之墓次。延之後五十餘年而孝文徙都,其孫數家即居於祖墓之北柏谷塢。



 袁式,字季祖,陳郡陽夏人,漢司徒滂之後。父深,晉侍中。式在南,歷武陵王遵諮議參軍。及劉裕執權,式歸姚興。及姚泓滅,歸魏,為上客,賜爵陽夏子。



 與司徒崔浩一面,便盡國士之交。時朝儀典章悉出於浩,浩以式博於故事,每所草創,恆顧訪之。性長者,雖羈旅飄泊,而清貧守度,不失士節。時人甚敬重之,皆呼曰袁諮議。至延和二年,衛大將軍、樂安王範為雍州刺史,詔式與中書侍郎高允俱為從事中郎。辭而獲免。



 式沈靖樂道,周覽書傳,
 至於詁訓《倉》、《雅》,偏所留懷。作《字釋》未就。以太安二年卒,贈豫州刺史,謚肅侯。



 子濟襲父爵,位魏郡太守,政有清稱。加寧遠將軍。及宋王劉昶開府,召為諮議參軍。



 毛脩之,字敬文,滎陽陽武人也。世仕晉。劉裕之平關中,留子義真鎮長安,以脩之為司馬。及義真敗,脩之沒統萬。太武平赫連昌,獲之。使領吳兵,以功拜吳兵將軍。脩之能為南人飲食,手自煎調,多所適意。太武親待之,累遷尚書,賜爵南郡公,常在太官主進御膳。從討和龍,時諸軍攻城,行宮人少,宋故將朱脩之為雲中將軍,欲率吳兵為逆。因入和龍,冀浮海南歸。以告修之,不聽,乃止。
 是日無脩之,大變幾作。硃脩之遂奔馮弘。脩之又以軍功,遷特進、撫軍大將軍,位次崔浩下。



 浩以其中國舊門,雖不博洽,猶涉獵書傳,與共論說之。次及陳壽《三國志》,云「有古良史風,其所著述,文義典正,班史以來無及壽者」。脩之曰:「昔在蜀中,聞長老言,壽曾為諸葛亮門下書佐,得撻百下,故其論武侯云:應變非其所長。」



 浩乃與論曰:「承祚之評亮,乃有故義過美之譽,非挾恨之言。夫亮之相備,英雄奮發之時,君臣相得,魚水為喻。而不能與曹氏爭天下,委棄荊州,退入巴蜀,守窮崎嶇之地,僭號邊夷之間,此策之下者。可以趙佗為偶,而以管、蕭之亞匹,
 不亦過乎!且亮既據蜀,弗量勢力,嚴威切法,控勒蜀人,欲以邊夷之眾,抗衡上國。



 出兵隴右,再攻祁山,一攻陳倉,疏遲失會,摧衄而反。後入秦川,更求野戰。魏人知其意,以不戰屈之。智窮勢盡,發病而死。由是言之,豈合古之善將,見可知難乎?」脩之謂浩言為然。後卒於外都大官,謚恭公。



 脩之在南有四子,唯子法仁入魏。文成初,為金部尚書,襲爵,轉殿中尚書。



 法仁言聲壯大,至於軍旅田狩,唱呼處分,振於山谷。卒,贈征東大將軍、南郡王,謚曰威。



 朱脩之者,仕宋為司徒從事中郎。守滑臺,為安頡所禽。太武善其固守,以宗室女妻之,以為雲中鎮將。後
 奔馮弘。弘送之江南。頡之剋滑,宋陳留太守嚴稜戍倉垣。及山陽公奚斤軍至潁川,稜率文武五百人詣斤降。明元嘉其誠款,賜爵郃陽侯,假荊州刺史。隨駕南討,還為上客。及太武踐阼,以歸化之功,除中山太守,有清廉稱。卒於家。子幼玉襲。稜舊書有傳,今附之云。脩之在宋顯達,事並具《南史》。



 唐和,字幼起,晉西宜安人也。父繇,以涼土喪亂,推涼武昭王霸于河右。及涼亡,和與兄契攜其甥武昭王孫寶,避難伊吾。招集人眾二千餘家,臣於蠕蠕。蠕蠕以契為伊吾王。



 經二十年,和與契遣使降魏,為蠕蠕所逼,遂擁
 部至高昌。蠕蠕遣部帥阿若討和,至白力城。和先攻高寧。契與阿若戰沒,和收餘眾,奔前部國。時沮渠安周屯橫截城,和攻拔之。斬安周兄子樹,又剋高寧、白力二城。遣使表狀。太武嘉之,屢賜之璽書。後和與前部王車伊洛破安周。太武使周公萬度歸討焉耆,詔和與伊洛率所領赴度歸,喻下柳驢以東六城。因共擊波居羅城,拔之。後同征龜茲,度歸令和鎮焉耆。時柳驢戍主乙真伽將叛,和徑入其城,禽斬乙真伽。由是西域剋平,和有力焉。



 正平元年,和詣闕。太武優寵之,待以為上客。文成以和歸誠先朝,封酒泉公。



 太安中,為濟州刺史,甚有稱
 績。徵為內都大官。評決獄訟,不加捶楚,察疑獲實者甚多,世以是稱之。卒,贈征西大將軍、太常卿、酒泉王,謚曰宣。



 子欽,字孟真,位陜州刺史。降爵為侯。卒,子景宣襲爵。卒於東都太守。



 契子玄達,性果毅,有父風。與叔父和歸闕,俱為上客,封晉昌公。獻文時,位華州刺史。太和十六年,降為侯。子崇,字繼祖,襲爵。



 寇讚,字奉國,上谷人也,因難徙馮翊萬年。父脩之,字延期,苻堅東萊太守。



 贊弟謙,有道術,太武敬重之,故追贈修之安西將軍、秦州刺史、馮翊公。賜命服,謚曰哀公。詔秦、雍二州為立碑墓。又贈脩之母為馮翊夫人,及宗從
 追贈太守、縣令、侯、子、男者十六人,其臨職者七郡、五縣。



 讚少以清潔知名。身長八尺,姿容嚴嶷,非禮不動。苻堅僕射韋華,州里高達,雖年時有異,恆以風味相待。華為馮翊太守,召為功曹。後除襄邑令。姚泓滅,秦、雍人千餘家推讚為主,歸魏。拜河南郡太守。其後秦、雍人來奔河南、滎陽、河內者,戶至萬數,拜讚南雍州刺史、軹縣侯,於洛陽立雍州之郡縣以撫之。由是流人襁負,自遠而至,參倍於前。進贊爵河南公,加安南將軍,領南蠻校尉,仍刺史。



 分洛、豫二州之僑郡以益之。雖位高爵重,接待不倦。



 初,贊之未貴,嘗從相者唐文相。文曰:「君額上黑子入幘,
 位當至方伯,封公。」及其貴也,文以百姓禮拜謁曰:「明公憶疇昔言乎?」延文坐曰:「往時卿言杜瓊不得官長,人咸謂不然。及瓊為盩厔令,卿猶言相中不見,而瓊果以暴疾,未拜而終。昔魏舒見主人兒死,自知己必至公。吾恆以卿言瓊之驗,亦復不息此望也。」乃賜文衣服良馬。贊在州十七年,甚收公私之譽。年老,求致仕。卒,遺令薄葬,斂以時服。太武悼惜之,謚曰宣穆。子元寶襲爵。



 元寶弟臻,字仙勝。年十二,遭父憂,居喪以孝稱。輕財好士。獻文末,為中川太守。時馮熙為洛州刺史,政號貪虐,仙勝微能附之,甚得其意。後為弘農太守。



 坐受納,為御史所彈,
 遂廢,卒於家。



 子祖訓,順陽太守。祖訓弟祖禮。兄弟並孝友敦穆,白首同居。父母亡雖久,猶於平生所處堂宇,備設幃帳几杖,以時節開堂列拜;垂涕陳薦,若宗廟焉。吉凶之事,必先啟告,遠出行反亦如之。祖禮,宣武末為河州刺史。在任數年,遇郤鐵匆反,又為城人詣都列其貪狀十六條。會赦免。久之,兼廷尉卿,又兼尚書。畏避勢家,承顏候色,不能有所執據。後蠻反於三鴨,為都督追討,戰歿。贈衛大將軍、七兵尚書、雍州刺史、昌平男。祖禮弟俊。



 俊字祖俊。性寬雅,幼有識量,好學彊記。性又廉恕,不以
 財利為心。家人曾賣物與人,而利得絹一匹。俊於後知之,乃曰:「得財失行,吾所不取。」訪主還之。以選為孝文帝挽郎,除奉朝請。大乘賊起,燕、趙擾亂,俊參護軍事東討,以功授員外散騎侍郎。累遷司空府主簿。時靈太后臨朝,減食祿官十分之一,造永寧佛寺,令俊典之。資費巨萬,主吏不能欺隱。寺成,又極壯麗。靈太后嘉之,除左軍將軍。孝昌中,朝議以國用不足,乃置鹽池都將,秩比上郡。前後居職者多有侵隱,乃以俊為之,仍主簿。



 永安初,華州人史底與司徒楊椿訟田。長史以椿勢貴,皆言椿直,欲以田給椿。



 俊曰:「史底窮人,楊公橫奪其地,若欲損
 不足以給有餘,見使雷同,未敢聞命。」



 遂以地還史底。孝莊帝後知之,嘉俊守正不撓,拜司馬;其附椿者咸責焉。



 二年,出為梁州刺史。人俗荒獷,多為盜賊。俊乃令郡縣為立庠序,勸其耕桑,敦以禮讓。數年之中,風俗頓革。梁遣其將曹琰之鎮魏興,繼日板築。琰之屢擾疆場,邊人患之。俊遣長史杜林道攻克其城,并禽琰之。琰之即梁大將景宗之季弟也。



 於是梁人懼焉。屬魏室多故,州又僻遠,梁人知無外援,遂大兵頓魏興,志圖攻取。



 俊撫厲將士,人思效命。梁人知其得眾心也,弗之敢逼。俊在州清苦,不事產業,其子等並徒步而還。吏人送俊,留連於
 道,久之乃得出界。



 大統三年,東魏授俊洛州刺史,俊因此乃謀歸闕。五年,將家及親屬四百口入關,拜秘書監。時軍國草創,墳典散逸,俊始選置令史,抄集經籍,四部群書,稍得周備。加鎮東將軍,封安西縣男。十七年,加散騎常侍,遂稱篤疾,不復朝覲。



 恭帝三年,賜姓若口引氏。孝閔帝踐阼,進爵為子。武成元年,進驃騎大將軍、開府儀同三司。



 俊年齒雖高,而志識未衰。教授子孫,必先典禮。明帝尚儒重德,特欽賞之,數加思賜,思與相見。俊不得已,乃入朝,帝與同席而坐,顧訪洛陽故事。俊身長八尺,鬚鬢皓然,容止端詳,音韻清朗。帝與之談論,不覺屢
 為之前膝。及俊辭還,帝親執其手,曰:「公年德俱尊,朕所欽尚。乞言之事,所望於公。宜數相見,以慰虛想。」以御輿令於帝前乘出。顧謂左右曰:「如此事,唯積善者可以致之。何止見重於今,亦將傳之萬古。」時人咸以為榮。卒年八十二。武帝歎惜之,贈本官,加冀定瀛三州諸軍事、冀州刺史,謚曰元。



 俊篤於仁義,期功之中有孤幼者,衣食豐約,并與之同。少為司徒崔光所知,光命其子勵與俊結友。俊每造光,常清談移日。小宗伯盧辯以俊業行俱崇,待以師友之禮。每有閑暇,輒詣俊宴語彌日。恒謂人曰:「不見安西君,煩憂不遣。」其為通人所敬重如此。子奉,
 位至儀同大將軍、順陽郡守、洵州刺史、昌國縣公。



 奉弟顒,少好學,最知名。居喪哀毀。位儀同大將軍,掌朝、布憲、典祀下大夫,小納言,濩澤郡公。



 酈範,字世則,范陽涿鹿人也。祖紹,慕容寶濮陽太守,以郡迎降,道武授兗州監軍。父嵩,天水太守。範,太武時,給事東宮。太武踐阼,追錄先朝舊勛,賜爵永寧男。以奉禮郎奉遷太武、景穆神主於太廟,進爵為子。為征南大將軍慕容白曜司馬。及定三齊,範多進策,白曜皆用其謀,遂表為青州刺史。進爵為侯,加冠軍將軍,還為尚書右丞。後除平東將軍、青州刺史,假范陽公。範前解州還京
 也,夜夢陰毛拂踝。他日說之。時齊人有占夢者史武進云:「公豪盛於齊下矣。」使君臨撫東秦,道光海岱,必當重牧全齊,再祿營丘矣。」範笑答曰:「吾將為卿必驗此夢。」果如言。時鎮將元伊利表範與外賊交通。孝文詔範曰:「鎮將伊利表卿造船市玉,與外賊交通,規陷卿罪,窺覦州任。有司推驗,虛實自顯,有罪者今伏其辜矣。卿其明為算略,勿復懷疑。」還朝,卒京師。謚曰穆。子道元。



 道元字善長。初襲爵永寧侯,例降為伯。御史中尉李彪以道元執法清刻,自太傅掾引為書侍御史。彪為僕射李沖所奏,道元以屬官坐免。景明中,為冀州鎮東府長
 史。刺史於勁,順皇后父也,西討關中,亦不至州,道元行事三年。為政嚴酷,吏人畏之,姦盜逃于他境。後試守魯陽郡,道元表立黌序,崇勸學教。詔曰:「魯陽本以蠻人,不立大學。今可聽之,以成良守文翁之化。」道元在郡,山蠻伏其威名,不敢為寇。延昌中,為東荊州刺史,威猛為政,如在冀州。蠻人詣闕訟其刻峻,請前刺史寇祖禮。及以遣戍兵七十人送道元還京,二人並坐免官。



 後為河南尹。明帝以沃野、懷朔、薄骨律、武川、撫冥、柔玄、懷荒、禦夷諸鎮並改為州,其郡、縣、戍名,令準古城邑。詔道元持節兼黃門侍郎,馳驛與大都督李崇籌宜置立,裁減去留。
 會諸鎮叛,不果而還。孝昌初,遣將攻揚州,刺史元法僧又於彭城反叛。詔道元持節、兼侍中、攝行臺尚書,節度諸軍,依僕射李平故事。軍至渦陽,敗退。道元追討,多有斬獲。後除御史中尉。



 道元素有嚴猛之稱,權豪始頗憚之。而不能有所糾正,聲望更損。司州牧、汝南王悅嬖近左右丘念,常與臥起。及選州官,多由於念。念常匿悅第,時還其家,道元密訪知,收念付獄。悅啟靈太后,請全念身,有敕赦之。道元遂盡其命,因以劾悅。



 時雍州刺史蕭寶夤反狀稍露,侍中、城陽王徽素忌道元,因諷朝廷,遣為關右大使。寶夤慮道元圖己,遣其行臺郎中郭子帙
 圍道元於陰盤驛亭。亭在岡下,常食岡下之井。既被圍,穿井十餘丈不得水。水盡力屈,賊遂踰墻而入。道元與其弟道闕二子俱被害。道元瞋目叱賊,厲聲而死。寶夤猶遣斂其父子,殯於長安城東。事平,喪還,贈吏部尚書、冀州刺史、安定縣男。



 道元好學,歷覽奇書,撰注《水經》四十卷,《本志》十三篇。又為《七聘》及諸文皆行於世。然兄弟不能篤睦,又多嫌忌,時論薄之。子孝友襲。



 道元第四弟道慎,字善季,涉歷史傳,有幹局。位正平太守,有能名。遷長樂相。卒,贈平州刺史。



 道慎弟道約,字善禮,樸質遲鈍,頗愛琴書。性多造請,好以榮利幹謁,乞丐不已,多為人
 所笑弄。坎壈於世,不免飢寒。晚歷東萊、魯陽二郡太守。為政清靜,吏人安之。



 範弟道峻子惲,字幼和。好學有文才,尤長吏乾。舉秀才,射策高第。歷位尚書外兵郎。行臺長孫承業引為行臺郎。惲頗兼武用,恆以功名自許。每進計於承業,多見納用。以功賞魏昌縣子。惲在軍啟求減身官爵,為父請贈,詔授征虜將軍、安州刺史。惲後與唐州刺史崔元珍固守平陽。爾朱榮稱兵赴闕,惲與元珍不從,為榮行臺郎中樊子鵠陷城,被害。所作文章,頗行於世。撰慕容氏書,不成。



 子懷則,司空長流參軍。



 韓秀,字白武,昌黎人也。祖宰,慕容俊謁者僕射。父景,皇
 始初歸魏,拜宣威將軍、騎都尉。秀歷位尚書郎,賜爵遂昌子。文成稱秀聰敏清辯,才任喉舌,遂命出納王言,並掌機密。行幸遊獵,隨侍左右。獻文即位,轉給事中,參征南慕容白曜軍事。延興中,尚書奏以敦煌一鎮,介遠西北,寇賊路衝,慮或不固,欲移就涼州。群臣會議,僉以為然。秀獨曰:「此蹙國之事,非闢土之宜。愚謂敦煌之立,其來已久,雖鄰彊寇,而兵人素習,循常置戍,足以自全。若徙就姑臧,慮人懷異意,或貪留重遷,情不願徙。脫引寇內侵,深為國患。且捨遠就近,遙防有闕。一旦廢罷,是啟戎心,則夷狄交構,互相來往。關右荒擾,烽警不息,邊役
 煩興,艱難方甚。」乃從秀議。後為平東將軍、青州刺史。卒,子務襲爵。



 務字道世,性端謹,有吏乾。為定州平北長史,頗有受納,為御史中尉李平所劾。付廷尉,會赦免。後除龍驤將軍、郢州刺史。務獻七寶床、象牙席。詔曰:「昔晉武帝焚雉頭裘,朕常嘉之。今務所獻,亦此之流也。奇麗之物,有乖風素,可付其家人。」後以詐表破賊,免官。久之,拜太中大夫,進號左將軍,卒。



 堯暄,字辟邪,上黨長子人也。本名鐘葵,後賜名暄。祖僧賴,道武平中山,與趙郡呂含首來歸國。暄聰了,美容貌。為千人軍將。太武以其恭謹,擢為中散。



 後兼北部尚書。
 于時始立三長,暄為東道十三州使,更比戶籍,賜獨車一乘,廄馬四匹。暄前後從征及出使檢案三十許度,皆有剋己奉公之稱。賞賜衣服、彩絹、奴婢等物,賜爵平陽伯。及改置百官,授太僕卿,轉大司農。卒於平城。孝文為之舉哀,贈相州刺史。初,暄至徐州,見州城樓觀,嫌其華盛,乃令往往毀徹,由是,後更損落。及孝文幸彭城,聞之,曰:「暄猶可追斬。」暄長子洪襲爵。



 洪子傑,字永壽。元象中,開府儀同三司、樂城縣公。



 洪弟遵,位臨洮太守。卒,謚曰思。



 遵弟榮,位員外散騎侍郎。



 子雄,字休武,少驍果,輕財重氣。位燕州刺史、平城縣伯。
 隨爾朱兆與齊神武戰,敗於廣阿,率所部據定州歸神武。其從兄傑為兆滄州刺史,亦遣使降。神武以其兄弟俱有誠款,使傑便為行瀛州事。使雄代傑為瀛州刺史,進爵為公。時禁網疏闊,官司相與聚斂。唯雄義然後取,接下以恩,甚為吏人所懷。



 魏孝武帝入關,雄為大都督。隨高昂破賀拔勝於穰城,仍除豫州刺史。元洪威據潁川叛。叛人趙繼宗殺潁川太守邵招,據樂口,北應洪威。雄討之,繼宗敗走。



 城內因雄之出,據州引西魏。雄復與行臺侯景討平之。



 梁將李洪芝、王當伯襲破平鄉城,雄並禽之。又破梁司州刺史陳慶之,復圍南荊州。東救未
 至,雄陷其城。梁以元慶和為魏王,侵擾南境。雄大破之於南頓。尋與行臺侯景破梁楚城。豫州人上書,更乞雄為刺史,復行豫州事。



 潁州長史賀若統執刺史田迅,據州降西魏。詔雄與廣州刺史趙育、揚州刺史是寶,隨行臺任祥攻之。西魏將怡鋒敗祥等,育、寶各還,據城降敵。雄收散卒,保大梁。周文帝遣其右丞韋孝寬等攻豫州,雄都督程多寶降之。執刺史馮邕,並雄家屬及部下妻子數千口,欲送長安。至樂口,友外兵參軍王恆伽、都督赫連俊等從大梁邀之。斬多寶,收雄家口還大梁。雄別破樂口,禽丞伯,進討縣瓠。復以雄行豫州事。西魏以是
 寶為揚州刺史,據項城,義州刺史韓顯據南頓。雄一日拔其二城,禽顯及長史岳,寶遁走。加驃騎大將軍、儀同三司,仍隨侯景平魯陽,復除豫州刺史。



 雄雖武將,性質寬厚,為政舉其大綱而已。在邊十年,屢有功績。愛人物,多所施與,亦以此稱。興和四年,卒於鄴,贈司徒,謚曰武恭。子師嗣。



 柳崇,字僧生,河東解人也。七世祖軌,晉廷尉卿。崇方雅有器量,身長八尺,美鬚明目,兼有學行。舉秀才,射策高第。解褐太尉主簿,轉尚書右外兵郎中。于時河東、河北二郡爭境。其間有鹽池之饒,虞阪之便,守宰百姓皆恐
 外割;公私朋競,紛囂臺府。孝文乃遣崇檢斷,上下息訟。屬荊、郢新附,南寇窺擾,又詔崇持節與州郡經略,加慰喻。還,遷太子洗馬、本郡中正。



 累遷河中太守。崇初屈郡,郡人張明失馬,疑執十餘人。崇見之,不問賊事,人別借以溫顏,更問其親老存不,農業多少,而微察其辭色。即獲真賊呂穆等二人,餘皆放遣。郡中畏服,境內怗然。卒於官,贈岐州刺史,謚曰穆。崇所製文章,寇亂遺失。長子慶和,性沉靜,不競於時。位給事中、本郡中正,卒。慶和弟楷,字士則。身長八尺,善草書,頗涉文史。位撫軍司馬。



 論曰:屈遵學知機。恆乃局量受委。張蒲、谷渾,文武為
 用,人世仍顯,不亦善乎?公孫表初則一介見知,終以輕薄致戾。軌始受探金之賞,末陷財利之嫌,鮮克有終,固不虛也。張濟使於四方,有延譽之美。李先學術嘉謀,荷遇三世。賈彞早播時譽。秀則不畏強御。竇瑾、李,時曰良幹。瑾以片言疑似,以夙故猜嫌,而嬰合門之戮,良可悲也。韓延之忠於所事,有國士之烈。袁式取遇崔公,以博雅而重。脩之晚著誠款。唐和萬里慕義。寇贊誠信見嘉。酈範智器而達。道元遭命,有銜須之風。韓秀議邊,得馭遠之算。堯暄聰察致位,禮加存沒。構崇素業有資,器行仍世。盛矣乎!



\end{pinyinscope}