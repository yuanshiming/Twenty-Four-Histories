\article{卷九十三列傳第八十一 僭偽附庸}

\begin{pinyinscope}

 夏赫連氏燕慕容氏後秦姚氏北燕馮氏西秦乞伏氏北涼沮渠氏梁蕭氏晉自永嘉之亂,宇縣瓜分,胡羯憑陵,積有年代,各言膺運,咸居大寶。竟而自相吞滅,終為魏臣。然魏自昭成已前,王迹未顯,至如劉石之徒,時代不接,舊書為傳,編之四夷,有欺耳目,無益緗素。且於時五馬浮江,正朔示改,《
 陽秋》記注,具存紀錄。雖朝政叢脞,而年代已多。太宗文皇帝爰動天文,大存刊勒,其時事相接,已編之《載記》。今斷自道武已來所吞併者,序其行事,紀其滅亡。其餘不相關涉,皆所不取。至如晉、宋、齊、梁雖曰偏據,年漸三百,鼎命相承。



 《魏書》命曰《島夷》,列之於傳,亦所不取。故不入今篇,蕭察雖云帝號,附庸周室,故從此編,次為《僭偽附庸傳》云爾。



 鐵弗劉武,南單于苗裔,左賢王去卑之孫,北部帥劉猛之從子,居於新興慮虒之北。北人謂胡父鮮卑母為「鐵弗」,因以號為姓。武父誥汁爰,世領部落。汁爰死,武代
 焉。武死,子務桓代領部落,與魏和通。務桓死,弟閼陋頭代立,密謀反叛。後務桓子悉勿祈遂閼陋頭而立。悉勿祈死,弟衛辰代立。



 衛辰,務桓之第三子也。既立,遣子朝獻,昭成以女妻之。衛辰潛通苻堅,堅以為左賢王。遣使請堅求田地。春去秋來,堅許之。後乃背堅,專心歸魏。舉兵伐堅,堅遣其將鄧羌討擒之。堅自至朔方,以衛辰為夏陽公,統其部落,衛辰復附於堅,昭成討大破之,遂走奔苻堅。堅送還朔方,遣兵戍之。



 昭成末,衛辰導苻堅寇魏南境,王師敗績。堅遂分國人為二部,自河以西,屬之衛辰;自河以東,屬之劉庫仁。堅後以衛辰為單于,督攝
 河西新類,屯於代來。



 慕容永據長子,拜衛辰使持節、都督河西諸軍事、大將軍、朔州牧、朔方王。姚萇亦遣使結好,拜衛辰使持節、都督北朔雜夷諸軍事、大將軍、大單于、河西王、幽州牧。



 登國中,衛辰遣子直力鞮寇南部,其眾八九萬。道武軍五六千人,為其所圍。



 帝乃以車為方營,並戰並前,大破之於鐵岐山南。直力鞮單騎而走。帝乘勝追之,自五原金津南度河,徑入其國。遂至衛辰所居悅跂城,衛辰父子驚遁。乃分遣陳留公元虔南至白鹽池,虜衛辰家屬;將軍伊謂至木根山,擒直力鞮。衛辰單騎遁走,為其部下所殺,傳首行宮。先是河水赤如血,
 衛辰惡之,及衛辰之亡,誅其族類,並投之於河。衛辰第三子屈丐奔薛干部帥太悉伏。



 屈丐,本名勃勃,明元改其名曰屈丐。北方言屈丐者卑下也。太悉伏送之姚興。



 興高平公破多羅沒弈於妻之以女。屈丐身長八尺五寸,興見而奇之。拜驍騎將軍,加奉車都尉,常參軍國大議,寵遇踰於勛舊。興弟濟南公邕言於興曰:「屈丐天性不仁,難以親育,寵之太甚,臣竊惑之。」興曰:「屈丐有濟世之才,吾方收其藝用,興之共平天下,有何不可?」乃以屈丐為安遠將軍,封陽川侯,使助沒弈于鎮高平。邑固諫以為不可。興乃止。以屈丐為持節、安北將軍、五原公,配
 以三交五部鮮卑二萬餘落,鎮朔方。



 道武末,屈丐襲殺沒弈於而并其眾,僭稱大夏天王,號年龍昇,置百官。興乃悔之。屈丐恥姓鐵弗,遂改為赫連氏,自云徽赫與天連。又號其支庶為鐵伐氏,云族剛銳如鐵,皆堪伐人。晉將劉裕攻長安,屈丐聞而喜曰:「姚泓豈能拒裕?裕必克之。待裕去後,吾取之如拾遺耳。」於是秣馬勵兵,休養士卒。及劉裕禽泓,留子義真守長安。屈丐伐之,大破義真,積人頭為京觀,號曰髑髏臺。遂僭皇帝於水霸上,號年為昌武,定都統萬,勒銘城南,頌其功德,以長安為南都。



 性驕虐,視人如草,蒸土以築城,鐵錐刺入一寸,即殺作人
 而並築之。所造兵器,匠呈必死:射甲不入,即斬弓人,如其入,便斬鎧匠,殺工匠數千人。常居城上,置弓劍於側,有所嫌忿,手自殺人。群臣忤視者,鑿其目,笑者決其脣,諫者謂之誹謗,先截其舌,而後斬之。議廢其子璝,璝自長安起兵攻屈丐,丐遣子太原公昌破璝殺之。屈丐以昌為太子。始光二年,屈丐死,昌僭立。



 昌字還國,一名折,屈丐之第二子也。既僭位,改年承光。太武聞屈丐死,諸子相攻,關中大亂,於是西伐。乃以輕騎一萬八千,濟河襲昌。時冬至之日,昌宴饗,王師奄到,上下驚擾。車駕次於黑水,去其城三十餘里,昌乃出戰。太武馳往擊之,昌
 退走入城,未閉門,軍士乘勝入其西宮,焚其西門,夜宿城北。明日分軍四出,徙萬餘家而還。



 後昌遣弟定與司空奚斤相持於長安,太武乘虛西伐,濟君子津,輕騎三萬,倍道兼行。群臣咸諫曰:「統萬城堅,非一日可拔。今輕軍討之,進不可剋,退無所資。不若步軍攻具,一時俱往。」帝曰:「夫用兵之術,攻城最下,不得已而用之。



 如其攻具一時俱往,賊必懼而堅守。若攻不時拔,則食盡兵疲,外無所掠,非上策也。朕以輕騎至其城下,彼先聞有步軍,步從見騎至,必當心閑。朕且羸師以誘之,若得一戰,擒之必矣。所以然者,軍士去家二千里,後有黃河之難,所
 謂置之死地而後生也。以是決戰則有餘,攻城則不足。」遂行,次於黑水,分軍伏於谷,而以少眾至其城下。昌將狄子玉來降,說:使人追其弟定,定曰:「城堅峻未可攻拔,待擒斤等,然後徐往,內外擊之,有何不濟?」昌以為然。太武惡之,退軍城北,示昌以弱,遣永昌王健及娥清等分騎五千,西掠居人。會軍士負罪,亡入昌城,言官軍糧盡,士卒食菜,輜重在後,步兵未至,擊之為便。昌信其言,引眾出城,步騎三萬。司徒長孫翰等咸言昌步陣難陷,宜避其鋒,且待步兵,一時奮擊。帝曰:「不然,遠來求賊,恐其不出。今避而不擊,彼奮我弱,非計也。」遂收軍偽北,引而
 疲之。昌以為退,鼓噪而前,舒陣為翼。行五六里,帝衝之,賊陣不動。稍前行,會有風起,方術官趙倪勸帝更待後日,崔浩叱之。帝乃分騎為左右以掎之。帝墜馬,賊已逼,帝騰馬刺殺其尚書斛黎文,殺騎賊十餘人。流矢中帝,帝奮擊不輟。



 昌軍大潰,不及入城,奔投上邽。遂剋其城。



 初,屈丐奢,好脩宮室,城高十仞,基厚三十步,上廣十步,宮牆五仞,其堅可以礪刀斧。臺榭高大,飛閣相連,皆彫鏤圖畫,被以綺繡,飾以丹青,窮極文采。



 帝顧謂左右曰:「蕞爾小國,而用人如此,雖欲不亡,其可得乎?」



 侍御史安頡禽昌,帝使侍中古弼迎昌至京師,舍之西宮門內,給
 以乘輿之副。



 又詔昌尚始平公主,假會稽公,封為秦王,坐謀反伏誅。



 昌弟定,小字直獖,屈丐之第五子也。凶暴無賴。昌敗,定奔於平涼,自稱尊號,改年勝光。定登陰槃山,望其本國,泣曰:「先帝以朕承大業者,豈有今日之事乎!使天假朕年,當與卿諸人建季興之業。」俄而群狐百數,鳴於其側,定命射之,無所獲。惡之曰:「所見亦大不臧,咄咄天道,復何言哉!」



 定與宋連和,遙分河北。自恆山以東,屬宋;恆山以西,屬定。太武親率輕騎襲平涼。定救平涼,方陣自固。帝四面圍之,斷其水草,定不得水,引眾下原,詔武衛將軍丘眷擊之。定眾潰,被創,單騎遁走,由其
 餘眾,乃西保上邽。神蒨四年,為吐谷渾慕璝所襲,禽定送京師,伏誅。



 徒河慕容廆,字弈洛瑰,本出昌黎。曾祖莫護跋,魏初,率諸部落入居遼西,從司馬宣王討公孫氏,拜率義王,始建王府於棘城之北。祖木延,從毌丘儉征高麗有功,始號左賢王。父涉歸,以勛進拜鮮卑單于,遷邑遼東。涉歸死,廆代領部落。



 以遼東僻遠,遷於徒河之青山。穆帝世,頗為東部之患。廆死,子晃嗣。



 晃字元真,號年為元年,自稱燕王。建國二年,昭成納晃女為后。四年,晃城和龍而都焉。征高麗大破之,遂入丸都,掘高麗王釗父利墓,載其
 尸,焚其宮室,毀丸都而歸。釗後稱臣,乃歸其父屍。晃死,子俊嗣。



 俊字宣英,既襲位,號年為元年。聞石氏亂,乃礪甲嚴兵,將為進取之計,徙都於薊。建國十五年,俊僭稱皇帝,置百官,號年天璽,國稱大燕。十六年,自薊遷都於鄴,號年光壽。俊死,第三子嗣。



 字景茂,號年建熙。政無綱紀。有神降於鄴,曰湘女,有聲,與人相接,數日而去。後苻堅遣將王猛代鄴,禽,封新興侯。道武之七年,苻堅敗於淮南。叔父垂叛堅,攻苻丕於鄴。弟濟北王泓先為北地長史,聞垂攻鄴,亡奔關東,還屯華陰,自稱雍州牧、濟北王;推垂為丞相、大司馬、吳王。堅遣子鉅
 鹿公睿伐泓。泓弟中山王沖,先為平陽太守,亦起兵河東,奔泓。泓眾至十萬,遣使謂堅,求分王天下。堅大怒,責。叩頭流血謝,堅待之如初,命以書招垂及泓、沖。



 密遣使謂泓:「勉建大業,可以吳王為相國;中山王為太宰,領大司馬;汝可為大將軍,領司徒,承制封拜。聽吾死問,汝便即尊位。」泓進向長安,年號燕興。



 泓謀臣高蓋、宿勤崇等以泓德望後沖,且持法苛峻,乃殺泓,立沖為皇太弟,承制行事,置百官。進據阿房。初,堅之滅燕,沖姊清河公主年十四,有殊色,堅納之。



 沖年十二,亦有龍陽之姿,堅又幸之。姊弟專寵。長安歌之曰:「一雌復一雄,雙飛
 入紫宮。」王猛切諫,乃出沖。及其母卒,葬之以燕后之禮。長安又謠曰:「鳳皇,鳳皇,止阿房。」時以鳳皇非梧桐不棲,非竹實不食,乃蒔梧竹數千株於阿城,以待鳳皇。沖小字鳳皇,至是,阿城終為堅賊。入見堅謝,因言二子昨婚,欲堅幸第,堅許之。出,術士王嘉曰:「椎蘆作蘧蒢,不成文章。會天大雨,不得殺羊。」言將殺堅而不果也。堅與群臣莫解。是夜大雨,晨不果出。事發,堅乃誅父子及宗族,城內鮮卑無少長男女皆殺之。



 廆弟運。運孫永,字叔明。既為苻堅所並,永徙於長安。家貧,夫妻常賣靴於市。及為堅所殺,沖乃自稱尊號,以永為小將軍。
 沖毒暴,及堅出如五將山,沖入長安,縱兵大掠,死者不可勝計。初,堅之未亂,關中忽然,無火而煙氣大起,方數十里,月餘不滅。堅每臨聽訟觀,令民有怨者,舉煙於城北,觀而錄之。長安為之詔曰:「欲得必存當舉煙。」關中謠曰:「長鞘馬鞭擊左股,太歲南行當復虜。」



 西人呼徒河為白虜,沖果據長安。樂之忘歸,且以慕容垂名威夙著,跨據山東,憚不敢進,眾咸怨之。登國元年,沖左將軍韓延因人之怨,殺沖,立沖將段隨為燕王,改年昌平。沖之入長安,王嘉謂之曰:「鳳皇,鳳皇,何不高飛還故鄉?無故在此取滅亡。」



 沖敗,其左僕射慕容恆與永潛謀,襲殺隨,立
 宜都王子覬為燕王,號年建明。



 率鮮卑男女三十餘萬口,乘輿服御,禮樂器物,去長安而東。以永為武衛將軍。恆弟護軍將軍韜,陰有貳志,誘覬殺之於臨晉。恆怒,去之。永與武衛將軍刁雲率眾攻韜。韜遣司馬宿勤黎逆戰,永執而戮之。韜懼,出奔恆營。恆立慕容沖子望為帝,改年建平。眾悉去望奔永,永執望殺之,立慕容泓子忠為帝,改年建武。忠以永為太尉,守尚書令,封河東公。東至聞喜,知慕容垂稱尊號,託以農要弗進,築燕熙城以自固。刁雲等又殺忠,推永為大都督、大將軍、大單于、雍秦梁涼四州牧、河東王,稱蕃於垂。



 永進據長子,僭稱帝,
 號年中興。垂攻丁零翟釗於滑臺,釗敗降永。永以釗為車騎大將軍、東郡王。歲餘,謀殺永,永誅之。垂來攻永,永敗,為前驅所獲,垂數而戮之。并斬永公卿已下刁雲、大逸豆歸等四十餘人。永所統新舊人戶、服御、圖書、器樂、珍寶,垂悉獲之。



 垂字道明,晃第五子也。甚見寵愛,常自謂諸弟子曰:「此兒闊達好奇,終能破人家,或能成人家。」故名霸,字道業。恩遇踰於俊。俊弗能平,及即王位,以垂墜馬傷齒,改名為缺,外以慕郤缺為名,內實惡之。尋以讖記之文,乃去夬,以垂為名。年十三,為偏將,所在征伐,勇冠三軍。俊平中原,垂為前鋒,累戰有大功。及俊僭尊
 號,封吳王。



 後以車騎大將軍敗桓溫於枋頭,威名大震,不容於,西奔苻堅。堅甚重之,拜冠軍將軍,封賓都侯。堅敗淮南,入於垂軍。子寶勸垂殺之,垂以堅遇之厚也,不聽。行至洛陽,請求拜墓,堅許之。遂起兵攻苻丕於鄴。垂稱燕王,置百官,年號燕元。



 登國元年,垂僭位,號年為建興。繕宗廟社稷於中山,盡有幽、冀、平州之地,遣使朝貢。三年,道武遣九原公儀使於垂,垂又遣使朝貢。四年,道武遣陳留公虔使於垂,垂又遣使朝貢。五年,又遣秦王觚使於垂,垂留觚不遣,遂絕行人。垂議討慕容水,太史令靳安言於垂曰:「彗星經尾、箕之分,燕當有野死之王。
 不出五年,其國必亡。歲在鶉火,必剋長子。」垂乃止。安出而謂人曰:「此眾既並,終不能久。」安蓋知道武之興也,而不敢言。先是,丁零翟遼叛垂,後遣使謝罪,垂不許。遼怒,遂自號大魏天王,屯滑臺,與垂相擊。死,子釗代之。及垂征剋滑臺,釗奔長子。垂議征長子,諸將咸諫。以永國未有釁,請他年。垂將從之,垂弟司徒、范陽王德固勸垂。垂曰:「司徒議與吾同,且吾投老,叩囊底智足以剋之,不復留逆賊以累子孫。」乃伐永剋之。



 十年,垂遣其太子寶來寇。始寶之來,垂已有疾。自到五原,道武斷其行路,父子問絕。帝乃詭其行人之辭,臨河告之曰:「汝父已死,何不
 遽還?」寶兄弟聞之憂怖,以為信然,於是士卒駭動。初,寶至幽州,其所乘車軸無故自折。占工靳安以為大凶,固勸令還,寶怒,不從。至是,問安。安曰:「速去可免。」寶愈恐。



 安退告人曰:「今將死於他鄉,尸骸委於草野,為烏鳶螻蟻所食,不復見家族。」



 十月,寶燒船夜遁。時河冰未成,寶謂帝不能度,不設斥候。十一月,天暴風寒,冰合,帝進軍濟河急追之。至參合陂西,靳安言於寶曰:「今日西北風動,是軍將至之應,宜兼行速去,不然必危。」其夜,帝部分眾軍,東西為掎角之勢。約勒士卒,束馬口,銜枚無聲。昧爽,眾軍齊進,日出登山,下臨其營。寶眾晨將東引,顧見軍
 至,遂驚擾。帝縱騎騰躡,馬者蹶倒冰上。寶及諸父兄弟,軍馬迸散,僅以身免。寶軍四五萬人,一時放仗,斂手就羈。擒其王公文武數千。垂復欲來寇,太史曰:「太白夕沒西方,數日後見東方,此為躁兵,先舉者亡。」垂不從,鑿山開道,至寶前敗所,見積骸如丘,設祭弔之。死者父兄子弟遂皆皋哭,聲震山川,垂慚忿嘔血,發病而還,死於上谷。寶僭立。



 寶字道裕,垂之第四子也。少輕果,無志操,好人佞己。為太子,砥厲自脩。



 垂妻段氏謂垂曰:「寶姿質雍容,柔而不斷,承平則為仁明之主,處難則非濟世之雄。今託以大業,未見克昌之美。遼西、高陽,兒之俊賢者,
 宜擇一以樹之。趙王驎姦詐負氣,常有輕寶之心,恐難作。」垂不納。寶聞,深以為恨。寶既僭位,年號永康。遣驎逼其母段氏自裁。段氏怒曰:「汝兄弟尚逼殺母,安能保社稷?吾豈惜死!」遂自殺。寶議以后謀廢嫡,稱無母之道,不宜成喪,群臣咸以為然。寶中書令眭邃執意抗言,寶從之而止。



 皇始元年,道武南伐。及剋信都,寶大懼,夜來犯營,帝擊破之。寶走中山,遂奔薊。寶子清河王會先守龍城,聞寶被圍,率眾赴難,逢寶於路。寶分奪其軍,以授弟遼西王農等。會怒,襲農殺之,勒兵攻寶。寶走龍城,會追圍之。侍御郎高雲襲敗會師,會奔中山。寶命雲為子,封
 夕陽公。會至中山,為慕容普鄰所殺。寶至龍城,垂舅蘭汗拒之,寶南走奔薊。汗復遣迎。寶以汗,垂之季舅,子盛又汗之婿也,必謂無二,乃還龍城。汗殺之,及子策等百餘人。汗自稱大都督、大單于、昌黎王,號年青龍。以盛子婿,哀而宥之。



 盛字道運,寶長子也。垂封為長樂公,寶僭立,進爵為王。蘭汗之殺寶也,以盛為侍中、左光祿大夫。盛乃間汗兄弟,使相疑害。李旱、衛雙、劉志、張貞等皆盛之舊暱,汗太子穆並引為腹心。盛結旱等,因汗、穆等醉,夜襲殺之。僭尊號,改年為建平,又號年為長樂。盛改稱庶人大王。盛以寶闇而不斷,遂峻極威刑,於是上下震
 局。前將軍段璣等夜鼓噪攻盛;傷之。遂輦升殿,召叔父河間公熙,屬以後事,熙未至而死。



 熙字道文,小字長生,垂之少子也。群臣與盛伯母丁氏議,以其家多難,宜立長君,遂廢盛子定,迎熙立之。熙立,殺定,年號光始。築龍騰苑,起雲山於苑內。



 又起逍遙宮、甘露殿,連房數百,觀閣相交。鑿天河渠,引水入宮。又為妻苻氏鑿曲光海、清涼池。季夏盛暑,不得休息,暍死者太半。熙遊城南,止大柳樹下,若有人呼曰:「大王且止。」熙惡之,伐其樹,下有蛇長丈餘。熙盡殺寶諸子,改年為建始。又為其妻起承華殿,負土於北門,土與穀同價。典軍杜靜載棺詣闕,上書
 極諫,熙大怒,斬之。熙妻當季夏思凍魚膾,仲冬須生地黃,切責不得,加有司大辟。苻氏死,熙擁其屍僵仆絕息,久而乃蘇,悲號擗踴,斬衰食粥。大斂之後,復啟而交接。制百官哭臨,沙門素服。令有司案檢,有淚者為忠,無淚者罪之,群臣莫不含辛以為淚。及葬,熙被髮徒步,從轜車毀城門而出。長老相謂曰:「慕容氏自毀其門,將不久矣。」衛中將軍馮跋兄弟閉門拒熙,執而殺之。立夕陽公雲為王。



 雲,寶之養子也,復姓高氏,年號正始。跋又殺雲自立。



 雲之立也,熙幽州刺史、上庸公慕容懿以遼西歸降。道武以懿為征東大將軍、平州牧、昌黎王。後坐反伏
 誅。



 晃少子德,字玄明,雅為兄垂所重。苻堅滅,以德為張掖太守。垂僭號,封范陽王,位司徒。寶即位,以德鎮鄴,大丞相。寶既東走,群僚勸德稱尊號,德不從。皇始二年,既拔中山,道武遣衛王儀攻鄴,德南走滑臺,自稱燕王,號年燕元,置百官。德冠軍將軍苻廣叛於乞活壘,德留兄子和守滑臺,率眾攻廣,斬之。而和長史李辯殺和,以城降魏。德無所據,用其尚書潘聰計,據青、齊,入都廣固,僭稱尊號,號年建平。女水竭,德聞而惡之,因而寢疾。兄子超請祈女水,德曰:「人君之命,豈女水所知?」乃以超為太子。德死,超僭立。



 超字祖明,德兄北海王納之子也。既
 僭位,號年太上。超南郊柴燎,焰起而煙不出,靈臺令張光告人曰:「今火盛而煙滅,國其亡乎!」天賜五年,晉將劉裕伐超,超將公孫五樓勸拒之於大峴,不從。裕入大峴,超戰於臨朐,為裕敗。退還廣固,圍之。廣固鬼夜哭,有流星長十餘丈,隕於廣固。城潰,裕執超。送建康市斬之。



 姚萇,字景茂,出於南安赤亭,燒當之後也,祖柯迴,助魏掎姜維於沓中,以功假綏戎校尉、西羌都督。父弋仲,晉永嘉之亂,東徙榆眉。劉曜以弋仲為平西將軍,平襄公。後隨石季龍遷於清河灄頭,勒以弋仲為奮武將軍,封襄平公。弋仲死,子襄代,屯於譙城。慕容俊以襄為豫州
 刺史、丹陽公,屯淮南。自稱大將軍、大單于,為晉將桓溫所敗,奔河東。後為苻眉所殺。



 弋仲有子四十二人,萇第二十四。隨兄襄征伐,襄甚奇之。襄敗,降於苻堅。



 從堅征伐,頻有功。堅伐晉,以萇為龍驤將軍,督益梁州諸軍事,謂萇曰:「朕本以龍驤建業,龍驤之號,初未假人,今特以相授。山南之事,一以委卿。」堅左將軍竇中進曰:「王者無戲言,此亦不臧之徵也,惟陛下察之。」堅默然。及慕容泓起兵華澤,堅遣子衛大將軍睿討之,戰敗,為泓所殺。時萇為睿司馬,懼罪奔馬牧。



 聚眾萬餘,自稱大將軍、大單于、萬年秦王,號年白雀。數月之間,眾至十餘萬。



 與慕
 容沖連和,進屯北地。苻堅出五將山,萇執而殺之。



 登國元年,僭稱皇帝,置百官,國號大秦,年曰建初。改長安曰常安,以其太子興鎮之。自擊苻登於安定,敗之。萇病,夢苻堅將天官使者、鬼兵數百,突入營中。萇懼,走後宮,宮人迎萇刺鬼,誤中萇陰。鬼相謂曰:「正中死處。」拔矛,出血石餘。寤而驚悸,遂患陰腫,刺之,出血如夢。萇乃狂言,或稱萇,「殺陛下者臣兄襄,非臣之罪,願不枉法。」萇死,子興襲位,秘不發喪。



 興字子略,萇長子也。既滅苻登,然後發喪行服。僭稱皇帝,年號皇初。天興元年,興去皇帝之號,降稱天王,號年洪始。興剋洛陽,以其弟東平公紹鎮之。
 三年,興遣使來聘,道武遣謁者僕射張濟使於興。天興五年夏,興遣其弟義陽公平率眾四萬侵平陽,攻乾壁六十餘日,陷之。七月,車駕親征。八月,次永安,平募遣勇將率精騎二百窺軍,為前鋒將長孫肥所禽,匹馬不反。平遂退走。帝急追,及於柴壁,圍之。興乃悉舉其眾,救平。帝增築重圍,內以防平之出,外以距興之入。



 又截汾曲為南北浮橋,乘西岸築圍。帝帥師度蒙坑南四十里,逆擊興。興晨行北引,未及安營,大軍卒至,興眾怖憂。帝知興氣挫,乃南絕蒙坑之口,東杜新阪之隘,守天度,屯賈山,令平水陸路絕,將坐甲而禽之。又令緣汾帶罔樹柵,
 以衛芻牧者。



 九月,興從汾西北下,憑壑為壘以自固。興又將數千騎乘西橋。官軍鉤取,以為薪蒸。興還壘,道武度其必攻西圍,乃命脩塹,增廣之。至夜,興果來攻,梯短不及,棄之塹中而還。興又分其眾,臨汾為壘,叩逼水門,與平相望。帝因截水中,興內外隔絕,士眾喪氣。於是平糧盡,窘急,夜悉眾將突西南而出。興列兵汾西,舉烽鼓噪,為平接援。帝簡諸軍精銳,屯汾西固守,南絕水口。興夜聞聲,望平力戰突免;平聞外鼓,望興攻圍引接。故但叫呼,虛相應和,莫敢逼圍。平不得出,窮逼,乃將二妾赴水死。興安遠將軍不蒙世、揚武將軍雷重等將士四千
 餘人隨平投水,帝令泅水鉤捕,無得免者。平眾三千餘人,皆斂手受執。擒興尚書右僕射狄伯支已下四十餘人。興遠來救,自觀其窮,力不能免,舉軍悲號,震動山谷,數日不止。頻遣使請和,帝不許,乃班師。興還長安。有雀數萬頭斗於興廟,毛羽折落,多有死者,月餘乃止。識者曰:「今雀鬥廟上,子孫當有爭亂者乎?」又興殿有聲如牛句。有二狐入長安,一登興殿屋,走入宮,一入市,求之不得。永興三年,興遣周寶來聘。五年,興遣使來聘,並請進女,明元許之。神瑞元年,興遣兼散騎常侍,尚書吏部郎嚴康來聘。二年,興遣散騎常侍、東武侯姚敞、尚書姚軌
 奉其西平公主於明元,明元以后禮納之。



 泰常元年,興死。長子泓,字元子,僭位,號年永和。晉將劉裕伐泓,長驅入關。泓戰敗請降,裕執之,於建康斬之。



 馮跋,字文起,小名乞直代,本出長樂信都。慕容永僭號長子,以跋父安為將。



 永為垂所滅,安東徙昌黎,家於長谷,遂同夷俗。



 跋飲酒至一石不亂,諸弟皆不修行業,唯跋恭慎。慕容熙僭號,以跋為殿中左監,稍遷衛中將軍。後坐事逃亡。既而熙政殘虐,人不堪命。跋乃與從兄萬泥等二十二人結謀,跋與二弟乘車,使婦人御,潛入龍城,匿於孫護之室,以誅熙。乃立夕陽公高雲為主。雲以
 跋為侍中、征北大將軍、開府儀同三司,封武邑公。事皆決跋兄弟。明元初,雲為左右所殺,跋乃自立為燕王,置百官,號年太平。於時永興元年也。跋撫納契丹等,諸落頗來附之。明元遣謁者於什門喻之,為跋所留。泰常三年,和龍城有赤氣蔽日,自寅至申。跋太史令張穆以為兵氣,勸跋還魏使,奉修職貢,跋不從。明元詔征東大將軍長孫道生討之,跋嬰城固守,道生不剋而還。



 神蒨二年,跋有疾,其長子永先死,立次子翼為世子,攝國事,勒兵以備非常。



 跋妾宋氏規立其子受居,深忌翼,謂之曰:「主上疾將瘳,奈何代父臨國乎!」翼遂還。樣氏矯絕內外,
 遣閽人傳問。翼及跋諸子、大臣並不得省疾,唯中給事胡福獨得出入,專掌禁衛。跋疾甚,福慮宋氏將成其計,乃言於跋弟弘。勒兵而入,跋驚怖而死。弘襲位,翼勒兵出戰不利,遂死。跋有子男百餘人,悉為弘所殺。



 弘字文通,跋之少弟也。跋立,為尚書右僕射,封中山公,領中領軍,內掌禁衛,外總朝政。歷位司徒。及自立,乃與宋氏通和。延和元年,太武親討之,弘嬰城固守。其營丘、遼東、成周、樂浪、帶方、玄菟六郡皆降,太武徙其人三萬餘家於幽州。其尚書郭深勸之歸誠進女,乞為附庸,保守宗廟。弘曰:「負釁在前,忿形已露,附降取死。不如守志,更圖所
 適也。」先是,弘廢其元妻王氏,黜世子崇,令鎮肥如,以後妻慕容氏子曰王仁為世子。崇母弟廣平公朗、樂陵公邈相謂曰:「禍將至矣!」於是遂出奔遼西,勸崇來降,崇納之。會太武使給事中王德陳示成敗,崇遣邈入朝。太武封崇遼西王,錄其國尚書事,遼西十郡,承制假授文官尚書、刺史,武官征虜已下。弘遣其將封羽率眾圍崇,太武詔永昌王健督諸軍救之。封羽又以凡城降,徙其人三千餘家而還。弘遣其尚書高顒請罪,乞以季女充掖庭。帝許之,徵其子王仁入朝,弘不遣。其散騎常侍劉訓諫,弘大怒,殺之。太武又詔樂平王丕等計之。日就蹙削,
 上下危懼。弘太常陽昬復勸弘請罪乞降,令王仁入侍。



 弘不聽,乃密求迎於高麗。太延二年,高麗遣將葛居盧等率眾迎之,弘乃擁其城內士女入於高麗。先是,其國有狼夜繞城群嗥。如是終歲。又有鼠集於城西,闐滿數里,西行,至水則在前者銜馬矢,迭相齧尾而度。宿軍地燃,一旬而滅,觸地生蛆,月餘乃止。和龍城生白毛,一尺二寸。



 弘至遼東,高麗遣使勞之曰:「龍城王馮君,爰適野次,士馬勞乎?」弘慚怒,稱制答讓之。高麗乃處之於平郭,尋徙北豐。弘素侮高麗,政刑賞罰,猶如其國。



 高麗乃奪其侍人,質任王仁。弘忿怨之,謀將用奔。太武又徵弘於
 高麗。乃殺之於北豐,子孫同時死者十餘人。弘子朗、邈。朗子熙,在《外戚傳》。



 乞伏國仁,隴西人也。其先如弗,自漠北南出。五世祖佑鄰,并兼諸部,眾漸盛。父司繁,擁部落降苻堅,堅以為南單于,又拜鎮西將軍,鎮勇士川。司繁死,國仁為將軍。及堅敗,國仁叔步頹叛於隴石。堅令國仁討之,步頹大悅,迎而推之,部眾十餘萬。道武時,私署大都督、大將軍、大單于、秦河二州牧,號年建義,署置官屬。分部內為十一郡,築勇士城以都之。



 國仁死,弟乾歸統事,自署大都督、大將軍、大單于、河南王,改年為太初,置百官。登國中,遷
 於金城。城門自壞,乾歸惡之,遷於苑川。尋為姚興所破,又奔枹罕,遂降姚興。拜為河州刺史,封歸義侯。尋遣還苑川。乾歸乃背姚興,私稱秦王,置百官,號年更始。遣使請援,明元許之。田于五溪,有梟集其手,尋為其兄子公府所殺。



 子熾盤殺公府,代統任。熾盤自稱大將軍、河南王,改年為永康。後襲禿髮傉檀於樂都,滅之,乃私署秦王,置百官,改年為建弘。後遣其尚書郎莫者胡、積射將軍乞伏又寅貢金二百斤,請伐赫連昌,太武許之。及統萬事平,熾盤乃遣其叔平遠將軍泥頭、弟安遠將軍安度質於京師。又使其中書侍郎王愷、丞相從事中郎烏
 訥闐奉表貢其方物。熾盤死,子慕末統任。



 慕末字安石跋。既立,改年為永弘。其尚書隴西辛進嘗隨熾盤遊後園,進彈鳥,丸誤傷慕末母面。至是,誅進五族二十七人。慕末弟殊羅蒸熾盤左夫人禿髮氏,慕末知而禁之,殊羅與叔父什夤謀殺慕末,使禿髮氏盜門鑰。鑰誤,門不開。門者以告,慕末收其黨,盡殺之。欲鞭什夤,什夤曰:「我負汝死,不負汝鞭。」慕末怒,刳其腹,投屍於河。什夤母弟白養及去列,頗有怒言,又殺之。政刑酷濫,內外崩離,部人多叛。



 後為赫連定所逼,遣王愷、烏訥闐請迎於太武。太武許以安定以西,平涼以東封之。慕末乃焚城邑,毀
 寶器,率戶萬五千至高田谷。為赫連定所拒,遂保南安。



 太武遣師迎之,慕末衛將軍吉毗固諫,以為不宜內徙,慕末從之。赫連定遣其北平公韋代率眾萬人攻南安。城內大饑,人相食。神蒨四年,慕末及宗族五百餘人出降,送於上邽,遂為定滅。



 大沮渠蒙遜,本張掖臨松盧水人也,匈奴有左沮渠官,蒙遜之先為此職,羌之酋豪曰大,故以官為氏,以大冠之。世居盧水為酋豪。遜高祖暉仲歸、曾祖遮,皆雄健有勇名。祖祁復延,封伏地王。父法弘,襲爵。苻氏以為中田護軍。



 蒙遜代父領部曲,有勇略,多計數,頗曉天文,為諸
 胡所推服,呂光自王於涼土,使蒙遜自領營人,配箱直。又以蒙遜叔父羅仇為西平太守。後遣其子慕率羅仇伐乞伏乾歸於枹罕,為乾歸所敗,殺之。蒙遜求還葬羅仇,因聚眾屯金山,與從兄晉昌太守男成共推建康太守段業為使持節、大都督、龍驤大將軍、涼州牧、建康公,稱神璽元年。業以蒙遜為張掖太守,封臨池侯,男成為輔國將軍,委以軍國之任。



 業又自稱涼王,以蒙遜為尚書左丞。忌蒙遜威名,微疏遠之。天興四年,蒙遜內不自安,請為西安太守。蒙遜欲激怒其眾,乃密誣告男成叛逆,業殺之。蒙遜泣而告眾,陳欲復讎之意。男成素有恩
 信,眾情怨憤,泣而從之。蒙遜因舉兵攻殺業,私署使持節、大都督、大將軍、涼州牧、張掖公,年號永安。居張掖。是月,涼武昭王亦起兵,年號庚子。



 永興中,蒙遜剋姑臧,遷居之。改號玄始元年,自稱河西王,置百官。頻遣使朝貢。蒙遜寢於新臺,閹人王懷祖斫蒙遜,傷足,蒙遜妻孟氏禽懷祖斬之。及聞晉滅姚泓,怒甚。有校郎言事於蒙遜,蒙遜曰:「汝聞劉裕入關,敢研研然也!」遂殺之。尋稱籓於晉。泰常中,蒙遜剋敦煌,改年承玄。後又稱蕃於宋,并求書。宋文帝並給之。蒙遜又就宋司徒王弘求《搜神記》,弘與之。



 神蒨中,遣尚書郎宗舒、左常侍高猛朝貢,上表稱
 臣。前後貢使相望。後遣子安周內侍。太武遣兼太常李順持節拜蒙遜為假節,加侍中、都督涼州西域羌戎諸軍事、太傅、行征西大將軍、涼州牧、涼王。使崔浩為冊書以褒賞之。蒙遜又改義和元年。延和二年四月,蒙遜死,詔遣使監護喪事,私謚武宣王。蒙遜性淫忌,忍於刑戮,閨庭之中,略無風禮。



 第三子牧犍統任,自稱河西王,遣使請朝命。并遣使通宋,受宋褒授。先是,太武遣李順迎蒙遜女為夫人,會蒙遜死,牧犍受蒙遜遺意,送妹於京師,並為右昭儀。改稱承和元年。太武又遣李順拜牧犍為使持節、侍中、都督涼沙河三州西域羌戎諸軍事、車
 騎將軍、開府儀同三司、領護西戎校尉、涼州刺史、河西王。牧犍以無功受賞,乃留順,上表乞安、平一號,優詔不許。牧犍尚太武妹武威公主,遣其相宋繇表謝,獻馬五百匹,黃金百斤。繇又表請公主及牧犍母妃后定號。朝議謂禮母以子貴,妻從夫爵,牧犍母宜稱河西國太后;公主於國內可稱王后,於京師則稱公主。詔從之。牧犍遣建節將軍沮渠旁周朝京師,太武遣侍中古弼、尚書李順賜其侍臣衣服有差,并徵世子封壇入侍。牧犍乃遣封壇朝京師。



 太延五年,太武遣尚書賀羅使涼州,且觀虛實。帝以牧犍雖稱籓致貢,而內多乖悖,於是親征
 之。詔公卿為書讓之,數其罪十二。官軍濟河,牧犍曰:「何故爾也?」用其左丞姚定國計,不肯出迎,求救於蠕蠕。遣大將董來萬餘人拒軍於城南,戰退。車駕至姑臧,遣使喻牧犍令出。牧犍聞蠕蠕內侵善無,幸車駕返旆,遂嬰城自守。牧犍兄子祖踰城出降,具知其情。太武乃引諸軍進攻,牧犍兄子萬年率麾下又來降。城拔,牧犍與左右文武,面縛請罪,詔釋其縛。徙涼州人三萬餘家於京師。



 初,太延中,有一老父投書於敦煌城東門,忽然不見。其書紙八字,文曰:「涼王三十年,若七年。」又於震電所得石,丹書曰:「河西,河西,三十年。破帶石,樂七年。」帶石青山
 名,在姑臧南。山祀傍泥陷不通,牧犍征南大將軍董來曰:「祀豈有知乎!」遂毀祀伐木,通道而行。牧犍立,果七年而滅。初,牧犍淫嫂李氏,兄弟三人傳嬖之。李與牧犍姊共毒公主,上遣醫乘傳救公主,得愈。上徵李氏,牧犍不遣,厚送居於酒泉。上大怒,既剋,猶以妹婿待之。其母死,以王太妃禮葬焉。又為蒙遜置守冢三十家,授牧犍征西大將軍,王如故。初,官軍未入之間,牧犍使人斫開府庫,取金銀珠玉及珍奇器物,不更封閉,百姓因之入盜,巨細蕩盡。



 有司求賊不得。真君八年,其所親人及守藏者告之,乃窮竟其事,搜其家中,番得所藏器物。又告牧
 犍父子多畜毒藥,前後隱竊殺人,乃有百數,姊妹皆為左道,朋行淫佚,曾無愧顏。始罽賓沙門曰曇無讖,東入鄯善,自云能使鬼療病,令婦人多子。與鄯善王妹曼頭陀林淫通,發覺,亡奔涼州。蒙遜寵之,號曰聖人。曇無讖以男女交接術教授婦女,蒙遜諸女、子婦,皆往受法。太武聞諸行人言曇無讖術,乃召之。蒙遜不遣,遂發露其事,拷訊殺之。至此,帝知之,於是賜昭儀沮渠氏死,誅其宗族。唯萬年及祖以前先降,得免。是年,人又告牧犍猶與故臣交通謀反,詔司徒崔浩就公主第賜牧犍死。與主決良久,乃自裁。葬以王禮,謚曰哀王。及公主薨,詔與
 牧犍合葬。公主無男,有女,以國甥得襲母爵為武威公主。



 蒙遜子季義,位東雍州刺史。真君中,與河東薛安都謀逆,召至京師,付其兄弟扼殺之。萬年、祖並以先降,萬年拜張掖王,祖廣武公。後坐謀逆,俱死。



 初,牧犍之敗,弟樂都太守安周南奔吐谷渾,太武遣鎮南將軍奚眷討之。牧犍弟酒泉太守無諱奔晉昌,乃使弋陽公元潔守酒泉。真君初,無諱圍酒泉,陷之。又圍張掖,不能剋,退保臨松。太武不伐,詔諭之。時永昌王健鎮涼州,無諱使其中尉梁偉詣健,求奉酒泉。又送潔及統帥兵士於健軍。二年,太武遣使拜無諱為征西大將軍、涼州牧、酒泉王。
 尋以無諱復規叛,遣南陽公奚眷討酒泉,剋之。無諱遂謀度流沙,遣安周西擊鄯善,鄯善欲降,會魏使者勸令拒守,安周不能剋,退保東城。三年春,鄯善王比龍西奔且末,其世子乃從安周。鄯善大亂。無諱遂度流沙,士卒渴死者太半,仍據鄯善。先是高昌太守闞爽為李寶舅唐契所攻,聞無諱至鄯善,使詐降,欲令無諱與唐契相擊。無諱留安周住鄯善,從焉耆東北趣高昌。會蠕蠕殺唐契,爽拒無諱。無諱將衛興奴遂屠其城。爽奔蠕蠕,無諱因留高昌。五年夏,無諱病死,安周立,為蠕蠕所并。



 梁帝蕭詧,字理孫,蘭陵人,武帝之孫,昭明太子統之第
 三子也。幼好學,善屬文,尤長佛義,特為梁武嘉賞。梁普通中,封曲江縣公。及昭明太子薨,封詧岳陽郡王,位東揚州刺史,領會稽太守。初,昭明卒,梁武捨詧兄弟而立簡文,內常愧之,故寵亞諸子。以會稽人物殷阜,一都之會,故有此授,以慰其心,詧既以其昆季不得為嗣,常懷不平。又以梁武衰老,朝多秕政,有敗亡之漸。遂蓄聚貨財,交通賓客,招募輕俠,折節下之。其勇敢者,多歸附焉。左右遂至數千人,皆厚加資給。中大同元年,除西中郎將、雍州刺史,都督五州諸軍事,寧蠻校尉。詧以襄陽形勝之地,又梁武創基之所,時平足以樹根本,時亂足以圖
 霸功,遂務修刑政。



 太清二年,梁武以詧兄河東王譽為湘州刺史,徙湘州刺史張纘為雍州。纘恃才輕譽,州府迎候有闕,譽深銜之,遂託疾不與相見。後聞侯景作亂,頗陵蹙纘。纘構譽及詧於梁元帝,元帝令其世子方等及王僧辯相繼攻譽。譽告於詧,詧聞之大怒。



 及梁元將援建業,令所督諸州並發兵赴都,詧遣府司馬劉方貴領兵為前軍,出漢口。



 及將發,梁元又使諮議參軍劉玨召詧自行,詧不從。而方貴潛與梁元相知,剋期襲詧。未及發,會詧以他事召方貴,謀泄,遂據樊城拒命。詧遣軍攻之。梁元乃厚資遣張纘,若將述職,而密援方貴。纘次
 大隄,而樊城已陷。詧擒方貴兄弟黨與,並斬之。詧時以譽危急,乃留諮議參軍蔡大寶守襄陽,率眾伐江陵以救之。梁元大懼,乃遣參軍庾奐謂詧曰:「以姪伐叔,逆順安在?」詧曰:「家兄無罪,屢被攻圍,七父若顧先恩,豈應若是?如能退兵湘水,吾便旋旆襄陽。」時攻柵不剋,會大雨暴至,平地四尺,眾頗離心。軍主杜岸、岸弟幼安及其兄子龕,以其屬降於江陵。



 詧夜遁歸襄陽,器械輜重多沒於湕水。詧恐不能自固,乃遣蔡大寶求附庸於西魏。



 時西魏大統十五年也。周文令丞相東閣祭酒榮權使焉。


是歲,梁元令柳仲禮圖襄陽,詧乃遣妃王氏及世子
 \gezhu{
  山尞}
 為
 質請救。周文令榮權報命,仍遣開府楊忠為援。十六年,忠擒仲禮,平漢東。西魏命詧發喪嗣位,使假散騎常侍鄭孝穆及榮權策命詧為梁王。乃於襄陽置百官,承制封拜。十七年,留尚書僕射蔡大寶守雍部而朝於京師。周文謂曰:「王之來此,頗由榮權。」乃召權見,曰:「權吉士也,寡人與之從事,未嘗見失信。」詧曰:「榮常道二國之言無私,故詧今者得歸誠魏闕耳。」



 魏恭帝元年,周文命柱國于謹伐江陵,詧以兵會之,及江陵平,周文命主梁嗣,居江陵東城,資以江陵一州之地。其襄陽所統,盡入於周。詧乃稱皇帝於其國,年號大定。追尊其父統為昭明
 皇帝,廟號高宗;統妃蔡氏為昭德皇后。又尊其所生母龔氏皇太后。立妻王氏為皇后,子巋為皇太子。其慶賞刑威,官方制度,並同王者。



 唯上疏則稱臣,奉朝廷正朔。至於爵命其下,亦依梁氏之舊。其戎章勛級,則又兼用柱國等官。又追贈叔父邵陵王綸太宰,謚曰壯武。贈兄河東王譽丞相,謚曰武桓。



 周文仍置江陵防主,統兵居於西城,外云助詧備禦,內實防詧。



 初,江陵滅,梁元將王琳據湘州,志圖匡復。及詧立,琳乃遣其將潘純陀、侯方兒來寇。詧禦之,純陀等退歸夏口。詧之四年,詧遣其大將軍王操略取王琳之長沙、武陵、南平
 等郡。五年,王琳又遣其將雷文柔襲陷監利郡,太守蔡大有死之。



 尋而琳與陳人相持,稱蕃乞師於詧,詧許之。師未出而琳軍敗,附於齊。是歲,其太子巋來朝京師。六年四月,大雨震,前殿崩,壓二百餘人。七年冬,有鵩鳥鳴於寢殿。八年二月,詧終於前殿,時年四十四。是歲,周保定二年也。八月,葬於平陵,謚曰宣皇帝,廟號中宗。



 詧少有大志,不拘小節,雖多猜忌,而知人善任使,撫將士有恩,能得其死力。



 性不飲酒,安於儉素。事母以孝聞。又不好聲色,尤惡見婦人,雖相去數步,亦云遙聞其臭。經御婦人之衣,更不著,並皆棄之。一幸姬媵,病臥累旬。又惡見
 人髮,白事者,必方便避之,擔輿者,冬月必須裹頭,夏日則加蓮葉帽。其在東揚州,頗放誕,省覽簿領,好為戲弄之言,以此獲譏於世。及江陵平,宿將尹德毅謂詧曰:「臣聞人主之行,與匹夫不同。匹夫者,飾小行,競小廉,以取名譽;人主者,定天下,安社稷,以成大功。今魏虜貪婪,罔顧弔伐之義,俘囚士庶,並充軍實。然此等戚屬,咸在江東。悠悠之人,可門到戶說?既塗炭至此,咸謂殿下為之。殿下既殺人父兄,孤人子弟,人盡讎也,又誰與為國?但魏之精銳,盡萃於此,犒師之禮,非無故事。若殿下為設享會,固請於謹等為歡,彼無我虞,當相率而至,預伏
 武士,因而斃之。江陵百姓,撫而安之,文武官僚,隨即銓授。魏人懾息,未敢送死;僧辯之徒,折簡可致。然後朝服濟江,入踐皇極,纘堯復禹,萬世一時。」詧謂德毅曰:「卿此策非不善也,然魏人待我甚厚,未可背德。若遽為卿計,則鄧祁侯所謂人將不食吾餘。」既而闔城長幼,被虜入關,又失襄陽之地。詧恨,乃曰:「不用德毅之言,以至於是!」又見邑居殘毀,干戈日用,恥其威略不振,常懷憂憤,乃著《愍時賦》,以見志焉。居常怏怏,每誦「老馬伏櫪,志在千里。烈士暮年,壯心不已」,未嘗不盱衡扼腕歎吒者久之。遂以憂憤發背而死。



 詧篤好文義,所著文集十五卷,內典《
 華嚴》、《般若》、《法華》、《金光明義疏》三十六卷,並行於世。武帝又命其太子巋嗣位,年號天保。



 巋字仁遠,詧之第三子也。機辯有文學,善於撫御,能得其下歡心。嗣位之元年,尊其祖母龔太后曰太皇太后,嫡母王皇后曰皇太后,所生曹貴嬪曰皇太妃,其年五月,其太皇太后薨,謚曰元太后。九月,其太妃又薨,謚曰孝皇太妃。二年,其皇太后薨,謚曰宣靜皇后。



 五年,陳湘州刺史華皎、巴州刺史戴僧朔並來附。皎送其子玄響為質於巋,仍請兵伐陳。巋上言其狀。武帝詔衛公直督荊州總管權景宣、大將軍元定等赴之。巋亦遣其柱國
 王操率水軍二萬,會皎於巴陵。既而與陳將吳明徹等戰於沌口,直軍不利,元定遂沒,巋大將軍李廣等亦為陳人所虜,長沙、巴陵並陷於陳。衛公直乃歸罪於巋之柱國殷亮。巋雖以退敗不獨罪亮,然不敢違命,遂誅之。吳明徹乘勝攻剋巋河東郡,獲其守將許孝敬。明年,明徹進寇江陵,引江水灌城。巋出頓紀南,以避其銳。江陵副總管高琳與其尚書僕射王操拒守。巋馬軍主馬武、吉徹等擊明徹,明徹退保公安,巋乃還江陵。巋之八年,陳又遣其司空章昭達來寇,江陵總管陸騰及巋之將士擊走之。昭達又寇竟陵之青泥,巋令其大將軍許世武赴
 援,大為昭達所破。



 初,華皎、戴僧朔從衛公直與陳人戰敗,率其麾下數百人歸於巋。巋以皎為司空,封江夏郡公;僧朔為車騎將軍,封吳興縣侯。巋之十年,皎將來朝,至襄陽,請衛公直曰:「梁主既失江南諸郡,人少國貧,朝廷興亡繼絕,理宜資贍。豈使齊桓、楚莊獨擅救衛復陳之美?望借數州,以裨梁國。」直然之,乃遣使言狀。帝許之,詔以基、平、鄀三州歸之於巋。



 及平齊,巋朝於鄴,帝雖以禮接之,然未之重也。巋知之,後因宴承間,乃陳其父荷周文拯救之恩,並敘二國艱虞,脣齒掎角之事。辭理辯暢,因涕泣交流,帝亦為之噓欷。自是大加賞異,禮遇日隆。後
 帝復與之宴,齊氏故臣叱列長叉亦預焉,帝指謂巋曰:「是登陴罵朕者也。」巋曰:「長叉未能輔桀,翻敢吠堯!」帝大笑。



 及酒酣,帝又命琵琶自彈之,仍謂巋曰:「當為梁主盡歡。」巋乃起請舞,帝曰:「王乃能為朕舞乎?」巋曰:「陛下既親撫五絃,臣何敢不同百獸?」帝大悅,賜雜繒萬段、良馬數十疋,并賜齊後主妓妾,及帝所乘五百里駿馬以遣之。及隋文帝執政,尉遲迥、王謙、司馬消難等各起兵。時巋將帥皆密請興師,與迥等為連衡之勢,進可以盡節於周氏,退可以席卷山南。巋以為不可。俄而消難奔陳,迥等相次破滅。隋文帝既踐極,恩禮彌厚,遣使賜金五百
 兩、銀千兩、布帛萬疋、馬五百疋。



 開皇二年,隋文帝備禮納巋女為晉王妃,又欲以其子瑒尚蘭陵公主,由是罷江陵總管,巋專制其國。四年,來朝長安。帝甚敬待之,詔巋位在王公之上。巋被服端麗,進退閑雅,天子矚目,百僚傾慕。帝賜巋縑萬疋,珍玩稱是。及還,親執其手謂之曰:「梁主久滯荊楚,未復舊都,朕當振旅長江,相送旋反。」巋拜謝而歸。五年五月,寢疾薨。臨終上表奉辭,并獻所服金裝劍,帝覽而嗟悼。巋在位二十三年。



 梁之臣子,葬之顯陵,謚曰孝明皇帝,廟號世宗。



 巋孝悌慈仁,有君人之量。四時祭享,未嘗不悲慕流涕。性尤儉約,御下有方,
 境內安之。所著文集及《孝經》、《周易義記》及《大小乘幽微》,並行於世。文帝又命其太子琮嗣位。



 琮字溫文,性俶儻不羈,博學有文義。兼善弓馬,遣人伏地持帖,琮奔馬射之,十發十中,持帖者亦不懼。初封東陽王,尋立為梁太子。及嗣位,帝賜以璽書,敦勉之。又賜梁之大臣璽書,誡勉之。時琮年號廣運,有識者曰:「運之為字,軍走也,吾君將奔走乎!」其年,琮遣大將軍戚昕以舟師襲陳公安,不剋而還。文帝徵琮叔父岑入朝,拜大將軍,封懷義公,因留不遣。復置江陵總管以監之。琮所署大將軍許世武密以城召陳將宜黃侯陳紀,謀泄,琮
 誅之。後二歲,上征琮入朝,率臣下二百餘人朝京師。江陵父老莫不殞涕曰:「吾君其不反矣!」上以琮來朝,遣武鄉公崔弘度將兵戍之。軍至鄀州,琮叔父巖及弟瓛等懼弘度掩襲之,遂引陳人至城下,虜居人而叛。於是廢梁國。上遣左僕射高熲安集之,曲赦江陵死罪,給復十年。



 梁二主各給守墓十戶,拜琮柱國,賜爵莒國公。



 自詧初即位,歲在乙亥,至是,歲在丁未,凡三十三載而亡。



 琮至煬帝嗣位,甚見親重,拜內史令,改封梁公。琮之宗族,緦麻以上,並隨才擢用,於是諸蕭昆弟,布列朝廷。琮性淡雅,不以職務自嬰,退朝縱酒而已。內史令楊約與琮同
 列,帝令約宣旨誡勵。約復以私情諭之,琮曰:「琮若復事事,則何異公哉?」約笑而退。約兄素時為尚書令,見琮嫁從父妹於鉗耳氏,謂曰:「公帝王之族,何乃適妹鉗耳氏?」琮曰:「前已嫁妹於侯莫陳氏,此復何疑?」素曰:「鉗耳,羌也,侯莫陳,虜也。何得相比?」琮曰:「以羌異虜,未之前聞。」素慚而止。琮雖羈旅,見北間豪貴,無所降下。常與賀若弼深友,弼既誅,復有童謠曰「蕭蕭亦復起」,帝由是忌之,遂廢於家。卒,贈左光祿大夫。



 子鉉,位襄城通守。復以琮弟子鉅為梁公。鉅小名曰藏,煬帝甚暱之,以為千牛。與宇文皛出入宮掖,伺詧內外。帝每有遊宴,鉅未嘗不從。遂於宮
 中,多行淫穢。江都之變,為宇文化及所殺。



 察子,追謚孝惠太子;巖,封安平王;岌,封東平王;岑,封河間王,後改封吳郡王。琮弟瓛,義興王;瑑,晉陵王;璟,臨海王;珣,南海王;瑒,義安王,瑀,新安王。



 察之居帝位,以蔡大寶為股肱,王操為腹心,魏益德、尹正、薛暉、許孝敬、薛宣為爪牙,甄玄成、劉盈、岑善方、傅淮、褚珪、蔡大業典眾務,張綰以舊齒處顯位,沈重以儒學蒙厚禮。自餘多所獎拔,咸盡其器能。及巋纂業,親賢並用。將相則華皎、殷亮、劉忠義,宗室則蕭欣、蕭翼,人望則蕭確、謝溫、柳洋、王湜、徐岳,外戚則王洋、王誦、殷璉,文章則劉孝勝、范迪、沈君游、君公、柳信言,政事則
 袁敞、柳莊、蔡延壽、甄詡、皇甫茲。故能保其疆土而和其人焉。今載詧子等及蔡大寶以下尤著者,附於左。其在梁、陳、隋已有傳,及巋諸子未任職者,則不兼錄。



 字道遠,詧之長子也。母曰宣靜皇后。詧之為梁王,立為世子。尋病卒。



 及詧稱帝,追謚焉。



 巖字義遠,詧第五子也。性仁厚,善撫接,歷尚書令、太尉、太傅。入陳,授東揚州刺史。及陳亡,百姓推巖為主。為總管宇文述所破,伏法於長安。



 岌,詧第六子也。性淳和,位至侍中、中衛將軍。巋之五年,卒。贈司空,謚曰孝。



 岑字智遠,詧第八子也。位至太尉。性簡貴,御下嚴整。及琮嗣位,自以望重屬尊,頗有不法。故隋文徵入朝,拜大將軍,封懷義郡公。



 瓛字欽文,巋第三子也。幼有令譽,能屬文。位荊州刺史,頗有能名。崔弘度兵至鄀州,瓛懼,與其叔父巖奔陳。陳主以為侍中、吳州刺史,甚得物情。三吳父老皆曰:「吾君之子。」陳亡,吳人推之為主。吳人見梁武、簡文及詧、巋等兄弟中並第三,而踐尊位。瓛自以巋第三子,深自矜負。有謝異者,頗知廢興,梁陳之際,言無不驗,江南人甚敬信之。及陳主被禽,異奔瓛,由是益為眾所歸。宇文述討之,
 瓛遣王褒守吳州,自將拒述。述遣兵別道襲褒,褒衣道士服,棄城而遁。瓛敗,將左右數人,逃於太湖,匿於人家。被執,述送長安斬之。



 璟,仕隋,尚衣奉御;瑒,衛尉卿、秘書監、陶丘侯;瑀,內史侍郎、河池太守。



 蔡大寶,字敬位,濟陽考城人。祖履,齊尚書祠部郎。父點,梁尚書儀曹郎、南兗州別駕。



 大寶少孤,而篤學不倦,善屬文。初以明經對策第一,解褐武陵王國左常侍。



 嘗以書乾僕射徐勉,勉大賞異,乃令與其子遊處,所有墳籍,盡以給之。遂博覽群書,學無不綜。詧初出第,勉仍薦大寶為侍讀,兼掌記室。尋除尚書儀曹郎。詧出鎮會稽,大
 寶詣選曹求諮議,不得,以為記室。大寶攘臂而出曰:「不為孫秀,非人也。」詧蒞襄陽,遷諮議參軍,謀謨皆自大寶出。及梁元與河東王譽結隙,詧令大寶使江陵以觀之。梁元素知大寶,見之甚悅,乃示所制《玄覽賦》,令注解焉。



 三日而畢。梁元大嗟賞之,贈遺甚厚。大寶還,白詧云:「湘東必有異圖,禍亂將作,不可下援臺城。」詧納之。及詧於江陵稱帝,為侍中、尚書令,參掌選事,進位柱國、軍師將軍,封安豐縣侯。巋嗣位,冊授司空、中書監、中權大將軍、領吏部尚書。固讓司空,許之,加特進。巋之三年,卒。及葬,巋三臨其喪。贈司徒,進爵為公,謚曰文凱,配食詧廟。



 大
 寶性嚴整,有智謀,雅達政事,文辭贍速。詧之章表、書記、教令、詔冊,並大寶專掌之。詧推心委任,以為謀主。時人以詧之有大寶,猶劉先主之有孔明焉。



 所著文集三十卷,及《尚書義疏》,並行於世。



 有四子。次於延壽有器識,博涉經籍,尤善當世之務。尚詧女宣城公主,歷中書郎、尚書右丞、吏部郎、御史中丞。從琮入隋。授開府儀同三司、秘書丞。終於成州刺史。



 大寶弟大業,字敬道。有至行,位散騎常侍、衛尉卿、都官尚書、太常卿。卒,贈金紫光祿大夫,謚曰簡。有五子,允恭最知名。位太子舍人。梁滅入陳,為尚書庫部郎。陳亡仕隋,起居舍人。



 王操,字子高,其先太原晉陽人,詧母龔氏之外弟也。性敦厚,有籌略。初為詧外兵參軍,親任亞於蔡大寶。及詧稱帝,歷五兵尚書、郢州刺史,進位柱國,封新康縣侯。巋嗣位,授鎮右將軍、尚書僕射。及吳明徹為寇,巋出頓紀南,操撫循將士,莫不用命。明徹既退,江陵獲全,操之力也。遷侍中、中衛將軍、尚書令、開府儀同三司,領荊州刺史。操既位居朝右,每自挹損,深得當時之譽。卒,巋舉哀於朝堂,流涕曰:「天不使吾平蕩江表,何奪吾賢相之速也!」及葬,親祖於瓦官門。贈司空,進爵為公,謚曰康節。



 有七子,次子衡最知名。有才學,位中書、黃門侍郎。



 魏益德,襄陽人也。有材幹,膽勇過人。詧稱帝,進位柱國,封上黃縣侯。卒,贈司空,謚曰忠壯,進爵為公。巋之五年,以益德配食察廟。



 尹正,其先天水人。詧蒞雍州,正為其府中兵參軍。禽張纘,獲杜岸,皆正之力。詧稱帝,除護軍將軍,位柱國,封新野縣侯。卒,贈開府儀同三司,謚曰剛。



 巋之五年,以正配食詧廟。



 子德毅,多權略,位大將軍。後以見疑賜死。



 甄玄成,字敬平,中山人。博達經史,善屬文。少為簡文所知。以錄事參軍隨詧鎮襄陽,轉中記室參軍,頗參政事。以江陵甲兵殷盛,遂懷貳心,密書與元帝,具申誠款。或
 有得其書,送於詧。詧深信佛法,常願不殺誦《法華經》人。玄成素誦《法華經》,遂以此獲免。詧後見之,常曰:「甄公好得《法華經》力。」後位吏部尚書,有文集二十卷。



 子詡,少沈敏,閑習政事。歷中書舍人、尚書右丞。從琮入隋,授開府儀同三司,終於太府少卿。



 岑善方,字思義,南陽棘陽人。祖惠甫,給事中。父昶,散騎侍郎。善方有器局,博綜經史。同刑獄參軍隨詧至襄陽。詧初請內附,以善方兼記室充使,往來凡數十反。魏恭帝二年,封長寧縣公。及詧稱帝,位散騎侍郎、起部尚書。善方性清慎,有當世幹能,故詧委以機密。卒,贈太常卿,
 謚曰敬。所著文集十卷。



 有七子,並有操行。之元、之利、之象最知名。之元太子舍人,早卒。之利仕隋,位零陵郡丞。之象仕隋,尚書虞部員外侍郎,邵陵、上宜、渭南、邯鄲四縣令。



 宗如周,南陽人。有才學,以府僚隨詧,後至度支尚書。如周面狹長,詧以《法華經》云:「聞經隨喜,面不俠長。」嘗戲之曰:「卿何為謗經?」如周踧躇,自陳不謗。詧又謂之如初。如周懼,出告蔡大寶。大寶知其旨,笑謂之曰:「君當不謗餘經,正應不信《法華》耳。」如周乃悟。又嘗有人訴事於如周,謂為經作如州官也。乃曰:「某有屈滯,故來訴如州官。」如
 周曰:「爾何小人,敢呼我名!」



 其人慚謝曰:「祗言如周官作如州,不知如州官名如周,早知如州官名如周,則不敢喚如周官作如州。」如周乃笑曰:「令卿自責,見侮反深。」眾咸服其寬雅。



 袁敞,陳郡人。祖昂,司空。父士俊,安成內史。敞少有識量,博涉文史。以吏部郎使詣周。時主者以敞班在陳使之後,敞固不從命曰:「昔陳之祖父,乃梁諸侯下吏,盜有江東。今周朝宗萬國,招攜以禮。若使梁之行人在陳之後,便恐彞倫失序。豈使臣之所望焉。」主者不能屈,遂以狀奏。周武帝善之,乃詔敞與陳使異日而進。使還,以稱旨,
 遷侍中。轉左戶尚書。從琮入隋,授開府儀同三司。終於譙州刺史。



 論曰:自金行運否,中原喪亂,元氏唯天所命,方一函復。鐵弗、徒何之輩,雖非行錄所歸,觀其遞為割據,亦一時之傑。然而卒至夷滅,可謂魏之驅除。梁主任術好謀,愛賢養士,蓋有英雄之志,霸王之略焉。及淮海版蕩,骨肉猜貳,擁眾自固,稱籓內款,終能據有全楚,中興頹運。雖士宇殊於舊邦,而位號同於曩日。



 眙厥自遠,享國雖短,可不謂賢哉!嗣子纂業,增修遣構,賞罰得衷,舉厝有方。



 密邇寇讎,則威略具舉;朝宗上國,則聲猷遠振。豈非繼
 世之令主乎?琮大去其邦,因而不反,遂為外戚。不事自持,蓋亦守滿之道也。



\end{pinyinscope}