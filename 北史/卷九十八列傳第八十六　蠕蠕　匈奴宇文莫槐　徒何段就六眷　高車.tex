\article{卷九十八列傳第八十六 蠕蠕 匈奴宇文莫槐 徒何段就六眷 高車}

\begin{pinyinscope}

 蠕蠕
 姓郁久閭氏。始神元之末,掠騎有得一奴,髮始齊眉,忘本姓名,其主字之曰木骨閭。「木骨閭」者,首禿也。「木骨閭」與「郁久閭」聲相近,故後子孫因以為氏。木骨閭既壯,免奴為騎卒。穆帝時,坐後期當斬,亡匿廣漠谿谷間,收合逋逃,得百餘人,依純突鄰部。木骨閭死,子車鹿會
 雄健,始有部眾,自號柔然。



 後太武以其無知,狀類於蟲,故改其號為蠕蠕。車鹿會既為部帥,歲貢馬畜、貂豽皮。冬則徙度漠南,夏則還居漠北。車鹿會死,子吐奴傀立。吐奴傀死,子跋提立。



 跋提死,子地粟袁立。



 地粟袁死,其部分為二。地粟袁長子匹候跋繼父,居東邊;次子縕紇提,別居西邊。及昭成崩,縕紇提附衛辰而貳於魏。魏登國中討之,蠕蠕移部遁走,追之及於大磧南床山下,大破之,虜其半部。匹候跋及部帥屋擊,各收餘落遁走。遣長孫嵩及長孫肥追之,度磧。嵩至平望川,大破屋擊,禽之,斬以徇。肥至涿邪山,及匹候跋,舉落請降。獲縕紇提
 子曷多汗及曷多汗兄誥歸之、社侖、斛律等,并宗黨數百人,分配諸部。縕紇提西遁,將歸衛辰。道武追之至跋那山,縕紇提復降,道武撫慰如舊。



 九年,曷多汗與社侖率部眾棄其父西走,長孫肥輕騎追之,至上郡跋那山,斬曷多汗,盡殆其眾,社侖數人奔匹候跋,匹候跋處之南鄙,去其庭五百里,令其子四人監之。既而社侖率其私屬,執匹候跋四子而叛,襲匹候跋。諸子收餘眾,亡依高車斛律部。社侖兇狡有權變,月餘,乃釋匹候跋,歸其諸子,欲聚而殲之。密舉兵襲匹候跋,殺匹候跋。子啟拔、吳頡等十五人,歸于道武。社侖既殺匹候跋,懼王師討
 之,乃掠五原以西諸部,北度大漠。道武以拔、頡為安遠將軍,平棘侯。社侖與姚興和親,道武遣材官將軍和突襲黜弗、素古延諸部,社侖遣騎素古延,突逆擊破之。



 社侖遠遁漠北,侵高車,深入其地,遂並諸部,凶勢益振。北徙弱洛水,始立軍法;千人為軍,軍置將一人;百人為幢,幢置帥一人。先登者賜以虜獲,退懦者以石擊首殺之,或臨時垂撻。無文記,將帥以羊屎粗討兵數,後頗知刻木為記。其西北有匈奴餘種,國尤富彊,部帥日拔也稽舉兵擊社侖。逆戰於頞根河,大破之。



 後盡為社侖所并。號為彊盛,隨水草畜牧。其西則焉耆之地,東則朝鮮之
 地,北則渡沙漠,窮瀚海,南則臨大磧。其常所會庭,敦煌、張掖之北。小國皆苦其寇抄,羈縻附之。於是自號豆代可汗。豆代,猶魏言駕馭開張也;可汗,猶魏言皇帝也。



 蠕蠕之俗,君及大臣因其行能,即為稱號,若中國立謚。既死之後,不復追稱。



 道武謂尚書崔宏曰:「蠕蠕之人,昔來號為頑囂,每來抄掠,駕牸牛奔遁,驅犍牛隨之,牸牛伏不能前。異部人有教其以犍牛易之來者,蠕蠕曰:『其母尚不能行,而況其子!』終於不易,遂為敵所虜。今社侖學中國,立法,置戰陣,卒成邊害。道家言『聖人生,大盜起』,信矣。」



 天興五年,社侖聞道武征姚興,遂犯塞,入自參合陂,南
 至豺山及善無北澤。



 時遣常山王遵以萬騎追之,不及。天賜中,社侖從弟悅代、大那等謀殺社侖而立大那。發覺,大那等來奔,以大那為冠軍將軍、西平侯,悅代為越騎校尉、易陽子。



 三年夏,社侖寇邊。永興元年冬,又犯塞。二年,明元討之,社侖遁走,道死。



 其子度拔年少,未能御眾,部落立社侖弟斛律,號藹苦蓋可汗,魏言姿質美好也。斛律北并賀術也骨國,東破譬曆辰部落。三年,斛律宗人悅侯咄牴乾等百數十人來降。斛律畏威自守,不敢南侵,北邊安靜。神瑞元年,與馮跋和媵親,跋娉斛律女為妻,將為交婚。斛律長兄子步鹿真謂斛律曰:「女小遠
 適,憂思生疾,可遣大臣樹黎、勿地延等女為媵。」斛律不許。步鹿真出,謂樹黎等曰:「斛律欲令汝女為媵,遠至他國。」黎遂共結謀,令勇士夜就斛律穹廬後,伺其出執之,與女俱嬪于和龍。乃立步鹿真。步鹿真立,委政樹黎。



 初,高車叱洛侯者,叛其渠帥,導社侖破諸部落,社侖德之,以為大人。步鹿真與社侖子社拔共至叱洛侯家,淫其少妻。少妻告步鹿真,叱洛侯欲舉大檀為主,遺大檀金馬勒為信。步鹿真聞之,歸發八千騎往圍,叱洛侯焚其珍寶,自刎而死。



 步鹿真遂掩大檀。大檀發軍執步鹿真及社拔,絞殺之,乃自立。



 大檀者,社侖季父僕渾之子,先
 統別部鎮於西界,能得眾心,國人推戴之,號牟汗紇升蓋可汗,魏言制勝也。斛律父子既至和龍,馮跋封為上谷侯。大檀率眾南徙犯塞,明元親討之,大檀懼而遁走。遣山陽侯奚斤等追之,遇寒雪,士眾凍死及墮指者十二三。及明元崩,太武即位,大檀聞而大喜,始光元年秋,乃寇雲中。太武親討之,三日二夜至雲中。大檀騎圍太武五十餘重,騎逼,馬首相次如堵焉。士卒大懼。太武顏色自若,眾情乃安。先是,大檀弟大那與社侖爭國,敗而來奔。大檀以大那子於陟斤為部帥。軍士射於陟斤殺之,大檀恐,乃還。二年,太武大舉征之,東西五道並進。平
 陽王長孫翰等從黑漠;汝陰公長孫道生從白黑兩漠間;車駕從中道;東平公娥清次西,從栗園;宜城王奚斤、將軍安原等西道,從爾寒山。諸軍至漠南,舍輜重,輕騎齎十五日糧,絕漠討之。大檀部落駭驚,北走。



 神蒨元年八月,大檀遣子將騎萬餘入塞,殺掠邊人而走,附國高車追擊破之。



 自廣寧還,追之不及。二年四月,太武練兵於南郊,將襲大檀。公卿大臣皆不願,術士張深、徐辯以天文說止帝,帝從崔浩計而行。會江南使還,稱宋文欲犯河南,謂行人曰:「汝疾還告魏主,歸我河南地,即當罷兵;不然,盡我將士之力。」帝聞而大笑,告公卿曰:「龜鱉小
 豎,自救不暇,何能為也?就使能來,若不先滅蠕蠕,便是坐待寇至,腹背受敵,非上策也。吾行決矣!」於是車駕出東道,向黑山;平陽王長孫翰從西道,向大娥山。同會賊庭。五月,次于沙漠南,舍輜重輕襲之。



 至慄水,大檀眾西奔。弟匹黎先典東落,將赴大檀,遇翰軍,翰縱騎擊之,殺其大人數百。大檀聞之震怖,將其族黨,焚燒廬舍,絕跡西走,莫知所至。於是國落四散,竄伏山谷,畜產布野,無人收視。太武緣慄水西行,過漢將竇憲故壘。六月,車駕次於菟園水,去平城三千七百餘里。分軍搜討,東至瀚海,西接張掖水,北度燕然山,東西五千餘里,南北三千
 里。高車諸部殺大檀種類前後歸降三十餘萬,俘獲首虜及戎馬百餘萬匹。八月,太武聞東部高車屯巳尼陂,人畜甚眾,去官軍千餘里,遂遣左僕射安原等往討之。暨巳尼陂,高車諸部望軍降者數十萬。大檀部落衰弱,因發疾而死。



 子吳提立,號敕連可汗,魏言神聖也。四年,遣使朝獻。先是,北鄙候騎獲吳提南偏邏者二十餘人,太武賜之衣服,遣歸。吳提上下感德,故朝貢焉。帝厚賓其使而遣之。延和三年二月,以吳提尚西海公主,又遣使者納吳提妹為夫人,又進為左昭儀。吳提遣其兄禿鹿傀及左右數百人來朝,獻馬二千匹。帝大悅,班賜甚
 厚。



 至太延二年,乃絕和犯塞,四年,車駕幸五原,遂征之。樂平王丕、河東公賀多羅督十五將出東道,永昌王健、宜都王穆壽督十五將出西道,車駕出中道。至浚稽山,分中道復為二道,陳留王崇從大澤向涿邪山,車駕從浚稽北向天山。西登子阜,刻石記行,不見蠕蠕而還。時漠北大旱,無水草,軍馬多死。



 五年,車駕西伐沮渠牧犍,宜都王穆壽輔景穆居守,長樂王嵇敬、建寧王崇二萬人鎮漠南,以備蠕蠕。吳提果犯塞。壽素不設備,賊至七介山,京邑大駭,爭奔中城。司空長孫道生拒之於吐頹山。吳提之寇也,留其兄乞列歸與北鎮諸軍相守,敬、崇
 等破乞列歸于陰山之北,獲乞列歸。歎曰:「沮渠陷我也!」獲其伯父他吾無鹿胡及其將帥五百人,斬首萬餘級。吳提聞而遁走,道生追之,至于漠南而還。



 真君四年,車駕幸漠南,分軍為四道:樂安王範、建寧王崇各統十五將出東道,樂平王丕督十五將出西道,車駕出中道,中山王辰領十五將為中軍後繼。車賀至鹿渾谷,與賊相遇。吳提遁走,追至頞根河擊破之。車駕至石水而還。五年,復幸漠南,欲襲吳提,吳提遠遁,乃止。



 吳提死,子吐賀真立,號處可汗,魏言唯也。十年正月,車賀北伐,高涼王那出東道,略陽王羯兒出西道,車駕與景穆自中道出涿
 邪山。吐賀真別部帥爾綿他拔等率千餘家來降。是時,軍行數千里,吐賀真新立,恐懼遠遁。九月,車駕北伐,高涼王出東道,略陽王羯兒出中道,與諸軍期會於地弗池。吐賀真悉國精銳,軍資甚盛,圍那數十重。那掘長圍堅守,相持數日。吐賀真數挑戰輒不利,以那眾少而固,疑大軍將至,解圍夜遁。那引軍追之,九日九夜,吐賀真益懼,棄輜重,踰穹隆嶺遠遁。那收其輜重,引軍還,與車駕會於廣澤。略陽王羯兒盡收其人戶、畜產百餘萬。自是,吐賀真遂單弱,遠竄,邊疆息警矣。太安四年,車駕北征,騎十萬,車十五萬兩,旌旗千里,遂渡大漠。吐賀真
 遠遁,其莫弗烏朱駕頹率眾數千落來降,乃刊石記功而還。太武征伐之後,意存休息;蠕蠕亦怖威北竄,不敢復南。



 和平五年,吐賀真死,子予成立,號受羅部真可汗,魏言惠也。自稱永康元年。



 率部侵塞,北鎮遊軍大破其眾。皇興四年,予成犯塞,車駕北討,京兆王子推、東陽公元丕督諸軍出西道,任城王云等督軍出東道,汝陰王賜、濟南公羅烏拔督軍為前鋒,隴西王源賀督諸軍為後繼。諸將會車駕于女水之濱,獻文親誓眾,詔諸將曰:「用兵在奇,不在眾也。卿等但為朕力戰,方略已在朕心。」乃選精兵五千人挑戰,多設奇兵以惑之,虜眾奔潰,逐
 北三十餘里,斬首五萬級,降者萬餘人,戎馬器械,不可稱計。旬有九日,往返六千餘里。改女水曰武川,遂作《北征頌》,刊石紀功。



 延興五年,予成求通婚聘,有司以予成數犯邊塞,請絕其使,發兵討之。帝曰:「蠕蠕譬若禽獸,貪而亡義,朕要當以信誠待物,不可抑絕也。予成知悔前非,遣使請和,求結姻援,安可孤其款意?」乃詔報曰:「所論婚事,今始一反,尋覽事理,未允厥中。夫男而下女,爻象所明,初婚之吉,敦崇禮聘,君子所以重人倫之本。不敬其初,令終難矣。」予成懷譎詐,終獻文世,更不求婚。



 太和元年四月,遣莫何去汾比拔等來獻良馬、貂裘。比拔
 等稱:「伏承天朝珍寶華麗甚積,求一觀之。」乃敕有司,出御府珍玩、金玉、文繡、器物,御廄文馬、奇禽、異獸及人間所宜用者,列之京肆,令其歷觀焉。比拔見之,自相謂曰:「大國富麗,一生所未見也。」二年二月,又遣比拔等朝貢,尋復請婚焉。孝文志在招納,許之。予成雖歲貢不絕,而款約不著,婚事亦停。



 九年,予成死,子豆侖立,號伏古敦可汗,魏言恒也。自稱太平元年。豆侖性殘暴好殺。其名臣侯醫垔、石洛候數以忠言諫之,又勸與魏通和,勿侵中國。豆侖怒,誣石洛候謀反,殺之,夷其三族。



 十六年八月,孝文遣陽平王頤、左僕射陸睿並為都督,領軍斛律
 桓等十二將七萬騎討豆侖。部內高車阿伏至羅率眾十餘萬西走,自立為主。豆侖與叔父那蓋為二道追之。豆侖出自浚稽山北而西,那蓋出自金山。豆侖頻為阿伏至羅所敗,那蓋累有勝捷。四人咸以那蓋為天所助,欲推那蓋為主。那蓋不從,眾彊之。那蓋曰:「我為臣不可,焉能為主?」眾乃殺豆侖母子,以尸示那蓋,乃襲位。



 那蓋號候其伏代庫者可汗,魏言悅樂也。自稱太安元年。



 那蓋死,子伏圖立,號他汗可汗,魏言緒也。自稱始平元年。正始三年,伏圖遣使紇奚勿六跋朝獻,請求通和。宣武不報其使,詔有司敕勿六跋曰:「蠕蠕遠祖社侖是大魏
 叛臣,往者包容,暫時通使。今蠕蠕衰微,有損疇日;大魏之德,方隆周、漢,跨據中原,指清八表。正以江南未平,權寬北略。通和之事,未容相許。



 若脩蕃禮,款誠昭著者,當不孤爾也。」永平元年,伏圖又遣勿六跋奉函書一封,并獻貂裘。宣武不納,依前喻遣。



 伏圖西征高車,為高車王彌俄突所殺。子醜奴立,號豆羅伏拔豆伐可汗,魏言彰制也,自稱建昌元年。永平四年九月,醜奴遣沙門洪宣奉獻珠像。延昌三年冬,宣武遣驍騎將軍馬義舒使於醜奴,未發而崩,事遂停寢。醜奴壯健,善用兵。四年,遣使俟斤尉比建朝貢。熙平元年,西征高車大破之,禽其主
 彌俄突,殺之,盡并叛者,國遂彊盛。二年,又遣使俟斤尉比建、紇奚勿六跋、鞏顧禮等朝貢。神龜元年二月,明帝臨顯陽殿,引顧禮等二十人於殿下,遣中書舍人徐紇宣詔,讓以蠕蠕蕃禮不備之意。



 初,豆侖之死也,那蓋為主,伏圖納豆侖之妻候呂陵氏,生醜奴、阿那環等六人。醜奴立後,忽亡一子,字祖惠,求募不能得。有屋引副升牟妻是豆渾地萬,年二十許,為醫巫,假託神鬼,先常為醜奴所信,出入去來。乃言:「此兒今在天上,我能呼得。」醜奴母子欣悅。後歲仲秋,在大澤中施帳屋,齋潔七日,祈請天神。



 經一宿,祖惠忽在帳中,自云恒在天上。醜奴母
 子抱之悲喜,大會國人,號地萬為聖女,納為可賀敦。授夫副升牟爵位,賜牛、馬、羊三千頭。地萬既挾左道,亦是有姿色,醜奴甚加重愛,信用其言,亂其國政。如是積歲,祖惠年長,其母問之。



 祖惠言:「我恒在地萬家,不嘗上天。上天者,地萬教也。」其母具以狀告醜奴。



 醜奴言地萬懸鑒遠事,不可不信,勿用讒言也。既而地萬恐懼,譖祖惠於醜奴,醜奴陰殺之。



 正光初,醜奴母遣莫何去汾李具列等絞殺地萬。醜奴怒,欲誅具列等。又阿至羅侵醜奴,醜奴擊之,軍敗還,為母與其大臣所殺,立醜奴弟阿那瑰為主。阿那瑰立經十日,其族兄俊力發示發率眾數
 萬以伐,阿那瑰戰敗,將弟乙居伐輕騎南走歸魏。阿那瑰母候呂陵氏及其二弟尋為示發所殺,而阿那瑰未之知也。



 九月,阿那瑰將至,明帝遣兼侍中陸希道為使主,兼散騎常侍孟威為使副,迎勞近畿。使司空公、京兆王繼至北中,侍中崔光、黃門郎元纂在近郊,並申宴勞,引至闕下。十月,明帝臨顯陽殿,引從五品已上清官、皇宗、籓國使客等,列於殿庭。王公已下及阿那瑰等入就庭中,北面。位定,謁者引王公已下升殿,阿那瑰位於籓王之下,又引特命之官及阿那瑰弟並二叔升,位於群官之下。遣中書舍人曹道宣詔勞問。阿那瑰啟云:「陛下
 優隆,命臣弟、叔等升殿預會。但臣有從兄,在北之日,官高於二叔,乞命升殿。」詔聽之,乃位於阿那瑰弟之下,二叔之上。



 宴將罷,阿那瑰執所啟立於座後。詔遣舍人常景問所欲言。阿那瑰求詣帝前,詔引之。阿那瑰再拜跽曰:「臣先世源由,出於大魏。」詔曰:「朕已具知。」阿那瑰起而言曰:「臣之先,逐草放牧,遂居漠北。」詔曰:「卿言未盡,可具陳之。」



 阿那瑰又言曰:「臣祖先已來,世居北土,雖復隔越山津,而乃恭心慕化,未能時宣者,正以高車悖逆,臣國擾攘,不暇遣使以宜遠誠。自頃年已前,漸定高車,及臣兄為主,故遣鞏顧禮等使來大魏,實欲虔脩籓禮。是以
 曹道芝北使之日,臣與主兄,即遣大臣五人,拜受詔命。臣兄弟本心,未及上徹。但高車從而侵暴,中有姦臣,因亂作逆,殺臣兄,立臣為主。裁過旬日,臣以陛下恩慈如天,是故倉卒輕身投國,歸命陛下。」詔曰:「具卿所陳,理猶未盡,可更言之。」阿那瑰再拜受詔,起而言曰:「臣以家難,輕來投闕,老母在彼,萬里分張,本國臣人,皆已迸散。



 陛下隆恩,有過天地,求乞兵馬,還向本國,誅翦叛逆,收集亡散。陛下慈念,賜借兵馬,老母若在,得生相見,以申母子之恩;如其死也,即得報讎,以雪大恥。



 臣當統臨餘人,奉事陛下,四時之貢,不敢闕絕。陛下聖顏難睹,敢不披
 陳?但所欲言者,口不能盡言。別有辭啟,謹以仰呈,原垂昭覽。」仍以啟付舍人常景,具以奏聞。



 尋封阿那瑰朔方郡公、蠕蠕王,賜以衣冕,加之軺、蓋,祿從儀衛,同於戚籓。



 十二月,明帝以阿那瑰國無定主,思還綏集,占請切至,詔議之。時朝臣意有同異,或言聽還,或言不可。領軍元叉為宰相,阿那瑰私以金百斤貨之,遂歸北。



 二年正月,阿那瑰等五十四人請辭,明帝臨西堂,引見阿那瑰及其叔伯兄弟五人,升階賜坐,遣中書舍人穆弼宣勞。阿那瑰等拜辭。詔賜阿那瑰細明光人馬鎧一具,鐵人馬鎧六具,露絲銀纏槊二張并白眊,赤漆槊十張并白眊,
 黑漆槊十張并幡,露絲弓二張并箭,朱漆柘弓六張并箭,黑漆弓十張并箭,赤漆楯幡并刀,黑漆楯六幡并刀,赤漆鼓角二十具,五色錦被二領,黃紬被褥三十具,私府繡袍一領并帽,內者緋納襖一領、緋袍二十領并帽,內者雜彩千段,緋納小口褲褶一具內中宛具,紫納大口褲褶一具內中宛具,百子帳十八具,黃布幕六張,新乾飯一百石,麥八石,榛五石,銅烏錥四枚、柔鐵烏錥二枚各受二斛。黑漆竹榼四枚各受五升,婢二口,父草馬五百疋,駝百二十頭,牸牛一百頭,羊五千口,朱畫盤器十合,粟二十萬石,至鎮給之。詔侍中崔光、黃門
 元纂,郭外勞遣。



 阿那瑰來奔之後,其從父兄俟力發婆羅門率數萬人入討示發,破之。示發走奔地豆干,為其所殺。推婆羅門為主,號彌偶可社句可汗,魏言安靜也。時安北將軍、懷朔鎮將楊鈞表:「傳聞彼人已立主,是阿那瑰同堂兄弟。夷人獸心,已相君長,恐未肯以殺兄之人,郊迎其弟。輕往虛反,徒損國威。自非廣加兵眾,無以送其入北。」二月,明帝詔舊經蠕蠕,使者牒云具仁往,喻婆羅門迎阿那瑰復籓之意。婆羅門殊自驕慢,無遜避之心,責具仁禮敬,具仁執節不屈。婆羅門遣大官莫何去汾、俟斤丘升頭六人,將兵二千隨具仁迎阿那瑰。五月,具
 仁還鎮,論彼事勢。阿那瑰慮不敢入,表求還京。



 會婆羅門為高車所逐,率十部落詣涼州歸降。於是蠕蠕數萬,相率迎阿那瑰。



 七月,阿那瑰啟云:「投化阿那瑰蠕蠕元退社、渾河旃等二人,以今月二十六日到鎮,云國土大亂,姓姓別住,迭相抄掠,當今北人,鵠望待拯。今乞依前恩,賜給精兵一萬,還令督率領,送臣磧北,撫定荒人。脫蒙所請,事必克濟。」詔付尚書、門下博議。八月,詔兼散騎常侍王遵業馳驛宣旨慰喻阿那瑰,并申賜賚。九月,蠕蠕後主俟匿伐來奔懷朔鎮,阿那瑰兄也,列稱規望乞軍,并請阿那瑰。



 十月,錄尚書事高陽王雍、尚書令李
 崇、侍中侯剛、尚書左僕射元欽、侍中元叉、侍中安豐王延明、吏部尚書元修義、尚書李彥、給事黃門侍郎元纂、給事黃門侍郎張烈、給事黃門侍序盧同等奏曰:「竊聞漢立南北單于,晉有東西之稱,皆所以相維禦難,為國籓籬。今臣等參議,以為懷朔鎮北,土名無結山吐若奚泉,敦煌北西海郡,即漢、晉舊鄣,二處寬平,原野彌沃。阿那瑰宜置西吐若奚泉,婆羅門宜置西海郡。各令總率部落,收離聚散。其爵號及資給所須,唯恩裁處。彼臣下之官,任其舊俗。阿那瑰所居既是境外,宜少優遣,以示威刑。計沃野、懷朔、武川鎮各差二百人,令當鎮軍主監
 率,給其糧仗,送至前所。仍於彼為其造構,功就聽還。諸於北來在婆羅門前投化者,令州鎮上佐,准程給糧,送詣懷朔阿那瑰,鎮與使人,量給食稟;在京館者,任其去留。阿那瑰草創,先無儲積,請給朔州麻子乾飯二千斛,官駝運送。婆羅門居於四海,既是境內,資衛不得同之。阿那瑰等新造籓屏,宜各遣使持節馳驛,先詣慰喻,并委經略。」明帝從之。



 十二月,詔安西將軍、廷尉元洪超兼尚書行臺,詣敦煌安置婆羅門。婆羅門尋與部眾謀叛投嚈噠。嚈噠三妻,皆婆羅門姊妹也。仍為州軍所討,禽之。



 三年十二月,阿那瑰上表,乞粟以為田種。詔給萬石。
 四年,阿那瑰眾大飢。



 入塞寇抄。明帝詔尚書左丞元孚兼行臺尚書,持節喻之,孚見阿那瑰。為其所執。



 以孚自隨,驅掠良口二千并公私驛馬、牛羊數十萬北遁,謝孚放還。詔驃騎大將軍、尚書令李崇等率騎十萬討之,出塞三千餘里,至瀚海,不及而還。俟匿伐至洛陽,明帝臨西堂引見之。五年,婆羅門死於洛南之館,詔贈使持節、鎮西將軍、秦州刺史、廣牧公。



 是歲,沃野鎮人破六韓拔陵反,諸鎮相應。孝昌元年春,阿那瑰率眾討之。詔遣牒云具仁齎雜物勞賜。阿那瑰拜受詔命,勒眾十萬,從武川鎮西向沃野,頻戰剋捷。四月,明帝又遣通直散騎常
 侍、中書舍人馮俊使阿那瑰,宣勞班賜有差。阿那瑰部落既和,土馬稍盛,乃號敕連頭兵伐可汗,魏言把攬也。十月,阿那瑰復遣郁久閭彌娥等朝貢。三年四月,阿那瑰遣人鞏鳳景等朝貢。及還,明帝詔之曰:「北鎮群狄,為逆不息,蠕蠕主為國立忠,助加誅討,言念誠心,無忘寢食。今知停在朔垂,與爾朱榮鄰接,其嚴勒部曲,勿相暴掠。又近得蠕蠕主啟,更欲為國東討。



 但蠕蠕主世居北漠,不宜炎夏,今可且停,聽待後敕。」蓋朝廷慮其反覆也。此後頻使朝貢。



 建義初,孝莊詔曰:「夫勛高者賞重,德厚者名隆。蠕蠕主阿那瑰鎮衛北籓,禦侮朔表,遂使陰
 山息警,弱水無塵,刊迹狼山,銘名瀚海。至誠既篤,勛緒莫酬,故宜標以殊禮,何容格以恒式。自今以後,讚拜不言名,上書不稱臣。」



 太昌元年六月,阿那瑰遣烏勾蘭樹升伐等朝貢,并為長子請尚公主。永熙二年四月,孝武詔以范陽王誨之長女瑯邪公主許之,未及成婚,帝入關。東、西魏競結阿那瑰為婚好。西魏文帝乃以孝武時舍人元翌女稱為化政公主,妻阿那瑰兄弟塔寒,又自納阿那瑰女為后,加以金帛誘之。阿那瑰遂留東魏使元整,不報信命。後遂率眾度河,又廢后為言,文帝不得已,遂敕廢后自殺。



 元象元年五月,阿那瑰掠幽州范陽,
 南至易水。九月,又掠肆州秀容,至於三推。又殺元整,轉謀侵害。東魏乃囚阿那瑰使溫豆拔等。祖武以阿那瑰兇狡,將撫懷之,乃遣其使人龍無駒北還,以通溫豆拔等音問。始阿那瑰殺元整,亦謂溫豆拔等不存,既見無駒,微懷感愧。興和二年春,復遣龍無駒等朝貢東魏。然猶未款誠。



 阿那瑰女妻文帝者遇疾死,齊神武因遣相府功曹參軍張徽纂使於阿那瑰,間說之。云文帝及周文既害孝武,又殺阿那瑰之女,妄以疏屬假公主之號,嫁彼為親。



 又阿那瑰度河西討時,周文燒草,使其馬飢,不得南進,此其逆詐反覆難信之狀。



 又論東魏正統所
 在,言其往者破亡歸命,魏朝保護,得存其國,以大義示之。兼詐阿那瑰云:近有赤鋪步落堅胡行於河西,為蠕蠕主所獲。云蠕蠕主問之:「汝從高王?為從黑獺?」一人言從黑獺,蠕蠕主殺之;二人言從高王,蠕蠕主放遣。此即蠕蠕主存大國宿昔仁義。彼女既見害,欺詐相待,不仁不信,宜見討伐。且守逆一方,未知歸順,朝廷亦欲加誅。彼若深念舊恩,以存和睦,當以天子懿親公主結成姻媾,為遣兵將,伐彼叛臣,為蠕蠕主雪恥報惡。



 徽纂既申齊神武意,阿那瑰乃召其大臣與議之,便歸誠於東魏。遣其俟利、莫何莫緣游大力等朝貢,因為其子庵羅辰
 請婚。靜帝詔兼散騎常侍太府卿羅念、兼通直散騎常侍中書舍人穆景相等使於阿那瑰。八月,阿那瑰遣莫何去折豆渾十升等朝貢,復因求婚。齊神武請遂其意,以招四遠。詔以常山王騭妹樂安公主許之,改封為蘭陵郡長公主。十二月,阿那瑰復遣折豆渾十升詣東魏請婚。三年四月,阿那瑰遣吐豆登郁久閭譬渾、俊利莫何折豆常侯煩等奉馬千疋,以為聘禮,請迎公主。詔兼宗正卿元壽、兼太常卿孟韶等送公主自晉陽北邁,資用器物,齊神武親自經紀,咸出豐水屋。阿那瑰遣其吐豆登郁久閭匿伏、俊利阿夷普掘、蒱提棄之伏等迎公主
 於新城之南。六月,齊神武慮阿那瑰難信,又以國事加重,躬送公主於樓煩之北,接勞其使,每皆隆厚。阿那瑰大喜,自是朝貢東魏相尋。四年,阿那瑰請以其孫女號鄰和公主妻齊神武第九子長廣公湛,靜帝詔為婚焉。阿那瑰遣其吐豆登郁久閭譬掘、俊利莫何游大刀送女於晉陽。武定四年,阿那瑰有愛女,號為公主,以齊神武威德日盛,又請致之,靜帝聞而詔神武納之。阿那瑰遣其吐豆發郁久閭汗拔姻姬等送女於晉陽。自此東魏邊塞無事,至於武定末,使貢相尋。



 始阿那瑰初復其國,盡禮朝廷。明帝之後,中原喪亂,未能外略,阿那瑰統
 率北方,頗為強盛,稍敢驕大,禮敬頗闕,遣使朝貢,不復稱臣。天平以來,逾自踞慢。汝陽王暹之為秦州也,遣其典簽齊人淳于覃使於阿那瑰。遂留之,親寵任事。



 阿那瑰因入洛陽,心慕中國,立官號,僭擬王者,遂有侍中、黃門之屬。以覃為祕書監、黃門郎,掌其文墨。覃教阿那瑰,轉至不遜,每奉國書,鄰敵抗禮。及齊受東魏禪,亦歲時往來不絕。



 天保三年,阿那瑰為突厥所破,自殺。其太子庵羅辰及瑰從弟登注俟利、登注子庫提,並擁眾奔齊。其餘眾立注次子鐵伐為主。四年,齊文宣送登注及子庫提還北。鐵伐尋為契丹所殺,其國人仍立登注為主。
 又為大人阿富提等所殺,其國人復立庫提為主。是歲,復為突厥所攻,舉國奔齊。文宣乃北討突厥,迎納蠕蠕,廢其主庫提,立阿那瑰子庵羅辰為主,致之馬邑川,給其廩餼、繒帛。親追突厥於朔方,突厥請降,許之而還。於是蠕蠕貢獻不絕。



 五年三月,庵羅辰叛,文宣親討,大破之。庵羅辰父子北遁。四月,寇肆州。



 帝自晉陽討之,至恒州黃瓜堆,虜散走。時大軍已還。帝麾下千餘騎,遇蠕蠕別部數萬,四面圍逼。帝神色自若,指畫形勢,虜眾披靡,遂縱兵潰圍而出。虜退走,追擊之,伏尸二十五里,獲庵羅辰妻子及生口三萬餘人。五月,帝又北討蠕蠕,大破
 之。六月,蠕蠕帥部眾東徙,將南侵,帝帥輕騎於金川下邀擊,蠕蠕聞而遠遁。



 六年六月,文宣又親討蠕蠕。七月,帝頓白道,留輜重,親率輕騎五千追蠕蠕,躬犯矢石,頻大破之,遂至沃野,大獲而還。



 是時,蠕蠕既累為突厥所破,以西魏恭帝二年,遂率部千餘家奔關中。突厥既恃兵強,又藉西魏和好,恐其遺類依憑大國,使驛相繼,請盡殺以甘心。周文議許之,遂收縛蠕蠕主已下三千餘人付突厥使,於青門外斬之。中男以下免,並配王公家。



 匈奴宇文莫槐,出遼東塞外,其先南單于之遠屬也,世為東部大人。其語與鮮卑頗異。人皆翦髮而留其頂上,
 以為首飾,長過數寸則截短之。婦女被長襦及足,而無裳焉。秋收烏頭為毒藥,以射禽獸。莫槐虐用其人,為部下所殺,更立其弟普撥為大人。蓋撥死,子丘不勤立,尚平帝女。兵不勤死,子莫廆立。本名犯道武諱。



 莫廆遣弟屈雲攻慕容廆,慕容廆擊破之。又遣別部素延伐慕容廆於棘城,復為慕容廆所破,時莫廆部眾彊盛,自稱單于,塞外諸部咸憚之。



 莫廆死,子遜暱延立,率眾攻慕容廆於棘城。廆子翰先戍於外,遜暱延謂其眾曰:「翰素果勇,必為人患,宜先取之,城不足憂也。」乃分騎數千襲翰,翰聞之,使人詐為段末波使者,逆謂遜暱延曰:「翰數
 為吾患,久思除之,今聞來討,甚善。



 戒嚴相待,宜兼路早赴。」翰設伏待之。遜暱延以為信然。長驅不備,至於伏所,為翰所虜。翰馳使告,乘勝遂進,及晨而至。廆亦盡銳應之。遜暱延見而方嚴,率眾逆擊戰,前鋒始交,而翰已入其營,縱火燎之,眾乃大潰,遜暱延單馬奔還,悉俘其眾。遜暱延父子世雄漠北,又先得玉璽三紐,自言為天所相,每自誇大。及此敗也,乃卑辭厚幣,遣使朝貢於昭帝,帝嘉之,以女妻焉。



 遜暱延死,子乞得龜立。復伐慕容廆,廆拒之。惠帝三年,乞得龜屯保澆水,固壘不戰,遣其兄悉跋堆襲廆子仁于柏林。仁逆擊,斬悉跋堆。廆又攻
 乞得龜克之,乞得龜單騎夜奔,悉虜其眾。乘勝長驅,入其固城,收資財億計,徙部人數萬戶以歸。先是,海出大龜,枯死於平郭,至是而乞得龜敗。



 別部人逸豆歸殺乞得龜而自立,與慕容晃迭相攻擊。遣其國相莫渾伐晃,而莫渾荒酒縱獵,為晃所破,死者萬餘人。建國八年,晃伐逸豆歸,逸豆歸拒之。為晃所敗,殺其驍將涉亦干。逸豆歸遠遁漠北,遂奔高麗。晃徙其部眾五千餘落於昌黎,自是散滅矣。



 徒何段就六眷,出於遼西。其伯祖日陸眷,因亂被賣為漁陽烏丸大人庫辱官家奴。諸大人集會幽州,皆持唾
 壺,唯庫辱官獨無,乃唾日陸眷口中。日陸眷因咽之,西向拜天曰:「願使主君之智慧祿相,盡移入我腹中。」其後漁陽大飢。庫辱官以日陸眷為健,使將人詣遼西逐食,詔誘亡叛,遂至彊盛。日陸眷死,弟乞珍代立。



 乞珍死,子務目塵代立,即就六眷父也。據遼西之地而臣於晉。其所統三萬餘家,控弦上馬四五萬騎。穆帝時,幽州刺史王浚以段氏數為己用,深德之,乃表封務目塵為遼西公,假大單于印綬。浚使務目塵率萬餘騎伐石勒於常山封龍山下,大破之。



 務目塵死,就六眷立。就六眷與弟疋磾、從弟末波等率五萬餘騎圍石勒於襄國。



 勒登城
 望之,見將士皆釋伏寢臥,無警備之意。勒因其懈怠,選募勇健,穿城突出,直衝末波,生禽之。置之座上,與飲宴盡歡,約為父子,盟誓而遣之。末波既得免,就六眷等遂攝軍而還,不復報浚,歸于遼西。自此以後,末波常不敢南向溲焉。人問其故,末波曰:「吾父在南。」其感勒不害己也如此。



 就六眷死,其子幼弱,疋磾與劉琨世子群奔喪。疋磾陰卷甲而往,欲殺其叔羽鱗及末波而奪其國。末波等知之,遣軍逆擊疋磾。劉群為末波所獲。疋磾走還薊,懼琨禽己,請琨宴會,因執而害之。疋磾既殺劉琨,與羽鱗、末波自相攻擊,部眾乖離。欲擁其眾徙保上谷,阻
 軍都之險,以距末波等。平文帝聞之,陰嚴精騎,將擊之。疋磾恐懼,南奔樂陵。後石勒遣石季龍擊段文鴦于樂陵,破之,生禽文鴦。



 疋磾遂率其屬及諸塢壁降于石勒。



 末波自稱幽州刺史,屯遼西。末波死,國人因立陸眷弟護遼為主。烈帝時,假護遼驃騎大將軍、幽州刺史、大單于、北平公,弟鬱蘭撫軍將軍、冀州刺史、勃海公。建國元年,石季龍征護遼於遼西,護遼奔於平岡山,遂投慕容晃,晃殺之。鬱蘭奔石季龍,以所徙鮮卑五千人配之,使屯令支。鬱蘭死,子龕代之。及冉閔之亂,龕率眾南移,遂據齊地。慕容俊使弟玄恭率眾伐龕於廣固,執龕送之
 薊。俊毒其目而殺之,坑其徒三千餘人。



 高車,蓋古赤狄之餘種也。初號為狄歷,北方以為高車、丁零。其語略與匈奴同而時有小異。或云:其先匈奴甥也。其種有狄氏、袁紇氏、斛律氏、解批氏、護骨氏、異奇斤氏。俗云:匈奴單于生二女,姿容甚美,國人皆以為神。單于曰:「吾有此女,安可配人?將以與天。」乃於國北無人之地築高臺,置二女其上曰:「請天自迎之。」經三年,其母欲迎之。單于曰:「不可,未徹之間耳。」復一年,乃有一老狼,晝夜守臺嗥呼,因穿臺下為空穴,經時不去。其小女曰:「吾父處我於此,欲以與天,而今狼來,或是神物,無使之然。」
 將下就之。其姊夫驚曰:「此是畜生,無乃辱父母?」妹不從,下為狼妻而產子。後遂滋繁成國。故其人好引聲長歌,又似狼嗥。



 無都統大帥,當種各有君長。為性麤猛,黨類同心,至於寇難,翕然相依。鬥無行陣,頭別衝突,乍出乍入,不能堅戰。其俗,蹲踞褻黷,無所忌避。婚姻用牛馬納聘以為榮,結言既定,男黨營車闌馬,令女黨恣取上馬,袒乘出闌,馬主立闌外,振手驚馬,不墜者即取之,墜則更取,數滿乃止。俗無穀,不作酒。迎婦之日,男女相將,振馬酪熟肉節解。主人延賓,亦無行位,穹廬前叢坐,飲宴終日,復留其宿。明日,將婦歸。既而夫黨還入其家馬群,
 極取良馬,父母弟兄雖惜,終無言者。頗諱取寡婦,而優憐之。其畜產自有記識,雖闌縱在野,終無妄取。俗不清潔,喜致震霆。每震,則叫呼射天而棄之移去。來歲秋,馬肥,復相率候於震所,埋羖羊,燃火拔刀,女巫祝說,似如中國祓除,而群隊馳馬旋繞,百匝乃止。人持一束柳桋回,豎之,以乳酪灌焉。婦人以皮裹羊骸,戴之首上,縈屈髮鬢而綴之,有似軒冕。其死亡葬送,掘地作坎,坐尸於中,張臂引弓,佩刀挾槊,無異於生,而露坎不掩。時有震死及疫癘,則為之祈福;若安全無他,則為報賽。多殺雜畜,燒骨以燎,走馬繞旋,多者數百匝。男女無小大,皆集
 會。平吉之人,則歌舞作樂;死喪之家,則悲吟哭泣。其遷徙隨水草,衣皮食肉,牛、羊畜產,盡與蠕蠕同。唯車輪高大,輻數至多。



 徙於鹿渾海西北百餘里,部落彊大,常與蠕蠕為敵,亦每侵盜于魏。魏道武襲之,大破其諸部。後道武復度弱洛水,西行至鹿渾海,停駕簡輕騎,西北行百餘里,襲破之,虜獲生口、牛馬羊二十餘萬。復討其餘種於狼山,大破之。車駕北巡,分命諸將為東西二道,道武親勒六軍從中道,自駮髯水西北,徇略其部,諸軍同時雲合,破其雜種三十餘落。衛王儀別督諸將從西北絕漠千餘里,復破其遺迸七部。於是高車大懼,諸部震駭。
 道武自牛川南引,大校獵,以高車為圍,騎徒遮列,周七百餘里,聚雜獸於其中,因驅至平城,即以高車眾起鹿苑,南因臺陰,北距長城,東包白登,屬之西山。尋而高車侄利曷莫弗敕力犍率其九百餘落內附,拜敕力犍為揚威將軍,置司馬、參軍,賜穀二萬斛。後高車解批莫弗幡豆建復率其部三十餘落內附,亦拜為威遠將軍,置司馬、參軍,賜衣服,歲給廩食。



 蠕蠕社崙破敗之後,收拾部落,轉徙廣漠之北,侵入高車之地。斛律部帥倍侯利患之,曰:「新崙新集,兵貧馬少,易與耳!」乃舉眾掩擊,入其國落。高車昧利,不顧後患,分其廬室,妻其婦女,安息寢臥不起。
 社崙登高望見,乃招集亡散得千人,晨振殺之,走而脫者十二三。倍侯利遂奔魏,賜爵孟都公。侯利質直,勇健過人,奮戈陷陣,有異於眾。北方人善用五十蓍筮吉兇,每中,故得親幸,賞賜豐厚,命其少子曷堂內侍。及倍侯利卒,道武悼惜,葬以魏禮,謚曰忠壯王。後詔將軍伊謂帥二萬騎北襲高車餘種袁紇烏,頻破之。道武時,分散諸部,唯高車以類粗獷,不任使役,故得別為部落。



 後太武征蠕蠕,破之而還。至漠南,聞高車東部在巳尼陂,人畜甚眾,去官軍千餘里,將
 遣左僕射安原等討之。司徒長孫翰、尚書令劉潔等諫,太武不聽。乃遣原等并發新附高車合萬騎,至于巳尼陂,高車諸部望軍而降者數十萬落,獲馬牛羊亦百餘萬,皆徙置漠南千里之地。乘高車,逐水草,畜牧蕃息,數年之後,漸知粒食,歲致獻貢。由是國家馬及牛、羊遂至于賤,氈皮委積。文成時,五部高車合聚祭天,眾至數萬,大會走馬,殺牲游繞,歌吟忻忻。其俗稱自前世以來,無盛於此會。車駕臨幸,莫不忻悅。後孝文召高車之眾,隨車賀南討,高車不願南行,遂推袁紇樹者為主,相率北叛,游踐金陵。都督宇文福追討,大敗而還。又詔平北將
 軍、江陽王繼為都督討之。繼先遣人慰勞樹者。樹者入蠕蠕。尋悔,相率而降。



 高車之族又有十二姓:一曰泣伏利氏,二曰吐盧氏,三曰乙旃氏,四曰大連氏,五曰窟賀氏,六曰達薄氏,七曰阿崙氏,八曰莫允氏,九曰俟分氏,十曰副伏羅氏,十一曰乞袁氏,十二曰右叔沛氏。



 先是,副伏羅部為蠕蠕所役屬。豆崙之世,蠕蠕亂離,國部分散,副伏羅阿伏至羅與從弟窮奇俱統領高車之眾十餘萬落。太和十一年,豆崙犯塞,阿伏至羅等固諫不從,怒率所部之眾西叛,至前部西北,自立為王。國人號之曰候婁匐勒,猶魏言大天子也;窮奇號候倍,猶魏言
 儲主也。二人和穆,分部而立,阿伏至羅居北,窮奇在南。豆崙追討之,頻為阿伏至羅所敗,乃引眾東徙。十四年,阿伏至羅遣商胡越者至京師,以二箭奉貢。云:「蠕蠕為天子之賊,臣諫之不從,遂叛來此,而自豎立,當為天子討除蠕蠕。」孝文未之信也,遣使者于提往觀虛實。阿伏至羅與窮奇遣使者薄頡隨提來朝,貢其方物。詔員外散騎侍郎可足渾長生復與于提使高車,各賜繡褲褶一具,雜綵百匹。



 窮奇後為嚈噠所殺,虜其子彌俄突等。其眾分散,或來奔附,或投蠕蠕。詔遣宣威將軍、羽林監孟威撫納降人,置之高平鎮。阿伏至羅長子蒸阿伏
 至羅餘妻,謀害阿伏至羅,阿伏至羅殺之。阿伏至羅又殘暴,大失眾心,眾共殺之,立其宗人跋利延為主。歲餘,嚈噠伐高車,將納彌俄突。國人殺跋利延,迎彌俄突而立之。



 彌俄突既立,復遣朝貢,又奉表獻金方一、銀方一、金杖二、馬七匹、駝十頭。



 詔使者慕容坦賜彌俄突雜彩六十匹。宣武詔之曰:「卿遠據沙外,頻申誠款,覽捐忠志,特所欽嘉。蠕蠕、嚈噠、吐谷渾所以交通者,皆路由高昌,掎角相接。今高昌內附,遣使迎引。蠕蠕往來路絕,姦勢。不得妄令群小敢有陵犯,擁塞王人,罪在不赦。彌俄突尋與蠕蠕主伏圖戰於蒲類海北,為伏圖所敗,西走三
 百餘里。伏圖次於伊吾北山。先是,高昌王麴嘉表求內徙,宣武遣孟威迎之。至伊吾,蠕蠕見威軍,怖而遁走。彌俄突聞其離駭;追擊大破之,殺伏圖於薄類海北,割其髮,送於孟威。又遣使獻龍馬五匹,金、銀、貂皮及諸方物。詔東城子于亮報之,賜樂器一部、樂工八十人、赤紬十匹,雜彩六十匹。彌俄突遣其莫何去汾屋引叱賀真貢其方物。



 明帝初,彌俄突與蠕蠕主醜奴戰敗,被禽。醜奴繫其兩腳於駑馬之上,頓曳殺之,漆其頭為飲器。其部眾悉入嚈噠。經數年,嚈噠聽彌俄突弟伊匐還國。伊匐既復國,遣使奉表,於是詔遣使者谷楷等拜為鎮西將
 軍、西海郡開國公、高車王。伊匐復大破蠕蠕,蠕蠕主婆羅門走投涼州。正光中,伊匐遣使朝貢,因乞朱畫步挽一乘并幔褥、鞦必一副、傘扇各一枚、青曲蓋五枚、赤漆扇五枚、鼓角十枚,詔給之。伊匐後與蠕蠕戰,敗歸,其弟越居殺伊匐而自立。天平中,越居復為蠕蠕所破,伊匐子比適復殺越居而自立。興和中,比適又為蠕蠕所破,越居子去賓自蠕蠕奔東魏。齊神武欲招納遠人,上言封去賓為高車王,拜安北將軍、肆州刺史。既而病死。



 初,道武時有吐突鄰部在女水上,常與解如部相為脣齒,不供職事。登國三年,道武親西征,度弱洛水,復西行趣
 其國。至女水上,討解如部落,破之。明年春,盡略徙其部落畜產而還。



 又有紇突鄰,與紇奚世同部落,而各有大人長帥,擁集種類,常為寇於意辛山,登國五年,道武勒眾親討焉。慕容駖率師來會,大破之。紇突鄰大人屋地鞬,紇奚大人庫寒等皆舉部歸降。皇始二年,車駕伐中山,軍於柏肆。慕容寶夜來攻營,軍人驚,走還於國。路由并州,遂反,將攻晉陽,并州刺史元延討平之。紇突鄰部帥匿物尼、紇奚部帥叱奴根等復聚黨反於陰館,南安公元順討之不剋,死者數千人。



 道武聞之,遣安遠將軍庾嶽還討匿物尼等,皆殄之。



 又有侯呂鄰部,眾萬餘口,
 常依險畜牧。登國中,其大人叱伐為寇於苦水河。



 八年夏,道武大破之,并禽其別帥焉古延等。



 薛干部常屯聚于三城之間,及滅衛辰後,其部帥太悉伏望軍歸順,道武撫安之。



 車駕還,衛辰子屈丐奔其部。道武聞之,使使詔太悉伏執送之。太悉伏出屈丐以示使者曰:「今窮而見投,寧與俱亡,何忍送之!」遂不遣。道武大怒,車駕親討之。



 會太悉伏先出擊曹覆寅,官軍乘虛,遂屠其城,獲太悉伏妻子、珍寶,徙其人而還。



 太悉伏來赴不及,送奔姚興。未幾,亡歸嶺北。上郡以西諸鮮卑、雜胡聞而皆應之。



 天賜五年,屈丐盡劫掠總服之。及平統萬,薛干種類皆
 得為編戶矣。



 而牽屯山鮮卑別種破多蘭部世傳主部落。至木易乾,有武力壯勇,劫掠左右,西及金城,東侵安定,數年間,諸種患之。天興四年,遣常山王遵討之於高平。木易乾將數千騎棄國遁走,盡徙其人於京師、餘種分迸,其後,為赫連屈丐所滅。



 又黜弗、素古延等諸部,富而不恭。天興五年,材官將軍和突率六千騎襲而獲之。



 又越勤倍泥部,永興五年,轉牧跋那山西。七月,遣奚斤討破之,徙其人而還。



 論曰:周之獫狁,漢之匈奴。其作害中國,故久矣。魏、晉之世,種族瓜分,去來沙漠之陲,窺擾鄣塞之際,猶皆東胡
 之緒餘,冒頓之枝葉。至如蠕蠕者,匈奴之裔,根本莫尋,逃形集丑,自小為大,風馳鳥赴,倏來忽往,代京由之屢駭,戎車所以不寧。是故魏氏祖宗,揚威曜武,驅其畜產,收其部落,翦之窮發之野,逐之無人之鄉。豈好肆兵極銳,兇器不戢?蓋亦急病除惡,事不得已。其狡狄強弱之由,猾虜服叛之跡,故備錄云。



\end{pinyinscope}