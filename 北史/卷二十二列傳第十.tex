\article{卷二十二列傳第十}

\begin{pinyinscope}

 長孫嵩五世孫儉儉子平長孫道生曾孫幼幼孫曾熾熾弟晟從祖紹遠紹遠子覽弟澄長孫肥長孫嵩,代人也。父仁,昭成時為南部大人。嵩寬雅有器度,昭成賜名焉。年十
 四,代父統事。昭成末年,諸部乖亂,苻堅使劉庫仁攝國事,嵩與元他等率部眾歸之。



 劉顯之謀難也,嵩率舊人及庶師七百餘家叛顯走。將至五原,時寔君之子渥亦聚眾自立,嵩欲歸之。見於烏渥,稱逆父之子,勸嵩歸道武。嵩未決,烏渥回其牛首,嵩僶俛從之,見道武於二漢亭。道武以為南部大人,累著軍功。後從征中山,除冀州刺史,賜爵鉅鹿公。歷侍中、司徒、相州刺史,封南平公。所在著稱。明元即位,山陽侯奚斤、北新侯安同、白馬侯崔宏等八人坐止車門右,聽理萬機,故世號八公。



 晉將劉裕
 之伐姚泓,明元假嵩節,督山東諸軍事,傅詣平原,緣河北岸列軍,次於畔城。軍頗失利。詔假裕道。裕於舟中望嵩麾蓋,遺以酃酒及江南食物。嵩皆送京師。詔嵩厚答之。又敕簡精兵為戰備,若裕西過者,便率精銳,南出彭、沛;如不時過,但引軍隨之。彼至崤、陜間,必與姚泓相持,一死一傷,眾力疲弊。比及秋月,徐乃乘之,則裕
 首可不戰而縣。於是叔孫建等。尋河趣洛,遂入關。嵩與建等自城皋南濟,晉諸屯戍皆望塵奔潰。裕克長安,嵩乃班師。



 明元寢疾,問後事於嵩。嵩曰:「立長則順,以德則人服。今長皇
 子
 賢而世嫡,天所命也,請立。」乃定策,詔太武臨朝監國,嵩為左輔。



 太武即位,進爵北平王、司州中正。詔問公卿:「赫連、蠕蠕,征討何先?」



 嵩與平陽王長孫翰、司空奚斤等曰:「赫連土居,未能為患。蠕蠕世為邊害,宜先討大檀。及則收其畜產,足以富國;不及則校獵陰山,多殺禽獸,皮肉筋角以充軍實,亦愈於
 破一小國。」太常崔浩曰:「大檀遷徙鳥逝,疾追則不足經久,大眾則不能及之。赫連屈丐土宇不過千里,其刑政殘害,人神所棄,宜先討之。」尚書劉潔、武京侯安原請先平馮跋。帝默然,遂西巡狩。後聞屈丐死,關中大亂,議欲征之。嵩等曰:「彼若城守,以逸待勞。大檀聞之,乘虛而寇,危道也。」帝乃問幽徵於天師寇謙之。勸行,杜超之贊成,崔浩又言西伐利。嵩等固諫不可,帝大怒,責嵩在官貪污,使武士頓辱。尋遷太尉。久之,加柱國大將軍。自是輿駕征伐。嵩以元老,多留鎮京師,坐朝堂平斷刑獄。薨,年八十,謚曰宣王。後孝文追錄先朝功臣,以嵩配饗廟庭。



 子頹,善騎射,彎弓三百斤。襲爵,加侍中、征南大將軍。有罪黜為戍兵。後復爵。薨,謚曰安王。



 子敦,字孝友,位北鎮都將。坐黷貨,降為公。孝文時,自訟先世勳重,復其王爵。薨。謚簡王。



 子道,字念僧,襲爵。久之,隨例降為公,位左衛將軍。卒,謚慎。



 子悅,襲爵。建義初,復本王爵,尋降為公,位光祿少卿。卒,贈司空。



 嵩五世孫儉,仕周知名。儉,本名慶明。曾祖地汾,安東將軍、臨川公。祖酌,恆州刺史。父戫,員外散騎侍郎,早卒。



 儉方正有操行,神彩嚴肅,雖在私室,終日儼然。性不妄交,非其同志,雖貴遊造門,亦不與相見。太昌中,邊方騷動,儉初假東夏州防城大
 都督,從爾朱天光破宿勤明達等,以功賜爵索盧侯。周文臨夏州,以為錄事參軍事,深敬器之。及賀拔岳被害,周文赴平涼,凡有經綸謀策,儉皆參預。從平侯莫陳悅,留儉為秦州長史、防城大都督,委以後事,別封信都縣伯。渭州刺史可硃渾元奔東魏後,河渭間人情離阻。刺史李弼令儉權鎮渭州。儉將十餘騎冒難赴之。復隨機安撫,羌胡悅服。



 轉夏州刺史。甚得人和。時西夏州仍未內屬,而東魏遣許和為刺史。儉以信義招之,和乃歸附。即以儉為西夏州刺史,總統三夏州諸軍事。



 荊襄初附,周文表授儉都督三荊等十二州諸軍事、荊州刺史、東
 南道行臺僕射。



 所部鄭縣令泉璨為百姓所訟,推按獲實。儉即大集寮屬,遂於事前引己過,肉袒自罰,捨璨不問。於是屬城肅勵,莫敢犯法。魏文帝璽書勞之。周文又與儉書曰:「近聞公部內縣令有罪,遂自杖三十,用肅群下,聞之嘉歎良久不可言。」儉清正率下,兼懷仁恕。有竊盜者,原情得實,誨而放之。荊蠻舊俗,少不敬長。儉殷勤勸導,風俗大革。務廣耕桑,兼習武事,故邊境無虞,人安其業。吏人表請為儉構清德樓,樹碑刻頌,朝議許之。吏人又以儉秩滿,恐有代至,詣闕乞留儉,朝廷嘉而許之,在州遂歷七載。



 征授大行臺尚書,兼相府司馬。常與
 群公侍坐,及退,周文謂左右曰:「此人閑雅,孤每與語,常肅然畏敬,恐有所失。」他日,周文謂儉曰:「名實須相稱,尚書志安貧素,可改名儉,以彰雅操。」遷尚書左僕射,加侍中。後除東南道行臺僕射、大都督十五州諸軍事、荊州刺史。時梁岳陽王蕭詧內附,初遣使入朝。至荊州,儉於事列軍儀,具戎服,以賓主禮見使。容貌魁偉,音聲如鐘,大為鮮卑語,遣人傳譯以答問。客惶恐不敢仰視。日晚,儉乃著裙襦紗帽,引客宴於別齋。因敘梁國喪亂,朝廷招攜之意,發言可觀。使人大悅,出曰:「吾所不能測也。」



 魏廢帝二年,授東南道大都督、荊襄等三十三州鎮防
 諸軍事。及梁元帝嗣位於江陵,外敦鄰睦,內懷異計。儉密啟陳攻取之謀。於是征儉入朝,問以經略。儉陳謀,周文深然之,乃命還州,密為之備。尋令柱國于謹伐江陵。事平,以儉元謀,賞奴婢三百口。遂令儉鎮江陵,進爵昌寧郡公。後移鎮荊州,授總管荊襄等五十二州諸軍事、行荊州刺史。及周閔帝初,趙貴等將圖晉公護,儉長子僧衍預其謀,坐死。護乃征儉,拜小塚宰。保定四年,拜柱國。朝議以儉操行清白,勳績隆重,乃下詔褒美之,兼賜以雜採粟麥,以彰其美。



 天和初,轉陜州,總管七州諸軍事、陜州刺史。儉嘗詣闕奏事,時大雪,雪中待報,自旦達
 暮,竟無惰容。其謹愨若此。以疾還京,詔以儉舊居狹隘,賜甲第一區。



 後薨於夏州總管。臨終遺令:斂以時服,素車載柩,不設儀仗,親友贈襚,一無所受。諸子並奉行之。又遺啟請葬周文帝陵側,并以所賜宅還官,詔皆從之。贈本官,加涼瓜等十州諸軍事、涼州刺史,追封鄫國公,謚曰文。荊州人儀同趙超等六百九十七人,詣闕請為儉立廟樹碑,詔許之。



 建德元年,詔曰:「故柱國、鄫國公儉,臨終審正,爰吐德音。以所居之宅本因上賜,制度宏麗,非諸子所居,請以還官,更遷他所。昔叔敖辭沃壤之地,蕭何就窮僻之鄉。以古方令,無慚曩哲。而有司未達大
 體,遽以其第外給。夫追善念功,先王令典,豈得遂其謙挹,致乖懲勸!令以本宅還其妻子,俾清風遠播,無替聿修。」



 次子隆,位司金中大夫。從長潮公元定代陳,沒江南,卒。隆弟平,最知名。



 平字處均,美容儀,有器幹,頗覽書記,為周衛王侍讀。時武帝逼於宇文護,與衛王謀誅之,王常使平通意於帝。護誅,拜開府儀同三司。宣帝置東京官屬,以平為少司寇,與宗伯趙芬分掌六府。隋文龍潛時,與平情好款洽。及為丞相,恩禮彌厚。時賀若弼鎮壽陽,帝恐其懷貳,遣平代之為揚州總管,賜爵襄陽公。弼果不從,平麾壯士
 執弼,送京師。



 隋開皇三年,徵拜度支尚書。平見天下州縣多罹水旱,百姓不給,奏令人間每秋家出粟麥一石以下,貧富為差,儲之閭里,以備凶年,名曰義倉。帝深嘉納。自是州里豐衍。後轉工部尚書,名曰稱職。時有人告大都督邴紹非毀朝庭為憒憒者,上怒,將斬之。平進諫曰:「諺云:『不癡不聾,不作大家翁。』此言雖小,可以喻大。邴紹之言,不應聞奏。陛下又復誅之,恐百代之後,有虧聖德。」上於是赦紹。因敕群臣,誹謗之罪,勿復以聞。後突厥達頭可汗與都藍可汗相攻,各遣使請援。上遣平持節宣諭,令其和解。平至,陳利害,遂各解兵。可汗贈平馬二
 百疋。



 還,進所得馬,上盡以賜之。未幾,遇譴,以尚書檢校汴州事,尋除汴州刺史,後歷許、貝二州,俱有善政。



 鄴都俗薄,前後刺史,多不稱職。朝庭以平為相州刺史,甚有能名。在州數年,坐正月十五日百姓大戲,畫衣裳鍪甲象,上怒免之。俄而上念平鎮淮南時事,進位大將軍,拜太常卿、吏部尚書。卒官,謚曰康。



 子師孝,性輕狡好利,數犯法。上以其不克負荷,遣使弔平。以師孝為勃海郡主簿。屬大業之季,恣行貪濁,一郡苦之。後為王世充所害。



 長孫道生,嵩從子也。忠厚廉謹,道武愛其慎重,使掌機密。與賀毗等四人,內侍左右,出入詔命。明元即位,除南
 統將軍、冀州刺史。後取人美女以獻,明元切責之,以舊臣不加罪黜。太武即位,進爵汝陰公,遷廷尉卿。從征蠕蠕,與尉眷等率眾出白黑兩漠間,大捷而還。太武征赫連昌,道生與司徒長孫翰、宗正娥清為前驅,遂平其國。昌弟定走保平涼,宋遣將到彥之、王仲德寇河南以救定。詔道生與丹陽王太之屯河上以禦之。遂誘宋將檀道濟,邀其前後,追至歷城而還。除司空,加侍中,進封上黨王。薨,年八十二,贈太尉,謚曰靖。



 道生廉約,身為三司,而衣不華飾,食不兼味。一熊皮鄣泥,數十年不易,時人比之晏嬰。第宅卑陋。出鎮後,其子弟頗更修繕,起堂廡。
 道生還,歎曰:「昔霍去病以匈奴未滅,無有家為。今強寇尚遊魂漠北,吾豈可安坐華美也!」乃切責子弟,令毀其宅。太武世,所在著績,每建大議,多合時機。為將有權略,善待士眾。帝命歌工歷頌群臣曰:「智如崔浩,廉如道生。」及年老,頗惑其妻孟氏,以此見譏。與從父嵩俱為三公,當世以為榮。



 子瓬,位少卿,早卒。瓬子觀,少以壯勇知名,後襲祖爵上黨王。時異姓諸王襲爵,多降為公,帝以其祖道生佐命先朝,故特不降。以征西大將軍、假司空,督河西七鎮諸軍討吐谷渾。部帥拾寅遁藏,焚其所居城邑而還。孝文初,拜殿中尚書、侍中。吐谷渾又侵逼,復假
 觀司空討降之。後為征南大將軍。薨,謚曰定。葬禮依其祖靖王故事,陪葬雲中金陵。



 子冀歸,六歲襲爵,降為公。孝文以其幼承家業,賜名幼,字承業。承業聰敏有才藝,虛心愛士,為前將軍,從孝文南討。宣武時,為揚州刺史、假鎮南大將軍、都督淮南諸軍事。梁將裴邃、虞鴻襲據壽春,承業諸子驍果,邃頗難之,號曰:「鐵小兒」。詔河間王琛總眾援之。琛欲決戰,承業以雨久,更須持重。琛弗從,遂戰,為賊所乘,承業後殿。



 初,承業既總強兵,久不決戰,議者疑有異圖。朝庭重遣河間王琛及臨淮王彧、尚書李憲等三都督,外聲助承業,
 內實防之。



 會鮮于修禮反於中山,以承業為大都督北討。尋以本使達鄴城,詔承業解行臺,罷大使,遣河間王琛為大都督,酈道元為行臺。承業遣子子裕奉表,稱與琛同在淮南,俱當國難。琛敗臣全,遂生私隙。且臨機奪帥,非策所長。書奏,不納。琛與承業前到呼沲,承業未欲戰,而琛不從。行達五鹿,為修禮邀擊,琛不赴之,賊總至,遂大敗。承業與琛並除名。尋而正平郡蜀反,復假承業鎮西將軍、討蜀都督。



 頻戰有功,除平東將軍,復本爵。後除尚書右僕射。未幾,雍州刺史蕭寶夤據州反,復以承業為行臺討之。承業時背疽未愈,靈太后勞之曰:「卿疹
 源如此,朕欲相停,更無可寄,如何?」承業答曰:「死而後已,敢不自力。」時子彥亦患腳癉,扶杖入辭。尚書僕射元順顧相謂曰:「吾等備為大臣,各居寵位,危難之日,病者先行,無乃不可乎!」莫有對者。



 時薛鳳賢反於正平,薛修義屯聚河東,分據鹽池,攻圍蒲阪,東西連結,以應寶夤。承業乃據河東。時有詔廢鹽池稅,承業上表曰:「鹽池天資賄貨,密邇京畿,唯須寶而護之,均贍以理。今四境多虞,府藏罄竭,然冀、定二州,且亡且亂,常調之絹,不復可收。仰惟府庫,有出無入,必須經綸,出入相補。略論鹽稅,一年之中,準絹而言,猶不應減三十萬疋也。便是移冀、定
 二州置於畿甸,今若廢之,事同再失。臣前仰違嚴旨,不先討關賊,徑解河東者,非是閑長安而急蒲阪。蒲阪一陷,沒失鹽池,三軍口命,濟贍理絕。天助大魏,茲計不爽。昔高祖昇平之年,無所乏少,猶創置鹽官而加典護。非為物而競利,恐由利而亂俗也。況今王公素餐,百官尸祿,租征六年之粟,調折來歲之資。此皆出入私財,奪人膂力,豈是願言,事不獲已。臣輒符司監將尉,還率所部,依常收稅,更聽後敕。」及雍州平,除雍州刺史。



 孝莊初,封上黨王,尋改馮翊王,後降為郡公。遷司徒公,加侍中、兼尚書令、大行臺,仍鎮長安。節閔立,遷太尉公、錄尚書事。
 及韓陵之敗,斛斯椿先據河橋,謀誅爾朱。使承業入洛,啟節閔誅世隆兄弟之意。孝武初,轉太傅,以定策功,更封開國子。承業表請迴授其姨兄廷尉卿元洪超次子惲。初,承業生而母亡,為洪超母所撫養,是以求讓。許之。



 武帝入關,承業時鎮武牢,亦隨赴長安,位太師、錄尚書事,封上黨王。大統元年,薨,贈假黃鉞、大丞相、都督三十州諸軍事、雍州刺史,謚曰文宣。



 承業少輕俠,鬥雞走馬,力爭殺人,因亡抵龍門將陳興德家。會赦,乃免。因以後妻羅前夫女呂氏妻興德兄興恩以報之。羅年大承業十餘歲,酷妒忌。承業雅相敬愛,無姬妾。童侍之中在承
 業左右嫌疑致死者,乃有數四。前妻張氏二子,子彥、子裕。羅生三子:紹遠、士亮、季亮。兄弟皆雄武。



 子彥本名雋,有膂力,以累從父征討功,封槐里縣子。孝武帝與齊神武構隙,加子彥中軍大都督、行臺僕射,鎮恆農,以為心膂。及從帝入關,封高平郡公,位儀同三司。以從征竇泰、戰沙苑功,加開府、侍中。及東復舊京,以子彥兼尚書令、行司州牧,留鎮洛陽。後以不利,班師。大統七年,拜太子太傅。



 子彥少常墜馬折臂,肘上骨起寸餘。乃命開肉鋸骨,流血數升,言戲自若。時以為踰於關羽。末年石發,舉體生瘡,雖親戚兄弟以為惡疾。子彥曰:「惡疾如此,難以
 自明。世無良醫,吾其死矣!嘗聞惡疾蝮蛇螫之不痛,試為求之,當令兄弟知我。」及於南山得蛇,以股觸之,痛楚號叫,俄而腫死。文帝聞之。慟哭曰:「失我良將!」贈雍州刺史。



 子裕位衛尉少卿。啟捨凡階十七級,為子義貞求官。除左將軍,加通直散騎常侍。又以父勳,封平原縣伯。



 義貞弟兕,字若汗。性機辯,強記博聞,雅重賓游,尤善談論。從魏孝武西遷,別封鄴縣侯。周天和初,進驃騎大將軍、開府儀同三司。歷熊、絳二州刺史,並有能名。襲爵平原縣公。卒,子熾嗣。



 熾字仲光,性敏慧,美姿容,頗涉群書,兼長武藝。建德初,
 周武帝崇尚道法,求學兼經史者為通道館學士,熾應其選。隋文帝作相,自御正上士擢為丞相府功曹參軍,加大都督,封陽平縣子,遷稍伯下大夫。以平王謙,拜儀同三司。及帝受禪,熾率官屬先入清宮,即授內史舍人、上儀同三司,攝東宮右庶子,出入兩宮,甚被委遇。累遷太常少卿,改封饒陽縣子,進位開府儀同三司,改授吏部侍郎。大業中,歷位大理卿、戶部尚書。吐谷渾寇張掖,令熾擊之,追至青海,以功授銀青光祿大夫。六年,帝幸江都宮,留熾東都居守,攝左候衛將軍。卒官,謚曰靜。子安世,通事謁者。熾弟晟。



 晟字季晟,性通敏,略涉書記,善彈工射,矯捷過人。年十八,仕周為司衛上士。初未知名,唯隋文帝一見深異焉,謂曰:「長孫武藝逸群,又多奇略。後之名將,非此子邪?」及突厥攝圖請婚,周以趙王招女妻之。周與攝圖各相誇競,妙選驍勇以充使者,因遣晟副汝南公宇文神慶送千金公主至其牙。前後使人數十輩,攝圖多不禮之,獨愛晟,每共游獵,留之竟歲。嘗有二雕,飛而爭肉,因以箭兩隻與晟,請射取之。晟馳往,遇雕相玃,遂一發雙貫焉。攝圖喜,命諸子弟貴人皆相親友,冀暱近之,以學彈射。其弟處羅侯號突利設,尤得眾心,為攝圖所忌,密託心
 腹,陰與晟盟。晟與之游獵,因察山川形勢,部眾強弱。皆盡知之。還,拜奉車都尉。



 開皇元年,攝圖曰:「我,周家親也。今隋公自立而不能制,何面目見可賀敦!」



 因與高寶寧攻陷臨渝鎮,約諸面部落,謀共南侵。文帝新立,由是大懼,修長城,發兵屯北境。命陰壽鎮幽州、虞慶則鎮並州,屯兵為之備。



 晟先知攝圖、玷厥、阿波、突利等叔侄兄弟各統強兵,俱號可汗,分居四面,內懷猜忌,外示和同,難以力征,易可離間。因上書曰:「臣於周末,忝充外使,匈奴倚伏,實所具知。玷厥之於攝圖,兵強而位下,外名相屬,內隙已彰,鼓動其情,必將自戰。又處羅侯者,攝圖之弟,
 姦多而勢弱,曲取眾心,國人愛之,因為攝圖所忌。又阿波首鼠,介在其間,頗畏攝圖,受其牽率,唯強是與,未有定心。



 宜遠交而近攻,離強而合弱。通使玷厥,說合阿波,則攝圖回兵,自防右地。又引處羅,遣連奚、霫,則攝圖分眾,還備左方。道尾猜嫌,腹心離阻,十數年後,承釁討之,必可一舉而空其國。」



 上省表大悅,因召與語。晟口陳形勢,手畫山川,寫其虛實,皆如指掌。上深嗟異,皆納用焉。因遣太僕元暉出伊吾道,使詣玷厥,賜以狼頭纛,謬為欽敬。玷厥使來,引居攝圖使上。反間既行,果相猜貳。授晟車騎將軍,出黃龍道,齎敝賜奚、霫、契丹等,遣為鄉導,
 得至處羅侯所,深布心腹,誘令內附。



 二年,攝圖號四十萬騎,自蘭州入,至于周盤,破達奚長儒軍。更欲南入,玷厥不從,引兵而去。時晟又說染干詐告攝圖曰:「鐵勒等反,欲襲其牙。」攝圖乃懼,迴兵出塞。



 後數年,突厥大入,發八道元帥出拒之。阿波至涼州,與竇榮定戰,賊帥累北。



 時晟為偏將,使謂之曰:「攝圖每來,戰皆大勝。阿波纔入,便即致敗,此乃突厥之恥。且攝圖之與阿波,兵勢本敵,今攝圖日勝,為眾所崇,阿波不利,為國生辱。



 攝圖必當因此以罪歸於阿波,成其夙計,滅北牙矣。」阿波使至,晟又謂曰:「今達頭與隋連和,而攝圖不能制。可汗何不依
 附天子,連結達頭,相合為強?此萬全之計。豈若喪兵負罪,歸就攝圖,受其戮辱耶!」阿波納之,因留塞上。後使人隨晟入朝。時攝圖與衛王軍遇,戰於白道,敗走。至磧,聞阿波懷貳,乃掩北牙,盡獲其眾而殺其母。阿波還無所歸。西奔玷厥,乞師十餘萬,東擊攝圖,復得故地。



 收散卒,與攝圖相攻。阿波頻勝,其勢益強。攝圖又遣使朝貢,公主自請改姓,乞為帝女,上許之。



 四年,遣晟副虞慶則使于攝圖,賜公主姓為楊氏,改封大義公主。攝圖奉詔,不肯起拜。晟進曰:「突厥與隋俱是大國天子,可汗不起,安敢違意。但可賀敦為帝女,則可汗是大隋女媚,奈何不
 敬婦公?」攝圖笑謂其達官曰:「須拜婦公。」



 乃拜受詔。使還稱旨,授儀同三司、左勳衛車騎將軍。



 七年,攝圖死,遣晟持節拜其弟處羅侯為莫何可汗,以其子雍閭為葉護可汗。



 處羅侯因晟奏曰:「阿波為天所滅,與五六千騎在山谷間,當取之以獻。」時召文武議焉。樂安公元諧曰:「請就彼梟首,以懲其惡。」武陽公李充請生將入朝,顯戮而示百姓。上問晟,晟曰:「阿波之惡,非負國家。因其困窮,取而為戮,恐非招遠之道。不如兩存之。」上曰:「善。」



 八年,處羅侯死,遣晟往弔,仍齎陳國所獻寶器,以賜雍閭。



 十三年,流人楊欽亡入突厥,詐言彭國公劉昶共宇文氏女謀
 欲反隋,遣其來密告公主。雍閭信之,乃不修貢。又遣晟出使,微觀察焉。公主見晟,言辭不遜,又遣所私胡人安遂迦共欽計議,扇惑雍閭。晟還,以狀奏。又遣晟往索欽,雍閭欲勿與,謬曰:「客內無此色人。」晟乃貨其達官,知欽所在,夜掩獲之,以示雍閭。



 因發公主私事。國人大恥。雍閭執遂迦等,並以付晟。使還,上大喜,加授開府,仍遣入蕃,涖殺大義公主。雍閭又表請婚,僉議將許之。晟奏曰:「臣觀雍閭反覆無信,特共玷厥有隙,所以依倚國家。縱與為婚,終當必叛。今若得尚公主,承藉威靈,玷厥、染干必又受其徵發。強而更反,後恐難圖。且染干者,處羅侯
 之子,素有誠款,于今兩世。臣前與相見,亦乞通婚,不如許之,招令南徙。兵少力弱,易可撫馴,使敵雍閭,以為邊捍。」上曰:「善。」又遣慰喻染干,許尚公主。



 十七年,染干遣使隨晟來逆女。以宗女封安義公主以妻之。晟說染干南徙,居度斤舊鎮。雍閭疾之,亟來抄略。染干伺知動靜,輒遣奏聞,是以賊來,每先有備。



 十九年,染干因晟奏雍閭作攻具。欲打大同城。詔發六總管,並取漢王節度,分道出塞討之。雍閭懼,復共達頭同盟,合力掩襲染干,大戰於大長城下。染干敗績,其兄弟子姪盡見殺,而部落亡散。染干與晟獨以五騎逼夜南走。至旦,行百餘里,收得
 數百騎。乃相與謀曰:「今兵敗入朝,一降人耳,大隋天子豈禮我乎!玷厥雖來,本無冤隙,若往投之,必相存濟。」晟知懷貳,乃密遣使者入伏遠鎮,令速舉烽。染干見四烽俱發,問晟:「城上烽然,何也?」晟紿之曰:「城高地迥,必遙見賊來。我國家法,若賊少,舉二烽;來多,舉三烽;大逼,舉四烽。使見賊多而又近耳。」染干大懼,謂其眾曰:「追兵已逼,且可投城。」既入鎮,晟留其達官執室以領其眾,自將染干馳驛入朝。帝大喜,進晟左勳衛驃騎將軍,持節護突厥。晟遣降虜覘候雍閭,知其牙內屢有災變;夜見赤虹,光照數百里。天狗隕,雨血三日;流星墜其營內,有聲如
 雷。每夜自驚,言隋師且至。並遣奏知。尋以染干為意彌豆啟人可汗。賜射於武安殿,選善射者十二人,分為兩朋。啟人曰:「臣由長孫大使得見天子,今日賜射,願入其朋。」許之。給箭,六發皆入鹿,啟人之朋竟勝。時有鳶群飛,上曰:「公善彈,為我取之。」十發俱中,並應丸而落。是日,百官獲賚,晟獨居多。尋遣領五萬人,於朔州築大利城以處染干。安義公主死,持節送義城公主,復以妻之。晟又奏:「染干部落歸者既眾,雖在長城內,猶被雍閭抄略。往來辛苦,不得寧居。請徙五原,以河為固。於夏、勝兩州間,東西至河,南北四百里,掘為橫塹。令處其內,任情放牧,
 免於抄掠,人必自安。」上並從之。



 二十年,都藍大亂,為部下所殺。晟因奏曰:「賊內攜離,其主被殺。乘此招誘,必並來降。請遣染干部下,分頭招尉。」上許之,果盡來附。達頭恐怖,又大集兵。詔晟部領降人,為秦州行軍總管,取晉王廣節度,出討達頭。達頭與王相抗,晟進策曰:「突厥飲泉,易可行毒。」因取諸藥,毒水上流。達頭人畜飲之多死,大驚曰:「天雨惡水,其亡我乎!」因夜遁。晟追之,斬首千餘級,俘百餘口。王大喜,引晟入內,同宴極歡。有突厥達官來降,時亦預坐。說言突厥之內,大畏長孫總管,聞其弓聲謂為霹靂,見其走馬稱為閃電。王笑曰:「將軍震怒,威
 行域外,遂與雷霆為比,一何壯哉!」師旋,授上開府儀同三司,復遣還大利城,安撫新附。



 仁壽元年,晟表奏曰:「臣夜登城樓,望見磧北有赤氣,長百餘里,皆如雨足,下垂被地。謹驗兵書,此名灑血。其下之國,必且破亡。欲滅匈奴,宜在今日。」



 詔楊素為行軍元帥,晟為受降使者,送染干北伐。



 二年,軍次北河,逢賊帥思力俟斤等領兵拒戰,晟與大將軍梁默擊走之,賊眾多降。晟又教染干分遣使者,往北方鐵勒等部,招攜取之。三年,有鐵勒思結、伏具、渾、斛薛、阿拔、僕骨等十餘部,盡背達頭來降附。達頭眾大潰,西奔吐谷渾。



 晟送染干,安置于磧口。事畢,入朝。



 遇文帝崩,匿喪未發。煬帝引晟於大行前委以內衙宿衛,知門禁事,即日拜左領軍將軍。遇楊諒作逆,敕以本官為相州刺史,發山東兵馬,與李雄等共經略之。



 晟辭以子行布在逆地。帝曰:「公終不以兒害義,其勿辭也。」於是馳遣赴相州。



 諒破,追還,轉武衛將軍。



 大業三年,煬帝幸榆林,欲出塞外,陳兵耀武,經突厥中,指于涿郡。仍恐染干驚懼,先遣晟往喻旨,稱述帝意。染干聽之,因召所部諸國,奚、霫、室韋等種落數十,酋長咸萃。晟見牙中草穢,欲令染干親自除之,示諸部落,以明威重。乃指帳前草曰:「此根大香。」染干遽取嗅之,曰:「殊不香也。」曰:「國家法,
 天子行幸所在,諸侯並躬親灑掃,耘除御路,以表至敬之心。今牙中蕪穢,謂是留香草耳。」染干乃悟,曰:「奴罪過!奴之骨肉,皆天子賜也。得效筋力,豈敢有辭?特以旁人不知法耳。」遂拔所佩刀,親自芟草。其貴人及諸產落爭放效之。乃發榆林北境,至於其牙,又東達於薊,長三千里,廣百餘步,舉國就役而開御道。



 帝聞益喜焉。後除淮陽太守,未赴任,復為右驍衛將軍。



 五年,卒,年五十八,帝悼惜之。後突厥圍鴈門,帝歎曰:「向使長孫晟在,不令匈奴至此!」



 晟好奇計,務立功名。性至孝,居憂毀瘠,為朝士所稱。大唐貞觀中,追贈司空、上柱國,謚曰獻。少子無忌
 嗣。其長子行布,亦多謀略,有父風。起家漢王諒庫直。後遇諒并州起逆,率從南拒官軍,留行布守城。遂與豆盧毓閉門拒守諒,城陷,遇害。次子恆安,以兄功授鷹揚郎將。



 紹遠字師,少名仁。寬容有大度,雅好墳籍,聰慧過人。父承業作牧壽春,時紹遠年十三。承業管記有王碩者,文學士也,聞紹遠強記,遂白承業,求驗之。承業命試之。碩乃試以《禮記月令》。於是紹遠讀數紙,纔一遍,誦之若流。碩歎服之。起家司徒府參軍事。後以別將討平河東蜀薛,封東阿縣伯。



 魏孝武西遷,紹遠隨承業奔赴,以功別
 封文安縣子。大統二年,除太常卿,遷中書令,仍襲父爵。後例降為公,改馮翊郡。恭帝二年,累遷錄尚書事。周文每謂群臣曰:「長孫公任使處,令人無反顧憂;漢之蕭、寇,何足多也。其容止堂堂,足為當今模楷。」六官建,拜大司樂。周閔踐祚,復封上黨郡公。



 初,紹遠為太常,廣召工人,創造樂器,唯黃鐘不調,每恒恨之。嘗經韓使君佛寺,聞浮圖三層上鐸鳴,其音雅合宮調,因取而配奏,方始克諧。乃啟明帝曰:「魏氏來宅秦、雍,雖祖述樂章,然黃鐘為君,天子之正位,往經創造,歷稔無成。



 方知水行將季,木運伊始,天命有歸,靈樂自降。此蓋乾坤祐助,宗廟致感。
 方當降物和神,祚隆萬世。」詔曰:「朕以菲薄,何德可以當之。此蓋天地祖宗之祐,亦由公達鑒所致也。」俄改授禮部中大夫。時猶因魏氏舊樂,未遑更造,但去小呂,加大呂而已。紹遠上疏陳雅樂,詔並行之。紹遠所奏樂,以八為數。故梁黃門侍郎裴正上書,以為昔者大舜欲聞七始,下洎周武,爰制七音。持林鐘作黃鐘,以為正調之首。詔與紹遠詳議。



 正曰:「天子用八,非無典故;縣而不擊,未聞厥理。且黃鐘為天,大呂為地,太蔟為人。今縣黃鐘而擊太蔟,便是虛天位專用人矣。」



 紹遠曰:「夫天不言,四時行焉。地不言,萬物生焉。人感中和之氣,居變通之道。今
 縣黃鐘而擊太蔟,是天子端拱,群司奉職。從此而議,何往不可?」



 正曰:「案《呂氏春秋》曰:『楚之衰也,為作巫音;齊之衰也,為作大呂。』且大呂以下七鐘,皆是林鐘之調,何得稱為十一月調?專用六月之均,便是欲迎仲冬,猶行季夏。以此而奏,深非至理。」



 紹遠曰:「卿之所言,似欲求勝。若窮理盡性,自伐更深。何者?案《周禮》祀天樂云:『黃鐘為宮,大呂為角。』此則大呂之用,宛而成章。雖知引呂氏之小文,不覺失周公之大禮。且今縣大呂,則有黃鐘、林鐘,二均乃備。春夏則奏林鐘,秋冬則奏黃鐘,作黃鐘不擊大呂,作林鐘不擊黃鐘。此所謂左之右之;君子宜之,右之
 左之,君子有之。而卿不縣大呂,止有黃鐘一宮,便是季夏之時仍作仲冬之調。



 以此為至理,無乃不可乎!然《周禮》又云:『乃奏黃鐘,歌大呂,以祀天神。』謂五帝及日月星辰也。王者各以夏之正月,祀感帝於南郊。又朝日以春分,夕月以秋分,依如正禮,並用仲冬之調。又曰:『奏太蔟,歌應鐘,以祭地祇。』謂神州及社稷。以春秋二仲,依如正禮,唯奏孟春之宮。自外四望、山川、先妣、先祖,並各周宮,不依月變。略舉大綱,則三隅可反。然則還相為宮,雖有其義,引《禮》取證,乃不月別變宮。且黃鐘為君,則陽之正位,若隨時變易,是君無定體。而卿用林鐘,以為正調,便
 是君臣易位,陰陽相反。正之名器,將何取焉?」



 正曰:「今用林鐘為黃鐘者,實得相生之義。既清且韻,妙合真體。然八音平濁,何足可稱?」



 紹遠曰:「天者陽位,故其音平而濁,濁則君聲。地者陰位,故其音急而清,清則臣調。然急清者於體易絕,平濁者在義可久。可久可大,王者之基。至於鄭、衛新聲,非不清韻,若欲施之聖世,吾所不取也。」於是遂定,以八為數焉。



 尋拜京兆尹,歷少保、小司空,出為河州刺史。河右戎落,向化日近,同姓婚姻,因以成俗。紹遠導之以禮,大革弊風。政存簡恕,百姓悅服。入為小宗伯。



 武帝讀史書,見武王克殷而作七始,又欲廢八縣七,
 並除黃鐘之正宮,用林鐘為調首。紹遠奏云:「天子縣八,百王共軌。下逮周武,甫修七始之音。詳諸經義,又無廢八之典。且黃鐘為君,天子正位,今欲廢之,未見其可。臣案《周禮》奏黃鐘,歌大呂,此則先聖之弘範,不易之明證。願勿輕變古典,趣改樂章。」帝默然久之,曰:「朕欲廢八縣七者,所望體本求直,豈茍易名。當更思其義。」後竟行七音。



 屬紹遠遘疾,未獲面陳。慮有司遽捐樂器,乃與樂部齊樹書曰:「伏聞朝廷前議,而欲廢八縣七。然則天子縣八,有自來矣。古先聖,殊塗一致;逮周武克殷,逆取順守,專用干戈,事乖揖讓。反求經義,是用七音,蓋非萬代不
 易之典。其縣八筍虡,不得毀之。宜待吾疾瘳,當別奏聞。」此後紹遠疾篤,乃命其子覽曰:「夫黃鐘者,天子之宮。大呂者,皇后之位。今廢黃鐘之位,是祿去王室。若用林鐘為首,是政出私門。將恐八百之祚,不得同姬周之永也。吾既為人臣,義無寢默,必輿疾固爭闕庭。」後疾甚,乃上遺表曰:「謹案《春秋》隱公《傳》云:『天子用八。』周禮云天子縣二八,倕氏之鐘十六,母句氏之磬十六。漢成帝獲古磬十六。



 《周禮圖》縣十六。此數事者,照爛典章。揚搉而言,足為龜鏡。伏惟陛下受圖蒼帝,接統玄精,秦、漢以還,獨為稱首。至如周武,有事干戈,臣獨鄙之,而況陛下。以臣自
 揣餘息,匪夕伊朝。伏願珍御萬機,不勞改八從七。」帝省表涕零,重贈柱國大將軍,謚曰獻,號樂祖,配饗廟庭。子覽嗣。



 覽字休因,性弘雅,有器度,喜慍不形於色。略涉書記,尤曉鐘律。周明帝時,為大都督。明帝以覽性質淳和,堪為師表,使事魯公,甚見親善。及魯公即位,是為武帝,超拜車騎大將軍。每公卿上奏,必令省讀。覽有口辯,聲氣雄壯,凡所宣傳,百僚屬目。帝每嘉嘆之。覽初名善,帝謂曰:「朕以萬機委卿先覽。」遂賜名焉。及誅宇文護,以功進封薛國公,累遷小司空。從平齊,進位柱國。武帝崩,受遺
 輔政。宣帝時,位上柱國、大司徒,歷同、涇二州刺史。隋文帝為丞相,轉宜州刺史。開皇二年,將有事於江南,徵為東南道行軍元帥,統八總管出壽陽,水陸俱進。師臨江,陳人大駭。會陳宣帝殂,覽欲乘釁滅之,監軍高熲以禮不伐喪,乃還。



 文帝命覽與安德王楊雄、上柱國元諧、李充、左僕射高熲、右衛大將軍虞慶則、吳州總管賀若弼等同宴。上曰:「朕昔在周朝,備展誠節。但苦被猜忌,每致寒心。



 為臣若此,竟何情賴!朕與公等,共享終吉。罪非謀逆,一無所問。朕亦知公至誠侍太子,宜數參見之。柱臣素望,實屬於公。宜識朕意。」其恩禮如此。又為蜀王秀納
 覽女為妃。後為涇州刺史。卒官。



 子洪嗣,位宋順臨三州刺史、司農少卿、北平太守。



 澄字士亮,年十歲,司徒李琰之見而奇之,遂以女妻焉。十四從父承業征討,有智謀,勇冠諸將。以功封西華縣侯。及長,容貌魁岸,風儀溫雅。魏大統中,歷位豫、渭二州刺史。以軍功,別封永寧縣伯,尋進覆津縣侯。



 魏文帝與周文及群公宴,從容曰:「《孝經》一卷,人行之本,諸君宜各引《孝經》之要言。」澄應聲曰:「夙夜匪懈,以事一人。」座中有人次云:「匡救其惡。」既出西閣,周文深嘆澄之合機,而譴其次答者。



 周孝閔帝踐阼,拜大將軍,進爵義門郡公。出
 為玉壁總管,頗有威信。卒於鎮,贈柱國,謚曰簡。自喪初至及葬,明帝三臨之。典祀中大夫宇文容諫曰:「君臨臣喪,自有節制。今乘輿屢降,恐乖典禮。」帝不從。其為上所追惜如此。子嶸嗣。



 瓬弟禮,少以父任為散騎侍郎,與襄城公盧魯元等內侍。恭敏有才志。太武寵信之,曰:「其父親近吾祖,子在我左右,不亦宜乎。」



 長孫肥,代人也。昭成時,年十三,以選內侍。少有雅度,果毅少言。道武之在獨孤及賀蘭部,常侍從,禦侮左右,帝深信仗之。登國初,與莫題等俱為大將,屢有軍功。後從平中山,以功賜爵瑯邪公。遷衛尉卿,改爵盧鄉。時中山
 太守仇儒不樂內徙,亡匿趙郡,推趙準為主。妄造妖言云:「燕東傾,趙當續。欲知其名,淮水不足。」準喜而從之,自號鉅鹿公,儒為長史。據關城,連引丁零,殺害長吏。



 肥討破準於九門,斬仇儒,禽準。詔以儒肉食準,傳送京師,轅之於市,夷其族。



 除肥兗州刺史。



 姚平之寇平陽,道武征肥與毗陵王順等為前鋒。平退保柴壁,帝進攻屠之。遣肥還鎮兗州,撫慰河南,威信著於淮泗。善策謀,勇冠諸將。前後征討,未嘗失敗。



 故每有大難,令肥當之;南平中原,西摧羌寇,肥功居多,賞賜千計。後降爵藍田侯。卒,謚曰武,陪葬金陵。子翰襲爵。



 翰少有父風。道武時,以善騎
 射,為獵郎。明元之在外,翰與元磨渾等潛謀奉迎。明元即位,與磨渾等拾遺左右。以功累遷平南將軍。率眾鎮北境,威名甚著。



 太武即位,封平陽王。蠕蠕大檀之入寇雲中,太武親征之。遣翰與東平公娥清出長川討大檀。大檀北遁,追擊剋獲而還。遷司徒。從襲赫連昌,破之。翰清正嚴明,喜撫將士,薨,太武為之流涕,親臨其喪。喪禮依安城王叔孫俊故事。謚曰威,陪葬金陵。



 子成襲爵,降為公,位南部尚書。卒,陪葬金陵。翰弟陵,位駕部尚書。性寬厚,好學愛士。封吳郡公,贈吳郡王。謚恭,陪葬金陵。



 論曰:昭成之末,眾叛親離。長孫嵩寬厚沈毅,任重王室,
 歷事累世,邈為元老。生則宗臣,歿祀清廟,美矣!儉器識明允,智謀通贍,堂堂焉有公輔之望,謇謇焉有王臣之節。而處朝廷之日少,在方岳之日多,何哉?平識具該通,出內流譽,取諸開物成務,蓋亦有隋之榱桷也。道生恭慎廉約,兼著威名,見知明主,聲入歌奏。二公並列,暉炫朝野,門祉世祿,榮被後昆。雖漢世八王,無以方其茂績;張氏七葉,不能譬此重光。子彥勇烈絕倫,紹遠樂聲特妙;熾乃早稱英俊,覽乃獨擅雄辯。不然則何以並統師旅,俱司禮閣,鐘鼎不墜,且公且侯?晟體資英武,兼包奇略。因機制變,懷彼戎夷,傾巢盡落,屈膝稽顙。塞垣絕鳴
 鏑之旅,渭橋有單于之拜。惠流邊朔,功光王府,保茲世祿,不亦宜乎!肥結發內侍,雄武自立,軍鋒所指,罔不棄散,關、張萬人敵,未足多也。翰有父風,不殞先構,臨喪加禮,抑有由哉!



\end{pinyinscope}