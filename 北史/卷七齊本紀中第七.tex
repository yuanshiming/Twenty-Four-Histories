\article{卷七齊本紀中第七}

\begin{pinyinscope}

 顯祖文宣皇帝諱洋,字子進,神武第二子,文襄之母弟也。武明太后初孕帝,每夜有赤光照室,太后私怪之。及產,命之曰侯尼于。鮮卑言有相子也。以生於晉陽,一名晉陽樂。時神武家徒壁立,后與親姻相對,共憂寒餒。帝生始數月,尚未能言,欻然曰:「得活。」太后及左右大驚,不敢言。及長,黑色,大頰兌下,鱗身重踝。瞻視審定,不好戲弄,深沈有大度。晉陽有沙門,乍愚乍智,時人不測,呼為
 阿禿師。太后見諸子焉,歷問祿位。至帝,再三舉手指天而已,口無所言,見者異之。神武嘗從諸子過鳳陽門,有龍在上,唯神武與帝見之。內雖明敏,貌若不足。文襄每嗤之曰:「此人亦得富貴,相法亦何由可解。」神武以帝貌陋,神彩不甚發揚。曾問以時事,帝略有所辨,儻語一事,必得事衷。又嘗令諸子,各使理亂絲,帝獨抽刀斬之,曰:「亂者須斬。」神武以為然。又各配兵四出,而使彭樂率甲騎偽攻之。文襄等怖撓,帝勒眾與彭樂相格,樂免胄言情,猶禽之以獻。由是神武稱異之,謂長史薛琡曰:「此兒意識過吾。」琡亦私怪之。幼時,師事范陽盧景裕,默識過
 人,未嘗有所自明,景裕不能測也。天平二年,封太原郡公,累遷尚書左僕射。後從文襄行過遼陽山,獨見天門開,餘無人見者。



 武定五年,神武崩。猶祕凶事,眾情疑駭。帝雖內嬰巨痛,外若平常,人情頗安。魏帝授帝尚書令、中書監、京畿大都督。



 七年八月,文襄遇賊。帝在城東雙堂,事出倉卒,內外震駭。帝神色不變,指麾部分,自臠斬群賊而漆其首,祕不發喪。徐言奴反,大將軍被傷,無大吉也。當時內外,莫不驚異。乃諷魏朝立皇太子,因以大赦。乃赴晉陽總庶政。帝內雖明察,外若不了,老臣宿將皆輕帝。於是帝推誠接下,務從寬厚,事有不便者咸蠲
 省焉,群情始服。



 八年正月辛酉,魏帝為文襄舉哀於東堂。戊辰,詔進帝位使持節、丞相、都督中外諸軍、錄尚書事、大行臺、齊郡王,食邑一萬戶。三月庚申,又進封齊王,食冀州之勃海、長樂、安德、武邑、瀛州之河間五郡,邑十萬戶。帝自居晉陽,寢室每夜有光如晝。既為王,夢人以筆點己額。旦日,以語館客王曇哲,曰:「吾其退乎?」曇哲拜賀曰:「王上加點為主,當進也。」五月辛亥,帝如鄴。光州獲九尾狐以獻。甲寅,魏帝遣兼太尉彭城王韶、司空潘相樂奉冊,進帝位相國,總百揆。



 以冀州之勃海、長樂、安德、武邑;瀛州之河間、高陽、章武;定州之中山、常山、博陵十
 郡,邑二十萬戶,加九錫殊禮,齊王如故。丙辰,魏帝遜位別宮,又使兼太尉彭城王韶、兼司空敬顯俊奉冊禪位,致璽書於帝,并奉皇帝璽綬,禪代之禮,一依唐、虞、漢、魏故事。帝累表固辭,詔不許。於是尚書令高隆之率百僚勸進。



 天保元年夏五月戊午,皇帝即位於南郊,升壇,柴燎告天。是日,鄴下獲赤雀,獻于郊所。事畢還宮,御太極前殿,大赦,改元。百官進兩大階;六州緣邊職人三大階。自魏孝莊已後,百官絕祿,至是復給焉。己未,詔封魏帝為中山王。追尊皇祖文穆王為文穆皇帝;皇祖妣為文穆皇后;皇考獻武王為獻武皇帝;皇兄文襄王為文襄
 皇帝。命有司議祖宗以聞。辛酉,尊王太后為皇太后。乙酉,降魏朝封爵各有差。其信都從義,及宣力霸朝者,又西來人,并武定六年以來南來投化者,不在降限。辛未,遣大使於四方觀察風俗,問人疾苦。甲戌,遷神主於太廟。六月辛巳,詔改封崇聖侯孔長為恭聖侯,邑一百戶,以奉孔子祀。并下魯郡,以時修葺廟宇。



 又詔:吉凶車服制度,各為等差,具立條式,使儉而獲中。分遣使人致祭於五岳、四瀆,其堯祠、舜廟下及孔父、老君等載於祀典者,咸秩罔遺。又詔:冀州之勃海、長樂二郡,先帝始封之國,義旗初起之地,并州之太原、青州之齊郡,霸朝所在,
 王命是基。君子有作,貴不忘本,齊郡、勃海,可並復一年,長樂復二年,太原復三年。壬午,詔故太傅孫騰、故太保尉景、故大司馬婁昭、故司徒高敖曹、故尚書左僕射慕容紹宗、故領軍萬俟干、故定州刺史段榮、故御史中尉劉貴、故御史中尉竇泰、故殷州刺史劉豐、故濟州刺史蔡俊等,並左右先帝,經贊皇基,或不幸早殂,或隕身王事,可遣使者就墓致祭,并撫問妻子。又詔封宗室,太尉高岳為清河王,太保高隆之為平原王,開府儀同三司高歸彥為平秦王,徐州刺史高思宗為上洛王,營州刺史高長弼為廣武王,兼武衛將軍高普為武興王,兼武
 衛將軍高子瑗為平昌王,兼北中郎將高顯國為襄樂王,前太子庶子高睿為趙郡王,揚州縣開國公高孝緒為修城王。又詔封功臣,太師庫狄乾為章武王,大司馬斛律金為咸陽王,並州刺史賀拔仁為安定王,殷州刺史韓軌為安德王,瀛州刺史可朱渾道元為扶風王,司徒公彭樂為陳留王,司空公潘相樂為河東王。癸未,詔封諸弟,青州刺史浚為永安王;尚書左僕射淹為平陽王,定州刺史浟為彭城王,儀同三司演為常山王,冀州刺史渙為上黨王,儀同三司淯為襄城王,儀同三司湛為長廣王,湝為任城王,湜為高陽王,濟為博陵王,凝為
 新平王,潤為馮翊王,洽為漢陽王。丁亥,詔立王子殷為皇太子,王后李氏為皇后。庚寅,詔以太師庫狄乾為太宰,司徒彭樂為太尉,司空潘相樂為司徒,開府儀同三司司馬子如為司空。己亥,以皇太子初入東宮,赦畿內及并州死罪已下,降餘州死罪已下囚。秋七月辛亥,尊文襄妃元氏為文襄皇后,宮曰靜德。



 又封文襄子孝琬為河間王,孝瑜為河南王。乙卯,以尚書令、平原王封隆之為錄尚書事,尚書左僕射、平陽王淹為尚書令,改御史中尉還為中丞。詔魏御府所有珍奇雜彩常所不給人者,悉送內後園,以供七日宴賜。八月,詔郡國修立黌
 序,廣延髦俊,敦述儒風。其國子學生,亦依舊銓補。往者文襄皇帝所運蔡邕石經五十二枚,移置學館,依次修立。又詔求直言正諫之士,待以不次。命牧人之官,廣勸農桑。



 庚寅,詔曰:「朕以虛薄,嗣弘王業,思所以贊揚盛績,播之萬古。雖史官執筆,有聞無墜,猶恐緒言遺美,時或未書。在位王公、文武大小,降及庶人,爰至僧徒,或親奉音旨,或承傳旁說,凡可載之文籍,悉條封上。」甲午,詔曰:「魏世議定《麟趾格》,遂為通制,官司施用,猶未盡善。群官可更論討新令。未成之間,仍以舊格從事。」九月癸丑,以領東夷校尉、遼東郡開國公、高麗王成為使持節、侍中、
 驃騎大將軍、領護東夷校尉,王、公如故。丁卯,詔以梁侍中、使持節、假黃鉞、都督中外諸軍事、大將軍、承制邵陵王蕭綸為梁王。庚午,幸晉陽。是日,皇太子入居涼風堂,監國。冬十月己卯,法駕,御金輅,入晉陽宮,朝皇太后於內殿。



 辛巳,曲赦并州太原郡晉陽縣及相國府四獄囚。乙酉,以特進元紹為尚書左僕射,并州刺史段紹為右僕射。壬辰,罷相國府,留騎兵、外兵曹,各立一省,別掌機密。



 十一月,周文帝帥師至陜城,分騎北度至建州。甲寅,梁湘東王蕭繹遣使朝貢。丙寅,帝親戎出次城東,周文帝見軍容嚴盛,歎曰:「高歡不死矣!」遂班師。十二月辛丑,
 車駕至自晉陽。是歲,高麗、蠕蠕、吐谷渾、庫莫奚並遣使朝貢。



 二年春正月丁未,梁湘東王蕭繹遣使朝貢。辛亥,祀圓丘,以神武皇帝配。癸亥,親耕籍田。乙丑,享太廟。二月壬辰,太尉彭樂謀反,伏誅。三月丙午,襄城王淯薨。己未,詔梁承制湘東王繹為梁使持節、假黃鉞、相國,建梁臺,總百揆、承制梁王。庚申,司空司馬子如坐事免。是月,梁交、梁、義、新四州刺史,各以地內附。西魏文帝崩。夏四月壬辰,梁王蕭繹遣使朝貢。六月庚午,以前司空司馬子如為太尉。秋七月己卯,改顯陽殿還為昭陽殿。辛卯,改殷
 州為趙州以避太子之諱。是月,侯景廢梁簡文帝,立蕭棟為主。九月壬申,免諸伎作屯牧雜色役隸之徒為白戶。癸巳,行幸趙、定二州,因至晉陽。冬十月戊申,起宣光、建始、嘉福、仁壽諸殿。庚申,蕭繹遣使朝貢。丁卯,文襄皇帝神主入于廟。十一月,侯景廢梁主棟,僭即偽位於建鄴,自稱曰漢。十二月,中山王殂。是歲,蠕蠕、室韋、高麗並遣使朝貢。



 三年春正月丙申,帝親討庫莫奚於代郡,大破之,以其口配山東為百姓。二月,蠕蠕主阿那瑰為突厥所破。瑰自殺。其太子庵羅辰及瑰從弟登注俟刑、登注子庫提
 並擁眾來奔。蠕蠕餘眾立注次子鐵伐為主。辛丑,契丹遣使朝貢。三月戊子,詔清河王岳、司徒潘相樂、行臺辛術帥師南伐。癸巳,詔進梁王蕭繹為梁主。夏四月壬申,東南道行臺辛術於廣陵送傳國八璽。甲申,以吏部尚書楊愔為尚書右僕射。六月己亥,清河王岳等班師。乙卯,車駕幸晉陽。冬十月乙未,次黃櫨嶺。仍起長城,北至社於戍,四百餘里,立三十六戍。十一月辛巳,梁主蕭繹即位於江陵,是為元帝,遣使來聘。十二月壬子,車駕還宮。戊午,幸晉陽。是歲,西魏廢帝元年。



 四年春正月丙子,山胡圍離石戍,帝親討之。未至而逃。
 因巡三堆戍,大狩而旋。戊寅,庫莫奚遣使朝貢。自魏末用永安錢,又有數品,皆輕濫。己丑,鑄新錢,文曰常平五銖。二月,送蠕蠕鐵伐父登注及子庫提還北。鐵伐尋為契丹所殺。國人復立登注為主。仍為其大人阿富提等所殺,國人復立庫提為主。夏四月,車駕還宮。



 戊午,西南有大聲如雷。五月庚午,校獵於林慮山。戊子,還宮。六月甲辰,章武王庫狄乾薨。秋,北巡冀、定、幽、安,仍北討契丹。冬十月丁酉,車駕至平州,遂西道趣長塹。甲辰,帝步踰山嶺,為士卒先,指麾奮擊,大破契丹。是行也,帝露頭袒身,晝夜不息,行千餘里。唯食肉飲水,氣色彌厲。丁巳,登
 碣石山,臨滄海。十一月己未,帝自平州還,遂如晉陽。閏月壬寅,梁人來聘。十二月己未,突厥復攻蠕蠕,蠕蠕舉國來奔。癸亥,帝北討突厥,迎納蠕蠕。乃廢其主庫提,立阿那瑰子庵羅辰為主,置之馬邑川。追突厥於朔方,突厥請降,許之而還。自是貢獻相繼。



 五年春正月癸丑,帝討山胡大破之。男子十二已上皆斬,女子及幼弱以賞軍。



 遂平石樓。石樓絕險,自魏代所不能至。於是遠近山胡,莫不懾伏。是役也,有都督戰傷,其什長路暉禮不能救,帝命刳其五藏,使九人分食之,肉及穢惡皆盡。自是始行威虐。是月,周文帝廢西魏
 帝而立齊王廓,是為恭帝。三月,蠕蠕庵羅辰叛,帝親討大破之,辰父子北遁。太保賀拔仁坐違緩,拔其髮,免為庶人,使負炭輸晉陽宮。夏四月,蠕蠕寇肆州。丁巳,帝自晉陽討之,至恒州。時虜騎散走,大軍已還,帝帥麾下二千餘騎為殿,夜宿黃瓜堆。蠕蠕別部數萬騎,扣鞍而進,四面圍逼。



 帝安睡,平明方起。神色自若,指畫軍形,潰圍而出。虜走,追擊之,伏尸二十里。



 獲庵羅辰妻子、生口三萬餘。五月丁亥,地豆干、契丹並遣使朝貢。丁未,北討蠕蠕,又大破之。六月,蠕蠕遠遁。秋七月戊子,肅慎遣使朝貢。壬辰,降罪人。庚戌,至自北伐。八月庚午,以司州牧、清
 河王岳為太保,以安德王軌為大司馬,以扶風王可朱渾道元為大將軍,以司空尉粲為司徒,以太子少師侯莫陳相為司空,以尚書令、平陽王淹為錄尚書事,以常山王演為尚書令,以上黨王渙為尚書右僕射。



 丁丑,行幸晉陽。辛巳,錄尚書事、平原王高隆之薨。封冀州刺史段韶為平原王。



 是月,詔常山王演、上黨王渙、清河王岳、平原王段韶率眾於洛陽西南築伐惡城、新城、嚴城、河南城四鎮。九月,帝親自臨幸,欲以致西師。西師不出,乃如晉陽。



 冬十月,西魏攻陷江陵,殺梁元帝。梁將王僧辯在建業,推其晉安王蕭方智為太宰、都督中外諸軍事、
 承制置百官。十二月庚申,車駕北巡,至達速嶺,親覽山川險要,將起長城。是歲,西魏恭帝元年。



 六年春正月壬寅,清河王岳度江,剋夏首。梁司徒、郢州刺史陸法和請降。詔以梁貞陽侯蕭明為梁主,遣尚書右僕射、上黨王渙送之江南。二月甲子,以陸法和為使持節、都督十州諸軍事、太尉、大都督、西南道大行臺。三月丙戌,上黨王渙剋東關,斬梁將裴之橫。丙申,車駕至自晉陽。封文襄二子,孝珩為廣寧王,延宗為安德王。戊戌,帝臨昭陽殿決獄。是月,發寡婦以配軍士築長城。夏五月,蕭明入于建業。六月甲子,河東王潘相樂薨。壬申,
 帝親討蠕蠕。甲戌,諸軍大會祁連池。乙亥,出塞,至庫狄谷。百餘里無水泉,六軍渴乏,俄而大雨。秋七月己卯,帝頓白道,留輜重,親率輕騎五千,追蠕蠕。壬午,及之懷朔鎮。帝躬犯矢石,頻大破之,遂至沃野。壬辰,還晉陽。九月己卯,車駕至自晉陽。冬十月,梁將陳霸先襲殺王僧辯,廢蕭明,復立蕭方智為主。辛亥,行幸晉陽。十一月,梁秦州刺史徐嗣徽、南豫州刺史任約等襲據石頭城,並以州內附。壬辰,大都督蕭軌帥眾至江,遣都督柳達摩等度江,鎮石頭。己亥,太保、清河王岳薨。柳達摩為霸先攻逼,以石自頭降。是歲,高麗、庫莫奚並遣使朝貢。詔發夫一
 百八十萬人築城,自幽州北夏口,西至恒州,九百餘里。



 七年春正月辛丑,封司空侯莫陳相為白水郡王。車駕至自晉陽。於鄴城西馬射,大集眾庶觀之。二月辛未,詔常山王演等於涼風堂讀尚書奏案,論定得失,帝親決之。三月丁酉,大都督蕭軌等帥眾濟江。夏四月乙丑,儀同三司婁睿討魯陽蠻,大破之。丁卯,造金華殿。五月,漢陽王洽薨。帝以肉為斷慈,遂不復食。六月乙卯,蕭軌等與梁師戰於鐘山西,遇霖雨失利,軌及都督李希光、王敬寶、東方老、軍司裴英起並沒,士卒還者十二三。乙丑,梁湘州刺史王琳獻馴象。秋七月。乙亥,周文帝殂。是月,
 發山東寡婦二千六百人配軍士,有夫而濫奪者十二三。十一月壬子,併省州三,郡一百五十三。縣五百八十九,鎮三,戍二十六。十二月庚子,魏恭帝遜位於周。是歲,庫莫奚、契丹遣使朝貢。修廣三臺宮殿。先是,自西河總秦戍築長城東至海。前後所築,東西凡三千餘里,六十里一戍,其要害置州鎮凡二十五所。



 八年春三月,大熱,人或慄死。夏四月庚午,詔禁取蝦蟹蜆蛤之類,唯許私家捕魚。乙酉,詔公私禁取鷹鷂。以太師、咸陽王斛律金為右丞相,以前大將軍、扶風王可朱渾道元為太傅,以開府儀同三司賀拔仁為太保,尚書
 令、常山王演為司空,以錄尚書事、長廣王湛為尚書令,以尚書右僕射楊愔為左僕射,以并省尚書左僕射崔暹為右僕射,以上黨王渙為錄尚書事。是月,帝在城東馬射,敕京師士女悉赴觀,不赴省,罪以軍法,七日乃止。五月辛酉,冀州人劉向於鄴謀逆,黨與皆伏誅。秋八月己巳,庫莫奚遣使朝貢。庚辰,詔丘郊禘祫時祭,皆市取少牢,不得刲割,有司監視,必令豐備。農社、先蠶,酒肉而已。雩、衣某、風、雨、司人、司祿、靈星雜祀,果餅酒脯。唯當務盡誠敬,義同如在。辛巳,制榷酤。自夏至九月,河北六州、河南十三州、畿內八郡大蝗,飛至鄴,蔽日,聲如風雨。甲辰,
 詔今年遭蝗處,免租。冬十月乙亥,梁主蕭方智遜位於陳。陳武帝遣使稱籓朝貢。是歲,周閔帝元年。周塚宰宇文護殺閔帝而立明帝,又改元焉。初於長城內築重城。庫洛拔而東,至於塢紇戍,凡四百餘里。



 九年春二月丁亥,降罪人。己丑,詔燎野限以仲冬,不得他時行火,損昆蟲草木。三月丁酉,車駕至自晉陽。夏四月辛巳,大赦。是月,北豫州刺史司馬消難以城叛于周。大旱,帝以祈雨不降,毀西門豹祠,掘其塚。五月辛丑,以尚書令、長廣王湛為錄尚書事,以驃騎大將軍、平秦王歸彥為右僕射。甲辰,以前左僕射楊愔為尚書令。六月
 乙丑,帝自晉陽北巡。己巳,至祁連池。戊寅,還晉陽。是夏,山東大蝗,差人夫捕而坑之。秋七月辛丑,給畿內老人劉奴等九百四十三人版職及杖帽,各有差。戊申,詔趙、燕、瀛、定、南營五州,及司州廣平、清河二郡,去年螽澇損田,兼春夏少雨,苗稼薄者,免今年租稅。八月乙丑,車駕至自晉陽。甲戌,行幸晉陽。先是,發丁匠三十餘萬人營三臺於鄴,因其舊基而高博之,大起宮室及遊豫園。至是,三臺成。改銅爵曰金鳳,金武曰聖應,冰井曰崇光。冬十一月甲午,車駕至自晉陽。登三臺,御乾象殿,朝宴群臣。以新宮成,丁酉,大赦內外,文武官並進一大階。丁巳,
 梁湘州刺史王琳遣使請立蕭莊為梁主,仍以江州內屬,令莊居之。十二月癸酉,詔以梁王蕭莊為梁主,進居九派。戊寅,以太傅可朱渾道元為太師,以司徒尉粲為太尉,以冀州刺史段韶為司空,以錄尚書事、常山王演為大司馬,以錄尚書事、長廣王湛為司徒。起大莊嚴寺。是歲,殺永安王浚、上黨王渙。



 十年春正月戊戌,以司空侯莫陳相為大將軍。辛丑,太尉長樂郡公尉粲、肆州刺史濮陽公婁仲遠並進爵為王。甲寅,行幸遼陽甘露寺。二月丙戌,帝於甘露寺禪居深觀,唯軍國大政奏聞。三月戊戌,以侍中高德正為尚
 書右僕射。丙辰,車駕至自遼陽。是月,梁主蕭莊至郢州,遣使朝貢。夏閏四月丁酉,以司州牧、彭城王浟為兼司空,以侍中、高陽王水是為尚書左僕射。乙巳,以兼司空、彭城王浟為兼太尉,攝司空事,封皇子紹廉為長樂王。五月癸未,誅始平公元世、東平公元景式等二十五家,禁止特進元韶等十九家。尋並誅之,男子無少長皆斬,所殺三千人,並投漳水。六月,陳武帝殂。秋八月戊戌,封皇子紹義為廣陽王。以尚書右僕射、河間王孝琬為左僕射。癸卯,詔諸軍人,或有父祖改姓冒入元氏,或假託攜認,妄稱姓元者,不問世數遠近,悉聽改復本姓。是月,殺
 左僕射高德正。九月己巳,行幸晉陽。冬十月甲午,帝暴崩於晉陽宮德陽堂,時年三十一。遺詔,凶事一從儉約,喪月之斷,限以三十六日。嗣子百僚,內外遐邇,奉制割情,悉從公除。癸卯,發喪,僉於宣德殿。十一月辛未,梓宮還鄴。十二月乙酉,殯於太極前殿。乾明元年二月丙申,葬於武寧陵,謚曰文宣帝,廟號顯祖。



 帝沈敏有遠量,外若不遠,內鑒甚明。文襄年長英秀,神武特所愛重,百僚承風,莫不震懼。而帝善自晦跡,言不出口,恒自貶退,言咸順從。故深見輕,雖家人亦以為不及。文襄嗣業,帝以次長見猜嫌。帝后李氏色美,每預宴會,容貌遠過靖德
 皇后,文襄彌不平焉。帝每為后私營服玩,小佳,文襄即令逼取。后恚,有時未與。帝笑曰:「此物猶應可求,兄須,何容吝。」文襄或愧而不取,便恭受,亦無飾讓。每退朝還第,輙閉閣靜坐,雖對妻子,能竟日不言。或袒跣奔躍。后問其故,對曰:「為爾漫戲。」此蓋習勞而不肯言也。所寢至夜曾有光,巨細可察,后驚告帝。帝曰:「慎勿妄言。」自此唯與后寢,侍御皆令出外。文襄崩,祕不發喪。



 其後漸露,魏帝竊謂左右曰:「大將軍此殂,似是天意,威權當歸王室矣。」及帝將赴晉陽,親入辭謁於昭陽殿,從者千人,居前持劍者十餘輩。帝在殿下數十步立,而衛士升階已二百
 許人,皆攘袂扣刃,若對嚴敵。帝令主者傳奏,須詣晉陽。言訖,再拜而出。魏帝失色,目送帝曰:「此人似不能見容,吾不知死在何日。」及至并州,慰諭將士,措辭款實。眾皆欣然,曰:「誰謂左僕射翻不減令公。」令公即指文襄也。時訛言上黨出聖人。帝聞之,將從一郡。而郡人張思進上言,殿下生於南宮,坊名上黨,即是上黨出聖人,帝悅而止。先是童謠曰:「一束槁,兩頭然,河邊羖䍽飛上天。」槁然兩頭,於文為高。河邊羖䍽為水邊羊,指帝名也。於是徐之才盛陳宜受禪。帝曰:「先父亡兄,功德如此,尚終北面,吾又何敢當。」之才曰:「正為不及父兄,須早升九五。如其
 不作,人將生心,且讖云:「羊飲盟津角掛天。」盟津,水也,羊飲水,王名也,角挂天,大位也。又陽平郡界面星驛傍有大水,土人常見群羊數百,立臥其中,就視不見。事與讖合,願王勿疑。」帝以問高德正。德正又贊成之,於是始決。乃使李密卜之,遇《大橫》,曰:「大吉,漢文帝之封也。」帝乃鑄象以卜之,一寫而成。使段韶問斛律金於肆州,金來朝,深言不可,以鎧曹宗景業首陳符命,請殺之。乃議於太后前。太后謂諸貴曰:「我兒獰直,必自無此意,直高德正樂禍,教之耳。」帝意決,乃整兵而東。使高德正之鄴,諷喻公卿,莫有應者。司馬子如逆帝於遼陽,固言未可。杜弼
 亦抱馬諫。帝欲還,尚食丞李集曰:「此行事非小,而言還?」帝偽言使向東門殺之,而別令賜絹十疋。四月,夜,禾生於魏帝銅研,旦長數寸,有穗。五月,帝復東赴鄴,令左右曰:「異言者斬。」是月,光州獻九尾狐。帝至鄴城南,召入,并齎板策。旦,高隆之進謁曰:「用此何為?」帝作色曰:「我自作事,若欲族滅耶!」隆之謝而退。於是乃作圓丘,備法物,草禪讓事。



 及登極之後,神明轉茂,外柔內剛,果於斷割,人莫能窺。又特明吏事,留心政術,簡靖寬和,坦於任使。故楊愔等得盡於匡贊,朝政粲然。兼以法馭下,不避權貴。或有違犯,不容勛戚,內外莫不肅然。至於軍國機策,獨決
 懷抱,規謀宏遠,有人君大略。又以三方鼎峙,繕甲練兵,左右宿衛,置百保軍士。每臨行陣,親當矢石。鋒刃交接,唯恐前敵不多。屢犯艱厄,常致剋捷。嘗追及蠕蠕,令都督高阿那肱率騎數千,塞其走道。時虜軍猶盛,五萬餘人。肱以兵少請益,帝更減其半騎。



 那肱奮擊。遂大破之。虜主踰越巖谷,僅以身免。都督高元海、王師羅並無武藝,先稱怯弱;一旦交鋒,有踰驍壯。嘗於東山游宴,以關隴未平,投盃震怒。召魏收於前,立為詔書,宣示遠近,將事西行。是歲,周文帝殂,西人震恐,常為度隴之計。



 既征伐四剋,威振戎夏。六七年後,以功業自矜。遂留情耽湎,
 肆行淫暴。或躬自鼓舞,歌謳不息,從旦通宵,以夜繼晝;或袒露形體,塗傅粉黛,散髮胡服,雜衣錦彩,拔刃張弓,游行市肆。勛戚之第,朝夕臨幸。時乘鹿車、白象、駱駝、牛、驢,並不施鞍勒。或盛暑炎赫,日中暴身;隆冬酷寒,去衣馳走。從者不堪,帝居之自若。街坐巷宿,處處游行。多使劉桃枝、崔季舒負之而行。或擔胡鼓而拍之。親戚貴臣,左右近習,侍從錯雜,無復差等。徵集淫嫗,悉去衣裳,分付從官,朝夕臨視。或聚棘為馬,紐草為索,逼遣乘騎,牽引來去,流血灑地,以為娛樂。



 凡諸殺害,多令支解。或焚之於火,或投之於河。沈酗既久,彌以狂惑。每至將醉,輒
 拔劍挂手,或張弓傅矢,或執持牟槊。游行高廛阜,問婦人曰:「天子何如?」



 答曰:「顛顛癡癡,何成天子。」帝乃殺之。或馳騁衢路,散擲錢物,恣人拾取,爭競喧譁,方以為喜。



 太后嘗在北宮,坐一小榻。帝時已醉,手自舉床,后便墜落,頗有傷損。醒悟之後,大懷慚恨。遂令多聚柴火,將入其中。太后驚懼,親自持挽。又設地席,令平秦王高歸彥執杖,口自責疏,脫背就罰。敕歸彥:「杖不出血,當即斬汝。」太后涕泣,前自抱之。帝流涕苦請,不肯受於太后。太后聽許,方捨背杖,笞腳五十,莫不至到。衣冠拜謝,悲不自勝,因此戒酒。一旬,還復如初。自是耽湎轉劇。遂幸李后家,以
 鳴鏑射后母崔,正中其頰。因罵曰:「吾醉時尚不識太后,老婢何事!」



 馬鞭亂打一百有餘。三臺構木高二十七丈,兩棟相距二百餘尺,工匠危怯,皆繫繩自防。帝登脊疾走,都無怖畏。時復雅舞,折旋中節;傍人見者,莫不寒心。又召死囚,以席為翅,從臺飛下,免其罪戮。果敢不慮者,盡皆獲全。疑怯猶豫者,或致損跌。



 沈酗既久,轉虧本性。怒大司農穆子容,使之脫衣而伏,親射之。不中,以橛貫其下竅,入腸。雖以楊愔為宰輔,使進廁籌。以其體肥,呼為楊大肚。馬鞭鞭其背,流血浹袍。以刀子剺其腹槁。崔季舒託俳言曰:「老小公子惡戲?」因掣刀子而去之。又置愔
 於棺中,載以轜車,幾下釘者數四。曾至彭城王浟宅,謂其母爾朱曰:「憶汝辱我母婿時,向何由可耐。」手自刃殺。又至故僕射崔暹第,謂暹妻李曰:「頗憶暹不?」李曰:「結髮義深,實懷追憶。」帝曰:「若憶時,自往看也。」



 親自斬之,棄頭牆外。嘗在晉陽,以槊戲刺都督尉子耀,應手而死。在三臺太光殿上,鋸殺都督穆嵩。又幸開府暴顯家,有都督韓哲無罪,忽眾中召,斬之數段。魏樂安王元昂,后之姊婿。其妻有色,帝數幸之,欲納為昭儀。召昂令伏,以鳴鏑射一百餘下,凝血垂將一石,竟至於死。後帝自往弔哭,於喪次逼擁其妻。仍令從官脫衣助襚,兼錢彩,號為信
 物。一日所得,將踰巨萬。后啼不食,乞讓位於姊。太后又為言,帝意乃釋。所幸薛嬪,甚被寵愛。忽意其經與高岳私通,無故斬首,藏之於懷。於東山宴,勸酬始合,忽探出頭,投於柈上。支解其屍,弄其為琵琶。一座驚怖,莫不喪膽。帝方收取,對之流淚云:「佳人難再得,甚可惜也。」載屍以出,被髮步哭而隨之。至有閭巷庸猥,人無識知者,忽令召斬鄴下。繫徒罪至大辟,簡取隨駕,號為供御囚,手自刃殺,持以為戲。凡所屠害,動多支解,或投之烈火,或棄之漳流。兼以外築長城,內營臺殿,賞費過度,天下騷然;內外慘慘,各懷怨毒。而素嚴斷臨下,加之默識強
 記,百僚戰慄,不敢為非。曾有典御丞李集面諫,比帝有甚於桀紂。帝令縛置流中。沈沒久之,復令以出,謂曰:「吾何如桀紂?」



 集曰:「回來彌不及矣。」帝又令沈之,引出更問,如此數四,集對如初。帝大笑曰:「天下有如此癡漢!方知龍逢、比干,非是俊物。」遂解放之。又被引入見,似有所諫,帝令將出腰斬。其或斬或赦,莫能測焉。



 初,帝登阼,改年為天保。士有深識者曰:「天保之字,為一大人只十,帝其不過十乎。」又先是謠云:「馬子入石室,三千六百日。」帝以午年生,故曰「馬子」。三臺,石季龍舊居,故曰「石室」。三千六百日,十年也。又,帝曾問太山道士曰:「吾得幾年為天子?」
 答曰:「得三十年。」道士出後,帝謂李后曰:「十年十月十日,得非三十也?吾甚畏之,過此無慮。人生有死,何得致惜,但憐正道尚幼,人將奪之耳。」帝及期而崩,濟南竟不終位,時以為知命。曾幸晉陽,夜宿杠門嶺。嶺有數株栢樹,皆將千年,枝葉嫩茂,似有神物所託。時帝已被酒,向嶺瞋罵。射中一株,未幾,立枯而死。又出言屢中,時人故謂之神靈。雖為猖獗,不專云昏暴。末年遂不能進食,唯數飲酒,曲蘗成災,因而致斃。先是,霍州發楚夷王女塚,尸如生焉。得珠襦玉匣,帝珍之,還以斂焉。始祖珽以險薄多過,帝數罪之,每謂為老賊。及武成時,珽被任遇,乃說
 武成曰:「文宣甚暴,何得稱文?



 既非創業,何得稱祖?若宣帝為祖,陛下萬歲後將何以稱?」武成溺於珽說,天統初,有詔改謚景烈,廟號威宗。武平初,趙彥深執政,又奏復帝本謚,廟號顯祖云。



 廢帝殷,字正道,小名道人,文宣帝之長子也。母曰李皇后。天保元年,立為皇太子,時年六歲。性敏慧,初學反語,於跡字下注云「自反」。時侍者未達其故,太子曰:「跡字足傍亦為跡,豈非自反邪。」嘗宴北宮,獨令河間王勿入,左右問其故,太子曰:「世宗遇賊處,河間王復何宜在此。」文宣每言:「太子得漢家性質,不似我」,欲廢之,立太原王。初
 詔國子博士李寶鼎傅之。寶鼎卒,復詔國子博士邢峙侍講。太子雖富於春秋,而溫裕開朗,有人君之度。貫綜經業,省覽時政,甚有美名。七年冬,文宣召朝臣文學者及禮學官於宮宴會,令以經義相質,親自臨聽。太子手筆措問,在坐莫不歎美。九年,文宣在晉陽,太子監國,集諸儒講《孝經》。令楊愔傳旨謂國子助教許散愁曰:「先生在世,何以自資?」對曰:「散愁自少以來,不登孌童之床,不入季女之室,服膺簡策,不知老之將至。平生素懷,若斯而已。」太子曰:「顏子縮屋稱貞,柳下嫗而不亂,未若此翁白首不娶者也。」



 乃齎絹百疋。後文宣登金鳳臺,召太子
 使手刃囚。太子惻然有難色,再三不斷其首。



 文宣怒,親以馬鞭撞太子三下。由是氣悸語吃,精神時復昏擾。



 十年十月,文宣崩,癸卯,太子即帝位於晉陽宣德殿,大赦。內外百官普加汎級,亡官失爵,聽復資品。庚戌,尊皇太后為太皇太后,皇后為皇太后。詔九州軍人七十已上,授以板職。武官年六十已上,及癃病不堪驅使者,並皆放免。土木營造金銅鐵諸雜作工,一切停罷。十一月乙卯,以右丞相、咸陽王斛律金為左丞相。



 以錄尚書事、常山王演為太傅。以司徒、長廣王湛為太尉。以司空段韶為司徒,以平陽王淹為司空。高陽王湜為尚書左僕射,
 河間王孝琬為司州牧,侍中燕子獻為右僕射。戊午,分命使者,巡省四方,求政得失,省察風俗,問人疾苦。十二月戊戌,改封上黨王紹仁為漁陽王,廣陽王紹義為范陽王,長樂王紹廣為隴西王。是歲,周武成元年。



 乾明元年春正月癸丑朔,改元。己未,詔寬徭賦。癸亥,高陽王湜薨。是月,車駕至自晉陽。己亥,以太傅、常山王演為太師、錄尚書事,以太尉、長廣王湛為大司馬、并省錄尚書事。以尚書左僕射、平秦王歸彥為司空,趙郡王睿為尚書左僕射。詔諸元良口配沒宮內及賜人者,並放免。甲辰,帝幸芳林園,親錄囚徒,死罪已下,降免各有差。
 乙巳,太師、常山王演矯詔誅尚書令楊愔、尚書右僕射燕子獻、領軍大將軍可朱渾天和、侍中宋欽道、散騎常侍鄭子默。戊申,以常山王演為大丞相、都督中外諸軍、錄尚書事。以大司馬、長廣王湛為太傅、京畿大都督。以司徒段韶為大將軍。以前司空、平陽王淹為太尉。以司空、平秦王歸彥為司徒,彭城王浟為尚書令。又以高麗王世子湯為使持節、領東夷校尉、遼東郡公、高麗王。是月,王琳為陳所敗,蕭莊自拔至和州。三月甲寅,詔軍國事皆申晉陽,稟大丞相常山王規算。壬申,封文襄第二子孝珩為廣寧王,第三子長恭為蘭陵王。夏四月癸亥,
 詔河南定、冀、趙、瀛、滄、南膠、光、南青九州,往因螽水,頗傷時稼,遣使分塗贍恤。是月,周明帝崩。五月壬子,以開府儀同三司劉洪徽為尚書右僕射。秋八月壬午,太皇太后令廢帝為濟南王,全食一郡。以大丞相、常山王演入纂大統。是日,王居別宮。皇建二年九月,殂於晉陽,時年十七。



 帝聰慧夙成,寬厚仁智,天保間,雅有令名。及承大位,楊愔、燕子獻、宋欽道等同輔。以常山王地親望重,內外畏服。加以文宣初崩之日,太后本欲立之,故愔等並懷猜忌。常山王憂悵,乃白太后,誅其黨。時平秦王歸彥亦預謀焉。皇建二年秋,天文告變,歸彥慮有後害,仍白
 孝昭,以王當咎,乃遣歸彥馳駟至晉陽害之。



 王薨後,孝昭不豫,見文宣為祟。孝昭深惡之,厭勝術備設而無益也。薨三旬而孝昭崩。大寧二年,葬於武寧之西北,謚閔悼王。初,文宣命邢邵制帝名殷字正道,從而尤之,「殷家弟及,『正』字一止,吾身後兒不得也。」邵懼,請改焉。文宣不許,曰:「天也。」因謂昭帝曰:「奪時但奪,慎勿殺也。」



 孝昭皇帝演字延安,神武皇帝第六子,文宣皇帝之母弟也。幼而英峙,早有大成之量,武明皇太后早所愛重。魏元象元年,封常山郡公。及文襄執政,遣中書侍郎李同軌就霸府為諸弟師。帝所覽文籍,源其指歸,而不好
 辭彩。每歎云:「雖盟津之師左驂震而不衄」,以為能。遂篤志讀《漢書》,至《李陵傳》,恒壯其所為焉。聰敏過人。所與游處,一知其家諱,終身未嘗誤犯。同軌病卒,又命開府長流參軍刁柔代之,性嚴褊,不適誘訓之宜,中被遣出。帝送出閣,慘然斂容,淚數行下,左右莫不歔欷。其敬業重舊如此。



 天保初,進爵為王。五年,除并省尚書令。帝善斷割,長思理,省內畏服。七年,從文宣還鄴。文宣以尚書奏事,多有異同,令帝與朝臣先論定得失,然後敷奏。



 帝長於政術,割斷咸盡其理,文宣歎重之。八年,轉司空、錄尚書事。九年,除大司馬,仍錄尚書事。時文宣溺於游宴,帝
 憂憤,表於神色。文宣覺之,謂帝曰:「但令汝在,我何為不縱樂?」帝唯啼泣拜伏,竟無所言。文宣亦大悲,抵盃於地曰:「汝似嫌我,自今敢進酒者斬之!」因取所御盃,盡皆壞棄。後益沈湎,或入諸貴戚家,角力批拉,不限貴賤。唯常山王至,內外肅然。帝又密撰事條,將諫。



 其友王晞以為不可。帝不從,因間極言,遂逢大怒。順成后本魏朝宗室,文宣欲帝離之。陰為帝廣求淑媛,望移其寵。帝雖承旨有納,而情義彌重。帝性頗嚴,尚書郎中剖斷有失,輒加捶楚,令史姦慝,便即考竟。文宣乃立帝於前,以刀環擬脅。



 召被立罰者,臨以白刃,求帝之短,咸無所陳,方見解
 釋。自是不許笞箠郎中。後賜帝魏時宮人,醒而忘之。謂帝擅取,遂令刀環亂築,因此致困。皇太后日夜啼泣,文宣不知所為。先是禁友王晞,乃捨之,令侍帝。帝月餘漸瘳,不敢復諫。及文宣崩,帝居禁中護喪事。幼主即位,乃即朝班。除太傅、錄尚書事,朝政皆決於帝。



 月餘,乃居籓邸。自是,詔敕多不關帝。客或言於帝曰:「鷙鳥捨巢,必有探卵之患,今日之地,何宜屢出。」



 乾明元年,從廢帝赴鄴,居於領軍府。時楊愔、燕子獻、可朱渾天和、宋欽道、鄭子默等以帝威望既重,內懼權逼,請以帝為太師、司州牧、錄尚書事;長廣王湛為大司馬、錄并省尚書事,解京畿
 大都督。帝既以尊親而見猜斥,乃與長廣王期獵,謀之於野。三月甲戌,帝初上省。旦,發領軍府,大風暴起,壞所御車幔。帝甚惡之。及至省,朝士咸集。坐定,酒數行,於坐執尚書令楊愔、右僕射燕子獻、領軍可朱渾天和、侍中宋欽道等於坐。帝戎服與平原王段韶、平秦王高歸彥、領軍劉洪徽入自雲龍門,於中書省前遇散騎常侍鄭子默,又執之,同斬於御府之內。帝至東閣門,都督成休寧抽刃呵帝。帝令高歸彥喻之,休寧厲聲大呼不從。歸彥既為領軍,素為兵士所服,悉皆弛杖。休寧方歎息而罷。帝入至昭陽殿,幼主、太皇太后、皇太后並出臨御坐。
 帝奏愔等罪,求伏專擅之辜。時庭中及兩廊下衛士二千餘人,皆被甲待詔。武衛娥永樂武力絕倫,又被文宣重遇,撫刃思效。廢帝吃訥,兼倉卒,不知所言。太皇太后又為皇太后誓,言帝無異志,唯云逼而已。高歸彥敕勞衛士戒嚴,永樂乃內刀而泣。帝乃令歸彥引侍衛之士向華林園,以京畿軍入守門閣,斬娥永樂於園。詔以帝為大丞相、都督中外諸軍、錄尚書事。相府佐史進位一等。帝尋如晉陽。有詔,軍國大政,咸諮決焉。帝既當大位,知無不為,擇其令典,考綜名實。廢帝恭己以聽政。太皇太后尋下令廢少主,命帝統大業。



 皇建元年八月壬午,皇帝即位於晉陽宣德殿。大赦,改乾明元年為皇建。詔奉太皇太后還稱皇太后,皇太后稱文宣皇后,宮曰昭信。乙酉,詔自太祖創業已來,諸有佐命功臣,子孫絕滅,國統不傳者,有司搜訪近親,以名聞,當量為立後。諸郡國老人,各授板職,賜黃帽鳩杖。又詔謇正之士,並聽進見陳事;軍人戰亡死王事者,以時申聞,當加榮贈。督將朝士名望素高,位歷通顯,天保以來未蒙追贈者。



 亦皆錄奏。又以延尉、中丞,執法所在,繩違案罪,不得舞文弄法。其官奴婢年六十已上,免為庶人。戊子,以太傅、長廣王湛為右丞相。以太尉、平陽王淹
 為太傅。



 以尚書令、彭城王浟為大司馬。壬辰,詔分遣大使,巡省四方。觀察風俗,問人疾苦,考求得失,搜訪賢良。甲午,詔曰:「昔武王剋殷,先封往代。兩漢魏晉,無廢茲典。及元氏統歷,不率舊章。朕纂承大業,思弘古典。但二王三恪,舊說不同,可議定是非,列名條奏。其禮儀體式,亦仰議之。」又詔國子寺可備立官屬,依舊置生,講習經典,歲時考試。其文襄帝所運石經,宜即施列於學館。外州大學,亦仰典司,勤加督課。丙申,詔九州勛人有重封者,聽分授子弟,以廣骨肉之恩。九月壬申,詔議定三祖樂。冬十一月辛亥,立妃元氏為皇后,世子百年為皇太子。
 賜天下為父後者,爵一級。癸丑,有司奏太祖獻武皇帝廟宜奏《武德之樂》,舞《昭烈之舞》;太宗文襄皇帝廟宜奏《文德之樂》,舞《宣政之舞》;高祖文宣皇帝廟宜奏《文正之樂》,舞《光大之舞》。詔曰:「可。」庚申,詔以故太師尉景、故太師竇泰、故太師太原王婁昭、故太宰章武王庫狄干、故太尉段榮、故太師萬俟普、故司徒蔡俊、故太師高乾、故司徒莫多婁貸文、故太保劉貴、故太保封祖裔、故廣州刺史王懷十二人配饗太祖廟庭;故太師清河王岳、故太宰安德王韓軌、故太宰扶風王可朱渾道元、故太師高昂、故大司馬劉豐、故太師萬俟受洛干、故太尉慕容紹
 宗十一人配饗世宗廟庭;故太尉河東王潘相樂、故司空薛修義、故太傅破六韓常三人配饗高祖廟庭。是月,帝親戎北討庫莫奚,出長城。虜奔遁,分兵致討,大獲牛馬,括總入晉陽宮。十二月丙午,車駕至晉陽。



 二年春正月辛亥,祀圓丘。壬子,禘於太廟。癸丑,詔降罪人各有差。二月丁丑,詔內外執事之官從五品已上、及三府主簿錄事參軍、諸王文學、侍御史、廷尉三官、尚書郎中、中書舍人,每二年之內,各舉一人。冬十月丙子,以尚書令、彭城王浟為太保,長樂王尉粲為太尉。己酉,野雉栖于前殿之庭。十一月甲辰,詔曰:「朕嬰此暴疾,奄忽
 無逮。今嗣子沖眇,未閑政術,社稷業重,理歸上德。右丞相、長廣王湛,研機測化,體道居宗,人雄之望,海內瞻仰,同胞共氣,家國所憑。可遣尚書左僕射、趙郡王睿喻旨,徵王統茲大寶。其喪紀之禮,一同漢文,三十六日,悉從公除。山陵施用,務從儉約。」先是,帝不豫而無闕聽覽,是日,崩於晉陽宮,時年二十七。大寧元年閏十二月癸卯,梓宮還鄴,上謚曰孝昭皇帝。庚午,葬於文靜陵。



 帝聰敏有識度,深沈能斷,不可窺測。身長八尺,腰帶十圍,儀望風表,迥然獨秀。自居臺省,留心政術,閑明簿領,吏所不逮。及正位宸居,彌所克勵,輕徭薄賦,勤恤人隱。內無私
 寵,外收人物,雖后父,位亦特進無別。日昃臨明,務知人之善惡。每訪問左右,冀獲直言。曾問舍人裴澤在外議論得失,澤率爾對曰:「陛下聰明至公,自可遠侔古昔,而有識之士,咸言傷細,帝王之度,頗為未弘。」



 帝笑曰:「誠如卿言。朕初臨萬機,慮不周悉,故致爾耳。此事安可久行,恐後又嫌疏漏。」澤因被寵遇。其樂聞過也如此。趙郡王睿與庫狄顯安侍坐,帝曰:「須拔我同堂弟,顯安我親姑子,今序家人禮,除君臣之敬,可言我之不逮。」顯安曰:「陛下多妄言。」曰:「若何?」對曰:「陛下昔見文宣以馬鞭撻人,常以為非,而今行之,非妄言邪?」帝握其手謝之。又使直言,
 對曰:「陛下太細,天子乃更似吏。」帝曰:「朕甚知之,然無法來久,將整之以至無為耳。」又問王晞,晞,答如顯安,皆從容受納。性至孝。太后不豫,出居南宮,帝行不正履,容色貶悴,衣不解帶,殆將四旬。殿去南宮五百餘步,雞鳴而去,辰時方還,來去徒行,不乘輿輦。太后所苦小增,便即寢伏閣外,食飲藥物,盡皆躬親。太后嘗心痛,不自堪忍,帝立侍帷前,以爪掐手心,血流出袖。友愛諸弟,無君臣之隔。雄勇有謀。于時國富兵強,將雪神武遺恨,意在頓駕平陽,為進取之策。遠圖不遂,惜哉。



 初,帝與濟南約,不相害。及輿駕在晉陽,武成鎮鄴。望氣者云「鄴城有天子
 氣。」帝恐濟南復興,乃密行鴆毒。濟南不從,乃扼而殺之。後頗愧悔。初苦內熱,頻進渴散。時有尚書令史姓趙,於鄴見文宣從楊愔、燕子獻等西行,言相與復仇。



 帝在晉陽宮,與毛夫人亦見焉。遂漸危篤,備禳厭之事,或煮油四灑,或持炬燒逐。



 諸厲方出殿梁,山騎棟上,歌呼自若,了無懼容。時有天狗下,乃於其所講武以厭之,有兔驚馬,帝墜而絕肋。太后視疾,問濟南所在者三,帝不對。太后怒曰:「殺去邪!不用吾言,死其宜矣。」臨終之際,唯扶服床枕,叩頭求哀。遣使詔追長廣王入纂大統。又手書云:「宜將吾妻子置一好處,勿學前人也。」



 論曰:神武平定四方,威權在己。遷鄴之後,雖主祭有人,號令所加,政皆自出。文宣因循鴻業,內外葉從。自朝及野,群心屬望。東魏之地,舉國樂推,曾未期月,遂登宸極。始則存心政事,風化肅然,數年之間,朝野安乂。其後縱酒肆欲,事極猖狂,昏邪殘暴,近代未有。饗國不永,實由斯疾。濟南繼業,大革其弊,風教粲然,搢紳稱幸。股肱輔弼,雖懷厥誠,既不能贊弘道德,和睦親懿;又不能遠慮防身,深謀衛主。應斷不斷,自取其災。臣既誅夷,君尋廢辱,皆任非其器之所致爾。孝昭早居臺閣,故事通明;人吏之間,無所不委。文宣崩後,大革前弊。及臨尊極,留心
 更深,時人服其明而譏其細也。情好稽古,率由禮度;將封先代之胤,且敦學校之風;徵召才賢,文武畢集。於時周氏朝政,移於宰臣;主將相猜,不無危殆。乃眷關右,實懷兼並之志。經謀宏曠,諒近代之明主。而降年不永,其故何哉?豈幽顯之塗,別有復報;將齊之基宇,止在於斯。帝欲大之,天不許也?



\end{pinyinscope}