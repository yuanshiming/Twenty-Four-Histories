\article{卷七十七列傳第六十五}

\begin{pinyinscope}

 裴政李諤鮑宏高構榮毗陸知命梁毗柳彧趙綽杜整裴政,字德表,河東聞喜人也。祖邃,父之禮,並《南史》有傳。政幼聰明,博聞強記,達於從政,為當世所稱。仕梁,以軍功封為夷陵侯,給事黃門侍郎。及魏軍圍荊州,政在外見獲,蕭察謂政曰:「我,武皇帝之孫,不可為爾君乎?爾何煩殉身於七父。若從我計,則貴及子孫,不然,分腰領矣。」
 鎖之,送至城下,使謂元帝曰:「王僧辯聞臺城破,已自為帝。王琳孤弱,不能復來。」政許之。既而告城中曰:「援兵大至,吾以間使被禽,當以碎身報國。監者擊其口,終不易辭。



 察怒,命趣行戮。蔡大業諫曰:「此人之望也,殺之,則荊州不可下。因得釋。會江陵平,與城中朝士俱送京師。周文聞其忠,援員外散騎侍郎,引入相府。命與盧辯依《周禮》建六官,并攢次朝儀,車服器用,多遵古禮,革漢、魏之法,事並施行。尋授刑部下大夫,轉少司憲。政明習故事,又參定周律。能飲酒,至數斗不亂。



 簿案盈几,剖決如流,用法寬平,無有冤濫。囚徒犯極刑者,乃許其妻子入獄
 就之。



 至冬,將行決,皆曰:「裴大夫致我於死,死無所恨。」又善鐘律,嘗與長孫紹遠論樂,事在《紹遠傳》。



 隋開皇元年,為率更令,加上儀同三司。詔與蘇威等修定律令。採魏、晉刑典,下至齊、梁,沿革輕重,取其折衷。同撰著者十餘人,凡疑滯不通,皆取決於政。



 進位散騎常侍,轉左庶子。多所匡正,見稱純愨,東宮凡有大事,皆以委之。右庶子劉榮,性甚專固。時武職交番,通事舍人趙元愷作辭見帳,未及成。太子再三催促,榮令元愷口奏,不須造帳。及奏,太子問:「名帳安在?」元愷云:「稟承劉榮,不聽造帳。」太子即以詰榮,榮便拒諱,太子付政推問。未及奏狀,阿附榮
 者先言於太子曰:「政欲陷榮,推事不實。」太子召責之,政曰:「凡推事有兩,一察情,一據證,審其曲直,以定是非。臣察榮位高任重,縱實語元愷,蓋是纖介之愆,計不須諱。又察元愷,受制於榮,豈敢以無端之言妄相點累。二人之情理正相似。元愷引左衛率崔茜等證,茜款狀悉與元愷符同。察情既敵,須以證定。臣謂榮語元愷非虛。」太子亦不罪榮,而稱政平直。



 政好面折人短,而退無後言。時雲定興數入侍太子,為奇服異器,進奉後宮,又緣女寵,來往無節。政數切諫,太子不納。政謂定興曰:「公所為不合禮度。又元妃暴薨,道路籍籍,此於太子非令名也。
 願公自引退,不然將及禍。」定興怒,以告太子,太子益疏政。由是出為襄州總管,妻子不之官,所受秩奉,散給僚吏。



 人犯罪者,陰悉知之,或竟歲不發,至再三犯,乃因都會時,於眾中召出,親案其罪,五人處死,流、徒者甚眾。合境惶懾,令行禁止,稱為神明,爾後不修囹圄,殆無諍訟。卒於官。著《承聖實錄》十卷。及太子廢,文帝追憶之曰:「向遣裴政、劉行本在,共匡弼之,猶應不令至此。」



 子南金,位膳部郎,學涉有文藻,以輕財貴義稱。



 李諤,字士恢,趙郡人也。博學解屬文。仕齊,為中書舍人,有口辯,每接對陳使。周平齊,拜天官都上士。諤見隋文
 帝有帝王志操,深自結納。及帝為丞相,甚見親待,訪以得失。時兵革屢動,國用虛耗,諤上《重穀論》以諷焉。帝納之。



 及受禪,歷比部、考功二曹侍郎,賜爵南和伯。諤性公方,明時務。遷書侍御史。



 上謂群臣曰:「朕昔為大司馬,每求外職,李諤陳十二策,苦勸不許,朕遂決意在內。今此事業,諤之力也。」賜物二千段。



 諤見禮教凋弊,公卿薨亡,其愛妾侍婢,子孫輒嫁賣之,遂成風俗,乃上書曰:「臣聞追遠慎終,人德歸厚,三年無改,方稱為孝。如聞大臣之內,有父祖亡沒,日月未久,子孫無賴,引其妓妾,嫁賣取財,有一於此,實損風化。妾雖微賤,親承衣履,服斬三年,
 古今通式。豈容遽褫衰絰,強傅鉛華,泣辭靈几之前,送付他人之室?凡在見者,猶致傷心,況乎人子,能堪斯忍!復有朝廷重臣,位望通貴,平生交舊,情若弟兄。及其亡沒,杳同行路,朝聞其死,夕規其妾,方便求娉,以得為限。無廉恥之心,棄友朋之義。且居家理務,可移於官,既不正私,何能贊務?」



 上覽而嘉之。五品已上妻妾不得改醮,始於此也。



 諤又以時文體尚輕薄,流宕忘反,上書曰:臣聞古先哲王之化人也,必變其視聽,防其嗜慾,塞其邪放之心,示以淳和之路。五教六行,為訓人之本,《詩》、《書》、《禮》、《易》,為道義之門。故能家復孝慈,人知禮讓,正俗調風,莫
 大於此。其有上書獻賦,制誄鐫銘,皆以褒德序賢,明勳證理。茍非懲勸,義不徒然。降及後代,風教漸落。魏之三祖,更尚文詞,忽君人之大道,好彫蟲之小藝。下之從上,有同影響,競騁文華,遂成風俗。



 江左齊、梁,其弊彌甚,貴賤賢愚,唯務吟詠。遂復遺理存異,尋虛逐微,競一韻之奇,爭一字之巧。連篇累牘,不出月露之形,積案盈箱,唯是風雲之狀。世俗以此相高,朝廷據茲擢士。祿利之路既開,愛尚之情愈篤。於是閭里童昏,貴游總卯,未窺六甲,先製五言。至如羲皇、舜、禹之典,伊、傅、周、孔之說,不復關心,何嘗入耳。以傲誕為清虛,以緣情為勳績,指儒素
 為古拙,用詞賦為君子。故文筆日繁,其政日亂,良由棄大聖之軌模,構無用以為用也。捐本逐末,流遍華壤,遞相師祖,久而愈扇。



 及大隋受命,聖道聿興,屏黜浮詞,遏止華偽。自非懷經抱持,志道依仁,不得引預搢紳,參廁纓冕。開皇四年,普詔天下,公私文翰,並宜實錄。其年九月,泗州刺史司馬幼之文表華艷,付所司推罪。自是公卿大臣感知正道,莫不鑽仰墳素,棄絕華綺,擇先王之令典,行大道於茲世。



 如聞外州遠縣,仍踵弊風,選吏舉人,未遵典則。宗黨稱孝,鄉曲歸仁,學必典謨,交不茍合,則擯落私門,不加收齒;其學不稽古,逐俗隨時,作輕薄
 之篇章,結朋黨而求譽。則選充吏職,舉送天朝。蓋由縣令、刺史,未行風教,猶挾私情,不存公道。臣既忝憲司,職當糾察。若聞風即劾,恐挂綱者多,請勒有司,普加搜訪,有如此者,具狀送臺。



 諤又以當官者好自矜伐,復上奏具陳其弊。請加罪黜,以懲風軌。上以諤前後所奏頒示天下,四海靡然向風,深革其弊。諤在職數年,務存大體,不尚嚴猛,由是無剛謇之譽,而潛有匡正之志。



 邳公蘇威以臨道店舍,乃求利之徒,事業污雜,非敦本之義。遂奏約遣歸農。



 有願依舊者,在所州縣,錄附市籍,仍撤毀舊店,並令遠道,限以時日。時逢冬塞,莫敢陳訴。諤因別
 使,見其如此,以農工有業,各附所安,逆旅之與旗亭,自古非同一概,即附市籍,於理不可。且行旅之所依託,豈容一朝而廢?徒為勞擾,於事非宜。遂專決之,並令依舊。使還詣闕,然後奏聞。文帝善之曰:「體國之臣,當如此矣。」以年老,出拜通州刺史,甚有惠政,人夷悅服。卒官。



 四子。世子大方襲爵,最有才器。大業初,判內史舍人。次大體、大鈞,並位尚書郎。



 鮑宏,字潤身,東海郯人也。父機,以才學知名。仕梁,位書侍御史。宏七歲而孤,為兄泉之所愛育。年十二,能屬文,嘗和湘東王繹詩,繹嗟賞不已,引為中記室。累遷通直
 散騎侍郎。江陵平,歸于周,明帝甚禮之,引為麟趾殿學士。累遷遂伯下大夫。與杜子暉聘陳,謀伐齊,陳遂出兵度江以侵齊。帝嘗問宏取齊策,宏以為「先皇往日,出師洛陽,彼有其備,每不克捷。如臣計者,進兵汾、潞,直掩晉陽,出其不虞,以為上策。」帝從之。及定山東,除小御正,賜爵平遙縣伯,加儀同。隋文帝作相,奉使山南。會王謙舉兵於蜀,路次潼州,為謙將達奚惎所執,逼送成都,竟不屈節。謙敗,馳傳入京,文帝嘉之,賜以金帶。及受禪,加開府,進爵為公。歷利、邛二州刺史,秩滿還京。時有尉義臣者,其父崇不從尉遲迥,從復與突厥戰死。上嘉之,將賜姓
 金氏。訪及群下,宏曰:「昔項伯不同項羽,漢高賜其姓劉氏,秦真父能死難,魏武賜姓曹氏。請賜以皇族。」帝曰:「善。」因賜義臣姓楊。後授均州刺史,以目疾免,卒于家。



 初,周武帝敕宏修《皇室譜》一部,分為《帝緒》、《疏屬》、《賜姓》三篇。



 有集十卷,行於世。



 高構,字孝基,北海人也。性滑稽多智,辯給過人,好讀書,工吏事。仕齊,歷蘭陵、平原二郡太守。齊滅,周武帝以為許州司馬。隋文帝受禪,累遷戶部侍郎。



 時內史侍郎晉平東與兄子長茂爭嫡,尚書省不以斷,朝臣三議不決。構斷而合理,上以為能,召入內殿,勞之曰:「我聞尚書郎
 上應列宿,觀卿才識,方知古人之言信矣。嫡庶者,禮教之所重,我讀卿判數遍,詞理愜當,意所不能及也。」賜米百石。由是知名。



 馮翊武鄉女子焦氏既啞又聾,嫁之不售。嘗樵菜於野,為人所犯而有孕,遂生一男。時年六歲,莫知其姓,於是申省。構判曰:「母不能言,窮究理絕。案《風俗通》,姓有九種,或氏於爵,或氏所居。此兒生在武鄉,可以武為姓。」尋遷雍州司馬,以明斷見稱。歲餘,轉吏部侍郎,號為稱職。復徙雍州司馬,坐事左轉盩厔令,甚有能名。上善之,復拜雍州司馬。仁壽初,又為吏部侍郎,以公事免。



 煬帝立,召令復位。時為吏部者多以不稱去職,唯構
 最有能名,前後典選之官,皆出其下,時人以構好劇談,頗謂輕薄,然其內懷方雅,特為吏部尚書牛弘所重。



 後以老病解職,弘時典選,凡將有所擢用,輒遣人就第問其可不。河東薛道衡才高當世,每稱構有清鑒,所為文筆,必先以草呈觀構而後出之。構有所詆訶,道衡未嘗不嗟伏。大業七年,終于家。所舉薦杜如晦、房玄齡等,後皆自致公輔,論者稱構有知人之鑒。



 開皇中,昌黎豆盧實為黃門會郎,稱為慎密。河東裴術為右丞,多所糾正。河內士燮、平原東方舉、安定皇甫聿道,俱為刑部,並執法平允。京兆韋焜為戶部郎,屢進讜言。南陽韓則為延州,
 甚有惠政。此等事行遺闕,皆有吏乾,為當時所稱。



 榮毗,字子諶,北平無終人也。父權,魏兵部尚書。毗少剛鯁,有局量,涉獵群言。仕周,位內史下士。隋開皇中,累遷殿內局監。時以華陰多盜賊,妙選長史,楊素薦毗為華州長史,世號為能。素之田宅,多在華陰,左右放縱,毗以法繩之,無所寬貸。毗因朝集,素謂之曰:「素之舉卿,適以自罰也?」毗答曰:「奉法一心者,但恐累公所舉。」素笑曰:「前言戲耳。卿之奉法,素之望也。」時晉王在揚州,每令人密覘京師消息,遣張衡於路次往往置馬坊,以畜牧為辭,實給私人也。



 州縣莫敢違,毗獨遏絕其事。上聞而嘉之,
 賚絹百匹,轉蒲州司馬。



 漢王諒之反也,河東豪傑以城應諒。刺史丘和覺變,遁歸關中。長史渤海高義明謂毗曰:「河東國之東門,若失之,則為難不細。在中雖復匈匈,非悉反也。但收桀黠者十餘人斬之,自當立定耳。」毗然之。義明馳馬追和,將與協計。至城西門,為渤海所殺,毗亦被執。及諒平,拜書侍御史,帝謂曰:「今日之舉,馬坊之事也。無改汝心。」帝亦敬之。毗在朝侃然正色,為百僚所憚。後以母憂去職。歲餘,起令視事。尋卒官。贈鴻臚少卿。



 毗兄建緒,性甚亮直,兼有學業。仕周,為載師下大夫、儀同三司。及平齊之始,留鎮鄴城,因著《齊紀》三十卷。建緒
 與文帝有舊,及為丞相,加位開府,拜息州刺史。將之官,時帝陰有禪代之計,因謂建緒曰:「且躊躇,當共取富貴耳。」



 建緒自以周之大夫,因義形於色曰:「明公此旨,非僕所聞。」帝不悅。建緒遂行。



 開皇初來朝,上謂之曰:「卿亦悔不?」建緒稽首曰:「臣位非徐廣,情類楊彪。」



 上笑曰:「朕雖不解書語,亦知卿此言不遜也。」兼始、洪二州刺史,俱有能名。



 陸知命,字仲通,吳郡富春人也。父敖,陳散騎常侍。知命性好學,通識大體,以貞介自持。仕陳,為太學博士、南獄正。及陳滅,歸於家。會高智慧等作亂于江左,晉王廣鎮
 江都,以其三吳之望召令諷諭反者。以功拜儀同三司,賜以田宅,復用其弟恪為汧陽令。知命以恪非百里才,上表陳讓,朝廷許之。時見天下一統,知命詣朝堂上表,請使高麗以宣示皇風,使彼君臣面縛闕下。書奏,天子異之。歲餘,授普寧鎮將。人或言其正直者。由是待詔於御史臺。煬帝嗣位,拜書侍御史,侃然正色,為百僚所憚。帝甚敬之。後坐事免。歲餘,復職。時齊王暕頗驕縱,暱近小人,知命奏劾之,暕竟得罪,百僚震慄。遼東之役,為東暆道受降使者,卒於師。



 贈御史大夫。



 梁毗,字景和,安定烏氏人也。祖越,魏涇、豫、洛三州刺史,
 郃陽縣公。父茂,周滄、兗二州刺史。毗性剛謇,頗有學涉。仕周,累遷布憲下大夫。宣政中,封易陽縣子,遷武藏大夫。隋文帝受禪,進爵為侯。開皇初,以鯁正,拜書侍御史,名為稱職。轉大興令,遷雍州贊務。毗既出憲司,復典京邑,直道而行,無所回避,頗失權貴心,由是出為西寧州刺史,改封邯鄲縣侯。在州十一年。



 先是,蠻夷酋長皆服金冠,以金多者為豪俊,由是遞相陵辱,每尋干戈,邊境略無寧歲。毗患之,後因諸酋長相率以金遺之,於是置金座側,對之慟哭,謂曰:「此飢不可食,寒不可衣,汝等以此相滅。今將此來,欲殺我邪!」無所納,悉以還之。於是
 蠻夷感悟,遂不相攻。文帝聞而善之,徵為散騎常侍、大理卿。處法平允,時人稱之。歲餘,進位上開府。毗見左僕射楊素貴重擅權,百僚震懾,恐為國患,因上封事曰:「竊見左僕射越國公素,幸遇愈重,權勢日隆,所私皆非忠讜,所進咸是親戚,子弟布州,兼州連縣。天下無事,容息姦圖,四海稍虞,必為禍始。



 夫姦臣擅命,有漸而來。王莽資之於積年,桓玄基之於易世,而卒殄漢祀,終傾晉祚。陛下若以素為阿衡,臣恐其心未必伊尹也。」帝大怒,命有司禁止,親自詰之。



 毗極言曰:「素既擅權寵,作威作福,將領之處,殺戮無道。又太子、蜀王罪廢之日,百僚無不震
 悚,唯素揚眉奮肘,喜見容色,利國家有事以為身幸。」毗發言謇謇,有誠亮之節,帝無以屈也,乃釋之。素自此恩寵漸衰。但素任寄隆重,多所折挫,當時朝士無不懾伏;有敢與相是非,辭氣不撓者,獨毗與柳彧及尚書左丞李綱而已。後上不復專委於素,蓋由察毗之言。



 煬帝即位,遷刑部尚書,並攝御史大夫事。奏劾宇文述和私役部兵,帝議免述罪,毗固爭,因忤旨,遂令張衡代為大夫。毗憂憤卒。帝令吏部尚書牛弘弔之。



 子敬真,位大理司直。時煬帝欲成光祿大夫魚俱羅罪,令敬真案其獄,遂希旨陷之極刑。未幾,敬真有疾,見俱羅為祟而死。



 柳彧,字幼文,河東人也。世居襄陽。父仲禮,《南史》有傳。仲禮,梁敗見囚于周,復家河東。彧少好學,頗涉經史。周大冢宰宇文護引為中外府記室,久而出為寧州總管掾。武帝親總萬機,彧詣闕求試。帝異之,以為司武中士。轉鄭令。



 平齊之後,帝賞從官,留京者不預。彧上表曰:「今太平告始,信賞宜明,酬勳報勞,務先有本。屠城破邑,出自聖規,斬將搴旗,必由神略。若負戈擐甲,徵扞劬勞。至於鎮撫國家,宿衛為重。俱稟成算,非專己能,留從事同,功勞須等。」於是留守並加品級。



 隋文帝受禪,歷尚書虞部、屯田二侍郎。時制三品已上,門皆列戟。左僕射高熲
 子弘德封應國公,申牒請戟。彧判曰:「僕射之子更不異居,父之戟槊已列門外,尊有厭卑之義,子有避父之禮,豈容外門既設,內閣又施?」事竟不行。熲聞而歎伏。後遷書侍御史,當朝正色,甚為百僚敬憚。上嘉其婞直,謂曰:「大丈夫當立名於世,無容容而已。」賜錢十萬,米百石。



 時刺史多任武將,類不稱職,彧上表曰:「伏見詔書以上柱國和乾子為杞州刺史,其人年垂八十。鐘鳴漏盡。前在趙州,暗於職務,政由群小,賄賂公行。百姓吁嗟,歌謠滿道,乃云:『老禾不早殺,餘種穢良田。』古人云:『耕當問奴,織當問婢。』此言各有所能也。乾子弓馬武用,是其所長;臨
 人蒞職,非其所解。如謂優老尚年,自可厚賜金帛,若令刺舉,所損殊大。臣死而後已,敢不竭誠。」上善之,乾子竟免。有應州刺史唐君明,居母喪,娶雍州長史厙狄士文之從父妹。彧劾之曰:「君明忽劬勞之痛,惑嬿爾之親,冒此苴縗,命彼褕翟。不義不暱,《春秋》載其將亡:無禮無儀,詩人欲其遄死。士文贊務神州,名位通顯,棄二姓之重匹,違六禮之軌儀。請禁錮終身,以懲風俗。」二家竟坐得罪。隋承喪亂之後,風俗頹壞,彧多所矯正,上甚嘉之。又見上勤於聽受,百僚奏請多有煩碎,因上疏諫曰:「人君出令,誡在煩數。是以舜任五臣,堯咨四岳,設官分職,各
 有司存,垂拱無為,天下以乂。所謂勞於求賢,逸於任使。比見事無大小,咸關聖職。陛下留心政道,無憚疲勞,至乃營造細小之事,出給輕微之物,一日之內,酬答百司,至乃日旰忘食貧,分夜未寢,動以文簿,憂勞聖躬。伏願思臣至言,少減煩務。」上覽而嘉之。以其家,敕有司與之築宅,因曰:「柳彧正直之士,國之龜寶也。」其見重如此。



 右僕射楊素當途顯貴,百僚懾憚,無敢忤者,嘗以少譴,敕送南臺。素恃貴,坐彧床。彧從外來,見素如此,於階下端笏整容曰:「奉敕推公罪。」素遽下。彧據案坐,立素於庭前,辯詰事狀。素由是銜之。彧時方為上所信任,故素未有
 以中之。



 彧見近代以來,都邑百姓每至正月十五日,作角抵戲,遞相誇競,至於糜費財力,上奏請禁絕之曰:「竊見京邑,爰及外州,每以正月望夜,充街塞陌,鳴鼓聒天,燎炬照地,人戴獸面,男為女服,倡優雜伎,詭狀異形。外內共觀,曾不相避。



 竭貲破產,競此一時。盡室並孥,無問貴賤,男女混雜,緇素不分。穢行因此而生,盜賊由斯而起。非益於化,實損於人。請頒天下,並即禁斷。」詔可其奏。



 是歲,持節巡河北五十二州,奏免長吏贓汙不稱職者二百餘人,州縣肅然,莫不震懼。上嘉之,賜絹布二百匹,氈三十領,拜儀同三司,歲餘,加員外散騎常侍。



 仁壽初,
 持節巡省太原道十九州。及還,賜絹百五十匹。



 彧嘗得博陵李文博所撰《政道集》十卷,蜀王秀遣人求之。彧送之於秀,秀復賜彧奴婢十口。及秀得罪,楊素奏彧以內臣交通諸侯,除名,配戍懷遠鎮。行達高陽,有詔徵進。至晉陽,遇漢王諒作亂,遣使馳召彧入城。而諒反形已露,彧入城,度不得免,遂詐中惡不食,自稱危篤。諒怒囚之。及諒敗,楊素奏彧心懷兩端,以候事變,迹雖不反,心實同逆。坐徙敦煌。素卒,乃自申理,有詔徵還。卒於道。



 有子紹,為介休令。



 趙綽,字士倬,河東人也。性質直剛毅。周初為天官府史,
 以恭謹恪勤,擢授夏官府下士。稍以明乾見知,為內史中士。父艱去職,哀毀骨立,世稱其孝。隋文帝為丞相,知其清正,引為錄事參軍。遷掌朝大夫,從行軍總管是云暉擊叛蠻,以功拜儀同。



 文帝受禪,授大理丞。處法平允,考績連最。歷大理正、尚書都官侍郎,每有奏讞,正色侃然,漸見禮重。上以盜賊不禁,將重其法,綽進諫曰:「律者天下之大信,其可失乎!」上忻然納之,因謂曰:「若更有聞見,宜數言之。」遷大理少卿。



 故陳將蕭摩訶,其子世略在江南作亂,摩訶當從坐。上曰:「世略年未二十,亦何能為!以其名將之子,為人逼耳。」因赦摩訶。綽固諫不可,上不
 能奪,欲待綽去而赦之,因命綽退食。綽曰:「臣奏獄未決,不敢退朝。」上曰:「大理其為朕特放摩訶也。」因命左右釋之。刑部侍郎辛亶嘗衣緋褌,俗云利官,上以為厭蠱,將斬之,綽曰:「據法不當死,臣不敢奉詔。」上怒甚,謂曰:「卿惜辛亶而不自惜也?」命左僕射高熲將綽斬之。綽曰:「陛下寧可殺臣,不可殺辛亶。」至朝堂,解衣當斬。上使人謂綽曰:「竟如何?」對曰:「執法一心,不敢惜死。」上拂衣入,良久乃釋之。明日,謝綽,勞勉之,賜物三百段。



 時上禁行惡錢,有二人在市以惡錢易好者,武候執以聞,上悉令斬之。綽諫曰:「此人坐當杖,殺之非法。」上曰:「不關卿事。」綽曰:「陛下
 不以臣愚闇,置在法司,欲妄殺人,豈得不關臣事?」上曰:「撼大木不動者,當退。」對曰:「臣望感天心,何論動木!」上復曰:「啜羹者,熱則置之。天子之感,欲相挫邪?」



 綽拜而益前,訶之不肯退。上遂入。書侍御史柳彧復上奏切諫,上乃止。上以綽有誠直之心,每引入閣中,或遇上與皇后同榻,即呼綽坐,評論得失。前後賞賜以萬計。後進開府,贈其父為蔡州刺史。



 時河東薛胄為大理卿,俱名平恕。然胄斷獄以情,而綽守法,俱為稱職。上每謂綽曰:「朕於卿無所愛惜,但卿骨相不當貴耳。」仁壽中,卒官,上為之流涕,中使弔祭,鴻臚監護喪事。二子元方、元襲。



 杜整,字皇育,京兆杜陵人也。祖盛,魏潁川太守。父闢,渭州刺史。整少有風概,九歲丁父憂,哀毀骨立,事母以孝聞。及長,驍勇有膂力,好讀《孫吳兵法》。



 魏大統末,襲爵武鄉侯。周文引為親信。累遷儀同三司、武州刺史。從武帝平齊,加上儀同,進爵平原縣公,入為勳曹中大夫。隋文帝為丞相,進位開府。及帝受禪,加上開府,進封長廣郡公,拜左武衛將軍。開皇六年,突厥犯塞,詔衛王爽北伐,以整為行軍總管,兼元帥長史。至合川,無虜而還。密進取陳策,上善之,以為行軍總管,鎮襄陽。卒,上傷之,謚曰襄。



 子楷嗣,位開府。



 整弟肅,亦有志行,位北地太守。



 論曰:大廈之構,非一本之枝,帝王之功,非一士之略,長短殊用,大小異宜,咨棁棟梁,莫可棄也。裴政、李諤、鮑宏、高構、榮毗、陸知命等,或文能道義,或才足乾時,識用顯於當年,故事留於臺閣。參之有隋多士,取其開物成務,皆廊廟之榱桷,亦北辰之眾星也。趙綽居大理,囹圄無冤。柳彧之處憲臺,奸邪自肅。



 然不畏御,梁毗得之矣。邦之司直,柳彧近之矣。杜整以聲績著美,其有以取之乎!



\end{pinyinscope}