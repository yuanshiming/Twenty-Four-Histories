\article{卷十八列傳第六 景穆十二王下}

\begin{pinyinscope}

 任城王雲,和平五年封。少聰慧,年五歲,景穆崩,號哭不絕聲。太武抱之泣曰:「汝何知而有成人意也!」獻文時,拜都督中外諸軍事、中都大官,聽訟,甚收時譽。及獻文欲禪位於京兆王子推,王公卿士莫敢先言。雲進曰:「父子相傳久矣,皇魏未之有革。」太尉源賀又進以為不可,願思任城之言。東陽公元丕等進曰:「皇太子雖聖德夙彰,
 然實沖幼。陛下欲隆獨善,其若宗廟何?」帝曰:「儲宮正統,群公相之,有何不可?」於是傳位孝文。



 後蠕蠕犯塞,雲為中軍大都督,從獻文討之。過大磧,雲曰:「夷狄之馬初不見武頭盾,若令此盾在前,破之必矣。」帝從之,命敕勒首領,執手勞遣之。於是相率而歌,方駕而前。大破之,獲其兇首。後仇池氐反,又命雲討平之。除開府、徐州刺史。雲以太妃蓋氏薨,表求解任。獻文不許。雲悲號動疾,乃許之。性善撫接,深得徐方之心,為百姓所追戀,送遺錢貨,一無所受。



 再遷冀州刺史,甚得下情。於是合州請戶輸絹五尺、粟五升,以報雲恩。孝文嘉之,詔宣告天下,使知
 勸勵。遷長安鎮都大將、雍州刺史。雲廉謹自修,留心庶獄,挫抑豪強,劫盜止息,州人頌之者千餘人。太和五年,薨於州,遺令薄葬,勿受贈襚,諸子奉遵其旨。謚曰康,陪葬雲中之金陵。



 長子澄,字道鏡,少好學,美鬢髮,善舉止,言辭清辯,響若縣鐘。康王薨,居喪以孝聞。襲封,加征北大將軍。以氏羌反叛,除征南大將軍、梁州刺史。文明太后引見誡厲之,顧謂中書令李沖曰:「此兒風神吐發,當為宗室領袖,是行當不辱命,我不妄也。」澄至州,誘導懷附,西南款順。加侍中,賜衣一襲,乘黃馬一匹,以旌其能。轉開府、徐州刺史,甚著聲績。朝京師,引見於皇信堂。
 孝文詔澄曰:「昔鄭子產鑄刑書而晉叔向非之。此二人皆賢士,得失竟誰?」對曰:「鄭國寡弱,攝於強鄰,人情去就,非刑莫制,故鑄刑書以示威。雖乖古式,合今權道。」



 帝方革變,深善其對,笑曰:「任城當欲為魏子產也。朕方創改朝制,當與任城共萬世之功。」後徵為中書令,改授尚書令。齊庾蓽來朝,見澄音韻遒雅,風儀秀逸,謂主客郎張彞曰:「往魏任城以武著稱,今魏任城乃以文見美也。」



 時詔延四廟之子,下逮玄孫之胄,申宗宴於皇信堂。不以爵秩為列,悉序昭穆為次,用家人之禮。帝曰:「行禮已畢、欲令宗室各言其志,可率賦詩。」特命澄為七言連韻,與
 孝文往復賭賽,遂至極歡,際夜乃罷。



 後帝外示南討,意在謀遷,齊於明堂左個。詔太常卿王諶,親令龜卜易筮南伐之事,其兆遇《革》。澄進曰:「《易》言革者更也,將欲革君臣之命,湯、武得之為吉。陛下帝有天下,今日卜征,不得云革命,未可全為吉也。」帝厲聲曰:「此象云大人武變,何言不吉也!」車駕還宮,便召澄,未及升階,遙謂曰:「向者之《革》,今更欲論之。明堂之忿,懼眾人競言,沮我大計,故厲色怖文武耳。」



 乃獨謂澄曰:「國家興自北土,徙居平城,雖富有四海,文軌未一。此間用武之地,非可興文。崤函帝宅,河洛王里,因茲大舉,光宅中原,任城意以為何如?」澄
 深贊成其事。帝曰:「任城便是我之子房。」加撫軍大將軍、太子少保,又兼尚書左僕射。及車駕幸洛陽,定遷都之策,詔澄馳驛向北,問彼百司,論擇可否。曰:「近論《革》,今真所謂革也。」澄既至代都,眾聞遷詔,莫不驚駭。澄援引今古,徐以曉之,眾乃開伏。遂南馳還報,會車駕於滑臺。帝大悅曰:「若非任城,朕事業不得就也。」從幸鄴宮。除吏部尚書。



 及車駕自代北巡,留澄銓簡舊臣。初,魏自公侯以下,動有萬數,冗散無事。



 澄品為三等,量其優劣,盡其能否之用,咸無怨者。駕還洛京,復兼右僕射。



 帝至北芒,遂幸洪池,命澄侍升龍舟。帝曰:「朕昨夜夢一老公,拜立路
 左,云晉侍中嵇紹,故此奉迎,神爽卑懼,似有求焉。」澄曰:「陛下經殷墟而弔比干,至洛陽而遺嵇紹,當是希恩而感夢。」帝曰:「朕既有此夢,或如任城所言。」於是求其兆域,遣使弔祭焉。



 齊明帝既廢弒自立,其雍州刺史曹武請以襄陽內附。車駕將自赴之,引澄及咸陽王禧、彭城王勰、司徒馮誕、司空穆亮、鎮南李沖等議之。禧等或云宜行,或言宜止。帝曰:「眾人意見不等,宜有客主,共相起發。任城與鎮南為應留之議,朕當為宜行之論。諸公坐聽,長者從之。」於是帝往復數交,駕遂南征,不從澄及李沖等言。後從征至縣瓠,以疾篤還京。



 車駕還洛,引見王公
 侍臣於清徽堂。帝曰:「此堂成來,未與王公行宴樂之禮。



 今與諸賢,欲無高而不升,無小而不入。」因之流化渠。帝曰:「此曲水者,取乾道曲成,萬物無滯。」次之洗煩池。帝曰:「此池亦有嘉魚。」澄曰:「所謂『魚在在藻,有頒其首。』」帝曰:「且取『王在靈沼,於牣魚躍。』」次之觀德殿。



 帝曰:「射以觀德,故遂命之。」次之凝閑堂。帝曰:「此堂取夫子閑居之義。不可縱奢以忘儉,自安以忘危,故此堂後作茅茨堂。」謂李沖曰:「此東曰步元廡,西曰遊凱廡。此坐雖無唐堯之君,卿等當無愧於元、凱。」沖對曰:「臣既遭唐堯之君,敢辭元、凱之譽?」帝曰:「光景垂落,朕同宗有載考之義,卿等將出,何
 得默爾德音。」即命黃門侍郎崔光、郭祚、通直郎刑巒、崔休等賦詩言志。燭至,公卿辭退,李沖再拜上於萬歲壽。帝曰:「卿等以燭至致辭,復獻於萬壽,朕報卿以《南山》之詩。」乃曰:「燭至辭退,庶姓之禮;在夜載考,宗族之義。卿等且還,朕與諸王宗室欲成此夜飲。」後坐公事免官。尋兼吏部尚書。



 恆州刺史穆泰在州謀反,授澄節,銅武、竹使符,御仗左右,仍行恆州事。行達雁門,遣書侍御史李煥先赴。至即禽泰,窮其黨與,罪人皆得。鉅鹿公陸睿、安樂侯元隆等百餘人並獄禁。具狀表聞。帝覽表,乃大悅曰:「我任城可謂社稷臣,正復皋陶斷獄,豈能過之?」顧咸陽
 王等曰:「汝等脫當其處,不能辦此也。」車駕尋幸平城。勞澄,引見逆徒,無一人稱枉。時人莫不歎之。帝謂左右曰:「必也無訟,今日見之。」以澄正尚書。



 車駕南伐,留澄居守,復兼右僕射。澄表請以國秩一歲租帛,助供軍資,詔受其半。帝復幸鄴。見公卿曰:「朕昨入城,見車上婦人冠帽而著小襦襖者,尚書何為不察?」澄曰:「著者猶少。」帝曰:「任城欲令全著乎?一言可以喪邦,其斯之謂。可命史官書之。」又曰:「王者不降佐於蒼昊,拔才而用之。朕失於舉人,任一群婦女輩,當更銓簡耳。任城在省,為舉天下綱維,為當署事而已?」澄曰:「臣實署事而已。」帝曰:「如此,便一令
 史足矣,何待任城?」尋除尚書左僕射,從駕南伐。孝文崩,受顧命。



 宣武初,有降人嚴叔懋告尚書令王肅遣孔思達潛通齊國,為叛逆。澄信之,乃表肅將叛,輒下禁止。咸陽、北海二王奏澄擅禁宰輔,免官還第。尋除開府、揚州刺史。下車封孫叔敖之墓,毀蔣子文之廟;上表請修復皇宗之學,開四門之教。詔從之。



 先是,朝議有南伐之計,以蕭寶夤為東揚州刺史,據東城;陳伯之為江州刺史,戍陽石。以澄總督二鎮,授之節度。澄於是遣統軍傅豎眼、王神念等進次大峴、東關、九山、淮陵,皆分部諸將,倍道據之。澄總勒大眾,絡繹相接,所在克捷,詔書褒美。既
 而遇雨,淮水暴長,澄引歸壽春。還既狼狽,失兵四千餘人。澄頻表解州,帝不許。有司奏奪其開府,又降三階。



 轉鎮北大將軍、定州刺史。初,百姓每有橫調,恆煩苦之。前後牧守未能蠲除,澄多所省減。又明黜陟賞罰之法,表減公園之地以給無業貧人,布絹不任衣者禁不聽造,百姓欣賴焉。母孟太妃薨,居喪過毀,當世稱之。服闋,除太子太保。



 時高肇當朝,猜忌賢戚。澄為肇間構,常恐不全,乃終日昏飲,以示荒敗。所作詭越,時謂為狂。宣武夜崩,時事倉卒,高肇擁兵於外。明帝沖幼,朝野不安。



 澄雖疏斥,而朝望所屬。領軍于忠、侍中崔光等奏澄為尚書
 令,於是眾心欣服。尋遷司空,加侍中,俄詔領尚書令。



 登表上《皇誥宗制》并《訓詁》各一卷,欲太后覽之,思勸誡之益。又奏利國濟人所宜振舉者十條:一曰律度量衡,公私不同,所宜一之;二曰宜興學校,以明黜陟之法;三曰宜興滅繼絕,各舉所知;四曰五調之外,一不煩人,任人之力,不過三日;五曰臨人之官,皆須黜陟,以旌賞罰;六曰逃亡代輸,去來年久者,若非伎作,任聽即住;七曰邊兵逃走,或實陷沒,皆須精檢,三長及近親,若實隱之,徵其代輸,不隱勿論;八曰工商世業之戶,復徵租調,無以堪濟,今請免之,使專其業;九曰三長禁姦,不得隔越相
 領,戶不滿者,隨近并合;十曰羽林武賁,邊方有事,暫可赴戰,常戍宜遣番兵代之。靈太后下其奏,百僚議之,事有同否。



 時四中郎將兵數寡弱,不足以襟帶京師。澄奏宜以東中帶滎陽郡,南中帶魯陽郡,西中帶恆農郡,北中帶河內郡,選二品、三品親賢兼稱者居之。省非急之作,配以強兵。如此則深根固本,強幹弱枝之義也。靈太后將從之,從議者不同,乃止。



 尋以疾患,表求解任,不許。



 澄以北邊鎮將選舉彌輕,恐賊虜窺邊,山陵危迫,奏求重鎮將之選,修警備之嚴,詔不從。後賊虜入寇,至於舊都,鎮將多非其人;所在叛亂,犯逼山陵,如澄所慮。



 澄奏:「
 都城府寺猶未周悉,今軍旅初寧,無宜發眾,請取諸職人及司州郡縣犯十杖以上、百鞭以下收贖之物,絹一匹輸磚二百,以漸修造。」詔從之。太傅、清河王懌表駮其事,遂寢不行。



 澄又奏:「司州牧、高陽王臣雍拷殺奉朝請韓元昭、前門下錄事姚敬賢,雖因公事,理實未盡。何者?若昭等狀彰,死罪以定,應刑於都市,與眾棄之。如其疑似不分,情理未究,不宜以三清九流之官,杖下便死,輕絕人命,傷理敗法。往年在州,於大市鞭殺五人,及檢賊狀,全無寸尺。今復酷害,一至於此。朝野云云,咸懷驚愕。若生殺在下,虐專於臣,人君之權,安所復用?請以見事
 付廷尉推究,驗其為劫之狀,察其拷殺之理。」詔從之。澄當官無所回避。又奏墾田授受之制八條,甚有綱貫。西哉嚈噠、波斯諸國,各因公使,並遺澄駿馬一匹。澄請付太僕,以充國閑。詔曰:「王廉貞之德,有過楚相,可敕付廄,以成君子大哉之美。」



 御史中尉、東平王匡奏請取景明元年以來內外考簿、吏部除書、中兵勳案並諸殿最,欲以案校竊階盜官之人。靈太后許之。澄表以為「御史之體,風聞是司。至於昌勛妄階,皆有處別。若一處有風謠,即應攝其一簿,研檢虛實。若差殊不同,偽情自露,然後繩以典刑,人誰不服?豈有移一省之事,窮革世之尤,如
 此求過,誰堪其罪?斯實聖朝所宜重慎也。」靈太后納之,乃止。後遷司徒公,侍中、尚書令如故。



 神龜元年,詔加女侍中貂蟬,同外侍中之飾。澄上表諫曰:「高祖、世宗皆有女侍中官,未見綴金蟬於象珥,極鼲貂於鬢髮。江南偽晉穆何后有女尚書而加貂榼,此乃衰亂之世,妖妄之服。且婦人而服男子之服,至陰而陽,故自穆、哀以降,國統二絕。因是劉裕所以篡逆。禮容舉措,風化之本,請依常儀,追還前詔。」帝從之。



 時太后銳於興繕,在京師則起永寧、太上公等佛寺,工費不少,外州各造五級佛圖。又數為一切齋會,施物動至萬計。百姓疲於土木之功,金
 銀之價為之踴上。



 削奪百官祿力,費損庫藏。兼曲賚左右,日有數千。澄上表極言得失。雖卒不從,常優答禮之。政無大小,皆引參預。澄亦盡心匡輔,事有不便於人者,必於諫諍,殷勤不已,內外咸敬憚之。



 二年,薨,贈假黃鉞、使持節、都督中外諸軍事、太傅、領太尉公,加以殊禮,備九錫,依晉大司馬齊王攸故事,謚曰文宣王。澄之葬也,凶飾甚盛。靈太后親送郊外,停輿悲哭,哀慟左右。百官會赴千餘人,莫不欷歔,當時以為哀榮之極。第四子彞襲。



 彞字子倫,繼室馮氏所生,頗有父風。拜通直散騎常侍。及元叉專權而彞恥於託附,故不得顯職。莊帝初,河
 陰遇害。贈儀同三司、青州刺史,謚曰文。



 彞庶長兄順,字子和。年九歲,師事樂安陳豐,初書王羲之《小學篇》數千言,晝夜誦之,旬有五日,一皆通徹。豐奇之。白澄曰:「豐十五從師,迄於白首,耳目所經,未見此比,江夏黃童不得無雙也。」澄笑曰:「藍田生玉,何容不爾。」



 十六通《杜氏春秋》,下帷讀書,篤志愛古。性謇愕,淡於榮利,好飲酒,解鼓琴。



 每長吟永歎,托詠虛室。宣武時,上《魏道頌》,文多不載。起家為給事中。時高肇權重,天下人士望塵拜伏。順曾懷刺詣肇門,門者以其年少,答云:「在坐大有貴客」。不肯為通。順叱之曰:「任城王兒可是賤也?」及見,直往登床,捧手
 抗禮,王公先達莫不怪懾;而順辭吐傲然,若無所睹。肇謂眾賓曰:「此兒豪氣尚爾,況其父乎!」及去,肇加敬送之。澄聞之大怒,杖之數十。後拜太常少卿,以父憂去職,哭泣歐血,身自負土。時年二十五,便有白髮,免喪抽去,不復更生,世人以為孝思所致。



 尋除給事黃門侍郎。時領軍元叉威勢尤盛,凡有遷授,莫不造門謝謁。順拜表而已,曾不詣叉。叉謂順曰:「卿何得聊不見我?」順正色曰:「天子富於春秋,委政宗輔,叔父宜以至公為心,舉士報國。如何賣恩,責人私謝,豈所望也!」至於朝論得失,順常鯁言正議,曾不阿旨。由此見憚,出除恆州刺史。順謂叉曰:「
 北鎮紛紜,方為國梗,請假都督,為國屏捍。」叉心疑難,不欲授以兵官,謂順曰:「此朝廷之事,非我所裁。」順曰:「叔父既殺生由己,自言天歷應在我躬,何得復有朝廷?」叉彌忿憚之。轉齊州刺史。順自負有才,不得居內,每懷鬱怏,形於言色。遂縱酒自娛,不親政事。叉解領軍,徵為給事黃門侍郎。親友郊迎,賀其得入。順曰:「不患不入,正恐入而復出耳。」俄兼殿中尚書,轉侍中。初,中山王熙起兵討元叉,不果而誅。及靈太后反政,方得改葬。順侍坐西遊園,因奏太后曰:「臣昨往看中山家葬,非唯宗親哀其冤酷,行路士庶見一家十喪,皆為青族旐,莫不酸泣。」叉妻時
 在太后側,順指之曰:「陛下奈何以一妹之故,不伏元叉之罪,使天下懷冤?」太后默然不語。



 就德興於營州反,使尚書盧同往討之,大敗而還。屬侍中穆紹與順侍坐,因論同之罪。同先有近宅借紹,紹頗欲為言。順勃然曰:「盧同終將無罪!」太后曰:「何得如侍中之言?」順曰:「同有好宅與要勢侍中,豈慮罪也?」紹慚,不敢復言。



 靈太后頗事妝飾,數出遊幸,順面諍之曰:「禮,婦人喪夫,自稱未亡人,首去珠珥,衣不被採。陛下母臨天下,年垂不惑,過修容飾,何以示後世?」靈太后慚而還入,召順責之曰:「千里相征,豈欲眾中見辱也!」順曰:「陛下盛服炫容,不畏天下所笑,
 何恥臣之一言乎!」



 初,城陽王徽慕順才名,偏相賞納。而廣陽王深通徽妻于氏,大為嫌隙。及深自定州被徵,入為吏部尚書,兼中領軍,順為詔書,辭頗優美。徽疑順為深左右,由是與徐紇間順於靈太后。出順為護軍將軍、太常卿。順奉辭於西遊園,徽、紇侍側。順指謂靈太后曰:「此人魏之宰嚭,魏國不滅,終不死亡。」紇協肩而出。順因抗聲叱之曰:「一介刀筆小人,正堪為幾案之吏,寧應忝茲執戟,虧我彞倫!」



 遂振衣而起。靈太后默而不言。時追論順父顧託之功,增任城王彝邑二千戶,又析彞邑五百以封順為東阿縣公。順疾徽等間之,遂為《蒼蠅賦》。屬
 疾在家,杜絕慶弔。



 後除吏部尚書,兼右僕射,與城陽王徽同日拜職。舍人鄭儼於止車門外先謁徽,後拜順。順怒曰:「卿是佞人,當拜佞王。我是直人,不受曲拜。」儼深懷謝。順曰:「卿是高門子弟,而為北宮幸臣,僕射李思沖尚與王洛誠同傳,以此度之,卿亦應繼其卷下。」見者為之震動,而順安然自得。及上省,登階向榻,見榻甚故,問都令史徐仵起。仵起曰:「此榻曾經先王坐。」順即哽塞,涕泗交流,久而不能言,遂令換之。



 時三公曹令史朱暉素事錄尚書、高陽王雍,雍欲以為廷尉評,頻煩託順,順不為用。雍遂下命用之,順投之於地。雍聞之,大怒,昧爽坐都
 ,召尚書及丞郎畢集,欲待順至,於眾挫之。順日高方至。雍攘袂撫几而言曰:「身天子之子,天子之弟,天子之叔,天子之相,四海之內,親尊莫二。元順何人,以身成命投棄於地!」



 順鬚鬢俱張,仰面看屋,憤氣奔湧,長歔而不言。久之,搖一白羽扇,徐而謂雍曰:「高祖遷宅中土,創定九流,官方清濁,軌儀萬古。而朱暉小人,身為省吏,何合為廷尉清官?殿下既先皇同氣,誠宜遵旨,自有恆規,而復踰之也?」雍曰:「身為丞相、錄尚書,如何不得用一人為官?」順曰:「庖人雖不理庖,尸祝不越樽俎而代之。未聞有別旨令殿下參選事。」順又厲聲曰:「殿下必如是,順當依
 事奏聞。」



 雍遂笑而言曰:「豈可以朱暉小人,便相忿恨。」遂起,呼順入室,與之極飲。順之亢毅不撓,皆此類也。後兼左僕射。



 汆朱榮之奉莊帝,召百官悉至河陰。素聞順數諫諍,惜其亮直,謂朱瑞曰:「可語元僕射,但在省,不須來。」順不達其旨,聞害衣冠,遂便出走,為陵戶鮮于康奴所害。家徒四壁,無物僉,止有書數千卷而已。門下通事令史王才達裂裳覆之。莊帝還宮,遣黃門侍郎山偉巡喻京邑。偉臨順喪,悲慟無已。既還,莊帝怪其聲散,偉以狀對。莊帝敕侍中元祉曰:「宗室喪亡非一,不可周贍。元僕射清苦之節,死乃益彰,特贈絹百匹,餘不得為例。」贈尚
 書令、司徒公,謚曰文烈。



 初,帝在籓,順夢一段黑雲從西北直來,觸東南上日月俱破,復翳諸星,天地盡闇。俄而雲消霧散,便有日出自西南隅,甚明凈,云長樂王日。尋見莊帝從閶闔門入,登太極殿,唱萬歲者三,百官咸加朝服謁帝,唯順集書省步廊西槐樹下,脫衣冠臥。既寤,告元暉業曰:「吾昨夜夢,於我殊自不佳。」說夢,因解之曰:「黑雲,氣之惡者,是北方之色,終當必有北敵,以亂京師,害二宮,殘毀百僚。



 何者?日,君象也。月,后象也。眾星,百官象也。以此言之,京邑其當禍乎?昔劉曜破晉室以為髑髏臺,前途之事,得無此乎?雖然,彭城王勰有文德於天
 下,今夢其兒為天子,積德必報,此必然矣!但恨其得之不久。所以然者,出自西南,以時易年,不過三載。但恨我不見之。何者?我夢臥槐樹下,槐字木傍鬼,身與鬼并,復解冠冕,此寧不死乎!然亡後乃得三公贈耳。」皆如其夢。順撰《帝錄》二十卷,詩賦表頌數十篇,並多亡失。



 長子朗,時年十七,枕戈潛伏積年,乃手刃康奴,以首祭順墓,然後詣闕請罪。



 朝廷嘉而不問。朗位司徒屬。天平中,為奴所害,贈尚書右僕射。



 順弟紀,字子綱,隨孝武入關中,位尚書左僕射、華山郡王。



 澄弟嵩,字道岳,孝文時,位步兵校尉。大司馬、安定王休薨,未及卒哭,嵩便遊田。帝聞而
 大怒,詔曰:「嵩,大司馬薨殂甫爾,便以鷹鷂自娛,有如父之痛,無猶子之情,捐心棄禮,何其太速!便可免官。」後兼武衛將軍。



 孝文南伐,齊將陳顯達率眾拒戰,嵩身備三仗,免胄直前,勇冠三軍。將士從之,顯達奔潰。帝大悅曰:「任城康王大有福德,文武頓出其門。」以功賜爵高平縣侯。初,孝文之發洛也,馮皇后以罪幽於宮內。既平顯達,回次穀唐原,帝疾甚,將賜后死,曰:「使人不易可得。」顧謂任城王澄曰:「任城必不負我,嵩亦當不負任城,可使嵩也。」於是引嵩入內,親詔遣之。宣武即位,為揚州刺史,威名大振。後并妻穆氏為蒼頭李太伯等所害。謚曰剛侯。



 第二子世俊,頗有乾用,而無行業。襲爵。孝莊時,遷吏部尚書。爾朱兆寇京師,詔世俊以本官為都督,守河橋。及兆至河,世俊初無拒守意,便隔岸遙拜。遂將船五艘迎兆軍,兆因得入。京都破殘,皆世俊之罪,時論疾之。尤為爾朱世隆所暱。孝武初,改封武陽縣子。世俊居選曹,不能厲心,多所受納,為中尉彈糾,坐免官。孝靜時,位尚書令。世俊輕薄,好去就。興和中,薨。贈太尉,謚曰躁戾。



 南安王楨,皇興二年封。孝文時,累遷長安鎮都大將、雍州刺史。楨性忠謹。



 其母疾篤,憂毀異常,遂有白雉遊其庭前。帝聞其致感,賜帛千匹以褒美之。征赴講武,引見
 於皇信堂,戒之曰:「公孝行著於私庭,令問彰於邦國,既國之懿親,終無貧賤之慮。所宜慎者略有三事:一者恃親驕矜,違禮僭度;二者傲慢貪奢,不恤政事;三者飲酒遊逸,不擇交友。三者不去,患禍將生。」而楨不能遵奉,後乃聚斂肆情。孝文以楨孝養聞名內外,特加原恕,削除封爵,以庶人歸第,禁錮終身。



 以議定遷都,復封南安王,為鎮北大將軍、相州刺史。帝餞楨於華林都亭,詔並賦詩。不能者,並可聽射,當使武士彎弓,文人下筆。帝送楨下階,流涕而別。



 太和二十年五月,至鄴。上日,暴雨大風,凍死者數十人。楨又以旱,祈雨於群神。



 鄴城有石季龍
 廟,人奉祀之。楨告神像云:「三日不雨,當加鞭罰。」請雨不驗,遂鞭像一百。是月,疽發背薨,謚曰惠。及恆州刺史穆泰謀反,楨知而不告。雖薨,猶追奪爵封,國除。



 子英,性識聰敏,善騎射,解音律,微曉醫術。孝文時,為梁州刺史。帝南伐,為漢中別道都將。後大駕臨鐘離,英以大駕親動,勢傾東南,漢中有可乘之會,表求追討,帝許之。以功遷安南大將軍,賜爵廣武伯。



 宣武即位,拜吏部尚書,以前後軍功,進爵常山侯。尋詔英率眾南討,大破梁曹景宗軍。梁司州刺史蔡道恭憂死,三關戍棄城而走。初,孝文平漢陽,英有戰功,許復其封。及為陳顯達所敗,遂寢。是
 役也,宣武大悅,乃復之,改封中山王。



 既而梁入寇肥梁,詔英率眾十萬討之,所在皆以便宜從事。英表陳事機,乃擊破陰陵,斬梁將二十五人,及虜首五千餘級。又頻破梁軍於梁城,斬其支將四十二人,殺獲及溺死者將五萬。梁中軍大將軍臨川王蕭宏、尚書左僕射柳惔等大將五人沿淮東走。凡收米四十萬石。英追奔至馬頭,梁馬頭戍主委城遁走,遂圍鐘離。詔以師行已久,命英為振旅之意。英表:「期至二月將末,三月之初,理在必剋。但自此月一日已來,霖雨連并,可謂天違人願。然王者行師,舉動不易,不可以少致暌淹,便生異議。願聞朝廷,
 特開遠略,少復賜寬,假以日月,無使為山之功,中途而廢。」及四月,水盛破橋,英及諸將狼狽奔退,士眾沒者十有五六。英至揚州,遣使送節及衣冠、貂蟬、章綬,詔以付典。有司奏英經算失圖,案劾處死。詔恕死為百姓。



 後京兆王愉反,復英王封,除使持節、假征東將軍、都督冀州諸軍事。英未發而冀州已平。



 時郢州中從事督榮祖潛引梁軍,以義陽應之,三關之戍並據城降梁。郢州刺史婁悅嬰城自守。縣瓠人白早生等殺豫州刺史司馬悅,據城南叛。梁將齊茍兒率眾守縣瓠。悅子尚華陽公主,并為所劫。詔英使持節、都督南征諸軍事、假征南將軍,
 出自汝南。帝以刑巒頻破早生,詔英南赴義陽。英以眾少,累表請軍,帝不許。而英輒與邢巒分兵共攻縣瓠,IN之,乃引軍而南。既次義陽,將取三關。英策之曰:「三關相須如左右手,若IN一關,而二關不待攻而定。攻難不如易,東關易攻,宜須先取,即黃石公所謂戰如風發,攻如河決也。」英恐其並力於東,乃使長史李華率五統向西關,分其兵勢,身督諸軍向東關。果如英策。凡禽其大將六人、支將二十人、卒七千、米四十萬石,軍資稱是。還朝,除尚書僕射。薨,贈司徒公,謚獻武王。



 英子熙,字真興,好學俊爽,有文才,聲著於世。然輕躁浮動,英深慮非保
 家之主,常欲廢之,立第四子略。略固請乃止。累遷光祿勛。時領軍于忠執政。熙,忠之婿也,故歲中驟遷。後授相州刺史。熙以七月上,其日大風寒雨,凍死者二十餘人,驢馬數十匹。熙聞其祖父前事,心惡之。又有蛆生其庭。初,熙兄弟並為清河王懌所暱,及劉騰、元叉隔絕二宮,矯詔殺懌,熙乃起兵討之。熙起兵甫十日,為其長史柳元章、別駕游荊、魏郡太守李孝怡執熙置之高樓,并其子弟。叉遣尚書左丞盧同斬之於鄴街,傳首京師。始熙妃于氏知熙必敗,不從其謀,自初哭泣不絕,至於熙死。



 熙既籓王,加有文學,風氣甚高。始鎮鄴,知友才學之士
 袁翻、李琰之、李神俊、王誦兄弟、裴敬憲等咸餞於河梁,賦詩告別。及將死,復與知故書,恨志意不遂。時人矜之。又,熙於任城王澄薨前,夢有人告之曰:「任城當死,死後二百日外,君亦不免。若其不信,試看任城家。」熙夢中顧瞻任城第舍,四面牆崩,無遺堵焉。熙惡之,覺而以告所親。及熙之死也,果如所夢。熙兄弟三人,每從英征伐,在軍貪暴,或因迎降逐北,至有斬殺無辜,多增首級,以為功狀。又于忠誣郭祚、裴植也,忠意未決害之,由熙勸獎,遂至極法,世以為冤。及熙之禍,識者以為有報應焉。靈太后反政,贈太尉公,謚曰文莊王。



 熙弟略,字人雋興,位給
 事黃門侍郎。熙敗,略潛行,自託舊識河內司馬始賓。



 始賓便為荻筏,夜與略俱渡盟津,詣上黨屯留縣栗法光家。法光素敦信義,忻而納之。略舊識刁雙,時為西河太守,略復歸之。停止經年,雙乃令從子昌送略潛遁江左。梁武甚禮敬之,封中山王,宣城太守。俄而徐州刺史元法僧據城南叛,梁乃以略為大都督,令詣彭城接誘初附。尋徵略與法僧同還。略雖在江南,自以家禍,晨夜哭泣,身若居喪。又惡法僧為人,與法僧言,未嘗一笑。



 梁復除略衡州刺史,未行。會其豫章王綜以城歸國,綜長史江革、司馬祖恆、將士五千人,悉見禽虜。明帝敕有司悉遣
 革等還南,因以征略,梁乃備禮遣之。明帝詔光祿大夫刁雙境首勞問,除略侍中、義陽王。還達石人驛亭,詔宗室親黨、內外百官先相識者,迎之近郊。其司馬始賓除給事中,領直侯,栗法光本縣令,刁昌東平太守,刁雙西兗州刺史。略所經一食一宿處,無不霑賞。



 尋改封東平王,後為尚書令。靈太后甚寵任之,其見委信,殆與元徽相埒。於時天下多事,軍國萬端。略守常自保,無他裨益,唯具臣而已。爾朱滎,略之姑夫,略素所輕忽。略又黨於鄭儼、徐紇,榮兼銜之。榮入洛也,見害於河陰。加贈太保、司空公,謚曰文貞。



 英弟怡,位鄯善鎮將。在鎮貪暴,為
 有司所糾,逃免,卒。莊帝初,以爾朱榮婦兄,贈太尉、扶風王。子肅,封魯郡王。



 肅弟曄,字華興,小字盆子。性輕躁,有膂力。莊帝初,封長廣王。爾朱榮死,世隆等推曄為主,年號建明。尋為世隆廢。節閔立,封為東海王。孝武初,被殺。



 城陽王長壽,皇興二年封,位沃野鎮都大將,甚有威名。薨,謚康王。子鸞襲。



 鸞字宣明,身長八尺,腰帶十圍。以武藝稱,頻為北都大將。孝文初,除使持節、征南大將軍。與安南將軍盧陽烏、李佐攻赭陽不剋,敗退,降為定襄縣王。後以留守功,還復本封。宣武時,為定州刺史。鸞愛樂佛道,繕起佛寺,勸率百姓,大為土木之勞,公私費擾,頗
 為人患。宣武聞之,詔奪祿一周。薨,謚懷王。



 子徽,字顯順,粗涉文史,頗有吏才。宣武時,襲封,為河內太守。在郡清整,有時譽。明帝時,為并州刺史。先是,州界夏霜,安業者少,徽輒開倉振之,文武咸共諫止。徽曰:「昔汲長孺郡守耳,尚輒開倉,救人災弊。況我皇家親近,受委大籓,豈可拘法而不救人困也?」先給後奏。明帝嘉之,加安北將軍。汾州山胡舊多劫掠,自徽為郡,群胡自相戒,勿得侵擾鄰州。汾、肆之人多來詣徽投訴,願得口判。除秦州刺史,還都,吏人泣涕攀車,不能自已。徽車馬羸弊,皆京來舊物,見者莫不歎其清儉。



 改授度支尚書,兼吏部尚書,尋
 為正。徽以選舉法期在得人,限以停年,有乖舊體。但行之日久,難以頓革,以德同者盡年,勞等者進德,于時稱為中平。除侍中,餘官如故。徽表乞守一官。天下士子莫不歎息,咸曰:「城陽離選,貧者復何所希!」怨嗟之聲,俄然上徹。還令兼吏部尚書。累遷尚書令。



 時靈太后專制,朝綱頹褫,徽既居寵任,無所匡弼。與鄭儼之徒,更相阿黨。



 外似柔謹,內多猜忌,睚眥之忿,必思報復,識者疾之。又不能防閑其妻于氏,遂與廣陽王深姦通。及深受任軍府,每有表啟,論徽罪過,雖涉誣毀,頗亦實焉。



 莊帝踐阼,拜司州牧。尋除司徒,仍領牧。元顥之入洛,徽從莊帝北
 巡。及車駕還宮,以與謀之功,除侍中、大司馬、太尉公,加羽葆鼓吹,增邑通前二萬戶。



 徽表辭官封,前後屢上。徽為莊帝親待,內懼爾朱榮等,故有此辭。莊帝識其意,聽其辭封,不許讓官。徽後妻,莊帝舅女。侍中李彧,帝之姊婿。徽性佞媚,善自取容,挾內外之意,宗室親寵,莫與比焉。遂與彧等勸帝圖榮。莊帝亦先有意。榮死,世隆等屯據不解。除徽太保,仍大司馬、宗師、錄尚書事,總統內外。徽本意謂榮死後枝葉散亡。及爾朱宗族聚結謀難,徽算略無出,憂怖而已。性多嫉妒,不欲人居其前。每入參謀議,獨與帝決。朝臣有上軍國籌策者,並勸帝不納。乃
 云:「小賊何慮不除?」又惜財用,於時有所賞錫,咸出薄少,或多而中減,與而復追。



 莊帝雅自約狹,尤亦徽所贊成。太府少卿李苗,徽司徒時司馬也,徽待之頗厚。苗每致忠言,徽多不採納。苗謂人曰:「城陽本自蜂目,而豺聲復將露也。」及爾朱兆之入,禁衛奔散,莊帝步出雲龍門,徽乘馬奔度,帝頻呼之,徽不顧而去。遂走山南,至故吏寇彌宅。彌外雖容納,內不自安,乃怖徽云:「官捕將至。」令其避他所,使人於路邀害,送屍於爾朱兆。孝武初,贈使持節、侍中、太師、錄尚書事、司州牧,謚曰文獻。子延襲爵。齊受禪,例降。



 章武王太洛,皇興二年薨,追贈征北大將軍、章武郡王,謚曰敬。無子。孝文初,以南安惠王第二子彬為後。



 彬字豹兒,勇健有將用。為夏州刺史,以貪婪削封。後除汾州刺史。胡六百餘人保險謀反。彬請兵二萬,帝大怒曰:「必須大眾者,則先斬刺史,然後發兵!」



 彬奉詔大懼,身先將士,討胡平之。卒,贈散騎常侍。



 子融,字永興,儀貌壯麗,性通率有豪氣。宣武初,復先爵,累遷河南尹。融性尤貪欲,恣情聚斂,為中尉糾彈,削除官爵。汾、夏山胡叛逆,連結正平、平陽。



 詔復融前封,征東將軍、持節、都督以討之。融寡於經略,為胡所敗。後賊帥鮮于修禮寇暴瀛、定二州,
 長孫承業等討之失利。除融車騎將軍,為前驅左軍都督,與廣陽王深等共討修禮。師度交津,葛榮殺修禮而自立,轉營至白牛邏,輕騎擊融,於陣見殺。贈司空公。尋以融死王事,進贈司徒公,加前後部鼓吹,謚莊武。子景哲襲。景哲弟朗,即廢帝也。



 樂陵王胡兒,和平四年薨,追封樂陵王,謚曰康。無子。獻文詔胡兒兄汝陰王天賜之第二子永全後之。襲封後,改名思譽。孝文時,為鎮北大將軍。穆泰陰謀不軌,思譽知而不告,削封為庶人。太和末,復王封。薨,謚密王。子景略襲,位豳州刺史。薨,謚惠王。



 安定王休,皇興二年封。少聰敏。為外都大官,斷獄有稱。車駕南伐,領大司馬。孝文親行諸軍,遇休以三盜人徇六軍,將斬之,有詔赦之。休執曰:「不斬何以息盜?」詔曰:「王者之體,亦時有非常之澤,雖違軍法,可特原之。」休乃奉詔。帝謂司徒馮誕曰:「大司馬嚴而執法,諸軍不可不慎。」於是六軍肅然。定都洛邑,休從駕幸鄴,命休率從駕文武迎家于平城,帝親餞休於漳水之北。十八年,休寢疾,帝幸其第,流涕問疾,中使醫藥相望于路。及薨至殯,車駕三臨。帝至其門,改服錫衰,素弁加絰。皇太子百官皆從行吊禮。謚曰靖王。詔贈假黃鉞,加羽葆鼓吹,悉準三
 老尉元之儀。帝親送出郭,慟哭而返。諸王恩禮莫比。宣武世,配饗廟庭。



 次子燮襲,拜太中大夫,除華州刺史。燮表曰:「謹惟州居李潤堡,雖是少梁舊地,晉芮錫壤,然胡夷內附,遂為戎落。竊以馮翊古城,實惟西籓奧府;面華、渭,包原澤;井淺地平,樵牧饒廣。採材華陰,陸運七十,伐木龍門,順流而下。



 陪削舊雉,功省力易。丁不十錢之費,人無八旬之勤。損輕益重,乞垂昭鑒。」遂詔曰:「一勞永逸,便可聽移。」薨於州,贈朔州刺史。



 子超,字化生,襲。時以胡國珍封安定公,改封北平王,後復本封。爾朱榮入洛,避難見害。



 超弟琰,字伏寶,大統中,封宋安王。薨,謚曰懿。子
 景山。



 景山字寶岳,少有器局,幹略過人。周景帝時,以軍功累遷開府儀同三司。從武帝平齊,以功拜大將軍、平原郡公、亳州總管。法令明肅,賊盜屏迹,部內大清。



 徵為候正。宣帝嗣位,從上柱國韋孝寬經略淮南。鄖州總管宇文亮反,以輕兵襲孝寬。寬為亮所薄,景山擊破之。以功拜亳州總管。



 隋文帝為丞相,尉遲迥作亂。榮州刺史宇文胄與迥通謀,陰以書諷景山。景山執使,封書詣相府,進位上大將軍。以軍功,遷安州總管,進柱國。隋文帝受禪,拜上柱國。明年,大舉伐陳,以景山為行軍元帥,出漢口。
 將濟江,會陳宣帝殂,有詔班師。景山大著威名,甚為敵人所憚。後數載,坐事免。卒于家,贈梁州總管,謚曰襄。子成壽嗣。



 成壽便弓馬,為秦王庫直。大業中,為西平郡通守。



 燮弟願平,清狂無行。宣武初,為給事中,悖惡日甚,殺人劫盜,公私咸患。



 帝以戚近,不忍致之法;免官,禁之別館。館名悉思堂,冀其克念。帝崩,乃得出。



 靈太后臨朝,以其不悛,還於別館,依前禁錮。久之,離禁還家,付宗師嚴加誨獎。



 後拜通直散騎常侍、前將軍。坐裸其妻王氏於其男女前,又強姦妻妹於妻母之側,御史中尉侯剛案以不道,處絞刑。會赦免,黜為員外常侍。卒。



 論曰:陽平諸子,頤乃忠壯。京兆之胤,心妻實有聲。匡之謇直,有足稱矣!



 當獻文將禪,可謂國之大節。康王毅然廷諍,德音孔昭,一言興邦,斯之謂歟!文宣貞固俊遠,鬱為宗傑,身用累朝。寧濟夷險,社稷是任,其梁棟之望乎!順蹇諤俶儻,有汲黯之風,不用於時,橫招非命,惜矣!嵩有行陣之氣,俊乃裂冠之徒。



 南安原始要終,善不掩惡。英將帥之用,著聲於時。熙、略兄弟,早播人譽,或才疏志大,或器狹任廣。咸不能就其功名,俱至非命,惜也!康王不永,鸞起家聲。



 徽飾智矯情,外諂內忌,永安之禍,誰任其責?宛其死也,固其宜哉!章武、樂陵,蓋不足數。靖王聽斷
 威重,見稱於太和,美矣!



\end{pinyinscope}