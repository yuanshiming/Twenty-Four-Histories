\article{卷七十三列傳第六十一}

\begin{pinyinscope}

 梁士彥元諧虞慶則元胄達奚長儒賀婁子乾兄詮史萬歲劉方馮昱王楊武通陳永貴房兆杜彥周搖獨孤楷弟盛乞伏慧張威和洪陰壽子世師骨儀楊義臣梁士彥,字相如,安定烏氏人也。少任俠,好讀兵書,頗涉經史。周武帝將平東夏,聞其勇決,自扶風郡守除為九
 曲鎮將,進位上開府,封建威縣公,齊人甚憚之。後以熊州刺史從武帝拔晉州,進位大將軍,除晉州刺史。及帝還後,齊後主親攻圍之,樓堞皆盡,短兵相接。士彥慷慨自若,謂將士曰:「死在今日,吾為爾先!」



 於是勇烈齊奮,呼聲動地,無不一當百。齊師少卻。乃令妻妾及軍人子女,晝夜修城,三日而就。武帝六軍亦至,齊師圍解。士彥見帝,持帝鬚泣,帝亦為之流涕。



 時帝欲班師,士彥叩馬諫。帝從之,執其手曰:「朕有晉州,為平齊之基,宜善守之。」及齊平,封郕國公,位上柱國、雍州主簿。宣帝即位,除徐州總管。與烏丸軌禽陳將吳明徹、裴忌於呂梁,略定淮南
 地。



 隋文帝作相,轉亳州總管。尉遲迥反,為行軍總管,及韋孝寬擊之,令家僮梁默等為前鋒,士彥繼之,所當皆破。及迥平,除相州刺史。深見忌,徵還京師。閑居無事,恃功懷怨,與宇文忻、劉昉等謀反。將率僮僕候上享廟之際以發機。復欲於蒲州起事,略取河北,捉黎陽關,塞河陽路,劫調布為牟甲,募盜賊為戰士。其甥裴通知而奏之。帝未發其事,授晉州刺史,欲觀其志。士彥欣然謂昉等曰:「天也!」又請儀同薛摩兒為長史,帝從之。後與公卿朝謁,帝令執士彥、忻、昉等於行間,詰之狀,猶不伏,捕薛摩兒至對之。摩兒具論始末云:「第二子剛垂泣苦諫,第
 三子叔諧曰:『作猛獸須成斑。』」士彥失色,顧曰:「汝殺我!」於是伏誅,時年七十二。有子五人。



 操字孟德,位上開府、義鄉縣公,早卒。



 剛字永固,位大將軍、通政縣公、涇州刺史。以諫父獲免,徙瓜州。叔諧坐士彥誅。



 梁默者,士彥之蒼頭也,驍武絕人。士彥每從征伐,常與默陷陣。仕周,位開府。開皇末,以行軍總管從楊素征突厥,進位大將軍。又從平楊諒,授柱國。大業五年,從煬帝征吐谷渾,力戰死之。贈光祿大夫。



 元諧,河南洛陽人也,家世貴盛。諧性豪俠,有氣調。少與隋文帝同受業於國子,甚相友愛。後以軍功,累遷大將
 軍。及帝為相,引致左右。諧謂帝曰:「公無黨,譬如水間一堵牆,大危矣。公其勉之!」及帝受禪,顧諧笑曰:「水間墻竟何如也?」進位上大將軍,封樂安郡公。奉詔參修律令。



 時吐谷渾將定城王鐘利旁率騎度河,連結黨項。諧率兵出鄯州,趣青海,邀其歸路。相遇於豐利山,諧擊走之,又破其太子可博汗。其名王十七人、公侯十三人,各率其所部來降。詔授上柱國,別封一子縣公。諧拜寧州刺史,頗有威惠。然性剛愎,好排詆,不能取媚於左右。嘗言於上曰:「臣一心事主,不曲取人意。」上曰:「宜終此言。」後以公事免。



 時上柱國王誼有功於國,與諧俱無位任,每相往來。
 胡僧告諧、誼謀反,帝按其事,無狀,慰諭釋之。未幾,誼誅,諧漸被疏忌。然以龍潛之舊,每預朝請,恩禮無虧。及平陳,百僚大宴,諧進曰:「陛下威德遠被,臣前請突厥可汗為候正,陳叔寶為令史,今可用臣言。帝曰:「朕平陳國,本以除逆,非欲誇誕。公之所奏,殊非朕心。突厥不知山川,何能警候?叔寶昏醉,寧堪驅使?」諧默然而退。



 後數歲,有人告諧與從父弟上開府滂、臨澤侯田鸞、上儀同祁緒等謀反。帝令按其事。有司奏:「諧謀令祁緒勒黨項兵,既繼巴蜀。時廣平王雄、左僕射高熲二人用事,諧欲譖去之,云:『左執法星動已四年矣,狀一奏,高熲必死。』又言:『太
 白犯月,光芒相照,主殺大臣,雄必當之。』諧與滂嘗同謁帝,私謂滂曰:『我是主人,殿上者賊也。』因令滂望氣,滂曰:『彼雲似蹲狗走鹿,不如我輩有福德雲。』」帝大怒,諧、滂、鸞、緒並伏誅,籍沒其家。



 虞慶則,京兆櫟陽人也,本姓魚。其先仕赫連氏,遂家靈武,世為北邊豪傑。



 父祥,周靈武太守。慶則幼雄毅,性俶儻,身長八尺,有膽智,善鮮卑語,身被重鎧,帶兩鞬,左右馳射,本州豪俠皆敬憚之。初以射獵為事,中更折節讀書,常慕傅介子、班仲升之為人。仕周,為中外府外兵參軍事,襲爵沁源縣公。越王盛討平稽胡,將班師,內史下
 大夫高熲與盛謀,須文武幹略者鎮遏之,表請慶則,於是拜石州總管。甚有威惠,稽胡慕義歸者八千餘戶。



 開皇元年,歷位內史監、吏部尚書、京兆尹,封彭城郡公,營新都總監。二年,突厥入寇,慶則為元帥討之。部分失所,士卒多寒凍,墮指者千餘人。偏將達奚長儒率騎兵二千人別道邀賊,為虜所圍,慶則按營不救。由是長儒孤軍獨戰,死者十八九。上弗之責也。尋遷尚書右僕射。後突厥主攝圖將內附,請一重臣充使,詔慶則往。攝圖恃強,慶則責以往事,攝圖不服。其介長孫晟又說諭之,攝圖及弟葉護皆拜受詔,因稱臣朝貢,請永為籓附。初,慶
 則出使,帝敕曰:「我欲存立突厥,彼送公馬,但取五三疋。」攝圖見慶則,贈馬千疋,又以女妻之。帝以慶則功高,皆無所問。授上柱國,封魯國公,食任城縣千戶,以彭城公迴授第二子義。



 平陳後,帝幸晉王第,置酒會群臣。高熲等奉觴上壽。帝曰:「高熲平江南,虞慶則平突厥,可謂茂功矣。」楊素曰:「皆由至尊威德所被。」慶則曰:「楊素前出兵武牢、硤石,若非至尊威德,亦無剋理。」遂互相長短。御史欲彈之,帝曰:「今日計功為樂,並不須劾。」帝觀群臣宴射,慶則進曰:「臣蒙賚酒,令盡樂,御史在側,恐醉被彈。」帝賜御史酒,遣之出。慶則奉觴上壽,極歡。帝謂諸公曰:「飲此
 酒,願我與公等子孫常如今日,世守富貴。」九年,轉為右衛大將軍,尋改為右武候大將軍。



 十七年,嶺南人李世賢據州反,議欲討之。諸將二三請行,皆不許。帝顧謂慶則曰:「位居宰相,爵為上公,國家有賊,遂無行意,何也?」慶則拜謝恐懼,帝乃遣焉。為桂州道行軍總管,以婦弟趙什柱為隨府長史。什柱與慶則愛妾通,恐事彰,乃宣言:「慶則不欲此行。」帝聞之。先是,朝臣出征,帝皆宴別,禮賜遣之。



 慶則南討辭帝,帝色不悅,慶則由是怏怏不得志。暨平世賢還,歸桂鎮,觀眺山川形勢,曰:「此誠險固,加以足糧。若守得其人,攻不可拔。」遂使什柱馳詣京奏事,觀
 帝顏色。什柱至京,因告慶則謀反。帝按驗之,於是伏誅。拜什柱為大將軍。



 慶則子孝仁,幼豪俠任氣,拜儀同,領晉王親信。坐父事除名。煬帝嗣位,以籓邸之舊,授候衛長史,兼領金谷監,監禁苑。有巧思,頗稱旨。大業九年,伐遼,遷都水丞,充使監運,頗有功。然性奢華,以駱駝負函盛水養魚而自給。後或告其為不軌,遂見誅。



 元胄,河南洛陽人,魏昭成帝之六代孫也。祖順,魏濮陽王。父雄,武陵王。



 胄少英果,多武藝,美鬚眉,有不可犯之色。周齊王憲見而壯之,引致左右,數從征伐。官至大將軍。隋文帝初被召入,將受顧託,先呼胄,次命陶澄,並委
 以腹心,恆宿臥內。及為丞相,每典軍在禁中,又引弟威俱入侍衛。



 周趙王招謀害帝,帝不之知,乃將酒肴詣其宅。趙王引帝入寢室,左右不得從,唯楊弘與胄兄弟坐於戶側。趙王令其二子進瓜,因將刺帝。及酒酣,趙王欲生變,以佩刀子剌瓜,連啖帝,將為不利。胄進曰:「相府有事,不可久留。」趙王呵之曰:「我與丞相言,汝何為者!」叱之使卻。胄瞋目憤氣,扣刀入衛。趙王問其姓名,胄以實對。趙王曰:「汝非昔事齊王者乎?誠壯士也!」因賜之酒,曰:「吾豈有不善之意邪?卿何猜警如是!」趙王偽吐,將入後閤,胄恐其為變,扶令上座,如此者再三。趙王稱喉乾,命胄
 就廚取飲,胄不動。會滕王逌後至,帝降階迎之,胄耳語勸帝速去。帝猶不悟,曰:「彼無兵馬,復何能為?」胄曰:「兵馬悉他家物,一先下手,大事便去。胄不辭死,死何益邪?」復入坐。胄聞屋後有被甲聲,遽請曰:「相府事殷,公何得如此?」因扶帝下床,趣而去。趙王將追帝,胄以身蔽戶,王不得出。帝及門,胄自後而至。趙王恨不時發,彈指出血。及誅趙王,賞賜不可勝計。



 帝受禪,封武陵郡公,拜左衛將軍。尋遷右衛大將軍。帝從容曰:「保護朕躬,成此基業,元胄功也。」歷豫、亳、淅三州刺史。時突厥屢為邊患,朝廷以胄素有威名,拜靈州總管,北夷甚憚焉。徵為右衛大將
 軍,親顧益隆。嘗正月十五日,帝與近臣登高,時胄下直,馳詔召之。及見,謂曰:「公與外人登高,未若就朕也。」



 賜宴極歡。晉王廣每致禮焉。房陵王之廢也,胄預其謀。帝正窮東宮事,左衛大將軍元旻苦諫,楊素乃譖之。帝大怒,執旻於仗。胄時當下直,不去,因奏曰:「臣向不下直者,為防元旻耳。」復以此言激怒帝,帝遂誅旻。



 蜀王秀之得罪,胄坐與交通,除名。煬帝即位,不得調。時慈州刺史上官政坐事徙嶺南,將軍丘和亦以罪廢。胄與和有舊,因數從之游,酒酣,謂和曰:「上官政誠壯士也,今徙嶺表,得無大事乎?」因自拊腹曰:「若是公者,不徒然矣!」



 和明日奏之,
 胄竟坐死。於是徵政為驍騎將軍。拜和代州刺史。



 達奚長儒,字富仁,代人也。祖俟,魏定州刺史。父慶,驃騎大將軍、儀同三司。長儒少懷節操,膽烈過人。十五,襲爵樂安公。為周文帝引為親信,以質直恭朴,授子都督。數有戰功。天和中,除渭南郡守,位驃騎大將軍、開府儀同三司。



 從武帝平齊,遷上開府,進爵成安郡公,別封一子縣公。宣政元年,除左將軍勇猛中大夫。後與烏丸軌圍陳將吳明徹於呂梁,陳援軍至,軌令長儒拒之。長儒取車輪數百,繫以大石,沉之清水,連轂相次以待之。船艦礙輪不得進,長儒縱奇兵大破之,獲吳明徹,以功進位
 大將軍。尋授行軍總管,北巡沙塞,卒與虜遇,大破之。



 文帝作相,王謙舉兵於蜀,沙氐楊永安扇動利、興、武、文、沙、龍等穴州以應謙,詔長儒擊破之。謙二子自京師逃歸其父,長儒並捕斬之。文帝受禪,進位上大將軍,封蘄郡公。



 開皇二年,突厥沙缽略可汗并弟葉護及籓那可汗寇掠而南,詔以長儒為行軍總管擊之。遇於周槃,眾寡不敵,軍中大懼,長儒慷慨,神色愈烈。為虜所衝突,散而復聚,且戰且行,轉鬥三日,五兵咸盡,士卒以拳毆之,手皆骨見,殺傷萬計,虜氣稍奪,於是解去。長儒身被五瘡,通中者二,其戰士死者十八九。突厥本欲大掠秦、隴,既
 逢長儒,兵皆力戰,虜意大沮,明日,於戰處焚屍慟哭而去。文帝下詔褒美,授上柱國,餘勛迴授一子。其戰亡將士,皆贈官三轉,子孫襲之。歷寧、鄜二州刺史,母憂去職。長儒性至孝,水漿不入口五日,毀悴過禮,殆將滅性,天子嘉歎。起為夏州總管,匈奴憚之,不敢窺塞。以病免。又除襄州總管,轉蘭州。



 文帝遣涼州總管獨孤羅、原州總管元褒、靈州總管賀若誼等發卒備胡,皆受長儒節度。長儒率眾出祁連山北,西至蒲類海,無虜而還。轉荊州總管,帝謂曰:「江陵國之南門,今以委卿,朕無慮也。」卒官。謚曰威。



 子皓,大業中,位太僕少卿。



 賀婁子幹,字萬壽,本代人也。隨魏氏南遷,世居關右。祖道成,魏侍中、太子太傅。父景賢,右衛大將軍。子乾少以驍武知名。仕周,累遷少司水,以勤勞封思安縣子。大象中,除秦州刺史,進爵為伯。及尉遲迥為亂,子乾從韋孝寬討之。



 遇賊圍懷州,子幹與宇文述等擊破之。文帝大悅,手書慰勉。其後每戰先登。及破鄴城,與崔弘度逐迥至樓上。進位上開府,封武川縣公,以思安縣伯別封子皎。



 開皇元年,進爵鉅鹿郡公。其年,吐谷渾寇涼州,子乾以行軍總管從上柱國元諧擊之,功最,優詔褒美,即令子乾鎮涼州。其年,突厥寇蘭州,子幹拒之,至可洛峐山,
 與賊相遇。賊眾甚盛,子幹阻川為營,賊軍不得水數日,人馬甚弊,從擊,大破之。於是冊授上大將軍,徵授營新都副監,尋拜工部尚書。其年,突厥復犯塞,以行軍總管從竇榮定擊之。子幹別路破賊,文帝嘉之,遣優詔勞免之。子幹請入朝,詔令馳驛奉見。吐谷渾復寇邊,命子幹討之。入掠其國,二旬而還。



 文帝以隴西頻被寇掠,甚患之。又彼俗不設村塢,敕子幹勒人為堡,營田積穀,以備不虞。子幹上書曰:「比見屯田之所,獲少費多。但隴右之人,以畜牧為事,若更屯聚,彌不獲安。但使鎮戍連接,烽候相望,人雖散居,必無所慮。」帝從之。



 帝以子幹習邊事,
 授榆關總管,遷雲州刺史,甚為虜所憚。後數年,突厥雍虞閭遣使請降,並獻羊馬。詔以子乾為行軍總管,出西北道應接之。還,拜雲州總管,以突厥所獻馬百匹、羊千口以賜之,乃下書曰:「自公守北門,風塵不警。突厥所獻,還以賜公。」母憂去職。朝廷以榆關重鎮,尋起視事。卒官。文帝傷惜久之。



 贈懷、魏等四州剌史,謚曰懷。子善柱嗣。



 子乾兄詮,亦有才器。位銀青光祿大夫、鄭純深等三州刺史、北地太守、東安郡公。



 史萬歲,京兆杜陵人也。父靜,周滄州刺史。萬歲少英武,善騎射,驍健若飛。



 好讀兵書,兼精占候。年十五,逢周、齊
 戰於芒山,萬歲從父在軍,旗鼓正相望,萬歲令左右趣裝急去。俄而周兵大敗,其父由是奇之。及平齊之役,其父戰沒,萬歲以忠臣子,拜開府儀同三司,襲爵太平縣公。



 尉遲迥之亂,萬歲從梁士彥擊之。軍次馮翊,見群鴈飛來,萬歲謂士彥請射行中第三者。射之,應弦而落,三軍莫不悅服。及與迥軍遇,每戰先登。鄴城之陣,官軍稍卻,萬歲乃馳馬奮擊,殺數十人,眾亦齊力,官軍復振。迥平,以功拜上大將軍。



 開皇初,大將軍爾朱勣以謀反伏誅,萬歲頗關涉,坐除名,配敦煌為戍卒。其戍主甚驍武,每單騎深入突厥中,輒大剋獲,突厥莫敢當。其人深自
 矜負,數罵辱萬歲。萬歲患之,自言亦有武用。戍主試令騎射,笑曰:「小人定可。」萬歲因請弓馬,復掠突厥中,大得六畜而歸。戍主始善之,每與同行,輒入突厥數百里,名讋北夷。竇榮定之擊突厥,萬歲詣轅門請自效。榮定素聞其名,見而大悅。因遣人謂突厥,當各遣一壯士決勝負。突厥許諾,因遣一騎挑戰。榮定遣萬歲出應之,萬歲馳斬其首而還。突厥大驚,遂引軍去。由是拜上儀同,領車騎將軍。平陳之役,以功加上開府。及高智慧等作亂江南,以行軍總管從楊素擊之。萬歲自東陽別道而進,踰嶺越海,攻陷溪洞不可勝數。前後七百餘戰,轉鬥千
 里,寂無聲問者十旬,遠近皆以萬歲為沒。萬歲乃置書竹筒中,浮之水。汲者得之,以言於素。大悅,上其事。文帝歎嗟。還,拜左領軍將軍。



 先是,南寧夷爨玩降,拜昆州刺史,既而復叛。遂以萬歲為行軍總管擊之。入蜻蛉川,經弄凍,次小勃弄、大勃弄,至於南中。賊前後屯據要害,萬歲皆擊破之。



 行數百里,見諸葛亮紀功碑,銘其背曰:「萬歲後,勝我者過此。」萬歲令左右倒其碑而進。度西二河,入渠濫川,行千餘里,破其三十餘部。諸夷大懼,遣使請降,獻明珠徑寸。於是勒石頌美隋德。萬歲請將爨玩入朝,詔許之。爨玩陰有二心,不欲詣闕,因賂萬歲金寶,萬
 歲乃捨玩而還。蜀王在益州,知其受賂,遣使將索之。



 萬歲聞而悉以所得金寶沉之於江,索無所獲。以功進柱國。晉王廣甚欽敬之,待以交友之禮。上知為晉王所善,令萬歲督晉王府軍事。明年,爨玩復反。蜀王秀奏萬歲受賂縱賊,致生邊患。上令窮之,事皆驗,罪當死。上數之,萬歲曰:「臣留玩者,恐其州有變,留以鎮撫。臣還至瀘水,詔書方到,由是不將入朝,實不受賂。」



 上以萬歲心有欺隱,大怒,顧有司曰:「將斬之。」萬歲懼而服罪,頓首請命。左僕射高熲、左衛大將軍元旻等進曰:「史萬歲雄略過人,每行兵用師之處,未嘗不身先士卒,雖古名將未能過
 也。」上意稍解,於是除名。歲餘,復官爵。尋拜河州刺史,復領行軍總管以備胡。



 開皇末,突厥達頭可汗犯塞,上令晉王及楊素出靈武道,漢王諒與萬歲出馬邑道。萬歲率柱國張定和、大將軍李藥王、楊義臣等出塞,至大斤山,遇虜。達頭遣使問曰:「隋將為誰?」候騎曰:「史萬歲也。」突厥復曰:「得非敦煌戍卒乎?」



 候騎曰:「是也。」達頭聞而引去。萬歲馳追百餘里乃及,擊大破之,逐北入磧數百里,虜遁逃而還。楊素害其功,譖萬歲云:「突厥本降,初不為寇。」遂寢其功。



 萬歲數抗表陳狀,上未之悟。會上從仁壽宮初還京師,廢皇太子,窮東宮黨與。上問萬歲所在,萬歲
 實在朝堂,楊素見上方怒,因曰:「萬歲謁東宮矣。」以激怒上。



 上謂信然,令召萬歲。時所將士卒在朝堂稱冤者數百人,萬歲謂曰:「吾今日為汝極言於上。」及見上,言將士有功。為朝廷所抑,詞氣憤厲,忤上。上大怒,命左右Ξ殺之。既而追悔不及,因下詔罪狀之。萬歲死之日,天下士庶聞者,識與不識,無不冤惜。



 萬歲為將,不修營伍,令士卒各隨所安,無警夜之備,虜亦不敢犯。臨陣對敵,應變無方,號為良將。子懷義嗣。



 劉方,京兆長安人也。性剛決,有膽氣。仁周,承御上士,以戰功拜上儀同。



 隋文帝為丞相,方從韋孝寬破尉遲迥
 於相州,以功加開府,賜爵河陰縣侯。文帝受禪,進爵為公。開皇三年,從衛王爽破突厥於白道,進位大將軍。後歷甘、瓜二州刺史。仁壽中,交州俚人李佛子作亂,據越王故城。左僕射楊素言方有將帥略,於是詔方為交州道行軍總管,統二十七營而進。法令嚴肅,然仁而愛士。長史、度支侍郎敬德亮從軍至尹州,疾甚,不能進,留之州館。分別之際,方哀其危篤,流涕嗚咽,感動行路。論者多之,稱為良將。至都隆嶺,遇賊,方遣營主宋纂、何貴、嚴願等破之。進兵臨佛子,先令人諭以禍福,佛子乃降,送於京師。其有桀黠恐為亂者,皆斬之。尋授驩州道行軍
 總管,以尚書右丞李綱為司馬,經略林邑。方遣欽州刺史寧長真、驩州刺史李暈、上開府秦雄以步騎出越常,方親率大將軍張愻、司馬李綱舟師趣北境。大業元年正月,軍至海口。林邑王梵志遣兵守險,方擊走之。



 師次闍梨江,賊據南岸立柵,方盛陳旗幟,擊金鼓,賊懼而潰。既度江,行三十里,賊乘巨象,四面而至。方以弩射象,象中瘡,卻蹂其陣,賊奔柵,因攻破之。於是濟區栗,進至大緣江,所擊皆破。經馬援銅柱,南行八日,至其國都。林邑王梵志棄城奔海,獲其廟主金人,污其宮室,刻石紀功而還。士卒腳腫死者十四五。方在道遇患卒,帝甚傷惜
 之,下詔褒美,贈上柱國、盧國公。子通仁嗣。



 開皇中,有馮昱、王、楊武通、陳永貴、房兆,俱為邊將,名顯當時。



 並不知何許人。昱多權略,有武藝。文帝初為丞相,以行軍總管與王誼、李威等討平叛蠻,拜柱國。開皇初,又以行軍總管屯乙弗泊備胡,每戰常大剋捷。驍通善謝,每以行軍總管屯兵江北以禦陳,為陳人所憚。伐陳之役,及高智慧反,攻討皆有殊績。位柱國、白水郡公。



 武通,弘農華陰人,性果烈,善馳射。數以行軍總管討西南夷,以功封白水郡公,拜左武衛將軍。時黨項羌屢為邊患,朝廷以其有威名,使鎮邊,歷岷、蘭二州總管。復與周法
 尚討嘉州叛獠,法尚軍初不利,武通為賊斷歸路。於是束馬懸車,出賊不意,頻戰破之。賊知其孤軍無援,傾部落而至。武通轉鬥數百里,為賊所拒,四面路絕。武通輕騎挑戰,墜馬,為賊所執,殺而啖之。



 永貴,隴右胡人,本姓白,以勇烈,為文帝所親愛。數以行軍總管領邊,每戰必單騎陷陣。位柱國、蘭利二州總管,封北陳郡公。



 兆,代人,本姓屋引氏,剛毅有武略。頻為行軍總管攻胡,以功位至柱國、徐州總管。並史失其事。



 杜彥,雲中人也。父遷,葛榮之亂,徙家于幽。彥性勇決,善騎射。仕周,以軍功累遷隴州刺史,賜爵永安縣伯。隋文
 帝為丞相,從韋孝寬擊尉遲迥,以功進位上開府,改封襄武縣侯,拜魏郡太守。開皇初,授丹州刺史,進爵為公。徵為左武衛將軍。平陳之役,以行軍總管與韓擒相繼而進。及陳平,賜物五千段,粟六千石,進位柱國,賜子寶安爵昌陽縣公。高智慧等之作亂,復以行軍總管從楊素討平之,斬其渠帥。賊李陀擁眾據彭山,彥襲擊破之,斬陀,傳其首。又擊徐州、宜封二洞,悉平。賜奴婢百餘口。拜洪州總管,有能名。及雲州總管賀婁子乾卒,上悼惜者久之,因謂侍臣曰:「榆林國之重鎮,安得子乾之輩乎?」後數日,上曰:「莫過杜彥。」於是徵拜雲州總管。北夷畏憚,
 胡馬不敢至塞。後朝廷追錄前功,賜子寶虔爵承縣公。十八年,遼東之役,以行軍總管從漢王至營州。上以彥曉習軍旅,令總統五十營事。及還,拜朔州總管。突厥寇雲州,上令楊素擊走之,猶恐為邊患,復拜彥雲州總管。以疾徵還,卒。



 子寶虔,大業末,至文城郡丞。



 周搖,字世安,河南洛陽人也。其先與魏同源,初姓普乃,及居洛陽,改為周氏。曾祖拔拔,祖右六肱,俱為北平王。父恕延,歷行臺僕射、南荊州總管。搖少剛毅,有武藝,性謹厚,動遵法度。仕魏,位開府儀同三司。周閔帝受禪,賜姓車非氏,封金水郡公。歷鳳、楚二州刺史,吏人安之。從
 平齊,以戰功超授柱國,進封夔國公。未幾,拜晉州總管。時隋文帝為定州總管,文獻皇后自京師赴州,路經搖所,主禮甚薄。既而白后曰:「公廨甚富於財,限法不敢輒費。又王臣無得效私。」



 其質直如此。帝以其奉法,每嘉之。及為丞相,徙封濟北郡公,拜豫州總管。帝受禪,復姓周氏。開皇初,突厥寇邊,燕、薊多被其患,前總管李崇為虜所殺,上思所以鎮之,曰:「無以加周搖。」拜為幽州總管、六州五十鎮諸軍事。搖修障塞,謹斥候,邊人安之。徙尋、襄二州總管,俱有能名,進上柱國。以老乞骸骨,上勞之曰:「公歷仕三代,保茲遐壽,良足善也。」賜坐褥。歸第,終於家,
 謚曰恭。



 獨孤楷,字修則,不知何許人也,本姓李氏。父屯,從齊神武帝與周師戰於沙苑,齊師敗績,因為柱國獨孤信所禽,配為士伍,給使信家,漸得親近,因賜姓獨孤氏。楷少謹厚,便弄馬槊,為宇文護執刀。數從征伐,賜爵廣阿縣公。拜右侍下大夫。從韋孝寬平淮南,以功賜子景雲爵西河縣公。隋文帝為丞相,進開府,領親信兵。及受禪,拜右監門將軍。進封汝陽郡公。仁壽初,出為原州總管。時蜀王秀鎮益州,上徵之,猶豫未發。朝廷恐秀生變,拜楷益州總管,馳傳代之。秀果有異志,楷諷諭久之,乃就路。
 楷察秀有悔色,因勒兵為備。秀至興樂,去益州四十餘里,將反襲楷,密使覘之,知不可犯而止。楷在益州,甚有惠政,蜀中父老于今稱之。煬帝既位,轉並州總管。遇疾喪明,上表乞骸骨。帝曰:「公先朝舊臣,臥以鎮之,無勞躬親簿領也。」以其長子凌雲監省郡事。其見重如此。轉長平太守。卒,謚曰恭。子凌雲、平雲、彥雲,皆知名。



 楷弟盛,性剛烈,有膽略。以落邸之舊,累遷右屯衛將軍。宇文化及之亂,裴虔通引兵至成象殿,宿衛者皆釋仗走。盛謂虔通曰:「何物兵?形勢太異!」虔通曰:「事已然,不預將軍事。」盛罵曰:「老賊,何物語!」不及被甲,與左右十餘人逆拒之,為
 亂兵所殺。越王侗稱制,贈光祿大夫、紀國公,謚曰武節。



 乞伏慧,字令和,馬邑鮮卑人也。祖周,魏銀青光祿大夫;父纂,金紫光祿大夫。並為第一領人酋長。慧少慷慨,有大節,便弓馬,好鷹犬。齊文襄時,為行臺左丞,累遷太僕卿,自永寧縣公封宜人郡王。其兄貴和,又以軍功為王。一門二王,稱為貴顯。周武平齊,授使持節、開府儀同大將軍,拜佽飛右旅下大夫,轉熊渠中大夫。從韋孝寬擊尉遲惇於武陟,以功授大將軍。及破尉遲迥,進位柱國,賜爵西河郡公。請以官爵讓兄,朝廷不許,論者義之。隋文帝受禪,拜曹州刺史。曹土舊俗,人多姦隱,戶口簿帳,
 恆不以實。慧下車按察,得戶數萬。遷涼州總管。先是,突厥屢為寇抄,慧嚴警烽燧,遠為斥候,虜竟不入境。後為荊州總管,又領潭桂二州總管、三十一州諸軍事。其俗輕剽,慧躬行朴素以矯之,風化大洽。曾見人以珣捕魚者,出絹買而放之,其仁心如此。百姓美之,號其處曰西河公珣。煬帝即位,為天水太守。大業五年,征吐谷渾。郡濱西境,人苦勞役,又遇帝西巡,坐御道不整,獻食疏薄,帝大怒,命左右斬之。見其無髮,乃釋之。除名,卒于家。



 張威,不知何許人也。父琛,魏弘農太守。威少倜儻,有大志,善騎射,膂力過人。仕周,以軍功位柱國、京兆尹,封長
 壽縣公。王謙作亂,隋文帝以威為行軍總管,從梁睿擊之。軍次通谷,謙守將李三王拒守。睿以威為先鋒。三王閉壘不戰,威令人激怒之,三王果出陣。威令壯士奮擊,三王軍潰。大兵繼進至開遠,謙將趙儼眾十萬,連營三十里。威鑿山通道,攻其背,儼敗走,追至成都。及謙平,進位上柱國、瀘州總管。隋文帝受禪,拜幽、洛二州總管,改封晉熙郡公。尋拜河北道臺僕射,後督晉王軍府事。遷青州總管。在青州頗事產業,遣家奴於人間鬻蘆菔根,其奴緣此侵擾百姓。上深加譴責,坐廢於家。後從上祠太山,到洛陽,上責讓之,因問威所執笏安在。威頓首曰:「
 臣負罪,無顏復執,謹藏於家。」上曰:「可持來。」威明日奉笏以見,上曰:「公雖不遵法度,功效實多,今還公笏。」於是復拜洛州刺史。後改封皖城郡公,轉相州刺史。卒。



 子植,大業中,位至武賁郎將。



 和洪,汝南人也。勇烈過人。仕周,以軍功位車騎大將軍、儀同三司。時龍州蠻任公忻、李國立等,聚眾為亂,刺史獨孤善不能禦。朝議以洪有武略,代善為刺史。月餘,斬公忻、國立等,皆平之。後從武帝平齊,位上儀同,賜爵北平侯,拜左勳曹下大夫。柱國王軌之禽吳明徹也,洪有功焉,加位開府,遷折衝中大夫。尉遲迥作亂,洪以行軍
 總管從韋孝寬擊之,以功封廣武郡公。時東夏初平,物情尚梗,隋文帝以洪有威名,令領冀州事,甚得人和。後拜泗州刺史。屬突厥寇邊,詔洪為北道行軍總管,擊走之,追虜至磧而還。後遷徐州總管。卒。



 陰壽字羅雲,武威人也。父嵩,周夏州刺史。壽少果烈,有武幹,性謹厚。從周武帝平齊,位開府。隋文帝為丞相,引為掾。尉遲迥亂,文帝以韋孝寬為元帥擊之,命壽監軍。時孝寬有疾,不能親總戎事,每臥帳中,遣婦人傳教命,三軍綱紀,皆取決於壽。以功進位上柱國。尋拜幽州總管,封趙郡公。先是,齊之疏屬高寶寧,周武帝拜為營州
 刺史,性桀黠,得華夷心。及文帝為丞相,遂連契丹、靺鞨舉兵反。



 帝以中原多故,未遑進討,諭之不下。開皇初,又引突厥攻圍北平。至是,令壽討之。寶寧棄城奔于磧北,黃龍諸縣悉平。壽班師,留開府成道昂鎮之。壽患寶寧攻道昂,乃重購獲之,北邊遂安。卒官,贈司空。



 子世師,少有節概,性忠厚,多武藝。以功臣子拜儀同。煬帝嗣位,拜張掖太守,深為戎狄所憚。後拜樓煩太守,遷左翊衛將軍。與代王留守京師。及義軍至,世師自以世荷隋恩,遂拒守不下。及城平,與京兆郡丞骨儀等見誅。



 骨儀,天竺胡人。性剛鯁,有不可奪之志,開皇初,為御史,
 處法平當,不為勢利所回。煬帝嗣位,遷尚書左司郎。于時朝政漸亂,貨賄公行,凡當樞要之職,無問貴賤,並家累金寶。天下士大夫莫不變節,而儀勵志守常,介然獨立。帝嘉其清苦,拜京兆郡丞,公方彌著。時刑部尚書衛玄兼領京兆內史,頗行詭道,輒為儀所執正。玄雖不便之,不能傷。及義兵至,玄恐禍及,辭老病。儀與世師同心協契,父子並誅,其後絕。世師有子弘智等,各以年幼獲全。



 楊義臣,代人也,本姓尉遲氏。父崇,仕周,為儀同大將軍,以兵鎮恆山。時隋文帝為定州總管,崇知帝相貌非常,
 每自結納,帝甚親待之。及為丞相,尉遲迥亂,崇以宗族故,自囚,遣使請罪。帝下書慰諭之,即令馳驛入朝,恆置左右。開皇初,封秦興公。歲餘,從行軍總管達奚長儒擊突厥於周槃,力戰而死。贈大將軍、豫州刺史,以義臣襲崇官爵。時義臣尚幼,養於宮中,未弱冠,奉詔宿衛如千牛者數年,賞賜甚厚。上嘗言及恩舊,顧義臣嗟嘆久之,因下詔賜義臣姓楊氏,編之屬籍,為皇從孫。未幾,拜陜州刺史。義臣性謹厚,能騎射,有將領才。後突厥達頭可汗犯塞,以行軍總管出白道,大破之。明年,突厥又寇邊,義臣擊之,追至大斤山,與虜遇。時太平公史萬歲亦至,
 與義臣合擊大破之。萬歲為楊素所陷,義臣功竟不錄。



 煬帝嗣位,漢王諒反,時代州總管李景被諒將喬鐘葵所圍,義臣時為朔州總管,奉詔救之。鐘葵見義臣兵少,悉眾拒之。時鐘葵亞將王拔驍勇,善用槊,射者不能中,每以數騎陷陣。義臣患之,募能當拔者。有車騎將軍楊思恩請當之。義臣見思恩氣貌雄勇,顧之曰:「壯士也!」賜以危酒。思恩望見拔立於陣後,投觴於地,策馬赴之。再往不剋,所從騎士退,思恩為拔所殺。拔遂乘之,義臣軍北者十餘里。



 於是購得思恩屍,義臣哭之甚慟,三軍莫不下泣,所從騎士皆腰斬。義臣自以兵少,悉取軍中牛
 驢,得數千頭,復令數百人,人持一鼓,潛驅之磵谷間,出其不意。義臣晡後復與鐘葵戰,兵初合,命驅牛驢者疾進。一時鳴鼓,埃塵張天,鐘葵軍不知所以,以為伏兵發,因大潰,從擊破之。以功進位上大將軍。累遷太僕卿。從征吐谷渾,令義臣屯琵琶峽,連營八十里,南接元壽,北連段文振,合圍吐谷渾主於覆袁川。復從征遼東,以軍將指肅慎道。至鴨綠水,與乙支文德戰,每為先鋒,一日七捷。後與諸軍俱敗,竟坐免。俄而復位。明年,以為軍副。與大將軍宇文述趣平壤。至鴨綠水,會楊玄感作亂班師,檢校趙郡太守。妖賊向海公作亂,寇扶風、安定間,義
 臣奉詔擊平之。尋從帝復征遼東,進位左光祿大夫。時勃海高士達、清河張金稱並相聚為盜,攻陷郡縣。帝遣將軍段達討之,不能剋,詔義臣率遼東還兵擊之,大破士達,斬金稱。又收降賊,入豆子,討賊格謙禽之,以狀聞奏。帝惡其威名,遽追入朝,賊由是復盛。義臣以功進位光祿大夫,尋拜禮部尚書。卒于官。



 論曰:昔韓信愆垓下之期,則項王不滅;英布無淮南之舉,則漢道未隆。以二子之勳庸,咸憤怨而菹戮,況乃無古人之殊績,而懷悖逆之心者乎!梁士彥遭雲雷之會,以勇略成名,遂貪天之功以為己力。執者倦矣,施者未
 厭,將生厲階,求逞其欲。及茲顛墜,自取之也。元諧、虞慶則、元胄,或契闊艱危,或綢繆恩舊,將安將樂,漸見遺忘,內懷怏怏,矜伐不已。雖時主之刻薄,亦言語以速禍乎!然隋文佐命元功,鮮有終其天命,配享清廟,寂爾無聞。斯蓋草創帝圖,事出權道,本異同心,故久而愈薄。其牽牛蹊田,雖則有罪,奪之非道,能無怨乎?皆深文巧詆,致之刑辟,帝沈猜之心,固已甚矣。求其餘慶,不亦難哉!長儒以步卒二千,抗十萬之眾,師殲矢盡,勇氣彌厲,壯矣哉!子乾西涉青海,北臨玄塞,胡夷懾憚,亦有可稱。萬歲實懷智勇,善撫士卒,人皆樂死,師不疲勞。北卻匈奴,南
 平夷獠,兵鋒所指,威警絕域。論功仗氣,犯忤貴臣,偏聽生奸,死非其罪,人皆痛惜,有李廣之風焉。劉方號令無私,臨軍嚴肅,克翦林邑,遂清南海,徼外百蠻,無思不服。杜彥東夏南服,屢有戰功,作鎮朔垂,胡塵不起。周搖以質直見知,獨孤楷以恤人流譽。盛蹈履之地,可以追蹤古人。乞伏慧能以國讓,亦云美矣。而慧以供帳不厚,至於放黜,君方逞欲,罰亦深哉!陰世師遭天所廢,舍命無改,雖異先覺,頗同後凋。義臣時屬擾攘,功成三捷,而以功見忌,得沒亦為幸也。



\end{pinyinscope}