\article{卷七十一列傳第五十九 隋宗室諸王}

\begin{pinyinscope}

 蔡
 景王整滕穆王瓚道宣王嵩衛昭王爽河間王弘義城公處綱離石太守子崇文帝四王煬帝三子蔡景王整,隋文帝之次弟也。文帝四弟,唯整及滕穆王瓚與帝同生,次道宣王嵩,次衛昭王爽並異母。整,周明帝時以武元軍功,賜爵陳留郡公。位開府、車騎大將軍。
 從武帝平齊。力戰而死。文帝初居武元之憂,率諸弟負土為墳,人植一栢,四根鬱茂,西北一根整栽者獨黃。後因大風雨,并根失之,果終不吉。文帝作相,贈柱國、大司徒、八州剌史。及受禪,追封謚焉。



 子智積襲。又封其弟智明為高陽郡公,智才開封縣公。尋拜智積開府儀同三司,授同州刺史,儀衛資送甚盛。



 整娶同郡尉遲綱女,生智積。開皇中,有司奏智積將葬尉太妃,帝曰:「昔幾殺我。我有同生二弟,並倚婦家勢,常憎疾我。我向之笑云:『爾既嗔我,不可與爾角嗔。』並云:『阿兄止倚頭額。』時有醫師邊隱逐勢,言我後百日當病癲。二弟私喜。以告父母。父
 母泣謂我曰:『爾二弟大劇,不能愛兄。』我因言:『一日有天下,當改其姓。夫不受其親而愛他人者,謂之悖德,當改之為悖。』父母許我此言。父母亡後,二弟及婦又讒我,言於晉公。于時每還,欲入門,常不喜,如見獄門。託以患氣,常鎖閤靜坐,唯食至時暫開閤。每飛言入耳,竊云『復未邪?』當時實不可耐,羨人無兄弟。世間貧家兄弟多相愛,由相假藉;達官兄弟多相憎,爭名利故也。」



 智積在同州,未嘗嬉戲游獵,聽政之暇,端坐讀書。門無私謁。有侍讀公孫尚義,山東儒士,府佐楊君英、蕭德言,並有文學,時延於坐。所設唯餅果,酒纔三酌。家有女妓,唯年節嘉慶
 奏於太妃前。始,文帝龍潛時,與景王不睦,太妃尉氏又與獨孤皇后不相諧,以是智積常懷危懼,每自貶損。帝亦以是哀憐之。人或勸智積為產業,智積曰:「昔平原露朽財帛,苦其多也。吾幸無可露,何更營乎!」有五男,止教讀《論語》、《孝經》而已,亦不令交通賓客。或問其故,智積曰:「恐兒子有才能以致禍也。」開皇二十年,徵還京,無他職任,闔門自守,非朝覲不出。煬帝即位,滕王綸、衛王集並以讒構得罪,高陽公智明亦以交通奪爵,智積愈懼。大業三年,授弘農太守,委政僚佐,清靜自居。及楊玄感作逆,自東都引軍而西,智積謂官屬曰:「玄感欲西圖關中,
 若成其計,則根本固矣。當以計縻之,使不得進。不出一旬,自可禽耳。』及玄感軍至城下,智積登陴詈辱之,玄感怒甚,留攻之。城門為賊所燒,智積乃更益火,賊不得入。數日,宇文述等軍至,合擊破之。尋拜宗正卿。



 十二年,從駕江都,寢疾。帝時疏薄骨肉,智積每不自安,及遇患,不呼醫。



 臨終,謂所親曰:「吾今日始知得保首領沒於地矣!」時人哀之。有子道玄。



 滕穆王瓚,字恆生,一名慧。仕周,以武元軍功,封竟陵郡公,尚周武帝妹順陽公主。保定四年,累遷納言。瓚貴公子,又尚公主,美姿容,好書愛士,甚有當時譽,時人號曰
 楊三郎。武帝甚親愛之。平齊之役,諸王咸從,留瓚居守,謂曰:「六府事殷,一以相付,朕無西顧之憂矣。」宣帝即位,遷吏部中大夫,加上儀同。



 宣帝崩,文帝入禁中,將總朝政,令廢太子勇召之。瓚素與帝不協,不從,曰:「作隋國公恐不能保,何乃更為族滅事邪!」文帝作相,拜大宗伯,典脩禮律,進位上柱國、邵國公。瓚見帝執政,恐為家禍,陰有圖帝計,帝每優容之。及受禪,立為滕王,拜雍州牧。帝數與同坐,呼為阿三。後坐事去牧,以王就第。



 瓚妃宇文氏,素與獨孤皇后不平,至是鬱鬱不得志,陰有咒詛。帝命瓚出之。



 瓚不忍離絕,固請。帝不得已,從之,宇文氏竟
 除屬籍。由是恩禮更薄。開皇十一年,從幸栗園,坐樹下,方飲酒,鼻忽流血,暴薨。時年四十四。人皆以為遇鳩。



 子綸嗣。



 綸字斌褵,性弘厚,美姿容,頗知鐘律。文帝受禪,封邵國公。明年,拜邵州刺史。晉王廣納妃於梁,詔綸致禮,甚為梁人所敬。



 綸以穆王故,當文帝世,每不自安。煬帝即位,尤被猜忌。綸憂懼,呼術者王琛問之。琛答曰:「王相祿不凡。滕即騰也,此字足為善應。」有沙門惠恩、崛多等,頗解占候,綸每與交通,嘗令些三人為厭勝法。有人告綸怨望咒詛,帝令黃門侍郎王弘窮驗之。弘希旨奏綸厭蠱
 惡逆,坐當死。帝令卿議之,司徒楊素等曰:「綸懷惡之由,積自家世。惟皇運之始,四海同心,在於孔懷,彌須協力。其先乃離阻大謀,棄同即異。父悖於前,子逆於後,為惡有將,其罪莫大。請依前科。」



 帝以皇族不忍,除名徙邊郡。



 大業七年,帝征遼東,綸欲上表,請從軍自效,為郡司所遏。示幾,徙珠崖。



 及天下大亂,為賊林仕弘逼,攜妻子竄儋耳。後歸國,封懷化縣公。尋病卒。



 綸弟坦,字文褵,初封竟陵郡公,坐綸徙長沙。



 坦弟猛,字武褵,徙衡山。



 猛弟溫,字明褵,初徙零陵。溫好學,解屬文,既而作《零陵賦》以自寄,其詞哀思。帝見而怒之,轉徙南海。



 溫弟詵,字弘褵,
 前亦徙零陵。帝以其脩謹,襲封滕王,以奉穆王嗣。大業末,於江都為宇文化及所害。



 道宣王嵩,在周以武元軍功,賜爵興城公。早卒。文帝受禪,追封謚焉。以滕穆王瓚子靜襲。卒,謚曰悼。無子,以蔡王智積子世澄襲。



 衛昭王爽,字師仁,小字明達。在周以武元軍功,於襁褓中封同安郡公。六歲而武元崩,為獻皇后所養,由是寵愛特異諸弟。年十七,為內史上大夫。文帝執政,授蒲州刺史、柱國。及受禪,立為衛王,所生李氏為太妃。爽位雍州牧、右領軍大將軍、權領并州總管、上柱國、涼州總管。
 爽美風儀,有器局,政甚有聲。大軍北伐,河間王弘、豆盧勣、竇榮定、高熲、虞慶則等分道而進,以爽為元帥,俱受爽節度。親率李充等四將出朔州,遇沙缽略可汗於白道,接戰,大破之,沙缽略中重瘡而遁。帝大悅,賜爽真食梁安縣千戶。六年,復為元帥,步騎十五萬出合川,突厥遁逃。徵為納言。帝甚重之。未幾,爽疾,帝使薛榮宗視之,雲眾鬼為厲。爽令左右驅逐之。居數日,有鬼物來擊榮宗,走下階而斃。其夜爽薨,年二十五。贈太尉、冀州刺史。子集嗣。



 集字文會,初封遂安王,尋襲封衛王。煬帝時,諸侯王恩禮漸薄,猜防日甚,集憂懼,乃呼術者俞普明章
 醮以祈福助。有人告集咒詛,憲司希旨,鍛成其獄,奏集惡逆,坐當死。詔下其議,楊素等曰:「集密懷左道,厭蠱君親,是君父之罪人,非臣子之所赦,請論如律。」時滕王綸坐與相連,帝不忍加誅,除名遠徙邊郡。天下亂,不知所終。



 河間王弘,字辟惡,文帝從祖弟也。祖愛敬,早卒。父元孫,少孤,隨母郭氏養於舅族。及武元帝與周文建義關中,元孫時在鄴,懼為齊人所誅,因假外家姓為郭氏。元孫死,齊為周滅,弘始入關。與文帝相得,帝哀之,為買田宅。



 弘性明悟,有文武幹略。數從征伐,累遷開府儀同三司。文
 帝為丞相,常置左右,委以心腹。帝詣周趙王宅,將及於難,弘時立於戶外,以衛文帝。尋加上開府,賜爵永康縣公。及愛禪,拜大將軍,進爵郡公。尋贈其父柱國、尚書令、河間郡公。



 其年,立弘為河間王,拜右衛大將軍。尋進柱國,以行軍元帥出靈州道征突厥,大破之。拜寧州總管,進上柱國。政尚清靜,甚有恩惠。遷蒲州刺史,得以便宜從事。



 時河東多盜賊,弘奏為盜者百餘人,投之邊裔,州境恬然,號為良吏。每晉王廣入朝,弘輒領揚州總管,及王歸籓,弘復還蒲州。在州十餘年,風教大洽。煬帝嗣位,拜太子太保。歲餘,薨。大業六年,追封郇王。子慶嗣。



 慶
 傾曲善候時變。帝猜忌骨肉,滕王綸等皆被廢放,唯慶獲全。累遷滎陽太守,頗有政績。及李密據洛口倉。滎陽諸縣多應密。慶勒兵拒守。歲餘,城中糧盡,兵勢日蹙。密遺慶書曰:「王之先世,家住山東,本姓郭氏,乃非楊族。婁敬之於漢高,殊非血胤;呂布之於董卓,良異天親。芝焚蕙歎,事不同此。江都荒湎,流宕忘歸,骨肉崩離,人神怨憤。舉烽火於驪山,諸侯莫至;浮膠船於漢水,還日未期。



 王獨守孤城,援絕千里,糧餱支計,僅有月餘,弊卒之多,纔盈數百。有何恃賴,欲相抗拒?求枯魚於市肆,既事非虛;因歸鴈以運糧,竟知何日!止恐禍生匕首,釁發蕭墻,
 空以七尺之軀,懸賞千金之購,可為酸鼻者也。幸能三思,自求多福。」



 于時江都敗問亦至,慶得書,遂降于密,改姓為郭氏。密破,歸東都,又為楊氏,越王侗不之責也。及侗稱制,拜宗正卿。



 世充既僭偽號,降爵為郇國公,復為郭氏。世充以兄女妻之,署滎州刺史。及世充將敗,慶欲將妻同歸長安,其妻曰:「國家以妾奉箕帚於公者,欲以申厚意,結公心耳。今父叔窮迫,家國阽危,而不顧婚姻,孤負付屬,為全身之計,非妾所能責公也。妾若至長安,公家一婢耳,何用妾為!顧送還東都,君之惠也。」慶不許。其妻遂沐浴靚莊。仰藥而死。慶遂歸國,為宜州刺史、郇
 國公,復姓楊氏。其嫡母元太妃,年老,兩目喪明,世充斬之。



 義城公處綱,文帝族父也。生長北邊,少習騎射。在周,以軍功拜上儀同。文帝受禪,贈其父鐘葵柱國、尚書令、義城縣公,以處綱襲焉。累遷右領軍將軍。綱雖無才藝,而性質直,在官強濟,亦為當時所稱。拜蒲州刺史,吏人悅之。卒於秦州總管,謚曰恭。



 弟處樂,官至洛州刺史。漢王諒反,朝廷以為二心,廢錮不齒。



 離石太守子崇,武元帝族弟也。父盆生,贈荊刺史。子崇少好學,涉獵書記,有風儀,愛賢好士。開皇初,拜儀同,
 以車騎將軍恒典宿衛,後為司門侍郎。煬帝嗣位,累遷候衛將軍。坐事免。未幾,復檢校將軍事。從帝幸汾陽宮,子崇知突厥必為寇,屢請早還京師,不納。尋有鴈門之圍。及賊退,帝怒之曰:「子崇怯懦,妄有陳請,驚動我眾心,不可居爪牙寄。」出為離石郡太守,有能名。自是突屢寇邊塞,胡賊劉六兒復擁眾劫掠郡境,子崇表請兵鎮遏。帝復大怒,令子崇行長城。



 子崇行百餘里,四面路絕,不得進而歸。



 歲餘,朔方梁師都、馬邑劉武周等各作亂,郡中諸胡復反。子崇患之,言欲朝集,遂與心腹數百人自孟門關將還京師。遇道路隔絕,退歸離石。左右聞太
 原兵起,不復入城,各叛去。子崇悉收叛者父兄斬之。後數日,義兵至,城中應之。城陷,為讎家所殺。



 文帝五男,皆文獻皇后所生。長曰房陵王勇,次煬帝,次秦孝王俊,次庶人秀,次庶人諒。



 房陵王勇,小名睍地伐。周世以武元軍功,封博平縣侯。及文帝輔政,立為世子,拜大將軍、左司衛,封長寧郡公。出為洛州總管、東京少塚宰,總統舊齊之地。



 後徵還京師,進上柱國、大司馬,領內史御正,諸禁衛皆屬焉。文帝受禪,立為皇太子,軍國政事及尚書奏死罪已下,皆令勇參決。帝以山東人多流冗,遣使案檢,又欲徙人北實邊
 塞。勇上書諫,以為「戀土懷舊,人之本情,波迸流離,蓋不獲已。



 有齊之末,主闇時昏,周平東夏,繼以威虐,人不堪命,致有逃亡,非厭家鄉,原為羈旅。若假以數歲,沐浴皇風,逃竄之徒,自然歸本。雖北夷犯邊,令所在嚴固,何待遷配,以致勞擾?」上覽而嘉之。時晉王廣亦表言不可,帝遂止。是後時政不便,多所損益,帝每納之。帝常從容謂群臣曰:「前世皇王,溺於嬖幸,廢立之所由生。朕傍無姬侍,五子同母,可謂真兄弟也。豈若前代,多諸內寵,孽子忿爭,為亡國之道邪!」



 勇頗好學,解屬詞賦,性寬仁和厚,率意任情,無矯飾之行。引明克讓、姚察、陸開明等為之
 賓友。勇嘗文飾蜀鎧,帝見而不悅,恐致奢侈之漸,因誡之曰:「我歷觀前代帝王,未有奢華而能長久者。汝當儲后,若不上稱帝心,下合人意,何以承宗廟之重,居兆人之上?吾昔衣服,各留一物,時復看以自警戒。又擬分賜汝兄弟。恐汝以今日皇太子之心,忘昔時之事,故令高熲賜汝我舊所帶刀子一枚,并菹醬一合,汝昔作上士時所常食如此。若存憶前事,應知我心。」



 後經冬至,百官朝勇,勇張樂受賀。帝知之,問朝臣:「近聞至節,內外百官相率朝東宮,是何禮也?」太常少卿辛亶對曰:「於東宮是賀,不得言朝。帝曰:「改節稱賀,正可三數十人,逐情各去,
 何因有司徵召,一朝普集,太子法服設樂以待之?東宮如此,殊乖禮制。」乃下詔曰:「皇太子雖居上嗣,義兼臣子,而諸方岳牧正冬朝賀,任土作貢,別上東宮。事非典則,宜悉停斷。」



 自此恩寵始衰,漸生凝阻。時帝令選強宗入上臺宿衛,高熲奏:「若盡取強者,恐東宮宿衛太劣。」帝作色曰:「我有時行動,宿衛須得雄毅。太子毓德東宮,左右何須強武?如我商量,恒於交番之日,分向東宮上下,團伍不別,豈非好事邪?



 我熟見前代,公不須仍踵舊風!」蓋疑熲男尚勇女,形於此言,以防之。



 勇多內寵,昭訓雲氏嬖幸,禮匹於嫡。而妃元氏無寵,嘗遇心疾,二日而薨。



 獻
 皇后意有他故,甚責望勇。又自妃薨,雲昭訓專擅內政,后彌不平,頗求勇罪過。



 晉王廣知之,彌自矯飾,姬妾恆備員數,唯與蕭妃居處。皇后由是薄勇,愈稱晉王德行,後晉王來朝,車馬侍從,皆為儉素,接朝臣,禮極卑屈,聲名籍甚,冠於諸王。臨還揚州,入內辭皇后,因哽咽流涕,伏不能興。皇后泫然泣下,相對歔欷。



 王曰:「臣性識愚下,常守平生昆弟之意,不知何罪,失愛東宮,恆蓄盛怒,欲加屠陷。每恐讒譖出於杼軸,鳩毒遇於盃杓。」皇后忿然曰:「睍地伐漸不可耐,我為伊索得元家女,望隆基業,竟不聞作夫妻,專寵阿雲,有如許豚犬。前新婦本無病痛,
 忽爾暴亡,遣人投藥,致此夭逝。事已如此,我亦不窮。何因復於汝處發如此意?我在尚爾,我死後當魚肉汝乎?每思東宮竟無正嫡,至尊千秋萬歲後,遣汝等兄弟向陽雲兒前再拜問訊,此是幾許大苦痛邪!」晉王又拜,嗚咽不能止,皇后亦悲不自勝。此別之後,知皇后意移,始構奪宗之計。因引張衡定策,遣褒公宇文述深交楊約,令喻旨於越公素,具言皇后此語。素瞿然曰:「但不知皇后如何?但如所言,吾又何為者!」後數日,素入侍宴,微稱晉王孝悌恭儉有禮,用此揣皇后意,后泣曰:「公言是也。我兒大孝順,每聞至尊及我遣內使到,必迎於境首。
 又其新婦亦大可憐,我使婢去,常與同寢共食。豈如睍地伐共阿雲相對而坐,終日酣宴,暱近小人,疑阻骨肉!我所以益憐阿鷿者,嘗恐暗地殺之。」素既知意,盛言太子不才。皇后遂遺素金,始有廢立之意。



 勇頗知其謀,憂懼,計無所出。聞新豐人王輔賢能占候,召而問之。輔賢曰:「白虹貫東宮門,太白襲月,皇太子廢退象也。」以銅鐵五兵造諸厭勝。又於後園內作庶人村,屋宇卑陋,太子時於中寢息,布衣草褥,冀以當之。帝知其不安,在仁壽宮,使楊素觀勇,素至東宮,偃息未入,勇束帶待之,故亦不進以怒勇,勇銜之,形於言色。素還,言勇怨望,恐有他
 變。帝甚疑之。皇后又遣人伺覘東宮,纖介事皆聞奏,因加媒蘗,構成其罪。帝惑之,遂疏忌勇。乃於玄武門達至德門量置人候,以伺動靜,皆隨事奏聞。又東宮宿衛人,侍官已上,名籍悉令屬諸衛府,有健兒者咸屏去之。晉王又令段達私貨東宮幸臣姬威,令取太子消息,密告楊素。於是內外宣謗,過失日聞。段達脅姬威曰:「東宮罪過,主上皆已知之。已奉密詔,定當廢立。君能告之,則大富貴。」威遂許諾。



 開皇二十年,車駕至自仁壽宮,御大興殿,謂侍臣曰:「我新還京師,應開懷歡樂,不知何意,翻悒然愁苦。」吏部尚書牛弘對曰:「由臣等不稱職,故至尊憂
 勞。帝既數聞讒譖,疑朝臣具委,故有斯問,冀聞太子之愆。弘既此對,大乖本指。



 帝因作色謂東宮官屬曰:「仁壽宮去此不遠,令我每還京師,嚴備如入敵國。我為患利,不脫衣臥。夜欲得近廁,故在後房。恐有驚急,還就前殿。豈非爾輩欲壞我家國邪!」乃執唐令則等數人,付所司訊鞫。令楊素陳東宮事狀,以告近臣。素顯言之曰:「奉敕向京,令皇太子檢校劉居士餘黨。太子忿然作色,肉戰淚下,云:『居士黨已盡,遣我何處窮討?爾作右僕射,受委自求,何關我事!』又云:『昔大事不遂,我先被誅。今作天子,竟乃令我不如弟,一事已上,不得自由。』因長歎回視云:『
 我大覺身妨!』又云:『諸王皆得奴,獨不與我!』乃向西北奮頭,喃喃細語。」帝曰:「此兒不堪妨承嗣久矣。皇后恆勸我廢,我以布素時生,復長子,望其漸改,隱忍至今。勇昔從南兗州來,語衛王曰:『阿娘不與我一好婦女,亦是可恨。』因指皇后侍兒曰:『皆我物。』此言幾許異事!其婦初亡,即以斗帳安餘老嫗。新婦初亡,我深疑使馬嗣明藥殺。我曾責之,便懟曰:『會當殺元孝矩。』此欲害我而遷怒耳。初,長寧誕育,朕與皇后共抱養之,自懷彼此,連遣來索。且雲定興女,在外私合而生,想此由來,何必是其體胤?昔晉太子取屠家女,其兒即好屠割。今儻非類,便亂宗祐。又
 劉金驎,佞人也,呼定興作家翁。定興愚人,受其此語。我前解金驎者,為其此事。勇昔在宮,引曹妙達共定興女同宴,妙達在外云『我今得勸妃酒。』直以其諸子偏庶,畏人不服,故逆縱之,欲收天下望耳。我雖德慚堯舜,終不以萬姓付不肖子。我恆畏其加害,加防大敵,令欲廢之,以安天下。」左衛大將軍元旻諫曰:「廢立大事,天子無貳言,詔旨若行,後悔無及。讒言罔極,惟陛下察之。」辭直爭強,聲色俱厲,帝不答。



 時姬威又表告太子非法,帝使威盡言。威對曰:「皇太子由來共臣語,唯意在驕奢,欲得樊川以至散關,總規為苑。兼云:『昔漢武將起上林苑,東
 方朔諫,賜朔黃金百斤,幾許可笑!我實無金輒賜此等。若有諫者,正當斬之,不過殺百許人,自然永息。』前蘇孝慈解左衛率,皇太子奮髯揚肘曰:『大丈夫當有一日,終不忘之,決當快意。』又宮內所須,尚書多執法不與,便怒曰:『僕射已下五人,會展三人腳,便使知慢我之禍。』又於苑內築一小城,春夏秋冬作役不輟,營起亭殿,朝造夕改。每云:『至尊嗔我多側庶,高緯、陳叔寶豈是孽子乎?』嘗令師姥卜吉凶,語臣曰:『至尊忌在十八年,此期促矣。』」帝泫然曰:「誰非父母生,乃至於此!我有舊使婦女,令看東宮。奏云:『勿令廣平王至皇太子處。東宮憎婦,亦廣平王
 教之。』元贊亦知其陰惡,勸我於左藏東加置兩隊。初平陳後,宮人好者悉配春坊,如聞不知厭足,於外更有求訪。朕近覽《齊書》,見高歡縱其兒子,不勝忿憤,安可效尤!」於是勇及諸子皆被禁錮,部分收其黨與。楊素舞文鍛煉,以成其獄。勇由是遂敗。



 居數日,有司承素意,奏「元旻身備宿衛,常曲事於勇,情有附託。在仁壽宮,裴弘將勇書於朝堂與旻,題封云,勿令人見。」帝曰:「朕在仁壽宮。有纖小事,東宮必知,疾於驛馬,懌之甚久,豈非此徒邪?」遣武士執旻及弘付法。



 先是,勇嘗於仁壽宮參起居還,途中見一枯槐樹,根幹蟠錯,大且五六圍,顧左右曰:「此堪
 作何器用?」或對曰:「古槐尤堪取火。」於時衛士皆佩火燧,勇因令匠者造數千枚,欲以分賜左右。至是,獲於庫。又藥藏局貯艾數斛,亦搜得之。



 大將為怪,以問姬威。威曰:「太子此意別有所在。比令長寧王已下,詣仁壽宮還,每常急行,一宿便至。恒飼馬千匹,云徑往捉城門,自然餓死。」素以威言詰勇,勇不服曰:「竊聞公家馬數萬匹,勇忝備位太子,有馬千匹,乃是反乎?」素又發泄東宮服玩似加琱飾者,悉陳於庭,以示文帝群官,為太子罪。帝曰:「前簿王世積,得婦女領巾,狀似槊幡,當時遍示百官,欲以為戒。今我兒乃自為之。領巾為槊幡,此是服妖。」使將諸
 物示勇以詰之。皇后又責之罪。帝使使問勇,勇不服。



 太史令袁充進曰:「臣觀天文,皇太子當廢。」上曰:「玄象久見矣。」群臣無敢言者。於是使人召勇。勇見使者,驚曰:「得無殺我邪?」帝戎服陳兵,御武德殿,集百官立於東面,諸親立於西面,引勇及諸子烈於殿庭。命薛道衡宣詔廢勇及其男女為王、公主者並為庶人。命道衡謂勇曰:「爾之罪惡,人神所棄,欲求不廢,其可得邪!」勇再拜曰:「臣合尸之都市,為將來鑒誡。幸蒙哀憐,得全性命」。



 言畢,泣下流襟,既而舞蹈而去。左右莫不憫默。



 又下詔:「左衛大將軍元旻,任掌禁兵,委以心膂,乃包藏姦伏,離間君親,崇長
 厲階,最為魁首。太子左庶子唐令則,策名儲貳,位長宮僚,諂曲取容,音技自進,躬執樂器,親教內人,贊成驕侈,導引非法。太子家令鄒文騰,專行左道,偏被親暱,占問國家,希覬災禍。左衛率司馬夏侯福,內事諂諛,外作威勢,陵侮上下,褻濁宮闈。典膳監元淹,謬陳愛憎,開示怨隙,進引妖巫,營事厭禱。前吏部侍郎蕭子寶,往居省閣,舊非宮臣,進畫姦謀,要射榮利。前主璽下士何竦,假託玄象,妄說妖怪,志圖禍亂,心在速發;兼諸奇服,皆竦規模,增長驕奢,糜費百姓。此之七人,為害斯甚,並處斬刑,妻妾子孫皆沒官。車騎將軍閻毗、東郡公崔君綽、游騎
 尉沈福寶、瀛州人章仇太翼等四人,所為之事,並是悖逆,論其狀迹,罪合極刑。但未能盡戮,並特免死,各決杖一百,身及妻子資財田宅悉沒官。副將作大匠高龍叉,預追番丁,輒配東宮使役,營造亭舍,進入春坊;率更令晉文建、通直散騎侍郎判司農少卿事元衡,料度之外,私自出給,虛破丁功,擅割園地。並處自盡,」於是集群官于廣陽門外,宣詔以戮之。乃移勇於內史省,給五品料食。



 立晉王廣為皇太子,仍以勇付之,復囚於東宮。賜楊素物三千段,元胄、楊約並千段,楊難敵五百段,皆鞫勇之功賞也。



 時文林郎楊孝政上盡諫,言:「皇太子為小人
 所誤,不宜廢黜。」帝怒,撻其胸。尋而貝州長史裴肅表稱:「庶人罪黜已久,當克已自新,請封一小國。」帝知勇黜不允天下情,乃徵肅入朝,具陳廢立意。



 時勇自以廢非其罪,頻請見上,面申冤屈。皇太子遏不得聞。勇於是升樹叫,聞於帝,冀得引見。楊素因奏言:「勇情志昏亂,又癲鬼所著,不可復收。」帝以為然,卒不得見。帝遇疾於仁壽宮,皇太子入侍醫,姦亂事聞於帝。帝抵床曰:「枉廢我兒!」遣追勇。未及發使而崩,秘不發喪。遽收柳述、元巖,繫大理獄,偽敕賜庶人死。追封房陵王,不為立嗣。



 勇有十男:雲昭訓生長寧王儼、平原王裕、安城王筠。高良娣生安平
 王嶷、襄城王恪。王良媛生高陽王該、建安王韶。成姬生潁川王煚。後宮生孝實、孝範。



 初,儼誕,帝聞之曰:「此乃皇太孫,何乃生不得地!」雲定興奏曰:「天生龍種,所以因雲而出。」時人以為敏對。六歲,封長寧郡王。勇敗,並坐廢。上表求宿衛,辭情哀切,帝覽之惻然。楊素進曰:「伏願聖心同於螫手,不宜留意。」



 煬帝踐祚,儼常從行,遇鴆卒。諸弟分徙嶺外,皆敕殺之。



 秦王俊,字阿祗。開皇元年,立為秦王。二年,拜上柱國、河南道行臺尚書令、洛州刺史,時年十二。加右武衛大將軍,領關東兵。三年,遷秦州總管,隴右諸州盡隸焉。俊仁恕
 慈愛,崇敬佛道,請為沙門,不許。六年,遷山南道行臺尚書令。



 伐陳之役,為山南道行軍元帥,督三十總管,水陸十餘萬,屯漢口,為上流節度。



 尋授揚州總管、四十四州諸軍事,鎮廣陵。轉并州總管、二十四州諸軍事。初頗有令問,文帝聞而大悅。後漸奢侈,違犯制度,出錢求息。帝遣按其事,與相連坐者百餘人。於是盛脩宮室,窮極侈麗。俊有巧思,每親運斤斧,工巧之器,飾以珠玉。



 為妃作七寶幕籬,重不可戴,以馬負之而行。徵役無已。置渾天儀、測景表。又為水殿,香塗粉壁,玉砌金堦,梁柱楣棟之間,周以明鏡,間以寶珠,極瑩飾之美。



 每與賓客伎女絃
 歌於上。



 俊頗好內,妃崔氏性妒,甚不平之,遂於瓜中進毒。俊由是遇疾,徵還京師。



 以俊奢縱,免官,以王就第。左武衛將軍劉昇諫曰:「秦王非有他過,但費官物、營廨舍而已。臣謂可容。」帝曰:「法不可違。」昇固諫,帝忿然作色,升乃止。



 楊素復進諫,以秦王過不應至此。帝曰:「我是五兒之父,非兆人之父。若如公意,何不別制天子兒律!以周公為人,尚誅管、蔡。我誠不及周公遠矣,安能虧法乎!」



 卒不許。



 俊疾篤,含銀,銀色變,以為遇蠱。未能起,遣使奉表陳謝。帝責以失德。大都督皇甫統上表請復王官,不許。歲餘,以疾篤,復拜上柱國。二十年六月,薨於秦邸。帝哭
 之數聲而已,曰:「晉王前送一鹿,我令作脯,擬賜秦王。今亡。可置靈坐之前。心已許之,不可虧信。」帝及后往視,見大蜘蛛、大蛷螋從枕頭出,求之不見。窮之,知妃所為也。俊所為侈麗物悉命焚之。敕送終之具,務從儉約,以為從世法。王府僚佐請立碑,帝曰:「欲求名,一卷史書足矣,何用碑為!若子孫不能保家,徒與人作鎮石耳。」



 妃崔氏以毒王故,下詔廢絕,賜死於其家。子浩,崔氏所生也。以其母譴死,遂不得立。於是以秦國官為喪主。俊長女永豐公主,年十三,遭父憂,哀慕盡禮,免喪,遂絕酒肉。每忌日,輒流涕不食。有開府王延者,性忠厚,領俊親信兵十
 餘年,俊甚禮之。及俊疾,延恆在閤下,衣不解帶。俊薨,勺飲不入口者數日,羸頓骨立。帝聞憫之,賜以御藥,授驃騎將軍,典宿衛。俊葬日,延號慟而絕。帝嗟異之,令通事舍人弔祭,詔葬延於俊墓側。



 煬帝即位,立浩為秦王,以奉孝王嗣。封浩弟湛濟北侯。後以浩為河陽都尉。



 楊玄感作逆之際,左翊衛大將軍宇文述勒兵討之。至河陽,脩啟於浩,浩詣述營,共相往復,有司劾浩以諸侯交通內臣,竟坐廢免。宇文化及弒逆,立浩為帝。化及敗於黎陽,北走魏縣,自僭為帝,因而害之。



 湛驍果有膽烈。大業初,為滎陽太守,坐浩免,亦為化及所害。



 庶
 人秀,開皇元年,立為越王。未幾,徙封於蜀,拜柱國、益州總管、二十四州諸軍事。二年,進上柱國、西南道行臺尚書令,本官如故。歲餘而罷。十二年,入為內史令、右領軍大將軍。尋出鎮於蜀。



 秀有膽氣,容貌瑰偉,美有鬚髯,多武藝,甚為朝臣所憚。帝每謂文獻皇后曰:「秀必以惡終。我在當無慮,至兄弟必反。」兵部侍郎元衡使於蜀,秀深結於衡,以左右為請。衡既還京師,請益左右,帝不許。大將軍劉噲之討西爨,帝令上開府楊武通將兵繼進。秀使嬖人萬知先為武通行軍司馬,帝以秀任非其人,譴責之,因謂群臣曰:「壞我法者,必在子孫。譬如猛獸,物不
 能害,反為毛間蟲所損食耳。」



 於是遂分秀所統。



 秀漸奢侈,違犯制度,車馬被服擬於天子。及太子勇廢,秀甚不平。皇太子恐秀終為後變,陰令楊素求其罪狀而譖之。仁壽二年,徵還京師,見不與語。明日,使使切讓之。皇太子及諸王流涕庭謝,帝曰:「頃者俊糜費財物,我以父道訓之。



 今秀蠹害生靈,當以君道繩之。」乃下以法。開府慶整諫曰:「庶人勇既廢,秦王已薨,陛下兒子無多,何至如是!蜀王性甚耿介,今被責,恐不自全。」帝大怒,欲斷其舌。因謂群臣曰:「當斬秀於市以謝百姓。」乃令楊素、蘇威、牛弘、柳述、趙綽推之。太子陰作偶人,書帝及漢王姓字,縛
 手釘心,令人埋之華山下,令楊素發之。又作檄文曰「逆臣賊子,專弄威柄,陛下唯守虛器,一無所知」,陳甲兵之盛,云「指期問罪」,置秀集中,因以聞奏。帝曰:「天下寧有是邪!」乃廢為庶人,幽之內侍省,不得與妻子相見,令給獠婢二人驅使之。與連坐百餘人。



 秀既幽逼,憤懣不知所為,乃上表陳己愆,請與其愛子爪子相見,并請賜一穴,今骸骨有所。帝乃下詔數其罪曰:「汝地居臣子,情兼家國,庸蜀險要,委以鎮之。



 汝乃干紀亂常,懷惡樂禍,闢睨二宮,佇望災釁,容納不逞,結構異端。我有不和,汝便覘候,望我不起,便有異心。皇太子,汝兄也,次當建立,汝假
 托妖言,乃云不終其位。妄稱鬼怪,又道不得入宮,自言骨相非人臣,德業堪承重器。妄道清城出聖,欲己當之,詐稱益州龍見,託言吉兆。重述木易之姓,更脩成都之宮。



 妄說禾乃之名,以當八千之運。橫生京師妖異,以證父兄之災;妄造蜀地徵祥,以符已身之籙。汝豈不欲得國家惡也?天下亂也?輒造白玉之珽,又為白羽之箭,文物服飾,豈似有君?鳩集左道,符書厭鎮,漢王與汝,親則弟也,乃畫其形像,題其姓名,縛手釘心,枷鎖杻械。仍云請西岳華山慈父聖母神兵九億萬騎,收楊諒魂神,閉在華山下,勿令散蕩。我之於汝,親則父也,復云請西岳
 華山慈父聖母,賜為開化楊堅夫妻,回心歡喜。又畫我形像,縛手撮頭,仍云請西岳神兵收楊堅魂神。



 如此形狀,我今不知楊諒、楊堅是汝何親也!包藏兇匿,圖謀不軌,逆臣之迹也。



 希父之災,以為身幸,賊子之心也。懷非分之望,肆毒心於兄,悖惡之行也。嫉妒於弟,無惡不為,無孔懷之情也。違犯制度,壞亂之極也。多殺不辜,豺狼之暴也。



 剝削人庶,酷虐之甚也。唯求財貨,市井之業也。專事妖邪,頑囂之性也。弗克負荷,不材之器也。凡此十者,滅天理,逆人倫,汝皆為之,不祥之甚也。欲免患禍,長守富貴,其可得乎!」後聽與其子同處。煬帝即位,禁錮如
 初。宇文化及之弒逆也,欲立秀為帝,群議不許。於是害之,並其諸子。



 庶人諒,字德章,一名傑,小字益錢。開皇元年,立為漢王。十二年,為雍州牧,加上柱國、右衛大將軍。轉左衛大將軍。十七年,出為並州總管,帝幸溫湯而送之。自山以東,至于滄海,南拒黃河,五十二州盡隸焉。特許以便宜,不拘律令。



 十八年,起遼東之役,以諒為行軍元帥。至遼水,師遇疾疫,不利而還。十九年,突厥犯塞,以諒為行軍元帥,竟不臨戎。文帝甚寵愛之。



 諒自以居天下精兵處,以太子讒廢,居常怏怏,陰有異圖。遂諷帝云:「突厥方強,太
 原即為重鎮,宜脩武備。」帝從之。於是大發工役,繕脩器械,貯納於并州。招集亡命,左右私人,殆將數萬。王頍者,梁將王僧辯之子,少倜儻,有奇略,為諒諮議參軍。蕭摩訶者,陳氏舊將。二人俱不得志,每鬱鬱思亂,並為諒親善。



 及蜀王以罪廢,諒愈不自安。會文帝崩,使車騎屈突通徵之,不赴,遂發兵反。



 總管司馬皇甫誕諫,諒怒,收繫之。王頍說諒曰:「王所部將吏家屬盡在關西,若用此等,即宜長驅深入,直據京都,所謂疾雷不及掩耳。若但欲割據舊齊之地,宜任東人。」諒不能專之。乃兼用二策,唱言:「楊素反,將誅之。」



 總管府兵曹河東裴文安說諒曰:「井
 陘以西,是王掌據內,山東士馬,亦為我有,宜悉發之。分遣羸兵,屯守要路,仍令隨方略地;率其精銳,直入蒲津。文安請為前鋒,王以大軍繼後,風行電擊,頓於霸上,咸陽以東可指麾而定。京師震擾,兵不暇集,上下相疑,群情離駭,我即陳兵號令,誰敢不從!旬日之間,事可定矣。」



 諒大悅。於是遣所署大將軍餘公理將兵出太谷,以趣河陽。大將軍綦良出滏口,以趣黎陽。大將軍鄧建出井陘,以略燕、趙。柱國喬鐘馗出鴈門。署文安為柱國,紇單貴、王聃、大將軍茹茹天保、侯莫陳惠直指京師。未至蒲津百餘里。諒忽改圖,令紇單貴斷河橋,守蒲州,而召
 文安。文安至曰:「兵機詭速,本欲出其不意。王既不行,文安又返,使彼計成,大事去矣。」諒不對。於是從亂者十九州,乃以王聃為蒲州刺史,裴文安為晉州,薛粹為絳州,梁菩薩為潞州,韋道正為韓州,張伯英為澤州。遣偽署大將軍常倫進兵絳州,遇晉州司法仲孝俊之子,謂曰:「吾曉天文遁甲,今年起兵,得晉地者王。」孝俊聞之曰:「皇太子常為晉王,故曰晉地,非謂反徒也。」時潞州有官羊生羔,二首相背,以為諒之咎徵。



 煬帝遣楊素率騎五千,襲王聃、紇單貴於蒲州,破之,於是率步騎四萬趣太原。



 諒使趙子開守高壁,楊素擊走之。諒大懼,拒素於蒿澤。
 屬天大雨,諒欲旋師,王頍諫曰:「楊素懸軍,士馬疲弊,王以銳卒親戎擊之,其勢必舉。今見敵而還,示人以怯,阻戰士之心,益西軍之氣,願必勿還。」諒不從,退守清源。素進擊之,諒與官兵大戰,死者萬八千人。諒退保並州,楊素進擊之,諒乃降。百僚奏諒罪當死,帝曰:「朕終鮮兄弟,情不忍言,欲屈法恕諒一死。」於是除名,絕其屬籍,竟以幽死。



 先是,并州謠言:「一張紙,兩張紙,客量小兒作天子。」時偽署官告身皆一紙,別授則二紙。諒聞謠喜曰:「我幼字阿客,『量』與『諒』同音,吾於皇家最小。」以為應之。



 子顥,因而禁錮。宇文化及弒逆之際,遇害。



 煬帝三男:蕭皇后生元德太子昭、齊王暕。蕭嬪生趙王杲。



 元德太子昭,煬帝長子也。初,文帝以開皇三年四月庚午,夢神自天而降,云是天神將生降。寤,召納言蘇威以告之。及聞蕭妃在並州有娠,迎置大興宮之客省。



 明年正月戊辰而生昭,養於宮中,號大曹主。三歲時,於玄武門弄石師子,文帝與文獻皇后至其所。文帝適患腰痛,舉手馮后,昭因避去,如此者再三。文帝歎曰:「天生長者,誰復教乎!」由是大奇之。文帝嘗謂曰:「當為爾娶婦。」應聲而泣。



 文帝問其故,對曰:「漢王未婚時,恆在至尊所,一朝娶婦,便則出外。懼將違離,是以啼耳。」上嘆其有至性,
 特鐘愛焉。年十二,立為河南王。仁壽初,徙為晉王。



 拜內史令,兼左衛大將軍。轉雍州牧。煬帝即位,便幸洛陽宮,昭留守京師。及大業元年,帝遣使者立為皇太子。



 昭有武力,能引強。性謙沖,言色恂恂,未嘗忿怒。其有深可嫌責者,但云「大不是」。所膳不許多品,帷席極於儉素。臣吏有老父母,必親問其安否,歲時皆有惠賜。其仁愛如此。明年,朝於洛陽,後數月,將還京師,願得少留,帝不許。



 拜請無數,體素肥,因致勞疾。帝令巫者視之,云房陵王為祟。未幾而薨,時年二十三。先是,太史奏言楚分有喪,於是改封越公楊素於楚。及昭薨日,而素亦薨,蓋隋、楚同
 分也。詔內史侍郎虞世基為哀冊文,帝深追悼之。



 昭妃慈州刺史博陵崔弘昇女。後秦王妃以蠱毒獲譴,昭奏曰:「惡逆者,乃新婦之姑,請離之。」乃娶滑國公京兆韋壽女為妃。昭有子三人:韋妃生恭皇帝,大劉良娣生燕王倓,小劉良娣生越王侗。



 倓字仁安,敏慧美咨容,煬帝於諸孫中特所鐘愛,常置左右。性好讀書,尤重儒素,造次所及,有若成人。良娣早終,每忌日未嘗不流涕嗚咽,帝由是益奇之。



 宇文化及弒逆之際,倓覺變,欲入奏,恐露其事,因與梁公蕭鉅、千牛宇文晶等穿芳林門側水竇入。至玄武門,詭奏曰:「臣卒中惡,命懸俄頃,請得面辭,死
 無所恨。」冀見帝,為司宮者所遏,竟不得聞。俄而難作,遇害,時年十六。



 越王侗,字仁謹,美姿容,性寬厚。大業三年,立為越王。帝每巡幸,侗常留守東都。楊玄感反,與戶部尚書樊子蓋拒之。事平,朝於高陽,拜高陽太守。俄以本官留守東都。十三年,帝幸江都,復令侗與金紫光祿大夫段達、太府卿元文都、攝戶部尚書韋津、右武衛將軍皇甫無逸等總留臺事。



 宇文化及之弒逆,文都等議尊立侗,大赦,改元曰皇泰。謚帝曰明,廟號世祖,追尊元德太子為孝成皇帝,廟號世宗,尊其母劉良娣為皇太后。以段達為納
 言、右翊衛大將軍、攝禮部尚書,王世充為納言、左翊衛大將軍、攝吏部尚書,元文都為內史令、左驍衛大將軍,盧楚亦內史令,皇甫無逸為兵部尚書、右武衛大將軍,郭文懿為內史侍郎,趙長文為黃門侍郎,委以機務,為金書鐵券,藏之宮掖。于時洛陽稱段達等為「七貴」。



 未幾,宇文化及以秦王浩為天子,來次彭城,所經城邑,多從逆黨。侗懼,遣使者蓋琮、馬公政招懷李密。密遂請降,侗大忻悅,禮其使甚厚。即拜密為太尉、尚書令、魏國公,令拒化及。仍下書曰:我大隋之有天下,於茲三十八載。高祖文皇帝聖略神功。載造區夏。世祖明皇帝則天法地,
 混一華戎。東暨蟠木,西通細柳,前踰丹徼,後越幽都,日月之所臨,風雨之所至,圓首方足,稟氣食毛,莫不盡入提封,皆為臣妾。加以寶貺畢集,雲瑞咸臻,作樂制禮,移風易俗。智周寰海,萬物咸受其賜;道濟天下,百姓用而不知。世祖往因歷試,統臨南服,自居皇極,順茲望幸。所以往歲省方,展禮肆觀,停鑾駐蹕,按駕清道,八屯如昔,七萃不移。豈意釁起非常,逮於軒陛,災生不意,廷及冕旒。奉諱之日,五情崩殞,攀號荼毒,不能自勝。



 且聞之自古,代有屯剝,賊臣逆子,何世無之。至如宇文化及,世傳庸品。其父述,往屬時來,早沾厚遇,賜以昏媾,置之公輔。
 位尊九命,祿重萬鐘,禮極人臣,榮冠世表,徒承海岳之恩,未有涓塵之答。化及以此下材,夙蒙顧眄,出入外內,奉望階墀。昔陪籓國,統領衛兵,及從升皇祚,陪列九卿。但本性兇狠,恣其貪穢,或交結惡黨,或侵掠商貨,事重刑簽,狀盈獄簡。在上不遺簪履,恩加草芥,應至死辜,每蒙恕免。三經除解,尋復本職;再徙邊裔,仍即追還。生成之恩,昊天罔極;獎擢之義,人事罕聞。化及梟獍為心,禽獸不若,從毒興禍,傾覆行宮。



 諸王兄弟,一時殘酷,痛暴行路,世不忍言。有窮之在夏時,犬戎之於周世,釁辱之極,亦未是過。朕所以刻骨崩心,飲膽嘗血,瞻天視地,無
 處自容。



 今王公卿士,庶尹百辟,咸以大寶鴻名,不可顛墜,元兇巨猾,須早夷殄,翼戴朕躬,嗣守寶位。顧惟寡薄,志不逮此。今者出黼扆而仗旄鉞,釋衰麻而擐甲胄,銜冤誓眾,忍淚臨兵,指日遄征,以平大盜。且化及偽立秦王之子,幽遏比於拘囚;其身自稱霸相,專擅擬於九五。履踐禁御,據有宮關,昂首揚眉,初無慚色。衣冠朝望,外懼兇威,志士誠臣,內懷憤怨。以我義師,順彼天道,梟夷醜族,匪夕伊朝。



 太尉、尚書令魏公,丹誠內發,宏略外舉,率勤王之師,討違天之逆。果毅爭先,熊羆競進,金鼓振讋,若火焚毛,鋒刃從橫,如湯沃雪。魏公志存匡濟,投袂
 前驅,朕親御六軍,星言繼軌。以此眾戰,以斯順舉,擘山可以動,射石可以入。



 況賊擁此人徒,皆有離德,京都侍衛,西憶鄉家,江左淳人,南思邦邑。比來表書駱驛,人信相尋。若王師一臨,舊章暫睹,自應解甲倒戈,冰銷棄散。且聞化及自恣,天奪其心,殺戮不辜,挫辱人士,莫不道路以目,號天跼地。朕今復仇雪恥,梟轅者一人,拯溺救焚,所哀者士庶。唯望天鑒孔殷,祐我宗社,億兆感義,俱會朕心。梟戮元兇,策勳飲至,四海交泰,稱朕意焉。兵衛軍機,並受魏公節度。



 密見使者,大悅,北面拜伏,臣禮甚恭,遂東拒化及。



 七貴頗不協。未幾,元文都、盧楚、郭文懿、
 趙長文等為世充所殺,皇甫無逸遁歸京師。世充詣侗所陳謝,辭情哀苦。侗以為至誠,命之上殿,被髮為盟,誓無貳志。自是侗無所關預。及世充破李密,眾望益歸之,遂自為鄭王,總百揆,加九錫,備法物,侗不能禁。段達、雲定興等十人入見侗曰:「天命不常,鄭王功德甚盛,願陛下遵唐、虞之迹。」侗怒曰:「天下者,高祖之天下,東都者,世祖之東都。若隋德未衰,此言不可而發。必天命有改,亦何論於禪讓!公等或先朝舊臣,或勤王立節,忽有斯言,朕亦何望!」神色凜然,侍衛者莫不流汗。既而退朝,對良娣而泣。世充更使謂曰:「今海內未定,須得長君,待四方
 乂安,復子明辟。必若前盟,義不違負。」侗不得已,遜位於世充,遂被幽於含涼殿。世充僭偽號,封潞國公。



 有宇文儒童、裴仁基等謀誅世充,復尊立侗。事泄,並見害。世充兄世惲因勸世充害侗。世充遣其姪行本齎鳩詣侗曰:「願皇帝飲此酒。」侗知不免,請與母相見,不許。遂布席焚香禮佛,祝曰:「從今以去,不生帝王尊貴家。」及仰藥,不能時絕,更以帛縊之。世充偽謚曰恭皇帝。



 齊王暕,字世朏出,小字阿孩。美容儀,疏眉目,少為文帝所愛。開皇中,立為豫章王。及長,頗涉經史。尤工騎射。初為內史令。仁壽中,拜揚州總管、江淮以南諸軍事。煬帝即
 位,進封齊王。大業二年,帝初入東都,盛陳鹵簿,柬為軍導。轉豫州牧。俄而元德太子薨,朝野注望,咸以暕當嗣。帝又敕吏部尚書牛弘妙選官屬,公卿由是多進子弟。明年,轉雍州牧,尋徙河南尹、開府儀同三司。元德太子左右二萬餘人悉隸於暕,寵遇益隆。自樂平公主及諸戚屬競來致禮,百官稱謁,填咽道路。



 暕頗驕恣,暱近小人,所行多不法。遣喬令則、劉虔安、裴該、皇甫諶、厙狄仲錡、陳智偉等采求聲色狗馬。令則等因此放縱,方人家有女者,輒矯暕命呼之,載入暕宅,因緣藏匿,恣行淫穢而後遣之。仲錡、智偉二人詣隴西,撾炙諸胡,責其名馬,得
 數匹以進於暕。暕令還主,仲錡等詐言王賜,將歸家,暕不之知也。又樂平公主嘗奏帝,云柳氏女美者,帝未有所答。久之,主復以柳氏進暕,暕納之。



 後帝問主柳氏女所在,主曰:「在齊王所。」帝不悅。暕於東都營第,大門無故崩,應事栿中折,識者以為不祥。後從帝幸榆林,暕督後軍,步騎五萬,恆與帝相去數十里而舍。會帝於汾陽宮大獵,詔暕以千騎入圍。暕大獲麋鹿以獻,而帝未有得也,怒從官,皆言為暕左右所遏,獸不得前。帝於是怒,求暕罪失。時制縣令無故不得出境,有伊闕令皇甫詡幸於暕,違禁將之汾陽宮;又京兆人達奚通有妾王氏善
 歌,貴游宴聚,多或要致,於是展轉亦入王家。御史韋德裕希旨劾暕。帝令甲士千餘,大索暕第,因窮其事。



 暕妃韋氏,戶部尚書沖之女也,早卒。暕遂與妃姊元氏婦通,生一女。外人皆不得知,陰引喬令則於第內酣宴,令則稱慶,脫暕帽以為歡。召相工遍視後庭,相工指妃姊曰:「此產子者當為皇后,貴不可言。」時國無儲副,暕自謂次當得立。



 又以元德太子有三子,內常不安,陰挾左道,為厭勝事。至是,皆發。帝大怒,斬令則等數人,妃姊賜死,暕府僚皆斥之邊遠。時趙王杲猶在孩孺,帝謂侍臣曰:「朕唯有暕一子,不然者,當肆諸市朝,以明國憲也。」



 暕自是
 恩寵日衰,雖為京尹,不復關預時政。帝恆令武賁郎將一人監其府事,暕有微失,輒奏之。帝亦慮暕生變,所給左右,皆以老弱備員而已。柬每懷危懼,心不自安。又帝在江都宮元會,暕具法服將朝,無故有血從裳中而下;又坐齋中,見群鼠數十,至前而死,視皆無頭。暕甚惡之。俄而化及作亂,兵將犯蹕,帝聞之,顧蕭后曰:「得非阿孩也?」其見疏忌如此。化及復令人捕暕,時尚臥未起,賊進,暕驚曰:「是何人?」莫有報者。暕猶謂帝令捕之,曰:「詔使且緩,兒不負國家!」



 賊曳至街,斬之,及其二子亦遇害。暕竟不知殺者為誰。時年三十四。



 有遺腹子愍,與蕭后同入
 突厥,處羅可汗號為隋王。中國人沒入北蕃者,悉配之以為部落,以定襄城處之。及突厥滅,乃獲之。貞觀中,位至尚衣奉御,永徽初,卒。



 趙王杲,小字季子。年七歲,以大業九年封趙王。尋授光祿大夫,歷河南尹,行江都太守。杲聰令,美容儀,帝有所制詞賦,杲多能誦之。性至孝,嘗見帝風動,不進膳,杲亦終日不食。又蕭后嘗灸,杲先請試炷,后不許之。杲泣請曰:「后所服藥,皆蒙嘗之。今灸,願聽嘗炷。」悲咽不已。后為停灸,由是尤鐘愛。後遇化及反,杲在帝側,號慟不已。裴虔通使斬之帝前而血湔御服。時年十二。



 論曰:周建懿親,漢開盤石,內以敦睦九族,外以輯寧億兆,深根固本,崇獎王室,安則有以同其樂,衰則有以恤其危,所由來久矣。自魏、晉已下,多失厥中,不遵王度,各徇所私。抑之則勢齊於匹夫,抗之則權侔於萬乘,矯枉過正,非一時也。得失詳於前史,不復究而論焉。隋文昆弟之恩,素非篤睦,閨房之隙,又不相容。至於二世承基,茲弊愈甚。是以滕穆暴薨,人皆竊議,蔡王將沒,自以為幸。



 唯衛王養於獻后,故任遇特隆,而諸子遷流莫知死所,悲夫!其錫以茅土,稱為盤石,特無甲兵之衛,居與皁吏為伍。外內無虞,顛危不暇,時逢多難,將何望哉!



 河間
 屬乃葭莩,地非寵逼,故高位厚秩,與時終始。楊慶二三其德,志在茍生,變本宗如反掌,棄慈母若遺迹,及身而絕,固宜然矣。文帝五子,莫有終其天年。房陵資於骨肉之親,篤於君臣之義,經綸締構,契闊夷險,撫軍臨國,凡二十年。雖三善未稱,而視膳無闕。恩寵既變,讒言間之,顧復之慈,頓隔於人理;父子之道,遂滅於天性,隋室將亡之效,眾庶皆知之矣。《慎子》曰:「一兔走街,百人逐之;積兔於市,過者不顧。」豈其無欲哉?分定故也。房陵分定久矣,而帝一朝易之,開逆亂之源,長覬覦之望。又維城肇建,崇其威重,恃寵而驕,厚自封植,進之既踰制,退之不
 以道,俊以憂卒,實此之由。俄屬天步方艱,讒人已勝,尺布斗粟,莫肯相容。秀窺岷、蜀之阻,諒起晉陽之甲,成茲亂常之釁,蓋亦有以動之也。



 《棠棣》之詩徒賦,有庳之封無期,或幽囚於囹圄,或顛殞於鳩毒。本根既絕,枝葉畢翦,十有餘年,宗社淪陷。自古廢嫡立庶,覆族傾宗者多矣,考其亂亡之禍,未若有隋之酷。《詩》云:「殷鑒不遠,在夏后之世。」後之有國有家者,可不深戒哉!元德謹重,有君人之量,降年不永,哀哉!齊王敏慧可稱,志不及遠,頗懷驕僭,故帝疏而忌之,內無父子之親,貌展君臣之敬。身非積善,國有餘殃,至令趙及燕、越,皆不得死,悲夫!



\end{pinyinscope}