\article{卷十六列傳第四 道武七王 明元六王 太武五王}

\begin{pinyinscope}

 道武皇帝十男:
 宣穆劉後生明元皇帝;賀夫人生清河王紹;大王夫人生陽平王熙;王夫人生河南王曜;河間王修、長樂王處文二王母氏闕;段夫人生廣平王連、京兆王黎;皇子渾及聰母氏並闕,皆早薨,無傳。



 清河王紹字受洛拔,天興六年封。性兇狠險悖,好劫剝行人,斫射犬豕,以為戲樂。有孕婦,紹剖觀其胎。道武嘗怒之,倒懸井中,垂死乃出。明元常以義方責之,由此不協。而紹母賀夫人有譴,帝將殺之。會日暮,未決。賀氏密告急於紹,紹乃與帳下及宦者數人逾宮犯禁。帝驚起,求弓刀不及,暴崩。明日,宮門至日中不開,紹稱詔召百寮於西宮端門前北面,紹從門扇間謂曰:「我
 有父,亦有兄,公卿欲從誰也?」王公以下皆失色,莫有對者。良久,南平公長孫嵩曰:「臣等不審登遐狀。」唯陰平公元烈哭泣而去。於是朝野兇兇,人懷異志。肥如侯賀護舉烽於安陽城北,故賀蘭部人皆往赴之。其餘舊部,亦率子弟,招集故人,往往相聚。紹聞人情不安,乃出布帛班賜王公以下。



 先是,明元在外,聞變
 乃還,潛於山中,使人夜告北新侯安同,眾皆響應。衛士執送紹,於是賜紹母子死,誅帳下閹官、宮人為內應者十數人。其先犯乘輿者,群臣於城南都街生臠食之。
 紹時年十
 六。紹母
 即獻明皇后妹也,美而艷。道武如賀蘭部,見而悅之,告獻明后請納焉。后曰:「不可。此過美,不善,且已有夫。」



 帝密令人殺其夫而納之,生紹,終致大逆焉。



 陽平王熙,天興六年封,聰達有雅操。明元練兵於東部,詔熙督十二軍校閱,甚得軍儀,賞賜隆厚。泰常六年,薨,帝哀慟不已。長子佗襲爵。



 佗性忠厚,武藝無過者。後改
 封淮南王,鎮武牢,威名甚著。孝文時,位司徒,賜安車几杖,入朝不趨。太和十二年,薨。時孝文有事太廟,始薦,聞之,廢祭,輿駕親臨哀慟,禮賵有加,謚曰靖王。



 世子吐萬早卒。



 子僖王顯襲祖爵,薨。



 子世遵襲。孝明時,為荊州刺史。在邊境,前代以來,互相抄掠,世遵到州,不聽侵擾。其弟均時在荊州,為朝陽戍主。有南戍主妻,三月三日遊戲沔水側,均輒遣部曲掠取。世遵聞之,責均,遂移還本戍,吳人感荷。後頗行貨賄,散費邊儲,是以聲名有損。薨於定州刺史,謚曰康王。



 吐萬弟鐘葵,早卒。



 長子法壽,累遷安州刺史。法壽先令所親,微服入境,觀察風俗。下車
 便大行賞罰,於是境內肅然。後於河陰遇害。



 子慶智,性貪鄙。為太尉主簿,事無大小,得物然後判,或十數錢,或二十錢,得便取之,府中號為「十錢主簿。」



 法壽弟法僧,位益州刺史,殺戮自任,威怒無恆。王、賈諸姓,州內人士,法僧皆召為卒伍,無所假縱。於是合境皆反,招引外寇。後拜徐州刺史。法僧本附元叉,以驕恣,恐禍及己,將謀為逆。時領主書兼舍人張文伯奉使徐州,法僧謂曰:「我欲與卿去危就安,能從我否?」文伯曰:「安能棄孝義而從叛逆也!」法僧將殺之,文伯罵曰:「僕寧死見文陵松柏,不能生作背國之虜!」法僧殺之。孝昌元年,法僧殺行臺高諒,
 反於彭城。自稱尊號,改元天啟。大軍致討,法僧奔梁。其武官三千餘人戍彭城者,法僧皆印額為奴,逼將南度。梁武帝授法僧司空,封始安郡王,尋改封宋王,甚見優寵。又進位太尉,仍立為魏主。不行,授開府儀同三司、郢州刺史,乃徵為太尉。卒於梁,謚曰襄厲王。子景隆、景仲。



 景隆初封丹楊公,位廣州刺史,徙徐州,改封彭城王。丁父憂,襲封宋王,又為廣州刺史。卒。梁復以景仲為廣州刺史,封枝江縣公。侯景作亂,遣誘召之,許奉為主。景仲將應之,為西江督護陳霸先所攻,乃縊而死。



 河南王曜,天興六年封。五歲,嘗射雀於道武前,中之,帝
 驚歎焉。及長,武藝絕人,與陽平王熙等並督諸軍講武,眾咸服其勇。薨。



 長子提襲。驍烈有父風,改封潁川王。迎昭儀于塞北。時年十六,有夙成之量,殊域敬焉。後改封武昌,累遷統萬鎮都大將,甚見寵待。薨,謚曰成王。



 長子平原襲爵。忠果有智略。為齊州刺史,善於懷撫。孝文時,妖賊司馬小君自稱晉後,屯聚平陵,年號聖君。平原身自討擊,禽小君,送京師斬之。又有妖人劉舉,自稱天子,復討斬之。時歲頻不登,齊人饑饉,平原以私米三千餘斛為粥,以全人命。北州戍卒一千餘人,還者皆給路糧,百姓咸稱詠之。遷征南大將軍、開府、雍州刺史,鎮長安。
 薨,謚曰簡王。



 長子和,字善意,襲爵。初,和聘乙氏公主女為妃,生子顯,薄之。以公主故,不得遣出。因忿,遂自落髮為沙門。既不幸其母,乃捨顯,以爵讓其次弟鑒。鑒固辭。公主以其外孫不得襲爵,訴於孝文。孝文詔鑒終之後,令顯襲爵,鑒乃受之。



 鑒字紹達,沉重少言,寬和好士。為齊州刺史。時革變之始,鑒上書遵孝文之旨,採齊之舊風。軌制粲然,皆合規矩。孝文下詔褒美,班之天下,一如鑒所上。



 齊人愛詠,咸曰耳目更新。



 孝文崩後,和罷沙門歸俗。棄其妻子,納一寡婦曹氏為妻。曹氏年長,大和十五歲,攜男女五人,隨鑒至歷城,干亂政事。和與曹及五
 子七處受納,鑒皆順其意,言無不從。於是獄以賄成,取受狼籍,齊人苦之,鑒名大損。轉徐州刺史。屬徐、兗大水,人多飢餓,鑒表加賑恤,人賴以濟。先是,京兆王愉為徐州,王既年少,長史盧陽烏寬以馭下,郡縣多不奉法。鑒表梁郡太守程靈虯虐政殘人,盜寇並起。



 詔免靈虯,於是徐境肅然。薨,謚悼王。



 和與鑒子伯崇競求承襲,詔聽和襲,位東郡太守。先是,郡人孫天恩家豪富,嘗與和爭地,遣奴客打和垂死。至此,和誣天恩與北賊來往,父子兄弟一時俱戮,資財田宅皆沒於官。天恩宗從欲詣闕訴冤,以和元叉之親,不敢告列。和語其郡人曰:「我覓一
 州,亦應可得。念此小人,痛入骨髓,故乞此郡,以報宿怨,此後更不求富貴。」識者曰:「王當沒於此矣!」薨,贈相州刺史。



 河間王脩,天賜四年封。薨,無子,太武詔河南王曜子羯兒襲,改封略陽王。



 正平初,有罪賜死,爵除。



 長樂王處文,天賜四年封。聰辯夙成。年十四,薨。明元悼傷之,自小僉至葬,常親臨哀慟。陪葬金陵,無子,爵除。



 廣平王連,天賜四年封。薨,無子,太武以陽平王熙第二子渾為南平王,以繼連後。渾好弓馬,射鳥輒歷飛而中之,日射兔得五十頭。太武嘗命左右分射,勝者中的籌
 滿,詔渾解之,三發皆中。帝大悅,器其藝能,常引侍左右。累遷涼州鎮將、都督西戎諸軍事、領護西域校尉,恩著涼土。更滿還京,父老皆涕泣追送,如違所親。薨。



 子飛襲。後賜名霄。身長九尺,腰帶十圍,容貌魁偉,雅有風則。貞白卓然,好直言正諫,朝臣憚之。孝文特垂欽重,除宗正卿。詔曰:「自今奏事,諸臣相稱,可云姓名;唯南平王一人,可直言其封。」遷左光祿大夫。薨,賜東園第一祕器。



 孝文緦衰臨霄喪,宴不舉樂,謚曰安王。子纂襲。



 京兆王黎,天賜四年封。薨。子吐相襲,改封江陽王。薨,無子。



 獻文以南平王霄第二子繼字世仁為後,襲封江陽
 王。宣武時,為青州刺史。為家僮取人女為婦妾,又以良人為婢,為御史所彈,坐免官爵。及靈太后臨朝,繼子叉先納太后妹,復繼本封;後徙封京兆王,歷司徒,加侍中。繼,孝文時已歷內外顯任,靈太后臨朝,入居心膂,歷轉台司。頻表遜位,轉太保,侍中如故,加前後部鼓吹。詔以至節,禮有朝慶,繼位高年宿,可依齊郡王簡故事,朝訖引坐,免其拜伏。轉太傅,侍中如故。時叉執殺生之權,拜受之日,送者傾朝,有識者為之致懼。又詔令乘步挽至殿廷,兩人扶侍,禮與丞相高陽王埒。後除使持節、侍中、太師、大將軍、錄尚書事、大都督、節度西道諸軍事。及出
 師,車駕臨餞,傾朝祖送。



 尋加太尉公。及班師,繼啟求還復封江陽,詔從之。繼晚更貪婪,牧守令長新除赴官,無不受納貨賄,以相託付。妻子各別請屬,至乃郡縣微吏,亦不獲平心選舉。



 憑叉威勢,法官不敢糾擿,天下患之。叉黜,繼廢於家。初,爾朱榮之為直寢,數以名馬奉叉,叉接以恩意,榮甚德之。建義初,復以繼為太師、司州牧。永安元年,薨,贈假黃鉞都督九州諸軍,錄尚書事、大丞相如故,謚曰武烈。



 叉字伯俊,小字夜叉。靈太后臨朝,以叉妹夫,除通直郎。叉妻封新平君,後遷馮翊君,拜女侍中。叉女夭,靈太后詔贈鄉主。叉累加侍中、領軍將軍。既在
 門下,兼總禁兵,深為靈太后所信委。太傅、清河王懌以親賢輔政,每欲斥黜之。叉遂令通直郎宋維,告司染都尉韓文殊欲謀逆立懌,懌坐禁止。後窮案無實,懌雖得免,猶以兵衛守於宮西別館。久之,叉恐懌終為己害,乃與侍中劉騰密謀,詐取主食中黃門胡度、胡定列,誣懌云:「貨度等金帛,令以毒藥置御食中以害帝。」騰以具奏。明帝信之,乃御顯陽殿。騰閉永巷門,靈太后不得出。懌入,遇叉於含章殿後,命宗士及直齋執懌衣袂,將入含章東省。騰稱詔集公卿議,以大逆論。咸畏叉,無敢異者。唯僕射游肇執意不同。叉、騰持公卿議入奏,夜中殺懌。
 於是假為靈太后辭遜詔,叉遂與太師、高陽王雍等輔政。常直禁中,明帝呼為姨父。自後百寮重跡。後帝徙御徽音殿,叉亦入居殿右,曲盡佞媚,遂出入禁中,恒令勇士持刀劍以自先後。叉於千秋門外廠下施木闌檻,有時出入,止息其中,腹心防守,以備竊發。



 初,叉之專政,矯情自飾,勞謙待士。得志之後,便自驕愎,耽酒好色,與奪任情。乃於禁中自作別庫掌握之,珍寶充牣其中。叉曾臥婦人於食輿,以巴覆之。



 輿入禁內,出亦如之,直衛雖知,莫敢言者。姑姊婦女,朋淫無別。政事怠墮,綱紀不舉。州鎮多非其人,於是天下遂亂矣。叉自知不法,恐被廢黜,
 乃陰遣弟洪業召武州人姬庫根等與之聚宴。遂為誓盟,欲令為亂,朝廷必以己為大將軍往伐,因以共為表裏,如此可得自立。根等然其言,乃厚遺根等,遣還州,與洪業買馬。



 從劉騰死後,防衛微緩。叉頗亦自寬,時宿於外,每日出遊,留連他邑。靈太后微察知之。正光五年秋,靈太后對明帝謂群臣,求出家於嵩山閑居寺,欲自下髮。



 帝與群臣大懼,叩頭泣涕。遂與太后密謀圖之。乃對叉流涕,敘太后欲出家憂怖之心。叉乃勸帝從太后意。於是太后數御顯陽,二宮無復禁礙。舉其親元法僧為徐州刺史,法僧據州反叛。靈太后數以為言,叉深愧悔。
 丞相、高陽王雍雖位重於叉,而甚畏憚。會太后與帝遊洛水,遂幸雍第,定圖叉之計。後雍從帝朝太后,乃進言叉父子權重。太后曰:「然。元郎若忠於朝廷,何故不去領軍,以餘官輔政?」叉聞之甚懼,免冠求解。乃以叉為儀同三司、尚書令、侍中、領左右。



 叉雖去兵權,然總任內外,不慮黜廢。又有閹人張景嵩、劉思逸、屯弘昶、伏景謀廢叉。嵩以帝嬪潘外憐有幸,說云,元叉欲害之。嬪泣訴於帝云:「叉非直欲殺妾,亦將害陛下。」帝信之。後叉出宿,遂解其侍中。旦欲入宮,門者不納。尋除名。



 初,咸陽王禧以逆見誅,其子樹,梁封為鄴王。及法僧反叛後,樹遺公卿百
 寮書,暴叉過惡,言:「叉本名夜叉,弟羅實名羅剎。夜叉、羅剎,此鬼食人,非遇黑風,事同飄墮。鳴呼魏境!離此二災。惡木盜泉,不息不飲,勝名梟稱,不入不為。況昆季此名,表能噬物,日露久矣,始信斯言。」叉為遠近所惡如此。



 其後靈太后顧謂侍臣曰:「劉騰、元叉昔邀朕索鐵券,望得不死,朕賴不與。」



 中書舍人韓子順對曰:「臣聞殺活,豈計與否。陛下昔雖不與,何解今日不殺?」



 靈太后憮然。未幾,有人告叉及其弟爪謀反。先遣其從弟洪業率六鎮降戶反定州;叉令勾魯陽諸蠻侵擾伊闕,叉兄弟為內應,起有日矣,得其手書。靈太后以妹婿故,未忍便決。群臣
 固執不已,明帝又以為言,太后乃從之。於是叉及弟爪並賜死於家。



 太后猶以妹故,復追贈尚書令、冀州刺史。叉子舒,祕書郎。叉死後,亡奔梁,官至征北大將軍、青冀二州刺史。



 子善,亦名善住。少隨父至江南,性好學,通涉《五經》,尤明《左氏傳》。



 侯景之亂,善歸周,武帝甚禮之,以為太子宮尹,賜爵江陽縣公,每執經以授太子。



 隋開皇初,拜內史侍郎,凡有敷奏,詞氣抑揚,觀者屬目。陳使袁雅來聘,上令善就館受書。雅出門不拜。善論舊事有拜之儀,雅未能對。遂拜,成禮而去。後遷國子祭酒。上嘗親臨釋奠,令善
 講《孝經》,於是敷陳義理,兼之以諫。上大悅曰:「聞江陽之說,更起朕心。」齎絹一百匹,衣一襲。善之通博,在何妥之下,然以風流醖藉,俯仰可觀,音韻清朗,由是為後進所歸。妥每懷不平,心欲屈善,因講《春秋》。初發題,諸儒畢集,善私謂妥曰:「名望已定,幸無相苦。」妥然之。及就講肆,妥遂引古今滯義以難善,多不能對。二人由是有隙。



 善以高熲有宰相之具,嘗言於上曰:「楊素粗疏,蘇威怯懦,元胄、元旻,正似鴨耳。可以付社稷者,唯獨高熲。」上初然之。及熲得罪,上以善言為熲游說,深責望之。善憂懼,先患消渴,於是病頓而卒。



 叉弟羅,字仲綱。雖父兄貴盛,而虛
 己接物。累遷青州刺史。叉當朝專政,羅望傾四海,於時才名之士王元景、邢子才、季獎等咸為其賓客,從遊青土。罷州,入為守正卿。叉死後,羅通叉妻,時人穢之,或云其救命之計也。孝武時,位尚書令、開府儀同三司、梁州刺史。孝靜初,梁遣將圍逼,羅以州降,封南郡王。及侯景自立,以羅為開府儀同三司、尚書令,改封江陽王。梁元帝滅景,周文帝求羅,遂得還。除開府儀同三司、侍中、少師,襲爵江陽王。舒子善住,在後從南入關,羅乃以爵還善住,改封羅為固道郡公。



 羅弟爽,字景哲。少而機警,位給事黃門侍郎、金紫光祿大夫。卒,謚曰懿。



 爽弟蠻,仕齊,
 歷位兼度支尚書,行潁州事。坐不為繼母服,為左丞所彈。後除開府儀同三司。齊天保十年,大誅元氏。昭帝元后,蠻之女也,為苦請,自市追免之,賜姓步六孤氏。卒,贈司空。蠻弟爪,字景邕,位給事中,與兄叉同時誅。



 繼弟羅侯,遷洛之際,以墳陵在北,遂家於燕州之昌平郡。內豐資產,唯以意得為適。不入京師,在賓客往來者,必厚相禮遺,豪據北方,甚有聲稱。以叉執權,尤不樂入仕,就拜昌平太守。



 明元皇帝七男:杜密皇后生太武皇帝;大慕容夫人生樂平戾王丕;安定殤王彌闕母氏;慕容夫人生樂安宣
 王範;尹夫人生永昌莊王健;建寧王崇、新興王俊二王並闕母氏。



 樂平王丕,少有才幹。泰常七年封,拜車騎大將軍。後督河西、高平諸軍討南秦王楊難當。軍至略陽,禁令齊肅,所過無私,百姓爭致牛酒。難當懼,還仇池。



 而諸將議曰:「若不誅豪帥,軍還之後,必聚而為寇。」又以大眾遠出,不有所掠,則無以充軍實,賞將士。將從之,時中書侍郎高元參丕軍事,諫曰:「今若誅之,是傷其向化之心,恐大軍一還,為亂必速。」丕以為然,於是綏懷初附,秋豪無犯。



 初,馮弘之奔高麗,太武詔遣送之,高麗不遣。太武怒,將討
 之。丕上疏以為和龍新定,宜復之,使廣脩農殖,以饒軍實,然後進圖,可一舉而滅。帝納之,乃止。後坐劉潔事,以憂薨,事在《潔傳》,謚曰戾王。子拔襲爵。後坐事賜死,國除。



 丕之薨及日者董道秀之死也,高元遂著《筮論》曰:「昔明元末,起白臺,其高二十餘丈。樂平王嘗夢登其上,四望無所見。王以問日者董道秀。筮之,曰:「大吉」。王默而有喜色。後事發,王遂憂死,而道秀棄市。道秀若推六爻以對王曰:「易稱亢龍有悔。窮高日亢,高而無人,不為善也。」夫如是,則上寧於王,下保於己,福祿方至,豈有禍哉?今舍於本而從其末,咎釁之至,不亦宜乎!」



 安定王彌,泰常七年封。薨,謚曰殤王。無子,國除。



 樂安王範,泰常七年封。雅性沉厚。太武以長安形勝之地,乃拜範為衛大將軍、開府義同三司、長安鎮都大將。範謙恭惠下,推心撫納,百姓稱之。時秦土新離寇賊,流亡者相繼,請崇易簡之禮,帝納之。於是遂寬徭,與人休息。後劉潔之謀,範聞而不告。事發,因疾暴薨長子良,太武未有子,嘗曰:「兄弟之子猶子。」親撫養之。長而壯勇多知,嘗參軍國大計。文成時,襲王,拜長安鎮都大將、雍州刺史,為內都大官。薨,謚曰簡王。



 永昌王健,泰常七年封。健姿貌魁壯,所在征戰,常有大
 功。才藝比陳留桓王而智略過之。從太武破赫連昌,遂西略至木根山。討和龍,健別攻拔建德。後平叛胡白龍餘黨于西海。太武襲蠕蠕,越涿邪山,詔健殿後。矢不虛發,所中皆應弦而斃,威震漠北。尋從平涼州,健功居多。又討破禿髮保周,自殺,傳首京師。復降沮渠無諱。薨,謚曰莊王。子仁襲。仁亦驍勇有父風,太武奇之。後與濮陽王閭若文謀為不軌,發覺,賜死,國除。



 建寧王崇,泰常七年封。文成時,封崇子麗濟南王。後與京兆王杜元寶謀逆,父子並賜死。



 新興王俊,泰常七年封。少善騎射,多藝。坐法,削爵為公。
 俊好酒色,多越法度。又以母先遇罪死,而己被貶削,恒懷怨望,頗有悖心。後事發,賜死,國除。



 太武皇帝十一男:賀皇后生景穆帝;越椒房生晉王伏羅;舒椒房生東平王翰;弗椒房生臨淮王譚;伏椒房生廣陽王建;閭左昭儀生吳王餘;其小兒、貓兒、真、彪頭、龍頭並闕母氏,皆早薨,無傳。



 晉王伏羅,真君三年封,加車騎大將軍。後督高平、涼州諸軍討吐谷渾慕利延。



 軍至樂都,謂諸將曰:「若從正道,恐軍聲先振,必當遠遁。潛軍出其非意,此鄧艾禽蜀之計也。」諸將咸難之。伏羅曰:「夫將軍制勝,萬里擇利,專之
 可也。」



 遂間道行。至大母橋,慕利延眾驚,奔白蘭。慕利延兄子拾寅走河西,降其一萬餘落。八年,薨,無子,國除。



 東平王翰,真君三年封秦王,拜侍中、中軍大將軍,參典都曹事。忠貞雅正,百僚憚之。太傅高元以翰年少,作《諸侯箴》以遺之,翰覽之大悅。後鎮枹罕,羌戎敬服。改封東平王。太武崩,諸大臣等議欲立翰,而中常侍宗愛與翰不協,矯太后令立南安王餘,遂殺翰。子道符襲爵,拜長安鎮都大將。皇興元年,謀反,司馬段太陽斬之,傅首京師。



 臨淮王譚,真君三年封燕王,拜侍中,參都曹事。後改封
 臨淮王。薨,謚宣王。



 子提襲,為梁州刺史。以貪縱削除,加罰,徙配北鎮。久之,提子員外郎穎免冠請解所居官,代父邊戍,孝文不許。後昭提從駕南伐。至洛陽,參定遷都之議。



 尋卒,以預參遷都功,追封長鄉縣侯。宣武時,贈雍州刺史,謚曰「懿」。



 提子昌,字法顯。好文學。居父母喪,哀號孺慕,非感行人。宣武時,復封臨淮王,未拜而薨。贈齊州刺史,謚曰康王,追改封濟南王。



 子彧,字文若,紹封。彧少有才學,當時甚美。侍中崔光見而謂人曰:「黑頭三公,當此人也。」少與從兄安豐王延明、中山王熙,並以宗室博古文學齊名,時人莫能定其優劣。尚書郎范陽盧道
 將謂吏部清河崔休曰:「三人才學雖並優美,然安豐少於造次,中山皂白太多,未若濟南風流寬雅。」時人為之語曰:「三王楚琳瑯,未若濟南備員方。」彧姿制閑裕,吐發流美。瑯邪王誦,有名人也,見之未嘗不心醉忘疲。奏鄭廟歌詞,時稱其美。除給事黃門侍郎。



 彧本名亮,字仕明,時侍中穆紹與彧同署,避紹父諱,啟求改名。詔曰:「仕明風神運吐,常自以比荀文若,可名彧,以取定體相倫之美。」彧求復本封,詔許復封臨淮,寄食相州魏郡。又長兼御史中尉。彧以為倫敘得之,不謝。領軍于忠忿,言之朝廷曰:「臨淮雖復風流可觀,而無骨鯁之操,中尉之任,恐非
 所堪。」遂去威儀,單車而還,朝流為之歎息。累遷侍中、衛將軍、左光祿大夫,兼尚書左僕射,攝選。後以本官為東道行臺。會爾朱榮入洛,殺害元氏,彧撫膺慟哭,遂奔梁。梁武遣其舍人陳建孫迎接,並觀彧為人。建孫稱彧風神閑俊。梁武亦先聞名,深相器待。見彧於樂遊園,因設宴樂。彧聞聲歔欷,涕淚交下,梁武為之不樂。自前後奔叛,皆候旨稱魏為偽,唯彧表啟常云魏臨淮王。梁武體彧雅性,不以為責。及知莊帝踐阼,彧以母老請還,辭旨懇切。梁武惜其人才,又難違其意,遣其僕射徐勉私勸彧留。彧曰:「死猶願北,況於生也?」梁武乃以禮遣。彧性至
 孝。自經違離,不進酒肉;憔悴容貌,見者傷之。歷位尚書令、大司馬,兼錄尚書。



 莊帝追崇武宣王為文穆皇帝,廟號肅祖,母李妃為文穆皇后。將遷神主於太廟,以孝文為伯考。彧表諫,以為:「漢祖創業,香街有太上之廟;光武中興,南頓立春陵之寢。元帝之於光武,疏為絕服,猶尚身奉子道,入繼大宗。高祖之於聖躬,親實猶子,陛下既纂洪緒,豈宜加伯考之名?且漢宣之繼孝昭,斯乃上後叔祖,豈忘宗承考妣?蓋以大義斯奪。及金德將興,宣王受寄,景王意在毀冕,文王心規裂冠。雖祭則魏主,而權歸晉室。昆之與季,實傾曹氏。且子元宣王冢胤,文王成
 其大業。故晉武繼文祖武,宣有伯考之稱。以今類古,恐或非儔。高祖德溢寰中,道超無外。肅祖雖勳格宇宙,猶曾奉贄稱臣。穆后稟德坤元,復將配享乾位。此乃君臣並筵,嫂叔同室,歷觀墳籍,未有其事。」時莊帝意銳,朝臣無敢言者,唯彧與吏部尚書李神俊並有表聞。詔報曰:「文穆皇帝勳格四表,道邁百王,是用考循舊範,恭上尊號。王表云漢太上於香街,南頓於春陵。漢高不因瓜瓞之緒,光武又無世及之德,皆身受符命,不由父祖。別廟異寢,於理何差?文穆皇帝天眷人宅,曆數有歸。朕忝承下武,遂主神器。既帝業有統,漢氏非倫。若以昔況今,不
 當移寢。



 則魏太祖、晉景帝雖王跡已顯,皆以人臣而終,豈得與餘帝別廟,有闕餘序?漢郡國立廟者,欲尊高祖之德,使饗遍天下」非關太廟神主,獨在外祠薦。漢宣之父,亦非勳德所出,雖不追尊,不亦可乎?伯考之名,自是尊卑之稱,何必準古而言非類也。復云君臣同列,嫂叔共室。當以文穆皇帝昔遂臣道,以此為疑。《禮》『天子元子猶士』,禘祫豈不得同室乎?且晉文、景共為一代,議者云世限七,主無定數。昭穆既同,明有共室之理。禮既有祔,嫂叔何嫌?《禮》,大祖、禰一廟,豈無婦舅共室也?若專以共室為疑,容可更議遷毀。」莊帝既逼諸妹之請,此詞意黃
 門侍郎常景、中書侍郎邢子才所贊成也。又追尊兄彭城王為孝宣帝。彧又面諫曰:「陛下作而不法,後世何觀?歷尋書籍,未有其事。」帝不從。及神主入廟。復敕百官悉陪從,一依乘輿之式。彧上表以為:「爰自中古,迄於下葉,崇尚君親,褒明功懿,乃有皇號,終無帝名。今若去帝,直留皇名,求之古義,少有依準。」又不納。



 爾朱榮死,除彧司徒公。及爾朱兆率眾奄至,出東掖門,為賊所獲。見兆,辭色不屈,為群胡所毆,薨。孝武帝末,贈大將軍、太師、太尉公、錄尚書事,謚曰文穆。彧美風韻,善進止,衣冠之下,雅有容則。博覽群書,不為章句,所制文藻,雖多亡失,猶有
 傳於世者。然居官不能清白,所進舉止於親婭,為識者所譏。無子。



 弟孝友,少有時譽,襲爵臨淮王,累遷滄州刺史。為政溫和,好行小惠,不能清白,而無所侵犯,百姓亦以此便之。魏靜帝宴齊文襄於華林園,孝友因醉自譽,又云:「陛下許賜臣能。」帝笑曰:「朕恒聞王自道清。」文襄曰:「臨淮王雅旨舍罪。」於是君臣俱笑而不罪。孝友明於政理,嘗奏表曰:令制百家為黨族,二十家為閭,五家為比鄰。百家之內,有帥二十五,徵發皆免,苦樂不均。羊少狼多,復有蠶食。此之為弊久矣。京邑諸坊,或七八百家,唯一里正、二史,庶事無闕,而況外州乎?請依舊置,三正之
 名不改,而百家為於,四閭,閭二比,計族少十二丁,得十二匹貲絹。略計見管之戶,應二萬餘族,一歲出貲絹二十四萬匹。十五丁出一番兵,計得一萬六千兵。此富國安人之道也。



 古諸侯娶九女,士有一妻二妾。《晉令》:諸王置妾八人;郡君、侯,妾六人。



 《官品令》:第一、第二品有四妾;第三、第四有三妾;第五、第六有二妾;第七、第八有一妾。所以陰教聿脩,繼祠有廣。廣繼嗣,孝也。脩陰教,禮也。而聖朝忽棄此數,由來漸久,將相多尚公主,王侯娶后族,故無妾媵,習以為常。婦人多幸,生逢今世,舉朝略是無妾,天下殆皆一妻。設令人強志廣娶,則家道離索,身事
 迍邅,內外親知共相嗤怪。凡今之人,通無準節。父母嫁女,則教之以妒;姑姊逢迎,必相勸以忌。持制夫為婦德,以能妒為女工。自云受人欺,畏他笑我。王公猶自一心,以下何敢二意!夫妒忌之心生,則妻妾之禮廢;妻妾之禮廢,則姦淫之兆興,斯臣之所以毒恨者也。請以王、公、第一品娶八,通妻以備九女;稱事二品備七;三品、四品備五;五品、六品則一妻二妾。限以一周,悉令充數。若不充數,及待妾非禮,使妻妒加捶撻,免所居官。其妻無子而不娶妾,斯則自絕,無以血食祖父,請科不孝之罪,離遣其妻。



 臣之赤心,義唯家國,欲使吉凶無不合禮,貴賤
 各有其宜。省人帥以出兵丁,立倉儲以豐穀食。設賞格以禽姦盜,行典令以示朝章。庶使足食足兵,人信之矣。



 又冒申妻妾之數,正欲使王侯將相,功臣子弟,苗胤滿朝,傳祚無窮,此臣之志也。



 詔付有司,議奏不同。



 孝友又言:「今人生為皁隸,葬擬王侯,存沒異途,無復節制。崇壯丘隴,盛飾祭儀,鄰里相榮,稱為至孝。又夫婦之始,王化所先,共食合瓢,足以成禮。而今之富者彌奢,同牢之設,甚於祭槃。累魚成山,山有林木,林木之上,鸞鳳斯存。



 徒有煩勞,終成委棄,仰惟天意,其或不然。請自茲以後,若婚葬過禮者,以違旨論。官司不加糾劾,即與同罪。



 孝友
 在尹積年,以法自守,甚著聲稱。然性無骨鯁,善事權勢,為正直者所譏。



 齊天保初,準例降爵,封臨淮縣公,拜光祿大夫。二年冬,被詔入晉陽宮,出與元暉業同被害。



 昌弟孚,字秀和,少有令譽。侍中游肇、並州刺史高聰、司徒崔光等見孚,咸曰:「此子當準的人物,恨吾徒衰暮,不及見耳。」累遷兼尚書右丞。靈太后臨朝,宦者干政,孚乃總括古今名妃賢后,凡為四卷,奏之。遷左丞。



 蠕蠕主阿那瑰既得反國,其人大飢,相率入塞,阿那瑰上表請臺振給。詔孚為北道行臺,詣彼振恤,孚陳便宜表曰:皮服之人,未嘗粒食,宜從俗因利,拯其所無。昔漢建武中,單于
 款塞,時轉河東米Я二萬五千斛、牛羊三萬六千頭以給之。斯則前代和戎,撫新柔遠之長策也。



 乞以牸牛產羊,糊其口食。且畜牧繁息,是其所便;毛血之利,惠兼衣食。



 又尚書奏云:如其仍住七州,隨寬置之。臣謂人情戀本,寧肯徙內?若依臣請,給振雜畜,愛本重鄉,必還舊土。如其不然,禁留益損。假令逼徙,事非久計。何者?人面獸心,去留難測。既易水草,痾恙將多;憂愁致困,死亡必甚。兼其餘類,尚在沙磧;脫出狂勃,翻歸舊巢,必殘掠邑里,遺毒百姓。亂而方塞,未若杜其未萌。又貿遷起於上古,交易行於中世。漢與胡通,亦立關市。今北人阻饑,命懸
 溝壑;公給之外,必求市易。彼若願求,宜見聽許。



 又云:營大者不計小名,圖遠者弗拘近利。雖戎狄衰盛,歷代不同,叛服之情,略可論討。周之北伐,僅獲中規;漢氏外攘,裁收下策。昔在代京,恆為重備,將帥勞止,甲士疲力。計前世苦之,力未能致。今天祚大魏,亂亡在彼。朝廷垂天覆之恩,廓大造之德,鳩其散亡,禮送令反,宜因此時,善思遠策。竊以理雖萬變,可以一觀;來事雖懸,易以往卜。昔漢宣之世,呼韓款塞,漢遣董忠、韓昌領邊郡士馬,送出朔方,因留衛助。又光武時,亦令中郎將段彬置安集掾史,隨單于所在,參察動靜。斯皆守吉之元龜,安邊之
 勝策。計今朝廷成功,不減曩時,蠕蠕國弊,亦同疇日。宜準昔成謀,略依舊事,借其所閑地,聽使田牧。粗置官屬,示相慰撫。嚴戒邊兵,以見保衛。馭以仁寬,縻以久策。使親不至矯詐,疏不容叛反。今北鎮諸將,舊常云一人代外邏,因令防察。所謂天子有道,守在四夷者也。



 又云:先人有奪人之心,待降如受彊敵。武非尋外,亦以防內。若從處分割配,諸州鎮遼遠,非轉輸可到,悔叛之情,變起難測。又居人畜業,布在原野,戎夷性貪,見則思盜;防彼肅此,少兵不堪。渾流之際,易相干犯。驅之還本,未必樂去,配州內徙,復不肯從。既其如此,為費必大。



 朝廷不許。



 孚持白武幡勞阿那瑰於柔玄、懷荒二鎮間。阿那環瑰眾號三十萬,陰有異意,遂拘留孚。載以韞車,日給酪一升、肉一段。每集其眾,坐孚車廂,稱為行臺,甚加禮敬。阿那瑰遂南過,至舊京。後遣孚等還,因上表謝罪。有司以孚事下廷尉,丞高謙之云孚辱命,處孚流罪。



 後拜冀州刺史。孚勸課農桑,境內稱慈父,鄰州號曰神君。先是,州人張孟都、張洪建、馬潘、崔獨憐、張叔緒、崔醜、張天宜、崔思哲等八人,皆屯保林野,不臣王命,州郡號曰八王。孚至,皆請入城,願致死效力。後為葛榮所陷,為榮所執。



 兄祐為防城都督,兄子禮為錄事參軍。榮欲先害子禮,孚請
 先死以贖子禮,叩頭流血,榮乃捨之。又大集將士,議其死事。孚兄弟各誣己引過,爭相為死。又孟都、潘紹等數百人皆叩頭就法,請活使君。榮曰:「此魏之誠臣義士也。」凡同禁五百人,皆得免。榮卒,還除冀州刺史。元顥入洛,授孚東道行臺、彭城郡王。孚封顥逆書送朝廷,天子嘉之。顥卒,封孚萬年鄉男。



 永安末,樂器殘缺,莊帝命孚監儀注。孚上表曰:昔太和中,中書監高閭、太樂令公孫崇脩造金石,數十年間,乃奏成功。時大集儒生,考其得失。太常卿劉芳請別營造,久而方就。復召公卿量校合否,論者沸騰,莫有適從。登被旨敕,並見施用。往歲大軍入洛,
 戎馬交馳,所有樂器,亡失垂盡。臣至太樂署,問太樂令張乾龜等,云承前以來,置宮懸四箱,栒虡六架,東北架編黃鐘之磬十四。雖器名黃鐘,而聲實夷則;考之音制,不甚諧韻。姑洗懸於東北,太蔟編於西北,蕤賓列於西南。並皆器象差位,調律不和。又有儀鐘十四,虡懸架首,初不叩擊,今便刪廢,以從正則。臣今據《周禮鳧氏》脩廣之規,《磬氏》倨句之法,吹律求聲,叩鐘求音;損除繁雜,討論實錄。依十二月為十二宮。



 各準辰次,當位懸設。月聲既備,隨用擊奏。則會還相為宮之義,又得律呂相生之體。今量鐘磬之數,各以十二架為定。



 奏可。于時搢紳之
 士,咸往觀聽,靡不咨嗟歎服而反。太傅、錄尚書長孫承業妙解聲律,特復稱善。



 復從孝武帝入關,除尚書左僕射、扶風郡王。尋監國史。歷位司空、兼尚書令、太保。時蠕蠕主與孚相識,先請見孚,然後遣女。於是乃使孚行。蠕蠕君臣見孚,莫不歡悅,奉皇后來歸。



 孚性機辯,好酒,貌短而禿。周文帝偏所眷顧,嘗於室內置酒十瓨,瓨餘一斛,上皆加帽,欲戲孚。孚適入室,見即驚喜,曰:「吾兄弟輩甚無禮,何為竊入王家,匡坐相對?宜早還宅也。」因持酒歸。周文撫手大笑。後遇風患,手足不隨,口不能言,乃左手畫地作字,乞解所任。三奏不許。遷太傅。薨。帝親臨,百
 官赴弔。



 贈大司馬、錄尚書事,謚曰文簡。



 子端嗣,位大行臺尚書、華州刺史。性疏佷,頗以基地驕物,時論鄙之。



 廣陽王建,真君三年封楚王,後改封廣陽。薨,謚曰簡王。子石侯襲,薨,謚曰哀王。子遺興襲,薨,謚曰定王。無子。



 石侯弟嘉,少沉敏,喜慍不形於色,兼有武略。孝文初,拜徐州刺史,甚有威惠。後封廣陽王,以紹建後。孝文南伐,詔嘉斷均口。嘉違失指授,令賊得免。帝怒責之曰:「叔祖定非世孫,何太不上類也!」及將大漸,遺詔以嘉為尚書左僕射,與咸陽王禧等輔政。遷司州牧。嘉表請於京四面築坊三百二十,各周一千二百步,乞發三正復丁,以充
 茲役。雖有暫勞,奸盜永止。詔從之。拜衛大將軍、尚書令,除儀同三司。



 嘉好飲酒,或沉醉,在宣武前言笑自得,無所顧忌。帝尊年老,常優容之。與彭城、北海、高陽諸王,每入宴集,極懽彌夜,數加賞賜。帝亦時幸其第。性好儀飾,車服鮮華。既居儀同,又任端首,出入容衛,道路榮之。後遷司空,轉司徒。



 嘉好立功名,有益公私,多所敷奏,帝雅委付之。愛敬人物,後來才俊未為時知者,侍坐之次,轉加談引,時人以此稱之。薨,遺命薄葬。宣武悼惜之,贈侍中、太保,謚曰懿烈。



 嘉後妃宜都王穆壽孫女,司空從妹也。聰明婦人。及為嘉妃,多所匡贊,光益家道。



 子深,字知
 遠,襲爵。孝明初,拜肆州刺史。預行恩信,胡人便之,劫盜止息。



 後為恒州刺史,在州多所受納,政以賄成。私家有馬千匹者,必取百匹,以此為恒。



 累遷殿中尚書,未拜。坐淫城陽王徽妃于氏,為徽表訟。詔付丞相、高陽王雍等宗室議決其罪,以王還第。



 及沃野鎮人破六韓拔陵反叛,臨淮王彧討之失利,詔深為北道大都督,受尚書令李崇節度。時東道都督崔暹敗於白道,深等諸軍退還朔州。深上書曰:邊豎構逆,以成紛梗,其所由來,非一朝也。昔皇始以移防為重,盛簡親賢,擁麾作鎮,配以高門子弟,以死防遏。不但不廢仕宦,至乃偏得復除。當時人
 物,忻慕為之。及太和在歷。僕射李沖當官任事,涼州土人,悉免廝役;豐沛舊門,仍防邊戍。自非得罪當世,莫肯與之為伍。征鎮驅使為虞候、白直,一生推遷,不過軍主。然其往世房分,留居京者,得上品通官;在鎮者,便為清途所隔。或投彼有北,以御魑魅,多復逃胡鄉。乃峻邊兵之格,鎮人浮遊在外,皆聽流兵捉之。於是少年不得從師,長者不得遊宦。獨為匪人,言者流涕。



 自定鼎伊洛,邊任益輕,唯底滯凡才,出為鎮將。轉相模習,專事聚斂。或有諸方奸吏,犯罪配邊,為之指蹤,過弄官府;政以賄立,莫能自改。咸言姦吏為此,無不切齒增怒。及阿那瑰背
 恩,縱掠竊奔,命師追之。十五萬眾度沙漠,不日而還。



 邊人見此援師,便自意輕中國。尚書令臣崇時即申聞,求改鎮為州,將允其願,抑亦先覺。朝廷未許。而高闕戍主,率下失和,拔陵殺之為逆命;攻城掠地,所見必誅。王師屢北,賊黨日盛。此段之舉,指望銷平。其崔暹雙輪不反,臣崇與臣,逡巡復路。今者相與,還次雲中。馬首是瞻,未便西邁。將士之情,莫不解體。今日所慮,非止西北,將恐諸鎮尋亦如此。天下之事何易可量!



 時不納其策。東西部敕勒之叛,朝議更思深言。遣兼黃門侍郎酈道元為大使,欲復鎮為州,以順人望。會六鎮盡叛,不得施行。深
 後上言:「今六鎮俱叛,二部高車亦同惡黨,以疲兵討之,必不制敵。請簡選兵,或留守恒州要處。更為後圖。」



 及李崇征還,深專總戎政。拔陵避蠕蠕,南移度河。先是,別將李叔仁以拔陵來逼,請求迎援,深赴之,前後降附二十萬人。深與行臺元纂表求恒州北別立郡縣,安置降戶,隨宜振賚,息其亂心。不從。詔遣黃門侍郎楊置分散之於冀、定、瀛三州就食。深謂纂曰:「此輩復為乞活矣。禍亂當由此作。」



 既而鮮于脩禮叛於定州,杜洛周反於幽州,其餘降戶,猶在恒州,遂欲推深為主。深乃上書乞還京師。令左衛將軍楊津代深為都督,以深為侍中、右衛將軍、
 定州刺史。時中山太守趙叔隆、別駕崔融討賊失利,臺使劉審考核未訖,會賊逼中山,深乃令叔隆防境。審馳驛還京,云深擅相放縱。城陽王徽與深有隙,因此構之。乃徵深為吏部尚書、兼中領軍。及深至都,明帝不欲使徽、深相憾,敕因宴會,令相和解。徽銜不已。



 後河間王琛等為鮮于脩禮所敗,乃除深儀同三司、大都督。章武王融為左都督,裴衍為右都督,並受深節度。徽因奏靈太后構深曰:「廣陽以愛子握兵在外,不可測也。」乃敕章武王等潛相防備。融遂以敕示深。深懼,事無大小,不敢自決。靈太后聞之,乃使問深意狀,乃具言曰:往者元叉執權,
 移天徙日,而徽託附,無翼而飛。今大明反政,任寄唯重,以徽褊心,銜臣次骨。臣以疏滯,遠離京輦,被其構阻,無所不為。然臣昔不在其後,自此以來,翻成陵谷。徽遂一歲八遷,位居宰相;臣乃積年淹滯,有功不錄。



 自徽執政以來,非但抑臣而已,北征之勳,皆被擁塞。將士告捷,終無片賞;雖為表請,多不蒙遂。前留元摽據乎盛樂,後被重圍,析骸易子,倒懸一隅。嬰城二載,賊散之後,依階乞官,徽乃盤退,不允所請。而徐州下邳戍主賈勳,法僧叛後,暫被圍逼,固守之勛,比之未重,乃立得州,即授開國。天下之事,其流一也,功同賞異,不平謂何!又驃騎李崇
 北征之日,啟募八州之人,聽用關西之格。及臣在後,依此科賞。復言北道征者,不得同於關西。定襄陵廟之至重,平城守國之要鎮,若計此而論功,亦何負於秦楚?但以嫉臣之故,便欲望風排抑。



 然其當途以來,何直退勳而已。但是隨臣征者,即便為所嫉。統軍袁叔和曾經省訴,徽初言有理,又聞北征隸臣為統,應時變色。復令臣兄子仲顯異端訟臣,緝緝翩翩,謀相誹謗。言臣惡者,接以恩顏;稱臣善者,即被嫌責。甄琛曾理臣屈,乃視之若仇讎;徐紇頗言臣短,即待之如親戚。又驃騎長史祖瑩,昔在軍中,妄增首級;矯亂戎行,蠹害軍府,獲罪有司,避
 命山澤。直以謗臣之故,徽乃還雪其罪。



 臣府司馬劉敬,比送降人,既到定州,翻然背叛,賊如決河,豈其能擁。且以臣府參僚,不免身首異處。徽既怒遷,捨其元惡,及胥徒。從臣行者,莫不悚懼。頃恒州之人,乞臣為刺史,徽乃斐然言不可測。及降戶結謀,臣頻表啟,徽乃因執言此事。及向定州,遠彼姦惡,又復論臣將有異志。翻覆如此,欲相陷沒。致令國朝,遽賜遷代。賊起之由,誰使然也?



 徽既優幸,任隆一世,慕勢之徒,於臣何有!是故餘人攝選,車馬填門。及臣居邊,賓遊罕至。臣近比為慮其為梗,是以孜孜乞赴京闕。屬流人舉斧,元戎垂翅,復從後命,自
 安無所;僶俛先驅,不敢辭事。及臣出都,行塵未滅,已聞在後,復生異議。言臣將兒自隨,證為可疑之兆。忽稱此以構亂。悠悠之人,復傳音響,言左軍臣融、右軍臣衍皆受密敕,伺察臣事。徽既用心如此,臣將何以自安?竊以天步未夷,國難猶梗;方伯之任,於斯為急。徽昔臨籓,乃有人譽,及居端右,無聞焉爾。今求出之為州,使得申其利用。徽若外從所長,臣無內慮之切。脫蒙,公私幸甚。



 深以兵士頻經退散,人無鬥情,連營轉柵,日行十里。行達交津,隔水而陣。



 賊脩禮常與葛榮謀,後稍信朔州人毛普賢,榮常銜之。普賢昔為深統軍,及在交津,深傳入諭
 之,普賢乃有降意。又使錄事參軍元晏說賊程殺鬼。果相猜貳。葛榮遂殺普賢、脩禮而自立。榮以新得大眾,上下未安,遂北度瀛州。深便率眾北轉。榮東攻章武王融,融戰敗於白牛還。深遂退走,趣定州。聞刺史楊津疑其有異志,乃止於州南佛寺。停二日夜,乃召都督毛謚等六七人,臂肩為約,危難之際,期相拯恤。



 謚疑深意異,乃密告津,云深謀不軌。津遣謚討深。深走出,謚叫噪追躡。深與左右行至博陵郡界,逢賊遊騎,乃引詣葛榮。賊徒見深,頗有喜者。榮新自立,內惡之,乃害深。莊帝追復王爵,贈司徒公,謚曰忠武。



 子湛,字士淵,少有風尚。孝莊初,襲
 封。孝靜初,累遷冀州刺史。所在聚斂,風政不立。入為侍中,後行司州牧。時齊神武作相,以湛頗有器望,啟超拜太尉公。



 薨,贈假黃鉞、大司馬、尚書令,謚曰文獻。初,湛名位漸重,留連聲色,始以婢紫光遺尚書郎中宋遊道,後乃私耽,出為冀州,竊而攜去。遊道大致紛紜,乃云紫光湛父所寵,湛母遺己。將致公文,久乃停息。論者兩非之。



 湛弟瑾,尚書祠部郎。後謀殺齊文襄。事泄,合門伏法。



 湛子法輪,紫光所生也。齊王矜湛覆滅,乃啟原之,復其爵土。



 南安王餘,真君三年封吳王,後改封南安王。太武暴崩,
 中常侍宗愛矯皇太后令迎立之,然後發喪。大赦,改年為永平。餘自以非次而立,厚遺群下,取悅於眾。



 為長夜之飲,聲樂不絕。旬月之間,帑藏空罄。尤好弋獵,出入無度。邊方告難,餘不恤之,百姓憤惋,而餘晏如也。宗愛權恣日甚,內外憚之。餘疑愛變,謀奪其權。愛因餘祭廟,夜殺餘。文成葬以王禮,謚曰隱。



 論曰:梟獍為物,天實生之。觀夫元紹所懷,蓋亦特鐘沴氣。平陽以降,並多夭促;英才武略,未顯高年。靖、簡二王,為時稱首。鑒既有聲,渾亦見器。霄、繼荷遇太和之日,名位豈妄及哉!叉階緣寵私,遂亂天下,殺身全祀,固為幸
 焉。



 樂平、樂安俱以將領自效,竟以憂迫而逝,驗克終之為鮮。莊王才力智謀,一時之傑,與夫建寧、新興,不同日也。



 太武之子,秦、晉才賢。而翰之遇酷,倚伏豈可量矣。臨淮之後,彧為盛德;廣陽之世,嘉實為美,深之闕惡於元徽,所謂盜憎之義。作之見殺,不基晚歟!



\end{pinyinscope}