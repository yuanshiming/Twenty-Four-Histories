\article{卷六十五列傳第五十三}

\begin{pinyinscope}

 達奚武若干惠怡峰劉亮王德赫連達韓果蔡祐常善辛威厙狄昌梁椿梁臺田弘子仁恭孫德懋達奚武,字成興,代人也。祖眷,父長,並為鎮將。武少倜儻好馳射,賀拔岳徵關右,引為別將。及岳為侯莫陳悅所
 害,武與趙貴收屍歸平涼,同翊載周文帝。



 從平悅,封須昌縣伯。大統初,自大丞相府中兵參軍出為東秦州刺史。齊神武與竇泰、高敖曹三道來侵,周文欲并兵擊泰,諸將多異議,唯武及蘇綽與周文意同,遂禽之。周文進圖弘農,遣武從兩騎覘候。武與其候奇遇,即交戰,斬六級,獲三人而反。齊神武趣沙苑,周文復遣武覘之。武從三騎,皆衣敵人衣,至暮,下馬潛聽其軍號,歷營若警夜者,有不如法者,往往撻之。具知敵情以告,周文遂從破之。



 進爵高陽郡公。



 四年,周文援洛陽,武為前鋒,與李弼破莫多婁貸文。又進至河橋,力戰,斬其司徒高敖曹。再
 遷雍州刺史。復從戰芒山,時大軍不利,齊神武乘勝進軍至陜。



 武禦之,乃退。十七年,詔武經略漢川。梁梁州刺史宜豐侯蕭脩固守南鄭。武圍之,脩請服。會梁武陵王遣其將楊乾運等救脩,脩更不下。武擊走乾運,脩乃降。自斂門以北悉平。明年,振旅還京師。朝議欲以武為柱國,武曰:「我作柱國,不應在元子孝前。」固辭。以大將軍出鎮玉壁。



 周孝閔帝踐祚,授柱國、大司寇。齊北豫州刺史司馬消難舉州來附,詔武與楊忠迎消難以歸。武成實,轉大宗伯,進封鄭國公。齊將斛律敦侵汾、絳,武禦之,敦退。武築柏壁城,留開府權嚴、薛羽生守之。保定三年,遷太保。
 其年,大軍東伐,隨公楊忠引突厥自北道,武以三萬騎自東道期會晉陽。武至平陽,後期不進,而忠已還,武尚未知。齊將斛律明月遣武書曰:「鴻鶴已翔於寥廓,羅者猶視於沮澤也。」武覽書,乃班師。出為同州刺史。明年,從晉公護東伐。時尉遲迥圍洛陽,為敵所敗。武與齊王憲於芒山禦之。至夜,收軍。憲欲待明更戰。武曰:「洛陽軍散,人情駭動,不因夜速還,明日欲歸不得。」憲從之,遂全軍而返。天和三年,轉太傅。



 武微時,奢侈好華飾。及居重位,不持威儀,行常單馬,左右從一兩人而已,門外不施戟,恒晝掩一扉。或謂曰:「公位冠群后,何輕率若是?」武曰:「吾昔
 在布衣,豈望富貴!今日富貴,不可頓忘疇昔。且天下未平,國恩未報,安可過事威容乎?」言者慚而退。武之在同州,時旱,武帝敕武祀華岳。岳廟舊在山下,常所祈禱。武謂僚屬曰:「吾備位三公,不能燮理陰陽,不可同於眾人,在常祀所,必須登峰展誠,尋其聖奧。」岳既高峻,人跡罕通。武年逾六十,唯將數人攀藤而上,於是稽首祈請。晚不得還,即於岳上藉草而宿。夢一白衣來執武手曰:「快辛苦。」甚相嘉尚。武遂驚覺,益用祗肅。至旦,雲霧四起,俄而澍雨,遠近霑洽。



 武帝聞之,璽書勞武,賜綵百匹。



 武性貪吝,其為大司寇也,在庫有萬釘金帶,當時寶之,武因
 入庫,乃取以歸。



 主得白晉公護,護以武勳重,不彰其過,因而賜之。時論深鄙焉。薨,贈太傅、十五州諸軍事、同州刺史,謚曰桓。子震嗣。



 震字猛略。少驍勇,走及奔馬。周文嘗於渭北校獵,時有兔過周文前,震與諸將競射之,馬倒而墜。震足不傾躓,因步走射之,一發中兔。顧馬纔起,遂回身騰上。周文喜曰:「非此父不生此子。」乃賜震雜綵一百段。後封魏昌縣公。明帝初,拜司右中大夫,加驃騎大將軍、開府儀同三司。武成初,進爵廣平郡公,除華州刺史。震雖出自膏腴,少習武藝,然頗有政術。天和六年,拜柱國。建德初,襲爵鄭國公。從平鄴,賜妾二人、女樂一
 部,拜大宗伯。震父嘗為此職,時論榮之。宣政中,出為原州總管。隋開皇初,薨於家。



 震弟惎,大象末,為益州刺史,與王謙據蜀起兵,被誅。



 若干惠,字惠保,代武川人也。其先與魏俱起,以國為姓。父樹利周,從魏廣陽王深征葛榮,戰沒,贈冀州刺史。惠以別將從賀拔岳,以功封北平縣男。及岳為侯莫陳悅所害,惠與寇洛、趙貴等同謀翊戴周文。仍從平悅,拜直閣將軍。從禽竇泰,復弘農,破沙苑,惠每先登陷陣。加侍中、開府儀同三司、封長樂郡公。大統四年,從魏文帝東巡洛陽,與齊神武戰於河橋,力戰破之。七年,遷領軍。及
 高仲密舉北豫州來附,周文迎之。軍至洛陽,齊神武屯於芒山。惠為右軍,與中軍大破之。齊神武兵乃萃左軍,軍將趙貴等戰不利。會日暮,齊神武進兵攻惠,惠擊之,皆披靡。至夜中,神武騎復來追惠。惠徐下馬,顧命廚人營食。食訖,謂左右曰:「長安死,此中死,異乎?」乃建旗鳴角,收軍而還。神武追騎憚惠,疑有伏兵,不敢逼。至弘農,見周文,陳賊形勢,恨其垂成之功,虧於一簣,噓唏不自勝。周文壯之,遷司空。惠性剛質,有勇力,容貌魁岸。善於撫御,將士莫不懷恩。及侯景內附,朝議欲收輯河南,令惠以本官鎮魯陽。遇病,薨於軍。



 惠於諸將年最少。早喪父,事
 母以孝聞。周文嘗造射堂新成,與諸將宴射。惠竊歎曰:「親老矣,何時辦此!」周文聞之,即日徙堂於惠宅。其見重如此。及薨,為流涕久之。惠喪至,又臨撫焉。加贈秦州刺史,謚曰武烈。子鳳嗣。



 鳳字達摩,有識度。襲父爵長樂郡公,尚周文女。位開府儀同三司、大馭中大夫。後錄惠佐命功,封鳳徐國公,拜柱國。



 怡峰,字景阜,遼西人也。本姓默台,因避難改焉。高祖寬,燕遼西郡守,魏道武時歸朝,拜羽真,賜爵長蛇公。曾祖文,冀州刺史。峰少以驍勇聞。從賀拔岳討萬俟醜奴,賜爵蒲陰縣男。岳被害,峰與趙貴等同謀翊戴周文,進爵
 為伯。及齊神武與孝武帝構隙,文帝令峰與都督趙貴赴洛陽。至潼關,孝武西遷,峰即從周文帝拔迴洛,復潼關。後以討曹泥功,進爵華陽縣公。又從破竇泰於小關。復弘農,破沙苑,進爵樂陵郡公。仍與元季海、獨孤信復洛陽。東魏行臺任祥率步騎萬餘攻潁川,峰復以輕騎五百邀擊,大破之。自是威名轉盛。加授開府儀同三司。及周文與東魏戰河橋,時峰為左軍,不利,與李遠先還,周文遂班師。詔原其罪。拜夏州刺史。大統十五年,東魏圍潁川,峰與趙貴赴援。至南陽,病卒。峰沈毅有膽略,得士卒心,當時號驍將。周文嗟悼者久之。贈華州刺史,
 謚曰襄威。



 子昂嗣。位開府儀同三司。朝廷追錄峰功,封昂郡公。



 昂弟光,少以峰勛,賜爵安平縣侯,加開府儀同三司。



 光弟春,少知名,位吏部下大夫、儀同三司。



 劉亮,中山人也,本名道德。父特真,位領人酋長。魏大統中,以亮著勳,追贈恒州刺史。亮少倜儻,有從橫計略,姿貌魁傑,見者憚之。以都督從賀拔岳西征,以功封廣興縣子。侯莫陳悅害岳,亮與諸將謀迎周文。及平悅後,悅黨豳州刺史孫定兒仍據州不下,眾至數萬。周文令亮襲之。定兒以義兵猶遠,未為之備。亮乃輕將二十騎,先豎一纛於近城高嶺,即馳入城中。定兒方置酒高會,卒
 見亮至,眾皆駭愕。亮乃麾兵斬定兒,懸首州門,號令賊黨。仍遙指城外纛,命二騎曰:「出追大軍。」賊黨兇懼,一時降服。及周文置十二軍,簡諸將領之,亮領一軍。每征討,常與怡峰俱為騎將。以復潼關功,封饒陽縣伯。尋加侍中。從禽竇泰,復弘農,虞沙苑,並力戰有功。遷開府儀同三司、大都督,進爵長廣公。以母憂去職,居喪毀瘠。周文嗟其至性,每憂惜之。起復本官。亮以勇敢見知,為當時名將,兼屢陳謀策,多合機宜。周文謂曰:「卿文武兼資,即孤之孔明也。」乃賜名亮,并賜姓侯莫陳氏。出為東雍州刺史,為政清靜,百姓安之。卒於州。喪還京,周文親臨之,
 泣而謂人曰:「股肱喪矣,腹心何寄!」令鴻臚卿臨護喪事,追贈太尉,謚曰襄。



 後配餉周文廟廷。子昶嗣。



 昶尚周文女西河長公主,大象中,位柱國、秦靈二州總管,以亮功封彭國公。



 隋開皇中,坐事死。



 昶弟靜,天水郡守。靜弟恭,開府儀同三司、饒陽縣伯。恭弟幹,上儀同三司、褒中侯。



 王德,字天恩,代武川人也。少善騎射,雖不經師訓,以孝悌稱。初從爾朱榮討元顥,賜爵同官縣子。又從賀拔岳討平萬俟醜奴,別封深澤縣男。及侯莫陳悅害岳,德與寇洛等議,翊戴周文,於是除平涼郡守。德雖不知書,至於斷決處分,良吏無以過。涇州所部五郡,德常為最。及孝
 武西遷,進封下博縣伯,行東雍州事。



 在州未幾,百姓懷之。賜姓烏丸氏。大統元年,進爵為公,加車騎大將軍、儀同三司、北雍州刺史。後常從周文征伐,累有戰功,加開府、侍中,進爵河間郡公。先是河、渭間種羌屢叛,以德有威名,拜河州刺史。群羌率服。後卒於涇州刺史,謚曰獻。德性厚重廉慎,言行無擇。母幾年百歲,後德終。



 子慶嗣,小名公奴。性謹厚,位開府儀同三司。初德喪父,貧無以葬,乃賣公奴并一女以營葬事。因遭兵亂,不復相知。及德在平涼,始得之,遂名曰慶。



 赫連達,字朔周,盛樂人,勃勃之後也。曾祖庫多汗,因避
 難改姓杜氏。達性剛鯁有膽力。少從賀拔岳征討有功,賜爵長廣鄉男。及岳為侯莫陳悅所害,趙貴建議迎周文,達贊成其議,請輕騎告周文,仍迎之。諸將或欲南追賀拔勝,或云東告朝廷。達又曰:「此皆遠水不救近火,何足道哉!」謀遂定,令達馳往。周文見達慟哭,遂以數百騎南赴平涼,令達率騎據彈箏峽。時百姓惶懼奔散者,軍爭欲掠之。



 達止之,乃撫以恩信,人皆悅附。周文聞而嘉之。加平東將軍。周文謂諸將曰:「當清水公遇禍之日,君等性命懸於賊手。杜朔周冒萬死之難,遠來見及,遂得同雪讎恥。勞而不酬,何以勸善?」乃賜馬二百疋。孝武入
 關,褒敘勛義,以達首迎元帥,匡復秦、隴,進爵魏昌縣伯。從儀同李虎破曹泥。後復弘農,戰沙苑,皆有功。詔復姓赫連。以達勛望兼隆,乃除雲州刺史,進爵為公。從大將軍達奚武攻漢中。梁宜豐侯蕭脩拒守積時,後乃送款。開府賀蘭願德等以其食盡,欲急攻取之。



 達曰:「不戰而獲城,策之上也。無容利其子女,貪其財帛,仁者不為。如其困獸猶鬥,則成敗未可知。」武遂受脩降。師還,遷驃騎大將軍、開府儀同三司,加侍中,進爵藍田縣公。保定初,為大將軍、夏州總管。達雖非文吏,然性質直,遵奉法度,輕於鞭撻,而重慎死罪。性又廉儉。邊境胡人或饋達羊,
 達欲招異類,報以繒帛。主司請用官物。達曰:「羊入我廚,物出官庫,是欺上也。」命取私帛與之。



 識者嘉其仁恕。尋進爵樂川郡公,位柱國。薨。



 子遷嗣。位大將軍、蒲州刺史。



 韓果,字阿六拔,代武川人也。少驍雄,善騎射。賀拔岳西征,引為帳內,擊萬俟醜奴。後從周文討平侯莫陳悅。大統初,累進爵為石城公。果性強記,兼有權略,善伺敵虛實,揣知情狀。有潛匿溪谷欲為間偵者,果登高望之,所疑處,往必有獲。周文由是以果為虞候都督。每從征行,常領候騎,晝夜巡察,略不眠寢。從平竇泰於潼關,周文因其規畫,軍以勝返,賞真珠金帶一條。又從復弘農,破
 沙苑,戰河橋,並有功。歷朔、安二州刺史。從戰芒山,軍還,除河東郡守。又從大將軍破稽胡於北山。胡地險阻,人迹罕至,果進兵窮討,散其種落。稽胡憚果勁勇驕捷,號為著翅人。周文聞之,笑曰:「著翅之名,寧滅飛將。」累遷開府儀同三司、進爵褒中郡公。保定三年,拜少師,進位柱國。天和初,授華州刺史。為政寬簡,吏人稱之。薨。



 子明嗣。為黎州刺史,與尉遲迥同謀反,被誅。



 蔡祐,字承先,其先陳留圉人也。曾祖紹為夏州鎮將,徙居高平,因家焉。父襲,名著西州。魏正光中,萬俟醜奴亂關中,襲乃背賊歸洛陽。拜齊安郡守。及孝武西遷,始拔
 難西歸。賜爵平舒縣伯,除岐、雍二州刺史。祐性聰敏,有行檢。襲之背賊東歸,祐年十四,事母以孝聞。及長,有膂力。周文在原州,召為帳下親信。



 及遷夏州,以祐為都督。侯莫陳悅害賀拔岳,諸將迎周文,周文將赴之。夏州首望彌姐元進等陰有異計。周文微知之,召元進等入計事,既而目祐。祐即出外,衣甲持刀直入,叱元進而斬之,并其黨伏誅。一坐皆戰慄。於是與諸將盟,同心誅悅。



 周文以此重之,謂祐曰:「吾今以爾為子,爾其父事我。」慄迎孝武於潼關,以前後功封萇鄉縣伯。後從禽竇泰,復弘農,戰沙苑,皆有功。授平東將軍、太中大夫。



 又從戰河橋,
 祐下馬步斗,左右勸乘馬以備急卒。祐怒之曰:「丞相養我如子,今日豈以性命為念?」遂率左右十餘人,齊聲大呼,殺傷甚多。敵以其無繼,圍之十餘重。祐乃彎弓持滿,四面拒之。東魏人乃募厚甲長刀者,直進取祐。去祐可三十步,左右勸射之。祐曰:「吾曹性命,在此一矢耳,豈虛發哉!」敵人可十步,祐乃射之,中其面,應弦而倒,便以槊刺殺之。敵乃稍卻。祐乃徐引退。是戰也,西軍不利,周文已還。祐至弘農,夜與周文會。周文字之曰:「承先,爾來吾無憂矣!」



 周文驚,不得寢,枕祐股上乃安。以功進爵為公,授京兆郡守。



 高仲密舉北豫來附,周文率軍援之,與齊
 神武遇於芒山。祐時著明光鐵鎧,所向無敵。齊人咸曰:「此是鐵猛獸也。」皆避之。歷青、原二州刺史,尋除大都督。



 遭父憂,請終喪紀,弗許。累遷開府儀同三司,加侍中,賜姓大利稽氏,進爵懷寧郡公。六官建,授兵部中大夫。周文不豫,祐與晉公護、賀蘭祥等侍疾。及周文崩,祐悲慕不已,遂得氣疾。



 周孝閔帝踐祚,拜少保。祐與尉遲綱俱掌禁兵。時帝信任司會李植等,謀害晉公護。祐每泣諫,帝不聽。尋而帝廢。明帝之為公子也,與祐特相友暱,及即位,禮遇彌隆。加拜小司馬。御膳每有異味,輒以賜祐,群臣朝宴,每被別留,或至昏夜,列炬鳴笳,送祐還宅。祐以
 過蒙殊遇,常辭疾避之。至於婚姻,尤不願結於權要。尋以本官權鎮原州。頃之,授宜州刺史。未之部,卒於原州。



 祐少與鄉人李穆布衣齊名,常相謂曰:「大丈夫當建立功名,以取富貴,安能久處貧賤。」言訖,各大笑。後皆如言。及從征伐,為士卒先。軍還,諸將爭功,祐終無所競。周文每歎之曰:「承先口不言勛,孤當代其論敘。」性節儉,所得祿秩,皆散宗族,身死之日,家無餘財。贈柱國大將軍、原州都督,謚曰莊。子正嗣。



 祐弟澤,頗好學,有幹能。後為雲阜州刺史,以不從司馬消難被害。



 常善,高陽人也。家本豪族。魏孝昌中,從爾朱榮入洛,封
 房城縣男。後周文平侯莫陳悅,除天水郡守。累遷驃騎大將軍、開府儀同三司、西安州刺史,轉蔚州刺史。頻蒞二籓,有政績。進爵永陽郡公。周孝閔帝踐祚,拜大將軍、寧州總管。



 保定二年,入為小司徒。卒,贈柱國大將軍、都督、延州刺史。子昂和嗣。



 辛威,隴西人也。少慷慨有志略。初從賀拔岳征伐有功,假輔國將軍、都督。



 及周文統岳眾,見威奇之,引為帳內,封白土縣伯,後進爵為公。累遷開府儀同三司,賜姓普屯氏。出為鄜州刺史。威時望既重,朝廷以桑梓榮之,遷河州刺史、本州大中正。頻領二鎮,頗得人和。周孝閔帝
 踐祚,拜大將軍,進爵枹罕郡公。宣政元年,進位上柱國。大象二年,進封宿國公,復為少傅。薨。威性持重,有威嚴。



 歷官數十年,未嘗有過,故得以身名終。兼其家有義,五世同居,時以此稱之。



 子永達嗣。位儀同大將軍。



 厙狄昌,字恃德,神武人也。少便弓馬,有膂力。及長,進止閑雅,膽氣壯烈,每以將帥自許。從爾朱天光定關中。天光敗,又從賀拔岳征討。及岳被害,昌與諸將議翊戴周文。從平侯莫陳悅,賜爵陰盤縣子。後從迎孝武,復潼關,改封長子縣子。大統初,累遷開府儀同三司,進爵方城公。六官建,授稍伯中大夫。周孝閔帝踐祚,拜大將軍。
 卒。



 梁椿,字千年,代人也。初從爾朱榮入洛,又從賀拔岳討平萬俟醜奴,仍從周文平侯草陳悅。大統中,累以戰功封東平郡公,位開府儀同三司。周孝閔帝踐祚,除華州刺史,改封清陵郡公。保定元年,拜大將軍,卒於位。贈都督、恒州刺史,謚曰烈。椿性果毅,善於撫納,所獲賞物,分賜麾下,故每踐敵場,咸得其死力。



 雅好儉素,不營貲產,時論以此稱焉。



 子明,以椿功賜爵豐陽縣公。後襲椿爵,舊封回授弟朗。



 梁臺,字洛都,萇池人也。少果敢,有志操。從爾朱天光平
 關、隴,賜爵隴城鄉男。及天光敗於韓陵,賀拔岳又引為心膂。岳為侯莫陳悅所害,臺與諸將翊戴周文。從平悅,累功授潁州刺史,賜姓賀蘭氏。累遷驃騎大將軍、開府儀同、侍中。



 周孝閔帝踐祚,進爵中部縣公。保定四年,拜大將軍。時大軍圍洛陽,久不拔。齊騎奄至,齊公憲禦之。有數人為敵所執,已去。臺單馬突入,射殺兩人,敵皆披靡,被執者遂還。齊公憲每歎曰:「梁臺果毅膽決,不可及也。」五年,拜鄜州刺史。



 臺性疏通,恕以待物,至於蒞人,尤以惠愛為心。不過識千餘字,口占書啟,詞意可觀。年過六十,猶能被甲跨馬,足不躡鐙,馳射弋獵,矢不虛發。後
 以疾卒。



 田弘,字廣略,高平人也。少慷慨,有謀略。初陷萬俟醜奴。爾朱天光入關,弘自原州歸順。及周文統眾,弘求謁見,乃論時事,即處以爪牙之任。又以迎孝武功,封鶉陰縣子。周文嘗以所著鐵甲賜弘,云:「天下若定,還將此甲示孤也。」



 累功賜姓紇干氏,授原州刺史。以弘勛望兼至,故以衣錦榮之。周文在同州,文武並集,乃謂之曰:「人人如弘盡心,天下豈不早定?」即授車騎大將軍、儀同三司。



 魏廢帝元年,加驃騎大將軍、開府儀同三司。平蜀後,梁信州刺史蕭韶等未從朝化,詔弘討平之。又討西平反羌
 及鳳州叛氐等,並破之。每臨陣,推鋒直前,身被一百餘箭,破骨者九,馬被十槊。朝廷壯之。周孝閔踐阼,進爵鴈門郡公。保定元年,出為岷州刺史。弘雖武將,而動遵法式,百姓賴安之。三年,從隨公楊忠伐齊,拜大將軍。後進柱國大將軍,歷位大司空、少保、襄州總管。薨于州。子仁恭嗣。



 仁恭,字長貴。性寬仁,有局度。歷位幽州總管。隋文帝受禪,進上柱國,拜太子太師,甚見親重。嘗幸其第,宴飲極歡,禮賜甚厚。尋奉詔營太廟,進爵觀國公,拜右武衛大將軍,轉左武衛大將軍。卒官,贈司空,謚曰敬。子世師嗣。



 次子德懋,少以孝友知名。開皇初,以父軍功賜爵平原郡公,授太子千牛備身。



 丁父艱,哀毀骨立,廬於墓側,負土成墳。帝聞而嘉之,遣員外散騎侍郎元志就弔焉。復降璽書存問,賜帛及米,下詔表其閭。大業中,位尚書駕部郎,卒官。



 時有玉城郡公王景、鮮虞縣公謝慶恩並位上柱國;大義公辛遵及其弟韶並位柱國。隋文帝以其俱佐命功臣,特加崇貴,親禮與仁恭等,事皆亡失云。



 論曰:周文接喪亂之際,乘戰爭之餘,發跡平涼,撫征關右。于時外虞孔熾,內難方殷,羽檄交馳,戎軒屢駕,終能蕩清逋孽,克固鴻基。雖稟算於廟堂,實責成於將帥。達
 奚武、若干惠、怡峰、劉亮、王德、赫連達、韓果、蔡祐、常善、辛威、厙狄昌、梁椿、梁臺、田弘等,並兼資勇略,咸會風雲,或效績中權,或立功方面,均分休戚,同濟艱危,可謂國之爪牙,朝之禦侮者也。而武協規文後,得雋小關,周瑜赤壁之謀,賈詡烏巢之策,何以能尚?一言興邦,斯之謂矣。惠、德本以果毅知名,而能率由孝道,雖圖史所嘆,何以加焉?勇者不必有仁,斯不然矣。



 以赫連達之先識而加之以仁恕,蔡祐之敢勇而終之以不伐,斯豈企及之所致乎,抑亦天性而已。仁恭出內榮顯,豈徒然哉。德懋道協天經,亦足嘉矣。



\end{pinyinscope}