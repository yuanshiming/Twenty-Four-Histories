\article{卷二十一列傳第九}

\begin{pinyinscope}

 燕鳳許謙崔宏子浩張袞弟恂鄧彥海燕鳳,字子章,代人也。少好學,博綜經史,明習陰陽讖緯。昭成素聞其名,使以禮致之,鳳不應聘。及軍圍代,謂城人曰:「鳳不來者,將屠之。」代人懼,遂送鳳。昭成待以賓禮。後拜代王左長史,參決國事。又以經授獻明帝。



 嘗使苻堅,堅問鳳曰:「代王何如人?」對曰:「寬和仁愛,經略高遠,一時雄主也。常有并吞天下之志。」堅曰:「卿輩北人,無剛甲
 利兵,敵弱則進,敵強則退,安能并兼邪?」鳳曰:「北人壯悍,上馬持三仗,驅馳若飛。主上雄雋,率服北土,控弦百萬,號令若一。軍無輜重樵爨之苦,輕行速捷,因敵取資。此南方所以疲弊,北方所以常勝也。」堅曰:「彼國人馬多少?」鳳曰:「控弦之士數十萬,見馬一百萬匹。」堅曰:「卿言人眾則可,說馬太多。」鳳曰:「雲中川自東山至西河二百里,北山至南山百餘里,每歲孟秋,馬常大集,略為滿川。以此推之,使人言猶未盡。」鳳還,堅厚加贈遺。



 及昭成崩,道武將遷長安。鳳以道武幼弱,固請于苻堅曰:「代主初崩,臣子亡叛,遺孫沖幼,莫相輔立。其別部大人劉庫仁勇而
 有智,鐵弗衛辰狡猾多端,皆不可獨任。宜分部為二,令人統之。兩人素有深仇,其勢莫能先發,此禦邊之上策。



 待其孫長,乃存而立之,是陛下大惠於亡國也。」堅從之。鳳尋東還。及道武即位,歷吏部郎、給事黃門侍郎、行臺尚書,甚見器重。明元世,與崔宏、封懿、梁越等入講經傳,出議朝政。太武初,以舊勳賜爵平舒侯。卒,子才襲。



 許謙,字元遜,代人也。少有文才,善天文圖讖學。建國時,將家歸附,昭成擢為代王郎中令,兼掌文記。與燕鳳俱授獻明帝經。昭成崩後,謙徙長安。苻堅從弟行唐公洛鎮和龍,請謙之鎮。未幾,以繼母老,辭歸。登國初,遂歸道武,
 以為右司馬,與張兗等參贊初基。慕容寶之來寇也,道武使謙告難於姚興。興遣將楊佛嵩來援。佛嵩稽緩,道武命謙為書遺之,佛嵩乃倍道兼行。道武大悅,賜謙爵關內侯。寶敗,佛嵩乃還。及慕容垂死,謙上書勸進。并州平,以謙為陽曲護軍,賜爵平舒侯。卒,贈幽州刺史、高陽公,謚曰文。



 子洛陽襲爵。明元追錄謙功,以洛陽為鴈門太守。洛陽家田三生嘉禾,皆異畝同穎。太武善之,進爵北地公。卒,謚曰恭。



 崔宏,字玄伯,清河東武城人,魏司空林之六世孫也。祖悅,仕石季龍,位司徒右長史。父潛,仕慕容,為黃門侍
 郎。並以才學稱。



 宏少有雋才,號曰冀州神童。苻融之牧冀州,虛心禮敬。拜陽平公侍郎、領冀州從事。出總庶事,入為賓友,眾務修理,處斷無滯。苻堅聞之,徵為太子舍人。



 辭以母疾,不就。左遷著作佐郎。太原郝軒名知人,稱宏有王佐之材,近代所未有也。堅亡,避難齊魯間,為丁零翟釗及晉叛將張願所留。郝軒歎曰:「斯人也,遇斯時,不因扶搖之勢,而與鴳雀飛沈,豈不惜哉!」



 仕慕容垂,為吏部郎、尚書左丞、高陽內史,所歷著稱。立身雅正,雖在兵亂,獨厲志篤學,不以資產為意,妻子不免飢寒。



 道武征慕容寶,次中山。棄郡走海濱。帝素聞其名,遣求。及至,
 以為黃門侍郎,與張兗對總機要,草創制度。時晉使來聘,帝將報之,詔有司議國號。宏議曰:「三皇、五帝之立號也,或因所生之土,或以封國之名。故虞、夏、商、周始皆諸侯,及聖德既隆,萬國宗戴,稱號隨本,不復更立。唯商人屢徙,改號曰殷。然猶兼行,不廢始基之號。故《詩》云『殷商之旅』,此其義也。國家雖統北方廣漠之土,逮于陛下,應運龍飛。雖曰舊邦,受命惟新。以是登國之初改代曰魏。慕容永亦奉進魏土。夫魏者大名州之上國,斯乃革命之徵驗,利見之玄符也。臣愚以為宜號為魏。」道武從之,於是稱魏。



 及帝幸鄴,歷問故事。宏應對若流,帝善之。還
 次恒嶺,帝親登山頂,撫慰新人,適遇宏扶老母登嶺,賜以牛米。因詔諸徙人不能自進者,給以車牛。遷吏部尚書。時命有司制官爵,撰朝儀,協音樂,定律令,申科禁,宏總而裁之,以為永式。



 及置八部大夫,以擬八坐。宏通署三十六曹,如令、僕統事。深被信任,勢傾朝廷。



 約儉自居,不營產業,家徒四壁;出無車乘,朝晡步上。母年七十,供養無重膳。



 帝聞,益重之,厚加饋賜。時人亦或譏其過約,而宏居之愈甚。常引問古今舊事,王者制度,宏陳古人制作之體,及往代廢興之由,甚合上意。未嘗謇諤忤旨,亦不諂諛茍容。及道武季年,大臣多犯威怒,宏獨無譴
 者,由於此也。



 帝曾引宏講論《漢書》,至婁敬說漢祖,欲以魯元公主妻匈奴,善之,嗟嘆者良久。是以諸公主皆嫁於賓附之國,朝臣子弟、良族美彥不得尚焉。尚書職罷,賜宏爵白馬侯,加周兵將軍。與舊功臣庾岳、奚斤等同班,而信寵過之。



 道武崩,明元未即位,清河王紹因人心不安,大出財帛,班賜朝士。宏獨不受紹財,長孫嵩以下咸愧焉。詔遣使者循行郡國,糾察守宰不如法者,令宏與宜都公穆觀等案之,帝稱其平當。又詔宏與長孫嵩等朝堂決刑獄。



 明元以郡國豪右大人蠹害,乃優詔征之。人多戀本,而長吏逼遣之。於是輕薄少年,因相扇動,
 所在聚結。西河、建興盜賊並起,守宰討之不能禁。帝乃引宏及北新侯安同、壽光侯叔孫建、武元城侯元屈等問焉。宏欲大赦以紓之。屈曰:「不如先誅首惡,赦其黨類。」宏曰:「王者臨天下,以安人為本,何顧小曲直也。夫赦雖非正道,而可以權行。若赦而不改,誅之不晚。」明元從之。



 神瑞初,詔宏與南平公嵩等坐止車門右,聽理機事。并州胡數萬南掠河內,遣將軍公孫表等討之,敗績。帝問計於群臣。宏曰:「表等諸軍,不為不足,但失於處分,故使小盜假息耳。胡眾雖多,而無猛健主將,所謂千奴共一詹也。宜得大將素為胡所服信者,將數百騎,就攝表軍
 以討之。賊聞,必望風震怖。壽光侯建,前在并州,諸將莫及。」帝從之,遂平胡寇。尋拜天部大人,進爵為公。泰常三年夏,宏病篤,帝遣侍中穆觀就受遺言,侍臣問疾,一夜數返。卒,追贈司空,謚文貞公。喪禮一依安城王叔孫俊故事。詔群臣及附國渠帥皆會葬,自親王以外,盡命拜送。子浩襲。太和中,孝文追錄先朝功臣,以宏配饗廟廷。



 浩字伯深,少好學。博覽經史,玄象陰陽百家之言,無不該覽。研精義理,時人莫及。弱冠為通直郎,稍遷著作郎。道武以其工書,常置左右。道武季年,威嚴頗峻,宮省左右,多以微過得罪,莫不逃避,隱匿目下之變。浩獨恭勤
 不怠,或終日不歸。帝知之,輒命賜以御粥。其砥直任時,不為窮通改節若此。明元初,拜博士祭酒,賜爵武城子。常授帝經書,每至郊祀,父子並乘軒軺,時人榮之。明元好陰陽術數,聞浩說《易》及《洪範》五行,善之。因命筮吉兇,參觀天文,考定疑惑。浩總核天人之際,舉其綱紀者,數家多有應驗。恒與軍國大謀,甚為寵密。時有兔在後宮,檢無從得入,帝令浩推之。浩以為當有鄰國貢嬪嬙者。明年,姚興果獻女。



 神瑞二年,秋穀不登,太史令王亮、蘇坦因華陰公主等言:「讖書云:國家當都鄴,大樂五十年。」勸帝遷都於鄴,可救今年之饑。帝以問浩。浩曰:「非長久
 策也。東州之人,常謂國家居廣漠之地,人畜無算,號稱牛毛之眾。今留守舊都,分家南徙,恐不滿諸州之地。參居郡縣,處榛林之下,不便水土,疾疫死傷,情見事露,則百姓意阻。四方聞之,有輕侮之意,屈丐及蠕蠕必提挈而來。雲中、平城則有危殆之事,阻隔恒、代,千里之際,須欲救援,赴之甚難。如此,則聲實俱損矣。今居北方,假令山東有變,輕騎南出,耀威桑梓之中,誰知多少?百姓見之,望塵震伏。此是國家威制諸夏之長策也。至春草生,乳酪將出,兼有菜果,足接來秋。若得中熟,事則濟矣。」帝深然之。復使中貴人問浩曰:「今既無以至來秋,或復不
 熟,將如之何?」浩曰:「可簡窮下之戶,諸州就穀。若秋無年,願更圖也。



 但不可遷都。」帝於是分人詣山東三州就食,出倉穀以稟之。來年遂大熟,賜浩妾各一人,及御衣綿絹等。初,姚興死之前歲,太史奏熒惑在匏瓜星中,一夜忽然亡失,不知所在。或謂下入危亡之國,將為童謠妖言,而後行其災禍。帝乃召諸碩儒,與史官求其所詣。浩對曰:「案《春秋左氏傅》說神降于萃,其至之日,各其物也。



 請以日辰推之。庚午之夕,辛未之朝,天有陰雲,熒惑之亡,當在此二日之內。庚與午,皆主於秦,辛為西夷。今姚興據咸陽,是熒惑入秦矣。」諸人皆作色曰:「天上失星,人
 安能知其所詣,而妄說無征之言!」浩笑而不應。後八十餘日,熒惑果出東井,留守盤旋。秦中大旱赤地,昆明池水竭。童謠訛言,國中喧擾。明年,姚興死,二子交兵,三年國滅。於是諸人乃服。



 泰常元年,晉將劉裕伐姚泓,欲水斥河西上,求假道。詔群臣議之。外朝公卿咸曰:「函谷天險,裕何能西入?揚言伐姚,意或難測。宜先發軍斷河上流,勿令西過。」內朝咸同外計,帝將從之。浩曰:「此非上策也。司馬休之徒擾其荊州,劉裕切齒久矣。今興死子幼,乘其危亡而伐之,臣觀其意,必自入關。勁躁之人,不顧後患。今若塞其西路,裕必上岸北侵。如此則姚無事而我
 受敵矣。蠕蠕內寇,人食又乏,發軍赴南,則北冠進擊;若其救北,則南州復危,未若假之水道,縱裕西入。然後興兵塞其東歸之路。所謂卞莊刺彪,兩得之勢也。使裕勝也,必德我假道之惠;令姚氏勝也,亦不失救鄰之名。縱裕得關中,懸遠難守。彼不能守,終為我物。今不勞兵馬,坐觀成敗,鬥兩彪而收長久之利,上策也。夫為國之計,擇利為之,豈顧婚姻,酬一女子之惠也?假國家棄恒山以南,裕必不能發吳越之兵爭守河北也。」議者猶曰:「裕西入函谷,則進退路窮,腹背受敵。北上岸,則姚軍必不出關助我。揚聲西行,意在北進,其勢然也。」帝遂從群議,
 遣長孫嵩拒之。戰於畔城,為晉將朱超石所敗。帝恨不用浩言。



 二年,晉齊郡太守王懿來降。陳計,稱劉裕在洛,勸以軍絕其後路,則裕軍不戰而可克。書奏,帝善之。會浩在前,進講書傳。帝問浩曰:「裕西伐已至潼關,卿觀事得濟否?」浩曰:「姚興好養虛名而無實用,子泓又病,眾叛親離。乘其危亡,兵精將勇,克之必矣。」帝曰:「裕武能何如慕容垂?」浩曰:「垂承父祖之資,生便尊貴。同類歸之,若夜蛾之赴火;少加倚仗,便足立功。劉裕挺出寒微,不因一卒之用,奮臂大呼,而夷滅桓玄。北禽慕容超,南摧盧循。裕若平姚而篡其主。秦地戎夷混並,裕亦不能守之。秦
 地亦終當為國家所有。」帝曰:「裕已入關,不能進,不能退,我遣精騎南襲彭城、壽春,裕亦何能自立?」浩曰:「今西北二寇未殄,陛下不可親御六師。長孫嵩有經國之用,無進取之能,非劉裕敵也。臣謂待之不晚。」帝笑曰:「卿量之已審矣。」浩曰:「臣常私論近世人物,不敢不上聞。若王猛之經國,苻堅之管仲也;慕容恪之輔少主,慕容之霍光也;劉裕之平逆亂,司馬德宗之曹操也。」帝曰:「卿謂先帝如何?」浩曰:「太祖用漠北淳朴之人,南入漢地,變風易俗,化洽四海。自與羲、農、舜、禹齊烈,臣豈能仰名。」



 帝曰:「屈丐何如?」浩曰:「屈丐家國夷滅,一身孤寄,為姚氏封植。不
 思樹黨強鄰,報復仇恥,乃結蠕蠕,背德於姚。撅豎小人,無大經略,正可殘暴,終為人殘滅耳。」帝大悅,說至中夜。賜浩縹醪酒十斛,水精戎鹽一兩,曰:「朕味卿言,若此鹽酒,故與卿同其味也。」



 三年,彗星出天津,入太微,經北斗,絡紫微,犯天棓。八十餘日,至天漢而滅。帝復召諸儒、術士問之,曰:「災咎將在何國?朕甚畏之。」浩曰:「災異由人而起,人無,妖不自作。《漢書》載王莽篡位之前,彗星出入,正與今同。



 國家主尊臣卑,人無異望。是為僭晉將滅,劉裕篡之之應也。」諸人莫能易浩言,帝深然之。五年,宋果代晉,南鎮上宋改元赦書。時帝幸東南舄水齒池,射鳥,聞
 之,驛馳召浩,告曰:「往年卿言彗星之占驗矣。朕今日始信天道。」初,浩父疾篤。乃翦爪截髮,夜在庭中仰禱斗極,為父請命,求以身代。叩頭流血,歲餘不息,家人罕有知者。及父終,居喪盡禮,時人稱之。襲爵白馬公。



 自朝廷禮儀,優文策詔,軍國書記,盡關於浩。浩能為雅說,不長屬文,而留心於制度科律及經術之言。作《家祭法》,次序五宗,蒸嘗之禮,豐儉之節,義理可觀。性不好莊老之書,每讀不過數十行,輒棄之,曰:「此矯誣之說,不近人情,必非老子所作。老聃習禮,仲尼所師,豈設敗法之言以亂先王之教。袁生所謂家人筐篋中物,不可揚於王庭。」



 帝恒
 有微疾,而災異屢見,乃使中貴人密問浩曰:「今茲日蝕於胃、昴,盡光趙、代之分野。朕疾疹彌年,恐一旦奄忽,諸子並少,其為我設圖後計。」浩曰:「陛下春秋富盛,聖業方融,德以除災,幸就平愈。昔宋景見災修德,熒惑退舍。



 願陛下遺諸憂慮,恬神保和,無以闇昧之說,致損聖思。必不得已,請陳瞽言。自聖化龍興,不崇儲貳,是以永興之始,社稷幾危。今宜早建東宮,選公卿忠賢陛下素所委仗者,使為師傅;左右信臣簡在帝心者,以充賓友。入總萬機,出統戎政,監國撫軍,六柄在手。若此,則陛下可以優游無為,頤神養壽。此乃萬代之令典,塞禍之大備也。
 今長皇子諱,年漸一紀,明睿溫和,眾情所繫,時登儲副,則天下幸甚。立子以長,禮之大經,若須並大,成人而擇,倒錯天倫,則生履霜堅冰之禍。



 自古以來,載籍所記,興衰存亡,鮮不由此。」帝納之,於是使浩奉策告宗廟,令太武為國副主,居正殿臨朝。司徒長孫嵩、高陽公奚斤、北新公安同為左輔,坐東廂,西面。浩與太尉穆觀、散騎常侍丘堆為右弼,坐西廂,東面。百寮總己以聽焉。



 明元居西宮,時隱而窺之,聽其決斷。大悅,謂左右侍臣曰:「長孫嵩宿德舊臣,歷事四世,功存社稷;奚斤辯捷智謀,名聞遐邇;安同曉解俗情,明於校練;穆觀達政事要,識吾旨
 趣;崔浩博聞強識,精於天人之會;丘堆雖無大用,然在公專謹。



 以六人輔吾子,足以經國。吾與汝曹遊行四境,伐叛柔服,可以得志於天下矣。」



 群臣時奏事所疑。帝曰:「此非我所知,當決之於汝曹國主也。」



 會聞宋武帝殂,帝欲取洛陽、武牢、滑臺。浩曰:「陛下不以劉裕欻起,納其使貢,裕亦敬事陛下。不幸今死,乘喪伐之,雖得之,不令。《春秋》晉士丐侵齊,聞齊侯卒,乃還。君子大其不伐喪,以為恩足以感孝子,義足以動諸侯。今國家未能一舉而定江南,宜遣人弔祭,恤其凶災,布義風於天下,令德之事也。且裕新死,黨與未離,不如緩之,待其惡稔。如其強臣
 爭權,變難必起,然後命將揚威,可不勞士卒而收淮北之地。」帝銳意南伐,語浩曰:「劉裕因姚興死而滅其國。裕死,我伐之,何為不可!」浩固執曰:「興死,二子交爭,裕乃伐之。」帝大怒,不從。



 遂遣奚斤等南伐,議於監國之前曰:「先攻城,先略地?」斤請先攻城。浩曰:「南人長於固守,苻氏攻襄陽,經年不拔。今以大國之力,攻其小城,若不時剋,挫損軍勢,危道也。不如分軍略地,至淮為限,列置守宰,收斂租穀。滑臺、武牢反在軍北,絕望南救,必沿河東走。若或不然,即是囿中之物。」公孫表請先圖其城。斤等濟河,先攻滑臺,經時不拔,表請濟師。帝怒,乃親南巡,拜浩為
 相州刺史,隨軍謀主。及車駕還,浩從幸西河、太原,下臨河流,傍覽川城,慨然有感。



 遂與同寮論五等郡縣之是非,考秦皇、漢武之違失。時伏其言。



 天師寇謙之每與浩言,聞其論古興亡之迹,常自夜達旦,竦意斂容,深美之,曰:「斯人言也惠,皆可底行,亦當今之皋陶也。但人貴遠賤近,不能深察之耳。」



 因謂浩曰:「吾當兼攸儒教,輔助太平真君,而學不稽古。為吾撰列王者政典,并論其大要。」浩乃著書二十餘篇,上推太初,下盡秦、漢變弊之迹,大旨先以復五等為本。太武,左右忌浩正直,共排毀之。帝雖知其能,不免群議,故浩以公歸第。



 及有疑議,召問焉。
 浩纖妍白皙如美婦人。性敏達,長於謀計,自比張良,謂己稽古過之。既歸第,因欲修服食養性術,而寇謙之有《神中錄圖新經》,浩因師事之。



 始光中,進爵東郡公,拜太常卿。時議伐赫連昌,群臣皆以為難,唯浩曰:「往年以來,熒惑再守羽林,越鉤陳,其占秦亡。又今年五星并出東方,利以西伐。



 天應人和,時會並集,不可不進。」帝乃使奚斤等擊蒲阪,而親率輕騎掠其都城,大獲而還。後復討昌,次其城下,收眾偽退。昌鼓噪而前,舒陣為兩翼。會有風雨從東南來,揚沙昏冥,宦者趙倪進曰:「今風雨從賊後來,我向彼背,天不助人。



 又將士飢渴,願陛下攝騎避
 之,更待後日。」浩叱之曰:「是何言歟!千里制勝,一日之中,豈得變易?賊前行不止,後已離絕,宜分軍隱山,掩擊不意。風道在人,豈有常也?」帝曰:「善。」分騎奮擊,昌軍大潰。



 神二年,議擊蠕蠕,朝臣內外盡不欲行,保太后亦固止帝,帝皆不聽。唯浩讚成之。尚書令劉潔、左僕射安原等乃使黃門侍郎仇齊推赫連昌太史張深、徐辯說帝曰:「今年己巳,三陰之歲,歲星襲月,太白在西方,不可舉兵。北伐必敗,雖克不利於上。」又群臣共贊深等云:「深少時常諫苻堅不可南征,堅不從而敗。今天時人事都不和協,如何舉動?」帝意不快,乃召浩與深等辯之。



 浩難深曰:「
 陽者德也,陰者刑也,故月蝕修刑。夫王者之用刑,大則陳之原野,小則肆之市朝。戰伐者,用刑之大者也。以此言之,三陰用兵,蓋得其類,修刑之義也。歲星襲月,年饑人流,應在他國,遠期十二年。太白行蒼龍宿,於天文為東,不妨北伐。深等俗生,志意淺近,牽於術數,不達大體,難與遠圖。臣觀天文,比年以來,月行掩昴,至今猶然。其占,三年天子大破旄頭之國。蠕蠕、高車,旄頭之眾也。夫聖明御時,能行非常之事。古人語曰:『非常之原,黎人懼焉;及其成功,天下晏然。』願陛下勿疑。」深等慚曰:「蠕蠕荒外無用之物,得其地不可耕而食,得其人不可臣而使。
 輕疾無常,難得而制,有何汲汲而勞苦士馬。」



 浩曰:「深言天時,是其所職;若論形勢,非彼所知。斯乃漢世舊說常談,施之於今,不合事宜。何以言之?夫蠕蠕者,舊是國家北邊叛隸,今誅其元惡,收其善人,令復舊位,非無用也。漠北高涼,不生蚊蚋,水草美善,夏則北遷,田牧其地,非不可耕而食也。蠕蠕子弟來降,貴者尚公主,賤者將軍、大夫,居列滿朝。



 又高車號為名騎,非不可臣而畜也。夫以南人追之,則患其輕疾;於國兵則不然。



 何者?彼能遠走,我亦能遠逐,非難制也。往數入塞,國人震驚。今夏不乘虛掩進,破滅其國,至秋復來,不得安臥。自太宗之世,
 迄於今日,無歲不警,豈不汲汲乎哉?世人皆謂深、辯通解數術,明決成敗,臣請試之。問其西國未滅之前,有何亡徵?知而不言,是其不忠;若實不知,是其無術。」



 時赫連昌在坐,深等自以無先言,慚不能對。帝大悅,謂公卿曰:「吾意決矣。



 亡國之臣不可與謀,信哉!」而保太后猶疑之。復令群臣至保太后前評議,帝命浩善曉之令寤。



 既罷朝,或有尤浩曰:「吳賊侵南,舍之北伐,師行千里,其誰不知?蠕蠕遠遁,前無所獲,後有南侵之患,此危道也。」浩曰:「今年不摧蠕蠕,則無以禦南賊。自國家并西國以來,南人恐懼,揚聲動眾,以衛淮北。彼北我南,彼征我息,其勢
 然矣。北破蠕蠕,往還之間,故不見其至也。何以言之?劉裕得關中,留其愛子,精兵數萬,良將勁卒,猶不能固守,舉軍盡沒,號哭之聲至今未已。如何正當國家休明之世,士馬強盛之時,而欲以駒犢齒虎口也?設國家與之河南,彼必不能守之。自量不能守,是以必不來。若或有眾,備邊之軍耳。夫見瓶水凍,知天下之寒;嘗肉一臠,識鑊中之昧。物有其類,可推而得。且蠕蠕恃遠,謂國家力不能至,自寬來久。故夏則散眾放畜,秋肥乃聚,背寒向溫,南來寇抄。今掩其不備,大軍卒至,必驚駭,望塵奔走。牡馬護牧,牝馬戀駒;驅馳難制,不得水草;未過數日,朋
 聚而困弊,可一舉而滅。暫勞永逸,時不可失也。唯患上無此意。今聖慮已決,如何止之?」遂行。天師謂浩曰:「是行可果乎?」浩曰:「必克。但恐諸將瑣瑣,前後顧慮,不能乘勝深入,使不全舉耳。」



 及軍到,入其境,蠕蠕先不設備。於是分軍搜討,東西五千里,南北三千里,所虜及獲畜產車廬數百萬。高車殺蠕蠕種類歸降者三十餘萬落。虜遂散亂。帝沿弱水,西至涿邪山,諸大將果慮深入有伏兵,勸帝止。天師以浩曩日言,固勸帝窮討,帝不聽。後有降人言:「蠕蠕大檀先被疾,不知所為,乃焚穹廬,科車自載,將百人入山南走。人畜窘聚,方六十里,無人領統。相去
 百八十里,追軍不至,乃徐西遁,唯此得免。」聞涼州賈胡言:「若復前行二日,則盡滅之矣。」帝深恨之。



 大軍既還,南軍竟不能動,如浩所料。



 浩明識天文,好觀星變。常置金銀銅鋌於酢器中,令青,夜有所見,即以鋌畫紙作字,以記其異。太武每幸浩第,多問以異事。或倉卒不及束帶,奉進蔬食,不暇精美,帝為舉匕箸,或立嘗而還。其見寵愛如此。於是引浩出入臥內。加侍中、特進、撫軍大將軍、左光祿大夫,以賞謀謨之功。帝從容謂浩曰:「卿才智深博,事朕祖考,忠著三世,朕故延卿自近。其思盡規諫,勿有隱懷。朕雖當時遷怒,若或不用,久可不深思卿言也?」
 因令歌工歷頌群臣,事在《長孫道生傳》。又召新降高車渠帥數百人,賜酒食於前。指浩以示之曰:「汝曹視此人纖尪懦弱,手不能彎弓持矛,其胸中所懷,乃踰於兵甲。朕始時雖有征討之志,而慮不自決,前後剋捷,皆此人導吾令到此矣。」乃敕諸尚書曰:「凡軍國大計,卿等所不能決,皆先咨浩然後行。」



 俄而南籓諸將表宋師欲犯河南,請兵三萬,先其未發逆擊之。因誅河北流人在界上者,絕其鄉導,足以挫其銳氣,使不敢深入。詔公卿議之,咸言宜許。浩曰:「此不可從也。往年國家大破蠕蠕,馬力有餘。南賊喪精,常恐輕兵奄至,故揚聲動眾,以備不虞,
 非敢先發。又南土下濕,夏月蒸暑,非行師之時。且彼先嚴有備,必堅城固守。屯軍攻之,則糧食不給;分兵肆討,則無以應敵。未見其利。就使能來,待其勞倦,秋涼馬肥,因敵取食,徐往擊之,萬全之計。在朝群臣及西北守將,從陛下征討,西滅赫連,北破蠕蠕,多獲美女珍寶,馬畜成群;南鎮諸將,聞而生羨,亦欲南抄,以取資財。是以妄張賊勢,披毛求瑕,冀得肆心。既不獲聽,故數稱賊動以恐朝廷。背公存私,為國生事,非忠也。」帝從浩議。



 南鎮諸將表賊至,而自陳兵少,求簡幽州以南戍兵佐寧。就漳水造船,嚴以為備。公卿議者僉然,欲遣騎五千,并假署
 司馬楚之、魯軌、韓延之等,令誘引邊人。



 浩曰:「非上策也。彼聞幽州已南,精兵悉發,大造舟船,輕騎在後,欲存立司馬,誅除宋族,必舉國駭擾,懼於滅亡,當悉發精銳,來備北境。後審知官軍有聲無實,恃其先聚,必喜而前行,徑來至河,肆其侵暴。則我守將,無以禦之。若彼有見機之人,善設權譎,乘間深入,虞我國虛,生變不難。非制敵之良計。今公卿欲以威力攘賊,乃所以招令速至也。夫張虛聲而召實害,此之謂矣。不可不思,後悔無及。



 我使在彼,期四月前還,可待使至,審而後發,猶未晚也。楚之人徒,是彼所忌,將奪其國,彼安得端坐視之?故楚之往
 則彼來,楚之止則彼息,其勢然也。且楚之等瑣才,能招合輕薄無賴,而不能成就大功。為國生事,使兵連禍結,必此之群矣。



 臣嘗聞魯軌說姚興,求入荊州。至則散敗,乃不免蠻賊掠賣為奴,使禍及姚泓,已然之效。」



 浩又陳天時不利於彼,曰:「今茲害氣在揚州,不宜先舉兵,一也。午歲自刑,先發者傷,二也。日蝕滅光,晝昏星見,飛鳥墮落,宿當斗、牛,憂在危亡,三也。



 熒惑伏匿於翼、軫,戒亂及喪,四也。太白未出,進兵者敗,五也。夫興國之君,先修人事,次盡地利,後觀天時,故萬舉而萬全,國安而身盛。今宋新國,是人事未周也;災變屢見,是天時不協也;舟行
 水涸,是地利不盡也。三事無一成,自守猶或不安,何得先發而攻人哉?彼必聽我虛聲而嚴,我亦承彼嚴而動,兩推其咎,皆自以為應敵。兵法當分災,迎受害氣,未可舉動也。」帝不能違眾,乃從公卿議。



 浩復固爭,不從。遂遣陽平王杜超鎮鄴,瑯邪王司馬楚之等屯潁川。於是寇來遂疾,到彥之自清水入河,水斥流西行,分兵列守南岸,西至潼關。



 帝聞赫連定與宋縣分河北,乃先討赫連。群臣皆曰:「義隆軍猶在河中,舍之西行,前寇未可必剋;而義隆乘虛,則東州敗矣。」帝疑焉,問計於浩。浩曰:「義隆與赫連定同惡相連,招結馮跋,牽引蠕蠕,規肆逆心,虛相
 唱和。義隆望定進,定待義隆前,皆莫敢先入。以臣觀之,有似連雞,不得俱飛,無能為害也。臣始謂義隆軍屯住河中,兩道北上,東道向冀州,西道衝鄴。如此則陛下當自致討,不得徐行。今則不然,東西列兵,徑二千里中,一處不過千,形分勢弱。以此觀之,儜兒情見,正望固河自守,免死為幸,無北度意也。赫連定殘根易摧,擬之必僕。



 定之後,東出潼關,席卷而前,威震南極,江淮以北無立草矣。聖策獨發,非愚近所及,願陛下必行無疑。」



 平涼既平,其日宴會,帝執浩手以示蒙遜使曰:「所云崔公,此是也。才略之美,當今無比。朕行止必問,成敗決焉,若合
 符契。」



 後冠軍安頡軍還,獻南俘,因說南賊之言云:「宋敕其諸將,若北國兵動,先其未至,徑前入河。若其不動,住彭城勿進。」如浩所量。帝謂公卿曰:「卿輩前謂我用浩計為謬,驚怖固諫。常勝之家,自謂踰人遠矣,至於歸終,乃不能及。」



 遷浩司徒。



 時方士祁纖奏立四王,以日東西南北為名,欲以致禎吉,除災異。詔浩與學士議之。浩曰:「先王建國,以作籓屏,不應假名其福。夫日月運轉,周歷四方,京師所居,在於其內。四王之稱,實奄邦畿,名之則逆,不可承用。」先是,纖奏改代為萬年,浩曰:「昔太祖道武皇帝應期受命,開拓洪業,諸所制宜,無不循古。



 以始封代
 土,後稱為魏。故代、魏兼用,猶彼殷、商。國家積德,著在圖史,當享萬億,不待假名以為益也。纖之所聞,皆非正義。」帝從之。



 時河西王沮渠牧犍內有貳意,帝將討焉,先問於浩。浩對曰:「牧犍惡心已露,不可不誅。官軍往年北伐,雖不IN獲,實無所損。于時行者,內外軍馬三十萬匹,計在道死傷,不滿八千。歲常羸死,恒不減萬,乃不少於前。而遠方承虛,便謂大損,不能復振。今出其不圖,大軍卒至,必驚懼騷擾,不知所出,擒之必矣。牧犍幼弱,諸弟驕恣,爭權縱橫,人心離解。加以比年以來,天災地變,都在秦、涼,成滅之國也。」



 帝命公卿議之,恒農王奚斤等三十
 餘人皆表曰:「牧犍西垂下國,雖心不為純臣,然繼父修職貢,朝廷接以蕃禮。又王姬釐降,罪未甚彰,謂且羈縻而已。令士馬勞止,可宜小息。又其地鹵斥,略無水草,大軍既到,不得久停。彼聞軍來,必完聚城守,攻則難拔,野無所掠。」於是尚書古弼、李順之徒皆曰:「自溫闈河以西至於涼州,地純枯石,了無水草,不見流川。皆言姑臧城南天梯山上,冬有積雪深一丈,至春夏消液,下流成川,引以溉灌。彼聞軍至,決此渠口,水不通流,則致渴乏。去城百里之內,赤地無草,不任久停軍馬。斤等議是也。」帝乃命浩以其前言與斤共相難抑。諸人不復餘言,唯曰
 彼無水草。浩曰:「《漢書地理志》稱『涼州之畜,為天下饒』,若無水草,何以畜牧?又漢人為居,終不於無水草之地築城郭立郡縣也。又雪之消液,裁不斂塵,何得通渠引漕,溉灌數百萬頃乎?此言大詆誣於人矣。」



 李順等復曰:「吾曹目見,何可共辯?」浩曰:「汝曹受人金錢,欲為之辭,謂我目不見便可欺也!」帝隱聽,聞之乃出,親見斤等。辭旨嚴厲,形於神色。群臣乃不敢復言。於是遂討涼州,平之。多饒水草,如浩所言。



 乃詔浩總理史務,務從實錄。於是監祕書事,以中書侍郎高允、散騎侍郎張偉參著作,續成前紀。至於損益褒貶,折衷潤色,浩所總焉。浩有鑒識,以
 人倫為己任。明元、太武之世,征海內賢才,起自仄陋。及所得外國遠方名士,拔而用之,皆浩之由也。至於禮樂憲章,皆歸宗於浩。



 及景穆始總百揆,浩復與宜都王穆壽輔政事。又將討蠕蠕,劉潔復致異議。帝愈欲討之,乃召問浩。浩對曰:「往擊蠕蠕,師不多日,潔等各欲回還。後獲尚書,云軍還之時,去賊三十里,是潔等之計過矣。夫北土多積雪,至冬時,常避寒南徙。



 若因其時,潛軍而出,必與之遇。既與之遇,則可禽獲。」帝以為然。乃分軍四道,諸將俱會鹿渾海。期日有定,而潔恨計不用,沮誤諸將,無功而還。



 帝西巡至東雍,親臨汾曲,觀叛賊薛永宗壘,
 進軍圍之。永宗出兵欲戰,帝問浩曰:「今日可擊否?」浩曰:「永宗未知陛下自來,人心安固。北風迅疾,宜急擊之,須臾必破。若待明日,恐見官軍盛大,必夜遁走。」帝從之,永宗潰滅。車駕濟河,前驅告賊在渭北。帝至洛水橋,賊已夜遁。詔問浩曰:「蓋吳在長安北九十里,渭北地空,穀草不備,欲度渭南西行,何如?」浩曰:「蓋吳營去此六十里,賊魁所在。擊蛇之法,當先破頭,頭破則尾豈能動?宜乘勢先擊吳。今軍往,一日便到。吳平之後,迴向長安,亦一日而至。一日之乏,未便損傷。愚謂宜從北道。



 若從南道,則蓋吳徐入北山,卒未可平。」帝不從,乃度渭南。吳聞帝至,
 盡散入北山,果如浩言。軍無所剋,帝悔之。後以浩輔東宮之勤,賜繒絮布各千段。



 帝蒐於河西,詔浩詣行所議軍事。浩表曰:「昔漢武患匈奴強盛,故開涼州五郡,通西域,廣農積穀,為滅賊之資,東西迭擊。故漢未疲而匈奴已弊,後遂入朝。



 昔平涼州,臣愚以為北賊未平,征役不息,可不徙其人,案前世故事,計之長者。



 若徙其人,則土地空虛,雖有鎮戍,適可禦邊而已。至於大舉,軍資必乏。陛下以此事闊遠,竟不施用。如臣愚意,猶如前議,募徙豪彊大家,充實涼土。軍舉之日,東西齊勢,此計之得者。」



 浩又上《五寅元曆》。表曰:「太宗即位元年,敕臣解《急就章》、《
 孝經》、《論語》、《詩》、《尚書》、《春秋》、《禮記》、《周易》,三年成訖。復詔臣學天文星曆、《易》式、九宮,無不盡看。三十九年,晝夜無廢。臣稟性弱劣,力不及健婦人,更無餘能,是以專心思書,忘寢與食。至乃夢共鬼爭義,遂得周公、孔子之要術。始知古人有虛有實,妄語者多,真正者少。自秦始皇燒書之後,經典絕滅。漢高祖以來,世人妄造曆術者十餘家,皆不得天道之正。大誤四千,小誤甚多,不可言盡。臣愍其如此。今遭陛下太平之世,除偽從真,宜改誤歷,以從天道。



 是以臣前奏造曆,今始成訖,謹以奏。惟恩省察,以臣曆術,宣示中書博士,然後施用。非但時人,天地鬼神
 知臣得正,可以益國家萬世之名,過於三皇、五帝矣。」



 浩又以《晉書》諸家並多誤,著《晉後書》,未就,傳世者五十餘卷。



 初,道武詔祕書郎鄧彥海著國記十餘卷,編年次事,體例未成,逮于明元,廢不著述。神蒨二年,詔集諸文人摭錄國書。浩及弟覽、高讜、鄧穎、晁繼、范享、黃輔等共參著作,敘成國書三十卷。著作令史太原閔堪、趙郡卻標素諂事浩,乃請立石,銘載國書,以彰直筆。并勒浩所注《五經》。浩贊成之,景穆善焉。遂營於天郊東三里,方百步,用功三百萬乃訖。



 浩書國事備而不典,而石銘顯在衢路,北人咸悉忿毒,相與構浩於帝。帝大怒,使有司案浩,
 取秘書郎及長歷生數百人意狀。浩服受賕。真君十一年六月,誅浩。



 清河崔氏無遠近,及范陽盧氏、太原郭氏、河東柳氏,皆浩之姻親,盡夷其族。其祕書郎史以下盡死。



 浩始弱冠,太原郭逸以女妻之。浩晚成,不曜華採,故時人未知。逸妻王氏,宋鎮北將軍王仲德姊也。每奇浩才能,自以為得婿。俄而女亡,王氏深以傷恨,復欲以少女繼昏。逸及親屬以為不可,王氏固執與之。逸不能違,遂重結好。浩非毀佛法,而妻郭氏敬好釋典,時時讀誦。浩怒,取而焚之,捐灰廁中。及浩幽執,被置檻內,送於城南,使衛士數十人溲其上,呼聲嗷嗷,聞于行路。自宰司
 之被戮辱,未有如浩者,世皆以為報應之驗。



 初,浩害李順,基萌已成,夜夢以火爇順寢室,火作而順死。浩與室家群立觀之。俄而順弟息號哭而出,曰:「此輩吾賊也!」以戈擊之,悉投於河。寤而以告館客馮景仁,曰:「此真不善也。夫以火爇人,暴之極也。且兆始惡者有終殃,積不善者無餘慶。厲階成矣,公其圖之。」浩曰:「吾方思之。」而不能悛,至是而族。



 浩既工書,人多託寫《急就章》,從少至老,初不憚勞。所書蓋以百數,必稱「馮代彊」,以示不敢犯國。其謹也如此。浩書體勢及其先人,而巧妙不如也。世寶其迹,多裁割綴連,以為摹楷。



 浩母,盧諶孫女也。浩著《食經
 序》曰:「余自少及長,耳目聞見,諸母諸姑所修婦功,無不蘊習酒食。朝夕養舅姑,四時供祭祀,雖有功力,不任僮使,常手自親焉。昔遭喪亂,饑饉仍臻,食稟蔬餬口,不能具其物用,十餘年間,不復備設。



 先妣慮久廢忘,後生無所知見,而少不習書,乃占授為九篇。文辭約舉,婉而成章,聰辯彊記,皆此類也。親沒之後,遇國龍興之會,平暴除亂,拓定四方。余備位台鉉,與參大謀。賞獲豐厚,牛羊蓋澤;貲累巨萬,衣則重錦,食則粱肉。遠惟平生,思季路負米之時,不可復得。故序遺文,垂示來世。」



 浩弟簡,字仲亮,一名覽。好學,少以善書知名。道武初,歷中書侍郎,爵五
 等侯,參著作事。卒。簡弟恬,字叔玄,小名白。位豫州刺史,爵武陽侯。坐浩伏誅。



 宏祖悅,與范陽盧諶並以博藝齊名。諶法鐘繇,悅法衛瓘,而俱習索靖之草,皆盡其妙。諶傳子偃,偃傳子邈;悅傳子潛,潛傳子宏。世不替業,故魏初重崔、盧之書。宏自非朝廷文誥,四方書檄,初不妄染,故世無遺文。尤善草隸,為世摹楷,行押特盡精巧,而不見遺迹。始宏因苻氏亂,欲避地江南,為張願所獲,本圖不遂。乃作詩以自傷,而不行於時,蓋懼罪也。浩誅,中書侍郎高允受敕收浩家書,始見此詩,允知其意。允孫綽錄於允集。



 初,宏父潛為兄渾等誄手筆本草,延昌初,著
 作佐郎王遵業買書於市,遇得之。



 年將二百,寶其書迹,深藏祕之。武定中,遵業子松年將以遺黃門郎崔季舒,人多摹拓之。左光祿大夫姚元標以工書知名於時,見潛書,以為過於浩也。



 宏弟徽,字玄猷,少有文才,與勃海高演俱知名。歷位秘書監,賜爵貝丘侯。



 樂安王範鎮長安,選舊德之士與範俱,以徽為平西將軍副將,行樂安王傅,進爵濟南公。徽為政務存大體,不親小事。性好人倫。引接賓客,或談及平生,或講論道義,誨誘後進,終日不止。以疾,徵還京師,卒,謚曰元公,士類無不歎惜。



 始清河崔寬祖肜,隨晉南陽王保避地隴右,遂仕西涼及沮
 渠氏。



 肜生剖,字伯宗,每慷慨有懷東土。常歎曰:「風雨如晦,雞鳴不已,吾所庶幾!」及太武西巡,剖乃總率同義,使子寬送款。太武嘉之,拜寬岐陽令,賜爵延水男。遣使與寬俱西,撫慰初附。征部詣京師,未至而卒。文成以剖誠著先朝,贈涼州刺史、武陵公,謚曰元。



 寬字景仁,還京,封安國子,位弘農太守。初,寬通款見浩,浩與相齒次,厚存接之。及浩誅,以遠來疏族,獨得不坐。遂家于武城,居司空林舊墟,以一子繼浩。與浩弟覽妻封氏相奉如親。寬後襲爵武陵公,陜城鎮將。三崤地險,人多寇劫。



 而寬性滑稽,誘接豪右,宿盜魁帥,與相交結。傾衿待遇,不逆細
 微,莫不感其意氣。時官無祿力,唯取給於人,寬善撫納,招致禮遺,大有取受,而與之者無恨。



 又恒農出漆蠟竹木之饒,路與南通,貿易來往,家產豐富,而百姓樂之。諸鎮之中,號曰能政。及解鎮,人人追戀,詣闕上疏者三百餘人。卒,遺言薄葬,斂以時服。



 長子衡,字伯玉,少以孝行著稱。學崔浩書,頗亦類焉。天安元年,擢為內秘書中散。班下詔命及御所覽書,多其迹也。衡舉李沖、李元愷、程駿等,終為名器。



 承明元年,遷內都坐令,善折獄,孝文嘉之。太和二年,襲爵武陵公。衡涉獵書史,頗為文筆。蠕蠕時犯塞,衡上書陳備御之方、便國利人之策凡五十餘
 條。除秦州刺史,徙爵齊郡公。先是,河東年饑,劫盜大起。衡至,修龔遂法,勸課農桑,周年間,寇盜止息。卒,贈冀州刺史,謚惠公。衡五子。



 長子敞,字公世,襲爵,例降為侯,為平原相。敞性狷急,與刺史楊椿迭相表列,敞坐免官。宣武初,為鉅鹿太守。弟朏之逆,敞為黃木軍主韓文殊所藏。其家悉見籍沒,唯敞妻李氏以公主之甥,自隨奴婢田宅二百餘口得免。正光中,普釋禁錮,敞復爵郡侯,卒於趙郡太守。



 敞弟鐘,字公祿,奉朝請。弟朏之逆,以出後被原。歷司徒右長史、金紫光祿大夫、冀州大中正。敞亡後,鐘貪其財,誣敞息子積等三人非兄胤,辭訴累歲,人
 士疾之。爾朱世隆為尚書令,奏除其官,終身勿齒。朏好學,有文才,為京兆王愉錄事參軍,與愉同逆,伏法。



 宏同郡董謐。謐父京,與同郡崔康時、廣陽霍原等,俱以碩學,播名遼海。謐好學,傳父業。中山平,入朝,拜儀曹郎,撰朝覲、饗宴、郊廟、社稷之儀。



 張袞,字洪龍,上谷沮陽人也。祖翼,父卓,位並太守。袞篤實好學,有文才。



 道武為代王,選為左長史。從追蠕蠕五六百里。諸部帥因兗言糧盡,不宜深入。帝問袞:「殺副馬足三日食乎?」皆言足。帝乃倍道追及於廣漠赤地南床山下,大破之。既而帝問袞曰:「卿曹外人,知我前問三日
 糧意乎?蠕蠕奔走數日,畜產失飲,至水必留。計其道程,三日足及。輕騎卒至,出其不意,彼必驚散,其勢然矣。」



 部帥聞之,咸曰:「聖策,非所及也。」袞常參大謀,每告人曰:「主上天資傑邁,必能囊括六合。夫遭風雲之會,不建騰跳之功者,非人豪也。」遂策名委質,竭誠伏事。時劉顯地廣兵彊,跨有朔裔,會其兄弟乖離,共相疑阻。袞言於道武曰:「顯志大意高,今因其內釁,宜速乘之。」帝從之,遂破走顯。又從破賀訥。道武登勿居山遊宴,從官請聚石為峰,以記功德,乃命袞為文。



 慕容寶之來寇也,袞言於道武曰:「寶乘滑臺功,因長子捷,頌財竭力,難與爭鋒,宜羸師
 以侈其心。」帝從之,果破之參合。遷給事黃門侍郎。道武南伐,次中山,袞遺寶書,喻以成敗。寶見書,大懼,遂奔和龍。既剋中山,聽入八議,拜幽州刺史,賜爵臨渭侯,百姓安之。



 天興初,徵還京師。後與崔逞答晉將郗恢書失旨,黜為尚書令史。袞遇創業之初,始以才謀見任,率心奉上,不顧嫌疑。道武曾問南州人於袞,袞與盧溥州里,數稱薦之。又未嘗與崔逞相識,聞風稱美。及中山平,盧溥聚黨為逆,崔逞答書不允,並乖本言,故忿之。



 袞年過七十,闔門守靜,手執經書,刊定乖失。愛好人物,善誘無倦,士類以此高之。永興二年,卒。太武後追錄舊勳,遣大鴻
 臚即墓策贈太保,謚文康公。



 子度,少有學尚,襲爵臨渭侯,卒於中都大官。



 度子白澤,年十一,遭母憂,以孝聞。長而博學。文成初,除殿中曹給事中,甚見寵任。白澤本字鐘葵,獻文賜名白澤,納其女為嬪。出行雍州刺史。清心少欲,人吏安之。獻文詔諸監臨官取所監羊一口、酒一斛者,罪至大辟;與者以從坐論。



 糾得尚書以下罪狀者,各隨所糾官輕重而授之。白澤上表,以為此法若行之不已,恐姦人窺望,勞臣懈節,請依律令舊法。獻文納之。太和初,懷州人伊祁茍初三十餘人謀反,文明皇太后欲盡誅一城人。白澤諫,以為《周書》父子兄弟罪不相及,
 不誣十室,而況一州。后從之,乃止。轉散騎常侍、殿中尚書。卒,贈相州刺史、廣平公,謚曰簡。



 長子倫,字天念,大司農少卿、燕州大中正。熙平中,蠕蠕主醜奴遣使來朝,抗敵國之禮,不修臣敬。朝議將依漢答匈奴故事,遣使報之。倫表以為:「虜雖慕德,亦來觀我。懼之以彊,儻或歸附;示之以弱,窺覦或起。《春秋》所謂以我卜也。高祖、世宗知其若此,來既莫逆,去又不追。必其委贄玉帛之辰,屈膝籓方之禮,則豐其勞賄,藉以珍物。至於王人遠役,銜命虜庭,優以匹敵之尊,加之想望之寵,恐徒生虜慢,無益聖朝。」不從。孝莊初,卒於大司農卿。



 袞弟恂。
 恂字洪讓,隨兄袞歸北,參代王軍事。說道武宜收中土士庶之望,以建大業。帝深加器異。皇始初,拜中書侍郎。帷幄密謀,頗亦參預。賜爵平皋子,出為廣平太守。恂招集離散,勸課農桑,流人歸者數千戶。遷常山太守。恂開建學校,優禮儒士,吏人歌詠之。時喪亂之後,罕能克厲者,唯恂當官清白,仁恕臨下,百姓親愛之,政為當時第一。明元即位,徵拜太中大夫。卒。恂性清儉,死日家無餘財。贈并州刺史、平皋侯,謚曰宣。



 子純,字道尚,襲爵。坐事除。



 純弟代,字定燕,陳留、北平二郡太守。卒,贈營州刺史,謚惠侯。代所歷著稱,有父遺風。



 代子萇年,為汝南太守。
 郡人劉崇之兄弟分析,家貧,唯一牛,爭不能決,訟於郡庭。萇年悽而見之,謂曰:「汝曹當以一牛,故致此競;脫有二牛,必不爭。」



 乃以己牛一頭賜之。於是境中各相戒約,咸敦敬讓。卒于郡。子琛,字寶貴,少有孝行,位至太子翊軍校尉。卒。



 鄧彥海,安定人也。祖羌,苻堅車騎將軍。父翼,河間相。慕容垂之圍鄴,以為冀州刺史,爵真定侯。拒對使者曰:「先君忠于秦室,翼豈可先叛乎?忠臣不事二主,未敢聞命。」垂遣喻之曰:「吾與車騎結為異姓兄弟,卿亦猶吾子弟,安得辭乎?」翼曰:「冀州宜任親賢,翼請他役效命。」垂乃用
 為河間太守。後卒於趙郡內史。



 彥海性貞素,言行可復,博覽經書,長於《易》筮。道武定中原,擢為著作郎,再遷尚書吏部郎。彥海明解制度,多識故事,與尚書崔宏參定朝儀、律令、音樂,及軍國文記、詔策多是彥海所為。賜爵下博子。道武詔彥海撰國記十餘卷,唯次年月,起居行事而已,未有體例。彥海謹於朝事,未嘗忤旨。其從父弟暉時為尚書郎,兇俠好奇,與定陵侯和跋厚。跋有罪誅,其子弟奔長安。或告暉將送出之,由是道武疑知情,遂賜彥海死。既而悔之。時人咸愍惜焉。



 子穎襲爵,稍遷中書侍郎。太武詔太常卿崔浩集諸文學撰述國書,穎與
 浩弟覽等俱參著作事。太武幸漠南,高車莫弗庫若干率騎數萬餘,驅鹿百餘萬詣行所。詔穎為文,銘於漠南,以記功德。兼散騎常侍,使宋。進爵為侯。卒,謚曰文恭。子怡襲爵,位荊州刺史,賜爵南陽公。卒。



 子侍,孝文賜名述,位齊州刺史。初改置百官,始重公府元佐,以述為太傅元丕長史。座於司空長史。謚曰貞。



 論曰:昭成、道武之時,雲雷方始,至於經邦緯俗,文武兼資。燕鳳博識多聞,首膺禮命。許謙才術俱美,驅馳艱虞。不然,何以成帝業也。崔宏家世雋偉,仍屬權輿,總機任重,守正成務,禮從清廟,固其宜也。浩才藝通博,究覽天
 文,政事籌策,時莫之二。此其所以自比於子房焉。屬明元為政之秋,太武經營之日,言聽計從,寧廓區夏,遇既深矣,勤亦茂哉。謀雖蓋世,威未震主,末途邂逅,遂不自全。豈鳥盡弓藏,人惡其上,將器盈必概,陰害貽禍,何斯人而遭斯酷乎?至若張袞才策,不免其戾,彥海貞白,禍非其罪,亦足痛云。洪讓世著循吏,家風良可貴矣。



\end{pinyinscope}