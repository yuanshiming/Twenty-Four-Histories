\article{卷八十七列傳第七十五 酷吏}

\begin{pinyinscope}

 於
 洛侯胡泥李洪之子神張赦提趙霸崔暹邸珍田式燕榮元弘嗣王文同夫為國之體有四焉:一曰仁義,二曰禮制,三曰法令,四曰刑罰。仁義、禮制,教之本也;法令、刑罰,教之末也。無本不立,無末不成。然教化遠而刑罰近,可以助化而不可
 以專行,可以立威而不可以繁用。老子曰:「其政察察,其人缺缺。」



 又曰:「法令滋章,盜賊多有。」然則,令之煩苛,吏之嚴酷,不可致化,百世可知。考覽前載,有時而用之矣。



 昔秦任獄吏,赭衣滿道。漢革其風,矯枉過正,禁綱疏闊,遂漏吞舟。故大姦巨猾,犯義悖禮。郅都、寧成之倫,猛氣奮發,摧拉凶邪,一切以救時弊。雖乖教義,或有所取焉。于洛侯之徒,前書編之《酷吏》。或因餘緒,或以微功,遭遇時來,忝竊高位。肆其褊性,多行無禮,君子小人,咸罹其毒。凡所蒞職,莫不懍然。



 居其下者,視之如蛇虺;過其境者,逃之如寇仇。與人之恩,心非好善;加人之罪,事非疾惡。
 其所笞辱,多在無辜。察其所為,豺狼之不若也。其禁姦除猾,殆與郅、寧之倫異乎。君子賤之,故編於《酷吏》。



 魏有於洛侯、胡泥、李洪之、高遵、張赦提、羊祉、崔暹、酈道元、谷楷。齊有邸珍、宋游道、盧斐、畢義雲。《周書》不立此篇。《隋書》有庫狄士文、田式、燕榮、趙仲卿、崔弘度、元弘嗣、王文同。今檢高遵、羊祉、酈道元、谷楷、宋游道、盧斐、畢義雲、庫狄士文、趙仲卿、崔弘度各從其家傳,其餘並列於此云。



 於洛侯,代人也。為秦州刺史,貪酷安忍。部人富熾奪人呂勝脛纏一具,洛侯輒鞭富熾一百,截其右腕。百姓王隴客刺殺人王羌奴、王愈二人,依律罪死。而洛侯生拔
 隴客舌,刺其本,并刺胸腹二十餘瘡。隴客不堪苦痛,隨刀戰動。乃立四柱,磔其手足。命將絕,始斬其首,支解四體,分懸道路。見者無不傷楚歎愕。百姓王元壽等一時反叛。有司糾劾,孝文詔使者於州常刑人處,宣告兵人,然後斬洛侯以謝百姓。



 胡泥,代人也。歷官至司衛監,賜爵永成侯。泥率勒禁中,不憚豪貴。殿中尚書叔孫侯頭應內直而闕於一時,泥以法繩之。侯頭恃寵,遂與口諍。孝文聞而嘉焉,賜泥衣服一襲,出為幽州刺史,假范陽。以北平陽尼碩學,遂表薦之。轉為定州刺史。以暴虐,刑罰酷濫,受納貨賄,徽還
 戮之。將就法,孝文臨太華殿引見,遣侍臣宣詔責之,遂就家賜盡。



 李洪之,本名文通,恒農人也。少為沙門,晚乃還俗。真君中,為狄道護軍,賜爵安陽男。會永昌王仁隨太武南征,得元后姊妹二人,洪之潛相餉遺,結為兄弟,遂便如親。頗得元后在南兄弟名字,乃改名洪之。及仁坐事誅,元后入宮,得幸於文成,生獻文。元后臨崩,太后問其親,因言洪之為兄。與相訣經日,具條列南方諸兄珍之等,手以付洪之。遂號為獻文親舅。太安中,珍之等兄弟至都,與洪之相見,敘元后平生故事,計長幼為昆季。以外戚
 為河內太守,進爵任城侯,威儀一同刺史。河內北連上黨,南接武牢,地險人悍,數為劫害,長吏不能禁。洪之至郡,嚴設科防,募斬賊者,便加重賞,勤勸務本,盜賊止息。誅鋤姦黨,過為酷虐。後為懷州刺史,封漢郡公,徵拜內都大官。河西羌胡領部落反叛,獻文親征,命洪之與侍中、東郡王陸定總統諸軍。輿駕至并州,詔洪之為河西都將,討山胡。皆保險距戰,洪之築壘於石樓南白雞原以對之。時諸將悉欲進攻,洪之乃開以大信,聽其復業。胡人遂降。獻文嘉之。遷拜尚書、外都大官。



 後為使持節、安南將軍、秦、益二州刺史。至任,設禁姦之制。有帶刃行
 者,罪與劫同。輕重品格,各有條章。於是大饗州中豪傑長老,示之法制。乃夜密遣騎分部覆諸要路,有犯禁者,輒捉送州,宣告斬決。其中枉見殺害者,至有百數。赤葩渴郎羌深居山谷,雖相羈縻,王人罕到。洪之芟山為道,廣十餘步,示以軍行之勢。乃興軍臨其境,山人驚駭。洪之將數十騎至其里閭,撫其妻子,問所疾苦,因資遺之。眾羌喜悅,求編課調,所入十倍於常。洪之善御戎夷,頗有威惠,而刻害之聲,聞於朝野。



 初,洪之微時妻張氏,亦聰強婦人,自貧賤至富貴,多所補益,有男女幾十人。



 洪之後得劉芳從姊,重之,疏張氏。亦多所產育。為兩宅別
 居,偏厚劉室,由是二妻妒競,兩宅母子,往來如仇。及蒞西州,以劉自隨。



 洪之素非廉清,每有受納。時孝文始建祿制,法禁嚴峻,遂鎖洪之赴京,親臨太華,庭集群臣數之。以其大臣,聽在家自裁。洪之志性慷慨,多所堪忍。疹病炙療,艾炷圍將二寸,首足十餘處,一時俱下,言笑自若,接賓不輟。及臨盡,沐浴衣幍,防卒扶持,出入遍巡家庭,如是再三,泣歎良久,乃臥而引藥。



 始洪之託為元后兄,公私自同外戚。至此罪後,孝文乃稍對百官辯其誣假。而諸李猶善相視,恩紀如親。洪之始見元后,計年為兄。及珍之等至,洪之以元后素定長幼,其呼拜坐,皆如
 家人。暮年,數延攜之宴飲。醉酣之後,時或言及本末,洪之則起而加敬,笑語自若。富貴赫奕,舅戚之家。遂棄宗,專附珍之等。後頗存振本屬,而猶不顯然。劉氏四子。



 長子神,少有膽略,以氣尚為名。以軍功封長樂縣男,累遷平東將軍、太中大夫。孝昌中,行相州事,尋正加撫軍。葛榮盡銳攻之,久不能剋。會葛榮見禽,以功進爵為公。元顥入洛,莊帝北巡,以神為侍中。又除殿中尚書,仍行相州事。車駕還宮,改封安康郡公。普泰元年,進驃騎大將軍、儀同三司、相州大中正。薨,贈司徒公,冀州刺史。子士。齊受禪,例降。



 張赦提,中山安喜人也。性雄武,有規畫。初為武卉中郎。時京畿盜魁,首稱豹子、彪子,並善弓馬,於靈丘、應門間聚為劫害。至乃斬人首,射其口,刺人臍,引腸繞樹而共射之,以為戲笑。其暴酷如此。軍騎掩捕,久弗能獲,行者患焉。赦提為逐賊軍將,未幾而獲彪子、豹子及其黨與,盡送京師,斬於闕下,自是清靜。



 其靈丘羅思祖,宗門豪溢,家處隘險,多止亡命,與之為劫。獻文怒之,孥戮其家。



 而思祖家黨,相率寇盜。赦提募求捕逐。以赦提為遊徼軍將,前後擒獲,殺之略盡。



 因此,濫有屠害,尤為忍酷。既資前稱,又藉此功,除幽州刺史,假安喜侯。赦提克己厲
 約,遂有清稱。後頗縱妻段氏,多有受納。命僧尼因事通請,貪虐流聞。中散李真香出使幽州,採訪牧守政績。真香驗案其罪,赦提懼死欲逃。其妻姑為太尉、東陽王丕妻,恃丕親貴,自許詣丕申訴求助,謂赦提曰:「當為訴理,幸得申雪,願寬憂,不為異計。」赦提以此,差自解慰。段乃陳列:真香昔嘗因假而過幽州,知赦提有好牛,從索不果。令臺使止挾前事,故威逼部下,拷楚過極,橫以無辜,證成誣罪。執事恐有不盡,使駕部令趙秦州重往究訊,事狀如前,處赦提大辟。孝文詔賜死於第。將就盡,命妻而責之曰:「貪濁穢吾者卿也,又安吾而不得免禍,九泉
 之下,當為仇仇矣。」



 又有華山太守趙霸,酷暴非理。大使崔光奏霸云:「不遵憲度,威虐任情,至乃手擊吏人,僚屬奔走,不可以君人字下,納之軌物。輒禁止在州。」詔免所居官。



 崔暹,字元欽,本云清河東武城人也,世家于滎陽、潁川之間。性猛酷,少仁恕,姦猾好利,能事勢家。初以秀才累遷南袞州刺史,盜用官瓦,贓污狼籍,為御史中尉李平所糾,免官。後行豫州事,尋即真。遣子析戶,分隸三縣,廣占田宅,藏匿官奴,障吝陂葦,侵盜公私,為御史中尉王顯所彈,免官。後累遷瀛州刺史。



 貪暴安忍,人庶患之。嘗
 出獵州北,單騎至人村,有汲水婦人,暹令飲馬,因問曰:「崔瀛州何如?」婦人不知是暹,答曰:「百姓何罪!得如此癩兒刺史。」暹默然而去。以不稱職,被解還京。武川鎮反,詔暹為都督,李崇討之。違崇節度,為賊所敗,單騎潛還。禁於廷尉,以女妓園田貨元叉獲免。建義初,遇害於河陰。贈司徒公、冀州刺史,追封武津縣公。



 子瓚,字祖珍,位兼尚書左丞,卒。瓚妻,莊帝姊也,後封襄城長公主,故特贈瓚冀州刺史。子茂,字祖昂,襲祖爵。



 邸珍,字安寶,本中山上曲陽人也,魏太和中,徙居武州鎮。孝昌中,六鎮兵起,珍遂從杜洛周賊。洛周為葛榮所
 吞,珍入榮軍。榮為爾朱榮所破,珍與其餘黨,俱徙并州。從齊神武出山東。神武起義信都,拜珍長史,封上曲縣侯,除殷州刺史。



 珍求取無厭,大為州人所疾苦。後兼尚書右僕射、大行臺,節度諸軍事,擊梁州將成景攜等,解東行圍,回軍彭城。珍御下殘酷,士眾離心,至於土人豪族,遇之無禮,遂為州人所害。後贈定州刺史、司空公。



 田式,字顯標,馮翊下邽人也。祖安興、父長樂,仕魏,俱為本郡太守。式性剛果,多武藝,拳勇絕人。仕周,位渭南太守,政尚嚴猛,吏人重足而立,無敢違法。遷本郡太守,親故屏跡,請託不行。周武帝聞而善之,進位儀同三司,賜
 爵信都縣公,擢拜延州刺史。從平齊,以功授上開府,徙為建州刺史,改封梁泉縣公。



 後從韋孝寬討尉遲迥,以功拜大將軍,進爵武山郡公。及隋文帝受禪,拜襄州總管。



 專以立威為務,每視事於外,必盛氣以待之。其下官屬,股慄無敢仰視。有犯禁者,雖至親暱,無所容貸。其女婿京兆杜寧自長安省之,式誡寧無出外。寧久之不得還,竊上北樓,以暢羈思。式知之,杖寧五十。其所愛奴,嘗詣式白事,有蟲上其衣衿,揮袖拂去之,式以為慢己,立棒殺之。或僚吏姦贓,部內劫盜者,無問輕重,悉禁地阱中,寢處糞穢,令受苦毒。自非身死,終不得出。每赦書到
 州,式未暇省讀,先召獄卒殺重囚,然後宣示百姓。其刻暴如此。由是為上所譴,除名。式慚恚不食,妻子至其所輒怒,唯侍僮二人,給使左右。從家中索椒,欲自殺,家人不與。陰遣侍僮詣市買毒藥,妻子又奪棄之。式恚臥,其子信時為儀同,至式前流涕曰:「大人既是朝廷重臣,又無大過,比見公卿放辱者多矣,旋復外用,大人何能久乎?乃至於此!」式欻起抽刀斫信,信避之,刃中於門。上知之,以式為罪己之深,復其官爵,尋拜廣州總管,卒官。



 燕榮,字貴公,華陰弘農人也。父侃,周大將軍榮性剛嚴,有武藝。仕周,為內侍上士。從武帝伐齊,以功授開府儀
 同三司,封高邑縣公。隋文帝受禪,進位大將軍,進封落叢郡公,拜晉州刺史。尋從河間王弘擊突厥,以功拜上柱國,遷青州總管。在州,選絕有力者為伍伯。吏人過之者,必加詰問,輒楚撻之,創多見骨。



 姦盜屏跡,境內肅然。他州縣人經其界者,畏若寇仇,不敢休息。後因入朝覲,特加恩遇。榮以母老,請每歲入朝,上許之。



 伐陳之役,以為行軍總管,率水軍自東萊傍海入太湖,取吳郡。既破丹陽,吳人共立蕭瓛,為宇文述所敗,退保包山。榮率精甲躡之,瓛敗走,為榮所執。事平,檢校揚州總管。尋徵為武候將軍,後除幽州總管。



 榮性嚴酷,有威容,長吏見者,
 莫不惶懼自失。范陽盧氏,世為著姓,榮皆署為吏卒,以屈辱之。鞭笞左右,動至千數,流血盈前,飲啖自若。嘗按部,道次見叢荊,堪為笞箠,命取之,輒以試人。人或自陳無咎,榮曰:』後有罪,當免。」



 及後犯細過,將撾之,人曰:「前日被杖,許有罪宥之。」榮曰:「無過尚爾,況有過邪!」榜捶如舊。榮每巡省管內,聞人吏妻有美色,輒舍其室而淫之,貪暴放縱日甚。時元弘嗣除幽州長史,懼辱,固辭。上知之,敕榮曰:「弘嗣杖十已上罪,皆奏聞。」榮忿曰:「豎子何敢弄我!」及遣弘嗣監納倉粟,颺得一糠一秕,罰之,每笞不滿十,然一日中或至三數。如是歷年,怨隙日構。榮遂收付
 獄,禁絕其糧。



 弘嗣饑,抽衣絮雜水咽之。其妻詣闕稱冤,上遣考功侍郎劉士龍馳驛鞫問,奏榮毒虐,又贓穢狼籍,遂征還京,賜死。先是,榮家寢室無故有蛆數斛從地墳出。未幾,榮死於蛆出之處。有子詢。



 元弘嗣,河南洛陽人也。祖剛,魏漁陽王。父經,周漁陽郡公。弘嗣少襲爵,十八為左親衛。開皇元年,從晉王平陳,以功授上儀同。後除觀州長史,以嚴峻任事,州人多怨之。轉幽州。時總管燕榮肆虐於弘嗣,每笞辱。弘嗣心不伏,遂被禁。



 及榮誅,弘嗣為政,酷又甚之。每鞫囚,多以酢灌鼻,或椓弋其下竅。無敢隱情,姦偽屏息。仁壽末,授木
 工監,修營東都。大業初,煬帝潛有遼東意,遣弘嗣於東萊海口監造船。諸州役丁苦其捶楚,官人當作,晝夜立水中,略不敢息,自腰已下無不蛆生,死者十三四。尋遷黃門侍郎,轉殿中少監。遼東之役,進位金紫光祿大夫。後奴賊寇隴西,詔弘嗣擊之。及玄感反,弘嗣屯兵安定。或告之謀應玄感,代王侑遣執送行在所。以無反釋。帝疑之,除名徙日南,道死。有子仁觀。



 王文同,京兆頻陽人也。性明辯,有乾用。開皇中,以軍功拜儀同,授桂州司馬。煬帝嗣位,為光祿少卿。以忤旨,出為恒山郡贊務。有一人豪猾,每持長吏長短,前後守令
 咸憚之。文同下車,聞其名而數之。因令剡木為大橛,埋之於庭,出尺餘,四面各埋小橛,令其人踣心於木橛上,縛四支於小橛,以棒打其背,應時潰爛。郡中大駭,吏人懾氣。及帝征遼東,令文同巡察河北諸郡,文同見沙門齋戒菜食者,以為妖妄,皆收繫之。北至河間,召郡官人。小有遲違者,輒覆面於地而捶殺之。求沙門相聚講論及長老共為佛會者數百人,文同以為聚結惑眾,盡斬之。又悉裸僧尼,驗有淫狀非童男女者數千人,復將殺之。郡中士女,號哭於路,諸郡驚駭,各奏其事。帝聞大怒,遣使者違奚善意馳鎖之,斬於河間,以謝百姓。仇人剖
 其棺,臠其肉啖之,斯須咸盡。



 論曰:士之立名,其途不一,或以循良進,或以嚴酷顯。故寬猛相資,德刑互設。然不嚴而化,君子所先。於洛侯等為惡不同,同歸於酷,肆其毒螫,多行殘忍。



 賤人肌膚,同諸木石;輕人性命,甚於芻狗。長惡不悛,鮮有不及。故或身嬰罪戮,或憂恚俱殞,異術皆斃,各其宜焉。凡百君子,以為有天道矣。



\end{pinyinscope}