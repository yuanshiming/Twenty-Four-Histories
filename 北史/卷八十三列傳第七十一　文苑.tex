\article{卷八十三列傳第七十一 文苑}

\begin{pinyinscope}

 溫
 子
 昇荀濟祖鴻勳李廣樊遜荀士遜王褒庾信顏之推弟之儀虞世基柳許善心李文博明克讓劉臻諸葛潁王貞虞綽王胄庾自直潘徽常德志尹式劉善經
 祖君彥孔德紹劉斌《易》曰:「觀乎天文,以察時變;觀乎人文,以化成天下。」然則文之為用其大矣哉!逖聽三古,彌綸百代,若乃《墳》、《素》所紀,靡得而云;《典》、《謨》已降,遺風可述。至於制禮作樂,騰實飛聲,善乎。言之不文,行之豈能遠也。是以曲阜之多才多藝,監二代以正其源;闕里之性與天道,修《六經》以維其末。用能窮神知化,稱首於千古;經邦緯俗,藏用於百代。至哉,斯固聖人之述作也。逮乎兩周道喪,七十義乖。淹中、稷下,八儒、三墨之異,漆園、黍谷,名、法、兵、農之別,雖雅誥奧義,或未盡善,考其遺跡,亦賢達之流乎。其離讒放逐之臣,塗窮後門之士,道感軻而未遇,志鬱抑而不
 申。憤激委約之中,飛文魏闕之下,奮迅泥滓,自致青雲,振沈溺於一朝,流風聲於千載者往往而有矣。



 漢自孝武之後,雅尚斯文,揚葩振藻者如林,而二馬、王、楊為之傑。東京之朝,茲道逾扇,咀徵含商者成市,而班、傅、張、蔡為之雄。當塗受命,尤好蟲篆;金行勃興,無替前烈。曹、王、陳、阮負宏衍之思,挺棟幹於鄧林;潘、陸、張、左擅侈麗之才,飾羽儀於鳳穴。斯並高視當世,連衡孔門。雖時運推移,質文屢變,譬猶六代並奏,易俗之用無爽;九源競逐,一致之理同歸。歷選前英,於斯為盛。



 既而中州板蕩,戎狄交侵,僭偽相屬,生靈塗炭,故文章黜焉。其能潛思於
 戰爭之間,揮翰於鋒鏑之下,亦有時而間出矣。若乃魯徵、杜廣、徐光、尹弼之儔,知名於二趙;宋該、封弈、朱彤、梁讜之屬,見重於燕、秦。然皆迫於倉卒,牽於戰陣,章奏符檄,則粲然可觀;體物緣情,則寂寥於世。非其才有優劣,時運然也。至於朔方之地,蕞爾夷俗,胡義周之頌國都,足稱宏麗。區區河右,而學者埒於中原,劉延明之銘酒泉,可謂清典。子曰:「十室之邑,必有忠信。」豈徒言哉。



 洎乎有魏,定鼎沙朔。南包河、淮,西吞關、隴。當時之士,有許謙、崔宏、宏子浩、高允、高閭、游雅等,先後之間,聲實俱茂,詞義典正,有永嘉之遺烈焉。



 及太和在運,銳情文學,固以
 頡頏漢徹,跨躡曹丕,氣韻高遠,艷藻獨構。衣冠仰止,咸慕新風,律調頗殊,曲度遂改。辭罕泉源,言多胸臆,潤古彫今,有所未遇。



 是故雅言麗則之奇,綺合繡聯之美,眇歷歲年,未聞獨得。既而陳郡袁翻、河內常景,晚拔疇類,稍革其風。及明皇御歷,文雅大盛,學者如牛毛,成者如麟角。孔子曰:「才難。」不其然也?于時陳郡袁翻、翻弟躍、河東裴敬憲、弟莊伯、莊伯族弟伯茂、范陽盧觀、弟仲宣、頓丘李諧、勃海高肅、河間邢臧、趙國李騫,彫琢瓊瑤,刻削杞梓,並為龍光,俱稱鴻翼。樂安孫彥舉、濟陰溫子昇,並自孤寒,鬱然特起。咸能綜採繁縟,興屬清華。比於建安
 之徐、陳、應、劉,元元之潘、張、左、束,各一時也。



 有齊自霸業云啟,廣延髦俊,開四門以賓之,頓八紘以掩之。鄴都之下,煙霏霧集。河間邢子才、鉅鹿魏伯起、范陽盧元明、鉅鹿魏季景、清河崔長儒、河間邢子明、范陽祖孝徵、中山杜輔玄、北平陽子烈並其流也。復有范陽祖鴻勳,亦參文士之列。及天保中,李愔、陸仰、崔瞻、陸元規並在中書,參掌綸誥。其李廣、樊遜、李德林、盧詢祖、盧思道始以文章著名。皇建之朝,常侍王晞獨擅其美。河清、天統之辰,杜臺卿、劉逖、魏騫亦參詔敕。自李愔已下,在省唯撰述除官詔旨,其關涉軍國文翰,多是魏收作之。及在武平,
 李若、荀士遜、李德林、薛道衡並為中書侍郎,典司綸綍。



 後主雖溺於群小,然頗好詠詩,幼時嘗讀詩賦,語人云:「終有解作此理不?」



 初因畫屏風,敕通直郎蕭放及晉陵王孝式錄古賢烈士及近代輕艷諸詩以充圖畫,帝彌重之。後復追齊州錄事參軍蕭愨、趙州功曹參軍顏之推同入撰錄,猶依霸朝,謂之館客。放及之推意欲更廣其事,又因祖珽輔政,愛重之推,又託鄧長顒漸說後主,屬意斯文。三年,祖珽奏立文林館,於是更召引文學士,謂之待詔文林館焉。珽又奏撰《御覽》,詔珽及特進魏收、太子太師徐之才、中書令崔劼、散騎常侍張凋、中書監
 陽休之監撰。珽等奏追通直散騎侍郎韋道遜、陸乂、太子舍人王劭、衛尉丞李孝基、殿中侍御史魏澹、中散大夫劉仲威、袁奭、國子博士朱才、奉車都尉眭道閑、考功郎中崔子樞、左外兵郎薛道衡、并省主客郎中盧思道、司空東閣祭酒崔德立、太傅行參軍崔儦、太學博士諸葛漢、奉朝請鄭公超、殿中侍御史鄭子信等入館撰書,並敕放、愨、之推等同入撰例。復命散騎常侍封孝琰、前樂陵太守鄭元禮、衛尉少卿杜臺卿、通直散騎常侍楊訓、前南兗州長史羊肅、通直散騎侍郎馬元熙、并省三公郎中劉氏、開府行參軍李師上、溫君悠入館,亦令撰
 書。後復命特進崔季舒、前仁州刺史劉逖、散騎常侍李孝貞、中書侍郎李德林續入待詔。尋又詔諸人各舉所知。又有前濟州長史李翥、前廣武太守魏謇、前西兗州司馬蕭溉、前幽州長史陸仁惠、鄭州司馬江旰、前通直散騎侍郎辛德源、陸開明、通直郎封孝騫、太尉掾張德沖、并省右戶郎元行恭、司徒戶曹參軍古道子、前司空功曹參軍劉顗、獲嘉令崔德儒、給事中李元楷、晉州中從事陽師孝、太尉中兵參軍劉儒行、司空祭酒陽辟疆、司公士曹參軍盧公順、司空中兵參軍周子深、開府行參軍王友伯、崔君洽、魏師謇並入館待詔。又敕僕射段
 孝言亦入焉。《御覽》成後,所撰錄人亦有不得待詔,付所司處分者。凡此諸人,亦有文學膚淺,附會親識,妄相推薦者十三四焉。雖然,當時操筆之徒,搜求略盡。其外如廣平宋孝王、信都劉善經輩三數人,論其才性,入館諸賢亦十三四不逮之。



 周氏創業,運屬陵夷,纂遺文於既喪,聘奇士如弗及。是以蘇亮、蘇綽、盧柔、唐瑾、元偉、李昶之徒,咸奮鱗翼,自致青紫。然綽之建言,務存質朴,遂糠秕魏、晉,憲章虞、夏,雖屬辭有師古之美,矯枉非適時之用,故莫能常行焉。既而革車電邁,渚宮雲撤,梁、荊之風,扇於關右,狂簡之徒,斐然成俗,流宕忘反,無所取裁。



 夫
 人有六情,稟五常之秀;情感六氣,順四時之序。蓋文之所起,情發於中。



 而自漢、魏以來,迄乎晉、宋,其體屢變,前哲論之詳矣。暨永明、天監之際,太和、天保之間,洛陽、江左,文雅尤盛,彼此好尚,互有異同。江左宮商發越,貴於清綺;河朔詞義貞剛,重乎氣質。氣質則理勝其詞,清綺則文過其意。理深者便於時用,文華者宜於詠歌。此其南北詞人得失之大較也。若能掇彼清音,簡茲累句,各去所短,合其兩長,則文質彬彬,盡美盡善矣。



 梁自大同之後,雅道淪缺,漸乖典則,爭馳新巧。簡文、湘東啟其淫放,徐陵、庾信分路揚鑣。其意淺而繁,其文匿而彩,詞尚
 輕險,情多哀思,格以延陵之聽,蓋亦亡國之音也。



 隋文初統萬機,每念斲凋為樸,發號施令,咸去浮華。然時俗詞藻、猶多淫麗;故憲臺執法,屢飛霜簡。煬帝初習藝文,有非輕側,暨乎即位,一變其體。《與越公書》、《建東都詔》、《冬至受朝詩》及《擬飲馬長城窟》,並存雅體,歸於典制,雖意在驕淫,而詞無浮蕩。故當時綴文之士,遂得依而取正焉。所謂能言者未必能行,蓋亦君子不以人廢言也。


爰自東帝歸秦,逮乎青蓋入洛,四隩咸暨,九州攸同。江、漢英靈,燕、趙奇俊,並該天綱之中,俱為大國之寶。言刈其楚,片善無遺,潤水圓流,不能十數,才之難也,不其然乎。
 時之文人,見稱當世者,則齊人范陽盧思道、安平李德林、河東薛道衡、趙郡李元操、鉅鹿魏澹,陳人會稽虞世基、河東柳
 \gezhu{
  巧言}
 、高陽許善心等,或鷹揚河朔,或獨步漢南,俱騁龍光,並驅雲路矣。



 《魏書》序袁躍、裴敬憲、盧觀、封肅、邢臧、裴伯茂、邢昕、溫子升為《文苑傳》,今唯取子昇,其餘並各附其家傳。《齊書》敘祖鴻勳、李廣、樊遜、劉逖、荀士遜、顏之推為《文苑傳》,今唯取祖、李、樊、荀,其餘亦各附其家傳。《周書》不立此傳,今取王褒、庾信列於此篇。顏之推竟從齊入周,故列在王、庾之下。


顏之儀既之推之弟,故列在之推之末。《隋書》序劉臻、崔儦、王頍、諸葛潁、王貞、孫萬
 壽、虞綽、王胄、庾自直、潘徽為《文學傳》,今檢崔儦、王頍、孫萬壽各從其家傳,其餘編之此篇,并取虞世基、許善心、柳
 \gezhu{
  巧言}
 、明克讓冠之於此,以備《文苑傳》云。



 溫子昇,字鵬舉,自云太原人,晉大將軍嶠之後也。世居江左。祖恭之,宋彭城王義康戶曹,避難歸魏,家于濟陰冤句,因為其郡縣人焉。父暉,兗州左將軍長史,行濟陰郡事。



 子昇初受學於崔靈恩、劉蘭。精勤,以夜繼晝,晝夜不倦。長乃博覽百家,文章清婉。為廣陽王深賤客,在馬坊教諸奴子書。作《侯山祠堂碑文》,常景見而善之,故詣深謝之。景曰:「頃見溫生。」深怪問之。景曰:「溫生是大才士。」
 深由是稍知之。



 熙平初,中尉、東平王匡博召辭人以充御史。同時射策者八百餘人,子昇與盧仲宣、孫搴等二十四人為高第。於是預選者爭相引決,匡使子昇當之,皆受屈而去。



 搴謂人曰:「朝來靡旗亂轍者,皆子昇逐北。」遂補御史,時年二十二。臺中彈文皆委焉。以憂去任。服闋,還為朝請。後李神俊行荊州事,引兼錄事參軍。被徽赴省,神俊表留不遣。吏部郎中李獎退表不許,曰:「昔伯瑜之不應留,王朗所以發歎。宜速遣赴,無踵彥雲前失。」於是還省。及廣陽王深為東北道行臺,召為郎中。



 黃門郎徐紇受四方表啟,答之敏速,於深獨沈思,曰:「彼有溫
 郎中,才藻可畏。」



 高車破走,珍寶盈滿,子昇取絹四十疋。深軍敗,子昇為葛榮所得。榮下都督和洛興與子昇舊識,以數十騎潛送子昇,得達冀州。還京,李楷執其手曰:「卿今得免,足使夷甫慚德。」自是無復宦情,閉門讀書,厲精不已。



 及孝莊即位,以子升為南主客郎中,修起居注。曾一日不直,上黨王天穆時錄尚書事,將加捶撻,子昇遂逃遁。天穆甚怒,奏人代之。莊帝曰:「當世才子不過數人,豈容為此便相放黜?」乃寢其奏。及天穆將討邢杲,召子昇同行,子昇未敢應。天穆謂人曰:「吾欲收其才用,豈懷前忿也?今復不來,便須南走越,北走胡耳!」子昇不得
 已而見之。加伏波將軍。為行臺郎中。天穆深知賞之。元顥入洛,天穆召子升問曰:「即欲向京師?為隨我北度?」對曰:「主上以武牢失守,致此狼狽。元顥新入,人情未安,今往討之,必有征無戰。王若剋復京師,奉迎大駕,桓、文之舉也。捨此北度,竊為大王惜之。」天穆善之而不能用,遣子昇還洛,顥以為中書舍人。莊帝還宮,為顥任使者多被廢黜,而子昇復為舍人。天穆每謂子升曰:「恨不用卿前計。」除正員郎,仍舍人。及帝殺爾朱榮也,子升預謀,當時赦詔,子昇詞也。榮入內,遇子昇把詔書,問:「是何文字?」子升顏色不變,曰:「敕。」榮不視之。爾朱兆入洛,子昇懼禍
 逃匿。



 永熙中為侍讀,兼舍人、鎮南將軍、金紫光祿大夫。遷散騎常侍、中軍大將軍,後領本州大中正。梁使張皋寫子昇文筆,傳於江外,梁武稱之曰:「曹植、陸機復生於北土,恨我辭人,數窮百六。」陽夏守傅摽使吐谷渾,見其國主床頭有書數卷,乃是子升文也。濟陰王暉業嘗云:「江左文人,宋有顏延之、謝靈運,梁有沈約、任昉,我子昇足以陵顏轢謝,含任吐沈。」楊遵彥作《文德論》,以為古今辭人皆負才遺行,澆薄險忌,唯邢子才、王元景、溫子升彬彬有德素。



 齊文襄引子昇為大將軍諮議。子升前為中書郎,嘗詣梁客館受國書,自以不修容止,謂人曰:「詩
 章易作,逋峭難為。」文襄館客元僅曰:「諸人當賀,推子昇合陳辭。」子升久忸怩,乃推陸操焉。及元僅、劉思逸、荀濟等作亂,文襄疑子昇知其謀。方使之作《神武碑》。文既成,乃餓諸晉陽獄,食弊襦而死。棄屍路隅,沒其家口。太尉長史宋游道收葬之,又為集其文筆為三十五卷。



 子昇外恬靜,與物無競,言有準的,不妄毀譽。而內深險,事故之際,好豫其間,所以終致禍敗。又撰《永安記》三卷。無子。



 弟子盛,州主簿,有文才,年二十餘卒。



 荀濟,字子通。其先潁川人,世居江左。濟初與梁武帝布衣交。知梁武當王,然負氣不服,謂人曰:「會楯上磨墨作
 檄文。」或稱其才於梁武,梁武曰:「此人好亂者也。」濟又上書譏佛法,言營費太甚。梁武將誅之,遂奔魏,館于崔甗家。



 及是見執。楊愔音謂曰:「遲暮何為然?」濟曰:「叱叱,氣耳,何關遲暮!」乃下辯曰:「自傷年幾摧頹,恐功名不立。舍兒女之情,起風雲之事,故挾天子,誅權臣。」齊文襄惜其才,將不殺,親謂曰:「荀公何意反?」濟曰:「奉詔誅將軍高澄,何為反!」於是燔殺之。鄴下士大夫多傳濟音韻。



 祖鴻勳,涿郡范陽人也。父慎,仕魏,歷鴈門、咸陽二郡太守,政有能名。卒於金紫光祿大夫、贈中書監、幽州刺史,謚惠侯。鴻勳弱冠,與同郡盧文符並為州主簿。僕射、臨
 淮王彧表薦其文學,除奉朝請。人曰:「臨淮舉卿,竟不相謝,恐非其宜。」鴻勳曰:「為國舉才,臨淮之務,祖鴻勳何事從而識之。」彧聞而喜曰:「吾得其人矣。」後咸陽王徽奏鴻勳為司徒法曹參軍事。及赴洛,徽謂曰:「臨淮相舉,竟不到門,今來何也?」鴻勳曰:「今來赴職,非為謝恩。」轉廷尉正,去官歸鄉里。齊神武嘗徽至并州,作《晉祠記》,好事者玩其文。位至高陽太守。在官清素,妻子不免寒餒。時議高之。齊天保初,卒官。



 李廣,字弘基,范陽人也。其先自遼東徙焉。廣博涉群書,有才思。少與趙郡李謇齊名,為邢、魏之亞,而訥於言,敏
 於行。中尉崔暹精選御史,皆是世胃,廣獨以才學兼侍御史,修國史。南臺文奏,多其辭也。齊文宣初嗣霸業,命掌書記。



 天保初,欲以為中書郎,遇其病篤而止。廣嘗欲早朝,假寐,忽驚覺,謂其妻曰:「吾向似睡非睡,忽見一人出吾身中,語云:『君用心過苦,非精神所堪,今辭君去。』」因而恍忽不樂,數日便遇疾,積年不起。廣雅有鑒識,度量弘遠,坦率無私,為士流所愛,時共贍遺之,賴以自給。竟以疾終。嘗薦畢義雲於崔暹。廣卒後,義雲集其文筆七卷,託魏收為之序。



 樊遜,字孝謙,河東北猗氏人也。祖琰、父衡,並無官宦。而
 衡性至孝,喪父,負土成墳,植柏方數十畝,朝夕號慕。遜少好學。其兄仲以造氈為業,亦常優饒之。



 遜自責曰:「為人弟,獨愛安逸,可不愧於心乎!』欲同勤事業。母馮氏謂曰:「汝欲謹小行邪?」遜感母言,遂專心典籍,恆書壁作「見賢思齊」四字以自勸。



 遜貌醜陋,有才氣。屬本州淪陷,寓居鄴中,為臨漳小吏。縣令裴鑒蒞官清苦,致白雀等瑞。遜上《清德頌》十首,鑒大加賞重,擢為主簿。仍薦之於右僕射崔暹,與遼東李廣、勃海封孝琰等為暹賓客。人有譏其靜默不能趨時者。遜常服東方朔之言:「陸沈世俗,避世金馬」,遂借陸沈公子為主人,擬《客難》制《客誨》以自
 廣。後崔暹大會客,大司馬、襄城王旭時亦在坐,欲命府僚。暹指遜曰:「此人學富才高,兼之佳行,可為王參軍也。」旭目之曰:「豈能就耶?」遜曰:「家無蔭第,不敢當此。」武定七年,齊文襄崩,暹為文宣徙於邊,賓客咸散,遜遂徙居陳留。梁州刺史劉殺鬼以遜兼錄事參軍事。遜仍舉秀才。尚書案舊令,下州三載一舉秀才,為三年已貢開封人鄭祖獻,計至此年未合。兼別駕王聰抗辭爭議,右丞陽斐不能卻。尚書令高隆之曰:「雖遜才學優異,待明年非遠。」遜竟還本州。天保元年,本州復召舉秀才。三年春,會朝堂對策。策罷,中書郎張子融奏入。至四年五月,遜與
 定州秀才李子宣等對策三年不調,被付外。上書請從罷,詔不報。梁州重舉遜為秀才。五年正月,制詔問焉。尚書擢第,以遜為當時第一。十二月,清河王岳為大行臺,率眾南討,以遜從軍。明年,文宣納梁貞陽侯蕭明為梁主,岳假遜大行臺郎中,使於江南,與蕭修、侯瑱和解。遜往還五日,得修等報書,岳因與修盟於江上。大軍還鄴,遜仍被都官尚書崔昂舉薦。詔付尚書,考為清平勤幹,送吏部。



 七年,詔令校定群書,供皇太子。遜與冀州秀才高乾和,瀛州秀才馬敬德、許散愁、韓同寶,洛州秀才傅懷德,懷州秀才古道子,廣平郡孝廉李漢子,勃海郡
 孝廉鮑長暄,陽平郡孝廉景孫,前梁州府主簿王九元、前開府水曹參軍周子深等十一人同被尚書召共刊定。時秘府書籍紕繆者多,遜乃議曰:「案漢中壘校尉劉向受詔校書,每一書竟,表上,輒言臣向書、長水校尉臣參書、太常博士書、中外書合若干本,以相比校,然後殺青。今所仇校,供擬極重,出自蘭臺,御諸甲館。向之故事,見存府閣。即欲刊定,必藉眾本。太常卿邢子才、太子少傅魏收、吏部尚書辛術、司農少卿穆子容、前黃門郎司馬子瑞、故國子祭酒李業興並是多書之家,請牒借本參校。」秘書監尉瑾移尚書都坐,凡所得別本三千餘卷。《
 五經》諸史殆無遺闕。



 於時魏收作《庫狄乾碑序》,令孝謙為之銘,陸仰不知,以為收合作也。陸操、伏渾卒,楊愔使孝謙代己作書以告晉陽朝士,令魏潤色之,收不能改一字。八年,減東西二省官,更定選,員不過三百,參者二三千人。楊愔言於眾曰:「後生清俊,莫過盧思道;文章成就,莫過樊孝謙;几案斷割,莫過崔成之。」遂以思道長兼員外郎,三人並員外將軍。孝謙辭曰:「門族寒陋,訪第必不成,乞補員外司馬督。」



 愔曰:「才高不依常例。」特奏用之。



 清河初,為主書,參典詔策。天統元年,加員外郎。居七八日,行過轜車,嚬眉下淚,指方相曰:「何日更相煩君一到?」
 數日而卒,雇方相送葬,仍前所逢者。



 孝謙死後,定州秀才荀士遜繼為主書,才名相亞。



 茹瞻字孝博,東安人。南州舉秀才。清朗剛直。楊愔將用之,曰:「今日之選,不可無茹生。」卒於侍御史。



 荀士遜,廣平人也。好學,有思理,為文清典,見賞知音。武定末,舉司州秀才,迄齊天保,十年不調。皇建中,馬敬德薦為主書,轉中書舍人。狀貌甚醜,以文辭見重。嘗有事須奏,遇武成在後庭,因左右傳通,傳通者不得士遜姓名,乃云「醜舍人」。帝曰:「必士遜也。」看封題果是,內人莫不歡笑。累遷中書侍郎,號為稱職。與李若等撰《典言》,行於
 世。齊亡年卒。



 王褒,字子深,瑯邪臨沂人也。曾祖儉、祖騫、父規,並《南史》有傳。褒識量淹通,志懷沈靜,美威儀,善談笑,博覽史傳,七歲能屬文。外祖梁司空袁昂愛之,謂賓客曰:「此兒當成吾宅相。」弱冠舉秀才,除祕書郎、太子舍人。梁國子祭酒蕭子雲,褒之姑夫也,特善草隸。褒少以姻戚,去來其家,遂相模範,而名亞子雲,並見重於時。武帝嘉其才藝,遂以弟鄱陽王恢女妻之。襲爵南昌縣侯,歷位祕書丞、宣城王文學、安城內史。及侯景陷建鄴,褒輯寧所部,見稱於時。轉南平內史。梁元帝嗣位,褒有舊,召拜吏部尚
 書、右僕射,仍遷左丞,兼參掌。褒既名家,文學優贍,當時咸共推挹,故位望隆重,寵遇日甚。而愈自謙損,不以位地矜物,時論稱之。



 初,元帝平侯景及禽武陵王紀後,以建鄴凋殘,時江陵殷盛,便欲安之。又其政府臣僚皆楚人也,並願即都鄢郢。嘗召群臣議之。鎮軍將軍胡僧祐、吏部尚書宗懍、太府卿黃羅漢、御史中丞劉玨等曰:「建鄴王氣已盡,又荊南地又有天子氣,遷徙非宜。」元帝深以為然。褒性謹慎,知元帝多猜忌,弗敢公言其非。後因清閑,密諫,言辭甚切。元帝意好荊楚,已從僧祐等策,竟不用。及魏征江陵,元帝授褒都督城西諸軍事。柵破,從
 元帝入金城。俄而元帝出降,褒遂與眾俱出,見柱國于謹,甚禮之。褒曾作《燕歌》,妙盡塞北寒苦之狀,元帝及諸文士並和之,而競為悽切之辭,至此方驗焉。褒與王克、劉玨、宗懍、殷不害等數十人俱至長安,周文喜曰:「昔平吳之利,二陸而已;今定楚之功,群賢畢至,可謂過之矣。」又謂褒及王克曰:「吾即王氏甥也,卿等並吾之舅氏,當以親戚為情,勿以去鄉介意。」



 於是授褒及殷不害等車騎大將軍、儀同三司。常從容上席,資餼甚厚。褒等亦並荷恩眄,忘羈旅焉。



 周孝閔帝踐阼,封石泉縣子。明帝即位,篤好文學,時褒與庾信才名最高,特加親待。帝每遊
 宴,命褒賦詩談論,恆在左右。尋加開府儀同三司。保定中,除內史中大夫。武帝作《象經》,令褒注之,引據該洽,甚見稱賞。褒有器局,雅識政體,既累世在江東為宰輔,帝亦以此重之。建德以後,頗參朝議,凡大詔冊,皆令褒具草。東宮既建,授太子少保,遷少司空,仍掌綸誥。乘輿行幸,褒常侍從。



 初,褒與梁處士汝南周弘讓相善,及讓兄弘正自陳來聘,帝許褒等通親知音問,褒贈弘讓詩并書焉。尋出為宜州刺史,卒於位。子鼒。



 庾信,字子山,南陽新野人。祖易、父肩吾,並《南史》有傳。信幼而俊邁,聰敏絕倫,博覽群書,尤善《春秋左氏傳》。身長
 八尺,腰帶十圍,容止頹然,有過人者。父肩吾,為梁太子中庶子,掌管記。東海徐摛為右衛率。摛子陵及信並為抄撰學士。父子東宮,出入禁闥,恩禮莫與比隆。既文並綺艷,故世號為徐、庾體焉。當時後進,競相模範,每有一文,都下莫不傳誦。累遷通直散騎常侍,聘于東魏。文章辭令,盛為鄴下所稱。還為東宮學士,領建康令。



 侯景作亂,梁簡文帝命信率宮中文武千餘人營於硃雀航。及景至,信以眾先退。



 臺城陷後,信奔於江陵。梁元帝承制,除御史中丞。及即位,轉右衛將軍,封武康縣侯,加散騎侍郎,聘于西魏。屬大軍南討,遂留長安。江陵平,累遷儀
 同三司。



 周孝閔帝踐阼,封臨清縣子,除司水下大夫。出為弘農郡守。遷驃騎大將軍、開府儀同三司、司憲中大夫。進爵義城縣侯。俄拜洛州刺史。信為政簡靜,吏人安之。



 時陳氏與周通好,南北流寓之士,各許還其舊國。陳氏乃請王褒及信等十數人。武帝唯放王克、殷不害等,信及褒並惜而不遣。尋徵為司宗中大夫。明帝、武帝並雅好文學,信特蒙恩禮。至於趙、滕諸王,周旋款至,有若布衣之交。群公碑誌,多相託焉。唯王褒頗與信埒,自餘文人,莫有逮者。



 信雖位望通顯,常作鄉關之思,乃作《哀江南賦》以致其意。大象初,以疾去職。隋開皇元年卒。有
 文集二十卷。文帝悼之,贈本官,加荊、雍二州刺史。子立嗣。



 顏之推,字介,瑯邪臨沂人也。祖見遠、父協,並以義烈稱。世善《周官》、《左氏》學,俱《南史》有傳。之推年十二,遇梁湘東王自講《莊》、《老》,之推便預門徒。虛談非其所好,還習《禮》、《傳》。博覽書史,無不該洽,辭情典麗,甚為西府所稱。湘東王以為其國右常侍,加鎮西墨曹參軍。好飲酒,多任縱,不修邊幅,時論以此少之。湘東遣世子方諸鎮郢州,以之推為中撫軍府外兵參軍,掌管記。遇侯景陷郢州,頻欲殺之,賴其行臺郎中王則以免。景平,還江陵。時湘東即
 位,以之推為散騎侍郎,奏舍人事。後為周軍所破,大將軍李穆重之,送往弘農,令掌其兄陽平公遠書翰。遇河水暴長,具船將妻子奔齊,經砥柱之險,時人稱其勇決。文宣見,悅之,即除奉朝請,引於內館中,侍從左右,頗被顧眄。後從至天泉池,以為中書舍人,令中書郎段孝信將敕示之推。之推營外飲酒,孝信還以狀言,文宣乃曰:「且停。」由是遂寢。後待詔文林館,除司徒錄事參軍。之推聰穎機悟,博識有才辯,工尺牘,應對閑明,大為祖珽所重,令掌知館事,判署文書。



 遷通直散騎常侍,俄領中書舍人。帝時有取索,恒令中使傳旨,之推稟承宣告,館中
 皆受進止。所進文書,皆是其封署,於進賢門奏之,待報方出。兼善於文字,監校繕寫,處事勤敏,號為稱職,帝甚加恩接。為勛要者所嫉,常欲害之。崔季舒等將諫也,之推取急還宅,故不連署。及召集諫人,之推亦被喚入,勘無名,得免。



 尋除黃門侍郎。



 及周兵陷晉陽,帝輕騎還鄴,窘急,計無所從。之推因宦者侍中鄧長顒進奔陳策,仍勸募吳士千餘人以為左右,取青、徐路共投陳國。帝納之,以告丞相高阿那肱等。阿那肱不願入陳。乃云吳士難信,勸帝送珍寶累重向青州,且守三齊地。若不可保,徐浮海南度。雖不從之推策,然猶以為平原太守,令守
 河津。



 齊亡入周。大象末,為御史上士。隋開皇中,太子召為文學,深見禮重,尋以疾終。有文集三十卷,撰《家訓》二十篇,並行於世。之推在齊有二子,長曰思魯,次曰敏楚,蓋不忘本也。《之推集》,思魯自為序。



 弟之儀,字升。幼穎悟,三歲能讀《孝經》。及長,博涉群書,好為詞賦。嘗獻梁元帝《荊州頌》,辭致雅贍。帝手敕曰:「枚乘二葉,俱得游梁;應貞兩世,並稱文學。我求才子,鯁慰良深。」



 江陵平,之儀隨例遷長安,周明帝以為麟趾學士。稍遷司書上士。武帝初建東宮,盛選師傅,以之儀為侍讀。太子後征吐谷渾,在軍有過行,鄭譯等並以不能匡弼
 坐譴,唯之儀以累諫獲賞。即拜小宮尹,封平陽縣男。宣帝即位,遷上儀同大將軍、御正中大夫,進爵為公。帝後刑政乖僻,昏縱日甚。之儀犯顏驟諫,雖不見納,終亦不止,深為帝所忌。然以恩舊,每優容之。及帝殺王軌,之儀固諫。帝怒,欲并致之於法。後以其諒直無私,乃舍之。



 宣帝崩,劉昉、鄭譯等矯遺詔,以隋文帝為丞相輔少主。之儀知非帝旨,拒而弗從。昉等草詔,署訖,逼之儀署。之儀厲聲謂昉等曰:「主上升遐,嗣子幼沖,阿衡之任,宜在宗英。方今賢戚之內,趙王最長,以親以德,合膺重寄。公等備受朝恩,當盡忠報國,柰何一旦欲以神器假人!之儀
 有死而已,不能誣罔先帝。」於是昉等知不可屈,乃代之儀署而行之。隋文帝後索符璽,之儀又正色曰:「此天子之物,自有主者,宰相何故索之?」於是文帝大怒,命引出,將戮之。然以其人望,乃止。出為西疆郡守。



 及踐極,詔徵還京師,進爵新野郡公。開皇五年,拜集州刺史。在州清靜,夷夏悅之。明年代還,遂優游不仕。十年正月,之儀例入朝。文帝望而識之,命引至御坐,謂之曰:「見危授命,臨大節而不可奪。古人所難,何以加卿。」乃賜錢十萬、米一百石。十一年卒。有《文集》十卷,行於世。



 虞世基,字懋世,會稽餘姚人也。父荔,《南史》有傳。世基幼
 恬靜,喜慍不形於色,博學有高才,兼善草隸。陳中書令孔奐見而歎曰:「南金之貴,屬在斯人。」



 少傅徐陵聞其名,召之,世基不往。後因公會,陵一見而奇之,顧朝士曰:「當今潘、陸也。」因以弟女妻焉。仕陳,累遷尚書左丞。陳主嘗於莫府山校獵,令世基為《講武賦》,於坐奏之。陳主嘉之,賜馬一匹。



 及陳滅,入隋為通直郎,直內史省。貧無產業,每傭書養親,怏怏不平。嘗為五言詩以見情,文理心妻切,世以為工,作者無不吟詠。未幾拜內史舍人。煬帝即位,顧遇彌隆。祕書監河東柳顧言,博學有才,罕所推謝,至是與世基相見,歎曰:「海內當共推此一人,非吾儕所及
 也。」俄遷內史侍郎。以母憂去職,哀毀骨立。



 有詔起令視事。拜見之日,殆不能起,令左右扶之。哀其羸瘠,詔令進肉。世基食,輒悲哽不能下筋。帝使謂曰:「方相委任,宜為國惜身。」前後敦勸者數矣。帝重其才,親禮逾厚,專典機密,與納言蘇威、左翊衛大將軍宇文述、黃門侍郎裴矩、御史大夫裴蘊等參掌朝政。時天下多事,四方表奏,日有百數。帝方凝重,事不廷決。入閣之後,始召世基口授節度。世基至省,方為敕書,日旦百紙,無所遺繆。



 遼東之役,進位金紫光祿大夫。後從幸鴈門,為突厥所圍。戰士多敗。世基勸帝為賞格,親自撫循,乃下詔停遼東事。帝
 從之,師乃復振。及圍解,勳格不行,又下伐遼之詔,由是言其詐眾,朝野離心。帝幸江都,次鞏縣,世基以盜賊日盛,請發兵屯洛口倉,以備不虞。帝不從,但答云:「卿是書生,定猶恇怯。」于時天下大亂,世基知帝不可諫正,又以高熲、張衡等相繼誅戮,懼禍及己,雖居近侍,唯諂取容,不敢忤意。盜賊日甚,郡縣多沒,世基知帝惡數聞之,後有告敗者,乃抑損表狀,不以實聞。是後外間有變,帝弗之知也。嘗遣太僕卿楊義臣捕盜河北,降賊數十萬,列狀上聞。帝歎曰:「我初不聞賊頓如此,義臣列降賊何多也?」世基曰:「鼠竊雖多,未足為慮。義臣剋之,擁兵不少,久
 在閫外,此最非宜。」帝曰:「卿言是也。」遽追義臣,放其兵散。又越王侗遣太常丞元善達間行賊中,詣江都奏事,稱:「李密有眾數萬,圍逼京都。賊據洛口倉,城內無食。若陛下速還,烏合必散。不然者,東都決沒。」因歔欷嗚咽,帝為改容。世基見帝色憂,進曰:「越王年小,此輩誑之。若如所言,善達何緣得至?」帝勃然怒曰:「善達小人,敢廷辱我!」因使經賊中,向東陽催運。善達遂為群盜所殺。此後外人杜口,莫敢以賊聞奏。



 世基氣貌沈審,言多合意,是以特見親愛,朝臣無與為比。其繼室孫氏,性驕淫,世基惑之,恣意奢靡,彫飾器服,無復素士之風。孫復攜前夫子夏
 侯儼入世基舍,而頑鄙無賴,為其聚斂,鬻官賣獄,賄賂公行,其門如市,金寶盈積。其弟世南素國士,而清貧不立,未曾有所贍。由是為論者所譏。朝野咸共疾怨。宇文化及之弒逆也,世基乃見害。



 長子肅,好學才藝,時人稱有家風。弱冠早沒。



 肅弟熙,大業末為符璽郎。次子柔、晦,並宣義郎。化及將亂之夕,宗人虞伋知而告熙曰:「事勢已然,吾將濟卿南度,且得免禍,同死何益。」熙曰:「棄父背君,求生何地,感尊之懷,自此訣矣。」及難作,兄弟競請先死,行刑人先世基殺之。


柳
 \gezhu{
  巧言}
 ,字顧言,河東人也。世仕江南,居襄陽。祖惔,《南史》有
 傳。
 \gezhu{
  巧言}
 少聰敏,解屬文,好讀書,所覽將萬卷。仕梁,為著作佐郎。後蕭察據荊州,以為侍中,領國子祭酒、吏部尚書。及梁國廢,拜開府,為內史侍郎。以無吏乾,轉晉王諮議參軍。王好文雅,招引才學之士諸葛潁、虞世南、王胄、朱瑒等百餘人以充學士,而
 \gezhu{
  巧言}
 為之冠。王以師友處之,每有文什,必令其潤色,然後示人。嘗朝京還,作《歸籓賦》,命
 \gezhu{
  巧言}
 為序,詞甚典麗。初王屬文,效庾信體,及見
 \gezhu{
  巧言}
 後,文體遂變。


仁壽初,引為東宮學士,加通直散騎常侍,檢校洗馬,甚見親重。每召入臥內,與之宴謔。
 \gezhu{
  巧言}
 尤俊辯,多在侍從,有所顧問,應答如響。性嗜酒,言雜誹諧。



 由是彌為太
 子所親狎。以其好內典,令撰《法華玄宗》,為二十卷上之。太子大悅,賞賜優洽,儕輩莫比。


煬帝嗣位,拜祕書監,封漢南縣公。帝退朝後,便命入問,言宴諷讀,終日而罷。常每與嬪后對酒,時逢興會,輒遣命之至,與同榻共席,恩比友朋。帝猶恨不能夜召,乃命匠刻木為偶人,施機關,能坐起拜伏,以像
 \gezhu{
  巧言}
 。帝每月下對飲酒,輒令宮人置於座,與相酬酢,而為歡笑。從幸揚州,卒,帝傷惜者久之。贈大將軍,謚曰康。


\gezhu{
  巧言}
 撰《晉王北伐記》十五卷,有集十卷行於世。



 許善心,字務本,高陽北新城人也。祖茂、父亨,並《南史》有
 傳。善心九歲而孤,為母范氏所鞠養。幼聰明,有思理,所聞輒能記,多聞默識,為當世所稱。



 家有舊書萬餘卷,皆遍通涉。十五解屬文,為箋上父友徐陵,陵大奇之,謂人曰:「此神童也。」太子詹事江總舉秀才,對策高第,授度支郎中,補撰史學士。禎明二年,加通直散騎常侍聘隋。遇文帝伐陳,禮成而不獲反命。累表請辭,上不許。



 留縶賓館。及陳亡,上遣使告之。善心素服號哭於西階下,藉草東向,經三日,敕書唁焉。明日,有詔就館拜通直散騎常侍,賜衣一襲。善心哭盡哀,入房改服,復出北面立,垂涕再拜受詔。明日,乃朝服泣於殿下,悲不能興。上顧左右
 曰:「我平陳國,唯獲此人。既能懷其舊君,即我誠臣也。」敕以本官直門下省,賜物千段、草馬二十匹。從幸太山,還,授虞部侍郎。



 十六年,有神雀降於含章闥,上召百官賜宴,告以此瑞。善心於坐請紙筆,製《神雀頌》奏之。上甚悅曰:「我見神雀,共皇后觀之。今且召公等入,適述此事。



 善心於坐始知,即能成頌。文不加點,筆不停毫,常聞此言,今見其事。」因賜物二百段。十七年,除秘書丞。時祕藏圖籍,尚多淆亂。善心效阮孝緒《七錄》,更制《七林》,各總敘冠於篇首。又於部錄之下明作者之意,區分類例焉。又奏追李文博、陸從典等學者十許人,正定經史錯謬。仁壽
 元年,攝黃門侍郎。二年,加攝太常少卿,與牛弘等議定禮樂,祕書丞、黃門並如故。四年,留守京師。帝崩於仁壽宮,煬帝祕不發喪,先易留宮人,出除巖州刺史。逢漢王諒反,不之任。大業元年,轉禮部侍郎,奏薦儒者徐文遠為國子博士,包愷、陸德明、褚徽、魯世達之輩,並加品秩,授為學官。其年,副納言楊達為冀州道大使,以稱旨,賜物五百段。



 左衛大將軍宇文述每日借本部兵數十人以供私役,常半日而罷。御史大夫梁毗奏劾之。上方以腹心委述,初付法官推,千餘人皆稱被役。經二十餘日,法官候伺上旨,乃言役不滿日,其數雖多,不合通計,縱
 令有實,亦無罪。諸兵士聞之,更云初不被役。上欲釋之,付議虛實,百僚咸議為虛。善心以為述於仗衛之所,抽兵私役,雖不滿日,闕於宿衛,與常役所部,情狀乃殊。又兵多下番,散還本府,分道追至,不謀同辭。今殆一月,方始翻覆,姦狀分明,此何可捨?蘇威、楊汪等二十餘人同善心議,其餘皆議免罪。煬帝可免者之奏。後數月,述譖善心曰:「陳叔寶卒,善心共周羅、虞世基、袁充、蔡徵等同往送葬。善心為祭文,謂為『陛下』。



 敢於今日加叔寶尊號。」召問有實,自援古例,事得釋,而甚惡之。又太史奏帝即位年與堯時符合,善心議以國哀甫爾,不宜稱賀。述
 諷御史劾之,左遷給事郎,降品二等。



 四年,撰《方物志》,奏之。七年,從至涿郡。帝方自御戎以東討,善心上封事,忤旨免官。其年復徵守給事郎。帝嘗言及文帝受命之符,因問鬼神之事,敕善心與崔祖濬撰《靈異記》十卷。



 初,善心父撰著《梁史》,未就而歿。善心述成父志,修續家書。其《序傳》末述制作之意,曰:謹按太素將萌,洪荒初判。乾儀資始,辰象所以正時;坤載厚生,品物於焉播氣。參三才而育德,肖二統而降靈。有黎人焉,為之君長;有貴賤矣,為其宗極。



 保上天之眷命,膺下土之樂推,莫不執太方,振長策,感召風雲,驅馳英俊。干戈揖讓,取之也殊功;鼎
 玉龜符,成之也一致。革命創制,竹素之道稍彰;紀事記言,筆墨之官漸著。炎、農以往,存其名而漏其迹;黃、軒以來,晦其文而顯其質。登丘納麓,具訓誥及典謨;貫昴入房,傳夏正與殷祀。洎辨方正位,論時計功。南北左右,兼四名之別;《檮杌》、《乘》車,擅一家之稱。國惡雖諱,君舉必書。故賊子亂臣,天下大懼,元龜明鏡,昭然可察。及三郊遞襲,五勝相沿,俱稱百穀之王,並以四海自任。重光累德,何世無哉。



 逮有梁之興,君臨天下,江左建國,莫斯為盛。受命在於一君,繼統傳乎四主。



 克昌四十八載,餘祚五十六年。武皇帝出自諸生,爰升寶歷。拯百王之弊,救萬
 姓之危。反澆季之末流,登上皇之獨道。朝多君子,野無遺賢,禮樂必備,憲章咸舉,弘深慈於不殺,濟大忍於無刑。蕩蕩巍巍,可為稱首。屬陰戎入潁,羯胡侵洛。沸騰墋黷,三季之所未聞;掃地滔天,一元之所巨厄。廊廟有序,翦成狐兔之場;珪帛有儀,碎夫犬羊之手。福善積而身禍,仁義存而國亡,豈天道歟?豈人事歟?嘗別論之,在於《序論》之卷。



 先君昔在前代,早懷述作,凡撰《齊書》為五十卷;《梁書》紀傳,隨事勒成及闕而未就者,目錄注為一百八卷。梁室交喪,墳籍銷盡。冢壁皆殘,不準無所盜;帷囊同毀,陳農何以求!秦儒既坑,先王之道將墜;漢臣徒請,
 口授之文亦絕。所撰之書,一時亡散。有陳初建,詔為史官,補闕拾遺,心識口誦,依舊目錄,更加修撰,且成百卷,已有六帙五十八卷上祕閣訖。



 善心早嬰荼蓼,弗克荷薪,太建之末,頻抗表聞,至德之初,蒙授史任。方願緗素採訪,門庭記錄,俯勵弱才,仰成先志。而單宗少強近,虛室類原、顏,退屏無所交游,棲遲不求進益。假班嗣之書,徒聞其語;給王隱之筆,未見其人。加以庸瑣涼能,孤陋末學,參職郎署,兼撰《陳史》,致此書延時,未即成續。禎明二年,以臺郎入聘,屬本邑淪覆,他鄉播遷,行人失時,將命不復。望都亭而長慟,遷別館而懸壺。家史舊書,在後
 蕩盡。今止有六卷獲存,又並缺落失次。自入京邑以求,隨見補葺,略成七十卷:四《帝紀》八卷,《后妃》一卷,三《太子錄》一卷,為一帙十卷;《宗室王侯列傳》一帙十卷;《具臣列傳》二帙二十卷;《外戚傳》一卷,《孝德傳》一卷,《誠臣傳》一卷,《文苑傳》二卷,《儒林傳》二卷,《逸人傳》一卷,《數術傳》一卷,《籓臣傳》一卷,合一帙十卷;《止足傳》一卷,《列女傳》一卷,《權幸傳》一卷,《羯賊傳》二卷,《逆臣傳》二卷,《叛臣傳》二卷,《敘傳論述》一卷,合一帙十卷。凡稱史臣者皆先君所言,下稱名案者皆善心補闕。別為《敘論》一篇,託于《敘傳》之末。



 十年,又從至懷遠鎮,加授朝散大夫。突厥圍鴈門,攝左親侍
 武賁郎將,領江南兵宿衛殿省。駕幸江都,追敘前勳,授通議大夫,詔還本品,行給事郎。



 十四年,化及弒逆之日,隋官盡詣朝堂謁賀,善心獨不至。許弘仁馳告曰:「天子已崩,宇文將軍攝政,合朝文武,莫不咸集。天道人事,自有代終,何預叔而低徊若此?」善心怒之,不肯隨去。弘仁返走上馬,泣而言曰:「將軍於叔全無惡意,忽自求死,豈不痛哉!」還告唐奉議,以狀白化及,遣人就宅執至朝堂。化及令釋之,善心不舞蹈而出。化及目送之,曰:「此大負氣。」命捉來,罵云:「我好欲放你,敢如此不遜!」其黨輒牽曳,遂害之。及越王稱制,贈左光祿大夫,封高陽縣公,謚曰
 文節。



 善心母范氏,梁太子中舍人孝才之女也。少寡,養孤,博學有高節。隋文帝知之,敕尚食每獻時新,常遣分賜。嘗詔范入內,侍皇后講讀。封永樂郡君。及善心遇禍,范氏九十有二,臨喪不哭,撫柩曰:「能死國難,我有兒矣。」因臥不食,後十餘日亦終。



 李文博,博陵人。性貞介鯁直,好學不倦。至於教義名理,特所留心。每讀書至安危得失,忠臣烈士,未嘗不反覆吟玩。開皇中,為羽騎尉。特為吏部侍郎薛道衡所知,恒令在事帷中,披檢書史,并察己行事。若遇政教善事,即抄撰記錄,如選用疏謬,即委之臧否。道衡每得其語,
 莫不忻然從之。



 後直祕書內省,典校群籍。守道居貧,晏如也。雖衣食乏絕。而清操愈厲,不妄通賓客,恆以禮法自處,儕輩莫不敬焉。道衡知其貧,每延于家,給以資費。文博商略古今政教得失,如指諸掌。然無吏乾。稍遷校書郎,出為縣丞,遂得下考,數歲不調。道衡為司隸大夫,遇之東都尚書省,甚嗟愍之,奏為從事。因謂齊王司馬李綱曰:「今日遂遇文博,得奏用之。」以為歡笑。其見賞知音如此。



 在洛下,曾詣房玄齡,相送出衢路。玄齡謂曰:「公生平志尚,唯在正直。今既得為從事,故應有會素心。比來激濁揚清,所為多少?」文博遂奮臂厲聲曰:「夫清其流
 者必潔其源,正其末者須端其本。今政源混亂,雖日免十貪郡守,亦何所益!」其率直疾惡,不知忌諱,皆如此類。時朝政浸壞,人多贓賄,唯文博不改其操。論者以此貴之。遭亂播遷,不知所終。



 初,文博在內省校書,虞世基子亦在其內,盛飾容服而未有所知。文博因從容問之年紀,答云十八。文博乃謂曰:「昔賈誼當此之年,議論何事?君今徒事儀容,欲何為者?」又秦孝王妃生男,文帝大喜,頒賜群官各有差。文博家道屢空,人謂其悅賞。乃云:「賞罰之設,功過所歸。今王妃生男,於群官何事,乃妄受賞也!」



 其循名責實,錄過計功,必使賞罰不濫,功過無隱皆
 爾。



 文博本為經學,後讀史書,於諸子及論,尤所該洽。性長議論,亦善屬文。著《政道集》十卷,大行於世。



 開皇中,又有魏郡侯白,字君素,好學有捷才,性滑稽,尤辯俊。舉秀才,為儒林郎。通侻不持威儀,好為俳諧雜說。人多愛狎之,所在處,觀者如市。楊素甚狎之。素嘗與牛弘退朝,白謂素曰:「日之夕矣。」素大笑曰:「以我為『牛羊下來』邪!」文帝聞其名,召與語,悅之,令於祕書修國史。每將擢用,輒曰「白不勝官』而止。後給五品食,月餘而死。時人傷其薄命。著《旌異記》十五卷,行於世。



 明克讓,字弘道,平原鬲人也。世仕江左。祖僧紹、父山賓,
 並《南史》有傳。



 克讓少儒雅,善談論,博涉書史,所覽將萬卷,《三禮》、《論語》,尤所研精,龜策歷象,咸得其要。年十四,釋褐湘東王法曹參軍。時舍人朱異在儀賢堂講《老子》,克讓預焉。堂邊有修竹,異令克讓詠之。克讓攬筆輒成,卒章曰:「非君多愛賞,誰貴此貞心?」異甚奇之。仕梁,位中書侍郎。梁滅,歸長安,引為麟趾殿學士。周武帝即位,為露門學士,令與太史官屬正定新歷。累遷司調大夫,賜爵歷城縣伯。隋文帝受禪,位率更令,進爵為侯。太子以師道處之,恩禮甚厚,每有四方珍味,輒以賜之。時東宮盛徵天下才學士。至於博物洽聞,皆出其下。詔與太常牛
 弘等修禮議樂。當朝典故,多所裁正。以疾去官,加通直散騎常侍,卒。上甚惜之,二宮贈賻甚厚。



 所著《孝經義疏》一部,《古今帝代記》一卷,《文類》四卷,《續名僧記》一卷,集二十卷。



 子餘慶,位司門郎。越王侗稱制,為國子祭酒。



 克讓叔少遐,博涉群書,有詞藻。仕梁,位都官尚書。入齊,甚為名流王元景、陽休之等所禮。皇建中,拜中庶子。卒,贈中書令、揚州司馬。



 劉臻,字宣摯,沛國相人也。父顯,《南史》有傳。臻年十八,舉秀才,為邵陵王東閣祭酒。元帝時,遷中書舍人。江陵平,歸魏為中書侍郎。周冢宰宇文護辟為中外府記室,軍
 書羽檄,多成其手。後為露門學士,授大都督,封饒陽縣子。歷藍田令、畿伯下大夫。隋文帝受禪,進位儀同三司。左僕射高熲之伐陳也,以臻隨軍主文翰,進爵為伯。皇太子勇引為學士,甚親狎之。



 臻無吏乾,又性惚怳,耽經覃思。至於世事,多所遺忘。有劉訥者,亦任儀同,俱為太子學士,情好甚密。臻住城南,訥住城東。臻嘗欲尋訥,謂從者曰:「汝知劉儀同家乎?」從者不知尋訥,謂臻還家,因答曰:「知。」於是引之而去。既扣門,臻尚未悟,謂至訥家,乃據鞍大呼曰:「劉儀同可出矣。」其子迎門,臻驚曰:「汝亦來邪?」其子答曰:「此是大人家。」於是顧眄久之,乃悟,叱從者:「
 汝大無意,吾欲造劉訥耳!」性好啖蜆,以音同父諱,呼為扁螺,其疏放多此類也。



 精於兩《漢書》,時人稱為《漢》聖。開皇十八年,卒。有集十卷,行於世。



 諸葛潁,字漢,丹楊建康人也。祖銓,梁零陵太守。父規,義陽太守。潁年十八能屬文,起家邵陵王參軍事,轉記室。侯景之亂,奔齊,歷學士、太子舍人。周氏平齊,不得調,杜門不出者十餘年。習《易》、《圖緯》、《蒼》《雅》、《莊》《老》頗得其要,清辯有俊才。晉王廣素聞其名,引為參軍事,轉記室。及王為太子,除藥藏郎。



 煬帝即位,遷著作郎,甚見親倖,出入臥內。帝每賜之曲宴,輒與皇后嬪御連席共榻。潁因間隙,
 多所譖毀,是以時人謂之「冶葛」。後錄恩舊,授朝散大夫。



 帝嘗賜潁詩,其卒章曰:「參翰長洲苑,侍講肅成門,名理窮研核,英華恣討論。



 實錄資平允,傳芳導後昆。」其待遇如此。從征吐谷渾,加正議大夫。從駕北巡,卒於道。


潁性褊急,與柳
 \gezhu{
  巧言}
 每相忿鬩。帝屢責怒之,而猶不止。於後帝亦薄之。有集二十卷,撰《鑾駕北巡記》三卷,《幸江都道里記》一卷,《洛陽古今記》一卷,《馬名錄》二卷,並行於世。有子嘉會。



 王貞,字孝逸,梁郡陳留人也。少聰敏,七歲好學,善《毛詩》、《禮記》、《左氏傳》、《周易》,諸史百家無不畢覽。善屬文,不事產
 業,每以諷讀為娛。



 開皇初,汴州刺史樊叔略引為主簿。後舉秀才,授縣尉。非其好也,謝病于家。煬帝即位,齊王暕鎮江都,聞其名,以書召之。及至,以客禮待之,索其文集。貞上三十三卷,為啟陳謝。齊王覽集,甚善之,賜良馬四匹。貞復上《江都賦》,王賜錢十萬貫、良馬二匹。未幾,以疾甚還鄉,終於家。



 虞綽,字士裕,會稽餘姚人也。父孝曾,陳始興王咨議。綽身長八尺,姿儀甚偉,博學有俊才,尤工草隸。陳左衛將軍傅縡,有盛名於世,見綽詞賦,歎美之。



 仕陳,為太學博士,遷永陽王記室。及陳亡,晉王廣引為學士。大業初,轉
 為祕書學士,奉詔與祕書郎虞世南,著作佐郎庾自直等撰《長洲玉鏡》等書十餘部。綽所筆削,帝未嘗不稱善,而官竟不遷。初為校書郎,以籓邸左右,授宣惠尉,遷著作佐郎。與虞世南、庾自直、蔡允恭等四人常直禁中,以文翰待詔,恩眄隆洽。從征遼東,帝舍臨海,頻見大鳥,異之,詔綽為銘。帝覽而善之,命有司勒於海上。以度遼功,授建節尉。



 綽恃才任氣,無所降下。著作郎諸葛潁以學業倖於帝,綽每輕侮之,由是有隙。



 帝嘗問綽於潁,潁曰:「虞綽粗疏人也。」帝頷之。時禮部尚書楊玄感稱其貴踞,虛己禮之,與結布衣之友。綽數從之遊。其族人虞世南
 誡之曰:「上性猜忌,而君過厚玄感。若與絕交者,帝知君改悔,可以無咎。不然終當見禍。」綽不從。尋有告綽以禁內兵書借玄感,帝甚銜之。及玄感敗,其妓妾並入宮,帝因問之曰:「玄感平常時與何人交往?」其妾以虞綽對。帝令大理卿鄭善果窮理其事。綽曰:「羈旅薄游,與玄感文酒談款,實無他謀。」帝怒不解,徙綽于邊。綽至長安而亡。吏逮之急,於是潛度江,變姓名,自稱吳卓。游東陽,抵信安令天水辛大德舍。歲餘,綽與人爭田相訟,因有識綽者而告之,竟為吏所執,坐斬江都。所有詞賦,並行於世。



 大德為令,誅翦群盜,甚得人和。與綽俱為使者所執,其
 妻泣曰:「每諫君無匿學士。今日之事,豈不哀哉!」大德笑曰:「我本圖脫長者,乃為人告之,吾罪也,當死以謝綽。」會有詔,死罪得以擊賊自效。信安吏人詣使者叩頭曰:「辛君人命所懸,不然亦無信安矣。」使者留之以討賊。帝怒,斬使者。大德獲全。



 王胄,字承基,瑯邪臨沂人也。祖筠、父祥,並《南史》有傳。胄少有逸才,仕陳,歷太子舍人、東陽王文學。及陳滅,晉王廣引為博士。仁壽末,從劉方擊林邑,以功授帥都督。大業初,為著作佐郎,以文詞為煬帝所重。帝嘗自東都還京師,賜天下大酺四日。為五言詩,詔群官詩成者奏之。
 帝覽胄詩而善之,因謂侍臣曰:「氣高致遠,歸之於胄;詞清體潤,其在世基;意密理新,惟庾自直。過此者未可以言詩也。」帝所有篇什,多令繼和。與虞綽齊名,同志友善,于時後進之士,咸以二人為準的。從征遼東,進授朝散大夫。



 胄性疏率不倫,自恃才伐,鬱鬱於官。每負氣陵傲,忽略時人。為諸葛潁所嫉,屢譖之於帝,帝愛其才而不罪。禮部尚書楊玄感虛襟與交,數游其第。及玄感敗,與虞綽徙邊。胄遂亡匿,潛還江左。為吏所捕,坐誅。所著詞賦,多行於世。



 兄翽,字元恭。博學多通,少有盛名於江左。仕陳,歷太子
 洗馬、中舍人。陳亡,與胄俱為學士。煬帝即位,授秘書郎,卒於官。



 庾自直,潁川人。父持,《南史》有傳。少好學,沈靜寡欲。仕陳,歷豫章王府外兵參軍、記室。陳亡入關,不得調。晉王廣聞之,引為學士。大業初,授著作佐郎。自直解屬文,於五言詩尤善。性恭慎,不妄交游。特為帝所愛,有篇章必先示自直,令其詆訶。自直所難,帝輒改之。或至於再三,俟其稱善,然後方出。其見親禮如此。後以本官知起居舍人事。化及作逆,與之北上,自載露車中,感激發病卒。有文集十卷,行於世。



 潘徽,字伯彥,吳郡人也。性聰敏,少受《禮》於鄭灼,受《毛詩》於施公,受《書》於張沖,講《莊》、《老》於張譏,並通大義。尤精《三史》。善屬文,能持論。中書令江總引致文儒之士,徽一詣總,甚敬之。釋褐新蔡王國侍郎,選為客館令。隋遣魏澹聘於陳,陳人使徽接對之。澹將反命,為啟於陳主曰:「敬奉弘慈,曲垂餞送。」徽以餞送為重,敬奉為輕,卻其啟而不奏。澹曰:「《曲禮》云:主敬客。《詩》曰:『維桑與梓,必恭敬止』。《孝經》:『宗廟致敬。』又云:『不敬其親,謂之悖禮。』孔子敬天之怒,成湯聖敬日躋。宗廟極重,上天極高,父極尊,君極貴,四者咸同一敬,《五經》未有異文。不知以敬為輕,竟何所據?」



 徽難之曰:「向所論敬字,本不全以為輕,但施用處殊,義成通別。禮主於敬,此是通言。猶如男子冠而字之,注云:『成人,敬其名也。』《春秋》有冀缺,夫妻亦云相敬。於子則有敬名之義,在夫亦有敬妻之說,此可復並謂極高極尊乎?至若敬謝諸公,固非尊地;公子敬愛,止施賓友;敬問敬報,彌見雷同;敬聽敬酬,何關貴隔。當知敬之為義,雖是不輕,但敬之於語,則有時混漫。今云敬奉,所以成疑。聊舉一隅,未為深據。」澹不能對,遂從而改焉。



 及陳滅,為州博士。秦王俊聞其名,召為學士。嘗從俊朝京師。在塗,令徽於馬上為賦,行一驛而成,其名曰《述恩賦》。俊覽而
 善之。復令為《萬字文》。又遣撰集字書,名為《韻纂》,徽為之序。俊薨,晉王廣復引為揚州博士,令與諸儒撰《江都集禮》一部,復令徽為序。煬帝嗣位,徽與著作郎陸從典、太常博士褚亮、歐陽詢等助越公楊素撰《魏書》,會素薨而止。授京兆郡博士。楊玄感兄弟重之,數相往來。及玄感敗,凡所交關,多罹其患。徽以玄感故人,為帝所不悅。有司希旨,出徽為西海郡威定縣主簿。意甚不平,行至隴頭,發病而卒。



 隋時有常得志、尹式、劉善經、祖君彥、孔德紹、劉斌,並有才名,事多遺逸。



 常得志,京兆人。隋秦王記室。及王薨,過故第,為五言詩,
 辭理悲壯,甚為時人所重。復為《兄弟論》,義理可稱。



 尹式,河間人。仁壽中,官至漢王記室。漢王阻兵,式自殺。其族人正卿、彥卿亦俱有俊才,名顯於世。



 劉善經,河間人。歷著作佐郎、太子舍人。著《酬德傳》三十卷,《諸劉譜》三十卷,《四聲指歸》一卷,行於世。



 祖君彥,見其父珽傳。



 孔德紹,會稽人。有清才,官至京城縣丞。竇建德署為中書令,專典書檄。及建德敗,伏誅。



 劉斌,南陽人。祖之遴,《南史》有傳。斌頗有詞藻,官至信都司功書佐。竇建德署為中書舍人。建德敗,復為劉黑闥
 中書侍郎。與黑闥亡歸突厥,不知所終。


論曰:古人之所貴名不朽者,蓋重言之尚存。王褒、庾信、顏之推、虞世基、柳
 \gezhu{
  巧言}
 、許善心、明克讓、劉臻、王貞、虞綽、王胄等,並極南土譽望,又加之以才名,其為貴顯,固其宜也。自餘或位下人微,居常亦何能自達。及其靈蛇可握,天綱俱頓,並編緗素,咸貫辭林。雖其位可下,其身可殺,千載之外,貴賤一焉。



 非此道也,孰云能致?凡百士子,可不務乎!



\end{pinyinscope}