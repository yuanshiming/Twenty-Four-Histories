\article{卷七十四列傳第六十二}

\begin{pinyinscope}

 劉昉柳裘皇甫績郭衍張衡楊汪裴蘊袁充李雄劉昉,博陵望都人也。父孟良,仕魏,位大司農卿。從武帝入關,為梁州刺史。



 昉輕狡,有姦數。周武帝時,以功臣子入侍皇太子。及宣帝嗣位,以技佞見狎,出入宮掖,寵冠一時。位小御正,與御正中大夫顏之儀並見親信。及帝不悆,召昉及之儀俱入臥內,屬以後事。帝失喑,不復能
 言。昉見靜帝幼沖,又素奇隋文帝。時文帝以后父故,有重名於天下,昉遂與鄭譯謀,引帝輔政。帝固讓,不敢當,昉曰:「公若為,當速為之。如不為,昉自為也。」帝乃從之,及帝為丞相,以昉為司馬。



 時宣帝弟漢王贊居要沖,每與帝同帳而坐。昉飾美妓進贊,贊甚悅之。昉因說贊曰:「大王,先帝之弟,時望所歸。孺子幼沖,豈堪大事!今先帝初崩,群情尚擾,王且歸第。待事寧後,入為天子,此萬全計也。」贊時年未弱冠,性識庸下,以為信然,遂從之。文帝以昉有定策功,拜上大將軍,封黃國公,與沛國公鄭譯皆為心膂。



 前後賞賜鉅萬,出入以甲士自衛,朝野傾矚,稱
 為黃、沛。時人語曰:「劉昉牽前,鄭譯推後。」



 昉自恃功,有驕色。然性粗疏,溺於財利,富商大賈朝夕盈門。于時尉遲迥起兵,帝令韋孝寬討之。至武陟,諸將不一。帝欲遣昉、譯一人往監軍,因謂之曰:「須得心膂以統大軍,公兩人誰行?」昉辭未嘗為將,譯以母老為請,帝不懌。而高熲請行,遂遣之。由是恩禮漸薄。又王謙、司馬消難相繼反,文帝憂之,忘寢與食。昉逸遊縱酒,不以職司為意,相府事多所遺落。帝深銜之,以高熲代司馬。是後益見疏忌。及受禪,進柱國,改封舒國公,閑居無事,不復任使。昉自以佐命元功,中被疏遠,甚不自安。後遇京師饑,上命禁
 酒。昉使妾賃屋,當壚酤酒。治書侍御史梁毗劾奏之,有詔不問。昉鬱鬱不得志。



 時上柱國梁士彥、宇文忻俱失職怨望,時昉並與之交,數相往來。士彥妻有美色,昉與私通,士彥不之知也,情好彌協,遂相與謀反,許推士彥為帝。後事泄,帝窮問之。昉自知不免,默無所封。詔誅之曰:上柱國郕國公梁士彥、杞國公宇文忻、柱國舒國公劉昉等,朕受命之初,並展勤力,酬勛報效,榮高祿重。朝夕宴言,備知朕意。但心如溪壑,志等豺狼,不荷朝恩,忽謀逆亂。



 士彥稱有相者,云其應籙,年過六十,必據九五。初平尉遲迥,暫臨相州,已有反心,彰於道路。朕即遣人
 代之,不聲其罪。入京之後,逆意轉深。忻、昉之徒,言相扶助。士彥許率僮僕,剋期不遠,欲於蒲州起事。即斷河橋,捉黎陽之關,塞河陽之路。自謂一朝奮發,無人當者,其第二子剛,每常苦諫,第三子叔諧,固深勸獎。朕既聞知,猶恐枉濫,及授晉部之任,欲驗蒲州之情。士彥得以欣然,云是天贊。



 忻往定鄴城,自矜不已,位極人臣,猶恨賞薄。朕深念其功,不計無禮,任以武候,授以領軍,寄之爪牙,委之心腹。忻密為異計,樹黨宮闈,多奏交友,入參宿衛。朕推心待物,言必依許。為而弗止,心跡漸彰,仍解禁兵,令其改悔。而志規不逞,愈結於懷,乃與士彥情意偏
 厚,俱營賊逆,逢則交謀。委士彥河東,自許關右,蒲津事建,即望從征討,兩軍結東西之旅,一舉合連橫之勢,然後北破晉陽,還圖宗社。



 昉入佐相府,便為非法,三度事發,二度其婦自論。常云姓是「卯金刀」,名是「一萬日」,劉氏應王,為萬日天子。朕訓之導之,望其脩改。口請自新,志存如舊,亦與士彥情好深重,逆節姦心,盡探肝膈。嘗共士彥論太白所犯,問東井之間,思秦地之亂,訪軒轅之裏,願宮掖之災。唯侍蒲阪事興,欲在關內應接,殘賊之策,千端萬緒。



 惟忻及昉,名位並高,寧肯北面曲躬,臣於士彥?乃是各懷不遜,圖成亂階,一得擾攘之基,方逞吞
 并之事。士彥、忻、昉身為謀首,叔諧贊成父意,議實難容,並已處盡。士彥、忻、昉兄弟叔姪,特恕其命。



 臨刑,至朝堂,宇文忻見高熲,向之叩頭求哀。昉勃然謂忻曰:「事形如此,何叩頭之有!」於是伏誅,籍沒其家。後數日,帝素服臨射殿,盡取三家資物置於前,命百僚射取之,以為鑒戒云。



 柳裘,字茂和,河東解人,南齊司空世隆之曾孫也。祖惔,梁尚書左僕射。父明,太子舍人、義興太守。裘少聰慧,弱冠有令名。在梁,歷位尚書郎、駙馬都尉。



 梁元帝為魏軍所逼,遣裘請和於魏。俄而江陵平,遂入關中。周明、武間,
 自麟趾學士累遷太子侍讀,封昌樂縣侯。宣帝即位,進爵為公,轉御飾大夫。及帝不悆,留侍禁中,與劉昉、韋暮、皇甫績同謀引隋文帝,曰:「時不可失,今事已然,宜早定大計。天與不取,反受其殃。」帝從之。進上開府、內史大夫,委以機密。及尉遲迥作亂,天下騷動,并州總管李穆頗懷猶豫,帝令裘往喻之。裘見穆盛陳利害,穆遂歸心。以奉使功,賜彩三百匹,金九環帶一腰。時司馬消難奔陳,帝即令裘隨便安集淮南,賜馬及雜物。開皇元年,進位大將軍,拜許州刺史。在官清簡,人懷之,轉曹州刺史。後帝思裘定策功,欲加榮秩,將徵之,顧朝臣曰:「曹州刺史
 何當入朝」?或曰:「即今冬也。」乃止。裘尋卒,帝傷惜者久之,謚曰安。子惠音嗣。



 皇甫績,字功明,安定朝那人也。祖穆,魏隴東太守。父道,周湖州刺史、雍州都督。績三歲而孤,為外祖韋孝寬所鞠養。孝寬以諸子墮業,督以嚴訓,愍績孤幼,特捨之。績歎曰:「我無庭訓,養於外氏,不能克躬勵已,何以成立!」深自感激,命左右自杖三十。孝寬聞而對之流涕。於是專精好學,略涉經史。周武帝為魯公時,引為侍讀。建德初,轉宮尹中士。武帝嘗避暑雲陽宮,時宣帝為太子監國。



 衛剌王作亂,城門已閉,百僚多有遁者。績聞難赴之,於
 玄武門遇皇太子,下樓執績手,悲喜交集。帝聞而善之,遷小宮尹。宣政初,錄前後功,封義陽縣男,累轉御正下士。宣帝崩,隋文帝總己,績有力焉。加上開府,轉內史中大夫,進封郡公。



 拜大將軍。開皇元年,出為豫州刺史。尋拜都官尚書。轉晉州刺史。將之官,稽首言陳有三可滅。帝問其狀,績曰:「大吞小,一也。以有道伐無道,二也。納叛臣蕭巖,於我有詞,三也。陛下若命鷹揚之將,臣請預戎行。」上嘉勞而遣之。陳平,拜蘇州刺史。高智慧作亂江南,州人顧子元等發兵應之,因以攻績,相持八旬。子元素感績恩,於冬至日遣使奉牛酒。績遺之書。子元得書,於
 城下頓首陳謝。楊素援兵至,合擊破之。拜信州總管。俄以病乞骸骨,詔徵還京師,賜以御藥,中使相望,顧問不絕。卒於家,謚曰安。



 子偲嗣。大業中,位尚書主爵郎。



 郭衍,字彥文,自云太原介休人也。父崇,以舍人從魏孝武帝入關,位侍中。



 衍少驍武,善騎射。建德中,以軍功累遷儀同大將軍。又從周武帝平並州,以功加開府,封武強縣公,賜姓叱羅氏。宣政元年,為右中軍熊渠中大夫。尉遲迥之亂,從韋孝寬討之,以功授上柱國,封武山郡公。密勸隋文帝殺周室諸王,早行禪代,由是大被親暱。開皇元年,衍復舊姓為郭氏。突厥犯塞,以衍為行軍總
 管,領兵屯平涼。數歲,虜不入境。徵為開漕渠大監。部率水工,鑿渠引渭水,經大興城北,東至潼關,漕運四百餘里,關中賴之,名曰富人渠。五年,授瀛州刺史,遇秋霖大水,其屬縣多致漂沒,人皆上高樹,依大塚。衍親備船筏,並齊糧食拯救之,民多獲濟。衍先開倉賑恤,後始聞奏。上大善之,遷授朔州總管。所部有恆安鎮,北接蕃境,常勞轉運。衍乃選沃饒地,置屯田,歲嬴萬餘石,人免轉輸之勞。又築桑乾鎮,皆稱旨。十年,從晉王廣出鎮揚州。遇江表構逆,命衍為總管,先屯京口。於貴洲南與賊戰,敗之。仍討東陽、永嘉、宣城、黟、歙諸洞,盡平之。授蔣州刺
 史。



 衍臨下甚倨,事上甚卑。晉王愛暱之,宴賜隆厚。遷洪州總管。王有奪宗之謀,託衍心腹,遣宇文述以情告之。衍大喜曰:「若所謀事果,自可為皇太子。如其不諧,亦須據淮海,復梁、陳之舊。副君酒客,其如我何!」王因召衍,陰共計議。



 又恐人疑無故來往,託以妻患癭,王妃蕭氏有術能療之。以狀奏帝,聽共妻向江都,往來無度。衍又詐稱廣州俚反,王乃奏衍行兵討之。由是大脩甲仗,陰養士卒。及王入為太子,徵授左監門率,轉左宗衛率。文帝於仁壽宮將大漸,太子與楊素矯詔令衍、宇文述領東宮兵,帖上臺宿衛,門禁並由之。及上崩,漢王起逆,而京
 師空虛,使衍馳還,總兵居守。



 大業元年,拜左武衛大將軍。帝幸江都,令統左軍,改授光祿大夫,又從征吐谷渾,出金山道,納降二萬餘戶。衍能揣上意,阿諛順旨,帝每謂人曰:「唯郭衍心與朕同。」又嘗勸帝取樂,五日一視事,無得效高祖空自劬勞。帝從之,益稱其孝順。初,新令行,衍封爵從例除。六年,以恩舊封真定侯。從往江都,卒。贈左衛大將軍。謚曰襄。



 長子臻,武牙郎將。次子嗣本,孝昌令。



 張衡,字建平,河內人也。祖嶷,魏河陽太守。父允,周萬州刺史。衡幼懷志尚,有骨梗風。十五,詣太學受業,研精覃
 思,為同輩所推。周武帝居太后憂,與左右出獵,衡露髻輿櫬,扣馬切諫。帝嘉焉,賜衣一襲,馬一匹,擢拜漢王侍讀。



 衡又就沈重受《三禮》,略究大旨。累遷掌朝大夫。



 隋文帝受禪,拜司門侍郎。及晉王廣為河北行臺,衡歷刑部、度支二曹郎。行臺廢,拜並州總管掾。王轉牧揚州,衡復為掾。王甚親任之,衡亦竭慮盡誠。奪宗之計,多衡所建。遷揚州總管司馬。熙州李英林反,署置百官,以衡為行軍總管討平之,拜開府。及王為皇太子,拜衡右庶子。



 煬帝嗣位,除給事黃門侍郎、銀青光祿大夫。遷御史大夫,甚見親重。大業三年,帝幸榆林郡,還至太原,謂衡曰:「朕
 欲過公宅,可為朕作主人也。」衡馳至河內,與宗族具牛酒。帝上太行,開直道九十里,以抵其宅。帝悅其山泉,留宴三日,因謂衡曰:「往從先皇拜太山之始,途經洛陽,瞻望於此,深恨不得相過,不謂今日得諧宿願。」衡俯伏辭謝,奉觴上壽。帝益歡,賜其宅傍田三十頃、良馬一匹、金帶、縑彩六百段、衣一襲、御食器一具。衡固讓,帝曰:「天子所至稱幸者,蓋為此也,不足為辭。」衡復獻食於帝,帝令頒賜公卿,下至衛士,無不霑給。衡以籓邸之舊,恩寵莫與為比,頗自驕貴。明年,帝幸汾陽宮。時帝欲大汾陽宮,令衡與紀弘整具圖奏之。衡承間進諫,以比年勞役,百
 姓疲敝為請。帝意甚不平。後嘗目衡謂侍臣曰:「張衡自謂由甚計畫,令我有天下。」時齊王暕失愛於上,帝密令人求其罪。有人譖暕違制,將伊闕令皇甫詡從之汾陽宮。又錄前幸涿郡及祠恆岳時,父老謁見者,衣冠不整。帝譴衡以憲司皆不能舉正,出為榆林太守。



 明年,帝復幸汾陽宮,衡督役築樓煩城,因而謁帝。帝惡衡不損瘦,以為不念咎,因謂曰:「公甚肥澤,宜且還郡。」衡復之榆林。俄而敕衡督役江都宮。有人詣衡訟宮監者,衡不為理,還以訟書付監,其人大為監所困。禮部尚書楊玄感使至江都,其人詣玄感稱冤。玄感固以衡為不可。」及與相
 見,未有所言,又先謂玄感曰:「薛道衡真為枉死。」玄感具上其事。江都郡丞王世充又奏衡頻減頓具。帝怒,鎖衡詣江都市,將斬之。既而除名,放還田里。帝每令親人覘衡所為。



 八年,帝自遼東還都,妄言衡怨望,謗訕朝政,帝賜死於家。臨死,大言曰:「我為人作何物事,而望久活!」監刑者塞耳,促令殺之。武德初,以為死非其罪,贈大將軍、南陽郡公,謚曰忠。子希玄。



 楊汪,字元度,本弘農華陰人也。曾祖順,居河東。父琛,儀同三司。及汪貴,追贈平鄉縣公。汪少凶疏,與人群鬥,拳所毆擊,無不顛踣。長更折節勤學,專精《左氏傳》,通《三禮》。
 解褐周冀王侍讀,王甚重之,每曰:「楊侍讀德業優深,孤之穆生也。」後問《禮》於沈重,受《漢書》於劉臻,二人曰:「吾弗如也。」



 由是知名。累遷夏官府都上士。



 隋文帝居相,引知兵事,遷掌朝下大夫。及受禪,賜爵平鄉縣伯,歷秦州總管府長史。每聽政暇,必延生徒講授,時人稱之。入為尚書兵部侍郎。數年,帝謂諫議大夫王達曰:「卿為我覓一好左丞。」達遂私於汪曰:「我當薦君為左丞,若事果,當以良田相報也。」汪以達言奏之,達竟獲罪,卒拜汪尚書左丞。汪明習法令,果於剖斷,當時號為稱職。未幾,坐事免。後拜洛州長史,轉荊州長史。煬帝即位,追為尚書左丞,
 尋守大理卿。視事二日,帝將親省囚徒。時繫囚二百餘人,汪通宵究審,詰朝而奏,曲盡事情,一無遺誤,帝甚嘉之。歲餘,拜國子祭酒。帝令百僚就學,與汪講論。天下通儒碩學多萃焉,論難鋒起,皆不能屈。帝令御史書其問答奏之,省而大悅,賜良馬一匹。後加銀青光祿大夫。



 及楊玄感反,河南贊務裴弘策出師禦之,戰不利,奔還,遇汪而屏人交語。既而留守樊子蓋斬弘策,以狀奏汪,帝疑之,出為梁郡通守。後煬帝崩,王世充推越王侗為主,徵拜吏部尚書,頗見親委。及世充僭號,汪復用事。世充平,遂以兇黨伏誅。



 裴蘊,河東聞喜人也。祖之平,父忌,並《南史》有傳。忌在陳,與吳明徹同見俘于周,周賜爵江夏公,在隋十餘年而卒。蘊明辯有吏乾,仕陳,歷直閣將軍、興寧令。以父在北,陰奉表於隋文帝,請為內應。及陳平,上悉閱江南衣冠之士,次至蘊,以夙有向化心,超授儀同。僕射高熲不悟上旨,諫曰:「蘊無功於國,寵踰倫輩,臣未見其可。」又加上儀同,復諫。上曰:「可加開府。」乃不敢復言。



 即日拜開府儀同三司,禮賜優洽。歷洋、直隸三州刺史,俱有能名。



 大業初,考績連最。煬帝聞其善政,徵為太常少卿。初,文帝不好聲技,遣牛弘定樂,非正聲清商及九部四舞之
 色,皆罷遣從百姓。至是,蘊揣知帝意,奏括天下周、齊、梁、陳樂家子弟,皆為樂戶。其六品已下,至於凡庶,有善音樂及倡優百戲者,皆直太常。是後異技淫聲咸萃樂府,皆置博士,遞相教傳,增益樂人至三萬餘。帝大悅,遷戶部侍郎。時猶承文帝和平後,禁網疏闊,戶口多漏。或年及成丁,猶詐為小,未至於老,已免租賦。蘊歷為刺史,素知其情,因是條奏,皆令貌閱。若一人不實,則官司解職,鄉正、里長皆遠流配。又許民相告,若糾得一丁者,令被糾之家代輸賦役。是歲大業五年也。諸郡計帳,進丁二十四萬三千,新附口六十四萬一千五百。帝臨朝覽狀,
 謂百官曰:「前代無好人,致此罔冒。今進民口皆從實者,全由裴蘊一人用心。古語云,得賢而理,驗之信矣。」由是漸見親委,拜京兆贊務,發手適纖毫,吏民懾憚。



 未幾,擢授御史大夫,與裴矩、虞世基參掌機密。蘊善候伺人主微意,若欲罪者,則曲法順情,鍛成其罪;所欲宥者,則附從輕典,因而釋之。是後大小之獄皆以付蘊,憲部、大理莫敢與奪,必稟承進止,然後決斷。蘊亦機辯,所論法理,言若懸河,或重或輕,皆由其口,剖析明敏,時人不能致詰。楊玄感之反也,帝遣蘊推其黨與,謂蘊曰:「玄感一呼,從者十萬。益知天下人不欲多,多即相聚為盜耳。



 不盡加
 誅,則後無以勸。」蘊由是乃峻法理之,所戮者數萬人,皆籍沒其家。帝大稱善,賜奴婢十五口。司隸大夫薛道衡以忤意獲譴,蘊知帝惡之,乃奏曰:「道衡負才恃舊,有無君之心。見詔書每下,便腹非私議,推惡於國,妄造禍端。論其罪名,似如隱昧,源其情意,深為悖逆。」帝曰:「然。我少時與此人相隨行役,輕我童稚,共高熲、賀若弼等外擅威權。自知罪當誣罔,及我即位,懷不自安,賴天下無事,未得反耳。公論其逆。妙體本心。」於是誅道衡。又帝問蘇威以討遼之策,威不願帝復行,且欲令帝知天下多賊,乃詭答:「今者之役,不願發兵,但詔赦群盜,自可得數十
 萬。遣關內奴賊及山東歷山飛、張金稱等頭別為一軍,出遼西道;諸河南賊王薄、孟讓等十餘頭,並給舟楫,浮滄海道。必喜於免罪,競務立功,一歲之間,可滅高麗矣。」帝不懌曰:「我去尚猶未克,鼠竊安能濟乎!」威出後,蘊奏曰:「此大不遜,天下何處有許多賊!」帝悟曰:「老革多姦,將賊脅我。欲搭其口,但隱忍之,誠極難耐。」蘊知上意,遣張行本奏威罪惡,帝付蘊推鞫之,乃處其死。帝曰:「未忍便殺。」遂父子及孫三世並除名。



 蘊又欲重己權勢,令虞世基奏罷司隸刺史以下官屬,增置御史百餘人。於是引致姦黠,共為朋黨,郡縣有不附者,陰中之。于時軍國多
 務,凡是興師動眾,京都留守,及與諸蕃互市,皆令御史監之。賓客附隸,遍於郡國,侵擾百姓,帝弗之知也。以度遼之役,進位銀青光祿大夫。及司馬德戡將為亂也,江陽長張惠紹夜弛告之。蘊共惠紹謀,欲矯詔發郭下兵民,盡取榮公護兒節度,收在外逆黨宇文化及等,仍發羽林殿腳,遣范富婁等入自西苑,取梁公蕭鉅及燕王處分,扣門援帝。謀議已定,遣報虞世基。世基疑反者不實,抑其計。須臾,難作。蘊嘆曰:「謀及播郎,竟誤人事!」遂見害。子愔,為尚輦直長,亦同日死。



 袁充,字德符,本陳郡陽夏人也。其後寓居丹陽。祖昂,父
 君正,俱為梁侍中。



 充少警悟,年十餘歲,其父黨至門,時冬初,充尚衣葛衫。客戲充曰:「袁郎子,絺兮綌兮,淒其以風。」充應聲答曰:「唯絺與綌,服之無斁。」以是大見嗟賞。



 仕陳,年十七,為秘書郎。歷太子舍人、晉安王文學、吏部侍郎、散騎常侍。及陳滅歸國,歷蒙、鄜二州司馬。充性好道術,頗解占候,由是領太史令。時上將廢皇太子,正窮東宮官屬,充見上雅信符應,因希旨進曰:「比觀玄象,皇太子當廢。」



 上然之。充復表奏隋興以後,日景漸長,曰:「開皇元年,冬至日影一丈二尺七寸二分,自爾漸短。至十七年,冬至影一丈二尺六寸三分。四年冬至,在洛陽測影,
 一丈二尺八寸八分。二年,夏至影一尺四寸八分,自爾漸短。至十六年,夏至影一尺四寸五分。《周官》以土圭之法正日影,日至之影尺有五寸。鄭玄云:『冬至之影一丈三尺。』今十六年夏至之影,短於舊影五分,十七年冬至之影,短於舊影三寸七分。日去極近,則影短而日長;去極遠,則影長而日短。行內道,則去極近;外道,則去極遠。《堯典》曰:『日短星昴,以正仲冬。』據昴星昏中,則知堯時仲冬,日在須女十度。以曆數推之,開皇已來冬至,日在斗十一度,與唐堯之代,去極並近。謹案《春秋元命包》云:『日月出內道,璇璣得常,天帝崇靈,聖王相功。』京房《別對》曰:『太平
 日行上道,升平行次道,霸世行下道。』伏惟大隋啟運,上感乾元,影短日長,振古未之有也。」上大悅,告天下。將作役功,因加程課,丁匠苦之。



 仁壽初,充言上本命與陰陽律呂合者六十餘條而奏之,因上表曰:「皇帝載誕之初,非止神光瑞氣,嘉祥應感。至於本命行年,生月生日,並與天地日月、陰陽律呂,運轉相符,表裏合會。此誕聖之異,寶曆之元。今與物更新,改年仁壽,歲月日子,還共誕聖之時並同,明合天地之心,得仁壽之理。故知洪基長算,永永無窮。」上大悅,賞賜優崇,儕輩莫之比。



 仁壽四年甲子歲,煬帝初即位,充及太史丞高智寶奏言:「去歲冬
 至,日景逾長。今歲皇帝即位,與堯受命年合。昔唐堯受命四十九年,到上元第一紀甲子,天正十一月庚戍冬至;陛下即位,其年即當上元第一紀甲子,天正十一月庚戍冬至,正與唐堯同。自放勛以來,凡經八上元,其間綿代,未有仁壽甲子之合。謹案:第一紀甲子,太一在一宮,天目居武德,陰陽歷數,並得符同唐堯。唐堯丙辰生,丙子年受命,止合三五。未若己丑甲子,支乾並當六合。允一元三統之期,合五紀九章之會,共帝堯同其數,與皇唐比其蹤。信所謂皇哉唐哉,唐哉皇哉者矣。」仍諷齊王暕率百官拜表奉賀。後熒惑守太微者數旬,時繕脩
 宮室,征役繁重,充乃上表稱「陛下脩德,熒惑退舍」。百僚畢賀。帝大喜,前後賞賜將萬計。時軍國多務,充候帝意欲有所為,便奏稱天文見象,須有改作,以是取媚於上。大業六年,遷內史舍人。從征遼東,拜朝請大夫、秘書少監。



 後天下大亂,帝初罹鴈門之厄,又盜賊益起,心不自安。充復託天文,上表陳嘉瑞以媚上曰:伏惟陛下握錄圖而馭黔首,提萬善而化八紘,以百姓為心,匪一人受慶,先天罔違所欲,後天必奉其時。是以初膺寶曆,正當上元之紀;乾之初九,又與本命符會。斯則聖人冥契,故能動合天經。謹案去年已來,玄象星瑞,毫釐無爽。謹錄
 尤異,上天降祥、破突厥等狀七事。



 其一,去八月二十八日夜,大流星如斗,出王良北,正落突厥營,聲如崩牆。



 其二,八月二十九日夜,復有大流星如斗,出羽林,向北流,正當北方。依占,頻二夜流星墜賊所,賊必敗散。其三,九月四日夜,頻有兩星大如斗,出北斗魁,向東北流。依占,北斗主殺伐,賊必破敗。其四,歲星主福德,頻行京都二處分野。



 依占,國家之福。其五,去七月內,熒惑守羽林,九月七日已退舍。依占,不出三日,賊必敗散。其六,去年十一月二十日夜,有流星赤如火,從東北向西南,落賊帥盧明月營,破其橦車。其七,十二月十五日夜,通漢鎮北
 有赤氣互北方,突厥將亡之應也。依勘《城錄》,河南、洛陽並當甲子,與乾元初九爻及上元甲子符合。



 此是福地,永無所慮。旋觀往政,側聞前古,彼則異時間出,今則一朝總萃。豈非天贊有道,助殲兇孽?方清九夷於東濊,沉五狄於北溟,告成岱岳,無為汾水。



 書奏,帝大悅,超拜秘書令。親待逾暱,每欲征討,充皆預知之,乃假託星象,獎成帝意,在位者皆切患之。宇文化及弒逆之際,并誅充。



 李雄,勃海蓚人也。父棠,名列《誠義傳》。雄少慷慨,有壯志。弱冠,從周武帝平齊,以功授帥都督。隋文帝作相,從韋孝寬破尉遲迥,拜上開府,賜爵建昌縣公。伐陳之後,以
 功進位大將軍。歷郴江二州刺史,並有能名。後坐事免。漢王諒之反,煬帝將發幽州兵討之。時竇抗為幽州總管,帝恐其貳,問可任者於楊素。



 素遂進雄,授上大將軍,拜廉州刺史。馳至幽州,止傳舍,召募得千餘人。抗恃素貴,不時相見。雄遣人諭之,後二日,抗從鐵騎二千來詣雄所。雄伏甲禽抗,悉發幽州兵步騎三萬,自井陘討諒。遷幽州總管。尋徵拜戶部尚書。雄明辯有器幹,帝甚任之。新羅嘗遣使朝貢,雄至朝堂與語,因問其冠制所由。其使者曰:「古弁遺象,安有大國君子不識?」雄因曰:「中國無禮,求諸四夷。」使者曰:「自至已來,此言外未見無禮。」憲司
 以雄失辭,奏劾其事,竟坐免。俄而復職。從幸江都,帝以仗衛不整,顧雄部伍之。雄立指麾,六軍肅然。帝大悅曰:「公真武侯才也。」



 尋轉右候衛大將軍。復坐事除名。遼東之役,帝令從軍自效,因從來護兒自東萊將指滄海。會楊玄感反於黎陽,帝疑之,詔鎖雄送行在所。雄殺使亡歸玄感,玄感每與計焉。及玄感敗,伏誅,籍沒其家。



 論曰:隋文肇基王業,劉昉實啟其謀,於時當軸執鈞,物無異論。不能忘身急病,以義斷恩,方乃慮難求全,偷安懷祿。其在周也,靡忠貞之節;其奉隋也,愧竭命之誠。非義掩其前功,蓄怨興其後釁,而望不陷刑辟,保貴全生,
 難矣。柳裘、皇甫績,因人成事,好亂樂禍,大運光啟,並參樞要。斯固在人欲其悅己,在我欲其罵人,理自然也。晏嬰有言曰:「一心可以事百君,百心不可以事一君。」於昉等見之矣。郭衍,文皇締構之始,當爪牙之寄;煬帝經綸之際,參心膂之謀。而如脂如韋,以水濟水,君所謂可,亦曰可焉,君所謂不,亦曰不焉,功雖居多,名不見重。然則立身行道,可不慎歟!語曰:「無為權首,將受其咎。」又曰:「無始禍,無兆亂。」夫忠為令德,施非其人尚或不可,況託足邪徑,又不得其人者歟!



 張衡奪宗之計,實兆其謀,夫動不以順,能無及於此也?楊汪以學業自許,其終不令,惜
 乎!裴蘊素懷奸險,巧於附會,作威作福,唯利是視,滅亡之禍,其可免乎!



 袁充少在江東,初以警悟見許,委質隋氏,更以玄象自矜,要求時幸,干進附入,變動星占,謬增晷景,厚誣天道,亂常侮眾。刑茲勿舍,其在斯乎!李雄斯言為玷,取譏夷翟,以亂從亂,何救誅夷。



\end{pinyinscope}