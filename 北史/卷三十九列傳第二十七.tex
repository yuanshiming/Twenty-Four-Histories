\article{卷三十九列傳第二十七}

\begin{pinyinscope}

 薛安都劉休賓房法壽曾孫豹玄孫彥謙族子景伯畢眾敬曾孫義雲羊祉子深孫肅弟子敦烈薛安都,字休達,河東汾陰人也。父廣,晉上黨太守。安都少驍勇,善騎射,頗結輕俠,諸兄患之。安都乃求以一身分出,不取片資,兄許之,居於別廄。遠近交遊者爭有送遺,馬牛衣服什物充滿其庭。真君五年,與東雍州刺史沮渠康謀逆,事發奔宋。



 在南以武力見敘,遇宋孝武起
 江州,遂以為將。和平六年,宋湘東王殺其主子業而自立,是為明帝。群情不協,共立子業弟晉安王子勛。安都與沈文秀、崔道固、常珍奇等舉兵應之。宋明帝遺將張永討安都。安都遣使降魏,請兵救援,遣第四子道次為質。獻文乃遣鎮東大將軍尉元等赴之,拜安都鎮南大將軍、徐州刺史,賜爵河東公。元等既入彭城,安都中悔,謀圖元等。元知之,遂不果發。安都因重貨元等,委罪於女婿裴祖隆。元乃殺祖隆而隱安都謀。



 皇興二年,與畢眾敬朝于京師,甚見禮重。子姪群從並處上客,皆封侯,至於門生,無不收敘。又為起第宅,館宇崇麗,資給甚厚。
 卒,贈假黃鉞、秦州刺史、河東王,謚曰康。



 子道襲爵,位平州刺史,政有聲稱。歷相、秦二州刺史,卒。道弟道異,亦以勛為第一客。早卒,贈秦州刺史、安邑侯。道異弟道次,既質京師,賜爵安邑侯,位秦州刺史,進河南公。



 安都從祖弟真度,初亦與安都南奔;及從安都來降,為上客。太和初,賜爵河北侯,出為平州刺史,假陽平公,後降為伯。歷荊州、東荊州刺史。初遷洛後,真度每獻計勸先取樊、鄧,後攻南陽,故大為帝所賞。改封臨晉縣伯,轉豫州刺史。



 景明初,豫州大饑,真度表輒日別出倉米五十斛為粥,救其甚者。詔曰:「真度所表,甚有憂濟百姓之意,宜
 在拯恤。」歷華、荊二州刺史,入為大司農卿。正始初,除揚州刺史。還朝,除金紫光祿大夫,加散騎常侍,改封敷西。
 卒,贈
 左光祿大夫,謚曰莊。有子十二人,嫡子懷徹襲封。



 初,真度有女妓數十人。每集賓客,輒命之絲竹歌舞,不輟於前,盡聲色之適。



 庶長子懷吉,居喪過周,以父妓十餘人并樂器獻之,宣武納焉。



 懷吉好勇,有膂力,雖不善書學,亦解達時事。卒於汾州刺史。懷吉本不厲清節,及為汾州,偏有聚納之響。自以支庶,餌誘勝己,共為婚姻。多攜親戚,悉令同行,兼為之彌縫,恣其取受。而將勞賓客,曲盡物情,送去迎來,不避寒熱。性少言,每有接對,但
 默然而返。既指授先期明人馬之數,左右密已記錄。俄而酒饌相尋,芻粟繼至,逮于將別,贈以錢縑,下及廝庸,咸過本望。



 真度諸子既多,其母非一;同產相朋,因有憎愛。興和中,遂致訴列,云以毒藥相害。顯在公府,發揚疵釁,時人恥焉。



 劉休賓,字處乾,本平原人也。祖昶,從慕容德度河,家于北海都昌縣。父奉伯,宋北海太守。休賓少好學,有文才。仕宋為兗州刺史。娶崔邪利女,生子文曄。



 崔氏先歸寧在魯郡。邪利之降,文曄母子與俱入魏。及慕容白曜軍至,休賓不降。



 白曜請崔氏與文曄至,以報休賓。又執休
 賓兄延和妻子巡視城下。休賓答白曜,許待歷城降,當即歸順。密遣主簿尹文達向歷城,觀魏軍形勢。文達詣白曜,詐祗侯。



 白曜令文達往升城,見其妻子。文曄哭泣,以爪髮為信。文達回,復經白曜,誓約而還,見休賓。休賓撫爪髮泣,復遣文達與白曜期。白曜喜,以酒灌地,啟告山河,誓不負休賓。文達還謂休賓,可早決計。休賓於是告兄子聞慰。聞慰固執不可,遂差本契。白曜尋遣著作佐郎許赤彪夜至梁鄒南門,告城上人曰:「休賓遣文達頻造僕射許降,何得無信!」於是城內遂相維持,欲降不得。歷城降,休賓乃出請命。



 及立平齊郡,乃以梁鄒人為
 懷寧縣,以休賓為令。延興二年卒。



 文曄有志尚,綜覽群書,輕財重義。太和中,坐從兄聞慰南叛,被徙北邊,孝文特聽還代。帝曾幸方山,文曄大言求見,申父功厚賞屈。於是賜爵都昌子,深見待遇,拜協律中郎。卒於高陽太守,贈兗州刺史,謚曰貞。



 休賓叔父旋之,其妻許氏生二子法鳳、法武,而旋之早卒。東陽平,許氏攜二子入魏,孤貧不自立,母子並出家為尼僧。既而反俗,俱奔江南。法武後改名峻,字孝標,《南史》有傳。



 房法壽,小名烏頭,清河東武城人也。曾祖諶,仕燕,位太尉掾。隨慕容氏遷于齊,子孫因家之,遂為東清河繹幕
 人焉。法壽幼孤,少好射獵,輕率勇果,結諸群小為劫盜,宗族患之。弱冠,州迎主簿。後以母老,不復應州郡命,常盜殺豬羊以供母。招集壯士,恆有數百。仕宋為魏郡太守。法壽從祖弟崇吉,母妻為慕容白曜所獲,託法壽為計,法壽與崇吉歸款於白曜。詔以法壽為平遠將軍,與韓騏驎對為冀州刺史。及歷城、梁鄒降,法壽、崇吉等與崔道固、劉休賓俱至京師。以法壽為上客,崇吉為次客,崔、劉為下客。法壽供給亞於薛安都等,以功賜爵壯武侯,給以田宅奴婢。性愛酒,好施,親舊賓客率同飢飽,坎壈常不豐足。畢眾敬等皆尚其通愛。卒,贈青州刺史,謚
 敬侯。



 子伯祖襲,例降為伯,歷齊郡內史。伯祖訚弱,委事於功曹張僧皓,大有受納,伯祖衣食不充。後卒於幽州輔國府長史,免官,卒。子翼,大城戍主,帶宗安太守,襲爵壯武侯。



 翼子豹,字仲幹。體貌魁岸,美音儀。年十七,州辟主簿。王思政入據潁川,慕容紹宗出討,豹為紹宗開府主簿兼行臺郎中。紹宗自云有水厄,遂於戰艦中浴,并自投於水,冀以厭當之。豹白紹宗曰:「夫命也在天,豈人理所能延保。公若實有水厄,非禳辟所能卻;若其實無,何禳之有。今三軍之事,在於明公,唯應達命任理,以保元吉。方
 乃乘船入水,云以防災,豈如岸上指麾,以保萬全也。」紹宗笑曰:「不能免俗,為復爾耳。」未幾而紹宗遇溺,時論以為知微。清河中,除謁者僕射,拜西河太守。地接周境,俗雜稽胡,豹政貴清靜,甚著聲績。遷博陵太守,亦有能名。又遷樂陵太守,風教修理,稱為美政。郡瀕海,水味多鹹苦。豹命鑿一井,遂得甘泉,遐邇以為政化所致。豹罷歸後,井味復鹹。齊滅,遂還本鄉,丘園自養。頻被徵命,固辭以疾。每牧守初臨,必遣致禮,官佐邑宰皆投刺申敬。終於家,無子,以兄熊子彥詡嗣。彥詡明辯有學識,位殿中侍御史,千乘、益都二縣令,有惠政。熊字子威,性至孝,聰
 郎有節概。州辟主簿,行清河、廣川二郡事。七子。



 長子彥詢最知名,以魏勛門嫡孫,賜爵永始縣子,特為叔豹所愛重。病卒,豹取急,親送柩還鄉;悲痛傷惜,以為喪當家之寶。初,彥詢少時為監館,嘗接陳使江總。及陳滅,總入關,見彥詢弟彥謙曰:「公是監館弟邪?」因慘然曰:「昔因將命,得申言款。」彥詢所贈總詩,今見載《總集》。



 彥謙早孤,不識父,為母兄鞠養。長兄彥詢,雅有清鑒,以彥謙天性穎悟,每奇之,親教讀書。年七歲,誦數萬言,為宗黨所異。十五出後叔父子貞,事所繼有踰本生。子貞哀之,撫養甚厚。後丁繼母憂,勺飲不入口者五日。事伯父豹,竭盡心
 力,每四時珍果,弗敢先嘗。遇期功之戚,必蔬食終禮,宗從取則焉。其後受學于博士尹琳,手不釋卷,遂通涉《五經》。解屬文,雅有詞辯,風概高人。



 年十八,屬齊廣寧王孝珩為齊州刺史,辟為主簿。時禁網疏闊,州郡之職,尤多縱弛。及彥謙在職,清簡守法,州境肅然,莫不敬憚。及周師入鄴,齊主東奔,以彥謙為齊州中從事。彥謙痛本朝傾覆,將糾率忠義,潛謀匡輔,事不果而止。齊亡,歸於家。周武帝遣柱國辛遵為齊州刺史,為賊帥輔帶劍所執。彥謙以書諭之,帶劍慚懼,送遵還州,諸賊並各歸首。及隋文受禪之後,遂優游鄉曲,誓無仕心。



 開皇七年,刺史
 韋藝固薦之,不得已而應命。吏部尚書盧愷一見重之,擢授承奉郎,俄遷監察御史。後屬陳平,奉詔安撫泉、括等十州。以銜命稱旨,賜物百段、米百石、衣一襲、奴婢七口。



 遷秦州總管錄事參軍。因朝集時,左僕射高熲定考課。彥謙謂熲曰:「《書》稱三載考績,黜陟幽明。唐、虞以降,代有其法,黜陟合理,褒貶無虧,便是進必得賢,退皆不肖。如或舛謬,法乃虛設。比見諸州考校,執見不同,進退多少,參差不類。況復愛憎肆意,致乖平坦。清介孤直,未必高第;卑諂巧官,翻居上等。



 真偽混淆,是非瞀亂。宰貴既不精練,斟酌取捨,曾經驅使者,多以蒙識獲成;未歷臺
 省者,皆為不知被退。又四方懸遠,難可詳悉,唯準量人數,半破半成。徒計官員之少多,莫顧善惡之眾寡。俗求允當,其道無由。明公鑒達幽微,平心遇物,今年考校,必無阿枉,脫有前件數事,未審何以裁之?唯顧遠布耳目,精加採訪。



 褒秋毫之善,貶纖介之惡。非直有光至道,亦足標獎賢能。」詞氣侃然,觀者屬目。



 熲為之動容,深見嗟賞。因歷問河西、隴右官人景行,彥謙對之如響。熲謂諸州總管、刺史曰:「與公言,不如獨共秦州考使語。」後數日,熲言於帝,帝弗能用。



 以秩滿,遷長葛縣令,甚有惠化,百姓號為慈父。仁壽中,帝令持節使者巡行州縣,察長吏
 能不。以彥謙為天下第一,超授鄀州司馬。吏人號哭相謂曰:「房明府今去,吾屬何用生為!」其後百姓思之,立碑頌德。鄀州久無刺史,州務皆歸彥謙,名有異政。內史侍郎薛道衡,一代文宗,位望清顯。所與交結,皆海內名賢。



 重彥謙為人,深加友敬。及為襄州總管,辭翰往來,交錯道路。煬帝嗣位,道衡轉牧番州,路經彥謙所,留連數日,屑涕而別。



 黃門侍郎張衡亦與彥謙相善。于時帝營東都,窮極侈麗,天下失望。又漢王構逆,罹罪者多。彥謙見衡當塗而不能匡救,書諭之曰:竊聞賞者所以勸善,刑者所以懲惡。故疏賤之人,有善必賞;尊賢之戚,犯惡必
 刑。未有罰則避親,賞則遺賤者也。今國家祗承靈命,作人父母,刑賞曲直,升聞於天,夤畏照臨,亦宜謹肅。故文王云:「我其夙夜畏天之威。」以此而論,雖州、國有殊,高下懸邈,憂人慎法,其理一也。



 至如并州釁逆,須有甄明。若楊諒實以詔命不通,慮宗社危逼,徵兵聚眾,非為干紀,則當原其本情,議其刑罰;上副聖主友于之意,下曉愚人疑惑之心。若審知外內無虞,嗣后纂統,而好亂樂禍,妄有覬覦,則管、蔡之誅,當在於諒。同惡相濟,無所逃罪;梟縣孥戮,國有常刑。遂使籍沒流移,恐為冤濫。恢恢天網,豈其然乎!罪疑從輕,斯義安在!昔叔向置鬻獄之死,
 晉國所嘉;釋之斷犯蹕之刑,漢文稱善。羊舌寧不愛弟,廷尉非茍違君,俱以執法無私,不容輕重。



 且聖人大寶,是曰神器,茍非天命,不可妄得。故蚩尤、項籍之驍勇,伊尹、霍光之權勢,李老、孔丘之才智,呂望、孫武之兵術,吳、楚連盤石之據,產、祿承母弟之基,不應歷運之兆,終無帝主之位。況乎蕞爾一隅,蜂扇蟻聚,楊諒之愚鄙,群小之凶慝,而欲憑陵畿甸,覬幸非望者哉。開闢以降,書契云及,帝皇之跡,可得而詳。自非積德累仁,豐功厚利,孰能道洽幽顯,義感靈祗?是以古之哲王,昧旦丕顯,履冰在念,御朽兢懷。逮叔世驕荒,曾無戒懼,肆於人上,騁嗜
 奔欲,不司具載,謂略陳之。



 曩者,齊、陳二國,並居大位。自謂與天地合德,日月齊明,罔念憂虞,不恤刑政。近臣懷寵,稱善而隱惡;史官曲筆,掩瑕而錄美。是以人庶呼嗟,終閉塞於視聽;公卿虛譽,日敷陳於左右。法網嚴密,刑辟日多,賦役煩興,老幼疲苦。昔鄭有子產,齊有晏嬰,楚有叔敖,晉有士會,凡此小國,尚足名臣,齊、陳之強,豈無良佐?但以執政壅蔽,懷私殉軀,忘國憂家,外同內忌。設有正直之士,才堪乾時,於己非宜,即加擯棄;儻遇諂佞之輩,行多穢慝,於我有益,遽蒙薦舉。以此求賢,何從而至。夫賢材者,非尚膂力,豈繫文華,唯須正身負戴。確乎
 不動,譬棟之處屋,如骨之在身,所謂棟梁骨鯁之材也。齊、陳不任骨鯁,信近讒諛,天高聽卑,監其淫僻。故總收神器,歸我大隋。向使二國祗敬上玄,惠恤鰥寡,委任方直,斥遠浮華,卑菲為心,惻隱是務,河朔強富,江湖險隔,各保其業,人不思亂,泰山之固,弗可動也。然而寢臥積薪,宴安鴆毒,遂使禾黍生廟,務露沾衣,吊影撫心,何嗟及矣!故《詩》云:「殷之未喪師,克配上帝。宜鑒于殷,駿命不易。」萬機之事,何者不須熟慮哉。



 伏惟皇帝望雲就日,仁孝夙彰,錫社分珪,大成規矩。及總統淮海,盛德日新,當璧之符,遐邇僉屬。纘曆甫爾,寬仁已布,率土蒼生,翹足
 而喜。並州之亂,變起倉卒,職由楊諒詭惑,詿誤吏人;非有構怨本朝,棄德從賊者也。而有司將帥,稱其願反,非止誣陷良善,亦恐大玷皇猷。



 足下宿當重寄,早預心膂,粵自籓邸,柱石見知,方當書名竹帛,傳芳萬古,稷、契、伊、呂,彼獨何人。既屬明時,須存謇諤,立當世之大誡,作將來之憲範,豈容曲順人主,以愛虧刑;又使脅從之徒,橫貽罪譴。忝蒙眷遇,輒寫微誠,野人愚瞽,不知忌諱。



 衡得書,歎息而不敢奏聞。



 彥謙知王綱不振,遂去官,隱居不仕。將結構蒙山之下,以求其志。會置司隸官,盛選天下知名之士。朝廷以彥謙公方宿著,時望所歸,徵授司隸
 刺史。彥謙亦慨然有澄清天下之志,凡所薦舉,皆人倫表式。其有彈射,當之者曾無怨言。司隸別駕劉灹陵上侮下,訐以為直,刺史憚之,皆為之拜。唯彥謙執志不撓,抗禮長揖。



 有識嘉之,灹亦不恨。



 大業九年,從駕度遼,監扶餘道軍事。其後隋政漸亂,莫不變節,彥謙直道守常,頗為執政者所嫉。出為涇陽令,終於官。



 彥謙居家,每子姪定省,常為講說督勉之,亹癖不倦。家有舊業,資產素殷,又前後居官所得俸祿,皆以周恤親友,家無餘財。車服器用,務存素儉。自少及長,一言一行,未嘗涉私。雖致屢空,怡然自得。嘗從容獨笑,顧謂其子玄齡曰:「人皆因
 祿富,我獨以官貧。所遺子孫,在於清白耳。」所有文筆,恢廓閑雅,有古人之深致。又善草隸,人有得其尺牘者,皆寶玩之。太原王劭、北海高構、蓨縣李綱、中山郎茂、郎穎、河東柳彧、薛孺,皆一時知名雅澹之士,彥謙並與為友。雖冠蓋成列,而門無雜賓。體資文雅,深達政務,有識者咸以遠大許之。



 初,開皇中平陳之後,天下一統,論者咸云將致太平。彥謙私謂所親趙郡李少通曰:「主上性多忌剋,不納諫諍。太子卑弱,諸王擅威。在朝惟行苛酷之政,未弘遠大之體,天下雖安,方憂危亂。」少通初謂不然。及仁壽、大業之際,其言皆驗。貞觀初,以子玄齡著勛庸,
 贈徐州都督、臨淄縣公,謚曰定。



 伯祖弟幼愍,安豐、新蔡二郡太守,坐事奪官。居家,忽聞門有客聲,出無所見,還至庭中,為家群犬所噬,卒。



 景伯字良暉,法壽族子也。祖元慶,仕宋。歷七郡太守,後為沈文秀青州建威府司馬。宋明帝之殺廢帝子業,子業弟子勛起兵。文秀後歸子勛,元慶不同,為文秀所害。父愛親,獻文時,三齊平,隨例內徙,為平齊人。以父非命,疏服終身。



 景伯生於桑乾,少喪父,以孝聞。家貧,傭書自給,養母甚謹。尚書盧陽烏稱之於李沖。沖時典選,拔為奉朝請。累遷齊州輔國長史。會刺史亡,敕行州事。政存
 寬簡,百姓安之。後除清河太守。郡人劉簡武曾失禮於景伯,聞其臨郡,闔家逃亡。景伯督切屬縣,追捕禽之。即署其子為西曹掾,令喻山賊。賊以景伯不念舊惡,一時俱下,論者稱之。舊制,守令六年為限。限滿將代,郡人韓靈和等三百餘人表訴乞留,復加二載。後為司空長史,以母疾去官。



 景伯性復淳和。涉獵經史,諸弟宗之,如事嚴親。及弟亡,蔬食終喪,期不內御,憂毀之容,有如居重。其次弟景先亡,其幼弟景遠期年哭臨,亦不內寢。鄉里為之語曰:「有義有禮,房家兄弟。」廷尉卿崔光韶好標榜人物,無所推尚,每云景伯有士大夫之行業。及母亡,景
 伯居喪,不食鹽菜。因此遂為水病,積年不愈。



 卒於家,贈左將軍、齊州刺史。



 景伯子文烈,位司徒左長史,與從父弟逸祐並有名。



 文烈性溫柔,未嘗嗔怒。為吏部郎時,經霖雨絕糧,遣婢糴米,因爾逃竄,三四日方還。文烈徐謂曰:「舉家無食,汝何處來?」竟無捶撻。子山基,仕隋,歷戶部、考功侍郎,並著能名,見稱於時。



 景先字光胄,幼孤貧,無資從師,其母自授《毛詩》、《曲禮》。年十二,請其母曰:「豈可使兄傭賃以供景先也?請自求衣,然後就學。」母哀其小,不許。



 苦請乃從之。遂得一羊裘,忻然自足。晝則樵蘇,夜誦經史,遂大通贍。



 太和中,例得還鄉,解褐太學博士。時太
 常劉芳、侍中崔光當世儒宗,歎其精博,奏兼著作佐郎,修國史。侍中穆紹又啟景先撰《宣武起居注》。累遷步兵校尉,領尚書郎、齊州中正,所歷皆有當官稱。



 景先沈敏方正,事兄恭謹,出告反面,晨昏參省,側立移時,兄亦危坐,相敬如賓。兄曾寢疾,景先侍湯藥,衣冠不解,形容毀瘁。親友見者,莫不哀之。卒,特贈洛州刺史,謚曰文。景先作《五經疑問》百餘篇,其語典該。符璽郎王神貴益之,名為《辯疑》,合成十卷,亦有可觀。節閔帝時,奏上之。帝親自執卷,與神貴往復,嘉其用心。子延祐,武定末太子家令,後隸魏收修史。



 景遠字叔遐,重然諾,好施與。頻歲凶儉,
 分贍宗親;又於通衢以飼餓者,存濟甚眾。平原劉郁行經齊、兗之境,忽遇劫賊,已殺十餘人。次至郁,呼曰:「與君鄉近,何忍見殺。」賊曰:「若言鄉里,親親是誰?」郁曰:「齊州主簿房陽是我姨兄。」陽是景遠小字。賊曰:「我食其粥得活,何得殺其親。」遂還衣物,蒙活者二十餘人。



 景遠好史傳,不為章句。天性小急,不類家風。然事二兄至謹,撫養兄孤,恩訓甚篤。益州刺史傅豎眼慕其名義,啟為昭武府功曹參軍。以母老不應,豎眼頗恨之。卒於家。子敬道,永熙中開府參軍。



 畢眾敬,小名奈,東平須昌人也。少好弓馬射獵,交結輕
 果,常於疆境盜掠為業。仕宋,位太山太守。湘東王彧殺其主子業而自立,是為明帝。遣眾敬詣兗州募人。到彭城,刺史薛安都召與密謀,云:「晉安有上流之名,且孝武第三子,當共卿西從晉安。」眾敬從之。東平太守申纂據無鹽城,不與之同。及宋明平子勛,授纂兗州刺史。會有人發眾敬父墓,令其母骸首散落。眾敬發喪行服,疑纂所為。弟眾愛,為薛安都長史,亦遣人密至濟陰,掘纂父墓,以相報答。



 及安都以城入魏,眾敬不同其謀。子元賓以母并百口悉在彭城,恐交致禍,日夜啼泣,遣請眾敬,眾敬猶未從之。眾敬先已遣表謝宋,宋明授眾敬兗州
 刺史,而以元賓有他罪,獨不捨之。眾敬拔刀破柱曰:「皓首之年,唯有此子,今不原貸,何用獨全!」及尉元至,乃以城降。元遣將入城,事定。眾敬悔恚,數日不食。皇興初,就拜散騎常侍、兗州刺史,賜爵東平公,與中書侍郎李璨對為刺史。慕容白曜攻剋無鹽,獲申纂,無殺纂意。而城中火起,纂為所燒死。眾敬聞剋無鹽,懼不殺纂,乃與白曜書,并表朝廷,云家酷由纂。聞纂死。乃悅。二年,與薛安都朝京師,賜甲第一區。後復為兗州刺史,徵還京師。



 眾敬善自奉養,食膳豐華,必致他方遠味。年已七十,髮鬚皓白,而氣力未衰,跨鞍馳騁,有若少壯。篤於姻類,深有
 國士之風。張讜之亡,躬往營視,有若至親。



 太和中,孝文賓禮舊老,眾敬與高允引至方山。雖文武奢儉,好尚不同,然亦與允甚相愛敬,接膝談款,有若平生。後以篤老,乞還桑梓,朝廷許之。眾敬臨還,獻真珠榼四具、銀裝劍一口、刺彪矛一枚、仙人文綾一百疋。文明太后與帝引見於皇信堂,賜以酒饌車馬絹等,勞遣之。卒於兗州。



 子元賓,少豪俠有武幹,涉獵書史。與父同建勳誠,至京師,俱為上賓,賜爵須昌侯。後拜兗州刺史,假彭城公。父子相代為本州,當世榮之。時眾敬以老還鄉,常呼元賓為使君。每元賓聽政時,乘板輿出至元賓所,先遣左右敕
 不聽起,觀其斷決,忻忻然喜見顏色。眾敬善持家業,猶能督課田產,大致儲積。元賓為政清平,善撫人物,百姓愛樂之。以父憂解任,喪中,遙授長兼殿中尚書。卒,贈衛尉卿,謚曰平。



 元賓入魏,初娶東平劉氏,有四子,祖朽、祖髦、祖歸、祖旋。賜妻元氏,生二子,祖榮、祖暉。祖朽最長,祖暉次祖髦。故事,前妻雖先有子,後賜之妻子皆承嫡。所以劉氏先亡,祖暉不服重。元氏後卒,祖朽等三年終禮。



 祖榮早卒,子義允襲祖爵東平公,例降為侯。卒,子僧安襲。



 祖朽身長八尺,腰帶十圍。涉獵經史,好為文詠,善與人交。襲父爵須昌侯,例降為伯。以本州中正為統軍,隸
 邢巒討梁師,以功封南城縣男。歷散騎侍郎、中書侍郎。神龜末,除東豫州刺史。祖朽善撫邊,清平有信,百姓稱之。後為瀛州刺史,卒。贈吏部尚書、兗州刺史。無子,以弟祖歸子義暢為後,襲爵。



 義暢傾巧無士業,善通時要,位中書侍郎、兗州大中正。後除散騎常待,坐事伏法。祖髦以兄祖朽別封南城,以須昌伯回授之,位東平太守,卒於本州別駕。



 祖暉早有器幹,為豳州刺史,以全守勛,封新昌縣子。逢蕭寶夤退敗,祖暉拔城,東趣華陰,坐免官爵。尋行豳州事。建義中,詔復州、爵。後為賊宿勤明達所攻沒。長子義勰襲爵,齊受禪,例降。義勰弟義雲。



 義雲小字陀兒,少粗俠。家在兗州北境,常劫掠行旅,州里患之。晚方折節從官,累遷尚書都官郎中。性嚴酷,事多幹了。齊文襄作相,以為稱職,令普勾偽官,專以車輻考掠,所獲甚多,然大起怨謗。曾為司州吏所訟,云其有所減截,并改換文書。文襄以其推偽,眾人怨望,並無所問。乃拘吏,數而斬之。因此銳情訊鞫,威名日盛。



 文宣受禪,除書侍御史,彈射不避勳親。累選御史中丞,繩劾更切。然豪橫不平,頻被怨訟。前為汲郡太守翟嵩啟列:義雲從父兄僧明負官債,先任京畿長史,不受其屬,立限切徵,由此挾嫌,數遣御史過郡訪察,欲相推繩。又坐私
 藏工匠,家有十餘機織錦,并造金銀器物,乃被禁止。尋見釋,以為司徒左長史。



 尚書左丞司馬子瑞奏彈義雲,稱:「天保元年四月,竇氏皇姨祖載日,內外百官赴第吊省;義雲唯遣御史投名,身遂不赴。又義雲啟云:『喪婦孤貧。後娶李世安女為妻。世安身雖父服未終,其女為祖已就平吉,特乞闇迎,不敢備禮。』及義雲成婚之夕,眾禮備設,剋日拜閣;鳴騶清路,盛列羽儀;兼差臺吏二十人,責其鮮服,侍從車後。直是茍求成婚,誣罔干上。義雲資產宅宇,足稱豪室,忽通孤貧,亦為矯詐。又駕幸晉陽,都坐判:『拜起居表,四品以下五品以上,令預前一日赴南
 都署表;三品以上,臨日署訖。』義雲乃乖例,署表之日,索表就家先署,臨日遂稱私忌不來。」於是詔付廷尉科罪。尋敕免推。子瑞又奏彈義雲事十餘條,多煩碎,罪止罰金,不至除免。



 子瑞從兄消難為北豫州刺史。義雲遣御史張子階詣州采風聞,先禁其典簽家客等。消難危懼,遂叛入周。時論歸罪義雲,云其規報子瑞。事亦上聞。爾前宴賞,義雲常預,從此後集見稍疏,聲望大損。乾明初,子瑞遷御史中丞。鄭子默正被任用,義雲之姑即子默祖母,遂除度支尚書,攝左丞。子默誅後,左丞便解。



 孝昭赴晉陽,高元海留鄴,義雲深相依附。知其信向釋氏,常
 隨之聽講,為此款密,無所不至。及孝昭大漸,顧命武成。高歸彥至都,武成猶致疑惑。元海遣犢車迎義雲入北宮參審,遂與元海等勸進。仍從幸晉陽,參預時政。尋除兗州刺史,給後部鼓吹,即本州也。軒昂自得,意望銓衡之舉,見諸人自陳,逆許引接。又言離別暫時,非久在州。先有鐃吹,至於按部行游,兩部並用。猶作書與元海,論敘時事。元海入內,不覺遺落,給事中李孝貞得而奏之。為此,元海漸疏,孝貞因是兼中書舍人。又高歸彥起逆,義雲在州私集人馬,并聚甲仗,將以自防,實無他意,為人密啟。及歸彥被擒,又列其朋黨專擅,為此追還。武成猶
 錄其往誠,竟不加罪,除兼七兵尚書。



 義雲性豪縱,頗以施惠為心。累世本州刺史,家富於財,士之匱乏者,多有拯濟。及貴,恣情驕侈,營造第宅宏壯,未幾而成。閨門穢雜,聲遍朝野。為郎時,與左丞宋游道因公事忿競。游道廷辱之,云:「《雄狐》之詩,千載為汝。」義雲一無所答。然酷暴殘忍,非人理所及。為家尤甚,子姓僕隸,恆瘡痍遍體。



 有孽子善昭,性至凶頑,與義雲侍婢姦通。搒掠無數,為其著籠頭,繫之庭樹,食以芻秣,十餘日乃釋之。夜中,義雲被賊害,即善昭所佩刀也,遺之於善昭庭中。



 善昭聞難奔哭。家人得佩刀,善昭怖,便走出,投平恩墅舍。旦日,武
 成令舍人是蘭子暢就宅推之。爾前,義雲新納少室範陽盧氏,有色貌。子暢疑盧姦人所為,將加栲掠。盧具列善昭云爾。乃收捕,繫臨漳獄,將斬之。邢邵上言,此乃大逆,義雲又是朝貴,不可發。乃斬之於獄,棄尸漳水。



 祖歸位建寧太守。子義遠,位平原太守。義遠弟義顯、義攜,性並豪率。天平以後,梁使人還往,經歷兗城。前後州將以義攜兄弟善營鮭膳,器物鮮華,常兼長史,接宴賓客。祖旋,太尉行參軍。卒,贈都官尚書、齊兗二州刺史。



 眾敬弟眾愛,隨兄歸魏,以勳為第一客,賜爵鉅平侯。卒,贈徐州刺史。謚曰康。



 子聞慰,字子安。有器幹,襲爵,例降為伯。延
 昌初,累遷清河內史,固以疾辭。後試守廣平內史。正光初,相州刺史中山王熙起兵,謀誅元叉。聞慰斬其使,發兵拒之。叉以為忠於己,遷滄州刺史,甚有政績。後除散騎常侍、東道行臺,尋為都督、安樂王鑒軍司馬,攻元法僧,敗。奔還京師,被劾,遇赦免。卒,贈散騎常侍、兗州刺史,伯如故,謚曰恭。



 子祖彥,字修賢。涉獵書傳,風度閑雅,為時所知。以侍卸史為元法僧監軍,法僧反,被逼南入。後還,歷中書侍郎,襲爵鉅平伯。卒,贈尚書右僕射、兗州刺史。祖彥弟祖哲,祕書郎。諸畢當朝,不乏榮貴,但幃薄不修,為時所鄙。



 申纂者,本魏郡人,申鍾曾孫也。皇始初,道
 武平中山,纂舉室南奔,家於濟陰。及在無鹽,仕宋為兗州刺史。既敗,子景義入魏。



 羊祉,字靈祐,太山鉅平人,晉太僕卿琇之六世孫也。父規之,宋任城令。太武南討,至鄒山,規之與魯郡太守崔邪利及其屬縣徐遜、愛猛之等俱降,賜爵鉅平子,拜鴈門太守。



 祉性剛愎,好刑名。為司空令、輔國長史,襲爵鉅平子。侵盜公資,私營居宅,有司按之,抵死。孝文特恕遠徙。後還。景明初,為將作都將,加左軍將軍。四年,持節為梁州軍司,討叛氐。正始二年,王師伐蜀,以祉假節龍驤將軍、益州刺史,出劍閣而還。又以本將軍為秦、梁二州
 刺史,加征虜將軍。天性酷忍,又不清潔,坐掠人為奴婢。為御史中尉王顯所彈,免。高肇執政,祉復被起為光祿大夫,假平南將軍、持節,領步騎三萬,先驅趣涪。未至,宣武崩,班師。夜中引軍,山有二徑,軍人迷而失路,祉便斬隊副楊明達,梟首路側。為中尉元昭所劾,會赦免。後加平北將軍,未拜而卒。贈安東將軍、兗州刺史。



 太常少卿元端、博士劉臺龍議謚曰:「祉志存埋輪,不避強禦;及贊戎律,熊武斯裁;仗節撫籓,邊夷識德,化沾殊類,襁負懷仁。謹依謚法,布德行剛曰景,宜謚為景。」侍中侯剛、給事黃門侍郎元纂等駮曰:「臣聞唯名與器,弗可妄假。



 定謚
 準行,必當其迹。按祉志性急酷,所在過威,布德罕聞,暴聲屢發。而禮官虛述,謚之為景,非直失於一人,實毀朝則。請還付外,準行更量虛實。」靈太后令曰:「依駮便議。」元端、臺龍上言:「竊惟謚者行之迹,狀者迹之稱。然尚書銓衡是司,釐品庶物,若狀與迹乖,應抑而不受,錄其實狀,然後下寺,依謚法準狀科上。豈有捨其行迹,外有所求,去狀去稱,將何所準。檢祉以母老辭籓,乃降手詔云:『卿綏撫有年,聲實兼著,安邊寧境,實稱朝望。』及其沒也,又加顯贈,言祉誠著累朝,效彰出內,作牧岷區,字萌之績驟聞。詔冊褒美,無替倫望。然君子使人,器之,義無求備。
 德有數德,優劣不同,剛而能剋,亦為德焉。謹依謚法,布德行剛曰景,謂前議為允。」司徙右長史張烈、主簿李枿刺稱:「按祉歷官累朝,當官允稱。委捍西南,邊隅靖遏,準行易名,獎誡攸在,竊謂無虧體例。」尚書李詔又述奏以府寺為允,靈太后可其奏。



 祉自當官,不憚彊禦。朝廷以為剛斷,時有檢覆,每令出使。然好慕刑名,頗為深文,所經之處,人號天狗下。及出將臨州,並無恩潤,兵人患其嚴虐。子深。



 深字文泉,早有風尚,學涉經史,兼長几案。少與隴西李神俊同志相友。自司空記室參軍,再遷尚書駕部郎中。
 于時沙汰郎官,務精才實,深以才堪見留。在公明斷,尚書僕射崔亮、吏部尚書甄琛咸敬重之。明帝行釋奠之禮,講《孝經》,深儕輩中獨蒙引聽,時論美之。



 正光末,北地人車金雀等率羌、胡反叛,高平賊宿勤明達寇豳、破諸州,北海王顥為都督、行臺討之。以深為行臺右丞、軍司,仍領郎中。顥敗,還京。頃之,遷尚書左丞。蕭寶夤反,攻圍華州,王平、薛鳳賢等作逆。敕深兼給事黃門侍郎,與大行臺、僕射長孫承業共會潼關,規模進止。事平,以功賜爵新泰男。靈太后曾幸芒山,集僧尼齋會,公卿盡在坐。太后引見深,欣然勞問之。顧謂左右曰:「羊深真忠臣也。」
 舉坐傾心。



 莊帝踐阼,除太府卿,又為二兗行臺。深處分軍國,損益隨機,亦有時譽。初爾朱榮殺害朝士,深第七弟侃為太山太守。性粗武,遂率鄉人外招梁寇。深在彭城,忽得侃書,招深同逆。深慨然流涕,斬使人,并收表聞。莊帝乃下詔褒其忠烈,令還朝受敕。乃歸京師,除名。久之,除金紫光祿大夫。元顥入洛,以深兼黃門侍郎。



 景平,免官。普泰初,為散騎常侍、衛將軍、右光祿大夫,監起居注。



 自天下多事,東西二省,官員委積。節閔帝敕深與常侍盧道虔、元晏、元法壽選人補定,自奉朝請以上,各有沙汰。尋兼侍中。節閔帝甚親待之。時膠序廢替,名教陵
 遲。深乃上疏,請修立國學,廣延胄子,帝善之。孝武初,除中書令。永熙三年,以深兼御史中尉、東道軍司。及帝入關,深與樊子鵠不從齊神武,起兵於兗州,子鵠署深為齊州刺史。天平二年正月,東魏軍討破之,斬於陣。



 深子肅,武定末儀同、開府、東閣祭酒。以學尚知名。乾明初,為冀州中從事。



 趙郡王為巡省大使,肅以遲緩不任職解。朝議以肅無罪,尋復之。武平中,入文林館撰書。尋為武德郡守。



 祉弟靈引,好法律。李彪為中丞,以為書侍御史,固辭,彪頗銜之。及為三公郎,坐兄祉事知而不糾,彪劾奏免官。甚為尚書令高肇所暱。京兆王愉與肇深
 相嫌忌。及愉出鎮冀州,肇與靈引為愉長史,以相間伺。靈引私恃肇勢,每折於愉。及愉作逆,先斬靈引於門。時論云:「非直愉自不臣,抑亦由肇及靈引所致。」事平,贈平東將軍、兗州刺史,謚曰威。



 子敦,字元禮,性尚閑素,學涉書史。以父死王中,除給事中。出為本州別駕。



 公平正直,見非法,終不判署。後為衛將軍、廣平太守,甚有能名。姦吏局蹐,秋毫無犯。雅性清儉,屬歲饑,家饋未至,使人外尋陂澤,採藕根食之。遇有疾苦,家人解衣質米以供之。然政尚威嚴。朝廷以其清白,賜穀一千斛,絹一百匹。卒官,吏人奔哭,莫不悲慟。贈
 衛大將軍、吏部尚書、兗州刺史,謚曰貞。武定初,齊神武以敦及中山太守蘇淑在官奉法,清約自居,宜見追褒,仍上言請加旌錄。詔各賞帛一百匹,粟五百斛,下郡國,咸使聞知。



 靈引弟瑩,字靈珍,兗州別駕從事。子烈。



 烈字信卿,少通敏,頗自修立,有成人風。好讀書,能言名理,以玄學知名。



 魏孝昌末,烈從兄侃為太山太守,據郡起兵外叛。烈潛知共謀,深懼家禍,與從兄廣平太守敦馳赴洛陽告難。朝廷將加厚賞,烈告人云:「譬如斬手全軀,所存者大故爾,豈有幸從兄之敗,以為己利乎。」卒無所受。



 天保中,累遷尚書祠部、左右戶郎中,在官咸為稱
 職。除陽平太守,有能名。



 時頻有災蝗,犬牙不入陽平境,敕書褒美焉。遷光祿少卿、兗州大中正。天平初,除義州刺史,以老還鄉,卒于家。



 烈家傳素業,閨門修飭,為世所稱。一門女不再醮。魏太和中,於兗州造一尼寺,女寡居無子者,並出家為尼,咸存戒行。烈天統中與尚書畢義雲爭兗州大中正。



 義雲盛稱門代累世,本州刺史,卿世為我家故史。烈云:「自畢軌被誅以還,寂無人物。近日刺史,皆疆場之上,彼此而得,何足為言。豈若我之漢河南尹、晉朝太傅,名德學行,百世傳美。且男清女貞,足以相冠,自外多可稱也。」蓋譏義雲之帷薄焉。



 烈弟修,有才幹,
 卒於尚書左丞。子玄正。武平末,將作丞。隋開皇中,戶部侍郎。卒於隴西郡贊務。



 論曰:薛安都一武夫耳,雖輕於去就,實啟東南。事窘圖變,而竟保寵祿,優矣。休賓窮而委質;孝標名重東南;法壽拓落不羈,克昌厥後;景伯兄弟儒素,良可稱乎。眾敬舉地納誠,榮曜朝國;人位並列,無乏於時。羊祉剛酷之風,得死為幸。深以才幹從事,聲跡可稱。敦、烈持己所遵,殆時彥也。



\end{pinyinscope}