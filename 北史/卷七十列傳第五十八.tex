\article{卷七十列傳第五十八}

\begin{pinyinscope}

 韓褒趙肅子軌
 張軌李彥郭彥梁昕皇甫璠子誕辛慶之族子昂王子直杜杲呂思禮徐招檀翥孟信宗懍劉璠子祥兄子行本柳遐子莊韓褒,字弘業,潁川潁陽人也。祖環,魏平涼郡守、安定郡公。父演,恒州刺史。褒少有志尚,好學而不守章句。其師
 怪問之,對曰:「文字之間,常奉訓誘,至於商較異同,請從所好。」師因此奇之。及長,涉獵經史,深沈有遠略。屬魏室喪亂,避地夏州。時周文帝為刺史,素聞其名,待以客禮。及賀拔岳為侯莫陳悅所害,諸將遣使迎周文。周文問以去留之計,褒曰:「此天授也,何可疑乎!」周文納焉。及為丞相,引為錄事參軍。賜姓侯呂陵氏。大統初,遷行臺左丞,賜爵三水縣伯、丞相府從事中郎,出鎮淅、酈。居二年,徵拜丞相府司馬,進爵為侯。



 出為北雍州刺史。州帶北山,多有盜賊。褒密訪之,並豪右所為也,而陽不之知。厚加禮遇,謂曰:「刺史起自書生,安知督盜?所賴卿等共分
 其憂耳。」乃悉召傑黠少年素為鄉里患者,置為主帥,分其地界,有盜發而不獲者,以故縱論。於是諸被署者莫不惶懼,皆首伏曰:「前盜發者,並某等為之。」所有徒侶,皆列其姓名,或亡命隱匿者,並悉言其所在。褒乃取盜名簿藏之,因大榜州門曰:「自知行盜者,可急來首,即除其罪。盡今月不首者,顯戮其身,籍沒妻子,以賞前首者。」



 旬日之間,諸盜咸悉首盡。褒取名簿勘之,一無差異,並原其罪,許以自新,由是群盜屏息。入為給事黃門侍郎,遷侍中,除都督、西涼州刺史。羌胡之俗,輕貧弱,尚豪富。豪富之家,侵漁百姓,同於僕錄。故貧者日削,豪者益富。褒
 乃悉募貧人,以充兵士,優復其家,蠲免徭賦。又調富人財物以振給之。每西城商貨至,又先盡貧者市之。於是貧富漸均,戶口殷實。廢帝元年,為會州刺史。後以驃騎大將軍、開府儀同三司,進爵為公,累遷汾州刺史。



 先是,齊寇數入,人廢耕桑,前後刺史,莫能防扞。褒至,適會寇來,乃不下屬縣。人既不備,以故多被抄掠。齊人喜於不覺,以為州先未集兵,今還必不能追躡,由是益懈,不為營壘。褒已先勒精銳,伏北山中,分據險阻,邀其歸路。乘其眾怠,縱伏擊之,盡獲其眾。故事,獲生口者,並送京師,褒因是奏曰:「所獲賊眾,不足為多,俘而辱之,但益其忿
 耳。請一切放還,以德報怨。」有詔許焉。自此抄兵頗息,遷河州總管,仍轉鳳州刺史。尋以年老請致事,詔許之。天和五年,拜少保。褒歷事三帝,以忠厚見知。武帝深相敬重,常以師道處之,每入朝見,必有詔令坐,然始論政事。卒,贈涇、岐、燕三州刺史,謚曰貞。



 子繼伯嗣。仕隋,位終衛尉少卿。



 趙肅,字慶壅,河南洛陽人也。世仕河西。及沮渠氏滅,曾祖武始歸於魏,賜爵金城侯。祖興,中書博士。父申侯,舉秀才,為後軍府主簿。肅早有操行,知名於時。孝昌中,起家殿中侍御史,累遷左將軍、太中大夫。東魏天平初,除
 新安郡守,秩滿還洛陽。大統三年,獨孤信東討,肅率宗人為向導。授司州別駕,監督糧儲,軍用不匱。周文帝聞之,謂人曰:「趙肅可謂洛陽主人也。」九年,行華山郡事。



 十三年,除廷尉少卿。明年元日,當行朝禮,非有封爵者不得預焉。肅時未有茅土,左僕射長孫儉啟周文請之,周文乃召肅謂曰:「歲初行禮,豈得使卿不預!



 然何為不早言也?」於是令肅自選封名。肅曰:「河清乃太平之應,竊所願也。」



 於是封清河縣子。十六年,除廷尉卿,加征東將軍。肅久在理官。執心平允,凡所處斷,咸得其情。廉慎自居,不營產業,時人以此稱之。十七年,進位車騎大將軍、儀同
 三司、散騎常侍,賜姓乙弗氏。先是,周文命肅撰法律,肅積思累年,遂感心疾。去職,卒於家。子軌。



 軌少好學,有行檢。周蔡王引為記室,以清苦聞。隋文帝受禪,為齊州別駕,有能名。其東鄰有桑,葚落其家,軌遣人悉拾還其主,戒其諸子曰:「吾非以此求名,意者非機杼物,不願侵人。汝等宜以為戒。」在州考績連最。持節使者郃陽公梁子恭上狀,文帝賜以米帛甚優,令入朝。父老將送者,各揮涕曰:「別駕在官,水火不與百姓交,是以不敢以盃酒相送。公清如水,請酌一盃水奉餞。」軌受飲之。



 至京,詔與牛弘撰定律令格式。



 時衛王爽為原州總
 管,召為司馬。在道夜行,其左右馬逸入田中,暴人禾。軌駐馬待明,訪知禾主,酬直而去。原州人吏聞之,莫不改操。後檢校硤州刺史,甚有恩惠。轉壽州總管長史。芍陂舊有五門堰,蕪穢不通。軌勸課吏人,更開三十六門,灌田五千餘頃,人賴其利。秩滿歸,卒于家。子弘安、弘智,並知名。



 張軌,字元軌,濟北臨邑人也。父崇,高平令。軌少好學,志識開朗。初在洛陽,家貧,與樂安孫樹仁為莫逆之友,每易衣而出,以此見稱。軌常謂所親曰:「秦、雍之間,必有王者。」爾朱氏敗後,遂杖策入關。賀拔岳以軌為記室參軍。



 典機密。尋轉倉曹。時穀糴踴貴,或有請貸官倉者,軌曰:「以私害公,非吾宿志。



 濟人之難,詎得相違?」乃賣所服衣物,糴粟以振其乏。及岳被害,周文帝以軌為都督,從征侯莫陳悅。悅平,使於洛陽,見領軍斛斯椿。椿曰:「高歡逆謀,已傳行路,人情西望,以日為年,未知宇文何如賀拔也?」軌曰:「宇文公文足經國,武足定亂,至於高識遠度,非愚管所測。椿曰:「誠如卿言,真可恃也。」周文為行臺,授軌郎中。孝武西遷,除中書舍人,封壽張縣子,肅著作佐郎,修起居注,遷給事黃門侍郎,兼吏部郎中。出為河北郡守。在郡三年,聲績甚著,臨人政術,有循吏之美。大統間
 言宰人者,多推尚之。入為丞相府從事中郎,行武功郡事。章武公導出鎮秦州,以軌為長史。廢帝元年,進車騎大將軍、儀同三司、散騎常侍。



 二年,賜姓宇文氏,行南秦州事。恭帝二年,徵拜度支尚書,復除隴右府長史。卒於位,謚曰質。軌性清素,臨終之日,家無餘財,唯有書數百卷。



 子肅,周明帝初為宣納上士,轉中外府記室參軍、中山公訓侍讀。早有才名,性頗輕猾,時人比之魏諷。卒以罪考竟終。



 李彥,字彥士,梁郡下邑人也。祖光之,魏淮南郡守。父靜,南青州刺史。彥少有節操,好學慕古。孝昌中,解褐奉朝
 請。孝武入關,兼著作佐郎,修起居注。



 大統初,除通直散騎侍郎,累遷左戶郎中。十二年,省三十六曹為十二部,改授戶部郎中,封平陽縣子。廢帝初,拜尚書右丞,轉左丞。彥在尚書十有五載,屬軍國草創,庶務殷繁,留心省閣,未嘗懈怠。斷決如流,略無疑滯。臺閣莫不歎其公勤,服其明察。遷給事黃門侍郎,仍左丞。賜姓宇文氏。出為鄜州刺史。六官建,改授軍司馬,進爵為伯。彥性謙恭,有禮節,雖居顯要,於親黨之間恂如也。輕財重義,好施愛士,時論以此稱之,
 然素多疾,而勤於蒞職,雖沈頓枕席,猶理務不輟,遂至於卒。謚曰敬。



 彥臨終遺誡其子等曰:「昔人以窾木為櫝,葛累為緘,下不亂泉,上不泄臭,實吾平生之志也。但事既矯枉,恐為世士所譏。今可斂以時服,葬於磽脊之地,勿用明器、芻塗及儀衛等。爾其今哉。」朝廷嘉焉。不奪其志。



 子昇明嗣。少歷顯職。大象末,太府中大夫、儀同大將軍。仕隋,終於齊州刺史。



 子仁政,長安縣長。義軍至,以罪誅。



 郭彥,太原陽曲人也。其先從官關右,遂居馮翊。父胤,靈武令。彥少知名。



 周文帝臨雍州,辟為西曹書佐。累遷虞
 部郎中。大統十二年,初選當州首望,統領鄉兵,除帥都督。以居郎官著稱,封龍門縣子,進大都督。恭帝元年,除兵部尚書,仍以本兵從柱國於謹南伐江陵。進驃騎大將軍、開府儀同三司,進爵為伯。六官建,拜戶部中大夫。周孝閔帝踐祚,出為澧州刺史。蠻左生梗,不營農業。彥勸以耕稼,人皆務本,亡命之徒,咸從賦役。先是,以澧州糧儲乏少,每令荊州遞送。自彥蒞職,倉庾充實,無復轉輸之勞。齊南安城主馮顯密遣使歸降,其眾未之知也。柱國宇文貴令彥率兵應接。時齊人先令顯率所部送糧南下,彥懼其眾不從命,乃於路邀之,顯因得自拔。其眾
 果拒戰,彥縱兵奮擊,並虜獲之。以南安無備,即引軍掩襲,遂有其城。晉公護嘉之,進爵懷德縣公。入為工部中大夫。保定四年,晉公護東討,彥從尉遲迥攻洛陽,迥復令彥與權景宣出汝南。及軍次豫州,使彥鎮之。天和中,為隴右總管府長史。卒於官。贈小司空、宜鄜丹三州刺史。



 梁昕,字元明,安定烏氏人也。世為關中著姓。其先因官,徙居京兆之盩厔。



 祖重耳,漳縣令。父勸儒,中散大夫,贈涇州刺史。昕少溫恭,見稱州里。從爾朱天光征討,拜右將軍、太中大夫。周文帝迎魏孝武,軍次雍州,昕以三輔
 望族上謁。



 周文見昕容貌瑰偉,深賞異之,即授右府長流參軍。累遷丞相府主簿。大統十二年,除河南郡守,遷東荊州刺史。昕撫以仁惠,蠻夷悅之。封安定縣子。周孝閔帝踐祚,進位驃騎大將軍、開府儀同三司。明帝初,進爵胡城縣伯。天和初,拜工部中大夫,出為陜州總管府長史。昕性溫裕,有幹能,歷官內外,咸著聲稱。尋卒官。贈大將軍,謚曰貞。



 昕弟榮,位計部下大夫、開府儀同三司、朝那縣伯。贈涇、寧、幽三州刺史,謚曰靜。



 子蠙,仕隋,為給事郎。貞觀中,終於鄭州刺史。



 皇甫璠,字景瑜,安定三水人也。世為西州著姓,後徙居
 京兆。父和,本州中從事。大統末,追贈散騎常侍、儀同三司、涇州刺史。璠少忠謹,有幹略,永安中,辟州都督。周文帝為牧,補主簿,以勤事被知。大統四年,引為丞相府行參軍。周孝閔帝踐祚,為守廟下大夫、長樂縣子。保定中,為鴻州刺史,入為小納言。累遷蕃部中大夫,進驃騎大將軍、開府儀同三司。璠性平和,小心奉法,安貞守志,恆以清白自處,當時稱為善人。建德三年,為隨州刺史,政存簡惠,百姓安之,卒官,贈交、渭二州刺史,謚曰恭。



 子諒,少知名。大象中,位吏部下大夫。諒弟誕。



 誕字玄慮,少剛毅,有器局,開皇中,累遷治書侍御史,朝
 臣入不肅憚焉。後為尚書左丞。時漢王諒為並州總管,朝廷盛選僚佐,拜誕並州總管司馬,總府政事,一以諮之,諒甚敬焉。及煬帝即位,諒用諮議王頍謀,發兵作亂。誕數諫止,諒不納。誕因流涕,以死固請。諒怒囚之。及楊素將至,諒屯清源以拒之。諒主簿豆盧毓出誕於獄,協謀閉城拒諒。諒襲擊破之,並抗節遇害。帝以亡身殉國,嘉悼者久之。詔贈柱國,封弘義公,謚曰明。



 子無逸嗣。尋為淯陽太守,甚有聲稱。大業初,令行,舊爵例除。以無逸誠義之後,賜爵平輿侯。入為刑部侍郎,守右武衛將軍。



 初,漢王諒之反,州縣莫不響應。有嵐州司馬陶世模、繁
 畤令敬釗,並抗節不從。



 世模,京兆人。性明敏,有器幹。仁壽初,為嵐州司馬。諒反,刺史喬鐘葵將赴之,世模以義拒之。臨之以兵,辭氣不撓,鐘葵義而釋之。軍吏請斬之,於是被囚。及諒平,拜開府,授大興令。從衛玄擊楊玄感,以功進位銀青光祿大夫。



 釗字積善,河東蒲阪人。父元約,周布憲中大夫。釗,仁壽中為繁畤令,甚有能名。漢王諒反,師陷其城,賊帥墨弼執送偽將喬鐘葵,署為代州總管司馬。釗正色拒之,誓之以死。會鐘葵敗,釗遂免。卒於朝邑令。



 辛慶之,字餘慶,隴西狄道人也。世為隴右著姓。父顯宗,
 馮翊郡守,贈雍州刺史。慶之少以文學徵詣洛陽,對策第一,除祕書郎。屬爾朱氏作亂,魏孝莊帝令司空楊津為北道行臺,節度山東諸軍以討之。津啟慶之為行臺左丞,與參謀議。至鄴,聞孝莊帝崩,遂出兗、冀間,謀結義徒,以赴國難。尋而節閔帝立,乃還洛陽。



 及賀拔岳為行臺,復啟慶之為行臺吏部郎。大統初,從周文帝東討,為行臺左丞。



 六年,行河東郡事。九年,入為丞相府右長史,兼給事黃門侍郎,除度支尚書,復行河東郡事。遷南荊州刺史,加儀同三司。慶之位遇雖隆,而率性儉素,車馬衣服亦不尚華侈。志量淹和,有儒者風度,特為當時所
 重。又以其經明行修。令與盧誕等教授諸王。慶帝二年,拜祕書監。卒官。子加陵,主寢上士。慶之族子昂。



 昂字進君。數歲便有成人志行。有善相人者,謂其父仲略曰:「公家雖世載冠冕,然名德富貴,莫有及此兒者。仲略亦重昂志氣。深以為然。年十八,侯景辟為行臺郎中。景後來附,昂遂入朝,除丞相府行參軍。後追論歸朝勛,封襄城縣男。



 及尉遲迥伐蜀,昂占募從軍。蜀平,迥表昂為龍州長史,領龍安郡事。州帶山谷,舊俗生梗。昂威惠洽著,吏人畏而愛之。成都一方之會,風俗舛雜,迥以昂達於從政,復表昂行成都令。昂到縣。便與諸生與祭文翁
 學堂,因共歡宴,謂諸生曰:「子孝臣忠,師嚴友信,立身之要,如斯而已。若不事斯語,何以成名?各宜自勉,克成令譽。」昂言切理至,諸生等並深感悟,歸而告其父老曰:「辛君教誡如此,不可違之。」於是井邑肅然,咸從其化。遷梓潼郡守。六官建,入為司隸上士,襲爵繁昌縣公。



 保定二年,為小吏部。時益州殷阜,軍國所資,經途艱險,每苦劫盜。詔昂使於益、梁,軍人之務皆委決焉。昂撫導荒梗,頗得寧靜。天和初,陸騰討信州蠻,詔昂便於通、渠等州運糧饋之。時臨、信、楚、合等諸州人庶多從逆,昂諭以禍福,赴者如歸。乃令老弱負糧,壯夫拒戰,莫有怨者。使還,屬
 巴州萬榮郡人反叛,圍郡城,昂於是遂募通、開二州,得三千人,倍道兼行,出其不意。又令其眾皆作中國歌,直趣賊壘。謂有大軍赴救,望風瓦解。朝廷嘉其權以濟事,詔梁州總管、杞國公亮即於軍中賞昂奴婢二十口,繒彩四百匹。又以昂威信布於宕梁,遂表為渠州刺史。轉通州。推誠布信,甚得夷獠歡心。秩滿還京,首領皆隨昂詣闕朝覲。以昂化洽夷落,進位驃騎大將軍、開府儀同三司。時晉公護執政,昂稍被護親待,武帝頗銜之。及誅護,加之捶楚,因此遂卒。



 昂族人仲景,好學,有雅量。其高祖欽,後趙吏部尚書、雍州刺史,子孫因家焉。父歡,魏隴
 州刺史、朱陽公。仲景年十八,舉文學,對策高第。拜司空府主簿。



 建德中,位內史下大夫、開府儀同三司。卒于家。子衡。



 王子直,字孝正,京兆杜陵人也。世為郡右族。父琳,州主簿、東雍州長史。



 子直性節儉,有幹能。魏正光中,州辟主簿,起家奉朝請。永安初,拜鴻臚少卿。



 孝武西遷,封山北縣男。大統初,漢熾屠各阻兵於南山,與隴東屠各共為脣齒。周文帝令子直率涇州步騎五千討破之。賜書勞問,除尚書左外兵郎中,兼中書舍人。



 從解洛陽圍,經河橋戰,兼尚書左丞,出為秦總管府司馬。時涼州刺史
 宇文仲和據州逆命,子直從隴右大都督獨孤信討平之。復入為大行臺郎中,兼丞相府記室,除太子中庶子,領齊王友。尋行馮翊郡事。廢帝元年,拜使持節、大都督,行瓜州事。



 務以德政化人,西土悅附。恭帝初,徵拜黃門侍郎。卒官。



 子宣禮,柱國府參軍。



 杜杲,字子暉,京兆杜陵人也,祖建,魏輔國將軍,贈蒙州刺史。父皎,儀同三司、武都郡守。杲學涉經史,有當世幹略,其族父攢,清貞有識鑒,深器重之,常曰:「吾家千里駒也。」攢時仕魏,為黃門侍郎,兼度支尚書、衛大將軍、西道大行臺,尚孝武妹新豐公主,因薦之朝廷。永熙三年,起
 家奉朝請。周明帝初,為修城郡守。屬鳳州人仇周貢等構亂,攻逼修城,杲信洽於人,部內遂無叛者。尋率郡兵與開府趙昶合勢,並破平之。入為司會上士。



 初,陳文帝弟安成王頊為質於梁,及江陵平,頊隨例遷長安。陳人請之,周文帝許而未遣。至是,帝欲歸之,命杲使焉。陳文帝大悅,即遣使報聘,并賂黔中數州地,仍請畫野分疆,永敦鄰好。以杲奉使稱旨,進授都督,行小御伯,更往分界。



 陳於是歸魯山郡。帝乃拜頊柱國大將軍,詔杲送之還國。陳文帝謂杲曰:「家弟今蒙禮遣,實是周朝之惠。然不還魯山,亦恐未能及此。」杲答曰:「安成之在關中,乃咸
 陽一布衣耳。然是陳之介弟,其價豈止一城?本朝親睦九族,恕己及物,上遵太祖遺旨,下思繼好之義,所以發德音者,蓋為此也。若知止侔魯山,固當不貪一鎮。況魯山梁之舊地,梁即本朝籓臣,若以始末言之,魯山自合歸國。云以尋常之土,易已骨肉之親,使臣猶謂不可,何以聞諸朝廷!」陳文帝慚恧久之,乃曰:「前言戲之耳!」自是接遇有加常禮。及還,引升殿,親降御座,執手以別。朝廷嘉之,授大都督、小載師下大夫,行小納言,復聘於陳。及華皎來附,詔令衛公直、都督元定等援之。定等並沒。自是連兵不息,東南搔動。武帝授杲御正中大夫,使陳,論
 保境息人之意。陳宣帝遣其黃門侍郎徐陵謂杲曰:「兩國通好,彼朝受我叛人,何也?」杲曰:「陳主昔在本朝,非慕義而至,主上授以柱國,位極人臣,子女玉帛,備禮將送,今主社稷,孰謂非恩?郝烈之徒,邊人狂狡,曾未報德,而先納之。今受華氏,正是相報。過自彼始,豈在本朝!」陵曰:「彼納華皎,志圖吞噬。此受郝烈。容之而已。且華皎方州列將。竊邑叛亡。郝烈一百許戶,脫身逃竄。



 大小有異,豈得同年而語乎?」杲曰:「大小雖殊,受降一也。若論先後,本朝無失。」陵曰:「周朝送主上還國,既以為恩,衛公共元定度江,孰云非怨?計恩與怨,亦足相埒。」杲曰:「元定等軍敗
 身囚,其怨已滅。陳主負扆馮玉,其恩猶在。



 且怨由彼國,恩起本朝,以怨酬恩,未之聞也。」陵笑而不答。杲因陳和通之便,陵具以聞。陳宣許之,遂遣使來聘。



 建德初,授司城中大夫,仍使於陳。宣帝謂杲曰:「長湖公軍人等雖築館處之,然恐不能無北風之戀。王褒、庾信之徒既羈旅關中,亦當有南枝之思耳。」杲揣陳宣意欲以元定軍將士易王褒等,乃答之曰:「長湖總戎失律,臨雖茍免,既不死節,安用此為!且猶牛之一毛,何能損益。本朝之議,初未及此。」陳宣帝乃止。及杲還,至石頭,又遣謂之曰:「若欲合從,共圖齊氏,能以樊、鄧見與,方可表信。」



 杲答曰:「合從
 圖齊,豈唯弊邑之利?必須城鎮,宜待得之於齊。先索漢南,使臣不敢聞命。」還,除司倉中大夫,又使於陳。杲有辭辯,閑於占對,前後將命,陳人不能屈,陳宣帝甚敬異之。時元定已卒,乃禮送開府賀拔華及定棺樞,杲受之以歸。除河東郡守,遷溫州刺史,賜爵義興縣伯。大象元年,徵拜御正中大夫,復使陳。二年,除申州刺史,加開府儀同大將軍,進爵為侯。除同州刺史。隋開皇元年,以杲為同州總管,進爵為公。俄遷工部尚書。二年,除西南道行臺兵部尚書。尋以疾卒。



 子運,大象末,宣納上士。



 杲兄長暉,位儀同三司。



 呂思禮,東平壽張人也。性溫潤,不雜交遊。年十四,受學於徐遵明,長於論難,諸生為之語曰:「講《書》論《易》鋒難敵。」十九,舉秀才,對策高第,除相州功曹參軍。葛榮圍鄴,思禮有守禦勛,賜爵平陵縣伯,除欒城令。普泰中,僕射司馬子如薦為尚書二千石郎中。尋以地寒被出,兼國子博士。乃求為關西大行臺郎中,與姚幼瑜、茹文就俱入關。為行臺賀拔岳所重,專掌機密,甚得時譽。岳為侯莫陳悅所害,趙貴等議遣赫連達迎周文帝,思禮預其謀。及周文為關西大都督,以思禮為府長史,尋除行臺右丞。以迎魏孝武功,封汶陽縣子,加冠軍將軍。拜黃門侍
 郎。魏文帝即位,領著作郎,除安東將軍、都官尚書,兼七兵、殿中二曹事。



 從禽竇泰,進爵為侯。大統四年,以謗訕朝政賜死。



 思禮好學有才,雖務兼軍國,而手不釋卷。晝理政事,夜即讀書,令蒼頭執燭,燭燼夜有數升。沙苑之捷,命為露布,食頃便成,周文歎其工而且速。所為碑誄表頌,並傳於世。七年,追贈車騎將軍、定州刺史。



 子亶嗣。大象中,位至駕部下大夫。



 時有博陵崔騰,早有名譽,歷職清顯,為丞相府長史,亦以投書謗議賜死。



 徐招,字思賢,高平金鄉人也。世為著姓。招少好法律及朝廷舊事,發言措筆,常欲辯析秋毫,初入洛陽,雖未登仕,
 已為時知,朝廷疑事多預議焉。延昌中,從征浮山堰有功,賜爵高文男。及廣陽王深北討鮮于修禮,啟為員外散騎侍郎、深府長流參軍。招陳策請離間之,葛榮竟殺修理,自為魁帥。以功進爵為侯。永安初,射策甲科,除員外散騎常侍,領尚書儀曹郎中。招少習吏事,未能精究朝儀,常恨才達,恐名迹不立。久之,方轉二千石郎中。爾硃榮死,爾朱世隆屯兵河橋,莊帝以招為行臺左丞,自武牢北度,引馬場、河內之眾以抗世隆。後爾朱兆得招,鎖送洛陽,仲遠數招罪,將斬之。招曰:「不虧君命,得死為幸。」仲遠重之,曰:「凡人受命,理各為主。今若為戮,何以勸
 人臣?」乃釋之,用為行臺右丞。及仲遠南奔,招獨還洛。永熙末,從孝武入關中,拜給事黃門侍郎,兼尚書右丞。時朝廷播遷,典章遺闕,至於臺省法式,皆招所記,論者多焉。大統三年,拜驃騎將軍、侍中。時文帝舅子王起化犯罪死,有詔追贈,招執奏正之。後卒於度支尚書。子山雲嗣。



 檀翥,字鳳翔,高平金鄉人也。六世祖毓,晉步兵校尉。父江,始還北,仁至太常少卿,贈兗州刺史。翥十歲喪父,還京師宅,與營人雜居。雖幼孤寒,不與鄰人來往。好讀書,解屬文,能鼓琴,早為瑯邪王誦所知。年十九,以名家子
 為魏明帝挽郎。後客游三輔,時毛遐為行臺,鎮北雍,表翥為行臺郎中。莊帝既誅爾朱榮,遐使翥詣亦師,因除著作佐郎,郎中如故。後孝武帝西幸,除兼中書舍人,修國史。



 大統初,又兼著作佐郎。以守關迎賀勳,封高唐子。後坐談論輕躁,為黃門侍郎徐招所糾,死於廷尉獄。



 孟信,字脩仁,廣川索盧人也。家世貧寒,頗傳學業。信常曰:「窮則變,變則通。吾家世傳儒學,而未有通官,當由儒非世務也。」遂感激,棄書從軍。永熙末,除奉朝請。從孝武帝入關,封東州子,趙平太守。政尚寬和,權豪無犯。山中老人會以酒饋之,信和顏接引,殷勤勞問,乃自出酒,
 以鐵鐺溫之,素木盤盛蕪青菹,唯此而已。又以一鐺借老人,但執一盃,各自斟酌,申酬酢之意,謂老人曰:「吾至郡來,無人以一物見遺,今卿獨有此餉。且食菜已久,欲為卿受一髆耳。



 酒既自有,不能相費。」老人大悅,再拜,擘進之。酒盡方別。及去官,居貧無食。唯有一老牛,其兄子賣之,擬供薪米。券契已訖,市法應知牛主住在所。信適從外來,見買牛人,方知其賣也。因告之曰:「此牛先來有病,小用便發,君不須也。」杖其兄子二十。買牛人嗟異良久,呼信曰:「孟公,但見與牛,未必須其力也。」苦請不得,乃罷。買牛者,周文帝帳下人,周文深歎異焉。未幾,舉
 為太子少師,後遷太子太傅,儒者榮之。特加車騎大將軍、儀同三司、散騎常侍。辭老請退,周文不奪其志,賜車馬、几杖、衣服、床帳。卒於家。贈冀州刺史,謚曰戴。



 子儒。



 宗懍,字元懍,南陽涅陽人也。八世祖孫,永嘉亂,討陳敏有功,封柴桑縣侯,除宜都郡守。卒官。子孫因居江陵。父高之,梁山陰令。懍少聰敏,好讀書,晝夜不倦,語輒引古事,鄉里呼為「小兒學士」。梁大同六年,舉秀才。以不及二宮元會,例不對策。及梁元帝鎮荊州,謂長史劉之遴曰:「貴鄉多士,為舉一有意少年。」



 之遴以懍應命,即日引見,令兼記室。嘗夕被召宿省,使製《龍川廟碑》,一夜便就。詰
 朝呈上,梁元帝嘆美之。後歷臨汝、建城、廣晉三縣令。遭母憂去職,哭輒歐血,兩旬之內,絕而復蘇者三。每旦有君烏數千集于廬舍,候哭而來,哭止而去,時論以為孝感所致。梁元帝即位,擢為尚書侍郎,封信安縣侯,累遷吏部尚書。懍父高之先為南臺書侍御史,犯憲。懍願父釋罪,當終身菜食。高之理雪,故懍菜食,鄉里稱之。在元帝府,府中多言其矯。至是,大進魚肉,國子祭酒沛國劉玨讓之曰:「本知卿不忠,猶謂卿孝。今日便是忠孝並無。」懍不能對。懍博學有才藻,口未嘗譽人,朋友以此少之。初,侯景平後,梁元帝議還建鄴,唯懍勸都渚宮,以鄉在
 荊州故也。及江陵平,與王褒等入關。周文帝以懍名重南土,甚禮之。周孝閔帝踐祚,拜車騎大將軍、儀同三司。明帝即位,又與王褒等在麟趾刊定群書,數蒙宴賜。



 保定中,卒。有集二十卷行於世。



 劉璠,字寶義,沛人也。六世祖敏,以永嘉亂,徙居廣陵。父臧,性方正,篤志好學,居家以孝聞。仕梁,為著作郎。璠九歲而孤,居喪合禮。少好讀書,兼善文筆。十七,為上黃侯蕭曄所器重。范陽張綰,梁之外戚,才高口辯,見推於世。



 以曄懿貴,亦假借之。璠年少未仕,而負才使氣,不為之屈。綰嘗於新渝侯宅,因酒後詬京兆杜杲曰:「寒士不遜。」
 璠厲色曰:「此坐誰非寒士?」璠本意在綰,而曄以為屬己,辭色不平。璠曰:「何王之門不可曳長裾也!」遂拂衣而去。曄謝之,乃止。後隨曄在淮南。璠母在建康遘疾,璠弗之知。嘗忽一日舉身楚痛,尋而家信至,云其母病。璠即號泣戒道,絕而又蘇。當身痛之辰,即母死之日。居喪毀瘠,遂感風氣,服闋後一年,猶杖而後起。及曄終於毗陵,故吏多分散,璠獨奉曄喪還都,墳成乃退。梁簡文寺在東宮,遇曄素重,諸不送者多被劾責,唯璠獨被優嘗賞。解褐王國常侍,非其好也。



 璠少慷慨,好功名,志欲立事邊城,不樂隨牒平進。曾宜豐侯蕭脩出為北徐州刺史,即請為其
 輕車府主簿,兼記室參軍。脩為梁州,又板為中記室,補華陽太守。



 屬侯景度江,梁室大亂,脩以璠有才略,甚親委之。時寇難繁興,未有所定,璠乃喟然賦詩以見志。其末章曰:「隨會平王室,夷吾匡霸功。虛薄而時用,徒然慕昔風。」脩開府,置佐史,以璠為諮議參軍,仍領記室。梁元帝承制,授樹功將軍、鎮西府諮議參軍。賜書曰:「鄧禹文學,尚或執戈;葛洪書生,且云破賊。前脩無遠,屬望良深。」元帝尋以脩紹鄱陽之封,且為雍州刺史,復以璠為為脩平北府司馬。



 及武陵王紀稱制於蜀,以璠為中書侍郎。遣召璠,使者八反,乃至蜀。又以為黃門侍郎,令長史劉
 孝勝深布心腹,使工畫《陳平度河歸漢圖》以遺之。璠苦求還,中記室韋登私曰:「殿下忍而蓄憾,足下不留,將致大禍。脫使盜遮於葭萌,則卿殆矣。孰若共構大廈,使身名俱美哉!」璠正色曰:「卿欲緩頰於我邪?我與府侯分義已定,豈以寵辱夷險易其心乎!丈夫立志,當死生以之耳。殿下方布大義於天下,終不逞志於一人。」紀知不為己用,乃厚贈而遣之。臨別,紀又解其佩刀贈璠曰:「想見物思人。」璠曰:「敢不奉揚威靈,剋翦姦宄。」紀於是遣使拜脩為益州刺史,封隨郡王,以璠為府長史,加蜀郡太守。



 還至白馬西,屬達奚武軍已至南鄭,璠不得入城,遂降
 武。周文帝素聞其名,先戒武曰:「勿使劉璠死。」故武先令璠赴闕。周文見之如舊,謂僕射申徽曰:「劉璠佳士,古人何以過之!」徽曰:「晉人滅吳,利在二陸。明公今平梁漢,得劉璠也。」時南鄭尚拒守,達奚武請屠之,周文將許焉,唯令全脩一家而已。璠乃請之於朝,周文怒而不許也。璠泣而固請,移時不退。柳仲禮侍側,曰:「此烈士也。」周文既納蕭脩降,又許其反國。脩至長安累月,未之遣也。璠因侍宴,周文曰:「我於古誰比?」曰:「常以公命世英主,湯、武莫逮。今日所見,曾是齊桓、晉文之不若。」周文曰:「我不得比湯、武,望與伊、周為匹,何桓、文之不若乎?」



 對曰:「齊桓存三
 亡國,晉文不失信於伐原。」語未終,周文撫掌曰:「我解爾意,欲激我耳。」即命遣脩。脩請與璠俱還,周文不許。以璠為中外府記室,遷黃門侍郎、儀同三司。嘗臥疾居家,對雪興感,乃作《雪賦》以遂志焉。初,蕭脩在漢中與蕭紀箋,及答西魏書、移襄陽文,皆璠辭也。



 周明帝初,授內史中大夫,掌綸誥。尋封平陽縣子。在職清白簡亮,不合於時。



 左遷同和郡守。璠善於撫御,蒞職未期,生羌降附者五百餘家。前後郡守多經營以致貲產,唯璠秋毫無所取。妻子並隨羌俗,食麥衣皮,始終不改。洮陽、洪和二郡羌常越境詣璠番訟理。蔡公廣時鎮隴右,嘉其善政。及遷鎮
 陜州,欲啟璠自隨,羌人樂從者七百人,聞者莫不嘆異。陳公純作鎮隴右,引為總管府司錄,甚禮敬之。



 卒於官。著梁典三十卷,有集二十卷,行於世。子祥。



 祥字休徵。幼聰慧,賓客見者皆號神童。事嫡母以至孝聞。其伯父黃門郎璆,有名江左,在嶺南,聞而奇之,乃令名祥字休徵。後以字行於世。十歲能屬文,十二通《五經》。仕梁,為宜豐侯記室參軍。江陵平,隨例入關中。齊公憲召為記室,府中書記皆令掌之。封漢安縣子。憲進爵為王,以休徵為王友。俄除內史上士。武帝東征,休徵陪侍帷幄,平齊露布即休徵文也。累遷車騎大將軍、儀同大
 將軍。歷長安、萬年二縣令,頗獲時譽。卒於官。初,璠所選《梁典》始就,未及刊定而卒,臨終謂休徵曰:「能成我志,其在此書乎!」休徵脩定繕寫。勒成一家,行於世。



 行本,璠兄子也。父環,仕梁,歷職清顯。行本起家梁武陵王國常侍。遇蕭脩以梁州北附,遂與叔父璠歸周,寓居新豐。每以諷讀為事,精力忘疲,雖衣食乏絕,晏如也。性剛烈,有不可奪之志。周大冢宰宇文護引為中外府記室。武帝親總萬機,轉御正中士,兼領起居注。累遷掌朝下大夫。周代故事,天子臨軒,掌朝典筆硯,持至御坐,則承御大夫取進之。及行本為掌朝,將進筆於帝,承御復
 欲取之。行本抗聲曰:「筆不可得。」帝驚視問之,行本曰:「臣聞設官分職,各有司存。臣既不得佩承御刀,承御亦焉得取臣筆?」帝曰:「然。」因令二司各行所職。及宣帝嗣位,多失德,行本切諫忤旨,出為河內太守。及尉遲迥作亂,攻懷州,行本率吏人拒之,拜儀同,賜爵文安縣子。



 隋文帝踐祚,拜諫議大夫,檢校中書侍郎。上嘗怒一郎,於殿前笞之。行本進曰:「此人素清,其過又小。」上不顧。行本正當上前曰:「陛下不以臣不肖,令臣在左右。臣言若是,陛下安得不聽?臣言若非,當致之於理,安得輕臣而不顧?



 臣所言非私!」因置笏於地而退,上斂容謝之,遂原所笞者。



 時天下大同,四夷內附,行本以黨項羌密邇封域,最為後服,上表劾其使者曰:「臣聞南蠻遵校尉之統,西域仰都護之威。比見西羌,鼠竊狗盜,不父不子,無君無臣,異類殊方,於斯為下。不悟羈縻之惠,詎知含養之恩,狼戾為心,獨乖正朔。



 使人近至,請付推科。」上奇其志。雍州別駕元肇言於上曰:「有一州吏,受人饋錢二百文,律令杖一百。然臣下車之始,與其為約。此吏故違,請加徒一年。」行本駁之曰:「律令之行,蓋發明詔。今肇乃敢重其教命,輕忽憲章,虧法取威,非人臣之禮。」上嘉之,賜絹百匹。



 拜太子左庶子,領書侍御史如故。皇太子虛襟敬憚。時唐
 令則為左庶子,太子暱狎之,每令以弦歌教內人。行本責之曰:「庶子當匡太子以正道,何嬖暱房帷之間哉!」令則甚慚而不能改。時沛國劉臻、平原明克讓、河南陸爽等並以文學為太子所親。行本怒其不能調護,每謂三人曰:「卿等正解讀書耳。」時左術率長史夏侯福為太子所暱,嘗於閤內與太子戲。福大笑,聲聞於外。行本時在閤下聞之,待其出,數之曰:「汝何小人,敢為褻慢!」因付執法者推之。太子為請,乃釋之。



 太子嘗得良馬,令福乘而觀之。太子甚悅,因欲令行本復乘。行本正色曰:「至尊置臣於庶子位,欲輔導殿下以正道,非為殿下作弄臣。」太
 子慚而止。復以本官領大興令,權貴憚其方正,無敢至其門者。由是請託路絕,吏人懷之。未幾,卒于官,上甚傷惜之。及太子廢,上曰:「嗟乎!若使劉行本在,勇當不及此乎!」行本無子。



 柳遐,字子昇,河東解人,宋太尉元景從孫也。祖叔珍,義陽內史,事見《南史》。父季遠,梁宜都太守。遐幼而爽邁,神彩嶷然,髫歲便有成人之量。篤好文學,動合規矩。其世父慶遠特器異之,謂曰:「吾昔逮事伯父太尉公,嘗謂吾云:『我昨夢汝登一樓,甚峻麗,吾以坐席與汝。汝後名宦必達,恨吾不及見耳。』吾向聊復晝寢,又夢將昔時坐席
 還以賜汝,汝之官位當復及吾。特宜勉勵,以應嘉祥也。」梁西昌侯藻鎮雍州,遐時年十二,以百姓禮脩謁,風儀端肅,進止詳雅。藻羨之,試遣左右踐遐衣裾,欲觀其舉措。遐徐步稍前,曾不顧盼。仕梁稍遷尚書功論郎。陳郡謝舉時為僕射,引遐與語,甚嘉之,顧謂人曰:「江漢英靈見於此矣。」



 岳陽王蕭詧於襄陽承制,授遐吏部郎,賜爵聞喜公。尋進位持節、侍中、驃騎大將軍、開府儀同三司。及詧踐帝位於江陵,以襄陽來歸,辭詧曰:「陛下中興鼎業,龍飛舊楚。臣昔因幸會,早奉名節,理當以身許國,期之始終。自晉氏南遷臣宗族蓋寡,從祖太尉、世父儀同、
 從父司空,並以位望隆重,遂家于金陵;唯留先臣獨守墳栢,嘗誡臣等,使不違此志。今襄陽既入北朝,臣若陪隨鑾蹕,進則無益塵露,退則有虧先旨。」詧重違其志,遂許之,因留鄉里,以經籍自娛。



 周文帝、明帝頻征,固辭以疾。及詧殂,遐舉哀,行舊臣之服。保定中,又徵之,遐始入朝,授驃騎大將軍、開府儀同三司、霍州刺史。遐導人務先以德,再三不用命者,乃微加貶異,示恥而已。其下感而化之,不復為過,咸曰:「我君仁惠如此,其可欺乎!」卒,贈金、安二州刺史。



 遐有至行。初為州主簿,其父卒于揚州,遐自襄陽奔赴,六日而至,哀感行路,毀悴不可識。後奉
 喪西歸。中流風起,舟中人相顧失色。遐抱棺號慟,訴天求哀,俄頃風止浪息。其母嘗乳間發疽,醫云:「此疾無可救理,唯得人吮膿,或望微止其痛。」遐應聲即吮,旬日遂瘳。咸以為孝感所致。性又溫裕,略無喜慍之容。弘獎名教,未嘗論人之短。尤尚施與,家無餘財。臨終遺誡簿葬,其子等並奉行之。



 有十子,靖、莊最知名。



 靖字思休,少方雅,博覽墳籍。仕梁,正員郎。隨遐入周,授大都督,歷河南、德廣二郡守。所居皆有政術,吏人畏而愛之。然性愛閑素,其於名利澹如也。及秩滿還鄉,便有終焉之志。隋文帝踐極,特詔征之,以疾固辭。優游不仕,
 閉門自守,所對唯琴書如已。足不歷園庭,殆將十載。子弟奉之若嚴君焉。其有過者,靖必下帷自責,於是長幼相率拜謝於庭,靖然後見之,勖以禮法。鄉里亦慕而化之,或有不善者,皆曰:「唯恐柳德廣知也。」時論方之王烈。前後總管到官,皆親至靖家問疾,遂以為故事。秦王俊臨州,賚以几杖,並致衣物。靖唯受几杖,餘並固辭。



 其為當時所重如此。開皇中,壽終。



 莊字思敬,少有器量,博覽墳籍,兼善辭令。濟陽蔡大寶有重名於江左,時為岳陽王蕭詧諮議,見莊,歎曰:「襄陽水鏡,復在於茲!」大寶遂以其女妻之。俄而察辟為參軍。
 及詧稱帝,累遷鴻臚卿。及隋文帝輔政,蕭巋令莊奉書入關。時三方構難,文帝懼巋有異志,及莊還,謂曰:「孤昔以開府從役江陵,深蒙梁主殊眷。



 今主幼時艱,猥蒙顧託。梁主弈業重光,委誠朝廷,而今已後,方見松筠之節。君還申孤此意於梁主也。」遂執莊手而別。時梁之將帥咸請與尉遲迥連衡,進可盡節於周氏,退可席卷山南,唯巋疑不可。會莊至自長安,申文帝結託之意,遂言於巋曰:「今尉遲迥雖曰舊將,昏耄已甚。消難、王謙常人之下者,非有匡合之才。況山東、庸蜀從化日近,周室之恩未洽於朝廷。臣料之,迥等終當覆滅,隨公必私周國,未
 若保境息人,以觀其變。」巋深以為然。未幾,消難奔陳,迥及謙相次就戮。



 巋謂莊曰:「近若從眾言,社稷已不守矣。」文帝踐祚,莊又入朝,帝深慰勉之。



 及為晉王廣納妃于梁,莊因是往來四五反,前後賜物數千段。梁國廢,授開府儀同三司,除給事黃門侍郎。



 莊明習舊章,雅達政事,凡所駁正,帝莫不稱善。蘇威為納言,重莊器識,常奏帝云:「江南人有學業者,多不習世務;習世務者,又無學業。能兼之者,不過柳莊。」高熲亦與莊甚厚。莊與陳茂同官,不能降意。茂見上及朝臣多屬意於莊,心每不平。帝與茂有舊,譖愬頗行。尚書省嘗奏犯罪人,依法合流,而上
 處以大辟。



 莊據法執之,帝不從,由是忤旨。俄屬尚藥進丸藥不稱旨,茂因奏莊不親監,帝怒。



 十一年,徐璒等反於江南,詔莊以行軍總管長史,隨軍討之。璒平,即授饒州刺史,甚有能名。卒於官。



 論曰:韓褒奉事三帝,以忠厚知名。趙肅平允當官,張軌循良播美,李彥譽流省閣,郭彥信著蠻貃,歷官出納,並當時之選也。梁昕、皇甫璠、辛慶之、王子直、杜杲之徒,並關右之舊族。或紆組登朝,獲當官之譽,或張旃出境,有專對之才,既茂國猷,克隆家業,美矣!魏文帝云「文人不護細行。」其呂思禮之謂乎!徐招、檀翥、孟信各以才學自
 業,又加之以清介,並志能之士也。宗懍才辭幹局,見重梁元,逮乎播越秦中,不預政事,豈亡國俘虜不與圖存者乎?梁氏據有江東五十餘載,挾策紀事,蓋亦多人。劉璠學思通博,有著述之譽,雖傳疑傳信,頗有詳略,而屬辭比事,為一家之言。行本正色抗言,具存乎骨鯁。柳遐立身之道,進退有節,觀其眷戀墳隴,其孝可移於朝廷;盡禮舊主,其忠可事於新君。夫能推此類以求賢,則知人幾於易矣。莊亮直之風,不殞門表,忠而獲謗,蓋亦自古有之。



\end{pinyinscope}