\article{卷十三列傳第一 後妃上}

\begin{pinyinscope}

 魏神元皇后竇氏文帝皇后封氏桓皇后惟氏平文皇后王氏昭成皇后慕容氏獻明皇后賀氏道武皇后慕容氏道武宣穆皇後劉氏明元昭哀皇後姚氏明元密皇后杜氏太武皇后赫連氏太武敬哀皇后賀氏景穆恭皇后鬱久閭氏文成文明皇后馮氏文成元皇后李氏獻文思皇后李氏孝文貞皇後林氏孝文廢皇后馮氏孝文幽皇后馮氏孝文文昭皇后高氏宣武順皇后于氏宣武皇后高氏宣武靈皇後胡氏孝明皇後胡氏孝武皇后高氏文帝文皇后乙弗氏文帝悼皇后鬱久閭氏廢帝皇后宇文氏恭帝皇
 后若干氏孝靜皇后高氏漢因秦制,帝之祖母曰太皇太后,母曰皇太后,妃曰皇后,餘則多稱夫人,隨世增損,非如《周禮》有夫人、嬪婦、御妻之數焉。魏、晉相因,時有升降,前史言之具矣。



 魏氏王業之兆,雖始於神元,然自昭成之前,未具言六宮之典;而章、
 平、思、昭、穆、惠、煬、烈八帝妃後無聞。道武追尊祖妣,皆從帝謚為皇后。始立中宮,餘妾或稱夫人,多少無限,然皆有品次。太武稍增左右昭儀及貴人、椒房等,後庭漸已多矣。又魏故事,將立皇后,必令手鑄金人,以成者為吉,不則不得立也。又太武、文成,保母劬勞之恩,並極尊崇之義,雖事乖典禮,而觀過知仁。孝
 文改定內官:左右昭儀位視大司馬,三夫人視三公,三嬪視三卿,六嬪視六卿,世婦視中大夫,御女視元士。後置女職,以典內司。內司視尚書令、僕。作司、大監、女侍中三官視二品。監、女尚書、美人,女史、女賢人、女書
 史、書女、小書女五官視三品。中才人、供人、中使、女生才人、恭使宮人視四品。青衣、女酒、女饗、女食、奚官女奴視五品。及齊神武、文襄,俱未踐尊極。神武嫡妻稱妃,其所娉蠕蠕女稱為蠕蠕公主。
 文
 襄既尚魏朝公主,故無別號。兩宮自餘姬侍,並稱娘而已。文宣後庭雖有夫人、嬪、御之稱,然未具員數。孝昭內職甚少,唯楊嬪才貌兼美,復是貴家,襄城王母桑氏有
 德行,並蒙恩禮,其餘無聞焉。



 河清新令:內命婦依古制有三夫人、九嬪、二十七世婦、八十一御女。又準漢制置昭儀,有左右二人,比丞相。其弘德、正德、崇德為三夫人,比三公。光猷、昭訓、隆徽為上嬪,比三卿。宣徽、凝暉、宣明、順華、凝華、光訓為下嬪、比六卿。正華、令側、修訓、曜儀、明淑、芳華、敬婉、昭華、光正、昭寧、貞範、弘徽、和德、弘猷、茂光、明信、靜訓、曜德、廣訓、暉範、敬訓、芳猷、婉華、明範、艷儀、暉則、敬信為二十七世婦,比從三品。穆光、茂德、貞懿、曜光、貞凝、光範、令儀、內範、穆閨、婉德、明婉、艷婉、妙範、暉章、敬茂、靜肅、瓊章、穆華、慎儀、妙儀、明懿、崇明、麗則、婉儀、彭媛、
 修閑、修靜、弘慎、艷光、漪容、徽淑、秀儀、芳婉、貞慎、明艷、貞穆、修範、肅容、茂儀、英淑、弘艷、正信、凝婉、英範、懷順、修媛、良則、瑤章、訓成、潤儀、寧訓、淑懿、柔則、穆儀、修禮、昭慎、貞媛、肅閨、敬順、柔華、昭順、敬寧、明訓、弘儀、崇敬、修敬、承閑、昭容、麗儀、閑華、思柔、媛光、懷德、良媛、淑猗、茂範、良信、艷華、徽娥、肅儀、妙則為八十一御女,比正四品。武成好內,並具其員,自外又置才人,採女,以為散號。後王既立二后,昭儀以下皆倍其數。又置左右娥英,比左右丞相,降昭儀比二大夫。尋又置淑妃一人,比相國。周氏率由姬制,內職有序。



 文帝創基,修衽席以儉約;武皇嗣歷,節情
 欲於矯枉。宮闈有貫魚之美,戚里無私溺之尤,可謂得君人之體也。宣皇外行其志,內逞其欲,溪壑難滿,採擇無厭;恩之所加,莫限廝皁;榮之所及,無隔險詖。於是升蘭殿以正位,踐椒庭而齊體者,非一人焉;階房帷而拖青紫,緣恩倖而擁玉帛,非一族焉。雖辛、癸之荒淫,趙、李之傾惑,曾未足比其仿佛也。人厭苛政,弊事實多,文帝之祀忽諸,特由於此。



 隋文思革前弊,大矯其違,唯皇后當室,傍無私寵;婦官位號,未詳備焉。開皇二年著內官之式,略依《周禮》,省減其數。嬪三員,掌教四德,視正三品;世婦九員,掌賓客祭祀,視正五品;女御三十八員,掌女
 功絲枲,視正七品。又采漢、晉舊儀,置六尚、六司、六典,遞相統攝,以掌宮掖之政。一曰尚宮,掌導引皇后及閨閣稟賜。管司令三人,掌圖籍法式,糾察宣奏;典琮三人,掌琮璽器玩。二曰尚儀,掌禮儀教學。管司樂三人,掌音律之事;典贊三人,掌導引內外命婦朝見。



 三曰尚服,掌服章寶藏。管司飾三人,掌簪珥花嚴;典櫛三人,掌巾櫛膏沐。四曰尚食,掌進膳先嘗。管司醫三人,掌方藥卜筮;典器三人,掌樽彞器皿。五曰尚寢,掌帷帳床褥。管司筵三人,掌鋪設灑掃;典執三人,掌扇傘燈燭。六曰尚工,掌營造百役。管司製三人,掌衣服裁縫;典會三人,掌財帛出
 入。六尚各三員視從九品,六司視勳品,六典視流外二品。



 初,文獻皇后功參歷試,外預朝政,內擅宮闈;懷嫉妒之心,虛嬪妾之位;不設三妃,防其上逼。自嬪以下,置六十員。加又抑損服章,降其品秩。至文獻崩後,始置貴人三員,增嬪至九員,世婦二十七員,御女八十一員。貴人等關掌宮闈之務,六尚以下皆分泰焉。



 煬帝時,后妃嬪御無釐婦職,唯端容麗飾,陪從宴遊而已。帝又參詳典故,自製嘉名,著之於令。貴妃、淑妃、德妃是為三夫人,品正第一。順儀、順容、順華、修儀、修容、修華、充儀、充容、充華,是為九嬪,品正第二。婕妤一十二員,品正第三。美人、才
 人一十五員,品正第四,是為世婦。寶林二十員,品正第五。御女二十四員,品正第六。採女三十七員,品正第七,是為女御,總一百二十,以敘於宴寢。又有承衣刀人,皆趨侍左右,並無員數,視六品以下。時又增置女官,準尚書省,以六局管二十四司。一曰尚宮局,管司言,掌宣傳奏啟;司簿,掌名錄計度;司正,掌格式推罰;司闈,掌門閣管籥。二曰尚儀局,管司籍,掌經史教學,紙筆几案;司樂,掌音律;司賓,掌賓客;司贊,掌禮儀贊相導引。三曰尚服局,管司璽,掌琮璽符節;司衣,掌衣服;司飾,掌湯沐巾櫛玩弄;司仗,掌仗衛戎器。



 四曰尚食局,管司膳,掌膳羞;司
 醖,掌酒醴益醢;司藥,掌醫巫藥劑;司饎,掌廩餼柴炭。五曰尚寢局,管司設,掌床席帷帳,鋪設灑掃;司輿,掌輿輦傘扇,執持羽儀;司苑,掌園禦種植,蔬菜瓜果;司燈,掌火燭。六曰尚工局,管司製,掌營造裁縫;司寶,掌金玉珠璣錢貨;司彩,掌繒帛;司織,掌織染。六尚二十二司,員各二人,唯司樂、司膳員各四人。每司又置典及掌,以貳其職。六尚十人,品從第五。司二十八人,品從第六。典二十八人,品從第七。掌二十八人,品從第九。女史流外,量局閑劇,多者十人以下,無定員數。聯事分職,各有司存焉。



 魏神元皇后竇氏,沒鹿回部大人賓之女也。賓臨終,誡
 其二子速侯、回題,令善事帝。及賓卒,速侯等欲因帝會喪為變。語泄,帝聞之,晨起以佩刀殺后。馳使告速侯等,言后暴崩。速侯等來赴,因執殺之。文帝皇后封氏,生桓、穆二帝,早崩。桓帝立,乃葬焉。文成初,穿天泉池,獲一石銘,稱桓帝葬母氏,遠近赴會二十餘萬。有司以聞,命藏之太廟。次妃蘭氏,是生思帝。



 桓皇后惟氏,生三子,長曰普根,次惠帝,次煬帝。平文崩。后攝國事,時人謂之曰「女國」。后性猛忌,平文之崩,后所為也。



 平文皇后王氏,廣寧人也。年十三,因事入宮,得幸於平
 文,生昭成帝。平文崩,昭成在襁褓,時國有內難,將害帝子。后匿帝於褲中,咒曰:「若天祚未終者,汝無聲。」遂良久不啼,得免於難。昭成初欲定都於水壘源川,築城郭,起宮室,議不決。后聞之曰:「國自上世,遷徙為業。今事難之後,基業未固,若郭而居,一旦寇來,難卒遷動。」乃止。烈帝之崩,國祚殆危;興復大業,后之力也。崩,葬雲中金陵。道武即位,配饗太廟。



 昭成皇后慕容氏,慕容晃之女也。初,帝納晃妹為妃,未幾而崩。晃後請繼好。



 遣大人長孫秩逆后,晃送于境上。后至,有寵,生獻明帝及秦明王。后性聰敏多智,專夕理
 內,每事多從。初,昭成遣衛辰兄悉勿祈還部落也,后誡之曰:「汝還,必深防衛辰。辰姦猾,終當滅汝。」悉勿祈死,其子果為衛辰所殺,卒如后言。建國二十三年,崩。道武即位,配饗太廟。



 獻明皇后賀氏,東部大人野干女也。少以容儀選入東宮,生道武。苻洛之內侮也,后與道武及故臣吏避難北徙。俄而高車來抄掠,后乘車避賊而南,中路失道,乃仰天曰:「國家胤胄豈正爾絕滅也!惟神靈扶助。」遂馳,輪正不傾。行百餘里,至七個山南而免難。



 後劉顯使人將害帝,帝姑為顯弟亢掞妻,知之,密以告后。梁眷亦來告難。
 后乃令帝去之。后夜飲顯醉,向晨,故驚廄中群馬,使起視馬,后泣謂曰:「吾諸子始皆在此,今盡亡失,汝等誰殺之?」故顯使不急追。道武得至賀蘭部,群情未甚歸附。后從弟外朝大人悅舉部隨從,供奉盡禮。顯怒,將害后,后奔亢掞家,匿神車中三日。亢掞舉室請救,乃得免。會劉顯部亂,始得亡歸。後后弟染干忌道武之得人心,舉兵圍逼行宮。后出謂染干曰:「汝等今安所置我,而欲殺吾子也?」染干慚而去。後后少子秦王觚使於燕,慕容垂止之。后以觚不反,憂念寢疾。皇始元年,崩,祔葬于盛樂金陵。後追加尊謚,配饗焉。



 道武皇后慕容氏,寶之季女也。中山平,入充掖庭,得幸。左丞相、衛王儀等奏請立皇后。帝從儀,令后鑄金人成,乃立之。封后母孟為漂陽君。後崩。



 道武宣穆皇后劉氏,劉眷女也。登國初,納為夫人,生華陰公主,後生明元。



 后專理內事,寵待有加,以鑄金人不成,故不登后位。魏故事,後宮產子,將為儲貳,其母皆賜死。道武末年,后以舊法薨。明元即位,追遵謚位,配饗太廟。自此後,宮人為帝母,皆正配饗焉。



 明元昭哀皇后姚氏,姚興女西平長公主也。明元以后納之,後為夫人。后以鑄金人不成,未升尊位,然帝寵禮
 如后。是後猶欲正位,后謙不當。泰常五年,薨。



 帝追恨之,贈皇后璽綬而加謚焉。葬雲中金陵。



 明元密皇后杜氏,魏郡鄴人,陽平王超之妹也。初以良家子選入太子宮,有寵,生太武。及明元即位,拜貴嬪。泰常五年,薨,謚曰貴嬪,葬雲中金陵。太武即位,追尊號謚,配饗太廟。又立廟于鄴,刺史四時薦祀。以魏郡,太后所生之邑,復其調役。後甘露降於廟庭。文成時,相州刺史高閭表修后廟,詔曰:「婦人外成,理無獨祀;陰必配陽,以成天地,未聞有莘之國立大姒之饗。此乃先皇所立,一時之至感,非經世之遠制,便可罷祀。」先是,太武保母竇
 氏,初以夫家坐事誅,與二女俱入宮;操行純備,進退以禮,明元命為太武保母。性仁慈,帝感其恩訓,奉養不異所生。及即位,尊為皇太后,封其弟漏頭為遼東王。太后訓釐內外,甚有聲稱。



 性恬素寡欲,喜怒不形於色,好揚人之善,隱人之過。帝征涼州,蠕蠕吳堤入寇,太后命諸將擊走之。真君元年,崩。詔天下大臨三日,太保盧魯元監護喪事,謚曰惠。葬崞山,從后意也。初,后嘗登崞山,顧謂左右曰:「吾母養帝躬,敬神而愛人,若死而不滅,必不為賤鬼。然於先朝,本無位次,不可違禮以從園陵。此山之上,可以終託。」故葬焉。別立后寢廟於崞山,建碑頌德。



 太武皇后赫連氏,屈丐女也。太武平統萬,納后及二妹,俱為貴人,後立為皇后。文成初,崩,祔葬金陵。



 太武敬哀皇后賀氏,代人也。初為夫人,生景穆。神蒨元年,薨,追贈貴嬪,葬雲中金陵。後追號尊謚,配饗太廟。



 景穆恭皇后郁久閭氏,河東王毗妹也。少以才,選入東宮。有寵,生文成皇帝而薨。文成即位,追尊號謚,葬雲中金陵,配饗太廟。



 又文成乳母常氏,本遼西人,因事入宮,乳帝,有劬勞保護之功。文成即位,尊為保太后,尋尊為皇太后,告於郊廟。和平元年,崩。詔天下大臨三日,謚曰昭。



 葬於廣寧磨笄山,俗謂之鳴雞山,太后遺志也。依惠
 太后故事,別立寢廟,置守陵二百家,樹碑頌德。



 文成文明皇后馮氏,長樂信都人也。父朗,秦、雍二州刺史、西城郡公。母樂浪王氏。后生於長安,有神光之異。朗坐事誅,后遂入宮。太武左昭儀,后之姑也,雅有母德撫養教訓。年十四,文成踐極,以選為貴人,後立為皇后。文成崩,故事國有大喪,三日後御服器物一以燒焚,百官及中宮皆號泣而臨之。后悲叫自投火,左右救之,良久乃蘇。獻文即位,尊為皇太后。丞相乙渾謀逆,獻文年十二,居于諒闇,太后密定大策,誅渾,遂臨朝聽政。及孝文生,太后躬親撫養。是後罷令不聽政事。太后行不正,內
 寵李弈,獻文因事誅之。太后不得意,遂害帝。承明元年,尊曰太皇太后,復臨朝聽政。后性聰達。自入宮掖,粗學書計;及登尊極,省決萬機。孝文詔罷鷹師曹,以其地為太后立報德佛寺。太后與孝文遊於方山,顧川阜有終焉之志。因謂群臣曰:「舜葬蒼梧,二妃不從,豈必遠祔山陵,然後為貴哉?吾百歲後,神其安此。」孝文乃詔有司營建壽陵於方山,又起永固石室,將終為清廟焉。太和五年起作,八年而成,刊石立碑,頌太后功德。



 太后以帝富於春秋,乃作《勸戒歌》三百餘章,又作《皇誥》十八篇,文多不載。太后立文宣王廟於長安,又立思燕佛圖於龍城,
 皆刊石立碑。太后又制,內屬五廟之孫、外戚六親緦麻,皆受復除。性儉素,不好華飾,躬御縵繪而已。宰人上膳,案裁徑尺,羞膳滋味,減於故事十分之八。太后嘗以體不安,服庵閭子,宰人昏而進粥,有蜒在焉,后舉匕得之。帝時侍側,大怒,將加極罰,太后笑而釋之。



 自太后臨朝專政,孝文雅性孝謹,不欲參決;事無巨細,一稟於太后。太后多智,猜忍,能行大事;殺戮賞罰,決之俄頃,多有不關帝者。是以威福兼作,震動內外。故杞道德、王遇、張祐、苻承祖等拔自微閹,歲中而至王公。王睿出入臥內,數年便為宰輔。賞齎千萬億計,金書鐵券,許以不死之
 詔。李沖以器能受任,亦由見寵幃幄,密加錫齎,不可勝數。后性嚴明,假有寵侍,亦無所縱。左右纖介之愆,動加棰楚,多至百餘,少亦數十。然性不宿憾,尋亦待之如初;或因此更加富貴,是以人人懷於利欲,至死而不思退。



 太后曾與孝文幸靈泉池,宴群臣及蕃國使人、諸方渠帥,各令為其方舞。孝文上壽,太后忻然作歌,帝亦和歌,遂命群臣各言其志,於是和歌者九十人。太后外禮人望,元丕、游明根等頒賜金帛輿馬;每至褒美睿等,皆引丕參之,以示無私。



 又自以過失,懼人議己,小有疑忌,便見誅戮。迄后之崩,孝文不知所生。至如李、李惠之徒,
 猜嫌覆滅者十餘家,死者數百人,率多枉濫,天下冤之。



 十四年,崩於太和殿,年四十九。其日有雄雉集于太華殿。帝酌飲不入口五日,毀慕過禮。謚曰文明太皇太后。葬于永固陵,日中而反,虞於鑒玄殿。詔曰:「尊旨從儉,不申罔極之痛;稱情允禮,仰損儉訓之德;進退思惟,倍用崩感。又山陵之節,亦有成命;內則方丈,外裁奄坎。脫於孝子之心有所不盡者,室中可二丈,墳不得過三十步。今以陵萬世所仰,復廣為六十步。孤負遺旨,益以痛絕!其幽房大小,棺槨質約,不設明器。至於素帳縵茵瓷瓦之物,亦皆不置。此則遵先志,從冊令。俱奉遺事,而有從
 有違,未達者或以致怪。梓宮之裏,玄堂之內,聖靈所憑,已一一奉遵,仰昭儉德。其餘外事,有所不從,以盡痛慕之情。其宣示遠近,著告群司,上明儉誨之美,下彰違命之失。」及卒哭,孝文服衰,近臣從服。三司以下外臣衰服者,變服就練;七品以下,盡除即吉。設附祭於太和殿,公卿以下始親公事。帝毀瘠,絕酒肉不御者三年。



 初,帝孝於太后,乃於永固陵東北里餘營壽宮,遂有終焉瞻望之志。及遷洛陽,乃自表瀍西以為山園之所,而方山虛宮號曰萬年堂云。



 文成元皇后李氏,梁國蒙縣人,頓丘王峻之妹也。后
 之生也,有異於常。父方叔,恆言此女當大貴。及長,姿質美麗。太武南征,永昌王仁出壽春,軍至后宅,因得后。及仁鎮長安,遇事誅,后與其家人送平城宮。高祖登白樓望見,美之。乃下臺,后得幸於齋庫中,遂有娠。常太后後問后,知之,時守庫者亦私書於壁記之,別加驗問,皆符同。及生獻文,拜貴人。太安二年,太后令依故事。令后具條記在南兄弟,及引所結宗兄洪之,悉以付託。臨決,每一稱兄弟,拊胸慟泣,遂薨。後謚曰元皇后,葬金陵,配饗太廟。



 獻文思皇后李氏,中山安喜人,南郡王惠之女也。姿德
 婉淑,年十八。以選入東宮。獻文即位,為夫人,生孝文帝。皇興三年,薨,葬金陵。承明元年,追崇號謚,配饗太廟。



 孝文貞皇后林氏,平涼人也。父勝,位平涼太守。叔父金閭,起自閹官。獻文初,為定州刺史,為乙渾所誅,及勝兄弟皆死。勝無子,有二女入掖庭。后容色美麗,得幸於孝文,生皇子恂。以恂將為儲貳,太和七年,后依舊制薨。帝仁恕不欲襲前事,而稟文明太后意,故不果行。謚曰貞皇后,葬金陵。及恂以罪賜死,有司奏追廢后為庶人。



 孝文廢皇后馮氏,太師熙之女也。太和十七年,孝文既終喪,太尉元丕等表以長秋未建,六宮無主,請正內位。
 孝文從之,立后為皇后,恩遇甚厚。孝文後重引后姊昭儀至洛,稍有寵,后禮愛漸衰。昭儀自以年長,且前入宮掖,素見待念,輕后而不率妾禮。后雖性不妒忌,時有愧恨之色。照儀規為內主,譖構百端,尋廢后為庶人。后貞謹有德操,遂為練行尼,後終於瑤光佛寺。



 孝文幽皇后亦馮熙女。母曰常氏。本賤微,得幸於熙,熙元妃公主薨後,遂主家事。生后與北平公夙。文明太皇太后欲家世貴寵,乃簡熙二女,俱入掖庭,時年十四。其一早卒。后有姿媚,偏見愛幸。未幾,疾病。太后乃遣還家為尼,帝猶留念焉。歲餘而太后崩,帝服終,頗存訪之。又
 聞后素疹痊除,遣閹官雙三念璽書勞問,遂迎赴洛陽。及至,寵愛過本初。當夕,宮人稀復進見。拜為左昭儀,後立為皇后。



 帝頻歲南征,后遂與中官高菩薩私亂。及帝在汝南不豫,后便公然醜恣,中常侍雙蒙等為其心腹。是時彭城公主,宋王劉昶子婦也,年少嫠居。北平公馮夙,后之周母弟也。后求婚於孝文,孝文許之。公主志不願,后欲強之,婚有日矣。公主密與侍婢及僮從十餘人,乘輕車,冒霖雨,赴懸瓠,奉謁孝文,自陳本意。因言后與菩薩亂狀。帝聞,因駭愕,未之信,而秘匿之。此後后漸憂懼。與母常氏求託女巫,禱厭孝文疾不起;一旦得如文
 明太后輔少主稱命者,賞報不貲。又取三牲,宮中祆祠,假言祈福,專為左道。母常或自詣宮中,或遣侍婢與相報答。



 帝至洛,執問菩薩、雙蒙等,具得情狀。帝以疾臥含溫室,夜引后,并列菩薩等於戶外。后臨入,令搜衣中,稱有寸刃便斬。后頓首泣謝,乃賜坐東楹,去御筵二丈餘。孝文令菩薩等陳狀,又讓后曰:「汝有妖術,可具言之。」后乞屏左右,有所密狀。孝文敕中常侍悉出,唯令長秋卿白整在側,取衛直刀拄之。后猶不言。



 孝文乃以綿堅塞整耳,自小語再三呼整,無所應,乃令后言。事隱,人莫知之。高祖乃喚彭城、北海二王令入坐,言:「昔是汝嫂,今便
 他人,但入勿避。」又曰:「此老嫗欲白刃插我肋上,可窮問本未,勿有所難。」又云:「馮家女不能復相廢逐,且使在宮中空坐,有心乃能自死,汝等勿謂吾猶有情也。」帝素至孝,猶以文明太后故,未行廢。二王出,乃賜后辭死訣,再拜稽首涕泣。及入宮後,帝命中官有問於后,后罵曰:「我天子婦,當面對,豈令汝傳也!」帝怒,敕后母常入,示與后狀,常撻之百餘乃止。



 帝尋南伐,且留京師。雖以罪失寵,而夫人嬪妾奉之如法。唯令世宗在東宮,無朝謁之事。帝疾甚,謂彭城王勰曰:「後宮久乖陰德,自絕於天,吾死後可賜自盡別宮,葬以后禮,庶掩馮門之大過。」帝崩,梓
 宮達魯陽,乃行遺詔。北海王詳奉宣遺旨,長秋卿白整等入授后藥。后走呼,不肯引決,曰:「官豈有此也!是此諸王輩殺我耳。」整等執持強之,乃含椒而盡。梓宮次洛南,咸陽王禧等知審死,相視曰:「若無遺詔,我兄弟亦當作計去之。豈可令失行婦人宰制天下,殺我輩也?」



 謚曰幽皇后,葬長陵塋內。



 孝文文昭皇后高氏,司徒公肇之妹也。父揚,母蓋氏,凡四男三女,皆生於東裔。孝文初,乃舉室西歸。近龍城鎮,鎮表后德色婉艷。及至,文明太后親幸北部曹見后,奇之,入掖庭,時年十三。初,后幼曾夢在堂內立,而日光自
 窗中照之,灼灼而熱,后東西避之,光猶斜照不已。如是數夕,怪之,以白其父颺。颺以問遼東人閔宗。宗曰:「此奇徵也。昔有夢月入懷,猶生天子,況日照之徵!此女將被帝命,誕育人君之象也。」後生宣武及廣平王懷、長樂公主。馮昭儀寵盛,密有母養帝心。后自代如洛陽,暴薨於汲郡之共縣,或云昭儀所賊也。宣武之為皇太子,二日一朝幽后,后拊念慈愛有加。孝文出征,宣武入朝,必久留后宮,親視櫛沐,母道隆備。



 其後有司奏請加號,謚曰文昭貴人。孝文從之。宣武踐阼,追尊配饗。后先葬在長陵東南,陵制卑局,因就起山陵,號終寧陵。置邑戶五百
 家。明帝時,更上尊號太后,以同漢、晉之典,正姑婦之禮,廟號如舊文昭。遷靈櫬於長陵兆內西北六十步。初,開終寧陵數丈,於梓宮上獲大蛇,長丈餘,黑色,頭有王字,蟄而不動;靈櫬既遷,還置蛇舊處。



 宣武順皇后于氏,太尉烈弟勁之女也。宣武始親政事,烈時為領軍,總心膂之任。以嬪御未備,左右諷諭,稱后有容德,帝乃迎入為貴人。時年十四,甚見寵愛,立為皇后。」后靜默寬容,性不妒忌。生皇子,三歲夭沒。其後暴崩,宮禁事秘,莫能知悉,而世議歸咎于高夫人。葬永泰陵,謚曰順皇后。



 宣武皇后高氏,文昭皇后弟偃之女也。宣武納為貴嬪,生皇子,早夭;又生建德公主。後拜為皇后,甚見禮重。性妒忌,宮人希得進御。及明帝即位,上尊號曰皇太后。尋為尼,居瑤光寺。非大節慶不入宮中。建德公主始五六歲,靈太后出覲母武邑君,時天文有變,靈太后欲以當禍,是夜暴崩,天下冤之。喪還瑤光佛寺,殯皆以尼禮。



 初,孝文幽后之寵也,欲專其愛,後宮接御,多見阻遏。孝文時言于近臣,稱婦人妒防,雖王者亦不能免,況士庶乎。宣武高后悍忌,嬪御有至帝崩不蒙侍接者。



 由是在洛二十餘年,皇子全育者唯明帝而已。



 宣武靈皇后胡氏,安定臨涇人,司徒國珍女也。母皇甫氏,產后之日,赤光四照。京兆山北縣有趙胡者,善於卜相,國珍問之,胡云:「賢女有大貴之表,方為天地母,生天地主,勿過三人知也。」后姑為尼,頗能講道。宣武初,入講禁中。



 積歲,諷左右稱后有姿行。帝聞之,乃召入掖庭,為充華世婦。而椒庭之中,以國舊制,相與祈祝,皆願生諸王、公主,不願生太子。唯后每稱:「夫人等言,何緣畏一身之死而令皇家不育塚嫡也?」明帝在孕,同列猶以故事相恐,勸為諸計。后固意確然,幽夜獨誓,但使所懷是男,次第當長子,子生,身死不辭。既誕明帝,進為充華嬪。先是,宣武頻喪
 皇子,自以年長,深加慎護,為擇乳保,皆取良家宜子者,養於別宮,皇后及充華皆莫得而撫視焉。



 及明帝踐阼,尊后為皇太妃,後尊為皇太后。臨朝聽政,猶曰殿下,下令行事。



 後改令稱詔,群臣上書曰陛下,自稱曰朕。太后以明帝沖幼,未堪親祭,欲傍《周禮》夫人與君交獻之義,代行祭禮。禮官博議以為不可,而太后欲以幃幔自鄣,觀三公行事。重問侍中崔光,光便據漢和熹鄧后薦祭故事。太后大悅,遂攝行初祀。



 太后性聰悟,多才藝,姑既為尼,幼相依託,略得佛經大義。親覽萬機,手筆斷決。



 幸西林園法流堂,命侍臣射,不能者罰之。又自射針孔,中之,大悅,賜
 左右布帛有差。先是,太后敕造申訟車,時御焉。出自雲龍大司馬門,從宮西北,入自千秋門,以納冤訟。又親策孝、秀、州郡計吏於朝堂。太后與明帝幸華林園,宴群臣於都亭曲水,令王公以下賦七言詩。太后詩曰:「化光造物含氣貞。」明帝詩曰:「恭己無為賴慈英。」王公以下賜帛有差。太后父薨,百僚表請公除,太后不許。



 尋幸永寧寺,觀建剎於九級之基,僧尼士女赴者數萬人。及改葬文昭高后,太后不欲令明帝主事,乃自為喪主。出至終寧陵,親奠遣事,還哭於太極殿,至於訖事,皆自主焉。後幸嵩高山,夫人、九嬪、公主以下從者數百人,升于頂中。廢
 諸淫祀,而胡天神不在其例。尋幸闕口溫水,登雞頭山,自射象牙簪,一發中之,敕示文武。



 時太后逼幸清河王懌,淫亂肆情,為天下所惡。領軍元叉、長秋卿劉騰等奉明帝於顯陽殿,幽太后於北宮,於禁中殺懌。其後太后從子都統僧敬與備身左右張車渠等數十人謀殺叉,復奉太后臨朝。事不克,僧敬坐徙邊,車渠等死,胡氏多免黜。



 後明帝朝太后於西林園,宴文武侍臣,飲至日夕,叉乃起至太后前自陳,外云太后欲害己及騰。太后答云:「無此語。」遂至于極昏。太后乃起執明帝手下堂,言:「母子不聚久,今暮共一宿,諸大臣送我入。」太后與帝向東
 北小閣,左衛將軍奚康生謀殺叉不果。



 自劉騰死,叉又寬怠,太后與明帝及高陽王雍為計,解叉領軍。太后復臨朝,大赦改元。自是朝政疏緩,威恩不立,天下牧守,所在貪婪。鄭儼污亂宮掖,勢傾海內;李神軌、徐紇並見親侍,一二年中,位總禁要。手握王爵,輕重在心,宣淫於朝,為四方之所穢。文武解體,所在亂逆,土崩魚爛,由於此矣。僧敬又因聚集親族,遂涕泣諫曰:「陛下母儀海內,豈宜輕脫如此!」太后大怒,自是不召僧敬。



 內為朋黨,防弊耳目,明帝所親幸者,太后多以事害焉。有蜜多道人,能胡語,帝置於左右。太后慮其傳致消息,三月三日,於城南大
 巷中殺之,方懸賞募賊。又於禁中殺領左右、鴻臚少卿谷會、紹達,並帝所親也。母子之間,嫌隙屢起。鄭儼慮禍,乃與太后計,因潘嬪生女,妄言皇子,便大赦,為武泰元年,復陰行鴆毒。其年二月,明帝暴崩,乃奉潘嬪女,言太子即位。經數日,見人心已安,始言潘嬪本實生女,今宜更擇嗣君,遂立臨洮王子釗為主,年始二、三歲,天下愕然。



 及爾朱榮稱兵度河,太后盡召明帝六宮,皆令入道,太后亦自落髮。榮遣騎拘送太后及幼主於河陰。太后對榮多所陳說,榮拂衣而起。太后及幼主並沈於河。太后妹馮翊君收瘞於雙靈寺。武帝時,始葬以后禮,而追
 加謚曰靈。



 孝明皇后胡氏,靈太后從兄冀州刺史盛之女。靈太后欲榮重門族,故立為皇后。



 明帝頗有酒德,專嬖充華潘氏,后及嬪御並無過寵。太后為帝選納,抑屈人流。時博陵崔孝芬、范陽盧道約、隴西李瓚等女,俱為世婦。諸人訴訟,咸見忿責。武泰初,后既入道,遂居於瑤光寺。



 孝武皇后高氏,齊神武長女也。帝見立,乃納為后。及帝西幸關中,降為彭城王韶妃。



 文帝文皇后乙弗氏,河南洛陽人也。其先世為吐谷渾渠帥,居青海,號青海王。



 涼州平,后之高祖莫瑰擁部落
 入附,拜定州刺史,封西平公。自莫瑰後,三世尚公主,女乃多為王妃,甚見貴重。父瑗,儀同三司、兗州刺史。母淮陽長公主,孝文之第四女也。后美容儀,少言笑,年數歲,父母異之,指示諸親曰:「生女何妨也。



 若此者,實勝男。」年十六,文帝納為妃。及帝即位,以大統元年冊為皇后。后性好節儉,蔬食故衣,珠玉羅綺絕於服玩。又仁恕不為嫉妒之心,帝益重之。生男女十二人,多早夭,唯太子及武都王戊存焉。時新都關中,務欲東討,蠕蠕寇邊,未遑北伐,故帝結婚以撫之。於是更納悼后,命后遜居別宮,出家為尼。悼后猶懷猜忌,復徙后居秦州,依子秦州刺
 史武都王。帝雖限大計,恩好不忘,後密令養髮,有追還之意。然事祕禁,外無知者。六年春,蠕蠕舉國度河,前驅已過夏,頗有言虜為悼后之故興比役。帝曰:「豈有百萬之眾為一女子舉也?雖然,致此物論,朕亦何顏以見將帥邪!」乃遣中常侍曹寵齎手敕令后自盡。后奉敕,揮淚謂寵曰:「願至尊享千萬歲,天下康寧,死無恨也。」因命武都王前,與之決。遺語皇太子,辭皆悽愴,因慟哭久之。侍御咸垂涕失聲,莫能仰視。召僧設供,令侍婢數十人出家,手中落髮。事畢,乃入室,引被自覆而崩,年三十一。鑿麥積崖為龕而葬,神柩將入,有二叢雲先入龕中,頃之
 一滅一出,後號寂陵。及文帝山陵畢,手書云,萬歲後欲令兵配饗。公卿乃議追謚曰文皇后,祔於太廟。廢帝時,合葬於永陵。



 文帝悼皇后郁久閭氏,蠕蠕主阿那瑰之長女也。容貌端嚴,夙有成智。大統初,蠕蠕屢犯北邊,文帝乃與約,通好結婚,扶風王孚受使奉迎。蠕蠕俗以東為貴,后之來,營幕戶席,一皆東向。車七百乘,馬萬匹,駝千頭。到黑鹽池,魏朝鹵簿文物始至。孚奏請正南面,后日:「我未見魏主,故蠕蠕女也。魏仗向南,我自東面。」



 孚無以辭。



 四年正月,至京師,立為皇后,時年十四。六年,后懷孕將產,居於瑤
 華殿,聞上有狗吠聲,心甚惡之。又見婦人盛飾來至后所,后謂左右:「此為何人?」醫巫傍侍,悉無見者,時以為文後之靈。產訖而崩,年十六,葬於少陵原。十七年,合葬永陵。當會橫橋北,后梓宮先至鹿苑,帝巉輬後來,將就次所,軌折不進。



 廢帝皇后宇文氏,周文帝女也。后初產之日,有雲氣滿室,芬氳久之。幼有風神,好陳列女圖,置之左右。周文曰:「每見此女,良慰人意。」廢帝之為太子,納為妃。及即位,立為皇后。志操明秀,帝深重之,專寵後宮,不置嬪御。帝既廢崩,后亦以忠於魏室罹禍。



 恭帝皇后若干氏,司空長樂正公惠之女也。有容色,恭帝納之為妃。及即位,立為皇后。後出家為尼,在佛寺薨,竟無謚。



 孝靜皇后高氏,齊神武之第二女也。天平四年,詔娉以為皇后,神武前後固辭,帝不許。興和初,詔司徒孫騰、司空襄城王昶等奉詔致禮,以後駕迎於晉陽之丞相第。五月,立為皇后,大赦。齊受禪,降為中山王妃。後降於尚書左僕射楊遵彥。



\end{pinyinscope}