\article{卷七十五列傳第六十三}

\begin{pinyinscope}

 趙煚趙芬王韶元巖宇文弼伊婁謙李圓通陳茂郭榮龐晃李安楊尚希張煚蘇孝慈元壽趙煚,字通賢,天水西人也。祖超宗,魏河東太守。父仲懿,尚書左丞。



 煚少孤,養母至孝。年十四,有人盜伐其父墓中樹者,煚對之號慟,因執送官。



 見魏右僕射周惠達,長
 揖不拜,自述孤苦,涕淚交集,惠達為之隕涕歎息者久之。



 及長,沈深有器局,略涉書記。周文帝引為相府參軍事。從破洛陽。及班師,煚請留撫納亡叛,從之。煚於是帥所領與齊人前後五戰,斬獲甚眾,以功封平定縣男。



 累轉中書侍郎。



 周閔帝受禪,遷硤州刺史。蠻酋向天王以兵攻信陵、秭歸,煚襲擊破之,二郡獲全。時周人於江南岸置安蜀城以禦陳,屬霖雨數旬,城頹者百餘步。蠻酋鄭南鄉叛,引陳將吳明徹欲掩安蜀。議者皆勸煚益脩守禦,煚不從,乃遣使說誘江外生蠻向武陽,令乘虛掩襲南鄉所居,獲其父母妻子。南鄉聞之,其黨各散,陳兵
 亦遁。



 明年,吳明徹屢為寇患,煚與前後十六戰,每挫其鋒。以功授開府儀同三司,再遷戶部中大夫。周武帝欲收齊河南地,煚諫曰:「河南洛陽,四面受敵,縱得不可以守。請從河北直指太原,傾其巢穴,可一舉以定。」帝不納,師竟無功。尋從上柱國于翼自三鴉道伐陳,剋十九城而還。以讒毀,功不見錄。累遷御正上大夫。



 煚與宗伯斛斯徵素不協,徵後出為齊州刺史,坐事下獄,自知罪重,遂踰獄走。



 帝大怒,購之甚急。煚密奏曰:「徵自以罪重,懼死遁逃,若不北走匈奴,則南奔吳越。徵雖愚陋,久歷清顯,奔彼敵國,無益聖朝。今炎旱為災,可因茲大赦。」



 帝從
 之。徵賴而免,煚卒不言。



 隋文帝為丞相,加上開府,再遷大宗伯。及踐阼,煚授璽紱。進位大將軍,賜爵金城郡公,拜相州刺史。朝廷以煚習故事,徵拜尚書右僕射。未幾,以忤旨出為陜州刺史,轉冀州刺史,甚有威惠。煚嘗有疾,百姓奔馳,爭為祈禱,其得人情如此。冀州市多姦詐,煚為銅斗鐵尺。置於肆,百姓便之。帝聞而嘉焉,頒之天下,以為常法。嘗有人盜煚田中蒿,為吏所執。煚曰:「此乃刺史不能宣風化,彼何罪也?」慰諭遣之,令人載蒿一車賜盜者,盜愧過於重刑。帝幸洛陽,煚來朝,帝勞之。卒於官。



 子義臣嗣,位至太子洗馬。後同楊諒反,誅。



 趙芬,字士茂,天水西人也。父諒,周秦州刺史。芬少有辯智,頗涉經史。周引為相府鎧曹參軍。歷記室,累遷開府儀同三司。性彊濟,所居之職,皆有聲績。



 周武帝親總萬機,拜內史下大夫,轉小御正。明習故事,每朝廷有所疑議,眾不能決者,芬輒為評斷,莫不稱善。後為司會。及申國公李穆討齊,引為行軍長史,封淮安縣男。再遷東京小宗伯,鎮洛陽。隋文帝為丞相,尉遲迥與司馬消難陰謀往來,芬察知之,密白帝。由是深見親委,遷東京左僕射,進爵郡公。開皇初,罷東京官,拜尚書右僕射,與郢公王誼脩律令。俄兼內史令,甚見信任。未幾,以老病出
 為蒲州刺史,加金紫光祿大夫,仍領關東運漕,賜錢百萬、粟五千石而遣之。後數年,上表乞骸骨,徵還京師。賜以三驥軺車,几杖被褥,歸于家。皇太子又致巾帔。後數年,卒,帝遣使致祭,鴻臚監護喪事。



 子元恪嗣,位揚州總管司馬,左遷候衛長史。



 少子元楷,與元恪皆明幹世事。元楷,大業中為歷陽郡丞,與廬江郡丞徐仲宗俱竭百姓之產,以貢於帝。仲宗遷南郡丞;元楷超拜江都丞,兼領江都宮監。



 王韶,字子相,自云太原晉陽人也,世居京兆。祖諧,原州刺史。父諒,早卒。



 韶幼而方雅,頗好奇節,有識者異之。在
 周,累以軍功,官至車騎大將軍、儀同三司。復轉軍正。周武帝既拔晉州,意欲旋師,韶諫曰:「取亂侮亡,正在今日。方欲釋之而去,臣愚深所未解。」帝大悅。及齊平,進位開府,封晉陽縣公,賜口馬雜畜萬計。遷內史中大夫。宣帝即位,拜豐州刺史,改封昌樂縣公。



 隋文帝受禪,進爵項城郡公,轉靈州刺史,加位大將軍。晉王廣之鎮并州,除行臺右僕射,賜綵五百匹。韶性剛直,王甚憚之,每事諮詢,不敢違法度。韶嘗奉使檢行長城,後王穿池,起三山,韶既還,自鎖而諫,王謝而罷之。帝聞而嘉嘆,賜金百兩,並後宮四人。平陳之役,以本官為元帥府司馬。及剋金
 陵,韶即鎮焉。



 晉王廣班師,留韶於石頭防遏,委以後事。歲餘,徵還。帝謂公卿曰:「晉王以幼出籓,遂能剋平吳、越,王子相之力也。」於是進位柱國,賜奴婢三百口,錦絹五千段。及上幸並州,以其稱職,特加勞勉。後上謂曰:「自朕至此,公鬚鬢漸白,無乃憂勞所致?柱石之望,唯在於公,努力勉之!」韶辭謝,上勞而遣之。



 秦王俊為並州總管,仍為長史。歲餘,馳驛入京,勞弊而卒。帝甚傷惜之,謂秦王使者曰:「語爾王,我前令子相緩來,如何乃遣馳驛?殺我子相,豈不由汝!」



 言甚⼎妻愴。使有司為立宅,曰:「往者何用宅為?但以表我深心耳!」又曰:「子相受我委寄,十有餘年,終
 始不易。寵章未極,舍我而死乎!」發言流涕。因命取子相封事數十紙,傳示群臣曰:「其真言匡正,裨益甚多,吾每披尋,未嘗釋手。」煬帝即位,追贈司徒、尚書令、靈豳等十州刺史、魏公。子士隆嗣。



 士隆略知書計,尤便弓馬,慷慨有父風。大業世,頗見親重,位備身將軍,改封耿國公。越王侗稱帝,士隆率數千兵自江淮而至。會王世充僭號,甚禮重之,署尚書右僕射。憂憤,疽發背卒。



 元巖,字君山,河南洛陽人也。父禎,魏敷州刺史。巖好讀書,不守章名,剛鯁有器局,以名節自許,少與勃海高熲、太原王韶同志友善。仕周,為武賁給事。



 大冢宰宇文護
 見而器之,以為中外記室。累遷內史中大夫,封昌國縣伯。周宣帝嗣位,為政昏暴,京兆郡丞樂運輿櫬詣朝堂,陳帝八失,言甚切至。帝大怒,將戮之,朝臣莫有救者。巖謂人曰:「臧洪同日,尚可俱死,其況比干乎?若樂運不免,吾將與之俱斃。」詣閤請見,言於帝曰:「樂運知書奏必死,所以不顧身命者,欲取後世名。陛下若殺之,乃成其名,落其術內。不如勞而遣之,以廣聖度。」運因獲免。後帝將誅烏丸軌,巖不肯署詔。御正顏之儀切諫不入,巖進繼之,脫巾頓顙,三拜三進。帝曰:「汝欲黨烏丸軌耶?」巖曰:「臣非黨軌,正恐濫誅,失天下望。」



 帝怒,使閹豎搏其面,遂廢
 於家。



 隋文帝為丞相,加開府、戶部中大夫。及受禪,拜兵部尚書,進爵平昌郡公。



 巖性嚴重,明達世務,每有奏議,侃然正色,廷爭面折,無所回避,上及公卿皆敬憚之。時帝懲周代諸侯微弱,以致滅亡,由是分王諸子,權侔王室,以為盤石之固。



 遣晉王廣鎮並州,蜀王秀鎮益州。二王年並幼,選貞良有重望者為之僚佐。時嚴與王韶為河北道行臺右僕射,帝謂曰:「公宰相大器,今屈輔我兒,亦如曹參相齊之意。」及巖到官,法令明肅,吏人稱焉。蜀王好奢,嘗欲取獠口為閹人,又欲生剖死囚,取膽為樂。巖皆不奉教,排閣切諫,王輒謝而止。憚巖為人,每循法度。蜀
 中獄訟,巖所裁斷,莫不悅服。有得罪者,謂曰:「平昌公與罪,吾何怨焉?」上甚嘉之,賞賜優洽。卒于官,上悼惜久之。益州父老莫不隕涕,于今思之。



 巖卒後,蜀王為非法,造渾天儀,又共妃出獵,以彈彈人,多捕山獠充宦者,僚佐無能諫止。及秀得罪,上曰:「元巖若在,吾兒豈有是乎!」



 子弘嗣。歷給事郎、司朝謁者、北平通守。



 宇文弼,字公輔,河南洛陽人也,其先與周同出。祖直力勤,魏鉅鹿太守。父珍,周宕州刺史。弼慷慨有大節,博學多通。仕周,嘗奉使鄧至國及黑水、龍涸諸羌,前後降附三十餘部。及還,奉詔脩定五禮,書成奏之,賜田二頃、粟
 百石。累遷小吏部,擢八人為縣令,皆有異績,世以為知人。轉內史都上士。



 武帝將謀出兵河陽以伐齊,弼進策曰:「齊氏建國,于今累世,雖曰無道,尚有其人。今若用兵,須擇其地。河陽要衝,精兵所聚,盡力攻圍,恐難得志。彼汾之曲,戍小山平,攻之易拔,用武之地也。」帝不納,師竟無功。建德五年,大舉伐齊,卒用弼策。於是募三輔豪俠少年數百人為別陽,從帝攻拔晉州,身被三瘡,苦戰不息,帝奇而壯之。因從平齊,以功拜上儀同,封武威縣公。宣帝嗣位,為守廟大夫。時突厥寇甘州,帝令侯莫陳昶擊之。弼為監軍,謂昶曰:「宜選精騎,直趨祈連之西。賊若收軍,
 必自蓼泉之北,此地險隘,兼下濕,度其人馬,三日方度。



 彼勞我逸,破之必矣。若邀此路,真上策也。」昶不能用,西取合黎,大軍行遲,虜已出塞。其年,弼又從梁士彥攻拔壽陽,改封安樂縣公,除澮州刺史,轉南司州刺史。司馬消難之奔陳,弼追之不及。遇陳將樊毅,戰於漳口,自旦及午,三戰三捷。除黃州刺史,轉南定州刺史。



 開皇初,以前功封平昌縣公。入為尚書右丞。時西羌內附,詔弼持節安集,置鹽澤、蒲昌二郡而還。遷左丞,當官正色,為百僚所憚。三年,突厥寇甘州,以行軍司馬元帥竇榮定擊破之。還除太僕少卿,轉吏部侍郎。平陳之役。楊素出
 信州道,令弼持節為諸軍節度,仍領行軍總管。劉仁恩之破陳將呂仲肅也,弼有謀焉。加開府,擢拜刑部尚書,領太子虞候率。上嘗親臨釋奠。弼與博士論議,詞致清遠。上大悅,謂群臣曰:「朕今睹周公之制禮,見宣尼之論孝,實慰朕心。」時朝廷以晉陽為重鎮,並州總管必屬親王,其長史、司馬亦一時高選。前長史王韶卒,以弼有文武幹用,出為並州長史。十八年,遼東之役,授元帥漢王府司馬,仍領行軍總管。



 軍還,歷朔、代、吳三州總管,皆有能名。煬帝即位,拜刑部尚書,仍持節,巡省河北。還除泉州刺史。復徵拜刑部尚書,轉禮部尚書。



 弼既以才能著
 稱,歷職顯要,聲望甚重,物議多見推許。帝頗忌之。時帝漸好聲色,尤勤遠略,弼謂高熲曰:「昔周天元好聲色亡國,以今方之,不亦甚乎!」



 又言「長城之役,幸非急務」。有人奏之,坐誅,天下冤之。所著辭賦二十餘萬言,為《尚書》、《孝經注》行於世。有子儉瑗。



 伊婁謙,字彥恭,本鮮卑人也。其先世為酋長,隨魏南遷。祖信,中部太守。



 父靈,相隆二州刺史。謙性忠直,善辭令。仕周,累遷宣納上士、使持節、驃騎大將軍。武帝將伐齊,召入內殿,問以兵事。對曰:「偽齊僭擅,跋扈不恭,沉溺倡優,耽昏曲蘗。其折衝之將斛律明月已斃讒人之口,上
 下離心。若命六師齊進,臣之願也。」帝大笑,因使謙與小司寇拓跋偉聘齊觀釁。帝尋發兵。齊主知之,令其僕射陽休之責謙曰:「貴朝盛夏徵兵,馬首何向?」答曰:「僕拭玉之始,未聞興師。設復西增白帝之城,東益巴丘之戍,豈足怪哉!」謙參軍高遵以情輸齊,遂留謙不遣。帝既克并州,召謙勞之。乃執遵付謙,任令報復,謙頓首請赦之。帝曰:「卿可聚眾唾面,令知愧也。」謙跪曰:「遵罪又非唾面之責。」帝善其言而止。



 謙竟待遵如初。尋賜爵濟陽縣伯,累遷前驅中大夫。大象中,進爵為侯,位開府。



 隋文帝作相,授亳州總管,俄征還京。恥與逆人王謙同名,因爾稱字。
 文帝受禪,以彥恭為左武候將軍,俄拜大將軍,進爵為公。後出為澤州刺史,清約自處,甚得人和。以疾去職,吏人攀戀,行數百里不絕。卒于家。子傑嗣。



 李圓通,京兆涇陽人也。少孤賤,給使隋文帝家。及帝為隋公,擢授參軍事。



 初,帝少時,每宴客,恆令圓通監廚。圓通性嚴整,左右婢僕,咸所敬憚。唯世子乳母恃寵輕之,賓客未供,每有干請。圓通不許,或輒持去。圓通大怒,叱廚人撾之數十,叫聲徹於閣內,僚吏左右,代其失色。賓去後,帝知之,召圓通命坐賜食,從此獨善之,以為堪當大任。帝作相,賜爵懷昌男。授帥都督,進爵新安子,委以
 心膂。圓通多力勁捷,長於武用。周代諸王素憚帝,伺便圖為不利,賴圓通保護,獲免者數矣。帝深感之,由是參預政事,授相國外兵曹,仍領左親信。尋授上儀同。



 帝受禪,拜內史侍郎,領左衛長史,進爵為伯。歷左右庶子、給事黃門侍郎、尚書左丞,攝刑部尚書,深被任信。伐陳之役,以行軍總管從楊素出信州道,以功進位大將軍。改封萬安縣公,揚州總管長史。秦孝王仁柔自喜,少斷決,府中事多決於圓通。入為司農卿,遷刑部尚書,後復為並州長史。孝王以奢得罪,圓通亦坐免。



 尋檢校刑部尚書事。仁壽中,以勳舊進爵郡公。煬帝嗣位,拜兵部尚書。
 帝幸揚州,以圓通留守京師。判宇文述田還百姓,述訴其受賂。帝怒,坐是免官。圓通憂懼發病,卒。贈柱國,封爵悉如故。



 子孝常,大業末,為華陰令。武德初,以應義旗功,封義安王。



 又有陳茂者,河東猗氏人。家世寒微,質直恭謹,為州里所稱。文帝為隋國公,引為僚佐,待遇與圓通等。每令典家事,常稱旨。後從帝與齊師戰於晉州,賊甚盛,帝將挑戰,茂固止不得,因捉馬鞚。帝怒,拔刀斫其額,流血被面,詞氣不撓。帝感而謝之,厚加禮敬。帝為丞相,委以心膂。及受禪,拜給事黃門侍郎,封魏城縣男,每典機密。轉益州總管司馬,遷太府卿,進爵為伯。卒官。子政
 嗣。



 政字弘道,倜儻有文武大略,善鐘律,便弓馬。少養宮中,年十七,為太子千牛備身。京都大俠劉居士重政才氣,數從之游。圓通子孝常與政相善,並與居士交結。及居士伏誅,政及常從坐,上以功臣子,撻之二百而赦之。由是不得調。煬帝時,歷位協律郎、通事謁者、兵曹承務郎。帝以其才,甚重之。宇文化及之亂,以為太常卿。後歸大唐,為梁州總管,遇賊見殺。



 郭榮,字長榮,自云太原人也。父徽,仕魏為同州司馬。時武元皇帝為刺史,由是與隋文帝有舊。徽後位洵州刺史、安城縣公。及帝受禪,拜太僕卿,卒官。榮容貌魁岸,外
 疏內密,與交者多愛之。周大塚宰宇文護引為親信。護察榮謹愿,擢為中外府水曹參軍。齊寇屢侵,護令榮於汾州觀城勢,時汾州與姚襄鎮相去懸遠,榮以二城孤迥,勢不相救,請於州鎮間更築城以相控攝,護從之。俄而齊將段孝先攻陷姚襄、汾州二城,唯榮所立者獨能自守。護作浮橋出兵,孝先於上流縱大筏擊浮橋,護令榮督便水者引取其筏。以功授大都督。護又以稽胡數為寇亂,使綏集之。



 榮於上郡、延安築周昌、弘信、廣安、招遠、咸寧等五城以遏其要路,稽胡由是不能為寇。周武親總萬機,拜宣納中士。後從平齊,以功封平陽縣男。遷
 司水大夫。



 榮少與隋文帝親狎,帝嘗與夜坐月下,謂榮曰:「吾仰觀玄象,俯察人事,周歷已盡,我其代之。」榮深自結納。未幾,周宣崩,文帝總百揆,召榮,撫其背笑曰:「吾言驗未?」既拜相府樂曹參軍。俄以本官復領籓部大夫。文帝受禪,引為內史舍人,以龍潛之舊,進爵蒲城郡公,位上儀同。累遷通州刺史。仁壽初,西南夷獠多叛,詔榮領八州諸軍事、行軍總管討平之。



 煬帝即位,入為武候驃騎大將軍,以嚴正聞。後黔字首領田羅駒阻清江作亂,夷陵諸郡人夷多應者,詔榮擊平之。遷左候衛將軍。從帝西征吐谷渾,拜銀青光祿大夫。遼東之役,以功進左
 光祿大夫。明年,帝復事遼東,榮以為中國疲弊,萬乘不宜屢動,乃言於帝,請止行。帝不納。復從軍攻遼東城,榮親蒙矢石,晝夜不釋甲胄。帝知之大悅,每勞勉之。帝後以榮年老,欲出為郡。榮陳請不願。哀之,拜右候衛大將軍。後數日,帝謂百僚曰:「誠心純至如郭榮者,固無比矣。」楊玄感之亂,帝令馳守太原。明年,從帝至柳城,卒於懷遠鎮。帝為廢朝,贈兵部尚書,謚曰恭。子福善。



 龐晃,字元顯,榆林人也。父虯,周驃騎大將軍。晃少以良家子召補州都督。



 周文帝署大都督,領親信兵,常置左右。晃因徙居關中。後遷驃騎將軍,襲爵比陽侯。衛王直
 出領襄州,晃以本官從。尋與長湖公元定擊江南,孤軍深入,沒於陳。



 數年,衛王直遣晃弟車騎將軍元俊賚絹八百匹贖焉,乃得歸。拜上儀同,復事衛王。



 時隋文帝出為隨州刺史,路經襄陽,衛王令晃詣文帝。晃知帝非常人,深自結納。



 及帝去官歸京師,晃迎見於襄邑。帝甚歡,與晃同飯,晃因曰:「公相貌非常,名在圖籙,九五之日,幸願不忘。」帝笑曰:「何妄言也!」頃之,有一雄雉鳴於庭,帝令晃射之,曰:「中則有賞。然富貴之日,持以為驗。」文帝受禪,與晃言及之,晃再拜曰:「陛下君臨宇內,猶憶曩時之言?」上笑曰:「公此言何得忘也!」尋加上開府,拜右衛將軍。進
 爵為公。河間王弘之擊突厥。晃性剛悍。時廣平王雄當途用事,勢傾朝廷,晁每陵侮之。嘗於軍中臥,見雄不起,雄甚銜之。復與高熲有隙。二人屢譖晃,由是宿衛十餘年,官不得進。出為懷州刺史,遷原州總管,卒於官。帝為廢朝,謚曰敬。



 子長壽,頗知名,位驃騎將軍。



 李安,字玄德,隴西狄道人也。父蔚,仕周,為相燕恆三州刺史、襄武縣公。



 安美姿容,善騎射。天和中,襲爵襄武公,授儀同、小司右上士。隋文帝作相,引之左右,遷職方中大夫。復拜安弟哲為儀同。安叔父梁州刺史璋時在京師,與周趙王謀害帝,誘哲為內應。哲謂安曰:「寢之則不
 忠,言之則不義,失忠與義,何以立身?」安曰:「丞相,父也,其可背乎!」遂陰白之。及趙王等伏誅,將加官賞,安頓首曰:「豈可將叔父之命以求官賞?」於是俯伏流涕,悲不自勝。帝為之改容曰:「我為汝特存璋子。」乃命有司罪止璋身,帝亦為安隱其事而不言。尋授安開府,進封趙郡公,哲上儀同、黃臺縣男。



 文帝即位,歷內史侍郎、尚書左丞、黃門侍郎。平陳之役為楊素司馬,仍領行軍總管,率蜀兵順流東下。時陳人屯白沙,安謂諸將曰:「水戰非北人所長。今陳人依險泊船,必輕我無備。夜襲之,賊可破也。」安率眾先鋒,大破陳師。詔書勞勉,進位上大將軍、郢州刺
 史。轉鄧州刺史。求為內職,帝重違其意,除左領左右將軍。遷右領軍大將軍。拜哲開府儀同三司、備身將軍。兄弟俱典禁衛,恩信甚重。



 十八年,突厥犯塞,以安為行軍總管,從楊素擊之。安別出長川,會虜渡河,與戰破之。仁壽元年,出安為寧州刺史,哲為衛州刺史。安子瓊,哲子瑋,始自襁褓,乳養宮中,至是年八九歲,始命歸家。其親顧如是。帝嘗言及作相時事,因愍安兄弟滅親奉國,乃下詔曰:「先王立教,以義斷恩,割親愛之情,盡事君之道,用能弘獎大節,體此至公。往者朕登庸惟始,王業初基,寧州刺史趙郡公李安,其叔璋潛結籓枝,包藏不逞。安與弟
 哲深知逆順,披露丹心,凶謀既彰,罪人斯得。



 朕每念誠節,嘉之無已。但以事涉其親,猶有疑惑,欲使安等名教之方,自處有地。



 朕常為思審,遂致淹年。今更詳案聖典,求諸往事,父子天性,忠孝猶不並立,況復叔姪恩輕,情禮本有差降。忘私奉國,深得正理。宜錄舊勳,重弘賞命。」於是拜安、哲俱為柱國,賜縑各五十匹、馬百匹、羊千口。以哲為備身將軍。進封順陽郡公。安謂親族曰:「雖家獲全,而叔父遭禍,今奉此詔,悲愧交懷。」因歔欷悲感,不能自勝。先患水病,於是疾甚而卒。謚曰懷。子瓊嗣。少子孝恭,最知名。



 哲,煬帝時工部尚書,後坐事除名,配防嶺南,
 道卒。



 楊尚希,弘農人也。祖真,魏天水太守。父承寶,商直淅三州刺史。尚希齠齔而孤,年十一,辭母請受業長安。范陽盧辯見而異之,令入太學,專精不倦,同輩皆共推服。周文帝嘗親臨釋奠,尚希時年十八,令講《孝經》,詞旨可觀。文帝奇之,賜姓普六茹氏。擢為國子博士,累轉舍人上士。明、武世,歷太學博士、太子宮尹、計部中大夫。賜爵高都侯,東京司憲中大夫。撫慰山東、河北,至相州而宣帝崩,與相州總管尉遲迥發喪於館。尚希出謂左右曰:「蜀公哭不哀而視不安,將有他計。吾不去,將及於難。」遂夜
 遁。及明,迥方覺,令數十騎追不及,遂歸京師。隋文帝以尚希宗室之望,又背迥而至,待之甚厚。及迥屯兵武陟,遣尚希領宗室兵三千人鎮潼關。尋授司會中大夫。文帝受禪,拜度支尚書,進爵為公。歲餘,出為河南道行臺兵部尚書,加銀青光祿大夫。尚希時見天下州郡過多,上表以為「今郡縣倍多於古,或地無百里,數縣並置,或戶不滿千,二郡分領。具僚以眾,資費日多,吏卒又倍,租調歲減。清幹良材,百分無一,動須數萬,如何可充!所謂人少官多,十羊九牧。今存要去閑,併小為大,國家則不虧粟帛,選用則易得賢才。」帝覽而嘉之,遂罷天下諸郡。
 後歷位瀛州刺史、兵部、禮部二尚書,授上儀同。尚希性惇厚,兼以學業自通,甚有雅望,為朝廷所重。上時每旦臨朝,日側不倦,尚希諫以為「陛下宜舉大綱,責成宰輔。繁碎之務,非人主所宜親。」上歡然曰:「公愛我者。」尚希有足疾,謂曰:「蒲州出美酒,足堪養病,屈公臥臨之。」



 於是拜蒲州刺史,仍領本州宗團驃騎。尚希在州,甚有惠政,復引瀵水立隄防,開稻田數千頃,人賴其利。卒官。謚曰平。



 子旻嗣,後封丹水縣公,位安定郡丞。



 張煚,字士鴻,河間鄚人也。父羨,少好學,多所通涉,仕魏,為蕩難將軍。



 從武帝入關,累遷銀青光祿大夫。周文引
 為從事中郎,賜姓叱羅氏。歷司織大夫、雍州中從事、應州刺史、儀同三司,賜爵虞鄉縣公。復入為司成中大夫,典國史。



 周代公卿,類多武將,唯羨以素業自通,甚為當時所重。後以年老致仕。隋文帝受禪,欽其德望,以書征之。及謁見,敕令勿拜,扶杖升殿,上降榻執手,與之同坐,宴語久之,賜以几杖。會遷都龍首,羨上表勸以儉約,上優詔答之。卒,贈滄州刺史,謚曰定。所撰《老子》、《莊子》義,名《道言》,五十二篇。



 煚好學,有父風。仕魏,位員外侍郎。周文引為外兵曹。明、武世,位冢宰司錄,賜爵北平縣子。宣帝時,加儀同,進爵為伯。隋文帝為丞相,煚深自推結。帝以
 其有乾用,甚親遇之。及受禪,拜為尚書右丞,進爵為侯。遷太府少卿,領營新都監丞。丁父憂去職,柴毀骨立。未期,授儀同三司,襲爵虞鄉縣公。歷太府卿、戶部尚書。晉王廣為揚州總管,授煚司馬,加銀青光祿大夫。



 煚性和厚有識度,甚有當時譽。後拜冀州刺史,晉王廣頻表請之,復為晉王長史,檢校蔣州事。及晉王為皇太子,復為冀州刺史,位上開府,吏人悅服,稱為良二千石。卒官。子慧寶,官至絳郡丞。



 開皇中,有劉仁恩者,政績為天下第一,擢拜刑部尚書。以行軍總管從楊素伐陳,與素破陳將呂仲肅於荊門,仁恩計功居多,授上大將軍,甚有當
 時譽。馮翊郭均、上黨馮世期並明悟有幹略,相繼為兵部尚書。此三人俱顯名於世,然事行闕落,史莫能知。



 蘇孝慈,扶風人也。父武,周兗州刺史。孝慈少沉謹,有器幹,美容儀。仕周,位至工部中大夫,封臨水縣公。隋文帝受禪,進爵安平郡公,拜太府卿。於時王業初基,徵天下匠,纖微之巧,無不畢集。孝慈總其事,世以為能。歷位兵部尚書,待遇愈密。時皇太子勇頗知時政,上欲重宮官之望,多令大臣領其職,拜孝慈太子右衛率,尚書如故。及於陜州置常平倉,轉輸京下,以渭水多沙,乍深乍淺,乃決渭水為渠以屬河,令孝慈督其役,渠成,上善之,又
 領太子左衛率,仍判工部、戶部二尚書,稱為幹理。進位大將軍,轉工部尚書,率如故。先是,以百僚供費不足,臺省府寺咸置廨錢,收息取給。孝慈以為官與百姓爭利,非興化之道,表請公卿已下給職田各有差,上並納焉。及將廢太子,憚其在東宮,出為淅州刺史。太子以孝慈去,形於言色。遷洪州總管,俱有惠政。後桂林山越相聚為亂,詔孝慈為行軍總管,擊平之。卒官。子會昌。



 孝慈兄順,周眉州刺史。



 子沙羅,字子粹。仕周,以破尉遲迥功,授開府儀同三司,封通泰縣公。開皇中,歷位資、邛二州刺史,檢校利州總管。從史萬歲擊西爨,進位大將軍。尋檢
 校益州總管長史。及蜀王秀廢,沙羅坐除名。卒于家。子康嗣。



 元壽,字長壽,河南洛陽人也。祖敦,魏侍中、邵陵王。父寶,周涼州刺史。



 壽少孤,性仁孝,九歲喪父,哀毀骨立,宗族鄉黨咸異之。事母以孝聞。及長,方直,頗涉文史。周武成初,封隆城縣侯。保定四年,封儀隴縣侯,授儀同三司。隋開皇初,議伐陳,以壽有思理,使於淮浦監修船艦,以強濟見稱。累遷尚書左丞。



 文帝嘗出苑觀射,文武並從。開府蕭摩訶妻患且死,奏請遣子向江南收其家產,御史見而不言。壽奏劾之曰:「御史之官,義存糾察,直繩莫舉,
 憲典誰寄?今月五日,鑾輿徙蹕,親臨射苑,開府儀同三司蕭摩訶幸廁朝行,預觀盛禮,奏稱請遣子世略暫往江南重收家產。妻安遇患,彌留有日,安若長逝,世略不合此行。竊以人倫之義,伉儷為重,資愛之道,烏烏弗虧。摩訶遠念資財,近忘匹好,一言纔發,名教頓盡,而兼殿內侍御史臣韓徵之等親所聞見,竟不彈糾。若知非不舉,情涉阿縱;如不以為非,豈關理識?儀同三司、太子左庶子、檢校書侍御史臣劉行本虧失憲體,何所逃愆?臣謬膺朝寄,忝居左轄,無容寢默,謹以狀聞。」上嘉納之。後授太常少卿,出為基州刺史,有公廉稱。入為太府少卿,
 進位開府。煬帝嗣位,漢王諒反,左僕射楊素為行軍元帥,壽為長史。事平,以功授大將軍。遷太府卿。大業四年,拜內史令,從帝西討吐谷渾,壽率眾屯金山,東西連營三百餘里以圍渾主。



 還拜右光祿大夫。七年,兼左翊衛將軍。從征遼東,在道卒。帝哭之甚慟,贈尚書右僕射、光祿大夫,謚曰景。



 子敏,頗有才辯,而輕險多詐。壽卒,帝追思之,擢敏守內史舍人。交通博徒,數泄省中語。化及之反,敏創其謀,偽授內史侍郎,為沈光所殺。



 論曰:二趙明習故事,當世咸推,及居端右,無聞殊績。故知人之分器,各有量限,大小云異,不可相踰。晉蜀二王,
 帝之愛子,擅以權寵,莫拘憲法。王韶、元巖任當彼相,並見嚴憚,莫敢為非,謇諤之風有足稱矣。宇文弼宇量宏遠,聲望攸歸,斯言不密,以致傾殞,惜矣!伊婁謙志識弘深,不念舊惡,請赦高遵之罪,有君子風焉。李圓通、郭榮、龐晃等或陳力經綸之際,或自結龍潛之始,其所以高位厚秩,隆恩殊寵,豈徒然哉!李安雖則滅親,而於義亦已疏矣。楊尚希譽望隆重,張煚、蘇孝慈威稱貞幹,並擢自開皇之初,蓋當時之選也。元壽之彈行本,有意存夫名教。然其計功稱伐,蓋不足雲,端揆之贈,則為優矣。



\end{pinyinscope}