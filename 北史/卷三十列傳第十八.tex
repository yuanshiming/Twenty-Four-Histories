\article{卷三十列傳第十八}

\begin{pinyinscope}

 盧玄玄孫思道昌衡元明潛盧柔子愷盧觀弟仲宣彪從子文偉盧同子斐兄子景裕景裕弟辯光光子賁光從弟勇盧誕盧玄,字子真,范陽涿人也。曾祖諶,晉司空劉琨從事中郎。祖偃,父邈,並仁慕容氏。偃為營丘太守,邈為范陽太守,皆以儒雅稱。



 神蒨四年,太武辟召天下儒俊,以玄為首。授中書博士,遷侍郎,本州大中正。



 使馮弘,稱臣請附。
 外兄司徒崔浩每與言輒歎曰:「對子真,使我懷古之情更深。」



 浩大欲齊整人倫,分明姓族。玄曰:「創制立事,各有其時,樂為此者,詎幾人也?



 宜三思。」浩當時雖無以異之,竟於不納。浩敗,頗亦由此。後賜爵固安子,散騎常侍,使宋。宋文帝與之言,嘉嘆良久,曰:「中郎,卿曾祖也!」還,遇疾,歸鄉卒。贈平東將軍、幽州刺史、固安侯,謚曰宣。



 子度世,字子遷。幼聰達,有計數。為中書學生,應選東宮。弱冠,與從兄遐俱以學行為時流所重。遐特為崔浩所敬,位至尚書、光祿大夫、范陽子。



 度世後以崔浩事,棄官逃於高陽鄭羆家,羆匿之。使者囚羆長子,將加捶楚。



 羆誡之曰:「
 君子殺身以成仁,汝雖死勿言。」子奉父言,遂被拷掠,乃至火爇其體,因以物故,卒無所言。度世後令弟娶羆妹,以報其恩。太武臨江,宋文使其殿上將軍黃延年至。帝問曰:「盧度世坐與崔浩親通,逃命江表,應已至彼。」延年對曰:「都下無聞,當必不至。」帝詔東宮赦度世宗族逃亡籍沒者,度世乃出。拜中書侍郎,襲爵。



 興安初,兼太常卿,立保太后父遼西獻王廟,進爵為侯。後除散騎侍郎,使宋,應對宋侍中柳元景失衷。還,被禁劾,經年乃釋。除濟州刺史。州接邊境,將士數相侵掠。度世乃禁勒所統,還其俘虜,二境以寧。後坐事免。尋除青州刺史,未拜,卒,謚
 曰惠。四子,伯源、敏、昶、尚之。



 初,玄有五子,唯度世嫡,餘皆別生。崔浩之難,其庶兄弟恆欲害之,度世常深忿恨。及度世有子,每誡絕妾孽,以防後患。至伯源兄弟,婢妾生子,雖形貌相類,皆不舉接。為識者所非。



 伯源小名陽烏,性溫雅寡欲,有祖父風。敦尚學業,閨門和睦。襲侯爵,降為伯。累加秘書監、本州大中正。時孝文帝將立馮后,先問伯源。請更簡卜。帝曰:「以先后之姪,朕意已定。」伯源曰:「雖奉敕如此,然臣心實有未盡。」及朝臣集議,執意如前。馮誕有盛寵,深以為恨,伯源不以介懷。及孝文議伐齊,伯源表以為萬乘親戎,轉運難繼。詔雖不從,而優答之。
 尋以齊武帝殂,停師。



 時涇州羌叛,殘破城邑。伯源以步騎六千號三萬,徐行而進。未經三旬,賊眾逃散。降者數萬口,唯梟首惡,餘悉不問。詔兼侍中。



 初,伯源年十四,嘗詣長安。將還,餞送者五十餘人,別於渭北。有相者扶風人王逵曰:「諸君皆不如此盧郎,雖位不副實,然得聲名甚盛,望踰公輔。後二十餘年,當制命關右,願不相忘。」此行也,相者年過八十,詣軍門請見,言敘平生。



 未幾,守儀曹尚書。



 及齊雍州刺史曹武請降,乃以伯源為使持節、安南將軍,督前鋒諸軍,徑赴樊、鄧。辭以儒生不行軍事,帝不許。伯源曰:「臣恐曹武為周魴耳。陛下宜審之。」



 武果
 偽降。伯源乃進攻赭陽,師敗,坐免官爵。尋曹母憂。服闋,兼太尉長史。



 後為徐州京兆王愉兼長史。愉時年少,事無巨細,多決於伯源。伯源以誠信御物,甚得東南人和。南徐州刺史沈陵密謀叛,伯源屢有表聞,朝廷不納。陵果逃叛。



 陵之餘黨,伯源皆撫而赦之,唯歸罪於陵,由是眾心乃安。



 景明初,卒於祕書監,年四十八,贈幽州刺史,復本爵固安伯。謚曰懿。



 初,諶父志,法鐘繇書,子孫傳業,累世有能名。至邈以上,兼善草跡。伯源習家法,代京宮殿,多其所題。白馬公崔宏亦善書,世傳衛瓘體。魏初工書者,崔、盧二門。伯源與李沖特相友善,沖重伯源門風,
 伯源私沖才官,故結為婚姻,往來親密。至於伯源荷孝文意遇,頗亦由沖。伯源有八子。



 長子道將,字祖業。應襲父爵而讓第八弟道舒,詔不許。道將引清河王國常侍韓子熙讓弟采魯陽男之例,詔乃許之。道將涉獵經史,風氣謇諤,頗有文才,為一家後來之寇,諸父並敬憚之。彭城王勰、任城王澄皆虛衿相待。勰為中軍大將軍,辟行參軍。累遷燕郡太守。道將下車表樂毅、霍原之墓,為之立祠。優禮儒生,厲勸學業,敦課農桑,墾田歲倍。卒於司徒司馬,贈太常卿,謚曰獻。所為文筆數十篇。



 子懷祖,太學博士、員外散騎侍郎,卒。子莊,少有美名,位都水使
 者,卒官。



 懷祖弟懷仁,字子友,涉學有辭。性恬靜,蕭然有閑雅致。歷太尉記室、弘農郡守,不之任,卜居陳留界。所著詩賦銘頌二萬餘言,撰《中表實錄》二十卷。懷仁有行檢,善與人交。與瑯邪王衍、隴西李壽之情好相得。常語衍云:「昔太丘道廣,許劭知而不顧;嵇生峭立,鐘會遇而絕言。吾處季、孟之間,去其太甚。」衍以為然。



 子彥卿,有學尚,仕隋位御史。撰《後魏紀》三十卷。貞觀中位石門令、東宮學士。道將弟道亮,字仲業,隱居不仕。子思道。



 思道字子行,聰爽俊辯,通侻不羈。年十六,中山劉松為人作碑銘,以示思道。



 思道讀之,多所不解。乃感激讀書,
 師事河間邢子才。後復為文示松,松不能甚解。



 乃喟然歎曰:「學之有益,豈徒然哉!」因就魏收借異書。數年間,才學兼著。然不持操行,好輕侮人物。齊天保中,《魏史》成,思道多所非毀。由是前後再被笞辱,因而落泊不調。



 後左僕射楊遵彥薦之於朝,解褐司空行參軍、長兼員外散騎侍郎,直中書省。



 文宣帝崩,當朝文士各作挽歌十首,擇其善者而用之。魏收、陽休之、祖孝徵等不過得一二首,唯思道獨有八篇。故時人稱為「八米盧郎」。後漏泄省中語,出為丞相西閣祭酒。歷太子舍人、司徒錄事參軍。每居官,多被譴辱。後以擅用庫錢,免歸家。嘗於薊北,悵
 然感慨,為五言詩見意,世以為工。後為給事黃門侍郎,待詔文林館。



 周武帝平齊,授儀同三司,追赴長安。與同輩陽休之等數人作《聽蟬鳴篇》。



 思道所為,詞意清切,為時人所重。新野庾信遍覽諸同作者,而深歎美之。未幾,母疾,還鄉。遇同郡祖英伯及從兄昌期等舉兵作亂,思道豫焉。柱國宇文神舉討平之。思道罪當斬,已在死中。神舉素聞其名,引出,令作露布。援筆立成,文不加點。神舉嘉而宥之。後除掌教上士。隋文帝為丞相,遷武陽太守。位下,不得志,為《孤鴻賦》以寄其情。其序曰:余志學之歲,自鄉里遊京師,便見識知音,歷受群公之眷。年登弱
 冠,甫就朝列;談者過誤,遂竊虛名。通人楊令君、邢特進以下,皆分庭致敬,倒屣相接,翦拂吹噓,長其光價。而才本駑拙,性實疏懶,勢利貨殖,淡然不營。雖籠絆朝市,且三十載,而獨往之心,未始去懷抱也。



 攝生舛和,有少氣疾。分符坐嘯,作守東原。洪河之湄,沃野彌望,囂務既屏,魚鳥為鄰。有離群之鴻,為羅者所獲,野人馴養,貢之於餘。置諸池庭,朝夕賞玩,既用銷憂,兼以輕疾。《大易》稱「鴻漸於陸」,羽儀盛也。揚子曰「鴻飛冥冥」,騫翥高也。《淮南子》云「東歸碣石」,違溽暑也。平子賦「南翔衡陽」,避祁寒也。若其雅步清音,遠心高韻,鵷鸞已降,罕見其儔。而鎩翮墻
 陰,偶影獨立,唼喋秕稗,雞鶩為伍,不亦傷乎。



 餘五十之年,忽焉已至,永言身事,慨然多緒,乃為之賦,聊以自慰云。



 開皇初,以母老,表請解職,優詔許之。思道恃才地,多所陵轢,由是官途淪滯。既而又著《勞生論》,指切當世。歲餘,奉詔郊勞陳使。頃之,遭母憂。未幾,起為散騎侍郎,參內史侍郎事。于時,議置六卿,將除大理。思道上奏曰:「省有駕部,寺留太僕;省有刑部,寺除大理。斯則重畜產而賤刑名,誠為不可。」又陳殿庭非杖罰之所,朝臣犯笞罪,請以贖論。上悉嘉納之。是歲,卒于京師。上甚惜之,遣使吊祭焉。集二十卷,行於世。子赤松,大業中,位河東縣長。



 道亮弟道裕,字寧祖。少以學尚知名,風儀兼美。尚獻文女樂浪長公主,拜駙馬都尉。歷位中書侍郎、太子中庶子、幽州大中正,卒於涇州刺史,謚曰文。



 道裕弟道虔,字慶祖。粗閑經史,兼通算術。尚孝文女濟南長公主,拜駙馬都尉。公主驕淫,聲穢遐邇,無疾暴薨,時云道虔所害。宣武祕其事,不苦窮之。後靈太后追主薨事,黜道虔,令終身不仕。道虔外生李彧,尚莊帝姊豐亭公主,因相藉託。永安中,除輔國將軍、通直常侍。以議曆勛,賜爵臨淄伯。天平中,歷都官尚書、本州大中正,幽州刺史,加衛大將軍。卒官,贈尚書右僕射、司空公、瀛州刺史,謚曰文恭。



 道虔好《禮》學,難齊尚書令王儉《喪服集記》七十餘條。為尚書同寮於草屋下設雞黍之膳,談者以為高。昧旦將上省,必見其弟然後去。奴在馬上彈琵琶,道虔聞之,杖奴一百。公主二子,昌寓宇、昌仁。昌宇不慧,昌仁早卒。道虔又娶司馬氏,有子昌裕。後司馬氏見出,更娉元氏,甚聰悟,常升高座講《老子》。道虔從弟元明隔紗帷以聽焉。元氏生二子,昌斯、昌衡,昌衡最知名。



 昌衡字子均,小字龍子。沈靖有才識,風神淡雅,容止可法。博涉經史,工草行書。從弟思道,小字釋奴,宗中稱英妙,昌衡與之俱被推重。故幽州語曰:「盧家千里,釋奴、龍
 子。」仕魏,兼太尉外兵參軍。齊受禪,歷平恩令。右僕射祖孝徵薦為尚書金部郎。孝徵每曰:「吾用盧子均為尚書郎,自謂無愧幽明。」始天保中,尚書王昕以雅談獲罪,諸弟尚守而不墜。自茲以後,此道浸微。昌衡與頓丘李若、彭城劉氏、河南陸彥師、隴西辛德源、王循並為後進風流之士。後兼散騎侍郎,迎勞周使。周武平齊,授司玉中士,與大宗伯斛斯征修《禮令》。



 隋開皇初,拜尚書祠部侍郎。文帝嘗大集群下,令自陳功,人皆競進,昌衡獨無所言。左僕射高熲目而異之。陳使賀徹、周濆相繼來聘,朝廷每令昌衡接對之。



 未幾,出為徐州總管長史,甚有能
 名。吏部尚書蘇威考之曰:「德為世表,行為士則。」論之者以為美談。常行至浚儀,所乘馬為人牛所觸致死。牛主陳謝,求還價直。昌衡謂曰:「六畜相觸,自關常理,此豈人情也,君何謝焉?」拒而不受。性寬厚不校,皆此類也。轉壽州總管長史。宇文述甚敬之,委以州務。歲餘,遷金州刺史。仁壽中,奉詔持節為河南道巡省大使。及還,以奉使稱旨,授儀同三司,賜物二百段。昌衡自以年在縣車,上表乞骸骨,優詔不許。大業初,徵為太子左庶子,行詣洛陽,道卒。子寶素、寶胤。



 道虔弟道侃,字希祖,沈雅有學尚,位州主簿,卒。以弟道約子正達為後。



 道侃弟道和,字叔
 雍,兄弟之中,人望最下。位冀州中軍府中兵參軍,卒。子景猷,弘農太守。景猷子士彥,有風概,隋開皇中,為蜀王秀屬。以秀所為不軌,辭疾,終於家。



 道和弟道約,字季恭,位司徒屬、幽州大中正。興和末,除衛大將軍、兗州刺史,在州頗得人和。卒,贈儀同三司、幽州刺史。



 子正通,少有令譽,位開府諮議,卒。妻謝氏,與正通弟正思淫亂,為御史所劾,人士疾之。正思弟正山字公順,早以文學見知,為符璽郎,待詔文林館。正思兄弟以齊太后舅氏,武平中,並得優贈。



 道約弟道舒,字幼安,襲父爵,位中書侍郎,卒。子熙裕襲。熙裕清虛守道,有古人風,為親表所敬。



 伯
 源弟敏,字仲通,小字洪崖,少有大量。孝文器之,納其女為嬪。位儀曹郎,早卒,贈威遠將軍、范陽太守,謚曰靖。五子。



 長義僖,字遠慶,早有學尚,識度沈雅。年九歲喪父,便有至性,少為僕射李沖所歎美。起家秘書郎,累遷冠軍將軍、中散大夫,以母憂去職。幽州刺史王誦與之交款,每與故舊李神俊等書曰:「盧冠軍在此,時復惠存,輒連數日,得以諮詢政道。」其見重若此。後拜征虜將軍、太中大夫,散秩多年,澹然自得。李神俊勸其幹謁當途,義僖曰:「既學先王之道,貴行先王之志,何得茍求富貴也?」孝昌中,除散騎常侍。時靈太后臨朝,黃門侍郎李神軌勢
 傾朝野,求結婚姻。義僖慮其必敗,拒而不許。王誦謂義僖曰:「昔人不以一女易五男,卿易之也?」義僖曰:「所以不從,正為此耳。從,恐禍大而連速。」誦乃握義僖手曰:「我聞有命,不改以告人。」遂適他族。臨婚之夕,靈太后遣中常侍服景就家敕停,內外惶怖,義僖夷然自若。普泰中,除都官尚書、驃騎大將軍、左光祿大夫。



 義僖寬和畏慎,不妄交款。性清儉,不營財利。少時,幽州頻遭水旱,先有數萬石穀貸人,義僖以年穀不熟,乃燔其契,州閭悅其恩德。雖居顯位,每至困乏,麥飯蔬食,怡然甘之。卒,贈大將軍、儀同三司、瀛州刺史,謚曰孝簡。



 子遜之,清靖寡欲,位
 太尉記室參軍。義僖四弟,並遠不逮兄也。



 敏弟昶,字叔達,小字師顏,學涉經史,早有時譽。太和中,兼員外散騎常侍,使於齊。孝文詔昶曰:「密邇江揚,不早當晚,會是朕物。卿等欲言便言,無相疑難。」又敕副使王清石曰:「卿莫以南人語致疑盧昶。若彼先有知識,欲見但見,須論即論。昶正寬柔君子,無多文才,或主客命卿作詩,莫以昶不作,便罷也。凡使人以和為貴,勿相矜誇,見於色貌。」及至彼,遇齊明立,孝文南討,昶兄伯源為別道將。而齊明以朝廷加兵,遂酷遇之。昶等本非骨鯁,大怖,淚汗橫流。齊明以腐米臭魚莝豆供之。而謁者張思
 寧,辭氣謇愕,遂以壯烈死於館中。昶還,孝文責之曰:「銜命之禮,有死無辱,雖流放海隅,猶宜抱節致殞。卿不能長纓羈首,已是可恨。乃俛眉飲啄,自同犬馬。有生必死,修短幾何?卿若殺身成名,貽之竹素,何如甘彼芻菽,以辱君父。縱不能遠慚蘇武,寧不近愧思寧!」遂見罷黜。



 景明初,除中書侍郎,遷給事黃門侍郎、本州大中正、散騎常侍,兼尚書。時洛陽縣獲白鼠。昶奏,以為案《瑞典》,外鎮刺史二千石令長不祗上命,刻暴,百姓怨嗟,則白鼠至。因陳時政,多所勸誡。詔書褒美其意。轉侍中,又兼吏部尚書,尋即正,仍侍中。昶守職而已,無所激揚,與侍中元
 暉等更相朋附,為宣武所寵,時人鄙之。出為徐州刺史。昶既儒生,本少將略,又羊社子燮為昶司馬,專任戎事,掩昶耳目,將士怨之。朐山戍主傅文驥糧樵俱罄,以城降梁。昶見城降,先走退,諸軍相尋奔遁。遇大寒,軍人凍死及落手足者太半。自魏經略江右,唯中山王英敗於鐘離,昶於朐山失利,最為甚焉。宣武遣黃門甄琛馳驛鎖昶,窮其敗狀,詔以免官論。自餘將統以下,悉聽依赦復任。未幾,拜太常卿,仍除雍州刺史,進號鎮西將軍,加散騎常侍。卒官,謚曰穆。



 昶寬和矜恕,善於綏懷。其在徐州,戍兵有疾,親自檢恤,至番兵年滿不歸,容充後役,終
 昶一政,然後始還,人庶稱之。



 子元聿,字仲訓,無他才能。尚孝文女義陽長公主,拜駙馬都尉。位太尉司馬、光祿大夫。卒,贈中書監。子士晟,儀同開府掾。



 元聿第五弟元明,字幼章。涉歷群書,兼有文義,風彩閑潤,進退可觀。永安初,長兼尚書令、臨淮王彧欽愛之。及彧開府,引為兼屬,仍領部曲。孝武登阼,以郎任行禮,封城陽縣子,遷中書侍郎。永熙末,居洛東緱山,乃作《幽居賦》焉。



 於時,元明友人王由居潁川,忽夢由攜酒就之言別,賦詩為贈。及明,憶其詩十字,云:「自茲一去後,市朝不復遊。」元明歎曰:「由性不狎俗,旅寄人間,乃有
 今夢,詩復如此,必有他故。」經三日,果聞由為亂兵所害。尋其亡日,乃是發夢之夜。天平中,兼吏部郎中,副李諧使梁,南人稱之。還,拜尚書右丞,轉散騎常侍,監起居。積年在史館,了不措意。又兼黃門郎、本州大中正。



 元明善自標置,不妄交遊,飲酒賦詩,遇興忘返。性好玄理,作史子雜論數十篇,諸文別有集錄。少時,常從鄉還洛,途遇相州刺史、中山王熙。熙,博識之士,見而歎曰:「盧郎有如此風神,唯須誦《離騷》,飲美酒,自為佳器。」遂留之數日,贈帛及馬而別。元明凡三娶,次妻鄭氏與元明兄子士啟淫汙,元明不能離絕。



 又好以世地自矜,時論以此貶
 之。



 元明弟元緝,字幼緒,兇粗好酒,曾於婦氏飲宴,小有不平,手刃其客。位輔國將軍、司徒司馬,贈驃騎大將軍、吏部尚書、幽州刺史,謚曰宣。



 昶弟尚之,字季儒,小字羨夏。亦以儒素見重,位司徒左長史、前將軍、濟州刺史、光祿大夫。



 長子文甫,字元祐,涉歷文史,有名譽於時。位司空行參軍。文甫弟文翼,字仲祐,少甚輕躁,晚頗改節。以軍功賜爵范陽子,位太中大夫。文翼弟文符,字叔偉,性通率,位通直散騎侍郎。子潛。



 潛容貌瑰偉,善言談,少有成人志尚,累遷大將軍府中兵參軍,機事強濟,為文襄所知,言其終可大用。王思政見獲於潁川,文襄重其才
 識。潛常從容白文襄:「思政不能死節,何足可重?」文襄謂左右曰:「我有盧潛,便是更得一王思政。」



 天保中,除左戶郎中。坐譏議《魏書》,與王松年、李庶等俱被禁止。會清河王岳救江陵,特赦潛為岳行臺郎。還,歷中書、黃門侍郎。為奴誣告謀反,文宣明之,以奴付潛,潛不之責。黃門鄭子默奏潛從清河王岳南討,岳令潛說梁將侯瑱,大納瑱賂遺,還不奏聞。文宣杖潛一百,仍截其鬚,潛顏色不變。歷魏尹丞、懷州別駕、江州刺史,所在有善政。



 孝昭作相,以潛為揚州道行臺左丞。先是,梁將王琳擁其主蕭莊歸壽陽,朝廷以琳為揚州刺史,敕潛與琳為南討經
 略。後除行臺尚書、儀同三司。王琳銳意圖南,潛以為時事未可,由是與琳有隙,更相表列。武成追琳入鄴,除潛揚州刺史,領行臺尚書。潛在淮南十三年,大樹風績,為陳人所憚。陳主與其邊將書云:「盧潛猶在,卿宜深備之。」文宣初平淮南,給復十年,年滿後,逮天統、武平中,征稅頗雜。又高元海執政,斷漁獵,人家無以自資。諸商胡負官責息者,宦者陳德信縱其妄注淮南富家,令州縣徵責。又敕送突厥馬數千匹於揚州管內,令土豪貴買之,錢直始入。便出敕括江、淮間馬並送官廄。由是百姓騷擾,切齒嗟怨。潛隨事撫慰,兼行權略,故得寧靖。武平三
 年,徵為五兵尚書。揚州吏人以潛斷酒肉,篤信釋氏,大設僧會,以香花緣道流涕送之。潛歎曰:「正恐不久復來耳!」至鄴未幾,復為揚州道行臺尚書。



 四年,陳將吳明徹來寇,領軍封輔相赴援。陳兵及峴,輔相不從,潛固爭不得,憂憤發病,臥幕下,果敗。陳人遂圍壽陽,壅芍陂,以水灌之。詔王長春為南討都督。長春軍次河南,多給兵士糧,便鳴角欲引,而賤糴其米;及頓兵,更貴糶其米。



 乃與皮景和擁眾十萬於淮北,不進。壽陽城中青黑龍升天,城尋陷。潛及行臺僕射王貴顯、特進巴陵王王琳、扶風王可朱渾孝裕、武衛將軍奚永樂、儀同索景和、仁州刺
 史酈伯偉、霍州刺史封子繡、泰州刺史高子植、行臺左丞李騊駼等督將五十八,軍士一萬,皆沒焉。陳人殺王琳,餘皆囚於東冶。陳主欲知齊之虛實,乃出潛,曰:「囚本屬幽州,於河北最小,口有五十萬,落陳者,唯與酈伯偉二人耳。」



 時李騊駼將逃歸,並要潛。潛曰:「我此頭面,何可誑人?吾少時,相者云:沒在吳越地。死生已定,弟其行也。」因寄書與弟士邃曰:「吾夢汝以某月某日得患,某月某日漸損。」皆如其言。既而歎曰:「壽陽陷,吾以頸血濺城而死,佛教不聽自殺,故荏苒偷生,今可死矣!」於是閉氣而絕。其家購屍歸葬,贈開府儀同三司、尚書左僕射、兗州
 刺史。無子,以弟士邃子元孝嗣。



 潛雅性貞固。祖珽常要潛陷仁州刺史劉逖,許以高位。潛曰:「如此事,吾不為也。」行臺慕容恃德常所推重,有疾,謂其子曰:「盧尚書教我為人,有如昆弟。



 我死,持上騂馬與之。」其子以他馬往。恃德柩出門自停,不可動,巫祝以為恃德聲怒曰:「何不與盧尚書我所騎騂馬?」其子遽奉命,柩乃行。潛以馬價為營福事。



 其為時重如此。



 士邃字子淹,少為崔昂所知。昂云:「此昆季足為後生之俊,但恨其俱不讀書耳。」位尚書左右丞、吏部郎中、中山太守帶定州長史。齊亡後,卒。



 度世之為濟州也,魏初平升城。無鹽房崇吉母傅,度世繼
 外祖母兄之子婦也,兗州刺史申纂妻賈氏,崇吉之姑女也,皆亡破,老病憔悴。而度世推計中表,致其供恤。每覲見傅氏,跪問起居,隨時奉送衣被食物;亦存賑賈氏,供其服膳。青州既陷,諸崔墜落,多所收贖。及伯源、昶等,並循父風。遠親疏屬,敘為尊行長者,莫不畢拜致敬。閨門之禮,為世所推。謙退簡約,不與世競。父母亡後,同居共財,自祖至孫,家內百口。在洛時,有饑年,無以自贍,然尊卑怡穆,豐儉同之。親從昆季,常旦省諸父,出坐別室,暮乃入內。朝府之外,不妄交遊。其相勖以禮,如此。又一門三主,當世以為榮。伯源兄弟亡,及道將卒後,家風衰
 損。子孫多有非法,幃薄混穢,為時所鄙。



 度世從祖弟神寶,位中書博士。孝文為弟高陽王雍納其女為妃。



 初,玄從祖兄溥,慕容寶之末,統攝鄉部屯海濱,殺其鄉姻諸祖十餘人,稱征北大將軍、幽州刺史,攻掠郡縣。天興中,討禽之。



 溥玄孫洪,字曾孫。太和中,位中書博士,樂陵、陽平二郡太守,幽州中正。



 洪三子。長子崇,少立美名,有識者許之以遠大,卒於驃騎府法曹參軍。崇子柔。



 柔字子剛。少孤,為叔母所養,撫視甚於其子。柔盡心溫清,亦同己親,親族歎重之。性聰敏好學,未冠解屬文。但口吃,不能持論。頗使酒誕節,為世所譏。



 司徒、臨淮王彧
 見而器之,以女妻焉。



 及魏孝武與齊神武有隙,詔賀拔勝出牧荊州。柔謂因此可著功績,遂從勝之荊州。以柔為大行臺郎中,掌書記,軍之機務,柔多預之。及勝為太保,以柔為掾。



 孝武後召勝引兵赴洛,勝以問柔。柔曰:「高歡託晉陽之甲,意實難知。公宜席卷赴都,與決勝負,存沒以之,此忠之上策也。若北阻魯陽,南並舊楚,東連兗、豫,西接關中,帶甲十萬,觀釁而動,亦中策也。舉三荊之地,通款梁國,可以庇身,功名去矣,策之下者。」勝輕柔年少,笑而不應。



 及孝武西遷,東魏遣侯景襲穰。勝敗,遂南奔梁,柔亦從之。勝頻表梁武帝,求歸關中。梁武帝覽表,
 嘉其辭彩,既知柔所製,因遣舍人勞問,並遺縑錦。後與勝俱還,行至襄陽。齊神武懼勝西入,遣侯景以輕騎邀之。勝及柔懼,乃棄船山行,贏糧冒險,經數百里。時屬秋霖,徒侶凍餒者,太半至於死。大統二年,至長安,封容城縣男。周文帝引為行臺郎中,除從事中郎。與郎中蘇綽掌機密。時沙苑之役,大軍屢捷,汝、潁之間,多舉義來附。書翰往反,日百餘牒,柔隨機報答,皆合事宜。進爵為子。累遷中書侍郎,兼著作,撰直居注。後為黃門侍郎,周文知其貧,解衣賜之。後遷中書監。周孝閔帝踐阼,拜小內史大夫,進位開府儀同三司,卒於位。所作詩、頌、碑、銘、檄、
 表、啟行於世者數十篇。子愷嗣。



 愷字長仁。性孝友,神情穎悟,涉獵經史,有當世幹能,頗解屬文。周齊王憲引為記室。從憲伐齊,說齊柏社鎮下之。遷小吏部大夫。時染工王神歡者,以賂自進,冢宰宇文護擢為計部下大夫。愷諫曰:「古者,登高能賦,可為大夫;求賢審官,理須詳慎。今神歡出自染工,更無殊異,徒以家富自通,遂與晉紳並列,實恐鵜翼之刺,聞之外境。」護竟寢其事。轉內史下大夫。武帝在雲陽宮,敕諸屯簡老牛,欲以享士。愷諫曰:「昔田子方贖老馬,君子以為美談。向奉明敕,欲以老牛享士,有虧仁政。」帝美其言而止。
 轉禮部大夫,為聘陳使副。先是,行人多從其國禮,及愷為使,一依本朝,陳人莫能屈。建德四年,李穆攻拔軹關、柏崖二鎮,命愷作露布。帝讀大悅曰:「盧愷文章大進,荀景茜故是令君之子。」大象元年,拜東都吏部大夫。



 隋開皇初,加上儀同三司,除尚書吏部侍郎。進爵為侯,仍攝尚書左丞。每有敷奏,侃然正色,雖逢喜怒,不改其常。加散騎常侍。八年,上親考百僚,以愷為上,固讓不敢受。文帝曰:「當仁不讓,何愧之有?皆在朕心,無勞飾讓。」歲餘,拜禮部尚書,攝吏部尚書事。會國子博士何妥與右僕射蘇威不平,奏威陰事,愷坐與相連。憲司奏愷曰:「房恭懿
 者,尉遲迥之黨,不當仕進。威、愷二人,曲相薦達,累轉海州刺史。吏部預選者甚多,愷不即授官,皆注色而遣。威之從父弟徹、肅二人,並以鄉正徵詣吏部。徹文狀後至,而先任用。肅左足攣蹇,才用無算,愷以威故,授朝請郎。愷之朋黨,事甚明白。」上大怒曰:「愷敢將天官以為私惠!」



 愷免冠頓首曰:「皇太子將以通事舍人蘇夔為舍人。夔,威之子,臣以夔未當遷,固啟而止。臣若與威有私,豈當如此?」上曰:「威子,朝廷共知,卿乃固執,以徼身幸;至所不知,便行朋附。姦臣之行也。」於是除名,卒於家。自周氏以降,選無清濁。及愷攝吏部,與薛道衡、陸彥師等甄別士
 流,故涉黨錮之譖,遂及於此。



 崇弟仲義,字小黑,知名於世,位員外散騎侍郎、幽州刺史。崇兄弟官雖不達,婚姻常與玄家齊等。洪弟光宗,位尚書郎。光宗子觀。



 觀字伯舉。少好學,有俊才,舉秀才,射策甲科。除太學博士、著作佐郎。與太常少卿李神俊、光祿大夫王誦等在尚書上省,撰定朝儀。遷尚書儀曹郎中。孝昌元年卒。



 觀弟仲宣,小名金。才學優洽,乃踰於觀,但文體頗細。兄弟俱以文章顯,論者美之。位太尉屬。魏孝莊帝初,遇害河陰。及兄觀並無子,文集莫為撰次,罕有存者。仲宣弟叔彪。



 叔彪少機悟,豪率輕俠,好奇策,慕諸葛亮之為人。為賀拔勝荊州開府長史。



 勝不用其計,棄城奔梁。叔彪歸本縣,築室臨陂,優遊自適。齊文襄降辟書,辭疾不到。天保初,復徵。不得已,布裙露車至鄴。楊愔往候之,以為司徒諮議,辭疾不受。孝昭即位,召為中庶子,問以世事。叔彪勸討關西,畫地陳兵勢,請立重鎮於平陽,與彼蒲州相對,深溝高壘,運糧實之。帝深納之。又願自居平陽,成此謀略。帝命元文遙與叔虎參謀,撰《平西策》一卷。未幾,帝崩,事寢。武成即位,拜儀同三司,判都官尚書,出為金州刺史,遷太子詹事。



 叔彪在鄉時,有粟千石,每至春夏,鄉
 人無食者,令自載取。至秋,任還其價而不計。歲歲常得倍餘。既在朝通貴,自以年老,兒子又多,遂營一大屋,曰:「歌於斯,哭於斯。」魏收常來詣之,訪以洛京舊事。不待食而起,云:「難為子費。」叔彪留之,良久食至,但有粟飡葵菜,木碗盛之,片脯而已。所將僕從,亦盡設食,一與此同。



 齊滅,歸范陽。遭亂城陷,與族弟士邃皆以寒餒斃。周將宇文神舉以二人有名德,收而葬之。



 洪從弟附伯,附伯弟侍伯,並有學識。附伯位滄州平東府長史,侍伯南岐州刺史。侍伯從弟文偉。



 文偉字休族。父敞,位議郎,後以文偉勛,贈幽州刺史。文
 偉少孤,有志尚,頗涉經史。州辟主簿。年三十八,始舉秀才,除本州平北府長流參軍。說刺史裴俊案舊迹修督亢陂,溉田萬餘傾,人賴其利。俊脩立之功,多以委之。文偉既善於營理,兼展私力,家素貧儉,因此致富。及北方將亂,文偉積稻穀於范陽城,時經荒儉,多所振贍,彌為鄉里所歸。及韓樓據薊城,文偉率鄉閭守范陽。樓平,以功封大夏縣男,除范陽太守。



 莊帝崩,文偉與幽州刺史劉靈助同謀起義。靈助克瀛州,留文偉行州事,自率兵赴定州,為爾朱榮將侯深所敗。文偉走還本郡,仍與高乾兄弟相影響。屬神武至信都,文偉遣子懷道奉啟陳
 謝。中興初,除安州刺史,不之官,尋轉幽州刺史。安州刺史盧曹亦從靈助舉兵。靈助敗,因據幽州降爾朱兆。兆仍以為刺史,據城不下,文偉不得入。後除青州刺史。



 文偉輕財愛客,善於撫接;好為小惠。是以所在頗得人情。經紀生資,常若不足,致財積聚,承候寵要,餉遺不絕。卒,贈司徒公、尚書右僕射,謚曰孝威。



 子恭道,性溫良,頗有文學。位范陽郡太守,有德惠。先文偉卒。贈度支尚書,謚曰定。



 子詢祖,襲祖爵大夏男。有術學,文辭華美,為後生之俊。舉秀才,至鄴。趙郡李祖勳嘗宴諸文士。齊文宣使小黃門敕祖勳母曰:「蠕蠕既破,何無賀表?」使者待之。諸
 賓皆為表,詢祖俄頃便成。其詞云:「昔十萬橫行,樊將軍請而受屈;五千深入,李都尉降而不歸。」時重其工。後朝廷大遷除,同日催拜。詢祖立於東止車門外,為二十餘人作表,文不加點,辭理可觀。詢祖初襲爵,有宿德朝士謂曰:「大夏初成」,詢祖應聲曰:「且得燕雀相賀」。



 天保末,為築長城子使。自負其才,內懷鬱怏,遂毀容服如賤役者以見楊愔。



 愔曰:「故舊皆有所縻,唯大夏未加處分。」詢祖厲聲曰:「是誰之咎?」既至役所,作《築長城賦》以寄其意。其略曰:「板則紫栢,杵則木瓜,何期材而斯用也?



 草則離離靡靡,緣岡而殖。但使十步而有一芳,餘亦何辭間於荊
 棘。」邢邵常戲曰:「卿小年才學富盛,戴角者無上齒,恐卿不壽。」對曰:「詢祖初聞此言,實懷惕懼;見丈人蒼蒼在鬢,差以自安。」邵甚重其敏贍。既有口辯,好臧否人物。眾共嫉之,言其淫於從妹。宗人思道謂曰:「大夏何為招四海議?」詢祖曰:「骨肉還相殘,何況執玉帛者萬國。」與思道俱為北州人俊,魏收楊譽思道而以詢祖為不及。



 詢祖謂人曰:「見未能高飛者,借其羽毛;知逸勢沖天者,翦其翅翮。」既諸謗毀日至,素論皆薄其為人。長廣太守邢子廣曰:「詢祖有規檢禰衡,思道無冰稜文舉。」



 後頗折節。歷太子舍人、司徒記室,卒。有文集十卷,皆遺逸。



 恭道弟懷道,
 性輕率好酒,頗有慕尚。既家預義舉,神武親待之。卒於烏蘇鎮城都督。



 懷道弟宗道,性粗率,動作狂俠,位南營州刺史。嘗於晉陽置酒,賓遊滿座;中書舍人馬士達目其彈箜篌女妓,云手甚纖素,宗道即以遺之。士達固辭,宗道便命其家人,將解其腕,士達不得已而受之。將赴營州,於督亢城坡,大集鄉人,殺牛聚會。有一舊門人,醉言疏失,宗道令沈之於水。後坐酷濫除名。



 玄族子輔,字顯光,本州別駕。子同。



 同字叔倫,身長八尺,容貌魁偉,善於處世。太和中,起家北海王詳國常侍。



 熙平初,累遷尚書左丞。時相州刺史
 奚康生徵百姓歲調,皆長七八十尺,以邀憂公之譽,部內患之。同於歲祿,官給長絹。同乃舉案康生度外徵調。書奏,詔科康生罪,兼褒同在公之績。明帝世,朝政稍稀,人多竊冒軍功。同閱吏部勛書,因加檢核,得竊階者三百餘人。乃表言:竊見吏部勛簿,多皆改換,乃校中兵秦案,並復乖舛。愚謂罪雖恩免,猶須刊定。請遣一都令史,與令僕省事各一人,總集吏部、中兵二局勳簿,對句奏案。若名級相應者,即於黃素楷書大字,具件階級數,令本曹尚書以朱印印之。明造兩通,一關吏部,一留兵局,與奏案對掌。進則防揩洗之偽,退則無改易之理。



 從前
 以來,勳書上省,唯列姓名,不載本屬。致令竊濫之徒,輕為茍且。今請徵職白身,具列本州郡縣三長之所;其實官正職者,亦列官名曹別錄曆。皆仰本軍印記其上,然後印縫,各上所司。統將、都督,並皆印記,然後列上行臺。行臺關太尉。太尉檢練精實,乃始關刺。省重究括,然後奏申。奏出之日,黃素朱印,關付吏部。頃來,非但偷階冒名,改換勳簿而已,或一階再取,或易名受級,凡如此者,其人不少。良由吏部無法,防塞失方。何者?吏部加階之後,簿不注記,緣此之故,易生僥倖。自今敘階之後,名簿具注,加補日月,尚書印記,然後付曹,郎中別作抄目,遷
 代相付。此制一行,差止姦罔。



 詔從之。同又奏曰:臣伏思黃素勛簿,政可粗止姦偽,然在軍虛詐,猶未可盡。請自今在軍閱簿之日,行臺、軍司、監軍、都督各明立文案,處處記之。斬首成一階以上,即令給券。



 其券,一紙之上,當中大書,起行臺、統軍位號,勛人甲乙。斬三賊及被傷成階以上,亦具書於券,各盡一行,當行豎裂。其券,前後皆起年號日月,破某處陣,某官某勛,印記為驗。一支付勳人,一支付行臺。記至京,即送門下,別丞守錄。



 又自遷都以來,戎車屢捷,所以征勛轉多,敘不可盡者,良由歲久生姦,積年長偽,巧吏階緣,偷增遂甚。請自今為始,諸有勳
 簿已經奏賞者,即廣下遠近,云某處勳判,咸令知聞。立格酬敘,以三年為斷。其職人及出身,限內悉令銓除;實官及外號,隋才加授。庶使酬勤速申,立效者勸,事不經久,僥倖易息。或遭窮難,州無中正者,不在此限。又勛簿之示,徵還之日,即應申送。然頃來,行臺、督將至京始造,或一年二歲,方上勛書。姦偽之原,實自由此。於今以後,軍還之日,便通勛簿,不聽隔月。



 詔復依行。



 元叉之廢靈太后也,相州刺史、中山王熙起兵於鄴。敗之。叉以同為持節兼黃門侍郎慰勞使,乃就州刑熙。還,授正黃門。同善事在位,為叉所親。戮熙之日,深窮黨與,以希叉旨,論
 者非之。同兄琇,少多大言,常云公侯可致。至此,始為都水使者。同啟求回身二階以加琇。琇遂除安州刺史,論者稱之。營州城人就德興謀反,除同度支尚書,持節使營州慰勞,聽以便宜從事。同乃遣賊家口三十人,并免家奴為良,齎書喻之。德興乃降,安輯其人而還。德興復反,詔同為幽州刺史,兼尚書行臺,慰勞之。同慮德興難信,勒眾而往,為德興所擊,大敗而還。



 靈太后反政,以同叉黨,除名。莊帝踐祚,詔復本秩,除都官尚書,復兼七兵。



 以前慰勞德興功,封章武縣伯,正除七兵。轉殿中。普泰初,除侍中,進號驃騎將軍、左光祿大夫。同時久病,牽強
 啟乞儀同。



 初同之為黃門也,與節閔帝俱在門下,同異其為人,素相款託。帝以恩舊,許之,除儀同三司。永熙初,薨,贈尚書右僕射。四子,長子斐嗣。



 斐字子章,性殘忍,以彊斷知名。齊文襄引為大將軍府刑獄參軍,謂云:「狂簡,斐然成章,非嘉名字也。」天保中,稍遷尚書左丞,別典京畿詔獄。酷濫非人情所為,無問事之大小,拷掠過度,於大棒車輻下死者非一。或嚴冬至寒,置囚於冰雪之上;或盛夏酷熱,暴之日下,枉陷人致死者,前後百數人。伺察官人罪失,動即奏聞。朝士見之,莫不重跡屏氣,皆目之為校事。斐揚揚得志,言必自矜。
 後以謗史事,與李庶俱病鞭杖,死獄中。斐弟筠,青州中從事。



 同兄靜,好學有風度,飲酒至數斗不亂。終於太常丞。大統初,贈太僕卿、平州刺史。靜子景裕。



 景裕字仲孺,小字白頭。少敏,專經為學。居拒馬河,將一老婢作食,妻子不自隨從。又避地大寧山,不營世事。居無二業,唯在注解。其叔父同職居顯要,而景裕止於園舍,情均郊野。謙恭守道,貞素自得,由是世號居士。節閔初,除國子博士,參議正聲,其見親遇,待以不臣之禮。永熙初,以例解。天平中,還鄉里。



 與邢子才、魏季景、魏收、邢昕等同征赴鄴,景裕寓託僧寺,講聽不已。未幾,歸本郡。



 河間邢摩納與景裕從兄仲禮據鄉作逆,逼其同反,以應西魏。齊神武命都督賀拔仁討平之。聞景裕經明行著,驛馬特徵。既而舍之,使教諸子,在館十日一歸家,隨以鼎食。景裕風儀言行,雅見嗟賞。



 先是,景裕注《周易》、《尚書》、《孝經》、《論語》、《禮記》、《老子》,其《毛詩》、《春秋左氏》未訖。齊文襄入相,於第開講;招延時俊,令景裕解所注《易》。景裕理義精微,吐發閑雅。時有問難,或相詆訶,大聲厲色,言至不遜。



 而景裕神彩儼然,風誦如一,從容往復,無際可尋,由是士君子嗟美之。初,元顥入洛,以為中書郎。普泰中,復除國子博士。進退其間,未曾有得失之色。性清靜,淡於
 榮利,弊衣粗食,恬然自安,終日端嚴,如對賓客。興和中,補齊王開府屬,卒於晉陽。神武悼惜之。



 景裕雖不聚徒教授,所注《易》大行於世。又好釋氏,通其大義。天竺胡沙門道悕,每譯諸經論,輒託景裕為之序。景裕之敗也,繫晉陽獄,至心誦經,枷鎖自脫。是時,又有人負罪當死,夢沙門教講經,覺時如所夢,謂誦千遍,臨刑刀折。



 主者以聞,赦之。此經遂行,號曰《高王觀世音》。景裕弟辯。



 辯字景宣,少好學,博通經籍。正光初,舉秀才,為太學博士。以《大戴禮》未有解詁,辯乃注之。其兄景裕為當時碩儒,謂辯曰:「昔侍中注《小戴》,今汝注《大戴》,庶纂前修矣。」節
 閔帝立,除中書舍人。屬齊神武起兵信都,既破爾朱氏,遂鼓行指洛。節閔遣辯持節勞之於鄴。神武令辯見其所奉中興主,辯抗節不從。神武怒曰:「我舉大義,誅群醜,車駕在此,誰遣爾來?」辯抗言酬答,守節不撓。神武異之,舍而不逼。



 孝武即位,以辯為廣平王贊師。永熙二年,平等浮屠成,孝武會萬僧於寺。石佛低舉其頭,終日乃止。帝禮拜之。辯曰:「石立社移,自古有此,陛下何怪。」



 及帝入關,事起倉卒,辯不及至家,單馬而從。或問辯曰:「得辭家不?」辯曰:「門外之道,以義斷恩,復何辭也。」



 孝武至長安,封范陽縣公。歷位給事黃門侍郎,領著作,加本州大中正。
 周文帝以辯有儒術,甚禮之,朝廷大議,常召顧問。遷太子少保,領國子祭酒。趙青雀之亂,魏太子出居渭北,辯時隨從,亦不告家人。其執志敢決,皆此類也。尋除太常卿、太子少傅,轉少師,魏太子及諸王等皆行束修之禮,受業於辯,進爵范陽郡公。



 自孝武西遷,朝儀湮墜,于時朝廷憲章、乘輿法服、金石律呂、晷刻渾儀,皆令辯因時制宜。皆合軌度,多依古禮。性彊記默識,能斷大事,凡所創制,處之不疑。加驃騎大將軍、開府儀同三司,累遷尚書令。及建門官,為師氏中大夫。明帝即位,遷小宗伯,進位大將軍。



 帝嘗與諸公幸其第,儒者榮之。出為宜州刺
 史,以患不之部。卒,謚曰獻,配食文帝廟庭。子慎嗣。位復州刺史。慎弟詮,性趫捷,善騎射,位儀同三司。隋開皇初,以辯前代名德,追封沈國公。



 初,周文欲行《周官》,命蘇綽專掌其事。未幾而綽卒,乃令辯成之。於是依《周禮》建六官,革漢、魏之法。以魏恭帝三年,始命行之。六卿之外,置太師、太傅、太保各一人,是曰三孤。時未建東宮,其太子官員,改創未畢。尋又改典命為大司禮,置中大夫。自茲厥後,世有損益。武成元年,增御正四人,位上大夫。



 保定四年,改宗伯為納言,禮部為司宗,大司禮為禮部,大司樂為樂部。五年,左右武伯各置大夫一人。以建德元年,
 改置宿衛官員。二年,省六府諸司中大夫以下官,府置四司,以下大夫為官之長,士貳之。是歲,又增改東宮官員。三年,初置太子諫議大夫,員四人,文學十人。皇弟、皇子友,員各二人,學士六人。四年,又改置宿衛官員。其司武、司衛之類,皆後所增改。太子正宮尹之屬,亦後所創置。



 而典章散滅,弗可復知。宣帝嗣位,事不師古,官員班品,隨情變革。至如初置四輔官,及六府諸司復置中大夫,並御正、內史增置上大夫等,則今載於外史。餘則朝出夕改,莫能詳錄。



 於時,雖行《周禮》,內外眾職,又兼用秦、漢等官,今略舉其名號及命數,附之於左。其紀傳內更
 有餘官而於此不載者,亦史之闕文也。



 柱國、大將軍,建德四年增置上柱國、上大將軍也。正九命。



 驃騎大將軍、開府儀同三司,建德四年改為開府儀同大將軍,仍增上開府儀同大將軍;車騎大將軍、儀同三司,建德四年改為儀同大將軍,仍增上儀同大將軍;雍州牧。九命。



 驃騎大將軍、右光祿大夫,車騎將軍、左光祿大夫,戶三萬以上州刺史。正八命。



 征東、征南、征西、征北等將軍、右金紫光祿大夫;中軍、鎮軍、撫軍等將軍,左金紫光祿大夫;大都督;戶二萬以上州刺史;京兆尹。八命。



 平東、平西、平南、平北等將軍,右銀青光祿大夫;前、右、左、後等將軍,左銀
 青光祿大夫;帥都督;柱國大將軍府長史、司馬、司錄;戶一萬以上州刺史。



 正七命。



 冠軍將軍、太中大夫;輔國將軍、中散大夫;都督;戶五千以上州刺史;戶一萬五千以上郡守。七命。



 鎮遠將軍、諫議大夫;建忠將軍、誠議大夫;別將;開府長史、司馬、司錄;戶不滿五千以下州刺史;戶一萬以上郡守。正六命。



 中堅將軍、右中郎將;寧朔將軍、左中郎將;儀同府、正八命州長史,司馬,司錄;戶五千以上郡守;大呼藥。六命。



 寧遠將軍、右員外常侍;揚烈將軍、左員外常侍;統軍;驃騎車騎將軍府、八命州長史,司馬,司錄;柱國大將軍府中郎、掾、屬;戶一千以上郡守;長安、
 萬年縣令。正五命。



 伏波將軍、奉車都尉;輕車將軍、奉騎都尉;四征中鎮撫將軍府、正七命州長史,司馬,司錄;開府府中郎、掾、屬;戶不滿一千以下郡守;戶七千以上縣令;正八命州呼藥。五命。



 宣威將軍、武賁給事;明威將軍、冗從給事;儀同府中郎、掾、屬;柱國大將軍府列曹參軍;四平前左右後將軍府、七命州長史,司馬,司錄;正八命州別駕;戶四千以上縣令;八命州呼藥。正四命。



 襄威將軍、給事中;厲威將軍、奉朝請;軍主;開府列曹參軍;冠軍輔國將軍府、正六命州長史,司馬,司錄;正七命州別駕;正八命州中從事;七命郡丞;戶二千以上縣令;正七命
 州呼藥。四命。



 威烈將軍、右員外侍郎;討寇將軍、左員外侍郎;幢主;儀同府、正八命州列曹參軍;柱國大將軍府參軍;鎮遠建忠中堅寧朔將軍府長史,司馬;正六命州別駕;正七命州中從事;正六命郡丞;戶五百以上縣令;七命州呼藥。正三命。



 蕩寇將軍、武騎常侍;蕩難將軍、武騎侍郎;開府參軍,驃騎車騎將軍府、八命州列曹參軍,寧遠揚烈伏波輕車將軍府長史;正六命州中從事,六命郡丞;戶不滿五百以下縣令;戍主;正六命州呼藥。三命。



 殄寇將軍、強弩司馬;殄難將軍、積弩司馬;四征中鎮撫將軍府、正七命州列曹參軍;正五命郡丞。正二命。



 掃
 寇將軍、武騎司馬;掃難將軍、武威司馬;四平前右左後將軍府、七命州列曹參軍;五命郡丞;戍副。二命。



 曠野將軍、殿中司馬;橫野將軍、員外司馬;冠軍輔國將軍府、正六命州列曹參軍。正一命。



 武威將軍、淮海都尉;武邪將軍、山林都尉;鎮遠建忠中堅寧朔寧遠揚烈伏波輕車將軍府列曹參軍。一命。



 周制:封郡縣五等爵者,皆加開國;授柱國大將軍、開府、儀同者,並加使持節、大都督;其開府又加驃騎大將軍、侍中;其儀同又加車騎大將軍、散騎常侍;其授總管、刺史,則加使持節、諸軍事。以此為常。大象元年,詔總管、刺史及行兵者,加持節,餘悉罷之。
 辯所制定之後,又有改革。今粗附之云。辯弟光。



 光子景仁。性溫謹,博覽群書,精於《三禮》,善陰陽,解鐘律,又好玄言。



 孝昌初,釋褐司空府參軍事。及魏孝武西遷,光於山東立義,遙授晉州刺史。大統六年,攜家西入,除丞相府記室參軍,賜爵范陽縣伯。俄拜行臺郎中,專掌書記,改封安息縣伯。歷位京兆郡守、侍中、開府儀同三司、匠師中大夫,進爵燕郡公、虞州刺史,行陜州總管府長史,卒官。周武帝少嘗受業於光,故贈賻有加恆典,贈少傅,謚曰簡。



 光性崇佛道,至誠信敬。常從周文狩於檀臺山,時獵圍既合,帝遙指山上謂群公曰:「公等有所見
 不?」咸曰:無所見。」光獨曰:「見一桑門。」帝曰:「是也。」即解圍而還。令光於桑門立處造浮圖。掘基一丈,得瓦缽錫杖各一,帝稱歎,因立寺焉。及為京兆,而郡舍先是數有妖怪,前後郡將,無敢居者。光曰:「吉凶由人,妖不妄作。」遂人居之。未幾,光所乘馬忽升事,登床,南首而立。食器無故自破。光並不以介懷,其精誠守正如此。注《道德經章句》行於世。子賁。



 賁字子征。略涉書記,頗解鐘律。在周,襲爵燕郡公,歷位魯陽太守、太子少宮尹、儀同三司、司武上士。時隋文帝為大司馬,賁知帝非常人,深自推結。宣帝嗣位,加開府。
 及文帝被顧託,群情未一,引賁置左右。帝將之東第,百官皆不知所去,帝潛令賁部伍仗衛,因召公卿而謂曰:「欲富貴者當相隨來!」往往偶語,欲有去就。賁嚴兵而至,眾莫敢動。出崇陽門至東宮,門者拒不內,賁諭之不去,瞋目叱之,門者遂卻。既而帝得入,賁恆典宿衛,承間進說以應天順人之事,帝從之。及受禪,命賁清宮,因典宿衛。賁乃奏改周旗幟,更為嘉名,其青龍、騶虞、朱雀、玄武、千秋、萬歲之旗,皆賁所創也。尋拜散騎常侍,兼太子左庶子、左領軍將軍。



 及高熲、蘇威共掌朝政,賁甚不平。時柱國劉昉被疏忌,賁諷昉及上柱國元諧、李詢、華州刺
 史張賓等謀黜熲、威,五人相與輔政。又以晉王上之愛子,謀行廢立。



 復私謂皇太子曰:「賁將數謁殿下,恐為上譴,願察區區之心。」謀泄,昉等委罪於賓、賁。公卿奏二人坐當死,帝以龍潛之舊,不忍加誅,並除名。賓未幾卒。歲餘,賁復爵位,檢校太常卿。以古樂宮縣七八,損益不同,歷代通儒,議無定準,乃上表曰:「殷人以上,通用五音。周武克殷,得鶉火天駟之應,其音用七。漢興,加應鐘,故十六枚而在一虡。鄭玄注《周禮》,「二八十六為虡」,此則七八之義,其來遠矣。然世有沿革,用捨不同。至周武帝復改縣七,以林鐘為宮。夫樂者,政之本也,故移風易俗,莫善
 於樂,是以吳札觀而辯興亡。然則樂也者,所以動天地,感鬼神,情發於聲,安危斯應。周武以林鐘為宮,蓋將亡之徵也。且林鐘之管,即黃種下生之義。黃鐘,君也,而生於臣,明於皇朝九五之應。又陰者臣也,而居君位,更顯國家登極之祥。斯實冥數相符,非關人事。臣聞五帝不相沿樂,三王不相襲禮,此蓋隨時改制而不失雅正者也。」帝竟從之,改七縣八,黃鐘為宮。詔賁與儀同楊慶和刊定周、齊音律。



 未幾,歷郢、虢、懷三州刺史。在懷州決沁水東注,名曰利人渠,又派入溫縣,名曰溫潤渠,以溉舄鹹,人賴其利。後為齊州刺史,糶官米而自糶,坐除名。



 後
 從幸洛陽,帝從容謂曰:「我始為大司馬,及總百揆,頻繁左右,與卿足為恩舊。卿若無過,位與高熲齊。坐與凶人交構,由是廢黜。言念疇昔之恩,復處牧伯之位,何乃不思報效,以至於此!吾不忍殺卿,是屈法申私耳。」賁俯伏陳謝。



 詔復本官。後數日,對詔失旨,又自敘功績,有怨言。帝大怒,謂群臣曰:「吾將與賁一州,觀此,不可復用。」



 後皇太子為其言曰:「此輩並有佐命功,雖性行輕險,誠不可棄。」帝曰:「我抑屈之,全其命也。微劉昉、鄭譯及賁、柳裘、皇甫績等,則我不至此。然此等皆反覆子也。當周宣帝時,以無賴得幸。及帝大漸,顏之儀等請以趙王輔政,此輩
 行詐,顧命於我。我將為政,又欲亂之,故昉謀大逆於前,譯為巫蠱於後。如賁之例,皆不滿志,任之則不遜,致之則怨,自難信也,非我棄之。眾人見此,或有竊議,謂我薄於功臣,斯不然矣。」蘇威進曰:「漢光武欲全功臣,皆以列侯奉朝請,至尊仁育,復用此道以安之。」上曰:「然。」遂廢,卒於家。



 勇字季禮,景裕從弟也。父璧,魏下邳太守。勇初與景裕俱在學,其叔同曰:「白頭必以文通,季禮當以武達。興吾門者,二子也。」幽州反者僕骨邢以勇為本郡范陽王,時年十八。後葛榮又以勇為燕王。齊神武起兵,盧文偉召
 之,不應。爾朱氏滅,乃赴晉陽。



 神武署丞相主簿。屬山西霜儉,運山東租輸,皆令實載,違者罪之。令勇典其事。鄉郡公主虛僦千餘車,勇劾之。公主訴於神武,而勇守法不虧。神武謂郭秀曰:「盧勇懍懍,有不可犯色,真公人也。方當委之大事,豈止納租而已。」後行洛州事。



 元象初,官軍圍廣州,未拔,行臺侯景聞西魏救兵將至,集諸將議之。勇請進觀形勢,於是率百騎,各攏一馬。至大騩山,知西魏將李景和將至。勇乃多置幡旗於樹頭,分騎為數十隊,鳴角直前。禽西魏儀同程華,斬儀同王征蠻而還。



 再遷陽州刺史,鎮宜陽。叛人韓木蘭、陳忻等常為邊患,
 勇大破之。啟求入朝,神武賜勇書曰:「吾委卿陽州,安枕高臥,無西南之慮矣。表啟宜停,當使漢兒之中,無在卿前者。」卒,年三十二。勇有馬五百匹,私造甲仗,遺啟盡獻之。贈司空、冀州刺史,謚武貞。



 誕本名恭祖。曾祖晏,博學,善隸書,有名於世;仕慕容氏,位給事黃門侍郎,營丘、成周二郡守。祖壽,太子洗馬;慕容氏滅,入魏為魯郡守。



 父叔仁,年十八,州辟主簿,舉秀才,除員外郎。以親老乃辭歸就養。



 父母既沒哀毀六年,躬營墳壟,遂有終焉之志。景明中,被徵入洛,授武賁中郎將,非其好也。尋除鎮遠將軍、通直散騎常侍,並稱疾
 不朝。乃出為幽州司馬,又辭歸鄉里。當時咸稱其高尚焉。



 誕於度世為族弟。幼而通亮,博學,有祠彩。郡辟功曹,州舉秀才,不得。起家侍御史,累遷輔國大將軍、太中大夫、幽州別駕、北豫州都督府長史。時刺史高仲密以州歸西魏,遣大將軍李遠率軍赴援,誕與文武二千餘人奉候大軍。以功授鎮東將軍、金紫光祿大夫,封固安縣伯。尋加散騎侍郎,拜給事黃門侍郎。



 魏帝詔曰:「經師易求,人師難得。朕諸兒稍長,欲令卿為師。」於是親幸晉王第,敕晉王以下皆拜之於帝前,因賜名曰誕。加征東將軍、散騎常侍。周文帝又以誕儒宗學府,為當世所推,乃
 拜國子祭酒,進車騎大將軍、儀同三司。恭帝二年,除祕書監,後以疾卒。



 論曰:盧玄緒業著聞,首應旌命,子孫繼迹,為世盛門。其文武功列殆無足紀,而見重於時,聲高冠帶,蓋德業儒素有過人者。伯源兄弟亦有二方之風流,雅道家聲,諸子不逮。思道一代俊偉,而宦途寥落,雖曰窮通有命,抑亦不護細行之所致乎!潛及昌衡,雅素之紀,家風克嗣,堂構無虧。子剛使酒誕節,蓋亦明珠之類。



 長仁諫說可重,一簣而傾,惜矣!伯舉、仲宣,文雅俱劭。叔彪志尚宏遠,任俠好謀。文偉望重地華,早有志尚,間關夷險之際,終遇英
 雄之主,雖禮秩未弘,亦為佐命之一也。詢祖辭情艷發,早著聲名,負其才地,肆情矜矯,位遇未聞,弱年夭逝。若得終介眉壽,通塞未可量焉。叔倫質器洪厚,卷舒兼濟。子章殘忍為志,咎之徒也。景裕兄弟,雅業可宗,雖擇木異邦,而立名俱劭。辯捐益成務,其殆優乎。



 勇雖文武異趣,各其美也。賁二三其德,雖取悅於報己,而移之在我,亦安能其罵人。見遣末路,尚何足怪?誕不殞儒業,亦足稱云。



\end{pinyinscope}