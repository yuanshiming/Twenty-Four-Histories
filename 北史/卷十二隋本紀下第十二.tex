\article{卷十二隋本紀下第十二}

\begin{pinyinscope}

 煬皇帝諱廣,一名英,小字阿鷿,高祖第二子也。母曰文獻獨孤皇后。上美姿儀,少敏慧。高祖及后於諸子中,特所鐘愛。在周以高祖勛,封鴈門郡公。開皇元年,立為晉王,拜柱國、并州總管,時年十三。尋授武衛大將軍,進上柱國、河北道行臺尚書令,大將軍如故。高祖令項城公歆、安道公李徹輔導之。上好學,善屬文,沈深嚴重,朝野屬望。高祖密令善相者來和遍視諸子。和曰:「晉王眉
 上雙骨隆起,貴不可言。」既而高祖幸上所居第,見樂器弦多斷絕,又有塵埃,若不用者,以為不好聲妓之玩。上尤自矯飾,當時稱為仁孝。嘗觀獵遇雨,左右進油衣,上曰:「士卒皆霑濕,我獨衣此乎!」乃令持去。六年,轉淮南道行臺尚書令。其年,徵拜雍州牧、內史令。



 八年冬,大舉伐陳,以上為行軍元帥。及陳平,執陳湘州刺史施文慶、散騎常侍沈客卿、市令湯慧朗、刑法監徐析、尚書都令史暨慧,以其邪佞,有害于民,斬之石闕下以謝三吳。於是封府庫資財,無所取,天下稱賢。進位太尉,賜路車、乘馬、袞冕之服,玄珪、白璧各一雙。復拜並州總管。俄而江南
 高智慧等相聚作亂,徙上為揚州總管,鎮江都,每歲一朝。高祖之祠太山也,領武候大將軍。明年,歸籓,後數載,突厥寇邊,復為行軍元帥,出靈武。無虜而旋。及太子勇廢,立上為皇太子。是月,當受冊。高祖曰:「吾以大興公成帝業。」令上出舍大興。其夜,烈風大雪,地震山崩,民舍多壞,壓死者百餘口。仁壽初,奉詔巡撫東南。是後,高祖每避署仁壽宮,恆令上監國。



 四年七月,高祖崩,上即皇帝位於仁壽宮。八月,奉梓宮還京師。并州總管、漢王諒舉兵反,詔尚書左僕射楊素討平之。九月乙巳,以備身將軍崔彭為左領軍大將軍。十一月乙未,幸洛陽。丙申,發
 丁男十數萬掘塹,自龍門東接長平、汲郡,抵臨清關,度河,至浚儀,襄城,達于上洛,以置關防。癸丑,詔曰:乾道變化,陰陽所以消息;沿創不同,生靈所以順序。若使天意不變,施化何以成四時?人事不易,為政何以利萬姓?《易》不云乎,通其變,使民不倦。變則通,通則久。有德則可久,有功則可大。朕又聞之,安安而能遷,民用丕變。是故姬邑兩周,如武王之意;殷人五徙,成湯后之業。若不因民順天,功業見乎變,愛民治國者,可不謂歟。



 然雒邑自古之都,王畿之內,天地之所合,陰陽之所和;控以三河,固以四塞;水陸通,貢賦等。故漢祖曰:「吾行天下多矣,唯見
 雒陽。」自古皇王,何嘗不留意,所不都者,蓋有由焉。或以九州未一,或以困其府庫,作雒之制,所以未暇也。



 我有隋之始,便欲創茲懷、雒,日復一日,越暨于今。念茲在茲,興言感哽。朕肅膺寶歷,纂臨萬邦,遵而不失,心奉先志。今者,漢王諒悖逆,毒被山東;遂令州縣,或淪非所。由關河懸遠,兵不赴急。加以並州移戶,復在河南;周遷殷民,意在於此。況復南服遐遠,東夏殷大,因機順動,今也其時。群司百辟,僉諧厥議。



 但成周歔脊,弗堪胥宇。今可於伊雒營建東京,便即設官分職,以為民極也。夫宮室之制,本以便生人;上棟下宇,足以避風露。高臺廣廈,豈
 曰適形?故《傳》云:儉,德之恭;侈,惡之大。宣尼有云:與其不遜也,寧儉。豈謂瑤臺瓊室,方為宮殿者乎?土階采椽,而非帝王者乎?是知非天下以奉一人,乃一人以主天下也。



 民惟國本,本固邦寧。百姓足,孰與不足。今所營構,務從節儉。無令雕牆峻宇,復起於當今;俗使卑宮菲食,將貽於後世。有司明為條格,稱朕意焉。



 十二月乙丑,以右武衛將軍來護兒為右驍衛大將軍。戊辰,以柱國李景為右武衛大將軍,以右衛率周羅為右武候大將軍。



 大業元年春正月壬辰朔,大赦,改元。立妃蕭氏為皇後。改豫州為溱州,洛州為豫州。廢諸州總管府。丙申,立晉
 王昭為皇太子。丁酉,以上柱國宇文述為左衛大將軍,上柱國郭衍為左武衛大將軍,延壽公于仲文為右衛大將軍。己亥,以豫章王暕為豫州牧。戊申,發八使巡省風俗。下詔曰:昔者哲王之理天下也,其在愛民乎?既富而教,家給人足,故能風教淳厚,遠至邇安。理定功成,率由斯道。朕恭嗣寶位,撫育黎獻,夙夜戰兢,若臨川谷。雖則聿遵先緒,弗敢失墜,永言政術,多有缺然。況以四海之遠,兆民之眾,未獲親臨,問其疾苦。每慮幽仄莫舉,冤屈不申,一物失所,用傷和氣。萬方有罪,責在朕躬,所以興寤增歎,而夕惕載懷者也。今既布政惟始,宜存寬大。
 可分遣使人,巡省方俗,宣揚風化,薦拔淹滯,申達幽枉。孝悌力田,給以優復。鰥寡孤獨不能自存者,量加振濟。義夫節婦,旌表門閭。高年之老,加其板授,並依別條,賜以粟帛。篤疾之徒給侍丁者,雖有侍養之名,曾無賙贍之實,明加檢校,使得存養。



 若有名行顯著,操履修潔;及學業才能,一藝可取,咸宜訪採,將身入朝。所在州縣,以禮發遣。其蠹政害人,不便于時者,使還之日,具錄奏聞。



 己酉,以吳州總管宇文弼為刑部尚書。二月己卯,以尚書左僕射楊素為尚書令。



 三月丁未,詔尚書令楊素、納言楊達、將作大匠宇文愷營建東京,徙豫州郭下居民
 以實之。戊申,詔曰:「聽採輿頌,謀及黎庶,故能審政刑之得失。是知昧旦思治,欲使幽枉必達,彞倫有章。而牧宰任稱朝委,茍為僥倖,以求考課,虛立殿最,不存理實。綱紀於是不理,冤屈所以莫申。關河重阻,無由自達。朕故建立東京,躬親存問。今將巡歷淮海,觀省風俗。眷求讜言,徒繁詞翰,而鄉校之內,闕爾無聞;恇然夕惕,用勞興寢。其民下有知州縣官人政理苛刻,侵害百姓,背公徇私,不便於民者,聽詣朝堂封奏。庶乎四聰以達,天下無冤。」又於皁澗營顯仁宮,採海內奇禽異獸草木之類,以實園苑。徙天下富商大賈數萬家於東京。辛亥,發河南
 諸郡男女七百萬開通濟渠,自西苑引穀、洛水達於河,自板渚引河通於淮。庚申,遣黃門侍郎王弘、上儀同於士澄往江南採木,造龍舟、鳳抃、黃龍、赤艦樓船等數萬艘。



 夏四月癸亥,大將軍劉仲方擊林邑破之。五月庚戌,戶部尚書、義豐侯韋沖卒。甲子,熒惑入太微。秋七月丁酉,制戰亡之家,給復十年。丙午,滕王綸、衛王集並奪爵徙邊。閏七月甲子,以尚書令楊素為太子太師,安德王雄為太子太傅,河間王弘為太子太保。丙子,詔曰:君民建國,教學為先;移風易俗,必自茲始。而言絕義乖,多歷年代,進德修業,其道浸微。漢採坑焚之餘,不絕如線;晉
 承板蕩之運,掃地將盡。自時厥後,軍國多虞;雖復黌宇時建,示同愛禮;函丈或陳,殆為虛器。遂使紆青拖紫,非以學優;製錦操刀,類多牆面。上陵下替,綱維不立,雅缺道消,實由於此。朕纂承洪緒,思弘大訓。將欲尊師重道,用闡厥繇;講信修睦,敦獎名教。方今區宇平壹,文軌攸同,十步之內,必有芳草;四海之中,豈無孝、秀。諸在家及見入學者,若有篤志好古,耽典悅禮,學行優敏,堪膺時務,所在採訪,具以名聞。即當隨其器能,擢以不次。若研精經術,未顧進仕,可依其藝業深淺,門蔭高卑,雖未升朝,並量準給祿。庶夫恂恂善誘,不日成器,濟濟盈朝,何
 遠之有。其國子等學,亦宜申明舊制,教習生徒,具為課試之法,以盡砥礪之道。



 八月壬寅,上御龍舟幸江都,以左武衛大將軍郭衍為前軍,右武衛大將軍李景為後軍。文武官五品以上給樓船,九品以上給黃篾。舳艫相接,二百餘里。冬十月己丑,赦江、淮已南,揚州給復五年;舊總管內,給復三年。十一月己未,以大將軍崔仲方為禮部尚書。



 二年春正月辛酉,東京成,賜監督者有差。以大理卿梁毗為刑部尚書。丁卯,遣十使,併省州縣。二月丙戌,詔尚書令楊素、吏部尚書牛弘、大將軍宇文愷、內史侍郎虞
 世基、禮部侍郎許善心制定輿服。始備輦輅及五時副車。上常服皮弁,十有二琪。文官弁服,珮玉;五品已上,給犢車通憲;三公、親王加油絡。武官平巾幘,褲褶;三品已上,給瓟槊。下至胥吏,服色各有差。非庶人不得戎服。戊戌,置都尉官。三月庚午,車駕發江都。先是,太府少卿何稠、太府丞雲定興盛修儀仗,於是課州縣送羽毛。百姓求捕之,網羅被水陸,禽獸有堪氈毦之用者,殆無遺類。



 至是而成。夏四月庚戌,上自伊闕,陳法駕,備千乘萬騎,入於東京。辛亥,上御端門,大赦天下,免今年租賦。癸丑,以冀州刺史楊文思為民部尚書。五月甲寅,金紫光祿
 大夫、兵部尚書李通坐事免。乙卯,詔曰:「旌表先哲,式在饗祀。所以優禮賢能,顯彰遺愛。朕永鑒前修,尚想名德,何嘗不興歎九原,屬懷千載。其自古以來賢人君子,有能樹聲立德,佐世匡時,博利殊功,有益於人者,並宜營立祠宇,以時致祭。墳壟之處,不得侵踐。有司量為條式,稱朕意焉。」六月壬子,以尚書令、太子太師楊素為司徒。進封豫章王暕為齊王。秋七月癸丑,以衛尉卿衛玄為工部尚書。庚申,制百官不得計考增級。必有德行功能,灼然顯著者,擢之。壬戌,擢籓邸舊臣鮮于羅等二十七人,官爵有差。甲戌,皇太子昭薨。乙亥,上柱國、司徒、楚國
 公楊素薨。八月辛卯,封皇孫倓為燕王,侗為越王,侑為代王。九月乙丑,立秦王俊子浩為秦王。冬十月戊子,以靈州刺史段文振為兵部尚書。十二月庚寅,詔曰:「前代帝王,因時創業,君民建國,禮尊南面。面歷運推移,年代永久,丘壟殘毀,樵牧相趨;塋兆堙蕪,封樹莫辨。興言淪滅,有愴于懷。自古以來帝王陵墓,可給隨近十戶,蠲其雜役,以供守視。」



 三年春正月癸亥,敕並州逆人已流配而逃亡者,所獲之處,即宜斬決。丙子,長星竟天,出於東壁,二旬而止。是月,武陽郡上言河水清。二月己丑,慧星見於東井、文昌;
 歷大陵、五車、北河,入太微,掃帝座,前後百餘日而止。三月辛亥,車駕還京師。壬子,以大將軍姚辯為左衛將軍。癸丑,遣羽騎朱寬使於流求國。乙卯,河間王弘薨。夏四月庚辰,詔曰:「古者帝王觀風俗,皆所以憂勤兆庶,安集遐荒。自蕃夷內附,未遑親撫,山東經亂,須加存恤。今欲安輯河北,巡省趙、魏,所司依式。」甲申,頒律令,大赦天下,關內給復三年。壬辰,改州為郡。改度量衡,並依古式。改上柱國以下官為大夫。甲午,詔曰:天下之重,非獨理所安;帝王之功,豈一士之略。自古明君哲后,立政經邦,何嘗不選賢與能,振拔淹滯。周稱多士,漢號得人,尚想前
 風,載懷欽佇。朕負扆夙興,冕旒待旦。引領巖谷,置以周行;冀與群才,共康庶績。而匯茅寂漠,投竿罕至。豈美璞韜采,未值良工;將介石在懷,確乎難拔?永鑒則哲,憮然興歎。凡厥在位,譬諸股肱,若濟巨川,義同舟楫。豈得保茲寵祿,晦爾所知,優游卒歲,甚非謂也。祁大夫之舉善,良史以為至公;臧文仲之蔽賢,尼父譏其竊位。求諸往古,非無褒貶。宜思進善,用匡寡薄。夫孝悌有聞,人倫之本;德行敦厚,立身之基。或節義可稱,或操履清潔,所以激貪厲俗,有益風化。強毅正直,執憲不撓,學業優敏,文才美秀;並為廊廟之用,實乃瑚璉之資。才堪將略,則拔
 之以御侮;力有驍壯,則任之以爪牙。爰及一藝可取,亦宜採錄;若眾善畢舉,與時無棄。以此求理,庶幾非遠。文武有職事者,五品已上,宜依令十科舉人。有一於此,不必求備。朕當待以不次,隨才升用。其見任九品已上官者,不在舉送之限。



 丙申,車駕北巡狩。丁酉,以刑部尚書宇文弼為禮部尚書。戊戌,敕百司不得踐暴禾稼。其有須開為路者,有司計地所收,即以近倉酬賜,務從優厚。己亥,至赤岸澤,以太牢祭故太師李穆。五月丁巳,突厥啟民可汗遣子拓特勒來朝。戊午,發河北十餘郡丁男,自太行山達于並州,以通馳道。丙寅,啟民可汗遣其兄
 子毗黎伽特勒來朝。辛未,啟民可汗使請自入寒奉迎輿駕,上不許。癸酉,有星孛于文昌,上將星常皆動搖。六月辛巳,獵於連谷。丁亥,詔曰:聿追孝饗,德莫至焉;崇建寢廟,禮之大者。然則質文異代,損益殊時。學滅坑焚,經典散逸;憲章湮墜,廟堂制度,師說不同。所以世數多少,莫能是正,連室異宮,亦無定准。朕獲奉祖宗,欽承景業,永惟嚴配,冀隆大典。於是詢謀在位,博訪儒術。咸以為高祖文皇帝受天明命,奄有區夏。拯群飛於四海,革彫弊於百王。



 恤獄緩刑,生靈皆遂其性;輕徭薄賦,比屋各安其業。芟夷宇宙,混壹車書。東漸西被,無思不服;南征
 北怨,俱荷來蘇。駕毳乘風,歷代所弗至;辮髮左衽,聲教所罕及。莫不厥角關塞,頓顙闕庭;譯靡絕時,書無虛月。韜戈偃伯,天下晏如;嘉瑞休徵,表裏禔福。猗歟偉歟,無得而名者也。朕又聞之,德厚者流光,理辨者禮縟。是以周之文、武,漢之高、光,其典章特立,謚號斯重。豈非緣情稱述,即崇顯之義乎。高祖文皇帝宜別建廟宇,以彰巍巍之德;仍遵月祭,用表蒸蒸之懷。



 有司以時創造,務合典制。又名位既殊,禮亦異等。天子七廟,事著前經;諸侯二昭,義有差降。故知以多為貴,王者之禮,今可依用,貽厥後昆。



 戊子,次榆林郡。丁酉,啟民可汗來朝。己亥,吐谷
 渾、高昌並遣使貢方物。



 甲辰,上御北樓,觀漁于河,以宴百僚。秋七月辛亥,啟民可汗上表請變服,襲冠帶。詔啟民贊拜不名,在諸侯王上。甲寅,上於郡城東御大帳,其下備儀衛,建旌旗,宴啟民及其部落三千五百人。奏百戲之樂,賜啟民及其部落各有差。丙子,殺光祿大夫賀苦弼、禮部尚書宇文弼、太常卿高熲。尚書左僕射蘇威坐事免。發丁男百餘萬築長城,西鉅榆林,東至紫河,二旬而罷,死者十五六。八月壬午,車駕發榆林。乙酉,啟民飾盧清道以候乘輿,帝幸其帳。啟民奉觴上壽,宴賜極厚。上謂高麗使者曰:「歸語爾王,當早來朝見。不然者,吾
 與啟民巡彼土矣。」皇后亦幸義城公主帳。己丑,啟民可汗歸蕃。癸巳,入樓煩關。壬寅,次太原,詔營晉陽宮。



 九月己未,次濟源,幸御史大夫張衡宅,宴享極歡。己巳,至於東都。壬申,以齊王暕為河南尹、開府儀同三司。癸酉,以戶部尚書楊文思為納言。



 四年春正月乙巳,詔發河北諸郡男女百餘萬開永濟渠,引沁水南達于河,北通涿郡。庚戌,百僚大射於允武殿。丁卯,賜城內居民米各十石。壬申,以太府卿元壽為內史令,鴻臚卿楊玄感為禮部尚書。癸酉,以工部尚書衛玄為右武候大將軍,大理卿長孫熾為戶部尚書。二
 月己卯,遣司朝謁者崔毅使突厥處羅,致汗血馬。三月辛酉,以將作大匠宇文愷為工部尚書。壬戌,百濟、倭、赤土、迦羅含國並遣使貢方物。乙丑,車駕幸五原,因出塞,巡長城。丙寅,遣屯田主事常駿使赤土,致羅罽。夏四月丙午,以離石之汾源、臨泉,鴈門之秀容為樓煩郡。起汾陽宮。癸丑,以河內太守張定和為左屯衛大將軍。乙卯,詔曰:「突厥意利珍豆啟民可汗率領部落,保附關塞,遵奉朝禮,思改戎俗。頻入謁覲,屢有陳請。以氈牆毳幕,事窮荒陋;上棟下宇,願同比屋。誠心懇切,朕之所重。宜於萬壽戍置城造屋,其帷帳床褥以上,隨事量給,務從優
 厚,稱朕意焉。」五月壬申,蜀郡獲三足烏,張掖獲玄狐,各一。秋七月辛巳,發丁男二十餘萬築長城,自榆林谷而東。乙未,左翊衛大將軍宇文述破吐谷渾於曼頭、赤水。八月辛酉,親祠恆岳,河北道郡守畢集。大赦天下,車駕所經郡縣,免一年租調。九月辛未,徵天下鷹師,悉集東京,至者萬餘人。戊寅,慧星出五車,掃文昌,至房而滅。辛巳,詔免長城役者一年租賦。冬十月丙午,詔曰:「先師尼父,聖德在躬,誕發天縱之姿,憲章文武之道;命世膺期,蘊茲素王。而頹山之歎,忽踰於千祀;盛德之美,不在於百代。永惟懿範,宜有優崇。可立孔子後為紹聖侯,有司
 求其苗裔,錄以申上。」辛亥,詔曰:「昔周王下車,首封唐虞之胤;漢帝承歷,亦命殷周之後。皆所以褒立先代,憲章在昔。朕嗣膺景業,傍求雅訓,有一弘益,欽若令典。以為周兼夏殷,文質大備;漢有天下,車書混一;魏晉沿襲,風流未遠。並宜立後,以存繼絕之義。有司可求其胄緒,列聞。」乙卯,頒新式於天下。



 五年春正月丙子,改東京為東都。癸未,詔天下均田」戊子,上自東都還京師。



 己丑,制民間鐵叉搭鉤刃之類,皆禁絕之。太守每歲密上屬官景跡。二月戊戌,次于閿鄉。詔祭古帝王陵及開皇功臣墓。庚子制,漢魏、周官不
 得為蔭。辛丑,赤土國遣使貢方物。戊申,車駕至京師。丙辰,宴耆舊四百人於武德殿,頒賜各有差。



 己未,上御崇德殿之西院,愀然不悅,顧謂左右曰:「此先帝所居,實用增感,情所未安。於此院之西,別營一殿。」壬戌,制父母聽隨子之官。三月己巳,車駕西巡河右。庚午,有司言武功男子史永遵與從父昆弟同居,上嘉之,賜物一百段,米二百石,表其門閭。乙亥,幸扶風舊宅。夏四月己亥,大獵於隴西。壬寅,高麗、吐谷渾、伊吾並遣使來朝。乙巳,次狄道。黨項羌來貢方物。癸亥,出臨津關,度黃河,至西平,陳兵講武。五月乙亥,上大獵於延山。長圍周亙二千里。庚
 辰,入長寧谷。壬午,度星嶺。甲申,宴郡臣於金山之上。丙戌,梁浩亹,御馬度而橋壞,斬朝散大夫黃亙及督役者九人。吐谷渾主率眾保覆袁川。帝分命內史元壽南屯金山,兵部尚書段文振北屯雪山,太僕卿楊義臣東屯琵琶峽,將軍張壽西屯泥嶺,四面圍之。吐谷渾主伏允以數十騎遁出,遣其名王詐稱伏允,保車我真山。壬辰,詔右屯衛大將軍張定和往捕之。定和挺身挑戰,為賊所殺。亞將柳武建擊破之,斬首數百級。甲午,其仙頭王窮蹙,率男女十餘萬口來降。六月丁酉,遣左光祿大夫梁默、右翊衛將軍李瓊等追吐谷渾主,皆遇賊,死之。癸
 卯,經大斗拔谷。山路隘險,魚貫而出,風霰晦暝,與後宮相失。士卒凍死者大半。丙午,次張掖。辛亥,詔諸郡學業該通,才藝優洽;膂力驍壯,超絕等倫;在官勤奮,堪理政事;立性正直,不避強禦:四科舉人。壬子,高昌王曲伯雅來朝。伊吾吐屯設等獻西域數千里之地,上大悅。癸丑,置西海、河源、鄯善、且末等四郡。丙辰,上御觀風行殿,盛陳文物;奏九部樂,設魚龍曼延,宴高昌王、吐屯設於殿上,以寵異之。其蠻夷陪列者,三十餘國。戊午,大赦天下。開皇已來流配,悉放還鄉。晉陽逆黨,不在此例。隴右諸郡,給復三年。秋七月丁卯,置馬牧於青海渚中,以求龍
 種,無效而止。九月癸未,車駕入長安。冬十月癸亥,詔曰:「優德尚齒,載之典訓;尊事乞言,義彰膠序。鬻熊為師,無取筋力;方叔元老,克壯其猷。朕永言稽古,用求至理。是以龐眉黃髮,更令收敘;務簡秩優,無虧藥餌,庶等臥理,佇其弘益。今歲耆老赴集者,可於近郡處置。年七十已上,疾患沈滯不堪居職,即給賜帛,送還本郡。其官至七品以上者,量給廩以終厥身。」十一月丙子,車駕幸東都。



 六年春正月癸亥朔,旦,有盜數十人,皆素冠練衣,焚香持華,自稱彌勒佛。



 入自建國門,監門者皆稽首。既而奪衛士仗,將為亂。齊王暕遇而斬之。於是都下大索,與相
 連坐者千餘家。丁丑,角抵大戲於端門街,天下奇伎異藝畢集,終月而罷。帝數微服往觀之。己丑,倭國遣使貢方物。二月乙巳,武賁郎將陳棱、朝請大夫張鎮州擊流水破之。獻俘萬七千口,頒賜百官。乙卯,詔曰:「夫帝圖草創,王業艱難,咸依股肱,葉同心德;用能救厥頹運,克膺大寶。然後疇庸茂賞,開國承家;誓以山河,傳之不朽。近代凋喪,四海未壹。茅土妄假,名實相乖;歷茲永久,莫能懲革。皇運之初,百度伊始,猶循舊貫,未暇改作。今天下交泰,文軌攸同。



 宜率遵先典,永垂大訓。自今已後,唯有功勳,乃得賜封,仍令子孫承襲。」丙辰,改封安德王雄為
 觀王,河間王子慶為郇王。庚申,征魏、齊、周、陳樂人,悉配太常。三月癸亥,幸江都宮。甲子,以鴻臚卿史祥為左驍衛大將軍。夏四月丁未,宴江、淮已南父老,頒賜各有差。六月辛卯,室韋、赤土並遣使貢方物。壬辰,鴈門賊帥尉文通,聚眾三千,保於莫壁谷,遣鷹揚楊伯泉擊破之。甲寅,制江都太守,秩同京尹。冬十月壬申,刑部尚書梁毗卒。壬子,戶部尚書、銀青光祿大夫長孫熾卒。十二月己未,左光祿大夫、吏部尚書牛弘卒。辛酉,朱崖人王萬昌舉兵作亂,遣隴西太守韓洪討平之。



 七年春正月壬寅,左武衛大將軍、光祿大夫、真定侯郭
 衍卒。二月己未,上升釣臺,臨楊子津,大宴百僚,頒賜各有差。庚申,百濟遣使朝貢。乙亥,上自江都御龍舟入通濟渠,遂幸于涿郡。壬午,詔曰:「武有七德,先之以安民;政有六本,興之以教義。高麗虧失籓禮,將欲問罪遼左,恢宣勝略。雖懷伐國,仍事省力。今往涿郡,巡撫民俗。其河北諸郡及山西、山東年九十已上,版授太守;八十者,授縣令。三月丁亥,右光祿大夫、左屯衛大將軍姚辯卒。夏四月庚午,幸涿郡之臨朔宮。五月戊子,以武威太守樊子蓋為民部尚書。秋,大水,山東、河南漂沒三十餘郡,民相賣為奴婢。冬十月乙卯,底柱山崩,偃水逆流數十里。
 戊午,以東平太守吐萬緒為左屯衛大將軍。十二月己酉,突厥處羅多利可汗來朝,帝大悅,接以殊禮。



 于時,遼東戰士及饋運者填咽於道,晝夜不絕。苦役者,始為群盜。甲子,敕都尉、鷹揚與郡縣相知追捕,隨獲斬決之。



 八年春正月辛巳,大軍集于涿郡。以兵部尚書段文振為左候衛大將軍。壬午,下詔曰:天地大德,降繁霜於秋令;聖哲至仁,著兵甲於刑典。故知造化之有肅殺,義在無私;帝王之用干戈,蓋非獲已。版泉、丹浦,莫匪龔行;取亂覆昏,咸由順動。



 況乎甘野誓師,夏開承大禹之業;商郊問罪,周發成文王之志。永監載籍,屬當朕躬。粵我有
 隋,誕膺靈命。兼三才而建極,一六合而為家。提封所漸,細柳、蟠桃之外;聲教爰暨,紫舌、黃枝之域。遠至邇安,罔弗和會;功成理定,於是乎在。



 而高麗小醜,迷昏不恭。崇聚勃、碣之間,薦食遼、濊之境。雖復漢、魏誅夷,巢窟暫擾;亂離多阻,種落還集。萃川藪於前代,播實繁以迄今。眷彼華壤,翦為夷類。歷年永久,惡稔既盈;天道禍淫,亡徵已兆,亂常敗德,非可勝圖;掩慝懷姦,唯日不足。移告之嚴,未嘗面受;朝覲之禮,莫肯躬親。誘納亡叛,不知紀極;充斥邊垂,亟勞烽候。關析以之不靜,生人為之廢業。在昔薄伐,已漏天網。既緩前禽之戮,未即後服之誅。曾不
 懷恩,翻其長惡。乃兼契丹之黨。虔劉海戍;習鞨靺之服,侵軼遼西。又青丘之表,咸修職貢;碧海之濱,同稟正朔。遂復寇攘政琛贐,遏絕往來;虐及弗辜,誠而遇禍。軺軒奉使,爰暨海東;旌節所次,途經籓境;而擁塞道路,拒絕王人。無事君之心,豈為臣之禮?此而可忍,孰不可容!且法令苛酷,賦斂煩重。彊臣豪族,咸執國均;朋黨比周,以之成俗。賄貨如市,冤枉莫申。



 重以仍歲災凶,比屋饑饉;兵戈不息,徭役無期。力竭轉輸,身填溝壑。百姓愁苦,爰誰適從。境內哀惶,不勝其弊。回面內向,各懷性命之圖;黃髮稚齒,咸興酷毒之歎。省俗觀風,爰屈幽朔;弔人問罪,
 無俟再駕。親總六師,用申九伐。拯厥阽危,協從天意;殄茲逋穢,克嗣先謨。今宜授律啟行,分麾屆路;掩勃澥而雷震,及夫餘以電掃。比戈按甲,俟誓而後行;三令五申,必勝而後戰。左第一軍可鏤方道,第二軍可長岑道,第三軍可海冥道,第四軍可蓋馬道,第五軍可建安道,第六軍可南蘇道,第七軍可遼東道,第八軍可玄菟道,第九軍可扶餘道,第十軍可朝鮮道,第十一軍可沃沮道,第十二軍可樂浪道;右第一軍可粘蟬道,第二軍可含資道,第三軍可渾彌道,第四軍可臨屯道,第五軍可候城道,第六軍可提奚道,第七軍可踏頓道,第八軍可肅
 慎道,第九軍可碣石道,第十軍可東施道,第十一軍可帶方道,第十二軍可襄平道。凡此眾軍,先奉廟略。絡繹引途,總集平壤。莫非如豺如貔之勇,百戰百勝之雄。顧眄則山岳傾頹,叱吒則風雲騰鬱。腹心攸同,爪牙斯在。



 朕躬馭元戎,為其節度。涉遼而東,循海之右。解倒懸於遐裔,問疾苦於遺黎。其外輕齎游闕,隨機赴響;卷甲銜枚,出其不意。又滄海道軍,舟艫千里;高帆電逝,巨艦雲飛。橫斷沮江,逕造平壤。島嶼之望斯絕,坎井之路已窮。其餘被髮左衽之人,控弦待發;微、盧、彭、濮之旅,不謀同辭。杖順臨逆,人百其勇,以此眾戰,勢等摧枯。然則王者
 之師,義存止殺;聖人之教,必也勝殘。天罰有罪,本在元惡;人之多辟,脅從罔理。若高元泥首轅門,自歸司寇,即解縛焚櫬,弘之以恩。其餘巨人,願歸朝奉化,咸加慰撫,各安生業,隨才任用,無隔夷夏。營壘所次,務在整肅;萏堯有禁,秋毫勿犯。以布恩宥,以喻禍福。若其同惡相濟,抗拒官軍,國有常刑,俾無遺類。明加曉示,稱朕意焉!



 總一百一十三萬三千八百,號二百萬,其饋運者倍之。癸未,第一軍發,終四十日,引師乃盡。旌旗亙千里,近古出師之盛,未之有也。乙未,以右候衛大將軍衛玄為刑部尚書。甲辰,內史令元壽卒。二月甲寅,詔曰:「朕觀風燕裔,
 問罪遼濱,文武葉力,爪牙思奮;莫不執銳勤王,舍家從役。罕蓄倉廩之資,兼捐播殖之務。朕所以夕惕愀然,慮其匱乏。雖復素飽之眾,情在忘私;悅使之徒,宜從其厚。



 諸行從一品以下佽飛募人以上家口,郡縣宜數存問。若有糧食乏少,皆賑給之。或雖有田疇,貧弱不能自耕重,可於多丁富室,勤課相助。使夫居者有斂積之豐,行役無顧後之慮。」壬戎,司空、京兆尹、光祿大夫、觀王雄薨。三月辛卯,兵部尚書、左候衛大將軍段文振卒。癸巳,上御師。甲午,臨戎于遼水橋。戊戌,大軍為賊所拒,不果濟。右屯衛大將軍左光祿大夫麥鐵杖、武賁郎將錢士雄、孟
 金叉等皆死之。甲午,車駕度遼,大戰于東岸,擊賊破之,進圍遼東。乙未,大頓。見二大鳥,高丈餘,皓身朱足,游泳自若。上異之,命工圖寫,並立銘頌。五月戊午,納言楊達卒,于時,諸將各奉旨,不敢越機。既而高麗各固城守,攻之不下。六月己未,幸遼東,責怒諸將,止城西數里,御六合城。七月壬午,宇文述等敗績于薩水,右屯衛將軍薛世雄死之。九軍並陷,師奔還,亡者千餘騎。癸卯,班師。九月庚辰,上至東都,己丑,詔:「軍國異容,文武殊用,匡危拯難,則霸德攸興;化人成俗,則王道斯貴。時當撥亂,屠販可以登朝;世屬隆平,經術然後升仕。豐都爰肇,儒服
 無預於周行;建武之朝,功臣不參於吏職。自三方未一,四海交爭;不遑文教,唯尚武功。設官分職,罕以才授;班朝理人,乃由勳敘。莫非拔足行陣,出自勇夫。



 學敩之道,既所不習;政事之方,故亦無取。是非暗於在己,威福專於下吏。貪冒貨賄,不知紀極;蠹政害民,實由於此。自今已後,諸授勛官者,並不得回授文武職事。庶遵彼更張,取類於調瑟;求諸名製,不傷於美錦。若吏部輒擬用者,御史即宜糾彈。」冬十月戊寅,工部尚書宇文愷卒。十一月己卯,以宗女華容公主嫁于高昌王。辛巳,光祿大夫韓壽卒。甲申,敗將宇文述、於仲文等除名為民,斬尚書
 右丞劉士龍以謝天下。是歲,大旱疫,人多死,山東尤甚。密詔江、淮南諸郡,閱視民間童女姿質端麗者,每歲貢之。



 九年春正月丁丑,徵天下兵,募民為驍果,集于涿郡。壬午,賊帥杜彥永、王潤等陷平原郡,大掠而去。辛卯,置折衝、果毅、武勇、雄武等郎將官,以領驍果。



 乙未,平原李德逸聚眾數萬,稱阿舅賊,劫掠山東。靈武白榆妄稱奴賊,劫掠牧馬;北連突厥,隴右多被其患。遣將軍范貴討之,連年不能克。戊戌,大赦。己亥,遣代王侑、刑部尚書衛玄鎮京師。辛丑,以右驍衛將軍李渾為右驍衛大將軍。二
 月己未,濟北人韓進洛聚眾數萬為群盜。壬午,復宇文述等官爵,又徵兵討高麗。三月丙子,濟北人孟海公起兵為盜,眾至數萬。丁丑,發丁男十萬城大興。戊寅,幸遼東。以越王侗、工部尚書樊子蓋鎮東都。庚子,北海人郭方預聚徒為賊,自號盧公,眾至三萬,攻陷郡城,大掠而去。夏四月庚午,車駕度遼。壬申,遣宇文述、楊義臣趣平壤城。五月丁丑,熒惑入南斗。己卯,濟北人甄寶車聚眾萬餘,寇掠城邑。



 六月乙巳,禮部尚書楊玄感反於黎陽。丙辰,玄感逼東都。河南贊理裴弘策拒之,反為賊所敗。戊辰,兵部侍郎斛斯政奔於高麗。庚午,上班師。高麗犯
 後軍,敕右武衛大將軍李景為後拒,遣左翊衛大將軍宇文述、左候衛將軍屈突通等馳傳發兵,以討玄感。秋七月己卯,令所在發人城縣府驛。癸未,餘杭人劉元進舉兵反,眾至數萬。八月壬寅,左翊衛大將軍宇文述等破楊玄感於閿鄉,斬之。餘黨悉平。癸卯,吳人朱燮、晉陵人管崇擁眾十萬餘,自稱將軍,寇江左。甲辰,制驍果之家,蠲免賦役。丁未,詔郡縣城去道過五里已上者,徙就之。戊申,制盜賊籍沒其家。乙卯,賊帥陳瑱等三萬,攻陷信安郡。辛酉,司農卿、光祿大夫、葛國公趙元淑以罪伏誅。



 九月己卯,濟陰人吳海流、東海人彭孝才並舉兵為
 盜,眾數萬。庚辰,賊帥梁慧尚聚眾四萬,陷蒼梧郡。甲午,車駕次上谷。以供費不給,上大怒,免太守虞荷等官。



 丁酉,東陽人李三兒、向但子舉兵作亂,眾至萬餘。閏月己巳,幸博陵。庚午,上謂侍臣曰:「朕昔從先朝,周旋於此,年甫八歲。日月不居,倏經三紀,追惟曩昔,不可復希。」言未卒,流涕嗚咽。侍衛者皆泣下沾襟。冬十月丁丑,賊帥呂明星率眾數千圍東郡,武賁郎將費青奴擊斬之。乙酉,詔曰:「博陵昔為定州,地居衝要;先王歷試所基,王化斯遠。故以道冠《豳風》,義高姚邑。朕巡撫氓庶,爰屆茲邦,瞻望郊廛,緬懷敬止。思所以宣播慶澤,覃被下人;崇紀顯
 號,式光令緒。可改博陵為高陽郡,赦境內死罪以下,給復一年。」於是召高祖時故吏,皆量才授職。壬辰,以納言蘇威為開府儀同三司。朱燮、管崇推劉元進為天子,遣將軍吐萬緒、魚俱羅討之,連年不能剋。齊人孟讓、王薄等眾十餘萬,據長白山,攻剽諸郡。清河賊張金稱眾各數萬,勃海賊帥格謙,自號燕王,孫宣雅自號齊王,眾各十萬,山東苦之。丁亥,以右候衛將軍郭榮為右候衛大將軍。十一月己酉,右候衛將軍馮孝慈討張金稱於清河,反為所敗,孝慈死之。十二月甲辰,車裂楊玄感弟朝散大夫積善及黨與十餘人,仍焚而揚之。丁亥,扶風人
 向海明舉兵作亂,稱皇帝,建元白烏。



 遣太僕卿楊義臣擊破之。



 十年春正月甲寅,以宗女為信義公主,嫁於突厥曷娑那可汗。二月辛未,詔百寮議伐高麗,數日無敢言者。戊子,詔曰:「竭力王役,致身戎事,咸由徇義,莫匪勤誠。委命草芥,暴骸原野;興言念之,每懷愍惻。往年問罪,將屆遼濱;廟算勝略,具有進止。而諒昏兇,罔識成敗;高熲愎狠,本無智謀。臨三軍猶兒戲,視人命如草芥;不遵成規,坐貽撓退。遂令死亡者眾,不及埋藏。今宜遣使人,分道收葬。設祭於遼西郡,立道場一所。恩加泉壤,庶弭窮魂之
 冤;澤及枯骨,用弘仁者之惠。」辛卯,詔曰:黃帝五十二戰,成湯二十七征,方乃德施諸侯,令行天下;盧芳小盜,漢祖尚且親戎;隗囂餘燼,光武猶自登隴。豈不欲除暴止戈,勞而後逸者哉。朕纂承寶業,君臨天下;日月所照,風雨所霑;孰非我臣,獨隔聲教。蕞爾高麗,僻居荒裔;鴟張狠噬,侮慢不恭;抄竊我邊垂,侵逼我城鎮。是以去歲出軍,問罪遼、碣;殪長蛇於玄菟,戮封豕於襄平。扶餘眾軍,風馳電逝;追奔逐北,徑踰浿水。滄海舟楫,衝賊腹心;焚其城郭,汙其宮室。高元伏金質泥首,送款軍門。尋請入朝,歸罪司寇。朕以許其改過,乃詔班師。而長惡靡悛,宴安
 鴆毒。此而可忍,孰不可容。便可分命六師,百道俱進。朕當親執武節,臨御諸軍;秣馬九都,觀兵遼水;順天誅於海外,拯窮民於倒懸。征伐以正之;明德以誅之;止除元惡,餘無所問。若有識存亡之分,悟安危之機,翻然北首,自求多福。必其同惡相濟,抗拒王師,若火燎原,刑茲無赦。有司便宜宣布,咸使知聞。



 丁酉,扶風人唐弼舉兵反,眾十萬,推李弘為天子,自稱唐王。三月壬子,行幸涿郡。癸亥,次臨渝宮。親御戎服,祃祭黃帝,斬叛軍者以釁鼓。夏四月辛未,彭城賊張大彪聚眾數萬,保縣薄山為盜,遣榆林太守董純擊破斬之。甲午,車駕次北平。五月庚
 子,詔舉郡孝悌廉潔各十人。壬寅,賊帥宋世謨陷瑯邪。庚申,延安人劉迦論舉兵反,自稱皇王,建元大世。六月辛未,賊帥鄭文雅、林寶護等眾三萬,陷建安郡,太守楊景祥死之。秋七月癸丑,車駕次懷遠鎮。乙卯,曹國遣使貢方物。



 甲子,高麗遣使請降,囚送斛斯政。上大悅。八月己巳,班師。右衛大將軍、左光祿大夫鄭榮卒。冬十月丁卯,上至東都。己丑,還京師。十一月丙申,支解斛斯政於金光門外。乙巳,有事於南郊。己酉,賊帥司馬長安破長平郡。乙卯,離石胡劉苗王舉兵反,自稱天子,以其弟六兒為永安王,眾至數萬。將軍潘長文討之,不能克。是月,
 賊帥王德仁擁眾數萬,保林慮山為盜。十二月壬申,上如東都,其日大赦天下。戊子,入東都。庚寅,賊帥孟讓眾十餘萬,據都梁宮。遣江都丞王世充擊破之,盡虜其眾。



 十一年春正月甲午朔,宴百寮。突厥、新羅、靺鞨、畢大辭、訶咄、傅越、烏那曷、波臘吐火羅、俱慮建、忽論、靺鞨、訶多、沛汗、龜茲、疏勒、于闐、安國、曹國、何國、穆國、畢、衣密、失範延、伽折、契丹等國,並遣使朝貢。戊戌,武賁郎將高建毗破賊帥顏宣政於齊郡,虜男女數千口。乙卯,大會蠻夷,設魚龍曼延之樂,頒賜各有差。二月戊辰,賊帥楊仲緒等率眾萬餘攻北平,滑公李景破斬之。



 庚午,詔曰:「設險
 守國,著自前經;重門禦暴,事彰往策。所以宅土寧邦,禁邪固本。而近代戰爭,居人散逸,田疇無伍,郛郭不修。遂使遊惰實繁,寇攘未息。



 今天下平一,海內晏如;宜令人悉城居,田隨近給。使強弱相容,力役兼濟;穿窬無所厝其姦宄,雚蒲不得聚其逋逃。有司具為事條,務令得所。」丙子,王須拔反,自稱漫天王,國號燕。賊帥魏刀兒自稱歷山飛,眾各十餘萬,北連突厥,南寇趙。



 三月丁酉,殺右驍衛大將軍光祿大夫郕公李渾、將作監光祿大夫李敏,並族滅其家。



 癸卯,賊帥司馬長安破西河。己酉,幸太原,避暑汾陽宮。秋七月己亥,淮南人張起緒舉兵為盜,
 眾至三萬。辛丑,光祿大夫、右禦衛大將軍張壽卒。八月乙卯,巡北塞。戊辰,突厥始畢可汗率騎數十萬,謀襲乘輿;義成公主遣使告變。壬申,車駕馳幸鴈門。癸酉,突厥圍城,官軍頻戰不利。上大懼,欲率精騎潰圍而出;民部尚書樊子蓋固諫,乃止。齊王暕以後軍保于崞縣。甲申,詔天下諸郡募兵。於是守令各來赴難。九月甲辰,突厥解圍而去。丁未,曲赦太原、雁門死罪已下。冬十月壬戌,上至於東都。丁卯,彭城人魏騏驎聚眾萬餘為盜,寇魯郡。壬申,賊帥盧明月聚眾十餘萬,寇陳、汝間。東海賊李子通擁眾度淮,自號楚王,建元明政,寇江都。十一月乙
 卯,賊帥王須拔破高陽郡。十二月戊寅,有大流星如斛,墜明月營,破其衝車。庚辰,詔民部尚書樊子蓋發關中兵,討絳郡賊敬盤陁、柴保昌等,經年不能剋。譙郡人朱粲擁眾數十萬寇荊、襄,僭稱楚帝,建元昌達。漢南諸郡,多為所陷焉。



 十二年春正月甲午,鴈門人翟松柏起兵於靈丘,眾至數萬,轉攻傍縣。二月己未,真臘遣使貢方物。甲子夜,有二大鳥似雕,飛入大業殿,止于御幄,至明而去。



 癸亥,東海賊盧公暹率眾萬餘,保于蒼山。夏四月丁巳,顯陽門災。癸亥,魏刀兒所部將甄翟兒號歷山飛,眾十萬,轉寇
 太原。將軍潘長文討之,反為所敗,長文死之。五月丙戌朔,日有蝕之,既。癸巳,大流星殞於吳郡,為石。壬午,上於景華宮徵求螢火,得數斛,夜出遊山而放之,光遍巖谷。秋七月壬戌,民部尚書、光祿大夫、濟北公樊子蓋卒。甲子,幸江都宮,以越王侗、光祿大夫段達、太府卿元文都、檢校民部尚書韋津、右武衛將軍皇甫無逸、右司郎盧楚等總留守事。奉信郎崔民象以盜賊充斥,於建國門表諫不宜巡幸。上大怒,先解其頤,乃斬之。戊辰,馮翊人孫華自號總管,舉兵為盜。高涼通守洗寶徹舉兵作亂,嶺南溪洞多應之。己巳,熒惑守羽林,月餘乃退。車駕次
 汜水,奉信郎王愛仁以盜賊日盛,諫上,請還西京。



 上怒,斬之而行。八月乙巳,賊帥趙萬海眾數十萬,自恆山寇高陽。壬子,有大流星如斗,出王良、閣道,聲如壞牆。癸丑,大流星如甕,出羽林。九月丁酉,東海人杜伏威、揚州沈覓敵等作亂,眾至數萬,右禦衛將軍陳棱擊破之。戊午,有二枉矢,出北斗魁,委曲蛇形,注於南斗。壬戌,安定人荔非世雄殺臨涇令,舉兵作亂,自號將軍。冬十月己丑,開府儀同三司、左翊衛大將軍、光祿大夫、許公宇文述薨。



 十二月癸未,鄱陽賊操天成舉兵反,自號元興王,建元始興,攻陷豫章郡。乙酉,以右翊衛大將軍來護為開
 府儀同三司,行左翊衛大將軍。壬辰,鄱陽人林士弘自稱皇帝,國號楚,建元太平。攻陷九江、廬陵郡。唐公破甄翟兒於西河,虜男女千口。



 十三年春正月壬子,齊郡賊杜伏威率眾度淮,攻陷歷陽郡。丙辰,勃海賊竇建德設壇於河間之樂壽,自稱長樂王,建元丁丑。辛巳,賊帥徐圓郎率眾數千破東平郡。弘化人到GC成聚眾萬餘人為盜,傍郡苦之。二月壬午,朔方人梁師都殺郡丞唐世宗,據郡反,自稱大丞相。遣銀青光祿大夫張世隆擊之,反為所敗。戊子,賊帥王子英破上谷郡。己丑,馬邑校尉劉武周殺太守王仁恭,舉
 兵作亂,北連突厥,自稱定楊可汗。庚寅,賊帥李密、翟讓等陷興洛倉。越王侗遣武賁郎將劉長恭、光祿少卿房崱擊之,反為所敗,死者十五六。庚子,李密自號魏公,稱元年;開倉以賑群盜,眾至數十萬。河南諸郡,相繼皆陷焉。壬寅,劉武周破武賁郎將王智辯於桑乾鎮,智辯死之。三月戊午,廬江人張子路舉兵反,遣右禦衛將軍陳稜討平之。



 丁丑,賊帥李通德眾十萬寇廬江,左屯衛將軍張鎮州擊破之。夏四月癸未,金城校尉薛舉率眾反,自稱西秦霸王,建元秦興。攻陷隴右諸郡。己丑,賊帥孟讓夜入東都外郭,燒豐都市而去。癸巳,李密陷迥洛東
 倉。丁酉,賊帥房憲伯陷汝陰郡。是月,光祿大夫武賁郎將裴仁基、淮陽太守趙佗等,並以眾叛歸李密。五月辛酉夜,有流星如甕,墜於江都,甲子,唐公起義師於太原。丙寅,突厥數千寇太原,唐公擊破之。秋七月壬子,熒惑守積屍。丙辰,武威人李軌舉兵反,攻陷河曲諸郡,自稱涼王,建元安樂。八月辛巳,唐公破武牙郎將宋老生於霍邑,斬之。九月己丑,帝括江都人女、寡婦以配從兵。是月,武陽郡丞元寶藏以郡叛歸李密,與賊帥李文相攻陷黎陽倉。彗星見於營室。冬十月丁亥,太原陽世洛聚眾萬餘人。寇掠城邑。



 丙申,羅令蕭銑以縣反,鄱陽人董
 景珍以郡反,迎銑於羅縣,號為梁王,攻陷傍郡。



 戊戌,武賁郎將高毗敗濟北郡賊甄寶車於監山。十一月丙辰,唐公入京師。辛酉,遙尊帝為太上皇,立代王侑為帝,改元義寧。上起宮丹楊,將遜于江左。有烏鵲來巢幄帳,驅不能止。熒惑犯太微。有石自江浮入于楊子,日光四散如流血,上甚惡之。二年三月,右屯衛將軍宇文化及、武賁郎將司馬德戡、元禮、監門直閣裴虔通、將作少監宇文智及、武勇郎將趙行樞、鷹揚郎將孟景、內史舍人元敏、符璽郎李覆、牛方裕、千牛左右李孝本、弟孝質、直長許弘仁、薛世良、城門郎唐奉義、醫正張愷等,以驍果作
 亂,入犯宮闈。上崩于溫室,時年五十。蕭后令宮人撤床簀為棺,以埋之。化及發後,右禦衛將軍陳稜奉梓宮於成象殿,葬吳公臺下。發斂之始,容貌若生,眾咸異之。大唐平江南之後,改葬雷塘。



 初,上自以蕃王,次不當立,每矯情飾行,以釣虛名,陰有奪宗之計。時高祖雅重文獻皇后,而性忌妾媵;皇太子勇內多嬖幸,以此失愛。帝後庭有子皆不育之,示無私寵,取媚於后。大臣用事者,傾心與交。中使至第,無貴賤,皆曲承顏色,申以厚禮。婢僕往來者,無不稱其仁孝。又常私入宮掖,密謀於文獻后。楊素等因機構扇,遂成廢立。自高祖大漸暨諒闇之中,
 蒸淫無度。山陵始就,即事巡游。以天下承平日久,士馬全盛。慨然慕秦皇、漢武之事。乃盛理宮室,窮極侈靡。召募行人,分使絕域。諸蕃至者,厚加禮賜;有不恭命,以兵擊之。盛興屯田於玉門、柳城之外。課天下富室分道市武馬,疋直十餘萬。富強坐是而凍餒者,十家而九。



 性多詭譎。。所幸之處,不欲人知;每幸之所,輒數道置頓。四海珍羞殊味,水陸必備焉。求市者無遠不至。郡縣官人,競為獻食;豐厚者進擢,疏儉者獲罪。姦吏侵漁,內外虛竭;頭會箕斂,人不聊生。于時,軍國多務,日不暇給。帝方驕怠,惡聞政事;冤屈不理,奏請罕決。又猜忌臣下,無所專
 任。朝臣有不合意者,必構其罪而族滅之。高熲、賀若弼先皇心膂,參謀帷幄;張衡、李金才籓邸惟舊,績著經綸。惡其直道,忌其正義;求其無形之罪,加以丹頸之戮。其餘事君盡禮,謇謇匪躬;無辜無罪,橫受夷戮者,不可勝紀。政刑弛紊,賄貨公行,莫敢有言,道路以目。六軍不息,百役繁興;行者不歸,居者失業;人飢相食,邑落為墟,上弗之恤也。東西行幸,靡有定居;每以供費不給,逆收數年之賦。所至,唯與後宮流連耽湎,惟日不足。招迎姥媼,朝夕共肆醜言。又引少年,令與宮人穢亂。不軌不遜,以為娛樂。區宇之內,盜賊蜂起;劫掠從官,屠陷城邑。近臣
 互相掩蔽,皆隱賊數,不以實對。或有言賊多者,輒大被詰責。各求茍免,上下相蒙。每出師徒,敗亡相繼。戰士盡力,不加賞賜;百姓無辜,咸受屠戮。蒸庶積怨,天下土崩;至於就禽,而猶未之寤也。



 恭皇帝諱侑,元德太子之子也。母曰韋妃。性聰敏,有氣度。大業三年,立為陳王。後數載,徙為代王。及煬帝親征遼東,令於京師總留事。十一年,從幸晉陽,拜太原太守,尋鎮京師。義兵入長安,尊煬帝為太上皇,奉帝纂業。



 義寧元年十一月壬戌,上即皇帝位於大興殿。詔曰:「王道喪亂,天步不康;屬之於朕,逢此百罹。襁褓之歲,夙遭憫
 凶;孺子之辰,太上播越。興言感動,實疚于懷。太尉唐公,膺期作宰,糾合義兵,翼戴皇室。爰奉明詔,弼予幼沖,顯命光臨,天威咫尺。對揚尊號,悼心失圖;一人在遠,三讓不遂;僶勉南面,厝身無所。茍利社稷,莫敢或違;俯從群議,奉遵聖旨。可大赦天下。改大業十三年為義寧元年。十一月十六日昧爽以前,大辟罪已下,皆赦除之;常赦所不免者,不在赦限。」甲子,以光祿大夫、大將軍、太尉唐公為假黃鉞、使持節、大都督內外諸軍事、尚書令、大丞相,進封唐王。丙寅,詔曰:「朕惟孺子,未出深宮;太上遠巡,追蹤穆滿。時逢多難,委當尊極;辭不獲免,恭己臨朝。若
 涉大川,罔知所濟;民之情偽,曾未之聞。賴股肱戮力,上宰賢良;匡佐沖人,輔其不逮。軍國機務,事無大小;文武設官,位無貴賤;憲章賞罰,咸歸相府。庶績其凝,責成斯屬。」己巳,以唐王子隴西公建成為唐國世子;敦煌公為京兆尹,改封秦公;元吉為齊公。



 太原置鎮北府。乙亥,張掖康老和舉兵反。十二月癸未,薛舉自稱天子,寇扶風,秦公為元帥擊破之。丁亥,桂陽人曹武徹舉兵反,建元通聖。丁酉,義師禽驍衛大將軍屈突通於閿鄉。乙巳,賊帥張善安陷廬江郡。



 二年春正月丁未,詔唐王劍履上殿,入朝不趨,贊拜不
 名,加前後羽葆鼓吹。



 壬戌,將軍王世充為李密所敗,河內通守孟善誼、武賁郎將王辯、楊威、劉長恭、梁德、董智通皆死之。庚戌,河陽郡尉獨孤武都降於李密。三月丙辰,右屯衛將軍宇文化及弒太上皇於江都宮。右禦衛將軍獨孤盛死之。齊王暕、趙王杲、燕王倓、右翊衛大將軍宇文協、內史侍郎虞世基、御史大夫裴蘊、給事郎許善心皆遇害。化及立秦王浩為帝,自稱大丞相,朝士文武,皆受其官爵。光祿大夫宿公麥才、折衝郎將朝請大夫沈光同謀討賊,夜襲化及營,反為所害。戊辰,詔唐王備九錫之禮,加璽紱、遠游冠、綠綟綬,位在諸侯王上。唐國
 置丞相已下,一依舊式。五月乙巳朔,詔唐王冕十有二旒,建天子旌旗;出警入蹕,金根車,駕備五時副車,置旄頭雲罕車;舞八佾,設鐘虞宮縣。王后、王子、王女爵命之號,一遵舊典。戊午,詔曰:「天禍隋國,大行太上皇遇盜江都。憫予小子,哀號永感,仰惟荼毒,仇復靡申。相國唐王膺期命世,扶危拯溺;自北徂南,東征西怨。總九合於一匡,決百勝於千里。糾率夷夏,大庇氓黎;保義朕躬,繄王是賴。德侔造化,功格蒼旻;兆庶歸心,歷數斯在;屈為人臣,載違天命。當今九服崩離,三靈改卜,大運去矣,請避賢路。私僮命駕,須歸籓國。予本代王,及予而代,天之所
 廢,豈期如是。庶憑稽古之聖,以誅四凶;幸值惟新之恩,預充三恪。雪冤恥於皇祖,守禋祀為孝孫;朝聞夕殞,及泉無恨。今遵故事,遜於舊邸。庶官群辟,改事唐朝。宜依前典,趣上尊號。若釋重負,感泰兼懷。假手真人,俾除醜逆。」仍敕有司,凡有表奏,皆不得以聞。是日,上遜位於大唐。以為酅國公。武德二年夏五月崩,時年十五。



 史臣曰:煬帝爰在弱齡,早有志尚,南平吳會,北卻匈奴,昆弟之中,獨著聲績。於是矯情飾貌,肆厥姦回,故得獻後鐘心,文皇革慮。天方肇亂,遂升儲兩。



 踐峻極之榮基,承丕顯之休命。地廣三代,威振八紘。單于頓顙,越常重
 譯。赤仄之泉,流溢于都內;紅腐之粟,充積於塞下。負其富強之資,思逞無厭之欲。狹殷周之制度,尚秦漢之規摹。恃才矜己,傲狠明德。內懷險躁,外示凝簡。盛冠服以塞其奸,除諫官以掩其過。淫荒無度,法令滋彰;教絕四維,刑參五虐。誅鋤骨肉,屠劓忠良。受賞者莫見其功,為戮者莫聞其罪。驕怒之兵屢動,土木之功不息。頻出朔方,三駕遼左。旌旗萬里,征稅百端。猾吏侵漁,人弗堪命。乃急令暴賦以擾之,嚴刑峻法以臨之,甲兵威武以董之,自是海內騷然,無聊生矣。俄而玄感肇黎陽之亂,匈奴有鴈門之圍。天子方棄中土,遠之揚、越。姦宄乘釁,強
 弱相陵;關梁閉而不通,皇輿往而莫返。加之以師旅,因之以饑饉,流離道路,轉死溝壑,十七八焉。於是相聚雚蒲,蝟毛而起。大則跨州連郡,稱帝稱王;小則千百為群,攻城剽邑。流血成川澤,死人如亂麻;炊者不及析骸,食者不遑易子。茫茫九土,並為糜鹿之場;惵惵黔黎,俱充蛇豕之餌。四方萬里,簡書相續。猶謂鼠竊狗盜,不足為虞;上下相蒙,莫肯念亂。振蜉蝣之羽,窮長夜之樂。土崩魚爛,貫盈惡稔。



 普天之下,莫匪仇讎;左右之人,皆為敵國。終然不悟,同彼望夷;遂以萬乘之尊,死於疋夫之手。億兆靡感恩之士,九牧無勤王之師。子弟同就誅夷,體
 骨棄而莫掩。



 社稷顛隕,本枝殄絕。自肇有書契,以迄於茲,宇宙崩離,生靈塗炭,喪身滅國,未有若斯之甚也。《書》曰:「天作孽,猶可違;自作孽,不可逭。」《傳》曰:「吉兇由人,妖不妄作。」又曰:「兵猶火也,不戢將自焚。」觀隋室之存亡,斯言有徵矣。恭帝年在幼沖,遭家金難。一人失德,四海土崩;群盜螽起,豺狼塞路;南巢遂往,流彘不歸。既鐘百六之期,躬踐數終之運,謳歌有屬,笙鐘變響。雖欲不遵堯、舜之跡,庸可得乎。



\end{pinyinscope}