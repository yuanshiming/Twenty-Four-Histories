\article{卷二十四列傳第十二}

\begin{pinyinscope}

 崔逞子頤孫彧
 玄孫冏休五世孫六世孫贍麃逞兄遹王憲曾孫昕晞皓封懿族曾孫回回子隆之回弟肅回族弟述崔逞,字叔祖,清河東武城人,魏中尉琰之五世孫也。曾祖諒,晉中書令。祖遇,仕石氏,為特進。父瑜,黃門郎。逞少好學,有文才。仕慕容,補著作郎,撰《燕記》。遷黃門侍郎。滅,苻堅以為齊郡太守。堅敗,仕晉,歷清河、平原二郡太守。為翟遼所虜,以為中書令。慕容垂滅翟釗,以為秘
 書監。慕容寶東走和龍,為留臺吏部尚書。及慕容驎立,逞攜妻子歸魏。張袞先稱美之,由是道武禮遇甚厚。拜尚書,錄三十六曹,別給吏屬,居門下省。尋除御史中丞。



 道武攻中山,未剋,六軍乏糧,問計於逞。逞曰:「飛鴞食葚而改音,《詩》稱其事,可取以助糧。」帝雖銜其侮慢,然兵既須食,乃聽人以葚當租。逞又言:「可使軍人及時自取,過時則落盡。」帝怒曰:「內賊未平,兵人安可解甲收葚乎!」



 以中山未拔,故不加罪。及姚興侵晉,襄陽戍將郗恢馳使乞師於常山王遵,書云「賢兄武步中原」。道武以為悖君臣之體,敕逞與張袞為遵書答使,亦貶其主號以報之。
 逞、袞為書,乃云「貴主」。帝怒其失旨,黜袞,遂賜逞死。



 後晉荊州刺史司馬休之等數十人為桓玄所逐,皆將來奔。至陳留,聞逞被殺,分為二輩:一奔長安,一奔廣固。帝聞深悔,自是士人有過,多見優容。



 逞子毅、禕、嚴、頤。初,逞之內徙,終慮不免,乃使其妻張氏與四子歸慕容德於廣固,獨與小子頤在代京。及逞死,亦以此為譴。



 頤字太沖,散騎常侍,賜爵清河侯。太武聞宋以其兄諲為冀州刺史,乃曰:「義隆用其兄,我豈無冀州地邪?」乃以頤為冀州刺史。入為大鴻臚,持節策拜楊難當為南秦王。奉使數返,光揚朝命,太武善之。後與方士韋文秀詣
 王屋山造金丹,不就。真君初,卒。始崔浩與頤及榮陽太守模等,年皆相次。浩為長,次模,次頤。



 三人別祖,而模、頤為親。浩恃其家世魏、晉公卿,常侮模、頤。浩不信佛道,模深所歸向,雖糞壤中,禮拜形像。浩大笑曰:「持此頭顱,不凈處跪是胡神也!」



 模嘗謂人曰:「桃簡可欺我,何容輕我周兒也!」浩小名桃簡,頤小名周兒。太武頗聞之,故浩誅時,二家獲免。



 頤五子。少子睿以交通境外,伏誅。自逞之死,至睿之誅,三世,積五十餘年,在北一門盡矣。



 彧字文若,頤兄禕之孫也。父勳之,字寧國,位大司馬、外兵郎,贈通直郎。



 彧與兄相如俱自宋入魏。相如以才學
 知名,早卒。



 彧少逢隱沙門,教以《素問》、《甲乙》,遂善醫術。中山王英子略曾病,王顯等不能療。彧針之,抽針即愈,後位冀州別駕。性仁恕,見疹者,喜與療之。廣教門生,令多救療。其弟子清河趙約、勃海郝文法之徒,咸亦有名。



 彧子景哲,豪率,亦以醫術知名。仕魏,太中大夫、司徒長史。



 景哲子冏,字法峻,幼好學,泛覽經傅,多伎藝,尤工相術。仕魏為司空參軍。



 齊天保初,為尚藥典御。歷高陽太守、太子家令。武平中,為散騎常侍、假儀同三司。從幸晉陽,嘗謂中書侍郎李德林曰:「比日看高相王以下文武官人相表,俱盡其事,口不忍言。唯弟一人更應富貴,當在
 他國,不在本朝,吾不及見也。」其精如此。



 冏性廉謹,恭儉自修,所得俸秩,必分親故。終鴻臚卿。臨終,誡其二子曰:「夫恭儉福之輿,傲侈禍之機。乘福輿者浸以康休,蹈禍機者忽而傾覆,汝其誡歟!



 吾沒後,斂以時服,祭無牢餼,棺足周屍,瘞不泄露而已。」及卒,長子修遵父命。



 景哲弟景鳳,字鸞叔,位尚藥典御。



 休字惠盛。曾祖諲,仕宋位青、冀二州刺史。祖靈和,宋員外散騎侍郎。父宗伯,始還魏,追贈清河太守。休少孤貧,矯然自立。舉秀才,入京師,與宋弁、邢巒雅相知友。尚書王嶷欽其人望,為長子娉休姊,贍以財貨,由是少振。孝
 文納休妹為嬪。頻遷兼給事黃門侍郎。休勤學,公事軍旅之隙,手不釋卷。禮遇亞于宋弁、郭祚。孝文南伐,以北海王詳為尚書僕射,統留臺事,以休為尚書左丞。詔以北海年少,百揆務殷,便以委休。轉長史,兼給事黃門侍郎,參定禮儀。帝嘗閱故府,得舊冠,題曰:「南部尚書崔逞制」。顧謂休曰:「此卿家舊事也。」後從駕南行。



 及還,幸彭城,汎舟泗水,詔在侍筵,觀者榮之。



 宣武初,休以祖父未葬,弟夤又亡,固求出為勃海太守。性嚴明,雅長政體。



 下車先戮豪猾數人,姦盜莫不禽翦。清身率下,部內安之。時大儒張吾貴名盛山東,弟子恒千餘人,所在多不見容。
 休招延禮接,使肄業而還,儒者稱為口實。入為吏部郎中,遷散騎常侍,權兼選任,多所拔擢。廣平王懷數引談宴。以與諸王交游,免官。後為司徒右長史,公平清潔,甚得時譽。歷幽、青二州刺史,皆以清白稱,二州懷其德澤。入為度支、七兵、殿中三尚書。休久在臺閣,明習典故,每朝廷疑議,咸取正焉。諸公咸謂崔尚書下意處不可異也。卒,贈尚書右僕射,謚曰文貞。



 休少而謙退,事母孝謹。及為尚書,子仲文娶丞相高陽王雍女,女適領軍元叉庶長子舒,挾恃二家,志氣微改,陵藉同列。尚書令李崇、左僕射蕭寶夤、右僕射元欽皆以此憚下之。始休母房
 氏欲以休女妻其外孫邢氏,休乃違母情,以妻叉子,議者非之。子甗。



 甗字長儒,狀貌偉麗,善於容止。少知名。為魏宣武挽郎。釋褐太學博士,累遷散騎侍郎。坐事免歸鄉里。冀部豪傑之起,爭召甗兄弟,甗中立無所就。高敖曹以三百騎劫取之,以為師友。齊神武至信都,以為開府諮議參軍,歷給事黃門侍郎、衛將軍。神武入洛,議定廢立。太僕綦俊盛言節閔帝賢明,可主社稷。甗作色而前曰:「若其賢明,自可待我高王。既為逆胡所立,何得猶作天子?若從俊言,王師何名義舉?。由是節閔及中興主皆廢。更立平
 陽王,是為孝武。以建義功,封武城縣公。



 甗恃預義旗,頗自矜縱。尋以貪汙為御史糾劾,逃還鄉里。時清河多盜,齊文襄以石愷為太守,令得專殺。愷經甗宅,謂少年曰:「諸郎輩莫作賊,太守打殺人!」



 甗顧曰:「何不答府君:下官家作賊,止捉一天子牽臂下殿,捉一天子推上殿。不作偷驢摸犢賊。」及遇赦出,復為黃門。天平中,授徐州刺史,給廣宗部曲三百,清河部曲千。



 甗性暴慢。寵妾馮氏,長且姣,家人號曰成母,朝士邢子才等多姦之。至是假其威勢,恣情取受,風政不立。



 初,甗為常侍,求人修起居注,或曰:「魏收可。」甗曰:「收輕薄徒耳。」



 更引祖鴻勛為之。又欲
 陷收不孝之罪,乃以盧元明代收為中書郎。由是收銜之。及收聘梁,過徐州,甗備刺史鹵簿迎之,使人相聞收曰:「勿怪儀衛多,稽古力也。」



 收語蹇,急報曰:「崔徐州建義之勛,何稽古之有?」甗自以門伐素高,特不平此言。收乘宿憾,故以此挫之。罷徐州,除祕書監,以母憂去官。服終,兼太常卿,轉七兵尚書、清河邑中正。



 甗有文學,偉風貌,寡言辭,端嶷如神,以簡貴自處。齊神武言:「崔甗應作令僕,恨其精神太遒。趙郡李渾將聘梁,名輩畢萃,詩酒正謹,甗後到,一坐無復談話。鄭伯猷歎曰:「身長八尺,面如刻畫,謦欬為洪鐘饗,胸中貯千卷書,使人那得不畏服!」



 甗以籍地自矜,常與蕭祗、明少遐等高宴終日,獨無言。少遐晚謂甗曰:「驚風飄白日,忽然落西山。」甗亦無言,直曰「爾」。每謂盧元明曰:「天下盛門唯我與爾,博崔、趙李何事者哉!」崔暹聞而銜之。神武葬後,甗又竊言:「黃頷小兒堪當重任不?」暹外兄李慎以告暹。暹啟文襄,絕甗朝謁。甗要拜道左,文襄發怒曰:「黃頷兒何足拜也!」於是鎖甗赴晉陽,訊之,不服。暹引邢子才為證,子才執無此言。甗在禁謂邢曰:「卿知我意屬太丘不?」邢出,告甗子贍曰:「尊公意,正應欲結姻陳元康。」贍有新生女,乃許妻元康子。元康為言於文襄曰:「崔甗名望素重,不可以私語殺之。」
 文襄曰:「若免其性命,當徙之遐裔。」元康曰:「甗若在邊,或將外叛。以英賢資寇敵,非所宜也。」文襄曰:「既有季珪之罪,還令輸作可乎?」元康曰:「元康常讀《崔琰傳》,追恨魏武不弘。甗若在作所而殞,後世豈道公不殺也?」文襄曰:「然則奈何?」元康曰:「甗合死。朝野皆知。



 公誠能以寬濟猛,特輕其罰,則仁德彌著,天下歸心。」段孝先亦言甗勛舊,乃舍之。甗進謁奉謝,文襄猶怒曰:「我雖無堪,忝當大任,被卿以為黃頷小兒。金石可銷,此言難滅!」



 齊天保初,除侍中,監起居。以禪代之際,參掌儀禮,別封新豐縣男,回授第九弟子約。



 甗一門婚嫁,皆衣冠美族;吉凶儀範,為當
 時所稱。婁太后為博陵王納甗妹為妃,敕其使曰:「好作法用,勿使崔家笑人。」婚夕,文宣帝舉酒祝曰:「新婦宜男,孝順富貴。」甗跪對:「孝順乃自臣門,富貴恩由陛下。」五年,為東兗州刺史,復攜馮氏之部。為馮氏厭蠱,頗失精爽,尋遇偏風。馮氏受納狼籍,為御史劾,與甗俱召,詔付廷尉。諸囚多姦焉,獄中致競。尋別詔斬馮氏於都市,支解為九段。



 甗以疾卒獄中。



 甗歷覽群書,兼有辭藻。自中興迄於孝武,詔誥表檄多甗所為。然性侈,耽財色,於諸弟不能盡雍穆之美,世論以此譏之。素與魏收不協,收後專典國史,甗恐被惡言,乃悅之曰:「昔有班固,今則魏子。」
 收鼻笑之,憾不釋。甗子贍。



 贍字彥通,潔白,善容止,神彩嶷然,言不妄發,才學風流為後來之秀。初,潁川荀濟自江南入洛,贍學於濟,故得經史有師法。侍中李神俊雅有風譽,晚年無子,見贍,歎謂邢邵曰:「昨見崔甗兒,便為後生第一。我遂無此物,見此使人傷懷!」



 年十五,刺史高昂召署主簿,清河公高岳避為開府西閣祭酒。博陵崔暹為中尉,啟除侍御史。以父與暹隙,俄而去官。神武召與北海王晞俱為諸子賓友,仍為相府中兵參軍,轉主簿。文襄崩,秘未發喪,文宣命贍兼相府司馬,使鄴。



 魏孝靜帝以人日登雲龍門。與
 其父甗俱侍宴為詩。詔問邢邵等曰:「令贍此詩何如其父?」咸曰:「甗博雅弘麗,贍氣調清新,並詩人之冠冕。」宴罷,咸共嗟賞之,云:「今日之宴,併為崔贍父子。」楊愔欲引贍為中書侍,時盧思道直中書省,愔問其文藻優劣,思道曰:「崔贍文詞之美,實有可稱,但舉世重其風流,所以才華見沒。」愔云:「此言有理。」其日奏用之。愔又曰:「昔裴瓚晉世為中書郎,神情高邁,每於禁門出入,宿衛者皆肅然動容。崔生堂堂,亦當無愧裴子乎?」



 皇建元年,除給事黃門侍郎。與趙郡李概為莫逆之友。概將東還,贍遺之書曰:「仗氣使酒,我之常弊,詆訶指切,在卿尤甚。足下告
 歸,吾於何聞過也?」贍患氣,兼性遲重,雖居二省,竟不堪敷奏。



 孝昭踐阼,皇太子就傅受業,除太子中庶子,征赴晉陽。敕曰:「東宮弱年,未陶訓義。卿儀形風德,人之師表,故勞卿朝夕遊處,開發幼蒙。一物三善,皆以相寄。」贍專在東宮,調護講讀及進退禮度,皆歸委焉。太子納妃斛律氏,敕贍與鴻臚崔勵撰定婚禮儀注,主司以為後式。時詔議三恪之禮,太子少傅魏收為一議,朝士莫不雷同。贍別立異議,收讀訖笑而不言。贍正色曰:「聖上詔群臣議國家大典,少傅名位不輕,贍議若是,須贊其所長;若非,須詰其不允。何容讀國士議文,直此冷笑?崔贍居
 聖朝顯職,尚不免見疵,草萊諸生,欲云何自進!」贍容貌方嚴,詞旨雄辯;收慚遽,竟無一言。



 大寧元年,除衛尉少卿。尋兼散騎常侍,聘陳使主。行過彭城,讀道旁碑文未畢而絕倒。從者遙見,以為中惡。此碑乃贍父徐州時所立,故哀感焉。贍經熱病,面多瘢痕,然雍容可觀,辭韻溫雅,南人大相欽服。陳舍人劉師知見而心醉,乃言:「常侍,前朝通好之日何意不來?今日誰相對揚者!」其見重如此。還,襲爵武城公,再遷吏部郎中。因患耳,請急十餘日。舊式,百日不上,解官。吏部尚書尉瑾性偏急,以贍舉措舒緩,曹務煩劇,附驛奏聞。因見代,遂免歸。天統末,加驃
 騎大將軍,就拜銀青光祿大夫。卒,贈大理卿、濟州刺史,謚曰文。



 贍性簡傲,以才地自矜,所與周旋,皆一時名望。在御史臺,恆宅中送食,備盡珍羞;別室獨餐,處之自若。有一河東人士姓裴,亦為御史,伺贍食,便往造焉。



 贍不與交言,又不命匕箸。裴坐觀贍食罷而退。明日,自攜匕箸,恣情飲啖。贍謂曰:「我初不喚君食,亦不共君語,遂能不拘小節。昔劉毅在京口冒請鵝炙,豈亦異是?君定名士。」於是每與之同食。性方重,好讀書,酒後清言,聞者莫不傾耳。



 自天保以後,重吏事,謂容止醖籍者為潦倒,而贍終不改焉。常見選曹以劉逖為縣令,謂之曰:「官長正
 應子琮輩,乃復屈名人!」馮子琮聞之大怒。及其用事,幾敗焉。有集二十卷。



 甗弟仲文,有文學。太和中,為丞相掾。沙苑之敗,仲文持馬尾度河,波中乍沒乍出。神武望見,曰:「崔掾也。」遽遣船赴接。及至,謂曰:「卿為君為親,不顧萬死,可謂家之孝子,國之忠臣也。」後文襄欲使行青州,聞其多醉,乃止。



 天保初,甗為侍中,仲文為銀青光祿大夫,同日受拜,時云兩鳳連飛。嘗被敕召,宿酲未解。文宣怒,將罰之。試使為觀射詩十韻,操筆立成,乃原之。拜散騎常侍、光祿大夫。卒。子偃,太子洗馬、尚書郎。偃弟儦。



 儦字岐叔。少與范陽盧思道、隴西辛德源同志友善。每
 以讀書為務,負恃才地,大署其戶曰:「不讀五千卷者,無得入此室」初舉秀才,為員外散騎侍郎。遷殿中侍御史。與熊安生、馬敬德等議五禮,兼修律令。尋兼散騎侍郎,使陳。還,待詔文林館。歷尚書郎。與頓丘李若俱見稱重,時人語曰:「京師灼灼,崔儦、李若。」



 若每謂其子曰:「盧思道、崔儦,杳然崖岸,吾所重也,汝其師之。」思道與儦嘗酒後相調,儦曰:「偃邈無聞。」思道譏儦云:「高曾官薄。」齊亡,歸鄉。仕郡為功曹,補主簿。隋開皇四年,徵授給事郎,兼內史舍人。後兼通直散騎侍郎,聘陳。還,授員外散騎侍郎。以聾,常得無事,一醉輒八日。越國公楊素時方貴幸,重
 儦門地,為子玄縱娶其女為妻,娉禮甚厚。親迎之始,公卿滿坐,素令騎迎之。



 儦弊衣冠騎驢而至。素推令上坐,儦禮甚倨,言又不遜,素忿然拂衣而起,竟罷坐。



 後數日,儦方來謝,素待之如初。詔授易州刺史,或言其未合,乃追停。鹿語人曰:「易州刺史何必勝道義。」仁壽中,卒於京師。子世濟。



 仲文弟叔仁,輕俠重衿期。仕魏為潁州刺史。以貪污,為御史中丞高仲密劾,賜死於宅。臨刑,賦詩五絕,與諸弟訣別。不及其兄甗,以其不甚營救也。子彥武,有識用。隋開皇初,位魏州刺史。



 叔仁弟叔義,魏孝莊時為尚書庫部郎。初,叔義父休為青州刺史,放盜魁,令出
 其黨,遂以為門客。在洛陽,與兄叔仁鑄錢。事發,合家逃逸,叔義見執。時城陽王徽為司州牧,臨淮王彧以非其身罪,驟為致言。徽以求婚不得,遂停赦書而殺之。



 叔義弟子侃,以寄名從軍竊級為中書郎。為尚書左丞和子岳彈糾,失官。性兼使氣。從自修改,閉門讀書,當時稱為博洽。後兼通直散騎常侍,使梁,為陽斐副。



 恥居斐下,自負才地,呼斐為陽子,語輒折之。還,卒於路。子拯,位太子僕、武德郡守。



 子侃弟子植,位冀州別駕。走馬從禽,髮挂木而死。子珪。



 子植弟子聿,位東莞太守。



 子聿弟子約。五
 歲喪父,不肯食肉。後喪母,居喪哀毀骨立。人云:「崔九作孝,風吹即倒。」禫月,兄子度死,又百日不入房。長八尺餘,姿神俊異,潛觀梁使劉孝儀,賓從見者駭目。武定中,為平原公開府祭酒。與兄子贍俱詣晉陽,寄居佛寺。贍長於子約二歲,每退朝久立,子約馮几對之,儀望俱華,儼然相映。諸沙門竊窺之,以為二天人也。乾明中,為考功郎。病且卒,謂贍曰:「自諸兄歿,而門業頹替,居家大唯吾與爾。命之修短,曾何足悲。汝能免之,吾不餒矣。」



 休弟夤,字敬禮,位太子舍人。卒,贈樂安太守。妻,樂安王長女晉寧公主也,貞烈有德行。



 子愍,字長謙,幼聰敏。濟州刺史盧尚之欲以長女妻之,休子甗為長謙求尚之次女,曰:「
 家道多由婦人,欲令姊妹為妯娌。」尚之感其義,於是同日成婚。休誡諸子曰:「汝等宜皆一體,勿作同堂意。若不用吾言,鬼神不享汝祭祀。」休亡,枕中有書,如平生所誡,諸子奉焉。長謙與休第二子仲文同年而月長,其家謂之大二、小二。長謙少與太原王延業俱為著作佐郎,監典校書。後為青州司馬。賊圍城二百日,長謙書不廢,凡咨手抄八千餘紙,天文、律歷、醫方、卜相、風角、鳥言,靡不開解。晚頗以酒為損。遷司徒諮議,修起居注,加金紫光祿大夫。後兼散騎常侍,使梁。將行,謂人曰:「我厄在吳國,忌在酉年,今恐不免。」及還,未入境,卒。年二十八。贈南青
 州刺史。逞兄遹。



 遹字寧祖,亦有名於時。為慕容垂尚書左丞、范陽昌黎二郡太守。



 遹曾孫延壽,冀州主簿。輕財好施,甚收鄉曲譽。



 延壽子隆宗,簡率友悌,居喪以孝聞。位蘭陵、燕二郡太守。仁信待物,檢慎至誠,故見重於時。卒,贈齊州刺史,謚曰孝。



 子敬保,冀州儀同府從事中郎。卒,贈冀州刺史。



 敬保子子恒,位魯郡太守,早卒。



 子恒弟子安、子昇,武定中,連元瑾事伏法。



 逞宗人模,字思範,琰兄霸之後也。父遵,慕容垂少府卿。模仕宋為榮陽太守。



 神中,平滑臺,歸降,後賜爵武城男。模長者篤厚,不營榮利,雖為崔浩
 輕侮,而不為浩屈。與崔頤相親,往來如一家。



 始模在南,妻張氏有二子,仲智、季柔。模至京師,賜妻金氏,生子幼度。仲智等以父隔遠,乃聚貨規贖歸之。其母張曰:「汝父志懷無決,必不能來。」行人以賄至都,模果顧念幼度等,指謂行人曰:「何忍捨此輩,致為刑辱。當為爾取一人,使名位不減我。」乃授以申謨,宋東郡太守也。神。中被執,賜妻,生子靈度。申謨聞此,乃棄妻子走還江外。靈度刑為閽人。



 初,直君末,模兄協子邪利為宋魯郡太守,以郡降。賜爵臨淄子,拜廣寧太守,卒。邪利二子,懷順、次恩,仍居宋青州。懷順以父入魏,故不仕。及魏克青州,懷順迎
 邪利喪還青州云。



 王憲,字顯則,北海劇人也。其先姓田,秦始皇滅齊,田氏稱王家子孫,因以為氏。仍居海岱。祖猛,仕苻堅,位丞相。父休,河東太守。憲幼孤,隨伯父永在鄴。苻丕稱尊號,復以永為丞相。永為慕容永所殺,憲匿於清河人家。皇始中,乃歸魏。道武見之,曰:「此王猛孫也。」厚禮待之,以為本州中正,領選曹事,兼掌門下。太武即位,遷廷尉卿。出為上谷太守,賜爵高唐子。清身率下,風化大行。



 尋拜外都大官,復移中都。歷任二曹,斷獄稱旨。進爵劇縣侯。出為並州刺史,又進北海公。境內清肅。及還京師。以憲年老,
 特賜錦繡布帛,珍羞醴膳。天安初,卒,年八十九。謚曰康。子崇襲。



 崇弟嶷,字道長。孝文初,為南部尚書,在任十四年。時南州多事,訟者填門。



 嶷性儒緩不斷,終日昏睡。李、鄧宗慶等,號為明察,而二人終見誅戮。餘十數人或出或免,唯嶷卒得自保。時人語曰:「實癡實昏,終得保存。」後封華山公,入為內都大官,卒。子祖念襲爵。



 祖念弟雲,字羅漢,頗有風尚,位南兗州刺史。坐受所部荊山戍主杜虔財,又取官絹,因染遂有割易,御史糾劾。會赦免。卒官,贈豫州刺史,謚文昭。長子昕。



 昕字元景,少篤學,能誦書,日以中疊舉手極上為率。與
 太原王延業俱詣魏安豐王延明。延明歎美之。太尉、汝南王悅辟為騎兵參軍。舊事,王出則騎兵武服持刀陪從。昕恥之,未嘗肯依行列。悅好逸遊,或馳騁信宿,昕輒棄還。悅乃令騎兵在前,手為驅策。昕捨轡高拱,任馬所之,左右言其誕慢。悅曰:「府望唯在此賢,不可責也。」悅數散錢於地,令諸佐爭拾之,昕獨不拾。悅又散銀錢以目昕,乃取其一。悅與府寮飲酒,起自移床,人爭進手,昕獨執板卻立。悅作色曰:「我帝孫,帝子,帝弟,帝叔,今親起輿床,卿何偃蹇?。對曰:「元景位望微劣,不足使殿下式瞻儀形,安敢以親王僚採,從廝養之役。」悅謝焉。坐上皆引滿
 酣暢;昕先起,臥於閑室,頻召不至。悅乃自詣呼之,曰:「懷其才而忽府主,可謂仁乎?」昕曰:「商辛沈湎,其亡也忽諸。府主自忽傲,寮佐敢任其咎?」悅大笑而去。後除著作佐郎。以兵亂漸起,將避地海隅。侍中李琰之、黃門侍郎王遵業惜其名士,不容外任,奏除尚書右外兵郎中。出為光州長史,故免河陰之難。遷東萊太守。于時年凶,人多相食,昕勤恤人隱,多所全濟。昕少時與河間邢邵俱為元羅賓友,及守東萊,邵舉室就之。郡人以邵是邢杲從弟,會兵將執之。昕以身蔽伏其上,呼曰:「欲執子才,當先執我。」邵乃免。



 太昌初,還洛。吏部尚書李神俊奏言:「比因
 多故,常侍遂無員限。今以王元景等為常侍,定限八員。」加金紫光祿大夫。武帝或時袒露,與近臣戲狎,每見昕,即正冠而斂容焉。昕體素甚肥,遭喪後,遂終身羸瘠。楊愔重其德素,以為人之師表。元象元年,兼散騎常侍,聘梁,魏收為副,並為朝廷所重。使還,高隆之求貨不得,諷憲臺劾昕、收在江東大將商人市易,並坐禁止。齊文襄營救之。累遷祕書監。



 昕雅好清言,詞無淺俗。在東萊時,獲殺其同行侶者,詰之未服。昕謂曰:「彼物故不歸,卿無恙而反,何以自明?」邢邵後見文襄,說此言以為笑樂。昕聞之,詣邵曰:「卿不識造化。」還謂人曰:「子才應死,我罵之
 極深。」頃之,以被謗,左遷陽平太守。在郡有稱績。文襄謂人曰:「王元景殊獲我力,由吾數戲之,其在吏事,遂為良二千石。」



 齊文宣踐阼,拜七兵尚書。以參議禮,封宜君縣男。嘗有鮮卑聚語,崔昂戲問昕曰:「頗解此不?」昕曰:「樓羅,樓羅,實自難解。時唱染於,似道我輩。」



 文宣以昕疏誕,非濟世才,罵:「好門戶,惡人身!」又有讒之者,云:「王元景每嗟水運不應遂絕。」帝愈怒,乃下詔曰:「元景本自庸才,素無勛行,早霑纓紱,遂履清途。發自畿邦,超居詹事。俄佩龍文之劍,仍啟帶礪之書。語其器分,何因到此?誠宜清心勵己,少酬萬一。尚書百揆之本,庶務攸歸。元景與奪
 任情,威福在己。能使直而為枉,曲反成絃。害政損公,名義安在?偽賞賓郎之味,好詠輕薄之篇。自謂模擬傖楚,曲盡風制。推此為長,餘何足取。此而不繩,後將焉肅?



 在身官爵,宜從削奪。」於是徙幽州為百姓。昕任運窮通,不改其操。未幾,徵還,奉敕送蕭莊於梁為主。除銀青光祿大夫,判祠部尚書。



 帝怒臨漳令嵇曄及舍人李文師,以曄賜薛豐洛,文師賜崔士順為奴。鄭子默私誘昕曰:「自古無朝士作奴。」昕曰:「箕子為之奴,何言無也?」子默遂以昕言啟文宣,仍曰:「王元景比陛下於紂。」楊愔微為解之。帝謂愔曰:「王元景是爾博士,爾語皆元景所教。」帝後與
 朝臣酣飲,昕稱疾不至。帝遣騎執之,見其方搖膝吟詠,遂斬於御前,投屍漳水。天統末,追贈吏部尚書。有文集二十卷。子顗嗣。



 卒於燕郡太守。



 昕母清河崔氏,學識有風訓。生九子,皆風流醖籍,世號王氏九龍。昕弟暉、昭、晞、皓最知名。



 暉字元旭,少與昕齊名,兼多術藝。卒於中書舍人,贈兗州刺史。



 昭字仲亮,少好儒術,又頗以武藝自許。性敦篤,以友悌知名。卒於考功郎中。



 晞字叔朗,小名沙彌。幼而孝謹,淹雅有器度。好學不倦。美容儀,有風則。



 魏末,隨母兄東適海隅,與邢子良遊處。子良愛其清悟,與其在洛兩兄書曰:「賢弟彌郎,意識深
 遠,曠達不羈。簡於造次,言必詣理。吟詠情性,麗絕當時。恐足下方難為兄,不暇慮其不進也。」



 魏永安初,第二兄暉聘梁,啟晞釋褐,除員外散騎侍郎,徵署廣平王開府功曹史。晞願養母,竟不受署。母終後,仍屬遷鄴,遨遊鞏、洛,悅其山水。與范陽盧元明、鉅鹿魏季景結侶同契,往天陵山,浩然有終焉之志。及西魏將獨孤信入洛,署為開府記室。晞稱先被犬傷,困篤,不赴。有故人疑其所傷非猘,書勸令赴。晞復書曰:「辱告存念,見令起疾。循復眷旨,似疑吾所傷未必是猘。吾豈願其必猘?



 但理契無疑耳。就足下疑之,亦有過說。足下既疑其非猘,亦可疑其
 是猘,其疑半矣。若疑其是猘而營護,雖非猘亦無損。疑非猘而不療,儻是猘則難救。然則過療則致萬全,過不療或至於死。若王晞無可惜也,則不足取;既取之,便是可惜。奈何奪其萬全,任其或死!且將軍威德所被飆飛霧襲,方掩八紘,豈在一介?若必從隗始,先須濟其生靈。足下何不從容為將軍言也?」於是方得見寬。俄而信返,晞遂歸鄴。



 齊神武訪朝廷子弟忠孝謹密者,令與諸子遊。晞與清河崔贍、頓丘李度、范陽盧正通首應此選。文襄時為大將軍,握晞等手曰:「我弟並向成長,志識未定,近善狎惡,不能不移。吾弟不負義方,卿祿位常亞召
 弟;若茍使回邪,致相詿誤,罪及門族,非止一身。」晞隨神武到晉陽,補中外府功曹參軍,帶常山公演友。



 齊天保初,行太原郡事。及文宣昏逸,常山王數諫。帝疑王假辭於晞,欲加大辟。王私謂晞曰:「博士,明日當作一條事,為欲相活,亦圖自全,宜深體勿怪。」



 乃於眾中杖晞二十。帝尋發怒,聞晞得杖,以故不殺,髡鞭鉗配甲坊。居三年,王又固諫爭,大被毆撻,閉口不食。太后極憂之。帝謂左右曰:「儻小兒死,奈我老母何!」於是每問王疾,謂曰:「努力彊食,當以王晞還汝。」乃釋晞令往。王抱晞曰;「吾氣息惙然,恐不復相見!」晞流涕曰:「天道神明,豈令殿下遂斃此舍。



 至尊親為人兄,尊為人主,安可與計?殿下不食,太后亦不食,殿下縱不自惜,不惜太后乎?」言未卒,王彊坐而飯。晞由是得免徒,還為王友。



 王復錄尚書事。新除官者必詣王謝職,去必辭。晞言於王曰:「受爵天朝,拜恩私第,自古以為干紀。朝廷文武,出入辭謝,宜一約絕。主上顒顒,賴殿下扶翼。」



 王深納焉。常從容謂晞曰:「主上起居不恒,卿耳目所具,吾豈可以前逢一怒,遂爾結舌。卿宜為撰諫草,吾當伺便極諫。」晞遂條十餘事以呈,因切諫王曰:「今朝廷乃爾,欲學介子匹夫,輕一朝之命,狂藥令人不自覺,刀箭豈復識親疏?一旦禍出理外,將奈殿下家業
 何!奈皇太后何!乞且將順,日慎一日。」王歔欷不自勝,曰:「乃至是乎!」明日見晞,曰:「吾長夜九思,今便息意。」便命火對晞焚之。



 後王承間苦諫,遂致忤旨。帝使力士反接伏,白刃注頸,罵曰:「小子何知,欲以吏才非我!是誰教汝?」王曰:「天下噤口,除臣誰敢有言?」帝催遣捶楚,亂杖數十。會醉臥得解。爾後褻黷之好,遍於宗戚,所往留連,俾畫作夜;唯常山邸多無適而去。



 及帝崩,濟南嗣立。王謂晞曰:「一人垂拱,吾曹亦保優閑。」因言:「朝廷寬仁慈恕,真守文良主。」晞曰:「天保享詐,東宮委一胡人。今卒覽萬機,駕馭雄傑。如聖德幼沖,未堪雙難,而使他姓出納詔命,必權
 有所歸。殿下雖欲守籓職,其可得也?假令得遂沖退,自審家祚得保靈長不?」王默然,思念久之,曰:「何以處我?」晞曰:「周公抱成王朝諸侯,攝政七年,然後復子明辟。幸有故事,惟殿下慮之。」王曰:「我安敢自擬周公?」晞曰:「殿下今日地望,欲避周公得邪?」



 王不答。帝臨發,敕王從駕,除晞并州長史。



 及王至鄴,誅楊、燕等。詔以王為大丞相、都督中外諸軍事,督攝文武還并州。



 及至,延晞謂曰:「不早用卿言,使群小弄權,幾至傾覆。今君側雖獲暫清,終當何以處我?」晞曰:「殿下將往時地位,猶可以名教出處。今日事勢,遂關天時,非復人理所及。」有頃,奏趙郡王睿為左
 長史,晞為司馬。每夜載入,晝則不與語,以晞儒緩,恐不允武將之意。後進晞密室,曰:「比王侯諸貴每見煎迫,言我違天不祥,恐當或有變起,吾正欲以正法繩之。」晞曰:「朝廷比者疏遠親戚,寧思骨血之重。殿下倉卒所行,非復人臣之事。芒刺在背,交戟入頸,上下相疑,何由可久?且天道不恒,虧盈迭至,神機變化,肸蠻斯集。雖執謙挹,粃糠神器。便是違上玄之意,墜先人之基。」王曰:「卿何敢須發非所宜言!須致卿於法。」晞曰:「竊謂天時人事,同無異揆。是以冒犯雷霆,不憚斧鉞。今日得披肝膽,抑亦神明攸贊。」王曰:「拯難匡時,方俟聖哲,吾何敢私議,幸勿多
 言。」尋有詔,以丞相任重,普進府寮一班,晞以司馬領吏部郎中。丞相從事中郎陸杳將出使,臨別,握晞手曰:「相王功格區宇,天下樂推,歌謠滿道,物無異望。杳等伏隸,願披赤心。而忽奉外使,無由面盡短誠,寸心謹以仰白。」晞尋述杳言。王曰:「若內外咸有異望,趙彥深朝夕左右,何因都無所論?自以卿意試密與言之。」晞以事隙問彥深。曰:「我比亦驚此音謠,每欲陳聞,則口噤心戰。弟既發論,吾亦欲昧死一披肝膽。」因亦同勸。是時諸王公將相日敦請,四方岳牧表陳符命。乾明元年八月,昭帝踐阼。九月,除晞散騎常侍,仍領兼吏部郎中。



 後因奏事罷,帝
 從容曰:「比日何為自同外客,略不可見?自今假非局司,但有所懷,隨宜作一牒,候少隙即徑進也。」因敕尚書陽休之、鴻臚卿崔勵等三人,每日本職務罷,並入東廊。共舉錄歷代廢禮墜樂,職司廢置,朝饗異同,輿服增損,或道德高俊久在沈淪,或巧言眩俗,妖邪害政,爰及田市舟車、征稅通塞、婚葬儀軌、貴賤等衰,有不便於時而古今行用不已者,或自古利用而當今毀棄者,悉令詳思,以漸條奏。未待頓備,遇憶續聞。朝晡給典御食,畢景聽還。時百官請建東宮,敕未許,每令晞就東堂監視太子冠服,導引趨拜。尋拜為太子太傅。晞以局司奉璽授皇
 太子。太子釋奠,又兼中庶子。帝謂曰:「今既當劇職,不得尋常舒慢也。」



 帝將北征,敕問:「比何所聞?」晞曰:「道路傳言,車駕將行。」帝曰:「庫莫奚南侵,我未經親戎,因此聊欲習武。」晞曰:「鑾駕巡狩,為復何爾?若輕有征戰,恐天下失望。」帝曰:「此懦夫常慮,吾自當臨時斟酌。」帝使齋帥裴澤、主書蔡暉伺察群下,好相誣枉,朝士呼為裴、蔡。時二人奏:「車駕北征後,陽休之、王晞數與諸人遊宴,不以公事在懷。」帝杖休之、晞脛各四十。帝斬人於前,問晞曰:「此人合死不?」晞曰:「罪實合死,但恨其不得死地。臣聞刑人於市,與眾棄之;殿廷非殺戮之所。」帝改容曰:「自今當為王公
 改之。」



 帝欲以晞為侍中,苦辭不受。或勸晞勿自疏,晞曰:「我少年以來,閱要人多矣。充詘少時,鮮不敗績。且性實疏緩,不堪時務。人主恩私,何由可保?萬一披猖,求追無地。非不愛作熱官,但思之爛熟耳。」百官嘗賜射,晞中的,當得絹,為不書箭,有司不與。晞陶陶然曰:「我今段可謂武有餘文不足矣。」晞無子,帝將賜之妾。使小黃門就宅宣旨,皇后相聞晞妻。晞令妻答,妻終不言,晞以手撩胸而退。帝聞之笑。



 孝昭崩,晞哀慕殆不自勝,因以羸敗。武成本忿其儒緩,由是彌嫌之。因奏事,大被訶叱,而雅步晏然。歷東徐州刺史、祕書監。武平初,遷大鴻臚,加儀同
 三司,監修起居注,待詔文林館。性閑淡寡欲,雖王事鞅掌,而雅操不移。在并州,雖戎馬填閭,未嘗以世務為累。良辰美景,嘯詠遨遊,登臨山水,以談宴為事,人士謂之「方外司馬。」詣晉祠,賦詩曰:「日落應歸去,魚鳥見留連。」忽有相王使召,晞不時至。明日,丞相西閣祭酒盧思道謂晞曰:「昨被召已朱顏,得無以魚鳥致怪?」



 晞緩笑曰:「昨晚陶然,頗以酒漿被責。卿輩亦是留連之一物,豈直在魚鳥而已?」



 及晉陽陷敗,與同志避周兵東北走。山路險迥,懼有土賊,而晞溫酒服膏,曾不一廢。每不肯疾去,行侶尤之。晞曰:「莫尤我,我行事若不悔,久作三公矣。」



 齊亡,周
 武帝以晞為儀同大將軍、太子諫議大夫。隋開皇元年,卒於洛陽,年七十一。贈儀同三司、曹州刺史。



 皓字季高,少立名行,為士友所稱。遭母憂,居喪有至性。儒緩亦同諸兄。嘗從文宣北征,乘赤馬,旦蒙霜氣,遂不復識。自言失馬,虞候為求覓不得。須臾日出,馬體霜盡,繫在幕前,方云:「我馬尚在。」為司徒掾,在府聽午鼓,蹀躞待去。群寮嘲之曰:「王七思歸何太疾?」季高曰:「大鵬始欲舉,燕雀何啾唧?」



 嘲者曰:「誰家屋當頭,鋪首浪遊逸。」於是喧笑,季高不復得言。大寧初,兼散騎常侍、聘陳使主。天統末,修國史。尋除通直散騎常侍。卒,贈郢州刺史。子伯,
 奉朝請,待詔文林館。皓弟曄,字季炎,卒於滄州司馬。



 封懿,字處德,勃海蓚人也。曾祖釋,晉東夷校尉。父放,慕容吏部尚書。



 兄孚,慕容超太尉。懿有才器,能屬文,與孚雖器行有長短,而名位略齊。仕慕容寶,位中書令、戶部尚書。寶敗,歸魏,除給事黃門侍郎、都坐大官、章安子。道武引見,問以慕容舊事,懿應對疏慢,廢黜還家。明元初,復徵拜都坐大官,進爵為侯。卒官。懿撰《燕書》,頗行於世。



 子玄之,坐與司馬國璠、溫楷等謀亂,伏誅。臨刑,明元謂曰:「終不令絕汝種也,將宥汝一子。」玄之以弟虔之子磨奴字君明早孤,乞全其命。乃殺玄之四子,赦磨奴,刑
 為宦人。崔浩之誅也,太武謂磨奴曰:「汝本應全,所以致刑者,由浩也。」後為中曹監,使張掖,賜爵富城子。卒於懷州刺史,贈勃海公,謚曰定。以族子叔念為後。



 回字叔念,孝文賜名焉,慕容太尉奕之後也。父鑒。初,磨奴既以回為後,請於獻文。贈鑒寧遠將軍、滄水太守。回襲靡奴爵富城子。宣武時,累遷安州刺史。



 山人愿朴,父子賓旅同寢一室。回下車,勒令別處,其俗遂改。明帝時,為瀛州刺史。時大乘寇亂之後,加以水潦,表求振恤,免其兵調,州內賴之。歷度支、都官二尚書、冀州大中正。



 滎陽鄭雲諂事長秋卿劉騰,貨紫纈四百匹,得為安州
 刺史。除書旦出,晚往詣回,坐未定,問回:「安州興生,何事為便?」回曰:「卿荷國寵靈,位至方伯,雖不能拔園葵,去織婦,宜思方略以濟百姓,如何見造問興生乎?封回不為商賈,何以相示?」雲慚失色。



 轉七兵尚書,領御史中尉,劾奏尚書右僕射元欽與從兄麗妻崔氏姦通,時人稱之。後為殿中尚書、右光祿大夫。莊帝初,遇害河陰。贈司空公,謚曰孝宣。長子隆之。



 隆之字祖裔,小名皮,寬和有度量。延昌中,道人法慶作亂冀州,自號大乘,眾五萬人。隆之以開府中兵參軍與大都督元遙討之。獲法慶,賜爵武城子。累遷河內太守。
 未到郡,屬爾朱兆入洛,莊帝幽崩,隆之以父遇害,常懷報雪,因持節東歸,圖為義舉。遂與高乾等夜襲冀州,克之,乃推為刺史。及齊神武自晉陽東出,隆之遣子子繪隨高乾奉迎於滏口。



 中興初,拜吏部尚書。韓陵之役,留隆之鎮鄴城。未幾,徵為侍中,封安德郡公。于時朝議以爾朱榮宜配食明帝廟庭。隆之議曰:「榮為人臣,親行殺逆,豈有害人之母而與子對食之理?」以參議麟趾閣新制,又贈其妻祖氏范陽郡君。隆之表以先爵富城子及武城子轉授弟子孝琬等,朝廷嘉而從之。後為斛斯椿等所構,逃歸鄉里,齊神武召赴晉陽。



 魏孝靜立,除吏部
 尚書,尋加侍中。元象初,除冀州刺史,加開府,累遷尚書右僕射。及北豫州刺史高仲密將叛,陰招冀州豪望為內應。詔隆之馳驛慰撫,遂得安靜。隆之首參神武經略,奇謀皆密以啟聞,手書削槁,罕知於外。卒於齊州刺史,贈司徒。神武以追榮未盡,復啟贈太保,謚宣懿。神武後至冀州北境,次交津,追憶隆之,顧冀州行事司馬子如,言其德美,為之流涕。令以太牢就祭。隆之歷事五帝,以謹素見知。凡四為侍中,再為吏部尚書,一為僕射,四為冀州刺史。每臨冀部,州中舊齒咸曰:「我封公復來。」其得物情如此。子子繪嗣。



 子繪字仲藻,小名搔。性和理,有器
 局。釋褐秘書郎,累遷平陽太守,加散騎常侍。晉州北界霍山舊號千里徑者,山阪高峻,每大軍往來,士馬勞苦。子繪請於舊徑東谷別開一路。神武從之,仍令子繪修開,旬日而就。徵補大行臺吏部郎中。



 神武崩,祕未發喪,文襄以子繪為勃海太守。執其手曰:「誠知未允勛臣官望,但須鎮撫。且衣錦晝遊,古人所貴,宜善加經略,不勞習常太守向州參也。」仍聽收集部曲一千人。



 大寧三年,為都官尚書。高歸彥作逆,命子繪參贊軍事。賊平,敕子繪權行州事。徵拜儀同三司、尚書右僕射。卒,謚曰簡。子寶蓋襲。



 子繪弟子繡,位霍州刺史。陳將吳明徹侵淮南,
 子繡城陷,送揚州。齊亡後,逃歸。終於通州刺史。子繡外貌儒雅,而使氣難犯。兄女婿司空婁定遠為瀛州刺史,子繡為勃海太守。定遠過之,對妻及諸女宴集言戲,微有褻慢。子繡鳴鼓集眾將攻之,定遠免冠拜謝,久之乃釋。



 隆之弟興之,字祖胄。經明行修,恬素清靜。位瀛冀二州刺史、平北府長史。



 所歷有當官譽。卒,以隆之佐命功,贈殿中尚書、雍州刺史,謚曰文。



 子孝琬,字士茜。七歲而孤,為隆之鞠養;慈愛甚篤,隆之啟以父爵富城子授焉。位東宮洗馬。卒,贈太府少卿。



 孝琬性恬靜,頗好文詠。太子少師邢邵、七兵尚書王昕並先達高才,與孝琬年位
 懸隔,晚相逢遇,分好遂深。孝琬靈櫬言歸,二人送於郊外,悲哭悽慟,有感路人。



 孝琬弟孝琰,字士光,少修飭,學尚有風儀。位秘書丞、散騎常侍、聘陳使主,在道遙授中書侍郎。還,坐受魏收囑,牒其門客從行事發,付南都獄,決鞭二百,除名。後除並省吏部郎中、南陽王友,赴晉陽典機密。



 和士開母喪,託附者咸往奔哭。鄴中富商丁鄒、嚴興等並為義孝,有一士人亦在哭限。孝琰入弔,出謂人曰:「嚴興之南,丁鄒之北,有一朝士,號叫甚哀。」



 聞者傳之。士開知而大怒。其後會黃門郎李瑰奏南陽王綽驕恣,士開因譖之曰:「孝琰從綽出外,乘其副馬,捨離部伍,
 別行戲語。」時孝琰女為范陽王妃,為禮事,因假入辭。帝遂決馬鞭一百放出,又遣高阿那肱重決五十,幾死。還鄴,在集書省上下。自此沈廢。士開死後,為通直散騎常侍。後與周和好,以為聘周使副。



 祖珽輔政,奏入文林館撰御覽。



 孝琰文筆不高,但以風流自立,善談戲,威儀閑雅,容止進退,人皆慕之。以祖珽好自矜大,佞之云:「是衣冠宰相,異於餘人。」近習聞之,大以為恨。尋以本官兼尚書右丞。其所彈射,多承意旨。時有道人曇獻者,為皇太后所幸,賞賜隆厚,車服過度。又乞為沙門統,後主意不許,但太后欲之,遂得居任。然後主常憾焉。因僧尼他事,
 訴者辭引曇獻,上令有司推劾。孝琰案其受賄,致於極法,其家珍異悉以沒官。由是正授左丞,仍奏門下事。



 性頗簡傲,不諧時俗,意遇漸高,彌自矜誕,舉動舒遲,無所降屈,識者鄙之。



 與崔季舒等以正諫同死。子君確、君靜二人徙北邊,少子君嚴、君贊下蠶室。南安敗,君確等二人皆坐死。



 興之弟延之,字祖業,少明辯,有世用。封郯城子,位青州刺史,多所受納。



 後行晉州事。沙苑之敗,延之棄州北走,以隆之故,免其死。卒,贈尚書左僕射、司徒公,謚文恭。子纂嗣。



 鑒長子琳,字彥寶,位中書侍郎。與侍中、南平王馮誕等議定律令,有識者稱之。歷位太尉長史、
 司宗下大夫、南夏青二州刺史、光祿大夫。琳弟子肅。



 肅字元邕,博涉經史。太傅崔光見而賞焉。位尚書左中兵郎中。性恭儉,不妄交游,唯與崔勵、勵從兄鴻尤相親善。所制文章多亡失,存者十餘卷。



 懿從兄子愷,字思悌,奕之孫也。父勸,慕容垂侍中、太常卿。愷位給事黃門侍郎、散騎常侍。後入代都,名出懿子玄之右。俱坐司馬氏事死。愷妻,盧玄女也。



 愷子伯達,棄母及妻李氏南奔河表,改婚房氏。獻文末,伯達子休傑內入。祖母盧猶存,垂百歲矣。而李已死。休傑位冀州咸陽王府諮議參軍。



 回族叔軌,字廣度。好學,通覽經傳。與光祿大夫武邑孫
 惠蔚同志友善。惠蔚每推軌曰:「封生之於經義,吾所弗如者多矣。」頗自修潔,儀容甚偉。或曰:「學士不事修飾,此賢何獨如此?」軌聞,笑曰:「君子整其衣冠,尊其瞻視,何必蓬頭垢面而後為賢。」言者慚退。以兼員外散騎常侍銜命高麗。高麗王雲恃其偏遠,稱疾不親受詔。軌正色詰之,喻以大義,雲乃北面受旨。使還,轉考功郎中,除本郡中正。勃海太守崔休入為吏部郎中,以兄考事干軌。軌曰:「法者天下之事,不可以舊君故,虧之也。」休歎其守正。軌在臺中,稱為儒雅。除國子博士,假通直散騎常侍,慰勞汾州山胡。



 司空、清河王懌表修明堂、辟雍,詔百寮集
 議。軌議曰:《周官匠人》職云:夏后氏世室,殷人重屋,周人明堂,五室,九階,四戶,八窗。鄭玄曰:「或舉宗廟,或舉王寢,或舉明堂,互文以見同制。」然則三代明堂,其制一也。案周與夏、殷,損益不同。至於明堂,因而弗革,明五室之義,得天數矣。是以鄭玄又曰:「五室者,象五行也。」然則九階者法九土,四戶者達四時,八窗者通八風,誠不易之大範,有國之恒式。若其上圓下方以則天地,通水環宮以節觀者,茅蓋白盛為之質飾,赤綴白綴為之戶牖,皆典籍所載,制度之明義也。



 秦焚滅五典,非毀三代,變更先聖,不依舊憲。故《呂氏月令》見九室之義,大戴之《禮》著十
 二堂之文。漢承秦法,亦未能改,東西二京,俱為九室。是以《黃圖》、《白武通》、蔡邕、應劭等咸稱九室以象九州,十二堂以象十二辰。夫室以祭天,堂以布政。依行而祭,故室不過五;依時布政,故堂不踰四。州之與辰,非所可法。



 九與十二,厥用安在?今聖朝欲尊道訓人,備禮化物,宜則五室,以為永制。至如廟學之嫌,臺沼之雜,袁準之徒已論正矣。



 後卒於廷尉少卿。贈濟州刺史。



 初,軌深為郭祚所知,祚堂謂子景尚曰:「封軌、高綽二人,並幹國之才,必應遠至。吾平生不妄進舉,而每薦此二人,非直為國進賢,亦為汝等之津梁。」其見重如此。軌既以方直自業,高
 綽亦以風概立名。高肇拜司徒,綽送迎往來,軌竟不詣。綽顧不見軌,乃遽歸曰:「吾一生自謂無愆規矩,今日舉措不如封生遠矣。」



 軌以務德慎言,脩身之本,姦回讒佞,世之巨害。乃為《務德》、《慎言》、《遠佞》、《防姦》四戒。文多不載。



 長子偉伯,字君良,博學有才思。弱冠,除太學博士。每朝廷大議,偉伯參焉。



 雅為太保崔光、僕射游肇所知賞。太尉、清河王懌辟參軍事。懌親為《孝經解詁》,命偉伯為難例九條,皆發起隱漏。偉伯又討論《禮》、《傳》、《詩》、《易》疑事數十條,儒者咸稱之。時朝廷將經始明堂,廣集儒學,議其制度,九五之論,久而不定。偉伯乃搜檢經、緯,上《明堂圖說》六
 卷。又撰《封氏本錄》六卷。



 正光末,尚書僕射蕭寶夤為關西行臺,引為行臺郎。及寶夤為逆,偉伯與南平王固潛結關中豪右韋子粲等,謀舉義兵。事發,見殺。永安中,贈瀛州刺史,聽一子出身,無子,轉授弟翼。翼弟述。



 述字君義,有乾用。天平中,為三公郎中。時增損舊事,為《麟趾新格》,其名法科條皆述所刪定。齊受禪,累遷大理卿。河清三年,敕與錄尚書趙彥深、僕射魏收、尚書陽休之、國子祭酒馬敬德等議定律令。歷位度支、五兵、殿中三尚書。



 述久為法官,明解律令,議斷平允,深為時人所稱。而厚積財產,一無分饋。



 雖至親密友,貧病困篤,亦絕
 於拯濟。朝野物論甚鄙之。外貌方整,而不免請謁,回避進趣,頗致嗤駭。前妻河內司馬氏。一息為娶隴西李士元女,大輸財聘。及將成禮,猶競懸違。述忽取所供養像,對士元打像為誓。士元笑曰:「封公何處常得應急像,須誓便用?」一息娶范陽盧莊之女,述又經府訴云:「送騍乃嫌腳跛,評田則云鹹薄,銅器又嫌古廢。」皆為吝嗇所及,每致紛紜。子元茜,位太子舍人。



 述弟詢,字景文,窺涉經史,以清素自持。位尚書左丞、濟南太守。歷官皆有幹局才具,臨郡甚著聲績。隋開皇中卒。



 論曰:崔逞文學器識,當年之俊,忽微慮遠,俱以為災。休
 立身有本,當官著稱。長儒才望之美,禍因驕物,雖有周公之才,猶且為累。況未足諭其高下,能無及乎?贍詞韻溫雅,風神秀發,固人望也。王憲名公之孫,老見優異。元景昆季履道,標映人倫,美哉!封回克光家世,隆之勤勞霸業,子繪實隆堂構,可謂載德者矣。君義聚斂嗇吝,無乃鄙哉!



\end{pinyinscope}