\article{卷八十六列傳第七十四 循吏}

\begin{pinyinscope}

 張膺路邕閻慶胤明亮杜纂竇瑗蘇淑張華原孟業蘇瓊路去病梁彥光樊叔略公孫景茂辛公義柳儉郭絢敬肅劉曠王伽魏德深先王疆理天下,司牧黎元,刑法以禁其姦,禮教以防其
 欲,雖為政以德,理實殊塗,百慮一致,在斯而已。《書》云「知人則哲」,又云「無曠庶官」,言非其人為空官也。睿哲之后,必致清明之臣;昏亂之朝,多有貪殘之吏。嗜欲所召,影響從之。故五帝三王,不易人而化,皆在所由化之而已。蓋有無能之吏,無不可御之人焉。自罷侯置守,歷年永久,統以方牧,仍世相循,所以寬猛為用,庇人調俗。



 但廉平常迹,聲有難高;適時應務,招響必速。是故搏擊為侯,起不旋踵;懦弱貽咎,錄用無時。此則已然於前世矣!後之為吏,與世沈浮,叔季澆漓,姦巧多緒,居官蒞職,道各不同,故往籍述其賢能,以彰懲勸之道。



 案魏立《良吏傳》
 有張恂、鹿生、張膺、宋世景、路邕、閻慶胤、明亮、杜纂、裴佗、竇瑗、羊敦、蘇淑。齊立《循吏傳》有張華原、宋世良、郎基、孟業、崔伯謙、蘇瓊、房豹、路去病。《周書》不立此篇。隋《循吏傳》有梁彥光、樊叔略、趙軌、房恭懿、公孫景茂、辛公義、柳儉、劉曠、王伽、魏德深。其張恂、鹿生、宋世景、裴佗、羊敦、宋世良、郎基、崔伯謙、房豹、趙軌、房恭懿,各附其家傳,其餘皆依時代編緝,以備《循吏篇》云。



 張膺,不知何許人也。延興中,為魯郡太守。履行貞素,妻女樵採自供。孝文深嘉之。遷京兆太守,清白著稱,得吏人之忻心焉。



 路邕,陽平人也。宣武時,除東魏郡太守。蒞政清勤。經年儉,日出家粟,賑賜貧窘。靈太后下詔褒美,賜龍廄馬一匹、衣一襲、被褥一具。稍遷南青州刺史,卒。



 閻慶胤,不知何許人也。為東秦州敷城太守。頻年饑儉,慶胤歲常以家粟千石,賑恤貧窮,人賴以濟。部人陽寶龍一千餘人申頌美政。有司以聞,靈太后卒無褒賞。



 明亮,字文德,平原高昌人也。有識幹,歷員外常侍。延昌中,宣武臨朝堂,親自黜陟,授亮勇武將軍。亮進曰:「臣本官常侍,是第三清;今授臣勇武,其號至濁。且文武又殊,請更改授。」帝曰:「九流之內,人咸君子,卿獨欲乖眾,妄相
 清濁,所請未可。」亮曰:「今江左未賓,書軌宜一,方為陛下投命前驅,拓定吳會。官爵,陛下之所輕;賤命,微臣之所重。陛下方收所重,何惜所輕?」因請改授平遠將軍。帝曰:「運籌用武,然後遠人始平。卿但用武平之,何患不得平遠乎?」亮乃陳謝而退。除陽平太守。清白愛人,甚有惠政。轉汲郡太守,為政如前,舉宣遠近。卒,二郡人吏迄今追思之。



 杜纂,字榮孫,常山九門人也。少以清苦自立。時縣令齊羅喪亡,無親屬收殮,纂以私財殯葬,由是郡縣標其門閭。後居父喪盡禮。郡舉孝廉,稍除積弩將軍,從征新野。
 及南陽平,以功賜爵井陘男。賞帛五百匹,數日之中,散之知友,時人稱之。歷武都、漢陽二郡太守,並以清白為名。明帝初,拜清河內史。性儉約,尤愛貧老,問人疾苦,至有對之泣涕。勸督農桑,親自檢視,勤者賞以物帛,惰者加以罪譴。弔死問生,甚有恩紀。除東益州刺史,無御邊威略,群氐反叛,以失人和徵還。遷太中大夫。正光末,清河人房通等三百人頌纂德政,乞重臨郡,詔許之。孝昌中,為葛榮圍逼。以郡降,榮以為常山太守。榮滅,卒於家。



 纂所歷任,好行小惠,蔬食弊衣,多涉誣矯。而輕財潔己,終無受納,為百姓所思,號為良守。天平中,贈定州刺史。



 竇瑗,字世珍,遼西陽洛人也。自言本出扶風平陵,漢大將軍武曾孫崇為遼西太守,遂家焉。曾祖堪,慕容氏漁陽太守。祖表,馮弘城周太守,入魏。父冏,舉秀才,早卒。普泰初,瑗啟以身階級為父請贈,詔贈平州刺史。瑗年十七,便荷帙從師,遊學十載,始為御史。後兼太常博士,拜太原王爾朱榮官,榮留為北道大行臺左丞。以拜榮官,賞新昌男。從榮東平葛榮,封容城縣伯。瑗乞以容城伯讓兄叔珍,詔聽以新昌男轉授之。叔珍由是位至太山太守。爾朱世隆等立長廣王曄為主,南赴洛陽。至東郭外,世隆等遣瑗奏廢之,瑗執鞭獨入禁內,奏願行堯、舜
 事,曄遂禪廣陵。由是除給事黃門侍郎。



 孝武帝時,為廷尉卿。及釋奠開講,瑗與溫子昇、魏季景、李業興並為擿句。



 天平中,除廣宗太守,政有清白之稱。廣宗人情凶戾,累政咸見告訟。唯瑗一人,終始全潔。轉中山太守,聲譽甚美,為吏人所懷。及齊神武班書州郡,稱瑗政績,以為勸勵。後授平州刺史,在州政如臨郡。又為神武丞相府右長史。瑗無軍府斷割才,不甚稱職。又行晉州事。及還鄴,上表曰:「臣伏讀《麟趾新制》至三公曹第六十六條:『母殺其父,子不得告,告者死。』三返覆之,未得其門。何者?案律:『子孫告父母、祖父母者,死。』又漢宣云:『子匿父,孫慝大父母,皆
 勿論。』蓋謂父母、祖父母小者攘羊,甚者殺害之類,恩須相隱,律抑不言,法理如是,足見其直,未必指母殺父,止子不言也。今母殺父而子不告,便是知母而不知父,識比野人,義近禽獸。且母之於父,作合移天,既殺己之天,復殺子之天,二天頓毀,豈容頓默?此母之罪,義在不赦;下手之日,母恩即離。仍以母道不告,鄙臣所以致惑。如或有之,可臨時議罪,何用豫制斯條,用為訓誡?恐千載之下,談者喧譁,以明明大朝,有尊母卑父之論。以臣管見,實所不取。」詔付尚書。三公郎封君義立判云:「母殺其父,子復告母,母由告死,便是子殺。天下未有無母之國,
 不知此子,將欲何之?既於法無違,於事非害,宣布有司,謂不宜改。」瑗復難云:「局判云『母由告死,便是子殺。天下未有無母之國,不知此子,將欲何之。』瑗案典律,未聞母殺其父而子有隱母之義。既不告母,便是與殺父同。天下可有無父之國,此子獨得有所之乎?」事雖停寢。除大宗正卿。宗室及其寒士,相與輕之,瑗案法推正,甚見仇疾。官雖通顯,貧窘如初,清尚之操,為時所重。領本州大中正,兼廷尉卿,卒官。贈太僕卿、濟州刺史,謚曰明。



 蘇淑,字仲和,武邑人也。兄壽興,坐事為閹官,後拜河間太守,賜爵晉陽男。



 及壽興將卒,遂冒養淑為子。淑熙平
 中襲其爵。後除樂陵內史,在郡綏撫,甚有人譽。後謝病乞解,有詔聽之,人吏老幼訴乞淑者甚眾。後歷滎陽、中山二郡太守,卒。



 淑清心愛下,所歷三郡,皆為吏人所思,當時稱為良二千石。武定初,贈衛大將軍、都官尚書、瀛州刺史,謚曰懿。齊神武追美清操,與羊敦同見優賞。



 張華,原字國滿,代郡人也。少明敏,有器度。初為齊神武驃騎府法曹參軍,賜爵新城伯,累遷大丞相府屬。深被親待,每號令三軍,恆令宣諭意旨。尋除散騎常侍。周文始據雍州,神武使華原入關說焉。周文謂曰:「若能屈驥足於此,當共享富貴;不爾,命懸今日。」華原曰:「殞首而已,
 不敢聞命。」周文嘉其亮正,乃使東還。尋悔,遣追不及。神武以華原久而不返,每歎惜之,及聞其來,喜見於色。後除相府右長史,遷驃騎大將軍、特進,進爵為公,仍徙封新安。後為兗州刺史。華原有幹略,達政體。至州,乃廣布耳目,以威禁。境內大賊及鄰州亡命三百餘人,皆詣華原歸款。咸撫以恩信,放歸田里,於是人懷感附,寇盜寢息。州獄先有繫囚千餘人,華原科簡輕重,隨事決遣。至年暮,唯有重罪者數十人。華原各給假五日,曰:「期盡速還也。」囚等曰:「有君如是,何忍背之!」依期畢至。先是,州境數有猛獸為暴。自華原臨政,州東北七十里甑山中,忽
 有六駮食猛獸,咸以為化感所致。卒官,州人大小莫不號慕,為樹碑立祠,四時祭焉。贈司空公、尚書左僕射。子宰均嗣。



 孟業,字敬業,鉅鹿安國人也。家本寒微,少為州吏,性廉謹。同僚諸人,侵盜官絹,分三十匹與業,拒而不受。行臺郎中郭秀相禮接,方欲薦之,會秀卒。



 魏彭城王韶,齊神武之婿也,拜定州刺史,除業為典簽。長史劉仁之謂業曰:「我處其外,君居其內,同心戮力,庶有濟乎?」未幾,仁之入為中書令,臨路啟韶云:「殿下左右可信任者,唯有孟業,願專任之,餘人不可信也。」又與業別,執手曰:「令我出
 都,君便失援,恐君在後,不自保全,唯正與直,願君自勉。」



 業唯有一馬,瘦死。韶以業貧,令州府官人,同食馬肉,欲令厚相酬償。業固辭不敢。韶乃戲業曰:「卿邀名人也。」對曰:「業為典簽,州中要職,諸人欲相賄贍,止患無方便耳。今喚食肉,恐致聚斂,有損聲名,所以仰違明教。」後未旬日,韶左右王四德、董惟金並以馬死託肉,為長史裴英起密啟。神武有書與韶,大致誚讓。



 業尋被譖,出外行縣事。後神武書責韶云:「典簽姓孟者,極能用心,何乃令出外也!」及韶代下,業亦隨還,贈送一無所受。仁之後為西兗州,臨別謂吏部郎中崔暹曰:「貴州人士,唯有孟業,銓舉
 之次,不可忘也。」暹問業曰:「君往在定州,有何政,使劉西兗如此欽嘆?」業答曰:「唯知自脩也。」韶為并州刺史,業復為典簽,仍兼長史。



 齊天保初,清河王岳拜司州牧,召為法曹。業形貌短小,及謁見,岳心鄙其眇小,笑而不言。後尋業斷決處,謂曰:「卿斷決之明,可謂有過軀貌之用。」補河間王國郎中令。清貧自守,未曾有失。文宣謂侍中裴英起曰:「卿識河間王郎中孟業不?一昨見其國司文案,似是好人。」對曰:「昔與臣同事魏彭城王元韶。其人清忠正直,世所希有。」帝曰:「如公言者,比來便是大屈。」除中書舍人。文宣初唯得姓名,及因奏事,見其羸老,又質性敦
 樸,無升降之容,加之平緩,寡於方便。有一道士由吾道榮以術藝被迎,將入內,業為通名。忽於眾中抗聲奏云:「由吾道士不食五穀。」帝命推而下之。又令點檢百官,敷奏失所,帝遣人以馬鞭擊業頭,至於流血。然亦體其衰老,非力所堪。



 皇建二年,累遷東郡太守,以寬惠著名。其年夏,五官張凝因出使,得麥一莖五穗,其餘或三穗四穗共一莖者,合郡咸以政化所感,因即申上。至秋,復有東燕縣人班映祖,送嘉禾一莖九穗。河清三年,敕人間養驢,催買甚切。業曰:「吾既為人父母,豈可坐看此急。令宜權出庫錢,貸人取辦,後日有罪,吾自當之。」後為憲司
 所劾。被攝之日,郡人皆泣而隨之,迭相弔慰。送業度關者,有數百人,至黎陽郡西,方得辭決。攀援號哭,悲動行路。詣闕訴冤者非一人,敕乃放還。郡中父老,扣河迎接。



 武成親戎,自洛還鄴,道由東郡。業具牛酒,率人吏拜謁路旁,自稱:「糞土臣孟業,伏惟聖駕親行,有征無戰,謹上微禮。」便與人吏俱唱萬歲,導引前入,帝大嘉之。後除廣平太守,年既老,理政不如在東郡時。武平九年,為太中大夫,加衛將軍,尋卒。



 業志守質素,不尚浮華。為子結婚,為朝肺腑吒羅氏。其子以蔭得為平原王段孝先相府行參軍,乃令作今世服飾綺襦紈褲。吒羅家又恃姻婭,
 炫曜矜誇。業知而不禁,素望頗貶。



 蘇瓊,字珍之,長樂武強人也。父備,仕魏,至衛尉少卿。瓊幼時隨父在邊,嘗謁東荊州刺史曹芝,芝戲問曰:「卿欲官不?」對曰:「設官求人,非人求官。」



 芝異其對,署為府長流參軍。齊文襄以儀同開府,引為刑獄參軍,每加勉勞。并州嘗有強盜,長流參軍張龍推其事,所疑賊徒,並已拷伏,失物家並識認,唯不獲盜贓。文襄付瓊,更令窮審,乃別推得元景融等十餘人,并獲贓驗。文襄大笑,語前妄引賊者曰:「爾輩若不遇我好參軍,幾致枉死。」除南清河太守。郡多盜賊,及瓊至,姦盜止息。或外境姦非,輒從界
 中行過者,無不捉送。零陵縣人魏雙成,住處與畿內武城交錯,失牛,疑其村人魏子賓,列送至郡。一經窮問,知賓非盜,而便放之。雙成云:「府君放賊去,百姓牛何處可得?」瓊不理其語,密遣訪獲盜者。



 從此畜牧不收,云:「但存府君。」其鄰郡富家,將財物寄置界內以避盜。冀州繹幕縣人成氏大富,為賊攻急,告曰:「我物已寄蘇公矣」,賊遂去。平原郡有妖賊劉黑茍,構結徒侶,通於滄海。瓊所部人,連接村居,無相染累。鄰邑於此伏其德績。郡中舊賊一百餘人,悉充左右,人間善惡及長吏飲人一盃酒,無不即知。



 瓊性清慎,不發私書。道人道研為濟州沙門統,
 資產巨富,在郡多出息,常得郡縣為征。及欲求謁,度知其意,每見則談問玄理。研雖為債數來,無由啟口。其弟子問其故,研曰:「每見府君,徑將我入青雲間,何由得論地上事。」師徒還歸,遂焚責券。郡人趙潁,官至樂陵太守,年餘八十,致事歸。五月中,得新瓜一雙,自來奉。潁恃年老,苦請,遂便為留。乃致於事梁上,竟不割。人聞受趙潁餉瓜,欲貢新果,至門,問知潁瓜猶在,相顧而去。有百姓乙普明,兄弟爭田,積年不斷,各相援據,乃至百人。瓊召普明兄弟,對眾人諭之曰:「天下難得者兄弟,易求者田地。假令得地失兄弟心,如何?」因而下淚,諸證人莫不
 灑泣。普明兄弟叩頭,乞外更思,分異十年,遂還同住。



 每年春,總集大儒衛覬隆、田元鳳等講於郡學,朝吏文案之暇,悉令受書。時人指吏曹為學生屋。禁斷淫祠,婚姻喪葬,皆教令儉而衷禮。又蠶月預下綿絹度樣於部內,其兵賦次第,並立明式。至於調役,事必先辦,郡縣吏長,恆無十杖稽失。



 當時州郡,無不遣人至境,訪其政術。



 天保中,郡界大水,人災,絕食者千餘家。瓊普集郡中有粟家,自從貸粟,悉以給付飢者。州計戶徵租,復欲推其貸粟,綱紀謂瓊曰:「雖矜飢餒,恐罪累府君。」



 瓊曰:「一身獲罪且活千室,何所怨乎?」遂上表陳狀,使檢皆免,人戶保安。
 此等相撫兒子,咸言「府君生汝」。在郡六年,人庶懷之,遂無一人經州。前後四表,列為尤最。遭憂解職,故人贈遺,一無所受。尋起為司直、廷尉正,朝士嗟其屈,尚書辛術曰:「既直且正,名以定體,不慮不申。」初,瓊任清河太守,裴獻伯為濟州刺史。獻伯酷於用法,瓊恩於養人。房延祐為樂陵郡,過濟州。裴問其外聲,延祐云:「唯聞太守善,刺史惡。」裴云:「得人譽者非至公。」答云:「若爾,黃霸、襲遂,君之罪人也。」後有敕,州各舉清能。裴以前言,恐為瓊陷,瓊申其枉滯,議者尚其公平。畢儀雲為御史中丞,以猛暴任職,理官忌憚,莫敢有違。瓊推察務在得情,雪者甚眾。寺
 署臺案,始自於瓊。遷三公郎中。趙州及清河、南中有人頻告謀反,前後皆付瓊推檢,事多申雪。尚書崔昂謂瓊曰:「若欲立功名,當更思餘理。仍數雪反逆,身命何輕?」瓊正色曰:「所雪者冤枉,不放反逆。」昂大慚。京師為之語曰:「斷決無疑蘇珍之。」



 皇建中,賜爵安定縣男、徐州行臺左丞,行徐州事。徐州城中五級寺忽被盜銅像一百軀。有司徵檢,四鄰防宿及蹤跡所疑,逮繫數十人。瓊一時放遣,寺僧怨訴不為推賊。瓊遣僧,謝曰「但且還寺,得像自送。」爾後十日,抄賊姓名及贓處所,徑收掩,悉獲實驗。賊徒款引,道俗歎伏。舊制,以淮禁不聽商販輒度。淮南歲
 儉,啟聽淮北取糴。後淮北人飢,復請通糴淮南,遂得商估往還,彼此兼濟,水陸之利,通於河北。



 後為大理卿而齊亡,仕周,為博陵太守。隋開皇初卒。



 路去病,陽平人也。風神疏朗,儀表瑰異。齊河清初,為殿中侍御史,彈劾不避貴戚,以正直知名。敕用士人為縣宰,以去病為定州饒陽縣令。去病明閑時務,性頗嚴毅,人不敢欺,然至廉平,為吏人歎伏。武平四年,為成安縣令。都下有鄴、臨漳、成安三縣,輦轂之下,舊號難為。重以政亂時艱,綱紀不立,近臣內戚,請屬百端。去病消息事宜,以理抗答。勢要之徒,雖廝養小人,莫不憚其風格,亦
 不至嫌恨。自遷鄴以還,三縣令政術,去病獨為稱首。周武平齊,重其能官,與濟陰郡守公孫景茂二人不被替代,發詔褒揚。去病後以尉遲迥事。隋大業初,卒於冀氏縣令。



 梁彥光,字脩芝,安定烏氏人也。祖茂,魏秦、華二州刺史。父顯,周荊州刺史。彥光少岐嶷,有至性,其父每謂所親曰:「此兒有風骨,當興吾宗。」七歲時,父遇篤疾,醫云「餌五石可愈」。時求紫石英不得,彥光憂瘁,不知所為。忽於園中見一物,彥光所不識,怪而持歸,即紫石英也。親屬咸異之,以為至孝所感。魏大統末,入學,略涉經史,有規檢,
 造次必以禮。解褐秘書郎。周受禪,遷舍人上士。武帝時,累遷小馭下大夫。母憂去職,毀瘠過禮。未幾,起令視事,帝見其毀甚,嗟嘆久之。後為御正下大夫,從帝平齊,以功授開府、陽城縣公。宣帝即位,拜華州刺史,進封華陰郡公,以陽城公轉封一子。後拜柱國、青州刺史。屬帝崩,不之官。



 隋文帝受禪,以為岐州刺史,兼領宮監,甚有惠政,嘉禾連理,出於州境。上嘉其能,下詔褒美,賜粟五百斛、物三百段、御傘一枚,以厲清正。後轉相州刺史。



 彥光前在岐州,其俗頗質,以靜鎮之,合境大安,奏課連最,為天下第一。及居相部,如岐州法。鄴都雜俗,人多變詐,為
 之作歌,稱其不能理政。上聞而譴之,竟坐免。歲餘,拜趙州刺史。彥光曰:「臣前待罪相州,百姓呼為戴帽餳。臣自分廢黜,無復衣冠之望。不謂天恩復垂收採。請復為相州,改絃易調,庶有以變其風俗。」



 上從之,復為相州刺史。豪猾者聞彥光自請來,莫不嗤笑。彥光下車,發摘姦隱,有若神明,狡猾莫不潛竄,合境大駭。初,齊亡後,衣冠士人,多遷關內,唯技巧商販及樂戶之家,移實州郭。由是人情險詖,妄起風謠,訴訟官人,萬端千變。彥光欲革其弊,乃用秩俸之物,招致山東大儒,每鄉立學,非聖哲之書不得教授。常以季月召集之,親臨策試。有勤學異等,
 聰令有聞者,升堂設饌,其餘並坐廊下。



 有好諍訟惰業無成者,坐之庭中,設以草具。及大成當舉,行賓貢之禮;又於郊外祖道,并以財物資之。於是人皆剋勵,風俗大改。



 有滏陽人焦通,性酗酒,事親禮闕,為從弟所訟。彥光弗之罪,將至州學,令觀孔子廟中韓伯瑜母杖不痛,哀母力衰,對母悲泣之像。通遂感悟,悲愧若無容者。



 彥光訓喻而遣之,後改過勵行,卒為善士。吏人感悅,略無諍訟。卒官,贈冀定瀛青四州刺史,謚曰襄。



 子文謙嗣,弘雅有父風。以上柱國世子,例授儀同。歷上、饒二州刺史,遷鄱陽太守,稱為天下之最。徵拜戶部侍郎。遼東之役,領
 武賁郎將,為盧龍道軍副。



 會楊玄感作亂,其弟武賁郎將玄縱先隸文謙,玄感反問未至而玄縱逃走,文謙不之覺。坐是,配防桂林而卒。



 少子文讓,初封陽城縣公,後為鷹揚郎將。從衛玄擊楊玄感於東都,力戰而死,贈通議大夫。



 樊叔略,陳留人也。父觀,仕魏,為南兗州刺史、河陽侯,為高氏所誅。叔略被腐刑,給使殿省。身長九尺,有志氣。頗見忌,內不自安,遂奔關西。周文器之,引置左右,授都督,襲爵為侯。大冢宰宇文護執政,引為中尉。漸被委信,兼督內外,位開府儀同三司。護誅,齊王憲引為園苑監。數
 進兵謀,憲甚奇之。從武帝平齊,以功加上開府,封清鄉縣公,拜汴州刺史,號為明決。宣帝營建東都,以叔略有巧思,拜營構監。宮室制度,皆叔略所定。尉遲迥之亂,鎮大梁,以軍功拜大將軍,復為汴州刺史。隋文帝受禪,加位上大將軍,進爵安定郡公。在州數年,甚有聲稱。遷相州刺史,政為當時第一。上降璽書褒美之,賜以粟帛,班示天下。百姓為之語曰:「智無窮,清鄉公;上下正,樊安定。」徵拜司農卿,吏人莫不流涕,相與立碑頌德。自為司農,凡所種植,叔略別有條制,皆出人意表。朝廷有疑滯,公卿所未能決,叔略輒為評理。雖無學術,有所依據,然師
 心獨見,闇與理合。甚為上所親委,高熲、楊素禮遇之。叔略雖為司農,往往參督九卿事。性頗豪侈,每食方丈,備水陸。十四年,從祠太山。至洛陽,上令錄囚徒。將奏,晨至獄門,於馬上暴卒,上嗟悼久之。贈亳州刺史,謚曰襄。



 公孫景茂,字元蔚,河間阜城人也。容貌魁梧,少好學,博涉經史。在魏,察孝廉,射策甲科。稍遷太常博士,多所損益,時人稱為書庫。歷高唐令、大理正,俱有能名。齊滅,周武帝聞而召見,與語器之,授濟北太守。以母憂去職。開皇初,召拜汝南太守。郡廢,為曹州司馬,遷息州刺史。法令清靜,德化大行。屬平陳之役,征人在路病者,景茂減
 俸祿為饘粥湯藥,多方振濟之,賴全活者千數。上聞嘉之,詔宣示天下。十五年,上幸洛陽,景茂謁見。時七十七,上命升殿坐。問其年,哀其老,嗟嘆久之。景茂再拜曰:「呂望八十而遇文王,臣踰七十而逢陛下。」上甚悅,下詔褒美之,加上儀同三司,伊州刺史。明年,以疾徵,吏人號泣於道。及疾愈,復乞骸骨,又不許。轉道州刺史。悉以秩俸買牛犢雞豬,散惠孤弱不自存者。



 好單騎巡人,家至戶入,閱視百姓產業。有修理者,於都會時,乃褒揚稱述;如有過惡,隨即訓導,而不彰也。由是人行義讓,有無均通。男子相助耕耘,婦女相從紡織,大村或數百戶,皆如一
 家之務。其後請致仕,上優詔聽之。仁壽中,上明公楊紀出使河北,見景茂神力不衰,還以狀奏。於是就拜淄州刺史,賜以馬輦,便道之官。前後歷職,皆有德政,論者稱為良牧。大業初,卒官。年八十七,謚曰康。



 身死之日,諸州人吏赴喪者數千人。或不及葬,皆望墳慟哭,野祭而去。



 辛公義,隴西狄道人也。祖徽,魏徐州刺史。父季慶,青州刺史。公義早孤,為母氏所養,親授《書》、《傳》。周天和中,選良家子任太學生。武帝時,召入露門學,令受道義,每月集御前,令與大儒講論。上數嗟異,時輩慕之。建德初,授宣納中士。從平齊,累遷掌治上士、掃寇將軍。隋文帝作相,
 授內史上士,參掌機要。開皇元年,除主客侍郎,攝內史舍人,賜爵安陽縣男。轉駕部侍郎,使勾檢諸馬牧,所獲十餘萬匹。上喜曰:「唯我公義,奉國罄心。」



 從軍平陳,以功除岷州刺史。土俗畏病,若一人有疾,即合家避之,父子夫妻,不相看養,孝義道絕。由是病者多死。公義患之,欲變其俗。因分遣官人,巡檢部內,凡有疾病,皆以床輦來,安置事。暑月疫時,病人或至數百,廊悉滿。公義親設一榻,獨坐其間,終日連夕,對之理事。所得秩俸,盡用市藥,迎醫療之,躬勸其飲食,於是悉差。方召其親戚而喻之曰:「死生由命,不關相著,前汝棄之,所以死耳。今我
 聚病者,坐臥其間,若言相染,那得不死?病兒復差,汝等勿復信之。」諸病家子孫,慚謝而去。後人有過疾者,爭就使君,其家親屬,固留養之。



 始相慈愛,此風遂革,合境之內,呼為慈母。



 後遷並州刺史。下車,先至獄中,因露坐牢側,親自驗問。十餘日間,決斷咸盡。方還大,受領新訟。皆不立文案,遣當直佐僚一人,側坐訊問。事若不盡,應須禁者,公義即宿事,終不還閣。人或諫之曰:「此事有程,使君何自苦也?」



 答曰:「刺史無德可以導人,尚令百姓系於囹圄。豈有禁人在獄,而心自安乎!」



 罪人聞之,咸自款服。後有欲諍訟者,鄉閭父老遽相曉曰:「此蓋小事,何
 忍勤勞使君!」訟者多兩讓而止。時山東霖雨,自陳汝至於滄海,皆苦水災。境內犬牙,獨無所損。山出黃銀,獲之以獻,詔水部郎婁崱就公義禱焉,乃聞空中有金石絲竹之響。



 仁壽元年,追充揚州道黜陟大使。豫章王暕恐其部內官僚犯法,未入州境,豫令使屬之。公義答曰:「不敢有私。」及至揚州,皆無所縱捨,暕銜之。及煬帝即位,揚州長史王弘入為黃門郎,因言公義之短,竟去官。吏人守闕訴冤,相繼不絕。



 後數歲,帝悟,除內史侍郎。丁母憂,未幾起為司隸大夫,檢校右禦衛武賁郎將。



 從征至柳城郡,卒。子融。



 柳儉,字道約,河東解人也。祖元璋,魏司州大中正、相、華二州刺史。父裕,周聞喜令。儉有局量,立行清苦,為州里所敬,雖至親暱,無敢狎侮。仕周,歷宣納上士、畿伯大夫。及隋文帝受禪,擢拜水部侍郎,封率道縣伯。未幾,出為廣漢太守,甚有能名。俄而郡廢。時帝勵精思政,妙簡良能,出為牧宰。儉以仁明著稱,擢拜蓬州刺史。獄訟者庭決遣之,佐吏從容而已,獄無繫囚。蜀王秀時鎮益州,列上其事。遷邛州刺史。在職十餘年,人夷悅服。蜀王秀之得罪也,儉坐與交通,免職。及還鄉,妻子衣食不贍,見者咸嘆伏焉。煬帝嗣位,征之。於時,多以功臣任職,牧州領
 郡者,並帶戎資,唯儉起自良吏。帝嘉其績,特授朝散大夫,拜弘化太守,儉清節愈勵。大業五年,入朝,郡國畢集。帝謂納言蘇威、吏部尚書牛弘曰:「其中清名天下第一者,為誰?」威等以儉對。帝又問其次,威以涿郡贊務郭絢,潁川贊務敬肅等二人對。帝賜儉帛二百匹,絢、肅各一百匹,令天下朝集使送至郡邸,以旌異焉,論者美之。及大業末,盜賊峰起,數被攻逼。儉撫結人夷,卒無離叛,竟以保全。及義兵至長安,尊立恭帝,儉與留守李粲縞素,於州南向慟哭。既而歸京師,相國賜儉物三百段,就拜上大將軍。歲餘,卒於家,時年八十九。



 郭絢,河東安邑人,
 家世寒微。初為尚書令史,後以軍功拜儀同,歷數州司馬、長史,皆有能名。大業初,刑部尚書宇文幹巡省河北,引絢為副。煬帝將有事遼東,以涿郡為衝要,訪可任者。聞絢有幹局,拜涿郡贊務,吏人悅服。數載,遷為通守,兼領留守。及山東盜起,絢逐捕之,多所剋獲。時諸郡無復完者,唯涿郡獨全。後將兵擊竇建德於河間,戰死,人吏哭之,數月不息。



 敬肅,字敬儉,河東蒲阪人。少以貞介知名,釋褐州主簿。開皇初,為安陵令,有能名。擢拜秦州司馬,轉幽州長史。仁壽中,為衛州司馬,俱有異績。煬帝嗣位,遷潁川郡贊務。大業五年,朝東都。帝令司隸大夫薛
 道衡為天下郡官之狀,稱肅曰:「心如鐵石,老而彌篤。」時左翊衛大將軍宇文述當塗用事,其邑在潁州,每有書屬肅,肅未嘗開封,輒令使者持去。述賓客有放縱者,以法繩之,無所寬貸,由是述銜之。八年,朝於涿郡。帝以其年老,有能名,將擢為太守者數矣,輒為述所毀,不行。大業末,乞骸骨,優詔許之。去官之日,家無餘財。歲餘,終於家。



 劉曠,不知何許人也,性謹厚,每以誠恕應物。開皇初,為平鄉令,單騎之官。



 人有諍訟者,輒丁寧曉以義理,不加繩劾,各自引咎而去。所得俸祿,賑施窮乏。



 百姓感其德
 化,更相篤勵曰:「有君如此,何得為非?」在職七年,風教大洽。獄中無繫囚,諍訟絕息,囹圄皆生草,庭可張羅。及去官,吏人無少長號泣,沿路將送,數百里不絕。遷為臨潁令,清名善政為天下第一。尚書左僕射高熲言狀,上召之。及引見,勞之曰:「天下縣令固多矣,卿能獨異於眾,良足美也。」顧謂侍臣曰:「若不殊獎,何以勸人?」於是下優詔,擢拜莒州刺史。



 王伽,河間章武人也。開皇末,為齊州參軍。初無足稱,後被州使送流囚李參等七十餘人詣京師。時制,流人並枷鎖傳送。次滎陽,憫其辛苦,悉呼而謂之曰:「卿輩既犯
 國刑,虧損名教,身嬰縲紲,此其職也。今復重勞援卒,豈獨不愧於心哉!」參等辭謝。伽曰:「汝等雖犯憲法,枷鎖亦大苦辛,吾欲與汝等脫去,行至京師總集,能不違期不?」皆拜謝曰:「必不敢違。」伽於是悉脫枷,停援卒,與期曰:「某日當至京師。如致前卻,吾當為汝受死。」舍之而去。流人感悅,依期而至,一無離叛。上聞而驚異,召見與語,稱善久之。於是悉召流人,並令攜負妻子俱入,賜宴於殿庭而赦之。乃下詔曰:「凡在有生,含靈稟性,咸知好惡,並識是非。若臨以至誠,明加勸導,則俗必從化,人皆遷善。往以海內亂離,德教廢絕,官人無慈愛之心,兆庶懷奸詐
 之意,所以獄訟不息,澆薄難理。朕受命上天,安養萬姓,思導聖法,以德化人。朝夕孜孜,意本如此。而伽深識朕意,誠心宣導。參等感悟,自赴憲司。明率土之人,非為難教,良是官人不加示曉,致令陷罪,無由自新。若使官盡王伽之儔,人皆李參之輩,刑措不用,其何遠哉!」於是擢伽為雍令,政有能名。



 魏德深,本鉅鹿人也。祖沖,仕周,為刑部大夫、建州刺史,因家弘農。父毗,鬱林令。德深初為隋文帝挽郎,後歷馮翊郡書佐,武陽郡司戶、書佐,以能遷貴鄉長。為政清靜,不嚴而肅。會興遼東之役,征稅百端,使人往來,責成郡
 縣。於時王綱弛紊,吏多贓賄,所在征斂,人不堪命。唯德深一縣,有無相通,不竭其力,所求皆給,而百姓不擾。於時盜賊群起,武陽諸城,多被淪陷,唯貴鄉獨全。郡丞元寶藏受詔逐捕盜賊,每戰不利,則器械必盡,輒徵發於人,動以軍法從事,如此者數矣。其鄰城營造,皆聚於事,吏人遞相督責,晝夜喧囂,猶不能濟。德深各問其所欲,任隨便修營,官府寂然,恆若無事。唯約束長吏,所修不須過勝餘縣,使百姓勞苦。然在下各自竭心,常為諸縣之最。尋轉館陶長,貴鄉吏人聞之,相與言及其事,皆歔欷流涕,語不成聲。及將赴任,傾城送之,號泣之聲,道
 路不絕。



 既至館陶,闔境老幼,皆如見其父母。有猾人員外郎趙君實,與郡丞元寶藏深相交結,前後令長,未有不受其指麾者。自德深至縣,君實屏處於室,未嘗輒敢出門。逃竄之徒,歸來如市。貴鄉父老,冒涉艱險,詣闕請留德深,有詔許之。館陶父老,復詣郡相訟,以貴鄉文書為詐。郡不能決。會持節使者韋霽、杜整等至,兩縣詣使訟之,乃斷從貴鄉。貴鄉吏人,歌呼滿道,互相稱慶;館陶眾庶,合境悲泣,因從而居住者數百家。



 寶藏深害其能。會越王侗徵兵於郡,寶藏遂令德深率兵千人赴東都。俄而寶藏以武陽歸李密,德深所領皆武陽人也,以本
 土從賊,念其親戚,輒出都門,東向慟哭而反。人或謂之曰:「李密兵馬,近在金墉,去此二十餘里,汝必欲歸,誰能相禁?何為自苦如此!」其人皆垂泣曰:「我與魏明府同來,不忍棄去,豈以道路艱難乎!」其得人心如此。後與賊戰,沒於陣。貴鄉、館陶人庶,至今懷之。



 論曰:為政之道,寬猛相濟,猶寒暑迭代,俱成歲功者也。然存夫簡久,必藉寬平,大則致鼓腹之歡,小則有息肩之惠。故《詩》曰:「雖無德與汝,式歌且舞。」



 張膺等皆有寬仁之心,至誠待物,化行所屬,愛結人心,故得所去見思,所居而化。



 《詩》所謂「愷悌君子,人之父母」,豈徒然哉!



\end{pinyinscope}