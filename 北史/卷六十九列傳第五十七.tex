\article{卷六十九列傳第五十七}

\begin{pinyinscope}

 申徽陸通弟逞厙狄峙楊薦王慶趙剛子仲卿趙昶王悅趙文表元定楊標申徽,字世儀,魏郡人也。六世祖鐘,為後趙司徒。冉閔末,中原喪亂,鐘子邃避地江左。曾祖爽,仕宋,位雍州刺史。祖隆道。宋北兗州刺史。父明仁,郡功曹,早卒。徽少與母居,盡力孝養。及長,好經史。性審慎,不妄交遊。遭母憂,喪
 畢,乃歸於魏。元顥入洛,以元邃為東徐州刺史,邃引徽為主簿。顥敗,邃被檻車送洛陽,故吏賓客並委去,唯徽送之。及邃得免,乃廣集賓友,歎徽有古人風。



 尋除太尉府行參軍。



 孝武初,徽以洛陽兵難未已,遂間行入關見周文。周文與語,奇之,薦之於賀拔岳,岳亦雅相敬待,引為賓客。周文臨夏州,以徽為記室參軍,兼府主簿。周文察徽沉密有度量,每事信委之,乃為大行臺郎中。時軍國草創,幕府務殷,四方書檄皆徽之辭也。以迎孝武功,封博平縣子、本州大中正。大統初,進爵為侯。四年,拜中書舍人,修起居注。河橋之役,大軍不利,近侍之官分散
 者眾,徽獨不離左右,魏帝稱歎之。十年,遷給事黃門侍郎。



 先是,東陽王元榮為瓜州刺史。其女婿劉彥隨焉。及榮死,瓜州首望表榮子康為刺史,彥遂殺康而取其位。屬四方多難,朝廷不遑問罪,因授彥刺史。頻徵不奉詔,又南通吐谷渾,將圖叛逆。周文難於動眾,欲以權略致之,乃以徽為河西大使,密令圖彥。徽輕以五十騎行,既至,止於賓館。彥見徽單使,不以為疑,徽乃遣一人微勸彥歸朝,以揣其意,彥不從。徽又使贊成其住計,彥便從之,遂來至館。徽先與瓜州豪右密謀執彥,遂叱而縛之。彥辭無罪,徽數之曰:「君無尺寸之功,濫居方岳之重,恃
 遠背誕,不恭貢職,戮辱使人,輕忽詔命。計君之咎,實不容誅。



 但受詔之日,本令相送歸闕,所恨不得即申明罰,以謝邊遠耳。」於是宣詔慰勞吏人及彥所部,復云大軍續至,城內無敢動者。使還,遷都官尚書。



 十二年,瓜州刺史成慶為城人張保所殺,都督令狐延等起義逐保,啟請刺史。



 以徽信洽西土,拜假節、瓜州刺史。徽在州五稔,儉約率下,邊人樂而安之。十六年,徽兼尚書右僕射,加侍中、驃騎大將軍、開府儀同三司。廢帝二年,進爵為公,正右僕射,賜姓宇文氏。



 徽性勤至,凡所居官,案牘無大小皆親自省覽,以是事無稽滯,吏不得為姦。



 後雖歷公
 卿,此志不懈。出為襄州刺史。時南方初附,舊俗官人皆通餉遺。徽性廉慎,乃畫楊震像於寢室以自戒。及代還,人吏送者數十里不絕。徽自以無德於人,慨然懷愧,因賦詩,題於清水亭。長幼聞之,皆競來就讀,遞相謂曰:「此是申使君手迹。」並寫誦之。



 明帝以御正任總絲綸,更崇其秩為上大夫,員四人,號大御正,又以徽為之。



 歷小司空、少保,出為荊州刺史。入為小司徒、小宗伯。天和六年,上疏乞骸骨,詔許之。薨,贈泗州刺史,謚曰章。



 子康嗣。位瀘州刺史、司織下大夫、上開府。



 康弟敦,汝南郡守。敦弟靜,齊郡守。靜弟處,上開府、同昌縣侯。卒。



 陸通,字仲明,吳郡人也。曾祖載,從宋武帝平關中,軍還,留載隨其子義真鎮長安,遂沒赫連氏。魏太武平赫連氏,載仕魏,位中山郡守。父政,性致孝。其母吳人,好食魚。北土魚少,政求之常苦難。後宅側忽有泉出,而有魚,遂得以供膳。時人以為孝感所致,因謂其泉為孝魚泉。從爾朱天光討伐。及天光敗,歸周文。



 周文為行臺,以政為行臺左丞、原州長史,賜爵中都縣伯。大統中,卒。



 通少敦敏好學,有志節。幼從政在河西,遂逢寇難,與政相失。通乃自拔東歸,從爾朱榮。榮死,又從爾朱兆。及爾朱氏滅,及入關。周文時在夏州,引為帳內督。



 頃之,賀拔岳為侯
 莫陳悅所害。時有傳岳軍府已亡散者,周文憂之,通以為不然。



 居數日,問至,果如所策。自是愈見親禮,遂晝夜陪侍,家人罕見其面。通雖處機密,愈自恭謹,周文以此重之。後以迎孝武功,封都昌縣伯。



 大統元年,進爵為侯。從禽竇泰,復弘農。沙苑之後。力戰有功。又從解洛陽圍。軍還,屬趙青雀反於長安,周文將討之,以人馬疲弊,不可速行,又謂青雀等一時陸梁,不足為慮,乃云「我到長安,但輕騎臨之,必當面縛。」通進曰:「青雀等既以大軍不利,謂朝廷傾危,同惡相求,遂成反亂。然其逆謀久定,必無遷善之心。且其詐言大軍敗績,東寇將至,若以輕騎
 往,百姓謂為信然,更沮兆庶之望。



 大兵雖疲弊,精銳猶多,以明公之威,率思歸之眾,以順討逆,何慮不平?」周文深納之,因從平青雀。錄前後功,進爵為公,徐州刺史。以寇難未平,留不之部。



 與于謹討劉平伏,加大都督。從周文援玉壁,進儀同三司。



 九年,高仲密以地來附,通從若干惠戰於芒山。眾軍皆退,唯惠與通率所部力戰。至夜中乃陰引還,敵亦不敢逼。時授驃騎大將軍、開府僕同三司、太僕卿,賜姓部六孤氏,進爵綏德郡公。周孝閔踐祚,拜小司空。保定五年,累遷大司寇。



 通性柔謹,雖久處列位,常清慎自守。所得祿賜,盡與親故共之,家無餘財。



 常曰:「凡人患貧而不貴,不患貴而貧也。建德元年,轉大司馬。其年薨。通弟逞。



 逞字季明,初名彥,字世雄。魏文帝常從容謂之曰:「爾既溫裕,何因乃字世雄?且為世之雄,非所宜也。於爾兄弟又復不類。」遂改焉。逞少謹密,早有名譽。



 兄通先以軍功別受茅土,乃讓父爵中都縣伯令逞襲之。起家羽林監、周文內親信。



 時輩皆以驍勇自達,唯逞獨兼文雅,周文由此加禮遇焉。大統十四年,參大丞相府軍事,尋兼記室。保定初,累遷吏部中大夫,歷蕃部、御伯中大夫,進驃騎大將軍、開府儀同三司,徙授司宗中大夫,轉軍司馬。
 逞幹識詳明,歷任三府,所在著績。



 朝遷嘉之,進爵為公。



 天和三年,齊遣侍中斛斯文略、中書侍郎劉逖來聘。初脩鄰好,盛選行人,詔逞為使主,尹公正為副以報之。逞美容止,善辭令,敏而有禮,齊人稱焉。還屆近畿,詔令路車儀服郊迎而入,時人榮之。四年,除京兆尹。郡界有豕生數子,經旬而死。其家又有豶,遂乳養之,諸豚賴之以活,時論以逞仁政所致。俄遷司會中大夫,出為河州刺史。晉公護雅重其才,表為中外府司馬,頗委任之。尋復為司會,兼納言,遷小司馬。及護誅,坐免官。



 頃之,起為納言。又以疾不堪劇任,乃除宜州刺史。故事,刺史奉辭,例
 備鹵簿,逞以時屬農要,奏請停之。武帝深嘉焉,詔遂其所請,以彰雅操。逞在州有惠政,吏人稱之。東宮初建,授太子太保。卒,贈大將軍。子操嗣。



 厙狄峙,其先遼東人,本姓段,匹磾之後也,因避難改焉。後徙居代,世為豪右。祖凌,武威郡守。父貞,上洛郡守。峙少以弘厚知名,善騎射,有謀略。仕魏,位高陽郡守,政存仁恕,百姓頗悅之。孝武西遷,峙乃棄官從入關。大統元年,拜中書舍人,參掌機密,以恭謹見稱。遷黃門侍郎。時與東魏爭衡,蠕蠕乘虛,屢為邊患,朝議欲結和親,乃使峙往。峙狀貌魁梧,善於辭令,蠕蠕主雅信重之,自是不
 復為寇。周文謂峙曰:「昔魏絳和戎,見稱前史。以君方之,彼有愧色。」封高邑縣公。累遷驃騎大將軍、開府儀同三司,拜侍中。蠕蠕滅後,突厥強盛,雖與周通好,而外連齊氏。周文又令峙銜命喻之。突厥感悟,即執齊使歸諸京師。進爵安豐郡公,歷小司空、小司寇。



 明帝初,為益州刺史、都督三十一州諸軍事。峙性寬和,尚清靖,為夷獠所安。



 後為宜州刺史,入為少師。以年老,乞骸骨,詔許之。卒,謚曰定。



 子嶷嗣,少知名,位開府儀同三司、職方中大夫、蔡州刺史。卒官。



 嶷弟徵,從平齊,以功拜儀同大將軍,賜爵樂陵縣公。



 徵弟徽,亦以軍功至儀同大將軍、保城縣男。



 徽弟嶔,性弘厚,有局度,以齊右下大夫從武帝東伐,入並州。軍敗,侍臣殲焉。及帝之出,唯嶔侍從。以功授上儀同大將軍,遷開府,歷右宮伯,賜爵樂城縣侯。仕隋,位至戶部尚書。



 楊薦,字承略,秦郡寧夷人也。父寶,昌平郡守。薦幼孤,早有名譽,性廉謹,喜怒不形於色。魏永安中,隨爾朱天光入關討君賊,封高邑縣男。周文臨夏州,補帳內都督。及平侯莫陳悅,使薦入洛請事,孝武授周文關西大行臺,仍除薦直閤將軍。時馮翊長公主嫠居,孝武意欲歸諸周文,乃令武衛元毗喻旨。薦歸白周文,又遣薦入洛陽
 請之,孝武即許焉。孝武欲向關中,薦贊成其計。孝武曰:「卿歸語行臺迎我。」周文又遣薦與長史宇文測出關候接。孝武至長安。進爵清水縣子。



 大統元年,蠕蠕請和親,周文遣薦與楊寬使,並結婚而還。進爵為侯。又使薦納幣於蠕蠕。魏文帝郁久閭后崩,周文遣僕射趙善使蠕蠕,更請婚。善至夏州,聞蠕蠕貳於東魏,欲執使者,善懼,乃還。周文乃使薦往,賜黃金十斤,雜綵三百匹。



 薦至蠕蠕,責其背惠食言,並論結婚之意,蠕蠕感悟,乃遣使隨薦報命焉。及侯景來附,周文令薦助鎮遏。薦知景翻覆,遂求還,具陳事實,周文乃遣使密追助景之兵。尋而景
 叛。



 十六年,大軍東討,周文恐蠕蠕乘虛寇掠,乃遣薦往,更論和好,以安慰之。



 進使持節、驃騎大將軍、開府儀同三司,加侍中。周孝閔帝踐阼,除御伯大夫,進爵姚谷縣公,仍使突厥結婚。突厥可汗弟地頭可汗阿史那庫頭居東面,與齊通和,說其兄欲背先約。計謀已定,將以薦等送齊。薦知其意,乃正色責之,辭氣慷慨,涕泗橫流。可汗慘然良久曰:「幸無所疑,當共平東賊,然後發遣我女。」乃令薦先報命,仍請東討。以奉使稱旨,廷大將軍。保定四年,又納幣於突厥。還行小司馬,又行大司徒。從陳公純等逆女於突厥,進爵南安郡公。天和三年,遷總管梁
 州刺史。後以疾卒。



 王慶,字興慶,太原祁人也。父因,魏靈州刺史、懷德縣公。慶少開悟,有才略。初從周文征伐,復弘農,破沙苑,並有戰功,每獲殊賞。大統十年,授殿中將軍。周孝閔帝踐阼,晉公護引為典簽。慶樞機明辯,漸見親待,授大都督。武成元年,以前後功,賜爵始安縣男。二年,行小賓部。保定二年,使吐谷渾。與其分疆,仍論和好之事。渾主悅服,遣所親隨慶貢獻。



 初,突厥與周和親,許納女為后。而齊人知之,懼成合從之勢,亦遣使求婚,財饋甚厚。突厥貪其重賂,便許之。朝議以魏氏昔與蠕蠕結婚,遂為齊人離
 貳,今者復恐改變,欲遣使結之。遂授慶左武伯,副楊薦為使。是歲,遂興入并之役。慶乃引突厥騎,與隋公楊忠至太原而還。及齊人許送皇姑及世母,朝廷遂與通和。突厥聞之,復致疑阻,於是又遣慶往諭之。可汗感悅,結好如初。五年,復與宇文貴使突厥逆女。自此,以慶信著北蕃,頻歲出使。後更至突厥,屬其可汗暴殂,突厥謂慶曰:「前後使來,逢我國喪者,皆緌面表哀。況今二國和親,豈得不行此事!」



 慶抗辭不從。突厥見其守正,卒不敢逼,武帝聞而嘉之。錄慶前後使功,遷開府儀同三司、兵部中大夫,進爵為公。歷丹、中二州刺史,為政嚴肅,吏不敢
 犯。大象元年,授小司徒,加上大將軍、總管汾石二州五鎮諸軍事、汾州刺史。又除延州總管,進位柱國。開皇元年,進爵平昌郡公。卒于鎮,贈上柱國,謚曰莊。子淹嗣。



 趙剛,字僧慶,河南洛陽人也。祖寧,魏高平太守。父和,永平中,陵江將軍。



 南討度淮,聞父喪,輒還,所司將致之於法,和曰:「罔極之恩,終天莫報。若許安厝,禮畢而即罪戮,死且無恨。言訖號慟,悲感傍人。主司以聞,遂宥之。喪畢,除寧遠將軍。大統初,追贈膠州刺史。剛少機辯,有幹能,起家奉朝請,累遷金紫光祿大夫,領司徒府從事中郎,加閣內都督。及孝武與齊神武構隙,剛密奉旨,召東荊
 州刺史馮景昭。未及發,而神武已逼洛陽,孝武西遷。景昭集府僚文武,議其去就,司馬馮道和請據州待北方處分。剛抽刀投地曰:「公若為忠臣,可斬道和。



 如欲從賊,可見殺!」景昭感悟,遂率眾赴關右。屬侯景逼穰城,東荊州人楊歡等起兵應景,以其眾邀景昭於路。景昭戰敗,剛遂沒於蠻。後自贖免,乃見東魏東荊州刺史李魔憐,勸令歸關西。魔憐納之,使剛至并州,密觀事勢。神武引剛內宴,因令剛賚書申敕荊州。剛還報魔憐,仍說魔憐斬楊歡等,以州歸西。魔憐乃使剛入朝。大統初,剛於灞上見周文,具陳關東情實。周文嘉之,封陽邑縣子。論復
 東荊州功,進爵臨汝縣伯。



 初,賀拔勝、獨孤信以孝武西遷之後,並流寓江左。至是,剛言於魏文帝,請追而復之。乃以剛為兼給事黃門侍郎,使梁魏興,賚移書與其梁州刺史杜懷寶等。



 即與剛盟歃,受移送建康,仍遣人隨剛報命。是年,又詔剛使三荊,聽在所便宜從事。使還,稱旨,進爵武成縣侯,除大丞相府帳內都督。復使魏興,重申前命。尋而梁人禮送賀拔勝、獨孤信等。頃之,御史中尉董紹進策,請圖梁漢,以紹為行臺、梁州刺史。剛以為不可,而朝議已決,遂出軍。紹竟無功還,免為庶人。除剛潁州郡守。高仲密以北豫州來附,兼大行臺左丞,持節
 赴潁川節度義軍。師還,剛別破侯景前驅於南陸,復獲其郡守二人。時有流言,傳剛東叛,神武因設反間,聲遣迎接。剛乃率騎襲其丁塢,拔之,周文知剛無貳,乃加賚焉。除營州刺史,進爵為公。



 渭州人鄭五醜構逆,與叛羌傍乞鐵勿相應。令剛往鎮之。將發,魏文帝引見內寢,舉觴屬剛曰:「昔侯景在東,為卿所困。黠羌小醜,豈足勞卿謀慮也?」時五醜已克定夷鎮,所在立柵。剛至,並攻破之,散其黨與。五醜於是西奔鐵勿,剛又進破鐵勿偽廣寧郡。屬宇文貴等西討,詔以剛行渭州事,資給糧餼。加驃騎大將軍、開府儀同三司,入為光祿卿。六官建,拜膳部
 中大夫。



 周孝閔帝踐祚,進爵浮陽郡公,出為利州總管。沙州氐恃險逆命,剛再討復之。



 方州生獠,自此始從賦役。剛以信州濱江負阻,乃表請討之。詔剛率利、沙等十四州兵往經略焉。仍加授渠州刺史。剛初至,梁帥憚其軍威,相次降款。剛師出踰年,士卒疲弊,尋復亡叛。後遂以入功而還。又與所部儀同尹才失和,被徵赴闕,遇疾,卒於路。贈中、淅、涿三州刺史,謚曰成。子元卿,弟仲卿。



 仲卿性粗暴,有膂力。周齊王憲甚禮之。以軍功位上儀同,為畿伯中大夫。後以平王謙功,進位大將軍,封長垣縣公。隋文帝受禪,進河北郡公,尋拜石州刺史。



 法令嚴
 猛,纖介之失無所寬捨,鞭笞輒至二百。吏人戰慄無敢違犯,盜賊屏息,皆稱其能。遷朔州總管。時塞北盛興屯田,仲卿總統之。微有不理者,仲卿輒召主掌撻其胸背,或解衣倒曳於荊棘中,時人謂之於菟。事多克濟,由是收獲歲廣,邊戍無餽運之憂。



 會突厥啟人可汗求婚,上許之。仲卿因是間其骨肉,遂相攻擊。十七年,啟人窘迫,與隋使長孫晟投通漢鎮。仲卿率騎千餘援之,達頭不敢逼。潛遣人誘致啟人所部,至者二萬餘家。其年,從高熲指白道以擊達頭,仲卿為前鋒。至族蠡山,與虜遇,交戰七日,大破之。追奔至乞伏泊,復啟人。突厥悉眾而至,
 仲卿為方陣,四面拒戰,經五日。會高熲大兵至,合擊之,虜乃敗走。追度白道,踰秦山七百餘里。時突厥降者萬餘家,上令仲卿處之恒安。以功進上柱國。朝廷慮達頭掩襲啟人,令仲卿屯兵二萬以備之,代州總管韓洪、永康公李藥王、蔚州刺史劉隆等將步騎一萬鎮恒安,達頭來寇,韓洪軍大敗。仲鄉自樂寧鎮邀擊,斬千餘級。明年,督役築金河、定襄二城以居啟人。



 時有上表言仲卿酷暴,上命御史王偉按之,並實。惜其功,不罪,因勞之曰:「知公清正,為下所惡。」賜物五百段。仲卿益恣,由是免官。仁壽初,檢校司農卿。蜀王秀之得罪,奉詔往益州按之。
 秀賓客經過處,仲卿必深文致法,州縣長吏坐者太半。上以為能,賞奴婢五十口、黃金二百兩、米粟五千石、奇寶雜物稱是。



 煬帝嗣位,判兵部、工部二尚書事。卒官。謚曰肅。子世弘嗣。



 趙昶,字長舒,天水南安人也。曾祖襄,仕魏,至中山郡守,因家於代焉。昶少聰敏,有志節。弱冠,以材力聞。魏北中郎將高千鎮陜,以昶為長史、中軍都督。



 周文平弘農,擢為相府典簽。



 大統九年,大軍失律於芒山,清水氐酋李鼠仁自軍逃還,憑險作亂。周文將討之,先求可使者,遂令昶使焉。見鼠仁,喻以禍福。群兇或從或不,從其命者,
 復將加刃於昶。而昶神色自若,志氣彌厲。鼠仁感悟,遂相率降。氐梁道顯叛,攻南由,周文復遣昶慰喻之,道顯等皆即款附。東秦州刺史魏光因徙其豪帥三十餘人並部落於華州,周文即以昶為都督領之。先是,汾州胡叛,再遣昶慰勞之,皆知其虛實。及大軍往討,昶為先驅,遂破之。以功封章武縣伯。



 十五年,拜安夷郡守,帶長蛇鎮將。氐俗荒獷,昶威懷以禮,莫不悅服。期歲之後,樂從軍者千餘人。加授帥都督。時屬軍機,科發切急,氐情難之,復相率謀叛。昶又潛遣誘說,離間其情。因其攜貳,遂輕往臨之。群氐不知所為,咸來見昶,乃收其首逆者二
 十餘人斬之,餘眾遂定。朝廷嘉之,除大都督,行南秦州事。時氐帥蓋鬧等反,昶復討禽之。又與史寧破宕昌羌、獠二十餘萬。拜武州刺史。恭帝初,加驃騎大將軍、開府儀同三司。潭水羌叛,殺武陵、潭水二郡守。昶率儀同駱天人等討平之。



 周明帝初,鳳州人仇周貢、魏興等反,自號周公,破廣化郡,攻沒諸縣,分兵西入,圍廣業、脩城二郡。廣業郡守薛爽、脩城郡守杜杲等請昶為援。遣使報杲,為周貢黨樊伏興等所獲。興等知昶將至,解脩城圍,據泥功嶺,設六伏以待昶。昶至,遂遇其伏,合戰破之。廣業之圍亦解,昶追之至泥陽川而還。興州人段吒及氐
 酋姜多復反,攻沒郡縣,昶討斬之。昶自以被拔擢居將帥之任,傾心下士,虜獲氐、羌,撫而使之,皆為昶盡力。周文常曰:「不煩國家士馬而能威服氐、羌者,趙昶有之矣。」至是,明帝錄前後功,進爵長道郡公,賜姓宇文氏,賞勞甚厚。二年,徵拜賓部中大夫,行吏部。尋以疾卒。



 王悅,字眾喜,京兆藍田人也。少有氣幹,為州里所稱。周文初定關隴,悅率募鄉里從軍。屢有戰功。大統元年,除相府刑獄參軍。封藍田縣伯。四年,東魏將侯景攻圍洛陽,周文赴援,悅又率鄉里千餘人從軍至洛陽。將戰之夕,悅罄其行資,市牛饗戰士。悅所部盡力。斬獲居多。遷
 大行臺右丞,轉左丞。久居管轄,頗獲時譽。



 十三年,侯景據河南來附,仍請兵為援,周文先遣韋法保、賀蘭願德等帥眾助之。悅言於周文曰:「侯景之於高歡,始則篤鄉黨之情,末乃定君臣之契,位居上將,職重台司,論其分義,有同魚水。今歡始死,景便離貳,豈不知君臣之道有違,忠義之禮不足?蓋其圖既大,不恤小嫌。然尚能背德於高氏,豈肯盡節於朝廷?今若益之以勢,援之以兵,非唯侯景不為池中之物,亦恐朝廷貽笑將來也。」周文納之,乃遣追法保等,而景尋叛。後拜京兆郡守、散騎常侍,遷大行臺尚書。從達奚武征梁漢。軍出,武令悅說其城主
 楊賢,悅乃貽之書,賢於是遂降。悅又白武云:「白馬衝要,是必爭之地。今城守寡弱,易可圖也。若蜀兵更至,攻之實難。」武然之,即令悅率輕騎徑趣白馬。悅示其禍福,梁將深悟,遂以城降。時梁武陵王蕭紀果遣其將任珍奇,欲先據白馬。行次關城,聞其已降,乃還。及梁州平,周文即以悅行刺史事。招攜初附,人吏安之。



 廢帝二年,徵還本任。屬改行臺為中外府,尚書員廢,悅以儀同領兵還鄉里。



 悅既久居顯職,及此之還,私懷怏怏,猶陵駕鄉里,失於宗黨之情,其長子康恃悅舊望,遂自驕縱。所部軍人將有婚禮,康乃非理陵辱。軍人訴之,悅及康並坐除
 名,仍配遠防。及于謹伐江陵,令悅從軍展效。江陵平,因留鎮之。



 周孝閔帝踐祚,依例復官,授郢州刺史。尋拜使持節、驃騎大將軍、開府儀同三司、大都督、司水中大夫,進爵藍田縣侯。俄遷司憲中大夫,賜姓宇文氏,又進爵河北縣公。性儉約,不營生業,雖出內榮顯,家徒四壁而已,明帝手敕勞勉之。



 保定元年,卒於位。



 子康嗣,官至司邑下大夫。



 趙文表,其先天水西人也,後徙居南鄭。累世為二千石。父玨,性方嚴,有度量。位御伯中大夫,封昌國縣伯。贈虞、絳二州刺史,謚曰貞。文表少而脩謹,志存忠節。起家為
 周文親信,累遷左金紫光祿大夫。保定五年,授畿伯下大夫,遷許國公宇文貴府長史。尋拜車騎大將軍、儀同三司。仍從貴使突厥迎皇后,進止儀注,皆令文表典之。文表斟酌而行,皆合禮度。及皇后將入境,突厥託以馬瘦徐行,文表慮其為變,遂說突厥使羅莫緣曰:「后自發彼蕃,已淹時序,途經沙漠,人馬疲勞,且東寇每伺間隙,吐谷渾亦能為變。今君以可汗愛女,結姻上國,曾無防慮,豈人臣之體乎?」莫緣然之,遂倍道兼行,數日至甘州。以迎后功,別封伯陽縣伯。



 天和三年,除梁州總管府長史。所管地名恒稜者,方數百里,並夷、獠所居,恃其險固,
 常懷不軌,文表率眾討平之。遷蓬州刺史。政尚仁恕,夷、獠懷之。加驃騎大將軍、開府儀同三司。又加大將軍,進爵為公。大象中,拜吳州總管,時開府於顗為吳州刺史。及隋文帝執政,尉遲迥等舉兵,遠近騷然,人懷異望。顗自以族大,且為國家肺腑,懼文表負已,謀欲先之,乃稱疾不出。文表往問之,顗遂手刃文表,因令其吏人告云:「文表謀反。」仍馳啟其狀。帝以諸方未定,恐顗為變,遂授顗吳州總管以安之。後知文表無異志,雖不罪顗,而聽其子仁海襲爵。



 元定,字願安,河南洛陽人也。祖比,魏婺州刺史。父道龍,
 鉅鹿郡守。定惇厚少言,內沈審而外剛毅。從周文討侯莫陳悅,以功拜步兵校尉。孝武西遷,封高邑縣男。定有勇略,累從征伐,每戰必陷陣,然未嘗自言其功。周文深重之,諸將亦稱其長者。累加驃騎大將軍、開府儀同三司,進爵為公。廢帝二年,以宗室進封建城郡王。三年,行《周禮》,爵隨例降,改封長湖郡公。



 周明帝初,拜岷州刺史。威恩兼濟,甚得羌豪之情,先時生羌據險不賓者,至是並出山谷,從征賦焉。及定代還,羌豪等咸戀慕之。保定中,授左宮伯中大夫。



 久之,轉左武伯中大夫,進位大將軍。天和二年,陳湘州刺史華皎舉州歸梁,梁主欲因其隙
 更圖攻取,乃遣使請兵,詔定從衛公直率眾赴之。梁人與華皎皆為水軍,定為陸軍,直總督之。俱至夏口,而陳郢州堅守不下,直令定圍之。陳遣其將淳於量、徐度、吳明徹等水陸來拒,皎為陳人所敗,真得脫身歸梁。定既孤軍縣隔,進退無路,陳人乘勝,水陸逼之。定乃率所部,斫竹開路,且戰且行,欲趨湘州,而湘州已陷。徐度等知定窮迫,遣使偽與定通和,重為盟誓,許放還國。定疑其詭詐,欲力戰死之。而定長史長孫隆及諸將等多勸定和,定乃許之。於是與度等刑牲歃血,解仗就船。為度所執,所部眾軍亦被囚虜,送詣丹陽。居數月,憂憤發病卒。子樂
 嗣。



 楊標字顯進,正平高涼人也。祖貴、父猛,並為縣令。標少豪俠,有志氣。



 魏孝昌中,爾朱榮殺害朝士,大司馬、城陽王元徽逃難投標,標藏而免之。孝莊帝立,徽乃出,復為司馬。由是標以義烈,擢拜伏波將軍、給事中。元顥入洛,孝莊北度太行。及爾朱榮奉帝南討,至馬渚,標乃具船以濟王師。顥平,封肥如縣伯,加鎮遠將軍、步兵校尉、行濟北郡事。進都督、平東將軍、太中大夫。



 從孝武入關,進爵為侯,加撫軍將軍、銀青光祿大夫。時東魏遷鄴,周文欲知其所為,乃遣標間行詣鄴以觀察之。使還稱旨,授通直散騎常侍、車騎將軍。稽胡恃險不賓,屢行鈔竊,
 以標兼黃門侍郎,往慰撫之。標頗有權略,能得邊情,誘化酋梁,多來款附,乃有隨樹入朝者。時弘農為東魏守,標從周文攻拔之。然自河以北,猶附東魏。標父猛先為邵郡白水令,標與其豪右相知,請微行詣邵郡,舉兵以應朝廷。周文許之,標遂行。與土豪王覆憐等陰謀舉事,密相應會,內外俱發,遂拔邵郡,禽郡守程保及縣令四人,並斬之。眾議推標行郡事,標以因覆憐成事,遂表覆憐為邵郡守。以功授大行臺左丞,仍率義徒更為經略。於是遣諜人誘說東魏城堡,旬月之間,正平、河北、南汾、二絳、建州、太寧等諸城,並有請為內應者,大軍因攻而拔之。
 以標行正平郡事,左丞如故。齊神武敗於沙苑,其將韓軌、潘樂、可朱渾元等為殿,標分兵要截,殺傷甚眾。東雍州刺史司馬恭懼標威聲,棄城遁走。



 標遂移據東雍州。



 周文以標有謀略,堪委邊任,乃表行建州事。時建州遠在敵境,然標威恩夙著,所經之處,多贏糧附之。比至建州,眾已一萬。東魏州刺史車折於洛出兵逆戰,標擊敗之。又破其行臺斛律俱於州西,大獲甲仗及軍資,以給義士。由是威名大振。



 東魏遣太保尉景攻陷正平,復遣行臺薛脩義與斛律俱相會,於是敵眾漸盛。標以孤軍無援,且腹背受敵,謀欲拔還,復恐義徒背叛,遂偽為周
 文書,遣人若從外送來者,云已遣軍四道赴援。因令人漏泄,使所在知之。又分士人義酋,令各領所部四出鈔掠,擬供軍費。標分遣訖,遂於夜中拔還邵郡。朝廷嘉其權以全軍。既授建州刺史。時東魏以正平為東雍州,遣薛榮祖鎮之。乃先遣奇兵,急攻汾橋。榮祖果盡出城中戰士,於汾橋拒守。其夜,標從他道濟,遂襲剋之。進驃騎將軍。邵郡人以郡東叛,郡守郭武安脫身走免。標又率兵攻而復之。轉正平郡守。又擊破東魏南絳郡,虜其郡守屈僧珍。錄前後功,封郃陽縣伯。



 芒山之戰,標攻拔標栢谷塢,因即鎮之。及大軍不利,標亦拔還。而東魏將侯景率
 騎追標,標與儀同韋法保同心抗禦,且戰且前,景乃引退。周文嘉之,復授建州刺史,鎮軍箱。標久從軍役,末及葬父。至是,表請遷葬。詔贈其父車騎大將軍、儀同三司、晉州刺史,贈其母夏陽縣君,並給儀衛,州里榮之。及齊神武圍玉壁,別令侯景趣齊子嶺。標恐入寇邵郡,率騎禦之。景遠聞標至,斫木斷路者六十餘里,猶驚而不安,遂退還河陽,其見憚如此。十二年,進授大都督,加晉、建二州諸軍事。又攻破蓼塢,獲東魏將李顯,進儀同三司。尋加開府,復鎮邵郡。十六年,大軍東討,授大行臺尚書,率義眾先驅敵境,攻其四戍,拔之。時以齊軍不出,乃追
 標還。改封華陽縣侯。又於邵郡置邵州,以標為刺史,率所部兵鎮之。



 保定四年,遷少師。其年,大軍圍洛陽,詔標出軹關。然標自鎮東境二十餘年,數與齊人戰,每常克獲,以此遂有輕敵之心。時洛陽未下,而標深入敵境,又不設備。齊人奄至,大破標軍。標以眾敗,遂降於標齊。標之立勛也。有慷慨壯烈之志,及軍敗,遂就虜以求茍免,時論以此鄙之,朝廷猶錄其功,不以為罪,令其子襲爵。



 論曰:申徽局量深沉,文之以經史,陸通鑒悟明敏,飾之以溫恭,並夙奉龍顏,早蒙任遇,效宣提戟,功預披荊,義結周旋,恩生契闊。遂得入居端揆,出撫列籓。



 雖以識用
 成名,抑亦情兼惟舊。陸逞於戎旅之際,以文雅見知,出境播延譽之能,蒞官著從政之美,歷居顯要,豈徒然哉!厙狄峙建和戎之功,楊薦成入關之策,趙剛之克剪兇狡,趙昶之懷服氐、羌,王悅之料侯景,文表之譎突厥,或明稱先覺,或識表見機,觀其立功立事,皆一時志力之士也。元定敗亡,同黃權之無路;楊標攻勝,亦兵破而身囚。功名寥落,良可嗟矣!《易》曰:「師出以律,否臧兇。」



 《傳》曰:「不備不虞,不可以師。」其標之謂也!



\end{pinyinscope}