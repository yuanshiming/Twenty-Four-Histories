\article{卷八十五列傳第七十三 節義}

\begin{pinyinscope}

 於
 什門段進石文德汲固王玄威婁提劉渴侯朱長生于提馬八龍門文愛晁清劉侯仁石祖興邵洪哲王榮世胡小彪孫道登李几張安祖王閭劉業興蓋俊郭琰沓龍超乙速孤佛保
 李棠杜叔毗劉弘游元張須陀楊善會盧楚劉子翊堯君素陳孝意張季珣杜松贇郭世俊郎方貴《易》稱:「立人之道,曰仁與義。」蓋士之成名,在斯二者。故古人以天下為大,方身則輕;生為重矣,比義則輕。然則死有重於太山,貴其理全也;生有輕於鴻毛,重其義全也。故生無再得,死不可追。而仁道不遠,則殺身以徇;義重於生,則捐軀而踐。龍逢殞命於夏癸,比干竭節於商辛,申蒯斷臂於齊莊,弘演納肝於衛懿,漢之紀信、欒布,晉
 之向雄、嵇紹,並不憚於危亡,以蹈忠貞之節。雖功未存於社稷,力無救於顛墜。然視彼茍免之徒,貫三光而洞九泉矣。凡在立名之士,莫不庶幾焉。然至臨難忘身,見危授命,雖斯文不墜,而行之蓋寡。固知士之所重,信在慈乎。非夫內懷鐵石之心,外負陵霜之節,孰能行之若命,赴蹈如歸者乎!自魏訖隋,年餘二百,若迺歲寒見松柏,疾風知勁草,千載之後,懍懍猶生。豈獨聞彼伯夷,懦夫立志,亦冀將來君子,有所庶幾。



 《魏書》序于什門、段進、石文德、汲固、王玄威、婁提、劉渴侯、朱長生、馬八龍、門文愛、晁清、劉侯仁、石祖興、邵洪哲、王榮世、胡小彪、孫道登、
 李幾、張安祖、王閭以為《節義傳》,今又檢得郭琰、沓龍超、乙速孤佛保,及《周書孝節傳》李棠、杜叔毗附之。又案《齊書》不立此篇,而《隋書》序劉弘、皇甫誕、游元、馮慈明、張須阤、楊善會、獨孤盛、元文都、盧楚、劉子翊、堯君素為《誠節傳》。今皇甫誕、馮慈明、獨孤盛、元文都各附其家傳,其餘並附此篇,又檢取《隋書孝義傳》郎方貴、郭世俊亦附之,以備《節文傳》云。



 于什門,代人也。魏明元時為謁者,使喻馮跋。及至和龍,住外不入,使謂跋曰:「大魏皇帝有詔,須馮主出受,然後敢入。」跋使人牽逼令入。見跋不拜,跋令人按其項。什門
 曰:「馮主拜受詔,吾自以賓主致敬,何須苦見逼也?」與跋往復,聲氣厲然,初不撓屈。既而跋止什門。什門於群眾中回身背跋,披褲後襠以辱之。既而拘留,隨身衣裳,敗壞略盡,蟣虱被體。跋遺以衣服,拒而不受。歷二十四年。後馮弘上表稱臣,乃送什門歸。拜書侍御史。太武下詔褒美,比之蘇武,賜羊千口、帛千匹,進為上大夫,策告宗廟,班示天下。



 段進,不知何許人也。太武初,為白道守將。蠕蠕大檀入塞,圍之,力屈被執。



 進抗聲大罵,遂為賊殺。帝愍之,追贈安北將軍,賜爵顯美侯,謚曰莊。



 石文德,中山蒲陰人也。有行義。真君初,縣令黃宣在任喪亡。宣單貧,無期親。文德祖父苗以家財殯葬,持服三年。奉養宣妻二十餘載,及亡,又衰縗斂祔,率禮無闕。自苗逮文德,刺史守令卒官者,制服送之。五世同居,閨門雍睦。



 又梁州上言,天水白石縣人趙令安、孟蘭強等四世同居,行著州里。詔並標榜門閭。



 汲固,東郡梁城人也。為兗州從事。刺史李式坐事被收,吏人皆送至河上。時式子憲生始滿月。式大言於眾曰:「程嬰、仵臼何如人也?」固曰:「今古豈殊!」



 遂便潛還不顧,徑來入城,於式婦閨抱憲歸藏。及捕者收憲,屬有一婢產
 男,母以婢兒授之。事尋泄,固乃攜憲逃遁,遇赦始歸。憲即為固長育,至十餘歲,恒呼固夫婦為郎婆。後高祐為兗州刺史,嘉固節義,以為主簿。



 王玄威,恒農北陜人也。獻文崩,玄威立草廬於州城門外,衰裳蔬粥,哭踴無時。刺史茍頹以事表聞。詔令問狀,云:「先帝澤被蒼生,玄威不勝悲慕,戀心如此,不知禮式。」詔問玄威,欲有所訴,聽為表列。玄威云:「聞諱悲號,竊謂臣子同例,無所求謁。」及至百日,乃自竭家財,設四百人齋會。忌日,又設百僧供。



 至大除日,詔送白紬褲褶一具與玄威釋服,下州令表異焉。



 婁提,代人也。獻文時,為內三郎。獻文暴崩,提謂人曰:「聖主升遐,安用活為!」遂引佩刀自刺,幾死。文明太后詔賜帛二百匹。



 時有敕勒部人蛭拔寅,兄地于坐盜食官馬,依制命死。拔寅自誣己殺,兄又云實非弟殺。兄弟爭死,辭不能定,孝文昭原之。



 劉渴侯,不知何許人也。稟性剛烈。太和中,為徐州後軍,以力死戰,眾寡不敵,遂禽。瞋目大罵,終不降屈,為賊所殺。孝文贈立忠將軍、平州刺史、上庸侯,賜絹千匹、穀千斛。



 有嚴季者亦為軍校尉,與渴侯同殿,勢窮被執,終不降屈。後得逃還,除立節將軍,賜爵五等男。



 朱長生、于提者,並代人也。孝文時,長生為員外散騎常侍,與提俱使高車。



 既至,高車王阿伏至羅責長生等拜,長生拒之。阿伏至羅乃不以禮待。長生以金銀寶器奉之,至羅既受獻,長生曰:「為臣內附,宜盡臣禮,何得口云再拜,而實不拜。」呼出帳,命眾中拜。阿伏至羅慚其臣下,大怒曰:「帳中何不教我拜,而辱我於大眾?」奪長生等獻物,內之叢石,兵脅之曰:「為我臣則活,不降則殺汝!」



 長生與于提瞋目厲聲責之曰:「我為鬼,不為汝臣!」阿伏至羅大怒,絕其飲食。



 從者三十人皆求阿伏至羅,乃給以肉酪。長生與提又不從,乃各分徙之。三歲及放還。孝文以
 長生等守節,遠同蘇武,拜長生河內太守,提隴西太守,並賜爵五等男,從者皆為令長。



 馬八龍,武邑武強人也。輕財重義。友人武遂縣尹靈哲在軍喪亡,八龍聞即奔赴,負屍而歸,以家財殯葬,為制緦麻,撫其孤遺,恩如所生。州郡表列,詔表門閭。



 門文愛,汲郡山陽人也。早孤,供養伯父母以孝謹聞。伯父亡,服未終,伯母又亡。文愛居喪持服六年,哀毀骨立。鄉人魏仲賢等相與標其孝義。



 晁清,遼東人也。祖暉,濟州刺史、潁川公。清襲祖爵,例降為伯。為梁城戍將,梁師攻圍,糧盡城陷。清抗節不屈,為
 賊所殺。宣武褒美,贈樂陵太守,謚曰忠。子榮賓襲。



 劉侯仁,豫州人也。城人白早生殺刺史司馬悅,據城南叛。悅息朏,走投侯仁,賊雖重加購募,又嚴其捶撻,侯仁終無漏泄。朏遂免禍。事寧,有司奏其操行,請免府籍,敘一小縣。詔可。



 石祖興,常山九門人也。太守田文彪、縣令和真等喪亡,祖興自出家絹二百餘匹,營護喪事。州郡表列。孝文嘉之,賜爵二級為上造。後拜寧陵令,卒。吏部尚書李韶奏其節義,請加贈謚,以獎來者,靈太后如所奏。有司謚曰恭。



 邵洪哲,上谷沮陽人也。縣令范道榮先自朐城歸款,以除縣令。道榮鄉人徐孔明妄經公府,訟道榮非勳,道榮坐除名。羈旅孤貧,不能自理。洪哲不勝義憤,遂代道榮詣京師,明申曲直,經歷寒暑,不憚劬勞。道榮卒得復雪。



 又北鎮反亂,道榮孤單,無所歸附。洪哲兄伯川復率鄉人來相迎接,送達幽州。



 道榮感其誠節,訴省申聞。詔下州郡,標其里閭。



 王榮世,陽平館陶人也。為三城戍主、方城縣子。梁師攻圍,力窮,知不可全,乃先焚府庫,後殺妻妾。及賊陷城,與戍副鄧元興等俱以不屈被害。明帝下詔,褒美忠節,進
 榮世爵為伯,贈齊州刺史;元興開國子,贈洛州刺史。



 胡小彪,河南河陰人也。少有武氣。正光末,為統軍於晉壽。孝昌中,梁將樊文識等寇邊。益州刺史邴虯遣長史和安固守小劍,文熾圍之。虯命小彪與統軍崔珍寶同往防拒。文熾掩襲小彪、珍寶並禽之。文熾攻小劍未陷,乃將珍寶至城下,使謂和安曰:「南軍強盛,北救不來,豈若歸款,取其富貴?」和安命射之,乃退。



 復逼小彪與和安交言。小彪乃慷慨謂安曰:「我柵不防,為賊所虜。觀其兵士,勢不足言,努力堅守,魏行臺、傅梁州遣將已至。」賊以刀毆擊,言不得終,遂害之。



 三軍無不歎其壯節,哀其死
 亡。賊尋奔敗,禽其次將蕭世澄、陳文緒等一十一人。



 行臺魏子建壯其氣概,啟以世澄購其屍柩,乃獲骸骨歸葬之。



 遜道登,彭城呂縣人也。永安初,為梁將韋休等所虜。面縛臨刃,巡遠村塢,令其招降鄉曲。道登厲聲唱呼:「但當努力,賊無所能!」賊遂屠戮之。



 又荊州被圍,行臺宗靈恩遣使宗女等四人入城曉喻,為賊將所獲。執女等巡城,令其改辭。女等大言:「天軍垂至,堅守莫降。」賊忿,各刳其腹,然後斬首。二州表其節義。道登等並賜五品郡、五等子爵,聽子弟承襲,遣使詣所在弔祭。



 李几,博陵安平人也。七世共居同財。家有二十二房,一百九十八口,長幼濟濟,風禮著聞。至於作役,卑幼競集。鄉里嗟美,標其門閭。



 張安祖,河陽人也。襲世爵山北侯。時有元承貴,曾為河陽令。家貧,且赴尚書求選,逢天寒甚,遂凍死路側。一子年幼,停屍門巷,棺殮無託。安祖悲哭盡禮,買木為棺,手自營作,殮殯周給。朝野嘉歎。尚書聞奏,標其門閭。



 王閭,北海密人也。數世同居,有百口。又太山劉業興,四世同居,魯郡蓋俊,六世同居,並共財產,家門雍睦。鄉里敬異。有司申奏,皆標門閭。



 郭琰,字神寶,京槃人也。少喪父,事母以孝聞。孝武帝之居籓邸,琰以通俠被知。及即位,封新豐縣公,除洛州刺史。孝武西入,改封馮翊郡公,授行臺尚書、潼關大都督。大統中,齊神武遣大都督竇泰襲恒農。時琰為行臺,眾少戰敗,乃奔洛州。至刺史泉仙城守力窮,城將陷,乃仰天哭曰:「天乎!天乎!何由縱此長蛇,而不助順也?」言發涕流,不能自止。兵士見之,咸自厲憤。竟為東魏將高敖曹所禽。復謂敖曹曰:「天子之臣,乃為賊所執。」敖曹素聞其名,義不殺之,送於并州。見齊神武,言色不屈,見害。



 沓龍超,晉壽人也。性尚義俠,少為鄉里所重。永熙中,梁
 將樊文熾來寇益州,刺史傅和孤城固守。龍超每出戰,輒破之。時攻圍既久,糧矢方盡,刺史遣龍超夜出,請援於漢中,遂為文熾所得。許以封爵,使告城中曰:「外無援軍,宜早降。」



 乃置龍超於攻樓上。龍超乃告刺史曰:「援軍數萬,近在大寒。」文熾大怒,火炙殺之。至死,辭氣不撓。大統二年,詔贈龍驤將軍、巴州刺史。



 乙速孤佛保,北秀容胡酋也。少驍武,善射。孝武帝時,為直閣將軍。從入關,封蒲子縣公,并賜弓矢。大統初,梁將蘭欽來寇,遂陷漢中。佛保時為都督,統兵力戰。知將敗,乃先城未陷,仰天大哭曰:「此馬吾常所乘,此弓矢天恩
 賜我,豈可令賊得吾弓馬乎!」遂斬馬及弓,自刎而死。三軍莫不壯之。黃門郎趙僧慶時使漢中,聞,乃收運其屍致長安。天子歎感,詔著作錄之。



 李棠,字長卿,勃海蓚人也。祖伯貴,魏宣武時,官至魯郡守。有孝行,居父喪,哀戚過禮,遂以毀卒。宣武嘉之,贈勃海相。父元胄,員外散騎侍郎。棠幼孤,好學,有志操。高仲密為北豫州刺史,請棠為掾。仲密將圖西附。時東魏又遣鎮城奚壽興典兵事。仲密遂與堂謀殺壽興,率其眾據城,遣棠詣關中歸款。周文嘉之,封廣宗縣公,位給事黃門侍郎,加車騎大將軍、儀同三司、散騎常侍。從魏安
 公尉遲迥伐蜀,棠乃應募喻之。既入成都,蕭捴問迥軍中委曲,棠不對。捴乃苦辱之。



 棠曰:「我王者忠臣,有死而已,義不為爾移志也。」遂害之。子敞嗣。



 杜叔毗,字子弼,其先京兆杜陵人也,徙居襄陽。父漸,梁邊城太守。叔毗早歲而孤,事母以孝聞。仕梁,為宜豐侯蕭脩府中直兵參軍。周文令大將軍達奚武圍脩於南鄭,脩令叔毗詣闕請和。周文見而禮之。使未及還,而脩中直兵曹策、參軍劉曉謀以城降武。時叔毗兄君錫為脩中記室參軍,從子映錄事參軍,映弟晰中直兵參軍,各領部曲。策等忌之,懼不同己,遂誣以謀叛,擅加害焉。
 尋討策等禽之。



 城降,策至長安,叔毗朝夕號泣,具申冤狀。朝議以事在歸附之前,不可追罪。叔毗志在復仇,然恐坐及其母。母曰:「汝兄橫罹禍酷,痛切骨髓。若曹策朝死,吾以夕歿,亦所甘心。汝何疑焉?」叔毗拜受母言,後遂白日手刃策於京城,斷首瓠腹,解其支體,然後面縛請就戮焉。周文嘉其志氣,特命舍之。遭母憂,哀毀骨立,殆不勝喪。服闋,晉公護辟為中外府樂曹參軍。累遷陜州刺史。後從衛國公直南討,軍敗,為陳人所禽。陳人將降之,叔毗辭色不撓,遂被害。子廉卿。



 劉弘,字仲遠,彭城叢亭里人也。少好學,有羈檢,重節概。
 仕齊,位西楚州刺史。齊亡,周武帝以為本郡太守。及隋文帝平陳,以行車長史從總管吐萬緒度江,加上儀同,封濩澤縣公,拜泉州刺史。會高智慧亂,以兵攻州。弘城守,糧盡,煮犀甲腰帶及剝樹皮食之,一無離叛。賊欲降之,弘抗節彌厲。城陷,為賊所害。文帝聞而嘉歎者久之,賜物二千段。子長信,襲其官爵。



 游元,字楚客,廣平任城人也。父寶藏,位至郡守。元少聰敏。仕周,歷壽春令、譙州司馬,俱有能名。開皇中,為殿內侍御史。煬帝嗣位,遷尚度支郎。遼東之役,領左驍衛長史,為蓋牟道監軍,拜朝請大夫,兼書侍御史。宇文述
 等九軍敗績,帝令元主其獄。述時貴倖,勢傾朝廷,遣家僮造元,有所請屬,元不之見。他日,案述逾急,仍以屬請狀劾之。帝嘉其公正,賜朝服一襲。後奉使黎陽督運。楊玄感作逆,告以情。元引正義責之,遂見困,竟不屈節,見害。帝甚嘉之,贈銀青光祿大夫,拜其子仁宗為正議大夫、弋陽郡通守。



 張須陀,弘農閿鄉人也。性剛烈,有勇略。弱冠從史萬歲討西爨,以功授儀同。



 後從楊素擊平漢王諒,加開府。大業中,為齊郡贊務。會興遼東之役,歲飢,須陀將開倉賑給。官屬咸曰:「須待詔敕。」須陀曰:「如待報至,當委溝壑。吾
 若以此獲罪,死無所恨。」先開倉而後狀,帝嘉而不責。



 天下既承平日久,多不習兵。須陀獨勇決善戰,又長撫馭,得士卒心,號為名將。時賊帥王薄北連豆子賊孫宣雅、石祗闍、郝孝德等,眾十餘萬,攻章丘。須陀大破之,露布以聞。帝大悅,優詔褒揚,令使者圖畫其形容奏之。其年,賊裴長才,石子河等奄至城下,須陀與戰,長才敗走。後數旬,賊帥秦君弘、郭方預等園北海,須陀倍道而進,大敗之。司隸刺史裴操之上狀,帝遣使勞問之。



 十年,賊左孝友屯蹲狗山,須陀列八營以逼之。孝友窘迫,面縛來降。其黨解象、王良、鄭大彪、李脘等眾各萬計,須陀平
 之,威振東夏。以功遷齊郡通守,領河南道十二郡黜陟討捕大使。俄而賊盧明月眾十餘萬將寇河北,次祝阿。須陀邀擊,殺數千人。賊呂明星、師仁泰、霍小漢等眾各萬餘,擾濟北,須陀擊走之。尋將兵拒東郡賊翟讓,前後三十餘戰,每破走之。轉榮陽通守。



 時李密說讓取洛口倉,遂逼來滎陽。須陀拒之,讓懼而退,須陀乘之。密先伏數千人邀擊之,須陀敗,被圍,潰輒出,左右不能盡出,復入救之,往來數四,眾皆敗。乃仰天曰:「兵敗如此,何面見天子乎!」乃下馬戰死。其所部兵晝夜號哭,數日不止。帝令其子元備總父兵。元備時在齊郡,遇賊,竟不果行。



 楊善會,字敬仁,弘農華陰人也。父初,位毗陵太守。善會大業中為鄃令,以清正聞。俄而百姓聚起為盜,善會討之,往皆剋捷。後賊帥張金稱屯於縣界,善會每挫其鋒。煬帝遣將軍段達討金稱,善會進計於達,達不能用,軍竟敗。後進止一以謀之,乃大剋。金稱復引勃海賊孫宣雅、高士雅等破黎陽而還,善會邀破之。擢拜朝請大夫,清河郡丞。於時山東郡縣,陷沒相繼,能抗賊者,唯善會而已。前後七百餘陣,未嘗負敗。會太僕楊義臣討金稱見敗,取善會定策,與金稱戰,賊乃退走。善會捕斬之,傳首行在所。帝賜以尚方甲槊弓劍,進拜清河通守。復從
 楊義臣斬漳南賊帥高士達,傳首江都宮。帝下詔褒揚之。後為竇建德所陷。建德釋而禮之,用為貝州刺史。善會肆罵,臨之以兵,辭氣不撓,乃害之。清河士庶,莫不傷痛。



 盧楚,涿郡范陽人也。祖景祚,魏司空掾。楚少有才學,性鯁急,口秘,言語澀難。大業中,為尚書左司郎。當朝正色,甚為公卿所憚。及帝幸江都,東都官僚多不奉法。楚每存糾舉,無所回避。越王侗稱尊號,以楚為內史令、左備身將軍、尚書左丞、右光祿大夫,封涿郡公,與元文都等同心戮力以輔侗。及王世充作亂,兵犯太陽門。武衛將
 軍皇甫無逸斬關逃難,呼楚同去。楚曰:「僕與元公有約,若社稷有難,誓以俱死。今捨去不義。」及世充入,楚匿太官署,執之。世充奮袂令斬,於是鋒刃交下,支體糜碎。



 劉子翊,彭城叢亭里人也。父遍,齊徐州司馬。子翊少好學,頗解屬文。性剛謇,有吏乾。開皇中,為秦州司法參軍。因入考,楊素奏為侍御史。時永寧縣令李公孝,四歲喪母,九歲外繼。其後,父更別娶後妻,至是而亡。河間劉炫以為無撫育之恩,議不解任。子翊駮之曰:《傳》云:「繼母,同母也。」當以配父之尊,居母之位,齊杖之制,皆如親母。又「為人後者為其父母期」,服者,自以本生,非殊親之與繼
 也。父雖自處傍尊之地,於子之情,猶須隆其本重。是以令云:「為人後者,其父母,並解官申其心喪。父卒母嫁,為父後者雖不服,亦申心喪;其繼母嫁,不解官。」此專據嫁者生文耳。將知繼母在父之室,則制同親母。若謂非有撫育之恩,同之行路,何服之有乎?服既有之,心喪焉可獨異?三省令旨,其義甚明。今言令許不解,何其甚謬?



 且後人者為其父母期,未有變隔以親繼,親既等,故心喪不得有殊。《服問》云:「母出,則為繼母之黨服。」豈不以出母族絕,推而遠之;繼母配父,引而親之乎?



 子思曰:「為伋也妻,是為白也母;不為伋也妻,是不為白也母。」定知服以
 名重,情以父親。所以聖人敦之以孝慈,弘之以名義。是使子以名服,同之親母;繼母以義報,等之己生。



 如謂繼母之來,在子出之後,制有淺深者。考之經傳,未見其文。譬出後之人,所後者初亡,後之者至,此後可以無撫育之恩而不服重乎?昔長沙人王毖,漢末為上計詣京師。既而吳、魏隔絕,毖在內國,更娶,生子昌。毖死後,為東平相,始知吳之母亡。便情繫居重,不攝職事。於時議者,不以為非。然則繼之與前,於情無別。若要以撫育始生服制,王昌復何足云乎?又晉鎮南將軍羊祜無子,取弟子伊為子。祜薨,伊不服重。祜妻表聞,伊辭曰:「伯生存養己,
 伊不敢違。然無父命,故還本生。」尚書彭權議:「子之出養,必由父命,無命而出,是為叛子。」於是下詔從之。然則心服之制,不得緣恩而生也。



 論云:「禮者稱情而立文,杖義而設教。」還以此義,諭彼之情。稱情者如母之情,杖義者為子之義。分定然後能尊父順名,崇禮篤敬。茍以姆養之恩,始成母子。則恩由彼至,服自己來。則慈母如母,何待父令?又云:「繼母、慈母,本實路人,臨己養己,同之骨血。」基如斯言,子不由父,縱有恩育,得如母乎?其慈繼雖在三年之下,而居齊期之上。禮有倫例,服以稱情。繼母本以名服,豈藉恩之厚薄也。至於兄弟之子猶子也,私暱
 之心實殊,禮服之制無二。彼言「以」輕「如」



 重,因以不同;此謂如重之辭,即同重法。若使輕重不等,何得為「如」?律云「準枉法」者,但准其罪,「以枉法論」者,即同真法。律以弊刑,禮以設教。



 「准」者准擬之名,「以」者即真之稱。「如」、「以」二字,義用不殊,禮、律兩文,所防是一。將此明彼,足見其義。取譬伐柯,何遠之有。



 論云:「取子為後者,將以供承祧廟,奉養己身。不得使宗子歸其故宅,以子道事本父之後妻也。」然本父後妻,因父而得母稱。若如來旨,本父亦可無心喪乎?



 何直父之後妻也。



 論又云:「《禮》言舊君,其尊豈後君乎?已去其位,非復純臣,須言『舊』以殊之。別有所重,非復
 純孝,故言『其』已見之,目以『其父』之文,是名異也。」



 此又非通論。何以言之?「其」「舊」訓殊,所用亦別。「舊」者易新之稱,「其」



 者因彼之辭,安得以相類哉?至如《禮》云:「其父析薪,其子不克負荷。」《傳》云:「衛雖小,其君在焉。」若其父而有異,其君復有異乎?斯不然矣。



 今炫敢違禮乖令,侮聖干法,使出後之子,無情於本生,名義之分,有虧於風俗,徇飾非於明世,強媒蘗於《禮經》,雖欲揚己露才,不覺言之傷理。



 事奏,竟從子翊之議。



 歷新豐令、大理正,並有能名。擢授書侍御史。每朝廷疑議,子翊為之辯析,多出眾人意表。從幸江東。屬天下大亂,帝猶不悟。子翊因侍切諫,由是忤
 旨,令子翊為丹陽留守。



 尋遣於上江督運,為賊吳棋子所虜。子翊說之,因以眾降。復遣首領賊渡江,遇煬帝被殺,知而告之。子翊弗信,斬所言者。賊又請以為主,不從。因執至臨川城下,使告城中云「帝崩」。子翊乃易其言,於是見害。



 堯君素,魏郡湯陰人也。煬帝為晉王時,君素為左右。帝嗣位,累遷鷹揚郎將。



 大業末,從驍衛大將軍屈突通拒義師於河東。俄而通引兵南遁,置君素領河東通守。



 義師遣將呂紹宗、韋義節等攻之不克。及通軍敗,至城下呼之。君素見通,歔欷流涕,悲不自勝,左右皆哽咽。通亦泣
 下霑襟,因說君素早降以取富貴。君素以名義責之曰:「公縱不能遠慚主上,公所乘馬,即代王所賜也,公何面目乘之哉!」通曰:「吁!君素!我力屈而來。」君素曰:「方今力猶未屈,何用多言!」通慚而退。時圍甚急,行李斷絕。君素乃為木鵝,置表於頸,具論事勢,浮之黃河,沿流而下。河陽守者得之,達於東都。越王侗見而歎息,乃承制拜君素為金紫光祿大夫,密遣行人勞之。監門直閣龐玉、武衛將軍皇甫無逸前後自東都歸義,俱造城下,為陳利害。朝廷又賜金券,待以不死。君素卒無降心。其妻又至城下,謂曰:「隋室已亡,何苦取禍?」君素曰:「天下事非婦人所
 知。」引弓射之,應弦而倒。君素亦知事必不濟,每言及隋國,未嘗不歔欷。常謂將士曰:「吾是籓邸舊臣,至於大義,不得不死。今穀支數年,食盡,足知天下之事。必隋室傾敗,天命有歸,吾當斷頭以付諸君。」後頗得江都傾覆消息,又糧盡,男女相食,眾心離駭。白虹降於府門,兵器之端,夜皆光見。月餘,君素為左右所害。



 陳孝意、張季珣、杜松贇,並以誠節顯。



 孝意,河東人。大業初,為魯郡司法書佐,郡內號為廉平。太守蘇威嘗欲殺一囚,孝意固諫,不許。孝意因解衣先受死。良久,威意乃解,謝而遣之,漸加禮敬。



 及威為納言,奏孝意為侍御
 史。後以父憂去職,居喪過禮,有白鹿馴擾其廬,時人以為孝感。尋起授雁門郡丞。在郡菜食齋居,朝夕哀臨,每一發聲,未嘗不絕倒。



 柴毀骨立,見者哀之。時長吏多贓污,孝意清節彌厲。發姦摘伏,動若有神,吏人稱之。



 煬帝幸江都,馬邑劉武周殺太守王仁恭作亂,前郡丞楊長仁、鴈門令王隺等謀應賊。孝意知之,族滅其家,郡中戰慄。俄而武周來攻,孝意拒之,每致剋捷。



 但孤城無援,而孝意誓以必死。亦知帝必不反,每旦夕向詔敕庫俯伏涕流,悲動左右。糧盡,為校尉張世倫所殺,以城歸武周。



 張季珣,京兆人。父祥,少為隋文帝所知,引為丞相參軍,
 累遷并州司馬。及漢王諒反,遣其將劉建攻之,縱火燒其郭下。祥見百姓驚駭,其城西有王母廟,登城望之,再拜號泣曰:「百姓何罪,致此焚燒?神其有靈,可降雨相救。」言訖,廟上雲起,雨降而火遂滅。士卒感其至誠,莫不用命。援軍至,賊退。以功授開府。



 後卒於都水監。



 季珣少慷慨,有志節。大業末,為鷹揚郎將。所居據箕山為固,與洛口接。及李密陷倉城,遣兵呼之。季珣大罵。密怒,攻之,連年不能剋。經三年,資用盡,無薪,徹屋而爨,人皆穴處。季珣撫之,一無離叛。後士卒飢羸,為密所陷。季珣坐事,顏色自若,密遣兵禽送之。群賊曳令拜密。季珣曰:「吾雖
 敗軍將,猶是天子爪牙臣,何容拜賊!」密壯而釋之。翟讓從求金不得,殺之。



 其弟仲琰,為上洛令。及義兵起,城守,部下殺之以歸義。



 仲琰弟幼琮,為千牛左右。宇文化及亂,遇害。季珣世忠烈,兄弟俱死國難,論者賢之。



 杜松贇,北海人也。性剛烈,重名義。為石門府隊正。大業末,楊厚來攻北海縣,松贇覘賊被執。使謂城中,云「郡兵已破,宜早歸降」,松贇偽許之。既至城下,大呼曰:「我邂逅被執,非力屈也。官軍大來,賊旦暮禽翦。」賊以刀築其口,引之去。松贇罵厚曰:「老賊何敢辱賢良!」言未卒,賊斷其腰。城中望之,莫不流涕扼腕,銳氣益倍,北海卒完。優贈
 朝請大夫、本郡通守。



 郭世俊,字弘乂,太原文水人也。家門雍睦,七世同居,犬豕同乳,鳥鵲同巢,時人以為義感之應。州縣上其事,隋文帝遣平昌公宇文幹詣其家勞問。尚書侍御史柳彧巡省河北,表其門閭。漢王諒為并州總管,聞而嘉歎,賜其兄弟二十餘人衣各一襲。



 郎方貴,淮南人也。少有志尚,與從父弟雙貴同居。隋開皇中,方貴常於淮水津所寄渡,舟人怒之,撾方貴臂折。至家,雙貴問知之,恚恨,遂向津,毆殺船人。



 津者執送之。縣以方貴為首,當死,雙貴從坐,當流。兄弟爭為首坐,縣
 司不能斷,送詣州。兄弟各引死,州不能定。二人爭欲赴水死。州以狀聞。上聞,異之,特原其罪,表其門閭,賜物百段。後為州主簿。



 論曰:於什門等或臨危不撓,視死如歸;或赴險如夷,唯義有所在。其大則光國隆家,其小則損己利物。故其盛烈所著,與河海而爭流;峻節所標,共竹柏而俱茂。並蹈履之所致,身沒名立,豈徒然也!



\end{pinyinscope}