\article{卷五十六列傳第四十四}

\begin{pinyinscope}

 魏
 收魏長賢魏季景子澹魏蘭根族子愷魏收,字伯起,小字佛助,鉅鹿下曲陽人也。自序:漢初魏無知封高良侯,子均。均子恢。恢子彥。彥子歆,字子胡,幼孤,有志操,博洽經史,位終本郡太守。



 子悅,字處德,性沉厚,有度量,宣城公趙國李孝伯見而重之,以女妻焉。位濟陰太守,以善政稱。



 悅子子建,字敬忠,釋褐奉朝請,累
 遷太尉從事中郎。初,宣武時平氏,遂於武興立鎮,尋改為東益州。其後鎮將刺史,乖失人和,群氏作梗,遂為邊患。乃除子建東益州刺史。子建布以恩信,遠近清靜。正光五年,南北二秦城人莫折念生、韓祖香、張長命相繼構逆。僉以州城之人,莫不勁勇,同類悉反,宜先收其器械。



 子建以為城人數當行陣,盡皆驍果,安之足以為用,急之腹背為憂。乃悉召居城老壯,曉示之,并上言諸城人本非罪坐而來者,悉求聽免。明帝優詔從之。子建漸分其父兄子弟,外居郡戍,內外相顧,終獲保全。及秦賊乘勝,屯營黑水,子建乃潛使掩襲,前後斬獲甚眾,威名
 赫然。先反者,及此悉降。乃間使上聞,帝甚嘉之,詔子建兼尚書,為行臺刺史如故。於是威振蜀土。其梁、巴、二益、兩秦之事,皆所節度。



 梁州刺史傅豎眼子敬仲心以為愧,在洛大行貨賄,以圖行臺。先是子建亦屢求歸京師,至此,乃遣刺史唐永代焉。豎眼因為行臺。子建將還,群氏慕戀,相率斷道。主簿楊僧覆先行曉喻,諸氏忿曰:「我留刺史,爾送出也?」斫之數創,幾死。



 子建徐加慰譬,旬月方得前行。吏人贈遺,一無所受。而東益氏、蜀尋反,攻逼唐永,永棄城而走,乃喪一籓矣。初永之走,子建客有沙門云璨及鉅鹿人耿顯皆沒落氏手,及知子建之客,垂
 泣追衣物還之,送出白馬。遺愛所被如此。



 初,子建為前軍將軍,十年不徙,在洛閑暇,與吏部尚書李歆、歆從弟延寔頗為弈棋,時人謂為耽好。子建每曰:「棋於廉勇之際,得之深矣。且吾未為時用,博弈可也。」及一臨邊事,凡經五年,未曾對局。



 還洛後,累遷衛尉卿。初,元顥內逼,莊帝北幸,子建謂所親盧義僖曰:「北海自絕社稷,稱籓蕭衍,吾老矣,豈能為陪臣!」遂攜家口居洛南。顥平乃歸。先苦風痺,及此遂甚。以卿任有務,屢上書乞身,特除右光祿大夫。邢杲之平,太傅李延寔子侍中彧為大使,撫慰東土。時外戚貴盛,送客填門,子建亦往候別。延實曰:「小
 兒今行,何以相助?」子建曰:「益以盈滿為誡。」延實悵然久之。及莊帝殺爾朱榮,遇禍於河陰者,其家率相弔賀。太尉李虔第二子仁曜,子建之女婿,往亦見害。子建謂姨弟盧道虔曰:「朝廷誅翦權強,兇徒尚梗,未聞有奇謀異略,恐不可濟。此乃李門禍始,弔賀無乃匆匆!」及永安之後,李氏宗族流離,或遇誅夷,如其所慮。後歷左光祿大夫,加散騎常侍、驃騎大將軍。



 子建自出為籓牧,董司山南,居脂膏之中,遇天下多事,正身潔已,不以財利經懷。及歸京師,家人衣食,常不周贍,清素之迹,著於終始。性存重慎,不雜交游,唯與尚書盧義僖、姨弟涇州刺史盧
 道裕雅相親暱。及疾篤,顧敕二子曰:「死生大分,含氣所同。世有厚葬,吾平生不取;遽除裸身,又非吾意。氣絕之後,斂以時服。吾平生契闊,前後三娶,合葬之事,抑又非古。且汝二母,先在舊塋,墳地久固,已有定別。唯汝次母墓在外耳,可遷入兆域,依班而定行於吾墓之後,如此足矣,不須附合。當順吾心,勿令吾有遺恨。」永熙二年春,卒於洛陽孝義里舍,時年六十。又贈儀同三司、定州刺史,謚曰文靜。



 二子,收、祚。



 收少機警,不持細行。年十五,頗已屬文。及隨父赴邊,好習騎射,欲以武藝自達。滎陽鄭伯調之曰:「魏郎弄戟多少?」收慚,遂折節讀書。夏月,坐板
 床,隨樹陰諷誦。積年,床板為之銳減,而精力不輟。以文華顯。



 初除太學博士。及爾朱榮於河陰濫害朝士,收亦在圍中,以日晏獲免。吏部尚書李神雋重收才學,奏授司徒記室參軍。永安三年,除北主客郎中。節閔帝立,妙簡近侍,詔試收為封禪書。收下筆便就,不立槁草,文將千言,所改無幾。時黃門郎賈思同侍立,深奇之,白帝曰:「雖七步之才,無以過此。」遷散騎侍郎,尋敕典起居注,并脩國史,俄兼中書侍郎,時年二十六。



 孝武初,又詔收攝本職,文誥填積,事咸稱旨。黃門郎崔甗從齊神武入朝,熏灼於世,收初不詣門。甗為帝登阼赦云:「朕託體孝文。」
 收嗤其率直。正員郎李慎以告之,甗深忿忌。時節閔帝殂,令收為詔。甗乃宣言:收普泰世出入幃心屋,一日造詔,優為詞旨,然則義旗之士,盡為逆人。又收父老,合解官歸侍。南臺將加彈劾,賴尚書辛雄為言於中尉綦俊,乃解。收有賤生弟仲同,先未齒錄,因此怖懼,上籍,遣還鄉扶侍。孝武嘗大發士卒,狩於嵩山之南,旬有六日。時寒,朝野嗟怨。帝與從官及諸妃王,奇伎異飾,多非禮度。收欲言則懼,欲默不能已,乃上《南狩賦》以諷焉,年二十七。雖富言淫麗,而終歸雅正。帝手詔報焉,甚見褒美。



 鄭伯謂曰:「卿不遇老夫,猶應逐兔。」



 神武固讓天柱大將軍,魏
 帝敕收為詔,令遂所請。欲加相國,問收相國品秩,收以實對,帝遂止。收既未測主、相之意,以前事不安,求解,詔許焉。久之,除帝兄子廣平王贊開府從事中郎,收不敢辭,乃為《庭竹賦》以致已意。尋兼中書舍人。與濟陰溫子昇、河間邢子才齊譽,世號「三才」。時孝武內有間隙,收遂以疾固辭而免。舅崔孝芬怪而問之,收曰:「懼有晉陽之甲。」尋而神武南上,帝西入關。



 收兼通直散騎常侍,副王昕聘梁。昕風流文辯,收辭藻富逸,梁主及其群臣咸加敬異。先是,南北初和,李諧、盧元明首通使命,二人才器,並為鄰國所重。至此,梁主稱曰:「盧、李命世,王、魏中興,未
 知後來,復何如耳。」收在館,遂買吳婢入館;其部下有賣婢者,收亦喚取,遍行奸穢。梁朝館司,皆為之獲罪。人稱其才,而鄙其行。在途作《聘游賦》,辭甚美盛。使還,尚書右僕射高隆之求南貨於昕、收,不能如志,遂諷御史中尉高仲密禁止昕、收於其臺,久之得釋。



 及孫搴死,司馬子如薦收,召赴晉陽,以為中外府主簿。以受旨乖懺,頻被嫌責,加以箠楚,久不得志。會司馬子如奉使霸朝,收假其餘光。子如因宴戲言於神武曰:「魏收,天子中書郎,一國大才,願大王借與顏色。」由此轉府屬,然未甚優禮。



 收從叔季景有文學,歷官著名,並在收前,然收常所欺忽。季
 景、收初赴并,頓丘李庶者,故大司農諧之子也,以華辯見稱,曾謂收曰:「霸朝便有二魏。」收率爾曰:「以從叔見比,例邪輸之比卿。」邪輸者,故尚書令陳留公繼伯之子,愚癡有名,好自入市肆,高價買物,商買共所嗤玩。收忽以季景方之,不遜例多如此。



 收本以文才,必望穎脫見知,位既不遂,求脩國史。崔暹為言於文襄曰:「國史事重,公家父子霸王功業,皆須具載,非收不可。」文襄乃啟收兼散騎常侍,脩國史。武定二年,除正常侍,領兼中書侍郎,仍脩國史。



 魏帝宴百僚,問何故名「人日」,皆莫能知。收對曰:「晉議郎董勛答問禮俗云:正月一日為雞,二日為
 狗,三日為豬,四日為羊,五日為牛,六日為馬,七日為人。」時邢邵亦在側,甚恧焉。自魏、梁和好,書下紙每云:「想彼境內寧靜,此率土安和。」梁後使其書乃去「彼」字,自稱猶著「此」,欲示無外之意。收定報書云:「想境內清晏,今萬國安和。」梁人復書,依以為體。



 後神武入朝,靜帝授相國,固讓,令收為啟。啟成呈上,文襄時侍側,神武指收曰「此人當復為崔光。」四年,神武於西門豹祠宴集,謂司馬子如曰:「魏收為史官,書吾善惡,聞北便利時諸貴常餉史官飲食,司馬僕射頗曾餉不?」因共大笑。



 仍謂收曰:「卿勿見元康等在吾目下趨走,謂吾以為勤勞。我後世身名在卿
 手,勿謂我不知。」尋加兼著作郎。



 收昔在京洛,輕薄尤甚,人號云「魏收驚蛺蝶。」文襄曾游東山,令給事黃門侍郎顥等宴。文襄曰:「魏收恃才無宜適,須出其短。」往復數番,收忽大唱曰:「楊遵彥理屈,已倒。」愔從容曰:「我綽有餘暇,山立不動。若遇當塗,恐翩翩遂逝。」當塗者魏,翩翩者蝶也。文襄先知之,大笑稱善。文襄又曰:「向語猶微,宜更指斥。」愔應聲曰:「魏收在并作一篇詩,對眾讀訖,云:『打從叔季景出六百斗米,亦不辨此。』遠近所知,非敢妄說。」文襄喜曰:「我亦先聞。」眾人皆笑。收雖自申雪,不復抗拒,終身病之。



 侯景叛入梁,寇南境。文襄時在晉陽,令收為檄五
 十餘紙,不日而就。又檄梁朝,令送侯景,初夜執筆,三更便了,文過七紙。文襄善之。魏帝曾委秋大射,普令賦詩,收詩末云:「尺書徵建鄴,折簡召長安。」文襄壯之,顧謂人曰:「在朝今有魏收,便是國之光采。雅俗文墨,通達縱橫。我亦使子才、子昇,時有所作,至於詞氣,並不及之。吾或決有所懷,忘而不語,語而不盡,意有未及,收呈草,皆以周悉。此亦難有。」又敕兼主客郎,接梁使謝珽、徐陵。侯景既陷梁,梁鄱陽王範時為合州刺史,文襄敕收以書喻之。範得書,仍率部伍西上,州刺史崔聖念入據其城。文襄謂收曰:「今定一州,卿有其力,猶恨『尺書徵建鄴』未效
 耳。」



 文襄崩,文宣如晉陽,令與黃門郎崔季舒、高德正、吏部郎中尉瑾於北第參掌機密。轉祕書監,兼著作郎,又除定州大中正。時齊將受禪,楊愔奏收置之別館,令撰禪代詔冊諸文,遣徐之才守門,不聽出。



 天保元年,除中書令,仍兼著作郎,封富平縣子。二年,詔撰魏史。四年,除魏尹,故優以祿力,專在史閣,不知郡事。初,帝令群臣各言志,收曰:「臣願得直筆東觀,早出魏書。」故帝使收專其任。又詔平原王高隆之總監之,署名而已。



 帝敕收曰:「好直筆,我終不作魏太武誅史官。」



 始,魏初鄧彥海撰代記十餘卷,其後崔浩典史,游雅、高允、程駿、李彪、崔光、李琰之郎知
 世修其業。浩為編年體,彪始分作紀、表、志、傳,書猶未出。宜武時,命邢巒追撰孝文起居注,書至太和十四年。又命崔鴻、王遵業補續焉,下訖孝明,事甚委悉。濟陰王暉業撰辨宗室錄三十卷。收於是與通直常侍房延祐、司空司馬辛元植、國子博士刁柔、裴昂之、尚書郎高孝乾專總斟酌,以成魏書。辯定名稱,隨條甄舉。又搜採亡遺,綴續後事,備一代史籍,表而上聞之。勒成一代大典:凡十二紀,九十二列傳,合一百一十卷。五年三月,奏上之。秋,除梁州刺史。收以志未成,奏請終業,許之。十一月復奏十志:天象四卷,地形三卷,律歷二卷,禮樂四卷,食貨
 一卷,刑罰一卷,靈徵二卷,官氏二卷,釋老一卷,凡二十卷。續於紀傳,合一百三十卷。分為十二表,其史三十五例,二十五序,九十四論,前後二表一啟,皆獨出於收。



 收所引史官,恐其陵逼,唯取學流先相依附者。其房延祐、辛元植、眭仲讓雖夙涉朝位,並非史才;刁柔、裴昂之以儒業見知,全不堪編緝;高孝乾以左道求進。



 修史諸人,宗祖姻戚,多被書錄,飾以美言。收頗急,不甚能平,夙有怨者,多沒其善。每言:「何物小子,敢共魏收作色!舉之則使上天,按之當使入地。」初,收在神武時為太常少卿,脩國史,得陽休之助。因謝休之曰:「無以謝德,當為卿作佳
 傳。」休之父固,魏世為北平太守,以貪虐為中尉李平所彈獲罪,載在魏起居注。收書云:「固為北平,甚有惠政,坐公事免官。」又云:「李平深相敬重。」



 爾朱榮於魏為賊,收以高氏出自爾朱,且納榮子金,故減其惡而增其善,論云:「若脩德義之風,同韓、彭、伊、霍,夫何足數。」



 時論既言收著史不平,文宣詔收於尚書省與諸家子孫共加論討。前後投訴,百有餘人,云遺其世職位;或云其家不見記錄;或云妄有非毀。收皆隨狀答之。,范陽盧斐父同附出族祖玄傳下;頓丘李庶家傳,稱其本是梁國家人。斐、庶譏議,云史書不直。收性急,不勝其憤,啟誣其欲加屠害。
 帝大怒,親自詰責。斐曰:「臣父仕魏。位至儀同,功業顯著,名聞天下,與收無親,遂不立傳。博陵崔綽,位至本郡功曹,更無事跡,是收外親,乃為傳首。」收曰:「綽雖無位,道義可嘉,所以合傳。」帝曰:「卿何由知其好人?」收曰:「高允曾為綽贊,稱有道德。」帝曰:「司空才士,為人作讚,正應稱揚。亦如卿為人作文章,道其好者,豈能皆實?」



 收無以對,戰慄而已。但帝先重收才,不欲加罪。時太原王松年亦謗史,及斐、庶並獲罪,各被鞭配甲坊,或因以致死。盧思道亦抵罪。然猶以群口沸騰,敕魏史且勿施行,令群官博議。聽有家事者入署,不實者陳牒。於是眾口喧然,號為「穢
 史」,投牒者相次,收無以抗之。時左僕射楊愔、右僕射高德正二人勢傾朝野,與收皆親。



 收遂為其家並作傳,二人不欲言史不實,抑塞拆辭,終文宣世,更不重論。



 又尚書陸操嘗謂愔曰:「魏收魏書可謂博物宏才,有大功於魏室。」愔嘗謂收曰:「此謂不刊之書,傳之萬古。但恨論及諸家枝葉親姻,過為繁碎,與舊史體例不同耳。」收曰:「往因中原喪亂,人士譜牒遺逸略盡,是以具盡其枝派。望公觀過知仁,以免尤責。」



 八年夏,除太子少傅,監國史。復參脩律令。三臺成,文宣曰:「臺成,須有賦。」愔先以告收,收上《皇居新殿臺賦》,其文甚壯麗。時所作者自邢邵已下,
 咸不逮焉。收上賦前數日,乃告邢邵,邵後告人曰:「收甚惡人,不早言之。」帝曾游東山,敕收作詔,宣揚威德,譬喻關西。俄頃而訖,辭理宏壯,帝對百僚大嗟賞之。仍兼太子詹事。收娶其舅女,崔昂之妹,產一女,無子。魏太常劉芳孫女、中書郎崔肇師女,夫家坐事,帝並賜收為妻。時人比之賈充置左右夫人。然無子。



 後病甚,恐身後嫡媵不平,乃放二姬。及疾瘳追憶,作《懷離賦》以申意。



 文宣每以酣宴之次,云太子性懦,宗社事重,終當傳位常山。收謂楊愔曰:「古人云:太子國之根本,不可動搖。至尊三爵後,每言傳位常山,令臣下疑貳。



 若實,便須決行;若戲
 此言,魏收既忝師傅,正當守之以死,但恐國家不安。」愔以收言奏帝,自此便止。帝數宴喜,收每預侍從。皇太子之納鄭良娣也,有司備設牢饌。帝既酣飲,起而自毀覆之,仍詔收曰:「知我意不?」收曰:「臣愚謂良娣既東宮之妾,理不須牢,仰惟聖懷,緣此毀去。」帝大笑,握收手曰:「卿知我意。」



 安德王延宗納趙郡李祖收女為刀,後帝幸李宅宴,而妃母宋氏薦二石榴於帝前。問諸人,莫知其意,帝投之。收曰:「石榴房中多子,王新婚,妃母欲子孫眾多。」



 帝大喜,詔收:「卿還將來。」仍賜收美錦二疋。



 十年,除儀同三司。帝在宴席,口敕以為中書監,命中書郎李愔
 以收一代盛才,難於率爾,久而未訖。比成,帝已醉醒,遂不重言,愔仍不奏,事竟寢。及帝崩於晉陽,驛召收及中山太守陽休之參議吉凶之禮,并掌詔誥。仍除侍中,遷太常卿,文宣謚及廟號、陵名,皆收議也。



 及孝昭居中宰事,命收禁中為諸詔文,積日不出。轉中書監。皇建元年,除兼侍中、右光祿大夫,仍儀同,監史。收先副王昕使梁,不相協睦,時昕弟晞親密,而孝昭別令休之兼中書,在晉陽典誥詔,收留在鄴,蓋晞所為。收大不平,謂太子舍人盧詢祖:「若使卿作文誥,我亦不言。」又除祖珽為著作郎,欲以代收。司空主簿李翥,文詞士也,聞而
 告人曰:「詔誥悉歸陽子烈,著作復遣祖孝徵,文史頓失,恐魏公發背。」於時詔議二王三恪,收執王肅、杜預義,以元、司馬氏為二王,通曹備三恪。詔諸禮學之官皆執鄭玄五代之議。孝昭后姓元,議恪不欲廣及,故議從收。又除兼太子少傅,解侍中。



 帝以魏史未行,詔收更加研審,收奉詔,頗有改正。及詔行魏史,收以為直置祕閣,外人無由得見,於是命送一本付并省,一本付鄴下,任人寫之。



 太寧元年,加開府。河清二年,兼右僕射。時武成酣飲終日,朝事專委侍中高元海,凡庸不堪大任。以收才名振俗,都官尚書畢義雲長於斷割,乃虛心倚仗。收畏避,
 不能匡救,為議者所譏。帝於華林別起玄洲苑,備山水臺觀之麗,詔於閣上畫收,其見重如此。



 始收比溫子昇、邢邵稍為後進,邵既被疏出,子升以罪死,收遂大被任用,獨步一時。議論更相訾毀,各有朋黨。收每議陋邢文。邵又云:「江南任昉,文體本疏,魏收非直模擬,亦大偷竊。」收聞乃曰:「伊常於沈約集中作賊,何意道我偷任。」任、沈俱有重名,邢、魏各有所好。武平中,黃門郎顏之推以二公意問僕射祖珽。珽答曰:「見邢、魏之臧不,即是任、沈之優劣。」收以溫子昇全不作賦,邢雖有一兩首,又非所長,常云:「會須能作賦,始成大才士。唯以章表碑志自許,此
 外更同兒戲。」自武定二年以後,國家大事詔命,軍國文詞,皆收所作。每有警急,受詔立成。或時中使催促,收筆下有同宿構,敏速之工,邢、溫所不逮也。其參議典禮,與邢相埒。



 既而趙郡公增年獲免,收知而過之,事發除名。其年,又以託附陳使封孝琰,牒令其門客與行,遇昆侖舶至,得奇貨:猓然褥表、美玉盈尺等數十件。罪當流,以贖論。三年,起除清郡尹。尋遣黃門郎元文遙敕收曰:「卿舊人,事我家最久,前者之罪,情在可恕。比令卿為尹,非謂美授,但初起卿,斟酌如此。朕豈可用卿之才而忘卿身?待至十月,當還卿開府。」天統元年,除左光祿大夫。二
 年,行齊州刺史,尋為真。



 收以子姪年少,申以戒歷,著枕中篇。其詞曰:吾曾覽管子之書,其言曰:「任之重者莫如身,途之畏者莫如口,期之遠者莫如年。以重任行畏途至遠期,惟君子為能及矣。」追而味之,喟然長息。



 若夫岳立而重,有潛戴而不傾;山藏稱固,亦趨負而不停;呂梁獨浚,能行歌而匪惕;焦原作險,或躋踵而不驚。九陔方集,故眇然而迅舉;五紀當定,想窅乎而上征。茍任重也有度,則任之而愈固。乘危也有術,蓋乘之而靡恤。彼期遠而能通,果應之而可必。豈神理之獨爾,亦人事其如一。



 嗚呼!處天壤之間,勞死生之地,攻之以嗜欲,牽之以
 名利,粱肉不期而共臻,珠玉無足而俱致,於是乎驕奢仍作,危亡旋至。然同上智大賢,惟幾惟哲,或出或處,不常其時。其舒也濟世成務,其卷也聲銷迹滅。玉帛子女,椒蘭律呂,諂諛無所先;稱肉度骨,膏辱挑舌,怨惡莫之前。勳名共山河同久,志業與金石比堅。斯蓋厚棟不橈,游刃砉然。逮於厥德不常,喪其金璞,馳騖人世,鼓動流俗,挾湯日而謂寒,包溪壑而未足。源不清而流濁,表不端而影曲。嗟乎!膠漆詎堅,寒暑甚促,反利而成害,化榮而就辱,欣戚更來,得喪仍續。至有身禦魑魅,魂沉狴獄。



 詎非足力不強,迷在當局!孰可謂車戒前傾,人師先覺?



 聞諸君子,雅道之士,游遨經術,厭飫文史。筆有奇鋒,談有勝理。孝悌之至,神明通矣。審蹈而行,量路而止。自我及物,先人後已。情無繫於榮悴,心靡滯於慍喜。不養望於丘壑,不待價於城市。言行相顧,慎終猶始。有一於斯,鬱為羽儀。



 恪居展事,知無不為,或左或右,則髦士攸宜,無悔無吝,故高而不危。異乎勇進忘退,茍得患失;射千金之產,徼萬鐘之秩;投烈風之門,趣炎火之室。載蹶而墜其貽宴,或蹲乃喪其貞吉。可不畏歟!可不戒歟!



 門有倚禍,事不可不密;墻有伏寇,言不可而失。宜諦其言,宜端其行。言之不善,行之不正,鬼執強梁,人囚徑廷,幽奪
 其魄,明夭其命。不服非法,不行非道。公鼎為己信,私玉非身寶。過涅為紺,踰藍作青,持繩親直,置水觀平。時然後取,未若無欲,知止知足,庶免於辱。是以為必察其幾,舉必慎於微。知幾慮微,斯亡則稀;既察且慎,福祿攸歸。昔蘧瑗識四十九非,顏子鄰幾三月不違。跬步無已,至於千里;覆蕢而進,及於萬仞。故云行遠自邇,登高自卑,可大可久,與世推移。



 月滿如規,後夜則虧;槿榮於枝,望暮而萎。夫奚益而不損?孰有損而不害?



 益不欲多,利不欲大。唯居德者畏其甚,體真者懼其大。道尊則群謗集,任重而眾怨會。其達也則尼父棲遑,其忠也而周公狼狽。無曰人
 之我狹,在我不可而覆;無曰人之我厚,在我不可而咎。如山之大,無不有也;如谷之虛,無不受也。能剛能柔,重可負也;能信能順,險可走也;能智能愚,期可久也。



 周廟之人,三緘其口,漏邑在前,欹器留後,俾諸來裔,傳之坐右。



 其後群臣多言魏史不實,武成復敕更審。收又迴換,遂為盧同立傳,崔綽反更附出。楊愔家傳本云「有魏以來,一門而已」,至是改此八字。又先云「弘農華陰人」,乃改「自云弘農」以配王慧龍「自云太原人」,此其失也。尋除開府、中書監。武成崩,未發喪,在內諸公以後主即位有年,疑於赦令。諸公引收訪焉。收固執宜有恩澤,乃從之。掌
 詔誥,除尚書右僕射,總議監一禮事,位特進。收奏請趙彥深、和士開、徐之才共監,先以告士開,士開驚,辭以不學。收曰:「天下事皆由王,五禮非王不決。」士開謝而許之。多引文士令執筆,儒者馬敬德、熊安生、權會實主之。



 武平三年薨,贈司空、尚書左僕射,謚文貞。有集七十卷。



 收碩學大才,然性褊,不能達命體道。見當塗貴游,每以言色相悅。然提獎後輩,以名行為先,浮華輕險之徒,雖有才能,弗重也。初,河間邢子才、子明及季景與收,並以文章業,世稱大邢小魏,言尤俊也。收少子才十歲,子才每曰:「佛助,僚人之偉。」後收稍與子才爭名,文宣貶子才曰:「
 爾才不及魏收。」收益得志,自序云:「先稱溫、邢,後曰邢、魏。」然收內陋邢,心不許也。收既輕疾,好聲樂,善胡舞。文宣末,數於東山與諸優為獼猴與狗鬥,帝寵狎之。收外兄博陵崔巖嘗以雙聲嘲收曰:「遇魏收衰日愚魏。」魏答曰:「顏巖腥瘦,是誰所生,羊頤狗頰,頭團鼻平,飯房答籠,著孔嘲玎。」其辯捷不拘若是。既緣史筆,多憾於人,齊亡之歲,收塚被發,棄其骨於外。



 先養弟子仁表為嗣,位至尚書膳部郎中。隋開皇中,卒於溫縣令。



 子建族子惇,字仲讓。容貌魁偉,性通率。永安末,除安東將軍、光祿大夫。



 爾硃仲遠鎮東郡,以事捕惇,遇出外,執惇兄子胤而去。惇
 聞哭曰:「若害胤寧無吾也。」乃見仲遠,叩頭曰:「家事在惇,胤何知也?乞以身罪。」仲遠義而捨之。



 天平中,拜衛將軍,右光祿大夫,卒。



 惇叔偃,字盤蚪。有當世乾用,位驍騎將軍。性浮動,晚乃曲附高肇。彭城王勰之死也,偃構成其事,為時所惡。



 子質,字懷素。幼有立志,年十四,啟母求就徐遵明受業,母以其年幼,不許。



 質遂密將一奴,遠赴徐學,留書一紙,置所臥床。內外見之,相視悲歎。五六年中,便通諸經大義。自學言歸,生徒輻湊,皆同衣食,情若兄弟。後避葛榮難,客居趙國飛龍山,為亂賊所害。士友傷惜之。興和二年,侍中李俊、祕書監常景等三十二人申
 辭於尚書,為請贈謚。事下太常,博士考行,謚曰貞烈先生。



 魏長賢,收之族叔也。祖釗,本名顯義,字弘理,魏世祖賜名,仍命以顯義為字。雅性俊辯,博涉群書,有當世才,兼資文武,知名梁、楚、淮、泗之間。世祖南伐,聞而召之,既至,與語大悅。謂釗曰:「今我此行,是卿建功之日,勉之,勿憂不富貴也。」授內都直,侍左右。師次淮南,諸城未有下者。釗乃進曰:「陛下百萬之軍,風行電掃,攻城略地,所向無前,雖有智者,莫能為計。然而師次淮南,已經累日,義陽諸城,猶敢拒守,此非不懼亡滅,自謂必可保全也。但陛
 下卒徒果銳,殺掠尚多,人皆畏威,未甚懷惠,恐一旦降下,妻子不全,所以遲疑,未肯先發。臣請間入城內,見其豪右,宣達聖心,示以誠信,必當大小相率,面縛請罪。陛下拔其英楚,因而任之,此外諸城,可不勞兵而自定。」世祖大喜曰:「所以召卿,本為是耳。卿今所言,副吾所望。」釗遂夜入城中,示以危亡之期,開以生全之路,城中大小欣悅,明旦開門出降。自此而南,望塵款附。世祖謂釗曰:「卿之一言,踰於十萬之師。揚我信義,播於四表,實卿一人之力。」即授義陽太守、陵江將軍。又令釗與諸將,統兵討襲,所當無不摧破,軍中服其勇敢。世祖益喜,謂群臣
 曰:「中國士人,吾拔擢咸盡,文武膽略,未有若釗儔。」加授建忠將軍,追贈其父處順州刺史。時經略江左,方大用之,遇風疾發動,頻降醫藥,竟不痊復。卒時年六十四。



 父彥,字惠卿,博學善屬文。趙郡王乾避開府參軍,廣陵王羽辟記室,並不行。



 陳留公李崇甚重之,引為鎮西參軍事。崇討叛氏陽珍、叛蠻魯北燕,又請為記室參軍。中山王英討淮南,又請為記室參軍。軍還,求為著作郎,思樹不朽之業。以晉書作者多家,體制繁雜,欲正其紕繆,刪其遊辭,勒成一家之典。俄而彭城王聞李崇稱之,復請為掾,兼知主客郎中,書遂不成。王遇害,退歸田里。清
 河王復引為諮議。王勢高名重,深為權倖所疾,恐罹其禍,固辭以疾。肅宗初,拜驃騎長史,尋轉光州刺史。年六十八,卒。



 兄伯胤之歸也,留長賢與弟德振,使宦學於洛中。孝靜北遷,亦徙居鄴。博涉經史,詞藻清華,舉秀才,除汝南王悅參軍事。入齊,平陽王淹辟為法曹參軍,轉著作佐郎。更撰晉書,欲還成先志。



 河清中,上書譏刺時政,大忤權幸,為上黨屯留令。親故以長賢不相時而動,或為書以相規責。長賢復書曰:日者惠書,義高旨遠。誨僕以自求諸已,思不出位,國之大事,君與執政所圖。



 又謂僕祿不足以代耕,位不登於執戟,乾非其議,自貽悔咎。
 勤勤懇懇,誠見故人之心。靜言再思,無忘寤寐。



 僕雖固陋,亦嘗奉教於君子矣。以為士之立身,其路不一。故有負鼎俎以趨世,隱漁釣以待時,操築傅巖之下,取履圯橋之上者矣。或有釋賃車以匡霸業,委挽輅以定王基,由斬祛以見禮,因射鉤而受相者矣。或有三黜不移,屈身以直道;九死不侮,甘心於苦節者矣。皆奮於泥滓,自致青雲。雖事有萬殊,而理終一致,榷其大要,歸乎忠孝而已矣。



 夫孝則竭力所生,忠則致身所事,未有孝而遺其親,忠而後其君者也。僕自射策金馬,記言麟閣,寒暑迭運,五稔於茲。不能勒成一家,潤色鴻業,善述人事,功
 既闕如,顯親揚名,邈焉無冀。每一念之,曷云其已。自頃王室板蕩,彞倫攸斁,大臣持祿而莫諫,小臣畏罪而不言,虛痛朝危,空哀主辱。匪躬之故,徒聞其語;有犯無隱,未見其人。此梅福所以獻書,朱雲所以請劍者也。抑又聞之,嫠不恤緯而憂宗周之亡,女不懷歸而悲太子之少,況僕之先人,世傳儒業,訓僕以為子之道,歷僕以事君之節?今僕之委質,有年世矣,安可自同於匹庶,取笑於兒女子哉!是以腸一夕而九回,心終朝而百慮,懼當年之不立,恥沒世而無聞,慷慨懷古,自強不息,庶几伯夷之風,以立懦夫之志。吾子又謂僕干進務入,不畏友
 朋;居下訕上,欲益反損。僕誠不敏,以貽吾子之羞,默默茍容,又非平生之意。故願得鋤彼草茅,逐茲鳥雀,去一惡,樹一善,不違先旨,以沒九泉。求仁得仁,其誰敢怨?



 但言與不言在我,用與不用在時。若國道方屯,時不我與,以忠獲罪,以信見疑,貝錦成章,青蠅變色,良田敗於邪徑,黃金鑠於眾口,窮達運也,其如命何!



 吾子忠告之言,敢不敬承嘉惠。然則僕之所懷,未可一二為俗人道也。投筆而已,乂復何言!



 是出也,人皆為之怏怏,而長賢處之怡然,不屑懷抱,識者以此多焉。



 武平中,辭疾去職,終於齊代,不復出仕。周武平齊,搜揚才俊,辟書屢降,固以
 疾辭。卒年七十四。貞觀中,贈定州刺史。子征。



 魏季景,收族叔也。父鸞字雙和,為魏文賜名。有器幹,體貌魁偉,以有容儀,為奉車都尉。曾升輅車,觸毀金翼,斂容請罪。帝笑曰:「卿體貌過人,素不便習,何足懼也?」車駕南征漢陽,除鸞統軍。帝歷幸其營,嘆賞之。及在馬圈不豫,敕兼武衛將軍,領宿衛左右。景明中,六輔之廢,鸞頗預其事。後除光州刺史,更滿還朝,卒。謚曰夷。子季景少孤,清苦自立,博學有文才,弱冠有名京師。時邢子明稱有才學,殆與子才相侔,季景與收相亞,洛中號兩邢二魏。莊帝時,為中書侍郎。普泰中,為尚書右丞。季景善附
 會,宰要當朝,必先事其左右。爾朱世隆特賞愛之。於時才名甚盛,頗過其實。太昌中,位給事黃門侍郎,甚見信待,除定州大中正。孝武帝釋奠,季景與溫子昇、李業興、竇瑗等俱為摘句。天平初,因遷都,遂居柏人西山。內懷憂悔,乃為擇居賦。元象初,兼給事黃門侍郎,後兼散騎常侍,使梁。還,歷大司農卿、魏郡尹。卒,家無餘財,遺命薄葬,贈散騎常侍、衛尉卿。



 所著文筆二百餘篇。子澹知名。



 澹子彥深。年十五而孤,專精好學,高才善屬文。仕齊,殿中侍御史,預修五禮,及撰御覽。除殿中郎、中書舍人,與李德林脩國史。入周為納言中士。隋初,為行臺禮部侍
 郎,尋為聘陳使主。還,除太子舍人。廢太子勇深禮之,令注庚信集,撰笑苑,世稱博物。遷著作郎,仍為太子學士。



 帝以魏收所撰後魏書褒貶失實,平繪為中興書事不倫序,詔澹別成魏史。澹自道武下及恭帝,為十二紀,七十八列傳。別為史論及例,各一卷,合九十二卷。義例與魏收多所不同。



 其一曰:「臣聞天子者繼天立稱,終始絕。故穀梁傳:『太上不名。』曲禮:『天子不言出,諸侯不生名。』諸侯尚不生名,況天子乎?若為太子,必須書名。



 良由子者對父生稱,父前子名,禮之意也。至如馬遷,周之太子,並皆言名,漢之儲兩,俱沒其諱,以尊漢卑周,臣子之意
 也。竊謂雖立此理,恐非其義。何者?春秋、禮記,太子必書名,天王不言出,此仲尼之褒貶,皇王之稱謂,非當時與異代,遂為優劣也。班固、范曄、陳壽、王隱、沈約參差不同,尊卑失序。至於魏收諱儲君之名,書天子之字,過又甚焉。今所撰,諱皇帝名,書太子字,欲尊君卑臣,依春秋之義。」



 二曰「魏氏平文以前,部落之君長耳。太祖遠追二十八帝,並極崇高,違堯舜憲章,越周公典禮。但道武出自結繩,未師典誥,當須南董直筆,裁而正之;反更飾非,豈是觀過?但力微天女所誕,靈異絕世,尊為始祖,得禮之宜。平文、昭成,雄據塞表,英風漸盛,圖南之業,基自此始。
 長孫斤之亂也,兵交御坐,太子授命,昭成獲免。道武此時,后緡方娠,宗廟復存,社稷有主,大功大孝,實在獻明。此之三世,稱謚可也;自茲以外,未之敢聞。」



 其三曰:「幽王死於驪山,厲王出奔於彘,未嘗隱諱,直筆書之,欲以勸善懲惡,詒誡將來。而太武、獻文,並遭非命,前史立紀,不異天年,言論之間,頗露首尾。殺主害君,莫知姓名,逆臣賊子,何所懼哉?今分明直書,不敢回避。」



 四曰:「自晉德不競,宇宙分崩,或帝或王,各自署置。其生略如敵國,書死便同庶人。凡處華夏之地者,皆書曰卒,同之吳、楚。」



 澹又以為「司馬遷創立紀傳已來,述者非一,人無善惡,皆為
 立論。計在身行迹,具在正書,事既無奇,不足懲勸,再述乍同銘頌,重敘唯覺繁文。案丘明亞聖之才,發揚聖旨,言『君子曰』者,無非甚泰;其間尋常,直言而已。今所撰史,竊有慕焉,可為勸戒者,論其得失;其無損益者,所不論也。」上覽而善之。未幾而卒。有集三十卷。子罕言。



 澹弟彥玄,位洧州司馬。子滿行。



 魏蘭根,字蘭根,收族叔也。父伯成,中山太守。蘭根身長八尺,儀貌奇偉,博學高才,機警有識悟。起家北海王國侍郎。母憂,居喪有孝稱。將葬,常山郡境先有董卓祠,祠有柏樹,蘭根以卓凶逆,不應遺祠至今,乃啟刺史,請伐
 為槨。左右人言有靈,蘭根了無疑懼。父喪,廬於墓側,負土成墳,憂毀殆於滅性。



 正光末,尚書令李崇為大都督,討蠕蠕,以蘭根為長史。因說崇曰:「緣邊諸鎮,控攝長遠,昔時初置,地廣人稀,或徵發中原強宗子弟,或國之肺腑寄以爪牙。



 中年以來,有司乖實,號曰府戶,役同廝養,官婚班齒,致失清流。而本宗舊類,各各榮顯,顧瞻彼此,理當憤怨。宜改鎮立州,分置郡縣。凡是府戶,悉免為平人,入仕次第,一準其舊。此計若行,國家庶無北顧之慮。」崇以奏聞,事寢不報。



 孝昌初,為岐州刺史,從行臺蕭實夤討破宛川。俘其人為奴婢,以美女十人賞蘭根。蘭根
 辭曰:「此縣介於強虜,故成背叛。今當恤其飢寒,奈何並充僕隸?」



 於是盡以歸其父兄。部內麥多五穗。鄰州田鼠為災,犬牙不入岐境。及蕭寶夤敗於涇州,岐州人囚蘭根降賊。寶夤兵威復振,城人復斬賊刺史侯莫陳仲和,推蘭根復任。朝廷以蘭根得西土人心,加都督涇、岐、東秦、南岐四州諸軍事,兼四州行臺尚書。



 孝昌末,河北流人南度,以蘭根兼尚書,使齊、濟、二兗四州安撫,并置郡縣。



 蘭根甥邢杲反於青、光間,復詔蘭根慰勞。杲不下,仍隨元天穆討之。還,拜中書令。



 莊帝之將誅爾朱榮,蘭根泄之於兄子周達,周達告爾朱世隆。及榮死,蘭根憂,不
 知所出。時應詔王道習見信於莊帝,蘭根乃託附之,求出立功。乃兼尚書右僕射、河北行臺,定州率募鄉曲,欲防井陘。為榮將侯深所敗,走依勃海高乾。屬乾兄弟義舉,固在其中。神武以宿望深禮之。中興初,為尚書右僕射。神武將入洛陽,時廢立未決,令蘭根察節閔帝。帝神採高明,蘭根恐於後難測,遂與高乾兄弟及黃門侍郎崔甗同請。神武不得已,遂立武帝。太昌初,加侍中、開府儀同三司、鉅鹿縣侯,啟授兄子周達。蘭根既預勳業,位居端副,始敘復岐州勳,封永興侯。高乾之死,蘭根懼,以病免。天平初,言病篤,以開府儀同歸本鄉,門施行馬。武
 定三年,薨。贈司徒公,謚曰文宣。長子相如襲爵。



 相如性亢直,有文藻,與族兄愷齊名,雅為當時所貴。早卒。孝昭時,佐命功臣配饗,不及蘭根,次子敬仲表訴,竟不允。敬仲以才器稱,卒於章武太守。子餉,字孝衡。幼孤,學涉有時譽,居喪以孝聞。隋饒州司倉參軍事。子景義、景禮並有才行,鄉人呼為雙鳳,早卒。敬仲弟少政,位至洛州刺史。子孝該、孝幾。



 愷自散騎常侍遷青州長史,固辭。文宣大怒曰:「何物漢子,與官不就!」時帝已失德,朝廷為之懼,愷容色坦然。帝曰:「死與長史,任卿所擇。」答曰:「能殺臣者陛下,不受長史者愚臣。」帝謂楊愔曰:「何慮無人,苦用此
 漢!放還,永不須收。」由是積年沈廢。後遇愔於路,微自陳。愔云:「咸由中旨。」愷應聲曰;「雖復零雨自天,終待雲興四岳,公豈得言不知?」楊愔欣然曰:「此言極為簡要。」數日,除霍州刺史,在職有政理。後卒於膠州刺史。



 論曰:伯起少頗疏放,不拘行檢,及折節讀書,鬱為偉器。學博今古,才極從橫,體物之旨,尤為富贍,足以入相如之室,游尼父之門。勒成魏籍,追從班、馬,婉而有則,繁而不蕪,持論序言,鉤深致遠。但意存實錄,好抵陰私,到於親故之家,一無所說,不平之議,見於斯矣。王松年、李庶等並論正家門,未為謗議,遂憑附時宰,鼓動淫刑,庶因
 鞭撻而終,此公之失德。長賢思樹風聲,抗言昏俗,有硃子游之風。季景父子,雅業相傳,抑弓冶之義。蘭根道冠時英,功參霸業,亦一代之偉人也。



\end{pinyinscope}