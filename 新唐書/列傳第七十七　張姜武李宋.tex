\article{列傳第七十七 張姜武李宋}

\begin{pinyinscope}

 張鎰,字季權,一字公度,國子祭酒後胤五世孫也。父齊丘,朔方節度使、東都留守。鎰以廕授左衛兵曹參軍歌劇六部。主要著作有《論科學和藝術是否敗壞或增進道,郭子儀表為元帥府判官,遷累殿中侍御史。乾元初,華原令盧樅以公事譙責邑人齊令詵。令詵,宦人也,銜之,構樅罪。鎰按驗當免官,有司承風以死論。鎰不直之,乃白其母曰:「今理樅,樅免死而鎰坐貶。嘿則負官,貶則為太夫人憂,敢問所安?」母曰:「兒無累於道,吾所安也。」遂執正其罪,樅得流,鎰貶撫州司戶參軍。徙晉陵令。江西觀察使張鎬表為判官,遷屯田、右司二員外郎。居母喪,以孝聞。不妄交游,特與楊綰、崔祐甫善。



 大歷初,出為濠州刺史,政條清簡,延經術士講教生徒。比去,州升明經者四十人。李靈耀反於汴,鎰團閱鄉兵嚴守御,有詔褒美,擢侍御史,兼緣淮鎮守使。以最遷壽州刺史。歷江西、河中觀察使。不閱旬,改汴滑節度使,以病固辭,詔留私第。



 建中二年,拜中書侍郎、同中書門下平章事。明年,以兩河用兵,詔省薄御膳及皇太子食物,鎰因奏減堂餐錢及百官稟奉三分一,以助用度。時黜陟使裴伯言薦潞州處士田佐時,詔除右拾遺、集賢院直學士。鎰以為禮輕,恐士不勸,復詔州縣吏以絹百匹、粟百石就家致聘,佐時卒不至。



 郭子儀婿太僕卿趙縱為奴告,下御史劾治,而奴留內侍省。鎰奏言:「貞觀時有奴告其主謀反者,太宗曰:『謀反理不獨成,尚當有他人論之,豈藉奴告耶?』乃著令:奴告主者斬。由是賤不得干貴,下不得凌上,教本既修,悖亂不萌。頃者,長安令李濟以奴得罪,萬年令霍晏因婢坐譴。輿臺下類,主反畏之,悖慢成風,漸不可長。建中元年五月辛卯詔書:奴婢告主,非謀叛者,同自首法,並準律論。由是獄訴衰息。今縱事非叛逆,而奴留禁中,獨下縱獄,情所不厭。且將帥功孰大於子儀,塚土僅乾,兩婿前已得罪,縱復繼之,不數月斥其三婿。假令縱實犯法,事不緣奴,尚宜錄勛念亡,以從蕩宥,況為奴所訴耶?陛下方貴武臣以討賊,彼雖見寵一時,不能忘懷於異日也。」帝納之,貶縱循州司馬,杖奴死。鎰召子儀家僮數百,暴示奴尸。



 盧杞忌鎰剛直,欲去之。時硃泚以盧龍卒戍鳳翔,帝擇人以代,杞即謬曰:「鳳翔將校,班秩素高,非宰相信臣,不可鎮撫,臣宜行。」帝不許,杞復曰:「陛下必以臣容貌蕞陋,不為三軍所信,恐後生變,臣不敢自謀,惟陛下擇之。」帝乃顧鎰曰:「文武兼資,望重內外,無易卿者,其為朕撫盧龍士。」乃以中書侍郎為鳳翔、隴右節度使。鎰知為杞陰中,然辭窮,因再拜受詔。頃之,與吐蕃相尚結贊盟清水,約牛馬為牲。鎰恥與盟,將末殺其禮,乃紿語吐蕃,以羊豕犬代之。



 帝幸奉天,鎰罄家貲將自獻行在。而營將李楚琳者,嘗事硃泚,得其心。軍司馬齊映等謀曰:「楚琳必為亂。」乃遣屯隴州。楚琳知之,稽故未行。鎰以帝在外,心憂惑,謂已亟去,不為備。楚琳夜率其黨王汾、李卓、牛僧伽等作亂,齊映自竇出,齊抗托傭,皆免。鎰縋城走,不及遠,與二子為候騎所執,楚琳殺之,屬官王沼、張元度、柳遇、李漵皆死。詔贈鎰太子太傅。



 姜公輔,愛州日南人。第進士,補校書郎,以制策異等授右拾遺,為翰林學士。歲滿當遷,上書以母老賴祿而養,求兼京兆戶曹參軍事。公輔有高材,每進見,敷奏詳亮,德宗器之。



 硃滔助田悅也,以蜜裹書間道邀泚,太原馬燧獲之,泚不知也,召還京師。公輔諫曰:「陛下若不能坦懷待泚,不如誅之,養虎無自詒害。」不從。俄而涇師亂,帝自苑門出,公輔叩馬諫曰:「泚嘗帥涇原,得士心,向以滔叛奪之兵,居常怫鬱不自聊,請馳騎捕取以從,無為群兇得之。」帝倉卒不及聽。既行,欲駐鳳翔倚張鎰。公輔曰:「鎰雖信臣,然文吏也,所領皆硃泚部曲,漁陽突騎,泚若立,涇軍且有變,非萬全策也。」帝亦記桑道茂言,遂之奉天。不數日,鳳翔果亂,殺鎰。帝在奉天,有言泚反者,請為守備。盧杞曰:「泚忠正篤實,奈何言其叛,傷大臣心!請百口保之。」帝知群臣多勸杞奉迎乘輿者,乃詔諸道兵距城一舍止。公輔曰:「王者不嚴羽衛,無以重威靈。今禁旅單寡而士馬處外,為陛下危之。」帝曰:「善。」悉內諸軍。泚兵果至,如所言,乃擢公輔諫議大夫、同中書門下平章事。



 帝徙梁,唐安公主道薨。主性仁孝,許下嫁韋宥,以播遷未克也。帝悼之甚,詔厚其葬。公輔諫曰:「即平賊,主必歸葬,今行道宜從儉,以濟軍興。」帝怒,謂翰林學士陸贄曰:「唐安之葬,不欲事塋壟,令累甓為浮圖,費甚寡約,不容宰相關預,茍欲指朕過爾!」贄曰:「公輔官諫議,職宰相,獻替固其分。本立輔臣,朝夕納誨,微而弼之,乃其所也。」帝曰:「不然,朕以公輔才不足以相,而又自求解,朕既許之,內知且罷,故賣直售名爾。」遂下遷太子左庶子,以母喪解。復為右庶子。



 久不遷,陸贄為相,公輔數求官,贄密謂曰:「竇丞相嘗言,為公擬官屢矣,上輒不悅。」公輔懼,請為道士,未報。它日又言之,帝問故,公輔隱贄言,以參語對。帝怒,黜公輔泉州別駕,遣使齎詔讓參。順宗立,拜吉州刺史,未就官卒。憲宗時,贈禮部尚書。



 武元衡,字伯蒼。曾祖載德,則天皇后之族弟。祖平一,有名。元衡舉進士,累為華原令。畿輔鎮軍督將,皆驕橫橈政,元衡移疾去。德宗欽其才,召拜比部員外郎,歲內三遷至右司郎中,以詳整任職。擢為御史中丞。嘗對延英,帝目送之,曰:「是真宰相器!」



 順宗立,王叔文使人誘以為黨,拒不納。俄為山陵儀仗使,監察御史劉禹錫求為判官,元衡不與,叔文滋不悅。數日,改太子右庶子。會冊皇太子,元衡贊相,太子識之。及即位,是為憲宗,復拜中丞,進戶部侍郎。元和二年,拜門下侍郎、同中書門下平章事,兼判戶部事。帝素知元衡堅正有守,故眷禮信任異它相。浙西李錡求入覲,既又稱疾,欲賒其期。帝問宰相鄭絪,絪請聽之,元衡曰:「不可,錡自請入朝,詔既許之,而復不至,是可否在錡。陛下新即位,天下屬耳目,若奸臣得遂其私,則威令去矣。」帝然之,遽追錡。而錡計窮,果反。



 是時,蜀新定,高崇文為節度,不知吏治,帝難其代。詔元衡檢校吏部尚書,兼門下侍郎、同平章事,為劍南西川節度使,繇蕭縣伯封臨淮郡公,帝御安福門慰遣之。崇文去成都,盡以金帛、帟幕、伎樂、工巧行,蜀幾為空。元衡至,綏靖約束,儉己寬民,比三年,上下完實,蠻夷懷歸。雅性莊重,雖淡於接物,而開府極一時選。



 八年,召還秉政。李吉甫、李絳數爭事帝前,不葉,元衡獨持正無所違附,帝稱其長者。吉甫卒,淮、蔡用兵,帝悉以機政委之。王承宗上疏請赦吳元濟,使人白事中書,悖慢不恭,元衡叱去。承宗怨,數上章誣詆。未幾入朝,出靖安里第,夜漏未盡,賊乘暗呼曰:「滅燭!」射元衡中肩,復擊其左股,徒御格鬥不勝,皆駭走,遂害元衡,批顱骨持去。邏司傳噪盜殺宰相,連十餘里,達朝堂,百官恟懼,未知主名。少選,馬逸還第,中外乃審知。是日,仗入紫宸門,有司以聞,帝震驚,罷朝,坐延英見宰相,哀慟,為再不食。贈司徒,謚曰忠愍。詔金吾、府、縣大索,或傳言曰:「無搜賊,賊窮必亂。」又投書於道曰:「毋急我,我先殺汝。」故吏卒不窮捕。兵部侍郎許孟容言於帝曰:「國相橫尸路隅而盜不獲,為朝廷辱。」帝乃下詔:「能得賊者賞錢千萬,授五品官。與賊謀及舍賊能自言者亦賞。有不如詔,族之。」積錢東西市以募告者。於是左神策將軍王士則、左威衛將軍王士平以賊聞,捕得張晏等十八人,言為承宗所遣,皆斬之。逾月,東都防禦使呂元膺執淄青留邸賊門察、訾嘉珍,自言始謀殺元衡者,會晏先發,故藉之以告師道而竊其賞,帝密誅之。



 初,京師大恐,城門加兵誰何,其偉狀異服、燕趙言者,皆驗訊乃遣。公卿朝,以家奴持兵呵衛,宰相則金吾彀騎導翼,每過里門,搜索喧嘩。因詔寅漏上二刻乃傳點雲。



 從父弟儒衡。儒衡,字廷碩,姿狀秀偉,不妄言,與人交,終始一節。宰相鄭餘慶不事華潔,門下客多垢衣敗服,獨儒衡上謁,未嘗有所易,以莊詞正色見重於餘慶。元衡歿,帝待之益厚,累遷戶部郎中,知諫議大夫事,俄兼知制誥。皇甫鎛以宰相領度支,剝下以媚天子,儒衡疏其狀。鎛自訴於帝,帝曰:「乃欲報怨邪?」鎛不敢對。



 儒衡論議勁正,有風節,且將大用。宰相令狐楚忌之,會以狄兼謨為拾遺,楚自草制,引武后革命事,盛推仁傑功,以指切儒衡,且沮止之。儒衡泣見上曰:「臣祖平一,當天后時,避仕終老,不涉於累。」帝慰勉之,自是薄楚為人也。遷中書舍人。時元稹倚宦官,知制誥,儒衡鄙厭之。會食瓜,蠅集其上,儒衡揮以扇,曰:「適從何處來,遽集於此?」一坐皆失色。然以疾惡太分明,終不至大任,以兵部侍郎卒,年五十六,贈工部尚書。



 李絳,字深之,系本贊皇。擢進士、宏辭,補渭南尉,拜監察御史。元和二年,授翰林學士,俄知制誥。會李錡誅,憲宗將輦取其貲,絳與裴垍諫曰:「錡僭侈誅求,六州之人怨入骨髓。今元惡傳首,若因取其財,恐非遏亂略、惠綏困窮者。願賜本道,代貧民租賦。」制可。樞密使劉光琦議遣中人持赦令賜諸道,以裒饋餉,絳請付度支鹽鐵急遞以遣,息取求之弊。光琦引故事以對,帝曰:「故事是耶,當守之;不然,當改。可循舊哉!」



 帝嘗稱太宗、玄宗之盛:「朕不佞,欲庶幾二祖之道德風烈,無愧謚號,不為宗廟羞,何行而至此乎?」絳曰:「陛下誠能正身勵己,尊道德,遠邪佞,進忠直。與大臣言,敬而信,無使小人參焉;與賢者游,親而禮,無使不肖與焉。去官無益於治者,則材能出;斥宮女之希御者,則怨曠銷。將帥擇,士卒勇矣;官師公,吏治輯矣。法令行而下不違,教化篤而俗必遷。如是,可與祖宗合德,號稱中興,夫何遠之有?言之不行,無益也;行之不至,無益也。」帝曰:「美哉斯言,朕將書諸紳。」即詔絳與崔群、錢徵、韋弘景、白居易等搜次君臣成敗五十種,為連屏,張便坐。帝每閱視,顧左右曰:「而等宜作意,勿為如此事。」



 是時,盛興安國佛祠,幸臣吐突承璀請立石紀聖德焉,營構華廣,欲使絳為之頌,將遺錢千萬。絳上言:「陛下蕩積習之弊,四海延頸望德音,忽自立碑,示人以不廣。《易》稱:『大人與天地合德。』謂非文字所能盡,若令可述,是陛下美有分限。堯、舜至文、武,皆不傳其事,惟秦始刻嶧山,揚暴誅伐巡幸之勞,失道之君,不足為法。今安國有碑,若敘游觀,即非治要;述崇飾,又非政宜。請罷之。」帝怒,絳伏奏愈切,帝悟曰:「微絳,我不自知。」命百牛倒石,令使者勞諭絳。襄陽裴均違詔書,獻銀壺甕數百具,絳請歸之度支,示天下以信。帝可奏,仍赦均罪。時議還盧從史昭義,已而將復召之,從史以軍無見儲為解。李吉甫謂鄭絪漏其謀,帝召絳議,欲逐絪,絳為開白,乃免。



 絳見浴堂殿,帝曰:「比諫官多朋黨,論奏不實,皆陷謗訕,欲黜其尤者,若何?」絳曰:「此非陛下意,必憸人以此營誤上心。自古納諫昌,拒諫亡。夫人臣進言於上,豈易哉?君尊如天,臣卑如地,加有雷霆之威,彼晝度夜思,始欲陳十事,俄而去五六,及將以聞,則又憚而削其半,故上達者財十二。何哉?乾不測之禍,顧身無利耳。雖開納獎勵,尚恐不至,今乃欲譴訶之,使直士杜口,非社稷利也。」帝曰:「非卿言,我不知諫之益。」



 初,承璀討王承宗,議者皆言古無以宦人統師者,絳當制書,固爭,帝不能奪,止詔宰相授敕。承璀果無功還,加開府儀同三司。絳奏:「承璀喪師,當抵罪,今寵以崇秩,後有奔軍之將,蹈利干賞,陛下何以處之?」又數論宦官橫肆,方鎮進獻等事。自知言切,且斥去,悉取內署所上疏稿焚之,以俟命。帝果怒,絳謝曰:「陛下憐臣愚,處之腹心之地,而惜身不言,乃臣負陛下;若上犯聖顏,旁忤貴幸,因而獲罪,乃陛下負臣。」於是帝動容曰:「卿告朕以人所難言者,疾風知勁草,卿當之矣。」遂繇司勛郎中進中書舍人。翌日,賜金紫,親擇良笏與之,且曰:「異時膺顧托南面,當如此。」絳頓首。



 烏重胤縛盧從史,而承璀牒署昭義留後,絳曰:「澤潞據山東要害,磁、邢、洺跨兩河間,可制其合從。今孽豎就禽,方收威柄,遽以偏將蒞本軍,綱紀大紊矣。河南、北諸鎮,謂陛下啗以官爵,使逐其帥,其肯默然哉?宜以孟元陽為澤潞,而以重胤節度三城,兩河諸侯聞之,必欣然。」帝從之。



 張茂昭舉族入覲,絳上言:「任迪簡既往代,則士之從茂昭,皆為定人,宜亟授以官,且遣使者詔其麾下皆聽茂昭節度。」有詔拜河中節度使。會迪簡以帑廥匱竭,稍簡罷士之疲老者,人情不安,迪簡亦危,絳請斥禁帑絹十萬以濟事機。吳少誠病甚,絳建言:「淮西地不與賊接,若朝廷命帥,今乃其時,有如阻命,則決可討矣。然鎮、蔡不可並取,願赦承宗,趣立蔡功。」時江淮大旱,帝下赦令有所蠲弛,絳言:「江淮流亡,所貸未廣,而宮人猥積,有怨鬲之思,當大出之,以省經費。嶺南之俗,鬻子為業,可聽;非券劑取直者,如掠賣法,敕有司一切苛止。」帝皆順納。



 後閱月不賜對,絳謂:「大臣持祿不敢諫,小臣畏罪不敢言,管仲以為害霸最甚。今臣等飽食不言,無履危之患,自為計得矣,顧聖治如何?」有詔明日對三殿。帝嘗畋苑中,至蓬萊池,謂左右曰:「絳嘗以諫我,今可返也。」其見禮憚如此。



 帝怪前世任賢以致治,今無賢可任,何耶?對曰:「聖王選當代之人,極其才分,自可致治。豈借賢異代,治今日之人哉?天子不以己能蓋人,痛折節下士,則天下賢者乃出。」帝曰:「何知其必賢而任之?」對曰:「知人誠難,堯、舜以為病。然循其名,驗以事,所得十七。夫任官而辨廉,措事不阿容,無希望依違之辭,無邪媚愉悅之容,此近於賢矣。賢則當任,任則當久。賢者中立而寡助,舉其類則不肖者怨,杜邪徑則懷奸者疾,一制度則貴戚毀傷,正過失則人君疏忌。夫然,用賢豈容易哉?」帝曰:「卿言得之矣。」



 六年,罷學士,遷戶部侍郎,判本司。帝以戶部故有獻,而絳獨無有,何哉?答曰:「凡方鎮有地則有賦,或嗇用度易羨餘以為獻。臣乃為陛下謹出納,烏有羨贏哉?若以為獻,是徙東庫物實西庫,進官物結私恩。」帝瞿然悟。帝每有詢訪,隨事補益,所言無不聽,欲遂以相。而承璀寵方盛,忌其進,陰有毀短,帝乃出承璀淮南監軍。翌日,拜絳中書侍郎、同中書門下平章事。封高邑男。方江淮歲儉,民薦饑,有御史使還,奏不為災,帝以語絳,答曰:「方隅皆陛下大臣,奏孰不實?而御史茍悅陛下耳。凡君人者當任大臣,無使小臣得以間,願出其名顯責之。」李吉甫嘗盛贊天子威德,帝欣然,絳獨曰:「陛下自視今日何如漢文帝時?」帝曰:「朕安敢望文帝?」對曰:「是時賈誼以為措火積薪下,火未及然,因以為安,其憂如此。今法令所不及者五十餘州,西戎內訌,近以涇、隴為鄙,去京師遠不千里,烽燧相接也;加比水旱無年,倉廩空虛。誠陛下焦心銷志求濟時之略,渠便高枕而臥哉!」帝入謂左右曰:「絳言骨鯁,真宰相也。」遣使者賜酴醾酒。



 魏博田季安死,子懷諫弱,軍中請襲節度,吉甫議討之,絳曰:「不然,兩河所懼者,部將以兵圖己也,故委諸將總兵,皆使力敵任均,以相維制,不得為變。若主帥強,則足以制其命。今懷諫乳方臭,不能事,必假權於人,權重則怨生,向之權力均者,將起事生患矣。眾所歸必在寬厚簡易、軍中素所愛者,彼得立,不倚朝廷亦不能安。惟陛下蓄威以俟之。」俄而田興果立,以魏博聽命,帝大悅。吉甫復請命中人宣尉,因刺其變,徐議所宜。絳獨謂:「不如推誠撫納,即假旄節。它日使者持三軍表來,請與興,則制在彼,不在此,可奏與特授,安得同哉?」然帝重違吉甫,故詔張忠順持節往,而授興留後。絳固請曰:「如興萬有一不受命,即姑息,復如向時矣。」由是即拜興節度使。絳復曰:「王化不及魏博久矣,一日挈六州來歸,不大犒賞,人心不激。請斥禁錢百五十萬緡賜其軍。」有言太過者,絳曰:「假令舉十五萬眾,期歲而得六州,計所轉給三倍於費。今興天挺忠義,首變污俗,破兩河之膽,可嗇小費隳機事哉?」從之。



 帝患朋黨,以問絳。答曰:「自古人君最惡者朋黨,小人揣知,故常借口以激怒上心。朋黨者,尋之則無跡,言之則可疑。小人常以利動,不顧忠義;君子者,遇主知則進,疑則退,安其位不為它計,故常為奸人所乘。夫聖人同跡,賢者求類,是同道也,非黨也。陛下奉遵堯、舜、禹、湯之德,豈謂上與數千年君為黨耶?道德同耳。漢時名節骨鯁士,同心愛國,而宦官小人疾之,起黨錮之獄,訖亡天下。趨利之人,常為朋比,同其私也;守正之人,常遭構毀,違其私也。小人多,譖言常勝;正人少,直道常不勝。可不戒哉!」絳居中介特,尤為左右所不悅,遂因以自明。



 王播為鹽鐵使,而事月進。絳曰:「比禁天下正賦外不得有它獻,而播妄名羨餘,不出祿稟家貲,願悉付有司。」帝曰:「善。」訖絳在位,獻不入禁中。吐蕃犯涇州,掠人畜,絳因言:「濱塞虛籍多,實兵少。今京西、北神策鎮軍,本防盛秋,坐仰衣食,不使戰。事至之日,乃先稟中尉。夫兵不內御,要須應變,失毫厘,差千里。請分隸本道,則號令齊一,前戰不還踵矣。」然士卒樂兩軍姑息,宦者以為言,議遂寢。嘗盛夏對延英,帝汗浹衣,絳欲趨出,帝曰:「朕宮中所對,惟宦官、女子,欲與卿講天下事,乃其樂也。」



 絳或無所論諍,帝輒詰所以然。又言:「公等得無有姻故冗食者,當為惜官。」吉甫、權德輿皆稱無有。絳曰:「崔祐甫為宰相,不半歲除吏八百人。德宗曰:『多公姻故,何耶?』祐甫曰:『所問當與不當耳,非臣親舊,孰知其才?其不知者,安敢與官?』時以為名言。武后命官猥多,而開元中有名者皆出其選。古人言拔十得五,猶得其半。若情故自嫌,非聖主責成意。」帝曰:「誠然,在至當而已。」帝又問:「玄宗開元時致治,天寶則亂,何一君而相反耶?」絳曰:「治生於憂危,亂生於放肆。玄宗嘗歷試官守,知人之艱難,臨禦初,任用姚崇、宋璟,勵精聽納,故左右前後皆正人也。洎林甫、國忠得君,專引傾邪之人,分總要劇。於是上不聞直言,嗜欲日滋,內則盜臣勸以興利,外則武夫誘以開邊,天下騷動,故祿山乘隙而奮。此皆小人啟導,從逸而驕。系時主所行,無常治,亦無常亂。」帝曰:「凡人舉事,病不通於理,追咎其失,古人處此有道耶?」絳曰:「事或過差,聖哲所不免。天子有諫臣,所以救過。上下同體,猶手足之於心膂,交相為用。但矜能護失,常情所蔽,聖人改過不吝,願陛下以此處之。」



 教坊使稱密詔閱良家子及別宅婦人內禁中,京師囂然。絳將入言於帝,吉甫曰:「此諫官所論列。」絳曰:「公嘗病諫官論事,此難言者,欲移之耶?」吉甫乃欲諷詔使止之,絳以吉甫畏不敢諫,遂獨上疏。帝曰:「朕以丹王等無侍者,比命訪閭里,以貲致之,彼不諭朕意,故至嘩擾。」乃悉歸所取。



 以足疾求免,罷為禮部尚書。帝乃召承璀於淮南。絳雖去位,猶懷不能已,因上言:「北虜方強,其憂有五。彼蔑信重利,歲入馬求直,今則置不取,當貯他謀,一也。屯士不足,斥候不明,城無完堞,非可應卒,二也。今之營築,不詢眾謀,遠規塞外,城非要地,虜一入寇,應援艱阻,三也。比年通好,往來窺覘,河山兵甲,悉知之矣,若寇掠驅脅,援兵非十日不至,既至虜去,兵罷復來,四也。北狄、西戎久為仇敵,今回鶻思叛,脫相連約,數道並進,何以遏之?五也。」



 十年,出為華州刺史。承璀田多在部中,主奴擾民,絳捕系之。會遣五坊使,帝戒曰:「至華宜自戢;絳,大臣,有奏即行法矣。」州有捕鷂戶,歲責貢限,絳以為言,並勸止畋獵,有詔澤潞、太原、天威府並罷之。入為兵部尚書,母喪免。還授河中觀察使。河中故節制,而皇甫鎛惡絳,故薄其恩,議者不直。鎛得罪,復以兵部召。遷御史大夫。穆宗數游畋,絳率其屬叩延英切諫,不納。以疾辭,還兵部尚書,歷東都留守,徙東川節度使,復為留守。寶歷初,拜尚書左僕射。絳偉儀質,以直道進退,望冠一時,賢不肖太分,屢為讒邪所中。御史中丞王璠遇絳於道,不之避。絳引故事論列,宰相李逢吉右璠,下遷絳太子少師,分司東都。



 文宗立,召為太常卿,以檢校司空為山南西道節度使,累封趙郡公。四年,南蠻寇蜀道,詔絳募兵千人往赴,不半道,蠻已去,兵還。監軍使楊叔元者,素疾絳,遣人迎說軍曰:「將收募直而還為民。」士皆怒,乃噪而入,劫庫兵。絳方宴,不設備,遂握節登陴。或言縋城可以免,絳不從。牙將王景延力戰歿,絳遂遇害,年六十七。幕府趙存約、薛齊皆死。事聞,諫官崔戎等列絳冤,冊贈司徒,謚曰貞,賻禮甚厚。景延亦贈官,祿一子。大中初,詔史官差第元和將相,圖形凌煙閣,絳在焉,獨留中。絳所論事萬餘言,其甥夏侯孜以授蔣偕,次為七篇。



 子璋,字重禮。大中初擢進士第,闢盧鈞太原幕府。遷監察御史,奏太廟祫享復用宰相攝事。進起居郎。舊制,設次郊丘,太僕盤車載樂,召群臣臨觀,璋奏罷之。咸通中,累官尚書右丞、湖南宣歙觀察使。



 宋申錫,字慶臣,史失其何所人。少而孤,擢進士第,累闢節度府,後頻遷起居舍人,以禮部員外郎為翰林學士。敬宗時,拜侍講學士。長慶、寶歷間,風俗囂薄,驅煽朋黨,申錫素孤直少與,及進用,議者謂可以激浮競。



 文宗即位,再轉中書舍人,復為翰林學士。帝惡宦官權寵震主,再致宮禁之變,而王守澄典禁兵,偃蹇放肆,欲叕刂除本根,思可與決大議者。察申錫忠厚,因召對,俾與朝臣謀去守澄等,且倚以執政,申錫頓首謝。未幾拜尚書右丞,逾月進同中書門下平章事。乃除王璠京兆尹,密諭帝旨。璠漏言,而守澄黨鄭注得其謀。太和五年,遣軍候豆盧著誣告申錫與漳王謀反,守澄持奏浴堂,將遣騎二百屠申錫家,宦官馬存亮爭曰:「謀反者獨申錫耳,當召南司會議,不然,京師跂足亂矣。」守澄不能對。時二月晦,群司皆休,中人馳召宰相,馬奔乏死於道,易所乘以復命。申錫與牛僧孺、路隋、李宗閔至中書,中人唱曰:「所召無宋申錫。」申錫始知得罪,望延英門,以笏叩額還第。僧孺等見上出著告牒,皆駭愕不知所對。守澄捕申錫親吏張全真、家人買子緣信及十六宅典史,脅成其罪。帝乃罷申錫為太子右庶子,召三省官、御史中丞、大理卿、京兆尹會中書集賢院雜驗申錫反狀。京師嘩言相驚,久乃定。



 翌日,延英召宰相群官悉入,初議抵申錫死,僕射竇易直率然對曰:「人臣無將,將而必誅。」聞者不然。於是左散騎常侍崔玄亮、給事中李固言、諫議大夫王質、補闕盧鈞、舒元褒、羅泰、蔣系、裴休、竇宗直、韋溫,拾遺李群、韋端符、丁居晦、袁都等伏殿陛,請以獄付外。帝震怒,叱曰:「吾與公卿議矣,卿屬第出!」玄亮、固言執據愈切,涕泣懇到,繇是議貸申錫於嶺表。京兆尹崔琯、大理卿王正雅苦請出著與申錫劾正情狀,帝悟,乃貶申錫開州司馬,從而流死者數十百人,天下以為冤。擢豆盧著兼殿中侍御史。



 初,申錫既歸,易素服俟命外舍,其妻責謂曰;「公何負天子,乃反乎?」申錫曰:「吾起孤生,位宰相,蒙國厚恩,不能鉏奸亂,反為所陷,我豈反者乎?」初,申錫以清節進,疾要位者納賕餉,敗風俗,故自為近臣,凡四方賄謝一不受。既被罪,有司驗劾,悉得所還問遺書,朝野為咨閔。然在宰府無它謀略。七年,感憤卒,有詔歸葬。



 開成元年,李石因延英召對,從容言曰:「陛下之政,皆承天心,惟申錫之枉,久未原雪。」帝慚曰:「我當時亦悟其失,而詐忠者迫我以社稷計故耳。使逢漢昭、宣時,當不坐此。」因追復右丞、同中書門下平章事,贈兵部尚書,錄其子慎微為城固尉。會昌二年,賜謚曰貞。



 贊曰:鎰、元衡暴忠王室,絳巨德大臣,皆為賊奸所乘,不歿元身,蓋福善禍淫之訓有時而撓。雖然,賢者於忠誼,寧以一不幸,遽使慊然於其心哉!要躬可殞,而名與岱、崧等矣。公輔隙開,而猶納說焉。申錫謀小任大,顛沛從之,惜乎!



\end{pinyinscope}