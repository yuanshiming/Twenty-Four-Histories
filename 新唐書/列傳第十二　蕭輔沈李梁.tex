\article{列傳第十二 蕭輔沈李梁}

\begin{pinyinscope}

 蕭銑,後梁宣帝曾孫也。祖巖,開皇初叛隋降陳,陳亡,文帝誅之。銑少貧,傭書,事母孝。煬帝以外戚擢為羅川令。



 大業十三年,岳州校尉董景珍、雷世猛,旅帥鄭文秀、許玄徹、萬瓚、徐德基、郭華,沔人張繡等謀反隋,且推景珍為主,景珍曰:「吾素微,雖假名號,眾不厭。羅川令,故梁裔也,寬仁大度,有武皇遺風。且吾聞帝王之興,必有符命。隋冠帶悉號『起梁』,蕭氏中興象也。今推之,以應天順人,不亦可乎?」乃遣人告銑。銑即報景珍書曰:「我先君昔事隋,職貢無廢,乃貪我土宇,滅我宗祊,我是以痛心疾首,恫心疾首,思刷厥恥。傑今天誘乃衷,公等降心,將大復梁緒,徼福於先帝,吾敢不糾厲士眾以從公哉!」即募兵數千,揚言跡盜,將以應景珍。



 會潁川賊沈柳生寇縣,銑出戰不利,謂其下曰:「岳陽豪傑將推我為主,今天下叛隋,吾能守節獨完哉?且吾先人國於此,若徇其請復梁祚,因以半紙檄召群盜,誰敢不從?」眾悅。乃以十月稱梁公,旗幟服色悉用其舊。柳生以眾歸銑,用為車騎大將軍。不五日,遠近爭附,眾數萬,乃趨巴陵。景珍遣徐德基、郭華率強姓百迎謁,而先見柳生。柳生與其下謀曰:「梁公起,我最先附,勛第一。今岳陽兵眾而位多,誰肯為我下?不如殺德基,質其人,獨挾梁主以進,則吾誰先?」因殺德基,詣中軍白銑。銑驚曰:「今欲撥亂,遽自相屠,我不能為若主矣!」步出軍門。柳生懼,伏地請罪。銑責宥之,陳兵而進。景珍曰:「德其倡義竭誠,柳生擅殺之,不誅,無以為政。且兇賊與共處,必為亂。」銑因斬柳生。於是築壇城南,柴上帝,自稱梁王。有異鳥至,建元為鳳鳴。



 義寧二年,僭稱皇帝,署百官,一用梁故事。追謚從父琮為孝靖帝,祖巖河間忠烈王,父璿文憲王。封景珍晉王,雷世猛秦王,鄭文秀楚王,許玄徹燕王,萬瓚魯王,張繡齊王,楊道生宋王。隋將張鎮州、王仁壽擊銑,不能克,及隋亡,乃與寧長真等率嶺南州縣降於銑。時林士弘據江南,銑遣將蘇胡兒拔豫章,使楊道生取南郡,張繡略定嶺表。西至三峽,南交趾,北距漢水,皆附屬,勝兵四十萬。



 武德元年,徙都江陵,復園廟。引岑文本為中書侍郎,掌機密。遣道生攻峽州,刺史許紹擊破之,士死過半。



 三年,高祖詔夔州總管趙郡王孝恭討之,拔通、開二州,斬偽東平王闍提。諸將擅兵橫恣,銑恐浸不制,乃陽議休兵營農,以黜其權。大司馬董景珍之弟為將軍,怨之,謀作亂,事洩,被誅。景珍方鎮長沙,銑下書赦之,召還江陵。景珍懼,遣使詣孝恭,舉地降。銑遣張繡攻景珍,景珍曰:「前年醢彭越,往年殺韓信,獨不見乎!奈何相攻?」繡不答,圍之。景珍潰而走,麾下殺之。銑進繡為尚書令。繡恃功,亦驕蹇,銑又誅之。銑性外寬內忌,疾勝己者,於是大臣舊將皆疑間,多叛去,銑不能禁,由此愈弱。



 四年,詔孝恭與李靖率巴蜀兵順流下,廬江王瑗繇襄陽道,黔州刺史田世康出辰州道,會兵圖銑。偽將周法明以四州降,即詔為黃州總管,趨夏口道,攻安州,克之。偽將雷長潁以魯山降。銑乃遣將文士弘拒孝恭,戰清江口,孝恭大破之,獲鬥艦千艘,拔宜昌、當陽、枝江、松滋,偽江州將蓋彥舉以城降。孝恭、靖直逼其都。



 初,銑放兵,止留宿衛數千人,及倉卒追集,江、嶺回遠,未及赴。孝恭布長圍守之,數日,破其水城,取樓船數千。交州總管丘和、長史高士廉、司馬杜之松詣靖降。銑度救不至,謂其下曰:「天不祚梁乎?待窮而下,必害百姓。今城未拔,先出降,可免亂。諸人何患無君?」乃麾而令,守陴者皆慟。以太牢告於廟,率官屬緦衰布幘詣軍門,謝曰:「當死者銑爾,百姓非罪也,請無殺掠!」孝恭受之,護送京師。後數日,救兵至,且十餘萬。知銑降,乃送款。銑至,高祖讓之,對曰:「隋失其鹿,英雄競逐。銑無天命,故為陛下禽,猶田橫南面,豈負漢哉?」帝怒其不屈,詔斬都市,年三十九。自僭國至滅凡五年。



 贊曰:銑,故梁子孫,起文吏,掩東南而有之,荊、楚好亂,氣俗然也。觀銑武雖不足,文有餘矣,大抵盜仁義,詭世亂俗者,聖人所必誅。若銑力困計殫,以好言自釋於下,系虜在廷,抗辭不屈,偽辯易窮,卒以殊死,高祖聖矣哉!



 輔公祏,齊州臨濟人。隋季與鄉人杜伏威為盜,轉掠淮南。伏威兵浸盛,自號總管,以公祏為長史。賊李子通據江都,伏威使公祏以精卒數千度江擊之。子通拒戰,眾十倍,銳甚。公祏選甲士千人,操長刀居前,別以千人隨之,令曰:「卻者斬!」公祏以眾殿。俄而子通方陣而進,長刀千人皆決死鬥,公祏縱左右翼搏之,子通大潰,降其眾數千。伏威既遣使歸國,武德二年,詔授公祏淮南道行臺尚書左僕射,封舒國公。



 初,伏威與公祏少相愛,又兄事之,故軍中呼輔伯,尊禮略等。伏威稍忌之,乃署養子闞棱為左將軍,王雄誕為右將軍,推公祏為僕射,陰解其柄。公祏內怏怏不平,乃與故人左游仙偽學闢穀以自晦。



 六年,伏威入朝,留公祏居守,復令雄誕握兵副之,陰誡曰:「吾至京不失職,無容公祏為變。」後左游仙說公祏反,會雄誕以疾臥家,公祏奪其兵,紿言伏威移書令舉事。八月,遂僭位,國稱宋,即陳故宮都之;殺王雄誕,署百官,以左游仙為兵部尚書、東南道大使、越州總管;增修器械,轉廩食,遣將徐紹宗侵海州,陳正通寇壽陽。詔越郡王孝恭趨九江,嶺南大使李靖下宣城,懷州總管黃君漢出譙,齊州總管李世勣繇淮、泗討之。孝恭取蕪湖,下梁山三鎮。河南安撫大使任瑰拔揚子城,降偽將龍龕,遂據揚州。公祏復遣將馮惠亮、陳當世屯博望山,陳正通、徐紹宗屯青州山以拒戰,孝恭率諸將破之,惠亮、正通走,李靖躡追百餘里,眾悉潰,正通等以五百騎奔丹陽。公祏懼,棄城奔左游仙於會稽,兵尚數萬。夜至毘陵,能從者裁五百。偽將吳騷、孫安謀執之,公



 祏棄妻子斬關遁,與腹心士數十抵武康,野人執送丹陽,孝恭斬之,傳首京師擊李子通,始公祏佐伏威起據江東,距公祏死,凡十三年。



 沈法興,湖州武康人。父恪,陳廣州刺史。法興隋大業末為吳興郡守,東陽賊樓世幹略其郡,煬帝詔與太僕丞元祐討之。



 義寧二年,江都亂,法興自以世南土,屬姓數千家,遠近向服,乃與祐將孫士漢、陳果仁執祐,名誅宇文化及,三月發東陽,行收兵,趨江都,下餘杭,比至烏程,眾六萬。毘陵通守路道德拒之,法興約連和,因襲殺之,據其城,遂定江表十餘州,自署江南道總管。聞越王侗立,乃上書稱大司馬、錄尚書事、天門公,承制置百官,以陳果仁為司徒,孫士漢司空,蔣元超尚書左僕射,殷芊左丞,徐令言右丞,劉子翼選部侍郎,李百藥為掾。後聞侗被廢,高祖武德二年,稱梁王,建元為延康,易隋官儀,頗用陳氏故事。



 法興自意南方諸城可跂而平,專事威戮,下有細過即誅之,繇是將士攜解。俄遣子綸救陳棱,擊李子通,反為所敗。子通乘鋒度江,破京口。使將蔣元超戰庱亭,大敗,死之。法興懼,棄城與左右數百投吳郡賊聞人遂安,遂安遣將葉孝辯迎之。法興中悔,將殺孝辯,趨會稽,為所覺,懼,自沈於江。起義寧至武德,凡三年滅。



 李子通,沂州承人。少貧,以漁獵為生。居其鄉,見班白負戴,必代之,家有餘,則以賙人,而喜報仇。



 隋大業末,長白山賊左才相自號博山公,子通依之,以武力雄其間。鄉人有陷賊者,子通專經護之。方是時,群盜暴忍,獨子通仁愛,歸者遂多,不半歲,有徒萬人。才相畏忌,子通乃引眾度淮,與杜伏威合。為隋將來整所破,奔海陵,得眾二萬,自稱將軍。大業十一年僭號楚王。



 宇文化及殺煬帝,以右御衛將軍陳棱為江都太守,已而棱降,高祖授以總管,即守其郡。子通攻棱,棱窮,乞師於沈法興、杜伏威。伏威自將屯清流,法興遣子綸屯揚子,間數十里。子通納言毛文深請募吳人詐為法興兵夜襲伏威,二人遂交惡,無敢先戰者。子通得悉力取江都,遂據之,棱奔而免。子通僭即皇帝位,國號吳,建元明政。齊賊樂伯通先為化及守丹陽,即以眾萬餘降之,子通用為尚書左僕射。又敗法興兵,遂取晉陵,以法興所署掾李百藥為內史侍郎,典文檄,尚書左丞殷芊為太常卿,司禮樂,繇是江南士人多歸之。會伏威命輔公祏拔丹陽,進屯溧水,子通戰敗,糧且盡,棄江都,保京口,伏威盡得其地。俄東走太湖,裒散兵二萬人,復張,襲法興吳郡,破之。據餘杭。東舉會稽,南距嶺,西抵宣城,北太湖,悉有之。



 武德四年,伏威遣將王雄誕討子通。戰蘇州,敗績,退保餘杭,雄誕進傅城。子通窮。乃降,伏威受之,並樂伯通送京師。高祖薄其罪,賜宅一區、田五頃,賚予頗厚。及伏威來朝,子通語伯通曰:「東南未靖,而伏威來。我故兵多在江外,若收之,可建大功。」遂皆亡。及藍田,為關吏所獲,並伏誅。方子通等僭盛時,復有硃粲、林士弘、張善安亦竊名號於淮、楚間。



 硃粲,亳州城父人。初為縣史。大業中從軍,伐賊長白山,亡命去為盜,號「可達寒賊」,自稱迦樓羅王,眾十萬。度淮屠景陵、沔陽,轉剽山南,所至殘戮無遺噍。僭號楚帝,建元為昌達。攻拔南陽。



 義寧末,與山南撫慰使馬元規戰冠軍,大敗,收餘眾,復振,至二十萬。粲所克州縣皆發藏粟以食,遷徙無常,去輒燔廥聚,毀城郭,不務稼穡,專以劫為資。於是人大餒,死者系路,其軍亦匱,乃掠小兒烝食之。戒其徒曰:「味之珍寧有加人者?弟使佗國有人,我恤無儲哉!」勒所部略婦人孺兒分烹之,又稅諸城細弱以益糧。隋著作佐郎陸從典、通事舍人顏愍楚謫南陽,粲初引為賓客,後盡食兩家。俄而諸城懼,皆逃散。



 顯州首領楊士林、田瓚起兵攻粲,旁郡響赴,戰淮源,粲大敗,挈殘士奔菊潭,遣使乞降。高祖以前御史大夫段確假散騎常侍勞之。確醉,戲粲曰:「君膾人多矣,若為味?」粲曰:「啖嗜酒人,正似糟豚。」確悸,罵曰:「狂賊,歸朝乃一奴耳,復得噬人乎?」粲懼,收確於坐,並從者數十悉饔之,以饗左右。遂屠菊潭,奔王世充,署龍驤大將軍。東都平,斬洛水上。士庶競擲瓦礫擊其尸,須臾若塚。



 林士弘,饒州鄱陽人。隋季與鄉人操師乞起為盜。師乞自號元興王,建元天成,大業十二年據豫章,以士弘為大將軍。隋遣治書侍御史劉子翊討賊,射殺師乞,而士弘收其眾,復戰彭蠡,子翊敗,死之。遂大振,眾十餘萬,據虔州,自號南越王。俄僭號楚:稱皇帝,建元為太平。侍御史鄭大節以九江郡下之。士弘任其黨王戎為司空。臨川、廬陵、南康、宜春豪傑皆殺隋守令以附,北盡九江,南番禺,悉有之。後蕭銑以舟師破豫章,士弘獨有南昌、虔、循、潮之地。銑敗,其亡卒稍歸之,復振。趙郡王孝恭招慰,降循、潮二州。



 武德五年,士弘弟鄱陽王藥師以兵二萬圍循州,總管楊世略破斬之,士弘請降。王戎亦獻南昌地,詔戎為南昌州總管。士弘復遁保安城山,誘潰亡,謀復亂,袁人相聚應之,為張善安所察,以兵赴討。會士弘死,其黨乃解。



 張善安,兗州方與人。年十七,亡命為盜,轉掠淮南。會孟讓敗,得其散卒八百,襲破廬江郡。依林士弘,不見信,憾之,反襲士弘,焚其郛,去保南康。蕭銑取豫章,遣將蘇胡兒守之,善安奪其地,據以歸國,授洪州總管。



 武德六年反,輔公祏以為西南道大行臺。善安掠孫州,執總管王戎,襲殺黃州總管周法明。會李大亮兵至,為開曉禍福,答曰:「善安初不反,為部下詿誤。降,今易耳,恐不免,奈何?」大亮曰:「總管定降,吾固不疑。」因獨入其陣,與善安握手語,乃大喜,將數十騎詣大亮營。大亮引入,命壯士執之。騎皆驚,引去,悉兵來戰。大亮諭以善安自歸,無庸斗。其黨罵曰:「總管賣我!」遂潰。送善安京師,稱不與公祏謀,高祖赦之。公祏破,得其書,遂伏誅。



 梁師都,夏州朔方人。為郡豪姓。仕隋鷹揚府郎將。大業末罷歸,結徒起為盜,殺郡丞唐世宗,據郡稱大丞相,聯兵突厥。與隋將張世隆戰,敗之,因略定雕陰、弘化、延安。自為梁國,僭皇帝位,祭天於城南,坎地瘞玉得印,以為瑞,建元永隆。始畢可汗遺以狼頭纛,號大度毘伽可汗、解事天子,遂導突厥兵居河南地,拔鹽川郡。



 武德二年,寇靈州,長史楊則擊走之。又與突厥千騎營野豬嶺,延州總管段德操勒兵不戰,師都氣懈,遣兵進擊,戰酣,德操自以輕騎出其旁乘之,師都大潰,逐北二百里,俘馘甚眾。未幾,以步騎五千入寇,德操又盡屠其軍,降堡將張舉、劉旻。師都懼,遣尚書陸季覽說處羅可汗曰:「隋亡,中國裂為四五,勢均力弱,皆爭附突厥。今唐滅劉武周,國益大,兵方四出。師都將朝夕亡,然次亦及突厥,願可汁如魏孝文,兵引而南,師都請為鄉道。」處羅納之,令莫賀咄設入五原,泥步設與師都趨延州,處羅自攻太原,突利可汗與奚、霫、契丹、靺羯繇幽州道合,竇建德自滏口會晉、絳。已而處羅死,兵不出,又為德操所破。



 六年,其將賀遂、索周以所部十二州降。德操悉兵攻之,拔東城,師都保西城不敢出,求救於突厥頡利,頡利以勁兵萬騎赴之。先是,稽胡大帥劉屳成以眾附師都,因讒見殺,其下疑懼,乃多叛。師都日益蹙,遂往朝頡利,教使南略,故突厥盜邊無寧歲,遂窺渭橋。



 後突厥政亂,太宗以師都浸危,乃諭以書使歸,不從。詔夏州長史劍旻、司馬劍蘭經略之。獲生口,縱以為間,君臣離撓。出輕騎蹂其稼,城中饑虛。又天狗墮其城。辛獠兒、李正寶、馮端皆其健將,謀執師都降,不果,正寶挺身歸。



 貞觀二年,旻、蘭表可取狀,詔柴紹、薛萬均並力,令旻以勁卒直據朔方東城。頡利來援,會大雪,羊馬死,紹逆戰,破之,進屯城下。其從父弟洛仁斬師都降,擢洛仁為右驍衛將軍、朔方郡公。自起至滅十二年。以其地為夏州。始師都據郡時,劉季真、郭子和者亦俱起,子和自有傳。



 劉季真,離石胡人。父龍兒,大業十年舉兵自稱王,以季真為太子,弟六兒為永安王。鋒甚銳,將軍潘長文連年擊,不能下。後虎賁郎將梁德破殺龍兒,眾乃散。唐兵起,六兒復聚為盜,附劉武周,季真從之,自號太子王,六兒為拓定王,迭為邊害。西河公張綸、真鄉公李仲文合兵討之,季真降,詔以為石州總管,賜姓李,封彭山郡王。宋金剛戰澮州,勢未決,遂復連武周。及敗,秦王執六兒斬之,季真奔高滿政,俄被殺。



\end{pinyinscope}