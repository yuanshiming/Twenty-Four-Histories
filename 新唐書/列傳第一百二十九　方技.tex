\article{列傳第一百二十九 方技}

\begin{pinyinscope}

 李淳風甄權許胤宗張文仲袁天綱客師張憬藏乙弗私禮金梁鳳王遠知薛頤葉法善明崇儼尚獻甫嚴善思杜生張果邢和璞師夜光羅思遠姜撫桑道茂



 凡推步、卜、相、醫、巧,皆技也。能以技自顯地一世,亦悟之天,非積習致然。然士君子能之,則不迂,不泥,不矜,不神;小人能之,則迂而入諸拘礙,泥而弗通大方,矜以誇眾,神以誣人,故前聖不以為教,蓋吝之也。若李淳風諫太宗不濫誅,許胤宗不著方劑書,嚴譔諫不合乾陵,乃卓然有益於時者,茲可珍也。至遠知、果、撫等詭行紀怪,又技之下者焉。



 李淳風,岐州雍人。父播,仕隋高唐尉,棄官為道士,號黃冠子,以論譔自見。淳風幼爽秀,通群書,明步天歷算。貞觀初,與傅仁均爭歷法,議者多附淳風,故以將仁郎直太史局。制渾得儀,詆摭前世失,著《法象書》七篇上之。擢承務郎,遷太常博士,改太史丞,與諸儒修書,遷為令。太宗得秘讖,言「唐中弱,有女武代王」。以問淳風,對曰:「其兆既成,已在宮中。又四十年而王,王而夷唐子孫且盡。」帝曰:「我求而殺之,奈何?」對曰:「天之所命,不可去也,而王者果不死,徒使疑似之戳淫及無辜。且陛下所親愛,四十年而老,老則仁,雖受終易姓,而不能絕唐。若殺之,復生壯者,多殺而逞,則陛下子孫無遺種矣!」帝採其言,止。



 淳風於占候吉兇,若節契然,當世術家意有鬼神相之,非學習可致,終不能測也。以勞封昌樂縣男。奉詔與算博士梁述、助教王真儒等是正《五曹》、《孫子》等書,刊定注解,立於學官。撰《麟德歷》代《戊寅歷》,候者推最密。自秘閣郎中復為太史令,卒。所撰《典章文物志》、《乙巳占》等書傳於世。子該,孫仙宗,並擢太史令。



 唐初言歷者惟傅仁均。仁均,滑州人,終太史令。



 甄權,許州扶溝人。以母病,與弟立言究習方書,遂為高醫。仕隋為秘書省正字,稱疾免。魯州刺史庫狄嶔風痺不得挽弓,權使彀矢向堋立,金咸其肩隅,一進,曰:「可以射矣。」果如言。貞觀中,權已百歲,太宗幸其舍,視飲食,訪逮其術,擢朝散大夫,賜幾杖衣服。尋卒,年一百三歲。所撰《脈經》、《針方》、《明堂》等圖傳於時。



 立言仕為太常丞。杜淹苦流腫,帝遣視,曰:「去此十日,午漏上,且死。」如之,有道人必腹懣煩彌二歲,診曰:「腹有蠱,誤食發而然。」令餌雄黃一劑,少選,吐一蛇如拇,無目,燒之有發氣,乃愈。



 後以醫顯者,清漳宋俠、義興許胤宗、洛陽張文仲李虔縱、京兆韋慈藏。



 俠官朝散大夫,藥藏監。



 胤宗仕陳為新蔡王外兵參軍。王太后病風不能言,脈沉難對,醫家告術窮。胤宗曰:「餌液不可進。」即以黃耆、防風煮湯數十斛,置床下,氣如霧,熏薄之,是夕語。擢義興太守。武德初,累進散騎侍郎。關中多骨蒸疾,轉相染,得者皆死,胤宗療視必愈。或勸其著書貽後世者,答曰:「醫特意耳,思慮精則得之。脈之候幽而難明,吾意所解,口莫能宣也。古之上醫,要在視脈,病乃可識。病與藥值,唯用一物攻之,氣純而愈速。今之人不善為脈,以情度病,多其物以幸有功,譬獵不知兔,廣絡原野,冀一人獲之,術亦疏矣。一藥偶得,它味相制,弗能專力,此難愈之驗也。脈之妙處不可傳,虛著方劑,終無益於世,此吾所以不著書也。」卒年七十餘。



 文仲仕武后時,至尚藥奉御。特進蘇良嗣方朝,疾作,僕廷中。文仲診曰:「憂憤而成,若脅痛者,殆未可救。」頃告脅痛。又曰:「及心則貽。」俄心痛而死。文仲論風與氣尤精。後集諸言方者與共著書,詔王方慶監之。文仲曰:「風狀百二十四,氣狀八十,治不以時,則死及之。惟頭風與上氣、足氣,藥可常御。病風之人,春秋末月,可使洞利,乃不困劇,自餘須發則治,以時消息。」乃著《四時輕重術》凡十八種上之。



 虔縱官侍御醫,慈藏光祿卿。



 袁天綱,益州成都人。仕隋為鹽官令。仕隨為鹽官令《舊書》卷一九一《袁天綱傳》及《冊府》卷八六○均謂「隋大業中為資官令」。在洛陽,與杜淹、王珪、韋挺游,天綱謂淹曰:「公蘭臺、學堂全且博,將以文章顯。」謂珪「法令成,天地相臨,不十年官五品」;謂挺「面如虎,當以武處官」;「然三君久皆得譴,吾且見之」。淹以侍御史入天策為學士,珪太子中允,挺善隱太子,薦為左衛率。武德中,俱以事流雋州,見天綱,曰:「公等終且貴。杜位三品,難與言壽,王、韋亦三品,後於杜而壽過之,但晚節皆困。」見竇軌曰:「君伏犀貫玉枕,輔角完起,十年且顯,立功其在梁、益間邪!」軌後為益州行臺僕射,天綱復曰:「赤脈乾瞳,方語而浮赤入大宅,公為將必多殺,願自戒。」軌果坐事見召。天綱曰:「公毋憂,右輔澤而動,不久必還。」果還為都督。



 貞觀初,太宗召見曰:「古有君平,朕今得爾,何如?」對曰:「彼不逢時,臣固勝之。」武后之幼,天綱見其母曰:「夫人法生貴子。」乃見二子元慶、元爽,曰:「官三品,保家主也。」見韓國夫人,曰:「此女貴而不利夫。」後最幼,姆抱以見,紿以男,天綱視其步與目,驚曰:「龍瞳鳳頸,極貴驗也;若為女,當作天子。」帝在九成宮,令視岑文本,曰:「學堂瑩夷,眉過目,故文章振天下。首生骨未成,自前而視,法三品。肉不稱骨,非壽兆也。」張行成、馬周見,曰:「馬君伏犀貫腦,背若有負,貴驗也。近古君臣相遇未有及公者。然面澤赤而耳無根,後骨不隆,壽不長也。張晚得官,終位宰相。」其術精類如此。高士廉曰:「君終作何官?」謝曰:「僕及夏四月,數既盡。」如期以火山令卒。以火山令卒,按《舊書》卷一九一《袁天綱傳》、《冊府》卷八六○均謂武德初授火井令,「火山」疑是「火井」之訛。



 子客師,亦傳其術,為廩犧令。高宗置一鼠於奩,令術家射,皆曰鼠。客師獨曰:「強實鼠,然入則一,出則四。」發之,鼠生三子。嘗度江,叩舟而還,左右請故,曰:「舟中人鼻下氣皆墨,不可以濟。」俄有一男子,跛而負,直就舟,客師曰:「貴人在,吾可以濟。」江中風忽起,幾覆而免。跛男子乃婁師德也。



 時有長社人張憬藏,持與天綱埒。太子詹事蔣儼有所問,答曰:「公厄在三尺土下,盡六年而貴,六十位蒲州刺史,無有祿矣。」儼使高麗,為莫離支所囚,居土室六年還。及為蒲州,歲如期,則召掾史、妻子,告當死,俄詔聽致仕。劉仁軌與鄉人靖賢請占,憬藏答曰:「劉公當五品而譴,終位冠人臣。」謂賢曰:「君法客死。」仁軌為尚書僕射。賢猥曰:「我三子皆富田宅,吾何客死?」俄喪三子,盡鬻田宅,寄死友家。魏元忠尚少,往見憬藏,問之,久不答,元忠怒曰:「窮通有命,何預君邪?」拂衣去。憬藏遽起曰:「君之相在怒時,位必卿相。」姚崇、李迥秀、杜景往從之游,憬藏曰:「三人者皆宰相,然姚最貴。」郎中裴珪妻趙見之,憬藏曰:「夫人目修緩,法曰『豕視淫』,又曰『目有四白,五夫守宅』,夫人且得罪。」俄坐奸,沒入掖廷。裴光廷當國,憬藏以紙大署「臺」字投之,光廷曰:「吾既臺司矣,尚何事?」後三日,貶臺州刺史。



 隋末又有高唐人乙弗弘禮,當煬帝居籓,召見,弘禮賀曰:「大王為萬乘主,所戒在德而已。」及即位,悉詔諸術家坊處之,使弘禮總攝。海內浸亂,帝曰:「而昔言朕既驗,然終當奈何?」弘禮逡巡,帝知之,乃曰:「不言,且死!」弘禮曰:「臣觀人臣相與陛下類者不長,然聖人不相,故臣不能知。」由是敕有司監視,毋得與外語。



 薛大鼎坐事沒為奴,及貞觀時,有請於弘禮,答曰:「君,奴也,欲何事?」請解衣視之,弘禮指腰而下曰:「位方岳。」



 玄宗時有金梁鳳者,頗言人貴賤夭壽。裴冕為河西留後,梁鳳輒言:「不半歲兵起,君當以御史中丞除宰相。」又言:「一日向雒,一日向蜀,一日向朔方,此時公當國。」冕妖其言,絕之。俄而祿山反,冕以御史中丞召,因問三日,答曰:「雒日即滅,蜀曰不能久,朔方日愈明。」肅宗即位,而冕遂相,薦於帝,拜都水使者。梁鳳謂呂諲曰:「君且輔政,須大怖乃得。」諲責驛史,之,史突入射諲,兩矢風中,走而免,明年知政事。李揆、盧允毀服紿謁,梁鳳不許,二人語以情,梁鳳曰:「李自舍人閱歲而相,盧不過郎官。」揆已相,擢允吏部郎中。



 王遠知,系本瑯邪,後為揚州人。父曇選,為陳揚州刺史。母晝寢,夢鳳集其身,因有娠。浮屠寶志謂曇選曰:「生子當為世方士。」



 遠知少警敏,多通書傳,事陶弘景,傳其術,為道士。又從臧兢游。陳後主聞其名,召入重陽殿,辯論超詣,甚見咨挹。隋煬帝為晉王,鎮揚州,使人介以邀見,少選發白,俄復鬢,帝懼,遣之。後幸涿郡,詔遠知見臨朔宮,帝執弟子禮,咨質仙事,詔京師作玉清玄壇以處之。及幸揚州,遠知謂帝不宜遠京國,不省。



 高祖尚微,遠知密語天命。武德中,平王世充,秦王與房玄齡微服過之,遠知未識,迎語曰:「中有聖人,非王乎?」乃念以寶。遠知曰:「方為太平天子,願自愛。」太宗立,欲官之,苦辭。貞觀九年,詔潤州即茆山為觀,俾居之。璽詔曰:「省所奏,願還舊山,已別詔不違雅素,並敕立祠觀,以伸曩懷。未知先生早晚至江外,祠舍何當就功?令太史令薛頤等往宣朕意。」



 遠知多怪言,詫其弟子潘師正曰:「吾少也有累,不得上天,今署少室伯,吾將行。」即沐浴,加冠衣,若寢者,遂卒。或言壽蓋百二十六歲雲。遺命子紹業曰:「爾年六十五見天子,七十見女君。」調露中,紹業表其言,高宗召見,嗟賞,追贈遠知太中大夫,謚升真先生。武時復召見,皆如其年。又贈金紫光祿大夫。天授中改謚升玄。



 薛頤者,滑州人。當隋大業時為道士,善天步律歷。武德初,追直秦王府,密語曰:「德星舍秦分,王當帝天下。」王表為太史丞,稍遷令。貞觀時,太宗將封秦山,彗星見,賾因言:「臣商天意,陛下未可東。」亦會大臣上議,帝遂罷。固丐為道士,帝為築觀九〓山,號曰:「紫府」,拜賾太中大夫,往居之。即祠建清臺,候辰次災祥以聞,所上與太史李淳風合。數歲卒。



 高宗時,又有葉法善者,括州括蒼人。世為道士,傳陰陽、占繇、符架之術,能厭劾怪鬼。帝聞之,召詣京師,欲寵以官,不拜。留內齋場,禮賜殊縟。時帝悉召方士,化黃金治丹,法善上言:「丹不可遽就,徒費財與日,請核真偽。」帝許之,凡百餘人皆罷。嘗在東都凌空祠為壇以祭,都人悉往觀,有數十人自奔火中,眾大驚,救而免。法善笑曰:「此為魅所馮,吾以法攝之耳。」問而信,病亦皆已。其譎幻類若此。



 歷高、中二宗朝五十年,往來山中,時時召入禁內。雅不喜浮屠法,常力詆毀,議者淺其好習,然發衛高,卒叵之測。睿宗立,或言陰有助力。無天中,拜鴻廬卿,員外置,封越國公,舍景龍觀,追贈其父歙州刺史,寵映當世。開元八年卒。或言生隋大業丙子,死庚子,蓋百七歲雲。玄宗下詔褒悼,贈越州都督。



 明崇儼,洛州偃師人,梁國子祭酒山賓五世孫。少隨父恪令安喜,吏有能召鬼神者,盡傳其術。乾封初,應岳牧舉,調黃安丞,以奇技自名。高宗召見,甚悅,擢冀王府文學。試為窟室,使宮人奏樂其中,召崇儼問:「何祥邪?為我止之。」崇儼書桃木為二符,剚室上,樂即止,曰:「向見怪龍,怖而止。」盛夏,帝思雪,崇儼坐頃取以進,自云往陰山取之。四月,帝憶瓜,崇儼索百錢,須臾以瓜獻,曰:「得之緱氏老人圃中。」帝召老人問故,曰:「埋一瓜失之,土中得百錢。」



 累遷正諫大夫。帝令入閣供奉,每謁見,陳時政,多托鬼神為言。至為武后作厭勝事,又言章懷太子不德。儀鳳四年,為盜所刺於東都,好事者為言:「崇儼役鬼勞苦,為鬼所殺。」而太后疑太子使客殺之,故贈侍中,謚曰莊,擢子珪為秘書郎。命御史中丞崔謐等雜治,誣服者甚眾。及太子廢,死狀乃明。



 尚獻甫,衛州汲人,善占候。武后召見,由道士擢太史令,辭曰:「臣梗野,不可以事官長。」後改太史局為渾儀監,以獻甫為令,不隸秘書省。數問災異,又於上陽宮集術家撰《方域》等篇。長安二年,熒惑犯五諸侯,獻甫自陳:「五諸侯,太史位;臣命納音,金也;火,金之仇,臣且死。」後曰:「朕為卿厭之。」迂水衡都尉,謂曰:「水生金,卿無憂。」至秋卒,後嗟異,復以渾儀監為太史局云。



 嚴善思名譔,同州朝邑人,以字行。父延,與河東裴玄證、隴西李貞蔡靜皆通儒術,該曉圖識。善思傳延業,褚遂良、上官儀等奇其能。高宗封泰山,舉銷聲幽藪科及第,調襄陽尉。居親喪,廬墓,因隱居十年。武后時擢監察御史,兼右拾遺內供奉,數言天下事。方酷吏構大獄,以善思為詳審使,平活八百餘人,原千餘姓。長壽中,按囚司刑寺,罷疑不實者百人。來俊臣等疾之,誣以罪,適交趾,五歲得還。是時李淳風死,候家皆不效,乃詔善思以著作佐郎兼太史令。聖歷二年,熒惑入輿鬼,後問其占,對曰:「大臣當之。」是年王及善卒。長安中,熒惑入月,鎮犯天關,善思曰:「法當亂臣伏罪,而有下謀上之象。」歲餘,張柬之等起兵誅二張。遷給事中。



 後崩,將合葬乾陵,善思建言:「尊者先葬,卑者不得入。今啟乾陵,是以卑動尊,術家所忌。且玄關石門,冶金錮隙,非攻鑒不能開,神道幽靜,多所驚黷。若別攻隧以入其中,即往昔葬時神位前定,更且有害。曩營乾陵,國有大難,易姓建國二十餘年,今又營之,難且復生。合葬非古也,況事有不安,豈足循據?漢世皇后別起陵墓,魏、晉始合葬。漢積祀四百,魏、晉祚率不長,亦其驗也。今若更擇吉地,附近乾陵,取從葬之義。使神有知,無所不通;若其無知,合亦何益?山川精氣,上為列星。葬得其所,則神安而後嗣昌;失其宜,則神危而後嗣損。願割私愛,使社稷長久。」中宗不納。



 神龍中,武后喪公除,太常請大習樂,供郊廟,詔未許。善思奏曰:「樂者氣化,所以感天地、調五行。漢、魏喪禮,以日易月,蓋三年不為禮,禮必壞,三年不為樂,樂必崩。禮,陰也;樂,陽也。樂崩陽伏,禮廢陰愆,故變以適時,孝道之大。安人神,公也;茹哀戚,私也。王者不以私害公,請如太常奏。」帝從之。遷禮部侍郎。表皇后擅政,為社稷憂,求汝州刺史。嘗語姚崇曰:「韋氏禍且塗地,相王所居有華蓋紫氣,必位九五,公善護之。」及睿宗立,崇以語聞,召拜右散騎常侍。



 初,譙王重福徙均州,過汝,善思為刺史。及謀反,偽除禮部尚書。重福敗,坐關通論死,吏部尚書宋璟、戶部郎中李邕薄其罪,給事中韓思復固請,乃流靜州。始,善思為御史,中書舍人劉允濟為酷吏所陷,且死,善思力訟其冤,得免。戶部尚書王本立見之,曰:「祁奚之救叔向,嚴公有之。」後見允濟,語未嘗及之。思復之解善思也,亦不自德,時稱長者之報。後遇赦還。開元十六年卒。子向,乾元中為鳳翔尹,三世皆年八十五雲。



 杜生者,許州人。善《易》占。有亡奴者問所從追,戒曰:「自此行,逢使者,懇丐其鞭。若不可,則以情告。」其人果值使者於道,如生語,使者異之,曰:「去鞭,吾無以進馬,可折道傍〓代之。」乃往折〓,見亡奴伏其下,獲之。它日又有亡奴者,生戒持錢五百伺於道,見進鷂使者,可市其一,必得奴。俄而使至,其人以情告,使者以一與之,忽飛集灌莽上,往取之而得亡奴。眾以為神。



 時有浮屠泓者,黃州人。與天官侍郎張敬之善。敬之以武后在位,常指所服示子冠宗曰:「莽朝服耳。」俄冠宗以父應入三品,詣有司言狀。泓忽曰:「君無煩求三品也。」敬之大驚,已而知出冠宗意。敬之弟訥之疾殆,泓曰:「公弟當位三品,不足憂也。」已而愈。嘗為燕國公張說市宅,戒曰:「無穿東北,王隅也!」它日見說曰:「宅氣索然,云何?」與說共視,土隅有三坎丈餘,泓驚曰:「公富貴一世而已,諸子將不終。」說懼,將平之,泓曰:「客上無氣,與地脈不連,譬身瘡痏補它肉,無益也。」說子皆污賊死斥云。



 張果者,晦鄉里世系以自神,隱中條山,往來汾、晉間,世傳數百歲人。武后時,遣使召之,即死,後人復見居恆州山中。



 開元二十一年,刺史韋濟以聞。玄宗令通事舍人裴晤往迎,見晤輒氣絕僕,久乃蘇。晤不敢逼,馳白狀。帝更遣中書舍人徐嶠齎璽書邀禮,乃至東都,舍集賢院,肩輿入宮。帝親問治道神仙事,語秘不傳。果善息氣,能累日不食,數御美酒。嘗云:「我生堯丙子歲,位侍中。」其貌實年六七十。時有邢和璞者,善知人夭壽。師夜光者,善視鬼。帝令和璞推果生死,懵然莫知其端。帝召果密坐,使夜光視之,不見果所住。



 帝謂高力士曰:「吾聞飲堇無苦者,奇士也。」時天寒,因取以飲果,三進,頹然曰:「非佳酒也。」乃寢。頃視齒燋縮,顧左右取鐵如意擊墮之,藏帶中,更出藥傅其斷,良久,齒已生,粲然駢絜。帝益神之。欲以玉真公主降果,未言也。果忽謂秘書少監王迥質、太常少卿蕭華曰:「諺謂娶婦得公主,平地生公府,可畏也。」二人怪語不倫。俄有使至,傳詔曰:「玉真公主欲降先生。」果笑,固不奉詔。有詔圖形集賢院,懇辭還山,詔可。擢銀青光祿大夫,號通玄先生,賜帛三百匹,給扶侍二人。至恆山蒲吾縣,未幾卒,或言尸解。帝為立棲霞觀其所。



 夜光者,薊州人,少為浮屠。至長安,因九仙公主得召見溫泉,帝奇其辯,賜冠帶,授四門博士,賜緋衣、銀魚、金繒千數,得侍左右如幸臣。



 和璞喜黃老,作《潁陽書》,世傳之。



 天寶中,有孫甑生者,以技聞,能使石自斗,草為人騎馳走。楊貴妃喜觀之,數召入宮中。



 又有羅思遠,能自隱。帝學,不肯盡其術,試自隱,常餘衣帶,及思遠共試,則驗。厚錫金帛,然卒不得。帝怒,裹以襆,壓殺之。數日,有中使者自蜀還,逢思遠駕而西,笑曰:「上為戲何虐也!」



 姜撫,宋州人。自言通仟人不死術,隱居不出。開元末,太常卿韋縚祭名山,因訪隱民,還白撫已數百歲。召至東都,舍集賢院。因言:「服常春藤,使白發還鬢,則長生可致。藤生太湖最良,終南往往有之,不及也。」帝遣使者至太湖,多取以賜中朝老臣。因詔天下,使自求之。宰相裴耀卿奉觴上千萬歲壽,帝悅,御花萼棲宴群臣,出藤百奩,遍賜之。擢撫銀青光祿大夫,號沖和先生。撫又言:「終南山有旱藕,餌之延年。」狀類葛粉,帝作湯餅賜大臣。右驍衛將軍甘守誠能銘藥石,曰:「常春者,千歲藟也。旱藕,杜蒙也。方家久不用,撫易名以神之。民間以酒漬藤,飲者多暴死。」乃止。撫內慚悸,請求藥牢山,遂逃去。



 桑道茂者,寒人,失其系望。善太一遁甲術。乾元初,官軍圍安慶緒於相州,勢危甚,道茂在圍中,密語人曰:「三月壬申西師潰。」至期,九節度兵皆敗。後召待詔翰林。建中初,上言:「國家不出三年有厄會,奉天有王氣,宜高坦堞,為王者居,使可容萬乘者。」德宗素驗其數,詔京兆尹嚴郢發眾數千及神策兵城之。時盛夏趣功,人莫知其故。及硃泚反,帝蒙難奉天,賴以濟。



 李晟為右金吾大將軍,道茂齎一縑見晟,再拜曰:「公貴盛無比,然我命在公手,能見赦否?」晟大驚,不領其言。道茂出懷中一書,自具姓名,署其左曰:「為賊逼脅。」固請晟判,晟笑曰:「欲我何語?」道茂曰:「弟言準狀赦之。」晟勉從。已又以縑願易晟衫,請題衿膺曰:「它日為信。」再拜去。道茂果污硃泚偽官。晟收長安,與逆徒縛旗下,將就刑,出晟衫及書以示。晟為奏,原其死。



 是時籓鎮擅地無寧時,道茂曰:「年號元和,寇盜翦滅矣。」至憲宗乃驗。道茂居有二伯甚茂,曰:「人居而木蕃者去之,木盛則土衰,土衰則人病。」乃以鐵數十鈞埋其下,復曰:「後有發其地而死者。」大和中,溫造居之,發藏鐵而造死。杜佑與楊炎善。盧杞疾之,佑懼,以問道茂,答曰:「君歲中補外,則福壽叵涯矣。」俄拜饒州刺史,後終司徒。李泌病,道茂署於紙曰:「厄三月二日就饗,國與家吉而身危。」會中和日,泌雖篤,強入。德宗見泌不能步,詔歸第,卒。是日北軍謀亂,仗士禽斬之。李鵬為盛唐令,道茂曰:「君位止此,而塚息位宰相,次息亦大鎮,子孫百世。」鵬卒,後石至宰相,福歷七鎮,諸孫通顯云。



\end{pinyinscope}