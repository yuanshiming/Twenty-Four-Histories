\article{志第三 禮樂三}

\begin{pinyinscope}

 自周衰,禮樂壞於戰國而廢絕於秦。漢興,《六經》在者,皆錯亂、散亡、雜偽,而諸儒方共補緝先知先覺指認識事物先於眾人的人。語出《孟子·萬章,以意解詁,未得其真,而讖緯之書出以亂經矣。自鄭玄之徒,號稱大儒,皆主其說,學者由此牽惑沒溺,而時君不能斷決,以為有其舉之,莫可廢也。由是郊、丘、明堂之論,至於紛然而莫知所止。



 《禮》曰:「以禋祀祀昊天上帝。」此天也,玄以為天皇大帝者,北辰耀魄寶也。又曰:「兆五帝於四郊。」此五行精氣之神也,玄以為青帝靈威仰、赤帝赤熛怒、黃帝含樞紐、白帝白招拒、黑帝汁光紀者,五天也。由是有六天之說,後世莫能廢焉。



 唐初《貞觀禮》:冬至祀昊天上帝於圓丘,正月辛日祀感生帝靈威仰於南郊以祈穀,而孟夏雩於南郊,季秋大享於明堂。皆祀五天帝。至高宗時,禮官以謂太史《圓丘圖》,昊天上帝在壇上,而耀魄寶在壇第一等,則昊天上帝非耀魄寶可知,而祠令及《顯慶禮》猶著六天之說。顯慶二年,禮部尚書許敬宗與禮官等議曰:「六天出於緯書,而南郊、圓丘一也,玄以為二物;郊及明堂本以祭天,而玄皆以為祭太微五帝。《傳》曰:『凡祀,啟蟄而郊,郊而後耕。』故『郊祀后稷,以祈農事』。而玄謂周祭感帝靈威仰,配以後稷,因而祈穀。皆繆論也。」由是盡黜玄說,而南郊祈穀、孟夏雩、明堂大享皆祭昊天上帝。



 乾封元年,詔祈穀復祀感帝。二年,又詔明堂兼祀昊天上帝及五帝。開元中,起居舍人王仲丘議曰:「按《貞觀禮》祈穀祀感帝,而《顯慶禮》祀昊天上帝。《傳》曰:『郊而後耕。』《詩》曰:『噫嘻春夏,祈穀於上帝。』《禮記》亦曰:『上辛祈穀於上帝。』而鄭玄乃云:『天之五帝迭王,王者之興必感其一,因別祭尊之。故夏正之月,祭其所生之帝於南郊,以其祖配之。故周祭靈威仰,以後稷配,因以祈穀。』然則祈穀非祭之本意,乃因後稷為配爾,此非祈穀之本義也。夫祈穀,本以祭天也,然五帝者五行之精,所以生九穀也,宜於祈穀祭昊天而兼祭五帝。」又曰:「《月令》,大雩、大享帝,皆盛祭也。而孟夏雩、季秋大享,《貞觀禮》皆祭五方帝,而《顯慶禮》皆祭昊天上帝,宜兼用之以合大雩、大享之義。」既而蕭嵩等撰定《開元禮》,雖未能合古,而天神之位別矣。



 其配神之主,武德中,冬至及孟夏雩祭皇地祇於方丘、神州地祇於北郊,以景帝配;而上辛祈穀祀感帝於南郊,季秋祀五方天帝於明堂,以元帝配。貞觀初,圓丘、明堂、北郊以高祖配,而元帝惟配感帝。高宗永徽二年,以太宗配祀明堂,而有司乃以高祖配五天帝,太宗配五人帝。太尉長孫無忌等與禮官議,以謂:「自三代以來,歷漢、魏、晉、宋,無父子同配於明堂者。《祭法》曰:『周人禘嚳而郊稷,祖文王而宗武王。』鄭玄以祖宗合為一祭,謂祭五帝、五神於明堂,以文、武共配。而王肅駁曰:『古者祖功宗德,自是不毀之名,非謂配食於明堂。』《春秋傳》曰:『禘、郊、祖、宗、報,五者國之典祀也。』以此知祖、宗非一祭。」於是以高祖配於圓丘,太宗配於明堂。



 乾封二年,詔圓丘、五方、明堂、感帝、神州皆以高祖、太宗並配。則天垂拱元年,詔有司議,而成均助教孔玄義、太子右諭德沈伯儀、鳳閣舍人元萬頃範履冰議皆不同,而卒用萬頃、履冰之說。由是郊、丘諸祠,常以高祖、太宗、高宗並配。開元十一年,親享圓丘,中書令張說、衛尉少卿韋縚為禮儀使,乃以高祖配,而罷三祖並配。至二十年,蕭嵩等定禮,而祖宗之配定矣。



 寶應元年,太常卿杜鴻漸、禮儀使判官薛頎歸崇敬等言:「禘者,冬至祭天於圓丘,周人配以遠祖。唐高祖非始封之君,不得為太祖以配天地。而太祖景皇帝受封於唐,即殷之契、周之後稷也,請以太祖郊配天地。」諫議大夫黎幹以謂:「禘者,宗廟之事,非祭天,而太祖非受命之君,不宜作配。」為十詰十難以非之。書奏,不報。乃罷高祖,以景皇帝配。明年旱,言事者以為高祖不得配之過也。代宗疑之,詔群臣議。太常博士獨孤及議曰:「受命於神宗,禹也,而夏后氏祖顓頊而郊鯀;纘禹黜夏,湯也,而殷人郊冥而祖契;革命作周,武王也,而周人郊稷而祖文王。太祖景皇帝始封於唐,天所命也。」由是配享不易。嗚呼,禮之失也,豈獨緯書之罪哉!在於學者好為曲說,而人君一切臨時申其私意,以增多為盡禮,而不知煩數之為黷也。



 古者祭天於圓丘,在國之南,祭地於澤中之方丘,在國之北,所以順陰陽,因高下,而事天地以其類也。其方位既別,而其燎壇、瘞坎、樂舞變數亦皆不同,而後世有合祭之文。則天天冊萬歲元年,其享南郊,始合祭天地。



 睿宗即位,將有事於南郊,諫議大夫賈曾議曰:「《祭法》,有虞氏禘黃帝而郊嚳,夏后氏禘黃帝而郊鯀。郊之與廟,皆有禘也。禘於廟,則祖宗合食於太祖;禘於郊,則地祇群望皆合於圓丘,以始祖配享。蓋有事之大祭,非常祀也。《三輔故事》:『祭於圓丘,上帝、后土位皆南面。』則漢嘗合祭矣。」國子祭酒褚無量、司業郭山惲等皆以曾言為然。是時睿宗將祭地於北郊,故曾之議寢。



 玄宗既已定《開元禮》,天寶元年,遂合祭天地於南郊。是時,神仙道家之說興,陳王府參軍田同秀言:「玄元皇帝降丹鳳門。」乃建玄元廟。二月辛卯,親享玄元皇帝廟;甲午,親享太廟;丙申,有事於南郊。其後遂以為故事,終唐之世,莫能改也。為禮可不慎哉!



 夫男女之不相褻於內外也,況郊廟乎?中宗時,將享南郊,國子祭酒祝欽明言皇后當助祭,大常博士唐紹、蔣欽緒以為不可,左僕射韋巨源獨以欽明說為是。於是以皇后為亞獻,補大臣李嶠等女為齋娘,以執籩豆焉。至德宗貞元六年,又以皇太子為亞獻,親王為終獻。



 《孝經》曰:「宗祀文王於明堂,以配上帝。」而三代有其名而無其制度,故自漢以來,諸儒之論不一,至於莫知所從,則一切臨時增損,而不能合古。然推其本旨,要於布政交神於王者尊嚴之居而已,其制作何必與古同?然為之者至無所據依,乃引天地、四時、風氣、乾坤、五行、數象之類以為仿像,而眾說變不克成。



 隋無明堂,而季秋大亨,常寓雩壇。唐高祖、太宗時,寓於圓丘。貞觀中,禮部尚書豆盧寬、國子助教劉伯莊議:「從昆侖道上層以祭天,下層以布政。」而太子中允孔穎達以為非。侍中魏徵以謂:「五室重屋,上圓下方,上以祭天,下以布政。自前世儒者所言雖異,而以為如此者多同。至於高下廣狹丈尺之制,可以因事制宜也。」秘書監顏師古曰:「《周書》敘明堂有應門、雉門之制,以此知為王者之常居爾。其青陽、總章、玄堂、太廟、左右個,皆路寢之名也。《文王居明堂》之篇,帶弓蜀,禮高禖,九門磔禳,國有酒以合三族,推其事皆與《月令》合,則皆在路寢也。《大戴禮》曰在近郊,又曰文王之廟也,此奚足以取信哉?且門有皋、庫,豈得施於郊野?謂宜近在宮中。」征及師古等皆當世名儒,其論止於如此。



 高宗時改元總章,分萬年置明堂縣,示欲必立之。而議者益紛然,或以為五室,或以為九室。而高宗依兩議,以帟幕為之,與公卿臨觀,而議益不一。乃下詔率意班其制度,至取象黃琮,上設鴟尾,其言益不經,而明堂亦不能立。至則天始毀東都乾元殿,以其地立明堂,其制淫侈,無復可觀,皆不足記。其後火焚之,既而又復立。開元五年,復以為乾元殿而不毀。初,則天以木為瓦,夾紵漆之。二十五年,玄宗遣將作大匠康灊素毀之。灊素以為勞人,乃去其上層,易以真瓦。而迄唐之世,季秋大享,皆寓圓丘。



 《書》曰:「七世之廟,可以觀德。」而禮家之說,世數不同。然自《禮記》《王制》、《祭法》、《禮器》,大儒荀卿、劉歆、班固、王肅之徒,以為七廟者多。蓋自漢、魏以來,創業之君特起,其上世微,又無功德以備祖宗,故其初皆不能立七廟。



 唐武德元年,始立四廟,曰宣簡公、懿王、景皇帝、元皇帝。貞觀九年,高祖崩,太宗詔有司定議。諫議大夫硃子奢請立七廟,虛太祖之室以待。於是尚書八座議:「《禮》曰:『天子三昭三穆,與太祖之廟而七。』晉、宋、齊、梁皆立親廟六,此故事也。」制曰:「可。」於是祔弘農府君及高祖為六室。二十三年,太宗崩,弘農府君以世遠毀,藏夾室,遂祔太宗。及高宗崩,宣皇帝遷於夾室,而祔高宗。皆為六室。



 武氏亂敗,中宗神龍元年,已復京太廟,又立太廟於東都。議立始祖為七廟,而議者欲以涼武昭王為始祖。太常博士張齊賢議以為不可,因曰:「古者有天下者事七世,而始封之君謂之太祖。太祖之廟,百世不遷。至祫祭,則毀廟皆以昭穆合食於太祖。商祖玄王,周祖後稷,其世數遠,而遷廟之主皆出太祖後。故合食之序,尊卑不差。漢以高皇帝為太祖,而太上皇不在合食之列,為其尊於太祖也。魏以武帝為太祖,晉以宣帝為太祖,武、宣而上,廟室皆不合食於祫,至隋亦然。唐受天命,景皇帝始封之君,太祖也,以其世近,而在三昭三穆之內,而光皇帝以上,皆以屬尊不列合食。今宜以景皇帝為太祖,復祔宣皇帝為七室,而太祖以上四室皆不合食於祫。」博士劉承慶、尹知章議曰:「三昭三穆與太祖為七廟者,禮也。而王跡有淺深,太祖有遠近。太祖以功建,昭穆以親崇;有功者不遷,親盡者則毀。今以太祖近而廟數不備,乃欲於昭穆之外,遠立當遷之主以足七廟,而乖迭毀之義,不可。」天子下其議大臣,禮部尚書祝欽明兩用其言,於是以景皇帝為始祖,而不祔宣皇帝。已而以孝敬皇帝為義宗,祔於廟,由是為七室,而京太廟亦七室。中宗崩,中書令姚元之、吏部尚書宋璟以為:「義宗,追尊之帝,不宜列昭穆,而其葬在洛州,請立別廟於東都,而有司時享,其京廟神主藏於夾室」。由是祔中宗,而光皇帝不遷,遂為七室矣。



 睿宗崩,博士陳貞節、蘇獻等議曰:「古者兄弟不相為後,殷之盤庚,不序於陽甲;漢之光武,不嗣於孝成;而晉懷帝亦繼世祖而不繼惠帝。蓋兄弟相代,昭穆位同,至其當遷,不可兼毀二廟。荀卿子曰:『有天下者事七世。』謂從禰以上也。若傍容兄弟,上毀祖考,則天子有不得事七世者矣。孝和皇帝有中興之功而無後,宜如殷之陽甲,出為別廟,祔睿宗以繼高宗。」於是立中宗廟於太廟之西。



 開元十年,詔宣皇帝復祔於正室,謚為獻祖,並謚光皇帝為懿祖,又以中宗還祔太廟,於是太廟為九室。將親祔之,而遇雨不克行,乃命有司行事。寶應二年,祧獻祖、懿祖,祔玄宗、肅宗。自是之後,常為九室矣。



 代宗崩,禮儀使顏真卿議:「太祖、高祖、太宗皆不毀,而代祖元皇帝當遷。」於是遷元皇帝而祔代宗。德宗崩,禮儀使杜黃裳議:「高宗在三昭三穆外,當遷。」於是遷高宗而祔德宗,蓋以中、睿為昭穆矣。順宗崩,當遷中宗,而有司疑之,以謂則天革命,中宗中興之主也。博士王涇、史官蔣武皆以為中宗得失在己,非漢光武、晉元帝之比,不得為中興不遷之君。由是遷中宗而祔順宗。



 自憲宗、穆宗、敬宗、文宗四世祔廟,睿、玄、肅、代以次遷。至武宗崩,德宗以次當遷,而於世次為高祖,禮官始覺其非,以謂兄弟不相為後,不得為昭穆,乃議復祔代宗。而議者言:「已祧之主不得復入太廟。」禮官曰:「昔晉元、明之世,已遷豫章、潁川,後皆復祔,此故事也。」議者又言:「廟室有定數,而無後之主當置別廟。」禮官曰:「晉武帝時,景、文同廟,廟雖六代,其實七主。至元帝、明帝,廟皆十室,故賀循曰:『廟以容主為限,而無常數也。』」於是復祔代宗,而以敬宗、文宗、武宗同為一代。初,玄宗之復祔獻祖也,詔曰:「使親而不盡,遠而不祧。」蓋其率意而言爾,非本於禮也。而後之為說者,乃遷就其事,以謂三昭三穆與太祖祖功宗德三廟不遷為九廟者,周制也。及敬、文、武三宗為一代,故終唐之世,常為九代十一室焉。



 開元五年,太廟四室壞,奉其神主於太極殿,天子素服避正殿,輟朝三日。時將行幸東都,遂謁神主於太極殿而後行。安祿山之亂,宗廟為賊所焚,肅宗復京師,設次光順門外,向廟而哭,輟朝三日。其後黃巢陷京師,焚毀宗廟,而僖宗出奔,神主法物從行,皆為賊所掠。巢敗,復京師,素服哭於廟而後入。



 初,唐建東、西二都,而東都無廟。則天皇后僭號稱周,立周七廟於東都以祀武氏,改西京唐太廟為享德廟。神龍元年,中宗復位,遷武氏廟主於西京,為崇尊廟,而以東都武氏故廟為唐太廟,祔光皇帝以下七室而親享焉。由是東西二都皆有廟,歲時並享。其後安祿山陷兩京,宗廟皆焚毀。肅宗即位,西都建廟作主,而東都太廟毀為軍營,九室神主亡失,至大歷中,始於人間得之,寓於太微宮,不得祔享。自建中至於會昌,議者不一,或以為:「東西二京宜皆有廟,而舊主當瘞,虛其廟以俟,巡幸則載主而行。」或謂:「宜藏其神主於夾室。」或曰:「周豐、洛有廟者,因遷都乃立廟爾,今東都不因遷而立廟,非也。」又曰:「古者載主以行者,惟新遷一室之主爾,未有載群廟之主者也。」至武宗時,悉廢群議,詔有司擇日修東都廟。已而武宗崩,宣宗竟以太微神主祔東都廟焉。



 其追贈皇后、追尊皇太后、贈皇太子往往皆立別廟。其近於禮者,後世當求諸禮;其不合於禮而出其私意者,蓋其制作與其議論皆不足取焉。故不著也。



 宣宗已復河、湟三州七關,歸其功順宗、憲宗而加謚號。博士李稠請改作神主,易書新謚。右司郎中楊發等議,以謂:「古者已祔之主無改作,加謚追尊,非禮也,始於則天,然猶不改主易書,宜以新謚寶冊告於陵廟可也。」是時,宰相以謂士族之廟皆就易書,乃就舊主易書新謚焉。



 禘、祫,大祭也。祫以昭穆合食於太祖,而禘以審諦其尊卑,此祫、禘之義,而為禮者失之。至於年數不同,祖、宗失位,而議者莫知所從。《禮》曰:「三年一祫,五年一禘。」《傳》曰:「五年再殷祭。」高宗上元三年十月當祫,而有司疑其年數。太學博士史玄璨等議,以為:「新君喪畢而祫,明年而禘。自是之後,五年而再祭。蓋後禘去前禘五年,而祫常在禘後三年,禘常在祫後二年。魯宣公八年禘僖公,蓋二年喪畢而祫,明年而禘,至八年而再禘。昭公二十年禘,至二十五年又禘,此可知也。」議者以玄璨等言有經據,遂從之。睿宗崩,開元六年喪畢而祫,明年而禘。自是之後,祫、禘各自以年,不相通數。凡七祫五禘,至二十七年,禘、祫並在一歲,有司覺其非,乃議以為一禘一祫,五年再殷,宜通數。而禘後置祫,歲數遠近,二說不同。鄭玄用高堂隆先三而後二,徐邈先二後三。而邈以謂二禘相去為月六十,中分三十置一祫焉。此最為得,遂用其說。由是一禘一祫,在五年之間,合於再殷之義,而置禘先後,則不同焉。



 禮,禘、祫,太祖位於西而東向,其子孫列為昭穆,昭南向而穆北向。雖已毀廟之主,皆出而序於昭穆。殷、周之興,太祖世遠,而群廟之主皆出其後,故其禮易明。漢、魏以來,其興也暴,又其上世微,故創國之君為太祖而世近,毀廟之主皆在太祖之上,於是禘、祫不得如古。而漢、魏之制,太祖而上,毀廟之主皆不合食。



 唐興,以景皇帝為太祖,而世近在三昭三穆之內,至祫、禘,乃虛東向之位,而太祖與群廟列於昭穆。代宗即位,祔玄宗、肅宗,而遷獻祖、懿祖於夾室。於是太祖居第一室,禘、祫得正其位而東向,而獻、懿不合食。建中二年,太學博士陳京請為獻祖、懿祖立別廟,至禘、祫則享。禮儀使顏真卿議曰:「太祖景皇帝居百代不遷之尊,而禘、祫之時,暫居昭穆,屈己以奉祖宗可也。」乃引晉蔡謨議,以獻祖居東向,而懿祖、太祖以下左右為昭穆。由是議者紛然。



 貞元七年,太常卿裴鬱議,以太祖百代不遷,獻、懿二祖親盡廟遷而居東向,非是,請下百寮議。工部郎中張薦等議與真卿同。太子左庶子李嶸等七人曰:「真卿所用,晉蔡謨之議也,謨為『禹不先鯀』之說,雖有其言,當時不用。獻、懿二祖宜藏夾室,以合《祭法》『遠廟為祧,而壇、墠有禱則祭,無禱則止』之義。吏部郎中柳冕等十二人曰:「《周禮》有先公之祧,遷祖藏於後稷之廟,其周未受命之祧乎?又有先王之祧,其遷主藏於文、武之廟,其周已受命之祧乎?今獻祖、懿祖,猶周先公也,請築別廟以居之。」司勛員外郎裴樞曰:「建石室於寢園以藏神主,至禘、祫之歲則祭之。」考功員外郎陳京、同官縣尉仲子陵皆曰:「遷神主於德明、興聖廟。」京兆少尹韋武曰:「祫則獻祖東向,禘則太祖東向。」十一年,左司郎中陸淳曰:「議者多矣,不過三而已。一曰復太祖之正位,二曰並列昭穆而虛東向,三曰祫則獻祖,禘則太祖,迭居東向。而復正太祖之位為是。然太祖復位,則獻、懿之主宜有所歸。一曰藏諸夾室,二曰置之別廟,三曰遷於園寢,四曰祔於興聖。然而藏諸夾室,則無饗獻之期;置之別廟,則非《禮經》之文;遷於寢園,則亂宗廟之儀。唯祔於興聖為是。」至十九年,左僕射姚南仲等獻議五十七封,付都省集議。戶部尚書王紹等五十五人請遷懿祖祔興聖廟,議遂定,由是太祖始復東向之位。



 若諸臣之享其親,廟室、服器之數,視其品。開元十二年著令:一品、二品四廟,三品三廟,五品二廟,嫡士一廟,庶人祭於寢。及定禮:二品以上四廟,三品三廟,三品以上不須爵者亦四廟,四廟有始封為五廟,四品、五品有兼爵亦三廟,六品以下達於庶人,祭於寢。天寶十載,京官正員四品清望及四品、五品清官,聽立廟,勿限兼爵;雖品及而建廟未逮,亦聽寢祭。



 廟之制,三品以上九架,廈兩旁。三廟者五間,中為三室,左右廈一間,前後虛之,無重栱、藻井。室皆為石室一,於西墉三之一近南,距地四尺,容二主。廟垣周之,為南門、東門,門屋三室,而上間以廟,增建神廚於廟東之少南,齋院於東門之外少北,制勿逾於廟。三品以上有神主,五品以上有幾筵。牲以少牢,羊、豕一,六品以下特豚,不以祖禰貴賤,皆子孫之牲。牲闕,代以野獸。五品以上室異牲,六品以下共牲。二品以上室以籩豆十,三品以八,四品、五品以六。五品以上室皆簠二、簋二、甒二,鈃二、俎三、尊二、罍二、勺二、爵六,盤一、坫一、篚一、牙盤胙俎一。祭服,三品以上玄冕,五品以上爵弁,六品以下進賢冠,各以其服。



 凡祔皆給休五日,時享皆四日。散齋二日於正寢,致齋一日於廟,子孫陪者齋一宿於家。始廟則署主而祔,後喪闋乃祔,喪二十八月上旬卜而祔,始神事之矣。王公之主載以輅,夫人之主以翟車,其餘皆以輿。天子以四孟、臘享太廟,諸臣避之,祭仲而不臘。三歲一祫,五歲一禘。若祔、若常享、若禘祫,卜日、齋戒、省牲、視滌、濯鼎鑊,亨牲、實饌、三獻、飲福、受胙進退之數,大抵如宗廟之祀。以國官亞、終獻,無則以親賓,以子弟。



 其後不卜日,而筮用亥。祭寢者,春、秋以分,冬、夏以至日。若祭春分,則廢元日。然元正,歲之始,冬至,陽之復,二節最重。祭不欲數,乃廢春分,通為四。



 祠器以烏漆,差小常制。祭服以進賢冠,主婦花釵禮衣,後或改衣冠從公服,無則常服。



 凡祭之在廟、在寢,既畢,皆親賓子孫慰,主人以常服見。若宗子有故,庶子攝祭,則祝曰:「孝子某使介子某執其常事。」通祭三代,而宗子卑,則以上牲祭宗子家,祝曰:「孝子某為其介子某薦其常事。」庶子官尊而立廟,其主祭則以支庶封官依大宗主祭,兄陪於位。以廟由弟立,已不得延神也。或兄弟分官,則各祭考妣于正寢。



 古殤及無後皆祔食於祖,無祝而不拜,設坐祖左而西向。亞獻者奠,祝乃奠之,一獻而止。其後廟制設幄,當中南向,祔坐無所施,皆祭室戶外之東而西向。親伯叔之無後者示付曾祖,親昆弟及從父昆弟祔於祖,親子侄祔於禰。寢祭之位西上,祖東向而昭穆南北,則伯叔之祔者居禰下之穆位北向,昆弟、從父昆弟居祖下之昭位南向,子侄居伯叔之下穆位北向,以序尊卑。凡殤、無後,以周親及大功為斷。



 古者廟於大門內,秦出寢於陵側,故王公亦建廟於墓。既廟與居異,則宮中有喪而祭。三年之喪,齊衰、大功皆廢祭;外喪,齊衰以下行之。



\end{pinyinscope}