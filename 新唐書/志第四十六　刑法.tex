\article{志第四十六 刑法}

\begin{pinyinscope}

 古之為國者,議事以制,不為刑闢,懼民之知爭端也。後世作為刑書,惟恐不備首。祖述堯舜,憲章文武,崇尚「禮樂」、「仁義」、「忠吮、,俾民之知所避也。其為法雖殊,而用心則一,蓋皆欲民之無犯也。然未知夫導之以德、齊之以禮,而可使民遷善遠罪而不自知也。



 唐之刑書有四,曰:律、令、格、式。令者,尊卑貴賤之等數,國家之制度也;格者,百官有司之所常行之事也;式者,其所常守之法也。凡邦國之政,必從事於此三者。其有所違及人之為惡而入於罪戾者,一斷以律。律之為書,因隋之舊,為十有二篇:一曰名例,二曰衛禁,三曰職制,四曰戶婚,五曰廄庫,六曰擅興,七曰賊盜,八曰鬥訟,九曰詐偽,十曰雜律,十一曰捕亡,十二曰斷獄。



 其用刑有五:一曰笞。笞之為言恥也;凡過之小者,捶撻以恥之。漢用竹,後世更以楚。《書》曰「撲作教刑」是也。二曰杖。杖者,持也;可持以擊也。《書》曰「鞭作官刑」是也。三曰徒。徒者,奴也;蓋奴辱之。《周禮》曰:「其奴,男子入於罪隸,任之以事,寘之圜土而教之,量其罪之輕重,有年數而舍。四曰流。《書》云「流宥五刑」,謂不忍刑殺,宥之於遠也。五曰死。乃古大闢之刑也。



 自隋以前,死刑有五,曰:罄、絞、斬、梟、裂。而流、徒之刑,鞭笞兼用,數皆逾百。至隋始定為:笞刑五,自十至於五十;杖刑五,自六十至於百;徒刑五,自一年至於三年;流刑三,自一千里至於二千里;死刑二,絞、斬。除其鞭刑及梟首、軒裂之酷。又有議、請、減、贖、當、免之法。唐皆因之。然隋文帝性刻深,而煬帝昏亂,民不勝其毒。



 唐興,高祖入京師,約法十二條,惟殺人、劫盜、背軍、叛逆者死。及受禪,命納言劉文靜等損益律令。武德二年,頒新格五十三條,唯吏受賕、犯盜、詐冒府庫物,赦不原。凡斷屠日及正月、五月、九月不行刑。四年,高祖躬錄囚徒,以人因亂冒法者眾,盜非劫傷其主及征人逃亡、官吏枉法,皆原之。已而又詔僕射裴寂等十五人更撰律令,凡律五百,麗以五十三條。流罪三,皆加千里;居作三歲至二歲半者悉為一歲。餘無改焉。



 太宗即位,詔長孫無忌、房玄齡等復定舊令,議絞刑之屬五十,皆免死而斷右趾。既而又哀其斷毀支體,謂侍臣曰:「肉刑,前代除之久矣,今復斷人趾,吾不忍也。」王珪、蕭瑀、陳叔達對曰:「受刑者當死而獲生,豈憚去一趾?去趾,所以使見者知懼。今以死刑為斷趾,蓋寬之也。」帝曰:「公等更思之。」其後蜀王法曹參軍裴弘獻駁律令四十餘事,乃詔房玄齡與弘獻等重加刪定。玄齡等以謂「古者五刑,刖居其一。及肉刑既廢,今以笞、杖、徒、流、死為五刑,而又刖足,是六刑也。」於是除斷趾法,為加役流三千里,居作二年。



 太宗嘗覽《明堂針灸圖》,見人之五藏皆近背,針灸失所,則其害致死,嘆曰:「夫箠者,五刑之輕;死者,人之所重。安得犯至輕之刑而或致死?」遂詔罪人無得鞭背。



 五年,河內人李好德坐妖言下獄,大理丞張蘊古以為好德病狂瞀,法不當坐。治書侍御史權萬紀劾蘊古相州人,好德兄厚德方為相州刺史,故蘊古奏不以實。太宗怒,遽斬蘊古,既而大悔,因詔「死刑雖令即決,皆三覆奏」。久之,謂群臣曰:「死者不可復生。昔王世充殺鄭頲而猶能悔,近有府史取賕不多,朕殺之,是思之不審也。決囚雖三覆奏,而頃刻之間,何暇思慮?自今宜二日五覆奏。決日,尚食勿進酒肉,教坊太常輟教習,諸州死罪三覆奏,其日亦蔬食,務合禮撤樂、減膳之意。」



 故時律,兄弟分居,廕不相及,而連坐則俱死。同州人房強以弟謀反當從坐,帝因錄囚為之動容,曰:「反逆有二:興師動眾一也,惡言犯法二也。輕重固異,而鈞謂之反,連坐皆死,豈定法耶?」玄齡等議曰:「禮,孫為父尸,故祖有陰孫令,是祖孫重而兄弟輕。」於是令:「反逆者,祖孫與兄弟緣坐,皆配沒;惡言犯法者,兄弟配流而已。玄齡等遂與法司增損隋律,降大闢為流者九十二,流為徒者七十一,以為律;定令一千五百四十六條,以為令;又刪武德以來敕三千餘條為七百條,以為格;又取尚書省列曹及諸寺、監、十六衛計帳以為式。



 凡州縣皆有獄,而京兆、河南獄治京師,其諸司有罪及金吾捕者又有大理獄。京師之囚,刑部月一奏,御史巡行之。每歲立春至秋及大祭祀、致齊,朔望、上下弦、二十四氣、雨及夜未明,假日、斷屠月,皆停死刑。



 京師決死,涖以御史、金吾,在外則上佐,餘皆判官涖之。五品以上罪論死,乘車就刑,大理正涖之,或賜死於家。凡囚已刑,無親屬者,將作給棺,瘞於京城七里外,壙有甎銘,上揭以榜,家人得取以葬。



 諸獄之長官,五日一慮囚。夏置漿飲,月一沐之;疾病給醫藥,重者釋械,其家一人入侍,職事散官三品以上,婦女子孫二人入侍。



 天下疑獄讞大理寺不能決,尚書省眾議之,錄可為法者送秘書省。奏報不馳驛。經覆而決者,刑部歲以正月遣使巡覆,所至,閱獄囚杻校、糧餉,治不如法者。杻校鉗鎖皆有長短廣狹之制,量囚輕重用之。



 囚二十日一訊,三訊而止,數不過二百。



 凡杖,皆長三尺五寸,削去節目。訊杖,大頭徑三分二厘,小頭二分二厘。常行杖,大頭二分七厘,小頭一分七厘。笞杖,大頭二分,小頭一分有半。



 死罪校而加杻,官品勛階第七者,鎖禁之。輕罪及十歲以下至八十以上者、廢疾、侏儒、懷妊皆頌系以待斷。



 居作者著鉗若校,京師隸將作,女子隸少府縫作。旬給假一日,臘、寒食二日,毋出役院。病者釋鉗校、給假,疾差陪役。謀反者男女奴婢沒為官奴婢,隸司農,七十者免之。凡役,男子入于蔬圃,女子入於廚饎。



 流移人在道疾病,婦人免乳,祖父母、父母喪,男女奴婢死,皆給假,授程糧。



 非反逆緣坐,六歲縱之,特流者三歲縱之,有官者得復仕。



 初,太宗以古者斷獄,訊於三槐、九棘,乃詔:「死罪,中書、門下五品以上及尚書等平議之;三品以上犯公罪流、私罪徒,皆不追身。」凡所以纖悉條目,必本於仁恕。然自張蘊古之死也,法官以失出為誡,有失入者,又不加罪,自是吏法稍密。帝以問大理卿劉德威,對曰:「律,失入減三等,失出減五等。今失入無辜,而失出為大罪,故吏皆深文。」帝矍然,遂命失出入者皆如律,自此吏亦持平。



 十四年,詔流罪無遠近皆徙邊要州。後犯者浸少。十六年,又徙死罪以實西州,流者戍之,以罪輕重為更限。



 廣州都督賞仁弘嘗率鄉兵二千助高祖起,封長沙郡公。仁弘交通豪酋,納金寶,沒降獠為奴婢,又擅賦夷人。既還,有舟七十。或告其贓,法當死。帝哀其老且有功,因貸為庶人,乃召五品以上,謂曰:「賞罰所以代天行法,今朕寬仁弘死,是自弄法以負天也。人臣有過,請罪於君,君有過,宜請罪於天。其令有司設槁席於南郊三日,朕將請罪。」房玄齡等曰:「寬仁弘不以私而以功,何罪之請?」百僚頓首三請,乃止。



 太宗以英武定天下,然其天姿仁恕。初即位,有勸以威刑肅天下者,魏徵以為不可,因為上言王政本於仁恩,所以愛民厚俗之意,太宗欣然納之,遂以寬仁治天下,而於刑法尤慎。四年,天下斷死罪二十九人。六年,親錄囚徒,閔死罪者三百九十人,縱之還家,期以明年秋即刑;及期,囚皆詣朝堂,無後者,太宗嘉其誠信,悉原之。然嘗謂群臣曰:「吾聞語曰:一歲再赦,好人暗啞。吾有天下未嘗數赦者,不欲誘民於幸免也。」自房玄齡等更定律、令、格、式,訖太宗世,用之無所變改。



 高宗初即位,詔律學之士撰《律疏》。又詔長孫無忌等增損格敕,其曹司常務曰《留司格》,頒之天下曰《散頒格》。龍朔、儀鳳中,司刑太常伯李敬玄、左僕射劉仁軌相繼又加刊正。



 武后時,內史裴居道、鳳閣侍郎韋方質等又刪武德以後至於垂拱詔敕為新格,藏於有司,曰《垂拱留司格》。神龍元年,中書令韋安石又續其後至於神龍,為《散頒格》。睿宗即位,戶部尚書岑羲等又著《太極格》。



 玄宗開元三年,黃門監盧懷慎等又著《開元格》。至二十五年,中書令李林甫又著新格,凡所損益數千條,明年,吏部尚書宋璟又著後格,皆以開元名書。天寶四載,又詔刑部尚書蕭炅稍復增損之。肅宗、代宗無所改造。至德宗時,詔中書門下選律學之士,取至德以來制敕奏讞,掇其可為法者藏之,而不名書。



 憲宗時,刑部侍郎許孟容等刪天寶以後敕為《開元格後敕》。



 文宗命尚書省郎官各刪本司敕,而丞與侍郎覆視,中書門下參其可否而奏之,為《大和格後敕》。開成三年,刑部侍郎狄兼篸採開元二十六年以後至於開成制敕,刪其繁者,為《開成詳定格》。



 宣宗時,左衛率府倉曹參軍張戣以刑律分類為門,而附以格敕,為《大中刑律統類》,詔刑部頒行之。



 此其當世所施行而著見者,其餘有其書而不常行者,不足紀也。



 《書》曰:「慎乃出令。」蓋法令在簡,簡則明,行之在久,久則信,而中材之主,庸愚之吏,常莫克守之,而喜為變革。至其繁積,則雖有精明之士不能遍習,而吏得上下以為奸,此刑書之弊也。蓋自高宗以來,其大節鮮可紀,而格令之書,不勝其繁也。



 高宗既昏懦,而繼以武氏之亂,毒流天下,幾至於亡。自永徽以後,武氏已得志,而刑濫矣。當時大獄,以尚書刑部、御史臺、大理寺雜按,謂之「三司」,而法吏以慘酷為能,至不釋枷而笞棰以死者,皆不禁。律有杖百,凡五十九條,犯者或至死而杖未畢,乃詔除其四十九條,然無益也。武後已稱制,懼天下不服,欲制以威,乃修後周告密之法,詔官司受訊,有言密事者,馳驛奏之。自徐敬業、越王貞、瑯邪王沖等起兵討亂,武氏益恐。乃引酷吏周興、來俊臣輩典大獄,與侯思止、王弘義、郭弘霸、李敬仁、康韋、衛遂忠等集告事數百人,共為羅織,構陷無辜。自唐之宗室與朝廷之士,日被告捕,不可勝數,天下之人,為之仄足,如狄仁傑、魏元忠等皆幾不免。左臺御史周矩上疏曰:「比奸憸告訐,習以為常。推劾之吏,以深刻為功,鑿空爭能,相矜以虐。泥耳囊頭,摺脅簽爪,縣發燻耳,臥鄰穢溺,刻害支體,糜爛獄中,號曰『獄持』;閉絕食飲,晝夜使不得眠,號曰『宿囚』。殘賊威暴,取決目前。被誣者茍求得死,何所不至?為國者以仁為宗,以刑為助,周用仁而昌,秦用刑而亡。願陛下緩刑用仁,天下幸甚!」武后不納。麟臺正字陳子昂亦上書切諫,不省。及周興、來俊臣等誅死,後亦老,其意少衰,而狄仁傑、姚崇、宋璟、王及善相與論垂拱以來酷濫之冤,太后感寤,由是不復殺戮。然其毒虐所被,自古未之有也。大足元年,乃詔法司及推事使敢多作辯狀而加語者,以故入論。中宗、韋後繼以亂敗。



 玄宗自初即位,勵精政事,常自選太守、縣令,告戒以言,而良吏布州縣,民獲安樂,二十年間,號稱治平,衣食富足,人罕犯法。是歲刑部所斷天下死罪五十八人,往時大理獄,相傳鳥雀不棲,至是有鵲巢其庭樹,群臣稱賀,以為幾致刑錯。然而李林甫用事矣,自來俊臣誅後,至此始復起大獄,以誣陷所殺數十百人,如韋堅、李邕等皆一時名臣,天下冤之。而天子亦自喜邊功,遣將分出以擊蠻夷,兵數大敗,士卒死傷以萬計,國用耗乏,而轉漕輸送,遠近煩費,民力既弊,盜賊起而獄訟繁矣。天子方側然,詔曰:「徒非重刑,而役者寒暑不釋械系。杖,古以代肉刑也,或犯非巨蠹而棰以至死,其皆免,以配諸軍自效。民年八十以上及重疾有罪,皆勿坐。侍丁犯法,原之俾終養。」以此施德其民。然巨盜起,天下被其毒,民莫蒙其賜也。



 安、史之亂,偽官陸大鈞等背賊來歸,及慶緒奔河北,脅從者相率待罪闕下,自大臣陳希烈等合數百人。以御史大夫李峴、中丞崔器等為三司使,而肅宗方喜刑名,器亦刻深,乃以河南尹達奚珣等三十九人為重罪,斬於獨柳樹者十一人,珣及韋恆腰斬,陳希烈等賜自盡於獄中者七人,其餘決重杖死者二十一人。以歲除日行刑,集百官臨視,家屬流竄。初,史思明、高秀巖等皆自拔歸命,聞珣等被誅,懼不自安,乃復叛。而三司用刑連年,流貶相繼。及王璵為相,請詔三司推核未已者,一切免之。然河北叛人畏誅不降,兵連不解,朝廷屢起大獄。肅宗後亦悔,嘆曰:「朕為三司所誤。」臨崩,詔天下流人皆釋之。



 代宗性仁恕,常以至德以來用刑為戒。及河、洛平,下詔河北、河南吏民任偽官者,一切不問。得史朝義將士妻子四百餘人,皆赦之。僕固懷恩反,免其家,不緣坐。劇賊高玉聚徒南山,啗人數千,後擒獲,會赦,代宗將貸其死,公卿議請為菹醢,帝不從,卒杖殺之。諫者常諷帝政寬,故朝廷不肅。帝笑曰:「艱難時無以逮下,顧刑法峻急,有威無恩,朕不忍也。」即位五年,府縣寺獄無重囚。故時,別敕決人捶無數。寶應元年,詔曰:「凡制敕與一頓杖者,其數止四十;至到與一頓及重杖一頓、痛杖一頓者,皆止六十。」



 德宗性猜忌少恩,然用刑無大濫。刑部侍郎班宏言:「謀反、大逆及叛、惡逆四者,十惡之大也,犯者宜如律。其餘當斬、絞刑者,決重杖一頓處死,以代極法。」故時,死罪皆先決杖,其數或百或六十,於是悉罷之。



 憲宗英果明斷,自即位數誅方鎮,欲治僭叛,一以法度,然於用刑喜寬仁。是時,李吉甫、李絳為相。吉甫言:「治天下必任賞罰,陛下頻降赦令,蠲逋負,賑饑民,恩德至矣。然典刑未舉,中外有懈怠心。」絳曰:「今天下雖未大治,亦未甚亂,乃古平國用中典之時。自古欲治之君,必先德化,至暴亂之世,始專任刑法。吉甫之言過矣。」憲宗以為然。司空于頔亦諷帝用刑以收威柄,帝謂宰相曰:「頔懷奸謀,欲朕失人心也。」元和八年,詔:「兩京、關內、河東、河北、淮南、山南東西道死罪十惡、殺人、鑄錢、造印,若強盜持仗劫京兆界中及它盜贓逾三匹者,論如故。其餘死罪皆流天德五城,父祖子孫欲隨者,勿禁。」蓋刑者,政之輔也。政得其道,仁義興行,而禮讓成俗,然猶不敢廢刑,所以為民防也,寬之而已。今不隆其本、顧風俗謂何而廢常刑,是弛民之禁,啟其奸,由積水而決其防。故自玄宗廢徒杖刑,至是又廢死刑,民未知德,而徒以為幸也。



 穆宗童昏,然頗知慎刑法,每有司斷大獄,令中書舍人一人參酌而輕重之,號「參酌院」。大理少卿崔杞奏曰:「國家法度,高祖、太宗定制二百餘年矣。《周禮》:正月布刑,張之門閭及都鄙邦國,所以屢丁寧,使四方謹行之。大理寺,陛下守法之司也。今別設參酌之官,有司定罪,乃議其出入,是與奪系於人情,而法官不得守其職。昔子路問政,孔子曰:『必也正名乎。』臣以為參酌之名不正,宜廢。」乃罷之。



 大和六年,興平縣民上官興以醉殺人而逃,聞械其父,乃自歸。京兆尹杜悰、御史中丞宇文鼎以其就刑免父,請減死。詔兩省議,以為殺人者死,百王所守;若許以生,是誘之殺人也。諫官亦以為言。文宗以興免父囚,近於義,杖流靈州,君子以為失刑。文宗好治,躬自謹畏,然閹宦肆孽不能制。至誅殺大臣,夷滅其族,濫及者不可勝數,心知其冤,為之飲恨流涕,而莫能救止。蓋仁者制亂,而弱者縱之,然則剛強非不仁,而柔弱者仁之賊也。



 武宗用李德裕誅劉稹等,大刑舉矣,而性嚴刻。故時,竊盜無死,所以原民情迫於饑寒也,至是贓滿千錢者死,至宣宗乃罷之。而宣宗亦自喜刑名,常曰:「犯我法,雖子弟不宥也。」然少仁恩,唐德自是衰矣。



 蓋自高祖、太宗除隋虐亂,治以寬平,民樂其安,重於犯法,致治之美,幾乎三代之盛時。考其推心惻物,其可謂仁矣!自高宗、武后以來,毒流邦家,唐祚絕而復續。玄宗初勵精為政,二十年間,刑獄減省,歲斷死罪才五十八人。以此見致治雖難,勉之則易,未有為而不至者。自此以後,兵革遂興,國家多故,而人主規規,無復太宗之志。其雖有心於治者,亦不能講考大法,而性有寬猛,凡所更革,一切臨時茍且,或重或輕,徒為繁文,不足以示後世。而高祖、太宗之法,僅守而存。故自肅宗以來,所可書者幾希矣;懿宗以後,無所稱焉。



\end{pinyinscope}