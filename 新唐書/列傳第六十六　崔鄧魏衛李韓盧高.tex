\article{列傳第六十六 崔鄧魏衛李韓盧高}

\begin{pinyinscope}

 崔光遠,系出博陵,後徙靈昌。祖敬嗣,嗜酒摴博。中宗在房州,吏多肆慢不為禮觀的一個偉大革命,並強調它是一種探究人類發展過程的科,敬嗣為刺史,獨盡誠推奉,儲給豐衍,帝德之。及反正,有與敬嗣同姓名者,每擬官,帝輒超拜,後召見,悟非是。訪真敬嗣,已死,即授其子汪五品官。汪生光遠,勇決任氣,長六尺,瞳子白黑分明。開元末,為唐安令,與楊國忠善,累遷京兆少尹,為吐蕃吊祭使,還,會玄宗西狩,詔留光遠為京兆尹、西京留守、採訪使。乘輿已出,都人亂,火左藏大盈庫,爭輦財珍,至乘驢入宮殿者。光遠乃募官攝府、縣,誰何宮闕,斬十數人,乃定。因偽使其子東見祿山,而祿山先署張休為京兆尹,由是追休,授光遠故官。俄而同羅背賊,以廄馬二千出奔,賊將孫孝哲、安神威招之不得,神威憂死,官吏驚走,獄囚皆逸。光遠以為賊且走,命人守神威、孝哲等第,斬曳落河二人。孝哲馳白祿山,光遠懼,與長安令蘇震出開遠門,使人奔呼曰:「尹巡門!」門兵具器仗迎謁,至,皆斬之,募得百餘人,遂趨靈武。肅宗嘉之,擢拜御史大夫,復為京兆尹,遣到渭北募僑民。會賊黨剽涇陽,休祠房,椎牛呼飲。光遠刺知之,率兵夜趨其所,使百騎彀滿狙其前,命驍士合噪。賊醉,不能師,斬其徒二千,得馬千噭,俘一酋長以獻。自是,賊常避其鋒。扈帝還,改禮部尚書、鄴國公,封實戶三百。



 乾元元年,繇汴州刺史代蕭華為魏州節度使。初,郭子儀與賊戰汲郡,光遠裁率汴師千人援之,不甚力。及守魏,使將軍李處崟拒賊,子儀不救,戰不勝,奔還,賊因傅城下詭呼曰:「處崟召我而不出,何也?」光遠信之,斬處崟。處崟善戰,眾倚以為重,及死,人益危。魏城經袁知泰、能元皓等完築,牢甚,光遠不能守,夜潰圍出,奔京師。帝赦其罪,拜太子少保。會襄州將康楚元、張嘉延反,陷荊、襄諸州,因拜持節荊、襄招討,充山南東道兵馬都使。又徙鳳翔尹。先是,岐、隴賊郭愔等掠州縣,峙五堡,光遠至,遣官喻降之。既而沉飲不親事,愔等陰約黨項及奴剌、突厥,敗韋倫於秦、隴,殺監軍使。帝怒光遠無狀,召還。復使節度劍南。會段子璋反東川,李奐敗走成都,光遠進討平之。然不能禁士卒剽掠士女,至斷腕取金者,夷殺數千人。帝詔監軍按其罪,以憂卒。



 鄧景山,曹州人。本以文吏進,累至監察御史。至德初,擢拜青齊節度使,徙淮南。為政簡肅。有鼉集城門,鄧班語景山曰:「介物也。失所次,金不從革之象。其有兵乎?」未幾,宋州刺史劉殿反。初,展有異志,淮西節度使王仲昇表其狀,詔遷揚州長史兼江淮都統,密詔景山執送京師。展知之,擁兵二萬度淮。景山逆擊不勝,奔壽州,因引平盧節度副使田神功討展。神功兵至揚州,大掠居人,發塚墓,大食、波斯賈胡死者數千人。展叛凡三月平,追景山入朝,拜尚書左丞,以崔圓代之。



 王思禮在太原,儲廥贏衍,請輸半以實京師。會卒,管崇嗣代之,政弛不治,數月,為下盜費略盡。帝聞,即以景山為太原尹,封南陽郡公。至則振核紀綱,檢覆幹隱,眾大懼。而景山清約,子弟饌不過草具,用器止烏漆,待上賓惟豚、魚而已,取倉粟紅腐者食之,兼給麾下,麾下怨訕。左右白景山,景山曰:「此不食,留將安用邪?」因慢罵,士皆羞忿。有裨校抵死,諸將請贖,不許;其弟請代,不許;請納一馬贖,景山乃許減死。眾怒曰:「吾屬命才一馬直乎?」景山護失,叱遣之。少將黃抱節因眾怒作亂,景山遇害,時寶應元年也。肅宗以其統馭失方,不復究驗,遣使喻撫其軍,軍中請辛云京為節度,詔可。景山與劉晏善,其後家寒窶,晏屢經紀之,嫁其孤女。謚曰敬。



 崔瓘,博陵人,以士行修謹聞。累官至澧州刺史,不為煩苛,人便安之,流亡還歸,居二年,增戶數萬。詔特進五階,以寵異政。大歷中,遷湖南觀察使,時將吏習寬弛,不奉法,瓘稍以禮法繩裁之,下多怨。別將臧玠、判官達奚覯忿爭,覯曰:「今幸無事。」玠曰:「欲有事邪?」拂衣去,是夜以兵殺覯。瓘聞難,惶懼走,遇害,帝悼惜之。



 魏少游,字少游,邢州鉅鹿人,以吏乾稱。天寶末,累遷朔方水陸轉運副使。肅宗幸靈武,杜鴻漸等奉迎,而留少游繕治宮室。少游大為殿宇幄帟,皆象宮闕,諸王、公主悉有次舍,供擬窮水陸。又有千餘騎,鎧幟光鮮,振旅以入。帝見宮殿,不悅曰:「我至此欲就大事,安用是為?」稍命去之。除左司郎中。兩京平,封鉅鹿縣侯,遷陜州刺史。王師潰於鄴,河、洛震駭,少游鎮守自若。擢京兆尹。李輔國以其不附己,改衛尉卿。會率群臣馬助軍,少游與漢中王瑀持異,帝怒,貶渠州長史。復為京兆尹,始請:「中書門下省五品、尚書省四品、諸司正員三品、諸王、駙馬期以上親及婿若甥,不得任京兆官。」詔可。大歷二年,為江西觀察使,進刑部尚書,改封趙國公。六年卒,贈太子太師。



 少游四為京兆,雖無赫赫名,然善任人,緣飾規檢,有足稱者。



 衛伯玉,史失其何所人。少習武技,為有力。天寶中,從安西府,積勞至員外諸衛將軍。肅宗即位,慨然願立功,乃歸長安,領神策兵馬使,出鎮陜州行營。乾元二年,賊將李歸仁以騎五千入寇,伯玉與戰強子阪,破之,獲馬六百匹。遷羽林大將軍,徙四鎮、北庭行營節度使,俄為神策軍節度。史思明遣子朝義夜襲陜,將動京師,伯玉迎擊,破之於永寧。加特進,封河東郡公。廣德元年,代宗幸陜,以伯玉有幹略,可方面大事,乃拜荊南節度使,進封城陽郡王。大歷初,以母憂當代,諷將吏留己,復詔節度荊南,議者醜其留。十一年,歸京師。卒。



 李澄,遼東襄平人,隋蒲山公寬之遠胄。以勇剽隸江淮都統李峘府為偏將。又從永平節度李勉軍,勉帥汴,表澄滑州刺史。李希烈陷汴,勉走,澄以城降賊,希烈以為尚書令,節度永平軍。興元元年,澄遣盧融間道奉表詣行在。德宗嘉之,署帛詔內蜜丸,授澄刑部尚書、汴滑節度使,澄未即宣,乃行勒訓士馬。希烈疑,以養子六百戍之。賊急攻寧陵,邀澄至石柱,澄密令焚營為驚遁者,養子輩果乘以剽掠,澄盡斬之,以告,希烈不能詰。賊遣將翟崇暉率精兵寇陳州,未還,汴軍寡,澄度不能制己,又中官薛盈珍持節至,封澄武威郡王,賜實封,乃燔賊旗節自歸。希烈既失澄,而崇暉復敗,由是奔汝南。



 澄引兵將取汴,屯其北門不敢進,及劉洽師屯東門,賊將田懷珍納之。比澄入,洽已保子城矣。澄乃舍浚儀,兩軍士日爭忿,未能安。會鄭州賊將孫液送款於澄,澄遣子清馳赴。先此,河陽李芃使偏將雍希顥攻鄭,數殘剽,液拒之。及納清,希顥大怒,急攻鄭。清助守,殺河陽兵數千,希顥焚陽武去,澄遂如鄭。詔授清檢校太子賓客,易名克寧。貞元初,遷澄檢校尚書左僕射、養成軍節度使。二年卒,年五十四,贈司空。澄始封隴西公,後乃進王爵,每上章,必疊署二封,士大夫笑其野。



 澄之喪,克寧閟不發,閱旬日,欲自領事,其行軍司馬馬鉉不許,克寧殺之,墨絰,加卒嬰城,將為亂。劉洽以兵屯境上,遣使諭止,遂自戢,然道閉者半月。詔以賈耽代鎮,克寧乃護喪歸,悉索府中財夜出,軍士從剽之殆盡,澄柩至京,猶賜克寧莊一區、錢千緡、粟麥數千石云。



 韓全義,家素寒,史失其先世。興卒伍,以巧佞事宦者竇文場,擢累長武城使,進拜夏綏銀宥節度使,詔以長武兵赴之。全義素懦貪,無紀律,為下靳狎。詔未下,軍中遍知之,謀曰:「夏州沙磧,無樹藿生業,不可往。」是夜,噪而亂,全義縋以逸,殺其親將王棲巖、趙虔曜等,軍虞候高崇文誅亂首,眾乃定,全義得赴屯。



 吳少誠以蔡拒命,詔合十七鎮兵討之。時軍無帥統,惟以奄豎監之,遂敗於小溵。德宗以文場素為全義地,因用為淮西行營招討使,以陳許節度使上官水兌副之,諸鎮兵皆屬。全義無它方略,號令悉稟監軍,每議攻戰,宦豎十數紛爭帳中,小人好自異,互詆訾不能決。賊知之,數請戰。遇賊廣利城,方暑,地沮洳,士皆病癘,全義未嘗存之。既戰,師皆潰,退保五樓,賊移屯逼之,乃與監軍賈英秀等保溵水,不能固,又入屯陳州。是時,唯陳許將孟元陽、神策將蘇光榮守溵水,全義誘潞、滑州數大將殺之,然卒不振。宦人共掩其敗,帝不知。少誠度無能為,即謾書謝監軍,求洗前咎。帝下其議,宰相賈耽以為五樓之敗,賊不追者,以冀恩耳,請納其誠。帝然之。



 全義班師,過闕下,托疾不入謁。司馬崔放見帝,謝無功。帝曰;「全義誘少誠歸國,功大矣!何必殺敵乃為功邪?」還屯夏州,中人即第宴賚,然卒不見天子去。時恨帝失政,使奸人得肆云。憲宗在籓,疾之,既嗣位,全義大懼,願入覲,不復用,以太子少保致仕卒。其子獻女樂八人,帝不納,曰:「我方以儉治天下,惡用是為?」



 盧從史,其先在元魏時為盛族,後徙籍不常。父虔,好學,由進士第歷御史、秘書監。從史少好騎射,游澤、潞間,節度使李長榮署為督將。貞元後,蕃臣缺,德宗必取本軍所喜戴者授之。從史在潞,奸獪得士心,又善附迎中人,會長榮卒,即擢拜昭義節度副大使。既得志,浸恣不道,至奪部將妻,而能辯給粉澤其非。府屬孔戡等屢以直語爭刺,初唯唯,後益不從,皆引去。元和中,丁父喪未官,從史即獻計誅王承宗,陰向帝旨,繇是奪服,復領澤、潞。因詔討賊,而勒兵逗留,陰與承宗交,得其密號授軍中,又高芻粟直以售度支。既上書求兼宰相,且誣諸軍與賊通,兵未可進。憲宗患之。



 初,神策中尉吐突承璀與對壘,從史時過其營飲博,承璀多出寶帶、奇玩誇之。從史資沓猥,所玩悅必遺焉。從史喜,益狎不疑。帝用裴垍謀,敕承璀圖之。承璀伏壯士幕下,伺其來與語,士突起捽持出帳後,縛內車中。從者驚亂,斬數十人,諭以密詔,而大將烏重胤素忠果,部勒其眾,乃定。會夜,疾驅,未明出境,道路無知者。於是五年夏四月,有詔慰其軍,疏從史罪惡,貶驩州司馬,賜死。子繼宗等並徙嶺南。



 高霞寓,幽州範陽人。其先五代不異居,孝聞裏閭。德宗初,採訪使洪經綸言之,詔表闕於門。霞寓能讀《春秋》及兵法,頗以感概自尚,狡譎多變。往見長武城使高崇文,崇文異其才,檄任軍職。從擊劉闢,戰輒克,下鹿頭城,降李文悅、仇良輔等,追戰七盤城有功,禽闢於羊灌。擢拜彭州刺史。俄代崇文為長武城使,封感義郡王。



 元和中,以左威衛將軍隨吐突承璀討王承宗,諸將多覆軍,獨霞寓有功,詔藏所獲鎧仗於神策庫以旌之。承璀已執盧從史,其軍相驚,乃遣霞寓諭之,麾而大呼曰:「元惡縛矣,公等宜自安!」即脫鎧揖而前,眾遂定,欲留為帥,霞寓間道去。拜豐州刺史、三城都團練防禦使。



 討吳元濟也,析山南東道為兩鎮,以霞寓宿將,拜唐鄧隋節度使,遏賊南沖。霞寓雖悍,而寡謀,統制尤非所善,始引兵趨蕭陂,戰小勝,進至文城柵,賊偽北,逐之,為伏所掩,遂大敗,才以身免。詒貶歸州刺史。乃厚賂權宦,召為右衛大將軍,拜振武節度使。會吐蕃攻鹽、豐二州,霞寓以兵五千屯拂雲堆,虜引去。浚金河,溉鹵地數千頃。改左武衛大將軍,又節度邠寧,位檢校司徒。寶歷中,疽發首,不能事,以右金吾衛大將軍召,卒於道,贈太保。



 霞寓位既高,言多不遜,帝欲罷其兵,益自憂,乃上私第為佛祠,請署曰「懷恩」,以塞帝疑。俄又詺侮僚屬,作慢語斥訕大臣,其反覆自任類此。



\end{pinyinscope}