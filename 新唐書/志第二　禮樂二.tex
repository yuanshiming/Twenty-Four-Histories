\article{志第二 禮樂二}

\begin{pinyinscope}

 六曰進熟。皇帝既升,奠玉、幣。太官令帥進饌者奉饌,各陳於內壝門外。謁者引司徒出,詣饌所。司徒奉昊天上帝之俎,太官令引饌入門4卷在恩格斯逝世後由考茨基整理成3冊,分別於1904、1905,各至其陛。祝史俱進,跪,徹毛血之豆,降自東陛以出。諸太祝迎饌於壇上。司徒、太官令俱降自東陛以出。又進設外官、眾星之饌。皇帝詣罍洗,盥手,洗爵,升壇自南陛。司徒升自東陛,立於尊所。齋郎奉俎從升,立於司徒後。皇帝詣上帝尊所,執尊者舉冪,侍中贊酌泛齊,進昊天上帝前,北向跪,奠爵;興,少退,立。太祝持版進於神右,東向跪,讀祝文曰:「維某年歲次月朔日,嗣天子臣某,敢昭告於昊天上帝。」皇帝再拜。詣配帝酒尊所,執尊者舉冪,侍中取爵於坫以進,皇帝受爵,侍中贊酌泛齊,進高祖神堯皇帝前,東向跪,奠;興,少退,立。太祝持版進於左,北向跪,讀祝文曰:「維某年歲次月朔日,曾孫開元神武皇帝臣某,敢昭告於高祖神堯皇帝。」皇帝再拜。進昊天上帝前,北向立。太祝各以爵酌上尊福酒,合置一爵,太祝持爵授侍中以進,皇帝再拜,受爵,跪,祭酒,啐酒,奠爵,俯伏,興。太祝各帥齋郎進俎。太祝減神前胙肉,共置一俎,授司徒以進,皇帝受以授左右。皇帝跪,取爵,遂飲,卒爵。侍中進,受虛爵,復於坫。皇帝俯伏,興,再拜,降自南陛,復於位。文舞出,武舞入。初,皇帝將復位,謁者引太尉詣罍洗,盥手,洗瓠爵,自東陛升壇,詣昊天上帝著尊所,執尊者舉冪,太尉酌醴齊,進昊天上帝前,北向跪,奠爵;興,再拜。詣配帝犧尊所,取爵於坫,酌醴齊,進高祖神堯皇帝前,東向跪,奠爵;興,再拜。進昊天上帝前,北向立。諸太祝各以爵酌福酒,合置一爵,進於右,西向立。太尉再拜,受爵,跪,祭酒,遂飲,卒爵。太祝進,受虛爵,復於坫。太尉再拜,降,復位。初,太尉獻將畢,謁者引光祿卿詣罍洗,盥手,洗瓠爵,升,酌盎齊。終獻如亞獻。太尉將升獻,謁者七人分引五方帝及大明、夜明等獻官,詣罍洗,盥手,洗瓠爵,各由其陛升,酌泛齊,進,跪奠於神前。初,第一等獻官將升,謁者五人次引獻官各詣罍洗,盥、洗,各由其陛升壇,詣第二等內官酒尊所,酌醍齊以獻。贊者四人次引獻官詣罍洗,盥、洗,詣外官酒尊所,酌清酒以獻。贊者四人,次引獻官詣罍洗,盥、洗,詣眾星酒尊所,酌昔酒以獻。其祝史、齋郎酌酒助奠,皆如內官。上下諸祝各進,跪,徹豆,還尊所。奉禮郎曰:「賜胙。」贊者曰:「眾官再拜。」在位者皆再拜。大常卿前奏:「請再拜。」皇帝再拜。奉禮郎曰:「眾官再拜。」在位者皆再拜。樂作一成。太常卿前奏:「請就望燎位。」皇帝就位,南向立。上下諸祝各執篚,取玉、幣、祝版、禮物以上。齋郎以俎載牲體、稷、黍飯及爵酒,各由其陛降壇,詣柴壇,自南陛登,以幣、祝版、饌物置於柴上。戶內諸祝又以內官以下禮幣皆從燎。奉禮郎曰:「可燎。」東、西面各六人,以炬燎火。半柴,太常卿前曰:「禮畢。」皇帝還大次,出中壝門,殿中監前受鎮珪,以授尚衣奉御,殿中監又前受大珪。皇帝入次,謁者、贊引各引祀官,通事舍人分引從禮群官、諸方客使以次出。贊者引御史、太祝以下俱復執事位。奉禮郎曰:「再拜。」御史以下皆再拜,出。工人、二舞以次出。



 若宗廟,曰饋食。皇帝既升,祼,太官令出,帥進饌者奉饌,陳於東門之外,西向南上。謁者引司徒出,詣饌所,司徒奉獻祖之俎。太官引饌入自正門,至於太階。祝史俱進,徹毛血之豆,降自阼階以出。諸太祝迎饌於階上設之,乃取蕭、稷、黍擩於脂,燔於爐。太常卿引皇帝詣罍洗,盥手,洗爵,升自阼階,詣獻祖尊彞所,執尊者舉冪,侍中贊酌泛齊,進獻祖前,北向跪,奠爵。又詣尊所,侍中取爵於坫以進,酌泛齊,進神前,北向跪,奠爵,退立。太祝持版進於神右,東面跪,讀祝文曰:「維某年歲次月朔日,孝曾孫開元神武皇帝某,敢昭告於獻祖宣皇帝、祖妣宣莊皇后張氏。」皇帝再拜,又再拜。奠,詣懿祖尊彞,酌泛齊,進神前,南向跪,奠爵,少西,俯伏,興。又醉泛齊,進神前,南向跪,奠爵,少東,退立。祝史西面跪,讀祝文。皇帝再拜,又再拜。次奠太祖、代祖、高祖、太宗、高宗、中宗、睿宗,皆如懿祖。乃詣東序,西向立。司徒升自阼階,立於前楹間,北面東上。諸太祝各以爵酌上尊福酒,合置一爵,太祝持爵授侍中以進。皇帝再拜,受爵,跪,祭酒、啐酒,奠爵,俯伏,興。諸太祝各帥齋郎進俎,太祝減神前三牲胙肉,共置一俎上,以黍、稷飯共置一籩,授司徒以進;太祝又以胙肉授司徒以進。皇帝每受,以授左右,乃跪取爵,飲,卒爵。侍中進,受虛爵,以授太祝,復於坫。皇帝降自阼階,復於版位。文舞出,武舞入。初,皇帝將復位,太尉詣罍洗,盥手,洗爵,升自阼階,詣獻祖尊彞所,酌醴齊進神前,北向跪,奠爵,少東,興,再拜。又取爵於坫,酌醴齊進神前,北向跪,奠爵,少西,北向再拜。次奠懿祖、太祖、代祖、高祖、太宗、高宗、中宗、睿宗如獻祖。乃詣東序,西向立。諸太祝各以爵酌福酒,合置一爵,太祝持爵進於左,北向立。太尉再拜受爵,跪,祭酒,遂飲,卒爵。太祝進,受爵,復於坫。太尉興,再拜,復於位。初,太尉獻將畢,謁者引光祿卿詣罍洗,盥、洗,升,酌盎齊。終獻如亞獻。諸太祝各進。徹豆,還尊所。奉禮郎曰:「賜胙。」贊者曰:「眾官再拜。」在位者皆再拜。太常卿前奏:「請再拜。」皇帝再拜。奉禮郎曰:「眾官再拜。」在位者皆再拜。樂一成止。太常卿前曰:「禮畢。」皇帝出門,殿中監前受鎮珪。通事舍人、謁者、贊引各引享官、九廟子孫及從享群官、諸方客使以次出。贊引引御史、太祝以下俱復執事位。奉禮郎曰:「再拜。」御史以下皆再拜以出。工人、二舞以次出。太廟令與太祝、宮闈令帥腰輿升,納神主。其祝版燔於齋坊。



 七祀,各因其時享:司命、戶以春,灶以夏,中霤以季夏土王之日,門、厲以秋,行以冬。



 時享之日,太廟令布神席於廟庭西門之內道南,東向北上;設酒尊於東南,罍洗又於東南。太廟令、良愬令實尊篚,太官丞引饌,光祿卿升,終獻,獻官乃即事,一獻而止。其配享功臣,各位於其廟室太階之東,少南,西向,以北為上。壺尊二於座左,設洗於終獻洗東南,北向。以太官令奉饌,廟享已亞獻,然後獻官即事,而助奠者分奠,一獻而止。



 此冬至祀昊天上帝於圓丘、孟冬祫於太廟之禮,在乎壇壝、宗廟之間,禮盛而物備者莫過乎此也。其壇堂之上下、壝門之內外、次位之尊卑與其向立之方、出入降登之節,大抵可推而見,其盛且備者如此,則其小且略者又可推而知也。



 至於壇臽、神位、尊爵、玉幣、籩豆、簋簠、牲牢、冊祝之數皆略依古。



 四成,而成高八尺一寸,下成廣二十丈,而五減之,至於五丈,而十有二陛者,圓丘也。八觚三成,成高四尺,上廣十有六步,設八陛,上陛廣八尺,中陛一丈,下陛丈有二尺者,方丘也。高、廣皆四丈者,神州之壇也。其廣皆四丈,而高八尺者青帝、七尺者赤帝、五尺者黃帝、九尺者白帝、六尺者黑帝之壇也。廣四丈,高八尺者,朝日之壇也。為坎深三尺,縱廣四丈,壇於其中,高一尺,方廣四丈者,夕月之壇也。廣五丈,以五土為之者,社稷之壇也。高尺,廣丈,蠟壇也。高五尺,周四十步者,先農、先蠶之壇也。其高皆三尺,廣皆丈者,小祀之壇也。岳鎮、海瀆祭於其廟,無廟則為之壇於坎,廣一丈,四向為陛者,海瀆之壇也。廣二丈五尺,高三尺,四出陛者,古帝王之壇也。廣一丈,高一丈二尺,戶方六尺者,大祀之燎壇也。廣八尺,高一丈,戶方三尺者,中祀之燎壇也。廣五尺,戶方二尺者,小祀之燎壇也。皆開上南出。瘞坎皆在內壝之外壬地,南出陛,方,深足容物。此壇臽之制也。



 冬至祀昊天上帝於圓丘,以高祖神堯皇帝配。東方青帝靈威仰、南方赤帝赤熛怒、中央黃帝含樞紐、西方白帝白招拒、北方黑帝汁光紀及大明、夜明在壇之第一等。天皇大帝、北辰、北斗、天一、太一、紫微五帝座,並差在行位前。餘內官諸坐及五星、十二辰、河漢四十九坐,在第二等十有二陛之間。中官、市垣、帝座、七公、日星、帝席、大角、攝提、太微、五帝、太子、明堂、軒轅、三臺、五車、諸王、月星、織女、建星、天紀十七座及二十八宿,差在前列。其餘中官一百四十二座皆在第三等十二陛之間。外官一百五在內壝之內,眾星三百六十在內壝之外。正月上辛祈穀,祀昊天上帝,以高祖神堯皇帝配,五帝在四方之陛。孟夏雩,祀昊天上帝,以太宗文武聖皇帝配,五方帝在第一等,五帝在第二等,五官在壇下之東南。季秋祀昊天上帝,以睿宗大聖真皇帝配,五方帝在五室,五帝各在其左,五官在庭,各依其方。立春祀青帝,以太皞氏配,歲星、三辰在壇下之東北,七宿在西北,句芒在東南。立夏祀赤帝,以神農氏配,熒惑、三辰、七宿、祝融氏位如青帝。季夏土王之日祀黃帝,以軒轅氏配,鎮星、后土氏之位如赤帝。立秋祀白帝,以少昊氏配,太白、三辰、七宿、蓐收之位如赤帝。立冬祀黑帝,以顓頊氏配,辰星、三辰、七宿、玄冥氏之位如白帝。蠟祭百神,大明、夜明在壇上,神農、伊耆各在其壇上,後稷在壇東,五官、田畯各在其方,五星、十二次、二十八宿、五方之岳鎮、海瀆、山林、川澤、丘陵、墳衍、原隰、井泉各在其方之壇,龍、麟、硃鳥、騶虞、玄武、鱗、羽、裸、毛、介、水墉、坊、郵表叕、於菟、貓各在其方壇之後。夏至祭皇地祇,以高祖配,五方之岳鎮、海瀆、原隰、丘陵、墳衍在內壝之內,各居其方,而中岳以下在西南。孟冬祭神州地祇,以太宗配。社以後土,稷以後稷配。吉亥祭神農,以後稷配,而朝日、夕月無配。席,尊者以槁秸,卑者以莞。此神位之序也。



 以大尊實泛齊,著尊實醴齊,犧尊實盎齊,山罍實酒,皆二;以象尊實醍齊,壺尊實沈齊,皆二;山罍實酒四:以祀昊天上帝、皇地祇、神州地祇。以著尊實泛齊,牲尊實醴齊,象尊實盎齊,山罍實酒,皆二,以祀配帝。以著尊二實醴齊,以祀內官。以犧尊二實盎齊,以祀中官。以象尊二實醍齊,以祀外官。以壺尊二實昔酒,以祀眾星、日、月。以上皆有坫。迎氣,五方帝、五人帝以六尊,惟山罍皆減上帝之半。五方帝大享於明堂,太尊、著尊、牛羲尊、山罍各二。五方帝從祀於圓丘,以太尊實泛齊,皆二。五人帝從享於明堂,以著尊實醴齊,皆二。日、月,以太尊實醴齊,著尊實盎齊,皆二,以山罍實酒一。從祀於圓丘,以太尊二實泛齊。神州地祇從祀於方丘,以太尊二實泛齊。五官、五星、三辰、后稷,以象尊實醍齊;七宿,以壺尊實沈齊,皆二。蠟祭,神農、伊耆氏,以著尊皆二實盎齊。田畯、龍、麟、硃鳥、騶虞、玄武,以壺尊實沈齊。鱗、羽、裸、毛、介、丘陵、墳衍、原隰、井泉、水墉、坊、郵表叕、虎、貓、昆蟲、以散尊實清酒,皆二。岳鎮、海瀆,以山尊實醍齊。山、川、林、澤,以蜃尊實沈齊,皆二。伊耆氏以上皆有坫。太社,以太罍實醍齊,著尊實盎齊,皆二;山罍一。太稷,後稷氏亦如之。其餘中祀,皆以犧尊實醍齊,象尊實盎齊,山罍實酒,皆二,小祀,皆以象尊二實醍齊。宗廟祫享,室以斝彞實明水,黃彞實鬯,皆一;犧尊實泛齊,象尊實醴齊,著尊實盎齊,山罍實酒,皆二,設堂上。壺尊實醍齊,大尊實沈齊,山罍實酒,皆二,設堂下。禘享,雞彞、鳥彞一。時享,春、夏室以雞彞、鳥彞一,秋、冬以斝彞、黃彞一,皆有坫。七祀及功臣配享,以壺尊二實醍齊。別廟之享,春、夏以雞彞實明水,鳥彞實鬯,皆一;牲尊實醴齊,象尊實盎齊,山罍實酒,皆二。秋、冬以斝彞、黃彞,皆一;著尊、壺尊、山罍皆二。太子之廟,以犧尊實醴齊,象尊實盎齊,山罍實酒,皆二。凡祀,五齊之上尊,必皆實明水;山罍之上尊,必皆實明酒;小祀之上尊,亦實明水。此尊爵之數也。



 冬至,祀昊天上帝以蒼璧。上辛,明堂以四圭有邸,與配帝之幣皆以蒼,內官以下幣如方色。皇地祇以黃琮,與配帝之幣皆以黃。青帝以青圭,亦帝以赤璋,黃帝以黃琮,白帝以白琥,黑帝以黑璜;幣如其玉。日以圭、璧,幣以青;月以圭、璧,幣以白。神州、社、稷以兩圭有邸,幣以黑;岳鎮、海瀆以兩圭有邸,幣如其方色。神農之幣以赤,伊耆以黑,五星以方色,先農之幣以青,先蠶之幣以黑,配坐皆如之。它祀幣皆以白,其長丈八尺。此玉、幣之制也。



 冬至祀圓丘,昊天上帝、配帝,籩十二、豆十二、簋一、簠一、一、俎一。五方上帝、大明、夜明,籩八、豆八、簋一、簠一、一、俎一。五星、十二辰、河漢及內官、中官,籩二、豆二、簋一、簠一、俎一。外官眾星,籩、豆、簋、簠、俎各一。正月上辛,祈穀圓丘,昊天、配帝、五方帝,如冬至。孟夏雩祀圓丘,昊天、配帝、五方帝,如冬至。五人帝,籩四、豆四、簋一、簠一、俎一。五官,籩二、豆二、簋一、簠一、俎一。季秋大享明堂,如雩祀。立春祀青帝及太昊氏,籩豆皆十二、簋一、簠一、一、俎一。歲星、三辰、句芒、七宿,籩二、豆二、簋一、簠一、俎一。其赤帝、黃帝、白帝、黑帝皆如之。示昔祭百神,大明、夜明,籩十、豆十、簋一、簠一、一、俎一。神農、伊耆,籩、豆各四,簋、簠、、俎各一。五星、十二辰、后稷、五方田畯、岳鎮、海瀆、二十八宿、五方山林川澤,籩、豆各二,簋、簠、俎各一。丘陵、填衍、原隰、龍、麟、硃鳥、白虎、玄武、鱗、羽、毛、介、於菟等,籩、豆各一,簋、簠、俎各一。又井泉,籩、豆各一,簋、簠、俎各一。春分朝日,秋分夕月,籩十、豆十、簋一、簠一、一、俎一。四時祭風師、雨師、靈星、司中、司命、司人、司祿,籩八、豆八、簋一、簠一、俎一。夏至祭方丘,皇地示氏及配帝,豆皆十二、簋一、簠一、一、俎一。神州,籩四、豆四、簋一、簠一、一、俎一。其五岳、四鎮、四海、四瀆及五方山川林澤,籩二、豆二,簋、簠、俎各一。孟冬祭神州及配帝,籩豆皆十二、簋一、一、一、俎一。春、秋祭太社、太稷及配坐,籩豆皆十、簋二、簠二、鈃三、俎三。四時祭馬祖、馬社、先牧、馬步,籩豆皆八、簋一、簠一、俎一。時享太廟,每室籩豆皆十二、簋二、簠二、三、鈃三、俎三。七祀,籩二、豆二、簋二、簠二、俎一。祫享、功臣配享,如七祀。孟春祭帝社及配坐,籩豆皆十、簋二、簋二、三、鈃三、俎三。季春祭先蠶,籩豆皆十、簋二、簠二、三、鈃三、俎三。孟冬祭司寒,籩豆皆八、簋一、簠一、俎一。春、秋釋奠於孔宣父,先聖、先師,籩十、豆十、簋二、簠二、三、鈃三、俎三;若從祀,籩豆皆二、簋一、簠一、俎一。春、秋釋奠於齊太公、留侯,籩豆皆十、簋二、簠二、三、鈃三、俎三、仲春祭五龍,籩豆皆八、簋一、簠一、俎一。四時祭五岳、四鎮、四海、四瀆,各籩豆十、簋二、簠二、俎三。三年祭先代帝王及配坐,籩豆皆十、簋二、簠二、俎三。州縣祭社、稷、先聖,釋奠於先師,籩豆皆八、簋二、簠二、俎三。籩以石鹽、槁魚、棗、慄、榛、菱芡之實、鹿脯、白餅、黑餅、糗餌、粉騑。豆以菲菹盆棨、菁菹鹿棨、芹菹兔棨、芹菹魚棨、脾析菹豚胉。嵒食、糝食。中祀之籩無糗餌、粉棨,豆無嵒食、糝食。小祀之籩無白餅、黑餅、豆無脾析菹豚胉。凡用皆四者,籩以石鹽、棗實、慄黃、鹿脯;豆以芹菹兔棨、菁菹鹿棨。用皆二者,籩以慄黃、牛脯。豆以葵菹鹿棨。用皆一者,籩以牛脯,豆以鹿。用牛脯者,通以羊。凡簠、簋皆一者,簋以稷,簠以黍。用皆二者,簋以黍、稷,簠以稻、粱。實以大羹,鈃以肉羹。此籩、豆、簠、簋、、鈃之實也。



 昊天上帝,蒼犢;五方帝,方色犢;大明,青犢;夜明,白犢;神州地祇黑犢。配帝之犢:天以蒼,地以黃,神州以黑,皆一。宗廟、太社、太稷、帝社、先蠶、古帝王、岳鎮、海瀆,皆太牢;社、稷之牲以黑;五官、五星、三辰、七宿,皆少牢。蠟祭:神農氏、伊耆氏,少牢;後稷及五方、十二次、五官、五田畯、五岳、四鎮、海瀆、日、月,方以犢二;星辰以降,方皆少牢五;井泉皆羊一。非順成之方則闕。風師、雨師、靈星、司中、司命、司人、司祿、馬祖、先牧、馬社、馬步,皆羊一。司寒,黑牲一。凡牲在滌,大祀九旬,中祀三旬,小祀一旬,養而不卜。無方色則用純,必有副焉。省牲而犢鳴,則免之而用副。禁其棰柎,死則瘞之,創病者請代犢,告祈之牲不養。凡祀,皆以其日未明十五刻,太官令帥宰人以鸞刀割牲,祝史以豆斂毛血置饌所,祭則奉之以入,遂亨之。肉載以俎,皆升右胖體十一:前節三,肩、臂、臑;後節二,肫、胳;正脊一,脡泚一,橫脊一,正脅一,短脅一,代脅一,皆並骨。別祭用太牢者,酒二斗,脯一段,棨四合;用少牢者,酒減半。此牲牢之別也。



 祝版,其長一尺一分,廣八寸,厚二分,其木梓、楸。凡大祀、中祀,署版必拜。皇帝親祠,至大次,郊社令以祝版進署,受以出,奠於坫。宗廟則太廟令進之。若有司攝事,則進而御署,皇帝北向再拜,侍臣奉版,郊社令受以出。皇后親祠,則郊社令預送內侍,享前一日進署,後北向再拜,近侍奉以出,授內侍送享所。享日之平明,女祝奠於坫。此冊祝之制也。



\end{pinyinscope}