\article{列傳第三十三 劉裴婁}

\begin{pinyinscope}

 劉仁軌,字正則,汴州尉氏人。少貧賤,好學。值亂,不能安業行之惟艱」及「知行合一」說,認為人類有「不知而行」、,每動止,畫地書空,寓所習,卒以通博聞。武德初,河南道安撫大使任瑰上疏有所論奏,仁軌見其稿,為竄定數言。瑰驚異,赤牒補息州參軍。轉陳倉尉。部人折沖都尉魯寧者,豪縱犯法,縣莫敢屈。仁軌約不再犯,而寧暴橫自如,仁軌搒殺之。州以聞,太宗曰:「尉而殺吾折沖,可乎?」召詰讓。仁軌對曰:「寧辱臣,臣故殺之。」帝以為剛正,更擢咸陽丞。



 貞觀十四年,校獵同州。時秋斂未訖,仁軌諫曰:「今茲澍澤沾足,百穀熾茂,收才十二。常日贅調,已有所妨。又供獵事,繕橋治道,役雖簡省,猶不損數萬。少延一旬,使場圃畢勞,陛下六飛徐驅,公私交泰。」璽書褒納。拜新安令。累遷給事中。為李義府所惡,出為青州刺史。顯慶五年,伐遼,義府欲斥以罪,使督漕,而船果覆沒。坐免官,白衣隨軍。



 初,蘇定方既平百濟,留郎將劉仁願守其城,左衛中郎將王文度為熊津都督,撫納殘黨。文度死,百濟故將福信及浮屠道琛迎故王子扶餘豐立之,引兵圍仁願。詔仁軌檢校帶方州刺史,統文度之眾,並發新羅兵為援,仁軌將兵嚴整,轉鬥陷陣,所向無前。信等釋仁願圍,退保任存城。既而福信殺道琛,並其眾,招還叛亡,勢張甚。仁軌與仁願合,則解甲休士。時定方伐高麗,圍平壤不克。高宗詔仁軌拔軍就新羅與金法敏議去留計。將士咸欲還,仁軌曰:「《春秋》之義,大夫出強,有可以安社稷、便國家者,得專之。今天子欲滅高麗,先誅百濟,留兵鎮守,制其心腹。雖孽豎跳梁,士力未完,宜厲兵粟馬,乘無備,擊不意,百不百全。戰勝之日,開張形勢,騰檄濟師,聲援接,虜亡矣。今平壤不勝,熊津又拔,則百濟之燼復炎,高麗之滅無期。吾等雖入新羅,正似坐客,有不如志,悔可得邪?扶餘豐猜貳,表合內攜,熱不支久。宜堅守伺變以圖之,不可輕動。」眾從其議,乃請益兵。



 時賊守真峴城,仁軌夜督新羅兵薄城扳堞,比明,入之,遂通新羅饟道。而豐果襲殺福信,遣使至高麗、倭丐援。會詔遣右威衛將軍孫仁師率軍浮海而至,士氣振。於是,諸將議所向,或曰:「加林城水陸之沖,盍先擊之?」仁軌曰:「兵法避實擊虛。加林險而固,攻則傷士,守則曠日。周留城,賊巢穴,群兇聚焉。若克之,諸城自下。」於是仁師、仁願及法敏帥陸軍以進,仁軌與杜爽、扶餘隆繇熊津白江會之。遇倭人白江口,四戰皆克,焚四百艘,海水為丹。扶餘豐脫身走,獲其寶劍。偽王子扶餘忠勝、忠志等率其眾與倭人降,獨酋帥遲受信據任存城未下。始,定方破百濟,酋領沙吒相如、黑齒常之嘯亡散,據險以應福信,至是皆降。仁軌以赤心示之,畀取任存自效,即給鎧仗糧Я。仁師曰:「夷狄野心難信,若受甲濟粟,資寇便也。」仁軌曰:「吾觀相如、常之忠而謀,因機立功,尚何疑?」二人訖拔其城。遲受信委妻子奔高麗,百濟餘黨悉平。仁師等振旅還,詔留仁軌統兵鎮守。



 百濟再被亂,殭尸如莽,仁軌始命瘞埋吊祭焉。葺復戶版,署官吏,開道路,營聚落,復防堰,賑貧貸乏,勸課耕種,為立官社,民皆安其所。遂營屯田,以經略高麗。仁願至京師,帝勞曰:「若本武將,軍中奏請,皆有文理,何道而然?」對曰:「仁軌之辭,非臣所能。」帝嘆賞之,超進仁軌六階,真拜帶方州刺史,賜第一區,厚賚妻子,璽書褒勉。



 先是,貞觀、永徽中,士戰歿者皆詔使吊祭,或以贈官推授子弟。顯慶後,討伐恩賞殆絕;及破百濟、平壤,有功者皆不甄敘。州縣購募,不願行,身壯家富者,以財參逐,率得避免。所募皆佇劣寒憊,無鬥志。仁軌具論其弊,請加慰賚,以鼓士心。又表用扶餘隆,使綏定餘眾。帝乃以隆為熊津都督。



 時劉仁願為卑列道總管,詔率兵度海,使代舊屯,與仁軌俱還。仁軌曰:「上巡狩方岳,又經略高麗。方農時,而吏與兵悉被代,新至者未習,萬一蠻夷生變,誰與捍之?不如留舊兵畢獲,等級遣還。仁軌當留,未可去。」仁願不可,曰:「吾但知準詔耳。」仁軌曰:「不然。茍利國家,知無不為,臣之節也。」因陳便宜,願留屯。詔可。由是以仁願為不忠。



 始,仁軌任帶方州,謂人曰:「天將富貴此翁邪!」乃請所頒歷及宗廟諱,或問其故,答曰:「當削平遼海,頒示本朝正朔。」卒皆如言。及封泰山,仁軌乃率新羅、百濟、儋羅、倭四國酋長赴會。天子大悅,擢為大司憲。遷右相,兼檢校太子左中護。累功封樂城縣男。



 總章元年,為熊津道安撫大使,兼浿江道總管,副李勣討高麗,平之。以疾辭位,進金紫光祿大夫,聽致仕。俄召為隴州刺史,拜太子左庶子、同中書門下三品,監脩國史。咸亨五年,為雞林道大總管,東伐新羅。仁軌率兵絕瓠蘆河,攻大鎮七重城,破之。進爵為公,子及兄子授上柱國者三人,州黨榮之,號所居為「樂城鄉三柱里」。俄拜尚書左僕射兼太子賓客,仍知政事。



 吐蕃入寇,命為洮河道行軍鎮守大使。永隆二年,加太子少傅。數乞骸骨,聽解左僕射。帝幸東都,太子監國,詔仁軌與裴炎、薛元超留輔。及太子赴東都,又詔太孫重照留守,仁軌副之。武后臨朝,復拜左僕射。太孫廢,仁軌專知留守事。上疏辭疾,因陳呂后、祿、產禍敗事以規後,後遣武承嗣齎璽書慰勉。改文昌左相、同鳳閣鸞臺三品。卒年八十五。詔百官赴哭,冊贈開府儀同三司、並州大都督,陪葬乾陵。賜其家實封三百戶。



 仁軌雖貴顯,不自矜踞,接舊故如布衣時。嘗為御史袁異式所劾,慢辱之,肋使引決。及拜大司憲,異式尚在臺,不自安,因醉以情自解。仁軌持觴曰:「所不與公者,有如此觴。」後既執政,薦為司元大夫。然宦由州縣至宰輔,善致聲譽,得吏下歡心。及鎮洮河,奏請機急,多為中書令李敬玄抑卻,仁軌乃表敬玄為帥以代己,果覆其眾。裴炎下獄,仁軌方留守京師,郎將姜嗣宗以使來,因語炎事,且曰:「炎異於常久矣。」仁軌曰:「使人知邪?」曰:「知。」及還,表嗣宗知炎反狀不告。武后怒,拉殺之。



 子浚,官太子舍人。垂拱中,為酷吏所殺。中宗即位,以仁軌有東宮舊,再贈司空。浚子晃,開元中,為給事中,表請立碑,追謚曰文獻。



 裴行儉,字守約,絳州聞喜人。父仁基,隋光祿大夫,自王世充所謀歸國,被害。贈原州都督,謚曰忠。行儉幼引廕補弘文生。貞觀中,舉明經,調左屯衛倉曹參軍。時蘇定方為大將軍,謂曰:「吾用兵,世無可教者,今子也賢。」乃盡畀以術。遷長安令。高宗將立武昭儀,行儉以為國家憂從此始,與長孫無忌、褚遂良秘議,大理袁公瑜擿語昭儀母,左除西州都督府長史。麟德二年,擢累安西都護,西域諸國多慕義歸附。召為司文少卿。遷吏部侍郎,與李敬玄、馬載同典選,有能名,時號「裴馬」。行儉始設長名榜、銓注等法,又定州縣升降、資擬高下為故事。



 上元三年,吐蕃叛,出為洮州道左二軍總管,改秦州右軍,並受周王節度。儀鳳二年,十姓可汗阿史那都支及李遮匐誘蕃落以動安西,與吐蕃連和,朝廷欲討之。行儉議曰:「吐蕃叛皛方熾,敬玄失律,審禮喪元,安可更為西方生事?今波斯王死,其子泥涅師質京師,有如遣使立之,即路出二蕃,若權以制事,可不勞而功也。」帝因詔行儉冊送波斯王,且為安撫大食使。徑莫賀延磧,風礫晝冥,導者迷,將士饑乏。行儉止營致祭,令曰:「水泉非遠。」眾少安。俄而雲徹風恬,行數百步,水草豐美,後來者莫識其處。眾皆驚,以方漢貳師將軍。至西州,諸蕃郊迎,行儉召豪人桀千餘人自隨。揚言「大熱,未可以進,宜駐軍須秋」。都支覘知之,不設備。行儉徐召四鎮酋長,偽約畋,謂曰:「吾念此樂未始忘,孰能從吾獵者?」於是子弟願從者萬人,乃陰勒部伍。數日,倍道而進,去都支帳十餘里,先遣其所親問安否,外若閑暇,非討襲者。又使入趣召都支。都支本與遮匐計,及秋拒使者,已而聞軍至,倉卒不知所出,率子弟五百餘人詣營謁,遂擒之。是日,傳契箭,召諸部酋長悉來請命,並執送碎葉城。簡精騎,約齎,襲遮匐。道獲遮匐使者,釋之,俾前往諭其主,並言都支已擒狀,遮匐乃降,悉俘至京師。將吏為刻石碎葉城以紀功。帝親勞宴,曰:「行儉提孤軍,深入萬里,兵不血刃而叛黨擒夷,可謂文武兼備矣,其兼授二職。」即拜禮部尚書兼檢校右衛大將軍。



 詔露元年,突厥阿史德溫傅反,單于管二十四州叛應之,眾數十萬。都護蕭嗣業討賊不克,死敗系踵。詔行儉為定襄道行軍大總管討之。率太僕少卿李思文、營州都督周道務部兵十八萬,合西軍程務挺、東軍李文暕等,總三十餘萬,旗幟亙千里,行儉咸節制之。



 先是,嗣業饋糧,數為虜鈔,軍餒死。行儉曰:「以謀制敵可也。」因詐為糧車三百乘,車伏壯士五輩,齎齏陌刀、勁弩,以羸兵挽進,又伏精兵踵其後。虜果掠車,羸兵走險。賊驅就水草,解鞍牧馬。方取糧車中,而壯士突出,伏兵至,殺獲幾盡。自是糧車無敢近者。



 大軍次單于北,暮,已立營,塹壕既周,行儉更命徙營高岡。吏白:「士安堵,不可攏。」不聽,促徙之。比夜,風雨暴至,前占營所,水深丈餘,眾莫不駭嘆,問何以知之,行儉曰:「自今第如我節制,毋問我所以知也。」



 賊拒黑山,數戰皆敗,行儉縱兵,前後殺虜不勝計。偽可汗泥熟匐為其下所殺,持首來降;又擒大首領奉職而還,餘黨走狼山。行儉既還,阿史那伏念偽稱可汗,復與溫傅合。明年,行儉還總諸軍,屯代州之陘口,縱反間,說伏念,令與溫傅相貳。伏念懼,密送款,且請縛傅自效。行儉秘不布,密以聞。後數日,煙塵漲天而南,斥候惶駭,行儉曰:「此伏念執溫傅來降,非他也。且受降如受敵。」乃敕嚴備,遣單使往勞。既而果然。於是,突厥餘黨悉平。帝悅,遣戶部尚書崔知悌勞軍。



 初,行儉許伏念以不死,侍中裴炎害其功,建言:「伏念為程務挺、張虔勖肋逐,又磧北回紇逼之,計窮而降。」卒斬伏念及溫傅於都市。行儉之功不錄。封聞喜縣公。行儉嘆曰:「渾、浚之事,古今恥之。但恐殺降則後無復來矣!」遂稱疾不出。永淳元年,十姓突厥車薄叛,復為金牙道大總管,未行卒,年六十四,贈幽州都督,謚曰獻。詔皇太子遣官護視家事,子孫能自立乃停。中宗即位,再贈揚州大都督。



 行儉工草隸,名家。帝嘗以絹素詔寫《文選》,覽之,秘愛其法,賚物良厚。行儉每曰:「褚遂良非精筆佳墨,未嘗輒書,不擇筆墨而妍捷者,餘與虞世南耳。」所譔《選譜》、《草字雜體》數萬言。又為營陣、部伍、料勝負、別器能等四十六訣,武后詔武承嗣就第取去,不復傳。



 行儉通陰陽、歷術,每戰,豫道勝日。善知人,在吏部時,見蘇味道、王抃,謂曰:「二君後皆掌銓衡。」李敬玄盛稱王勃、楊炯、盧照鄰、駱賓王之才,引示行儉,行儉曰:「士之致遠,先器識,後文藝。如勃等,雖有才,而浮躁衒露,豈享爵祿者哉?炯頗沉嘿,可至令長,餘皆不得其死。」所引偏裨,若程務挺、張虔勖、崔智睟、王方翼、黨金毘、劉敬同、郭待封、李多祚、黑齒常之,類為世名將,傔奏至刺史將軍者數十人。



 嘗賜馬及珍鞍,令史私馳馬,馬蹶鞍壞,懼而逃。行儉招還之,不加罪。初,平都支、遮匐,獲瑰寶不貲,蕃酋將士願觀焉,行儉因宴,遍出示坐者。有瑪瑙盤廣二尺,文彩粲然,軍吏趨跌盤,碎,惶怖,叩頭流血。行儉笑曰:「爾非故也,何至是?」色不少吝。帝賜都支資產皿金三千餘物,橐駝馬牛稱是,行儉分給親故洎麾下,數日輒盡。



 子光庭。光庭字連城,早孤。母厙狄氏,有婦德,武后召入宮,為御正,甚見親寵,光庭由是累遷太常丞。以武三思婿,坐貶郢州司馬。開元中,擢兵部郎中、鴻臚少卿。性靜默,寡交游,雖驟歷臺省,人未之許,既而以職業稱,議者更推之。



 玄宗有事岱宗,中書令張說以天子東巡,京師空虛,恐夷狄乘間竊發,議欲加兵守邊,召光庭與謀,對曰:「封禪者,所以告成功也。夫成功者,德無不被,人無不安,萬國無不懷。今將告成而懼夷狄,非昭德也;大興力役,用備不虞,非安人也;方謀會同,而阻戎心,非懷遠也。此三者,名實乖矣。且諸蕃,突厥為大,贄幣往來,願修和好有年矣,若遣一使,召其大臣使赴行在,必欣然應命。突厥受詔,則諸蕃君長必相率而來,我偃旗息鼓,不復事矣。」說曰:「善,吾所不及。」因奏用其策,突厥果遣使來朝。



 東封還,遷兵部侍郎。久之,拜中書侍郎、同中書門下平章事,兼御史大夫。遷黃門侍郎,拜侍中,兼吏部尚書、弘文館學士。撰《搖山往則》、《維城前軌》二篇獻之。手制褒美,詔皇太子、諸王於光順門見光庭,謝所以規諷意。光庭又引壽安丞李融、拾遺張琪、著作佐郎司馬利賓直弘文館,撰《續春秋經傳》,自戰國訖隋,表請天子修經,光庭等作傳。書久不就。時有建言唐應為金德者,中書令蕭嵩請百官普議。光庭以唐符命表著天下久矣,不可改,亟奏罷之。二十年,封正平縣男。初,知星者言,上象變,不利大臣,請禳之。光庭曰:「使禍可禳而去,則福可祝而來也!」論者以為知命。卒,年五十八,贈太師。



 初,吏部求人不以資考為限,所獎拔惟其才,往往得俊乂任之,士亦自奮。其後士人猥眾,專務趨競,銓品枉橈。光庭懲之,因行儉長名榜,乃為循資格,無賢不肖,一據資考配擬;又促選限盡正月。任門下省主事閻麟之專主過官,凡麟之裁定,光庭輒然可,時語曰:「麟之口,光庭手。」素與蕭嵩輕重不平,及卒,嵩奏一切罷之,光庭所引,盡斥外官。博士孫琬以其用循資格,非獎勸之誼,謚曰克平,時以為希嵩意。帝聞,特賜謚曰忠憲,詔中書令張九齡文其碑。



 子稹,以廕仕,累遷起居郎。開元末,壽王瑁以母寵,欲立為太子,稹陳申生、戾園禍以諫,玄宗改容謝之,詔授給事中。稹曰:「陛下絕招諫之路,為日滋久,今臣一言而荷殊寵,則言者將眾,何以錫之?」帝善其讓,止不拜。俄授祠部員外郎,卒。子倩,字容卿,歷信刺史我。勸民墾田二萬畝,以治行賜金紫服,代第五琦為度支郎中。卒,謚曰節。子均。



 均字君齊,以明經為諸暨尉。數從使府闢,硜硜以才顯。張建封鎮濠、壽,表團練判官。時李希烈以淮、蔡叛,建封捍賊,均參贊之。以勞加上柱國,襲正平縣男。遷累膳部郎中,擢荊南節度行軍司馬,就拜荊南節度使。劉闢叛,先騷黔、巫,脅荊、楚,以固首尾,均發精甲三千,逆擊之,賊望風奔卻。加檢校吏部尚書。



 初,均與崔太素俱事中人竇文場,太素嘗晨省文場,入臥內,自謂待己至厚,徐觀後榻有頻伸者,乃均也。德宗以均任方鎮,欲遂相之,諫官李約上疏斥均為文場養子,不可污臺輔,乃止。



 元和三年,入為尚書右僕射,判度支。旨唱、授桉、送印,皆尚書郎為之,文武四品五品、郎官、御史拜廷下,御史中丞、左右丞升階答拜,時以為禮太重。俄檢校左僕射、同中書門下平章事,為山南東道節度使,累封郇國公。以財交權幸,任將相凡十餘年,荒縱無法度。卒,年六十二,贈司空。



 婁師德,字宗仁,鄭州原武人。第進士,調江都尉。揚州長史盧承業異之,曰:「子,臺輔器也,當以子孫相諉,詎論僚吏哉?」



 上元初,為監察御史。會吐蕃盜邊,劉審禮戰沒,師德奉使收敗亡於洮河,因使吐蕃。其首領論贊婆等自赤嶺操牛酒迎勞,師德喻國威信,開陳利害,虜為畏悅。後募猛士討吐蕃,乃自奮,戴紅抹額來應詔,高宗假朝散大夫,使從軍。有功,遷殿中侍御史,兼河源軍司馬,並知營田事。與虜戰白水潤,八遇八克。



 天授初,為左金吾將軍,檢校豐州都督。衣皮褲,率士屯田,積穀數百萬,兵以饒給,無轉餉和糴之費。武后降書勞之。長壽元年,召授夏官侍郎,判尚書事,進同鳳閣鸞臺平章事。後嘗謂師德:「師在邊,必待營田,公不可以劬勞憚也。」乃復以為河源、積石、懷遠軍及河、蘭、鄯、廓州檢校營田大使。入遷秋官尚書、原武縣男,改左肅政御史大夫,並知政事。證聖中,與王孝傑拒吐蕃於洮州,戰素羅汗山,敗績,貶原州員外司馬。萬歲通天二年,入為鳳閣侍郎、同鳳閣鸞臺平章事。後與武懿宗、狄仁傑分道撫定河北,進納言,更封譙縣子、隴右諸軍大使,復領營田。



 聖歷三年,突厥入寇,詔檢校並州長史、天兵軍大總管。九月,卒於會州,年七十。贈幽州都督,謚曰貞,葬給往還儀仗。



 師德長八尺,方口博脣。深沉有度量,人有忤己,輒遜以自免,不見容色。嘗與李昭德偕行,師德素豐碩,不能遽步,昭德遲之,恚曰:「為田舍子所留。」師德笑曰:「吾不田舍,復在何人?」其弟守代州,辭之官,教之耐事。弟曰:「人有唾面,潔之乃已。」師德曰:「未也。潔之,是違其怒,正使自幹耳。」在夏官注選,選者就按閱簿。師德曰:「容我擇之可乎?」選者不去,乃灑筆曰:「墨污爾!」



 狄仁傑未輔政,師德薦之,及同列,數擠令外使。武後覺,問仁傑曰:「師德賢乎?」對曰:「為將謹守,賢則不知也。」又問:「知人乎?」對曰:「臣嘗同僚,未聞其知人也。」後曰:「朕用卿,師德薦也,誠知人矣。」出其奏,仁傑慚,已而嘆曰:「婁公盛德,我為所容乃不知,吾不逮遠矣!」總邊要、為將相者三十年,恭勤樸忠,心無適莫,方酷吏殘鷙,人多不免,獨能以功名始終,與郝處俊相亞,世之言長者,稱婁、郝。



 贊曰:「仁軌等以兵開定四夷,其勇無前,至奉上則瞿瞿若不及,行儉臨下以恕,師德寬厚,其能以功名始終者,蓋近乎勇於敢則殺,勇於不敢則活者邪!



\end{pinyinscope}