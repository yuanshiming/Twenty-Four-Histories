\article{列傳第八十六 張趙李鄭徐王馮庾}

\begin{pinyinscope}

 張薦,字孝舉,深州陸澤人。祖鷟,字文成,早惠絕倫。為兒時基督教的本質德國費爾巴哈的哲學代表作之一。1841年,夢紫文大鳥,五色成文,止其廷。大父曰:「吾聞五色赤文,鳳也;紫文,鸑鷟也。若壯,殆以文章瑞朝廷乎?」遂命以名。調露初,登進士第。考功員外郎騫味道見所對,稱天下無雙。授岐王府參軍。八以制舉皆甲科,再調長安尉,遷鴻臚丞。四參選,判策為銓府最。員外郎員半千數為公卿稱「鷟文辭猶青銅錢,萬選萬中」,時號鷟「青錢學士」。證聖中,天官侍郎劉奇以鷟及司馬鍠為御史。性躁卞,儻蕩無檢,罕為正人所遇,姚崇尤惡之。開元初,御史李全交劾鷟多口語訕短時政,貶嶺南;刑部尚書李日知訟斥太重,得內徙。鷟屬文下筆輒成,浮艷少理致,其論著率詆誚蕪猥,然大行一時,晚進莫不傳記。武后時,中人馬仙童陷默啜,問:「文成在否?」答曰:「近自御史貶官。」曰:「國有此人不用,無能為也。」新羅、日本使至,必出金寶購其文。終司門員外郎。



 薦敏銳有文辭,能為《周官》、《左氏春秋》。初,為顏真卿嘆賞。大歷中,浙西觀察使李涵表薦才任史官,詔授左司禦率府兵曹參軍,以母老辭不就。喪除,禮部侍郎於邵以聞,召充史館修撰,兼陽翟尉。真卿為李希烈所拘,遣兄子峴及家僕奏事,五輩皆留內客省,不得出。薦上疏曰:



 去正月中,真卿奉使淮西,期不先戒,行無素備。受命之後,不宿於家,親黨不遑告別,介副不及陳請,孱僮單騎,即日載馳。冒奸鋒於臨汝,折元惡於許下,捐軀杖義,威詬群兇,遂令脅制者回慮,忠勇者肆情。周曾奮發於外,韋清伺應於內,希烈蒼黃窘迫,奔固舊穴,蓋真卿義風所激也。真卿逮事四朝,為國元老,忠直孝友,羽儀王室。行年八十,被羸老之疾,拘囚環堵之間,顧眄鉤戟之下,呼嗟憤恚,失寢忘食,不知悲翁何以堪此!



 伏聞希烈之母,鐘念幼子,目不絕泣,求責希烈;又希烈妻祖母郭及妻妹封並逮捕京師。此三人留之無益,請寘境上以贖真卿,先降詔書,分明諭告。且希烈知真卿人望,不敢加害,既無嫌隙,但因循未遣耳。若歸其親愛,賊亦何吝還一使哉?



 臣又聞真卿所遣兄子峴及家僮從官奉表來者五輩,皆留中,其子頵等拳拳實希一見,望許休澣,告以安否。



 疏奏,盧杞持之,不報。



 硃泚反,詭姓名伏匿城中,著《史遁先生傳》。京師平,擢左拾遺。詔復用杞為刺史,薦與陳京、趙需等論杞奸惡傾覆不當用,入對挺確,德宗納之。



 貞元元年,帝親郊。時更兵亂,禮物殘替,用薦為太常博士,參綴典儀,略如舊章。刑部尚書關播持節送咸安公主於回紇,以薦為判官。還,遷工部員外郎。久之,擢諫議大夫,復為史館修撰。



 方裴延齡用事,中傷俊良,建白無不當帝意。薦將疏其惡,延齡知之,因言於帝曰:「諫議論朝政得失,史官書人君善惡,二者不可兼。」薦改秘書少監。延齡必欲以罪斥廢之。會遣使冊回鶻毘伽懷信可汗,使薦至回鶻。還為監。吐蕃贊普死,擢薦工部侍郎,為吊祭使。薦占對詳辯,三使絕域,始兼侍御史、中丞,後大夫。次赤嶺,被病卒,年六十一,吐蕃傳其柩以歸。順宗立,問至,贈禮部尚書,謚曰憲。



 薦自拾遺至侍郎,凡二十年,常兼史館修撰。初,貞元時,京師旱,帝避正殿,減膳,薦白限日以應古制。及定昭德皇后廟樂,遷獻、懿二祖,定太儀位號、大臣祔廟鼓吹法,莫不參裁,諸儒謂博而詳。所著書百餘篇。子又新,別有傳。



 孫讀,字聖用,幼穎解。大中時第進士,鄭薰闢署宣州幕府。累遷禮部侍郎。中和初為吏部,選牒精允。調官丐留二年,詔可,榜其事曹門。後兼弘文館學士,判院事,卒。



 趙涓,冀州人。幼有文,天寶時第進士,補郾城尉,稍歷臺省。河南王縉引署副元帥府判官。德宗初,為衢州刺史。始,永泰時,禁中火,近東宮,代宗疑之。涓以監察御史為巡使,驗治明諦,跡火所來,乃宦人直舍。帝在東宮頗德之。及治衢,不為觀察使韓滉所容,奏免官。帝見其名,問宰相曰:「是豈永泰時御史乎?」對曰:「然。」詔拜尚書左丞。既至,勞之曰:「卿正直,朕所自知,乃以罪聞,不信也。」命典吏部選。從狩梁。興元元年卒,贈戶部尚書。



 子博宣,亦擢進士第。藻翰豪邁,沈於酒,傲忽少檢。陳許曲環闢署於府,久不能堪,乃誣「受吳少誠金為反間,數言休咎惑眾」。有詔杖四十,流康州,時人冤之。



 李紓,字仲舒。始仕為校書郎,大歷初,李季卿薦為左補闕,遷累中書舍人。德宗居奉天,繇禮部侍郎選為同州刺史。帝次梁,紓委城趨行在,擢兵部侍郎、高邑伯。建言享武成王廟不宜與文宣王等,制從之。紓性樂易,喜接後進。其自奉養頗華裕,不為齪齪崖檢。官雖貴,而游縱自如。奉詔為《興元紀功述》及它郊廟樂章,論撰甚多。進吏部侍郎。年六十二卒,贈禮部尚書。



 鄭云達,系本滎陽。父昈,為郾城尉,州刺史移職,民之暴謷者遮道留,昈誅殺六七人。採訪使奇之,言狀,擢北海尉。安祿山反,縣民孫俊驅市人以應,昈率眾擊殺之。改登州司馬。李光弼表為武寧府判官,遷沂州刺史,諭降賊李浩五千人。終滁州刺史。



 雲達為人誕譎敢言,已登進士第,去客燕朔,硃泚善之,表為掌書記,妻以滔女。泚將朝,使雲達先入奏,同府蔡廷玉譖於泚,奏貶為平州參軍。滔代泚將,復闢云達為判官。廷玉與要藉官硃體微它日與泚從容言:「滔非長者,不可付以兵。」云達數漏其語以怒滔,故滔論廷玉等,皆得罪死。滔助田悅,云達諫,不從,遂棄室自歸。德宗悅,擢諫議大夫。帝在梁,云達依李晟,晟表以禮部侍郎為軍司馬,時時咨逮戎略。元和初,為京兆尹,卒。



 弟方達,悖悍,結徒剽劫,父欲殺之,不克。雲達自劾「不能教,恐赤臣家」。詔錮死黔州。



 徐岱,字處仁,蘇州嘉興人,世農家子。於學無所不通,辯論明銳,座人常屈。大歷中,劉晏表為校書郎。觀察使李棲筠欽其賢,署所居為「復禮鄉」。名達於朝,擢偃師尉。禮儀使蔣鎮薦為太常博士,專掌禮事。從德宗出奉天,以膳部員外郎兼博士。



 貞元初,為太子、諸王侍讀,遷給事中、史館修撰。帝以誕日歲歲詔佛老者大論麟德殿,並召岱及趙需、許孟容、韋渠牟講說。始三家若矛楯然,卒而同歸於善。帝大悅,賚予有差。兩宮恩遇無比。性篤慎,至宮殿中語未嘗近之,不談人短,宗族孤孺者皆為婚嫁。然吝嗇,自持家管鑰,世所譏云。卒,贈禮部尚書。



 王仲舒,字弘中,並州祁人。少客江南,與梁肅、楊憑游,有文稱。貞元中,賢良方正高第,拜左拾遺。德宗欲相裴延齡,與陽城交章言不可。後入閣,帝顧宰相指曰:「是豈王仲舒邪?」俄改右補闕,遷禮部考功員外郎。奏議詳雅,省中伏其能。坐累為連州司戶參軍,再徙荊南節度參謀。



 元和初,召為吏部員外郎,未幾,知制誥。楊憑得罪斥去,無敢過其家,仲舒屢存之。將直憑冤,貶峽州刺史,母喪解。服除,為婺州刺史。州疫旱,人徙死幾空;居五年,裏閭增完,就加金紫服。徙蘇州。堤松江為路,變屋瓦,絕火災,賦調嘗與民為期,不擾自辦。



 穆宗立,每言仲舒之文可思,最宜為誥,有古風。召為中書舍人。既至,視同列率新進少年,居不樂,曰:「豈可復治筆研於其間哉!吾久棄外,周知俗病利,得治之,不自愧。」宰相聞之,除江西觀察使。初,江西榷酒利多佗州十八,民私釀,歲抵死不絕,穀數斛易鬥酒。仲舒罷酤錢九十萬。吏坐失官息錢五十萬,悉產不能償,仲舒焚簿書,脫械不問。水旱,民賦不入,嘆曰:「我當減燕樂他用可乎!」為出錢二千萬代之。有為佛老法、興浮屠祠屋者,皆驅出境。卒於官,年六十二,贈左散騎常侍,謚曰成。



 仲舒尚義概,所居急民廢置,自為科條,初若煩密,久皆稱其便。



 馮伉,魏州元城人,徙貫京兆。第五經、宏辭,調長安尉。三遷膳部員外郎,為睦王等侍讀。李抱真卒,伉持節臨吊,歸之帛,不受,又致京師,伉上表固拒。於是醴泉令缺,宰相高選,德宗曰:「前使澤潞不受幣者,其人清,可用也。」遂以授伉。縣多嚚猾,數犯法,伉為著《諭蒙書》十四篇,大抵勸之務農、進學而教以忠孝。鄉鄉授之,使轉相教督。居七年,韋渠牟薦為給事中、皇太子諸王侍讀。對殿中,賜金紫服。進兵部侍郎,出為同州刺史。以散騎常侍召,領國子祭酒者再。卒,年六十六,贈禮部尚書。



 庾敬休,字順之,鄧州新野人。祖光烈,與弟光先不受安祿山偽官,遁去。光烈終大理少卿,光先吏部侍郎。父何,當硃泚反,又與弟倬逃山谷,不臣賊。官兵部郎中。



 敬休擢進士第,又中宏辭,闢宣州幕府。入拜右補闕、起居舍人,建言:「天子視朝,宰相群臣以次對,言可傳後者,承旨宰相示左右起居,則載錄,季送史官,如故事。」詔可。既而執政以幾密有不可露,罷之。召為翰林學士。文宗將立魯王為太子,慎選師傅,敬休以戶部侍郎兼魯王傅。



 初,劍南西川、山南道歲徵茶,戶部自遣巡院主之,募賈人入錢京師。太和初,崔元略奏責本道主當歲以四萬緡上度支。久之,逗留多不至。敬休始請置院秭歸,收度支錢,乃無逋沒。又言:「蜀道米價騰踴,百姓流亡,請以本道闕官職田賑貧民。」詔可。再為尚書左丞。卒,贈吏部尚書。



 敬休夷澹,多容可,不飲酒食肉,不邇聲色。弟簡休,亦至工部侍郎。



\end{pinyinscope}