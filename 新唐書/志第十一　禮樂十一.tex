\article{志第十一 禮樂十一}

\begin{pinyinscope}

 聲無形而樂有器。古之作樂者,知夫器之必有弊,而聲不可以言傳,懼夫器失而聲遂亡也,乃多為之法以著之。故始求聲者以律楚、衛、魏國之間,多次制止戰爭。提出「兼相愛,交相,而造律者以黍。自一黍之廣,積而為分、寸;一黍之多,積而為龠、合;一黍之重,積而為銖、兩。此造律之本也。故為之長短之法,而著之於度;為之多少之法,而著之於量;為之輕重之法,而著之於權衡。是三物者,亦必有時而弊,則又總其法而著之於數。使其分寸、龠合、銖兩皆起於黃鐘,然後律、度、量、衡相用為表裏,使得律者可以制度、量、衡,因度、量、衡亦可以制律。不幸而皆亡,則推其法數而制之,用其長短、多少、輕重以相參考。四者既同,而聲必至,聲至而後樂可作矣。夫物用於有形而必弊,聲藏於無形而不竭,以有數之法求無形之聲,其法具存。無作則已,茍有作者,雖去聖人於千萬歲後,無不得焉。此古之君子知物之終始,而憂世之慮深,其多為之法而丁寧纖悉,可謂至矣。



 三代既亡,禮樂失其本,至其聲器、有司之守,亦以散亡。自漢以來,歷代莫不有樂,作者各因其所學,雖清濁高下時有不同,然不能出於法數。至其所以用於郊廟、朝廷,以接人神之歡,其金石之響,歌舞之容,則各因其功業治亂之所起,而本其風俗之所由。



 自漢、魏之亂,晉遷江南,中國遂沒於夷狄。至隋滅陳,始得其樂器,稍欲因而有作,而時君褊迫,不足以堪其事也。是時鄭譯、牛弘、辛彥之,何妥、蔡子元、于普明之徒,皆名知樂,相與撰定。依京房六十律。因而六之,為三百六十律,以當一歲之日,又以一律為七音,音為一調,凡十二律為八十四調,其說甚詳。而終隋之世,所用者黃鐘一宮,五夏,二舞、登歌,房中等十四調而已。



 《記》曰:「功成作樂,蓋王者未作樂之時,必因其舊而用之。唐興即用隋樂。武德九年,始詔太常少卿祖孝孫、協律郎竇璡等定樂。初,隋用黃鐘一宮,惟擊七鐘,其五鐘設而不擊,謂之啞鐘。唐協律郎張文收乃依古斷竹為十二律,高祖命與孝孫吹調五鐘,叩之而應,由是十二鐘皆用。孝孫又以十二月旋相為六十聲、八十四調。其法,因五音生二變,因變徵為正徵,因變宮為清宮。七音起黃鐘,終南呂,迭為綱紀。黃鐘之律,管長九寸,王於中宮土。半之,四寸五分,與清宮合,五音之首也。加以二變,循環無間。故一宮、二商、三角、四變徵、五徵、六羽、七變宮,其聲繇濁至清為一均。凡十二宮調,皆正宮也。正宮聲之下,無復濁音,故五音以宮為尊。十二商調,調有下聲一,謂宮也。十二角調,調有下聲二,宮、商也。十二徵調,調有下聲三,宮、商、角也。十二羽調,調有下聲四,宮、商、角、徵也。十二變徵調,居角音之後,正徵之前。十二變宮調,在羽音之後,清宮之前。雅樂成調,無出七聲,本宮遞相用。唯樂章則隨律定均,合以笙、磬,節以鐘、鼓。樂既成,奏之。



 太宗謂侍臣曰:「古者聖人沿情以作樂,國之興衰,未必由此。」御史大夫杜淹曰:「陳將亡也。有《玉樹後庭花》,齊將亡也,有《伴侶曲》,聞者悲泣,所謂亡國之音哀以思,以是觀之,亦樂之所起。」帝曰:夫聲之所感,各因人之哀樂。將亡之政,其民苦,故聞以悲。今《玉樹》、《伴侶》之曲尚存,為公奏之,知必不悲。」尚書右丞魏征進曰:「孔子稱:『樂云樂云,鐘鼓云乎哉。』樂在人和,不在音也。」十一年,張文收復請重正餘樂,帝不許,曰:「朕聞人和則樂和,隋末喪亂,雖改音律而樂不和。若百姓安樂,金石自諧矣。」



 文收既定樂,復鑄銅律三百六十、銅斛二、銅秤二、銅甌十四、稱尺一。斛左右耳與臀皆方,積十而登,以至於斛,與古玉尺、玉斗同。皆藏於太樂署。武后時,太常卿武延秀以為奇玩,乃獻之。及將考中宗廟樂,有司奏請出之,而稱尺已亡,其跡猶存,以常用度量校之,尺當六之五,量、衡皆三之一。至肅宗時,山東人魏延陵得律一,因中官李輔國獻之,云「太常諸樂調皆下,不合黃鐘,請悉更制諸鐘磬。」帝以為然,乃悉取太常諸樂器入於禁中,更加磨剡,凡二十五日而成。御三殿觀之,以還太常。然以漢律考之,黃鐘乃太簇也,當時議者以為非是。



 其後黃巢之亂,樂工逃散,金奏皆亡。昭宗即位,將謁郊廟,有司不知樂縣制度。太常博士殷盈孫按周法以算數除鎛鐘輕重高卬,黃鐘九寸五分,倍應鐘三寸三分半,凡四十八等。圖上口項之量及徑衡之圍。乃命鑄鎛鐘十二,編鐘二百四十。宰相張浚為脩奉樂縣使,求知聲者,得處士蕭承訓等,校石磬,合而擊拊之,音遂諧。



 唐為國而作樂之制尤簡,高祖、太宗即用隋樂與孝孫、文收所定而已。其後世所更者,樂章舞曲。至於昭宗,始得盈孫焉,故其議論罕所發明。若其樂歌廟舞,用於當世者,可以考也。



 樂縣之制,宮縣四面,天子用之。若祭祀,則前祀二日,大樂令設縣於壇南內壝之外,北向。東方,西方,磬虡起北,鐘虡次之。南方,北方,磬虡起西,鐘虡次之。鎛鐘十有二,在十二辰之位。樹雷鼓於北縣之內、道之左右,植建鼓於四隅。置柷、敔於縣內,柷在右,敔在左。設歌鐘、歌磬於壇上,南方北向。磬虡在西,鐘虡在東。琴、瑟、箏、築皆一,當磬虡之次,匏,竹在下。凡天神之類,皆以雷鼓;地祇之類,皆以靈鼓;人鬼之類,皆以路鼓。其設於庭,則在南,而登歌者在堂。若朝會,則加鐘磬十二虡,設鼓吹十二案於建鼓之外。案設羽葆鼓一,大鼓一,金錞一,歌、蕭、笳皆二。登歌,鐘、磬各一虡,節鼓一,歌者四人,琴、瑟、箏、築皆一,在堂上;笙、和、簫、篪、塤皆一,在堂下。若皇后享先蠶,則設十二大磬,以當辰位,而無路鼓。軒縣三百,皇太子用之。若釋奠於文宣王、武成王,亦用之。其制,去宮縣之南面。判縣二面,唐之舊禮,祭風伯、雨師、五岳、四瀆用之。其制,去軒縣之北面。皆植建鼓於東北、西北二隅。特縣,去判縣之西面,或陳於階間,有其制而無所用。



 凡橫者為簨,植者為虡。虡以縣鐘磬,皆十有六,周人謂之一堵,而唐隋謂之一虡。自隋以前,宮縣二十虡。及隋平陳,得梁故事用三十六虡,遂用之。唐初因隋舊,用三十六虡。高宗蓬萊宮成。增用七十二虡。至武后時省之。開元定禮,始依古著為二十虡。至昭宗時,宰相張浚已修樂縣,乃言:舊制,太清宮、南北郊、社稷及諸殿廷用二十虡,而太廟、含元殿用三十六虡,浚以為非古,而廟廷狹隘,不能容三十六,乃復用二十虡。而鐘虡四,以當甲丙庚壬,磬虡四,以當乙丁辛癸,與《開元禮》異,而不知其改制之時,或說以鐘磬應陰陽之位,此《禮經》所不著。



 凡樂八音,自漢以來,惟金以鐘定律呂,故其制度最詳,其餘七者,史官不記。至唐,獨宮縣與登歌、鼓吹十二案樂器有數,其餘皆略而不著,而其物名具在。八音:一曰金,為鎛鐘,為編鐘,為歌鐘,為錞,為鐃,為鐲,為鐸。二曰石,為大磬,為編磬,為歌磬。三曰土,為壎,為緌,緌,大壎也。四曰革,為雷鼓,為靈鼓,為路鼓,皆有鞀;為建鼓,為鞀鼓,為縣鼓,為節鼓,為拊,為相。五曰絲,為琴,為瑟,為頌瑟,頌瑟,箏也;為阮咸,為築。六曰木,為柷,為敔,為雅,為應。七曰匏,為笙,為竽,為巢,巢,大笙也;為和,和,小笙也。八曰竹,為簫,為管,為篪,為笛,為舂牘。此其樂器也。



 初,祖孝孫已定樂,乃曰大樂與天地同和者也,制《十二和》,以法天之成數,號《大唐雅樂》:一曰《豫和》二曰《順和》,三曰《永和》,四曰《肅和》,五曰《雍和》,六曰《壽和》,七曰《太和》,八曰《舒和》,九曰《昭和》,十曰《休和》,十一曰《正和》,十二曰《承和》。用於郊廟、朝廷,以和人神。孝孫已卒,張文收以為《十二和》之制未備,乃詔有司釐定,而文收考正律呂,超居郎呂才葉其聲音,樂曲遂備。自高宗以後,稍更其曲名。開元定禮,始復遵用孝孫《十二和》。其著於禮者:



 一曰《豫和》,以降天神。冬至祀圓丘,上辛祈穀,孟夏雩,季秋享明堂,朝日,夕月,巡狩告於圓丘,燔柴告至,封祀太山,類於上帝,皆以圜鐘為宮,三奏;黃鐘為角,太簇為徵,姑洗為羽,各一奏,文舞六成。五郊迎氣,黃帝以黃鐘為宮。赤帝以函鐘為徵,白帝以太簇為商,黑帝以南呂為羽,青帝以姑洗為角,皆文舞六成。



 二曰《順和》,以降地祇。夏至祭方丘,孟冬祭神州地祇,春秋社,巡狩告社,宜於社,禪社首,皆以函鐘為宮,太簇為角,姑洗為徵,南呂為羽。各三奏,文舞八成。望於山川,以蕤賓為宮,三奏。



 三曰《永和》,以降人鬼。時享、禘祫,有事而告謁於廟,皆以黃鐘為宮,三奏;大呂為角,太簇為征,應鐘為羽,各二奏。文舞九成。祀先農,皇太子釋奠,皆以姑洗為宮,文舞三成;送神,各以其曲一成。蠟兼天地人,以黃鐘奏《豫和》,蕤賓、姑洗、太族奏《順和》,無射、夷則奏《永和》,六均皆一成以降神,而送神以《豫和》。



 四曰《肅和》,登歌以奠玉帛。於天神,以大呂為宮;於地祇,以應鐘為宮;於宗廟,以圜鐘為宮;祀先農、釋奠,以南呂為宮;望於山川,以函鐘為宮。



 五曰《雍和》,凡祭祀以入俎。天神之俎,以黃鐘為宮;地祇之俎,以太簇為宮;人鬼之俎,以無射為宮。又以徹豆。凡祭祀,俎入之後,接神之曲亦如之。



 六曰《壽和》,以酌獻、飲福。以黃鐘為宮。



 七曰《太和》,以為行節。亦以黃鐘為宮。凡祭祀,天子入門而即位,與其升降,至於還次,行則作,止則止。其在朝廷,天子將自內出,撞黃鐘之鐘,右五鐘應,乃奏之,其禮畢,興而入,撞蕤賓之種,左五鐘應,乃奏之。皆以黃鐘為宮。



 八曰《舒和》,以出入二舞,及皇太子、王公、群后、國老若皇后之妾御、皇太子之宮臣,出入門則奏之。皆以太族之商。



 九曰《昭和》,皇帝、皇太子以舉酒。



 十曰《休和》,皇帝以飯,以肅拜三老,皇太子亦以飯。皆以其月之律均。



 十一曰《正和》,皇后受冊以行。



 十二曰《承和》,皇太子在其宮,有會以行。若駕出,則撞黃鐘,奏《太和》。出太極門而奏《採茨》,至於嘉德門而止。其還也亦然。



 初,隋有文舞、武舞,至祖孝孫定樂,更文舞曰《治康》,武舞曰《凱安》,舞者各六十四人。文舞:左籥右翟,與執纛而引者二人,皆委貌冠,黑素,絳領,廣袖,白褲,革帶,烏皮履。武舞:左干右戚,執旌居前者二人,執鞀執鐸皆二人,金錞二,輿者四人,奏者二人,執鐃二人,執相在左,執雅在右,皆二人夾導,服平冕,餘同文舞。朝會則武弁,平巾幘,廣袖,金甲,豹文褲,烏皮華。執干戚夾導,皆同郊廟。凡初獻,作文舞之舞;亞獻、終獻,作武舞之舞。太廟降神以文舞,每室酌獻,各用其廟之舞。禘祫遷廟之主合食,則舞亦如之。儀鳳二年,太常卿韋萬石定《凱安舞》六變:一變象龍興參墟;二變象克定關中;三變象東夏賓服;四變象江淮平;五變象獫狁伏從;六變復位以崇。象兵還振旅。



 初,太宗時,詔秘書監顏師古等撰定弘農府君至高祖太武皇帝六廟樂曲舞名。其後變更不一,而自獻祖而下廟舞,略可見也。獻祖曰《光大之舞》,懿祖曰《長發之舞》,太祖曰《大政之舞》,世祖曰《大成之舞》,高祖曰《大明之舞》,太宗曰《崇德之舞》,高宗曰《鈞天之舞》,中宗曰《太和之舞》,世祖曰《大成之舞》,高祖曰《大明之舞》,太宗曰《崇德之舞》,高宗曰《鈞天之舞》,中宗曰《太和之舞》,睿宗曰《景雲之舞》,玄宗曰《大運之舞》,肅宗曰《惟新之舞》,代宗曰《保大之舞》,德宗曰《文明之舞》,順宗曰《大順之舞》,憲宗曰《象德之舞》,穆宗曰《和寧之舞》,敬宗曰《大鈞之舞》,文宗曰《文成之舞》,武宗曰《大定之舞》,昭宗曰《咸寧之舞》,其餘闕而不著。



 唐之自制樂凡三大舞:一曰《七德舞》,二曰《九功舞》,三曰《上元舞》。



 《七德舞》者,本名《秦王破陣樂》。太宗為秦王,破劉武周,軍中相與作《秦王破陣樂》曲。及即位,宴會必奏之,謂侍臣曰:「雖發揚蹈厲,異乎文容,然功業由之,被於樂章,示不忘本也。」右僕射封德彞曰:「陛下以聖武戡難,陳樂象德,文容豈足道哉!」帝矍然曰:「朕雖以武功興,終以文德綏海內,謂文容不如蹈厲,斯過矣。」乃制舞圖,左圓右方,先偏後伍,交錯屈伸,以象魚麗、鵝鸛。命呂才以圖教樂工百二十八人,被銀甲執戟而舞,凡三變,每變為四陣,象擊刺往來,歌者和曰:「秦王破陣樂」。後令魏徵與員外散騎常侍褚亮、員外散騎常侍虞世南、太子右庶子李百藥更制歌辭,名曰《七德舞》。舞初成,觀者皆扼腕踴躍,諸將上壽,群臣稱萬歲,蠻夷在庭者請相率以舞。太常卿蕭瑀曰:「樂所以美盛德,形容而有所未盡,陛下破劉武周,薛舉、竇建德、王世充,原圖其狀以識。」帝曰:「方四海未定,攻伐以平禍亂,制樂陣其梗概而已。若備寫禽獲,今將相有嘗為其臣者,觀之有所不忍,我不為也。」自是元日、冬至朝會慶賀,與《九功舞》同奏。舞人更以進賢冠,虎文褲,崽蛇帶,鳥皮鞾,二人執旌居前。其後更號《神功破陣樂》。



 《九功舞》者,本名《功成慶善樂》。太宗生於慶善宮,貞觀六年幸之,宴從臣,賞賜閭里,同漢沛、宛。帝歡甚,賦詩,起居郎呂才被之管弦,名曰《功成慶善樂》,以童兒六十四人,冠進德冠,紫褲褶,長袖,漆髻,屣履而舞,號《九功舞》。進蹈安徐,以象文德。麟德二年詔:「郊廟、享宴奏文舞,用《功成慶善樂》,曳履,執紼,服褲褶,童子冠如故,武舞用《神功破陣樂》,衣甲,持戟,執纛者被金甲,八佾,加簫、笛、歌鼓,列坐縣南,若舞即與宮縣合奏。其宴樂二舞仍別設焉。」



 《上元舞》者,高宗所作也。舞者百八十人,衣畫雲五色衣,以象元氣。其樂有《上元》、《二儀》、《三才》、《四時》、《五行》、《六律》、《七政》、《八風》、《九宮》、《十洲》、《得一》、《慶雲》之曲,大祠享皆用之。至上元三年,詔:「惟圓丘,方澤、太廟乃用,餘皆罷。」又曰:「《神功破陣樂》不入雅樂,《功成慶善樂》不可降神,亦皆罷。」而效廟用《治康》、《凱安》如故。



 儀鳳二年,太常卿韋萬石奏:「請作《上元舞》,兼奏《破陣》、《慶善》二舞。而《破陣樂》五十二徧,著於雅樂者二徧;《慶善樂》五十徧,著於雅樂者一徧;《上元舞》二十九徧,皆著於雅樂。」又曰:「《雲門》、《大咸》、《大磬》、《大夏》,古文舞也。《大濩》、《大武》,古武舞也。為國家者,揖讓得天下,則先奏文舞;征伐得天下,則先奏武舞。《神功破陣樂》有武事之象,《功成慶善樂》有文事之象,用二舞,請先奏《神功破陣樂》。」初,朝會常奏《破陣舞》,高宗即位,不忍觀之,乃不設。後幸九成宮,置酒,韋萬石曰:「《破陣樂》舞,所以宣揚祖宗盛烈,以示後世,自陛下即位,寢而不作者久矣。禮,天子親總干戚,以舞先祖之樂。今《破陣樂》久廢,群下無所稱述,非所以發孝思也。」帝復令奏之,舞畢,嘆曰:「不見此樂垂三十年,追思王業勤勞若此,朕安可忘武功邪!」群臣皆稱萬歲。然遇饗燕奏二樂,天子必避位,坐者皆興。太常博士裴守真以謂「奏二舞時,天子不宜起立」。詔從之。及高宗崩,改《治康舞》曰《化康》以避諱。武後毀唐太廟。《七德》、《九功》之舞皆亡,唯其名存。自後復用隋文舞、武舞而已。



 燕樂。高祖即位,仍隋制設九部樂:《燕樂伎》,樂工舞人無變者。《清商伎》者,隋清樂也。有編鐘,編磬、獨弦琴,擊琴、瑟、奏琵琶、臥箜篌、築、箏、節鼓皆一;笙、笛、簫、篪、方響、跋膝皆二。歌二人,吹葉一人,舞者四人,並習《巴渝舞》。《西涼伎》,有編鐘、編磬皆一;彈箏、掃箏,臣箜篌、豎箜篌、琵琶。五弦笙、蕭、觱篥、小觱篥、笛、橫笛、腰鼓、齊鼓、簷鼓皆一;銅鈸二,貝一。白舞一人,方舞四人。《天竺伎》,有銅鼓,羯鼓、都曇鼓、毛員鼓,觱篥,橫笛,鳳首箜篌,琵琶、五弦,貝,紼一;銅鈸二,舞者二人。《高麗伎》,有彈箏、掃箏、鳳首箜篌、臥箜篌、豎箜篌、琵琶,以蛇皮為槽,厚寸餘,有鱗甲。楸木為面,象牙為捍撥,畫國王形。又有五弦、義觜、笛、笙、葫蘆笙、簫、小觱篥、桃皮觱篥、腰鼓、齊鼓、簷鼓、龜頭鼓、鐵版、貝、大觱篥。胡旋舞,舞者立球上,旋轉如風。《龜茲伎》,有彈箏、豎箜篌、琵琶、五弦、橫笛、笙、蕭、觱篥、答臘鼓、毛員鼓、都曇鼓,侯提鼓、雞婁鼓、腰鼓、齊鼓、簷鼓、貝,皆一;銅鈸二。舞者四人。設五方師子,高丈餘,飾以方色。每師子有十二人,畫衣,執紅拂,首加紅襪,謂之師子郎。《安國伎》,有豎箜篌、琵琶、五弦、橫笛、簫、觱篥、正鼓、和鼓、銅鈸,皆一;舞者二人。《疏勒伎》,有堅箜篌、琵琶、五弦、簫、橫笛、觱篥、答臘鼓、羯鼓、侯提鼓、腰鼓、雞婁鼓,皆一;舞者二人。《康國伎》,有正鼓、和鼓,皆一;笛、銅鈸,皆二。舞者二人。工人之服皆從其國。



 隋樂,每奏九部樂終,輒奏《文康樂》,一曰《禮畢》。虁騰時,命削去之,其後遂亡。及平高昌,收其樂。有豎箜篌、銅角,一;琵琶、五弦、橫笛、簫、觱篥、答臘鼓、腰鼓、雞婁鼓、羯鼓,皆二人。工人布巾,袷袍,錦襟,金銅帶,畫褲。舞者二人,黃袍袖,練襦,五色絳帶,金銅耳璫;赤鞾。自是初有十部樂。



 其後因內宴,詔長孫無忌制《傾杯曲》,魏徵制《樂社樂曲》,虞世南制《英雄樂曲》。帝之破竇建德也。乘馬名黃驄驃,及征高麗,死於道,頗哀惜之,命樂工制《黃驄疊曲》四曲,皆宮調也。



 五弦,如琵琶而小,北國所出,舊以木撥彈,樂工裴神符初以手彈,太宗悅甚,後人習為掃琵琶。



 高宗即位,景雲見,河水清,張文收採古誼為《景雲河清歌》,亦名燕樂。有玉磬、方響、掃箏、築、臥箜篌、大小箜篌、大小琵琶、大小五弦、吹葉、大小笙、大小觱篥、簫、銅鈸、長笛、尺八、短笛,皆一;毛員鼓、連鞉鼓、桴鼓、貝,皆二。每器工一人,歌二人。工人絳袍,金帶,烏鞾。舞者二十人。分四部:一《景雲舞》,二《慶善舞》,三《破陣舞》,四《承天舞》。《景雲樂》,舞八人,五色雲冠,錦袍,五色褲,金銅帶。《慶善樂》,舞四人,紫袍,白褲。《破陳樂》,舞四人,綾袍,絳褲。《承天樂》,舞四人,進德冠,紫袍,白褲。《景雲舞》,元會第一奏之。



 高宗以琴曲浸絕,雖有傳者,復失宮商,令有司脩習。太常丞呂才上言:「舜彈五弦之琴,哥《南風》之詩,是知琴操曲弄皆合於歌。今以御《雪詩》為《白雪歌》。古今奏正曲復有送聲,君唱臣和之義,以群臣所和詩十六韻為送聲十六節。」帝善之,乃命太常著於樂府。才復撰《琴歌》、《白雪》等曲,帝亦制歌詞十六,皆著樂府。



 帝將伐高麗,燕洛陽城門,觀屯營教舞,按新徵用武之勢,名曰《一戎大定樂》,舞者百四十人,被五採甲,持槊而舞,歌者和之,曰「八弦同軌樂。」象高麗平而天下大定也。及遼東平,行軍大總管李勣作《夷來賓》之曲以獻。



 調露二年,幸洛陽城南樓,宴群臣,太常奏《六合還淳》之舞,其容制不傳。



 高宗自以李氏老子之後也,於是命樂工制道調。



\end{pinyinscope}