\article{列傳第一百四十二下 回鶻下}

\begin{pinyinscope}

 回鶻之請昏,有司度費當五百萬,帝方內討強節度,故遣宗正少卿李誠、太常博士殷侑往諭不可。穆宗立,回鶻又使合達干等來固求昏題》等。,許之。俄而可汗死,使者臨冊所嗣為登囉羽錄沒蜜施句主毘伽崇德可汗。可汗已立,遣伊難珠、句錄、都督思結等以葉護公主來逆女,部渠二千人,納馬二萬、橐它千。四夷之使中國,其眾未嘗多此。詔許五百人至長安,餘留太原。詔以太和公主下降。主,憲宗女也。帝為主建府,以左金吾衛大將軍胡證、光祿卿李憲持節護送,太府卿李說為昏禮使,冊拜主為仁孝端麗明智上壽可敦,告於廟,天子禦通化門餞主,群臣班辭於道。公主出塞,距回鶻牙百里,可汗欲先與主由間道私見,胡證不可,虜人曰:「昔咸安公主行之。」證曰:「天子詔我送公主授可汗,今未見,不可先也。」乃止。於是可汗升樓坐,東向,下設毳幔以居公主,請襲胡衣,以一姆侍出,西向拜已,退即次,被可敦服,絳通裾大襦,冠金冠,前後銳,復出拜已,乃升曲輿,九相分負,右旋於廷者九,降輿升樓,與可汗聯坐,東向,群臣以次謁。可敦亦自建牙,以二相出入帳中。證等歸,可敦大宴,悲啼眷慕。可汗厚贈使者。



 是時,裴度方伐幽、鎮,回鶻使渠將李義節以兵三千佐天子平河北,議者懲艾前患,不聽,兵已及豐州,使者厚賜乃去。


敬宗即位之年,可汗死,其弟曷薩特勒立,遣使者冊為愛登里羅汨沒密施合毘伽昭禮可汗,賜幣十二車。文宗初,又賜馬直絹五十萬。大和六年,可汗為其下所殺,從子胡特勒立,使者來告。明年,遣左驍衛將軍唐弘實與嗣澤王溶持節冊為愛登里羅汨沒蜜施合句錄毘伽彰信可汗。開成四年,其相掘羅勿作難,引沙陀共攻可汗,可汗自殺,國人立
 \gezhu{
  廠盍}
 馺特勒為可汗。方歲饑,遂疫,又大雪,羊、馬多死,未及命。武宗即位,以嗣澤王溶臨告,乃知其國亂。



 俄而渠長句錄莫賀與黠戛斯合騎十萬攻回鶻城,殺可汗,誅掘羅勿,焚其牙,諸部潰其相馺職與厖特勒十五部奔葛邏祿,殘眾入吐蕃、安西。於是,可汗牙部十三姓奉烏介特勒為可汗,南保錯子山。黠戛斯已破回鶻,得太和公主;又自以李陵後,與唐同宗,故遣使者達干奉主來歸。烏介怒,追擊達干殺之,劫主南度磧,邊人大恐。進攻天德城,振武節度使劉沔屯雲伽關拒卻之。宰相李德裕建言:「回鶻曩有功,今饑且亂,可汗無歸,不可擊,宜遣使者贍安之。」帝用兵部郎中李拭行邊刺狀。於是,其相赤心與王子嗢沒斯、特勒那頡啜將其部欲自歸,而公主亦遣使者來言烏介已立,因請命。又大臣頡乾伽思等表假振武居公主、可汗。帝乃詔右金吾衛大將軍王會持節慰撫其眾,輸糧二萬斛,不許借振武,令中人好語開諭;又詔使者持冊往,潛稽其行,須變。



 明年,回鶻奉主至漠南,入雲、朔,剽橫水,殺掠甚眾,轉側天德、振武間,盜畜牧自如。乃召諸道兵合討。嗢沒斯以赤心奸桀,難得要領,即密約天德戍將田牟,誘赤心斬帳下。那頡啜收赤心眾七千帳東走振武、大同,因室韋、黑沙南窺幽州,節度使張仲武破之,悉得其眾。那頡啜走,烏介執而殺之。然烏介兵尚強,號十萬,駐牙大同北閭門山。而特勒厖俱遮、阿敦寧等凡四部,及將軍曹磨你眾三萬,因仲武降,嗢沒斯亦附使者送款。帝欲使助可汗復國,而可汗已攻雲州,劉沔與戰,敗績。嗢沒斯率三部及特勒、大酋二千騎詣振武降。詔拜嗢沒斯為右金吾衛大將軍,爵懷化郡王,以天德為歸義軍,即拜歸義軍使;阿歷支寧邊郡公,習勿啜昌化郡公,烏羅思寧朔郡公,並為冠軍大將軍、左威衛大將軍;愛邪勿寧塞郡公,為右領軍大將軍。加賜嗢沒斯牙旗、豹尾、刀器諸物,給其屬冠帶。詔宰相德裕採秦、漢以來興殊俗、忠效卓異者凡三十人,為《異域歸忠傳》寵賜之。嗢沒斯請留族太原,率昆弟為天子捍邊,帝命劉沔為列舍雲、朔間處其家。可汗遣使者藉兵欲還故廷,且假天德城,帝不許。可汗恚,進略大同川,轉戰攻雲州,刺史嬰壁不敢出。詔益發諸鎮兵屯太原以北。



 嗢沒斯等既朝,皆賜李氏,名嗢沒斯曰思忠,阿歷支曰思貞,習勿啜曰思義,烏羅思曰思禮;愛邪勿曰弘順,即拜歸義軍副使。於是,詔劉沔為回鶻南面招撫使,張仲武東面招撫使,思忠為河西黨項都將、西南面招討使,沔營雁門。又詔銀州刺史何清朝、蔚州刺史契苾通,以蕃、渾兵出振武,與沔、仲武合,稍逼回鶻。思忠數深入諭降其下。沔分沙陀兵益思忠,河中軍以騎五百益弘順。沔進次雲州,思忠屯保大柵率河中、陳許兵與回鶻戰,敗之。明年,又為弘順所破。沔與天德行營副使石雄料勁騎及沙陀、契苾等雜虜,夜出雲州,走馬邑,抵安眾塞,逢虜,與戰破之。烏介方薄振武,雄馳入,夜穴壘出鏖兵,烏介驚,引去,雄追北至殺胡山,烏介被創走。雄遇公主,奉主還,降特勒以下眾數萬,盡收輜帑及所賜詔書。可汗收所餘往依黑車子,詔弘順、清朝窮躡。弘順厚啖黑車子以利,募殺烏介。初,從可汗亡者既不能軍,往往詣幽州降,留者皆饑寒痕夷,裁數千。黑車子幸其殘,即殺烏介。其下又奉其弟遏捻特勒為可汗。帝詔德裕紀功銘石於幽州,以誇後世。



 思忠等以國亡,皆願入朝,見聽,遂罷歸義軍,擢思忠左監門衛上將軍兼撫王傅,兩稟其奉,賜第永樂坊,分其兵賜諸節度。虜人憚隸食諸道,據滹沱河叛,劉沔坑殺三千人。詔回鶻營功德使在二京者,悉冠帶之。有司收摩尼書若象燒於道,產貲入之官。



 遏捻可汗裒殘部五千,仰食於奚大酋碩舍朗。大中初,仲武討奚,破之,回鶻浸耗滅,所存名王貴臣五百餘,轉依室韋。仲武諭令羈致可汗等,遏捻懼,挾妻葛祿、子特勒毒斯馳九騎夜委眾西走,部人皆慟哭。室韋七姓析回鶻隸之。黠戛斯怒,與其相阿播將兵七萬擊室韋,悉收回鶻還磧北。遺帳伏山林間,狙盜諸蕃自給,稍歸厖特勒。是時,特勒已自稱可汗,居甘州,有磧西諸城。宣宗務綏柔荒遠,遣使者抵靈州省其酋長,回鶻因遣人隨使者來京師,帝即冊拜嗢祿登里邏汨沒蜜施合俱錄毘伽懷建可汗。後十餘年,一再獻方物。



 懿宗時,大酋僕固俊自北庭擊吐蕃,斬論尚熱盡取西州、輪臺等城,使達干米懷玉朝且獻俘,因請命,詔可。其後王室亂,貢會不常,史亡其傳。



 昭宗幸鳳翔,靈州節度使韓遜表回鶻請率兵赴難,翰林學士韓偓曰:「虜為國仇舊矣。自會昌時伺邊,羽翼未成,不得逞。今乘我危以冀幸,水可開也。」遂格不報。然其國卒不振,時時以玉、馬與邊州相市云。



 薛延陀者,先與薛種雜居,後滅延陀部有之,號薛延陀,姓一利咥氏。在鐵勒諸部最雄張,風俗大抵與突厥同。



 西突厥處羅可汗之殺鐵勒諸酋也,其下往往相率叛去,推契苾哥楞為易勿真莫賀可汗,據貪汗山,奉薛延陀乙失缽為野咥可汗,保燕末山。而突厥射匱可汗復強,二部黜可汗號往臣之。回紇、拔野古、阿跌、同羅、僕骨、白霫在鬱督軍山者,東附始畢可汗;乙失缽在金山者,西役葉護可汗。



 貞觀二年,葉護死,其國亂,乙失缽孫曰夷男,率部帳七萬附頡利可汗。後突厥衰,夷男反攻頡利,弱之,於是諸姓多叛頡利,歸之者共推為主,夷男不敢當。明年,太宗方圖頡利,遣游擊將軍喬師望曨路齎詔書、鼓纛,冊拜夷男為真珠毘伽可汗。夷男已受命,遣使謝,歸方物,乃樹牙鬱督軍山,直京師西北六千里,東靺鞨,西葉護突厥,南沙磧,北俱倫水,地大眾附,於是回紇等諸部莫不伏屬。其弟統特勒入朝,帝以精刀、寶鞭賜之曰:「下有大過者,以吾鞭鞭之。」夷男以為寵。頡利可汗之滅,塞隧空荒,夷男率其部稍東,保都尉楗山獨邏水之陰,遠京師才三千里而贏,東室韋,西金山,南突厥,北瀚海,蓋古匈奴地也。勝兵二十萬,以二子大度設、突利失分將之,號南、北部。七年間,使者八朝。帝恐後強大為患,欲產其禍,乃下詔拜其二子皆為小可汗。



 十五年,帝以李思摩為可汗,始度河,牙於漠南。夷男惡之,未發。方帝幸洛陽,將遂封泰山,夷男與其下謀曰:「天子封泰山,萬國皆助兵,悉會行在,邊鄣空單,思摩可取也。」乃使大度設勒兵二十萬,南絕漠,壁白道川,率一兵得四馬,擊思摩。思摩走朔州,言狀,且請師。於是詔營州都督張儉統所部與奚、霫、契丹乘其東,朔州道行軍總管李勣眾六萬、騎三千,營朔州,靈州道行軍總管李大亮眾四萬、騎五千,屯靈武,慶州道行軍總管張士貴眾萬七千出雲中,涼州道行軍總管李襲譽經略之。帝敕諸將曰:「延陀度漠,馬已疲。夫用兵者,見利疾進,不利亟去。今虜不急擊思摩,又不速還,勢必敗,卿等勿與戰,須其歸,可擊也。」既而延陀使者來,求與突厥平。帝曰:「我約漠以北,延陀制之,漠以南,突厥專之,有輒相掠,誅不赦。延陀父事我而首違詔,得非亂邪?而曰與突厥和,乃故約也,尚何請?」不報。



 大度設次長城,思摩已南走,大度設度不可得,乃遣人乘長城罵之。適會勣兵至,行盍屬天,遽率眾走赤柯,度青山,然道回遠,勣選敢死士與突騎徑臘河,趣白道,及大度設,尾之不置。大度設顧不脫,度諾真水,陣以待。先是,延陀擊沙缽羅及阿史那社爾,皆以徒戰勝,至是卻騎不用,率五人為伍,一執馬,四前鬥,令曰:「勝則騎而逐,負者死,沒其家以償戰士。」及戰,突厥兵迮,延陀騰逐,勣救之,延陀縱射,馬輒死。勣乃以步士百人為隊,搗其罅,虜潰,部將薛萬徹率勁騎先收執馬者,故延陀不能去,斬首數千級,獲馬萬五千。大度設亡去,萬徹追弗及。殘卒奔漠北,會雪甚,眾皸踣死者十八。始延陀能以術禬神致雪,冀困勣師,及是反自敝云。



 勣還入定襄,天子遣使者齎璽書勞問,賞功恤死。延陀之使留待命者,帝悉還之,曰:「歸語爾可汗,爾自負其強,以突厥為弱,厚誅斂之,又取首領以為質,且我為天下主,渠嘗賦發於爾邪?後有利害,當謹思,毋遽也。」延陀乃遣使謝罪,又遣其仲父沙缽羅獻馬三千,因請昏。帝曰:「延陀本一俟斤,我則立之,度其力孰與頡利比,而敢橈邊乎?」不許昏。



 明年,以使來益獻馬、牛、羊、橐它,固求昏。帝與大臣計曰:「延陀屈強,朕策顧有二:選士十萬擊之,使無遺種,百年計也;絕昏羈縻,使無邊憂,三十年計也。然則孰利?」房玄齡曰:「今大亂餘氓,痍破未完,戰雖勝,猶危道也。不如和親。」帝曰:「善。」許以新興公主下嫁,召突利失大享,群臣侍,陳寶器,奏《慶善》、《破陣》盛樂及十部伎,突利失頓首上千萬歲壽。詔夷男親迎,帝將幸靈州以成昏事。夷男大喜,詫曰:「我鐵勒部人耳,上以我為可汗,公主以女我,乘輿為我幸邊,誰與我榮?」乃搜賦諸下羊馬為貲。或說夷男曰:「可汗與唐,皆一國主,奈何往朝?有如見款,尚可悔?」夷男曰:「不然。吾聞唐天子有德,四方共臣之,藉獨留我,磧北亦須有主,然舍我而求它,非計也。」下乃不敢言。



 時帝詔有司受所獻,延陀無府庫,調斂於下,不亟集,又度磧,水草乏,馬羊多死,納貢後期,帝亦止行。畜口耗死僅半,議者謂:「夷狄嘗為中國私,今禮不具而與昏,恐後有輕中國心。」乃下詔絕昏,謝其使。或曰:「既許之,信不可失。」帝曰:「公等計非也。昔漢匈奴強,中國不抗,故飾子女嫁單于。今北狄弱,我能制之,而延陀方謹事我者,顧新立,倚我以服眾。彼同羅、僕骨力足制延陀而不發,懼我也。我又妻之,固中國婿,名重而援堅,諸部將歸之,戎狄野心,能自立則叛矣。今絕昏,使諸姓聞之,將爭擊延陀,亡可待也。」李思摩果侵掠之。延陀遣突利失寇定襄,詔李勣逐出塞。俄遣使請率師助伐高麗,以刺帝意,帝引使者謂曰:「歸語爾可汗,我父子東征,能寇邊者可即來。」夷男沮縮,不敢謀,以使謝,固請助軍。帝嘉答。高麗莫離支令靺鞨以厚利啖夷男,欲與連和,夷男氣素索,不發,亦會病死,帝為祭於行。



 始延陀請以庶子曳莽為突利失可汗,統東方;嫡子拔灼為肆葉護可汗,統西方。白道之役,曳莽實為之謀,國人多怨。及會葬,曳莽亟還部,拔灼分兵襲殺之,自立為頡利俱利失薛沙多彌可汗。方是時,王師猶在遼,因即寇邊。帝遣江夏王道宗屯朔州,代州都督薛萬徹與左驍衛大將軍阿史那社爾屯勝州,左武候大將軍薩孤吳仁屯靈州,執失思力與突厥掎角塞下,虜知有備,乃去。



 拔灼性卞克,多殺父時貴臣而任所親暱,國人不安,而阿波設與唐使者遇於靺鞨東鄙,小戰不利,還怖國人曰:「唐兵至矣!」眾大擾,諸部遂潰。多彌可汗以十餘騎遁去,依阿史那時健,俄為回紇所殺,盡屠其宗,眾五六萬奔西域,立真珠毘伽可汗昆弟子咄摩支,號伊特勿失可汗,遣使者上言:「願保鬱督軍山。」常詔兵部尚書崔敦禮與李勣尉安之,俾定其國。



 鐵勒諸部素伏延陀,而咄摩支雖衰孑,尚臣畏之。帝恐卒為患,詔勣等曰:「降則撫之,叛則擊之。」勣至,咄摩支大駭,陰欲拒戰,外好言乞降。勣知之,縱兵擊,斬五千餘級,系老孺三萬,遂滅其國。咄摩支聞天子使者蕭嗣業在回紇,身詣嗣業丐降,入朝,拜右武衛將軍,賜田宅。初,延陀將滅,有丐食於其部者,延客帳中,妻視客人而狼首,主不覺,客已食,妻語部人共追之,至鬱督軍山,見二人焉,曰:「我神也,薛延陀且滅。」追者懼,卻走,遂失之。至是果敗此山下。



 帝以延陀滅,欲並契苾等降之,復遣道宗率阿史那社爾等分部窮討,帝幸靈州,節度諸將。於是鐵勒十一部皆歸命天子,請吏內屬。道宗等徑磧擊延陀餘眾阿波達干,斬首千餘級,逐北二百里。萬徹抵北道,諭降回紇諸酋。虜所遣使踵及帝行在,凡數千人,上言:「天至尊為可汗,世世以奴事,死不恨。」帝剖其地為州縣,北荒遂平。諸姓有來朝者,帝勞曰:「爾來,若鼠得穴、魚得泉,我為爾深廣之。」又曰:「我在,天下四夷有不安安之,不樂樂之,如驥尾受蒼蠅,可使日千里也。」於是告功太廟,賜民三日酺。後三年,餘部叛,以右領軍大將軍執失思力討平之。至永徽時,延陀部亡散者悉還,高宗為置嵠彈州處安之。



 拔野古一曰拔野固,或為拔曳固,漫散磧北,地千里,直僕骨東,鄰於靺鞨。帳戶六萬,兵萬人。地有薦草,產良馬、精鐵。有川曰康干河,斷松投之,三年輒化為石,色蒼致,然節理猶在,世謂康干石者。俗嗜獵射,少耕獲,乘木逐鹿冰上。風俗大抵鐵勒也,言語少異。貞觀三年,與僕骨、同羅、奚、霫同入朝。二十一年,大俟利發屈利失舉部內屬,置幽陵都督府,拜屈利失右武衛大將軍,即為都督。顯慶時,與思結、僕固、同羅叛,以左武衛大將軍鄭仁泰擊之,斬其渠首。至天寶間,能自來朝。



 僕骨亦曰僕固,在多覽葛之東。帳戶三萬,兵萬人。地最北,俗梗驁,難召率。始臣突厥,後附薛延陀。延陀滅,其酋娑匐俟利發歌濫拔延始內屬,以其地為金微州,拜歌濫拔延為右武衛大將軍、州都督。開元初,為首領僕固所殺,詣朔方降,有司誅之。子曰懷恩,至德時以功至朔方節度使,自有傳。



 同羅在薛延陀北,多覽葛之東,距京師七千里而贏,勝兵三萬。貞觀二年,遣使者入朝。久之,請內屬,置龜林都督府,拜酋俟利發時健啜為左領軍大將軍,即授都督。安祿山反,劫其兵用之,號「曳落河」者也。曳落河,猶言健兒云。



 渾在諸部最南者。突厥頡利敗時,有俟利發阿貪支款塞。薛延陀之滅,大俟利發渾汪舉部內向,以其地為皋蘭都督府,後分東、西州。太宗以阿貪支於汪屬尊,遣譯者諷汪,汪欣然避位。帝嘉其讓,以阿貪支為右領軍衛大將軍、皋蘭州刺史,汪雲麾將軍兼俟利發為之副。阿貪支死,子回貴嗣。回貴死,子大壽嗣。大壽死,子釋之嗣。釋之鷙勇不凡,從哥舒翰拔石堡城,遷右武衛大將軍,封汝南郡公。李光弼保河陽,釋之以朔方都知兵馬使為裨將,進寧朔郡王,知朔方節度留後。僕固懷恩之走,聲為歸鎮。釋之曰:「是必眾潰。」將拒之,其甥張韶曰:「彼如悔禍還鎮,渠可不納?」釋之信之,乃納懷恩。懷恩已入,使韶殺釋之,收其軍。已而惡韶,罵曰:「若負舅,肯忠於我?」折其脛,囚死彌峨城。釋之子瑊,建中功臣也,自有傳。



 契苾亦曰契苾羽,在焉耆西北鷹娑川,多覽葛之南。其酋哥楞自號易勿真莫賀可汗,弟莫賀咄特勒,皆有勇。莫賀咄死,子何力尚紐率其部來歸,時貞觀六年也。詔處之甘、涼間,以其地為榆溪州。永徽四年,以其部為賀蘭都督府,隸燕然都護。何力有戰功,忠節臣也。大和中,其種帳附於振武云。



 多覽葛亦曰多濫,在薛延陀東,濱同羅水,勝兵萬人。延陀已滅,其酋俟斤多濫葛末與回紇皆朝,以其地為燕然都督府,授右衛大將軍,即為府都督。死,以多濫葛塞匐為大俟利發,繼為都督。



 阿跌,亦曰訶咥,或為鞬跌。始與拔野古等皆朝,以其地為雞田州。開元中,鞬跌思泰自突厥默啜所來降。其後,光進、光顏皆以戰功至大官,賜李氏,附屬籍,自有傳。



 葛邏祿本突厥諸族,在北庭西北、金山之西,跨僕固振水,包多怛嶺,與車鼻部接。有三族:一謀落,或為謀刺;二熾俟,或為婆匐;三踏實力。永徽初,高偘之伐車鼻可汗,三族皆內屬。顯慶二年,以謀落部為陰山都督府,熾俟部為大漠都督府,踏實力部為玄池都督府,即用其酋長為都督。後分熾俟部置金附州。三族當東、西突厥間,常視其興衰,附叛不常也。後稍南徙,自號「三姓葉護」,兵強,甘於斗,廷州以西諸突厥皆畏之。開元初,再來朝。天寶時,與回紇、拔悉蜜共攻殺烏蘇米施可汗,又與回紇擊拔悉蜜,走其可汗阿史那施於北庭,奔京師。葛祿與九姓復立回紇葉護,所謂懷仁可汗者也。於是葛祿之處烏德犍山者臣回紇,在金山、北庭者自立葉護,歲來朝。久之,葉護頓毘伽縛突厥叛酋阿布思,進封金山郡王。天寶間,凡五朝。至德後,葛邏祿浸盛,與回紇爭強,徙十姓可汗故地,盡有碎葉、怛邏斯諸城。然限回紇,故朝會不能自達於朝。



 拔悉蜜,貞觀二十三年始來朝。天寶初,與回紇葉護擊殺突厥可汗,立拔悉蜜大酋阿史那施為賀臘毘伽可汗,遣使者入謝,玄宗賜紫文袍、金鈿帶、魚袋。不三歲,為葛邏祿、回紇所破,奔北庭。後朝京師,拜左武衛將軍,地與眾歸回紇。



 都播,亦曰都波,其地北瀕小海,西堅昆,南回紇,分三部,皆自統制。其俗無歲時。結草為廬。無畜牧,不知稼穡,土多百合草,掇其根以飯,捕魚、鳥、獸食之。衣貂鹿皮,貧者緝鳥羽為服。其昏姻,富者納馬,貧者效鹿皮草根。死以木斂置山中,或系於樹,送葬哭泣,與突厥同。無刑罰,盜者倍輸其贓。貞觀二十一年,因骨利幹入朝,亦以使通中國。



 骨利幹處瀚海北,勝兵五千。草多百合。產良馬,首似橐它,筋骼壯大,日中馳數百里。其地北距海,去京師最遠,又北度海則晝長夜短,日入亨羊胛,熟,東方已明,蓋近日出處也。既入朝,詔遣雲麾將軍康蘇蜜勞答,以其地為玄闕州。其大酋俟斤因使者獻馬,帝取其異者號十驥,皆為美名:曰「騰霜白」,曰「雪驄』,曰「凝露驄」,曰「縣光驄」,曰「決波騟』,曰「飛霞驃」,曰「發電赤」,曰「流金瓜」,曰「翔麟紫」,曰「奔虹赤」,厚禮其使。龍朔中,以玄闕州更為余吾州,隸瀚海都督府。延載初,亦來朝。



 白霫居鮮卑故地,直京師東北五千里,與同羅、僕骨接。避薛延陀,保奧支水、冷陘山,南契丹,北烏羅渾,東靺鞨,西拔野古,地圓袤二千里,山繚其外,勝兵萬人。業射獵,以赤皮緣衣,婦貫銅釧,以子鈴綴襟。其部有三:曰居延,曰無若沒,曰潢水。其君長臣突厥頡利可汗為俟斤。貞觀中再來朝,後列其地為寘顏州,以別部為居延州,即用俟斤為刺史。顯慶五年,授酋長李含珠為居延都督。含珠死,弟厥都繼之。後無聞焉。



 斛薛處多濫葛北,勝兵萬人。奚結處同羅北,思結在延陀故牙,二部合兵凡二萬。既來朝,列其地州縣之。太宗時,北狄能自通者,又有烏羅渾,或曰烏洛侯,曰烏羅護,直京師東北六千里而贏,東靺鞨,西突厥,南契丹,北烏丸,大抵風俗皆靺鞨也。烏丸或曰古丸。



 又有鞠,或曰祴,居拔野古東北,有木無草,地多苔。無羊馬,人豢鹿若牛馬,惟食苔,俗以駕車。又以鹿皮為衣,聚木作屋,尊卑共居。



 又有俞折者,地差大,俗與拔野古相埒。少羊馬,多貂鼠。



 又有駁馬者,或曰弊剌,曰遏羅支,直突厥之北,距京師萬四千里。隨水草,然喜居山,勝兵三萬。地常積雪,木不雕。以馬耕田,馬色皆駁,因以名國云。北極於海,雖畜馬而不乘,資湩酪以食。好與結骨戰,人貌多似結骨,而語不相通。皆劗發,樺皮帽。構木類井幹,覆樺為室。各有小君長,不能相臣也。



 大漢者,處鞠之北,饒羊馬,人物頎大,故以自名。與鞠俱鄰於黠戛斯劍海之瀕。



 此皆古所未賓者,當貞觀逮永徽,奉貂馬入朝,或一再至。



 黠戛斯,古堅昆國也。地當伊吾之西,焉耆北,白山之旁。或曰居勿,曰結骨。其種雜丁零,乃匈奴西鄙也。匈奴封漢降將李陵為右賢王,衛律為丁零王。後郅支單于破堅昆,於時東距單于廷七千里,南車師五千里,郅支留都之。故後世得其地者訛為結骨,稍號紇骨,亦曰紇扢斯云。眾數十萬,勝兵八萬,直回紇西北三千里,南依貪漫山。地夏沮洳,冬積雪。人皆長大,赤發、析面、綠瞳,以黑發為不祥。黑瞳者,必曰陵苗裔也。男少女多,以環貫耳,俗趫伉,男子有男黥其手,女已嫁黥項。雜居多淫佚。



 謂歲首為茂師哀,以三哀為一時,以十二物紀年,如歲在寅則曰虎年。氣多寒,雖大河亦半冰。稼有禾、粟、大小麥、青稞,步磑以為面糜。穄以三月種,九月獲,以飯,以釀酒,而無果蔬。畜,馬至壯大,以善鬥者為頭馬,有橐它、牛、羊,牛為多,富農至數千。其獸有野馬、骨咄、黃羊、原羝、鹿、黑尾,黑尾者似麞,尾大而黑。魚,有蔑者長七八尺,莫痕者無骨,口出頤下。鳥,雁、鶩、烏鵲、鷹、隼。木,松、樺、榆、柳、蒲。松高者仰射不能及顛,而樺尤多。有金、鐵、錫,每雨,俗必得鐵,號迦沙,為兵絕犀利,常以輸突厥。其戰有弓矢、旗幟,其騎士析木為盾,蔽股足,又以圓盾傅肩,而捍矢刃。



 其君曰「阿熱」,遂姓阿熱氏,建一纛,下皆尚赤,餘以部落為之號。服貴貂、豽,阿熱冬帽貂,夏帽金扣,銳頂而卷末,諸下皆帽白氈,喜佩刀礪,賤者衣皮不帽,女衣毳毼、錦、罽、綾,蓋安西、北庭、大食所貿售也。阿熱駐牙青山,周柵代垣,聯氈為帳,號「密的支」,它首領居小帳。凡調兵,諸部役屬者悉行。內貂鼠、青鼠為賦。其官,宰相、都督、職使、長史、將軍、達干六等。宰相七,都督三、職使十,皆典兵;長史十五,將軍、達干無員。諸部食肉及馬酪,惟阿熱設餅餌。樂有笛、鼓、笙、觱篥、盤鈴。戲有弄駝、師子、馬伎、繩伎。祠神惟主水草,祭無時,呼巫為「甘」。昏嫁納羊馬以聘,富者或百千計。喪不剺面,三環尸哭,乃火之,收其骨,歲而乃墓,然後器泣有節。冬處室,木皮為覆。其文字言語,與回鶻正同。法最嚴,臨陣橈、奉使不稱、妄議國若盜者皆斷首;子為盜,以首著父頸,非死不脫。



 阿熱牙至回鶻牙所,橐它四十日行。使者道出天德右二百里許抵西受降城,北三百里許至鷿鵜泉,泉西北至回鶻牙千五百里許,而有東、西二道,泉之北,東道也。回鶻牙北六百里得仙娥河,河東北曰雪山,地多水泉。青山之東,有水曰劍河,偶艇以度,水悉東北流,經其國,合而北入於海。



 東至木馬突厥三部落,曰都播、彌列、哥餓支,其酋長皆為頡斤。樺皮覆室,多善馬,俗乘木馬馳冰上,以板藉足,屈木支腋,蹴輒百步,勢迅激。夜鈔盜,晝伏匿,堅昆之人得以役屬之。



 堅昆,本強國也,地與突厥等,突厥以女妻其酋豪,東至骨利幹,南吐蕃,西南葛邏祿。始隸薛延陀,延陀以頡利發一人監國。其酋長三人,曰訖悉輩,曰居沙波輩,曰阿米輩,共治其國,未始與中國通。貞觀二十二年,聞鐵勒等已入臣,即遣使者獻方物,其酋長俟利發失缽屈阿棧身入朝,太宗勞享之,謂群臣曰:「往渭橋斬三突厥,自謂功多,今俟利發在席,更覺過之。」俟利發酒酣,奏願得持笏,帝以其地為堅昆府,拜俟利發左屯衛大將軍,即為都督,隸燕然都護。高宗世,再來朝。景龍中,獻方物,中宗引使者勞之曰:「而國與我同宗,非它蕃比。」屬以酒,使者頓首。玄宗世,四朝獻。



 乾元中,為回紇所破,自是不能通中國。後狄語訛為黠戛斯,蓋回鶻謂之,若曰黃赤面云,又訛為戛戛斯。然常與大食、吐蕃、葛祿相依杖,吐蕃之往來者畏回鶻剽鈔,必住葛祿,以待黠戛斯護送。大食有重錦,其載二十橐它乃勝,既不可兼負,故裁為二十匹,每三歲一餉黠戛斯。而回鶻授其君長阿熱官為「毘伽頓頡斤」。



 回鶻稍衰,阿熱即自稱可汗。其母,突騎施女也,為母可敦;妻葛祿葉護女,為可敦。回鶻遣宰相伐之,不勝,挐斗二十年不解。阿熱恃勝,乃肆詈曰:「爾運盡矣!我將收爾金帳,於爾帳前馳我馬,植我旗,爾能抗,亟來,即不能,當疾去。」回鶻不能討,其將句錄莫賀導阿熱破殺回鶻可汗,諸特勒皆潰。阿熱身自將,焚其牙及公主所廬金帳者,回鶻可汗常坐也。乃悉收其寶貲,並得太和公主,遂徙牙牢山之南。牢山亦曰賭滿,距回鶻舊牙度馬行十五日。阿熱以公主唐貴女,遣使者衛送公主還朝,為回鶻烏介可汗邀取之,並殺使者。



 會昌中,阿熱以使者見殺,無以通於朝,復遣注吾合素上書言狀。注吾,虜姓也;合,言猛;素者,左也,謂武猛善左射者。行三歲至京師,武宗大悅,班渤海使者上,以其處窮遠,能脩職貢,命太僕卿趙蕃持節臨慰其國,詔宰相即鴻臚寺見使者,使譯官考山川國風。宰相德裕上言:「貞觀時,遠國皆來,中書侍郎顏師古請如周史臣集四夷朝事為《王會篇》。今黠戛斯大通中國,宜為《王會圖》以示後世。」有詔以鴻臚所得繢著之。又詔阿熱著宗正屬籍。



 是時,烏介可汗餘眾托黑車子,阿熱願乘秋馬肥擊取之,表天子請師。帝令給事中劉濛為巡邊使,朝廷亦以河、隴四鎮十八州久淪戎狄,幸回鶻破弱,吐蕃亂,相殘嚙,可乘其衰。乃以右散騎常侍李拭使黠戛斯,冊君長為宗英雄武誠明可汗。未行,而武宗崩。宣宗嗣位,欲如先帝意,或謂黠戛斯小種,不足與唐抗,詔宰相與臺省四品以上官議,皆曰:「回鶻盛時有冊號,今幸衰亡,又加黠戛斯,後且生患。」乃止。至大中元年,卒詔鴻臚卿李業持節冊黠戛斯為英武誠明可汗。逮咸通間,三來朝。然卒不能取回鶻。後之朝聘冊命,史臣失傳。



 贊曰:夷狄資悍貪,人外而獸內,惟剽奪是視。故湯、武之興,未嘗與共功,蓋疏而不戚也。太宗初興,嘗用突厥矣,不勝其暴,卒縛而臣之。肅宗用回紇矣,至略華人,辱太子,笞殺近臣,求索無倪。德宗又用吐蕃矣,劫平涼,敗上將,空破西陲。所謂引外禍平內亂者也。夫用之以權,制之以謀,惟太宗能之。若二主懦昏,狃而狎之,烏勝其弊哉!彼親之則責償也多,慊而不滿則滋怨,化以仁義則頑,示以法則忿,熟我險易則為患也博而慘,療餒以冶葛,何時可哉?故《春秋》許夷狄者,不一而足,信矣。



\end{pinyinscope}