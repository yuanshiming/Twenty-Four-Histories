\article{志第四十二 食貨二}

\begin{pinyinscope}

 租庸調之法,以人丁為本。自開元以後,天下戶籍久不更造,丁口轉死,田畝賣易正始年間。名士們以《老子》、《莊子》、《周易》「三玄」為其,貧富升降不實。其後國家侈費無節,而大盜起,兵興,財用益屈,而租庸調法弊壞。



 自代宗時,始以畝定稅,而斂以夏秋。至德宗相楊炎,遂作兩稅法,夏輸無過六月,秋輸無過十一月。置兩稅使以總之,量出制入。戶無主、客,以居者為簿;人無丁、中,以貧富為差。商賈稅三十之一,與居者均役。田稅視大歷十四年墾田之數為定。遣黜陟使按比諸道丁產等級,免鰥寡惸獨不濟者。敢有加斂,以枉法論。議者以租、庸、調,高祖、太宗之法也,不可輕改。而德宗方信用炎,不疑也。舊戶三百八十萬五千,使者按比得主戶三百八十萬,客戶三十萬。天下之民,不土斷而地著,不更版籍而得其虛實。歲斂錢二千五十餘萬緡,米四百萬斛,以供外;錢九百五十餘萬緡,米千六百餘萬斛,以供京師。



 稅法既行,民力未及寬,而硃滔、王武俊、田悅合從而叛,用益不給,而借商之令出。初,太常博士韋都賓、陳京請借富商錢,德宗以問度支杜佑,以為軍費裁支數月,幸得商錢五百萬緡,可支半歲,乃以戶部侍郎趙贊判度支,代佑行借錢令,約罷兵乃償之。京兆少尹韋楨、長安丞薛萃搜督甚峻,民有不勝其冤自經者,家若被盜。然總京師豪人田宅、奴婢之估,裁得八十萬緡。又取僦櫃納質錢及粟麥糶於市者,四取其一,長安為罷市,市民相率遮邀宰相哭訴,盧杞疾驅而過。韋楨懼,乃請錢不及百緡、粟麥不及五十斛者免,而所獲裁二百萬緡。淮南節度使陳少游增其本道稅錢,每緡二百,因詔天下皆增之。



 自太宗時置義倉及常平倉以備兇荒,高宗以後,稍假義倉以給他費,至神龍中略盡。玄宗即位,復置之。其後第五琦請天下常平倉皆置庫,以畜本錢。至是趙贊又言:「自軍興,常平倉廢垂三十年,兇荒潰散,餧死相食,不可勝紀。陛下即位,京城兩市置常平官,雖頻年少雨,米不騰貴,可推而廣之,宜兼儲布帛。請於兩都、江陵、成都、楊、汴、蘇、洪置常平輕重本錢,上至百萬緡,下至十萬,積米、粟、布、帛、絲、麻、貴則下價而出之,賤則加估而收之。諸道津會置吏,閱商賈錢,每緡稅二十,竹、木、茶、漆稅十之一,以贍常平本錢。」德宗納其策。屬軍用迫蹴,亦隨而耗竭,不能備常平之積。是時,諸道討賊,兵在外者,度支給出界糧。每軍以臺省官一人為糧料使,主供億。士卒出境,則給酒肉。一卒出境,兼三人之費。將士利之,逾境而屯。趙贊復請稅間架,算除陌。其法:屋二架為間,上間錢二千,中間一千,下間五百;匿一間,杖六十,告者賞錢五萬。除陌法:公私貿易,千錢舊算二十,加為五十;物兩相易者,約直為率。而民益愁怨。及涇原兵反,大言虖長安市中曰:「不奪爾商戶僦質,不稅爾間架、除陌矣。」於是間架、除陌、竹、木、茶、漆、鐵之稅皆罷。



 硃泚平,天下戶口三耗其二。貞元四年,詔天下兩稅審等第高下,三年一定戶。自初定兩稅,貨重錢輕,乃計錢而輸綾絹。既而物價愈下,所納愈多,絹匹為錢三千二百,其後一匹為錢一千六百,輸一者過二,雖賦不增舊,而民愈困矣。度支以稅物頒諸司,皆增本價為虛估給之,而繆以濫惡督州縣剝價,謂之折納。復有「進奉」、「宣索」之名,改科役曰「召雇」,率配曰「和市」,以巧避微文,比大歷之數再倍。又癘疫水旱,戶口減耗,刺史析戶,張虛數以寬責。逃死闕稅,取於居者,一室空而四鄰亦盡。戶版不緝,無浮游之禁,州縣行小惠以傾誘鄰境,新收者優假之,唯安居不遷之民,賦役日重。帝以問宰相陸贄,贄上疏請厘革其甚害者,大略有六:



 其一曰:



 國家賦役之法,曰租、曰調、曰庸。其取法遠,其斂財均,其域人固。有田則有租,有家則有調,有身則有庸,天下法制均壹,雖轉徙莫容其奸,故人無搖心。天寶之季,海內波蕩,版圖隳於避地,賦法壞於奉軍。賦役舊法,行之百年,人以為便。兵興,供億不常,誅求隳制,此時弊,非法弊也。時有弊而未理,法無弊而已更。兩稅新制,竭耗編,日日滋甚。陛下初即位,宜損上益下,嗇用節財,而摘郡邑,驗簿書,州取大歷中一年科率多者為兩稅定法,此總無名之暴賦而立常規也。夫財之所生,必因人力。兩稅以資產為宗,不以丁身為本,資產少者稅輕,多者稅重。不知有藏於襟懷囊篋,物貴而人莫窺者;有場圃、囷倉,直輕而眾以為富者;有流通蕃息之貨,數寡而日收其贏者;有廬舍器用,價高而終歲利寡者。計估算緡,失平長偽,挾輕費轉徙者脫徭稅,敦本業者困斂求。此誘之為奸,驅之避役也。今徭賦輕重相百,而以舊為準,重處流亡益多,輕處歸附益眾。有流亡則攤出,已重者愈重;有歸附則散出,已輕者愈輕。人嬰其弊。願詔有司與宰相量年支,有不急者罷之,廣費者節之。軍興加稅,諸道權宜所增,皆可停。稅物估價,宜視月平,至京與色樣符者,不得虛稱折估。有濫惡,罪官吏,勿督百姓。每道以知兩稅判官一人與度支參計戶數,量土地沃瘠、物產多少為二等,州等下者配錢少,高者配錢多。不變法而逋逃漸息矣。



 其二曰:



 播殖非力不成,故先王定賦以布、麻、繒、纊、百穀,勉人功也。又懼物失貴賤之平,交易難準,乃定貨泉以節輕重。蓋為國之利權,守之在官,不以任下。然則穀帛,人所為也;錢貨,官所為也。人所為者,租稅取焉;官所為者,賦斂舍焉。國朝著令,租出谷,庸出絹,調出繒、纊、布、麻,曷嘗禁人鑄錢而以錢為賦?今兩稅效算緡之末法,估資產為差,以錢穀定稅,折供雜物,歲目頗殊。所供非所業,所業非所供,增價以市所無,減價以貨所有,耕織之力有限,而物價貴賤無常。初定兩稅,萬錢為絹三匹,價貴而數不多。及給軍裝,計數不計價,此稅少國用不充也。近者萬錢為絹六匹,價賤而數加。計口蠶織不殊,而所輸倍,此供稅多人力不及也。宜令有司覆初定兩稅之歲絹、布定估,為布帛之數,復庸、調舊制,隨土所宜,各脩家技。物甚賤,所出不加;物甚貴,所入不減。且經費所資,在錢者獨月俸、資課,以錢數多少給布,廣鑄而禁用銅器,則錢不乏。有糴鹽以入直,榷酒以納資,何慮無所給哉!



 其三曰:



 廉使奏吏之能者有四科,一曰戶口增加,二曰田野墾闢,三曰稅錢長數,四曰率辦先期。夫貴戶口增加,詭情以誘奸浮,苛法以析親族,所誘者將議薄征則遽散,所析者不勝重稅而亡,有州縣破傷之病;貴田野墾闢,率民殖荒田,限年免租,新畝雖闢,舊畬蕪矣,人以免租年滿,復為污萊,有稼穡不增之病;貴稅錢長數,重困疲羸,捶骨瀝髓,茍媚聚斂之司,有不恤人之病;貴率辦先期,作威殘人,絲不容織,粟不暇舂,貧者奔迸,有不恕物之病:四病繇考核不切事情之過。驗之以實,則租賦所加,固有受其損者,此州若增客戶,彼郡必減居人。增處邀賞而稅數加,減處懼罪而稅數不降。國家設考課之法,非欲崇聚斂也。宜命有司詳考課績,州稅有定,徭役有等,覆實然後報戶部。若人益阜實,稅額有餘,據戶均減十三為上課,減二次之,減一又次之。若流亡多,加稅見戶者,殿亦如之。民納租以去歲輸數為常,罷據額所率者。增闢勿益租,廢耕不降數。定戶之際,視雜產以校之。田既有常租,則不宜復入兩稅。如此,不督課而人人樂耕矣。



 其四曰:



 明君不厚所資而害所養,故先人事而借其暇力,家給然後斂餘財。今督收迫促,蠶事方興而輸縑,農功未艾而斂穀。有者急賣而耗半直,無者求假費倍。定兩稅之初,期約未詳,屬征役多故,率先限以收。宜定稅期,隨風俗時候,務於紓人。



 其五曰:



 頃師旅亟興,官司所儲,唯給軍食,兇荒不遑賑救。人小乏則取息利,大乏則鬻田廬。劍穫始畢,執契行貸,饑歲室家相棄,乞為奴僕,猶莫之售,或縊死道途。天災流行,四方代有。稅茶錢積戶部者,宜計諸道戶口均之。穀麥熟則平糴,亦以義倉為名,主以巡院。時稔傷農,則優價廣糴,穀貴而止;小歉則借貸。循環斂散,使聚穀幸災者無以牟大利。



 其六曰:



 古者百畝地號一夫,蓋一夫授田不得過百畝,欲使人不廢業,田無曠耕。今富者萬畝,貧者無容足之居,依托強家,為其私屬,終歲服勞,常患不充。有田之家坐食租稅,京畿田畝稅五升,而私家收租畝一石,官取一,私取十,穡者安得足食?宜為占田條限,裁租價,損有餘,優不足,此安富恤窮之善經,不可舍也。



 贄言雖切,以讒逐,事無施行者。



 十二年,河南尹齊抗復論其弊,以為:「軍興,國用稍廣,隨要而稅,吏擾人勞。陛下變為兩稅,督納有時,貪暴無容其奸。二十年間,府庫充牛刃。但定稅之初,錢輕貨重,故陛下以錢為稅。今錢重貨輕,若更為稅名,以就其輕,其利有六:吏絕其奸,一也;人用不擾,二也;靜而獲利,三也;用不乏錢,四也;不勞而易知,五也;農桑自勸,六也。百姓本出布帛,而稅反配錢,至輸時復取布帛,更為三估計折,州縣升降成奸。若直定布帛,無估可折。蓋以錢為稅,則人力竭而有司不之覺。今兩稅出於農人,農人所有,唯布帛而已。用布帛處多,用錢處少,又有鼓鑄以助國計,何必取於農人哉?」疏入,亦不報。



 初,德宗居奉天,儲畜空窘,嘗遣卒視賊,以苦寒乞襦褲,帝不能致,剔親王帶金而鬻之。硃泚既平,於是帝屬意聚斂,常賦之外,進奉不息。劍南西川節度使韋皋有「日進」,江西觀察使李兼有「月進」,淮南節度使杜亞、宣歙觀察使劉贊、鎮海節度使王緯、李錡皆徼射恩澤,以常賦入貢,名為「羨餘」。至代易又有「進奉」。當是時,戶部錢物,所在州府及巡院皆得擅留,或矯密旨加斂,謫官吏、刻祿稟,增稅通津、死人及蔬果。凡代易進奉,取於稅入,十獻二三,無敢問者。常州刺史裴肅鬻薪炭案紙為進奉,得遷浙東觀察使。刺史進奉,自肅始也。劉贊卒於宣州,其判官嚴綬傾軍府為進奉,召為刑部員外郎。判官進奉,自綬始也。自裴延齡用事,益為天子積私財,而生民重困。延齡死,而人相賀。



 是時,宮中取物於市,以中官為宮市使。兩市置「白望」數十百人,以鹽估敝衣、絹帛,尺寸分裂酬其直。又索進奉門戶及腳價錢,有齎物入市而空歸者。每中官出,沽漿賣餅之家皆徹肆塞門。諫官御史數上疏諫,不聽,人不堪其弊。戶部侍郎蘇弁言:「京師游手數千萬家,無生業者仰宮市以活,奈何罷?」帝悅,以為然。京兆尹韋湊奏:「小人因宮市為奸,真偽難辨,宜下府縣供送。」帝許之。中官言百姓賴宮市以養者也,湊反得罪。



 順宗即位,乃罷宮市使及鹽鐵使月進;憲宗又罷除官受代進奉及諸道兩稅外榷率,分天下之賦以為三:一曰上供,二曰送使,三曰留州。宰相裴垍又令諸道節度、觀察調費取於所治州,不足則取於屬州,而屬州送使之餘與其上供者,皆輸度支。是時,因德宗府庫之積,頗約費用,天子身服澣濯。及劉闢、李錡既平,訾藏皆入內庫。山南東道節度使于頔、河東節度使王鍔進獻甚厚,翰林學士李絳嘗諫曰:「方鎮進獻,因緣為奸,以侵百姓,非聖政所宜。」帝喟然曰:「誠知非至德事,然兩河中夏貢賦之地,朝覲久廢,河、湟陷沒,烽候列於郊甸。方刷祖宗之恥,不忍重斂於人也。」然獨不知進獻之取於人者重矣。



 及討淮西,判度支楊於陵坐饋餫不繼貶,以司農卿皇甫鎛代之,由是益為刻剝。司農卿王遂、京兆尹李翛號能聚斂,乃以為宣歙、浙西觀察使,予之富饒之地,以辦財賦。鹽鐵使王播言:「劉晏領使時,自按租庸,然後知州縣錢谷利病虛實。」乃以副使程異巡江、淮,核州府上供錢穀。異至江、淮,得錢百八十五萬貫。其年,遂代播為鹽鐵使。是時,河北兵討王承宗,於是募人入粟河北、淮西者,自千斛以上皆授以官。度支鹽鐵與諸道貢獻尤甚,號「助軍錢」。及賊平,則有賀禮及助賞設物。群臣上尊號,又有獻賀物。



 穆宗即位,一切罷之,兩稅外加率一錢者,以枉法贓論。然自在籓邸時,習見用兵之弊,以謂戎臣武卒,法當姑息。及即位,自神策諸軍,非時賞賜,不可勝紀。已而幽州兵囚張弘靖,鎮州殺田弘正,兩鎮用兵,置南北供軍院。而行營軍十五萬,不能亢兩鎮萬餘之眾。而饋運不能給,帛粟未至而諸軍或強奪於道。



 蓋自建中定兩稅,而物輕錢重,民以為患,至是四十年。當時為絹二匹半者為八匹,大率加三倍。豪家大商,積錢以逐輕重,故農人日困,末業日增。帝亦以貨輕錢重,民困而用不充,詔百官議革其弊。而議者多請重挾銅之律。戶部尚書楊於陵曰:「王者制錢以權百貨,貿遷有無,通變不倦,使物無甚貴甚賤,其術非它,在上而已。何則?上之所重,人必從之。古者權之於上,今索之於下;昔散之四方,今藏之公府;昔廣鑄以資用,今減爐以廢功;昔行之於中原,今洩之於邊裔。又有閭井送終之唅,商賈貸舉之積,江湖壓覆之耗,則錢焉得不重,貨焉得不輕?開元中,天下鑄錢七十餘爐,歲盈百萬,今才十數爐,歲入十五萬而已。大歷以前,淄青、太原、魏博雜鉛鐵以通時用,嶺南雜以金、銀、丹砂、象齒,今一用泉貨,故錢不足。今宜使天下兩稅、榷酒、鹽利、上供及留州、送使錢,悉輸以布帛穀粟,則人寬於所求,然後出內府之積,收市廛之滯,廣山鑄之數,限邊裔之出,禁私家之積,則貨日重而錢日輕矣。」宰相善其議。由是兩稅、上供、留州,皆易以布帛、絲纊,租、庸、課、調不計錢而納布帛,唯鹽酒本以榷率計錢,與兩稅異,不可去錢。



 文宗大和九年,以天下回殘錢置常平義倉本錢,歲增市之。非遇水旱不增者,判官罰俸,書下考;州縣假借,以枉法論。



 文宗嘗召鹽倉御史崔虞問太倉粟數,對曰:「有粟二百五十萬石。」帝曰:「今歲費廣而所畜寡,奈何?」乃詔出使郎官、御史督察州縣壅遏錢穀者。時豪民侵噬產業不移戶,州縣不敢徭役,而征稅皆出下貧。至於依富室為奴客,役罰峻於州縣。長吏歲輒遣吏巡覆田稅,民苦其擾。



 武宗即位,廢浮圖法,天下毀寺四千六百、招提蘭若四萬,籍僧尼為民二十六萬五千人,奴婢十五萬人,田數千萬頃,大秦穆護、襖二千餘人。上都、東都每街留寺二,每寺僧三十人,諸道留僧以三等,不過二十人。腴田鬻錢送戶部,中下田給寺家奴婢丁壯者為兩稅戶,人十畝。以僧尼既盡,兩京悲田養病坊,給寺田十頃,諸州七頃,主以耆壽。



 自會昌末,置備邊庫,收度支、戶部、鹽鐵錢物。宣宗更號延資庫。初以度支郎中判之,至是以屬宰相,其任益重。戶部歲送錢帛二十萬,度支鹽鐵送者三十萬,諸道進奉助軍錢皆輸焉。



 懿宗時,雲南蠻數內寇,徙兵戍嶺南。淮北大水,征賦不能辦,人人思亂。及龐勛反,附者六七萬。自關東至海大旱,冬蔬皆盡,貧者以蓬子為面,槐葉為齏。乾符初,大水,山東饑。中官田令孜為神策中尉,怙權用事,督賦益急。王仙芝、黃巢等起,天下遂亂,公私困竭。昭宗在鳳翔,為梁兵所圍,城中人相食,父食其子,而天子食粥,六宮及宗室多餓死。其窮至於如此,遂以亡。



 初,乾元末,天下上計百六十九州,戶百九十三萬三千一百二十四,不課者百一十七萬四千五百九十二;口千六百九十九萬三百八十六,不課者千四百六十一萬九千五百八十七。減天寶戶五百九十八萬二千五百八十四,口三千五百九十二萬八千七百二十三。



 元和中,供歲賦者,浙西、浙東、宣歙、淮南、江西、鄂岳、福建、湖南八道,戶百四十四萬,比天寶才四之一。兵食於官者八十三萬,加天寶三之一,通以二戶養一兵。京西北、河北以屯兵廣,無上供。至長慶,戶三百三十五萬,而兵九十九萬,率三戶以奉一兵。至武宗即位,戶二百一十一萬四千九百六十。會昌末,戶增至四百九十五萬五千一百五十一。



 宣宗既復河、湟,天下兩稅、榷酒茶鹽錢,歲入九百二十二萬緡,歲之常費率少三百餘萬,有司遠取後年乃濟。及群盜起,諸鎮不復上計云。



\end{pinyinscope}