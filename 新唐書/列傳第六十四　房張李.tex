\article{列傳第六十四 房張李}

\begin{pinyinscope}

 房琯,字次律,河南河南人。父融,武后時,以正諫大夫同鳳閣鸞臺平章事;神龍元年行陰陽,陰陽太極」。另一是太極本於無極,含有「有生於,貶死高州。琯少好學,風度沈整,以廕補弘文生。與呂向偕隱陸渾山,十年不諧際人事。開元中,作《封禪書》,說宰相張說,說奇之,奏為校書郎。舉任縣令科,授盧氏令。拜監察御史,坐訊獄非是,貶睦州司戶參軍。復為縣,所至上德化,興長利,以治最顯。



 天寶五載,試給事中,封漳南縣男。時玄宗有逸志,數巡幸,廣溫泉為華清宮,環宮所置百司區署。以琯資機算,詔總經度驪山,疏巖剔藪,為天子游觀。未畢,坐善李適之、韋堅,斥為宜春太守。歷瑯邪、鄴、扶風三郡,頻遷憲部侍郎。十五載,帝狩蜀,琯馳至普安上謁,帝喜甚,即拜文部尚書、同中書門下平章事,從至成都,賜一子官。



 俄與韋見素、崔渙奉冊靈武,見肅宗,具言上皇所以傳付意,因道當時利病,箝索虜情,辭吐華暢,帝為改容。琯既有重名。帝傾意待之,機務一二與琯參決,諸將相莫敢望。於是,第五琦言財利幸,為江淮租庸使。琯諫曰:「往楊國忠聚斂,產怨天下。陛下即位,人未見德,今又寵琦,是一國忠死,一國忠生,無以示遠方。」帝曰:「六軍之命方急,無財則散。卿惡琦可也,何所取財?」琯不得對。北海太守賀蘭進明自河南至,詔攝御史大夫、嶺南節度使,入謝,帝曰:「朕語琯除正大夫,何為攝邪?」進明銜之,因曰:「陛下知晉亂乎?惟以尚虛名,任王衍為宰相,基祖浮華,不事天下事,故至於敗。方唐中興,當用實才,而琯性疏闊,大言無當,非宰相器。陛下待之厚,然孰肯為陛下用乎?」帝曰:「何哉?」對曰:「陛下頃為皇太子,太子出曰撫軍,入曰監國,而琯為聖皇建遣諸王為都統節度,乃謂陛下為元子而付以朔方、河東、河北空虛之地,永王、豐王乃統四節度。此於聖皇似忠,於陛下非忠也。琯意諸子一得天下,身不失恩,又多樹私黨,以副戎權,推此而言,豈肯盡誠於陛下乎?」帝入其語,始惡琯。以進明為御史大夫、河南節度使。



 會琯請自將平賊,帝猶倚以成功,乃詔琯持節招討西京、防禦蒲潼兩關兵馬節度等使,得自擇參佐。乃以兵部尚書王思禮、御史中丞鄧景山為副,戶部侍郎李揖為行軍司馬,中丞宋若思、起居郎知制誥賈至、右司郎中魏少游為判官,給事中劉秩為參謀。琯分三軍趨京師:楊希文將南軍,自宜壽入;劉悊將中軍,自武功入;李光進將北軍,自奉天入。琯身中軍先鋒。十月庚子,次便橋。辛丑,中軍、北軍遇賊陳濤斜,戰不利。琯欲持重有所伺,中人邢延恩促戰,故敗,士死麻葦。癸卯,率南軍復戰,遂大敗,希文、悊皆降賊。初,琯用春秋時戰法,以車二千乘繚營,騎步夾之。既戰,賊乘風噪,牛悉髀慄,賊投芻而火之,人畜焚燒,殺卒四萬,血丹野,殘眾才數千,不能軍。琯還走行在,見帝,肉袒請罪,帝宥之,使裒夷散,復圖進取。琯雅自負,以天下為己任,然用兵本非所長。其佐李揖、劉秩等皆儒生,未嘗更軍旅,琯每詫曰:「彼曳落河雖多,能當我劉秩乎?」帝雖恨琯喪師,而眷任未衰。



 崔圓自蜀來,最後見帝,琯謂帝不見省,易之。圓以金畀李輔國,不淹日被寵,遂怨琯。琯數稱疾不入。會御史大夫顏真卿劾奏諫議大夫李何忌不孝,琯素善何忌,不欲以惡名錮之,托被酒入朝,貶西平郡司馬。琴工董廷蘭出入琯所,琯暱之。廷蘭藉琯勢,數招賕謝,為有司劾治,琯訴於帝,帝因震怒,叱遣之,琯惶恐就第。罷為太子少師。從帝還都,封清河郡公。琯之廢,朝臣多言琯謀包文武,可復用,雖琯亦自謂當柄任,為天子立功。善琯者暴其言於朝。琯方日引劉秩、嚴武與宴語,移病自如。帝以琯虛言浮誕,內鞅鞅,挾黨背公,非大臣體。乾元元年,出琯為邠州刺史,逐秩、武等,因下詔陳其比周狀,喻敕中外。始,邠以武將領刺史,故綱目廢弛,即治府為營,吏攘民居相淆歡。琯至,一切革之,人以便安,政聲流聞。召拜太子賓客,遷禮部尚書,為晉、漢二州刺史。寶應二年,召拜刑部尚書,道病卒,贈太尉。



 琯有遠器,好談老子、浮屠法,喜賓客,高談有餘,而不切事。時天下多故,急於謀略攻取,帝以吏事繩下,而琯為相,遽欲從容鎮靜以輔治之,又知人不明,以取敗撓,故功名隳損云。



 贊曰:唐名儒多言琯德器,有王佐材,而史載行事,亦少貶矣。一舉喪師,訖不復振。原琯以忠誼自奮,片言悟主而取宰相,必有以過人者,用違所長,遂無成功。然盛名之下,為難居矣。夫名盛則責望備,實不副則訾咎深。使琯遭時承平,從容帷幄,不失為名宰。而倉卒濟難,事敗隙生,陷於浮虛比周之罪,名之為累也,戒哉!



 子孺復,幼頗能屬文,然狂縱不法。淮南節度使陳少游奏置幕府。多招術家言己三十當得宰相,以熏權近,希進取。後闢浙西韓滉府。兄宗偃喪自嶺外還,孺復不出臨吊。與妻鄭不相中,慈姆為言,乃具棺召家人生斂之;鄭方乳,促上道,鄭死於行。又娶崔昭女,崔悍媢,殺二侍兒,私瘞之。觀察使以聞,貶連州司馬,聽崔去。既又與崔通,請復合,詔許。未幾復離。終容州刺史。



 琯孫啟,以廕補鳳翔參軍事,累調萬年令,素贅附王叔文。貞元末,叔文用事,除容管經略使,陰許以荊南帥節。啟至荊湖,宿留不肯進,會叔文與韋執誼內忿爭,不果拜。俄而皇太子監國,啟惶駭就鎮。凡九年,改桂管觀察使。州邸以賂請有司飛驛送詔,既而憲宗自遣宦人持詔賜啟,啟畏使者邀重餉,即曰:「先五日已得詔。」使者紿請視,因馳歸以聞,貶太僕少卿。啟自陳獻使者南口十五,帝怒,殺宦人,貶啟虔州長史,死。始詔五管、福建、黔中道不得以口饋遺、博易,罷臘口等使。



 琯族孫式,擢進士第,累遷忠州刺史。韋皋表為雲南安撫副使、蜀州刺史。皋卒,劉闢反,式留不得行。賊平,高崇文保貸之,言諸朝,除吏部郎中。時河朔諸將劉濟、張茂昭等更相劾奏,帝欲和之,拜式給事中,使河北,還奏如旨。遷陜虢觀察使,改河南尹。會討王承宗鎮州,索餉車四千乘,民不能具。式建言:「歲兇人勞,不任調發。」又御史元稹亦言:「賊未禽,而河南民先困。」詔可,都鄙安之。改宣歙觀察使。卒,贈左散騎常侍,謚曰傾。吏部郎中韋乾度曰:「始式刺蜀州,劉闢構難,即謂闢曰:『向夢公為上相,儀衛甚盛,幸無相忘。』闢喜,以為祥。後闢發兵署牒,首曰闢,副曰式,參謀曰符載。大節已虧,不宜得謚。」博士李虞仲曰:「始闢反,為其用者皆救死其頸,可盡被惡名乎?如式,不能去,又不能死,可謂求生害仁者也。闢走西山,召所疑畏者盡殺之,式在其間,會救得免。而曰大節已虧,近於溢言。」謚乃定。



 張鎬,字從周,博州人。儀狀瑰偉,有大志,視經史猶漁獵,然好王霸大略。少事吳兢,兢器之。游京師,未知名,率嗜酒鼓琴自娛。人或邀之,杖策往,醉即返,不及世務。



 天寶末,楊國忠執政,求天下士為己重,聞鎬才,薦之。釋褐衣,拜左拾遺,歷侍御史。玄宗西狩,鎬徒步扈從。俄遣詣肅宗所。數論事,擢諫議大夫,尋拜中書侍郎、同中書門下平章事。時引內浮屠數百居禁中,號「內道場」,諷唄外聞,鎬諫曰:「天子之福,要在養人,以一函宇,美風化,未聞區區佛法而致太平。願陛下以無為為心,不以小乘撓聖慮。」帝然之。尋詔兼河南節度使,都統淮南諸軍事。賊圍宋州,張巡告急,鎬倍道進,檄濠州刺史閭丘曉趣救。曉愎撓,逗留不肯進,比鎬至淮口,而巡已陷。鎬怒,杖殺曉。帝還京師,封南陽郡公,詔以本軍鎮汴州,捕平殘寇。史思明提範陽獻順款,鎬揣其偽,密奏曰:「思明勢窮而服,包藏不測,可以計取,難以義招,不宜以威權假之。」又言:「滑州防禦使許叔冀狡獪,臨難必變,宜追還宿衛。」書入不省。時宦官絡繹出鎬境,未嘗降情結納。自範陽、滑州使還者,皆盛言思明、叔冀忠,而毀鎬無經略才。帝以鎬不切事機,遂罷宰相,授荊州大都督府長史。思明、叔冀後果叛,如鎬言。召拜太子賓客、左散騎常侍。坐市嗣岐王珍第,貶辰州司戶參軍。代宗初,起為撫州刺史,遷洪州觀察使,更封平原郡公。袁晁寇東境,江介震騷,鎬遣兵屯上饒,斬首二千級。又襲舒城賊楊昭,梟之。沉千載者,新安大豪,連結椎剽,州縣不能禽,鎬遣別將盡殄其眾。改江南西道觀察使,卒。



 鎬起布衣,二期至宰相。居身廉,不殖貲產。善待士,性簡重,論議有體。在位雖淺,而天下之人推為舊德云。



 李泌,字長源,魏八柱國弼六世孫,徙居京兆。七歲知為文。玄宗開元十六年,悉召能言佛、道、孔子者,相答難禁中。有員俶者,九歲升坐,詞辯注射,坐人皆屈。帝異之,曰:「半千孫,固當然。」因問:「童子豈有類若者?」俶跪奏:「臣舅子李泌。」帝即馳召之。泌既至,帝方與燕國公張說觀弈,因使說試其能。說請賦「方圓動靜」,泌逡巡曰:「願聞其略。」說因曰:「方若棋局,圓若棋子,動若棋生,靜若棋死。」泌即答曰:「方若行義,圓若用智,動若騁材,靜若得意。」說因賀帝得奇童。帝大悅曰:「是子精神,要大於身。」賜束帛,敕其家曰:「善視養之。」張九齡尤所獎愛,常引至臥內。九齡與嚴挺之、蕭誠善,挺之惡誠佞,勸九齡謝絕之。九齡忽獨念曰:「嚴太苦勁,然蕭軟美可喜。」方命左右召蕭,泌在旁,帥爾曰:「公起布衣,以直道至宰相,而喜軟美者乎?」九齡驚,改容謝之,因呼「小友」。及長,博學,善治《易》,常游嵩、華、終南間,慕神仙不死術。天寶中,詣闕獻《復明堂九鼎議》,帝憶其早惠,召講《老子》,有法,得待詔翰林,仍供奉東宮,皇太子遇之厚。嘗賦詩譏誚楊國忠、安祿山等,國忠疾之,詔斥置蘄春郡。



 肅宗即位靈武,物色求訪,會泌亦自至。已謁見,陳天下所以成敗事,帝悅,欲授以官,固辭,願以客從。入議國事,出陪輿輦,眾指曰:「著黃者聖人,著白者山人。」帝聞,因賜金紫,拜元帥廣平王行軍司馬。帝嘗曰「卿侍上皇,中為朕師,今下判廣平行軍,朕父子資卿道義」云。始,軍中謀帥,皆屬建寧王,泌密白帝曰:「建寧王誠賢,然廣平塚嗣,有君人量,豈使為吳太伯乎?」帝曰:「廣平為太子,何假元帥?」泌曰:「使元帥有功,陛下不以為儲副,得耶?太子從曰撫軍,守曰監國,今元帥乃撫軍也。」帝從之。



 初,帝在東宮,李林甫數構譖,勢危甚,及即位,怨之,欲掘塚焚骨。泌以天子而念宿嫌,示天下不廣,使脅從之徒得釋言於賊。帝不悅,曰:「往事卿忘之乎?」對曰:「臣念不在此。上皇有天下五十年,一旦失意,南方氣候惡,且春秋高,聞陛下錄故怨,將內慚不懌,萬有一感疾,是陛下以天下之廣不能安親也。」帝感悟,抱泌頸以泣曰:「朕不及此。」因從容問破賊期,對曰;「賊掠金帛子女,悉送範陽,有茍得心,渠能定中國邪?華人為之用者,獨周摯、高尚等數人,餘皆脅制偷合,至天下大計,非所知也。不出二年,無寇矣,陛下無欲速。夫王者之師,當務萬全,圖久安,使無後害。今詔李光弼守太原,出井陘,郭子儀取馮翊,入河東,則史思明、張忠志不敢離範陽、常山,安守忠、田乾真不敢離長安,是以三地禁其四將也。隨祿山者,獨阿史那承慶耳。使子儀毋取華,令賊得通關中,則北守範陽,西救長安,奔命數千里,其精卒勁騎,不逾年而弊。我常以逸待勞,來避其鋒,去翦其疲,以所征之兵會撫風,與太原、朔方軍互擊之。徐命建寧王為範陽節度大使,北並塞與光弼相掎角,以取範陽。賊失巢窟,當死河南諸將手。」帝然之。會西方兵大集,帝欲速得長安,曰:「今戰必勝,攻必取,何暇千里先事範陽乎?」泌曰:「必得兩京,則賊再強,我再困。且我所恃者,磧西突騎、西北諸戎耳。若先取京師,期必在春,關東早熱,馬且病,士皆思歸,不可以戰。賊得休士養徒,必復來南。此危道也。」帝不聽。



 二京平,帝奉迎上皇,自請歸東宮以遂子道。泌曰:「上皇不來矣。人臣尚七十而傳,況欲勞上皇以天下事乎。」帝曰:「奈何?」泌乃為群臣通奏,具言天子思戀晨昏,請促還以就孝養。上皇得初奏,答曰:「當與我劍南一道自奉,不復東矣。」帝甚憂。及再奏至,喜曰:「吾方得為天子父!」遂下誥戒行。



 崔圓、李輔國以泌親信,疾之。泌畏禍,願隱衡山。有詔給三品祿,賜隱士服,為治室廬。泌嘗取松樛枝以隱背,名曰「養和」,後得如龍形者,因以獻帝,四方爭效之。代宗立,召至,舍蓬萊殿書閣。初,泌無妻,不食肉,帝乃賜光福里第,強詔食肉,為娶朔方故留後李甥,昏日,敕北軍供帳。元載惡不附己,因江西觀察使魏少游請僚佐,載稱泌才,以試秘書少監充判官。載誅,帝召還。復為常袞所忌,出為楚州刺史,辭不行,帝亦留之。會澧州缺,袞盛言南方凋瘵,請輟泌治之,乃授澧、朗、峽團練使,徙杭州刺史,皆有風績。



 德宗在奉天,召赴行在,授左散騎常侍。時李懷光叛,歲又蝗旱,議者欲赦懷光。帝博問群臣,泌破一桐葉附使以進,曰:「陛下與懷光,君臣之分不可復合,如此葉矣。」由是不赦。始,硃泚亂,帝約吐蕃赴援,賂以安西、北庭。既而渾瑊與賊戰咸陽,泚大敗,吐蕃以師追北不甚力,因大掠武功而歸。京師平,來請如約。帝業許,欲遂與之。泌曰:「安西、北庭,控制西域五十七國及十姓突厥,皆悍兵處,以分吐蕃勢,使不得並兵東侵。今與其地,則關中危矣。且吐蕃向持兩端不戰,又掠我武功,乃賊也,奈何與之?」遂止。



 貞元元年,拜陜虢觀察使。泌始鑿山開車道至三門,以便饟漕。以勞,進檢校禮部尚書。淮西兵防秋屯鄜州,已而四千人亡歸,或曰吳少誠密招之。既入境,泌邀險悉擊殺之。三年,拜中書侍郎、同中書門下平章事,累封鄴縣侯。初,張延賞減天下吏員,人情愁怨,至流離死道路者。泌請復之,帝未從,因問:「今戶口減承平時幾何?」曰:「三之二。」帝曰:「人既雕耗,員何可復?」泌曰:「不然。戶口雖耗,而事多承平十倍。陛下欲省州縣則可,而吏員不可減。今州或參軍署券,縣佐史判案。所謂省官者,去其冗員,非常員也。」帝曰:「若何為冗員?」對曰:「州參軍無職事及兼、試額內官者。兼、試,自至德以來有之,比正員三之一,可悉罷。」帝乃許復吏員,而罷冗官。泌又條奏:「中朝官常侍、賓客十員,其六員可罷;左右贊善三十員,其二十員可罷。如舊制,諸王未出閤,官屬皆不除。而所收科奉,乃多於減員矣。」帝悅。是時,州刺史月奉至千緡,方鎮所取無藝,而京官祿寡薄,自方鎮入八座,至謂罷權。薛邕由左丞貶歙州刺史,家人恨降之晚。崔祐甫任吏部員外,求為洪州別駕。使府賓佐有所忤者,薦為郎官。其當遷臺閣者,皆以不赴取罪去。泌以為外太重,內太輕,乃請隨官閑劇,普增其奉,時以為宜。而竇參多沮亂其事,不能悉如所請。泌又白罷拾遺、補闕,帝雖不從,然因是不除諫官,唯用韓皋、歸登。泌因收其公廨錢,令二人寓食中書舍人署。凡三年,始以韋綬、梁肅為左右補闕。



 太子妃蕭母,郜國公主也,坐蠱媚,幽禁中,帝怒,責太子,太子不知所對。泌入,帝數稱舒王賢,泌揣帝有廢立意,因曰:「陛下有一子而疑之,乃欲立弟之子,臣不敢以古事爭。且十宅諸叔,陛下奉之若何?」帝赫然曰:「卿何知舒王非朕子?」對曰:「陛下昔為臣言之。陛下有嫡子以為疑,弟之子敢自信於陛下乎?」帝曰:「卿違朕意,不顧家族邪?」對曰:「臣衰老,位宰相,以諫而誅,分也。使太子廢,它日陛下悔曰『我惟一子殺之,泌不吾諫,吾亦殺爾子』,則臣絕祀矣。雖有兄弟子,非所歆也。」即噫嗚流涕。因稱:「昔太宗詔:『太子不道,籓王窺伺者,兩廢之。』陛下疑東宮而稱舒王賢,得無窺伺乎?若太子得罪,請亦廢之而立皇孫,千秋萬歲後,天下猶陛下子孫有也。且郜國為其女妒忌,而蠱惑東宮,豈可以妻母累太子乎?」執爭數十,意益堅,帝寤,太子乃得安。



 初,興元後國用大屈,封物皆三損二。舊制,堂封歲三千六百縑,後才千二百。至是,帝使還舊封。於是李晟、馬燧、渾瑊各食實封,悉讓送泌,泌不納。時方鎮私獻於帝,歲凡五十萬緡,其後稍損至三十萬,帝以用度乏問泌,泌請:「天下供錢歲百萬給宮中,勸不受私獻。凡詔旨須索,即代兩稅,則方鎮可以行法,天下紓矣。」



 帝嘗從容言:「盧杞清介敢言,然少學,不能廣朕以古道,人皆指其奸而朕不覺也。」對曰:「陛下能覺杞之惡,安致建中禍邪?李揆和蕃,顏真卿使希烈,其害舊德多矣。又楊炎罪不至死,杞擠陷之而相關播。懷光立功,逼使其叛。此欺天也。」帝曰:「卿言誠有之。然楊炎視朕如三尺童子,有所論奏,可則退,不許則辭官,非特杞惡之也。且建中亂,卿亦知桑道茂語乎?乃命當然。」對曰:「夫命者,已然之言。主相造命,不當言命。言命,則不復賞善罰惡矣。桀曰:『我生不有命自天。』武王數紂曰:『謂己有天命。』君而言命,則桀、紂矣。」帝曰:「朕請不復言命。」俄加集賢殿、崇文館大學士,修國史。泌建言:學士加大,始中宗時,及張說為之,固辭,乃以學士知院事。至崔圓復為大學士,亦引泌為讓而止。



 帝以「前世上巳、九日,皆大宴集,而寒食多與上巳同時,欲以三月名節,自我為古,若何而可?」泌請:「廢正月晦,以二月朔為中和節,因賜大臣戚里尺,謂之裁度。民間以青囊盛百穀瓜果種相問遺,號為獻生子。里閭釀宜春酒,以祭勾芒神,祈豐年。百官進農書,以示務本。」帝悅,乃著令,與上巳、九日為三令節,中外皆賜緡錢燕會。



 四年八月,月蝕東壁,泌曰:「東壁,圖書府,大臣當有憂者。吾以宰相兼學士,當之矣。昔燕國公張說由是以亡,又可免乎?」明年果卒,年六十八,贈太子太傅。



 泌出入中禁,事四君,數為權幸所疾,常以智免。好縱橫大言,時時讜議,能寤移人主。然常持黃老鬼神說,故為人所譏切。初,肅宗重陰陽巫祝,擢王璵執政,大抵興造工役,輒牽禁忌俗說。而黎幹以左道位京兆尹,嘗使禁工駢珠刺繡為乘輿服,舉焚之以為禳禬。德宗素不為然,及嗣位,罷內道場,除巫祝。代宗將葬,帝號送承天門,而轀車行不中道,問其故,有司曰:「陛下本命在午,故避之。」帝泣曰:「安有枉靈駕以謀身利?」命直午而行。又宣政廊壞,太卜言:「孟冬魁岡,不可營繕。」帝曰:「《春秋》『啟塞從時』,何魁岡為?」亟詔葺之。及桑道茂城奉天事驗,始尚時日拘忌,因進用泌,泌亦自有所建明。獨柳玭稱,兩京復,泌謀居多,其功乃大於魯連、範蠡云。



 子繁。繁少才警,無行。泌始起陽城官諸朝,故城重德泌而親厚於繁。及疏裴延齡,既具槁,以繁可信,夜使繁書。已封,盡能誦憶,乃錄以示延齡。明日,延齡白帝曰:「城以疏示於朝。」即擿其條以自訴解。城奏入,帝怒,遂不省。泌與梁肅善,故繁師事肅。及卒,烝其室,士議喧丑,由是擯棄積年。後為太常博士,權德輿為卿,奏斥之,改河南府士曹參軍。累遷隋州刺史,罷歸,不得調。敬宗誕日,詔與兵部侍郎丁公著、太常少卿陸亙入殿中,抗老、佛誦論。改大理少卿、弘文館學士。諫官御史交章彈治,乃出為亳州刺史。州有劇賊,剽室廬、略財貲為患,它刺史不能禽,繁有機略,悉知賊巢藪所在,一旦出兵捕斬之。議者責繁不先啟觀察府,為擅興。詔御史舒元輿按之,元輿與繁素隙,盡翻其獄,以為濫殺不辜,有詔賜死,京兆人皆冤之。繁下獄,知且死,恐先人功業泯滅,從吏求廢紙,筆握著家傳十篇,傳於世。



 贊曰:泌之為人也,異哉!其謀事近忠,其輕去近高,其自全近智,卒而建上宰,近立功立名者。觀肅宗披榛莽,立朝廷,單言暫謀有所寤合,皆付以政。當此時,泌於獻納為不少,又佐代宗收兩京,獨不見錄,寧二主不以宰相器之邪?德宗晚好鬼神事,乃獲用,蓋以怪自置而為之助也。繁為家傳,言泌本居鬼谷,而史臣謬言好鬼道,以自解釋。既又著泌數與靈仙接,言舉不經,則知當時議者切而不與,有為而然。繁言多浮侈,不可信,掇其近實者著於傳。至勸帝先事範陽,明太子無罪,亦不可誣也。



\end{pinyinscope}