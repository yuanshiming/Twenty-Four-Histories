\article{列傳第一百二十七 文藝中}

\begin{pinyinscope}

 李適,字子至,京兆萬年人。舉進士,再調猗氏尉。武後修《三教珠英》書,以李嶠、張昌宗為使子思與孟子並提,後世稱思孟學派。該學派注重內心省察的,取文學士綴集,於是適與王無競、尹元凱、富嘉謨、宋之問、沈佺期、閻朝隱、劉允濟在選。書成,遷戶部員外郎,俄兼脩書學士。景龍初,又擢脩文館學士。睿宗時,待詔宣光閣,再選工部侍郎。卒,年四十九,贈貝州刺史。



 嘗夢與人論大衍數,寤而曰:「吾壽盡此乎!」敕其子曰:「霸陵原西視京師,吾樂之,可營墓,樹十松焉。」及未病時,衣冠往,寢石榻上,置所撰《九經要句》及素琴於前,士貴其達。



 子季卿,亦能文,舉明經、博學宏辭,調鄠尉。肅宗時,為中書舍人,以累貶通州別駕。代宗立,還為京兆少尹,復授舍人,進吏部侍郎、河南江淮宣慰使。振拔幽滯,號振職。大歷中,終右散騎常侍,遺命以布車一乘葬,贈禮部尚書。季卿在朝,薦進才髦,與人交,有終始,恢博君子也。



 初,中宗景龍二年,始於脩文館置大學士四員、學士八員,直學士十二員,象四時、八節、十二月。於是李嶠、宗楚客、趙彥昭、韋嗣立為大學士,適、劉憲、崔湜、鄭愔、盧藏用、李乂、岑羲、劉子玄為學士,薛稷、馬懷素、宋之問、武平一、杜審言、沈佺期、閻朝隱為直學士,又召徐堅、韋元旦、徐彥伯、劉允濟等滿員。其後被選者不一。凡天子饗會游豫,唯宰相及學士得從。春幸梨園,並渭水祓除,則賜細柳圈闢癘;夏宴蒲萄園,賜硃櫻;秋登慈恩浮圖,獻菊花酒稱壽;冬幸新豐,歷白鹿觀,上驪山,賜浴湯池,給香粉蘭澤,從行給翔麟馬,品官黃衣各一。帝有所感即賦詩,學士皆屬和。當時人所歆慕,然皆狎猥佻佞,忘君臣禮法,惟以文華取幸。若韋元旦、劉允濟、沈佺期、宋之問、閻朝隱等無它稱,附篇左云。



 韋元旦,京兆萬年人。祖澄,越王府記室,撰《女誡》傳於時。元旦擢進士第,補東阿尉,遷左臺監察御史。與張易之有姻屬,易之敗,貶感義尉。俄召為主客員外郎,遷中書舍人。舅陸頌妻,韋后弟也,故元旦憑以復進云。



 劉允濟,字允濟,河南鞏人,其先出沛國,齊彭城郡丞瓛六世孫。少孤,事母尤孝。工文辭,與王勃齊名。舉進士,補下邽尉,累遷著作佐郎。採魯哀公後十二世接戰國為《魯後春秋》獻之,遷左史,兼直弘文館。



 武后明堂成,奏賦述功德,手詔褒咨,除著作郎。為來俊臣飛構當死,以母老丐餘年,系獄,會赦免,貶大庾尉。復為著作佐郎,修國史。常曰:「史官善惡必書,使驕主賊臣懼,此權顧輕哉?而班生受金,陳壽求米,僕乃視如浮雲耳。」遷鳳閣舍人,坐二張暱狎,除青州長史,有清白稱,巡察使路敬潛言狀。以內憂去官。服除,召為修文館學士,既久斥,喜甚,與家人樂飲,數日卒。



 沈佺期,字雲卿,相州內黃人。及進士第,由協律郎累除給事中,考功受賕,劾未究,會張易之敗,遂長流驩州。稍遷臺州錄事參軍事。入計,得召見,拜起居郎兼修文館直學士。既侍宴,帝詔學士等舞《回波》,佺期為弄辭悅帝,還賜牙、緋。尋歷中書舍人、太子少詹事。開元初卒。弟全交、全宇,皆有才章而不逮佺期。



 宋之問,字延清,一名少連,汾州人。父令文,高宗時為東臺詳正學士。之問偉儀貌,雄於辯。甫冠,武后召與楊炯分直習藝館。累轉尚方監丞、左奉宸內供奉。武後游洛南龍門,詔從臣賦詩,左史東方蚪詩先成,後賜錦袍,之問俄頃獻,後覽之嗟賞,更奪袍以賜。



 於時張易之等烝暱寵甚,之問與閻朝隱、沈佺期、劉允濟傾心媚附,易之所賦諸篇,盡之問、朝隱所為,至為易之奉溺器。及敗,貶瀧州,朝隱崖州,並參軍事。之問逃歸洛陽,匿張仲之家。會武三思復用事,仲之與王同皎謀殺三思安王室,之問得其實,令兄子曇與冉祖雍上急變,因丐贖罪,由是擢鴻臚主簿,天下醜其行。



 景龍中,遷考功員外郎,諂事太平公主,故見用。及安樂公主權盛,復往諧結,故太平深疾之。中宗將用為中書舍人,太平發其知貢舉時賕餉狼藉,下遷汴州長史,未行,改越州長史。頗自力為政。窮歷剡溪山,置酒賦詩,流布京師,人人傳諷。



 睿宗立,以獪險盈惡詔流欽州。祖雍歷中書舍人、刑部侍郎。倡飲省中,為御史劾奏,貶蘄州刺史。至是,亦流嶺南,並賜死桂州。之問得詔震汗,東西步,不引決。祖雍請使者曰:「之問有妻子,幸聽訣。」使者許之,而之問荒悸不能處家事。祖雍怒曰:「與公俱負國家當死,奈何遲回邪?」乃飲食洗沐就死。祖雍,江夏王道宗甥,及進士第,有名於時。



 魏建安後迄江左,詩律屢變,至沈約、庾信,以音韻相婉附,屬對精密。及之問、沈佺期,又加靡麗,回忌聲病,約句準篇,如錦繡成文,學者宗之,號為「沈宋」。語曰「蘇李居前,沈宋比肩」,謂蘇武、李陵也。



 初,之問父令文,富文辭,且工書,有力絕人,世稱「三絕」。都下有牛善觸,人莫敢嬰,令文直往拔取角,折其頸殺之。既之問以文章起,其弟之悌以𧾷喬勇聞,之愻精草隸,世謂皆得父一絕。



 之悌,長八尺。開元中,歷劍南節度使、太原尹。嘗坐事流硃鳶,會蠻陷驩州,授總管擊之。募壯士八人,被重甲,大呼薄賊曰:「獠動即死!」賊七百人皆伏不能興,遂平賊。



 之愻為連州參軍,刺史聞其善歌,使教婢,日執笏立簾外,唱吟自如。



 閻朝隱,字友倩,趙州欒城人,少與兄鏡幾、弟仙舟皆著名。連中進士、孝悌廉讓科,補陽武尉。中宗為太子,朝隱以舍人幸。性滑稽,屬辭奇詭,為武后所賞。累遷給事中、仗內供奉。後有疾,令往禱少室山,乃沐浴,伏身俎盤為犧,請代後疾。還奏,會後亦愈,大見褒賜。其資佞諂如此。景龍初,自崖州遇赦還,累遷著作郎。先天中,為秘書少監,坐事貶通州別駕,卒。



 尹元凱,瀛州樂壽人。由慈州司倉參軍坐事免,棲遲不出者三十年。與張說、盧藏用厚,詔起為右補闕。



 時又有富嘉謨、吳少微,皆知名。



 嘉謨,武功人,舉進士。長安中,累轉晉陽尉;少微,新安人,亦尉晉陽,尤相友善;有魏谷倚者,為太原主簿,並負文辭,時稱「北京三傑」。天下文章尚徐、庾,浮俚不競,獨嘉謨、少微本經術,雅厚雄邁,人爭慕之,號「吳富體」。豫修《三教珠英》。韋嗣立薦嘉謨、少微並為左臺監察御史。已而嘉謨死,少微方病,聞之為慟,亦卒。



 劉憲,字元度,宋州寧陵人。父思立,在高宗時為名御史。於時河南、北大旱,詔遣御史中丞崔謐等分道賑贍,思立建言:「蠶務未畢而遣使撫巡,所至不能無勞餞。又賑給須立簿最,稽出入,往返停滯,妨廢且廣。若無驛處,馬須豫集,以一馬勞數家,今農事待雨興作,輟日役,破歲計,本欲安存,更煩擾之。望且責州縣給貸,須秋遣使便。」詔聽,罷謐等行。遷考功員外郎。始議加明經帖、進士雜文。卒官下。



 憲擢進士,調河南尉,累進左臺監察御史。天授中,奉詔按來俊臣罪,憲疾其酷,欲痛繩之,反為所構,貶潾水令。俊臣死,召為給事中,轉中書舍人。坐善張易之,出為渝州刺史。除太僕少卿,脩國史,兼脩文館學士,遷太子詹事。時玄宗在東宮,雅意墳史,憲啟曰:「殿下位副君,有絕人之才,非以尋擿章句,要通大意而已。侍讀褚無量經明行脩,耆年宿望,宜數召問以察其言。」太子順納。會卒,贈兗州都督。



 武后時,敕吏部糊名考判,求高才,惟憲與王適、司馬鍠、梁載言入第二等。適,幽州人,終雍州司功參軍。鍠,河南人,神龍初,以中書侍郎卒。事繼母孝,奉祿不入私舍。與弟銓、伯父希象皆歷殿中侍御史。希象,剛直不諂,終主爵員外郎。載言,聊城人,歷鳳閣舍人,專知制誥,終懷州刺史。



 李邕,字泰和,揚州江都人。父善,有雅行,淹貫古今,不能屬辭,故人號「書簏」。顯慶中,累擢崇賢館直學士兼沛王侍讀。為《文選注》,敷析淵洽,表上之,賜賚頗渥。除潞王府記室參軍,為涇城令,坐與賀蘭敏之善,流姚州,遇赦還。居汴、鄭間講授,諸生四遠至,傳其業,號「《文選》學」。



 邕少知名。始善注《文選》,釋事而忘意。書成以問邕,邕不敢對,善詰之,邕意欲有所更,善曰:「試為我補益之。」邕附事見義,善以其不可奪,故兩書並行。既冠,見特進李嶠,自言「讀書未遍,願一見秘書」。嶠曰:「秘閣萬卷,豈時日能習邪?」邕固請,乃假直秘書。未幾辭去,嶠驚,試問奧篇隱帙,了辯如響。嶠嘆曰:「子且名家!」



 嶠為內史,與監察御史張廷珪薦邕文高氣方直,才任諫諍,乃召拜左拾遺。御史中丞宋璟劾張昌宗等反狀,武后不應,邕立階下大言曰:「璟所陳社稷大計,陛下當聽。」後色解,即可璟奏。邕出,或讓曰:「子位卑,一忤旨,禍不測。」邕曰:「不如是,名亦不傳。」



 中宗立,鄭普思以方技幸,擢秘書監。邕諫曰:「陛下躬政日淺,有九重之嚴,未聞道路橫議。今籍籍皆言普思馮詭惑,說妖祥,陛下不知,猥見驅使。孔子曰:『《詩》三百,一言以蔽之,曰:思無邪。』陛下誠以普思術可致長生,則爽鳩氏且因之永有天下,非陛下乃今可得;能致神人邪,秦、漢且因之永有天下,非陛下乃今可得;能致佛法邪,梁武帝且因之永有天下,非陛下乃今可得;能鬼道邪,墨翟、干寶且各獻其主,永有天下,非陛下乃今可得。自古堯、舜稱聖者,臣觀所以行,皆在人事,敦睦九族,平章百姓,不聞以鬼神道治天下,惟陛下省察。」不納。



 五王誅,坐善張柬之,出為南和令,貶富州司戶參軍事。韋氏平,召拜左臺殿中侍御史,彈劾任職,人頗憚之。譙王重福謀反,邕與洛州司馬崔日知捕支黨,遷戶部員外郎。岑羲、崔湜惡日用,而邕與之交,玄宗在東宮,邕及崔隱甫、倪若水同被禮遇,羲等忌之,貶邕舍城丞。玄宗即位,召為戶部郎中。張廷珪為黃門侍郎,而姜皎方幸,共援邕為御史中丞。姚崇疾邕險躁,左遷括州司馬,起為陳州刺史。



 帝封泰山還,邕見帝汴州,詔獻辭賦,帝悅。然矜肆,自謂且宰相。邕素輕張說,與相惡。會仇人告邕贓貸枉法,下獄當死。許昌男子孔璋上書天子曰:



 明主舉能而舍過,取才而棄行,烈士抗節,勇者不避死,故晉用林父不以過,漢任陳平不以行,禽息隕身不祈生,北郭碎首不愛死。向若林父誅,陳平死,百里不用,晏嬰見逐,是晉無赤狄之土,漢無天子之尊,秦不強,齊不霸矣。伏見陳州刺史邕,剛毅忠烈,難不茍免。往者折二張之角,挫韋氏之鋒,雖身受謫屈,而奸謀沮解,即邕有功於國。且邕所能者,拯孤恤窮,救乏賙急,家無私聚。今聞坐贓下吏,死在旦夕。臣聞生無益於國者,不若殺身以明賢。臣願以六尺之軀膏鈇鉞,以代邕死。臣與邕生平不款曲,臣知有邕,邕不知有臣,臣不逮邕明矣。夫知賢而舉,仁也;任人之患,義也。獲二善以死,臣又何求?伏惟陛下寬邕之死,使率德改行。興林父、曲逆之功,臣得瞑目;附禽息、北郭之跡,大願畢矣。若以陽和方始,重行大戮,則臣請伏劍,不敢煩有司,皇天后土,實聞臣言。昔吳、楚反,漢得劇孟則不憂,夫以一賢而敵七國之眾,伏惟敷含垢之道,棄瑕之義,遠思劇孟,近取於邕。況告成岱宗,天地更新,赦而復論,人誰無罪,惟明主圖之。臣聞士為知己者死,臣不為死者所知,而甘之死者,非特惜邕賢,亦以成陛下矜能之慈。



 疏奏,邕得減死,貶遵化尉,流璋嶺南。邕妻溫,復為邕請戍邊自贖,曰:



 邕少習文章,疾惡如仇,不容於眾,邪佞切齒,諸儒側目。頻謫遠郡,削跡朝端,不啻十載。歲時嘆戀,聞者傷懷。屬國家有事泰山,法駕旋路,邕獻牛酒,例蒙恩私。妾聞正人用則佞人憂,邕之禍端,故自此始。且邕比任外官,卒無一毀,天意暫顧,罪過旋生。諺曰:「士無賢不肖,入朝見疾。」惟陛下明察。邕初蒙訊責,便系牢戶,水不入口者逾五日,氣息奄奄,惟吏是聽。事生吏口,迫邕手書。貸人蠶種,以為枉法;市羅貢奉,指為奸贓。於時匭使朝堂,守捉嚴固,號天訴地,誰肯為聞?泣血去國,投身荒裔,永無還期。妾願使邕得充一卒,效力王事,膏塗朔邊,骨糞沙壤,成邕夙心。



 表入不省。



 邕後從中人楊思勖討嶺南賊有功,徙澧州司馬。開元二十三年,起為括州刺史,喜興利除害。復坐誣枉,且得罪,天子識其名,詔勿劾。後歷淄、滑二州刺史,上計京師。始,邕蚤有名,重義愛士,久斥外,不與士大夫接。既入朝,人間傳其眉目瑰異,至阡陌聚觀,後生望風內謁,門巷填隘。中人臨問,索所為文章,且進上。以讒媢不得留,出為汲郡、北海太守。



 天寶中,左驍衛兵曹參軍柳勣有罪下獄,邕嘗遺勣馬,故吉溫使引邕嘗以休咎相語,陰賂遺。宰相李林甫素忌邕,因傅以罪。詔刑部員外郎祁順之、監察御史羅希奭就郡杖殺之,時年七十。代宗時,贈秘書監。



 邕之文,於碑頌是所長,人奉金帛請其文,前後所受鉅萬計。邕雖詘不進,而文名天下,時稱李北海。盧藏用嘗謂:「邕如干將、莫邪,難與爭鋒,但虞傷缺耳。」後卒如言。杜甫知邕負謗死,作《八哀詩》,讀者傷之。邕資豪放,不能治細行,所在賄謝,畋游自肆,終以敗云。



 呂向,字子回,亡其世貫,或曰涇州人。少孤,托外祖母隱陸渾山。工草隸,能一筆環寫百字,若縈發然,世號「連錦書」。強志於學,每賣藥,即市閱書,遂通古今。



 玄宗開元十年,召入翰林,兼集賢院校理,侍太子及諸王為文章。時帝歲遣使採擇天下姝好,內之後宮,號「花鳥使」;向因奏《美人賦》以諷,帝善之,擢左拾遺。天子數校獵渭川,向又獻詩規諷,進左補闕。帝自為文,勒石西嶽,詔向為鐫勒使。



 以起居舍人從帝東巡,帝引頡利發及蕃夷酋長入仗內,賜弓矢射禽。向上言:「鴟梟不鳴,未為瑞鳥;豺虎雖伏,弗曰仁獸。況突厥安忍殘賊,莫顧君父,陛下震以武義,來以文德,勢不得不廷,故稽顙稱臣,奔命遣使。陛下引內從官,陪封禪盛禮,使飛矢於前,同獲獸之樂,是狎暱太過。或荊卿詭動,何羅竊發,逼嚴蹕,冒清塵,縱醢單于,污穹廬,何以塞責?」帝順納,詔蕃夷出仗。久之,遷主客郎中,專侍皇太子,眷賚良異。



 始,向之生,父岌客遠方不還。少喪母,失墓所在,將葬,巫者求得之。不知父在亡,招魂合諸墓。後有傳父猶在者,訪索累年不獲。它日自朝還,道見一老人,物色問之,果父也。下馬抱父足號慟,行人為流涕。帝聞,咨嘆,官岌朝散大夫,賜錦彩,給內教坊樂工,娛懌其心。卒,贈東平太守。



 向終喪,再遷中書舍人,改工部侍郎。卒,贈華陰太守。嘗以李善釋《文選》為繁釀,與呂延濟、劉良、張銑、李周翰等更為詁解,時號《五臣注》。



 王翰,字子羽,並州晉陽人。少豪健恃才,及進士第,然喜蒱酒。張嘉貞為本州長史,偉其人,厚遇之。翰自歌以舞屬嘉貞,神氣軒舉自如。張說至,禮益加。復舉直言極諫,調昌樂尉,又舉超拔君類。方說輔政,故召為秘書正字,擢通事舍人、駕部員外郎。家畜聲伎,目使頤令,自視王侯,人莫不惡之。說罷宰相,翰出為汝州長史,徙仙州別駕。日與才士豪俠飲樂游畋,伐鼓窮歡,坐貶道州司馬,卒。



 孫逖,博州武水人。後魏光祿大夫惠蔚,其先也。祖希壯,為韓王府典簽,四世傳一子,故無近屬。父嘉之,少孤,依外家,客涉、鞏間。垂拱初,詣洛陽獻書,不報。第進士,終襄邑令。



 逖幼有文,屬思警敏。年十五,見雍州長史崔日用,令賦土火爐,援筆成篇,理趣不凡,日用駭嘆,遂與定交。舉手筆俊拔、哲人奇士、隱淪屠釣及文藻宏麗等科。開元十年,又舉賢良方正。玄宗御洛城門引見,命戶部郎中蘇晉等第其文異等,擢左拾遺。張說命子均、垍往拜之。李邕負才,自陳州入計,裒其文示逖。



 李暠鎮太原,表置幕府。以起居舍人入為集賢院脩撰。時海內少事,帝賜群臣十日一燕,宰相蕭嵩會百官賦《天成》、《玄澤》、《維南有山》、《楊之華》、《三月》、《英英有蘭》、《和風》、《嘉木》等詩八篇,繼《雅》、《頌》體,使逖序所以然。改考功員外郎,取顏真卿、李華、蕭穎士、趙驊等,皆海內有名士。俄遷中書舍人。是時,嘉之且八十,猶為令,逖求降外官,增父秩。帝嘉納,拜嘉之宋州司馬,聽致仕。父喪闋,復拜舍人。開元間,蘇頲、齊浣、蘇晉、賈曾、韓休、許景先及逖典詔誥,為代言最,而逖尤精密,張九齡視其草,欲易一字,卒不能也。居職八年,判刑部侍郎,以病風乞解,徙太子左庶子,遂綿廢累年,徙少詹事。上元中卒,贈尚書右僕射,謚曰文。



 諸子成最知名。



 成,字思退,推廕仕累洛陽、長安令。兄宿為華州刺史,因悸病喑,成請告往視,不待報輒行,代宗嘉其悌,不責也。稍遷倉部郎中、京兆少尹。為信州刺史,歲大旱,發倉以賤直售民,故饑而不亡。再期增戶五千,詔書褒美。徙蘇州,改桂管觀察使,卒。



 成通經術,奏議據正。嘗有期喪,吊者至,成不易縗而見。客疑之,請故,答曰:「縗者,古居喪常服,去之則廢喪也。今而巾襆,失矣。」子公器,亦至邕管經略使。



 公器子簡,字樞中。元和初,登進士第,闢鎮國、荊南幕府。累遷左司、吏部二郎中,繇諫議大夫知制誥,進中書舍人。初,逖掌誥,至代宗時,宿又居職,逮簡凡三世。



 會昌初,遷尚書左丞,建言:



 班位以品秩為等差,今官兼臺省,位置遷誤,不可為法。元和元年,御史臺白奏,常參官兼大夫、中丞者,視檢校官,居本品同類官上。其後侍郎兼大夫者,皆在左、右丞上。當時侍郎兼大夫少,唯京兆尹兼之。京兆尹從三品,今位乃在本品同類官從三品卿、監上,太常、宗正卿正三品下。左丞乃正四品上,戶部侍郎正四品下,今戶部侍郎兼大夫當在本品同類正四品下,諸曹侍郎上,不宜居正四品丞、郎上。又右丞正四品下,吏部侍郎正四品上,今吏部侍郎位右丞之下。蓋以丞有繩轄之重,雖吏部品高,猶居其下,然則戶部侍郎雖兼大夫,安得居其上哉?今散官自將仕郎至開府、特進,每品正、從有上中下,名級各異,則正從上下不得謂之同品。京兆、河南司錄及諸府州錄事參軍事皆操紀律,正諸曹,與尚書省左、右丞紀綱六曹略等,假使諸曹掾因功勞加臺省官,安得位在司錄、錄事參軍上?且左丞糾射八坐,主省內禁令、宗廟祠祭事,御史不當,得彈奏之,良以臺官所奏,拘牽成例,不揣事之輕重。使理可循,雖無往比,自宜行之。否者,號曰舊章,正可改也。



 武宗詔兩省官詳議,皆從簡請。



 歷河中、興元、宣武節度使,檢校尚書右僕射、東都留守。而弟範亦為淄青節度使,世推顯家。



 李白,字太白,興聖皇帝九世孫。其先隋末以罪徙西域,神龍初,遁還,客巴西。白之生,母夢長庚星,因以命之。十歲通詩書,既長,隱岷山。州舉有道,不應。蘇頲為益州長史,見白異之,曰:「是子天才英特,少益以學,可比相如。」然喜縱橫術,擊劍,為任俠,輕財重施。更客任城,與孔巢父、韓準、裴政、張叔明、陶沔居徂徠山,日沈飲,號「竹溪六逸」。



 天寶初,南入會稽,與吳筠善,筠被召,故白亦至長安。往見賀知章,知章見其文,嘆曰:「子,謫仙人也!」言於玄宗,召見金鑾殿,論當世事,奏頌一篇。帝賜食,親為調羹,有詔供奉翰林。白猶與飲徒醉於市。帝坐沈香亭子,意有所感,欲得白為樂章;召入,而白已醉,左右以水靧面,稍解,援筆成文,婉麗精切無留思。帝愛其才,數宴見。白嘗侍帝,醉,使高力士脫靴。力士素貴,恥之,擿其詩以激楊貴妃,帝欲官白,妃輒沮止。白自知不為親近所容,益驁放不自脩,與知章、李適之、汝陽王璡、崔宗之、蘇晉、張旭、焦遂為「酒八仙人」。懇求還山,帝賜金放還。白浮游四方,嘗乘舟與崔宗之自採石至金陵,著宮錦袍坐舟中,旁若無人。



 安祿山反,轉側宿松、匡廬間,永王璘闢為府僚佐。璘起兵,逃還彭澤,璘敗,當誅。初,白游並州,見郭子儀,奇之。子儀嘗犯法,白為救免。至是子儀請解官以贖,有詔長流夜郎。會赦,還尋陽,坐事下獄。時宋若思將吳兵三千赴河南,道尋陽,釋囚闢為參謀,未幾辭職。李陽冰為當塗令,白依之。代宗立,以左拾遺召,而白已卒,年六十餘。



 白晚好黃老,度牛渚磯至姑孰,悅謝家青山,欲終焉。及卒,葬東麓。元和末,宣歙觀察使範傳正祭其塚,禁樵採。訪後裔,惟二孫女嫁為民妻,進止仍有風範,因泣曰:「先祖志在青山,頃葬東麓,非本意。」傳正為改葬,立二碑焉。告二女,將改妻士族,辭以孤窮失身,命也,不願更嫁。傳正嘉嘆,復其夫徭役。



 文宗時,詔以白歌詩、裴旻劍舞、張旭草書為「三絕」。



 旭,蘇州吳人。嗜酒,每大醉,呼叫狂走,乃下筆,或以頭濡墨而書,既醒自視,以為神,不可復得也,世呼「張顛」。



 初,仕為常熟尉,有老人陳牒求判,宿昔又來,旭怒其煩,責之。老人曰:「觀公筆奇妙,欲以藏家爾。」旭因問所藏,盡出其父書,旭視之,天下奇筆也,自是盡其法。旭自言,始見公主擔夫爭道,又聞鼓吹,而得筆法意,觀倡公孫舞《劍器》,得其神。後人論書,歐、虞、褚、陸皆有異論,至旭,無非短者。傳其法,惟崔邈、顏真卿云。



 旻嘗與幽州都督孫佺北伐,為奚所圍,旻舞刀立馬上,矢四集,皆迎刀而斷,奚大驚引去。後以龍華軍使守北平。北平多虎,旻善射,一日得虎三十一,休山下。有老父曰:「此彪也。稍北,有真虎,使將軍遇之,且敗。」旻不信,怒馬趨之。有虎出叢薄中,小而猛,據地大吼,旻馬闢易,弓矢皆墮,自是不復射。



 王維,字摩詰。九歲知屬辭,與弟縉齊名,資孝友。開元初,擢進士,調太樂丞,坐累為濟州司倉參軍。張九齡執政,擢右拾遺。歷監察御史。母喪,毀幾不生。服除,累遷給事中。



 安祿山反,玄宗西狩,維為賊得,以藥下利,陽喑。祿山素知其才,迎置洛陽,迫為給事中。祿山大宴凝碧池,悉召梨園諸工合樂,諸工皆泣,維聞悲甚,賦詩悼痛。賊平,皆下獄。或以詩聞行在,時縉位已顯,請削官贖維罪,肅宗亦自憐之,下遷太子中允。久之,遷中庶子,三遷尚書右丞。



 縉為蜀州刺史未還,維自表「己有五短,縉五長,臣在省戶,縉遠方,願歸所任官,放田里,使縉得還京師。」議者不之罪。久乃召縉為左散騎常侍。上元初卒,年六十一。疾甚,縉在鳳翔,作書與別,又遺親故書數幅,停筆而化。贈秘書監。



 維工草隸,善畫,名盛於開元、天寶間,豪英貴人虛左以迎,寧、薛諸王待若師友。畫思入神,至山水平遠,雲勢石色,繪工以為天機所到,學者不及也。客有以《按樂圖》示者,無題識,維徐曰:「此《霓裳》第三疊最初拍也。」客未然,引工按曲,乃信。



 兄弟皆篤志奉佛,食不葷,衣不文彩。別墅在輞川,地奇勝,有華子岡、欹湖、竹裏館、柳浪、茱萸沜、辛夷塢,與裴迪游其中,賦詩相酬為樂。喪妻不娶,孤居三十年。母亡,表輞川第為寺,終葬其西。



 寶應中,代宗語縉曰:「朕嘗於諸王座聞維樂章,今傳幾何?」遣中人王承華往取,縉裒集數十百篇上之。



 鄭虔,鄭州滎陽人。天寶初,為協律郎,集綴當世事,著書八十餘篇。有窺其稿者,上書告虔私撰國史,虔蒼黃焚之,坐謫十年。還京師,玄宗愛其才,欲置左右,以不事事,更為置廣文館,以虔為博士。虔聞命,不知廣文曹司何在,訴宰相,宰相曰:「上增國學,置廣文館,以居賢者,令後世言廣文博士自君始,不亦美乎?」虔乃就職。久之,雨壞廡舍,有司不復修完,寓治國子館,自是遂廢。



 初,虔追紬故書可志者得四十餘篇,國子司業蘇源明名其書為《會稡》。虔善圖山水,好書,常苦無紙,於是慈恩寺貯柿葉數屋,遂往日取葉肄書,歲久殆遍。嘗自寫其詩並畫以獻,帝大署其尾曰:「鄭虔三絕」。遷著作郎。



 安祿山反,遣張通儒劫百官置東都,偽授虔水部郎中,因稱風緩,求攝市令,潛以密章達靈武。賊平,與張通、王維並囚宣陽里。三人者,皆善畫,崔圓使繪齋壁,虔等方悸死,即極思祈解於圓,卒免死,貶臺州司戶參軍事,維止下選。後數年卒。



 虔學長於地理,山川險易、方隅物產、兵戍眾寡無不詳。嘗為《天寶軍防錄》,言典事該。諸儒服其善著書,時號「鄭廣文」。在官貧約甚,澹如也。杜甫嘗贈以詩曰「才名四十年,坐客寒無氈」云。



 有鄭相如者,自滄州來,師事虔,虔未之禮,間問何所業,相如曰:「聞孔子稱『繼周者百世可知』,僕亦能知之。」虔駭然,即曰:「開元盡三十年當改元,盡十五年天下亂,賊臣僭位,公當污偽官,願守節,可以免。」虔又問:「自謂云何?」答曰:「相如有官三年,死衢州。」是年及進士第,調信安尉。既三年,虔詢吏部,則相如果死。故虔念其言,終不附賊。



 蕭穎士,字茂挺,梁鄱陽王恢七世孫。祖晶,賢而有謀,任雅相伐高麗,表為記室。越王貞舉兵,杖策詣之,陳三策,王不用,晶度必敗,乃亡去,客死廣陵。



 穎士四歲屬文,十歲補太學生。觀書一覽即誦,通百家譜系、書籀學。開元二十三年,舉進士,對策第一。父旻,以莒丞抵罪,穎士往訴於府佐張惟一,惟一曰:「旻有佳兒,吾以旻獲譴不憾。」乃平宥之。



 天寶初,穎士補秘書正字。於時裴耀卿、席豫、張均、宋遙、韋述皆先進,器其材,與鈞禮,由是名播天下。奉使括遺書趙、衛間,淹久不報,為有司劾免,留客濮陽。於是尹徵、王恆、盧異、盧士式、賈邕、趙匡、閻士和、柳並等皆執弟子禮,以次授業,號蕭夫子。召為集賢校理。宰相李林甫欲見之,穎士方父喪,不詣。林甫嘗至故人舍邀潁士,穎士前往,哭門內以待,林甫不得已,前吊乃去。怒其不下己,調廣陵參軍事,穎士急中不能堪,作《伐櫻桃樹賦》曰:「擢無庸之瑣質,蒙本枝以自庇。雖先寢而或薦,非和羹之正味。」以譏林甫云。君子恨其褊。會母喪免,流播吳、越。



 嘗謂:「仲尼作《春秋》,為百王不易法,而司馬遷作本紀、書、表、世家、列傳,敘事依違,失褒貶體,不足以訓。」乃起漢元年訖隋義寧編年,依《春秋》義類為傳百篇。在魏書高貴崩,曰:「司馬昭弒帝於南闕。」在梁書陳受禪,曰:「陳霸先反。」又自以梁枝孫,而宣帝逆取順守,故武帝得血食三紀;昔曲沃篡晉,而文公為五伯,仲尼弗貶也。乃黜陳閏隋,以唐土德承梁火德,皆自斷,諸儒不與論也。有太原王緒者,僧辯裔孫,撰《永寧公輔梁書》,黜陳不帝,穎士佐之,亦著《梁蕭史譜》及作《梁不禪陳論》以發緒義例,使光明云。



 史官韋述薦穎士自代,召詣史館待制,穎士乘傳詣京師。而林甫方威福自擅,穎士遂不屈,愈見疾,俄免官,往來鄠、杜間。林甫死,更調河南府參軍事。倭國遣使入朝,自陳國人願得蕭夫子為師者,中書舍人張漸等諫不可而止。



 安祿山寵恣,穎士陰語柳並曰:「胡人負寵而驕,亂不久矣。東京其先陷乎!」即托疾游太室山。已而祿山反,穎士往見河南採訪使郭納,言御守計,納忽不用,嘆曰:「肉食者以兒戲御劇賊,難矣哉!」聞封常清陳兵東京,往觀之,不宿而還。因藏家書於箕、穎間,身走山南,節度使源洧闢掌書記。賊別校攻南陽,洧懼,欲退保江陵,穎士說曰:「官兵守潼關,財用急,必待江、淮轉餉乃足,餉道由漢、沔,則襄陽乃今天下喉襟,一日不守,則大事去矣。且列郡數十,人百萬,訓兵攘寇,社稷之功也。賊方專崤、陜,公何遽輕土地,欲取笑天下乎?」洧乃按甲不出。亦會祿山死,賊解去。洧卒,往客金陵,永王璘召之,不見。



 時盛王為淮南節度大使,留蜀不遣,副大使李承式玩兵不振。穎士與宰相崔圓書,以為:「今兵食所資在東南,但楚、越重山復江,自古中原擾則盜先起,宜時遣王以捍鎮江淮。」俄而劉展果反。賊圍雍丘,脅泗上軍,承式遣兵往救,大宴賓客,陳女樂。穎士曰:「天子暴露,豈臣下盡歡時邪?夫投兵不測,乃使觀聽華麗,一旦思歸,誰致其死哉?」弗納。崔圓聞之,即授揚州功曹參軍。至官,信宿去。後客死汝南逆旅,年五十二,門人共謚曰文元先生。



 穎士樂聞人善,以推引後進為己任,如李陽、李幼卿、皇甫冉、陸渭等數十人,由獎目,皆為名士。天下推知人,稱蕭功曹。嘗兄事元德秀,而友殷寅、顏真卿、柳芳、陸據、李華、邵軫、趙驊,時人語曰「殷、顏、柳、陸,李、蕭、邵、趙」,以能全其交也。所與游者,孔至、賈至、源行恭、張有略、族弟季遐、劉穎、韓拯、陳晉、孫益、韋建、韋收。獨華與齊名,世號「蕭、李」。嘗與華、據游洛龍門,讀路旁碑,穎士即誦,華再閱,據三乃能盡記。聞者謂三人才高下,此其分也。有奴事穎士十年,笞楚嚴慘,或勸其去,答曰:「非不能,愛其才耳。」穎士數稱班彪、皇甫謐、張華、劉琨、潘尼能尚古,而混流俗不自振,曹植、陸機所不逮也;又言裴子野善著書。所許可當世者,陳子昂、富嘉謨、盧藏用之文辭,董南事、孔述睿之博學而已。



 子存,字伯誠,亮直有父風。能文辭,與韓會、沈既濟、梁肅、徐岱等善。浙西觀察使李棲筠表常熟主簿。顏真卿在湖州,與存及陸鴻漸等討摭古今韻字所原,作書數百篇。建中初,由殿中侍御史四遷比部郎中。張滂主財賦,闢存留務京師。裴延齡與滂不協,存疾其奸,去官,風痺卒。



 韓愈少為存所知,自袁州還,過存廬山故居,而諸子前死,唯一女在,為經贍其家。



 殷寅者,陳郡人。邵軫者,汝南人。



 陸據,河南人,字德鄰,後周上庸公騰六世孫。神宇警邁,善物理。年三十始到京師,公卿愛其文,交譽之。天寶十三載,終司勛員外郎。



 柳並者,字伯存。大歷中,闢河東府掌書記,遷殿中侍御史。喪明,終於家。初,並與劉太真、尹徵、閻士和受業於穎士,而並好黃老。穎士常曰:「太真,吾入室者也,斯文不墜,寄是子云。徵博聞強識,士和鉤深致遠,吾弗逮已。並不受命而尚黃、老,予亦何誅?」



 並弟談,字中庸,穎士愛其才,以女妻之。



 士和字伯均,著《蘭陵先生誄》、《蕭夫子集論》,因榷歷世文章,而盛推穎士所長,以為「聞蕭氏風者,五尺童子羞稱曹、陸」。



 皇甫冉,字茂政,十歲便能屬文,張九齡嘆異之。與弟曾皆善詩。天寶中,踵登進士,授無錫尉。王縉為河南元帥,表掌書記。遷累右補闕,卒。



 曾,字孝常,歷監察御史。其名與冉相上下,當時比張氏景陽、孟陽雲。



 蘇源明,京兆武功人,初名預,字弱夫。少孤,寓居徐、兗。工文辭,有名天寶間。及進士第,更試集賢院。累遷太子諭德。出為東平太守。是時,濟陽郡太守李倰以郡瀕河,請增領宿城、中都二縣以紓民力。二縣,隸東平、魯郡者也。於是源明議廢濟陽,析三縣分隸濟南、東平、濮陽。詔河南採訪使會濮陽太守崔季重、魯郡太守李蘭、濟南太守田琦及源明、倰五太守議於東平,不能決。既而卒廢濟陽,以縣皆隸東平。召源明為國子司業。



 安祿山陷京師,源明以病不受偽署。肅宗復兩京,擢考功郎中、知制誥。是時,承大盜之餘,國用覂屈,宰相王璵以祈禬進,禁中禱祀窮日夜,中官用事,給養繁靡,群臣莫敢切諍。昭應令梁鎮上書勸帝罷淫祀,其他不暇及也。源明數陳政治得失。及史思明陷洛陽,有詔幸東京,將親征。源明因上疏極諫曰:



 淫雨積時,道路方梗,甚不可一也。自春大旱,秋苗耗半,斂獲未畢,先之以清道之役,申之以供頓之苦,甚不可二也。每立殿廊,見旌旗之下,餓夫執殳,僕於行間,日見二三;市井餒食孚求食,死於路旁,日見四五。甚不可三也。奸夫盜兒,連墻接棟,磨礪以須陛下之出,御史大夫必不能澄清禁止。甚不可四也。聖皇巡蜀之初,都內財貨、吏民資產,糜散於道路之手,至有乘馬駃驢入宣政、紫宸者。況陛下初有四海,威制不及曩時遠矣。今茲東行,殆賊臣誘掖陛下而已。《詩》曰「三星在霤」,謂危亡在於須臾,臣不勝嗚咽,為陛下痛之。願速罷幸,不然,窮氓樂禍,已扼腕於下。甚不可五也。方今河、洛驛騷,江湖叛渙,《詩》曰:「中原有菽,庶民採之。」彼思明、楚元,皆採菽之人也。陛下何遽輕萬乘而速成之邪?甚不可六也。大河南北,舉為寇盜,王公以下,廩稍匱絕,將士糧賜,僅支日月,而中官冗食,不減往年,梨園雜伎,愈盛今日,陛下未得穆然高枕,殆繇此也。自非中庸指使,太常正樂外,願一切放歸,給長牒勿事,須五六年後,隨事蠲省。今聚而仰給,甚不可七也。李光弼拔河陽,王思禮下晉原,衛伯玉拂焉耆,過析支,不日可至。御史大夫王玄志壓巫閭,臨幽都;汝州刺史田南金逾闕口,遏二室;鄧景山凌淮、泗,愾然而西。狂賊失勢,蹙於緱山之下,北不敢逾孟津,東不敢過[B124]子,計日反接而至矣。陛下不坐而受之,乃欲親征,徇一朝之怒,甚不可八也。王者之於天地神祇,享之以牲幣而已。記曰:「不祈方士。」彼淫巫愚祝,妄有關說,甚不可九也。天子順動,人皆幸之之謂幸,人皆病之之謂不幸。臣等屢怫視聽,聯伏赤墀之下,頓顙流涕而出,雖陛下優容貸罪,凡百之臣必昌言於朝,萬口謗於外,甚不可十也。臣聞子不諍於父,不孝也;臣不諍於君,不忠也。不孝不忠,為茍榮冒祿,圈牢之物不若也。臣雖至賤,不能委身圈牢之中,將使樵夫指而笑之。



 帝嘉其切直,遂罷東幸。後以秘書少監卒。



 源明雅善杜甫、鄭虔,其最稱者元結、梁肅。



 肅,字敬之,一字寬中。隋刑部尚書毘五世孫,世居陸渾。建中初,中文辭清麗科,擢太子校書郎。蕭復薦其材,授右拾遺,脩史,以母羸老不赴。杜佑闢淮南掌書記,召為監察御史,轉右補闕、翰林學士、皇太子諸王侍讀。卒,年四十一,贈禮部郎中。



\end{pinyinscope}