\article{志第二十五 五行二}

\begin{pinyinscope}

 《五行傳》曰:「治宮室,飾臺榭,內淫亂,犯親戚,侮父兄在後來的發展中分裂了,一些人繼續宣揚神秘宗教教義和戒,則稼穡不成。」謂土失其性,則有水旱之災,草木百穀不熟也。又曰:』思心不睿,是謂不聖。厥咎霧,厥罰常風,厥極兇短折。時則有脂夜之妖,時則有華孽、蠃蟲之孽,時則有牛禍,時則有心腹之痾,時則有黃眚黃祥,時則有木、火、金、水沴土。」



 ○稼穡不成



 貞觀元年,關內饑。



 總章二年,諸州四十餘饑,關中尤甚。



 儀鳳四年春,東都饑。



 調露元年秋,關中饑。



 永隆元年冬,東都饑。



 永淳元年,關中及山南州二十六饑,京師人相食。



 垂拱三年,天下饑。



 大足元年春,河南諸州饑。



 景龍二年春,饑。三年三月,饑。



 先天二年冬,京師、岐、隴、幽州饑。



 開元十六年,河北饑。



 乾元三年春,饑,米斗錢千五百。



 廣德二年秋,關輔饑,米斗千錢。



 永泰元年,饑,京師米斗千錢。



 貞元元年春,大饑,東都、河南、河北米斗千錢,死者相枕。二年五月,麥將登而雨霖,米斗千錢。十四年,京師及河南饑。十九年秋,關輔饑。



 元和七年春,饑。八年,廣州饑。九年春,關內饑。十一年,東都、陳許州饑。



 長慶二年,江淮饑。



 大和四年,河北及太原饑。六年春,劍南饑。九年春,饑,河北尤甚。



 開成四年,溫、臺、明等州饑。



 大中五年冬,湖南饑。六年夏,淮南饑,海陵、高郵民於官河中漉得異米,號「聖米」。九年秋,淮南饑。



 咸通三年夏,淮南、河南饑。九年秋,江左及關內饑,東都尤甚。



 乾符三年春,京師饑。



 中和二年,關內大饑。四年,關內大饑,人相食。



 光啟二年二月,荊、襄大饑,米斗三千錢,人相食。三年,揚州大饑,米鬥萬錢。



 大順二年春,淮南大饑。



 天祐元年十月,京師大饑。



 ○常風



 武德二年十二月壬子,大風拔木。《易》巽為風,「重巽以申命」。其及物也,象人君誥命,其鼓動於天地間,有時飛沙揚塵,怒也,發屋拔木者,怒甚也。其占:「大臣專恣而氣盛,眾逆同志,君行蒙暗,施於事則皆傷害,故常風。」又「飄風入宮闕,一日再三,若風聲如雷觸地而起,為兵將興。」



 貞觀十四年六月乙酉,大風拔木。



 咸亨四年八月己酉,大風落太廟鴟尾。



 永隆二年七月,雍州大風害稼。



 弘道元年十二月壬午晦,宋州大風拔木。



 嗣聖元年四月丁巳,寧州大風拔木。



 垂拱四年十月辛亥,大風拔木。



 永昌二年五月丁亥,大風拔木。



 神龍元年三月乙酉,睦州大風拔木。崔玄韋封博陵郡王也,大風折其輅蓋。二年六月乙亥,滑州大風拔木。



 景龍元年七月,郴州大風,發屋拔木。八月,宋州大風拔木,壞廬舍。二年十月辛亥,滑州暴風發屋。三年三月辛未,曹州大風拔木。



 開元二年六月,京師大風發屋,大木拔者十七八。四年六月辛未,京師、陜、華大風拔木。九年七月丙辰,揚州、潤州暴風雨,發屋拔木。十四年六月戊午,大風拔木發屋,端門鴟尾盡落。端門,號令所從出也。十九年六月乙酉,大風拔木。二十二年五月戊子,大風拔木。



 天寶十一載五月甲子,東京大風拔木。十三載三月辛酉,大風拔木。



 永泰元年三月辛亥,大風拔木。



 大歷七年五月乙酉,大風拔木。十年五月甲寅,大風拔木。



 貞元元年七月庚子,大風拔木。六年四月甲申,大風雨。八年五月己未,暴風發太廟屋瓦,毀門闕、官署、廬舍不可勝紀。十年六月辛未,大風拔木。十四年八月癸未,廣州大風,壞屋覆舟。



 元和元年六月丙申,大風拔木。三年四月壬申,大風毀含元殿欄檻二十七間。占為兵起。四年十月壬午,天有氣如煙,臭如燔皮,日昳大風而止。五年三月丙子,大風毀崇陵上宮衙殿鴟尾及神門戟竿六,壞行垣四十間。八年六月庚寅,京師大風雨,毀屋飄瓦,人多壓死者,丙申,富平大風,拔棗木千餘株。十二年春,青州一夕暴風自西北,天地晦冥,空中有若旌旗狀,屋瓦上如蹂躒聲。有日者占之曰:「不及五年,茲地當大殺戮。」



 長慶二年正月己酉,大風霾。十月,夏州大風,飛沙為堆,高及城堞。三年正月丁巳朔,大風,昏霾終日。四年六月庚寅,大風毀延喜門及景風門。



 大和八年六月癸未,暴風壞長安縣署及經行寺塔。九年四月辛丑,大風拔木萬株,墮含元殿四鴟尾,拔殿廷樹三,壞金吾仗舍,發城門樓觀內外三十餘所,光化門西城十數雉壞。



 開成三年正月戊辰,大風拔木。五年四月甲子,大風拔木;五月壬寅,亦如之;七月戊寅,亦如之。



 會昌元年三月,黔南大風飄瓦。



 咸通六年正月,絳州大風拔木,有十圍者。十一月己卯晦,潼關夜中大風,出如吼雷,河噴石鳴,群烏亂飛,重關傾側。十二月,大風拔木。



 乾符五年五月丁酉,大風拔木。



 廣明元年四月甲申,京師及東都、汝州雨雹,大風拔木。四年六月乙巳,太原大風雨,拔木千株,害稼百里。



 光化三年七月乙丑,洺州大風,拔木發屋。



 天復二年,升州大風,發屋飛大木。



 ○夜妖



 大和九年十一月戊辰,晝晦。



 咸通七年九月辛卯朔,天暗。



 乾符二年二月,宣武境內黑風,雨土。



 天祐元年閏四月乙未朔,大風,雨土。



 ○華孽



 延載元年九月,內出梨華一枝示宰相。萬木搖落而生華,陰陽黷也。《傳》曰:「天反時為災。」又近常燠也。



 神龍二年十月,陳州李有華,鮮茂如春。



 元和十一年十二月,桃杏華。



 大和二年九月,徐州、滑州李有華,實可食。



 會昌三年冬,沁源桃李華。



 廣明元年冬,桃李華,山華皆發。



 中和二年九月,太原諸山桃杏華,有實。



 景福中,滄州城塹中冰有文,如畫大樹、華葉芬敷者,時人以為其地當有兵難。近華孽也。



 ○嬴蟲之孽



 貞觀二十一年八月,萊州螟。



 開元二十二年八月,榆關虸蚄蟲害稼,入平州界,有群雀來食之,一日而盡。二十六年,榆關虸蚄蟲害稼,群雀來食之。三載,青州紫蟲食田,有鳥食之。



 廣德元年秋,虸蚄蟲害稼,關中尤甚,米斗千錢。



 貞元十年四月,江西溪澗魚頭皆戴蚯蚓。



 長慶四年,絳州虸蚄蟲害稼。



 大和元年秋,河東、同虢等州虸蚄蟲害稼。



 開成元年,京城有蟻聚,長五六十步,闊五尺至一丈,厚五寸至一尺者。四年,河南黑蟲食田。



 ○牛禍



 調露元年春,牛大疫。京房《易傳》曰;「牛少者穀不成。」又占曰:「金革動。」



 長安中,有獻牛無前膊,三足而行者。又有牛膊上生數足,蹄甲皆具者。武太后從姊之子司農卿宗晉卿家牛生三角。



 神龍元年春,牛疫。二年冬,牛大疫。



 先天初,洛陽市有牛,左脅有人手,長一尺,或牽之以乞丐。



 開元十五年春,河北牛大疫。



 大歷八年,武功、櫟陽民家牛生犢,二首。



 貞元二年,牛疫。四年二月,郊牛生犢,六足,足多者,下不一。郊所以奉天。七年,關輔牛大疫,死者十五六。



 咸通七年,荊州民家牛生犢,五足。十五年夏,渝州江陽有水牛生騾駒,駒死。



 光啟元年,河東有牛人言,其家殺而食之。二年,延州膚施有牛死復生。



 ○黃眚黃祥



 貞觀七年三月丁卯,雨土。二十年閏三月己酉,有黃雲闊一丈,東西際天。黃為土功。



 永徽三年三月辛巳,雨土。



 景龍元年六月庚午,陜州雨土。十二月丁丑,雨土。



 天寶十三載二月丁丑,雨黃土。



 大歷七年十二月丙寅,雨土。



 貞元二年四月甲戌,雨土。八年二月庚子,雨土。



 大和八年十月甲子,土霧晝昏,至於十一月癸丑。



 開成元年七月乙亥,雨土。



 咸通十四年三月癸巳,雨黃土。



 中和二年五月辛酉,大風,雨土。



 天復三年二月,雨土,天地昏霾。



 天祐元年閏四月甲辰,大風,雨土。



 ○木火金水沴土



 武德二年十月乙未,京師地震。陰盛而反常則地震,故其占為臣強,為后妃專恣,為夷犯華,為小人道長,為寇至,為叛臣。七年七月,雋州地震,山摧壅江,水噎流。



 貞觀七年十月乙丑,京師地震,十二年正月壬寅,松、叢二州地震,壞廬舍。二十年九月辛亥,靈州地震,有聲如雷。二十三年八月癸酉朔,河東地震,晉州尤其,壓殺五十餘人;乙亥,又震。十一月乙丑,又震。



 永徽元年四月己巳朔,晉州地震;己卯,又震。六月庚辰,又震,有聲如雷。二年十月,又震。十一月戊寅,定襄地震。帝始封晉王,初即位而地屢震,天下將由帝而動搖象也。



 儀鳳二年正月庚辰,京師地震。



 永淳元年十月甲子,京師地震。



 垂拱三年七月乙亥,京師地震。四年七月戊午,又震。八月戊戌,神都地震。



 延載元年四月壬戌,常州地震。



 大足元年七月乙亥,楊、楚、常、潤、蘇五州地震。二年八月辛亥,劍南六州地震。



 景龍四年五月丁丑,剡縣地震。



 景雲三年正月甲戌,並、汾、絳三州地震,壞廬舍,壓死百餘人。



 開元二十二年二月壬寅,秦州地震,西北隱隱有聲,坼而復合,經時不止,壞廬舍殆盡,壓死四千餘人。二十六年三月癸巳,京師地震。



 至德元載十一月辛亥朔,河西地震裂有聲,陷廬舍,張掖、酒泉尤甚,至二載三月癸亥乃止。



 大歷二年十一月壬申,京師地震,自東北來,其聲如雷者。三年五月丙戌,又震。十二年,恆、定二州地大震,三日乃止,束鹿、寧晉地裂數丈,沙石隨水流出平地,壞廬舍,壓死者數百人。



 建中元年四月己亥,京師地震。三年六月甲子,又震。四年四月甲子,又震。五月辛巳,又震。



 貞元二年五月己酉,又震。三年十一月丁丑夜,京師、東都、蒲、陜地震。四年正月庚戌朔夜,京師地震;辛亥、壬子、丁卯、戊辰、庚午、癸酉、甲戌、乙亥,皆震,金、房二州尤甚,江溢山裂,屋宇多壞,人皆露處。二月壬午,京師又震;甲申、乙酉,丙申,三月甲寅、己未、庚午、辛未,五月丙寅、丁卯,皆震。八月甲午,又震,有聲如雷;甲辰,又震。九年四月辛酉,又震,有聲如雷,河中、關輔尤甚,壞城壁廬舍,地裂水湧。十年四月戊申,京師地震。癸丑,又震,侍中渾瑊第有樹湧出,樹枝皆戴蚯蚓。十三年七月乙未,又震。



 元和七年八月,京師地震,草樹皆搖。九年三月丙辰,雋州地震,晝夜八十,壓死百餘人,地陷者三十里。十年十月,京師地震。十一年二月丁丑,又震。十五年正月,穆宗即位,戊辰,始朝群臣於宣政殿,是夜地震。



 大和二年正月壬申,地震;七年六月甲戌,又震。九年三月乙卯,京師地震,屋瓦皆墜,戶牖間有聲;開成元年二月乙亥,又震;二年十一月乙丑夜,又震;四年十一月甲戌,又震。



 會昌二年正月癸亥,宋、亳二州地震。十二月癸未,京師地震。



 大中三年十月辛巳,上都及振武、河西、天德、靈武、鹽夏等州地震,壞廬舍,壓死數十人。十二年八月丁巳,太原地震。



 咸通元年五月,上都地震。六年十二月,晉、絳二州地震,壞廬舍,地裂泉湧,泥出青色。八年正月丁未,河中、晉、絳三州地大震,壞廬舍,人有死者。十三年四月庚子朔,浙東、西地震。



 乾符三年六月乙丑,雄州地震,至七月辛巳止,州城廬舍盡壞,地陷水湧,傷死甚眾;是月,濮州地震。十二月,京師地震有聲。四年六月庚寅,雄州地震。六年二月,京師地震,有聲如雷,藍田山裂水湧。



 中和三年秋,晉州地震,有聲如雷。



 光啟二年春,成都地震,月中十數。占曰:「兵、饑。」十二月,魏州地震。



 乾寧二年三月庚午,河東地震。



 ○山摧



 貞觀八年七月,隴右山摧。山者高峻,自上而隕之象也。



 垂拱二年九月己巳,雍州新豐縣露臺鄉大風雨,震電,有山湧出,高二十丈,有池周三百畝,池中有龍鳳之形、麥之異,武后以為休應,名曰「慶山」。荊州人俞文俊上言:「天氣不和而寒暑隔,人氣不和而贅疣生,地氣不和而堆阜出。今陛下以女主居陽位,反易剛柔,故地氣隔塞,山變為災。陛下以為『慶山』,臣以為非慶也。宜側身脩德以答天譴,不然,恐災禍至。」後怒,流於嶺南。



 永昌中,華州赤水南岸大山,晝日忽風昏,有聲隱隱如雷,頃之漸移東數百步,擁赤水,壓張村民三十餘家,山高二百餘丈,水深三十丈,坡上草木宛然。《金滕》曰:「山徙者人君不用道,祿去公室,賞罰不由君,佞人執政,政在女主,不出五年,有走王。」



 開元十七年四月乙亥,大風震電,藍田山摧裂百餘步,畿內山也。國主山川,山摧川竭,亡之證也。占曰:「人君德消政易則然。」



 大歷九年十一月戊戌,同州夏陽有山徙於河上,聲如雷。十三年,郴州黃芩山摧,壓死者數百人。



 建中二年,霍山裂。



 元和八年五月丁丑,大隗山摧。十五年七月丁未,苑中土山摧,壓死二十人。



 光啟三年四月,維州山崩,累日不止,塵坌亙天,壅江水逆流。占曰:「國破。」



 ○山鳴



 武德二年三月,太行山聖人崖有聲。占曰:「有寇至。」



 開元二十八年六月,吐蕃圍安戎城,斷水路,城東山鳴石坼,湧泉二。



 ○土為變怪



 垂拱元年九月,淮南地生毛,或白或蒼,長者尺餘,遍居人床下,揚州尤甚,大如馬鬣,焚之臭如燎毛。占曰:「兵起,民不安。」



 長壽中,東都天宮寺泥像皆流汗霡葸。



 天寶十一載六月,虢州閺鄉黃河中女媧墓因大雨晦冥,失其所在,至乾元二年六月乙未夜,瀕河人聞有風雷聲,曉見其墓踴出,下有巨石,上有雙柳,各長丈餘,時號風陵堆。占曰:「塚墓自移,天下破。」十三載,汝州葉縣南有土塊斗,中有血出,數日不止。



 大歷六年四月戊寅,藍田西原地陷。



 建中初,魏州魏縣西四十里,地數畝忽長崇數尺。四年四月甲子,京師地生毛,或黃或白,有長尺餘者。



 貞元四年四月,淮南及河南地生毛。



 元和十二年四月,吳元濟郾城守將鄧懷金以城降,城自壞五十餘步。



 大和六年二月,蘇州地震,生白毛。



 長慶中,新都大道觀泥人生須數寸,拔之復生。



 咸通五年十月,貞陵隧道摧陷。神策軍有浮屠像,懿宗嘗跪禮之,像沒地四尺。



 《五行傳》曰:「好攻戰,輕百姓,飾城郭,侵邊境,則金不從革。」謂金失其性而為變怪也。又曰:「言之不從,是謂不乂。厥咎僭,厥罰常晹,厥極憂。時則有詩妖、訛言,時則有毛蟲之孽,時則有犬禍,時則有口舌之痾,時則有白眚白祥,惟木沴金。」



 ○金不從革



 堯君素為隋守蒲州,兵器夜皆有光如火。火鑠金,金所畏也,敗亡之象。劉武周據並州,兵勢甚盛,城上槊刃夜每有火光。



 貞觀十七年八月,涼州昌松縣鴻池谷有石五,青質白文成字曰:「高皇悔出多子李元王八十年太平天子李世民年千年太子李治書燕山人士樂大國主尚汪譂獎文仁邁千古大王五王六王七王十風毛才子七佛八菩薩及上界佛田天子文武貞觀昌大聖延四方上不治示孝仙戈八為善。」太宗遣使祭之曰:「天有成命,表瑞貞石,文字昭然,歷數惟永,既旌高廟之業,又錫眇身之祚。迨於皇太子治,亦降貞符,具紀姓氏。甫惟寡薄,彌增寅懼。」昔魏以土德代漢,涼州石有文。石,金類,以五勝推之,故時人謂為魏氏之妖,而晉室之瑞。唐亦土德王,石有文,事頗相類。然其文初不可曉,而後人因推已事以驗之。蓋武氏革命,自以為金德王,其「佛菩薩」者,慈氏金輪之號也;「樂太國主」則鎮國太平公主、安樂公主,皆以女亂國;其「五王六王七王」者,唐世十八之數。



 垂拱三年七月,魏州地出鐵如船數十丈;廣州雨金。金位正秋,為刑、為兵。占曰:「人君多殺無辜。一年兵災於朝。」



 開元二十三年十二月乙巳,龍池《聖德頌》石自鳴,其音清遠如鐘磬。石與金同類。《春秋傳》:「怨讟動於民,則有非言之物言。」石鳴,近石言也。



 天寶十載六月乙亥,大同殿前鐘自鳴。占曰:「庶雄為亂。」



 至德二載,昭陵石馬汗出。昔周武帝之克晉州也,齊有石像,汗流濕地,此其類也。



 乾元二年七月乙亥晝,渾天儀有液如汗下流。



 上元二年,楚州獻寶玉十三:曰「玄黃天符」,形如笏,長八寸,有孔,云闢兵疫;曰「玉雞毛」,白玉也;曰「穀璧」,亦白玉也,粟爛自然,無雕鐫跡;曰「西王母白環」二;曰「如意寶珠」,大如雞卵;曰「紅靺鞨」,大如巨粟;曰「瑯玕珠」二,形如玉環,四分缺一;曰「玉印」,大如半手,理如鹿,陷入印中;曰「皇后採桑鉤」,如箸屈其末;曰「雷公石斧」無孔;其一闕。凡十三;寘之日中,白氣連天。



 元和中,文水《武士皞碑》失其龜頭。翰林院有鈴,夜中文書入則引之,以代傳呼,長慶中,河北用兵,夜輒自鳴,與軍中息耗相應,聲急則軍事急,聲緩則軍事緩。資州有石方丈,走行數畝。



 大和三年,南蠻圍成都,毀玉晨殿為礧,有吼聲三,乃止。四年五月己卯,通化南北二門鎖不可開,鑰入,如有持之者,破其管,門乃啟;又浙西觀察使王璠治潤州城隍,中得方石,有刻文曰:「山有石,石有玉,玉有瑕,瑕即休。」



 廣明元年,華嶽廟玄宗御製碑隱隱然有聲,聞數里間,浹旬乃止。近石言也。



 光化三年冬,武德殿前鐘聲忽嘶嗄;天復元年九月,聲又變小。



 ○常暘



 武德三年夏,旱,至於八月乃雨。四年,自春不雨,至於七月。雨,少陰之氣,其氣毀則不雨。少陰者,金也,金為刑、為兵,刑不辜,兵不戢,則金氣毀,故常為旱。火為盛陽,陽氣強悍,故聖人制禮以節之。禮失則僭而驕炕,以導盛陽,火勝則金衰,故亦旱。於五行,土實制水,土功興則水氣壅閼,又常為旱。天官有東井,主水事,天漢、天江,亦水祥也。水與火仇,而受制於土,土火謫見,若日蝕過分而未至,與七曜循中道之南,皆旱祥也。七年秋,關內、河東旱。



 貞觀元年夏,山東大旱。二年春,旱。三年春、夏,旱。四年春,旱。自太上皇傳位至此,而比年水旱。九年秋,劍南、關東州二十四旱。十二年,吳、楚、巴、蜀州二十六旱;冬,不雨,至於明年五月。十七年春、夏,旱,二十一年秋,陜、絳、蒲、夔等州旱。二十二年秋,開、萬等州旱;冬,不雨,至於明年三月。



 永徽元年,京畿雍、同、絳等州十,旱。二年九月,不雨,至於明年二月。四年夏、秋,旱,光、婺、滁、潁等州尤甚。



 顯慶五年春,河北州二十二旱。



 總章元年,京師及山東、江淮大旱。二年七月,劍南州十九旱;冬,無雪。



 咸享元年春,旱;秋,復大旱。



 儀鳳二年夏,河南、河北旱。三年四月,旱。



 永隆二年,關中旱,霜,大饑。



 永淳元年,關中大旱,饑。二年夏,河南、河北旱。



 永昌元年三月,旱。



 神功元年,黃、隋等州旱。



 久視元年夏,關內、河東旱。



 長安二年春,不雨,至於六月。三年冬,無雪,至於明年二月。



 神龍二年冬,不雨,至於明年五月,京師、山東、河北、河南旱,饑。



 太極元年春,旱;七月復旱。



 開元二年春,大旱。十二年七月,河東、河北旱,帝親禱雨宮中,設壇席,暴立三日。九月蒲、同等州旱。十四年秋,諸道州十五旱。十五年,諸道州十七旱。十六年,東都、河南、宋亳等州旱。二十四年夏,旱。



 永泰元年春、夏,旱。二年,關內大旱,自三月不雨,至於六月。



 大歷六年春,旱,至於八月。



 建中三年,自五月不雨,至於七月。



 興元元年冬,大旱。



 貞元元年春,旱,無麥苗,至於八月,旱甚,灞滻將竭,井皆無水。六年春,關輔大旱,無麥苗;夏,淮南、浙西、福建等道大旱,井泉竭,人曷且疫,死者甚眾。七年,揚、楚、滁、壽、澧等州旱。十四年春,旱,無麥。十五年夏,旱。十八年夏,申、光、蔡州旱。十九年正月,不雨,至七月甲戌乃雨。



 永貞元年秋,江浙、淮南、荊南、湖南、鄂岳陳許等州二十六,旱。



 元和三年,淮南、江南、江西、湖南、廣南、山南東西皆旱。四年春、夏,大旱;秋,淮南、浙西、江西、江東旱。七年夏,揚、潤等州旱。八年夏,同、華二州旱。十五年夏,旱。



 寶歷元年秋,荊南、淮南、浙西、江西、湖南及宣、襄、鄂等州旱。



 太和元年夏,京畿、河中、同州旱。六年,河東、河南、關輔旱。七年秋,大旱。八年夏,江淮及陜、華等州旱。九年秋,京兆、河南、河中、陜華同等州旱。



 開成二年春、夏,旱。四年夏,旱,浙東尤甚。



 會昌五年春,旱。六年春,不雨;冬,又不雨,至明年二月。



 大中四年,大旱。



 咸通二年秋,淮南、河南不雨,至於明年六月。九年,江淮旱。十年夏,旱。十一年夏,旱。



 廣明元年春、夏,大旱。



 中和四年,江南大旱,饑,人相食。



 景福二年秋,大旱。



 光化三年冬,京師旱,至於四年春。



 ○詩妖



 竇建德未敗時,有謠曰:「豆入牛口,勢不得久。」



 貞觀十四年,交河道行軍大總管侯君集伐高昌。先是其國中有童謠曰;「高昌兵馬如霜雪,漢家兵馬如日月,日月照霜雪,回首自消滅。」



 永徽後,民歌《武媚娘曲》。



 調露初,京城民謠有「側堂堂,橈堂堂」之言。太常丞李嗣真曰:「側者,不正;橈者,不安。自隋以來,樂府有《堂堂曲》,再言堂者,唐再受命之象。」



 永淳元年七月,東都大雨,人多殍殕。先是童謠曰:「新禾不入箱,新麥不入場,迨及八九月,狗吠空垣墻。」



 高宗自調露中欲封嵩山,屬突厥叛而止;後又欲封,以吐蕃入寇遂停。時童謠曰:「嵩山凡幾層,不畏登不得,但恐不得登,三度徵兵馬,傍道打騰騰。」



 永徽末,里歌有《桑條韋也》、《女時韋也》樂。



 龍朔中,時人飲酒令曰;「子母相去離,連臺拗倒。」俗謂杯盤為子母,又名盤為臺。又里歌有《突厥鹽》。



 永淳後,民歌曰:「楊柳楊柳漫頭駝。」



 垂拱後,東都有《契苾兒歌》,皆淫艷之詞。契苾,張易之小字也。



 如意初,里歌曰:「黃麞黃麞草裏藏,彎弓射爾傷。」其後,王孝傑敗於黃麞谷。



 神龍以後,民謠曰:「山南烏鵲窠,山北金駱駝,鐮柯不鑿孔,斧子不施柯。」山南,唐也,烏鵲窠者,人居寡也;山北,胡也,金駱駝者,虜獲而重載也。安樂公主於洺州造安樂寺,童謠曰:「可憐安樂寺,了了樹頭懸。」



 景龍中,民謠曰:「黃牸犢子挽紖斷,兩足踏地奚斷,城南黃牸犢子韋。」又有《阿緯娘歌》。時又謠曰:「可憐聖善寺,身著綠毛衣,牽來河裏飲,踏殺鯉魚兒。」



 玄宗在潞州,有童謠曰;「羊頭山北作朝堂。」



 天寶中,有術士李遐周於玄都觀院廡間為詩曰:「燕市人皆去,函關馬不歸,人逢山下鬼,環上系羅衣。」而人皆不悟,近詩妖也。又祿山未反時,童謠曰;「燕燕飛上天,天上女兒鋪白氈,氈上有千錢。」時幽州又有謠曰:「舊來誇戴竿,今日不堪看,但看五月里,清水河邊見契丹。」



 德宗時,或為詩曰;「此水連涇水,雙眸血滿川,青牛逐硃虎,方號太平年。」近詩妖也。硃泚未敗前兩月,有童謠曰:「一隻箸,兩頭硃,五六月,化為且。」



 元和初,童謠曰;「打麥打麥三三三。」乃轉身曰:「舞了也。」



 大中末,京師小兒疊布漬水紐之向日,謂之曰「拔暈」。



 咸通七年,童謠曰:「草青青,被嚴霜,鵲始後,看顛狂。」十四年,咸都童謠曰;「咸通癸巳,出無所之,蛇去馬來,道路稍開,頭無片瓦,地有殘灰。」是歲,歲陰在巳,明年在午。巳,蛇也;午,馬也。



 僖宗時,童謠曰:「金色蝦蟆爭努眼,翻卻曹州天下反。」



 乾符六年,童謠曰;「八月無霜寒草青,將軍騎馬出空城,漢家天子西巡狩,猶向江東更索兵。」



 中和初,童謠曰:「黃巢走,泰山東,死在翁家翁。」



 ○訛言



 貞觀十七年七月,民訛言官遣棖棖殺人,以祭天狗,云:其來也,身衣狗皮,鐵爪,每於暗中取人心肝而去。於是更相震怖,每夜驚擾,皆引弓斂自防,無兵器者剡竹為之,郊外不敢獨行。太宗惡之,令通夜開諸坊門,宣旨慰諭,月餘乃止。



 武后時,民飲酒謳歌,曲終而不盡者,謂之「族鹽」。



 開元二十七年十月,改作東都明堂,訛言官取小兒埋明堂下,以為厭勝。村野兒童藏於山谷,都城騷然,或言兵至。玄宗惡之,遣使尉諭,久之乃止。



 天寶三載二月辛亥,有星如月,墜於東南,墜後有聲,京師訛言官遣棖棖捕人,取肝以祭天狗,人頗恐懼,畿內尤甚,遣使安諭之,與貞觀十七年占同。



 天寶後,詩人多為憂苦流寓之思,及寄興於江湖僧寺,而樂曲亦多以邊地為名,有《伊州》、《甘州》、《涼州》等,至其曲遍繁聲,皆謂之「入破」。又有《胡旋舞》,本出康居,以旋轉便捷為巧,時又尚之。破者,蓋破碎雲。



 建中三年秋,江淮訛言有毛人食其心,人情大恐。硃泚既僭號,名其舊第曰潛龍宮,移內府珍貨以實之。占者以為,《易》稱「潛龍勿用」,此敗祥也。



 大和九年,京師訛言鄭注為上合金丹,生取小兒心肝,密旨捕小兒無算。往往陰相告曰:「某處失幾兒矣。」方士言金丹可致神仙,蓋誕妄不經之語,或信而服之,則發熱多死,如有所戒云。小兒,無辜者,取其心肝,將有殺戮象。



 劉從諫未死時,潞州有狂人折腰於市曰;「石雄七千人至矣。」從諫捕斬之。



 咸通十四年秋,成都訛言有犬夷母鬼夜入人家,民皆恐,夜則聚坐。或曰某家見鬼,眼晃然如燈焰,民益懼。



 黃巢未入京師時,都人以黃米及黑豆屑蒸食之,為之「黃賊打黑賊」。僖宗時,里巷鬥者激怒,言:「任見右廂天子。」



 毛蟲之孽。



 永徽中,河源軍有狼三,晝入軍門,射之,斃。



 永淳中,嵐、勝州兔害稼,千萬為群,食苗盡,兔亦不復見。



 開元三年,有熊晝入揚州城。



 乾元二年十月,詔百官上勤政樓觀安西兵赴陜州,有狐出於樓上,獲之。



 大歷四年八月己卯,虎入京師長壽坊宰臣元載家廟,射殺之。虎,西方之屬,威猛吞噬,刑戮之象。六年八月丁丑,獲白兔於太極殿之內廊。占曰;「國有憂。白,喪祥也。」



 建中三年九月己亥夜,虎入宣陽里,傷人二,詰朝獲之。



 貞元二年二月乙丑,有野鹿至於含元殿前,獲之;壬申,又有鹿至於含元殿前,獲之。占曰:「有大喪。」四年三月癸亥,有鹿至京師西市門,獲之。



 開成四年四月,有麞出於太廟,獲之。



 ○犬禍



 武德三年,突厥處羅可汗將入冠,夜聞犬群嗥而不見犬。



 武后初,酷吏丘神勣家狗生子皆無首,當項有孔如口,晝夜鳴吠,俄失所在。



 神功元年,安國獻兩首犬。首多者,上不一也。



 天寶十一載,李林甫晨起盥飾將朝,取書囊視之,中有物如鼠,躍於地即變為狗,壯大雄目,張牙視林甫,林甫射之,中,殺然有聲,隨箭沒。



 貞元七年,趙州柏鄉民李崇貞家黃犬乳犢。



 會昌三年,定州深澤令家狗生角。



 大中初,狗生角。京房曰:「執正失將害之應。」又曰:「君子危陷,則狗生角。」



 咸通中,會稽有狗生而不能吠,擊之無聲。狗職吠以守御,其不能者,象鎮守者不能禦寇之兆。



 成汭為荊南節度使,城中犬皆夜吠,日者向隱以為城郭將丘墟。



 中和二年秋,丹徒狗與彘交。占曰;「諸侯有謀害國者。」



 ○白眚白祥



 調露元年十一月壬午,秦州神亭治北霧開如日初耀,有白鹿、白狼見。近白祥也。



 神龍二年四月己亥,雨毛於越州之鄮縣。占曰:「邪人進,賢人遁。」



 大歷二年七月甲戌日入時,有白氣亙天。九月戊午夜,白霧起西北,亙天。五年五月甲申,西北有白氣亙天。



 貞元二十年九月庚辰甲夜,有白氣八,東西際天。



 大和三年八月,西方有白氣如柱。七年十月己酉,西方又有白氣如柱者三。



 光啟二年四月,有白氣頭黑如發,自東南入於揚州滅。



 光化二年三月乙巳,日中有白氣亙天,自西南貫於東北。



 天復元年八月己亥,西方有白雲如履底,中出白氣如匹練,長五丈,上沖天,分為三彗,頭下垂。占曰:「天下有兵。白者,戰祥也。」



 ○木沴金



 神龍中,東都白馬寺鐵像頭無故自落於殿門外。



 天寶五載四月,宰臣李適之常列鼎具膳羞,中夜,鼎躍出相鬥不解,鼎耳及足皆折。



\end{pinyinscope}