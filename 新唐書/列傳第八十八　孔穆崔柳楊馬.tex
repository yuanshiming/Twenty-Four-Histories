\article{列傳第八十八 孔穆崔柳楊馬}

\begin{pinyinscope}

 孔巢父,字弱翁,孔子三十七世孫。少力學,隱徂來山。永王璘稱兵江淮,闢署幕府探討宇宙本體、懷疑現實走向虛無主義,無政府主義。提出,不應,鏟跡民伍。璘敗,知名。廣德中,李季卿宣撫江淮,薦為左衛兵曹參軍。三遷庫部員外郎。出為涇原行軍司馬。累拜湖南觀察使,未行,會普王為荊襄副元帥,署行軍司馬。俄而德宗狩奉天,行在擢給事中,為河中、陜、華招討使,累上破賊方略,帝嘉納。



 未幾,兼御史大夫,為魏博宣慰使。巢父辯而才,及見田悅,與言君臣大義,利害逆順,開曉其眾。是時,悅久不臣,下皆厭亂,雜然喜曰:「不圖今日還為王人!」酒中,悅起,自陳騎射工,曰:「陛下見用,何敵不摧!」巢父曰:「若爾,不蚤自歸,乃一劇賊耳。」悅曰:「能為劇賊,豈不能為功臣乎?」巢父曰:「國方多虞,待子而息。」悅謝焉。數日,田緒殺悅,與大將邢曹俊等聽命,巢父即以緒權知軍務,紓其難。



 李懷光據河中,帝復令巢父宣慰,罷其兵,以太子太保授之。懷光素服待命,巢父不止。眾忿曰:「太尉無官矣!」方宣詔,乃噪而合,害巢父,並殺中人啖守盈。初,巢父至,懷光以其使魏博而田悅死,疑其謀出巢父,故軍亂不肯救。帝聞震悼,贈尚書左僕射,謚曰忠。詔具禮收葬,賜其家粟帛,存恤之。



 從子戣、戡、戢。



 戣,字君嚴,擢進士第。鄭滑盧群闢為判官,群卒,攝手怱留務。監軍楊志謙雅自肆,眾皆恐。戣邀志謙至府,與對榻臥起,示不疑,志謙嚴憚不敢動。入為侍御史,累擢諫議大夫。條上四事:一、多冗官,二、吏不奉法,三、百姓田不盡墾,四、山澤榷酤為州縣弊。憲宗異其言。中人劉希光受賕二十萬緡,抵死,吐突承璀坐厚善,逐為淮南監軍。太子舍人李涉知帝意,投匭上言承璀有功不可棄。戣得副章,不肯受,面質讓之。涉更因左右以聞,戣劾奏涉結近幸,營罔上聽。有詔斥涉峽州司馬,宦寵側目,人為危之,戣自以適所志,軒軒甚得。



 俄兼太子侍讀,改給事中。江西觀察使李少和坐贓,獄寢不下;博陵崔易簡殺從父兄,鞫狀具。京兆尹左右之,翻其情。戣慷慨論正,貶少和,殺易簡,奪尹三月俸。再遷尚書左丞。信州刺史李位好黃老道,數祠禱,部將韋岳告位集方士圖不軌,監軍高重謙上急變,捕位劾禁中。戣奏:「刺史有罪,不容系仗內,請付有司。」詔還御史臺。戣與三司雜治,無反狀。嶽坐誣罔誅,貶位建州司馬。中人愈怒,故出為華州刺史。明州歲貢淡菜蚶蛤之屬,戣以為自海抵京師,道路役凡四十三萬人,奏罷之。歷大理卿、國子祭酒。



 會嶺南節度使崔詠死,帝謂裴度曰:「嘗論罷蚶菜者,誰歟?今安在?是可往,為朕求之。」度以戣對,即拜嶺南節度使。既至,免屬州逋負十八萬緡、米八萬斛、黃金稅歲八百兩。先是,屬刺史俸率三萬,又不時給,皆取部中自衣食。戣乃倍其俸,約不得為貪暴,稍以法繩之。南方鬻口為貨,掠人為奴婢,戣峻為之禁。親吏得嬰兒於道,收育之,戣論以死,由是閭里相約不敢犯。士之斥南不能北歸與有罪之後百餘族,才可用,用之,稟無告者,女子為嫁遣之。蕃舶泊步有下碇稅,始至有閱貨宴,所餉犀琲,下及僕隸,戣禁絕,無所求索。舊制,海商死者,官籍其貲,滿三月無妻子詣府,則沒入。戣以海道歲一往復,茍有驗者不為限,悉推與。



 自貞元中,黃洞諸蠻叛,久不平。容、桂二管利虜掠,幸有功,乃請合兵討之。戣固言不可,帝不聽,大發江、湖兵,會二管入討。士被瘴毒死者不勝計,安南乘之,殺都護李象古,而桂管裴行立、容管陽旻皆無功,憂死;獨戣不邀一旦功,交、廣晏然大治。



 穆宗立,以吏部侍郎召,改右散騎常侍,還為左丞,以老自乞。雅善韓愈,謂曰:「公尚壯,上三留,何去之果?」戣曰:「吾豈要君者?吾年,一宜去;吾為左丞,不能進退郎官,二宜去。」愈曰:「公無留資,何恃而歸?」曰:「吾負二宜去,尚奚顧子言?」愈嗟嘆,即上疏言:「臣與戣同在南省,數與戣相見,其為人守節清苦,論議正平。年七十,筋力耳目未衰,憂國忘家,用意至到。如戣輩,在朝不過三數人,陛下不宜茍順其求,不留自助也。《禮》:大夫七十致仕,若不得謝,則賜之幾杖安車,不必七十盡許致仕。今戣據禮求退,陛下若不聽許,亦無傷義,而有貪賢之美。」不報。以禮部尚書致仕,歲致羊酒如漢徵士禮。卒,年七十三。贈兵部尚書,謚曰貞。



 子遵孺,溫裕,仕為天平節度使。遵孺子緯。



 緯,字化文,少孤,依諸父。多與有名者游,才譽蚤成。擢進士第,東川崔慎由表置幕府。從崔鉉淮南,復從慎由守河中,再遷觀察判官。宰相楊收薦以長安尉直弘文館。遷監察御史,進禮部員外郎、兼集賢直學士。母喪解。還為右司員外郎。趙隱言其才,拜翰林學士,俄知制誥。頻遷戶部侍郎,擢御史中丞。緯方雅,疾惡若仇,中外聞風,未繩輒肅。三遷吏部侍郎。權要私謁至盈幾,一不省,當路不悅,改太常卿。



 從僖宗西到蜀,以刑部尚書判戶部。蕭遘雅不喜,坐調度不給,改太子少保。及帝避硃玫,次陳倉,惟黃門衛士數百扈乘輿。詔拜緯御史大夫,令趣百官至行在。時群臣露次盩厔,為盜剽脅,衣囊略盡。緯謁宰相,欲有所論,遘與裴澈怨田令孜,不欲行,辭不見。緯召御史曰:「吾等身被恩,誼不辭難,今詔群臣皆不至,夫與人布衣游,猶緩急相恤,況於君乎?」且泣下。御史亦辭方寇奪,丐衣食,請辦一日費而行。緯曰:「吾妻疾,旦暮盡,丈夫豈以家事後國事乎?公善自謀,吾行決矣。」往見李昌符曰:「詔書再至,而群臣顧未行。僕,大夫也,不敢後。願假兵護送天子所。」昌符具資裝送之。既及行在,緯策玫必反,建言關邑厄狹,不足駐六師,請幸梁州。即日去陳倉而玫兵至,微緯言幾不脫。進拜兵部侍郎、同中書門下平章事。玫平,從帝還,領諸道鹽鐵轉運使,累遷尚書左僕射,賜號「持危啟運保乂功臣」。鐵券恕十死,又賜天興良田、善和里第各一區,兼京畿營田使。



 昭宗即位,進司空。以太學焚殘,乃兼國子祭酒,完治之。加司徒,封魯國公。帝將郊見,中尉樞密使索宰相朝服,有司白中人無衣冠助祭事,中尉怒,責禮官必得。緯言:「中人不朝服,國典也。陛下欲假借之,則請以所兼官為之服。」諫官固執,帝召謂曰:「方舉大禮,為我容之。」進兼太保。時天武都頭李順節,疏暴人也,以浙西節度使兼平章事。臺史白:「已謝,當班見百官。」緯判止之。明日,順節盛服至,則無班,怏怏去。他日見緯,以為言,緯曰:「固疑公見望也。且百闢卿士,天子廷臣,班見宰相,以宰相為之長。公提天武健兒,據堂受禮,安乎?必欲用之,去都頭乃可。」順節慚縮不敢言。



 張濬將伐太原,帝不決,以問緯,緯助濬請。既濬敗,坐傅會,出為荊南節度使,俄貶均州刺史。二人皆密結硃全忠,全忠為請,詔聽所便,乃屏居華陰。李茂貞入殺韋昭度,帝惡大臣朋比,與籓臣交,更召緯入朝,再擢吏部尚書,以司空、門下侍郎復輔政。使者敦勸,力疾到京師,見帝嗚咽流涕,自陳衰疾不任事,乞歸田里。帝動容,詔使者送緯至堂視事。會天子出次石門,從至莎城,以病還都。家人召醫視,緯曰:「天下方亂,何久求生?」不肯服藥,卒,贈太尉。



 戡,字勝始,進士及第,補修武尉,以大理評事佐昭義李長榮節度府。長榮死,盧從史自別將代之,留署掌書記。從史稍得志,益驕,與王承宗、田緒陰相結,欲久連兵以固其位。戡始陰爭不從,則於會肆言以折之,從史始若受其言,後偃蹇不軌,戡遂以疾歸洛陽。未幾,李吉甫鎮揚州,表置幕府,戡未應。從史曰:「是欲舍我而從人邪?」即誣以事,奏三上,詔以衛尉丞分司東都。自貞元後,帥鎮劾奏僚佐,不驗輒斥。至是,給事中呂元膺執不可。憲宗遣使諭曰:「朕非不知戡,行用之矣。」未幾,卒,年五十七。從史敗,追贈司勛員外郎。



 戢,字方舉。初,父死難,詔與一子官,補修武尉,不受,以讓其兄戡。擢明經,書判高等,為校書郎、陽翟尉,累遷殿中侍御史,分司東都。昭義判官徐玫,故嘗助盧從史為跋扈者,從史敗,孟元陽代,欲復用之。戢移書昭義前系玫,乃上列其狀。帝怒,流玫播州。轉侍御史、庫部員外郎。始,硃泚以彭偃為中書舍人,偃子充符得不死,闢鄜坊府。或薦其能,召還京師。戢謂京兆尹裴武曰:「泚所下詔令皆偃為之,悖逆子不鳥竄獸伏,乃干譽求進乎?子盍效季孫行父逐莒僕以勉事君者?」武即逐出充符。拜京兆少尹,再遷為湖南觀察使,召授右散騎常侍、京兆尹。歲旱,文宗憂甚,戢躬祠曲江池,一夕大澍,帝悅,詔兼御史大夫。卒,贈工部尚書。



 子溫業,字遜志,擢進士第。大中時,為吏部侍郎。求外遷,宰相白敏中顧同列曰:「吾等可少警,孔吏部不樂居朝矣。」後為太子賓客。



 穆寧,懷州河內人。父元休,有名開元間,獻書天子,擢偃師丞,世以儒聞。寧剛正,氣節自任。以明經調鹽山尉。安祿山反,署劉道玄為景城守,寧募兵斬之,檄州縣並力捍賊。史思明略境,郡守召寧攝東光令御之。賊遣使誘寧,寧斬以徇,郡守恐怒賊令致死,即奪其兵,罷所攝。始,寧過平原,見顏真卿,嘗商賊必反。及是,聞真卿拒祿山,即遺真卿書曰:「夫子為衛君乎?」真卿喜,署寧河北採訪支使。寧以息屬其母弟曰:「茍不乏嗣,足矣!」即馳謁真卿曰:「先人有嗣矣,我可從公死。」既而賊攻平原,寧勸固守,真卿不從,夜亡過河,見肅宗行在。帝問狀,真卿對:「不用穆寧言,故至此。」帝異之,馳驛召寧,將以諫議大夫任之。會真卿以直忤旨,寧亦罷。



 上元初,為殿中侍御史,佐鹽鐵轉運,住埇橋。李光弼屯徐州,餉不至,檄取資糧,寧不與。光弼怒,召寧欲殺之。或勸寧去,寧曰:「避之失守,亂自我始,何所逃罪乎?」即往見光弼。光弼曰:「吾師眾數萬,為天子討賊,食乏則人散,君閉廩不救,欲潰吾兵耶?」答曰:「命寧主糧者,敕也,公可以檄取乎?今公求糧,而寧專饋;寧有求兵,而公亦專與乎?」光弼執其手謝曰:「吾固知不可,聊與君議耳。」時重其能守官。累遷鄂岳沔都團練及租庸鹽鐵轉運使。當是時,河漕不通,自漢、沔徑商山以入京師。淮西節度使李忠臣不奉法,設戍邏以征商賈,又縱兵剽行人,道路幾絕。與寧夾淮為治,憚寧威,掠劫為衰,漕賈得通。坐杖死沔州別駕,貶平集尉。



 大歷初,起為監察御史,三遷檢校秘書少監,兼和州刺史,治有狀。後刺史疾之,以天寶舊版校見戶,妄劾寧多逋亡,貶泉州司戶參軍事。子質訴其枉,三年始得通。詔御史覆視,實增戶數倍。召入拜太子右諭德。寧性不能事權右,毅然寡合,執政者惡之,雖直其誣,猶置散位。寧默不樂,唶曰:「時不我容,我不時徇,又可以進乎!」遂移疾,滿百日屢矣,親友強之,輒復一朝。德宗在奉天,奔詣行在,擢秘書少監,改太子右庶子。帝還京師,乃曰:「可以行吾志矣!」即罷歸東都。以秘書監致仕,卒。



 寧居家嚴,事寡姊恭甚。嘗撰家令訓諸子,人一通。又戒曰:「君子之事親,養志為大,吾志直道而已。茍枉而道,三牲五鼎非吾養也。」疾病不嘗藥,時稱知命。



 四子:贊、質、員、賞。寧之老,贊為御史中丞,質右補闕,員侍御史,賞監察御史,皆以守道行誼顯。先是,韓休家訓子侄至嚴。貞元間,言家法者,尚韓、穆二門云。



 贊,字相明,擢累侍御史,分司東都。陜虢觀察使盧嶽妻分貲不及妾子,妾訴之。中丞盧佋欲重妾罪,贊不聽。佋與宰相竇參共誣贊受金,捕送獄。弟賞上冤狀,詔三司覆治,無之,猶出為郴州刺史。參敗,召為刑部郎中,對延英,擢御史中丞。裴延齡判度支,屬吏受賕,具獄,欲曲貸吏,贊執不可。延齡白贊深文,貶饒州別駕。久之,拜州刺史。憲宗立,進宣歙觀察使,卒於官。贈工部尚書。



 質,性強直,舉賢良方正,條對詳切,頻擢至給事中,政事得失,未嘗不盡言。元和時,鹽鐵、轉運諸院擅系囚,笞掠嚴楚,人多死。質奏請與州縣吏參決,自是不冤。後論吐突承璀不宜為將,憲宗不悅,改太子左庶子。坐與楊憑善,出為開州刺史,卒。



 員,字與直,工為文章。杜亞留守東都,置佐其府,蚤卒。



 兄弟皆和粹,世以珍味目之:贊少俗,然有格,為「酪」;質美而多入,為「酥」;員為「醍醐」;賞為「乳腐」云。



 崔邠,字處仁,貝州武城人。父倕,三世一爨,當時言治家者推其法。至德初,獻賦行在,肅宗異其文,位吏部侍郎。



 邠第進士,復擢賢良方正,授渭南尉,遷補闕。上疏論裴延齡奸,以鯁亮知名。由中書舍人再遷吏部侍郎。性溫裕深密,行己又簡儉,憲宗器之,裴垍亦薦邠材可宰相。會病,遂不拜。久乃為太常卿,知吏部尚書銓。故事,太常始視事,大閱四部樂,都人縱觀。邠自第去帽,親導母輿,公卿見者皆避道,都人榮之。以母憂解,卒於喪,年六十。贈吏部尚書,謚曰文簡。



 弟酆、郾、郇、鄯、鄲。



 郾,字廣略,姿儀偉秀,人望而慕之,然不可狎也。中進士第,補集賢校書郎。累遷吏部員外郎,下不敢欺,每擬吏,親挾格,褒黜必當,寒遠無留才。三遷諫議大夫。穆宗立,荒於游畋,內酣蕩,昕曙不能朝。郾進曰:「十一聖之功德,四海之大,萬國之眾,其治其亂,系於陛下。自山以東百城,地千里,昨日得之,今日失之。西望戎壘,距宗廟十舍,百姓憔悴,畜積無有。願陛下親政事以幸天下。」帝動容慰謝,遷給事中。



 敬宗嗣位,拜翰林侍講學士,旋進中書舍人,謝曰:「陛下使臣侍講,歷半歲不一問經義。臣無功,不足副厚恩。」帝慚曰:「朕少間當請益。」高釴適在旁,因言:「陛下樂善而無所咨詢,天下之人不知有響儒意。」帝重咎謝,咸賜錦、幣。郾與高重類《六經》要言為十篇,上之,以便觀省。



 遷禮部侍郎,出為虢州觀察使。先是,上供財乏,則奪吏奉助輸,歲率八十萬。郾曰:「吏不能贍私,安暇恤民?吾不能獨治,安得自封?」即以府常費代之。又詔賦粟輸太倉者,歲數萬石,民困於輸,則又輦而致之河。郾乃旁流為大敖受粟,竇而注諸艚。民悅,忘輸之勞。改鄂、岳等州觀察使。自蔡人叛,鄂、岳常苦兵,江湖盜賊顯行。郾修治鎧仗,造蒙沖,駛追窮躡,上下千里,歲中悉捕平。又觀察浙西,遷檢校禮部尚書,卒於官。贈吏部尚書,謚曰德。



 郾不藏貲,有輒周給親舊,為治其昏喪。居家怡然,不訓子弟,子弟自化。室處庳漏,無步廡,至霖淖,則客蓋而屐以就外位。治虢以寬,經月不笞一人。及涖鄂,則嚴法峻誅,一不貸。或問其故,曰:「陜土瘠而民勞,吾撫之不暇,猶恐其擾;鄂土沃民剽,雜以夷俗,非用威莫能治。政所以貴知變者也。」聞者服焉。



 五子:瑤、瑰、瑾、珮、璆。瑤任禮部侍郎、浙西鄂岳觀察使。瑾禮部侍郎、湖南觀察使。瑰、珮俱達官。



 鄯,擢進士,累遷至左金吾衛大將軍,暴卒,以韓約代之。不閱旬,李訓亂,約死於難。世謂鄯之亡,崔氏積善報也。贈禮部尚書。



 鄲及進士第,補渭南尉。累除刑部郎中,出副杜元穎西川節度府。召入為工部侍郎、集賢殿學士。再遷吏部侍郎,由宣歙觀察使入為太常卿。文宗末,擢同中書門下平章事,改中書侍郎,罷為劍南西川節度使。宣宗初,以檢校尚書右僕射同平章事,節度淮南,卒於軍。



 崔氏四世緦麻同爨,兄弟六人至三品,邠、郾、鄲凡為禮部五,吏部再,唐興無有也。居光德里,構便齋,宣宗聞而嘆曰:「鄲一門孝友,可為士族法。」因題曰「德星堂」。後京兆民即其里為「德星社」云。



 柳公綽,字寬,京兆華原人。始生三日,伯父子華曰:「興吾門者,此兒也。」因小字起之。幼孝友,性質嚴重,起居皆有禮法。屬文典正,不讀非聖書。舉賢良方正直言極諫,補校書郎。間一年,再登其科,授渭南尉。歲歉饉,其家雖給,而每飯不過一器,歲豐乃復。或問之,答曰:「四方病饑,獨能飽乎?」累遷開州刺史,地接夷落,寇常逼其城,吏曰:「兵力不能制,願以右職署渠帥。」公綽曰:「若同惡邪?何可撓法!」立誅之,寇亦引去。遷侍御史、吏部員外郎。時武元衡節度劍南,與裴度俱為判官,尤相引重。召為吏部郎中。



 憲宗喜武功,且數出游畋,公綽奏《太醫箴》以諷曰:「天布寒暑,不私於人。品類既一,高卑以均。人謹好愛,能保其身。清靜無瑕,輝光以新。寒暑滿天地,浹肌膚於外;好愛在耳目,誘心知於內。端潔為堤,奔射猶敗。氣行無間,隙不在大。謂天高矣,氛蒙晦之;謂地厚矣,橫流潰之。飲食資身,過則生患;衣服稱德,侈則生慢。唯過與侈,心必隨之。氣與心流,疾乃伺之。畋游恣樂,流情蕩志。馳騁勞形,叱吒傷氣。不養其外,前脩所忌。人乘氣生,嗜欲以萌。氣離有患,氣完則成。巧必喪真,智實誘情。醫之上者,理於未然。患居慮後,防處事先。心靜樂行,體和道全。克施萬物,以享億年。聖人在上,各有攸處。臣司太醫,敢告諸御。」天子高其才,遣使謂曰:「卿言『氣行無間,隙不在大』,愛朕深者,當置之坐隅。」逾月,拜御史中丞。



 公綽本與裴垍善,李吉甫復當國,出為湖南觀察使。以地卑濕,不可迎養,求分司東都,不聽。後徙鄂岳觀察使。時方討吳元濟,詔發鄂岳卒五千,隸安州刺史李聽。公綽曰:「朝廷謂吾儒生不知兵邪!」即請自行,許之。引兵度江,抵安州,聽以軍禮迎謁。公綽謂曰:「公所以屬鞬負弩,豈非兵事邪?若褫戎容,則兩郡守耳,何所統壹哉?以公世將曉兵,吾且欲署職,以兵法從事。」聽曰:「唯命。」即以都知兵馬使、中軍先鋒、行營都虞候三牒授之,選兵六千屬焉,戒諸校曰:「行營事一決都將。」聽被用畏威,遂盡力,當時服其知權。軍出,公綽數省問其家,疾病生死厚給之,婦人敖蕩者,沉之江。軍中感服曰:「中丞為我知家事,敢不死戰!」故鄂軍每戰輒克。



 元和十一年,為李道古代還,除給事中。李師道平,遣宣諭鄆州,復命,拜京兆尹。方赴府,有神策校乘馬不避者,即時搒死。帝怒其專殺,公綽曰:「此非獨試臣,乃輕陛下法。」帝曰:「既死,不以聞,可乎?」公綽曰:「臣不當奏。在市死,職金吾;在坊死,職左右巡使。」帝乃解。以母喪去官。服除,為刑部侍郎,領鹽鐵轉運使,轉兵部,兼御史大夫。



 長慶元年,復為京兆尹。時幽、鎮用兵,補置諸將,使驛系道。公綽奏曰:「比館遞匱乏,驛置多闕。敕使衣緋紫者,所乘至三四十騎;黃綠者,不下十數。吏不得視券,隨口輒供。驛馬盡,乃掠奪民馬。怨嗟驚擾,行李殆絕。請著定限,以息其弊。」有詔中書條檢定數,由是吏得紓罪。宦官共惡疾之。改吏部侍郎,遷御史大夫。韓弘病,自河中還,詔百官問疾,弘遣子辭不能見,公綽謂曰:「上使百司省候,是謂異禮,宜力疾以見公卿,安可臥令子姓傳言耶?」弘懼,挾扶以出。



 改禮部尚書,以祖諱換左丞。俄檢校戶部尚書、山南東道節度使。行部至鄧,縣吏有納賄、舞文二人同系獄,縣令以公綽素持法,謂必殺貪者,公綽判曰:「贓吏犯法,法在;奸吏壞法,法亡。」誅舞文者。其廄馬害圉人,公綽殺之。或言良馬可愛,曰:「安有良馬而害人乎?」



 寶歷元年,就遷檢校左僕射。牛僧孺罷政事,為武昌節度使,公綽具軍容伏謁,左右諫止之,答曰:「奇章始去臺宰,方鎮重宰相,所以尊朝廷也。」有道士獻丹藥,問所從來,曰:「自薊門。」時硃克融方叛,遽曰:「惜哉,藥自賊境來,雖驗何益!」即棄藥而逐道士。入為刑部尚書,俄拜邠寧節度使。先是神策諸鎮列屯部中,不聽本道節制,故虜得窺間。公綽論所宜,因詔屯營緩急悉受節度。復為刑部尚書。京兆獄有姑鞭婦至死者,府欲殺之。公綽曰:「尊毆卑,非鬥也;且子在,以妻而戮其母,不順。」遂減論。



 太和四年,為河東節度。遭歲惡,撙節用度,輟宴飲,衣食與士卒鈞。北虜遣梅祿將軍李暢以馬萬匹來市,所過皆厚勞,飭兵以防襲奪。至太原,公綽獨使牙將單騎勞問,待以至意,闢牙門,令譯官引謁,宴不加常。暢德之,出涕,徐驅道中,不妄馳獵。陘北有沙陀部,勇武喜鬥,為九姓、六州所畏。公綽召其酋硃邪執宜,治廢柵十一,募兵三千留屯塞上,其妻、母來太原者,令夫人飲食問遺之。沙陀感恩,故悉力保鄣。



 以病乞代,授兵部尚書,不任朝請。忽顧左右召故吏韋長,眾謂屬諉以家事。及長至,乃曰:「為我白宰相,徐州專殺李聽親吏,非用高瑀不能安。」因瞑目不復語,後二日卒,年六十八。贈太子太保,謚曰元。



 公綽居喪毀慕,三年不澡沐。事後母薛謹甚,雖姻屬不知非薛所生。外兄薛宮早卒,為育其女嫁之。嘗曰:「吾蒞官未嘗以私喜怒加於人,子孫其昌乎!」與錢徽、蔣乂、杜元穎、薛存誠善,取士如許康佐、鄭朗、盧簡辭、崔璵、夏侯孜、李拭、韋長,皆知名顯貴云。



 子仲郢,字諭蒙。母韓,即皋女也,善訓子,故仲郢幼嗜學,嘗和熊膽丸,使夜咀咽以助勤。長工文,著《尚書二十四司箴》,為韓愈咨賞。元和末,及進士第,為校書郎。牛僧孺闢武昌幕府,有父風矩,僧孺嘆曰:「非積習名教,安及此邪?」入為監察御史,遷侍御史。有禁卒誣里人斫父墓柏,射殺之,吏以專殺論,而中尉護免其死,右補闕蔣系爭,不省。仲郢監罰,執曰:「賊不死,是亂典刑。」有詔御史蕭傑監之,傑復爭。遂獨詔京兆杖之,不監。朝廷嘉其守。



 會昌初,累轉吏部郎中。時詔減官冗長者,仲郢條簡浹日,損千二百五十員,議者厭伏。遷左諫議大夫。武宗延方士,築望仙臺,累諫諄切,帝遣中人愧諭。御史崔元藻以覆按吳湘獄得罪,仲郢切諫,宰相李德裕不為嫌,奏拜京兆尹。置權量於東西市,使貿易用之,禁私制者。北司吏入粟違約,仲郢殺而尸之,自是人無敢犯,政號嚴明。會廢浮屠法,盡壞銅象為錢。仲郢為鑄錢使,吏請以字識錢者,不答。既,淮南鑄會昌字,久之,僧反取為鐘鈸云。中書舍人紇干柷訴甥劉詡毆其母,詡為禁軍校,仲郢不待奏,即捕取之,死杖下,宦官以為言,改右散騎常侍,知吏部銓。德裕頗抑進士科,仲郢無所徇。是時,以進士選,無受惡官者。又當調者,持闕簿令自閱,即擬唱,吏無能為奸。



 宣宗初,德裕罷政事,坐所厚善,出為鄭州刺史。周墀鎮滑,而鄭為屬郡,高其績;及入相,薦授河南尹,召拜戶部侍郎。墀罷,它宰相惡仲郢,左遷秘書監。數月,復出河南尹,以寬惠為政。或言不類京兆時,答曰:「輦轂之下,先彈壓;郡邑之治,本惠養。烏可類乎?」擢劍南東川節度使。大吏邊章簡挾勢肆貪,前帥不能制,仲郢因事殺之,官下肅然。居五年,召為吏部侍郎,俄改兵部,領鹽鐵轉運使。有劉習者,以藥術進,詔署鹽官。仲郢以為醫有本色官,若委錢穀,名分不正。帝悟,乃賜縑遣還。



 大中十二年,辭疾,以刑部尚書罷使,轉戶部,封河東縣男,為山南西道節度使。南鄭令權弈以罪,仲郢杖之,六日死,貶雷州刺史。頃之,以太子賓客分司東都,起為虢州刺史,以檢校尚書左僕射東都留守。會盜發父墓,棄官歸華原。徙華州刺史,不拜。咸通五年,為天平節度使。初,仲郢為諫議大夫,後每遷,必烏集升平第,庭樹戟架皆滿,五日乃散。及是不復集。卒於鎮。



 仲郢方嚴,尚氣義,事親甚謹。李德裕貶死,家無祿,不自振;及領鹽鐵,遂取其兄子從質為推官,知蘇州院。宰相令狐綯持不可,乃移書開諭綯,綯感悟,從之。每私居內齋,束帶正色,服用簡素。父子更九鎮,五為京兆,再為河南,皆不奏瑞,不度浮屠。急於摘貪吏,濟單弱。每旱潦,必貸匱蠲負,里無逋家。衣冠孤女不能自歸者,斥稟為婚嫁。在朝,非慶吊不至宰相第。其跡略相同。



 家有書萬卷,所藏必三本:上者貯庫,其副常所閱,下者幼學焉。仲郢嘗手鈔《六經》,司馬遷、班固、範曄史皆一鈔,魏晉及南北朝史再,又類所鈔它書凡三十篇,號《柳氏自備》;旁錄仙佛書甚眾,皆楷小精真,無行字。



 子璞、珪、璧、玭。



 璞,字韜玉,學不營仕。著《春秋三氏異同義》,又述《天祚長歷》,斷自漢武帝紀元,為編年,以大政、大祥異、侵叛戰伐隨著之,閏位者附見其左,常謂「杜征南《春秋後序》述紀甲歷為得實,自餘史家皆差」,蔣系以為然。終著作郎。



 珪,字交玄。大中中,與璧繼擢進士,皆秀整而文,杜牧、李商隱稱之。杜悰鎮西川,表在幕府,久乃至。會悰徙淮南,歸其積俸,珪不納,悰舉故事為言,卒辭之。以藍田尉直弘文館,遷右拾遺,而給事中蕭仿、鄭裔綽謂珪不能事父,封還其詔。仲郢訴其子「冒處諫職為不可,謂不孝則誣。請勒就養」,詔可。始,公綽治家埒韓滉,及珪被廢,士人愧悵。終衛尉少卿。



 璧,字賓玉。馬植鎮汴州,闢管書記。又從李瓚桂州,規止其不法,瓚不聽,乃拂衣去。未幾,軍亂。擢右補闕,再轉屯田員外郎。僖宗幸蜀,授翰林學士,累遷右諫議大夫。



 玭以明經補秘書正字,由書判拔萃,累轉左補闕。高湜再鎮昭義,皆表為副,擢刑部員外郎。湜貶高要尉,玭三疏申理。湜後得稿嗟嘆,以為其言雖自辨不加也。出為嶺南節度副使。廨中橘熟,既食,乃納直於官。黃巢陷交、廣,逃還,除起居郎。巢入京師,奔行在,再遷中書舍人、御史中丞。文德元年,以吏部侍郎脩國史,拜御史大夫。直清有父風,昭宗欲倚以相,中官譖玭煩碎,非廊廟器,乃止。坐事貶瀘州刺史,卒。光化初,帝自華還,詔復官爵。



 玭嘗述家訓以戒子孫曰:



 夫門地高者,一事墜先訓,則異它人,雖生可以茍爵位,死不可見祖先地下。門高則自驕,族盛則人窺嫉。實蓺懿行,人未必信;纖瑕微累,十手爭指矣。所以修己不得不至,為學不得不堅。夫士君子生於世,己無能而望它人用,己無善而望它人愛,猶農夫鹵莽種之而怨天澤不潤,雖欲弗餒,可乎?余幼聞先公僕射言:立己以孝悌為基,恭默為本,畏怯為務,勤儉為法。肥家以忍順,保交以簡恭,廣記如不及,求名如儻來。蒞官則絜己省事,而後可以言家法;家法備,然後可以言養人。直不近禍,廉不沽名。憂與禍不偕,絜與富不並。董生有云:「吊者在門,賀者在閭。」言憂則恐懼,恐懼則福至。又曰:「賀者在門,吊者在閭。」言受福則驕奢,驕奢則禍至。故世族遠長與命位豐約,不假問龜蓍星數,在處心行事而已。



 昭國裏崔山南琯子孫之盛,仕族罕比。山南曾祖母長孫夫人年高無齒,祖母唐夫人事姑孝,每旦,櫛縰笄拜階下,升堂乳姑,長孫不粒食者數年。一日病,言無以報吾婦,冀子孫皆得如婦孝。然則崔之門安得不大乎?東都仁和裏裴尚書寬子孫眾盛,實為名閥。天后時,宰相魏玄同選尚書之先為婿,未成婚而魏陷羅織獄,家徙嶺表。及北還,女已逾笄。其家議無以為衣食資,願下發為尼。有一尼自外至,曰:「女福厚豐,必有令匹,子孫將遍天下,宜北歸。」家人遂不敢議。及荊門,則裴齎裝以迎矣。今勢利之徒,舍信誓如返掌,則裴之蕃衍,乃天之報施也。餘舊府高公先君兄弟三人,俱居清列,非速客不二羹胾,夕食,齕蔔瓠而已,皆保重名於世。



 永寧王相國涯居位,竇氏女歸,請曰:「玉工貨釵直七十萬錢。」王曰:「七十萬錢,豈於女惜?但釵直若此,乃妖物也,禍必隨之。」女不復敢言。後釵為馮球外郎妻首飾,涯曰:「為郎吏妻,首飾有七十萬錢,其可久乎!」馮為賈相國餗門人,賈有奴頗橫,馮愛賈,召奴責之,奴泣謝。未幾,馮晨謁賈,賈未出,有二青衣齎銀罌出,曰:「公恐君寒,奉地黃酒三杯。」馮悅,盡舉之。俄病渴且咽,因暴卒。賈為嘆息出涕,卒不知其由。明年,王、賈皆遘禍。噫,王以珍玩為物之妖,信知言矣,而不知恩權隆赫之妖甚於物邪?馮以卑位貪貨,不能正其家,忠於所事,不能保其身,不足言矣。賈之奴害客於墻廡間而不知,欲始終富貴,其得乎?舒相國元輿與李繁有隙,為御史,鞫譙獄,窮致繁罪,後舒亦及禍。今世人盛言宿業報應,曾不思視履考祥事歟?夫名門右族,莫不由祖考忠孝勤儉以成立之,莫不由子孫頑率奢傲以覆墜之。成立之難如升天,覆墜之易如燎毛。



 餘家本以學識禮法稱於士林,比見諸家於吉兇禮制有疑者,多取正焉。喪亂以來,門祚衰落,基構之重,屬於後生。夫行道之人,德行文學為根株,正直剛毅為柯葉。有根無葉,或可俟時;有葉無根,膏雨所不能活也。至於孝慈、友悌、忠信、篤行,乃食之醢醬,可一日無哉?



 其大概如此。



 公權,字誠懸,公綽弟也。年十二,工辭賦。元和初,擢進士第。李聽鎮夏州,表為掌書記。因入奏,穆宗曰:「朕嘗於佛廟見卿筆跡,思之久矣。」即拜右拾遺、侍書學士,再遷司封員外郎。帝問公權用筆法,對曰:「心正則筆正,筆正乃可法矣。」時帝荒縱,故公權及之。帝改容,悟其以筆諫也。公綽嘗寓書宰相李宗閔,言家弟本志儒學,先朝以侍書見用,頗類工祝,願徙散秩。乃改右司郎中、弘文館學士。



 文宗復召侍書,遷中書舍人,充翰林書詔學士。嘗夜召對子亭,燭窮而語未盡,宮人以蠟液濡紙繼之。從幸未央宮,帝駐輦,曰:「朕有一喜,邊戍賜衣久不時,今中春而衣已給。」公權為數十言稱賀,帝曰:「當賀我以詩。」宮人迫之,公權應聲成,文婉切而麗。詔令再賦,復無停思,天子甚悅,曰:「子建七步,爾乃三焉。」常與六學士對便殿,帝稱漢文帝恭儉,因舉袂曰:「此三澣矣!」學士皆賀,獨公權無言。帝問之,對曰:「人主當進賢退不肖,納諫諍,明賞罰。服澣濯之衣,此小節耳,非有益治道者。」異日,與周墀同對,論事不阿,墀為惴恐,公權益不奪,帝徐曰:「卿有諍臣風,可屈居諫議大夫。」乃自舍人下遷,仍為學士知制誥。



 開成三年,轉工部侍郎。召問得失,因言:「郭旼領邠寧,而議者頗有臧否。」帝曰:「旼,尚父從子,太皇太后季父,官無玷郵,自大金吾位方鎮,何所更議?」答曰:「旼誠勛舊,然人謂獻二女乃有是除,信乎?」帝曰:「女自參承太后,豈獻哉?」公權曰:「疑嫌間不可戶曉。」因引王珪諫廬江王妃事。是日,帝命中官自南內送女還旼家。其忠益多類此。遷學士承旨。



 武宗立,罷為右散騎常侍。宰相崔珙引為集賢院學士,知院事。李德裕不悅,左授太子詹事,改賓客。累封河東郡公,復為常侍,進至太子少師。大中十三年,天子元會,公權稍耄忘,先群臣稱賀,占奏忽謬,御史劾之,奪一季俸,議者恨其不歸事。咸通初,乃以太子太保致仕。卒,年八十八。贈太子太師。



 公權博貫經術,於《詩》、《書》、《左氏春秋》、《國語》、莊周書尤邃,每解一義,必數十百言。通音律,而不喜奏樂,曰:「聞之令人驕怠。」其書法結體勁媚,自目一家。文宗嘗召與聯句,帝曰:「人皆苦炎熱,我愛夏日長。」公權屬曰:「薰風自南來,殿閣生微涼。」它學士亦屬繼,帝獨諷公權者,以為詞情皆足,命題於殿壁,字率徑五寸,帝嘆曰:「鐘、王無以尚也!」其遷少師,宣宗召至御座前,書紙三番,作真、行、草三體,奇秘,賜以器幣,且詔自書謝章,無限真、行。當時大臣家碑志,非其筆,人以子孫為不孝。外夷入貢者,皆別署貨貝曰:「此購柳書。」嘗書京兆西明寺《金剛經》,有鐘、王、歐、虞、褚、陸諸家法,自為得意。凡公卿以書貺遺,蓋鉅禹,而主藏奴或盜用。嘗貯杯盂一笥,縢識如故而器皆亡,奴妄言叵測者,公權笑曰:「銀杯羽化矣!」不復詰。唯研、筆、圖籍,自鐍秘之。



 子華,公綽諸父也。始闢嚴武劍南府,累遷池州刺史。代宗將幸華清宮,先命完葺,欲以子華為京兆少尹,尹惡其剛方,沮解之,遂為昭應令、檢校金部郎中、修宮使。設棘圍於市,徇邑中曰:「民有得華清瓦石材用,投圍中,逾三日不還者死。」不終日,已山積矣,營辦略足。宰相元載有別墅,以奴主務,自稱郎將,怙勢縱暴,租賦未嘗入官。子華因奴入謁,收付獄,劾發宿罪,杖殺之,一邑震伏。載不敢怨,遣吏厚謝。預知其終,自為墓銘。



 子公器、公度。公度善攝生,年八十餘,有強力。常云:「吾初無術,但未嘗以氣海暖冷物,熟生物,不以元氣佐喜怒耳。」位光祿少卿。公器生遵,遵生燦,別有傳。



 楊於陵,字達夫,本漢太尉震之裔。父太清,倦宦,客河朔,死安祿山之亂。於陵始六歲,間關至江左,逮長,有奇志。十八擢進士,調句容主簿。節度使韓滉剛嚴少許可,獨奇於陵,謂妻柳曰:「吾求佳婿,無如於陵賢。」因以妻之。闢鄂岳、江西使府。滉居宰相,領財賦,權震中外。於陵隨府罷,避親不肯調,退廬建昌,以文書自娛樂。滉卒,乃入為膳部員外郎。以吏部判南曹,選者恃與宰相親,文書不如式,於陵駁其違,宰相怒,以南曹郎出使吊宣武軍。未幾,遷右司郎中,換吏部,出為絳州刺史。德宗雅聞其名,留拜中書舍人。時京兆李實恃恩暴橫,於陵與所善許孟容不離附,為所譖短,徙秘書少監。帝崩,宣遺詔於太原、幽州,節府獻遺無所納。拜華州刺史,遷浙東觀察使。越人饑,請出米三十萬石抍贍貧民,政聲流聞。



 入為京兆尹。先是,編民多竄北軍籍中,倚以橫閭里。於陵請限丁制,減三丁者不得著籍,奸人無所影賴,京師豪右大震。遷戶部侍郎。元和初,牛僧孺等以賢良方正對策,於陵被詔程其文,居第一,宰相惡其言,出為嶺南節度使。闢韋詞、李翱等在幕府,咨訪得失,教民陶瓦易蒲屋,以絕火患。監軍許遂振者,悍戾貪肆,憚於陵,不敢撓以私,則為飛語聞京師,憲宗不能無惑,有詔罷歸。遂振領留事,笞吏剔抉其贓,吏呼曰:「楊公尚拒他方賂遺,肯私官錢邪?」宰相裴垍亦為帝別白言之,乃授吏部侍郎,而遂振終得罪。



 初,吏部程判,別詔官參考,齊抗當國,罷之。至是,尚書鄭餘慶移疾,乃循舊制。於陵建言:「他官但第判能否,不知限員,有司計員為留遣之格,事不相謀,莫如勿置。」於是有詔三考官止較科目選,至常調悉還吏部。又請修甲歷,南曹置別簿相檢實,吏不能為奸。始奏選者納直給符告,居四年,凡調三千員,時謂為適。



 以兵部兼御史大夫,判度支。王師討淮西,於陵用所親為供軍使,主唐、鄧,而高霞寓騰牒度支,以餉道乏。及戰敗,詔責之,指以為言。帝怒,貶於陵郴州刺史。徙原王傅,復以戶部侍郎知吏部選。李師道平,詔宣慰淄青。朝廷始議分其地,而劉悟節度滑州,未出鄆,於陵趣使上道。還奏,帝悅其能。會浙西觀察使李閹死,皇甫鎛素忌於陵,薦以代翛,帝不之可。穆宗立,遷戶部尚書,為東都留守。數上疏乞身,不許。授太子少傅,封弘農郡公。俄以尚書左僕射致仕,詔賜實俸,讓不受。於陵器量方峻,進止有常度,節操堅明,始終不失其正,時人尊仰之。太和四年卒,年七十八。冊贈司空,謚曰貞孝。



 四子:景復仕至同州刺史,紹復中書舍人,師復大理卿,中子嗣復位宰相,自有傳。



 馬總,字會元,系出扶風。少孤窶,不妄交游。貞元中,闢署滑州姚南仲幕府,監軍薛盈珍誣南仲不法,總坐貶泉州別駕。盈珍入用事,福建觀察使柳冕希旨欲誅之,會刺史穆贊保護,乃免。徙恩王傅。



 元和中,以虔州刺史遷安南都護,廉清不撓,用儒術教其俗,政事嘉美,獠夷安之。建二銅柱於漢故處,鑱著唐德,以明伏波之裔。徙桂管經略觀察使,入為刑部侍郎。十二年,兼御史大夫,副裴度宣慰淮西。吳元濟禽,為彰義節度留後。蔡人習偽惡,相掉訐,獷戾有夷貊風。總為設教令,明賞罰,磨治洗汰,其俗一變。始奏改彰義為淮西,尋擢拜淮西節度使,徙忠武,改華州防禦、鎮國軍使。李師道平,析鄆、曹、濮等為一道,除總節度,賜號天平軍。



 長慶初,劉總上幽、鎮地,詔總徙天平,而召手怱還,將大用之。會總卒,穆宗以鄆人附賴總,復詔還鎮。二年,檢校尚書左僕射,入為戶部尚書。總篤學,雖吏事倥傯,書不去前,論著頗多。卒,贈右僕射,謚曰懿。



 贊曰:巢父恃正義,觸群不肖,謀不以權,遂喪其身。寧、邠皆所謂邦之司直者,後世卒蕃衍。公綽仁而勇,於陵方重,總沈懿,皆有大臣風,才堪宰相而用不至,果時有不幸邪?穆、崔、柳代為孝友聞家,君子之澤遠哉!



\end{pinyinscope}