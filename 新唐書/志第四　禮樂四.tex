\article{志第四 禮樂四}

\begin{pinyinscope}

 其非常祀,天子有時而行之者,曰封禪、巡守、視學、耕藉、拜陵。



 《文中子》曰:「封禪,非古也,其秦、漢之侈心乎?蓋其曠世不常行,而於禮無所本,故自漢以來,儒生學官論議不同,而至於不能決,則出於時君率意而行之爾。隋文帝嘗令牛弘、辛彥之等撰定儀注,為壇泰山下,設祭如南郊而已,未嘗升山也。



 唐太宗已平突厥,而年穀屢豐,群臣請封泰山。太宗初頗非之,已而遣中書侍郎杜正倫行太山上七十二君壇跡,以是歲兩河大水而止。其後群臣言封禪者多,乃命秘書少監顏師古、諫議大夫硃子奢等集當時名儒博士雜議,不能決。於是左僕射房玄齡、特進魏徵、中書令楊師道博採眾議奏上之,其議曰:「為壇於泰山下,祀昊天上帝。壇之廣十二丈,高丈二尺。玉牒長一尺三寸,廣、厚五寸。玉檢如之,厚減三寸。其印齒如璽,纏以金繩五周。玉策四,皆長一尺三寸,廣寸五分,厚五分,每策皆五簡,聯以金。昊天上帝配以太祖,皇地祇配以高祖。已祀而歸格於廟,盛以金匱。匱高六寸,廣足容之,制如表函,纏以金繩,封以金泥,印以受命之璽。而玉牒藏於山上,以方石三枚為再累,纏以金繩,封以石泥,印以受命之璽。其山上之圓壇,土以五色,高九尺,廣五丈,四面為一階。天子升自南階,而封玉牒。已封,而加以土,築為封,高一丈二尺,廣二丈。其禪社首亦如之。其石檢封以受命璽,而玉檢別制璽,方一寸二分,文如受命璽。以石距非經,不用。又為告至壇,方八十一尺,高三尺,四出陛,以燔柴告至,望秩群神。」遂著於禮,其他降禪、朝覲皆不著。至十五年,將東幸,行至洛陽,而彗星見,乃止。



 高宗乾封元年,封泰山,為圓壇山南四里,如圓丘,三壝,壇上飾以青,四方如其色,號封祀壇。玉策三,以玉為簡,長一尺二寸,廣一寸二分,厚三分,刻而金文。玉匱一,長一尺三寸,以藏上帝之冊;金匱二,以藏配帝之冊,纏以金繩五周,金泥、玉璽,璽方一寸二分,文如受命璽。石感:方石再累,皆方五尺,厚一尺,刻方其中以容玉匱。感旁施檢,刻深三寸三分,闊一尺,當繩刻深三分,闊一寸五分。石檢十枚,以檢石感,皆長三尺,闊一尺,厚七分;印齒三首,皆深四寸,當璽方五寸,當繩闊一寸五分。檢立於感旁,南方、北方皆三,東方、西方皆二,去感隅皆一尺。感纏以金繩五周,封以石泥。距石十二,分距感隅,皆再累,皆闊二尺,長一丈,斜刻其首,令與感隅相應。又為壇於山上,廣五丈,高九尺,四出陛,一壝,號登封壇。玉牒、玉檢、石感、石距、玉匱、石檢皆如之。為降禪壇於社首山上,八隅、一成、八陛如方丘,三壝。上飾以黃,四方如其色,其餘皆如登封。其議略定,而天子詔曰:「古今之制,文質不同。今封禪以玉牒、金繩,而瓦尊、匏爵、秸席,宜改從文。」於是昊天上帝褥以蒼,地祇褥以黃,配褥皆以紫,而尊爵亦更焉。



 是歲正月,天子祀昊天上帝於山下之封祀壇,以高祖、太宗配,如圓丘之禮。親封玉冊,置石感,聚五色土封之,徑一丈二尺,高尺。已事,升山。明日,又封玉冊於登封壇。又明日,祀皇地祇於社首山之降禪壇,如方丘之禮,以太穆皇后、文德皇后配,而以皇后武氏為亞獻,越國太妃燕氏為終獻,率六宮以登,其帷帟皆錦繡。群臣瞻望,多竊笑之。又明日,御朝覲壇以朝群臣,如元日之禮。乃詔立登封、降禪、朝覲之碑,名封祀壇曰舞鶴臺,登封壇曰萬歲臺,降禪壇曰景雲臺,以紀瑞焉。其後將封嵩岳,以吐蕃、突厥寇邊而止。



 永淳元年,又作奉天宮於嵩山南,遂幸焉。將以明年十一月封禪,詔諸儒國子司業李行偉、考功員外郎賈大隱等草具其儀,已而遇疾,不克封,至武後遂登封焉。



 玄宗開元十二年,四方治定,歲屢豐稔,群臣多言封禪,中書令張說又固請,乃下制以十三年有事泰山。於是說與右散騎常侍徐堅、太常少卿韋縚、秘書少監康子元、國子博士侯行果刊定儀注。立圓臺於山上,廣五丈,高九尺,土色各依其方。又於圓臺上起方壇,廣一丈二尺,高九尺,其壇臺四面為一階。又積柴為燎壇於圓臺之東南,量地之宜,柴高一丈二尺,方一丈,開上,南出戶六尺。又為圓壇於山下,三成、十二階,如圓丘之制。又積柴於壇南為燎壇,如山上。又為玉冊、玉匱、石咸,皆如高宗之制。玄宗初以謂升中於崇山,精享也,不可喧嘩。欲使亞獻已下皆行禮山下壇,召禮官講議。學士賀知章等言:「昊天上帝,君也;五方精帝,臣也。陛下享君於上,群臣祀臣於下,可謂變禮之中。然禮成於三,亞、終之獻,不可異也。」於是三獻皆升山,而五方帝及諸神皆祭山下壇。玄宗問:「前世何為秘玉牒?」知章曰:「玉牒以通意於天,前代或祈長年,希神仙,旨尚微密,故外莫知。」帝曰:「朕今為民祈福,無一秘請。」即出玉牒以示百寮。乃祀昊天上帝於山上壇,以高祖配。祀五帝以下諸神於山下,其祀禮皆如圓丘。而卜日、告天及廟、社、大駕所經及告至、問百年、朝覲,皆如巡狩之禮。



 其登山也,為大次於中道,止休三刻而後升。其已祭燔燎,侍中前跪稱:「具官臣某言,請封玉冊。」皇帝升自南陛,北向立。太尉進昊天上帝神座前,跪取玉冊,置於桉以進。皇帝受玉冊,跪內之玉匱,纏以金繩,封以金泥。侍中取受命寶跪以進。皇帝取寶以印玉匱,侍中受寶,以授符寶郎。太尉進,皇帝跪捧玉匱授太尉,太尉退,復位。太常卿前奏:「請再拜。」皇帝再拜,退入於次。太尉奉玉匱之桉於石堿南,北向立。執事者發石蓋,太尉奉玉匱,跪藏於石堿內。執事者覆石蓋,檢以石檢,纏以金繩,封以石泥,以玉寶遍印,引降復位。帥執事者以石距封固,又以五色土圜封。其配座玉牒封於金匱,皆如封玉匱。太尉奉金匱從降,俱復位。以金匱內太廟,藏於高祖神堯皇帝之石室。其禪於社首,皆如方丘之禮。



 天子將巡狩,告於其方之州曰:「皇帝以某月於某巡狩,各脩乃守,考乃職事。敢不敬戒,國有常刑。」將發,告於圓丘。前一日,皇帝齋,如郊祀。告昊天上帝,又告於太廟、社稷。具大駕鹵簿。所過州、縣,刺史、令候於境,通事舍人承制問高年,祭古帝王、名臣、烈士。既至,刺史、令皆先奉見。將作築告至圓壇於嶽下,四出陛,設昊天上帝、配帝位。天子至,執事皆齋一日。



 明日,望於嶽、鎮、海、瀆、山、川、林、澤、丘、陵、墳、衍、原、隰,所司為壇。設祭官次於東壝門外道南,北向;設饌幔內壝東門外道北,南向;設宮縣、登歌;為瘞臽。祭官、執事皆齋一日。岳、鎮、海、瀆、山、川、林、澤、丘、陵、墳、衍、原、隰之尊,在壇上南陛之東,北向。設玉篚及洗,設神坐壇上北方。獻官奠玉幣及爵於嶽神,祝史助奠鎮、海以下。



 明日,乃肆覲,將作於行宮南為壝。三分壝間之二在南、為壇於北,廣九丈六尺,高九尺,四出陛。設宮縣壇南、御坐壇上之北,解劍席南陛之西。文、武官次門外東、西,刺史、令次文官南,蕃客次武官南,列輦路壇南。文官九品位壇東南,武官西南,相向。刺史、令位壇南三分庭一,蕃客位於西。又設門外位,建牙旗於壝外,黃麾大仗屯門,鈒戟陳壝中。吏部主客戶部贊群官、客使就門外位。刺史、令贄其土之實,錦、綺、繒、布、葛、越皆五兩為束,飾以黃帕常貢之物皆篚,其屬執列令後。文武九品先入就位。皇帝乘輿入北壝門,繇北陛升壇,即坐,南向。刺史、蕃客皆入壝門,至位,再拜,奠贄,興,執贄。侍中降於刺史東北,皆拜。宣已,又拜。蕃客以舍人稱制如之。戶部導貢物入刺史前,龜首之,金次之,丹、漆、絲、纊四海九州之美物,重行陳。執者退,就東西文武前,側立。通事舍人導刺史一人,解劍脫舄,執贄升前,北向跪奏:「官封臣姓名等敢獻壤奠。」遂奠贄。舍人跪舉以東授所司,刺史劍、舄復位。初,刺史升奠贄,在庭者以次奠於位前,皆再拜。戶部尚書壇間北向跪,請以貢物付所司,侍中承制曰:「可。」所司受贄出東門。中書侍郎以州鎮表方一桉俟於西門外,給事中以瑞桉俟於東門外,乃就侍臣位。初,刺史將入,乃各引桉分進東、西陛下。刺史將升,中書令、黃門侍郎降立,既升,乃取表升。尚書既請受贄,中書令乃前跪讀,黃門侍郎、給事中進跪奏瑞,侍郎、給事中導桉退,文武、刺史、國客皆再拜。北向位者出就門外位。皇帝降北陛以入,東、西位者出。設會如正、至,刺史、蕃客入門,皆奏樂如上公。



 會之明日,考制度。太常卿採詩陳之,以觀風俗。命市納賈,以觀民之好惡。典禮者考時定日,同律,禮、樂、制度、衣服正之。山川神祇有不舉為不恭,宗廟有不慎為不孝,皆黜爵。革制度、衣服者為叛,有討。有功德於百姓者,爵賞之。



 皇帝視學,設大次於學堂後,皇太子次於大次東。設御座堂上,講榻北向。皇太子座御座東南,西向。文臣三品以上坐太子南,少退;武臣三品以上於講榻西南;執讀座於前楹,北向。侍講座執讀者西北、武官之前;論義座於講榻前,北向。執如意立於侍講之東,北向。三館學官座武官後。設堂下版位,脫履席西階下。皇太子位於東階東南,執經於西階西南,文、武三品以上分位於南,執如意者一人在執經者後,學生位於文、武后。



 其日,皇帝乘馬,祭酒帥監官、學生迎於道左。皇帝入次,執經、侍講、執如意者與文武、學生皆就位堂下。皇太子立於學堂門外,西向。侍中奏「外辨」。皇帝升北階,即坐。皇太子乃入就位,在位皆再拜。侍中敕皇太子、王公升,皆再拜,乃坐。執讀、執經釋義。執如意者以授侍講,秉詣論義坐,問所疑,退,以如意授執者,還坐,乃皆降。若賜會,則侍中宣制,皇帝返次。群官既會,皇帝還,監官、學生辭於道左。



 皇帝孟春吉亥享先農,遂以耕藉。前享一日,奉禮設御坐於壇東,西向;望瘞位於壇西南,北向;從官位於內壝東門之內道南,執事者居後;奉禮位於樂縣東北,贊者在南。又設御耕藉位於外壝南門之外十步所,南向;從耕三公、諸王、尚書、卿位於御坐東南,重行西向,以其推數為列。其三公、諸王、尚書、卿等非耕者位於耕者之東,重行,西向北上;介公、酅公於御位西南,南向北上。尚舍設御耒席於三公之北少西,南向。奉禮又設司農卿之位於南,少退;諸執耒耜者位於公卿耕者之後、非耕者之前,西向。御耒耜一具,三公耒耜三具,諸王、尚書、卿各三人合耒耜九具。以下耒耜,太常各令藉田農人執之。



 皇帝已享,乃以耕根車載耒耜於御者間,皇帝乘車自行宮降大次。乘黃令以耒耜授廩犧令,橫執之,左耜置於席,遂守之。皇帝將望瘞,謁者引三公及從耕侍耕者、司農卿與執耒耜者皆就位。皇帝出就耕位,南向立。廩犧令進耒席南,北向,解韜出耒,執以興,少退,北向立。司農卿進受之,以授侍中,奉以進。皇帝受之,耕三推。侍中前受耒耜,反之司農卿,卿反之廩犧令,令復耒於韜,執以興,復位。皇帝初耕,執耒者皆以耒耜授侍耕者。皇帝耕止,三公、諸王耕五推,尚書、卿九推。執耒者前受之。皇帝還,入自南門,出內壝東門,入大次。享官、從享者出,太常卿帥其屬耕於千畝。



 皇帝還宮,明日,班勞酒於太極殿,如元會,不賀,不為壽。藉田之穀。斂而鐘之神倉,以擬粢盛及五齊、三酒,穰槁以食牲。



 藉田祭先農,唐初為帝社,亦曰藉田壇。貞觀三年,太宗將親耕,給事中孔穎達議曰:「《禮》:『天子藉田南郊,諸侯東郊。』晉武帝猶東南,今帝社乃東壇,未合於古。」太宗曰:「《書》稱『平秩東作』,而青輅、黛耜,順春氣也。吾方位少陽,田宜於東郊。」乃耕於東郊。



 垂拱中,武后藉田壇曰先農壇。神龍元年,禮部尚書祝欽明議曰:「《周頌·載芟》:『春藉田而祈社稷。』《禮》:『天子為藉千畝,諸侯百畝。』則緣田為社,曰王社、侯社。今曰先農,失王社之義,宜正名為帝社。」太常少卿韋叔夏、博士張齊賢等議曰:「《祭法》,王者立太社,然後立王社。所置之地,則無傳也。漢興已有官社,未立官稷,乃立於官社之後,以夏禹配官社,以後稷配官稷。臣瓚曰:『《高紀》,立漢社稷,所謂太社也。官社配以禹,所謂王社也。至光武乃不立官稷,相承至今。』魏以官社為帝社,故摯虞謂魏氏故事立太社是也。晉或廢或置,皆無處所。或曰二社並處,而王社居西。崔氏、皇甫氏皆曰王社在藉田。按衛宏《漢儀》:『春始東耕於藉田,引詩先農,則神農也。』又《五經要義》曰:『壇於田,以祀先農如社。』魏秦靜議風伯、雨師、靈星、先農、社、稷為國六神。晉太始四年,耕於東郊,以太牢祀先農。周、隋舊儀及國朝先農皆祭神農於帝社,配以後稷。則王社,先農不可一也。今宜於藉田立帝社、帝稷,配以禹、棄,則先農、帝社並祠,葉於周之《載芟》之義。」欽明又議曰:「藉田之祭本王社。古之祀先農,句龍、后稷也。烈山之子亦謂之農,而周棄繼之,皆祀為稷,共工之子曰后土,湯勝夏,欲遷而不可。故二神,社、稷主也。黃帝以降,不以羲、農列常祀,豈社、稷而祭神農乎?社、稷之祭,不取神農耒耜大功,而專於共工、烈山,蓋以三皇洪荒之跡,無取為教。彼秦靜何人,而知社稷、先農為二,而藉田有二壇乎?先農、王社一也,皆後稷、句龍異名而分祭,牲以四牢。」欽明又言:「漢祀禹,謬也。今欲正王社、先農之號而未決,乃更加二祀,不可。」叔夏、齊賢等乃奏言:「經無先農,《禮》曰『王自為立社,曰王社。』先儒以為在藉田也。永徽中猶曰藉田,垂拱後乃為先農。然則先農與社一神,今先農壇請改曰帝社壇,以合古王社之義。其祭,準令以孟春吉亥祠后土,以句龍氏配。」於是為帝社壇,又立帝稷壇於西,如太社、太稷,而壇不設方色,以異於太社。



 開元十九年,停帝稷而祀神農氏於壇上,以後稷配。二十三年,親祀神農於東郊,配以句芒,遂躬耕盡壟止。



 肅宗乾元二年,詔去耒耜雕刻,命有司改造之。天子出通化門,釋犮而入壇,遂祭神農氏,以後稷配。冕而硃紘,躬九推焉。



 憲宗元和五年,詔以來歲正月藉田,太常脩撰韋公肅言:「藉田禮廢久矣,有司無可考。」乃據《禮經》,參採開元、乾元故事,為先農壇於藉田。皇帝夾侍二人、正衣二人,侍中一人奉耒耜,中書令一人、禮部尚書一人侍從,司農卿一人授耒耜於侍中,太僕卿一人執牛,左、右衛將軍各一人侍衛。三公以宰相攝,九卿以左右僕射、尚書、御史大夫攝,三諸侯以正員一品官及嗣王攝。推數一用古制。禮儀使一人、太常卿一人贊禮;三公、九卿、諸侯執牛三十人,用六品以下官,皆服褲褶。御耒耜二,並韜皆以青。其制度取合農用,不雕飾,畢日收之,藉耒耜丈席二。先農壇高五尺,廣五丈,四出陛,其色青。三公、九卿、諸侯耒十有五。御耒之牛四,其二,副也,並牛衣。每牛各一人,絳衣介幘,取閑農務者,禮司以人贊導之。執耒持耜,以高品中官二人,不褲褶。皇帝詣望耕位,通事舍人分導文、武就耕所。太常帥其屬,用庶人二十八,以郊社令一人押之。太常少卿一人,率庶人趨耕所。博士六人,分贊耕禮。司農少卿一人,督視庶人終千畝。廩犧令二人,間一人奉耒耜授司農卿,以五品、六品清官攝;一人掌耒耜,太常寺用本官。三公、九卿,諸侯耕牛四十,其十,副也,牛各一人。庶人耕牛四十,各二牛一人。庶人耒耜二十具、鍤二具,木為刃。主藉田縣令一人,具朝服,當耕時立田側,畢乃退。畿甸諸縣令先期集,以常服陪耕所,耆艾二十人,陪於庶人耕位南。三公從者各三人,九卿、諸侯從者各一人,以助耕。皆絳服介幘,用其本司隸。是時雖草具其儀如此,以水、旱、用兵而止。



 皇帝謁陵,行宮距陵十里,設坐於齋室,設小次於陵所道西南,大次於寢西南。侍臣次於大次西南,陪位者次又於西南,皆東向。文官於北,武官於南,朝集使又於其南,皆相地之宜。



 前行二日,遣太尉告於廟。皇帝至行宮,即齋室。陵令以玉冊進署。設御位於陵東南隅,西向,有岡麓之閡,則隨地之宜。又設位於寢宮之殿東陛之東南,西向。尊坫陳於堂戶東南。百官、行從、宗室、客使位神道左右,寢宮則分方序立大次前。



 其日,未明五刻,陳黃麾大仗於陵寢。三刻,行事官及宗室親五等、諸親三等以上並客使之當陪者就位。皇帝素服乘馬,華蓋、繖、扇,侍臣騎從,詣小次。步出次,至位,再拜。又再拜。在位皆再拜,又再拜。少選,太常卿請辭,皇帝再拜,又再拜。奉禮曰:「奉辭。」在位者再拜。皇帝還小次,乘馬詣大次,仗衛列立以俟行。百官、宗室、諸親、客使序立次前。皇帝步至寢宮南門,仗衛止。乃入,繇東序進殿陛東南位,再拜;升自東階,北向,再拜,又再拜。入省服玩,抆拭帳簀,進太牢之饌,加珍羞。皇帝出尊所,酌酒,入,三奠爵,北向立。太祝二人持玉冊於戶外,東向跪讀。皇帝再拜,又再拜,乃出戶,當前北向立。太常卿請辭,皇帝再拜,出東門,還大次,宿行宮。



 若太子、諸王、公主陪葬柏城者,皆祭寢殿東廡;功臣陪葬者,祭東序。為位奠饌,以有司行事。



 或皇后從謁,則設大次寢宮東,先朝妃嬪次於大次南,大長公主、諸親命婦之次又於其南,皆東向。以行帳具障謁所,內謁者設皇后位於寢宮東,大次前,少東。先朝妃嬪位西南,各於次東,司贊位妃嬪東北,皆東向。皇帝既發行宮,皇后乘四望車之大次,改服假髻,白練單衣。內典引導妃嬪以下就位。皇后再拜,陪者皆拜。少選,遂辭,又拜,陪者皆拜。皇后還寢東大次,陪者退。皇后鈿釵禮衣,乘輿詣寢宮,先朝妃嬪、大長公主以下從。至北門,降輿,入大次,詣寢殿前西階之西,妃嬪、公主位於西,司贊位妃嬪東北,皆東向。皇后再拜,在位者皆拜。皇后繇西階入室,詣先帝前再拜,復詣先後前再拜,進省先後服玩,退西廂,東向立,進食。皇帝出,乃降西階位。辭,再拜,妃嬪皆拜。詣大次更衣,皇帝過,乃出寢宮北門,乘車還。



 天子不躬謁,則以太常卿行陵。所司撰日,車府令具軺車一馬清道,青衣、團扇、曲蓋繖,列俟於太常寺門。設次陵南百步道東,西向。右校令具剃器以備汛掃。太常卿公服乘車,奉禮郎以下從。至次,設卿位兆門外之左,陵官位卿東南,執事又於其南,皆西向。奉禮郎位陵官之西,贊引二人居南。太常卿以下再拜,在位皆拜。謁者導卿,贊引導眾官入,奉行、復位皆拜。出,乘車之它陵。有芟治,則命之。



 凡國陵之制,皇祖以上至太祖陵,皆朔、望上食,元日、冬至、寒食、伏、臘、社各一祭。皇考陵,朔、望及節祭,而日進食。又薦新於諸陵,其物五十有六品。始將進御,所司必先以送太常與尚食,滋味薦之,如宗廟。



 貞觀十三年,太宗謁獻陵,帝至小次,降輿,納履,入闕門,西向再拜,慟哭俯伏殆不能興。禮畢,改服入寢宮,執饌以薦。閱高祖及太穆后服御,悲感左右。步出司馬北門,泥行二百步。



 永徽二年,有司言:「先帝時,獻陵既三年,惟朔、望、冬至、夏伏、臘、清明、社上食,今昭陵喪期畢,請上食如獻陵。」從之。六年正月朔,高宗謁昭陵,行哭就位,再拜擗踴畢,易服謁寢宮。入寢哭踴,進東階,西向拜號,久,乃薦太牢之饌,加珍羞,拜哭奠饌。閱服御而後辭,行哭出寢北門,御小輦還。



 顯慶五年,詔歲春、秋季一巡,宜以三公行陵,太常少卿貳之,太常給鹵薄,仍著於令。始,《貞觀禮》歲以春、秋仲月巡陵,至武后時,乃以四季月、生日、忌日遣使詣陵起居。景龍二年,右臺侍卿史唐紹上書曰:「禮不祭墓,唐家之制,春、秋仲月以使具鹵簿衣冠巡陵。天授之後,乃有起居,遂為故事。夫起居者,參候動止,事生之道,非陵寢法。請停四季及生日、忌日、節日起居,淮式二時巡陵。」手敕曰:「乾陵歲冬至、寒食以外使,二忌以內使朝奉。它陵如紹奏。」至是又獻、昭、乾陵皆日祭。太常博士彭景直上疏曰:「禮無日祭陵,惟宗廟月有祭。故王設廟、祧、壇、墠為親疏多少之數,立七廟、一壇、一墠。曰考廟、曰王考廟、曰皇考廟,曰顯考廟,皆月祭之。遠廟為祧,享嘗乃止。去祧為壇,去壇為墠,有禱焉祭之,無禱乃止。又譙周《祭志》:『天子始祖、高祖、曾祖、祖、考之廟,皆月朔加薦,以象平生朔食,謂之月祭,二祧之廟無月祭。』則古皆無日祭者。今諸陵朔、望食,則近於古之殷事;諸節日食,近於古之薦新。鄭注《禮記》:『殷事,月朔、半薦新之奠也。』又:『既大祥即四時焉。』此其祭皆在廟,近代始以朔、望諸節祭陵寢,唯四時及臘五享廟。考經據禮,固無日祭於陵。唯漢七廟議,京師自高祖下至宣帝,與太上皇、悼皇考陵旁立廟,園各有寢、便殿,故日祭於寢,月祭於便殿。元帝時,貢禹以禮節煩數,願罷郡國廟。丞相韋玄成等又議七廟外,寢園皆無復。議者亦以祭不欲數,宜復古四時祭於廟。後劉歆引《春秋傳》『日祭,月祀,時享,歲貢。祖禰則日祭,曾高則月祀,二祧則時享,壇、墠則歲貢』。後漢陵寢之祭無傳焉。魏、晉以降,皆不祭墓。國家諸陵日祭請停如禮。」疏奏,天子以語侍臣曰:「禮官言諸陵不當日進食。夫禮以人情沿革,何專古為?乾陵宜朝晡進奠如故。昭、獻二陵日一進,或所司苦於費,可減朕常膳為之。」



 開元十五年敕:「宣皇帝、光皇帝陵,以縣令檢校,州長官歲一巡。」又敕:「歲春、秋巡陵,公卿具仗出城,至陵十里復。」



 十七年,玄宗謁橋陵,至需垣西闕下馬,望陵涕泗,行及神午門,號慟再拜。且以三府兵馬供衛,遂謁定陵、獻陵、昭陵、乾陵乃還。



 二十三年,詔:「獻、昭、乾、定、橋五陵,朔、望上食,歲冬至、寒食各日設一祭。若節與朔、望、忌日合,即準節祭料。橋陵日進半羊食。」二十七年,敕公卿巡陵乘輅,其令太僕寺,陵給輅二乘及仗。明年,制:「以宣皇帝、光皇帝、景皇帝、元皇帝追尊號謚有制,而陵寢所奉未稱。建初、啟運陵如興寧、永康陵,置署官、陵戶,春、秋仲月,分命公卿巡謁。二十年詔:建初、啟運、興寧、永康陵,歲四時、八節,所司與陵署具食進。」天寶二年,始以九月朔薦衣於諸陵。又常以寒食薦餳粥、雞球、雷車,五月薦衣、扇。



 陵司舊曰署,十三載改獻、昭、乾、定、橋五陵署為臺,令為臺令,升舊一階。是後諸陵署皆稱臺。



 大歷十四年,禮儀使顏真卿奏:「今元陵請朔、望、節祭,日薦,如故事;泰陵惟朔、望、歲冬至、寒食、伏、臘、社一祭,而罷日食。」制曰:「可。」貞元四年,國子祭酒包佶言:「歲二月、八月,公卿朝拜諸陵,陵臺所由導至陵下,禮略,無以盡恭。」於是太常約舊禮草定曰:「所司先撰吉日,公卿輅車、鹵薄就太常寺發,抵陵南道東設次,西向北上。公卿既至次,奉禮郎設位北門外之左,陵官位其東南,執事官又於其南。謁者導公卿,典引導眾官就位,皆拜。公卿、眾官以次奉行,拜而還。」



 故事,朝陵公卿發,天子視事不廢。十六年,拜陵官發,會董晉卒,廢朝。是後公卿發,乃因之不視事。



 元和元年,禮儀使杜黃裳請如故事,豐陵日祭,崇陵唯祭朔、望、節日、伏、臘。二年,宰臣建言:「禮有著定,後世徇一時之慕,過於煩,並故陵廟有薦新,而節有遣使,請歲太廟以時享,朔、望上食,諸陵以朔、望奠,親陵以朝晡奠,其餘享及忌日告陵皆停。」



\end{pinyinscope}