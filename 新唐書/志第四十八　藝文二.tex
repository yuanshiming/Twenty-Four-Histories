\article{志第四十八 藝文二}

\begin{pinyinscope}

 乙部史錄,其類十三:一曰正史類,二曰編年類,三曰偽史類,四曰雜史類體,所以終歸倒向唯心主義。在「物理學」中,他是機械唯,五曰起居注類,六曰故事類,七曰職官類,八曰雜傳記類,九曰儀注類,十曰刑法類,十一曰目錄類,十二曰譜牒類,十三曰地理類。凡著錄五百七十一家,八百五十七部,一萬六千八百七十四卷;不著錄三百五十八家,一萬二千三百二十七卷。



 司馬遷《史記》一百三十卷



 裴駰集解《史記》八十卷



 徐廣《史記音義》十三卷



 鄒誕生《史記音》三卷



 班固《漢書》一百一十五卷



 服虔《漢書音訓》一卷



 應劭《漢書集解音義》二十四卷



 諸葛亮《論前漢事》一卷



 又《音》一卷



 孟康《漢書音義》九卷



 晉灼《漢書集注》十四卷



 又《音義》十七卷



 韋昭《漢書音義》七卷



 崔浩《漢書音義》二卷



 孔氏《漢書音義鈔》二卷孔文祥。



 劉嗣等《漢書音義》二十六卷



 夏侯泳《漢書音》二卷



 包愷《漢書音》十二卷



 蕭該《漢書音》十二卷



 陰景倫《漢書律歷志音義》一卷



 項岱《漢書敘傳》八卷



 劉寶《漢書駁義》二卷



 陸澄《漢書新注》一卷



 韋稜《漢書續訓》二卷



 姚察《漢書訓纂》三十卷



 顏游秦《漢書決疑》十二卷



 僧務靜《漢書正義》三十卷



 李喜《漢書辨惑》三十卷



 《漢書正名氏義》十二卷



 《漢書英華》八卷



 劉珍等《東觀漢記》一百二十六卷



 又《錄》一卷



 謝承《後漢書》一百三十卷



 又《錄》一卷



 薛瑩《後漢記》一百卷



 司馬彪《續漢書》八十三卷



 又《錄》一卷



 劉義慶《後漢書》五十八卷



 華嶠《後漢書》三十一卷



 謝沈《後漢書》一百二卷



 又《外傳》十卷



 袁山松《後漢書》一百一卷



 又《錄》一卷



 範曄《後漢書》九十二卷



 又《論贊》五卷



 劉昭補注《後漢書》五十八卷



 張瑩《漢南紀》五十八卷



 劉熙注範曄《後漢書》一百二十二卷



 蕭該《後漢書音》三卷



 劉芳《後漢書音》一卷



 臧兢《後漢書音》三卷



 王沈《魏書》四十七卷



 陳壽《魏國志》三十卷



 《蜀國志》十五卷



 《吳國志》二十一卷並裴松之注。



 韋昭《吳書》五十五卷



 王隱《晉書》八十九卷



 虞預《晉書》五十八卷



 硃鳳《晉書》十四卷



 謝靈運《晉書》三十五卷



 又《錄》一卷



 臧榮緒《晉書》一百一十卷



 干寶《晉書》二十二卷



 蕭子云《晉書》九卷



 何法盛《晉中興書》八十卷



 徐爰《宋書》四十二卷



 孫嚴《宋書》五十八卷



 沈約《宋書》一百卷



 王智深《宋書》三十卷



 魏收《後魏書》一百三十卷



 魏澹《後魏書》一百七卷



 李德林《北齊末脩書》二十四卷



 王劭《齊志》十七卷



 又《隋書》八十卷



 蕭子顯《齊書》六十卷



 劉陟《齊書》十三卷



 謝昊、姚察《梁書》三十四卷



 顧野王《陳書》二卷



 傅縡《陳書》三卷



 許子儒注《史記》一百三十卷



 又《音》三卷字文舉,叔牙子也。證聖天官侍郎、潁川縣男。



 劉伯莊《史記音義》二十卷



 《御銓定漢書》八十七卷



 高宗與郝處俊等撰。



 顧胤《漢書古今集義》二十卷



 顏師古注《漢書》一百二十卷



 章懷太子賢注《後漢書》一百卷賢命劉訥言、格希玄等注。



 韋機《後漢書音義》二十七卷



 《晉書》一百三十卷房玄齡、褚遂良、許敬宗、來濟、陸元仕、劉子翼、令狐德棻、李義府、薛元超、上官儀、崔行功、李淳風、辛丘馭、劉引之、陽仁卿、李延壽、張文恭、敬播、李安期、李懷儼、趙弘智等脩,而名為御撰。



 姚思廉《梁書》五十六卷



 《陳書》三十六卷皆魏徵等同撰。



 張大素《後魏書》一百卷



 又《北齊書》二十卷



 《隋書》三十二卷



 李百藥《北齊書》五十卷



 令狐德棻《後周書》五十卷



 《隋書》八十五卷



 《志》三十卷顏師古、孔穎達、於志寧、李淳風、韋安化、李延壽與德棻、敬播、趙弘智、魏徵等撰。



 王元感注《史記》一百三十卷



 徐堅注《史記》一百三十卷



 又《義林》二十卷



 陳伯宣注《史記》一百三十卷貞元中上。



 韓琬《續史記》一百三十卷



 司馬貞《史記索隱》三十卷開元潤州別駕。



 劉伯莊又撰《史記地名》二十卷



 《漢書音義》二十卷



 張守節《史記正義》三十卷



 竇群《史記名臣疏》三十四卷



 敬播注《漢書》四十卷



 又《漢書音義》十二卷



 元懷景《漢書議苑》卷亡。開元右庶子,武陵縣男。謚曰文。



 姚珽《漢書紹訓》四十卷



 李鎮注《史記》一百三十卷開元十七年上,授門下典儀。



 沈遵《漢書問答》五卷



 李善《漢書辨惑》二十卷



 徐堅《晉書》一百一十卷



 高希嶠注《晉書》一百三十卷開元二十年上,授清池主簿。



 何超《晉書音義》三卷處士。



 《武德貞觀兩朝史》八十卷長孫無忌、令狐德棻、顧胤等撰。



 吳兢又《齊史》十卷



 《梁史》十卷



 《陳史》五卷



 《周史》十卷



 《隋史》二十卷



 《唐書》一百卷



 又一百三十卷兢、韋述、柳芳、令狐



 峘、於休烈等撰。



 《國史》一百六卷



 又一百一十三卷



 裴安時《史記纂訓》二十卷



 又《元魏書》三十卷字適之,大中江陵少尹。



 凡集史五家,六部,一千二百二十二卷。高峻以下不著錄三家,四百四十卷。



 梁武帝《通史》六百二卷



 李延壽《南史》八十卷



 又《北史》一百卷



 高氏《小史》一百二十卷高峻,初六十卷,其子迥厘益。之。峻,元和中人。



 劉氏《洞史》二十卷劉權,忠州刺史晏曾孫。



 姚康復《統史》三百卷大中太子詹事。



 右正史類七十家,九十部,四千八十五卷。失姓名二家,王元感以下不著錄二十三家,一千七百九十卷。總七十三家,六十九部。



 《紀年》十四卷《汲塚書》。



 荀悅《漢紀》三十卷



 應劭等注荀悅《漢紀》三十卷



 崔浩《漢紀音義》三卷



 侯瑾《漢皇德紀》三十卷



 張璠《後漢紀》三十卷



 袁宏《後漢紀》三十卷



 張緬《後漢略》二十七卷



 劉艾《漢靈獻二帝紀》六卷



 袁曄《漢獻帝春秋》十卷



 樂資《山陽公載記》十卷



 習鑿齒《漢晉春秋》五十四卷



 《魏武本紀》四卷



 孫盛《魏武春秋》二十卷



 又《晉陽秋》二十二卷



 魏澹《魏紀》十二卷



 梁祚《魏書國紀》十卷



 環濟《吳紀》十卷



 陸機《晉帝紀》四卷



 干寶《晉紀》二十二卷



 劉協注干寶《晉紀》六十卷



 劉謙之《晉紀》二十卷



 曹嘉之《晉紀》十卷



 徐廣《晉紀》四十五卷



 鄧粲《晉紀》十一卷



 又《晉陽秋》三十二卷



 檀道鸞《晉春秋》二十卷



 蕭景暢《晉史草》三十卷



 郭季產《晉續紀》五卷



 《晉錄》五卷



 王智深《宋紀》三十卷



 裴子野《宋略》二十卷



 鮑衡卿《宋春秋》二十卷



 王琰《宋春秋》二十卷



 沈約《齊紀》二十卷



 吳均《齊春秋》三十卷



 謝昊《梁典》三十九卷



 劉璠《梁典》三十卷



 何之元《梁典》三十卷



 蕭韶《梁太清紀》十卷



 《皇帝紀》七卷



 《梁末代記》一卷



 臧嚴《棲鳳春秋》五卷



 姚最《梁昭後略》十卷



 《北齊記》二十卷



 王劭《北齊志》十七卷



 趙毅《隋大業略記》三卷



 杜延業《晉春秋略》二十卷



 張大素《隋後略》十卷



 柳芳《唐歷》四十卷



 《續唐歷》二十二卷韋澳、蔣偕、李荀、張彥遠、崔瑄撰,崔龜從監脩。



 吳兢《唐春秋》三十卷



 韋述《唐春秋》三十卷



 陸長源《唐春秋》六十卷



 陳岳《唐統紀》一百卷



 焦璐《唐朝年代記》十卷徐州從事,龐勛亂遇害。



 李仁實《通歷》七卷



 馬亹《通歷》十卷



 王氏《五位圖》十卷王起。



 《廣五運圖》卷亡。



 苗臺符《古今通要》四卷宣、懿時人。



 賈欽文《古今年代歷》一卷大中時人。



 曹圭《五運錄》十二卷



 張敦素《建元歷》二卷



 劉軻《帝王歷數謌》一卷字希仁,元和末進士第,洺州刺史。



 封演《古今年號錄》一卷天寶末進士第。



 韋美《嘉號錄》一卷中和中進士。



 柳璨《正閏位歷》三卷



 李匡文《兩漢至唐年紀》一卷昭宗時宗正少卿。



 右編年類四十一家,四十八部,九百四十七卷。失姓名四家,柳芳以下不著錄十九家,三百五十五卷。



 常璩《華陽國志》十三卷



 又《漢之書》十卷



 《蜀李書》九卷



 和包《漢趙紀》十四卷



 田融《趙石記》二十卷



 又《二石記》二十卷



 《苻朝雜記》一卷



 王度、隨翽《二石偽事》六卷



 《二石書》十卷



 範亨《燕書》二十卷



 王景暉《南燕錄》六卷



 張詮《南燕書》十卷



 高閭《燕志》十卷



 段龜龍《涼記》十卷



 《西河記》二卷



 張諮《涼記》十卷



 劉昺《涼書》十卷



 又《敦煌實錄》二十卷



 裴景仁《秦記》十一卷杜惠明注。



 《拓拔涼錄》十卷



 《桓玄偽事》二卷



 《鄴洛鼎峙記》十卷



 守節先生《天啟紀》十卷



 崔鴻《十六國春秋》一百二十卷



 蕭方《三十國春秋》三十卷



 李概《戰國春秋》二十卷



 蔡允恭《後梁春秋》十卷



 武敏之《三十國春秋》一百卷



 右偽史類一十七家,二十七部,五百四十二卷。失姓名三家。



 《古文鎖語》四卷



 《汲塚周書》十卷



 子貢《越絕書》十六卷



 孔晁注《周書》八卷



 何承天《春秋前傳》十卷



 又《春秋前傳雜語》十卷



 樂資《春秋後傳》三十卷



 孟儀注《周載》三十卷



 趙曄《吳越春秋》十二卷



 楊方《吳越春秋削煩》五卷



 《吳越記》六卷



 劉向《戰國策》三十二卷



 高誘注《戰國策》三十二卷



 延篤《戰國策論》一卷



 陸賈《楚漢春秋》九卷



 衛颯《史記要傳》十卷



 張瑩《史記正傳》九卷



 譙周《古史考》二十五卷



 王粲《漢書英雄記》十卷



 葛洪《史記鈔》十四卷



 又《漢書鈔》三十卷



 《後漢書鈔》三十卷



 張緬《後漢書略》二十五卷



 又《晉書鈔》三十卷



 範曄《後漢書纘》十三卷



 孔衍《春秋時國語》十卷



 又《春秋後國語》十卷



 《漢尚書》十卷



 《漢春秋》十卷



 《後漢尚書》六卷



 《後漢春秋》六卷



 《後魏尚書》十四卷



 《後魏春秋》九卷



 王越客《後漢文武釋論》二十卷



 袁希之《漢表》十卷



 張溫《三史要略》三十卷



 阮孝緒《正史削繁》十四卷



 王延秀《史要》二十八卷



 蕭肅《合史》二十卷



 又《錄》一卷



 王蔑《史漢要集》二卷



 司馬彪《九州春秋》九卷



 《後漢雜事》十卷



 魚豢《魏略》五十卷



 孫壽《魏陽秋異同》八卷



 《魏武本紀年歷》五卷



 王隱《刪補蜀記》七卷



 張勃《吳錄》三十卷



 李概《左史》六卷



 胡沖《吳朝人士品秩狀》八卷



 又《吳歷》六卷



 虞禹《吳士人行狀名品》二卷



 虞溥《江表傳》五卷



 徐眾《三國評》三卷



 王濤《三國志序評》三卷



 傅暢《晉諸公贊》二十二卷



 《晉歷》二卷



 荀綽《晉後略》五卷



 賈匪之《漢魏晉帝要紀》三卷



 郭頒《魏晉代說》十卷



 謝綽《宋拾遺錄》十卷



 孔思尚《宋齊語錄》十卷



 陰僧仁《梁撮要》三十卷



 宋孝王《關東風俗傳》六十三卷



 來奧《帝王本紀》十卷



 環濟《帝王略要》十二卷



 劉滔《先聖本紀》十卷



 楊曄《華夷帝王紀》三十七卷



 張愔等《帝系譜》二卷



 韋昭《洞紀》四卷



 皇甫謐《帝王代紀》十卷



 又《年歷》六卷



 何茂林《續帝王代紀》十卷



 《帝王代紀》十六卷



 《歷紀》十卷



 姚恭《年歷帝紀》二十六卷



 吉文甫《十五代略》十卷



 《代譜》四十八卷周武帝敕撰。



 諸葛耽《帝錄》十卷



 庾和之《歷代記》三十卷



 熊襄《十代記》十卷



 盧元福《帝王編年錄》五十一卷



 又《共和以來甲乙紀年》二卷



 趙弘禮《王業歷》二卷



 周樹《洞歷記》九卷



 徐整《三五歷紀》二卷



 又《通歷》二卷



 《雜歷》五卷



 孔衍《國志歷》五卷



 《長歷》十四卷



 《千年歷》二卷



 許氏《千歲歷》三卷



 陶弘景《帝王年歷》五卷



 羊瑗《分王年歷》五卷



 王嘉《拾遺錄》三卷



 又《拾遺記》十卷蕭綺錄。



 周祗《崇安記》二卷



 王韶之《崇安記》十卷



 鮑衡卿《乘輿飛龍記》二卷



 蕭大圓《淮海亂離志》四卷



 李仁實《通歷》七卷



 裴矩《隋開業平陳記》十二卷



 褚無量《帝王紀錄》三卷



 皇甫遵《吳越春秋傳》十卷



 盧彥卿《後魏紀》三十三卷



 劉允濟《魯後春秋》二十卷



 丘悅《三國典略》三十卷



 元行沖《魏典》三十卷



 員半千《三國春秋》二十卷



 李筌《閫外春秋》十卷



 李吉甫《六代略》三十卷



 張絢《古五代新記》二卷



 許嵩《建康實錄》二十卷



 《柳氏自備》三十卷柳仲郢。



 鄭韋《史俊》十卷



 呂才《隋記》二十卷



 丘啟期《隋記》十卷開元管城尉。



 杜寶《大業雜記》十卷



 杜儒童《隋季革命記》五卷武后時人。



 《劉氏行年記》二十卷劉仁軌。



 崔良佐《三國春秋》卷亡。良佐,深州安平人,日用從子。居共白鹿山,門人謚曰貞文孝父。



 裴遵度《王政記》



 楊岑《皇王寶運錄》並卷亡。岑,憲宗時人。



 《功臣錄》三十卷



 唐潁《稽典》一百三十卷開元中,潁罷臨汾尉,上之。張說奏留史館脩史,兼集賢待制。



 王彥威《唐典》七十卷



 吳兢《唐書備闕記》十卷



 《續皇王寶運錄》十卷韋昭度、楊涉撰。



 韓祐《續古今人表》十卷開元十七年上,授太常寺太祝。



 張薦《宰輔傳略》卷亡。



 蔣乂《大唐宰輔錄》七十卷



 又《凌煙功臣》、《秦府十八學士》、《史臣》等傳四十卷



 凌璠《唐錄政要》十二卷昭宗時江都尉。



 南卓《唐朝綱領圖》一卷字昭嗣,大中黔南觀察使。



 薛璫《唐聖運圖》二卷



 劉肅《大唐新語》十三卷元和中江都主簿。



 李肇《國史補》三卷翰林學士,坐薦柏耆,自中書舍人左遷將作少監。



 林恩《補國史》十卷僖宗時進士。



 《傳載》一卷



 《史遺》一卷



 溫大雅《今上王業記》六卷



 李延壽《太宗政典》三十卷



 吳兢《太宗勛史》一卷



 又《貞觀政要》十卷



 李康《明皇政錄》十卷



 鄭處誨《明皇雜錄》二卷



 鄭棨《開天傳信記》一卷



 溫畬《天寶亂離西幸記》一卷



 宋巨《明皇幸蜀記》一卷



 姚汝能《安祿山事跡》三卷華陰尉。



 包住《河洛春秋》二卷安祿山、史思明事。



 徐岱《奉天記》一卷德宗西狩事。



 崔光庭《德宗幸奉天錄》一卷



 趙元一《奉天錄》四卷



 張讀《建中西狩錄》十卷字聖用,僖宗時吏部侍郎。



 袁皓《興元聖功錄》三卷



 穀況《燕南記》三卷張孝忠事。



 路隋《平淮西記》一卷



 杜信《史略》三十卷



 又《閑居錄》三十卷



 鄭澥《涼國公平蔡錄》一卷字蘊士,李愬山南東道掌書記,開州刺史。



 薛圖存《河南記》一卷李師道事。



 李潛用《乙卯記》一卷李訓、鄭注事。



 《大和摧兇記》一卷



 《野史甘露記》二卷



 《開成紀事》二卷



 李石《開成承詔錄》二卷



 李德裕《次柳氏舊聞》一卷



 又《文武兩朝獻替記》三卷



 《會昌伐叛記》一卷



 《上黨紀叛》一卷劉從諫事。



 韓昱《壺關錄》三卷



 裴廷裕《東觀奏記》三卷大順中,詔脩宣、懿、僖實錄,以日歷注記亡缺,因摭宣宗政事奏記於監脩國史杜讓能。廷裕,字膺餘,昭宗時翰林學士、左散騎常侍,貶湖南,卒。



 令狐澄《貞陵遺事》二卷綯子也。乾符中書舍人。



 柳玭《續貞陵遺事》一卷



 鄭言《平剡錄》一卷裘甫事。言,字垂之,浙西觀察使王式從事,咸通翰林學士、戶部侍郎。



 張云《咸通解圍錄》一卷字景之,一字瑞卿,起居舍人。



 鄭樵《彭門紀亂》三卷龐勛事。



 王坤《驚聽錄》一卷黃巢事。



 郭廷誨《廣陵妖亂志》三卷高駢事。



 《乾寧會稽錄》一卷董昌事。



 韓偓《金鑾密記》五卷



 王振《汴水滔天錄》一卷昭宗時拾遺。



 公沙仲穆《大和野史》十卷起大和,盡龍紀。右雜史類八十八家,一百七部,一千八百二十八卷。失姓名八家,元行沖以下不著錄六十八家,八百六十一卷。



 郭璞《穆天子傳》六卷



 《漢獻帝起居注》五卷



 李軌《晉泰始起居注》二十卷



 又《晉咸寧起居注》二十二卷



 《晉太康起居注》二十二卷



 《晉永平起居注》八卷



 《晉咸和起居注》十八卷



 《晉咸康起居注》二十二卷



 劉道薈《晉起居注》三百二十卷



 《晉建武大興永昌起居注》二十二卷



 《晉建元起居注》四卷



 《晉永和起居注》二十四卷



 《晉升平起居注》十卷



 《晉隆和興寧起居注》五卷



 《晉太和起居注》六卷



 《晉咸安起居注》三卷



 《晉寧康起居注》六卷



 《晉太元起居注》五十二卷



 《晉崇寧起居注》十卷



 《晉元興起居注》九卷



 《晉義熙起居注》三十四卷



 《晉元熙起居注》二卷



 何始真《晉起居鈔》五十一卷



 《晉起居注鈔》二十四卷



 《宋永初起居注》六卷



 《宋景平起居注》三卷



 《宋元嘉起居注》七十一卷



 《宋孝建起居注》十七卷



 《宋大明起居注》十五卷



 《後魏起居注》二百七十六卷



 《齊永明起居注》二十五卷



 《梁大同七年起居注》十卷



 《陳起居注》四十一卷



 《隋開皇元年起居注》六卷



 王逡之《三代起居注鈔》十五卷



 《流別起居注》四十七卷



 溫大雅《大唐創業起居注》三卷



 《開元起居注》三千六百八十二卷失撰人名。



 姚修《時政記》四十卷



 凡實錄二十八部,三百四十五卷。劉知幾以下不著錄四百五十七卷。



 周興嗣《梁皇帝實錄》二卷



 謝昊《梁皇帝實錄》五卷



 《梁太清實錄》十卷



 《高祖實錄》二十卷敬播撰,房玄齡監脩,許敬宗刪改。



 《今上實錄》二十卷敬播、顧胤撰,房玄齡監脩。



 長孫無忌《貞觀實錄》四十卷



 許敬宗《皇帝實錄》三十卷



 《高宗後脩實錄》三十卷初,令狐德棻撰,止乾封,劉知幾、吳兢續成。



 韋述《高宗實錄》三十卷



 武後《高宗實綠》一百卷



 《則天皇後實錄》二十卷魏元忠、武三思、祝欽明、徐彥伯、柳沖、韋承慶、崔融、岑羲、徐堅撰,劉知幾、吳兢刪正。



 宗秦客《聖母神皇實錄》十八卷



 吳兢《中宗實錄》二十卷



 劉知幾《太上皇實錄》十卷



 吳兢《睿宗實錄》五卷



 張說《今上實錄》二十卷說與唐潁撰,次玄宗開元初事。



 《開元實錄》四十七卷失撰人名。



 《玄宗實錄》一百卷令狐峘撰,元載監脩。



 《肅宗實錄》三十卷元載監脩。



 令狐峘《代宗實錄》四十卷



 沈既濟《建中實錄》十卷



 《德宗實錄》五十卷蔣乂、樊紳、林寶、韋處厚、獨孤鬱撰,裴垍監脩。



 《順宗錄》五卷韓愈沈傳師。宇文籍撰,李吉甫監脩。



 《憲宗實錄》四十卷沈傳師、鄭澣、宇文籍、蔣系、李漢、陳夷行、蘇景胤撰,杜元穎、韋處厚、路隋監脩。景胤,弁子也,中書舍人。



 《穆宗實錄》二十卷蘇景胤、王彥威、楊漢公、蘇滌、裴休撰,路隋監脩。滌,字玄獻,冕子也,荊南節度使、吏部尚書。



 《敬宗實錄》十卷陳商、鄭亞撰,李讓夷監脩。商,字述聖,禮部侍郎、秘書監。



 《敬宗寶錄》十卷陳商卷亞撰李讓夷監脩。商,字述聖,禮部侍郎,秘書監。



 《文宗實錄》四十卷盧耽、蔣偕、王渢、盧告、牛叢撰,魏暮監脩。耽,字子嚴,一字子重,歷西川節度使、同中書門下平章事。渢,字中德,歷東都留守。告,字子有,弘宣子也,歷吏部侍郎。



 《武宗實錄》三十卷韋保衡臨脩。



 凡詔令一家,一十一部,三百五卷。失姓名十家,溫彥博以下不著錄十一家,二百二十二卷。



 《晉雜詔書》一百卷



 又二十八卷



 又六十六卷



 《晉詔書黃素制》五卷



 《晉定品雜制》一卷



 《晉太元副詔》二十一卷



 《晉崇安元興大亨副詔》八卷



 《晉義熙詔》二十二卷



 《寧永初詔》六卷



 《宋元嘉詔》二十一卷



 宋乾《詔集區別》二十七卷



 溫彥博《古今詔集》三十卷



 李義府《古今詔集》一百卷



 薛克構《聖朝詔集》三十卷



 《唐德音錄》三十卷



 《太平內制》五卷



 《明皇制詔錄》一卷



 《元和制集》十卷



 王起《寫宣》十卷



 馬文敏《王言會最》五卷



 《唐舊制編錄》六卷費氏集。



 《擬狀注制》十卷



 右起居注類六家,三十八部,一千二百七十二卷。失姓名二十六家,《開元起居注》以下不著錄三家,三千七百二十五卷。總七家,七十七部。



 《秦漢以來舊事》八卷



 《漢武帝故事》二卷



 韋氏《三輔舊事》一卷



 葛洪《西京雜記》二卷



 《建武故事》三卷



 《永平故事》二卷



 應劭《漢朝駁》三十卷



 《漢諸王奏事》十卷



 《漢魏吳蜀舊事》八卷



 《魏名臣奏事》三十卷



 《魏臺訪議》三卷



 《魏廷尉決事》十卷



 《南臺奏事》九卷



 《晉太始太康故事》八卷



 孔愉《晉建武咸和咸康故事》四卷



 《晉建武以來故事》三卷



 《晉氏故事》三卷



 《晉朝雜事》二卷



 《晉故事》四十三卷



 《晉諸雜故事》二十二卷



 《晉雜議》十卷



 《晉要事》三卷



 《晉宋舊事》一百三十卷



 車灌《晉脩復山陵故事》五卷



 盧綝《晉八王故事》十二卷



 又《晉四王起事》四卷



 張敞《晉東宮舊事》十卷



 範汪《尚書大事》二十一卷



 《華林故事名》一卷



 劉道薈《先朝故事》二十卷



 《交州雜故事》九卷



 《中興伐逆事》二卷



 溫子昇《魏永安故事》三卷



 蕭大圓《梁魏舊事》三十卷



 僧亡名《天正舊事》三卷



 應詹《江南故事》三卷



 《大司馬陶公故事》三卷



 《郗太尉為尚書令故事》三卷



 王愆期《救襄陽上都府事》一卷



 《春坊舊事》三卷



 武後《述聖紀》一卷



 杜正倫《春坊要錄》四卷



 王方慶《南宮故事》十二卷



 裴矩《鄴都故事》十卷



 馬亹《唐年小錄》八卷



 張齊賢《孝和中興故事》三卷



 盧若虛《南宮故事》三十卷



 令狐德棻《凌煙閣功臣故事》四卷



 敬播《文貞公傳事》四卷



 劉禕之《文貞公故事》六卷



 張大業《魏文貞故事》八卷



 王方慶《文貞公事錄》一卷



 李仁實《衛公平突厥故事》二卷



 謝偃《英公故事》四卷



 劉禕之《英國貞武公故事》四卷



 陳諫等《彭城公故事》一卷劉晏。



 《張九齡事跡》一卷



 《李渤事跡》一卷



 《杜悰事跡》一卷



 《吳湘事跡》一卷



 丘據《相國涼公錄》一卷李抱玉事。據,諫議大夫。



 右故事類十七家,四十三部,四百九十六卷。失姓名二十五家,裴矩以下不著錄十六家,九十卷。



 王隆《漢官解詁》三卷胡廣注。



 應劭《漢官》五卷



 《漢官儀》十卷



 蔡質《漢官典儀》一卷



 丁孚《漢官儀式選用》一卷



 荀攸等《魏官儀》一卷



 傅暢《晉公卿禮秩故事》九卷



 《百官名》十四卷



 干寶《司徒儀注》五卷



 陸機《晉惠帝百官名》三卷



 《晉官屬名》四卷



 《晉過江人士目》一卷



 衛禹《晉永嘉流士》二卷



 《登城三戰簿》三卷



 範曄《百官階次》一卷



 荀欽明《宋百官階次》三卷



 《宋百官春秋》六卷



 《魏官品令》一卷



 王珪之《齊職官儀》五十卷



 徐勉《梁選簿》三卷



 沈約《梁新定官品》十六卷



 《梁百官人名》十五卷



 《陳將軍簿》一卷



 《太建十一年百官簿狀》二卷



 郎楚之《隋官序錄》十二卷



 王道秀《百官春秋》十三卷



 郭演《職令古今百官注》十卷



 陶彥藻《職官要錄》三十六卷



 《職員舊事》三十卷



 王方慶《宮卿舊事》一卷



 《六典》三十卷開元十年,起居舍人陸堅被詔集賢院脩「六典」,玄宗手寫六條,曰理典、教典、禮典、政典、刑典、事典。張說知院,委徐堅,經歲無規制,乃命毋煚、余欽、咸廙業、孫季良、韋述參撰。始以令式象《周禮》六官為制。蕭嵩知院,加劉鄭蘭、蕭晟、盧若虛。張九齡知院,加陸善經。李林甫代九齡,加苑咸。二十六年書成。



 王方慶又撰《尚書考功簿》五卷



 又《尚書考功狀績簿》十卷



 《尚書科配簿》五卷



 《五省遷除》二十卷



 裴行儉《選譜》十卷



 《唐循資格》一卷天寶中定。



 沈既濟《選舉志》十卷



 梁載言《具員故事》十卷



 又《具員事跡》十卷



 杜英《師職該》二卷



 任戩《官品纂要》十卷



 溫大雅《大丞相唐王官屬記》二卷



 杜易簡《御史臺雜注》五卷



 韓琬《御史臺記》十二卷



 韋述《御史臺記》十卷



 又《集賢注記》三卷



 李構《御史臺故事》三卷



 劉貺《天官舊事》一卷



 柳芳《大唐宰相表》三卷



 馬宇《鳳池錄》五十卷



 賀蘭正元《輔佐記》十卷



 又《舉選衡鑒》三卷昭義判官,貞元十三年上。



 韋琯《國相事狀》七卷憲宗時人。



 張之緒《文昌損益》二卷德宗時人。



 李肇《翰林志》一卷



 李吉甫《元和國計簿》十卷



 又《元和百司舉要》一卷



 王涯《唐循資格》五卷



 韋處厚《大和國計》二十卷



 王彥威《占額圖》一卷



 孫結《大唐國照圖》一卷文宗時人。



 《大唐國要圖》五卷左僕射賈耽纂,監察御史褚球重脩。



 《翰林內志》一卷



 楊鉅《翰林學士院舊規》一卷字文碩,收子也。昭宗時翰林學士、吏部侍郎。



 右職官類十九家,二十六部,二百六十二卷。失姓名十家,《六典》以下不著錄二十九家,二百八十卷。



 趙岐《三輔決錄》、十卷摯虞注。



 魏文帝《海內士品錄》三卷



 《海內先賢傳》五卷魏明帝時撰。



 李氏《海內先賢行狀》三卷



 韋氏《四海耆舊傳》一卷



 《諸國先賢傳》一卷



 圈稱《陳留風俗傳》三卷



 蘇林《陳留耆舊傳》三卷



 劉昺《敦煌實錄》二十卷



 陳英宗《陳留先賢傳像贊》一卷



 江敞《陳留人物志》十五卷



 周斐《汝南先賢傳》五卷



 陸胤《志廣州先賢傳》七卷



 劉芳《廣州先賢傳》七卷



 徐整《豫章舊志》八卷



 又《豫章烈士傳》三卷



 華隔《廣陵烈士傳》一卷



 張勝《桂陽先賢畫贊》五卷



 硃育《會稽記》四卷



 虞預《會稽典錄》二十四卷



 謝承《會稽先賢傳》七卷



 賀氏《會稽先賢傳像贊》四卷



 鐘離岫《會稽後賢傳》三卷



 賀氏《會稽太守像贊》二卷



 陸凱《吳國先賢傳》五卷



 《吳國先賢像贊》三卷



 陳壽《益部耆舊傳》十四卷



 《益州耆舊雜傳記》二卷



 白褒《魯國先賢傳》十四卷



 張方《楚國先賢傳》十二卷



 高範《荊州先賢傳》三卷



 仲長統《山陽先賢傳》一卷



 範瑗《交州先賢傳》四卷



 習鑿齒《襄陽耆舊傳》五卷



 又《逸人高士傳》八卷



 王基《東萊耆舊傳》一卷



 王羲度《徐州先賢傳》九卷



 劉義慶《徐州先賢傳贊》八卷



 又一卷



 劉彧《長沙舊邦傳贊》四卷



 郭緣生《武昌先賢傳》三卷



 虞溥《江表傳》三卷



 崔蔚祖《海岱志》十卷



 吳均《吳郡錢塘先賢傳》五卷



 陽休之《幽州古今人物志》三十卷



 留叔先《東陽朝堂書贊》一卷



 《濟北先賢傳》一卷



 《廬江七賢傳》一卷



 《零陵先賢傳》一卷



 蕭廣濟《孝子傳》十五卷



 師覺授《孝子傳》八卷



 王韶之《孝子傳》十五卷



 又《贊》三卷



 宗躬《孝子傳》二十卷



 又《止足傳》十卷



 虞盤佐《孝子傳》一卷



 又《高士傳》二卷



 徐廣《孝子傳》三卷



 梁武帝《孝子傳》三十卷



 《雜孝子傳》二卷



 鄭緝之《孝子傳贊》十卷



 申秀《孝友傳》八卷



 元懌《顯忠錄》二十卷



 嵇康《聖賢高士傳》八卷



 皇甫謐《高士傳》十卷



 又《逸士傳》一卷



 《玄晏春秋》二卷



 《韋氏家傳》三卷



 周續之《上古以來聖賢高士傳贊》三卷



 劉晝《高才不遇傳》四卷



 周弘讓《續高士傳》八卷



 張顯《逸人傳》三卷



 鐘離儒《逸人傳》七卷



 袁宏《名士傳》三卷



 袁淑《真隱傳》二卷



 阮孝緒《高隱傳》十卷



 劉向《列士傳》二卷



 範晏《陰德傳》二卷



 齊竟陵文宣王子良《止足傳》十卷



 鐘岏《良吏傳》十卷



 《先儒傳》五卷



 殷系《英籓可錄事》三卷一云張萬賢撰。



 鄭忱《文林館記》十卷



 張騭《文士傳》五十卷



 梁元帝《孝德傳》三十卷



 又《忠臣傳》三十卷



 《全德志》一卷



 《丹楊尹傳》十卷



 《同姓名錄》一卷



 《懷舊志》九卷



 裴懷貴《兄弟傳》三卷



 《悼善列傳》四卷



 劉昭《幼童傳》十卷



 盧思道《知己傳》一卷



 孫敏《春秋列國名臣傳》九卷



 《孔子弟子傳》五卷



 《東方朔傳》八卷



 《李固別傳》七卷



 《梁冀傳》二卷



 郭沖《諸葛亮隱沒五事》一卷



 《何顒傳》一卷



 《曹瞞傳》一卷



 《毋丘儉記》三卷



 管辰《管輅傳》二卷



 戴逵《竹林七賢論》二卷



 孟仲暉《七賢傳》七卷



 《桓玄傳》二卷



 《雜傳》六十九卷



 又四十卷



 又九卷



 任昉《雜傳》一百二十卷



 《荊揚二州遷代記》四卷



 元暉等《秘錄》二百七十卷



 五孝恭《集記》一百卷



 《漢明帝畫贊》五十卷



 姚澹《四科傳贊》四卷



 《七國敘贊》十卷



 《益州文翁學堂圖》一卷



 荀伯子《荀氏家傳》十卷



 又《薛常侍傳》二卷



 《明氏世錄》六卷明粲。



 《漢南庾氏家傳》三卷庾守業。



 《褚氏家傳》一卷褚結撰,褚陶注。



 《殷氏家傳》三卷殷敬。



 《崔氏世傳》七卷崔鴻。



 《邵氏家傳》十卷



 《王氏家傳》二十一卷



 《江氏家傳》七卷江饒。



 《暨氏家傳》一卷



 《虞氏家傳》五卷虞覽。



 《裴氏家記》三卷裴松之。



 《諸葛傳》五卷



 《曹氏家傳》一卷曹毘。



 《諸王傳》一卷



 《陸史》十五卷陸煦。



 王劭《爾硃氏家傳》二卷



 《何妥家傳》二卷



 《裴若弼家傳》一卷



 令狐德棻《令狐家傳》一卷



 張大素《敦煌張氏家傳》二十卷



 魏征《自古諸侯王善惡錄》二卷



 章懷太子《列籓正論》三十卷



 鄭世翼《交游傳》二卷



 李襲譽《忠孝圖傳贊》二十卷



 許敬宗《文館詞林文人傳》一百卷



 崔玄韋《友義傳》十卷



 又《義士傳》十五卷



 傅弈《高識傳》十卷



 郎餘令《孝子後傳》三十卷



 平貞諲《養德傳》卷亡。



 徐堅《大隱傳》三卷



 裴朏《續文士傳》十卷開元中懷州司馬。



 李襲譽又撰《江東記》三十卷



 李義府《宦游記》七十卷



 王方慶《友悌錄》十五卷



 又《王氏訓誡》五卷



 《王氏列傳》十五卷



 《王氏尚書傳》五卷



 《魏文貞故書》十卷



 唐臨《冥報記》二卷



 李筌《中臺志》十卷



 盧詵《四公記》一卷一作梁載言。



 王瓘《廣軒轅本紀》三卷



 李渤《六賢圖贊》一卷



 陸龜蒙《小名錄》五卷



 張昌宗《古文紀年新傳》三卷昌宗,冀州南宮人,太子舍人。



 王緒《永寧公輔梁記》十卷緒,開元人,僧辯兄孫也,永寧即僧辯所封。



 賈閏甫《李密傳》三卷閏甫,密舊屬。



 顏師古《安興貴家傳》卷亡。



 《陸氏英賢征記》三卷陸師儒。



 李邕《狄仁傑傳》三卷



 郭湜《高氏外傳》一卷力士。湜,大歷大理司直。



 李翰《張巡姚訚傳》二卷



 陳翃《郭公家傳》八卷子儀。翃嘗為其寮屬,後又從事渾瑊河中幕。



 殷亮《顏氏家傳》一卷杲卿。



 殷仲容《顏氏行狀》一卷真卿。



 馬宇《段公別傳》二卷秀實。宇,元和秘書少監,史館脩撰。



 李繁《相國鄴侯家傳》十卷



 王起《李趙公行狀》一卷李吉甫。



 張茂樞《河東張氏家傳》三卷弘靖孫。



 崔氏《唐顯慶登科記》五卷失名。



 姚康《科第錄》十六卷字汝諧,南仲孫也。兵部郎中,金吾將軍。



 李弈《唐登科記》二卷



 《文場盛事》一卷



 張鷟《朝野僉載》二十卷自號浮休子。



 《封氏聞見記》五卷封演。



 劉餗《國朝傳記》三卷



 《國朝舊事》四十卷



 蘇特《唐代衣冠盛事錄》一卷



 李綽《尚書故實》一卷尚書即張延賞。



 《柳氏訓序》一卷柳玭。



 武平一《景龍文館記》十卷



 蕭叔和《天祚永歸記》一卷睿宗事。



 韋機《西征記》卷亡。



 韓琬《南征記》十卷



 凌準《邠志》二卷



 陸贄《遣使錄》一卷



 裴肅《平戎記》五卷休父。



 房千里《投荒雜錄》一卷字鵠舉,大和初進士第,高州刺史。



 杜佑《賓佐記》一卷



 《文宗朝備問》一卷



 黃璞《閩川名士傳》一卷字紹山,大順中進士第。



 魏征《祥瑞錄》十卷



 徐景《玉璽正錄》一卷



 《國寶傳》一卷



 許康佐《九鼎記》四卷



 顏師古《王會圖》卷亡。



 李德裕《異域歸忠傳》二卷



 《西蕃會盟記》三卷



 《西戎記》二卷



 《英雄錄》一卷



 趙珫《孝行志》二十卷字盈之,晉州岳陽人,會昌中。



 武誼《自古忠臣傳》二十卷字子思,楚州盱眙人,咸通中州從事。



 凡女訓十七家,二十四部,三百八十三卷。失姓名一家,王方慶以下不著錄五家,八十三卷。



 劉向《列女傳》十五卷曹大家注。



 皇甫謐《列女傳》六卷



 綦毋邃《列女傳》七卷



 劉熙《列女傳》八卷



 趙母《列女傳》七卷



 項宗《列女後傳》十卷



 曹植《列女傳頌》一卷



 孫夫人《列女傳序贊》一卷



 杜預《列女傳》十卷



 虞通之《後妃記》四卷



 又《妒記》二卷



 諸葛亮《貞潔記》一卷



 曹大家《女誡》一卷



 辛德源、王劭等《內訓》二十卷



 徐湛之《婦人訓解集》十卷



 《女訓集》六卷



 長孫皇后《女則要錄》十卷



 魏征《列女傳略》七卷



 武後《列女傳》一百卷



 又《孝女傳》二十卷



 《古今內範》一百卷



 《內範要略》十卷



 《保傅乳母傳》七卷



 《鳳樓新誡》二十卷



 王方慶《王氏女記》十卷



 又《王氏王嬪傳》五卷



 《續妒記》五卷



 尚宮宋氏《女論語》十篇



 薛蒙妻韋氏《續曹大家女訓》十二章韋溫女。蒙,字中明,開成中進士第。



 王摶妻楊氏《女誡》一卷



 右雜傳記類一百二十五家,一百四十六部,一千六百五十六卷。失姓名十四家,崔玄韋以下不著錄五十一家,二千五百七十四卷。總一百四十七家,一百五十一部。



 衛宏《漢舊儀》四卷



 董巴《大漢輿服志》一卷



 徐廣《車服雜注》一卷



 又《晉尚書儀曹新定儀注》四十一卷



 《晉儀注》三十九卷



 傅瑗《晉新定儀注》四十卷



 《晉尚書儀曹吉禮儀注》三卷



 《晉尚書儀曹事》九卷



 《晉雜儀注》二十一卷



 《宋尚書儀注》三十六卷



 《宋儀注》二卷



 張鏡《宋東宮儀記》二十三卷



 嚴植之《南齊儀注》二十八卷



 又《梁皇帝崩兇儀》十一卷



 《梁皇太子喪禮》五卷



 《梁王侯以下兇禮》九卷



 《士喪禮儀注》十四卷



 沈約《梁儀注》十卷



 又《梁祭地祗陰陽儀注》二卷



 鮑泉《新儀》三十卷



 明山賓等《梁吉禮》十八卷



 《梁吉禮儀注》四卷



 又十卷



 《梁尚書儀曹儀注》十八卷



 又二十卷



 《梁天子喪禮》七卷



 又五卷



 《梁大行皇帝皇后崩儀注》一卷



 《梁太子妃薨兇儀注》九卷



 《梁諸侯世子卒兇儀注》九卷



 《梁陳大行皇帝崩儀注》八卷



 賀瑒等《梁賓禮》一卷



 《梁賓禮儀注》十三卷



 陸璉《梁軍禮》四卷



 司馬褧《梁嘉禮》三十五卷



 又《嘉禮儀注》四十五卷



 《陳吉禮儀注》五十卷



 《陳雜吉儀注》三十卷



 《陳雜儀注》六卷



 《陳諸帝后崩儀注》五卷



 《陳雜儀注兇儀》十三卷



 《陳皇太后崩儀注》四卷儀曹撰。



 《陳皇太子妃薨儀注》五卷儀曹撰。



 張彥《陳賓禮儀注》六卷



 常景《後魏儀注》五十卷



 趙彥深《北齊吉禮》七十二卷



 《北齊皇太后喪禮》十卷



 高潁《隋吉禮》五十四卷



 牛弘、潘徽《隋江都集禮》一百二十卷



 《大賀鹵簿》一卷



 周遷《古今輿服雜事》十卷



 蕭子云《古今輿服雜事》二十卷



 《甲辰儀注》五卷



 摯虞《決疑要注》一卷



 崔豹《古今注》一卷



 《諸王國雜儀注》十卷



 《雜儀注》一百卷



 範汪《雜府州郡儀》十卷



 又《祭典》三卷



 何胤《喪服治禮儀注》九卷



 何點《理禮儀注》九卷



 《冠婚儀》四卷



 崔皓《婚儀祭儀》二卷



 何晏《魏明帝謚議》二卷



 《魏氏郊丘》三卷



 高堂隆《魏臺雜訪議》三卷



 《晉謚議》八卷



 《晉簡文謚議》四卷



 孔晁等《晉明堂郊社議》三卷



 蔡謨《晉七廟議》三卷



 干寶《雜議》五卷



 荀顗等《晉雜議》十卷



 王景之《要典》三十九卷



 王逸《齊典》四卷



 丘仲孚《皇典》五卷



 盧諶《雜祭注》六卷



 盧辨《祀典》五卷



 徐爰《家儀》一卷



 王儉《吉儀》二卷



 又《吊答書儀》十卷



 《皇室書儀》七卷



 鮑衡卿《皇室書儀》十三卷



 謝朏《書筆儀》二十卷



 謝允《書儀》二卷



 唐瑾《婦人書儀》八卷



 《童悟》十三卷



 紀僧真《玉璽譜》一卷



 姚察《傳國璽》十卷



 徐令言《玉璽正錄》一卷



 張大頤《明堂儀》一卷



 姚璠等《明堂儀注》三卷



 《皇太子方岳亞獻儀》二卷



 蕭子云《東宮雜事》二十卷



 陸開明、宇文愷《東宮典記》七十卷



 令狐德棻《皇帝封禪儀》六卷



 孟利貞《封禪錄》十卷



 裴守真《神嶽封禪儀注》十卷



 郭山惲《大享明堂儀注》二卷



 《親享太廟儀注》三卷



 裴矩、虞世南《大唐書儀》十卷



 竇維鍌《吉兇禮要》二十卷



 韋叔夏《五禮要記》三十卷



 王愨中《禮儀注》八卷



 楊炯《家禮》十卷



 《大唐儀禮》一百卷長孫無忌、房玄齡、魏徵、李百藥、顏師古、令狐德棻、孔穎達、於志寧等撰。《吉禮》六十篇,《賓禮》四篇,《軍禮》二十篇,《嘉禮》四十二篇,《兇禮》六篇,《國恤》五篇,總一百三十篇。貞觀十一年上。



 《永徽五禮》一百三十卷長孫無忌、侍中許敬宗、兼中書令李義府、黃門侍郎劉祥道、許圉師、太常卿韋琨、博士蕭楚材孔志約等撰。削《國恤》,以為豫兇事非臣子所宜論次,定著二百九十九篇。顯慶三年上。



 武後《紫宸禮要》十卷



 《開元禮》一百五十卷開元中,通事舍人王嵒請改《禮記》,附唐制度,張說引嵒就集賢書院詳議。說奏:「《禮記》,漢代舊文,不可更,請脩貞觀、永徽五禮為《開元禮》。命賈登、張烜、施敬本、李銳、王仲丘、陸善經、洪孝昌撰緝,蕭嵩總之。



 蕭嵩《開元禮義鏡》一百卷



 《開元禮京兆義羅》十卷



 《開元禮類釋》二十卷



 《開元禮百問》二卷



 顏真卿《禮樂集》十卷禮儀使所定。



 韋渠牟《貞元新集開元後禮》二十卷



 柳逞《唐禮纂要》六卷



 韋公肅《禮閣新儀》二十卷元和人。



 王彥威《元和曲臺禮》三十卷



 又《續曲臺禮》三十卷



 李弘澤《直禮》一卷林甫孫,開成太府卿



 韋述《東封記》一卷



 李襲譽《明堂序》一卷



 員半千《明堂新禮》三卷



 李嗣真《明堂新禮》十卷



 王涇《大唐郊祀錄》十卷貞元九年上,時為太常禮院脩撰。



 裴瑾《崇豐二陵集禮》卷亡。瑾,字封叔,光庭曾孫,元和吉州刺史。



 王方慶《三品官祔廟禮》二卷



 又《古今儀集》五十卷



 孟詵《家祭禮》一卷



 徐閏《家祭儀》一卷



 範傳式《寢堂時饗儀》一卷



 鄭正則《祠享儀》一卷



 周元陽《祭錄》一卷



 賈頊《家薦儀》一卷



 盧弘宣《家祭儀》卷亡。



 孫氏《仲享儀》一卷孫日用。



 劉孝孫《二儀實錄》一卷



 袁郊《二儀實錄衣服名義圖》一卷



 又《服飾變古元錄》一卷字之儀,滋子也。昭宗翰林學士。



 王晉《使範》一卷



 戴至德《喪服變服》一卷



 張戩《喪儀纂要》九卷



 孟詵《喪服正要》二卷



 商價《喪禮極議》一卷



 張薦《五服圖》卷亡。



 《葬王播儀》一卷



 仲子陵《五服圖》十卷貞元九年上。



 鄭氏《書儀》二卷鄭餘慶。



 裴茝《內外親族五服儀》二卷



 裴度《書儀》二卷



 又《書儀》三卷硃儔注。茝,元和太常少卿。



 杜有晉《書儀》二卷



 右儀注類六十一家,一百部,一千四百六十七卷。失姓名三十二家,竇維鍌以下不著錄四十九家,八百九十三卷。



 《漢建武律令故事》三卷



 《漢名臣奏》二十九卷



 《廷尉決事》二十卷



 《廷尉駁事》十一卷



 《廷尉雜詔書》二十六卷



 《南臺奏事》二十二卷



 應劭《漢朝議駁》三十卷



 陳壽《漢名臣奏事》三十卷



 《晉駁事》四卷



 《晉彈事》九卷



 賈充、杜預《刑法律本》二十一卷



 又《晉令》四十卷



 宗躬《齊永明律》八卷



 蔡法度《梁律》二十卷



 又《梁令》三十卷



 《梁科》二卷



 《條鈔晉宋齊梁律》二十卷



 範泉等《陳律》九卷



 又《陳令》三十卷



 《陳科》三十卷



 趙郡王叡《北齊律》二十卷



 《令》八卷



 《麟趾格》四卷文襄帝時撰。



 趙肅等《周律》二十五卷



 蘇綽《大統式》三卷



 張斐《律解》二十卷



 劉邵《律略論》五卷



 高熲等《隋律》十二卷



 牛弘等《隋開皇令》三十卷



 《隋大業律》十八卷



 《武德律》十二卷



 又《式》十四卷



 《令》三十一卷尚書左僕射裴寂、右僕射蕭瑀、大理卿崔善為、給事中王敬業、中書舍人劉林甫顏師古王孝達、涇州別駕靖延、太常丞丁孝烏、隋大理丞房軸、天策上將府參軍李桐客、太常博士徐上機等奉詔撰定。以五十三條附新律,餘無增改。武德七年上。



 《貞觀律》十二卷



 又《令》二十七卷



 《格》十八卷



 《留司格》一卷



 《式》三十三卷中書令房玄齡、右僕射長孫無忌、蜀王府法曹參軍裴弘獻等奉詔撰定。凡律五百條,令一千五百四十六條,格七百條。以尚書省諸曹為目,其常務留本司者,著為《留司格》。



 《永徽律》十二卷



 又《式》十四卷



 《式本》四卷



 《令》三十卷



 《散頒天下格》七卷



 《留本司行格》十八卷太尉無忌、司空李勣左僕射於志寧、右僕射張行成,侍中高季輔、黃門待郎宇文節柳奭、尚書右丞段寶玄、太常少卿令狐德棻、吏部侍郎高敬言、刑部侍郎劉燕客、給事中趙文恪、中書舍人李友益、少府丞張行實、太府丞王文端、大理丞元紹、刑部郎中賈敏行等奉詔撰定。分格為二部,以曹司常務為「行格」,天下所共為「散頒格」。永徽三年上。至龍朔二年,詔司刑太常伯源直心、少常伯李敬玄、司刑大夫李文禮復刪定,唯改官曹局名而已。題行格曰「留本司行格中本」,散頒格曰「天下散行格中本」。



 《律疏》三十卷無忌、李勣、於志寧、邢部尚書唐臨、大理卿段寶玄、尚書右丞劉燕客、御史中丞賈敏行等奉詔撰,永徽四年上。



 《永徽留本司格後》十一卷左僕射劉仁軌、右僕射戴至德、侍中張文瓘、中書令李敬玄、右庶子郝處俊、黃門侍郎來恆、左庶子高智周、右庶子李義琰、吏部侍郎裴行儉馬戴、兵部侍郎蕭德昭裴炎、工部侍郎李義琛、刑部侍郎張楚金、金部郎中盧律師等奉詔撰,儀鳳二年上。



 趙仁本《法例》二卷



 崔知悌《法例》二卷



 《垂拱式》二十卷



 又《格》十卷



 《新格》二卷



 《散頒格》三卷



 《留司格》六卷秋官尚書裴居道、夏官尚書同鳳閣鸞臺三品岑長倩、鳳閣侍郎同鳳閣鸞臺平章事韋方質、刪定官袁智弘、咸陽尉王守慎奉詔撰。加計帳、勾帳二式。垂拱元年上新格,武后制序。



 《刪垂拱式》二十卷



 又《散頒格》七卷中書令韋安石、禮部尚書同中書門下三品祝欽明、尚書右丞蘇瑰、兵部郎中狄光嗣等刪定,神龍元年上。



 《太極格》十卷戶部尚書同中書門下三品岑羲、中書侍郎同中書門下三品陸象先、右散騎常侍徐堅、右司郎中唐紹、刑部員外郎邵知新、大理寺丞陳義海、評事張名播、右衛長史張處斌、左衛率府倉曹參軍羅思貞、刑部主事閻義顓等刪定,太極元年上。



 《開元前格》十卷兵部尚書兼紫微令姚崇、黃門監盧懷慎、紫微侍郎兼刑部尚書李乂、紫微侍郎蘇廷頁、舍人呂延祚給事中魏奉古、大理評事高智靜、韓城縣丞侯郢璡、瀛州司法參軍閻義顓等奉詔刪定,開元三年上。



 《開元後格》十卷



 又《令》三十卷



 《式》二十卷吏部侍郎兼侍中宋璟、中書侍郎蘇頲、尚書左丞盧從願、吏部侍郎裴漼慕容珣、戶部侍郎楊滔、中書舍人劉令植、大理司直高智靜、幽州司功參軍侯郢璡等刪定,開元七年上。



 《格後長行敕》六卷侍中裴光庭、中書令蕭嵩等刪次,開元十九年上。



 《開元新格》十卷



 《格式律令事類》四十卷中書令李林甫、侍中牛仙客、御史中丞王敬從、左武衛胄曹參軍崔晃、衛州司戶參軍直中書陳承信、酸棗尉直刑部俞元杞等刪定,開元二十五年上。《度支長行旨》五卷



 王行先《律令手鑒》二卷



 元泳《式苑》四卷



 裴光庭《唐開元格令科要》一卷



 《元和格敕》三十卷權德輿、劉伯芻等集。



 《元和刪定制敕》三十卷許孟容、韋貫之、蔣乂、柳登等集。



 《大和格後敕》四十卷



 《格後敕》五十卷初,前大理丞謝登纂,凡六十卷。詔刑部詳定,去其繁復。大和七年上。



 狄兼謨《開成詳定格》十卷



 《大中刑法總要格後敕》六十卷刑部侍郎劉彖等纂。



 張戣《大中刑律統類》十二卷



 盧紓《刑法要錄》十卷裴向上之。



 張伾《判格》三卷



 李崇《法鑒》八卷



 右刑法類二十八家,六十一部,一千四卷。失姓名九家,自《開元新格》以下不著錄十三家,三百二十三卷。



 劉向《七略別錄》二十卷



 劉歆《七略》七卷



 荀勖《晉中經簿》十四卷



 又《新撰文章家集敘》五卷



 丘深之《晉義熙以來新集目錄》三卷



 王儉《宋元徽元年四部書目錄》四卷



 《今書七志》七十卷賀縱補注



 阮孝緒《七錄》十二卷



 丘賓卿《梁天監四年書目》四卷



 劉遵《梁東宮四部書目》四卷



 《陳天嘉四部書目》四卷



 牛弘《隋開皇四年書目》四卷



 王劭《隋開皇二十年書目》四卷



 殷淳《四部書目序錄》三十九卷



 楊松珍《史目》三卷



 摯虞《文章志》四卷



 宋明帝《晉江左文章志》二卷



 沈約《宋世文章志》二卷



 傅亮《續文章志》二卷



 《名手畫錄》一卷



 虞龢《法書目錄》六卷



 《群書四錄》二百卷殷踐猷、王愜、韋述、余欽、毋煚、劉彥直、王灣、王仲丘撰,元行沖上之。



 毋煚《古今書錄》四十卷



 韋述《集賢書目》一卷



 李肇《經史釋題》二卷



 宗諫注《十三代史目》十卷



 常寶鼎《文選著作人名目》三卷



 尹植《文樞秘要目》七卷鈔《文思博要》、《藝文類聚》為秘要。



 《唐書敘例目錄》一卷



 孫玉汝《唐列聖實錄目》二十五卷



 《吳氏西齋書目》一卷吳兢。



 《河南東齋史目》三卷



 蔣彧《新集書目》一卷



 杜信《東齋籍》二十卷字立言,元和國子司業。



 右目錄類十九家,二十二部,四百六卷。失姓名二家,毋煚以下不著錄十二家,一百一十四卷。



 宋衷《世本》四卷



 《世本別錄》一卷



 宋均注《帝譜世本》七卷



 王氏注《世本譜》二卷



 《漢氏帝王譜》二卷



 《齊永元中表簿》六卷



 《梁大同四年表簿》三卷



 《齊梁宗簿》三卷



 《梁親表譜》五卷



 《後魏皇帝宗族譜》四卷



 元暉業《後魏辨宗錄》二卷



 《後魏譜》二卷



 《後魏方司格》一卷



 《齊高氏譜》六卷



 《周宇文氏譜》一卷



 賈冠《國親皇太子親傳》四卷



 王儉《百家集譜》十卷



 王僧孺《百家譜》三十卷



 又《十八州譜》七百一十二卷



 徐勉《百官譜》二十卷



 賈執《百家譜》五卷



 又《姓氏英賢譜》一百卷



 何承天《姓苑》十卷



 賈希鏡《氏族要狀》十五卷



 《官族傳》十五卷



 《冀州姓族譜》七卷



 《洪州諸姓譜》九卷



 《袁州諸姓譜》七卷



 《司馬氏世家》二卷



 《楊氏譜》一卷



 《蘇氏譜》一卷



 《孫氏譜記》十五卷



 《韋氏譜》十卷韋鼎。



 《裴氏家牒》二十卷裴守真。



 《大唐氏族志》一百卷高士廉、韋挺、岑文本、令狐德棻撰。



 《姓氏譜》二百卷許敬宗、李義府、孔志約、陽仁卿、史玄道、呂才撰。



 柳沖《大唐姓族系錄》二百卷



 路敬淳《衣冠譜》六十卷



 又《著姓略記》二十卷



 王元感《姓氏實論》十卷



 崔日用《姓苑略》一卷



 岑羲《氏族錄》卷亡。



 王方慶《王氏家牒》十五卷



 又《家譜》二十卷



 《王氏著錄》十卷



 韋述《開元譜》二十卷



 《國朝宰相甲族》一卷



 《百家類例》三卷



 《唐新定諸家譜錄》一卷李林甫等。



 林寶《元和姓纂》十卷



 竇從一《系纂》七卷



 陳湘《姓林》五卷



 孔至《姓氏雜錄》一卷



 李利涉《唐官姓氏記》五卷初,十卷。利涉貶南方,亡其半。



 又編《古命氏》三卷



 柳璨《姓氏韻略》六卷



 蕭穎士《梁蕭史譜》二十卷



 柳芳《永泰新譜》二十卷一作《皇室新譜》。



 柳璟《續譜》十卷



 《皇唐玉牒》一百一十卷開成二年,李衢、李寶撰。



 《唐皇室維城錄》一卷



 李匡文《天潢源派譜》一卷



 又《唐偕日譜》一卷



 《玉牒行樓》一卷



 《皇孫郡王譜》一卷



 《元和縣主譜》一卷



 《家譜》一卷



 李衢《大唐皇室新譜》一卷



 黃恭之《孔子系葉傳》二卷



 《謝氏家譜》一卷



 《東萊呂氏家譜》一卷



 《薛氏家譜》一卷



 《顏氏家譜》一卷



 《虞氏家譜》一卷



 《孫氏家譜》一卷



 《吳郡陸氏宗系譜》一卷陸景獻。



 《劉氏譜考》三卷



 《劉氏家史》十五卷並劉子玄。



 《紀王慎家譜》一卷



 《蔣王惲家譜》一卷



 《李用休家譜》二卷紀王慎之後。



 《徐氏譜》一卷徐商。



 《徐義倫家譜》一卷



 《劉晏家譜》一卷



 《劉輿家譜》一卷



 《周長球家譜》一卷



 《施氏家譜》二卷



 《萬氏譜》一卷



 《滎陽鄭氏家譜》一卷



 《竇氏家譜》一卷懿宗時國子博士竇澄之。



 《鮮于氏家譜》一卷



 《趙郡東祖李氏家譜》二卷



 《李氏房從譜》一卷



 《韋氏諸房略》一卷韋綯。



 《諱行錄》一卷



 右譜牒類十七家,三十九部,一千六百一十七卷。王元感以下不著錄二十二家,三百三十三卷。



 《三輔黃圖》一卷



 《三輔舊事》三卷



 《漢宮閣簿》三卷



 《洛陽宮殿簿》三卷



 葛洪《西京雜記》二卷



 薛冥《西京記》三卷



 潘岳《關中記》一卷



 《職方記》十六卷



 陸機《洛陽記》一卷



 《晉太康土地記》十卷



 戴延之《洛陽記》一卷



 《太康州郡縣名》五卷



 《後魏洛陽記》五卷



 《後魏諸州記》二十卷



 楊佺期《洛城圖》一卷



 周處《風土記》十卷



 鄧基、陸澄《地理志》一百五十卷



 圈稱《陳留風俗傳》三卷



 任昉《地記》二百五十二卷



 楊雄《蜀王本記》一卷



 虞茂《區宇圖》一百二十八卷



 譙周《三巴記》一卷



 郎蔚之《隋圖經集記》一百卷



 李充《益州記》三卷



 《周地圖》一百三十卷



 郭仲產《荊州記》二卷



 《雜記》十二卷



 鮑堅《南雍州記》三卷



 《雜地志》五卷



 阮敘之《南兗州記》一卷



 《地理志書鈔》十卷



 山謙之《南徐州記》二卷



 《地域方丈圖》一卷



 劉損之《京口記》二卷



 《地域方尺圖》一卷



 孫處玄《潤州圖注》二十卷



 雷次宗《豫章記》一卷



 《京邦記》二卷



 鄭緝之《東陽記》一卷



 《分吳會丹楊三郡記》二卷



 張僧監《潯陽記》二卷



 《西河舊事》一卷



 李叔布《齊州記》四卷



 闞駰《十三州志》十四卷



 張勃《吳地記》一卷



 顧野王《輿地志》三十卷



 晏模《齊地記》二卷



 又《十國都城記》十卷



 陸翽《鄴中記》二卷



 周明帝《國都城記》九卷



 劉芳《徐地錄》一卷



 郭璞注《山海經》二十三卷



 梁元帝《職貢圖》一卷



 又《山海經圖贊》二卷



 又《荊南地志》二卷



 《山海經音》二卷



 王範《交廣二州記》一卷



 桑欽《水經》三卷一作郭璞撰。



 樊文深《中岳潁州志》五卷



 酈道元注《水經》四十卷



 《秣陵記》二卷



 僧道安《四海百川水源記》一卷



 《湘州記》四卷



 又一卷



 《湘州圖副記》一卷



 《江圖》二卷



 庾仲雍《江記》五卷



 諸葛潁《巡撫揚州記》七卷



 又《漢水記》五卷



 戴祚《西征記》二卷



 《尋江源記》五卷



 郭緣生《述征記》二卷



 劉澄之《永初山川古今記》二十卷



 姚最《述行記》二卷



 李氏《宜都山川記》一卷



 沈懷文《隨王入沔記》十卷



 沈瑩《臨海水土異物志》一卷



 《魏聘使行記》五卷



 楊孚《交州異物志》一卷



 李彤《聖賢塚墓記》一卷



 陳祈暢《異物志》一卷



 宋云《魏國以西十一國事》一卷



 萬震《南州異物志》一卷



 沈懷遠《南越志》五卷



 硃應《扶南異物志》一卷



 程士章《西域道里記》三卷



 《京兆郡方物志》二十卷



 常駿等《赤土國記》二卷



 《諸郡土俗物產記》十九卷



 王玄策《中天竺國行記》十卷



 《涼州異物志》二卷



 僧智猛《游行外國傳》一卷



 《廟記》一卷



 僧法盛《歷國傳》二卷



 薛泰《輿駕東幸記》一卷



 《日南傳》一卷



 《林邑國記》一卷



 《真臘國事》一卷



 《交州以來外國傳》一卷



 《奉使高麗記》一卷



 《西南蠻入朝首領記》一卷



 裴矩《高麗風俗》一卷



 鄧行儼《東都記》三十卷貞觀著作郎。



 《括地志》五百五十卷



 又《序略》五卷魏王泰命著作郎蕭德言、秘書郎顧胤、記室參軍蔣亞卿、功曹參軍謝偃蘇勖撰。



 《長安四年十道圖》十三卷



 《開元三年十道圖》十卷



 《劍南地圖》二卷



 李播《方志圖》卷亡。



 《西域國志》六十卷高宗遣使分往康國、吐火羅,訪其風俗物產,畫圖以聞。詔史官撰次,許敬宗領之,顯慶三年上。



 李吉甫《元和郡縣圖志》五十四卷



 又《十道圖》十卷



 《古今地名》三卷



 《刪水經》十卷



 梁載言《十道志》十六卷



 王方慶《九嵕山志》十卷



 賈耽《地圖》十卷



 又《皇華四達記》十卷



 《古今郡國縣道四夷述》四十卷



 《關中隴右山南九州別錄》六卷



 《貞元十道錄》四卷



 《吐蕃黃河錄》四卷



 韋澳《諸道山河地名要略》九卷一作《處分語》。



 劉之推、文括《九州要略》三卷



 《郡國志》十卷



 馬敬寔《諸道行程血脈圖》一卷



 鄧世隆《東都記》三十卷



 韋機《東都記》二十卷



 韋述《兩京新記》五卷



 《兩京道里記》三卷



 李仁實《戎州記》一卷



 盧鵂《嵩山記》一卷天寶人。



 馬溫《鄴都故事》二卷肅、代時人。



 劉公銳《鄴城新記》三卷



 張周封《華陽風俗錄》一卷字子望,西川節度使李德裕從事,試協律郎。



 盧求《成都記》五卷西川節度使白敏中從事。



 鄭韋《益州理亂記》三卷



 李璋《太原事跡記》十四卷



 張文規《吳興雜錄》七卷



 房千里《南方異物志》一卷



 孟琯《嶺南異物志》一卷



 劉恂《嶺表錄異》三卷



 餘知古《渚宮故事》十卷文宗時人。



 吳從政《襄沔記》三卷



 張氏《燕吳行役記》二卷宣宗時人,失名。



 韋宙《零陵錄》一卷



 張密《廬山雜記》一卷



 張容《九江新舊錄》三卷咸通人。



 莫休符《桂林風土記》三卷



 段公路《北戶雜錄》三卷文昌孫。



 林住《閩中記》十卷



 裴矩又撰《西域圖記》三卷



 顧愔《新羅國記》一卷大歷中,歸崇敬使新羅,愔為從事。



 張建章《渤海國記》三卷



 戴斗《諸蕃記》一卷



 達奚通《海南諸蕃行記》一卷



 袁滋《雲南記》五卷



 李繁《北荒君長錄》三卷



 高少逸《四夷朝貢錄》十卷



 呂述《黠戛斯朝貢圖傳》一卷字脩業,會昌秘書少監,商州刺史。



 樊綽《蠻書》十卷咸通嶺南西道節度使蔡襲從事。



 竇滂《雲南別錄》一卷



 《雲南行記》一卷



 徐云虔《南詔錄》三卷乾符中人。



 右地理類六十三家,一百六部,一千二百九十二卷。失姓名三十一家,李播以下不著錄五十三家,九百八十九卷。



\end{pinyinscope}