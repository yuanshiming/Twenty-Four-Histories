\article{志第十三上 儀衛上}

\begin{pinyinscope}

 唐制,天子居曰「衙」,行曰「駕」,皆有衛有嚴。羽葆、華蓋、旌旗、罕畢、車馬之眾盛矣,皆安徐而不嘩。其人君舉動必以扇人工智能又稱「智能模擬」。以研究和模擬人的智能活動,出入則撞鐘,庭設樂宮,道路有鹵簿、鼓吹。禮官百司必備物而後動,蓋所以為慎重也。故慎重則尊嚴,尊嚴則肅恭。夫儀衛所以尊君而肅臣,其聲容文採,雖非三代之制,至其盛也,有足取焉。



 衙。



 凡朝會之仗,三衛番上,分為五仗,號衙內五衛:一曰供奉仗,以左右衛為之;二曰親仗,以親衛為之;三曰勛仗,以勛衛為之;四曰翊仗,以翊衛為之;皆服鶡冠、緋衫裌;五曰散手仗,以親、勛、翊衛為之,服緋施裲襠,繡野馬;皆帶刀捉仗,列坐於東西廊下。



 每月以四十六人立內廊閣外,號曰內仗。以左右金吾將軍當上,中郎將一人押之,有押官,有知隊仗官。朝堂置左右引駕三衛六十人,以左右衛、三衛年長強直能糾劾者為之,分五番。有引駕佽飛六十六人,以佽飛、越騎、步射為之,分六番,每番皆有主帥一人。坐日引駕升殿,金吾大將軍各一人押之,號曰押引駕官。中郎將、郎將各一人,檢校引駕事。又有千牛仗,以千牛備身、備身左右為之。千牛備身冠進德冠、服褲褶;備身左右服如三衛。皆執御刀、弓箭,升殿列御座左、右。



 內外諸門以排道人帶刀捉仗而立,號曰立門仗。宣政左右門仗、內仗,皆分三番而立,號曰交番仗。諸衛有挾門隊、長槍隊。承天門內則左、右衛挾門隊列東、西廊下,門外則左、右驍衛門隊列東、西廊下。長樂、永安門內則左、右威衛挾門隊列東、西廊下,門外則左、右領軍衛挾門隊列東、西廊下。嘉德門內則左、右武衛挾門隊列東、西廊下。車駕出皇城,則挾門隊皆從。長槍隊有漆槍、木槍、白桿槍、樸頭槍。



 每夜,第一鼕黶,諸隊仗佩弓箭、胡祿,出鋪立廊下,按槊,張弓、捻箭、彀弩。第二鼕黶後,擊鐘訖,持更者舉槊,鐘聲絕則解仗。一點,持更人按槊,持弓者穩箭唱號,諸衛仗隊皆分更行探。宿衛門閣仗隊,鍪、甲、蕞,擐左襻,餘仗隊唯持更人蕞一具,供奉、散手仗亦持更、蕞、甲。



 每朝,第一鼕黶訖,持更卸皆舉,張弓者攝箭收弩,立門隊及諸隊仗皆立於廊下。第二鼕黶聲絕,按槊、弛弓、收鋪,諸門挾門隊立於階下。復一刻,立門仗皆復舊,內外仗隊立於階下。



 元日、冬至大朝會、宴見蕃國王,則供奉仗、散手仗立於殿上;黃麾仗、樂縣、五路、五副路、屬車、輿輦、繖二、翰一,陳於庭;扇一百五十有六,三衛三百人執之,陳於兩廂。



 黃麾仗,左、右廂各十二部,十二行。第一行,長戟,六色氅,領軍衛赤氅,威衛青氅、黑氅,武衛鶩氅、驍衛白氅,左右衛黃氅,黃地雲花襖、冒。第二行,儀鍠,五色幡,赤地雲花襖、冒。第三行,大槊,小孔雀氅,黑地雲花襖,冒。第四行,小戟、刀、楯,白地雲花襖、冒。第五行,短戟,大五色鸚鵡毛氅,青地雲花襖、冒。第六行,細射弓箭,赤地四色雲花襖、冒。第七行,小槊,小五色鸚鵡毛氅,黃地雲花襖、冒。第八行,金花硃滕絡楯刀,赤地雲花襖、冒。第九行,戎,雞毛氅,黑地雲花襖、冒。第十行,細射弓箭,白地雲花襖、冒。第十一行,大鋋,白毦,青地雲花襖、冒。第十二行,金花綠滕絡楯刀,赤地四色雲花襖、冒。十二行皆有行滕、鞋、襪。



 前黃麾仗,首左右廂各一部,部十二行,行十人,左右領軍衛折沖都尉各一人,領主帥各十人,師子袍、冒。次左右廂皆一部,部十二行,行十人,左右威衛果毅都尉各一人,領主帥各十人,豹文袍、帽。次廂各一部,部十二行,行十人,左右武衛折沖都尉各一人,主帥各十人。次廂各一部,部十二行,行十人,左右衛折沖都尉各一人,主帥各十人。次當御廂各一部,部十二行,行十人,左右衛果毅都尉各一人,主帥各十人。次後廂各一部,部十二行,行十人,左右驍衛折沖都尉各一人,主帥各十人。次後廂各一部,部十二行,行十人,左右武衛果毅都尉各一人。主帥各十人。次後左右廂各一部,部十二行,行十人,左右威衛折沖都尉各一人,主帥各十人。次後左右廂各一部,部十二行,行十人,左右威衛果毅都尉各一人,主帥各十人。次後左右廂各一部,部十二行,行十人,左右領軍衛果毅都尉各一人,主帥各十人。次盡後左右廂,軍衛、主帥各十人護後,被師子文袍冒。



 左右領軍衛黃麾仗,首尾廂皆絳引幡,二十引前,十掩後。十廂各獨揭鼓十二重,重二人,赤地雲花襖、冒,行滕、鞋、襪,居黃麾仗外。每黃麾仗一部,鼓一,左右衛、左右驍衛、左右武衛、左右威衛將軍各一人,大將軍各一人,左右領軍衛大將軍各一人檢校,被繡袍。



 次左右衛黃旗仗,立於兩階之次,鍪、甲、弓、箭、刀、楯皆黃,隊有主帥以下四十人,皆戎服,被大袍,二人引旗,一人執,二人夾,二十人執槊,餘佩弩、弓箭。第一麟旗隊,第二角端旗隊,第三赤熊旗隊,折沖都尉各一人檢校,戎服,被大袍,佩弓箭、橫刀。又有夾轂隊,廂各六隊,隊三十人,胡木鍪、毦、蜀鎧、懸鈴、覆膊、錦臂、白行滕、紫帶、鞋襪,持、楯、刀;廂各折沖都尉一人、果毅都尉二人檢校,冠進德冠,被紫縚連甲、緋繡葵花文袍。第一隊、第四隊,硃質鍪、鎧,緋褲。第二隊、第五隊,白質鍪、鎧,紫褲。第三隊、第六隊,黑質鍪、鎧,皁褲。



 次左右驍衛赤旗仗,坐於東西廊下,鍪、甲、弓、箭、刀、楯皆赤,主帥以下如左右衛。第一鳳旗隊,第二飛黃旗隊,折沖都尉各一人檢校。第三吉利旗隊,第四兕旗隊,第五太平旗隊,果毅都尉各一人檢校。



 又有親、勛、翊衛仗,廂各三隊壓角,隊皆有旗,一人執,二人引,二人夾,校尉以下翊衛以上三十五人,皆平巾幘、緋裲襠、大口褲,帶橫刀;執槊二十人,帶弩四人,帶弓箭十一人。第一隊鳳旗,大將軍各一人主之。第二隊飛黃旗,將軍各一人主之。第三隊吉利旗,郎將一人主之。



 次左右武衛白旗仗,居驍衛之次,鍪、甲、弓、箭、刀、楯皆白,主帥以下如左右衛。第一五牛旗隊,黃旗居內,赤、青居左,白、黑居右,各八人執。第二飛麟旗隊,第三駃騠旗隊,第四鸞旗隊,果毅都尉各一人檢校。第五犀牛旗隊,第六鵕鸃旗隊,第七騏驎旗隊,第八騼蜀旗隊,折沖都尉各一人檢校。持鈒沄,果毅都尉各一人、校尉二人檢校。前隊執銀裝長刀,紫黃綬紛。絳引幡一、金節十二,分左右。次罕、畢、硃雀幢、叉,青龍、白虎幢、道蓋、叉,各一。自絳引幡以下,執者服如黃麾。執罕、畢及幢者,平陵冠、硃衣、革帶。左罕右畢,左青龍右白虎。稱長一人,出則告警,服如黃麾。鈒、戟隊各一百四十四人,分左右三行應蹕,服如黃麾。果毅執青龍等旗,將軍各一人檢校;旅帥二人執銀裝長刀,紫黃綬紛,檢校後隊。



 次左右威衛黑旗仗,立於階下,鍪、甲、弓、箭、楯、槊皆黑,主帥以下如左右衛。第一黃龍負圖旗隊,第二黃鹿旗隊,第三騶牙旗隊,第四蒼烏旗隊,果毅都尉各一人檢校。



 次左右領軍衛青旗仗,居威衛之次,鍪、甲、弓、箭、楯、皆青,主帥以下如左右衛。第一應龍旗隊,第二玉馬旗隊,第三三角獸旗隊,果毅都尉各一人檢校;第四白狼旗隊,第五龍馬旗隊,第六金牛旗隊,折沖都尉各一人檢校。



 又有殳仗、步甲隊,將軍各一人檢校。殳仗左右廂千人,廂別二百五十人執殳,二百五十人執叉,皆赤地雲花襖、冒,行滕、鞋襪。殳、叉以次相間。左右領軍衛各一百六十人,左右武衛各一百人,左右威衛、左右驍衛、左右衛各八十人。左右廂有主帥三十八人,平巾幘、緋裲襠、大口褲,執儀刀。廂有左右衛各三人,左右驍衛、左右武衛、左右威衛、左右領軍衛各四人,以主殳仗,被豹文袍、冒;領軍衛、師子文袍。步甲隊從左右廂各四十八,前後皆二十四。每隊折沖都尉一人主之,被繡袍。每隊一人,戎服大袍,帶橫刀,執旗;二人引,二人夾,皆戎服大袍,帶弓箭橫刀。隊別三十人,被甲、臂韝、行滕、鞋襪。每一隊鍪、甲、覆膊、執弓箭,一隊胡木鍪及毦、蜀鎧、覆膊,執刀、楯、相間。第一隊,赤質鍪、甲,赤弓、箭,折沖都尉各一人主之,執鶡雞旗。第二隊,赤質鍪、鎧,赤刀、楯、,果毅都尉各一人主之,執豹旗。第三隊,青質鍪、甲,青弓、箭,折沖都尉各一人主之。第四隊,青質鍪、鎧,青刀、楯、,果毅都尉各一人主之。第五隊,黑質鍪、甲,黑弓、箭,左右威衛折沖都尉各一人主之。第六隊,黑質鍪、鎧,黑刀、楯、,果毅都尉各一人主之。第七隊,白質鍪、甲,白弓、箭,左右武衛折沖都尉各一人主之。第八隊,白質鍪、鎧,白刀、楯、果毅都尉各一人主之。第九隊,黃質鍪、甲,黃弓、箭,左右驍衛折沖都尉各一人主之。第十隊,黃質鍪、鎧,黃刀、楯、,果毅都尉各一人主之。第十一隊,黃質鍪、甲,黃弓、箭,左右衛折沖都尉各一人主之。第十二隊,黃質鍪、鎧,黃刀、楯、,果毅都尉各一人主之。次後第一隊,黃質鍪、鎧,黃刀、楯、,左右衛折沖都尉各一人主之。至第十二隊與前同。



 次左右金吾衛闢邪旗隊,折沖都尉各一人檢校。又有清游隊、硃雀隊、玄武隊。清游隊建白澤旗二,各一人執,帶橫刀;二人引,二人夾,皆帶弓箭、橫刀。左右金吾衛折沖都尉各一人,帶弓箭、橫刀,各領四十人,皆帶橫刀,二十人持槊,四人持弩,十六人帶弓箭。硃雀隊建硃雀旗,一人執,引、夾皆二人,金吾衛折沖都尉一人主之,領四十人,二十人持槊,四人持弩,十六人帶弓箭,又二人持槊,皆佩橫刀,槊以黃金塗末。龍旗十二,執者戎服大袍,副竿二,各一人執,戎服大袍,分左右,果毅都尉各一人主之。大將軍各一人檢校二隊。玄武隊建玄武旗,一人執,二人引,二人夾,平巾幘、黑裲襠、黑裌、大口褲,左右金吾衛折沖都尉各一人主之,各領五十人,持槊二十五人,持弩五人,帶弓箭二十人,又二人持槊。諸衛挾門隊、長槍隊與諸隊相間。



 朝日,殿上設黼扆、躡席、熏爐、香案。御史大夫領屬官至殿西廡,從官硃衣傳呼,促百官就班,文武列於兩觀。監察御史二人立於東、西朝堂磚道以涖之。平明,傳點畢,內門開。監察御史領百官入,夾階,監門校尉二人執門籍,曰:「唱籍」。既視籍,曰:「在」。入畢而止。次門亦如之。序班於通乾、觀象門南,武班居文班之次。入宣政門,文班自東門而入,武班自西門而入,至閣門亦如之。夾階校尉十人同唱,入畢而止。宰相、兩省官對班於香案前,百官班於殿庭左右,巡使二人分涖於鐘鼓樓下,先一品班,次二品班,次三品班,次四品班,次五品班。每班尚書省官為首。武班供奉者立於橫街之北,次千牛中郎將,次千牛將軍,次過狀中郎將一人,次接狀中郎將一人,次押柱中郎將一人,次押柱中郎一人,次排階中郎將一人,次押散手仗中郎將一人,次左右金吾衛大將軍。凡殿中省監、少監,尚衣、尚舍、尚輦奉御,分左右隨繖、扇而立。東宮官居上臺官之次,王府官又次之,唯三太、三少、賓客、庶子、王傅隨本品。侍中奏「外辦」,皇帝步出西序門,索扇,扇合。皇帝升御座,扇開。左右留扇各三。左右金吾將軍一人奏「左右廂內外平安」。通事舍人贊宰相兩省官再拜,升殿。內謁者承旨喚仗,左右羽林軍勘以木契,自東西閣而入。內侍省五品以上一人引之,左右衛大將軍、將軍各一人押之。二十人以下入,則不帶仗。三十人入,則左右廂監門各二人,千牛備身各四人,三衛各八人,金吾一人。百人入,則左右廂監門各六人,千牛備身各四人,三衛三十三人,金吾七人。二百人,則增以左右武衛、威衛、領軍衛、金吾衛、翊衛等。凡仗入,則左右廂加一人監捉永巷,御刀、弓箭。及三衛帶刀入,則曰:「仗入」;三衛不帶刀而入,則曰「監引入」。朝罷,皇帝步入東序門,然後放仗。內外仗隊,七刻乃下。常參、輟朝日,六刻即下。宴蕃客日,隊下,復立半仗於兩廊。朔望受朝及蕃客辭見,加纛、槊沄,儀仗減半。凡千牛仗立,則全仗立。太陽虧,昏塵大霧,則內外諸門皆立仗。泥雨,則延三刻傳點。



 駕。



 大駕鹵簿。天子將出,前二日,太樂令設宮縣之樂於庭。晝漏上五刻,駕發。前發七刻,擊一鼓為一嚴。前五刻,擊二鼓為再嚴,侍中版奏「請中嚴」。有司陳鹵簿。前二刻,擊三鼓為三嚴,諸衛各督其隊與鈒、戟以次入陳殿庭。通事舍人引群官立朝堂,侍中、中書令以下奉迎於西階。侍中負寶,乘黃令進路於太極殿西階,南向;千牛將軍一人執長刀立路前,北向;黃門侍郎一人立侍臣之前;贊者二人。既外辦,太僕卿攝衣而升,正立執轡。天子乘輿以出,降自西階,曲直華蓋,警蹕,侍衛。千牛將軍前執轡,天子升路,太僕卿授綏,侍中、中書令以下夾侍。黃門侍郎前奏「請發」。鑾駕動,警蹕,鼓傳音,黃門侍郎與贊者夾引而出,千牛將軍夾路而趨。



 駕出承天門,侍郎乘馬奏「駕少留,敕侍臣乘馬」。侍中前承制,退稱:「制曰可」。黃門侍郎退稱:「侍臣乘馬。」贊者承傳,侍臣皆乘。侍衛之官各督其屬左右翊駕,在黃麾內。符寶郎奉六寶與殿中後部從,在黃鉞內。侍中、中書令以下夾侍路前,贊者在供奉官內。侍臣乘畢,侍郎奏「請車右升」。侍中前承制,退稱:「制曰可」。侍郎復位,千牛將軍升。侍郎奏「請發」。萬年縣令先導,次京兆牧、太常卿、司徒、御史大夫、兵部尚書,皆乘路,鹵簿如本品。



 次清游隊。次左右金吾衛大將軍各一人,帶弓箭橫刀,檢校龍旗以前硃雀等隊,各二人持槊,騎夾。次左右金吾衛果毅都尉各一人,帶弓箭橫刀,領夾道鐵甲佽飛。次虞候佽飛四十八騎,平巾幘、緋裲襠、大口褲,帶弓箭、橫刀,夾道分左右,以屬黃麾仗。次外鐵甲佽飛二十四人,帶弓箭、橫刀,甲騎具裝,分左右廂,皆六重,以屬步甲隊。



 次硃雀隊。次指南車、記里鼓車、白鷺車、鸞旗車、闢惡車、皮軒車,皆四馬,有正道匠一人,駕士十四人,皆平巾幘、大口褲、緋衫。太卜令一人,居闢惡車,服如佽飛,執弓箭。左金吾衛隊正一人,居皮軒車,服平巾幘、緋裲襠,銀裝儀刀,紫黃綬紛,執弩。次引駕十二重,重二人,皆騎,帶橫刀。自皮軒車後,屬於細仗前,槊、弓箭相間,左右金吾衛果毅都尉各一人主之。次鼓吹。次黃麾仗一,執者武弁、硃衣、革帶,二人夾。次殿中侍御史二人導。次太史監一人,書令史一人,騎引相風、行漏輿。次相風輿,正道匠一人,輿士八人,服如正道匠。次扛鼓、金鉦,司辰、典事匠各一人,刻漏生四人,分左右。次行漏生,正道匠一人,輿士十四人。



 次持鈒前隊。次御馬二十四,分左右,各二人馭。次尚乘奉御二人,書令史二人,騎從。



 次左青龍右白虎旗,執者一人,服如正道匠,引、夾各二人,皆騎。次左右衛果毅都尉各一人,各領二十五騎,二十人執槊,四人持弩,一人帶弓箭,行儀刀仗前。次通事舍人,四人在左,四人在右。侍御史,一人在左,一人在右。御史中丞,一人在左,一人在右。左拾遺一人在左,右拾遺一人在右。左補闕一人在左,右補闕一人在右。起居郎一人在左,起居舍人一人在右。諫議大夫,一人在左,一人在右。給事中二人在左,中書舍人二人在右。黃門侍郎二人在左,中書侍郎二人在右。左散騎常侍一人在左,右散騎常侍一人在右。侍中二人在左,中書令二人在右。通事舍人以下,皆一人從。次香蹬一,有衣,繡以黃龍,執者四人,服如折沖都尉。



 次左右衛將軍二人,分左右,領班劍、儀刀,各一人從。次班劍、儀刀,左右廂各十二行:第一左右衛親衛各五十三人,第二左右衛親衛各五十五人,第三左右衛勛衛各五十七人,第四左右衛勛衛各五十九人,各執金銅裝班劍,纁硃綬紛;第五左右衛翊衛各六十一人,第六左右衛翊衛各六十三人,第七左右衛翊衛各六十五人,第八左右驍衛各六十七人,各執金銅裝儀刀,綠綟綬紛;第九左右武衛翊衛各六十九人,第十左右威衛翊衛各七十一人,第十一左右領軍衛翊衛各七十三人,第十二左右金吾衛翊衛各七十五人,各執銀裝儀刀,紫黃綬紛。自第一行有曲折三人陪後門,每行加一人,至第十二行曲折十四人。



 次左右廂,諸衛中郎將主之,執班劍、儀刀,領親、勛、翊衛。次左右衛郎將各一人,皆領散手翊衛三十人,佩橫刀,騎,居副仗槊翊衛內。次左右驍衛郎將各一人,各領翊衛二十八人,甲騎具裝,執副仗槊,居散手衛外。次左右衛供奉中郎將,郎將四人,各領親、勛、翊衛四十八人,帶橫刀,騎,分左右,居三衛仗內。



 次玉路,駕六馬,太僕卿馭之,駕士三十二人。凡五路,皆有副。駕士皆平巾幘、大口褲,衫從路色。玉路,服青衫。千牛衛將軍一人陪乘,執金裝長刀,左右衛大將軍各一人騎夾,皆一人從,居供奉官後。次千牛衛將軍一人,中郎將二人,皆一人從。次千牛備身、備身左右二人,騎,居玉路後,帶橫刀,執御刀、弓箭。次御馬二,各一人馭。次左右監門校尉二人,騎,執銀裝儀刀,居後門內。



 次衙門旗,二人執,四人夾,皆騎,赤綦襖、黃冒、黃袍。次左右監門校尉各十二人,騎,執銀裝儀刀,督後門,十二行,仗頭皆一人。次左右驍衛、翊衛各三隊,居副仗槊外。次左右衛夾轂,廂各六隊。



 次大繖二,執者騎,橫行,居衙門後。次雉尾障扇四,執者騎,夾繖。次腰輿,輿士八人。次小團雉尾扇四,方雉尾扇十二,花蓋二,皆執者一人,夾腰輿。自大繖以下,執者服皆如折沖都尉。次掌輦四人,引輦。次大輦一,主輦二百人,平巾幘、黃絲布衫、大口褲、紫誕帶、紫行滕、鞋襪。尚輦奉御二人,主腰輿,各書令史二人騎從。次殿中少監一人,督諸局供奉事,一人從。次諸司供奉官。次御馬二十四,各二人馭,分左右。次尚乘直長二人,平巾幘、緋褲褶,書令史二人騎從,居御馬後。



 次持鈒沄。次大繖二,雉尾扇八,夾繖左右橫行。次小雉尾扇。硃畫團扇,皆十二,左右橫行。次花蓋二,叉二。次俾倪十二,左右橫行。次玄武幢一,叉一,居絳麾內。次絳麾二,左右夾玄武幢。次細槊十二,孔雀為毦,左右橫行,居絳麾後。自鈒、戟以下,執者服如黃麾仗,唯玄武幢執者服如罕、畢。



 次後黃麾,執者一人,夾二人,皆騎。次殿中侍御史二人,分左右,各令史二人騎從,居黃麾後。次大角。次方輦一,主輦二百人。次小輦一,主輦六十人。次小輿一,奉輿十二人,服如主輦。次尚輦直長二人,分左右,檢校輦輿,皆書令史二人騎從。次左右武衛五牛旗輿五,赤青居左,黃居中,白黑居右,皆八人執之,平巾幘、大口褲,衫從旗色,左右威衛隊正各一人主之,騎,執銀裝長刀。次乘黃令一人,丞一人,分左右,檢校玉路,皆府史二人騎從。



 次金路、象路、革路、木路,皆駕六馬,駕士三十二人。次五副路,皆駕四馬,駕士三十八人。次耕根車,駕六馬,駕士三十二人。次安車、四望車,皆駕四馬,駕士二十四人。次羊車,駕果下馬一,小史十四人。次屬車十二乘,駕牛,駕士各八人。次門下、史書、秘書、殿中四省局官各一人,騎,分左右夾屬車,各五人從,唯符寶以十二人從。次黃鉞車,上建黃鉞,駕二馬,左武衛隊正一人在車,駕士十二人。次豹尾車,駕二馬,左武衛隊正一人在車,駕士十二人。次左右威衛折沖都尉各一人,各領掩後二百人步從,五十人為行,大戟五十人,刀、楯、五十人,弓箭五十人,弩五十人,皆黑鍪、甲、覆膊、臂韝,橫行。



 次左右領軍衛將軍二人,領步甲隊及殳仗,各二人執槊槊從。次前後左右廂步甲隊。次左右廂黃麾仗。次左右廂殳仗。



 次諸衛馬隊,左右廂各二十四。自十二旗後,屬於玄武隊,前後有主帥以下四十人,皆戎服大袍,二人引旗,一人執,二人夾,二十人執槊,餘佩弩、弓箭。第一闢邪旗,左右金吾衛折沖都尉各一人主之,皆戎服大袍,佩弓箭、橫刀,騎;第二應龍旗,第三玉馬旗,第四三角獸旗,左右領軍衛果毅都尉各一人主之;第五黃龍負圖旗,第六黃鹿旗,左右威衛折沖都尉各一人主之;第七飛麟旗,第八駃騠旗,第九鸞旗,左右武衛果毅都尉各一人主之;第十鳳旗,第十一飛黃旗,左右驍衛折沖都尉各一人主之;第十二麟旗,第十三角端旗,以當御,第十四赤熊旗,左右衛折沖都尉各一人主之;第十五兕旗,第十六太平旗,左右驍衛果毅都尉各一人主之;第十七犀牛旗,第十八鵕鸃旗,第十九騼蜀旗,左右武衛折沖都尉各一人主之;第二十騶牙旗,第二十一蒼烏旗,左右威衛果毅都尉各一人主之;第二十二白狼旗,第二十三龍馬旗,第二十四金牛旗,左右領軍衛折沖都尉各一人主之;其服皆如第一。



 次玄武隊。次衙門一,居玄武隊前、大戟隊後,執者二人,夾四人,皆騎,分左右,赤綦襖,黃袍,黃冒。次衙門左右廂,廂有五門,執、夾人同上。第一門,居左右威衛黑質步甲隊之後,白質步甲隊之前;第二門,居左右衛步甲隊之後,左右領軍衛黃麾仗之前;第三門,居左右武衛黃麾仗之後,左右驍衛黃麾仗之前;第四門,居左右領軍衛黃麾仗之後,左右衛步甲隊之前;第五門,居左右武衛白質步甲隊之後,黑質步甲隊之前。五門別當步甲隊黃麾仗前、馬隊後,各六人分左右,戎服大袍,帶弓箭、橫刀。



 凡衙門,皆監門校尉六人,分左右,執銀裝長刀,騎。左右監門衛大將軍、將軍、中郎將,廂各巡行。校尉一人,往來檢校諸門。中郎將各一人騎從。左右金吾衛將軍循仗檢校,各二人執槊騎從。左右金吾衛果毅都尉二人,糾察仗內不法,各一人騎從。



 駕所至,路南向,將軍降,立於路右,侍中前奏「請降路」。天子降,乘輿而入,繖、扇、華蓋,侍衛。



 駕還,一刻,擊一鼓為一嚴,仗衛還於塗。三刻,擊二鼓為再嚴,將士布隊仗,侍中奏「請中嚴」。五刻,擊三鼓為三嚴,黃門侍郎奏「請駕發」。鼓傳音,發駕,鼓吹振作。入門,太樂令命擊蕤賓之鐘,左五鐘皆應。鼓柷,奏《採茨》之樂。至太極門,戛敔,樂止。既入,鼓柷,奏《太和》之樂。回路南向,侍中請降路,乘輿乃入,繖、扇、侍御、警蹕如初。至門,戛吾又,樂止。皇帝入,侍中版奏「請解嚴」。叩鉦,將士皆休。



\end{pinyinscope}