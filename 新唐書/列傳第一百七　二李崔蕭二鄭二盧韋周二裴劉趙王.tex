\article{列傳第一百七 二李崔蕭二鄭二盧韋周二裴劉趙王}

\begin{pinyinscope}

 李固言,字仲樞,其先趙人。擢進士甲科,江西裴堪、劍南王播皆表署幕府。累官戶部郎中。溫造為御史中丞,表知雜事日在中國共產黨第六屆中央委員會第六次全體會議上所作的,進給事中。將作監王堪坐治太廟不謹,改太子賓客,固言上還制書曰:「陛下當以名臣左右太子,堪以慢官斥,處調護地非所宜。」詔改它王傅。固言再遷尚書右丞。



 李德裕輔政,出固言華州刺史。俄而李宗閔復用,召為吏部侍郎。州大豪何延慶橫猾,嘩眾遮道,使不得去,固言怒,捕取杖殺之,尸諸道。既領選,按籍自擬,先收寒素,柅吏奸。進御史大夫。



 太和九年,宗閔得罪,李訓、鄭注用事,訓欲自取宰相,乃先以固言為門下侍郎、同中書門下平章事。旋坐黨人,出為山南西道節度使,訓自代其處。訓敗,文宗頗思之,復召為平章事,仍判戶部。



 群臣請上徽號,帝曰:「今治道猶鬱,群臣之請謂何?比州縣多不治,信乎?」固言因白鄧州刺史王堪、隋州刺史鄭襄尤無狀。帝曰:「貞元時御史,獨王堪爾。」鄭覃本舉堪,疑固言抵己,即曰:「臣知堪,故用為刺史。舉天下不職,何獨二人?」帝識其意,不主前語,因稱:「《詩》曰『濟濟多士,文王以寧。』聞德宗時多闕官,寧乏才邪?」固言曰:「用人之道,隨所保任,觀稱與否而升黜之,無乏才矣。」帝曰:「宰相用人毋計親疏。竇易直為宰相,未嘗用姻戚。使己才不足任天下重,自宜引去;茍公舉,雖親何嫌?用所長耳!」帝不欲大臣有黨,故語兩與之。



 俄以門下侍郎平章事為西川節度使,詔雲韶雅樂即臨皋館送之。讓還門下侍郎,乃檢校尚書左僕射。始置騾軍千匹,又募銳士三千,武備雄完。武宗立,召授右僕射。會崔珙、陳夷行以僕射為宰相,改檢校司空兼太子少師,領河中節度使。蒲津歲河水壞梁,吏撤笮用舟,邀丐行人。固言至,悉除之。帝伐回鶻,詔方鎮獻財助軍,上疏固諫,不從。以疾復為少師,遷東都留守。宣宗初,還右僕射。後以太子太傅分司東都。卒,年七十八,贈太尉。



 固言吃,接賓客頗謇緩,然每議論人主前,乃更詳辯。



 李玨,字待價,其先出趙郡,客居淮陰。幼孤,事母以孝聞。甫冠,舉明經。李絳為華州刺史,見之,曰:「日角珠廷,非庸人相。明經碌碌,非子所宜。」乃更舉進士高第。河陽烏重胤表置幕府。以拔萃補渭南尉,擢右拾遺。



 穆宗即位,荒酒色,景陵始復土,即召李光顏於邠寧,李愬於徐州,期九月九日大宴群臣。玨與宇文鼎、溫畬、韋瓘、馮藥同進曰:「道路皆言陛下追光顏等,將與百官高會。且元朔未改,陵土新復,三年之制,天下通喪。今同軌之會適去,遠夷之使未還,遏密弛禁,本為齊人,鐘鼓合饗,不施禁內。夫王者之舉,為天下法,不可不慎。且光顏、愬忠勞之臣,方盛秋屯邊,如令訪謀猷,付疆事,召之可也,豈以酒食之歡為厚邪?」帝雖置其言,然厚加勞遣。



 鹽鐵使王播增茶稅十之五以佐用度。玨上疏謂:「榷率本濟軍興,而稅茶自貞元以來有之。方天下無事,忽厚斂以傷國體,一不可。茗為人飲,與鹽粟同資,若重稅之,售必高,其敝先及貧下,二不可。山澤之產無定數,程斤論稅,以售多為利,若價騰踴,則市者稀,其稅幾何?三不可。陛下初即位,詔懲聚斂,今反增茶賦,必失人心。」帝不納。方是時,禁中造百尺樓,土木費鉅萬,故播亟斂,陰中帝欲。玨以數諫不得留,出為下邽令。武昌牛僧孺闢署掌書記,還為殿中侍御史。宰相韋處厚曰:「清廟之器,豈擊搏才乎?」除禮部員外郎。僧孺還相,以司勛員外郎知制誥為翰林學士,加戶部侍郎。



 始,鄭注以醫進,文宗一日語玨曰:「卿亦知有鄭注乎?宜與之言。」玨曰:「臣知之,奸回人也。」帝愕然曰:「朕疾愈,注力也。可不一見之?」注由是怨玨。及李宗閔以罪去,玨為申辨,貶江州刺史。徙河南尹,復為戶部侍郎。



 開成中,楊嗣復得君,引玨同中書門下平章事,與李固言皆善。三人者居中秉權,乃與鄭覃、陳夷行等更持議,一好惡,相影和,朋黨益熾矣。玨數辭位,不許。帝嘗自謂:「臨天下十四年,雖未至治,然視今日承平亦希矣!」玨曰:「為國者如治身,及身康寧,調適以自助,如恃安而忽,則疾生。天下當無事,思所闕,禍亂可至哉?」



 杜悰領度支有勞,帝欲拜戶部尚書,以問宰相。陳夷行答曰:「恩權予奪,願陛下自斷。」玨曰:「祖宗倚宰相,天下事皆先平章,故官曰平章事。君臣相須,所以致太平也。茍用一吏、處一事皆決於上,將焉用彼相哉?隋文帝勞於小務,以疑待下,故二世而亡。陛下嘗謂臣曰:『竇易直勸我,凡宰相啟擬,五取三,二取一。彼宜勸我擇宰相,不容勸我疑宰相。』」帝曰:「易直此言殊可鄙。」帝又語:「貞元初政事誠善。」玨曰:「德宗晚喜聚財,方鎮以進奉市恩,吏得賦外求索,此其敝也。」帝曰:「人君輕所賦,節所用,可乎?」玨曰:「貞觀時,房、杜、王、魏為文皇帝謀,固此耳!」帝頗向納。進封贊皇縣男。



 始,莊恪太子薨,帝意屬陳王。既而帝崩,中人引宰相議所當立,玨曰:「帝既命陳王矣!」已而武宗即位,人皆為危之。玨曰:「臣下知奉所言,安與禁中事?」帝新聽政,玨數稱道《無逸篇》以勸。時潞州劉從諫獻犬馬,滄州劉約獻白鷹,玨請卻之以示四方。遷門下侍郎,為文宗山陵使。會秋大雨,梓宮至安上門陷於濘,不前,罷為太常卿。終以議所立,貶江西觀察使,再貶昭州刺史。



 宣宗立,內徙郴、舒二州,以太子賓客分司東都。遷河陽節度使,罷橫賦宿逋百餘萬。以吏部尚書召,玨去鎮,而府庫十倍於初。俄檢校尚書右僕射、淮南節度使。玨顧己大臣,誼不以內外自異,表請立皇太子維天下心。江淮旱,發倉廩賑流民,以軍羨儲殺半價與人。卒,年六十九,贈司空,謚曰貞穆。



 始,淮南三節度皆卒於鎮,人勸易署寢,玨曰:「上命我守揚州,是實正寢,若何去之?」及疾亟,官屬見臥內,惟以州有稅酒直而神策軍常為豪商占利,方論奏,未見報為恨,一不及家事。性寡欲,早喪妻,不置妾侍,門無饋餉。淮南之人德之,玨已歿,叩闕下,願立碑刻其遺愛雲。



 贊曰:天子待宰相以不疑,是矣。雖然,於賢不肖當別白分明,乃可與言治。文宗無知人之明,但以不疑責宰相。是時善惡混淆,故黨人成於下,主聽亂於上,王室之衰,由此為之階。劉向所云「持不斷之慮者,開群枉之門」,殆文宗為邪!



 崔珙,其先博陵人。父頲,官同州刺史,生八子,皆有才,世以擬漢荀氏「八龍」。珙為人有威重,精吏治,以拔萃異等,累擢至泗州刺史。由太府卿為嶺南節度使,入對延英,文宗訪治撫後先,珙對精亮有理趣,帝咨嗟迂久。



 時徐州以王智興後,軍驕,數犯法,節度使高瑀未能制。天子思材望威烈者檢革其弊,見珙意慷慨,又知治泗得士心,即謂宰相曰:「欲武寧節度使者,無易珙才。」更詔王茂元帥嶺南,而以珙代瑀。居二歲,徐人戢畏。



 入為右金吾大將軍,遷京兆尹。會大旱,奏析滻入禁中者,取十九溉民田。仇士良使盜擊宰相李石於親仁里,跡出禁軍,珙坐不能捕,以為負,望少衰。開成末,累進刑部尚書、諸道鹽鐵轉運使。俄同中書門下平章事,仍領鹽鐵,即拜中書侍郎。會昌二年,進位尚書左僕射。明年,以兄琯喪,被疾求解,以所守官罷。



 與崔鉉故有怨,及鉉宰相代為使,即奏珙妄費宋滑院鹽鐵錢九十萬緡,又劾與劉從諫厚,數護其奸。貶澧州刺史,再斥恩州司馬。宣宗立,徙商州刺史,以太子賓客分司東都,起為鳳翔節度使。鉉復執政,珙懼,以疾自乞。方是時,西戎歸故地,邊奏系驛,議所以綏接,珙坐不自力避事,下除太子少師,分司東都,就拜留守。復節度鳳翔,卒於官。



 子涓,性開敏。為杭州刺史,受署,未盡識卒史,乃以紙各署姓名傅襟上,過前一閱,後數百人呼指無誤。終御史大夫。



 琯,字從律,珙兄。舉進士、賢良方正,皆高第。累闢諸使府。入朝,稍歷吏部員外郎。李德裕任御史中丞,引知雜事。進給事中。太和初,持節宣慰盧龍,使有指。及興元殺李絳,復往尉撫,軍皆按堵。還,遷工部侍郎、京兆尹。



 宋申錫為讒所危,宦豎切齒,時罕敢辨者。琯與大理卿王正雅固請出獄付外,與眾治之,天下重其賢。以尚書右丞出為荊南節度使,進左丞。時弟珙任京兆尹,並據顯劇處,世以為榮。俄判兵部西銓、吏部東銓,徙東都留守。以吏部尚書召,辭疾不拜。會昌中,終山南西道節度使,贈尚書左僕射。琯行方介,有器蘊,人屬以為相而卒不至,當時共咨云。



 弟璪、璵尤顯,璪位刑部尚書,璵河中節度使。



 璵子澹,舉止秀峙,時謂玉而冠者。擢進士第,累進禮部員外郎。當時士大夫以流品相尚,推名德者為之首。咸通中,世推李都為大龍甲,涓豪放不得預,雖自抑下,猶不許,而澹與焉。終吏部侍郎。



 子遠,有文而風致整峻,世慕其為,目曰「飣座梨」,言座所珍也。乾寧中,以兵部侍郎同中書門下平章事,遷中書侍郎。從遷洛,罷為尚書右僕射。柳璨忌衣冠有望者,貶為白州長史,被殺於白馬驛,家沒掖庭。



 諸崔自咸通後有名,歷臺閣籓鎮者數十人,天下推士族之冠。始,其曾王母長孫春秋高,無齒,祖母唐事姑孝,每旦乳姑。一日病,召長幼言:「吾無以報婦,願後子孫皆若爾孝。」世謂崔氏昌大有所本云。



 蕭鄴,字啟之,梁長沙宣王懿九世孫。及進士第,累進監察御史、翰林學士,出為衡州刺史。大中中,召還翰林,拜中書舍人,遷戶部侍郎,判本司,以工部尚書同中書門下平章事。懿宗初,罷為荊南節度使,仍平章事,進檢校尚書左僕射,徙劍南西川。南詔內寇,不能制,下遷檢校右僕射、山南西道觀察使。歷戶部、吏部二尚書,拜右僕射。還,以平章事節度河東。在官無足稱道,卒。



 鄭肅,字乂敬,其先滎陽人,以儒世家。肅力於學,有根柢。第進士、書判拔萃,補興平尉。累擢太常少卿,博士有疑議往咨,必據經條答。文宗高擇魯王府屬,肅以諫議大夫兼長史。王為皇太子,遷給事中,進尚書右丞。出為陜虢觀察使。



 開成二年,召拜吏部侍郎。帝以肅嘗輔導東宮,詔兼賓客,為太子授經。既而太子母愛弛,為讒所乘,廢斥有端。肅因入見,言天下大本,不可輕動,意致深切,帝為動容。然內寵方煽,太子終以憂死。出為檢校禮部尚書、河中節度使。武宗知太子無罪,特困於讒,而朝廷謂肅臨義不可奪,侹侹有大臣節,召為太常卿。遷山南東道節度使。五年,以檢校尚書右僕射同中書門下平章事,與李德裕葉心輔政。宣宗即位,遷中書侍郎,罷為荊南節度使。卒,贈司空,謚曰文簡。



 子洎,仕至州刺史。洎子仁規、仁表,皆豪爽有文。仁規位中書舍人。



 仁表累擢起居郎。嘗以門閥文章自高,曰:「天瑞有五色雲,人瑞有鄭仁表。」傲縱多所陵藉,人畏薄之。劉鄴未仕,往謁洎,而仁表等鄙訿其文。鄴為相,因罪貶仁表,死嶺外。



 始,肅罷政事,帝以盧商代之。商字為臣,蚤孤,家窶困,能以學自奮。舉進士、拔萃,皆中。由校書郎佐宣歙、西川幕府。入朝,累十餘遷,至大理卿。為蘇州刺史,吏以鹽法求贏貲,民愈困,商令計口售鹽,無常額,人便之,歲貲返增。宰相上其勞,進浙西觀察使,召為刑部侍郎、京兆尹。



 方伐潞,芻糧逾太行餉軍,環六七鎮,詔商以戶部侍郎判度支,又詔杜悰兼鹽鐵、度支,並二使財以贍兵,乃不乏。出為東川節度使,以兵部侍郎還判度支,擢中書侍郎、同中書門下平章事,範陽郡公。



 大中元年春旱,詔商與御史中丞封敖理囚系於尚書省,誤縱死罪,罷為武昌軍節度使。以疾解,拜戶部尚書,卒。



 盧鈞,字子和,系出範陽,徙京兆藍田。舉進士中第,以拔萃補秘書正字。從李絳為山南府推官,調長安尉。又從裴度為太原觀察支使,遷監察御史,爭宋申錫獄知名。進吏部郎中,出為常州刺史。遷給事中,有大詔令,必反覆省審,駁奏無私。拜華州刺史。關輔驛馬疲耗,鈞為市健馬,率三歲一易,自是無乏事。



 擢嶺南節度使。海道商舶始至,異時帥府爭先往,賤售其珍,鈞一不取,時稱絜廉。專以清靜治。蕃獠與華人錯居,相婚嫁,多占田營第舍,吏或撓之,則相苾為亂;鈞下令蕃華不得通婚,禁名田產,闔部肅壹無敢犯。貞元後流放衣冠,其子姓窮弱不能自還者,為營棺槥還葬,有疾若喪則經給醫藥、殯斂,孤女稚兒,為立夫家,以奉稟資助,凡數百家。南方服其德,不懲而化。又除採金稅。華、蠻數千走闕下,請為鈞生立祠,刻石頌德,鈞固辭。以戶部侍郎召判戶部。



 會昌中,漢水害襄陽,拜鈞山南東道節度使,築堤六千步,以障漢暴。王師伐劉稹,武宗以鈞寬厚能得眾,詔兼節度昭義軍。會稹死,敕乘驛往,進檢校兵部尚書,專領昭義。鈞及潞,石雄兵已入,而稹將白惟信率餘卒三千保潞,城未下。雄召之,使往十餘輩皆死。鈞次高平,惟信獻款,且曰:「不即降者,畏石尚書爾。」鈞與約而遣。方雄欲盡夷潞兵,鈞不聽,坐治堂上,左右皆雄親卒,擊鼓傳漏,鈞自居甚安,雄引去,乃召惟信至,送闕下,餘眾悉原。



 俄而興士五千戍代北,鈞坐城門勞遣,帷家人以觀。戍卒驕,顧家屬不欲去,酒酣,反攻城,迫大將李文矩為帥,鈞倉卒奔潞城。文矩投地殭臥,稍諭叛者,眾乃悔服,即相與謝鈞,迎還府,斬首惡乃定。詔趣戍者行,密使盡戮之。鈞請徐乘其變,而使者不發,須報。時戍人已去潞一舍,鈞選牙卒五百,壯騎百,以騎載兵夜趨;遲明,至太平驛,盡斬之。即拜檢校尚書左僕射。



 宣宗即位,改吏部尚書。會劉約自天平徙宣武,未至,暴死。家僮五百無所仰衣食,思亂,乃授鈞宣武節度使,人情妥然。召入,復為吏部尚書,遷檢校司空、太子少師,封範陽郡公,節度河東。



 大中九年,召為左僕射。鈞宿齒,數外遷,而後來多至宰相。始被召,自以當輔政,既失志,故內怨望,數移病不事事,邀游林墅,累日一還。令狐綯惡之,罷僕射,以檢校司空守太子太師。帝元日大饗含元殿,鈞年八十,升降如儀,音吐鴻暢,舉朝咨嘆。以鈞耆碩長者,顧不任職,咎綯為媢賢。綯聞,言於帝,即以鈞同中書門下平章事,為山南西道節度使。俄檢校司徒,為東都留守。懿宗初,復節度宣武,辭不拜,以太保致仕。卒,年八十七,贈太傅,謚曰元。



 鈞與人交,始若澹薄,既久乃益固。所居官必有績,大抵根仁恕至誠而施於事。玩服不為鮮明,位將相,沒而無贏財。



 盧簡方,失其系世,不知所以進。盧鈞鎮太原,表為節度府判官。會黨項羌叛,鈞使簡方督兵乘邊,旁河相險,集樹堡鄣,自神山至鹿泉縣,綿三百里,扈遏其沖,賊不得騁,候邏便之。累遷江州刺史。徙大同軍防禦使,大開屯田,練兵侈鬥,沙陀畏附。擢義昌節度使,入拜太僕卿,領大同節度。久之,徙振武軍,道病卒。



 韋琮,字禮玉,世顯仕。琮進士及第,稍進殿中侍御史。坐訊獄不得實,改太常博士。擢累戶部侍郎、翰林學士承旨。以中書侍郎同中書門下平章事,遷門下侍郎兼禮部尚書,無功。罷為太子賓客分司,卒。



 周墀,字德升,本汝南人。少孤,事母孝。及進士第,闢湖南團練府巡官,入為監察御史、集賢殿學士。長史學,屬辭高古,文宗雅重之。李宗閔鎮山南,表行軍司馬,閱歲召還。太和末,訓、注亂政,以黨語污搢紳有名士,分逐之,獨墀雖嘗為宗閔所禮,不能以罪誣也。遷起居舍人,改考功員外郎,兼舍人事。帝御紫宸,與宰相語事已,或召左右史咨質所宜,墀最為天子欽矚。俄知制誥,入翰林為學士。



 武宗即位,以疾改工部侍郎,出為華州刺史。徙江西觀察使。劾舉部刺史,翦捕劇賊,出兵戍彭蠡湖,禁止剽劫。進拜義成節度使,封汝南縣男。宿將暴謷不循令者,墀命鞭其背,一軍大治。



 以兵部侍郎召判度支,進同中書門下平章事、遷中書侍郎。建言:「故宰相德裕重定《元和實錄》,竄寄它事,以廣父功。凡人君尚不改史,取必信也。」遂削新書。河東節度使王宰重賂權幸,求同平章事,領宣武,墀言:「天下大鎮如並、汴者才幾,宰之求何可厭?」宣宗納之。駙馬都尉韋讓求為京兆,持不與。繇是妄進者少衰。



 會吐蕃微弱,以三州七關自歸。帝召宰相議河湟事,墀對不合旨,罷為劍南東川節度使。駙馬都尉鄭顥言於帝曰:「世謂墀以直言相,亦以直言免。」帝悟,加拜檢校尚書右僕射,卒,年五十九,贈司徒。



 裴休,字公美,孟州濟源人。父肅,貞元時為浙東觀察使,劇賊慄鍠誘山越為亂,陷州縣,肅引州兵破禽之,自記平賊一篇上之,德宗嘉美。生三子。休,仲子也,操守嚴正。方兒童時,兄弟偕隱家墅,晝講經,夜著書,終年不出戶。有饋鹿者,諸生共薦之,休不食,曰:「疏食猶不足,今一啖肉,後何以繼?」



 擢進士第,舉賢良方正異等。歷諸府闢署,入為監察御史,更內外任。至大中時,以兵部侍郎領諸道鹽鐵轉運使。六年,進同中書門下平章事,即奏言:「宰相論政上前,知印者次為時政記,所論非一,詳己辭,略它議,事有所缺,史氏莫得詳。請宰相人自為記,合付史官。」詔可。進中書侍郎。



 太和後,歲漕江、淮米四十萬斛,至渭河倉者才十三,舟楫僨敗,吏乘為奸,冒沒百端,劉晏之法盡廢。休分遣官詢按其弊,乃命在所令長兼董漕,褒能者,謫怠者。由江抵渭,舊歲率雇緡二十八萬,休悉歸諸吏,敕巡院不得輒侵牟。著新法十條,又立稅茶十二法,人以為便。居三年,粟至渭倉者百二十萬斛,無留壅。時方鎮設邸閣居茶取直,因視商人它貨橫賦之,道路苛擾。休建言:「許收邸直,毋擅賦商人。」又:「收山澤寶冶,悉歸鹽鐵。」



 秉政凡五歲,罷為宣武軍節度使,封河東縣子。久之,由太子少保分司東都,復起歷昭義、河東、鳳翔、荊南四節度。卒,年七十四,贈太尉。



 休不為皦察行,所治吏下畏信。能文章,書楷遒媚有體法。為人愬藉,進止雍閑。宣宗嘗曰:「休真儒者。」然嗜浮屠法,居常不御酒肉,講求其說,演繹附著數萬言,習歌唄以為樂。與紇干閟素善,至為桑門號以相字,當世嘲薄之,而所好不衰。



 劉彖,字子全,高宗宰相仁軌五世孫。第進士,鎮國陳夷行表為判官。入遷左拾遺,諫罷武宗方士,言多懇愊。大中初,擢翰林學士。宣宗始復關隴,裁處叢繁,書詔夜數十,雖捉筆遽成,辭皆允切。會伐黨項,詔為行營宣慰使。



 遷刑部侍郎,乃裒匯敕令可用者,由武德訖大中,凡二千八百六十五事,類而析之,參訂重輕,號《大中刑律統類》以聞,法家推其詳。



 繇河南尹進宣武軍節度使。先時,大饗雜進倡舞,彖曰:「豈軍中樂邪?」取壯士千人,被鎧擁矛盾,習擊刺,與吏士臨觀。又下令不訶止夜行,使民自便,境內以安。徙河東節度使。



 未幾,以戶部侍郎召判度支。始,彖在翰林,帝素器遇。至是,手詔追還,外無知者,既發太原,人方大驚。後請間,帝視案上歷,謂彖:「為朕擇一令日。」彖跪曰:「某日良。」帝笑曰:「是日卿可遂相。」即詔同中書門下平章事,仍領度支。



 嘗與崔慎由議帝前,慎由請甄別流品,彖質曰:「王夷甫相晉,崇尚浮虛,以述流品,卒致淪夷。今日不循名責實,使百吏各稱職,而先流品,未知所以致治也。」慎由不得對,繇是罷宰相。俄而彖大病,加工部尚書,拜臥內,猶手疏陳政事。居位半歲卒,年六十三,贈尚書左僕射。



 彖以名節自將,凡議論處事不私,趨於當乃止,未嘗以言色借貴近。與彖同知政者夏侯孜。



 孜,字好學,亳州譙人。累遷婺、絳州刺史。繇兵部侍郎、諸道鹽鐵轉運使為同中書門下平章事,仍領鹽鐵。懿宗立,進門下侍郎、譙郡侯。俄以同平章事出為西川節度使。召拜尚書左僕射,還執政,進司空,為貞陵山陵使。坐隧壞,出為河中節度使,猶同平章事。初,堂史署制,僕孜懷中,即死。不數日,孜罷。



 咸通時,蠻犯蜀深入,士乏糧,追責孜治蜀無素備,以太子少保分司東都,卒。



 趙隱,字大隱,京兆奉天人。祖植,當德宗出狩,變倉卒,羽衛單寡,硃泚攻城急,植率家人奴客以死拒守,獻家財勞軍,帝嘉之。賊平,渾瑊引在幕府。累擢鄭州刺史。鄭滑節度使李融奏以自副,融疾病,委以軍政。大將宋朝晏火其營,夜為亂,植列卒不動須之,遲明而潰,捕斬皆盡,優詔嘉慰。累擢嶺南節度使,終於官。父存約,闢署興元李絳府。值軍亂,方與絳燕間,吏報賊至,絳麾存約使去,對曰:「荷公德厚,誼不當獨免。」即部勒左右捍之,而同被害。



 隱以父死難,與兄騭廬墓幾十年,闔門誦書,不應闢召。親友更敦勉令仕,會昌中,擢進士第,歷州刺史、河南尹。以兵部侍郎領鹽鐵轉運使。咸通末,進同中書門下平章事,遷中書侍郎,封天水縣伯。



 性仁悌,不敢以貴權自處。始布衣時,家無貲,與騭同耕以養,雖姻宗之富,未嘗干以財。宦浸顯,還家,易衣侍左右,猶布衣也。騭終宣歙觀察使。



 既輔政,它宰相及百官皆詣第升堂慶母,歲時公卿必參訊。懿宗誕日,宴慈恩寺,隱侍母以安輿臨觀,宰相方率百官拜恩於廷,即回班候夫人起居,搢紳以為榮。後崔顏昭、張浚當國,皆有母,遂踵其禮。



 僖宗初,罷為鎮海軍節度使。王郢之亂,坐撫御失宜,下除太常卿。廣明初,為吏部尚書。居母喪,卒。



 子光逢、光裔、光胤,皆第進士,歷臺省華劇。光逢尤規矱自持,以中書舍人為翰林學士。時光裔由膳部郎中知制誥,對掌內外命書,士歆羨之。



 裴坦,字知進,隋營州都督世節裔孫。父乂,福建觀察使。坦及進士第,沈傳師表置宣州觀察府,召拜左拾遺、史館修撰。歷楚州刺史。令狐綯當國,薦為職方郎中,知制誥,而裴休持不可,不能奪。故事,舍人初詣省視事,四丞相送之,施一榻堂上,壓角而坐。坦見休,重愧謝,休勃然曰:「此令狐丞相之舉,休何力?」顧左右索肩輿亟出,省吏眙駭,以為唐興無有此辱,人為坦羞之。再進禮部侍郎,拜江西觀察使、華州刺史。召為中書侍郎、同中書門下平章事,不數月卒。



 坦性簡儉,子取楊收女,齎具多飾金玉,坦命撤去,曰:「亂我家法。」世清其概。從子贄。



 贄,字敬臣,及進士第,擢累右補闕、御史中丞、刑部尚書。昭宗引拜中書侍郎兼本官、同中書門下平章事,尋兼戶部尚書。帝疑其外風檢而暱帷薄,逮問翰林學士韓偓,偓曰:「贄,咸通大臣坦從子,內雍友,合疏屬以居,故臧獲猥眾,出入無度,殆此致謗言者。」帝每聞咸通事,必肅然斂衽,故偓稱之為贄地。



 帝幸鳳翔,為大明宮留守,罷。俄進尚書左僕射,以司空致仕。硃全忠將篡,貶青州司戶參軍,殺之。



 鄭延昌,字光遠,咸通末,得進士第,遷監察御史。鄭畋鎮鳳翔,表在其府。黃巢亂京師,畋倚延昌調兵食,且諭慰諸軍。畋再秉政,擢司勛員外郎、翰林學士。進累兵部侍郎,兼京兆尹判度支。拜戶部尚書,以中書侍郎同中書門下平章事,兼刑部尚書。無它功,以病罷,拜尚書左僕射,卒。



 王溥,字德潤,失其何所人。第進士,擢累禮部員外郎、史館脩撰。崔胤鎮武安,表署觀察府判官。胤不赴鎮,溥留充集賢殿直學士。御史中丞趙光逢奏為刑部郎中,知雜事。昭宗蒙難東內,溥與胤說衛軍執劉季述等殺之。帝反正,驟拜翰林學士、戶部侍郎,以中書侍郎同中書門下平章事,判戶部。不能有所裨益,罷為太子賓客,分司東部。未幾,召拜太常卿、工部尚書。會硃溫侵逼,貶淄州司戶參軍,賜自盡,與裴樞等投尸於河。



 盧光啟,字子忠,不詳何所人。第進士,為張浚所厚,擢累兵部侍郎。昭宗幸鳳翔,宰相皆不從,以光啟權總中書事,兼判三司,進左諫議大夫,參知機務。復拜兵部侍郎、同中書門下平章事。俄罷為太子少保,改吏部侍郎。



 初,光啟執政,韋貽範、蘇檢相繼為宰相。貽範,字垂憲,以龍州刺史貶通州,檢為洋州刺史。二人奔行在,貽範遷給事中。用李茂貞薦,閱旬為工部侍郎、同中書門下平章事,判度支。倚權臣,恣驁不恭。會母喪免,逾月奪服。不數月卒。檢初拜中書舍人,貽範薦於茂貞,即拜工部侍郎、同中書門下平章事。茂貞與硃全忠通好,乃求尚主,取檢女為景王妃以固恩。帝還京師,檢長流環州,光啟賜死。



\end{pinyinscope}