\article{列傳第十四 屈實尉遲張秦唐段}

\begin{pinyinscope}

 屈突通,其先蓋昌黎徒何人,後家長安。仕隋為虎賁郎將。文帝命覆隴西牧簿,得隱馬二萬匹,帝怒州龍門(今山西河津)人。曾於河、汾之間授徒千人,時稱,收太僕卿慕容悉達、監牧官史千五百人,將悉殊死。通曰:「人命至重,死不復生。陛下以至仁育四海,豈容以畜產一日而戮千五百士?」帝叱之,通進頓首曰:「臣願身就戮,以延眾死。」帝寤,曰:「朕不明,乃至是。今當免悉達等,旌爾善言。」遂皆以減論。擢左武衛將軍。蒞官勁正,有犯法者,雖親無所回縱。其弟蓋為長安令,亦以方嚴顯。時為語曰:「寧食三鬥艾,不見屈突蓋;寧食三鬥蔥,不逢屈突通。」



 煬帝即位,遣持詔召漢王諒。先是,文帝與諒約,若璽書召,驗視敕字加點,又與玉麟符合,則就道。及是,書無驗,諒覺變,詰通,通占對無屈,竟得歸長安。大業中,與宇文述共破楊玄感,以功遷左驍衛大將軍。秦、隴盜起,授關內討捕大使。安定人劉迦論反,眾十餘萬據雕陰。通發關中兵擊之,次安定,初不與戰,軍中意其怯。通陽言旋師,而潛入上郡。賊未之覺,引而南,去通七十里舍,分兵徇地。通候其無備,夜簡精甲襲破之,斬迦論並首級萬餘,築京觀於上郡南山,虜老弱數萬口。後隋政益亂,盜賊多,士無鬥志,諸將多覆。通每向必持重,雖不大克,亦不敗負。帝南幸,使鎮長安。



 高祖起,代王遣通守河東,戰久不下,高祖留兵圍之,遂濟河,破其將桑顯和於飲馬泉。通大懼,乃留鷹揚郎將堯君素守蒲,將自武關趨藍田以援長安。至潼關,阻劉文靜兵不得進,相持月餘。通令顯和夜襲文靜,詰朝大戰,顯和縱兵破二壁,唯文靜一壁獨完,然數入壁,短兵接,文靜中流矢,軍垂敗,顯和以士疲,乃傳餐食,文靜因得分兵實二壁。會游軍數百騎自南山還,擊其背,三壁兵大呼,奮而出,顯和遂潰,盡得其眾。通勢蹙,或說之降,曰:「吾蒙國厚恩,事二主,安可逃難?獨有死報爾!」每自摩其頸曰:「要當為國家受人一刀!」其訓勉士卒必流涕,故力雖窮,而人尚為之感奮。帝遣其家僮往召,通趨斬之。俄聞京師平,家盡沒,乃留顯和保潼關,率兵將如洛。既行,而顯和來降。文靜遣竇琮、段志玄精騎追及於稠桑,通結陣拒之。琮縱其子壽往諭使降,通大呼曰:「昔與汝父子,今則仇也!」命左右射之,顯和呼其眾曰:「京師陷,諸君皆家關西,何為復東?」眾皆舍兵。通知不免,遂下馬東南向,再拜號哭曰:「臣力屈兵敗,不負陛下。」遂被禽,送長安。帝勞曰:「何相見晚邪?」泣曰:「通不能盡人臣之節,故至此,為本朝羞。」帝曰:「忠臣也!」釋之,授兵部尚書、蔣國公,為秦王行軍元帥長史。



 從平薛仁杲,時賊珍用山積,諸將爭得之,通獨無所取。帝聞,曰:「清以奉國,名定不虛。」特賚金銀六百兩、彩千段。判陜東道行臺左僕射,從討王世充。時通二子在洛,帝曰:「今以東略屬公,如二子何?」通曰:「臣老矣,不足當重任。然疇昔陛下釋俘累,加恩禮,以蒙更生,是時口與心誓,以死許國。今日之行,正當先驅,二兒死自其分,終不以私害義。」帝太息曰:「烈士徇節,吾今見之。」及竇建德來援賊,秦王分麾下半以屬通,俾與齊王圍洛。世充平,論功第一,拜陜東道大行臺右僕射,鎮東都。數歲,召為刑部尚書。自以不習文,固辭,改工部。建成之變,復檢校行臺僕射,馳鎮洛。貞觀初,行臺廢,為洛州都督,進左光祿大夫。卒,年七十二,贈尚書左僕射,謚曰忠。後詔配饗太宗廟廷。永徽中,贈司空。



 二子壽、詮,壽襲爵。太宗幸洛,思通忠節,故詮以少子拜果毅都尉,賜粟帛恤其家,終瀛州刺史。詮子仲翔,神龍中,復守瀛州。



 初,桂州都督李弘節亦以清慎顯。既歿,其家賣珠。太宗疑弘節實貪,欲追坐舉者。魏徵曰:「陛下過矣!且今號清白死不變者,屈突通、張道源。通二子來調,共一馬;道源子不能自存。審其清者不加恤,疑其濁者罪所舉,亦好善不篤矣。」帝曰:「朕未之思。」置不問。故通之清益顯云。



 尉遲敬德名恭,以字行,朔州善陽人。隋大業末,從軍高陽,積閱為朝散大夫。劉武周亂,以為偏將。與宋金剛南侵,得晉、澮等州,襲破永安王孝基,執獨孤懷恩等。武德二年,秦王戰柏壁,金剛敗奔突厥,敬德合餘眾守介休,王遣任城王道宗、宇文士及諭之,乃與尋相舉地降,引為右一府統軍,從擊王世充。



 會尋相叛,諸將疑敬德且亂,囚之。行臺左僕射屈突通、尚書殷開山曰:「敬德慓敢,今執之,猜貳已結,不即殺,後悔無及也。」王曰:「不然。敬德必叛,寧肯後尋相者邪?」釋之,引見臥內,曰:「丈夫以氣相許,小嫌不足置胸中,我終不以讒害良士。」因賜之金,曰:「必欲去,以為汝資。」是日獵榆窠,會世充自將兵數萬來戰,單雄信者,賊驍將也,騎直趨王,敬德躍馬大呼橫刺,雄信墜,乃翼王出,率兵還戰,大敗之,禽其將陳智略,獲排槊兵六千。王顧曰:「比眾人意公必叛,我獨保無它,何相報速邪?」賜金銀一篋。



 竇建德營板渚,王命李勣等為伏,親挾弓,令敬德執槊,略其壘,大呼致師。建德兵出,乃稍引卻,殺數十人,眾益進。伏發,大破之。時世充兄子琬使於建德,乘隋帝廄馬,鎧甲華整,出入軍中以誇眾。王望見,問「誰可取者?」敬德請與高甑生、梁建方三騎馳往,禽琬,引其馬以歸,賊不敢動。從討劉黑闥,賊以奇兵襲李勣,王勒兵掩其後,俄而賊眾四面合,敬德率壯士馳入賊,王乘陣亂乃得出。又破徐圓朗。以功授王府左二副護軍。



 隱太子嘗以書招之,贈金皿一車。辭曰:「敬德起幽賤,會天下喪亂,久陷逆地,秦王實生之,方以身徇恩。今於殿下無功,其敢當賜?若私許,則懷二心,徇利棄忠,殿下亦焉用之哉?」太子怒而止。敬德以聞。王曰:「公之心如山嶽然,雖積金至斗,豈能移之?然恐非自安計。」巢王果遣壯士刺之。敬德開門安臥,賊至,不敢入。因譖於高祖,將殺之,王固爭,得免。



 其後隱、巢計日急,敬德與長孫無忌入白曰:「大王不先決,社稷危矣!」王曰:「我惟同氣,所未忍。伺其發,而後以義討之,如何?」敬德曰:「人情畏死,眾以死奉王,此天授也。天與不取,反得其咎。大王即不聽,請從此亡,不能交手蒙戮。」無忌曰:「王不從敬德言,敬德亦非王有,今敗矣。」王曰:「寡人之謀,未可全棄,公更圖之。」敬德曰:「處事有疑非智,臨難不決非勇。王今自計如何?勇士八百人悉入宮控弦被甲矣,尚何辭?」後又與侯君集等懇熟勸進,計乃定。時房玄齡、杜如晦被斥在外,召不至。王怒曰:「是背我邪?」因解所佩刀反授之。謂曰:「即不從,可斬其首以來。」敬德遂往諭玄齡等,與入計議。



 隱太子死,敬德領騎七十趨玄武門,王馬逸,墜林下,元吉將奪弓窘王,敬德馳叱之,元吉走,遂射殺之。宮、府兵屯玄武門,戰不解,敬德持二首示之,乃去。時帝泛舟海池,王命敬德往侍,不解甲趨行在。帝驚曰:「今日之亂為誰?爾來何邪?」對曰:「秦王以太子、齊王作亂,舉兵誅之,恐陛下不安,遣臣宿衛。」帝意悅。於是南衙、北門兵與府兵尚雜斗,敬德請帝手詔諸軍聽秦王節度,內外始定。



 王為皇太子,授左衛率。時坐隱、巢者百餘家,將盡沒入之。敬德曰:「為惡者二人,今已誅,若又窮支黨,非取安之道。」由是普原。論功為第一,賜絹萬匹,舉齊府金幣、什器賜焉。除右武候大將軍,封吳國公,實封千三百戶。



 突厥入寇,授涇州道行軍總管。虜至涇陽,輕騎與戰,敗之。敬德所得財,必散之士卒。然婞直,頗以功自負,又廷質大臣得失,與宰相不平。出為襄州都督。累遷同州刺史。嘗侍宴慶善宮,有班其上者,敬德曰:「爾何功,坐我上?」任城王道宗解喻之,敬德勃然,擊道宗目幾眇。太宗不懌,罷,召讓曰:「朕觀漢史,嘗怪高祖時功臣少全者。今視卿所為,乃知韓、彭夷戮,非高祖過。國之大事,惟賞與罰,橫恩不可數得,勉自脩飭,悔可及乎!」敬德頓首謝。後改封鄂國,歷鄜、夏二州都督。老就第,授開府儀同三司,朝朔望。



 帝將討高麗,敬德上言:「乘輿至遼,太子次定州,兩京空虛,恐有玄感之變。夷貊小國,不足枉萬乘,願委之將臣,以時摧滅。」帝不納。詔以本官行太常卿,為左一馬軍總管。師還,復致仕。顯慶三年卒,年七十四。高宗詔京官五品以上及朝集使赴第臨吊,冊贈司徒、並州都督,謚曰忠武。給班劍、羽葆、鼓吹,陪葬昭陵。



 敬德晚節,謝賓客不與通。飭觀、沼,奏清商樂,自奉養甚厚。又餌雲母粉,為方士術延年。其戰,善避槊,每單騎入賊,雖群刺之不能傷,又能奪取賊槊還刺之。齊王元吉使去刃與之校,敬德請王加刃,而獨去之,卒不能中。帝嘗問:「奪槊與避槊孰難?」對曰:「奪槊難。」試使與齊王戲,少選,王三失槊,遂大愧服。



 張公謹,字弘慎,魏州繁水人。為王世充洧州長史,與刺史崔樞挈城歸天子,授檢校鄒州別駕,遷累右武候長史,未知名。李勣、尉遲敬德數啟秦王,乃引入府。王將討隱、巢亂,使卜人占之,公謹自外至,投龜於地曰:「凡卜以定猶豫,決嫌疑。今事無疑,何卜之為?卜而不吉,其可已乎?」王曰:「善。」隱太子死,其徒攻玄武門,銳甚,公謹獨閉關拒之。以功授左武候將軍,封定遠郡公,實封一千戶。



 貞觀初,為代州都督,置屯田以省饋運。數言時政得失,太宗多所採納。後副李靖經略突厥,條可取狀於帝曰:「頡利縱欲肆兇,誅害善良,暱近小人,此主昏於上,可取一也。別部同羅、僕骨、回紇、延陀之屬,皆自立君長,圖為反噬,此眾叛於下,可取二也。突利被疑,以輕騎免,拓設出討,眾敗無餘,欲谷喪師,無托足之地,此兵挫將敗,可取三也。北方霜旱,稟糧乏絕,可取四也。頡利疏突厥,親諸胡,胡性翻覆,大軍臨之,內必生變,可取五也。華人在北者甚眾,比聞屯聚,保據山險,王師之出,當有應者,可取六也。」帝然所謀。及破定襄,敗頡利,璽詔慰勞,進封鄒國公,改襄州都督,以惠政聞。卒官下,年四十九。帝將出次哭之,有司奏:「日在辰,不可。」帝曰:「君臣猶父子也,情感於內,安有所避。」遂哭之。詔贈左驍衛大將軍,謚曰襄。十三年,追改郯國公。永徽中,加贈荊州都督。



 子大素,龍朔中,歷東臺舍人,兼修國史,著書百餘篇,終懷州長史。次子大安,上元中,同中書門下三品。章懷太子令與劉訥言等共注範曄《漢書》。太子廢,故貶為普州刺史,終橫州司馬。子悱,仕玄宗時為集賢院判官,詔以其家所著《魏書》、《說林》入院,綴修所闕,累擢知圖書、括訪異書使,進國子司業,以累免官。



 秦瓊,字叔寶,以字顯,齊州歷城人。始為隋將來護兒帳內,母喪,護兒遣使襚吊之。吏怪曰:「士卒死喪,將軍未有所問,今獨吊叔寶何也?」護兒曰:「是子才而武,志節完整,豈久處卑賤邪?」



 俄從通守張須陀擊賊盧明月下邳,賊眾十餘萬,須陀所統才十之一,堅壁水敢進,糧盡,欲引去。須陀曰:「賊見兵卻,必悉眾追我,得銳士襲其營,且有利,誰為吾行者?」眾莫對。惟叔寶與羅士信奮行。乃分勁兵千人伏莽間,須陀委營遁,明月悉兵追躡。叔寶等馳叩賊營,門閉不得入,乃升樓拔賊旗幟,殺數十人,營中亂,即斬關納外兵,縱火焚三十餘屯。明月奔還,須陀回擊,大破之。又與孫宣雅戰海曲,先登。以前後功擢建節尉。



 從須陀擊李密滎陽。須陀死,率殘兵附裴仁基。仁基降密,密得叔寶大喜,以為帳內驃騎,待之甚厚。密與宇文化及戰黎陽,中矢墮馬,濱死,追兵至,獨叔寶捍衛得免。



 後歸王世充,署龍驤大將軍。與程咬金計曰:「世充多詐,數與下咒誓,乃巫嫗,非撥亂主也。」因約俱西走,策其馬謝世充曰:「自顧不能奉事,請從此辭。」賊不敢逼,於是來降。高祖俾事秦王府,王尤獎禮。從鎮長春宮,拜馬軍總管。戰美良川,破尉遲敬德,功多,帝賜以黃金瓶,勞曰:「卿不恤妻子而來歸我,且又立功,使朕肉可食,當割以啖爾,況子女玉帛乎!」尋授秦王右三統軍,走宋金剛於介休,拜上柱國。從討世充、建德、黑闥三盜,未嘗不身先鋒鏖陣,前無堅對。積賜金帛以千萬計,進封翼國公。每敵有驍將銳士震耀出入以誇眾者,秦王輒命叔寶往取之,躍馬挺槍刺於萬眾中,莫不如志,以是頗自負。及平隱、巢,功拜左武衛大將軍,實封七百戶。



 後稍移疾,嘗曰:「吾少長戎馬間,歷二百餘戰,數重創,出血且數斛,安得不病乎?」卒,贈徐州都督,陪葬昭陵。太宗詔有司琢石為人馬立墓前,以旌戰功。貞觀十三年,改封胡國公。



 後四年,詔司徒、趙國公無忌,司空、河間王孝恭,司空、萊國公如晦,司空、太子太師、鄭國公徵,司空、梁國公玄齡,開府儀同三司、鄂國公敬德,特進、衛國公靖,特進、宋國公瑀,輔國大將軍、褒國公志玄,輔國大將軍、夔國公弘基,尚書左僕射、蔣國公通,陜東道行臺右僕射、鄖國公開山,荊州都督、譙國公紹,荊州都督、邳國公順德,洛州都督、鄖國公亮,吏部尚書、陳國公君集,左驍衛大將玖惲郯國公公謹,左領嵕唵蠼鵩⒙鎷𢥞塚癲可惺欏⒂佬絲す濫希Р可惺欏⒂騫幔Р可惺欏④旃螅奰可惺欏⒂⒐珓蓿⑹灞Γ⑼夾瘟柩談蟆8咦謨闌樟輳彩怪錄爛紀夾瘟柩談笳叻財呷耍紜⑹苛r、志玄、弘基、世南、叔寶,皆始終著名者也。



 唐儉,字茂約,並州晉陽人。祖邕,北齊尚書左僕射。父鑒,隋戎州刺史;與高祖善,嘗偕典軍衛,故儉雅與秦王游,同在太原。儉爽邁少繩檢,然事親以孝聞。見隋政浸亂,陰說秦王建大計。高祖嘗召訪之,儉曰:「公日角龍庭,姓協圖讖,系天下望久矣。若外嘯豪傑,北招戎狄,右收燕、趙,濟河而南,以據秦、雍,湯、武之業也。」高祖曰:「湯、武之事豈可幾?然喪亂方剡,私當圖存,公欲拯溺者,吾方為公思之。」及大將軍府開,授記室參軍、渭北道元帥司馬。從定京師,為相國府記室,晉昌郡公。



 武德初,進內史舍人,遷中書侍郎、散騎常侍。呂崇茂以夏縣反,與劉武周連和,詔永安王孝基、獨孤懷恩,於筠率兵致討,儉以使適至軍。會孝基等為武周所虜,儉亦見禽。始,懷恩屯蒲州,陰與部將元君實謀反,會俱在賊中,君實私語儉曰:「獨孤尚書將舉兵圖大事,猶豫不發,故及此。所謂當斷不斷而受亂者。」俄而懷恩脫歸,詔復守蒲。君實曰:「獨孤拔難歸,再戍河上,寧其王者不死乎?」儉恐必亂,密遣劉世讓歸白發其謀。會高祖幸蒲津,舟及中流而世讓至,帝驚,曰:「豈非天也!」命趨還舟,捕反者,懷恩自殺,餘黨皆誅。俄而武周敗,亡入突厥。儉封府庫、籍兵甲以待秦王。帝嘉儉身幽辱而不忘朝廷,詔復舊官,仍為並州道安撫大使,許以便宜。盡簿懷恩貲產賜儉。還為禮部尚書、天策府長史、檢校黃門侍郎、莒國公。仍為遂州都督,食綿州六百戶。



 貞觀初,使突厥還,太宗謂儉曰:「卿觀頡利可取乎?」對曰:「銜國威靈,庶有成功。」四年,馳傳往誘使歸款,頡利許之,兵懈弛,李靖因襲破之,儉脫身還。



 歲餘,為民部尚書。從獵洛陽苑,群豕突出於林,帝射四發,輒殪四豕。一豕躍及鐙,儉投馬搏之。帝拔劍斷豕,顧笑曰:「天策長史不見上將擊賊邪,何懼之甚?」對曰:「漢祖以馬上得之,不以馬上治之。陛下神武定四方,豈復快心於一獸?」帝為罷獵。詔其子善識尚豫章公主。



 悸事,與賓客縱酒為樂。坐小法,貶光祿大夫。永徽初,致仕,加特進。顯慶初卒,年七十八。贈開府儀同三司、並州都督,陪葬昭陵,謚曰襄。少子觀,為河西令,知名。孫從心,神龍中,以其子脧娶太平公主女,擢累殿中監。脧太常少卿,坐太平黨誅。



 儉弟憲。憲字茂彞,仕隋為東宮左勛衛。太子廢,罷歸。不治細行,好馳獵,藏亡命,所交皆博徒輕俠。高祖領太原,頗親遇之,參與大議。義師起,授正議大夫,置左右,尤所信倚。封安富縣公。武德中,進累雲麾將軍,加郡公。貞觀中,終金紫光祿大夫。



 裔孫次,字文編。建中初,及進士第,歷侍御史。竇參數薦之,改禮部員外郎。參貶,出為開州刺史,積十年不遷。韋皋鎮蜀,表為副使,德宗諭皋罷之。次身在遠,久抑不得申,以為古忠臣賢士罹讒毀被放,至殺身,君且不悟者,因採獲其事,為《辯謗略》三篇上之。帝益怒,曰:「是乃以古昏主方我!」改夔州刺史。憲宗立,召還,授禮部郎中,知制誥,終中書舍人。憲宗雅惡朋比傾陷者,嘗覽《辯謗略》,善之。謂學士沈傳師曰:「凡君人者,宜所觀省。然次編錄未盡,卿可廣其書。」傳師乃與令狐楚、杜元穎論次,起周訖隋,增為十篇,更號《元和辨謗略》。



 子扶,字雲翔,仕歷屯田郎中。大和五年,為山南宣撫使。內鄉倉督鄧琬負度支漕米七千斛,吏責償之,系其父子至孫凡二十八年,九人死於獄,扶奏申釋之。詔切責鹽鐵、度支二使,天下監院償逋系三年以上者,皆原。進中書舍人,出為福州觀察使。濫殺人,風績不立。會卒,奴婢爭財,有司按其貲至十餘萬,時議蚩薄之。



 扶弟持,字德守,中進士第。大和中,為渭南尉,試京兆府進士。時尹杜悰欲以親故托之,持輒趨降階伏,悰語塞,乃止。累遷工部郎中,出為容州刺史。遷給事中,歷朔方、昭義節度使,卒。



 子彥謙字茂業,多通技藝,尤工為詩,負才無所屈。乾符末,避亂漢南。王重榮鎮河中,闢幕府,累表為副,歷晉、絳二州刺史。重榮軍亂、彥謙貶興元參軍事。節度使楊守亮表為判官,遷副使,終閬、壁二州刺史。



 段志玄,齊州臨淄人。父偃師,仕隋為太原司法書佐。從義師,官至郢州刺史。志玄姿質偉岸,少無賴,數犯法。大業末,從父客太原,以票果,諸惡少年畏之,為秦王所識。高祖興,以千人從,授右領大都督府軍頭。下霍邑、絳郡,攻永豐倉,椎鋒最。歷左光祿大夫。從劉文靜拒屈突通於潼關。文靜為桑顯和所襲,軍且潰,志玄率壯騎馳賊,殺十餘人,中流矢,忍不言,突擊自如,賊眾亂,軍乘之,唐兵復振。通敗走,與諸將躡獲於稠桑,以多,授樂游府車騎將軍。從討王世充,深入,馬跌,為賊禽。兩騎夾持其髻,將度洛,志玄忽騰而上,二人者俱墮,於是奪其馬馳歸,尾騎數百不敢近。破竇建德,平東都,遷秦王府右二護軍。隱太子嘗以金帛誘之,拒不納。秦王即位,累遷左驍衛大將軍,封樊國公,實封九百戶。詔率兵至青海奪吐谷渾牧馬,逗留,免。未幾復職。文德皇后之葬,與宇文士及勒兵衛章武門,太宗夜遣使至二將軍所,士及披戶內使,志玄拒曰:「軍門不夜開。」使者示手詔,志玄曰:「夜不能辨。」不納。比曙,帝嘆曰:「真將軍,周亞夫何以加!」改封褒國公,歷鎮軍大將軍。貞觀十六年疾,帝臨視,泣顧曰:「當與卿子五品官。」頓首謝,請與母弟,乃拜志感左衛郎將。及卒,帝哭之慟。贈輔國大將軍、揚州都督,陪葬昭陵,謚曰壯肅。三世孫文昌。



 文昌,字墨卿,一字景初,世客荊州。疏爽任義節,不為齷齪小行。節度使裴胄禮之。胄採古今禮要為書,數從文昌質判所疑。後依劍南節度韋皋,皋表為校書郎。宰相李吉甫才之,擢登封尉、集賢校理,再遷左補闕。憲宗數欲親用,頗為韋貫之奇詆,偃蹇不得進。貫之罷,引為翰林學士,遷中書舍人,遂為承旨。穆宗即位,屢召入思政殿顧問,率至夕乃出。俄拜中書侍郎、同中書門下平章事。未逾年,自表還政。授劍南西川節度使、同平章事。文昌素諳蜀利病,大抵治寬靜,間以威斷,不常任也,群蠻震服。長慶二年黔中蠻叛,觀察使崔元略以聞,文昌使一介開曉,蠻即引還,彭濮蠻大酋蹉祿來請立石刊誓,脩貢獻。入遷兵部尚書。文宗立,拜御史大夫,進封鄒平郡公。俄檢校尚書右僕射、平章事,節度淮南。太和四年,檢校左僕射,徙帥荊南。州或旱,禬解必雨;或久雨,遇出游必霽。民為語曰:「旱不苦,禱而雨;雨不愁,公出游。」南詔襲南安,帝以文昌得蠻夷心,詔使下檄尉讓,即日解而去。復節度西川。九年卒,贈太尉。文昌先墓在荊州,歲時享祠,必薦以音樂歌舞,習禮者譏其非,少羈窶,所向少諧。及居將相,享用奢侈,士議尤替。



 子成式,字柯古,推廕為校書郎。博學強記,多奇篇秘籍。侍父於蜀,以畋獵自放,文昌遣吏自其意諫止。明日以雉兔遍遺幕府,人為書,因所獲儷前世事,無復用者,眾大驚。擢累尚書郎,為吉州刺史,終太常少卿。著《酉陽書》數十篇。子安節,乾寧中,為國子司業。善樂律,能自度曲云。



 贊曰:屈突通盡節於隋,而為唐忠臣,何哉?惟其一心,故事兩君而無嫌也。敬德之來,太宗以赤心付之,桑廕不徙而大功立。君臣相遇,古人謂之千載,顧不諒哉!投幾之會,間不容穟,公謹所以抵龜而決也。



\end{pinyinscope}