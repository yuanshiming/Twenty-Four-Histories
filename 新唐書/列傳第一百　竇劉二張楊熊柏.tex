\article{列傳第一百 竇劉二張楊熊柏}

\begin{pinyinscope}

 竇群,字丹列,京兆金城人。父叔向,以詩自名,代宗時性,思維是精神實體(心靈)的屬性。斯賓諾莎認為思維和,位左拾遺。群兄弟皆擢進士第,獨群以處士客隱毘陵。母卒,嚙一指置棺中,廬墓次,終喪。從盧庇傳啖助《春秋》學,著書數十篇。蘇州刺史韋夏卿薦之朝,並表其書,報聞,不召。後夏卿入為京兆尹,復言之德宗,擢為左拾遺。時張薦持節使吐蕃,乃遷群侍御史,為薦判官,入見帝曰:「陛下即位二十年,始自草茅擢臣為拾遺,何其難也?以二十年難進之臣為和蕃判官,一何易?」帝壯其言,不遣。



 王叔文黨盛,雅不喜群,群亦心幸心幸不肯附。欲逐之,韋執誼不可,乃止。群往見叔文曰:「事有不可知者。」叔文曰:「奈何?」曰:「去年李實伐恩恃權,震赫中外,君此時逡巡路傍,江南一吏耳。今君又處實之勢,豈不思路傍復有如君者乎?」叔文悚然,亦卒不用。



 憲宗立,轉膳部員外郎,兼侍御史知雜事。出為唐州刺史。節度使于頔聞其名,與語,奇之,表以自副。武元衡、李吉甫皆所厚善,故召拜吏部郎中。元衡輔政,薦群代為中丞。群引呂溫、羊士諤為御史,吉甫以二人躁險,持不下。群忮狠,反怨吉甫。吉甫節度淮南,群謂失恩,因擠之。陳登者,善術,夜過吉甫家,群即捕登掠考,上言吉甫陰事。憲宗面覆登,得其情,大怒,將誅群,吉甫為救解,乃免,出為湖南觀察使。改黔中。會水壞城郛,調溪洞群蠻築作,因是群蠻亂,貶開州刺史。稍遷容管經略使。召還,卒於行,年五十五,贈左散騎常侍。



 群狠自用,果於復怨。始召,將大任之,眾皆懼,及聞其死,乃安。



 兄常、牟,弟庠、鞏,皆為郎,工詞章,為《聯珠集》行於時,義取昆弟若五星然。



 常,字中行,大歷中及進士第,不肯調,客廣陵,多所論著,隱居二十年。鎮州王武俊聞其才,奏闢不應。杜佑鎮淮南,署為參謀。歷朗夔江撫四州刺史、國子祭酒,致仕。卒,贈越州都督。



 牟,字貽周,累佐節度府。晚從昭義盧從史,從史浸驕,牟度不可諫,即移疾歸東都。從史敗,不以覺微避去自賢。位國子司業。



 庠,字胄卿,終婺州刺史。



 鞏,字友封,雅裕,有名於時。平居與人言若不出口,世號「囁嚅翁」。元稹節度武昌,奏鞏自副,卒。



 劉棲楚,其出寒鄙。為鎮州小吏,王承宗奇之,薦於李逢吉,繇鄧州司倉參軍擢右拾遺。逢吉之罷裴度、逐李紳,皆嗾而為奸者。敬宗立,視朝常晏,數游畋失德。棲楚諫曰:「惟前世王者初嗣位,皆親庶政,坐以待旦。陛下新即位,安臥寢內,日晏乃作。大行殯宮密邇,鼓吹之聲日聞諸朝。且憲宗及先帝皆長君,朝夕恪勤,四方猶有叛者。陛下以少主,踐祚未幾,惡德流布,恐福祚之不長也。臣以諫為官,使陛下負天下譏,請碎首以謝。」遂額叩龍墀,血被面。李逢吉傳詔:「毋叩頭,待詔旨。」棲楚捧首立,帝動容,揚袂使去。棲楚曰:「不聽臣言,臣請死於此。」有詔尉諭,乃出。遷起居郎,辭疾歸洛。後諫官對延英,帝問:「向廷爭者在邪?」以諫議大夫召。未幾,宣授刑部侍郎。故事,侍郎無宣授者,逢吉喜助己,故不次任之。



 數月,改京兆尹,峻誅罰,不避權豪。先是,諸惡少竄名北軍,凌藉衣冠,有罪則逃軍中,無敢捕。棲楚一切窮治,不閱旬,宿奸老蠹為斂跡。一日,軍士乘醉有所凌突,諸少年從旁噪曰:「癡男子,不記頭上尹邪?」



 然其性詭激,敢為怪行,乘險抵戲,若無顧藉,內實恃權怙寵以干進。詣宰相,厲色慢辭,韋處厚惡之,出為桂管觀察使。卒,贈左散騎常侍。



 張又新,字孔昭,工部侍郎薦之子。元和中,及進士高第,歷左右補闕。性傾邪。李逢吉用事,惡李紳,冀得其罪,求中朝兇果敢言者厚之,以危中紳。又新與拾遺李續、劉棲楚等為逢吉搏吠所憎,故有「八關十六子」之目。



 敬宗立,紳貶端州司馬,朝臣過宰相賀,閽者曰:「止,宰相方與補闕語,姑伺之。」及又新出,流汗揖百官曰:「端溪之事,竊不敢讓。」人皆闢易畏之。尋轉祠部員外郎。嘗買婢遷約,為牙儈搜索陵突,御史劾舉,逢吉庇之,事不窮治。及逢吉罷,領山南東道節度,表又新為行軍司馬。坐田伾事,貶汀州刺史。李訓有寵,又新復見用,遷刑部郎中,為申州刺史。訓死,復坐貶。終左司郎中。又新善文辭,再以諂附敗,喪其家聲云。



 楊虞卿,字師皋,虢州弘農人。父寧,有高操,談辯可喜。擢明經,調臨渙主簿,棄官還夏,與陽城為莫逆交。德宗以諫議大夫召城,城未拜,詔寧即諭,與俱來。陜虢觀察使李齊運表置幕府。齊運入為京兆尹,表奉先主簿,拜監察御史,坐累免。順宗初,召為殿中侍御史,終國子祭酒。



 虞卿第進士、博學宏辭,為校書郎。抵淮南,委婚幣焉,會陳商葬其先,貧不振,虞卿未嘗與游,悉所齎助之。擢累監察御史。



 穆宗初立,逸游荒恣,虞卿上疏曰:「烏鳶遭害仁鳥逝,誹謗不誅良臣進。臣敢冒誅獻瞽言。臣聞堯、舜以天下為憂,不以位為樂。況今北虜方梗,西戎弗靖,兩河有瘡痏之虞,五嶺罹氛厲之役。人之疾苦積下,朝之制度莫脩。邊亡見儲,國用浸屈,固未可以高枕而息也。陛下初臨萬幾,宜有憂天下心。當日見輔臣公卿百執事,垂意以問,使四方內外灼有所聞。而聽政六十日,入對延英,獨三數大臣承聖問而已,它內朝臣偕入齊出,無所咨詢。諫臣盈廷,忠言不聞,臣實羞之。蓋主恩疏而正路塞也。公卿大臣宜朝夕燕見,則君臣情接而治道得矣。今宰臣四五人,或頃刻侍坐,鞠躬隕越,隨旨上下,無能往來,此繇君太尊、臣太卑故也。公卿列位,雖陟降清地,曾未奉優眷、承下問。雖陛下神聖如五帝,猶宜周爰顧逮,惠以氣色,使支體相成,君臣昭明。陛下求治於宰相,宰相求治於臣等,進忠若趨利,論政若訴冤,此而不治,無有也。自古天子居危思安之心同,而居安慮危之心則異,故不得皆為聖明也。」時又有衡山布衣趙知微,亦上書指言帝倡優在側,馳騁無度,內作色荒,外作禽荒。辭頗危切,帝詔宰相尉謝。宰相因是賀天子納諫,然不能用也。俄詔行勞西北邊。還,遷侍御史,改禮部員外郎、史館脩撰。進吏部。會曹史李賨等鬻偽告,調官六十五員,贓千六百萬以上,虞卿發其奸,賨等系御史府。而虞卿親吏嘗受二百萬,亡命,私奴受三十萬,虞卿縛奴送獄。三司嚴休復、高釴、韋景休雜推,賨等皆誅死。虞卿坐不檢下免官。



 李宗閔、牛僧孺輔政,引為右司郎中、弘文館學士。再遷給事中。虞卿佞柔,善諧麗權幸,倚為奸利。歲舉選者,皆走門下,署第注員,無不得所欲,升沈在牙頰間。當時有蘇景胤、張元夫,而虞卿兄弟汝士、漢公為人所奔向,故語曰:「欲趨舉場,問蘇、張;蘇、張猶可,三楊殺我。」宗閔待之尤厚,就黨中為最能唱和者,以口語軒輊事機,故時號「黨魁」。



 德裕之相,出為常州刺史。宗閔復入,以工部侍郎召,遷京兆尹。太和九年,京師訛言鄭注為帝治丹,剔小兒肝心用之。民相驚,扃護兒曹。帝不悅,注亦內不安,而雅與虞卿有怨,即約李訓奏言:「語出虞卿家,因京兆騶伍布都下。」御史大夫李固言素嫉虞卿周比,因傅左端倪。帝大怒,下虞卿詔獄。於是諸子弟自囚闕下稱冤,虞卿得釋,貶虔州司戶參軍,死。



 子知退、知權、擅、堪、漢公,皆擢進士第,漢公最顯。



 漢公,字用乂。始闢興元李絳幕府,絳死,不與其禍。遷累戶部郎中、史館修撰,轉司封郎中。坐虞卿,下除舒州刺史,徙湖、亳、蘇三州。擢桂管、浙東觀察使。繇戶部侍郎拜荊南節度使,召為工部尚書。或劾漢公治荊南有貪贓,降秘書監。稍遷國子祭酒。



 宣宗擢為同州刺史。於是,給事中鄭裔綽、鄭公輿共奏漢公冒猥無廉概,不可處近輔,三還制書。帝它日凡門下論執駁正未嘗卻。漢公素結左右,有奧助。至是,帝惑不從,制卒行。會寒食宴近臣,帝自擊球為樂,巡勞從臣,見裔綽等曰:「省中議無不從,唯漢公事為有黨。」裔綽獨對:「同州,太宗興王地。陛下為人子孫當精擇守長付之,漢公既以墨敗,陛下容可舉劇部私貪人?」帝恚見顏間。翌日,斥裔綽為商州刺史。漢公自同州更宣武、天平兩節度使,卒。子籌、範,仕亦顯。



 汝士,字慕巢。中進士第,又擢宏辭。牛李待之善,引為中書舍人。開成初,繇兵部侍郎為東川節度使。時嗣復鎮西川,乃族昆弟,對擁旄節,世榮其門。終刑部尚書。



 子知溫、知至,悉以進士第入官。知溫終荊南節使。知至為宰相劉瞻所善,以比部郎中知制誥。瞻得罪,亦貶瓊州司馬,擢累戶部侍郎。



 楊氏自汝士後,貴赫為冠族。所居靜恭里,兄弟並列門戟。咸通後,在臺省方鎮率十餘人。



 張宿者,本寒人,自名諸生。憲宗為廣陵王時,因張茂宗薦尉,得出入邸中,誕譎敢言。及監撫,自布衣授左拾遺,交通權幸,四方賂遺滿門。數召對,不能慎密,坐漏禁中語,貶郴丞十餘年。



 累遷比部員外郎。宰相李逢吉數言其狡譎不可信,白為濠州刺史,宿上疏自言,留不遣。帝欲以為諫議大夫,逢吉曰:「諫議職要重,當待賢者。宿細人,不可使污是官。陛下必用之,請先去臣乃可。」帝不悅。後逢吉罷,詔權知諫議大夫,宰相崔群、王涯同請曰:「諫議大夫,前世或自山林、擢行伍任之者,然皆道義卓異於時。今宿望輕,若待以不次,未足以寵,適以累之也。」請授他官,不聽,使中人宣授焉。宿怨執政不與己,乃日肆讒甚,與皇甫鎛相附離,多中傷正人君子。元和末,持節至淄青,李師道願割地遣子入侍。既而悔,復遣宿往,暴卒於道,贈秘書監。



 熊望者,字原師,擢進士第。性險躁,以辯說游公卿間。劉棲楚為京兆尹,樹權勢,望日出入門下,為刺取事機,陰佐計畫。敬宗喜為歌詩,議置東頭學士,以備燕狎。棲楚薦望,未及用,帝崩。文宗立,韋處厚秉政,詔望因緣險薄,營密職,圖褻幸,言雚沸眾議,貶漳州司戶參軍。



 柏耆者,有縱橫學。父良器,為時威名將。耆志健而望高,急於立名。是時,王承宗以常山叛,朝廷厭兵,耆杖策詣淮西行營謁裴度,且言願得天子一節馳入鎮,可掉舌下之。度為言,乃以左拾遺往。既至,以大誼動承宗,至泣下。乃請獻二州,以二子入質。真擢耆左拾遺,由是聲震一時。遷起居舍人。王承元徙義成軍,遣諫議大夫鄭覃往慰成德軍,賚緡錢百萬。賚未至,舉軍嘩議,穆宗遣耆諭天子意,眾乃信悅。轉兵部郎中、諫議大夫。太和初,李同捷反,詔兩河諸鎮出兵,久無功。乃授耆德州行營諸軍計會使,與判官沈亞之諭旨。會橫海節度使李祐平德州,同捷窮,請降,祐使大將萬洪代守滄州,同捷未出也,耆以三百騎馳入滄,以事誅洪,與同捷朝京師。既行,諜言王廷湊欲以奇兵劫同捷,耆遂斬其首以獻。諸將嫉耆功,比奏攢詆,文宗不獲已,貶耆循州司戶參軍、亞之南康尉。宦人馬國亮譖耆受同捷先所得王稷女及奴婢珍貲。初,祐聞耆殺洪,大驚,疾遂劇。帝曰:「祐若死,是耆殺之。」至是,積前怒,詔長流愛州,賜死。



 贊曰:詩人斥譖人最甚,投之豺虎、有北,不置也。如群、棲楚輩則然,肆訐以示公,構黨以植私,其言纚纚若可聽,卒而入於敗亂也。孔子所謂「順非而澤」者歟,「利口覆邦家」者歟?耆掩眾取功,自速其死,哀哉!



\end{pinyinscope}