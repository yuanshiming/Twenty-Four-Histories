\article{志第二十四 五行一}

\begin{pinyinscope}

 萬物盈於天地之間,而其為物最大且多者有五:一曰水,二曰火,三曰木,四曰金有矛盾,只是形式不同而已。舊矛盾解決了,新矛盾又產生。,五曰土。其用於人也,非此五物不能以為生,而闕其一不可,是以聖王重焉。夫所謂五物者,其見象於天也為五星,分位於地也為五方,行於四時也為五德,稟於人也為五常,播於音律為五聲,發於文章為五色,而總其精氣之用謂之五行。



 自三代之後,數術之士興,而為災異之學者務極其說,至舉天地萬物動植,無大小,皆推其類而附之於五物,曰五行之屬。以謂人稟五行之全氣以生。故於物為最靈。其餘動植之類,各得其氣之偏者,其發為英華美實、氣臭滋味、羽毛鱗介、文採剛柔,亦皆得其一氣之盛。至其為變怪非常,失其本性,則推以事類吉兇影響,其說尤為委曲繁密。



 蓋王者之有天下也,順天地以治人,而取材於萬物以足用。若政得其道,而取不過度,則天地順成,萬物茂盛,而民以安樂,謂之至治。若政失其道,用物傷夭,民被其害而愁苦,則天地之氣沴,三光錯行,陰陽寒暑失節,以為水旱、蝗螟、風雹、雷火、山崩、水溢、泉竭、雪霜不時、雨非其物,或發為氛霧、虹蜺、光怪之類,此天地災異之大者,皆生於亂政。而考其所發,驗以人事,往往近其所失,而以類至。然時有推之不能合者,豈非天地之大,固有不可知者邪?若其諸物種類,不可勝數,下至細微家人里巷之占,有考於人事而合者,有漠然而無所應者,皆不足道。語曰;「迅雷風烈必變。」蓋君子之畏天也,見物有反常而為變者,失其本性,則思其有以致而為之戒懼,雖微不敢忽而已。至為災異之學者不然,莫不指事以為應。及其難合,則旁引曲取而遷就其說。



 蓋自漢儒董仲舒、劉向與其子歆之徒,皆以《春秋》、《洪範》為學,而失聖人之本意。至其不通也,父子之言自相戾。可勝嘆哉!昔者箕子為周武王陳禹所有《洪範》之書,條其事為九類,別其說為九章,謂之「九疇」。考其說初不相附屬,而向為《五行傳》,乃取其五事、皇極、庶證附於五行、以為八事皆屬五行歟,則至於八政、五紀、三德、稽疑、福、極之類,又不能附,至俾《洪範》之書失其倫理,有以見所謂旁引曲取而遷就其說也。然自漢以來,未有非之者。又其祥眚禍痾之說,自其數術之學,故略存之,庶幾深識博聞之士有以考而擇焉。



 夫所謂災者,被於物而可知者也,水旱、螟蝗之類是已。異者,不可知其所以然者也,日食、星孛、五石、六鷁之類是已。孔子於《春秋》,記災異而不著其事應,蓋慎之也。以謂天道遠,非諄諄以諭人,而君子見其變,則知天之所以譴告,恐懼脩省而已。若推其事應,則有合有不合,有同有不同。至於不合不同,則將使君子怠焉。以為偶然而不懼。此其深意也。蓋聖人慎而不言如此,而後世猶為曲說以妄意天,此其不可以傳也。故考次武德以來,略依《洪範五行傳》,著其災異,而削其事應云。



 《五行傳》曰;「田獵不宿,飲食不享,出入不節,奪民農時,及有奸謀,則木不曲直。」謂生不暢茂,多折槁,及為變怪而失其性也。又曰:「貌之不恭,是謂不肅。厥咎狂,厥罰常雨,厥極兇。時則有服妖,時則有龜孽,時則有雞禍,時則有下體生上之痾,時則有青眚青祥、鼠妖,惟金沴木。」



 木不曲直。



 武德四年,亳州老子祠枯樹復生枝葉。老子,唐祖也。占曰:「枯木復生,權臣執政。」眭孟以為有受命者。九年三月,順天門樓東柱已傾毀而自起。占曰:「木僕而自起,國之災。」



 永微二年十一月甲申,陰霧凝凍,封樹木,數日不解。劉向以為木少陽,貴臣象。此人將有害,則陰氣脅木先寒,故得雨而冰也。亦謂之樹介,介,兵象也。



 顯慶四年八月,有毛桃樹生李。李,國姓也。占曰:「木生異實,國主殃。」



 麟德元年十二月癸酉,氛霧終日不解。甲戌,雨木冰。



 儀鳳三年十一月乙未,昏霧四塞,連夜不解。丙申,雨木冰。



 垂拱四年三月,雨桂子於臺州,旬餘乃止。占曰:「天雨草木,人多死。」



 長壽二年十月,萬象神宮側檉杉皆變為柏。柏貫四時,不改柯易葉,有士君子之操;檉杉柔脆,小人性也。象小人居君子之位。



 延載元年十月癸酉,白霧,木冰。



 景龍四年三月庚申,雨木冰。



 景雲二年,高祖故第有柿樹,自天授中枯死,至是復生。



 開元二十一年,蓬州枯楊生李枝,有實,與顯慶中毛桃生李同。二十九年,亳州老子祠,枯樹復榮。是年十一月己巳,寒甚,雨木冰,數日不解。



 永泰元年三月庚子,夜霜,木有冰。



 大歷二年十一月,紛霧如雪,草木冰。九年,晉州神山縣慶唐觀枯檜復生。



 興元元年春,亳州真源縣有李樹,植已十四年,其長尺有八寸,至是枝忽上聳,高六尺,周回如蓋九尺餘。李,國姓也。占曰:「木生枝聳,國有寇盜。」是歲,中書省枯柳復榮。



 貞元元年十二月,雨木冰。四年正月,雨木於陳留,十里許,大如指,長寸餘,中空,所下者立如植。木生於下,而自上隕者,上下易位之象;碎而中空者,小人象;如植者,自立之象。二十年冬,雨木冰。



 元和十五年九月己酉,大雨,樹無風而摧者十五六,近木自拔也。占曰:「木自拔,國將亂。」



 長慶三年十一月丁丑,雨木冰;成都慄樹結實,食之如李。



 寶歷元年十一月丙申,雨木冰。



 大和三年,成都李樹生木瓜,空中不實。七年十二月丙戌,夜霧,木冰。



 開成四年九月辛丑,雨雪,木冰。十月己巳,亦如之。



 會昌元年十二月丁丑,雨木冰。四年正月己酉,雨木冰。庚戌,亦如之。



 咸通十四年四月,成都李實變為木瓜。時人以為:李,國姓也;變者,國奪於人之象。



 廣明二年春,眉州在檀樹已枯倒,一夕復生。



 ○常雨



 武德六年秋,關中久雨。少陽曰昜,少陰曰雨,陽德衰則陰氣勝,故常雨。



 貞觀十五年春,霖雨。



 永徽六年八月,京城大雨。



 顯慶元年八月,霖雨,更九旬乃止。



 開元二年五月壬子,久雨,崇京城門。十六年九月,關中久雨。害稼。



 天寶五載秋,大雨,十二載八月,久雨。十三載秋,大霖雨,害稼,六旬不止。九月,閉坊市北門,蓋井,禁婦人入街市,祭玄冥太社,崇明德門,壞京城垣屋殆盡,人亦乏食。



 至德二載三月癸亥,大雨,至甲戌乃止。



 上元元年四月,雨,訖閏月乃止。二年秋,霖雨連月,渠竇生魚。



 永泰元年九月丙午,大雨,至於丙寅。



 大歷四年四月,雨,至於九月,閉坊市北門,置土臺,臺上置壇,立黃幡以祈晴。六年八月,連雨,害秋稼。



 貞元二年正月乙未,大雨雪,至於庚子,平地數尺,雪上黃黑如塵。五月乙巳,雨,至於丙申,時大饑,至是麥將登,復大雨霖,眾心恐懼。十年春,雨,至閏四月,間止不過一二日。十一年秋,大雨。十九年八月己未,大霖雨。



 元和四年四月,冊皇太子寧,以雨沾服,罷。十月,再擇日冊,又以雨沾服罷。近常雨也。六年七月,霖雨害稼。十二年五月,連雨。八月壬申,雨,至於九月戊子。十五年二月癸未,大雨。八月,久雨,閉坊市北門。宋、滄、景等州大雨,自六月癸酉至於丁亥,廬舍漂沒殆盡。



 寶歷元年六月,雨,至於八月。



 大和四年夏,鄆、曹、濮等州雨,壞城郭廬舍殆盡。五年正月庚子朔,京城陰雪,彌旬。



 開成五年七月,霖雨,葬文宗,龍輴陷,不能進。



 大中十年四月,雨,至於九月。



 咸通九年六月,久雨、崇明德門。



 乾符五年秋,大霖雨,汾、澮及河溢流害稼。



 廣明元年秋八月,大霖雨。



 天復元年八月,久雨。



 服妖。



 唐初,宮人乘馬者,依周舊儀,著冪釭,全身障蔽,永徽後,乃用帷帽,施裙,及頸,頗為淺露,至神龍末,冪釭始絕,皆婦人預事之象。



 太尉長孫無忌以烏羊毛為渾脫氈帽,人多效之,謂之「趙公渾脫」。近服妖也。



 高宗嘗內宴,太平公主紫衫、玉帶、皁羅折上巾,具紛礪七事,歌舞於帝前。帝與武后笑曰:「女子不可為武官,何為此裝束?」近服妖也。



 武后時,嬖臣張易之為母臧作七寶帳,有魚龍鸞鳳之形,仍為象床、犀簟。



 安樂公主使尚方合百鳥毛織二裙,正視為一色,傍視為一色,日中為一色,影中為一色,而百鳥之狀皆見,以其一獻韋後。公主又以百獸毛為廌面,韋後則集鳥毛為之,皆具其鳥獸狀,工費巨萬。公主初出降,益州獻單絲碧羅籠裙,縷金為花鳥,細如絲發,大如黍米,眼鼻觜甲皆備,嘹視者方見之。皆服妖也。自作毛裙,貴臣富家多效之,江、嶺奇禽異獸毛羽採之殆盡。



 韋后妹嘗為豹頭枕以闢邪,白澤枕以闢魅,伏熊枕以宜男,亦服妖也。



 景龍三年十一月,郊祀,韋後為亞獻,以婦人為齋娘,以祭祀之服執事。近服妖也。



 中宗賜宰臣宗楚客等巾子樣,其制高而踣,即帝在籓邸時冠也,故時人號「英王踣」。踣,顛僕也。



 開元二十五年正月,道士尹愔為諫議大夫,衣道乾服視事,亦服妖也。



 天寶初,貴族及士民好為胡服胡帽,婦人則簪步搖釵,衿袖窄小。楊貴妃常以假鬢為首飾,而好服黃裙。近服妖也。時人為之語曰:「義髻拋河裏,黃裙逐水流。」



 元和末,婦人為圓鬟椎髻,不設鬢飾,不施硃粉,惟以烏膏注脣,狀似悲啼者。圓鬟者,上不自樹也;悲啼者,憂恤象也。



 文宗時,吳、越間織高頭草履,纖如經綾谷,前代所無。履,下物也,織草為之,又非正服,而被以文飾,蓋陰斜闒茸泰侈之象。



 乾符五年,雒陽人為帽,皆冠軍士所冠者。又內臣有刻木象頭以里襆頭,百官效之,工門如市,度木斫之曰:「此斫尚書頭,此斫將軍頭,此斫軍容頭。」近服妖也。



 僖宗時,內人束發極急,及在成都,蜀婦人效之,時謂為「囚髻」。



 唐末,京都婦人梳發,以兩鬢抱面,狀如椎髻,時謂之「拋家髻」。又世俗尚以琉璃為釵釧。近服妖也。拋家、流離,皆播遷之兆云。



 昭宗時,十六宅諸王華侈相尚,巾幘各自為制度,都人效之,則曰:「為我作某王頭。」識者以為不祥。



 ○龜孽



 大足初,虔州獲龜,六眼,一夕而失。



 肅宗上元二年,有鼉聚於揚州城門上,節度使鄧景山以問族弟珽,對曰:「鼉,介物,兵象也。」



 貞元三年,潤州魚鱉蔽江而下,皆無首。



 大和三年,魏博管內有蟲,狀如龜,其鳴晝夜不絕。近龜孽也。



 秦宗權在蔡州,州中地忽裂,有石出,高五六尺,廣袤丈餘,正如大龜。



 雞禍。



 垂拱三年七月,冀州雌雞化為雄。



 永昌元年正月,明州雌雞化為雄。八月,松州雌雞化為雄。



 景龍二年春,滑州匡城縣民家雞有三足。京房《易妖占》曰:「君用婦言,則雞生妖。」



 玄宗好鬥雞,貴臣、外戚皆尚之,貧者或弄木雞,識者以為:雞,酉屬,帝生之歲也;鬥者,兵象。近雞禍也。



 大中八年九月,考城縣民家雄雞化為雌,伏子而雄鳴。化為雌,王室將卑之象,反雌伏也。漢宣帝時,雌雞化為雄,至元帝而王氏始萌,蓋馴致其禍也。



 咸通六年七月,徐州彭城民家雞生角。角,兵象,雞,小畜,猶賤類也。



 下體生上之痾。



 咸通十四年七月,宋州襄邑有獵者得雉,五足,三足出背上。足出於背者,下乾上之象;五足者,眾也。



 青眚青祥。



 貞觀十七年四月,立晉王為太子,有青氣繞東宮殿。始冊命而有祲,不祥。十八年六月壬戌,有青黑氣廣六尺,貫於辰戌,其長亙天。



 大和九年,鄭注篋中藥化為蠅數萬飛去。注始以藥術進,化為蠅者,敗死之象。近青眚也。乾元三年六月,昏,西北有青氣三。



 ○鼠妖



 武德元年秋,李密、王世充隔洛水相拒,密營中鼠,一夕渡水盡去。占曰:「鼠無故皆夜去,邑有兵。」



 貞觀十三年,建州鼠害稼。二十一年,渝州鼠害稼。



 顯慶三年,長孫元忌第有大鼠見於庭,月餘出入無常,後忽然死。



 龍朔元年十一月,洛州貓鼠同處。鼠隱伏象盜竊,貓職捕嚙,而反與鼠同,象司盜者廢職容奸。



 弘道初,梁州倉有大鼠,長二尺餘,為貓所嚙,數百鼠反嚙貓。少選,聚萬餘鼠,州遣人捕擊殺之,餘皆去。



 景雲中,有蛇鼠鬥於右威衛營東街槐樹,蛇為鼠所傷。鬥者,兵象。



 景龍元年,基州鼠害稼。



 開元二年,韶州鼠害稼,千萬為群。



 天寶元年十月,魏郡貓鼠同乳。同乳者,甚於同處。



 大歷十三年六月,隴右節度使硃泚於兵家得貓鼠同乳以獻。



 大和三年,成都貓鼠相乳。



 開成四年,江西鼠害稼。



 咸通十二年正月,汾州孝義縣民家鼠多銜蒿芻巢樹上。鼠穴居,去穴登木,賤人將貴之象。



 乾符三年秋,河東諸州多鼠,穴屋、壞衣,三月止。鼠,盜也,天戒若曰:「將有盜矣。」



 乾寧末,陜州有蛇鼠鬥於南門之內,蛇死而鼠亡去。



 ○金沴木



 武德元年八月戊戌,突厥始畢可汁衙帳無故自壞。



 中宗即位,金雞竿折。樹雞竿所以肆赦,始發大號而雞竿折,不祥。



 神龍中,有群狐入御史大夫李承嘉第,其堂無故壞;又秉筆而管直裂,易之又裂。



 開元五年正月癸卯,太廟四室壞。



 天寶十四載十二月,哥舒翰帥師守潼關,前軍啟行,牙門旗至坊門,觸落槍刃,眾以為不祥。



 永秦二年三月辛酉,中書敕庫壞。



 貞元四年正月庚戌朔,德宗御含元殿受朝賀,質明,殿階及欄檻三十餘間自壞,衛士死者十餘人。含元路寢,大朝會之所御也;正月朔,一歲之元。王者之事,天所以儆者重矣。



 大和九年,鄭注為鳳翔節度使,將之鎮,出開遠門,旗竿折。



 光啟初,揚州府署門屋自壞,故隋之行臺門也,制度甚宏麗云。



 《五行傳》曰:「棄法律,逐功臣,殺太子,以妾為妻,則火不炎上。」謂火失其性而為災也。京房《易傳》曰:「上不儉,下不節,盛火數起,燔宮室。」蓋火主禮云。又曰:「視之不明,是謂不哲。厥咎舒,厥罰常燠,厥極疾。時則有草妖,時則有羽蟲之孽,時則有羊禍,時則有目痾,時則有赤眚赤祥,惟水沴火。」



 ○火不炎上



 貞觀四年正月癸巳,武德殿北院火。十三年三月壬寅,雲陽石燃,方丈,晝則如灰,夜則有光,投草木則焚,歷年乃止。火失其性而沴金也。二十三年三月,甲弩庫火。



 永徽五年十二月乙巳,尚書司勛庫火。



 顯慶元年九月戊辰,恩州、吉州火,焚倉廩、甲仗、民居二百餘家。十一月己巳,饒州火。



 證聖元年正月丙申夜,明堂火,武太后欲避正殿,徹樂。宰相姚以為火因人,非天災也,不宜貶損。後乃御端門觀酺,引建章故事,復作明堂以厭之。是歲,內庫災,燔二百餘區。



 萬歲登封元年三月壬寅,撫州火。



 久視元年八月壬子,平州火,燔千餘家。



 景龍四年二月,東都凌空觀災。



 開元五年十一月乙卯,定陵寢殿火。是歲,洪州、潭州災,延燒州署,州人見有物赤而暾暾飛來,旋即火發。十五年七月甲戌,興教門樓柱災。是年,衡州災,延燒三百餘家,州人見有物大如甕,赤如燭籠,所至火即發。十八年二月丙寅,大雨雪,俄而雷震,左飛龍廄災。占曰:「天火燒廄,兵大起。」十月乙丑,東都宮佛光寺火。



 天寶二年六月,東都應天門觀災,延燒左、右延福門,經日不滅。京房《易傳》曰:「君不思道,天火燔其宮室。」九載三月,華嶽廟災,時帝將封西嶽,以廟災乃止。十載八月丙辰,武庫災,燔兵器四十餘萬。武庫,甲兵之本也。



 寶應元年十二月己酉,太府左藏庫火。



 廣德元年十二月辛卯夜,鄂州大風,火發江中,焚舟三千艘,延及岸上民居二千餘家,死者數千人。



 大歷十年二月,莊嚴寺浮圖災。初有疾風震電,俄而火從浮圖中出。



 貞元元年,江陵度支院火,焚江東租賦百餘萬。十三年正月,東都尚書省火。十九年四月,家令寺火。



 二年七月,洪州火,燔民舍萬七千家。元和七年六月,鎮州甲仗庫災,主吏坐死者百餘人。八年,江陵大火。十一年十一月甲戌,元陵火。李師道起宮室於鄆州,將謀亂,既成而火。



 大和二年十一月甲辰,禁中昭德寺火,延至宣政東垣及門下省,宮人死者數百人。三年十月癸丑,仗內火。四年三月,陳州、許州火,燒萬餘家。十月,浙西火。十一月,揚州海陵火。八年三月,揚州火。皆燔民舍千區。五月己巳,飛龍神駒中廄火。十月,揚州市火,燔民舍數千區。十二月,禁中昭成寺火。



 開成二年六月,徐州火,延燒民居三百餘家。四年十二月乙卯,乾陵火。丁丑晦,揚州市火,燔民舍數千家。



 會昌元年五月,潞州市火。三年六月,西內神龍寺火;萬年縣東市火,焚廬舍甚眾。六年八月,葬武宗,辛未,靈駕次三原縣,夜大風,行宮幔城火。



 乾符四年十月,東都聖善寺火。



 大順二年六月乙酉,幽州市樓災,延及數百步。七月癸丑甲夜,汴州相國寺佛閣災。是日暮,微雨震電,或見有赤塊轉門譙藤網中,周而火作。頃之,赤塊北飛,轉佛閣藤網中,亦周而火作。既而大雨暴至,平地水深數尺,火益甚,延及民居,三日不滅。



 ○常燠



 天寶元年冬,無冰。先儒以為陰失節也。又曰:「知罪不誅,其罰燠,夏則暑殺人,冬則物華實。」蓋當寒反燠,象宜刑而賞之也。



 貞元十四年夏,大燠。



 元和九年六月,大燠。



 長慶二年冬,少雪,水不冰凍,草木萌荑如正月。



 廣明元年十一月,暖如仲春。



 ○草妖



 武德四年,益州獻芝草如人狀。占曰:』王德將衰,下人將起,則有木生為人狀。」草,亦木類也。



 景龍二年,岐州郿縣民王上賓家,有苦賣菜高三尺餘,上廣尺餘,厚二分。近草妖也。三年,內出蒜條,上重生蒜。蒜,惡草也;重生者,其類眾也。四年,京畿藍田山竹實如麥。占曰:「大饑。」



 開元二年,終南山竹有華,實如麥,嶺南亦然,竹並枯死,是歲大饑,民採食之。占曰:「國中竹、柏枯,不出三年有喪。」十七年,睦州竹實。



 天寶初,臨川郡人李嘉胤屋柱生芝草,狀如天尊像。



 上元二年七月甲辰,延英殿御座上生白芝,一莖三花。白,喪象也。



 大和九年冬,鄭注之金帶有菌生。近草妖也。



 開成四年六月,襄州山竹有實成米,民採食之。



 光啟元年七月,河中解、永樂生草,葉自相樛結,如旌旗之狀,時人以為「旗子草」。一年七月,鳳翔麟游草生如旗狀。占曰:「其野有兵。」



 ○羽蟲之孽



 武德初,隋將堯君素守蒲州,有鵲巢其砲機。



 貞觀十七年春,齊王祐為齊州刺史,好畜鴨,有豬嚙鴨,頭斷者四十餘。是歲四月丙戌,立晉王為太子,雌雉集太極殿前,雄雉集東宮顯德殿前。太極,三朝所會也。



 永徽四年,宋州人蔡道基舍傍有獸高丈餘,頭類羊,一角,鹿形,馬蹄,牛尾,五色,有翅。占曰:「鳥如畜形者,有大兵。」五年七月辛巳。萬年宮有小鳥如雀,生子大如鳲鳩。



 調露元年,鳴鵽群飛入塞,相繼蔽野,至二年正月,還復北飛,至靈夏北,悉墮地而死,視之皆無首。



 文明後,天下屢奏雌雉化為雄,或半化者。



 景龍四年六月辛巳朔,烏集太極殿梁,驅之不去。



 開元十三年十一月戊子,雄雉馴飛泰山齋宮內。封禪,所以告成功,祀事無重於此者,而野鳥馴飛,不忌禁衛,不祥。二十五年四月,濮州兩烏、兩鵲、兩鸜鵒同巢。隴州鵲哺慈烏。二十八年四月庚辰,慈烏巢宣政殿栱。辛巳,又巢宣政殿栱。



 天寶十三載,葉縣有鵲巢於車轍中。不巢木而巢地,失其所也。



 至德二載三月,安祿山將武令珣圍南陽,有鵲巢於城中砲機者三,雛成乃去。



 大歷八年九月,武功獲大鳥,肉翅狐首,四足有爪,長四尺餘,毛赤如蝙蝠,群鳥隨而噪之。近羽蟲孽也。十三年五月,左羽林軍有鸜鵒乳鵲二。



 貞元四年三月,中書省梧桐樹有鵲以泥為巢。鵲巢知歲次,於羽蟲為有知,今以泥露巢,遇風雨壞矣。是歲夏,鄭、汴境內烏皆群飛,集魏博田緒、淄青李納境內,銜木為城,高二三尺,方十里,緒、納惡而焚之,信宿又然,烏口皆流血。九年春,許州鵲哺烏雛。十年四月,有大鳥飛集宮中,食雜骨數日,獲之,不食死。六月辛未晦,水鳥集左藏庫。十三年十月,懷州鵊巢內有黃雀往來哺食。十四年秋,有異鳥,色青,類鳩、鵲,見於宋州郊外,所止之處,群鳥翼衛,朝夕嗛稻粱以哺之,睢陽人適野聚觀者旬日。十八年六月烏集徐州之滕縣,兼柴為城,中有白烏一,碧烏一。



 元和元年,常州鸛巢於平地。四年十二月,群烏夜集於太行山上。十三年春,淄青府署及城中烏、鵲互取其雛,各以哺子,更相搏擊,不能禁。



 寶歷元年十一月丙申,群烏夜鳴。



 開成元年閏五月丙戌,烏集唐安寺,逾月散。雀集玄法寺,燕集蕭望之塚。二年三月。真興門外鵲巢於古塚。鵲巢知避歲,而古占又以高下卜水旱,今不巢於木而穴於塚,不祥。秋,突厥鳥自塞北群飛入塞。五年六月,有禿鶖群飛集禁苑。鶖,水鳥也。



 會昌元年,潞州長子有白頸烏與鵲鬥。



 大中十年三月,舒州吳塘堰有眾禽成巢,闊七尺,高一尺。水禽山鳥,無不馴狎。中有如人面、綠毛,紺爪觜者,其聲曰「甘」,人謂之「甘蟲」。占曰:「有鳥非常,來宿於邑中,國有兵,人相食。」



 咸通七年,涇州靈臺百里戍有雀生燕,至大俱飛去。京房《易傳》曰:「賊臣在國,厥妖燕生雀。」雀生燕同說。十一年夏,雉集河內縣署。咸通中,吳、越有異鳥極大,四目三足,鳴山林,其聲曰「羅平」。占曰:「國有兵,人相食。」



 乾符四年春,廬江縣北鵲巢於地。六年夏,鴟、雉集於偃師南樓及縣署。劉向說:「野鳥入處,宮室將空。」



 廣明元年春,絳州翼城縣有鵂鶹鳥群飛集縣署,眾鳥逐而噪之。光啟元年、二年,復如之,鵂鶹,一名訓狐。



 中和元年三月,陳留有烏變為鵲。二年,有鵲變為烏。古者以烏卜軍之勝負。烏變為鵲,民從賊之象;鵲復變為烏,賊復為民之象。三年,新安縣吏家捕得雉養之,與雞馴,月餘相與鬥死。四年,臨淮漣水民家鷹化為鵝,而弗能游。鷹以鷙而擊,武臣象也;鵝雖毛羽清潔,而飛不能遠,無搏擊之用,充庖廚而已。



 光啟元年十二月,陜州平陸集津山有雉二首向背而連頸者,棲集津倉廡後,數月,群雉數百來鬥殺之。二年正月,閺鄉、湖城野雉及鳶夜鳴。七月,中條山鵲焚其巢。三年七月,鵲復焚巢。京房《易傳》曰:「人君暴虐,鳥焚其舍。」三年十月,慈州仵城梟與鴟鬥相殺。



 光化二年,幽州節度使劉仁恭屠貝州去,夜有鵂鶹鳥十數飛入帳中,逐去復來。



 昭宗時,有禿鶖鳥巢寢殿隅,帝親射殺之。



 天復二年,帝在鳳翔,十一月丁巳,日南至,夜驟風,有烏數千,迄明飛噪,數日不止。自車駕在岐,常有烏數萬棲殿前諸樹,岐人謂之神鴉。三年,宣州有鳥如雉而大,尾有火光如散星,集於戟門,明日大火,曹局皆盡,惟兵械存。



 ○羊禍



 義寧二年三月丙辰,麟游縣有羔生而無尾。是月乙丑,太原獻羖羊,無頭而不死。



 開元二年正月,原州獻肉角羊。二年三月,富平縣有肉角羊。



 會昌二年春,代州崞縣羊生二首連頭,兩尾。占曰:「二首,上不一也。」



 咸通三年夏,平陶民家羊生羔如犢。



 乾符二年,洛陽建春門外因暴雨,有物墮地如羖羊,不食,頃之入地中,其跡月餘不滅,或以為雨土也。占曰:「當旱。」



 ○赤眚赤祥



 武德七年,河間王孝恭征輔公袥,宴群帥於舟中,孝恭以金碗酌江水,將飲之,則化為血。孝恭曰:「碗中之血,公袥授首之祥。」



 武德初,突厥國中雨血三日。



 光宅初,宗室岐州刺史崇真之子橫、杭等夜宴,忽有氣如血腥。



 武后時,來俊臣家井水變赤如血,井中夜有籲嗟嘆惋聲,俊臣以木棧之,木忽自投十步外。長安中,並州晉祠水赤如血。



 中宗時,成王千里家有血點地,及奩箱上有血淋瀝,腥聞數步。又中郎將東夷人毛婆羅炊飯,一夕化為血。



 景龍二年七月癸巳,赤氣際天,光燭地,三日乃止。赤氣,血祥也。



 天寶六載,少陵原楊慎矜父墓封域內,草人皆流血,慎矜令浮屠史敬思禳之,退朝裸而桎梏於叢棘間,如是數旬而流血不止。十二載,李林甫第東北隅每夜火光起,或有如小兒持火出入者。近赤祥也。



 寶應元年八月庚午夜,有赤光亙天,貫紫微,漸移東北,彌漫半天。



 大歷十三年二月,太僕寺有泥像,左臂上有黑汗滴下,以紙承之,血也。



 貞元二年十一月壬午,日沒,有赤氣五,出於黑雲中,亙天。十二年九月癸卯,夜有赤氣如火,見北方,上至北斗。十七年,福州劍池水赤如血。二十一年正月甲戌,雨赤雪於京師。



 元和十四年二月,鄆州從事院門前地有血,方尺餘,色甚鮮赤,不知所從來,人以為自空而墮也。



 長慶元年七月戊午,河水赤,三日止。



 寶歷元年十二月乙酉夜,西北有霧起,須臾遍天,霧止,有赤氣,或淺或深,久而乃散。



 大和元年四月庚戌,北方有赤氣,中有數白氣間之。六月乙卯夜,西北有赤氣。八月癸卯,京師見赤氣滿天。二年閏三月乙卯,北方有赤氣如血。



 咸通七年,鄭州永福湖水赤如凝血者三日。



 乾符六年,中書政事堂忽旦有死人,血污滿地,不知主名。又御井水色赤而腥,渫之,得一死女子腐爛,近赤祥也。



 中和二年七月丙午夜,西北方赤氣如絳,際天。



 光啟元年正月,潤州江水赤,凡數日。



 ○水沴火



 幽州坊穀地常有火,長慶三年夏,遂積水為池。近水沴火也。



\end{pinyinscope}