\article{列傳第九十二 白裴崔韋二李皇甫王}

\begin{pinyinscope}

 白志貞者,本名琇珪,故太原史也。事節度使李光弼,硜硜自力,有智數。光弼善之月。未完稿。編入《列寧全集》第30卷。本文闡述從資本主,使與帳下議。代宗素聞,及光弼卒,擢累司農卿。在官十年,德宗以為敏,遂倚腹心,進授神策軍使,賜今名。有所建白,善窺億帝指,故言無不從。從狩奉天,以為行在都知兵馬使。懼李懷光暴其惡,乃與趙贊、盧杞等抑懷光不使朝。懷光反,論斥其奸,貶恩州司馬,贊播州司馬。稍徙閬州別駕。貞元二年,起為果州刺史,宰相李勉固諫,不許。明年,拜浙西觀察使,死於官。



 裴延齡,河中河東人。乾元末,為汜水尉,賊陷東都,去客江夏。華州刺史董晉表署判官,稍遷太常博士。盧杞秉政,引為膳部員外郎、集賢院直學士。崔造表知東都度支院。召為祠部郎中,不待命,輒還集賢院,宰相張延賞疾其易,出為昭應令。與尉交訴所賕,京兆尹鄭叔則佑尉,而御史中丞竇參善延齡,卒逐尹。德宗用參輔政,即擢延齡司農少卿。



 會班宏卒,假領度支。延齡素不善財計,乃廣鉤距,取宿奸老吏與謀,以固帝幸。因建言:「左藏,天下歲入不貲,耗登不可校。請列別舍,以檢盈虛。」於是以天下宿負八百萬緡析為負庫,抽貫三百萬緡為賸庫,樣物三十萬緡為季庫,帛以素出、以色入者為月庫。帝皆可之。然天下負皆窮人,償入無期,抽貫與給皆盡;樣物與帛固有籍,延齡但多其薄最吏員以詭帝,於財用無所加也。俄以戶部侍郎為真。又請以京兆苗錢市草千萬,俾民輸諸苑。宰相陸贄等以為非是,不從。京右偏故有閟葦地數頃,延齡妄言:「長安、咸陽間,得陂艿數百頃,願以為內廄牧地,水甘草薦與苑廄等。」帝信之,以問宰相,皆曰:「當無有。」帝遣使按覆,果詐。延齡大慚,帝不責也。



 京兆積歲和市不得直,尹李充請之官,延齡誣其妄,反令還輸,號曰「底折錢」。嘗請斂財以實府,帝曰:「安得而實之?」延齡曰:「開元、天寶間,戶口繁息,百司務殷,官且有缺者。比兵興,戶不半在,今一官治數司足矣。請後官闕不即補,收其稟以實帑簿。」



 它日,帝謂延齡曰:「朕所居浴堂殿,一棟將壓,念易之,未能也。」延齡曰:「宗廟至重,殿棟微矣。且陛下本分錢,用之亡窮,何所難哉?」帝驚曰:「本分錢奈何?」對曰:「此在經誼,愚儒不能知,臣能言之。按禮,天下賦三之:一以充乾豆,一以事賓客,一君之庖廚。陛下奉宗廟,能竭天下賦三之一乎?鴻臚禮賓,勞予四夷,用十一為有贏。陛下所御饔餼簡儉,以所餘為百官稟料飧錢,未盡也,則所不盡者為本分錢。以治殿數十尚不乏,況一棟哉!」帝頷曰:「人未嘗為朕言之。」又造神龍佛祠,須材五十尺者。延齡妄奏:「同州得大谷,木數十章,度皆八十尺,」帝曰:「吾聞開元時,近山無巨木,求之嵐、勝間。今何地之近、材之良邪?」延齡曰:「異材瑰產,處處有之,待聖主乃出。今生近輔,豈開元所當得也!」帝悅。



 是時,陸贄為宰相,帝素所信重,極論其譎妄不可任,帝以為排媢,愈益厚延齡。贄上疏列其狀,具言:「延齡嘗奏句獲乾隱二千萬緡,請舍別庫為羨餘,供天子私費,故上之興作廣,宣索多矣。延齡欲實其言,乃大搜市廛,奪所入獻,逮捕匠徒,迫脅就功,號曰『敕索』,弗仇其直,名曰『和雇』,弗與之庸。又度支出納,與太府交相關制,出物旬計,見物月計,符按覆核,有御史以監董之,則財用不得回隱。延齡乃言掊糞土得銀十三萬兩,它貨且百萬,已棄而獲,皆羨餘也,悉移舍以供別敕。太府卿韋少華劾其妄,陛下縱之不為治,此乃侵削兆民,為天子取怨於下。」又引建中橫斂多積致播遷者,其言甚深切。帝得奏不悅。會鹽鐵使張滂、京兆尹李充、司農卿李銛皆指延齡專以險偽罔上,帝怒,乃罷贄宰相,左除滂等官。



 時大旱,人情愁惴。延齡言:「贄等失權怨望,顯言歲饑民流、度支糧芻乏以激怒眾士。」它日,帝畋苑中,而神策軍訴度支不賦廄芻者,天子惑延齡言,乃下詔斥逐贄等,朝廷震恐。延齡又捕充所善吏張忠榜掠之,誣充「沒官錢五十萬緡,以餌結權幸,令妻以犢車載金餉贄。」忠具獄,其母投訴光順門匭,有詔御史審劾,一夕得狀,乃釋忠。延齡不得逞,復奏充妄用京兆錢穀,願下有司比句,以比部郎中崔元翰欲釋憾於贄也。賴刑部侍郎奚陟辨治,充等得不冤。



 延齡資苛刻,又劫於利,專剝下附上,肆騁譎怪。其進對,皆他人莫敢言,而延齡言之不疑,亦人之所未聞者。帝頗知其詐,但以其不隱,欲聞外事,故斷用不疑。延齡恃得君,謂必輔政,少所降下,至嫚罵邇臣,時人側目。屬疾臥第,載度支官物輸之家,無敢言。帝念之,使者日三輩往。死,年六十九。人語以相安,唯帝悼不已。冊贈太子太傅、上柱國。永貞初,度支建言:「延齡曩列別庫,分藏正物,無實益而有吏文之煩。」乃詔復以還左藏。元和中,有司謚曰繆。



 崔損,字至無,系本博陵。大歷間,中進士、博學宏辭,補校書郎、咸陽尉。避親,改大理評事。累勞至右諫議大夫。於時,宰相趙憬卒,盧邁屬疾,裴延齡素善損,薦之德宗。貞元十二年,以本官同中書門下平章事。始,中書虛位十日,議者謂選有德,及用損,中外悵失。而損性齪齪能自將,延英進見,不敢出一言及天下事。逾年,進門下侍郎。嘗以疾臥家久,賜絹三百為醫藥費。



 損無卓卓稱於人者,而歷二省華要至宰相。母殯而不葬,亦不展殯;女兄為尼,沒不臨喪。建中後,宰相無久任者,損以便柔遜願中帝意,乃留八年。帝亦知公議病其持祿,然憐遇彌渥。卒,贈太子太傅,謚曰靖。



 韋渠牟,京兆萬年人,工部侍郎述從子也。少警悟,工為詩,李白異之,授以古樂府。去為道士,不終,更為浮屠,已而復冠。浙西韓滉表試校書郎,進至四門博士。



 貞元十二年,德宗誕日,詔給事中徐岱、兵部郎中趙需、禮部郎中許孟容與渠牟及佛老二師並對麟德殿,質問大趣。渠牟有口辯,雖於三家未究解,然答問鋒生,帝聽之意動。遷秘書郎,進詩七百言。未浹旬,擢右補闕內供奉。始,同列易之,後數遣中人專召渠牟,由是皆屬目。歲中,至諫議大夫。大抵延英對,雖大臣率漏下二三刻止,渠牟每奏事,輒五六刻乃罷,天子歡甚。渠牟為人佻躁,志向浮淺,不根於道德仁義,特用憸巧中帝意,非有嘉謨正辭感悟得君也。



 自陸贄免,帝躬攬庶政,不復委權於下。宰相取充位、行文書而已,至守宰、御史,皆自推簡。然處深宮,所倚而信者裴延齡、李齊運、王紹、李實、韋執誼與渠牟等,其權侔人主。延齡、實皆奸虐,紹無所建明。渠牟後出,望最輕,張恩勢以動天下,召崔芋於茅山,超鄭隨布衣至補闕,引醴泉令馮伉為給事中、太子侍讀。帝既偏於任聽,士之浮競甘進者爭出其門,赫然勢焰可炙。再擢太常卿。卒,年五十三,贈刑部尚書,謚曰忠。所論著甚多,傳於時。



 李齊運者,蔣王惲孫。始補寧王府東閣祭酒,擢累監察御史,復闢江淮都統李峘府。由工部郎中為長安令,政頗修辦。宗正少卿李瀚從子有所訟,齊運於瀚為卑行,而不禮訟者。瀚怒,辱諸朝,齊運以聞,代宗貶瀚。由是稍擢京兆少尹。出為河中尹、晉絳慈隰觀察使。



 德宗出狩,李懷光還兵奔難,晝夜馳,及河中,士罷困,乃休三日。齊運悉所賦勞軍,牛酒豐甘,人人喜悅。及懷光反,還守河中,齊運棄城走。詔拜京兆尹。時李晟壁渭橋,齊運發民築城保,督芻粟以餉晟。賊平,頗有助。萬年丞源邃不事,齊運怒,捽辱之,死於廷。邃家告冤,御史大夫崔縱請窮治,帝不許。御史聯章深劾,齊運訴於帝,言為朋黨所擠。天子使宰相諭諫官御史,後毋得群署章以劾,然卒不直邃冤。



 久之,大蝗旱,齊運不能政,乃以韓洄代之,改宗正卿、閑廄宮苑使。進至禮部尚書。宰相內殿對已,齊運常次進,帝與參決大事。既無學,暗於大體,第以甘言阿匼而已。嘗薦李錡為浙西,受賂數十萬,又薦李詞為湖州刺史,人告其贓,帝置不問。齊運臥疾,滿歲不能謁,每除吏,往往遣使即家咨逮。晚以妾為妻,具冕服行禮,士人蚩之。卒,年七十二,贈尚書左僕射。



 李實,道王元慶四世孫。以廕仕,嗣曹王皋闢署江西府判官,遷蘄州刺史。皋節度山南東道,復從之。皋卒,實知後務,刻薄軍費,士怨怒,欲殺之,夜縋亡歸京師。



 累進司農卿,擢拜京兆尹,封嗣道王。怙寵而愎,不循法度。貞元二十年旱,關輔饑,實方務聚斂以結恩,民訴府上,一不問。德宗訪外疾苦,實詭曰:「歲雖旱,不害有秋。」乃峻責租調,人窮無告,至撤舍鬻苗輸於官。優人成輔端為俳語諷帝,實怒,奏賤工謗國,帝為殺之。或言:「古者,瞽誦箴諫,雖恢諧托諭,何誅焉?」帝悔,然不罪實。



 故事,京兆避臺官。實嘗與御史王播遇,而騶唱爭道。播鉤責從者,實怒,奏播為三原令,廷辱之。惡萬年令李眾,誣逐虔州司馬,以所善虞部員外郎房啟代之。其怙權作威若此。公卿為讒短遷斥者甚眾,專情謷色見顏間。權德輿為禮部,而實私薦士二十人,迫語曰:「應用此第,不爾,君且外遷!」德輿雖拒之,然常憚其誣。吏部每奏科目頗嚴密,以杜請托,實公詣曹劫請趙宗儒,無所畏。



 詔書蠲人逋租,實格詔固斂,畿民大困,官吏皆被榜罰,掊取二十萬緡。吏乞貸豪厘,輒死。按之無罪者,猥曰「死亦非枉」,復殺之。專以殘忍為政。順宗在諒暗,不逾月,實殺數十人於府。貶通州長史。市人爭懷瓦石邀劫之,實懼,夜遁去,長安中相賀。以赦令內移,死虢州。



 皇甫鎛,涇州臨涇人。貞元初,第進士,又擢制科,為監察御史。居喪游處不度,下除詹事府司直。久之,遷吏部員外郎,典南曹,鈐制吏奸,稍知名。進郎中,遷累司農卿,判度支,改戶部侍郎。憲宗方伐蔡,急於用度,鎛裒會嚴亟,以辦濟師,帝悅,進兼御史大夫。蔡平之明年,遂同中書門下平章事,猶領度支。



 鎛以吏道進,既由聚斂句剝為宰相,至雖市道皆嗤之。崔群、裴度以聞,帝怒,不聽。度乃表罷政事,極論鎛奸邪苛刻,天下怨之,將食其肉。且言:「天下安否系朝廷,朝廷輕重在輔相。今承宗削地,程權赴闕,韓弘輿疾討賊,非力能制之,顧朝廷處置能服其心也。若相鎛,則四方解矣。請授以浙西觀察使。」其辭切至。帝以天下略平,亦欲崇臺沼宮觀自娛樂,鎛與程異知帝意,故數貢羨財,陰佐所欲,又賂吐突承璀為奧援。故帝排眾論,決任之,反以度為朋黨,不內其言。



 鎛乃益以巧媚自固,建損內外官稟佐國用,給事中崔植上還詔書,乃止。帝斥內帑所餘,詔度支評直,鎛貴售之以給邊兵,故繒陳彩,觸手輒壞,士怨怒,聚焚之。裴度以其事聞,鎛指所著靴曰:「此內府所出,牢韌可服,彼言不可用,詐也。」帝信之。鎛銜度,乃與李逢吉、令狐楚合擠之,出度太原。又以崔群有天下重望,勁正敢言,後議帝號,鎛乃譖群抑損徽稱。帝怒,逐群湖南。



 鎛罷度支,進門下侍郎平章事。嘗與金吾將軍李道古共薦方士柳泌、浮屠大通為長年藥,帝惑之。穆宗在東宮,聞其奸妄,始聽政,集群臣於月華門,貶鎛崖州司戶參軍,死其所。



 泌者,本楊仁晝也,習方伎。道古薦於鎛,召入禁中,自云能致藥為不死者,因言:「天臺山靈仙所舍,多異草,願官天臺,求採之。」起徒步拜臺州刺史,賜金紫。諫臣固爭,以為列聖亦有寵方士,未嘗使牧民,帝曰:「煩一州而致長年於君父,何愛哉?」後不敢言。泌驅吏民採藥山谷間,鞭笞苛急,歲餘無所獲。懼詐窮,舉族遁去,浙東觀察使捕得。鎛與道古營解,乃復待詔翰林。帝餌泌藥,浸躁怒不常,宦侍懼,以弒崩。大通自言百五十歲,鎛敗,與泌皆誅。初,吏責泌妄,答曰:「皆道古教我。」解衣即刑,卒無它異。



 鎛之貶,前坊州刺史班肅以嘗僚,獨餞於野,朝廷義之,擢為司封員外郎。



 鎛弟鏞,字龢卿,第進士。鎛為相時,任河南少尹,見權寵太盛,每極言之,鎛不悅,乃求分司為太子右庶子。鎛敗,朝廷賢之,授國子祭酒。開成初,以太子少保卒。鏞能屬文,工詩。為人寡言正色,衣冠甚偉,不屑世務,所交皆知名士。著書數十篇。



 王播,字明易又,其先太原人。父恕為揚州倉曹參軍,遂家焉。播,貞元中與弟炎、起皆有名,並擢進士,而播、起舉賢良方正異等。補盩厔尉。以善治獄,御史中丞李汶薦為監察御史。雲陽丞源咸季坐賕免,賂有司復得調,播劾解其官。歷侍御史。李實為京兆尹,與播遇諸衢。故事,尹當避道揖,實不肯。播移文詆之。實大怒,表播為三原令,將折之,播受命,趨府謝如禮。邑中豪強犯法,未嘗輒貸,歲終課最。實重其才,更薦之,德宗將擢以要近,會母喪解。還,除駕部員外郎。長安令于頔奴客與民盜馬,吏系民而縱奴,播捕取,均其罰。遷工部郎中,知御史雜事。刺舉不阿,有能稱。關中饑,諸鎮或閉糴,播以為言,三輔不乏。歷虢州刺史。



 李巽領鹽鐵,奏以副己。擢御史中丞,歲終,改京兆尹。時禁屯列畿內者,出入屬鞬佩劍,奸人冒之以剽劫,又勛將家馳獵近郊,播請一切呵止,盜賊不能隱,皆走出境。憲宗以為能,進刑部侍郎,領諸道鹽鐵轉運使。是時,天下多故,大理議讞,科條叢繁,播悉置格律坐隅,商處重輕,剖決如流,吏不能竄其私。帝討淮西也,切於饋餉,播引程異自副,異尤通萬貨盈虛,使馳傳江淮,裒財用以給軍興,兵得無乏。帝嘉其功,超拜禮部尚書。稍以貲賄結宦要,中外以為言。



 播薦皇甫鎛,及鎛用事,更忌播,而以異代使,播罷守本官。久之,檢校戶部尚書,為劍南西川節度使。穆宗立,逐鎛,播求還。長慶初,召為刑部尚書,復領鹽鐵,進中書侍郎、同中書門下平章事。時權幸競進,播賴其力至宰相,專務將迎,居位無所裨益,復失河北,眾望不厭,乃以檢校尚書右僕射出為淮南節度使,仍領使職,不肯易印,詔聽自隨。是時,南方旱歉,人相食,播掊斂不少衰,民皆怨之。然浚七里港以便漕引,後賴其利。



 敬宗即位,即拜檢校司空,以王涯代使。播失職,見王守澄方得君,厚以金謝,守澄乘間薦之,天子有意復用播。於是諫議大夫獨孤朗、張仲方、起居郎孔敏行、柳公權、宋申錫、補闕韋仁實、劉敦儒、拾遺李景讓、薛廷老等見延英,言播傾邪關通帝左右狀,帝沖暗,不內其言,遂復領使,天下公議益不與。



 文宗立,就進檢校司徒。太和元年,入朝,拜左僕射,復輔政,累封太原郡公。時韋處厚當國,以獻替自任,天子向之。播專以錢穀進,不甚與事。居位四年,卒,年七十二,贈太尉,謚曰敬。



 播少孤貧,自刻苦,至成立,居官以強濟稱。天性勤吏職,每視簿領紛積於前,人所不堪者,播反用為樂。所署吏,茍無大罪,以歲勞增秩而已,卒不易所職。雅善占奏,雖數十事,未嘗書於笏。再領鹽鐵,嗜權利,不復初操。重賦取,以正額月進為羨餘,歲百萬緡。自淮南還,獻玉帶十有三、銀碗數千、綾四十萬,遂再得相云。



 起,字舉之,釋褐校書郎,補藍田尉。李吉甫闢為淮南掌書記,以殿中侍御史入兼集賢殿直學士。元和末,累遷中書舍人。數上疏諫穆宗畋游事,歲中考第一。錢徽坐貢舉失實貶,詔起覆核,起建言:「以所試送宰相閱可否,然後付有司。」詔可。議者謂起為失職。



 拜禮部侍郎。李朅叛,與播俱上疏請詔王智興討之,卒定其亂。賜金紫,拜河南尹,進吏部侍郎。方播以僕射居相,避選曹,改兵部,為集賢殿學士。拜陜虢觀察使。時亳州刺史李繁以擅誅賊抵罪,起言:「繁父有功,而二千石不宜償賊死。」不報。



 入拜尚書左丞,以戶部尚書判度支。靈武、邠、寧多曠土,奏為營田,以省饋輓。歷河中節度使。方蝗旱,粟價騰踴,起下令家得儲三十斛,斥其餘以市,否者死。神策士怙勢不從,寘於法。由是廥積咸出,民賴以生。召授兵部尚書。以檢校尚書右僕射為山南東道節度使。濱漢塘堰聯屬,吏弗完治,起至部,先修復,與民約為水令,遂無兇年。



 李訓為宰相,起門生也,欲引與共政,即加銀青光祿大夫,復以兵部尚書召判戶部。訓敗,起素長厚,人不以訓諉之,止罷其判。俄加皇太子侍讀。文宗上文,好古學,是時,鄭覃以經術進,起以敦博顯,帝數訪逮時政。因積雨,願寬逐臣過惡,又短鮑叔終身不忘人過,以解帝錮人意。俄兼太常卿、禮儀使。帝題詩太子笏以賜,詔畫像便殿,號「當世仲尼」,其寵遇如此。又使廣《五位圖》,俾太子知古今治亂。開成三年,入翰林,為侍講學士,改太子少師。



 起治生無檢,所得祿賜為僮婢盜有,貧不能自存。帝知之,詔月益仙韶院錢三十萬。議者謂與玩臣分給,可恥也。起賴其入,不克讓。



 武宗立,為章陵鹵簿使、東都留守。召為吏部尚書,判太常卿。帝患選士不得才,特命起典貢舉。進尚書左僕射,封魏郡公。凡四舉士,皆知名者,人伏其鑒。擢山南西道節度使、同中書門下平章事。以夙儒兼宰相秩,前世所罕。入辭,帝勞曰:「宰相無內外。公,國耆老,朕有闕,當以聞。」宴賜備厚。宣宗初,檢校司空,以疾願代,不許。卒,年八十八,贈太尉,謚曰文懿。喪還,命使者吊其家,葬及祥亦如之。



 起性友悌,播喪,哀戚加於人。嗜學,非寢食不輒廢。天下之書無不讀,一經目,弗忘也。莊恪太子薨,詔為哀冊,詞情淒惋,當世稱之。帝嘗以疑事令使者口質,起具榜子附使者上,凡成十篇,號曰《寫宣》。它撰集亦多。



 炎終太常博士。子鐸、鐐自有傳。



 起子龜、式。



 龜,字大年,性高簡,博知書傳,無貴胄氣。常以光福第賓客多,更住永達里,林木窮僻,構半隱亭以自適。侍父至河中,廬中條山,朔望一歸省,州人號「郎君穀」,未始以人事自嬰。武宗雅知之,以左拾遺召。入謝,自陳病不任職,詔許。終父喪,召為右補闕。再擢屯田員外郎,稱疾去。崔璵觀察宣歙,表為副,龜樂宛陵山水,故從之。入為祠部郎中、史館修撰。咸通中,知制誥。鐸為相,改太常少卿、同州刺史。牙將白約素暴橫,嘗嘩言月稟薄,以動士心為亂,龜捕殺之,人皆震心慄。徙浙東觀察使。初,式臨州有惠政,人聞其至,歡迎之。卒,贈工部尚書。



 子蕘,力學,有文辭,以鐸當國,不貢進士。終右司員外郎。



 式以廕為太子正字,擢賢良方正科,累遷殿中侍御史。少節儉,巧於宦,因鄭注以交王守澄,中丞歸融劾之,出為江陵少尹。



 大中中,為晉州刺史,飾郵傳,器用畢給。會河曲大歉,民流徙,佗州不納,獨式勞恤之,活數千人。時特峨胡亦饑,將入寇汾、澮,聞式嚴備,不敢道境,報其種落曰:「晉州刺史當避之!」以善最稱。



 徙安南都護。故都護田早作木柵,歲率緡錢,既不時完,而所責益急。式取一年賦市芍木,豎周十二里,罷歲賦外率以紓齊人。浚壕繚柵,外植刺竹,寇不可冒。後蠻兵入掠錦田步,式使譯者開諭,一昔去,謝曰:「我自縛叛獠,非為寇也。」忠武戍卒服短後褐,以黃冒首,南方號「黃頭軍」,天下銳卒也。初,交阯數有變,懼式威,不自安,嘩曰:「黃頭軍將度海襲我矣!」相率夜圍城,合噪:「請都護北歸,我當抗黃頭軍。」式徐被甲,引家僮乘城責讓,矢■交發,叛者走。翌日,盡捕斬之。初,容管災歉,不歲貢,式始上輸,大犒宴軍中。歸質外蕃,而占城、真臘慕義,悉入獻,亦還所掠王民。



 寧國劇賊仇甫亂,明越觀察使鄭祗德不能討,宰相選式往代,詔可,因至京師。懿宗問方略,對曰:「第假臣兵,寇不足平也。」左右宦要皆曰:「兵眾則饋多,當惜天下費。」式奏:「盜若猖狂,天誅不亟決,東南征賦闕矣,寧得以億萬計之乎?兵多則功速費寡。二者孰利?」帝顧左右曰:「宜與兵。」於是詔益許、滑、淮南兵。式發自光福里第,麾幟皆東靡,獵獵有聲,喜曰:「是謂得天時矣!」聞賊用騎兵,乃閱所部,得吐蕃、回鶻遷隸數百,發龍陂監牧馬起用之,集土團諸兒為向導,擒甫斬之。加檢校右散騎常侍。餘姚民徐澤專魚鹽之利,慈溪民陳瑊冒名仕至縣令,皆豪縱,州不能制。式曰:「甫竊發,不足畏;若澤、瑊,乃巨猾也。」窮治其奸,皆榜死。



 咸通三年,徐州銀刀軍亂,以式檢校工部尚書,徙武寧節度使,詔許、滑兵自隨。視事三日,悉以計誅亂兵。會詔降武寧為團練,罷歸。終左金吾大將軍。



 贊曰:裴延齡引經誼惑其主,以不忠為忠。德宗倚延齡、韋渠牟等商天下成敗,自謂明而卒陷不明。君臣回沈,可不戒哉!憲宗銳於立功,而皇甫鎛以聚斂取宰相。夫宰相者,乃天下選,彼暫勞一功,烏足勝任哉?中興之不終,有為而然。



\end{pinyinscope}