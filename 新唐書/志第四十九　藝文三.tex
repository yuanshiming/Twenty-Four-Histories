\article{志第四十九 藝文三}

\begin{pinyinscope}

 丙部子錄,其類十七:一曰儒家類,二曰道家類,三曰法家類,四曰名家類盡心知性孟子主張的一種反省內心的認識方法和修養方,五曰墨家類,六曰縱橫家類,七曰雜家類,八曰農家類,九曰小說類,十曰天文類,十一曰歷算類,十二曰兵書類,十三曰五行類,十四曰雜藝術類,十五曰類書類,十六曰明堂經脈類,十七曰醫術類。凡著錄六百九家,九百六十七部,一萬七千一百五十二卷;不著錄五百七家,五千六百一十五卷。



 《晏子春秋》七卷晏嬰。



 《曾子》二卷曾參。



 《子思子》七卷孔人及。



 《公孫尼子》一卷



 趙岐注《孟子》十四卷孟軻。



 劉熙注《孟子》七卷



 鄭玄注《孟子》七卷



 綦毋邃注《孟子》七卷



 《荀卿子》十二卷荀況。



 《董子》一卷董無心。



 《魯連子》一卷魯仲連。



 陸賈《新語》二卷



 賈誼《新書》十卷



 桓寬《鹽鐵論》十卷



 劉向《新序》三十卷



 又《說苑》三十卷



 《揚子法言》六卷揚雄。



 宋衷注《法言》十卷



 李軌注《法言》三卷



 陸績注《揚子太玄經》十二卷



 虞翻注《太玄經》十四卷



 範望注《太玄經》十二卷



 宋仲孚注《太玄經》十二卷



 蔡文邵注《太玄經》十卷



 《桓子新論》十七卷桓譚。



 王符《潛夫論》十卷



 《仲長子昌言》十卷仲長統。



 荀悅《申鑒》五卷



 《魏子》三卷魏朗。



 魏文帝《典論》五卷



 《徐氏中論》六卷徐幹。



 王粲《去伐論集》三卷



 王肅《政論》十卷



 《杜氏體論四卷杜恕。



 《顧子新論》五卷顧譚。



 文禮《通語》十卷殷興續。



 諸葛亮《集誡》二卷



 陸景《典訓》十卷



 《譙子法訓》八卷



 又《五教》五卷譙周。



 王嬰《古今通論》三卷



 《周生烈子》五卷



 《袁子正論》二十卷



 又《正書》二十五卷袁準。



 《孫氏成敗志》三卷孫毓。



 夏侯湛《新論》十卷



 楊泉《物理論》十六卷



 又《太元經》十四卷劉緝注。



 華譚《新論》十卷



 虞喜《志林新書》二十卷



 又《後林新書》十卷



 《顧子義訓》十卷顧夷。



 蔡洪《清化經》十卷



 干寶《正言》十卷



 又《立言》十卷



 蔡韶《閎論》二卷



 呂竦《要覽》五卷



 周舍《正覽》六卷



 劉徽《魯史欹器圖》一卷



 綦毋氏《誡林》三卷



 《顏氏家訓》七卷顏之推。



 李穆叔《典言》四卷



 王滂《百里昌言》二卷



 《崔子至言》六卷崔靈童。



 盧辯《墳典》三十卷



 王劭《讀書記》三十二卷



 王通《中說》五卷



 辛德源《正訓》二十卷



 《太宗序志》一卷



 又《帝範》四卷賈行注。



 高宗《天訓》四卷



 武後《紫樞要錄》十卷



 又《臣軌》二卷



 《百寮新誡》五卷



 《青宮紀要》三十卷



 《少陽政範》三十卷



 《列籓正論》三十卷



 章懷太子《春宮要錄》十卷



 又《修身要覽》十卷



 《君臣相起發事》三卷



 魏征《諫事》五卷



 又《自古諸侯王善惡錄》二卷



 張大玄《平臺百一寓言》三卷



 楊相如《君臣政理論》三卷



 陸善經注《孟子》七卷



 張鎰《孟子音義》三卷



 楊倞注《荀子》二十卷汝士子,大理評事。



 王涯注《太玄經》六卷



 員俶《太玄幽贊》十卷開元四年,京兆府童子,進書,召試及第,授散官文學,直弘文館。



 柳宗元注《楊子法言》十三卷



 李襲譽《五經妙言》四十卷



 鄭澣《經史要錄》二十卷



 劉貺《續說苑》十卷



 杜正倫《百行章》一卷



 憲宗《前代君臣事跡》十四篇



 武後《訓記雜載》十卷採《青宮紀要》、《維城典訓》、《古今內範》、《內範要略》等書為《雜載》云。



 《維城典訓》二十卷



 褚無量《翼善紀》卷亡。



 裴光庭《搖山往則》一卷



 又《維城前軌》一卷



 丁公著《皇太子諸王訓》十卷



 《六經法言》二十卷韋處厚、路隋撰。



 崔郾《諸經纂要》十卷



 於志寧《諫苑》二十卷



 王方慶《諫林》二十卷



 楊浚《聖典》三卷校書郎,開元中上。



 張九齡《千秋金鏡錄》五卷



 唐次《辨謗略》三卷



 《元和辨謗略》十卷令狐楚、沈傳師、杜元穎撰。



 裴潾《大和新脩辨謗略》三卷



 李仁實《格論》三卷



 趙冬曦《王政》三卷景龍二年上。



 馮中庸《政錄》十卷開元十九年上,授汜水尉。



 《賈子》一卷開元中藍田尉。失名。



 儲光羲《正論》十五卷兗州人,開元進士第,又詔中書試文章,歷監察御史,安祿山反,陷賊自歸。



 牛希濟《理源》二卷



 陸質《君臣圖翼》二十五卷



 李吉甫《古今說苑》十一卷



 李德裕《御臣要略》卷亡。



 丘光庭《康教論》一卷



 《元子》十卷



 又《浪說》七篇



 《漫說》七篇元結。



 杜信《元和子》二卷



 林慎思《伸蒙子》三卷咸通中人。



 《冀子》五卷冀重,字子泉,定州容城人。廣明脩武令。



 崔愨《儒玄論》三卷字敬之,後魏白馬侯浩七世孫,中和光祿丞。



 右儒家類六十九家,九十二部,七百九十一卷。陸善經以下不著錄三十九家,三百七十一卷。



 《鬻子》一卷鬻熊。



 《老子道德經》二卷李耳。



 又三卷



 河上公注《老子道德經》二卷



 王弼注《新記玄言道德》二卷



 又《老子指例略》二卷



 蜀才注《老子》二卷



 鐘會《注》二卷



 羊祜《注》二卷



 又《解釋》四卷



 孫登注《老子》二卷



 王尚《注》二卷



 袁真《注》二卷



 張憑《注》二卷



 劉仲融《注》二卷



 陶弘景《注》四卷



 樹鐘山《注》二卷



 李允願《注》二卷



 陳嗣古《注》二卷



 僧惠琳《注》二卷



 惠嚴《注》二卷



 鳩摩羅什《注》二卷



 義盈《注》二卷



 程韶《集注》二卷



 任真子《集注》四卷



 張道相《集注》四卷



 盧景裕、梁曠等《注》二卷



 安丘望之《老子章句》二卷



 又《道德經指趣》三卷



 王肅《玄言新記道德》二卷



 梁曠《道德經品》四卷



 嚴遵《指歸》十四卷



 何晏《講疏》四卷



 又《道德問》二卷



 梁武帝《講疏》四卷



 又《講疏》六卷



 顧歡《道德經義疏》四卷



 又《義疏治綱》一卷



 孟智周《義疏》五卷



 戴詵《義疏》六卷



 葛洪《老子道德經序訣》二卷



 韓莊《玄旨》八卷



 劉遺民《玄譜》一卷



 《節解》二卷



 《章門》一卷



 李軌《老子音》一卷



 《鶡冠子》三卷



 張湛注《列子》八卷列禦寇。



 郭象注《莊子》十卷莊周。



 向秀《注》二十卷



 崔譔《注》十卷



 司馬彪《注》二十一卷



 又《注音》一卷



 李頤《集解》二十卷



 王玄古《集解》二十卷



 李充《釋莊子論》二卷



 馮廓《老子指歸》十三卷



 又《莊子古今正義》十卷



 梁簡文帝《講疏》三十卷



 王穆《疏》十卷



 又《音》一卷



 《莊子疏》七卷



 《文子》十二卷



 《廣成子》十二卷商洛公撰,張太衡注。



 《唐子》十卷唐滂。



 《蘇子》七卷蘇彥。



 《宣子》二卷宣聘。



 《陸子》十卷陸雲。



 《抱樸子內篇》二十卷葛洪。



 《孫子》十二卷孫綽。



 《符子》三十卷符朗。



 《賀子》十卷賀道養。



 《牟子》二卷牟融。



 傅弈注《老子》二卷



 楊上善注《老子道德經》二卷



 又注《莊子》十卷



 《老子指略論》二卷太子文學。



 闢閭仁住注《老子》二卷聖歷司禮博士。



 賈大隱《老子述義》十卷



 陸德明《莊子文句義》二十卷



 玄宗注《道德經》二卷



 又《疏》八卷天寶中加號《玄通道德經》,世不稱之。



 盧藏用注《老子》二卷



 又注《莊子內外篇》十二卷



 邢南和注《老子》開元二十一年上。



 馮朝隱注《老子》



 白履忠注《老子》



 李播注《老子》



 尹知章注《老子》



 傅弈《老子音義》並卷亡。



 陸德明《老子疏》十五卷



 逄行珪注《鬻子》一卷鄭縣尉。



 陳庭玉《老子疏》開元二十年上,授校書郎。卷亡。



 陸希聲《道德經傳》四卷



 吳善經注《道德經》二卷貞元中人。



 楊上善《道德經三略論》三卷



 道士成玄英注《老子道德經》二卷



 又《開題序訣義疏》七卷



 注《莊子》三十卷



 《疏》十二卷玄英,字子實,陜州人,隱居東海。貞觀五年,召至京師。永徽中,流鬱州。書成,道王元慶遣文學賈鼎就授大義,嵩高山人李利涉為序,唯《老子注》、《莊子疏》著錄。



 張游朝《南華象罔說》十卷



 又《沖虛白馬非馬證》八卷張志和父。



 孫思邈注《老子》卷亡。



 又注《莊子》



 柳縱注《莊子》開元二十年上,授章懷太子廟丞。



 尹知章注《莊子》並卷亡。



 甘暉、魏包注《莊子》卷亡。開元末奉詔注。



 元載《南華通微》十卷



 張志和《太易》十五卷



 又《玄真子》十二卷韋詣作《內解》。



 陳庭玉《莊子疏》卷亡。



 道士李含光《老子莊子周易學記》三卷



 又《義略》三卷含光,揚州江都人,本姓弘,避孝敬皇帝諱改焉,天寶間人。



 張隱居《莊子指要》三十三篇名九垓,號渾淪子,代、德時人。



 帥夜光《三玄異義》三十卷幽州人。開元二十年上,授校書郎,直國子監。



 徐靈府注《文子》十二卷



 李暹訓注《文子》十二卷



 王士元《亢倉子》二卷天寶元年,詔號《莊子》為《南華真經》,《列子》為《沖虛真經》,《文子》為《通玄真經》,《亢桑子》為《洞靈真經》。然《亢桑子》求之不獲,襄陽處士王士元謂:「《莊子》作『庚桑子』。太史公、《列子》作『亢倉子』,其實一也。」取諸子文義類者補其亡。



 《無能子》三卷不著人名氏,光啟中隱民間。



 凡神仙三十五家,五十部,三百四十一卷。失姓名十三家,自《道藏音義》以下不著錄六十二家,二百六十五卷。



 尹喜《高士老君內傳》三卷



 玄景先生《老子道德簡要義》五卷



 梁簡文帝《老子私記》十卷



 戴詵《老子西升經義》一卷



 韋處玄集解《老子西升經》二卷



 《老子黃庭經》一卷



 《老子探真經》一卷



 《老君科律》一卷



 《老子宣時誡》一卷



 《老子入室經》一卷



 《老子華蓋觀天訣》一卷



 《老子消水經》一卷



 《老子神策百二十條經》一卷



 鬼谷先生《關令尹喜傳》一卷四皓注。



 《清虛真人王君內傳》一卷



 王萇《三天法師張君內傳》一卷



 李遵《茅君內傳》一卷



 呂先生《太極左仙公葛君內傳》一卷



 華嶠《紫陽真人周君傳》一卷



 趙昇等《仙人馬君陰君內傳》一卷



 鄭云千《清虛真人裴君內傳》一卷



 範邈《紫虛元君南嶽夫人內傳》一卷



 項宗《紫虛元君魏夫人內傳》一卷



 王羲之《許先生傳》一卷



 《九華真妃內記》一卷



 宋都能《嵩高少室寇天師傳》三卷



 《王喬傳》一卷



 《漢武帝傳》二卷



 劉向《列仙傳》二卷



 葛洪《神仙傳》十卷



 見素子《洞仙傳》十卷



 東方朔《神異經》二卷張華注。



 又《十洲記》一卷



 周季通《蘇君記》一卷



 梁曠《南華仙人莊子論》三十卷



 《南華真人道德論》三卷



 《任子道論》十卷任嘏。



 《顧道士論》三卷顧穀。



 姖威《渾輿經》一卷



 杜夷《幽求子》三十卷



 張譏《玄書通義》十卷



 陶弘景《登真隱訣》二十五卷



 又《真誥》十卷



 張湛《養生要集》十卷



 《養性傳》二卷



 張太衡《無名子》一卷



 劉道人《老子玄譜》一卷



 劉無待《同光子》八卷侯儼注。



 《靈人辛玄子自序》一卷



 《華陽子自序》一卷茅處玄。



 《無上秘要》七十二卷



 《道要》三十卷



 馬樞《學道傳》二十卷



 郭憲《漢武帝別國洞冥記》四卷



 《道藏音義目錄》一百一十三卷崔湜、薛稷、沈佺期、道士史崇玄等撰。



 《集注陰符經》一卷太公、範蠡、鬼谷子、張良、諸葛亮、李淳風、李筌、李洽、李鑒、李銳、楊晟。



 李靖《陰符機》一卷



 道士李少卿《十異九迷論》一卷



 道士劉進喜《老子通諸論》一卷



 又《顯正論》一卷



 張果《陰符經太無傳》一卷



 又《陰符經辨命論》一卷



 《氣訣》一卷



 《神仙得道靈藥經》一卷



 《罔象成名圖》一卷



 《丹砂訣》一卷開元二十二年上。



 韋弘《陰符經正卷》一卷



 李筌《驪山母傳陰符玄義》一卷筌,號少室山達觀子,於嵩山虎口巖石壁得《黃帝陰符》本,題云:「魏道士寇謙之傳諸名山。」筌至驪山,老母傳其說。



 葉靜能《太上北帝靈文》三卷



 李淳風注《泰乾秘要》三卷



 楊上器注《太上玄元皇帝聖紀》十卷



 崔少元《老子心鏡》一卷



 《皇天原太上老君現跡記》一卷



 文明元年老子降事。



 《呂氏老子昌言》二卷



 王方慶《神仙後傳》十卷



 《玄晉蘇元明太清石壁記》三卷乾元中,劍州司馬纂,失名。



 《議化胡經狀》一卷萬歲通天元年,僧惠澄上言乞毀《老子化胡經》,敕秋官侍郎劉如璿等議狀。



 《寧州通真觀二十七宿真形圖贊》一卷記天寶中,寧州羅川縣金華洞獲玉像,皆列宿之真,唯少氐宿,改縣為寧真事。



 道士令狐見堯《正一真人二十四治圖》一卷貞元人。



 孫思邈《馬陰二君內傳》一卷



 又《太清真人煉雲母訣》二卷



 《攝生真錄》一卷



 《養生要錄》一卷



 《氣訣》一卷



 《燒煉秘訣》一卷



 《龍虎通元訣》一卷



 《龍虎亂日篇》一卷



 《幽傳福壽論》一卷



 《枕中素書》一卷



 《會三教論》一卷



 《龍虎篇》一卷青羅子周希彭、少室山人孺登同注。



 硃少陽《道引錄》三卷浮山隱士,代、德時人。



 張志和《玄真子》二卷



 戴簡《真教元符》三卷



 楊嗣復《九徵心戒》一卷



 裴煜《延壽赤書》一卷



 紇干巘《序通解錄》一卷字咸一,大中江西觀察使。



 《守真子秦鑒語》一卷



 道士張仙庭《三洞瓊綱》三卷



 段世貴《演正一炁化圖》三卷



 女子胡愔《黃庭內景圖》一卷



 道士司馬承禎《坐忘論》一卷



 又《脩生養氣訣》一卷



 《洞元靈寶五嶽名山朝儀經》一卷



 賈參寥《莊子通真論》三卷垂拱中,隱武陵。



 白履忠注《黃庭內景經》卷亡。



 又《三玄精辨論》一卷



 吳筠《神仙可學論》一卷



 又《玄綱論》三卷



 《明真辨偽論》一卷



 《輔正除邪論》一卷



 《辨方正惑論》一卷



 《道釋優劣論》一卷



 《心目論》一卷



 《復淳化論》一卷



 《著生論》一卷



 《形神可固論》一卷



 李延章集《鄭綽錄中元論》一卷



 大和人。



 施肩吾《辨疑論》一卷



 睦州人,元和進士第,隱洪州西山。



 道士令狐見堯《玉笥山記》一卷



 道士李沖昭《南嶽小錄》一卷



 沈汾《續神仙傳》三卷



 道士胡慧超《神仙內傳》一卷



 《晉洪州西山十二真君內傳》一卷



 李渤《真系傳》一卷



 李遵《茅三君內傳》一卷



 道士胡法超《許遜脩行傳》一卷



 張說《洪崖先生傳》一卷



 張氳先生,唐初人。



 沖虛子《胡慧超傳》一卷失名。慧超,高宗時道士。



 《潘尊師傅》一卷師正。



 《蔡尊師傅》一卷名南玉,字叔寶,宋祠部尚書廓七世孫,歷金部員外郎,棄官入道。大歷中卒。



 劉谷神《葉法善傳》二卷



 正元師《謫仙崔少元傳》二卷



 陰日用《傅仙宗行記》一卷仙宗,開元資陽道士。



 謝良嗣《吳天師內傳》一卷吳筠。



 溫造《瞿童述》一卷大歷辰溪童子瞿柏庭升仙,造為朗州刺史,追述其事。



 李堅《東極真人傳》一卷果州謝自然。



 江積《八仙傳》一卷大中後事。



 王仲丘《攝生纂錄》一卷



 高福《攝生錄》三卷



 郭霽《攝生經》一卷



 上官翼《養生經》一卷



 康仲熊《服內元氣訣》一卷



 《氣經新舊服法》三卷



 《康真人氣訣》一卷



 《太元先生炁訣》一卷失名。大歷中,遇羅浮王公傳氣術。



 《菩提達磨胎息訣》一卷



 李林甫《唐朝煉大丹感應頌》一卷



 崔元真《靈沙受氣用藥訣》一卷



 又《雲母論》二卷天寶隱岷山。



 劉知古《日月元樞》一卷



 海蟾子《元英還金篇》一卷



 還陽子《太還丹金虎白龍論》一卷隱士,失姓名。



 陳少微《大洞煉真寶經脩伏丹砂妙訣》一卷



 嚴靜《大丹至論》一卷



 凡釋氏二十五家,四十部,三百九十五卷。失姓名一家,玄琬以下不著錄七十四家,九百四十一卷。



 蕭子良《凈注子》二十卷王融頌。



 僧僧祐《法苑集》十五卷



 又《弘明集》十四卷



 《釋迦譜》十卷



 《薩婆多師資傳》四卷



 虞孝敬《高僧傳》六卷



 又《內典博要》三十卷



 僧賢明《真言要集》十卷



 郭瑜《脩多羅法門》二十卷



 駱子義《經論纂要》十卷



 顧歡《夷夏論》二卷



 甄鸞《笑道論》三卷



 衛元嵩《齊三教論》七卷



 杜乂《甄正論》三卷



 李思慎《心鏡論》十卷



 裴子野《名僧錄》十五卷



 僧寶唱《名僧傳》二十卷



 又《比丘尼傳》四卷



 僧惠皎《高僧傳》十四卷



 僧道宗《續高僧傳》三十二卷



 陶弘景《草堂法師傳》一卷



 蕭回理《草堂法師傳》一卷



 《稠禪師傳》一卷



 陽衒之《洛陽伽藍記》五卷



 費長房《歷代三寶記》三卷長房,成都人,隋翻經學士。



 僧彥琮《崇正論》六卷



 又集《沙門不拜俗議》六卷



 《福田論》一卷



 道宣《統略凈住子》二卷



 又《通惑決疑錄》二卷



 《廣弘明集》三十卷



 《集古今佛道論衡》四卷



 《續高僧傳》二十卷起梁初,盡貞觀十九年。



 《後集續高僧傳》十卷



 《東夏三寶感通錄》三卷



 《大唐貞觀內典錄》十卷



 義凈《大唐西域求法高僧傳》二卷



 法琳《辯正論》八卷陳子良注。



 又《破邪論》二卷琳,姓陳氏。太史令傅弈請廢佛法,琳諍之,放死蜀中。



 復禮《十門辨惑論》二卷永隆二年,答太子文學權無二《釋典稽疑》。



 楊上善《六趣論》六卷



 又《三教銓衡》十卷



 僧玄琬《佛教後代國王賞罰三寶法》一卷



 又《安養蒼生論》一卷



 《三德論》一卷姓楊氏,新豐人。貞觀十年上。



 《入道方便門》二卷



 《眾經目錄》五卷



 《鏡諭論》一卷



 《無礙緣起》一卷



 《十種讀經儀》一卷



 《無盡藏儀》一卷



 《發戒緣起》二卷



 《法界僧圖》一卷



 《十不論》一卷



 《懺悔罪法》一卷



 《禮佛儀式》二卷



 李師政《內德論》一卷上黨人,貞觀門下典儀。



 僧法云《辨量三教論》三卷



 又《十王正業論》十卷絳州人。



 道宣又撰《注戒本》二卷



 《疏記》四卷



 《注羯磨》二卷



 《疏記》四卷



 《行事刪補律儀》三卷或六卷。



 釋門正行懺悔儀》三卷



 《釋門亡物輕重儀》二卷



 《釋門章服儀》二卷



 《釋門歸敬儀》二卷



 《釋門護法儀》二卷



 《釋氏譜略》二卷



 《聖跡見在圖贊》二卷



 《佛化東漸圖贊》二卷



 《釋迦方志》二卷



 僧彥琮《大唐京寺錄傳》十卷



 又《沙門不敬錄》六卷龍朔人,並隋有二彥琮。



 玄應《大唐眾經音義》二十五卷



 玄惲《敬福論》十卷



 又《略論》二卷



 《大小乘觀門》十卷



 《法苑珠林集》一百卷



 《四分律僧尼討要略》五卷



 《金剛般若經集注》三卷



 《百願文》一卷玄惲,本名道世。



 玄範注《金剛般若經》一卷



 又注《二帝三藏聖教序》一卷太宗、高宗。



 慧覺《華嚴十地維摩纘義章》十三卷姓範氏,武德人。



 行友《己知沙門傳》一卷序僧海順事。



 道岳《三藏本疏》二十二卷姓孟氏,河陽人,貞觀中。



 道基《雜心玄章並鈔》八卷



 又《大乘章鈔》八卷姓呂氏,東平人,貞觀時。



 智正《華嚴疏》十卷姓白氏,安喜人,貞觀中。



 慧凈《雜心玄文》三十卷姓房,隋國子博士徽遠從子。



 又《俱舍論文疏》三十卷



 《大莊嚴論文疏》三十卷



 《法華經纘述》十卷



 那提《大乘集議論》四十卷



 《釋疑論》一卷



 《注金剛般若經》一卷



 《諸經講序》一卷



 玄會《義源文本》四卷



 又《時文釋鈔》四卷



 《涅盤義章句》四卷字懷默,姓席氏,安定人,貞觀中。



 慧休《雜心玄章鈔疏》卷亡。姓樂氏,瀛州人。



 靈潤《涅般義疏》十三卷



 又《玄章》三卷



 《遍攝大乘論義鈔》十三卷



 《玄章》三卷姓梁氏,虞鄉人。



 辯相《攝論疏》五卷辯相,居凈影寺。



 玄裝《大唐西域記》十二卷姓陳氏,緱氏人。



 辯機《西域記》十二卷



 清徹《金陵塔寺記》三十六卷



 師哲《前代國王修行記》五卷盡中宗時。



 《大唐內典錄》十卷西明寺僧撰。



 毋煚《開元內外經錄》十卷道、釋書二千五百餘部,九千五百餘卷。



 智矩《寶林傳》十卷



 法常《攝論義疏》八卷



 又《玄章》五卷姓張氏,南陽人,貞觀末。



 慧能《金剛般若經口訣正義》一卷姓盧氏,曲江人。



 僧灌頂《私記天臺智者詞旨》一卷



 又《義記》一卷字法云,姓吳氏,章安人。



 道綽《凈土論》二卷姓衛氏,並州文水人。



 道綽《行圖》一卷



 智首《五部區分鈔》二十一卷姓皇甫氏。



 法礪《四分疏》十卷



 又《羯磨疏》三卷



 《舍懺儀》一卷



 《輕重儀》一卷姓李氏,趙郡人。



 慧滿《四分律疏》二十卷姓梁氏,京兆長安人。



 慧旻《十誦私記》十三卷



 又《僧尼行事》三卷



 《尼眾竭磨》二卷



 《菩薩戒義疏》四卷字玄素,河東人。



 空藏《大乘要句》三卷姓王氏,新豐人。



 道宗《續高僧傳》三十二卷



 玄宗注《金剛般若經》一卷



 道氤《御注金剛般若經疏宣演》三卷



 《高僧嬾殘傳》一卷天寶人。



 元偉《真門聖胄集》五卷



 僧法海《六祖法寶記》一卷



 辛崇《僧伽行狀》一卷



 神楷《維摩經疏》六卷



 靈湍《攝山棲霞寺記》一卷



 《破胡集》一卷會昌沙汰佛法詔敕。



 法藏《起信論疏》二卷



 《法琳別傳》二卷



 《大唐京師寺錄》卷亡。



 玄覺《永嘉集》十卷慶州刺史魏靖編次。



 懷海《禪門規式》一卷



 希運《傳心法要》一卷裴休集。



 玄嶷《甄正論》三卷



 光瑤注《僧肇論》二卷



 李繁《玄聖蘧廬》一卷



 白居易《八漸通真議》一卷



 《七科義狀》一卷雲南國使段立之問,僧悟達答。



 《棲賢法雋》一卷僧惠明與西川節度判官鄭愚、漢州刺史趙璘論怫書。



 《禪關八問》一卷楊士達問,唐宗美對。



 僧一行《釋氏系錄》一卷



 宗密《禪源諸詮集》一百一卷



 又《起信論》二卷



 《起信論鈔》三卷



 《原人論》一卷



 《圓覺經大小疏鈔》各一卷



 楚南《般若經品頌偈》一卷



 又《破邪論》一卷大順中人。



 希還《參同契》一卷



 良價《大乘經要》一卷



 又《激勵道俗頌偈》一卷



 光仁《四大頌》一卷



 又《略華嚴長者論》一卷



 無殷《垂誡》十卷



 神清《參元語錄》十卷



 智月《僧美》三卷



 惠可《達摩血脈》一卷



 靖邁《古今譯經圖紀》四卷



 智昇《續古今譯經圖紀》一卷



 又《續大唐內典錄》一卷



 《續古今佛道論衡》一卷



 《對寒山子詩》七卷天臺隱士。臺州刺史閭丘胤序,僧道翹集。寒山子隱唐興縣寒山巖,於國清寺與隱者拾得往還。



 龐蘊《詩偈》三卷字道玄,衡州衡陽人,貞元初人,三百餘篇。



 智閑《偈頌》一卷二百餘篇。



 李吉甫《一行傳》一卷



 王彥威《內典目錄》十二卷



 右道家類一百三十七家,七十四部,一千二百四十卷。失姓名三家,玄宗以下不著錄一百五十八家,一千三百三十八卷。



 總一百三十七家,一百七十四部。



 《管子》十九卷管仲。



 《商君書》五卷商鞅。或作《商子》。



 《慎子》十卷慎到撰,滕輔注。



 《申子》三卷申不害。



 《韓子》二十卷韓非。



 《晁氏新書》七卷晁錯。



 董仲舒《春秋決獄》十卷黃氏正。



 《崔氏政論》六卷崔寔。



 《劉氏法論》五卷劉廙。



 《阮子政論》五卷阮武。



 《劉氏政論》十卷劉邵。



 《桓氏世要論》十二卷桓範。



 《陳子要言》十四卷陳融。



 李文博《治道集》十卷



 邯鄲綽《五經析疑》三十卷



 尹知章注《管子》三十卷



 又注《韓子》卷亡。



 杜佑《管氏指略》二卷



 李敬玄《正論》三卷



 右法家類十五家,十五部,一百六十六卷。尹知章以下不著錄三家,三十五卷。



 《鄧析子》一卷



 《尹文子》一卷



 《公孫龍子》三卷



 陳嗣古注《公孫龍子》一卷



 劉邵《人物志》三卷



 劉炳注《人物志》三卷



 姚信《士緯》十卷



 魏文帝《士操》一卷



 盧毓《九州人士論》一卷



 範謐《辨名苑》十卷



 僧遠年《兼名苑》二十卷



 賈大隱注《公孫龍子》一卷



 趙武孟《河西人物志》十卷



 杜周士《廣人物志》三卷



 宋璲《吳興人物志》十卷字勝之,吳興烏程人,大中時。右名家類十二家,十二部,五十五卷。趙武孟以下不著錄三家,二十三卷。



 《墨子》十五卷墨翟。



 《隨巢子》一卷



 《胡非子》一卷



 右墨家類三家,三部,一十七卷。



 《鬼谷子》二卷蘇秦。



 樂臺注《鬼谷子》三卷



 梁元帝《補闕子》十卷



 尹知章注《鬼谷子》三卷



 右縱橫家類四家,四部,一十五卷。尹知章不著錄。



 《尉繚子》六卷



 《尸子》十卷尸佼。



 《呂氏春秋》二十六卷呂不韋撰,高誘注。



 許慎注《淮南子》二十一卷淮南王劉安。



 高誘注《淮南子》二十一卷



 又《淮南鴻烈音》二卷



 嚴尤《三將軍論》一卷



 王充《論衡》三十卷



 應邵《風俗通義》三十卷



 《蔣子萬機論》十卷蔣濟。



 杜恕《篤論》四卷



 鐘會《芻蕘論》五卷



 《傅子》一百二十卷傅玄。



 張儼《默記》三卷



 又《誓論》三十卷



 裴玄《新言》五卷



 蘇道《立言》十卷



 劉欽《新義》十八卷



 《秦子》三卷秦菁。



 張明《折言論》二十卷



 《古訓》十卷



 孔衍《說林》五卷



 《抱樸子外篇》二十卷葛洪。



 楊偉《時務論》十二卷



 範泰《古今善言》三十卷



 徐益壽《記聞》三卷



 《何子》五卷何楷。



 《劉子》十卷劉勰。



 梁元帝《金樓子》十卷



 硃澹遠《語麗》十卷



 又《語對》十卷



 《張公雜記》一卷張華。



 陸士衡《要覽》三卷



 郭義恭《廣志》二卷



 崔豹《古今注》三卷



 伏侯《古今注》三卷



 江邃《釋文》十卷



 盧辯《稱謂》五卷



 謝昊《物始》十卷



 任昉《文章始》一卷張績補。



 姚察《續文章始》一卷



 庾肩吾《採璧》三卷



 韋道孫《新略》十卷



 徐陵《名數》十卷



 沈約《袖中記》二卷



 範謐《典墳數集》十卷



 侯亶《祥瑞圖》八卷



 孟眾《張掖郡玄石圖》一卷



 高堂隆《張掖郡玄石圖》一卷



 孫柔之《瑞應圖記》三卷



 熊理《瑞應圖贊》三卷



 顧野王《符瑞圖》十卷



 又《祥瑞圖》十卷



 王劭《皇隋靈感志》十卷



 許善心《皇隋瑞文》十四卷



 何望之《諫林》十卷



 虞通之《善諫》二卷



 孟儀《子林》二十卷



 沈約《子鈔》三十卷



 庾仲容《子鈔》三十卷



 殷仲堪《論集》九十六卷



 崔宏《帝王集要》三十卷



 陸澄《述正論》十三卷



 又《缺文》十卷



 徐除《文府》七卷宗道寧注。



 劉守敬《四部言心》十卷



 《新舊傳》四卷



 《古今辨作錄》三卷



 《博覽》十五卷



 《部略》十五卷



 《翰墨林》十卷



 魏征《群書治要》五十卷



 《麟閣詞英》六十卷高宗時敕撰。



 硃敬則《十代興亡論》十卷



 薛克構《子林》三十卷



 虞世南《帝王略論》五卷



 劉伯莊《群書治要音》五卷



 張大素《說林》二十卷



 王方慶《續世說新書》十卷



 韓潭《統載》三十卷夏綏銀節度使。貞元十三年上。



 熊執易《化統》五百卷執易類九經為書,三十年乃成,未及上,卒於西川,武元衡將為寫進,妻薛藏之不許。



 李文成《博雅志》十三卷安國公興貴子。



 元懷景《屬文要義》十卷



 崔玄韋《行己要範》十卷



 盧藏用《子書要略》一卷



 馬亹《意林》三卷



 《魏氏手略》二十卷魏謩。



 辛之諤《敘訓》二卷開元十七年上,授長社尉。



 《博聞奇要》二十卷開元武功縣人徐闉上,詔試文章,留集賢院校理。



 周蒙《續古今注》三卷



 薛洪《古今精義》十五卷



 趙蕤《長短要術》十卷字太賓,梓州人。開元,召之不赴。



 杜佑《理道要訣》十卷



 賀蘭正元《用人權衡》十卷貞元十三年上。



 樊宗師《魁紀公》三十卷



 又《樊子》三十卷



 郭昭度《治書》十卷



 硃樸《致理書》十卷



 蘇源《治亂集》三卷唐末人。



 張薦《江左寓居錄》卷亡。



 張楚金《紳誡》三卷



 馮伉《諭蒙》一卷



 庾敬休《諭善錄》七卷



 蕭佚《牧宰政術》二卷耒陽令。



 魯人初《公侯政術》十卷魯人名初不著姓,大中人。



 李知保《檢志》三卷代宗信州司倉參軍。



 王範《續蒙求》三卷



 白廷翰《唐蒙求》三卷廣明人。



 李伉《系蒙》二卷



 盧景亮《三足記》二卷



 右雜家類六十四家,七十五部,一千一百三卷。失姓名六家,虞世南以下不著錄三十四家,八百一十六卷。



 《範子計然》十五卷範蠡問,計然答。



 《尹都尉書》三卷



 《汜勝之書》二卷



 崔寔《四民月令》一卷



 賈思勰《齊民要術》十卷



 宗懍《荊楚歲時記》一卷



 杜公贍《荊楚歲時記》二卷



 杜臺卿《玉燭寶典》十二卷



 王氏《四時錄》十二卷



 戴凱之《竹譜》一卷



 顧烜《錢譜》一卷



 浮丘公《相鶴經》一卷



 堯須跋《鷙擊錄》二十卷



 《相馬經》三卷



 伯樂《相馬經》一卷



 徐成等《相馬經》二卷



 諸葛潁《種植法》七十七卷



 又《相馬經》六十卷



 甯戚《相牛經》一卷



 範蠡《養魚經》一卷



 《禁苑實錄》一卷



 《鷹經》一卷



 《蠶經》一卷



 又二卷



 《相貝經》一卷



 武後《兆人本業》三卷



 王方慶《園庭草木疏》二十一卷



 《孫氏千金月令》三卷孫思邈。



 李淳風《演齊民要術》卷亡。



 李邕《金穀園記》一卷



 薛登《四時記》二十卷



 裴澄《乘輿月令》二十卷國子司業。貞元十一年上。



 王涯《月令圖》一軸



 李綽《秦中歲時記》一卷



 韋行規《保生月錄》一卷



 韓鄂《四時纂要》五卷



 《歲華紀麗》二卷



 右農家類十九家,二十六部,二百三十五卷。失姓名六家,王方慶以下不著錄十一家,六十六卷。



 《燕丹子》一卷燕太子。



 邯鄲淳《笑林》三卷



 裴子野《類林》三卷



 張華《博物志》十卷



 又《列異傳》一卷



 賈泉注《郭子》三卷郭澄之。



 劉義慶《世說》八卷



 又《小說》十卷



 劉孝標《續世說》十卷



 殷蕓《小說》十卷



 劉齊《釋俗語》八卷



 蕭賁《辨林》二十卷



 劉炫《酒孝經》一卷



 庾元威《座右方》三卷



 侯白《啟彥錄》十卷



 《雜語》五卷



 戴祚《甄異傳》三卷



 袁王壽《古異傳》三卷



 祖沖之《述異記》十卷



 劉質《近異錄》二卷



 干寶《搜神記》三十卷



 劉之遴《神錄》五卷



 梁元帝《妍神記》十卷



 祖臺之《志怪》四卷



 孔氏《志怪》四卷



 荀氏《靈鬼志》三卷



 謝氏《鬼神列傳》二卷



 劉義慶《幽明錄》三十卷



 東陽無疑《齊諧記》七卷



 吳均《續齊諧記》一卷



 王延秀《感應傳》八卷



 陸果《系應驗記》一卷



 王琰《冥祥記》一卷



 王曼潁《續冥祥記》十一卷



 劉泳《因果記》十卷



 顏之推《冤魂志》三卷



 又《集靈記》十卷



 《征應集》二卷



 侯君素《旌異記》十五卷



 唐臨《冥報記》二卷



 李恕《誡子拾遺》四卷



 《開元禦集誡子書》一卷



 王方慶《王氏神通記》十卷



 狄仁傑《家範》一卷



 《盧公家範》一卷盧僎。



 蘇瑰《中樞龜鏡》一卷



 姚元崇《六誡》一卷



 《事始》三卷劉孝孫、房德懋。



 劉睿《續事始》三卷



 元結《猗犴子》一卷



 趙自勔《造化權輿》六卷



 《通微子十物志》一卷



 吳筠《兩同書》一卷



 李涪《刊誤》二卷



 李匡文《資暇》三卷



 《炙轂子雜錄注解》五卷王叡。



 蘇鶚《演義》十卷



 又《杜陽雜編》三卷字德祥,光啟中進士第。



 《柳氏家學要錄》二卷柳珵。



 盧光啟《初舉子》一卷字子忠,相昭宗。



 劉訥言《俳諧集》十五卷



 陳翱《卓異記》一卷憲、穆時人。



 裴紫芝《續卓異記》一卷



 薛用弱《集異記》三卷字中勝,長慶光州刺史。



 李玫《纂異記》一卷大中時人。



 李亢《獨異志》十卷



 谷神子《博異志》三卷



 沈如筠《異物志》三卷



 《古異記》一卷



 劉餗《傳記》三卷一作《國史異纂》。



 牛肅《紀聞》十卷



 陳鴻《開元升平源》一卷字大亮,貞元主客郎中。



 張薦《靈怪集》二卷



 陸長源《辨疑志》三卷



 李繁《說纂》四卷



 戴少平《還魂記》一卷貞元待詔。



 牛僧孺《玄怪錄》十卷



 李復言《續玄怪錄》五卷



 陳翰《異聞集》十卷唐末屯田員外郎。



 鄭遂《洽聞記》一卷



 鐘輅《前定錄》一卷



 趙自勤《定命論》十卷天寶秘書監。



 呂道生《定命錄》二卷大和中,道生增趙自勤之說。



 溫畬《續定命錄》一卷



 胡璩《譚賓錄》十卷字子溫,文、武時人。



 韋絢《劉公嘉話錄》一卷絢,字文明,執誼子也,咸通義武軍節度使。劉公,禹錫也。



 《戎幕閑談》一卷



 趙璘《因話錄》六卷字澤章,大中衢州刺史。



 袁郊《甘澤謠》一卷



 溫庭筠《乾巽子》三卷



 又《採茶錄》一卷



 段成式《酉陽雜俎》三十卷



 《廬陵官下記》二卷



 康軿《劇談錄》三卷字駕言,乾符進士第。



 高彥休《闕史》三卷



 《盧子史錄》卷亡。



 又《逸史》三卷大中時人。



 李隱《大唐奇事記》十卷咸通中人。



 陳劭《通幽記》一卷



 範攄《雲溪友議》三卷咸通時,自稱五雲溪人。



 李躍《嵐齋集》二十五卷



 尉遲樞《南楚新聞》三卷並唐末人。



 張固《幽閑鼓吹》一卷



 《常侍言旨》一卷柳珵。



 《盧氏雜說》一卷



 《桂苑叢譚》一卷馮翊子子休。



 《樹萱錄》一卷



 《會昌解頤》四卷



 《松窗錄》一卷



 《芝田錄》一卷



 《玉泉子見聞真錄》五卷



 張讀《宣室志》十卷



 柳祥《瀟湘錄》十卷



 皇甫松《醉鄉日月》三卷



 何自然《笑林》三卷



 焦璐《窮神秘苑》十卷



 裴鉶《傳奇》三卷高駢從事。



 劉軻《牛羊日歷》一卷牛僧孺、楊虞卿事。檀欒子皇甫松序。



 《補江總白猿傳》一卷



 郭良輔《武孝經》一卷



 陸羽《茶經》三卷



 張又新《煎茶水記》一卷



 封演《續錢譜》一卷



 右小說家類三十九家,四十一部,三百八卷。失姓名二家,李恕以下不著錄七十八家,三百二十七卷。



 趙嬰注《周髀》一卷



 甄鸞注《周髀》一卷



 張衡《靈憲圖》一卷



 又《渾天儀》一卷



 王蕃《渾天象注》一卷



 姚信《昕天論》一卷



 《石氏星經簿贊》一卷石申。



 虞喜《安天論》一卷



 《甘氏四七法》一卷甘德。



 劉表《荊州星占》二卷



 劉叡《荊州星占》二十卷



 《天文集占》三卷



 祖恆之《天文錄》三十卷



 韓楊《天文要集》四十卷



 高文洪《天文橫圖》一卷



 吳雲《天文雜占》一卷



 陳卓《四方星占》一卷



 又《五星占》一卷



 《天文集占》七卷



 孫僧化等《星占》三十三卷



 史崇《十二次二十八宿星占》十二卷



 庾季才《靈臺秘苑》一百二十卷



 逄行珪《玄機內事》七卷



 《論二十八宿度數》一卷



 《五星兵法》一卷



 《黃道略星占》一卷



 《孝經內記星圖》一卷



 《周易分野星國》一卷



 李淳風釋《周髀》二卷



 又《乙巳占》十二卷



 《天文占》一卷



 《大象元文》一卷



 《乾坤秘奧》七卷



 《法象志》七卷



 《太白會運逆兆通代記圖》一卷淳風與袁天綱集。



 武密《古今通占鏡》三十卷



 《大唐開元占經》一百一十卷瞿曇悉達集。



 董和《通乾論》十五卷和,本名純,避憲宗名改。善歷算。裴胄為荊南節度,館之,著是書云。



 《長慶算五星所在宿度圖》一卷



 司天少監徐升。



 黃冠子李播《天文大象賦》一卷李臺集解。



 王希明《丹元子步天歌》一卷



 右天文類二十家,三十部,三百六卷。失姓名六家,李淳風《天文占》以下不著錄六家,一百七十五卷。



 劉向《九章重差》一卷



 徐岳《九章算術》九卷



 又《算經要用百法》一卷



 《數術記遺》一卷甄鸞注。



 張丘建《算經》一卷甄鸞注。



 董泉《三等數》一卷甄鸞注。



 夏侯陽《算經》一卷甄鸞注。



 甄鸞《九章算經》九卷



 又《五曹算經》五卷



 《七曜本起歷》五卷



 《七曜歷算》二卷



 《歷術》一卷



 韓延《夏侯陽算經》一卷



 又《五曹算經》五卷



 宋泉之《九經術疏》九卷



 劉徽《海島算經》一卷



 又《九章重差圖》一卷



 劉祐《九章雜算文》二卷



 陰景愉《七經算術通義》七卷



 信都芳《器準》三卷



 《黃鐘算法》四十卷



 劉歆《三統歷》一卷



 《四分歷》一卷



 《推漢書律歷志術》一卷



 劉洪《乾象歷術》三卷闞澤注。



 《乾象歷》三卷



 楊偉《魏景初歷》三卷



 何承天《宋元嘉歷》二卷



 又《刻漏經》一卷



 虞廣刂《梁大同歷》一卷



 吳伯善《陳七曜歷》五卷



 孫僧化《後魏永安歷》一卷



 李業興《後魏甲子歷》一卷



 《後魏武定歷》一卷



 宋景業《北齊天保歷》一卷



 《北齊甲子元歷》一卷



 王琛《周大象歷》二卷



 馬顯《周甲寅元歷》一卷



 《周甲子元歷》一卷



 劉孝孫《隋開皇歷》一卷



 又《七曜雜術》二卷



 李德林《隋開皇歷》一卷



 張胄玄《隋大業歷》一卷



 又《玄歷術》一卷



 《七曜歷疏》三卷



 劉焯《皇極歷》一卷



 趙匪又《河西壬辰元歷》一卷



 《河西甲寅元歷》一卷



 劉智《正歷》四卷薛夏訓。



 《姜氏歷術》三卷



 崔浩《律歷術》一卷



 《歷日義統》一卷



 《歷日吉兇注》一卷



 硃史《刻漏經》一卷



 宋景《刻漏經》一卷



 李淳風注《周髀算經》二卷



 又注《九章算術》九卷



 注《九章算經要略》一卷



 注《五經算術》二卷



 注《張丘建算經》三卷



 注《海島算經》一卷



 注《五曹孫子等算經》二十卷



 注《甄鸞孫子算經》三卷



 釋祖沖之《綴術》五卷



 《皇極歷》一卷



 傅仁均《大唐戊寅歷》一卷



 《唐麟德歷》一卷



 《麟德歷出生記》十卷



 王孝通《緝古算術》四卷太史丞李淳風注。



 《算經表序》一卷



 南宮說《光宅歷草》十卷



 瞿曇謙《大唐甲子元辰歷》一卷



 《大唐刻漏經》一卷



 王勃《千歲歷》卷亡。



 謝察微《算經》三卷



 江本《一位算法》二卷



 陳從運《得一算經》七卷



 魯靖《新集五曹時要術》三卷



 邢和璞《潁陽書》三卷隱潁陽石堂山。



 僧一行《開元大衍歷》一卷



 又《歷議》十卷



 《歷立成》十二卷



 《歷草》二十四卷



 《七政長歷》三卷



 《心機算術括》一卷黃棲巖注。



 《寶應五紀歷》四十卷



 《建中正元歷》二十八卷



 曹士蔿《七曜符天歷》一卷建中時人。



 《七曜符天人元歷》三卷



 龍受《算法》二卷貞元人。



 《長慶宣明歷》三十四卷



 《長慶宣明歷要略》一卷



 《宣明歷超捷例要略》一卷



 邊岡《景福崇玄歷》四十卷岡稱處士。



 《大衍通元鑒新歷》三卷貞元至大中。



 《大唐長歷》一卷



 《都利聿斯經》二卷貞元中,都利術士李彌乾傳自西天竺,有璩公者譯其文。



 陳輔《聿斯四門經》一卷



 右歷算類三十六家,七十五部,二百三十七卷失姓名五家,王勃以下不著錄十九家,二百二十六卷。



 《黃帝問玄女法》三卷



 《黃帝用兵法訣》一卷



 《黃帝兵法孤虛推記》一卷



 《黃帝太一兵歷》一卷



 《黃帝太公三宮法要訣》一卷



 《太公陰謀》三卷



 又《陰謀三十六用》一卷



 《金匱》二卷



 《六韜》六卷



 《當敵》一卷



 《周書陰符》九卷



 《周呂書》一卷



 田穰苴《司馬法》三卷



 魏武帝注《孫子》三卷



 又《續孫子兵法》二卷



 《兵書接要》七卷孫武。



 孟氏解《孫子》二卷



 沈友注《孫子》二卷



 賈詡注《吳子兵法》一卷吳起。



 《吳孫子三十二壘經》一卷



 伍子胥《兵法》一卷



 黃石公《三略》三卷



 又《陰謀乘斗魁剛行軍秘》一卷



 成氏《三略訓》三卷



 《張良經》一卷



 《張氏七篇》七卷張良



 魏文帝《兵書要略》十卷



 宋高祖《兵法要略》一卷



 司馬彪《兵記》十二卷



 孔衍《兵林》六卷



 葛洪《兵法孤虛月時秘要法》一卷



 梁武帝《兵法》一卷



 梁元帝《玉韜》十卷



 劉祐《金韜》十卷



 蕭吉《金海》四十七卷



 陶弘景《真人水鏡》十卷



 《握鏡》三卷



 王略《武林》一卷



 《許子新書軍勝》十卷



 樂產《王佐秘書》五卷



 後周齊王憲《兵書要略》十卷



 隋高祖《新撰兵書》三十卷



 解忠鯁《龍武玄兵圖》二卷



 《新兵法》二十四卷



 《用兵要術》一卷



 《太一兵法》一卷



 《兵法要訣》一卷



 《承神兵書》八卷



 《兵機》十五卷



 《兵書要略》十卷



 《用兵撮要》二卷



 《兵春秋》一卷



 《獸斗亭亭》一卷



 《玉帳經》一卷



 《三陰圖》一卷



 《兵法雲氣推占》一卷



 《武德圖五兵八陣法要》一卷



 李靖《六軍鏡》三卷



 員半千《臨戎孝經》二卷



 李淳風《縣鏡》十卷



 李筌注《孫子》二卷



 又《太白陰經》十卷



 《青囊括》一卷



 杜牧注《孫子》三卷



 陳皞注《孫子》一卷



 賈林注《孫子》一卷



 孫鐈注《吳子》一卷



 裴行儉《安置軍營行陣等四十六訣》一卷



 李嶠《軍謀前鑒》十卷



 郭元振《定遠安邊策》三卷



 吳兢《兵家正史》九卷



 李處祐《兵法》開元中左衛中郎將,奉詔撰。卷亡。



 鄭虔《天寶軍防錄》卷亡。



 劉秩《止戈記》七卷



 《至德新議》十二卷



 董承祖《至德元寶玉函經》十卷



 李光弼《統軍靈轄秘策》一卷一作《武記》



 裴守一《軍誡》三卷



 《裴子新令》二卷裴緒。



 韓滉《天事序議》一卷



 韋皋《開復西南夷事狀》十七卷



 範傳正《西陲要略》三卷



 王公亮《兵書》十八卷長慶元年上。商州刺史。



 《行師類要》七卷



 燕僧利正《長慶人事軍律》三卷



 李渤《御戎新錄》二十卷



 李德裕《西南備邊錄》十三卷



 杜希全《新集兵書要訣》三卷



 張道古《兵論》一卷字子美,景福進士第。



 右兵書類二十三家,六十部,三百一十九卷。失姓名十四家,李筌以下不著錄二十五家,一百六十三卷。



 史蘇《沈思經》一卷



 《焦氏周易林》十六卷焦贛。



 《京氏周易四時候》二卷京房。



 又《周易飛候》六卷



 《周易混沌》四卷



 《周易錯卦》八卷



 《逆刺》三卷



 《費氏周易逆刺占災異》十二卷費直。



 又《周易林》二卷



 《崔氏周易林》十六卷崔篆。



 鄭玄注《九宮行棋經》三卷



 管輅《周易林》四卷



 又《鳥情逆占》一卷



 張滿《周易林》七卷



 《許氏周易雜占》七卷許峻。



 尚廣《周易雜占》八卷



 《武氏周易雜占》八卷



 魏伯陽《周易參同契》二卷



 又《周易五相類》一卷



 《徐氏周易筮占》二十四卷徐苗。



 伏曼容《周易集林》十二卷



 伏氏《周易集林》一卷



 杜氏《新易林占》三卷



 梁運《周易雜占筮訣文》二卷



 虞翻《周易集林律歷》一卷



 郭璞《周易洞林解》三卷



 梁元帝《連山》三十卷



 又《洞林》三卷



 郭氏《易腦》一卷



 《周易立成占》六卷



 《易林》十四卷



 《周易新林》一卷



 《易律歷》一卷



 《周易服藥法》一卷



 《易三備》三卷



 又三卷



 《易髓》一卷



 《周易問》十卷



 《周易雜圖序》一卷



 《周易八卦斗內圖》一卷



 又三卷



 《周易內卦神筮法》二卷



 《周易雜筮占》四卷



 《老子神符易》一卷



 《孝經元辰》二卷



 《推元辰厄命》一卷



 《元辰章》三卷



 《元辰》一卷



 《雜元辰祿命》二卷



 《河祿命》二卷



 孫僧化《六甲開天歷》一卷



 翼奉《風角要候》一卷



 王琛《風角六情訣》一卷



 又《推產婦何時產法》一卷



 《九宮行棋立成》一卷



 《祿命書》二卷



 《遁甲開山圖》一卷



 劉孝恭《風角鳥情》二卷



 又《祿命書》二十卷



 《鳥情占》一卷



 《風角》十卷



 《九宮經角》三卷



 《婚嫁書》二卷



 《登壇經》一卷



 《太一大游歷》二卷



 《大游太一歷》一卷



 《曜靈經》一卷



 《七政歷》一卷



 《六壬歷》一卷



 《六壬擇非經》六卷



 《靈寶登圖》一卷



 梁主榮《光明符》十二卷



 《推二十四氣歷》一卷



 《太一歷》一卷



 曹氏《黃帝式經三十六用》一卷



 《玄女式經要訣》一卷



 董氏《大龍首式經》一卷



 《桓公式經》一卷



 宋琨《式經》一卷



 《六壬式經雜占》九卷



 《雷公式經》一卷



 《太一式經》二卷



 《太一式經雜占》十卷



 《黃帝式用常陽經》一卷



 《黃帝龍首經》二卷



 《黃帝集靈》三卷



 《黃帝降國》一卷



 《黃帝鬥歷》一卷



 《太史公萬歲歷》一卷司馬談。



 《萬歲歷祠》二卷



 任氏《千歲歷祠》二卷



 《舉百事要略》一卷



 張衡《黃帝飛鳥歷》一卷



 《太一飛鳥歷》一卷



 《太一九宮雜占》十卷



 《九宮經》三卷



 《堪輿歷注》二卷



 殷紹《黃帝四序堪輿》一卷



 《地節堪輿》二卷



 伍子胥《遁甲文》一卷



 信都芳《遁甲經》二卷



 葛洪《三元遁甲圖》三卷



 許昉《三元遁甲》六卷



 杜仲《三元遁甲》一卷



 榮氏《遁甲開山圖》二卷



 《遁甲經》十卷



 《遁甲囊中經》一卷



 《遁甲推要》一卷



 《遁甲秘要》一卷



 《遁甲九星歷》一卷



 《遁甲萬一訣》三卷



 《三元遁甲立成圖》二卷



 《遁甲立成法》三卷



 《遁甲九宮八門圖》一卷



 《遁甲三奇》三卷



 《陽遁甲》九卷



 《陰遁甲》九卷



 《遁甲三元九甲立成》一卷



 《白澤圖》一卷



 《武王須臾》二卷



 《師曠占書》一卷



 《東方朔占書》一卷



 《淮南王萬畢術》一卷



 樂產《神樞靈轄》十卷



 柳彥詢《龜經》三卷



 柳世隆《龜經》三卷



 劉寶真《龜經》一卷



 王弘禮《龜經》一卷



 莊道名《龜經》一卷



 蕭吉《五行記》五卷



 又《五姓宅經》二十卷



 《葬經》二卷



 王璨《新撰陰陽書》三十卷



 《青烏子》三卷



 《葬經》八卷



 又十卷



 《葬書地脈經》一卷



 《墓書五陰》一卷



 《雜墓圖》一卷



 《墓圖立成》一卷



 《六甲塚名雜忌要訣》二卷



 郭氏《五姓墓圖要訣》五卷



 《壇中伏尸》一卷



 胡君《玄女彈五音法相塚經》一卷



 《百怪書》一卷



 《祠灶經》一卷



 《解文》一卷



 周宣《占夢書》三卷



 又二卷



 孫思邈《龜經》一卷



 又《五兆算經》一卷



 《龜上五兆動搖經訣》一卷



 《福祿論》三卷



 李淳風《四民福祿論》三卷



 又《玄悟經》三卷



 《太一元鑒》五卷



 《占燈經》一卷



 《注鄭玄九旗飛變》一卷



 《三元經》一卷



 《太一樞會賦》一卷玄宗注。



 崔知悌《產圖》一卷



 呂才《陰陽書》五十三卷



 《廣濟陰陽百忌歷》一卷



 《大唐地理經》十卷貞觀中上。



 袁天綱《相書》七卷



 《要訣》三卷



 陳恭釗《天寶歷》二卷天寶中詔定。



 趙同珍《壇經》一卷



 黎幹《蓬瀛書》三卷



 賈躭《唐七聖歷》一卷



 李遠《龍紀聖異歷》一卷



 竇維鋈《廣古今五行記》三十卷



 濮陽夏樵子《五行志》五卷



 《祿命人元經》三卷



 楊龍光《推計祿命厄運詩》一卷



 王希明《太一金鏡式經》十卷開元中詔撰。



 僧一行《天一太一經》一卷



 又《遁甲十八局》一卷



 《太一局遁甲經》一卷



 《五音地理經》十五卷



 《六壬明鏡連珠歌》一卷



 《六壬髓經》三卷



 馬先《天寶太一靈應式記》五卷



 李鼎祚《連珠明鏡式經》十卷開耀中上之。



 蕭君靖《遁甲圖》開元僕寺主簿,奉詔撰。卷亡。



 司馬驤《遁甲符寶萬歲經國歷》一卷驤與弟裕同撰。



 曹士蔿《金匱經》三卷



 馬雄《絳囊經》一卷雄稱居士。



 李靖《玉帳經》一卷



 李筌《六壬大玉帳歌》十卷



 王叔政《推太歲行年吉兇厄》一卷



 由吾公裕《葬經》三卷



 孫季邕《葬範》三卷



 盧重元《夢書》四卷開元人。



 柳璨《夢雋》一卷



 右五行類六十家,一百六十部,六百四十七卷。失姓名六十五家,袁天綱以下不著錄二十五家,一百三十二卷。



 郝沖、虞譚法《投壺經》一卷



 魏文帝《皇博經》一卷



 《大小博法》二卷



 《大博經行棋戲法》二卷



 鮑宏《小博經》一卷



 《博塞經》一卷



 《雜博戲》五卷



 隋煬帝《二儀簿經》一卷



 範汪等注《棋品》五卷



 梁武帝《棋評》一卷



 《棋勢》六卷



 《圍棋後九品序錄》一卷



 《竹苑仙棋圖》一卷



 周武帝《象經》一卷



 何妥《象經》一卷



 王褒《象經》一卷



 王裕注《象經》一卷



 《今古術藝》十五卷



 《名手畫錄》一卷



 李嗣真《畫後品》一卷



 《禮圖等雜畫》五十六卷



 漢王元昌畫《漢賢王圖》



 閻立德畫《文成公主降蕃圖》



 《玉華宮圖》



 《鬥雞圖》



 閻立本畫《秦府十八學士圖》



 《凌煙閣功臣二十四人圖》



 範長壽畫《風俗圖》



 《醉道士圖》



 王定畫《本草訓誡圖》貞觀尚方令。



 檀智敏畫《游春戲藝圖》振武校尉。



 殷雰、韋無忝畫《皇朝九聖圖》



 《高祖及諸王圖》



 《太宗自定輦上圖》



 《開元十八學士圖》開元人。



 董萼畫《盤車圖》開元人,字重照。



 曹元廓畫《後周北齊梁陳隋武德貞觀永徽等朝臣圖》



 《高祖太宗諸子圖》



 《秦府學士圖》



 《凌煙圖》武後左尚方令。



 楊昇畫《望賢宮圖》



 《安祿山真》



 張萱畫《少女圖》



 《乳母將嬰兒圖》



 《按羯鼓圖》



 《秋千圖》並開元館畫直。



 談皎畫《武惠妃舞圖》



 《佳麗寒食圖》



 《佳麗伎樂圖》



 韓幹畫《龍朔功臣圖》



 《姚宋及安祿山圖》



 《相馬圖》



 《玄宗試馬圖》



 《寧王調馬打球圖》大梁人,太府寺丞。



 陳宏畫《安祿山圖》



 《玄宗馬射圖》



 《上黨十九瑞圖》永王府長史。



 王象畫《鹵簿圖》



 田琦畫《洪崖子橘木圖》德平子,汝南太守。



 竇師綸畫《內庫瑞錦對雉鬥羊翔鳳游麟圖》字希言,太宗秦王府諮議、相國錄事參軍,封陵陽公。



 韋鶠畫《天竺胡僧渡水放牧圖》鑾子。



 周昉畫《撲蝶》、《按箏》、《楊真人降真》、《五星》等圖各一卷字景玄。



 張彥遠《歷代名畫記》十卷



 姚最《續畫品》一卷



 裴孝源《畫品錄》一卷中書舍人,記貞觀、顯慶年事。



 顧況《畫評》一卷



 硃景玄《唐畫斷》三卷會昌人。



 竇蒙《畫拾遺》卷亡。



 吳恬《畫山水錄》卷亡。恬一名玢,字建康,青州人。



 王琚《射經》一卷



 張守忠《射記》一卷



 任權《弓箭論》一卷



 上官儀《投壺經》一卷



 王積薪《金穀園九局圖》一卷開元待詔。



 韋珽《棋圖》一卷



 呂才《大博經》二卷



 董叔經《博經》一卷貞元中上。



 李郃《骰子選格》三卷字中玄,賀州刺史。



 右雜藝術類十一家,二十部,一百四十二卷。失姓名八家,張彥遠以下不著錄一十六家,一百一十七卷。



 何承天並合《皇覽》一百二十二卷



 徐爰並合《皇覽》八十四卷



 劉孝標《類苑》一百二十卷



 劉杳《壽光書苑》二百卷



 徐勉《華林遍略》六百卷



 祖孝徵等《脩文殿御覽》三百六十卷



 虞綽等《長洲玉鏡》二百三十八卷



 諸葛潁《玄門寶海》一百二十卷



 張氏《書圖泉海》七十卷



 《要錄》六十卷



 《檢事書》一百六十卷



 《帝王要覽》二十卷



 《文思博要》一千二百卷



 《目》十二卷右僕射高士廉、左僕射房玄齡、特進魏徵、中書令楊師道、兼中書侍郎岑文本、禮部侍郎彥相時、國子司業硃子奢、博士劉伯莊、太學博士馬嘉運、給事中許敬宗、司文郎中崔行功、太常博士呂才、秘書丞李淳風、起居郎褚遂良、晉王友姚思廉、太子舍人司馬宅相等奉詔撰,貞觀十五年上。



 許敬宗《搖山玉彩》五百卷孝敬皇帝令太子少師許敬宗、司議郎孟利貞、崇賢館學士郭瑜,顧胤、右史董思恭等撰。



 《累璧》四百卷



 又《目錄》四卷許敬宗等撰,龍朔元年上。



 《東殿新書》二百卷許敬宗、李義府奉詔於武德內殿修撰。其書自《史記》至《晉書》,刪其繁辭。龍朔元年上,高宗製序。



 歐陽詢《藝文類聚》一百卷令狐德棻、袁朗、趙弘智等同脩。



 虞世南《北堂書鈔》一百七十三卷



 張大素《策府》五百八十二卷



 武後《玄覽》一百卷



 《三教珠英》一千三百卷



 《目》十三卷張昌宗、李嶠、崔湜、閻朝隱、徐彥伯、張說、沈佺期、宋之問、富嘉篸、喬侃、員半千、薛曜等撰。開成初改為《海內珠英》,武后所改字並復舊。



 孟利貞《碧玉芳林》四百五十卷



 《玉藻瓊林》一百卷



 王義方《筆海》十卷



 《玄宗事類》一百三十卷



 又《初學記》三十卷張說類集要事以教諸王,徐堅、韋述、余欽、施敬本、張烜、李銳、孫季良等分撰。



 是光乂《十九部書語類》十卷開元末,自秘書省正字上,授集賢院修撰,後賜姓齊。



 劉秩《政典》三十五卷



 杜佑《通典》二百卷



 蘇冕《會要》四十卷



 《續會要》四十卷楊紹復、裴德融、崔彖、薛逢、鄭言、周膚敏、薛廷望、於珪、於球等撰,崔弦監脩。



 陸贄《備舉文言》二十卷



 劉綺《莊集類》一百卷



 高丘《詞集類略》三十卷



 陸羽《警年》十卷



 張仲素《詞圃》十卷字繪之,元和翰林學士、中書舍人。



 《元氏類集》三百卷元稹。



 《白氏經史事類》三十卷白居易。一名《六貼》。



 《王氏千門》四十卷王洛賓。



 於立政《類林》十卷



 郭道規《事鑒》五十卷



 馬幼昌《穿楊集》四卷判目。



 盛均《十三家貼》均,字之材,泉州南安人,終昭州刺史。以《白氏六帖》未備而廣之,卷亡。



 竇蒙《青囊書》十卷國子司業。



 韋稔《瀛類》十卷



 《應用類對》十卷



 高測《韻對》十卷



 溫庭筠《學海》三十卷



 王博古《脩文海》十七卷



 李途《記室新書》三十卷



 孫翰《錦繡穀》五卷



 張楚金《翰苑》七卷



 皮氏《鹿門家鈔》九十卷皮日休,字襲美,咸通太常博士。



 劉揚名《戚苑纂要》十卷



 《戚苑英華》十卷袁說重脩。



 右類書類十七家,二十四部,七千二百八十八卷。失姓名三家,王義方以下不著錄三十二家,一千三百三十八卷。



 皇甫謐《皇帝三部針經》十二卷



 張子存《赤烏神針經》一卷



 《黃帝針灸經》十二卷



 《黃帝雜注針經》一卷



 《黃帝針經》十卷



 《玉匱針經》十二卷



 《龍銜素針經並孔穴蝦蟆圖》三卷



 《徐叔向針灸要鈔》一卷



 《黃帝明堂經》三卷



 《黃帝明堂》三卷



 楊玄注《黃帝明堂經》三卷



 《黃帝內經明堂》十三卷



 《黃帝十二經脈明堂五藏圖》一卷



 曹氏《黃帝十二經明堂偃側人圖》十二卷



 秦承祖《明堂圖》三卷



 《明堂孔穴》五卷



 秦越人《黃帝八十一難經》二卷



 全元起注《黃帝素問》九卷



 靈寶注《黃帝九靈經》十二卷



 《黃帝甲乙經》十二卷



 《黃帝流注脈經》一卷



 《三部四時五藏辨候診色脈經》一卷



 《脈經》十卷



 又二卷



 徐氏《脈經訣》三卷



 王子顒《脈經》二卷



 歧伯《灸經》一卷



 雷氏《灸經》一卷



 《五藏訣》一卷



 《五藏論》一卷



 賈和光《鈴和子》十卷



 王冰注《黃帝素問》二十四卷



 《釋文》一卷冰號啟元子。



 楊上善注《黃帝內經明堂類成》十三卷



 又《黃帝內經太素》三十卷



 甄權《脈經》一卷



 《針經鈔》三卷



 《針方》一卷



 《明堂人形圖》一卷



 米遂《明堂論》一卷



 右明堂經脈類一十六家,三十五部,二百三十一卷。失姓名十六家,甄權以下不著錄二家,七卷。



 《神農本草》三卷



 雷公集撰《神農本草》四卷



 《吳氏本草因》六卷吳普。



 《李氏本草》三卷



 原平仲《靈秀本草圖》六卷



 殷子嚴《本草音義》二卷



 《本草用藥要妙》九卷



 《本草病源合藥節度》五卷



 《本草要術》三卷



 《療癰疽耳眼本草要妙》五卷



 《桐君藥錄》三卷



 徐之才《雷公藥對》二卷



 僧行智《諸藥異名》十卷



 《藥類》二卷



 《藥目要用》二卷



 《四時採取諸藥及合和》四卷



 《名醫別錄》三卷



 吳景《諸病源候論》五十卷



 《巢氏諸病源候論》五十卷巢元方。



 徐嗣伯《雜病論》一卷



 又《徐氏落年方》三卷



 《彭祖養性經》一卷



 張湛《養生要集》十卷



 《延年秘錄》十二卷



 秦承祖《藥方》四十卷



 吳普集《華氏藥方》十卷華佗。



 葛洪《肘後救卒方》六卷



 梁武帝《坐右方》十卷



 《如意方》十卷



 陶弘景集注《神農本草》七卷



 又《效驗方》十卷



 《補肘後救卒備急方》六卷



 《太清玉石丹藥要集》三卷



 《太清諸草木方集要》三卷



 隋煬帝敕撰《四海類聚單方》十六卷



 王叔和《張仲景藥方》十五卷



 又《傷寒卒病論》十卷



 《阮河南方》十六卷阮炳。



 尹穆纂《範東陽雜藥方》一百七十卷範注。



 《胡居士治百病要方》三卷胡洽。



 徐叔向《雜療方》二十卷



 又《體療雜病方》六卷



 《腳弱方》八卷



 《解寒食方》十五卷



 《阮河南藥方》十七卷



 褚澄《雜藥方》十二卷



 陳山提《雜藥方》十卷



 謝泰《黃素方》二十五卷



 孝思《雜湯丸散方》五十七卷



 謝士太《刪繁方》十二卷



 徐之才《徐王八代效驗方》十卷



 又《家秘方》三卷



 範世英《千金方》三卷



 姚僧垣《集驗方》十卷



 陳延之《小品方》十二卷



 蘇游《玄感傳尸方》一卷



 又《太一鐵胤神丹方》三卷



 《俞氏治小兒方》四卷



 俞寶《小女節療方》一卷



 僧僧深《集方》三十卷



 僧鸞《調氣方》一卷



 龔慶宣《劉涓子男方》十卷



 甘浚之《療癰疽金瘡要方》十四卷



 甘伯齊《療癰疽金瘡要方》十二卷



 《雜藥方》六卷



 《雜丸方》一卷



 《名醫集驗方》三卷



 《百病膏方》十卷



 《雜湯方》八卷



 《療目方》五卷



 《寒食散方並消息節度》二卷



 《婦人方》十卷



 又二十卷



 《少女方》十卷



 《少女雜方》二十卷



 《類聚方》二千六百卷



 《種芝經》九卷



 《芝草圖》一卷



 諸葛潁《淮南王食經》一百三十卷



 《音》十三卷



 《食目》十卷



 盧仁宗《食經》三卷



 崔浩《食經》九卷



 竺暄《食經》四卷



 又十卷



 趙武《四時食法》一卷



 《太官食法》一卷



 《太官食方》十九卷



 《四時御食經》一卷



 抱樸子《太清神仙服食經》五卷



 沖和子《太清璇璣文》七卷



 《太清神丹中經》三卷



 《太清神仙服食經》五卷



 《太清諸丹藥要錄》四卷



 京裏先生《金匱仙藥錄》三卷



 《神仙服食經》十二卷



 《神仙藥食經》一卷



 《神仙服食方》十卷



 《神仙服食藥方》十卷



 《服玉法並禁忌》一卷



 《寒食散論》二卷



 葛仙公錄《狐子方金訣》二卷



 《狐子雜訣》三卷



 明月公《陵陽子秘訣》一卷



 黃公《神臨藥秘經》一卷



 《黃白秘法》一卷



 又二十卷



 葛氏《房中秘術》一卷



 沖和子《玉房秘訣》十卷張鼎



 《本草》二十卷



 《目錄》一卷



 《藥圖》二十卷



 《圖經》七卷顯慶四年,英國公李勣、太尉長孫無忌、兼侍中辛茂將、太子賓客弘文館學士許敬宗、禮部郎中兼太子洗馬弘文館大學士孔志約、尚藥奉御許孝崇胡子彖蔣季璋、尚藥局直長藺復珪許弘直、侍御醫巢孝儉、太子藥藏監蔣季瑜吳嗣宗、丞蔣義方、太醫令蔣季琬許弘、丞蔣茂昌、太常丞呂才賈文通、太史令李淳風、潞王府參軍吳師哲、禮部主事顏仁楚、右監門府長史蘇敬等撰。



 孔志約《本草音義》二十卷



 蘇敬《新脩本草》二十一卷



 又《新脩本草圖》二十六卷



 《本草音》三卷



 《本草圖經》七卷



 甄立言一作權。《本草音義》七卷



 又《本草藥性》三卷



 《古今錄驗方》五十卷



 孟詵《食療本草》三卷



 又《補養方》三卷



 《必效方》十卷



 宋俠《經心方》十卷



 《崔氏纂要方》十卷崔行功。



 崔知悌《骨蒸病灸方》一卷



 王方慶《新本草》四十一卷



 又《藥性要訣》五卷



 《袖中備急要方》三卷



 《嶺南急要方》二卷



 《針灸服藥禁忌》五卷



 李含光《本草音義》二卷



 陳藏器《本草拾遺》十卷開元中人。



 鄭虔《胡本草》七卷



 孫思邈《千金方》三十卷



 又《千金髓方》二十卷



 《千金翼方》三十卷



 《神枕方》一卷



 《醫家要妙》五卷



 《楊太僕醫方》一卷失名。天授二年上。



 衛嵩《醫門金寶鑒》三卷



 許詠《六十四問》一卷



 段元亮《病源手鏡》一卷



 《伏氏醫苑》一卷伏適。



 甘伯宗《名醫傳》七卷



 王超《仙人水鏡圖訣》一卷貞觀人。



 吳兢《五藏論應象》一卷



 裴璡《五藏論》一卷



 劉清海《五藏類合賦》五卷



 裴王廷《五色傍通五藏圖》一卷



 張文懿《藏府通元賦》一卷



 段元亮《五藏鏡源》四卷



 喻義纂《療癰疽要訣》一卷



 《瘡腫論》一卷



 沈泰之《癰疽論》二卷



 青溪子《萬病拾遺》三卷



 又《消渴論》一卷



 《腳氣論》三卷



 李暄《嶺南腳氣論》一卷



 又《方》一卷



 《腳氣論》一卷蘇鑒、徐玉等編集。



 鄭景岫《南中四時攝生論》一卷



 蘇游《鐵粉論》一卷



 陳元《北京要術》一卷元為太原少尹。



 司空輿《發焰錄》一卷圖父,大中時商州刺史。



 青羅子《道光通元秘要術》三卷失姓,咸通人。



 《乾寧晏先生制伏草石論》六卷晏封。



 江承宗《刪繁藥詠》三卷鳳翔節度要籍。



 玄宗《開元廣濟方》五卷



 劉貺《真人肘後方》三卷



 王燾《外臺秘要方》四十卷



 又《外臺要略》十卷



 德宗《貞元集要廣利方》五卷



 《陸氏集驗方》十五卷陸贄。



 賈耽《備急單方》一卷



 薛弘慶《兵部手集方》三卷兵部尚書李絳所傳方。弘慶,大和河中少尹。



 薛景晦《古今集驗方》十卷



 元和刑部郎中,貶道州刺史。



 劉禹錫《傳信方》二卷



 崔玄亮《海上集驗方》十卷



 《楊氏產乳集驗方》三卷楊歸厚,元和中,自左拾遺貶鳳州司馬、虢州刺史。方九百一十一。



 鄭注《藥方》一卷



 《韋氏集驗獨行方》十二卷韋宙。



 張文仲《隨身備急方》三卷



 蘇越《群方秘要》三卷



 李繼皋《南行方》三卷



 白仁敘《唐興集驗方》五卷



 包會《應驗方》一卷



 許孝宗《篋中方》三卷



 梅崇《獻方》五卷



 姚和眾《童子秘訣》三卷



 又《眾童延齡至寶方》十卷



 孫會《嬰孺方》十卷



 邵英俊《口齒論》一卷



 又《排玉集》二卷口齒方。



 李昭明《嵩臺集》三卷



 陽曄《膳夫經手錄》四卷



 嚴龜《食法》十卷震之後,鎮西軍節度使譔子也。昭宗時宣慰汴寨。



 右醫術類六十四家,一百二十部,四千四十六卷。失姓名三十八家,王方慶以下不著錄五十五家,四百八卷。



\end{pinyinscope}