\article{列傳第九 李密}

\begin{pinyinscope}

 李密,字玄邃,一字法主,其先遼東襄平人。曾祖弼,魏司徒其對策,由是開以儒學為正統學術之先聲。其學說以儒家思,賜姓徒何氏,入周為太師、魏國公。祖曜,邢國公。父寬,隋上柱國、蒲山郡公。遂家長安。



 密趣解雄遠,多策略,散家貲養客禮賢不愛藉。以廕為左親衛府大都督、東宮千牛備身。額銳角方,瞳子黑白明澈。煬帝見之,謂宇文述曰:「左仗下黑色小兒為誰?」曰:「蒲山公李寬子密。」帝曰:「此兒顧盼不常,無入衛。」它日,述諭密曰:「君世素貴,當以才學顯,何事三衛間哉!」密大喜,謝病去,感厲讀書。聞包愷在緱山,往從之。以蒲韉乘牛,掛《漢書》一帙角上,行且讀。越國公楊素適見於道,按轡躡其後,曰:「何書生勤如此?」密識素,下拜。問所讀,曰:「《項羽傳》。」因與語,奇之。歸謂子玄感曰:「吾觀密識度,非若等輩。」玄感遂傾心結納。嘗私自密曰:「上多忌,隋歷且不長,中原有一日警,公與我孰後先?」密曰:「決兩陣之勝,噫嗚咄嗟,足以讋敵,我不如公。攬天下英雄馭之,使遠近歸屬,公不如我。」



 大業九年,玄感舉兵黎陽,遣人入關迎密。密至,謀曰:「今天子遠在遼左,去幽州尚千里,南限鉅海,北阻強胡,號令所通,惟榆林一道爾。若鼓而入薊,直扼其喉,高麗抗其前,我乘其後,不旬月齎糧竭,舉麾召之,眾可盡取,然後傳檄而南,天下定矣,上計也。關中四塞之地,彼留守衛文升,易人耳。若徑行勿留,直保長安,據函、崤,東制諸夏,是隋亡襟帶,我勢萬全,中計也。若因近趣便,先取東都,頓兵堅城下,不可以勝負決,下計也。」玄感曰:「公之下計,乃吾上策。今百官家屬皆在洛,當先取之,以搖其心。且經城不拔,何以示武?」密計不行。玄感至東都,所戰必克,自謂功在旦暮。既獲內史舍人韋福嗣,遂任之,故謀不專密。福嗣恥見執,策議皆持兩端。密揣其貳,謂玄感曰:「福嗣窮,為我虜,志在觀望。公初舉大事,奸人在側,事必敗,請斬以徇。」不從。密謂所親曰:「玄感好反而不圖勝,吾屬虜矣!」福嗣果遁去。會左武候大將軍李子雄得罪,傳送行在,道殺使者,奔玄感,勸舉大號。玄感問密,密曰:「昔張耳諫陳勝自王,荀彧止魏武求九錫,皆見疑外。今密將無類之乎?然阿諛順旨,非義士也。且公雖屢勝,而郡縣未有應者,東都尚強,救兵踵來,公當率精甲,身定關中,奈何亟自帝?」玄感笑而止。



 及隋軍至,玄感曰:「策安決?」密曰:「元弘嗣方戍隴右,可陽言其反,使迎我,因引軍西。」從之。至陜,欲圍弘農宮,密曰:「今紿眾入關,機在速,而追兵踵我,若前不得據險,退無所守,何以共完!」玄感不聽。留攻三日,不能拔,引去,至閺鄉,追及而敗。



 密羸行入關,為邏所獲,與支黨護送帝所。密謂眾曰:「吾等至行在,且菹醢,今尚可以計脫,何為安就鼎鑊?」眾然之。乃令出所有金示監使曰:「即死,幸報德。」使者顧金,禁漸弛,益市酒,飲笑歡嘩,守者懈,密等遂夜亡去。抵平原,賊郝孝德不見禮,去之淮陽。歲饑,削木皮以食。變姓名為劉智遠,教授諸生自給,鬱鬱不得志,哀吟泣下。人有告太守趙佗者,佗捕之,遁免。往依胃婿雍丘令丘君明,轉匿大俠王季才家,為吏跡捕,復亡去。



 時東郡賊翟讓聚黨萬人,密因介其徒王伯當以策干讓曰:「今主昏於上,人怨於下,銳兵盡之遼海,和親絕於突厥,南巡流連,空棄關輔,此實劉、項挺興之會。足下資豪桀,士馬精勇,指罪誅暴,為天下先,楊氏不足亡也。」讓由是加禮,遣說諸賊,至輒下。因為讓計曰:「今稟無見糧,難以持久,卒遇敵,其亡無時。不如取滎陽,休兵館穀,待士逸馬肥,乃可與人爭利。」讓聽之,遂破金堤關,徇滎陽諸縣,皆下。滎陽太守楊慶、河南討捕大使張須陀合兵討讓,讓素憚須陀,欲引去。密曰:「須陀健而無謀,且驟勝易驕,吾為公破之。」讓不得已,陣而待。密率驍勇常何等二十人為游騎,伏千兵莽間。須陀素輕讓,引兵搏之,讓少卻,伏發,與游軍乘之,遂殺須陀。



 十三年,讓分兵與密,別為牙帳,號蒲山公。密持軍嚴,雖盛夏號令,士皆若負霜雪,然戰得金寶,盡散之,繇是人為用。復說讓曰:「今群豪競興,公宜先天下攘除群兇,寧常剽奪草間求活哉?若直取興洛倉,發粟以賑窮乏,百萬之眾一朝可附,霜王之業成矣。」讓曰:「僕起畎隴,志不及此,須君得倉,更議之。」



 二月,密以千人出陽城北,逾方山,自羅口拔興洛倉,據之,獲縣長柴孝和。開倉賑食,眾繦屬至數十萬。隋越王侗遣將劉長恭、房崱討密,又令裴仁基統兵出成皋西。密乃為十隊,跨洛水,抗東、西二軍。令單雄信、徐世勣、王伯當騎為左右翼,自引麾下急擊長恭等,破之。東都震恐,眾保太微城,臺寺俱滿。



 讓等乃推密為主,建號魏公。鞏南設壇場,即位,刑牲歃血,改元永平,大赦,其文移稱行軍元帥魏公府。以讓為司徒,邴元真左長史,房彥藻右長史,楊德方左司馬,鄭德韜右司馬,單雄信左武候大將軍,徐世勣右武候大將軍。祖君彥記室。城洛口,周四十里,居之。命護軍將軍田茂廣造雲■三百具,以機發石,為攻城械,號「將軍砲」。進逼東都,燒上春門。



 四月,隋虎牢將裴仁基、淮陽太守趙佗降,長白山賊孟讓以所部歸密。以仁基為上柱國,與讓率兵二萬襲回洛倉,守之。入都城掠居人,火天津橋。隋出軍乘之,仁基等敗,還保鞏。司馬楊德方戰死。密自督眾三萬,破隋軍於故城,復得回洛倉。俄而德韜死,乃以鄭頲為左司馬,鄭虔象右司馬。諸賊帥黎陽李文相、洹水張升、清河趙君德、平原郝孝德皆歸密,因襲取黎陽倉。永安大族周法明舉江、黃地附之,齊郡賊徐圓郎、任城大俠徐師仁來歸。密令幕府移檄州縣,列煬帝十罪,天下震動。



 護軍柴孝和說密曰:「秦地阻山帶河,項背之亡,漢得之王。今公以仁基壁回洛,翟讓保洛口,公束鎧倍道趨長安,百姓誰不郊迎?是征而不戰也。眾附兵強,然後東向,指捴豪傑,天下廓清無事矣。今遲之,恐為人先。」密曰:「僕懷此久,顧我部皆山東人,今未下洛,安肯與我偕西?且諸將皆群盜,不相統一,敗則掃地矣。」遂止。是時,隋軍益出,密負銳,急與之確,中流矢,臥營中,隋軍乘之,密眾潰,棄倉守洛口。



 高祖起師太原,密自謂主盟,遣將軍張仁則致書於帝,呼為兄,請以步騎會河內。帝覽書,笑曰:「密陸梁,不可折簡致之。吾方定京師,未能東略,若不與,是生一隋。密適為吾守成皋,拒東都兵,使不得西,更遣剽將莫如密。吾寧推順,使驕其志,我得留撫關中,大事濟矣。」令記室溫大雅作報書,厚禮尊讓。密大喜,示其下,曰:「唐公見推,顧天下無可慮者。」遂專事隋。



 九月,遣將李士才將兵十二萬,攻隋鷹揚郎將張珣河陰,舉之。珣極罵不屈死。齊方士徐鴻客上書勸密因士氣趨江都,挾帝以令天下。密異其言,具幣邀之,已亡去。煬帝遣王世充選卒十萬擊密,世充營洛西,戰不利,更陳洛北,登山以望洛口。密引度洛,與世充戰。密兵多騎與長槊,而北薄山,地隘騎迮不得騁。世充多短兵盾,蹙之,密軍卻,世充乘勝進攻密月城。密還洛南,引而西,突世充營,世充奔還。師徒多喪,孝和溺死洛水,密哭之慟。自是大小六十餘戰。



 翟讓部將王儒信憚密威望,勸讓自為大塚宰,總秉眾務,收密權。讓兄寬亦曰:「天子當自取,何乃授人?」密聞之,與鄭頲陰圖讓。會世充兵又至,讓出拒,少退;密馳助之,戰石子河,世充走。明日,高會饗士,讓至密所,密令房彥藻引其左右就別帳飲。密出名弓示讓,讓挽滿,遣劍士蔡建從後擊之,並殺其兄、侄及儒信。密馳入讓壁慰諭,士無敢動者,以徐世勣、單雄信、王伯當分統其兵。隋將楊慶守滎陽,因說下之。世充夜襲倉城,密伏甲殪其眾。



 義寧二年,世充復營洛北,為浮梁,絕水以戰,密以千騎迎擊,不勝。世充進薄其壘,密提敢死士數百邀之,世充大潰,士爭橋溺死者數萬,洛水為不流,殺大將六人,獨世充脫。會夜大雨雪,士卒殭死且盡。密乘銳拔偃師,脩金墉城居之,有眾三十萬。又與東都留守韋津戰上春門,執津於陣。將作大匠宇文愷子儒童、河南留守職方郎柳續、河陽都尉獨孤武都、河內郡丞柳燮皆降。於是海岱、江淮間爭響附,竇建德、硃粲、楊士林、孟海公、徐圓朗、盧祖尚、周法明等悉上表勸進,府官屬亦請之。密曰:「東都未平,且勿議。」



 五月,越王侗稱帝。六月,宇文化及擁兵十餘萬至黎陽。侗遣使授密太尉、尚書令、東南道大行臺行軍元帥、魏國公,令平化及而後入輔,密受之。乃引兵東追化及黎陽。密知化及乏食,利速戰,乃持重以老其兵,使徐世勣保黎陽倉,化及攻不可下。密與隔水陣,遙謂化及曰:「公家本戎隸破野頭爾,父子兄弟受隋恩,至妻公主。上有失德不能諫,又虐殺之,冒天下之惡,今安往?能即降,尚全後嗣。」化及默然良久,乃瞋目為鄙語辱密。密顧左右曰:「此庸人,圖為帝,吾當折箠驅之。」乃以輕騎五百焚其攻具,火終夜不滅。度化及糧盡,乃偽與和,化及喜,使軍恣食,既而密饋不至,乃寤。遂大戰童山下,密中矢,頓汲縣堅壁。化及勢窮,掠汲郡,趣魏縣。其將陳智略、張童仁等率所部兵歸密,前後相踵。



 初,化及留輜重東郡,遣所署刑部尚書王軌守之。至是,軌舉郡降密。由是引而西,遣使朝東都,執殺逆人於弘達獻於侗。侗召密入朝,至溫,聞世充殺元文都,乃止。遂歸金墉,拘侗使不遣。



 初,密既殺翟讓,心稍驕,不恤士,素無府庫財,軍戰勝,無所賜與,又厚撫新集,人心始離。民食興洛倉者,給授無檢,至負取不勝,委於道,踐輮狼扈。密喜,自謂足食。司倉賈潤甫諫曰:「人,國本;食,人天。今百姓饑捐,暴骨道路。公雖受命,然賴人之天以固國本。而稟取不節,敖庾之藏有時而儩,粟竭人散,胡仰而成功?」不聽。徐世勣數規其違,密內不喜,使出就屯,故下茍且無固志。初,世充乏食,密少帛,請交相易,難之。邴元真好利,陰勸密許焉。後世充士飽,降者益少,密悔而止。



 武德元年九月,世充悉眾決戰,先以騎數百度河,密遣迎戰,驍將十餘人皆被創返。明日,密留王伯當守金墉,自引精兵出偃師,北阻邙山待之。密議所便,裴仁基曰:「世充悉勁兵來,東都必虛,請選眾二萬向洛,世充必自拔歸,我整軍徐還。兵法所謂彼歸我出,彼出我歸,以疲之也。」密眩於眾,不能用。仁基擊地嘆曰:「公後必悔!」遂出兵陣。世充陰索貌類密者,使縛之。既兩軍接,埃霧囂塞,世充軍,江淮士,出入若飛,密兵心動。世充督眾疾戰,使牽類密者過陣,噪曰:「獲密矣!」士皆呼萬歲,密軍亂,遂潰。裴仁基、祖君彥皆為世充所禽,偃師劫鄭頲叛歸世充。密提眾萬餘馳洛口,將入城,邴元真已輸款世充,潛導其軍。密知不發,期世充度兵半洛水,掩擊之。候騎不時覺,比出,世充絕河矣。即引騎遁武牢,元真遂降,眾稍散。



 密將如黎陽,或曰:「向殺翟讓,世勣傷幾死,瘡猶未平,今可保乎?」時王伯當棄金墉屯河陽,密輕騎歸之,謂曰:「敗矣,久苦諸君,我今自刎以謝眾!」伯當抱密慟絕,眾皆泣,莫能仰視。密復曰:「幸不相棄,當共歸關中,密雖無功,諸君必富貴。」掾柳燮曰:「昔盆子歸漢,尚食均輸。公與唐同族,雖不共起,然遏隋歸路,使無西,故唐不戰而據京師,亦公功也。」密又謂伯當曰:「將軍族重,豈復與孤俱行哉?」伯當曰:「昔蕭何舉宗從漢,今不昆季盡行,以為愧。豈公一失利,輕去就哉?雖隕首穴胸,所甘已。」左右感動,遂來歸。



 初,密建號登壇,疾風鼓其衣,幾僕;及即位,狐鳴於旁,惡之。及將敗,鞏數有回風發於地,激砂礫上屬天,白日為晦;屯營群鼠相銜尾西北度洛,經月不絕。



 及入關,兵尚二萬。高祖使迎勞,冠蓋相望,密大喜,謂其徒曰:「吾所舉雖不就,而恩結百姓,山東連城數百,以吾故,當盡歸國。功不減竇融,豈不以臺司處我?」及至,拜光祿卿,封邢國公,殊怨望。帝嘗呼之弟,妻以表妹獨孤氏。後禮寢薄,執政者又求賄,滋不平。因朝會進食,謂王伯當曰:「往在洛口,嘗欲以崔君賢為光祿,不意身自為此。」



 未幾,聞故所部將多不附世充者,高祖詔密以本兵就黎陽招撫故部曲,經略東都,伯當以左武衛將軍為密副。馳驛東至稠桑驛,有詔復召密,密大懼,謀叛。伯當止之,不從,乃曰:「士立義,不以存亡易慮。公顧伯當厚,願畢命以報。今可同往,死生以之,然無益也。」乃簡驍勇數十人,衣婦人服,戴幕釭,藏刀裙下,詐為家婢妾者,入桃林傳舍,須臾變服出,據其城。掠畜產,趣南山而東,馳告張善相以兵應己。



 熊州副將盛彥師率步騎伏陸渾縣南邢公峴之下,密兵度,橫出擊,斬之,年三十七,伯當俱死,傳首京師。時徐世勣尚為密保黎陽,帝遣使持密首往招世勣。世勣表請收葬,詔歸其尸,乃發喪,具威儀,三軍縞素,以君禮葬黎陽山西南五里,墳高七仞。密素得士,哭多歐血者。



 邴元真之降也,世充以為行臺僕射,鎮滑州。密故將杜才幹恨其背密,偽以兵歸之,斬取其首,祭密塚,已乃歸國。



 單雄信,曹州濟陰人。與翟讓友善。能馬上用槍,密軍中號「飛將」。偃師敗,降世充,為大將。秦王圍東都,雄信拒戰,槍幾及王,徐世勣呵之曰:「秦王也!」遂退。後東都平,斬洛渚上。



 祖君彥,齊僕射孝徵子。博學強記,屬辭贍速。薛道衡嘗薦之隋文帝,帝曰:「是非殺斛律明月人兒邪?朕無用之。」煬帝立,尤忌知名士,遂調東都書佐,檢校宿城令,世謂祖宿城。負其才,常鬱鬱思亂。及為密草檄,乃深斥主闕。密敗,世充見之,曰:「汝為賊罵國足未?」君彥曰:「跖客可使刺由,但愧不至耳!」世充令撲之。既困臥樹下,世充已自欲盜隋,中悔,命醫許惠照往視之,欲其蘇。郎將王拔柱曰:「弄筆生有餘罪。」乃蹙其心,即死,戮尸於偃師。



 贊曰:或稱密似項羽,非也。羽興五年霸天下,密連兵數十百戰不能取東都。始玄感亂,密首勸取關中;及自立,亦不能鼓而西,宜其亡也。然禮賢得士,乃田橫徒歟,賢陳涉遠矣!噫,使密不為叛,其才雄亦不可容於時云。



\end{pinyinscope}