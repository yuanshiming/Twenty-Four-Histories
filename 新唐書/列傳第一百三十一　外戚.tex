\article{列傳第一百三十一 外戚}

\begin{pinyinscope}

 凡外戚成敗,視主德何如。主賢則共其榮,主否則先受其禍。故太宗檢貴幸,裁賞賜,貞觀時但對近代資本主義生產方式的產生和商品經濟的發展卻起了,內里無敗家。高、中二宗,柄移艷私,產亂朝廷,武、韋諸族,耄嬰頸血,一日同污鐵刃。玄宗初年,法行近親,里表修敕。天寶奪明,委政妃宗,階召反虜,遂喪天下。楊氏之誅,噍類不遺,蓋數十年之寵,不賞一日之慘,甲第厚貲,無救同坎之悲,寧不哀哉!代、德而降,閹尹參嬖,後宮雖多,無赫赫顯門,亦無刀鋸大戮。故用福甚者得禍酷,取名少者蒙責輕,理所固然。若乃長孫無忌之功,武平一之識,吳漵之忠,弗緣內寵者,自見別傳。



 獨孤懷恩,元貞皇后弟之子也。父整,仕隋為涿郡太守。懷恩之幼,隋文帝獻皇后以侄養宮中。逮長,稍學記書,而居財不訾,喜交豪猾博徒。為雩令,以疾免。



 高祖平京師,拜長安令,頗嚴明,如職而辦。帝受禪,擢工部尚書。初,虞州刺史韋義節擊堯君素於蒲州,不克,帝遣懷恩代將。性貪,寡算略,數戰無功,士喪沮,詔書切責,而懷恩稍怨望。帝嘗與戲曰:「弟姑子悉有天下,次當爾邪?」懷恩內喜,以為天命。既而居忽忽,吒曰:「我家渠獨女子富貴也?」因謀亂。是時,虞鄉南山多宿盜,而劉武周使宋金剛略澮州,帝發關中軍屬秦王,屯柏壁。由是懷恩與麾下元君寶、解令榮靜謀引王行本軍與武周連和,割河東以啖之,引群賊取永豐倉,絕秦王餉道,長驅三輔。會君素死,而行本得其兵,部畫已定,而夏人呂崇茂殺縣令應武周。帝敕懷恩與永安王孝基、陜州總管於筠、內史侍郎唐儉擊夏,為金剛所掩,諸將皆沒於賊。君寶與開府劉讓私侮懷恩曰:「不早舉大事,以及斯辱也。」故謀浸露。



 及秦王敗武周於美良川,懷恩逃歸,帝命率師攻蒲州。君寶聞曰:「王者不死,果其然!」唐儉知狀。會武周還劉讓求罷兵,因白發懷恩等奸。於時行本舉蒲州降,懷恩勒兵入城,帝方濟河而讓至,具得反狀。帝召之,懷恩不知也,單舟以來,即縛之,窮索黨與,縊死於獄,以首徇華陰市,籍入其家。



 武士獲字信,世殖貲,喜交結。高祖嘗領屯汾、晉,休其家,因被顧接。後留守太原,引為行軍司鎧參軍。募兵既集,以劉弘基、長孫順德統之。王威、高君雅私謂士訄曰:「弘基等皆背征三衛,罪當死,奈何授之兵?吾且劾系之。」士卬皞曰:「此皆唐公客,若爾,必大有嫌。」故威等疑不發。會司兵參軍田德平欲勸威劾募人狀,士訄脅謂曰:「討捕兵悉隸唐公,威、君雅無與,徒寄坐耳,何能為?」德平亦止。兵起,士卬皞不與謀也。以大將軍府鎧曹參軍從平京師,為光祿大夫、義原郡公。自言嘗夢帝騎而上天,帝笑曰:「爾故王威黨也,以能罷系劉弘基等,其意可錄,且嘗禮我,故酬汝以官。今胡迂妄媚我邪?」累遷工部尚書,進封應國公,歷利、荊二州都督。卒,贈禮部尚書,謚曰定。高宗永徽中,以士↓仲女為皇后,故崇贈並州都督、司徒、周國公。咸亨中,加贈太尉兼太子太師、太原郡王,配享高祖廟廷,列功臣上。後監朝,尊為忠孝太皇,建崇先府,置官屬,追王五世。後革命,更於東都立武氏七廟,追冊為帝,諸妣皆隨帝號曰皇后。先天中,有詔削士卬皞偽號,仍為太原王,廟遂廢。



 始,士訄娶相里氏,生子元慶、元爽。又娶楊氏,生三女。元女妻賀蘭氏,早寡。季女妻郭氏,不顯。士卬皞卒後,諸子事楊不盡禮,銜之。後立,封楊代國夫人,進為榮國,后姊韓國夫人。於時元慶已官宗正少卿,元爽少府少監,兄子惟良衛尉少卿。楊諷後上疏出元慶等於外,以示退讓。由是元慶斥龍州,元爽濠州,惟良始州。元慶死,元爽流振州。乾封時,惟良及弟淄州刺史懷運與岳牧集泰山下,於是韓國有女在宮中,帝尤愛幸。後欲並殺之,即導帝幸其母所,惟良等上食,後寘堇焉,賀蘭食之,暴死。後歸罪惟良等,誅之,諷有司改姓「蝮氏」,絕屬籍。元爽緣坐死,家屬投嶺外。



 後取賀蘭敏之為士訄後,賜氏武,襲封,擢累左侍極、蘭臺太史令,與名儒李嗣真等參與刊撰。敏之韶秀自喜,烝於榮國,挾所愛,佻橫多過失;榮國卒,後出珍幣建佛廬徼福,敏之乾匿自用;司衛少卿楊思儉女選為太子妃,告婚期矣,敏之聞其美,強私焉;楊喪未畢,褫衰粗,奏音樂;太平公主往來外家,宮人從者,敏之悉逼亂之。後疊數怒,至此暴其惡,流雷州,表復故姓,道中自經死。乃還元爽之子承嗣奉士皞後,宗屬悉原。



 士訄兄士梭、士逸。



 士棱,字彥威,少柔願,力於田。官司農少卿,宣城縣公,常主苑囿農稼事。卒,贈潭州都督,陪葬獻陵。



 士逸,字逖,有戰功,為齊王府戶曹參軍,六安縣公。從王守太原,為劉武周所執,嘗遣間人陳破賊計。賊平,擢授益州行臺左丞,數言當世得失,高祖嘉納之。終韶州刺史。



 承嗣既還,擢尚輦奉御,襲周國公,遷秘書監、禮部尚書。俄以太常卿同中書門下三品,未幾辭位。垂拱初,以春官尚書同鳳閣鸞臺平章事,改納言,代蘇良嗣為文昌左相。性暴輕忍禍,聞左司郎中喬知之婢窈娘美,且善歌,奪取之,知之作《綠珠篇》以諷,婢得詩恨死。承嗣怒,告酷吏殺之,殘其家。



 初,後擅政,中宗幽逐,承嗣自謂傳國及己,武氏當有天下,即諷後革命,去唐家子孫,誅大臣不附者,倡議追王先世,立宗廟。又王元慶曰梁王,謚憲;元爽魏王,謚德;後從父士讓楚王,謚僖;士逸蜀王,謚節。又贈兄子承業陳王。而承嗣為魏王,元慶子三思為梁王,士讓之孫攸寧為建昌王、攸歸九江王、攸望會稽王,士逸孫懿宗河內王、嗣宗臨川王、仁範河間王,仁範子載德潁川王,士棱孫攸暨千乘王,惟良子攸宜建安王、攸緒安平王、從子攸止恆安王、重規高平王,承嗣子延基南陽王、延秀淮陽王,三思子崇訓高陽王、崇烈新安王,承業子延暉嗣陳王、延祚咸安王。承嗣實封千戶,監脩國史。密諭后黨鳳閣舍人張嘉福,使洛州人上書請立己為皇太子,以觀後意。後問岑長倩、格輔元,皆執不宜。承嗣不得已,奏請責諭嘉福等,不罪也。怨長倩等,皆以罪誅。以特進罷。未幾,復同鳳閣鸞臺三品。承嗣為左相,而攸寧為納言,故皆罷。又與三思同三品,不及月俱免,復拜特進。後決意還太子矣。久之,遷太子太保,不得志,鞅鞅憤死,贈太尉、並州牧,謚曰宣。



 延基襲爵,後嫌斥其名,更曰繼魏王。長安初,與妻永泰郡主及邵王私語張易之兄弟事,後忿爭,語聞,後怒,令自殺,以延義代王。



 中宗復位,侍中敬暉等言諸武不當王,與君臣白奏:「事不兩大,武家諸王宜皆免。」帝柔昏不斷,又素畏太后,且欲悅安之,更言攸暨、三思皆與去二張功,以折暉等,才降封一級:三思王德靜郡,攸暨壽春,懿宗為耿國公,攸寧江國,攸望葉國,嗣宗管國,攸宜息國,重規鄶國,延義魏國,攸緒巢國,崇訓酆國,延祿為咸安郡公。直臣宋務光、蘇安恆上書言:「武諸王饗封,不厭人心。」帝不悟。



 載德終湖州刺史,謚武烈。攸歸歷司屬少卿,至齊州刺史,事母孝,姊亡期,不嘗五辛,語輒流涕。攸止絳州刺史。三人死太后時,不及削封。



 攸宜歷同州刺史,萬歲通天初,為清邊道行軍大總管。討契丹,後親餞白馬寺,師無功還,拜左羽林大將軍。景龍時,遷右羽林,卒。總禁兵前後十年。嗣宗終司衛卿。



 重規為汴、鄭二州刺史,未至,役人營繕,後怒,貶廬州刺史。自是著令:諸王為州,不得擅營治。突厥之叛,以重規為天兵中道大總管,與沙吒忠義、張仁亶引眾三十萬討之。左羽林大將軍閻敬容為西道後軍,兵十五萬後援。還為左金吾衛大將軍,終衛尉卿。



 延秀母本帶方人,坐其家沒入奚官,以姝惠,賜承嗣,生延秀。突厥默啜薦女和親,後令延秀納之,詔右豹韜大將軍閻知微、右武衛郎將楊鸞莊齎金幣送至突厥所。知微等潛約默啜執延秀進寇媯、檀,故延秀不得歸。神龍初,默啜請和,因延秀送款,還,封柏國公,左衛中郎將。宗兄崇訓尚安樂公主,數與宴暱,頗通突厥語。仿虜謳舞,姿度閑冶,主愛悅。會崇訓死,遂私侍主,後因尚焉。以太常卿兼右衛將軍,封恆國公。三思死,韋後復私延秀,故延秀益自肆。主府倉曹參軍何鳳說曰:「今天下系心武家,庶幾再興。且讖曰『黑衣神孫被天裳』,神孫非公尚誰哉?」因勸服阜衣惑眾。韋後敗,尚與主居禁中,同斬肅章門。攸望以太府卿貶死春州。諸武屬坐延秀誅徙者略盡,獨載德子平一以文章顯,與攸緒常避盛滿,故免,自有傳。



 攸寧,天授中擢累納言。逾年,以左羽林衛大將軍罷,俄還納言。久乃罷為冬官尚書。聖歷初,同鳳閣鸞臺平章事。自承嗣、三思罷政事,間一年,攸寧、三思復當國,置句使,苛取民貲產,毀族者凡十七八,呼天自冤。築大庫百餘舍聚所得財,一昔火,不遺一錢。以冬官尚書罷。神龍初,終岐州刺史,贈尚書右僕射。



 三思當太后時,累進夏官、春官尚書,監脩國史,爵為王。契丹陷營州,以榆關道安撫大使屯邊。還,同鳳閣鸞臺三品,逾月去位。又檢校內史,罷為太子少保,遷賓客,仍監國史。



 三思性傾諛,善迎諧主意,鉤探隱微,故後頗信任,數幸其第,賞予尤渥。薛、二張方烝蠱,三思痛屈節,為懷義御馬,倡言昌宗為王子晉後身,引公卿歌詠淫污,靦然媚人而不恥也。後春秋高,厭居宮中,三思欲因此市權,誘脅群不肖,即建營三陽宮於嵩山、興泰宮於萬壽山,請太后歲臨幸,己與二張扈侍馳騁,竊威福自私云。工役鉅萬萬,百姓愁嘆。



 崇訓之尚主也,三思方輔政,中宗居東宮,欲寵耀其下,乃令具親迎禮。宰相李嶠、蘇味道等及沈佺期、宋之問諸有名士,造作文辭,慢洩相矜,無復禮法。中宗復位,擢崇訓駙馬都尉、太常卿,兼左衛將軍。三思進位司空、同中書門下三品,加實戶五百。固辭,進開府儀同三司。會降封,裁減實戶。俄以太后遺詔還所減,而封崇訓鎬國公。



 初,桓彥範等已誅二張,薛季昶、劉幽求勸並誅三思等,不從。翌日,三思因韋後潛入宮中,反易國政,數日而彥範等皆失柄,所斥去者悉還。詔群臣復循太后法。三思建言:「大帝封泰山,則天皇后建明堂,封嵩山,二聖之美不可廢。」帝韙其言,遂更名五縣曰乾封、合宮、永昌、登封、告成云。明年春,大旱,帝遣三思、攸暨禱乾陵而雨,帝悅。三思因主請復崇恩廟,昊、順二陵,皆置令丞。其黨鄭愔上《聖感頌》,帝為刻石。補闕張景源建言:「母子承業,不可言中興,所下制書皆除之。」於是天下名祠改唐興、龍興雲。補闕權若訥又言:「制詔如貞觀故事。且太后遺訓,母儀也;太宗舊章,祖德也。沿襲當自近者始。」帝褒答。是時,起球場苑中,詔文武三品分朋為都,帝與皇后臨觀。崇訓與駙馬都尉楊慎交注膏築場,以利其澤,用功不訾,人苦之。



 三思既私韋後,又與上官昭容亂,內忌節愍太子,即與主謀廢之。太子懼,故發羽林兵圍三思第,並崇訓斬之,殺其黨十餘人。



 時疾三思奸亂竊國,比司馬懿。其忌阻正人特甚,嘗曰:「我不知何等名善人,唯與我者殆是哉。」與宗楚客兄弟、紀處訥、崔湜、甘元柬相驅煽,王同皎、周憬、張仲之等不勝憤,謀殺之,為冉祖雍、宋之愻、李悛所白,皆坐死。因逮染五王,而崔湜遣周利貞就殺之,故祖雍與御史姚紹之等五人,號「三思五狗」。司農少卿趙履溫、中書舍人鄭愔、長安令馬構、司勛郎中崔日用、監察御史李心曳托其權,熏炙內外,其尤干政事者,天下語曰:「崔、冉、鄭,亂時政。」以爵賞自相崇樹,凡構大獄,污點善良,破壞其宗,天下為蕩然。始韋月將、高軫上疏,極言三思過惡,有司殺月將,逐軫惡地。黃門侍郎宋璟執奏,俄見斥。其權大抵如此。



 既死,帝為舉哀,廢朝五日,贈太尉,復封梁王,謚曰宣。追封崇訓魯王,謚曰忠。主以太子首祭三思柩。睿宗立,以父子皆逆節,斫棺暴尸,夷其墓。



 懿宗以司農卿爵為郡王,歷懷、洛二州刺史。神功元年,孫萬榮敗王孝傑兵,詔懿宗為神兵道大總管討之,而婁師德、沙吒忠義並為總管,兵凡二十萬,次趙州。懿宗聞賊且至,懼不知所出,欲棄軍走,或勸曰:「賊雖眾,無輜載,以鈔剽為命,若按兵老之,擊其歸,可成大功。」懿宗不暇計,退保相州,賊遂進屠趙州。後萬榮死,懿宗復與婁師德撫循河北,人有自賊中歸者,一切抵死,先剔取膽,乃殺之,血沫前,而舉動自如。始萬榮入寇也,別帥何阿小陷冀州,殺人無餘種,以懿宗暴忍似之,故號稱「兩何」,相語曰:「唯此兩何,殺人最多。」



 初,懿宗天授間受詔訊大獄,誅大臣王公,皆深排巧引,內刑塹中,無有脫者。其險酷雖周、來等不能繼也。神龍初,遷太子詹事,終懷州刺史。



 攸暨自右衛中郎將尚太平公主,拜駙馬都尉,累遷右衛大將軍。天授中,自千乘郡王進封定王,實封戶六百。遷麟臺監司祀卿。長安中,降王壽春,加特進。中宗時,拜司徒,復王定,加戶千,固辭,進開府儀同三司。延秀之誅,降楚國公。攸暨沈謹和厚,於時無忤,專自奉養而已。景龍中卒,贈太尉、並州大都督,還定王,謚曰忠簡。坐公主大逆,夷其墓。



 韋溫者,中宗廢後庶人從父兄也。後父玄貞,歷普州參軍事,以女為皇太子妃,故擢累豫州刺史。帝幽廬陵,玄貞流死欽州,妻崔為蠻首寧承所殺,四子洵、浩、洞、泚同死容州,後二女弟逃還京師。帝復政,是日詔贈玄貞上洛郡王、太師、雍州牧、益州大都督,溫父玄儼魯國公、特進、並州大都督。遣使者迎玄貞喪,詔廣州都督周仁軌討寧承,斬其首祭崔柩,官仁軌左羽林大將軍,汝南郡公。柩至,帝與後登長樂宮望而哭,贈酆王,謚文獻,號廟曰褒德,陵曰榮先,置令丞,給百戶掃除。贈洵吏部尚書、汝南郡王,浩太常卿、武陵郡,洞衛尉卿、淮陽郡,泚太僕卿、上蔡郡,並葬京師。



 溫初試吏,坐贓斥。神龍初,擢宗正卿,遷禮部尚書,封魯國公。弟湑,自洛州戶曹參軍事連拜左羽林大將軍,曹國公。後大妹嫁陸頌,進國子祭酒。仲妹嫁嗣虢王邕。湑子捷尚成安公主,溫從弟濯尚定安公主,並拜駙馬都尉,捷為右羽林將軍。景龍三年,溫以太子少保同中書門下三品,遙領揚州大都督。溫既見天下事在手,欲自殖以牢其權,引用友黨不相一,公卿雖畏伏,然溫無能,不如諸武兇而熾也。



 湑初兼脩文館大學士,時熒惑久留羽林,後惡之,方湑從至溫泉,後毒殺之以塞變,厚贈司徒、並州大都督。湑兄弟頗以文詞進,帝方盛選文章侍從,與賦詩相娛樂,湑雖為學士,常在北軍,無所造作。



 有富商抵罪,萬年令李令質按之。濯馳救,令質不從,毀於帝。帝召令質至,左右為恐,令質從容曰:「濯於賊非親,但以貨為請,濯雖勢重,不如守陛下法,死無恨。」帝釋不責。



 帝崩,後專政,畏有變,敕溫盡總內外兵,守省中;又以從子播、捷從弟璿、高嵩分領左右羽林軍。溫與宗楚客、武延秀等說後托圖讖,韋氏當受命,謀殺少帝,內憚相王、太平公主屬尊,欲先除之,然後發其謀。而玄宗兵夜起,將軍葛福順攻玄武門,入羽林,斬播、璿、高、嵩,梟首以徇,軍中相率而應,無敢後。後死,遲旦斬溫,分捕諸韋子弟,無少長皆斬。



 周仁軌者,京兆萬年人,後母族也。方為並州長史,殘酷嗜殺戮。異日,見堂下有斷臂,惡之,送於野,數昔往視,故在。是月,韋後敗,使者誅仁軌,刑人舉刀,仁軌承以臂,墯地乃悟。



 睿宗夷玄貞、洵墳墓,民盜取寶玉略盡。天寶九載,復詔發掘,長安尉薛榮先往視,塚銘載葬日月,與發塚日月正同,而陵與尉名合雲。



 王仁皎,字鳴鶴,玄宗廢后父也。景龍中,以將帥舉,授甘泉府果毅,遷左衛中郎將。帝即位,以後故,擢將作大匠,進累開府儀同三司,封祁國公,食戶三百。仁皎避職不事,委遠名譽,厚奉養,積媵妾貲貨而已。卒年六十九,贈太尉、益州大都督,謚昭宣。官為治葬。柩行,帝禦望春亭過喪。詔張說文其碑,帝為題石。



 子守一,與後孿生,帝微時與雅舊,後詔尚清陽公主。從討太平主有功,由尚乘奉御遷殿中少監、晉國公,累進太子少保,襲父爵,被遇良渥。後廢,貶柳州別駕,至藍田,賜死。守一沓墨無顧藉,財蓄巨萬,皆籍入於官。



 楊國忠,太真妃之從祖兄,張易之之出也。嗜飲博,數丐貸於人,無行檢,不為姻族齒。年三十從蜀軍,以屯優當遷,節度使張宥惡其人,笞屈之,然卒以優為新都尉。罷去,益困,蜀大豪鮮於仲通頗資給之。從父玄琰死蜀州,國忠護視其家,因與妹通,所謂虢國夫人者。裒其貲,至成都摴蒲,一日費輒盡,乃亡去。久之,調扶風尉,不得志。復入蜀,劍南節度使章仇兼瓊與宰相李林甫不平,聞楊氏新有寵,思有以結納之為奧助,使仲通之長安,仲通辭,以國忠見,乾貌頎峻,口辯給,兼瓊喜,表為推官,使部春貢長安。將行,告曰:「郫有一日糧,君至,可取之也。」國忠至,乃得蜀貨百萬,即大喜。至京師,見群女弟,致贈遺。於時虢國新寡,國忠多分賂,宣淫不止。諸楊日為兼瓊譽,而言國忠善摴蒲,玄宗引見,擢金吾兵曹參軍、閑廄判官。兼瓊入為戶部尚書兼御史大夫,用其力也。國忠稍入供奉,常後出,專主薄簿,計算鉤畫,分銖不誤,帝悅曰:「度支郎才也。」累遷監察御史。



 李林甫興韋堅等獄,欲危太子,獄事畏卻,以國忠怙寵,搏鷙可用,倚之使按劾。國忠乃慘文峭詆,逮系連年,誣蔑被誅者百餘族,度可以危太子者,先林甫意陷之,皆中所欲。林甫方深阻固位,陰為指向,故國忠乘以為奸,肆意無所憚。虢國居中用事,帝所好惡,國忠必探知其微,帝以為能,擢兼度支員外郎。遷不淹年,領十五餘使,林甫始惡之。



 天寶七載,擢給事中、兼御史中丞,專判度支。會三妹封國夫人,兄銛擢鴻臚卿,與國忠皆列棨戟,而第舍華僭,彌跨都邑。時海內豐熾,州縣粟帛舉巨萬,國忠因言:古者二十七年耕,餘九年食,今天置太平,請在所出滯積,變輕齎,內富京師。又悉天下義倉及丁租、地課易布帛,以充天子禁藏。明年,帝詔百官觀庫物,積如丘山,賜群臣各有差,錫國忠紫衣、金魚,知太府卿事。



 初,楊慎矜引王鉷為御史中丞,已而有隙。鉷挾國忠共劾慎矜,抵不道,誅。由是權傾中外。吉溫為國忠謀奪林甫政,國忠即誣奏京兆尹蕭炅、御史中丞宋渾,逐之,皆林甫所厚善,林甫不能救,遂結怨。鉷寵方渥,位勢在國忠右,國忠忌之,因邢縡事,構鉷誅死,己代為京兆尹,悉領其使。即窮劾支黨,引林甫交私狀,牽連左逮,數以聞,帝始厭林甫,疏薄之。



 先此,南詔質子閤羅鳳亡去,帝欲討之,國忠薦鮮於仲通為蜀郡長史,率兵六萬討之。戰瀘川,舉軍沒,獨仲通挺身免。時國忠兼兵部侍郎,素德仲通,為匿其敗,更敘戰功,使白衣領職。因自請兼領劍南,詔拜劍南節度、支度、營田副大使,知節度事。俄加本道兼山南西道採訪處置使,開幕府,引竇華、張漸、宋昱、鄭昂、魏仲犀等自佐,而留京師。帝再幸左藏庫,班齎百官。出納判官魏仲犀言:「鳳集通訓門。」門直庫西,有詔改為鳳皇門,進仲犀殿中侍御史,屬吏率以「鳳凰優」得調。俄拜國忠御史大夫,因引仲通為京兆尹,己兼領吏部。



 國忠恥雲南無功,知為林甫掎摭,欲自解於帝,乃使麾下請己到屯,外示憂邊,以合上旨,實杜禁言路,林甫果奏遣之。及辭,泣訴為林甫中傷者,妃又為言,故帝益親之,豫計召日。然國忠就道,惴惴不自安。帝在華清宮,驛追國忠還。林甫病已困,入見床下,林甫曰:「死矣,公且入相,以後事屬公!」國忠懼其詐,不敢當,流汗被顏。林甫果死,遂拜右相,兼文部尚書、集賢院大學士、監脩國史、崇賢館大學士、太清太微宮使,而節度、採訪等使、判度支,不解也。國忠已得柄,則窮擿林甫奸事,碎其家。帝以為功,封魏國公,固讓魏,徙封衛。



 國忠既以宰相領選,始建罷長名,於銓日即定留放。故事,歲揭版南院為選式,選者自通,一辭不如式,輒不得調,故有十年不官者。國忠創押例,無賢不肖,用選深者先補官,牒文謬缺得再通,眾議翕然美之。先天以前,諸司官知政事者,午漏盡,還本司視事,兵、吏部尚書、侍郎分案注擬。開元末,宰相員少,任益尊,不復視本司事。吏部銓,故常三注三唱,自春止夏乃訖。而國忠陰使吏到第,預定其員,集百官尚書省注唱,一日畢,以誇神明,駭天下耳目者。自是資格紛謬,無復綱序。虢國居宣陽坊左,國忠在其南,自臺禁還,趣虢國第,郎官、御史白事者皆隨以至。居同第,出駢騎,相調笑,施施若禽獸然,不以為羞,道路為恥駭。明年大選,因就第唱補,帷女兄弟觀之,士之醜野蹇傴者,呼其名,輒笑於堂,聲徹諸外,士大夫詬恥之。先是,有司已定注,則過門下,侍中、給事中按閱,有不可,黜之。國忠則召左相陳希烈隅坐,給事中在旁,既對注,曰:「已過門下矣。」希烈不敢異。侍郎韋見素、張倚與本曹郎趨走堂下,抱案牒,國忠顧女弟曰:「紫袍二主事何如?」皆大噱。鮮於仲通等諷選者鄭怤願立碑省戶下以頌德,詔仲通為頌,帝為易數字,因以黃金識其處。



 帝常歲十月幸華清宮,春乃還,而諸楊湯沐館在宮東垣,連蔓相照,帝臨幸,必遍五家,賞齎不訾計,出有賜,曰「餞路」,返有勞,曰「軟腳」。遠近饋遺閹稚、歌兒、狗馬、金貝,踵疊其門。



 國忠由御史至宰相,凡領四十餘使,而度支、吏部事自叢伙,第署一字不能盡,故吏得輕重,顯賕公謁無所忌。國忠性疏侻捷給,硜硜處決樞務,自任不疑,盛氣驕愎,百僚莫敢相可否,官屬悉苛督句剝相槊。又便佞,專徇帝嗜欲,不顧天下成敗。帝雅意事邊,故身調兵食,取習文簿惡吏任之,軍凡須索,快成其手,又不能省視也。始,李林甫紿帝天下無事,請巳漏出休,許之。文書填水奏,坐家裁決。既成,敕吏持案詣左相陳希烈聯署,左相不敢詰,署惟謹。至國忠時,韋見素代希烈,循以為常。它年,大雨敗稼,帝憂之,國忠擇善禾以進,曰:「雨不為災。」扶風太守房琯上郡災,國忠怒,遣御史按之。後乃無敢以水旱聞,皆前伺國忠意乃敢啟。子暄舉明經,不中,禮部侍郎達奚珣遣子撫往見國忠,國忠方朝,見撫喜。已而聞暄當黜,詬曰:「生子不富貴耶?豈以一名為鼠輩所賣!」珣大驚,即致暄高第。俄與珣同列,猶吒官不進。



 國忠雖當國,常領劍南召募使,遣戍瀘南,餉路險乏,舉無還者。舊,勛戶免行,所以寵戰功。國忠令當行者先取勛家,故士無鬥志。凡募法,願奮者則籍之。國忠歲遣宋昱、鄭昂、韋儇以御史迫促,郡縣吏窮無以應,乃詭設餉召貧弱者,密縛置室中,衣絮衣,械而送屯,亡者以送吏代之,人人思亂。尋遣劍南留後李宓率兵十餘萬擊閤羅鳳,敗死西洱河,國忠矯為捷書上聞。自再興師,傾中國驍卒二十萬,踦屨無遺,天下冤之。



 安祿山方有寵,總重兵於邊,偃蹇不奉法,帝護之,下莫敢言。國忠知終不出己下,又恃內援,獨暴發反狀,帝疑以位相媢,不之信。祿山雖逆久,以帝遇之厚,故隱忍,伺帝一日晏駕則稱兵。及見帝劈國忠,甚畏不利己,故謀日急。俄而祿山授尚書右僕射,帝恐國忠不悅,故冊拜司空。祿山還幽州,覺國忠圖己,反謀遂決。國忠令客何盈、蹇昂刺求反狀,諷京兆尹李峴圍其第,捕祿山所善李超、安岱、李方來、王岷殺之,貶其黨吉溫於合浦。祿山上書自陳,而條上國忠大罪二十,帝歸過於峴,貶零陵太守,以尉祿山意。國忠寡謀矜躁,謂祿山跋扈不足圖,故激怒之使必反,以取信於帝,帝卒不悟。乃建言:「請以祿山為平章事,追入輔政,以賈循為使,節度範陽,呂知誨節度平盧,楊光翽節度河東。」已草詔,帝使謁者輔璆琳覘祿山,未還,帝致詔坐側。而璆琳納金,固言不反。帝謂國忠曰:「祿山無二心,前詔焚之矣。」祿山反,以誅國忠為名,帝欲自將而東,使皇太子監國,謂左右曰:「我欲行一事。」國忠揣帝且禪太子,歸謂女弟等曰:「太子監國,吾屬誅矣。」因聚泣,入訴於貴妃,妃以死邀帝,遂寢。祿山既發範陽,嘆吒曰:「國忠頭來何遲?」



 哥舒翰守潼關,按兵守險,國忠聞欲反己,疑之,乃從中督戰,翰不得已出關,遂大敗,降賊。書聞,是日帝自南內移仗未央宮。國忠見百官,鯁咽不自勝。監察御史高適請率百官子弟及募豪桀十萬拒守,眾以為不可。初,國忠聞難作,自以身帥劍南,豫置腹心梁、益間,為自完計。至是,帝召宰相計事,國忠曰:「幸蜀便。」帝然之。明日遲昕,帝出延秋門,群臣不知,猶上朝,唯三衛彍騎立仗,尚聞刻漏聲。國忠與韋見素、高力士及皇太子諸王數百人護帝。右龍武大將軍陳玄禮謀殺國忠,不克。進次馬嵬,將士疲,乏食,玄禮懼亂,召諸將曰:「今天子震蕩,社稷不守,使生人肝腦塗地,豈非國忠所致!欲誅之以謝天下,云何?」眾曰:「念之久矣,事行身死,固所願。」會吐蕃使有請於國忠,眾大呼曰:「國忠與吐蕃謀反!」衛騎合,國忠突出,或射中其頞,殺之,爭啖其肉且盡,梟首以徇。帝驚曰:「國忠遂反耶?」時吐蕃使亦殲矣。御史大夫魏方進責眾曰:「何故殺宰相?」眾怒,又殺之。



 四子:暄、昢、曉、晞。暄位太常卿、戶部侍郎,聞亂,下馬蹶,弩眾射之,身貫百矢,乃踣。昢尚萬春公主,位鴻臚卿,陷賊見殺。曉奔漢中,為漢中王瑀搒死。晞及國忠妻裴柔同奔陳倉,為追兵所斬。柔,故蜀倡也,並坎而瘞。



 其黨翰林學士張漸、竇華,中書舍人宋昱,吏部郎中鄭昂,俱走山谷,民爭其貲,富埒國忠。昱戀貲產,竊入都,為亂兵所殺;餘坐誅。



 國忠本名釗,以圖讖有「卯金刀」,當位御史中丞時,帝為改今名。



 李翛,字翛,起寒賤,由莊憲太后婭婿得進,歷坊、絳二州刺史。無它才,為政粗辦。性纖巧,飾廚傳,結納閹寺,求善譽。憲宗以為才,拜伺農卿,進京兆尹,專聚斂以固恩寵,數譖毀近臣,一時側目。太后崩,詔翛為橋道置頓使,嗇官費,物物裁損為可喜者。梓宮至灞橋,從官多不得食。始議更造渭城門,計錢三萬,翛以為勞,不聽,使鑿軌道深之,柱危不支,方過喪而門壞,轀輬僅免,徹門乃得行。翛妄奏車軸折,山陵使李逢吉劾罔上,請免官。方帝用兵而翛屢有所獻,得不坐,才詔奪稟,逢吉持之,乃削銀青一階。翌日,加賜黃金。帝以浙西富饒,欲掊捃遺利,以翛為觀察使。被疾還京師。元和十四年卒,士有相賀者。



 鄭光,孝明皇太后弟也。會昌末,夢御大車載日月行中衢,光輝洪洞照六合,寤而占之,工曰:「君且暴貴。」不闋月,宣宗即位,光興民伍,拜諸衛將軍,遷累平盧軍節度使,徙河中、鳳翔,又賜鄠、雲陽二縣良田。大中四年,詔除其租賦,宰相言:「國常賦,窶人下戶不免,柰何以外戚廢法?」帝悟,追格前詔。俄封其妾為夫人,光曉帝意,還詔不敢拜,帝嘉之。七年,來朝,對延英,占奏俚近,帝失所望,不悅,留為右羽林統軍兼太子太保。太后言其家空短,帝厚賜金繒,終不復委方鎮。卒,贈司徒,詔罷三日朝,群臣奉慰。御史大夫李景讓曰:「禮,外祖父母、舅服小功五月,伯叔父若兄弟齊縗期,所以疏外密內也。王者不可使外戚強。按王、公主喪不過三日,光宜少降。」詔罷二日。



 子漢卿,終義昌軍節度使。



\end{pinyinscope}