\article{列傳第一百三十五 籓鎮魏博}

\begin{pinyinscope}

 安、史亂天下,至肅宗大難略平,君臣皆幸安,故瓜分河北地,付授叛將出發,護養孽萌,以成禍根。亂人乘之,遂擅署吏,以賦稅自私,不朝獻於廷。效戰國,肱髀相依,以土地傳子孫,脅百姓,加鋸其頸,利怵逆污,遂使其人自視猶羌狄然。一寇死,一賊生,訖唐亡百餘年,卒不為王土。



 當其盛時,蔡附齊連,內裂河南地,為合從以抗天子。杜牧至以「山東,王不得,不王;霸不得,不霸;賊得之,故天下不安」。又曰:



 厥今天下何如哉?干戈朽,斧鉞鈍,含引混貸,照育逆孽,殆為故常。而執事大人曾不歷算周思,以為宿謀,方且嵬岸抑揚,自以為廣大繁昌莫己若也。嗚呼!其不知乎,其俟蹇頓顛傾而後為之支計乎?且天下幾里,列郡幾所,自河以北,蟠城數百,角奔為寇,伺吾人憔悴,天時不利,則將與其朋伍駭亂吾民於掌股之上。今者及吾之壯,不圖擒取,乃偷處恬逸,以為後世子孫背脅疽根,此復何也?



 議者曰:倔強之徒,吾以良將勁兵為銜策,高位美爵充飽其腸,安而不橈,外而不拘,猶豢虎狼而不拂其心,則忿氣不萌,此大歷、貞元所以守邦也。何必疾戰焚煎吾民,然後為快也?



 愚曰:大歷、貞元之間,有城數十,千百卒夫,則朝廷貸以法,故於是闊視大言,自樹一家,破制削法,角為尊奢。天子不問,有司不呵;王侯通爵,越祿受之;覲聘不來,幾杖扶之;逆息虜胤,皇子嬪之。地益廣,兵益強,僭擬益甚,侈心益昌。土田名器,分劃大盡,而賊夫貪心,未及畔岸,淫名越號,走兵四略,以飽其志。趙、魏、燕、齊,同日而起,梁、蔡、吳、蜀,躡而和之,其餘混澒軒囂,欲相效者,往往而是。運遭孝武,前英後傑,夕思朝議,故能大者誅鉏,小者惠來。大抵生人油然多欲,欲而不得則怒,怒則爭亂隨之。是以教笞於家,刑罰於國,征伐於天下,裁其欲而塞其爭也。大歷、貞元之間反此,提區區之有,而塞無涯之爭,是以首尾指支,幾不能相運掉也。凡今者不知非此,而反用以為經,將見為盜者非止於河北而已。嗚呼!大歷、貞元守邦之術,永戒之哉!



 魏博傳五世,至田弘正入朝,十年復亂,更四姓,傳十世,有州七。成德更二姓,傳五世,至王承元入朝明年,王廷水奏反,傳六世,有州四。盧龍更三姓,傳五世,至劉總入朝,六月,硃克融反,傳十二世,有州九。淄青傳五世而滅,有州十二。滄景傳三世,至程權入朝,十六年而李全略有之,至其子同捷而滅,有州四。宣武傳四世而滅,有州四。彰義傳三世而滅,有州三。澤潞傳三世而滅,有州五。雖然,跡其由來,事有因藉,地之輕重,視人謀臧否歟!今取擅興若世嗣者,為《籓鎮傳》。若田弘正、張孝忠等,暴忠納誠,以屏王室,自如別傳云。



 田承嗣,字承嗣,平州盧龍人。世事盧龍軍,以豪俠聞。隸安祿山麾下,破奚、契丹,累功至武衛將軍。祿山反,與張忠志為賊前驅,陷河、洛。嘗大雪,祿山按行諸屯,至其營,若無人,已而擐甲列卒,閱所籍,不缺一人,祿山異其能,使守潁川。



 郭子儀平東都,承嗣以郡降,俄而復叛。安慶緒奔鄴,承嗣自潁川來,與蔡希德、武令榔合兵六萬,慶緒復振,抗王師。歲餘,史思明亂,承嗣又為賊導,及朝義敗,與共保莫州。僕固瑒追北,承嗣急,乃詐朝義使自求救幽州。承嗣守莫,因執賊妻息降於瑒,厚以金帛反間瑒將士。瑒慮下生變,即約降。承嗣詐疾不出,瑒欲馳入取之,承嗣列千刀為備,瑒不得志,承嗣重賂之以免。乃與張忠志、李懷仙、薛嵩皆詣僕固懷恩謝,願備行間。朝廷以二賊繼亂,州縣殘析,數大赦,凡為賊詿誤,一切不問。當是時,懷恩功高,亦恐賊平則任不重,因建白承嗣等分帥河北,賜鐵券,誓不死。拜承嗣莫州刺史,三遷至貝博滄瀛等州節度使,檢校太尉。



 承嗣沈猜陰賊,不習禮義。既得志,即計戶口,重賦斂,歷兵繕甲,使老弱耕,壯者在軍,不數年,有眾十萬。又擇趫秀強力者萬人,號牙兵,自署置官吏,圖版稅入,皆私有之。又求兼宰相,代宗以寇亂甫平,多所含宥,因就加同中書門下平章事,封雁門郡王,寵其軍曰天雄,以魏州為大都督府,即授長史,詔子華尚永樂公主,冀結其心。而性著兇詭,愈不遜。



 大歷八年,相衛薛嵩死,弟萼求假節,牙將裴志清逐萼,萼以眾歸承嗣。而帝自用李承昭為相州刺史,未至,承嗣使人訹吏士反,陽言救,實襲取之。帝遣使者諭罷兵,承嗣不奉詔,遣將盧子期取洺州,楊光朝取衛州,脅刺史薛雄亂,不從,屠其家,悉四州兵財以歸,擅置守宰。逼使者行礠、相,遣劉渾從之,陰使從子悅諷諸將詣使者剺面請承嗣為帥,使人不敢詰,於是厚賞請己者。帝乃下詔貶承嗣永州刺史,許一子從,悅及諸子皆逐惡地。詔河東節度使薛兼訓、成德李寶臣、幽州硃滔、昭義李承昭、淄青李正己、淮西李忠臣、永平李勉、汴宋田神玉等兵六萬掎角進,若承嗣不承命,聽在所討執,以軍法從事。其下霍榮國以礠降。李正己攻拔德州,李忠臣攻衛,築偃月壁河上。承嗣列將往往攜阻,殺數十人乃定。帝又遣御史大夫李涵督諸節度並力。承嗣遣裴志清等攻冀州,志清以兵附成德,承嗣悉眾圍之,為寶臣所逐,火輜重,歸於貝,計益窮,不知所出,遣其下郝光朝奉表請委身北闕下。又使悅與盧子期將萬人攻礠州,屯東山。宣慰使韓朝彩等固守,兼訓以萬騎屯西山,成德、幽州各遣兵救礠。時承昭以神策射生繼進,入河東壘。諸軍進討,數有功,頗賞,天子使中人多出御服、良馬、黃白金萬計勞賚,使人供帳高會。諸軍少懈,而正己、寶臣二軍會棗強,更相見。會正己軍輒引去,忠臣乃棄月壘,濟河屯陽武。承昭使成德、幽州兵循東山襲子期軍,自閉壁以驕賊。子期分步騎萬人環承昭壁,以兵四千乘高望麾而進。河東將劉文英、辛忠臣等決戰,而成德、幽州兵繞出子期後,於是圍解。更陣高原,諸將與承昭夾攻,大戰臨水,賊敗,尸旁午數里,斬九千級、馬千匹,執子期及將士二千三百,旗纛器甲鼓角二十萬。諸軍乘勝進,距礠十里,暮而舍。承昭舉燧,朝彩出銳兵鼓噪薄魏營,斬首五百,悅驚,率餘兵夜走,盡棄旗幕鎧仗五千乘。成德將王武俊以子期歸寶臣,寶臣方攻洺州,因以示城下,降之,復徇瀛州,瀛州亦降。得兵萬人,粟二十萬石,獻子期京師,斬之。



 天子遣中人勞寶臣,不為禮,寶臣乃貳,反攻硃滔,與承嗣和,承嗣與之滄州。正己又請天子許承嗣入朝。十一年,帝遣諫議大夫杜亞持節至魏受其降,許闔門還京師,赦魏博所管與更始。承嗣逗留不至。其秋,復略滑州,敗李勉兵。會李靈耀以汴州叛,詔忠臣、勉、河陽馬燧合討。靈耀求救於魏,承嗣使悅將兵三萬赴之,敗勉將杜如江、正己將尹伯良,死者殆半,乘勝屯汴北郛,與靈耀合。燧、忠臣逆擊,破之,悅脫身遁,斬獲數萬。靈耀東走,欲歸承嗣,為如江所禽,並魏將常準獻京師。明年,承嗣上書請罪,有詔復官爵,子弟皆仍故官,復賜鐵券。



 承嗣盜有貝、搏、魏、衛、相、礠、洺七州,而未嘗北面天子。凡再興師,會國威中奪,窮而復縱,故承嗣得肆奸無怖忌。十四年死,年七十五,贈太保。



 悅,蚤孤,母更嫁平盧戍卒,悅隨母轉側淄、青間。承嗣得魏,訪獲之,年十三,拜伏有禮,承嗣異之,委以號令,裁處皆與承嗣意合。及長,剽悍善鬥冠軍中,賊忍狙詐,外飭行義,輕財重施,以鉤美譽,人皆附之。承嗣愛其才,將死,顧諸子弱,乃命悅知節度事,令諸子佐之。帝因詔悅自中軍兵馬使、府左司馬擢留後,俄檢校工部尚書,為節度使。



 悅始招致賢才,開館宇,禮天下士,外示恭順,陰濟其奸。帝晚年尤寬弛,悅所奏請無不從。德宗立,不假借方鎮,諸將稍惕息。會黜陟使洪經綸至河北,聞悅養士七萬,輒下符罷其四萬歸田畝。悅即奉命,因大集將士,以好言激之曰:「而等籍軍中久,仰縑廩養父母妻子,今罷去,何恃而生?」眾大哭。悅乃悉出家貲給之,各令還部,自此,魏人德悅。



 及劉晏死,籓帥益懼,又傳言帝且東封泰山,李勉遂城汴州;而李正己懼,率兵萬人屯曹州,乃遣人說悅同叛。悅因與梁崇義等阻兵連和,以王侑、扈趯、許士則為腹心;邢曹俊、孟希祐、李長春、符璘、康愔為爪牙。建中二年,鎮州李惟岳、淄青李納求襲節度,不許,悅為請,不答,遂合謀同叛。會於邵、令狐峘等表汰浮圖,悅乃詐其軍曰:「有詔閱軍之老疾疲弱者。」繇是舉軍咨怨。悅與納會濮陽,納分兵佐悅。



 會幽州硃滔等奉詔討惟岳,悅乃遣孟希祐以兵五千助惟岳;別遣康愔以兵八千攻邢州;楊朝光以兵五千壁盧畽,絕昭義餉道。悅自將兵數萬繼進,又使朝光攻臨洺將張伾。伾固守,食且盡,賞賜不足,乃飾愛女示眾曰:「庫廩竭矣,願以此女代賞。」士感泣,請死戰,大破悅軍。有詔河東馬燧、河陽李芃與昭義軍救伾。三節度次狗、明二山間,未進。伾急,以紙為風鳶,高百餘丈,過悅營上,悅使善射者射之,不能及。燧營噪迎之,得書言「三日不解,臨洺士且為悅食。」燧乃自壺關鼓而東,破盧畽,戰雙岡,禽賊大將盧子昌而殺朝光,悅遁保洹水。



 於是曹俊為貝州刺史,乃承嗣時舊將,果而謀。悅未得志,召問計安出,對曰:「兵法,十則攻,今公以逆乾順,勢不敵也。宜留兵萬人屯崞阜口,以遏西師,則舉河北二十四州,惟公所命。今攻臨洺,糧竭卒老,不見其可。」悅所暱扈趯、孟希祐等皆訾短之,故悅不聽其言。燧等距悅軍三十里,築壘相望。悅與納合兵三萬,陣洹水。燧引神策將李晟夾攻悅,悅大敗,死傷二萬計,引壯騎數十夜奔魏,其將李長春拒關不內,以須官軍。而三帥頓不進。明日,悅得入,殺長春,持佩刀立軍門,流涕曰:「悅藉伯父餘業,與君等同休戚。今敗亡及此,不敢圖全。然悅久稽天誅者,特以淄青、恆冀子弟不得承襲,既弗能報,乃至用兵,使士民塗炭。悅正緣母老不能自剄,願公等斬悅首以取富貴,無庸俱死。」乃自投於地。眾憐,皆抱持之曰:「今士馬之眾,尚可一戰,事脫不濟,死生以之。」悅收淚曰:「諸公不以悅喪敗,誓同存亡,縱身先地下,敢忘厚意乎?」乃斷發為誓,將士亦斷發,約為兄弟;乃率富民大家財及府庫所有,大行賜與。而李再春及其子瑤以博州降,悅從兄昂以洺州降,燧等受之、悅皆族昂等家。悅自視兵械乏,眾單耗,懼,不知所出,復召曹俊與之謀。曹俊為整軍完壘以振士氣,群心復堅,後十餘日,燧等始進薄城下。



 未幾,王武俊殺惟岳,而深州降硃滔,滔分兵守之。天子授武俊恆州刺史,以康日知為深、趙二州觀察使。武俊恨賞薄,滔怨不得深州,悅知二將可間,乃曨路使王侑、許士則說滔曰:「司徒奉詔討賊,不十日,拔束鹿,下深州,惟嶽勢蹙,故王大夫能得逆首。聞出幽州日,有詔破惟岳得其地即隸麾下,今乃以深州與康日知,是朝廷不信於公也。且上英武獨斷,有秦皇、漢武風,將誅豪桀,掃除河朔,不使父子相襲。又功臣劉晏等皆旋踵破滅,殺梁崇義,誅其口三百餘,血丹漢江。今日破魏,則取燕、趙如牽轅下馬耳。夫魏博全則燕、趙安,鄙州尚書必以死報德。且合從連衡,救災恤患,不朽之業也,尚書願上貝州以廣湯沐,使侑等奉簿最孔目,司徒朝至魏則夕入貝,惟孰計之。」滔心素欲得貝,即大喜,使侑先還告師期。



 先是,詔武俊出恆冀粟三十萬賜滔,使還幽州,以突騎五百助燧軍。武俊懼悅破,將起師北伐,不肯歸粟、馬。滔因使王郅說武俊曰:「天子以君善戰,天下無前,故分散粟、馬以弱君軍。今若舉魏博,則王師北向,漳、滏勢危。誠能連營南旆,解田悅於倒縣,大夫之利也,豈特粟不出窖,馬不離廄,又有排危之義,聲滿天下。大夫親斷逆首,血衊釁衣袖,日知不出趙城,何功於國,而坐兼二州。河北士以不得深州為大夫恥。」武俊既得深,亦喜,即日使使報滔。



 於是滔率兵二萬屯寧晉,武俊以兵萬五千會之。悅恃救至,使康愔督兵與王師戰御河上,大敗,棄甲走城。悅怒,閉門不內,蹈藉死塹中者甚眾。其夏,滔、武俊軍至,悅具牛酒迎犒。燧等營魏河西,武俊、滔、悅壁河東,起樓櫓營中,兩軍相持,自秋汔冬。燧遣晟以兵三千,自邢、趙與張孝忠合攻涿、莫二州,以絕幽、薊路。



 悅重德滔,欲推為盟主而臣之。滔不敢當,乃更議如七國故事。悅國號魏,僭稱魏王,以府為大名府,署子為府留後;以扈趯為留守,許士則為司武,曾穆司文,裴抗司禮,封演司刑,並為侍郎;劉士素為內史舍人,張瑜、孫光佐為給事中,邢曹俊、孟希祐為左右僕射,田晁、高緬為征西節度使,蔡濟、薛有倫為虎牙將軍,高崇節知軍前兵馬,夏侯為兵馬使。晁以兵數千助李納守鄆。明年夏,滔屯河間,留大將馬寔以兵萬人戍魏。會硃泚亂,帝出奉天,燧還太原,武俊等皆罷,悅餞之,厚遺武俊、寔,官屬皆有贈。



 興元元年,滔自將兵欲南度河助泚,使王郅見悅計事曰:「頃大王在重圍,孤與趙刻日赴王難以全魏、貝。今秦帝已據關中,孤以步騎十萬與回紇趨東都相應接,王能從孤濟河,合勢以取大梁,孤得西收鞏、陜,與秦兵會,天下可定也。則王與趙王永無南慮,為脣齒之國,幸速計之。」是時,悅聞天子已赦罪,復官爵,心不欲行,重遽絕滔,陽遣薛有倫報滔如約。滔大喜,復使舍人李琯申固所言,悅猶豫,許士則諫曰:「冀王勇決權略,一世之雄也,殺懷仙,屠希彩,言術兄使如京師而奪之權,有恩者誅,同謀者覆,彼心腹渠可量哉?今大王之親不加泚,勇不加懷仙、希彩也,而念恩不已,拘攣匹夫義,出且見禽。彼得魏博,北聯幽薊,南入梁、鄭而與泚合,其理然也。大王不如偽許出迎,遣州縣具牛酒,至則以事自解,不可顧恩取禍也。」悅然之。先是,武俊陰約悅背滔,使相望。及聞滔要悅西,使田秀馳說悅曰:』聞大王欲從滔度河,為泚掎角,非也。方泚未盜京師時,滔為列國,且自高,如得東都,與泚連禍,兵多勢張,返制於豎於乎?今日天子復官赦罪,乃王臣,豈舍天子而北面滔、泚耶!願大王閉壘不出,武俊須昭義軍出,為王討之。」悅因秀還,具道其謀,而遣曾穆報滔。滔喜,自河間悉師而南,逾貝州,次清河,使人報悅,悅不至。進屯永濟,使王郅等督之曰:「王約出館陶與大王會,乃濟河。」悅良久曰:「始約從王,今舉軍持悅曰:『魏比困侵掠,供擬屈竭。』以悅日拊循,猶恐人且攜間,一日去城邑,朝出夕變,且何歸?不然,悅不敢背約。今遣孟希祐悉兵五千助王。」因使其屬裴抗、盧南史報命。滔怒罵曰:「逆虜前日求救,許我貝州,我不取;尊我為天子,我與同為王;教我遠來而不出。是賊不擊,尚何誅?」乃囚抗等,使馬寔取數縣,已而釋抗還之,悅兵不敢出,遂圍貝州。滔取武城,通德、棣,供軍饋,盡囚諸縣官吏,唯清陽不下,滔圍之。寔拔清平,殺五百人,俘男女貲財去。



 於是李抱真、武俊約出兵救魏。會有詔拜悅檢校尚書右僕射,封濟陽郡王,而給事中孔巢父持節宣勞。始悅阻兵凡四年,狂愎少謀,亟戰數北,死者什八,士苦之,且厭兵。既巢父至,莫不欣然。悅與巢父張飲,門階皆徹衛。至夜分,從弟緒與族人私語曰:「僕射妄起兵,幾赤吾族。以金帛厚天下,而不至兄弟。」或諫止之,緒怒,殺諫者,乃與左右逾垣入。悅方醉,寢酣。緒挺刃升堂,二弟諫止,緒斬之,因手刺悅,並殺基母妻。悅死,年三十四。比明,以悅命召許士則、蔡濟計事,至則殺之。劉忠信者,悅常使防督緒直寢門,緒呼曰:「忠信刺僕射,與扈趯反。」眾執之,語曰:「無之。」支已殊絕。



 緒字緒,承嗣第六子。悅待諸弟無所間,使緒主牙軍,而兇險多過,嘗笞勖之。悅於飲食衣服,儉嗇有節,緒常苦不足,頗怨望,故作難。悅既死,懼眾不附,以其徒數百將出奔,邢曹俊率眾追還。緒乃下令軍中曰:「我先王子,能立我者賞。」眾乃共推緒為留後,歸罪扈趯,斬其首以徇。復殺悅親信薛有倫等數十人,因巢父遣使者聽命天子。滔聞悅死,以兵五千合寔軍,進攻魏州。寔瀕王莽河壁,南距河,東抵博州,殺略甚眾。使人入魏招緒降。緒新篡,而寔圍且急,乃遣使以好言見滔,滔許與盟。曾穆勸緒絕滔,而緒部分亦定,乃乘城戰,武俊、抱真各脩好如悅時。詔即拜緒節度使。寔圍魏凡三月,滔敗走。



 貞元元年,以嘉誠公主降緒,拜駙馬都尉。李希烈平,以功賜一子八品官。緒猜忌,殺兄弟姑妹凡數人。兄朝,仕李納為齊州刺史。或言納將入之魏以代緒,緒厚賂納,且召朝,朝以死請不行,乃送之京師,過滑,緒將篡取之,賈耽以兵援接,乃免。



 累遷檢校尚書左僕射、常山郡王,又徙王雁門,實封五百戶,加同中書門下平章事。暴疾死,年三十三,贈司空。少子季安嗣。



 季安字夔。母微賤,公主命為己子,寵冠諸兄。數歲,為左衛胄曹參軍、節度副使。緒死時,年十五,匿喪觀變,軍中推為留後,因授節度使。除喪,加檢校尚書右僕射,進位檢校司空,俄同中書門下平章事。季安畏主之嚴,頗循禮法。及主薨,始自恣,擊鞠從禽,酣嗜欲,軍中事率意輕重,官屬進諫皆不納。



 會詔中尉吐突承璀以神策兵討王承宗,季安謀曰:「王師不跨河二十五年,今越魏伐趙,趙誠虜,魏亦虜矣,奈何?」或請以五千騎決除君憂。季安曰:「善,沮軍者斬!」時幽州劉濟將譚忠適使魏,聞之,入見季安曰:「往年王師取蜀取吳,算不失一,是宰相謀也。今伐趙,不使耆臣宿將而付中臣,不起天下甲而出秦甲,君知誰為之謀?此上自為謀,以誇服臣下。若師未叩趙,而先碎於魏,是上之謀不及下,且能不恥!既恥且怒,必任智畫,仗猛將,再舉涉河。鑒前之敗,必不越魏誅趙;校罪輕重,必不先趙後魏。是上不上,下不下,當魏而來也。」季安曰:「計安出?」忠曰:「王師入魏,君厚犒之。悉甲伐趙,而陰遺趙書曰:『魏若伐趙,為賣友;魏若與趙,為反君。賣友反君,魏不忍受。執事能弛陴鄣,遺一城,魏得持之獻捷天子以為符,此使魏北得以奉趙,西得以為臣,不世之利也。』趙不拒君,則魏安矣。」季安然之,遣大將率兵會王師伐承宗,糧餉自辦,取堂陽以報,加太子太保。



 有丘絳者,父時賓佐,與同府侯臧爭權,季安怒,斥為下縣尉,俄召還,先坎道左,既至,生瘞之。忍酷無忌憚,大抵如此,死年三十二,贈太尉。



 妻元誼女,召諸將立其子懷諫,最幼,不能事,政決於私奴蔣士則,數易置諸將,軍中怒,取田興為留後,所謂田弘正者,以懷諫歸第,殺士則等十餘人。季安既葬,送懷諫京師,授右監門衛將軍,寵錫蕃渥。緒弟縉、華顯於朝。



 縉字雲長,貞元十年入朝,授左驍衛將軍,封扶風郡公。元和中,拜夏綏銀節度使。始開元時,置宥州,扼寇路,久而廢,縉復城之。王師伐蔡,縉上橐它牛馬助軍。吐蕃寇豐州,縉設伏邀其歸,俘斬過當。入為左衛大將軍,李聽代之。聽劾縉盜沒軍糧四萬斛,強取羌人羊馬,故吐蕃得乘隙。貶衡王傅。俄而吐蕃又攻鹽州,貶房州司馬。長慶初,終左領軍衛將軍。華,太常少卿,尚永樂、新都二公主。



 田氏自承嗣至懷諫,四世,凡四十九年。



 史憲誠,其先奚也,內徙靈武,為建康人。三世署魏博將,祖及父爵皆為王。憲誠始以趫敢從父軍,田弘正討李師道,將先鋒兵四千濟河,拔城柵,師踵進,乘勝逐北,傅鄆堞。師道傳首,以功兼御史中丞。



 長慶二年,田布之自殺也,軍亂且囂。時憲誠為中軍兵馬使,頗言河朔舊事以搖其眾,眾乃逼還府,擅總軍務。穆宗以硃克融、王廷水奏方盜幽、鎮,未有以制,即以節度使授之。憲誠外詫王命,而陰結幽、鎮,依以自固。時李騕方亂,私與交通,數助請旄節,城馬頭,具舟黎陽,示將濟師者。會天子遣司門郎中韋文恪宣慰,憲誠見使者禮倨,言辭悖慢。俄聞斬騕,更恭謹謂文恪曰:「我本奚,如狗也,唯知識主,雖日加箠不忍離。」其譎獪類此。進檢校司空。



 與李全略為婚家,大和中,其子同捷反,潛以糧餉資之。文宗申約,使者相望,因進同中書門下平章事。憲誠使大將至京師偵事,作謾言自大,宰相韋處厚折其詐,遣去。憲誠懼,出兵從王師討之,復遣大將丌志沼率師二萬攻德州。時王廷水奏援同捷,陰誘志沼以利。志沼反,屯永濟,兵銳甚,諸鎮共御之。憲誠告急,天子詔義武李聽進討。於是志沼與廷水奏合兵劫貝州,為聽所敗,奔廷水奏。滄景平,憲誠不自安,請納地,進檢校司徒兼侍中,徙河中,封千乘郡公,以李聽代。



 初,憲誠將以族行,懼魏軍之留,問策於弟憲忠,憲忠教分相、衛,請置帥,因以弱魏。復請詔聽引軍聲圖志沼而假道清河,帝從之。憲誠因欲倚聽公去魏,及聽次清河,魏人驚,憲忠曰:「彼假道取賊,吾軍無負朝廷,何懼為?」乃稍安。然魏素聚兵清河,聽至,悉出其甲,將入魏,魏軍聞之懼,明日盡甲而出。聽按軍館陶不進。眾謂憲誠賣己,曰:「紿我以沽恩耶?」夜攻殺之,並監軍史良佐,推何進滔為帥以請,詔贈憲誠太尉,實大和三年。憲誠起,凡七年,死。



 何進滔,靈武人,世為本軍校。少客魏,委質軍中,事田弘正。弘正攻王承宗,夜以兵壓鎮州。承宗使健將以鐵冒面,引精騎千餘馳魏壁。進滔率猛士逐之,幾獲,鎮人大懼。從討李師道,以功兼侍御史。憲誠死,軍中傳言虖曰:「得何公事之,軍安矣!」進滔下令曰:「公等既迫我,當聽吾令。」眾唯唯。「孰殺前使及監軍者,疏出之。」凡斬九十餘人,釋脅從者。素服臨哭,將吏皆入吊。詔拜留後,俄進授節度使。居魏十餘年,民安之。進累檢校司徒、同中書門下平章事。開成五年死,贈太傅,謚曰定。



 子重順襲。武宗詔河陽李執方、滄州劉約諭朝京師,或割地自效,不聽命。時帝新即位,重起兵,乃授福王綰節度大使,以重順自副,賜名弘敬。帝討劉稹,加東面詔討使。弘敬倚稹相脣齒,無深入意,詔因稱其事母孝,在軍久,宜亟戰。弘敬亦自如。及王宰逾乾河攻澤州,天子慮稹起山東兵,命弘敬掎角塞其道,不奉詔。王元逵克邢州,攻上黨,弘敬不得已,乃出師。未幾,宰統陳許兵假道收磁州,弘敬懼,乃進戰,拔平恩,詔檢校尚書左僕射。澤潞平,加同中書門下平章事。懿宗初,兼中書令,封楚國公。咸通七年死,贈太師。



 子全暤襲,明年,拜節度使。平龐勛,以功遷檢校司空、同中書門下平章事。母喪,納所賜節,願行喪,詔不許。全暤年少好殺戮,下有小罪,鮮縱貰,人人危懼。後軍中相傳晙減糧帛,眾遂叛,全暤單騎遁,眾推韓君雄以總軍事,而殺全暤,實咸通十一年。詔贈太保。



 自進滔至全暤,凡三世,四十二年。



 懿宗更以普王為大使,擢君雄留後。君雄,魏州人。不五月,進副大使,三遷檢校司空。僖宗即位,進同中書門下平章事,賜名允中。死年六十一,贈太尉。



 子簡,襲留後。俄授節度使,進累檢校太尉、同中書門下平章事,封魏郡王。帝在蜀,天下亂,簡恃強完,欲拓地,覬望非常。時諸葛爽為黃巢守河陽,簡攻之,爽走,即戍以兵,以略邢、洺而歸。東攻鄆,鄆將曹存實出戰,敗死,其將硃宣率眾以守,久不下,爽乘其隙,復取河陽。簡還攻之,爽迎擊新鄉,簡大敗,樂彥禛以一軍先還,簡奔歸,疽發背死。彥禎代之。再世,凡十二年。



 彥禎者,亦魏人。簡時,歷博州刺史,下河陽有功,遷澶州。魏人立之,詔檢校工部尚書,領留後,進節度使,累加檢校尚書左僕射、同中書門下平章事。



 彥禎喜儒術,引公乘億、李山甫皆在幕府。嗣襄王煴之亂,彥禎使山甫往見鎮州王鎔,欲合幽、邢、滄諸鎮同盟拒賊,鎔厚謝,卒不克。彥禎見王室微,頗驕滿不軌,大興其眾,城魏周八十里,一月畢,人怨其殘。子從訓,資兇悖,劫王鐸,取其家,魏人不直。又聚亡命五百人,號「子將」,出入臥內,軍中藉藉惡之。從訓懼,易服奔近縣,彥禎即以為六州指揮使、相州刺史,輦兵械泉布,跡接於道,軍中益貳。彥禎常夢解佩帶覆而行,既寤,曰:「此神告我,下將有背乎?」已而軍亂,果囚彥禎,迫為桑門,尋殺之,推大將趙文弁總留後。



 從訓求救於硃全忠,全忠為起師,次內黃。從訓自相州以軍三萬傅城,文弁不敢出,眾懼,殺之,更推羅弘信帥軍。弘信出戰,從訓敗,裒餘眾壁洹水,弘信遣將程公佐擊斬之,梟首軍門,實文德元年。彥禎起,凡七年。



 羅弘信,字德孚,魏州貴鄉人。善騎射,狀貌雄偉。為裨將,主馬牧。魏有巫告弘信曰:「白頭老人使謝君,君當有是地。」弘信曰:「神欲危我耶?」文弁死,眾曰:「孰願主吾軍者?」弘信輒曰:「神命我矣!」眾環視,以為宜,遂立之。詔擢知留後,再遷節度使,加檢校司空、同中書門下平章事、豫章郡公。



 硃全忠討黃巢,餉粟三萬斛、馬二百匹。秦宗權亂,復詔弘信以粟二萬斛助軍,未輸,檢校工部尚書雷鄴來責粟,弘信素脅於牙軍,擅殺鄴。全忠以檄譙讓,弘信不敢報。大順初,全忠討太原李克用,遣將趙昌嗣見弘信假糧馬;又議屯邢、洺,假道相、衛,弘信不納。全忠使丁會、龐師古、葛從周、霍存等引萬騎度河,弘信壁內黃,凡五戰皆敗,禽大將馬武等,乃厚幣求和。方全忠圖河北,欲結納弘信,乃還兵。



 全忠攻兗鄆,硃宣求援於克用,遣李存信率兵救之,請道屯莘,其下侵魏芻牧,弘信不平。克用欲合鎮、定兵營河曲,搤魏、滑路,弘信馳告全忠,請禁游舸,絕往來。久之,魏人不至,全忠疑其紿,自將至滑州。弘信來告曰:「魏人未動者,正欲緩圖之。」全忠遂屯曹。太原將李瑭救宣,復壁莘,弘信厭其暴,而瑭溝壘自固。全忠遣使謂曰:「晉人志並河朔,師還,為公憂之。」弘信乃攻瑭,告全忠師期,全忠將趨滑為援,次封丘,而弘信已破瑭。克用怒,以兵掠魏博。全忠將侯言屯洹水,克用兵數求戰,言不敢出,全忠以葛從周代將。從周為暗竇,每克用兵至,輒出精卒薄戰,必捷。克用逾洹西北挑戰,從周大破之,禽其子落落,乃引去。然侵魏不已,大戰白龍潭,弘信敗,克用追薄魏門而還。弘信乃乞師全忠,全忠遣將壁洹水救魏。克用游兵剽相、魏,民死十九,弘信不堪其偪。光化元年,如全忠告亟。全忠復遣葛從周將兵追躡,拔洺州,執其刺史邢行恭;復攻邢,馬師素自拔走;遂圍礠州,袁奉韜自殺。不五日,取三州,斬首二萬級,禽其將百餘人,自是克用兵不出。



 始全忠亟討兗鄆,懼弘信貳,故歲時賂遺良厚。弘信每有饋答,全忠引其使北面拜受,兄事之,弘信以為厚己,故推心焉。



 進累檢校太師,守侍中,徙臨清郡王。光化元年死,年六十三,贈太師,追封北平王,謚曰莊肅。子紹威襲。



 紹威字端己。少有英氣,性精悍,吏事明辦。既領留後,昭宗即詔嗣父節度,加累檢校太尉,號「忠勤宣力致聖功臣」。幽州劉仁恭引兵攻鎮、冀,遂掠魏,紹威告急於全忠,全忠自將與仁恭戰內黃,日中,大破之,斬首三萬級。葛從周方守邢,亦敗其眾於魏縣。仁恭以眾十萬陷貝州,全忠使李思安屯內黃,從周悉軍入魏。仁恭攻魏,從周以五百騎出鬥,謂門者曰:「前有強敵,不可易。」命闔扉。士死戰,執仁恭將二人。仁恭使別將攻內黃,為思安所敗。從周乘勝破八壁,追北至臨清。仁恭乃還滄州,與李克用圖魏。紹威與全忠連兵伐滄州,從周攻拔德州,進薄浮陽。仁恭以兵至,監軍蔣玄暉請須其入壁,食盡可取。從周曰:「兵在機,機在上將,豈監軍所知!」逆戰老鴉堤,破之,斬首五萬,獲其將百餘人。又戰唐昌範橋,六遇輒勝。仁恭約和,乃還。紹威德全忠,故奉事愈固。全忠遷帝洛陽,命諸鎮治宮闕,而紹威營太廟,加侍中,封鄴王。



 魏牙軍,起田承嗣募軍中子弟為之,父子世襲,姻黨盤互,悍驕不顧法令,憲誠等皆所立,有不慊,輒害之無噍類。厚給稟,姑息不能制。時語曰:「長安天子,魏府牙軍。」謂其勢強也。紹威懲曩禍,雖外示優假,而內不堪。俄而小校李公佺作亂,不克,奔滄州。紹威乃決策屠翦,遣楊利言與全忠謀。全忠乃遣苻道昭將兵合魏軍二萬攻滄州,求公佺,又遣李思安助戰,魏軍不之疑。紹威子,全忠婿也,會女卒,使馬嗣勛來助葬,選長直千人納盟器,實甲以入。全忠自滑濟河,聲言督滄景行營。紹威欲出迎,假銳兵以入,軍中勸毋出而止。紹威遣人潛入庫,斷紘解甲,注夜,將奴客數百與嗣勛攻之,軍趨庫得兵,不可戰,因夷滅凡八千族,闉市為空。平明,全忠亦至,聞事定,馳入軍。魏兵在行者聞變,於是史仁遇保高唐,李重霸屯宗縣,分據貝、澶、衛等六州。仁遇自稱魏博留後,全忠解滄州兵以攻高唐,仁遇引眾走,為游騎所獲,支解之,進拔博、澶二州。李重霸走,俄斬其首,相、衛皆降。



 紹威雖除其偪,然勢弱,為全忠牽制,比州刺史矣,內悒悒悔恨。全忠兵在滄州,紹威主饋輓,自鄴至長蘆五百里,不絕於道。全忠還,紹威建元帥行府,極土木壯麗,全忠大悅。紹威間說曰:「邠、岐、太原皆狂譎,以復唐室為言。王宜自取神器,專天下之望。」全忠歸,乃受禪。



 紹威多聚書,至萬卷。江東羅隱工為詩,紹威厚幣結之,通譜系昭穆,因目己所為詩為「偷江東集」云。



 贊曰:田承嗣幾禽矣,李寶臣怒承倩而釋魏。建中之際,三將軍持銳躪血,功無成者。四叛連勢,兵結難作,天子不能守宗廟。傳及弘正,去污入朝,數年復亂,唐終不得魏。與夫豎刁亂齊,孰為輕重?



\end{pinyinscope}