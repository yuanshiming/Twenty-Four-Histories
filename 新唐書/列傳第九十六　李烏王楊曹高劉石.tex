\article{列傳第九十六 李烏王楊曹高劉石}

\begin{pinyinscope}

 李光進,其先河曲諸部,姓阿跌氏。貞觀中內屬,以其地為雞田州,世襲刺史三品。宣揚「黑、白、赤」三統的歷史循環論,否認社會的,隸朔方軍。



 光進與弟光顏少依舍利葛旃,葛旃妻,其女兄也。初,葛旃殺僕固瑒,歸河東辛云京,遂與光進俱家太原。以沈果稱。從馬燧救臨洺,戰洹水有功。歷前後軍牙門將、兼御史大夫、代州刺史。元和四年,王承宗、範希朝引師救易定,表光進為都將。時光顏亦至大夫,故軍中呼「大小大夫」。俄檢校工部尚書,為振武節度使,賜姓以光寵之;別詔光顏拜洺州刺史。弟兄榮冠當時。光進徙靈武,卒,年六十五,贈尚書左僕射。



 有至性,居母喪,三年不歸寢。光顏先娶,而母委以家事。及光進娶,母已亡,弟婦籍貲貯、納管鑰於姒,光進命反之,曰:「婦逮事姑,且嘗命主家事,不可改。」因相持泣,乃如初。



 光顏,字光遠。葛旃少教以騎射,每嘆其天資票健,己所不逮。長從河東軍為裨將,節度使馬燧謂曰:「若有奇相,終必光大。」解所佩劍贈之。討李懷光、楊惠琳,戰有功。從高崇文平劍南,數搴旗蹈軍,出入若神,益知名。進兼御史大夫,歷代、洺二州刺史。



 元和九年討蔡,以陳州刺史充忠武軍都知兵馬使。始逾月,擢本軍節度使,詔以其軍當一面。光顏乃壁溵水。明年,大破賊時曲。初,賊晨壓其營以陣,眾不得出,光顏毀其柵,將數騎突入賊中,反往一再,眾識光顏,矢集其身如蝟。子攬馬鞅諫無深入,光顏挺刃叱之,於是士爭奮,賊乃潰北。當此時,諸鎮兵環蔡十餘屯,相顧不肯前,獨光顏先敗賊。始,裴度宣慰諸軍還,為憲宗言:「光顏勇而義,必立功。」



 俄又與烏重胤破賊小溵河。初,都統韓弘約諸軍攻賊,賊先薄重胤壘,重胤中矛創甚,請救於光顏。光顏策賊既出,則小溵河之堡可乘,且重胤不可破。遣大將田穎、宋朝隱襲其城,夷之,賊失贅聚。弘怒不救重胤,違節度,取穎等將戮之,舉軍惜其材,光顏不敢拒。會中人景忠信至,知其然,即矯詔械系在所,馳以聞,有詔釋之。弘及光顏更以表言,帝謂弘使曰:「違都統令當死,但以功可贖,赦之以為後圖。」弘不悅。自是與弘有隙。



 十一年,屢困賊,遂拔凌雲柵。捷奏入,帝大悅,厚賚其使。進檢校尚書左僕射。十二年四月,敗賊於郾城,死者什三,數其甲凡三萬,悉畫雷公符、斗星,署曰:「破城北軍。」郾守將鄧懷金大恐,其令董昌齡因是勸懷金降,且來請曰:「城中兵父母妻子皆質賊,有如不戰而屈,且赤族。請公攻城,我舉火求援,援至,公迎破之,我以城下。」光顏許之。賊已北,昌齡奉偽印,懷金率諸將素服開門待。光顏入之,城自壞者五十版。



 弘素蹇縱,陰挾賊自重,且惡光顏忠力,思有以撓衊之。飭名姝,教歌舞、六博,襦衣屬珠琲,舉止光麗,費百鉅萬,遣使以遺光顏,曰:「公以君暴露於外,恭進侍者,慰君征行之勤。」光顏約旦日納焉。乃大合將校置酒,引使者以侍姝至,秀曼都雅,一軍驚視。光顏徐曰:「我去室家久,以為公憂,誠無以報德。然戰士皆棄妻子,蹈白刃,奈何獨以女色為樂?為我謝公:天子於光顏恩厚,誓不與賊同生!」指心曰:「雖死不貳。」因嗚咽泣下,將卒數萬皆感激流涕,乃厚賂使者還之,於是士氣益勵。



 裴度築赫連城於沲口,率輕騎觀之。賊以奇兵自五溝至,大呼薄戰,城為震壞,度危甚,光顏力戰卻之。先是,光顏策賊必至,密遣田布伏精騎溝下,扼其歸。賊敗,棄騎去,顛死溝中者千餘。由是賊悉銳士當光顏,而李愬得乘虛入蔡矣。董重質棄洄曲軍降愬,光顏躍馬入賊營大呼,眾萬餘人投甲請命。賊平,加檢校司空。入朝,召對麟德殿,賜與蕃渥,命宴其第,歸芻米二十車。



 帝討李師道,徙義成節度使,許以忠武兵自隨。不三旬,再敗賊濮陽,拔斗門,斬數千級。上言許、鄭兵合不可用,遂復鎮忠武。吐蕃入寇,徙邠寧軍。時虜毀鹽州城,使光顏復城之,亦以忠武兵從。初,田縉鎮夏州,以叨沓開邊隙,故黨項引吐蕃圍涇州,郝玭力戰破之。光顏聞賊至,料兵以赴,邠人慢言忷忷,騰噪不肯行。光顏為陳說大義,感慨流涕,聞者亦泣下,遽即路,虜走出塞。



 穆宗立,召還,賜開化里第,加同中書門下平章事。還軍,賚況不貲,以寵示群臣。俄徙鳳翔。帝將伐鎮州,復還忠武,又兼深冀行營節度使。宰相百官班餞,帝禦通化門臨送,賜珍器、良馬、玉帶。光顏提軍深入,而饋運不至,有詔以滄、景、德、棣州益之。光顏以宰相處置失宜,辭兼領,亦會赦王廷湊,復所治。李騕亂汴州,詔總軍出討,朝受命,暮即戎。翌日,拔尉氏。與汴人戰琵琶溝,未陣,薄之,賊走。騕平,進兼侍中。敬宗初,真拜司徒、河東節度。寶歷二年卒,年六十六,贈太尉,謚曰忠,賻賜良厚。及葬,文宗以其功高,復賜帛二千匹。



 光顏性忠義,善撫士,其下樂為用。許師勁悍,常為諸軍鋒,故數立勛。王仙芝、黃巢反,諸道告急,多請以助守。大校曹師罕以千五百人隸招討使宋威,張貫以四千人隸副使曾元裕。僖宗倚許軍以屏蔽東都,有請以為援,率不報。大將張自勉討雲南、黨項;龐勛亂,解圍壽州,戰淮口,以功累擢右威衛上將軍。至是表請討賊,詔乘傳赴軍,解宋州圍。威忌自勉成功,請以隸麾下,且欲殺之。宰相得其謀,不聽,以自勉代元裕。



 烏重胤,字保君,河東將承玭子也。少為潞牙將,兼左司馬。節度使盧從史奉詔討王承宗,陰與賊連。吐突承璀將圖之,以告重胤,乃縛從史。帳下士持兵合言雚,重胤叱曰:「天子有命,從者賞,違者斬!」士斂手還部無敢動。憲宗嘉其功,擢河陽節度使,封張掖郡公。



 帝討淮蔡,詔重胤以兵壓賊境,割汝州隸其軍,與李光顏相掎角。大小百餘戰,凡三年,賊平,再遷檢校司空,進邠國公。徙橫海軍,建言:「河朔能拒朝命者,蓋刺史失權,鎮將領軍能作威福也。使刺史得職,大帥雖有祿山、思明之奸,能據一州為叛哉?臣所管三州,輒還刺史職,各主其兵。」因請廢景州。法制脩立,時以為宜。



 討王廷湊也,出屯深州,方朝廷號令乖迕,賊浸不制,重胤久不敢進。穆宗以為觀望,詔杜叔良代之,以重胤為太子太保。長慶末,以檢校司徒、同中書門下平章事為山南西道節度使。召至京師,改節天平軍。文宗初,真拜司徒。李同捷請襲父位,帝方務靜安,授同捷兗海,以重胤耆將,兼節度滄景,以齊州隸軍。未幾卒,年六十七,贈太尉,謚懿穆。



 重胤出行伍,善撫士,與下同甘苦。蔡將李端降重胤,蔡人執其妻殺之,妻呼曰:「善事烏僕射!」得士心大抵如此。待官屬有禮,當時有名士如溫造、石洪皆在幕府。既歿,士二十餘人刲股以祭。



 子漢弘嗣爵。居母喪,奪為左領軍衛將軍,固辭。帝嘉許之。



 石洪者,字浚川,其先姓烏石蘭,後獨以石為氏。有至行,舉明經,為黃州錄事參軍,罷歸東都,十餘年隱居不出。公卿數薦,皆不答。重胤鎮河陽,求賢者以自重,或薦洪,重胤曰:「彼無求於人,其肯為我來邪?」乃具書幣邀闢,洪亦謂重胤知己,故欣然戒行。重胤喜其至,禮之。後詔書召為昭應尉、集賢校理。



 又有李珙者,世儒家,珙獨尚材武,有崖岸。嘗至澤潞見李抱真,欲署牙將,聞其使酒,不用。都將王虔休曰:「珙奇士,不能用,即殺之,無為它人得也。」抱真不納。虔休代節度,引為將。重胤禽從史,珙將救之,既聞謀出朝廷,乃止。重胤愛其才,討淮西也,表為行營都將。終右武衛上將軍。



 王沛,許州許昌人。少勇決,為節度使上官涚所器,妻以女,署牙門將。涚卒,它婿田偁脅涚子襲領其軍,謀殺監軍。沛知其計,密告之,支黨悉禽。德宗嘉美,即拜行軍司馬。而劉昌裔領節度,奏沛為監察御史,有詔護涚喪還京師。帝召見嘆息,以為功異等,嫌昌裔所請薄,謂沛曰:「吾意殊未厭,爾歸矣,方使別奏。」沛未至許,拜兼御史中丞。



 李光顏討吳元濟,奇沛風概,署行營兵馬使,使將勁兵別屯,數破賊有功。時詔書趣戰,諸將觀望,不敢度溵以壁。沛引兵五千夜濟合流,扼賊沖,遂城以居。於是河陽、宣武、太原、魏博等軍繼度,圍郾城。沛先結壘與賊對,蔡將鄧懷金遂降。蔡平,加兼大夫。復從光顏定淄青。及光彥鎮邠,詔分許兵往戍,沛又為都將,救鹽州,敗吐蕃,以功擢寧州刺史。徙陳州。



 李騕之亂,以忠武節度副使率師討騕,加檢校右散騎常侍,進拜兗海沂密節度使。是時新建府,俗獷驁,沛明示法制,搜閱以時,軍政大治。以檢校工部尚書徙忠武。太和元年卒,贈尚書右僕射。



 子逢,從父征伐,累功署忠武都知兵馬使。太和中,入為諸衛將軍。從劉沔、石雄破回鶻於天德,有士二千人未嘗戰,欲冒賞賜,逢不與。或為請之,答曰:「士奮死取賞,若無功而賞,何哉?」武宗以逢用法嚴,使宰相李德裕讓之,逢曰:「戰者,前踏白刃,不以法,人孰用命?」討劉稹也,為太原道行營將,領陳許兵七千屯翼城。稹平,加檢校右散騎常侍。後亦至忠武節度使云。



 楊元卿,史失其何所人。少孤,慷慨有術略。客江海上,時時高論,人謂狂生。吳少誠跋扈蔡州,元卿以褐衣見,署劇縣,俄召入幕府。又事少陽。每奏事至京師,頗為宰相李吉甫慰納。元卿還,與少陽言君臣大義以動其心,賊黨惡而共構之,判官蘇肇保救,乃免。然元卿陰橈少陽事,而輸款朝廷。及元濟擅襲節度,元卿欲困其財使不振,謬說曰:「先公吝於財,諸將至寒餒。府之有亡,我具知之。君若大賜將士以自固,又卑辭厚禮邀事諸鎮,則諸將悅,庶幾助我。吾為君持表見天子,安有不從者?」元濟許之。既至,則具條賊虛實,請敕諸道執元濟誅之。元濟覺,乃殺其妻並四子,圬為一堋射之,肇亦被害。



 憲宗拜元卿岳王府司馬,與李愬議僑置蔡州,以元卿為刺史,優納降附,壞賊黨與。元卿入見,願假度支錢及它奏請,不合旨;又裴度以諸將討蔡三年,功且成,若又以州與元卿,恐觖望生事,議格。更授光祿少卿。蔡平,超拜左金吾衛將軍。建言:「淮西多怪珍寶帶,往取必得。」帝曰:「我討賊,為人除害。賊平,我求得矣,焉用寶!止勿復言。」出為汾州刺史,復入為金吾。



 長慶初,鎮、魏易帥,元卿具道所以成敗事,穆宗久乃悟,賜白玉帶,擢涇原渭節度使。元卿墾發屯田五千頃,屯築高垣,牢鍵閉,寇至,耕者保垣以守。居六年,涇人德之。徙節河陽。何進滔亂魏博,元卿請自齎三月糧舉軍出討,文宗嘉美,加檢校司空。獻粟二十萬石,助天子經費。進光祿大夫。徙宣武軍。太和七年,以疾歸東都,授太子太保。卒,贈司徒。然性憸巧,所至聚斂,諧結權近,故累更方任云。



 子延宗,開成中為磁州刺史,與河陽兵謀逐帥自立。事敗,詔以元卿嘗毀家歸忠,全其宗,杖死延宗於京兆府,賜還田產。



 曹華,宋州楚丘人。始從宣武軍。縛亂將李乃送闕下,節度使董晉署為牙將。後避仇奔東都,會吳少誠叛,留守王翃署華襄城戍將。華浚隍埤堞,日與賊搏,數禽馘,賊憚之。憲宗初,累拜檢校右散騎常侍,召至京師,賜矛甲繒錦,還屯。拜寧州刺史,未行,屬吳元濟不受命,詔河陽懷汝節度使烏重胤討之,重胤請華自副。戰青陵城,賊大奔,拔凌雲柵,以功封陳留郡王。



 蔡平,進棣州刺史。州與鄆比,時賊略定滴河,華遽逐賊,斬二千級,復其縣。又募群盜可用者,貸死,補屯卒,使據孔道。賊至,輒擊卻之,不敢北。擢橫海節度副使。時朝廷披鄆為三鎮。其明年,兗海軍亂,殺觀察使王遂,詔華往代。視事三日,合軍大饗,幕甲士於廡,酒中,令曰:「天子以鄆人參別而戍,有轉徒勞,欲厚賞之。請鄆人右,州兵左。」既而出州兵,乃闔門大言曰:「天子有命,誅殺帥者!」甲起於幕,環之。凡斬千二百人,血流殷渠,赤氛冒門高丈餘。海、沂之人,重足屏息。



 華惡沂地褊,請治兗,許之。自李正己盜,齊、魯俗益污驁,華下令曰:「鄒、魯禮義鄉,不可忘本。」乃身見儒士,春秋祀孔子祠,立學官講誦,斥家貲佐贍給,人乃知教,成就諸生,仕諸朝。鎮人害田弘正,華亟請以本軍進討,不從。進華檢校工部尚書,就充節度使。



 李騕叛,以兵取宋州。華不待命,以兵逆擊,破之。騕平,檢校尚書右僕射,徙鎮義成軍。盜殺商賈,吏捕得,乃華嬖人。華怒,斷其頸以祭死者。卒,年六十九,贈左僕射。



 華雖出戎伍,而動必由禮,愛重士大夫,不以貴倨人,至廝豎必待以誠信,人以為難。



 高瑀,冀州蓚人。少沈邃,喜言兵。釋褐右金吾胄曹參軍,累遷陳、蔡二州刺史,入為太僕卿。忠武節度使王沛死,衛軍諸將多自謂得之。宰相裴度、韋處厚以瑀治陳、蔡素有狀,習軍中情偽,欲任之。會其軍表丐瑀,乃檢校左散騎常侍,領忠武節度使。自大歷後,擇帥悉出宦人中尉,所輸貨至鉅萬,貧者假貸富人,既得所欲,則椎膏血,倍以酬息,十常六七。及瑀有命,士相告曰:「韋、裴作相,天下無債帥。」州比水旱無年,瑀相地宜,築堤庸百八十里,時其鐘洩,民賴不饑。再加檢校尚書右僕射。六年,徙節武寧軍。以刑部尚書召,辭疾,拜太子少傅。不閱月,復詔節度忠武,卒於鎮,贈司空。



 瑀寬和,居官無赫然譽,所至稱治,士人懷之。



 劉沔,字子汪,徐州彭城人。父廷珍,以羽林軍扈德宗奉天,以戰功官左驍衛大將軍、東陽郡王。沔少孤,客振武,節度使範希朝署牙將。軍中大會,沔捉刀立堂下,希朝奇之,召謂曰:「後日必處吾坐。」希朝卒,入為神策將。太和末,遷累大將軍,擢涇原節度使,徙振武。開成三年,突厥劫營田,沔發吐渾、契苾、沙陀部萬人擊之,賊一轡無返者,悉頒所獲馬羊於戰卒,築都護府西北四壘。進檢校戶部尚書。



 武宗立,遷檢校尚書左僕射。回鶻寇天德,詔以兵據雲伽關,虜引去。會昌二年,又掠太原、振武,天子使兵部郎中李拭調兵食,因視諸將能否,拭獨稱沔,乃拜河東節度兼招撫回鶻使,進屯雁門關。虜寇雲州,沔擊之,斬七裨將,敗其眾。以還太和公主功,加檢校司空。議者恨其薄,又進金紫光祿大夫,賜一子官。虜殘眾走,詔沔追北,仍錄李靖賜之。軍還,次代州,歸義軍降虜三千,使隸食諸道,不受詔,據滹沱河叛,沔悉禽誅之。



 劉稹阻命,詔沔南討,屯榆社。沔素與張仲武不協,時方追幽州兵,故徙義成。會王宰逗留,宰相李德裕表沔鎮河陽,以滑兵二千壁萬善,居宰肘腋下,激之俾出軍。稹平,進檢校司徒,徙忠武節度使。以病改太子少保,不任謁,拜太子太傅致仕。卒,年六十五,贈司徒。



 石雄,徐州人,系寒,不知其先所來。少為牙校,敢毅善戰,氣蓋軍中。王智興討李同捷,收棣州,使雄先驅度河,鼓行無前。初,徐軍惡智興苛酷,謀逐之而立雄。智興懼變,因立功奏除州刺史,詔以為壁州刺史。智興由是殺雄素所善百餘人,誣雄陰結士搖亂,請以軍法論。文宗素知其能,不殺,流白州。徙為陳州長史。黨項擾河西,召雄隸振武劉沔軍,破羌有勞,帝難智興,久不擢。



 會昌初,回鶻入寇,連年掠雲、朔,牙五原塞下。詔雄為天德防禦副使,兼朔州刺史,佐劉沔屯雲州。沔召雄謀曰:「虜離散,當掃除久矣。國家以公主故,不欲亟攻。我若徑趨其牙,彼不及備,必棄公主走,我當迎主歸。有如不捷,吾則死之。」雄曰:「諾。」即選沙陀李國昌及契苾、拓拔雜虜三千騎,夜發馬邑,旦登振武城望之,見罽車十餘乘,從者硃碧衣,諜者曰:「公主帳也。」雄潛使喻之曰:「天子取公主,兵合,第無動。」雄穴城夜出,縱牛馬鼓噪,直搗烏介帳。可汗大駭,單騎走,追至殺胡山,斬首萬級,獲馬牛羊不貲,迎公主還。進豐州防禦使。



 武寧李彥佐討劉稹逗留,以雄為晉絳行營諸軍副使,助彥佐。是時,王宰屯萬善,劉沔屯石會關,顧望莫先進。雄受命,即勒兵越烏嶺,破賊五壁,斬獲千計,賊大震。雄臨財廉,每朝廷賜與,輒置軍門,自取一匹縑,餘悉分士伍,由是眾感發,無不奮。武宗喜曰:「今將帥義而勇罕雄比者。」就拜行營節度使,代彥佐。徙河中。稹危蹙,其大將郭誼密獻款,請斬稹首自歸。眾疑其詐,雄大言曰:「稹之叛,誼為謀主。今欲殺稹,乃誼自謀,又何疑?」雄以七千人徑薄潞,受誼降。進檢校兵部尚書,徙河陽。初,雄討稹,水次見白鷺,謂眾曰:「使吾射中其目,當成功。」一發如言。帝聞,下詔褒美。



 宣宗立,徙鎮鳳翔。雄素為李德裕識拔。王宰者,智興子,於雄故有隙。潞之役,雄功最多,宰惡之,數欲沮陷。會德裕罷宰相,因代歸。白敏中猥曰:「黑山、天井功,所酬已厭。」拜神武統軍。失勢怏怏卒。



 贊曰:世皆謂李愬提孤旅入蔡縛賊為奇功,殊未知光顏於平蔡為多也。是時,賊戰日窘,盡取銳卒護光顏,憑空堞以居,故愬能乘一切勢,出賊不意。然則無光顏之勝,愬烏能奮哉?



\end{pinyinscope}