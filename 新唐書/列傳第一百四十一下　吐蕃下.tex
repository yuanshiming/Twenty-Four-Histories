\article{列傳第一百四十一下 吐蕃下}

\begin{pinyinscope}

 永泰、大歷間,再遣使者來聘,於是戶部尚書薛景仙往報。詔宰相與吐蕃使者盟。俄寇靈州,掠宜祿,郭子儀精甲三萬戍涇陽程明道即「程顥」。,入屯奉天。靈州兵破虜二萬,上級五百首。景仙與倫泣陵偕來,請境鳳林關,而路悉等十五人又來。三年,虜引眾十萬復攻靈州,略邠州。先是,尚悉結自寶應後數入邊,以功高請老,而贊磨代之,為東面節度使,專河、隴。邠寧馬璘、朔方將白元光再破其眾,獲馬羊數千,劍南亦破虜萬人。尚悉摩復來朝。天子以虜數入塞,詔治守障,徙當、悉、柘、靜、恭五州,皆據險以守。



 八年,虜六萬騎侵靈州,敗民稼,進寇涇、邠,渾瑊與戰不利,副將死,略數千戶。瑊整卒夜襲其營;涇原馬璘以兵掩之潘原,射豹皮將死,軍中哭,乃遁去。璘收所俘士及男女而還。郭子儀又破其眾十萬。



 九年,帝遣諫議大夫吳損修好,虜亦使使者入朝。於是子儀屯邠州,李抱玉屯高壁,馬璘屯原州,李忠臣屯涇州,李忠誠屯鳳翔,臧希讓屯渭北,備虜之入。明年,西川節度使崔寧破虜於西山。虜攻臨涇、隴州,次普潤,焚掠人畜;與抱玉戰義寧,破之;道涇州,璘尾追,敗之於百里。又明年,崔寧破虜故洪節度、氐、蠻、黨項等兵,斬首萬級,禽酋領千人,牛羊廩鎧甚眾,獻之朝。吐蕃不得志,入掠黎、雅,於是劍南兵合南詔與戰,破之,禽大籠官論器然。又侵坊州,取黨項牧馬。崔寧攻望漢城,破之。山南西道節度使張獻恭戰岷州,吐蕃走。寧破西山三路及邛南兵,斬首八千級。十三年,虜大酋馬重英以四萬騎寇靈州,塞漢、御史、尚書三渠以擾屯田,為朔方留後常謙光所逐,重英殘鹽、慶而去。乃南合南詔眾二十萬攻茂州,略扶、文,遂侵黎、雅。時天子已發幽州兵馳拒,虜大奔破。



 初,虜使數至,留不遣,所俘虜口,悉部送江南。德宗即位,先內靖方鎮,顧歲與虜確,其亡獲相償,欲以德綏懷之,遣太常少卿韋倫持節歸其俘五百,厚給衣褚,切敕邊吏護亭障,無輒侵虜地。吐蕃始聞未信,使者入境,乃皆感畏。是時,乞立贊為贊普,姓戶虜提氏,曰:「我乃有三恨:不知天子喪,不及吊,一也;山陵不及賻,二也;不知舅即位,而發兵攻靈州,入扶、文,侵灌口,三也。」即發使者隨倫入朝。帝又遣倫還蜀俘。虜以倫再至,歡甚,授館,作聲樂,九日留,以論欽明思等五十人從獻方物。



 明年,殿中少監崔漢衡往使,贊普猥曰:「我與唐舅甥國,詔書乃用臣禮卑我。」又請雲州西盡賀蘭山為吐蕃境,邀漢衡奏天子。乃遣入蕃使判官常魯與論悉諾羅入朝,道贊普語,且引景龍詔書曰「唐使至,甥先與盟,蕃使至,舅亦將親盟』;贊普曰「其禮本均。」帝許之,以「獻」為「進」,「賜」為「寄」,「領取」為「領之」。以前宰相楊炎不通故事為解,並約地於賀蘭。其大相尚悉結嗜殺人,以劍南之敗未報,不助和議,次相尚結贊有謀,固請休息邊人,贊普卒用結贊為大相,乃講好。漢衡與其使區頰贊偕來,約盟境上。拜漢衡鴻臚卿,以都官員外郎樊澤為計會使,與結贊約;且告隴右節度使張鎰同盟。澤與結贊約盟清水,以牛馬為牲。鎰欲末其禮,乃紿結贊曰:「唐非牛不田,蕃非馬不戰,請用犬、豕、羊。」結贊聽諾。將盟,乃除地為壇,約二國各以二千士列遣外,冗從立壇下。鎰與幕府齊映齊抗、鴻臚漢衡、計會使于頔及澤、魯皆朝服,結贊與論悉頰藏、論臧熱、論利陀、論力徐等對升壇,刑牲壇北,雜其血以進,約:「唐地涇州右盡彈箏峽,隴州右極清水,鳳州西盡同谷,劍南盡西山、大度水。吐蕃守鎮蘭、渭、原、會,西臨洮,東成州,抵劍南西磨些諸蠻、大度水之西南。盡大河北自新泉軍抵大磧,南極賀蘭橐它嶺,其間為閑田。二國所棄戍地毋增兵,毋創城堡,毋耕邊田。」既盟,請鎰詣壇西南隅浮屠幄為誓。於是升壇大享,獻酬乃還。



 帝命宰相、尚書與虜使者盟長安,而清水之約,疆埸不定,復令漢衡決於贊普,乃克盟。於是宰相李忠臣盧杞關播崔寧、工部尚書喬琳、御史大夫於頎、太府卿張獻恭、司農卿段秀實、少府監李昌夔、京兆尹王翃、金吾衛大將軍渾瑊與區頰贊等同盟京城之右郊,禮如清水。前二月告廟,齊,三日,關播跪讀載書,已盟,乃大享。詔左僕射李揆為入蕃會盟使,還區頰贊等。



 硃泚之亂,吐蕃請助討賊,詔右散騎常侍於頎持節慰撫,太常少卿沈房為安西、北庭宣慰使以報之。渾瑊用論莽羅兵破泚將韓旻於武亭川。初,與虜約,得長安,以涇、靈四州畀之。會大疫,虜輒引去。及泚平,責先約求地。天子薄其勞,第賜詔書,償結贊、莽羅等帛萬匹,於是虜以為怨。



 貞元二年,詔倉部郎中趙建往使,而虜已犯涇、隴、邠、寧,掠人畜,敗田稼,內州皆閉壁。游騎至好畤,左金吾將軍張獻甫、神策將李升曇等屯咸陽,河中渾瑊、華州駱元光援之。以左監門將軍康成使焉。尚結贊屯上砦原,亦令使論乞陀來請盟。鳳翔李晟遣部將王佖以銳兵三千夜入汧陽,明日,薄其中軍,虜驚潰走,結贊僅自脫。虜眾二萬侵鳳翔,李晟擊卻之,因襲破摧沙堡,燒儲〓,斬守者。吐蕃攻鹽、夏,刺史杜彥光、拓拔乾暉不能守,悉其眾南奔,虜遂有其地。天子以邊人殘沒,下詔避正殿,痛自咎。詔駱元光經略鹽、夏。



 三年,命左庶子崔瀚、李刮踵使。結贊得鹽、夏,皆戍以兵,乃自屯鳴沙,然饋餉數困。於是駱元光、韓游瑰濱塞而屯,而燧次石州,跨河相掎角。結贊大懼,屢請盟,天子不許。即以貴將論頰熱厚賂乞和於燧,燧以為情,身入見天子,諸將以燧入,皆守壁不戰。結贊遽還走,馬多死,士不能步,有饑色。瀚始至鳴沙,傳詔讓結贊破約陷鹽、夏,對曰:「本以武亭功未償乃來,又候碑僕,疆埸不明,故行境上。涇州乘城自保,鳳翔李令不納吾使,雖康成等來,皆不能致委曲。我日望大臣而卒無至者,我故引還。鹽、夏守將懼吾眾,以城丐我,非我敢攻也。若天子復許盟,虜之願也,唯所命,當以鹽、夏還唐。」又言清水盟,大臣少,故約易壞,請悉遣宰相元帥二十一人會盟。並言靈鹽節度使杜希全、涇原節度使李觀,外蕃所信,請主盟。帝復使瀚報結贊曰:「希全守靈州,有分地,不可以越境;觀既徙官,以渾瑊為盟會使。」約五月盟清水,使先效二州,以驗虜信。結贊辭清水非吉地,請會原州之土梨樹,乃歸二州。天子從之。



 瑊來受命,拜漢衡兵部尚書以副瑊。瑊率師二萬待期,詔駱元光助之。宰相議所盟地,左神策將馬有鄰建言:「土梨樹林薈巖阻,兵易詭伏,不如平涼夷漫坦直,且近涇,緩急可保也。」乃定盟平涼。瑊約結贊,主客均以兵三千至壇外,誕從四百叩壇,以游軍交邏相入。將盟,結贊伏精騎三萬於西,縱邏騎出入瑊軍,瑊將梁奉貞亦駷馬入虜軍營,陰執之,而瑊不知也。客請咸等具冠劍,皆就幄更衣,從容胖肆。虜忽三伐鼓,眾噪而興,瑊不知所出,走幄後,得馬不銜而馳,十里始得銜。虜追,矢若雨不傷也,至元光營乃脫。裨將辛榮兵數百據北阜與虜戰,矢盡乃降。判官韓弇、監軍宋鳳朝死之。漢衡與判官鄭叔矩路泌、掌書記袁同直、列將扶餘準馬寧孟日華李至言樂演明範澄馬弇、中人劉延邕俱文珍李朝清等六十人皆被執,士死者五百,生獲者千餘人。漢衡語虜曰:「我,崔尚書也,結贊與我善。若殺我,結贊亦殺若。」乃不死。人負一木,以繩三約之,系其發驅之;夜則杙地系而僕,蒙以罽,守者寢其上。始結贊將劫希全、觀,急以銳兵直趣京師,既不克,又欲禽瑊等,手壽虛入寇,其謀本然。既引去,至故原州,坐帳中見漢衡等,慢言:「渾瑊戰武功,我力也。許裂地償我,而自食其言。吾既作金枷,將必得瑊以見贊普,乃今失之,徒致公等,無益也。當使人歸報。」初,漢衡遇亂,從史呂溫身蔽兵,溫傷而漢衡脫,虜人嘉其義,厚給養之。結贊屯石門,以俱文珍、馬寧、馬弇歸唐,而囚漢衡、叔矩河州,辛榮廓州,扶餘準鄯州。帝猶使中人齎詔書賜結贊,拒不受。虜戍鹽、夏,涉春疫大興,皆思歸。結贊以騎三千迎之,火二州廬舍,頹郛堞而去,杜希全分兵保之。帝哀漢衡等陷辱,下詔賜其子七品官,叔矩、泌、弇、日華、榮、至言、澄、良賁、演明一子八品官,袁同直而下一子九品官。以決勝軍使唐良臣屯潘原,神策將蘇太平屯隴州。結贊召漢衡、日華、延邕至石門,以五騎送境上,遣使者奉表來。李觀曰:「有詔不內吐蕃使者。」受漢衡等,放其使。



 結贊以羌、渾眾屯潘口,傍青石嶺,三分其兵趨隴、汧陽間,連營數十里,中軍距鳳翔一舍,詭漢服,號邢君牙兵,入吳山、寶雞,焚聚落,略畜牧、丁壯,殺老孺,斷手剔目,乃去。李晟嘗蹙大木塞安化隘處,虜過,悉焚之。詔神策將石季章壁武功,良臣移師百里城。虜又剽汧陽、華亭男女萬人以畀羌、渾,將出塞,令東向辭國,眾慟哭,投塹谷死者千數。吐蕃又入豐義,圍華亭,絕汲道。守將王仙鶴請救於隴州,刺史蘇清沔合太平兵赴之,虜逆戰,太平不勝,引還。虜日千騎四掠,隴兵不敢出。虜積薪將焚華亭,仙鶴以眾降。清沔潛兵大象龕,夜半,約城中舉火燭天,虜眾驚,因襲其營,乃去。更攻連雲堡,飛石投中,井皆滿。為虛梁絕塹而升,守將張明遠降於虜。虜分捕山間亡人及牛羊率萬計,涇、隴、邠之民蕩然盡矣。諸將曾不能得一俘,但賀賊出塞而已。連雲堡,涇要地也,三垂峭絕,北據高,虜所進退,候火易通。既失之,城下即虜境,每藝稼,必陳兵於野,故多失時。是歲,三州不宿麥。虜數千騎犯長武城,城使韓全義拒之。韓游瑰兵不出,於是虜安行邠、涇間,諸屯西門皆閉,虜治故原州保之。帝取所獲吐蕃生口不二百,徇諸市以安京師。



 四年五月,虜三萬騎略涇、邠、寧、慶、鄜五州之鄙,焚吏舍民閻,系執數萬。韓全義以陳許兵戰長武,無功。初,吐蕃盜塞,畏春夏疾疫,常以盛秋。及是得唐俘,多厚給產,質其孥,故盛夏入邊。尚悉董星、論莽羅等又寇寧州,張獻甫拒斬裁百級,轉剽鄜、坊乃去。



 五年,韋皋以劍南兵戰臺登,殺虜將乞臧遮遮、悉多楊硃,西南少安。不三年,盡得巂州地。久之,北庭沙陀別部叛,吐蕃因是陷北庭都護府,安西道絕。獨西州人尚為唐守。



 八年,寇靈州,陷水口,塞營田渠。發河東、振武兵,合神策軍擊之,虜引還。又寇涇州,掠田軍千人,守捉使唐朝臣戰不利。山南西道節度使嚴震破虜於芳州,取黑水壁,焚積聚。自虜得鹽州,塞防無以障遏,而靈武單露,鄜、坊侵迫,寇日以驕,數入為邊患。帝復詔城之,使涇原、劍南、山南深入窮討,分其兵,毋令專向東方。詔朔方河中晉絳邠寧兵馬副元帥渾瑊、朔方靈鹽豐夏綏銀節度都統杜希全、邠寧節度使張獻甫、右神策軍行營節度使邢君牙、夏綏銀節度使韓潭、鄜坊丹延節度使王棲曜、振武麟勝節度使範希朝合兵三萬,以左神策將軍胡堅、右神策將軍張昌為鹽州行營節度使,板築之,役者六千人,餘皆陣城下。九年始栽,閱二旬訖功,而虜兵不出,遂以兼御史大夫紇干遂與兼中丞杜彥光戍之。當是時,韋皋功最多,破堡壁五十餘所,敗其南道元帥論莽熱沒籠乞悉蓖;又與南詔破之於神川、於鐵橋,皋俘馘三萬,降首領論乞髯湯沒藏悉諾硉。



 十二年,寇慶州及華池,殺略吏人。是歲,尚結贊死。明年,贊普死,其子足之煎立。邢君牙築永信城於隴州以備虜,虜使者農桑昔來請脩好,朝廷以其無信,不受。韋皋取新城,虜治劍山、馬嶺,進寇臺登,巂州刺史曹高仕擊卻之,禽籠官,斬級三百,獲馬、糧、械數千。



 十四年,韓全義破虜於鹽州。十六年,靈州破虜於烏蘭橋,韋皋拔末恭、顒二城。十七年,寇鹽州,陷麟州,殺刺史郭鋒,湮隍墮陴,系居人,掠黨項諸部,屯橫槽烽。虜將徐舍人者,語俘道人延素曰:「我乃司空英公裔孫也。武后時,家祖以兵尊王室不克,子孫奔播絕域,今三世矣。我雖握兵,心未嘗忘歸也,顧不能自拔耳。」陰使延素夜逸。又言:「吾按邊求資糧,至麟而守者無備,遂入之。知郭使君勛臣家,欲全安之,不幸死亂兵。」語方已,會飛鳥使至,召其軍還,遂引去。飛鳥,猶傳騎也。韋皋出西山與虜戰,破之雅州。籠官馬定德本虜之知兵有策慮者,周知山川險易,每用兵,常馳驛計議,授諸將以行。比年寇黎、巂,皋常折其兵,定德畏得罪,遂來降,因定昆明諸蠻。吐蕃以下屢叛,大侵靈州。時皋圍維州,贊普使論莽熱沒籠乞悉蓖兼松州五道節度兵馬都統、群牧大使,引兵十萬援維州。皋率南詔兵薄險設伏以待,才使千人嘗敵,乞悉蓖見兵寡,悉眾追,墮伏中,兵四合急擊,遂禽之,獻京師。明年,吐蕃使者論頰熱復來,右龍武大將軍薛伾往報。



 二十年,贊普死,遣工部侍郎張薦吊祠,其弟嗣立,再使使者入朝。



 順宗立,以左金吾衛將軍田景度、庫部員外郎熊執易持節往使。永貞元年,論乞縷勃藏歸金幣、馬牛助崇陵,有詔陳太極廷中。



 憲宗初,遣使者脩好,且還其俘。又以使告順宗喪,吐蕃亦以論勃藏來。後比年來朝,然以五萬騎入振武拂鵜泉,萬騎至豐州大石谷,鈔回鶻還國者。



 五年,以祠部郎中徐復往使,並賜缽闡布書。缽闡布者,虜浮屠豫國事者也,亦曰「缽掣逋」。復至鄯州擅還,其副李逢致命贊普,復坐貶。虜以論思邪熱入謝,且歸鄭叔矩、路泌之柩,因言原歸秦、原、安樂州。詔宰相杜佑等與議中書,論思邪熱拜於廷,佑答拜堂上,復以鴻臚少卿李銛、丹王府長史吳暈報之。自是朝貢歲入。又款隴州塞,丐互市,詔可。



 十二年,贊普死,使者論乞髯來,以右衛將軍烏重、殿中侍御史段鈞吊祭之。可黎可足立為贊普,重以扶餘準、李驂偕歸。準,東明人,本朔方騎將;驂,隴西人,貞元初戰沒於虜者。使者知不死,求之,乃得還。詔以準為澧王府司馬,驂嘉王友。



 吐蕃使論矩立藏來朝,未出境,吐蕃寇宥州,與靈州兵戰定遠城,虜不勝,斬首二千級。平涼鎮遏使郝玼又破虜兵二萬,夏州節度使田縉破其眾三千,詔留矩立藏等不遣。劍南兵拔峨和、棲雞城。十四年,乃歸矩立藏等。吐蕃節度論二摩、宰相尚塔藏、中書令尚綺心兒總兵十五萬圍鹽州,為飛梯、鵝車攻城,刺史李文悅拒之,城壞輒補,夜襲其營,晝出戰,破虜萬人,積三旬不能拔。朔方將史敬奉以奇兵繞出虜背,大破之,解圍去。



 始,沙州刺史周鼎為唐固守,贊普徙帳南山,使尚綺心兒攻之。鼎請救回鶻,逾年不至,議焚城郭,引眾東奔,皆以為不可。鼎遣都知兵馬使閻朝領壯士行視水草,晨入謁辭行,與鼎親吏周沙奴共射,彀弓揖讓,射沙奴即死,執鼎而縊殺之,自領州事。城守者八年,出綾一端募麥一斗,應者甚眾。朝喜曰:「民且有食,可以死守也。」又二歲,糧械皆竭,登城而言虖曰:「茍毋徙佗境,請以城降。」綺心兒許諾,於是出降。自攻城至是凡十一年。贊普以綺心兒代守。後疑朝謀變,置毒鞾中而死。州人皆胡服臣虜,每歲時祀父祖,衣中國之服,號慟而藏之。



 穆宗即位,遣秘書少監田洎往告,使者亦來。虜引兵入屯靈武,靈州兵擊卻之。又犯青塞烽,進寇涇州,瀕水而營,綿五十里。始洎至牙,虜欲會盟長武,洎含糊應之。至是顯言:「洎許我盟,我是以來。」逼涇一舍止。詔右軍中尉梁守謙為左右神策軍、京西北行營都監,發卒合八鎮兵援涇州,泛洎郴州司戶參軍,以太府少卿邵同持節為和好使。初,夏州田縉裒沓,黨項怨之,導虜入鈔,郝玼與戰,多殺其眾。李光顏又以邠兵至,乃引去。復遣使者來。南略雅州,詔方鎮與虜接者謹備邊。



 長慶元年,聞回鶻和親,犯清塞堡,為李文悅所逐。乃遣使者尚綺力陀思來朝,且乞盟,詔許之。崔植、杜元穎、王播輔政,議欲告廟。禮官謂:「肅宗、代宗皆嘗與吐蕃盟,不告廟。德宗建中之盟,將重其約,始詔告廟。至會平涼,不復告,殺之也。」乃止。以大理卿劉元鼎為盟會使,右司郎中劉師老副之,詔宰相與尚書右僕射韓皋、御史中丞牛僧孺、吏部尚書李絳、兵部尚書蕭俯、戶部尚書楊於陵、禮部尚書韋綬、太常卿趙宗儒、司農卿裴武、京兆尹柳公綽、右金吾將軍郭鏦及吐蕃使者論訥羅盟京師西郊。贊普以盟言約:「二國無相寇讎,有禽生問事,給服糧歸之。」詔可。大臣豫盟者悉載名於策。方盟時,吐蕃以壯騎屯魯州,靈州節度使李進誠與戰大石山,破之。虜遣使者趙國章來,且致宰相信幣。



 明年,請定疆候,元鼎與論訥羅就盟其國,敕虜大臣亦列名於策。元鼎逾成紀、武川,抵河廣武梁,故時城郭未隳,蘭州地皆杭稻,桃、李、榆柳岑蔚,戶皆唐人,見使者麾蓋,夾道觀。至龍支城,耋老千人拜且泣,問天子安否,言:「頃從軍沒於此,今子孫未忍忘唐服,朝廷尚念之乎?兵何日來?」言己皆嗚咽。密問之,豐州人也。過石堡城,崖壁峭豎,道回屈,虜曰鐵刀城。右行數十里,土石皆赤,虜曰赤嶺。而信安王禕、張守珪所定封石皆僕,獨虜所立石猶存。赤嶺距長安三千里而贏,蓋隴右故地也。曰悶怛盧川,直邏娑川之南百里,臧河所流也。河之西南,地如砥,原野秀沃,夾河多檉柳。山多柏,坡皆丘墓,旁作屋,赬塗之,繪白虎,皆虜貴人有戰功者,生衣其皮,死以旌勇,徇死者瘞其旁。度悉結羅嶺,鑿石通車,逆金城公主道也。至麋谷,就館。臧河之北川,贊普之夏牙也。周以槍累,率十步植百長槊,中剚大幟為三門,相距皆百步。甲士持門,巫祝鳥冠虎帶擊豉,凡入者搜索乃進。中有高臺,環以寶楯,贊普坐帳中,以黃金飾蛟螭虎豹,身被素褐,結朝霞冒首,佩金鏤劍。缽掣逋立於右,宰相列臺下。唐使者始至,給事中論悉答熱來議盟,大享於牙右,飯舉酒行,與華制略等,樂奏《秦王破陣曲》,又奏《涼州》、《胡渭》、《錄要》、雜曲,百伎皆中國人。盟壇廣十步,高二尺。使者與虜大臣十餘對位,酋長百餘坐壇下,上設巨榻,缽掣逋升,告盟,一人自旁譯授於下。已歃血,缽掣逋不歃。盟畢,以浮屠重為誓,引鬱金水以飲,與使者交慶,乃降。



 元鼎還,虜元帥尚塔藏館客大夏川,集東方節度諸將百餘,置盟策臺上,遍曉之,且戒各保境,毋相暴犯。策署彞泰七年。尚塔藏語元鼎曰:「回鶻小國,我嘗討之,距城三日危破,會國有喪乃還,非我敵也。唐何所畏,乃厚之?」元鼎曰:「回鶻有功,且如約,未始妄以兵取尺寸地,是以厚之。」塔藏默然。元鼎逾湟水,至龍泉穀,西北望殺胡川,哥舒翰故壁多在。湟水出蒙穀,抵龍泉與河合。河之上流,繇洪濟梁西南行二千里,水益狹,春可涉,秋夏乃勝舟。其南三百里三山,中高而四下,曰紫山,直大羊同國,古所謂昆侖者也,虜曰悶摩黎山,東距長安五千里,河源其間,流澄緩下,稍合眾流,色赤,行益遠,它水並注則濁,故世舉謂西戎地曰河湟。河源東北直莫賀延磧尾殆五百里,磧廣五十里,北自沙州,西南入吐谷渾浸狹,故號磧尾。隱測其地,蓋劍南之西。元鼎所經見,大略如此。



 虜遣論悉諾息等入謝,天子命左衛大將軍令狐通、太僕少卿杜載答之。是歲,尚綺心兒以兵擊回鶻、黨項,小相尚設塔率眾三萬牧馬木蘭梁。比歲,使者獻金盎、銀冶犀、鹿,貢犛牛。



 寶歷至大和,再遣使者朝。五年,維州守將悉怛謀挈城以降,劍南西川節度使李德裕受之,收符章仗鎧,更遣將虞藏儉據之。州南抵江陽岷山,西北望隴山,一面崖,三涯江,虜號無憂城,為西南要捍。會牛僧孺當國,議還悉怛謀,歸其城。吐蕃夷誅無遺種,以怖諸戎。自是比五年虜使來,必報。所貢有玉帶、金皿、獺褐、犛牛尾、霞氈、馬、羊、橐它。



 贊普立幾三十年,病不事,委任大臣,故不能抗中國,邊候晏然。死,以弟達磨嗣。達磨嗜酒,好畋獵,喜內,且兇愎少恩,政益亂。開成四年,遣太子詹事李景儒往使,吐蕃以論集熱來朝,獻玉器羊馬。自是國中地震裂,水泉湧,岷山崩;洮水逆流三日,鼠食稼,人饑疫,死者相枕藉。鄯、廓間夜聞鼙鼓聲,人相驚。



 會昌二年,贊普死,論贊熱等來告,天子命將作監李璟吊祠。無子,以妃綝兄尚延力子乞離胡為贊普,始三歲,妃共治其國。大相結都那見乞離胡不肯拜,曰:「贊普支屬尚多,何至立綝氏子邪?」哭而出,用事者共殺之。



 別將尚恐熱為落門川討擊使,姓末,名農力,「熱」猶中國號「郎」也,譎詭善幻,約三部得萬騎,擊鄯州節度使尚婢婢,略地至渭州,與宰相尚與思羅戰薄寒山。思羅敗走松州,合蘇毘、吐渾、羊同兵八萬保洮河自守,恐熱謂蘇毘等曰:「宰相兄弟殺贊普,天神使我舉義兵誅不道,爾屬乃助逆背國耶?」蘇毘等疑而不戰,恐熱麾輕騎涉河,諸部先降,並其眾至十餘萬,禽思羅縊殺之。



 婢婢,姓沒盧,名贊心牙,羊同國人,世為吐蕃貴相,寬厚,略通書記,不喜仕,贊普強官之。三年,國人以贊普立非是,皆叛去。恐熱自號宰相,以兵二十萬擊婢婢,鼓鼙、牛馬、橐它聯千餘里,至鎮西軍,大風雷電,部將震死者十餘人,羊、馬、橐它亦數百。恐熱惡之,按軍不進。婢婢聞之,厚幣詒書約〓,恐熱大喜曰:「婢婢,書生,焉知軍事。我為贊普,當以家居宰相處之。」於是退營大夏川。婢婢遣將厖結心、莽羅薛呂擊恐熱於河州之南,伏兵四萬,結心據山射書極罵,恐熱怒甚,盛兵出鬥。結心偽北,恐熱追至數十里,莽羅薛呂以伏兵衷擊,大風雨,河溢,溺死甚眾,恐熱單騎而逃。既不得志,尤猜忍殺戮,部將岌藏、豐贊皆降,婢婢厚遇之。明年,恐熱復攻鄯州,婢婢分兵五道拒守,恐熱保東谷山,堅壁不出。岌藏繚以重柵,斷汲道,旬日,恐熱走薄寒山,募散卒稍至,得數千人,復戰鶡雞山,再戰南谷,皆大敗。兵拿仍歲不解。



 大中三年,婢婢屯兵河源,聞恐熱謀度河,急擊之,為恐熱所敗。婢婢統銳兵扼橋,亦不勝,焚橋而還。恐熱間出雞頂嶺關,馮硤為梁攻婢婢,至白土嶺,敗其將尚鐸羅榻藏,進戰犛牛硤。婢婢將燭盧鞏力欲負硤自固以困恐熱,大將磨離羆子不從,乃辭疾先歸。羆子急擊恐熱,一戰而死。婢婢糧盡,引眾趨甘州西境,以拓拔懷光居守,恐熱麾下多歸之。



 恐熱大略鄯、廓、瓜、肅、伊、西等州,所過捕戮,積尸狼藉,麾下內怨,皆欲圖之。乃揚聲將請唐兵五十萬共定其亂,保渭州,求冊為贊普,奉表歸唐。宣宗詔太僕卿陸耽持節慰勞,命涇原、靈武、鳳翔、邠寧、振武等兵迎援。恐熱既至,詔尚書左丞李景讓就問所欲。恐熱倨誇自大,且求河渭節度使,帝不許。還過咸陽橋,咄嘆曰:「我舉大事,覬得濟此河與唐分境。」於是復趨落門川收散卒,將寇邊,會久雨糧絕,恐熱還奔廓州。



 於是鳳翔節度使李玭復清水;涇原節度使康季榮復原州,取石門等六關,得人畜幾萬;靈武節度使李欽取安樂州,詔為威州;邠寧節度使張欽緒復蕭關;鳳翔收秦州;山南西道節度使鄭涯得扶州。鳳翔兵與吐蕃戰隴州,斬首五百級。是歲,河、隴高年千餘見闕下,天子為御延喜樓,賜冠帶,皆爭解辮易服。因詔差賜四道兵,錄有勞者;三州七關地腴衍者,聽民墾藝,貸五歲賦;溫池委度支榷其鹽,以贍邊;四道兵能營田者為給牛種,戍者倍其資餉,再歲一代;商賈往來於邊者,關鎮毋何留;兵欲墾田,與民同。



 初,太宗平薛仁杲,得隴上地;虜李軌,得涼州;破吐谷渾、高昌,開四鎮。玄宗繼收黃河積石、宛秀等軍,中國無斥候警者幾四十年。輪臺、伊吾屯田,禾菽彌望。開遠門揭候署曰「西極道九千九百里」,示戍人無萬里行也。乾元後,隴右、劍南西山三州七關軍鎮監牧三百所皆失之。憲宗常覽天下圖,見河湟舊封,赫然思經略之,未暇也。至是群臣奏言:「王者建功立業,必有以光表於世者。今不勤一卒,血一刃,而河湟自歸,請上天子尊號。」帝曰:「憲宗嘗念河、湟,業未就而殂落。今當述祖宗之烈,其議上順、憲二廟謚號,誇顯後世。」又詔:「朕姑息民,其山外諸州,須後經營之。」



 明年,沙州首領張義潮奉瓜、沙、伊、肅、甘等十一州地圖以獻。始,義潮陰結豪英歸唐,一日,眾擐甲噪州門,漢人皆助之,虜守者驚走,遂攝州事。繕甲兵,耕且戰,悉復餘州。以部校十輩皆操挺,內表其中,東北走天德城,防禦使李丕以聞。帝嘉其忠,命使者齎詔收慰,擢義潮沙州防禦使,俄號歸義軍,遂為節度使。其後河、渭州虜將尚延心以國破亡,亦獻款。秦州刺史高駢誘降延心及渾末部萬帳,遂收二州,拜延心武衛將軍。駢收鳳林關,以延心為河、渭等州都游弈使。



 咸通二年,義潮奉涼州來歸。七年,北庭回鶻僕固俊擊取西州,收諸部。鄯州城使張季顒與尚恐熱戰,破之,收器鎧以獻。吐番餘眾犯邠、寧,節度使薛弘宗卻之。會僕固俊與吐蕃大戰,斬恐熱首,傳京師。



 八年,義潮入朝,為右神武統軍,賜第及田,命族子淮深守歸義。十三年卒。沙州以長史曹義金領州務,遂授歸義節度使。後中原多故,王命不及,甘州為回鶻所並,歸義諸城多沒。



 渾末,亦曰嗢末,吐蕃奴部也。虜法,出師必發豪室,皆以奴從,平居散處耕牧。及恐熱亂,無所歸,共相嘯合數千人,以嗢末自號,居甘、肅、瓜、沙、河、渭、岷、廓、疊、宕間,其近蕃牙者最勇,而馬尤良云。



 贊曰:唐興,四夷有弗率者,皆利兵移之,蹙其牙,犁其廷而後已。惟吐蕃、回鶻號強雄,為中國患最久。贊普遂盡盜河湟,薄王畿為東境,犯京師,掠近輔,殘馘華人。謀夫虓帥,圜視共計,卒不得要領。晚節二姓自亡,而唐亦衰焉。夫外撫內寧,惟聖人不讓。玄宗有逸德,而拓地太大,務遠功,忽近虞,逆賊一奮,中原封裂,訖二百年不得復完,而至陵夷。然則內先自治,釋四夷為外懼,守成之良資也。



\end{pinyinscope}