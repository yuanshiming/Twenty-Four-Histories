\article{列傳第二十五 陳楊封裴宇文鄭權閻蔣姜張}

\begin{pinyinscope}

 陳叔達,字子聰,陳宣帝子也。少封義陽王,歷丹楊尹、都官尚書。入隋,久不試。大業中篇。中文本依照1928年俄文版編排(譯文採自《斯大林全,授內史舍人,出為絳郡通守。高祖西師,以郡聽命,授丞相府主簿,封漢東郡公。與溫大雅同管機秘,方禪代時,書冊誥詔,皆其筆也。武德初,授黃門侍郎,判納言,封江國公。



 叔達明辯,善為容,每占奏,縉紳屬目。江左士客長安,或汨滯,多薦諸朝。嘗賜食,得蒲萄,不舉,帝問之,對曰:「臣母病渴,求不能致,願歸奉之。」帝流涕曰:「卿有母遺乎?」因賜之,又賚物百段。貞觀初,與蕭瑀爭殿中,坐忿誶不恭,免官。未幾,居母喪,又有疾,太宗憂之,遣使禁卻吊者。喪除,為遂州都督,病不拜。頃之,擢禮部尚書。始,太子建成等鬩間太宗,帝惑之,叔達極意救辯,至是謂曰:「武德內難,卿有讜言,故以此報。」叔達謝曰:「豈獨為陛下,乃社稷計耳。」後閨薄汗漫,為有司露劾,帝以名臣為護掩,授散秩歸第。卒,謚曰繆。久之,贈戶部尚書,更謚曰忠。



 楊恭仁,隋觀王雄子也。仁壽中,累遷甘州刺史,臨事不苛細,徼人安之。文帝謂雄曰:「匪特朕得人,乃卿善教子矣。」大業初,轉吏部侍郎。楊玄感叛,詔率兵經略,與玄感戰破陵,敗之。遂與屈突通追獲賊。煬帝召見曰:「比聞與賊戰尤力,向但知卿奉法,而乃勇決如此,朕用自愧。」蘇威曰:「仁者必有勇,殆謂此邪。」時威及宇文述、裴蘊、裴矩參掌選事,皆受賕不法,恭仁素廉正,故惡之,出為河南道大使,使捕寇賊。至譙郡,為硃粲所敗,奔江都。宇文化及弒逆,署吏部尚書,為化及守魏縣。元寶藏執送京師,高祖素知之,授黃門侍郎,封觀國公。尋為涼州總管。



 恭仁久乘邊,習種落情偽,悉心綏慰,由蔥嶺以東,皆奉貢贄。就加納言。突厥頡利率眾數萬獵其境,恭仁應機設拒,張疑屯虛幟示之,頡利懼而走。瓜州刺史賀拔行威叛,朝廷未即討。恭仁募趫蕩,倍道進,賊不虞其來,遂克二城。縱所俘還之,眾感悅,遂相與縛行威降。召拜吏部尚書,兼中書令,檢校涼州諸軍事。遷左衛大將軍。武德末,拜雍州牧、揚州大都督府長史。遷洛州都督。太宗勞謂曰:「洛陽要重,朕子弟不為少,恐非所任,故以委公。」



 恭仁性沖厚,以禮自閑衛,未嘗與物忤,時人方漢石慶。既貴,不以勢尚人,故譽望益重。病,乞骸骨,詔以特進歸第。卒,贈潭州都督,陪葬昭陵,謚曰孝。



 子思訓襲爵。顯慶中,歷右屯衛將軍。從高宗幸並州。右衛大將軍慕容寶節夜邀思訓與謀亂,思訓不敢對。寶節懼,毒酒以進,思訓死。妻訴之,流寶節嶺表,至龍門,追斬之。乃詔以寘毒人者重其法。



 思訓孫睿交,尚長寧公主,豫誅張易之,賜實封五百戶。神龍中為秘書監,貶絳州別駕。



 師道字景猷,恭仁弟。清警有才思。客洛陽,為王世充所拘,間歸高祖,授上儀同,為備身左右。尚桂陽公主,除吏部侍郎。改太常卿,封安德郡公。貞觀十年,拜侍中,參豫朝政,親遇隆渥。性周謹,未嘗語禁省事。嘗曰:「吾讀《孔光傳》,想其餘風,或庶幾云。」太宗數訪群臣才行,師道雖有所推進,而乏甄品。久之,遷中書令。太子承乾得罪,詔與長孫無忌等雜治其獄。師道妻異姓子趙節與承乾通謀,乃微諷帝,欲活之。帝怒,罷為吏部尚書。師道起貴胄,四海人物,非所練悉,至銓署,專抑勢貴親黨以遠嫌,用人多違其才,不為時所稱。帝亦曰:「師道資性純淑,自應無過,而實怯懦,罕更事,緩急不得其力。」從征高麗,攝中書令。軍還,頗不職,改工部尚書,復為太常卿。



 師道善草隸,工詩,每與有名士燕集,歌詠自適。帝見其詩,為擿諷嗟賞。後賜宴,帝曰:「聞公每酣賞,捉筆賦詩,如宿構者,試為朕為之。」師道再拜,少選輒成,無所竄定,一坐嗟伏。卒,贈吏部尚書、並州都督,謚曰懿,陪葬昭陵,詔為立碑。



 子豫之,尚巢王元吉女壽春縣主。居母喪,與永嘉公主亂,為主婿竇奉節所殺。



 執柔,恭仁從孫,歷地官尚書。武后母,即恭仁叔父達之女。及臨朝,武承嗣、攸寧相繼用事。後曰:「要欲我家及外氏常一人為宰相。」乃以執柔同中書門下三品。未幾,卒。



 弟執一,亦以誅張易之功封河東郡公,累官右金吾衛大將軍。



 始,雄在隋,以同姓貴;自武德後,恭仁兄弟名位益盛;又以武后外家尊寵,凡尚主者三人,女為王妃五人,贈皇后一人,三品以上者二十餘人。



 封倫,字德彞,以字顯,觀州蓚人。祖隆,北齊太子太保。倫年方少,舅盧思道曰:「是兒識略過人,當自致卿相。」隋開皇末,江南亂,內史令楊素討之,署倫行軍記室。泊海上,素召計事,倫墜水,免,易衣以見,訖不言。久乃素知,問故,謝曰:「私事也,所不敢白。」素異其為,以從妹妻之。素營仁壽宮,表為土工監,規構鴻侈。宮成,文帝怒曰:「素殫百姓力,為吾掊怨天下。」素大懼。倫曰:「毋恐,皇后至,自當免。」明日,帝果勞素曰:「公知吾夫婦老,無以自娛樂,而盛飾此宮邪?」因大悅。素退問:「何料而知?」倫曰:「上節儉,故始見必怒。然雅聽後言。後,婦人,惟侈麗是好。後悅,則帝安矣。」素曰:「吾不及也。」素負才勢,多所凌藉,惟於倫降禮賞接,或與論天下事,袞袞不倦,每撫其床曰:「封郎終當據此。」薦之帝,擢內史舍人。



 虞世基得幸煬帝,然不悉吏事,處可失宜。倫陰為裁畫,內以諂承主意,百官章奏若忤旨,則寢不聞;外以峻文繩天下,有功當賞,輒抑不行。由是世基之寵日隆,而隋政日壞矣。宇文化及亂,持帝出宮,使倫數帝罪,帝曰:「卿,士人,何至是!」倫羞縮去。化及署為內史令,從至聊城,知化及敗,及結士及,得出護餉道。化及死,遂與士及來降。高祖知其諧附逆黨,方切讓,使就舍。倫以秘策幹帝,帝悅,更拜內史舍人。遷侍郎兼內史令。



 秦王討王世充,命倫參謀軍事。時兵久不決,帝欲班師,王遣倫西見帝曰:「賊地雖多,羈縻不相使,所用命者洛陽爾,計窮力屈,死在旦暮。今解而西,則賊勢磐結,後難以圖。」帝納之。賊平,帝謂侍臣曰:「始議東討,時多沮解者,唯秦王謂必克,倫贊其行,雖張華葉策晉武,亦何以加於是!」封平原縣公,判天策府司馬。初,竇建德援洛,王將趣虎牢,倫與蕭瑀諫不可,至是入賀。王笑曰:「不用公言,今日幸而捷,豈智者千慮或有失乎?」倫謝素不及。頃之,突厥寇太原,且遣使和親。帝問計,群臣咸請許之可紓戰。倫曰:「不然。彼有輕中國心,謂我不能戰,若乘其怠擊之,勢必勝,勝而後和,威德兩全。今雖不戰,後必復來。臣以為擊之便。」詔可。尋檢校吏部尚書,進封趙國公,徙密國。



 太宗立,拜尚書右僕射,實封六百戶。始,倫之歸,蕭瑀數薦之。及是,瑀為左僕射,每議事,倫初堅定,至帝前輒變易,由是有隙。貞觀元年,遘疾,臥尚書省,帝親臨視,命尚輦送還第。卒,年六十,贈司空,謚曰明。



 倫資險佞內狹,數刺人主意,陰導而陽合之。外謹順,居處衣服陋素,而交宮府,賄贈狼藉。然善矯飾,居之自如,人莫能探其膺肺。隱、刺之亂,數進忠策,太宗以為誠,橫賜累萬。又密言於高祖曰:「秦王恃功,頡頏太子下,若不早立,則亟圖之。」情白太子曰:「為四海不顧其親,乞羹者謂何?」及高祖議廢立,倫固諫止。當時語秘無知者,卒後,事浸聞。十七年,治書侍御史唐臨追劾奸狀,帝下其議百官。民部尚書唐儉等議:「倫寵極生前,而罪暴身後,所歷官不可盡奪,請還贈改謚,以懲憸壬。」有詔奪司空,削食封,改謚為繆。



 子言道,尚淮南長公主,官至宋州刺史。



 裴矩,字弘大,絳州聞喜人。父訥之,為齊太子舍人。矩在乳而孤,及長好學,有文藻智數。再補高平王文學。齊亡,不得調。隋高祖為定州總管,召補記室,以母憂去職。高祖已受禪,遷給事郎,奏舍人事。帝伐陳,為元帥記室。江左平,詔矩巡撫嶺南,未行,而高智慧等亂,道不通,帝難其遣,矩請速進,許之。次南康,得兵數千人。是時,俚帥王仲宣逼廣州,遣別將圍東衡州,矩與將軍鹿願赴之。賊立九壁,屯大庾嶺,矩進擊,破之。賊懼,釋東衡州之圍,據願長嶺,又擊破之,斬其帥。自南海趣廣州,仲宣懼,潰去。綏集二十餘州,承制署渠帥為刺史、縣令。還報,帝大悅,詔升殿勞苦之。拜開府,爵聞喜縣公,賜賚異等。遷累內史侍郎。時突厥強盛,都藍與突利構難,屢犯塞,詔太平公史萬歲為行軍總管,出定襄道,以矩為長史。破達頭可汗而萬歲誅,矩功不見錄。還為尚書左丞,遷吏部侍郎,名稱職。



 煬帝時,西域諸國悉至張掖交市,帝令矩護視。矩知帝勤遠略,乃訪諸商胡國俗、山川險易,撰《西域圖記》三篇,合四十四國,凡裂三道:北道起伊吾,徑蒲類、鐵勒、突厥可汗廷,亂北流河至拂菻;中道起高昌、焉耆、龜茲、疏勒,逾蔥嶺,鏺汗、蘇對沙那、康、曹、何、大小安、穆諸國,至波斯;南道起鄯善、于闐、硃俱波、喝般陀,亦度蔥嶺,涉護密、吐火羅、挹怛、忛延、漕國,至北婆羅門。皆竟西海。諸國亦自有空道交通。既還,奏之。帝引內矩,問西方事,矩盛言:「胡多瑰怪名寶,俗土著,易並吞。」帝由是甘心四夷,委矩經略。再遷黃門侍郎,參豫朝政。



 大業三年,帝有事恆山,西方來助祭者十餘國。矩遣人說高昌、伊吾等,啗以厚利,使入朝。帝西巡燕支山,高昌等二十七國謁道左,皆使佩金玉,服錦罽,奏樂歌舞,令士女盛飾縱觀,亙數十里,示中國強富。後遂破吐谷渾,拓地數千里,遣兵出戍,歲委輸巨億萬計。帝謂矩有綏懷略,擢銀青光祿大夫。帝在東都,矩以蠻夷朝貢踵至,諷帝悉召天下奇倡怪伎,大陳端門前,曳錦縠、珥金琲者十餘萬,百官都人列繒樓幔閣夾道,被服光麗。廛邸皆供帳,池酒林皪。譯長縱蠻夷與民貿易,所在令邀飲食,相娛樂。蠻夷嗟咨,謂中國為「仙晨帝所」。天子以為誠,謂宇文述、牛弘曰:「矩所建白,皆朕之志,要未發,矩輒先聞,非悉心奉國,疇能是邪?」又助城伊吾,脅處羅入朝。帝益喜,賜貂裘、西胡珍器。從帝巡塞北,幸啟民帳。時高麗遣使先在突厥,啟民引見帝。矩因奏言:「高麗本孤竹國,周以封箕子,漢分三郡,今乃不臣,先帝疾之,欲討久矣。方陛下時,安得不事?今其使朝突厥,及見啟民,舉國臣服,脅令入朝,可致也。請面詔其使,令歸語王,有如旅拒,方率突厥誅之。」帝納焉。高麗不聽命,征遼自此始。王師再臨遼,皆從,以勞加右光祿大夫。時綱紀汩振,宇文述、虞世基用事,官以賄遷,唯矩挺節無穢聲,世頗稱之。



 矩以始畢可汗眾漸盛,建請以宗女嫁叱吉設,建為南面可汗,分其勢。叱吉不敢受。始畢聞之,稍怨望。矩又言:「突厥淳陋,易離間,但內多群胡教導之。臣聞史蜀胡悉尤有謀,幸於始畢,請殺之。」帝曰:「善。」矩因詭計召胡悉受賜,斬馬邑下,報始畢曰:「史蜀胡悉背可汗,我所共惡,今既誅之。」始畢知狀,由是不朝。後帝北巡,始畢率騎十萬圍帝雁門,詔矩與虞世基宿朝堂待顧問。圍解,從幸江都宮。時盜賊蜂結,郡縣上奏不可計,矩言於帝。帝怒,遣詣京師,以疾解。俄而高祖入關,帝令虞世基問方略,矩曰:「唯願陛下亟西,天下定矣。」



 矩性勤謹,未嘗忤物,見天下方亂,其待遇士尤厚,雖廝役皆得其歡。是時,衛兵數逃去,帝憂之,以問矩。矩曰:「今乘輿淹狩已二年,諸驍果皆無家,人無匹合,則不久安,臣請皆聽納室。」帝笑曰:「公定多智。」因詔矩盡召江都女子、孀家,恣將士所欲,即配之,人情翕然相悅,曰:「裴公惠也!」宇文化及亂,眾劫矩。賊皆曰:「裴黃門無豫也。」既而眾以秦王子浩為帝,詔矩為侍內,隨而北。化及僭位,署矩尚書右僕射,為河北道安撫大使。又為竇建德所獲,建德以矩隋舊臣,遇之厚。建德起群盜,非有君臣制度,矩為略制朝儀,不閱月,憲章擬王者,建德尊禮之。建德敗,來朝,擢殿中侍御史,爵安邑縣公。累遷太子詹事、檢校侍中。時突厥數盜邊,高祖遣使約西突厥連和,突厥因請婚。帝曰:「彼勢與我絕,緩急不為用,奈何?」矩曰:「然北虜方熾,歲苦邊,若權順許,以示外援,須我完實更議之。」帝然其計。隱太子敗,餘黨保宮城不解。秦王遣矩諭之,乃聽命。遷民部尚書。



 太宗即位,疾貪吏,欲痛懲乂之,乃間遣人遺諸曹,一史受饋縑,帝怒,詔殺之。矩曰:「吏受賕,死固宜。然陛下以計紿之,因即行法,所謂罔人以罪,非道之以德之誼。」帝悅,為群臣言之,曰:「矩遂能廷爭,不面從,物物若此,天下有不治哉?」年八十,精明不忘,多識故事,見重於時。貞觀元年卒,贈絳州刺史,謚曰敬。



 宇文士及,字仁人,京兆長安人。父述,為隋右衛大將軍。開皇末,以述勛封新城縣公。文帝引入臥內,與語,奇之。詔尚煬帝女南陽公主,為尚輦奉御,從幸江都,以父喪免,起為鴻臚少卿。其兄化及謀弒逆,以主婿忌之,弗告。已弒帝,乃封蜀王。



 初,士及為奉御,而高祖任殿中少監,雅自款結。及從化及至黎陽,帝手書召之。士及亦遣家童間道走長安,通諄勤,且獻金鐶。帝悅曰:「我嘗與士及共事,今以此獻,是將來矣。」化及兵日蹙,士及勸歸命,不從,乃與封倫詭求督餉。俄而化及敗,於是濟北豪傑謀起齊兵擊竇建德以收河北,觀形勢,士及不納,與倫等自歸。帝讓之曰:「汝兄弟率思歸之人為入關計,爾得時,我父子,尚肯相假乎?今欲何地自處?」士及謝曰:「臣罪當死,但臣往在涿郡,嘗與陛下夜論世事,頃又奉所獻,冀以此贖罪。」帝笑謂裴寂曰:「彼與我論天下事,逮今六七年,公等皆在其後。」時士及女弟為昭儀,有寵,由是見親禮,授上儀同。從秦王平宋金剛,錄功,復隋舊封,以宗室女妻之,遷王府驃騎將軍。從討王世充等,進爵郢國公。武德八年,權檢校侍中,兼太子詹事。王即位,拜中書令,真食益州七百戶,以本官檢校涼州都督。時突厥數入寇,士及欲立威以鎮耀邊鄙,每出入,盛陳兵衛,又痛折節下士。或告其反,訊無狀,召為殿中監,以疾改蒲州刺史。政尚寬簡,人皆宜之。擢右衛大將軍。太宗延入閤語,或至夜分出,遇休沐,往往馳召。士及益自謹,其妻嘗問遽召何所事,士及卒不對。帝嘗玩禁中樹曰:「此嘉木也!」士及從旁美嘆。帝正色曰:「魏徵常勸我遠佞人,不識佞人為誰,乃今信然。」謝曰:「南衙群臣面折廷爭,陛下不得舉手。今臣幸在左右,不少有將順,雖貴為天子,亦何聊?」帝意解。又嘗割肉,以餅拭手,帝屢目,陽若不省,徐啗之。其機悟率類此。後以雅舊,別封一子新城縣公。久之,復為殿中監。卒,贈左衛大將軍、涼州都督,陪葬昭陵。士及撫幼弟、孤兄子,以友睦稱。好周恤親戚故人,然過自奉養,服玩食飲必極豐侈。有司謚曰恭,黃門侍郎劉洎曰:「士及居家侈肆,不可謂恭。」乃改曰縱。



 贊曰:封倫、裴矩,其奸足以亡隋,其知反以佐唐,何哉?惟奸人多才能,與時而成敗也。妖禽孽狐,當晝則伏自如,得夜乃為之祥。若倫偽行匿情,死乃暴聞,免兩觀之誅,幸矣。太宗知士及之佞,為游言自解,亦不能斥。彼中材之主,求不惑於佞,難哉!



 鄭善果,鄭州滎澤人。祖在魏為顯家。父誠,周大將軍、開封縣公,討尉遲迥,戰死。善果方九歲,以死事子襲爵,家人為其幼,弗告也;及受詔,號哭不自勝。隋開皇初,進封武德郡公。年十四,為沂州刺史。累轉魯郡太守。



 善果母崔,賢明曉政治,嘗坐閤內聽善果處決,或當理則悅,有不可,則引至床下,責愧之。故善果所至有績,號清吏。嘗與武威太守樊子蓋考為天下第一,煬帝賜物千段、黃金百兩。再遷大理卿。突厥圍帝雁門,以守御功拜右光祿大夫。從幸江都。宇文化及弒逆,署民部尚書,從至聊城。淮安王神通攻之,善果督戰,中流矢。神通解。俄為竇建德所獲,王琮讓之曰:「公,隋大臣,自尊夫人亡,名稱衰。今以忠臣子為逆賊徇命至傷夷,謂何?」善果慚,欲自殺,或止之,得不死。建德不之禮,乃歸神通。送京師,擢太子左庶子,更封滎陽郡公。數為太子陳得失。未幾,檢校大理卿,兼民部尚書。奉法持正,風績顯公卿間。詔與裴寂等十人每奏事若侍得升殿,而從父兄元亦與,時以為榮。坐事免。會山東平,持節為招撫大使。以選舉失實除名。後歷刑部尚書。貞觀初,出為岐州刺史,以累去。復拜江州刺史,卒。



 元,字德芳,隋沛國公譯之子。性察慧,愛尚文藝。以父功拜儀同,襲爵。累遷右衛將軍,更封莘國公。大業末,出為文城郡守。高祖兵興,遣將張綸西略地,攻拔其城,系致軍門,釋之,授太常卿。與襄武王琛使突厥,還為參旗將軍。元習軍旅事,帝令教諸屯軍法。劉武周將宋金剛與突厥處羅可汗犄角寇汾、晉,元諭罷可汗兵,不聽,乃進為武周援。會暴疾,其下意元置毒,囚之。處羅死,頡利立,留帳中數年。帝既許可汗婚,元始得還。帝勞曰:「卿不辱於虜,可輩蘇武、張騫矣。」拜鴻臚卿,母喪免。



 會突厥提精騎數十萬,身自將攻太原,詔即苫次起元持節往勞。既至,虜以不信咎中國,元隨語折讓,無所屈,徐乃數其背約,突厥愧服。因好謂頡利曰:「突厥得唐地無所用,唐得突厥不可臣而使,兩不為用而相攻伐,何哉?今掠財資,劫人口,皆入所部,可汗一不得,豈若僕旗接好,則金玉重幣一歸可汗。且唐有天下,約可汗為兄弟,使驛銜箠於道,今坐受其利不肯,乃蔑德貽怨,自取勞苦,若何?」頡利當其言,引還。太宗賜書曰:「知公口伐,可汗如約,遂使邊火息燧,朕何惜金石賜於公哉!」貞觀三年,復使突厥,還言:「夷狄以馬羊準盛衰,今突厥六畜不蕃,人色若菜,牙內飯粟化為血,不三年必亡。」無幾,突厥果敗。後轉左武侯大將軍,坐事免。起為宜州刺史,以老致仕。卒,贈幽州刺史,謚曰簡。



 元幹敏,所至常有譽。五聘絕域,危不脫,終不自為解。然譯事後母不謹,隋文帝嘗賜《孝經》愧勖之;至元亦不以孝聞,士醜其行。從孫杲,知名武後世,終天官侍郎。



 權萬紀,其先出天水,後徙京兆,為萬年人。父琢玠,隋匡州刺史,以愨願聞。萬紀悻直廉約,自潮州刺史擢治書侍御史。尚書右僕射房玄齡、侍中王珪掌內外官考,萬紀劾其不平,太宗按狀,珪不伏。魏徵奏言:「房玄齡等皆大臣,所考有私,萬紀在考堂無訂正,今而彈發,非誠心為國者。」帝乃置之,然以為不阿貴近,繇是獎禮。萬紀又建言:「宇文智及受隋恩,賊殺其君,萬世共棄,今其子乃任千牛,請斥屏以懲不軌。」帝從之。萬紀與侍御史李仁發既以言得進,頗掉罄自肆,眾情懍懍。徵奏:「萬紀等暗大體,詆訐彈射皆不實,陛下收其一切,遂敢附下罔上,釣強直名,迷奪聖明,以小謀大,群下離心。如玄齡等且不得申,況疏賤之臣哉?」帝寤,徙萬紀散騎常侍,而免仁發。數年,復召萬紀為持書御史,即奏言:「宣、饒部中可鑿山冶銀,歲取數百萬。」帝讓曰:「天子所乏,嘉謀善政有益於下者。公不推賢進善,乃以利規我,欲方我漢桓、靈邪?」斥使還第。



 久之,由御史中丞進尚書左丞,出為西韓州刺史。徙吳王長史。王畏其直,善遇之。齊王祐不奉法,帝素奇萬紀能左右吳王者,乃徙為祐長史。祐暱比群小,萬紀驟諫不入,即條過失以聞。帝遣劉德威按問,因召祐入朝。祐恐,與所嬖燕弘亮謀殺之,而萬紀先引道。祐遣弘亮馳彀騎追擊,斬首,殊支體,投圊中。又殺典軍韋文振。文振本以校尉從帝征伐,以質謹自將,帝使事祐,典廄馬,切諫不納,輒見萬紀道之,故祐內嘗忿疾。萬紀死,文振懼,馳去,追騎獲之。祐平,贈萬紀齊州都督、武都郡公,食二千戶,謚曰敢,文振左武衛將軍、襄陽縣公,食千戶。



 萬紀子玄初,高宗時兵部侍郎。



 懷恩,萬紀族孫。祖弘壽,為隋臨汾司倉書佐,高祖平京師,擢太僕卿、盧國公,卒,謚曰恭。故懷恩以廕累遷尚乘奉御,襲爵。馭人安畢羅為高宗所寵,見帝,戲慢不恭,懷恩奏事,適見之,退杖四十。帝嗟賞曰:「良吏也!」擢萬年令。賞罰明,見惡輒取。時語曰:「寧飲三鬥塵,無逢權懷恩。」其姿狀沈毅,每盛服,妻子不敢仰視。更慶、萊、衛、邢、宋五州刺史,洛州長史。所居威名赫然,吏重足立。嘗過汴州,時刺史楊德干亦以嚴稱,與懷恩名相埒。汴橋新成,立木中途,止過車者。懷恩適過之,示德乾曰:「民不可止邪,焉用此?」德乾慚服。遷益州大都督府長史,卒。



 從子楚璧,為左領軍衛兵曹參軍。玄宗在東都,楚璧乃與李迥秀子齊損、陳倉尉盧玢、左屯營長上折沖周履濟等謀反,以兄子梁山詐為襄王子,號光帝,擁營兵百餘夜入官城,欲劫留守王志愔,不克。遲明,兵斬楚璧等,傳首東都,籍其家。



 閻讓,字立德,以字行,京兆萬年人。父毘,為隋殿內少監,本以工藝進,故立德與弟立本皆機巧有思。武德初,為秦王府士曹參軍,從平東都。遷尚衣奉御,制袞冕六服、腰輿、傘扇,咸有典法。貞觀初,歷將作少匠、大安縣男。護治獻陵,拜大匠。文德皇后崩,攝司空,營昭陵,坐弛職免。起為博州刺史。太宗幸洛陽,詔立德按爽塏建離宮清暑,乃度地汝州西山,控汝水,睨廣成澤,號襄城宮,役凡百餘萬。宮成,煩燠不可居,帝廢之,以賜百姓,坐免官。



 未幾,復為大匠,即洪州造浮海大航五百艘,遂從征遼,攝殿中監,規築土山,破安市城。師還,至遼澤,亙二百里,淖不可通,立德築道為橋梁,無留行。帝悅,賜予良厚。又營翠微、玉華二宮,擢工部尚書。帝崩,復攝司空,典陵事,以勞進爵大安縣公。永征五年,高宗幸萬年宮,留守京師,領徒四萬治京城。卒,贈吏部尚書、並州都督,陪葬昭陵,謚曰康。



 立本,顯慶中以將作大匠代立德為工部尚書。總章元年,以司平太常伯拜右相、博陵縣男。初,太宗與侍臣泛舟春苑池,見異鳥容與波上,悅之,詔坐者賦詩,而召立本侔狀。閤外傳呼畫師閻立本,是時已為主爵郎中,俯伏池左,研吮丹粉,望坐者羞悵流汗。歸戒其子曰:「吾少讀書,文辭不減儕輩,今獨以畫見名,與廝役等,若曹慎毋習!」然性所好,雖被訾屈,亦不能罷也。既輔政,但以應務俗材,無宰相器。時姜恪以戰功擢左相,故時人有「左相宣威沙漠,右相馳譽丹青」之嘲。咸亨元年,官復舊名,改中書令。卒,謚曰文貞。立德孫知微,曾孫用之。



 知微,聖歷初為豹韜衛將軍。武后時,突厥默啜請和親,後遣知微攝春官尚書,持金帛護送武延秀聘其女。默啜怒非天子子,囚延秀,挾知微入寇趙、定,尊之如可汗,以示華人,自河以北蕭然。朝廷以知微賣國,夷其族。知微不知,逃還。武后業已然,乃曰:「惡臣疾子,賜百官甘心焉。」於是骨斷臠分,非要職者不能得。子則先,以武三思婿免死。玄宗在籓時,以善割蒙寵。開元中,有司奏擬供奉,姚元崇以為則先刑戮家,又逆人姻屬,不可留京師。詔曰:「朕在外日,嘗驅使,宜令供奉。」



 用之,初為彭州參軍,嘗攝錄事,一日糾愆謬不法數十事,太守以為材。後舉通事舍人,累遷右衛郎將,知引駕仗。金吾將軍李質升殿不解刀,呵卻之,請按以法,左右震悚。始,有司以三衛執扇登殿,用之奏三衛皆趫悍,不宜升陛邇御坐,請以宦者代,遂為故事。天寶中,女為義王玼妃。終左金吾將軍。



 蔣儼,常州義興人。擢明經第,為左屯衛兵曹參軍。太宗將伐高麗,募為使者,人皆憚行,儼奮曰:「以天子雄武,四夷畏威,蕞爾國敢圖王人?有如不幸,固吾死所也。」遂請行。為莫離支所囚,以兵脅之,不屈,內窟室中。高麗平,乃得歸。帝奇其節,授朝散大夫。為幽州司馬,劉祥道以巡察使到部,表最狀,擢會州刺史。再遷殿中少監,數陳時政病利,高宗輒優納。進蒲州刺史,戶產充夥,訴犴積年不平,前刺史踵以罪去,儼至,發隱禁奸,號良二千石。永隆二年,以老致仕。未幾,復召為太僕卿,以父諱辭官,徙太子右衛副率。



 中宗在東宮,儼數爭過失,不見用。自以總調護,不應諫。於是田游巖興處士為洗馬,太子所尊禮,儼詒書責之曰:「太子年鼎盛,聖道有所未盡,足下受調護之寄,居責言之地,唯唯悠悠,不出一談。向使不餐王粟,僕何敢議?今祿及親矣,尚何酬塞?」游巖愧不能答。儼尋徙右衛大將軍,封義興縣子,以太子詹事致仕。卒,年七十八。中宗立,以舊恩贈禮部尚書。



 韋弘機,京兆萬年人。祖元禮,隋浙州刺史。弘機仕貞觀時為左千牛胄曹參軍,使西突厥,冊拜同俄設為可汗。會石國叛,道梗,三年不得歸。裂裾錄所過諸國風俗、物產,為《西征記》。比還,太宗問外國事,即上其書。帝大悅,擢朝散大夫。累遷殿中監。顯慶中,為檀州刺史,以邊人陋僻,不知文儒貴,乃脩學官,畫孔子、七十二子、漢晉名儒象,自為贊,敦勸生徒,繇是大化。契苾何力討高麗。次灤水,會暴漲,師留三日。弘機輸給資糧,軍無饑,高宗善之,擢司農少卿,主東都營田苑。宦者犯法,杖乃奏,帝嗟賞,賜絹五十匹,曰:「後有犯,治之,毋奏。」遷司農卿。



 太子弘薨,詔蒲州刺史李沖寂治陵,成而玄堂厄,不容終具,將更為之。役者過期不遣,眾怨,夜燒營去。帝詔弘機嗣作,弘機令開隧左右為四便房,撙制禮物,裁工程,不多改作,如期而辦。帝嘗言:「兩都,我東西宅,然因隋宮室日僕不完,朕將更作,奈財用何?」弘機即言:「臣任司農十年,省惜常費,積三十萬緡,以治宮室,可不勞而成。」帝大悅,詔兼將作、少府二官,督營繕。初作宿羽、高山等宮,徙洛中橋於長夏門,廢利涉橋,人便之。天子乃登洛北絕岸,延眺良久,嘆其美,詔即其地營宮,所謂上陽者。尚書左僕射劉仁軌謂侍御史狄仁傑曰:「古天子陂池臺榭皆深宮復禁,不欲百姓見之,恐傷其心。而今列岸謻廊亙王城外,豈愛君哉?」弘機猥曰:「天下有道,百官奉職,任輔弼者,則思獻替事。我乃府藏臣,守官而已。」仁傑非之。俄坐家人犯盜,劾免官。



 初,東都方士硃欽遂為武后所寵,奸贓狼藉。弘機白:「欽遂假中宮驅策,依倚形勢,虧紊皇明,為禍亂之漸。」帝遣中使慰諭,敕毋漏言,逐欽遂於邊,後恨之。永淳中,帝幸東都,至芳桂宮,召弘機使白衣檢校園苑,將復任之,為後犄而止。終檢校司農少卿事。



 孫岳子、景駿。景駿別傳。



 岳子,武后時為汝州司馬,以辨治稱。召授尚舍奉御,入見,後賞其能,曰:「卿家事,朕悉知之。」因問舊故,至家人皆不忘。出為太原令,以不習武固辭,忤旨,下遷宋州長史。歷廬、海等州刺史,皆著風跡,恩嚴兩施。睿宗立,召為殿中少監,恩遇尤異。竇懷貞等誅,而嶽子舊與經過,為姜晈所劾,貶渠州別駕。起授陜州刺史,卒。孫皋,別有傳。



 姜師度,魏州魏人。擢明經,調丹陵尉、龍崗令,有清白稱。神龍初,試為易州刺史、河北道巡察,兼支度營田使。好興作,始廝溝於薊門,以限奚、契丹,循魏武帝故跡,並海鑿平虜渠,以通餉路,罷海運,省功多。遷司農卿。出為陜州刺史。太原倉水陸運所湊,轉屬諸河,師度使依高為廥,而注米於舟,以故人不勞。拜太子詹事。



 玄宗徙營州治柳城,拜營田支度脩築使。進為河中尹。安邑鹽池涸廢,師度大發卒,洫引其流,置鹽屯,公私收利不貲。徙同州刺史。又派洛灌朝邑、河西二縣,閼河以灌通靈陂,收棄地二千頃為上田,置十餘屯。帝幸長春宮,嘉其功,下詔褒美,加金紫光祿大夫,賜帛三百匹。進將作大匠。左拾遺劉彤建榷天下鹽鐵利內之官,免貧民賦,詔戶部侍郎強循與師度並假御史中丞,會諸道按察使議所以榷之之法,俄為議者沮,閣不行。卒,年七十餘。



 師度喜渠漕,所至繇役紛紜,不能皆便,然所就必為後世利。是時太史令傅孝忠以知星顯,時為語曰:「孝忠知仰天,師度知相地。」嘲所嗜也。



 強循字季先,鳳州人。仕累雍州司士參軍。華原無泉,人畜多曷死。循教人渠水以浸田,一方利之,號強公渠。詔書褒予甚厚。歷大理少卿、太子右庶子。為政辦給,不為威嚴,遇人盡信不疑,然當時恨其少文云。



 張知謇,子匪躬,幽州方城人,徙家岐。兄弟五人,知玄、知晦、知泰、知默,皆明經高第,曉吏治,清介有守,公卿爭為引重。調露時,知謇監察御史裏行,知默左臺侍御史。知謇歷十一州刺史,所蒞有威嚴,武后降璽書存問。萬歲通天中,自德州刺史入計,後奇其貌,詔工圖之,稱其兄弟容而才,謂之兩絕。又門皆列戟,白雀巢其廷,後數寵賜。知泰歷益州長史、中臺左丞、兵部侍郎,封陳留縣公。



 中宗在房州,禁察苛嚴。知謇與董玄質、崔敬嗣繼為刺史,供儗保戴不少弛。帝復位,拜知謇左衛將軍,加雲麾將軍,封範陽郡公;知泰御史臺大夫,加銀青光祿大夫,封漁陽郡公。伯仲華首同貴,時以為榮。知泰忤武三思,故出為並州刺史、天兵軍使。終魏州刺史,謚曰定。知謇歷東都副留守、左右羽林大將軍、同華州刺史,大理卿致仕。年八十,開元時卒。



 知謇敏且亮,惡請謁求進,士或不才冒位,視之若仇。每敕子孫「經不明不得舉」,家法可稱云。



 武后革命,知泰奏置東都諸關十七所,譏斂出入。百姓驚駭,樵米踴貴,卒罷不用,議者羞薄之。



 知默與監察御史王守慎、來俊臣、周興掌詔獄,數陷大臣。守慎雖其甥,惡鞫引之暴,不得去,請度為浮屠,後許之。而知默卒陷酷吏,子孫禁錮,為張氏羞。



 知玄子景升,知泰子景佚,開元中皆顯官。



\end{pinyinscope}