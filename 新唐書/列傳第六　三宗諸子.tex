\article{列傳第六 三宗諸子}

\begin{pinyinscope}

 高宗八子:後宮劉生忠,鄭生孝,楊生上金,蕭淑妃生素節,武後生弘、賢、中宗皇帝、睿宗皇帝。



 燕王忠,字正本。帝始為太子而忠生,宴宮中,俄而太宗臨幸,詔宮臣曰:「朕始有孫,欲共為樂。」酒酣,帝起舞,以屬群臣,在位皆舞,賚賜有差。貞觀二十年,始王陳。永徽初,拜雍州牧。王皇后無子,後舅柳奭說後,以忠母微,立之必親己,後然之,請於帝;又奭與褚遂良、韓瑗、長孫無忌、於志寧等繼請,遂立為皇太子。後廢,武后子弘甫三歲,許敬宗希後旨,建言:「國有正嫡,太子宜同漢劉強故事。」帝召見敬宗曰:「立嫡若何?」對曰:「正本則萬事治,太子,國本也。且東宮所出微,今知有正嫡,不自安;竊位而不自安,非社稷計。」帝曰:「忠固自讓。」敬宗曰:「能為太伯,不亦善乎?」於是降封梁王、梁州都督,賜甲第,實封戶二千,物二萬段。俄徙房州刺史。忠寢懼不聊生,至衣婦人衣,備刺客。數有妖夢,嘗自占。事露,廢為庶人,囚黔州承乾故宅。麟德初,宦者王伏勝得罪於武後,敬宗乃誣忠及上官儀與伏勝謀反,賜死,年二十二。無子。明年,太子弘表請收葬,許之。神龍初,追封,又贈太尉、揚州大都督。



 原悼王孝,永徽元年,始王許,與杞、雍二王同封。早薨。神龍初,追封及謚。



 澤王上金,始王杞。永徽三年,遙領益州大都督。歷鄜、壽二州刺史。武后疾其母,故有司誣奏,削封邑,徙置澧州。久之,後陽若可喜者,表杞王上金、鄱陽王素節聽朝集,義陽、宣城二公主各增夫秩。由是上金為沔州刺史,素節岳州刺史,然卒不朝。高宗崩,詔上金、素節、二公主赴哀。文明元年,徙王畢,又徙王澤。歷五州刺史。載初中,武承嗣諷周興誣上金、素節謀反,召系御史獄。上金聞素節已被殺,即雉經,七子並流死顯州。神龍初,追還官爵,以子義珣嗣王。義珣始被謫,匿身為傭保,而嗣許王瓘〗利其爵邑,告義珣假冒,復流嶺外。開元初,以素節子璆為後,而玉真公主表義珣實上金子,乃奪璆爵,復使義珣嗣王,拜率更令。薨,子潓嗣。



 許王素節,始王雍,授雍州牧。方羈丱,即誦書日千言。師事徐齊聃,淬勉自強,帝愛之。轉岐州刺史,更王郇。母被譖死,出素節為申州刺史。乾封初,詔素節病無入朝。而實不病,乃著《忠孝論》自明。倉曹參軍張柬之以聞,欲帝省其誣,武后滋不悅,坐受賕降王鄱陽,削封戶什七,徙置袁州,錮終身。儀鳳三年,為岳州刺史,更王葛,又徙王,歷三州刺史。與上金同追逮赴都,道聞遭喪哭者,謂左右曰:「病死何可得,而須哭哉?」至龍門驛被縊,年四十三,葬以庶人禮。子瑛等九人並誅,惟琳、瓘、璆、欽古尚幼,長囚雷州。中宗復位,追故封,又贈開府儀同三司、許州刺史,陪葬乾陵。詔瓘嗣王,實封戶四百。開元初,封琳為嗣越王,璆嗣澤王。琳至右監門衛將軍,子隨封夔國公。瓘為衛尉卿,以抑上金子不得封,貶鄂州別駕。因詔外繼嗣王者皆歸宗,乃以嗣江王禕為信安王,嗣蜀王示俞為廣漢王,嗣密王徹為濮陽王,嗣曹王臻為濟國公,嗣趙王琚為中山王,武陽王繼宗為澧國公。瓘累遷太子詹事。薨,贈蜀郡大都督。二子解、需皆幼,以璆子益嗣。天寶十四載,解始襲封王。



 璆,初封嗣澤王,降為郢國公,宮宗正、光祿卿,進封褒信王。初,張九齡撰《龍池頌》,刊石興慶宮,宗子以為不稱盛德,更命璆為頌,建花萼樓北。天寶初,復拜宗正卿。性友弟聰敏,宗子有一善,無不薦延,故宗室在省闥者多璆所啟。薨,贈江陵郡大都督。三子:謙為郢國公、梓州刺史,巽汝南郡公。欽古封巳國公,子賁嗣。



 孝敬皇帝弘,永徽六年始王代,與潞王同封。顯慶元年,立為皇太子。受《春秋左氏》於率更令郭瑜,至楚世子商臣弒其君,喟而廢卷曰:「聖人垂訓,何書此邪?」瑜曰:「孔子作《春秋》,善惡必書,褒善以勸,貶惡以誡,故商臣之罪雖千載猶不得滅。」弘曰:「然所不忍聞,願讀它書。」瑜拜曰:「里名勝母,曾子不入。殿下睿孝天資,黜兇悖之跡,不存視聽。臣聞安上治民,莫善於禮,故孔子稱『不學禮,無以立』。請改受《禮》。」太子曰:「善。」四年,加元服。又命賓客許敬宗、右庶子許圉師、中書侍郎上官儀、中舍人楊思儉即文思殿摘採古今文章,號《瑤山玉彩》,凡五百篇。書奏,帝賜物三萬段,餘臣賜有差。又詔五日一赴光順門決事。總章元年,釋採國學,請贈顏回為太子少師,曾參太子少保,制可。會有司以征遼士亡命及亡命不即首者,身殊死,家屬沒官。弘諫以為「士遇病不及期,或被略若溺、壓死,而軍法不因戰亡,則同隊悉坐,法家曰亡命,而家屬與真亡者同沒。《傳》曰:『與殺不辜,寧失不經。』臣請條別其科,無使淪胥」。詔可。帝幸東都,詔監國。時關中饑,弘視廡下兵食有榆皮、蓬實者,悄然命家令寺給米。義陽、宣城二公主以母故幽掖廷,四十不嫁,弘聞眙惻,建請下降。武后怒,即以當上衛士配之,由是失愛。又請以同州沙苑分假貧民。會納妃裴,而有司奏贄用白雁,適苑中獲之,帝喜曰:「漢獲硃雁,為樂府歌。今得白雁為婚贄,婚乃人倫首,我則無慚。」禮畢,曲赦岐州。帝嘗語侍臣:「弘仁孝,賓禮大臣,未嘗有過。」而後將騁志,弘奏請數怫旨。上元二年,從幸合璧宮,遇耽薨,年二十四,天下莫不痛之。詔曰:「太子嬰沈瘵,朕須其痊復,將遜於位。弘性仁厚,既承命,因感結,疾日以加。宜申往命,謚為孝敬皇帝。」葬緱氏,墓號恭陵,制度盡用天子禮,百官從權制三十六日釋服。帝自制《睿德紀》,刻石陵側。營陵費巨億,人厭苦之,投石傷所部官司,至相率亡去。妃薨,謚哀皇后。無子。永昌初,以楚王隆基嗣。中宗立,詔以主祔太廟,號義宗。開元中,有司奏:「孝敬皇帝宜建廟東都,以謚名廟。」詔可。於是罷義宗號。妃即裴居道女,有婦德,而居道以妃故拜內史納言,歷太子少保、翼國公,為酷吏所陷,下獄死。



 章懷太子賢字明允。容止端重,少為帝愛。甫數歲,讀書一覽輒不忘,至《論語》「賢賢易色」,一再誦之。帝問故,對曰:「性實愛此。」帝語李世勣,稱其夙敏。始王潞,歷幽州都督、雍州牧。徙王沛,累進揚州大都督、右衛大將軍。更名德。徙王雍,仍領雍州牧、涼州大都督,實封千戶。上元年,復名賢。是時,皇太子薨,其六月,立賢為皇太子。俄詔監國,賢於處決尤明審,朝廷稱焉,帝手敕褒賜。賢又招集諸儒:左庶子張大安、洗馬劉訥言、洛州司戶參軍格希玄、學士許叔牙成玄一史藏諸周寶寧等,共注範曄《後漢書》。書奏,帝優賜段物數萬。時正諫大夫明崇儼以左道為武后所信,崇儼言英王類太宗,而相王貴,賢聞,惡之。宮人或傳賢乃后姊韓國夫人所生,賢益疑,而後撰《少陽政範》、《孝子傳》賜賢,數以書讓勒,愈不安。調露中,天子在東都,崇儼為盜所殺,後疑出賢謀,遣人發太子陰事,詔薛元超、裴炎、高智周雜治之,獲甲數百首於東宮。帝素愛賢,薄其罪,後曰:「賢懷逆,大義滅親,不可赦。」乃廢為庶人,焚甲天津橋,貶大安普州刺史,流訥言於振州,坐徙者十餘人。開耀元年,徙賢巴州。武後得政,詔左金吾將軍丘神勣檢衛賢第,迫令自殺,年三十四。後舉哀顯福門,貶神勣疊州刺史,追復舊王。神龍初,贈司徒,遣使迎喪,陪葬乾陵。睿宗立,追贈皇太子及謚。三子:光順、守禮、守義。光順為樂安王,徙義豐,被誅。守義為犍為王,徙封桂陽,薨。先天中,追封光順莒王,守義畢王。



 守禮嗣王,始名光仁,授太子洗馬。武后革命,畏疾宗室,而守禮以父得罪,與睿宗諸子閉處宮中十餘年。睿宗封相王,許出外邸,於是守禮等始居外,改司議郎。中宗即位,復故封,拜光祿卿,實封戶五百。唐隆元年,進封邠王。睿宗立,兼檢校左金吾衛大將軍,出為幽州刺史,遙兼單于大都護,遷司空。開元初,累為州刺史。時寧、申、岐、薛王同為刺史,皆擇僚首持綱紀。守禮惟弋獵酣樂,不領事,故源乾曜、袁嘉祚、潘好禮皆為邠府長史、州佐,督檢之。後還諸王京師,守禮以外支為王,不甚才而多寵嬖,子六十餘人,無可稱者。常負息錢數百萬。或勸少治居產,守禮曰:「豈天子兄無葬者邪?」諸王每白上以為歡。岐王嘗奏守禮知雨,昜帝問故,答曰:「臣無它,當天后時,太子被罪,臣幽宮中,歲被敕杖凡四三,累創痕膚,前雨則沈懣,霽則佳,以此知之。」因泣下,帝為惻然。薨,年七十,贈太尉。子承宏、承寧、承寀可記者。承宏,爵廣武王,坐交非其人,貶房州別駕,還為宗正卿。廣德元年,吐蕃入京師,天子如陜,虜宰相馬重英立承宏為帝,以翰林學士於可封、霍瑰為宰相。賊退,詔放承宏於華州,死。承寧封嗣邠王。承寀,煌王,拜宗正卿,與僕固懷恩使回紇和親,即納其女為妃,封毘伽公主。薨,贈司空。唐制:嗣郡王加四品階,親王子服緋。開元中,張九齡奏:「寧、薛及邠王三子為王者賜紫,餘皆服緋,官不越六局郎,王府掾屬仍員外置。」後從帝至蜀者皆服紫。



 中宗四子:韋庶人生重潤,後宮生重福、重俊、殤帝。



 懿德太子重潤,本名重照,避武后諱改焉。帝為皇太子時,生東宮,高宗喜甚,乳月滿,為大赦天下,改元永淳。是歲,立為皇太孫,開府置官屬。帝問吏部侍郎裴敬彞、郎中王方慶,對曰:「禮有嫡子,無嫡孫。漢、魏太子在,子但封王。晉立愍懷子為皇太孫,齊立文惠子為皇太孫,皆居東宮。今有太子,又立太孫,於古無有。」帝曰:「自我作古若何?」對曰:「禮,君子抱孫不抱子,孫可以為王父尸者,昭穆同也。陛下肇建皇孫,本支千億之慶。」帝悅,詔議官屬。敬彞等奏置師、傅、友、文學、祭酒、左右長史、東西曹掾、主簿、管記、司錄、六曹等官,加王府一級,然卒不補。將封嵩山,召太子赴東都,以太孫留守京師。中宗失位,太孫府廢,貶庶人,別囚之。帝復位,封邵王。大足中,張易之兄弟得幸武後,或譖重潤與其女弟永泰郡主及主婿竊議,後怒,杖殺之,年十九。重潤秀容儀,以孝愛稱,誅不緣罪,人皆流涕。神龍初,追贈皇太子及謚,陪葬乾陵,號墓為陵,贈主為公主。



 譙王重福,高宗時王唐昌郡,徙封平恩。長安末乃進王。神龍初,韋庶人譖與張易之兄弟陷重潤,貶濮州員外刺史,徙合、均二州,不領事。景龍三年,中宗親郊,赦天下,十惡者咸宥,流人得還。重福不得歸,自陳「蒼生皆自新,而一子擯棄,皇天平分,固若此乎!」不報。韋後得政,詔左屯衛大將軍趙承恩、薛思簡以兵護守。睿宗立,徙集州,未行,洛陽男子張靈均說重福曰:「大王居嫡長,當為天子。相王雖平大難,安可越居大位?昔漢誅諸呂,乃東迎代王。今百官士庶皆願王來。王若陰幸東都,殺留守,擁兵西據陜,徇河南、河北,天下可圖也。」重福又遣靈均與其黨鄭愔計,愔亦密招重福為天子,豫尊睿宗為皇季叔,重茂皇太弟,制稱中元克復元年,愔自署左丞相,知內外文武事,以靈均為右丞相、天柱大將軍,知出征事,其餘以次除署。重福自均州與靈均乘驛趨東都,舍駙馬裴巽家。洛陽令候巽,重福驚,遽出,欲劫左右屯營兵,至天津橋,願從者數百人。侍御史李邕遇之,先馳至右屯營,呼曰:「譙王得罪先帝,擅入都為亂。公等勉立功取富貴。」稍稍閉皇城諸門以拒。重福徇右營不能動,趨左掖門,已闔,怒,縱火燒之。左營兵寢逼,眾遂潰,重福走山谷。明日,留守裴談總兵大索,投漕渠死,年三十一,礫其尸。帝詔以三品禮葬。



 節愍太子重俊,聖歷三年王義興,神龍初,王衛,拜洛州牧,實封千戶。俄領揚州大都督。明年為皇太子,與太后喪,殺冊禮,詔在籓食封,歲納東宮。給事中盧粲上言:「太子與列國同入封,不可為法。」詔罷之。重俊性明果,然少法度。既楊璬、武崇訓為賓客,二人馮貴寵,無學術,惟狗馬蹴踘相戲暱。左庶子姚珽數上疏諍導,右庶子平貞慎又獻《孝經議》、《養德》等傳,太子納而不克用。武三思挾韋後勢,將圖逆,內忌太子,而崇訓又三思子,尚安樂公主,常教主辱重俊,以非韋出,詈為奴,數請廢,自為皇太女。三年七月,重俊恚忿,遂率李多祚洎左羽林將軍李思沖、李承況、獨孤禕之、沙吒忠義,矯發左羽林及千騎兵殺三思、崇訓並其黨十餘人,使左金吾大將軍成王千里守宮城,自率兵趨肅章門,斬關入,索韋后、安樂公主、昭容上官所在。後挾帝升玄武門,宰相楊再思、蘇瑰、李嶠及宗楚客、紀處訥統兵二千餘人守太極殿,帝召右羽林將軍劉仁景等率留軍飛騎百人拒之,多祚兵不得進。帝據檻語千騎曰:「爾乃我爪牙,何忽為亂?能斬賊者有賞。」於是士倒戈斬多祚,餘黨潰。重俊亡入終南山,欲奔突厥,楚客遣果毅趙思慎追之,重俊憩於野,為左右所殺。詔殊首朝堂,獻太廟,並以告三思、崇訓柩。睿宗立,加贈謚,陪葬定陵。



 初,重俊被害,官屬莫敢視,惟永和丞甯嘉勖號哭,解衣裹其首,時人義之;楚客怒,收付獄,貶平興丞,卒。至是,亦贈永和令。



 重俊子宗暉,景雲三年封湖陽郡王,天寶中,至太常員外卿,薨。



 睿宗六子:肅明皇后生憲,宮人柳生捴,昭成皇后生玄宗皇帝,崔孺人生範,王德妃生業,後宮生隆悌。



 讓皇帝憲,始王永平。文明元年,武后以睿宗為皇帝,故憲立為皇太子;睿宗降為皇嗣,更冊為皇孫,與諸王皆出閤,開府置官屬。長壽二年,降王壽春,與衡陽、巴陵、彭城三王同封,復詔入閤。中宗立,改王蔡,固辭不敢當。唐隆元年,進封宋。



 睿宗將建東宮,以憲嫡長,又嘗為太子,而楚王有大功,故久不定。憲辭曰:「儲副,天下公器,時平則先嫡,國難則先功,重社稷也。使付授非宜,海內失望,臣以死請。」因涕泣固讓。時大臣亦言楚王有定社稷功,且聖庶抗嫡,不宜更議。帝嘉憲讓,遂許之,立楚王為皇太子,以憲為雍州牧、揚州大都督、太子太師,實封至二千戶,賜甲第,物段五千,良馬二十,奴婢十房,上田三十頃。進尚書左僕射,又兼司徒。讓司徒,更為太子賓客。



 時太平公主有醜圖,姚元崇、宋璟白帝,請出憲及申王成義為刺史,以銷釋陰計,乃以司徒兼蒲州刺史,進司空。玄宗既討定蕭、岑之難,進憲位太尉,贈千戶,固辭,更授開府儀同三司,解太尉、揚州大都督。徙王寧,又兼太常卿。開元十四年,表解卿。久之,復為太尉。歷澤、岐、涇三州刺史,累封至五千五百戶。二十九年薨。



 初,帝五子列第東都積善坊,號「五王子宅」。及賜第上都隆慶坊,亦號「五王宅」。玄宗為太子,嘗制大衾長枕,將與諸王共之。睿宗知,喜甚。及先天後,盡以隆慶舊邸為興慶宮,而賜憲及薛王第於勝業坊,申、岐二王居安興坊,環列宮側。天子於宮西、南置樓,其西署曰「花萼相輝之樓」,南曰「勤政務本之樓」,帝時時登之,聞諸王作樂,必亟召升樓,與同榻坐,或就幸第,賦詩燕嬉,賜金帛侑歡。諸王日朝側門,既歸,即具樂縱飲,擊球、鬥雞、馳鷹犬為樂,如是歲月不絕,所至輒中使勞賜相踵,世謂天子友悌,古無有者。帝於敦睦蓋天性然,雖讒邪亂其間,而卒無以搖。時有鶺鴒千數集麟德殿廷樹,翔棲浹日。左清道率府長史魏光乘作頌,以為天子友悌之祥。帝喜,亦為作頌。



 憲尤謹畏,未嘗干政而與人交,帝益信重,嘗以書賜憲等曰:「魏文帝詩:『西山一何高,高高殊無極。上有兩仟童,不飲亦不食。賜我一丸藥,光耀有五色。服之四五日,身體生羽翼。』朕每言服藥而求羽翼,寧如兄弟天生之羽翼乎?陳思王之才,足以經國,絕其朝謁,卒使憂死,魏祚未終,司馬氏奪之,豈神丸效耶?虞舜至聖,舍象傲以親九族,九族既睦,平章百姓。今數千載,天下歸善焉,此朕廢寢忘食所慕嘆也。頃因餘暇,選仟錄得神方,云餌之必壽。今持此藥,願與兄弟共之,偕至長齡,永永無極也。」後申王等相繼薨,唯憲在,帝親待愈益厚。每生日必幸其第為壽,往往留宿;居常無日不賜遺,尚食總監及四方所獻酒酪異饌;皆分餉之。憲嘗請歲盡錄賜目付史官,必數百紙。後有疾,護醫將膳,騎相望也。僧崇一者療之,少損,帝喜甚,賜緋袍、銀魚。已而疾寢劇,薨,年六十三。帝失聲號慟,左右皆泣下。



 帝以憲實推天下,有高世之行,非大號不稱,乃追謚讓皇帝,遣尚書左丞相裴耀卿、太常卿韋縚持節奉冊。其子璡表陳憲宿素退讓,不敢當大號。制不許。及斂,出天子服一稱,詔右監門大將軍高力士以手書寘靈坐,贈妃元為恭皇后,葬橋陵旁。及葬,敕中使諭璡等,送終之具,使眾見之,示以儉薄。所司請如諸陵,設千味食內壙中,監護使耀卿建言:「尚食料水陸千餘種及馬、牛、驢、犢、麞、鹿、鵝、鴨、魚、雁體節之味,並藥酒三十名,盛夏胎養,不可多殺,考求禮據,無所憑依。陛下每申讓帝之志,務存約素,請蠲省折衷。」詔可。既發引,大雨,有詔慶王潭等涉塗泥,步送十里,號其墓曰惠陵。



 憲嘗從帝按舞萬歲樓,帝從復道上見衛士已食,棄其餘竇中。帝怒,詔高力士杖殺之,憲從容曰:「從復道上窺人之私,恐士不自安,且失大體,豈以性命輕於余饗乎?」帝遽止,謂力士曰:「王於我,可謂有急難也。不然,且誤殺士。」又涼州獻新曲,帝御便坐,召諸王觀之。憲曰:「曲雖佳,然宮離而不屬,商亂而暴,君卑逼下,臣僭犯上。發於忽微,形於音聲,播之詠歌,見於人事,臣恐一日有播遷之禍。」帝默然。及安、史亂,世乃思憲審音云。



 憲本名成器,避昭成太后謚,與申王成義俱改今名。憲子十九人,其聞者璡、嗣莊、琳、瑀。



 璡、嗣莊、琳、瑀。



 璡眉宇秀整,性謹絜,善射,帝愛之。封汝陽王,歷太僕卿。與賀知章、褚庭誨、梁涉等善。薨,贈太子太師。



 嗣莊幼有令名,為太子諭德,封濟陰王。薨,贈幽州大都督。



 琳以秘書監為嗣寧王,從天子幸蜀,薨。



 瑀早有材望,偉儀觀。始封隴西郡公。從帝幸蜀,至河池,封漢中王,山南西道防禦使。乾元初,寧國公主降回紇,詔瑀以特進、太常卿持節冊拜回紇為威遠可汗。瑀亦知音,嘗早朝過永興里,聞笛音,顧左右曰:「是太常工乎?」曰:「然。」它日識之,曰:「何故臥吹?」笛工驚謝。又聞康昆侖奏琵琶,曰:「琵聲多,琶聲少,是未可彈五十四絲大弦也。」樂家以自下逆鼓曰琵,自上順鼓曰琶云。肅宗詔收群臣馬助戰,瑀與魏少游等持不可。帝怒,貶蓬州長史。薨,贈太子太師,謚曰宣。孫景儉。



 景儉字寬中。及進士第。強記多聞,善言古成敗王霸大略,高自負,於士大夫無所屈。王叔文等更譽之,以為管仲、諸葛亮比。叔文敗,景儉以母喪得不坐。韋夏卿守東都,闢幕府。竇群任中丞,引為監察御史,群貶,景儉亦為江陵戶曹參軍。累擢忠州刺史。元和末,入朝,不見用,復為澧州刺史。素與元稹、李紳善。二人方在翰林,言其才。及延英奉辭,景儉自陳見抑遠,穆宗憐之,追詔為倉部員外郎,不遣。閱月,拜諫議大夫。性矜誕,使酒縱氣,語侵宰相,蕭俯、段文昌訴於帝,貶建州刺史。稹得君,為之助,故還為諫議大夫。與馮宿、楊嗣復、溫造、李肇等集史官獨孤朗所,景儉醉,至中書,慢罵宰相王播、崔植、杜元穎,吏為遜言厚謝,乃去,坐貶漳州刺史,宿等皆逐矣。未及漳,稹輔政,改楚州刺史,議者謂景儉辱丞相,貶未至即遷,非是。稹懼,改少府少監,悉還宿等。景儉既湮厄不得志,卒。然其為人輕財,篤於義,既沒,士悵悼之。



 惠莊太子捴,本名成義。初生,武后以母賤,欲不齒,以示浮屠萬回,回詭曰:「此西土樹神,宜兄弟。」後喜,乃畜之。垂拱三年,始王恆,與衛、趙二王同封。俄改王衡陽。睿宗立,進王申,與岐、薛二王同封。累遷右衛、金吾二大將軍,實封至千戶。進司徒,兼益州大都督,四為州刺史。開元八年,停刺史,復為司徒。薨,冊書贈太子及謚,陪葬橋陵。捴性寬裕,儀貌環重。無嗣,詔以讓帝子珣嗣,為懷寧王,徙封同安。薨。天寶中,復以讓帝子嗣。



 惠文太子範,始名隆範。玄宗立,與薛王隆業避帝諱去二名。初王鄭,改封衛。俄降封巴陵,進王岐,為太常卿、並州大都督、左羽林大將軍。從玄宗誅太平公主,以功賜封,與薛王業並滿五千戶。歷為州刺史,遷太子太傅。開元十四年薨,冊書贈太子及謚,陪葬橋陵。帝哭之慟,徹常膳至累旬,群臣勉請乃復。



 範好學,工書,愛儒士,無貴賤為盡禮。與閻朝隱、劉廷琦、張諤、鄭繇等善,常飲酒賦詩相娛樂。又聚書畫,皆世所珍者。初,隋亡,禁內圖書湮放,唐興募訪,稍稍復出,藏秘府。長安初,張易之奏天下善工潢治,乃密使摹肖,殆不可辨,竊其真藏於家。既誅,悉為薛稷取去,稷又敗,範得之,後卒為火所焚。駙馬都尉裴虛己善讖緯,坐私與範游,徙嶺南,廷琦貶雅州司戶,諤為山茌丞,然帝於範無少間也,謂左右曰:「兄弟情天至,於我豈有異哉!趨競者強相附,我終不以為纖介。」時王毛仲等起賤微,暴貴,諸王見必加禮,獨範接之自如。子瑾嗣。



 瑾落魄不飭名檢,沈酒色,歷太僕卿,封河東王,暴薨,贈太子少師。天寶中,復以薛王子略陽公珍為嗣岐王。



 珍儀觀豐偉,為宗正員外卿,與蔚州鎮將硃融善。融嘗言珍似上皇,因有陰謀,往語金吾將軍邢濟曰:「關外寇近,京師草草,奈何?」濟曰:「我金吾,天子押衙,以死生從,安自脫?」融曰:「見嗣岐王無慮矣。」濟以聞,肅宗詔廢珍為庶人,賜死,融黨皆誅,擢濟為桂管防禦使。



 惠宣太子業,始王趙,降封中山,授都水使者。徙鼓城,兼陳州別駕,進王薛,為羽林大將軍、荊州大都督。以好學授秘書監。開元初,進太子少保,即拜太保,累歷州刺史。



 初,母早終,從母賢妃鞠之。八年,迎賢妃外邸,事之甚謹。其女弟淮陽、涼國二公主亦早卒,撫甥與己子均,帝益愛之。嘗被疾,帝自祝禬。既愈,幸其第,置酒賦詩為初生歡。帝嘗不豫,業妃弟內直郎韋賓與殿中監皇甫恂妄言休咎事,賓坐死,恂貶錦州刺史。妃恐,降服待罪,業亦不敢入謁,帝聞,遽召之,業伏殿下請罪,帝趨就執其手曰:「吾所猜於兄弟者,天地共咎之!」遂復燕歡,仍諭妃復位。俄進司徒。二十二年,業有疾,帝憂之,一昔容發為變,因假寢,夢獲方,寤而業少閑,邠王守禮等請以事付史官。及薨,帝悲不能食,冊書加贈及謚,陪葬橋陵。



 十一子,其聞者瑗、瑒、肙。帝後追思業,引見瑗等,傷之,乃下詔共賜實封千戶。瑗為樂安王。瑒滎陽王、宗正卿。肙為嗣薛王,歷鴻臚卿。天寶中,肙舅韋堅為李林甫所構,坐貶夷陵別駕,徙置夜郎、南浦。及安祿山亂,乃還京師。



 曾孫知柔,嗣王,再為宗正卿。久之,擢京兆尹。始,鄭、白渠梗壅,民不得歲。知柔調三輔,治復舊道,灌浸如約,遂無旱虞,民詣闕請立石紀功,知柔固讓得止。加累檢校司徒、同中書門下平章事。又詔營緝太廟,判度支,充諸道鹽鐵轉運使。昭宗出莎城,獨知柔從,乘輿器用庖頓皆主之,大細畢給。性儉約,雖位通顯,無居第。未幾,出拜清海軍節度使,在鎮廉潔,貢獻時入,進檢校太傅,兼侍中。仕凡四紀,常為宗室冠。卒於鎮。



 隋王隆悌,始封汝南王。早薨,睿宗追王,贈荊州大都督,爵不傳。



 贊曰:中宗失道,身為母所廢,妻所弒,而四子皆不得其死,嗣亦不傳,殆天穢其德而絕之,何耶?彼固自絕於天云爾。睿宗有聖子,一受命,一追帝,三贈太子,天與之報,福流無窮,盛歟!



\end{pinyinscope}