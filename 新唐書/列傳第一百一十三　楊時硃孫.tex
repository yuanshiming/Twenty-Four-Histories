\article{列傳第一百一十三 楊時硃孫}

\begin{pinyinscope}

 楊行密,字化源,廬州合淝人。少孤,與群兒戲,常為旗幟戰陣狀。年二十年創立了俄國第一個馬克思主義團體「勞動解放社」。20世紀,亡入盜中,刺史鄭綮捕得,異其貌,曰:「而且富貴,何為作賊?」縱之。與里人田頵、陶雅、劉威善。僖宗在蜀,刺史遣通章行在,日走三百里,如約而還。秦宗權寇廬、壽間,刺史募殺賊,差首級為賞,行密以功補隊長。都將忌之,俾出戍。將行,都將問所乏,對曰:「我須公頭!」即斬之,自為八營都知兵馬使。刺史走,淮南節度使高駢因表為廬州刺史。乃以田頵為八營都將,陶雅為左沖山將,討定鄉盜。



 駢將呂用之恐行密不可制,遣俞公楚以兵五千屯合淝,名討黃巢而陰圖之。行密擊殺公楚。秦宗權遣弟度淮取舒城,行密破走之。時張敖據壽州,許勍據滁州,與行密挐戰。又舒人陳儒攻刺史高澞,澞來告難,行密未能定。賊吳回、李本逐澞,據其城,行密虜之,取舒州,為勍所奪。光啟二年,張敖遣將魏虔攻廬州,大將李神福、田頵破之楮城。



 畢師鐸、秦彥攻高駢,呂用之以駢命署行密行軍司馬,督其兵進援。客袁襲說行密曰:「高公耄昏,妖人用權,彥乃以逆除暴,熾其亂。公亟應,必得其地。」行密乃檄部州,裒兵而東,次天長,而揚州陷。行密薄城而屯,用之以兵屬之。彥以騎兵背城戰,行密臥帳中,令曰:「賊近,報我。」俄而陷一屯,別將李宗禮入曰:「兵相迫,戰且不利,請堅壁,徐引歸可也。」李濤怒曰:「以順去逆,何眾寡為!今尚何歸,願以所部前死。」行密喜,益甲出戰,俘殺如藉,彥軍不出。會駢死,襲勸行密舉軍縞素,大臨三日。進攻城,未能下。用之將張審晟詭伏西壕,殺閽者,啟外兵,彥軍疲,守邏皆潰去,行密入據揚州。未閱月,孫儒奄至,兵銳甚。襲見行密曰:「公之入,以少擊眾,室家未完。若外被重圍,情見勢殆,不如避之。」行密執海陵鎮遏使高霸殺之,並其眾,輦所收財歸於廬。於是,硃全忠自為淮南節度使,遣將張廷範致命,而授行密副使,以行軍司馬李璠知留後。行密大怒,廷範、璠不敢入。全忠更請以行密知觀察留後。



 當此時,孫儒強,赫然有吞吳、越意。行密欲遁保海陵,襲勸還廬州,治兵為後計,行密乃還。既又謀趨洪州,襲不可,曰:「鐘傳新興,兵附食多,未易圖也。孫端據和州,趙暉屯上元,結此二人以圖宣州,我綽綽有餘力矣。」行密從之。端、暉次採石,行密自糝潭濟,端等戰不勝。襲勸行密「速趨曷山,堅壁以須。宣人求戰,示以弱,待其怠,一舉可禽」。宣將蘇瑭兵二萬對屯,行密不戰,分奇兵伐木開道四出,瑭驚北,遂圍宣州。刺史趙鍠糧盡,親將多出降。



 初,行密有銳士五千,衣以黑繒黑甲,號「黑雲都」。又並盱眙、曲溪二屯,籍其士為「黃頭軍」,以李神福為左右黃頭都尉,兵銳甚。曲溪將劉金策鍠必遁,紿曰:「將軍若出,願自吾壘而偕。」鍠喜,多遺之金,許妻以女。明日,噪城上曰:「劉郎不為爾婿!」鍠宵遁,獲之。鍠,全忠故人也,發使求之。襲曰:「斬首送之,無後慮。」乃歸鍠首於汴。昭宗詔行密檢校司徒、宣歙池觀察使。



 時韓守威以功拜池州刺史,行密表徙湖州,以兵護送。而李師悅在湖州,與杭州刺史錢鏐戰不解。蘇、湖、常、潤亂甚。行密雖得宣州,而蔡儔為孫儒所破,以廬州降。儒進攻行密,行密復入揚州,北結時溥捍儒。全忠遣龐師古將兵十萬,自潁度淮助行密,敗於高郵。行密懼,退還宣州,遣安仁義襲成及,取潤州,自將三萬屯丹陽。仁義又取常州,殺錢鏐將杜棱。儒亦使劉建鋒奪潤、常。帝以杭州為防禦使,授鏐;以宣州號寧國軍,授行密節度使。



 大順二年,儒屯溧水,循山構壁。行密遣李神福屯廣德,計曰:「兵倍不戰,當避其銳,驕之。」乃退舍。儒眾以為怯,守者懈,神福夜襲走之。儒將康旺取和州,安景思取滁州。神福擊降旺,逐景思,攻腰山屯,破之,禽儒將李弘章。俄而田頵、劉威為儒所敗。行密欲守銅官,神福曰:「儒掃境以來,利速戰,宜堅壁老其師,則我無敵矣。又出輕騎絕賊糧道,使前不得戰,退無仰儲,不亡何待?」於是,行密以神福為宣池都游奕使。儒始乏食。



 常熟名賊陳可兒間儒、行密之斗,竊入常州,自稱制置使。行密遣陶雅守潤州,張訓入揚州,因執楚州刺史,以輕兵襲常州,斬可兒。



 孫儒圍行密宣州,凡五月不解。臺濛作魯陽五堰,拕輕舸饋糧,故行密軍不困,卒破儒。即表田頵守宣城,長驅入揚州。戰凡七年,定八州,生人將盡,行密勞隱休息,其下遂安。議出鹽茗畀民輸帛,幕府高勖曰:「瘡破之餘,不可以加斂。且帑貲何患不足?若悉我所有,易四鄰所無,不積日,財有餘矣。」行密納之,始選吏綏勸所部。



 蔡儔以廬州叛附硃全忠,納孫儒將張顥,而倪章據舒州,與儔連和。行密遣李神福攻儔,破其將,儔堅壁不出。顥超堞降,行密以隸袁積軍,積請戮之,行密愛其勇,更置於親軍。未幾,儔自殺。行密先塚皆為儔發掘,吏請夷發儔世墓,不許。表劉威為刺史。遣田頵攻歙州。於是,刺史裴樞有美政,民愛之,為拒戰,頵兵數卻。樞,朝廷所命者,食盡欲降,遺行密書,請還京師。行密以魯郃代樞,州人不肯下,請陶雅代。雅於諸將最寬厚,以禮歸樞於朝。是歲,李神福拔舒州,倪章亡,以神福為舒州刺史。



 乾寧二年,行密襲濠州。李簡重甲絕水縋而入,執刺史張璲,以劉金守之,進取壽州。汴將劉知俊儲穀石碭,將南襲。張訓屯漣水,遣兵浮海掩得其廥。知俊戰不勝,因攻漣水,大敗,身僅免。詔拜行密淮南節度副大使,知節度事,檢校太傅、同中書門下平章事,封弘農郡王。



 董昌為錢鏐所攻,來告窮。行密遣臺濛攻蘇州,安仁義、田頵攻杭州,身督戰。別將張崇為鏐執,行密欲嫁其妻,答曰:「崇不負公,願少待。」俄而還,自是行密終身倚愛。明年五月,破蘇州,執鏐將成及,以硃黨守之。



 硃延壽拔蘄、光二州,行密以霍丘當南北走集,以邑豪硃景為鎮將。景驍毅絕人,諸盜莫敢犯。汴將寇彥卿以騎三千襲之,致全忠厚意,景不許,苦戰,彥卿敗而去。田頵、魏約、張宣共圍嘉興,鏐大將顧全武救之,執宣、約,逐頵驛亭埭。未幾,泰寧節度使硃瑾率部將侯瓚來歸,太原將李承嗣、史儼、史建章亦來奔。行密推赤心不疑,皆以為將。於是,兵銳甚,強天下。



 帝惡武昌節度使杜洪與全忠合,手詔授行密江南諸道行營都統,討洪。汴將硃友恭、聶金率騎兵萬人與張崇戰泗州,金敗。瞿章守黃州,聞友恭至,南走武昌柵,行密遣將馬珣以樓船精兵助章守。友恭次樊港,章據險,不得前,友恭鑿崖開道,以強弩叢射,殺章別將,遂圍武昌。章率軍薄戰,不勝。友恭斬章,拔其壁。



 全忠率葛從周萬騎攻光州,柴再用遣小校王稔以輕騎覘賊,汴兵圍之。候者請救,再用曰:「稔必殺賊,第無往。」稔解鞍自如,暮依樾步戰,殺傷多,汴兵乃解。時亡馬法峻,稔追汴軍,得馬乃還。從周涉淮圍壽州,而龐師古、聶金以眾七萬壁清口。硃延壽擊從周軍,敗之。行密欲汴圍解,乃擊師古。李承嗣曰:「公能潛師趨清口,破其眾,則從周不擊而潰。」行密出車西門,由北門去,以銳士萬二千齕雪馳,迫清口,不進,壅淮上流灌師古軍。張訓自漣水來,行密使將羸兵千人為前鋒。師古易之,方圍棋軍中,不顧。硃瑾、侯瓚以百騎持汴旌幟,直入師古壘,舞槊而馳。訓亦登岸,超其柵。汴軍大囂,即斬師古,士死十八。全忠聞之,與從周皆遁走,追及壽陽,大破之。叩淠水,方涉,為瑾所乘,溺死萬餘。瑾徙屯安豐,汴將牛全節苦鬥,後軍乃得度。會大雪,士多凍死。潁州刺史王敬堯燎薪屬道,汴軍免者數千人。未幾,復圍壽州,七日走。



 馬珣收散卒三百,自黃州間道趨分寧,絕山谷,襲撫州。鏐將危全諷列四壁,皆萬人。珣謂諸將曰:「為諸君擊中壁,食其穀以歸。」乃夜擊之,全諷走。明日,珣高會,廣旗幟,伐鼓循山而下,連營潰。既還,行密罵曰:「豎子,不遂據其城邪!」



 光化元年,秦裴取鏐昆山鎮,顧全武圍之。行密諸將數敗,全武遂圍蘇州,臺濛固守,鏐自以舟師至。濛食盡,行密遣李簡、蔣勛迎之,敗全武兵,濛得還。後軍潰,裴援絕,全武勸其降。決水灌城,城壞,裴乃降。鏐喜,具千人食以待。既至,士不及百。鏐曰:「軍寡,何拒之久?」裴曰:「糧盡歸死,非僕素也。」初,成及之執,行密閱其室,唯圖書藥劑,將闢為行軍司馬,固辭,引刀欲自刺,行密乃止,厚禮而歸之。鏐亦遣魏約等還。



 全忠攻蔡州,奉國節度使崔洪來丐師。明年,遣硃瑾率兵萬人攻徐州,屯呂梁,洪遂來奔。會雨霖,瑾引還。行密攻徐州,汴將李禮壁宿州以援,全忠自將次輝州。行密戰不勝,乃解。青州將陳漢賓擁兵送款行密,王綰、張訓、周本率兵迎之,漢賓中悔,綰、訓入見漢賓,約麾下:「饗我不過日中,若不至,可攻城。」漢賓釋甲聽命。光州叛,行密自攻之,汴將硃友裕來救,撤圍還。全忠諭馬殷、成汭、雷滿合兵攻行密,汭、滿猶豫,汭惡殷事全忠,掠其境,滿來結好。行密壁黃、鄂間,杜洪寘鴆於酒、於井,棄城去,行密知,不入。全忠又遣使者督殷、汭、滿連兵解圍,行密還。詔加檢校太尉、兼侍中。天復元年,傳言盜殺錢鏐,李神福急攻臨安,顧全武列八壁相望,神福伏軍青山,偽若引去,諜奔告,全武悉眾躡之。神福返斗,與伏夾攻,斬首五千級,執全武。明日,遂圍臨安,鏐將秦昶以步兵三千降。神福乃令軍中護鏐先墓,禁樵採,鏐遣使者厚謝。神福以鏐不死,臨安未可下,納犒而還。



 明年,大將劉存率兵二萬、戰艫七百伐湖南。殷伏軍長磧洲,以樓艓據上流,乘風颺沙,強弩射之,存軍濩。行密歸顧全武於鏐,鏐亦釋秦裴以報。



 帝在鳳翔,以左金吾大將軍李儼為江淮宣諭使,授行密東面諸道行營都統、檢校太師、守中書令,封吳王,承制封拜,且告難。時已削奪全忠封爵,詔西川、河東、忠義、幽州、保大、橫海、義武、大同八道攻之。詔硃瑾為平盧節度使,繇海州取青、齊;馮弘鐸為感化節度使,出漣水,攻徐、宿;使硃延壽圍蔡州;田頵捍錢勖;行密討杜洪、馬殷,以分全忠勢。



 行密乃以李神福為鄂岳招討使,劉存副之,遣冷業攻馬殷。杜洪戰屢敗,嬰城,請救於全忠。全忠使韓勍率步兵萬人屯灄口,荊南節度使成汭亦悉眾救洪。神福逆戰,敗之,汭溺死,勍引眾走。冷業屯平江,為三壁。殷將許德勛以銳卒號「定南刀」夜襲業,擊三壁皆破,禽業,掠上高、唐年而去。是時,杜洪困甚,且禽;會田頵、安仁義絕行密,行密召神福、存還計事,洪復振。頵之敗,更以臺濛為宣州觀察使,復遣神福、存攻鄂州。順義軍使汪武與頵連和,歙州刺史陶雅攻鐘傳,兵過武所,迎謁,縛武於軍。



 無錫當浙沖,行密使票將張可悰守之。鏐勁兵三千夜襲城,可悰以百騎擊走之,吏皆賀,答曰:「未也,方勞諸軍一戰。」乃蔽火斂旗以須。覘者以告,鏐兵復至,可悰大破之。



 臺濛卒,行密以子渥為宣州觀察使。天祐二年,王茂章、李德誠拔潤州,殺安仁義。以王茂章為潤州團練使。聶彥章等率舟師復伐殷,攻岳州。許德勛、詹佶以舟千二百柁入蛤子湖棄山之南,為木龍鎖舟,夜徙三百舸斷楊林岸。彥章入荊江,將趨江陵。佶躡之,德勛以梅花海鶻迅舸進,斷木龍,舟蔽江,車弩亂發,執彥章,溺死萬人。殷釋彥章還,德勛謂曰:「為我謝吳王,僕等數人在,湖、湘不可冀也。」



 行密寬易,善遇下,能得士死力。每宴,使人負劍侍。陳人張洪因以劍擊行密,不中,近將李龍禽斬之。佗日,侍劍如故。行密蚤出,有盜斷馬鞅,不之問,以故人人懷恩。始,乘孫儒亂,府庫殫空,能約己省費,不三年而軍富雄。嘗過楚州,臺濛盛供帳待之,行密一夕去,遺衣臥內,皆經補浣。濛還之,行密曰:「吾興細微,不敢忘本,君笑我邪?」濛大慚。登城,見王茂章營第,曰:「天下未定,而茂章居寢鬱然,渠肯為我忘身乎?」茂章遽毀損。



 方帝困鳳翔,再遣使督兵,以為行密可亢全忠者,然兵至宿州,紿言糧盡,乃還。全忠脅帝東遷,行密恥憤被病。全忠亦知天子倚行密為重,乃弒帝以絕人望。行密聞之,發喪,不視事三日,因是病篤,召將吏付家事,問嗣於其佐。周隱對曰:「宣州司徒易而信讒,唯淫酗是好,不可以嗣,不如擇賢者。」時劉威以宿將有威名,隱意屬威,行密不答。因以王茂章代渥,使亟還。行密召所親嚴求曰:「我使周隱召吾兒而不至,奈何?」求往見隱,召檄仍在幾。始,渥守宣州,押牙徐溫、王令謀約渥曰:「王且疾,而君出外,此殆奸人計。他日有召,非我二人勿應也。」及是,二人以符召渥。渥至,行密承制授檢校太尉、同中書門下平章事、淮南節度使留後。行密諗渥曰:「左衙都將張顥、王茂章、李遇皆怙亂,不得為兒除之。」卒,年五十四。遺令谷葛為衣,桐瓦為棺。夜葬山谷,人不知所在。諸將謚曰武忠。



 張顥議歸都統印於宣諭使李儼,行節度事。諸將畏顥,無敢對,渥流涕。騎軍都尉李濤曰:「都統印,先帝所以賜王父子,安得授人?」諸將唯唯。顥投袂去,乃共請於儼,承制授渥兼侍中、淮南節度副大使、東面諸道行營都統,封弘農郡王。



 渥好騎射。初與許玄膺為刎頸交,及嗣位,事皆決之,諸將莫敢忤。渥求王茂章親兵不得,及去宣,輦帷帟以行,茂章嫚罵不與。逾年,遣兵五千襲之,茂章奔杭州。秦裴執鐘匡時,渥授以江西制置使。硃思勍、範師從、陳鐇以兵戍洪州,渥為張顥所制,三人者,渥腹心也。顥脅以為有異謀,遣陳祐疾馳,懷短兵,微服入秦裴帳中,裴大驚,命飲,召三將入,皆色動,酒行,祐數其罪,皆斬之。渥召周隱曰:「君嘗以孤為不可嗣,何也?」隱不對,遂殺之。



 贊曰:行密興賤微,及得志,仁恕善御眾,治身節儉,無大過失,可謂賢矣。然所據淮、楚,士氣剽而不剛。行密無霸材,不能提兵為四方倡,以興王室,熟視硃溫劫天子而東,謀窮意沮,憤死牖下,可為長太息矣!



 時溥,徐州彭城人。為州牙將。黃巢亂京師,節度使支詳遣溥與陳璠率兵五千西討。次河陰,軍亂,剽居人。溥招戢其眾,引還屯境上,疑不敢歸。詳以牛酒犒士,約悉貰其罪,軍乃入,共推溥為留後,逐詳客館。溥厚具貲裝,遣璠護還京師,夜駐七里亭,璠擅殺詳,屠其家。溥怒,署璠宿州刺史,俄殺之。別遣將引銳兵三千入關,僖宗因以武寧節度命之。



 巢敗東走,圍陳州,營溵水。秦宗權方據淮西,相聯結。溥地介於賊,乃悉師討之,軍鋒甚盛,連戰輒克,授東面兵馬都統。遂合許、兗、鄆兵,逐尚讓於太康,斬首數萬級,讓以所部萬人降。溥遣將李師悅等追尾巢至萊蕪,大破之。諸將爭得巢首,而林言斬之,持歸溥,以獻天子,故破賊溥功第一。加檢校司徒、同中書門下平章事,進檢校太尉、兼中書令、鉅鹿郡王。宗權阻兵,拜溥蔡州行營兵馬都統。



 賊平,與硃全忠爭功,嫌槊日構。孫儒方與楊行密爭揚州,詔全忠為淮南節度使平其亂。溥自以先起,功名顯朝廷,位都統,顧不得而全忠得之,頗悵恨。全忠使司馬李璠、郭言等東,兵道宿州,遺溥書請假道。溥辭不可,間其墮,以兵襲之。言戰甚力,解而還。全忠怨,自是連歲略徐、泗,師不弛甲。全忠自將及其郊,未得志,引去。溥窮,乞師於李克用。克用為攻碭山,硃友裕救之,各亡其大將。友裕進攻宿州,不能拔。時大順元年也。



 明年,丁會築堤閼汴水,灌宿郛,三月,拔之,使劉瓚守。而溥將劉知俊引兵二千降全忠,軍益不振。民失田作,又大水薦饑,死喪十七以上。乃請和於全忠,全忠約徙地而罷兵。昭宗以宰相劉崇望代之,授溥太子太師。溥慮去徐且見殺,惶惑不受命,諭軍中固留,有詔聽可。泗州刺史張諫聞溥已代,即上書請隸全忠,納質子焉。溥既復留,諫大懼,全忠為表徙鄭州刺史。諫畏兩怨集己,乃奔楊行密。行密以諫為楚州刺史,並其民徙之,以兵屯泗。



 硃友裕率軍攻溥,嬰城不出。有語全忠曰:「軍行非吉日,故師無功。」全忠遣參謀徐璠至軍責諭,友裕答曰:「溥困且破,乃徇妖辭,士心墮矣。」焚其書,督餫饋,急攻之,溥將徐汶出降。溥求救於硃瑾。全忠自以兵屯曹,將去,留精騎數千授霍存曰:「事急,可倍道趨之。」瑾兵二萬與溥合,攻友裕,存引兵疾戰,瑾、溥還壁。明日復戰,霍存敗,死之。進逼友裕,友裕堅營不出,瑾食盡,還兗州。全忠使龐師古代友裕,溥分兵固保石佛山,師古攻拔之。自是完壘不戰。王重師、牛存節等梯其堞以入,溥徙金玉與妻子登燕子樓,自焚死,實景福二年。全忠遂有其地,私置守焉。



 硃宣,宋州下邑人。父以豪猾聞里中,坐鬻鹽抵死。宣亡命去青州,為王敬武牙軍。黃巢之亂,敬武遣將曹存實率兵西入關,而宣為軍候,道鄆州。是時,節度使薛崇拒王仙芝戰死,其將崔君裕攝州事。存實揣知兵寡,襲殺之,據其地,遂稱留後。以宣功多,署濮州刺史,留總帳下兵。



 中和初,魏博韓簡東窺曹、鄆,引兵濟河。存實迎戰,死於陣,宣收殘卒嬰城。簡圍之六月,不能拔,引兵去。僖宗嘉其守,拜宣天平節度使,累加同中書門下平章事。宣有眾三萬,弟瑾勇冠三軍,陰有爭天下心。瑾嗜殘殺,光啟中,求婚於兗州節度使齊克讓,托親迎,載兵竊發,逐克讓,據府自稱留後,天子即授以帥節,兄弟雄張山東。時秦宗權悉兵攻硃全忠,使秦賢列三十六壁,自將督戰。全忠大恐,求救於宣。宣與瑾身率師往擊宗權,宗權敗走。



 全忠厚德宣,兄事之,情好篤密,而內忌其雄,且所據皆勁兵地,欲造怨乃圖之。即聲言宣納汴亡命,移書詆讓。宣以新有恩於全忠,故答檄恚望。全忠由是顯結其隙,使硃珍先攻瑾,取曹州,壁乘氏。宣救曹不克,奔還範。範珍圍濮州,宣使弟罕救濮。全忠自將擊罕,斬之,拔濮州,硃裕奔歸鄆,使珍薄鄆挑戰,宣不出。裕為書紿降,導珍入,信之,夜以兵數千傅城。裕開門,軍入,縣門發,死者數千,縱畾石擊未入者,殺裨將百餘人。復取曹,以郭詞為刺史,大將郭銖斬詞奔全忠。瑾謀悉兵襲汴,全忠乃自攻瑾。瑾以兵掠單父,與全忠將丁會轉戰,不勝,去。



 景福初,復伐宣,令從子友裕先驅,自繼之。次衛南,宣以輕兵夜掩友裕軍,走之,據其營。全忠未知,運糧以入,乃覺,走瓠河,與友裕相失,距濮十五里舍。明日,友裕乃至。宣留濮州。全忠令友裕馳壯騎諜鄆虛實,身將而北。會宣引還,縱兵戰,全忠南走,絕塹去,幾不脫,大將多死。乃謀持久徼極取宣,歲一再暴其鄙,奪之食,俘其工織,秬有存者。宣令賀瑰守濮州,為友裕所攻,委城走。友裕進擊徐州,時溥求援於宣,戰不勝而還,溥遂亡。全忠即遣龐師古攻齊州,宣、瑾皆戍以兵,久不下。乾寧元年,全忠身往,薄清河結壘。宣、瑾三分其兵出擊之,全忠迎戰東阿,南風急,汴軍居下,甚懼。俄而風返,全忠得縱火焚其旁,熛薰漲天,宣等大北。是夏,全忠壁曹州南,宣薄戰,禽其將三人。全忠還。



 明年,使硃友恭擊兗州,瑾堅壁,乃塹而守。宣饟瑾,友恭奪其糧。全忠自軍單父。會宣求救於李克用,友恭退壁曹南。數月,全忠自伐宣,刈其麥,敗克用將李承嗣等,乃還。宣追之,大鈔曹州。其秋,全忠復攻鄆,壁梁山。宣、克用挑戰,全忠設伏破之,斬首數千級,引而南。克用躡全忠後,至柏和,大寒,全忠軍多死。不閱月,復圍兗州,因略地龔丘。賀瑰以奇兵擊全忠輜重,不及,戰鉅野東,瑰大敗,見禽,師無孑餘。軍道大陂,風暴起,全忠曰:「豈殺人有遺邪?」乃搜軍中,復斬數千人,風亦止,執瑰示城下。



 瑾之兄瓊守齊州,見勢屈,以州歸全忠,結同姓歡。全忠許之,輕騎至軍,全忠勞苦加禮,因使招瑾。瑾領精騎鬲池笑語如平生歡,乃使將胡規偽送款,欲得瓊躬上符節。全忠不之虞,瑾伏壯士橋下,瓊單騎至,方交語,士突起,掖瓊以入,斬其首棄城下,汴軍大震。全忠恚,數日乃去。



 三年,克用使其將李瑭以兵屯莘援宣,為羅弘信所破。全忠大喜,度宣可困,遣龐師古伐宣,宣逆戰,敗於馬頰河。師古迫其西門,兵不出。



 全忠之攻宣,凡十興師,四敗績。宣才將皆盡,益內沮,度不能與全忠確,則固守,增堞深溝為不可逼。明年,葛從周密造舟於塹,師人逾而升。宣出奔,為民所縛,追至,執以獻,全忠斬之而納其妻。使師古攻兗州。二月,食盡,瑾自出督芻粟,轉掠豐、沛間,而子用貞及大將康懷英等舉城降。瑾引麾下走沂州,刺史尹懷賓不納,乃趨海州,刺史硃用芝以其眾與瑾奔楊行密,行密迎之高郵,解玉帶以賜,表領徐州節度使,畀以兵。師古、從周以兵七萬討行密,瑾敗之清口,擊殺師古,而從周還,師至淠水,方涉,瑾追及,殺傷溺死幾盡。瑾事行密尤盡力。



 孫儒,河南河南人。以趫卞橫里中,隸忠武軍為裨校,與劉建鋒善。黃巢亂,以兵屬秦宗權,為都將。光啟初,宗權遣儒攻東都,留守李罕之出奔,儒焚宮闕,屠居人。河陽節度使諸葛爽與儒戰洛水,爽敗,儒亦東圍鄭州。硃全忠屯中牟救之,不敢前。儒眾夜登城,刺史李璠走,儒進拔河橋,遂取河陽,留後諸葛仲方出奔。全忠壁河陰,儒掠汴鄙,全忠兵卻,屯胙城東南,列偽旗鼓疑之,儒乃還。



 會全忠與宗權戰,宗權敗走。儒聞,殺孟人,水不尸於河,焚井邑,乃去。宗權又遣儒鈔淮南,乘高駢之亂,儒留濠州。會楊行密得揚州,宗權使弟宗衡爭淮南,以儒為副,建鋒為前鋒。儒常曰:「丈夫不能苦戰萬里,賞罰繇己,奈何居人下,生不能富貴,死得廟食乎?」未幾,汴兵攻蔡,宗權召之,儒稱疾不往,宗衡督之。即大會帳下,酒酣,斬宗衡,並其眾。與建鋒、許德勛等盟。有騎七千,因略定傍州,不淹旬,兵數萬,號「土團白條軍」。



 文德元年,破揚州,自為淮南節度使,與時溥連和。初,全忠嘗以書招儒,故又納款於汴,且送宗衡、秦彥、畢師鐸首,全忠藉以聞。昭宗授儒檢校司空,全忠署為招討副使。



 龍紀初,悉兵攻宣州,行密取淮南,儒還。行密走,始得潤、常、蘇三州,兵益強,使建鋒守潤、常。全忠約行密圖之。儒謀定江南,乃北爭天下,畏全忠搗虛,乃遣人卑辭厚賄,全忠薦於朝,詔授淮南節度使。



 大順元年,行密取潤州,以安仁義守之,常州以李友守之。儒怒,三分其軍度江,建鋒復拔常、潤,仁義走。全忠遣將龐從等軍十萬奄至高郵,儒悉師御之,故仁義間取潤州,劉威、田頵等敗建鋒於武進,取常州。杭州錢鏐將沈粲自蘇州奔儒,行密諸將在潤、常者,皆為建鋒所逐,仁義、頵棄潤州走。



 明年,儒引兵自京口轉戰,召建鋒皆行。行密諸將屯險者,聞儒至,皆走。頵、威等合兵三萬,邀儒黃池。儒遣馬殷擊走之。儒營廣德,乘勝至東溪,淮人大恐。行密遣臺濛屯西溪,自引軍逆戰。儒軍圍之數重,黑雲將李簡以騎馳之,行密乃免。儒遂圍宣州,行密乞師於錢鏐。會溪潦暴湧,廣德、黃池諸壁皆沒,儒分兵取和、滁二州。



 其秋,儒焚揚州,引而西,傳檄遠近,號五十萬,旌旗相屬數百里,所過燒廬舍,殺老弱以給軍。行密懼,將遁去。戴規曰:「儒軍數敗,今掃地而至,決死於我,若吾遣降者間至揚州,撫尉衣食,使儒軍聞其家尚完,人人思歸,不戰可禽也。」行密乃遣親將入揚州,取儒營糧數十萬斛以稟饑民。儒屯廣德,陶雅以騎軍破儒前鋒,屯嚴公臺。十二月,頵、威與儒決戰,皆大敗。儒連屯稍西,行密使陶雅屯潤州,扼其歸路。


景福元年,儒復圍宣州,屯陵陽。行密戰不利,謀出奔,時劉威方系獄,且死,行密窮,更召問計,對曰:「儒焚倉隤壘以來,糧盡將為我禽。若勁兵背城,坐制其困。」李神福亦請據險邀儒糧。行密乃分兵攻廣德,壁而絕饟道。軍適大疫,儒病
 \gezhu{
  疒占}
 ,遣建鋒、殷鈔諸縣。行密知城下兵寡,乃晨出,率仁義、頵背城決戰,破五十壁。會暴澍且冥,儒軍大敗。儒病甚,股弁不能興。頵執儒獻行密,諸將皆降。儒就刑於市,見劉威曰:「中君之謀。」儒嘗引鑒搔首曰:「此頭不久當入京師。」至是,傳首闕下。建鋒、殷哭之,相語曰:「公常有志廟食,吾等有土,當廟以報德。」及殷據湖南,表儒贈司徒、樂安郡王,立廟以祀。



\end{pinyinscope}