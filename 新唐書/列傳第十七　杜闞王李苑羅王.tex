\article{列傳第十七 杜闞王李苑羅王}

\begin{pinyinscope}

 杜伏威,齊州章丘人。少豪蕩,不治生貲,與里人輔公祏約刎頸交。公示石數盜姑家牧羊以饋伏威,縣跡捕急出,乃相與亡命為盜,時年十六。伏威狡譎多算,每剽劫,眾用其策皆效。嘗營護諸盜,出為導,入為殿,故其黨愛服,共推為主。



 隋大業九年,入長白山,依賊左君行,不得意,舍去,轉剽淮南,稱將軍。下邳賊苗海潮擁眾鈔暴,伏威遣公祏脅諭曰:「天下共苦隋,豪桀相與興義,然力弱勢分不相統,若合以為強,則無事隋矣。公能為主,吾且從,不然,一戰以決。」海潮懼,即以眾下之。江都留守遣校尉宋顥將兵捕擊,伏威與戰,偽北,誘顥墮葭榛澤中,順風縱火迫之,步騎燒死幾盡。海陵賊趙破陣聞伏威兵少,輕之,召使並力。伏威引親將十人操牛酒謁,勒公祏嚴兵待變。破陣引伏威入幕,置酒,悉召酋首高會。伏威突斬破陣,眾眙駭不及救,復殺數十人,下皆畏服,公祏兵亦至,遂並其眾,至數萬。攻安宜,屠之。隋遣虎牙郎將來整戰於黃花輪,伏威大敗,身重創,與公祏財有眾數百,亡去,行收卒得八千,與虎牙郎將公孫上哲戰鹽城,覆其軍。



 煬帝遣右御衛將軍陳棱以精兵討之,棱不敢戰,伏威遺以婦人服,書稱陳姥,怒其軍。棱果悉兵至,伏威迎出挑戰,棱軍射中其額,伏威怒曰:「不殺汝,矢不拔!」遂馳入棱陣,大呼沖擊,眾披靡,獲所射將,使拔箭已,斬之,攜其首入棱軍示之,又殺數十人,遂大潰,棱走而免。



 進破高郵,引兵度淮,攻歷陽,據之,稱總管。分兵徇屬縣,皆下,江淮群盜爭附。伏威選敢死士五千,號「上募」,寵厚之,與均甘苦,每攻取,必先登,戰罷,閱創在背者殺之。所虜獲必分與麾下,士有戰死,以其妻殉,故人自奮戰,無完敵。宇文化及以為歷陽太守,不受。徙丹陽,自稱大行臺。始進用士人,繕利兵械,薄賦斂,除殉葬法,民奸若盜及吏受賕,雖輕,皆殺無赦。上表越王侗,侗以為東南道大總管,封楚王。



 是時,秦王方討王世充,遣使招懷,伏威乃獻款。高祖授以東南道行臺尚書令、江淮安撫大使、上柱國、吳王,賜姓,豫屬籍,以其子德俊為山陽公,賜帛五千段,馬三百匹。伏威遣其將陳正通、徐紹宗以兵會,取世充之梁郡。又遣將王雄誕討李子通於杭州,禽以獻。破汪華於歙州。盡有江東、淮南地,南屬嶺,東至於海。秦王已平劉黑闥,師次曹、兗,伏威懼,乃入朝。詔拜太子太保兼行臺尚書令,留京師,位在齊王元吉上,以寵之。



 伏威好神仙長年術,餌雲母被毒,武德七年二月,暴卒。初,公祏反,矯伏威令以紿眾,趙郡王孝恭既平公祏,得反書以聞。高祖追其官,削屬籍,沒入家產。貞觀元年,太宗知其冤,詔復官爵,以公禮葬,仍還其子封。



 伏威有養子三十人,皆壯士,屬以兵,與同衣食,唯闞棱、王雄誕知名。



 闞棱,伏威邑人也。貌魁雄,善用兩刃刀,其長丈,名曰「陌刀」,一揮殺數人,前無堅對。伏威據江淮,以戰功顯,署左將軍。部兵皆群盜,橫相侵牟,棱案罪殺之,雖親故無脫者,至道不舉遺。從伏威入朝,拜左領軍將軍、越州都督。公祏反,棱與南討,青山之戰,與陳正通遇,陣方接,棱脫兜鍪謂眾曰:「不識我邪?何敢戰!」其徒多棱舊部,氣遂索,至有拜者。公祏破,棱功多,然頗自伐。公祏被禽,乃誣與己謀;又伏威、王雄誕及棱貲產在丹陽者當原,而趙郡王孝恭悉籍入之,棱自訴,忤孝恭。遂以謀反誅。



 王雄誕,曹州濟陰人。少強果,膂力絕人。伏威之起,用其計,戰多克,署驃騎將軍。



 初,伏威度淮與李子通合,後子通憚其才,襲之,伏威被創墮馬,雄誕負逃葭澤中,裒嘯散亡,又為隋將來整所窘,眾復潰。別將西門君儀妻王勇決而力,負伏威走,雄誕總麾下壯士十餘人從之。追兵至,雄誕還拒,數被創,氣彌厲,伏威遂脫。闞棱年長於雄誕,故軍中號棱「大將軍」,雄誕「小將軍」。



 後伏威令輔公祏擊子通,以雄誕、棱為副,戰溧水,子通敗,公祏乘勝追之,反為所擠,士皆走壁。雄誕曰:「子通狃於勝,無營壘,今急擊之,必克。」公祏不從。雄誕獨提私卒數百,銜枚夜往,乘風火之,子通大敗走,度太湖。武德四年,與子通戰蘇州,卻之。子通以精兵保獨松嶺,雄誕遣將陳當率千兵出不意,乘高蔽崦,張疑幟,夜縛炬於樹,遍山澤。子通懼,燒營遁,保餘杭,雄誕追禽之。



 歙守汪華在郡稱王且十年,雄誕還師攻之,華以勁甲出新安洞拒戰,雄誕伏兵山谷,以弱卒數千斗,輒走壁,華來攻,壁中奮殊死,不可下。會暮還,雄誕伏兵已據洞口,不得歸,遽面縛降。蘇賊聞人遂安據昆山,無所屬,伏威使討之,雄誕以邑險而完,攻之引日,遂單騎造壘門,陳國威靈,因開曉禍福,遂安即降。以前後功授歙州總管,封宜春郡公。



 伏威入朝,以兵屬雄誕。輔公祏將反,患其異己,縱反間,陽言得伏威教,責雄誕貳。雄誕素質直,信之,乃歸臥疾。公祏奪其兵,遣西門君儀諭計,雄誕始悔寤,曰:「天下方靖,王在京師,當謹守籓,奈何為族夷事?雄誕雖死,誼不從!」公祏遂縊之。



 雄誕愛人,善撫士,能致下死力,每破城邑,整眾山立,無絲毫犯。死之日,江南士庶為流涕。高祖嘉其節,以子世果襲宜春郡封。太宗立,優詔贈左驍衛大將軍、越州都督,謚曰忠。世果,垂拱初至廣州都督、安西大都護。



 張士貴,虢州盧氏人,本名忽峍。彎弓百五十斤,左右射無空發。隋大業末,起為盜,攻剽城邑,當時患之,號「忽峍賊」。高祖移檄招之,士貴即降,拜右光祿大夫。從征伐有功,賜爵新野縣公。又從平洛,授虢州刺史。帝曰:「顧令卿衣錦晝游耳。」進封虢國公、右屯衛大將軍。貞觀七年,為龔州道行軍總管,破反獠還,太宗聞其冒矢石先登,勞之曰:「嘗聞以忠報國者不顧身,於公見之。」累遷左領軍大將軍。顯慶初,卒,贈荊州都督,陪葬昭陵。



 李子和,同州蒲城人,本郭氏。為隋左翊衛,以罪徙榆林。大業末,郡饑,子和與死士十八人執丞王才,數以不恤下,斬之,開倉賑窮乏。自號永樂王,建元醜平,號其父為太公,以弟子政為尚書令,子端、子升為左右僕射,有騎兵二千。南連梁師都,北事突厥,納弟為質。始畢可汗冊子和為平楊天子,不敢當,乃更署為屋利設。武德元年獻款,授靈州總管、金河郡公,徙郕國公。襲師都寧朔城,克之。又伺突厥虛實,陰以章聞,為虜邏騎所獲,處羅可汗怒,囚子升,於是子和危畏,舉部南徙,詔內延州故城。五年,從平劉黑闥有功,賜姓,拜右武衛將軍。十一年,為婺州刺史,徙夷國公。顯慶初,轉黔州都督,乞骸骨,許之,進金紫光祿大夫,卒。



 苑君璋,馬邑豪也,以趫雄自奮。劉武周以兵入寇,君璋曰:「唐以一州兵掇取三輔,所向風靡,此殆天命,非人謀,不可爭也。太原而南多巖阻,今束甲深入,無踵軍,有失不可償,不如連突厥與唐合從,南面稱孤,上策也。」武周不聽,使君璋守朔州,引眾內侵,未幾敗,泣曰:「廢君言,乃至此!」即與共趨突厥。



 武周死,突厥以君璋為大行臺,統武周部曲,使鬱射設監兵,與舊將高滿政夜襲代州,不克。高祖遣使招之,賜鐵券,約不死。君璋拒命,進寇代州,刺史王孝德拒卻之。滿政勸君璋曰:「夷狄無禮,豈可北面臣之?請盡殺其眾以歸唐。」君璋不從。而馬邑困於兵,人厭亂,滿政因眾不忍,夜脅君璋,君璋奔突厥。滿政以城歸,詔拜朔州總管,封榮國公。君璋引突厥攻陷馬邑,殺滿政,夷其黨,乃去,退保恆安。其部皆中國人,多叛去,君璋窮,乃降,自請鄣虜贖罪。



 高祖遣雁門人元普賜金券,會頡利亦召之,意猶豫。子孝政諫曰:「大人許唐降,又貳頡利,自取亡也。今糧盡眾攜,不即決,恐衿肘變生,孝政不忍見禍之酷也!」即單騎南奔,君璋喻返之,召眾與議。恆安人郭子威曰:「恆安故王者都,山川足以自固,突厥方強,我援之,可觀天下變,何遽降?」君璋然之,執元普送突厥,頡利德之,遣以錦裘羊馬。其下怨,投書於門曰:「不早附唐,父子誅。」孝政懼,欲自歸,為君璋所拘。與突厥寇馬邑,犯太原,邊人苦之。見頡利政亂,知將亡,遂率所部降,頡利追,擊走其兵。



 入朝,拜安州都督,封芮國公,食五百戶,賜帛四千匹。君璋不曉書,然天資習事,歷職有惠稱。貞觀中,卒。



 羅藝,字子廷,襄州襄陽人,家京兆之雲陽。父榮,隋監門將軍。藝剛愎不仁,勇攻戰,善用槊。大業中,以戰力補虎賁郎將。遼東之役,李景以武衛大將軍督饟北平,詔藝以兵屬,分部嚴一。然任氣,嘗慢侮景,頻為景辱。


天下盜起,涿郡號富饒,伐遼兵仗多在,而倉
 \gezhu{
  廣寺}
 盈羨,又臨朔宮多珍寶,屯師且數萬,苦盜賊侵掠,留守將趙什住、賀蘭誼、晉文衍等不能支。藝捍寇,數破卻之,勇常冠軍,為諸將忌畏。藝陰自計,因出師,詭說眾曰:「吾軍討賊數有功,而食乏。官粟若山,而留守不賑恤,豈安人強眾意邪?」士皆怨。既還,郡丞出郊謁,藝執之,陳兵入,什住等懼,爭聽命。藝即發庫貲賜戰士,倉粟給窮人,境內大悅。殺異己者渤海太守唐禕等,威動北邊,柳城、懷遠並歸附。黜柳城太守楊林甫,改郡曰營州,以襄平太守鄧皓為總管,藝自稱幽州總管。



 宇文化及至山東,遣使招藝,藝曰:「我隋舊臣,今大行顛覆,義不辱於賊。」斬使者,為煬帝發喪三日。時竇建德、高開道亦遣使於藝,藝謂官屬曰:「建德等皆劇賊,不足共功名,唐公起兵據關中,民望所系,王業必成,吾決歸之。敢異議者戮!」會張道源撫輯山東,亦諭藝降,武德二年,乃奉表以地歸。詔封燕王,賜姓,豫屬籍。數與建德戰,多所禽馘。秦王擊劉黑闥,高祖詔藝弟監門將軍壽以兵從,藝自率眾數萬破劉什善、張君立於徐河。黑闥引突厥入寇,藝復以兵與皇太子建成會洺州,遂請入朝。帝厚禮之,拜左翊衛大將軍。


藝負其功,且貴重不少屈,秦王左右嘗至其營,藝
 \gezhu{
  疒只}
 辱之。高祖怒,以屬吏,久乃釋。時突厥放橫,藉藝威名欲憚虜,詔以本官領天節軍將,鎮涇州。



 太宗即位,進開府儀同三司。藝內懼,乃圖反,詭言閱武。兵既集,稱被密詔入朝,軍至豳,治中趙慈皓出謁,遂據州。帝命長孫無忌、尉遲敬德擊之,未至,慈皓與統軍楊岌謀誅藝,藝覺,執慈皓。岌居外,即攻之,藝敗,棄妻子,從數百騎奔突厥。抵寧州,騎稍亡,左右斬之,傳首,梟於都市。壽時為利州都督,亦及誅。



 先是,濟陰女子李,自言通鬼道,能愈疾,四方惑之,詔取致京師。嘗往來藝家,謂藝妻孟曰:「妃相貴,當母天下。」孟令視藝,又曰:「妃之貴由於王,貴色且發。」藝妻信之,亦贊以反,既敗,與李皆斬。



 王君廓,並州石艾人。少孤貧,為駔儈,無行,善盜。嘗負竹笱如魚具,內置逆刺,見鬻繒者,以笱囊其頭,不可脫,乃奪繒去,而主不辨也,鄉里患之。



 大業末,欲聚兵為盜,請與叔俱,不從,乃誣鄰人通叔母者,與叔共殺之,遂皆亡命。眾稍集,掠夏、長平。河東丞丁榮拒之,且遣使慰召。君廓見使,謬為欲歸首者。榮輕之,因陳兵登山,君廓悉伏甲山谷中。榮軍還,掩擊,破之。又與賊韋寶、鄧豹等掠虞鄉,宋老生與戰,君廓不利,保方山,老生列營迫之。君郭糧盡,詐請降,與老生隔澗語,祈請哀到。老生為感動,稍緩之,君廓一昔遁去。



 高祖兵起,召之,不從。歸李密,密不甚禮,乃歸國。授上柱國、假河內太守、常山郡公,遷遼州刺史,徙封上谷,從戰東都有功,為右武衛將軍。詔勞之曰:「爾以十三人破賊萬,自古以少制眾,無有也!」賜雜彩百段。別下軒轅、羅川二縣,破世充將魏隱,擊糧道緱氏,沈米艘三十柁。



 進爵彭國公,鎮幽州。擊突厥,俘斬二千,獲馬五千匹。入朝,帝賜所乘馬,令自廷中乘以出,謂侍臣曰:「昔藺相如叱秦王,目眥皆烈。君廓往擊建德,李勣遏之,至發憤大呼,鼻耳皆流血,其勇何特古人哉!朕當不以例賞。」乃賜錦袍金帶,還幽州。



 會大都督廬江王瑗反,欲奪君廓兵以委王詵。君廓本紿瑗使亂為己功,乃從數騎候詵,留騎於外,曰:「聞呼聲則入。」乃獨款詵,詐曰:「有急變,當白!」詵方沐,握發出,即斬之,因執瑗。以功授幽州都督,瑗家口悉賜之,進左光祿大夫,賜帛千段。



 居職不守法度,長史李玄道數以法繩督,猜惑不自安。會被召,至渭南,殺驛史,亡奔突厥,野人斬之。太宗顧前功,為收葬,待其家如初。御史大夫溫彥博奏:「君廓叛臣,不宜食封邑,有司失所宜言。」乃貶為庶人。



\end{pinyinscope}