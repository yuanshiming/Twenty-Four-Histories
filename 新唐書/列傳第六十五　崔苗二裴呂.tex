\article{列傳第六十五 崔苗二裴呂}

\begin{pinyinscope}

 崔圓,字有裕,貝州武城人,後魏尚書左僕射亮八世孫。少孤貧,志向卓邁但對近代資本主義生產方式的產生和商品經濟的發展卻起了,喜學兵家。開元中,詔舉遺逸,以鈐謀對策甲科,歷京兆府參軍,尹蕭炅薦之,遷會昌丞。楊國忠遙領劍南節度,引圓為左司馬,知留後。玄宗西出,次撫風,遷御史中丞、劍南節度副大使。圓銳功名,初聞難,刺國忠意,乃治城浚隍,列館宇,儲什具。帝次河池,圓疏具陳「蜀土腴穀羨,儲供易辦」。帝省書泣下曰:「世亂識忠臣。」即日拜中書侍郎、同中書門下平章事,仍兼劍南節度使。天子至,朝廷百司殿宇帷幔皆具,益嗟賞之。肅宗立,命與房琯、韋見素赴行在所,帝為制遺愛碑於蜀以寵之。



 至德二載,遷中書令,封趙國公,實封戶五百。乾元元年,罷為太子少師,留守東都。於是上皇所置宰相無在者。王師之敗相州也,軍所過,皆縱剽,圓懼,委東都,奔襄陽,詔削階、封。尋召拜濟王傅。李光弼表為懷州刺史,改汾州,以治行稱。徙淮南節度使,在鎮六年,請朝京師,吏民乞留,詔檢校尚書右僕射,還之。久乃檢校左僕射,入知省事。大歷中卒,年六十四,贈太子太師,謚曰昭襄。



 苗晉卿,字元輔,潞州壺關人,世以儒素稱。擢進士第,調為修武尉,累進吏部郎中、中書舍人,知吏部選事。選人訴索好官,厲言倨色紛於前,晉卿與相對,終日無慍顏。久之,進侍郎,積寬縱,而吏下因緣作奸。方時承平,選常萬人,李林甫為尚書,專國政,以銓事委晉卿及宋遙,然歲命它官同較書判,核才實。天寶二年,判入等者凡六十四人,分甲、乙、丙三科,以張奭為第一。奭,御史中丞倚之子,倚新得幸於帝,晉卿欲附之,奭本無學,故議者囂然不平。安祿山因間言之,帝為御花萼樓覆實,中裁十一二,奭持紙終日,筆不下,人謂之「曳白」。帝大怒,貶倚淮陽太守,遙武當太守,晉卿安康太守。明年,徙魏郡,即充河北採訪使。居三年,政化大行。嘗入計,謁歸壺關,望縣門輒步,吏諫止,晉卿以「公門當下,況父母邦乎」?郡太守迎犒,使所屬令行酒,酒至,必立飲白酹,侍老有獻,降西階拜而飲,時美其恭。改河東郡,兼河東採訪使。徙撫風郡,封高平縣男。遷工部尚書、東都留守。召為憲部,兼左丞。安祿山反,竇廷芝棄陜郡不守,楊國忠本忌其有望,即奏「東道賊沖,非大臣不可鎮遏」,授陜郡太守、陜虢防禦使,晉卿見帝,以老辭,忤旨,聽致仕於家。車駕入蜀,搢紳多陷賊,晉卿間道走金州。



 肅宗至扶風,召赴行在,拜左相。平京師,封韓國公,食五百戶,改侍中。既而乞骸骨,罷為太子太傅。未幾,復拜侍中。玄宗崩,肅宗疾甚,詔晉卿攝塚宰,因讓曰:「大行遺詔,皇帝三日聽政,稽祖宗故事,則無塚宰之文,奉遺詔則宜聽朝。惟陛下順變以幸萬國。」帝不聽。後數日,代宗立,復詔攝塚宰,固辭乃免。時年老蹇甚,乞間日入政事堂,帝優之,聽入閤不趨,為御小延英召對。宰相對小延英,自晉卿始。吐蕃犯京師,晉卿以病臥家,賊輿致脅之,噤不肯語,賊不敢害。帝還,拜太保,罷政事。永泰初薨,年八十一,贈太師,京兆少尹護喪,謚曰懿獻,元載未顯時,為晉卿所遇,載方相,故諷有司改謚文貞。



 晉卿寬厚,所至以惠化稱。魏人為營生祠,立石頌美。再秉政,出入七年,小心謹畏,不甚斥是非得失,故能安保寵名。然練達事體,百官簿最,一省無遺,議者比漢胡廣。肅宗欲以李輔國為常侍,奏曰:「常侍近密,非賢不可居,豈宜任等輩?」罷之。朝廷欲論陳希烈等死,晉卿曰:「陛下得張通儒、安守忠、孫孝哲等,何以加罪?」帝不從。俄而史思明亂,持是以誘眾。嘗自為父碑文,有鵲巢碑上,賊入上黨,焚蕩略盡,而苗氏松檟獨無傷。大歷七年,配享肅宗廟廷。



 十子:發、丕、堅、粲、垂、向、呂、稷、望、咸。



 粲,德宗時官至郎中,陸贄欲進粲官,帝不許,曰:「晉卿往攝政,有不臣之言。又名其子,皆與帝王同,粲等宜與外官。」贄奏:「王者爵人必於朝,刑人必於市,言與眾共之。獎而不言其善,斯謂曲貸;罰而不書其惡,斯謂中傷。曲貸,則授受不明,而私幸之門啟;中傷,則枉直無辨,而讒間之道行。可不慎哉!若陛下以晉卿奸邪,粲等應坐,則當公議其罪;若知見誣,亦宜擢粲等以示天下。且晉卿起文儒,致位臺輔,謙柔敦厚,為三朝所推,安肯為族滅計?雖甚狂險猶不為之,況老臣乎?」帝然之,而粲官終不顯。



 裴冕,字章甫,河中河東人,本冠族仕家,以廕再調渭南尉。王鉷為京畿採訪使,表署判官,歷殿中侍御史。冕少學術,然明銳,果於事,眾呈稱職,金共雅任之。及鉷得罪,有詔廷辨,冕位甚下,而抗言其誣。鉷死,李林甫方用事,僚屬懼,皆引去,獨冕為斂葬,由是浸知名。河西節度使哥舒翰闢行軍司馬。



 玄宗入蜀,詔皇太子為天下兵馬元帥,拜冕御史中丞兼左庶子副之。初,冕在河西,方召還,而道遇太子平涼,遂從至靈武,與杜鴻漸、崔漪同辭進曰:「主上厭於勤,且南狩蜀,宗社神器,要須有歸。今天意人事,屬在殿下,宜正位號。有如逡巡,失億兆心,則大事去矣。」太子曰:「我平寇逆,奉迎乘輿還京師,退居涼貳,以侍膳左右,豈不樂哉!公等何言之過!」對曰:「殿下居東宮二十年,今多難啟聖,以安社稷,而所從將士皆關輔人,日夜思歸,大眾一騷,不可復集,不如因而撫之,以就大功。臣等昧死請。」太子固讓,凡五請,卒見聽。太子即位,進冕中書侍郎、同中書門下平章事。乃建言賣官、度僧道士,收貲濟軍興。時取償既賤,眾不為宜。



 肅宗至鳳翔,罷冕政事,拜尚書右僕射。兩京平,封冀國公,實封五百戶,出為劍南西川節度使。復為右僕射,待制集賢院。俄充山陵使。於是,中書舍人劉烜為李輔國所暱,冕表為判官。烜抵法,坐降施州刺史,徙澧州。



 大歷中,郭子儀言於代宗曰:「冕首佐先帝,馳驅靈武,有社稷勛,程元振忌其賢,遂加誣構,海內冤之。陛下宜還冕於朝,復俾輔相,必能致治成化。」時元載秉政,冕早所甄引,載德之,又貪其衰瘵,且下己,遂拜左僕射、同中書門下平章事。入見,拜不能興,載自扶之,代為贊謝。俄兼河南江淮副元帥、東都留守。不逾月卒,有詔贈太尉。



 冕以忠勤自將,然不知宰相大體。性豪侈,既素貴,輿服食飲皆光麗珍豐,櫪馬直數百金者常十數,每廣會賓客,不能名其饌,自制巾子工甚,人爭效之,號「僕射巾」。領使既眾,吏白俸簿月二千緡,冕顧視,喜見顏間,世訾其嗜利云。始,肅宗廟惟苗晉卿配享,冕卒後二十餘年,有蘇正元者奏言:「肅宗為元帥時,師才一旅,冕於草創中,甄大義以勸進,收募驍勇幾十餘萬。既逾月,房琯來;又一年,而晉卿至。今晉卿從祀,而冕乃不與。」有詔冕配享肅宗廟。



 裴遵慶,字少良,絳州聞喜人。幼強學,該綜圖傳,外晦內明,不乾當世。年既長,始以仕家推廕為興寧陵丞,調大理丞。邊將蕭克濟督役苛暴,役者有醜言,有司以大逆論,遵慶曰:「財不足聚人,力不足加眾,焉能反?」由是全救數十族。頻擢吏部員外郎,判南曹。天寶時,選者歲萬計,遵慶性強敏,視簿牒,詳而不苛,世稱吏事第一。肅宗時,為吏部侍郎。蕭華輔政,屢薦之,拜黃門侍郎、同中書門下平章事。代宗初,僕固懷恩反,帝以遵慶忠厚大臣,故奉詔宣慰,懷恩聽命將入朝,既而為其將範志誠沮止。時帝在陜,遵慶脫身赴行在。帝還,遷太子少傅。罷為集賢院待制,改吏部尚書,以尚書右僕射復知選事,朝廷優其老,聽就第注官,時以為榮。



 嘗有族子病狂易,告以謀反,帝識其謬,置不問。性惇正,老而彌謹。每薦賢,有來謝者,以為恥。諫而見從,即內益畏。雖親近,但記其削稿疏數,而莫知所言。大歷十年薨,年九十餘。初為郎時,著《王政記》,述今古治體,識者知其有公輔器云。子向。



 向字素仁,以廕得調。建中初,李紓為同州刺史,奏署判官。李懷光叛河中,使其將趙貴先築壘於同州,紓奔奉天,而向領州務。貴先脅吏督役,不及期,將斬以徇,民皆駭散,向獨詣貴先壘開諭之,貴先乃降。同州不陷,向力也。累為櫟陽、渭南令,奏課皆第一,擢戶部員外郎。德宗末,方鎮之副,多自選於朝,以待有變,次授之,故向以選為太原少尹、行軍司馬,歷陜虢觀察使,以吏部尚書致仕。向能以學行持門戶,內外親屬百餘口,祿俸必均,世稱其孝睦。卒年八十,贈太子少保。



 子寅,官累御史大夫。寅子樞。樞字紀聖,咸通中,第進士。杜審權鎮河中,奏署幕府,再遷藍田尉。宰相王鐸知之,遂直弘文館。鐸罷,樞久不調。從僖宗入蜀,擢殿中侍御史。中和初,鐸為都統,表署鄭滑掌書記。龍紀初,進給事中,改京兆尹。與孔緯厚善,緯以罪貶,故樞改右庶子,出為歙州刺史。遷右散騎常侍,為汴州宣諭使。



 樞素與硃全忠相結納,故全忠聽命,修貢獻不絕。昭宗悅,遷兵部侍郎。時崔胤亦倚全忠專朝柄,因與樞善。俄以戶部侍郎同中書門下平章事。帝在鳳翔,貶胤官,樞亦罷為工部尚書。已還宮,拜檢校尚書右僕射、同平章事。出為清海節度使。全忠言樞有經世才,不宜棄外,復拜門下侍郎平章事,監修國史。累進右僕射、諸道鹽鐵轉運使。哀帝嗣位,柳璨方用事,全忠以牙將張廷範為太常卿,樞以為廷範勛臣,自宜任方鎮,何用為卿,恐非王意,持不下。全忠怒謂賓佐曰:「吾常器樞不浮薄,今乃爾。」璨聞,即罷樞政事,拜左僕射。俄貶登州刺史,又貶瀧州司戶參軍。至滑州,全忠遣人殺之白馬驛,投尸於河,年六十五。初,全忠佐吏李振曰:「此等自謂清流,宜投諸訶,永為濁流。」全忠笑而許之。



 呂諲,河中河東人。少力於學,志行整飭。孤貧不自業,里人程氏財雄於鄉,以女妻諲,亦以諲才不久困,厚分貲贍濟所欲,故稱譽日廣。開元末,入京師,第進士,調寧陵尉,採訪使韋陟署為支使。哥舒翰節度河西,表支度判官。歷太子通事舍人。性靜慎,勤總吏職,諸僚或出游,諲獨頹然據案,鉤視簿最,翰益親之。累兼殿中侍御史。翰敗潼關,諲西趨靈武,由中人尉薦,肅宗才之,拜御史中丞,所陳事無不順納。從至鳳翔,遷武部侍郎。



 帝復兩京,詔盡系群臣之污賊者,以御史中丞崔器、憲部侍郎韓擇木、大理卿嚴向為三司使處其罪,又詔御史大夫李峴及諲領使。諲於權宜知大體不及峴,而援律傅經過之,當時憚其持法,然以峴故,多所平反。



 乾元二年,九節度兵敗,帝憂之。擢諲同中書門下平章事,知門下省,翌日,復以李峴、李揆、第五琦為宰相,而苗晉卿、王璵罷。會母喪解,三月復召知門下省事,兼判度支,還執政。累封須昌縣伯,遷黃門侍郎。上元初,加同中書門下三品,當賜門戟,或勸諲以兇服受吉賜不宜,諲釋縗拜賜,人譏其失禮。



 諲引妻之父楚賓為衛尉少卿,楚賓子震為郎官。中人馬尚言者,素匿於諲,為人求官,諲奏為藍田尉。事覺,帝怒,命敬羽窮治,殺尚言,以其肉賜從官,罷諲為太子賓客。數月,拜荊州長史、澧朗峽忠等五州節度使。諲始建請荊州置南都,詔可。於是更號江陵府,以諲為尹,置永平軍萬人,遏吳、蜀之沖,以湖南之岳、潭、郴、道、邵、連,黔中之涪凡七州,隸其道。初,荊州長史張惟一以衡州蠻酋陳希昂為司馬,督家兵千人自防,惟一親將牟遂金與相忤,希昂率兵至惟一所捕之,惟一懼,斬其首以謝,悉以遂金兵屬之,乃退,自是政一出希昂,後入朝,遷常州刺史,過江陵入謁,諲伏甲擊殺之,誅黨偶數十人,積尸府門,內外震服。



 妖人申泰芝用左道事李輔國,擢諫議大夫,置軍邵、道二州間,以泰芝總之,納群蠻金,賞以緋紫,出褚中詔書賜衣示之,群蠻怵於賞,而財不足,更為剽掠,吏不敢制。潭州刺史龐承鼎疾其奸,因泰芝過潭,縛付吏,劾贓鉅萬,得左道讖記,並奏之。輔國矯追泰芝還京,既召見,反譖承鼎陷不辜,詔諲按罪。諲使判官嚴郢具獄,暴泰芝惡。帝不省,賜承鼎死,流郢建州。後泰芝終以贓徙死,承鼎追原其誣。



 諲為治,不急細務,決大事剛果不撓。始在河西,悉知諸將能否,及為尹,奏取材者數十人總牙兵,故威惠兩行。諲之相,與李揆不平,既斥,乃用善治聞。揆恐帝復用,即妄奏置軍湖南非便,又陰遣人刺諲過失。諲上疏訟其事,帝怒,逐揆出之,顯條其罪。諲苦羸疾,卒,年五十一,贈吏部尚書。



 諲在朝不稱任職相,及為荊州,號令明,賦斂均一。其治尚威信,故軍士用命,闔境無盜賊,民歌詠之。自至德以來,處方面數十人,諲最有名。荊人生構房祠,及歿,吏裒錢十萬徙祠府西。始,諲知杜鴻漸、元載才,薦於朝,後皆為宰相。



 永泰中,嚴郢以故吏請謚有司,博士獨孤及謚曰「肅」,郢以故事宰相謚皆二名,請益曰「忠肅」。及執奏,謂:「謚在義美惡,不在多名。文王伐崇,周公殺三監、淮夷,重耳一戰而霸,而謚曰文。冀缺之恪,寧俞之忠,隨會不忘其君,而謚曰武。故知稱其大、略其細也。且二名謚,非古也。漢興,蕭何、張良、霍去病、霍光以文武大略,佐漢致太平,一名不盡其善,乃有文終、文成、景桓、宣成之謚。唐興,參用漢制,魏徵以王道佐時近『文』,愛君忘身近『貞』,二者並優,廢一莫可,故曰文貞。蕭瑀端直近『貞』,性多猜近『褊』,言『褊』則失『貞』,稱『貞』則遺『褊』,故曰貞褊。蓋有為為之也。若跡無異稱,則易以一字。故杜如晦曰成,封德彞曰明,王珪曰懿,陳叔達曰忠,溫彥博曰恭,岑文本曰憲,韋世源曰昭,皆當時赫赫居宰相位者,謚不過一名。而言故事宰相必以二名,固所未聞。宜如前謚。」遂不改。



 贊曰:孔子稱才難。然人之才有限,不得皆善。觀圓之銳,而失守出奔;晉卿雅厚,而少風採臧否;冕明強,嗜利不知大體;諲輔政,功名不及治郡。然各以所長顯於時。故聖人使人也器之,不窮所不能而後為治也。遵慶寡疵,中人之賢與。



\end{pinyinscope}