\article{列傳第八十九 歸奚三崔盧二薛衛胡丁二王殷}

\begin{pinyinscope}

 歸崇敬,字正禮,蘇州吳人。治禮家學,多識容典,擢明經。遭父喪不少合理的思想。如提出關於物質和運動的統一、化學元素,孝聞鄉里。調國子直講。天寶中,舉博通墳典科,對策第一,遷四門博士。有詔舉才可宰百里者,復策高等,授左拾遺。肅宗次靈武,再遷起居郎、贊善大夫、史館修撰、兼集賢殿校理,修國史、儀注。以貧求解。歷同州長史、潤州別駕。未幾,有事橋陵、建陵,召還參掌儀典。改主客員外郎,復兼修撰。



 代宗幸陜,召問得失,崇敬極陳:「生人疲敝,當以儉化天下,則國富而兵可用。」時百官朝朔望,皆服褲褶,崇敬非之,建言:「三代逮漢無其制,隋以來,始有服者,事不稽古,宜停。」詔可。又言:「東都太廟不當置木主,按《禮》:『虞主用桑,練主用慄』,作慄主則瘞桑主,猶天無二日,土無二王也。東都太廟,本武后所建,以祀諸武,中宗去主存廟,以備行幸遷都之置。且商遷都前八後五,不必每都別立神主也。若曰神主已經奉祀,不得一日而廢,則桑主以虞,至練祭而埋之,明是不然。」時有方士巨彭祖建言:「唐家土德,請以四季月郊祀天地。」詔禮官儒者雜議。崇敬議:「《禮》以先立秋十八日迎黃靈,祀黃帝,黃帝於五行為土,而火為母,故火用事之末而祭之,三季月則否。彭祖牽緯候說,事詭不經,不可用。」又議:「五人帝於國家為前後,無君臣義,天子祭宜毋稱臣,祭而稱臣,於天帝無異。」又:「春秋釋奠孔子,祝版皇帝署,北面揖,以為太重。宜準武王受丹書於師尚父,行東面之禮。」事皆施行。



 大歷初,授倉部郎中,充吊祭冊立新羅使。海道風濤,舟幾壞,眾驚,謀以單舸載而免,答曰:「今共舟數十百人,我何忍獨濟哉?」少選,風息。先是,使外國多齎金帛,貿舉所無,崇敬囊橐惟衾衣,東夷傳其清德。還,授國子司業、兼集賢學士。八年,遣祀衡山,未至,而哥舒晃亂廣州,監察御史憚之,請望祀而還,崇敬正色曰:「君命豈有畏邪?」遂往。



 皇太子欲臨國學行齒胄禮,崇敬以學與官名皆不正,乃建議:



 古天子學曰闢雍。以制言之,壅水環繚如璧然;以誼言之,以禮樂明和天下云爾。在《禮》為澤宮,故前世或曰璧池,或曰璧沼,亦言學省。漢光武立明堂、闢雍、靈臺,號「三雍宮」。晉武帝臨闢雍,行鄉飲酒禮,別立國子學,以殊士庶。永嘉南遷,唯有國子學。隋大業中,更名國子監。今聲明之盛,闢雍獨闕,請以國子監為闢雍省。祭酒、司業之名,非學官所宜。業者,栒虡大版,今學不教樂,於義無當。請以祭酒為太師氏,位三品;司業為左師、右師,位四品。



 近世明經,不課其義,先取帖經,顓門廢業,傳受義絕。請以《禮記》、《左氏春秋》為大經,《周官》、《儀禮》、《毛詩》為中經,《尚書》、《周易》為小經,各置博士一員。《公羊》、《穀梁春秋》共準一中經,通置博士一員。博士兼通《孝經》、《論語》,依章疏講解。德行純絜、文詞雅正、形容莊重可為師表者,委四品以上各舉所知,在外給傳,七十者安車蒲輪敦遣。國子、太學、四門三館,各立五經博士,品秩、生徒有差。舊博士、助教、直講、經直、律館、算館助教,請皆罷。



 教授法。學生謁師,贄用暇脩一束、酒一壺、衫布一裁,色如師所服。師出中門,延入與坐,割脩酒,三爵止。乃發篋出經,摳衣前請,師為說經大略,然後就室,朝晡請益。師二時堂上訓授道義,示以文行忠信、孝悌睦友。旬省、月試、時考、歲貢,眡生徒及第多少為博士考課上下。有不率教者,檟楚之,國子移禮部,為太學生;太學又不變,徙之四門;四門不變,徙本州之學;復不變,繇役如初,終身不齒。雖率教,九年學不成者,亦歸之本州。



 禮部考試法。請罷帖經,於所習經問大義二十而得十八,《論語》、《孝經》十得八,為通;策三道,以本經對,通二為及第。其孝行聞鄉里者,舉解具言,試日義闕一二,許兼收焉。天下鄉貢如之。習業考試,並以明經為名,得第授官,與進士同。



 有詔尚書省集百官議。皆以習俗久,制度難分明,省禁非外司所宜名,《周官》世職者稱氏,國學非世官,不得名闢雍省、太師氏。大抵憚改作,故無施行者。



 坐史給稟錢不實,貶饒州司馬。德宗立,召還,復拜國子司業,稍遷翰林學士、左散騎常侍,充皇太子侍讀,又兼晉王元帥參謀,封餘姚郡公。田悅、李納稟命,持節宣慰,稱旨。表歸上塚,寵賜繒帛,儒生以為榮。遷工部尚書,仍前職。年老,以兵部尚書致仕。卒,年八十八,贈尚書左僕射,謚曰宣。論撰數十篇。子登。



 登,字沖之,事繼母篤孝。大歷中,舉孝廉高第。貞元初,策賢良,為右拾遺。裴延齡得幸,德宗欲遂以相,右補闕熊執易疏論之,以示登,登動容曰:「願竄吾名,雷霆之下,君難獨處。」故同列有所諫正,輒聯署無所回諱。轉右補闕、起居舍人,凡十五年,僚類有出其下而進趨,自喜得顯官,惟登與右拾遺蔣武退然遠權勢,終不以淹晚概懷。遷兵部員外郎。



 順宗為皇太子,登父子侍讀,及即位,以東宮恩超拜給事中,遷工部侍郎,復為皇太子、諸王侍讀,獻《龍樓箴》以諷。徙左散騎常侍,入謝。憲宗問政所先,登知帝睿而果於斷,勸順納諫爭,內外傳為讜言。後判國子祭酒事,進工部尚書,累封長洲縣男。卒,年六十七,贈太子少師,謚曰憲。



 登性溫恕,家僮為馬所𧾷是,笞折馬足,登知,不加責。有遺金石不死藥者,紿曰已嘗,及登服幾死,訊之,乃未之嘗,人皆為怒,而登不為慍。常慕陸象先為人,世亦許其類云。子融。



 融,字章之,元和中,及進士第,累遷左拾遺。事文宗為翰林學士,進至戶部侍郎。開成初,拜御史中丞。湖南觀察使盧周仁以南方屢火,取羨錢億萬進京師。融劾奏:「天下一家,中外之財皆陛下府庫。周仁陳小利,假異端,公違詔書,徇私希恩。恐海內效之,因緣漁刻,生人受弊,罪始周仁。請重責,還所進,代貧民租入。」詔不從,置錢河陰院以虞水旱。初,戶部員外郎盧元中、左司員外郎判戶部案姚康受平糴官秦季元絹六千匹,貸乾沒錢八千萬,俱貶嶺南尉。數年,金部員外郎韓益判度支,子弟受賕三百萬,未入者半。帝問融:「益所犯與盧元中、姚康孰甚?」對曰:「元中等枉失庫錢,益所坐子弟受賄,事異法輕。」故益止貶梧州參軍。融遷京兆尹,李固言為相,惡之,徙秘書監。固言罷,擢權知兵部侍郎。歲間,出為山南西道節度使,徙東川。還,歷兵部尚書,累封晉陵郡公。



 會昌後,儒臣少,朝廷禮典多本融議。辭疾,以太子少傅分司東都。大中七年,卒,贈尚書左僕射。



 奚陟,字殷卿,其先自譙亳西徙,故為京兆人。少篤志,通群書。大歷末,擢進士、文辭清麗科,授弘文館校書郎。德宗立,諫議大夫崔河圖持節使吐蕃,表陟自副,以親老辭不拜。楊炎輔政,召授左拾遺。居親喪,毀瘠過禮。硃泚反,走間道及車駕於興元,拜起居郎、翰林學士,不就職。賊平,改太子司議郎,歷金部、吏部員外。會左右丞缺,轉左司郎中。



 貞元八年,遷中書舍人。於是江南、淮西皆大水,詔陟勞問循慰,所至人人便安。中書吏倚宰相勢,常姑息,獨陟遇之無假借。先是,右省雜給眡職田稟,主事與拾遺等,陟以奉稍為率,由是吏官有差。中書令李晟有紙筆猥料積於省,它日以遺舍人,而雜事舍人常私有之,陟均舍寮無厚薄。雖細務,皆身親其勞,久益強力,人以為難。



 遷刑部侍郎。京兆尹李充有美政,裴延齡惡之,誣劾充比陸贄,數遺金帛,當抵罪,又乾沒京兆錢六十八萬緡,請付比部鉤校。時郎中崔元翰怨贄,揣延齡指,逮系搒掠甚急,內以險文。陟持平無所上下,具獄上,且言:「京兆錢給縣館傳,餘以度支符用度略盡。」充既免,元翰不得意,以恚死。



 陟尋知吏部選事,遷侍郎。銓綜平允,時謂與李朝隱略等,不能擿發清明如裴行儉、盧從願也。十五年,病癰,帝遣醫療視,敕曰:「陟,賢臣,為我善治之。」卒,年五十五,贈禮部尚書。



 陟少自底厲,著名節。常薦權德輿為起居舍人知制誥,楊於陵為郎中,其後皆有名。



 子敬玄,位左補闕。



 崔衍,字著,深州安平人。父倫,字敘,居父喪,跣護柩行千里,道路為流涕,廬塚彌年。服除,及進士第,歷吏部員外郎。安祿山反,陷於賊,不污偽官,使子弟間表賊事。賊平,下遷晉州長史。李齊物訟其忠,授長安令,封武邑縣男。寶應二年,以右庶子使吐蕃,虜背約,留二歲,執倫至涇州,逼為書約城中降,倫不從,更囚邏娑城,閱六歲,終不屈,乃許還。代宗見之,為感動嗚咽。即具陳虜情偽、山川險易,指畫帝前,人服其詳。遷尚書左丞,以疾改太子賓客。卒,年七十一,贈工部尚書,謚曰敬。



 衍,天寶末擢明經,調富平尉。繼母李不慈,倫自吐蕃歸,李敝衣以見,問故,曰:「衍不吾給。」倫怒,召衍,將袒而鞭之,衍涕泣無所陳。倫弟殷趨白:「衍所稟舉送夫人所,尚何雲!」倫悟,繇是譖無入。調清源令,勸民力田,懷附流亡,觀察使馬燧表其能,徙美原。父卒,事李益謹,歲為李子郃償負不勝計,故官刺史,妻子僅免饑寒。



 歷蘇、虢二州。虢居陜、華間,而賦數倍入,衍白太重。裴延齡領度支,方聚斂,私謂衍:「前刺史無發明,公當止。」衍不聽,復奏:「州部多巖田,又郵傳劇道,屬歲無秋,民舉流亡,不蠲減租額,人無生理。臣見長吏之患,在因循不以聞。不患陛下不憂恤也,患申請不實,不患朝廷不矜貸也。陛下拔臣大州,寧欲視民困而顧望不言哉?」德宗公其言,為詔度支減賦。遷宣歙池觀察使,簡靜為百姓所懷。幕府奏聘皆有名士,後多顯於時。卒,年六十九,贈工部尚書。衍儉約畏法,室無妾媵,祿稍周於親族,葬埋嫁娶,倚以濟者數十家。及卒,不能蕆喪,表諸朝,賜賻帛三百段,米粟稱之。



 先是,天下以進奉結主恩,州藏耗竭,韋皋、劉贊、裴肅為之倡。贊死,衍代之。舊貢金錫凡十八品,皆倍直市於州,民匱,多逃去,衍至,蠲革之。居十年,嗇用度,府庫充衍。及穆贊代州,以錢四十萬緡假民賦,故雖旱,人不流捐,由衍蓄積有素也。路應為觀察使,以衍有惠在民,言狀。元和元年,詔書褒美,賜一子官云。謚曰懿。



 盧景亮,字長晦,幽州範陽人。少孤,學無不覽。第進士、宏辭,授秘書郎。張延賞節度荊南,表為枝江尉,掌書記。入遷右補闕。硃泚反,景亮勸德宗曰:「陛下罪己不至,則感人不深。」帝然之。景亮志義崒然,多激發,與穆質同在諫爭地,書數上,鯁毅無所回。宰相李泌劾景亮等嘗眾會,漏所上語言,引善在己,即有惡歸之君。帝怒,貶為朗州司馬,質亦斥去,廢抑二十年。至憲宗時,由和州別駕召還,再遷中書舍人。



 景亮善屬文,根於忠仁,有經國志。嘗謂:「人君足食足兵而又得士,天下可為也。」乃興軒、頊以來至唐,叕刂治道之要,著書上下篇,號《三足記》。又作《答問》,言輓運大較及陳西戎利害,切指當世。公卿伏其達古今云。元和初卒,贈禮部侍郎。憲宗時,以直諫知名者,又有王源中。源中,字正蒙。擢進士、宏辭,累遷左補闕。是時,中官領禁兵,數亂法,捕臺府吏屬系軍中。源中上言:「臺憲者,紀綱地;府縣,責成之所。設吏有罪,宜歸有司,無令北軍亂南衙,麾下重於仗內。」帝納之。累轉戶部郎中、侍郎,擢翰林學士,進承旨學士。



 源中嗜酒,帝召之,醉不能見。及寤,憂其慢,不悔不得進也。他日,又如之,遂失帝意。以疾自言,出為山南西道節度使,入拜刑部侍郎。未幾,領天平節度使。開成三年卒,贈尚書右僕射。



 源中澹名利,率身治人,約而簡,當時咨美。



 薛蘋,河中寶鼎人。七世祖道實,為隋禮部尚書。父順為奉天尉,與楊國忠有舊,及用事,將引之,輒謝絕。



 蘋以吏最拜長安令,歷虢州刺史。憲宗時,奏最,擢湖南觀察使,徙浙東,以治行遷浙西,加御史大夫,累封河東郡公。所居守法度,務在安人。治身觳薄,所衣綠袍更十年,至緋衣乃易。居三鎮,聲樂不聞於家,所得祿,即分散親屬故人,而無餘藏。除左散騎常侍,年七十致仕。是時有年過蘋不肯去,故論者高蘋。居四年,卒,贈工部尚書,謚曰宣。蘋於文章中長於詩。



 兄芳,有器幹。萊與莘,其母代宗從母也。以外戚奉朝請,皆贊善大夫。



 莘子膺,太和初,為右補闕內供奉。其弟齊,佐興元李絳幕府,絳遇害,齊死於難。膺聞,不及請,馳赴之,哀甚,聞者垂泣。後歷工部員外郎。



 衛次公,字從周,河中河東人。舉進士,禮部侍郎潘炎異之,曰:「國器也。」高其第。調渭南尉。嚴震在興元,闢佐其府。累遷殿中侍御史。貞元中,擢左補闕、翰林學士。德宗崩,與鄭絪皆召至金鑾殿。時皇太子久疾,禁中或傳更議所立,眾失色。次公曰:「太子雖久疾,塚嫡也,內外系心久矣。必不得已,宜立廣陵王。」絪隨贊之,議乃定。



 順宗立,王叔文等用事,輕弄威柄,次公與絪多所持正。知禮部貢舉,斥華取實,不為權力侵橈。由中書舍人充史館修撰,改兵部侍郎。絪以宰相罷,坐與善,下除太子賓客。久乃為陜、虢州觀察使,蠲橫租錢歲三百萬。復入為兵部侍郎。故英公李勣、大理卿徐有功之孫,皆以負不得調,次公召見曰:「子之祖,勛在王府,寧限常格乎?」即優補而遣。進尚書左丞。時方討蔡,數建請罷兵,帝將相之,制稿具而蔡捷書至,乃追止。以檢校工部尚書為淮南節度使。久之,召還,道病卒,年六十六,贈太子少傅,謚曰敬。



 次公本善琴,方未顯時,京兆尹李齊運使子與游,請授之法,次公拒絕,因終身不復鼓。其節尚終始完絜。



 子洙,舉進士,尚臨真公主,檢校秘書少監、駙馬都尉。文宗曰:「洙起名家,以文進,宜諫官寵之。」乃為左拾遺,歷義成節度使。咸通中卒。



 薛戎,字元夫,河中寶鼎人。客毘陵陽羨山,年四十餘不仕。江西觀察使李衡闢署幕府,三返乃肯應。故宰相齊映代衡,奏留之。府罷,復歸陽羨。福建觀察使柳冕闢佐其府。先是,馬總佐鄭滑府,監軍宦人誣劾之,貶泉州別駕。冕欲除總以附幸家,即使戎攝刺史,按置其罪。戎曰:「以是待我耶?我始不願仕,正謂此爾!」不肯從,還白其狀。冕怒,據案引戎入,戎叱引者曰:「見賓客乃爾乎?」由東廂進。冕度未可屈,揖而去,囚之它館,環兵脅辱之,累月,戎終不為屈。淮南節度使杜佑聞之,書責冕,會冕亦病死,得解,自放江湖間。



 復為籓府交奏,稍遷河南令。吐突承璀討鎮州,所過吏迎廷畏不及,治道前驅,惟戎境內按故無所治迓。留府卒犯令者,縛置獄,留守怒,遣將略出之,不與。累遷浙東觀察使,所部州觸酒禁者罪當死,橘未貢先鬻者死,戎弛其禁。卒治下,年七十五,贈左散騎常侍。



 戎為吏,不尚約束詭名譽,其有善,歸之所部。故居官時無灼灼可驚者,已罷則懷之。悉奉稟賙濟內外親,無疏遠皆歸之。既病,以所有分遺之曰:「吾死矣,可持為歸資!」眾皆哭而去。



 弟放,端厚寡言。第進士,擢累兵部郎中。穆宗為太子,拜侍讀,及即位,參贊機命。帝謂曰:「小子新立,懼不克荷,先生宜相,以輔不逮。」放叩頭曰:「臣庸淺,不足塵大任,自有賢能處之。」帝美其誠,進工部侍郎、集賢學士,寵待尤至。改刑部侍郎。



 帝嘗問:「朕欲學經與史,何先?」放曰:「《六經》者,聖人之言,孔子所發明,天人之極也。《史記》道成敗得失,亦足以鑒,然謬於是非,非《六經》比。」帝曰:「吾聞學者白首不能通一經,安得其要乎?」對曰:「《論語》,《六經》之菁華也;《孝經》,人倫之本也。漢時《論語》首立於學官。光武令虎賁士皆習《孝經》,玄宗為注訓,蓋人知孝慈,則氣感和樂也。」帝曰:「聖人以孝為至德要道,信然。」終江西觀察使,謚曰簡。



 胡證,字啟中,河中河東人。舉進士第,渾瑊美其才,又以鄉府奏寘幕下。繇殿中侍御史為韶州刺史,以母老辭,為太子舍人。更從襄陽于頔,署掌書記。入為戶部郎中。田弘正以魏博內屬,請使自副,詔兼御史中丞,為弘正副使。入遷諫議大夫。



 元和九年,黨項屢擾邊,而單于都護府累更武將,職事廢,證以儒而勇選拜振武軍節度使。道河中,時趙宗儒為帥,以州民入謁,里人榮之。居四年,召任金吾大將軍,又充京西、京北巡邊使。



 太和公主降回鶻,以檢校工部尚書為和親使。舊制,行人有私覿禮,縣官不能具,召富人子納貲於使而命之官。證請儉受省費,以絕鬻官之濫。次漠南,虜人欲屈脅之,且言使者必易胡服,又欲主便道疾驅者,證固不從,以唐官儀自將,訖不辱命。還,拜工部侍郎,改京兆尹、左散騎常侍。寶歷初,以戶部尚書判度支,固辭,拜嶺南節度使。卒,年七十一,贈尚書右僕射。



 廣有舶貝奇寶,證厚殖財自奉,養奴數百人,營第脩行里,彌亙閭陌,車服器用珍侈,遂號京師高訾。素與賈餗善。李訓敗,衛軍利其財,聲言餗匿其家,爭入剽劫,執其子溵內左軍,至斬以徇。



 證旅力絕人。晉公裴度未顯時,羸服私飲,為武士所窘。證聞,突入坐客上,引觥三釂,客皆失色。因取鐵燈檠,摘枝葉,擽合其跗,橫膝上,謂客曰:「我欲為酒令,飲不釂者,以此擊之!」眾唯唯。證一飲輒數升,次授客,客流離盤杓不能盡,證欲擊之,諸惡少叩頭請去,證悉驅出。故時人稱其俠。



 丁公著,字平子,蘇州吳人。三歲喪母。甫七歲,見鄰媼抱子,哀感不肯食,請於父緒,願絕粒學老子道,父聽之。稍長,父勉敕就學。舉明經高第,授集賢校書郎,不滿秩輒去,侍養於家。父喪,負土作塚,貌力臒心叕,見者憂其死孝。觀察使薛蘋表上至行,詔刺史吊問,賜粟帛,旌闕其閭。淮南節度使李吉甫表授太子文學,兼集賢校理。會入輔政,擢為右補闕,遷直學士,充皇太子、諸王侍讀,因著《太子諸王訓》十篇。



 穆宗立,未聽政,召居禁中,條詢治理,且許以相。公著陳讓牢切,乃擢給事中,遷工部侍郎,知吏部選事。公著內知帝欲進用,故辭疾求外,遷授浙西觀察使,徙為河南尹,治以清靜聞。四遷禮部尚書、翰林侍講學士。長慶中,浙東災癘,拜觀察使,詔賜米七萬斛,使賑饑捐。久之,入為太常卿。太和中,以病丐身還鄉里,卒,年六十四,贈尚書右僕射。



 公著清約守道,每進一官,輒憂見顏間。四十喪妻,終身不畜妾。及卒,天下惜之。



 崔弘禮,字從周,系出博陵,北齊左僕射懷遠六世孫。磊磊有大志,通兵略。過宣武,從劉玄佐獵夷門,玄佐酒酣,顧曰:「崔生獨不知此樂邪?」弘禮笑曰:「我固喜武,請為公歡。」玄佐臂鷹與弘禮馳逐,急緩在手,一軍驚曰:「安得此奇客?」玄佐大悅,欲留之,固辭,厚為資餉。至京師,所善李觀病且死,弘禮殫褚為治喪,葬畢乃去。



 及進士第,平判異等。靈武李欒表為判官,以親老不應,更署東都留守呂元膺參謀。時天子討蔡,李師道謀襲洛,脅沮朝廷以釋蔡危。弘禮為箝揣賊情,部分設張,東都卒無患。遷留守判官,擢忻、汾二州刺史。田弘正請朝,表弘禮徙衛州,兼魏博節度副使。伐李師道,弘正多所咨逮。還魏博,又表為相州刺史。



 長慶初,張弘靜鎮幽州,詔弘禮往副。未及行,軍亂,改絳州刺史。李朅反於汴,詔徙河南尹,倚以捍賊。遷河陽節度使,治河內秦渠,溉田千頃,歲收八萬斛。徙華州刺史,改天平節度使。



 李同捷叛,與李聽合師討之。至濮州,大將李萬瑀、劉寀擁兵自固,弘禮表萬瑀守沂州,寀守黃州,奪其兵,擊賊禹城,破之,獲鎧裝數十萬。時徐泗節度使王智興檄兗、海、鄆、曹、淄、青當徐道者,出車五千乘,轉粟饋軍,弘禮度道遠,乃自兗開盲山故渠,自黃隊抵青丘,師人大濟。李祐以鄭滑兵三千入齊而潰,弘禮悉斬之,為出鄆兵二千,祐遂大破賊,尸藉十餘里,祐望鄆拜曰:「活我者崔公也!」加檢校尚書左僕射,徙東都留守。召還,以病自乞,改刑部尚書,復為留守。卒,年六十五,贈司空。



 弘禮短於治民,少愛利,晚頗務多積,素議訿之。



 崔玄亮,字晦叔,磁州昭義人。貞元初,擢進士第,累署諸鎮幕府。父喪,客高郵,臥苫終制,地下濕,因得痺病,不樂進取。元和初,召為監察御史,累轉駕部員外郎。清慎介特,澹如也。稍遷密歙二州刺史。歙人馬牛生駒犢,官籍蹄噭,故吏得為奸,玄亮焚其籍,一不問。民山處,輸租者苦之,下令許計斛輸錢,民賴其利。歷湖、曹二州,辭曹不拜。太和四年,繇太常少卿改諫議大夫,朝廷推為宿望,拜右散騎常侍。每遷官,輒讓形於色。



 鄭注構宋申錫,捕逮倉卒,內外震駭。玄亮率諫官叩延英苦諍,反復數百言,文宗未諭,玄亮置笏在陛曰:「孟軻有言:『眾人皆曰殺之,未可也;卿大夫皆曰殺之,未可也;天下皆曰殺之,然後察之,乃寘於法。』今殺一凡庶,當稽典律,況欲誅宰相乎?臣為陛下惜天下法,不為申錫言也。」俯伏流涕,帝感悟,眾亦服其不橈,由此名重朝廷。頃之,移疾歸東都,召為虢州刺史。卒,年六十六,贈禮部尚書。



 玄亮晚好黃老清靜術,故所居官未久輒去。遺言:「山東士人利便近,皆葬兩都,吾族未嘗遷,當歸葬滏陽,正首丘之義。」諸子如命。



 王質,字華卿。五世祖通為隋大儒。質少孤,客壽春,力耕以養母。講學不倦,諸生從授業者甚眾。年逾四十,偃蹇無進取意,姻友苦勸以仕,乃舉進士,中甲科。繇秘書省正字累佐帥府,五遷侍御史,繇山南西道節度副使再轉諫議大夫。宋申錫之得罪,質與諫官伏閣,文宗開延英召見,泣涕陳諫,帝稍寤,申錫得不死。為宦豎所惡,出虢州刺史。李德裕素器之,擢給事中、河南尹,徙宣歙觀察使。卒,年六十八,贈左散騎常侍,謚曰定。



 質清白畏慎,為政必先究風俗,所至有惠愛。雖與德裕厚善,而中立自將,不為黨。奏署幕府者,若河東裴夷直、天水趙皙、隴西李行方、梁國劉蕡,皆一時選云。



 殷侑,陳州人。幼有志於學,不治貲產。長通經術,以講道為娛。貞元末,及五經第,其學長於《禮》,擢太常博士。元和八年,回鶻請和親,朝廷以仰費廣劇,欲紓以期。詔侑、宗正少卿李孝誠使回鶻,可汗驕甚,盛陳甲兵,欲臣使者,侑不為屈。已傳命,虜責其倨,宣言欲留不遣,眾色怖,侑徐曰:「可汗,唐婿,欲坐屈使者拜,乃可汗無禮,非使臣倨也。」虜憚其言,不敢逼。還,遷虞部員外郎。



 王承宗叛,遣侑招諭,承宗聽命。進諫議大夫。侑論朝廷治亂得失,前後凡八十四通,以語切,出為桂管觀察使。寶歷元年,徙江西。所至以潔廉稱。入為衛尉卿。



 文宗即位,李同捷叛,而王廷湊陰為脣齒,兵久不解,詔五品以上官議尚書省。帝銳欲討賊,群臣無敢異論者,獨侑請舍廷湊而專事同捷,且言:「願以宗社安危為計,善師攻心為武,含垢安人為遠圖,網漏吞舟為至誡。」帝不納,然內嘉尚。



 同捷平,以侑嘗為滄州行軍司馬,遂拜義昌軍節度使。於時痍荒之餘,骸骨蔽野,墟里生荊棘,侑單身之官,安足粗淡,與下共勞苦,以仁惠為治。歲中,流戶襁屬而還,遂為營田,丐耕牛三萬,詔度支賜帛四萬匹佐其市。初,州兵三萬,仰稟度支,侑始至一歲,自以賦入贍其半,二歲則周用,乃奏罷度支所賜。戶口滋饒,廥儲盈腐,上下便安,請立石紀政。以勞加檢校吏部尚書。



 六年,徙天平節度。自李師道亂,朝廷雖析三鎮,然務安反側,賦入盡為軍貲,無輸王府者。侑以餉軍有贏,當上送官,乃裁制經費,歲以錢十五萬緡、粟五萬石歸有司。加檢校尚書右僕射。御史大夫溫造劾侑違制,擅賦斂民為無名之獻,詔以庾承宣代還。會濮州掾崔元武受吏賕,又率屬邑奉錢,增私馬估售官,疊三罪計絹百二十匹。大理以入私馬一重,削三官,刑部覆訊當流,未決。侑奏:「三犯不同,坐所重。律,頻贓者累論。元武犯皆枉法,當死。」詔用覆訊,流元武賀州。帝嘉侑守法,進刑部尚書,以造所奏不直,復用為天平節度。



 開成元年,再召為刑部尚書。時李訓、鄭注已誅,帝問侑治安術,侑言:「朝廷宜任耆德,毋輕用新進。」帝善之,賜彩三百匹。初,鹽鐵度支使屬官悉得以罪人系在所獄,或私置牢院,而州縣不聞知,歲千百數,不時決。侑奏許州縣糾列所系,申本道觀察使,並具獄上聞。許之,賜黃金十斤,以酬直言。



 涇原節度使硃叔夜坐侵牟士卒,贓數萬,家畜兵器,罷為左武衛大將軍。侑薄其罪,天子由是疏之,賜叔夜死,出侑為山南東道節度使。坐減兵不先論啟,左遷太子賓客,分司東都。俄領忠武節度。卒,年七十二,贈司空。



 侑以經術進,臨事銳敏,有強直名。晚節內冀臺輔,稍務交結,而素望少衰云。孫盈孫。



 盈孫,廣明初,為成都諸曹參軍。僖宗至蜀,聞有禮學,擢太常博士。光啟三年,帝將還京,而七廟焚殘,告享無所。盈孫白宰相:「始乘輿西,有司盡載神主以行,至鄠,悉為盜奪。今天子還宮,宜前具其禮。」宰相建言,脩復宗廟,功費廣,請與禮官議。時佗博士不在,獨盈孫從,議曰:「故廟十一室,二十三楹,楹十一梁,垣墉廣袤稱之。今朝廷多難,宜少變禮。按至德時作神主長安殿,饗告如宗廟,廟成乃祔。今正衙外無它殿,伏聞詔旨以少府監寓太廟,請因增完為十一室,其三太后廟,權舍西南夾廡,須廟成議遷。」詔可。自是神主、樂縣,皆所創定,舊學禮家當其議。



 龍紀元年,昭宗郊祠,兩中尉及樞密皆以宰相服侍上。盈孫奏言:「先世典令,無內官朝服侍祠。必欲之,當隨所攝資品,雖無授據,猶免僭逼。」詔可。時喪亂後,制度凋紊,追補容典,皆盈孫折衷焉。終大理卿,贈吏部尚書。



 王彥威,其先出太原。少孤,家無貲,自力於學。舉明經甲科。淹識古今典禮,未得調,求為太常散吏,卿知其經生,補檢討官。彥威採獲隋以來下訖唐凡禮沿革,皆條次匯分,號《元和新禮》上之。有詔拜博士。



 憲宗以正月崩,有司議葬用十二月下宿,彥威建言:「天子之葬七月,《春秋》之義。志崩不志葬,必其時也。舉天下葬一人,故過期不葬則譏之。高祖、中宗葬皆六月,太宗四月,高宗九月,睿、代二宗皆五月,德宗十月,順宗七月,惟玄、肅二宗皆十二月,有為為之,非常典也。且葬畢而虞,虞而卒哭,卒哭而祔,皆卜日。今葬卜歲暮,則畢祔在明年正月,是改元慶賜皆廢矣。」有詔更用五月。



 淮南李夷簡上言:「大行皇帝功高,宜稱祖。」穆宗下其議,彥威奏:「古者始封為太祖,由太祖而降,則又祖有功,宗有德。故夏人祖顓頊而宗禹,商人祖契而宗湯,周人祖文王而宗武王。魏晉而下,務欲推美,自始祖外並建列祖之議,叔世亂象,不可以為訓。唐本周禮,以景皇帝為太祖,祖神堯而宗太宗,自高宗後咸稱宗,以為成法。不然,太宗致升平,玄宗清內難,肅宗收復兩都,皆撥亂反正,猶不稱祖。今當本三代之制,黜魏晉亂法,大行廟號宜稱宗。」制可。又舊事,祔廟必告於太極殿,然後奉主入廟,既事則已,而有司祔主畢,又還告太極殿。彥威以為不可,執政怒,坐祝辭誤,奪二季俸,削一階。彥威終不回屈。後累擢司封郎中、弘文館學士、諫議大夫。



 李師道既平,其十二州賦法未均,詔彥威為勘定兩稅使,差量纖悉,人不為煩。還,兼史館修撰。



 興平民上官興殺人亡命,吏囚其父。興聞,自首請罪。京兆尹杜悰、御史中丞宇文鼎以自歸死免父之囚,可勸風俗,議減死。彥威上言:「殺人者死,百王共守。原而不殺,是教殺人。」有詔貸死,彥威詣宰相據法爭論,下遷河南少尹。俄改司農卿。



 李宗閔執政,雅善之,進拜平盧節度使。開成初,召為戶部侍郎,判度支。彥威於儒學固該邃,亦善吏事,但經總財用,出入米鹽,非所長也。而性剛訐自恃,嘗見文宗,顯奏曰:「百口家知有歲計,而軍用一切不可謹邪?臣按見財,量入以為出,隨色占費,終歲用之,無毫厘差。假令臣一旦迷愚,欲自欺沒,亦不可得。」因上《占頟圖》。又言:「至德迄元和,天下觀察者十,節度者二十有九,防御者四,經略者三,大都通邑皆有兵,最凡八十餘萬。長慶籍戶三百五十萬,而兵乃九十九萬,率三戶資一兵。今舉天下之入,歲三千五百萬,上供者三之一,又三之二則衣賜仰給焉。自留州留使外,餘四十萬眾,皆仰度支。」又為《供軍圖》上之。彥威雖自謂楗柅奸冒,著定其費,於利害無益也。



 始,神策軍多以稟縑于度支取直,吏私增賈厚給之,經用益耗。開成初,有詔禁止。時宦者仇士良、魚弘志方用事,彥威乃奏復與直,悅媚士良等。又效王播貢羨贏以冀速進。會邊兵訴所賜不時,縑皆敝惡,攝吏送臺獄,而彥威視事自如,及詔停務,始惶恐就第。貶衛尉卿。



 俄檢校禮部尚書,為忠武節度使,毀山房三千餘所,盜無所容。徙節宣武,封北海縣子。性強敏,善著書,頗行於時。卒,贈尚書右僕射,謚曰靖。



 贊曰:韓愈稱:「郡邑通得祀社稷、孔子。獨孔子用王者事,以門人為配,天子以下,北面拜跪薦祭,禮如親弟子者。句龍、棄以功,孔子則以德,固自有次第。」崇敬乃請東揖,以殺太重。方是時,公卿無韓愈之賢,無有折其非是者。道州刺史薛伯高嘗謂:「夫子稱顏回為庶幾,其從於陳、蔡者,亦各有號,出於一時,後世坐祀十人以為哲,豈夫子志哉?」觀七十子之賢,未有加於十人,坐而祀之,始於開元,非特牽於一時之稱號。《記》曰:「祭,有其舉之,莫敢廢也。」如崇敬誠不知禮,尊君以媚世,歷朝循而不改矣。伯高之語,柳宗元志之於其書,必有辨其妄者。



\end{pinyinscope}