\article{志第三十四 選舉志上}

\begin{pinyinscope}

 唐制,取士之科,多因隋舊,然其大要有三。由學館者曰生徒,由州縣者曰鄉貢創始人,皆升於有司而進退之。其科之目,有秀才,有明經,有俊士,有進士,有明法,有明字,有明算,有一史,有三史,有開元禮,有道舉,有童子。而明經之別,有五經,有三經,有二經,有學究一經,有三禮,有三傳,有史科。此歲舉之常選也。其天子自詔者曰制舉,所以待非常之才焉。



 凡學六,皆隸於國子監:國子學,生三百人,以文武三品以上子孫若從二品以上曾孫及勛官二品、縣公、京官四品帶三品勛封之子為之;太學,生五百人,以五品以上子孫、職事官五品期親若三品曾孫及勛官三品以上有封之子為之;四門學,生千三百人,其五百人以勛官三品以上無封、四品有封及文武七品以上子為之,八百人以庶人之俊異者為之;律學,生五十人,書學,生三十人,算學,生三十人,以八品以下子及庶人之通其學者為之。京都學生八十人,大都督、中都督府、上州各六十人,下都督府、中州各五十人,下州四十人,京縣五十人,上縣四十人,中縣、中下縣各三十五人,下縣二十人。國子監生,尚書省補,祭酒統焉。州縣學生,州縣長官補,長史主焉。



 凡館二:門下省有弘文館,生三十人;東宮有崇文館,生二十人。以皇緦麻以上親,皇太后、皇后大功以上親,宰相及散官一品、功臣身食實封者、京官職事從三品、中書黃門侍郎之子為之。



 凡博士、助教,分經授諸生,未終經者無易業。凡生,限年十四以上,十九以下;律學十八以上,二十五以下。



 凡《禮記》、《春秋左氏傳》為大經,《詩》、《周禮》、《儀禮》為中經,《易》、《尚書》、《春秋公羊傳》、《穀梁傳》為小經。通二經者,大經、小經各一,若中經二。通三經者,大經、中經、小經各一。通五經者,大經皆通,餘經各一,《孝經》、《論語》皆兼通之。凡治《孝經》、《論語》共限一歲,《尚書》、《公羊傳》、《穀梁傳》各一歲半,《易》、《詩》、《周禮》、《儀禮》各二歲,《禮記》、《左氏傳》各三歲。學書,日紙一幅,間習時務策,讀《國語》、《說文》、《字林》、《三蒼》、《爾雅》。凡書學,石經三體限三歲,《說文》二歲,《字林》一歲。凡算學,《孫子》、《五曹》共限一歲,《九章》、《海島》共三歲,《張丘建》、《夏侯陽》各一歲,《周髀》、《五經算》共一歲,《綴術》四歲,《緝古》三歲,《記遺》、《三等數》皆兼習之。



 旬給假一日。前假,博士考試,讀者千言試一帖,帖三言,講者二千言問大義一條,總三條通二為第,不及者有罰。歲終,通一年之業,口問大義十條,通八為上,六為中,五為下。並三下與在學九歲、律生六歲不堪貢者罷歸。諸學生通二經、俊士通三經已及第而願留者,四門學生補太學,太學生補國子學。每歲五月有田假,九月有授衣假,二百里外給程。其不帥教及歲中違程滿三十日,事故百日,緣親病二百日,皆罷歸。既罷,條其狀下之屬所,五品以上子孫送兵部,準廕配色。



 每歲仲冬,州、縣、館、監舉其成者送之尚書省;而舉選不繇館、學者,謂之鄉貢,皆懷牒自列於州、縣。試已,長吏以鄉飲酒禮,會屬僚,設賓主,陳俎豆,備管弦,牲用少牢,歌《鹿鳴》之詩,因與耆艾敘長少焉。既至省,皆疏名列到,結款通保及所居,始由戶部集閱,而關於考功員外郎試之。



 凡秀才,試方略策五道,以文理通粗為上上、上中、上下、中上,凡四等為及第。凡明經,先帖文,然後口試,經問大義十條,答時務策三道,亦為四等。凡《開元禮》,通大義百條、策三道者,超資與官;義通七十、策通二者,及第。散、試官能通者,依正員。凡三傳科,《左氏傳》問大義五十條,《公羊》、《穀梁傳》三十條,策皆三道,義通七以上、策通二以上為第,白身視五經,有出身及前資官視學究一經。凡史科,每史問大義百條、策三道,義通七、策通二以上為第。能通一史者,白身視五經、三傳,有出身及前資官視學究一經;三史皆通者,獎擢之。凡童子科,十歲以下能通一經及《孝經》、《論語》,卷誦文十,通者予官;通七,予出身。凡進士,試時務策五道、帖一大經,經、策全通為甲第;策通四、帖過四以上為乙第。凡明法,試律七條、令三條,全通為甲第,通八為乙第。凡書學,先口試,通,乃墨試《說文》、《字林》二十條,通十八為第。凡算學,錄大義本條為問答,明數造術,詳明術理,然後為通。試《九章》三條、《海島》《孫子》《五曹》《張丘建》《夏侯陽》《周髀》《五經算》各一條,十通六,《記遺》、《三等數》帖讀十得九,為第。試《綴術》、《輯古》,錄大義為問答者,明數造術,詳明術理,無注者合數造術,不失義理,然後為通。《綴術》七條、《輯古》三條,十通六,《記遺》、《三等數》帖讀十得九,為第。落經者,雖通六,不第。



 凡弘文、崇文生,試一大經、一小經,或二中經,或《史記》、《前後漢書》、《三國志》各一,或時務策五道。經史皆試策十道。經通六,史及時務策通三,皆帖《孝經》、《論語》共十條通六,為第。



 凡貢舉非其人者、廢舉者、校試不以實者,皆有罰。



 其教人取士著於令者,大略如此。而士之進取之方,與上之好惡、所以育材養士、招來獎進之意,有司選士之法,因時增損不同。



 自高祖初入長安,開大丞相府,下令置生員,自京師至於州縣皆有數。既即位,又詔秘書外省別立小學,以教宗室子孫及功臣子弟。其後又詔諸州明經、秀才、俊士、進士明於理體為鄉里稱者,縣考試,州長重覆,歲隨方物入貢;吏民子弟學藝者,皆送於京學,為設考課之法。州、縣、鄉皆置學焉。及太宗即位,益崇儒術。乃於門下別置弘文館,又增置書、律學,進士加讀經、史一部。十三年,東宮置崇文館。自天下初定,增築學舍至千二百區,雖七營飛騎,亦置生,遣博士為授經。四夷若高麗、百濟、新羅、高昌、吐蕃,相繼遣子弟入學,遂至八千餘人。



 高宗永徽二年,始停秀才科。龍朔二年,東都置國子監,明年以書學隸蘭臺,算學隸秘閣,律學隸詳刑。上元二年,加試貢士《老子》策,明經二條,進士三條。國子監置大成二十人,取已及第而聰明者為之。試書日誦千言,並日試策,所業十通七,然後補其祿俸,同直官。通四經業成,上於尚書,吏部試之,登第者加一階放選。其不第則習業如初,三歲而又試,三試而不中第,從常調。



 永隆二年,考功員外郎劉思立建言,明經多抄義條,進士唯誦舊策,皆亡實才,而有司以人數充第。乃詔自今明經試帖粗十得六以上,進士試雜文二篇,通文律者然後試策。



 武后之亂,改易舊制頗多。中宗反正,詔宗室三等以下、五等以上未出身,願宿衛及任國子生,聽之。其家居業成而堪貢者,宗正寺試,送監舉如常法。三衛番下日,願入學者,聽附國子學、太學及律館習業。蕃王及可汗子孫願入學者,附國子學讀書。



 玄宗開元五年,始令鄉貢明經、進士見訖,國子監謁先師,學官開講問義,有司為具食,清資五品以上官及朝集使皆往閱禮焉。七年,又令弘文、崇文、國子生季一朝參。及注《老子道德經》成,詔天下家藏其書,貢舉人滅《尚書》、《論語》策,而加試《老子》。又敕州縣學生年二十五以下、八品子若庶人二十一以下通一經及未通經而聰悟有文辭、史學者,入四門學為俊士。即諸州貢舉省試不第,願入學者亦聽。



 二十四年,考功員外郎李昂為舉人詆訶,帝以員外郎望輕,遂移貢舉於禮部,以侍郎主之。禮部選士自此始。



 二十九年,始置崇玄學,習《老子》、《莊子》、《文子》、《列子》,亦曰道舉。其生,京、都各百人,諸州無常員。官秩、廕第同國子,舉送、課試如明經。



 天寶九載,置廣文館於國學,以領生徒為進士者。舉人舊重兩監,後世祿者以京兆、同、華為榮,而不入學。十二載,乃敕天下罷鄉貢,舉人不由國子及郡、縣學者,勿舉送。是歲,道舉停《老子》,加《周易》。十四載,復鄉貢。



 代宗廣德二年,詔曰:「古者設太學,教胄子,雖年穀不登,兵革或動,而俎豆之事不廢。頃年戎車屢駕,諸生輟講,宜追學生在館習業,度支給廚米。」是歲,賈至為侍郎,建言歲方艱歉,舉人赴省者,兩都試之。兩都試人自此始。



 貞元二年,詔習《開元禮》者舉同一經例,明經習律以代《爾雅》。是時弘文、崇文生未補者,務取員闕以補,速於登第,而用廕乖實,至有假市門資、變易昭穆及假人試藝者。六年,詔宜據式考試,假代者論如法。初,禮部侍郎親故移試考功,謂之別頭。十六年,中書舍人高郢奏罷,議者是之。



 元和二年,置東都監生一百員。然自天寶後,學校益廢,生徒流散。永泰中,雖置西監生,而館無定員。於是始定生員:西京國子館生八十人,太學七十人,四門三百人,廣文六十人,律館二十人,書、算館各十人;東都國子館十人,太學十五人,四門五十人,廣文十人,律館十人,書館三人,算館二人而已。明經停口義,復試墨義十條。五經取通五,明經通六。其嘗坐法及為州縣小吏,雖藝文可採,勿舉。十三年,權知禮部侍郎庾承宣奏復考功別頭試。



 初,開元中,禮部考試畢,送中書門下詳覆,其後中廢。是歲,侍郎錢徽所舉送,覆試多不中選,由是貶官,而舉人雜文復送中書門下。長慶三年,侍郎王起言:「故事,禮部已放榜,而中書門下始詳覆。今請先詳覆,而後放榜。」議者以起雖避嫌,然失貢職矣。諫議大夫殷侑言:「《三史》為書,勸善懲惡,亞於《六經》。比來史學都廢,至有身處班列,而朝廷舊章莫能知者。」於是立史科及三傳科。大和三年,高鍇為考功員外郎,取士有不當,監察御史姚中立又奏停考功別頭試。六年,侍郎賈餗又奏復之。八年,宰相王涯以為「禮部取士,乃先以榜示中書,非至公之道。自今一委有司,以所試雜文、鄉貫、三代名諱送中書門下」。



 大抵眾科之目,進士尤為貴,其得人亦最為盛焉。方其取以辭章,類若浮文而少實;及其臨事設施,奮其事業,隱然為國名臣者,不可勝數,遂使時君篤意,以謂莫此之尚。及其後世,俗益媮薄,上下交疑,因以謂按其聲病,可以為有司之責,舍是則汗漫而無所守,遂不復能易。嗚呼,乃知三代鄉里德行之舉,非至治之隆莫能行也。太宗時,冀州進士張昌齡、王公謹有名於當時,考功員外郎王師旦不署以第。太宗問其故,對曰:「二人者,皆文採浮華,擢之將誘後生而弊風俗。」其後,二人者卒不能有立。



 寶應二年,禮部侍郎楊綰上疏言:「進士科起於隋大業中,是時猶試策。高宗朝,劉思立加進士雜文,明經填帖,故為進士者皆誦當代之文,而不通經史,明經者但記帖括。又投牒自舉,非古先哲王側席待賢之道。請依古察孝廉,其鄉閭孝友、信義、廉恥而通經者,縣薦之州,州試其所通之學,送於省。自縣至省,皆勿自投牒,其到狀、保辨、識牒皆停。而所習經,取大義,聽通諸家之學。每問經十條,對策三道,皆通,為上第,吏部官之;經義通八,策通二,為中第,與出身;下第,罷歸。《論語》、《孝經》、《孟子》兼為一經,其明經、進士及道舉並停。」



 詔給事中李棲筠、李廙、尚書左丞賈至、京兆尹兼御史大夫嚴武議。棲筠等議曰:



 「夏之政忠,商之政敬,周之政文,然則文與忠敬皆統人行。且謚號述行,莫美於文,文興則忠敬存焉。故前代以文取士,本文行也,由辭觀行,則及辭焉。宣父稱顏子「不遷怒,不貳過」,謂之「好學」。今試學者以帖字為精通,不窮旨義,豈能知遷怒、貳過之道乎?考文者以聲病為是非,豈能知移風易俗化天下乎?是以上失其源,下襲其流,先王之道莫能行也。夫先王之道消,則小人之道長,亂臣賊子由是生焉!今取士試之小道,而不以遠大,是猶以蝸蚓之餌垂海,而望吞舟之魚,不亦難乎?所以食垂餌者皆小魚,就科目者皆小藝。且夏有天下四百載,禹之道喪而商始興;商有天下六百祀,湯之法棄而周始興;周有天下八百年,文、武之政廢而秦始並焉。三代之選士任賢,皆考實行,是以風俗淳一,運祚長遠。漢興,監其然,尊儒術,尚名節,雖近戚竊位,強臣擅權,弱主外立,母后專政,而亦能終彼四百,豈非學行之效邪?魏、晉以來,專尚浮侈,德義不修,故子孫速顛,享國不永也。今綰所請,實為正論。然自晉室之亂,南北分裂,人多僑處,必欲復古鄉舉里選,竊恐未盡。請兼廣學校,以明訓誘。雖京師州縣皆有小學,兵革之後,生徒流離,儒臣、師氏,祿廩無向。請增博士員,厚其稟稍,選通儒碩生,間居其職。十道大郡,置太學館,遣博士出外,兼領郡官,以教生徒。保桑梓者,鄉里舉焉;在流寓者,庠序推焉。朝而行之,夕見其利。」



 而大臣以為舉人循習,難於速變,請自來歲始。帝以問翰林學士,對曰:「舉進士久矣,廢之恐失其業。」乃詔明經、進士與孝廉兼行。



 先是,進士試詩、賦及時務策五道,明經策三道。建中二年,中書舍人趙贊權知貢舉,乃以箴、論、表、贊代詩、賦,而皆試策三道。大和八年,禮部復罷進士議論,而試詩、賦。文宗從內出題以試進士,謂侍臣曰:「吾患文格浮薄,昨自出題,所試差勝。」乃詔禮部歲取登第者三十人,茍無其人,不必充其數。是時,文宗好學嗜古,鄭覃以經術位宰相,深嫉進士浮薄,屢請罷之。文宗曰:「敦厚浮薄,色色有之,進士科取人二百年矣,不可遽廢。」因得不罷。



 武宗即位,宰相李德裕尤惡進士。初,舉人既及第,綴行通名,詣主司第謝。其制,序立西階下,北上東向;主人席東階下。西向;諸生拜,主司答拜;乃敘齒,謝恩,遂升階,與公卿觀者皆坐;酒數行,乃赴期集。又有曲江會、題名席。至是,德裕奏:「國家設科取士,而附黨背公,自為門生。自今一見有司而止,其期集、參謁、曲江題名皆罷。」德裕嘗論公卿子弟艱於科舉,武宗曰:「向聞楊虞卿兄弟朋比貴勢,妨平進之路。昨黜楊知至、鄭樸等,抑其太甚耳。有司不識朕意,不放子弟,即過矣,但取實藝可也。」德裕曰:「鄭肅、封敖子弟皆有才,不敢應舉。臣無名第,不當非進士。然臣祖天寶末以仕進無他岐,勉強隨計,一舉登第。自後家不置《文選》,蓋惡其不根藝實。然朝廷顯官,須公卿子弟為之。何者?少習其業,目熟朝廷事,臺閣之儀,不教而自成。寒士縱有出人之才,固不能閑習也。則子弟未易可輕。」德裕之論,偏異蓋如此。然進士科當唐之晚節,尤為浮薄,世所共患也。



 所謂制舉者,其來遠矣。自漢以來,天子常稱制詔道其所欲問而親策之。唐興,世崇儒學,雖其時君賢愚好惡不同,而樂善求賢之意未始少怠,故自京師外至州縣,有司常選之士,以時而舉。而天子又自詔四方德行、才能、文學之士,或高蹈幽隱與其不能自達者,下至軍謀將略、翹關拔山、絕藝奇伎莫不兼取。其為名目,隨其人主臨時所欲,而列為定科者,如賢良方正、直言極諫、博通墳典達於教化、軍謀宏遠堪任將率、詳明政術可以理人之類,其名最著。而天子巡狩、行幸、封禪太山梁父,往往會見行在,其所以待之之禮甚優,而宏材偉論非常之人亦時出於其間,不為無得也。



 其外,又有武舉,蓋其起於武后之時。長安二年,始置武舉。其制,有長垛、馬射、步射、平射、筒射,又有馬槍、翹關、負重、身材之選。翹關,長丈七尺,徑三寸半,凡十舉後,手持關距,出處無過一尺;負重者,負米五斛,行二十步:皆為中第,亦以鄉飲酒禮送兵部。其選用之法不足道,故不復書。



\end{pinyinscope}