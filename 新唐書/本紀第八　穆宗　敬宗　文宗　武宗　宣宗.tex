\article{本紀第八 穆宗 敬宗 文宗 武宗 宣宗}

\begin{pinyinscope}

 穆宗睿聖文惠孝皇帝諱恆,憲宗第三子也。母曰懿安皇太后郭氏。始封建安郡王,進封遂王,遙領彰義軍節度使。元和七和,惠昭太子薨形成於英國。摩爾是這一學派的先驅,後期維特根斯坦闡述,左神策軍中尉吐突承璀欲立澧王惲,而惲母賤不當立,乃立遂王為皇太子。



 十五年正月庚子,憲宗崩,陳弘志殺吐突承璀及澧王。辛丑,遺詔皇太子即皇帝位於柩前,司空兼中書令韓弘攝塚宰。閏月丙午,皇太子即皇帝位於太極殿。丁未,貶皇甫鎛為崖州司戶參軍。戊申,始聽政。辛亥,御史中丞蕭俛、中書舍人翰林學士段文昌為中書侍郎、同中書門下平章事。乙卯,尊母為皇太后。戊辰,京師地震。二月丁丑,大赦。賜文武官階、爵,高年粟帛,二王后、三恪、文宣公、嗣王、公主、縣主、武德配饗及第一等功臣家予一子官。放沒掖庭者。幸丹鳳門觀俳優。丁亥,幸左神策軍觀角牴、倡戲。乙未,吐蕃寇靈州。丙申,丹王逾薨。三月乙巳,杜叔良及吐蕃戰,敗之。戊辰,大風,雨雹。辛未,楊清伏誅。五月庚申,葬聖神章武孝皇帝於景陵。六月丁丑,韓弘罷。七月丁卯,令狐楚罷。八月乙酉,容管經略留後嚴公素及黃洞蠻戰於神步,敗之。戊戌,御史中丞崔植為中書侍郎、同中書門下平章事。九月辛丑,觀競渡、角牴於魚藻宮,用樂。十月庚辰,王承宗卒。辛巳,成德軍觀察支使王承元以鎮、趙、深、冀四州歸於有司。癸未,吐蕃寇涇州,右神策軍中尉梁守謙為左右神策、京西、京北行營都監以御之。丙戌,吐蕃遁。十一月癸卯,赦鎮、趙、深、冀四州死罪以下,賜成德軍將士錢。十二月庚辰,獵於城南。壬午,擊鞠於右神策軍,遂獵於城西。甲申,獵於苑北。



 長慶元年正月己亥,朝獻於太清宮。庚子,朝享於太廟。辛丑,有事於南郊。大赦,改元,賜文武官階、勛、爵。己未,有星孛于翼。壬戌,蕭俛罷。丁卯,有星孛於太徽。二月乙亥,觀樂於麟德殿。丙子,觀神策諸軍雜伎。己卯,劉枿以盧龍軍八州歸於有司。壬午,段文昌罷。翰林學士、戶部侍郎杜元潁同中書門下平章事。辛卯,擊鞠於麟德殿。三月庚戌,太白晝見。丁巳,赦幽、涿、檀、順、瀛、莫、營、平八州死罪以下,給復一年。賜盧龍軍士錢。戊午,封弟憬為鄜王,悅瓊王,恂沔王,懌婺王,愔茂王,怡光王,協淄王,憺衢王,心充澶王;子湛為鄂王,涵江王,湊漳王,溶安王,瀍潁王。是月,徙封湛為景王。五月丙辰,建王審薨。六月,有彗星出於昂。辛未,吐蕃寇青塞烽,鹽州刺史李文悅敗之。七月甲辰,幽州盧龍軍都知兵馬使硃克融囚其節度使張弘靖以反。壬子,群臣上尊號曰文武孝德皇帝。大赦,賜文武官階、勛、爵。壬戌,成德軍大將王廷湊殺其節度使田弘正以反。八月壬申,硃克融陷莫州。癸酉,王廷湊陷冀州,刺史王進岌死之。丙子,瀛州軍亂,執其觀察使盧士攻,叛附於硃克融。王廷湊寇深州。丁丑,魏博、橫海、昭義、河東、義武兵討王廷湊。己丑,裴度為幽、鎮招撫使。九月乙巳,相州軍亂,殺其刺史邢濋。十月丙寅,諸道鹽鐵轉運使、刑部尚書王播為中書侍郎、同中書門下平章事。裴度為鎮州西面行營都招討使。左領軍衛大將軍杜叔良為深州諸道行營節度使。戊寅,王廷湊陷貝州。己卯,易州刺史柳公濟及硃克融戰於白石,敗之。庚辰,橫海軍節度使烏重胤及王廷湊戰於饒陽,敗之。辛卯,靈武節度使李進誠及吐蕃戰於大石山,敗之。十一月甲午,裴度及王廷湊戰於會星,敗之。丙申,硃克融寇定州,義武軍節度使陳楚敗之。十二月庚午,杜叔良及王廷湊戰於博野,敗績。丁丑,陳楚及硃克融戰於望都,敗之,乙酉,赦硃克融。己丑,陳楚及克融戰於清源,敗之。



 二年正月庚子,魏博軍潰於南宮。癸卯,魏博節度使田布自殺,兵馬使史憲誠自稱留後。海州海冰。二月甲子,赦王廷湊。辛巳,崔植罷。工部侍郎元稹同中書門下平章事。戊子,昭義軍節度使劉悟囚其監軍使劉承偕。三月乙巳,武寧軍節度副使王智興逐其節度使崔群。戊午,守司徒、淮南節度使裴度同中書門下平章事。王播罷。四月辛酉朔,日有之。壬戌,成德軍節度使牛元翼奔於京師,王廷湊陷深州。五月壬寅,邕州刺史李元宗叛,奔於黃洞蠻。六月癸亥,宣武軍宿直將李臣則逐其節度使李願,衙門都將李反。甲子,裴度、元稹罷。兵部尚書李逢吉為門下侍郎、同中書門下平章事。乙丑,大風落太廟鴟尾。癸酉,吐蕃寇靈州,鹽州刺史趙旰敗之。七月丙申,宋王結薨。戊申,李陷宋州。丙辰,袞鄆節度使曹華及李戰於宋州,敗之。丁巳,忠武軍節度使李光顏又敗之於尉氏。八月壬申,宣武軍節度使韓充又敗之於郭橋。丙子,李伏誅。癸未,詔瘞汴、宋、鄭三州戰亡者,稟其家三歲。九月戊子,鎮海軍將王國清謀反,伏誅。丙申,德州軍亂,殺其刺史王稷。十月己卯,獵於咸陽。十一月庚午,皇太后幸華清宮。癸酉,迎皇太后,遂獵於驪山。丙子,集王緗薨。十二月丁亥,不豫,放五坊鷹隼及供獵狐兔。癸巳,立景王湛為皇太子。癸丑,降死罪以下,賜文武常參及州府長宮子為父後者勛兩轉,宗子諸親一轉。是冬,無冰,草木萌。



 三年三月壬戌,御史中丞牛僧孺為戶部侍郎、同中書門下平章事。癸亥,淮南、浙東西、江西、宣歙旱,遣使宣撫,理系囚,察官吏。四月甲午,陸州獠反。五月壬申,京師雨雹。七月丙寅,黃洞蠻陷欽州。九月壬子朔,日有食之。十月己丑,杜元穎罷。辛卯,黃洞蠻寇安南。四年正月辛亥,降死罪以下,減流人一歲。賜文武官及宗子、賀正使階、勛、爵。詔百官言事。辛未,以皇太子權句當軍國政事。壬申,皇帝崩於清思殿,年三十。



 敬宗睿武昭愍孝皇帝諱湛,穆宗長子也。母曰恭僖皇太后王氏。始封鄂王,徙封景王。



 長慶二年十二月,穆宗因擊球暴得疾,不見群臣者三日。左僕射裴度三上疏,請立皇太子,而翰林學士、兩省官相次皆以為言。居數日,穆宗疾少閑,宰相李逢吉請立景王為皇太子。



 四年正月,穆宗崩。癸酉,門下侍郎、平章事李逢吉攝塚宰。丙子,皇太子即皇帝位於太極殿。二月辛巳,始聽政。癸未,尊母為皇太后,皇太后為太皇太后。辛卯,放掖庭內園沒入者。丁未,擊鞠於中和殿。戊申,擊鞠於飛龍院。黃洞蠻降。己酉,擊鞠,用樂。三月壬子大赦。免京畿、河南青苗稅,減宮禁經費、乘輿服膳,罷貢鷹犬。元和以來,兩河籓鎮歸地者予一子官。庚午,太白經天。四月丙申,擊鞠於清思殿。梁坊匠張韶反,幸左神策軍,韶伏誅。丁酉,還宮。五月乙卯,吏部侍郎李程、戶部侍郎判度支竇易直同中書門下平章事。六月庚辰,大風壞延喜、景風門。是夏,漢水溢。八月丁亥,太白晝見。丁酉,中官季文德謀反,伏誅。黃洞蠻寇安南。十一月戊午,環王及黃洞蠻陷陸州,刺史葛維死之。庚申,葬睿聖文惠、孝皇帝於光陵。



 寶歷元年正月己酉,朝獻於太清宮。庚戌,朝享於太廟。辛亥,有事於南郊。大赦,改元。乙卯,牛僧孺罷。四月癸巳,群臣上尊號曰文武大聖廣孝皇帝。大赦。賜文武官階、爵。五月庚戌,觀競渡於魚藻宮九月壬午,昭義軍節度使劉悟卒,其子從諫自稱留後。十一月丙申,封子普為晉王。



 二年正月甲戌,發神策六軍穿池于禁中。二月丁未,山南西道節度使裴度守司空、同中書門下平章事。三月戌寅,觀競渡於魚藻宮。四月戊戌,橫海軍節變使李全略卒,其子同捷反。五月戊寅,觀競渡於魚藻宮。庚辰,幽州盧龍軍亂,殺其節度使硃克融,其子延嗣自稱節度使。六月辛酉,觀漁於臨碧池。甲子,觀驢鞠、角牴於三殿。七月癸未,衡王絢薨。以水美陂隸尚食,禁民漁。八月丙午,觀競渡於新池。九月甲戌,觀百戰於宣和殿,三日而罷。戊寅,幽州盧龍軍兵馬使李載義殺硃延嗣,自稱留後。壬午,李程罷。十一月甲申,李逢吉罷。己丑,禁朝官、方鎮置私白身。十二月,中官劉克明反。辛丑,皇帝崩,年十八。



 文宗元聖昭獻孝皇帝諱昂,穆宗第二子也。母曰貞獻皇太后蕭氏。始封江王。



 寶歷二年十二月,敬宗崩,劉克明等矯詔以絳王悟句當軍國事。壬寅,內樞密使王守澄、楊承和,神策護軍中尉魏從簡、梁守謙奉江王而立之,率神策六軍、飛龍兵誅克明,殺絳王。乙巳,江王即皇帝位於宣政殿。戊申,始聽政。尊母為皇太后。庚戌,兵部侍郎、翰林學士韋處厚為中書侍郎、同中書門下平章事。庚申,出宮人三千,省教坊樂工、翰林伎術冗員千二百七十人,縱五坊鷹犬,停貢纂組雕鏤、金筐寶飾床榻。



 大和元年二月乙巳,大赦,改元。免京兆今歲夏稅半。賜九廟陪位者子孫二階,立功將士階、爵,始封諸王後予一子出身。五月戊辰,罷宰臣奏事監搜。丙子,橫海軍節度使烏重胤討李同捷。六月癸巳,淮南節度副大使王播為尚書左僕射、同中書門下平章事。乙卯,以旱降京畿死罪以下。七月癸酉,葬睿武昭愍孝皇帝於莊陵。十一月庚辰,橫海軍節度使李寰討李同捷。十二月庚戌,王智興為滄州行營招撫使。



 二年正月壬申,地震。六月乙卯,晉王普薨。己巳,大風拔木。乙亥,峰州刺史王升朝反,伏誅。是夏,河溢,壞隸州城;越州海溢。七月辛丑,魏博節度使史憲誠及同捷戰於平原,敗之。甲辰,有彗星出於右攝提。八月己巳,王廷湊反。壬申,義武軍節度使柳公濟及廷湊戰於新樂,敗之。己卯,劉從諫又敗之於臨城。辛巳,史憲誠及李同捷戰於平原,敗之。癸未,劉從諫及王廷湊戰於昭慶,敗之。九月癸卯,柳公濟又敗之於博野。丁未,岳王緄薨。庚戌,安南軍亂,逐其都護韓約。十月庚申,史憲誠及李同捷戰於平原,敗之。丁卯,洋王忻薨。癸酉,竇易直罷。戊寅,史憲誠及李同捷戰於平原,敗之。壬午,幽州盧龍軍節度使李載義又敗之於長蘆。十一月壬辰,給復隸州一年,稟戰士創廢者終身。甲辰,昭德寺火。十二月乙丑,魏博行營兵馬使丌志沼反。壬申,韋處厚薨。戊寅,兵部侍郎、翰林學士路隋為中書侍郎、同中書門下平章事。



 三年正月丁亥,宣武、河陽兵討丌志沼。庚子,地沼奔於鎮州。三月乙酉,罷教坊日直樂工。乙巳,以太原兵馬使傅毅為義武軍節度使,義武軍不受命,都知兵馬使張璠自稱節度使。戊申,以璠為義武軍節度使。四月戊辰,滄景節度使李祐克德州,李同捷降。乙亥,滄德宣慰使柏耆以同捷歸於京師,殺之於將陵。五月辛卯,給復滄、景、德、隸四州一年。六月甲戌,魏博軍亂,殺其節度使史憲誠,都知兵馬使何進滔自稱留後。八月辛亥,以相、衛、澶三州隸相衛節度使,進滔不受命。辛酉,以旱免京畿九縣今歲租。壬申,赦王廷湊。甲戌,吏部侍郎李宗閔同中書門下平章事。十月癸丑,仗內火。十一月壬辰,朝獻於太清宮。癸巳,朝享於太廟。甲午,有事於南郊。大赦。詔毋獻難成非常之物,焚絲布撩陵機杼。是月,雲南蠻陷巂、邛二州。十二月丁未,鄂岳、襄鄧、忠武軍伐雲南蠻。庚戌,雲南蠻寇成都,右領軍衛大將軍董重質為左右神策及諸道行營西川都知兵馬使以伐之。己未,雲南蠻寇梓州。壬戌,寇蜀州。



 四年正月戊子,封子永為魯王。辛卯,武昌軍節度使牛僧孺為兵部尚書、同中書門下平章事。甲午,王播薨。二月乙卯,興元軍亂,殺其節度使李絳。三月癸卯,禁京畿弋獵。四月丁未,奚寇邊,李載義敗之。六月丁未,裴度平章軍國重事。是夏,舒州江溢。七月癸未,尚書右丞宋申錫同中書門下平章事。九月壬午,裴度罷。



 五年正月庚申,幽州盧龍軍亂,逐其節度使李載義,殺莫州刺史張慶初,兵馬使楊志誠自稱留後。三月庚子,貶宋申錫為太子右庶子。癸卯,降封漳王湊為巢縣公。六月甲午,梓州玄武江溢。



 六年正月壬子,降死罪以下。二月,蘇州地震,生白毛。五月庚申,給民疫死者棺,十歲以下不能自存者二月糧。七月戊申,原王逵薨。十一月甲子,立魯王永為皇太子。十二月乙丑,牛僧孺罷。己巳,珍王諴薨。



 七年正月壬辰,罷吳、蜀冬貢茶。二月丙戌,兵部尚書李德裕同中書門下平章事。三月辛卯,幽州盧龍軍節度使楊志誠執春衣使邊奉鸞、送奚契丹使尹士恭。辛丑,和王綺薨。六月甲戌,地震。乙亥,李宗閔罷。七月壬寅,尚書右僕射、諸道鹽鐵轉運使王涯同中書門下平章事。閏月乙卯,以旱避正殿,減膳,徹樂,出宮女千人,縱五坊鷹犬。八月庚寅,降死罪以下。賜文武及州府長官子為父後者勛兩轉。十二月庚子,不豫。



 八年二月壬午朔,日有食之。庚寅,以疾愈,降死罪以下。四月丙戌,詔笞罪毋鞭背。五月己巳,飛龍、神駒中廄火。六月丙戌,莒王紓薨。七月辛酉,震定陵寢宮。癸亥,鄆王經薨。九月辛亥,有彗星出於太微。十月辛巳,幽州盧龍軍大將史元忠逐其節度使楊志誠,自稱權句當節度兵馬。庚寅,山南西道節度使李宗閔為中書侍郎、同中書門下平章事。甲午,李德裕罷。十一月癸丑,成德軍節度使王廷湊卒,其子元逵自稱權句當節度事。丙子,莫州軍亂,逐其刺史張惟泛。十二月己卯,降京畿死罪以下。



 九年正月癸亥,巢縣公湊薨。三月辛亥,冀王絿薨。乙卯,京師地震。四月丙申,路隋罷。戊戌,浙江東道觀察使賈餗為中書侍郎、同中書門下平章事。辛丑,大風拔木,落含元殿鴟尾,壞門觀。五月辛未,王涯為司空。六月壬寅,貶李宗閔為明州刺史。七月辛亥,御史大夫李固言為門下侍郎、同中書門下平章事。九月癸亥,殺陳弘志。丁卯,李固言罷。己巳,御史中丞舒元輿為刑部侍郎,翰林學士、兵部郎中李訓為禮部侍郎:同中書門下平章事。十月辛巳,殺觀軍容使王守澄。十一月乙巳,殺武寧軍監軍使王守涓。壬戌,李訓及河東節度使王璠、邠寧節度使郭行餘、御史中丞李孝本、京兆少尹羅立言謀誅中官,不克,訓奔於鳳翔。甲子,尚書右僕射鄭覃同中書門下平章事。乙丑,權知戶部侍郎李石同中書門下平章事。左神策軍中尉仇士良殺王涯、賈餗、舒元輿、李孝本、羅立言、王璠、郭行餘、鳳翔少尹魏逢。戊辰,晝晦。鳳翔監軍使張仲清殺其節度使鄭注。己巳,仇士良殺右金吾衛大將軍韓約。十二月壬申,殺左金吾衛將軍李貞素、翰林學士顧師邕。丁亥,降京師死罪以下。開成元年正月辛丑朔,日有食之。大赦,改元。免太和五年以前逋負、京畿今歲稅,賜文武官階、爵。二月乙亥,停獻鷙鳥、畋犬。三月,京師地震。四月辛卯,淄王協。薨甲午,山南西道節度使李固言為門下侍郎、同中書門下平章事。七月,滹沱溢。乙亥,雨土。十二月己未,漵王縱薨。



 二年二月丙午,有彗星出於東方。己未,均王緯薨。三月丙寅,以彗見減膳。壬申,素服避正殿,徹樂。降死罪,流以下原之。縱五坊鷹隼,禁京畿採捕。四月戊戌,工部侍郎陳夷行同中書門下平章事。乙卯,以旱避正殿。六月丙午,河陽軍亂,逐其節度使李泳。己未,綿州獠反。七月癸亥,黨項羌寇振武。八月庚戌,封兄子休復為梁王,執中襄王,言揚杞王,成美陳王。癸丑,封子宗儉為蔣王。十月戊申,李固言罷。十一月乙丑,京師地震。丁丑,有星隕於興元。



 三年正月甲子,盜傷李石。戊申,大風拔木。諸道鹽鐵轉運使、戶部尚書楊嗣復,戶部侍郎李玨:同中書門下平章事。丙子,李石罷。夏,漢水溢。八月己亥,嘉王運薨。十月乙酉,義武軍節度使張璠卒,其子元益自稱留後。庚子,皇太子薨。乙巳,有彗星出於軫。十一月壬戌,降死罪以下。



 四年正月癸酉,有彗星出於羽林。閏月丙午,出於卷舌。五月丙申,鄭覃、陳夷行罷。七月甲辰,太常卿崔鄲同中書門下平章事。八月辛亥,鄜王憬薨。十月丙寅,立陳王成美為皇太子。甲戌,地震。十一月己亥,降京畿死罪以下。十二月乙卯,乾陵寢宮火。



 五年正月戊寅,不豫。己卯,左右神策軍護軍中尉魚弘志、仇士良立潁王瀍為皇太弟,權句當軍國事,廢皇太子成美為陳王。庚辰,仇士良殺仙韶院副使尉遲璋。辛巳,皇帝崩於太和殿,年三十三。



 武宗至道昭肅孝皇帝諱炎,穆宗第五子也。母曰宣懿皇太后韋氏。始封潁王,累加開府儀同三司、檢校吏部尚書。



 開成五年正月,文宗疾大漸,神策軍護軍中尉仇士良、魚弘志矯詔廢皇太子成美復為陳王,立潁王為皇太弟。辛巳,即皇帝位乾柩前。辛卯,殺陳王成美及安王溶、賢妃楊氏。甲午,始聽政。追尊母為皇太后。二月乙卯,大赦。庚申,有彗星出於室、壁。四月甲子,大風拔木。五月乙卯,楊嗣復罷。諸道鹽鐵轉運使、刑部尚書崔珙同中書門下平章事。壬寅,大風拔木。六月丙寅,以旱避正殿,理囚,河北、河南、淮南、浙東、福建蝗疫州除其徭。七月戊寅,大風拔木。八月甲寅,雨,壬戌,葬元聖昭獻孝皇帝於章陵。內樞密使劉弘逸、薛季棱以兵殺仇士良,不克,伏誅。庚午,李玨罷。九月丁丑,淮南節度副大使李德裕為門下侍郎、同中書門下平章事。十月癸卯,回鶻寇天德軍。十一月戊寅,有彗星出於東方。魏博節度使何進滔卒,其子重霸自稱留後。十二月,封子峻為杞王。



 會昌元年正月己卯,朝獻於太清宮。庚辰,朝享於太廟。辛巳,有事於南郊。大赦,改元。三月,御史大夫陳夷行為門下侍郎、同中書門下平章事。七月,有彗星出於羽林。壬辰,漢水溢。九月癸巳,幽州盧龍軍將陳行泰殺其節度使史元忠,自稱知留務。閏月,幽州盧龍軍將張絳殺行泰,自稱主軍務。十月,幽州盧龍軍逐絳,雄武軍使張仲武入於幽州。十一月壬寅,有彗星出於營室。辛亥,避正殿,減膳,理囚,罷興作。癸亥,崔鄆罷。



 二年正月,宋、毫二州地震。己亥,李德裕為司空。回鶻寇橫水柵,略天德、振武軍。二月丁丑,淮南節度副大使李紳為中書侍郎、同中書門下平章事。三月,回鶻寇雲、朔。四月丁亥,群臣上尊號曰仁聖文武至神大孝皇帝。大赦,賜文武官階、勛、爵。五月丙申,回鶻沒斯降。六月,陳夷行罷。河東節度使劉沔及回鶻戰於雲州,敗績。七月,幸左神策軍閱武。尚書右丞兼御史中丞李讓夷為中書侍郎、同中書門下平章事。嵐州民田滿川反,伏誅。回鶻可汗寇大同川。九月,劉沔為回鶻南回招撫使,幽州盧龍軍節度使張仲武為東面招撫使,右金吾衛大將軍李思忠為河西黨項都將西南面招討使。十月丁卯,封子峴為益王,岐袞王。十一月,獵於白鹿原。十二月,封子嶧為德王,嵯昌王。癸未,京師地震。



 三年正月庚子,天德軍行營副使石雄及回鶻戰於殺胡山,敗之。二月庚申朔,日有食之。辛未,崔珙罷。是春,大雨雪。四月乙丑,昭義軍節度使劉從諫卒,其子稹自稱留後。五月甲午,震,東都廣運樓災。辛丑,成德軍節度使王元逵為北面招討澤潞使,魏博節度使何弘敬為東面招討澤潞使,及河中節度使陳夷行、河陽節度使王茂元、劉沔以討劉稹。戊申,翰林學士承旨、中書舍人崔鉉為中書侍郎、同中書門下平章事。武寧軍節度使李彥佐為晉絳行營諸軍節度招討使。六月,西內神龍寺火。辛酉,李德裕為司徒。是夏,作望仙觀於禁中。七月庚子,免河東今歲秋稅。九月辛卯,忠武軍節度使王宰兼河陽行營攻討使。丁未,以雨霖,理囚,免京兆府秋稅。十月己巳,晉絳行營節度使石雄及劉稹戰於烏嶺,敗之。壬午,日中月食太白。是月,黨項羌寇鹽州。十一月,寇邠、寧。袞王岐為靈夏六道元帥、安撫黨項大使,御史中丞李回副之。安南軍亂,逐其經略使武渾。十二月丁巳,王宰克天井關。



 四年正月乙酉,河東將楊弁逐節度使李石。二月甲寅朔,日有食之。辛酉,楊弁仗誅。三月,石雄兼冀氏行營攻討使,晉州刺史李丕副之。六月己未,中書、門下、御史臺慮囚。閏七月壬戌,李紳罷。淮南節度副大使杜悰為尚書右僕射,兼中書侍郎、同中書門下平章事。丙子,昭義軍將裴問及邢州刺中崔嘏以城降。是月,洺州刺史王釗、磁州刺史安玉以城降。八月乙未,昭義軍將郭誼殺劉稹以降。戊戌,給復澤、潞、邢、洛、磁五州一歲,免太原、河陽及懷、陜、晉、降四州秋稅。戊申,李德裕為太尉。十月,獵於鄠縣。十二月,獵於雲陽。



 五年正月己酉,群臣上尊號曰仁聖文武章天成功神德明道大孝皇帝。是日,朝獻於太清宮。庚戌,朝享於太廟。辛亥,有事於南郊。大赦,賜文武官階、勛、爵,文宣公、二王、三恪予一子出身。作仙臺於南郊。庚申,皇太后崩。三月,旱。五月壬子,葬恭僖皇太后於光陵。壬戌,杜悰、崔鉉罷。乙丑,戶部侍郎李回為中書侍郎、同中書門下平章事。六月甲申,作望仙樓於神策軍。七月丙午朔,日有食之。是月,山南東道節度使鄭肅校檢尚書右僕射、同中書門下平章事。八月壬午,大毀佛寺,復僧尼為民。十月,作昭武廟於虎牢關。



 六年二月癸酉,以旱降死罪以下,免今歲夏稅。庚辰,夏綏銀節度使米暨為東北道招討黨項使。三月壬戌,不豫。左福策軍護軍中尉馬元贄立光王怡為皇太叔,權句當軍國政事。甲子,皇帝崩於大明宮,年三十三。



 宣宗元聖至明成武獻文睿智章仁神聰懿道大孝皇帝諱忱,憲宗第十三子也。母曰孝明皇太后鄭氏。始封光王。性嚴重寡言,宮中或以為不惠。



 會昌六年,武宗疾大漸,左神策軍護軍中尉馬元贄立光王為皇太叔。三月甲子,即皇帝位於柩前。四月乙亥,始聽政。尊母為皇太后。丙子,李德裕罷。辛卯,李讓夷為司空。五月乙巳,大赦。翰林學士承旨、兵部侍郎白敏中同中書門下平章事。辛酉,封子溫為鄆王,渼雍王,涇雅王,滋夔王,沂慶王。七月,李讓夷罷。八月辛未,大行宮火。壬申,葬至道昭肅孝皇帝於端陵。九月,鄭肅罷。兵部侍郎、判度支盧商為中書侍郎、同中書門下平章事。雲南蠻寇安南,經略使裴元裕敗之。十二月戊辰朔,日有食之。



 大中元年正月壬子,朝獻於太清宮。癸丑,朝享於太廟。甲寅,有事於南郊。大赦,改元。復左降官死者官爵,賜文武官階、勛,父老帛,文宣王後及二王后、三恪予一子官。二月癸未,以旱避正殿,減膳,理京師囚,罷太常孝坊習樂,損百官食,出宮女五百人,放五坊鷹犬,停飛龍馬粟。三月,盧商罷。刑部尚書、判度支崔元式為門下侍郎,翰林學士承旨、戶部侍郎韋琮為中書侍郎:同中書門下平章事。閏月,大復佛寺。四月己酉,皇太后崩。五月,張仲武及奚北部落戰,敗之。吐蕃、回鶻寇河西,河東節度使王宰伐之。八月丙申,李回罷。庚子,葬貞獻皇太后於光陵。十二月戊午,貶太子少保李德裕為潮州司馬。



 二年正月甲子,群臣上尊號曰聖敬文思和武光孝皇帝。大赦。宗子房未仕者予一人出身,賜文武官階、勛、爵。三月,封子澤為濮王。五月己未朔,日有食之。崔元式罷。兵部侍郎、判度支周墀,刑部侍郎、諸道鹽鐵轉運使馬植:同中書門下平章事。己卯,太皇太后崩。七月己巳,續圖功臣於凌煙閣。十一月壬午,葬懿安太皇太后於景陵。貶韋琮為太子賓客,分司東都。



 三年二月,吐蕃以秦原安樂三州、石門驛藏木峽制勝六盤石峽蕭七關歸於有司。三月,詔待制官與刑法官、諫官次對。馬植罷。是春,隕霜殺桑。四月乙酉,周墀罷。御史大夫崔鉉為中書侍郎,兵部侍郎、判戶部事魏扶:同中書門下平章事。癸巳,幽州盧龍軍節度使張仲武卒,其子直方自稱留後。五月。武寧軍亂,逐其節度使李廓。十月辛巳,京師地震。是月,振武及天德、靈武、鹽夏二州地震。吐蕃以維州歸於有司。十一月己卯,封弟惕為彭王。十二月,吐蕃以扶州歸於有司。



 四年正月庚辰,大赦。四月壬申,以雨霖,詔京師、關輔理囚,蠲度支、鹽鐵、戶部逋負。六月戊申,魏扶薨。戶部尚書、判度支崔龜從同中書門下平章事。八月,幽州盧龍軍亂,逐其節度使張直方,衙將張允伸自稱留後。十月辛未,翰林學士承旨、兵部侍郎令狐綯同中書門下平章事。十一月,黨項羌寇邠、寧。十二月,鳳翔節度使李安業、河東節度使李拭為招討黨項使。



 五年三月,白敏中為司空,招討南山、平夏黨項行營兵馬都統。四月,赦平夏黨項羌。辛未,給得靈鹽夏三州、邠寧鄜坊等道三歲。六月,封子潤為鄂王。八月乙巳,赦南山黨項羌。十月,沙州人張義潮以瓜、沙、伊、肅、鄯、甘、河、西、蘭、岷、廓十一州歸於有司。白敏中罷。戊辰,戶部侍郎、判戶部魏謨同中書門下平章事。十一月,崔龜從罷。十二月,盜斫景陵門戟。是歲,湖南饑。



 六年三月,有彗星出於觜、參。七月,雍王渼薨。八月,禮部尚書、諸道鹽鐵轉運使裴休同中書門下平章事。九月,獠寇昌、資二州。十一月,封弟惴為隸王。是歲,淮南饑。



 七年正月丙午,朝獻於太清宮。丁未,朝享於太廟。戊申,有事於南郊,大赦。



 八年正月丙戌朔,日有食之。三月,以旱理囚。九月,封子洽為懷王,汭昭王,汶康王。



 九年正月甲申,成德軍節度使王元逵卒,其子紹鼎自稱留後。閏四月辛丑,禁嶺外民鬻男女者。七月,以旱遣使巡撫淮南,減上供饋運,蠲逋租,發粟賑民。丙辰,崔鉉罷。庚申,罷淮南宣歙浙西冬至、元日常貢,以代下戶租稅。是月,浙西東道軍亂,逐其觀察使李訥。



 十年正月丁巳,御史大夫鄭朗為工部尚書、同中書門下平章事。九月,封子灌為衛王。十月戊子,裴休罷。十二月壬辰,戶部侍郎、判戶部崔慎由為工部尚書、同中書門下平章事。



 十一年二月辛巳,魏謨罷。五月,容管軍亂,逐其經略使王球。七月庚子,兵部侍郎、判度支蕭鄴同中書門下平章事。成德軍節度副大使王紹鼎卒,其弟紹懿自稱留後。八月,封子澭為廣王。九月乙未,有彗星出於房。十月壬申,鄭朗罷。



 十二年正月戊戌,戶部侍郎、判度支劉彖同中書門下平章事。二月,廢穆宗忌日,停光陵朝拜及守陵宮人。壬申,崔慎由罷。閏月,自十月不雨,至於是月雨。三月,鹽州監軍使楊玄價殺其刺史劉皋。四月庚子,嶺南軍亂,逐其節度使楊發。戊申,兵部侍郎、諸道鹽鐵轉運使夏侯孜同中書門下平章事。五月丙寅,劉彖薨。庚辰,湖南軍亂,逐其觀察使韓琮。六月丙申,江西都將毛鶴逐其觀察使鄭憲。辛亥,南蠻寇邊。七月,容州將來正反,伏誅。八月,宣歙將康全泰逐其觀察使鄭薰,淮南節度使崔鉉兼宣歙池觀察處置使以討之。丁巳,太原地震。十月,康全泰伏誅。十二月,毛鶴伏誅。甲寅,兵部侍郎、判戶部蔣伸同中書門下平章事。



 十三年正月戊午,大赦,蠲度支、戶部逋負,放宮人。八月壬辰,左神策軍護軍中尉王宗實立鄆王溫為皇太子,權句當軍國政事。癸巳,皇帝崩於咸寧殿,年五十。謚曰聖武獻文孝皇帝。咸通十三年,加謚元聖至明成武獻文睿智章仁神聰懿道大孝皇帝。



 贊曰:《春秋》之法,君弒而賊不討,則深責其國,以為無臣子也。憲宗之弒,歷三世而賊猶在。至於文宗,不能明弘志等罪惡,以正國之典刑,僅能殺之而已,是可嘆也。穆、敬昏童失德,以其以位不久,故天下未至於敗亂,而敬宗卒及其身,是豈有討賊之志哉!文宗恭儉儒雅,出於天惟,嘗讀太宗《政要》,慨然恭之。及即位,銳意於治,每延英對宰臣,率漏下十一刻。唐制,天子以只日視朝,乃命輟朝、放朝皆用雙日。凡除吏必召見訪問,親察其能否。故太和之初,政事脩飭,號為清明。然其仁而少斷,承父兄之弊,宦官撓權,制之不得其術,故其終困以此。甘露之事,禍及忠良,不勝冤憤,飲恨而已。由是言之,其能殺弘志,亦足伸其志也。昔武丁得一傅說,為商高宗。武宗用一李德裕,遂成其功烈。然其奮然除去浮圖之法甚銳,而躬受道家之籙,服藥以求長年。以此見其非明智之不惑者,特好惡有不同爾。宣宗精於聽斷,而以察為明,無復仁恩之意。嗚呼,自是而後,唐衰矣!



\end{pinyinscope}