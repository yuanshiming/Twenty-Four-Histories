\article{志第十四 車服}

\begin{pinyinscope}

 唐初受命,車、服皆因隋舊。武德四年,始著車輿、衣服之令,上得兼下,下不得擬上。



 凡天子之車:



 曰玉路者,祭祀、納后所乘也,青質,玉飾末;金路者,饗、射、祀還、飲至所乘也,赤質,金飾末;象路者,行道所乘也,黃質,象飾末;革路者,臨兵、巡守所乘也,白質,鞔以革;木路者,蒐田所乘也,黑質,漆之。五路皆重輿,左青龍,右白虎,金鳳翅,畫苣文鳥獸,黃屋左纛。金鳳一、鈴二在軾前,鸞十二在衡。龍輈前設鄣塵。青蓋三層,繡飾。上設博山方鏡,下圓鏡。樹羽。輪金根、硃班、重牙。左建旗,十有二旒,畫升龍,其長曳地,青繡綢杠。右載闟戟,長四尺,廣三尺,黻文。旗首金龍銜錦結綬及緌帶,垂鈴。金鍐方釳,插翟尾五焦,鏤錫,鞶纓十二就。旌旗、蓋、鞶纓,皆從路質,唯蓋里皆用黃。五路皆有副。



 耕根車者,耕藉所乘也,青質,三重蓋,餘如玉路。安車者,臨幸所乘也,金飾重輿,曲壁,紫油纁,硃里通幰,硃絲絡網,硃鞶纓,硃覆發具絡,駕赤騮。副路、耕根車、安車,皆八鸞。四望車者,拜陵、臨吊所乘也,制如安車,青油纁,硃里通幰,硃絲絡網。又有屬車十乘:一曰指南車,二曰記里鼓車,三曰白鷺車,四曰鸞旗車,五曰闢惡車,六曰皮軒車,七曰羊車,與耕根車、四望車、安車為十乘。行幸陳於鹵簿,則分前後;大朝會,則分左右。



 皇后之車六:



 重翟車者,受冊、從祀、饗廟所乘也,青質,青油纁,硃里通幰,繡紫絡帶及帷,八鸞,鏤錫,鞶纓十二就,金鍐方釳,樹翟羽,硃總。



 厭翟車者,親桑所乘也,赤質,紫油纁,硃里通幰,紅錦絡帶及帷。



 翟車者,歸寧所乘也,黃油纁,黃裏通幰,白紅錦絡帶及帷。三車皆金飾末,輪畫硃牙,箱飾翟羽,硃絲絡網,鞶、纓色皆從車質。



 安車者,臨幸所乘也,制如金路,紫油纁,硃里通幰。



 四望車者,拜陵、臨喪所乘也,青油纁,硃里通幰。



 金根車者,常行所乘也,紫油纁,硃里通幰。



 夫人乘厭翟車,九嬪乘翟車,婕妤以下乘安車。外命婦、公主、王妃乘厭翟車。一品乘白銅飾犢車,青油纁,硃里通幰,硃絲絡網。二品以下去油纁、絡網。四品有青偏幰。



 皇太子之車三:



 金路者,從祀、朝賀、納妃所乘也,赤質,金飾末,重較,箱畫苣文鳥獸,黃屋,伏鹿軾,龍輈,金鳳一,在軾前,設鄣塵,硃黃蓋里,輪畫硃牙。左建旗九旒,右載闟戟,旗首金龍銜結綬及鈴緌,入鸞二鈴,金鍐方釳,樹翟尾五焦,鏤錫,鞶纓九就。



 軺車者,五日常服、朝饗、宮臣、出入行道所乘也。



 四望車者,臨吊所乘也。二車皆金飾末,紫油纁,硃里通幰。



 親王及武職,一品有象路,青油纁,硃里通幰,硃絲絡網。二品、三品有革路,硃里青通幰。四品有木路,五品有軺車,皆碧裏青偏幰。象飾末,班輪,八鸞,左建旗,畫升龍,右載闟戟。革路、木路,左建旃。軺車,曲壁,碧裏青通幰。諸路,硃質、硃蓋、硃旗、硃班輪。一品之旃九旒,二品八旒,三品七旒,四品六旒,鞶纓就亦如之。三品以上珂九子,四品七子,五品五子,六品以下去通幰及珂。



 王公車路,藏於太僕,受制、行冊命、巡陵、昏葬則給之。餘皆以騎代車。



 凡天子之服十四:



 大裘冕者,祀天地之服也。廣八寸,長一尺二寸,以板為之,黑表,纁里,無旒,金飾玉簪導,組帶為纓,色如其綬,黈纊充耳。大裘,繒表,黑羔表為緣,纁里,黑領、襟、襟緣,硃裳,白紗中單,皁領,青襟、襈、裾,硃襪,赤舄。鹿盧玉具劍,火珠鏢首,白玉雙佩。黑組大雙綬,黑質,黑、黃、赤、白、縹、綠為純,為備天地四方之色。廣一尺,長二丈四尺,五百首。紛廣二寸四分,長六尺四寸,色如綬。又有小雙綬,長二尺六寸,色如大綬,而首半之,間施三玉環。革帶,以白皮為之,以屬佩、綬、印章。鞶囊,亦曰鞶帶,博三寸半,加金鏤玉鉤褵。大帶,以素為之,以硃為里,在腰及垂皆有裨,上以硃錦,貴正色也,下以綠錦,賤間色也,博四寸。紐約,貴賤皆用青組,博三寸。黻以繒為之,隨裳色,上廣一尺,以象天數,下廣二尺以象地數,長三尺,硃質,畫龍、火、山三章,以象三才,其頸五寸,兩角有肩,廣二寸,以屬革帶。朝服謂之鞸,冕服謂之黻。



 袞冕者,踐祚、饗廟、徵還、遣將、飲至、加元服、納后、元日受朝賀、臨軒冊拜王公之服也。廣一尺二寸,長二尺四寸,金飾玉簪導,垂白珠十二旒,硃絲組帶為纓,色如綬。深青衣纁裳,十二章:日、月、星辰、山、龍、華蟲、火、宗彞八章在衣;藻、粉米、黼、黻四章在裳。衣畫,裳繡,以象天地之色也。自山、龍以下,每章一行為等,每行十二。衣、褾、領,畫以升龍,白紗中單,黻領,青褾、襈、裾,韍繡龍、山、火三章,舄加金飾。



 冕者,有事遠主之服也。八旒,七章:華蟲、火、宗彞三章在衣;藻、粉米、黼、黻四章在裳。



 毳冕者,祭海岳之服也。七旒,五章:宗彞、藻、粉米在衣;黼、黻在裳。



 絺冕者,祭社稷、饗先農之服也。六旒,三章:絺、粉米在衣;黼、黻在裳。



 玄冕者,蠟祭百神、朝日、夕月之服也。五旒,裳刺黼一章。自袞冕以下,其制一也,簪導、劍、佩、綬皆同。



 通天冠者,冬至受朝賀、祭還、燕群臣、養老之服也。二十四梁,附蟬十二首,施珠翠、金博山,黑介幘,組纓翠緌,玉、犀簪導,絳紗袍,硃裏紅羅裳,白紗中單,硃領、褾、襈、裾,白裙、襦,絳紗蔽膝,白羅方心曲領,白蔑,黑舄。白假帶,其制垂二絳帛,以變祭服之大帶。天子未加元服,以空頂黑介幘,雙童髻,雙玉導,加寶飾。三品以上亦加寶飾,五品以上雙玉導,金飾,六品以下無飾。



 緇布冠者,始冠之服也。天子五梁,三品以上三梁,五品以上二梁,九品以上一梁。



 武弁者,講武、出征、搜狩、大射、祃、類、宜社、賞祖、罰社、纂嚴之服也。有金附蟬,平巾幘。



 弁服者,朔日受朝之服也。以鹿皮為之,有攀以持發,十有二綦,玉簪導,絳紗衣,素裳,白玉雙佩,革帶之後有鞶囊,以盛小雙綬,白韈,烏皮履。



 黑介幘者,拜陵之服也。無飾,白紗單衣,白裙、襦,革帶,素韈,烏皮履。



 白紗帽者,視朝、聽訟、宴見賓客之服也。以烏紗為之,白裙、襦,白韈,烏皮履。



 平巾幘者,乘馬之服也。金飾,玉簪導,冠支以玉,紫褶,白褲,玉具裝,珠寶鈿帶,有鞾。



 白帢者,臨喪之服也。白紗單衣,烏皮履。



 皇后之服三:



 褘衣者,受冊、助祭、朝會大事之服也。深青織成為之,畫翬,赤質,五色,十二等。素紗中單,黼領,硃羅縠褾、襈,蔽膝隨裳色,以緅領為緣,用翟為章,三等。青衣,革帶、大帶隨衣色,裨、紐約、佩、綬如天子,青韈,舄加金飾。



 鞠衣者,親蠶之服也。黃羅為之,不畫,蔽膝、大帶、革帶、舄隨衣色,餘同褘衣。



 鈿釵襢衣者,燕見賓客之服也。十一鈿,服用雜色而不畫,加雙佩小綬,去舄加履,首飾大小華十二樹,以象袞冕之旒,又有兩博鬢。



 皇太子之服六:



 袞冕者,從祀、謁廟、加元服、納妃之服也。白珠九旒,紅絲組為纓,犀簪導,青纊充耳。黑衣纁裳,凡九章:龍、山、華蟲、火、宗彞在衣,藻、粉米、黼、黻在裳。白紗中單,黼領,青褾、襈、裾。革帶金鉤褵,大帶,瑜玉雙佩。硃組雙大綬,硃質,赤、白、縹、紺為純,長一丈八尺,廣九寸,三百二十首。黻隨裳色,有火、山二章。白韈,赤舄,硃履,加金塗銀扣飾。鹿盧玉具劍如天子。



 遠游冠者,謁廟、還宮、元日朔日入朝、釋奠之服也。以具服,遠游冠三梁,加金博山,附蟬九首,施珠翠,黑介幘,發纓翠緌,犀簪導,絳紗袍,紅裳,白紗中單,黑領、褾、襈、裾,白裙、襦,白假帶,方心曲領,絳紗蔽膝,白韈,黑舄。朔日入朝,通服褲褶。



 公服者,五日常朝、元日冬至受朝之服也。遠游冠,絳紗單衣,白裙、襦,革帶金鉤褵,假帶,瑜玉只佩,方心,紛,金縷鞶囊,純長六尺四寸,廣二寸四分,色如大綬。



 烏紗帽者,視事及燕見賓客之服也。白裙、襦,烏皮履。



 弁服者,朔、望視事之服也。鹿皮為之,犀簪導,組纓九綦,絳紗衣,素裳,革帶,鞶囊,小綬,雙佩。自具服以下,皆白韈,烏皮履。



 平巾幘者,乘馬之服也。金飾,犀簪導,紫裙,白褲,起梁珠寶鈿帶,鞾。進德冠者,亦乘馬之服也。九綦,加金飾,有褲褶,常服則有白裙、襦。



 皇太子妃之服有三:



 褕翟者,受冊、助祭、朝會大事之服也。青織成。文為搖翟,青質,五色九等。素紗中單,黼領,硃羅縠褾、襈,蔽膝隨裳色,用緅為領緣,以翟為章二等。青衣,革帶、大帶隨衣色,不硃里,青韈,舄加金飾,佩、綬如皇太子。



 鞠衣者,從蠶之服也。以黃羅為之,制如褕翟,無雉,蔽膝、大帶隨衣色。



 鈿釵襢衣者,燕見賓客之服也。九鈿,其服用雜色,制如鞠衣,加雙佩,小綬,去舄加履,首飾花九樹,有兩博鬢。



 群臣之服二十有一:



 袞冕者,一品之服也。九旒,青綦為珠,貫三採玉,以組為纓,色如其綬。青纊充耳,寶飾角簪導。青衣纁裳,九章:龍、山、華蟲、火、宗彞在衣,藻、粉米、黼、黻在裳,皆絳為繡遍衣。白紗中單,黼領,青褾、襈、裾。硃韈,赤舄。革帶鉤褵,大帶,黻隨裳色。金寶玉飾劍鏢首,山玄玉佩。綠綟綬,綠質,綠、紫、黃、赤為純,長一丈八尺,廣九寸,二百四十首。郊祀太尉攝事亦服之。



 冕者,二品之服也。八旒,青衣纁裳,七章:華蟲、火、宗彞在衣;藻、粉米、黼、黻在裳,銀裝劍,佩水蒼玉,紫綬、紫質,紫、黃、赤為純,長一丈六尺,廣八寸,一百八十首。革帶之後有金鏤鞶囊,金飾劍,水蒼玉佩,硃韈,赤舄。



 毳冕者,三品之服也。七旒,寶飾角簪導,五章:宗彞、藻、粉米在衣;黼、黻在裳。韍二章:山、火。紫綬如二品,金銀鏤鞶囊,金飾劍,水蒼玉佩,硃韈,赤舄。



 絺冕者,四品之服也。六旒,三章:粉米在衣;黼、黻在裳,中單,青領。韍,山一章。銀鏤鞶囊。自三品以下皆青綬,青質,青、白、紅為純,長一丈四尺,廣七寸,一百四十首,金飾劍,水蒼玉佩,硃韈,赤舄。



 玄冕者,五品之服也。以羅為之,五旒,衣、韍無章,裳刺黻一章。角簪導,青衣纁裳,其服用紬。大帶及裨,外黑內黃,黑綬紺質,青紺為純,長一丈二尺,廣六寸,一百二十首。象笏,上圓下方,六品以竹木,上挫下方。金飾劍,水蒼玉佩,硃韈,赤舄。三品以下私祭皆服之。



 平冕者,郊廟武舞郎之服也。黑衣絳裳,革帶,烏皮履。



 爵弁者,六品以下九品以上從祀之服也。以紬為之,無旒,黑纓,角簪導,青衣纁裳,白紗中單,青領、褾、襈、裾,革帶鉤褵,大帶及裨內外皆緇,爵韡,白韈,赤履。五品以上私祭皆服之。



 武弁者,武官朝參、殿庭武舞郎、堂下鼓人、鼓吹桉工之服也。有平巾幘,武舞緋絲布大袖,白練衣盍襠,螣蛇起梁帶,豹文大口褲,烏皮鞾。鼓人硃褷衣,革帶,烏皮履。鼓吹桉工加白練衣盍襠。



 弁服者,文官九品公事之服也。以鹿皮為之,通用烏紗,牙簪導。纓:一品九綦,二品八綦,三品七綦,四品六綦,五品五綦,犀簪導,皆硃衣素裳,革帶,鞶囊,小綬,雙佩,白韈,烏皮履。六品以下去綦及鞶囊、綬、佩。六品、七品綠衣,八品、九品青衣。



 進賢冠者,文武朝參、三老五更之服也。黑介幘,青緌。紛長六尺四寸,廣四寸,色如其綬。三品以上三梁,五品以上兩梁,九品以上及國官一梁,六品以下私祭皆服之。侍中、中書令、左右散騎常侍有黃金璫,附蟬,貂尾。侍左者左珥,侍右者右珥。諸州大中正一梁絳紗公服。殿庭文舞郎,黃紗袍,黑領、襈,白練衣盍襠,白布大口褲,革帶,烏皮履。



 遠游冠者,親王之服也。黑介幘,三梁,青緌,金鉤褵大帶,金寶飾劍,玉鏢首,纁硃綬,硃質,赤、黃、縹、紺為純,長一丈八尺,廣九寸,二百四十首。黃金璫,附蟬,諸王則否。



 法冠者,御史大夫、中丞、御史之服也。一名解廌冠。



 高山冠者,內侍省內謁者、親王司閣、謁者之服也。



 委貌冠者,郊廟文舞郎之服也。有黑絲布大袖,白練領、褾,絳布大口褲,革帶,烏皮履。



 卻非冠者,亭長、門僕之服也。



 平巾幘者,武官、衛官公事之服也。金飾,五品以上兼用玉,大口褲,烏皮鞾,白練裙、襦,起梁帶。陪大仗,有裲襠、螣蛇。朝集從事、州縣佐史、岳瀆祝史、外州品子、庶民任掌事者服之,有緋褶、大口褲,紫附褷。文武官騎馬服之,則去裲襠、珣蛇。褲褶之制:五品以上,細綾及羅為之,六品以下,小綾為之,三品以上紫,五品以上緋,七品以上綠,九品以上碧。裲襠之制:一當胸,一當背,短袖覆膊。螣蛇之制:以錦為表,長八尺,中實以綿,象蛇形。起梁帶之制:三品以上,玉梁寶鈿,五品以上,金梁寶鈿,六品以下,金飾隱起而已。



 黑介幘者,國官視品、府佐謁府、國子大學四門生俊士參見之服也。簪導,白紗單衣,青襟、褾、領,革帶,烏皮履。未冠者,冠則空頂黑介幘,雙童髻,去革帶。書算律學生、州縣學生朝參,則服烏紗帽,白裙、襦,青領。未冠者童子髻。



 介幘者,流外官、行署三品以下、登歌工人之服也。絳公服,以縵緋為之,制如絳紗單衣,方心曲領,革帶鉤褵,假帶,韈,烏皮履。九品以上則絳褷衣,制如絳公服而狹,袖形直如褵,不垂,緋褶大口褲,紫附褵,去方心曲領、假帶。登歌工人,硃連裳,革帶,烏皮履。殿庭加白練衣盍襠。



 平巾綠幘者,尚食局主膳,典膳局典食,太官署、食官署供膳、奉觶之服也。青絲布褲褶。羊車小史,五辮髻,紫碧腰襻,青耳屩。漏刻生、漏童,總角髻,皆青絲布褲褶。



 具服者,五品以上陪祭、朝饗、拜表、大事之服也,亦曰朝服。冠幘,簪導,絳紗單衣,白紗中單,黑領、袖,黑褾、襈、裾,白裙、襦,革帶金鉤褵,假帶,曲領方心,絳紗蔽膝,白韈,烏皮舄,劍,紛,鞶囊,雙佩,雙綬。六品以下去劍、佩、綬,七品以上以白筆代簪,八品、九品去白筆,白紗中單,以履代舄。



 從省服者,五品以上公事、朔望朝謁、見東宮之服也,亦曰公服。冠幘纓,簪導,絳紗單衣,白裙、襦,革帶鉤褵,假帶,方心,韈,履,紛,鞶囊,雙佩,烏皮履。六品以下去紛、鞶囊、雙佩。三品以上有公爵者,嫡子之婚,假絺冕。五品以上子孫,九品以上子,爵弁。庶人婚,假絳公服。



 命婦之服六:



 翟衣者,內命婦受冊、從蠶、朝會,外命婦嫁及受冊、從蠶、大朝會之服也。青質,繡翟,編次於衣及裳,重為九等。青紗中單,黼領,硃縠褾、襈、裾,蔽膝隨裳色,以緅為領緣,加文繡,重雉為章二等。大帶隨衣色,以青衣,革帶,青韈,舄,佩,綬,兩博鬢飾以寶鈿。一品翟九等,花釵九樹;二品翟八等,花釵八樹;三品翟七等,花釵七樹;四品翟六等,花釵六樹;五品翟五等,花釵五樹。寶鈿視花樹之數。



 鈿釵禮衣者,內命婦常參、外命婦朝參、辭見、禮會之服也。制同翟衣,加雙佩、小綬,去舄,加履。一品九鈿,二品八鈿,三品七鈿,四品六鈿,五品五鈿。



 禮衣者,六尚、寶林、御女、採女、女官七品以上大事之服也。通用雜色,制如鈿釵禮衣,唯無首飾、佩、綬。



 公服者,常供奉之服也。去中單、蔽膝、大帶,九品以上大事、常供奉亦如之。半袖裙襦者,東宮女史常供奉之服也。公主、王妃佩、綬同諸王。



 花釵禮衣者,親王納妃所給之服也。



 大袖連裳者,六品以下妻,九品以上女嫁服也。青質,素紗中單,蔽膝、大帶、革帶,韈、履同裳色,花釵,覆笄,兩博鬢,以金銀雜寶飾之。庶人女嫁有花釵,以金銀琉璃塗飾之。連裳,青質,青衣,革帶,韈、履同裳色。



 婦人燕服視夫。百官女嫁、廟見攝母服。五品以上媵降妻一等,妾降媵一等,六品以下妾降妻一等。



 天子有傳國璽及八璽,皆玉為之。神璽以鎮中國,藏而不用。受命璽以封禪禮神,皇帝行璽以報王公書,皇帝之璽以勞王公,皇帝信璽以召王公,天子行璽以報四夷書,天子之璽以勞四夷,天子信璽以召兵四夷,皆泥封。大朝會則符璽郎進神璽、受命璽於御座,行幸則合八璽為五輿,函封從於黃鉞之內。



 太皇太后、皇太后、皇后、皇太子及妃,璽皆金為之,藏而不用。太皇太后、皇太后封令書以宮官印,皇后以內侍省印,皇太子以左春坊印,妃以內坊印。



 初,太宗刻受命玄璽,以白玉為螭首,文曰:「皇天景命,有德者昌。」至武後,改諸璽皆為寶。中宗即位,復為璽。開元六年,復為寶。天寶初,改璽書為寶書。十載,改傳國寶為承天大寶。



 初,高祖入長安,罷隋竹使符,班銀菟符,其後改為銅魚符,以起軍旅、易守長,京都留守、折沖府、捉兵鎮守之所及左右金吾、宮苑總監、牧監皆給之。畿內則左三右一,畿外則左五右一,左者進內,右者在外,用始第一,周而復始。宮殿門、城門,給交魚符、巡魚符。左廂、右廂給開門符、閉門符。亦左符進內,右符監門掌之。蕃國亦給之,雄雌各十二,銘以國名,雄者進內,雌者付其國。朝貢使各齎其月魚而至,不合者劾奏。



 傳信符者,以給垂阜驛,通制命。皇太子監國給雙龍符,左右皆十。兩京、北都留守給麟符,左二十,右十九。東方諸州給青龍符,南方諸州硃雀符,西方諸州騶虞符,北方諸州玄武符,皆左四右三。左者進內,右者付外。行軍所亦給之。



 隨身魚符者,以明貴賤,應召命,左二右一,左者進內,右者隨身。皇太子以玉契召,勘合乃赴。親王以金,庶官以銅,皆題其位、姓名。官有貳者加左右,皆盛以魚袋,三品以上飾以金,五品以上飾以銀。刻姓名者,去官納之,不刻者傳佩相付。



 有傳符、銅魚符者,給封符印,發驛、封符及封魚函用之。有銅魚而無傳符者,給封函,還符、封函用之。



 天子巡幸,則京師、東都留守給留守印,諸司從行者,給行從印。



 木契符者,以重鎮守、慎出納,畿內左右皆三,畿外左右皆五。皇帝巡幸,太子監國,有軍旅之事則用之,王公征討皆給焉,左右各十九。太極殿前刻漏所,亦以左契給之,右以授承天門監門,晝夜勘合,然後鳴鼓。玄武門苑內諸門有喚人木契,左以進內,右以授監門,有敕召者用之。魚契所降,皆有敕書。尚書省符,與左同乃用。



 大將出,賜旌以顓賞,節以顓殺。旌以絳帛五丈,粉畫虎,有銅龍一,首纏緋幡,紫縑為袋,油囊為表。節,縣畫木盤三,相去數寸,隅垂赤麻,餘與旌同。



 高宗給五品以上隨身魚銀袋,以防召命之詐,出內必合之。三品以上金飾袋。垂拱中,都督、刺史始賜魚。天授二年,改佩魚皆為龜。其後,三品以上龜袋飾以金,四品以銀,五品以銅。中宗初,罷龜袋,復給以魚。郡王、嗣王亦佩金魚袋。景龍中,令特進佩魚,散官佩魚自此始也。然員外、試、檢校官,猶不佩魚。景雲中,詔衣紫者魚袋以金飾之,衣緋者以銀飾之。開元初,附馬都尉從五品者假紫、金魚袋,都督、刺史品卑者假緋、魚袋,五品以上檢校、試、判官皆佩魚。中書令張嘉貞奏,致仕者佩魚終身,自是百官賞緋、紫,必兼魚袋,謂之章服。當時服硃紫、佩魚者眾矣。



 初,隋文帝聽朝之服,以赭黃文綾袍,烏紗帽,折上巾,六合鞾,與貴臣通服。唯天子之帶有十三鐶,文官又有平頭小樣巾,百官常服同於庶人。



 至唐高祖,以赭黃袍、巾帶為常服。腰帶者,搢垂頭以下,名曰金宅尾,取順下之義。一品、二品銙以金,六品以上以犀,九品以上以銀,庶人以鐵。既而天子袍衫稍用赤、黃,遂禁臣民服。親王及三品、二王後,服大科綾羅,色用紫,飾以玉。五品以上服小科綾羅,色用硃,飾以金。六品以上服絲布交梭雙紃綾,色用黃。六品、七品服用綠,飾以銀。八品、九品服用青,飾以鍮石。勛官之服,隨其品而加佩刀、礪、紛帨。流外官、庶人、部曲、奴婢,則服紬絹施布,色用黃白,飾以鐵、銅。



 太宗時,又命七品服龜甲雙巨十花綾,色用綠。九品服絲布雜綾,色用青。是時士人以棠苧襴衫為上服,貴女功之始也。一命以黃,再命以黑,三命以纁,四命以綠,五命以紫。士服短褐,庶人以白。中書令馬周上議:「《禮》無服衫之文,三代之制有深衣。請加襴、袖、褾、襈,為士人上服。開骻者名曰缺骻衫,庶人服之。」又請:「裹頭者,左右各三襵,以象三才,重系前腳,以象二儀。」詔皆從之。太尉長孫無忌又議:「服袍者下加襴,緋、紫、綠皆視其品,庶人以白。」



 太宗嘗以襆頭起於後周,便武事者也。方天下偃兵,採古制為翼善冠,自服之。又制進德冠以賜貴臣,玉綦,制如弁服,以金飾梁,花趺,三品以上加金絡,五品以上附山雲。自是元日、冬至、朔、望視朝,服翼善冠,衣白練裙襦。常服則有褲褶與平巾幘,通用翼善冠。進德冠制如襆頭,皇太子乘馬則服進德冠,九綦,加金飾,犀簪導,亦有褲褶,燕服用紫。其後朔、望視朝,仍用弁服。



 顯慶元年,長孫無忌等曰:「武德初,撰《衣服令》,天子祀天地服大裘冕。按周郊被袞以象天。戴冕藻十有二旒,與大裘異。《月令》:孟冬,天子始裘以禦寒。若啟蟄祈穀,冬至報天,服裘可也。季夏迎氣,龍見而雩,如之何可服?故歷代唯服袞章。漢明帝始採《周官》、《禮記》制祀天地之服,天子備十二章,後魏、周、隋皆如之。伏請郊祀天地服袞冕,罷大裘。又新禮,皇帝祭社稷服絺冕,四旒,三章;祭日月服玄冕,三旒,衣無章。按令文,四品、五品之服也。三公亞獻皆服袞,孤卿服毳、,是天子同於大夫,君少臣多,非禮之中。且天子十二為節以法天,烏有四旒三章之服?若諸臣助祭,冕與王同,是貴賤無分也。若降王一等,則王服玄冕,群臣服爵弁,既屈天子,又貶公卿。《周禮》此文,久不用矣,猶祭祀之有尸侑,以君親而拜臣子,硩蔟、蟈氏之職,不通行者蓋多,故漢魏承用袞冕。今新禮,親祭日月,服五品之服,請循歷代故事,諸祭皆用袞冕。」制曰:「可。」無忌等又曰:「禮,皇帝為諸臣及五服親舉哀,素服,今服白袷,禮令乖舛。且白袷出近代,不可用。」乃改以素服。自是鷩冕以下,天子不復用,而白袷廢矣。其後以紫為三品之服,金玉帶銙十三;緋為四品之服,金帶銙十一;淺緋為五品之服,金帶銙十;深綠為六品之服,淺綠為七品之服,皆銀帶銙九;深青為八品之服,淺青為九品之服,皆鍮石帶銙八;黃為流外官及庶人之服,銅鐵帶銙七。



 武后擅政,多賜群臣巾子、繡袍,勒以回文之銘,皆無法度,不足紀。至中宗,又賜百官英王踣樣巾,其制高而踣,帝在籓時冠也。其後文官以紫黑施為巾。賜供奉官及諸司長官,則有羅巾、圓頭巾子,後遂不改。



 初,職事官三品以上賜金裝刀、礪石,一品以下則有手巾、算袋、佩刀、礪石。至睿宗時,罷佩刀、礪石,而武官五品以上佩韘七事,佩刀、刀子、蠣石、契苾真、噦厥針筒、火石是也。



 時皇太子將釋奠,有司草儀注,從臣皆乘馬著衣冠,左庶子劉子玄議曰:「古大夫乘車,以馬為騑服,魏、晉朝士駕牛車。如李廣北征,解鞍憩息;馬援南伐,據鞍顧眄。則鞍馬行於軍旅,戎服所便。江左尚書郎乘馬,則御史治之。顏延年罷官,騎馬出入,世稱放誕。近古專車則衣朝服,單馬則衣褻服。皇家巡謁陵廟,冊命王公,則盛服冠履,乘路車。士庶有以衣冠親迎者,亦時服箱。其餘貴賤,皆以騎代車。比者,法駕所幸,侍臣朝服乘馬。今既舍車,而冠履不易,何者?褒衣、博帶、革履、高冠,車中之服也。韈而鐙,跣而乘,非唯盩古,亦自取驚蹶。議者以秘閣梁《南郊圖》,有衣冠乘馬者,此圖後人所為也。古今圖畫多矣,如畫群公祖二疏,而有曳芒屩者;畫昭君入匈奴,而婦人有施帷冒者。夫芒屩出於水鄉,非京華所有;帷冒創於隋代,非漢宮所用。豈可因二畫以為故實乎?謂乘馬衣冠宜省。」太子從之,編於令。



 開元初,將有事南郊,中書令張說請遵古制用大裘,乃命有司制二冕。玄宗以大裘樸略,不可通寒暑,廢而不服。自是元正朝會用袞冕、通天冠,百官朔、望朝參,外官衙日,則佩算袋,餘日則否。玄宗謁五陵,初用素服,朔、望朝顓用常服。弁服、翼善冠皆廢。



 唐初,賞硃紫者服於軍中,其後軍將亦賞以假緋紫,有從戎缺骻之服,不在軍者服長袍,或無官而冒衣綠。有詔殿中侍御史糾察。諸衛大將軍、中郎將以下給袍者,皆易其繡文:千牛衛以瑞牛,左右衛以瑞馬,驍衛以虎,武衛以鷹,威衛以豹,領軍衛以白澤,金吾衛以闢邪。行六品者,冠去綦珠,五品去鞶囊、雙佩,襆頭用羅縠。



 婦人服從夫、子,五等以上親及五品以上母、妻,服紫衣,腰襻褾緣用錦繡。九品以上母、妻,服硃衣。流外及庶人不服綾、羅、縠、五色線鞾、履。凡襉色衣不過十二破,渾色衣不過六破。



 二十五年,御史大夫李適之建議:「冬至、元日大禮,朝參官及六品清官服硃衣,六品以下通服褲褶。」天寶中,御史中丞吉溫建議:「京官朔、望朝參,衣硃褲褶,五品以上有珂傘。」德宗嘗賜節度使時服,以雕銜綬帶,謂其行列有序,牧人有威儀也。元和十二年,太子少師鄭餘慶言:「百官服朝服者多誤。自今唯職事官五品兼六品以上散官者,則有佩、劍、綬,其餘皆省。」



 初,婦人施冪釭以蔽身,永徽中,始用帷冒,施裙及頸,坐簷以代乘車。命婦朝謁,則以駝駕車。數下詔禁而不止。武后時,帷冒益盛,中宗後乃無復冪釭矣。宮人從駕,皆胡冒乘馬,海內佼又之,至露髻馳騁,而帷冒亦廢,有衣男子衣而鞾,如奚、契丹之服。武德間,婦人曳履及線鞾。開元中,初有線鞋,侍兒則著履,奴婢服襴衫,而士女衣胡服,其後安祿山反,當時以為服妖之應。



 巴、蜀婦人出入有兜籠,乾元初,蕃將又以兜籠易負,遂以代車。


文宗即位,以四方車服僭奢,下詔準儀制令,品秩勛勞為等級。職事官服綠、青、碧,勛官諸司則佩刀、礪、紛、帨。諸親朝賀宴會之服:一品、二品服玉及通犀,三品服花犀、班犀。車馬無飾金銀。衣曳地不過二寸,袖不過一尺三寸。婦人裙不過五幅,曳地不過三寸,襦袖不過一尺五寸。袍襖之制:三品以上服綾,以鶻銜瑞草,雁銜綬帶及雙孔雀;四品、五品服綾,以地黃交枝;六品以下服綾,小窠無文及隔織、獨織。一品導從以七騎;二品、三品以五騎;四品以三騎;五品以二騎;六品以一騎。五品以上及節度使冊拜、婚會,則車有幰。外命婦一品、二品、三品乘金銅飾犢車,簷舁以八人,三品舁以六人;四品、五品乘白銅飾犢車,簷舁以四人;胥吏、商賈之妻老者乘葦
 \gezhu{
  厶大}
 車,兜籠舁以二人。度支、戶部,鹽鐵門官等服細葛布,無紋綾,綠暗銀藍鐵帶,鞍、轡、銜、鐙以鍮石。未有官者,服粗葛布、官施,綠銅鐵帶,乘蜀馬、鐵鐙。行官服紫粗布、施,藍鐵帶。中官不衣紗縠綾羅,諸司小兒不服大巾,商賈、庶人、僧、道士不乘馬。婦人衣青碧纈、平頭小花草履、彩帛縵成履,而禁高髻、險妝、去眉、開額及吳越高頭草履。王公之居,不施重栱、藻井。三品堂五間九架;門三間五架,五品堂五間七架,門三間兩架;六品、七品堂三間五架,庶人四架,而門皆一間兩架。常參官施懸魚、對鳳、瓦獸、通栿乳梁。詔下,人多怨者。京兆尹杜悰條易行者為寬限,而事遂不行。唯淮南觀察使李德裕令管內婦人衣袖四尺者闊一尺五寸,裙曳地四五寸者減三寸。



 開成末,定制:宰相、三公、師保、尚書令、僕射、諸司長官及致仕官,疾病許乘簷,如漢、魏載輿、步輿之制,三品以上官及刺史,有疾暫乘,不得舍驛。



\end{pinyinscope}