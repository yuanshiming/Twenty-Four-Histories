\article{列傳第一百四十上 突厥上}

\begin{pinyinscope}

 夷狄為中國患,尚矣。在前世者,史家類能言之。唐興,蠻夷更盛衰,嘗與中國亢衡者有四:突厥、吐蕃、回鶻、雲南是也。方其時明社會整體和個人發展的統一性。,群臣獻議盈廷,或聽或置,班然可睹也。



 劉貺以為:



 嚴尤辯而未詳,班固詳而未盡,榷其至當,周得上策,秦得其中,漢無策。何以言之?荒服之外,聲教所不逮,其叛不為之勞師,其降不為之釋備,嚴守御,險走集,使其為寇不能也,為臣不得也。「惠此中夏,以綏四方」,周之道也,故曰周得上策。《易》稱:「王侯設險以固其國。」築長城,脩障塞,所以設險也。趙簡子起長城備胡,燕、秦亦築長城限中外,益理城塹,城全國滅,人歸咎焉。後魏築長城,議者以為人治一步,方千里,役三十萬人,不旬朔而獲久逸,故曰秦得中策。漢以宗女嫁匈奴,而高祖亦審魯元不能止趙王之逆謀,謂能息匈奴之叛,非也。且冒頓手弒其親,而冀其不與外祖爭強,豈不惑哉?然則知和親非久安計而為之者,以天下初定,紓歲月之禍耳。武帝時,中國艾安,胡寇益希,疏而絕之,此其時也。方更糜耗華夏,連兵積年,故嚴尤以為下策。然而漢至昭、宣,武士練習,斥候精明,匈奴收跡遠徙,猶襲奉春之過舉,傾府藏給西北,歲二億七十萬。皇室淑女,嬪於穹廬;掖庭良人,降於沙漠。夫貢子女方物,臣僕之職也。《詩》曰:「莫敢不來享,莫敢不來王。」荒服稱其來,不言往也。公及吳盟,諱而不書。奈何以天子之尊,與匈奴約為兄弟,帝女之號,與胡媼並御;蒸母報子,從其污俗?中國異於蠻夷者,有父子男女之別也。婉冶之姿,毀節異類,垢辱甚矣。漢之君臣,莫之恥也。魏、晉羌狄居塞垣,資奉逾昔。百人之酋,千口之長,賜金印紫綬,食王侯之俸。牧馬之童,乘羊之隸,齎毳毼邀利者,相錯於路。耒耨之利,絲枲所生,散於數萬里之外。胡夷歲驕,華夏日蹙。方其強也,竭人力以征之;其服也,養之如初。病則受養,強則內攻,中國為羌胡服役且千載,可不悲哉!誠能移其財以賞戍卒,則民富;移其爵以餌守臣,則將良。富利歸於我,危亡移於彼,無納女之辱,無傳送之勞。棄此而不為,故曰漢無策。嚴尤謂古無上策,謂不能臣妾之也,誠能之而不用耳。秦無策,謂攘狄而亡國也。秦亡,非攘狄也。漢得下策,謂伐胡而人病。人既病矣,又役人而奉之,無策也。故曰嚴尤辯而未詳也。班固謂「其來慕義,則接以禮讓。」何者?禮讓以交君子,非所以接禽獸夷狄也。纖麗外散,則戎羯之心生;戎羯之心生,則侵盜之本也。聖人飲食聲樂不與之共,來朝坐於門外,舌人體委以食之,不使知馨香嘉味也。漢氏習玩驕虜,使其悅燕、趙之色,甘太官之珍,服以文綺羅紈,供之則增求,絕之則招怨,是飽豺狼以良肉,而縱其獵噬也。華人步卒利險阻,虜人騎兵利平地,堅守無與追奔競逐,來則杜險使不得進,去則閉險使不得還,沖以長戟,臨以強弩,非求勝也,譬諸蟲豸虺蜴,何禮讓之接哉?故曰班固詳而未盡者,此也。



 杜佑謂:



 秦以區區關中滅六強國,今竭萬方之財,上奉京師,外有犬戎恁陵,陷城數百,內有兵革未寧,三紀矣。豈制置異術,古今殊時乎?周制,步百為畝,畝百給一夫。商鞅佐秦,以為地利不盡,更以二百四十步為畝,百畝給一夫。又以秦地曠而人寡,晉地狹而人伙,誘三晉之人耕而優其田宅,復及子孫,使秦人應敵於外,非農與戰不得入官。大率百人以五十人為農,五十人習戰,故兵強國富。其後仕宦途多,末業日滋。今大率百人才十人為農,餘皆習佗技。又秦、漢鄭渠溉田四萬頃,白渠溉田四千五百頃,永徽中,兩渠灌浸不過萬頃,大歷初,減至六千畝。畝晙一斛,歲少四五百萬斛。地利耗,人力散,欲求強富,不可得也。漢時,長安北七百里即匈奴之地,侵掠未嘗暫息。計其舉國之眾,不過漢一大郡,鼉錯請備障塞,故北邊妥安。今潼關之西,隴山之東,鄜坊之南,終南之北,十餘州之地,已數十萬家。吐蕃綿力薄材,食鮮藝拙,不及中國遠甚,誠能復兩渠之饒,誘農夫趣耕,擇險要,繕城壘,屯田蓄力,河、隴可復,豈唯自守而已。



 至佑孫牧亦曰:



 天下無事時,大臣偷處榮逸,戰士離落,兵甲鈍弊,車馬刓弱,天下雜然盜發,則疾驅以戰,是謂宿敗之師。此不搜練之過,其敗一也。百人荷戈,仰食縣官,則挾千夫之名,大將小裨操其餘贏,以虜壯為幸,執兵者常少,糜食者常多,築壘未乾,公囊已虛。此不責實之過,其敗二也。戰小勝則張皇其功,奔走獻狀以邀賞,或一日再賜,一月累封,凱還未歌,書品已崇,爵命極矣,田宮廣矣,金繒溢矣,子孫官矣,肯外死勤於我哉?此賞厚之過,其敗三也。多喪兵士,顛翻大都,則跳身而來,刺邦而去,回視刀鋸、菜色甚安,一歲未更,已立於壇墀之上。此輕罰之過,其敗四也。大將將兵,柄不得專,一曰為偃月,一曰為魚麗,三軍萬夫,環旋翔佯,愰駭之間,虜騎乘之。此不專任之過,其敗五也。元和時,團兵數十萬以誅蔡,天下乾耗,四歲然後能取之,蓋五敗不去也。長慶初,盜子若孫悉來走命,未幾而燕、趙亂,引師起將,五敗益甚,不能加威於反虜。二杜之論如此。



 廣德、建中間,吐蕃再飲馬岷江,常以南詔為前鋒,操倍尋之戟,且戰且進,蜀兵折刃吞鏃,不能斃一戎。戎兵日深,疫死日眾,自度不能留,輒引去。蜀人語曰:「西戎尚可,南蠻殘我。」至韋皋鑿青溪道以和群蠻,使道蜀入貢,擇子弟習書算於成都,業成而去,習知山川要害。文宗時,大入成都,自越巂以北八百里,民畜為空,又敗卒貧民因緣掠殺,官不能禁。自是群蠻常有屠蜀之心,蜀民苦於重征者,亦欲啟之以幸非常。歲發戍卒,不習山川之險,緩步一舍,已呵然流汗。為將者刻薄自入,給帛則以疏易良,賦粟以沙參粒,故邊卒怨望而巴、蜀危憂。孫樵謂:「宜詔嚴道、沈黎、越巂三州,度要害,募卒以守。且兵籍於州則易役,卒出於邊則習險,相地分屯,春耕夏蠶以資衣食,秋冬嚴壁以俟寇。歲遣廉吏視卒之有無,則官無饋運,吏無牟盜。」此其備御之策可施行者,著之於篇。



 凡突厥、吐蕃、回鶻以盛衰先後為次;東夷、西域又次之,跡用兵之輕重也;終之以南蠻,記唐所繇亡雲。



 突厥阿史那氏,蓋古匈奴北部也。居金山之陽,臣於蠕蠕,種裔繁衍。至吐門,遂強大,更號可汗,猶單于也,妻曰可敦。其地三垂薄海,南抵大漠。其別部典兵者曰設,子弟曰特勒,大臣曰葉護,曰屈律啜、曰阿波、曰俟利發、曰吐屯、曰俟斤、曰閻洪達、曰頡利發、曰達干,凡二十八等,皆世其官而無員限。衛士曰附離。可汗建廷都斤山,牙門樹金狼頭纛,坐常東向。



 隋大業之亂,始畢可汗咄吉嗣立,華人多往依之,契丹、室韋、吐谷渾、高昌皆役屬,竇建德、薛舉、劉武周、梁師都、李軌、王世充等倔起虎視,悉臣尊之。控弦且百萬,戎狄熾強,古未有也。高祖起太原,遣府司馬劉文靜往聘,與連和,始畢使特勒康稍利獻馬二千、兵五百來會。帝平京師,遂恃功,使者每來多橫驕。武德元年,骨咄祿特勒來朝,帝宴太極殿,為奏九部樂,引升御坐。是歲,始畢牙帳自破,帝問內史令蕭瑀,瑀曰:「魏文帝幸許,城門無故壞,是年文帝崩,豈其類耶?」二年,始畢自將度河,至夏州,與賊梁師都合,又佐劉武周以五百騎入句注,將侵太原。會病死,帝為發哀長樂門,詔群臣即館吊其使,遣使者持段物三萬賻之。子什缽苾幼,不克立,以為泥步設,使居東偏,立其弟俟利弗設,是為處羅可汗。



 處羅復妻隋義成公主,遣使來告,則又潛通王世充,潞州總管李襲譽擊斬其使,取牛羊萬餘。處羅迎隋蕭皇后及齊王暕之子正道於竇建德所,因立正道為隋王,奉隋後,隋人沒者隸之,行其正朔,置百官,居定襄,眾萬人。秦王討武周也,處羅以弟步利設騎二千會並州三日,多掠城中婦人女子去,總管李仲文不能制,以俱儉特勒助屯。明年,謀取並州置楊正道,卜之,不吉,左右諫止,處羅曰:「我先人失國,賴隋以存,今忘之,不祥。卜不吉,神詎無知乎?我自決之。」會天雨血三日,國中犬夜群號,求之不見,遂有疾,公主餌以五石,俄疽發死。主以子奧射設陋弱,棄不立,更取其弟咄苾嗣,是為頡利可汗。



 頡利始為莫賀咄設,牙直五原北。薛舉陷平涼,與連和,帝患之,遣光祿卿宇文歆賂頡利,使與舉絕;隋五原太守張長遜以所部五城附虜,歆並說還五原地。皆見聽,且發兵舉長遜所部會秦王軍。太子建成議廢豐州,並割榆中地。於是處羅子鬱射設以所部萬帳入處河南,以靈州為塞。



 頡利又妻義成,以始畢子什缽苾為突利可汗,使居東。義成,楊諧女也,其弟善經亦依突厥,與王世充使者王文素共說頡利曰:「往啟民兄弟爭國,賴隋得復位,子孫有國。今天子非文帝後,宜立正道以報隋厚德。」頡利然之,故歲入寇。然倚父兄餘資,兵銳馬多,〓然驕氣,直出百蠻上,視中國為不足與,書辭悖嫚,多須求。帝方經略天下,故屈禮,多所舍貸,贈齎不貲,然而不厭無厓之求也。



 四年,頡利率萬騎與苑君璋合寇雁門,定襄王李大恩擊卻之。頡利執我使者漢陽公瑰、太常卿鄭元、左驍衛大將軍長孫順德,帝亦囚其使與相當。由是寇代州,敗行軍總管王孝基,略河東,犯原州,穿延州塞,諸將與戰,不能有所俘。



 明年,還順德等,且請和,贄魚膠,紿云:「固二國之好也。」帝雖未情,釋其使特勒熱寒等,厚與金還之。大恩上言:「突厥饑,馬邑可圖也。」詔殿中少監獨孤晟共擊之。晟後約,大恩不敢進,屯新城,頡利自將數萬騎與劉黑闥合圍之,大恩沒,士死者數千人。進擊忻州,為李高遷所破。黑闥以突厥萬人擾山東,又殘定州。頡利未得志,乃率十五萬騎入雁門,圍並州,深鈔汾、潞,取男女五千,分數千騎轉掠原、靈間。於是太子建成將兵出豳州道,秦王將兵出蒲州道擊之;李子和以兵趨雲中,掩可汗後;段德操出夏州,狙其歸。並州總管襄邑王神符戰汾東,斬虜五百首,取馬二千;汾州刺史蕭顗獻俘五千。虜陷大震關,縱兵掠弘州,總管宇文歆、靈州楊師道拒之,獲馬、橐它數千。頡利聞秦王且至,引出塞,王師還。又明年,與黑闥、君璋等小小入寇定、匡、原、朔等州,與屯將相勝負。帝遣太子建成復屯北邊、秦王屯並州備虜,久乃罷。俄又破代地一屯,進擊渭、豳二州,取馬邑,不有也,復請和,歸我馬邑。



 七年,攻原、朔二州,入代地,不勝,更與君璋合攻隴州及陰般城,分擊並地,秦王與齊王元吉屯豳州道以備胡。君璋與虜出入原、朔、忻、並地,剽系騷然,數為諸將驅逐。其八月,頡利與突利兵悉起,自原州連營而南,所在震恐,秦王、齊王拒之。



 初,關中霖潦,餉道絕,軍次豳州,可汗萬騎奄至,陣五龍阪,以數百騎挑戰,舉軍失色。秦王馳百騎掠陣,大言曰:「國家於突厥無負,何為深入?我,秦王也,故來自與可汗決,若固戰,我才百騎耳,徒廣殺傷,無益也。」頡利笑不答。又馳騎語突利曰:「爾往與我盟,急難相助,今無香火情邪?能一決乎?」突利亦不對。王將絕水前,頡利見兵少,又聞與突利語,陰相忌,即遣使者來曰:「王毋苦,我固不戰,將與王議事耳。」於是引卻。秦王縱反間,突利乃歸心,不欲戰,頡利亦無以強之,乃遣突利及夾畢特勒思摩請和,帝許之。突利遂自托於王為昆弟。帝見思摩,引升御榻,思摩頓首辭,帝曰:「我見若猶頡利也。」乃聽命。



 突厥既歲盜邊,或說帝曰:「虜數內寇者,以府庫子女所在,我能去長安,則戎心止矣。」帝使中書侍郎宇文士及逾南山,按行樊、鄧,將徙都焉。群臣贊遷,秦王獨曰:「夷狄自古為中國患,未聞周、漢為遷也。願假數年,請取可汗以報。」帝乃止。頡利已和,亦會甚雨,弓矢皆弛惡,遂解而還。帝會群臣問所以備邊者,將作大匠於筠請五原、靈武置舟師於河,扼其入。中書侍郎溫彥博曰:「魏為長塹遏匈奴,今可用。」帝使桑顯和塹邊大道,召江南船工大發卒治戰艦。頡利遣使來,願款北樓關請互市,帝不能拒。帝始兼天下,罷十二軍,尚文治,至是以虜患方張,乃復置之,以練卒搜騎。



 八年,頡利攻靈、朔,與代州都督藺〓戰新城,〓敗績。於是張瑾兵屯石嶺,李高遷屯大谷,秦王屯蒲州道。初,帝待突厥用敵國禮,及是,怒曰:「往吾以天下未定,厚於虜以紓吾邊。今卒敗約,朕將擊滅之,毋須姑息。」命有司更所與書為詔若敕。瑾未至屯,虜已逾石嶺,圍並州,攻靈州,轉擾潞、沁。李靖以兵出潞州道,行軍總管任瑰屯太行。瑾戰大谷,敗績,中書侍郎溫彥博陷於賊,鄆州都督張德政死之。遂攻廣武,為任城王道宗破。其欲谷設掠綏州,請和去。敗並州數縣,入蘭、鄯、彭州諸屯,或小勝,不能制。俄寇原州,折威將軍楊屯擊之,且發士屯大谷。



 九年,攻原、靈,又圍涼州,進犯涇、原,李靖與戰靈州,虜引去。寇西會州,圍烏城,翔徉隴、渭間,平道將軍柴紹破之於秦州,斬一特勒、三大將,虜千級。大抵虜得志則深入,負則請和,不恥也。其七月,頡利自將十萬騎襲武功,京師戒嚴。攻高陵,尉遲敬德與戰涇陽,獲俟斤烏沒啜,斬首千餘級。頡利遣謀臣執失思力入朝以覘我,因誇說曰:「二可汗兵百萬,今至矣!」太宗曰:「我與可汗嘗面約和,爾則背之。且義師之初,爾父子身從我,遺賜玉帛多至不可計,何妄以兵入我都畿,自誇盛強耶?今我當先戮爾矣!」思力懼,請命,蕭瑀、封德彞諫帝,不如禮遣之,帝不許,系於門下省。乃與侍中高士廉、中書令房玄齡、將軍周範等馳六騎出玄武門,幸渭上,與可汗隔水語,且責其負約。群酋見帝,皆驚,下馬拜。俄而眾軍至,旗鎧光明,部隊靜嚴,虜大駭。帝與頡利按轡,即麾軍卻而陣焉。蕭瑀以帝輕敵,叩馬諫,帝曰:「我思熟矣,非爾所知也。夫突厥掃地入寇,以我新有內難,謂不能師。我若闔城,彼且大掠吾境,故我獨出,示無所畏,又盛兵使知必戰,不意我能沮其始謀。彼入吾地既深,懼不能返,故與戰則克,和則固,制賊之命,在此舉矣!」是日,頡利果請和,許之。翌日,刑白馬,與頡利盟便橋上,突厥引還。蕭瑀曰:「頡利之來,諸將多請與戰,陛下不聽,既而虜自退,其策奈何?」帝曰:「突厥眾而不整,君臣惟利是視,可汗在水西,而酋帥皆來謁我,我醉而縛之,其勢易甚。又我敕長孫無忌、李靖潛師幽州以須,若大軍躡其後,伏邀諸前,取之反覆掌耳。然我新即位,為國者要在安靜,一與虜校,殺傷必多,彼敗未及亡,懼而脩德,與我為怨,其可當耶?今僕械卷鎧,啖以玉帛,虜志必驕,驕則亡之端也,故曰『將欲取之,必固與之』。瑀再拜曰:「非臣愚所逮也!」乃詔殿中監豆盧寬、將軍趙綽護送突厥,頡利獻馬三千匹、羊萬頭,帝不納,詔歸所俘於我。



 貞觀元年,薛延陀、回紇、拔野古諸部皆叛,使突利討之,不勝,輕騎走,頡利怒,囚之,突利由是怨望。是歲大雪,羊馬多凍死,人饑,懼王師乘其敝,即引兵入朔州地,聲言會獵。議者請責其敗約,因伐之,帝曰:「匹夫不可為不信,況國乎?我既與之盟,豈利其災,邀險以取之耶?須其無禮於我,乃伐之。」



 明年,突利自陳為頡利所攻,求救。帝曰:「朕與頡利盟,又與突利有昆弟約,不可不救,奈何?」兵部尚書杜如晦曰:「夷狄無信,我雖如約,彼常負之,今亂而擊之,侮亡之道也。」乃詔將軍周範壁太原經略之,頡利亦擁兵窺邊。或請築古長城,發民乘塞。帝曰:「突厥盛夏而霜,五日並出,三月連明,赤氣滿野,彼見災而不務德,不畏天也。遷徙無常,六畜多死,不用地也。俗死則焚,今葬皆起墓,背父祖命,謾鬼神也。與突利不睦,內相攻殘,不和於親也。有是四者,將亡矣,當為公等取之,安在築障塞乎?」突厥俗素質略,頡利得華士趙德言,才其人,委信之,稍專國;又委政諸胡,斥遠宗族不用,興師歲入邊,下不堪苦。胡性冒沓,數翻覆不信,號令無常。歲大饑,裒斂苛重,諸部愈貳。



 又明年,屬部薛延陀自稱可汗,以使來。詔兵部尚書李靖擊虜馬邑,頡利走,九俟斤以眾降,拔野古、僕骨、同羅諸部、習奚渠長皆來朝。於是詔並州都督李世勣出通漠道,李靖出定襄道,左武衛大將軍柴紹出金河道,靈州大都督任城王道宗出大同道,幽州都督衛孝節出恆安道,營州都督薛萬淑出暢武道,凡六總管,師十餘萬,皆授靖節度以討之。道宗戰靈州,俘人畜萬計,突利及鬱射設、廕奈特勒帥所部來奔,捷書日夜至,帝謂群臣曰:「往國家初定,太上皇以百姓故,奉突厥,詭而臣之,朕常痛心病首,思一刷恥於天下,今天誘諸將,所向輒克,朕其遂有成功乎!」



 四年正月,靖進屯惡陽嶺,夜襲頡利,頡利驚,退牙磧口,大酋康蘇蜜等以隋蕭皇后、楊正道降。或言中國人嘗密通書於後,中書舍人陽文瓘請劾治。帝曰:「天下未一,人或當思隋,今反側既安,何足治耶?」置勿劾。頡利窘,走保鐵山,兵猶數萬,令執失思力來,陽為哀言謝罪,請內屬,帝詔鴻臚卿唐儉、將軍安脩仁等持節慰撫。靖知儉在虜所,虜必安,乃襲擊之,盡獲其眾,頡利得千里馬,獨奔沙缽羅,行軍副總管張寶相禽之。沙缽羅設、蘇尼失以眾降,其國遂亡,復定襄、恆安地,斥境至大漠矣。



 頡利至京師,告俘太廟,帝御順天樓,陳仗衛,士民縱觀,吏執可汗至,帝曰:「而罪有五:而父國破,賴隋以安,不以一鏃力助之,使其廟社不血食,一也;與我鄰而棄信擾邊,二也;恃兵不戢,部落攜怨,三也;賊華民,暴禾稼,四也;許和親而遷延自遁,五也。朕殺爾非無名,顧渭上盟未之忘,故不窮責也。」乃悉還其家屬,館於太僕,稟食之。



 思結俟斤以四萬眾降,可汗弟欲谷設奔高昌,既而亦來降。伊吾城之長素臣突厥,舉七城以獻,因其地為西伊州。制詔:突厥往逢癘疫,長城之南,暴骨如丘,有司其以酒脯祭,為瘞藏之。又詔:隋亂,華民多沒於虜,遣使者以金帛贖男女八萬口,還為平民。



 頡利不室處,常設穹廬廷中,久鬱鬱不自憀,與家人悲歌相泣下,狀貌羸省。帝見憐之,以虢州負山多麕麋,有射獵之娛,乃拜為刺史,辭不往,遂授右衛大將軍,賜美田宅。帝曰:「昔啟民失國,隋文帝不■粟帛,興士眾,營護而存立之,至始畢稍強,則以兵圍煬帝雁門,今其滅者,殆背德忘義致然耶?」頡利子疊羅支,有至性,既舍京師,諸婦得品供,羅支預焉;其母最後至,不得給,羅支不敢嘗品肉。帝聞,嘆曰:「天稟仁孝,詎限華夷哉!」厚賜之,遂給母肉。



 八年,頡利死,贈歸義王,謚曰荒,詔國人葬之,從其禮,火尸,起塚灞東。其臣胡祿達官吐谷渾邪者,頡利母婆施之媵臣也,頡利始生,以授渾邪,至是哀慟,乃自殺。帝異之,贈中郎將,命葬頡利塚旁,詔中書侍郎岑文本刻其事於頡利、渾邪之墓碑。俄蘇尼失亦以死殉。尼失者,啟民可汗弟也。始畢以為沙缽羅設,帳部五萬,牙直靈州西北,姿雄趫,以仁惠御下,人多歸之;頡利政亂,其部獨不貳。突利降,頡利以為小可汗。頡利已敗,乃舉眾來,漠南地遂空,授北寧州都督、右衛大將軍,封懷德王云。



 頡利之亡,其下或走薛延陀,或入西域,而來降者尚十餘萬,詔議所宜,咸言:「突厥擾中國久,今天喪之,非慕義自歸,請悉籍降俘,內兗、豫閑處,使習耕織,百萬之虜,可化為齊人,是中國有加戶,而漠北遂空也。」中書令溫彥博請:「如漢建武時,置降匈奴留五原塞,全其部落,以為捍蔽,不革其俗,因而撫之,實空虛之地,且示無所猜。若內兗、豫,則乖本性,非函育之道。」秘書監魏徵建言:「突厥世為中國仇,今其來降,不即誅滅,當遣還河北。彼鳥獸野心,非我族類,弱則伏,強則叛,其天性也。且秦、漢以銳師猛將擊取河南地為郡縣者,以不欲使近中國也。陛下奈何以河南居之?且降者十萬,若令數年,孳息略倍,而近在畿甸,心腹疾也。」彥博曰:「不然,天子於四夷,若天地養萬物,覆載全安之,今突厥破滅,餘種歸命,不加哀憐而棄之,非天地蒙覆之義,而有阻四夷之嫌。臣謂處以河南,蓋死而生之,亡而存之,彼世將懷德,何叛之為?」徵曰:「魏時有胡落分處近郡,晉已平吳,郭欽、江統勸武帝逐出之,不能用。劉、石之亂,卒傾中夏。陛下必欲引突厥居河南,所謂養虎自遺患者也。」彥博曰:「聖人之道無不通,故曰『有教無類』。彼創殘之餘,以窮歸我,我援護之,收處內地,將教以禮法,職以耕農,又選酋良入宿衛,何患之恤?且光武置南單于,卒無叛亡。」於是中書侍郎顏師古、給事中杜楚客、禮部侍郎李百藥等皆勸帝不如使處河北,樹首長,俾統部落,視地多少,令不相臣,國小權分,終不得亢衡中國,長轡遠馭之道也。帝主彥博語,卒度朔方地,自幽州屬靈州,建順、祐、化、長四州為都督府,剖頡利故地,左置定襄都督、右置雲中都督二府統之。擢酋豪為將軍、郎將者五百人,奉朝請者且百員,入長安自籍者數千戶。乃以突利可汗為順州都督,令率其下就部。



 突利初為泥步設,得隋淮南公主以為妻。頡利之立,用次弟為延陀設,主延陀部,步利設主霫部,統特勒主胡部,斛特勒主斛薛部,以突利可汗主契丹、靺鞨部,樹牙南直幽州,東方之眾皆屬焉。突利斂取無法,下不附,故薛延陀、奚、霫等皆內屬,頡利遣擊之,又大敗,眾騷離,頡利囚捶之,久乃赦。突利嘗自結於太宗,及頡利衰,驟追兵於突利,不肯從,因起相攻。突利請入朝,帝謂左右曰:「古為國者勞己以憂人,則系祚長;役人以奉己,則亡。今突厥喪亂,由可汗不君,突利雖至親,不自保而來。夷狄弱則邊境安,然觀彼亡,我不可以無懼,有不逮者,禍可紓乎!」突利至,禮見良厚,輟膳以賜之,拜右衛大將軍,封北平郡王,食戶七百。及為都督,太宗敕曰:「而祖啟民破亡,隋則復之,棄德不報,而父始畢反為隋敵。爾今窮來歸我,所以不立爾為可汗,鑒前敗也。我欲中國安,爾宗族不亡,故授爾都督,毋相侵掠,長為我北籓。」突利頓首聽命。後入朝,死並州道中,年二十九,帝為舉哀,亦詔文本文其墓,子賀邏鶻嗣。



 帝幸九成宮,突利弟結社率以郎將宿衛,陰結種人謀反,劫賀邏鶻北還,謂其黨曰:「我聞晉王丁夜得闢仗出,我乘間突進,可犯行在。」是夕,大風冥,王不出,結社率恐謀漏,即射中營,噪而殺人,衛十等共擊之,乃走,殺廄人盜馬,欲度渭,徼邏禽斬之,赦賀邏鶻,投嶺外。於是群臣更言處突厥中國非是,帝亦患之,乃立阿史那思摩為乙彌泥孰俟利苾可汗,賜氏李,樹牙河北,悉徙突厥還故地。



 思摩,頡利族人也,父曰咄六設。始,啟民奔隋,磧北諸部奉思摩為可汗,啟民歸國,乃去可汗號。性開敏,善占對,始畢、處羅皆愛之。然以貌似胡,疑非阿史那種,故但為夾畢特勒,而不得為設。武德初,數以使者來,高祖嘉其誠,封和順郡王。及諸部納款,思摩獨留,與頡利俱禽,太宗以為忠,授右武候大將軍、化州都督,統頡利故部居河南,徙懷化郡王。及是將徙,內畏薛延陀,不敢出塞。帝詔司農卿郭嗣本持節賜延陀書,言:「中國禮義,未始滅人國,以頡利暴殘,伐而取之,非貪其地與人也。故處降部於河南,薦草美泉,利其畜牧,眾日孳蕃,今復以思摩為可汗,還其故疆。延陀受命在前,長於突厥,舉磧以北,延陀主之;其南,突厥保之。各守而境,無相鈔犯,有負約,我自以兵誅之。」思摩乃行,帝為置酒,引思摩前曰:「蒔一草一木,見其溺廡以為喜,況我養爾部人,息爾馬羊,不減昔乎!爾父母墳墓在河北,今復舊廷,故宴以慰行。」思摩泣下,奉觴上萬歲壽,且言:「破亡之餘,陛下使存骨舊鄉,願子孫世世事唐,以報厚德。」於是趙郡王孝恭、鴻臚卿劉善就思摩部,築壇場河上,拜受冊,賜鼓纛,又詔左屯衛將軍阿史那忠為左賢王,左武衛將軍阿史那泥孰為右賢王,相之。



 薛延陀聞突厥之北,恐其眾奔亡度磧,勒兵以待。及使者至,乃謝曰:「天子詔毋相侵,謹頓首奉詔。然突厥酣亂翻覆,其未亡時殺中國人如麻,陛下滅其國,謂宜收種落皆為奴婢,以償唐人。乃養之如子,而結社率竟反,此不可信明甚。後有亂,請終為陛下誅之。」十五年,思摩帥眾十餘萬、勝兵四萬、馬九萬匹始度河,牙於故定襄城,其地南大河,北白道,畜牧廣衍,龍荒之最壤,故突厥爭利之。思摩遣使謝曰:「蒙恩立為落長,實望世世為國一犬,守吠天子北門,有如延陀侵逼,願入保長城。」詔許之。



 居三年,不能得其眾,下多攜背,思摩慚,因入朝願留宿衛,更拜右武衛將軍。從伐遼,中流矢,帝為吮血,其顧厚類此。還,卒京師,贈兵部尚書、夏州都督,陪葬昭陵,築墳象白道山,為刊其勞,碑於化州。



 右賢王阿史那泥孰,蘇尼失子也。始歸國,妻以宗女,賜名忠。及從思摩出塞,思慕中國,見使者必流涕求入侍,許之。



 思摩既不能國,殘眾稍稍南度河,分處勝、夏二州。帝伐遼,或言突厥處河南,邇京師,請帝無東。帝曰:「夫為君者,豈有猜貳哉!湯、武化桀、紂之民,無不遷善,有隋無道,舉天下皆叛,非止夷狄也。朕閔突厥之亡,內河南以振贍之,彼不近走延陀而遠歸我,懷我深矣,朕策五十年中國無突厥患。」思摩眾既南,車鼻可汗乃盜有其地。



 車鼻,亦阿史那族,而突利部人也,名斛勃,世為小可汗。頡利敗,諸部欲共君長之,會薛延陀稱可汗,乃往歸焉。其為人沈果有智數,眾頗便附,延陀畏逼,將殺之,乃率所部遯去,騎數千尾追,不勝。竄金山之北,三垂斗絕,惟一面可容車騎,壤土夷博,即據之,勝兵三萬,自稱乙注車鼻可汗,距長安萬里,西葛邏祿,北結骨,皆並統之,時時出掠延陀人畜。延陀後衰,車鼻勢益張。



 二十一年,遣子沙缽羅特勒獻方物,且請身入朝。帝遣雲麾將軍安調遮、右屯衛郎將韓華往迎之,至則車鼻偃然無入朝意,華謀與葛邏祿共劫之,車鼻覺,華與車鼻子陟苾特勒鬥死,調遮被殺。帝怒,遣右驍衛郎將高偘發回紇、僕骨兵擊之,其大酋長歌邏祿泥孰闕俟利發、處木昆莫賀咄俟斤等以次降。偘師攻阿息山,部落不肯戰,車鼻攜愛妾,從數百騎走;追至金山,獲之,獻京師。高宗責曰:「頡利敗,爾不輔,無親也;延陀破,爾遯亡,不忠也。而罪當死,然朕見先帝所獲酋長必宥之,今原而死。」乃釋縛,數俘社廟,又見昭陵。拜左武衛將軍,賜居第,處其眾鬱督軍山,詔建狼山都督府統之。初,其子羯漫陀泣諫車鼻,請歸國,不聽。乃遣子庵鑠入朝,後來降,拜左屯衛將軍,建新黎州,使領其眾。於是突厥盡為封疆臣矣。始置單于都護府領狼山雲中桑乾三都督、蘇農等二十四州,瀚海都護府領金微新黎等七都督、仙萼賀蘭等八州。即擢領酋為都督、刺史。麟德初,改燕然為瀚海都護府,領回紇,徙故瀚海都護府於古雲中城,號雲中都護府,磧以北蕃州悉隸瀚海,南隸雲中。雲中者,義成公主所居也,頡利滅,李靖徙突厥羸破數百帳居之,以阿史德為之長,眾稍盛,即建言願以諸王為可汗,遙統之。帝曰:「今可汗,古單于也。」乃改雲中府為單于大都護府,以殷王旭輪為單于都護。帝封禪,都督葛邏祿叱利等三十餘人皆從至泰山下,已封,詔勒名於封禪碑云。凡三十年北方無戎馬警。



 調露初,單于府大酋溫傅、奉職二部反,立阿史那泥孰匐為可汗,二十四州酋長皆叛應之。乃以鴻臚卿單于大都護府長史蕭嗣業、左領軍衛將軍苑大智、右千牛衛將軍李景嘉討之,恃勝不設備,會雨雪,士皸寒,反為虜襲,大敗,殺略萬餘人,大智等收餘卒,行且戰,乃免。於是嗣業流桂州,餘坐免官。更拜禮部尚書裴行儉為定襄道行軍大總管,率太僕少卿李思文、營州都督周道務、西軍程務挺、東軍李文暕,士無慮三十萬,捕擊反者。詔右金吾將軍曹懷舜屯井陘,右武衛將軍崔獻屯絳、龍門。明年,行儉戰黑山,大破之,其下斬泥孰匐,以首降,禽溫傅、奉職以還,餘眾保狼山。始虜未叛,鳴〓群飛入塞,吏曰:「所謂突厥雀者,南飛,胡必至。」比春還,悉墮靈、夏間,率無首,泥孰果亡。狼山眾掠雲州,都督竇懷哲、右領軍中郎將程務挺逐出之。



 永隆中,溫傅部又迎頡利族子伏念於夏州,走度河,立為可汗,諸部響應。明年,遂寇原、慶二州。復詔行儉為大總管,以右武衛將軍曹懷舜、幽州都督李文暕副之。諜者紿言伏念、溫傅保黑沙,饑甚,可輕騎取也。懷舜獨信之,輕兵倍道至黑沙,乃不見虜,得薛延陀餘部,降之;引還至長城,遇溫傅與戰,所殺相當。行儉兵壁代之陘口,縱反間,故伏念、溫傅相貳,因遣兵擊伏念,敗之。伏念走,與懷舜遇,行且戰一日,為伏念所破,棄軍奔雲中,士為虜所乘,死不可算,皆南首僕。懷舜殺牲與伏念盟,乃免。伏念益北,留輜重妻子保金牙山,以輕騎將襲懷舜,會行儉遣部將掩得其輜重,比還,無所歸,乃北走保細沙。行儉縱單于鎮兵躡之,伏念意王師不能遠,不設備,及兵至,惶駭不得戰,遂遣使間道詣行儉,執溫傅降,行儉虜之,送京師,斬東市。



 永淳元年,骨咄祿又反。



 骨咄祿,頡利族人也,雲中都督舍利元英之部酋,世襲吐屯。伏念敗,乃嘯亡散,保總材山,又治黑沙城,有眾五千,盜九姓畜馬,稍強大,乃自立為可汗,以弟默啜為殺,咄悉匐為葉護。時單于府檢校降戶部落阿史德元珍者,為長史王本立所囚。會骨咄祿來寇,元珍請諭還諸部贖罪,許之。至即降骨咄祿,與為謀,遂以為阿波達干,悉屬以兵。乃寇單于府北鄙,遂攻並州,殺嵐州刺史王德茂,分掠定州,北平刺史霍王元軌擊卻之。又攻媯州,圍單于都護府,殺司馬張行師,攻蔚州,殺刺史李思儉,執豐州都督崔知辯。詔右武衛將軍程務挺為單于道安撫大使備邊。



 嗣聖、垂拱間,連寇朔、代,掠吏士。左玉鈴衛中郎將淳于處平為陽曲道總管,將擊賊總材山,至忻州與賊遇,鏖戰不利,死者五千人。更以天官尚書韋待價為燕然道大總管討之。明年,入昌平,右鷹揚衛大將軍黑齒常之擊卻之。復入朔州地,常之與戰黃花堆,虜敗,追奔四十里,遯過磧。右監門衛中郎將〓寶璧當追,意虜即破,欲幸取功,乃募諜出塞二千里,間虜無備,趨襲之。將至,漏言於軍,虜得整眾出,皆死戰,大敗,寶璧跳還,舉軍沒。武后怒,誅寶璧,改骨咄祿曰不卒祿。俄而元珍攻突騎施,戰死。



 天授初,骨咄祿死,其子幼,不得立。默啜自立為可汗,篡位數年,始攻靈州,多殺略士民。武后以薛懷義為朔方道行軍大總管,內史李昭德為行軍長史,鳳閣鸞臺平章事蘇味道為司馬,率朔方道總管契苾明、雁門道總管王孝傑、威化道總管李多祚、豐安道總管陳令英、瀚海道總管田揚名等凡十八將軍兵出塞,雜華蕃步騎擊之,不見虜,還。俄詔孝傑為朔方道行軍總管備邊。



 契丹李盡忠等反,默啜請擊賊自效,詔可。授左衛大將軍、歸國公,以左豹韜衛將軍閻知微即部冊拜遷善可汗。默啜乃引兵擊契丹,會盡忠死,襲松漠部落,盡得孫萬榮妻子輜重,酋長崩潰。後美其攻,復詔知微持節冊默啜為特進、頡跌利施大單于、立功報國可汗。未及命,俄攻靈、勝二州,縱殺略,為屯將所敗。又遣使者謝,請為後子,復言有女,願女諸王,且求六州降戶。初,突厥內屬者分處豐、勝、靈、夏、朔、代間,謂之河曲六州降人。默啜又請粟田種十萬斛,農器三千具,鐵數萬斤,後不許,宰相李嶠亦言不可。默啜怨,為慢言,執使者司賓卿田歸道。於是納言姚等建請與之,乃歸粟、器、降人數千帳,繇是突厥遂強。



 詔淮陽王武延秀聘其女為妃,詔知微攝春官尚書,與司賓卿楊鸞莊持節護送。默啜猥曰:「我以女嫁唐天子子,今乃後家子乎!且我世附唐,今聞其子孫獨二人在,我當立之。」即囚延秀等,妄號知微為可汗,自將十萬騎南向擊靜難、平狄、清夷等軍,靜難軍使慕容玄崱以兵五千降。虜入圍媯、檀,後詔司屬卿武重規為天兵中道大總管,右武威衛將軍沙吒忠義為天兵西道總管,幽州都督張仁亶為天兵東道總管,兵凡三十萬擊之;右羽林大將軍閻敬容、李多祚為天兵西道後軍總管,兵亦十五萬。默啜破蔚州飛狐,進殘定州,殺刺史孫彥高,焚廬舍,鄉聚為空。後怒,下詔購斬默啜者王之,更號曰斬啜。虜圍趙州,長史唐波若應之,入殺刺史高睿,進攻相州。詔沙吒忠義為河北道前軍總管,李多祚為後軍總管,將軍嵎夷公福富順為奇兵總管,擊虜。時中宗還自房陵,為皇太子,拜行軍大元帥,以納言狄仁傑為副,文昌右丞宋玄爽為長史,左肅政臺御史中丞霍獻可為司馬,右肅政臺御史中丞吉頊為監軍使,將軍扶餘文宣等六人為子總管。未行,默啜聞之,取趙、定所掠男女八九萬悉坑之,出五回道去,所過人畜、金幣、子女盡剽有之,諸將皆顧望不敢戰,獨狄仁傑以兵追之,不及。



 默啜負勝輕中國,有驕志,大抵兵與頡利時略等,地縱廣萬里,諸蕃悉往聽命。復立咄悉匐為左察,骨咄祿子默矩為右察,皆統兵二萬;子匐俱為小可汗,位兩察上,典處木昆等十姓兵四萬,號拓西可汗。歲入邊,戍兵不得休,乃高選魏元忠檢校並州長史為天兵軍大總管,婁師德副之,按屯以待。又徙元忠靈武道行軍大總管,備虜。



 默啜剽隴右牧馬萬匹去,俄復盜邊,詔安北大都護相王為天兵道大元帥,率並州長史武攸宜、夏州都督薛訥與元忠擊虜,兵未出,默啜去。明年,寇鹽、夏,掠羊馬十萬,攻石嶺,遂圍並州。以雍州長史薛季昶為持節山東防禦大使,節度滄、瀛、幽、易、恆、定、媯、檀、平等九州之軍,以瀛州都督張仁亶統諸州及清夷、障塞軍之兵,與季昶掎角,又以相王為安北道行軍元帥,監諸將,王留不行。虜入代、忻,仍殺略。



 長安三年,遣使者莫賀達干請進女女皇太子子,後使平恩郡王重俊、義興郡王重明盛服立諸朝。默啜更遣大酋移力貪汗獻馬千匹,謝許婚,後渥禮其使。中宗始即位,入攻嗚沙,於是靈武軍大總管沙吒忠義與戰,不勝,死者幾萬人,虜遂入原、會,多取牧馬。帝詔絕昏,購斬默啜者王以國、官諸衛大將軍。默啜殺我行人鴻臚卿臧思言,詔左屯衛大將軍張仁亶為朔方道大總管屯邊。明年,始築三受降城於河外,障絕寇路。久之,以唐休璟代屯。睿宗初立,又請和親,詔取宋王成器女為金山公主下嫁。會左羽林大將軍孫佺等與奚戰冷陘,為奚所執,獻諸默啜,默啜殺之,更以刑部尚書郭元振代休璟。



 玄宗立,絕和親。默啜乃遣子楊我支特勒入宿衛,固求昏,以蜀王女南和縣主妻之,下書諭尉可汗。明年,使子移涅可汗引同俄特勒、火拔頡利發石失畢精騎攻北庭,都護郭虔瓘擊之,斬同俄城下,虜奔解。火拔不敢歸,攜妻子來奔,拜左武衛大將軍、燕山郡王,號其妻為金山公主,賜齎優縟。楊我支死,詔宗親三等以上吊其家。是時突厥再上書求昏,帝未報。



 初,景雲中,默啜西滅娑葛,遂役屬契丹、奚,因虐用其下。既年老,愈昏暴,部落怨畔。十姓左五咄陸、右五弩失畢俟斤皆請降。葛邏祿、胡屋、鼠尼施三姓,大漠都督特進硃斯,陰山都督謀落匐雞,玄池都督蹋實力胡鼻率眾內附;詔處其眾於金山。以右羽林軍大將軍薛訥為涼州鎮軍大總管,節度赤水、建康、河源等軍,屯涼州,以都督楊執一副之:右衛大將軍郭虔瓘為朔州鎮軍大總管,節度和戎、大武、並州之北等軍,屯並州,以長史王晙副之。撫新附,檢鈔暴。默啜屢擊葛邏祿等,詔在所都護、總管掎角應援。虜勢浸削。其婿高麗莫離支高文簡,與〓跌都督思太,吐谷渾大酋慕容道奴,鬱射施大酋鶻屈頡斤、苾悉頡力,高麗大酋高拱毅,合萬餘帳相踵款邊,詔內之河南;引拜文簡左衛大將軍、遼西郡王,思太特進、右衛大將軍兼〓跌都督、樓煩郡公,道奴左武衛將軍兼刺史、雲中郡公,鶻屈頡斤左驍衛將軍兼刺史、陰山郡公,苾悉頡力左武衛將軍兼刺史、雁門郡公,拱毅左領軍衛將軍兼刺史、平城郡公,將軍皆員外置,賜各有差。



 默啜討九姓,戰磧北,九姓潰,人畜皆死,思結等部來降,帝悉官之。拜薛訥朔方道行軍大總管,太僕卿呂延祚、靈州刺史杜賓客佐之,備邊。詔金山、大漠、陰山、玄池都督等共圖取默啜,班賞格,賜物諭之。默啜又討九姓拔野古,戰獨樂河,拔野古大敗,默啜輕歸不為備,道大林中,拔曳固殘眾突出,擊默啜,斬之,乃與入蕃使郝靈佺傳首京師。



 骨咄祿子闕特勒合故部,攻殺小可汗及宗族略盡,立其兄默棘連,是為毘伽可汗。



\end{pinyinscope}