\article{志第一 禮樂一}

\begin{pinyinscope}

 由三代而上,治出於一,而禮樂達於天下;由三代而下,治出於二,而禮樂為虛名。古者Datta)。印度哲學家、社會活動家和印度教的改革家。以法號,宮室車輿以為居,衣裳冕弁以為服,尊爵俎豆以為器,金石絲竹以為樂,以適郊廟,以臨朝廷,以事神而治民。其歲時聚會以為朝覲、聘問,歡欣交接以為射鄉、食饗,合眾興事以為師田、學校,下至里閭田畝,吉兇哀樂,凡民之事,莫不一出於禮。由之以教其民為孝慈、友悌、忠信、仁義者,常不出於居處、動作、衣服、飲食之間。蓋其朝夕從事者,無非乎此也。此所謂治出於一,而禮樂達天下,使天下安習而行之,不知所以遷善遠罪而成俗也。



 及三代已亡,遭秦變古,後之有天下者,自天子百官名號位序、國家制度、宮車服器一切用秦,其間雖有欲治之主,思所改作,不能超然遠復三代之上,而牽其時俗,稍即以損益,大抵安於茍簡而已。其朝夕從事,則以簿書、獄訟、兵食為急,曰:「此為政也,所以治民。」至於三代禮樂,具其名物而藏於有司,時出而用之郊廟、朝廷,曰:「此為禮也,所以教民。」此所謂治出於二,而禮樂為虛名。故自漢以來,史官所記事物名數、降登揖讓、拜俯伏興之節,皆有司之事爾,所謂禮之末節也。然用之郊廟、朝廷,自搢紳、大夫從事其間者,皆莫能曉習,而天下之人至於老死未嘗見也,況欲識禮樂之盛,曉然諭其意而被其教化以成俗乎?嗚呼!習其器而不知其意,忘其本而存其末,又不能備具,所謂朝覲、聘問、射鄉、食饗、師田、學校、冠婚、喪葬之禮在者幾何?自梁以來,始以其當時所行傅於《周官》五禮之名,各立一家之學。



 唐初,即用隋禮,至太宗時,中書令房玄齡、秘書監魏徵,與禮官、學士等因隋之禮,增以天子上陵、朝廟、養老、大射、講武、讀時令、納皇后、皇太子入學、太常行陵、合朔、陳兵太社等,為《吉禮》六十一篇,《賓禮》四篇,《軍禮》二十篇,《嘉禮》四十二篇,《兇禮》十一篇,是為《貞觀禮》。高宗又詔太尉長孫無忌、中書令杜正倫李義府、中書侍郎李友益、黃門侍郎劉祥道許圉師、太子賓客許敬宗、太常卿韋琨等增之為一百三十卷,是為《顯慶禮》。其文雜以式令,而義府、敬宗方得幸,多希旨傅會。事既施行,議者皆以為非。上元三年,詔復用《貞觀禮》。由是終高宗世,《貞觀》、《顯慶》二禮兼行。而有司臨事,遠引古義,與二禮參考增損之,無復定制。武氏、中宗繼以亂敗,無可言者,博士掌禮,備官而已。



 玄宗開元十年,以國子司業韋縚為禮儀使,以掌五禮。十四年,通事舍人王嵒上疏,請刪去《禮記》舊文而益以今事,詔付集賢院議。學士張說以為《禮記》不刊之書,去聖久遠,不可改易,而唐《貞觀》《顯慶禮》,儀注前後不同,宜加折衷,以為唐禮。乃詔集賢院學士右散騎常侍徐堅、左拾遺李銳及太常博士施敬本撰述,歷年未就而銳卒,蕭嵩代銳為學士,奏起居舍人王仲丘撰定,為一百五十卷,是為《大唐開元禮》。由是,唐之五禮之文始備,而後世用之,雖時小有損益,不能過也。



 貞元中,太常禮院脩撰王涇考次歷代郊廟沿革之制及其工歌祝號,而圖其壇屋陟降之序,為《郊祀錄》十卷。元和十一年,秘書郎、脩撰韋公肅又錄開元已後禮文,損益為《禮閣新儀》三十卷。十三年,太常博士王彥威為《曲臺新禮》三十卷,又採元和以來王公士民昏祭喪葬之禮為《續曲臺禮》三十卷。嗚呼,考其文記,可謂備矣,以之施於貞觀、開元之間,亦可謂盛矣,而不能至三代之隆者,具其文而意不在焉,此所謂「禮樂為虛名」也哉!



 五禮:



 一曰吉禮。



 大祀:天、地、宗廟、五帝及追尊之帝、後。中祀:社、稷、日、月、星、辰、岳、鎮、海、瀆、帝社、先蠶、七祀、文宣、武成王及古帝王、贈太子。小祀:司中、司命、司人、司祿、風伯、雨師、靈星、山林、川澤、司寒、馬祖、先牧、馬社、馬步,州縣之社稷、釋奠。而天子親祠者二十有四。三歲一祫,五歲一禘,當其歲則舉。其餘二十有二,一歲之間不能遍舉,則有司攝事。其非常祀者,有時而行之。而皇后、皇太子歲行事者各一,其餘皆有司行事。



 凡歲之常祀二十有二:冬至、正月上辛,祈穀;孟夏,雩祀昊天上帝於圓丘;季秋,大享於明堂;臘,蠟百神於南郊;春分,朝日於東郊;秋分,夕月於西郊;夏至,祭地祇於方丘;孟冬,祭神州、地祇於北郊;仲春、仲秋上戊,祭於太社;立春、立夏、季夏之土王、立秋、立冬,祀五帝於四郊;孟春、孟夏、孟秋、孟冬、臘,享於太廟;孟春吉亥,享先農,遂以耕籍。



 凡祭祀之節有六:一曰卜日,二曰齋戒,三曰陳設,四曰省牲器,五曰奠玉帛、宗廟之晨裸,六曰進熟、饋食。



 一曰卜日。凡大祀、中祀無常日者卜,小祀則筮,皆於太廟。



 卜日,前祀四十有五日,卜於廟南門之外,布卜席闑西閾外。太常卿立門東,太卜正占者立門西,卜正奠龜於席西首,灼龜之具在龜北,乃執龜立席東,北向。太卜令進受龜,詣卿示高,卿受視已,令受龜,少退俟命。卿曰:「皇帝以某日祗祀於某。」令曰:「諾。」遂還席,西向坐。命龜曰:「假爾太龜,有常。」興,授卜正龜。卜正負東扉坐,作龜,興。令進,受龜,示卿。卿受,反之。令復位,東向,占之,不釋龜,進告於卿曰:「某日從。」乃以龜還卜正。凡卜日必舉初旬;不吉,即繇中及下,如初儀。若筮日,則卜正啟韇出策,兼執之,受命還席,以韇擊策,述命曰:「假爾太筮,有常。」乃釋韇坐策,執卦以示,如卜儀。小祀筮日,則太卜令蒞之,日吉乃用,遇廢務皆勿避。



 二曰齋戒。其別有三:曰散齋,曰致齋,曰清齋。大祀,散齋四日,致齋三日;中祀,散齋三日,致齋二日;小祀,散齋二日,致齋一日。



 大祀,前期七日,太尉誓百官於尚書省曰:「某日祀某神祇於某所,各揚其職。不供其事,國有常刑。」於是乃齋。皇帝散齋於別殿;致齋,其二日於太極殿,一日於行宮。前致齋一日,尚舍奉御設御幄於太極殿西序及室內,皆東向。尚舍直長張帷於前楹下。致齋之日,質明,諸衛勒所部屯門列仗。晝漏上水一刻,侍中版奏「請中嚴」。諸衛之屬各督其隊入陳於殿庭,通事舍人引文武五品已上褲褶陪位,諸侍衛之官服其器服,諸侍臣齋者結佩,詣閣奉迎。二刻,侍中版奏「外辦」。三刻,皇帝服袞冕,結佩,乘輿出自西房,曲直華蓋,警蹕侍衛,即御座,東向,侍臣夾侍。一刻頃,侍中前跪奏稱:「侍中臣某言,請就齋室。」皇帝降座入室,文武侍臣還本司,陪位者以次出。凡豫祀之官,散齋理事如舊,唯不吊喪問疾,不作樂,不判署刑殺文書,不行刑罰,不預穢惡。致齋,唯行祀事,其祀官已齋而闕者攝。其餘清齋一日。



 三曰陳設。其別有五:有待事之次,有即事之位,有門外之位,有牲器之位,有席神之位。



 前祀三日,尚舍直長施大次於外壝東門之內道北,南向。衛尉設文武侍臣之次於其前,左右相向。設祀官次於東壝之外道南,從祀文官九品於其東,東方、南方朝集使又於其東,蕃客又於其東,重行異位,北向西上。介公、酅公於西壝之外道南,武官九品於其西,西方、北方朝集使又於其西,蕃客又於其西,東上。其褒聖侯若在朝,位於文官三品下。設陳饌幔於內壝東西門之外道北,南向;北門之外道東,西向。



 明日,奉禮郎設御位於壇之東南,西向;望燎位當柴壇之北,南向;祀官公卿位於內壝東門之內道南,分獻之官於公卿之南,執事者又於其後,異位重行,西向北上。御史位於壇下,一在東南,西向,一在西南,東向。奉禮郎位於樂縣東北,贊者在南,差退,皆西向。又設奉禮郎、贊者位於燎壇東北,西向。皆北上。協律郎位於壇上南陛之西,東向。大樂令位於北縣之間,當壇北向。從祀文官九品位於執事之南,東方、南方朝集使又於其南,蕃客又於其南,西向北上。介公、酅公位於中壝西門之內道南,武官九品又於其南,西方、北方朝集使又於其南,蕃客又於其南,東向北上。所以即而行事也。



 又設祀官及從祀群官位於東西壝門之外,如設次,所以省牲及祀之日將入而序立也。



 設牲榜於東壝之外,當門西向。蒼牲一居前,又蒼牲一、又青牲一在北,少退南上。次赤牲一、次黃牲一、白牲一、玄牲一、又赤牲一、白牲一在南,少退北上。廩犧令位於牲西南,祝史陪其後,皆北向。諸太祝位於牲東,各當牲後,祝史陪其後,西向。太常卿位於牲前少北,卿史位於其西,皆南向。



 又設酒尊之位。上帝,太尊、著尊、犧尊、山罍各二,在壇上東南隅,北向;象尊、壺尊、山罍各二,在壇下南陛之東,北向,俱西上。配帝,著尊、犧尊、象尊、山罍各二,在壇上,於上帝酒尊之東,北向西上。五帝、日、月各太尊二,在第一等。內官每陛間各象尊二,在第二等。中官每陛間各壺尊二,在第三等。外官每道間各概尊二,於下壇下。眾星每道間各散尊二,於內壝之外。凡尊,設於神座之左而右向。尊皆加勺冪,五帝、日、月以上,皆有坫,以置爵也。設御洗於午陛東南,亞獻、終獻同洗於卯陛之南,皆北向。罍水在洗東,篚在洗西,南肆。篚,實以巾爵也。分獻,罍、洗、篚、冪各於其方陛道之左,內向。執尊、罍、篚、冪者,各立於其後。玉幣之篚於壇上下尊坫之所。



 前祀一日,晡後,太史令、郊社令各常服,帥其屬升,設昊天上帝神座於壇上北方,南向。席以稿秸。高祖神堯皇帝神座於東方,西向,席以莞。五方帝、日、月於壇第一等,青帝於東陛之北,赤帝於南陛之東,黃帝於南陛之西,白帝於西陛之南,黑帝於北陛之西,大明於東陛之南。夜明於西陛之北,席皆以稿秸。五星、十二辰、河漢及內官五十有五於第二等十有二陛之間,各依其方,席皆內向。其內官有北辰座於東陛之北,曜魄寶於北陛之西,北斗於南陛之東,天一、太一皆在北斗之東,五帝內座於曜魄寶之東,皆差在前。二十八宿及中官一百五十有九於第三等,其二十八宿及帝座、七公、日星、帝席、大角、攝提、太微、太子、明堂、軒轅、三臺、五車、諸王,月星、織女、建星、天紀等一十有七皆差在前。外官一百有五於內壝之內,眾星三百六十於內壝之外,各依方次十有二道之間,席皆以莞。



 若在宗廟,則前享三日,尚舍直長施大次於廟東門之外道北,南向。守宮設文武侍臣次於其後,文左武右,俱南向。設諸享官、九廟子孫於齋坊內道東近南,西向北上。文官九品又於其南,東方、南方蕃客又於其南,西向北上。介公、巂阜公於廟西門之外,近南。武官九品於其南,西方、北方蕃客又於其南,東向北上。



 前享一日,奉禮郎設御位於廟東南,西向。設享官公卿位於東門之內道南,執事者位於其後,西向北上。卿史位於廟堂之下,一在東南,西向;一在西南,東向。令史各陪其後。奉禮郎位於樂縣東北,贊者二人,在南差退,俱西向。協律郎位於廟堂上前楹之間,近西,東向。太樂令位於北縣之間,北向。設從享之官位,九廟子孫於享官公卿之南,昭、穆異位。文官九品以上,又於其南,東方、南方蕃客又於其南,西向北上。介公、酅公位於西門之內道南,武官九品於其南,少西,西方、北方蕃客又於其南,東向北上。設牲榜於東門之外,如郊之位。設尊彞之位於廟堂之上下,每座斝彞一,黃彞一,犧尊、象尊、著尊、山罍各二,在堂上,皆於神座之左。獻祖、太祖、高祖、高宗尊彞在前楹間,北向;懿祖、代祖、太宗、中宗、睿宗尊彞在戶外,南向。各有坫焉。其壺尊二、太尊二、山罍四,皆在堂下階間,北向西上;簋、鈃、籩、豆在堂上,俱東側階之北。每座四簋居前,四簠次之,六登次之,六鈃次之,籩、豆為後,皆以南為上,屈陳而下。御洗在東階東南,亞獻又於東南,俱北向;罍水在洗東,篚在洗西,南肆。



 享日,未明五刻,太廟令服其服,布昭、穆之座於戶外,自西序以東:獻祖、太祖、高祖、高宗皆北廂南向,懿祖、代祖、太宗、中宗、睿宗南廂北向。每座黼扆,莞席紛純,藻席畫純,次席黼純,左右幾。



 四曰省牲器。省牲之日,午後十刻,去壇二百步所,禁行人。晡後二刻,郊社令、丞帥府史三人及齋郎,以尊、坫、罍、洗、篚、冪入設於位。三刻,謁者、贊引各引祀官、公卿及牲皆就位。謁者引司空,贊引引御史,入詣壇東陛,升,行掃除於上,降,行樂縣於下。初,司空將升,謁者引太常卿,贊引引御史,入詣壇東陛。升,視滌濯,降,就省牲位,南向立。廩犧令少前,曰:「請省牲。」太常卿省牲。廩犧令北面舉手曰:「腯。」諸太祝各循牲一匝,西向舉手曰:「充。」諸太祝與廩犧令以次牽牲詣廚,授太官。謁者引光祿卿詣廚,省鼎鑊,申視濯溉。祀官御史省饌具,乃還齋所。祀日,未明十五刻,太官令帥宰人以鸞刀割牲,祝史以豆取毛血,各置於饌所,遂烹牲。其於廟亦如之。



 五曰奠玉帛。祀日,未明三刻,郊社令、良愬令各帥其屬入實尊、罍,太祝以玉幣置於篚,太官令帥進饌者實諸籩、豆、簋、簠於饌幔。未明二刻,奉禮郎帥贊者先入就位。贊者引御史、博士、諸太祝及令史、祝史與執事者,入自東門壇南,北向西上。奉禮郎曰:「再拜。」贊者承傳,御史以下皆再拜。執尊、罍、篚、冪者各就位。贊者引御史、諸太祝升壇東陛。御史一人,太祝二人,行掃除於上,及第一等;御史一人,太祝七人,行掃除於下。未明一刻,謁者、贊引各引群臣就門外位。太樂令帥工人、二舞以次入,文舞陳於縣內,武舞立於縣南。謁者引司空入,奉禮郎曰:「再拜。」司空再拜,升自東陛,行掃除於上,降,行樂縣於下。謁者、贊引各引群臣入就位。初,未明三刻,諸衛列大駕仗衛。侍中版奏「請中嚴」。乘黃令進玉輅於行宮南門外,南向。未明一刻,侍中版奏「外辦」。皇帝服袞冕,乘輿以出。皇帝升輅,如初。黃門侍郎奏「請進發」。至大次門外,南向。侍中請降輅。皇帝降輅,乘輿之大次。半刻頃,太常博士引太常卿位於大次外,當門北向。侍中版奏「外辦」。質明,皇帝服大裘而冕,博士引太常卿,太常卿引皇帝至中壝門外。殿中監進大珪,尚衣奉御又以鎮珪授殿中監以進。皇帝搢大珪、執鎮珪,禮部尚書與近侍者從。皇帝至版位,西向立。太常卿前奏:「請再拜。」皇帝再拜。奉禮郎曰:「眾官再拜。」在位者皆再拜。太常卿前曰:「有司謹具,請行事。」協律郎跪,俯伏,舉麾,樂舞六成。偃麾,戛敔,樂止。太常卿前奏:「請再拜。」皇帝再拜。奉禮郎曰:「眾官再拜」在位者皆再拜。諸太祝跪取玉幣於篚,各立於尊所。皇帝升壇自南陛,北向立。太祝以玉幣授侍中,東向以進。皇帝搢鎮珪受之,跪奠於昊天上帝,俯伏,興,少退,再拜,立於西方,東向。太祝以幣授侍中以進,皇帝受幣,跪奠於高祖神堯皇帝,俯伏,興,拜,降自南陛,復於位。皇帝將奠配帝之幣,謁者七人,分引獻官奉玉幣俱進,跪奠於諸神之位;祝史、齋郎助奠。初,眾官再拜,祝史各奉毛血之豆入,各由其陛升,諸太祝迎取於壇上奠之,退立於尊所。



 若宗廟,曰晨裸。享日,未明四刻,太廟令、良愬令各帥其屬入實尊、罍,太官令帥進饌者實諸籩、豆、簋、簠。未明三刻,奉禮郎帥贊者先入就位。贊者引御史、博士、宮闈令、太祝及令史、祝史與執事者,入自東門,當階間,北向西上。奉禮郎曰:「再拜。」御史以下皆再拜。執尊、罍、篚、冪者各就位。贊者引御史、諸太祝升自東階,行掃除於堂上,令史、祝史行掃除於下。太廟令帥其屬陳瑞物太階之西,上瑞為前列,次瑞次之,下瑞為後,又陳伐國寶器亦如之,皆北向西上,藉以席。未明二刻,陳腰輿於東階之東,每室各二,皆西向北上。贊者引太廟令、太祝,宮闈令帥內外執事者,以腰輿升自東階,入獻祖室,開臽室。太祝、宮闈令奉神主各置於輿,出,置於座,次出懿祖以下神主如獻祖。鑾駕將至,謁者、贊者各引享官,通事舍人分引從享群官、九廟子孫、諸方客使,皆就門外位。鑾駕至大次門外,回輅南向。將軍降,立於輅右。侍中請降輅,皇帝降輅,乘輿之大次。通事舍人引文武五品以上從享之官皆就門外位。大樂令帥工人、二舞入。謁者引司空入,就位。奉禮郎曰:「再拜。」司空再拜,升自東階,行掃除於堂上,降,行樂縣於下。初,司空行樂縣,謁者、贊引各引享官,通事舍人分引九廟子孫、從享群官、諸方客使入,就位。皇帝停大次半刻頃,侍中版奏「外辦」。皇帝出。太常卿引皇帝至廟門外,殿中監進鎮珪,皇帝執鎮圭。近侍者從入,皇帝至版位,西向立。太常卿前曰:「再拜。」皇帝再拜。奉禮郎曰:「眾官再拜。在位者皆再拜。,太常卿前曰:「有司謹具,請行事。」協律郎舉麾,鼓柷,樂舞九成。偃麾,戛敔,樂止。太常卿曰:「再拜。」皇帝再拜。奉禮郎曰:「眾官再拜。」在位者皆再拜。皇帝詣罍洗,侍中跪取匜,興,沃水;又跪取盤,興,承水。皇帝搢珪。盥手。黃門侍郎跪,取巾於篚,興,以帨受巾,跪奠於篚。又取瓚於篚,興,以進,皇帝受瓚。侍中酌水奉盤,皇帝洗瓚,黃門侍郎授巾如初。黃帝拭瓚,升自阼階,就獻祖尊彞所。執尊者舉冪,侍中贊酌鬱酒,進獻祖神座前,北向跪,以鬯祼地奠之,俯伏,興,少退,北向再拜。又就懿祖尊彞所,執尊者舉冪,侍中取瓚於坫以進,皇帝受瓚,侍中贊酌鬱酒,進懿祖神座前,南向跪,以鬯祼地奠之。次祼太祖以下,皆如懿祖。皇帝降自阼階,復於版位。初,群官已再拜,祝史各奉毛、血及肝、膋之豆立於東門外,齋郎奉爐炭、蕭、稷、黍各立於其後,以次入自正門,升自太階。諸太祝各迎取毛、血、肝、膋於階上,進奠於神座前。祝史退立於尊所,齋郎奉爐炭置於神座之左,其蕭、稷、黍各置於其下,降,自阼階以出。諸太祝取肝、膋燔於爐,還尊所。



\end{pinyinscope}