\article{列傳第九十八 裴度}

\begin{pinyinscope}

 裴度,字中立,河東聞喜人。貞元初,擢進士第,以宏辭補校書郎。舉賢良方正異等真假。邏輯實證主義者在運用這一原則時,又分為強形式的,調河陰尉。遷監察御史,論權嬖梗切,出為河南功曹參軍。武元衡帥西川,表掌節度府書記。召為起居舍人。



 元和六年,以司封員外郎知制誥。田弘正效魏、博六州於朝,憲宗遣度宣諭,弘正知度為帝高選,故郊迎趨跽受命,且請遍至屬州,布揚天子德澤,魏人由是歡服。還,拜中書舍人。久之,進御史中丞。宣徽五坊小使方秋閱鷹狗,所過撓官司,厚得餉謝乃去。下邽令裴寰,才吏也,不為禮,因構寰出醜言,送詔獄,當大不恭。宰相武元衡婉辭諍,帝怒未置。度見延英,言寰無辜,帝恚曰:「寰誠無罪,杖小使;小使無罪,且杖寰。」度曰:「責若此固宜,第寰為令,惜陛下百姓,安可罪?」帝色霽,乃釋寰。



 王師討蔡,以度視行營諸軍,還,奏攻取策,與帝意合。且問諸將才否,度對:「李光顏義而勇,當有成功。」不三日,光顏破時曲兵,帝嘆度知言。進兼刑部侍郎。



 王承宗、李師道謀緩蔡兵,乃伏盜京師,刺用事大臣,已害宰相元衡,又擊度,刃三進,斷靴,刜背裂中單,又傷首,度冒氈,得不死。哄導駭伏,獨騶王義持賊大呼,賊斷義手。度墜溝,賊意已死,因亡去。議者欲罷度,安二鎮反側,帝怒曰:「度得全,天也!若罷之,是賊計適行。吾倚度,足破三賊矣!」度亦以權紀未張,王室陵遲,常憤愧無死所。自行營歸,知賊曲折,帝益信杖。及病創一再旬,分衛兵護第,存候踵路。疾愈,詔毋須宣政衙,即對延英,拜中書侍郎、同中書門下平章事。時方連諸道兵,環挐不解,內外大恐,人累息。及度當國,外內始安。由是討賊益急。



 始,德宗時尚何伺,中朝士相過,金吾輒飛啟,宰相至闔門謝賓客。度以時多故,宜延天下髦英咨籌策,乃建請還第與士大夫相見,詔可。會莊憲太后崩,為禮儀使。帝不聽政,議置塚宰,度曰:「塚宰,商、周六官首,秉統百僚,王者諒暗,有權聽之制。歷世官廢,故國朝置否不常,不宜徇空名,稽樞務。」乃詔百司權聽中書門下處可。



 王鍔死,家奴告鍔子稷易父奏末,冒遺獻。帝留奴仗內,遣使者如東都按責其貲。度諫曰:「自鍔死,數有獻。今因告訐而檢省其私,臣恐天下將帥聞之,有以家為計者。」帝悟,殺二奴,還使者。



 於時,討蔡數不利,群臣爭請罷兵,錢徽、蕭俛尤確苦。度奏:「病在腹心,不時去,且為大患。不然,兩河亦將視此為逆順。」會唐鄧節度使高霞寓戰卻,它相揣帝厭兵,欲赦賊,鉤上指。帝曰:「一勝一負,兵家常勢。若師常利,則古何憚用兵耶?雖累聖亦不應留賊付朕。今但論帥臣勇怯、兵強弱、處置何如耳,渠一敗便沮成計乎?」於是左右不能容其間。十二年,宰相逢吉、涯建言:「餉億煩匱,宜休師。」唯度請身督戰,帝獨目度留,曰:「果為朕行乎?」度俯伏流涕曰:「臣誓不與賊偕存。」即拜門下侍郎、平章事、彰義軍節度、淮西宣慰招討處置使。



 度以韓弘領都統,乃上還招討以避弘,然實行都統事。又制詔有異辭,欲激賊怒弘者,意弘怏怏則度無與共功。度請易其辭,窒疑間之嫌。於是表馬總為宣慰副使,韓愈行軍司馬,李正封、馮宿、李宗閔備兩使幕府。入對延英,曰:「主憂臣辱,義在必死。賊未授首,臣無還期。」帝壯之,為流涕。及行,禦通化門臨遣,賜通天御帶,發神策騎三百為衛。初,逢吉忌度,帝惡居中撓沮,出之外。



 度屯郾城,勞諸軍,宣朝廷厚意,士奮於勇。是時,諸道兵悉中官統監,自處進退。度奏罷之,使將得顓制,號令一,戰氣倍。未幾,李愬夜入懸瓠城,縛吳元濟以報。度遣馬總先入蔡,明日,統洄曲降卒萬人持節徐進,撫定其人。初,元濟禁偶語於道,夜不然燭,酒食相饋遺者以軍法論。度視事,下令唯盜賊、鬥死抵法,餘一蠲除,往來不限晝夜,民始知有生之樂。度以蔡牙卒侍帳下,或謂:「反側未安,不可去備。」度笑曰:「吾為彰義節度,元惡已擒,人皆吾人也!」眾感泣。既而申、光平定,以馬總為留後。



 度入朝,會帝以二劍付監軍梁守謙,使悉誅賊將。度遇諸郾城,復與入蔡,商罪議誅。守謙請如詔,度固不然,騰奏申解,全宥者甚眾。策勛進金紫光祿大夫、弘文館大學士、上柱國、晉國公,戶三千,復知政事。



 程異、皇甫鎛以言財賦幸,俄得宰相。度三上書極論不可,帝不納。自上印,又不聽。纖人始得乘罅。



 初,蔡平,王承宗懼,度遣辯士柏耆脅說,乃獻德、棣二州,納質子。又諭程權入覲。始判滄、景、德、棣為一鎮,朝廷命帥,而承宗勢乃離。



 李師道怙強,度密勸帝誅之。乃詔宣武、義成、武寧、橫海四節度會田弘正致討。弘正請自黎陽濟,合諸節度兵,宰相皆謂宜。度曰:「魏博軍度黎陽,即叩賊境,封畛比聯,易生顧望,是自戰其地。弘正、光顏素少斷,士心盤桓,果不可用。不如養威河北,須霜降水落,絕陽劉,深抵鄆,以營陽穀,則人人殊死,賊勢窮矣。」上曰:「善。」詔弘正如度言。弘正奉詔,師道果禽。



 大賈張陟負五坊息錢,上命坊使楊朝汶收其家簿,閱貸錢雖已償,悉鉤止,根引數十百人,列箠挺脅不承。又獲盧大夫逋券,捕盧坦家客責償,久乃悟盧群券。坦子上訴,朝汶讕語:「錢入禁中,何可得?」御史中丞蕭俛及諫官列陳中人橫恣,度亦極言之。時方討鄆,帝曰:「姑議東軍,此細事,我自處辦。」度曰:「兵事不理,止山東;中人橫暴,將亂都下。」帝不悅,徐乃悟,讓朝汶曰:「以爾,使我羞見宰相!」命殺之,而原系者。繇是京師澄肅。



 帝嘗語:「臣事君,當勵善底公,朕惡夫樹黨者。」度曰:「君子小人以類而聚,未有無徒者。君子之徒同德,小人之徒同惡,外甚類,中實遠,在陛下觀所行則辨。」帝曰:「言者大抵若此,朕豈易辨之?」度退,喜曰:「上以為難辨則易,以為易辨則難,君子小人行判矣。」已而卒為異、鎛所構,以檢校尚書右僕射兼門下侍郎平章事為河東節度使。



 穆宗即位,進檢校司空。硃克融、王廷湊亂河朔,加度鎮州行營招討使。時帝以李光顏、烏重胤爪牙將,倚以擊賊,兵十餘萬,有所畏,無尺寸功。度既受命,入賊境,數斬將以聞。俄兼押北山諸蕃使。時元稹顯結宦官魏弘簡求執政,憚度復當國,因經制軍事,數居中持梗,不使有功。度恐亂作,即上書痛暴稹過惡。帝不得已,罷弘簡、稹近職。俄擢稹宰相,以度守司空、平章事、東都留守。諫官叩延英,言不可罷度兵,搖眾心。帝不召。於是交章極論,未之省。



 會中人使幽、鎮還,言:「軍中謂度在朝,而兩河諸侯忠者懷,強者畏。今居東,人人失望。」帝悟,詔度由太原朝京師。及陛見,始陳二賊畔換,受命無功,並陳所以入覲意,感概流涕。伏未起,謁者欲宣旨,帝遽曰:「朕當延英待卿!」始,議者謂度無奧援,且久外,為奸憸拫抑,慮帝未能明其忠。及進見,辭切氣怡,卓然當天子意。在位聞者皆竦,毅將貴臣至齎咨出涕。舊儀,閣中群臣未退,宰相不奏事,稱賀則謁者答。帝以度勛德,故待以殊禮。度之行,移克融、廷湊書,開說諄沓,傅以大誼,二人不敢桀,皆願罷兵。帝方憂深州圍,欲必出牛元翼,更使度騰書布旨。或曰:「賊知度失兵柄,必背約顧望。」帝釋然,乃拜度守司徒,領淮南節度使。



 會昭義監軍劉承偕慢劉悟,舉軍嘩怒,執承偕,悟拘以聞。帝怒,問度:「何施而可?」度頓首謝:「籓臣不與政。」辭不對。帝強之,度曰:「臣素知承偕怙寵,悟不能堪,嘗以書訴臣。是時,中人趙弘亮在行營知狀,欲持悟書以奏,陛下亦知之邪?」帝曰:「我不及知。顧悟誠惡之,胡不自聞,何哉?」度曰:「雖悟得聞,恐陛下不必聽。且臣視天顏不咫尺,比尚未能決,千里單言,可悟聖聽哉?」帝亟曰:「前語姑置,直謂今日奈何?」度曰:「必欲收忠義心,使帥臣死節,獨斬承偕,則四方群盜隱然破膽矣。」帝曰:「顧太后養為子,且我何愛?更言其次。」度曰:「投諸荒裔可乎?」帝曰:「可。」悟果出承偕,昭義遂安。



 是時,徐州王智興逐崔群,諸軍盤互河北,進退未一。議者交口請相度,乃以本官兼中書侍郎、平章事。權佞側目,謂李逢吉險賊善謀,可以構度,共諷帝自襄陽召逢吉還,拜兵部尚書。度居位再閱月,果為逢吉所間,罷為左僕射。帝暴風眩,中外不聞問者凡三日。度數請到內殿,求立太子,翼日乃見帝,遂立景王為嗣。逢吉既代相,思有以牙孽之,引所厚李仲言、張又新、李續、張權輿等,內結宦官,種支黨,醜沮日聞,乃出度山南西道節度使,奪平章事。



 長慶四年,王廷湊屠元翼之家,敬宗嗟惋,嘆宰輔非其人,使兇賊熾肆。學士韋處厚上疏曰:「臣聞汲黯在朝,淮南寢謀;干木處魏,諸侯息兵。王霸之理,以一士止百萬之師,一賢制千里之難。裴度元勛巨德,文武兼備,若位巖廟,委參決,必使戎虜畏威,幽、鎮自臣。管仲曰:『人離而聽之則愚,合而聽之則聖。』治亂之本,非有他術。陛下當饋而嘆,恨無蕭、曹,今一裴度擯棄於外,所以馮唐知漢文帝有頗、牧不能用也。」帝感悟,謂處厚曰:「度累為宰相,而官無平章事,謂何?」處厚具道其由,帝於是復度兼平章事。帝雖孺蒙,然注意度,中人至度所,必丁寧尉安,且示召期。寶歷二年,度請入朝,逢吉黨大懼,權輿作偽謠云:「非衣小兒坦其腹,天上有口被驅逐。」以度平元濟也。都城東西岡六,民間以為乾數,而度第平樂里,直第五岡。權輿乃言:「度名應圖讖,第據岡原,不召而來,其意可見。」欲以傾度。天子獨能明其誣,詔復使輔政。



 先是,帝將幸東都,大臣切諫,不納。帝恚曰:「朕意決矣!雖從官宮人自挾糗,無擾百姓。」趣有司檢料行宮,中外莫敢言。度從容奏:「國家建別都,本備巡幸。自艱難以來,宮闕、署屯、百司之區,荒圮弗治,假歲月完新,然後可行。倉卒無備,有司且得罪。」帝悅曰:「群臣諫朕不及此。如卿言,誠有未便,安用往邪?」因止行。



 汴宋觀察使令狐楚言亳州聖水出,飲者疾輒愈。度判曰:「妖由人興,水不自作。」命在所禁塞。



 硃克融執賜衣使者楊文端,詭言慢己,並訴所賜濫惡,又丐假度支帛三十萬匹,不者,軍必有變;且請遣工五千助治東都,須天子東巡。帝怒,患之,欲遣重臣臨慰。度曰:「克融無恚而悖,是將亡。譬猛虎自哮躍山林,憑窟穴則然,勢不得離其處,人亦不為懼。陛下無庸遣重使,第以詔書言:『中人倨驕,須還,我自責譴。春服不謹,方詰有司。所上工宜即遣,已詔在所供擬。』此則賊謀窮矣。陛下若未能然,則答:『宮室營繕既有序,毋遣工為重勞。朝廷緣召發,乃有賜與,朕無所愛,獨與範陽,體不可爾。』」帝曰:「善。」用度次策。克融聽命,歸文端。未幾,軍亂,殺克融。



 帝縱弛,日晏坐朝。度諫曰:「此陛下月率六七臨朝,天下人知勤政,河朔賊臣皆聳畏。近開延英益稀,恐萬機奏稟,有所壅閼。夫頤養之道,常順適時候,則六氣平和,萬壽可保。道家法:『春夏蚤起,取雞鳴時,秋冬晏起,取日出時。蓋在陽,勝之以陰;在陰,勝之以陽。今方居盛夏,謂宜詰旦數坐,廣加延問;漏及巳午,則炎赫可畏,聖躬勞矣。」帝嘉納,為數視朝。



 未幾,判度支。帝崩,定策誅劉克明等,迎立江王,是為文宗。加門下侍郎。李全略死,子同捷求襲滄景軍。度奏討平之,即陳:「調兵食非宰相事,請罷度支歸有司。」奏可。進階開府儀同三司,賜實封戶三百。度懇讓不得可,乃受實封。



 太和四年,數引疾不任機重,願上政事。帝擇上醫護治,中人日勞問相躡,乃詔進司徒、平章軍國重事,須疾已,三日若五日一至中書。度讓免冊禮。度自見功高位極,不能無慮,稍詭跡避禍。於是牛僧孺、李宗閔同輔政,媢度勛業久居上,欲有所逞,乃共訾其跡損短之,因度辭位,即白帝進兼侍中,出為山南東道節度使。白罷元和所置臨漢監,數千馬納之校,以善田四百頃還襄人。頃之,固請老,不許。



 八年,徙東都留守,俄加中書令。李訓之禍,宦官肆威以逞,凡訓、注宗婭賓客悉收逮,訊報苛慘。度上疏申理,全活數十姓。武德縣主藏史盜錢亡命,捕不得。河陽節度使溫造獄其令王賞責負,系三年,母死弗許喪。度為帝言之,賞得釋。



 時閹豎擅威,天子擁虛器,搢紳道喪,度不復有經濟意,乃治第東都集賢里,沼石林叢,岑繚幽勝。午橋作別墅,具燠館涼臺,號「綠野堂」,激波其下。度野服蕭散,與白居易、劉禹錫為文章、把酒,窮晝夜相歡,不問人間事。而帝知度年雖及,神明不衰,每大臣自洛來,必問度安否。



 開成二年,復以本官節度河東。度牢辭老疾,帝命吏部郎中盧弘宣諭意曰:「為朕臥護北門可也。」趣上道,度乃之鎮。易定節度使張璠卒,軍中將立其子元益,度乃遣使曉譬禍福,元益懼,束身歸朝。



 三年,以病丐還東都。真拜中書令,臥家未克謝,有詔先給俸料。上巳宴群臣曲江,度不赴,帝賜詩曰:「注想待元老,識君恨不早。我家柱石衰,憂來學丘禱。」別詔曰:「方春慎疾為難,勉醫藥自持。朕集中欲見公詩,故示此,異日可進。」使者及門而度薨,年七十六。帝聞震悼,以詩置靈幾。冊贈太傅,謚文忠,賵禮優縟,命京兆尹鄭復護喪。度臨終,自為銘志。帝怪無遺奏,敕家人索之,得半槁,以儲貳為請,無私言。會昌元年,加贈太師。大中初,詔配享憲宗廟廷。



 度退然,才中人,而神觀邁爽,操守堅正,善占對。既有功,名震四夷。使外國者,其君長必問度年今幾、狀貌孰似、天子用否。其威譽德業比郭汾陽,而用不用常為天下重輕。事四朝,以全德始終。及歿,天下莫不思其風烈。葬管城,逮今廟食。



 五子,識、諗知名。



 識,字通理,性敏悟,凡經目未始忘。推廕補京兆參軍,擢累大理少卿。王師討劉稹,為供軍使。稹平,改司農卿,進湖南觀察使。入拜大理卿,襲晉國公半封。為涇原節度使。



 時蕃酋尚恐熱上三州七關,列屯分守。宣宗擇名臣,以識帥涇原,畢諴帥邠寧,李福帥夏州,帝親臨遣。識至,治堡障,整戎器,開屯田。初,將士守邊,或積歲不得還。識與立戍限,滿者代;親七十,近戍。由是人感悅。加檢校刑部尚書,徙鳳翔、忠武、天平、邠寧、靈武等軍。進檢校尚書右僕射。靈武地斥鹵無井,識誓神而鑿之,果得泉。歷六節度,所蒞皆有可述。卒,贈司空,謚曰昭。



 諗有文,籍廕累官考功員外郎。宣宗訪元和宰相子,思度勛望,故待諗有加。為翰林學士,累遷工部侍郎,詔加承旨。適會帝幸其院,諗即稱謝。帝曰:「可歸與妻子相慶。」取御奩果以賜,諗舉衣跽受。帝顧宮人取巾裹賜之。後為太子少師,封河東郡公。黃巢盜國,迫以偽官,不從,遇害。



 贊曰:憲宗討蔡,出入四年。元濟外連奸臣,刺宰相及用事者,沮駭朝謀。惟天子赫然排群議,任度政事,倚以討賊。身督戰,遂平淮西。非度破賊之難,任度之為難也。韓愈頌其功曰:「凡此蔡功,惟斷乃成。」其知言哉!穆宗不君,憸人腐夫乘釁鐫詆,而度遂無顯功。非前智後愚,用不用,勢當然矣。前史稱度晚節頗浮沉為自安計,是不然。《大雅》曰:「既明且哲,以保其身。」度何訿云。



\end{pinyinscope}