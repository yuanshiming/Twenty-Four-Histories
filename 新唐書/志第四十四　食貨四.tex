\article{志第四十四 食貨四}

\begin{pinyinscope}

 唐有鹽池十八,井六百四十,皆隸度支。蒲州安邑、解縣有池五,總曰「兩池」,歲得鹽萬斛分為分析判斷和綜合判斷。但前者未擴大新的內容,後者不,以供京師。鹽州五原有烏池、白池、瓦池、細項池,靈州有溫泉池、兩井池、長尾池、五泉池、紅桃池、回樂池、弘靜池、會州有河池,三州皆輸米以代鹽。安北都護府有胡落池,歲得鹽萬四千斛,以給振武、天德。黔州有井四十一,成州、巂州井各一,果、閬、開、通井百二十三,山南西院領之。邛、眉、嘉有井十三,劍南西川院領之。梓、遂、綿、合、昌、渝、瀘、資、榮、陵、簡有井四百六十,劍南東川院領之。皆隨月督課。幽州、大同橫野軍有鹽屯,每屯有丁有兵,歲得鹽二千八百斛,下者千五百斛。負海州歲免租為鹽二萬斛以輸司農。青、楚、海、滄、棣、杭、蘇等州,以鹽價市輕貨,亦輸司農。



 天寶、至德間,鹽每斗十錢。乾元元年,鹽鐵、鑄錢使第五琦初變鹽法,就山海井灶近利之地置監院,游民業鹽者為亭戶,免雜徭。盜鬻者論以法。及琦為諸州榷鹽鐵使,盡榷天下鹽,斗加時價百錢而出之,為錢一百一十。



 自兵起,流庸未復,稅賦不足供費,鹽鐵使劉晏以為因民所急而稅之,則國足用。於是上鹽法輕重之宜,以鹽吏多則州縣擾,出鹽鄉因舊監置吏,亭戶糶商人,縱其所之。江、嶺去鹽遠者,有常平鹽,每商人不至,則減價以糶民,官收厚利而人不知貴。晏又以鹽生霖潦則鹵薄,旱則土溜墳,乃隨時為令,遣吏曉導,倍於勸農。吳、越、揚、楚鹽廩至數千,積鹽二萬餘石。有漣水、湖州、越州、杭州四場,嘉興、海陵、鹽城、新亭、臨平、蘭亭、永嘉、大昌、候官、富都十監,歲得錢百餘萬緡,以當百餘州之賦。自淮北置巡院十三,曰揚州、陳許、汴州、廬壽、白沙、淮西、甬橋、浙西、宋州、泗州、嶺南、兗鄆、鄭滑,捕私鹽者,奸盜為之衰息。然諸道加榷鹽錢,商人舟所過有稅。晏奏罷州縣率稅,禁堰埭邀以利者。晏之始至也,鹽利歲才四十萬緡,至大歷末,六百餘萬緡。天下之賦,鹽利居半,宮闈服御、軍饟、百官祿俸皆仰給焉。明年而晏罷。



 貞元四年,淮南節度使陳少游奏加民賦,自此江淮鹽每斗亦增二百,為錢三百一十,其後復增六十,河中兩池鹽每斗為錢三百七十。江淮豪賈射利,或時倍之,官收不能過半,民始怨矣。



 劉晏鹽法既成,商人納絹以代鹽利者,每緡加錢二百,以備將士春服。包佶為汴東水陸運、兩稅、鹽鐵使,許以漆器、玳瑁、綾綺代鹽價,雖不可用者亦高估而售之,廣虛數以罔上。亭戶冒法,私鬻不絕,巡捕之卒,遍於州縣。鹽估益貴,商人乘時射利,遠鄉貧民困高估,至有淡食者。巡吏既多,官冗傷財,當時病之。其後軍費日增,鹽價浸貴,有以穀數斗易鹽一升。私糴犯法,未嘗少息。



 順宗時始減江淮鹽價,每斗為錢二百五十,河中兩池鹽斗錢三百。增雲安、渙陽、塗鞬三監。其後鹽鐵使李錡奏江淮鹽斗減錢十以便民,未幾復舊。方是時,錡盛貢獻以固寵,朝廷大臣皆餌以厚貨,鹽鐵之利積於私室,而國用耗屈,榷鹽法大壞,多為虛估,率千錢不滿百三十而已。兵部侍郎李巽為使,以鹽利皆歸度支,物無虛估,天下糶鹽稅茶,其贏六百六十五萬緡。初歲之利,如劉晏之季年,其後則三倍晏時矣。兩池鹽利,歲收百五十餘萬緡。四方豪商猾賈、雜處解縣,主以郎官,其佐貳皆御史。鹽民田園籍於縣,而令不得以縣民治之。



 憲宗之討淮西也,度支使皇甫鎛加劍南東西兩川、山南西道鹽估以供軍。貞元中,盜鬻兩池鹽一石者死,至元和中,減死流天德五城,鎛奏論死如初。一斗以上杖背,沒其車驢,能捕斗鹽者賞千錢;節度觀察使以判官、州以司錄錄事參軍察私鹽,漏一石以上罰課料;鬻兩池鹽者,坊市居邸主人、市儈皆論坐;盜刮鹻土一斗,比鹽一升。州縣團保相察,比於貞元加酷矣。自兵興,河北鹽法羈縻而已。至皇甫鎛,又奏置榷鹽使,如江淮榷法,犯禁歲多。及田弘正舉魏博歸朝廷,穆宗命河北罷榷鹽。戶部侍郎張平叔議榷鹽法弊,請糶鹽可以富國,詔公卿議其可否。中書舍人韋處厚、兵部侍郎韓愈條詰之,以為不可,平叔屈服。是時奉天鹵池生水柏,以灰一斛得鹽十二斤,利倍鹻鹵。文帝時,採灰一斗,比鹽一斤論罪。開成末,詔私鹽月再犯者,易縣令,罰刺史俸;十犯,則罰觀察、判官課科。



 宣宗即位,茶、鹽之法益密,糶鹽少、私盜多者,謫觀察、判官,不計十犯。戶部侍郎、判度支盧弘止以兩池鹽法敝,遣巡院官司空輿更立新法,其課倍入,遷榷鹽使。以壕籬者,鹽池之堤禁,有盜壞與鬻鹻皆死,鹽盜持弓矢者亦皆死刑。兵部侍郎、判度支周墀又言:「兩池鹽盜販者,跡其居處,保、社按罪。鬻五石,市二石,亭戶盜糶二石,皆死。」是時江、吳群盜,以所剽物易茶鹽,不受者焚其室廬,吏不敢枝梧,鎮戍、場鋪、堰埭以關通致富。宣宗乃擇嘗更兩畿輔望縣令者為監院官。戶部侍郎裴休為鹽鐵使,上鹽法八事,其法皆施行,兩池榷課大增。



 其後兵遍天下,諸鎮擅利,兩池為河中節度使王重榮所有,歲貢鹽三千車。中官田令孜募新軍五十四都,餫轉不足,乃倡議兩池復歸鹽鐵使,而重榮不奉詔,至舉兵反,僖宗為再出,然而卒不能奪。



 唐初無酒禁。乾元元年,京師酒貴,肅宗以稟食方屈,乃禁京城酤酒,期以麥熟如初。二年,饑,復禁酤,非光祿祭祀、燕蕃客,不御酒。廣德二年,定天下酤戶以月收稅。建中元年,罷之。三年,復禁民酤,以佐軍費,置肆釀酒,斛收直三千,州縣總領,醨薄私釀者論其罪。尋以京師四方所湊,罷榷。貞元二年,復禁京城、畿縣酒,天下置肆以酤者,斗錢百五十,免其徭役,獨淮南、忠武、宣武、河東榷麴而已。元和六年,罷京師酤肆,以榷酒錢隨兩稅青苗斂之。大和八年,遂罷京師榷酤。凡天下榷酒為錢百五十六萬餘緡,在襄費居三之一,貧戶逃酤不在焉。昭宗世,以用度不足,易京畿近鎮麴法,復榷酒以贍軍,鳳翔節度使李茂貞方顓其利,按兵請入奏利害,天子遽罷之。



 初,德宗納戶部侍郎趙贊議,稅天下茶、漆、竹、木,十取一,以為常平本錢。及出奉天,乃悼悔,下詔亟罷之。及硃泚平,佞臣希意興利者益進。貞元八年,以水災減稅,明年,諸道鹽鐵使張滂奏:出茶州縣若山及商人要路,以三等定估,十稅其一。自是歲得錢四十萬緡,然水旱亦未嘗拯之也。



 穆宗即位,兩鎮用兵,帑藏空虛,禁中起百尺樓,費不可勝計。鹽鐵使王播圖寵以自幸,乃增天下茶稅,率百錢增五十。江淮、浙東西、嶺南、福建、荊襄茶,播自領之,兩川以戶部領之。天下茶加斤至二十兩,播又奏加取焉。右拾遺李玨上疏諫曰:「榷率起於養兵,今邊境無虞,而厚斂傷民,不可一也;茗飲,人之所資,重稅則價必增,貧弱益困,不可二也;山澤之饒,其出不訾,論稅以售多為利,價騰踴則市者稀,不可三也。」其後王涯判二使,置榷茶使,徙民茶樹於官場,焚其舊積者,天下大怨。令狐楚代為鹽鐵使兼榷茶使,復令納榷,加價而已。李石為相,以茶稅皆歸鹽鐵,復貞元之制。



 武宗即位,鹽鐵轉運使崔珙又增江淮茶稅。是時茶商所過州縣有重稅,或掠奪舟車,露積雨中,諸道置邸以收稅,謂之「搨地錢」,故私販益起。大中初,鹽鐵轉運使裴休著條約:私鬻三犯皆三百斤,乃論死;長行群旅,茶雖少皆死;雇載三犯至五百斤、居舍儈保四犯至千斤者,皆死;園戶私鬻百斤以上,杖背,三犯,加重徭;伐園失業者,刺史、縣令以縱私鹽論。廬、壽、淮南皆加半稅,私商給自首之帖,天下稅茶增倍貞元。江淮茶為大摸,一斤至五十兩。諸道鹽鐵使於悰每斤增稅錢五,謂之「剩茶錢」,自是斤兩復舊。



 凡銀、銅、鐵、錫之冶一百六十八。陜、宣、潤、饒、衢、信五州,銀冶五十八,銅冶九十六,鐵山五,錫山二,銅山四。汾州礬山七。麟德二年,廢陜州銅冶四十八。



 開元十五年,初稅伊陽五重山銀、錫。德宗時戶部侍郎韓洄建議,山澤之利宜歸王者,自是皆隸鹽鐵使。



 元和初,天下銀治廢者四十,歲採銀萬二千兩,銅二十六萬六千斤,鐵二百七萬斤,錫五萬斤,鉛無常數。



 開成元年,復以山澤之利歸州縣,刺史選吏主之。其後諸州牟利以自殖,舉天下不過七萬餘緡,不能當一縣之茶稅。及宣宗增河湟戍兵衣絹五十二萬餘匹。鹽鐵轉運使裴休請復歸鹽鐵使以供國用,增銀冶二、鐵山七十一,廢銅冶二十七、鉛山一。天下歲率銀二萬五千兩、銅六十五萬五千斤、鉛十一萬四千斤、錫萬七千斤、鐵五十三萬二千斤。



 隋末行五銖白錢,天下盜起,私鑄錢行。千錢初重二斤,其後愈輕,不及一斤,鐵葉、皮紙皆以為錢。高祖入長安,民間行線環錢,其制輕小,凡八九萬才滿半斛。



 武德四年,鑄「開元通寶」,徑八分,重二銖四參,積十錢重一兩,得輕重大小之中,其文以八分、篆、隸三體。洛、並、幽、益、桂等州皆置監。賜秦王、齊王三爐,右僕射裴寂一爐以鑄。盜鑄者論死,沒其家屬。



 其後盜鑄漸起。顯慶五年,以惡錢多,官為市之,以一善錢售五惡錢,民間藏惡錢以待禁馳。乾封元年,改鑄「乾封泉寶」錢,徑寸,重二銖六分,以一當舊錢之十。逾年而舊錢多廢。明年,以商賈不通,米帛踴貴,復行開元通寶錢,天下皆鑄之。然私錢犯法日蕃,有以舟筏鑄江中者。詔所在納惡錢,而奸亦不息。儀鳳中,瀕江民多私鑄錢為業,詔巡江官督捕,載銅、錫、鑞過百斤者沒官。四年,命東都糶米粟,斗別納惡錢百,少府、司農毀之。是時鑄多錢賤,米粟踴貴,乃罷少府鑄,尋復舊。永淳元年,私鑄者抵死,鄰、保、里、坊、村正皆從坐。武后時,錢非穿穴及鐵錫銅液,皆得用之,熟銅、排斗、沙澀之錢皆售,自是盜鑄蜂起,江淮游民依大山陂海以鑄,吏莫能捕。



 先天之際,兩京錢益濫,郴、衡錢才有輪郭,鐵錫五銖之屬皆可用之。或熔錫摸錢,須臾百十。開元初,宰相宋璟請禁惡錢,行二銖四參錢,毀舊錢不可用者。江淮有官爐錢、偏爐錢、棱錢、時錢,遣監察御史蕭隱之使江淮,率戶出惡錢,捕責甚峻,上青錢皆輸官,小惡者沈江湖,市井不通,物價益貴,隱之坐貶官。宋璟又請出米十萬斛收惡錢,少府毀之。十一年,詔所在加鑄,禁賣銅錫及造銅器者。二十年,千錢以重六斤四兩為率,每錢重二銖四參,禁缺頓、沙澀、蕩染、白強、黑強之錢。首者,官為市之。銅一斤為錢八十。



 二十二年,宰相張九齡建議:「古者以布帛菽粟不可尺寸抄勺而均,乃為錢以通貿易。官鑄所入無幾,而工費多,宜縱民鑄。」議下百官,宰相裴耀卿、黃門侍郎李林甫、河南少尹蕭炅、秘書監崔沔皆以為「嚴斷惡錢則人知禁,稅銅折役則官冶可成,計估度庸則私錢以利薄而自息。若許私鑄,則下皆棄農而競利矣。」左監門衛錄事參軍事劉秩曰:「今之錢,古之下幣也。若舍之任人,則上無以御下,下無以事上,不可一也;物賤傷農,錢輕傷賈,物重則錢輕,錢輕由乎物多,多則作法收之使少,物少則作法布之使輕,奈何假人?不可二也;鑄錢不雜鉛鐵則無利,雜則錢惡。今塞私鑄之路,人猶冒死,況設陷井誘之?不可三也;鑄錢無利則人不鑄,有利則去南畝者眾,不可四也;人富則不可以賞勸、貧則不可以威禁,法不行,人不理,繇貧富不齊,若得鑄錢,貧者服役於富室,富室乘而益恣,不可五也。夫錢重繇人日滋於前,而爐不加舊。公錢與銅價頗等,故破重錢為輕錢,銅之不贍,在採用者眾也。銅之為兵不如鐵,為器不如漆。禁銅則人無所用,盜鑄者少,公錢不破,人不犯死,錢又日增,是一舉而四美兼也。」是時公卿皆以縱民鑄為不便,於是下詔禁惡錢而已。信安郡王禕復言國用不足,請縱私鑄,議者皆畏禕帝弟之貴,莫敢與抗,獨倉部郎中韋伯陽以為不可,禕議亦格。



 二十六年,宣、潤等州初置錢監,兩京用錢稍善,米粟價益下。其後錢又漸惡,詔出銅所在置監,鑄「開元通寶」錢,京師庫藏皆滿。天下盜鑄益起,廣陵、丹楊、宣城尤甚。京師權豪,歲歲取之,舟車相屬。江淮偏爐錢數十種,雜以鐵錫,輕漫無復錢形。公鑄者號官爐錢,一以當偏爐錢七八,富商往往藏之,以易江淮私鑄者。兩京錢有鵝眼、古文、線環之別,每貫重不過三四斤,至翦鐵而緡之。宰相李林甫請出絹布三百萬匹,平估收錢,物價踴貴,訴者日萬人。兵部侍郎楊國忠欲招權以市恩,揚鞭市門曰:「行當復之。」明日,詔復行舊錢。天寶十一載,又出錢三十萬緡易兩市惡錢,出左藏庫排斗錢,許民易之。國忠又言錢非鐵錫、銅沙、穿穴、古文,皆得用之。



 是時增調農人鑄錢,既非所習,皆不聊生。內作判官韋倫請厚價募工,繇是役用減而鼓鑄多。天下爐九十九:絳州三十,揚、潤、宣、鄂、蔚皆十,益、鄧、郴皆五,洋州三,定州一。每爐歲鑄錢三千三百緡,役丁匠三十,費銅二萬一千二百斤、鑞三千七百斤、錫五百斤。每千錢費錢七百五十。天下歲鑄三十二萬七千緡。



 肅宗乾元元年,經費不給,鑄錢使第五琦鑄「乾元重寶」錢,徑一寸,每緡重十斤,與開元通寶參用,以一當十,亦號「乾元十當錢」。先是諸爐鑄錢窳薄,熔破錢及佛像,謂之「盤陀」,皆鑄為私錢,犯者杖死。第五琦為相,復命絳州諸爐鑄重輪乾元錢,徑一寸二分,其文亦曰:「乾元重寶」,背之外郭為重輪,每緡重十二斤,與開元通寶錢並行,以一當五十。是時民間行三錢,大而重棱者亦號「重棱錢」。法既屢易,物價騰踴,米斗錢至七千,餓死者滿道。



 初,有「虛錢」,京師人人私鑄,並小錢,壞鐘、像,犯禁者愈眾。鄭叔清為京兆尹,數月榜死者八百餘人。肅宗以新錢不便,命百官集議,不能改。上元元年,減重輪錢以一當三十,開元舊錢與乾元十當錢,皆以一當十,碾磑鬻受,得為實錢,虛錢交易皆用十當錢,由是錢有虛實之名。



 史思明據東都,亦鑄「得一元寶」錢,徑一寸四分,以一當開元通寶之百。既而惡「得一」非長祚之兆,改其文曰「順天元寶」。



 代宗即位,乾元重寶錢以一當二,重輪錢以一當三,凡三日而大小錢皆以一當一。自第五琦更鑄,犯法者日數百,州縣不能禁止,至是人甚便之。其後民間乾元、重棱二錢鑄為器,不復出矣。當時議者以為:「自天寶至今,戶九百餘萬。《王制》:上農夫食九人,中農夫七人。以中農夫計之,為六千三百萬人。少壯相均,人食米二升,日費米百二十六萬斛,歲費四萬五千三百六十萬斛,而衣倍之,吉兇之禮再倍,餘三年之儲以備水旱兇災,當米十三萬六千八十萬斛,以貴賤豐儉相當,則米之直與錢鈞也。田以高下肥瘠豐耗為率,一頃出米五十餘斛,當田二千七百二十一萬六千頃。而錢亦歲毀於棺瓶埋藏焚溺,其間銅貴錢賤,有鑄以為器者,不出十年錢幾盡,不足周當世之用。」諸道鹽鐵轉運使劉晏以江、嶺諸州,任土所出,皆重粗賤弱之貨,輸京師不足以供道路之直。於是積之江淮,易銅鉛薪炭,廣鑄錢,歲得十餘萬緡,輸京師及荊、揚二州,自是錢日增矣。



 大歷七年,禁天下鑄銅器。建中初,戶部侍郎韓洄以商州紅崖冶銅多,請復洛源廢監,起十爐,歲鑄錢七萬二千緡,每千錢費九百。德宗從之。


江淮多鉛錫錢,以銅
 \gezhu{
  湯皿}
 外,不盈斤兩,帛價益貴。銷千錢為銅六斤,鑄器則斤得錢六百,故銷鑄者多,而錢益耗。判度支趙贊採連州白銅鑄大錢,一當十,以權輕重。貞元初,駱谷、散關禁行人以一錢出者。諸道鹽鐵使張滂奏禁江淮鑄銅為器,惟鑄鑒而已。十年,詔天下鑄銅器,每器一斤,其直不得過百六十,銷錢者以盜鑄論。然而民間錢益少,繒帛價輕,州縣禁錢不出境,商賈皆絕。浙西觀察使李若初請通錢往來,而京師商賈齎錢四方貿易者不可勝計。詔復禁之。二十年,命市井交易,以綾、羅、絹、布、雜貨與錢兼用。憲宗以錢少,復禁用銅器。時商賈至京師,委錢諸道進奏院及諸軍、諸使富家,以輕裝趨四方,合券乃取之,號「飛錢」。京兆尹裴武請禁與商賈飛錢者,廋索諸坊,二人為保。



 鹽鐵使李巽以郴州平陽銅坑二百八十餘,復置桂陽監,以兩爐日鑄錢二十萬。天下歲鑄錢十三萬五千緡。命商賈蓄錢者,皆出以市貨;天下有銀之山必有銅,唯銀無益於人,五嶺以北,採銀一兩者流他州,官吏論罪。元和四年,京師用錢緡少二十及有鉛錫錢者,捕之;非交易而錢行衢路者,不問。復詔採五嶺銀坑,禁錢出嶺。六年,貿易錢十緡以上者,參用布帛。蔚州三河冶距飛狐故監二十里而近,河東節度使王鍔置爐,疏拒馬河水鑄錢,工費尤省,以刺史李聽為使,以五爐鑄,每爐月鑄錢三十萬,自是河東錫錢皆廢。



 自京師禁飛錢,家有滯藏,物價浸輕。判度支盧坦、兵部尚書判戶部事王紹、鹽鐵使王播請許商人於戶部、度支、鹽鐵三司飛錢,每千錢增給百錢,然商人無至者。復許與商人敵貫而易之,然錢重帛輕如故。憲宗為之出內庫錢五十萬緡市布帛,每匹加舊估十之一。會吳元濟、王承宗連衡拒命,以七道兵討之,經費屈竭。皇甫鎛建議,內外用錢每緡墊二十外,復抽五十送度支以贍軍。十二年,復給京兆府錢五十萬緡市布帛,而富家錢過五千貫者死,王公重貶,沒入於官,以五之一賞告者。京師區肆所積,皆方鎮錢,少亦五十萬緡,乃爭市第宅。然富賈倚左右神策軍官錢為名,府縣不敢劾問。民間墊陌有至七十者,鉛錫錢益多,吏捕犯者,多屬諸軍、諸使,言虖集市人強奪,毆傷吏卒。京兆尹崔元略請犯者本軍、本使涖決,帝不能用,詔送本軍、本使,而京兆府遣人涖決。穆宗即位,京師鬻金銀十兩亦墊一兩,糴米鹽百錢墊七八。京兆尹柳公綽以嚴法禁止之。尋以所在用錢墊陌不一,詔從俗所宜,內外給用,每緡墊八十。



 寶歷初,河南尹王起請銷錢為佛像者以盜鑄錢論。大和三年,詔佛像以鉛、錫、土、木為之,飾帶以金銀、鍮石、烏油、藍鐵,唯鑒、磬、釘、鐶、鈕得用銅,餘皆禁之,盜鑄者死。是時峻鉛錫錢之禁。告千錢者賞以五千。



 四年,詔積錢以七千緡為率,十萬緡者期以一年出之,二十萬以二年。凡交易百緡以上者,匹帛米粟居半。河南府、揚州、江陵府以都會之劇,約束如京師。未幾皆罷。



 八年,河東錫錢復起,鹽鐵使王涯置飛狐鑄錢院於蔚州,天下歲鑄錢不及十萬緡。文宗病幣輕錢重,詔方鎮縱錢穀交易。時雖禁銅為器,而江淮、嶺南列肆鬻之,鑄千錢為器,售利數倍。宰相李玨請加爐鑄錢,於是禁銅器,官一切為市之。天下銅坑五十,歲採銅二十六萬六千斤。及武宗廢浮屠法,永平監官李鬱彥請以銅像、鐘、磬、金盧、鐸皆歸巡院,州縣銅益多矣。鹽鐵使以工有常力,不足以加鑄,許諸道觀察使皆得置錢坊。淮南節度使李紳請天下以州名鑄錢,京師為京錢,大小徑寸,如開元通寶,交易禁用舊錢。會宣宗即位,盡黜會昌之政,新錢以字可辨,復鑄為像。



 昭宗末,京師用錢八百五十為貫,每百才八十五,河南府以八十為百云。



\end{pinyinscope}