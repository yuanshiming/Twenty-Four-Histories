\article{本紀第二 太宗}

\begin{pinyinscope}

 太宗文武大聖大廣孝皇帝諱世民,高祖次子也。母曰太穆皇后竇氏。生而不驚。方四歲,有書生謁高祖曰:「公在相法,貴人也,然必有貴子。」及見太宗禮記》,即今本《禮記》。,曰:「龍鳳之姿,天日之表,其年幾冠,必能濟世安民。」書生巳辭去,高祖懼其語洩,使人追殺之,而不知其所往,因以為神。乃採其語,名之曰世民。



 大業中,突厥圍煬帝雁門,煬帝從圍中以木系詔書,投汾水而下,募兵赴援。太宗時年十六,往應募,隸將軍云定興,謂定興曰:「虐敢圍吾天子者,以為無援故也。今宜先後吾軍為數十里,使其晝見旌旗,夜聞鉦鼓,以為大至,則可不擊而走之。不然,知我虛實,則勝敗未可知也。」定興從之。軍至崞縣,突厥候騎見其軍來不絕,果馳告始畢可汗曰:「救兵大至矣!」遂引去。高祖擊歷山飛,陷其圍中,太宗輕騎取之而出,遂奮擊,大破之。



 太宗為人聰明英武,有大志,而能屈節下士。時天下已亂,盜賊起,知隋必亡,乃推財養士,結納豪傑。長孫順德、劉弘基等,皆因事亡命,匿之。又與晉陽令劉文靜尤善,文靜坐李密事系獄,太宗夜就獄中見之,與圖大事。時百姓避賊多入城,城中幾萬人,文靜為令久,知其豪傑,因共部署。計已定,乃因裴寂告高祖。高祖初不許,已而許之。



 高祖已起兵,建大將軍府。太宗率兵徇西河,斬其郡丞高德儒。拜右領軍大都督,封敦煌郡公。唐兵西,將至霍邑,會天久雨,糧且盡,高祖謀欲還兵太原。太宗諫曰:「義師為天下起也,宜直入咸陽,號令天下。今還守一城,是為賊爾。」高祖不納。太宗哭於軍門,高祖驚,召問之,對曰:「還則眾散於前,而敵乘於後,死亡須臾,所以悲爾。」高祖寤,曰:「起事者汝也,成敗惟汝。」時左軍已先返,即與隴西公建成分追之。夜半,太宗失道入山谷,棄其馬,步而及其兵,與俱還。高祖乃將而前,遲明至霍邑。宋老生不出,太宗從數騎傅其城,舉鞭指麾,若將圍之者。老生怒,出,背城陣。高祖率建成居其東,太宗及柴紹居其南。老生兵薄東陣,建成墜馬,老生乘之,高祖軍卻。太宗自南原馳下阪,分兵斷其軍為二,而出其陣後,老生兵敗走,遂斬之。進次涇陽,擊胡賊劉鷂子,破之。唐兵攻長安,太宗屯金城坊,攻其西北,遂克之。義寧元年,為光祿大夫、唐國內史,徙封秦國公,食邑萬戶。薛舉攻扶風,太宗擊敗之,斬首萬餘級,遂略地至隴右。二年,為右元帥,徙封趙國公,率兵十萬攻東都,不克而還,設三伏於三王陵,敗隋將段達兵萬人。



 武德元年,為尚書令、右翊衛大將軍,進封秦王。薛舉寇涇州,太宗為西討元帥,進位雍州牧。七月,太宗有疾,諸將為舉所敗。八月,太宗疾間,復屯於高坑城,相持六十餘日。已而舉死,其子仁杲率其眾求戰,太宗按軍不動。久之,仁杲糧盡,眾稍離叛,太宗曰:「可矣!」乃遣行軍總管梁實柵淺水原。仁杲將宗羅睺擊實,太宗遣將軍龐玉救實,玉軍幾敗,太宗率兵出其後,羅睺敗走,太宗追之,至其城下,仁杲乃出降。師還,高祖遣李密馳傳勞之於豳州。密見太宗,不敢仰視,退而嘆曰:「真英主也!」獻捷太廟,拜右武候大將軍、太尉、使持節、陜東道大行臺尚書令,詔蒲、陜、河北諸總管兵皆受其節度。



 二年正月,鎮長春宮,進拜左武候大將軍、涼州總管。是時,劉武周據並州,宋金剛陷滄州,王行本據蒲州,而夏縣人呂崇茂殺縣令以應武周。高祖懼,詔諸將棄河東以守關中。太宗以為不可棄,願得兵三萬可以破賊。高祖於是悉發關中兵益之。十一月,出龍門關,屯於柏壁。



 三年四月,擊敗宋金剛於柏壁。金剛走介州,太宗追之,一百夜馳二百里,宿於雀鼠谷之西原。軍士皆饑,太宗不食者二日,行至浩州乃得食,而金剛將尉遲敬德、尋相等皆來降。劉武周懼,奔於突厥,其將楊伏念舉並州降。高祖遣蕭瑀即軍中拜太宗益州道行臺尚書令。七月,討王世充,敗之於北邙。



 四年二月,竇建德率兵十萬以援世充,太宗敗建德於虎牢,執之,世充乃降。六月,凱旋,太宗被金甲,陳鐵騎一萬、介士三萬,前後鼓吹,獻俘於太廟。高祖以謂太宗功高,古官號不足以稱,乃加號天策上將,領司徒、陜東道大行臺尚書令,位在王公上,增邑戶至三萬,賜袞冕、金輅、雙璧、黃金六千斤,前後鼓吹九部之樂,班劍四十人。



 五年正月,討劉黑闥於洺州,敗之。黑闥既降,已而復反。高祖怒,命太子建成取山東男子十五以上悉坑之,驅其小弱婦女以實關中。太宗切諫,以為不可,遂已。加拜左右十二衛大將軍。



 七年,突厥寇邊,太宗與遇於豳州,從百騎與其可汗語,乃盟而去。



 八年,進位中書令。初,高祖起太原,非其本意,而事出太宗。及取天下,破宋金剛、王世充、竇建德等,太宗切益高,而高祖屢許以為太子。太子建成懼廢,與齊王元吉謀害太宗,未發。



 九年六月,大宗以兵入玄武門,殺太子建成及齊王元吉。高祖大驚,乃以太宗為皇太子。八月甲子,即皇帝位於東宮顯德殿。遣裴寂告於南效。大赦,武德流人還之。賜文武官勛、爵。免關內及蒲、芮、虞、泰、陜、鼎六州二歲租,給復天下一年。民八十以上賜粟帛,百歲加版授。廢潼關以東瀕河諸關。癸酉,放宮女三千餘人。丙子,立妃長孫氏為皇后。癸未,突厥寇便橋。乙酉,及突厥頡利盟於便橋。九月壬子,禁私家妖神淫祀、占卜非龜易五兆者。十月丙辰朔,日有食之。癸亥,立中山郡王承乾為皇太子。庚辰,蕭瑀、陳叔達罷。十一月庚寅,降宗室郡王非有功者爵為縣公。十二月癸酉,慮囚。是歲,進封子長沙郡王恪為漢王,宜陽郡王祐楚王。



 貞觀元年正月乙酉,改元。辛丑,燕郡王李藝反於涇州,伏誅。二月丁巳,詔民男二十、女十五以上無夫家者,州縣以禮聘娶;貧不能自行者,鄉里富人及親戚資送之;鰥夫六十、寡婦五十、婦人有子若守節者勿強。三月癸巳,皇后親蠶。丙午,詔:「齊僕射崔季舒、黃門侍郎郭遵、尚書右丞封孝琰以極言蒙難,季舒子剛、遵子雲、孝琰子君遵並及淫刑,宜免內侍,褒敘以官。」閏月癸丑朔,日有食之。四月癸巳,涼州都督、長樂郡王幼良有罪,伏誅。五月癸丑,敕中書令、侍中朝堂受訟辭,有陳事者悉上封。六月辛丑,封德彞薨。甲辰,太子少師蕭瑀為尚書左僕射。是夏,山東旱,免今歲租。七月壬子,吏部尚書長孫無忌為尚書右僕射。八月河南、隴右邊州霜。宇文士及檢校涼州都督。戊戌,貶高士廉為安州大都督。九月庚戌朔,日有食之。辛酉,遣使諸州行損田,賑問下戶。御史大夫杜淹檢校吏部尚書,參議朝政。宇文士及罷。辛未,幽州都督王君廓奔於突厥。十月丁酉,以歲饑減膳。十一月己未,許子弟年十九以下隨父兄之官所。十二月壬午,蕭瑀罷。戊申,利州都督李孝常、右武衛將軍劉德裕謀反,伏誅。



 二年正月辛亥,長孫元忌罷。兵部尚書杜如晦檢校侍中,總監東宮兵馬事。癸丑,吐谷渾寇岷州,都督李道彥敗之。丁巳,徙封恪為蜀王,泰越王,祐燕王。庚午,刑部尚書李靖檢校中書令。二月戊戌,外官上考者給祿。三月戊申朔,日有食之。壬子,命中書門下五品以上及尚書議決死罪。壬戌,李靖為關內道行軍大總管,以備薛延陀。己巳,遣使巡關內,出金寶贖饑民鬻子者還之。庚午,以旱蝗責躬,大赦。癸酉,雨。四月己卯,瘞隋人暴骸。壬寅,朔方人梁洛仁殺梁師都以降。六月甲申,詔出使官稟食其家。庚寅,以子治生,賜是日生子者粟。辛卯,辰州刺史裴虔通以弒隋煬帝削爵,流驩州。七月戊申,萊州刺史牛方裕、絳州刺史薛世良、廣州長史唐奉義、虎牙郎將高元禮,以宇文化及之黨,皆除名,徙於邊。八月甲戌,省冤獄於朝堂。辛丑,立二王后廟,置國官。九月壬子,以有年,賜酺三日。十月庚辰,杜淹薨。戊子,殺瀛州刺史盧祖尚。十一月辛酉,有事於南郊。十二月壬辰,黃門侍郎王珪守侍中。癸巳,禁五品以上過市。



 三年正月丙午,以旱避正殿。癸丑,官得上下考者,給祿一年。戊午,享於太廟。癸亥,耕藉田。辛未,裴寂罷。二月戊寅,房玄齡為尚書左僕射,杜如晦為右僕射,尚書右丞魏徵為秘書監,參預朝政。三月己酉,慮囚。四月乙亥,太上皇徙居於大安宮。甲午,始御太極殿。戊戌,賜孝義之家粟五斛,八十以上二斛,九十以上三斛,百歲加絹二匹,婦人正月以來產子者粟一斛。五月乙丑,周王元方薨。六月戊寅,以旱慮囚。己卯,大風拔木。壬午,詔文武官言事。八月己巳朔,日有食之。丁亥,李靖為定襄道行軍大總管,以伐突厥。九月丁巳,華州刺史柴紹為勝州道行軍總管,以伐突厥。十一月庚申,並州都督李世勣為通漠道行軍總管,華州刺史柴紹為金河道行軍總管,任城郡王道宗為大同道行軍總管,幽州都督衛孝節為恆安道行軍總管,營州都督薛萬淑為暢武道行軍總管,以伐突厥。十二月癸末,杜如晦罷。閏月癸丑,為死丘者立浮屠祠。辛酉,慮囚。是歲,中國人歸自塞外及開四夷為州縣者百二十餘萬人。



 四年正月丁卯朔,日有食之。癸巳,武德殿北院火。二月己亥,幸溫湯。甲辰,李靖及突厥戰於陰山,敗之。丙午,至自溫湯。甲寅,大赦,賜酺五日。御史大夫溫彥博為中書令,王珪為侍中;民部尚書戴胄檢校吏部尚書,參豫朝政;太常卿蕭瑀為御史大夫,與宰臣參議朝政。丁巳,以旱詔公卿言事。三月甲午,李靖俘突厥頡利可汗以獻。四月戊戌,西北君長請上號為「天可汗」。六月乙卯,發卒治洛陽宮。七月甲子朔,日有食之。癸酉,蕭瑀罷。甲戌,太上皇不豫,廢朝。辛卯,疾愈,賜都督刺史文武官及民年八十以上、教子表門閭者有差。八月甲寅,李靖為尚書右僕射。九月庚午,瘞長城南隋人暴骨。己卯,如隴州。壬午,禁芻牧於古明君、賢臣、烈士之墓者。十月壬辰,赦岐、隴二州,免今歲租賦,降咸陽、始平、武功死罪以下。辛卯,獵於貴泉穀。甲辰,獵於魚龍川,獻獲於大安宮。乙卯,免武功今歲租賦。十一月壬戌,右衛大將軍侯君集為兵部尚書,參議朝政。甲子,至自隴州。戊寅,除鞭背刑。十二月甲辰,獵於鹿苑。乙巳,至自鹿苑。是歲,天下斷死罪者二十九人。



 五年正月癸酉,獵於昆明池。丙子,至自昆明池,獻獲於大安宮。二月己酉,封弟元裕為鄶王,元名譙王,靈夔魏王,元祥許王,元曉密王。庚戌,封子愔為梁王,貞漢王,惲郯王,治晉王,慎申王,囂江王,簡代王。四月壬辰,代王簡薨。五月乙丑,以金帛購隋人沒於突厥者,以還其家。八月甲辰,遣使高麗,祭隋人戰亡者。戊申,殺大理丞張蘊古。十一月丙子,有事於南郊。十二月丁亥,詔:「決死刑,京師五覆奏,諸州三覆奏,其日尚食毋進酒肉。」壬寅,幸溫湯。癸卯,獵於驪山,賜新豐高年帛。戊申,至自溫湯。癸丑,赦關內。



 六年正月乙卯朔,日有食之。癸西,靜州山獠反,右武衛將軍李子和敗之。三月,侯君集罷。戊辰,如九成宮。丁丑,降雍、岐、豳三州死罪以下,賜民八十以上粟帛。五月,魏徵檢校侍中。六月己亥,豐王元亨薨。辛亥,江王囂薨。七月己巳,詔天下行鄉飲酒。九月己酉,幸慶善宮。十月,侯君集起復。卯,至自慶善宮。十二月辛未,慮囚,縱死罪者歸其家。是歲,諸羌內屬者三十萬人。



 七年正月戊子,斥宇文化及黨人之子孫勿齒。辛丑,賜京城酺三日。三月丁卯,雨土。三月戊子,王珪罷。庚寅,魏徵為侍中。五月癸未,如九成宮。六月辛亥,戴胄薨。八月辛未,東酉洞獠寇邊,右屯衛大將軍張士貴為龔州道行軍總管以討之。九月,縱囚來歸,皆赦之。十月庚申,至自九成宮。乙丑,京師地震。十一月壬辰,開府儀同三司長孫無忌為司空。十二月甲寅,幸芙蓉園。丙辰,獵於少陵原。戊午,至自少陵原。



 八年正月辛丑,張士貴及獠戰,敗之。壬寅,遣使循省天下。二月乙巳,皇太子加元服。丙午,降死罪以下,賜五品以上子為父後者爵一級,民酺三日。三月庚辰,如九成宮。五月辛未朔,日有食之。是夏,吐谷渾寇涼州,左驍衛大將軍段志玄為酉海道行軍總管,左驍衛將軍樊興為赤水道行軍總管,以伐之。七月,隴右山崩。八月甲子,有星孛於虛、危。十月,作永安宮。甲子,至自九成宮。十一月辛未,李靖罷。己丑,吐谷渾寇涼州,執行人鴻臚丞趙德楷。十二月辛丑,特進李靖為西海道行軍大總管,侯君集為積石道行軍總管,任城郡王道宗為鄯善道行軍總管,膠東郡公道彥為赤水道行軍總管,涼州都督李大亮為且末道行軍總管,利州刺史高甑生為鹽澤道行軍總管,以伐吐谷渾。丁卯,從太上皇閱武於城西。



 九年正月,黨項羌叛。二月,長孫無忌罷。三月庚辰,洮州羌殺刺史孔長秀,附於吐谷渾。壬午,大赦。乙酉,高甑生及羌人戰,敗之。閏四月丙寅朔,日有食之。五月,長孫無忌起復。庚子,太上皇崩,皇太子聽政。壬子,李靖及吐谷渾戰,敗之。七月庚子,鹽澤道行軍副總管劉德敏及羌人戰,敗之。十月庚寅,葬太武皇帝於獻陵。十一月壬戌,特進蕭瑀參豫朝政。



 十年正月甲午,復聽政。癸丑,徙封元景為荊王,元昌漢王,元禮徐王,元嘉韓王,元則彭王,元懿鄭王,元軌霍王,元鳳虢王,元慶道王,靈夔燕王,恪吳王,泰魏王,祐齊王,愔蜀王,惲蔣王,貞越王,慎紀王。三月癸丑,出諸王為都督。六月壬申,溫彥博為尚書右僕射,太常卿楊師道為侍中。魏徵罷為特進,知門下省事,參議朝章國典。己卯,皇后崩。十一月庚寅,葬文德皇后於昭陵。十二月,蕭瑀罷。庚辰,慮囚。



 十一年正月丁亥,徙封元裕為鄧王,元名舒王。庚子,作飛仙宮。乙卯,免雍州今歲租賦。二月丁巳,營九摐山為陵,賜功臣、密戚陪塋地及秘器。甲子,如洛陽宮。乙丑,給民百歲以上侍五人。壬午,獵於鹿臺嶺。三月丙戌朔,日有食之。癸卯,降洛州囚見徒,免一歲租、調。辛亥,獵於廣成澤。癸丑,如洛陽宮。』六月甲寅,溫彥博薨。丁巳,幸明德宮。己未,以諸王為世封刺史。戊辰,以功臣為世封刺史。己巳,徙封元祥為江王。七月癸未,大雨,水,穀、洛溢。乙未,詔百官言事。壬寅,廢明德宮之玄圃院,賜遭水家。丙午,給亳州老子廟、兗州孔子廟戶各二十以奉享,復涼武昭王近墓戶二十以守衛。九月丁亥,河溢,壞陜州河北縣,毀河陽中潭,幸白司馬阪觀之,賜瀕河遭水家粟帛。十月癸丑,賜先朝謀臣武將及親戚亡者塋陪獻陵。十一月辛卯,如懷州。乙未,獵於濟源麥山。丙午,如洛陽宮。



 十二年正月乙未,叢州地震。癸卯,松州地震。二月癸亥,如河北縣,觀底柱。甲子,巫州獠反,夔州都督齊善行敗之。乙丑,如陜州。丁卯,觀鹽池。庚午,如蒲州。甲戌,如長春宮。免朝邑今歲租賦,降囚罪。乙亥,獵於河濱。閏月庚辰朔,日有食之。丙戌,至自長春宮。七月癸酉,吏部尚書高士廉為尚書右僕射。八月壬寅,吐蕃寇松州,侯君集為當彌道行軍大總管,率三總管兵以伐之。九月辛亥,闊水道行軍總管牛進達及吐蕃戰於松州,敗之。十月己卯,獵於給平,賜高年粟帛。乙未,至自始平。鈞州山獠反,桂州都督張寶德敗之。十一月己巳,明州山獠反,交州都督李道彥敗之。十二月辛巳,壁州山獠反,右武候將軍上官懷仁討之。是歲,滁、豪二州野蠶成繭。



 十三年正月乙巳,拜獻陵,赦三原及行從,免縣人今歲租賦,賜宿衛陵邑郎將、三原令爵一級。丁未,至自獻陵。二月庚子,停世封刺史。三月乙丑,有星孛于畢、昴。四月戊寅,如九成宮。甲申,中郎將阿史那結社率反,伏誅。壬寅,雲陽石然。五月甲寅,以旱避正殿,詔五品以上言事,減膳,罷役,理囚,賑乏,乃雨。六月丙申,封弟元嬰為滕王。八月辛未朔,日有食之。十月甲申,至自九成宮。十一月辛亥,楊師道為中書令。戊辰,尚書左丞劉洎為黃門侍郎,參知政事。十二月壬申,侯君集為交河道行軍大總管,以伐高昌。乙亥,封子福為趙王。壬辰,獵於咸陽。癸巳,至自咸陽。是歲,滁州野蠶成繭。



 十四年正月庚子,有司讀時令。甲寅,幸魏王泰第,赦雍州長安縣,免延康里今歲租賦。二月丁丑,觀釋奠於國學,赦大理、萬年縣,賜學官高第生帛。壬午,幸溫湯。辛卯,至自溫湯。乙未,求梁皇偘褚仲都、周熊安生沈重、陳沈文阿周弘正張譏、隋何妥劉焯劉炫之後。三月,羅、竇二州獠反,廣州總管黨仁弘敗之。五月壬寅,徙封靈夔為魯王。六月,滁州野蠶成繭。乙酉,大風拔木。八月庚午,作襄城宮。癸酉,侯君集克高昌。九月癸卯,赦高昌部及士卒父子犯死、期犯流、大功犯徒、小功緦麻犯杖,皆原之。閏十月乙未,如同州。甲辰,獵於堯山。庚戌,至自同州。十一月甲子,有事於南郊。十二月丁酉,侯君集俘高昌王以獻,賜酺三日。癸卯,獵於樊川。乙巳,至自樊川。



 十五年正月辛巳,如洛陽宮,次溫湯。衛士崔卿、刁文懿謀反,伏誅。三月戊辰,如襄城宮。四月辛卯,詔以來歲二月有事於泰山。乙未,免洛州今歲租,還戶故給復者加給一年,賜民八十以上物,惸獨鰥寡疾病不能自存者米二斛。慮囚。六月己酉,有星孛於太微。丙辰,停封泰山,避正殿,減膳。七月丙寅,宥周、隋名臣及忠列子孫貞觀以後流配者。十月辛卯,獵於伊闕。壬辰,如洛陽宮。十一癸酉,薛延陀寇邊,兵部尚書李世勣為朔州道行軍總管,右衛大將軍李大亮為靈州道行軍總管,涼州都督李襲譽為涼州道行軍總管,以伐之。十二月戊子,至自洛陽宮。庚子,命三品以上嫡子事東宮。辛丑,慮囚。甲辰,李世勣及薛延陀戰於諾真水,敗之。乙巳,贈戰亡將士官三轉。



 十六年正月乙丑,遣使安撫西州。戊辰,募戍西州者,前犯流死亡匿,聽自首以應募。辛未,徙天下死罪囚實西州。中書舍人岑文本為中書侍郎,專典機密。六月戊戌,太白晝見。七月戊午,長孫無忌為司徒,房玄齡為司空。十一月丙辰,獵於武功。壬戌,獵於岐山之陽。甲子,賜所過六縣高年孤疾氈衾粟帛,遂幸慶善宮。庚午,至自慶善宮。十二月癸卯,幸溫湯。甲辰,獵於驪山。乙己,至自溫湯。



 十七年正月戊辰,魏徵薨。代州都督劉蘭謀反,伏誅。二月己亥,慮囚。戊申,圖功臣於凌煙閣。三月壬子,禁送終違令式者。丙辰,齊王祐反,李世勣討之。甲子,以旱遣使覆囚決獄。乙丑,齊王祐伏誅,縱復齊州一年。四月乙酉,廢皇太子為庶人,漢王元昌、侯君集等伏誅丙。戌,立晉王治為皇太子,大赦,賜文武官及五品以上子為父後者爵一級,民八十以上粟帛,酺三日。丁亥,楊師道罷。己丑,特進蕭瑀為太子太保,李世勣為太子詹事:同中書門下三品。庚寅,謝承乾之過於太廟。癸巳,降封魏王泰為東萊郡王。六月己卯朔,日有食之。壬辰,葬隋恭帝。甲午,以旱避正殿,減膳,詔京官五品以上言事。丁酉,高士廉同中書門下三品,平章政事。閏月丁巳,詔皇太子典左右屯營兵。丙子,徙封泰為順陽郡王。七月丁酉,房玄齡罷。八月庚戌,工部尚書張亮為刑部尚書,參豫朝政。十月丁未,建諸州邸於京城。丁巳,房玄齡起復。十一月己卯,有事於南郊。壬午,賜酺三日,以涼州獲瑞石,赦涼州。十二月庚申,幸溫湯。庚午,至自溫湯。



 十八年正月乙未,如鐘官城。庚子,如鄠。壬寅,幸溫湯。二月己酉,如零口。乙卯,至自零口。丁巳,給復突厥、高昌部人隸諸州者二年。四月辛亥,如九成宮。七月甲午,營州都督張儉率幽、營兵及契丹、奚以伐高麗。八月壬子,安酉都護郭孝恪為西州道行軍總管,以伐焉耆。甲子,至自九成宮。丁卯,劉洎為侍中,岑文本為中書令,中書侍郎馬周守中書令。九月,黃門侍郎褚遂良參豫朝政。辛卯,郭孝恪及焉耆戰,敗之。十月辛丑朔,日有食之。癸卯,宴雍州父老於上林苑,賜粟帛。甲寅,如洛陽宮。己巳,獵於天池。十一月戊寅,慮囚。庚辰,遣使巡問鄭、汝、懷、澤四州高年,宴賜之。甲午,張亮為平壤道行軍大總管,李世勣、馬周為遼東道行軍大總管,率十六總管兵以伐高麗。十二月壬寅,庶人承乾卒。戊午,李思摩部落叛。



 十九年二月庚戌,如洛陽宮,以伐高麗。癸丑,射虎於武德北山。乙卯,皇太子監國於定州。丁巳,賜所過高年鰥寡粟帛,贈比干太師,謚忠烈。三月壬辰,長孫無忌攝侍中,吏部尚書楊師道攝中書令。四月癸卯,誓師於幽州,大饗軍。丁未,岑文本薨。癸亥,李世勣克蓋牟城。五月己巳,平壤道行軍總管程名振克沙卑城。庚午,次遼澤,瘞隋人戰亡者。乙亥,遼東道行軍總管張君乂有罪,伏誅。丁丑,軍於馬首山。甲申,克遼東城。六月丁酉,克白巖城。已未,大敗高麗於安市城東南山,左武衛將軍王君愕死之。辛酉,賜酺三日。七月壬申,葬死事官,加爵四級,以一子襲。九月癸未,班師。十月丙午,次營州,以太牢祭死事者。丙辰,皇太子迎謁於臨渝關。戊午,次漢武臺,刻石紀功。十一月癸酉,大饗軍於幽州。庚辰,次易州。癸未,平壤道行軍總管張文乾有罪,伏誅。丙戌,次定州。丁亥,貶楊師道為工部尚書。十二月戊申,次並州。己未,薛延陀寇夏州,左領軍大將軍執失思力敗之。庚申,殺劉洎。



 二十年正月辛未,夏州都督喬師望及薛延陀戰,敗之。丁丑,遣使二十二人,以六條黜陟於天下。庚辰,赦並州,起義時編戶給復三年,後附者一年。二月甲午,從伐高麗無功者,皆賜勛一轉。庚申,賜所過高年鰥寡粟。三月己巳,至自高麗。庚午,不豫,皇太子聽政。己丑,張亮謀反,伏誅。閏月癸巳朔,日有食之。六月乙亥,江夏郡王道宗、李世勣伐薛延陀。七月辛亥,疾愈。李世勣及薛延陀戰,敗之。八月甲子,封孫忠為陳王。己巳,如靈州。庚辰,次涇州,賜高年鰥寡粟帛。丙戌,逾隴山關,次瓦亭,觀牧馬。丁亥,許陪陵者子孫從葬。九月辛卯,遣使巡察嶺南。甲辰,鐵勒諸部請上號為「可汗」。辛亥,靈州地震。十月,貶蕭瑀為商州刺史。丙戌,至自靈州。十一月己丑,詔:「祭祀、表疏,籓客、兵馬、宿衛行魚契給驛,授五品以上官及除解,決死罪,皆以聞,餘委皇太子。」



 二十一年正月壬辰,高士廉薨。丁酉,詔以來歲二月有事於泰山。甲寅,以鐵勒諸部為州縣,賜京師酺三日。慮囚,降死罪以下。二月丁丑,皇太子釋菜於太學。三月戊子,左武衛大將軍牛進達為青丘道行軍大總管,李世勣為遼東道行軍大總管,率三總管兵以伐高麗。四月乙丑,作翠微宮。五月戊子,幸翠微宮。壬辰,命百司決事於皇太子。庚戌,李世勣克南蘇、木底城。六月丁丑,遣使鐵勒諸部購中國人陷沒者。七月乙未,牛進達克石城。丙申,作玉華宮。庚戌,至自翠微宮。八月,泉州海溢。壬戌,停封泰山。九月丁酉,封子明為曹王。十月癸丑,褚遂良罷。十一月癸卯,進封泰為濮王。十二月戊寅,左驍衛大將軍契苾何力為昆丘道行軍大總管,率三總管兵以伐龜茲。



 二十二年正月庚寅,馬周薨。戊戌,幸溫湯。己亥,中書舍人崔仁師為中書侍郎,參知機務。丙午,左武衛大將軍薛萬徹為青丘道行軍大總管,以伐高麗。長孫無忌檢校中書令,知尚書、門下省事。戊申,至自溫湯。二月褚遂良起復。乙卯,見京城父老,勞之,蠲今歲半租,畿縣三之一。丁卯,詔度遼水有功未酬勛而犯罪者與成官同。乙亥,幸玉華宮。己卯,獵於華原。流崔仁師於連州。三月丁亥,赦宜君給復縣人自玉華宮苑中遷者三年。四月丁巳,松州蠻叛,右武候將軍梁建方敗之。六月丙寅,張行成存問河北從軍者家,令州縣為營農。丙子,薛萬徹及高麗戰於泊灼城,敗之。七月甲申,太白晝見。壬辰,殺華州刺史李君羨。癸卯,房玄齡薨。八月己酉朔,日有食之。辛未,執失思力伐薛延陀餘部於金山。九月庚辰,昆丘道行軍總管阿史那社爾及薛延陀餘部處月、處蜜戰,敗之。己亥,褚遂良為中書令。壬寅,眉、邛、雅三州獠反,茂州都督張士貴討之。十月癸丑,至自玉華宮。己巳,阿史那社爾及龜茲戰,敗之。十二月辛未,降長安、萬年徒罪以下。閏月癸巳,慮囚。



 二十三年正月辛亥,阿史那社爾俘龜茲王以獻。三月己未,自冬旱,至是雨。辛酉,大赦。丁卯,不豫,命皇太子聽政於金液門。四月巳亥,幸翠微宮。五月戊午,貶李世勣為疊州都督。己巳,皇帝崩於含風殿,年五十三。庚午,奉大行御馬輿還京師。禮部尚書於志寧為侍中,太子少詹事張行成兼侍中,高季輔兼中書令。壬申,發喪,謚曰文。上元元年,改謚文武聖皇帝;天寶八載,謚文武大聖皇帝;十三載,增謚文武大聖大廣孝皇帝。



 贊曰:甚矣,至治之君不世出也!禹有天下,傳十有六王,而少康有中興之業。湯有天下,傳二十八王,而其甚盛者,號稱三宗。武王有天下,傳三十六王,而成、康之治與宣之功,其餘無所稱焉。雖《詩》、《書》所載,時有闕略,然三代千有七百餘年,傳七十餘君,其卓然著見於後世者,此六七君而已。嗚呼,可謂難得也!唐有天下,傳世二十,其可稱者三君,玄宗、憲宗皆不克其終,盛哉,太宗之烈也!其除隋之亂,比跡湯、武;致治之美,庶幾成、康。自古功德兼隆,由漢以來未之有也。至其牽於多愛,復立浮圖,好大喜功,勤兵於遠,此中材庸主之所常為。然《春秋》之法,常責備於賢者,是以後世君子之欲成人之美者,莫不嘆息於斯焉。



\end{pinyinscope}