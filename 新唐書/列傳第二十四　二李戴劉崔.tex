\article{列傳第二十四 二李戴劉崔}

\begin{pinyinscope}

 李綱,字文紀,觀州蓚人。少慷慨,尚風節。始名瑗,慕張綱為人井,它使人陷入「概念的自我發展的圈套」。主張通過民主改,改焉。仕周為齊王憲參軍事。宣帝將殺憲,召僚屬誣左其罪,綱矢死無橈辭。及憲誅,露車載尸,故吏奔匿,綱撫棺號慟,為瘞訖,乃去。



 事隋為太子洗馬。太子勇宴宮臣,左庶子唐令則奏琵琶,又歌《武媚娘曲》。綱曰:「令則官調護,乃自比倡優,進淫聲,惑視聽,誠使上聞之,豈不為殿下累乎?臣請正其罪。」勇曰:「置之,我欲為樂耳!」後勇廢,文帝切讓,官屬無敢對,綱獨曰:「陛下不素教,故太子至此。太子資中人,得賢者輔而善,得不肖導而惡,奈何歌舞鷹犬纖兒使日侍側?何特太子罪邪?」帝曰:「以汝為洗馬,何不擇人?」綱曰:「臣非東宮得言者。」帝曰:「朕過矣!」擢尚書右丞。時楊素、蘇威用事,綱據正不詭迎隨,素等多憾。會大將軍劉方討林邑,素言林邑多珍貲,非綱不可任,遂署行軍司馬。方揣素指,數危辱之,幾殆。軍還,不得調。稍除齊王府司馬。復詔出南海,應接林邑。久不召,乃身入奏。威劾綱擅去所部,以屬吏。會赦免,屏居鄠。大業末,賊帥何潘仁劫為長史。



 高祖平京師,綱上謁,授丞相府司錄參軍,封新昌縣公,領選舉。受禪,拜禮部尚書兼太子詹事。齊王元吉為並州總管,縱左右攘奪,民愁苦,宇文歆諫,不聽,騰狀顯言,王坐免。俄而復留,下危惴。劉武周入太原,元吉懼,棄軍奔京師,並州陷。帝怒,謂綱曰:「王年少,不習事,故以歆及竇誕佐之。太原,興王地,兵十萬,粟支十年,奈何一旦棄去?歆建此計,我當斬於軍。」綱曰:「王過惡,誕養成之。歆事王淺,有闕必諍。今賴歆計,使陛下不失愛子,且有功,又可加罪乎?」翼日,帝悟,引綱升御榻,勞曰:「卿不言,我幾濫罰。」於是釋歆,然猶貸誕也。帝以舞工安叱奴為散騎常侍,綱諫曰:「周家均工樂胥不得預士伍,雖復妙如師襄,才如子野,皆繼世不易業。故魏武使禰衡擊鼓,衡先解朝衣,曰:『不敢以先王法服為伶人衣。』齊高緯封曹妙達為王,以安馬駒開府,有國家者,可為鑒戒。今新造天下,開太平之基,功臣賞未及遍,高才猶伏草茅,而先令舞胡鳴玉曳組,位五品,趨丹地,殆非創業垂統、貽子孫之道也。」帝不納。



 綱在東宮,太子建成尤加禮,嘗游溫湯,綱疾不從。有進魚者,太子使膾之,唐儉、趙元楷自言其能。太子曰:「操刀膾鯉和鼎味,公等善之。若弼諧審諭,固屬綱矣。」遣使賜絹二百匹。後太子浸狎亡賴,猜間朝廷,綱頻諫不見聽,遂乞骸骨。帝罵曰:「卿為潘仁長史,而羞朕尚書邪?」綱頓首曰:「潘仁,賊也,志殘殺,然每諫輒止,為其長史,故無愧。陛下功成,厚自伐,臣言如持水內石,敢久為尚書乎?且臣事東宮,東宮又與臣忤,是以上印綬。」帝謝曰:「知公直士,幸卒輔吾兒。」乃拜太子少保,尚書、詹事如故。綱上書太子曰:「綱老矣,幸未就木,備位保傅,冀得效愚鄙。日殿下飲酒過量,非養生之道。凡為人子,務孝謹,以慰上心,不宜聽受邪說,與朝廷生槊間。」太子覽書不懌,所為益縱。綱悒悒不自賴,固請老,優詔解尚書。帝以綱隋名臣,手敕未嘗名。



 貞觀四年,復為少師。以足疾賜步輿,聽乘至閤,問以政事。詣東宮,太子承乾為拜,每聽政,必詔綱與房玄齡、王珪侍坐。嘗言曰:「托六尺之孤,寄百里之命,古人為難,綱以為易!」故發言陳事,毅然不可奪。及疾,帝遣玄齡至家存問。明年卒,年八十五,贈開府儀同三司,謚曰貞,太子為立碑。



 初,齊王憲女嫠居,綱厚恤之。及卒,女被發號哭,如喪其親然。綱在隋,宦不進,筮之得《鼎》。筮人曰:「君當為卿輔,然待易姓乃如志。仕不知退,折足為敗。」故綱雖顯於唐,數稱疾辭位云。孫安仁、安靜。



 安仁,永徽中為太子左庶子,太子忠廢還邸,寮屬奔散,獨安仁泣拜而去。終恆州刺史。安靜,天授中為右衛將軍。武氏革命,群臣皆勸進,安靜獨無所請。及收系獄,來俊臣問狀,安靜曰:「正以我唐舊臣,殺之可也。若詰其狀,吾誰欺?」俊臣誣殺之。會昌中,錄忠臣後,訪子孫已絕,乃贈安靜太子少師。自綱五世同居,安仁、安靜復以義烈聞,世稱李氏不衰。



 李大亮,京兆涇陽人。祖琰,為魏度支尚書。大亮有文武才略,隋末,署龐玉行軍兵曹。李密寇東都,玉戰敗,大亮被禽。賊將張弼異之,就執百餘人皆死,獨釋大亮,引與語,遂定交。



 高祖入關,大亮自歸,授土門令。方歲饑,境多盜賊。大亮招亡散,撫貧瘠,賣所乘馬,稍稍資業之,勸墾田,歲大熟。間出擊盜,所至輒平。秦王行北境,下書獎勞,賜馬五乘,帛五十段。頃之,胡賊大至,大亮度不能拒,乃單馬詣營說豪帥,為分別禍福,賊眾感服,遂相率降。大亮殺所乘馬與之食,至步而返。帝聞之悅,擢金州總管府司馬。王弘烈據襄陽,詔大亮安撫樊、鄧,因圖之,進擊,下十餘城。遷安州刺史。復使徇廣州,至九江,會輔公祏反,以計禽其將張善安。公祏方圍猷州,刺史左難當固守,大亮率兵擊走之。遷越州都督。



 貞觀初,徙交州,封武陽縣男。召授太府卿,復出涼州都督。嘗有臺使見名鷹,諷大亮獻之。大亮密表曰:「陛下絕畋獵久矣,而使者求鷹。信陛下意邪,乃乖昔旨;如其擅求,是使非其才。」太宗報書曰:「有臣如此,朕何憂!古人以一言之重訂千金,今賜胡瓶一,雖亡千鎰,乃朕所自御。」又賜荀悅《漢紀》,曰:「悅論議深博,極為政之體,公宜繹味之。」時突厥亡,帝遂欲懷四夷,諸部降者,人賜袍一領、帛五匹,首領拜將軍、中郎將,列五品者贏百員。又置降胡河南。詔大亮為西北道安撫大使,使以綏大度設、拓設、泥熟特勒及七姓種落之未附者,峙糧磧口賑其饑。大亮上言:「臣聞欲綏遠者必自近。中國,天下本根,四夷猶枝葉也。殘本根,厚枝葉,而曰求安,未之有也。屬者突厥傾國入朝,陛下不即俘江淮變其俗,而加賜物帛,悉官之,引處內地,豈久安計哉?今伊吾雖臣,遠在荒鹵。臣以為諸稱籓請附者,宜羈縻受之,使居塞外,畏威懷德,永為籓臣。謂之荒服者,故臣而不內,所謂行虛惠,收實福。河西積困夷狄,州縣蕭條,加因隋亂,殘耗已甚。臣愚願停招慰,省勞役,使邊人得就農畮,此中國利也。」帝納其計。



 八年,為劍南道巡省大使。會討吐谷渾,為河東道行軍總管,與李靖俱出北道,涉青海,觀河源,與虜遇蜀渾山,大戰,破之,俘其名王,獲雜畜數萬,進爵為公。拜右衛大將軍。晉王為皇太子,詔大亮兼右衛率,又兼工部尚書,身三職,宿衛兩宮。每番直,常假寐。帝勞曰:「公在,我得酣臥。」



 十八年,幸洛陽,詔副房玄齡居守。玄齡稱「有王陵、周勃節,可倚大事」。俄寢疾,帝親和藥,驛賜之。臨終,表請罷遼東役;又言京師宗廟所在,願以關中為意。就稿,嘆曰:「吾聞男子不死婦人手!」命屏左右,言終卒,年五十九。將斂,家無珠玉為含,惟貯米五斛、布三十端。帝哭為慟。贈兵部尚書、秦州都督,謚曰懿,陪葬昭陵。



 大亮性忠謹,外若不能言,而內剛烈,不可干非其義。對天子爭是非,無回撓。至妻子未始見惰容,事兄嫂以禮聞。位通顯,居陋狹甚。在越州寫書數百卷,及去,留都督署。初,破公祏,以功賜奴婢百口,謂曰:「而曹皆衣冠子女,不幸破亡,吾何忍錄而為隸乎?」縱遣之。高祖聞,咨美,更賜俚婢二十。後破吐谷渾,復賜奴婢百五十口,悉以遺親戚。葬宗族無後者三十餘柩,貲襚加焉。嘗以張弼脫其死,及貴,念有以報之。時弼為將作丞,匿不見,大亮求之不能得。一日,識諸塗,持弼泣,悉推家財與之,弼拒不受。乃言於帝曰:「臣及事陛下,張弼力也,願悉臣官爵授之。」帝為遷弼中郎將、代州都督。世皆賢大亮能報,而多弼不自伐也。歿後,所育孤姓為大亮行服如所親者十餘人。



 兄子道裕,貞觀末為將作匠。有告張亮反者,詔百官議。皆言亮當誅,獨道裕謂反形未具。帝怒不暇省,斬之。歲餘,刑部侍郎缺,宰相屢進名,不可。帝曰:「朕得之矣。是嘗議張亮者,朕時雖不從,今尚悔之。」遂命道裕。終大理卿。



 大亮族孫迥秀。迥秀,字茂之。及進士第,又中英才傑出科。調相州參軍事。累轉考功員外郎。武后愛其材,遷鳳閣舍人。大足初,檢校夏官侍郎,仍領選,銓汰文武,號稱職,進同鳳閣鸞臺平章事。張易之兄弟貴驕,因橈意諧媚,士論頓減。俄坐贓貶廬州刺史。易之誅,貶衡州長史。中宗即位,召授將作少監。累遷鴻臚卿、脩文館學士。出朔方道行軍大總管,還拜兵部尚書。卒,年五十,贈侍中。迥秀少聰悟,多通賓客。喜飲酒,雖多不亂,當時稱其風流。母少賤,妻嘗詈媵婢,母聞不樂,迥秀即出其妻。或問之,答曰:「娶婦要欲事姑,茍違顏色,何可留?」武后嘗遣內人候其母,或迎置宮中。後所居堂產芝草,犬乳鄰貓,中宗以為孝感,旌大門閭。子齊損,開元中以謀逆誅。



 戴胄,字玄胤,相州安陽人。性堅正,幹局明強,善簿最。隋末,為門下錄事,納言蘇威、黃門侍郎裴矩厚禮之。為越王侗給事郎。王世充謀篡,胄說曰:「君臣大分均父子,休戚同之。公當社稷之任,與存與亡,正在今日。願尊輔王室,擬伊、周以幸天下。」世充詭曰:「善。」俄肋九錫,胄又切諫,不納。出為鄭州長史,使與王行本守武牢。秦王攻拔之,引為府士曹參軍,封武昌縣男。大理少卿缺,太宗曰:「大理,人命所系,胄清直,其人哉。」即日命胄。長孫無忌被召,不解佩刀入東上閤。尚書右僕射封德彞論監門校尉不覺,罪當死,無忌贖。胄曰:「校尉與無忌罪均,臣子於尊極不稱誤。法著:御湯劑、飲食、舟船,雖誤皆死。陛下錄無忌功,原之可也。若罰無忌,殺校尉,不可謂刑。」帝曰:「法為天下公,朕安得阿親戚!」詔復議,德彞固執,帝將可。胄曰:「不然。校尉緣無忌以致罪,法當輕;若皆誤,不得獨死。」繇是與校尉皆免。



 時選者盛集,有詭資廕冒牒取調者,詔許自首;不首,罪當死。俄有詐得者,獄具,胄以法當流。帝曰:「朕詔不首者死,而今當流,是示天下不以信,卿賣獄邪?」胄曰:「陛下登殺之,非臣所及。既屬臣,敢虧法乎?」帝曰:「卿自守法,而使我失信,奈何?」胄曰:「法者,布大信於人;言乃一時喜怒所發。陛下以一朝忿,將殺之,既知不可而寘於法,此忍小忿、存大信也。若阿忿違信,臣為陛下惜之。」帝大感寤,從其言。胄犯顏據正,數查,參處法意,至析秋毫,隨類指擿,言若泉湧,帝益重之。遷尚書左丞。矜其貧,特詔賜錢十萬。會僕射蕭瑀免,封德彞卒,帝謂胄曰:「尚書總國綱維,失一事,天下有受其弊者。今以令、僕委卿,宜副朕舉。」胄明敏,長於操決,無宿疑。議者美其振職,謂武德以來殆無其輩。復拜諫議大夫,與魏徵更日供奉。進民部尚書。杜如晦遺言,請以選舉委胄,由是檢校吏部尚書。然好抑文雅,獎法吏,時以寡學為訾。



 貞觀四年,以本官參豫朝政,進爵郡公。帝將脩復洛陽宮,胄上疏諫曰:「比關中、河外置軍圍,強夫富室悉為兵,九成之役又興,司農、將作見丁無幾。大亂之後,戶口單破,一人就役,舉室捐業。籍軍者督戎仗,課役者責糧齎,竭貲經紀,猶不能濟。七月以來,霖潦未止,濱河南北,田正洿下,年之有亡未可知。壯者盡行,賦調不給,則帑藏虛矣。今宮殿足庇風雨、容羽衛,數年後成,猶不謂晚,何憚而遽自生勞擾邪?」帝覽奏,罷役。胄所敷內,緣政得失,咸有可觀。奏已,即削稿,秘外莫知。帝嘗謂左右曰:「胄於我非肺腑親,然事之機切無不聞,惟其忠概所激耳。」



 七年,卒,帝為舉哀,贈尚書右僕射,追封道國公,謚曰忠;以第舍陋不容祭,詔有司為立廟。聘其女為道王妃。房玄齡、魏徵與胄善,每至生平故處,輒流涕。



 胄無子,以兄子至德為後。



 至德,乾封中累遷西臺侍郎、同東西臺三品。閱十數年,父子繼為宰相,世詫其榮。高宗嘗為飛白書賜侍臣,賜至德曰:「泛洪源,俟舟楫」,郝處俊曰:「飛九霄,假六翮」,李敬玄曰「資啟沃,罄丹誠」,崔知悌曰「竭忠節,贊皇猷」,皆見意於辭云。遷尚書右僕射。時劉仁軌為左,人有所訴,率優容之;至德乃詰究本末,理直者密為奏,終不顯私恩。由是,當時多稱仁軌者,號仁軌為「解事僕射」。嘗更日聽訟,有嫗詣省,至德已收牒,嫗乃復取,曰:「初以為解事僕射,今乃非是。」至德笑還之。人伏其長者。或以問,至德答曰:「慶賞刑罰,人主之柄,為臣豈得與人主爭也!」帝知,嘆美之。儀鳳四年卒,詔百官哭其第。贈開府儀同三司、並州大都督,謚曰恭。



 劉洎,字思道,荊州江陵人。初為蕭銑黃門侍郎,南略地嶺表,下五十城,未還而銑敗,遂以城自歸,授南康州都督府長史。



 貞觀七年,擢給事中,封清苑縣男,轉治書侍御史。於時,尚書省詔敕稽壅,按成復下,彌年不能決。洎言:「尚書,萬機本,貞觀初未有令、僕,職並務繁,左丞戴胄、右丞魏徵,應事彈舉,無所回橈,百司震肅不敢懈。比者勛親在位,品非其任,功勢相傾,雖欲自強,先懼囂謗。故郎中嘿奪,惟事咨稟;尚書依違,不得專裁。管轄玩弛,綱紀不振。今宜精選左右丞、兩司郎中,使皆得人,非惟救曠滯之弊,固當矯拂趨競也。」未幾,拜尚書右丞。洎健於職,於是尚書復治如征時。累加銀青光祿大夫、散騎常侍,攝黃門侍郎。



 太宗好持論,與公卿言古今事,必往復難詰、究臧否。洎諫曰:「帝王之與臣庶,聖哲之與庸愚,等級遼絕,勢不倫擬。故課愚對聖,持卑抗尊,雖思自強,不可得已。陛下降慈旨,假柔顏,虛心聽納,猶恐群臣惴縮不敢進。況以神機天辯,飾辭援古而迮其議哉!夫天以無言為尊,聖以不言為德,皆弗欲煩也。且多記損心,多語耗氣,心氣內損,形神外勞,初雖無覺,久且為弊。且今之雍平,陛下力行所至耳。欲其長久,匪由辯博,但當忘愛憎,慎取舍,若貞觀初可矣!」手詔答曰:「非慮無以臨下,非言無以述慮。雖然,驕人輕物,恐由榷論致之。若形神心氣,不為勞也。」



 皇太子初立,洎謂宜尊賢重道,上書曰:「太子宗祧是系,善惡之習,興亡在焉。弗勤於始,將悔於末。故晁錯上書,令通政術;賈誼奏計,務知禮教。今太子孝友仁愛,挺自天姿,然春秋鼎盛,學當有漸。以陛下多才多藝,尚垂精厲志,以博異聞,而太子優游,坐棄白日。陛下每退朝,引見群臣,訪以今古,咨以得失;而太子處內,不接正人,不聞正論,臣所未諭。古者,問安而退,以廣敬也;異宮而處,以遠嫌也。間者,太子一入侍,逾句不出,師傅寮採,具員而已,非所謂愛之也。臣愚以為授以良書,娛以佳賓,使耳所未聞,睹所未見,儲德愈光,群生之福也。」帝於是敕洎與岑文本、馬周遞日直東宮。帝嘗怒苑西監穆裕,有詔斬朝堂,皇太子驟諫。帝喜曰:「朕始得魏徵,朝夕進諫。徵亡,劉洎、岑文本、馬周、褚遂良繼之。兒在吾膝前,見朕悅諫熟矣,故有今日言也。誠習以性成哉!」稍遷侍中。帝忽謂群臣曰:「朕今欲聞己過,卿等為朕言之。」長孫無忌、李勣、楊師道同辭對曰:「陛下以盛德致太平,臣等愚不見其過。」洎曰:「然頃上書有不稱旨,或面窮詰,無不羞汗,恐非所以進言者路。」帝曰:「卿言善,朕能改之。」



 及征遼東,詔兼太子左庶子、檢校民部尚書,輔皇太子監國。帝曰:「以卿輔太子,社稷安危在焉,宜識朕意。」洎曰:「願無憂!即大臣有罪,臣謹按法誅之。」帝怪其語謬,戒曰:「君不密則失臣,臣不密則失身。卿性疏而果,恐以此敗。」洎與褚遂良不相中。帝還,不豫,洎與馬周入候,出見遂良,泣曰:「上體患癰,殊可懼!」遂良即誣奏「洎曰:國家不足慮,正當輔少主行伊、霍事,大臣有異者,誅之。」帝愈,召洎問狀,洎引馬周為左。遂良執不已,帝惑之,乃賜死。方死時,索筆牘,欲自言,有司不敢與。帝後知之,有司皆得罪。顯慶中,其子弘業詣闕訴遂良譖死狀,李義府右之。高宗問近臣,給事中樂彥瑋曰:「辨之,是暴先帝過刑。」事寢。文明初,詔復官爵。



 彥瑋,字德珪,長安人。麟德元年,以西臺侍郎同東西臺三品。數月,罷為大司憲。卒,贈齊州都督。



 贊曰:「劉洎之才之烈,《易》所謂「王臣蹇蹇」者。然性剛疏,輔太子,欲身任安危,以言掩其眾,為媢忌所乘,卒陷罪誅。嗚呼!以太宗之明,蔽於所忿,洎之忠不能自申於上,況其下哉?古人以言為戒,可不慎歟!



 崔仁師,定州安喜人。武德初擢制舉,調管州錄事參軍。陳叔達薦仁師才任史官,遷右武衛錄事參軍,與脩梁、魏史。貞觀初,改殿中侍御史。時青州有男子謀逆,有司捕支黨,累系填獄,詔仁師按覆。始至,悉去囚械,為具食,飲湯瀋,以情訊之,坐止魁惡十餘人,它悉原縱。大理少卿孫伏伽謂曰:「原雪者眾,誰肯讓死?就決而事變,奈何?」仁師曰:「治獄主仁恕,故諺稱『殺人刖足,亦皆有禮』。豈有知枉不申,為身謀哉?使吾以一介易十囚命,固吾願也!」及敕使覆訊,諸囚咸叩頭曰:「崔公仁恕,必無枉者。」舉無異辭。由是知名。遷度支郎中。嘗口陳移用費數千名,太宗怪之,詔黃門侍郎杜正倫持簿,使仁師對唱,無一謬。帝奇之。時校書郎王玄度注《尚書》、《毛詩》,抵孔、鄭舊學,請遂廢。詔諸儒大議,博士以下不能詰。河間王孝恭請與孔、鄭並行,仁師以玄度不經,條不合大義者奏之。玄度報罷。



 遷給事中。時有司以律「反逆者緣坐兄弟沒官」為輕,詔八坐議。咸言漢、魏、晉謀反夷三族,請改從死。仁師曰:「父子天屬,足累其心,此而不恤,何愛兄弟?」房玄齡曰:「祖有廕孫義,則孫祖親重,而兄弟屬輕。今應重者流而輕者死,非用刑意。」遂不改。



 後密請魏王為太子,失帝旨,左遷鴻臚少卿。稍進民部侍郎。及征遼東,副韋挺知海運,又別知河南漕事。仁師以漕路回遠,恐所輸不時至,以便宜發近海租賦餉軍。坐運卒亡命不以聞,除名。帝還至中山,起為中書舍人、檢校刑部侍郎。幸翠微宮,上《清暑賦》以諷。帝稱善,賜帛五十段。二十二年,遷中書侍郎,參知機務,被遇尤渥。中書令褚遂良忌之,會有伏閤訴者,仁師不時上,帝大怒,流連州。永徽初,授簡州刺史,卒。



 子挹,挹子湜。湜字澄瀾。少以文詞稱。第進士,擢累左補闕,稍遷考功員外郎。時桓彥範等當國,畏武三思槊構,引湜使陰汋其奸。中宗稍疏功臣,三思日益寵,湜反以彥範等計告三思,驟遷中書舍人。彥範等被徙,又說三思速殺之以絕人望。三思問誰可使者,乃進其外兄周利貞。利貞往,彥範等皆死。擢利貞御史中丞。湜附托昭容上官氏,數與宣淫於外。景龍二年,遷兵部侍郎,而挹為禮部侍郎。武德以來,父子同為侍郎,惟挹、湜云。俄拜中書侍郎、檢校吏部侍郎、同中書門下平章事,與鄭愔同典選。納賂遺,銓品無序,為御史李尚隱劾奏,貶江州司馬。上官與安樂公主從中申護之,改襄州刺史。未幾,入為尚書左丞。韋氏稱制,復以吏部侍郎同中書門下三品。睿宗立,出為華州刺史。俄除太子詹事。



 初,湜建言山南可引丹水通漕至商州,自商鑱山出石門,抵北藍田,可通挽道。中宗以湜充使,開大昌關,役徒數萬,死者十五。禁舊道不得行,而新道為夏潦奔豗,數摧壓不通。至是論功,加銀青光祿大夫。景雲中,太平公主引為同中書門下三品。進拜中書公。時挹以戶部尚書得謝,而性貪,數為人請托以干湜。湜多不從,由是父子相失。



 玄宗在東宮,數至其第申款密。湜陰附主,時人危之,為寒毛。門下客獻《海鷗賦》以諷,湜稱善而不自悛。帝將誅蕭至忠等,召湜示腹心。弟澄諫曰:「上有所問,慎無隱。」湜不從。及見,對問失旨。至忠等誅,湜徙嶺外。時雍州長史李晉亦坐誅,嘆曰:「此本湜謀,今我死而湜生,何也?」又宮人元稱嘗與湜謀進■於帝。追及荊州賜死,年四十三。



 初,在襄州,與譙王數相問遺。王敗,湜當死,賴劉幽求、張說護免。及為宰相,陷幽求嶺表,密諷廣州都督周利貞殺之,不克。又與太平公主逐張說。其猜毒詭險殆天性,雖蠆虺不若也。



 與弟液、澄、從兄涖並以文翰居要官。每宴私,自比東晉王、謝。嘗曰:「吾一門入仕,歷官未嘗不為第一。丈夫當先據要路以制人,豈能默默受制於人哉!」故進趣不已,至於敗。湜執政時,年三十八,嘗暮出端門,緩轡諷詩。張說見之,嘆曰:「文與位固可致,其年不可及也。」



 液字潤甫,尤工五言詩,湜嘆,因字呼曰:「海子,我家龜龍也!」官至殿中侍御史。坐湜當流,亡命郢州,作《幽征賦》以見意,詞甚典麗。遇赦還,卒。子論,有吏乾,乾元中為州刺史,以治行稱。大歷末,遷同州刺史,為黜陟使庾何所按,議者不直何,故復用為衢州刺史。德宗以舊族耆年,擢大理卿,卒。



 澄本名滌,玄宗改焉。帝在籓,與同里居。出潞州,賓友餞者止國門,而澄獨從至華。及即位,寵暱甚。湜既誅,帝仍念之,用為秘書監。開元二年,欲贈其父挹吏部尚書,宰相持不可,遂用四品禮葬,贈和州刺史。澄侍左右,與諸王不讓席坐,性滑稽善辯,帝恐漏禁中語,以「慎密」字親署笏端。累遷金紫光祿大夫,封安喜縣子。卒,贈兗州刺史。



\end{pinyinscope}