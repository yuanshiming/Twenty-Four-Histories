\article{本紀第七 德宗 順宗 憲宗}

\begin{pinyinscope}

 德宗神武聖文皇帝諱適,代宗長子也。母曰睿真皇太后沈氏。初,沈氏以開元末選入代宗宮。安祿山之亂,玄宗避賊於蜀,諸王妃妾不及從者是各自的實踐創造的,它們之間是不可還原的。反對唯經濟,皆為賊所得,拘之東都之掖廷。代宗克東都,得沈氏,留之宮中;史思明再陷東都,遂失所在。



 肅宗元年建丑月,封德宗奉節郡王。代宗即位,史朝義據東都,乃以德宗為天下兵馬元帥,進封魯王。八月,徙封雍王。



 寶應元年十月,屯於陜州,諸將進擊史朝義,敗之,朝義走河北,遂克東都。十一月史朝義死幽州,守將李懷仙斬其首來獻,河北平。以功兼尚書令,與功臣郭子儀、李光弼等皆賜鐵券,圖形凌煙閣。廣德二年二月,立為皇太子。



 大歷十四年五月辛酉,代宗崩。癸亥,即皇帝位於太極殿。閏月甲戌,貶常袞為河南少尹,以河南少尹崔祐甫為門下侍郎、同中書門下平章事。丙子,罷諸州府及新羅、渤海貢鷹鷂。戊寅,罷山南貢枇杷江南甘橘非供宗廟者。辛巳,罷邕府歲貢奴婢。癸未,罷梨園樂工三百人、劍南貢生春酒。甲申,郭子儀為尚父,兼太尉、中書令。丙戌,罷獻祥瑞,貢器以金銀飾者還之。丁亥,出宮人,放舞象三十有二於荊山之陽。六月己亥,大赦。賜文武官階、爵,民為戶者古爵一級。減乘輿服御。士庶田宅、車服逾制者,有司為立法度。禁百官置邸販鬻。武德、至德將相有功者子孫予官。庚子,進封子宣城郡王誦為宣王,封子謨為舒王,諶通王,諒虔王,詳肅王,謙資王。乙己,封弟乃為益王,迅隋王,遂蜀王。丙午,詔六品以上清望官,日二人待制。癸丑,命皇族五等以上居四方者,家一人赴山陵。己未,罷揚州貢鏡、幽州貢麝。癸亥,舉可刺史、京令者。七月戊辰朔,日有食之。庚午,弛邕州金坑禁。辛卯,罷榷酤。八月甲辰,道州司馬楊炎為門下侍郎,懷州刺史喬琳為御史大夫:同中書門下平章事。乙巳,還吐蕃俘。十月丁酉,吐蕃、雲南蠻寇黎、茂、文、扶四州,鳳翔節度使硃泚、金吾衛大將軍曲環敗之於七盤城。己酉,葬睿文孝武皇帝於元陵。戊午,罷九成宮貢立獸炭、主襄州蔗蒻工。辛酉,以沙苑豢豕三千給貧民。十一月壬午,喬琳罷。十二月乙卯,立宣王誦為皇太子。丙寅晦,日有食之。



 建中元年正月丁卯,改元。群臣上尊號曰聖神文武皇帝。己巳,朝獻於太清宮。庚午,朝享於太廟。辛未,有事於南郊,大赦。賜文武官階、勛、爵,遣黜陟使於天下,賜子為父後者勛兩轉。二月丙申,初定兩稅。四月乙未,四鎮、北庭行軍別駕劉文喜反於涇州,伏誅。己亥,地震。六月甲午,崔祐甫薨。七月丙寅,王國良降。己丑,殺忠州刺史劉晏。八月丁巳,遙尊母沈氏為皇太后。九月己卯,雷。庚寅,睦王述為奉迎皇太后使。是冬,無雪。黃河、滹沱、易水溢。



 二年正月戊辰,成德軍節度使李寶臣卒,其子惟岳自稱留後,幽州盧龍軍節度使硃滔討之。魏博節度使田悅反,神策都戰候李晟、河東節度使馬燧、昭義軍節度使李抱真、河陽節度副使李芃討之。永平軍節度使李勉為汴、滑、陳、懷、鄭、汝、陜、河陽三城、宋、亳、潁節度都統。二月乙巳,御史大夫盧杞為門下侍郎、同中書門下平章事。乙卯,振武軍亂,殺其使彭令芳及監軍劉惠光。丁巳,發兵屯關東,誓師於望春樓。山南東道節度使梁崇義反。五月,京師雨雹。庚申。置待詔官三十人。六月,熒惑、太白鬥於東井。癸巳,淮寧軍節度使李希烈為漢南、漢北兵馬招討使,以討梁崇義。辛丑,郭子儀薨。七月庚申,楊炎罷。檢校尚書右僕射侯希逸為司空,前永平軍節度使張鎰為中書侍郎、同中書門下平章事。侯希逸薨。癸未,馬燧、李抱真及田悅戰於臨洺,敗之。八月,劍南西川節度使張延賞、東川節度使王叔邕、山南東道節度使賈耽、刑南節度使李昌巙、陳少游討梁崇義,以李希烈為諸軍都統。辛卯,平盧軍節度使李正己卒,其子納自稱留後。壬子,梁崇義伏誅。九月,李納陷宋州。李惟岳將張孝忠以易、定二州降。壬戌,賜立功士卒帛,稟死事家三歲。十月戊申,李納將李洧以徐州降。十一月辛酉,納寇徐州,宣武軍節度使劉洽敗之於七里溝。丁丑,馬燧及田悅戰於雙岡,敗之。甲申,李納將王涉以海州降。十二月丁酉,馬萬通以密州降。馬燧為魏博招討使。是歲,殺崖州司馬楊炎。



 三年正月丙寅,硃滔、成德軍節度使張孝忠及李惟岳戰於束鹿,敗之。辛未,減常膳及太子諸王食物。復榷酤。癸未,李納陷海、密二州。閏月乙未,李惟岳將康日知以趙州降。甲辰,惟岳伏誅,其將楊榮國以深州降。庚戌,馬燧及田悅戰於洹水,敗之。是月,悅將李再春以博州降,田昂以洺州降。二月戊午,李惟岳將楊政義以定州降。甲戌,給復易、定、深、趙、恆、冀六州三年,赦吏民為李惟岳迫脅者。己卯,震通化門。四月戊午,李納將李士真以德、棣二州降。甲子,借商錢。甲戌,昭義軍節度副使盧玄卿為魏博、澶相招討使。戊寅,張鎰罷。壬午,殺殿中侍御史鄭詹。是月,硃滔反,陷德、棣二州。五月辛卯,朔方軍節度使李懷光討田悅。六月甲子,京師地震。恆冀觀察使王武俊反。辛巳,李懷光、馬燧、李芃、李抱真及硃滔、王武俊、田悅戰於連篋山,敗績。七月壬辰,殿中丞李雲端謀反,伏誅。癸巳,停借商錢令。八月癸丑,演州司馬李孟秋、峰州刺史皮岸反,伏誅。九月丁亥,初稅商錢、茶、漆、竹、木。十月丙辰,吏部侍郎關播為中書侍郎、同中書門下平章事。李希烈反。丙子,肅王詳薨。



 四年正月丁亥,鳳翔節度使張鎰及吐蕃尚結贊盟於清水。庚寅,李希烈陷汝州,執刺史李元平。戊戌,東都、汝行營節度使哥舒曜討李希烈。二月丁卯,克汝州。三月辛卯,李希烈寇鄂州,刺史李兼敗之。丁酉,荊南節度使張伯儀及李希烈戰於安州,敗績。四月庚申,李勉為淮西招討處置使,哥舒曜副之;張伯儀為淮西應援招討使,賈耽,江南西道節度使嗣曹王皋副之。甲子,京師地震,生毛。丙子,哥舒曜及李希烈戰於潁橋,敗之。五月辛巳,京師地震。乙酉,潁王璬薨。乙未,劉洽為淄青、袞鄆招討制置使。六月庚戌,稅屋間架,算除陌錢。丁卯,徒封逾為丹王,遘簡王。七月,馬燧為魏博、澶相節度招討使。壬辰,盧杞、關播、李忠臣及吐蕃區頰贊盟於京師。八月丁未,李希烈寇襄城。乙卯,希烈將曹季昌以隋州降。庚申,有星隕於京師。九月丙戌,神策軍行營兵馬使劉德信及李希烈戰於扈澗,敗績。庚子,舒王謨為荊襄、江西、沔鄂節度諸軍行營兵馬都元帥,徙封普王。十月,涇原節度使姚令言反,犯京師。戊申,如奉天。硃泚反。庚戌,泚殺司農卿段秀實及左驍衛將軍劉海賓。鳳翔後營將李楚琳殺其節度使張鎰,自稱留後。癸丑,李希烈陷襄城,宣武軍兵馬使高翼死之。甲寅,硃泚殺涇原節度都虞候何明禮。乙卯,殺尚書右僕射崔寧。丁巳,戶部尚書蕭復為吏部尚書,吏部郎中劉從一為刑部侍郎,京兆府戶曹參軍、翰林學士姜公輔為諫議大夫:同中書門下平章事。硃泚犯奉天,禁軍敗績於城東。辛酉,靈鹽節度留後杜希全、鄜坊節度使李建徽及硃泚戰於漠谷,敗績。癸亥,劉德信及泚戰於思子陵,敗之。甲子,行在都虞候渾瑊及泚戰於城下,敗之,左龍武軍大將軍呂希倩死之。乙丑,將軍高重傑死之。是月,商州軍亂,殺其刺史謝良輔。十一月,劍南西山兵馬使張朏逐其節度使張延賞,朏伏誅。癸巳,李懷光及硃泚戰於魯店,敗之。懷光為中書令、朔方邠寧同華陜虢河中晉絳慈隰行營兵馬副元帥。十二月,硃泚陷華州。壬戌,貶盧杞為新州司馬。庚午,李希烈陷汴、鄭二州。



 興元元年正月癸酉,大赦,改元。去「聖神文武」號。復李希烈、田悅、王武俊、李納官爵。赴奉天收京城將士有罪減三等,子孫減二等,在行營者賜勛五轉。賜文武官階、勛、爵。罷間架、竹木茶漆稅及除陌錢。給復奉天五年,城中十年。關播罷。丙戌,吏部侍郎盧翰為兵部侍郎、同中書門下平章事。戊子,蕭復為山南東西、荊湖、淮南、江西、鄂岳、浙江東西、福建、嶺南宣慰安撫使。戊戌,劉洽為汴滑宋毫都統副使。二月甲子,李懷光為太尉,懷光反。丁卯,如梁州。懷光將孟庭保以兵來追,左衛大將軍侯仲莊取之於驛店。三月,李懷光奪鄜坊京畿金商節度使李建徽、神策軍兵馬使陽惠元兵,惠元死之。癸酉,魏博兵馬使田緒殺其節度使田悅。自稱留後。甲戌,李懷光殺左廂兵馬使張名振、右武鋒兵馬使石演芬。丁亥,李晟為京畿、渭北、鄜坊丹延節度招討使,神策行營兵馬使尚可孤為神策、京畿、渭南、商州節度招討使。壬辰,次梁州。丁酉,劉洽權知汴滑宋毫都統兵馬事。己亥,渾瑊為朔方、邠寧、振武、永平、奉天行營兵馬副元帥。四月,李懷光陷坊州。甲辰,李晟為京畿、渭北、商華兵馬副元帥。甲寅,姜公輔罷。涇原兵馬使田希鹽殺其節度使馮河清,自稱留後。乙丑,渾瑊及硃泚戰於武亭川,敗之。丁卯,義王此薨。是月,坊州刺史竇覦克坊州。五月癸酉,涇王侹薨。丙子,李抱真、王武俊及硃滔戰於經城,敗之。壬辰,尚可孤及硃泚戰於藍田之西,敗之。乙未,李晟又敗之於苑北。戊戌,又敗之於白華,復京師。六月癸卯,姚令言伏誅。甲辰,硃泚伏誅。己酉,李晟為司徒、中書令。癸丑,以梁州為興元府,給復一年,耆老加版授。甲寅,渾瑊為侍中。己巳,給復洋州一年,加給興元一年,免鳳州今歲稅,父老加版授。七月丙子,次鳳翔,免今歲秋稅,八十以上版授刺史,餘授上佐。丁丑,葬宗室遇害者。壬午,至自興元。丁亥,李懷光殺宣慰使孔巢父。辛卯,大赦。賜百官將士階、勛、爵,收京城者升八資。給復京兆府一年。是月,嗣曹王皋及李希烈戰於應山,敗之。八月癸卯,李晟為鳳翔隴右諸軍、涇原四鎮北庭行營兵馬副元帥,馬燧為晉、慈、隰諸軍行營兵馬副元帥,渾瑊為河中、同絳、陜虢諸軍行營兵馬副元帥。丙午,渾堿兼朔方行營兵馬副元帥。己酉,延王玢、隋王迅薨。十月辛丑,李勉檢校司徒、同中書門下平章事。閏月戊子,李希烈將李澄以滑州降。十一月癸卯,劉洽、邠隴行營節度使曲環及李希烈戰於陳州,敗之。戊午,克汴州。乙丑,蕭復罷。十二月乙酉,渾瑊及李懷光戰於乾坑,敗績。是歲,陳王珪薨。



 貞元元年正月丁酉,大赦,改元。罷榷稅。三月,李懷光殺步軍兵馬使田仙浩、都戰候呂嗚嶽。丁未,李希烈陷鄧州,殺唐鄧隋招討使黃金嶽。是春,旱。四月乙丑,徙封誼為舒王。壬午,渾瑊及李懷光戰於長春宮,敗之。丙戌,馬燧、渾瑊為河中招撫使。六月己丑,幽州盧龍軍節度使硃滔卒,涿州刺史劉怦自稱留後。辛卯,劍南西川節度使張延賞為中書侍郎、同中書門下平章事。戊子,馬燧及李懷光戰於陶城,敗之。七月,灞、滻竭。庚子,大風拔木。八月,襲封配饗功臣子孫。甲子,以旱避正殿,減膳。甲戌,李懷光伏誅。己卯,給復河中、同絳二州一年。馬燧為侍中,張延賞罷。丙戌,李希烈殺宣慰使顏真卿。九月辛亥,劉從一罷。庚申,幽州盧龍軍節度使劉怦卒,其子濟自稱留後。是秋,雨木冰。十一月癸卯,有事於南郊,大赦,賜奉天興元扈從百官、收京將士階、勛、爵。



 二年正月丙申,詔減御膳之半,賙貧乏者授以官。壬寅,盧翰罷。吏部侍郎劉滋為左散騎常侍,給事中崔造,中書舍人齊映:同中書門下平章事。二月癸亥,山南東道節度使樊澤及李希烈戰於泌河,敗之。四月丙寅,希烈伏誅。甲戌,雨土。甲申,給復淮西二年。五月,李希烈將李惠登以隋州降。己酉,地震。六月癸未,滄州刺史程日華卒,其子懷直自稱觀察留後。是月,淮西兵馬使吳少誠殺其節度使陳仙奇,自稱留後。七月,李希烈將薛翼以唐州降,侯召以光州降。八月丙子,大雨雹。丙戌,吐蕃寇邠、寧、涇、隴四州。九月乙巳,寇好畤,李晟敗之於汧陽。十月癸酉,邠寧節度使韓游瑰又敗之於平川。十一月甲午,立淑妃王氏為皇后。丁酉,皇后崩。辛丑,吐蕃陷鹽州。十二月丁巳,隱夏州。馬燧為綏、銀、麟、勝招討使。庚申,崔造罷。甲戌,以吐蕃寇邊,避正殿。



 三年正月壬寅,尚書左僕射張延賞同中書門下平章事。壬子,劉滋罷。貶齊映為夔州刺史。兵部侍郎柳渾同中書門下平章事。二月己卯,華州潼關節度使駱元光克鹽、夏二州。甲申,葬昭德皇后於靖陵。三月丁未,李晟為太尉。辛亥,馬燧罷副元帥。五月,揚州江溢。吳少誠殺申州刺史張伯元、殿中侍御史鄭常。閏月辛未,渾瑊及吐蕃盟於平涼,吐蕃執會盟副使、兵部尚書崔漢衡,殺判官、殿中侍御史韓弇。戊寅,太白晝見。六月,吐蕃寇鹽、夏二州。丙戌,馬燧為司徒,前陜虢觀察使李泌為中書侍郎、同中書門下平章事。七月甲子,朔方節度使杜希全為朔方、靈鹽、豐夏綏銀節度都統。壬申,張延賞薨。八月辛巳朔,日有食之。己丑,柳渾罷。戊申,吐蕃寇青石嶺,隴州刺史蘇清沔敗之。庚戌,禁大馬出蒲、潼、武關。九月丁巳,吐蕃寇汧陽。丙寅,陷華亭及連雲堡。十月甲申,寇豐義,韓游瑰敗之。乙酉,寇長武城,城使韓全義敗之。壬辰,射生將韓欽緒謀反,伏誅。十一月己卯,京師、東都、河中地震。十二月庚辰,獵於新店。



 四年正月庚戌朔,京師地震。大赦,刺史予一子官,增戶墾田者加階,縣令減選,九品以上官言事。壬申,劉玄佐為四鎮北庭行營、涇原節度副元帥。是月,金、房二州地震,江溢山裂。雨木冰於陳留。四月,河南、淮海地生毛。己亥,福建軍亂,逐其觀察使吳詵,大將郝誡溢自稱留後。五月,吐蕃寇涇、邠、寧、慶、鄜五州。六月己亥,封子原為邕王。七月庚戌,渾瑊為邠、寧慶副元帥。癸丑,寧州軍亂,邠寧都虞候楊朝晟敗之。己未,奚、室韋寇振武。是月,河水黑。八月,灞水溢。九月庚申,吐蕃寇寧州,邠寧節度使張獻甫敗之。冬,築夾城。是歲,京師城震二十。



 五年正月甲辰朔,日有食之。二月庚子,大理卿董晉為門下侍郎,御史大夫竇參為中書侍郎:同中書門下平章事。三月甲辰,李泌薨。夏,吐蕃寇長武城,韓全義敗之於佛堂原。九月丙午,劍南西川節度使韋皋敗吐蕃於臺登北谷,克巂州。十月,嶺南節度使李復克瓊州。



 六年春,旱。閏四月乙卯,詔常參官、畿縣令言事。免京光府夏稅。八月辛丑,殺皇太子妃蕭氏。十一月戊辰,朝獻於太清宮。己巳,朝享於太廟。庚午,有事於南郊。賜文武官階、爵。降囚罪,徒以下原之。葬戰亡暴骨者。是歲,吐蕃陷北庭都護府,節度使楊襲古奔於西州。



 七年正月己巳,襄王僙薨。四月,安南首領杜英翰反,伏誅。五月甲申,端王遇薨。九月,回鶻殺楊襲古。十二月戊戌,睦王述薨。是冬,無雪。



 八年二月庚子,雨土。三月甲申,宣武軍節度使劉玄佐卒,其子士寧自稱留後。四月,吐蕃寇靈州。丁亥,殺左諫議大夫知制誥吳通玄。乙未,貶竇參為郴州別駕。尚書左丞趙憬、兵部侍郎陸贄為中書侍郎、同中書門下平章事。五月己未,大風發太廟屋瓦。癸酉,平盧軍節度使李納卒,其子師古自稱留後。六月,淮水溢。吐蕃寇連雲堡,大將王進用死之。九月丁巳,韋皋及吐蕃戰於維州,敗之。十一月壬子朔,日有食之。庚午,山南西道節度使巖震及吐蕃戰於黑水堡,敗之。是月,幽州盧龍軍節度使劉濟及其弟瀛州刺史澭戰於瀛州,澭敗奔於京師。十二月甲辰,獵於城東。



 九年正月癸卯,復稅茶。四月辛酉,關輔、河中地震。五月甲辰,義成軍節度使賈耽為尚書右僕射,尚書右丞盧邁:同中書門下平章事。丙午,董晉罷。八月庚戌,李晟薨。十一月癸未,朝獻於太清宮。甲申,朝享於太廟。乙酉,有事於南郊,大赦。十二月丙辰,宣武軍將李萬榮逐其節度使劉士寧,自稱留後。



 十年正月壬辰,南詔蠻敗吐蕃於神川,來獻捷。四月癸卯朔,赦京城。戊申,地震。癸丑,又震。是月,太白晝見。六月丙寅,韋皋敗吐蕃,克峨和城。自春不雨至於是月。辛未,雨,大風拔木。七月,西原蠻叛。八月,陷欽、橫、潯、貴四州。十月,昭義軍節度留後王虔休及攝洺州刺史元誼戰於雞澤,敗之。十二月丙辰,獵於城南。壬戌,貶陸贄為太子賓客。



 十一年四月丙寅,奚寇平州,劉濟敗之於青都山。五月庚午,中書門下慮囚。八月辛亥,馬燧薨。九月,橫海軍兵馬使程懷信逐其兄節度使懷直,自稱留後。十月,朗、蜀二州江溢。十二月戊辰,獵於苑中。



 十二年二月己卯,吐蕃寇巂州,刺史曹高仕敗之。三月丙辰,韶王暹薨。四月庚午,魏博節度使田緒卒,其子季安自稱留後。六月己丑,宣武軍節度使李萬榮卒,其子乃自稱兵馬使,伏誅。七月戊戌,韓王迥薨。八月己未朔,日有食之。丙戌,趙憬薨。九月,吐蕃寇慶州。十月甲戌,右諫議大夫崔損、給事中趙宗儒同中書門下平章事。



 十三年正月壬寅,吐蕃請和。四月辛酉,以旱慮囚。壬戌,雩於興慶宮。五月壬寅,吐蕃寇巂州,曹高仕敗之。庚戌,義寧軍亂,殺其將常楚客。七月乙未,京師地震。九月己丑,盧邁罷。



 十四年三月丙申,鳳翔監軍使西門去奢殺其將夏侯衍。五月己酉,始雷。閏月辛亥,有星隕於西北。辛酉,長武城軍亂,逐其使韓全義。六月丙申,歸化堡軍亂,逐其將張國誠,涇原節度使劉昌敗之。七月壬申,趙宗儒罷。工部侍郎鄭餘慶為中書侍郎、同中書門下平章事。九月丁卯,杞王倕薨。十二月壬寅,明州將慄鍠殺其刺史盧云以反。是冬,無雪,京師饑。



 十五年正月甲寅,雅王逸薨。壬戌,郴州藍山崩。二月乙酉,宣武軍亂,殺節度行軍司馬陸長源,宋州刺史劉逸準自稱留後。三月甲寅,彰義軍節度使吳少誠反,陷唐州,守將張嘉瑜死之。四月乙未,慄鍠伏誅。九月乙巳,陳許節度留後上官涚及吳少誠戰於臨潁,敗績。丙午,少誠寇許州。庚戌,宣武軍節度使劉全諒卒,都知兵馬使韓弘自稱留後。丙辰,宣武、河陽、鄭滑、東都汝、成德、幽州、淄青、魏博、易定、澤潞、河東、淮南、徐泗、山南東西、鄂岳軍討吳少誠。十月己丑,邕王原薨。十一月丁未,山南東道節度使于頔及吳少誠戰於吳房,敗之。陳許節度使上官涚又敗之於柴籬。辛亥,安黃節度使伊慎又敗之於鐘山。十二月庚午,壽州刺史王宗又敗之於秋柵。辛未,渾瑊薨。乙未,諸道兵潰於小溵河。



 十六年正月乙巳,易定兵及吳少誠戰,敗績。二月乙酉,鹽夏綏銀節度使韓全義為蔡州行營招討處置使,上官涚副之。四月丁亥,黔中宴設將傅近逐其觀察使吳士宗。五月庚戌,韓全義及吳少誠戰於廣利城,敗績。壬子,徐泗濠節度使張建封卒,其子愔自稱知軍事。七月丁巳,伊慎及吳少誠戰於申州,敗之。己未,韋皋克吐蕃末恭城。丙寅,韓全義及吳少誠戰於五樓,敗績。八月,劉濟及其弟涿州刺史源戰於涿州,源敗,執之。己丑,殺遂州別駕崔位。韋皋克吐蕃顒城。九月庚戌,貶鄭餘慶為郴州司馬。庚申,太常卿齊抗為中書侍郎、同中書門下平章事。十月辛未,殺通州別駕崔河圖。是歲,京師饑。



 十七年二月丁酉,大雨雹。己亥,霜。乙巳,韋皋及吐蕃戰於鹿危山,敗之。戊申,大雨雹,震電。庚戌,大雪,雨雹。五月壬戌朔,日有食之。六月丙申,寧州軍亂,殺其刺史劉南金。己亥,浙西觀察使李錡殺上封事人崔善貞。丁巳,成德軍節度使王武俊卒,其子士真自稱留後。七月,隕霜殺菽。戊寅,吐蕃寇鹽州。己丑,陷麟州,刺史郭鋒死之。九月乙亥,韋鋋敗吐蕃於雅州,克木波城。是歲,嘉王運薨。



 十八年七月乙亥,罷正衙奏事。十二月,環王陷歡、愛二州。



 十九年二月己亥,安南將王季元逐其經略使裴泰,兵馬使趙均敗之。三月壬子,淮南節度使杜佑檢校司空、同中書門下平章事。七月己未,齊抗罷。自正月不雨至於是月。甲戌,雨。閏十月庚戌,鹽州將李庭俊反,伏誅。丁巳,崔損薨。十二月庚申,太常卿高郢為中書侍郎,吏部侍郎鄭珣瑜為門下侍郎:同中書門下平章事。



 二十年二月庚戌,大雨雹。七月癸酉,大雨雹。冬,雨木冰。



 二十一年正月癸巳,皇帝崩於會寧殿,年六十四。



 順宗至德弘道大聖大安孝皇帝諱誦,德宗長子也。母曰昭德皇后王氏。始封宣城郡王,大歷十四年六月,進封宣王。十二月乙卯,立為皇太子。



 為人寬仁,喜學藝,善隸書,禮重師傅,見輒先拜。從德宗幸奉天,常執弓矢居左右。郜國公主以蠱事得罪,太子妃,其女也,德宗疑之,幾廢者屢矣,賴李泌保護,乃免。後侍宴魚藻宮,張水嬉彩艦,宮人為棹歌,眾樂間發,德宗歡甚,顧太子曰:「今日何如?」太子誦《詩》「好樂無荒」以為對。及裴延齡、韋渠牟用事,世皆畏其為相,太子每候顏色,陳其不可。故二人者卒不得用。



 貞元二十年,太子病風且瘖。



 二十一年正月,不能朝。是時,德宗不豫,諸王皆侍左右,惟太子臥病,不能見,德宗悲傷涕泣,疾有加。癸巳,德宗崩。丙申,即皇帝位於太極殿。二月癸卯,朝群臣於紫宸門。辛亥,吏部侍郎韋執誼為尚書左丞、同中書門下平章事。甲子,大赦。罷宮市。民百歲版授下州刺史,婦人郡君;九十以上上佐,婦人縣君。乙丑,罷鹽鐵使月進。三月庚午,放後宮三百人。癸酉,放後宮及教坊女妓六百人。癸巳,立廣陵郡王純為皇太子。四月壬寅,封弟諤為欽王,諴珍王;進封子建康郡王經郯王,洋川郡王緯均王,臨淮郡王縱漵王,弘農郡王紓莒王,漢東郡王綱密王,晉陵郡王枿郇王,郡王約邵王,雲安郡王結宋王,宣城郡王緗集王,德陽郡王絿冀王,河東郡王綺和王;封子絢為衡王,纁會王,綰福王,紱撫王,緄岳王,紳袁王,綸桂王,繟翼王。戊申,以冊皇太子,降死罪以下,賜文武官子為父後者勛兩轉。七月辛卯,橫海軍節度使程懷信卒,其子執恭自稱留後。乙未,皇太子權句當軍國政事。太常卿杜黃裳為門下侍郎,左金吾衛大將軍袁滋為中書侍郎:同中書門下平章事。鄭珣瑜、高郢罷。



 永貞元年八月庚子,立皇太子為皇帝,自稱曰太上皇。辛丑,改元。降死罪以下。立良娣王氏為太上皇后。



 元和元年正月,皇帝率群臣上尊號曰應乾聖壽太上皇。是月,崩於咸寧殿,年四十六,謚曰至德大聖大安孝皇帝。大中三年,增謚至德弘道大聖大安孝皇帝。



 憲宗昭文章武大聖至神孝皇帝諱純,順宗長子也。母曰莊憲皇太后王氏。貞元四年六月己亥,封廣陵郡王。二十一年三月,立為皇太子。



 永貞元年八月,順宗詔立為皇帝。乙巳,即皇帝位於太極殿。丁未,始聽政。庚戌,罷獻祥瑞。癸丑,劍南西川節度使韋皋卒,行軍司馬劉闢自稱留後。戊午,天有聲於西北。己未,袁滋為劍南西川、山南西道安撫大使。癸亥,尚書左丞鄭餘慶同中書門下平章事。九月己巳,罷教坊樂工正員官。十月丁酉,為曾太皇太后舉哀。賈耽薨。戊戌,舒王誼薨。袁滋罷。己酉,葬神武聖文皇帝於崇陵。十一月己巳,祔睿真皇后於元陵寢宮。壬申,貶韋執誼為崖州司馬。夏綏銀節度留後楊惠琳反。十二月壬戌,中書舍人鄭絪為中書侍郎、同中書門下平章事。



 元和元年正月丁卯,大赦,改元。賜文武官階、勛、爵,民高年者米帛羊酒。癸未,長武城使高崇文為左神策行營節度使,率左右神策京西行營兵馬使李元奕、山南西道節度使嚴礪、劍南東川節度使李康以討劉闢。甲申,太上皇崩。劉闢陷梓州,執李康。三月丙子,高宗文克梓州。辛巳,楊惠琳伏誅。四月丁未,杜佑為司徒。壬戌,邵王約薨。初令尚書省六品、諸司四品以上職事官,太子師傅、賓客、詹事,王府傅,日二人待制。五月辛卯,尊母為皇太后。六月癸巳,降死罪以下。賜百姓有父母祖父母八十以上者粟二斛、物二段,九十以上粟三斛、物三段。丙申,大風拔木。丁酉,高崇文及劉闢戰於鹿頭柵,敗之。癸卯,嚴礪又敗之於石碑穀。閏月壬戌,平盧軍節度使李師古卒,其弟師道自稱留後。七月壬寅,葬至德大聖大安孝皇帝於豐陵。癸丑,高崇文及劉闢戰於玄武,敗之。八月丁卯,進封子平原郡王寧為鄧王,同安郡王寬澧王,延安郡王宥遂王,彭城郡王察深王,高密郡王寰洋王,文安郡王寮絳王;封子審為建王。九月丙午,嚴礪及劉闢戰於神泉,敗之。辛亥,高崇文克成都。十月甲子,減劍南東西川、山南西道今歲賦,釋脅從將吏。葬陣亡者,稟其家五歲。戊子,劉闢伏誅。十一月庚戌,鄭餘慶罷。是歲,召王偲薨。



 二年正月己丑,朝獻於太清宮。庚寅,朝享於太廟。辛卯,有事於南郊,大赦。賜文武官勛、爵,文宣公、二王后、三恪、公主、諸王一子官,高年米帛羊酒加版授。乙巳,杜黃裳罷。己酉,御史中丞武元衡為門下侍郎,中書舍人李吉甫為中書侍郎:同中書門下平章事。二月己巳,罷兩省官次對。癸酉,邕管經略使路恕敗黃洞蠻,執其首領黃承慶。九月乙酉,密王綢薨。十月,鎮海軍節度使李錡反,殺留後王澹。乙丑,淮南節度使王鍔為諸道行營兵馬招討使以討之。丁卯,武元衡罷。癸酉,鎮海軍兵馬使張子良執李錡。己卯,免潤州今歲稅。十一月甲申,李錡伏誅。十二月丙寅,劍南西川節度使高崇文為邠寧節度、京西諸軍都統。



 三年正月癸巳,群臣上尊號曰睿聖文武皇帝,大赦。罷諸道受代進奉錢。三月癸巳,郇王枿薨。四月壬申,大風壞含元殿西闕檻。六月,西原蠻首領黃少卿降。七月辛巳朔,日有食之。九月庚寅,山南東道節度使于頔為司空、同中書門下平章事。丙申,戶部侍郎裴垍為中書侍郎、同中書門下平章事。戊戌,李吉甫罷。



 四年正月壬午,免山南東道、淮南、江西、浙東、湖南、荊南今歲稅。戊子,簡王遘薨。二月丁卯,鄭絪罷。給事中李籓為門下侍郎、同中書門下平章事。三月乙酉,成德軍節度使王士真卒,其子承宗自稱留後。閏月己酉,以旱降京師死罪非殺人者,禁刺史境內榷率、諸道旨條外進獻、嶺南黔中福建掠良民為奴婢者,省飛龍廄馬。己未,雨。丁卯,立鄧王寧為皇太子。七月癸亥,吐蕃請和。八月丙申,環王寇安南,都護張舟敗之。十月辛巳,成德軍節度使王承宗反,執保信軍節度使薛昌朝。癸未,左神策軍護軍中尉吐突承璀為左右神策、河陽、浙西、宣歙、鎮州行營兵馬招討處置使以討之。戊子,承璀為鎮州招討宣慰使。癸巳,降死罪以下,賜文武官子為父後者勛兩轉。十一月己巳,彰義軍節度使吳少誠卒,其弟少陽自稱留後。



 五年正月己巳,左神策軍大將軍酈定進及王承宗戰,死之。三月甲子,大風拔木。四月丁亥,河東節度使範希朝、義武軍節度使張茂昭及王承宗戰於木刀溝,敗之。七月丁未,赦王承宗。乙卯,幽州盧龍軍節度使劉濟卒,其子總自稱留後。九月丙寅,太常卿權德輿為禮部尚書、同中書門下平章事。十月,張茂昭以易、定二州歸於有司。辛巳,義武軍都虞候楊伯玉反,伏誅。是月,義武軍兵馬使張佐元反,伏誅。十一月甲辰,會王纁薨。庚申,裴垍罷。



 六年正月庚申,淮南節度使李吉甫為中書侍郎、同中書門下平章事。二月壬申,李籓罷。己丑,忻王造薨。三月戊戌,有星隕於鄆州。十二月己丑,戶部侍郎李絳為中書侍郎、同中書門下平章事。閏月辛卯,辰、漵州首領張伯靖反,寇播、費二州。辛亥,皇太子薨。



 七年正月癸酉,振武河溢,毀東受降城。四月癸巳,詔民田畮樹桑二。六月癸巳,杜佑罷。七月乙亥,立遂王宥為皇太子。八月戊戌,魏博節度使田季安卒,其子懷諫自稱知軍府事。九月,京師地震。十月乙未,魏博軍以田季安之將田興知軍事。庚戌,降死罪以下,賜文武官子為父後者勛兩轉。是月,魏博節度使田興以六州歸於有司。十一月辛酉,赦魏、博、貝、衛、澶、相六州,給復一年,賜高年、孤獨、廢疾粟帛,賞軍士。



 八年正月辛未,權德輿罷。二月丁酉,貶于頔為恩王傅。三月甲子,劍南西川節度使武元衡為門下侍郎、同中書門下平章事。四月己亥,黔中經略使崔能討張伯靖。五月癸亥,荊南節度使嚴綬討伯靖。丁丑,大隗山崩。六月辛卯,渭水溢。辛丑,出宮人。七月己巳,劍南東川節度使潘孟陽討張伯靖。八月辛巳,湖南觀察使柳公綽討伯靖。丁未,伯靖降。十二月庚寅,振武將楊遵憲反,逐其節度使李進賢。



 九年二月癸卯,李絳罷。三月丙辰巂州地震。丁卯,隕霜殺桑。五月乙丑,桂王綸薨。癸酉,以旱免京畿夏稅。六月壬寅,河中節度使張弘靖為刑部尚書、同中書門下平章事。閏八月丙辰,彰義軍節度使吳少陽卒,其子元濟自稱知軍事。九月丁亥,山南東道節度使嚴綬、忠武軍都知兵馬使李光顏、壽州團練使李文通、河陽節度使烏重胤討之。十月,太白晝見。丙午,李吉甫薨。甲子,嚴綬為申、光、蔡招撫使。十一月戊子,罷京兆府獵獻狐兔。十二月,詔刑部、大理官朔望入對。戊辰,尚書右丞韋貫之同中書門下平章事。



 十年正月乙酉,宣武軍節度使韓弘為司徒。二月甲辰,嚴綬及吳元濟戰於磁丘,敗績。自冬不雨至於是月。丙午,雪。壬戌,河東戍將劉輔殺豐州刺史燕重旰,伏誅。三月庚子,忠武軍節度使李光顏及吳元濟戰於臨潁,敗之。四月甲辰,又敗之於南頓。五月丙申,又敗之於時曲。六月癸卯,盜殺武元衡。戊申,京師大索。乙丑,御史中丞裴度為中書侍郎、同中書門下平章事。七月甲戌,王承宗有罪,絕其朝貢。八月己亥朔,日有食之。丁未,李師道將訾嘉珍反於東都,留守呂元膺敗之。乙丑,李光顏及吳元濟戰於時曲,敗績。九月癸酉,韓弘為淮酉行營兵馬都統。十月,地震。十一月壬申,李光顏、烏重胤及吳元濟戰於小溵河,敗之。丁丑,李文通又敗之於固始。戊寅,盜焚獻陵寢宮。十二月甲辰,武寧軍都押衙王智興及李師道戰於平陰,敗之。是歲,丹王逾薨。



 十一年正月己巳,張弘靖罷。乙亥,幽州盧龍軍節度使劉總及王承宗戰於武強,敗之。癸未,免鄰賊州二歲稅。甲申,盜斷建陵門戟。二月庚子,王承宗焚蔚州。乙巳,中書舍人李逢吉為門下侍郎、同中書門下平章事。乙丑,地震。三月庚午,皇太后崩。四月庚子,李光顏、烏重胤及吳元濟戰於凌雲柵,敗之。乙卯,劉總及王承宗戰於深州,敗之。己未,免京畿二歲逋稅。五月丁卯,宥州軍亂,逐其刺史駱怡,夏綏銀節度使田縉敗之。丁亥,雲南蠻寇安南。六月,蜜州海溢。甲辰,唐鄧節度使高霞寓及吳元濟戰於鐵城,敗績。七月壬午,韓弘及元濟戰於郾城。敗之。丙戌,免淮西領賊州夏稅。八月甲午,渭水溢。壬寅,韋貫之罷。戊申,西原蠻陷賓、巒二州。己未,昭義軍節度使郗士美及王承宗戰於柏鄉,敗之。庚申,葬莊憲皇太后於豐陵。十一月乙丑,邕管經略使韋悅克賓、巒二州。甲戌,元陵火。十二月丁未,翰林學士、工部侍郎王涯為中書侍郎、同中書門下平章事。己未,西原蠻陷巖州。是冬,桃李華。



 十二年正月丁丑,地震。戊子,有彗星出於畢。四月辛卯,唐鄧隋節度使李愬及吳元濟戰於嵖岈山,敗之。乙未,李光顏又敗之於郾城。五月酉,李愬又敗之於張柴。七月丙辰,裴度為淮西宣慰處置使,戶部侍郎崔群為中書侍郎;同中書門下平章事。八月癸亥,烏重胤及吳元濟戰於賈店,敗績。九月丁未,李逢吉罷。甲寅,李愬及吳元濟戰於吳房,敗之。十月癸酉,克蔡州。甲戌,淮南節度使李庸阜為門下侍郎、同中書門下平章事。甲申,給復準西二年,免旁州來歲夏稅。葬戰士,稟其家五年。十一月丙戌,吳元濟伏誅。甲午,恩王連薨。是歲,容管經略使陽旻克欽、橫、潯、貴四州。



 十三年正月乙酉,大赦,免元和二年以前逋負,賜高年米帛羊酒。三月戊戌,御史大夫李夷簡為門下侍郎、同中書門下平章事。李庸阜罷。己酉,橫海軍節度使程權以滄、景二州歸於有司,權朝於京師。四月甲寅,王承宗獻德、棣二州。庚辰,赦承宗。六月癸丑朔,日有食之。癸亥,給復德、棣、滄、景四州一年。辛未,淮水溢。七月乙酉,宣武、魏博、義成、橫海軍討李師道。辛丑李夷簡罷。八月壬子,王涯罷。九月甲辰,戶部侍郎皇甫鎛,諸道鹽鐵轉運使程異為工部侍郎:同中書門下平章事。十月壬戌,吐蕃寇宥州,靈武節度使杜叔良敗之於定遠城。十一月丁亥,命山人柳泌為臺州刺史以求藥。十二月,庚戌,迎佛骨於鳳翔。



 十四年正月丙午,田弘正及李師道戰於陽谷,敗之。二月戊午,師道伏誅。四月辛未,程異薨。丙子,裴度罷。七月戊寅,韓弘以汴、宋、毫、潁四州歸於有司,弘朝於京師。己丑,群臣上尊號曰元和聖文神武法天應道皇帝。大赦,賜文武官階、勛、爵。遣黜陟使於天下。辛卯,沂海將王弁殺其觀察使王遂,處稱留後。丁酉,河陽節度使令狐楚為中書侍郎、同中書門下平章事。八月己酉,韓弘為中書令。九月戊寅,王弁伏誅。十月壬戌,安南將楊清殺其都護李象古以反。癸酉,吐蕃寇鹽州。十一月辛卯,朔方將史敬奉及吐蕃戰於瓠蘆河,敗之。十二月乙卯,崔群罷。



 十五年正月,宦者陳弘志等反。庚子,皇帝崩,年四十三,謚曰聖神章武孝皇帝。大中三年,加謚昭文章武大聖至神孝皇帝。



 贊曰:德宗猜忌刻薄,以強明自任,恥見屈於正論,而忘受欺於奸諛。故其疑蕭復之輕己,謂姜公輔為賣直,而不能容;用盧杞、趙贊,則至於敗亂,而終不悔。及奉天之難,深自懲艾,遂行姑息之政。由是朝廷益弱,而方鎮愈強,至於唐亡,其患以此。憲宗剛明果斷,自初即位,慨然發慎,志平僭叛,能用忠謀,不惑群議,卒收成功。自吳元濟誅,強籓悍將皆欲悔過而效順。當此之時,唐之威令,幾於復振,則其為優劣,不待較而可知也。及其晚節,信用非人,不終其業,而身罹不測之禍,則尤甚於德宗。鳴呼!小人之能敗國也,不必愚君暗主,雖聰明聖智,茍有惑焉,未有不為患者也。昔韓愈言,順宗在東宮二十年,天下陰受其賜。然享國日淺,不幸疾病,莫克有為,亦可以悲夫!



\end{pinyinscope}