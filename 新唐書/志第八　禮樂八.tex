\article{志第八 禮樂八}

\begin{pinyinscope}

 皇帝納皇后。



 制命太尉為使,宗正卿為副,吏部署承以戒之。前一日,有司展縣、設桉、陳車輿於太極殿廷,如元日。文武九品、朝集、蕃客之位,皆如冠禮。設使者受命位於大橫街南道東,西上,副少退,北面。侍中請「中嚴」。群臣入就位。使、副入,立於門外道東,西面。黃門侍郎引幡、節,中書侍郎引制書桉,立於左延明門內道北,西面北上。乃奏「外辦」。皇帝袞冕御輿,出自西房,即御座。使、副入,就位。典儀曰:「再拜。」在位者皆再拜。侍中前承制,降,詣使者東北,西面曰:「有制。」使、副再拜。侍中宣制曰:「納某官某氏女為皇后,命公等持節行納採等禮。」使、副又拜。主節立於使者東北,西面,以節授黃門侍郎,侍郎以授使者,付於主節,立於後。中書侍郎引制書桉立於使者東北,以制書授使者,置於桉。典儀曰:「再拜。」在位者皆再拜。使、副出,持節者前導,持桉者次之。侍中奏「禮畢」。皇帝入,在位者以次出。初,使、副乘輅,鼓吹備而不作,從者乘車以從。其制書以油絡網犢車載之。其日大昕,使、副至於次,主人受於廟若寢。布神席於室戶外之西,莞筵紛純,加藻席畫純,南向,右雕幾。使、副立於門西,北上,持幡、節者立於北,少退,制桉立於南,執雁者又在其南,皆東面。主人立於大門內,西面。儐者北面,受命於左,出,立於門東,西面,曰:「敢請事。」使者曰::「某奉制納採。」儐者入告。主人曰:「臣某之女若如人,既蒙制訪,臣某不敢辭。」儐者出告,入引主人出,迎使者於大門外之南,北面再拜。使者不答。主人揖使、副先入,至於階。使、副入,導以幡、節,桉、雁從之。幡、節立西階之西,東面;使者由階升,立於兩楹間,南面;副在西南,持桉及執雁者又在西南,皆東面。主人升阼階,當使者前,北面立。持桉者以桉進,授使者以制書,節脫衣,制者曰:「有制。」主人再拜。宣制,主人降詣階間,北面,再拜稽首,升,進,北面受制書,以授左右。使者授雁,主人再拜,進,受雁,以授左右。儐者引答表桉進,立於主人後,少西,以表授主人。主人進,授使者,退復位,再拜。節加衣。謁者引使、副降自西階以出。



 制文以版,長一尺二寸,博四寸,厚八分,後家答版亦如之。



 問名。使者既出,遂立於內門外之西,東面;主人立於內門內東廂,西面。儐者出請事,使者曰:「將加卜筮,奉制問名。」儐者入告。主人曰:「臣某之子若如人,既蒙制訪,臣某不敢辭。」儐者出告,入,引主人出,迎使者以入,授主人以制書,答表皆如納採。使、副降自西階以出,立於內門外之西,東面;主人立於東階下,西向。儐者出請事,使者曰:「禮畢。」儐者入告,主人曰;「某公奉制至於某之室,某有先人之禮,請禮從者。」儐者出告,使者曰:「某既得將事,敢辭。」儐者入告,主人曰:「先人之禮,敢固以請。」儐者出告,使者曰:「某辭不得命,敢不從。」儐者入告,遂引主人升,立於序端。掌事者徹幾,設二筵東上。設甒醴於東房西牖下,加杓冪,坫在尊北;實觶二,角柶二,籩、豆各一,實以脯棨,在坫北。又設洗於東南。主人降迎使者,西面揖,先入。使、副入門而左,主人入門而右。至階,主人曰:「請某位升。」使者曰:「某敢辭。」主人又曰:「固請某位升。」使者曰:「某敢固辭。」主人又曰:「終請某位升。」使者曰:「敢終辭。」主人升自阼階,使、副升自西階,北面立。主人阼階上,北面再拜。受幾於序端。掌事者內拂幾三,奉兩端西北向以進。主人東南向,外拂幾三,振袂,內執之,掌事者一人又執幾以從,主人進,西北向。使者序進,迎受於筵前,東南向以俟。主人還東階上,北面再拜送。使者以幾跪進,北面跪,各設於坐左,退於西階上,北面東上,答拜,立於階西,東面南上。贊者二人俱升,取觶降,盥手,洗觶,升,賓醴,加柶於觶,覆之,面葉,出房,南面。主人受醴,面柄,進使者筵前西,北面立。又贊者執觶以從。使者西階上,北面各一拜,序進筵前東,南面。主人又以次授醴,使者受,俱復西階上位。主人退,復東階上,北面一拜送。掌事者以次薦脯棨於筵前。使者各進,升筵,皆坐,左執觶,右取脯,擩於棨,祭於籩、豆之間,各以柶祭醴三,始扱一祭,又扱再祭,興;各以柶兼諸觶上,躐降筵於西階上,俱北面坐,啐醴,建柶,各奠觶於薦,遂拜,執觶興。主人答拜。使者進,升筵坐,各奠觶於薦東。降筵,序立於西階上,東面南上。掌事者牽馬入,陳於門內,三分庭一在南,北首西上。又掌事者奉幣篚,升自東階,以授主人,受於序端,進西面位。掌事者一人,又奉幣篚,立於主人之後。使者西階上,俱北面再拜。主人進詣楹間,南面立,使者序進,立於主人之西,俱南面。主人以幣篚授使者,使者受,退立於西階上,東面。執幣者又以授主人,主人受,以授使副,使副受之,退立於使者之北,俱東面。主人還東階上,北面再拜送。使者降自西階,從者訝受幣篚。使者當庭實揖馬以出,牽馬者從出。使者出大門外之西,東面立。從者訝受馬。。主人出門東,西面再拜送。使者退,主人入,立於東階下,西面。儐者告於主人曰:「賓不顧矣。」主人反於寢。」使者奉答表詣闕。



 納吉。使者之辭曰:「加諸卜筮,占曰日從,制使某也入告。」主人之辭曰:「臣某之女若如人,龜筮云吉,臣預在焉,臣某謹奉典制。」其餘皆如納採。



 納徵。其日,使者至於主人之門外,執事者入,布幕於內門之外,玄纁束陳於幕上,六馬陳於幕南,北首西上。執事者奉穀珪以櫝,俟於幕東,西面。謁者引使者及主人立於大門之內外。儐者進受命,出請事。使者曰;「某奉制納徵。」儐者入告,主人曰:「奉制賜臣以重禮,臣某祗奉典制。」儐者出告,入,引主人出,迎使者入。執事者坐,啟櫝取珪,加於玄纁。牽馬者從入,三分庭一在南,北首西上。執珪者在馬西,俱北面。其餘皆如納採。



 冊后。



 前一日,守宮設使者次於後氏大門外之西,尚舍設尚宮以下次於後氏閣外道西,東向,障以行帷。其日,臨軒命使,如納採。奉禮設使者位於大門外之西,東向;使副及內侍位於使者之南,舉冊桉及寶綬者在南,差退,持節者在使者之北,少退,俱東向。設主人位於大門外之南,北面。使者以下及主人位於內門外,亦如之。設內謁者監位於內門外主人之南,西面。司贊位於東階東南,掌贊二人在南,差退,俱西向。又置一桉於閣外。使、副乘輅,持節,備儀仗,鼓吹備而不作。內僕進重翟以下於大門之外道西,東向,以北為上。諸衛令其屬布後儀仗。使者出次,就位。主人朝服立於東階下,西面。儐者受命,出請事。使者曰:「某奉制,授皇后備物典冊。」儐者入告,主人出,迎於大門外,北面再拜,使者不答拜。使者入門而左,持節者前導,持桉者次之。主人入門而右,至內門外位。奉冊寶桉者進,授使副冊寶。內侍進使者前,西面受冊寶,東面授內謁者監,持入,立於閣外之西,東面跪置於桉。尚宮以下入閣,奉後首飾、褘衣,傅姆贊出,尚宮引降立於庭中,北面。尚宮跪取冊,尚服跪取寶綬,立於後之右,西向。司言、司寶各一人立於後左,東向。尚宮曰:「有制。」尚儀曰:「再拜。」皇后再拜。宣冊。尚儀曰:「再拜。。」皇后又再拜。尚宮授皇后以冊,受以授司言。尚服又授以寶綬,受以授司寶。皇后升坐,內官以下俱降立於庭,重行相向,西上。司贊曰:「再拜。」掌贊承傳,皆再拜。諸應侍衛者各升,立於侍位。尚儀前跪奏曰:「禮畢。」皇后降坐以入。使者復命。



 其遣使者奉迎。其日,侍中版奏「請中嚴」。皇帝服冕出,升所御殿,文武之官五品已上立於東西朝堂。奉迎前一日,守宮設使者次於大門之外道右,設使副及內侍次於使者次西,俱南向。尚舍設宮人次於閣外道西。奉禮設使、副、持桉執雁者、持節者及奉禮、贊者位,如冊后。又設內侍位於大門外道左,西面。又設宮人以下位於堂前。使、副朝服,乘輅持節,至大門外次,宮人等各之次奉迎。尚儀奏「請皇后中嚴」。傅姆導皇后,尚宮前引,出,升堂。皇后將出,主婦出於房外之西,南向。文武奉迎者皆陪立於大門之外,文官在東,武官在西,皆北上。謁者引使者詣大門外位,主人立於內門外堂前東階下,西面。儐者受命,出請事,使者曰:「某奉制,以今吉辰,率職奉迎。」儐者入告,主人曰;「臣謹奉典制。」儐者出告,入,引主人出門南,北面再拜。謁者引入至內門外堂西階,使者先升,位於兩楹間,南面;副在西,持桉、執雁者在西南,俱東面。主人升東階,詣使者前,北面立,使、副授以制書,曰:「有制。」主人再拜。使者宣制,主人降詣階間,北面再拜稽首。升,進,北面受制書。主人再拜,北面立。使、副授以雁,主人再拜,進受,仍北面立。儐者引二人對舉答表桉進,主人以表授使、副,再拜,降自西階以出,復門外位。奉禮曰:「再拜。」贊者承傳,使、副俱再拜。使者曰;「令月吉日,臣某等承制,率職奉迎。」內侍受以入,傳於司言,司言受以奏聞。尚儀奏請皇后再拜。主人入,升自東階,進,西面誡之曰:「戒之敬之,夙夜無違命。」主人退,立於東階上,西面。母誡於西階上,施衿結帨,曰:「勉之敬之,夙夜無違命。」皇后升輿以降,升重翟以幾,姆加景,內宮侍從及內侍導引,應乘車從者如鹵簿。皇后車出大門外,以次乘車馬引從。



 同牢之日,內侍之屬設皇后大次於皇帝所御殿門外之東,南向。將夕,尚寢設皇帝御幄於室內之奧,東向。鋪地席重茵,施屏障。初昏,尚食設洗於東階,東西當東霤,南北以堂深。後洗於東房,近北。設饌於東房西墉下,籩、豆各二十四,簋、簠各二,登各三,俎三。尊於室內北牖下,玄酒在西。又尊於房戶外之東,無玄酒。坫在南,加四爵,合巹。器皆烏漆,巹以匏。皇后入大門,鳴鐘鼓。從永巷至大次前,回車南向,施步障。尚儀進,當車前跪請降車。皇后降,入次。尚宮引詣殿門之外,西向立。尚儀跪奏「外辦,請降坐禮迎」。皇帝降坐,尚宮前引,詣門內之西,東面揖後以入。尚食酌玄酒三注於尊,尚寢設席於室內之西,東向。皇帝導後升自西階,入室即席,東向立。皇后入,立於尊西,南面。皇帝盥於西洗,後盥於北洗。饌入,設醬於席前,菹棨在其北;俎三設於豆東,豕俎特在北。尚食設黍於醬東,稷、稻、粱又在東;設棨湆於醬南。設後對醬於東,當特俎,菹棨在其南,北上;設黍於豕俎北,其西稷、稻、粱,設湆於醬北。尚食啟會郤于簠簋之南,對簠簋於北,加匕箸,尚寢設對席於饌東。尚食跪奏「饌具」。皇帝揖皇后升,對席,西面,皆坐。尚食跪取韭擩棨授皇帝,取菹擩棨授皇后,俱受,祭於豆間。尚食又取黍實於左手,遍取稷、稻、粱反於右手,授皇帝,又取黍、稷、稻、粱授皇后,俱受,祭於豆間。又各取鸑絕末授帝、後,俱祭於豆間。尚食各以鸑加於俎。司飾二人以巾授皇帝及皇后,俱涚手。尚食各跪品嘗饌,移黍置於席上,以次授鸑脊,帝、後皆食,三飯,卒食。尚食二人俱盥手洗爵於房,入室,酌於尊,以授帝、後,俱受,祭。尚食各以肝從,皆奠爵、振祭、嚌之。尚食皆受,實於俎、豆。各取爵,皆飲。尚儀受虛爵,奠於坫。再酳如初,三酳用巹,如再酳。尚食俱降東階,洗爵,升,酌於戶外,進,北面奠爵,興,再拜,跪取爵祭酒,遂飲卒爵,奠,遂拜,執爵興,降,奠於篚。尚儀北面跪,奏稱:「禮畢,興。」帝、後俱興。尚宮引皇帝入東房,釋冕服,御常服;尚宮引皇后入幄,脫服。尚宮引皇帝入。尚食徹饌,設於東房,如初。皇后從者餕皇帝之饌,皇帝侍者餕皇后之饌。



 皇太子納妃。



 皇帝遣使者至於主人之家,不持節,無制書。其納採、問名、納吉、納徵、告期,皆如後禮。



 其冊妃。前一日,主人設使者次大門之外道右,南向;又設宮人次於使者西南,俱東向,障以行帷。奉禮設使者位於大門外之西,副及內侍又於其南,舉冊桉及璽綬,命服者又南,差退,俱東向。設主人位於門南,北面。又設位於內門外,如之。設典內位於內門外主人之南,西面。宮人位於門外使者之後,重行東向,以北為上,障以行帷。設贊者二人位於東階東南,西向。典內預置一桉於閣外。使、副朝服,乘輅持節,鼓吹備而不作。至妃氏大門外次,掌嚴奉褕翟衣及首飾,內廄尉進厭翟於大門之外道西,東向,以北為上。諸衛帥其屬布儀仗。使者出次,持節前導,及宮人、典內皆就位。主人朝服,出迎於大門之外,北面再拜。使者入門而左,持桉從之。主人入門而右,至內門外位。奉冊寶桉者進,授使副冊寶,內侍西面受之,東面授典內,典內持入,跪置於閣內之桉。奉衣服及侍衛者從入,皆立於典內之南,俱東面。傅姆贊妃出,立於庭中,北面。掌書跪取玉寶,南向。掌嚴奉首飾、褕翟,與諸宮官侍衛者以次入。司則前贊妃再拜,北面受冊寶於掌書,南向授妃,妃以授司閨。司則又贊再拜,乃請妃升坐。宮官以下皆降立於庭,重行北面,西上。贊者曰:「再拜。」皆再拜。司則前啟「禮畢」。妃降座,入於室。主人儐使者如禮賓之儀。



 臨軒醮戒。前一日,衛尉設次於東朝堂之北,西向。又設宮官次於重明門外。其日,皇太子服袞冕出,升金輅,至承天門降輅,就次。前一日,有司設御座於太極殿阼階上,西向。設群官次於朝堂,展縣,陳車輅。其日,尚舍設皇太子席位於戶牖間,南向,莞席、藻席。尚食設酒尊於東序下,又陳籩脯一、豆棨一,在尊西。晡前三刻,設群官版位於內,奉禮設版位於外,如朝禮。侍中版奏「請中嚴」。前三刻,諸侍衛之官侍中、中書令以下俱詣閣奉迎。典儀帥贊者先入就位,吏部、兵部贊群官出次,就門外位。侍中版奏「外辦」。皇帝服通天冠、絳紗袍,乘輿出自西房,即御座西向。群官入就位。典儀曰:「再拜。」贊者承傳,在位者皆再拜。皇太子入縣南,典儀曰:「再拜。」贊者承傳,皇太子再拜。詣階,脫舄,升席西,南面立。尚食酌酒於序,進詣皇太子西,東面立。皇太子再拜,受爵。尚食又薦脯棨於席前。皇太子升席坐,左執爵,右取脯,擩於棨,祭於籩、豆之間。右祭酒,興,降席西,南面坐,啐酒,奠爵,興,再拜,執爵興。奉御受虛爵,直長徹薦,還於房。皇太子進,當御座前,東面立。皇帝命之曰:「往迎尒相,承我宗事,勖帥以敬。」皇太子曰:「臣謹奉制旨。」遂再拜,降自西階,納舄,出門。典儀曰:「再拜。」贊者承傳,在位者皆再拜,以次出。侍中前跪奏「禮畢」。皇帝入。



 皇太子既受命,執燭、前馬、鼓吹,至於妃氏大門外道西之次,回輅南向。左庶子跪奏,降輅之次。主人設幾筵。妃服褕翟、花釵,立於東房,主婦立於房戶外之西,南向。主人公服出,立於大門之內,西向。在廟則祭服。左庶子跪奏「請就位」。皇太子立於門西,東面。儐者受命出請事,左庶子承傳跪奏,皇太子曰:「以茲初昏,某奉制承命。」左庶子俯伏,興,傳於儐者,入告,主人曰:「某謹敬具以須。」儐者出,傳於左庶子以奏。儐者入,引主人迎於門外之東,西面再拜,皇太子答再拜。主人揖皇太子先入,掌畜者以雁授左庶子,以授皇太子,執雁入。及內門,主人讓曰:「請皇太子入。」皇太子曰:「某弗敢先。」主人又固請,皇太子又曰:「某固弗敢先。」主人揖,皇太子入門而左,主人入門而右。及內門,主人揖入,及內霤,當曲揖,當階揖,皇太子皆報揖。至於階,主人曰:「請皇太子升。」皇太子曰:「某敢辭。」主人固請,皇太子又曰:「某敢固辭。」主人終請,皇太子又曰:「某終辭。」主人揖,皇太子報揖。主人升,立於阼階上,西面。皇太子升,進當房戶前,北面,跪奠雁,再拜,降,出。主人不降送。內廄尉進,厭翟於內門外,傅姆導妃,司則前引,出於母左。師姆在右,保姆在左。父少進,西面戒之曰:「必有正焉。若衣花。」命之曰:「戒之敬之,夙夜無違命。」母戒之西階上,施衿結帨,命之曰:「勉之敬之,夙夜無違命。」庶母及門內施鞶,申之以父母之命,命之曰:「敬恭聽宗父母之言,夙夜無愆。視諸衿鞶。」妃既出內門,至輅後,皇太子授綏,姆辭不受,曰:「未教,不足與為禮。」妃升輅,乘以幾,姆加景。皇太子馭輪三周,馭者代之。皇太子出大門,乘輅還宮,妃次於後。主人使其屬送妃,以族從。



 同牢之日,司閨設妃次於閣內道東,南向。設皇太子御幄於內殿室內西廂,東向。設席重茵,施屏障。設同牢之席於室內,皇太子之席西廂,東向,妃席東廂,西向。席間量容牢饌。設洗於東階東南,設妃洗於東房近北。饌於東房西墉下,籩、豆各二十,簠、簋各二,鈃各三,瓦登一,俎三。尊在室內北墉下,玄酒在西。又設尊於房戶外之東,無玄酒。篚在南,實四爵,合巹。皇太子車至左閣,回輅南向,左庶子跪奏「請降輅」。入,俟於內殿門外之東,西面。妃至左閣外,回輅南向,司則請妃降輅,前後扇、燭。就次,立於內殿門西,東面。皇太子揖以入,升自西階,妃從升。執扇、燭者陳於東、西階內。皇太子即席,東向立,妃西向立。司饌進詣階間,跪奏「具牢饌」,司則承令曰:「諾。」遂設饌如皇后同牢之禮。司饌跪奏「饌具」。皇太子及妃俱坐。司饌跪,取脯,取韭菹,皆擩於棨,授皇太子,又取授妃,俱受,祭於籩、豆之間。司饌跪取黍實於左手,遍取稷反於右手,授皇太子,又授妃,各受,祭於菹棨之間。司饌各立,取鸑皆絕末,跪授太子及妃,俱受,又祭於菹棨之間。司饌俱以鸑加於俎。掌嚴授皇太子妃巾,涚手。以柶扱上鈃遍擩之,祭於上豆之間。司饌品嘗妃饌,移黍置於席上,以次跪授脊。皇太子及妃皆食以湆醬,三飯,卒食。司饌北面請進酒,司則承令曰;「諾。」司饌二人俱盥手洗爵於房,入室,酌於尊,北面立。皇太子及妃俱興,再拜。一人進授皇太子,一人授妃,皇太子及妃俱坐,祭酒,舉酒,司饌各以肝從,司則進受虛爵,奠於篚。司饌又俱洗爵,酌酒,再酳,皇太子及妃俱受爵飲。三酳用巹,如再酳。皇太子及妃立於席後,司則俱降東階,洗爵,升,酌於戶外,北面,俱奠爵,興,再拜。皇太子及妃俱答拜。司則坐,取爵祭酒,遂飲,啐爵,奠,遂拜,執爵興,降,奠爵於篚。司饌奏「徹饌」。司則前跪奏稱:「司則妾姓言,請殿下入。」皇太子入於東房,釋冕服,著褲褶。司則啟妃入幃幄,皇太子乃入室。媵餕皇太子之饌,御餕妃之饌。



 親王納妃。



 其納採、問名、納吉、納徵、請期,使者公服,乘犢車,至於妃氏之家,主人受於廟若寢。其賓主相見,儐贊出入升降,與其禮賓者,大抵皆如皇太子之使,而無副。其聘,以玄纁束、乘馬,玉以璋。冊命之日,使者持節,有副。



 親迎。王袞冕輅車,至於妃氏之門外,主人布席於室戶外之西,西上,右幾。又席於戶內,南向。設甒醴於東房東北隅,篚在尊南,實觶一、角柶一,脯棨又在其南。妃於房內即席,南向立,姆立於右。主人立於戶之東,西面。內贊者以觶酌醴,加柶,覆之,面柄,進筵前,北面。妃降席西,南面再拜,受觶。內贊者薦脯棨,妃升席,跪,左執觶,右取脯,擩於棨,祭於籩、豆之間,遂以又柶祭醴三,始扱一祭,又扱再祭,興,筵末跪,啐醴,建柶,奠觶,降筵西,南面再拜,就席立。主人乃迎賓。其餘皆如皇太子之迎。



 初昏,設洗於東階東南,又設妃洗於東房近北。饌於東房,障以帷。豆十六,簠、簋各二,璟各二、俎三,羊、豕臘,羊、豕節折,尊、坫於室內北墉下,玄酒在西。又設尊於房戶外之東,無玄酒,坫在南,賓以四爵,合巹。王至,降車以俟;妃至,降車北面立。王南面揖妃以入,及寢門,又揖以入。贊者酌玄酒三注於尊,妃從者設席於奧,東向。王導妃升自西階,入於室,即席東面立。妃入,立於尊西,南面。王盥於南洗,妃從者沃之;妃盥於北洗,王從者沃之。俱復位,立。贊者設饌入,西面,告「饌具。」王揖妃,即對席,西面,皆坐。其先祭而後飯,乃酳祭,至於燭入,皆如太子納妃之禮。公主出降。禮皆如王妃,而納採、問名、納吉、納徵,請期,主人皆受於寢。其賓之辭曰:「國恩貺室於某公之子,某公有先人之禮,使某也請。」主人命賓曰:「寡人有先皇之禮」云。



 其諸臣之子,一品至於三品為一等,玄纁束、乘馬,玉以璋。四品至於五品為一等,玄纁束、兩馬,無璋。六品至於九品為一等,玄纁束、儷皮二,而無馬。儷皮二,內攝之,毛在內,左首,立於幕南。其餘納採、問名、納吉、納徵、請期,大抵皆如親王納妃。



 其親迎之日,大昕,婿之父、女之父告於禰廟若寢。將行,布席於東序,西向;又席於戶牖之間,南向。父公服,坐於東序,西向。子服其上服:一品袞冕,二品勣冕,三品毳冕,四品絺冕,五品玄冕,六品爵弁。庶人絳公服。升自西階,進立於席西,南向。贊者酌酒進,北面以授子,子再拜受爵。贊者薦脯棨於席前,子升席,跪,左執爵,右取脯、擩於棨,祭於籩、豆之間。右祭酒,執爵興,降席西,南面跪,卒爵,再拜,執爵興。贊者受虛爵還尊所。子進,立於父席前,東面、父命之曰:「往迎爾相,承我宗事,勖率敬,先妣之嗣,若則有常。」庶子但云:「往迎爾相,勖率以敬。」子再拜曰:「不敢忘命。」又再拜,降,出,乃迎。



 初昏,設洗、陳饌皆如親王。牲用少牢及臘,三俎、二籩、二簠,其豆數:一品十六,二品十四,三品十二。婿及婦共牢,婦之簋、簠及豆、登之數,各視其夫。尊於室中北墉下,設尊於房戶外之東,加冪、勺,無玄酒。夫婦酌於內,尊四,爵兩,巹凡六,夫婦各三酳。主人乘革輅,至於婦氏大門外。女準其夫服,花釵、翟衣,入於房,以觶酌醴,如王妃。主人迎賓以入,遂同牢,皆如親王納妃之禮。



 質明,布舅席於東序,西向;布姑席於房戶外之西,南向。舅姑即席,婦執棗、粟入,升自西階,東面再拜,進,跪奠於舅席前,舅撫之,婦退,復位,又再拜。降自西階,受腶脩,升,進,北面再拜,進,跪奠於姑席前,姑舉之,婦退,復位,又再拜。婦席於姑西少北,南向。側尊甒醴於房內東壁下,籩、豆一,實以脯颭,在尊北。設洗於東房近北。婦立於席西,南面。內贊者盥手,洗觶,酌醴,加柶,面柄,北面立於婦前。婦進,東面拜受,復位。內贊者西階上,北面拜送,乃薦脯棨。婦升席,坐,左執觶,右取脯,擩於棨,祭於籩、豆之間,以柶祭醴三,始扱一祭,又扱再祭,加柶於觶,面葉,興,降席西,東面坐,啐醴,建柶,興,拜。內贊者答拜。婦進升席,跪,奠觶於豆東,取脯,降自西階以出,授婦氏從人於寢門外。



 盥饋。舅、姑入於室,婦盥饋。布席於室之奧,舅、姑共席坐,俱東面南上。贊者設尊於室內北墉下,饌於房內西墉下,如同牢。牲醴皆節折,右載之於舅俎,左載之於姑俎。婦入,升自西階,入房,以醬進。其他饌,從者設之,皆加匕箸。俎入,設於豆東。贊者各授箸,舅、姑各以篚菹擩於醬,祭於籩、豆之間,又祭飯訖,乃食。三飯,卒食。婦入於房,盥手洗爵,入室,酌酒酳舅,進奠爵舅席前少東,西面再拜,舅取爵祭酒,飲之。婦受爵出戶,入房,奠於右。盥手洗爵,酌酒酳姑。設婦席於室內北墉下,尊東面,婦徹饌,設於席前如初,西上。婦進,西面再拜,退,升席,南向坐。將餕,舅命易醬,內贊者易之。婦乃餕姑饌,婦祭,內贊者助之。既祭,乃食,三飯,卒食。內贊者洗爵酌酒酳,婦降席,西面再拜,受爵,升席坐,祭酒,飲訖,執爵興,降席東,南面立。內贊者受爵,奠於坫。婦進,西面再拜,受爵,升席坐,祭酒,飲訖,執爵興,降席東,南面立。內贊者受,奠於篚,婦進,西面再拜。舅、姑先降自西階,婦降自阼階。凡庶子婦,舅不降,而婦降自西階以出。



\end{pinyinscope}