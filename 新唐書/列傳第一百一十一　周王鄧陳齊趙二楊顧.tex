\article{列傳第一百一十一 周王鄧陳齊趙二楊顧}

\begin{pinyinscope}

 周寶,字上珪,平州盧龍人。曾祖待選,為魯城令,安祿山反輯實證主義的派別。以萊欣巴赫和亨普爾(CarlGustav,率縣人拒戰,死之。祖光濟,事平盧節度希逸為牙將,每戰,得攻魯城者,必手屠之。歷左贊善大夫,從李洧以徐州歸天子。父懷義,通書記,擢累檢校工部尚書、天德西城防禦使,以徙城事不為宰相李吉甫所助,以憂死。



 寶藉廕為千牛備身。天平節度使殷侑嘗為懷義參軍,寶從之,為部將。會昌時,選方鎮才校入宿衛,與高駢皆隸右神策軍,歷良原鎮使,以善擊球,俱備軍將,駢以兄事寶。寶強毅,未嘗詘意於人。官不進,自請以球見,武宗稱其能,擢金吾將軍。以球喪一目。進檢校工部尚書、涇原節度使。務耕力,聚糧二十萬斛,號良將。



 黃巢據宣、歙,徙寶鎮海軍節度兼南面招討使。巢聞,出採石,略揚州。僖宗入蜀,加檢校司空。時群盜所在盤結,柳超據常熟,王敖據昆山,王騰據華亭,宋可復據無錫。寶練卒自守,發杭州兵戍縣鎮,判八都:石鏡都,董昌主之;清平都,陳晟主之;於潛都,吳文舉主之;鹽官都,徐及主之;新登都,杜棱主之;唐山都,饒京主之;富春都,文禹主之;龍泉都,凌文舉主之。



 中和二年,進同中書門上平章事,兼天下租庸副使,封汝南郡王。寶和裕,喜接士,以京師陷賊,將赴難,益募兵,號「後樓都」。明年,董昌據杭州,柳超自常熟入睦州,刺史韋諸殺之。四年,餘杭鎮使陳晟攻諸,諸以州授晟。寶子璵統後樓都,孱不能馭軍,部伍橫肆。寶亦稍惑色,不恤事,以婿楊茂實為蘇州刺史,重斂,人不聊。田令孜以趙載代之,茂實不受命。寶表留,不聽,乃殘郛署、污垣牖去。詔以王蘊代載,載留潤州。



 初,鎮海將張鬱以擊球事寶。光啟初,劇賊剽昆山,寶遣鬱領兵三百戍海上,鬱醉而叛。王蘊謂州兵還休,不設備,鬱遂大掠,蘊嬰城守。寶遣將拓拔從討定之。鬱保常熟,因攻常州,刺史劉革迎降,眾稍集。寶遣將丁從實督兵攻之,鬱走海陵,依鎮遏使高霸,從實遂據常州。及董昌徙義勝軍節度使,寶承制擢杭州都將錢鏐領州事。宣州賊李君旺陷義興守之。是時,右散騎常侍沈誥使至江南,負田令孜勢,震暴州縣。嗣襄王下令搜令孜黨,寶收誥及趙載殺之。



 高駢領鹽鐵,闢寶子佶為支使,寶亦表駢從子在幕府。駢為都統,浸不禮寶,寶銜之。帝在蜀,淮南絕貢賦,謾言道浙西為寶剽阻。帝知其誣,不直駢,自是顯隙。駢出屯東塘,約西定京師,寶喜,將赴之,或曰:「高氏欲圖公地。」寶未信。駢遣人請會金山,謀執寶,寶答曰:「平時且不聞境上會,況上蒙塵,宗廟焚辱,寧高會時耶?我非李康,不能為人作功勛、欺朝廷也。」駢遣人切讓,寶亦詬絕之。



 會部將劉浩、刁頵與度支催勘使、太子左庶子薛朗叛,寶方寢,外兵格鬥,火照城中。寶驚出,諭曰:「為吾用則吾兵,否則寇也。六州皆我鎮,何往不適?」乃自青陽門出奔。士大掠,官屬崔綰、陸鍔、田倍皆死。浩奉朗領府事。寶至奔牛埭,駢饋以愬葛,諷其且亡也。寶抵於地曰:「公有呂用之,難方作,無誚我!」即奔常州依丁從實,召後樓都,無一士至者。



 錢鏐遣杜棱、成及攻薛朗,棱子建徽攻從實,聲言迎寶,擊破賊君旺,取船八百艘,遂圍常州,從實奔海陵。鏐具橐鞬迎寶,舍樟亭。未幾,殺之,不淹月,而駢為畢師鐸所囚。寶死,年七十四,贈太保。鏐以杜棱守常州。文德元年拔潤州,劉浩亡,不知所在,執朗,剖其心祭寶,使阮結守潤州。楊行密殺高霸,而張鬱、丁從實皆死。



 初,黃巢平,時溥遣小史李師悅上符璽,拜湖州刺史。昭宗時,遷忠國軍節度使。董昌反,師悅連和,與鏐有隙,而結好於行密,安仁義次潤州,復助之。乾寧三年卒,子繼徽代,以地附行密,其將沈攸謂不可,繼徽乃奔揚州。



 陳晟據睦州十八年死,弟詢代立,畏鏐忌己,因徐綰亂,與田頵通。鏐割桐廬隸杭州,詢遂絕鏐,攻蘭溪,鏐使方永珍擊詢。天祐元年,行密遣將闞晊、陶雅救之,執鏐弟鎰、大將王求、顧全武等。未幾,鏐將楊習攻婺州,詢乃奔楊渥,渥以金師會守之。及鏐破衢州,師會走,鏐取其地。



 王處存,京兆萬年人。世籍神策軍,家勝業里,為天下高貲。父宗,巧射利,侈靡自奉,僮千人,以此奮,累除檢校司空、金吾大將軍,遙領興元節度使。



 處存自右軍鎮使歷檢校刑部尚書、定州制置使,累遷義武節度使。黃巢陷京師,處存號哭,不俟詔,分麾下兵二千間道至山南衛乘輿。外約王重榮連盟,進屯渭橋,而涇州行軍司馬唐弘夫亦屯渭北。詔處存檢校尚書右僕射督戰,俄拜東南面行營招討使。中和二年授京城東面都統。每痛國難未夷,語輒流涕,軍中多處存義,愈為之用。素善李克用,又故婚好,遣使十輩曉譬迎勸,卒共平京師。王鐸差興復功,以勤王舉義處存為第一,收城破賊克用為第一。遷檢校司空。復出兵三千屬大將張公慶會諸軍捕巢泰山,滅之。進檢校司徒、同中書門下平章事。



 田令孜討王重榮,徙處存節度河中,上書言:「重榮有大功,不可改易,搖諸侯之心。」不納,趣上道。軍次晉州,刺史冀君武閉門不內,而重榮拒詔。



 處存臨事通便宜,有大將風。幽、鎮兵悍馬強,其地勢也,而易定介於其間,侵軼歲至。及李匡威得志,謀並取之。處存善修鄰歡,內撫民有恩,痛折節下賢,協穆太原以自助,遠近同心。歲時講兵,與諸鎮抗,無能侵軋者。累加侍中、檢校太尉。卒,年六十五,贈太子太師,謚曰忠肅。



 三軍跡河朔舊事,推子郜由副使為留後,昭宗從之。累拜節度使,加檢校司空、同中書門下平章事,又進太保。



 光化三年,硃全忠使張存敬攻幽州,以瓦橋濘潦,道祁溝關。郜方與劉守光厚,乃畀叔處直兵擾其尾,令騎將甄瓊章次義豐,而存敬游奕騎已至,且戰且引十餘里,執瓊章。而氏叔琮下深澤,執大將馬少安,圍祁州,屠之,斬刺史楊約,休兵十日。處直壁沙河,存敬軍河北,挑戰,處直不出,涉河乃戰,處直大敗,亡大將十五,士死者數萬。存敬收械甲以賦戰士,而焚其餘,遂圍定州。郜斬親吏梁汶,移書存敬,且請盟。俄而外郛陷,郜以其族奔太原,使處直主留後。全忠亦至,處直辭曰:「敝邑事上未嘗不忠,事鄰未嘗不禮,弗虞君之見攻也。」全忠責何故事克用,答曰:「太原藉兄弟之舊,修好往來,常道也。君茍為罪,請改圖。」全忠許之。處直以從孫為質,上所持節,即獻絹三十萬,具牛酒犒師。存敬取成而還。全忠表處直為節度留後、檢校尚書左僕射。



 郜至太原,克用表為檢校太尉,卒。



 處直,字允明,天復初為太原郡王。



 鄧處訥,字沖韞,邵州龍潭人。少從江西人閔頊防秋安南。中和元年還,道潭州,逐觀察使李裕,召諸州戍校徇曰:「天下未定,今與君等安護州邑,以待天子命,若何?」眾稱善。乃推頊為留後,請諸朝。僖宗方在蜀,遣使者撫慰。當是時,撫州刺史鐘傳據洪州,議者欲二盜相噬,即復置鎮南軍,擢頊節度使。頊悟,不受命。更為檢校尚書右僕射、欽化軍節度使,以處訥為邵州刺史。



 朗州武陵人雷滿者,本漁師,有勇力。時武陵諸蠻數叛,荊南節度使高駢擢滿為裨將,將鎮蠻軍從駢淮南。逃歸,與里人區景思獵大澤中,嘯亡命少年千人,署伍長,自號「朗團軍」。推滿為帥,景思為司馬,襲州,殺刺史崔翥。詔授朗州兵馬留後。歲略江陵,焚廬落,劫居人。俄進武貞軍節度使。先是,陬溪人周岳與滿狎,因獵宰肉不平而鬥,欲殺滿,不克。見滿已據州,悉眾趨衡州,逐刺史徐顥,詔授衡州刺史。石門峒酋向瑰聞滿得志,亦集夷獠數千,屠牛勞眾,操長刀柘弩寇州縣,自稱「朗北團」。陷澧州,殺刺史呂自牧,自稱刺史。



 頊既強大,且治人有恩,哀徐顥窮,率兵納之。向瑰召梅山十峒獠斷邵州道,頊掩其營。周岳羸軍誘戰,頊墮伏中,故大敗。淮西將黃皓殺頊。岳聞亂,以輕兵入潭州,自稱欽化軍節度使。處訥聞之哭,諸將入吊。處訥曰:「與君等荷僕射恩,若合一州之兵問周岳罪,奈何?」眾曰:「善。」於是礪甲訓兵,積八年,結雷滿為援,攻岳斬之,自稱留後。昭宗詔拜武安軍節度使。



 不三日,會劉建鋒、馬殷兵至,攻澧陵,處訥遣邵州豪傑蔣勛、鄧繼崇率兵三千斷龍回關。勛以牛酒犒師,殷說勛曰:「劉公勇智絕人,術家言當興翼、軫間。今精兵十萬,攻必下,戰必克,收敗眾以餉軍,公裒鄉兵捍關,殆矣。不如下之,富貴可得也。」勛謂然。又其下畏建鋒虐,夜棄甲走。建鋒至關,曰:「此天意也!」盡用邵旗鎧趨潭州。守者以為勛軍,納之。既入,處訥方宴,執而殺之。建鋒許勛賞,未及行,遣請,弗許,勛怒,率鄧繼崇攻湘鄉,取邵州,進壁定勝、武安。建鋒使殷督諸將擊之,殷大敗,走江滸。鄉人夏侯陟教殷以奇兵出迪田,逾澗山,據江為壁,伏兵於莽,誘勛度江。勛見士未陣,爭出鬥,殷分兵襲其壁,麾瀕江軍夾擊,勛大敗,拔定勝一壁,進圍邵州。未下而建鋒死,殷代為節度使。勛請和,不許,卒禽勛斬之。



 是時,道州蠻酋蔡結、何庾,衡人楊師遠各據州叛。宿人魯景仁從黃巢為盜,至廣州,病不能去,以千騎留連州,眾饑,從蔡結求糧,乃相倚杖,與州戍將黃行存誘工商四五千人據連州。郴人陳彥謙殺刺史董岳,發官帑募士,自稱都統,勝兵四千。零陵人唐行旻乘巢亂,脅眾自防,盜永州,殺刺史鄭蔚,與景仁合從,數遣諜殷虛實,完壘自守。



 殷遣將李瓊攻永州,殺行旻。李瑭攻道州,蔡結約峒獠為援,久不勝,謀曰:「蠻所恃,林藪耳。」乃屯大川,伐山焚林,獠驚走。城陷,執蔡結、何庾,殷斬之。李瓊出耒陽、常寧,攻郴州,陳彥謙出戰,軍亂不能陣,斬彥謙。進圍連州,魯景仁乘城守,三日不下,夜焚其門入之,景仁自刺死。



 頊,字公謹。滿,字秉仁。岳,字峻昭。行旻,字昌圖。



 滿不修飭,每宴使客,抵寶器潭中,曰:「此水府也,蛟龍所憑,吾能沒焉。」乃裸入水,俄取器以出。累遷檢校太尉、同中書門下平章事。天復元年卒。子彥威自立。間荊南節度使成汭兵出。襲江陵,入之,焚樓船,殘墟落,數千里無人跡。弟彥恭,結忠義節度趙匡凝以逐彥威,據江陵。匡凝弟匡明擊之,還走朗州。



 陳儒,江陵人。世為牙右職。廣明元年,以鄭紹業為荊南節度使,時朗州刺史段彥謨方據荊南,紹業憚之,逾半歲乃至。僖宗入蜀,召紹業還行在,以彥謨代節度。彥謨與監軍硃敬玫不平,謀殺之。敬玫覺,先率兵入其府,彥謨方寢,拔劍縋城奔親軍壘,不得入,彥謨曰:「而等負我!」俄見害,親屬僚佐皆死。敬玫以少尹李燧為留後,且誣彥謨以罪。帝遣中人似先元錫、王魯琪慰撫,密戒曰:「若敬玫可誅,誅之,以爾代而魯琪為副。」敬玫盛兵出迎,元錫等不敢發而還。復詔鄭紹業為節度使,逗留不進。



 敬玫署儒領府事。明年,遷檢校工部尚書,為節度使,進檢校右僕射。敬玫有悍卒三千,號「忠勇軍」,暴甚,儒不能制。初,紹業將申屠琮率兵五千援京師,既歸,儒告以忠勇撓治,琮請除之。大將程君從聞之,率眾奔澧州,琮追斬百餘人,軍乃潰。已而琮復顓軍。雷滿三以兵薄城,儒厚啖以利,乃去。



 淮南將張瑰、韓師德據復、岳二州,自署刺史。儒請瑰攝行軍司馬,師德攝節度副使,共擊滿。師德兵上峽,大略去。瑰引兵逐儒,儒將奔行在,既又劫還,囚之。瑰,滑州人,暴勇而殘,荊故將夷戮幾盡。時以楊玄晦代敬玫監軍,召敬玫還成都,懼帝治前罪,稱疾自解。前此數殺大將富商,故積賄,每曝衣,紈繡不可計。瑰見心動,遣卒賊之。敬玫衣黃衣,盜刺其腹死。



 秦宗言來寇,馬步使趙匡欲奉儒出,瑰覺之,殺匡而絕儒食,七日死。瑰固壘二歲,樵蘇皆盡,米斗錢四十千,計抔而食,號為「通腸」。疫死者,爭啖其尸,縣首於戶以備饌。軍中甲鼓無遺,夜擊闔為警。宗言不能下,乃解去。二年,宗權遣趙德諲攻瑰,瑰求救於歸州刺史郭禹,禹率峽州刺史潘章解圍。明年,德諲又至,諸將困於戰,城遂陷,瑰死,人無識者,並尸於井。復州長史陳璠從瑰至江陵,密斷瑰首置囊中,走京師獻之,授安州刺史。



 劉巨容,徐州人。為州大將。龐勛之反,自拔歸,授埇橋鎮遏使。浙西突陣將王郢反,攻明州,巨容以筒箭射郢死,拜明州刺史,徙楚州團練使。



 黃巢亂江淮,授蘄黃招討副使,徙襄州行軍司馬、檢校右散騎常侍。巢據荊南,俄遷山南東道節度使以捍巢,屯團林。江西招討使曹全晸與巨容守荊門關,與賊戰,巨容偽北,巢追之,伏興林樾間,賊大敗,執賊將十三人,轉鬥一舍,虜獲不可計。巢浮江東奔,巨容追之,率十俘八,以功遷檢校禮部尚書。諸將欲乘勝追斬巢,巨容止曰:「朝家多負人,有危難,不愛惜官賞,事平即忘之,不如留賊,為富貴作地。」諸將謂然,故巢復熾。及陷兩京,巨容合諸道兵討之,授南面行營招討使,累兼天下兵馬先鋒開道供軍糧料使、檢校司空,封彭城縣侯。



 巨容明吏治。時僖宗在蜀,公卿多因巨容護赴行在。山南西道節度使鹿晏弘為禁軍所逐,引麾下東出襄、鄧。秦宗權遣趙德諲合晏弘兵攻襄州,巨容不能守,奔成都。



 始,揚州人申屠生能化黃金,高駢客之,為呂用之所譖,亡奔襄、漢,駢遣吏捕得,生見巨容自言其術,巨容留不遣。田令孜之弟道襄州,巨容出金誇之。及在蜀,匿生,使術不得傳,令孜恨之。龍紀元年,殺巨容,夷其宗,生並死。



 巨容部將馮行襲者,均州武當人,以謀勇稱里中。中和初,鄉豪孫喜聚眾數千人,謀攻城。行襲伏士江隩,以單舟迎喜曰:「州人思得將軍久矣。顧將軍兵多必剽掠,若留眾江北,以輕騎進,我為鄉導,城可下。」喜信之。既度江,吏出迎,伏甲興,行襲擊喜,斬之,眾皆潰。行襲乘勝逐刺史呂燁,據均州,巨容因表為刺史。



 帝在蜀,均之右有長山,當襄、漢貢道,有劇賊據險劫獻物,行襲平之。武定節度使楊守忠表為行軍司馬,使領兵搤谷口以通秦、蜀。鳳翔李茂貞養子繼臻據金州,行襲攻拔之,昭宗即授金州防禦使。時山南西道節度使楊守亮將襲京師,道金、商,行襲逆戰破之,就擢戎昭軍節度使。硃全忠圍鳳翔,神策中尉韓全誨遣中人二十輩督江、淮兵過其州,行襲方附全忠,盡殺之,收詔書送全忠。



 天祐二年,王建遣將王思綰攻行襲,敗其兵,州大將金行全出降,行襲奔均州。建以行全為子,更名宗朗,授觀察使,以渠、巴、開三州隸之。宗朗不能守,焚郭邑去。全忠以行襲不足禦建,遣別將屯金州。行襲議徙戎昭軍於均州,以金、房為隸。全忠以金人不樂行襲,以馮恭領州,罷防禦使而廢戎昭軍。



 趙德諲,蔡州人。從秦宗權為右將,以討黃巢功授申州刺史。光啟初,與秦誥、鹿晏弘合兵攻襄州,節度使劉巨容奔成都。宗權假德諲山南東道節度留後,進攻荊南,悉收寶貲,留裨將王建肇守之,遺人才數百室。明年,歸州刺史郭禹來討,建肇納之,奔黔州。德諲失荊南,又度宗權必敗,舉地附硃全忠。全忠方為蔡州四面行營都統,即表以自副,加忠義軍節度使。宗權平,加中書令,封淮安郡王,卒。子匡凝嗣。



 匡凝,字光儀,由唐州刺史自為山南東道節度留後,昭宗即授節度使,不三年,以威惠聞。累遷檢校太尉兼中書令。匡凝矜嚴盛飾,前後持鑒自照。



 全忠之敗清口,匡凝與奉國節度使崔洪、河東李克用、淮南楊行密約合兵攻全忠。會方城鎮遏使度軫奔全忠,發其謀。全忠移書切責,使氏叔琮攻唐州,刺史趙匡璠降。進圍隋州,執刺史趙匡璘,斬首五千級;拔鄧州,執刺史國湘。匡凝懼,乞盟。



 全忠使親將陳俊、王紳入叔琮軍,崔洪留之,紳亡歸。洪與行密欲邀友恭軍,不克。會河東客伊超使淮南還,過蔡,洪亦留之,因是並俊送全忠,以部將苛拘為解,遣兄賢入質,全忠還之,質洪子於汴。全忠使賢調蔡卒二千出戍。將行,大將崔景思不悅,殺賢,洪懼,驅民趨申州,遂奔行密,麾鼓亙百餘里。武昌杜洪邀之,弗及,蔡士多亡去,從者才二千人。



 天祐元年,封匡凝為楚王。時諸道不上供,唯匡凝歲貢賦天子。全忠方圖天下,遣人諭止之,匡凝流涕曰:「吾為國屏翰,渠敢有他志!」副使王筠勸絕全忠,全忠怒,出兵攻之。弟匡明大破汴軍於鄧州,因勸匡凝與王建連和。及荊南成汭敗,匡凝取江陵,表匡明為荊南節度留後,有詔拜檢校司徒、荊南節度行軍司馬。



 全忠以其兵分可圖也,乃使楊師厚攻匡凝,自將中軍繼之,屯臨漢。匡凝遣客謝,囚不遣,敗荊南救兵,俘其將。全忠循江而南,師厚繇陰谷伐木為梁。匡凝以兵二萬瀕江戰,大敗,乃燔州,單舸夜奔揚州。行密見之曰:「君在鎮,輕車重馬輸於賊,今敗乃歸我邪?」筠自殺。全忠以師厚為山南東道節度留後,遂趨江陵。匡明亦謀奔淮南,子承規諫曰:「昔諸葛兄弟分仕二國,若適揚州,是自取疑也。」匡明謂然,乃趨成都,王建待以賓禮,授武信軍節度使,分其眾為崇義、勇義、順義、廣義四都,全忠遂有荊南。



 楊守亮,曹州人,本姓訾,名亮。與弟信俱從王仙芝為盜。亮身長七尺餘,色如鐵。仙芝死,又事徐唐莒,劫剽洪、饒二州。楊復光平江西,得其兄弟,養為假子,以信養於弟復恭家,曰守亮、守信。復恭收京師。守亮以戰多,拜山南西道節度使、檢校太保,守信興平軍節度使,並同中書門下平章事。復恭又以假子守貞為龍劍節度使,守忠為武定軍節度使,守厚為綿州刺史。



 初,硃玫取興、鳳州,虢州刺史滿存以兵赴行在,復收二州,昭宗擢為感義軍節度使,累檢校司徒、同中書門下平章事,與復恭四假子及利閬觀察使席儔等共攻王建。建軍已圍楊晟,分軍逼守厚,軍未成列而敗。先是,守貞、守忠聞建兵出,拔眾奔綿州,並力共攻東川,弗勝。建將華洪以兵萬人壁綿州之郊,敗守忠、守厚,二人分道行,收兵趨閬州。



 始,復恭敗,依守亮。而鳳翔李茂貞、邠寧王行瑜、鎮國韓建等共劾守亮納叛人,請以鎮兵討之。茂貞自為興元節度使,以書誚責宰相。帝為削守亮官爵,因詔茂貞問罪。滿存來救不克,以眾入興元。茂貞拔興、鳳、洋三州,破守亮於西,乘勝入興元。復恭挾諸假子及存奔閬州。洪進圍之。帝以徐彥若帥鳳翔,以興元授茂貞。茂貞不肯拜,帝乃以其子繼密為興元節度使。



 俄而洪拔閬州,守亮等皆挺身走,將北奔太原,趨商山,饑甚,丐食於野,為邏戍所縛,見韓建,守亮視建左右八百人皆常隸己,語建曰:「此屬吾養之素厚,無一為我死。公無費衣食,不如殺之。」建許諾。復曰:「公幸貸我,俾生見天子,陳先人功,萬有一不死。」建檻車送京師,吏縛以帛,內球於口。帝御延喜樓問反狀,守亮不得語,頷而已。左右白服罪,即執獻太廟,斬獨柳下,梟於市。守厚死巴州,麾下兵多歸王建。存奔京師,為左武衛大將軍。



 楊晟,不詳宗系。隸鳳翔軍,節度使李昌符畏其勇,欲殺之,妾周擿使亡去,隸神策軍為都校。僖宗在陳倉,邠寧硃玫遣萬騎合昌符追行在,乃擢晟感義軍節度使、檢校司空,守大散關。玫兵攻關,晟數卻,戰潘氏,遂大敗,內外無固志。帝更徙興元,晟西奔,玫取興、鳳二州。晟襲文州,逐刺史,據成、龍、茂等州。



 王建攻成都,田令孜以晟故將,與連和,假威戎軍節度使,守彭州。晟擊建,無功引還。且畏建圖己,乃約山南西道節度使楊守亮兄弟合謀拒建,掠新繁,焚漢州,又攻東川顧彥暉,為建兵所逐。建使王宗裕率騎五萬圍晟,食四郊麥,掠民資產。晟假子實以騎八千降於建,建以奇兵襲楊守厚等,皆亡去。晟開門決戰,大敗,遂約降。建饋十羊,晟曰:「以我為機上肉乎?」不出。建築甬道屬陴以入,斬晟首。



 晟有仁心,下懷其恩,雖城中食盡,無叛者。初,昌符死,晟得其妾周,母事之,周請為妻,晟固辭,旦夕問省,乃視事。愛將安師建者,勇而有禮,既就執,建顧曰:「爾報楊司徒足矣,能從我乎?」謝曰:「司徒誓同死生,不忍復戴日月。」三謂不回,乃戮之。



 顧彥朗、彥暉者,豐州人,並為天德軍小校。其使蔡京以兄弟有封侯相,每厚禮之,使子贈賚,稍稍進秩。黃巢亂長安,率軍同復京師。



 彥朗遷累右衛大將軍。光啟中,擢拜東川節度使、檢校太保、同中書門下平章事。至劍門,陳敬瑄吏奪其節,彥朗不得入,保利州。敬瑄誣劾彥朗擅興兵掠西境。僖宗下詔申曉講和,乃得到軍。署彥暉漢州刺史。



 初,楊守亮忌壁州刺史王建兇暴,欲逐之。建聞,合溪洞豪酋取閬州,擊利州,刺史走,即據二州,守亮不能制。彥朗與建雅舊,陰助貲饟。建攻成都,彥朗挾故憾,與並力,道路鄣梗。敬瑄告難於朝,帝詔和解,又敕李茂貞鐫諭。



 會彥朗卒,彥暉自知留後。明年為節度使。中人送節,為綿州刺史楊守厚所留。守厚發兵攻梓州,彥暉告急於建,建使李簡救之,戒曰:「賊破,並取彥暉,無須再往也。」簡破守厚軍,彥暉辭疾,不克取。建素有吞噬心,以彥朗與婚婭,久未忍。及彥暉,則交好愈疏,而境上關賦相稽詬,建怒。景福元年,遂攻彥暉。彥暉請救於楊守亮,遣楊子彥戍梓,執建大將王宗弼,彥暉責曰:「王公何以見討?君為大將,不諫云何?」宗弼謝罪,即解縛,使就館,帟幕衾服皆具,更養為子,改名琛。明年,建將華洪破綿州,守厚走,得彥暉節。時詔已進彥暉檢校司空、東川節度使矣。



 乾寧二年,昭宗在石門,督彥暉、建赴行在。建率兵二十萬次綿州,即劾彥暉劫輜運,回襲之。彥暉不敢出,但遣人塞建舟路,建遂擊取巴、閬、蓬、渠、通、果、龍、利八州。帝遣中人為兩川宣諭協和使。建奉詔還,而兵不解。彥暉謀窘,因大略漢、眉、資、簡等州。李茂貞亦欲爭其地,使子興元節度使繼密引軍救彥暉,以窺東川。四年,華洪眾五萬攻彥暉,取渝、昌、普三州,壁梓州南,敗彥暉兵,奪鎧馬八百,凡五十戰,圍遂固。帝仍遣左諫議大夫李洵諭止,建拒命。帝以嗣郯王戒丕鎮鳳翔,徙茂貞代建,皆不奉詔。



 梓有鏡堂,世稱其麗,彥暉嘗會諸將堂上,養子瑤尤親信,彥暉以所佩劍號「疥癆賓」佩之,使侍左右。嘗語諸將曰:「與公等生死同之,違者先齒『疥癆賓』!」眾曰:「諾。」及圍急,瑤請聚親信飲,得同死。彥暉顧王琛曰:「爾非我舊,可自求生。」指頹垣令逸。彥暉手殺妻子,乃自刎,宗族諸將皆死,麾下兵猶七萬。



 初,韋昭度為招討使,彥暉、建皆為大校。彥暉詳緩有儒者風,建左右髡發黥面若鬼,見者皆笑。至是錄笑者皆殺之。私署洪為東川節度留後。



 贊曰:《詩》云「戎狄是膺,荊舒是懲」,嫉其為中國之害也。春秋之世,楚滅陳、鄭,而卒復其祀,聖人善之。處存平黃巢,定京師,功冠諸將。昭宗嘗有意都襄陽,依趙凝以自全。大抵唐室屏翰,皆為硃溫所翦覆,過於夷狄、荊舒之為害也,甚矣。



\end{pinyinscope}