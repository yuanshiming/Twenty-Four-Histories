\article{志第三十九上 百官四上}

\begin{pinyinscope}

 ○十六衛



 △左右衛



 上將軍各一人,從二品;大將軍各一人,正三品;將軍各二人,從三品。掌宮禁宿衛,凡五府及外府皆總制焉。凡五府三衛及折沖府驍騎番上者,受其名簿而配以職。皇帝御正殿,則守諸門及內廂宿衛仗。非上日,亦將軍一人押仗,將軍缺,以中郎將代將軍,掌貳上將軍之事。左右驍衛、左右武衛、左右威衛、左右領軍、左右金吾、左右監門衛上將軍以下,品同。武德五年,改左右翊衛曰左右衛府,左右驍騎衛曰左右驍騎府,左右屯衛曰左右威衛,左右御衛曰左右領軍衛,左右備身府曰左右府,唯左右武衛府、左右監門府、左右候衛,仍隋不改。顯慶五年,改左右府曰左右千牛府。龍朔二年,左右衛府、驍衛府、武衛府,皆省「府」字,左右威衛曰左右武威衛,左右領軍衛曰左右戎衛,左右候衛曰左右金吾衛,左右監門府曰左右監門衛,左右千牛府曰左右奉宸衛,後又曰左右千牛衛。咸亨元年,改左右戎衛曰領軍衛。武後光宅元年,改左右驍衛曰左右武威,左右武衛曰左右鷹揚衛,左右威衛曰左右豹韜衛,左右領軍衛曰左右玉鈐衛。貞元二年,初置十六衛上將軍。左右衛有錄事一人,府一人,史二人,亭長八人,掌固四人。



 長史各一人,從六品上。掌判諸曹、五府、外府稟祿,卒伍、軍團之名數,器械、車馬之多少,小事得專達,每歲秋,贊大將軍考課。



 錄事參軍事各一人,正八品上。掌受諸曹及五府、外府之事,句稽抄目,印給紙筆。



 倉曹參軍事各二人,正八品下。掌五府文官勛考、假使、祿俸、公廨、田園、食料、醫藥、過所。自倉曹以下同品。有府二人,史四人。兵曹,府四人,史七人。騎曹,府二人,史四人。胄曹,府三人,史三人。武后長安初,改鎧曹曰胄曹,中宗即位復舊,先天元年又曰胄曹。開元初,諸衛司倉、司兵、騎兵參軍,改曰倉曹、兵曹、騎曹、胄曹參軍事。兵曹參軍事各二人,掌五府武官宿衛番第,受其名數,而大將軍配焉。騎曹參軍事各一人,掌外府雜畜簿帳、牧養。凡府馬承直,以遠近分七番,月一易之。以敕出宮城者,給馬。胄曹參軍事各一人,掌兵械、公廨興繕、罰謫。大朝會行從,則受黃質甲鎧、弓矢於衛尉。



 奉車都尉,掌馭副車。有其名而無其人,大陳設則它官攝。駙馬都尉無定員,與奉車都尉皆從五品下。



 司階各二人,正六品上;中候各三人,正七品下;司戈各五人,正八品下;執戟各五人,正九品下;長上各二十五人,從九品下。武後天授二年,置諸衛司階、中候、司戈、執戟,謂之四色官。



 親衛之府一:曰親府。勛衛之府二:一曰勛一府,二曰勛二府。翊衛之府二:一曰翊一府,二曰翊二府。凡五府:每府中郎將一人,正四品下;左右郎將一人,正五品上;親衛,正七品上;勛衛,從七品上;翊衛,正八品上。總四千九百六十三人。兵曹參軍事各一人,正九品上;校尉各五人,正六品上。每校尉有旅帥二人,從六品上;每旅帥各有隊正二十人,正七品上,副隊正二十人,正七品下。五府中郎將掌領校尉、旅帥、親衛、勛衛之屬宿衛者,而總其府事;左右郎將貳焉。番上者,以名簿上於大將軍而配以職。武德、貞觀世重資廕,二品、三品子,補親衛;二品曾孫、三品孫、四品子、職事官五品子若孫、勛官三品以上有封及國公子,補勛衛及率府親衛;四品孫、五品及上柱國子,補翊衛及率府勛衛;勛官二品及縣男以上、散官五品以上子若孫,補諸衛及率府翊衛。王府執仗親事、執乘親事,每月番上者數千人,宿衛內廡及城門,給稟食。執扇三衛三百人,擇少壯肩膊齊、儀容整美者,本衛印臂,送殿中省肄習,仗下,每番三衛一人,為太僕寺引輅。其後入官路艱,三衛非權勢子弟輒退番,柱國子有白首不得進者;流外雖鄙,不數年給祿稟。故三衛益賤,人罕趨之。有錄事一人,府一人,史三人。唐親衛、勛衛置驃騎將軍、車騎將軍,翊衛置車騎將軍。武德七年,改驃騎將軍為中郎將,車騎將軍皆為郎將,分左右,以親衛曰一府,勛衛、翊衛曰二府,謂之三府衛。諸衛翊衛及率府親、勛衛,亦曰三衛。永徽三年,避太子諱,改中郎將曰旅賁郎,郎將曰翊軍郎。太子廢,復舊。



 △左右驍衛



 上將軍各一人,大將軍各一人,將軍各二人。掌同左右衛。凡翊府之翊衛、外府豹騎番上者,分配之。凡分兵守諸門,在皇城四面、宮城內外,則與左右衛分知助鋪。長史各一人,錄事參軍事各一人,倉曹參軍事各二人,兵曹參軍事各二人,騎曹參軍事各一人,胄曹參軍事各一人,左右司階各二人,左右中候各三人,左右司戈各五人,左右執戟各五人。左右翊中郎將府中郎將各一人,左郎將各一人,右郎將各一人,兵曹參軍事各一人,校尉各五人,旅帥各十人,隊正各二十人,副隊正各二十人。有錄事一人,史二人,亭長二人,掌固四人。倉曹,府二人,史二人;兵曹,府三人,史五人;騎曹,府二人,史四人;胄曹,府三人,史三人。左右翊中郎將府錄事一人、府一人、史二人。



 △左右武衛



 上將軍各一人,大將軍各一人,將軍各二人,掌同左右衛。凡翊府之翊衛、外府熊渠番上者,分配之。長史各一人,錄事參軍事各一人,倉曹參軍事各二人,兵曹參軍事各二人,騎曹參軍事各一人,胄曹參軍事各一人,左右司階各二人,左右中候各三人,左右司戈各五人,左右執戟各五人,長上各二十五人。左右翊中郎將府官,同驍衛。有稱長二人,錄事一人,史二人,亭長二人,掌固四人。倉曹,府二人,史四人;兵曹,府三人,史五人;騎曹,府二人,史四人;胄曹,府三人,史三人。稱長掌唱警,為應蹕之節。



 △左右威衛



 上將軍各一人,大將軍各一人,將軍各二人,掌同左右衛。凡翊府之翊衛、外府羽林番上者,分配之。凡分兵主守,則知皇城東面助鋪。長史各一人,錄事參軍事各一人,倉曹參軍事各二人,兵曹參軍事各二人,騎曹參軍事各一人,胄曹參軍事各一人,左右司階各二人,左右中候各三人,左右司戈各五人,左右執戟各五人,長上各二十五人。左右翊中郎將府官,同驍衛。有錄事一人,史二人,亭長二人,掌固四人。倉曹,府二人,史四人;兵曹,府三人,史五人;騎曹,府二人,史四人;胄曹,府三人,史三人。



 △左右領軍衛



 上將軍各一人,大將軍各一人,將軍各二人,掌同左右衛。凡翊府之翊衛、外府射聲番上者,分配之。凡分兵主守,則知皇城西面助鋪及京城、苑城諸門。長史各一人,錄事參軍事各一人,倉曹參軍事各二人,兵曹參軍事各二人,騎曹參軍事各一人,胄曹參軍事各一人,左右司階各二人,左右中候各三人,左右司戈各五人,左右執戟各五人,長上各二十五人。左右翊中郎將府官,同驍衛。有錄事一人,史二人,亭長二人,掌固四人。倉曹,府二人,史四人;兵曹,府三人,史五人;騎曹,府二人,史四人;胄曹,府三人,史三人。



 △左右金吾衛



 上將軍各一人,大將軍各一人,將軍各二人。掌宮中、京城巡警,烽候、道路、水草之宜。凡翊府之翊衛及外府佽飛番上,皆屬焉。師田,則執左右營之禁,南衙宿衛官將軍以下及千牛番上者,皆配以職。大功役,則與御史循行。凡敝幕、故氈,以給病坊。兵曹參軍事,掌翊府、外府武官,兼掌獵師。騎曹參軍事,掌外府雜畜簿帳、牧養之事。胄曹參軍事,掌同左右衛。大朝會行從,給青龍旗、槊於衛尉。長史各一人,錄事參軍事各一人,倉曹參軍事各二人,兵曹參軍事各二人,騎曹參軍事各一人,胄曹參軍事各一人,左右司階各二人,左右中候各三人,左右司戈各五人,左右執戟各五人,左右街使各一人,判官各二人。左右翊中郎將府官如驍衛。有錄事一人,史二人。倉曹,府二人,史四人;兵曹,府三人,史五人;騎曹,府二人,史四人;胄曹,府三人,史三人。左右街典二人,引駕仗三衛六十人,引駕佽飛六十六人,大角手六百人。隋有察非掾,至唐廢。



 左右翊中郎將府中郎將,掌領府屬,督京城左右六街鋪巡警,以果毅二人助巡探。入閤日,中郎將一人升殿受狀,衛士六百為大角手,六番閱習,吹大角為昏明之節,諸營壘候以進退。



 左右街使,掌分察六街徼巡。凡城門坊角,有武候鋪,衛士、彍騎分守,大城門百人,大鋪三十人,小城門二十人,小鋪五人,日暮,鼓八百聲而門閉;乙夜,街使以騎卒循行囂襜,武官暗探;五更二點,鼓自內發,諸街鼓承振,坊市門皆啟,鼓三千撾,辨色而止。



 △左右監門衛



 上將軍各一人,大將軍各一人,將軍各二人。掌諸門禁衛及門籍。文武官九品以上,每月送籍於引駕仗及監門衛,衛以帳報內門。凡朝參、奏事、待詔官及繖扇儀仗出入者,閱其數。以物貨器用入宮者,有籍有傍。左監門將軍判入,右監門將軍判出,月一易其籍。行幸,則率屬於衙門監守。



 長史,掌判諸曹及禁門,巡視出入而司其籍、傍。餘同左右衛。兵曹參軍事兼掌倉曹,胄曹兼掌騎曹。



 左右翊中郎將府中郎將,掌涖宮殿城門,皆左入右出。中郎將各四人,長史各一人,錄事參軍事各一人,兵曹參軍事各一人,胄曹參軍事各一人。有錄事一人,史二人,亭長二人,掌固二人。兵曹,府三人,史五人;胄曹,府三人,史四人。監門校尉三百二十人,直長六百八十人,長入長上二十人,直長長上二十人。監門校尉掌敘出入。唐改監門府郎將為將軍。



 △左右千牛衛



 上將軍各一人,大將軍各一人,將軍各二人。掌侍衛及供御兵仗。以千牛備身左右執弓箭宿衛,以主仗守戎器。朝日,領備身左右升殿列侍。親射,則率屬以從。胄曹參軍事掌甲仗。凡御仗之物二百一十有九,羽儀之物三百,自千牛以下分掌之。上日,執御弓箭者亦自備以入宿。主仗每月上,則配以職,行從則兼騎曹。中郎將各二人,長史各一人,錄事參軍事各一人,兵曹參軍事各一人,胄曹參軍事各一人。唐改備身郎將曰將軍,備身將曰中郎將,千牛左右、備身左右曰千牛備身。初置備身主仗。有錄事一人,史二人,亭長二人,掌固四人。兵曹,府一人,史二人;胄曹,府一人,史一人。千牛備身十二人,備身左右十二人,備身一百人,主仗一百五十人。千牛備身掌執御刀,服花鈿繡衣綠,執象笏,宿衛侍從。備身左右掌執御弓矢,宿衛侍從。備身,掌宿衛侍從。主仗,掌守供御兵仗。



 左右翊中郎將府中郎將,掌供奉侍衛。凡千牛及備身左右以御刀仗升殿供奉者,皆上將軍領之,中郎將佐其職。有口敕,通事舍人承傳,聲不下聞者,中郎將宣告。



 ○諸衛折沖都尉府



 每府折沖都尉一人,上府正四品上,中府從四品下,下府正五品下。左右果毅都尉各一人,上府從五品下,中府正六品上,下府正六品下。別將各一人,上府正七品下,中府從七品上,下府從七品下。長史各一人,上府正七品下,中府從七品上,下府從七品下。兵曹參軍事各一人,上府正八品下,中府正九品下,下府從九品上。校尉五人,從七品下。旅帥十人,從八品上。隊正二十人,正九品下;副隊正二十人,從九品下。



 折沖都尉掌領屬備宿衛,師役則總戎具、資糧、點習,以三百人為團,一校尉領之。捉鋪持更者,晨夜有行人必問,不應則彈弓而向之,復不應則旁射,又不應則射之。晝以排門人遠望,暮夜以持更人遠聽。有眾而囂,則告主帥。



 左右果毅都尉,掌貳都尉。每府有錄事一人,府一人,史二人。丘曹,府二人,史三人。每隊正領兵五十人。武德元年,改鷹揚郎將曰軍頭,正四品下;鷹擊郎將曰府副,正五品上;司馬曰長史,正八品下;校尉,正六品下,旅帥,正七品下。廢越騎、步兵二校尉及察非掾。又改軍頭曰驃騎將軍,府副曰車騎將軍,皆為府。諸率府置驃騎將軍五人、車騎將軍十人。二年,以車騎將軍府隸驃騎府,置十二軍,分關內諸府皆隸焉。每軍,將軍一人,副一人。至六年廢。七年,改驃騎將軍府為統軍府,車騎將軍為別將。八年,復置十二軍。貞觀十年,改統軍府曰折沖都尉,別將曰果毅都尉。軍坊置坊主一人,檢校戶口,勸課農桑,以本坊五品勛官為之。三輔及近畿州都督府皆置府,凡六百三十三。永徽中,廢長史,置司馬一人,總司兵、司騎二局。武后垂拱中,以千二百人為上府,千人為中府,八百人為下府,赤縣為赤府,畿縣為畿府。聖歷元年,廢司馬,置長史、兵曹參軍事,又有別將一人,從六品下,居果毅都尉之次,其後分左右各一人,尋廢。久之,復置一人,降其品。開元初,衛士為武士,諸衛折沖、果毅、別將,擇有行者為展仗押官。右羽林軍十五人,左羽林軍二十五人,衣服同色。諸衛有弩手,左右驍衛各八十五人,餘衛各八十三人。



 △左右羽林軍



 大將軍各一人,正三品;將軍各三人,從三品。掌統北衙禁兵,督攝左右廂飛騎儀仗。大朝會,則周衛階陛。巡幸,則夾馳道為內仗。凡飛騎番上者,配其職。有敕上南衙者,大將軍承墨敕,白移於金吾,引駕仗官與監門奏覆,降墨敕,然後乃得入。長史各一人,從六品上;錄事參軍事各一人,正八品上;倉曹參軍事各一人,兼總騎曹事;兵曹參軍事各一人;胄曹參軍事各一人。自倉曹參軍以下,皆正八品下。司階各二人,正六品上;中候各三人,正七品下;司戈各五人,正八品上;執戟各五人,正九品下;長上各十人。左右翊衛中郎將府中郎將一人,正四品下;左右中郎一人,左右郎將一人,皆正五品上;兵曹參軍事一人,正九品上;校尉五人,旅帥十人,隊正二十人,副隊正二十人。有錄事一人,史二人,亭長二人,掌固四人。倉曹、兵曹各府二人、史四人;胄曹,府、史各二人。左右翊中郎將府,錄事一人,府一人,史二人;倉曹、兵曹各府二人,史四人;胄曹,府、史各二人。



 △左右龍武軍



 大將軍各一人,正二品;統軍各一人,正三品;將軍三人,從三品。掌同羽林。長史、錄事參軍事、倉曹參軍事、兵曹參軍事、胄曹參軍事各一人,司階各二人,中候各三人,司戈、執戟各五人,長上各十人。景雲元年,置龍武將軍。興元元年,六軍各置統軍。貞元三年,龍武軍增將軍一員,有錄事一人,史二人,亭長二人,掌固四人。倉曹,府二人,史四人;兵曹,府二人,史四人;胄曹,府、史各二人。



 △左右神武軍



 大將軍各一人,正二品;統軍各一人,正三品;將軍三人,從三品。總衙前射生兵。長史、錄事參軍事、倉曹參軍事、兵曹參軍事、胄曹參軍事各一人,司階各二人,中候各三人,司戈、執戟各五人,長上各十人。有錄事一人,史二人,倉曹、兵曹、胄曹府、史,皆如龍武軍。開元二十六年,分羽林置左右神武軍,尋廢;至德二年復置。



 △左右神策軍



 大將軍各一人,正二品;統軍各二人,正三品;將軍各四人,從三品。掌衛兵及內外八鎮兵。護軍中尉各一人,中護軍各一人,判官各三人,都句判官二人,句覆官各一人,表奏官各一人,支計官各一人,孔目官各二人,驅使官各二人。自長史以下,員數如龍武軍。左右龍武、左右神武、左右神策,號六軍。貞元二年,神策軍置大將軍、將軍,十四年置統軍,品秩同六軍。始,殿前左右神威軍,有大將軍二人,正二品;統軍二人,從三品;將軍二人,從五品。元和初,為一軍,號天威軍。八年廢,以軍隸神策,有馬軍、步軍將軍及指揮使等,以馬軍大將軍知軍事。天復三年廢神策軍,四年復置神策軍。



 △東宮官



 太子太師、太傅、太保各一人,從一品。掌輔導皇太子。每見,迎拜殿門,三師答拜,每門必讓,三師坐,太子乃坐。與三師書,前名惶恐,後名惶恐再拜。太子出,則乘路備鹵簿以從。



 少師、少傅、少保各一人,從二品。掌曉三師德行,以諭皇太子,奉太子以觀三師之道德。自太師以下唯其人,不必備。先天元年開府,置令、丞各一人,隸詹事府。尋廢。



 太子賓客四人,正三品。掌侍從規諫,贊相禮儀,宴會則上齒。侍讀,無常員,掌講導經學。貞觀十八年,以宰相兼賓客。開元中,定員四人。太宗時,晉王府有侍讀,及為太子亦置焉。其後,或置或否。開元初,十王宅引辭學工書者入教,亦為侍讀。



 △詹事府



 太子詹事一人,正三品;少詹事一人,正四品上。掌統三寺、十率府之政,少詹事為之貳。皇太子書稱令,庶子以下署名奉行,書案、畫日。丞二人,正六品上。掌判府事,知文武官簿、假使。凡敕令及尚書省、二坊符牒下東宮諸司者,皆發焉。主簿一人,從七品上;錄事二人,正九品下。隋廢詹事府。武德初復置。龍朔二年曰端尹府,詹事曰端尹,少詹事曰少尹。武後光宅元年改曰宮尹府,詹事曰宮尹,少詹事曰少尹。有令史九人,書令史十八人。



 司直二人,正七品上。掌糾劾宮寮及率府之兵。皇太子朝,則分知東西班。監國,則詹事、庶子為三司使,司直一人與司議郎、舍人分日受理啟狀。太子出,則分察鹵簿之內。有令史一人,書令史二人,亭長四人,掌固六人。



 △左春坊



 左庶子二人,正四品上;中允二人,正五品下。掌侍從贊相,駁正啟奏。總司經、典膳、藥藏、內直、典設、宮門六局。皇太子出,則版奏外辦、中嚴;入則解嚴。凡令書下,則與中允、司議郎等畫諾、覆審,留所畫以為案,更寫印署,注令諾,送詹事府。司議郎二人,正六品上。掌侍從規諫,駁正啟奏。凡皇太子出入、朝謁、從祀、釋奠、講學、監國之命,可傳於史冊者,錄為記注;宮坊祥眚,官長除拜、薨卒,歲終則錄送史館。左諭德一人,正四品下。掌諭皇太子以道德,隨事諷贊。皇太子朝宮臣,則列侍左階,出入騎從。左贊善大夫五人,正五品上。掌傳令,諷過失,贊禮儀,以經教授諸郡王。錄事二人,從八品下;主事三人,從九品下。隋有內允。武德三年改曰中舍人,隸門下坊。貞觀初曰中允,十八年置司議郎。永徽三年,避皇太子名,復改中允曰內允。太子廢,復舊。龍朔二年,改門下坊曰左春坊,左庶子曰左中護,中允曰左贊善大夫,司議郎分左右,置左右諭德各一人。咸亨元年,皆復舊,司議郎不分左右,其後諭德廢而司議郎復分。儀鳳四年,置左右贊善大夫各十人,以同姓為之。景雲二年,始兼用庶姓,改門下坊曰左春坊,復置諭德,庶子以比侍中,中允以比門下侍郎,司議郎以比給事中,贊善大夫以比諫議大夫,諭德以比散騎常侍。右坊,則庶子以比中書令,中舍人以比中書侍郎,太子監國則庶子比尚書令。有令史六人,書令史十二人,傳令四人,掌儀二人,贊者三人,亭長三人,掌固十人。



 △崇文館



 學士二人,掌經籍圖書,教授諸生,課試舉送如弘文館。校書郎二人,從九品下。掌校理書籍。貞觀十三年置崇賢館。顯慶元年,置學生二十人。上元二年,避太子名,改曰崇文館。有學士、直學士及讎校,皆無常員,無其人則庶子領館事。開元七年,改讎校曰校書郎。乾元初,以宰相為學士,總館事,貞元八年,隸左春坊。有館生十五人,書直一人,令史二人,書令史二人,典書二人,拓書手二人,楷書手十人,熟紙匠一人,裝潢匠二人,筆匠一人。



 △司經局



 洗馬二人,從五品下。掌經籍,出入侍從。圖書上東宮者,皆受而藏之。文學三人,正六品下。分知經籍,侍奉文章。校書四人,正九品下;正字二人,從九品上。掌校刊經史。唐改太子正書曰正字。龍朔三年,改司經局曰桂坊,罷隸左春坊,領崇賢館,比御史臺;以詹事一人為令,比御史大夫,司直二人比侍御史,以洗馬為司經大夫。置文學四人,錄事一人,正九品下。三年,改司經大夫曰桂坊大夫,糾正違失。咸亨元年,復隸左春坊,省錄事。有書令史二人,書吏二人,典書四人,楷書二十五人,掌固六人,裝潢匠二人,熟紙匠、筆匠各一人。



 △典膳局



 典膳郎二人,從六品下;丞二人,正八品上。掌進膳、嘗食,丞為之貳。每夕,更直於廚。龍朔二年,改典膳監曰典膳郎。有書令史二人,書吏四人,主食六人,典食二百人,掌固四人。



 △藥藏局



 藥藏郎二人,從六品下;丞二人,正八品上。掌和醫藥,丞為之貳。皇太子有疾,侍醫診候議方。藥將進,宮臣涖嘗,如尚藥局之職。有書令史一人,書吏二人,侍醫四人,典藥二人,藥童六人,掌固四人。



 △內直局



 內直郎二人,從六品下;丞二人,正八品下。掌符璽、衣服、繖扇、幾案、筆硯、垣墻。龍朔二年,改監曰內直郎,副監曰丞。有令史一人,書吏三人,典服十二人,典扇八人,典翰八人,掌固六人。武德中,有典璽四人,開元中廢。



 △典設局



 典設郎四人,從六品下;丞二人,正八品下。掌湯沐、燈燭、汛掃、鋪設。凡皇太子散齋別殿、致齋正殿,前一日設幄坐於東序及室內,張帷前楹。龍朔二年,改齋帥局曰典設局,齋帥曰郎。有書令史二人,書吏四人,幕士二百四十五人,掌固十二人。



 △宮門局



 宮門郎二人,從六品下;丞二人,正八品下。掌宮門管鑰。凡夜漏盡,擊漏鼓而開;夜漏上水一刻,擊漏鼓而閉。歲終行儺,則先一刻而啟。皇太子不在,則闔正門;還仗,如常。凡宮中,明時不鼓。龍朔三年,改宮門監曰宮門郎。有書令史一人,書吏二人,門僕百人,掌固四人。



 △右春坊



 右庶子二人,正四品下;中舍人二人,正五品下。掌侍從、獻納、啟奏,中舍人為之貳。皇太子監國,下令書則畫日,至春坊則庶子宣傳,中舍人奉行。太子舍人四人,正六品上。掌行令書、表啟。諸臣上皇太子,大事以箋,小事以啟,其封題皆上右春坊通事舍人以進。通事舍人八人,正七品下。掌導宮臣辭見,承令勞問。右諭德一人,右贊善大夫五人,錄事一人,主事二人,品皆如左春坊。隋內舍人隸典書坊。武德初改曰中舍人,管記舍人曰太子舍人。永徽元年,避太子名,復改中舍人曰內舍人。龍朔二年,改典書坊曰右春坊,右庶子曰右中護,中舍人曰右贊善大夫,舍人曰右司議郎。有令史九人,書令史十八人,傳令四人,典謁四人,亭長六人,掌固十人。



 △家令寺



 家令一人,從四品上。掌飲膳、倉儲。總食官、典倉、司藏三署。皇太子出入,則乘軺車為導;祭祀、賓客,則供酒食;賜予,則奉金玉、貨幣。凡床幾、茵席、器物,非取於將作、少府者,皆供焉。丞二人,從七品下,掌判寺事。凡三署出納,皆刺於詹事。莊宅、田園,審肥脊為收斂之數。宮、朝、坊、府土木營繕,則下於司藏。主簿一人,正九品下。唐改司府令曰家令。龍朔二年,改家令寺曰宮府寺,家令曰大夫。有錄事一人,府十人,史二十人,亭長四人,掌固四人,雜匠百人。



 △食官署



 令一人,從八品下;丞二人,從九品下。掌飲膳、酒醴。凡四時供送設食皆顓焉。供六品以下元日、寒食、冬至食於家令廚者。有府二人,史四人,掌膳四人,供膳百四十人,奉觶三十人。



 △典倉署



 令一人,從八品下;丞二人,從九品下。掌九穀、醯醢、庶羞、器皿、燈燭。凡園圃樹藝,皆受令焉。每月籍出納上於寺,歲終上詹事府。給戶奴婢、番戶、雜戶資糧衣服。有府三人,史五人,園丞二人,史二人。



 △司藏署



 令一人,從八品下;丞二人,從九品下。掌庫藏財貨出納、營繕。有府三人,史四人,計史一人。



 △率更寺



 令一人,從四品上。掌宗族次序、禮樂、刑罰及漏刻之政。太子釋奠、講學、齒胄,則總其儀;出入,乘軺車為導,居家令之次。坊、寺、府有罪者,論罰,庶人杖以下,皆送大理。皇太子未立,則斷於大理。丞一人,從七品上。掌貳令事。宮臣有犯理於率更者,躬問蔽罪而上於詹事。主簿一人,正九品下。掌印局。凡宗族不序,禮儀不節,音律不諧,漏刻不審,刑名不法,皆舉而正之。決囚,則與丞同涖。龍朔二年,改曰司更寺,令曰司更大夫。有錄事一人,府三人,史四人,漏刻博士三人,掌漏六人,漏童二十人,典鐘、典鼓各十二人,亭長四人,掌固四人。漏刻博士掌教漏刻。



 △僕寺



 僕一人,從四品上。掌車輿、乘騎、儀仗、喪葬,總廄牧署。太子出,則率廄牧令進路,親馭。丞一人,從七品上。掌判寺事。凡馬畜芻粟,歲以季夏上於詹事,以時出入而節其數。主簿一人,正九品下。掌廄牧畜養、車騎駕馭、儀仗。龍朔二年,改曰馭僕寺,僕曰大夫。有進馬十一人,錄事一人,府三人,史五人,亭長三人,掌固三人。



 △廄牧署



 令一人,從八品下;丞二人,從九品下。掌車馬、閑廄、牧畜。皇太子出,則率典乘先期習路馬,率駕士馭車乘,既出,進路,式路車於西閤外,南向以俟。凡群牧隸東宮者,皆受其職事。典乘四人,從九品下。有府三人,史六人,翼馭十人,駕士十五人,掌閑六百人,獸醫十人,主酪三十人。翼馭掌調馬執馭。



 △太子左右率府



 率各一人,正四品上;副率各二人,從四品上。掌兵仗、儀衛。凡諸曹及三府、外府皆隸焉。元日、冬至,皇太子朝宮臣、諸方使,則率衛府之屬為衛。每月三府三衛及五府超乘番上者,配以職。武德五年,改左右侍率曰左右衛率府,左右武侍衛率曰左右宗衛率府,左右宮門將曰左右監門率府。龍朔二年,改左右衛率府曰左右典戎衛,左右宗衛率府曰左右司禦率府,左右虞候率府曰左右清道衛,左右內率府曰左右奉裕衛,左右監門率府曰左右崇掖衛。武后垂拱中,改左右監門率府曰左右鶴禁衛。神龍元年,改左右司禦率府曰左右宗衛府,左右清道衛曰左右虞候率府。景雲二年,左右宗衛府復曰左右司禦率府。開元初,左右虞候率府復曰左右清道率府。



 長史各一人,正七品上。掌判諸曹府。季秋以屬官功狀上於率,而為考課。



 錄事參軍事各一人,從八品上;倉曹參軍事、兵曹參軍事、胄曹參軍事、騎曹參軍事各一人,從八品下。倉曹掌文官簿書,兵曹掌武官簿書,胄曹掌器械、公廨營繕。司階各一人,從六品上;中候各二人,從七品下;司戈各二人,從八品上;執戟各三人,散長上各十人,從九品下。左右司御、清道、監門、內率府,自率以下品同。有錄事一人,府一人,史一人。倉曹,府一人,史二人;兵曹、胄曹,各府二人,史三人;騎曹,府五人,史七人。亭長、掌固各二人。



 △親府、勛府、翊府三府



 每府中郎將各一人,從四品上;左右郎將各一人,正五品下。中郎將、郎將,掌其府校尉、旅帥及親、勛、翊衛之屬宿衛,而總其事。



 兵曹參軍事各一人,從九品上。掌判句。大朝會及皇太子出,則從鹵簿而涖其儀。親衛從七品上,勛衛正八品上,翊衛從八品上,員皆亡。校尉各五人,從六品上;旅帥各十人,正七品下;隊正各二十人,從八品上。武德元年,改功曹曰親衛,義曹曰勛衛,良曹曰翊衛,置三府,有錄事二人,府、史各一人。



 △太子左右司禦率府



 率各一人,正四品上;副率各二人,從四品上。掌同左右衛。凡諸曹及外府旅賁番上者隸焉。長史各一人,正七品上;錄事參軍事各一人,從八品上;倉曹參軍事、兵曹參軍事、胄曹參軍事、騎曹參軍事各一人,從八品下;司階各一人,中候各二人,司戈各二人,執戟各三人。親衛、勛衛、翊衛三府中郎將以下,如左右衛率府。有錄事一人,史二人。倉曹,府一人,史二人;兵曹,府二人,史三人;胄曹,府、史各二人。亭長一人,掌固二人。



 △太子左右清道率府



 率各一人,副率各二人。掌晝夜巡警。凡諸曹及外府直蕩番上者隸焉。皇太子出入,則以清游隊先導,後拒隊為殿。長史各一人,錄事參軍事各一人,從八品上;倉曹參軍事、兵曹參軍事、胄曹參軍事各一人,從八品下;左右司階各一人,左右中候各二人,左右司戈各一人,左右執戟各三人。親衛、勛衛、翊衛三府中郎將以下,如左右衛率府。有錄事一人,史二人,亭長二人,掌固二人。倉曹,府一人,史二人;兵曹,府二人,史三人;胄曹,府二人,史二人。細引押仗五十人。



 △太子左右監門率府



 率各一人,副率各二人。掌諸門禁衛。凡財物、器用,出者有籍。長史各一人,錄事參軍事各一人,正九品上;兵曹參軍事各一人,正九品下,兼領倉曹;胄曹參軍事各一人,正九品下;監門直長七十八人,從七品下。唐改宮門將曰監門率,直事曰直長。有錄事一人,史二人,亭長一人,掌固二人。兵曹,府二人,史二人;胄曹,府二人,史三人。



 △太子左右內率府



 率各一人,副率各一人。掌千牛供奉之事。皇太子坐日,領千牛升殿。射於射宮,則千牛奉弓矢立東階,西面;率奉弓,副率奉矢、決拾。北面張弓,左執付,右執蕭以進,副率以弓拂巾而進,各退立於位。既射,左內率啟其中否。長史各一人,錄事參軍事各一人,正九品上;兵曹參軍事各一人,正九品下,兼領倉曹。胄曹參軍事各一人,正九品下;千牛各四十四人,從七品上。唐置兵曹,改司使左右復曰千牛備身,主射左右復曰備身左右,弓箭備身去弓箭之名。龍朔二年,改千牛備身曰奉裕。開元中,千牛備身、備身左右,並為千牛。有備身二十八人,主仗四十人,錄事一人,史二人。兵曹,府一人,史二人。胄曹,府一人,史一人。



\end{pinyinscope}