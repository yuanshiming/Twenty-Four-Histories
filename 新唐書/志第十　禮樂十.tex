\article{志第十 禮樂十}

\begin{pinyinscope}

 五曰兇禮。



 《周禮》五禮,二曰兇禮。唐初,徙其次第五,而李義府、許敬宗以為兇事非臣子所宜言,遂去其《國恤》一篇,由是天子兇禮闕焉。至國有大故,則皆臨時採掇附比以從事,事已,則諱而不傳,故後世無考焉。至開元制禮,惟著天子賑恤水旱、遣使問疾、吊死、舉哀、除服、臨喪、冊贈之類,若五服與諸臣之喪葬、衰麻、哭泣,則頗詳焉。



 凡四方之水、旱、蝗,天子遣使者持節至其州,位於庭,使者南面,持節在其東南,長官北面,寮佐、正長、老人在其後,再拜,以授制書。其問疾亦如之,其主人迎使者於門外,使者東面,主人西面,再拜而入。其問婦人之疾,則受勞問者北面。



 若舉哀之日,為位於別殿,文武三品以上入,哭於庭,四品以下哭於門外。有司版奏「中嚴」、「外辦」。皇帝已變服而哭,然後百官內外在位者皆哭,十五舉音,哭止而奉慰。其除服如之。皇帝服:一品錫衰,三品以上緦衰,四品以下疑衰。服期者,三朝晡止;大功,朝晡止;小功以下,一哀止。晡,百官不集。若為蕃國君長之喪,則設次於城外,向其國而哭,五舉音止。



 若臨喪,則設大次於其門西,設素裀榻於堂上。皇帝小駕、鹵簿,乘四望車,警蹕,鼓吹備而不作。皇帝至大次,易素服,從官皆易服,侍臣則不。皇帝出次,喪主人免絰、釋杖、哭門外,望見乘輿,止哭而再拜,先入門右,西向。皇帝至堂,升自阼階,即哭位。巫、祝各一人先升,巫執桃立於東南,祝執茢立於西南,戈者四人先後隨升。喪主人入廷再拜,敕引乃升,立戶內之東,西向。皇帝出,喪主人門外拜送。皇帝變服於次,乃還廬。文、武常服。皇帝升車,鼓吹不作而入。其以敕使冊贈,則受冊於朝堂,載以犢車,備鹵薄,至第。妃主以內侍為使。贈者以蠟印畫綬。冊贈必因其啟葬,既葬則受於靈寢,既除則受於廟。主人公服而不哭,或單衣而介幘。受必有祭。未廟,受之寢。



 五服之制。



 斬衰三年。正服:子為父,女子子在室與已嫁而反室為父。加服:嫡孫為後者為祖,父為長子。義服:為人後者為所後父,妻為夫,妾為君,國官為君。王公以下三月而葬,葬而虞,三虞而卒哭。十三月小祥,二十五月大祥,二十七月禫祭。



 齊衰三年。正服:子,父在為母。加服:為祖後者,祖卒則為祖母,母為長子。義服:為繼母、慈母,繼母為長子,妾為君之長子。



 齊衰杖周。降服:父卒,母嫁及出妻之子為母,報,服亦如之。正服:為祖後者,祖在為祖母。義服:父卒,繼母嫁,從,為之服報;夫為妻。



 齊衰不杖周。正服:為祖父母,為伯叔父,為兄弟,為眾子,為兄弟之子及女子子在室與適人者,為嫡孫,為姑、姊妹與無夫子,報,女子子與適人為祖父母,妾為其子。加服:女子子適人者為兄弟之為父後者。降服:妾為其父母,為人後者為其父母,報,女子子適人者為其父母。義服:為伯叔母,為繼父同居者,妾為嫡妻,妾為君之庶子,婦為舅、姑,為夫之兄弟之子,舅、姑為嫡婦。



 齊衰五月。正服:為曾祖父母,女子子在室及嫁者亦如之。



 齊衰三月。正服:為高祖父母,女子子在室及嫁者亦如之。義服:為繼父不同居者。



 其父卒母嫁,出妻之子為母,鋼為祖後,祖在為祖母,雖周除,仍心喪三年。



 大功,長殤九月,中殤七月。正服:為子、女子子之長殤、中殤,為叔父之長殤、中殤,為姑、姊妹之長殤、中殤,為兄弟之長殤、中殤,為嫡孫之長殤、中殤,為兄弟之子、女子之長殤、中殤。義服:為夫之兄弟之子、女子子之長殤、中殤。成人九月正服:為從兄弟,為庶孫。降服:為女子子適人者,為姑、姊妹適人者報;出母為女子子適人者,為兄弟之女適人者報;為人後者為其兄弟與姑、姊妹在室者報。義服:為夫之祖父母與伯叔父母報,為夫之兄弟女適人者報;夫為人後者,其妻為本生舅、姑,為眾子之婦。



 小功五月殤。正服:為子、女子子之下殤,為叔父之下殤,為姑、姊妹之下殤,為兄弟之下殤,為嫡孫之下殤,為兄弟之子、女子子之下殤,為從兄弟姊妹之長殤,為庶孫之長殤。降服:為人後者為其兄弟之長殤,出嫁姑為侄之長殤,為人後者為其姑、姊妹之長殤。義服:為夫之兄弟之子、女子子之下殤,為夫之叔父之長殤。成人正服:為從祖祖父報,為從祖父報,為從祖姑、姊妹在室者報,為從祖兄弟報,為從祖祖姑在室者報,為外祖父母,為舅及從母報。降服:為從父姊妹適人者報,為孫女適人者,為人後者為其姑、姊妹適人者報。義服:為從祖祖母報,為從祖母報,為夫之姑、姊妹在室及適人者報,娣姒婦報,為同母異父兄弟姊妹報,為嫡母之父母兄弟從母,為庶母慈己者,為嫡孫之婦,母出為繼母之父兄弟從母,嫂叔報。



 緦麻三月殤。正服:為從父兄弟姊妹之中殤、下殤,為庶孫之中殤、下殤,為從祖叔父之長殤,為從祖兄弟之長殤,為舅及從母之長殤,為從父兄弟之子之長殤,為兄弟之孫長殤,為從祖姑、姊妹之長殤。降服:為人後者為其兄弟之中殤、下殤,為侄之中殤、下殤,出嫁姑為之報,為人後者為其姑、姊妹之中殤、下殤。義服:為人後者為從父兄弟之長殤,為夫之叔父之中殤、下殤,為夫之姑、姊妹之長殤。成人正服:為族兄弟,為族曾祖父報,為族祖父報,為族父報,為外孫,為曾孫、玄孫,為從母兄弟姊妹,為姑之子,為舅之子,為族曾祖姑在室者報,為族祖姑在室者報,為族姑在室者報。降服:為從祖姑、姊妹適人者報,女子子適人者為從祖父報,庶子為父後者為其母,為從祖姑適人者報,為人後者為外祖父母,為兄弟之孫女適人者報。義服:為族曾祖母報,為族祖母報,為族母報,為庶孫之婦,女子子適人者為從祖伯叔母,為庶母,為乳母,為婿,為妻之父母,為夫之曾祖高祖父母,為夫之從祖祖父母報,為夫之從祖父母報,為夫之外祖父母報,為從祖兄弟之子,為夫之從父兄弟之妻,為夫之從父姊妹在室及適人者,為夫之舅及從母報。改葬:子為父母,妻妾為其夫,其冠服杖屨皆依《儀禮》。皇家所絕傍親無服者,皇弟、皇子為之皆降一等。



 初,太宗嘗以同爨緦而嫂叔乃無服,舅與從母親等而異服,詔侍中魏徵、禮部侍郎令狐德棻等議:「舅為母族,姨乃外戚它姓,舅固為重,而服止一時,姨喪乃五月,古人未達者也。於是服曾祖父母齊衰三月者,增以齊衰五月;適子婦大功,增以期;眾子婦小功,增以大功;嫂叔服以小功五月報;其弟妻及夫兄亦以小功;舅服緦,請與從母增以小功。」然《律疏》舅報甥,服猶緦。顯慶中,長孫無忌以為甥為舅服同從母,則舅宜進同從母報。又古庶母緦,今無服,且庶母之子,昆弟也,為之杖齊,是同氣而吉兇異,自是亦改服緦。上元元年,武後請「父在,服母三年」。開元五年,右補闕盧履冰言:「《禮》,父在為母期,而服三年,非也,請如舊章。」乃詔並議舅及嫂叔服,久而不能決。二十年,中書令蕭嵩等改脩五禮,於是父在為母齊衰三年。



 諸臣之喪。



 有疾,齋於正寢,臥東首北墉下。疾困,去衣,加新衣,徹藥,清掃內外。四人坐而持手足,遺言則書之,為屬纊。氣絕,寢於地。男子白布衣,被發徒跣;婦人女子青縑衣,去首飾;齊衰以下,丈夫素冠。主人坐於床東,啼踴無數。眾主人在其後,兄弟之子以下又在其後,皆西面南上,哭。妻坐於狀西,妾及女子在其後,哭踴無數。兄弟之女以下又在其後,皆東面南上,籍槁坐哭。內外之際,隔以行帷。祖父以下為帷東北壁下,南面西上;祖母以下為帷西北壁,南面東上。外姻丈夫於戶外東,北面西上;婦人於主婦西北,南面東上。諸內喪,則尊行丈夫及外親丈夫席位於前堂,若戶外之左右,俱南面。宗親戶東,西上;外親戶西,東上。凡喪,皆以服精粗為序,國官位於門內之東,重行北面西上,俱紵巾帕頭,舒薦坐;參佐位於門內之西,重行北面東上,素服,皆舒席坐,哭。斬衰,三日不食;齊衰,二日不食;大功,三不食;小功、緦麻,再不食。



 復於正寢。復者三人,以死者之上服左荷之,升自前東溜,當屋履危,北面西上。左執領,右執腰,招以左。每招,長聲呼「某復」,三呼止。投衣於前,承以篋,升自阼階,入以覆尸。乃設床於室戶內之西,去腳、簟、枕,施幄,去裙。遷尸於狀,南首,覆用斂衾,去死衣,楔齒以角柶,綴足以燕幾,校在南。其內外哭位如始死之儀。乃奠以脯、醢,酒用吉器。升自阼階,奠於尸東當腢。內喪,則贊者皆受於戶外而設之。



 沐浴。掘坎於階間。近西,南順,廣尺,長二尺,深三尺,南其壤,為役灶於西墻下,東向,以俟煮沐。新盆、瓶、六鬲皆濯之,陳於西階下。沐巾一,浴巾二,用絺若紘,實於,櫛實於箱若簟,浴衣實於篋,皆具於西序下,南上。水淅稷米,取汁煮之,又汲為湯以俟浴。以盆盛潘及沐盤,升自西階,授沐者,沐者執潘及盤入。主人皆出於戶東,北面西上;主婦以下戶西,北面東上。俱立哭。其尊行者,丈夫於主人之東,北面西上;婦人於主婦之西,北面東上。俱坐哭。婦人以帳。乃沐櫛。束發用組。挋用巾。浴則四人抗衾,二人浴,拭用巾,挋用浴衣。設狀於尸東,衽下莞上簟。浴者舉尸,易狀,設枕,翦鬢斷爪如生,盛以小囊,大斂內於棺中。楔齒之柶、浴巾皆埋於坎。寘之。衣以明衣裳,以方巾覆面,仍以大斂之衾覆之。內外入就位,哭。



 乃襲。襲衣三稱,西領南上,明衣裳,舄一;帛巾一,方尺八寸;充耳,白纊;面衣,玄方尺,纁里,組系;握手,玄纁里,長尺二寸,廣五寸,削約於內旁寸,著以綿組系。庶襚繼陳,不用。將襲,具狀席於西階西,內外皆出哭,如浴。襲者以狀升,入設於尸東,布枕席,陳襲於席。祝去巾,加面衣,設充耳、握手,納舄若履。既襲,覆以大斂之衾,內外入哭。



 乃含。贊者奉盤水及,一品至於三品,飯用梁,含用璧;四品至於五品,飯用稷,含用碧;六品至於九品,飯用梁,含用貝。升堂,含者盥手於戶外,洗梁、璧實於,執以入,祝從入,北面,徹枕,去衾,受,奠於尸東。含者坐於狀東,西面,鑿巾,納飯、含於尸口。既含,主人復位。



 乃為明旌,以絳廣充幅,一品至於三品,長九尺,韜杠,銘曰「某官封之柩」,置於西階上;四品至於五品,長八尺;六品至於九品,長六尺。



 鑿木為重,一品至於三品,長八尺,橫者半之,三分庭一在南;四品至於五品,長七尺;六品至於九品,長六尺。以沐之米為粥,實於鬲,蓋以疏布,系以竹幹,縣於重木。覆用葦席,北面,屈兩端交後,西端在上,綴以竹幹。祝取銘置於重,殯堂前楹下,夾以葦席。



 小斂。衣一十九稱,朝服一,笏一,陳於東序,西領北上。設奠於東堂下,甒二,實以醴、酒,觶二,角柶一,少牢,臘三,籩、豆俎各八。設盆盥於饌東,布巾。贊者闢脯醢之,奠於尸狀西南。乃斂,具狀席於堂西,設盆盥西階之西,如東方。斂者盥,與執服者以斂衣人,喪者東西皆少退,內外哭。已斂,覆以夷衾。設狀於堂上兩楹間,衽下莞上簟,有枕。卒斂,開帷,主人以下西面憑哭,主婦以下東面憑哭,退。乃斂發而奠。贊者盥手奉饌至階,升,設於尸東,醴、酒奠於饌南西上,其俎,祝受巾巾之。奠者徹襲,奠,自西階降出。下帷,內外俱坐哭。有國官、僚佐者,以官代哭;無者以親疏為之。夜則為燎於庭,厥明滅燎。



 乃大斂。衣三十稱,上服一稱,冕具簪、導、纓,內喪則有花釵,衾一,西領南上。設奠如小斂,甒加勺,篚在東南。籩、豆、俎皆有幕,用功布。棺入,內外皆止哭,升棺於殯所,乃哭。熬八篚,黍、稷、梁、稻各二,皆加魚、臘。燭俟於饌東,設盆盥於東階東南。祝盥訖,升自阼階。徹巾,執巾者以待於阼階下。祝盥,贊者徹小斂之饌,降自西階,設於序西南,當西溜,如設於堂上。乃適於東階下新饌所,帷堂內外皆少退,立哭。御者斂,加冠若花釵,覆以衾。開帷,喪者東西憑哭如小斂,諸親憑哭。斂者四人舉狀,男女從,奉尸斂於棺,乃加蓋,覆以夷衾,內外皆復位如初。設熬穀,首足各一篚,傍各三篚,以木覆棺上,乃塗之,設帟於殯上,祝取銘置於殯。乃奠。執巾、幾席者升自阼階,入,設於室之西南隅,東面。又幾,巾已加,贊者以饌升,入室,西面,設於席前。祝加巾於俎,奠者降自西階以出。下帷,內外皆就位哭。



 既殯,設靈座於下室西間,東向,施狀、幾、案、屏、帳、服飾,以時上膳羞及湯沐如平生。殷奠之日,不饋於下室。



 廬在殯堂東廊下,近南,設苫塊。齊衰於其南,為堊室。俱北戶,翦蒲為席,不緣;大功又於其南,張帷,席以蒲;小功、緦麻又於其南,設狀,席以蒲。婦人次於西房。



 三日成服,內外皆哭,盡哀。乃降就次,服其服,無服者仍素服。相者引主人以下俱杖升,立於殯,內外皆哭。諸子孫跪哭尊者之前,祖父撫之,女子子對立而哭,唯諸父不撫。尊者出,主人以下降立阼階。



 朔望殷奠,饌於東堂下,瓦甒二,實醴及酒,角觶二,木柶一,少牢及臘三俎,二簋、二簠、二鈃,六籩、六豆。其日,不饋於下室。



 葬有期,前一日之夕,除葦障,設賓次於大門外之右,南向,啟殯之日,主人及諸子皆去冠,以紵巾帕頭,就位哭。祝衰服執功布,升自東階,詣殯南,北向,內外止哭,三聲噫嘻,乃曰:「謹以吉辰啟殯。」既告,內外哭。祝取銘置於重。掌事者升,徹殯塗,設席於柩東,升柩於席。又設席柩東,祝以功布升,拂柩,覆用夷衾,周設帷,開戶東向。主人以下升,哭於帷東,西向,俱南上。諸祖父以下哭於帷東北壁下,諸祖母以下哭於帷西北壁下;外姻丈夫帷東上,婦人帷西。祝與進饌者各以奠升,設於柩東席上,祝酌醴奠之。



 陳器用。啟之夕,發引前五刻,搥一鼓為一嚴,陳布吉、兇儀仗,方相、志石、大棺車及明器以下,陳於柩車之前。一品引四、披六、鐸左右各八、黼翣二、黻翣二、畫翣二,二品三品引二、披四、鐸左右各六、黼翣二、畫翣二,四品五品引二、披二、鐸左右各四、黼翣二、畫翣二,六品至於九品披二、鐸二、畫翣二。



 二刻頃,搥二鼓為二嚴,掌饌者徹啟奠以出,內外俱立,哭。執紼者皆入,掌事者徹帷,持翣者升,以翣障柩。執紼者升,執鐸者夾西階立,執纛者入,當西階南,北面立。掌事者取重出,倚於門外之東。執旌者立於纛南,北面。搥三鼓為三嚴,靈車進於內門外,南向,祝以腰輿詣靈座前,西向跪告。腰輿降自西階,以詣靈車。腰輿退。執鐸者振鐸,降就階間,南向。持翣者障以翣。執纛者卻行而引,輴止則北面立;執旌者亦漸而南,輴止,北面。主人以下以次從。



 輴在庭。輴至庭,主人及諸子以下立哭於輴東北,西向南上;祖父以下立哭於輴東北,南面西上;異姓之丈夫立哭於主人東南,西面北上。婦人以次從降,妻、妾、女子子以下立哭於輴西,東面南上;祖母以下立哭於輴西北,南向東上;異姓之婦人立哭於主婦西南,東面北上。內外之際,障以行帷。國官立哭於執紼者東,北面西上;僚佐立哭於執紼者西南,北面東上。祝帥執饌者設祖奠於輴東,如大斂。祝酌奠、進饌,北面跪曰:「永遷之禮,靈辰不留,謹奉旋車,式遵祖道,尚饗。」



 輴出,升車,執披者執前後披,紼者引輴出,旌先、纛次,主人以下從,哭於車盾後。輴出,到輀車,執紼者解屬於輀車,設帷障於輴後,遂升柩。祝與執饌者設遣奠於柩東,如祖奠。



 既奠,掌事者以蒲葦苞牲體下節五,以繩束之,盛以盤,載於輿前。方相、大棺車、輴車,明器輿、下帳輿、米輿、酒脯醢輿、苞牲輿、食輿為六輿,銘旌、纛、鐸、輀車以次行。賓有贈者,既祖奠,賓立於大門外西廂,東面,從者以篚奉玄纁立於西南,以馬陳於賓東南,北首西上。相者入,受命出,西面曰:「敢請事。」賓曰:「某敢賵。」相者入告,出曰:「孤某須矣。」執篚者奠,取幣以授賓。牽馬者先入,陳於輴車南,北首西上。賓入,由馬西當輴車南,北面立,內外止哭。賓曰:「某謚封若某位,將歸幽宅,敢致賵。」乃哭,內外皆哭。主人拜、稽顙。賓進輴東,西面,奠幣於車上,西出,主人拜、稽顙送之。



 喪至於墓所,下柩。進輴車於柩車之後,張帷,下柩於輴。丈夫在西,憑以哭。卑者拜辭,主人以下婦人皆障以行帷,哭於羨道西,東面北上。



 入墓。施行席於壙戶內之西,執紼者屬紼於輴,遂下柩於壙戶內席上,北首,覆以夷衾。



 輴出,持翣入,倚翣於壙內兩廂,遂以帳張於柩東,南向。米、酒、脯於東北,食盤設於前,醯、醢設於盤南,苞牲置於四隅,明器設於右。



 在壙。掌事者以玄纁授主人,主人授祝,奉以入,奠於靈座,主人拜、稽顙。施銘旌、志石於壙門之內,掩戶,設關鑰,遂復土三。主人以下稽顙哭,退,俱就靈所哭。掌儀者祭后土於墓左。



 反哭。既下柩於壙,搥一鼓為一嚴,掩戶;搥二鼓為再嚴,內外就靈所;搥三鼓為三嚴,徹酒、脯之奠,追靈車於帷外,陳布儀仗如來儀。腰輿入,少頃出,詣靈車後。靈車發引,內外從哭如來儀。出墓門,尊者乘,去墓百步,卑者乘以哭。靈車至於西階下,南向。祝以腰輿詣靈車後。少頃,升,入詣靈座前;主人以下從升,立於靈座東,西面南上;內外俱升。諸祖父以下哭於帷東北壁下,南面;妻及女子以下婦人哭於靈西,東面;諸祖母以下哭於帷西北壁下,南面;外姻哭於南廂,丈夫帷東,婦人帷西,皆北面;吊者哭於堂上,西面。主人以下出就次,沐浴以俟虞,斬衰者沐而不櫛。



 虞。主用桑,長尺,方四寸,孔徑九分,鳥漆匱,置於靈座,在寢室內戶西,東向,素幾在右。設洗於西階西南,瓦甒二、設於北牖下,醴、酒在東。喪者既沐,升靈所。主人及諸子倚仗於戶外,入,哭於位如初。饌入,如殷奠,升自東階。主人盥手洗爵,酌醴,西面跪奠,哭止。祝跪讀祝,主人哭拜,內外應拜者皆哭拜。乃出,杖降西階,還次。間日再虞,後日三虞,禮如初。



 小祥。毀廬為堊室,設蒲席。堊室者除之,席地。主人及諸子沐浴櫛翦,去首絰,練冠,妻妾女子去腰絰。主用慄,祭如虞禮。



 大祥之祭如小祥。間月而禫,釋祥服,而禫祭如大祥。既祥而還外寢。妻妾女子還於寢。食有醢、醬,既禫而飲醴酒,食乾肉。



 祔廟,筮日。將祔,掌事者為坎室於始祖廟室西壁,主人及亞獻以下散齊三日,致齊一日。前一日,主人以酒、脯告遞遷之主,乃遷置於幄坐,又奠酒、脯以安神。掌饌者徹膳以出,掌廟者以次櫝神主納於坎室。又設考之祔坐於曾祖室內東壁下,西向,右幾。設主人位於東南,西面。設子孫位於南門內道東,北面西上。設亞獻、終獻位於主人東南。設掌事以下位於終獻東南,俱西面北上。設贊唱者位於主人西南,西面。設酒尊於堂上室戶之東南,北向西上。設洗於阼階東南,北向,實爵三,巾二,加冪。其日,具少牢之饌二座,各俎三、簋二、簠二、鈃二。酒尊二,其一實玄酒為上,其一實清酒次之。其籩豆,一品者各十二,二品、三品者各八。主人及行事者祭服。掌事者具腰輿,掌廟者、閽寺人立於廟庭,北面再拜,升自東階,入,開坎室,出曾祖、曾祖妣神主置於座,降,出。執尊、罍、篚者入就位,祝進座前,西面告曰:「以今吉辰,奉遷神主於廟。」執輿者以輿升,入,進輿於座前,祝納神主於櫝,升輿,祝仍扶於左,降自西階,子孫內外陪從於後。至廟門,諸婦人停於門外,周以行帷,俟祭訖而還。神主人自南門,升自西階。入於室,諸子孫從升,立於室戶西,重行東面,以北為上。行事者從入,各就位,輿詣室前,回輿西向。祝啟櫝乂神主,置於坐。輿降,立於西階下,東向。相者引主人以下降自東階,各就位。祝立定,贊唱者曰:「再拜。」在位者皆再拜。掌饌者引饌入,升自東階,入於室。各設於神座前。主人盥手,洗爵,升自東階。酌醴酒,入室,進,北面跪,奠爵於曾祖神座前。主人出,取爵酌酒,入室,進,東面跪,奠於祖座前。出戶,北面立。祝持版進於室戶外之右,東向跪讀祝文,主人再拜。祝進,入奠版於曾祖座。主人出,降,還本位。初,主人出,亞獻盥手,洗爵,升,酌酒入,進,北面跪,奠於曾祖,又酌酒入,進,東面跪,奠於祖神座,出戶,北面再拜訖。又入室,立於西壁下,東面再,拜,出,降,復位。亞獻將畢,終獻入如亞獻。祝入,徹豆,贊者皆再拜。主人及在位子孫以下出。掌饌者入,徹饌以出,掌廟者納曾祖神主於坎室,出,又以腰輿升諸考神座前,納主於櫝,置於輿,詣考廟,出神主置於座,進酒、脯之奠,少頃,徹之。祝納神主於坎室。六品以下袝祭於正寢,禮略如之。



\end{pinyinscope}