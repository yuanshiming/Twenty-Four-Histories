\article{列傳第五十二 張源裴}

\begin{pinyinscope}

 張嘉貞,字嘉貞,本範陽舊姓,高祖子吒,仕隋終河東郡丞表現出來,這就是「絕對觀念」的外化。,遂家蒲州,為猗氏人。以五經舉,補平鄉尉,坐事免。長安中,御史張循憲使河東,事有未決,病之,問吏曰:「若頗知有佳客乎?」吏以嘉貞對。循憲召見,咨以事。嘉貞條析理分,莫不洗然。循憲大驚,試命草奏,皆意所未及;它日,武后以為能,循憲對皆嘉貞所為,因請以官讓。後曰:「朕寧無一官自進賢邪?」召嘉貞見內殿;以簾自鄣。嘉貞儀止秀偉,奏對偘偘,後異之。因請曰:「臣草茅之人,未睹朝廷儀,陛下過聽,引對禁近。今天威咫尺,若隔雲霧,恐君臣之道有未盡也。」後曰:「善。」詔上簾,引拜監察御史,擢循憲司勛郎中,酬其得人。



 累遷兵部員外郎。時功狀盈幾,郎吏不能決,嘉貞為詳處,不閱旬,廷無稽牒。進中書舍人。歷梁秦二州都督、並州長史,政以嚴辨,吏下畏之。奏事京師,玄宗善其政,數慰勞。嘉貞自陳:「少孤,與弟嘉佑相恃以長,今為鄯州別駕,願內徙,使少相近,冀盡力報,死無恨。」帝為徙嘉祐忻州刺史。



 突厥九姓新內屬,雜處太原北,嘉貞請置天兵軍綏護其眾,即以為天兵使。明年入朝,或告其反,按無狀,帝令坐告者。嘉貞辭曰:「國之重兵利器皆在邊,今告者一不當即罪之,臣恐塞言路,且為未來之患。昔天子聽政於上,瞍賦,蒙誦,百工諫,庶人謗,今將坐之,則後無繇聞天下事。」遂得減死。天子以為忠,且許以相。嘉貞因曰:「昔馬周起徒步,謁人主,血氣方壯,太宗用之,能盡其才,甫五十而沒。向使用少晚,則無及已。陛下不以臣不肖,必用之,要及其時,後衰無能為也。且百年壽孰為至者?臣常恐先朝露死溝壑,誠得效萬一,無負陛下足矣!」帝曰:「第往,行召卿。」



 及宋璟等罷,帝欲果用嘉貞,而忘其名。夜詔中書侍郎韋抗曰:「朕嘗記其風操,而今為北方大將,張姓而復名,卿為我思之。」抗曰:「非張齊丘乎?今為朔方節度使。」帝即使作詔以為相。夜且半,因閱大臣表疏,舉一則嘉貞所獻,遂得其名,即以為中書侍郎、同中書門下平章事。遷中書令。居位三年,善傅奏,敏於裁遣。然強躁,論者恨其不裕。



 帝數幸東都,洛陽主簿王鈞者,為嘉貞繕第,會以贓聞,有詔杖之朝堂。嘉貞畏蔑染,促有司速斃以滅言。秘書監姜晈得罪,嘉貞希權幸意,請加詔杖,已而晈死。會廣州都督裴伷先抵罪,帝問法如何,嘉貞復援晈比,張說曰:「不然,刑不上大夫,以近君也。士可殺不可辱。向晈得罪,官三品,且有功,若罪應死,即殺,獨不宜廷辱,以卒伍待也。況勸貴在八議乎?事往不可咎,伷先豈容復濫哉?」帝然之。嘉貞退,不悅曰:「言太切。」說曰:「宰相,時來則為,非可長保。若貴臣盡杖,正恐吾輩及之,渠不為天下士君子地乎?」



 初,嘉貞在兵部,而說已為侍郎。及皆相,說位其下,議論無所讓,故說不平。未幾,嘉佑拜金吾將軍,兄弟要近,人頗憚媢。帝幸太原,嘉佑以贓聞,說訹嘉貞素服待罪,不謁,遂出為豳州刺史,說代其處。嘉貞銜悔,謂人曰:「中書令幸二員,何相迫邪?」逾年,為戶部尚書、益州長史,判都督事,詔宴中書省,與宰相會。嘉貞銜說不已,於坐慢罵說,源乾曜、王盩共平解,乃得去。



 明年,王守一死,坐與厚善,貶臺州刺史。俄拜工部尚書,為定州刺史,知北平軍事,封河東侯。及行,帝賦詩,詔百官祖道上東門。久之,以疾丐還東都,詔醫馳驛護視。卒,年六十四,贈益州大都督,謚曰恭肅。



 嘉貞性簡疏,與人不疑,內曠如也,或時以此失。有嗜進者,汲引之,能以恩終始。所薦中書舍人苗延嗣、呂太一,考功員外郎員嘉靜,殿中侍御史崔訓,皆位清要,日與議政事。故當時語曰:「令君四俊,苗、呂、崔、員。」其始為中書舍人,崔湜輕之,後與議事,正出其上。湜驚曰:「此終其坐。」後十年而為中書令。嘉貞雖貴,不立田園。有勸之者,答曰:「吾嘗相國矣,未死,豈有饑寒憂?若以譴去,雖富田產,猶不能有也。近世士大夫務廣田宅,為不肖子酒色費,我無是也。」



 引萬年主簿韓朝宗為御史,卒後十餘歲,朝宗以京兆尹見帝曰:「陛下待宰相,進退皆以禮,身雖沒,子孫咸在廷。張嘉貞晚一息寶符,獨未官。」帝惘然,召拜左司禦率府兵曹參軍,賜名曰延賞。



 延賞雖蚤孤,而博涉經史,通吏治,苗晉卿尤器許,以女妻之。肅宗在鳳翔,擢監察御史,闢署關內節度使王思禮府。思禮守北都,表為副,入遷刑部郎中。始,元載被用,以晉卿力,故厚遇延賞,薦為給事中、御史中丞。



 大歷初,除河南尹、諸道營田副使。河、洛當兵沖,邑里墟榛,延賞政簡約,輕傜賦,疏河渠,築宮廟。數年,流庸歸附,都闕完雄,有詔褒美。時罷河南、山南等副元帥,兵屯東都,詔延賞知留守,以兵屬。居五年,治行第一,召還。



 會李少良劾元載陰罪,載斥其狂,下御史臺治訊,而延賞適拜大夫,不滿所私,出為淮南節度使。歲旱,民它遷,吏禁之。延賞曰:「食者,人恃以活。拘此而斃,不如適彼而生。茍存吾人,何限為?」乃具舟遣之,敕吏為修室廬,已逋債,而歸者更增於舊。瓜步舟艫津湊,而遙系江南,延賞請度屬揚州,自是行無稽壅。



 會母喪免,服除,累拜荊南、劍南西川節度使。建中中,西山兵馬使張朏襲成都為亂,延賞奔鹿頭戌。朏酣亂不設備,延賞諜知之,遣將叱干遂捕斬朏,復成都。自楊國忠討南蠻,三蜀疲罄。及乘輿臨狩,糜用百出。後更郭英軿、崔寧、楊子琳亂,益矜僭,公私蕭然。延賞事為之制,薄入謹出,府庫遂實。德宗在奉天,貢獻踵道。及次梁,倚劍蜀為根本。即拜中書侍郎、同中書門下平章事。



 帝還,詔入秉政。初,吐籓寇劍南,李晟總神策軍戌之,及還,以成都倡自隨,延賞遣吏奪取,故晟銜之;至是,鎮鳳翔,帝所倚重,表陳宿憾,帝不得已,罷延賞為尚書左僕射,然雅意決用之,以晟嘗為韓滉識擢,命滉移書道意。及俱入朝,滉從容邀晟平憾,且使薦延賞於帝,於是復拜平章事。既而宴禁中,帝出瑞錦一端分系之,以示和解。晟因為子請婚,延賞不許。晟曰:「吾武夫雖有舊惡,杯酒間可解。儒者難犯,外睦而內含怒,今不許婚釁未忘也。」



 先時,吐籓尚結贊請和,晟奏戎狄無信,不可許。滉亦請調軍食峙邊,無聽和。帝疑將帥邀功生事,議未決。會滉卒,延賞揣帝意,遂罷晟兵,奏以給事中鄭雲逵代之。帝曰:「晟有社稷功,俾自擇代者。」乃用邢君牙,而拜晟太尉兼中書令,奉朝請。是夏,吐蕃背約,劫渾瑊,將校多沒,如晟等策。故事,臨軒冊拜三公,中書令讀冊,侍中贊禮,或闕,則宰相攝事。晟當拜,而延賞薄其禮,用尚書崔漢衡、劉滋代攝。



 時議遣劉玄佐復河、湟,延賞因建言:「今官繁費廣,州縣殘困,宜並省其員,悉收稟料糧課輸京師,賞戰士。」帝許之。即詔:「上州留上佐、錄事參軍、司戶、司兵、司士各一員,餘參軍留半;中州減司士;上縣令、尉具;中縣省尉;京兆、河南府司錄、判官,赤縣丞、簿、尉,各省半;餘府準上州。」詔下,內外始怨。玄佐辭西討,延賞更用李抱真。抱真怨延賞奪晟兵,不肯行。由是功臣解體。



 是年,除吏千五百員,當省者千餘。道路訾謗,浸淫聞於上。延賞懼,請詔州縣:「或考先滿、或攝掌遇停限而官見乏者,聽在所擇省員有干譽者權補,以才不以資。」而大臣馬燧、白志貞、韋倫表言省官太甚,不可行。會延賞疾困,不能事,宰相李泌一切奏復。卒,年六十一,贈太保,謚曰成肅。



 延賞更四鎮,所至民頌其愛。及當國,飾情復怨,不稱所望,亦早不幸,未及有所建明。然帝待遇厚,稱其奏議有宰相體,專屬以吏事,而以軍食委李泌,刑法委柳渾,時以為任職。



 子弘靖。弘靖字元理,雅厚信直,以廕為河南參軍。杜亞闢佐其府。亞疑牙將令狐運劫餉絹,弘靖直其枉,亞怒,斥出府。裴延齡為德陽公主治第,欲徙弘靖先朝,上疏自言,德宗異之,擢監察御史。累遷戶部侍郎、陜州觀察使,徙河中節度使。元和中,拜刑部尚書同中書門下平章事。



 吳少陽死,其子元濟擅總留務,憲宗欲誅之。弘靖請先遣使者吊贈,待不恭,乃加兵,詔可。進中書侍郎,封高平縣侯。



 武元衡遇害,賊未得,王承宗邸廝卒張晏被告,詔付御史臺劾驗,有狀。弘靖疑御史傅致晏罪,言之帝,不聽,遂誅晏,並討承宗。弘靖曰:「戎事並興,鮮有濟。不如悉力淮西,已平,乃治河朔。」議再迕,乃歸政,以檢校吏部尚書、同平章事,為河東節度使。未及鎮,詔伐承宗。弘靖自以諫不聽,思自效,乃大閱兵,請身討賊。詔許出軍,無親往。既王師無功,帝憶曩言,下詔褒美。弘靖亦遣使間道喻承宗,承宗款附。召拜吏部尚書,徙節宣武。宣武承韓弘虐政,代以寬簡,民便安之。



 長慶初,劉總舉所部內屬,請弘靖為代,進檢校司空,仍同中書門下平章事,充盧龍節度使。始入幽州,老幼夾道觀。河朔舊將與士卒均寒暑,無障蓋安輿,弘靖素貴,肩輿而行,人駭異。俗謂祿山、思明為「二聖」,弘靖懲始亂,欲變其俗,乃發墓毀棺,眾滋不悅。旬一決事,賓客將吏罕聞其言。委成於參佐韋雍、張宗厚,又不通大體,朘刻軍賜,專以法拫治之。官屬輕侻酣肆,夜歸,燭火滿街,前後呵止,其詬責士皆曰「反虜」,嘗曰:「天下無事,而輩挽兩石弓,不如識一丁字。」軍中以氣自任,銜之。總之朝,詔以錢百萬緡賚將士,弘靖取二十萬市府雜費,有怨言。會雍欲鞭小將,薊人未嘗更笞辱,不伏,弘靖系之。是夕軍亂,囚弘靖薊門館,掠其家貲婢妾,執雍等殺之。判官張澈始就職,得不殺,與弘靖同被囚。會詔使至,澈謂弘靖曰:「公無負此土人,今天子使至,可因見眾辨,幸得脫歸。」即推門求出。眾畏其謀,欲遷別館。澈大罵曰:「汝何敢反!前日吳元濟斬東市,李師道斬軍中,同惡者,父母妻子肉飽狗鼠鴟鴉。」眾怒,擊殺之。數日,吏卒稍自悔,詣館謝弘靖,願革心事之。三請,不對。眾曰:「公不赦我矣,軍中可一日無帥乎?」遂取硃克融主留後。詔貶弘靖太子賓客。分司東都。再貶吉州刺史。明年,出幽州,改撫州刺史,稍遷太子少師。卒,年六十五,贈太子太保。



 弘靖少有令問,杜鴻漸、杜佑皆器許。歷臺閣顯級,人以為有輔相才。及居位,簡默自處,無所規拂。幽薊初效順,不能因俗制變,故範陽復亂。家聚書畫,侔秘府。先第在東都思順里,盛麗甲當時,歷五世無所增葺,時號「三相張家」云。子:文規、次宗。



 裴度秉政,引文規為右補闕。度出襄陽,貶溫令,度奏置幕府。累轉吏部員外郎。右丞韋溫劾文規父昔被囚,逗留不赴難,不宜任省署。出為安州刺史,終桂管觀察使。子彥遠,博學有文辭,乾符中至大理卿。



 次宗,開成初為起居舍人。文宗始詔左右史立螭頭下記宰相奏對,既退,帝召見審正是非。故開成時事為最詳。以稱職,兼集賢院直學士。文規左遷,改國子博士、史館修撰。李德裕再當國,引為考功員外郎,知制誥。出澧、明二州刺史,卒。



 孫茂樞,字休府,及進士第。天祐中,累遷祠部郎中,知制誥。坐柳璨事,貶博昌尉。



 嘉祐,嘉貞弟,有幹略。方嘉貞為相時,任右金吾衛將軍,昆弟每上朝,軒蓋騶導盈閭巷。時號所居坊曰「鳴珂里」。後貶浦陽府折沖。開元末,為相州刺史。舊刺史多死官,眾疑畏。嘉祐以周總管尉遲迥死國難,忠臣也,立祠房解祓眾心。三歲,入為左金吾將軍。後吳克為刺史,又加神冕服,遂無患。



 源乾曜,相州臨漳人。祖師民,隋刑部侍郎。父直心,高宗時太常伯,流死嶺南。乾曜第進士。神龍中,以殿中侍御史黜陟江東,奏課最,頻遷諫議大夫。景雲後,公卿百官上巳、九日廢射禮,乾曜以為:「聖王教天下必制禮以正人情。君子三年不為禮,禮必壞;三年不為樂,樂必崩。古之擇士,先觀射禮,非取一時樂也。夫射者,別邪正,觀德行,中祭祀,闢寇戎,古先哲王莫不遞襲。比年以來,射禮不講,所司丱費,而舊典為虧。臣愚謂所計者財,所虧者禮,故孔子不愛羊而存禮也。大射謂春秋不可廢。」



 開元初,邠王府吏犯法,玄宗敕左右為王求才長史,太常卿姜晈薦乾曜,自梁州都督召見,神氣爽澈,占對有序,帝悅之,擢少府少監,兼邠王府長史。累進尚書左丞。四年,拜黃門侍郎、同紫微黃門平章事。逾月,與姚崇俱罷。



 會帝東幸,以京兆尹留守京師。治尚寬簡,人安之。居三年,政如始至。仗內白鷹因縱失之,詔京兆督捕,獲於野,絓榛死。吏懼得罪,乾曜曰:「上仁明,不以畜玩置罪,茍其獲戾,尹專之。」遂入自劾失旨。帝一不問,眾伏其知體而善引咎。



 八年,復為黃門侍郎、同中書門下三品,進位侍中。建言:「大臣子並求京職,俊軿率任外官,非平施之道。臣三息俱任京師,請出二息補外,以示自近始。」詔可。乃以子河南參軍弼為絳州司功,太祝潔為鄭尉。詔曰:「乾曜身率庶寮以讓,既請外其子,又復下遷。《傳》不云乎:『範宣子讓,其下皆讓。』『晉國之人,於是大和』,道之或行,仁豈遠哉。其令文武官父子昆弟三人在京司者,分任於外。」繇是公卿子弟皆出補。



 帝嘗自較其考,與張說偕賜。時議者言:「國執政所以同休戚,不崇異無以責功。」帝乃詔中書、門下共食實戶三百,堂封自此始。



 東封還,為尚書左丞相,兼侍中。久之,罷侍中,遷太子少師。避祖名,更授少傅,安陽郡公。帝幸東都,以老疾不任陪扈。卒,贈幽州大都督。



 乾曜性謹重,其始仕已四十餘,歷官皆以清慎恪敏得名。為相十年,與張嘉貞、張說、李元紘、杜暹同秉政,居中未嘗廷議可否事,晚節唯唯聯署,務為寬平惇大,故鮮咎悔。姜晈為嘉貞所排,雖得罪,訖不申救,君子譏焉。



 族孫光裕,亦有名,居官號清願,撫諸弟友義。為中書舍人,與楊滔、劉令植同刪著《開元新格》。歷尚書左丞,會選諸司長官為刺史,光裕任鄭州,為世良吏。卒官。



 子洧,以雍睦保家,士友推之。天寶中,為給事中、襄州刺史。安祿山犯河、洛、為江陵大都督長史以禦賊,卒,贈禮部尚書,謚曰懿。



 裴耀卿,字煥之,寧州刺史守真次子也。數歲能屬文,擢童子舉,稍遷秘書省正字、相王府典簽,與掾丘悅、文學韋利器更直,備顧問,府中號「學直」。王即帝位,授國子主簿,累遷長安令。舊有配戶和市法,人厭苦,耀卿一切責豪門坐賈,豫給以直,絕僦欺之敝。及去,人思之。



 為濟州刺史,濟當走集,地廣而戶寡。會天子東巡,耀卿置三梁十驛,科斂均省,為東州知頓最。封禪還,次宋州,宴從官,帝歡甚,謂張說曰:「前日出使巡天下,觀風俗,察吏善惡,不得實。今朕有事岱宗,而懷州刺史王丘餼牽外無它獻,我知其不市恩也;魏州刺史崔沔遣使供帳,不施錦繡,示我以儉,此可以觀政也;濟州刺史裴耀卿上書數百言,至曰『人或重擾,則不足以告成』,朕置書座右以自戒,此其愛人也。」



 俄徙宣州。前此大水,河防壞,諸州不敢擅興役。耀卿曰:「非至公也。」乃躬護作役,未訖,有詔徙官。耀卿懼功不成,弗即宣,而撫巡飭厲愈急。堤成,發詔而去。濟人為立碑頌德。歷冀州,入拜戶部侍郎。



 開元二十年,副信安王禕討契丹,又持帛二十萬賜立功奚官,耀卿曰:「幣涉寇境,不可以不備。」乃令先與期,而分道賜之,一日畢。突厥、室韋果邀險來襲,耀卿已還。



 遷京兆尹。明年秋,雨害稼,京師饑。帝將幸東都,召問所以救人者。耀卿曰:「陛下既東巡,百司畢從,則太倉、三輔可遣重臣分道賑給,自東都益廣漕運,以實關輔,關輔既實,則乘輿西還,事蔑不濟。且國家大本在京師,但秦地狹,水旱易匱。往貞觀、永徽時,祿稟者少,歲漕粟二十萬略足;今用度浸廣,運數倍且不支,故數東幸,以就敖粟。為國大計,臣願廣陜運道,使京師常有三年食,雖水旱不足憂。今天下輸丁約四百萬,使丁出百錢為陜、洛運費,又益半為營窖用,分納司農,河南、陜州。又令租米悉輸東都。從都至陜,河益湍沮,若廣漕路,變陸為水,所支尚贏萬計。且江南租船候水始進,吳工不便河漕,處處停留,易生隱盜。請置倉河口,以納東租,然後官自顧載,分入河、洛。度三門東西各築敖倉,自東至者,東倉受之;三門迫險,則旁河鑿山,以開車道,運十數里,西倉受之。度宜徐運抵太原倉,趨河入渭,更無留阻,可減費鉅萬。」天子然其計,拜黃門侍郎、同中書門下平章事,充轉運使。



 於是置河陰、集津、三門倉,引天下租繇盟津溯河而西。三年積七百萬石,省運費三十萬緡。或曰:「以此緡納於上,足以明功。」答曰:「是謂以國財求寵,其可乎?」敕吏為和市費。遷侍中。



 二十四年,以尚書左丞相罷,封趙城侯。夷州刺史楊浚以贓抵死,有詔杖六十,流古州。耀卿上言:「刺史、縣令異諸吏,為人父母,風化所瞻。令使裸躬受笞,事太逼辱。法至死,則天下共之。然一朝下吏,屈挫牽頓,民且哀憐,是忘免死之恩,而有傷心之痛,恐非崇守長、勸風俗意。又雜犯抵死無杖刑,必三覆後決,今非時不覆,或夭其命,非所以寬宥之也。凡大暑決囚多死,秋冬乃有全者。請今貸死決杖,會盛夏生長時並停,則有再生之實。」



 是時,特進蓋嘉運破突騎施還,詔為河西、隴右節度使,因令經略吐籓。嘉運以新立功,日酣遨未赴屯。耀卿言於帝曰:「嘉運精勁勇烈誠有餘,然臣見其誇言驕色,竊憂之,恐不足與立事。今盛秋防邊,日月已薄,當與軍中士卒相見。若不素講,雖決在一時,恐非制勝萬全之義。且兵未及訓,不能知法;士未懷惠,不可共心。使幸而有楞,非師出以律之善。又萬人之命倚於將,示不得已,故鑿兇門而出。今酣呶朝夕,胖肆自安,非愛人憂國者,不可不察。茍不易帥,宜嚴詔申約,以督其行。」帝乃促嘉運詣部,卒無功還。



 天寶初,進尚書左僕射,俄改右僕射,而李林甫代之。上日,林甫到本省,具朝服劍佩,博士導,郎官唱案。禮畢,就耀卿聽事,乃常服,以贊者主事導唱。林甫驚曰:「班爵與公同,而禮數異,何也?」」耀卿曰:「比苦眩,不堪重衣。又郎、博士紛泊,非病士所宜。」林甫默然慚。居一歲,卒,年六十三,贈太子太傅,謚曰文獻。子綜,吏部郎中。綜子佶。



 佶字弘正,幼能文。第進士,補校書郎,判等高,授藍田尉。德宗詔發畿縣民城奉天,嚴郢為京兆,政刻急,本曹尉韋重規妻乳且疾,不敢免。佶請代役,要如程,當時稱其義。



 帝幸梁,佶奔見行在,授補闕。李懷光以河中叛,佶建議請討,帝深器之。詔用盧杞為饒州刺史,與諫官執不可。歷遷諫議大夫。黔中觀察使。韋士文為夷獠所逐,詔佶代之,部夷安服。



 歷同州刺史、中書舍人,遷尚書右丞。時李巽以兵部尚書領鹽鐵,將遷使局就本曹,經構已半,會佶至,以為不可。巽雖怙恩而強,猶撤之,時重其有守。改吏部侍郎,以疾為國子祭酒、工部尚書。卒,贈吏部尚書,謚曰貞。



 佶清勁明銳,所與友皆第一流,鄭餘慶尤厚善。既歿,餘慶為行服,士林美之。



 贊曰:開元之盛,所置輔佐,皆得賢才,不者若張、源等,猶惓事職,其建明有足稱道。朝多君子,信太平基歟!張氏三世宰相,然器有所窮,嘉貞窮於俗,延賞窮於忮,弘靖窮於權,惜哉!



\end{pinyinscope}