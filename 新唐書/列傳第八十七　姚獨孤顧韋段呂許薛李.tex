\article{列傳第八十七 姚獨孤顧韋段呂許薛李}

\begin{pinyinscope}

 姚南仲,華州下邽人。乾元初,擢制科,授太子校書。遷累右補闕。大歷十年,獨孤皇后崩克思寫於1875年4月至5月初。1891年首次發表。編入《馬,代宗悼痛,詔近城為陵,以朝夕臨望。南仲上疏曰:「臣聞人臣宅於家,帝王宅於國。長安乃祖宗所宅,其可興鑿建陵其側乎?夫葬者,藏也,欲人之不得見也。今西近宮闕,南迫大道。使近而可視,歿而復生,雖宮以待之可也。如令骨肉歸土,魂無不之,雖欲自近,了復何益?且王者必據高明,燭幽隱,先皇所以因龍首而建望春也。今起陵目前,心一感傷,累日不能平。且匹夫向隅,滿堂不樂,況萬乘乎,天下謂何?陛下謚后以貞懿,而終以褻近,臣竊惑焉。今國人皆曰后陵在邇,陛下將日省而時望焉,斯有損聖德,無益先後,欲寵反辱,惟陛下孰計。」疏奏,帝嘉納,進五品階以酬讜言。



 坐善宰相常袞,出為海鹽令。浙西觀察使韓滉表為推官,擢殿中侍御史內供奉。召還,四遷為御史中丞,改給事中、陜虢觀察使。拜義成節度使。監軍薛盈珍恃權橈政,不能逞,因毀南仲於朝,德宗惑之。俄遣小使程務盈誣表以罪。會南仲裨將曹文洽入奏,知其語,則晨夜追至長樂驛,及之,與同舍,夜殺務盈,投其誣於廁。為二書,一抵南仲,一治南仲冤,且自言殺務盈狀,乃自殺。驛吏以聞,帝駭異。南仲不自安,固請入朝。帝勞曰:「盈珍橈卿政邪?」曰:「不橈臣政,臣隳陛下法耳。如盈珍輩,所在有之,雖使羊、杜復生,撫百姓,御三軍,必不能成愷悌之化而正師律也。」帝默然。乃授尚書右僕射。貞元十九年卒,年七十五,贈太子太保,謚曰貞。



 初,崔位、馬少微者,俱在南仲幕府。盈珍之譖也,出位為遂州別駕。東川觀察使王叔邕希旨奏位,殺之。復出少微補外,使宦官護送,度江,投之水雲。



 獨孤及,字至之,河南洛陽人。為兒時,讀《孝經》,父試之曰:「兒志何語?」對曰:「立身行道,揚名於後世。」宗黨奇之。天寶末,以道舉高第補華陰尉,闢江淮都統李峘府,掌書記。



 代宗以左拾遺召,既至,上疏陳政曰:



 陛下屢發德音,使左右侍臣得直言極諫。壬辰詔書,召裴冕等十有三人集賢殿待制,以備詢問。此五帝盛德也。然頃者陛下雖容其直,而不錄其言,所上封皆寢不報。有容下之名,無聽諫之實,遂使諫者稍稍自鉗口,飽食相招為祿仕,此忠鯁之人所以竊嘆,而臣亦恥之。十室之邑,必有忠信,況朝廷之大,卿大夫之眾,陛下選授之精歟!假令不能如文王之多士,其中豈不有溫故知新,可懋陳政要而億則屢中者?陛下議政之際,曾不採其一說,堯之疇咨,禹之昌言,豈若是邪?昔堯設謗木於五達之衢,孔子曰:「以能問於不能,以多問於寡。」然則多聞闕疑,不恥下問,聖人之心也。願陛下以堯、孔心為心,日降清問,其不可者罷之,可者議於朝,與執事者共之。使知之必言,言之必行,行之必公,則君臣無私論,朝廷無私政,陛下以此辨可否於獻替,而建太平之階可也。



 師興不息十年矣,人之生產,空於杼軸。擁兵者第館亙街陌,奴婢厭酒肉,而貧人羸餓就役,剝膚及髓。長安城中,白晝椎剽,吏不敢詰。官亂職廢,將墮卒暴,百揆隳剌,如沸粥紛麻。民不敢訴於有司,有司不敢聞陛下,茹毒飲痛,窮而無告。今其心顒顒,獨恃於麥,麥不登,則易子咬骨矣。陛下不以此時厲精更始,思所以救之之術,忍令宗廟有累卵之危,萬姓悼心失圖,臣實懼焉。去年十一月丁巳夜,星隕如雨,昨清明降霜,三月苦熱,錯繆顛倒,沴莫大焉。此下陵上替,怨讟之氣取之也。天意丁寧譴戒,以警陛下,宜反躬罪己,旁求賢良者而師友之,黜貪佞不肖者,下哀痛之詔,去天下疾苦,廢無用之官,罷不急之費,禁止暴兵,節用愛人,兢兢乾乾,以徼福於上下,必能使天感神應,反妖災為和氣矣。



 又言:



 減江淮、山南諸道兵以贍國用,陛下初不以臣言為愚,然許即施行,及今未有沛然之詔,臣竊遲之。今天下唯朔方、隴西有吐蕃、僕固之虞,邠、涇、鳳翔兵足以當之矣。自此而往,東洎海,南至番禺,西盡巴蜀,無鼠竊之盜,而兵不為解。傾天下之貨,竭天下之谷,以給不用之軍,為無端之費,臣不知其故。假令居安思危,以備不虞,自可厄害之地,俾置屯御,悉休其餘,以糧儲扉屨之資充疲人貢賦,歲可以減國租半。陛下豈遲疑於改作,逡巡於舊貫,使大議有所壅,而率土之患日甚一日?是益其弊而厚其疾也。夫療癰者,必決之使潰。今兵之為患,猶癰也,不以漸戢之,其害滋大,大而圖之,必力倍而功寡,豈《易》「不俟終日」之義邪?



 俄改太常博士。或言景皇帝不宜為太祖,及據禮條上。謚呂諲、盧弈、郭知運等無浮美,無隱惡,得褒貶之正。遷禮部員外郎,歷濠、舒二州刺史。歲饑旱,鄰郡庸亡什四以上,舒人獨安。以治課加檢校司封郎中,賜金紫。徙常州,甘露降其廷。卒,年五十三,謚曰憲。



 及喜鑒拔後進,如梁肅、高參、崔元翰、陳京、唐次、齊抗皆師事之。性孝友。其為文彰明善惡,長於論議。晚嗜琴,有眼疾,不肯治,欲聽之專也。子朗、鬱。



 朗,字用晦,由處士闢署江西、宣歙、浙東三府。元和中,擢右拾遺。建言:「宜用觀察使領本道鹽鐵,罷場監管榷吏,除百姓之患。」不聽。盜殺武元衡,朗請貶京兆尹,誅捕賊吏。因勸罷兵,忤憲宗意,貶興元戶曹參軍。久乃拜殿中侍御史,兼史館修撰。坐與李景儉飲,景儉使酒慢宰相,出為韶州刺史。召還,再遷諫議大夫。



 敬宗初,宦官毆鄠令崔發雞乾下,朗請誅首惡以正常法。王播賂權近,還判鹽鐵,朗連疏論執。遷御史中丞。故事,選御史皆中丞自請。是時,崔晁、鄭居中繇宰相力得監察御史,朗拒不納,晁、居中卒改他官。侍御史李道樞醉謁朗,朗劾不虔,下除司議郎。會殿中王源植貶官,朗直其枉,書五上不報,即自劾執法不稱,願罷去。帝遣中人尉諭不許。文宗初,遷工部侍郎,出為福建觀察使,創發背卒,贈右散騎常侍。



 鬱,字古風,始生而孤,與朗育於伯父汜。擢進士第,最為權德輿所稱,以女妻之。元和初,舉制科高等,拜右拾遺,俄兼史館修撰,進右補闕。吐突承璀討王承宗,鬱執不可,挺議鯁固,號稱職。擢翰林學士。德輿輔政,以嫌去內職,拜考功員外郎,仍兼修撰。憲宗嘆德輿乃有佳婿,詔宰相高選世族,故杜悰尚岐陽公主,然帝猶謂不如德輿之得鬱也。俄知制誥。德輿去位,還為學士。九年,以疾辭禁近,徙秘書少監,屏居鄠,卒,年四十,贈絳州刺史。鬱有雅名,帝遇之厚,議者亦謂當宰相,共以早世惜之。



 子庠,字賢府,喪父始十歲,有至性,聞呼父官及吊客來,輒號慟幾絕。後舉進士,仕至尚書丞。



 顧少連,字夷仲,蘇州吳人。舉進士,尤為禮部侍郎薛邕所器,擢上第,以拔萃補登封主簿。邑有虎孽,民患之,少連命塞陷阱,獨移文嶽神,虎不為害。御史大夫於頎薦為監察御史。德宗幸奉天,徒步詣謁,授水部員外郎、翰林學士。再遷中書舍人,閱十年,以謹密稱。嘗請徙先兆於洛,帝重遠去,詔遣其子往,且命中人護蕆葬役。



 歷吏部侍郎。裴延齡方橫,無敢忤者。嘗與少連會田鎬第,酒酣,少連挺笏曰:「段秀實笏擊賊臣,今吾笏將擊奸臣!」奮且前,元友直在坐,歡解之。改京兆尹。政尚寬簡,不為灼灼名。先是,京畿租賦薄厚不能一,少連以法均之。遷吏部尚書,封本縣男,徙兵部。為東都留守,表禁苑及汝閑田募耕以便民,閱武力,利鎧仗,號良吏。卒,年六十二,贈尚書右僕射,謚曰敬。



 始,少連攜少子師閔奔行在,有詔同止翰林院,車駕還,授同州參軍。



 韋夏卿,字雲客,京兆萬年人。少邃於學,善文辭。大歷中,與弟正卿同舉賢良方正,皆策高等。授高陵主簿,累遷刑部員外郎。時仍歲旱蝗,詔以郎官宰畿甸,授奉天令,課第一,改長安令。轉吏部員外郎、郎中,擢給事中,出為常、蘇二州刺史。徐州節度使張建封疾甚,詔夏卿為徐泗行軍司馬,且代之。未至,而建封卒,徐軍立其子愔為留後,召夏卿為吏部侍郎。



 時從弟執誼在翰林,嘗受人金,有所干請,密以金內夏卿懷中,夏卿毀懷不受,曰:「吾與爾賴先人遺德,致位及此,顧當是哉?」執誼大慚。轉京兆尹、太子賓客,檢校工部尚書,為東都留守,辭疾,改太子少保。卒,年六十四,贈尚書左僕射,謚曰獻。



 夏卿性通簡,好古有遠韻,談說多聞。晚歲將罷歸,署其居曰「大隱洞」。與齊映、穆贊、贊弟員友善,雖同游,終年不見其喜慍。撫孤侄恩逾己子。為政務通理,不甚作條教。所闢士如路隋、張賈、李景儉等,至宰相達官,故世稱知人。



 正卿子瓘,字茂弘,及進士第,仕累中書舍人。與李德裕善,德裕任宰相,罕接士,唯瓘往請無間也。李宗閔惡之,德裕罷,貶為明州長史。會昌末,累遷楚州刺史,終桂管觀察使。



 段平仲,字秉庸,本武威人,隋民部尚書達六世孫。擢進士第。杜佑、李復之節度淮南,連表掌書記。擢監察御史。磊落有氣節,嗜酒敢言。是時,德宗春秋高,躬自聽斷,天下事有所壅隔,群臣畏帝苛察,無敢言。平仲常曰:「上聰明神武,但臣下畏怯,自為循默爾。使我一日得召見,宜大有開納。」會京師旱,詔擇御史、郎官開倉振恤。平仲與考功員外郎陳歸被選,同得對,粗陳振恤事,帝察其意有所畜,以歸在側未言。事訖,平仲方獨進,帝乃並留歸,正色問之,雜以它語,平仲錯牾不得言,乃謬稱名,帝怒,叱去之。蒼黃向幄後,歸趨降招之,乃得去。由是坐廢七年,然名由此顯。



 元和初,為諫議大夫,憲宗使吐突承璀討鎮州,亟疏爭,不可。及還,無功,又請斬之。再遷尚書右丞。朝廷有得失,未嘗不論奏,世推其敢直云。終太子左庶子。



 贊曰:君有常尊,臣有定卑,自然之勢也。然臣不自通於上,君不降而逮諸下,則治不得成而功不彰。返是而天下之務粲焉幾矣。德宗察察,欲折伏臣下,自為聰明,而治愈疏。段平仲一忤上,蒼惶失對,而猶以取名,何哉?下知所職,而上喪其所以為上也。故聖王屈己從諫,君臣兩得其美,知道之本歟!



 呂元膺,字景夫,鄆州東平人。姿儀瑰秀,有器識。始游京師,謁故宰相齊映,映嘆曰:「吾不及識婁、郝,殆斯人類乎!」策賢良高第,調安邑尉,闢長春宮判官。李懷光亂河中,輒解去。論惟明節度渭北,表佐其府。惟明卒,王棲曜代之,德宗敕棲曜留元膺自佐,入拜殿中侍御史。歷右司員外郎。出為蘄州刺史,嘗錄囚,囚或白:「父母在,明日歲旦不得省,為恨。」因泣,元膺惻然,悉釋械歸之,而戒還期。吏白「不可」,答曰:「吾以信待人,人豈我違?」如期而至。自是群盜感愧,悉避境去。



 元和中,累擢給事中。俄為同州刺史。既謝,帝逮問政事,所對詳詣。明日,謂宰相曰:「元膺直氣讜言,宜留左右,奈何出之?」李籓、裴垍謝,因言:「陛下及此,乃宗社無疆之休。臣等昧死請留元膺給事左右。」未幾,兼皇太子侍讀,進御史中丞。拜鄂岳觀察使。嘗夜登城,守者不許。左右曰:「中丞也。」對曰:「夜不可辨。」乃還。明日,擢守者為大將。入拜尚書左丞。度支使潘孟陽、太府卿王遂交相惡,乃除孟陽散騎常侍,遂鄧州刺史,詔辭無所輕重。元膺上其詔,請明枉直,以顯褒懲。



 江西裴堪按虔州刺史李將順受賕,不覆訊而貶。元膺曰:「觀察使奏部刺史,不加覆,雖當誅,猶不可為天下法。」請遣御史按問,宰相不能奪。



 選拜東都留守。故事,留守賜旗甲,至元膺不給。或上言:「用兵討淮西,東都近賊,損其儀,沮威望,請比華、汝、壽三州。」帝不聽,並三州罷之。留守不賜旗甲,自此始。都有李師道留邸,邸兵與山棚謀竊發,事覺,元膺禽破之。始,盜發,都人震恐,守兵弱不足恃,元膺坐城門指縱部分,意氣閑舒,人賴以安。東畿西南通鄧、虢,川谷曠深,多麋鹿,人業射獵而不事農,遷徙無常,皆趫悍善鬥,號曰「山棚」。權德輿居守,將羈縻之,未克。至是,元膺募為山河子弟,使衛宮城,詔可。



 改河中節度使。時方鎮多姑息,獨元膺秉正自將,監軍及中人往來者,無不嚴憚。入拜吏部侍郎。正色立朝,有臺宰望,處事裁宜,人服其有禮。以疾改太子賓客。居官始終無訾缺。卒,年七十二,贈吏部尚書。



 許孟容,字公範,京兆長安人。擢進士異等,又第明經,調校書郎。闢武寧張建封府。李納以兵拒境,建封遣使諭止,前後三輩往,皆不聽。乃使孟容見納,敷引逆順,納即悔謝,為罷兵。表為濠州刺史。



 德宗知其能,召拜禮部員外郎。公主子求補崇文生者,孟容固謂不可,主訴之,帝問狀,以著令對。帝嘉其守,擢郎中。累遷給事中。京兆上言「好畤風雹害稼」,帝遣宦人覆視,不實,奪尹以下俸。孟容曰:「府縣上事不實,罪應罰。然陛下遣宦者覆視,紊綱紀。宜更擇御史一人參驗,乃可。」不聽。



 浙東觀察使裴肅諉判官齊總暴斂以厚獻,厭天子所欲。會肅卒,帝擢總自大理評事兼監察御史為衢州刺史。衢,大州也。孟容還制曰:「方用兵處,有不待次而擢者。今衢不他虞,總無功越進超授,群議謂何?且總本判官,今詔書乃言『權知留後,攝都團練副使』,初無制授,尤不見其可。假令總有可錄,宜暴課最,解中外之惑。」會補闕王武陵等亦執爭,於是詔中停。帝召謂曰:「使百執事皆如卿,朕何憂邪?」自袁高爭盧杞後,凡十八年,門下無議可否者。至孟容數論駁,四方知天子開納多士,浩然想見其風。



 貞元十九年夏,大旱,上疏言:「陛下齋居損膳,具牲玉,走群望,而天意未答,豈豐歉有定,陰陽適然乎?竊惟天人交感之際,系教令順民與否。今戶部錢非度支歲計,本備緩急,若取一百萬緡代京兆一歲賦,則京圻無流亡,振災為福。又應省察流移征防當還未還,役作禁錮當釋未釋;負逋饋送,當免免之;沈滯鬱抑,當伸伸之;以順人奉天。若是而神弗祐、歲弗稔,未之聞也。」先是,為裴延齡、李齊運流斥者,雖十年弗內移,故孟容因旱及之。帝始不悅,改太常少卿。



 元和初,再遷尚書右丞、京兆尹。神策軍自興元後,日驕恣,府縣不能制。軍吏李昱貸富人錢八百萬,三歲不肯歸。孟容遣吏捕詰,與之期使償,曰:「不如期,且死!」一軍盡驚,訴於朝。憲宗詔以昱付軍治之,再遣使,皆不聽,奏曰:「不奉詔,臣當誅。然臣職司輦轂,當為陛下抑豪強。錢未盡輸,昱不可得。」帝嘉其守正,許之。京師豪右大震。



 累遷吏部侍郎。盜殺武元衡,孟容白宰相曰:「漢有一汲黯,奸臣寢謀。今朝廷無有過失,而狂賊敢爾,尚謂國有人乎?願白天子,起裴中丞輔政,使主兵柄,索賊黨,罪人得矣。」後數日,果相度。俄以尚書左丞宣慰汴宋陳許河陽行營,拜東都留守。卒,年七十六,贈太子少保,謚曰憲。



 孟容方勁有禮學,每所折衷,咸得其正。好提腋士,天下清議上之。



 弟季同,始署西川韋皋府判官。劉闢反,棄妻子歸,拜監察御史。歷長安令,再遷兵部郎中。孟容為禮部侍郎,徙季同京兆少尹。時京兆尹元義方出為鄜坊觀察使,奏劾宰相李絳與季同舉進士為同年,才數月輒徙。帝以問絳,絳曰:「進士、明經,歲大抵百人,吏部得官至千人,私謂為同年,本非親與舊也。今季同以兄嫌徙少尹,豈臣所助邪?且忠臣事君,不以私害公,設有才,雖親舊當白用。避嫌不用,乃臣下身謀,非天子用人意。」帝然之。終宣歙觀察使。



 薛存誠,字資明,河中寶鼎人。中進士第。擢累監察御史。元和初,討劉闢,郵傳事叢,詔以中人為館驛使,存誠以為害體甚,奏罷之。轉殿中侍御史,累遷給事中。瓊林庫廣籍工徒,存誠曰:「此奸人羼名以避征役,不可許。」又神策軍與咸陽尉袁儋不平,誣奏之,儋被罰。二敕皆執不下。憲宗悅,遣使勞之,拜御史中丞。浮屠鑒虛者,自貞元中關通賂遺,倚宦豎為奸,會坐于頔、杜黃裳家事,逮捕下獄。存誠窮劾之,得贓數十萬,當以大闢。權近更保救於帝,有詔釋之,存誠不聽。明日,詔使詣臺諭曰:「朕須此囚面詰,非赦也。」存誠奏曰:「獄已具,陛下必欲召赦之,請先殺臣乃可。不然,臣不敢奉詔。」鑒虛卒抵死。江西監軍高重昌妄劾信州刺史李位謀反,追付仗內詰狀。存誠一日三表,請付位御史臺。及按,果無實。



 未幾,復為給事中。會御史中丞闕,帝謂宰相曰:「持憲無易存誠者。」乃復命之。會暴卒,帝悼惜,贈刑部侍郎。存誠性和易,於人無所不容,及當官,毅然不可奪。子廷老。



 廷老,字商叟,及進士第,讜正有父風。寶歷中,為右拾遺。敬宗政日僻,嘗與舒元褒、李漢入閣論奏曰:「比除拜不由宰司擬進,恐綱紀浸壞,奸邪放肆。」帝厲語曰:「更論何事?」元褒曰:「宮中興作太甚。」帝色變,曰:「興作何所?」元褒不能對。廷老曰:「臣等以諫為職,有聞即應論奏。然見外輦材瓦絕多,知有所營。」帝曰:「已諭。」時造清思院,殿中用銅鑒三千,薄金十萬餅,故廷老等懇言之。尋加史館脩撰。



 鄭注用事,嶺南節度使鄭權附之,悉盜公庫寶貨輸注家為謝。廷老表按權罪,由是中人切齒。又論李逢吉黨張權輿、程昔範不宜居諫爭官,逢吉怒。會廷老告滿百日,出為臨晉令。文宗立,召為殿中侍御史。李讓夷數薦之,拜翰林學士。日酣飲,不持檢操,帝不悅,並讓夷罷之。開成三年,遷給事中。在公卿間,侃侃不干虛譽,推為正人。卒,贈刑部侍郎。



 子保遜,第進士,擢累給事中。



 保遜子昭緯,乾寧中,至禮部侍郎。性輕率,坐事貶磎州刺史。



 李遜,字友道,魏申公發之後,趙郡所謂申公房者,客居荊州。始署山南東道掌書記,累遷濠州刺史。初,濠州兵謀殺其將楊騰,騰走揚州,因滅騰家,曹亡剽劫。遜至,鐫諭利害,眾釋鎧自歸。觀察使旨限外浮斂,遜一不應。入為虞部郎中。由衢州刺史以政最擢浙東觀察使。當貞元初,福建軍亂,前觀察使奏益兵三千屯於境,以折閩沖,遂為長戍,幾二十年。遜署事,即停其兵。



 入為給事中。故事,天子以畸日聽政,對群臣。遜奏:「陛下求治,而下有所陳,當不時上,豈宜限以日?如是,畢歲得望天子者幾何?」憲宗悅,從之。遷戶部侍郎。



 代嚴綬為山南東道節度使。時方討蔡,析山南東道為兩節度:以唐、鄧、隋三州授高霞寓,得專攻討,而遜督襄、復、郢、均、房五州賦饋之。初,襄陽兵隸霞寓者多逃還,後霞寓戰賊不勝,言為遜所橈。帝欲按狀,宰相請置不問,下遷太子賓客。中人誣之,更貶恩王傅。久乃歷京兆尹、國子祭酒。以檢校禮部尚書為忠武節度使。時吳元濟始平,治條疏纇,遜召會大眾,申嚴約束,明諭賞罰,上下皆感畏,眾遂安。遜於為政,抑強植弱,貧富均一,所至有績可紀。



 長慶初,幽、鎮繼亂,遜首建誅討計,不聽。詔以兵萬人會行營,即日上道,先諸軍至,由是進檢校吏部尚書。未幾,徙節鳳翔,過京師,以疾求解為刑部尚書。卒,年六十三,贈尚書右僕射,謚曰貞。



 子方玄,字景業,第進士。裴誼奏署江西府判官。有大獄,論死者十餘囚,方玄刺審其冤,悉平貸之。累為池州刺史。鉤檢戶籍,所以差量徭賦者,皆有科品程章,吏不得私。常曰:「沈約年八十,手寫簿書,蓋為此云。」終處州刺史。



 遜弟建,字杓直,與兄俱客荊州。鄉人爭鬥,不詣府而詣建,平決無頗。母憐其孝,每字之曰:「犬委子勸吾食,吾輒飽;進藥,吾意其瘳。」貞元中,補校書郎。德宗思得文學者,或以建聞,帝問左右,宰相鄭珣瑜曰:「臣為吏部時,當補校書者八人,它皆藉貴勢以請,建獨無有。」帝喜,擢左拾遺、翰林學士。



 順宗立,李師古以兵侵曹州,建作詔諭還之,詞不假借。王叔文欲更之,建不可。左除太子詹事,改殿中侍御史。以兵部郎中知制誥。宰相有竄定詔稿者,亟請解職,除京兆少尹。會遜被讒,建申治之,出為澧州刺史。召拜刑部侍郎。卒,贈工部尚書。



 初,建為學時,家苦貧。兄造知其賢,為營丐,使成就之。故遜、建皆舉進士。後雖通顯,未嘗治垣屋,以清儉稱。



 建子訥,字敦止,及進士第。遷累中書舍人,為浙東觀察使。性疏卞,遇士不以禮,為下所逐,貶朗州刺史。召為河南尹。時久雨,洛暴漲,訥行水魏王堤,懼漂泊,疾馳去,水遂大毀民廬。議者薄其材。初,訥居與宰相楊收接,收欲市訥冗舍以廣第,訥叱曰:「先人舊廬,為權貴優笑地邪?」凡三為華州刺史,歷兵部尚書,以太子太傅卒。遺命葬不請鹵簿,避贈謚,詔聽。



\end{pinyinscope}