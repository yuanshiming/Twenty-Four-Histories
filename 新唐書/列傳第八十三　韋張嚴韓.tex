\article{列傳第八十三 韋張嚴韓}

\begin{pinyinscope}

 韋皋,字城武,京兆萬年人。六代祖範,有勛力周、隋間。皋始仕為建陵挽郎,諸帥府更闢批判晚期馬克思。,擢監察御史。張鎰節度鳳翔,署營田判官。以殿中侍御史知隴州行營留事。



 德宗狩奉天,李楚琳殺鎰,劫眾叛歸硃泚,隴州刺史郝通奔降楚琳。始,泚以範陽軍鎮鳳翔,既歸節,而留兵五百戍隴上,以部將牛雲光督之。至是,雲光謀請皋為帥,將劫以臣泚。別將翟曄伺知以白皋。雲光懼不克,率眾出奔,至汧陽,遇泚奴使皋所,謂雲光曰:「太尉已為天子,使我以御史中丞授皋。若聽,固吾人也,不受,可遂誅之。請以兵俱。」許之。皋迎勞,先納奴,偽受泚詔。即讓雲光曰:「既去而復,何也?」對曰:「向未知公之命,故去;今還,願與公同生死。」皋曰:「大使固善,茍無它圖,請釋甲以安眾,而後可入也。」雲光以皋諸生,亡能為,乃命士委仗鎧,皋受而內其卒。明日,置酒大會,奴、雲光與其下至,皋伏甲左右廡,酒行,盡殺之,以其首徇。泚復使它奴拜皋鳳翔節度使,皋亦斬之及從騎三人,縱一人使報泚。帝聞,乃授皋隴州刺史,置奉義軍,拜節度使,寵其功。皋遣兄平及弇繼至奉天,士氣益壯,乃築壇血牲與士盟曰:「協力一心,以誅元惡,有渝此盟,神其殛之。」又馳使吐蕃與連和,隴坻遂安。帝自梁、洋還,召為左金吾衛將軍,遷大將軍。



 貞元初,代張延賞為劍南西川節度使。初,雲南蠻羈附吐蕃,其盜塞必以蠻為鄉道。皋計得雲南則斬虜右支,乃間使招徠之,稍稍通西南夷。明年,蠻大首領苴那時以王爵讓其兄子烏星。始,烏星幼,那時攝領其部,故請歸爵。皋上言:「禮讓行於殊俗,則怫戾者化,願皆封以示褒進。」詔可。又明年,雲南款邊求內屬,約東蠻鬼主驃傍、苴夢沖等絕吐蕃盟。五年,東蠻斷瀘水橋攻吐蕃,請皋濟師。皋遣精卒二千,與蠻共破吐蕃於臺登,殺青海大酋乞臧遮遮、臘城酋悉多楊硃及論東柴等,虜墜死崖谷不可計,多獲牛馬鎧裝。遮遮,尚結贊之子,虜貴將悍雄者也;既敗,酋長百餘行哭隨之。悍將已亡,則屯柵以次降定。進檢校吏部尚書。



 初,東蠻地二千里,勝兵常數萬,南倚閣羅鳳,西結吐蕃,狙勢強弱為患,皋能綏服之,故戰有功。詔以那時為順政王、夢沖懷化王、驃傍和義王,刻「兩林」、「勿鄧」等印以賜之。而夢沖復與吐蕃盟,皋遣別將蘇峞召之,詰其叛,斬於琵琶川,立次鬼主樣棄等,蠻部震服。乃建安夷軍於資州,維制諸蠻;城龍溪於西山,保納降羌。



 九年,天子城鹽州,策虜且來撓襲,詔皋出師牽維之。乃命大將董勔、張芬分出西山、靈關,破峨和、通鶴、定廉城,逾的博嶺,遂圍維州,搏棲雞,攻下羊溪等三城,取劍山屯焚之。南道元帥論莽熱來援,與戰,破其軍,進收白岸,乃城鹽州。詔皋休士。以功為檢校尚書右僕射、扶風縣伯。



 於是西山羌女、訶陵、南水、白狗、逋租、弱水、清遠、咄霸八國酋長,皆因皋請入朝。乃遣幕府崔佐時由石門趣雲南,而南詔復通。石門者,隋史萬歲南征道也,天寶中,鮮於仲通下兵南溪,道遂閉。至是蠻徑北谷,近吐蕃,故皋治復之。繇黎州出邛部,直云南,置青溪關,號曰「南道」。乃詔皋統押近界諸蠻、西山八國、雲南安撫使。俄進同中書門下平章事。



 十三年,復巂州。吐蕃怨,完壘造舟謀擾邊,皋輒破卻之。自是曩貢、臘城等九節度嬰嬰、籠官馬定德與大將舉落皆降,昆明管些蠻又內附。贊普怒,遂北掠靈、朔,破麟州以取償焉。帝詔皋深入以橈虜。皋遣大將陳泊等出三奇,崔堯臣趨石門無衣山,仇冕、董振走維州,邢玼出黃崖略棲雞、老翁城,高倜、王英俊繇峨和、清溪道薄故松州,元膺出濕山、成溪,臧守至道黎、巂,韋良金趨平夷,路惟明自靈壁、夏陽攻逋租、偏松城,王有道涉大度河,陳孝陽率蠻苴那時等道西瀘攻昆明、諾濟,師無慮五萬,以八月悉出塞。十月,大破吐蕃,拔其保鎮捕候,追奔轉戰千里,遂圍維州。吐蕃釋靈、朔兵,使論莽熱以內大相兼東境五節度大使,率雜虜十萬來救。師伏以待,虜乘勝深入,師噪而奮,虜大潰,生禽莽熱獻諸朝。帝悅,進檢校司徒兼中書令、南康郡王,帝制紀功碑褒賜之。



 順宗立,詔檢校太尉。會王叔文等干政,皋遣劉闢來京師謁叔文曰:「公使私於君,請盡領劍南,則惟君之報。不然,惟君之怨。」叔文怒,欲斬闢,闢遁去。皋知叔文多釁,又自以大臣可與國大議,即上表請皇太子監國,又上箋太子,暴叔文、伾之奸,且勸進。會大臣繼請,太子遂受禪,因投殛奸黨。是歲,皋暴卒,年六十一,贈太師,謚曰忠武。



 皋治蜀二十一年,數出師,凡破吐蕃四十八萬,禽殺節度、都督、城主、籠官千五百,斬首五萬餘級,獲牛羊二十五萬,收器械六百三十萬,其功烈為西南劇。善拊士,至雖昏嫁皆厚資之,婿給錦衣,女給銀塗衣,賜各萬錢,死喪者稱是。其僚掾官雖顯,不使還朝,即署屬州刺史,自以侈橫,務蓋藏之。故劉闢階其厲,卒以叛。朝廷欲追繩其咎,而不與皋者詆所進兵皆鏤「定秦」字,有陸暢者上言:「臣向在蜀,知『定秦』者,匠名也。」繇是議息。暢,字達夫,皋雅所厚禮。始,天寶時,李白為《蜀道難》篇以斥嚴武,暢更為《蜀道易》以美皋焉。



 始,皋務私其民,列州互除租,凡三歲一復。皋沒,蜀人德之,見其遺象必拜。凡刻石著皋名者,皆鑱其文尊諱之。



 兄聿,弟平。聿以廕調南陵尉,遷秘書郎,以父嫌名換太子司議郎,闢淮南杜佑府。元和初,為國子司業。劉闢與盧文若反,皋子行式娶文若女弟,聿不以聞。闢平,行式妻當沒掖庭,有司並按聿,或以道遠不應坐,乃皆赦之。終太子右庶子。平與皋斬硃泚使者,間走奉天上功,擢萬年尉。



 平子正貫,字公理,少孤,皋謂能大其門,名曰臧孫。推廕為單父尉,不得意,棄官去,改今名。舉賢良方正異等,除太子校書郎,調華原尉。後又中詳閑吏治科,遷萬年主簿,擢累司農卿。坐尚食乏供,貶均州刺史。久之,進壽州團練使。宣宗立,以治當最,拜京兆尹、同州刺史。俄擢嶺南節度使。南海舶賈始至,大帥必取象犀明珠,上珍而售以下直。正貫既至,無所取,吏咨其清。南方風俗右鬼,正貫毀淫祠,教民毋妄祈。會海水溢,人爭咎撤祠事,以為神不厭,正貫登城沃酒以誓曰:「不當神意,長人任其咎,無逮下民。」俄而水去,民乃信之。居鎮三歲,既病,遺令無厚葬,無用鼓吹,無請謚。卒,年六十八,贈工部尚書。



 劉闢者,字太初,擢進士宏詞科,佐韋皋府,遷累御史中丞、支度副使。皋卒,闢主後務,諷諸將徼旄節,憲宗以給事中召之,不奉詔。時帝新即位,欲靜鎮四方,即拜檢校工部尚書、劍南西川節度使。闢意帝可動,益驁蹇,吐不臣語,求統三川,欲以所善盧文若節度東川,即以兵取梓州。且以術家言五福、太一舍於蜀,乃造大樓以祈祥。帝始重征討,而宰相杜黃裳勸帝,且言:「闢,妄書生耳,可鼓而俘也。」薦高崇文、李元弈等將神策行營兵皆西,使嚴礪、李康掎角之。



 詔許自新,闢不聽。崇文取東川,帝乃下詔奪其官,進破鹿頭關,遂下成都。闢從數十騎走,至羊灌田,自投水,不能死,騎將酈定進禽之。文若先殺其族,縋石自沈於江,失其尸。檻車送闢京師,尚冀不死,食飲於道晏然。將至都,神策以兵迎之,系其首,曳而入,驚曰:「何至是邪?」帝御興安樓受俘,詔詰反狀,闢曰:「臣不敢反,五院子弟為惡,不能制。」詔問:「遣使賜節何不受?」乃伏罪。獻廟社,徇於市,斬於城西南獨柳下。子超郎等九人,與部將崔綱以次誅。



 始,闢嘗病,見問疾者必以手行入其口,闢即裂食之。唯盧文若至,如平常,故益與之厚,而皆夷族。



 張建封,字本立,鄧州南陽人,客隱兗州。父玠,少任俠。安祿山反,使李廷偉脅徇山東,魯郡太守韓擇木迎館之。玠率豪桀段絳等集兵,將斬以徇,擇木不許,唯司兵參軍張孚助其謀,乃殺廷偉並其黨以聞。擇木、孚皆受賞,而玠去之江南,不自言功。



 建封少喜文章,能辯論,慷慨尚氣,自許以功名顯。李光弼鎮河南,盜起蘇、常間,殘掠鄉縣。代宗詔中人馬日新與光弼麾下皆討。建封見中人,請前喻賊,可不須戰。因到賊屯開譬禍福,一日降數千人,縱還田里,由是知名。湖南觀察使韋之晉闢署參謀,授左清道兵曹參軍,不樂職,輒去。令狐彰節度滑亳,奏置幕府,彰不朝覲,建封非之。往見轉運使劉晏,晏奏試大理評事,使筦漕務,歲餘罷。時馬燧為三城鎮遏使,雅知之,表為判官,擢監察御史。燧伐李靈耀,軍中事多所諏訪,從鎮河東,授侍御史,即表其能於朝。楊炎將任以要職,盧杞不喜,出為岳州刺史。



 李希烈既破梁崇義,跋扈不臣,壽州刺史崔昭與相聞,德宗召宰相選代昭者,杞倉卒不暇取它吏,即白用建封。希烈數敗王師,張甚,遂僭即天子位,淮南節度使陳少游陰附之。希烈遣將楊豐齎偽赦二,畀建封、少游。豐至,建封縛致軍中,會中人來,對之斬其首,因送偽書於行在。少游聞之,恚汗不自處,建封乃劾其附賊狀,帝方蒙難,不暇治也。希烈又署杜少誠為淮南節度使,約破壽州,以趣江都。建封壁霍丘秋柵拒之,賊不能東。遷團練使。帝還自梁,少游卒憂死。進兼御史大夫、濠壽廬觀察使。是時,四方尚多故,乃繕陴隍,益治兵,四鄙附悅。希烈使票帥悍卒來戰,建封皆沮衄之。賊平,進封階,又任一子正員官。



 貞元四年,拜御史大夫、徐泗濠節度使。始,李洧以徐降,洧卒,高承宗、獨孤華代之,地迫於寇,常困肸不支。於是李泌建言:「東南漕自淮達諸汴,徐之埇橋為江、淮計口,今徐州刺史高明應甚少,脫為李納所並,以梗餉路,是失江、淮也。請以建封代之,益與濠、泗二州。夫徐地重而兵勁,若帥又賢,即淄青震矣。」帝曰:「善。」繇是徐復為雄鎮。久之,檢校尚書右僕射。十三年,來朝,帝不待日召見延英殿,詔會朝赴大夫班,以示殊寵,建封賦《朝天行》以獻。帝眷遇異等,賜名馬珍具。



 是時,宦者主宮市,置數十百人閱物廛左,謂之「白望」。無詔文驗核,但稱宮市,則莫敢誰何,大率與直十不償一。又邀閽闥所奉及腳傭,至有重荷趨肆而徒返者。有農賣一驢薪,宦人以數尺帛易之,又取它費,且驅驢入宮,而農納薪辭帛,欲亟去,不許,恚曰:「惟有死耳!」遂擊宦者。有司執之以聞,帝黜宦人,賜農帛十匹,然宮市不廢也。諫臣交章列上,皆不納,故建封請間為帝言之,帝頗順聽。會詔書蠲民逋賦,帝問何如,答曰:「殘逋積負,決無可斂,雖蠲除之,百姓尚無所益。」又陳:「河東節度使李說、華州刺史盧征皆病不能事,左右得以為奸。右金吾大將軍李翰好刺細事規寵,人疾惡之。」帝悉嘉可。未幾,制詔:「官師過從,人情之常,自今金吾勿以聞。」



 元巳,賜宴曲江,特詔與宰相同榻食。其還鎮,帝賦詩以餞,於時雖馬燧、渾瑊、劉玄佐、李抱真等勛寵卓越,未有以詩餞者。帝又使左右以所持鞭賜之,曰:「卿節誼歲寒弗渝,故用此為況。」建封又賦詩以自警勵。十六年,以病求代,詔韋夏卿代之,未至而建封卒。年六十六,冊贈司徒。



 治徐凡十年,躬於所事,一軍大治。善容人過,至健黠亦未嘗曲法假之。其言忠義感激,故下皆畏悅。性樂士,賢不肖游其門者禮必均,故其往如歸。許孟容、韓愈皆奏署幕府,有文章傳於時。



 子愔,始以廕補虢州參軍事。建封卒,府佐鄭通誠者攝留事,畏其軍亂,因浙西戍兵過徐,謀引以為援。舉軍怒,斧庫取兵,環府大噪,殺通誠及大將數人,乃表於朝,請愔為留後,假旄節。帝不許,披濠、泗隸淮南,詔杜佑討徐亂。泗州刺史張伾以兵攻埇橋,與徐軍遇,伾大敗。帝未有以制,乃授愔右驍衛將軍、徐州刺史,知留後。以伾為泗州留後,杜兼為濠州留後。俄進愔武寧軍節度使。



 元和初,以疾求代,召為工部尚書,以王紹節度武寧,還濠、泗隸徐。徐人喜,遂不敢亂,而愔得行。未逾境,卒。愔治徐七年,其政稱治。贈尚書右僕射。



 嚴震,字遐聞,梓州鹽亭人。本農家子,以財役裏閭。至德、乾元中,數出貲助邊,得為州長史。西川節度使嚴武知其才,署押衙,遷恆王府司馬,委以軍府眾務。武卒,罷歸。會東川節度使李叔明表為渝州刺史,震以叔明姻家,移疾去。山南西道節度府又表為鳳州刺史。母喪解。起為興、鳳兩州團練使,好興利除害。建中中,劍南黜陟使韋楨狀震治行為山南第一,乃賜上下考,封鄖國公。治鳳十四年,號稱清嚴,遠邇咨美。遷山南西道節度使。



 硃泚反,遣腹心穆廷光等遺帛書誘之,震即斬以聞。是時,李懷光與賊連和,奉天危蹙,帝欲徙蹕山南,震聞,馳表奉迎,遣大將張用誠以兵五千捍衛。用誠至盩厔有反計,帝憂之,會震牙將馬勛嗣至,帝告以故,勛曰:「臣請歸取節度符召之,即不受,斬其首以復命。」帝悅,使計日往。勛還得符,請壯士五人與偕,出駱谷,用誠以為未知其謀,以數百騎迓勛館之,左右嚴侍。勛未發,陰令焚草館外,士寒爭附火,勛從容引符示之,曰:「大夫召君。」用誠懼,將走,壯士自後禽之。用誠子斫勛傷首,左右捍刀得免,遂僕用誠,而格殺其子。勛即軍中,士皆擐甲矣。勛昌言曰:「若父母妻子在梁州,今棄之而反,何所利邪?大夫取用誠爾,若等無與!」眾乃服,不敢動。即縛用誠送於震,杖殺之,而拔其副以統師。始,勛赴行在,逾半日期,帝頗憂。比至,大喜。翌日,發奉天。既入駱谷,懷光以騎追襲,賴山南兵以免。尋加檢校戶部尚書、馮翊郡王,實封二百戶。



 天子至梁州,宰相以為地貧無所仰給,請進幸成都。震曰;「山南密邇畿輔,李晟銳於收復,方藉六師為聲援,今引而西,則諸將顧望,責功無期。」帝未決,會晟表至,亦請駐蹕梁、洋,議遂定。然梁、漢間刀耕火耨,民採穭為食,雖領十五郡,而賦入才比東方數大縣。自安、史後,山賊剽掠,戶口流散,震隨宜勸課,鳩斂有法,民不煩擾,而行在供億具焉。車駕將還,加檢校尚書左僕射。詔改梁州為興元府,即用震為尹,加實封二百戶。久之,進同中書門下平章事。貞元十五年卒,年七十六,贈太保,謚曰忠穆。



 從孫譔,與宰相楊收善。咸通中,繇桂管觀察使擢為江西節度使,改號鎮南軍。時南蠻內寇,詔譔募士三萬備之。或言譔廣補卒,擅納縑廩。及收得罪,韋保衡以譔素善收,賕賄狼藉,遣使按覆,詔賜死。



 韓弘,滑州匡城人。少孤,依其舅劉玄佐。舉明經不中,從外家學騎射。由諸曹試大理評事,為宋州南城將。事劉全諒,署都知兵馬使。貞元十五年,全諒死,軍中思玄佐,以弘才武,共立為留後,請監軍表諸朝。詔檢校工部尚書,充宣武節度副大使,知節度事。



 先是,曲環死,吳少誠與全諒謀襲陳許,使數輩仍在館。弘始得帥,欲以忠自表於眾,即驅出少誠使斬之,選卒三千,會諸軍擊少誠,敗之。汴自劉士寧以來,軍益驕,及殺陸長源,主帥勢輕,不可制。弘察軍中素恣橫者劉鍔等三百人,一日,數其罪斬之牙門,流血丹道,弘言笑自如。自是訖弘去,無一敢肆者。李師古屯曹州,以謀鄭、滑,或告:「師古治道矣,兵且至,請備之。」弘曰:「師來不除道也。」師古情得,乃引去。累授檢校司空、同中書門下平章事。弘以官與太原王鍔等,詒書宰相,恥為鍔下。憲宗方用兵淮西,藉其重,更授檢校司徒,班鍔上。



 嚴綬以王師敗,乃拜弘淮西諸軍行營都統,使捍兩河,而令李光顏、烏重胤擊賊。弘不親屯,遣子公武領兵三千屬光顏,然陰為逗橈計,以危國邀功者,每諸將告捷,輒累日不怡。元濟平,以功加兼侍中,封許國公。李師道誅,弘大懼,因請入朝,冊拜司徒、中書令,以足疾,命中人掖拜,固願留京師。帝崩,攝塚宰。俄出為河中節度使。以病請還,復拜司徒、中書令。卒,年五十八,贈太尉,謚曰隱。



 始,弘自汴來朝,獻馬三千、絹五十萬、它錦彩三萬,而汴之庫廄錢尚百萬緡,絹亦百餘萬,馬七千,糧三百萬斛,兵械不可數。弘為人莊重寡言,罪殺人,問法何如,不自為輕重,沉謀勇斷,故少誠、師道等皆憚之。詔使至,或驁侮不為禮。齊、蔡平,勢屈而後請覲,然天子尊寵異等,能以名位始終,亦其天幸。



 子公武,字從偃。起家衛尉主簿,為宣武行營兵馬使,以討蔡功檢校左散騎常侍、鄜坊等州節度使。弘入朝,為右金吾將軍。弘出河中,弘弟充徙宣武,乃曰:「二父居重鎮,我以孺子又當執金吾職乎?」因固辭,改右驍衛大將軍。性恭遜,不以富貴自處。卒,贈戶部尚書,謚曰恭。



 充,本名璀,少亦依舅家。李元為河陽節度使,署牙將。元改昭義,又從之。元嘗謂賓佐曰:「充後當貴,諸君必善事之。」未幾,弘領宣武,召主親兵,元曰:「我知君舊矣,吾兒不才,無足累君者,二女方幼,以為托。」遂辭去。累授御史大夫。



 弘峻法,人人不自保。充謙慎無少懈,念弘在鎮久,不入見天子,身又得士,不自安,因請入宿衛,弘許之,不即遣。後因獵,單騎走洛陽,朝廷亮其節,擢右金吾衛將軍,轉大將軍,斥軍士虛名不如令者七百人。歷少府監、鄜坊等州節度使。



 穆宗立,幽、鎮、魏復亂,王承元以冀兵二千屯滑州,朝廷恐冀兵相訹為叛,徙承元鄜坊,而授充檢校尚書左僕射,為義成軍節度使。會汴軍逐李願,以李朅主留事。帝謂充素為汴士悅向,詔節度宣武,兼統義成兵討朅。戰郭橋,破之。會李質斬朅,遂入汴。初,陳許李光顏亦奉詔討朅,屯尉氏,意先得汴,欲俘掠以餌軍,而汴監軍姚文壽亦欲內光顏。充聞其謀,馳至城下,汴人望見充,歡躍無復貳者。



 始,帝遣人問破賊期,充對:「汴,天下咽喉,臣頗習其人,然王師臨之,一月可破。」方二旬即克。帝喜曰:「充料敵若神。」加檢校司空。籍朅所脅為兵者三萬,悉縱之。又責首亂者千餘,斥出境,令曰:「敢後者斬!」由是內外按堵,汴人愛賴之。卒,年五十五,贈司徒,謚曰肅。



 充雖將家,性儉節,歷三鎮,居處服玩如儒生,乘機決策無餘悔,世推善將。李元沒,充為嫁二女,周其家。自弘去汴,監軍選軍中敢士二千直閣下,日秩酒肴,物力幾屈,然不敢廢。充未入時,李質總軍事,乃曰:「韓公至而頓去二千人食,豈不失人心乎?不去,且無以繼,可以弊事遺吾帥乎!」因悉罷之而後迎充。



 李質者,節士也。始為牙將,及朅為留後,邀帥節,勸之不從。朅疽發於首,委質以兵,遂禽朅。終金吾將軍。



 贊曰:皋、建封、弘本諸生,震興田畝間,未有以異人,及投隙龍驤,皆為國梁楹,光奮一時。使不遭遇,與庸夫汩汩並胔而腐可也。皋、弘雖陰慝,卒能以誠言自解,長沒天年,宜哉!



\end{pinyinscope}