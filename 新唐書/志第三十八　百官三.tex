\article{志第三十八 百官三}

\begin{pinyinscope}

 ○御史臺



 大夫一人,正三品;中丞二人,正四品下。大夫掌以刑法典章糾正百官之罪惡,中丞為之貳。其屬有三院:一曰臺院,侍御史隸焉;二曰殿院,殿中侍御史隸焉;三曰察院,監察御史隸焉。凡冤而無告者,三司詰之。三司,謂御史大夫、中書、門下也。大事奏裁,小事專達。凡有彈劾,御史以白大夫,大事以方幅,小事署名而已。有制覆囚,則與刑部尚書平閱。行幸,乘路車為導。朝會,則率其屬正百官之班序,遲明列於兩觀,監察御史二人押班,侍御史顓舉不如法者。文武官職事九品以上及二王後,朝朔望。文官五品以上及兩省供奉官、監察御史、員外郎、太常博士日參,號常參官。武官三品以上三日一朝,號九參官;五品以上及折沖當番者五日一朝,號六參官。弘文、崇文館、國子監學生四時參。凡諸王入朝及以恩追至者,日參。九品以上,自十月至二月,褲褶以朝;五品以上有珂,蕃官及四品非清官則否。凡朝位以官,職事同者先爵,爵同以齒,致仕官居上;職事與散官、勛官合班,則文散官居職事之下,武散官次之,勛官又次之;官同者,異姓為後。親王、嗣王任文武官者,從其班,官卑者從王品;郡王任三品以下職事者,居同階品之上。非任文武官者,嗣王居太子太保之下,郡王次之,國公居三品之下,郡公居從三品之下,縣公居四品之下,侯居從四品之下,伯居五品之下,子居從五品之上,男居從五品之下。以前官召見者,居本品見任之上,以理解者,居同品之下。本司參集者,以職事為上下。文武三品非職事官者,朝參名簿,皆稱曰諸公。凡出,不逾四面關則不辭見。都督、刺史、都護既辭,候旨於側門。左右僕射、侍中、中書令初拜,以表讓。中書門下五品以上及諸司長官,謝於正衙,復進狀謝於側門。兩班三品以朔望朝,就食廊下,殿中侍御史二人為使涖之。高宗改治書侍御史中丞,以避帝名;龍朔二年,改御史臺曰憲臺,大夫曰大司憲,中丞曰司憲大夫。武後文明元年,改御史臺曰肅政臺。光宅元年,分左右臺:左臺知百司,監軍旅;右臺察州縣、省風俗。尋命左臺兼察州縣。兩臺歲再發使八人,春曰風俗,秋曰廉察,以四十八條察州縣。兩臺御史,有假、有檢校、有員外、有試,至神龍初皆廢。景雲三年,以兩臺望齊,糾舉苛察,百僚厭其煩,乃廢右臺。延和元年復置,歲中以尚書省隸左臺,月餘而右臺復廢。至德後,諸道使府參佐,皆以御史為之,謂之外臺;復有檢校、里行、內供奉,或兼或攝,諸使下官亦如之。會昌初,升大夫、中丞品。東都留臺,有中丞一人、侍御史一人、殿中侍御史二人、監察御史三人;元和後,不置中丞,以侍御史、殿中侍御史、監察御史主留臺務,而三院御史亦不常備。



 侍御史六人,從六品下。掌糾舉百寮及入閤承詔,知推、彈、雜事。凡三司理事,與給事中、中書舍人更直朝堂。若三司所按而非其長官,則與刑部郎中、員外郎、大理司直、評事往訊。彈劾,則大夫、中丞押奏。大事,法冠、硃衣、纁裳、白紗中單;小事常服。久次者一人知雜事,謂之雜端,殿中監察職掌、進名、遷改及令史考第,臺內事顓決,亦號臺端。次一人知公廨。次一人知彈。分京城諸司及諸州為東、西:次一人知西推、贓贖、三司受事,號副端;次一人知東推、理匭等,有不糾舉者罰之;以殿中侍御史第一人同知東推,蒞太倉出納;第二人同知西推,蒞左藏出納。號四推御史。只日,臺院受事;雙日,殿院受事。次侍御史一人,分司東都臺。凡御史以下遇長官於路,去戴下馬,長官斂轡止之。出入行止,殿中以下視以為法,先後有罰。入朝,則與殿中侍御史隨仗分入,東則居侍中、黃門侍郎、給事中之次,西則居中書令、侍郎、舍人之次,各居中丞、大夫下。每一人東向承詔五日,有旨召御史,不呼名則承詔者出。樂彥瑋為大夫,以嘗召兩御史,乃加副承詔一人,闕則殿中承乏。監察御史分日直朝堂,入自側門,非奏事不至殿庭;正門無籍;天授中,詔側門置籍,得至殿庭;開元七年,又詔隨仗入閤。分左右巡,糾察違失,左巡知京城內,右巡知京城外,盡雍、洛二州之境,月一代,將晦,即巡刑部、大理、東西徒坊、金吾、縣獄。搜狩,則監圍,察斷絕失禽者。其後,以殿中掌左右巡;尋以務劇,選用京畿縣尉。又置御史裏行使、侍御史裏行使、殿中里行使、監察裏行使,以未為正官,無員數。唐法,殿中侍御史遷拜及職事,與侍御史鈞。開元以降,權屬侍御史,而殿中兼知庫藏、宮門內事。故事,御史臺不受訟,有訴可聞者略其姓名,托以風聞。其後,御史嫉惡者少,通狀壅絕。十四年,乃定授事御史一人,知其日劾狀,題告事人姓名。其後,宰相以御史權重,建議彈奏先白中丞、大夫,復通狀中書、門下,然後得奏。自是御史之任輕矣。建中元年,以侍御史分掌公廨、推、彈,自是雜端之任輕矣。元和八年,命四推御史受事,周而復始,罷東西分日之限。隋末,廢殿內侍御史;義寧元年,承相府置察非掾二人;武德元年,改曰殿中侍御史;龍朔元年,置監察御史裏行;武後文明元年,置殿中里行,後亦顓以里行名官;長安二年,置內供奉。



 主簿一人,從七品下。掌印,受事發辰,核臺務,主公廨及戶奴婢、勛散官之職。錄事二人,從九品下。有主事二人。臺院有令史七十八人,書令史二十五人,亭長六人,掌固十二人。殿院有令史八人,書令史十八人。察院有計史三十四人,令史十人,掌固十二人。



 殿中侍御史九人,從七品下。掌殿庭供奉之儀,京畿諸州兵皆隸焉。正班,列於閤門之外,糾離班、語不肅者。元日、冬至朝會,則乘馬、具服、戴黑豸升殿。巡幸,則往來門旗之內,檢校文物虧失者。一人同知東推,監太倉出納;一人同知西推,監左藏出納;二人為廊下食使;二人分知左右巡;三人內供奉。



 監察御史十五人,正八品下。掌分察百寮,巡按州縣,獄訟、軍戎、祭祀、營作、太府出納皆蒞焉;知朝堂左右廂及百司綱目。



 凡十道巡按,以判官二人為佐,務繁則有支使。其一,察官人善惡;其二,察戶口流散,籍帳隱沒,賦役不均;其三,察農桑不勤,倉庫減耗;其四,察妖猾盜賊,不事生業,為私蠹害;其五,察德行孝悌,茂才異等,藏器晦跡,應時用者;其六,察黠吏豪宗兼並縱暴,貧弱冤苦不能自申者。凡戰伐大克獲,則數俘馘、審功賞,然後奏之。屯田、鑄錢,嶺南、黔府選補,亦視功過糾察。決囚徒,則與中書舍人、金吾將軍蒞之。國忌齋,則與殿中侍御史分察寺觀。蒞宴射、習射及大祠、中祠,視不如儀者以聞。



 初,開元中,兼巡傳驛,至二十五年,以監察御史檢校兩京館驛。大歷十四年,兩京以御史一人知館驛,號館驛使。監察御史分察尚書省六司,繇下第一人為始,出使亦然。興元元年,以第一人察吏部、禮部,兼監祭使;第二人察兵部、工部,兼館驛使;第三人察戶部、刑部。歲終議殿最。元和中,以新人不出使無以觀能否,乃命顓察尚書省,號曰六察官。開元十九年,以監察御史二人蒞太倉、左藏庫。三院御史,皆初領繁劇外府推事。其後,以殿中侍御史上一人為監太倉使,第二人為監左藏庫使。



 凡諸使下三院御史內供奉,其班居正臺監察御史之上。



 ○太常寺



 卿一人,正三品;少卿二人,正四品上。掌禮樂、郊廟、社稷之事,總郊社、太樂、鼓吹、太醫、太卜、廩犧、諸祠廟等署,少卿為之貳。凡大禮,則贊引;有司攝事,則為亞獻;三公行園陵,則為副;大祭祀,省牲、器,則謁者為之導;小祀及公卿嘉禮,命謁者贊相。凡巡幸、出師、克獲,皆擇日告太廟。凡藏大享之器服,有四院:一曰天府院,藏瑞應及伐國所獲之寶,禘祫則陳於廟庭;二曰御衣院,藏天子祭服;三曰樂縣院,藏六樂之器;四曰神廚院,藏御廩及諸器官奴婢。初,有衣冠署,令,正八品上;貞觀元年,署廢。高宗即位,改治禮郎曰奉禮郎,以避帝名;龍朔二年,改太常寺曰奉常寺,九寺卿皆曰正卿,少卿曰大夫。武後光宅元年,復改太常寺曰司常寺。



 丞二人,從五品下。掌判寺事。凡享太廟,則修七祀於西門之內。主簿二人,從七品上。



 博士四人,從七品上。掌辨五禮;按王公、三品以上功過善惡為之謚;大禮,則贊卿導引。



 太祝六人,正九品上。掌出納神主,祭祀則跪讀祝文;卿省牲則循牲告充,牽以授太官。



 奉禮郎二人,從九品上。掌君臣版位,以奉朝會、祭祀之禮。宗廟則設皇帝位於庭,九廟子孫列焉,昭、穆異位,去爵從齒。凡樽、彞、勺、冪、篚、坫、簠、簋、登,鈃、籩、豆,皆辨其位。凡祭祀、朝會,在位拜跪之節,皆贊導之。公卿巡行諸陵,則主其威儀鼓吹,而相其禮。



 協律郎二人,正八品上。掌和律呂。錄事二人,從九品上;八寺錄事品同。有禮院修撰、檢討官各一人,府十一人,史二十三人,謁者十人,贊引二十人,贊者四人,祝史六人,贊者十六人。太常寺、禮院禮生各三十五人,亭長八人,掌固十二人。



 △兩京郊社署



 令各一人,從七品下;丞各一人,從八品上。令掌五郊、社稷、明堂之位,與奉禮郎設樽、罍、篚、冪,而太官令實之。立燎壇,積柴。合朔有變,則巡察四門,以俟變過,明則罷。有府二人,史四人,典事五人,掌固五人,門僕八人,齋郎百一十人。齋郎掌供郊廟之役。太廟九室,室有長三人,以主樽、罍、篚、冪、鎖鑰,又有罍洗二人;郊壇有掌坐二十四人,以主神御之物。皆禮部奏補。凡室長十年、掌座十二年,皆授官。祭饗而員少,兼取三館學生,皆絳衣絳幘。更一番者,戶部下蠲符,歲一申考諸署所擇者,太常以十月申解於禮部,如貢舉法,帖《論語》及一大經。中第者錄奏吏部注冬集散官,否者番上如初。六試而絀,授散官。唐初,以郊社、太樂、鼓吹、太醫、太官、左藏、乘黃、典廄、典客、上林、太倉、平準、常平、典牧、左尚、右尚為上署,鉤盾、右藏、織染、掌冶為中署,珍羞、良醞、掌醞、守宮、武器、車府、司儀、崇玄、導官、甄官、河渠、弩坊、甲坊、舟楫、太卜、廩犧、中校、左校、右校為下署。



 △太樂署



 令二人,從七品下;丞一人,從八品下;樂正八人,從九品下。令掌調鐘律,以供祭饗。凡習樂,立師以教,而歲考其師之課業為三等,以上禮部。十年大校,未成,則五年而校,以番上下。有故及不任供奉,則輸資錢,以充伎衣樂器之用。散樂,閏月人出資錢百六十,長上者復繇役,音聲人納資者歲錢二千。博士教之,功多者為上第,功少者為中第,不勤者為下第,禮部覆之。十五年有五上考、七中考者,授散官,直本司,年滿考少者,不敘。教長上弟子四考,難色二人、次難色二人業成者,進考,得難曲五十以上任供奉者為業成。習難色大部伎三年而成,次部二年而成,易色小部伎一年而成,皆入等第三為業成。業成、行脩謹者,為助教;博士缺,以次補之。長上及別教未得十曲,給資三之一;不成者隸鼓吹署。習大小橫吹,難色四番而成,易色三番而成;不成者,博士有謫。內教博士及弟子長教者,給資錢而留之。武德後,置內教坊於禁中。武後如意元年,改曰雲韶府,以中官為使。開元二年,又置內教坊於蓬萊宮側,有音聲博士、第一曹博士、第二曹博士。京都置左右教坊,掌俳優雜技。自是不隸太常,以中官為教坊使。唐改太樂為樂正,有府三人,史六人,典事八人,掌固六人,文武二舞郎一百四十人,散樂三百八十二人,仗內散樂一千人,音聲人一萬二十七人。有別教院。開成三年,改法曲所處院曰仙韶院。



 △鼓吹署



 令二人,從七品下;丞二人,從八品下;樂正四人,從九品下。令掌鼓吹之節。合朔有變,則帥工人設五鼓於太社,執麾旒於四門之塾,置龍床,有變則舉麾擊鼓,變復而止。馬射,設㧏鼓金鉦,施龍床。大儺,帥鼓角以助侲子之唱。有府三人,史六人,典事四人,掌固四人。唐並清商、鼓吹為一署,增令一人。



 △太醫署



 令二人,從七品下;丞二人,醫監四人,並從八品下;醫正八人,從九品下。令掌醫療之法,其屬有四:一曰醫師,二曰針師,三曰按摩師,四曰咒禁師。皆教以博士,考試登用如國子監。醫師、醫正、醫工療病,書其全之多少為考課。歲給藥以防民疾。凡陵寢廟皆儲以藥,尚藥、太常醫各一人受之。宮人患坊有藥庫,監門蒞出給;醫師、醫監、醫正番別一人蒞坊。凡課藥之州,置採藥師一人。京師以良田為園,庶人十六以上為藥園生,業成者為師。凡藥,辨其所出,擇其良者進焉。有府二人,史四人,主藥八人,藥童二十四人,藥園師二人,藥園生八人,掌固四人,醫師二十人,醫工百人,醫生四十人,典藥一人,針工二十人,針生二十人,按摩工五十六人,按摩生十五人,咒禁師二人,咒禁工八人,咒禁生十人。



 醫博士一人,正八品上;助教一人,從九品上。掌教授諸生以《本草》、《甲乙》、《脈經》,分而為業:一曰體療,二曰瘡腫,三曰少小,四曰耳目口齒,五曰角法。



 針博士一人,從八品上;助教一人,針師十人,並從九品下。掌教針生以經脈、孔穴,教如醫生。



 按摩博士一人,按摩師四人,並從九品下。掌教導引之法以除疾,損傷折跌者,正之。



 咒禁博士一人,從九品下。掌教咒禁祓除為厲者,齋戒以受焉。



 △太卜署



 令一人,從七品下;丞二人,從八品下;卜正、博士各二人,從九品下。掌卜筮之法:一曰龜,二曰五兆,三曰易,四曰式。祭祀、大事,率卜正卜日,示高於卿,退而命龜,既灼而占,先上旬,次中旬,次下旬。小祀、小事者,則卜正示高、命龜、作,而太卜令佐蒞之。季冬,帥侲子堂贈大儺,天子六隊,太子二隊,方相氏右執戈、左執楯而導之,唱十二神名,以逐惡鬼,儺者出,礫雄雞於宮門、城門。有卜助教二人,卜師二十人,巫師十五人,卜筮生四十五人,府一人,史二人,掌固二人。



 △廩犧署



 令一人,從八品下;丞二人,正九品下。掌犧牲粢盛之事。祀用太牢者,三牲加酒、脯、醢,與太祝牽牲就榜位,卿省牲,則北面告腯,以授太官。籍田,則供耒於司農卿,卿以授侍中;籍田所收以供粢盛、五齊、三酒之用,以餘及槁飼犧牲。有府一人,史二人,典事二人,掌固二人。



 △汾祠署



 令一人,從七品下;丞一人,從八品上。掌享祭灑掃之制。有府二人,史四人,廟幹二人。開元二十一年置署。



 三皇五帝以前帝王、三皇、五帝、周文王、周武王、漢高祖、兩京武成王廟



 令一人,從六品下;丞一人,正八品下。掌開闔、灑掃、釋奠之禮。有錄事一人,府二人,史四人,廟幹二人,掌固四人,門僕八人。神龍二年,兩京置齊太公廟署,其後廢;開元十九年復置。天寶三載,初置周文王廟署;六載,置三皇五帝廟署;七載,置三皇五帝以前帝王廟署;九載,置周武王、漢高祖廟署。上元元年,改齊太公署為武成王廟署,硃全忠曰武明。



 ○光祿寺



 卿一人,從三品;少卿二人,從四品上;丞二人,從六品上;主簿二人,從七品上。掌酒醴膳羞之政,總太官、珍羞、良醞、掌醢四署。凡祭祀,省牲鑊、濯溉;三公攝祭,則為終獻。朝會宴享,則節其等差。錄事二人。龍朔二年,改光祿寺曰司宰寺。武後光宅元年,曰司膳寺。有府十一人,史二十一人,亭長六人,掌固六人。



 △太官署



 令二人,從七品下;丞四人,從八品下。掌供祠宴朝會膳食。祭日,令白卿詣廚省牲鑊,取明水、明火,帥宰人割牲,取毛血實豆,遂烹。又實簠簋,設於饌幕之內。有府四人,史八人,監膳十人,監膳史十五人,供膳二千四百人,掌固四人。



 △珍羞署



 令一人,正八品下;丞二人,正九品下。掌供祭祀、朝會、賓客之庶羞,榛慄、脯脩、魚鹽、菱芡之名數。武后垂拱元年,改肴藏署曰珍羞署,神龍元年復舊,開元元年又改。有府三人,史六人,典書八人,餳匠五人,掌固四人。



 △良醞署



 令二人,正八品下;丞二人,正九品下。掌供五齊、三酒。享太廟,則供鬱鬯以實六彞;進御,則供春暴、秋清、酴麋、桑落之酒。有府三人,史六人,監事二人,掌醞二十人,酒匠十三人,奉觶百二十人,掌固四人。



 △掌醢署



 令一人,正八品下;丞二人,正九品下。掌供醢醯之物:一曰鹿醢,二曰兔醢,三曰羊醢,四曰魚醢。宗廟,用菹以實豆;賓客、百官,用醯醬以和羹。有府二人,史二人,主醢十人,醬匠二十三人,酢匠十二人,豉匠十二人,菹醯匠八人,掌固四人。



 ○衛尉寺



 卿一人,從三品;少卿二人,從四品上;丞二人,從六品上。掌器械文物,總武庫、武器、守宮三署。兵器入者,皆籍其名數。祭祀、朝會,則供羽儀、節鉞、金鼓,帷帟、茵席。凡供宮衛者,歲再閱,有敝則脩於少府。主簿二人,從七品上。錄事一人。龍朔二年,改曰司衛寺。武後光宅元年又改。有府六人,史十一人,亭長四人,掌固六人。



 丞,掌判寺事,辨器械出納之數。大事承制敕,小事則聽於尚書省。



 △兩京武庫署



 令各二人,從六品下;丞各二人,從八品下。掌藏兵械。有赦,建金雞,置鼓宮城門之右,大理及府縣囚徒至,則擊之。監事各一人,正九品上。諸署監事,品同。有府各六人,史各六人,典事各二人,掌固各五人。開元二十五年,東都亦置署。



 △武器署



 令一人,正八品下;丞二人,正九品下。掌外戎器。祭祀、巡幸,則納於武庫。給六品以上葬鹵簿、棨戟。凡戟,廟、社、宮、殿之門二十有四,東宮之門一十八,一品之門十六,二品及京兆河南太原尹、大都督、大都護之門十四,三品及上都督、中都督、上都護、上州之門十二,下都督、下都護、中州、下州之門各十。衣幡壞者,五歲一易之。薨卒者既葬,追還。監事二人。有府二人,史六人,典事二人,掌固四人。貞觀中,東都亦置署。



 △守宮署



 令一人,正八品下;丞二人,正九品下。掌供帳帟。祭祀、巡幸,則設王公百官之位。吏部、兵部、禮部試貢舉人,則供帷幕。王公婚禮,亦供帳具。京諸司長上官,以品給其床罽。供蕃客帷帟,則題歲月。席壽三年,氈壽五年,褥壽七年;不及期而壞,有罰。監事二人。有府二人,史四人,掌設六人,幕士八十人,掌固四人。



 ○宗正寺



 卿一人,從三品;少卿二人,從四品上;丞二人,從六品上。掌天子族親屬籍,以別昭穆;領陵臺、宗玄二署。凡親有五等,先定於司封:一曰皇帝周親、皇后父母,視三品;二曰皇帝大功親、小功尊屬,太皇太后、皇太后、皇後周親,視四品;三曰皇帝小功親、緦麻尊屬,太皇太后、皇太后、皇后大功親,視五品;四曰皇帝緦麻親、袒免尊屬,太皇太后、皇太后、皇后小功親;五曰皇帝袒免親,太皇太后小功卑屬,皇太后、皇后緦麻親,視六品。皇帝親之夫婦男女,降本親二等,餘親降三等,尊屬進一等,降而過五等者不為親。諸王、大長公主、長公主親,本品;嗣王、郡王非三等親者,亦視五品;駙馬都尉,視諸親。祭祀、冊命、朝會,陪位、襲封者皆以簿書上司封。主簿二人,從七品上。知圖譜官一人,脩玉牒官一人,知宗子表疏官一人,錄事二人。武德二年,置宗師一人,後省。龍朔二年,改宗正寺曰司宗寺。武後光宅元年曰司屬寺。有府五人,史五人,亭長四人,掌固四人。京都太廟齋郎各一百三十人,門僕各三十三人,主簿、錄事各二人。



 △諸陵臺



 令各一人,從五品上;丞各一人,從七品下。建初、啟運、興寧、永康陵,令各一人,從七品下;丞各一人,從八品下。掌守衛山陵。凡陪葬,以文武分左右,子孫從父祖者亦如之;宮人陪葬,則陵戶成墳。諸陵四至有封,禁民葬,唯故墳不毀。開元二十四年,以宗廟所奉不可名以署,太常少卿韋縚奏廢太廟署,以少卿一人知太廟事。二十五年,濮陽王徹為宗正卿,恩遇甚厚,建議以宗正司屬籍,乃請以陵寢、宗廟隸宗正。天寶十二載,駙馬都尉張垍為太常卿,得幸,又以太廟諸陵署隸太常。十載,改獻、昭、乾、定、橋五陵署為臺,升令品,永康、興寧二陵稱署如故。至德二年,復以陵廟隸宗正。永泰元年,太常卿姜慶初復奏以陵廟隸太常,大歷二年復舊。陵臺有錄事各一人,府各二人,史各四人,主衣、主輦、主藥各四人,典事各三人,掌固各二人,陵戶各三百人,昭陵、乾陵、橋陵增百人。諸陵有錄事各一人,府各一人,史各二人,典事各二人,掌固各二人,陵戶各百人。



 △諸太子廟



 令各一人,從八品上;丞各一人,正九品下;錄事各一人。令掌灑掃開闔之節,四時享祭焉。有府各一人,史各二人,典事各二人,掌固各一人。



 △諸太子陵



 令各一人,從八品下;丞各一人,從九品下;錄事各一人。有府各一人,史各二人,典事各二人,掌固各一人,陵戶各三十人。太常舊有太廟署,令一人,從七品下;丞二人,從七品下;齋郎二十四人。



 △崇玄署



 令一人,正八品下;丞一人,正九品下。掌京都諸觀名數與道士帳籍、齋醮之事。新羅、日本僧入朝學問,九年不還者編諸籍。道士、女官、僧、尼,見天子必拜。凡止民家,不過三夜。出逾宿者,立案連署,不過七日,路遠者州縣給程。天下觀一千六百八十七,道士七百七十六,女官九百八十八;寺五千三百五十八,僧七萬五千五百二十四,尼五萬五百七十六。兩京度僧、尼、道士、女官,御史一人涖之。每三歲州、縣為籍,一以留縣,一以留州;僧、尼,一以上祠部,道士、女官,一以上宗正,一以上司封。有府二人,史三人,典事六人,掌固二人,崇玄學博士一人、學生百人。隋以署隸鴻臚,又有道場、玄壇。唐置諸寺觀監,隸鴻臚寺,每寺觀有監一人。貞觀中,廢寺觀監。上元二年,置漆園監,尋廢。開元二十五年,置崇玄學於玄元皇帝廟。天寶元年,兩京置博士、助教各一員,學生百人,每祠享,以學生代齋郎。二載,改崇玄學曰崇玄館,博士曰學士,助教曰直學士,置大學士一人,以宰相為之,領兩京玄元宮及道院,改天下崇玄學為通道學,博士曰道德博士,未幾而罷。寶應、永泰間,學生存者亡幾。大歷三年,復增至百人。初,天下僧、尼、道士、女官,皆隸鴻臚寺,武後延載元年,以僧、尼隸祠部。開元二十四年,道士、女官隸宗正寺,天寶二載,以道士隸司封。貞元四年,崇玄館罷大學士,後復置左右街大功德使、東都功德使、修功德使,總僧、尼之籍及功役。元和二年,以道士、女官隸左右街功德使。會昌二年,以僧、尼隸主客,太清宮置玄元館,亦有學士,至六年廢,而僧、尼復隸兩街功德使。



 ○太僕寺



 卿一人,從三品;少卿二人,從四品上;丞四人,從六品上;主簿二人,從七品上;錄事二人。卿掌廄牧、輦輿之政,總乘黃、典廄、典牧、車府四署及諸監牧。行幸,供五路屬車。凡監牧籍帳,歲受而會之,上駕部以議考課。永徽中,太僕寺曰司馭寺,武後光宅元年改曰司僕寺。有府十七人,史三十四人,獸醫六百人,獸醫博士四人,學生百人,亭長四人,掌固六人。



 △乘黃署



 令一人,從七品下;丞一人,從八品下。掌供車路及馴馭之法。凡有事,前期四十日,率駕士調習,尚乘隨路色供馬;前期二十日,調習於內侍省。有府一人,史二人,駕士一百四十人,羊車小史十四人,掌固六人。



 △典廄署



 令二人,從七品下;丞四人,從八品下。掌飼馬牛、給養雜畜。良馬一丁,中馬二丁,駑馬三丁,乳駒、乳犢十給一丁。有府四人,史八人,主乘六人,典事八人,執馭百人,駕士八百人,掌固六人。



 △典牧署



 令三人,正八品上;丞六人,從九品上。掌諸牧雜畜給納及酥酪脯臘之事。群牧所送羊犢,以供廩犧、尚食。監事八人。有府四人,史八人,典事十六人,主酪七十四人,駕士百六十人,掌固四人。



 △車府署



 令一人,正八品下;丞一人,正九品下。掌王公以下車路及馴馭之法。從官三品以上婚、葬,給駕士。凡路車之馬牛,率馭士調習。有府一人,史二人,典書四人,馭士百七十五人,掌固六人。



 △諸牧監



 上牧監:監各一人,從五品下;副監各二人,正六品下;丞各二人,正八品上;主簿各一人,正九品下。中牧監:監,正六品下;副監,從六品下;丞,從八品上;主簿,從九品上。下牧監:監,從六品下;副監,正七品下;丞,正九品上;主簿,從九品下。中牧監副監、丞,減上牧監一員。南使、西使,丞各三人,從七品下;錄事各一人,從九品下。北使、鹽州使,丞各二人,從七品下。掌群牧孳課。凡馬五千為上監,三千為中監,不及為下監。馬牛之群,有牧長,有尉。馬之駑、良,皆著籍,良馬稱左,駑馬稱右。每歲孟秋,群牧使以諸監之籍合為一,以仲秋上於寺,送細馬,則有牽夫、識馬小兒、獸醫等。凡馬游牝以三月,駒犢在牧者,三歲別群。孳生過分有賞,死耗亦以率除之。歲終監牧使巡按,以功過相除為考課。上牧監,有錄事各一人,府各三人,史各六人,典事各八人,掌固各四人。中牧監,減府一人,史、典事各減三人。下牧監,典事、掌固減二人。南使、西使,錄事、史各一人,府各五人,史各九人;北使、鹽州使,錄事以下員數及品,如南使。麟德中,置八使,分總監坊。秦、蘭、原、渭四州及河曲之地。凡監四十有八:南使有監十五,西使有監十六,北使有監七,鹽州使有監八,嵐州使有監二。自京師西屬隴右,有七馬坊,置隴右三使領之。又有沙苑、樓煩、天馬監。沙苑監掌畜隴右諸牧牛羊,給宴祭及尚食所用,每歲與典牧署供焉。自監以下,品數如下牧監。至開元二十三年,廢監。



 △東宮九牧監



 丞二人,正八品上;錄事一人,從九品下。掌牧養馬牛,以供皇太子之用。有錄事史各一人,府三人,史六人。初,監有監、副監、丞、主簿、錄事各一人,府二人,史四人,典事四人,掌固二人。自監以下,品同下牧監。又有馬牧使,有丞以下官。



 ○大理寺



 卿一人,從三品;少卿二人,從五品下。掌折獄、詳刑。凡罪抵流、死,皆上刑部,覆於中書、門下。系者五日一慮。



 龍朔二年,改曰詳刑寺;武後光宅元年,改曰司刑寺;中宗時廢獄丞。有府二十八人,史五十六人,司直史十二人,評事史二十四人,獄史六人,亭長四人,掌固十八人,問事百人。



 正二人,從五品下。掌議獄,正科條。凡丞斷罪不當,則以法正之。五品以上論者,蒞決。巡幸則留總持寺事。



 丞六人,從六品上。掌分判寺事,正刑之輕重。徒以上囚,則呼與家屬告罪,問其服否。



 主簿二人,從七品上。掌印,省署鈔目,句檢稽失。凡官吏抵罪及雪免,皆立簿。私罪贖銅一斤,公罪二斤,皆為一負;十負為一殿。每歲吏部、兵部牒覆選人殿負,錄報焉。



 獄丞二人,從九品下。掌率獄史,知囚徒。貴賤、男女異獄。五品以上月一沐,暑則置漿。禁紙筆、金刃、錢物、杵梃入者。囚病,給醫藥,重者脫械鎖,家人入侍。



 司直六人,從六品上;評事八人,從八品下。掌出使推按。凡承制推訊長史,當停務禁錮者,請魚書以往。錄事二人。



 ○鴻臚寺



 卿一人,從三品;少卿二人,從四品上;丞二人,從六品上。掌賓客及兇儀之事。領典客、司儀二署。凡四夷君長,以蕃望高下為簿,朝見辨其等位,第三等居武官三品之下,第四等居五品之下,第五等居六品之下,有官者居本班。御史察食料。二王後、夷狄君長襲官爵者,辨嫡庶。諸蕃封命,則受冊而往。海外諸蕃朝賀進貢使有下從,留其半於境;繇海路朝者,廣州擇首領一人、左右二人入朝;所獻之物,先上其數於鴻臚。凡客還,鴻臚籍衣齎賜物多少以報主客,給過所。蕃客奏事,具至日月及所奏之宜,方別為狀,月一奏,為簿,以副藏鴻臚。獻馬,則殿中、太僕寺涖閱,良者入殿中,駑病入太僕。獻藥者,鴻臚寺驗覆,少府監定價之高下。鷹、鶻、狗、豹無估,則鴻臚定所報輕重。凡獻物,皆客執以見,駝馬則陳於朝堂,不足進者州縣留之。皇帝、皇太子為五服親及大臣發哀臨吊,則卿贊相。大臣一品葬,以卿護;二品,以少卿;三品,以丞。皆司儀示以禮制。主簿一人,從七品上。錄事二人。龍朔二年,改鴻臚寺曰同文寺,武後光宅元年,改曰司賓寺。有府五人,史十人,亭長四人,掌固六人。



 △典客署



 令一人,從七品下;丞三人,從八品下。掌二王後介公、酅公之版籍及四夷歸化在籓者,朝貢、宴享、送迎皆預焉。酋渠首領朝見者,給稟食;病,則遣醫給湯藥;喪,則給以所須;還蕃賜物,則佐其受領,教拜謝之節。有典客十三人,府四人,史八人,掌固二人。



 掌客十五人,正九品上。掌送迎蕃客,顓蒞館舍。



 △司儀署



 令一人,正八品下;丞一人,正九品下。掌兇禮喪葬之具。京官職事三品以上、散官二品以上祖父母、父母喪,職事散官五品以上、都督、刺史卒於京師,及五品死王事者,將葬,祭以少牢,率齋郎執俎豆以往。三品以上贈以束帛,黑一、纁二,一品加乘馬;既引,遣使贈於郭門之外,皆有束帛,一品加璧。五品以上葬,給營墓夫。有司儀六人,府二人,史四人,掌設十八人,齋郎三十人,掌固四人,幕士六十人。



 ○司農寺



 卿一人,從三品;少卿二人,從四品上。掌倉儲委積之事。總上林、太倉、鉤盾、霡官四署及諸倉、司竹、諸湯、宮苑、鹽池、諸屯等監。凡京都百司官吏祿稟、朝會、蔡祀所須,皆供焉。藉田,則進耒耜。



 丞六人,從六品上。總判寺事。凡租及槁秸至京都者,閱而納焉。官戶奴婢有技能者配諸司,婦人入掖庭,以類相偶,行宮監牧及賜王公、公主皆取之。凡孳生雞彘,以戶奴婢課養。俘口則配輕使,始至給稟食。主簿二人,從七品上;錄事二人。龍朔二年,改司農寺曰司稼寺。有府三十八人,史七十六人,計史三人,亭長九人,掌固七人。



 △上林署



 令二人,從七品下;丞四人,從八品下。掌苑囿園池。植果蔬,以供朝會、祭祀及尚食諸司常料。季冬,藏冰千段,先立春三日納之冰井,以黑牡、秬黍祭司寒,仲春啟冰亦如之。監事十人。有府七人,史十四人,典事二十四人,掌固五人。



 △太倉署



 令三人,從七品下;丞五人,從八品下;監事八人。掌廩藏之事。有府十人,史二十人,典事二十四人,掌固八人。



 △鉤盾署



 令二人,正八品上;丞四人,正九品上;監事十人,掌供薪炭、鵝鴨、蒲藺、陂池藪澤之物,以給祭祀、朝會、饗燕賓客。有府七人,史十四人,典事十九人,掌固五人。


△
 \gezhu{
  道禾}
 官署


令二人,正八品下;丞四人,正九品上;監事十人。掌
 \gezhu{
  道禾}
 擇米麥。凡九穀,皆隨精粗差其耗損而供焉。有府八人,史十六人,典事二十四人,掌固五人。初有御細倉督、麴面倉督,貞觀中省。



 △太原、永豐、龍門等倉



 每倉監一人,正七品下;丞二人,從八品上。掌倉廩儲積。凡出納帳籍,歲終上寺。有錄事一人,府三人,史六人,典事八人,掌固六人;龍門等倉,減府一人,史、典事、掌固各減二人。



 △司竹



 監一人,從六品下;副監一人,正七品下;丞二人,正八品上。掌植竹、葦,供宮中百司簾篚之屬,歲以筍供尚食。有錄事一人,府二人,史四人,典事三十人,掌固四人,葦園匠一百人。



 △慶善、石門、溫泉湯等監



 每監監一人,從六品下;丞一人,正七品下。掌湯池、宮禁、防堰及偫粟芻、脩調度,以備供奉。王公以下湯館,視貴賤為差。凡近湯所潤瓜蔬,先時而熟者,以薦陵廟。有錄事一人,府一人,史二人,掌固四人。



 △京都諸宮苑總監



 監各一人,從五品下;副監各一人,從六品下;丞各二人,從七品下;主簿各二人,從九品上。掌苑內宮館、園池、禽魚、果木。凡官屬人畜出入,皆有籍。有錄事各二人,府各八人,史各十六人,亭長各四人,掌固各六人,獸醫各五人。



 △京都諸園苑監、苑四面監



 監各一人,從六品下;副監各一人,從七品下;丞各二人,正八品下。掌完葺苑面、宮館、園池與種蒔、蕃養六畜之事。顯慶二年,改青城宮監曰東都苑北面監,明德宮監曰東都苑南面監,洛陽宮農圃監曰東都苑東面監,倉貨監曰東都苑西面監。有錄事各一人,府各三人,史各六人,典事各六人,掌固各六人。



 △九成宮總監



 監一人,從五品下;副監一人,從六品下;丞一人,從七品下;主簿一人,從九品上。掌脩完宮苑,供進煉餌之事。有錄事一人,府三人,自監以下,品同宮苑。武德初,改隋仁壽宮監曰九成宮監。



 △諸鹽池監



 監一人,正七品下,掌鹽功簿帳。有錄事一人,史二人。



 △諸屯



 監一人,從七品下;丞一人,從八品下。掌營種屯田,句會功課及畜產簿帳,以水旱蝝蝗定課。屯主勸率營農,督斂地課。有錄事一人,府一人,史二人,典事二人,掌固四人。每屯主一人,屯副一人,主簿一人,錄事一人,府三人,史五人。



 ○太府寺



 卿一人,從三品;少卿二人,從四品上。掌財貨、廩藏、貿易,總京都四市、左右藏、常平七署。凡四方貢賦、百官俸秩,謹其出納。賦物任土所出,定精粗之差,祭祀幣帛皆供焉。龍朔二年,改太府寺曰外府寺。武後光宅元年,改曰司府寺。中宗即位,復曰太府寺。有府二十五人,史五十人,計史四人,亭長七人,掌固七人。



 丞四人,從六品上。掌判寺事。凡元日、冬至以方物陳於庭者,受而進之。會賜及別敕六品以下賜者,給於朝堂。以一人主左、右藏署帳,凡在署為簿,在寺為帳,三月一報金部。



 主簿二人,從七品上。掌印,省鈔目,句檢稽失,平權衡度量,歲以八月印署,然後用之。錄事二人。



 △兩京諸市署



 令一人,從六品上;丞二人,正八品上。掌財貨交易、度量器物,辨其真偽輕重。市肆皆建標築土為候,禁榷固及參市自殖者。凡市,日中擊鼓三百以會眾,日入前七刻,擊鉦三百而散。有果毅巡。平貨物為三等之直,十日為簿。車駕行幸,則立市於頓側互市,有衛士五十人,以察非常。有錄事一人,府三人,史七人,典事三人,掌固一人。



 △左藏署



 令三人,從七品下;丞五人,從八品下;監事八人。掌錢帛、雜彩。天下賦調,卿及御史監閱。有府九人,史十八人,典事十二人,掌固八人。



 △右藏署



 令二人,正八品上;丞三人,正九品上;監事四人。掌金玉、珠寶、銅鐵、骨角、齒毛、彩畫。有府五人,史十二人,典事七人,掌固十人。



 △常平署



 令一人,從七品上;丞二人,從八品下;監事五人。掌平糴、倉儲、出納。有府四人,史八人,典事五人,掌固六人。顯慶三年,置署。武后時,東都亦置署。



 ○國子監



 祭酒一人,從三品;司業二人,從四品下。掌儒學訓導之政,總國子、太學、廣文、四門、律、書、算凡七學。天子視學,皇太子齒胄,則講義。釋奠,執經論議,奏京文武七品以上觀禮。凡授經,以《周易》、《尚書》、《周禮》、《儀禮》、《禮記》、《毛詩》、《春秋左氏傳》、《公羊傳》、《穀梁傳》各為一經,兼習《孝經》、《論語》、《老子》,歲終,考學官訓導多少為殿最。



 丞一人,從六品下,掌判監事。每歲,七學生業成,與司業、祭酒蒞試,登第者上於禮部。



 主簿一人,從七品下。掌印,句督監事。七學生不率教者,舉而免之。錄事一人,從九品下。武德初,以國子監曰國子學,隸太常寺,貞觀二年復曰監。龍朔二年,改國子監曰司成館,祭酒曰大司成,司業曰少司成。咸亨元年復曰監。垂拱元年,改國子監曰成均監。有府七人,史十三人,亭長六人,掌固八人。



 △國子學



 博士五人,正五品上。掌教三品以上及國公子孫、從二品以上曾孫為生者。五分其經以為業:《周禮》、《儀禮》、《禮記》、《毛詩》、《春秋左氏傳》各六十人,暇則習隸書、《國語》、《說文》、《字林》、《三倉》、《爾雅》。每歲通兩經。求仕者,上於監;秀才、進士亦如之。學生以長幼為序,習正業之外,教吉、兇二禮,公私有事則相儀。龍朔二年,改博士曰宣業。有大成十人,學生八十人,典學四人,廟幹二人,掌固四人,東都學生十五人。



 助教五人,從六品上。掌佐博士分經教授。



 直講四人,掌佐博士、助教以經術講授。



 五經博士各二人,正五品上。掌以其經之學教國子。《周易》、《尚書》、《毛詩》、《左氏春秋》、《禮記》為五經,《論語》、《孝經》、《爾雅》不立學官,附中經而已。



 △太學



 博士六人,正六品上;助教六人,從七品上。掌教五品以上及郡縣公子孫、從三品曾孫為生者,五分其經以為業,每經百人。有學生七十人,典學四人,掌固六人,東都學生十五人。



 △廣文館



 博士四人,助教二人。掌領國子學生業進士者。有學生六十人,東都十人。天寶九載,置廣文館,有知進士助教,後罷知進士之名。



 △四門館



 博士六人,正七品上;助教六人,從八品上;直講四人。掌教七品以上、侯伯子男子為生及庶人子為俊士生者。有學生三百人,典學四人,掌固六人;東都學生五十人。



 △律學



 博士三人,從八品下;助教一人,從九品下。掌教八品以下及庶人子為生者。律令為顓業,兼習格式法例。隋,律學隸大理寺,博士八人。武德初,隸國子監,尋廢;貞觀六年復置,顯慶三年又廢,以博士以下隸大理寺;龍朔二年復置。有學生二十人,典學二人。元和初,東都置學生五人。



 △書學



 博士二人,從九品下;助教一人。掌教八品以下及庶人子為生者。石經、《說文》、《字林》為顓業,兼習餘書。武德初,廢書學,貞觀二年復置,顯慶三年又廢,以博士以下隸秘書省,龍朔二年復。有學生十人,典學二人,東都學生三人。



 △算學



 博士二人,從九品下;助教一人。掌教八品以下及庶人子為生者。二分其經以為業:《九章》、《海島》、《孫子》、《五曹》、《張丘建》、《夏侯陽》、《周髀》、《五經算》、《綴術》、《緝古》為顓業,兼習《記遺》、《三等數》。凡六學束脩之禮、督課、試舉,皆如國子學;助教以下所掌亦如之。唐廢算學,顯慶元年復置,三年又廢,以博士以下隸太史局。龍朔二年復。有學生十人,典學一人,東都學生二人。



 ○少府



 監一人,從三品;少監二人,從四品下。掌百工技巧之政。總中尚、左尚、右尚、織染、掌冶五署及諸冶、鑄錢、互市等監。供天子器御、后妃服飾及郊廟圭玉、百官儀物。凡武庫袍襦,皆識其輕重乃藏之,冬至、元日以給衛士。諸州市牛皮、角以供用,牧畜角、筋、腦、革悉輸焉。細鏤之工,教以四年;車路樂器之工三年;平漫刀槊之工二年,矢鏃竹漆屈柳之工半焉;冠冕弁幘之工九月。教作者傳家技,四季以令丞試之,歲終以監試之,皆物勒工名。丞六人,從六品下。掌判監事。給五署所須金石、齒革、羽毛、竹木,所入之物,各以名數州土為籍。工役眾寡難易有等差,而均其勞逸。主簿二人,從七品下;錄事二人,從九品上。武德初,廢監,以諸署隸太府寺。貞觀元年復置。龍朔二年改曰內府監,武后垂拱元年曰尚方監。有府二十七人,史十七人,計史三人,亭長八人,掌固六人,短蕃匠五千二十九人,綾綿坊巧兒三百六十五人,內作使綾匠八十三人,掖庭綾匠百五十人,內作巧兒四十二人,配京都諸司諸使雜匠百二十五人。



 △中尚署



 令一人,從七品下;丞二人,從八品下。掌供郊祀圭璧及天子器玩、后妃服飾雕文錯彩之制。凡金木齒革羽毛,任土以時而供。赦日,樹金雞於仗南,竿長七丈,有雞高四尺,黃金飾首,銜絳幡長七尺,承以彩盤,維以絳繩,將作監供焉。擊㧏鼓千聲,集百官、父老、囚徒。坊小兒得雞首者官以錢購,或取絳幡而已。歲二月,獻牙尺;寒食,獻;五月,獻綬帶;夏至,獻雷車;七月,獻鈿針;臘日,獻口脂;唯筆、琴瑟弦月獻;金銀暨紙非旨不獻。制魚袋以給百官;蕃客賜寶鈿帶魚袋,則授鴻臚寺丞、主簿。監作四人,從九品下。凡監作,皆同品。有府九人,史十八人,典事四人,掌固四人。唐改內尚方署曰中尚方署。武後改少府監曰尚方監,而中左右尚方、織染方、掌冶方五署,皆去方以避監。自是不改矣。有金銀作坊院。



 △左尚署



 令一人,從七品下;丞五人,從八品下。掌供翟扇、蓋繖、五路、五副、七輦、十二車,及皇太后、皇太子、公主、王妃、內外命婦、王公之車路。凡畫素刻鏤與宮中蠟炬雜作,皆領之。監作六人。有府七人,史二十人,典事十八人,掌固十四人。



 △右尚署



 令二人,從七品下;丞四人,從八品下。掌供十二閑馬之轡。每歲取於京兆、河南府,加飾乃進。凡五品三部之帳,刀劍、斧鉞、甲胄、紙筆、茵席、履舄,皆儗其用,皮毛之工亦領焉。監作六人。有府七人,史二十人,典事十三人,掌固十人。



 △織染署



 令一人,正八品上;丞二人,正九品上。掌供冠冕、組綬及織糸任、色染。錦、羅、紗、縠、綾、紬、施、絹、布,皆廣尺有八寸,四丈為匹。布五丈為端,綿六兩為屯,絲五兩為絇,麻三斤為綟。凡綾錦文織,禁示於外。高品一人專蒞之,歲奏用度及所織。每掖庭經錦,則給酒羊。七月七日,祭杼。監作六人。有府六人,史十四人,典事十一人,掌固五人。



 △掌冶署



 令一人,正八品上;丞二人,正九品上。掌範鎔金銀銅鐵及塗飾琉璃玉作。銅鐵人得採,而官收以稅,唯鑞官市。邊州不置鐵冶,器用所須,皆官供。凡諸冶成器,上數於少府監,然後給之。監作二人。有府六人,史十二人,典事二十三人,掌固四人。



 △諸冶監



 令各一人,正七品下;丞各一人,從八品上。掌鑄兵農之器,給軍士、屯田居民,唯興農冶顓供隴右監牧。監作四人。有錄事一人,府一人,史二人,典事二人,掌固四人。太原冶,減監作二人。



 △諸鑄錢監



 監各一人,副監各二人,丞各一人。以所在都督、刺史判焉;副監,上佐;丞,以判司;監事以參軍及縣尉為之。監事各一人。有錄事各一人,府各三人,史各四人,典事各五人。凡鑄錢有七監,會昌中,增至八監,每道置鑄錢坊一。大中初,三監廢。



 △互市監



 每監監一人,從六品下;丞一人,正八品下。掌蕃國交易之事。隋以監隸四方館。唐隸少府。貞觀六年,改交市監曰互市監,副監曰丞,武后垂拱元年曰通市監。有錄事一人,府二人,史四人,價人四人,掌固八人。



 ○將作監



 監二人,從三品;少監二人,從四品下。掌土木工匠之政,總左校、右校、中校、甄官等署,百工等監。大明、興慶、上陽宮,中書、門下、六軍仗舍、閑廄,謂之內作;郊廟、城門、省、寺、臺、監、十六衛、東宮、王府諸廨,謂之外作。自十月距二月,休冶功;自冬至距九月,休土功。凡治宮廟,太常擇日以聞。



 丞四人,從六品下。掌判監事。凡外營繕、大事則聽制敕,小事則須省符。功有長短,役有輕重。自四月距七月,為長功;二月、三月、八月、九月,為中功;自十月距正月,為短功。長上匠,州率資錢以酬雇。軍器則勒歲月與工姓名。武德初,改令曰大匠,少令曰少匠。龍朔二年,改將作監曰繕工監,大匠曰大監,少匠曰少監。咸亨元年,繕工監曰營繕監。天寶十一載,改大匠曰大監,少匠曰少監。有府十四人,史二十八人,計史三人,亭長四人,掌固六人,短蕃匠一萬二千七百四十四人,明資匠二百六十人。



 主簿二人,從七品下。掌官吏糧料、俸食,假使必由之。諸司供署監物有闕,舉焉。錄事二人,從九品上。



 △左校署



 令二人,從八品下;丞一人,正九品下。掌梓匠之事。樂縣、簨弶、兵械、喪葬儀物皆供焉。宮室之制,自天子至士庶有等差,官脩者左校為之。監作十人。有府六人,史十二人,監作十二人。



 △右校署



 令二人,正八品下;丞三人,正九品下。掌版築、塗泥、丹堊、匽廁之事。有所須,則審其多少而市之。監作十人。有府五人,史十人,典事二十四人。



 △中校署



 令一人,從八品下;丞三人,正九品下。掌供舟軍、兵械、雜器。行幸陳設則供竿柱,閑廄系秣則供行槽,禱祀則供棘葛,內外營作所須皆取焉。監牧車牛,有年支芻豆,則受之以給車坊。監事四人。武后時,改曰營繕署。垂拱元年復舊,尋廢。開元初復置。有府二人,史六人,典事八人,掌固二人。



 △甄官署



 令一人,從八品下;丞二人,正九品下。掌琢石、陶土之事,供石磬、人、獸、碑、柱、碾、磑、瓶、缶之器,敕葬則供明器。監作四人。有府五人,史十人,典事十八人。



 百工、就穀、庫谷、斜谷、太陰、伊陽監,監各一人,正七品下;副監一人,從七品下;丞一人,正八品上。掌採伐材木。監作四人。武德初,置百工監,掌舟車及營造雜作,有監、少監各一人,丞四人,主簿一人。又置就穀、庫谷、斜谷、太陰、伊陽五監。貞觀中,廢百工監。高宗置百工署,掌東都土木瓦石之功。開元十五年為監。有錄事一人,府一人,史三人,典事二十人。



 ○軍器監



 監一人,正四品上;丞一人,正七品上。掌繕甲弩,以時輸武庫。總署二:一曰弩坊,二曰甲坊。主簿一人,正八品下;錄事一人,從九品下。武德初,有武器監一人,正八品下。掌兵杖、廄牧。少監一人,丞二人,主簿一人。七年廢軍器監,八年復置,九年又廢。貞觀六年,廢武器監。開元以前,軍器皆出左尚署,三年置軍器監,十一年復廢為甲弩坊,隸少府,十六年復為監。有府八人,史十二人,亭長二人,掌固四人。



 △弩坊署



 令一人,正八品下;丞一人,正九品下。掌出納矛槊、弓矢、排弩、刃鏃、雜作及工匠。監作二人。有府二人,史五人,典事二人。貞觀六年,改弓弩署為弩坊署,甲鎧署為甲坊署。



 △甲坊署



 令一人,正八品下;丞一人,正九品下。掌出納甲胄、■繩、筋角、雜作及工匠。監作二人。有府二人,史五人,典事二人。



 ○都水監



 使者二人,正五品上。掌川澤、津梁、渠堰、陂池之政,總河渠、諸津監署。凡漁捕有禁,溉田自遠始,先稻後陸,渠長、斗門長節其多少而均焉。府縣以官督察。丞二人,從七品上。掌判監事。凡京畿諸水,因灌溉盜費者有禁。水入內之餘,則均王公百官。主簿一人,眾八品下。掌運漕、漁捕程,會而糾舉之。武德初,廢都水監為署。貞觀六年復為監,改令曰使者。龍朔二年,改都水監曰司津監,使者曰監。武后垂拱元年,改都水監曰水衡監,使者曰都尉。開元二十五年,不隸將作監。有錄事一人,府五人,史十人,亭長一人,掌固四人。初,貞觀六年,置舟楫署,有令一人,正八品下,掌舟楫、運漕;漕正一人,府三人,史六人,監漕一人,漕史一人,典事六人,掌固八人。上元二年,置丞二人,正九品下,掌運漕隱失。開元二十六年,署廢。



 △河渠署



 令一人,正八品下;丞一人,正九品上。掌河渠、陂池、堤堰、魚醢之事。凡溝渠開塞,漁捕時禁,皆顓之。饗宗廟,則供魚鮍;祀昊天上帝,有司攝事,則供腥魚。日供尚食及給中書、門下,歲供諸司及東宮之冬藏。渭河三百里內漁釣者,五坊捕治之。供祠祀,則自便橋至東渭橋禁民漁。三元日,非供祠不採魚。唐有河堤使者。貞觀初改曰河堤謁者。有府三人,史六人,典事三人,每渠及斗門有長一人,掌固三人,魚師十二人。初,有監漕十人,從九品上,大歷後省。興成、五門、六門、龍首、涇堰、滋堤,凡六堰,皆有丞一人,從九品下。府一人,史二人,典事二人,掌固二人。貞觀六年皆廢。



 河堤謁者六人,正八品下。掌完堤堰、利溝瀆、漁捕之事。涇、渭、白渠,以京兆少尹一人督視。



 △諸津



 令各一人,正九品上;丞二人,從九品下。掌天下津濟舟梁。灞橋、永濟橋,以勛官散官一人蒞之;天津橋、中橋,則以衛士拚掃。凡舟渠之備,皆先儗其半,袽塞、竹,所在供焉。唐改津尉曰令,有錄事一人,府一人,史二人,典事三人,津吏五人,橋丁各三十人,匠各八人。京兆、河南諸津,隸都水監;便橋、渭橋、萬年三橋,有丞一人,從九品下;府一人,史十人,典事二人,掌固二人。貞觀中廢。



\end{pinyinscope}