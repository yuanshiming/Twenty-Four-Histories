\article{列傳第四十二 裴劉魏李吉}

\begin{pinyinscope}

 裴炎,字子隆,絳州聞喜人。寬厚,寡言笑,有奇節。補弘文生資本論馬克思的主要著作。寫於19世紀40—80年代。共,休澣,它生或出游,炎讀書不廢。有司欲薦狀,以業未就,辭不舉,服勤十年,尤通《左氏春秋》。舉明經及第。補濮州司倉參軍,歷御史、起居舍人,浸遷黃門侍郎。調露二年,同中書門下三品。進拜侍中。高宗幸東都,留皇太子京師,以炎調護。帝不豫,太子監國,詔炎與劉齊賢、郭正一於東宮平章政事,及大漸,受遺輔太子,是為中宗。改中書令。舊,宰相議事門下省,號政事堂,長孫無忌以司空、房玄齡以僕射、魏徵以太子太師皆知門下省事,至炎,以中書令執政事筆,故徙政事堂於中書省。



 中宗欲以後父韋玄貞為侍中及授乳媼子五品官,炎固執不從,帝怒曰:「我意讓國與玄貞,豈不可?何惜侍中邪?」炎懼,因與武后謀廢帝。後命炎洎劉禕之率羽林將軍程務挺、張虔勖勒兵入宮,宣太后令,扶帝下殿,帝曰:「我何罪?」後曰:「以天下與玄貞,安得無罪?」乃廢帝為盧陵王,更立豫王為皇帝。以定策功,封永清縣男。



 後已持政,稍自肆,於是武承嗣請立七廟,追王其先,炎諫曰:「太后天下母,以盛德臨朝,宜存至公,不容追王祖考,示自私。且獨不見呂氏事乎!」後曰:「呂氏之王,權屬生人,今追崇先世,在亡跡異,安得同哉!」炎曰:「蔓草難圖,漸不可長。」後不悅而罷。承嗣又諷太后誅韓王元嘉、魯王靈夔,以絕宗室望,劉禕之、韋仁約畏默不敢言,炎獨固爭,後愈銜怒。未幾,賜爵河東縣侯。



 豫王雖為帝,未嘗省天下事。炎謀乘太后出游龍門,以兵執之,還政天子。會久雨,太后不出而止。徐敬業兵興,後議討之,炎曰:「天子年長矣,不豫政,故豎子有辭。今若復子明闢,賊不討而解。」御史崔詧曰:「炎受顧托,身總大權,聞亂不討,乃請太后歸政,此必有異圖。」後乃捕炎送詔獄,遣御史大夫騫味道、御史魚承曄參鞫之。鳳閣侍郎胡元範曰:「炎社稷臣,有功於國,悉心事上,天下所知,臣明其不反。」納言劉齊賢、左衛率蔣儼繼辨之,後曰:「炎反有端,顧卿未知耳。」元範、齊賢曰:「若炎反,臣輩亦反矣。」後曰:「朕知炎反,卿輩不反。」遂斬於都亭驛。



 炎被劾,或勉其遜辭,炎曰:「宰相下獄,理不可全。」卒不折節,籍其家,無儋石之贏。初,炎見裴行儉破突厥有功,沮薄之,乃斬降虜阿史那伏念等五十餘人,議者恨其媢克,且使國家失信四夷,以為陰禍有知雲。睿宗立,贈太尉、益州大都督,謚曰忠。



 元範者,申州義陽人。介廉有才,以炎故,流死巂州。



 炎從子伷先。伷先未冠,推廕為太僕丞。炎死,坐流嶺南。上變求面陳得失,後召見,盛氣待之,曰:「炎謀反,法當誅,尚何道?」伷先對曰:「陛下唐家婦,身荷先帝顧命,今雖臨朝,當責任大臣,須東宮年就德成,復子明闢,奈何遽王諸武、斥宗室?炎為唐忠臣,而戮逮子孫,海內憤怨。臣愚謂陛下宜還太子東宮,罷諸武權。不然,豪桀乘時而動,不可不懼!」後怒,命曳出,杖之朝堂,長流瀼州。



 歲餘,逃歸,為吏跡捕,流北庭。無復名檢,專居賄,五年至數千萬。娶降胡女為妻,妻有黃金、駿馬、牛羊,以財自雄。養客數百人。自北庭屬京師,多其客,候朝廷事,聞知十常七八。時補闕李秦授為武后謀曰:「讖言『代武者劉』,劉無強姓,殆流人乎?今大臣流放者數萬族,使之葉亂,社稷憂也。」後謂然,夜拜秦授考功員外郎,分走使者,賜墨詔,慰安流人,實命殺之。伷先前知,以橐駝載金幣、賓客奔突厥。行未遠,都護遣兵追之,與格鬥,為所執。械系獄,以狀聞。會武後度流人已誅,畏天下姍誚,更遣使者安撫十道,以好言自解釋曰:「前使使慰安有罪,而不曉朕意,擅誅殺,殘忍不道,朕甚自咎。今流人存者一切縱還。」繇是伷先得不死。



 中宗復位,求炎後,授先太子詹事丞。遷秦、桂、廣三州都督。坐累且誅,賴宰相張說右之,免官。久乃擢範陽節度使,太原、京兆尹。以京師官冗,奏罷畿縣員外及試官。進工部尚書。年八十六,以東京留守累封翼城縣公,卒官下。



 劉禕之,字希美,常州晉陵人。父子翼,字小心,在隋為著作郎。峭直有行,嘗面折僚友短,退無餘訾。李伯藥曰;「子翼詈人,人都不憾。」貞觀初,召之,辭以母老,詔許終養。江南道巡察使李襲譽嘉其孝,表所居為孝慈里。母已喪,召拜吳王府功曹參軍,終著作郎、弘文館直學士。



 禕之少與孟利貞、高智周、郭正一俱以文辭稱,號「劉孟高郭」,並直昭文館。俄遷右史、弘文館直學士。上元中,與元萬頃等偕召入禁中,論次新書凡千餘篇。高宗又密與參決時政,以分宰相權,時謂「北門學士」。兄懿之,亦給事中,同兩省。先是,姊為內官,武后遣至外家問疾,禕之因賀蘭敏之私省之,坐流巂州。後為丐還,除中書舍人。



 儀鳳中,吐蕃寇邊,帝訪侍臣所以置之、討之之宜,人人異謀,示之獨勸帝:「夷狄猶禽獸,雖被馮陵,不足校,願戢威,紓百姓之急。」帝內其言。俄拜相王府司馬。檢校中書侍郎,帝謂曰:「卿家忠孝,朕子賴卿以師矩,冀蓬在麻不扶而挺也。」



 後既立王為帝,以其參奉大議,愈親之,擢中書侍郎、同中書門下三品,賜爵臨淮縣男。方是時,詔令叢繁,禕之思致華敏,裁可占授,少選可待也。司門員外郎房先敏坐累貶衛州司馬,訴於相府,內史騫味道謂曰:「太后旨。」禕之曰:「乃上從有司所奏云。」後聞,以味道歸非於上,貶青州刺史,加禕之太中大夫,賜物百段。後因曰:「君為元首,臣為股肱,以手足疾移於腹背,尚為一體乎?示之引咎於已,忠臣也。」納言王德真推順曰:「戴至德無異才,惟能歸善於君,為時所服。」後曰:「善。」後私語鳳閣舍人賈大隱曰:「後能廢昏立明,盍反政以安天下?」大隱表其言,後怒曰:「禕之乃負我!」垂拱中,或告禕之,受歸誠州都督孫萬榮金,與許敬宗妾私通,太后遣肅州刺史王本立鞫治,以敕示禕之,禕之曰:「不經鳳閣鸞臺,何謂之敕!」後以為拒制使,賜死於家,年五十七。



 初,禕之得罪,睿宗以舊屬申理之,姻友冀得釋。禕之曰:「吾死矣。太后威福由己,而帝營救,速吾禍也!」在獄上疏自陳。臨誅,洗沐,神色自若。命其子執筆占為表,子號塞不能書,禕之乃自捉筆,得數紙,詞懇哀到,人皆傷之。麟臺郎郭翰、太子文學周思鈞悵嘆其文,後惡之,貶翰巫州司法參軍,思鈞播州司倉參軍。睿宗嗣位,贈禕之中書令。



 翰者,嘗為御史,巡察隴右。多所按劾。次寧州,時狄仁傑為刺史,民爭言有異政。翰就館,以筆紙置於案,謂僚屬曰:「入其境,其政可知,願薦使君美於朝,毋久留。」即命駕去。性寬簡,讀《老子》至「和其光,同其塵」,慨然曰:「大雅君子,以保其身。」乃辭憲官,改麟臺郎云。



 魏玄同,字和初,定州鼓城人。祖士廓,仕齊為輕車將軍。玄同進十擢第,調長安令。累官司列大夫。坐與上官儀善,流嶺外。既廢,不自護藉,乃馳逐為生事。上元初,會赦還,工部尚書劉審禮表其材,拜岐州長史。再遷吏部侍郎。永淳元年,詔與中書、門下同承受進止平章事。封鉅鹿男。上疏言選舉法弊曰:



 方今人不加富、盜賊未衰、禮誼浸薄者,下吏不稱職,庶官非其才,取人之道有所未盡也。武德、貞觀,庶事草創,人物固乏。天祚大聖,享國永年,異人間出。諸色人流,歲以千計,官有常員,人無定限,選集猥至,十不收一,取舍淆紊。



 夏、商以前,制度多闕。至周,煥然可觀。諸侯之臣不皆命天子,王朝庶官不專一職。穆王以伯冏為太僕正,命曰:「慎簡乃僚。」此乃自擇下吏之言也。太僕正,特中大夫耳,尚以僚屬委之,則三公、九卿亦當然也。故太宰、內史並掌爵祿廢置,司徒、司馬別掌興賢詔事。是分任群司而統以數職,王命其大者,而自擇其小者。



 漢制,諸侯自置吏四百石以下,其傅、相大臣則漢為置之;州郡掾史、督郵、從事,悉任之牧守。



 自魏、晉以後,始歸吏部,而迄於今。以刀筆量才,簿書察行,法與世弊,其來久矣。尺丈之量,鍾庾之器,非所及則不能度,非所受則無以容,況天下之大、士類之眾,可委數人手乎!又尸厥任者,間非其選,至為人擇官,為身擇利,下筆系親疏,措情觀勢要,悠悠風塵,此焉奔競,使百行折之一面,九能斷之數言,不亦難乎。



 且臣聞蒞官者,不可以無學。傳曰:「學以從政,不聞以政入學。」今貴戚子弟一皆早仕,弘文、崇賢、千牛、輦腳之類,程較既淺,技能亦薄,而門閥有素,資望自高。夫所謂胄子者,必裁諸學,少則受業,長而入官,然後移家事國,謂之德進。夫少仕則不務學,輕試則無才。又勛官、三衛、流外之屬,不待州縣之舉,直取書判,非先德後言之誼。



 臣聞國之用人,如人用財,貧者止糟糠,富者餘粱肉。故當衰弊乏賢,則磨策朽鈍以馭之;太平多士,則遴柬髦俊而使之。今選者猥多,宜以簡練為急。竊見制書,三品至九品並得薦十,此誠仄席旁求意也。但褒貶不明,故上不憂黜責,下不盡搜揚,莫慎所舉,而茍以應命。且惟賢知賢,聖人篤論。皋陶既舉,不仁者遠。身茍濫進,庸及知人?不擇舉者之緊,而責所舉之濫,不可得已。以陛下聖明,國家德業,而不建經久之策,但顧望魏、晉遺風,臣竊惑之。願少遵周、漢之規,以分吏部選,即所用詳,所失鮮矣。



 不納。進拜文昌左丞、鸞臺侍郎、同鳳閣鸞臺三品。遷地官尚書,檢校納言。玄同與裴炎締交,能保終始,故號「耐久朋」。



 先是,狄仁傑督太原運,失米萬斛,將坐誅,玄同救免。而河陽令周興未知也,數於朝堂聽命。玄同曰:「明府可去矣,毋久留。」興以為沮己,銜之,至是誣玄同言「太后老矣,當復皇嗣」。後不察,賜死於家,年七十三。初,監察御史房濟監刑,謂曰:「丈人盍上變?冀召見,得自陳。」玄同曰;「人殺與鬼殺等耳,不能為告事人!」玄同子恬,字安禮,事親以孝聞。第進士,為御史主簿。開元中。至潁王傅。



 李昭德,雍州長安人。父乾祐,貞觀初為殿中侍御史。鄃令裴仁軌私役門卒,太宗欲斬之,乾祐曰;「法令與天下共之,非陛下獨有也。仁軌以輕罪致極刑,非畫一之制。刑罰不中,則民無所措手足。」帝意解,繇是免死。遷侍御史。母卒,廬墓側,負土成墳。帝遣使就吊,表異其閭。歷治書侍御史,有能名。永徽初,擢御史大夫,為褚遂良所惡,出為邢、魏二州刺史。乾祐雖強直,而暱小人。嘗為書與所善吏,刺取朝廷事,迷隱其辭,為吏所賣,遂良白發於朝,坐流驩州。臺拜滄州刺史。入為司刑太常伯,舉雍州司功參軍崔擢為尚書郎,不得報,私語擢所以然。後擢犯罪,告乾祐漏禁中語以自贖,詔免官,卒。



 昭德強幹有父風,擢明經,累官御史中丞。永昌初,坐事貶振州陵水尉。還為夏官侍郎。如意元年,拜鳳閣侍郎、同鳳閣鸞臺平章事。武後營神都,昭德規創文昌臺及定鼎、上東諸門,標置華壯。洛有二橋,司農卿韋機徙其一直長夏門,民利之,其一橋廢,省巨萬計。然洛水歲淙嚙之,繕者告勞。昭德始累石代柱,銳其前,廝殺暴濤,水不能怒,自是無患。俄檢校內史。薛懷義討突厥,以昭德為行軍長史,不見虜還。



 武承嗣任文昌左相,昭德諫曰;「承嗣已王,不宜典機衡,以惑眾庶。且父子猶相篡奪,況姑侄乎?」後矍然曰;「我未之思也。」乃罷承嗣為太子少保。洛陽人王慶之率險佞數百人請以承嗣為皇太子,後不許;固請,後遣昭德詰其故。昭德笞殺慶之,餘黨散走。因奏曰:「自古有侄為天子而為姑立廟乎?以親親言之,天皇,陛下夫也;皇嗣,陛下子也。當傳之子孫為萬世計。陛下承天皇顧托而有天下,又立承嗣,臣見天皇不來食矣。」後乃止。承嗣恨,譖短之。後曰;「吾任昭德而獲安枕,是代我勞,非而所知也。」有人獲洛水白石而赤文者,獻闕下曰:「此石赤心,故以獻。」昭德叱曰;「洛水餘石豈盡能反邪?」時來俊臣、侯思止舞文法,數誅陷大臣,人皆懾懼。昭德每奏其誣罔不道狀,卒榜殺思止,其黨稍摧沮。



 然昭德頗怙權,為眾指目。魯王府功曹參軍丘愔上疏曰;「臣聞魏冉誅庶族以安秦,忠也。弱諸侯以強國,功也;然出入自專,擊斷無忌,威震人主,不聞有王,張祿一言而卒用憂死。向使昭王不即覺悟,則秦之霸業或不傳子孫。陛下天授以前,萬機獨斷,公卿百執具職而已。自長壽以來,厭怠細政,擢委昭德,乘總權綱,而才小任重,負氣強愎,聾盲下民,芻狗同列,刻薄慶賞,多所矯虔,聲威翕習,天下杜口。臣伏見南臺敕目,群臣奏請,陛下制已曰『可』,而昭德建言不可,制又從之。且人臣參奉機密,獻可替否,事或便利,不豫咨謀,而畫可已行,方興駁異,是陽露擅命,以示於人,歸美引咎,誼不類此,一切奏讞,皆承風指,陰相傅會。臣觀其膽,乃大於身,鼻息所沖,上拂雲漢。夫小家治生,有千百之貲,將以托人,尚憂失授,況天下之重,可輕委寄乎?履霜堅冰,須防其漸。大權一去,收之良難。願陛下察臣之言。」又果毅鄧注著《石論》數千言,述其專恣,鳳閣舍人逢弘敏以聞。後由是惡之,謂姚曰:「誠如所言,昭德固負國矣!」乃貶欽州南賓尉。俄召授監察御史。



 萬歲通天二年,來俊臣誣以逆謀,既而俊臣亦下獄,同日誅。時甚雨,眾庶莫不冤昭德而快俊臣。神龍二年,贈左御史大夫。建中三年,加贈司空。



 吉頊,洛州河南人。長七尺,性陰克,敢言事。舉進士及第。調明堂尉。父哲為易州刺史,坐賕當死,頊往見武承嗣,自陳有二女弟,請侍王巾盥者。承嗣喜,以犢車迎之。三日未言,問其故,答曰:「父犯法且死,故憂之。」承嗣為表貸哲死,遷頊龍為監。



 劉思禮謀反,頊上變事,後命武懿宗雜訊,因諷囚引近臣高閥生平所牾者凡三十六姓,捕系詔獄,搒楚百慘,以成其獄,同日論死,天下冤之。擢右肅政臺中丞。



 來俊臣下獄,司刑當以死,狀三日不下。頊從武後游苑中,因間言:「臣為陛下耳目,知俊臣狀入不出,人以為疑。」後曰:「朕以俊臣有功,徐思之。」頊曰:「於安遠告虺貞反,今為成州司馬。俊臣誣殺忠良,罪惡如山,國蟊賊也,尚何惜?」於是後斬俊臣,而召安遠為尚食奉御。



 突厥陷趙、定,授檢校相州刺史,且募兵制虜南向。頊辭不知武,後曰:「賊方走,藉卿坐鎮耳。」初,太原溫彬茂死高宗時,封一笥書,諉妻曰:「吾死後,須年及垂拱獻之。」垂拱初,妻上其書,言後革命事及突厥至趙去,故後知虜且還。頊至,募士無應者,俄詔以皇太子為元帥,應募日數千。頊還言狀,後曰:「人心若是邪?卿可為群臣道之。」頊誦語於朝,諸武惡之。



 始,頊善張易之、殿中少監田歸道、鳳閣舍人薛稷、正諫大夫員半千、夏官侍郎李迥秀,皆為控鶴內供奉。頊又強敏,故後倚為腹心。聖歷二年,進天官侍郎、同鳳閣鸞臺平章事。為刺史時,武懿宗討契丹,退保相州。後爭功殿中,懿宗陋短俯僂,頊嚴語侵之,無所容假。後怒曰:「我在,乃藉諸武,它日安可保?」銜之。



 張易之兄弟以寵盛,思自全,問頊計安出。頊曰:「公家以幸進,非有大功於天下,勢必危。吾有不朽策,願效之,非止保身,且世世不絕胙。」易之流涕請,頊曰:「天下思唐久矣!廬陵斥外,相王幽閉。上春秋高,武諸王,非海內屬意。公盍從容請相王、廬陵,以副人望?易吊為賀之資也。」易之、昌宗乘間如頊教,後意乃定。既而知頊與謀,召見問狀,頊對:「廬陵、相王皆陛下子,先帝顧托於陛下,當速有所付。」乃還中宗。



 明年,頊坐弟冒偽官貶琰川尉,及辭,召見,泣曰:「臣去國,無復再謁,願有所言。然疾棘,請須臾間。」後命坐,頊曰:「水土皆一盎,有爭乎?」曰:「無。」曰:「以為塗,有爭乎?」曰:「無。」曰:「以塗為佛與道,有爭乎?」曰:「有之。」頊頓首曰:「雖臣亦以為有。夫皇子、外戚,有分則兩安。今太子再立,而外家諸王並封,陛下何以和之?貴賤親疏之不明,是驅使必爭,臣知兩不安矣。」後曰:「朕知之,業已然,且奈何?」頊尋徙始豐尉,客江都,卒。



 中宗之立,頊實倡之,會得罪,無知者。睿宗初,有發明其忠,乃下詔贈御史大夫。



 贊曰:異乎,炎之暗於幾也!知中宗之不君,不知武后之盜朝,假虎翼而責其搏人,死固宜哉!昭德、頊進不以道,君子恥之。雖然,一情區區,抑武興唐,其助有端,則賢炎遠矣。禕之、玄同漏言及誅,不失所以事君者云。



\end{pinyinscope}