\article{列傳第五十五 裴陽宋楊崔李解}

\begin{pinyinscope}

 裴漼,絳州聞喜著姓。父琰之,永徽中為同州司戶參軍,年甚少,不主曹務社會主義社會,在這個社會中,個人的利益與集體利益一致。,刺史李崇義內輕之,鐫諭曰:「同,三輔,吏事繁,子盍求便官?毋留此!」琰之唯唯。吏白積案數百,崇義讓使趣斷,琰之曰:「何至逼人?」乃命吏連紙進筆為省決,一日畢,既與奪當理,而筆詞勁妙。崇義驚曰:「子何自晦,成吾過耶?」由是名動一州,號「霹靂手」。後為永年令,有惠政,吏刻石頌美。以倉部郎中病廢。漼侍疾十餘年,不肯仕。琰之沒,始擢明經,調陳留主簿,遷監察御史。



 時崔湜、鄭愔典吏部,坐奸贓,為李尚隱所劾,詔漼按訊,而安樂公主、上官昭容為阿右,漼執正其罪,天下稱之。累進中書舍人。睿宗造金仙、玉真二觀,時旱甚,役不止,漼上言:「春夏毋聚大眾,起大役,不可興土功,妨農事。若役使乖度,則有疾疫水旱之災,此天人常應也。今自冬徂春,雨不時降,人心憔然,莫知所出,而土木方興,時之孽,職為此發。今東作雲始,丁壯就功,妨多益少,饑寒有漸。《春秋》莊公三十一年冬,不雨,是時歲三築臺;僖公二十一年夏,大旱,是時作南門。陛下以四方為念,宜下明制,令二京營作、和市木石,一切停止。有如農桑失時,戶口流散,雖寺觀營立,能救饑寒敝哉!」不報。遷兵部侍郎。以銓總勞,特授一子官。開元五年,為吏部侍郎,甄拔士為多。拜御史大夫。



 漼雅與張說善,說方宰相,數薦之,漼長於敷奏,天子亦自重焉,擢吏部尚書。世儉素,而晚節稍畜伎妾,為奢侈事,議者以為缺。改太子賓客。卒,贈禮部尚書,謚曰懿。從祖弟寬。



 寬,性通敏,工騎射、彈棋、投壺,略通書記。景雲中,為潤州參軍事。刺史韋詵有女,擇所宜歸,會休日登樓,見人於後圃有所瘞藏者,訪諸吏,曰:「參軍裴寬居也。」與偕來,詵問狀,答曰:「寬義不以包苴汙家,適有人以鹿為餉,致而去,不敢自欺,故瘞之。」詵嗟異,乃引為按察判官,許妻以女。歸語妻曰:「常求佳婿,今得矣。」明日,幃其族使觀之。寬時衣碧,瘠而長,既入,族人皆笑,呼為「碧鸛雀」。詵曰:「愛其女,必以為賢公侯妻也,何可以貌求人?」卒妻寬。



 舉拔萃,為河南丞,遷長安尉。宇文融為侍御史,括天下田,奏為江東覆田判官。改太常博士。禮部建言忌日享廟應用樂,寬自以情立議曰:「廟尊忌卑則作樂,廟卑忌尊則備而不奏。」中書令張說善之,請如寬議。遷刑部員外郎。萬騎將軍馬崇白日殺人,而王毛仲方以貴幸,將鬻其獄,寬固執不肯從。河西節度使蕭嵩表為判官,歷兵部侍郎。宰相裴耀卿領江淮運,列倉河陰,奏寬為戶部侍郎自副。遷吏部。出為蒲州刺史,州久旱,寬入境輒雨。徙河南尹,不屈附權貴,河南大治。由金吾大將軍授太原尹,玄宗賦詩褒餞。天寶初,由陳留太守拜範陽節度使。時北平軍使烏承恩,虜酋也,與中人通,數冒賄,寬以法繩治。檀州刺史何僧獻生口數十,寬悉歸之,故夷夏感附。



 三載,用安祿山守範陽,召寬為戶部尚書,兼御史大夫。裴敦復平海賊還,廣張功簿,寬密白其妄。會河北部將入朝,盛譽寬政,且言華虜猶思之,帝嗟賞,眷倚加厚。李林甫恐其遂相,又惡寬善李適之,乃漏寬語以激敦復,敦復任氣而疏,以林甫為誠。先是,寬以所善請於敦復,即欲白發其言,林甫趣之。敦復未及聞,扈幸溫泉宮。而其下裨將程藏曜、曹鑒自以他事系臺,寬捕按之,敦復謂寬求致其罪,遽以金五百兩賂貴妃姊,因得事聞於帝,由是貶寬睢陽太守。及韋堅獄起,寬復坐親,貶安陸別駕。林甫任羅希奭殺李適之也,亦使過安陸,將怖殺寬,寬叩頭祈哀,希奭乃去。寬懼終見殺,丐為浮屠,不許。稍遷東海太守,徙馮翊,入為禮部尚書。卒,年七十五,贈太子太傅。



 寬兄弟八人,皆擢明經,任臺、省、州刺史。雅性友愛,於東都治第,八院相對,甥侄亦有名稱,常擊鼓會飯。其為政務清簡,所蒞人愛之,世皆冀其得宰相。天寶間稱舊德,以寬為首。然惑於佛,喜與桑門游,習誦其書,老彌篤云。子住。



 住,字士明,擢明經,調河南參軍事。性通綽,舉止不煩。累遷京兆倉曹參軍。虢王巨表署襄、鄧營田判官。母喪,居東都。會史思明亂,逃山谷間。思明故為寬將,德寬舊恩,且聞住名,遣捕騎跡獲之,喜甚,呼為「郎君」,偽授御史中丞。賊殘殺宗室,住陰緩之。全活者數百人。又嘗疏賊虛實於朝,事洩,思明恨罵,危死而免。賊平,除太子中允,遷考功郎中,數燕見奏事。



 代宗幸陜,住徒步挾考功南曹印赴行在,帝曰:「疾風知勁草,果可信。」將用為御史中丞,為元載沮卻,故拜河東租庸、鹽鐵使。時關輔旱,住入計,帝召至便殿,問榷酤利歲出內幾何,住久不對。帝復問,曰:「臣有所思。」帝曰:「何邪?」住曰:「臣自河東來,涉三百里,而農人愁嘆,穀菽未種。誠謂陛下軫念元元,先訪疾苦,而乃責臣以利。孟子曰:『治國者,仁義而已,何以利為?』故未敢即對。」帝曰:「微公言,朕不聞此。」拜左司郎中,數訪政事。載忌之,出為虔州刺史,歷饒、盧、亳三州,除右金吾將軍。



 德宗新即位,以刑名治天下,百吏震服。時大行將蕆陵事,禁屠殺,尚父郭子儀家奴宰羊,住列奏,帝謂不畏強御,善之,或曰:「尚父有社稷功,豈不為庇之?」住笑曰:「非君所知。尚父方貴盛,上新即位,必謂黨附者眾。今發其細過,以明不恃權耳。吾上以盡事君之道,下以安大臣,不亦可乎?」



 時朝堂別置三司決庶獄,辨爭者輒擊登聞鼓。住上疏曰:「諫鼓、謗木之設,所以達幽枉,延直言。今詭猾之人,輕動天聽,爭纖微,若然者,安用吏治乎?」帝然之,於是悉歸有司。住惡法吏舞文,或挾宿怨為重輕,因獻《獄官箴》以諷。坐所善誅,貶閬州司馬。俄召為太子右庶子,進兵部侍郎,至河南尹、東都副留守。凡五世為河南,住視事未嘗敢當正處。以寬厚和易為治,不鞫人以贓。卒,年七十五,贈禮部尚書。



 寬弟子胄,字胤叔,擢明經,佐李抱玉鳳翔幕府。不得意,謝歸,更從宣歙觀察使陳少游,抱玉怒,劾貶桐廬尉。時李棲筠觀察浙西,幕府皆一時高選。判官許鳴謙名知人,見崔造及胄,器之,白棲筠取胄為支使。



 代宗惡宰相元載怙權,召棲筠為御史大夫,欲以相,棲筠引胄殿中侍御史,尤為載所惡。會棲筠卒,胄護喪歸洛陽,人為危之,胄屹然不沮惴。少游復表為淮南觀察判官。載誅,始拜刑部員外郎,遷宣州刺史。楊炎當國,為載復仇,窮摭所惡。會胄部人積胄雜奉為贓,炎遣員寓蔓劾峭詆,貶汀州司馬。稍遷京兆少尹,以父名不拜,換國子司業。遷江西觀察使。初,李兼嘗罷南昌卒千餘人,收資稟為月進,胄白罷之。樊澤徙襄州,宰相議所代,德宗雅記胄才,遂拜荊南節度使。



 是時,方鎮爭剝下希恩,制重錦異綾,名貢奉,有中使者,即悉公帑市歡。胄待之有節,獻餉直不數金,宴勞止三爵。是時武臣多粗暴庸人,待賓介不以禮,少失意,則以罪中傷之,胄亦劾斥其管記,世恨胄之流於俗。卒,年七十五,贈尚書右僕射,謚曰成。



 陽嶠,其先北平人,世徙洛陽,北齊尚書右僕射休之四世孫。舉八科皆中,調將陵尉,累遷詹事司直。長安中,左右御史中丞桓彥範、袁恕己爭取為御史。楊再思素與嶠善,知其意不樂彈抨事,為語彥範,彥範曰:「為官擇人,豈待情樂乎?唯不樂者固與之,以伸難進、抑躁求也。」遂為右臺侍御史。久乃遷國子司業。嶠資謹飭好學,喜誘勸後生、修講舍,人以為善職。



 睿宗立,進尚書右丞。時議建都督府,擇最吏,故嶠為涇州都督。議罷,歷魏州刺史、荊州長史、本道按察使,率以清白聞。魏州人嫠耳闕下,請嶠為刺史,故再治魏。入為國子祭酒,封北平縣伯。引尹知章、範行恭、趙玄默為學官,皆名儒冠云。生徒游惰者至督以鞭楚,人怨之,乘夜毆嶠道中,事聞,詔捕毆者殺之。嶠撫孤侄與子均,常語人曰:「吾備位方伯,而心亦昔時一尉耳。」以老致仕。卒,謚曰敬。



 宋慶禮,洺州永年人。擢明經,補衛尉。武后詔侍御史桓彥範行河北,鄣斷居庸、五回等路,以支突厥,召慶禮與議,見其方略,器之。俄遷大理評事,為嶺南採訪使。時崖、振五州首領更相掠,民苦於兵,使者至,輒苦瘴癘,莫敢往。慶禮身到其境,諭首領大誼,皆釋仇相親,州土以安,罷戍卒五千。歷監察、殿中侍御史。以習識邊事,拜河東、河北營田使。善騎,日能馳數百里。性甘於勞苦,然好興作,濱塞掘阱植兵,以邀虜徑,議者蚩其不切事。稍遷貝州刺史,復為河北支度營田使。



 初,營州都督府治柳城,扼制奚、契丹。武后時,趙文翽失兩籓情,攻殘其府,更治東漁陽城。玄宗時,奚、契丹款附,帝欲復治故城,宋璟固爭不可,獨慶禮執處其利,乃詔與太子詹事姜師度、左驍衛將軍邵宏等為使,築裁三旬畢。俄兼營州都督,開屯田八十餘所,追拔漁陽、淄青沒戶還舊田宅,又集商胡立邸肆。不數年,倉廥充,居人籓輯。卒,贈工部尚書。



 慶禮為政嚴,少私,吏畏威不敢犯。太常博士張星以好巧自是,謚曰「專」。禮部員外郎張九齡申駁曰:「慶禮國勞臣,在邊垂三十年。往城營州,士才數千,無甲兵強衛,指期而往,不失所慮,遂罷海運,收歲儲,邊亭晏然。其功可推,不當丑謚。」慶禮兄子辭玉亦自詣闕訴。改謚曰敬。



 楊瑒,字瑤光,華州華陰人。五世祖縉,為陳中書舍人,名屬義,終交、愛九州都督、武康郡公。子林甫代領都督,隋滅陳,逾三年乃降,徙長安。林甫字衛卿,為柳城太守,高祖軍興,遣其子琮招之,挈郡以來,授檢校總管,足疾不能造朝。帝以絳州寒涼,拜刺史,累封宜春郡公。琮字孝璋,為上津令。會天下亂,去官,與秦王同里居。武德初,為王府參軍,兼庫直。隱太子事平,詔親王、宰相一人入宴,而琮獨預,太宗賜《懷昔賦》,申以恩意。歷沔、綏二州刺史。姆饋孺子以餅,妻偽受而棄之垣外,人咨其廉。



 瑒始為麟游令,時竇懷貞大營金仙、玉直二觀,檄取畿內嘗負逆人貲者,暴斂之以佐費,瑒拒不應。懷貞怒曰:「縣令而拒大夫命乎?」瑒曰:「所論者民冤抑也,位高下乎何取?」懷貞壯其對,為止。初,韋後表民二十二為丁限,及敗,有司追趣其課,瑒執不可,曰:「韋氏當國,擅擢士大夫,赦罪人,皆不改,奚獨取已寬之人重斂其租?非所以保下之宜。」遂止不課,由是名顯當世。



 擢累侍御史。京兆尹崔日知貪沓不法,瑒與大夫李傑謀劾舉之,反為日知先構。瑒廷奏曰:「肅繩之司,一為恐脅所屈,開奸人謀,則御史府可廢。」玄宗直之,令傑還視事,而逐日知。



 瑒進歷御史中丞、戶部侍郎。帝嘗召宰相大臣議天下戶版延英殿,瑒言利病尤詳,帝咨賞。於是宇文融建檢脫戶餘口,瑒執不便。融方貴,公卿唵默唯唯,獨瑒抗議,故出為華州刺史。帝封太山,集樂工山下,居喪者亦在行。瑒謂起苴絰使和鐘律,非人情所堪,帝許,乃免。



 入為國子祭酒,表大儒王迥質、尹子路、白履忠等三人教授國子。有詔迥質諫議大夫、皇太子侍讀;履忠老不任職,拜朝散大夫罷歸;子路直弘文館。皆有名。瑒奏:「有司帖試明經,不質大義,乃取年頭、月尾、孤經、絕句,且今習《春秋》三家、《儀禮》者才十二,恐諸家廢無日,請帖平文以存學家,其能通者稍加優宦,獎孤學。」從之,因詔以三家《傳》、《儀禮》出身者不任散官,遂著令。生徒為瑒立頌太學門。



 又言:「古者卿大夫子弟及諸侯歲貢小學之異者入太學,漸漬禮樂,知朝廷君臣之序,班以品類,分以師長,三德四教,學成然後爵之。唐興,二監舉者千百數,當選者十之二,考功覆校以第,謂經明行修,故無多少之限。今考功限天下明經、進士歲百人,二監之得無幾,然則學徒費官稟,而博士濫天祿者也。且以流外及諸色仕者歲二千,過明經、進士十倍,胥史浮虛之徒,眊先王禮義,非得與服勤道業者挈長短、絕輕重也。國家啟庠序,廣化導,將有以用而勸進之。有司為限約以黜退之,欲望俊乂在朝,難矣。」帝然其言。再遷大理卿,以疾辭,改左散騎常侍。卒,年六十八,贈戶部尚書,謚曰貞。



 瑒常嘆士大夫不能用古禮,因其家冠、婚、喪、祭,乃據舊典為之節文,揖讓威儀,哭踴衰殺,無有違者。在官清白,吏請立石紀德,瑒曰:「事益於人,書名史氏足矣。若碑頌者,徒遣後人作碇石耳。」



 瑒伯父志操,頗剛簡,未遇時,著《閑居賦》自托,常曰:「得田十頃、僮婢十人,下有兄弟布粟之資,上可供先公伏臘足矣。」位終司屬卿、安平縣男。瑒從父兄晏,精《孝經》學,常手寫數十篇,可教者輒遺之。



 崔隱甫,貝州武城人。隋散騎侍郎人麃曾孫。解褐左玉鈐衛兵曹參軍,遷殿侍御史內供奉。浮屠惠範倚太平公主脅人子女,隱甫劾狀,反為所擠,貶邛州司馬。玄宗立,擢汾州長史,兼河東道支度營田使,遷洛陽令。梨園弟子胡雛善笛,有寵,嘗負罪匿禁中。帝以他事召隱甫,從容指曰:「就卿丐此人。」對曰:「陛下輕臣而重樂工,請解官。」再拜出,帝遽謝,與胡雛,隱甫殺之,有詔貰死,不及矣。賜隱甫百縑。



 孫佺敗績于奚,擢隱甫並州司馬護邊,會兄逸甫疾甚,未及行,詔責逗留,下除河南令。累拜華州刺史、太原尹,入為河南尹。居三歲,進拜御史大夫。初,臺無獄,凡有囚則系大理。貞觀時,李乾祐為大夫,始置獄,由是中丞、侍御史皆得系人。隱甫執故事,廢掘諸獄。其後患囚往來或漏洩,復系之廚院云。臺中自監察御史而下,舊皆得顓事,無所承諮。隱甫始一切令歸稟乃得行,有忤意輒劾正,多貶絀者,臺吏側目,威名赫然。帝嘗詔校外官歲考。異時必委曲參審,竟春未定。隱甫一日會朝集使,詢逮檢實,其暮皆訖,議者服其敏。帝嘗謂曰:「卿為大夫,天下以為稱職。」



 張說當國,隱甫素惡之,乃與中丞宇文融、李林甫暴其過,不宜處位,說賜罷;然帝嫉朋黨,免其官,使侍母。歲餘,復為大夫。遷刑部尚書,兼河南尹。帝還京師,即拜東都留守。累封清河郡公。卒,贈益州大都督,謚曰忠。



 始,帝欲相隱甫也,謂曰:「牛仙客可與語,卿常見否?」對曰:「未也。」帝曰:「可見之。」隱甫終不詣。他日又問,對如初。帝乃不用。子弟或問故,答曰:「吾不以其人微易之也,其材不逮中人,可與之對耶?」隱甫所至絜介自守,明吏治,在職以強正稱云。



 贊曰:嚴挺之拒宰相不肯見李林甫,崔隱甫違詔不屈牛仙客,信剛者乎!二人坐是皆不得相,彼亦各申其志也。管夷吾以編棧諭之,信曲與直不相函哉!



 李尚隱,其先出趙郡,徙貫萬年。年二十,舉明經,再調下邽主簿,州刺史姚班說其能,器之。神龍中,左臺中丞侯令德為關內黜陟使,尚隱佐之,以最擢左臺監察御史。於是,崔湜、鄭愔典吏部選,附勢幸,銓擬不平,至逆用三年員闕,材廉者軋不進,俄而相踵知政事,尚隱與御史李懷讓顯劾其罪,湜等皆斥去。睦州刺史馮昭泰性鷙刻,人憚其強,嘗誣系桐廬令李師旦二百餘家為妖蠱,有詔御史覆驗,皆稱病不肯往。尚隱曰:「善良方蒙枉,不為申明,可乎?」因請行,果推雪其冤。湜、愔復當路,乃出尚隱為伊闕令,懷讓魏令。湜等伏誅,玄宗知尚隱方嚴,由定州司馬擢吏部員外郎,懷讓自河陽令拜兵部員外郎。懷讓,蓚人,後歷給事中。



 尚隱以將作少監營橋陵,封高邑縣男。未幾,進御史中丞。御史王旭招權,稍不制,仇家告其罪,尚隱窮治,具得奸贓,無假借,遂抵罪。進兵部侍郎。俄出為蒲州刺史。浮屠懷照者,自言母夢日入懷生己,鏤石著驗,聞人馮待徵等助實其言。尚隱劾處妖妄,詔流懷照播州。再遷河南尹。



 尚隱性剛亮,論議皆披心示誠,處事分明,御下不苛密。尤詳練故實,前後制令,誦記略無遺。妖賊劉定高夜犯通洛門,尚隱坐不素覺,左遷桂州都督。帝遣使勞曰:「知卿忠公,然國法須爾。」因賜雜彩百匹遣之。遷廣州都督、五府經略使。及還,人或袖金以贈,尚隱曰:「吾自性分不可易,非畏人知也。」



 代王丘為御史大夫。時司農卿陳思問引屬吏多小人,乾隱錢穀,尚隱按其違,贓累鉅萬,思問流死嶺南。改尚隱太子詹事。不閱旬,進戶部尚書。前後更揚、益二州長史、東都留守,爵高邑伯。開元二十八年,以太子賓客卒,年七十五,謚曰貞。



 尚隱三入御史府,輒繩惡吏,不以殘摯失名,所發當也,素議歸重。仕官未嘗以過謫,惟劾詆幸臣及坐小法左遷,復見用,以循吏終始云。



 自開元二十二年置京畿採訪處置等使,用中丞盧奐為之,尚隱以大夫不充使。永泰以後,大夫王翊、崔渙、李涵、崔寧、盧杞乃為之。



 解琬,魏州元城人。舉幽素科,中之,調新政尉。後自成都丞奏事稱旨,躐除監察御史,以喪免。武後顧琬習邊事,迫追西撫羌夷,琬因乞終喪,後嘉許之,詔服除赴屯。遷侍御史,安撫烏質勒及十姓部落,以功擢御史中丞,兼北庭都護、西域安撫使。琬與郭元振善,宗楚客惡之,左授滄州刺史。為政引大體,部人順附。



 景龍中,遷御史大夫,兼朔方行軍大總管。前後乘邊積二十年,大抵務農習戰,多為長利,華虜安之。景雲二年,復為朔方軍大總管,分遣隨軍要籍官河陽丞張冠宗、肥鄉令韋景駿、普安令於處忠料三城兵,省其戍十萬人。改右武衛大將軍,兼檢校晉州刺史、濟南縣男。以老丐骸骨,不待報輒去,優詔以金紫光祿大夫聽致仕,準品給全祿,璽書勞問。會吐番騷邊,復召授左散騎常侍,詔與虜定經界,因諧輯十姓降戶。琬建言吐蕃不可以信約,請調兵十萬屯秦、渭間,防遏其奸。是冬,吐蕃果入寇,為秦渭兵擊走之。俄復請老,不許,遷太子賓客。年八十餘,開元五年,終同州刺史。



\end{pinyinscope}