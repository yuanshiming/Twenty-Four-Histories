\article{列傳第一百四十三 沙陀}

\begin{pinyinscope}

 沙陀,西突厥別部處月種也。始,突厥東西部分治烏孫故地,與處月、處蜜雜居。貞觀七年,太宗以鼓纛立利邲咄陸可汗羊傳》制定義例。系統闡發《春秋》中的「微言大義」。又有,而族人步真觖望,謀並其弟彌射乃自立。彌射懼,率處月等入朝。而步真勢窮,亦歸國。其留者,咄陸以射匱特勒劫越之子賀魯統之。



 西突厥浸強,內相攻,其大酋乙毘咄陸可汗建廷鏃曷山之西,號北庭,而處月等又隸屬之。處月居金娑山之陽,蒲類之東,有大磧,名沙陀,故號沙陀突厥云。



 咄陸寇伊州,引二部兵圍天山,安西都護郭孝恪擊走之,拔處月俟斤之城。後乙毘可汗敗,奔吐火羅。賀魯來降,詔拜瑤池都督,徙其部庭州之莫賀城。處月硃邪闕俟斤阿厥亦請內屬。



 永徽初,賀魯反,而硃邪孤注亦殺招慰使連和,引兵據牢山。於是射脾俟斤沙陀那速不肯從,高宗以賀魯所領授之。明年,弓月道總管梁建方、契苾何力引兵斬孤注,俘九千人。又明年,廢瑤池都督府,即處月地置金滿、沙陀二州,皆領都督。賀魯亡,安撫大使阿史那彌射次伊麗水,而處月來歸。乃置昆陵都護府,統咄陸部,以彌射為都護。



 龍朔初,以處月酋沙陀金山從武衛將軍薛仁貴討鐵勒,授墨離軍討擊使。長安二年,進為金滿州都督,累封張掖郡公。金山死,子輔國嗣。先天初避吐蕃,徙部北庭,率其下入朝。開元二年,復領金滿州都督,封其母鼠尼施為鄯國夫人。輔國累爵永壽郡王。死,子骨咄支嗣。



 天寶初,回紇內附,以骨咄支兼回紇副都護。從肅宗平安祿山,拜特進、驍衛上將軍。死,子盡忠嗣,累遷金吾衛大將軍、酒泉縣公。至德、寶應間,中國多故,北庭、西州閉不通,朝奏使皆道出回紇,而虜多漁擷,尤苦之,雖沙陀之倚北庭者,亦困其暴斂。



 貞元中,沙陀部七千帳附吐蕃,與共寇北庭,陷之。吐蕃徙其部甘州,以盡忠為軍大論。吐蕃寇邊,常以沙陀為前鋒。



 久之,回鶻取涼州,吐蕃疑盡忠持兩端,議徙沙陀於河外,舉部愁恐。盡忠與硃邪執宜謀,曰:「我世為唐臣,不幸陷污,今若走蕭關自歸,不愈於絕種乎?」盡忠曰:「善。」元和三年,悉眾三萬落循烏德鞬山而東。吐蕃追之。行且戰,旁洮水,奏石門,轉鬥不解,部眾略盡,盡忠死之。執宜裒瘢傷,士裁二千,騎七百,雜畜橐它千計,款靈州塞。節度使範希朝以聞。詔處其部鹽州,置陰山府,以執宜為府兵馬使。沙陀素健鬥,希朝欲藉以捍虜,為市牛羊,廣畜牧,休養之。其童耄自鳳翔、興元、太原道歸者,皆還其部。盡忠弟葛勒阿波率殘部七百叩振武降,授左武衛大將軍,兼陰山府都督。



 執宜朝長安,賜金幣袍馬萬計,授特進、金吾衛將軍。然議者以靈武迫吐蕃,恐後反覆生變,又濱邊,益口則食翔價。頃之,希朝鎮太原,因詔沙陀舉軍從之。希朝乃料其勁騎千二百,號沙陀軍,置軍使,而處餘眾於定襄川。執宜乃保神武川之黃花堆,更號陰山北沙陀。是時,天子伐鎮州,執宜以軍七百為前鋒,王承宗眾數萬伏木刀溝,與執宜遇,飛矢雨集。執宜提軍橫貫賊陣鏖斗,李光顏等乘之,斬首萬級。鎮兵解,進蔚州刺史。王鍔節度太原,建言:「硃邪族孳熾,散居北川,恐啟野心,願析其族隸諸州,勢分易弱也。」遂建十府以處沙陀。八年,回鶻過磧南取西城、柳谷,詔執宜屯天德。明年,伐吳元濟,又詔執宜隸李光顏,破蔡人時曲,拔凌雲柵。元濟平,授檢校刑部尚書,猶隸光顏軍。長慶初,伐鎮州,悉發沙陀,與易定軍掎角,破賊深州。執宜入朝,留宿衛,拜金吾衛將軍。大和中,柳公綽領河東,奏陘北沙陀素為九姓、六州所畏,請委執宜治雲、朔塞下廢府十一,料部人三千御北邊,號代北行營,授執宜陰山府都督、代北行營招撫使,隸河東節度。



 執宜死,子赤心嗣。開成四年,回鶻徑磧口,抵榆林塞。宰相掘羅勿以良馬三百遺赤心,約共攻彰信可汗。可汗死,節度使劉沔以沙陀擊回鶻於殺胡山。久之,伐潞,誅劉稹,詔赤心率代北騎軍三千隸石雄為前軍,破石會關,助王宰下天井,合太原軍,次榆社,與監軍使呂義忠禽楊弁。潞州平,遷朔州刺史,仍為代北軍使。



 大中初,吐蕃合黨項及回鶻殘眾寇河西,太原王宰統代北諸軍進討,沙陀常深入,冠諸軍。赤心所向,虜輒披靡,曰:「吾見赤馬將軍火生頭上。」始,沙陀臣吐蕃,其左老右壯,溷男女,略與同,而馳射趫悍過之,虜倚其兵,常苦邊。及歸國,吐蕃由此亦衰。宣宗已復三州、七關,征西戍皆罷,乃遷赤心蔚州刺史、雲州守捉使。



 龐勛亂,詔義成康承訓為行營招討使,赤心以突騎三千從。承訓兵絕渙水,遇伏,墮圍中幾沒,赤心以騎五百掀出之。勛欲速戰,眾八萬,短兵接,赤心勒勁騎突賊,與官軍夾擊,敗之。其弟赤衰以千騎追之亳東。勛平,進大同軍節度使,賜氏李,名國昌,預鄭王屬籍,賜親仁里甲第。回鶻叩榆林,擾靈、鹽,詔國昌為鄜延節度使。又寇天德,乃徙節振武,進檢校司徒。王仙芝陷荊、襄,朝廷發諸州兵討捕,國昌遣劉遷統云中突騎逐賊,數有功。



 乾符三年,段文楚為代北水陸發運、雲州防禦使。是時無年,文楚晙損用度,下皆怨。邊校程懷信、王行審、蓋寓、李存璋、薛鐵山、康君立等曹議曰:「世多難,丈夫當投罅立功。段公乃儒者,難共計。沙陀雄勁,李振武父子勇冠軍,我若推之,無不應,則代北唾手可定。拾取富貴若何?」咸曰:「善!」乃夜謁國昌子雲中守捉使克用曰:「歲艱稟食削,吾等不忍餓死,公家威德著聞,請誅虐帥,安部內。」克用許之,募得士萬人,趨雲州,次鬥雞臺。城中執文楚至,殺之;據州以聞,共丐克用為大同防禦留後。不許,發諸道兵進捕,諸道不甚力,而黃巢方引度江,朝廷度未能制,乃赦之,以國昌為大同軍防禦使。國昌不受命,詔河東節度使崔彥昭、幽州張公素共擊之,無功。



 國昌與黨項戰,未決,大同川吐渾赫連鐸襲振武,盡取其貲械。國昌窮,挈騎五百還雲州,州不納,鐸遂取之。克用轉側蔚、朔間,裒兵才三千,屯新城,鐸引萬人圍之,隧而攻,三日不拔,鐸兵殺傷甚。國昌自蔚州來,鐸引去。僖宗以鐸領大同節度,畀討國昌。六年,詔昭義李鈞為北面招討使,督潞、太原兵屯代州;幽州李可舉會鐸攻蔚州,國昌以一隊當之。克用分兵抵遮虜城拒鈞,天大雪,士戺僕,鈞眾潰,還代州,軍遂亂,鈞死於兵。廣明元年,以李琢為蔚、朔招討都統,率兵數萬屯代州。克用使傅文達調蔚、朔兵,朔州刺史高文集縛以送琢。琢進攻蔚州,國昌敗,與克用舉宗奔達靼。鐸密畀酋長圖之,克用得其計,因豪桀大會馳射,百步外針芒木葉無不中,部人大驚,即倡言:「今黃巢北寇,為中原患,一日天子赦我,願與公等南向定天下,庸能終老沙磧哉!」達靼知不留,乃止。



 巢攻潼關,入京師,詔河東監軍陳景思發代北軍。時沙陀都督李友金屯興唐軍,薩葛首領米海萬、安慶都督史敬存屯感義軍,克用客塞下,眾數千無所屬。景思聞天子西,乃與友金料騎五千入居絳,兵擅劫帑自私。還代州,益募士三萬,屯崞西,士囂縱,友金不能制,謀曰:「今合大眾,不得威名宿將,且無功。吾兄司徒父子,材而雄,眾所推畏,比得罪於朝,僑戍北部不敢還。今若召之使將兵,代北豪英,一呼可集,整行伍,鼓而南,賊不足平也。」景思曰:「善!」乃丐赦國昌,使討賊贖罪。有詔拜克用代州刺史、忻代兵馬留後,促本軍討賊。克用募達靼萬人,趨代州,將南道太原。節度使鄭從讜塞石嶺關,不得前,克用儳道至太原,營城下五日,邀糧貲,從讜不答,乃大略,還屯代州。



 中和二年,蔚州刺史蘇祐會赫連鐸兵將攻代州,克用率騎五百先襲蔚州,下之。祐屯美女谷,鐸與幽州李可舉眾七萬攻蔚州,譙柵相屬。克用直搗營,入蔚州,燔府庫,棄而去,屯雁門。國昌自達靼率兵歸代州。擾汾、並、樓煩,不釋鎧。帝詔克用還軍朔州。



 於是義武節度使王處存、河中節度使王重榮傳詔招克用同討巢。克用喜,即大閱雁門,得忻、代、蔚、朔、達靼眾三萬、騎五千而南。於是國昌守代州。鄭從讜不肯假道,克用軍傅太原而營,奉幣馬遺從讜,身從數騎呼曰:「我且西,願與公一言。」從讜升陴慰勉,歸貨幣饔餼。克用乃自陰地趨晉,會河中。帝聞,擢克用雁門節度、神策天寧軍鎮遏、忻代觀察使。明年,宰相王鐸承制,授克用東北面行營都統,河東監軍陳景思為監軍使。克用使弟克脩領彀騎五百度河,克用自夏陽濟,留薛阿檀扼津口,次同州,壁乾坑,與賊戰梁田坡,敗之。進壁渭橋,遂收京師。功第一,進同中書門下平章事、隴西郡公;國昌為代北軍節度使。未幾,以克用領河東節度。



 黃巢與秦宗權合寇河南。四年,克用率河東、代北兵將自澤、潞下天井關,河陽諸葛爽堙井以拒,克用乃由河中濟,趨許州,合徐、汴兵破尚讓於太康。戰西華,又破之。賊走,河南平。追北曹州,還過汴,硃全忠邀之,克用留兵於郊,入舍上源館。夜帳飲,全忠自佐饔,進貲寶,握手諄勞。是時,全忠忌克用桀邁難制,則連車外環,陳兵道左右。克用醉。乃攻館,下拒戰,親將郭景銖滅燭扶克用,徐告之,尚被酒,乃引弓射。會煙囂四合,大震電,克用與薛志勤等間關升南譙門,縋走營。部下死者數百人,所獲賊乘輿物盡亡之。克用整眾歸太原,益訓兵,將報仇,使弟克勤以萬騎屯河中,乃請擊全忠。使者八返,內外震恐,帝使內謁慰解。尋進位檢校太傅、隴西郡王。



 光啟元年,幽州李可舉、鎮州王景崇言:「易定故燕、趙境,請取分之。」於是可舉攻易州,下之;景崇攻無極。易定節度使王處存求救於克用,克用自將救無極,敗鎮人,攻馬頭,固新城。鎮兵走,處存復取易州。鳳翔李昌符、邠寧硃玫與全忠連和,觀軍容使田令孜惡克用與王重榮合,建言:「不可處近輔,請授王處存河中,而徙重榮於易定,則克用孤矣。」帝從之。重榮以告,克用怒曰:「我當從公提鼓出汜水關誅全忠,回殲穴鼠耳。」重榮計曰:「公兵朝出關,則邠、岐兵夕傅吾堞,願先治邠、岐。」克用乃表言:「玫、昌符連全忠為亂,請以兵十五萬度河梟二豎,然後平汴雪大恥,願陛下戒嚴,無為賊所搖。」帝遣使慰止,背相望也。克用不奉詔,玫亦引邠、鳳兵營沙苑。克用薄戰,玫敗,夜亡去。克用還河中,天子出趣鳳翔,道傳兵且至,即趣寶雞。克用與重榮聯章請還宮,願留兵衛京師,即還鎮。帝懼,走大散關,駐興元。克用引歸。嗣襄王煴偽詔至太原,克用燔之,執其使,間道奉表興元。始,朝廷意玫結克用迫乘輿,及表至,示群臣,因騰曉山南諸鎮,行在少安。王行瑜斬玫,克用以千騎經略京畿。三年,國昌卒。俄而昭宗即位,進克用檢校太師兼侍中。



 大順初,克用自攻赫連鐸於雲州,拔東郛。幽州李匡威以兵三萬救之,殺其將安金俊,克用走。鐸與匡威共建言:「山南亂,克用實首之。今乘其敗,可伐而取也。」全忠亦請與河北三鎮共討之。宰相張浚是其計,乃下制削克用官爵、屬籍,以浚為兵馬招討、制置、宣慰使,京兆尹孫揆副之,樞密使駱全諲為行營都監,華州節度使韓建為行營馬步都虞候兼供軍糧料使,王鎔領河東東面,全忠南面,李匡威北面,並為行營招討使。鐸副匡威,先薄戰。克用追潞兵,不肯行,共殺守將李克恭,送款於汴,南首闕下。更詔揆為昭義節度使,克用將李存孝邀揆長子殺之。匡威、鐸並吐蕃、黠戛斯眾十萬攻遮虜軍,殺其將劉胡子。克用乃屯渾河川,存孝與鐸戰樂安,鐸敗走。浚入陰地關,壁汾、隰,薛鐵山、李承嗣營洪洞迎戰。存孝次趙城,韓建夜出壯士三百乘其營,存孝伏以待,建兵大奔。存孝攻絳州,未下,晉州刺史張行恭棄城走,建與浚遁還。明年,克用奉表自陳,乃復拜檢校太師、守中書令、隴西郡王。



 克用悉兵攻鐸雲州,以騎將薛阿檀為前軍,設伏河上。鐸縱騎追阿檀,遇伏而奔。鐸亡入吐渾。克用取雲州,以部將石善友為刺史、大同軍防禦使。



 景福初,鎮州王鎔攻堯山,克用使李嗣勛擊之,斬級三萬,克用遂拔天長,略常山,度滹沱,燔其郛。徇地至趙,取鼓、槁二城。赫連鐸眾八萬攻天成軍,克用飛檄發軍太原,匡威已壁雲州北郊,克用自神堆引軍夜入雲州,死戰,走之。乾寧元年,克用次新城,鐸膝行詣軍門降,克用鞭而縱之。進下武州,攻新州。李匡籌引步騎七萬救之,克用迎戰,斬首萬級,俘少將三百,徇城下,新州降。取媯州,匡籌棄幽州走。明年,幽州降,克用以劉仁恭為留後,乃旋。



 王行瑜、韓建、李茂貞連兵南闕下,殺李溪。克用盡調北部兵度河,拔絳州,斬刺史王瑤。次河中,王珂謁於道。同州王行約奔京師。圍韓建於華州,京師震動,帝為幸石門、莎城,遣內謁郗廷昱慰勞,且言茂貞屯盩厔,行瑜屯興平,克用乃進營渭橋。帝以嗣延王戒丕、嗣丹王允詔克用擊邠、鳳。克用奉詔,屯渭北,遣史儼以票騎三千護石門,且令王珂輸河中粟備行在。帝以赤詔嘉答,進克用諸道兵馬都招討使,命二嗣王兄事之,令促討行瑜。克用請帝還京師,以二千騎衛乘輿。時宮室煨殘,駐尚書省,百官喪馬,克用進乘輿金具裝二駟,又上百乘給從官。進太師、兼中書令、邠寧四面行營都統。



 行瑜堅壁梨園,茂貞自率師三萬逼咸陽而屯。克用請帝責茂貞罷兵,因削官爵,願與河中共討之。帝詔弟事行瑜,貸茂貞,俾結好。硃詔賜魏國夫人陳氏。陳,襄陽人也,善書,帝所愛,欲急平賊,故予之。茂貞以兵援龍泉,克用使李罕之、李存審夜引兵劫其餉,援兵亡,行瑜潰而走,追殺萬計。行瑜入邠州,丐歸款,克用使史儼入其城。行瑜死慶州,傳自京師。帝悉論幕府官屬及諸子功,封爵之,克用賜號「忠貞平難功臣」,進封晉王。



 克用屯雲陽,遣李習吉入朝,且請與王珂悉力討茂貞,帝不許。克用私於使者曰:「叛根不除,憂未艾也。」天子發度支錢三十萬緡勞其軍。時鄆州硃宣兄弟為全忠所困,使來告,克用請道於魏救之。兵解復鬥,克用自將而往,使李存信率兵三萬與史儼等次於莘,為魏兵所破。克用怒,大略相、魏去。



 始,茂貞畏克用見討,修貢獻如籓臣。及克用還,絕貢獻,與韓建謀以兵入朝。帝懼,詔克用進衛京師。帝謀度河幸太原,遣延王入克用軍促迎天子。既次渭北,建固請幸華州。克用謂王曰:「患本於不斷,顧上自為之。」李存信攻魏,葛從周引眾三萬來援,戰洹水上,汴人夜坎諸野,哄合,克用子落落馬陷而顛,克用救之,亦顛;追兵迫,射之乃免。存信已傅魏城,克用並力,羅弘信以捉生逆戰,為克用所敗,追及郛,叩闔而還。於是陜州王珙攻河中,李嗣昭援珂,再戰再勝,珙圍解。



 帝使延王持節至太原,謂克用曰:「不用卿計,故逮此,無可言者。今我寄於華,百司群官無所托,非卿尚誰與憂?不則不復見宗廟矣!」王至太原,克用留累月,每大張飲,王必以舞屬克用,因陳國事,涕數行下,冀感動之。時劉仁恭據幽州,貳於克用,數召兵不應,克用以書讓之,仁恭得書,抵於地,遂顯絕。故克用內憂幽州,以好辭謝王,不復有西意。俄自將屯蔚州,會晨大雺冥,仁恭來薄戰,克用大敗,走太原,大將多死。



 全忠奪邢、磁、洺三州,茂貞度克用沮橈,無能出師,乃與韓建謾好,致書言帝暴露累年,請共治宮室迎天子。初,長安自石門之奔,宮殿焚圮,及岐人再逆,火閭里皆盡,宮城昏夜狐貍鳴啼,無人跡。帝幸華西溪,望舊京必泫然流涕,左右淒塞不得語。王建方盜兩川,茂貞欲披其鄙私之,數南師,不暇東,而全忠繕治洛陽,茂貞因約克用共其勞,克用辭窮,乃出貲為助。



 光化初,帝還京師,詔克用與全忠解仇,宰相徐彥若、崔胤皆勸之。克用勢已折,然尚以功高位全忠上,恥先下之,時王鎔方睦於汴,乃遺書鎔,使為己倡。全忠即遣使奉書幣恭甚,克用亦報之。然汴日益張,窮鬥不置。王珙請汴兵攻河中,克用使李嗣昭、張漢瑜援之,汴兵走。葛從周取承天軍,氏叔琮取遼州、樂平,進壁榆次,克用使周德威逐出之。李嗣昭以步騎三萬下太行,略河內,拔懷州,進攻河陽,汴人閻寶救之,嗣昭退保懷。天復元年,全忠取晉、絳,逼河中,王珂告急,使相望,汴人扼空道,晉兵不得前,遂虜珂。珂妻,克用女,不能救,全忠遂有河中,克用朝貢道亦梗。



 全忠知克用迮不振,乃大舉攻太原,分遣銳將氏叔琮等率魏博、兗鄆、邢洺、義武、晉絳兵環入之,晉城邑多下。會大雨,汴兵糧乏,士瘧癘,遂解。克用雖內憤悒,憚全忠強難與爭,乃厚致幣馬謝,復請修好。全忠遂取同、華,屯渭上。帝如鳳翔,李茂貞、韓全誨請召克用入衛。克用間道遣使者奔問,並詒書全忠勸還汴,全忠不答。



 克用率兵趨平陽,攻吉上堡,破汴軍於晉州。李嗣昭、周德威下慈、隰,進屯河中。汴將硃友寧以兵十萬壁其南,全忠自屯晉州。晉人聞全忠至,皆失色。時有虹貫德威營,氏叔琮薄壘疾鬥,晉兵大敗,仗械輜儲皆盡。友寧長驅略汾、慈、隰州,皆下,遂圍太原,攻西門。德威、嗣昭循山挈餘眾得歸,克用大恐,身荷版築,率士拒守,陰於嗣昭、德威謀奔雲州。李存信曰:「不如依北蕃。」國昌妻劉語克用曰:「聞王欲委城入蕃,審乎?計誰出?」曰:「存信等為此。」劉曰:「彼牧羊奴,安辦遠計。王常笑王行瑜失城走而死,若何效之?且王頃居達靼,危不免。必一朝去此,禍不旋跬,渠能及北虜哉?」克用悟,乃止。居數日,散士復集。嗣昭夜擾友寧營,汴人驚,引去。德威追之,抵白壁關,復收慈、隰、汾三州。三年,克用攻晉州,聞帝自鳳翔還京師,乃去。雲州都將王敬暉殺刺史劉再立,以地予劉仁恭;李嗣昭討之。仁恭援敬暉,嗣昭壁樂安,欲戰,仁恭取敬暉,棄城去。



 帝東遷,詔至太原,克用泣謂其下曰:「乘輿不復西矣。」遣使者奔問行在,俄加號「協盟同力功臣。」李茂貞、王建與邠州楊崇本遣使者來約義舉,克用顧籓鎮皆附汴,不可與共功,惟契丹阿保機尚可用,乃卑辭召之。保機身到雲中,與克用會,約為兄弟,留十日去,遺馬千匹、牛羊萬計,期冬大舉度河,會昭宗弒而止。四年,王建、李茂貞約克用大舉。建將康晏步騎二萬與克用監軍張承業會鳳翔,是時汴將王重師守長安,劉知俊守同州,與戰長安西,建兵敗,遂不振。



 唐亡,建與淮南楊渥請克用自王一方,須賊平訪唐宗室立之。建請悉蜀工制乘輿御物。克用答曰:「自王,非吾志也。」建又勸茂貞王岐,茂貞孱褊,亦不敢當,但侈府第、僭宮禁而已。建、渥乃自王。是歲,克用有疾,城門自壞,明年卒。



 贊曰:沙陀始歸命天子,仰哺於邊,世喋血助征討,常為邊兵雄。至克用逢王室亂,遂有太原。虜性惇固,少它腸,自負材果,欲經營天下而不克也。兵雖勝,然數敗;地雖得,輒復失,故熟視帝劫遷,縮頸羞汗,偷景待殭,不亦鄙乎!賴其子慓銳,抑而復振。是時,提兵托勤王者五族,然卒亡硃氏為唐滌恥者,沙陀也。使克用稍知古今,能如齊桓、晉文,唐遽亡乎哉?



\end{pinyinscope}