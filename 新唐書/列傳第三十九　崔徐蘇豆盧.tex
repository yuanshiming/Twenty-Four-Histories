\article{列傳第三十九 崔徐蘇豆盧}

\begin{pinyinscope}

 崔融,字安成,齊州全節人。擢八科高第。累補宮門丞、崇文館學士。中宗為太子時,選侍讀,典東朝章疏。武后幸嵩高不能與之混同。隨著實踐和科學的發展,人們對具體物質形,見融銘《啟母碣》,嘆美之。及已封,即命銘《朝覲碑》。授著作佐郎,遷右史,進鳳閣舍人。時有司議關市,行人盡征之,融上疏謂:「《周官》九賦,其七曰關市。以市多淫巧,而關通末游,欲止抑之,故加稅耳。然唯斂工商,而不及往來。今一切通取,則事不師古。且四人異業舊矣,復動而搖之。市者,兼受善惡也。若甚,則細人無所容,細人無所容,久必為亂。天下之關必險道,市必要津,豪宗、惡少在焉,聞一旦變法,或致騷動,恐南走蠻,北走狄。今江津、河滸列鋪率稅,檢覆稽留,加主司僦略邀丐,則商人廢業。魏、晉、齊、隋所不行,況陛下乎?有如師興費廣,雖倍算商旅、加斂齊人可也。」後納之。



 張易之兄弟頗延文學士,融與李嶠、蘇味道、麟臺少監王紹宗降節佞附。易之誅,貶袁州刺史。召授國子司業。與脩《武後實錄》勞,封清河縣子。融為文華婉,當時未有輩者。朝廷大筆,多手敕委之,其《洛出寶圖頌》尤工。譔《武后哀冊》最高麗,絕筆而死,時謂思苦神竭雲。年五十四。贈衛州刺史,謚曰文。膳部員外郎杜審言為融所獎引,為服緦麻。



 六子,其聞者禹錫、翹。禹錫,開元中,中書舍人,贈定州刺史,謚曰貞。翹,禮部尚書,贈荊州大都督,謚曰成。



 孫巨,右補闕,亦有文。



 曾孫從。從字子乂,少孤貧,與兄能偕隱太原山中。會歲饑,拾橡實以飯,講學不廢。擢進士第。從山南嚴震府為推官,以母喪免。兄弟廬墓,手藝松柏。喪闋,不應闢命。久之,韋皋引為西山運務使。奏遷判官,攝守邛州。前刺史有以盜系獄,辭已具。從疑其冤,縱不治,俄得真盜。皋卒,劉闢反,欲並東川。從以書諭止闢,闢怒,從乃募兵嬰城守。闢方悉兵拒高崇文,戰而敗,從完州自如。盧坦表宣州副使。



 入為殿中侍御史,遷吏部員外郎。異時,史給選者成牒,以先後丐賕,從一限出之,後遂為法。裴度為御史中丞,奏以右司郎中知雜事。度已相,代為中丞。所彈治,不屈權幸。事系臺閣而付仗內者,必請還有司。薦引御史,務取質重廉退者。李翛以寵得京兆尹,為莊憲太后山陵橋道使,務以減末徭費為功,至不治道輴車留渭橋,久不得進。從三劾之,無少貸。



 俄授陜虢觀察使。遷尚書右丞。王承宗請割德、棣而遣子入侍也,憲宗選堪使者,以命從。議者謂承宗狠譎,非單使可屈。次魏,田弘正請以五百騎從,辭之,惟童騎十數,疾趨鎮。集軍士球場宣詔,為陳逆順大節禍福之效,音辭暢厲,士感動,承宗自失,貌愈恭,至泣下,即按二州戶口、符印上之。還為山南西道節度使。帝欲遂相,監軍使揣知,為用事者求金,從不肯答,用是不得相。長慶初,繇尚書左丞領鄜坊節度。屬部多神策屯軍,數亂法驕橫,吏不能制,從一繩以法,下皆重足畏之。黨項互市羊馬,類先遺帥守,從獨不取,而厚慰待之,羌不敢盜境。寶歷初,為東都留守。故事,留司官入宮城門列晨衙見留守。吏誕傲,久廢,至是復行。



 召拜戶部尚書。宰相李宗閔以從裴度、李德裕所善,內不喜。從求致仕,除太子賓客,分司東都,告滿百日去。於是眾嘩語不平,宗閔懼,復授檢校尚書左僕射、淮南節度副大使,知節度事。揚州凡交易貲產、奴婢有貫率錢,畜羊有口算,又貿曲牟其贏,以佐用度,從皆蠲除之。官吏俸帛常加倍以給,獨節度使則否,從皆與之同。大和六年卒,年七十二。下有刲股肉以祭者。贈司空,謚曰貞。



 從為人嚴偉,立朝棱棱有風望,不喜交權利,忠厚而讓。階品當立門戟,終不請。位方鎮,內無聲妓娛玩。士大夫賢之。



 能,字子才。硃泚之亂,渾瑊以朔方軍戰武功,引佐幕府。進累侍御史。河東鄭儋表為判官。累遷黔中觀察使,以讒坐貶。從為中丞,奏以自代。繇將作監授嶺南節度使,與從皆秉節居鎮,世傳為榮。卒,年六十八,贈禮部尚書。



 從子慎由、安潛。能子彥曾。



 慎由,字敬止。聰警強記,資端厚,有父風採。繇進士第擢賢良方正異等。鄭滑高銖闢府判官。入為右拾遺,進翰林學士。授湖南觀察使。召還,由刑部侍郎領浙西。入遷戶部侍郎,判戶部。始,慎由苦目疾,不得視,醫為治刮,適愈而召。



 俄進工部尚書、同中書門下平章事。與蕭鄴有隙,鄴輔政,引劉彖,而出慎由為東川節度使。初,宣宗餌長年藥,病渴且中躁,而國嗣未立。帝對宰相欲肆赦,患無其端。慎由曰:「太子,天下本。若立之,赦為有名。」帝惡之,不答。鄴等乘是譖去之,時大中十二年也。



 咸通初,徙華州刺史,改河中節度使。以吏部尚書請老,授太子太保,分司東都。卒,贈司空,謚曰貞。子胤,別傳。



 安潛,字進之。進士擢第。咸通中,歷江西觀察、忠武節度使。乾符初,王仙芝寇河南,安潛募人增陴繕械,不以力費仰朝廷。首請會兵討捕,號令精明,賊畏之,不犯陳許境。使大將張自勉將兵七千援宋州。時宋威屯曹州,而官軍數卻,賊圍宋益急。自勉收南月城,斬賊二千級,仙芝夜解去。宰相鄭畋建言:「請以陳許兵三千隸宋威。」而威忌自勉,乞盡得安潛軍,使自勉隸麾下。畋謂威有疑忿,必殺自勉,奏言:「今以兵悉畀威,是自勉以功受辱。安潛抗賊有功,乃取銳兵付威,後有緩急,何以戰?是勞不蒙賞,無以示天下。」詔止以四千付威,餘還自勉。



 俄代高駢領西川節度。吏倚駢為奸利者,安潛皆誅之,數更除繆政,於是盜賊衰,蜀民以安。宰相盧攜素厚駢,乃誣以罪,罷為太子賓客,分司東都。



 僖宗避賊劍南,召為太子少師。王鐸任都統,表以自副。鐸解兵,安潛復為少師、東都留守。青州王敬武卒,詔拜平盧節度使,檢校太師兼侍中。會敬武子師範專地,不得入而還。後遷太子太傅。卒,贈太子太師,謚貞孝。



 安潛於吏事尤長,雖位將相,閱具獄,未嘗不身聽之。



 彥曾,咸通初,繇太僕卿為徐州觀察使。曉律令,然卞急,為政剛猛。徐軍素驕,而彥曾長於撫民,短治軍,士多怨之。



 初,蠻寇五管,陷交趾,詔節度使孟球募兵三千往屯,以八百人戍桂林。舊制,三年一更。至期請代,而彥曾親吏尹戡、徐行儉貪不恤士,乃議稟賜乏,請無發兵,復留屯一年。戍者怒,殺都將王仲甫,脅糧料判官龐勛為將,取庫兵,剽湘、衡,虜丁壯,合眾千餘北還,自浙西趨淮南,達泗口。所過先遣俳兒弄木偶,伺人情,以防邀遏。彥曾命牙將田厚簡慰勞,而用都虞候元密伏甲任山館擊賊。勛遣吏紿言士思歸,不敢遏,請至府解甲自歸,彥曾斬其吏。勛陷宿州,發廥錢募兵,亡命者從亂如歸,船千艘,與騎夾岸,噪而進。彥曾料丁男乘城。或勸率眾奔兗州,彥曾曰:「我,方帥也,奉命守此,惟有死爾。」斬議者一人號於眾。俄而勛傅城,城中大霧如墮。彥曾悉誅賊家屬,勛眾四面超墉入,囚彥曾大彭館。有曹君長者說勛曰:「貴者不並處,今朝廷未以留後命公,蓋觀察使存爾。」勛乃殺彥曾於寢,自監軍使逮官屬皆死。始,彥曾治第鄭州,引水灌沼,水十步忽化為血。署張佛筵,液蜜為人,一昔鼠嚙皆斷首。徐有子亭,下瀦水為沱,彥曾導清河灌之,鐫石龍首注溜,蔽以屋。徐人謂屋覆龍,於文為「龐」;清河,崔望也,為吞噬云。贈刑部尚書。乾符中,錄其子祐之為滎陽尉。



 徐吏有路審中者,彥曾知其能,頗任之。既遇害,賂守卒,斂藏其尸。張玄稔攻徐州;審中率死士應官軍,開南白門,官兵入,因得破勛。後位嵐州刺史。鄭畋謂審中節貫神明,請擢為右羽林將軍,詔可。



 有許鐸者,罷武城令,客於徐,勛脅以官,不從。彥曾官屬被囚,鐸潛饋資糧,及死,為收瘞,匿免其子弟,賊平,乃皆歸其喪。詔拜石首令,賜銀緋。僚官焦璐、溫廷皓、李棁、崔蘊、柳秦、盧崇嗣、韋廷範贈官有差,錄其子官之。



 徐彥伯,兗州瑕丘人,名洪,以字顯。七歲能為文。結廬太行山下。薛元超安撫河北,表其賢,對策高第。調永壽尉、蒲州司兵參軍。時司戶韋皓善判,司士李亙工書,而彥伯屬辭,時稱「河東三絕」。遷職方員外郎,奉迎中宗房州,進給事中。武后撰《三教珠英》,取文辭士,皆天下選,而彥伯、李嶠居首。遷宗正卿,出為齊州刺史。帝復位,改太常少卿。以脩《武後實錄》勞,封高平縣子。為衛州刺史,政善狀,璽書嘉勞。移蒲州,以近畿,會郊祭,上《南郊賦》一篇,辭致黃縟。擢脩文館學士、工部侍郎。歷太子賓客。以疾乞骸骨,許之。開元二年卒。



 彥伯事寡嫂謹,撫諸侄同己子。秉筆累朝,後來翕然慕仿。晚為文稍強澀,然當時不及也。



 始,武后時,大獄興,王公卿士以語言為酷吏所引,死徙不可計。彥伯著《樞機論》以謂:「言者,德之柄,行之主,志之端,身之文也。君子之樞機,動則物應,得失之見也。可以濟身,亦以覆身,否泰榮辱一系之。能審思而應,精慮而動,擇其交以後談,則悔吝何由而生?怨惡何由而至?如此乃可以言也。」以為戒世云。



 蘇味道,趙州欒城人。九歲能屬辭,與里人李嶠俱以文翰顯,時號「蘇李」。逮冠,州舉進士,中第。累調咸陽尉。吏部侍郎裴行儉才之,會征突厥,引管書記。裴居道為左金吾衛將軍,倩味道作章,攬筆而具,閑徹清密,當時盛傳。



 延載中,以鳳閣舍人檢校侍郎、同鳳閣鸞臺平章事,歲餘為真。證聖元年,與張錫俱坐法系司刑獄。錫雖下吏,氣象自如,味道獨席地飯蔬,為危惴可憐者。武后聞,放錫嶺南,才降味道集州刺史。召為天官侍郎。聖歷初,復以鳳閣侍郎、同鳳閣鸞臺三品。更葬其親,有詔州縣治喪事。味道因役庸過程,遂侵毀鄉人墓田,蕭至忠劾之,貶坊州刺史。遷益州大都督府長史。張易之敗,坐黨附,貶眉州刺史。復還益州長史,未就道卒,年五十八,贈冀州刺史。



 味道練臺閣故事,善占奏。然其為相,特具位,未嘗有所發明,脂韋自營而已。常謂人曰:「決事不欲明白,誤則有悔,摸棱持兩端可也。」故世號「摸棱手」。性友愛。其弟味元,味元嘗請托不遂,因慢折之,味道怡然不屑。所論著行於時。



 豆盧欽望,雍州萬年人。祖寬,隋文帝外孫,為梁泉令。高祖定關中,與郡守蕭瑀率豪姓進款。擢累殿中監。子懷讓,尚萬春公主。詔寬用魏太和詔,去「豆」姓,著「盧」。貞觀中,遷禮部尚書、左衛大將軍,芮國公。卒,贈特進、並州都督,陪葬昭陵,謚曰定。復其舊姓。



 欽望累官越州都督、司賓卿。長壽二年,拜內史,封芮國公。李昭德被罪,有司劾奏欽望阿順昭德不執正,附臣罔君,貶趙州刺史。入為司府卿,遷秋官尚書。中宗還東宮,拜太子宮尹。進文昌右相、同鳳閣鸞臺三品。罷為太子賓客。帝復位,擢尚書左僕射、平章軍國重事。欽望居宰相積十餘年,方易之、三思等怙勢宣烝,窺間王室,戮忠戚,觖冀非常,不能有所裁抑,獨謹身諄諄自全。進開府儀同三司,檢校安國相王府長史。卒,年八十,贈司空、並州大都督,陪葬乾陵,謚曰元。



 武后時,宰相又有史務滋、崔元綜、周允元,略可述者附左方。



 史務滋,宣州溧陽人。累吏勞,遷司賓卿,進拜納言。後革命,詔務滋等十人分行天下。雅州刺史劉行實兄弟為侍御史來子珣誣其反,詔務滋與來俊臣雜治,俊臣言務滋與囚善,掩其反狀,後命俊臣並治,遂自殺。



 崔元綜,鄭州新鄭人。祖君肅,武德中為黃門侍郎、鴻臚卿。元綜,天授初以鸞臺侍郎、同鳳閣鸞臺平章事。性恪慎,坐政事堂,束帶,終日不休偃,尤護細概。外若謹厚,而中刻薄。每受制鞫獄,必澡垢索疵,不入死不肯止,人畏鄙之。未幾,坐事流振州,搢紳為慶。會赦還,除監察御史。遷蒲州刺史,致仕。善攝生,年九十餘卒。



 周允元字汝良,豫州安城人。自右肅政御史中丞,拜檢校鳳閣侍郎、同鳳閣鸞臺平章事。武後宴宰相,詔陳書傳善言,允元曰:「恥其君不如堯、舜。」武三思劾奏語指斥,後曰:「聞其言足以誡,安得為過?」卒,贈貝州刺史。



\end{pinyinscope}