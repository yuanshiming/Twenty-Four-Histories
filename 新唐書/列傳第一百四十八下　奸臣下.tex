\article{列傳第一百四十八下 奸臣下}

\begin{pinyinscope}

 盧杞,字子良。父弈,見《忠義傳》。杞有口才,體陋甚,鬼貌藍色理論傾向的哲學家的統稱。他們認為,真正的知識不是來自,不恥惡衣菲食,人未悟其不情,咸謂有祖風節。藉廕為清道率府兵曹參軍,僕固懷恩闢朔方府掌書記,病免。補鴻臚丞,出為忠州刺史。上謁節度府衛伯玉,伯玉不喜,乃謝歸。稍遷吏部郎中,為虢州刺史。奏言虢有官豕三千為民患。德宗曰:「徙之沙苑。」杞曰:「同州亦陛下百姓,臣謂食之便。」帝曰:「守虢而憂它州,宰相材也。」詔以豕賜貧民,遂有意柄任矣。俄召為御史中丞,論奏無不合。逾年遷大夫,不閱旬,擢門下侍郎、同中書門下平章事。



 既得志,險賊浸露。賢者媢,能者忌,小忤己,不傅死地不止。將大樹威,脅眾市權為自固者。楊炎與杞俱輔政,炎鄙杞才下,不悅,未半歲,譖罷炎。時大理卿嚴郢與炎有隙,即擢郢御史大夫以自助,炎卒逐死。張鎰材裕忠懿,帝所倚愛,未有以間。會隴右用兵,杞乃見帝,偽請行,帝不可,即薦鎰守鳳翔。既又惡郢。時幽州硃滔與泚有違言,誣其軍司馬蔡廷玉間鬩,請殺之。俄而滔反,帝欲斥之以悅滔,下御史鄭詹按狀,貶柳州司戶參軍,敕吏護送。廷玉疑送滔所,因自沈於河。杞奏,恐泚疑為詔所殺,願下詹三司雜治,並劾大夫郢。初,詹善張鎰,每伺杞間,獨詣鎰,杞知之。它日伺詹來,即徑至鎰便坐。詹趨避,杞遽及機事,鎰不得已,曰:「鄭侍御在。」杞陽驚曰:「向所言,非外所得聞。」至是並按。有詔詹杖死,流郢費州。杜佑判度支,帝尤寵禮。杞短毀百緒,訖貶蘇州刺史。李希烈反,杞素惡顏真卿挺正敢言,即令宣慰其軍,卒為賊害。故宰相李揆有雅望,畏復用,遣為吐蕃會盟使,卒於行。李洧以徐州降,有所經略,使人誤先白鎰,杞怒,沮解之,不使有功。其狙害隱毒,天下無不痛憤,以杞得君,故不敢言。



 是時兵屯河南、北,挐不解,財用日急。於是度支條軍所仰給,月費緡百餘萬,而藏錢才支三月。杞乃以戶部侍郎趙贊判度支,其黨韋都賓等建言:「商賈儲錢千萬,聽自業;過千萬者,貣其贏以濟軍。軍罷,約取償於官。」帝許之。京兆暴責其期,校吏頸大搜廛里,疑占列不盡,則笞掠之,人不勝冤,自殞溝瀆者相望,京師囂然不闋日。然悉田宅奴婢之直,緡止八十萬。又僦、質舍、居貿粟者,四貣其一,僅至二百萬。而長安為閉肆,民皆邀宰相祈訴。杞無以諭,驅而去。帝知民愁忿,而所得不足給師,罷之。贊術窮,於是間架、除陌之暴縱矣。其法:屋二架為間,差稅之,上者二千,中千,下五百,吏執籌入第室計之,隱不盡,率二架抵罪,告者以錢五萬畀之。凡公私貿易,舊法率千錢算二十,請加五十,主儈注所售,入其算有司;其自相市,為私籍自言,隱不盡,率千錢沒二萬,告者以萬錢畀之。由是主儈得操其私以為奸,公上所入常不得半,而恨誹之聲滿天下。及涇師亂,呼於市曰:「不奪而商人僦質矣,不稅而間架、除陌矣!」其倡和造作以召怨挻亂,皆杞為之。



 帝出奉天,杞與關播從。後數日,崔寧自賊中來,以播遷事指杞,杞即誣寧反,帝殺之。靈武杜希全率鹽、夏二州士六千來赴,帝議所從道,杞請道漠穀。渾瑊曰:「不然,彼多險,且為賊乘,不如道乾陵北,逾雞子堆而屯,與為掎角,賊可破矣。」帝從杞議,賊果拒隘,兵不得入,奔還邠州。



 李懷光自河北還,數破賊,泚解去。或謂王翃、趙贊曰:「聞懷光嘗斥宰相不能謀,度支賦斂重,而京兆刻損軍賜,宜誅之以謝天下。方懷光有功,上必聽用其言,公等殆矣!」二人以白杞。杞懼,即譎帝曰:「懷光勛在宗社,賊憚之破膽,今因其威,可一舉而定。若許來朝,則犒賜留連,賊得裒整殘餘為完守計,圖之實難,不如席勝使平京師,破竹之勢也。」帝然之。詔懷光無朝,進屯便橋。懷光自以千里勤難,有大功,為奸臣沮間,不一見天子,內怏怏無所發,遂謀反,因暴言杞等罪惡。士議嘩沸,皆指目杞,帝始寤,貶為新州司馬。



 始,帝即位,以崔祐甫為相,專以道德導主意,故建中初綱紀張設,赫然有貞觀風。及杞相,乃諷帝以刑名繩天下,亂敗踵及。其陰害矯譎,雖國屯主辱,猶謷然肆為之。後雖斥,然帝念之不衰。及興元赦令,俄徙吉州長史。杞乃曰:「上必復用我。」貞元元年,詔拜饒州刺史。給事中袁高當行詔書,不肯草,白宰相曰:「杞反易天常,使萬乘播遷,幸赦不誅,又委大州,失天下望。」宰相不悅,乃召它舍人作制,高固執不得下。於是諫臣趙需、裴佶、宇文炫、盧景亮、張薦等眾對,極言杞罪四海共棄,今復用之,忠臣寒膺,良士痛骨,必且階禍。其言懇到。帝語宰相曰:「授杞小州可乎?」李勉曰:「陛下與大州亦無難,如四方之謗何?」乃詔為澧州別駕。後散騎常侍李泌見,帝曰:「高等論杞事,朕可之矣!」泌頓首賀曰:「比日外謂陛下漢之桓、靈,今乃知堯、舜主也。」帝喜。杞遂死澧州。



 初,尚父郭子儀病甚,百官造省,不屏姬侍。及杞至,則屏之,隱幾而待。家人怪問其故,子儀曰:「彼外陋內險,左右見必笑,使後得權,吾族無類矣!」



 崔胤,字垂休,宰相慎由子也。擢進士第,累遷中書舍人、御史中丞。喜陰計,附離權強,其外自處若簡重,而中險譎可畏。崔昭緯屢薦之,由戶部侍郎同中書門下平章事。方王珙兄弟爭河中,以胤為節度使,不得赴,半歲,復以中書侍郎留輔政。及昭緯以罪誅,罷為武安節度使。陸扆當國,時王室不競,南、北司各樹黨結籓鎮,內相凌脅。胤素厚硃全忠,委心結之。全忠為言胤有功,不宜處外,故還相而逐扆。



 光化初,昭宗至自華,務安反側,而胤陰為全忠地,俾擅兵四討。帝醜其行,罷為吏部尚書,復倚扆以相。會清海無帥,因拜胤清海節度使。始,昭緯死,皆王摶等白發其奸,胤坐是賜罷,內銜憾。既與摶同宰相,胤議悉去中官,摶不助,請徐圖之。及是不欲外除,即漏其語於全忠,令露劾摶交敕使共危國,罪當誅。胤次湖南,召還守司空、門下侍郎、平章事,兼領度支、鹽鐵、戶部使,而賜摶死,並誅中尉宋道弼、景務修,繇是權震天下,雖宦官亦累息。至是,四拜宰相,世謂「崔四入」。



 劉季述幽帝東內,奉德王監國,畏全忠強,雖深怨胤,不敢殺,止罷政事。胤趣全忠以師西,問所以幽帝狀。全忠乃使張存敬攻河中,掠晉、絳。神策軍大將孫德昭常忿閹尹廢辱天子,胤令判官石戩與游,乘間伺察。德昭飲酣必泣,胤揣得其情,乃使戩說曰:「自季述廢天子,天下之人未嘗忘,武夫義臣搏手憤惋。今謀反者特季述、仲先耳,它人劫於威,無與也。君能乘此誅二豎,復天子,取功名乎?即不早計,將有無之者。」德昭感寤,乃告以胤謀。德昭許諾,胤斬帶為誓。俄而季述、仲先誅,以功進司徒,不就,復輔政,並還使領。帝德之,延見或不名,以字呼之,寵遇無比。



 天復元年,全忠已取河中,進逼同、華。中尉韓全誨以胤與全忠善,恐導之翦除君側,乃白罷政事,未及免,倉卒挾帝幸鳳翔。胤怨帝見廢,不肯從,召全忠以兵迎天子,令太子太師盧渥率群臣迎全忠。始,全忠至華,遣幕府裴鑄奏事。帝不得已,聽來朝。至是胤為之謀,乃以兵迫行在。帝下詔趣還鎮,因詔遣渥等俱西。全忠上表具言:「向書詔皆出宰相,乃今知非陛下意,為所詿誤。師業入關,請得與李茂貞約釋憾以迎乘輿。」茂貞劾奏:「胤畜死士,用度支使榷利,令親信陳班與京兆府募兵保所居坊。天子出次,遣使者五輩往召,安臥不動,一奉表陳謝。」時帝見全忠表,亦大恚,因下詔顯責之,以工部尚書罷知政事,胤出居華州。



 初,天復後宦官尤屈事胤,事無不咨。每議政禁中,至繼以燭,請盡誅中官,以宮人掌內司事。韓全誨等密知之,共於帝前求哀。乃詔胤後當密封,無口陳。中官益恐,滋欲得其謀,乃求知書美人宗柔等內左右以刺陰事。胤計稍露,宦者或相泣無憀,不自安,劫幸之謀固矣。



 居華時,為全忠數畫丑計。全忠引兵還屯河中,胤迎謁渭橋,奉觴為全忠壽,自歌以箅酒。會茂貞殺全誨等,與全忠約和。帝急召之,墨詔者四、硃札三,皆辭疾。及帝出鳳翔,幸全忠軍,乃迎謁於道,復拜平章事,進位司徒,兼判六軍諸衛事,詔徙家舍右軍,賜帷帳器用十車。胤遂奏:「高祖、太宗無內侍典軍,天寶後宦人浸盛,德宗分羽林衛為左右神策軍,令宦者主之,以二千人為率。其後參掌機密,至內務百司悉歸中人,共相彌縫為不法,朝廷微弱,禍始於此。請罷左右神策、內諸司使、諸道監軍。」於是中外宦官悉誅,天子傳導詔命,只用宮人寵顏等。



 帝之在鳳翔,以廬光啟、蘇檢為相,胤皆逐殺之,分斥從幸近臣陸扆等三十餘人,惟裴贄孤立可制,留與偕秉政。帝動靜一決於胤,無敢言者。胤議以皇子為元帥,全忠副之,示褒崇其功。全忠內利輝王沖幼,故胤藉以請。帝曰:「濮王長,若何?」還禁中,召翰林學士韓偓以謀。偓陰佐胤,卒不能卻。全忠還東,到長樂,群臣班辭,胤獨至霸橋置酒,乙夜乃還。帝即召問:「全忠安否?」與飲,命宮人為舞劍曲,戊夜乃出,賜二宮人,固讓乃許。是時天子孤危,威令盡去,胤之劫持類如此。進侍中、魏國公。



 自鳳翔還,揣全忠將篡奪,顧己宰相,恐一日及禍,欲握兵自固,謬謂全忠曰:「京師迫茂貞,不可無備,須募軍以守。今左右龍武、羽林、神策,播幸之餘,無見兵。請軍置四步將,將二百五十人;一騎將,將百人。使番休遞侍。」以京兆尹鄭元規為六軍諸衛副使,陳班為威遠軍使,募卒於市。全忠知其意,陽相然許。胤乃毀浮圖,取銅鐵為兵仗。全忠陰令汴人數百應募,以其子友倫入宿衛。會為球戲,墜馬死,全忠疑胤陰計,大怒。時傳胤將挾帝幸荊、襄,而全忠方謀脅乘輿都洛,懼其異議,密表胤專權亂國,請誅之。即罷為太子少傅。全忠令其子友諒以兵圍開化坊第,殺胤,汴士皆突出,市人爭投瓦礫擊其尸,年五十一,元規、陳班等皆死,實天復四年正月。



 胤罷凡三日死,死十日,全忠脅帝遷洛,發長安居人悉東,徹屋木自渭循河下。老幼系路,啼號不絕,皆大罵曰:「國賊崔胤導全忠賣社稷,使我及此!」先是,全忠雖據河南,顧強諸侯相持,未敢決移國。及胤間內隙,與相結,得梯其禍,取朝權以成強大,終亡天下,胤身屠宗滅。世言慎由晚無子,遇異浮屠,以術求,乃生胤,字緇郎。及為相,其季父安潛唶曰:「吾父兄刻苦以持門戶,終為緇郎壞之!」



 崔昭緯字蘊曜,其先清河人。及進士第。至昭宗時仕浸顯,以戶部侍郎同中書門下平章事,居位凡八年,累進尚書右僕射。性險刻,密結中人,外連強諸侯,內制天子以固其權。令族人鋋事王行瑜邠寧幕府。每它宰相建議,或詔令有不便於己,必使鋋密告行瑜,使上書訾訐,己則陰阿助之。方是時,帝室微,人主若贅斿然。始,帝委杜讓能調兵食以討鳳翔,昭緯方倚李茂貞、行瑜為重,陰得其計,則走告之,激使稱兵向闕,遂殺讓能。後又導三鎮兵殺韋昭度等。帝性剛明,不堪忍,會誅行瑜,乃罷昭緯為右僕射。復請硃全忠薦己,又厚賂諸王,為所奏,貶梧州司馬,下詔條其五罪,賜死。行次江陵,使者至,斬之。鋋亦誅。



 柳璨字炤之,公綽族孫也。為人鄙野,其家不以諸柳齒。少孤貧,好學,晝採薪給費,夜然葉照書,強記,多所通涉。譏訶劉子玄《史通》,著《析微》,時或稱之。顏蕘判史館,引為直學士,由是益知名。遷左拾遺。昭宗好文,待李磎最厚,磎死,內常求似磎者。或薦璨才高,試文,帝稱善,擢翰林學士。



 崔胤死,昭宗密許璨宰相,外無知者。日暮自禁中出,騶士傳呼宰相,人皆大驚。明日,帝謂學士承旨張文蔚曰:「璨材可用,今擢為相,應授何官?」對曰:「用賢不計資。」帝曰:「諫議大夫可乎?」曰:「唯唯。」遂以諫議大夫同中書門下平章事。起布衣,至是不四歲,其暴貴近世所未有。裴樞、獨孤損、崔遠皆宿望舊臣,與同位,頗輕之,璨內以為怨。硃全忠圖篡殺,宿衛士皆汴人,璨一厚結之,與蔣玄暉、張廷範尤相得。既挾全忠,故朝權皆歸之。進中書侍郎、判戶部,封河東縣男。



 天祐二年,長星出太微、文昌間,占者曰:「君臣皆不利,宜多殺以塞天變。」玄暉、廷範乃與璨謀殺大臣宿有望者。璨手疏所仇媢若獨孤損等三十餘人,皆誅死,天下以為冤。全忠聞之,不善也。其後急於九錫,宣徽北院使王殷者構璨等,言其有貳,故禮不至。玄暉懼,自往辨解。全忠怒罵曰:「爾與柳璨輩沮我,不由九錫,作天子不得邪?」璨懼,即脅哀帝曰:「人望歸元帥矣,陛下宜揖讓以授終。」璨請自行,進拜司空,為冊禮使,即日進道。及玄暉死,而全忠恚璨背己,貶登州刺史,俄除名為民,流崖州,尋斬之。臨刑悔吒曰:「負國賊柳璨,死宜矣!」弟瑀、瑊皆榜死。



 玄暉者,少賤,不得其系著。事硃全忠為腹心。昭宗東遷,玄暉為樞密使。帝駐陜州,術家言星緯不常,且有大變,宜須冬幸洛。帝度全忠必篡,命衛官高瑰持帛詔賜王建,告以脅遷,且言:「全忠以兵二萬治洛陽,將盡去我左右,君宜與茂貞、克用、行密同盟,傳檄襄、魏、幽、鎮,使各以軍迎我還京師。」又詔全忠:「後方娠,須十月乃東。」全忠知帝有謀,遣寇彥卿趣迫。天子不得已,遂行。抵谷水,全忠盡殺左右黃門、內園小兒五百人,悉以汴兵為衛。初,全忠至鳳翔,侵邠州,節度使楊崇本降,質其家。崇本妻美,全忠與亂,故崇本怒。至是遣使者會克用、茂貞,南告趙匡凝及建,同舉兵問劫遷狀,全忠大懼。



 帝自出關,畏不測,常默坐流涕。玄暉與張廷範內言冋,必以告全忠。全忠恨帝無傳禪意,乃謀弒以絕人望,因令其屬李振諭玄暉。玄暉與龍武統軍硃友恭、氏叔琮夜選勇士百人叩行在,言有急奏,請見帝。宮門開,門留十士以守。至椒蘭院中,夫人裴貞一啟關,殺之,乃趨殿下。玄暉曰:「上安在?」昭儀季漸榮曰:「院使毋傷宅家,寧殺我!」士持劍入,帝聞,遽單衣走,環柱,遂弒之。漸榮以身蔽帝,亦死。復執後,後求哀。玄暉以全忠所弒者帝也,乃釋後。明日,宰相請對,日晏不出。玄暉矯遺詔,言帝夜與昭儀博,為貞一、漸榮所弒,出二人首。全忠自河中來朝,振曰:「晉文帝殺高貴鄉公,歸罪成濟。今宜誅友恭等,解天下謗。」全忠趨西內臨,對嗣天子自言弒逆非本謀,皆友恭等罪,因泣下,請討罪人。是時洛城旱,米斗直錢六百,軍有掠糴者,都人怨,故因以悅眾,執友恭、叔琮斬之。全忠邀九錫,玄暉自持詔趨汴言之。還洛不淹日,全忠矯詔收付有司車裂之,貶為兇逆百姓,焚尸都門外。



 廷範者,以優人為全忠所愛,扈東遷為御營使,進金吾衛將軍、河南尹。全忠欲以為太常卿,宰相裴樞持不可,繇是樞罷去。柳璨希旨下詔,責中外不得妄言流品清濁,卒用廷範太常卿。會天子將郊,以為修樂縣使,又與蘇楷等駁昭宗謚。全忠恚九錫緩也,王殷譖其與璨等祀天祁延唐祚,及玄暉死、璨誅,即貶廷範萊州司戶參軍,軒於河南市。



 叔琮亦汴州人,中和末隸感化軍,以騎士奮,性沈壯有膽力。從全忠擊黃巢陳、許間,名右諸將,得為親校。與時溥、硃宣戰,以多累表檢校尚書右僕射,為宿州刺史。攻趙匡凝於襄陽,不克。又與李克用戰洹水,遷曹州刺史。天復初,拔澤、潞,擊太原,授晉慈觀察使。全忠屯鳳翔,克用襲絳州,攻臨汾,叔琮以二壯士類沙陀者牧馬於原,與克用軍偕行,伺隙各禽一虜還。克用大驚,疑有伏,遂退屯蒲。會硃友寧以兵三萬來援,叔琮曰:「賊遁矣,無以立功。」乃潛師夜獵游騎,殺數百,進破其壘,俘斬萬級,收馬三千,遂長驅取汾州,轉戰薄太原而還。遷檢校司空,再進為保大軍節度使。



 全忠欲遷帝於洛,表為右龍武統軍。與弒帝,故全忠請貶白州司戶參軍,斬之。叔琮將死,呼曰:「硃溫賣我以取容天下,神理謂何?」



 友恭者,本李彥威也。壽州人,客汴州。殖財任俠,全忠愛而子畜之。領長劍都,積功,表為檢校尚書左僕射。乾寧中,授汝州刺史,檢校司空。楊行密侵鄂州,友恭將兵萬餘援杜洪,至江州,還攻黃州,入之,獲行密將,俘斬萬計。又襲安州,殺守將。遷潁州刺史、感化軍節度留後。帝東遷,為左龍武統軍,貶崖州司戶參軍。臨刑曰:「溫殺我,當亦滅族!」又語張廷範曰:「公行及此」云。



 贊曰:木將壞,蟲實生之;國將亡,妖實產之。故三宰嘯兇牝奪辰,林甫將蕃黃屋奔,鬼質敗謀興元蹙,崔、柳倒持李宗覆。嗚呼,有國家者,可不戒哉!



\end{pinyinscope}