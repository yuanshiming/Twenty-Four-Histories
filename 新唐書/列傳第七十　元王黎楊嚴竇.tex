\article{列傳第七十 元王黎楊嚴竇}

\begin{pinyinscope}

 元載,字公輔,鳳翔岐山人。父昇,本景氏。曹王明妃元氏賜田在扶風,昇主其租入賴於個人內心的體驗來把握。主張對社會歷史現象的研究,主,有勞,請於妃,冒為元氏。載少孤,既長,嗜學,工屬文。天寶初,下詔舉明莊、老、列、文四子學者,載策入高第,補新平尉。韋鑒監選黔中,苗晉卿東都留守,皆署判官,浸以名聞。至德初,江都採訪使李希言表載自副,擢祠部員外郎、洪州刺史。入為度支郎中,占奏敏給,肅宗異之。累遷戶部侍郎,充度支、江淮轉運等使。



 帝不豫,李輔國用事,輔國妻,載宗女也,因相締暱。會京兆尹缺,輔國白用載,載意屬國柄,固辭,輔國曉之,翌日,拜同中書門下平章事,領使如故。代宗立,輔國勢愈重,數稱其才,進拜中書侍郎、許昌縣子。載以度支繁浩,有吏事督責,損威寵,乃悉天下錢穀委劉晏。未幾,判天下元帥行軍司馬。



 盜殺李輔國,載陰與其謀。乃復結中人董秀,厚啖以金,使刺取密旨,帝有所屬,必先知之,探微揣端,無不諧契,故帝任不疑。華原令顧繇上封白發其私,帝方倚以當國,乃斥繇,除名為民。魚朝恩驕橫震天下,與載不葉,憚之,雖帝亦銜恚,乃乘間奏誅朝恩,帝畏有變,載結其愛將為助。朝恩已誅,載得意甚,益矜肆。時擬奏文武官功狀多謬舛,載虞有司駁正,乃請別敕授六品以下官,吏部、兵部即附甲團奏,不須檢勘,欲示權出於己。又與王縉請以河中為中都,裒關輔河東十州稅奉京師,選兵五萬屯中都,鎮御四方,杪秋行幸,上春還,可以避羌戎患。載以議入,即從,前敕所由吏於河中經圖宮殿,築私第。帝聞,惡之,置其議。



 初,四鎮北庭行營節度使寄治涇州,大歷八年,吐蕃寇邠寧,議者謂三輔以西無襟帶之固,而涇州散地不足守。載嘗在西州,具知河西、隴右要領,乃言於帝曰:「國家西境極於潘原,吐蕃防戍乃在摧沙堡,而原州界其間,草薦水甘,舊壘存焉,比吐蕃毀夷垣墉,棄不居,其右則監牧故地,巨塹長壕,重復深固。原州雖早霜不可蓺,而平涼在其東,獨耕一縣,可以足食。請徙京西軍戍原州,乘間築作,二旬可訖,貯粟一歲。戎人夏牧青海上,羽書比至,則我功集矣。徙子儀大軍在涇,以為根本,分兵守石門、木峽,隴山之關,北抵於河,皆連山峻險,寇不可越。稍置鳴沙縣、豐安軍為之羽翼,北帶靈武五城,為之形勢,然後舉隴右之地,以至安西,是謂斷西戎脛,朝廷高枕矣。」因圖上地形,使吏間入原州度水泉,計徒庸,車乘畚閘之器悉具。而田神功沮短其議,乃曰:「興師料敵,老將所難,陛下信一書生言,舉國從之,誤矣。」帝由是疑不決。



 載智略開果,久得君,以為文武才略莫己若。外委主書卓英倩、李待榮,內劫婦言,縱諸子關通貨賄。京師要司及方面,皆擠遣忠良,進貪猥。凡仕進乾請,不結子弟,則謁主書。城中開南北二第,室宇奢廣,當時為冠。近郊作觀榭,帳帟什器不徙而供。膏腴別墅,疆畛相望,且數十區。名姝異伎,雖禁中不逮。帝盡得其狀。載嘗獨見,帝深戒之,謷然不悛。客有賦《都盧尋橦篇》諷其危,載泣下而不知悟。會李少良上書詆其醜狀,載怒,奏殺少良,道路目語,不敢復議。載由是非黨與不復接,生平道義交皆謝絕。



 帝積怒,大歷十二年三月庚辰,仗下,帝御延英殿,遣左金吾大將軍吳湊收載及王縉,系政事堂,分捕親吏、諸子下獄。詔吏部尚書劉晏、御史大夫李涵、散騎常侍蕭昕、兵部侍郎袁騕、禮部侍郎常袞、諫議大夫杜亞訊狀,而責辨端目皆出禁中。遣中使臨詰陰事,皆服。乃下詔賜載自盡,妻王及子揚州兵曹參軍伯和、祠部員外郎仲武、校書郎季能並賜死,發其祖、父塚,棺棄尸,毀私廟主及大寧、安仁里二第,以賜百官署舍,破東都第助治禁苑。



 王氏,河西節度使忠嗣女,悍驕戾沓,載叵禁。而諸子牟賊,聚斂無涯藝,輕浮者奔走。爭蓄妓妾,為倡優褻戲,親族環觀不愧也。及死,行路無嗟隱者。籍其家,鐘乳五百兩,詔分賜中書、門下臺省官,胡椒至八百石,它物稱是。女真一,少為尼,沒入掖庭。德宗時,始告以載死,號踴投地,左右呵止,帝曰:「安有聞親喪責其哀殞乎?」命扶出。



 帝為太子也,實用載議。興元元年,詔復其官,聽改葬。故吏許初、楊晈、紀慆等合貲以葬,謚曰荒,後改曰成縱。載敗,董秀、卓英倩、李待榮、術者李季連悉論死。其它與載厚善坐貶者,若楊炎、王昂、宋晦、韓洄、王定、包佶、徐縯、裴冀、王紀、韓會等凡數十百人。



 英倩弟英璘,家金州,州人緣以授官者亦百餘,豪制鄉曲,聚無賴少年以伺變,恃載權,牧宰莫敢問。載誅,英璘盜庫兵據險以叛。詔發禁兵及山南西道兵二千討捕,刺史孫道平禽殺之。詔給復其州三年。



 李少良者,以吏治由諸帥府遷累殿中侍御史。罷,游京師,不見調,憤載不法,疏論其惡,帝留少良客省,欲究其事。其友韋頌者候之,漏言於陸珽。載召珽問知之,乃奏下少良御史臺,劾其漏禁中語,並與頌、珽論殺之。珽,善經子,與頌及少良善,又狎載子弟親黨,故載廉得其謀。初,載盛時,人皆疾厭之。大歷八年,有晉州男子郇謨以麻總發,持竹笥、葦席,行哭長安東市,人問之,曰:「我有字三十,欲以獻上,字言一事,即不中,以笥貯尸,席裹而棄之。」京兆以聞,帝召見,賜以衣,館內客省,問狀,多譏切載。其言「團」者,願罷諸州團練使,其言「監」者,請罷諸道監軍,大抵類此。先是,天下兵興,凡要州權署團練、刺史。載用事,授刺史者悉帶團練以悅人心,故謨指而刺云。



 王縉,字夏卿,本太原祁人,後客河中。少好學,與兄維俱以名聞。舉草澤、文辭清麗科上第,歷侍御史、武部員外郎。祿山亂,擢太原少尹,佐李光弼,以功加憲部侍郎,遷兵部。史朝義平,詔宣慰河北,使還有指,俄拜黃門侍郎、同中書門下平章事。進侍中,持節都統河南、淮西、山南東道諸節度行營事。辭侍中,加東都留守。歲餘,拜河南副元帥,損軍資錢四十萬緡,營完宮室。硃希彩殺李懷仙也,詔拜盧龍節度使,至幽州,委軍於希彩乃還。會辛云京卒,兼領河東節度,讓還河南副元帥、東都留守。太原將王無縱、張奉璋恃功,以縉儒者易之,不如律令,縉斬以徇,諸將股慄。再歲還,以本官復知政事。時元載專朝,天子拱手,縉曲意附離,無敢忤。又恃才多所狎侮,雖載亦疾其凌靳也。京兆尹黎幹數論執,載惡之,縉折幹曰:「尹,南方孤生,安曉朝廷事?」



 縉素奉佛,不茹葷食肉,晚節尤謹。妻死,以道政里第為佛祠,諸道節度、觀察使來朝,必邀至其所,諷令出財佐營作。初,代宗喜祠祀,而未重浮屠法,每從容問所以然,縉與元載盛陳福業報應,帝意向之。繇是禁中祀佛,諷唄齋薰,號「內道場」,引內沙門日百餘,饌供珍滋,出入乘廄馬,度支具稟給。或夷狄入寇,必合眾沙門誦《護國仁王經》為禳厭,幸其去,則橫加錫與,不知紀極。胡人官至卿監、封國公者,著籍禁省,勢傾公王,群居賴寵,更相凌奪,凡京畿上田美產,多歸浮屠。雖藏奸宿亂踵相逮,而帝終不悟,詔天下官司不得棰辱僧尼。初,五臺山祠鑄銅為瓦,金塗之,費億萬計。縉給中書符,遣浮屠數十輩行州縣,斂丐貲貨。縉為上言:「國家慶祚靈長,福報所馮,雖時多難,無足道者。祿山、思明毒亂方煽,而皆有子禍,僕固懷恩臨亂而踣,西戎內寇,未及擊輒去,非人事也。」故帝信愈篤。七月望日,宮中造盂蘭盆,綴飾鏐飲琲,設高祖以下七聖位,幡節、衣冠皆具,各以帝號識其幡,自禁內分詣道佛祠,鐃吹鼓舞,奔走相屬。是日立仗,百官班光順門,奉迎導從,歲以為常。群臣承風,皆言生死報應,故人事置而不脩,大歷政刑,日以堙陵,由縉與元載、杜鴻漸倡之也。



 性貪冒,縱親戚尼姏招納財賄,猥屑相稽,若市賈然。及敗,劉晏等鞫其罪,同載論死,晏曰:「重刑再覆,有國常典,況大臣乎!法有首從,不容俱死。」於是以聞,上憫其耄,不加刑,乃貶括州刺史。久之,遷太子賓客,分司東都。建中二年死,年八十二。



 黎幹,戎州人。善星緯術,得待詔翰林,擢累諫議大夫,封壽春公。自負其辯,沾沾喜議論。初,唐家郊祭天地,以高祖神堯皇帝配。寶應元年,杜鴻漸為太常卿、禮儀使,於是禮儀判官薛頎、集賢校理歸崇敬等共建:「神堯獨受命之主,非始封君,不得冒太祖配天地。景皇帝受封於唐,即商之契、周之後稷,請奉景皇帝配天地,於禮宜甚。」乾非之,乃上《十詰》、《十難》,傅經誼,抵鄭玄,以折頎、崇敬等,曰:「頎等引禘者至日祭天於圓丘,周人以遠祖配,今宜以景皇帝為始祖,配昊天圓丘。臣乾一詰:《國語》稱有虞氏、夏后氏並禘黃帝,商禘舜,周禘嚳。二詰:《商頌『《長發》,大禘也』。三詰:《周頌》『《雍》,禘太祖也』。四詰:《祭法》,虞、夏並禘黃帝,商、周俱禘嚳。五詰:《大傳》『不王不禘,王者禘其祖之所自出,以其祖配之』。六詰:《爾雅》『禘,大祭也』。七詰:《家語》『凡四代帝王所郊,皆以配天;所謂禘,五年大祭也』。八詰:盧植以『禘,祭名。禘,諦也,事取明諦,故云』。九詰:王肅言『禘,五年大祭』。十詰:郭璞亦云。此經傳先儒皆不言祭昊天於圓丘,根證章章,故臣謂禘止五年宗廟大祭,了無疑晦。」



 其《十難》,一曰:「《周頌》《雍》之序曰:『禘,祭太祖也。』鄭玄說『禘,大祭也。太祖,謂文王也』。《商頌》『《長發》,大禘也』玄曰;『大禘,祭天也。』商、周兩《頌》,同文異解,索玄之意,以禘加『大』,因曰『祭天』。臣謂《春秋》『大事於太廟』,雖曰『大』,得祭天乎?虞、夏、商、周禘黃帝與嚳,《禮》『不王不禘』,皆不言『大』,玄安得稱祭天乎?《長發》所頌,不及嚳與感生帝,故知不為祭天侑嚳明矣。商、周五帝大祭見於經者甚詳,而禘主廟,不主天。今背孔子之訓言,取玄之偏誼,誣繆祀典,不見其可。」



 二曰:「『不王不禘,王者禘其祖之所自出,以其祖配之』。此言惟天子當禘。如虞、夏出黃帝,商、周出嚳,以近祖配之。自出之祖無廟,及自外至。自外至者,同之天地,得主而止。又自出者在母亦然。《春秋傳》『陳,則我周之自出』。詎可謂出太微五帝乎?玄以一『禘』為三誼,在《祭法》則曰『祭昊天於圓丘』。在《春秋傳》則『郊以後稷配靈威仰』。在《商頌》曰『祭天』。在《周頌》則禘曰『大於四時祭,而小於祫』。本末駁舛,臆判自私,不足以訓。」



 三曰:「商、周之前,禘所自出。自漢、魏以來,曠千餘歲,其禮不講。蓋玄所說不當於經,不質於聖,先儒置之不用,是為棄言。」



 四曰:「今禮家行於世者,皆本玄學。臣請取玄之隙,還破頎等所建。頎等曰:『景皇帝為始祖,以配天。』按《王制》『天子七廟』。玄曰:『周禮也。太祖與文、武之祧,合親廟四而七。商氏六廟,契與湯合二昭二穆而六。』據玄,則夏不以鮌、顓頊、昌意為始祖,是又與玄乖背。自古未有以人臣為始祖者,唯商以契,周以稷。夫稷、契皆天子元妃子。簡狄吞玄鳥卵而生契,契佐禹有大功,舜封之商,其《詩》曰:『天命玄鳥,降而生商,宅殷土芒芒。』後稷母曰姜嫄,出野履巨跡而生稷,稷勤稼穡,堯舉為農師,舜封之邰,號曰后稷。其詩曰:『履帝武敏歆,攸介攸止。』『即有邰家室。』舜、禹有天下,契、稷在焉。《傳》曰:『功施於人則祀之,以死勤事則祀之。』契為司徒,而人輯睦,稷勤百穀而山死,皆在祀典。及子孫而有天下,故尊而祖之。」



 五曰:「既用玄說,小德配寡,而後稷止配一帝,不得全配五帝。今以景帝配昊天,於玄為可為不可乎?」



 六曰:「眾詰臣曰:『上帝一帝,《周官》:祀天旅上帝,祀地旅四望。旅,眾也。則上帝是五帝。』臣曰:『否,旅有眾義,出於《爾雅》。又為祭名,亦曰陳也。如前所詰,旅上帝為五帝,則季氏旅於泰山可得為四鎮邪?』」



 七曰:「援玄之言,則景帝親盡,主應在祧,反配天地,禮不相值。夫所謂始祖者,經綸草昧,功普體大,以比元氣含覆廣大者也。故曰萬物之始,天也;人之始,祖也;日之始,至也。掃地而祭,則質;器用陶匏,則性;牲用犢,則誠;兆於南郊,則就陽。至尊至質,不敢同於先祖也。」《白虎通議》曰:『祭天歲一者何?事之不敢黷也。』故因歲之陽氣始達而祭之。今一歲四祭,黷莫大焉。上帝五帝,祀闕不舉,怠孰甚焉?黷與怠,皆失也。臣聞親有限,祖有常,聖人制禮不以情變。唐家累聖,歷祀百年,非不知景帝為始封。當時通儒鉅工尊高祖以配天,宗太宗以配上帝,人神克厭,為日既久。乃今以神堯降侑含樞紐,而太宗仍配上帝,則樞紐上帝佐也。以子先父,非天地祖宗之意。」



 八曰:「景皇帝非造我區夏,不得與夏之禹、商之契、周之稷、漢高帝、魏武帝、晉宣帝、唐神堯皇帝並功,則陟配圜丘,上與天匹,曾謂圜丘不如林放乎?」



 九曰:「魏以武帝、晉以宣帝為始祖者,夫操與懿皆人傑也。擁天下強兵,挾弱主,制海內之命,名雖為臣,勢實為君,後世因之以成帝業,尊而祖之,不亦可乎?」



 十曰:「神堯拯隋室之亂,振臂大呼,濟人塗炭,汛掃蕩攘,群兇無餘,出入不數年而成王業,漢祖之功不能加焉。夏以禹,漢以高帝,我以神堯為始祖,訂夏法漢,於義何嫌?今頎、崇敬革天對,易祖廟,事之大者不稽於古,難以疑文僻說定之。臣官以諫為名,不敢不盡愚。」



 議聞,代宗不韙其言。其後名儒大議,而景帝配天卒著於禮。



 俄遷京兆尹,頗以治稱。京師苦樵薪乏,乾度開漕渠,興南山谷口,尾入於苑,以便運載。帝為御安福門觀之。乾密具同船作倡優水嬉,冀以媚帝。久之,渠不就。俄改刑部侍郎。魚朝恩敗,坐交通,出為桂管觀察使。大歷八年,復召為京兆尹。時大旱,乾造土龍,自與巫覡對舞,彌月不應。又禱孔子廟,帝笑曰:「丘之禱久矣。」使毀土龍,帝減膳節用,既而霪雨。十三年,涇水擁隔,請開鄭、白支渠,復秦、漢故道以溉民田,廢碾磑八十餘所。



 乾性貪暴,既復用,不暇念治,專徇財色,附會嬖近,挾左道希主恩,帝甚惑之。德宗在東宮,干與宦者特進劉忠翼陰謀,幾危宗嗣。及即位,又詭道希進,密乘車謁忠翼。事覺,除名長流,既行,市人數百群噪投礫從之,俄賜死藍田驛。



 忠翼本名清潭,與左衛將軍董秀皆有寵於代宗。當盛時,爵賞在其口吻,掊冒財賄,貲產累皆巨萬。至是,積前罪,並及誅。



 楊炎,字公南,鳳翔天興人。曾祖大寶,武德初為龍門令,劉武周攻之,死於守,贈全節侯。祖哲,以孝行稱。父播,舉進士,退居求志,玄宗召拜諫議大夫,棄官歸養。肅宗時,即家拜散騎常侍,號玄靖先生。炎美須眉,峻風宇,文藻雄蔚,然豪爽尚氣。河西節度使呂崇賁闢掌書記。神烏令李太簡嘗醉辱之,炎令左右反接,搒二百餘,幾死,崇賁愛其才,不問。李光弼表為判官,不應。召拜起居舍人,固辭。父喪,廬墓側,號慕不廢聲,有紫芝白雀之祥,詔表其閭。炎三世以孝行聞,至門樹六闕,古所未有。終喪,為司勛員外郎,遷中書舍人,與常袞同時知制誥。袞長於除書,而炎善德音,自開元後言制詔者,稱「常楊」云。宰相元載與炎同郡,炎又元出也,故擢炎吏部侍郎、史館脩撰。載當國,陰擇才可代己者,引以自近,初得禮部侍郎劉單,會卒,復取吏部侍郎薛邕,邕坐事貶,後得炎,親重無比。會載敗,坐貶道州司馬。



 德宗在東宮,雅知其名,又嘗得炎所為《李楷洛碑》,寘於壁,日諷玩之。及即位,崔祐甫薦炎可器任,即拜門下侍郎、同中書門下平章事。



 舊制,天下財賦皆入左藏庫,而太府四時以數聞,尚書比部覆出納,舉無干欺。及第五琦為度支、鹽鐵使,京師豪將求取無節,琦不能禁,乃悉租賦進大盈內庫。天子以給取為便,故不復出。自是天下公賦為人君私藏,有司不得計贏少。而宦官以冗名持簿者三百人,奉給其間,根柢連結不可動。及炎為相,言於帝曰:「財賦者,邦國大本,而生人之喉命,天下治亂重輕系焉。先朝權制,以中人領其職,五尺宦豎,操邦之柄,豐儉盈虛,雖大臣不得知,則無以計天下利害。陛下至德,惟人是恤,參計敝蠹,莫與斯甚。臣請出之,以歸有司。度宮中經費一歲幾何,量數奉入,不敢以闕。如此,然後可以議政,惟陛下審察。」帝從之。乃詔歲中裁取以入大盈,度支具數先聞。



 初,定令有租賦庸調法,自開元承平久,不為版籍,法度玩敝。而丁口轉死,田畝換易,貧富升降,悉非向時,而戶部歲以空文上之。又戍邊者,蠲其租、庸,六歲免歸。玄宗事夷狄,戍者多死,邊將諱不以聞,故貫籍不除。天寶中,王金共為戶口使,方務聚斂,以其籍存而丁不在,是隱課不出,乃按舊籍,除當免者,積三十年,責其租、庸,人苦無告,故法遂大敝。至德後,天下兵起,因以饑癘,百役並作,人戶凋耗,版圖空虛。軍國之用,仰給於度支、轉運使;四方征鎮,又自給於節度、都團練使。賦斂之司數四,莫相統攝,綱目大壞。朝廷不能覆諸使,諸使不能覆諸州。四方貢獻,悉入內庫,權臣巧吏,因得旁緣,公托進獻,私為贓盜者,動萬萬計。河南、山東、荊襄、劍南重兵處,皆厚自奉養,王賦所入無幾。科斂凡數百名,廢者不削,重者不去,新舊仍積,不知其涯。百姓竭膏血,鬻親愛,旬輸月送,無有休息。吏因其苛,蠶食於人。富人多丁者,以宦、學、釋、老得免,貧人無所入則丁存。故課免於上,而賦增於下。是以天下殘瘁,蕩為浮人,鄉居地著者百不四五。炎疾其敝,乃請為「兩稅法」以一其制。凡百役之費,一錢之斂,先度其數而賦於人,量出制入。戶無主客,以見居為簿;人無丁中,以貧富為差。不居處而行商者,在所州縣稅三十之一,度所取與居者均,使無僥利。居人之稅,秋夏兩入之,俗有不便者三之。其租、庸、雜徭悉省,而丁額不廢。其田畝之稅,率以大歷十四年墾田之數為準,而均收之。夏稅盡六月,秋稅盡十一月,歲終以戶賦增失進退長吏,而尚書度支總焉。帝善之,使諭中外。議者沮詰,以為租庸令行數百年,不可輕改。帝不聽。天下果利之。自是人不土斷而地著,賦不加斂而增入,版籍不造而得其虛實,吏不誡而奸無所取,輕重之權始歸朝廷矣。



 炎興嶺表,以單議悟天子,中外翕然屬望為賢相。居數月,崔祐甫疾,不能事,喬琳免,炎獨當國,遂多變祐甫之政,減薄護元陵功優,人始不悅。又請開豐州陵陽渠,發畿縣民役作,閭里騷然,渠卒不就。



 素德元載,思有以報之,於是復議城原州,節度使段秀實謂「安邊卻敵,宜以緩計,方農事,不可遽興功。」炎怒,追秀實為司農卿,以邠寧李懷光督作,遣硃泚、崔寧統兵各萬人翼之。詔書下,涇軍恚曰:「吾軍為國西屏十餘年。始自邠土,農桑地著之安,徙此榛莽中,手披足踐,既立城壘,則又投之塞外,且安寘此乎?」又懷光持法嚴,舉軍畏之。裨將劉文喜因人之怨,乃上疏求秀實、硃泚為使。詔以泚代懷光,文喜不奉詔,閉城拒守,質其子吐蕃以求援。時方煬旱,人情騷攜,群臣皆請赦文喜,帝不聽。詔減服御給軍,且趣師涇州,士當受春服者皆即賜。命泚、懷光率軍攻之,壘環其州。別將劉海賓斬文喜,獻其首,涇州平,而原卒不能城。又以劉晏劾載,已坐貶,乃出晏忠州,用庾準為荊南節度使,誣晏殺之,朝野側目。李正己表請晏罪,炎懼,乃遣腹心分走諸道:裴冀使東都、河陽、魏博,孫成使澤潞、礠邢、幽州,盧東美使河南、淄青,李舟使山南、湖南,王定使淮西。聲言宣慰,而實自辯解,言「晏往嘗傅會奸邪,謀立獨孤妃為後,帝自惡之,非它過也」。帝聞,使中人復其言於正己,還報信然,於是帝意銜之,未發也。



 會盧杞以門下侍郎同中書門下平章事,進炎中書侍郎,同秉政。杞無術學,貌麼陋,炎薄之,托疾不與會食,杞陰為憾。舊制,中書舍人分押尚書六曹,以平奏報。開元初,廢其職。杞請復之,炎固以為不可,杞益怒。又密啟主書過咎,逐之,炎曰:「主書,吾局吏也,吾當自治之,奈何相侵邪?」始,炎還朝,道襄、漢,因勸梁崇義入朝,後又使李舟邀說之,崇義益反側。及其叛,議者歸咎炎,以為趣成之。帝欲以淮西李希烈統諸軍致討,炎曰:「希烈始與李忠臣為子,逐忠臣取其位,此可以任乎?居無尺寸功,猶倔強不奉法,設使平賊,陛下將何以制之?」帝不能平,恚曰:「氍不能食吾言。」遂用希烈。又嘗訪群臣可大任者,杞薦張鎰、嚴郢,而炎舉崔昭、趙惠伯。帝以炎論議疏闊,遂罷為尚書左僕射。既謝,對延英訖,不至中書,杞怒,益欲中之。



 先是,嚴郢為京兆尹,不附炎,炎諷御史張著劾之,罷兼御史中丞。源休與郢不善,自流人擢休為京兆少尹,令伺郢過。休反與郢善,炎怒。會張光晟謀殺回紇酋帥,乃使休使回紇。郢坐度田不實,下除大理卿。至是炎罷,其子弘業賕賂狼藉,故杞引郢為御史大夫按之,並得它過。惠伯為河南尹時,嘗市炎第為官廨。御史劾炎宰相抑吏市私第,貴取其直。杞召大理正田晉評罪,晉曰:「宰相於庶官比監臨,計羨利,罪奪官。」杞怒,謫晉衡州司馬。於是當監主自盜,罪絞。開元時,蕭嵩嘗度曲江南,欲立私廟,以為天子臨幸處乃止,後炎復取以立廟。飛語云:「地有王氣,故炎取之。」帝聞,震怒,會獄具,詔三司同覆,貶崖州司馬同正。未至百里,賜死,年五十五。貶惠伯多田尉,亦殺之。



 初,炎矯飭志節,頗得名。既傅會元載抵罪,俄而得政,然忮害根中,不能自止。眥睚必讎,果於用私,終以此及禍。自道州還也,家人以綠袍木簡棄之,炎止曰:「吾嶺上一逐吏,超登上臺,可常哉?且有非常之福,必有非常之禍,安可棄是乎?」及貶,還所服。久之,詔復其官,謚肅愍,左丞孔駁之,更曰平厲。



 庾準者,常州人。無學術,以柔媚自進,得幸於王縉,驟至中書舍人,時流嗤薄之。再遷尚書右丞。縉得罪,出為汝州刺史。復入為司農卿。又善炎,故炎使節度荊南;晏已誣死,引為尚書左丞。建中三年卒,贈工部尚書。



 嚴郢,字叔敖,華州華陰人。父正誨,以才吏更七郡、終江南西道採訪使。郢及進士第,補太常協律郎,守東都太廟。祿山亂,郢取神主秘於家,至德初,定洛陽,有司得以奉迎還廟,擢大理司直。呂諲鎮江陵,表為判官。方士申泰芝以術得幸肅宗,遨游湖、衡間,以妖幻詭眾,奸贓鉅萬,潭州刺史龐承鼎按治。帝不信,召還泰芝,下承鼎江陵獄。郢具言泰芝左道,帝遣中人與諲雜訊有狀,帝不為然。御史中丞敬羽白貸泰芝,郢方入朝,亟辨之。帝怒,叱郢去。郢復曰:「承鼎劾泰芝詭沓有實,泰芝言承鼎驗左不存。今緩有罪,急無罪,臣死不敢如詔。」帝卒殺承鼎,流郢建州。泰芝後坐妖妄不道誅。代宗初,追還承鼎官,召郢為監察御史,連署帥府司馬。郭子儀表為關內、河東副元帥府判官,遷行軍司馬。子儀鎮邠州,檄郢主留務。河中士卒不樂戍邠,多逃還。郢取渠首尸之,乃定。歲餘,召至京師,元載薦之帝,時載得罪,不見用。御史大夫李棲筠亦薦郢,帝曰:「是元載所厚,可乎?」答曰:「如郢材力,陛下不自取,而留為奸人用邪?」即日拜河南尹、水陸運使。大歷末,進拜京兆尹。嚴明持法令,疾惡撫窮,敢誅殺,盜賊一衰,減隸官匠丁數百千人,號稱職尹。



 宰相楊炎請屯田豐州,發關輔民鑿陵陽渠,郢習朔邊病利,即奏:「舊屯肥饒地,今十不墾一,水田甚廣,力不及而廢。若發二京關輔民浚豐渠營田,擾而無利。請以內苑蒔稻驗之,秦地膏腴,田上上,耕者皆畿人,月一代,功甚易,又人給錢月八千,糧不在,然有司常募不能足。合府縣共之,計一農歲錢九萬六千,米月七斛二斗,大抵歲僦丁三百,錢二千八百八十萬,米二千一百六十斛,臣恐終歲獲不酬費。況二千里發人出塞,而歲一代乎?又自太原轉糧以哺,私出資費倍之,是虛畿甸,事空徭也。」郢又言:「五城舊屯地至廣,請以鑿渠糧俾諸城,夏貸冬輸,取渠工布帛給田者,令據直轉穀,則關輔免調發,而諸城闢田。」炎不許,渠卒不成,棄之。



 御史臺請天下斷獄一切待報,唯殺人許償死,論徒者得悉徙邊。郢言:「罪人徙邊,即流也。流有三,而一用之,誠難。且殺人外猶有十惡、偽造用符印、強光火諸盜,今一徙之,法太輕,不足禁惡。又罪抵徒,科別差殊,或毆傷、夫婦離非義絕、養男別姓、立嫡不如式、私度關、冒戶等不可悉,而與十惡同徙,即輕重不倫。又按,京師天下聚,論徒者至廣,例不覆讞,今若悉待報,有司斷決有程,月不啻五千獄,正恐牒按填委,章程紊撓。且邊及近邊犯死徒流者,若何為差?請下有司更議。」炎惡異己,陰諷御史張著劾郢匿發民浚渠,使怨歸上。系金吾。長安中日數千人遮建福門訟郢冤,帝微知之,削兼御史中丞。人知郢得原,皆迎拜。會秋旱,郢請蠲租稅,炎令度支御史按覆,以不實,罷為大理卿。



 炎之罷,盧杞引郢為御史大夫,共謀炎罪。即逮捕河中觀察使趙惠伯下獄,楚掠慘棘,鍛成其罪,卒逐炎崖州,惠伯費州。天下以郢挾宰相報仇為不直。然杞用郢敗炎,內忌郢才,因按蔡廷玉事,殺御史鄭詹,出郢為費州刺史。道逢柩殯,問之,或曰:「趙惠伯之殯。」郢內慚,忽忽歲餘卒。



 竇參,字時中,刑部尚書誕四世孫。學律令,為人矜嚴悻直,果於斷。以廕累為萬年尉。同舍當夕直者,聞親疾惶遽,參為代之。會失囚,京兆按直簿劾其人,參曰:「彼以不及謁而往,參當坐。」乃貶江夏尉,人皆義之。遷奉先尉。男子曹芬兄弟隸北軍,醉暴其妹,父救不止,恚赴井死。參當兄弟重闢,眾請俟免喪,參曰:「父繇子死,若以喪延,是殺父不坐。」皆榜殺之,一縣畏伏。



 進大理司直,按江淮獄揚州,節度使陳少游偃蹇不郊迎,遣軍吏致問,參厲辭譙讓,少游慚,往謁參,參不顧即去。婺州刺史鄧珽盜贓八千緡,宰相右珽,欲免輸其財,詔百官集尚書省議,多希意為助,參獨持法,卒輸入之。遷監察御史。湖南判官馬彞發部令贓千萬,令之子因權幸誣奏彞,參往按,直其侵衊。彞後佐曹王皋,以干直聞者也。



 入為御史中丞,舉劾無所回忌。德宗數召見,語天下事,或決大議,帝器之。然多與宰相駁異,數為排卻,卒無以傷。參由是無所憚,或率情制事矣。時定百官班稟,參嘗為大理司直,故多其入,使在丞上。惡詹事李昇,抑其班在諸府少尹下。中外稍惡其專。



 進兼戶部侍郎。民家生豕二首四足,有司欲以聞,參曰:「此乃豕禍。」屏不奏。陳少游死,子請襲封,參大署省門曰:「少游位將相,以艱危易節,上含垢不忍發,其息容得傳襲邪?」神策將軍孟華戰有功,或誣以反,龍武將軍李建玉陷吐蕃自拔歸,部曲告與虜通,皆論死。參悉治出之,人始屬望。



 俄以中書侍郎同中書門下平章事,領度支、鹽鐵使。每延英對,它相罷,參必留,以度支為言,實專政也。然參無學術,不能稽古立事,惟樹親黨,多所言冋察,四方畏之。於是淄青李納厚饋參,外示嚴畏,實賂帝親近為間,故左右爭毀短之。



 申,其族子也,為給事中,參親愛,每除吏多訪申,申因得招賂,漏禁密語,故申所至,人目為「喜鵲」。帝聞,以戒參,且曰:「是必為累,不如斥之。」參以情訴曰:「臣無強子姓,申雖疏屬,無它惡。」帝曰:「而雖自保,如外言何?」參固陳丐。



 初,陸贄與參不平,吳通玄兄弟皆在翰林,與贄軒輊不得,申舅嗣虢王則之與通微等善,遂共譖贄。帝得其奸,逐申為道州司馬。不浹日,貶參郴州別駕。宣武劉士寧餉參絹五千,湖南觀察使李巽故與參隙,以狀聞,又中人為之驗左,帝大怒,以為外交戎臣,欲殺參。贄雖怨,然亦以殺之太重,乃貶驩州司馬,逐其息景伯於泉州,女尼於郴州,沒入貲產奴婢。帝又欲殺申、則之及屬人榮,贄固爭:「法有首從,首原則從減。榮與參雖善,然初無邪僻,數激憤有直言,晚頗疏忌,請貶榮遠官,申、則之除名流嶺南。」詔可。時宦侍謗沮不已,參竟賜死於邕州,年六十。而杖殺申,免榮死,諸竇並逐雲。



 吳通玄者,海州人,與弟通微皆博學善文章。父道瓘,以道士詔授太子諸王經,故通玄等皆得侍太子游,太子待之甚善。始,通玄舉神童,補秘書正字。又擢文辭清麗科,調同州司戶參軍。德宗立,弟兄踵召為翰林學士。頃之,通微遷職方郎中,通玄起居舍人,並知制誥。凡帝有撰述,非通玄筆未嘗慊。



 與陸贄、吉中孚、韋執誼並位。贄文高有謀,特為帝器遇,且更險難,有功。通玄等特以東宮恩舊進,暱而不禮,見贄驟擢,頗媢恨。贄自恃勁正,屢短通玄於帝前,欲斥遠之,即建言:「承平時,工藝書畫之冗,皆待詔翰林而無學士,至德以來,命集賢學士入禁中草書詔,待進止於翰林院,因以名官。今四方無事,制書職分宜歸中書舍人,請罷學士。」帝不許。通玄怨日結,謀奪其內職。會贄權知兵部侍郎,主貢舉,乃命為真。貞元十年,通玄拜諫議大夫,自以久次,當得中書舍人,大怨望。贄與竇參交惡,參從子申從舅嗣虢王則之方為金吾將軍,故申介之使結通玄兄弟,共危贄。而通玄以宗室女為外婦,帝知,未及責。則之飛謗云:「贄試進士,受賄謝。」帝惡誣構,大怒,罷參宰相,逐則之昭州司馬,通玄泉州司馬。又銜淫污近屬事,自詰之,不敢答,賜死長城驛。贄遂相矣。



 通玄死,通微白衣待罪於門,帝宥之,內懼禍,不敢行喪服。



 贊曰:元載、楊炎各以才資奮,適主暗庸,故致位輔相。若其翦閹尹,城原州以謀西夏,還左藏有司,一租賦以檢制有亡,誠有取焉。然載本與輔國以利合,險刻著諸心,溪壑之欲,發乎無厭。炎牽連載勢,興醜裔,秉國維綱,返為載復讎,釋言於君,卒與妻子並誅,暴先骨,殛命於道,蓋自取之也。夫奸人多才,未始不為患,故酆舒以俊死,而鄧析以辯亡。若兩人者,所謂多才者邪!縉言福業報應,參得君自私,無可論者。《易》稱「鼎折足,其刑剭諒哉!



\end{pinyinscope}