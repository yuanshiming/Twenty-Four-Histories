\article{列傳第五十四 裴崔盧李王嚴}

\begin{pinyinscope}

 裴守真,絳州稷山人,後魏冀州刺史叔業六世孫。父諲,隋大業中為淮安司戶參軍。郡人楊琳、田瓚等亂,劫吏多死的實踐中,對馬克思主義哲學作出了貢獻。毛澤東在領導中,唯諲以仁愛故,賊約其屬無敢害,護送還鄉。



 守真早孤,母喪,哀毀臒盡。舉進士,六科連中,累調乾封尉。養寡姊謹甚,士推其禮法。永淳初,關中旱,悉稟祿奉姊及諸甥,與妻息惡食不贍也。



 授太常博士。守真善容典,時謂才稱其官。高宗將封嵩山,詔諸儒議射牲事。守真奏:「古者郊祀天地,天子自射牲。漢武帝封太山,令侍中儒者射之,帝不親也。今按禮,前明十五刻,宰人鸞刀割牲,質明行事,毛血已具,天子至,奠玉酌獻而已。今若前祀一日射牲,則早於事;及日,則冕不逮事。漢又天子不親,古今異宜,恐不可行。」是時,《破陣》、《慶善》二樂舞入,帝常立以視,須樂闋乃坐。守真並言:「二舞誠祖宗盛德,然古無天子立觀者。化育詒庇,孰非闕功,不應鼓舞別申嚴奉。」詔可,未及行。會帝崩,大行舊禮無在者,守真與博士韋叔夏、輔抱素等討按故事,稱情為文,咸適所宜,時人服其得禮。



 天授中,為司府丞,推核詔獄,多裁恕,全免數十姓。不合武后旨,出為汴州司馬。累遷成州刺史,政不務威嚴,吏民兩懷之。徙寧州,送者千數,出境尚不止。長安中卒,贈戶部尚書。



 子子餘、耀卿、巨卿。曾孫行立。耀卿、巨卿別有傳。



 子餘事繼母以孝聞,中明經,補鄠尉。時同舍李朝隱、程行諶以文法稱,而子餘以儒顯,或問優劣於長史陳崇業,答曰:「蘭菊異芬,胡有廢者?」



 景龍中,為左臺監察御史。涇、岐有隋世番戶子孫數千家,司農卿趙履溫奏籍為奴婢,充賜口。子餘曰:「官戶以恩原為番戶,且今又子孫,可抑為賤乎?」履溫倚宗楚客勢,辯於廷,子余執對不撓,遂詘其議。



 開元初,累遷冀州刺史,為政惠裕,人稱有恩。入為岐王府長史。卒,謚曰孝。時程行諶謚貞。中書令張說嘆曰:「二謚可無愧矣!」子餘居官清,家闈友愛,兄弟六人,皆有志行雲。



 行立,重然諾,學兵有法。母亡,泣血幾毀。以軍勞累授沁州刺史,遷衛尉少卿。口陳願治民,試一縣自效,除河東令,寬猛時當。由蘄州刺史遷安南經略使。環王國叛人李樂山謀廢其君,來乞兵,行立不受,命部將杜英策討斬之,歸其孥,蠻人悅服。英策及範廷芝者,皆奚洞豪也,隸於軍,它經略使多假借,暴恣乾治,行立陰把其罪,貸之,許自效,故能得英策死力。廷芝嘗休沐,久不還,行立召之,約曰:「軍法,逾日者斬,異時復然,爾且死!」後廷芝逾期,行立笞殺之,以尸還範氏,更為擇良子弟以代,於是威聲風行。徙桂管觀察使。黃家洞賊叛,行立討平之。俄代桂仲武為安南都護。銳於立功,為時所訾。召還,道卒,年四十七,贈右散騎常侍。



 崔沔,字善沖,京兆長安人,後周隴州刺史士約四世孫,自博陵徙焉。純謹無二言,事親篤孝,有才章。擢進士。舉賢良方正高第,不中者誦訾之,武后敕有司覆試,對益工,遂為第一。再補陸渾主簿,入調吏部,侍郎岑義嘆曰:「君今郤詵也!」薦為左補闕。性舒遲,進止雍如也,當官則正言,不可得而詘。睿宗召授中書舍人,以母病東都不忍去,固辭求侍,更表陸渾尉郭鄰、太樂丞封希顏、處士李喜以代己處。詔改虞部郎中,俄檢校御史中丞。請發太倉粟及減苑囿鳥獸所給以賑貧乏,人賴其利。監察御史宋宣遠與盧懷慎姻家,恃以弄法;姚崇子彞留司東都,通賓客,招賄賂。沔將按劾,崇、懷慎方執政,共薦沔有史才,轉著作郎,去其權,蓋憚之也。久之,為太子左庶子。母亡,受吊廬前,賓客未嘗至柩室。語人曰:「平生非至親不升堂入謁,豈以存亡變禮邪?」中書令張說數稱之。服除,遷中書侍郎。



 玄宗以仙州數喪刺史,欲廢之,沔請治舞陽,舞陽,故樊噲國也,更為樊州,帝不納,州卒廢。沔既喜論得失,或曰:「今中書宰相承制,雖侍郎貳之,取充位而已。」沔曰:「百官分職,上下相維,以成至治,豈可俯首懷祿邪?」凡詔敕曹事,多所異同,說不悅,出為魏州刺史。雨潦敗稼,沔弛禁便人。召還。分掌吏部十銓,以左散騎常侍為集賢修撰,歷秘書監、太子賓客。



 是時,太常議加宗廟籩豆,又欲增喪服,於是卿韋縚請坐增籩豆至十二;外祖服大功,舅小功,堂姨若舅、舅母袒免。沔曰:「祭祀上矣,古者飲食必先嚴獻。未有火化,故有毛血之薦,未有曲蘗,故有玄酒之奠。後王作為酒醴、犧牲以致馨香,故有三牲、八簋、五齊、九獻。神道主敬,可備而不敢廢也,雖曰備物,而節制存焉。鈃俎、籩豆、簠簋尊罍之實,皆周時饌,其用通宴饗賓客,而周公與毛血、玄酒同薦於先祖。晉盧諶家祭禮,所薦皆晉時常食,不純用古。此聖賢變文而通其情也。然當時飲食不可闕於祭,明矣。國家清廟時享,禮饌具設,周制也,古物存焉。園寢上食,時膳備列,漢法也,它珍極焉。職貢來祭,致遠物也。有新必薦,順時令也。苑囿躬稼所收,搜狩親中,莫不薦而後食,盡誠敬民。若此至矣,無以加矣。諸珍羞鮮物,第敕有司悉使著於令,因宜而薦,不必加籩豆以為嗛也。大羹,古食也,盛於古器。和羹,常饌也,盛於時器。毛血盛於盤,玄酒盛於尊。未有薦時饌而用古器者,繇古質而今文,便事也。故加籩豆未足盡天下美物,而措諸朝,徒近侈耳。魯丹桓宮之楹,刻其桷,《春秋》非之。班固稱:『墨家出於清廟,是以貴儉。』然清廟不奢,舊矣。太常所請,臣所未安。」



 又太常言:「爵小不及合,執持至難。」沔曰:「禮有以小為貴者,獻以爵是也。然今不及制,則非禮,自有司之陋也。隨失制宜,不待議而革雲。」又言:「禮本於家正,家正而天下定。家不可以貳,故父以尊崇,母以厭降。是以內服齊斬,外服緦,尊名所加,不過一等,今古不易之道也。昔辛有適伊川,見被發而祭,知其將戎,禮先亡也。比制《唐禮》,推廣舅恩,故弘道以來,國命再移於外姓,本禮驗亡,可不戒哉!」時職方郎中韋述、戶部郎中楊伯成、禮部員外郎楊仲昌、監門兵曹參軍劉秩等議與沔合,又詔中書門下參裁,於是宗廟籩豆坐各六,姨若舅小功,舅母緦麻,堂姨袒免,餘仍舊制。



 每朝廷有疑議,皆咨逮取衷。卒,年六十七,贈禮部尚書,謚曰孝。沔儉約自持,祿稟隨散宗族,不治居宅,嘗作《陋室銘》以見志。子祐甫至宰相,別傳。



 盧從願,字子龔。六世祖昶,仕後魏為度支尚書,自範陽徙臨漳,故從願為臨漳人。擢明經,為夏尉。又舉制科高第,拜右拾遺,遷監察御史,為山南黜陟巡撫使,還奏稱旨,累進中書舍人。



 睿宗立,拜吏部侍郎。吏選自中宗後綱紀耗蕩,從願精力於官,偽牒詭功,擿檢無所遺,銓總六年,以平允聞。帝異之,特官其一子。從願請贈其父敬一為鄭州長史,制可。初,高宗時,吏部號稱職者裴行儉、馬載,及是,從願與李朝隱為有名,故號「前有裴、馬,後有盧、李」。



 開元四年,玄宗悉召縣令策於廷,考下第者罷之。從願坐擬選失實,下遷豫州刺史。政嚴簡,奏課為天下第一,寶書勞問,賜絹百匹。召為工部侍郎,遷尚書左丞、中書侍郎,以工部尚書留守東都,代韋抗為刑部尚書。數充校考使,升退詳確。



 御史中丞宇文融方用事,將以括田戶功為上下考,從願不許,融恨之,乃密白「從願盛殖產,占良田數百頃」,帝自此薄之,目為多田翁。後欲用為相屢矣,卒以是止。十八年,復為東都留守,坐子起居郎論輸糴於官取利多,貶絳州刺史,遷太子賓客。二十年,河北饑,詔為宣撫處置使,發倉廥賑饑民。使還,乞骸骨,授吏部尚書致仕,給全祿終身。卒,贈益州大都督,謚曰文。



 李朝隱,字光國,京兆三原人。明法中第,調臨汾尉,擢至大理丞。武三思構五王,而侍御史鄭愔請誅之,朝隱獨以「不經鞫實,不宜輕用法」,忤旨,貶嶺南丑地。宰相韋巨源、李嶠言於中宗曰:「朝隱素清正,一日遠逐,恐駭天下。」帝更以為聞喜令。



 遷侍御史、吏部員外郎。時政出權幸,不關兩省而內授官,但斜封其狀付中書,即宣所司。朝隱執罷千四百員,怨誹嘩騰,朝隱胖然無避屈。遷長安令,宦官閭興貴有所干請,曳去之。睿宗嘉嘆,後御承天門,對百官及朝集使褒諭其能,使遍聞之。進太中大夫一階,賜中上考、絹百匹,以旌剛烈。成安公主奪民園,不酬直,朝隱取主奴杖之,由是權豪斂伏。為執政所擠,出通州都督,徙絳州刺史。開元初,遷吏部侍郎,銓敘明審,與盧從願並授一子官。久之,以策縣令有下第,降滑州刺史,徙同州。玄宗東幸,召見慰勞,賜以衣、帛。擢河南尹,政嚴清,奸人不容息。太子舅趙常奴怙勢橫閭里,朝隱曰:「此不繩,不可為政。」執而悟辱之,帝賜書慰勉。



 入為大理卿。武強令裴景仙丐贓五千匹,亡命,帝怒,詔殺之。朝隱曰:「景仙,其先寂有國功,載初時,家為酷吏所破,誅夷略盡,而景仙獨存,且承嫡,於法當請。又丐乞贓無死比,藉當死坐,猶將宥之,使私廟之祀無餒魂可也。」帝不許,固請曰:「生殺之柄,人主專之;條別輕重,有司當守。且贓惟枉法抵死,今丐贓即斬,後有枉法,亦又何加?且近發德音,杖者聽減,流者給程,豈一景仙獨過常法?」有詔決杖百,流嶺南。



 朝隱更授岐州刺史,母喪解。召為揚州大都督府長史,固辭,見聽。時年已衰,而篤於孝,自致毀瘠,士人以為難。明年,詔書敦遣揚州就職。還為大理卿,封金城伯,代崔隱甫為御史大夫。天下以其有素望,每大夫缺,冀朝隱得之。及居職,不爭引大體,惟先細務,由是名少衰。進太常卿,出為嶺南採訪處置使,兼判廣州。卒於官,贈吏部尚書,官給車槥北還,謚曰貞。



 王丘,字仲山,同晈從子也。父同晊,終太子左庶子。丘十一擢童子科,它童皆專經,而獨屬文,由是知名。及冠,舉制科中第,授奉禮郎。氣象清古,行修絜,於詞賦尤高。族人方慶及魏元忠更薦之,自偃師主簿擢監察御史。



 開元初,遷考功員外郎。考功異時多請托,進者濫冒,歲數百人。丘務核實材,登科才滿百,議者謂自武後至是數十年,採錄精明無丘比。其後席豫、嚴挺之亦有稱,然出丘下。遷紫微舍人、吏部侍郎,典選,復號平允。其獎用如山陰尉孫逖、桃林尉張鏡微、湖城尉張晉明、進士王泠然,皆一時茂秀。久之,為黃門侍郎。



 會山東旱饑,議以中朝臣為刺史,制詔:「皋陶稱:『在知人,在安民。』皆念存邦本,朝乾夕惕,無忘一日。今長吏或未稱,蒼生謂何?深思循良,以革頹敝,宜重刺史之選,自朝廷始。」乃以丘與中書付郎崔沔等並為山東刺史。而丘守懷州,尤清嚴,為下畏慕。入知吏部選,改尚書左丞,以父喪解。服除,為右散騎常侍,仍知制誥。裴光庭卒,蕭嵩與丘善,將引與當國,丘固辭,盛推韓休行能。及休秉政,薦為御史大夫。丘訥於言,所白奏帝多不喜,改太子賓客,襲父封。以疾徙禮部尚書,致仕。



 丘更履華劇,而所守清約,未嘗通饋遺,室宅童騎敝陋,既老,藥餌不自給。帝嘆之,以謂有古人節,下制給全祿以旌絜吏。天寶二年卒,贈荊州大都督,謚曰文。



 嚴挺之,名浚,以字行,華州華陰人。少好學,姿質軒秀。舉進士,並擢制科,調義興尉,號材吏。姚崇為州刺史,異之。崇執政,引為右拾遺。



 睿宗好音律,每聽忘倦。先天二年正月望夜,胡人婆陀請然百千燈,因弛門禁,又追賜元年酺,帝御延喜、安福門縱觀,晝夜不息,閱月未止。挺之上疏諫,以為:「酺者因人所利,合醵為歡也,不使靡敝。今暴衣冠,羅伎樂,雜鄭、衛之音,縱倡優之玩,不深戒慎,使有司跛倚,下人罷劇,府縣裏閻課賦苛嚴,呼嗟道路,貿壞家產,營百戲,擾方春之業,欲同其樂而反遺之患。」乃陳「五不可」,誠意忠到,帝納焉。



 侍御史任正名恃風憲,至廷中責詈衣冠,挺之讓其不敬,反為所劾,貶萬州員外參軍事。開元中,為考功員外郎,累進給事中,典貢舉,時號平允。會杜暹、李元紘為相,不相中。暹善挺之,而元紘善宋遙,用為中書舍人。遙校吏部判,取舍與挺之異,言於元紘,元紘屢詰譙,挺之厲言曰:「公位相國,而愛憎反任小人乎?」元紘曰:「小人為誰?」曰:「宋遙也。」由是出為登州刺史,改太原少尹。



 初,殿中監王毛仲持節抵太原朔方籍兵馬,後累年,仍移太原取兵仗,挺之不肯應,且以毛仲寵幸,久恐有變,密啟於帝。俄改濮、汴二州刺史,所治皆嚴威,吏至重足脅息。會毛仲敗死,帝以挺之言忠,召為刑部侍郎,遷太府卿。



 宰相張九齡雅知之,用為尚書左丞,知吏部選。李林甫與九齡同輔政,以九齡方得君,諂事之,內實不善也。戶部侍郎蕭炅,林甫所引,不知書,嘗與挺之言,稱蒸嘗伏臘,乃為「伏獵」。挺之白九齡:「省中而有伏獵侍郎乎!」乃出炅岐州刺史,林甫恨之。九齡欲引以輔政,使往謁林甫,挺之負正,陋其為人,凡三年,非公事不造也,林甫益怨。會挺之有所諉於蔚州刺史王元琰,林甫使人暴其語禁中,下除洛州刺史,徙絳州。



 天寶初,帝顧林甫曰:「嚴挺之安在?此其材可用。」林甫退召其弟損之與道舊,諄諄款曲,且許美官,因曰:「天子視絳州厚,要當以事自解歸,得見上,且大用。」因紿挺之使稱疾,願就醫京師。林甫已得奏,即言挺之春秋高,有疾,幸閑官得自養。帝恨吒久之,乃以為員外詹事,詔歸東都。挺之鬱鬱成疾,乃自為文志墓,遺令薄葬,斂以時服。



 挺之重交游,許與生死不易,嫁故人孤女數十人,當時重之。然溺志於佛,與浮屠惠義善,義卒,衰服送其喪,已乃自葬於其塔左,君子以為偏。子武。



 武,字季鷹。幼豪爽。母裴不為挺之所答,獨厚其妾英。武始八歲,怪問其母,母語之故。武奮然以鐵鎚就英寢,碎其首。左右驚白挺之曰:「郎戲殺英。」武辭曰:「安有大臣厚妾而薄妻者,兒故殺之,非戲也。」父奇之,曰:「真嚴挺之子!」然數禁敕。武讀書不甚究其義,以廕調太原府參軍事,累遷殿中侍御史。從玄宗入蜀,擢諫議大夫。至德初,赴肅宗行在,房琯以其名臣子,薦為給事中。已收長安,拜京兆少尹。坐琯事貶巴州刺史。久之,遷東川節度使。上皇合劍南為一道,擢武成都尹、劍南節度使。還,拜京兆尹,為二聖山陵橋道使,封鄭國公。遷黃門侍郎。與元載厚相結,求宰相不遂,復節度劍南。破吐籓七萬眾於當狗城,遂收鹽川。加檢校吏部尚書。



 武在蜀頗放肆,用度無藝,或一言之悅,賞至百萬。蜀雖號富饒,而峻掊亟斂,閭里為空,然虜亦不敢近境。梓州刺史章彞始為武判官,因小忿殺之。琯以故宰相為巡內刺史,武慢倨不為禮。最厚杜甫,然欲殺甫數矣。李白為《蜀道難》者,乃為房與杜危之也。永泰初卒,母哭,且曰:「而今而後,吾知免為官婢矣。」年四十,贈尚書左僕射。



 挺之從孫綬。綬父丹,嘗為劍南鹽鐵、青苗、租庸使,以武在蜀,辭不拜。綬擢進士第,以侍御史副劉贊為宣歙團練使。贊卒,綬總留事,悉庫物以獻,召為刑部員外郎。賓佐進奉由綬始。



 河東節度使李說病,軍司馬鄭儋總其政,說卒,代為節度。時德宗務姑息,方鎮若帥死,不它命,即用軍司馬代之,以和厭眾情。至是,帝頗憶綬所獻,故擢為河東司馬。明年,儋卒,即檢校工部尚書,代其使。憲宗立,楊惠琳反夏州,劉闢反蜀,綬建言:「天子始即位,不可失威,請必誅。」選銳兵,遣大將李光顏助討賊。二賊平,檢校尚書左僕射,封扶風郡公,進司空。在鎮九年,尚寬惠,治稱流聞,士馬孳息。嘗大閱,旗幟周七十里,回鶻梅錄將軍在會,聞金鼓震伏。入為尚書右僕射。



 綬既名胄,於吏事有方略,然銳進趣,素議薄之。始就廊下食,在百官上,帝使中人賜含桃,綬見拜之,為御史劾奏,綬慚懼待罪,詔釋綬而貶中人。出為荊南節度使,封鄭國公。



 漵州蠻張伯靖殺吏,據辰、錦州,連九洞自固,詔綬進討。綬勒兵出次,遣將齎檄開曉,群蠻悉降。吳元濟反,僉以綬明恕可大事,乃徙山南東道節度使,加淮西招撫使。綬引師壓賊境,多出金帛賞士,以厚賂謝中人,招聲援,既未有以制賊,閉屯彌年不戰。宰相裴度謂綬非將才,以太子少保召還,檢校司徒,判光祿卿事,進少傅。卒,年七十七,贈太保。



 綬才不逾中人,然歷三鎮,所奏闢及綬時位將相者九人。初,綬未顯,過於■鄉尉李達,達不禮,方飯它客,不召綬。後達罷彭城令,過並州,晨入謁,不知綬也。綬方大宴賓客,召達至,戒客勿起,讓曰:「吾昔羈旅■鄉,君方召客食而不顧我,今我召客亦不敢留君。」達慚,不得去,左右引出,悸而瘖,臥館數月,其佐令狐楚為請,乃免。



 河東李進賢者,善畜牧,家高貲,得幸於綬,署牙門將。元和中,進賢累為振武節度使,闢綬子澈為判官。澈年少,治苛刻,軍中苦之。回鶻入闢鵜泉,進賢發兵討之,吏廩糧不實,次鳴砂,焚殺其將楊遵憲而還。進賢大怒,眾懼,因燔城門,攻進賢,左右拒戰不勝,縋而去,奔靖邊軍。乃殺澈而屠進賢家。詔以夏綏銀節度使張煦代之,誅亂首數百人乃定。



\end{pinyinscope}