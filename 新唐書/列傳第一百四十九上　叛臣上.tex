\article{列傳第一百四十九上 叛臣上}

\begin{pinyinscope}

 僕固懷恩,鐵勒部人。貞觀二十年,鐵勒九姓大首領率眾降,分置瀚海、燕然、金微、幽陵等九都督府,別為蕃州要著作有《數學原理》(與其學生羅素合著)、《科學與近代世,以僕骨歌濫拔延為右武衛大將軍、金微都督,訛為僕固氏。生乙李啜;乙李啜生懷恩,世襲都督。



 懷恩善戰鬥,曉識戎情,部分謹嚴。安祿山反,從朔方節度使郭子儀討賊雲中,破之;敗薛忠義於背度山,殺七千騎,禽忠義子,下馬邑。進會李光弼,戰常山、趙郡、沙河、嘉山,走史思明。肅宗即位,與子儀赴靈武。時同羅部落叛,祿山北掠朔方,子儀率懷恩迎擊。懷恩子玢戰敗降虜,已而自拔歸,懷恩怒,叱斬之。將士股慄,皆殊死戰,遂破其眾,收馬、橐它、器械甚眾。帝又詔與敦煌王承寀使回紇請師,回紇聽命。至德二載,從子儀下馮翊、河東,走賊將崔乾祐,襲潼關,破之。賊將安守忠、李歸仁苦戰二日,王師敗績。懷恩至渭水,無舟,抱馬鬣以逸,收散卒還河東。子儀赴鳳翔,歸仁以勁兵邀戰三原,子儀使懷恩與王升、陳回光、渾釋之、李國貞五將軍伏白渠下,賊至遇伏,敗而走。又戰清渠,不利,引還。



 時回紇使葉護、帝得以四千騎濟師,南蠻、大食等兵亦踵至。帝乃詔廣平王為元帥,使懷恩統回紇兵,從王戰香積寺北。賊以一軍伏營左,懷恩馳掩之,馘斬無遺者,賊氣沮。既合戰,以回紇夾攻賊,戰酣,脫甲援矛直搗陣,殺十餘人,眾驚靡,亦會李嗣業鏖斗尤力,賊大崩敗。會日暮,懷恩見王曰:「賊必棄城走,願假壯騎二百,縛安守忠、李歸仁等致麾下。」王曰:「將軍戰疲,且休矣;迨明,與將軍圖之。」對曰:「守忠等皆天下驍賊,驟勝而敗,此天與我也,奈何縱之?使復得眾,必為我患,雖悔無逮。」王不從,固請,通夕四五反。遲明,諜者至,守忠等果遁去。又從王破賊於新店。以復兩京有殊功,詔加開府儀同三司、鴻臚卿,封豐國公,賜封二百戶。



 從郭子儀破安太清,下懷、衛二州,攻相州,戰愁思岡,常為先鋒,勇冠軍中。乾元二年,拜朔方行營節度使,進封大寧郡王。



 懷恩為人雄重寡言,應對舒緩,然剛決犯上,始居偏裨,意有不合,雖主將必折詬。其麾下皆蕃、漢勁卒,恃功多不法。子儀政寬,能優容之。及李光弼代子儀,懷恩仍為副。光弼守河陽,攻懷州,降安太清。又子瑒,亦善鬥,以儀同三司將兵,每深入多殺,賊憚其勇,號猛將。太清妻有色,瑒劫致於幕,光弼命歸之,不聽,以卒環守。復馳騎趨之,射殺七人,奪妻還太清。懷恩怒曰:「公乃為賊殺官卒邪?」光弼持法嚴,少假貸。初,會軍汜水,朔方將張用濟後至,斬纛下。懷恩心憚光弼,自用濟誅,常邑邑不樂。及光弼與史思明戰邙山,不用令,以覆王師。帝思其功,召入為工部尚書,寵以殊禮。代宗立,拜隴右節度使,未行,改朔方行營節度,以副子儀。



 初,肅宗以寧國公主下嫁毘伽闕可汗,又為少子請婚,故以懷恩女妻之。少子立,號登里可汗,而懷恩女為可敦。寶應元年,帝召兵於回紇,而登里可汗已為史朝義所誘,引眾十萬盜塞,關中大震。帝遣殿中監藥子昂勞之,可汗因請見懷恩及其母,有詔報可。懷恩避嫌不往,帝賜鐵券,手詔固遣,乃行。與可汗會太原,可汗大悅,遂請和,助討朝義,即引兵屯陜州,待師期。



 於是雍王以元帥為中軍,拜懷恩同中書門下平章事為之副,乃與左殺為先鋒。時諸節度皆以兵會,次黃水,賊堅壁自固。懷恩陣西原,多張旗■,使突騎與回紇稍南出繚賊左,舉旗為應,破賊壁,死者數萬。朝義擁精騎十萬來援,埋根決戰,短兵接,殺獲相當。魚朝恩令射生五百攢矢注射,賊多死,而陣堅不可犯。馬璘怒,單騎援旗直進,奪兩盾,賊闢易,大軍乘以入,眾囂不止,朝義敗。斬首萬六千級,禽四千餘人,降者三萬。轉戰石榴園、老子祠,賊再敗,自相奔蹂死,填尚書谷幾滿,朝義輕騎走。懷恩進收東都、河陽,封府庫,無所私。釋賊所署許叔冀、王伷等,眾皆按堵。留回紇屯河陽,使瑒及北庭兵馬將高輔成以萬騎逐北,懷恩常壓賊而次。至鄭州,再戰再捷,賊帥張獻誠以汴州降,下滑州。朝義至衛州,與其黨田承嗣、李進超、李達廬合,有眾四萬,據河以戰。瑒濟師登岸薄之,賊黨奔潰。進次昌樂,朝義逸,偽帥達廬降,薛高、李寶臣舉相、衛、深、定等九州獻款。朝義至貝州,得其黨薛忠義,引眾三萬拒瑒於臨清。賊氣盛,瑒勒兵挫其鋒,令高彥崇、渾日進、李光逸設三伏以待,賊半度,伏發,擊之,朝義走。會回紇以輕騎至,瑒卷甲馳之,大戰下博,賊背水陣,師奔擊,賊大崩,積尸蔽流而下。朝義退守莫州。於是都知兵馬使薛兼訓、郝廷玉、兗鄆節度使辛云京會師城下,朝義與田承嗣數挑戰,不勝,臨陣斬偽黨敬榮。朝義懼,率殘眾奔幽州。王追躡,朝義走平州,自經死,河北平。懷恩與諸將皆罷兵,以功遷尚書左僕射兼中書令、河北副元帥、朔方節度使,加封戶四百。



 初,帝有詔但取朝義,其它一切赦之。故薛嵩、張忠志、李懷仙、田承嗣見懷恩皆叩頭,願效力行伍。懷恩自見功高,且賊平則勢輕,不能固寵,乃悉請裂河北分大鎮以授之,潛結其心以為助,嵩等卒據以為患云。



 未幾,加太子少師,增戶五百,第一區,與一子五品官。詔護回紇歸國,道太原,辛云京內忌懷恩,又以其與回紇親,疑可汗見襲,閉關不敢犒軍。懷恩既父子新立功,舉河朔若拾遺,名出諸將遠甚,而為云京所拒,大怒,表上其狀。頓軍汾州,使裨將李光逸以兵守祁,李懷光據晉州,張如嶽據沁州,高暉等十餘人自從。會監軍駱奉先自云京所歸,云京已厚結其歡,因言懷恩與可汗約反狀明白。奉先過懷恩,升堂拜母,母讓曰:「若與我兒約兄弟,今何自親云京?然前事勿論,自今宜如初。」酒酣,懷恩舞,奉先厚納以幣。懷恩未及酬,奉先亟辭去,懷恩即遣左右匿其馬。奉先疑圖己,乘夜遁歸。懷恩驚,追與其馬。奉先還,具奏懷恩反狀,懷恩亦請誅云京、奉先,詔兩解之。懷恩之過潞,李抱玉贈以幣馬,懷恩答之。俄抱玉表懷恩私有所結。



 廣德初,進拜太保,與一子三品、一子四品官,增封戶五百。瑒與一子五品官,封戶百。仍賜鐵券,以名藏太廟,畫象凌煙閣。又以瑒檢校兵部尚書、朔方行營節度使。然懷恩怏怏,又性強固,不肯為讒毀屈,無以自解,乃上書陳情曰:「臣世本夷人,少蒙上皇驅策。祿山之亂,臣以偏裨決死靜難,杖天威神,克滅強胡。思明繼逆,先帝委臣以兵,誓雪國讎,攻城野戰,身先士卒,兄弟死於陣,子姓沒於軍,九族之內,十不一在,而存者創痍滿身。陛下龍潛時,親總師旅,臣事麾下,悉臣之愚。是時數以微功,已為李輔國讒間,幾至毀家。陛下即位,知臣負謗,遂開獨見之明,杜眾多之口,拔臣於汧、隴,任臣以朔方,游魂反乾,朽骨再肉。前日回紇入塞,士人未曉,京輔震驚,陛下詔臣至太原勞問,許臣一切處置,因得與可汗計議,分道用兵,收復東都,掃蕩燕、薊。時可汗在洛,為魚朝恩猜阻,已失歡心。及臣護送回紇,云京閉城不出,潛使攘竊,蕃夷怨怒,彌縫百端,乃得返國。臣還汾州,休息士馬云京亦不使一介相聞,畏臣劾奏,故構為飛謗,以起異端。陛下不垂明察,欲使忠直之臣,陷讒邪之黨,臣所為拊心泣血者也。然臣之罪有六,無所逃死:往者同羅背逆,以騷河曲,兵連不解,臣不顧老母,從先帝於行在,募兵討賊,同羅奔殄,是臣不忠於國,罪一也;斬子玢以令士眾,舍天性之愛,是臣不忠於國,罪二也;二女遠嫁,為國和親,合從殄滅,是臣不忠於國,罪三也;又與子瑒躬履行陣,志寧邦家,是臣不忠於國,罪四也;河北新附,諸鎮皆握強兵,臣之撫綏,反側時定,是臣不忠於國,罪五也;協和回紇,戡定中原,二陵復土,使陛下勤孝兩全,是臣不忠於國,罪六也。」又言:「來瑱之誅,不暴其罪,天下為疑。四方奏請,陛下皆云與驃騎議之,可否不出宰相。」詞言慢很,帝一不為慊,且欲其悔過,故推心待之。詔宰相裴遵慶臨諭詔旨,因察其去就。



 遵慶至,懷恩抱其足,泣且訴。遵慶道帝所以不疑,即勸入朝,懷恩許諾。副將範志誠諫,以為「嫌隙成矣,奈何入不測之朝,獨不見來瑱、李光弼乎?二臣功高不賞,瑱已及誅。」懷恩乃止。欲使一子入宿衛,志誠固止。御史大夫王翊使回紇還,懷恩慮洩其交通狀,因留不遣。即使瑒攻雲京,云京敗,進攻榆次。



 初,帝幸陜,顏真卿請奉詔召懷恩。至是,帝使往,辭曰:「臣往請行,時也,今無及矣!」帝問故,對曰:「頃陛下避狄於陜,臣見懷恩,責以《春秋》義,不奔問官守,故懷恩來朝,以助討賊,則其辭順。今陛下即宮京邑,懷恩進不勤王,退不釋眾,其辭曲,必不來矣!」「然則奈何?」曰:「今言懷恩反者,獨辛云京、李抱玉、駱奉先、魚朝恩四人耳,自餘盛言其枉。然懷恩將士,皆郭子儀舊部曲,陛下若以子儀代之,喻以逆順,必相率而歸。」從之。



 子儀至河中,瑒攻榆次,未拔,追兵於祁,責其緩,鞭之,眾怒。是夕,偏將焦暉、白玉等斬其首,獻闕下。懷恩聞,以告母。母曰:「我戒汝勿反,國家酬汝不淺,今眾變,禍且及我,奈何?」懷恩再拜出,母提刀逐之曰:「吾為國殺此賊,取其心以謝軍中。」懷恩走,乃與部曲三百北度河,走靈武,稍稍引亡命,軍復振。帝念舊勛,不加罪,詔輦其母歸京師,厚恤之,以壽終。又下詔拜懷恩太保兼中書令、大寧郡王,罷餘官。



 懷恩固惡不能改,遂誘吐蕃十萬入塞,豐州守將戰死。進掠涇、邠,祭來瑱墓。度涇水,邠寧節度使白孝德御之,覆其陣,懷恩泣曰:「曩皆為我子,反為人致死於我。」入侵奉天,子儀拒退之。永泰元年,帝集天下兵防秋。懷恩誘合諸蕃號二十萬入寇,吐蕃自北道逼醴泉,搖奉天;任敷、鄭廷、郝德自東道寇奉先,以窺同州;羌、渾、奴剌自西道略盩厔,趣鳳翔。京師震駭。詔子儀屯涇陽,渾日進、白元光屯奉天,李光進屯雲陽,馬璘、郝廷玉屯便橋,董秦屯東渭橋,駱奉先、李日越屯盩厔,李抱玉屯鳳翔,周智光屯同州,杜冕屯坊州,帝御六軍屯苑中,下詔親征。懷恩至鳴沙,病甚,還死靈武,部曲焚其尸以葬。部將張韶、徐璜玉不能定其軍,皆前死。範志誠統眾寇涇陽。時諸屯堅壁,大雨,溪垘流潰,賊不得進。吐蕃既持久,又與回紇爭長,更相疑,莫適先進,因焚廬舍,驅男女數萬去。周智光邀戰澄城,破之,收馬牛軍資萬計。回紇乃詣子儀降,請擊吐蕃自效。子儀分兵隨之,破其眾於涇州。任敷走,羌、渾詣李抱玉降。



 始,懷恩立功,門內死王事者四十六人。及拒命,士不弛甲凡三年。帝隱忍,數下詔,未嘗聲其反。及死,為之惻然曰:「懷恩不反,為左右所誤耳!」俄而從子名臣以千騎降。大歷四年,冊懷恩幼女為崇徽公主,嫁回紇云。



 周智光,少賤,失其先系,以騎射從軍,起行間為裨將。魚朝恩鎮陜州,與相暱款,數稱薦之,累遷同、華二州節度使。



 永泰元年,吐蕃、回紇、黨項羌、渾、奴剌眾十餘萬寇奉天,智光邀戰澄城,破之,獲駝馬軍貲萬計,逐北至鄜州。素與杜冕仇嫌,時冕屯坊州,家在鄜,智光入殺刺史張麟,害冕宗屬八十人,火民三千舍而去。朝廷召,懼不赴。更詔冕使梁州避讎,冀其來。偃然不聽命,聚不逞數萬,恣剽掠以甘其欲,結固之。殺陜州監軍張志斌及前虢州刺史龐充。初,志斌自陜入奏,智光慢不為禮,志斌責之,怒曰:「僕固懷恩豈反者邪?皆鼠輩弄威福趣之禍也。我本不反,今為爾反!」遂叱斬志斌,饗帳下。時崔圓自淮南納方物百萬,盜頡其半;天下貢奉輸漕,劫留之;士沿調當西者懼何詰,間道走同者,遣部將邀捕斬之。代宗未暴其罪,命中使余元仙持詔拜尚書左僕射。既受詔,恚語曰:「吾有大功,上不與平章事,且同、華地狹,不足申腳,若加陜、虢、商、鄜、坊五州,差可。」因言:「諸子皆彎弓二百斤,有萬人敵,挾天子令諸侯,非智光尚誰可?」即歷詆大臣,元仙震汗。徐遺百縑遣之。自立生祠,俾其下襘賽。



 大歷二年,帝詔郭子儀密圖之。同、華路閉,詔書不能通,乃召子儀婿趙縱受口詔,書帛內蜜丸,遣家童走間道傳詔。子儀得詔,聲言討之。未行,其眾大攜,部將李漢惠自同州降子儀。乃貶智光澧州刺史,聽百人隨身,貸將吏一切不問。尋為帳下斬其首,並斬子元耀、元乾來獻,詔梟首皇城南街。判官邵賁、別將蔣羅漢並伏誅。敕有司具儀告太清宮、太廟、七陵。



 先是,淮西李忠臣入朝,次潼關,聞智光反,率兵討之。會敗,忠臣因入華大掠,自赤水至潼關畜產財物皆盡,官吏至衣紙自蔽、累日不食者。



 梁崇義,京兆長安人。以概量業於市,力能舒鉤。後為羽林射生,事來瑱。沉默寡言。瑱自襄陽朝京師,分諸將戍福昌、南陽。瑱誅,戍者潰,崇義自南陽勒眾還襄州,與李昭、薛南陽相讓為長,眾曰:「非梁卿莫可。」遂總其軍,殺昭及南陽,脅制眾心。代宗因即拜節度使。舉七州兵二萬,與田承嗣、李正己、薛嵩、李寶臣相輔車,根牙槃結。然獨以地褊兵少,法令最治,折節遇士以自振,襄、漢間人識教義。親厚數諷入朝,答曰:「來公有大功,畏閹豎讒,逡巡辭召。至代宗立,不待駕而朝,即見族。吾釁盈矣,若何欲見上乎?」



 建中元年,李希烈請討之。崇義懼,整飭軍旅。男子郭昔上變事,德宗欲示以信,流昔遠方,詔金部員外郎李舟諭旨。初,劉文喜之難,舟奉詔入涇州,俄而帳下斬文喜以聞,四方傳舟能覆軍殺將,反側者皆惡之。舟至,以入朝勸崇義,崇義不悅。明年,遣使尉撫諸道,舟復如崇義所,遂不肯內,請易它使。更命給事中廬翰往,崇義益不安,跋扈甚,諫者多死。朝廷以不疑示天下,乃加同中書門下平章事,妻及子悉封賞,賜鐵券,擢其將蘭杲為鄧州刺史,遣御史張著以手詔召崇義。崇義使卒持滿,乃受命。杲奉詔不敢發,詣崇義自言。崇義對著號哭,遂拒詔。



 帝命李希烈率諸道兵進討。崇義先攻江陵,欲通黔、嶺,敗於四望而還。殺希烈臨漢屯兵千餘,希烈怒,引兵循漢而上。崇義使翟崇暉、杜少誠戰蠻水,折北至涑口,大敗。二將降,希烈寵之,使部降兵徇襄陽,約百姓按堵。崇義閉壁,守者斬關出,不可止,乃與妻赴井死,傳首京師。希烈誅其親族及軍從臨漢役者三千人。



 崇義孫叔明,養於李納,後從劉悟為昭義將,從諫死,遣進旄節,有詔誅之。



 李懷光,渤海靺鞨人,本姓茹。父常,徙幽州,為朔方部將,以戰多賜姓,更名嘉慶。



 懷光在軍,積勞至開府儀同三司,為都虞候。勇鷙敢誅殺,雖親屬犯法,無所回貸。節度使郭子儀仁厚,不親事,以紀綱委懷光,軍中畏之。會母喪,起兼邠、寧、慶都將。德宗罷子儀副元帥,以所部兵分諸將,故懷光檢校刑部尚書,為寧、慶、晉、絳、慈、隰等州節度使。引眾城長武,據原首,臨涇水,以扼吐蕃空道,自是不敢南侵。建中初,楊炎欲城原州,使懷光兼帥涇原,遂其功。原州宿將史抗、溫儒雅等,故子儀麾下,嘗在懷光右,及處其下,意鬱鬱,懷光因罪誅之,由是涇軍迎畏。劉文喜者,因眾懼,遂叛。詔與硃泚討平之,加檢校太子少師。明年,徙朔方節度使,實封戶四百,仍領邠寧。



 時馬燧、李抱真討田悅,未克,詔懷光以朔方兵萬五千並力。懷光至魏,未及營,與硃滔等戰連篋山,為賊所敗,悅因決水灌軍,燧等退屯魏縣。尋進同中書門下平章事,益戶二百。與滔等相持,久不戰。



 帝狩奉天,懷光率所部奔命。方雨淖,奮厲軍士倍道進,自蒲津絕河,敗泚軍於醴泉。將抵奉天,前遣裨將張韶以蠟韜表,隨賊攻城,叩壘呼曰:「我朔方使也!」縋而上,比登,身被數十矢。時帝被圍急,聞之喜,即持韶大號城上,人心乃安。又敗賊於魯店,泚解圍去。進加副元帥、中書令。



 懷光為人疏而愎,誦言:「宰相謀議乖剌,度支賦斂重,京兆尹刻薄軍食,天下之亂皆由此。吾見上,且請誅之。」或以告王翃,翃等計:「懷光有大功,上且訪以得失,使其言入,豈不殆哉!」遂告盧杞,杞即說帝曰:「懷光兵威已振,逆賊破膽,若席勝,可一舉滅賊。今入朝,則必宴勞留連,賊得從容完備,卒難圖也!」帝不得其情,因然之。乃敕懷光屯便橋,督諸將進討。懷光自以徑千里赴難,為奸臣拫隔不得朝,頗恚悵,去屯咸陽。明日,李晟會陳濤斜,壁壘未具,賊大至。晟說懷光曰:「賊保宮苑,攻之良難。今敢離窟穴,與公薄戰,此天以賊賜公也。」懷光曰:「吾馬未秣,士未飯,可遽戰哉?姑養吾勇以待之。」晟不得已,閉壁不出。懷光數暴杞等罪,帝為貶杞與趙贊、白志貞,又劾奏中人翟文秀,亦殺之以尉懷光。然益自疑,堅壁八旬不出戰,屢詔使進軍,以伺釁為解,陰連硃泚。



 初,崔漢衡使吐蕃求助兵,尚結贊曰:「吾法,進軍以本兵大臣為信。今制書不署懷光,未敢前。」帝乃命翰林學士陸贄詣懷光議事,懷光陳三不可,且言:「吐蕃舍人馬重英陷長安,贊普責其不焚爇,今其來,必肆宿志,一不可。彼雲引兵五萬,既用其人,則同漢士,儻邀我厚賞,何以致之?二不可。虜人雖來,義不先用,勒兵自固,以觀成敗,王師勝則分功,敗則圖變,狡詐多端,不可信,三不可。」卒不肯署。又謾罵贄曰:「爾何能?」



 興元元年,詔加太尉,賜鐵券。懷光赫然怒曰:「凡疑人臣反,則賜券。今授懷光,是使反也!」抵於地。時部將韓游瑰將兵衛奉天,懷光約令為變,游瑰以聞。數日,又密書趣之,門者捕送。又遣將趙升鸞諜於奉天,升鸞告渾瑊曰:「懷光遣達奚承俊火乾陵,使我為內應,以脅乘輿。」瑊白發其奸,請帝決幸梁州。帝令瑊戒嚴,未畢,帝自西門出,詔戴休顏守奉天。懷光遣將孟廷寶、惠靜壽、孫福率輕騎趨南山,糧料使張增遇之。三人計曰:「吾屬以叛聞,不如緩軍,彼怒,不過不吾將耳。」使增紿眾曰:「由此東,吾有見糧可食也。」廷寶等引而東,縱卒大掠,而百官遂入駱谷。追帝不及。還白懷光,懷光怒,悉罷其兵。懷光乃奪李建徽、陽惠元等軍,屯好畤,然其下稍稍攜貳。泚始憚之,至是欲遂臣懷光。懷光怒,告絕,益不安,乃引兵掠涇陽、三原、富平,遂如河中,留張昕守咸陽。而孟涉、段威勇擁兵降李晟,韓游瑰殺昕,以邠州歸。戴休顏自奉天令於軍曰:「懷光反。」乃城守。



 有詔以懷光為太子太保,許其麾下擇功高者一人統其兵。不奉詔。懷光至河中,取同、絳二州,按兵觀望。京師平,命給事中孔巢父、中人啖守盈召之,皆為懷光帳下所害,於是繕兵嚴守。帝乃遣渾瑊討之。度支欲罷其軍歲中稟賜,帝曰:「朔方軍累有功,豈以懷光拒命而眾不被恩邪?」詔所司別貯縑錢,須事定乃給。瑊破同州,屯軍不得進,數為懷光所衄。帝以河東節度使馬燧威名白著,乃拜副元帥,與瑊及鎮國駱元光、邠寧韓游瑰、鄜坊唐朝臣會兵進討。燧拔絳州,諸軍遂圍河中。



 貞元元年八月,朔方部將牛名俊斬懷光,傳首以獻,年五十七。帝念其功,詔許一子嗣,賜莊、第各一區,聽以禮葬,妻王徙澧州。初,懷光死,其子琟盡殺其弟乃死,故懷光無後。五年,詔曰:「懷舊念功,仁之大也;興滅繼絕,義之至也。昔蔡叔圯族,周封其子;韓信乾紀,漢爵其孥;侯君集不率,太宗存其祀。考先王之道,烈祖之訓,皆以刑佐德,俾人向方。曩者盜臣竊發,朕狩近郊,懷光夙駕千里,奔命行在,假雷霆之威,破虎狼之眾。守節靡終,潛構禍胎,大戮所加,自貽伊戚,孤魂無歸,懷之恍然。宜以外孫燕賜姓李,名曰承緒,以左衛率府胄曹參軍繼懷光後。」乃賜錢百萬,置田墓側,以備祭享;還妻王,使就養雲。



 陳少游,博州博平人。幼習老子、莊周書,為崇玄生,諸儒推為都講。有冒者欲對廣眾切問以屈少游。及升坐,音吐清辯,據引淹該,問窮而對有餘。大學士陳希烈高其能。既擢第,補南平令,治有聲。累遷侍御史、回紇糧料使,加檢校職方員外郎充使。檢校郎官自少游始。僕固懷恩表署河北副元帥判官,遷晉、鄭二州刺史。



 少游長權變,所至一切幹濟,賄謝權幸,以是數遷。李抱玉表澤潞副使,為陳鄭留後。永泰中,復奏為隴右行軍司馬,擢桂管觀察使。少游不樂遠去,規徙近鎮。時宦官董秀有寵,掌樞近,少游乃宿其里,候歸沐,入謁,因鄙語諂謂秀曰:「七郎親屬幾何?月費幾何?」秀謝曰:「族甚大,歲用常過百萬。」少游曰:「審如是,奉入不足為數日費,當數外營乃辦耳。吾雖不才,請獨取濟,歲輸錢五千萬。今具其半,請先入之。」秀大喜,與厚相結。少游因泣曰:「嶺南瘴癘,恐不得生還見顏色。」秀遽曰:「公美才,不當遠出,請少待。」時少游已納賂元載子仲武,於是內外更薦之,改宣歙池觀察使。大歷五年,徙浙東,封潁川縣子,遷淮南節度使。



 喜譎數,行小惠,群吏任職。三總籓,皆天下富饒處,以是斂求貿易無虛日,積財寶巨億萬。初結元載,賂金帛歲無慮十萬緡;又事宦官駱奉先、劉清潭、吳承倩及秀,故能久其任。後載以過見疑,少游亦疏之。載子伯和謫揚州,少游陽善之,陰奏其罪,代宗以為忠。建中初,朝廷經費不充,始請本道稅錢千增二百,鹽斗加百錢,度支因請諸道並增焉。李納拒命,少游出師收徐、海等州,俄棄之,退屯盱眙。累進檢校尚書左僕射,賜封戶三百,加同中書門下平章事。時宰相關播、盧杞與少游有雅故,故驟兼臺司。



 德宗幸奉天,度支汴東兩稅使包佶寓揚州,所儲財賦八百萬緡將輸京師,少游意硃泚勢盛,不遽平,欲肋取其財,使判官崔就佶索文簿,貸二百萬緡。佶以非敕命,拒之。怒曰:「君善,得為劉長卿;不爾,為崔眾矣!」長卿嘗任租庸使,為吳仲孺所囚,崔眾以倨李光弼被殺,故以為言。佶謁少游,欲諫止,不得語,即遣去,於是財用悉為少游所掠。佶奔白沙,少游遣幕中房孺復召之。佶驚走度江,伏妻子案牘中以免。佶有御遏兵三千,令高越、元甫將焉,少游奪之。能隨佶者,至上元,復為韓滉所留。佶但諸史如江、鄂州,以表內蠟丸以聞。會少游使至,帝詰其事,辭以不知。時禍難煽結,帝未能制,乃曰:「少游,國守臣,取佶之財,防它盜耳,庸何傷!」遠近聞之,咸稱帝得其機云。少游聞之,果自安不疑。



 李希烈陷汴,聲言襲江淮。少游懼,遣參謀溫述送款曰:「豪、壽、舒、廬,既韜刃卷鎧,惟君命。」又使巡官趙詵如鄆州,厚結李納。希烈僭號,遣將楊豐齎偽赦令送少游。壽州刺史張建封邏得之,斬豐,以偽赦送行在。會佶入朝,具言少游脅財賦狀。少游慚,上表言所取以贍軍興,請償之。而州府殘破,不能償,乃與腹心吏設法重稅,民皆苦之。劉洽取汴州,得希烈偽起居注,書「某月日,陳少游上表歸順。」少游聞,羞悸發病死,年六十一,贈太尉。



 贊曰:懷恩與賊百戰,闔宗死事至四十六人,遂汛掃燕、趙無餘埃,功高威重,不能防患,兇德根於心,弗得其所輒發,果於犯上,惜哉!其母拔刀逐賊,烈婦人也。懷光提萬眾,振天子於難,一為讒人所沮,忿戾不自還,身首殊分,然讒人亦可疾矣,所謂「交亂四國」者也。



 李錡,淄川王孝同五世孫。以父國貞廕調鳳翔府參軍。貞元初,遷至宗正少卿。嘗與卿李乾爭議,錡以直不坐,德宗兩置之。自雅王傅出為杭、湖二州刺史。方李齊運用事,錡以賂結其歡,居三歲,遷潤州刺史、浙西觀察、諸道鹽鐵轉運使。多積奇寶,歲時奉獻,德宗暱之。錡因恃恩驁橫,天下攉酒漕運,錡得專之,故朝廷用事臣,錡以利交,餘皆乾沒於私,國計日耗。浙西布衣崔善貞上書闕下暴其罪,帝械以賜錡;錡豫浚大坎,至則並械瘞坎中,聞者切齒。



 錡得志,無所憚,圖久安計,乃益募兵,選善射者為一屯,號「挽硬隨身」,以胡、奚雜類虯須者為一將,號「蕃落健兒」,皆錡腹心,稟給十倍,使號錡為假父,故樂為其用。帝於是復鎮海軍,以錡為節度使,罷領鹽鐵轉運。錡喜得節,而忘其權去,暴踞日甚,屬吏死不以過甚眾;又逼污良家,寮佐力諫不能得,遽遁去。



 憲宗即位,不假借方鎮,故倔強者稍稍入朝。錡不自安,亦三請覲。有詔拜尚書左僕射,以御史大夫李元素代之。中使馳驛勞問,兼撫慰其軍。錡署判官王澹為留後。錡無入朝意,稱疾遷延不即行。澹及中使數趣之,錡不悅,乘澹視事有所變更者,諷親兵圖澹。因給冬服,錡坐幄中,以挽硬、蕃落自衛,澹與中使入謁,既出,眾持刃謾罵,殺澹食之。監軍使遣牙將趙琦慰諭,又食之。以兵注中使頸,錡陽驚扈解,乃囚別館。蕃落兵,薛頡主之;挽硬兵,李鈞主之。又以公孫玠、韓運分總餘軍。室五劍,授管內鎮將,令殺五州刺史。屬別將庾伯良兵三千築石頭城,謀據江左。



 常州刺史顏防用其客李雲謀,矯詔稱招討副使,殺鎮將李深,傳檄蘇、杭、湖、睦四州同討錡。湖州辛秘亦殺鎮將趙惟忠。而蘇州李素為鎮將姚志安所執,釘舷上,獻於錡,錡敗而免。



 憲宗以淮南節度使王鍔為諸道行營兵馬招討處置使,中官薛尚衍為都監招討宣慰使,發宣武、武寧、武昌、淮南、宣歙、江西、浙東兵,自宣、杭、信三州進討。初,錡以宣州富饒,遣四院隨身兵馬使張子良、李奉仙、田少卿領兵三千分下宣、歙、池,錡甥裴行立雖預謀,而欲效順,故相與約還兵執錡,行立應於內。子良等既行,其夕,諭軍中曰:「僕射反矣,精兵四面皆至,常、湖鎮將乾首通衢,勢蹙且敗,吾輩徒死,不如轉禍希福。」部眾大悅,遂回趣城。行立舉火,內外合噪,行立攻牙門。錡大驚,左右曰:「城外兵馬至。」錡曰:「何人邪?」曰:「張中丞也。」錡怒甚,曰:「門外兵何人也?」曰:「裴侍御也。」錡拊膺曰:「行立亦叛吾邪!」跣足逃於女樓下。李鈞引兵三百趨出庭院格鬥,行立兵貫出其中,斬鈞,傳首城下。錡聞之,舉族慟哭。子良以監軍命曉諭城中逆順,且呼錡束身還朝,左右以幕縋而出之。錡以僕射召,數日而反狀至,下詔削官爵,明日而敗,送京師。神策兵自長樂驛護至闕下,帝御興安門問罪,對曰:「張子良教臣反,非臣意也。」帝曰:「爾以宗臣為節度使,不能斬子良然後入朝邪?」錡不能對。以其日與子師回腰斬於城西南,年六十七。尸數日,帝出黃衣二襲,葬以庶人禮。



 擢子良檢校工部尚書、左金吾將軍,封南陽郡王,賜名奉國;田少卿檢校左散騎常侍、左羽林將軍,代國公;李奉仙檢校右常侍、右羽林將軍,邠國公;裴行立泌州刺史。贈王澹給事中,趙琦和州刺史,崔善貞睦州司馬。削錡屬籍,從弟宋州刺史銛、通事舍人銑、從子師偃流嶺南。



 贊曰:語曰「出入之吝,謂之有司」,賤之也。德宗平硃泚,京師府藏耗竭,諸道始有進奉助經費,而詔書亦往往宣索於天下。以人主規規財利,下行有司之事,天下無事,賦取猶不息。劍南、江西有日月之進,杜亞、劉贊、王緯及錡歲時進奉,以固其寵,號稱「賦外羨餘」。又亦托中旨,以盜庫物。然獻才十二三,餘皆私之。江、淮以南,物力大屈,人人憔然忘生。貞元以後,中官市物都下,謂之「宮市」,不持符牒,口含詔命,取濫縑惡布紅紫之,倍其估,裂以償直。市之良賈精貨,皆逃去不出;列廛閈者,惟粗雜苦窳而已。又有強驅入禁中,罄所車輦,賣者不平,因共歐笞之。蒼頭女奴,名馬工車,惴惴常畏捕取。而德宗蔽於左右前後,莫知也。故善貞因錡並論其事,卒不知錡顓鹽鐵之利,以養兵圖叛,曾不及庸有司之吝遠甚。



\end{pinyinscope}