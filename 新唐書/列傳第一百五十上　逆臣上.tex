\article{列傳第一百五十上 逆臣上}

\begin{pinyinscope}

 安祿山,營州柳城胡也,本姓康。母阿史德,為覡,居突厥中自然科學家在日常生活和科學實踐中自發產生的唯物主義觀,禱子於軋犖山,虜所謂鬥戰神者,既而妊。及生,有光照穹廬,野獸盡鳴,望氣者言其祥。範陽節度使張仁願遣搜廬帳,欲盡殺之,匿而免。母以神所命,遂字軋犖山。少孤,隨母嫁虜將安延偃。開元初,偃攜以歸國,與將軍安道買亡子偕來,得依其家,故道買子安節厚德偃,約兩家子為兄弟,乃冒姓安,更名祿山。及長,忮忍多智,善億測人情,通六蕃語,為互市郎。



 張守珪節度幽州,祿山盜羊而獲,守珪將殺之,呼曰:「公不欲滅兩蕃邪?何殺我?」守珪壯其語,又見偉而皙,釋之,與史思明俱為捉生。知山川水泉處,嘗以五騎禽契丹數十人,守珪異之,稍益其兵,有討輒克,拔為偏將。守珪醜其肥,由是不敢飽,因養為子。後以平盧兵馬使擢特進、幽州節度副使。



 於是御史中丞張利貞採訪河北,祿山百計諛媚,多出金諧結左右為私恩。利貞入朝,盛言祿山能,乃授營州都督、平盧軍使、順化州刺史。使者往來,陰以賂中其嗜,一口更譽,玄宗始才之。天寶元年,以平廬為節度,祿山為之使,兼柳城太守,押兩蕃、渤海、黑水四府經略使。明年,入朝,奏對稱旨,進驃騎大將軍。又明年,代裴寬為範陽節度、河北採訪使,仍領平盧軍。祿山北還,詔中書門下尚書三省正員長官、御史中丞餞鴻臚亭。



 四載,奚、契丹殺公主以叛,祿山幸邀功,肆其侵,於是兩蕃貳。祿山起軍擊契丹,還奏;「夢李靖、李勣求食於臣,乃祠北郡,芝生於梁。」其詭誕敢言不疑如此。席豫為河北黜陟使,言祿山賢。時宰相李林甫嫌儒臣以戰功進,尊寵間己,乃請顓用蕃將,故帝寵祿山益牢,群議不能軋,卒亂天下,林甫啟之也。



 祿山陽為愚不敏蓋其奸,承間奏曰:「臣生蕃戎,寵榮過甚,無異材可用,願以身為陛下死。」天子以為誠,憐之。令見皇太子,不拜。左右擿語之,祿山曰:「臣不識朝廷儀,皇太子何官也?」帝曰:「吾百歲後付以位。」謝曰:「臣愚,知陛下不知太子,罪萬死。」乃再拜。時楊貴妃有寵,祿山請為妃養兒,帝許之。其拜,必先妃后帝,帝怪之,答曰:「蕃人先母后父。」帝大悅,命與楊銛及三夫人約為兄弟。繇是祿山有亂天下意,令麾下劉駱谷居京師,伺朝廷隙。



 六載,進御史大夫,封妻段為夫人,有國。林甫以宰相貴甚,群臣無敢鈞禮,惟祿山倚恩,入謁倨。林甫欲諷寤之,使與王鉷偕,鉷亦位大夫,林甫見金共,鉷趨拜卑約,祿山惕然,不覺自罄折。林甫與語,揣其意,迎剖其端,祿山大駭,以為神,每見,雖盛寒必流汗。林甫稍厚之,引至中書,覆以己袍。祿山德林甫,呼十郎。駱穀每奏事還,先問:「十郎何如?」有好言輒喜;若謂「大夫好檢校」,則反手據床曰:「我且死!」優人李龜年為帝學之,帝以為樂。



 晚益肥,腹緩及膝,奮兩肩若挽牽者乃能行,作《胡旋舞》帝前,乃疾如風。帝視其腹曰:「胡腹中何有而大?」答曰:「唯赤心耳!」每乘驛入朝,半道必易馬,號「大夫換馬臺」,不爾,馬輒僕,故馬必能負五石馳者乃勝載。帝為祿山起第京師,以中人督役,戒曰:「善為部署,祿山眼孔大,毋令笑我。」為瑣戶交疏,臺觀沼池華僭,帟幕率緹繡,金銀為榜筐、爪籬,大抵服御雖乘輿不能過。帝登勤政樓,幄坐之左張金雞大障,前置特榻,詔祿山坐,褰其幄,以示尊寵。太子諫曰:「自古幄坐非人臣當得,陛下寵祿山過甚,必驕。」帝曰:「胡有異相,我欲厭之。」



 時太平久,人忘戰,帝春秋高,嬖艷鉗固,李林甫、楊國忠更持權,綱紀大亂。祿山計天下可取,逆謀日熾,每過朝堂龍尾道,南北睥睨,久乃去。更築壘範陽北,號雄武城,峙兵積穀。養同羅、降奚、契丹曳落河八千人為假子,教家奴善弓矢者數百,畜單于、護真大馬三萬,牛羊五萬,引張通儒、李廷堅、平洌、李史魚、獨孤問俗署幕府,以高尚典書記,嚴莊掌簿最,阿史那承慶、安太清、安守忠、李歸仁、孫孝哲、蔡希德、牛廷玠、向潤客、高邈、李欽湊、李立節、崔乾祐、尹子奇、何千年、武令珣、能元皓、田承嗣、田乾真皆拔行伍,署大將。潛遣賈胡行諸道,歲輸財百萬。至大會,祿山踞重床,燎香,陳怪珍,胡人數百侍左右,引見諸賈,陳犧牲,女巫鼓舞於前以自神。陰令群賈市錦彩硃紫服數萬為叛資。月進牛、橐駝、鷹、狗、奇禽異物,以蠱帝心,而人不聊。自以無功而貴,見天子盛開邊,乃紿契丹諸酋,大置酒,毒焉,既酣,悉斬其首,先後殺數千人,獻馘闕下。帝不知,賜鐵券,封柳城郡公。又贈延偃範陽大都督,進祿山東平郡王。



 九載,兼河北道採訪處置使,賜永寧園為邸。入朝,楊國忠兄弟姊弟廷之新豐,給玉食;至湯,將校皆賜浴。帝幸望春宮以待,獻俘八千,詔賜永穆公主池觀為游燕地。徙新第,請墨敕召宰相宴。是日,帝將擊球,乃置會,命宰相皆赴。帝獵苑中,獲鮮禽,必馳賜。詔上谷郡置五爐,許鑄錢。又求兼河東,遂拜雲中太守、河東節度使。既兼制三道,意益侈。男子凡十一,帝以慶宗為太僕卿,慶緒鴻臚卿,慶長秘書監。



 十一載,率河東兵討契丹,告奚曰:「彼背盟,我將討之,爾助我乎?」奚為出徒兵二千鄉導。至土護真河,祿山計曰:「道雖遠,我疾趨賊,乘其不備,破之固矣。」乃敕人持一繩,欲盡縛契丹。晝夜行三百里,次天門嶺,會雨甚,弓弛矢脫不可用。祿山督戰急,大將何思德曰:「士方疲,宜少息,使使者盛陳利以脅賊,賊必降。」祿山怒,欲斬以令軍,乃請戰。思德貌類祿山,及戰,虜叢矛注矢邀取之,傳言祿山獲矣。奚聞亦叛,夾攻祿山營,士略盡。祿山中流矢,引奚兒數十,棄眾走山而墜,慶緒、孫孝哲掖出之,夜走平廬。部將史定方以兵鏖戰,虜解圍去。



 祿山不得志,乃悉兵號二十萬討契丹以報。帝聞,詔朔方節度使阿布思以師會。布思者,九姓首領也,偉貌多權略,開元初,為默啜所困,內屬,帝寵之。祿山雅忌其才,不相下,欲襲取之,故表請自助。布思懼而叛,轉入漠北,祿山不進,輒班師。會布思為回紇所掠,奔葛邏祿,祿山厚募其部落降之。葛邏祿懼,執布思送北庭,獻之京師。祿山已得布思眾,則兵雄天下,愈偃肆。皇太子及宰相屢言祿山反,帝不信。是時國忠疑隙已深,建言追還朝,以驗厥狀。祿山揣得其謀,乃馳入謁,帝意遂安,凡國忠所陳,無入者。



 十三載,來謁華清宮,對帝泣曰:「臣蕃人,不識文字,陛下擢以不次,國忠必欲殺臣以甘心。」帝慰解之。拜尚書左僕射,賜實封千戶,奴婢第產稱是,詔還鎮。又請為閑廄、隴右群牧等使,表吉溫自副。其軍中有功位將軍者五百人,中郎將二千人。祿山之還,帝禦望春亭以餞,斥御服賜之。祿山大驚,不自安,疾驅去。至淇門,輕艫循流下,萬夫挽繂而助,日三百里。既總閑牧,因擇良馬內範陽,又奪張文儼馬牧,反狀明白。人告言者,帝必縛與之。



 明年,國忠謀授祿山同中書門下平章事,召還朝。制未下,帝使中官輔璆琳賜大柑,因察非常。祿山厚賂之,還言無它,帝遂不召。未幾事洩,帝托它罪殺之,自是始疑。然祿山亦懼朝廷圖己,每使者至,稱疾不出,嚴衛然後見。黜陟使裴士淹行部至範陽,再旬不見,既而使武士挾引,無復臣禮,士淹宣詔還,不敢言。帝賜慶宗娶宗室女,手詔祿山觀禮,辭疾甚。獻馬三千匹,騶靮自倍,車三百乘,乘三士,因欲襲京師。河南尹達奚珣極言毋內騶兵,詔可。帝賜書曰:「為卿別治一湯,可會十月,朕待卿華清宮。」使至,祿山踞床曰:「天子安穩否?」乃送使者別館。使還,言曰:「臣幾死!」



 冬十一月,反範陽,詭言奉密詔討楊國忠,騰榜郡縣,以高尚、嚴莊為謀主,孫孝哲、高邈、張通儒、通晤為腹心,兵凡十五萬,號二十萬,師行日六十里。先三日,合大將置酒,觀繪圖,起燕至洛,山川險易攻守悉具,人人賜金帛,並授圖,約曰:「違者斬!」至是,如所素。祿山從牙門部曲百餘騎次城北,祭先塚而行。使賈循主留務,呂知誨守平廬,高秀巖守大同。燕老人叩馬諫,祿山使嚴莊好謂曰:「吾憂國之危,非私也。」禮遣之。因下令:「有沮軍者夷三族!」凡七日,反書聞,帝方在華清宮,中外失色。車駕還京師,斬慶宗,賜其妻康死,榮義郡主亦死。下詔切責祿山,許自歸。祿山答書慢甚,叵可忍。賊遣高邈、臧均以射生騎二十馳入太原,劫取尹楊光翽殺之,以張獻誠守定州。



 祿山謀逆十餘年,凡降蕃夷皆接以恩;有不服者,假兵脅制之;所得士,釋縛給湯沐、衣服,或重譯以達,故蕃夷情偽悉得之。祿山通夷語,躬自尉撫,皆釋俘囚為戰士,故其下樂輸死,所戰無前。邈最有謀,勸祿山取李光弼為左司馬,不納。既而悔之,憂見顏色,久而曰:「史思明可當之。」賊之未反,邈為謀,聲進生口,直取洛陽,無殺光翽,天下當未有知者,賊不從。何千年亦勸賊令高秀巖以兵三萬出振武,下朔方,誘諸蕃,取鹽、夏、鄜、坊,使李歸仁、張通儒以兵二萬道雲中,取太原,團弩士萬五千入蒲關,以動關中;勸祿山自將兵五萬梁河陽,取洛陽,使蔡希德、賈循以兵二萬絕海收淄、青,以搖江淮;則天下無復事矣。祿山弗用。



 時兵暴起,州縣發官鎧仗,皆穿朽鈍折不可用,持梃斗,弗能亢,吏皆棄城匿,或自殺,不則就禽,日不絕。禁衛皆市井徒,既授甲,不能脫弓示蜀、劍夬,乃發左藏庫繒帛大募兵。以封常清為範陽、平盧節度使,郭子儀為朔方節度、關內支度副大使,右羽林大將軍王承業為太原尹,衛尉卿張介然為汴州刺史,金吾將軍程千里為潞州長史,以榮王為元帥,高仙芝副之,馳驛討賊。



 祿山至鉅鹿,欲止,驚曰:「鹿,吾名。」去之沙河,或言如漢高祖不宿柏人以佞賊。賊投草頹樹於河,以長繩維舟集槎以結,冰一昔合,遂濟河,陷靈昌郡。又三日,下陳留、滎陽。次罌子穀,將軍荔非守瑜邀之,殺數百人,流矢及祿山輿,乃不敢前,更出谷南。守瑜矢盡,死於河。敗封常清,取東都,常清奔陜。殺留守李憕、御史中丞盧弈。河南尹達奚珣臣於賊。時高仙芝屯陜,聞常清敗,棄甲保潼關,太守竇廷芝奔河東。常山太守顏杲卿殺賊將李欽湊,禽高邈、何千年,於是趙郡、鉅鹿、廣平、清河、河間、景城六郡皆為國守,祿山所有才廬龍、密雲、漁陽、汲、鄴、陳留、滎陽、陜郡、臨汝而已。



 賊之據東京,見宮闕尊雄,銳情僭號,故兵久不西,而諸道兵得稍集。尹子奇屯陳留,欲東略,會濟南太守李隨、單父尉賈賁、濮陽人尚衡、東平太守嗣吳王祗、真源令張巡相繼起兵,旬日眾數萬。子奇至襄邑而還。



 明年正月,僭稱雄武皇帝,國號燕,建元聖武,子慶緒王晉,慶和王鄭,達奚珣為左相,張通儒為右相,嚴莊為御史大夫,署拜百官。復取常山,殺顏杲卿。安思義屯真定,會李光弼出土門救常山,思義降,博陵亦拔,唯稿城、九門二縣為賊守。史思明、李立節、蔡希德圍饒陽,不克,引軍攻石邑,張奉璋固守。朔方節度使郭子儀自雲中引兵與光弼合,敗思明於九門,李立節死,希德奔鉅鹿;思明奔趙郡,自鼓城襲博陵,復據之。光弼拔趙郡,還圍博陵,軍恆陽。希德請濟師於賊,賊以二萬騎涉滹沱入博陵,牛廷玠發媯、檀等兵萬人來助,思明益強,與光弼戰,敗於嘉山。光弼收郡十三,河南諸郡皆嚴兵守,潼關不開。



 祿山懼,谷還範陽,召嚴莊、高尚責曰:「我起,而曹謂萬全。今四方兵日盛,自關以西,不跬步進,爾謀何在,尚見我為?」遣尚等出。凡數日,田乾真自潼關來,勸祿山曰:「自古興王,戰皆有勝負,乃成大業,無一舉而得者。今四方兵雖多,非我敵也。有如事不成,吾擁數萬眾,尚可橫行天下,為十年計。且高尚、嚴莊,佐命元勛也,陛下何遽絕之,使自為患邪?」祿山喜,道其小字曰:「阿浩,非汝孰悟我!然則奈何?」乾真曰:「召而尉安之。」乃內尚等,與飲宴,祿山自歌,君臣如初。即遣孫孝哲、安神威西攻長安。會高仙芝等死,哥舒翰守潼關,為乾祐所敗,囚之。賊不謂天子能遽去,駐兵潼關,十日乃西。時行在已至扶風,於是汧、隴以東,皆沒於賊。祿山以張通儒守東京,乾真為京兆尹,使安守忠屯苑中。



 祿山未至長安,士人皆逃入山谷,東西駱驛二百里。宮嬪散匿行哭,將相第家委寶貨不貲,群不逞爭取之,累日不能盡。又剽左藏大盈庫,百司帑藏竭,乃火其餘。祿山至,怒,乃大索三日,民間財貲盡掠之,府縣因株根牽連,句剝苛急,百姓愈騷。祿山怨慶宗死,乃取帝近屬自霍國長公主、諸王妃妾、子孫姻婿等百餘人害之,以祭慶宗。群臣從天子者,誅滅其宗。虜性得所欲則肆為殘虐,人益不附。諸大將欲有咨決,皆因嚴莊以見。御下少恩,雖腹心雅故,皆為仇敵。郡縣相與殺守將,迎王師,前後反覆十數,城邑墟矣。



 肅宗治兵靈武,天下日跂首待。長安相傳太子西來矣,人聞輒東走,圜里至空,都畿豪桀殺賊吏自歸者無虛日,賊斬刈懲之不能止。又賊將類剽勇無遠謀,日縱酒,嗜聲色財利,車駕危得入蜀,終無進躡之患。



 帳下李豬兒者,本降豎,幼事祿山謹甚,使為閹人,愈親信。祿山腹大垂膝,每易衣,左右共舉之,豬兒為結帶。雖華清賜浴,亦許自隨。及老,愈肥,曲隱常瘡。既叛,不能無恚懼,至是目復盲,俄又得疽疾,尤卞躁,左右給侍,無罪輒死,或棰掠何辱,豬兒尤數,雖嚴莊親倚,時時遭笞靳,故二人深怨祿山。初,慶緒善騎射,未冠為鴻臚卿。賊僭號,嬖段夫人,愛其子慶恩,欲立之。慶緒懼不立,莊亦疑難作不利己,私語慶緒曰:「君聞大義滅親乎?自古固有不得已而為者。」慶緒陰曉曰:「唯唯。」又語豬兒曰:「汝事上罪可數乎?不行大事,死無日!」遂與定謀。至德二載正月朔,祿山朝群臣,創甚,罷。是夜,莊、慶緒持兵扈門,豬兒入帳下,以大刀斫其腹。祿山盲,捫佩刀不得,振幄柱呼曰:「是家賊!」俄而腸潰於床,即死,年五十餘罽,包以氈赩,埋床下。因傳疾甚,偽詔立慶緒為皇太子,又矯稱祿山傳位慶緒,乃偽尊太上皇。



 既襲偽位,改載初元年,即縱樂飲酒,委政於莊而兄事之。以張通儒、安守忠等屯長安,史思明領範陽,鎮恆陽軍,牛廷玠屯安陽,張志忠戍井陘,各募兵。



 於是廣平王率師東討,李嗣業將前軍,郭子儀將中軍,王思禮將後軍,回紇葉護以兵從。通儒等裒兵十萬陣長安中,賊皆奚,素畏回紇,既合,驚且囂。王分精兵與嗣業合擊之,守忠等大敗,引而東,通儒棄妻子奔陜郡。王師入長安,思禮清宮。僕固懷恩以回紇、南蠻、大食兵前驅,王悉師追賊,莊自將兵十萬與通儒合,鉦鼓震百餘里。尹子奇已殺張巡,悉眾十萬來,並力營陜西,次曲沃。先是回紇傍南山設伏,按軍北崦以待。莊大戰新店,以騎挑戰,六遇輒北,王師逐之,入賊壘。賊張兩翼攻之,追兵沒,王師亂,幾不能軍。嗣業馳,殊死鬥,回紇自南山繚擊其背,賊驚,遂亂。王師復振,合攻之,殺掠不勝算,賊大敗,追奔五十餘里,尸髀藉藉滿坑壑,鎧仗狼扈,自陜屬於洛。莊跳還,與慶緒、守忠、通儒等劫殘軍走鄴郡。



 王入洛陽,大陳兵天津橋。偽侍中陳希烈等三百人素服叩頭待罪,王勞曰:「公等脅污,非反也,天子有詔赦罪,皆復而官。」眾大喜。於是陳留殺賊將尹子奇以降。莊妻薛舍獲嘉,紿言永王女,詣營,及見王,辭曰:「莊欲降,願得一信。」王與子儀謀,莊若至者,餘黨可諭而下,乃約莊賜鐵券。莊乃降,乘驛至京師,肅宗引見,釋其死,授司農卿。阿史那承慶其以眾三萬奔恆、趙,或趨範陽,其從慶緒者,痍卒才千餘。



 會蔡希德自上黨,田承嗣自潁川,武令珣自南陽,各以眾來,邢、衛、洺、魏募兵稍稍集,眾六萬,賊復振。以相州為成安府,太守為尹,改元天和,以高尚、平洌為宰相,崔乾佑、孫孝哲、牛廷玠為將,以阿史那承慶為獻城郡王,安守忠左威衛大將軍,阿史那從禮左羽林大將軍。然部黨益攜解,由是能元皓以偽淄青節度使、高秀巖以河東節度使並納順。德州刺史王暕、貝州刺史宇文寬皆背賊自歸,河北諸軍各嬰城守,賊使蔡希德、安雄俊、安太清等以兵攻陷之,戮於市、膾其肉。



 慶緒懼人之貳己,設壇加載書、踠血與群臣盟。然承慶等十餘人送密款,有詔以承慶為太保、定襄郡王,守忠左羽林軍大將軍、歸德郡王,從禮太傅、順義郡王,蔡希德德州刺史,李廷讓邢州刺史,苻敬超洺州刺史,楊宗太子左諭德,任瑗明州刺史,獨孤允陳州刺史,楊日休洋州刺史,恭榮光岐陽令;自裨校等,數數為國間賊。而慶緒治宮室、觀榭、塘沼,泛樓舡為水嬉,長夜飲。通儒等爭權不能一,凡有建白,眾共訾沮之。希德最有謀,剛狷,謀殺慶緒為內應,通儒以它事斬之,麾下數千皆亡去。希德素得士,舉軍恨嘆。慶緒以乾佑為天下兵馬使,權震中外,愎悍少恩,士不附。



 乾元元年秋九月,帝詔郭子儀率九節度兵凡二十萬討慶緒,攻衛州,遂度河,師背水壁而待。慶緒遣安太清拒戰,聞衛州已圍,則鼓而南,作三軍:乾佑將上軍,雄俊、王福德佐之;田承嗣將下軍,榮敬佐之;慶緒自將中軍,孫孝哲、薛嵩佐之。既戰,王師偽卻,慶緒逐之,遇伏而潰慶。緒走,獲其弟慶和,斬於京師。子儀引軍躡賊,戰愁思崗,賊復敗,自是銳兵盡矣。因嬰鄴自固,使薛嵩以厚幣求救於史思明。思明遣李歸仁將兵萬三千壁滏陽,未進,而王師圍已固,築浚城隍三周,決安陽水灌城。城中棧而處,糧盡,易口以食,米斗錢七萬餘,一鼠錢數千,屑松飼馬,隤墻取麥秸,濯糞取芻,城中欲降不得。賊更以太清代乾佑將。



 於是思明有眾十三萬,三分其軍趨鄴。明年三月,營安陽。慶緒急,乃遣太清奉皇帝璽綬讓思明。思明以書示軍中,咸呼萬歲,乃約慶緒為兄弟,還其書,慶緒大悅。王師不利,九節度奔還,子儀斷河陽橋,戍穀水。思明進屯鄴南。慶緒收官軍餘餉,尚十餘萬石。召孝哲等謀拒思明,諸將皆曰:「今日安得復背史王乎?」通儒、尚、洌皆請自往謝思明,慶緒許諾。思明見,為流涕,厚禮遣還。三日,慶緒未出,思明請慶緒歃血盟,不得已,以五百騎詣思明軍。先此,思明令軍中擐甲待,慶緒至,再拜伏地謝曰:「臣不克負荷,棄兩都,陷重圍,不意大王以太上皇故,暴師遠來,臣之罪,唯王圖之。」思明恚曰:「兵利不利亦何事,而為人子,殺父求位,非大逆邪?吾乃為太上皇討賊。」顧左右牽出斬之。慶緒數目周萬志,萬志進曰:「慶緒為君矣,宜賜死。」乃並四弟縊。又誅尚、孝哲、乾佑,殊而膊之。思明改葬祿山以王禮,偽謚燕剌王。祿山父子僭位凡三年而滅。



 初,祿山陷東京,以張萬頃為河南尹,士人宗室賴以免者眾,肅宗嘉其仁,拜濮陽太守。帝以賊國讎,惡聞其姓,京師坊里有「安」字者,悉易之。



 高尚者,雍奴人。母老,丐食自給,尚客河朔不肯歸。與令狐潮相善,淫其婢,生一女,遂留居。然篤學善文辭,嘗喟然謂汝南周銑曰:「吾當作賊死,不能齕草根求活也。」李齊物為新平太守,薦諸朝,贐錢三萬,介之見高力士。力士以為才,置門下,家事一咨之,諷近臣表其能,擢左領軍倉曹參軍。



 力士語祿山,表為平盧掌書記,因出入臥內。祿山喜睡,尚嘗執筆侍,通昔不寢,繇是親愛。遂與嚴莊語圖讖,導祿山反。陷東都,偽拜中書侍郎。大抵賊所下赦令,皆尚為之。嚴莊降後,尚獨典政事,至偽侍中。



 孫孝哲者,契丹部人。母冶色,祿山通之,故孝哲得狎近。長七尺,伉健有謀。祿山對側門俟召,衣帶絕,不知所為,孝哲箴縷素具,徐為紉綻,祿山大悅。尤能先事取情。祿山魁大,非孝哲縫衣不能勝。天寶末,官大將軍。



 賊僭位,偽拜殿中監、閑廄使,爵為王,與嚴莊爭寵不平。裘馬光侈,食輒珍滋。賊令監張通儒等守長安,人皆目之。殺妃、主、宗室子百餘人,窮誅楊國忠、高力士黨與及與賊忤者不勝計,剔首析肢,流離道衢。祿山死,莊奪其使以與鄧季陽。慶緒之奔,莊懼為所圖,因降。



 有商胡康廉者,天寶中為安南都護,附楊國忠,官將軍。上元中,出家貲佐山南驛稟,肅宗喜其濟,許之,累試鴻臚卿。婿在賊中,有告其畔,坐誅。事連莊,系獄,貶難江尉。京兆尹劉晏發吏防其家,莊恨之。俄詔釋罪,莊入見代宗,誣晏常矜功怨上,漏禁中事,晏遂貶云。



 史思明,寧夷州突厥種,初名窣于,玄宗賜其名。姿臒露,鳶肩傴背,僨目側鼻,寡須發,躁健譎狡。與安祿山共鄉里,生先祿山一日,故長相善。少事特進烏知義,以輕騎覘賊,多所禽馘。通六蕃譯,亦為互市郎。頃之,負官錢,無以償,將走奚。未至,為邏騎所困,欲殺之,紿曰:「我使人也,若聞殺天子使者,其國不祥,不如以我見王,王活我,功自汝得。」邏以為然,送至王所,不拜,曰:「天子使見小國君不拜,禮也。」王怒,然疑真使者,卒授館,待以禮。將還,令百人從入朝。奚有部將瑣高者,名聞國中,思明欲禽以贖罪,訹王曰:「從我者雖多,無足與見天子者,惟高材,可與至中國。」王悅,命高將帳下三百俱。既至平廬,遣謂戍主曰:「奚兵數百,外稱入朝,內實盜,請備之。」主潛師迎犒,殺其眾,囚高以獻。幽州節度使張守珪奇其功,表折沖,與祿山俱為捉生。



 天寶初,累功至將軍、知平廬軍事。入奏,帝賜坐與語,奇之。問年,曰:「四十矣。」撫其背曰:「爾貴在晚,勉之!」遷大將軍、北平太守。從祿山討契丹,祿山敗,單騎走師州,殺其下左賢哥解、魚承仙自解。思明逃山中,再閱旬,裒散卒得七百,追見祿山平盧,祿山喜,握手曰:「計而死矣,今故在,吾何憂!」思明親密曰:「吾聞進退在時,向蚤出,隨哥解地下矣。」契丹取師州,守捉使劉客奴亡去,祿山使思明擊走之,表平盧兵馬使。



 思明少賤,鄉里易之。大豪辛氏有女,方求婿,窺思明,告其親曰:「必嫁我思明。」宗屬不可,女固以歸。思明亦負曰:「自我得婦,官不休,生男子多,殆且貴乎!」



 祿山反,使思明略定河北,會賈循死,留思明守範陽,而常山顏杲卿等傳檄拒賊,祿山使向潤客等代,遣思明攻常山,九日執杲卿。進薄饒陽,盧全誠拒守,河間、景城、平原、樂安、清河、博平六郡稍募兵自固。河間李奐以兵七千救饒陽,景城李持兵八千助河間,平原顏真卿以兵六千助清河,悉為思明所敗,子杞死之,饒陽愈堅。會李光弼收常山,思明遽解圍迎戰,晝夜行二百里,相持久不決。郭子儀取趙郡,合兵攻賊。凡再戰,皆大敗,走入博陵。光弼追傅城,幾拔。屬潼關潰,肅宗召朔方、河東兵,光弼引還,使王俌守常山。賊尾追光弼於井陘,敗歸。攻平盧,劉正臣輕之,不設備,敗保北平,兵貲二千乘皆沒。思明得其銳卒,張甚,謀攻常山。俌欲降,諸將殺之,遣使至信都迎刺史鳥承恩鎮守,不聽。思明攻土門,城中伏甲詭降,賊登城,伏起,賊殲;思明中戟,扶以免。復攻陷之,焚廬舍,種誅其人。取稿城,守將白嘉祐走趙郡,思明圍之五日,入之,嘉祐奔太原,思明再陷常山。賊別帥尹子奇圍河間,顏真卿遣和琳將兵萬餘往救之。於是北風號勁,鼓之,士不進。賊縱擊,大敗,執琳,引眾攻城,禽李奐。又拔景城,李赴河死。招樂安,降之。遂攻平原,未至,真卿棄郡去。進破清河,執太守王懷忠,入博平,遂圍信都。初,賊先獲承恩母、妻及子,故承恩降,而兵尚五萬,騎三千。擊饒陽,李系自燔死。



 思明兵所向,縱其下椎剽,淫奪人妻女,以是士最奮。是時,舉河北悉入賊,生人貲產掃地,壯齎負,老嬰則殺之,殺人以為戲。祿山偽署範陽節度使。始,麾下騎才二千,同羅步曳落河止三千,既數勝,兵最強,狺然有噬江、漢心。以精卒五萬畀尹子奇,度河劫北海以震淮、徐。會回紇襲範陽,範陽閉不出,子奇乃還救,遂不克。至德二載,與蔡希德、高秀巖合兵十萬攻太原。是時,李光弼使部將張奉璋以兵守故關,思明攻陷之,奉璋走樂平。思明取攻具山東,奉璋匿士廣陽,改服紿為賊使者,責其後期,斬數人,引眾得還太原。時光弼固守且十月,不能拔。而安慶緒襲位,賜姓安,名榮國,爵媯川郡王。



 賊之陷兩京,常以橐它載禁府珍寶貯範陽,如丘阜然。思明見富強,心間然驕,欲自取之。已而慶緒敗走相州,殘士三萬北歸,無所屬,思明擊殺數千人,降之。慶緒知其貳,使阿史那承慶、安守忠、李立節詣思明議事,且共圖之。判官耿仁智欲以大誼動賊,請間曰:「公貴且賢,無待下為之謀,然請一言而死。」思明曰:「為我言之。」對曰:「方祿山強,誰敢不服?大夫事之,固無罪。今天子聰明勇智,有少康、宣王風,公誠發使輸誠,無不納,此轉禍入福之秋也。」思明曰:「善。」承慶等未知,以五千騎來,思明介而勞,前謂曰:「公等至,士不勝喜,然邊兵素憚使者威,不自安,請弛弓以入。」從之。思明從承慶等飲,即拘之,收其兵,給貲以遣,斬守忠、立節以徇。



 李光弼聞其絕慶緒,使人招之。前此烏承恩已歸國,帝遣鐫諭之,思明使牙門金如意奉十三郡兵八萬籍歸於朝,於是高秀巖以河東自歸。有詔思明為歸義郡王、範陽長史、河北節度使,諸子並列卿;以秀巖為雲中太守,亦官其諸子。遣承恩與中人李思敬尉撫,趣討殘賊。思明乃遣張忠志守幽州,假薛萼以恆州刺史,招趙州刺史陸濟使降,授朝義兵五千守冀州,假令狐彰博州刺史,戍滑州。



 然思明外順命,內實通賊,益募兵。帝知之,以其常事承恩父知義,冀其無嫌,即擢承恩為河北節度副大使,使圖思明。承恩至範陽,羸服夜過諸將,陰諗以謀,諸將返以告思明,疑未有以驗。會承恩與思敬奏事還,思明留館之,幃所寢床,伏二人焉。承恩子入見,因留臥。夜半,語其子曰:「吾受命除此逆胡。」二人白思明,乃執承恩,探衣囊得賜阿史那承慶鐵券及光弼牒,又得薄紙書數番,皆當誅將士姓名。賊大詬曰:「我何負於爾,至是邪!」故答曰:「此太尉光弼謀,上不知也。」思明召官吏於廷,西向哭曰:「臣赤心不負國,何至殺臣?」因榜殺承恩父子及支黨二百餘人,囚思敬以聞。帝遣使諭曰:「事出承恩,非朕與光弼意。」又聞三司議陳希烈等死,思明懼曰:「希烈等皆大臣,上皇棄而西,既復位,此等宜見勞,返殺之,況我本從祿山反乎?」諸將皆勸賊表天子誅光弼。思明使耿仁智、張不矜上疏請斬光弼,不然,且攻太原。疏入於函,仁智輒易去。左右密白思明,執二人曰:「負我邪!」命斬之。既又欲貸死,復召責曰:「仁智事我三十年,今日我忘爾邪?」仁智怒曰:「人固有死,大夫納邪說,再圖反,我雖生不如死!」思明怒,捶殺之。九節度圍相州急,慶緒間道求救,思明懼王師,未敢進。俄而蕭華舉魏州歸天子,崔光遠代守,思明乃引兵擊魏,拔之,殺數萬人。



 乾元二年正月朔,築壇,僭稱大聖周王,建元應天,以周贄為司馬;救相州,卻王師,殺慶緒,並其眾,欲遂西略,虞根本未固,即留朝義守相州,自引還。夏四月,更國號大燕,建元順天,自稱應天皇帝。妻辛為皇后,以朝義為懷王,周贄為相,李歸仁為將;號範陽為燕京,洛陽周京,長安秦京。更以州為郡,鑄「順天得一」錢。欲郊及藉田,聘儒生講制度。或上書言:「北有兩蕃,西有二都,勝負未可知,而為太平事,難矣。」思明不悅,遂祠祀上帝。是日大風,不能郊。



 留子朝清守幽州,使阿史那玉、向貢、張通儒、高如震、高久仁、王東武等輔之。兵四出寇河南,身出濮陽,使令狐彰絕黎陽,朝義出白高,周萬志自胡良度河圍汴州。於是節度使許叔冀,濮州刺史董秦,梁浦、田神功皆附賊,即命叔冀與李祥守汴州,徙秦等家屬平盧,使浦、神功下江、淮,約曰:「得地,人取貲二艫。」思明乘勝鼓行,西陷洛陽,破汝、鄭、滑三州,圍李光弼河陽,不能拔。使安太清取懷州以守,光弼攻之,太清降。思明又遣田承嗣擊申、光等州,王同芝擊陳,許敬釭擊兗、鄆,薛萼擊曹。上元二年二月,思明以計敗光弼兵於北邙,王師棄河陽、懷州,京師震恐,益兵屯陜州。思明遂西,使朝義為先鋒,身自宜陽繼進。



 朝義攻陜,敗於姜子阪,退壁永寧。思明大怒,召朝義並駱悅、蔡文景、許季常,將誅而釋之,詫曰:「朝義怯,不能成我事!」欲追朝清自副。又敕朝義築三角城居糧,終日畢,未塓而思明至,怒不如約,辭曰:「士疲少息耳。」思明曰:「汝惜士而違我令邪?」據鞍畢塓乃去,顧曰:「朝下陜,夕斬是賊。」朝義懼。思明居傳舍,令所愛曹將軍擊刁斗呵衛。駱悅等被讓,即共說朝義曰:「向兵敗,悅與王死無日,不如召曹將軍同計大事。」朝義面不應。悅曰:「王誠不忍,吾等且歸唐,不得事王矣。」朝義許之,令季常以言動曹將軍。曹將軍畏諸將,不敢拒。思明愛優諢,寢食常在側,優者以其忍,恨之。是夜思明驚,據床叱吒。優問故,答曰:「我夢群鹿度水,鹿死而水乾,云何?」俄如匽,優相謂曰:「胡命盡乎!」少選,悅麾下周子俊射其臂,墜,問難所起,曰:「懷王也。」思明曰:「旦日失言,宜有此。然殺我太早,使我不得至長安。」大呼懷王三,曰:「囚我可也,無取殺父名!」復罵曹將軍曰:「胡誤我!」左右反接縛之,送柳泉傳舍。悅還報,朝義曰:「驚聖人否?損聖人否?」悅曰:「無有。」時周贄、許叔冀以後軍屯福昌,季常,叔冀子也,朝義令告之。贄聞,驚僕地。賊領兵還,贄等出迎,悅惡其貳,乃殺贄。次柳泉,悅畏眾不厭,縊殺思明,以氈裹尸,橐它負還東京。朝義乃即位,建元顯聖。



 初,思明諸子無嫡庶分,以少者為尊。朝義,孽長子,寬厚,下多附者。及難起,陰令向貢、阿史那玉圖朝清。朝清喜田獵,戕虐似思明,淫酗過之,養帳下三千人,皆剽賊輕死。貢紿計曰:「聞上欲以王為太子,且車駕在遠,王宜入侍。」朝清謂然,趣帳下出治裝,貢使高久仁、高如震率壯士入牙城。朝清問其故,或曰:「軍叛矣。」乃擐甲登樓,責貢等,士陣樓下,朝清自射殺數人,阿史那玉軍偽北,朝清下,被執,與母辛俱死。張通儒不知,引兵戰城中,數日不克,亦死。貢攝軍事,未幾,玉襲殺之,自為長史,治殺朝清罪,乃梟久仁,徇於軍。如震懼,擁兵拒守。五日,玉敗走武清,朝義使人招之,至東都,凡胡面者,無長少悉誅。以李懷仙為幽州節度使,斬如震,幽州乃定。



 朝義虛懷禮下,事皆決大臣,然無經略才。當此時,洛陽諸郡人相食,城邑榛墟,又諸將皆祿山舊臣,與思明故輩行,恥為朝義屈,召兵輒不至,欲還幽州。



 會雍王以河東、朔方、回紇兵十餘萬討賊,僕固懷恩與回紇左殺為先鋒,魚朝恩、郭英乂殿,入自黽池,李抱玉薄河陽,李光弼徑陳留,合兵。始,代宗召南北軍諸將問所以討賊計,開府儀同三司管崇嗣曰:「我得回紇,無不勝。」帝曰:「未也。」右金吾大將軍薛景仙曰:「我若不勝,請以勇士二萬椎鋒死賊。」帝員:「壯矣!」右金吾大將軍長孫全緒曰:「賊若背城戰,破之必矣;若閉城留死,未可取也。且回紇短於攻城,持久勢且沮。我若休士張勢以綴賊,使光弼取陳留,抱玉搗河北,先斷其手足,然後縱間賊中,彼脅從者相疑,則滅可待。」帝曰:「善。」命潼關、陜戒嚴。師次洛陽,馳兵下懷州,王師部伍靜嚴,賊有懼色。



 朝義以師十萬距橫水,戰大敗,俘馘凡六萬,委牛馬器甲不可計。朝義燒明堂,東奔汴州,偽節度使張獻誠不納,自濮北趣幽州。東都再更亂,英乂、朝恩等不能戢軍,與回紇縱掠,延及鄭、汝,閭井至無煙。方冽寒,人皆連紙褫書為裳示俞。賊走至下博,僕固瑒追及之,朝義復敗。河東戍將李竭誠、成德李令崇皆背賊掎角戰。至漳水,無舟,諸將勸降,朝義不悅。田承嗣請環車為營,內女子車中,以輜重次之,伏兵以待。既戰而卻,王師逐之,爭貲寶,賊引奇兵繞出,又伏發,王師卻數十里止。朝義遂走莫州,瑒追圍之。閱四旬,賊八戰八奔。明年正月,閱精兵,欲決死。承嗣謂朝義:「不如身將驍銳還幽州,因懷仙悉兵五萬還戰,聲勢外張,勝可萬全。臣請堅守,雖瑒之強,不遽下。」朝義然納,以騎五千夜出,比行,握承嗣手,以存亡為托。承嗣頓首流涕。將行,復曰:「闔門百口,母老子稚,今付公矣。」承嗣聽命。少選,集諸將曰:「吾與公等事燕,下河北百五十餘城,發人塚墓,焚人室廬,掠人玉帛,壯者死鋒刃,弱者填溝壑,公門華胄,為我廝隸,齊姜、宋子,為我掃除。今天降鑒,吾等安所歸命?自古禍福亦不常,能改往修今,是轉危即安矣。旦日且出降,公等謂何?」眾咸曰:「善。」黎明,使人號城上曰:「朝義夜半走矣,胡不追賊?」信未信,承嗣將朝義母及妻孺詣瑒壘,於是諸軍率輕兵追之。



 朝義至範陽,懷仙部將李抱忠閉壁不受,曰:「頃既受命天子,一年之中,且降且叛,二三孰甚焉!」朝義告饑,抱忠饋於野。朝義飯,軍亦飯,飯已,軍子弟稍稍辭去。朝義流涕罵承嗣曰:「老奴誤我!」去至梁鄉,拜思明墓,東走廣陽,不受。謀奔兩蕃,懷仙招之,自漁陽回止幽州,縊死醫巫閭祠下。懷仙斬其首傳長安,召故將收其尸。懷仙改服出次哭之,士皆號慟。及葬,莫知其所。偽恆州刺史張忠志、趙州刺史盧俶、定州刺史程元勝、徐州刺史劉如伶、相州節度使薛嵩及懷仙、承嗣等皆舉其地以歸。思明父子僭號凡四年滅。朝義死,部送將士妻口百餘於官,有司請隸司農,帝曰:「是皆良家子,脅掠至此。」命稟食還其親;無所歸者,官為資遣。



 贊曰:祿山、思明興夷奴餓俘,假天子恩幸,遂亂天下。彼能以臣反君,而其子亦能賊殺其父,事之好還,天道固然。然生民厄會,必假手於人者,故二賊暴興而亟滅。張謂譏劉裕「近希曹、馬,遠棄桓、文,禍徒及於兩朝,福未盈於三載,八葉傳其世嗣,六君不以壽終,天之報施,其明驗乎!」杜牧謂:「相工稱隨文帝當為帝者,後篡竊果得之。周末,楊氏為作八柱國,公侯相襲久矣,一旦以男子偷竊位號,不三二十年,壯老嬰兒皆不得其死。彼知相法者,當曰此必為楊氏之禍,乃可為善相人。」張、杜確論,至今多稱誦之。如祿山、思明,希劉裕、楊堅而不至者,是以著其論。



\end{pinyinscope}