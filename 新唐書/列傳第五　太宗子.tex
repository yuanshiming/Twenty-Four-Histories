\article{列傳第五 太宗子}

\begin{pinyinscope}

 太宗十四子:文德皇后生承乾,又生第四子泰、高宗皇帝,後宮生寬,楊妃生恪,又生第六子愔邀請赴瑞典講學,次年卒於此。他是近代哲學創始人之一,反,陰妃生祐,王氏生惲,燕妃生貞,又生第十一子囂,韋妃生慎,後宮生簡,楊妃生福,楊氏生明。



 常山愍王承乾字高明,生承乾殿,即以命之。武德三年,始王常山郡,與長沙、宜都二王同封。俄徙中山。太宗即位,立為皇太子。甫八歲,特敏惠,帝愛之。在諒暗,使裁決庶政,有大體,後每行幸,則令監國。及長,好聲色慢游,然懼帝,秘其跡。臨朝,言諄諄必忠孝,退乃與群不逞狎慢。左右或進諫,危坐斂容,痛自咎,飾非辯給,諫者拜答不暇,故人人以為賢而莫之察。後過惡寢聞,宮臣若孔穎達、令狐德棻、於志寧、張玄素、趙弘智、王仁表、崔知機等皆天下選,每規爭承乾,帝必厚賜金帛,欲以厲其心。承乾慠不悛,往往遣人陰圖害之。時魏王泰有美名,帝愛重。而承乾病足,不良行,且懼廢,與泰交惡。泰亦謀奪長,各樹黨。東宮有俳兒,善姿首,承乾嬖愛,帝聞震怒,收兒殺之,坐死者數人。承乾意為泰告,望甚。內念兒不已,築室圖其象,贈官樹碑,為起塚苑中,朝夕祭。承乾至其處裴回,涕數行下,愈怨懟,稱疾不朝,累數月。又使戶奴數十百人習音聲,學胡人椎髻,剪彩為舞衣,尋橦跳劍,鼓鞞聲通晝夜不絕。造大銅爐、六熟鼎,招亡奴盜取人牛馬,親視烹燖,召所幸廝養共食之。又好突厥言及所服,選貌類胡者,被以羊裘,辮發,五人建一落,張氈舍,造五狼頭纛,分戟為陣,系幡旗,設穹廬自居,使諸部斂羊以烹,抽佩刀割肉相啗。承乾身作可汗死。使眾號哭剺面,奔馬環臨之。忽復起曰:「使我有天下,將數萬騎到金城,然後解發,委身思摩,當一設,顧不快邪!」左右私相語,以為妖。又襞氈為鎧,列丹幟,勒部陣,與漢王元昌分統,大呼擊刺為樂。不用命者,披樹抶之,或至死,輕者輒腐之。嘗曰:「我作天子,當肆吾欲;有諫者,我殺之,殺五百人,豈不定?」又召壯士左衛副率封師進、刺客張師政、紇干承基等謀殺魏王泰,不克,遂與元昌、侯君集、李安儼、趙節、杜荷鑱臂血■之,謀以兵入西宮。貞觀十七年,齊王祐反齊州,承乾謂承基等:「我宮西墻,去大內正可二十步棘耳,豈與齊州等?」會承基連齊王事系獄當死,即上變。帝詔長孫無忌、房玄齡、蕭瑀、李勣、孫伏伽、岑文本、馬周、褚遂良雜治,廢為庶人,徙黔州。十九年死,帝為廢朝,葬以國公禮。子象,為懷州別駕,厥鄂州別駕。開元中,象子適之為宰相,贈還承乾始王,象越州都督、郇國公。楚王寬,武德三年,出後楚哀王,蚤薨,貞觀初追封。



 鬱林王恪,始王長沙,俄進封漢。貞觀二年徙蜀,與越、燕二王同封。不之國,久乃為齊州都督。帝謂左右曰:』吾於恪豈不欲常見之?但令早有定分,使外作籓屏,吾百歲後,庶兄弟無危亡憂。」十年,改王吳,與魏、齊、蜀、蔣、越、紀六王同徙封。授安州都督。帝賜書曰:「汝惟茂親,勉思所以籓王室,以義制事,以禮制心。外之為君臣,內之為父子,今當去膝下,不遺汝珍,而遺汝以言,其念之哉!」坐與乳媼子博塞,罷都督,削封戶三百。高宗即位,拜司空、梁州都督。恪善騎射,有文武才。其母隋煬帝女,地親望高,中外所向。帝初以晉王為太子,又欲立恪,長孫無忌固爭,帝曰:「公豈以非己甥邪?且兒英果類我,若保護舅氏,未可知。」無忌曰:「晉王仁厚,守文之良主,且舉棋不定則敗,況儲位乎?」帝乃止。故無忌常惡之。永徽中,房遺愛謀反,因遂誅恪,以絕天下望。臨刑呼曰:「社稷有靈,無忌且族滅!」四子,仁、瑋、琨、璄並流嶺表。顯慶五年,追王鬱林,為立廟,以河間王孝恭孫榮為鬱林縣侯以嗣。神龍初,贈司空,備禮改葬。光宅中,仁遇赦還,適會榮以罪斥,故得襲鬱林縣男,歷岳州別駕,爵郡公。嘗使江左,州人遺以金,拒不內。武后遣使者勞曰:「兒,吾家千里駒。」更名千里。自天授後,宗室賢者多株剪,唯千里詭躁不情,數進符瑞諸異物,得免。中宗反正,改王成紀。未幾,進王成。節愍太子誅武三思,千里與其子天水王禧率數十人斬右延明門以入。太子敗,誅死,籍其家,改氏「蝮」。睿宗立,詔還氏及官爵。瑋蚤卒,中宗追封朗陵王。子示玄,出繼蜀王愔。開元中,以傍繼國改封廣漢郡王,遷太僕卿同正員,薨。



 琨,武后時歷六州刺史,皆有名。聖歷中,為嶺南招慰使,安輯反獠,甚得其宜。卒,贈司衛卿。神龍初,贈張掖郡王。開元中,以子禕貴,追封吳王。



 禕少有志尚,事繼母謹,撫異母弟祗,以友稱。當襲封,固讓祗,中宗嘉其意,特封嗣江王,以繼囂後。開元時,亦以傍繼徙信安郡王。累為州刺史,治嚴辦。遷禮部尚書、朔方節度使。



 初,吐蕃據石堡城,數盜塞,詔禕與河西、隴右議攻取。既到屯,諏日進師。或謂:「城險,賊所愛,必固守。今兵深入,有如不捷,吾軍必奔,不如持重伺賊勢。」禕曰:「人臣之節,豈憚險不進乎?必眾寡不敵者,吾以死繼之。」於是分兵迮賊路,督諸將倍道進,遂拔之。自是河、隴諸軍游弈,拓地至千里。玄宗喜,更號其城曰振武軍。契丹牙官可突於叛,詔拜忠王為河北道行軍元帥討之,敕禕以副。王不行,故禕率裴耀卿諸將分道出範陽北,擊二蕃,破之,禽酋長以還,餘部竄伏。加開府儀同三司,領關內支度營田採訪處置使,授二子官。禕功多,執政害之,賞不讎,為當時所恨。久之,擢兵部尚書,為朔方節度大使。坐事下除衢州刺史。歷滑、懷二州。天寶初,以太子少師致仕。明年,遷太師,未拜,薨。禕治家嚴,教子有法度,故峘、嶧、峴皆顯。



 峘性質厚,歷宦有美名,以王孫封趙國公。楊國忠亂政,悉斥不附己者。峘由考功郎中拜睢陽太守,以清簡為二千石最。方入計,而玄宗入蜀,即走行在。除武部侍郎,兼御史大夫。俄拜蜀郡太守、劍南節度採訪使。郭千仞反,與陳玄禮共討平之。上皇還京,遷戶部尚書,改越國。乾元元年,持節都統江淮節度宣慰觀察使。都統之號,自峘始。明年,宋州刺史劉展有異志,詔拜展為淮南節度使,密詔峘與楊州長史鄧景山圖之。時展強扈,既受詔,即悉兵度淮,峘、景山拒之,戰壽春,敗績,峘走丹陽。詔貶袁州司馬,卒於官,贈揚州大都督。弟峴別傳。



 祗封嗣吳王,出為東平太守。安祿山反,河南、陳留、滎陽、靈昌相繼陷,祗募兵拒賊,玄宗壯之。累遷陳留太守,持節河南道節度採訪使。歷太僕、宗正卿。代宗大歷時,祗既宗室老,以太子賓客為集賢院待制。是時,勛望大臣無職事者皆得待詔於院,給飧錢署舍以厚其禮,自左僕射裴冕等十三人為之。子巘,以廕補五品官。祗薨,兄岵得罪,乃以巘嗣王。累至宗正卿,檢校刑部尚書。薨,贈太子少保。性介直,面刺人短。歷官清白,居室不能庇風雨。收恤甥侄,慈愛過人,家無留儲,公卿合賻乃克葬。璄,神龍初封歸政郡王,歷宗正卿,坐千里事,貶南州司馬。



 濮恭王泰字惠褒。始王宜都,徙封衛,繼懷王後。又徙封越,為揚州大都督。再遷雍州牧、左武候大將軍。改王魏。帝以泰好士,善屬文,詔即府置文學館,得自引學士。又以泰大腰腹,聽乘小輿至朝。司馬蘇勖勸泰延賓客著書,如古賢王。泰乃奏撰《括地志》,於是引著作郎蕭德言、秘書郎顧胤、記室參軍蔣亞卿、功曹參軍謝偃等撰次。衛尉供帳,光祿給食,士有文學者多與,而貴游子弟更相因藉,門若市然。泰悟其過,欲速成,乃分道計州,繙緝疏錄,凡五百五十篇,歷四期成。詔藏秘閣,所賜萬段。後帝幸泰延康坊第,曲赦長安死罪,免坊人一年租,府僚以差賜帛。又泰月稟過皇太子遠甚,諫議大夫褚遂良諫曰:「聖人尊嫡卑庶,謂之儲君,故用物不會,與王共之,庶子不得為比,所以塞嫌萌,杜禍源。先王法制,本諸人情,知有國家者必有嫡庶,庶子雖愛,不得過嫡子。如當親者疏,當尊者卑,則私恩害公,惑志亂國。今魏王稟料過東宮,議者以為非是。昔漢竇太后愛梁王,封四十餘城。王築苑三百里,治宮室,為復道,費財巨萬,出警人蹕,一不得意,遂發病死。宣帝亦驕淮陽王,幾至於敗,輔以退讓之臣,乃克免。今魏王新出閤,且當示以節儉,自可在後月加歲增。又宜擇師傅,教以謙儉,勉以文學,就成德器,此所謂聖人之教,不肅而成也。」帝又敕泰入居武德殿,侍中魏徵亦言:「王為陛下愛子,欲安全之,則不當使居嫌疑之地。今武德殿在東宮之西,昔海陵居之矣,論者為不可。雖時與事異,人之多言,尚或可畏。又王之心亦弗遑舍,願罷之,成王以寵為懼之美。」帝悟,乃止。時皇太子承乾病蹇,泰以計傾之,乃引駙馬都尉柴令武、房遺愛等布腹心,而韋挺、杜楚客相繼攝府事。二人者,為泰要結中朝臣,津介賂遺,群臣更附為朋黨。承乾懼,陰遣人稱泰府典簽詣玄武門上封,帝省之,書言泰罪,帝怒,即遣捕詰,不獲。既而太子敗,帝陰許立泰,岑文本、劉洎請遂立泰為太子。長孫無忌固欲立晉王,帝以太原石文有「治萬吉」,復欲從無忌。泰微知之,因語晉王:「爾善元昌,得無及乎?」王憂甚,帝怪之,以故對,帝憮然悟。會召承乾譴勒,承乾曰:「臣貴為太子,尚何求?但為泰所圖,與朝臣謀自安爾。無狀之人,遂教臣為不軌事。若泰為太子,正使其得計耳。」帝曰:「是也,有如立泰,則副君可詭求而得。使泰也立,承乾、治俱死;治也立,泰、承乾可無它。」即幽泰將作監,解雍州牧、相州都督、左武候大將軍,降王東萊。因詔:「自今太子不道、籓王窺望者,兩棄之,著為令。」然帝猶謂無忌曰:「公勸我立雉奴,雉奴仁懦,得無為宗社憂,奈何?」雉奴,高宗小字。泰尋改王順陽,居均州之鄖鄉。帝嘗持泰表語左右曰:「泰文辭可喜,豈非才士?我心念泰無已時,但為社稷計,遣居外,使兩相完也。」二十一年進王濮。高宗即位,詔泰開府置僚屬,車服羞膳異等。薨鄖鄉,年三十五,贈太尉、雍州牧。二子:欣、徽。欣嗣王,武后時為酷吏所陷,貶昭州別駕,薨。子嶠,神龍初得嗣王。開元中為國子祭酒,以罪貶鄧州別駕,薨。徽封新安郡王。



 庶人祐字贊。武德八年,王宜陽,進王楚,又王燕,已乃封齊,領齊州都督。貞觀十一年始歸國。以明年入朝,以疾留京師。其舅尚乘直長陰弘智,憸人也,說祐曰:「王兄弟多,即上萬歲後,何以自全?要須得士自助。」乃引客燕弘亮謁祐,祐悅,賜金帛,使募劍客。十五年還州。初,帝用王府長史、司馬,必取骨鯁敢言者,有過失輒聞。而祐溺群小,好弋獵,長史薛大鼎屢諫不聽,帝以輔王無狀,免之,更用權萬紀。萬紀性剛急,以法繩祐。有昝君謨、梁猛虎者,騎射得幸,萬紀斥之,祐私引與狎暱。帝數以書讓祐,萬紀恐並獲罪,即說祐曰:「王,上愛子,上欲王改悔,故數教責王。誠能飭躬引咎,萬紀請入朝言之,上意宜解。」祐因上書謝罪。萬紀見帝,言祐且自新,帝悅,厚賜萬紀,而仍譙戒祐。祐聞萬紀見勞,而己蒙責,以為賣己,益不平。會萬紀又以疑貳系君謨等,制祐不出國門,悉暴祐罪於朝,祐不勝忿。有詔刑部尚書劉德威臨訊,頗實,帝召祐、萬紀還京師。祐與燕弘亮等謀,射殺萬紀,支解之。左右勸祐遂發兵,乃募城中男子年十五以上悉發,私署左右上柱國,光祿大夫,開府儀同三司,托東、托西等王,斥庫貲行賞,驅人築堞浚隍,繕甲兵。人惡之,皆夜縋亡去。詔兵部尚書李勣與劉德威發便道兵討之。祐日夜引弘亮等五人對其妃宴樂。語官軍,則弘亮妄言:「王毋憂,右手持酒啗,左手刀拂之。」祐信愛弘亮,聞之喜。帝手敕祐曰:「吾常戒汝勿近小人,正為此耳。往吾子,今國仇,我上慚皇天,下愧後土。」題畢,涕而遣。祐檄諸縣,縣輒以聞。祐窮蹙,上表曰:「臣,帝子也,為萬紀讒構,上天降靈,罪人斯得。臣狂失心,惝怳驚悸,左右無兵,即欲顛走,所以頗仗械以自衛護。」時勣未至,而青、淄等州兵已集。或勸祐虜子女走豆子為盜,計未決,兵曹杜行敏夜勒兵鑿垣入,祐與弘亮等閉門拒,至日中,行敏呼曰:「吾為國討賊,不速降,且焚。」士積薪,祐乃出,執送京師。賜死內侍省,貶為庶人,葬以國公禮。詔齊州給復一年,擢行敏巴州刺史,封南陽郡公。祐喜養鬥鴨,方未反,貍齚鴨四十餘,絕其頭去。及敗,牽連誅死者凡四十餘人。祐之亂,州人羅石頭數祐罪,以刀直前刺祐,不克,殺之。詔贈亳州刺史。祐嘗引騎徇邑聚,野人高君狀曰;「上親平寇難,土地甲兵不勝計。今王以數千人為亂,猶一手搖泰山,又如君父何?」祐擊禽之,愧其言,不能殺。詔擢榆社令。



 蜀悼王愔,貞觀五年始王梁,與郯、漢、申、江、代五王同封。徙王蜀,實封八百戶。出為岐州刺史。數畋游,為非法,帝頻責教,不悛,怒曰:「禽獸可擾於人,鐵石可為器,愔曾不如之!」乃削封戶及國官半,徙虢州。久之,還戶,增至千。復出馳弋,敗民稼。典軍楊道整叩馬諫,愔捽擊之。御史大夫李乾祐劾愔罪,高宗怒,貶黃州刺史。擢道整匡道府折沖都尉。吳王恪得罪,愔以母弟廢為庶人,徙巴州。俄封涪陵王,薨。咸亨初,復爵士,贈益州大都督,陪葬昭陵,以子璠嗣王。璠,武后時謫死歸誠州。神龍初,以朗陵王瑋子示俞嗣。



 蔣王惲,始王郯,又徙王蔣,拜安州都督,賜實封千戶。永徽三年,徙梁州。惲造器物服玩,多至四百車,所經州縣騷然護送,為有司劾奏,詔貸不問。上元中,遷箕州刺史。錄事參軍張君徹誣告惲反,詔使者按驗,惲惶懼自殺。高宗知其枉,斬君徹,贈惲司空、荊州大都督,陪葬昭陵。三子:煒、煌、休道。煒初王汝南郡,惲薨,遂嗣王,為武后所害。神龍初,以嫡孫紹宗為嗣蔣王,薨,子欽福嗣,為率更令。煌封蔡國公。孫之芳,有令譽,安祿山奏為範陽司馬。祿山反,自拔歸京師。歷工部侍郎、太子右庶子。廣德初,詔兼御史大夫使吐蕃,被留二歲乃得歸。拜禮部尚書,改太子賓客。休道子琚,神龍初封嗣趙王,開元中改王中山。



 越王貞,始王漢,後徙原,已乃封越。貞善騎射,涉文史,有吏乾,為宗室材王。武后初,遷累太子太傅、豫州刺史。中宗廢居房陵,貞乃與韓王元嘉及王子黃公譔,魯王靈夔、王子範陽王藹,霍王元軌、王子江都王緒,及子瑯邪王沖計議反正。垂拱四年,明堂成,悉追宗室行享禮,共疑後遂大誅戮不遺種,事且急,譔乃矯帝璽書賜沖曰:「朕幽縶,諸王宜即起兵。」於是命長史蕭德琮募兵,告諸王師期。八月,沖先發,諸王莫有應者,獨貞將兵攻上蔡,破之,而沖已敗。貞稍徇屬縣,得士七千,列五營:貞為中營,以裴守德為大將軍,領中營;趙成美為左中郎將,領左營;閭弘道為右中郎將,領右營;安摩訶為郎將,領後軍;王孝志為右將軍,領前軍。以韋慶禮為司馬,署官五百。然肋誘無鬥志,家童皆佩符以闢兵。九月,後遣左豹韜衛大將軍曲崇裕、夏官尚書岑長倩率兵十萬討之,以鳳閣侍郎張光輔為諸軍節度,乃下詔削貞父子屬籍,改氏「虺」。崇裕等次豫州,貞少子規及裴守德拒戰,兵潰,貞乃閉門守。守德者,驍勇士。貞始起,以女妻之,委以腹心。至是,欲殺貞自贖。會軍薄城,家人白貞:「今事乃爾,王豈受戮辱者邪?」即仰藥死。規自殺,守德與主俱縊。起凡二十日敗。始,貞臨水自鑒,不見其首,惡之,未幾及禍。沖,貞長子也。好學,勇而才,累遷博州刺史。初發,有士五千,度河趣武水,武水令告急魏州,州遣莘令馬玄素領兵先乘城,沖攻之,因風,積薪焚其門,火作風反,眾心沮解,其屬董元寂誦言:「王與國家戰,乃反爾。」沖斬以徇,眾懼,遂潰,唯家僮數十從之,乃走博州,為當關刺死。後命丘神勣討之,兵未至,沖已死,起七日敗。二弟:茜、溫。茜,常山公,坐死。溫以前告,流嶺南。初,貞騰檄壽州刺史趙瑰,諭以興兵且假道。瑰得檄,許為應,瑰妻常樂長公主亦趣諸王蚤立功,故瑰與主皆死。濟州刺史薛顗與其弟紹謀應沖,率所部庸、調,治兵募士,沖敗,下獄死。顗,駙馬都尉瓘之子,母城陽長公主,封河東縣侯。紹尚太平公主,擢累右玉鈐衛員外將軍,以主婿不加戮,餓死河南獄。神龍初,敬暉等奏沖父子死社稷,請復爵土,為武三思等沮罷。開元四年,乃復爵土,有司謚死不忘君曰敬。五年,又詔:「王嗣絕國除,朕甚悼焉。其以貞從孫故許王子夔國公琳嗣王,奉王祀。」琳薨,爵不傳。貞最幼息珍子謫嶺表,數世不能歸。開成中,女孫持四世柩北還,求祔王塋。詔嘉憫,敕宗正寺、京兆府為訪其兆,非陪陵者聽葬。女名元真,為道士。



 紀王慎,始王申,後徙紀,食戶八百。貞觀中,遷襄州刺史,以治當最,天子璽書勞勉,人為立石頌德。二十三年,進戶至千。文明初,累遷太子太師、貝州刺史。慎少好學,善星步,與越王齊名,當世號「紀越」。初,貞連諸王起兵,慎知時未可,獨拒不與合。將就誅而免,改氏「虺」,載以檻車,謫巴州,薨於道。七子:續、琮、睿、秀、獻、欽、證。續與秀最知名。續王東平,歷和州刺史,薨。琮義陽王,睿楚國公,秀襄陽郡公,獻廣化郡公,欽建平郡公,五人並為武后所殺。神龍初,以證嗣王,擢左驍衛將軍,薨。子行同嗣。琮三子:行遠、行芳、行休。始,琮與二弟同死桂林。開元四年,行休請身迎柩,既至,無封樹,議者謂不可復得。行休歸,地布席以祈。是夜夢王乘舟,舟判為二。既而適野,見東洲中斷,乃悟焉。又靈堂鎖一夕莖自屈,管上有指跡,一奇二並。使卜人筮之,曰:「屈,於文為尸出;指者,示也;一奇二並,三殯也。先王告之矣。」乃趣其所,發之如言,而一節獨闕。行休號而寢,夢琮告曰:「在洛南洲。」明日,直殯南得之。於是以三喪歸,陪葬昭陵,贈琮陳州刺史。永昌時,行遠、行芳斥巂州,六道使至,行遠先就戮,行芳幼當赦,抱持請代,遂與俱死,西南人稱死悌云。慎女東光縣主,始八歲,聞慎有疾,不食,父哀之,紿云已愈,主察顏色未平,終不肯御,內外稱之。長適太子司議郎裴仲將。時妃、主多恃貴,以奢豫相矜,主獨儉素,姊弟誚曰:「人生富貴在得志,獨勤苦,欲何求?」答曰:「我幼好禮,今行之不違,非得志謂何?且自古賢妃淑女以恭遜著名,驕縱敗德,況榮寵貴盛,儻來物也,可恃以凌人乎?」及王死,號慟,嘔血數升。免喪,絕膏沐者二十年。始,諸王、妃、主自垂拱後被害者皆槁掩之。神龍初,詔州縣普加求訪,祭以牲牢,復官爵,諸王皆陪葬昭、獻二陵。主聞,感慟,卒,敕其子曰:「為我謝親戚,酷憤已雪,下見先王無恨矣!」中宗為舉哀章善門,下詔褒揚。



 江殤王囂,封之明年薨,無後。



 代王簡,已封薨,無後。



 趙王福,貞觀十三年始王,出後隱太子。累遷梁州都督,實封八百戶。薨,贈司空、並州都督,陪葬昭陵。無子,神龍初,以蔣王惲孫思順嗣王。



 曹王明,母本巢王妃,帝寵之,欲立為後,魏徵諫曰:「陛下不可以辰贏自累。」乃止。貞觀二十一年,始王曹,累為都督、刺史。高宗詔出後巢王。永隆中,坐太子賢事,降王零陵,徙黔州。都督謝祐逼殺之,帝聞,悼甚,黔官吏皆坐免。景雲中,陪葬昭陵。三子:俊、傑、備。俊嗣王,南州別駕,傑為黎國公,垂拱時並及誅。神龍初,以傑子胤為嗣曹王。是時,諸王子孫自嶺外還,入見中宗,皆號慟,帝為泣下。初,武后時,壯者誅死,幼皆沒為官奴,或匿人間庸保。至是,相繼出,帝隨屬遠近封拜云。後備自南還,詔停胤封而封備,歷衛尉少卿同正員,薨。開元十二年,復封胤。薨,子戢嗣,位左衛率府中郎將。子皋嗣。



 皋字子蘭,少補左司禦兵曹參軍。天寶十一載嗣王。事母太妃鄭以孝聞。安祿山反,奉母逃民間,間走蜀,謁玄宗,由都水使者遷左領軍將軍。上元初旱歉,皋祿不足養,請補外,不許,乃故抵輕法,貶溫州長史,俄攝州事。州大饑,發官廩數十萬石賑餓者,僚史叩庭請先以聞,皋曰:「人日不再食且死,可俟命後發哉?茍殺我而活眾,其利大矣!」既貸,乃自劾,優詔開許,就進少府監。時殿中侍御史李鈞與其弟京兆法曹參軍鍔宦既遂,不肯還鄉,母窮不自給。皋行縣見之,嘆曰:「入則孝,出則悌,有餘力則學。若二子者可與事君乎哉?」舉劾之,並錮死。召還,未得見,即上書言治道,詔授衡州刺史,為觀察使謾劾,貶潮州。會楊炎起道州為宰相,知皋直,復用為衡州刺史。初,御史覆訊,皋懼憂其母,出則囚服,入乃衣冠,貌言如平常。及為潮,以遷入告。至是復位,乃言其實。建中元年,進拜湖南觀察使。前帥辛京杲貪虐,使部將王國良戍武岡,賴其富,即劾以死,國良恐,據縣反,斂荊、黔、洪、桂兵討之,再歲不能下。皋至,遺書曰:「觀將軍非敢大逆者,特逃讒抗死爾!將軍遇我,可以降,我固為京杲誣者,幸蒙雪,何忍以兵加將軍哉?以為不然,我以陣術破將軍陣,以攻法屠將軍城,非將軍所度也。」國良得書,喜且畏,因請降,然內尚首鼠。皋即日單騎稱使者造國良壘,賊延使者入,皋大呼其軍曰:「有識曹王者乎?乃我也。來受良降,良今安在?」一軍愕眙,不敢動,國良迎拜,叩頭請罪。皋執手,約為昆弟,則盡焚攻守具,散其兵。有詔赦之,賜名惟新。



 明年,持母喪至江陵。會梁崇義反,奪為左衛大將軍,復觀察湖南。李希烈反,遷江西節度使。受命日,不宿家,至豫章,大令將吏曰:「有功未申與懷器謀不發者,皆自言。」得裨校伊慎、李伯潛、劉旻,悉補大將。擢王鍔為中軍,以馬彞、許孟容為幕府。治戰艦,裒兵二萬,以士二千五百委慎等教之。自將五百人,教以秦兵團力法,聯其賞罰,弛張如一,乃約以五百人擊慎卒二千五百,莫能當其鋒,即盡以教之。初,慎嘗從希烈平襄州,至是,希烈懼為皋用,即反間,德宗信之,將誅慎,皋請赦之,使自效。會與賊夾江陣,皋勉慎立功,以所乘馬及其鎧賜之,使將先鋒,斬賊數百級,乃免。



 賊柵蔡山不可攻,皋聲言西取蘄,引兵艦循崖溯江上。賊聞,以羸師保柵,悉軍行江北,與皋直。西去蔡山三百里,皋遣步士悉登舟,順流下,攻蔡山,拔之。間一日,賊救至,遂大敗,乃取蘄州,降其將李良,平黃州,兵益振。



 會舒王為元帥,授皋前軍兵馬使。俄而天子狩奉天,鹽鐵使包佶為陳少游所窘,以運艚溯江,次蘄口,希烈使杜少誠將步騎三萬將絕江道,皋遣伊慎兵七千御於永安,走之。以功進工部尚書。帝駐梁州,皋之貢助相望。以天子處外,乃不敢居城府,出屯西塞山大洲,徙郡縣為軍市。改戶部尚書。又遣伊慎、王鍔攻安州,未下,希烈遣劉戒虛以步騎八千援之,皋命李伯潛迎擊於應山,俘之,遂下安州,斬偽刺史王嘉祥。希烈別遣兵援隋州,皋破之厲鄉,因下平靜、白雁關,賊遂不敢南略。遷荊南節度使,賜實封三百戶。凡戰大小三十二,取州五、縣二十,斬首三萬三千,禽生萬六千,未嘗敗。師所過,不敢伐桑棗、踐禾稼。朝廷仰食江淮,而西道出九江,至大別,皆與賊接,皋轉戰數千里,餉路遂通,江漢倚皋為固。淮西平,乃請護喪歸東都,帝走中人賵吊。訖葬來朝,還就鎮。初,江陵東北傍漢有古鄣,不治,歲輒溢。皋修塞之,得其下良田五千頃。規江南廢洲為廬舍,構二橋跨江,而流人自占者二千餘家。繇荊抵樂鄉二百里,其間墟聚凡數十,不井飲,皋始命鑿井以便人。貞元初,吳少誠擅蔡,故徙皋鎮山南東道,割隋、汝以益軍,練兵峙糧,市回鶻馬以益戰騎,歲時大畋以教士,少誠畏之。皋性勤儉,能知人疾苦。參聽微隱,盡得吏下短長,其賞罰必信。所至常平物估,豪舉不得擅其利。教為戰艦,挾二輪蹈之,鼓水疾進,駛於陣馬。有所造作,皆用省而利長。以物遺人,必自視衡量,庫帛皆印署,以杜吏謾。扶鳳馬彞未知名,皋識之,卒以正直稱。張柬之有園圃在襄陽,皋嘗宴集,將市取之。彞曰:「漢陽有中興功,今遺業當百世共保,奈何使其子孫鬻乎?」皋謝曰:「主吏失詞,以為君羞,微君安得聞此言?」卒年六十,贈尚書右僕射,謚曰成。皋嘗自創意為欹器,以颻木上出五觚,下銳圓,為盂形,所容二豆,少則水弱,多則強,中則水器力均,雖動搖,乃不覆雲。



 子象古、道古。



 象古,元和中,自衡州刺史擢安南都護,貪縱不法。驩州刺史楊清者,蠻酋也,象古忌其豪,召為牙門將,常鬱鬱思亂。會討黃賊,象古發甲助之,乃授清兵三千。清與子志烈還襲安南,殺象古並其家。詔赦清為瓊州刺史,以桂仲武為都護。清拒命,仲武分諭渠酋,兵皆附,破城,斬清,夷其族。



 道古,舉進士,獻書闕下,擢校書郎、集賢院學士。累遷司門員外郎,歷利、隋、唐、睦四州刺史。柳公綽鎮鄂岳,為飛譖上聞,憲宗欲代之。裴度言:「嗣曹王皋嘗能以江漢兵制李希烈,威惠在人,今以其子將,必有功。」會道古自黔中觀察使入朝,乃代公綽,倍道入其軍,公綽惶遽出,財貲皆被奪。元和十二年,攻申州,破其郛,進圍中城。守卒夜驅女子登而噪,發懸門以出,道古眾亂,多死於賊。李聽守安州,未嘗敗,道古誣逐之。自將出穆陵關,士卒驕,不能制,又度支錢道古悉以饋權幸,故賜不給,其下怨怒,戰不甚力,賊亦易之。故再入申,不能下,卒無功。淮西平,加檢校御史大夫,召為宗正卿、左金吾將軍。帝喜服餌,道古欲自媚,而所善柳泌自謂能化金為不死藥,乃因宰相皇甫鎛以聞,俄會帝崩。穆宗為太子,惡之,既立,誅泌,貶鎛,斥道古為循州司馬。終以服丹歐血死。長慶初,詔還其官。道古巧於宦,便佞傾下,游公卿間,常與奕博,偽不勝,厚進所償,嗜利者多得其歡心,故少盜美名。及死,賣宅以葬。



\end{pinyinscope}