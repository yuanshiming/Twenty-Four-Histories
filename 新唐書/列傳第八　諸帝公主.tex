\article{列傳第八 諸帝公主}

\begin{pinyinscope}

 世祖一女。



 同安公主,高祖同母媦也。下嫁隋州刺史王裕。貞觀時,以屬尊進大長公主。嘗有疾,太宗躬省視,賜縑五百,姆侍皆有賚予。永徽初,賜實戶三百。薨年八十六。裕,隋司徒柬之子,終開府儀同三司。



 高祖十九女。



 長沙公主,下嫁馮少師。



 襄陽公主,下嫁竇誕。



 平陽昭公主,太穆皇后所生,下嫁柴紹。初,高祖兵興,主居長安,紹曰:「尊公將以兵清京師,我欲往,恐不能偕,奈何?」主曰:「公行矣,我自為計。」紹詭道走並州,主奔鄠,發家貲招南山亡命,得數百人以應帝。於是,名賊何潘仁壁司竹園,殺行人,稱總管,主遣家奴馬三寶喻降之,共攻鄠阜。別部賊李仲文、向善志、丘師利等各持所領會戲下,因略地盩厔、武功、始平,下之。乃申法誓眾,禁剽奪,遠近咸附,勒兵七萬,威振關中。帝度河,紹以數百騎並南山來迎,主引精兵萬人與秦王會渭北。紹及主對置幕府,分定京師,號「娘子軍」。帝即位,以功給賚不涯。武德六年薨,葬加前後部羽葆、鼓吹、大路、麾幢、虎賁、甲卒、班劍。太常議:「婦人葬,古無鼓吹。」帝不從,曰:「鼓吹,軍樂也。往者主身執金鼓,參佐命,於古有邪?宜用之。」



 高密公主,下嫁長孫孝政,又嫁段綸。綸,隋兵部尚書文振子,為工部尚書、杞國公。永徽六年主薨,遺命:「吾葬必令墓東向,以望獻陵,冀不忘孝也。」



 長廣公主,始封桂陽。下嫁趙慈景。慈景,隴西人,帝美其姿制,故妻之。帝起兵,或勸亡去,對曰:「母以我為命,且安往?」吏捕系於獄。帝平京師,引拜開化郡公,為相國府文學。進兵部侍郎。為華州刺史。討堯君素戰死、贈秦州刺史,謚曰忠。公主更嫁楊師道。聰悟有思,工為詩,豪侈自肆,晚稍折節,以壽薨。



 長沙公主,始封萬春。下嫁豆盧寬子懷讓。



 房陵公主,始封永嘉。下嫁竇奉節,又嫁賀蘭僧伽。



 九江公主,下嫁執失思力。



 廬陵公主,下嫁喬師望,為同州刺史。



 南昌公主,下嫁蘇勖。



 安平公主,下嫁楊思敬。



 淮南公主,下嫁封道言。



 真定公主,下嫁崔恭禮。



 衡陽公主,下嫁阿史那社爾。



 丹陽公主,下嫁薛萬徹。萬徹蠢甚,公主羞,不與同席者數月。太宗聞,笑焉,為置酒,悉召它婿與萬徹從容語,握槊賭所佩刀,陽不勝,遂解賜之。主喜,命同載以歸。



 臨海公主,下嫁裴律師。



 館陶公主,下嫁崔宣慶。



 安定公主,始封千金。下嫁溫挺。挺死,又嫁鄭敬玄。



 常樂公主,下嫁趙瑰。生女,為周王妃,武後殺之。逐瑰括州刺史,徙壽州。越王貞將舉兵,遺瑰書假道,瑰將應之。主進使者曰:「為我謝王,與其進,不與其退,。若諸王皆丈夫,不應掩久至是。我聞楊氏篡周,尉遲迥乃周出,猶能連突厥,使天下響震,況諸王國懿親,宗社所托,不舍生取義,尚何須邪?人臣同國患為忠,不同為逆,王等勉之。」王敗,周興劾瑰與主連謀,皆被殺。



 太宗二十一女。



 襄城公主,下嫁蕭銳。性孝睦,動循矩法,帝敕諸公主視為師式。有司告營別第,辭曰:「婦事舅姑如父母,異宮則定省闕。」止葺故第,門列雙戟而已。銳卒,更嫁姜簡。永徽二年薨,高宗舉哀於命婦朝堂,遣工部侍郎丘行淹馳驛吊祭,陪葬昭陵。喪次故城,帝登樓望哭以送柩。



 汝南公主,蚤薨。



 南平公主,下嫁王敬直,以累斥嶺南,更嫁劉玄意。



 遂安公主,下嫁竇逵。逵死,又嫁王大禮。



 長樂公主,下嫁長孫沖。帝以長孫皇后所生,故敕有司裝齎視長公主而倍之。魏徵曰:「昔漢明帝封諸王曰:『朕子安得同先帝子乎?』然則長公主者,尊公主矣。制有等差,渠可越也?」帝以語後,後曰:「嘗聞陛下厚禮徵而未知也,今聞其言,乃納主於義,社稷臣也。妾於陛下,夫婦之重,有所言,猶候顏色,況臣下情隔禮殊,而敢犯嚴顏陳忠言哉!願許之,與天下為公。」帝大悅,因請齎帛四十匹、錢四十萬即徵家賜之。



 豫章公主,下嫁唐義識。



 北景公主,始封巴陵。下嫁柴令武,坐與房遺愛謀反,同主賜死。顯慶中追贈,立廟於墓,四時祭以少牢。



 普安公主,下嫁史仁表。



 東陽公主,下嫁高履行。高宗即位,進為大長公主。韋正矩之誅,主坐婚家,斥徙集州。又坐章懷太子累,奪邑封。以長孫無忌舅族也,故武后惡之,垂拱中,並二子徙置巫州。



 臨川公主,韋貴妃所生。下嫁周道務。主工籀隸,能屬文。高宗立,上《孝德頌》,帝下詔褒答。永徽初,進長公主,恩賞卓異。永淳初薨。道務,殿中大監、譙郡公範之子。初,道務孺褓時,以功臣子養宮中。範卒,還第,毀瘠如成人。復內之,年十四乃得出。歷營州都督,檢校右驍衛將軍。謚曰襄。



 清河公主名敬,字德賢,下嫁程懷亮,薨麟德時,陪葬昭陵。懷亮,知節子也,終寧遠將軍。



 蘭陵公主名淑,字麗貞,下嫁竇懷悊,薨顯慶時。懷悊官兗州都督,太穆皇后之族子。



 晉安公主,下嫁韋思安,又嫁楊仁輅。



 安康公主,下嫁獨孤謀。



 新興公主,下嫁長孫曦。



 城陽公主,下嫁杜荷,坐太子承乾事誅,又嫁薛瓘。初,主之婚,帝使卜之,繇曰:「二火皆食,始同榮,末同戚,請晝昏則吉。」馬周諫曰:「朝謁以朝,思相戒也;講習以晝,思相成也;燕飲以昃,思相歡也;婚合以夜,思相親也。故上下有成,內外有親,動息有時,吉兇有儀。今先亂其始,不可為也。夫卜所以決疑,若黷禮慢先,聖人所不用。」帝乃止。麟德初,瓘歷左奉宸衛將軍。主坐巫蠱,斥瓘房州刺史,主從之官。咸亨中,主薨而瓘卒,雙柩還京師。子顗,封河東縣侯、濟州刺史。瑯邪王沖起兵,顗與弟紹以所部庸、調作兵募士,且應之。沖敗,殺都吏以滅口。事洩,下獄俱死。



 合浦公主,始封高陽。下嫁房玄齡子遺愛。主,帝所愛,故禮異它婿。主負所愛而驕。房遺直以嫡當拜銀青光祿大夫,讓弟遺愛,帝不許。玄齡卒,主導遺愛異貲,既而反譖之,遺直自言,帝痛讓主,乃免。自是稍疏外,主怏怏。會御史劾盜,得浮屠辯機金寶神枕,自言主所賜。初,浮屠廬主之封地,會主與遺愛獵,見而悅之,具帳其廬,與之亂,更以二女子從遺愛,私餉億計。至是,浮屠殊死,殺奴婢十餘。主益望,帝崩無哀容。又浮屠智勖迎占禍福,惠弘能視鬼,道士李晃高醫,皆私侍主。主使掖廷令陳玄運伺宮省禨祥,步星次。永徽中,與遺愛謀反,賜死。顯慶時追贈。



 金山公主,蚤薨。



 晉陽公主字明達,幼字兕子,文德皇后所生。未嘗見喜慍色。帝有所怒責,必伺顏徐徐辯解,故省中多蒙其惠,莫不譽愛。後崩,時主始孩,不之識;及五歲,經後所游地,哀不自勝。帝諸子,唯晉王及主最少,故親畜之。王每出閤,主送至虔化門;泣而別。王勝衣,班於朝,主泣曰:「兄今與群臣同列,不得在內乎?」帝亦為流涕。主臨帝飛白書,下不能辨。薨年十二。帝閱三旬不常膳,日數十哀,因以臒羸。群臣進勉,帝曰:「朕渠不知悲愛無益?而不能已,我亦不知其所以然。」因詔有司簿主湯沐餘貲,營佛祠墓側。



 常山公主,未及下嫁,薨顯慶時。



 新城公主,晉陽母弟也。下嫁長孫詮,詮以罪徙巂州。更嫁韋正矩,為奉冕大夫,遇主不以禮。俄而主暴薨,高宗詔三司雜治,正矩不能辯,伏誅。以皇后禮葬昭陵旁。



 高宗三女。



 義陽公主,蕭淑妃所生,下嫁權毅。



 高安公主,義陽母弟也。始封宣城。下嫁潁州刺史王勖。天授中,勖為武后所誅。神龍初,進冊長公主,實封千戶,開府置官屬。睿宗立,增戶千。薨開元時,玄宗哭於暉政門,遣大鴻臚持節赴吊,京兆尹攝鴻臚護喪事。



 太平公主,則天皇后所生,後愛之傾諸女。榮國夫人死,後丐主為道士,以幸冥福。儀鳳中,吐蕃請主下嫁,後不欲棄之夷,乃真築宮,如方士薰戒,以拒和親事。久之,主衣紫袍玉帶,折上巾,具紛礪,歌舞帝前。帝及後大笑曰:「兒不為武官,何遽爾?」主曰:「以賜駙馬可乎?」帝識其意,擇薛紹尚之。假萬年縣為婚館,門隘不能容翟車,有司毀垣以入,自興安門設燎相屬,道樾為枯。紹死,更嫁武承嗣,會承嗣小疾,罷昏。後殺武攸暨妻,以配主。主方額廣頤,多陰謀,後常謂「類我」。而主內與謀,外檢畏,終後世無它訾。



 永淳之前,親王食實戶八百,增至千輒止;公主不過三百,而主獨加戶五十。及聖歷時,進及三千戶。預誅二張功,增號鎮國,與相王均封五千,而薛、武二家女皆食實封。主與相王衛王成王、長寧安樂二公主給衛士,環第十步一區,持兵呵衛,僭肖宮省。神龍時,與長寧、安樂、宜城、新都、定安、金城凡七公主,皆開府置官屬,視親王。安樂戶至三千,長寧二千五百,府不置長史。宜城、定安非韋後所生,戶止二千。主三子:崇簡、崇敏、崇行,皆拜三品。



 韋后、上官昭容用事,自以謀出主下遠甚,憚之。主亦自以軋而可勝,故益橫。於是推進天下士,謂儒者多窶狹,厚持金帛謝之,以動大議,遠近翕然響之。



 玄宗將誅韋氏,主與秘計,遣子崇簡從。事定,將立相王,未有以發其端者。主顧溫王乃兒子,可劫以為功,乃入見王曰:「天下事歸相王,此非兒所坐。」乃掖王下,取乘輿服進睿宗。睿宗即位,主權由此震天下,加實封至萬戶,三子封王,餘皆祭酒、九卿。主每奏事,漏數徙乃得退,所言皆從。有所論薦,或自寒冗躐進至侍從,旋踵將相。朝廷大政事非關決不下,聞不朝,則宰相就第咨判,天子殆畫可而已。主侍武后久,善策人主微指,先事逢合,無不中。田園遍近甸,皆上腴。吳、蜀、嶺嶠市作器用,州縣護送,道相望也。天下珍滋譎怪充於家,供帳聲伎與天子等。侍兒曳紈穀者數百,奴伯嫗監千人,隴右牧馬至萬匹。



 長安浮屠慧範畜貲千萬,諧結權近,本善張易之。及易之誅,或言其豫謀者,於是封上庸郡公,月給奉稍。主乳媼與通,奏擢三品御史大夫。御史魏傳弓劾其奸贓四十萬,請論死。中宗欲赦之,進曰:「刑賞,國大事,陛下賞已妄加矣,又欲廢刑,天下其謂何?」帝不得已,削銀青階。大夫薛謙光劾慧範不法,不可貸,主為申理,故謙光等反得罪。



 玄宗以太子監國,使宋王、岐王總禁兵。主恚權分,乘輦至光範門,召宰相白廢太子。於是宋璟、姚元之不悅,請出主東都,帝不許,詔主居蒲州。主大望,太子懼,奏斥璟、元之以銷戢怨嫌。監察御史慕容珣復劾慧範事,帝疑珣離間骨肉,貶密州司馬。主居外四月,太子表追還京師。



 時宰相七人,五出主門下。又左羽林大將軍常元楷、知羽林軍李慈皆私謁主。主內忌太子明,又宰相皆其黨,乃有逆謀。先天二年,與尚書左僕射竇懷貞、侍中岑羲、中書令蕭至忠崔湜、太子少保薛稷、雍州長史李晉、右散騎常侍昭文館學士賈膺福、鴻臚卿唐晙及元楷、慈、慧範等謀廢太子,使元楷、慈舉羽林兵入武德殿殺太子,懷貞、羲、至忠舉兵南衙為應。既有日矣,太子得其奸,召岐王、薛王、兵部尚書郭元振、將軍王毛仲、殿中少監姜晈、中書侍郎王琚、吏部侍郎崔日用定策。前一日,因毛仲取內閑馬三百,率太僕少卿李令問王守一、內侍高力士、果毅李守德叩虔化門,梟元楷、慈於北闕下,縛膺福內客省,執羲、至忠至朝堂,斬之,因大赦天下。主聞變,亡入南山,三日不出,賜死於第。諸子及黨與死者數十人。簿其田貲,瑰寶若山,督子貸,凡三年不能盡。崇簡素知主謀,苦諫,主怒,榜掠尤楚,至是復官爵,賜氏李。始,主作觀池樂游原,以為盛集,既敗,賜寧、申、岐、薛四王,都人歲祓禊其地。



 中宗八女。



 新都公主,下嫁武延暉。



 宜城公主,始封義安郡主。下嫁裴巽。巽有嬖姝,主恚,刖耳劓鼻,且斷巽發。帝怒,斥為縣主,巽左遷。久之,復故封。神龍元年,與長寧、新寧、義安、安樂、新平五郡主皆進封。



 定安公主,始封新寧郡。下嫁王同皎。同皎得罪,神龍時,又嫁韋濯。濯即韋皇后從祖弟,以衛尉少卿誅,更嫁太府卿崔銑。主薨,王同皎子請與父合葬,給事中夏侯銛曰:「主義絕王廟,恩成崔室,逝者有知,同皎將拒諸泉。」銑或訴於帝,乃止。銛坐是貶瀘州都督。



 長寧公主,韋庶人所生,下嫁楊慎交。造第東都,使楊務廉營總。第成,府財幾竭,乃擢務廉將作大匠。又取西京高士廉第、左金吾衛故營合為宅,右屬都城,左頫大道,作三重樓以馮觀,築山浚池。帝及後數臨幸,置酒賦詩。又並坊西隙地廣鞠場。東都廢永昌縣,主丐其治為府,以地瀕洛,築鄣之,崇臺、蜚觀相聯屬。無慮費二十萬。魏王泰故第,東西盡一坊,瀦沼三百畝,泰薨,以與民。至是,主丐得之,亭閣華詭捋西京。內倚母愛,寵傾一朝,與安樂宜城二主、後胃郕國崇國夫人爭任事,賕謁紛紜。東都第成,不及居,韋氏敗,斥慎交絳州別駕,主偕往,乃請以東都第為景雲祠,而西京鬻第,評木石直,為錢二十億萬。開元十六年,慎交死,主更嫁蘇彥伯。務廉卒坐贓數十萬,廢終身。



 永壽公主,下嫁韋金歲。蚤薨,長安初追贈。



 永泰公主,以郡主下嫁武延基。大足中,忤張易之,為武后所殺。帝追贈,以禮改葬,號墓為陵。



 安樂公主,最幼女。帝遷房陵而主生,解衣以褓之,名曰裹兒。姝秀辯敏,後尤愛之。下嫁武崇訓。帝復位,光艷動天下,侯王柄臣多出其門。嘗作詔,箝其前,請帝署可,帝笑從之。又請為皇太女,左僕射魏元忠諫不可,主曰:「元忠,山東木強,烏足論國事?阿武子尚為天子,天子女有不可乎?」與太平等七公主皆開府,而主府官屬尤濫,皆出屠販,納訾售官,降墨敕斜封授之,故號「斜封官」。主營第及安樂佛廬,皆憲寫宮省,而工緻過之。嘗請昆明池為私沼,帝曰:「先帝未有以與人者。」主不悅,自鑿定昆池,延袤數里。定,言可抗訂之也。司農卿趙履溫為繕治,累石肖華山,隥彴橫邪,回淵九折,以石瀵水。又為寶爐,鏤怪獸神禽,間以璖貝珊瑚,不可涯計。崇訓死,主素與武延秀亂,即嫁之。是日,假後車輅,自宮送至第,帝與後為御安福門臨觀,詔雍州長史竇懷貞為禮會使,弘文學士為儐,相王障車,捐賜金帛不貲。翌日,大會群臣太極殿,主被翠服出,向天子再拜,南面拜公卿,公卿皆伏地稽首。武攸暨與太平公主偶舞為帝壽。賜群臣帛數十萬。帝御承天門,大赦,因賜民酺三日,內外官賜勛,緣禮官屬兼階、爵。奪臨川長公主宅以為第,旁徹民廬,怨聲囂然。第成,禁藏空殫,假萬騎仗、內音樂送主還第,天子親幸,宴近臣。崇訓子方數歲,拜太常卿,封鎬國公,實封戶五百。公主滿孺月,帝、後復幸第,大赦天下。時主與長寧、定安三家廝臺掠民子女為奴婢,左臺侍御史袁從一縛送獄,主入訴,帝為手詔喻免。從一曰:「陛下納主訴,縱奴騶掠平民,何以治天下?臣知放奴則免禍,劾奴則得罪於主,然不忍屈陛下法,自偷生也。」不納。臨淄王誅庶人,主方覽鏡作眉,聞亂,走至右延明門,兵及,斬其首。追貶為「悖逆庶人」。睿宗即位,詔以二品禮葬之。趙履溫諂事主,嘗褫朝服,以項挽車。庶人死,蹈舞承天門呼萬歲,臨淄王斬之,父子同刑。百姓疾其興役,割取肉去。



 成安公主,字季姜。始封新平。下嫁韋捷。捷以韋後從子誅,主後薨。



 睿宗十一女。



 壽昌公主,下嫁崔真。



 安興昭懷公主,蚤薨。



 荊山公主,下嫁薛伯陽。



 淮陽公主,下嫁王承慶。



 代國公主名華,字華婉,劉皇后所生。下嫁鄭萬鈞。



 涼國公主字華莊,始封仙源。下嫁薛伯陽。



 薛國公主,始封清陽。下嫁王守一。守一誅,更嫁裴巽。



 鄎國公主,崔貴妃所生。三歲而妃薨,哭泣不食三日,如成人。始封荊山。下嫁薛儆,又嫁鄭孝義。開元初,封邑至千四百戶。



 金仙公主,始封西城縣主。景雲初進封。太極元年,與玉真公主皆為道士,築觀京師,又方士史崇玄為師。崇玄本寒人,事太平公主,得出入禁中,拜鴻臚卿,聲勢光重。觀始興,詔崇玄護作,日萬人。群浮屠疾之,以錢數十萬賂狂人段謙冒入承天門,升太極殿,自稱天子。有司執之,辭曰:「崇玄使我來。」詔流嶺南,且敕浮屠、方士無兩競。太平敗,崇玄伏誅。



 玉真公主字持盈,始封崇昌縣主。俄進號上清玄都大洞三景師。天寶三載,上言曰:「先帝許妾舍家,今仍叨主第,食租賦,誠願去公主號,罷邑司,歸之王府。」玄宗不許。又言:「妾,高宗之孫,睿宗之女,陛下之女弟,於天下不為賤,何必名系主號、資湯沐,然後為貴、?請入數百家之產,延十年之命。」帝知至意,乃許之。薨寶應時。



 霍國公主,下嫁裴虛己。



 玄宗二十九女。



 永穆公主,下嫁王繇。



 常芬公主,下嫁張去奢。



 孝昌公主,蚤薨。



 唐昌公主,下嫁薛銹。



 靈昌公主,蚤薨。



 常山公主,下嫁薛譚,又嫁竇澤。



 萬安公主,天寶時為道士。



 開元新制:長公主封戶二千,帝妹戶千,率以三丁為限;皇子王戶二千,主半之。左右以為薄。帝曰:「百姓租賦非我有,士出萬死,賞不過束帛,女何功而享多戶邪?使知儉嗇,不亦可乎?」於是,公主所稟殆不給車服。後咸宜以母愛益封至千戶,諸主皆增,自是著於令。主不下嫁,亦封千戶,有司給奴婢如令。



 上仙公主,蚤薨。



 懷思公主,蚤薨,葬築臺,號登真。



 晉國公主,始封高都。下嫁崔惠童。貞元元年,與衛、楚、宋、齊、宿、蕭、鄧、紀、郜國九公主同徙封。



 新昌公主,下嫁蕭衡。



 臨晉公主,皇甫淑妃所生。下嫁郭潛曜。薨大歷時。



 衛國公主,始封建平。下嫁豆盧建,又嫁楊說。薨貞元時。



 真陽公主,下嫁源清,又嫁蘇震。



 信成公主,下嫁獨孤明。



 楚國公主,始封壽春。下嫁吳澄江。上皇居西宮,獨主得入侍。興元元年,請為道士,詔可,賜名上善。



 普康公主,蚤薨。咸通九年追封。



 昌樂公主,高才人所生。下嫁竇鍔。薨大歷時。



 永寧公主,下嫁裴齊丘。



 宋國公主,始封平昌。下嫁溫西華,又嫁楊徽。薨元和時。



 齊國公主,始封興信,徙封寧親。下嫁張垍,又嫁裴潁,末嫁楊敷。薨貞元時。



 咸直公主,貞順皇后所生。下嫁楊洄,又嫁崔嵩。薨興元時。



 宜春公主,蚤薨。



 廣寧公主,董芳儀所生。下嫁程昌胤,又嫁蘇克貞。薨大歷時。



 萬春公主,杜美人所生。下嫁楊朏,又嫁楊錡。薨大歷時。



 太華公主,貞順皇后所生。下嫁楊錡。薨天寶時。



 壽光公主,下嫁郭液。



 樂城公主,下嫁薛履謙,坐嗣岐王珍事誅。



 新平公主,常才人所生。幼智敏,習知圖訓,帝賢之。下嫁裴玪,又嫁姜慶初。慶初得罪,主幽禁中。薨大歷時。



 壽安公主,曹野那姬所生。孕九月而育,帝惡之,詔衣羽人服。代宗以廣平王入謁,帝字呼主曰:「蟲娘,汝後可與名王在靈州請封。」下嫁蘇發。



 肅宗七女。



 宿國公主,始封長樂。下嫁豆盧湛。



 蕭國公主,始封寧國。下嫁鄭巽,又嫁薛康衡。乾元元年,降回紇英武威遠可汗,乃置府。二年,還朝。貞元中,讓府屬,更置邑司。



 和政公主,章敬太后所生。生三歲,後崩,養於韋妃。性敏惠,事妃有孝稱。下嫁柳潭。安祿山陷京師,寧國公主方嫠居,主棄三子,奪潭馬以載寧國,身與潭步,日百里,潭躬水薪,主射爨,以奉寧國。初,潭兄澄之妻,楊貴妃姊也,勢幸傾朝,公主未嘗干以私;及死,撫其子如所生。從玄宗至蜀,始封,遷潭駙馬都尉。郭千仞反,玄宗御玄英樓諭降之,不聽。潭率折沖張義童等殊死鬥,主彀弓授潭,潭手斬賊五十級,平之。肅宗有疾,主侍左右勤勞,詔賜田,以女弟寶章主未有賜,固讓不敢當。阿布思之妻隸掖廷,帝宴,使衣綠衣為倡。主諫曰:「布思誠逆人,妻不容近至尊;無罪,不可與群倡處。」帝為免出之。自兵興,財用耗,主以貿易取奇贏千萬澹軍。及帝山陵,又進邑入千萬。代宗初立,屢陳人間利病、國家盛衰事,天子鄉納。吐蕃犯京師,主避地南奔,次商於,遇群盜,主諭以禍福,皆稽顙願為奴。代宗以主貧,詔諸節度餉億,主一不取。親紉綻裳衣,諸子不服紈絺。廣德時,吐蕃再入寇,主方妊,入語備邊計,潭固止,主曰:「君獨無兄乎?」入見內殿。翌日,免乳而薨。



 郯國公主,始封大寧。下嫁張清。薨貞元時。



 紀國公主,始封宜寧。下嫁鄭沛。薨元和時。



 永和公主,韋妃所生。始封寶章。下嫁王詮。薨大歷時。



 郜國公主,始封延光。下嫁裴徽,又嫁蕭升。升卒,主與彭州司馬李萬亂,而蜀州別駕蕭鼎、澧陽令韋惲、太子詹事李皆私侍主家。久之,奸聞。德宗怒,幽主它第,杖殺萬,斥鼎、惲、弁嶺表。貞元四年,又以厭蠱廢。六年薨。子位,坐為蠱祝,囚端州,佩、儒、偲囚房州,前生子駙馬都尉裴液囚錦州。主女為皇太子妃,帝畏妃怨望,將殺之,未發,會主薨,太子屬疾,乃殺妃以厭災,謚曰惠。



 代宗十八女。



 靈仙公主,蚤薨,追封。



 真定公主,蚤薨,追封。



 永清公主,下嫁裴仿。



 齊國昭懿公主,崔貴妃所生。始封升平。下嫁郭曖。大歷末,寰內民訴涇水為磑壅不得溉田,京兆尹黎幹以請,詔撤磑以水與民。時主及曖家皆有磑,丐留,帝曰:「吾為蒼生,若可為諸戚唱!」即日毀,由是廢者八十所。憲宗即位,獻女伎,帝曰:「太上皇不受獻,朕何敢違?」還之。薨元和時,贈號國,賜謚。穆宗立,復贈封。



 華陽公主,貞懿皇后所生。韶悟過人,帝愛之。視帝所喜,必善遇;所惡,曲全之。大歷七年,以病丐為道士,號瓊華真人。病甚,嚙帝指傷。薨,追封。



 玉清公主,蚤薨,追封。



 嘉豐公主,下嫁高怡。與普寧公主同降,有司具冊禮光順門,以雨不克,罷。薨建中時。



 長林公主,下嫁衛尉少卿沈明。貞元二年具冊禮,德宗不御正殿,不設樂,遂為故事。薨元和時。



 太和公主,蚤薨,追封。



 趙國莊懿公主,始封武清。貞元元年,徙封嘉誠。下嫁魏博節度使田緒,德宗幸望春亭臨餞。厭翟敝不可乘,以金根代之。公主出降,乘金根車,自主始。薨元和時,贈封及謚。



 玉虛公主,蚤薨。



 普寧公主,下嫁吳士廣。



 晉陽公主,下嫁太常少卿裴液。薨大和時。



 義清公主,下嫁秘書少監柳杲。



 壽昌公主,下嫁光祿少卿竇克良。薨貞元時。



 新都公主,貞元十二年下嫁田華,具禮光順門,五禮由是廢。



 西平公主,蚤薨。



 章寧公主,蚤薨。



 德宗十一女。



 韓國貞穆公主,昭德皇后所生。幼謹孝,帝愛之。始封唐安。將下嫁秘書少監韋宥,未克而硃泚亂,從至城固薨,加封謚。



 魏國憲穆公主,始封義陽。下嫁王士平。主恣橫不法,帝幽之禁中;錮士平於第,久之,拜安州刺史,坐交中人貶賀州司戶參軍。門下客蔡南史、獨孤申叔為主作《團雪散雪辭》狀離曠意。帝聞,怒,捕南史等逐之,幾廢時士科。薨,追封及謚。



 鄭國莊穆公主,始封義章。下嫁張孝忠子茂宗。薨,加贈及謚。



 臨真公主,下嫁秘書少監薛釗。薨元和時。



 永陽公主,下嫁殿中少監崔諲。



 普寧公主,蚤薨。



 文安公主,丐為道士。薨大和時。



 燕國襄穆公主,始封咸安。下降回紇武義成功可汗,置府。薨元和時,追封及謚。



 義川公主,蚤薨。



 宜都公主,下嫁殿中少監柳昱。薨貞元時。



 晉平公主,蚤薨。



 順宗十一女。



 漢陽公主名暢,莊憲皇后所生。始封德陽郡主。下嫁郭鏦。辭歸第,涕泣不自勝,德宗曰:「兒有不足邪?」對曰:「思相離,無他恨也。」帝亦泣,顧太子曰:「真而子也。」



 永貞元年,與諸公主皆進封。時戚近爭為奢詡事,主獨以儉,常用鐵簪畫壁,記田租所入。文宗尤惡世流侈,因主入,問曰:「姑所服,何年法也?今之弊,何代而然?」對曰:「妾自貞元時辭宮,所服皆當時賜,未嘗改變。元和後,數用兵,悉出禁藏纖麗物賞戰士,由是散於人間,內外相矜,忸以成風。若陛下示所好於下,誰敢不變?」帝悅,詔宮人視主衣制廣狹,遍諭諸主,且敕京兆尹禁切浮靡。主嘗誨諸女曰:「先姑有言,吾與若皆帝子,驕盈貴侈,可戒不可恃。」開成五年薨。



 梁國恭靖公主,與漢陽同生。始封咸寧郡主,徙普安。下嫁鄭何。薨,追封及謚。



 東陽公主,始封信安郡主。下嫁崔杞。



 西河公主,始封武陵郡主。下嫁沈翬。薨咸通時。



 雲安公主,亦漢陽同生。下嫁劉士涇。



 襄陽公主,始封晉康縣主。下嫁張孝忠子克禮。主縱恣,常微行市里。有薛樞、薛渾、李元本皆得私侍,而渾尤愛,至謁渾母如姑。有司欲致詰,多與金,使不得發。克禮以聞,穆宗幽主禁中。元本乃功臣惟簡子,故貸死,流象州,樞、渾崖州。



 潯陽公主,崔昭儀所生。大和三年,與平恩、邵陽二公主並為道士,歲賜封物七百匹。



 臨汝公主,崔昭訓所生。蚤薨。



 虢國公主,始封清源郡主,徙陽安。下嫁王承系。薨,追封。



 平恩公主,蚤薨。



 邵陽公主,蚤薨。



 憲宗十八女。



 梁國惠康公主,始封普寧。帝特愛之。下嫁於季友。元和中,徙永昌。薨,詔追封及謚。將葬,度支奏義陽、義章公主葬用錢四千萬,有詔減千萬。



 永嘉公主,為道士。



 衡陽公主,蚤薨。



 宣城公主,下嫁沈。



 鄭國溫儀公主,始封汾陽。下嫁韋讓。薨,追封及謚。



 岐陽莊淑公主,懿安皇后所生。下嫁杜悰,帝為御正殿臨遣,繇西朝堂出,復御延喜門,止主車,大賜賓從金錢。開第昌化里,疏龍首池為沼。後家上尚父大通里亭為主別館。貴震當世。然主事舅姑以禮聞,所賜奴婢偃蹇,皆上還,丐直自市。悰為澧州刺史,主與偕,從者不二十婢,乘驢,不肉食,州縣供具,拒不受。姑寢疾,主不解衣,藥糜不嘗不進。開成中,悰自忠武入朝,主疾侵,曰:「願朝興慶宮,雖死於道,不恨。」道薨。



 陳留公主,下嫁裴損。損為太子諭德。



 真寧公主,下嫁薛翃。



 南康公主,下嫁沈汾。薨咸通時。



 臨真公主,始封襄城。下嫁衛洙。薨咸通時。



 普康公主,蚤薨。



 真源公主,始封安陵。下嫁杜中立。



 永順公主,下嫁劉弘景。



 安平公主,下嫁劉異。宣宗即位,宰相以異為平盧節度使,帝曰:「朕唯一妹,欲時見之。」乃止。後隨異居外,歲時輒乘馹入朝。薨乾符時。



 永安公主,長慶初,許下嫁回鶻保義可汗,會可汗死,止不行。太和中,丐為道士,詔賜邑印,如尋陽公主故事,且歸婚貲。



 義寧公主,未及下嫁薨。



 定安公主,始封太和。下嫁回鶻崇德可汗。會昌三年來歸,詔宗正卿李仍叔、秘書監李踐方等告景陵。主次太原,詔使勞問系塗,以黠戛斯所獻白貂皮、玉指環往賜。至京師,詔百官迎謁再拜。故事:邑司官承命答拜,有司議:「邑司官卑,不可當。」群臣請以主左右上媵戴鬢帛承拜,兩襠持命。又詔神策軍四百具鹵簿,群臣班迓。主乘輅謁憲、穆二室,欷歔流涕,退詣光順門易服、褫冠金奠待罪,自言和親無狀。帝使中人勞慰,復冠金奠乃入,群臣賀天子。又詣興慶宮。明日,主謁太皇太后。進封長公主,遂廢太和府。主始至,宣城以下七主不出迎,武宗怒。差奪封絹贖罪。宰相建言:「禮始中壺,行天下,王化之美也,請載於史,示後世。」詔可。



 貴鄉公主,蚤薨。



 穆宗八女。



 義豐公主,武貴妃所生。下嫁韋處仁。薨咸通時。



 淮陽公主,張昭儀所生。下嫁柳正元。



 延安公主,下嫁竇浣。



 金堂公主,始封晉陵。下嫁郭仲恭。薨乾符時。



 清源公主,薨太和時。



 饒陽公主,下嫁郭仲詞。



 義昌公主,為道士。薨咸通時。



 安康公主,為道士。乾符四年,以主在外頗擾人,詔與永興、天長、寧國、興唐四主還南內。



 敬宗三女。



 永興公主。



 天長公主。



 寧國公主,薨廣明時。



 文宗四女。



 興唐公主。



 西平公主。



 郎寧公主,薨咸通時。



 光化公主,薨廣明時。



 武宗七女。



 昌樂公主。



 壽春公主。



 長寧公主,薨大中時。



 延慶公主。



 靜樂公主,薨咸通時。



 樂溫公主。



 永清公主,薨咸通時。



 宣宗十一女。



 萬壽公主,下嫁鄭顥。主,帝所愛,前此下詔:「先王制禮,貴賤共之。萬壽公主奉舅姑,宜從士人法。」舊制:車輿以鐐金釦飾。帝曰:「我以儉率天下,宜自近始,易以銅。」主每進見,帝必諄勉篤誨,曰:「無鄙夫家,無干時事。」又曰:「太平、安樂之禍,不可不戒!」故諸主祗畏,爭為可喜事。帝遂詔:「夫婦,教化之端。其公主、縣主有子而寡,不得復嫁。」



 永福公主。



 齊國恭懷公主,始封西華。下嫁嚴祁。祁為刑部侍郎。主薨大中時,追贈及謚。



 廣德公主,下嫁於琮。初,琮尚永福公主,主與帝食,怒折匕箸,帝曰:「此可為士人妻乎?」更許琮尚主。琮為黃巢所害,主泣曰:「今日誼不獨存,賊宜殺我!」巢不許,乃縊室中。主治家有禮法,嘗從琮貶韶州,侍者才數人,卻州縣饋遺。凡內外冠、婚、喪、祭,主皆身答勞,疏戚咸得其心,為世聞婦。



 義和公主。



 饒安公主。



 盛唐公主。



 平原公主,薨咸通時,已而追封。



 唐陽公主。



 許昌莊肅公主,下嫁柳陟。薨中和時。



 豐陽公主。



 懿宗八女。



 衛國文懿公主,郭淑妃所生。始封同昌。下嫁韋保衡。咸通十年薨。帝既素所愛,自制挽歌,群臣畢和。又許百官祭以金貝、寓車、廞服,火之,民爭取煨以汰寶。及葬,帝與妃坐延興門,哭以過柩,仗衛彌數十里,冶金為俑,怪寶千計實墓中,與乳保同葬。追封及謚。



 安化公主。



 普康公主。



 昌元公主,薨咸通時。



 昌寧公主。



 金華公主。



 仁壽公主。



 永壽公主。



 僖宗二女。



 唐興公主。



 永平公主。



 昭宗十一女。



 新安公主。



 平原公主,積善皇后所生。帝在鳳翔,以主下嫁李茂貞子繼



 偘,後謂不可。帝曰:「不爾,我無安所!」是日,宴內殿,茂貞坐帝東南,主拜殿上。繼偘族兄弟皆西向立,主遍拜之。及帝還,硃全忠移茂貞書,取主還京師。



 信都公主。



 益昌公主。



 唐興公主。



 德清公主。



 太康公主。



 永明公主,蚤薨。



 新興公主。



 普安公主。



 樂平公主。



 贊曰:婦人內夫家,雖天姬之貴,史官猶外而不詳。又僖、昭之亂,典策埃滅,故諸帝公主降日、薨年,粗得其概,亡者闕而不書。



\end{pinyinscope}