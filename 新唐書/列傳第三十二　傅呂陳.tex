\article{列傳第三十二 傅呂陳}

\begin{pinyinscope}

 傅弈,相州鄴人。隋開皇中,以儀曹事漢王諒。諒反,問弈:「今茲熒惑入井,果若何?」對曰:「東井辯證法的革命精神。主要著作除上述者外,還有《精神現象,黃道所由,熒惑之舍,烏足怪邪?若入地上井,乃為災。」諒怒。俄及敗,弈以對免,徙扶風。



 高祖為扶風太守,禮之。及即位,拜太史丞。會令庾儉以父質占候忤煬帝死,懲其事,恥以術宦,薦弈自代。弈遷令,與儉同列,數排毀之,儉不為恨。於是人多儉仁,罪弈遽且忿。



 時國制草具,多仍隋舊,弈謂承亂世之後,當有變更,乃上言:「龍紀、火官,黃帝廢之,《咸池》、《六英》,堯不相沿,禹弗行舜政,周弗襲湯禮。《易》稱『巳日乃孚,革而信也』,故曰『革之時大矣哉』。有隋之季,違天害民,專峻刑法,殺戮賢俊,天下兆庶同心叛之。陛下撥亂反正,而官名、律令一用隋舊。且懲沸羹者吹冷齏,傷弓之鳥驚曲木,況天下久苦隋暴,安得不新其耳目哉?改正朔,易服色,變律令,革官名,功極作樂,治終制禮,使民知盛德之隆,此其時也。然官貴簡約,夏後官百不如虞氏五十,周三百不如商之百。」又曰:「夏有亂政而作《禹刑》,商有亂政而作《湯刑》,周有亂政而作《九刑》。衛鞅為秦制法,增鑿顛、抽脅、鑊烹等六篇,始皇為挾書律,此失於煩,不可不監。」



 是時,太僕卿張道源建言:「官曹文簿繁總易欺,請減之以鈐吏奸。」公卿舉不為然,弈獨是之,為眾沮訾,不得行。



 武德七年,上疏極詆浮圖法曰:



 西域之法,無君臣父子,以三塗六道嚇愚期庸。追既往之罪,窺將來之福,至有身陷惡逆,獄中禮佛,口誦梵言,以圖偷免。且生死壽夭,本諸自然;刑德威福,系之人主。今其徒矯托,皆云由佛,攘天理,竊主權。《書》曰:「惟闢作福,惟闢作威,惟闢玉食。」臣有作福作威玉食,害於而家,兇於而國。



 五帝三王,未有佛法,君明臣忠,年祚長久。至漢明帝始立胡祠,然惟西域桑門自傳其教。西晉以上,不許中國髡發事胡。至石、苻亂華,乃弛厥禁,主庸臣佞,政虐祚短,事佛致然。梁武、齊襄尤足為戒。昔褒姒一女,營惑幽王,能亡其國,況今僧尼十萬,刻繪泥像,以惑天下,有不亡乎?陛下以十萬之眾,自相夫婦,十年滋產,十年教訓,兵農兩足,利可勝既邪?昔高齊章仇子他言僧尼塔廟,外見毀宰臣,內見疾妃嬙,陽讒陰謗,卒死都市,周武帝入齊,封寵其墓,臣竊賢之。



 又上十二論,言益痛切。帝下弈議有司,唯道源佐其請。中書令蕭瑀曰:「佛,聖人也,非聖人者無法,請誅之。」弈曰:「禮,始事親,終事君。而佛逃父出家,以匹夫抗天子,以繼體悖所親。瑀非出空桑,乃尊其言,蓋所謂非孝者無親。」瑀不答,但合爪曰:「地獄正為是人設矣。」帝善弈對,未及行,會傳位止。



 初,九年,太白躔秦分,弈奏秦王當有天下,帝以奏付王。及太宗即位,召賜食,謂曰:「向所奏,幾敗我!雖然,自今毋有所諱而不盡言。」又嘗問:「卿拒佛法,奈何?」弈曰:「佛,西胡黠人爾,欺訹夷狄以自神。至入中國,而鏚兒幻夫摸象莊、老以文飾之,有害國家,而無補百姓也。」帝異之。



 貞觀十三年,卒,年八十五。弈病,未嘗問醫,忽酣臥,蹶然悟曰:「吾死矣乎!」即自志曰:「傅弈,青山白雲人也。以醉死,嗚乎!」遺言戒子:「《六經》名教言,若可習也;妖胡之法,慎勿為。吾死當惈葬。」弈雖善數,然嘗自言其學不可以傳。又注《老子》,並集晉、魏以來與佛議駁者為《高識篇》。武德時,所改漏刻,定十二軍號,皆詔弈云。



 呂才,博州清平人。貞觀時,祖孝孫增損樂律,與音家王長通、白明達更質難,不能決。太宗詔侍臣舉善音者,中書令溫彥博白才天悟絕人,聞見一接,輒窮其妙;侍中王珪、魏徵盛稱才制尺八凡十二枚,長短不同,與律諧契。即召才直弘文館,參論樂事。



 帝嘗覽周武帝《三局象經》,不能通,或言太子洗馬蔡允恭能之,召問允恭,少通其略,老乃忘。試問才,退一昔即解,具圖以聞。允恭記其舊,與才正同,由是知名。擢累太常博士。



 帝病陰陽家所傳書多謬偽淺惡,世益拘畏,命才與宿學老師刪落煩訛,掇可用者為五十三篇,合舊書四十七,凡百篇,詔頒天下。才於持議儒而不俚,以經誼推處其驗術,諸家共訶短之,又舉世相惑以禍福,終莫悟云。



 才之言不甚文,要欲救俗失,切時事,俾易曉也。故叕刂其三篇。



 《卜宅篇》曰:



 《易》稱「上古穴居而野處,後世聖人易之以宮室。蓋取諸《大壯》」。殷、周時有卜擇之文,《詩》稱「相其陰陽」,《書》卜洛食。近世乃有五姓,謂宮也,商也,角也,徵也,羽也,以為天下萬物悉配屬之,以處吉兇,然言皆不類。如張、王為商,武、庾為羽,是以音相諧附;至柳為宮,趙為角,則又不然。其間一姓而兩屬,復姓數字不得所歸。是直野人巫師說爾。按《堪輿經》,黃帝對天老,始言五姓。且黃帝時獨姬、姜數姓耳,後世賜族者浸多,然管、蔡、郕、霍、魯、衛、毛、聃、郜、雍、曹、滕、畢、原、酆、郇本之姬姓,孔、殷、宋、華、向、蕭、亳、皇甫本之子姓,至因官命氏,因邑賜族,本同末異,叵為配宮商哉?春秋以陳、衛、秦為水姓,齊、鄭、宋為火姓,或所出之祖,所分之星,所居之地,以著由來,非宮、商、角、徵、羽相管攝也。



 《祿命篇》曰:



 漢宋忠、賈誼譏司馬季主曰:「卜筮者高人祿命,以悅人心;矯言禍福,以規人財。」王充曰:「見骨體,知命祿;見命祿,知骨體。」此則言祿命尚矣。推索本原,固不其然。「積善之家,必有餘慶」,豈建祿而後吉乎?「積惡之家,必有餘殃」,豈劫殺而後災乎?「皇天無親,常與善人」,天人之交如影響。「有夏多罪,天命剿絕」;宋景脩德,妖星退舍。「學也祿在某中」,不生當建學。文王憂勤損壽,非初值空亡;長平坑降卒,非俱犯三刑;南陽多近親,非俱當六合;歷陽成湖,不共河魁;蜀郡炎火,不盡災厄。世有同建與祿,而貴賤殊域;共命若胎,而夭壽異科。魯桓公六年七月,子同生,是為莊公。按歷,歲在乙亥,月建申,然則值祿空亡,據法應窮賤。又觸句絞六害,偝驛馬,身克驛馬三刑,法無官。命火也,生當病鄉,法曰「為人尪弱矬陋」,而《詩》言莊公曰:「猗嗟昌兮,頎而長兮。美目揚兮,巧趨蹌兮。」唯向命一物,法當壽,而公薨止四十五。一不驗。秦昭襄王四十八年,始皇帝生以正月,故名政。是歲壬寅正月,命偝祿,於法無官,假得祿,奴婢應少。又破驛馬三刑,身克驛馬,法望官不到。命金也,正月為絕,無始有終,老而吉。又建命生,法當壽,帝崩時不過五十。二不驗。漢武帝以乙酉歲七月七日平旦生,當祿空亡,於法無官。雖向驛馬,乃隔四辰,法少無官,老而吉;武帝即位,年十六,末年戶口減耗。三不驗。後魏高祖孝文皇帝生皇興元年八月,是歲丁未,為偝祿命與驛馬三刑,身克驛馬,於法無官。又生父死中,法不見父,而孝文受其父顯祖之禪。禮,君未逾年,不得正位,故天子無父,事三老也。孝文率天下生子墓中,法宜嫡子,雖有次子,當早卒,而高祖長子先被弒,次子義隆享國。又生祖祿下,法得嫡孫財若祿;其孫劭、浚皆篡逆,幾失宗祧。五不驗。



 《葬篇》曰:



 《易》稱:「古之葬者,衣之以薪,不封不樹,喪期無數,後世聖人易之以棺郭。蓋取諸《大過》。」《經》曰:「葬者,藏也,欲人之弗得見也。」又曰:「卜其宅兆,而安厝之。」以是為感慕之所也,魂神之宅也。朝市貿遷不可知,石泉頹嚙不可常,是其謀及卜筮,庶無後艱,斯則備於慎終之禮也。後代葬說出於巫史,一物有失,便謂災及死生,多為妨禁,以售其術,附妄憑妖,至其書乃有百二十家。《春秋》:「王者七日而殯,七月而葬;諸侯五日而殯,五月而葬;大夫三月,士庶人逾月而已。」貴賤不同,禮亦異數。此直為赴吊遠近之期,量事制法。故先期而葬,謂之不懷也;後期不葬,謂之殆禮也。此則葬有定期,不擇年與月,一也。又曰:「丁巳,葬定公,雨,不克葬,至於戊午襄事。」君子善之。《禮》「卜先遠日」者,自末而進,避不懷也。今法己亥日用葬最兇,春秋是日葬者二十餘族。此葬不擇日,二也。《禮》:「周尚赤,大事用旦;殷尚白,大事用日中;夏尚黑,大事用昏。」大事者何?喪禮也。此直取當代所尚,而不擇時早晚也。鄭卿子產及子太叔葬簡公。於是,司墓大夫室當柩路,若壞其室,即平旦而堋;不壞其室,即日中而堋。子產不欲壞室,欲待日中。子太叔曰:「若日中而堋,恐久勞諸侯大夫來會葬者。」然子產、太叔不問時之得失,惟論人事可否而已。曾子曰:「葬逢日蝕,舍於路左,待明而行。」所以備非常也。按法,葬家多取乾、艮二時,乃近夜半,文與禮乖。此葬不擇時,三也。《經》曰:「立身行道,揚名於後世,以顯父母。」《易》謂:「聖人之大寶曰位,何以守位曰仁。」而法曰:「官爵富貴,葬可致也;年壽修促,子姓蕃衍,葬可招也。」夫日慎一日,澤及無疆;德則不建,而祚乃無永。臧孫有後於魯,不聞葬得吉也;若敖絕祀於荊,不聞葬得兇也。此葬有吉兇不可信,四也。今法皆據五姓為之。古之葬,並在國都之北,趙氏之葬,在九原,漢家山陵,或散處諸域,又何上利下利、大墓小墓為哉?然劉之子孫,本支不絕,趙後與六國等王。此則葬用五姓不可信,五也。且人有初賤而後貴、始泰而終否者。子文為令尹,三仕三已,展禽三黜於士師。彼塚墓已定而不改,此名位不常,何也?故知榮辱升降,事關諸人,而不由於葬,六也。世之人為葬巫所欺,忘擗踴荼毒,以期徼幸。由是相塋隴,希官爵;擇日時,規財利。謂辰日不哭,欣然而受吊;謂同屬不得臨壙,吉服避送其親。詭斁禮俗,不可以法,七也。



 帝又詔造《方域圖》及教飛騎戰陣圖,屢稱旨。擢太常丞。麟德中,以太子司更大夫卒。生平豫修書及著述甚多。



 子方毅,七歲能誦經。太宗聞其敏,召見,奇之,賜束帛。長為右衛鎧曹參軍。母喪,以毀卒。布車從母葬,通人郎餘令以白粥、玄酒、生芻祭路隅,世共哀之。



 陳子昂,字伯玉,梓州射洪人。其先居新城,六世祖太樂,當齊時。兄弟競豪人桀,梁武帝命為郡司馬。父元敬,世高貲,歲饑,出粟萬石賑鄉里。舉明經,調文林郎。



 子昂十八未知書,以富家子,尚氣決,弋博自如。它日入鄉校,感悔,即痛修飭。文明初,舉進士。時高宗崩,將遷梓宮長安,於是,關中無歲,子昂盛言東都勝塏,可營山陵。上書曰:



 「臣聞秦據咸陽,漢都長安,山河為固,而天下服者,以北假胡、宛之利,南資巴、蜀之饒,轉關東之粟,而收山西之寶,長羈利策,橫制宇宙。今則不然,燕、代迫匈奴,巴、隴嬰吐蕃,西老千里贏糧,北丁十五乘塞,歲月奔命,秦之首尾不完,所餘獨三輔間耳。頃遭荒饉,百姓薦饑,薄河而右,惟有赤地;循隴以北,不逢青草。父兄轉徙,妻子流離。賴天悔禍,去年薄稔,羸耗之餘,幾不沉命。然流亡未還,白骨縱橫,阡陌無主,至於蓄積,猶可哀傷。陛下以先帝遺意,方大駕長驅,按節西京,千乘萬騎,何從仰給?山陵穿復,必資徒役,率臒弊之眾,興數萬之軍,調發近畿,督抶稚老,鏟山輦石,驅以就功,春作無時,何望有秋?雕氓遺噍,再罹艱苦,有不堪其困,則逸為盜賊,揭梃叫呼,可不深圖哉!



 且天子以四海為家,舜葬蒼梧,禹葬會稽,豈愛夷裔而鄙中國耶?示無外也。周平王、漢光武都洛,而山陵寢廟並在西土者,實以時有不可,故遺小存大,去禍取福也。今景山崇秀,北對嵩、邙,右眄汝、海,祝融、太昊之故墟在焉。園陵之美,復何以加?且太原廥巨萬之倉,洛口儲天下之粟,乃欲舍而不顧,儻鼠竊狗盜,西入陜郊,東犯虎牢。取敖倉一抔粟,陛下何與遏之?



 武後奇其才,召見金華殿。子昂貌柔野,少威儀,而占對慷慨,擢麟臺正字。



 垂拱初,詔問群臣「調元氣當以何道?」子昂因是勸後興明堂、大學,即上言:



 臣聞之於師曰:「元氣,天地之始,萬物之祖,王政之大端也。天地莫大於陰陽,萬物莫靈於人,王政莫先於安人。故人安則陰陽和,陰陽和則天地平,天地平則元氣正。先王以人之通於天也,於是養成群生,順天德,使人樂其業,甘其食,美其服,然後天瑞降,地符升,風雨時,草木茂遂。故顓頊、唐、虞不敢荒寧,其《書》曰:「百姓昭明,協和萬邦,黎人於變時雍。乃命羲和,欽若昊天,歷象日月星辰,敬授人時。」和之得也。夏、商之衰,桀、紂昏暴,陰陽乖行,天地震怒,山川神鬼,發妖見災,疾疫大興,終以滅亡,和之失也。迨周文、武創業,誠信忠厚加於百姓,故成、康刑措四十餘年,天人方和。而幽、厲亂常,苛慝暴虐,詬黷天地,川塚沸崩,人用愁怨。其《詩》曰:「昊天不惠,降此大戾」,不先不後,為虐為瘵,顧不哀哉!近隋煬帝恃四海之富,鑿渠決河,自伊、洛屬之揚州,疲生人之力,洩天地之藏,中國之難起,故身死人手,宗廟為墟。逆元氣之理也。臣觀禍亂之動,天人之際,先師之說,昭然著明,不可欺也。



 陛下含天地之德,日月之明,眇然遠思,欲求太和,此伏羲氏所以為三皇首也。昔者,天皇大帝攬元符,東封太山,然未建明堂,享上帝,使萬世鴻業闕而不昭,殆留此盛德,以發揮陛下哉!臣謂和元氣,睦人倫,舍此則無以為也。昔黃帝合宮,有虞總期,堯衢室,夏世室,皆所以調元氣,治陰陽也。臣聞明堂有天地之制,陰陽之統,二十四氣、八風、十二月、四時、五行、二十八宿,莫不率備。王者政失則災,政順則祥。臣願陛下為唐恢萬世之業,相國南郊,建明堂,與天下更始,按《周禮》、《月令》而成之。乃月孟春,乘鸞輅,駕蒼龍,朝在三公、九卿、大夫於青陽左個,負斧扆,馮玉幾,聽天下之政。躬藉田、親蠶以勸農桑,養三老、五更以教孝悌,明訟恤獄以息淫刑,脩文德以止干戈,察孝廉以除貪吏。後宮非妃嬪御女者,出之;珠玉錦繡、雕琢伎巧無益者,棄之;巫鬼淫祀營惑於人者,禁之。臣謂不數期且見太平雲。



 又言:



 陛下方興大化,而太學久廢,堂皇埃蕪,《詩》、《書》不聞,明詔尚未及之,愚臣所以私恨也。太學者,政教之地也,君臣上下之取則也,俎豆揖讓之所興也,天子於此得賢臣焉。今委而不論,雖欲睦人倫,興治綱,失之本而求之末,不可得也。「君子三年不為禮,禮必壞,三年不為樂,樂必崩」,奈何為天下而輕禮樂哉?願引胄子使歸太學,國家之大務不可廢已。



 後召見,賜筆札中書省,令條上利害。子昂對三事。其一言:



 九道出大使巡按天下,申黜陟,求人瘼,臣謂計有未盡也。且陛下發使,必欲使百姓知天子夙夜憂勤之也,群臣知考績而任之也,奸暴不逞知將除之也,則莫如擇仁可以恤孤、明可以振滯、剛不避強御、智足以照奸者,然後以為使,故輶軒未動,而天下翹然待之矣。今使且未出,道路之人皆已指笑,欲望進賢下不肖,豈可得邪?宰相奉詔書,有遣使之名,無任使之實。使愈出,天下愈弊,徒令百姓治道路,送往迎來,不見其益也。臣願陛下更選有威重風概為眾推者,因御前殿,以使者之禮禮之,諄諄戒敕所以出使之意,乃授以節。自京師及州縣,登拔才良,求人瘼,宣布上意,令若家見而戶曉。昔堯、舜不下席而化天下,蓋黜陟幽明能折衷者。陛下知難得人,則不如少出使。彼煩數而無益於化,是烹小鮮而數撓之矣。



 其二言:



 刺史、縣令,政教之首。陛下布德澤,下詔書,必待刺史、縣令謹宣而奉行之。不得其人,則委棄有司,掛墻屋耳,百姓安得知之?一州得才刺史,十萬戶賴其福;得不才刺史,十萬戶受其困。國家興衰,在此職也。今吏部調縣令如補一尉,但計資考,不求賢良。有如不次用人,則天下囂然相謗矣,狃於常而不變也。故庸人皆任縣令,教化之陵遲,顧不甚哉!



 其三言:



 天下有危機,禍福因之而生。機靜則有福,動則有禍,百姓安則樂生,不安則輕生者是也。今軍旅之弊,夫妻不得安,父子不相養,五六年矣。自劍南盡河、隴,山東由青、徐、曹、汴,河北舉滄、瀛、趙、鄚,或困水旱,或頓兵疫,死亡流離略盡,尚賴陛下憫其失職,凡兵戍調發,一切罷之,使人得妻子相見,父兄相保,可謂能靜其機也。然臣恐將相有貪夷狄利,以廣地強武說陛下者,欲動其機,機動則禍構。宜脩文德,去刑罰,勸農桑,以息疲民。蠻夷知中國有聖王,必累譯至矣。



 於時,吐蕃、九姓叛,詔田揚名發金山道十姓兵討之。十姓君長以三萬騎戰,有功,遂請入朝。後責其嘗不奉命擅破回紇,不聽。子昂上疏曰:



 國家能制十姓者,繇九姓強大,臣服中國,故勢微弱,委命下吏。今九姓叛亡,北蕃喪亂,君長無主,回紇殘破,磧北諸姓已非國有,欲犄角亡叛,唯金山諸蕃共為形勢。有司乃以揚名擅破回紇,歸十姓之罪,拒而遣還,不使入朝,恐非羈戎之長策也。夫戎有鳥獸心,親之則順,疑之則亂,今阻其善意,則十姓內無國家親信之恩,外有回紇報仇之患,懷不自安,鳥駭狼顧,則河西諸蕃自此拒命矣。且夷狄相攻,中國之福。今回紇已破,既無可言;十姓非罪,又不當絕。罪止揚名,足以慰其酋領矣。



 近詔同城權置安北府,其地當磧南口,制匈奴之沖,常為劇鎮。臣頃聞磧北突厥之歸者已千餘帳,來者未止,甘州降戶四千帳,亦置同城。今磧北喪亂、荒饉之餘,無所存仰,陛下開府招納,誠覆全戎狄之仁也。然同城本無儲峙,而降附蕃落不免寒饑,更相劫掠。今安北有官牛羊六千,粟麥萬斛,城孤兵少,降者日眾,不加救恤,盜劫日多。夫人情以求生為急,今有粟麥牛羊為之餌,而不救其死,安得不為盜乎?盜興則安北不全,甘、涼以往,蹺以待陷,後為邊患,禍未可量。是則誘使亂,誨之盜也。且夷狄代有雄人桀,與中國抗,有如勃起,招合遺散,眾將系興,此國家大機,不可失也。



 又謂:



 河西諸州,軍興以來,公私儲蓄,尤可嗟痛。涼州歲食六萬斛,屯田所收不能償墾。陛下欲制河西,定亂戎,此州空虛,未可動也。甘州所積四十萬斛,觀其山川,誠河西喉咽地,北當九姓,南逼吐蕃,奸回不測,伺我邊罅。故甘州地廣粟多,左右受敵,但戶止三千,勝兵者少,屯田廣夷,倉庾豐衍,瓜、肅以西,皆仰其餫,一旬不往,士已枵饑。是河西之命系於甘州矣。且其四十餘屯,水泉良沃,不待天時,歲取二十萬斛,但人力寡乏,未盡墾發。異時吐番不敢東侵者,繇甘、涼士馬強盛,以振其入。今甘州積粟萬計,兵少不足以制賊,若吐蕃敢大入,燔蓄穀,蹂諸屯,則河西諸州,我何以守?宜益屯兵,外得以防盜,內得以營農,取數年之收,可飽士百萬,則天兵所臨,何求不得哉?



 其後吐蕃果入寇,終後世為邊患最甚。



 後方謀開蜀山,由雅州道翦生羌,因以襲吐蕃。子昂上書以七驗諫止之,曰:



 臣聞亂生必由於怨。雅州羌未嘗一日為盜,今無罪蒙戮,怨必甚,怨甚則蜂駭且亡,而邊邑連兵,守備不解,蜀之禍構矣。東漢喪敗,亂始諸羌,一驗也。吐蕃黠獪,抗天誅者二十餘年。前日薛仁貴、郭待封以十萬眾敗大非川,一甲不返;李敬玄、劉審禮舉十八萬眾困青海,身執賊廷,關、隴為空。今乃欲建李處一為上將,驅疲兵襲不可幸之吐蕃,舉為賊笑,二驗也。夫事有求利而得害者。昔蜀與中國不通,秦以金牛、美女啖蜀侯,侯使五丁力士棧褒斜,鑿通谷,迎秦之饋。秦隨以兵,而地入中州,三驗也。吐蕃愛蜀富,思盜之矣,徒以障隊隘絕,頓餓喙不得噬。今撤山羌,開阪險,使賊得收奔亡以攻邊,是除道待賊,舉蜀以遺之,四驗也。蜀為西南一都會,國之寶府,又人富粟多,浮江而下,可濟中國。今圖僥幸之利,以事西羌,得羌地不足耕,得羌財不足富。是過殺無辜之眾,以傷陛下之仁,五驗也。蜀所恃,有險也;蜀所安,無役也。今開蜀險,役蜀人,險開則便寇,人役則傷財。臣恐未及見羌,而奸盜在其中矣。異時益州長史李崇真托言吐蕃寇松州,天子為盛軍師,趣轉餉以備之。不三年,巴、蜀大困,不見一賊,而崇真奸贓已鉅萬。今得非有奸臣圖利,復以生羌為資?六驗也。蜀士尪孱不知兵,一虜持矛,百人不敢當。若西戎不即破滅,臣見蜀之邊垂且不守,而為羌夷所暴,七驗也。國家近廢安北,拔單于,棄龜茲、疏勒,天下以為務仁不務廣,務養不務殺,行太古三皇事。今徇貪夫之議,誅無罪之羌,遺全蜀患,此臣所未諭。方山東饑,關隴弊,生人流亡,誠陛下寧靜思和天人之時,安可動甲兵、興大役,以自生亂?又西軍失守,北屯不利,邊人駭情,今復舉輿師投不測,小人徒知議夷狄之利,非帝王至德也。善為天下者,計大而不計小,務德而不務刑,據安念危,值利思害。願陛下審計之。



 後復召見,使論為政之要,適時不便者,毋援上古,角空言。子昂乃奏八科:一措刑,二官人,三知賢,四去疑,五招諫,六勸賞,七息兵,八安宗子。其大榷謂:



 今百度已備,但刑急罔密,非為政之要。凡大人初制天下,必有兇亂叛逆之人為我驅除,以明天誅。兇叛已滅,則順人情,赦過宥罪。蓋刑以禁亂,亂靜而刑息,不為承平設也。太平之人,樂德而惡刑,刑之所加,人必慘怛,故聖人貴措刑也。比大赦,澡蕩群罪,天下蒙慶,咸得自新。近日詔獄稍滋,鉤捕支黨,株蔓推窮,蓋獄吏不識天意,以抵慘刻。誠宜廣愷悌之道,敕法慎罰,省白誣冤,此太平安人之務也。



 官人惟賢,政所以治也。然君子小人各尚其類。若陛下好賢而不任,任而不能信,信而不能終,終而不賞,雖有賢人,終不肯至,又不肯勸。反是,則天下之賢集矣。



 議者乃云「賢不可知,人不易識」。臣以為固易知,固易識。夫尚德行者無兇險,務公正者無邪朋,廉者憎貪,信者疾偽,智不為愚者謀,勇不為怯者死,猶鸞隼不接翼,薰蕕不共氣,其理自然。何者?以德並兇,勢不相入;以正攻佞,勢不相利;以廉勸貪,勢不相售;以信質偽,勢不相和。智者尚謀,愚者所不聽;勇者徇死,怯者所不從。此趣向之反也。賢人未嘗不思效用,顧無其類則難進,是以湮汩於時。誠能信任俊良,知左右有灼然賢行者,賜之尊爵厚祿,使以類相舉,則天下之理得矣。



 陛下知得賢須任,今未能者,蓋以常信任者不效。如裴炎、劉禕之、周思茂、騫味道固蒙用矣,皆孤恩前死,以是陛下疑於信賢。臣固不然。昔人有以噎得病,乃欲絕食,不知食絕而身殞。賢人於國,猶食在人,人不可以一噎而止餐,國不可以謬一賢而遠正士,此神鑒所知也。



 聖人大德,在能納諫,太宗德參三王,而能容魏徵之直。今誠有敢諫骨鯁之臣,陛下廣延順納,以新盛德,則萬世有述。



 臣聞勞臣不賞,不可勸功;死士不賞,不可勸勇。今或勤勞死難,名爵不及;偷榮尸祿,寵秩妄加,非所以表庸勵行者也。願表顯徇節,勵勉百僚。古之賞一人,千萬人悅者,蓋云當也。



 今事之最大者,患兵甲歲興,賦役不省,興師十萬,則百萬之家不得安業。自有事北狄,於今十年,不聞中國之勝。以庸將御冗兵,徭役日廣,兵甲日敝。願審量損益,計利害,勢有不可,毋虛出兵,則人安矣。



 虺賊干紀,自取屠滅,罪止魁逆,無復緣坐,宗室子弟,皆得更生。然臣願陛下重曉慰之,使明知天子慈仁,下得自安。臣聞人情不能自明則疑,疑則懼,懼則罪生。惟賜愷悌之德,使居無過之地。



 俄遷右衛胄曹參軍。



 後既稱皇帝,改號周,子昂上《周受命頌》以媚悅後。雖數召見問政事,論亦詳切,故奏聞輒罷。以母喪去官,服終,擢右拾遺。



 子昂多病,居職不樂。會武攸宜討契丹,高置幕府,表子昂參謀。次漁陽,前軍敗,舉軍震恐,攸宜輕易無將略,子昂諫曰:「陛下發天下兵以屬大王,安危成敗在此舉,安可忽哉?」今大王法制不立,如小兒戲。願審智愚,量勇怯,度眾寡,以長攻短,此刷恥之道也。夫按軍尚威嚴,擇親信以虞不測。大王提重兵精甲,屯之境上,硃亥竊發之變,良可懼也。王能聽愚計,分麾下萬人為前驅,契丹小醜,指日可擒。」攸宜以其儒者,謝不納。居數日,復進計,攸宜怒,徙署軍曹。子昂知不合,不復言。



 聖歷初,以父老,表解官歸侍,詔以官供養。會父喪,廬塚次,每哀慟,聞者為涕。縣令段簡貪暴,聞其富,欲害子昂,家人納錢二十萬緡,簡薄其賂,捕送獄中。子昂之見捕,自筮,卦成,驚曰:「天命不祐,吾殆死乎!」果死獄中,年四十三。



 子昂資褊躁,然輕財好施,篤朋友,與陸餘慶、王無競、房融、崔泰之、盧藏用、趙元最厚。



 唐興,文章承徐、庾餘風,天下祖尚,子昂始變雅正。初,為《感遇詩》三十八章,王適曰:「是必為海內文宗。」乃請交。子昂所論著,當世以為法。大歷中,東川節度使李叔明為立旌德碑於梓州,而學堂至今猶存。



 子光,復與趙元子少微相善,俱以文稱。光終商州刺史。子易甫、簡甫,皆位御史。



 王無競者,字仲烈,世徙東萊,宋太尉弘之遠裔。家足於財,頗負氣豪縱。擢下筆成章科,調欒城尉,三遷監察御史,改殿中。會朝,宰相宗楚客、楊再思離立偶語,無競揚笏曰:「朝禮上敬,公等大臣,不宜慢常典。」楚客怒,徙無競太子舍人。



 神龍初,詆權幸,出為蘇州司馬。張易之等誅,坐嘗交往,貶廣州,仇家矯制搒殺之。



 趙元省,字貞固,河間人。祖掞,號通儒,在隋,與同郡劉焯俱召至京師,補黎陽長,徙居汲。



 元少負志略,好論辯。來游雒陽,士爭慕向,所以造謝皆縉紳選。武後方稱制,懼不容其高,調宜祿尉。到職,非公事不言,彈琴蒔藥,如隱者之操。自傷位不配才,卒年四十九。其友魏元忠、孟詵、宋之問、崔璩等共謚昭夷先生。



 贊曰:「子昂說武後興明堂太學,其言甚高,殊可怪笑。後竊威柄,誅大臣、宗室,肋逼長君而奪之權。子昂乃以王者之術勉之,卒為婦人訕侮不用,可謂薦圭璧於房闥,以脂澤污漫之也。瞽者不見泰山,聾者不聞震霆,子昂之於言,其聾瞽歟。



\end{pinyinscope}