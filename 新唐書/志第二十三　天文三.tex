\article{志第二十三 天文三}

\begin{pinyinscope}

 ○月五星凌犯及星變



 隋大業十三年六月,鎮星贏而旅於參。參,唐星也。李淳風曰:「鎮星主福,未當居而居,所宿國吉。」



 義寧二年三月丙午,熒惑入東井。占曰:「大人憂。」



 武德元年五月庚午,太白晝見。占曰:「兵起,臣強。」六月丙子,熒惑犯右執法。占曰:「執法,大臣象。」二年七月戊寅,月犯牽牛。凡月與列宿相犯,其宿地憂。牽牛,吳、越分。九月庚寅,大白晝見。冬,熒惑守五諸侯。六年七月癸卯,熒惑犯輿鬼西南星。占曰:「大臣有誅。」七年六月,熒惑犯右執法。七月戊寅,歲星犯畢。占曰:「邊有兵。」八年九月癸丑,熒惑入太微。太微,天廷也。冬,太白入南斗。斗主爵祿。九年五月,太白晝見;六月丁巳,經天;己未,又經天。在秦分。丙寅,月犯氐。氐為天子宿宮。己卯,太白晝見;七月辛亥,晝見;甲寅,晝見;八月丁巳,晝見。太白,上公;經天者,陰乘陽也。



 貞觀三年三月丁丑,歲星逆行入氐。占曰:「人君治宮室過度。」一曰;「饑。」五年五月庚申,鎮星犯鍵閉。占為腹心喉舌臣。九年四月丙午,熒惑犯軒轅。十年四月癸酉,復犯之。占曰:「熒惑主禮,禮失而後罰出焉。」軒轅為後宮。十一年二月癸未,熒惑入輿鬼。占曰;「賊在大人側。」十二年六月辛卯,熒惑入東井。占曰:「旱。」十三年五月乙巳,犯右執法。六月,太白犯東井北轅。井,京師分也。十四年十一月壬午,月入太微。占曰;「君不安。」十五年二月,芝惑逆行,犯太微東上相。十六年五月,太白犯畢左股,畢為邊將;六月戊戌,晝見。九月己未,熒惑犯太微西上將;十月丙戌,入太微,犯左執法。十七年二月,犯鍵閉;三月丁巳,守心前星;癸酉,逆行犯鉤鈐。熒惑常以十月入太微,受制而出,伺其所守犯,天子所誅也。鍵閉為腹心喉舌臣,鉤鈐以開闔天心,皆貴臣象。十八年十一月乙未,月掩鉤鈐。十九年七月壬午,太白入太微,是夜月掩南斗,太白遂犯左執法,光芒相及箕、鬥間。漢津,高麗地也。太白為兵,亦罰星也。二十年七月丁未,歲星守東壁。占曰:「五穀以水傷。」二十一年四月戊寅,月犯熒惑。占曰:「貴臣死。」十二月丁丑,月食昴。占曰:「天子破匈奴。」二十二年五月丁亥,犯右執法。七月,太白晝見。乙巳,鎮星守東井。占曰;「旱。」閏十二月辛巳,太白犯建星。占曰:「大臣相譖。」



 永徽元年二月己丑,熒惑犯東井。上曰:「旱。」四月己巳,月犯五諸侯,熒惑犯輿鬼。占曰;「諸侯兇。」五月己未,太白晝見。二年六月己丑,太白入太微,犯右執法;九月甲午,犯心前星。十二月乙未,太白晝見。三年正月壬戌,犯牽牛。牽牛為將軍,又吳、越分也。丁亥,歲星掩太微上將。二月己丑,熒惑犯五諸侯;五月戊子,掩右執法。四年六月己丑,太白晝見。六年七月乙亥,歲星守尾。占曰;「人主以嬪為後。」己丑,熒惑入輿鬼;八月丁卯,入軒轅。



 顯慶元年四月丁酉,太白犯東井北轅。占曰:「秦有兵。」五年二月甲午,熒惑入南斗;六月戊申,復犯之。南斗,天廟;去復來者,其事大且久也。



 龍朔元年六月辛巳,太白晝見經天;九月癸卯,犯左執法。二年七月己丑,熒惑守羽林,羽林,禁兵也;三年正月己卯,犯天街。占曰:「政塞奸出。」六月乙酉,太白入東井。占曰:「君失政,大臣有誅。」



 麟德二年三月戊午,熒惑犯東井;四月壬寅,入輿鬼,犯質星。



 乾封元年八月乙巳,熒惑入東井。二年五月庚申,入軒轅。三年正月乙巳,月犯軒轅大星。



 咸亨元年四月癸卯,月犯東井。占曰;「人主憂。」七月壬申,熒惑入東井。占曰:「旱。」丙申,月犯熒惑。占曰;「貴人死。」十二月丙子,熒惑入太微;二年四月戊辰,復犯。太微垣,將相位也。五年六月壬寅,太白入東井。



 上元二年正月甲寅,熒惑犯房。占曰;「君有憂。」一曰;「有喪。」二年正月丁卯,太白犯牽牛。占曰:「將軍兇。」



 儀鳳二年八月辛亥,太白犯軒轅左角。左角,貴相也。三年十月戊寅,熒惑犯鉤鈐;四年四月戊午,入羽林。占曰「軍憂。」



 調露元年七月辛巳,入天囷。



 永隆元年五月癸未,犯輿鬼。丁酉,太白晝見經天。是謂陰乘陽,陽,君道也。



 永淳元年五月丁巳,辰星犯軒轅。九月庚戌,熒惑入輿鬼,犯質星;十一月乙未,復犯輿鬼。去而復來,是謂「句巳」。



 垂拱元年四月癸未,辰星犯東井北轅。辰星為廷尉,東井為法令,失道則相犯也。十二月戊子,月掩軒轅大星;二年三月丙辰,復犯之。



 萬歲通天元年十一月乙丑,歲星犯司怪。占曰:「水旱不時。」



 聖歷元年五月庚午,太白犯天關。天關主邊事。二年,熒惑入輿鬼。三年三月辛亥,歲星犯左執法。



 久視元年十二月甲戌晦,熒惑犯軒轅。



 自乾封二年後,月及熒惑、太白、辰星凌犯軒轅者六。



 長安二年熒惑犯五諸侯。渾儀監尚獻甫奏;「臣命在金,五諸侯太史之位,火克金,臣將死矣。」武后曰:「朕為卿禳之,以獻甫為水衡都尉,水生金,又去太史之位,卿無憂矣。」是秋,獻甫卒。四年,熒惑入月,鎮星犯天關。



 神龍元年三月癸巳,熒惑犯天田,占曰「旱」;七月辛巳,掩氐西南星,占曰「賊臣在內」。二年閏正月丁卯,月掩軒轅後星。九月壬子,熒惑犯左執法。己巳,月犯軒轅後星;十一月辛亥,犯昴,占曰「胡王死」。戊午,熒惑入氐;十二月丁酉,犯天江,占曰「旱」。三年五月戊戌,太白入輿鬼中。占曰:「大臣有誅。」



 景龍三年六月癸巳,太白晝見在東井。京師分也。四年二月癸未,熒惑犯天街。五月甲子,月犯五諸侯。



 景雲二年三月壬申,太白入羽林。八月己未,歲星犯執法。



 太極元年三月壬申,熒惑入東井。



 先天元年八月甲子,太白襲月。占曰:「太白,兵象;月,大臣體。」二年十一月丙子,熒惑犯司怪。



 開元二年七月己丑,太白犯輿鬼東南星。七年六月甲戌,太白犯東井鉞星。占曰:「斧鉞用。」八年三月庚午,犯東井北轅;五月甲子,犯軒轅。十一年十一月丁卯,歲星犯進賢。十四年十月甲寅,太白晝見。二十五年六月壬戌,熒惑犯房。二十七年七月辛丑,犯南斗。占曰:「貴相兇。」



 天寶十三載五月,熒惑守心五旬餘。占曰:「主去其宮。」十四載十二月,月食歲星在東井。占曰:「其國亡。」東井,京師分也。



 至德二載七月己酉。太白晝見經天,至於十一月戊午不見,歷秦、周、楚、鄭、宋、燕之分。十二月,歲星犯軒轅大星。占曰:「女主謀君。」



 乾元元年五月癸未,月掩心前星,占曰「太子憂」;六月癸丑,入南斗魁中,占曰「大人憂」。二年正月癸未,歲星蝕月在翼,楚分也,一曰:「饑。」二月丙辰,月犯心中星。占曰:「主命惡之。」



 上元元年五月癸丑,月掩昴。占曰;「胡王死。」八月己酉,太白犯進賢。十二月癸未,歲星掩房。占曰:「將相憂。」三年建子月癸巳,月掩昴,出昴北;八月丁卯,又掩昴。



 寶應二年四月己丑,月掩歲星。占曰:「饑。」



 永泰元年九月辛卯,太白晝見經天。



 大歷二年七月癸亥,熒惑入氐,其色赤黃。乙丑,鎮星犯水位。占曰:「有水災。」乙亥,歲星犯司怪。八月壬午,月入氐;丙申,犯畢。九月戊甲,歲星守東井。占皆為有兵。乙丑,熒惑犯南斗。在燕分。十二月丁丑,犯壘壁。占曰:「兵起。」三年正月壬子,月掩畢;八月己未,復掩畢;辛酉,入東井。九月壬申,歲星入輿鬼。占曰:「歲星為貴臣,輿鬼主死喪。」丁丑,熒惑入太微,二旬而出。己卯,太白犯左執法。四年二月壬寅,熒惑守房上相;丙午,有芒角;三月壬午,逆行入氐中。是月,鎮星犯輿鬼。七月戊辰,熒惑犯次相;九月丁卯,犯建星。占曰:「大臣相譖。」五年二月乙巳,歲星入軒轅。六月丁酉,月犯進賢;庚子,犯氐。庚戌,太白入東井。六年七月乙巳,月掩畢,入畢中;壬子,月犯太微。八月甲戌,熒惑犯鄭星。庚辰,月入太微。九月壬辰,熒惑犯哭星;庚子,犯泣星。是夜,月掩畢;丁未,入太微;十月丁卯,掩畢。己巳,熒惑犯壘壁。甲戌,月入軒轅。占曰「憂在後宮」;十一月壬寅,入太微;丙午掩氐;十二月己巳,入太微;七年正月乙未,犯軒轅;二月戊午,掩天關。占曰:「亂臣更天子法令。」己巳,熒惑犯天街;四月丁巳,入東井。辛未,歲星犯左角。占曰:「天下之道不通。」壬申,月入羽林;五月丙戌,入太微。八年四月癸丑,歲星掩房。占曰:「將相憂。」又宋分也。甲寅,熒惑入壘壁;五月庚辰,入羽林。七月己卯,太白入東井,留七日,非常度也。占曰:「秦有兵。」乙未,月入畢中。癸未,入羽林。己丑,太白入太微。占曰:「兵入天廷。」八月晝見。十月丁巳,月掩畢;壬戌,入輿鬼,掩質星。庚午,月及太白入氐中。占曰;「君有哭泣事。」十一月己卯,月入羽林。癸未,太白入房。占曰:「白衣會。」不曰犯而曰入,蓋鉤鈐間。癸丑,月掩天關;甲寅,入東井;癸酉,入羽林。九年三月丁未,熒惑入東井。四月丁丑,月入太微。五月己未,太白入軒轅。占曰:「憂在後宮。」六月己卯,月掩南斗;庚辰,入太微;七月甲辰,掩房;辛亥,入羽林;壬戌,入輿鬼。九月辛丑,太白入南斗。占曰:「有反臣。」又曰:「有赦。」甲子,熒惑入氐。宋分也。十月戊子,歲星入南斗。占曰:「大臣有誅。」十二月戊辰,月入羽林。十年三月庚戌,熒惑入壘壁;四月甲子,入羽林。八月戊辰,月入太微。十一年閏八月丁酉,太白晝見經天。十二年正月乙丑,月掩軒轅;癸酉,掩心前星,宋分也」丙子,入南斗魁中。二月乙未,鎮星入氐中。占曰:「其分兵喪。」李正己地也。三月壬戌,月入太微;四月乙未,掩心前星;五月丙辰,入太微;戍戌,入羽林;七月庚戌,入南斗。乙亥,熒惑入東井。十月壬辰,月掩昴;庚子,入太微;十一月乙卯,入羽林;十二月壬午,復入羽林。自六年至此,月入太微者十有二,入羽林者八;熒惑三入東井,再入羽林,三入壘壁;月,太白、歲星,皆入南斗魁中。十四年春,歲星入東井。



 建中元年十一月,月食歲星,在秦分。占曰:「其國亡。」是月,歲星食天尸。天尸,輿鬼中星。占曰:「有妖言,小人在位,君王失樞,死者太半。」三年七月,月掩心中星。



 貞元四年五月丁卯,月犯歲星在營室。六月癸卯,熒惑逆行入羽林。占曰:「軍有憂。」六年五月戊辰,月犯太白,間容一指。占曰:「大將死。」十年四月,太白晝見。十一年七月,熒惑、太白相繼犯太微上將。十三年二月戊辰,太白入昴。三月庚寅,月犯太白。十九年三月,熒惑入南斗,色如血。斗,吳、越分;色如血者,旱祥也。二十一年正月己酉,太白犯昴。趙分也。



 永貞元年十二月丙午,月犯畢,己酉,歲星犯太微西垣。將相位也。



 元和元年十月,太白入南斗;十二月,復犯之。斗,吳分也。二年正月癸丑,月犯太白於女、虛。二月壬申,月掩歲星。占曰:「大臣死。」四月丙子,太白犯東井北轅。己卯,月犯房上相。三年三月乙未,鎮星蝕月在氐。占曰:「其地主死。」四年九月癸亥,太白犯南斗。七年正月辛未,月掩熒惑。五月癸亥,熒惑犯右執法。六月己亥,月犯南斗魁。八年七月癸酉,月犯五諸侯。十月己丑,熒惑犯太微西上將;十二月,掩左執法。九年二月丁酉,月犯心中星;七月辛亥,掩心中星。占曰:「其宿地兇。」心,豫州分。壬辰,月掩軒轅。是月,太白入南斗,至十月出,乃晝見。熒惑入南斗中,因留,犯之。南斗,天廟,又丞相位也。十年八月丙午,月入南斗魁中。十一年二月丙辰,月掩心。是月,熒惑入氐,因逆行。三月己丑,月犯鎮星在女。齊分也。四月丙辰,太白犯輿鬼。占曰:「有僇臣。」六月甲辰,月掩心後星。是月,熒惑復入氐,是謂「句巳」。十一月戊寅,月犯歲星;十二月甲午,犯鎮星在危。亦齊分也。十二年三月丁丑,月犯心。十三年正月乙未,歲星逆行,犯太微西上將。三月,熒惑入南斗,因逆留,至於七月,在南斗中,大如五升器,色赤而怒,乃東行,非常也。八月甲戌,太白犯左執法。乙巳,熒惑犯哭星。十月甲子,月犯昴。趙分也。十四年正月癸卯,月犯南斗魁。占曰;「相兇。」五月丙戌,月犯心中星;七月乙酉,掩心中星;十五年正月丙申,復犯中星。四月,太白犯昴。七月庚申,熒惑逆行入羽林。八月己卯,月掩牽年。吳、越分也。十一月壬子,月犯東井北轅。



 長慶元年正月丙午,月掩東井鉞,遂犯南轅第一星。二月乙亥,太白犯昴。趙分也。丁亥,月犯歲星在尾。占曰:「大臣死。」燕分也。三月庚戌,太白犯五車,因晝見,至於七月。以歷度推之,在唐及趙、魏之分。占曰:「兵起。」七月壬寅,月掩房次相。九月乙巳,太白犯左執法。二年九月,太白晝見。熒惑守天囷,六旬餘乃去。占曰:「天囷,上帝之藏,耗祥也。十月,熒惑犯鎮星於昴。甲子,月掩牽牛中星。占曰;「吳、越兇。」十一月丁丑,掩左角;十二月,復掩之。占曰:「將死。」甲寅,月犯太白於南斗。四年三月庚午,太白犯東井北轅,遂入井中,晝見經天,七日而出,因犯輿鬼。京師分也。五月乙亥,月掩畢大星。六月丙戌,鎮星依歷在觜觿,贏行至參六度,當居不居,失行而前,遂犯井鉞。占曰:「所居宿久,國福厚;易,福薄。」又曰;「嬴,為王不寧;鉞主斬刈而又犯之,其占重。」癸未,熒惑犯東井;丁亥,入井中。己丑,太白犯軒轅右角,因晝見,至於九月。占曰:「相兇。」十月辛巳,月入畢口。十一月,熒惑逆行向參,鎮星守天關。十二月戊子,月掩東井。



 寶歷元年四月壬寅,熒惑入輿鬼,掩積尸;七月癸卯,犯執法。甲辰,鎮星犯東井。甲子,月掩畢大星。癸未,太白犯南斗。丙戌,月犯畢;十月辛亥,犯天囷。十一月庚辰,鎮星復犯東井。癸未,月犯東井。二年正月甲申,犯左執法;戊子,入於氐。二月丙午,犯畢。五月甲午,熒惑犯昴。六月,太白犯昴。七月壬申,月犯畢。八月庚戌,熒惑犯輿鬼。



 大和元年正月庚午,月掩畢;三月癸丑,入畢口,掩大星。月變於畢者,自寶歷元年九月,及茲而五。五月,月掩熒惑在太微西坦。丙戌,熒惑犯右執法。



 大和二年正月庚午,月掩鎮星。七月甲辰,熒惑掩輿鬼質星。十月丁卯,月掩東井北轅。三年二月乙卯,太白犯昴。壬申,熒惑掩右執法;七月,入於氐;十月,入於南斗。四年四月庚申,月掩南斗杓次星。十一月辛未,熒惑犯右執法。五年二月甲申,月掩熒惑。三月,熒惑犯南斗杓次星。六年四月辛未,月掩鎮星於端門。己丑,太白晝見。七月戊戌,月掩心大星;辛丑,掩南斗杓次星。七年五月甲辰,熒惑守心中星。六月丙子,月掩心中星,遂犯熒惑。七月甲午,月掩心中星;丙申,掩南斗口第二星。九月丁巳,入於箕;戊辰,入於南斗。癸酉,太白入南斗。冬,鎮星守角;八年二月始去。七月戊子,月犯昴。十月庚子,熒惑、鎮星合於亢。十二月丙戌,月掩昴。是歲,月入南斗者五。占曰:「大人憂。」九年夏,太白晝見,自軒轅至於翼、軫。六月庚寅,月掩歲星在危而暈;十月庚辰,月復掩歲星在危。



 開成元年正月甲辰,太白掩建星。占曰:「大臣相譖。」六月丁未,月掩心前星;八月乙巳,入南斗。二年正月壬申,月掩昴。二月己亥,月掩太白於昴中。六月甲寅,月掩昴而暈,太白亦有暈。六月己酉,大星晝見。庚申,太白入於東井。七月壬申,月入南斗;丁亥,掩太白於柳。八月壬子,太白入太微,遂犯左、右執法。九月丙子,月掩昴;三年二月己酉,掩心前星。二月戊午,熒惑入東井;三月乙酉,入輿鬼。五月辛酉,太白犯輿鬼。庚午,月犯心中星。甲寅,太白犯右執法。七月乙丑,月掩心前星。十月辛卯,太白犯南斗。四年二月丁卯,月掩歲星於畢;三月乙酉,掩東井。七月乙未,月犯熒惑。占曰;「貴臣死。」八月壬申,熒惑犯鉞,遂入東井。十月戊午,辰星入南斗魁中。占曰:「大赦。」五年春,木當王,而歲星小暗無光。占曰:「有大喪。」二月壬申,熒惑入輿鬼。四月,太白、歲星入輿鬼。五月,辰星見於七星,色赤如火。七月乙酉,月掩鎮星。



 會昌元年閏八月丁酉,熒惑入輿鬼中。占曰:「有兵喪。」十二月庚午,月犯太白於羽林;二年正月壬戌,掩太白於羽林。六月丙寅,太白犯東井。十月丙戌,月掩歲星於角。三年三月丙申,又掩歲星於角。七月癸巳,熒惑入東井,色蒼赤,動搖井中;八月丁丑,犯輿鬼。十月壬午晝,月食太白於亢。四年二月,歲星守房,掩上相;熒惑逆行,守軒轅,四旬乃去。庚申,月掩畢大星。十月癸未,太白與熒惑合,遂入南斗。五年二月壬午,太白掩昴;五月辛酉,入畢口;八月壬午,犯軒轅大星。九月癸巳,熒惑犯太微上將。六年二月丁丑,犯畢大星。丁亥,月出無光,犯熒惑於太微,頃之,乃稍有光,遂犯左執法;丙申,掩牽牛南星,遂犯歲星。牽牛,揚州分。



 大中十一年八月,熒惑犯東井。



 咸通十年春,熒惑逆行,守心。



 乾符二年四月庚辰,太白晝見在昴。三年七月,常星晝見。四年七月,月犯房。六年冬,歲星入南斗魁中。占曰;「有反臣。」



 光啟二年四月,熒惑犯月角。



 文德元年七月丙午,月入南斗。八月,熒惑守輿鬼。占曰:「多戰死。」



 龍紀元年七月甲辰,月犯心。



 乾寧二年七月癸亥,熒惑犯心。



 光化二年,鎮星入南斗。三年八月壬申,太白應見在氐,不見,至九月丁亥乃見,是謂當出不出。十一月丁未,太白犯月,因晝見。



 天復元年五月自丁酉至於己亥,太白晝見經天,在井度。十月,大角五色散搖,煌煌如火。占曰:「王者惡之。」二年五月甲子,太白襲熒惑在軒轅後星上,太白遂犯端門,又犯長垣中星。占曰:「賊臣謀亂,京畿大戰。」十月甲戌,太白夕見在斗,去地一丈而墜。占曰:「兵聚其下。」又曰;「山摧石裂,大水竭。」庚子,辰星見氐中,小而不明。占曰:「負海之國大水。」是歲,鎮星守虛。三年二月始去虛。十一月丙戌,太白在南斗,去地五尺許,色小而黃,至明年正月乃高十丈,光芒甚大。是冬,熒惑徘徊於東井間,久而不去。京師分也。



 天祐元年二月辛卯,太白夕見昴西,色赤,炎焰如火;壬辰,有三角如花而動搖。占曰:「有反,城有火災,胡兵起。」六月甲午,太白在張,芒角甚大;癸丑,句巳,犯水位。自夏及秋,大角五色散搖,煌煌然。占同天復初。三年八月丙午,歲星在哭星上,生黃白氣如孛狀。



 ○五星聚合



 武德元年七月丙午,鎮星、太白、辰星聚於東井。關中分也。二年三月丙申,鎮星、太白、辰星復聚於東井。九年六月己卯,歲星、辰星合於東井。占曰:「為變謀。」



 貞觀十八年五月,太白、辰星合於東井。占曰:「為兵謀。」十九年六月丙辰,太宗征高麗,次安市城,太白、辰星合於東井。《史記》曰:太白為主,辰星為客,為蠻夷,出相從而兵在野為戰。



 永徽元年七月辛酉,歲星、太白合於柳。在秦分。占曰:「兵起。」



 景龍元年十月丙寅,太白、熒惑合於虛、危。占曰;「有喪。



 景雲二年七月,鎮星、太白合於張。占曰;「內兵。」



 太極元年四月,熒惑、太白合於東井。



 天寶九載八月,五星聚於尾、箕,熒惑先至而又先去。尾、箕,燕分也。占曰:「有德則慶,無德則殃。」十四載二月,熒惑、太白鬥於畢、昴、井、鬼間,至四月乃伏。十五載五月,熒惑、鎮星同在虛、危,中天芒角大動搖。占者以為北方之宿,子午相沖,災在南方。



 至德二載四月壬寅,歲星、熒惑、太白、辰星聚於鶉首,從歲星也。罰星先去,而歲星留。占曰;「歲星、熒惑為陽,太白、辰星為陰。陰主外邦,陽主中邦,陽與陰合,中外相連以兵。」八月,太白芒怒,掩歲星於鶉火,又晝見經天。鶉火,周分也。



 乾元元年四月,熒惑、鎮星、太白聚於營室。太史南宮沛奏;「其地戰不勝。」衛分也。



 大歷三年七月壬申,五星並出東方。占曰;「中國利。」八年閏十一月壬寅,太白、辰星合於危。齊分也。十年正月甲寅,歲星、熒惑合於南斗。占曰:「饑、旱。」吳、越分也。一曰:「不可用兵。」七月庚辰,太白、辰星合於柳。京師分也。



 建中二年六月,熒惑、太白鬥於東井。四年六月,熒惑、太白復鬥於東井。京師分也。金、火、罰星斗者,戰象也。



 興元元年春,熒惑守歲星在角、亢。占曰;「有反臣。」角、亢,鄭也。



 貞元四年五月乙亥,歲星、熒惑、鎮星聚於營室。占曰:「其國亡。」地在衛分。六年閏三月庚申,太白、辰星合於東井。占為兵憂。戊寅,熒惑犯鎮星在奎。魯分也。



 元和九年十月辛未,熒惑犯鎮星,又與太白合於女。在齊分。十年六月辛未,歲星、熒惑、太白、辰星合於東井。占曰:「中外相連以兵。」十一年五月丁卯,歲星、辰星合於東井」六月己未,復合於東井。占曰;「為變謀而更事。」十一月戊子,鎮星、熒惑合於虛、危。十二月,鎮星、太白、辰星聚於危。皆齊分也。十四年八月丁丑,歲星、太白、辰星聚於軫。占曰:「兵喪。」在楚分與南方夷貊之國。十五年三月,鎮星、太白合於奎。占曰:「內兵。」徐州分也。十二月,熒惑、鎮星合於奎。占曰:「主憂。」



 長慶二年二月甲戌,歲星、熒惑合於南斗。占曰:「饑、旱。」八月丙寅,熒惑犯鎮星在昴、畢,因留相守。占曰;「主憂。」四年八月庚辰,熒惑犯鎮星於東井,鎮星即失行犯鉞。而熒惑復往犯之。占曰:「內亂。」



 寶歷二年八月丁未,熒惑、鎮星復合於東井、輿鬼間。



 大和二年九月,歲星、熒惑、鎮星聚於七星。三年四月壬申,歲星犯鎮星。占曰:「饑。」四年五月丙午,歲星、太白合於東井。六年正月,太白、熒惑合於羽林。十月,太白、熒惑、鎮星聚於軫。八年七月庚寅,太白、熒惑合相犯,推歷度在翼,近太微。占曰:「兵起。」



 開成三年六月丁亥,太白犯熒惑於張。占曰:「有喪。」四年正月丁巳,熒惑、太白、辰星聚於南斗,推歷度在燕分。占曰;「內外兵喪,改立王公。」冬,歲星、熒惑俱逆行失色,合於東井。京師分也。



 會昌二年六月乙丑,熒惑犯歲星於翼。占曰;「旱。」四年十月癸未,太白、熒惑合於南斗。



 咸通中,熒惑、鎮星、太白、辰星聚於畢、昴,在趙,魏之分。詔鎮州王景崇被袞冕,軍府稱臣以厭之。



 文德元年八月,歲星、鎮星、太白聚於張,周分也。占曰:「內外有兵。」為河內、河東地。



 光化三年十月,太白、鎮星合於南斗。占曰:「吳、越有兵。」



\end{pinyinscope}