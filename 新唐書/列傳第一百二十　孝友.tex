\article{列傳第一百二十 孝友}

\begin{pinyinscope}

 唐受命二百八十八年,以孝悌名通朝廷者,多閭巷刺草之民,皆得書於史官。



 萬年王世貴,長安嚴待封,涇陽田伯明,華原韓難陀,華州王瞿曇,鄭縣辛法汪、郭士舉、張長、郭士度、鄭迪、柳仁忠、能君德、劉崇、甘元爽、韓子尚、韓思約,下邽張萬徹,朝邑申屠思恭、呂昂,鶉觚張元亮,靈臺孫智和,新平馮猛將,宜川司馬芬,洛交周崇俊,洛川何善宜,博陵崔定仁,冀州燕遺倩,貝州馬衡,滄州鄭士才,清池孫楚信、劉賢,渤海邊鳳舉,瀛州硃寶積,樂陵蘇伏念,邯鄲章征,雞澤馮仁海、郭守素,文安董相,武邑王達多、張丘感、張藝朗暨孫師才、張義節,沙河趙君惠,南樂穀感德,魏縣毛仁,武城茹智達,歷亭王師威、李肆仁,臨河李文綢,湯陰後斥奴,鼓城彭思義、陳屺、田堤岳,太原盧遺仁、王知道,蒲州賈孝才,解縣衛玄表,南嶽張利見,安邑曹文行、孫懷應、相里志降、楊王操、邵玄同、張衡、曹存勛、李文褒、董文海、李文秀、張仙兒、張公憲,虞鄉董敬直,河東張金城、呂神通、呂雲、呂志挺、呂元光、趙舉、張祐、姚熾、張師德、馮巨源、杜山藏,河西郭文政,伊闕任仲濟、源榮璧,汴州張士巖,陳留家師諒、董允恭,尉氏楊思貞,中牟潘良瑗暨子季通,陽武時惠珣,封丘楊嵩珪,許田李頤道,胙城蔡洪、石善雄暨孫彥威,朗山胡君才,徐州皇甫恆,彭城尹務榮,荊州劉寶,長壽史摶,益州焦懷肅、郭景華,郪縣曹少微,涪城趙煙,資陽趙光寓、黃昪,梓潼馬冬王、秦舉、王興嗣,依政樊漪,巴西韋士宗、文博熒暨子詮,南鄭李貞古,巢縣張進昭,萬載廖洪,南陵蘇仲方,鄱陽張言贊,樂平謝惟勤、沈普、姜崌,上饒鮑嘉福、虞熔真,句容張常洧,弋陽張球、李營暨子凝孫楚,貴溪黃舟,建昌熊士贍,臨江袁鳴,贛縣謝俊,餘杭何公弁、章成緬、方宗,建德何起門,桐廬祝希進,諸暨張萬和,蕭山李渭、許伯會、戴恭、俞僅,信安徐知新、徐惠諲,東陽應先、唐君祐,睦州許利川,建陽劉常,邵武黃亙、張巨篯、吳海,泉山黃嘉猷,永泰王奭,皆事親居喪著至行者。萬年宋興貴,奉先張郛,澧陽張仁興,櫟陽董思寵,湖城閻旻,高平雍仙高,湖城閻酆,正平周思藝、張子英,曲沃張君密、秦德方、馬玄操、李君則,太平趙德儼,隴西陳嗣,北海呂元簡,經城宋洸之,單父劉九江,無棣徐文亮,樂陵吳正表,河間劉宣、董永,安邑任君義、衛開,龍門梁神義、賀見涉、張奇異,鄭縣王元緒、寇元童,舒城徐行周,睦州方良琨,桐廬戴元益,高安宋練,涇縣萬晏,弋陽李植,繁昌王丕,皆數世同居者。天子皆旌表門閭,賜粟帛,州縣存問,復賦稅,有授以官者。



 唐時陳藏器著《本草拾遺》,謂人肉治羸疾,自是民間以父母疾,多刲股肉而進。又有京兆張阿九、趙言,奉天趙正言、滑清泌,羽林飛騎啖榮祿,鄭縣吳孝友,華陰尹義華,潞州張光玼,解縣南鍛,河東李忠孝、韓放,鄢陵任客奴,絳縣張子英,平原楊仙朝,樂工段日升,河東將陳涉,襄陽馮子,城固雍孫八,虞鄉張抱玉、骨英秀,榆次馮秀誠,封丘楊嵩珪、劉浩,清池硃庭玉、弟庭金,繁昌硃心存,歙縣黃芮,左千牛薛鋒及河陽劉士約,或給帛,或旌表門閭,皆名在國史。善乎!韓愈之論也,曰:「父母疾,亨藥餌,以是為孝,未聞毀支體者也。茍不傷義,則聖賢先眾而為之。是不幸因而且死,則毀傷滅絕之罪有歸矣,安可旌其門以表異之?」雖然,委巷之陋,非有學術禮義之資,能忘身以及其親,出於誠心,亦足稱者。故列十七八焉。廣明後,方鎮凌法,誇地千里,事不上聞,孝悌篤行之士,旌命所不及。載小說者,名字不參見他書,不可錄。若李知本、張志寬之屬,承上順下,有禮讓君子之風,故輯而序之。張士巖父病,藥須鯉魚,冬月冰合,有獺銜魚至前,得以供父,父遂愈。母病癰,士巖吮血。父亡,廬墓,有虎狼依之。焦懷肅母病,每嘗其唾,若味異,輒悲號幾絕。母終,水漿不入口五日,負土成墳,廬守,日一食,杖然後起。繼母沒,亦如之。張進昭,母患狐刺,左手墮而終。及殯,進昭截左腕廬於墓。張公藝九世同居,北齊東安王永樂、隋大使梁子恭躬慰撫,表其門。高宗有事太山,臨幸其居,問本末,書「忍」字以對,天子為流涕,賜縑帛而去。四人名頗著,詳見於篇。



 李知本,趙州元氏人,元魏洛州刺史靈六世孫。父孝端,仕隋為獲嘉丞。與族弟太沖俱有世閥,而太沖官婚最高,鄉人語曰:「太沖無兄,孝端無弟。」



 知本涉經術,事親篤至,與弟知隱雍順,子孫百餘,至貲用僮僕無間也。大業末,盜賊過閭不入,相戒曰:「無犯義門。」往依者五百餘室,皆以免。貞觀初,知隱為伊闕丞,知夏津令。開元中,孫瑱為給事中、揚州長史。知隱孫顒,有文辭,至太常少卿。從祖兄弟位給事中,凡四人。



 張志寬,蒲州安邑人。居父喪而毀,州里稱之。王君廓兵略地,不暴其閭,倚全者百許姓。後為里正,忽詣縣稱母疾求急,令問狀,對曰:「母有疾,志寬輒病,是以知之。」令謂其妄,系於獄,馳驗如言,乃慰遣之。母終,負土成墳,手蒔松柏。高祖遣使者就吊,拜員外散騎常侍,賜物四十段,表其閭。



 劉君良,瀛州饒陽人。四世同居,族兄弟猶同產也,門內鬥粟尺帛無所私。隋大業末,荒饉,妻勸其異居,因易置庭樹鳥雛,令斗且鳴,家人怪之,妻曰:「天下亂,禽鳥不相容,況人邪!」君良即與兄弟別處。月餘,密知其計,因斥去妻,曰:「爾破吾家!」召兄弟流涕以告,更復同居。天下亂,鄉人共依之,眾築為堡,因號義成堡。武德中,深州別駕楊弘業至其居,凡六院共一庖,子弟皆有禮節,嘆挹而去。貞觀六年,表異門閭。



 王少玄,博州聊城人。父隋末死亂兵,遺腹生少玄。甫十歲,問父所在,母以告,即哀泣求尸。時野中白骨覆壓,或曰:「以子血漬而滲者,父胔也。」少玄鑱膚,閱旬而獲,遂以葬。創甚,彌年乃興。貞觀中,州言狀,拜徐王府參軍。



 任敬臣,字希古,棣州人。五歲喪母,哀毀天至。七歲,問父英曰:「若何可以報母?」英曰:「揚名顯親可也。」乃刻志從學。汝南任處權見其文,驚曰:「孔子稱顏回之賢,以為弗如也。吾非古人,然見此兒,信不可及。」十六,刺史崔樞欲舉秀才,自以學未廣,遁去。又三年卒業,舉孝廉,授著作局正字。父亡,數殞絕,繼母曰:「而不勝喪,謂孝可乎?」敬臣更進饘粥。服除,遷秘書郎。休沐,闔門誦書。監虞世南器其人,歲終,書上考,固辭。召為弘文館學士,俄授越王府西閣祭酒。當代,王再表留,進朝請郎。舉制科,擢許王文學。復為弘文館學士,終太子舍人。



 支叔才,定州人。隋末荒饉,夜丐食野中,還進母,為賊執,欲殺之,告以情,賊閔其孝,為解縛。母病癰,叔才吮瘡注藥。及亡,廬墓,有白鵲止廬傍。高宗時,表異其家。



 至德間,有常州人王遇、弟遐俱為賊執,將釋一人,兄弟相讓死,賊感其意,盡縱之。



 程袁師,宋州人。母病,十旬不褫帶,藥不嘗不進。代弟戍洛州。母終,聞訃,日走二百里,因負土築墳,號臒,人不復識。改葬曾門以來,閱二十年乃畢。常有白狼、黃蛇馴墓左,每哭,群鳥鳴翔。永徽中,刺史狀諸朝,詔吏敦駕。既至,不願仕,授儒林郎,還之。



 武弘度,士畐兄之子,補相州司兵參軍。永徽中,父卒,自徐州被發徒跣趨喪所,負土築塋,晨夕號,日一溢米。素芝產廬前,貍擾其旁。高宗下詔褒美,旌其門。



 宋思禮,字過庭,事繼母徐為聞孝。補蕭縣主簿。會大旱,井池涸,母羸疾,非泉水不適口,思禮憂懼且禱,忽有泉出諸庭,味甘寒,日不乏汲。縣人異之,尉柳晃為刻石頌其孝感。



 鄭潛曜者,父萬鈞,駙馬都尉、滎陽郡公。母,代國長公主。開元中,主寢疾,潛曜侍左右,造次不去,累三月不靧面。主疾侵,刺血為書請諸神,丐以身代。火書,而「神許」二字獨不化。翌日主愈,戒左右無敢言。後尚臨晉長公主,歷太僕光祿卿。



 元讓,雍州武功人。擢明經,以母病不肯調,侍膳不出閭數十年。母終,廬墓次,廢櫛沐,飯菜飲水。咸亨中,太子監國,下令表闕於門。永淳初,巡察使表讓孝悌卓越,擢太子右內率府長史。歲滿,還鄉里,人有所訟,皆詣讓判。中宗在東宮,召拜司議郎,入謁,武后望謂曰:「卿孝於家,必能忠於國,宜以治道輔吾子。」尋卒。



 裴敬彞,絳州聞喜人。曾祖子通,隋開皇中以太中大夫居母喪,哭喪明,有白烏巢塚樾。兄弟八人皆為名孝,詔表門闕,世謂「義門裴氏」。



 敬彞七歲能文章,性謹敏,宗族重之,號「甘露頂」。父智周,補臨黃令,為下所訟。敬彞年十四,詣巡察使唐臨直枉,臨奇之,試命作賦,賦工。父罪已釋,表敬彞於朝,補陳王府典簽。一日,忽泣涕謂左右曰:「大人病痛,吾輒然,今心悸而痛,事叵測。」乃請急,倍道歸,而父已卒,羸毀逾禮。乾封初,遷累監察御史。母病,醫許仁則者鐍不能乘,敬彞自為輿往迎。既居喪,詔贈縑帛,官為作靈輿。終服,以著作郎兼修國史。歷中書舍人、太子左庶子。武后時,為酷吏所陷,死嶺南。



 梁文貞,虢州閺鄉人。少從軍守邊,逮還,親已亡。自傷不得養,即穿壙為門,晨夕汛掃,廬墓左,喑默三十年,家人有所問,畫文以對。會官改新道,出文貞廬前,行旅見之,皆為流涕。有甘露降塋木,白兔馴擾,縣令刊石紀之。開元中,刺史許景先表文貞孝絕倫類,詔付史官。



 沈季詮,字子平,洪州豫章人。少孤,事母孝,未嘗與人爭,皆以為怯。季詮曰:「吾怯乎?為人子者,可遺憂於親乎哉!」貞觀中,侍母度江,遇暴風,母溺死,季詮號呼投江中,少選,持母臂浮出水上。都督謝叔方具禮祭而葬之。



 許伯會,越州蕭山人。或曰玄度十二世孫。舉孝廉。上元中,為衡陽博士。母喪,負土成墳,不御絮帛、嘗滋味。野火將逮塋樹,悲號於天,俄而雨,火滅。歲旱,泉湧廬前,靈芝生。



 陳集原,瀧州開陽人。世為酋長。父龍樹,為欽州刺史,有疾,即集原輒不食。及亡,嘔血數升,即塋作廬,盡以田貲讓兄弟,里人高之。武后時,歷右豹韜衛大將軍。



 陸南金,蘇州吳人。祖士季,從同郡顧野王學《左氏春秋》、《司馬史》、《班氏漢書》。仕隋為越王侗記室兼侍讀。侗稱制,擢著作郎。時王世充將篡逆,侗謂士季曰:「隋有天下三十年,朝果無忠臣乎?」士季對曰:「見危授命,臣宿志也。請因啟事為陛下殺之。」謀洩,停侍讀,乃不克。貞觀初,終太學博士兼弘文館學士。



 南金仕為太常奉禮郎。開元初,少卿廬崇道抵罪徙嶺南,逃還東都。南金居母喪,崇道偽稱吊客,入而道其情,南金匿之。俄為仇人跡告,詔侍御史王旭捕按,南金當重法,弟趙璧詣旭自言:「匿崇道者我也,請死。」南金固言弟自誣不情,旭怪之,趙璧曰:「母未葬,妹未歸,兄能辦之,我生無益,不如死。」旭驚,上狀。玄宗皆宥之。



 南金知書史,履操謹完。張說、陸象先以賢謂之,由庫部員外以痼疾改太子洗馬,卒。



 張琇,河中解人。父審素,為巂州都督,有陳纂仁者,誣其冒戰級、私庸兵。玄宗疑之,詔監察御史楊汪即按。纂仁復告審素與總管董堂禮謀反。於是汪收審素系雅州獄,馳至巂州按反狀。堂禮不勝忿,殺纂仁,以兵七百圍汪,脅使露章雪審素罪。既而吏共斬堂禮,汪得出,遂當審素實反,斬之,沒其家。琇與兄瑝尚幼,徙嶺南。久之,逃還。汪更名萬頃。瑝時年十三,琇少二歲。夜狙萬頃於魏王池,瑝斫其馬,萬頃驚,不及斗,為琇所殺。條所以殺萬頃狀系於斧,奔江南,將殺構父罪者,然後詣有司。道汜水,吏捕以聞。中書令張九齡等皆稱其孝烈,宜貸死,侍中裴耀卿等陳不可,帝亦謂然,謂九齡曰:「孝子者,義不顧命。殺之可成其志,赦之則虧律。凡為子,孰不願孝?轉相仇殺,遂無已時。」卒用耀卿議,議者以為冤。帝下詔申諭,乃殺之。臨刑賜食,瑝不能進,琇色自如,曰:「下見先人,復何恨!」人莫不閔之,為誄揭於道,斂錢為葬北邙,尚恐仇人發之,作疑塚,使不知其處。



 太宗時,有即墨人王君操,父隋末為鄉人李君則所殺,亡命去,時君操尚幼。至貞觀時,朝世更易,而君操窶孤,仇家無所憚,詣州自言。君操密挾刃殺之,剔其心肝啖立盡,趨告刺史曰:「父死兇手,歷二十年不克報,乃今刷憤,願歸死有司。」州上狀,帝為貸死。



 高宗時,絳州人趙師舉父為人殺,師舉幼,母改嫁,仇家不疑。師舉長,為人庸,夜讀書。久之,手殺仇人,詣官自陳,帝原之。



 永徽初,同官人同蹄智壽父為族人所害,智壽與弟智爽候諸塗,擊殺之,相率歸有司爭為首,有司不能決者三年。或言弟始謀,乃論死,臨刑曰:「仇已報,死不恨。」智壽自投地委頓,身無完膚,舐智爽血盡乃已,見者傷之。



 武后時,下邽人徐元慶父爽為縣尉趙師韞所殺,元慶變姓名為驛家保。久之,師韞以御史舍亭下,元慶手殺之,自囚詣官。後欲赦死,左拾遺陳子昂議曰:



 先王立禮以進人,明罰以齊政。枕干仇敵,人子義也;誅罪禁亂,王政綱也。然無義不可訓人,亂綱不可明法。聖人修禮治內,飭法防外,使守法者不以禮廢刑,居禮者不以法傷義,然後暴亂銷,廉恥興,天下所以直道而行也。



 元慶報父仇,束身歸罪,雖古烈士何以加?然殺人者死,畫一之制也,法不可二,元慶宜伏辜。《傳》曰:「父仇不同天。」勸人之教也。教之不茍,元慶宜赦。



 臣聞刑所以生,遏亂也;仁所以利,崇德也。今報父之仇,非亂也;行子之道,仁也。仁而無利,與同亂誅,是曰能刑,未可以訓。然則邪由正生,治必亂作,故禮防不勝,先王以制刑也。今義元慶之節,則廢刑也。跡元慶所以能義動天下,以其忘生而趨其德也。若釋罪以利其生,是奪其德,虧其義,非所謂殺身成仁、全死忘生之節。臣謂宜正國之典,寘之以刑,然後旌閭墓可也。



 時韙其言。後禮部員外郎柳宗元駁曰:



 禮之大本,以防亂也。若曰:無為賊虐,凡為子者殺無赦。刑之大本,亦以防亂也。若曰:無為賊虐,凡為治者殺無赦。其本則合,其用則異。旌與誅,不得並也。誅其可旌,茲謂濫,黷刑甚矣;旌其可誅,茲謂僭,壞禮甚矣。



 若師韞獨以私怨,奮吏氣,虐非辜,州牧不知罪,刑官不知問,上下蒙冒,號不聞。而元慶能處心積慮以沖仇人之胸,介然自克,即死無憾,是守禮而行義也。執事者宜有慚色,將謝之不暇,而又何誅焉?



 其或父不免於罪,師韞之誅,不愆於法,是非死於吏也,是死於法也。法其可仇乎?仇天子之法,而戕奉法之吏,是悖驁而凌上也。執而誅之,所以正邦典,而又何旌焉?



 禮之所謂仇者,冤抑沈痛而號無告也,非謂抵罪觸法,陷於大戮,而曰彼殺之我乃殺之,不議曲直,暴寡脅弱而已。《春秋傳》曰:「父不受誅,子復仇可也;父受誅,子復仇,此推刃之道。復仇不除害。」今若取此以斷兩下相殺,則合於禮矣。



 且夫不忘仇,孝也;不愛死,義也。元慶能不越於禮,服孝死義,是必達理而聞道者也。夫達理聞道之人,豈其以王法為敵仇者哉!議者反以為戮,黷刑壞禮,其不可以為典明矣。請下臣議附於令,有斷斯獄者,不宜以前議從事。



 憲宗時,衢州人餘常安父、叔皆為里人謝全所殺。常安八歲,已能謀復仇。十有七年,卒殺全。刺史元錫奏輕比,刑部尚書李鄘執不可,卒抵死。



 又富平人梁悅父為秦果所殺,悅殺仇,詣縣請罪。詔曰:「在《禮》父仇不同天,而法殺人必死。禮、法,王教大端也,二說異焉。下尚書省議。」職方員外郎韓愈曰:



 子復父仇,見於《春秋》、於《禮記》、《周官》,若子史,不勝數,未有非而罪者。最宜詳於律,而律無條,非闕文也。蓋以為不許復仇,則傷孝子之心;許復仇,則人將倚法顓殺,無以禁止。夫律雖本於聖人,然執而行之者,有司也。經之所明者,制有司者也。丁寧其義於經而深沒其文於律者,將使法吏一斷於法,而經術之士得引經以議也。



 《周官》曰:「凡殺人而義者,令勿仇,仇之則死。」義者,宜也。明殺人而不得其宜者,子得復仇也。此百姓之相仇者也。公羊子曰:「父不受誅,子復仇可也。」不受誅者,罪不當誅也。誅者,上施下之辭,非百姓相殺也。《周官》曰:「凡報仇讎者,書於士,殺之無罪。」言將復仇,必先言於官,則無罪也。



 復仇之名雖同,而其事各異。或百姓相仇,如《周官》所稱,可議於今者;或為官吏所誅,如《公羊》所稱,不可行於今者。《周官》所稱:將復仇先告於士,若孤稚羸弱,抱微志而伺敵人之便,恐不能自言,未可以為斷於今也。然則殺之與赦不可一,宜定其制曰:「有復父仇者,事發,具其事下尚書省,集議以聞,酌處之。」則經無失指矣。



 有詔以悅申冤,請罪詣公門,流循州。



 穆宗世,京兆人康買得,年十四,父憲責錢於雲陽張蒞,蒞醉,拉憲危死。買得以蒞趫悍,度救不足解,則舉鍤擊其首,三日蒞死。刑部侍郎孫革建言:「買得救父難不為暴,度不解而擊不為兇。先王制刑,必先父子之親。《春秋》原心定罪,《周書》諸罰有權。買得孝性天至,宜賜矜宥。」有詔減死。



 侯知道、程俱羅者,靈州靈武人。居親喪,穿壙作塚,皆身執其勞,鄉人助者,即哭而卻之。廬墳次,哭泣無節,知道七年、俱羅三年不止。知道垢塵積首,率夜半傅墳,踴而哭,鳥獸為悲號。李華作《二孝贊》表其行,曰:「厥初生人,有君有親。孝親為子,忠君為臣。兆自天命,降及人倫。背死不義,忘生不仁。過及智就,為之禮文。至哉侯氏,創巨病殷。手足胼胝,以成高墳。夜黑飆動,如臨鬼神。哭無常聲,迥徹蒼旻。苴斬三年,爾獨終身。嗟嗟程生,其哀也均。顧後絕配,瞻前無鄰。」



 又有何澄粹者,池州人。親病日錮,俗尚鬼,病者不進藥。澄粹剔股肉進,親疾為瘳。後親沒,伏於墓,哭踴無數,以毀卒,當時號「青陽孝子」,士為作誄甚眾。



 壽州安豐李興亦有至行,柳宗元為作《孝門銘》,曰:「壽州刺史臣承思言:『九月丁亥,安豐令上所部編戶氓興,父被惡疾,歲月就亟,興自刃股肉,假托饋獻,父老病已不能啖,宿而死。興號呼撫臆,口鼻垂血,捧土就墳,沾漬涕洟。墳左作小廬,蒙以苫茨,伏匿其中,扶服頓踴,晝夜哭訴。孝誠幽達,神為見異,廬上產紫芝、白芝,廬中醴泉湧。此皆陛下孝治神化,陰中其心,而克致斯事。謹按興匹庶賤陋,循習淺下,性非文字所導,生與耨耒為業,而能鐘彼醇孝,超出古烈,天意神道,猶錫瑞物以表殊異。伏惟陛下有唐堯如神之德,宜加旌褒,合於上下。請表其里閭,刻石明白,宣延風美,觀示後祀,永無極。臣昧死請。』制曰可。銘曰:『懿厥孝思,茲惟淑靈。稟承粹和,篤守天經。泣侍羸疾,默禱隱冥。引刃自向,殘肌敗形。羞膳奉進,憂勞孝誠。惟時高高,曾不視聽。創巨痛仍,號於穹旻。捧土濡涕,頓首成墳。搯膺腐眥,寒暑在廬。草木悴死,鳥獸踟躕。殊類異族,亦相其哀。肇有二位,孝道爰興。克脩厥猷,載籍是登。在帝有虞,以孝烝烝。仲尼述經,以教於曾。惟昔魯侯,見命夷宮。亦有考叔,寤莊稱純。顯顯李氏,實與之倫。哀嗟道路,涕慕裡鄰。神錫秘祉,三秀靈泉。帝命薦加,亦表其門。統一上下,交贊天人。建此碑號,億齡揚芬。』」



 許法慎,滄州清池人。甫三歲,已有知。時母病,不飲乳,慘慘有憂色。或以珍餌詭悅之,輒不食,還以進母。後親喪,常廬於塋,有甘露、嘉禾、靈芝、木連理、白兔之祥。天寶中,表異其閭。



 林攢,泉州莆田人。貞元初,仕為福唐尉。母羸老,未及迎而病。攢聞,棄官還。及母亡,水漿不入口五日。自埏甓作塚,廬其右,有白烏來,甘露降。觀察使李若初遣官屬驗實,會露晞,里人失色,攢哭曰:「天所降露,禍我邪?」俄而露復集,烏亦回翔。詔作二闕於母墓前,又表其閭,蠲徭役,時號「闕下林家」。



 陳饒奴,饒州人。年十二,親並亡,窶弱居喪,又歲饑,或教其分弟妹,可全性命。饒奴流涕,身丐訴相全養。刺史李復異之,給資儲,署其門曰「孝友童子」。



 王博武,許州人。會昌中,侍母至廣州,及沙湧口,暴風,母溺死,博武自投於水。嶺南節度使盧貞俾吏沈罟,獲二尸焉,乃葬之,表其墓曰「孝子墓」。詔為刻石。



 萬敬儒,廬州人。三世同居,喪親,廬墓,刺血寫浮屠書,斷手二指,輒復生。州改所居曰成孝鄉廣孝聚。大中時,表其家。



 章全益,梓州涪城人。少孤,為兄全啟所鞠。母病,全啟刲股膳母而愈。及全啟亡,全益服斬衰,斷手一指以報。不畜妻,僮僕處一室,賣藥自業,世傳能作黃金。居成都四十年,號章孝子,卒,年九十八。



 贊曰:聖人治天下有道,曰「要在孝悌而已」。父父也,子子也,兄兄也,弟弟也,推而之國,國而之天下,建一善而百行從,其失則以法繩之。故曰:「孝者天下大本,法其末也。」至匹夫單人,行孝一概,而兇盜不敢凌,天子喟而旌之者,以其教孝而求忠也。故裒而著於篇。



\end{pinyinscope}