\article{列傳第一百二十二 循吏}

\begin{pinyinscope}

 治者,君也;求所以治者,民也;推君之治而濟之民,吏也。故吏良則法平政成,不良則王道馳而敗矣。在堯、舜時抽象的「本質世界」,它是最高的實在領域,物理世界依存於,曰「九德咸事」也,「百工惟時」也;在周文、武時,曰「《棫樸》,能官人也」,「《南山有臺》,樂得賢也」;是循吏之效也。堯、舜,五帝之盛帝,文、武,三王之顯王,不能去是而治,後世可乎哉?



 唐興,承隋亂離,祓荒荼,始擇用州刺史、縣令。太宗嘗曰:「朕思天下事,丙夜不安枕,永惟治人之本,莫重刺史,故錄姓名於屏風,臥興對之,得才否狀,輒疏之下方,以擬廢置。」又詔內外官五品以上舉任縣令者。於是官得其人,民去嘆愁、就妥安。都督、刺史,其職察州縣,間遣使者循行天下,劾舉不職。始,都督、刺史皆天子臨軒冊授。後不復冊,然猶受命日對便殿,賜衣物,乃遣。玄宗開元時,已辭,仍詣側門候進止,所以光寵守臣,以責其功。初,刺史準京官得佩魚,品卑者假緋、魚。開元中,又錮廢酷吏,懲無良,群臣化之,革苛嬈之風,爭以惠利顯。復詔:三省侍郎缺,擇嘗任刺史者;郎官缺,擇嘗任縣令者。至宰相名臣,莫不孜孜言長人不可輕授亟易。是以授受之間,雖不能皆善,而所得十五。故協氣嘉生,薰為太平,垂祀三百,與漢相埒。致之之術,非循吏謂何?故條次治宜,以著厥庸。若將相大臣兼以勛閥著者,名見本篇,不列於茲。



 韋仁壽,京兆萬年人。隋大業末,為蜀郡司法書佐,斷獄平,得罪者皆自以韋君所論,死無恨。高祖入關,遣使者徇定蜀,承制擢仁壽巂州都督府長史。南寧州納款,朝廷歲遣使撫接,至率貪沓,邊人苦之,多叛去。帝素聞仁壽治理,詔檢校南寧州都督,寄治越巂,詔歲一按行尉勞。仁壽將兵五百人,循西洱河,開地數千里,稱詔置七州十五縣,酋豪皆來賓見,即授以牧宰,威令簡嚴,人人安悅。將還,酋長泣曰:「天子藉公鎮撫,奈何欲去我?」仁壽以池壁未立為解,諸酋即相率築城起廨,甫旬略具。仁壽乃告以實曰:「吾奉詔第撫循,庸敢擅留?」夷夏父老乃悲啼祖行,遣子弟隨貢方物,天子大悅。仁壽請徙治南寧州,假兵遂撫定,詔可,敕益州給兵護送。刺史竇軌疾其功,訹言山獠方叛,未可以遠略,不時遣。歲餘,卒。



 陳君賓,陳鄱陽王伯山子也。仕隋為襄國通守。武德初,挈郡聽命,封東陽郡公,遷邢州刺史。貞觀初,徙鄧州。州承喪亂後,百姓流冗,君賓加意勞徠,不期月,皆還自業。明年,四方霜潦,獨君賓所治有年,儲倉充羨,蒲、虞二州民就食其境。太宗下詔勞之曰:「去年關內六州穀不登,餱糧少,令析民房逐食。聞刺史與百姓識朕此懷,務相安養,還有贏糧,出布帛贈遺行者。此知水旱常數,更相拯贍,禮讓興行,海內之人皆為兄弟,變澆薄之風,朕顧何憂?已命有司錄刺史以下功最;百姓養戶,免今年調物。」是歲,入為太府少卿,轉少府少監,坐事免。起為虔州刺史,卒。



 張允濟,青州北海人。仕隋為武陽令,以愛利為行。元武民以牸牛依婦家者,久之,孳十餘犢,將歸,而婦家不與牛。民訴縣,縣不能決,乃詣允濟,允濟曰:「若自有令,吾何與為?」民泣訴其抑,允濟因令左右縛民,蒙其首,過婦家,云捕盜牛者,命盡出民家牛,質所來,婦家不知,遽曰:「此婿家牛,我無豫。」即遣左右撤蒙,曰:「可以此牛還婿家。」婦家叩頭服罪,元武吏大慚。允濟過道旁,有姥廬守所蒔蔥,因教曰:「第還舍,脫有盜,當告令。」姥謝歸。俄大亡蔥,允濟召十里內男女盡至,物色驗之,果得盜者。有行人夜發,遺袍道中,行十餘里乃寤,人曰:「吾境未嘗拾遺,可還取之。」既而得袍。舉政尤異,遷高陽郡丞,郡缺太守,獨統郡事,吏下畏悅。賊帥王須拔攻郡,於是糧屈,吏食槐葉槁節,無叛者。貞觀初,累遷刑部侍郎,封武城縣男,擢幽州刺史,卒。



 時又有李桐客者,亦以治稱。初仕隋,為門下錄事。煬帝在江都,以四方日亂,謀徙都丹陽,召群臣議。左右希意,以為江左且望幸,若巡狩勒石紀功,復禹舊跡,顧不其然。桐客獨曰:「吳會卑濕而〓,不足奉萬乘、給三軍,吳人力屈,無以堪命,且逾越險阻,非社稷福。御史劾以訕毀,幾得罪而免。為宇文化及脅,將至黎陽,又陷竇建德。賊平,授秦王府法曹參軍。貞觀初,累為通、巴二州刺史,治尚清平,民呼為慈父。桐客,冀州衡水人。



 李素立,趙州高邑人。曾祖義深,仕北齊為梁州刺史。父政藻,為隋水部郎,使淮南,死於盜。素立仕武德初,擢監察御史。民犯法不及死,高祖欲殺之,素立諫曰:「三尺法,天下所共有,一動搖,則人無以措手足。方大業經始,奈何輦轂下先棄刑書鴻煒」帝嘉納,由是恩顧特異。以親喪解官,起授七品清要,有司擬雍州司戶參軍,帝曰:「要而不清。」復擬秘書郎,帝曰:「清而不要。」乃授侍御史。貞觀中,轉揚州大都督府司馬。



 初,突厥鐵勒部內附,即其地為瀚海都護府,詔素立領之。於是,闕泥熟別部數梗邊,素立以不足用兵,遣使諭降,夷人感其惠,率馬牛以獻,素立止受酒一杯,歸其餘。乃開屯田,立署次,虜益畏威。歷太僕、鴻臚卿,累封高邑縣侯。出為綿州刺史。永徽初,徙蒲州,將行,還所餘儲籺並什器於州,齎家書就道。會卒,高宗特廢朝一日,謚曰平。



 孫至遠,始名鵬。而素立方奉使,謂家人曰:「古有待事名子,吾此役可命子孫矣。」遂以名之。少秀晤,能治《尚書》、《左氏春秋》,未見杜預《釋例》而作《編記》,大趣略同。復撰《周書》,起後稷至赧,為傳紀,令狐德棻許其良史。始調蒲州參軍,累補乾封尉。上元時,制策高第,授明堂主簿。以喪解官,既除,調鴻臚主簿。奏戎狄簿領,高宗悅,擢監察御史裏行。忤貴幸,外遷,久乃歷司勛、吏部員外郎中。遷天官侍郎,知選事,疾令史受賄謝,多所絀易,吏肅然斂手。有王忠者,被放,吏謬書其姓為「士」,欲擬訖增成之,至遠曰:「調者三萬,無士姓,此必王忠。」吏叩頭服罪。至遠之知選,以內史李昭德進,人或勸其往謝,答曰:「公以公用我,奈何欲謝以私?」卒不詣。故昭德銜之,出為壁州刺史。卒,年四十八。



 至遠父休烈,亦有文,終郪令,年四十九。世嘆其父子材不盡雲。至遠見桓彥範,力言其賢。盧從願尚少,高以評目。許弟從遠且貴,豫言其位,以驗所至。蘇頲,其出也,少失母,至遠愛視甚謹,以女妻之。友兄弟,事寡姊有禮,世稱其德。



 從遠清密有學,神龍初,歷中書令、太府卿,累封趙郡公,謚曰懿。兄弟皆德望相埒。又從父游道,武后時冬官尚書、同鳳閣鸞臺三品。



 至遠子畬,字玉田,少聰警。初歷汜水主簿,遇事蜂銳,雖廝豎,一閱輒記姓名、居業。黜陟使路敬潛薦其清白,擢右臺監察御史裏行。臺廢,授監察御史,累轉國子司業。事母謹,累世同居,長幼有禮。畬妻物故,時母病,恐悲傷,約家人無以哭聞母所,朝夕省侍無憂色。母終,毀而卒。



 從遠子巖,年十餘歲,會中宗祀明堂,以近臣子弟執籩豆,巖進止中禮,授右宗衛兵曹參軍。歷洛陽尉,累遷兵部郎中。發扶風兵應姚、巂,稱旨,遷諫議大夫,封贊皇縣伯。終兵部侍郎。巖善草隸。為參軍時制一裘,服終身。



 薛大鼎,字重臣,蒲州汾陰人。父粹,為隋介州長史,與漢王諒同反,誅。大鼎貰為官奴,流辰州,用戰功得還。高祖兵興,謁見龍門,因說帝絕龍門,軍永豐倉就食,傳檄遠近,據天府,示豪桀,為拊背扼喉計,帝奇之。時諸將已決策先攻河東,故議置。授大將軍府察非掾。出為山南道副大使,開屯田以實倉廩。趙郡王孝恭討輔公祏,以大鼎為饒州道軍師,引兵度彭蠡湖,以功遷浩州刺史。累徙滄州。無棣渠久廞塞,大鼎浚治屬之海,商賈流行,里民歌曰:「新溝通,舟楫利。屬滄海,魚鹽至。昔徒行,今騁駟。美哉薛公德滂被!」又疏長蘆、漳、衡三渠,洩污潦,水不為害。是時,鄭德本在瀛州,賈敦頤為冀州,皆有治名故河北稱「鐺腳刺史」。永徽中,遷銀青光祿大夫,行荊州大都督長史。卒,謚曰恭。



 子克構,有器識,永隆初,歷戶部郎中。族人黃門侍郎顗,以弟紹尚太平公主,問於克構,答曰:「室有傲婦,善士所惡。夫惟淑德,以配君子,無患可矣。」顗不敢沮,而紹卒誅。陳思忠居父喪,詔奪服,客往吊,思忠辭以辰日不見。克構曰:「事親者,避嫌可也;既孤矣,則無不哭。」世服其言。天授中,遷麟臺監。坐弟為酷吏所陷,流死嶺南。



 賈敦頤,曹州冤句人。貞觀時,數歷州刺史,資廉潔。入朝,常盡室行,車一乘,敝甚,羸馬繩羈,道上不知其刺史也。久之,為洛州司馬,以公累下獄,太宗貰之,有司執不貰,帝曰:「人孰無過,吾去太甚者。若悉繩以法,雖子不得於父,況臣得事其君乎?」遂獲原。徙瀛州刺史,州瀕滹沱、滱二水,歲湓溢,壞室廬,浸洳數百里。敦頤為立堰庸,水不能暴,百姓利之。時弟敦實為饒陽令,政清靜,吏民嘉美。舊制,大功之嫌不連官,朝廷以其兄弟治行相高,故不徙以示寵。永徽中,遷洛州。洛多豪右,占田類逾制,敦頤舉沒者三千餘頃,以賦貧民,發奸擿伏,下無能欺。卒於官。



 咸亨初,敦實為洛州長史,亦寬惠,人心懷向。洛陽令楊德乾矜酷烈,杖殺人以立威,敦實喻止,曰:「政在養人,傷生過多,雖能,不足貴也。」德乾為衰減。始,洛人為敦頤刻碑大市旁,及敦實入為太子右庶子,人復為立碑其側,故號「常棣碑」。歷懷州刺史,有美跡。永淳初致仕,病篤,子孫迎醫,敦實不肯見,曰:「未聞良醫能治老也。」卒,年九十餘。子膺福,左散騎常侍、昭文館學士,以竇懷貞黨誅。



 德干歷澤、齊、汴、相四州刺史,有威嚴#急語曰:「寧食三斗炭,不逢楊德乾。」天授初,子神讓與徐敬業起兵,皆及誅。



 田仁會,雍州長安人。祖軌,隋幽州刺史,封信都郡公。父弘襲封,至陵州刺史。仁會擢制舉,仕累左武候中郎將。太宗征遼東,而薛延陀以數萬騎掩河內,詔仁會與執失思力率兵擊敗之,尾逐數百里,延陀幾生得,璽書嘉尉。永徽中,為平州刺史,歲旱,自暴以祈,而雨大至,谷遂登。人歌曰:「父母育我兮田使君,挺精誠兮上天聞,中田致雨兮山出雲,倉廩實兮禮義申,願君常在兮不患貧。」五遷勝州都督,境有夙賊,依山剽行人,仁會發騎捕格,夷之。城門夜開,道無寇跡。入為太府少卿,遷右金吾將軍。所得祿,估有贏,輒入之官,人以為尚名。然資強摯疾惡,晝夜循行,有絲毫奸必發,廷中謫罰日數百,京師無貴賤舉憚之。有女巫傳鬼道惑眾,自言能活死人,市里尊神,仁會劾徙於邊。轉右衛將軍,以年老乞骸骨。卒,年七十八,謚曰威。



 子歸道,明經及第,累擢通事舍人內供奉、左衛郎將。突厥默啜請和,武后詔將軍閻知微冊可汗號,持節往。默啜又遣使謝,知微遇諸道,即與緋袍銀帶,因表使者即到,請備禮廷賜。歸道諫曰:「虜背惠積年,今悔過入朝,解辮削衽宜待天旨。而知微擅賜,使朝廷何以加之?宜敕初服,須天子命。小國使者,不足備禮迓之。」後從焉。默啜將至單于都護府,詔歸道攝司賓卿往勞。默啜請六胡州及都護府地不得,大怨望,執歸道將害之。歸道色不撓,詈且讓,為陳禍福,默啜亦悔。會有詔賜默啜粟三萬石,彩五萬段,農器三千,且許結婚,於是更以禮遣歸道。既還,具陳默啜不臣狀,請備邊。已而果反,乃擢歸道夏官侍郎,益親信。



 遷左金吾將軍、司膳卿,押千騎宿衛玄武門。桓彥範等誅二張,而歸道不豫聞,及索騎士,拒不應。事平,彥範欲誅之,以辭直,免,還私第。然中宗壯其守,召拜太僕少卿,遷殿中少監、右金吾將軍。卒,贈輔國大將軍,追封原國公,謚曰烈,帝自為文以祭。



 子賓庭,開元時至光祿卿。



 裴懷古,壽州壽春人。儀鳳中,上書闕下,補下邽主簿,遷監察御史。姚、巂道蠻反,命懷古馳驛往懷輯之,申明誅賞,歸者日千計。俄縛首惡,遂定南方,蠻夏立石著功。恆州浮屠為其徒誣告祝詛不道,武后怒,命按誅之。懷古得其枉,為後申訴,不聽,因曰:「陛下法與天下畫一,豈使臣殺無辜以希盛旨哉?即其人有不臣狀,臣何情寬之?」後意解,得不誅。



 閻知微之使突厥,懷古監其軍。默啜脅知微稱可汗,又欲官懷古,不肯拜,將殺之。辭曰:「守忠而死與毀節以生孰愈?請就斬,不避也。」遂囚軍中,因得亡,而素尪弱,不能騎,宛轉山谷間,僅達並州。時長史武重規縱暴,左右妄殺人取賞,見懷古至,急執之。有果毅嘗識懷古,疾呼曰:「裴御史也。」遂免。遷祠部員外郎。



 姚、巂酋等叩闕下,願得懷古鎮安遠夷,拜姚州都督,以疾辭。始安賊歐陽倩眾數萬,剽沒州縣,以懷古為桂州都督招尉討擊使,未逾嶺,逆以書諭禍福,賊迎降,自陳為吏侵而反。懷古知其誠,以為示不疑,可破其謀,乃輕騎赴之。或曰:「獠夷難親,備之且不信,況易之哉!」答曰:「忠信可通神明,況裔人耶!」身至壁撫諭,倩等大喜,悉歸所掠出降,雖諸洞素翻覆者,亦牽連根附,嶺外平。



 徙相州刺史、並州大都督長史,所至吏民懷愛。神龍中,召為左羽林大將軍,未至官,還為並州。人知其還,攜扶老稚出迎。崔宣道始代為長史,亦野次。懷古不欲厚愧宣道,使人驅迎者還,而來者愈眾,得人心類如此。俄轉幽州都督,綏懷兩蕃,將舉落內屬,會以左威衛大將軍召,而孫佺代之,而佺不知兵,遂敗其師。卒於官。



 懷古清介審慎,在幽州時,韓琬以監察御史監軍,稱其「馭士信,臨財廉,國名將」云。



 韋景駿,司農少卿弘機孫。中明經。神龍中,歷肥鄉令。縣北瀕漳,連年泛溢,人苦之。舊防迫漕渠,雖峭岸,隨即壞決。景駿相地勢,益南千步,因高築鄣,水至堤趾輒去,其北燥為腴田。又維艚以梁其上,而廢長橋,功少費約,後遂為法。方河北饑,身巡閭里,勸人通有無,教導撫循,縣民獨免流散。及去,人立石著其功。後為貴鄉令,有母子相訟者,景駿曰:「令少不天,常自痛。爾幸有親,而忘孝邪?教之不孚,令之罪也。」因嗚咽流涕,付授《孝經》,使習大義。於是母子感悟,請自新,遂為孝子。當時治有名者:景駿與清漳令馮元淑、臨洺令楊茂謙三人。



 景駿後數年為趙州長史,道出肥鄉,民喜,爭奉酒食迎犒,有小兒亦在中。景駿曰:「方兒曹未生,而吾去邑,非有舊恩,何故來?」對曰:「耆老為我言,學廬、館舍、橋鄣皆公所治,意公為古人,今幸親見,所以來。」景駿為留終日。後遷房州刺史。州窮險,有蠻夷風,無學校,好祀淫鬼,景駿為諸生貢舉,通隘道,作傳舍,罷祠房無名者。景駿之治民,求所以便之,類如此。。轉奉天令,未行,卒。



 茂謙擢制舉,授左拾遺內供奉,為吏介而勤,歷秘書郎。始竇懷貞雅重其材,及執政,薦為大理正、左臺御史中丞。開元初,出為魏州刺史、河北道按察使。與司馬張懷玉同鄉,長相善,洎晚有隙,掉訐短長,左遷桂州都督。徙廣州。卒。



 景駿子述,自有傳。



 李惠登,營州柳城人,為平盧軍裨將。安祿山亂,從董秦泛海,略定滄、棣等州。輕兵遠鬥,賊不支,戰輒北。史思明反,惠登陷賊,以計挺身走山南,依來瑱,表試金吾衛將軍。李希烈反,屬以兵二千,使屯隋州,惠登挈州以歸,即拜刺史。州數被亂,野如藝,人無處業。惠登雖樸素無學術,而視人所謂利者行之,所謂害者去之,率心所安,暗與古合。政清靜,居二十年,田畝闢,戶口日增,人歌舞之。於是節度使于頔狀其績,詔加御史大夫,升隋為上州。俄檢校國子祭酒,卒,贈洪州都督。



 羅珦,越州會稽人。寶應初,詣闕上書,授太常寺太祝。曹王皋領江西、荊襄節度使,常署幕府,遷累副使。皋卒,軍亂,劫府軍,珦取首惡十餘人斬以徇,環棘廷中,俾投所劫庫物,一日皆滿,乃貰餘黨。召為奉天令。中官出入系道,吏緣以犯禁,珦搒笞之,雖死不置,自是屏息。擢廬州刺史。民間病者,舍醫藥,禱淫祀,珦下令止之。修學宮,政教簡易,有芝草、白雀。淮南節度使杜佑上治狀,賜金紫服。再遷京兆尹,請減平糴半,以常賦充之,人賴其利。以老病求解,徙太子賓客,累封襄陽縣男。卒,謚曰夷。



 子讓,字景宣,以文學蚤有譽。舉進士、宏辭、賢良方正,皆高第,為咸陽尉。父喪,幾毀滅。服除,布衣糲飯,不應闢署十餘年。淮南節度使李鄘即所居敦請置幕府,除監察御史,位給事中,累遷福建觀察使,兼御史中丞。有仁惠名。或以婢遺讓者,問所從,答曰:「女兄九人皆為官所賣,留者獨老母耳。」讓慘然,為燹券,召母歸之。入為散騎常侍,拜江西觀察使。卒,年七十一,贈禮部尚書。



 韋丹,字文明,京兆萬年人,周大司空孝寬六世孫。高祖琨,以洗馬事太子承乾,諫不聽。太宗才之,擢給事中。高宗在東宮,為中舍人,封武陽縣侯。孝敬為太子,琨以右中護為詹事。卒,贈秦州都督,謚曰貞。



 丹蚤孤,從外祖顏真卿學,擢明經,調安遠令,以讓庶兄,入紫閣山事從父能。復舉《五經》高第,歷咸陽尉,張獻甫表佐邠寧幕府。順宗為太子,以殿中侍御史召為舍人。新羅國君死,詔拜司封郎中往吊。故事,使外國,賜州縣十官,賣以取貲,號「私覿官」。丹曰:「使外國,不足於資,宜上請,安有貿官受錢?」即具疏所宜費,帝命有司與之,因著令。未行,而新羅立君死,還為容州刺史。教民耕織,止惰游,興學校,民貧自鬻者,贖歸之,禁吏不得掠為隸。始城州,周十三里,屯田二十四所,教種茶、麥,仁化大行。遷河南少尹,未至,徙義成軍司馬。以諫議大夫召,有直名。



 劉闢反,議者欲釋不誅,丹上疏,以為「孝文世,法廢人慢,當濟以威,今不誅闢,則可使者唯兩京耳」。憲宗褒美。會闢圍梓州,乃授丹劍南東川節度使,代李康。至漢中,上言康守方盡力,不可易。召還議蜀事。闢去梓,因以讓高崇文,乃拜晉慈隰州觀察使,封咸陽郡公。閱歲,自陳所治三州,非要害地,不足張職,為國家費,不如屬之河東,帝從之。



 徙為江南西道觀察使。丹計口受俸,委餘於官,罷八州冗食者,收其財。始,民不知為瓦屋,草茨竹椽,久燥則戛而焚。丹召工教為陶,聚材於場,度其費為估,不取贏利。人能為屋者,受材瓦於官,免半賦,徐取其償;逃未復者,官為為之;貧不能者,畀以財;身往勸督。置南北市,為營以舍軍,歲中旱,募人就功,厚與直,給其食。為衢南北夾兩營,東西七里。以廢倉為新廄,馬息不死。築堤捍江,長十二里,竇以疏漲。凡為陂塘五百九十八所,灌田萬二千頃。有吏主倉十年,丹覆其糧,亡三千斛,丹曰:「吏豈自費邪?」籍其家,盡得文記,乃權吏所奪,召諸吏曰:「若恃權取於倉,罪也。與若期,一月還之。」皆頓首謝,及期無敢違。有卒違令當死,釋不誅,去,上書告丹不法,詔丹解官待辨。會卒,年五十八。驗卒所告,皆不實,丹治狀愈明。



 太和中,裴誼觀察江西,上言為丹立祠堂,刻石紀功,不報。宣宗讀《元和實錄》,見丹政事卓然,它日與宰相語:「元和時治民孰第一?」周墀對:「臣嘗守江西,韋丹有大功,德被八州,歿四十年,老幼思之不忘。」乃詔觀察使紇干蒦上丹功狀,命刻功於碑。



 子宙,推廕累調河南府司錄參軍,李玨表河陽幕府。宣宗謂宰相墀曰:「丹有子否?」以宙對。帝曰:「與好官。」乃拜侍御史,三遷度支郎中。


盧鈞節度太原,表宙為副。是時,回鶻已破諸部,入塞下,剽殺吏民。鈞欲得信重吏視邊,宙請往。自定襄、雁門、五原,絕武州塞,略云中,逾句注,遍見酋豪,鐫諭之;視亭障守卒,增其稟;約吏不得擅以兵侵諸戎,犯者死,於是三部六蕃諸種皆信悅。召拜吏部郎中。出為永州刺史。州方災歉,乃斥官下什用所以供刺史者,得九十餘萬錢,為市糧餉。俗不知法,多觸罪,宙為書制律,並種植為生之宜,戶給之。州負嶺,轉餉艱險,每饑,人輒莩死,宙始築常平倉,收穀羨餘以待乏。罷冗役九百四十四員。縣舊置吏督賦,宙俾民自輸,家十相保,常先期。湘源生零陵香,歲市上供,人苦之,宙為奏罷。民貧無牛,以力耕,宙為置社,二十家月會錢若干,探名得者先市牛,以是為準,久之,牛不乏。立學官,取仕家子弟十五人充之。初,俚民婚,出財會賓客,號「破酒」,晝夜集,多至數百人,貧者猶數十;力不足,則不迎,至淫奔者。宙條約,使略如禮,俗遂改。邑中少年,常以七月擊鼓,群入民家,號「行盜」,皆迎為辦具,謂之「起盆」,後為解素,喧呼
 \gezhu{
  疒只}
 斗。宙至,一切禁之。



 還為大理少卿。久之,拜江西觀察使,政簡易,南方以為世官。遷嶺南節度使。南詔陷交趾,撫兵積備,以干聞。加檢校尚書左僕射、同中書門下平章事。咸通中卒。



 宙弟岫,字伯起,亦有名。宙在嶺南,以從女妻小校劉謙,或諫止之,岫曰:「吾子孫或當依之。」謙後以功為封州刺史,生二子,即隱、龔。盧攜舉進士,陋甚,岫獨謂攜必大用。攜執政,岫自泗州刺史擢福建觀察使云。



 盧弘宣,字子章。元和中,擢進士第。鄭權帥襄陽,闢署幕府。李愬代權,又二人交憾。弘宣始謁愬,愬敕左右謹衛,既與語,見其沖遠,不覺洗然。裴度留守東都,表為判官,遷累給事中。駙馬都尉韋處仁拜虢州刺史,弘宣謂非所任,還詔不下。



 開成中,山南、江西大水,詔弘宣與吏部郎中崔瑨分道賑恤,使有指。還,遷京兆尹、刑部侍郎。拜劍南東川節度使。時歲饑,盜贅結,酋豪自王,偽署官吏,發敖廥招亡命,聯蓬、瀘、嘉、榮諸州,訹蠻落搖亂,根株磐熾。弘宣下檄脅諭,賊黨稍降,其黠強者署軍中,孱無能還之農。魁長逃入峽中,吏捕誅之。徙義武節度使。弘宣性寬厚,政目簡省,人便安之,然犯者不甚貸。河朔故法,偶語軍中則死,弘宣使除之。初,詔賜其軍粟三十萬斛,貯飛狐,弘宣計輓費不能滿直,敕吏守之。明年春,大旱,教民隨力往取,時幽、魏饑甚,獨易、定自如。至秋,悉收所貸,軍食以饒。歷工部尚書、秘書監,以太子少傅致仕。卒,年七十七,贈尚書右僕射。弘宣患士庶人家祭無定儀,乃合十二家法,損益其當,次以為書。



 子告,字子有,及進士第,終給事中。



 薛元賞,亡里系所來。太和初,自司農少卿,出為漢州刺史。時李德裕為劍南西川節度使,會維州降,德裕受之以聞,牛僧孺沮其議,執還之。元賞上書極言可因撫之,潰虜膺腹,不可失。不省。段文昌代德裕,狀元賞治當最。遷累司農卿、京兆尹。出為武寧節度使,罷泗口猥稅,人以為便。俄徙邠寧。



 會昌中,德裕當國,復拜京兆尹。都市多俠少年,以黛墨鑱膚,誇詭力,剽奪坊閭。元賞到府三日,收惡少,杖死三十餘輩,陳諸市,餘黨懼,爭以火滅其文。元賞長吏事,能推言時弊,件白之。禁屯怙勢擾府縣,元賞數與爭,不少縱,由是軍暴折戢,百姓賴安。就加檢校吏部尚書。閱歲,進工部尚書,領諸道鹽鐵轉運使。德裕用元賞弟元龜為京兆少尹,知府事。宣宗立,罷德裕,而元龜坐貶崖州司戶參軍,元賞下除袁王傅。久之,復拜昭義節度使,卒。



 何易於,不詳何所人及所以進。為益昌令。縣距州四十里,刺史崔樸常乘春與賓屬泛舟出益昌旁,索民挽繂,易於身引舟,樸驚問狀,易於曰:「方春,百姓耕且蠶,惟令不事,可任其勞。」樸愧,與賓客疾驅去。鹽鐵官榷取茶利,詔下,所在毋敢隱。易於視詔書曰:「益昌人不征茶且不可活,矧厚賦毒之乎?」命吏閣詔,吏曰:「天子詔何敢拒?吏坐死,公得免竄邪?」對曰:「吾敢愛一身,移暴於民乎?亦不使罪爾曹。」即自焚之。觀察使素賢之,不劾也。民有死喪不能具葬者,以俸敕吏為辦。召高年坐,以問政得失。凡鬥民在廷,易於丁寧指曉枉直,杖楚遣之,不以付吏,獄三年無囚。督賦役不忍迫下戶,或以俸代輸。饋給往來,傳符外一無所進,故無異稱。以中上考,遷羅江令。刺史裴休嘗至其邑,導侍不過三人,廉約蓋資性云。



\end{pinyinscope}