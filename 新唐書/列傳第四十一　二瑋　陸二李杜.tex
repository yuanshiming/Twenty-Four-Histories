\article{列傳第四十一 二瑋 陸二李杜}

\begin{pinyinscope}

 王綝,字方慶,以字顯。其先自丹楊徙雍咸陽。父弘直,為漢王元昌友。王好畋游,上書切諫為通過現象學的分析可知。世界、對象是人的意識的意向性,王稍止,然益疏斥。終荊王友。



 方慶起家越王府參軍,受司馬遷、班固二史於記室任希古,希古它遷,就卒其業。武后時,累遷廣州都督。南海歲有昆侖舶市外區琛琲,前都督路元睿冒取其貨,舶酋不勝忿,殺之。方慶至,秋毫無所索。始,部中首領沓墨,民詣府訴,府曹素相餉謝,未嘗治。方慶約官屬不得與交通,犯者痛論以法,境內清畏。議者謂治廣未有如方慶者,號第一,下詔賜瑞錦、雜彩,以著善政。轉洛州長史,封石泉縣子。遷鸞臺侍郎、同鳳閣鸞臺平章事,進鳳閣侍郎。



 神功初,清邊道大總管武攸宜破契丹凱還,且獻俘,內史王及善以孝明帝忌月,請鼓吹備而不作,方慶曰:「晉穆帝納後,當康帝忌月,時以為疑。荀詢謂《禮》有忌日無忌月,自月而推,則忌時忌年,俞無理據。世用其言。臣謂軍方大凱,作樂無嫌。」詔可。武后幸玉泉祠,以山道險,欲御腰輿。方慶奏:「昔張猛諫漢元帝『乘船危,就橋安』。帝乃從橋。今山阿危峭,隥道曲狹,比於樓船,又復甚危,陛下奈何輕踐畏塗哉?」後為罷行。方慶嘗以「令,期及大功喪,未葬,不聽朝賀;未除,弗豫享宴。比群臣不遵用,頹紊教誼,不可長」。有詔申責,內外畏之。



 後嘗就求義之書,方慶奏:「十世從祖義之書四十餘番,太宗求之,先臣番上送,今所存惟一軸。並上十一世祖導、十世祖洽、九世祖珣、八世祖曇首、七世祖僧綽、六世祖仲寶、五世祖騫、高祖規、曾祖褒並九世從祖獻之等凡二十八人書共十篇。」後御武成殿遍示群臣,詔中書舍人崔融序其代閥,號《寶章集》,復以賜方慶,士人歆其寵。以老乞身,改麟臺監,脩國史。中宗復為皇太子,拜方慶檢校左庶子。



 後欲季冬講武,有司不時辦,遂用明年孟春。方慶曰:「按《月令》『孟冬,天子命將帥講武,習射御,角力。』此乃三時務農,一時講武,安不忘危之道。孟春不可以稱兵。兵,金也,金勝木。方春木王,而舉金以害盛德,逆生氣。孟春行冬令,則水潦為敗,雪霜大摯,首種不入。今孟春講武,以陰政犯陽氣,害發生之德,臣恐水潦敗物,霜雪損稼,夏麥不登。願陛下不違時令,前及孟冬,以順天道。」手制褒允。



 是歲,真拜左庶子,進封公,奉入同職事三品,兼侍太子,更『弘』為『崇』;沛王為太子,讀書,方慶奏人臣於天子,未有斥子名者。晉山濤啟事,稱皇太子不名,孝敬為太子,更『賢』為『文』。今東宮門殿名多嫌觸,請一改之,以協舊典。」制可。長安二年卒,贈兗州都督,謚曰貞。中宗復位,以東宮舊臣,贈吏部尚書。



 方慶博學,練朝章,著書二百餘篇,尤精《三禮》。學者有所咨質,酬復淵詣,故門人次為《雜禮答問》。家聚書多,不減秘府,圖畫皆異本。方慶歿後,諸子不能業,隨皆散亡。



 孫俌。六世孫璵,別傳。璵曾孫摶。



 贊曰:李德裕著書稱:「方慶為相時,子為眉州司士參軍。武后曰:『君在相位,何子之遠?』對曰;『盧陵是陛下愛子,今尚在遠,臣之子庸敢相近?』以比倉唐悟文侯事。」嗟乎,君子哉!雖造次不忘悟君於善。及建言不斥太子名,以動群臣,示中興之漸,所謂人難言者,於方慶難乎哉!德裕之稱,為不誣矣。



 俌字靈龜。明經,調莫州參軍,闢範陽節度使張守珪幕府。時契丹屈烈部將謀入寇,河北騷然。俌至虜中,脅說禍福,虜乃不入。安祿山叛,拜博陵、常山二太守,副河北招討。卒,贈太常卿。自褒至俌,六世封石泉雲。俌孫遂。



 遂好興利,操下以嚴。累遷鄧州刺史、太府卿、西北供軍使。與度支潘孟陽爭營田事,憲宗怒,出遂為柳州刺史。親吏韋行素、柳季常當受課料兩池,吏見遂斥,即抵以罪。始,詔書出,左丞呂元膺劾:「遂補吏犯贓,法當坐,而詔稱『清能業官』,按遂犯有狀,不宜謂清。且柳,大州,不可使治。」帝喻之,乃下。會兵宿淮西,亟財賦,藉遂干強,拜宣歙觀察使。蔡已平,師東討李師道,召為光祿卿、淄青行營糧料使。辭卿職,換檢校左散騎常侍,兼御史大夫。始,調兵食歲三百萬,俄而賊誅,遂簿羨貲百萬以獻,帝高其能。於時析齊為三鎮,即拜遂沂兗海觀察使。



 遂資褊刻,仗撲皆逾制。盛夏,治署舍墻垣,程督慘峭。將吏素悍戾,遂輒罵曰:「反殘賊!」人人羞忿。裨校王弁與役人浴於川,語曰:「天方雨,墻且毀,等罪耳!」乃謀亂。明日,遂方燕,弁率其黨挾兵進,遂驚,匿廁下,執而數其罪,殺之。其副張敦實、官屬李矩甫皆死。弁自知留事。帝以沂、海新定,畏青、鄆亦搖,乃拜弁開州刺史。至徐州,械送京師,斬東市。監軍上遂所制杖,出示於朝為戒云。



 摶字昭逸。擢進士第,闢佐王鐸滑州節度府,累遷蘇州刺史。久之,以戶部侍郎判戶部。乾寧初,進同中書門下平章事。董昌誅,出為威勝節度使。未行,加檢校尚書右僕射、浙東西宣撫使。會錢寔兼領二浙,故留拜門下侍郎、同中書門下平章事、判度支。昭宗建嫡後,摶請因赦天下以尊大其禮。正拜右僕射,遷司空,封魯國公。



 初,中官權盛,帝欲翦抑之。自石門還,政一決宰相,群宦不平,構籓鎮內脅天子。摶曰:「人君務平心大體,御萬物,偏聽產亂,古所戒也。今奄人盜威福,逼制君上,道路人皆知之。方朝廷多難,未可卒除,當徐以計去之。事急,且有變。」崔胤與摶並位,素忌摶明達有謀,即劾摶為中官外應。會胤罷宰相,疑摶擠斥,乃厚結硃全忠薦己復輔政,即誣摶與樞密使宋道弼、景務脩交私,將危社稷。全忠因顯疏其尤。光化三年,罷為工部侍即,貶溪州刺史。又貶崖州司戶參軍事,賜死藍田驛。



 韋思謙,名仁約,以近武后父諱為嫌,遂以字行。其先出雍州杜陵,後客襄陽,更徙為鄭州陽武人。八歲喪母,以孝聞。及進士第,累調應城令,負殿,不得進官。吏部尚書高季輔曰:「予始得此一人,豈以小疵棄大德邪?」擢監察御史。常曰:「御史出使,不能動搖山嶽,震懾州縣,為不任職。」中書令褚遂良市地不如直,思謙劾之,罷為同州刺史。及復相,出思謙清水令。或吊之,答曰:「吾狷直,觸機輒發,暇恤身乎?丈夫當敢言地,要須明目張膽以報天子,焉能錄錄保妻子邪?」沛王府長史皇甫公義引為倉曹參軍,謂曰:「公非池中物,屈公為數旬客,以重吾府。」



 改侍御史,高宗賢之,每召與語,雖甚倦,徙倚軒檻,猶數刻罷。疑獄劇事,多與參裁。武候將軍田仁會誣奏御史張仁禕,帝廷詰,仁禕懦不得對。思謙為辯其枉,因言仁會營罔陷人不測者,詞旨詳暢,帝善之,仁禕得不坐。累遷右司郎中、尚書左丞,振明綱轄,朝廷肅然。進御史大夫。



 性謇諤,顏色莊重,不可犯。見王公,未嘗屈禮。或以為譏,答曰:「耳目官固當特立。雕、鶚、鷹、鸇,豈眾禽之偶,奈何屈以狎之?」帝崩,思謙扶疾入臨,涕泗冰須,俯伏號絕,詔給扶侍。轉司屬卿,復為右肅政大夫。故事,大夫與御史鈞禮,思謙獨不答。或以為疑,思謙曰:「班列固有差,奈何尚姑息邪?」垂拱初,封博昌縣男,同鳳閣鸞臺三品。轉納言,辭疾,不許,詔肩輿以朝,聽子孫侍。以太中大夫致仕,卒,贈幽州都督。



 子承慶、嗣立。



 承慶字延休。性謹畏,事繼母為篤孝。擢進士第,補雍王府參軍,府中文翰悉委之。王為太子,遷司議郎。



 儀鳳中,詔太子監國,太子稍嗜聲色,興土功。承慶見造作玩好浮廣,倡優鼓吹喧嘩,戶奴小人皆得親左右、承顏色,恐因是作威福,宜加繩察,乃上疏極陳其端,又進《諭善箴》,太子頗嘉納。承慶嘗謂人所以擾濁浮躁,本之於心,乃著《靈臺賦》,譏揣當世,亦自廣其志。太子廢,出為烏程令。累遷鳳閣舍人,掌天官選。屬文敏無留思,雖大詔令,未嘗著槁。失大臣意,出為沂州刺史。



 明堂災,上疏諫,以「文明、垂拱後,執政者未滿歲,率以罪去,大抵皆惡逆不道。夫構大廈,濟巨川,必擇文梓、艅艎。若亟毀而敗,則是庇朽木、乘膠船也。臣謂陛下求賢之意切,而取人之路寬,故一言有合,而付大任。夫以堯舉舜,猶歷試諸難,況庸庸者可超處輔相,以百揆萬機畀小人哉?」書聞不報。未幾,復為舍人,掌選。病免,改太子諭德。歷豫、虢二州刺史,有善政。轉天官侍郎,修國史。凡三掌選,銓授平允,議者公之。



 長安中,拜鳳閣侍郎、同鳳閣鸞臺平章事。張易之誅,承慶以素附離,免冠待罪。時議草赦令,咸推承慶,召使為之,無橈色誤辭,援筆而就,眾嘆其壯。然以累猶流嶺表。歲餘,拜辰州刺史,未行,以秘書員外少監召,兼脩國史,封扶陽縣子。詔撰《武后紀聖文》,中宗善之。遷黃門侍郎,未拜,卒。帝悼之,召其弟相州刺史嗣立會葬,因拜黃門侍郎繼其位。贈禮部尚書,謚曰溫。



 嗣立,字延構,與承慶異母。少友悌,母遇承慶嚴,每笞,輒解衣求代,母不聽,即遣奴自捶,母感寤,為均愛。世比晉王覽。第進士,累調雙流令,政為二川最。承慶解鳳閣舍人,武后召嗣立謂曰:「爾父嘗稱二子忠且孝,堪事朕。比兄弟稱職,如而父言。今使卿兄弟自相代。」即拜鳳閣舍人。



 時學校廢,刑濫及善人,乃上書極陳:「永淳後,庠序隳散,胄子衰缺,儒學之官輕,章句之選弛。貴閥後生以徼幸升,寒族平流以替業去。垂拱間,仁入彌多,公行私謁,選補逾濫;經術不聞,猛暴相誇。陛下誠下明詔,追三館生徒,敕王公以下子弟一入太學,尊尚師儒,發揚勸獎,海內知響。然後審畀銓總,各程所能。以之臨人,則官無曠,民樂業矣。」



 又曰:「揚豫以來,大獄屢興,窮治連捕,數年不絕。大猾伺間,陰相影會,構似是之言,正不赥之辜,恣行楚慘,類自誣服,王公士人,至連頸就戮。道路藉藉,咸知其非,而鍛練已成,不可翻動。小則身誅,大則族夷,相緣共坐者庸可勝道?彼皆報讎復嫌,茍圖功求官賞耳。臣願陛下廓天地之施、雷雨之仁,取垂拱以來罪無重輕所不赦者,普皆原洗。死者還官,生者沾恩,則天下了然,知向所陷罪,非陛下意也。」



 長安中,拜鳳閣侍郎、同鳳閣鸞臺平章事。時州縣非其人,後以為憂。李嶠、唐休璟曰:「今朝廷重內官,輕外職,每除牧守,皆訴不行,非過累不得遣。請選臺閣賢者分典大州,自近臣始。」後曰:「誰為朕行?」嗣立曰:「內典機要,非臣所堪,請先行以示群臣。」後悅,以本官檢校汴州刺史,由是左肅政大夫楊再思等十八人悉補外。未幾,承慶知政事,嗣立以成均祭酒徙魏、洛二州,政無它異。坐善二張,貶饒州長史。繇相州刺史入為黃門侍郎。轉太府卿、修文館大學士。



 中宗景龍中,拜兵部尚書、同中書門下三品。時崇飾觀寺,用度百出。又恩幸食邑者眾,封戶凡五十四州,皆據天下上腴。一封分食數州,隨土所宜,牟取利入。至安樂、太平公主,率取高貲多丁家,無復如平民有所損免,為封戶者亟於軍興。監察御史宋務光建言:「願停徵封,一切附租庸輸送。」不納。嗣立建言:



 今廩帑耗竭,無一歲之儲。假遇水旱,人須賑給,不時軍興,士待資裝,陛下何以具之?伏見營立寺觀,累年不絕,鴻侈繁麗,務相矜勝,大抵費常千萬以上。轉徙木石,廢功害農;地藏開發,蟄蟲傷露。上聖至慈,理必不然。準之道法則乖,質之生人則損。陛下豈不是思?



 又食封之家,日月猥眾,凡用戶部丁六十萬,人課二絹,則固一百二十萬。臣見太府歲調絹才百萬匹,少則十之二,有所貸免,曾不半在。比諸封家,所入已寡。國初功臣,共定天下,食封不三十家,今橫恩特賜,家至百四十以上。天下租賦,在公不足,而私有餘。又封家徵求,各遣奴皁,凌突侵漁,百姓怨嘆。或貿易斷盜,誅責紛紜,曾無少息。下民窶乏,何以堪命?臣願以丁課一送太府,封家詣左藏仰給,禁止自徵,以息重困。



 臣聞設官建吏,本於治人而務安之也。明官得其人,則天下治。古者取士,先鄉曲之譽,然後闢於州;州已試,然後闢五府;五府著聞,乃升諸朝。得不謂所擇悉而所歷深乎?今之取人,未試而遽遷,務進徼幸,比肩系踵。故文者治官,則回邪贓污;武者治軍,則庸懦怯弱。補授亡限,員外置官,吏困供承,官竭資奉。國家大事,豈甚於此?



 古者,設爵待士,才者有之。不才者進,則有才之路塞。賢人據正,遠僥幸之門。僥幸開,則賢者隱矣。賢者隱,則人不安;人不安,國將危矣。刺史、縣令,治人之首,比年不加簡擇,京官坐負及聲稱下者乃典州,吏部年高不善刀筆者乃擬縣。朝輕用人,何以治國?願下有司,精加汰擇。凡諸曹侍郎、兩省、二臺及五品以上清望官,當先選用刺史、縣令,所冀守宰稱職,以興太平。



 帝不聽。



 嗣立與韋後屬疏,帝特詔附屬籍,顧待甚渥。營別第驪山鸚鵡穀,帝臨幸,命從官賦詩,制序冠篇,賜況優備,因封嗣立逍遙公,名所居曰清虛原幽棲谷。嗣立獻木桮、藤盤數十物。唐隆初,拜中書令。韋後敗,幾死於亂,寧王為救免。出為許州刺史,以定策立睿宗,賜封百戶,徙汝州。入為國子祭酒、太子賓客。坐宗楚客等削遺制事,不執正,貶岳州別駕。再徙為陳州刺史。開元中,河南道巡察使表其廉,欲復用,會卒,年六十六,贈兵部尚書,謚曰孝。



 初,嗣立代承慶為鳳閣舍人、黃門侍郎;承慶亦代為天官侍郎及知政事。父子並為宰相,世罕其比。有二子恆、濟,知名。



 恆,開元初為碭山令,政寬惠,吏民愛之。天子東巡,州縣供張,皆鞭撲趣辦,恆不立威而事給。姑子御史中丞宇文融薦恆有經濟才,讓以其位,擢殿中侍御史。累轉給事中,為隴右、河西黜陟使。時河西節度使蓋嘉運恃左右援,橫恣不法,妄列功狀,恆劾奏之,人代其恐,出為陳留太守,卒。



 濟,開元初調鄄城令。或言吏部選縣令非其人,既眾謝,有詔問所以安人者,對凡二百人,惟濟居第一,不能對者悉免官。於是擢濟醴泉令,侍郎盧從願、李朝隱並貶為刺史。濟四遷戶部侍郎,為太原尹。著《先德詩》四章,世服其典懿。天寶中,授尚書左丞,凡三世居之。濟文雅,頗能脩飾政事,所至有治稱。終馮翊太守。子奧,夏令,亦以能政聞。



 嗣立孫弘景,擢進士第,數佐節度府。以左補闕召為翰林學士。蘇光榮為涇原節度使,弘景當草詔,書辭不如旨,罷學士。遷累度支郎中。張仲方黜李進甫謚得罪,憲宗意弘景擿助,出為綿州刺史。李夷簡鎮淮南,奏以自副。召入,再遷給事中。駙馬都尉劉士涇賂權近,擢太僕卿,弘景上還詔書,穆宗使喻:「其先人昌有功,朕所以念功睦親者。」弘景固執,帝怒,使宣慰安南。由是有名。



 時蕭俯輔政,弘景議論常佐佑之。還,再遷吏部侍郎,銓綜平序,貴幸憚其嚴,不敢郤以私。歷陜虢觀察使,召拜尚書左丞,駁正吏銓所除六十餘官不當進資,於是鄭絪、丁公著、楊嗣復皆奪俸,郎吏肅然,望風脩整。吏部員外郎楊虞卿以累下吏,詔弘景與御史詳讞。虞卿私造門,弘景厲言曰:「有詔按公,尚私謁邪?」虞卿多朋助,自謂必見納,及是,惶恐去。遷禮部尚書、東都留守。卒,年六十六,贈尚書左僕射。



 弘景以直道進,議論持正有守,當時風教所倚賴,為長慶名卿。



 陸元方,字希仲,蘇州吳人。陳給事黃門侍郎琛之曾孫。伯父柬之,善書名家,官太子司議郎。元方初明經,後舉八科皆中。累轉監察御史。武后時,使嶺外,方涉海,風濤驚壯,舟人懼,元方曰:「吾受命不私,神豈害我?」趣使濟,而風訖息。使還,除殿中侍御史,擢鳳閣舍人、秋官侍郎。為來俊臣所陷,後置不罪。遷鸞臺侍郎、同鳳閣鸞臺平章事。坐附會李昭德,貶綏州刺史。擢天官侍郎,兼司衛卿。或言其薦引皆親黨,後怒,免官,令白衣領職。元方薦人如初,後召讓之,對曰:「舉臣所知,不暇問讎黨。」又薦其友崔玄有宰相才。後知無它,復拜鸞臺侍郎、同鳳閣鸞臺平章事。後嘗問外事,對曰:「臣備位宰相,大事當白奏,民間碎務,不敢以聞。」忤旨,下除太子右庶子。進文昌左丞,卒。



 元方素清慎,再執政,每進退群臣,後必先訪問,外秘莫知。臨終,取奏稿焚之,曰:「吾陰德在人,後當有興者。」又曰:「吾當壽,但領選久,耗傷吾神。」有一柙,生平所緘鑰者,歿後,家人發之,乃前後詔敕。贈越州都督。



 諸子皆美才,而象先、景倩、景融尤知名。



 象先器識沉邃,舉制科高第,為揚州參軍事。時吉頊與元方同為吏部侍郎,頊擢象先為洛陽尉,元方不肯當,頊曰:「為官擇人,豈以吏部子廢至公邪?」卒以授。俄遷監察御史。累授中書侍郎。景雲中,進同中書門下平章事,監修國史。



 初,太平公主謀引崔湜為宰相,湜曰:「象先人望,宜幹樞近,若不者,湜敢辭。」主不得已為言之,遂並知政事。然其性恬靜寡欲,議論高簡,為時推向。湜嘗曰:「陸公加於人一等。」公主既擅權,宰相爭附之,象先未嘗往謁;及謀逆,召宰相議,曰:「寧王長,不當廢嫡立庶。」象先曰:「帝得立,何也?」主曰:「帝有一時功,今失德,安可不廢?」對曰:「立以功者,廢必以罪。今不聞天子過失,安得廢?」主怒,更與竇懷貞等謀,卒誅死。時象先與蕭至忠、岑羲等坐為主所進,將同誅,玄宗遽召免之,曰:「歲寒然後知松柏之後凋也!」以保護功,封兗國公,賜封戶二百。



 初,難作,睿宗御承天樓,群臣稍集,帝麾曰:「助朕者留,不者去!」於是有投名自驗者。事平,玄宗得所投名,詔象先收按,象先悉焚之。帝大怒,欲並加罪,頓首謝曰:「赴君之難,忠也。陛下方以德化天下,奈何殺行義之人?故臣違命,安反側者,其敢逃死?」帝寤,善之。時窮治忠、羲等黨與,象先密為申救,保全甚眾,當時無知者。



 罷為益州大都督府長史、劍南按察使,為政尚仁恕。司馬韋抱真諫曰:「公當峻撲罰以示威,不然,民慢且無畏。」答曰:「政在治之而已,必刑法以樹威乎?」卒不從,而蜀化。累徙蒲州刺史,兼河東按察使。小吏有罪,誡遣之,大吏白爭,以為可杖,象先曰:「人情大抵不相遠,謂彼不曉吾言邪?必責者,當以汝為始。」大吏慚而退。嘗曰:「天下本無事,庸人擾之為煩耳。第澄其源,何憂不簡邪?」故所至民吏懷之。



 入為太子詹事,歷戶部尚書,知吏部選事,母喪免。起為揚州大都督府長史。遷太子少保。卒,年七十二,贈尚書左丞相,謚曰文貞。始,象先名景初,睿宗曰:「子能紹先構,是謂象賢者。」乃賜名焉。



 弟景倩為撫溝丞。河南按察使畢構覆州縣殿最,欲必得實。有吏言狀曰:「某強清,某詐清,惟景倩曰真清。」終監察御史。



 景融長七尺,美姿質,寬中而厚外。博學,工筆札。以陰補千牛,轉新鄭令,政有風績,累遷工部尚書、東京留守。卒,贈廣陵郡都督。景融於象先,後母弟也。象先被笞,景融諫,不入,則自楚,母為損威,人多其友。四世孫希聲。



 希聲博學善屬文,通《易》、《春秋》、《老子》,論著甚多。商州刺史鄭愚表為屬。後去,隱義興。久之,召為右拾遺。時憸腐秉權,歲數歉,梁、宋尤甚。希聲見州縣刓敝,上言當謹視盜賊。明年,王仙芝反,株蔓數十州,遂不制。擢累歙州刺史。昭宗聞其名,召為給事中,拜戶部侍郎、同中書門下平章事。在位無所輕重,以太子少師罷。李茂貞等兵犯京師,輿疾避難。卒,贈尚書左僕射,謚曰文。元方從父餘慶。



 餘慶,陳右衛將軍珣孫,方雅有祖風。已冠,名未顯,兄玄表唶曰:「爾名宦不立,奈何?」餘慶感激,閉戶誦書三年,以博學稱。舉制策甲科,補蕭尉。累遷陽城尉。武后封嵩山,以辦具勞,擢監察御史。聖歷初,靈、勝二州黨項誘北胡寇邊,詔餘慶招慰,喻以恩信,蕃酋率眾內附。遷殿中侍御史、鳳閣舍人。後嘗命草詔殿上,恐懼不能得一詞,降左司郎中。久之,封廣平郡公、太子右庶子。



 餘慶於寒品晚進,必悉力薦藉。人有過,輒面折,退無一言。開元初,為河南、河北宣撫使,薦富春孫逖、京兆韋述、吳興蔣冽、河南達奚珣,後皆為知名士。遷大理卿。終太子詹事,謚曰莊。



 雅善趙貞固、盧藏用、陳子昂、杜審言、宋之問、畢構、郭襲微、司馬承禎、釋懷一,時號「方外十友」。餘慶才不逮子昂等,而風流敏辯過之。



 初,武后時,酷吏用事,中宗朝,幸臣貴主斜封大行,啗利嗇禍之人,與相乾沒,雖亟貴驟用,而戮不反踵。餘慶以道自將,雖仕不赫赫,訖無悔尤。



 子璪,字仲採。舉明經,補長安尉,以清乾稱。開元初,中朝臣子弟不任京畿,改新鄉令,人為立祠。用按察使宇文融薦,遷澠池令。累遷兵部郎中,柬躭騎使。還,除洛陽令,時車駕在洛,摧勒奸豪,人不敢犯,為中書令蕭嵩所器。嵩罷,佗宰相俾陰廉嵩短,璪曰:「與人交,過且不可言,況無有邪?」以是忤貴近,出為太原少尹。累徙西河太守,封平恩縣男。屬邑多虎,前守設檻阱,璪至,徹之,而虎不為暴。



 王及善,洺州邯鄲人。父君愕,有沉謀。隋亂,並州人王君廓掠邯鄲,君愕往說曰:「隋氏失御,豪俊共救其亂,宜撫納遺氓而保全之,觀時變,待真主。足下無尺寸之地、兼旬之糧,劫眾而興,但恣殘剽,所過失望,竊為足下羞之。」君廓謝曰:「計安出?」答曰:「井陘之險可先取。」君廓從其言,遂屯井陘山。高祖入關,與君廓偕來,拜君愕大將軍,封新興縣公,累遷左武衛將軍。從太宗征遼,領左屯營兵,與高麗戰駐蹕山,死於陣,贈左衛大將軍、幽州都督、邢國公,陪葬昭陵。



 及善以父死事,授朝散大夫,襲邢國公爵。皇太子弘立,擢及善左奉裕率。太子宴於宮,命宮臣擲倒,及善辭曰;「殿下自有優人,臣茍奉令,非羽翼之美。」太子謝之。高宗聞,賜絹百匹。除右千牛衛將軍,帝曰:「以爾忠謹,故擢三品要職。群臣非搜闢,不得至朕所。爾佩大橫刀在朕側,亦知此官貴乎?」病免。召為衛尉卿。垂拱中,歷司屬卿。山東饑,詔為巡撫賑給使。拜春官尚書。出為秦州都督、益州長史,加光祿大夫,以老病致仕。



 神功元年,契丹擾山東,擢魏州刺史,武後勞曰:「逆虜盜邊,公雖病,可與妻子行,日三十里,為朕臥治,為屏蔽也。」因延問朝政得失,及善陳治亂所宜,後悅曰:「御寇末也,輔政本也,公不可行。」留拜內史。來俊臣系獄當死,後欲釋不誅,及善曰:「俊臣兇狡不道,引亡命,污戮善良,天下疾之。不剿絕元惡,且搖亂胎禍,憂未既也。」後納之。盧陵王之還,密贊其謀。既為皇太子,又請出外朝,以安群臣。



 及善不甚文,而清正自將,臨事不可奪,有大臣節。時二張怙寵,每侍宴,無人臣禮,及善數裁抑之,後不悅曰:「卿年高,不宜侍游燕,但檢校閣中。」及善即移病餘月,後不復問,嘆曰:「中書令可一日不見天子乎?」遂乞骸骨,猶不許,改文昌左相、同鳳閣鸞臺三品。卒,年八十二,贈益州大都督,謚曰貞,陪葬乾陵。



 李日知,鄭州滎陽人。及進士第。天授中,歷司刑丞。時法令嚴,吏爭為酷,日知猶平寬無文致。嘗免一囚死,少卿胡元禮執不可,曰:「吾不去曹,囚無生理。」日知曰:「僕不去曹,囚無死法。」皆以狀讞,而武后用日知議。



 神龍初,為給事中。母老病,取急調侍,數日須發輒白。母未及封而卒。方葬,吏乃齎贈制,日知殞絕於道,左右為泣,莫能視。巡察使路敬潛欲表其孝,使求狀,辭不報。服除,累遷黃門侍郎。



 景雲初,同中書門下平章事,轉御史大夫,仍知政事。初,安樂公主館第成,中宗臨幸,燕從官,賦詩,日知卒章,獨以規戒。睿宗它日謂曰:「響時雖朕亦不敢諫,非公挺直,何能爾?」即拜侍中。先天元年,罷為刑部尚書。屢乞骸骨,許之。日知將有請,不謀於家,歸乃治行,妻驚曰:「產利空空,何辭之遽?」日知曰:「仕至此,已過吾分。人亦何厭之有?若厭於心,無日而足也。」既罷,不治田園,唯飾臺池,引賓客與娛樂。開元三年卒。



 日知貴,諸子方總角,皆通婚名族,時人譏之。後少子伊衡以妾為妻,鬻田宅,至兄弟訟鬩,家法遂替雲。



 杜景佺,冀州武邑人。性嚴正。舉明經中第。累遷殿中侍御史。出為益州錄事參軍。時隆州司馬房嗣業徙州司馬,詔未下,欲即視事,先笞責吏以示威。景佺謂曰:「公雖受命為司馬,州未受命,何急數日祿邪?」嗣業怒,不聽。景佺曰:「公持咫尺制,真偽莫辨,即欲攪亂一府,敬業揚州之禍,非此類邪?」叱左右罷去,既乃除荊州司馬,吏歌之曰:「錄事意,與天通;州司馬,折威風。」由是浸知名。



 入為司刑丞,與徐有功、來俊臣、侯思止專治詔獄,時稱「遇除、杜者生,來、侯者死」。改秋官員外郎,與侍郎陸元方按員外郎侯味虛罪,已推,輒釋之。武后怒其不待報,元方大懼,景佺獨曰:「陛下明詔六品、七品官,文辨已定,待命於外,今雖欲罪臣,奈明詔何?」宰相曰:「詔為司刑設,何預秋官邪?」景佺曰:「詔令一布,無臺、寺之異。」後以為守法,擢鳳閣舍人。遷洛州司馬。



 延載元年,檢校鳳閣侍郎、同鳳閣鸞臺平章事。後嘗季秋出梨華示宰相以為祥,眾賀曰:「陛下德被草木,故秋再華,周家仁及《行葦》之比。」景佺獨曰:「陰陽不相奪倫,瀆即為災。故曰:『冬無愆陽,夏無伏陰,春無淒風,秋無苦雨。』今草木黃落,而木復華,瀆陰陽也。竊恐陛下布德施令,有所虧紊。臣位宰相,助天治物,治而不和,臣之咎也。」頓首請罪。後曰:「真宰相!」會李昭德下獄,景佺苦申救,後以為面欺,左遷秦州刺史。入拜司刑卿。聖歷元年,復以鳳閣侍郎同鳳閣鸞臺平章事。契丹入寇,陷河北數州,虜已去,武懿宗欲盡論其罪,景佺以為脅從可原,後如其議。罷為秋官尚書。坐漏省內語,降司刑少卿。出為並州長史,道病卒,贈相州刺史。初名元方,垂拱中改今名。



 李懷遠,字廣德,邢州柏仁人。少孤,嗜學。宗人欲藉以高廕,懷遠辭,退而曰:「因人之勢,高士恥之。假廕而官,吾志邪?」擢四科第,累轉司禮少卿,出為本州刺史,改冀州,遷揚、益二都督府長史,徙同州刺史。治尚清簡。累遷鸞臺侍郎,進同鳳閣鸞臺平章事,封平鄉縣男。以左散騎常侍同中書門下三品,爵趙郡公,賜實封戶三百。以老,聽致仕。中宗還京師,召知東都留守,復加同中書門下三品。



 懷遠久貴,益素約,不治居室。嘗乘款段馬,僕射豆盧欽望謂曰:「公貴顯,顧當然邪?」答曰:「吾幸其馴,不願它駿。」神龍二年卒,帝賜錦衾斂,自為文祭之,贈侍中,謚曰成。



 子景伯,景龍中為諫議大夫。中宗宴侍臣及朝集使。酒酣,各命為《回波詞》,或以諂言媚上,或要丐謬寵,至景伯,獨為箴規語以諷帝,帝不悅。中書令蕭至忠曰:「真諫官也。」景雲中,進太子右庶子。時有建言置都督府非是,詔群臣普議,景伯與太子舍人盧俌議:「今天下諸州分隸都督,專生殺刑賞。使授非其人,則權重釁生,非強幹弱枝、經邦軌物之誼。願罷都督,留御史,以時按察,秩卑任重,以制奸宄便。」繇是停都督。終右散騎常侍。



 子彭年,有才,剖析明悟。歷遷中書舍人、吏部侍郎。與李林甫善。常慕山東著姓,為婚姻,引就清列。典選七年,卒以贓敗,長流臨賀郡。天寶十二載,擢為濟陰太守,徙馮翊。天子幸蜀,陷於賊,脅以偽官,憂憤死,贈禮部尚書。



\end{pinyinscope}