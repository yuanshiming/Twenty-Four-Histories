\article{本紀第三 高宗}

\begin{pinyinscope}

 高宗天皇大聖大弘孝皇帝諱治,字為善,太宗第九子也。母曰文德皇后長孫氏。始封晉王,貞觀七年,遙領並州都督。十七年(今陜西眉縣)橫渠鎮人。世稱橫渠先生。理學主要創始人之,太子承乾廢,而魏王泰次當立,亦以罪黜,乃立子治為皇太子。太宗嘗命皇太子游觀習射,太子辭以非所好,願得奉至尊,居膝下。太宗大喜,乃營寢殿側為別院,使太子居之。太宗每視朝,皇太子常侍,觀決庶政。



 二十三年,太宗有疾,詔皇太子聽政於金液門。四月,從幸翠微宮。太宗崩,以羽檄發六府甲士四千,衛皇太子入於京師。六月甲戌,即皇帝位於柩前。大赦,賜文武官勛一轉,民八十以上粟帛,給復雍州及比歲供軍所一年。癸未,長孫無忌為太尉。癸巳,檢校洛州刺史李勣為開府儀同三司,參掌機密。八月癸酉,河東地震。乙亥,又震。庚辰,遣使存問河東,給復二年,賜壓死者人絹三匹。庚寅,葬文皇帝於昭陵。九月甲寅,荊王元景為司徒,吳王恪為司空。乙卯,李勣為尚書左僕射、同中書門下三品。十一月乙丑,晉州地震。左翊衛郎將高偘伐突厥。是冬,無雪。



 永徽元年正月辛丑,改元。丙午,立妃王氏為皇后。張行成為侍中。二月辛卯,封子孝為許王,上金杞王,素節雍王。四月己巳,晉州地震。五月巳未,太白晝見。六月,高偘及突厥戰於金山,敗之。庚辰,晉州地震,詔五品以上言事。七月辛酉,以旱慮囚。八月戊辰,給五品以上解官充侍者半祿,加賜帛。庚午,降死罪以下。九月癸卯,高偘俘突厥車鼻可汗以獻。十月戊辰,李勣罷左僕射。十一月己未,貶褚遂良為同州刺史。十二月庚午,琰州獠寇邊,梓州都督謝萬歲死之。



 二年正月戊戌,開義倉以賑民。乙巳,黃門侍郎宇文節、中書侍郎柳奭同中書門下三品。乙卯,瑤池都督阿史那賀魯叛。四月乙丑,命有司毋進肉食,訖於五月。七月丁未,賀魯寇庭州,左武衛大將軍梁建方、右驍衛大將軍契苾何力為弓月道行軍總管以伐之。八月己巳,高季輔為侍中;於志寧為尚書左僕射,張行成為右僕射,同中書門下三品。己卯,白水蠻冠邊,左領軍將軍趙孝祖為郎州道行軍總管以伐之。九月癸卯,以同州苦泉牧地賜貧民。十月辛卯,晉州地震。十一月辛酉,有事於南郊。癸酉,禁進犬馬鷹鶻。戊寅,忻州地震。甲申,雨木冰。是月,竇州、義州蠻寇邊,桂州都督劉伯英敗之。趙孝祖及白水蠻戰於羅仵候山,敗之。十二月乙未,太白晝見。壬子,處月硃邪孤注殺招慰使單道惠,叛附於賀魯。是冬,無雪。



 三年正月癸亥,梁建方及處月戰於牢山,敗之。甲子,以旱避正殿,減膳,降囚罪,徒以下原之。己巳,褚遂良為吏部尚書、同中書門下三品。丙子,享於太廟。丁亥,耕藉田。三月辛巳,雨土。宇文節為侍中,柳奭守中書令。四月庚寅,趙孝祖及白水蠻戰,敗之。甲午,彭王元則薨。是月,兵部侍郎韓瑗為黃門侍郎、同中書門下三品。五月庚申,求齊侍中崔季舒、給事黃門侍郎裴澤、隋儀同三司豆盧毓、御史中丞游楚客子孫官之。七月丁巳,立陳王忠為皇太子,大赦,賜五品以上子為父後者勛一轉,民酺三日。九月丙辰,求周司沐大夫裴融、尚書左丞封孝琰子孫官之。是月,中書侍郎來濟同中書門下三品。十二月癸巳,濮王泰薨。



 四年二月甲申,駙馬都尉房遺愛薛萬徹柴令武、高陽巴陵公主謀反,伏誅;殺荊王元景、吳王恪。乙酉,流宇文節於桂州。戊子,廢蜀王愔為庶人。己亥,徐王元禮為司徒,李勣為司空。四月壬寅,以旱慮囚,遣使決天下獄,減殿中、太僕馬粟,詔文武官言事。甲辰,避正殿,減膳。六月己丑,太白晝見。八月己亥,隕石於馮翊十有八。九月壬戌,張行成薨。甲戌,褚遂良為尚書右僕射。十月庚子,幸溫湯。甲辰,赦新豐。乙巳,至自溫湯。戊申,睦州女子陳碩真反,婺州刺史崔義玄討之。十一月庚戌,陳碩真伏誅。癸丑,兵部尚書崔敦禮為侍中。丁巳,柳奭為中書令。十二月庚子,高季輔薨。



 五年正月丙寅,以旱詔文武官、朝集使言事。三月戊午,如萬年宮。乙丑,次鳳泉湯。辛未,赦岐州及所過徒罪以下。六月癸亥,柳奭罷。丙寅,河北大水,遣使慮囚。八月己未,詔免麟游、岐陽今歲課役,岐州及供頓縣半歲。九月丁酉,至自萬年宮。十月癸卯,築京師羅郭,起觀於九門。



 六年正月壬申,拜昭陵,赦醴泉及行從,免縣今歲租、調,陵所宿衛進爵一級,令、丞加一階。癸酉,以少牢祭陪葬者。甲戌,至自昭陵。庚寅,封子弘為代王,賢潞王。二月乙巳,皇太子加元服,降死罪以下,賜酺三日,五品以上為父後者勛一轉。乙丑,營州都督程名振、左衛中郎將蘇定方伐高麗。五月壬午,及高麗戰於貴端水,敗之。癸未,左屯衛大將軍程知節為蔥山道行軍大總管,以伐賀魯。壬辰,韓瑗為侍中,來濟為中書令。七月乙酉,崔敦禮為中書令。是月,中書舍人李義府為中書侍郎,參知政事。九月庚午,貶褚遂良為潭州都督。乙酉,洛水溢。十月,齊州黃河溢。己酉,廢皇后為庶人。乙卯,立宸妃武氏為皇后。丁巳,大赦,賜民八十以上粟帛。十一月己巳,皇后見於太廟。戊子,停諸州貢珠。癸巳,詔禁吏酷法及為隱名書者。是冬,皇后殺王庶人。



 顯慶元年正月辛未,廢皇太子為梁王,立代王弘為皇太子。壬申,大赦,改元,賜五品以上子為父後者勛一轉,民酺三日,八十以上粟帛,丙戌,禁胡人為幻戲者。甲午,放宮人。三月辛巳,皇后親蠶。丙戌,戶部侍郎杜正倫為黃門侍郎、同中書門下三品。四月壬寅,詔五品以上老疾不以罪者同致仕。壬子,矩州人謝無零反,伏誅。七月癸未,崔敦禮為太子少師、同中書門下三品。八月丙申,崔敦禮薨。辛丑,程知節及賀魯部歌邏祿、處月戰於榆慕穀,敗之。九月庚辰,括州海溢。癸未,程知節及賀魯戰於怛篤城,敗之。十一月乙丑,以子顯生,賜京官、朝集使勛一轉。自八月霜且雨至於是月。是歲,龜茲大將羯獵顛附於賀魯,左屯衛大將軍楊胄伐之。



 二年閏正月壬寅,如洛陽宮。庚戌,右屯衛將軍蘇定方為伊麗道行軍總管,以伐賀魯。二月癸亥,降洛州囚罪,徒以下原之,免民一歲租、調,賜百歲以上氈衾粟帛。庚午,封子顯為周王。壬申,徙封素節為郇王。三月戊申,禁舅姑拜公主,父母拜王妃。癸丑,李義府兼中書令。五月丙申,幸明德宮。七月丁亥,如洛陽宮。八月丁卯,貶韓瑗為振州刺史,來濟為臺州刺史。辛未,衛尉卿許敬宗為侍中。九月庚寅,杜正倫兼中書令。十一月戊戌,如許州。甲辰,遣使慮所過州縣囚。乙巳,獵於滍南。壬子,講武於新鄭,赦鄭州,免一歲租賦,賜八十以上粟帛,其嘗事高祖任佐史者以名聞。十二月乙卯,如洛陽宮。丁巳,蘇定方敗賀魯於金牙山,執之。丁卯,以洛陽宮為東都。



 三年正月戊申,楊胄及龜茲羯獵顛戰於泥師域,敗之。二月甲戌,至自東都。戊寅,慮囚。六月壬子,程名振及高麗戰於赤烽鎮,敗之。十一月乙酉,貶杜正倫為橫州刺史,李義府普州刺史。戊子,許敬宗權檢校中書令。甲午,蘇定方俘賀魯以獻。戊戌,許敬宗為中書令,大理卿辛茂將兼侍中。



 四年三月壬午,昆陵都護阿史那彌射及西突厥真珠葉護戰於雙河,敗之。四月丙辰,於志寧為太子太師、同中書門下三品。乙丑,黃門侍郎許圉師同中書門下三品。戊辰,流長孫無忌於黔州。於志寧罷。五月己卯,許圉師為中書侍郎、同中書門下三品。丙申,兵部尚書任雅相、度支尚書盧承慶參知政事。戊戌,殺涼州都督長史趙持滿。七月己丑,以旱避正殿。壬辰,慮囚。八月壬子,李義府為吏部尚書、同中書門下三品。十月丙午,皇太子加元服,大赦,賜五品以上子孫為父祖後者勛一轉,民酺三日。閏月戊寅,如東都,皇太子監國。辛巳,詔所過供頓免今歲租賦之半,賜民八十以上氈衾粟帛。十一月丙午,許圉師為左散騎常侍、檢校侍中。戊午,辛茂將薨。癸亥,賀魯部悉結闕俟斤都曼寇邊,左驍衛大將軍蘇定方為安撫大使以伐之。盧承慶同中書門下三品。



 五年正月癸卯,蘇定方俘都曼以獻。甲子,如並州。己巳,次長平,賜父老布帛。二月丙戌,赦並州及所過州縣,義旗初嘗任五品以上葬並州者祭之,加佐命功臣食別封者子孫二階,大將軍府僚佐存者一階,民年八十以上版授刺史、縣令,賜酺三日。甲午,祠舊宅。三月丙午,皇后宴親族鄰里於朝堂,會命婦於內殿。賜從官五品以上、並州長史司馬勛一轉。婦人八十以上版授郡君,賜氈衾粟帛。己酉,講武於城西。辛亥,左武衛大將軍蘇定方為神兵道行軍總管,新羅王金春秋為嵎夷道行軍總管,率三將軍及新羅兵以伐百濟。四月癸巳,如東都。五月辛丑,作八關宮。戊辰,定襄都督阿史德樞賓為沙磚道行軍總管,以伐契丹。六月庚午朔,日有食之。七月乙巳,廢梁王忠為庶人。丁卯,盧承慶罷。八月庚辰,蘇定方及百濟戰,敗之。壬午,左武衛大將軍鄭仁泰及悉結、拔也固、僕骨同羅戰,敗之。癸未,赦神兵道大總管以下軍士及其家,賜民酺三日。十一月戊戌,蘇定方俘百濟王以獻。甲寅,如許州。十二月辛未,獵於安樂川。己卯,如東都。壬午,左驍衛大將軍契苾何力為浿江道行軍大總管,蘇定方為遼東道行軍大總管,左驍衛將軍劉伯英為平壤道行軍大總管,以伐高麗。阿史德樞賓及奚、契丹戰,敗之。



 龍朔元年正月戊午,鴻臚卿蕭嗣業為扶餘道行軍總管,以伐高麗。二月乙未,改元,赦洛州。四月庚辰,任雅相為浿江道行軍總管,契苾何力為遼東道行軍總管,蘇定方為平壤道行軍總管,蕭嗣業為扶餘道行軍總管,右驍衛將軍程名振為鏤方道行軍總管,左驍衛將軍龐孝泰為沃沮道行軍總管,率三十五軍以伐高麗。甲午晦,日有食之。六月辛巳,太白經天。八月甲戌,蘇定方及高麗戰於浿江,敗之。九月癸卯,及皇后幸李勣、許圉師第。壬子,徙封賢為沛王。十月丁卯,獵於陸渾。戊辰,獵於非山。癸酉,如東都。鄭仁泰為鐵勒道行軍大總管,蕭嗣業為仙崿道行軍大總管,左驍衛大將軍阿史那忠為長岑道行軍大總管,以伐鐵勒。



 二年二月甲子,大易官名。甲戌,任雅相薨。戊寅,龐孝泰及高麗戰於蛇水,死之。三月庚寅,鄭仁泰及鐵勒戰於天山,敗之。乙巳,如河北縣。辛亥,如蒲州。癸丑,如同州。四月庚申,至自同州。辛巳,作蓬萊宮。六月癸亥,禁宗戚獻纂組雕鏤。七月戊子,以子旭輪生滿月,大赦,賜酺三日。右威衛將軍孫仁師為熊津道行軍總管,以伐百濟。戊戌,李義府罷。八月壬寅,許敬宗為太子少師、同東西臺三品。九月丁丑,李義府起復。十月丁酉,幸溫湯,皇太子監國。丁未,至自溫湯。庚戌,西臺侍郎上官儀同東西臺三品。十一月辛未,貶許圉師為虔州刺史。癸酉,封子旭輪為殷王。是歲,右衛將軍蘇海政為蒨海道行軍總管,以伐龜茲。海政殺昆陵都護阿史那彌射。



 三年正月乙丑,李義府為右相。二月減百官一月俸,賦雍、同等十五州民錢,以作蓬萊宮。乙亥,殺駙馬都尉韋正矩。庚戌,慮囚。四月戊子,流李義府於巂州。五月壬午,柳州蠻叛,冀州都督長史劉伯英以嶺南兵伐之。六月,吐蕃攻吐谷渾,涼州都督鄭仁泰為青海道行軍大總管以救之。八月癸卯,有彗星出於左攝提。戊申,詔百寮言事。遣按察大使於十道。九月戊午,孫仁師及百濟戰於白江,敗之。十月辛巳,詔皇太子五日一至光順門,監諸司奏事,小事決之。十一月甲戌,雨木冰。十二月庚子,改明年為麟德元年,降京師、雍州諸縣死罪以下。壬寅,安西都護高賢為行軍總管,以伐弓月。



 麟德元年二月戊子,如福昌宮。癸卯,如萬年宮。四月壬午,道王元慶薨。五月戊申,許王孝薨。丙寅,以旱避正殿。七月丁未,詔以三年正月有事於泰山。八月己卯,幸舊第,降萬年縣死罪以下。壬午,至自萬年宮。丁亥,司列太常伯劉祥道兼右相,大司憲竇德玄為司元太常伯、檢校左相。十二月丙戌,殺上官儀。戊子,殺庶人忠。劉祥道罷。太子右中護樂彥瑋、西臺侍郎孫處約同知軍國政事。是冬,無雪。



 二年二月壬午如東都。三月甲寅,司戎太常伯姜恪同東酉臺三品。戊午,遣使慮京師諸司及雍、洛二州囚。閏月癸酉,日有食之。是春,疏勒、弓月、吐蕃攻於闐,酉州都督崔智辯、左武衛將軍曹繼叔救之。四月丙午,赦桂、廣、黔三都督府。丙寅,講武於邙山之陽。戊辰,左侍極陸敦信檢校右相,孫處約、樂彥瑋罷。七月己丑,鄧王元裕薨。十月壬戌,帶方州刺史劉仁軌為大司憲兼知政事。丁卯,如泰山。大有年。



 乾封元年正月戊辰,封於泰山。庚午,禪於社首,以皇后為亞獻。壬申,大赦,改元。賜文武官階、勛、爵。民年八十以上版授下州刺史、司馬、縣令,婦人郡、縣君;七十以上至八十,賜古爵一級。民酺七日,女子百戶牛酒。免所過今年租賦,給復齊州一年半、兗州二年。辛卯,幸曲阜,祠孔子,贈太師。二月己未,如亳州,祠老子,追號太上玄元皇帝,縣人宗姓給復一年。四月甲辰,至自亳州。庚戌,陸敦信罷。六月壬寅,高麗泉男生請內附,右驍衛大將軍契苾何力為遼東安撫大使,率兵援之。左金吾衛將軍龐同善、營州都督高偘為遼東道行軍總管,左武衛將軍薛仁貴、左監門衛將軍李謹行為後援。七月乙丑,徙封旭輪為豫王。庚午,劉仁軌兼右相。八月辛丑,竇德玄薨。丁未,殺始州刺史武惟良、淄州刺史武懷運。九月,龐同善及高麗戰,敗之。十二月己酉,李勣為遼東道行臺大總管,率六總管兵以伐高麗。



 二年正月丁丑,以旱避正殿,減膳,慮囚。二月丁酉,涪陵郡王愔薨。辛丑,禁工商乘馬。六月乙卯,西臺侍郎楊武戴至德、東臺侍郎李安期、司列少常伯趙仁本同東西臺三品。東臺舍人張文瓘參知政事。七月己卯,以旱避正殿。減膳,遣使慮囚。八月己丑朔,日有食之。辛亥,李安期罷。九月庚申,以餌藥,皇太子監國。辛未,李勣及高麗戰於新城,敗之。是歲,嶺南洞獠陷瓊州。



 總章元年正月壬子,劉仁軌為遼東道副大總管兼安撫大使、浿江道行軍總管。二月丁巳,皇太子釋奠於國學。戊寅,如九成宮。壬午,李勣敗高麗,克扶餘、南蘇、木底、蒼巖城。三月庚寅,大赦,改元。四月乙卯,贈顏回太子少師,曾參太子少保。丙辰,有彗星出於五車,避正殿;減膳,撤樂,詔內外官言事。庚申,以太原元從西府功臣為二等:第一功後官無五品者,授其子若孫一人,有至四品五品者加二階,有三品以上加爵三等;第二功後官無五品者,授其子若孫從六品一人,有至五品者加一階,六品者二階,三品以上爵一等。辛巳,楊武薨。八月癸酉,至自九成宮。九月癸巳,李勣敗高麗王高藏,執之。十二月丁巳,俘高藏以獻。丁卯,有事於南郊。甲戌,姜恪檢校左相,司平太常伯閻立本守右相。



 二年二月辛酉,右肅機李敬玄為西臺侍郎,張文瓘為東臺侍郎:同東西臺三品。三月丙戌,東臺侍郎郝處俊同東西臺三品。癸巳,皇后親蠶。四月己酉,如九成宮。六月戊申朔,日有食之。七月癸巳,左衛大將軍契苾何力為烏海道行軍大總管,以援吐谷渾。九月庚寅,括州海溢。壬寅,如岐州。乙巳,赦岐州,賜高年粟帛。十月丁巳,至自岐州。十一月丁亥,徙封旭輪為冀王,改名輪。十二月戊申,李勣薨。是冬,無雪。



 咸亨元年正月丁丑,劉仁軌罷。二月戊申,慮囚。丁巳,東南有聲若雷。三月甲戌,大赦,改元。壬辰,許敬宗罷。四月癸卯,吐蕃陷龜茲撥換城。廢安西四鎮。己酉,李敬玄罷。辛亥,右威衛大將軍薛仁貴為邏娑道行軍大總管,以伐吐蕃。庚午,如九成宮。雍州大雨雹。高麗酋長鉗牟岑叛,寇邊,左監門衛大將軍高偘為東州道行軍總管,右領軍衛大將軍李謹行為燕山道行軍總管,以伐之。六月壬寅朔,日有食之。七月甲戌,以雍、華、蒲、同四州旱,遣使慮囚,減中御諸廄馬。戊子,李敬玄起復。薛仁貴及吐蕃戰於大非川,敗績。八月庚戌,以穀貴禁酒。丁巳,至自九成宮。甲子,趙王福薨。丙寅,以旱避正殿,減膳。九月丁丑,給復雍、華、同、岐、邠、隴六州一年。閏月癸卯,皇后以旱請避位。甲寅,姜恪為涼州道行軍大總管,以伐吐蕃。十月庚辰,詔文武官言事。乙未,趙仁本罷。十二月庚寅,復官名。是歲,大饑。



 二年正月乙巳,如東都,皇太子監國。二月辛未,遣使存問諸州。四月戊子,大風,雨雹。六月癸巳,以旱慮囚。九月,地震。丙申,徐王元禮薨。十月丙子,求明禮樂之士。十一月甲午朔,日有食之。庚戌,如許州,遣使存問所過疾老鰥寡,慮囚。十二月癸酉,獵於昆陽。丙戌,如東都。是歲,姜恪為侍中,閻立本為中書令。



 三年正月辛丑,姚州蠻寇邊,太子右衛副率梁積壽為姚州道行軍總管以伐之。二月己卯,姜恪薨。四月壬申,校旗於洛水之陰。九月癸卯,徙封賢為雍王。十月己未,皇太子監國。十一月戊子朔,日有食之。甲辰,至自東都。十二月,金紫光祿大夫致仕劉仁軌為太子左庶子、同中書門下三品。



 四年正月丙辰,鄭王元懿薨。四月丙子,如九成宮。閏五月丁卯,禁作甗捕魚、營圈取獸者。八月辛丑,以不豫詔皇太子聽諸司啟事。己酉,大風落太廟鴟尾。十月壬午,閻立本薨。乙未,以皇太子納妃,赦岐州,賜酺三日。乙巳,至自九成宮。



 上元元年二月壬午,劉仁軌為雞林道行軍大總管,以伐新羅。三月辛亥朔,日有食之。己巳,皇后親蠶。八月壬辰,皇帝稱天皇,皇后稱天後。追尊六代祖宣簡公為宣皇帝,妣張氏曰宣莊皇后;五代祖懿王為光皇帝,妣賈氏曰光懿皇后。增高祖、太宗及後謚。大赦,改元,賜酺三日。十一月丙午,如東都。己酉,獨於華山曲武原。十二月癸未,蔣王惲自殺。



 二年正月己未,給復雍、同、華、岐、隴五州一年。辛未,吐蕃請和。二月,劉仁軌及新羅戰於七重城,敗之。三月丁巳,天后親蠶。四月辛巳,天後殺周王顯妃趙氏。丙戌,以旱避正殿,減膳,撤樂,詔百官言事。己亥,天後殺皇太子。五月戊申,追號皇太子為孝敬皇帝。六月戊寅,立雍王賢為皇太子,大赦。七月辛亥,杞王上金免官,削封邑。八月庚寅,葬孝敬皇帝於恭陵。丁酉,詔婦人為宮官者歲一見其親。庚子,張文瓘為侍中,郝處俊為中書令,劉仁軌為尚書左僕射,戴至德為右僕射。十月庚辰,雍州雨雹。壬午,有彗星出於角、亢。



 儀鳳元年正月壬戌,徙封輪為相王。丁卯,納州獠寇邊。二月丁亥,如汝州溫湯,遣使慮免汝州輕系。三月癸卯,黃門侍郎來恆、中書侍郎薛元超同中書門下三品。甲辰,如東都,免汝州今歲半租,賜民八十以上帛。閏月己巳。吐蕃寇鄯、廓、河、芳四州,左監門衛中郎將令狐智通伐之。乙酉,周王顯為洮河道行軍元帥,領左衛大將軍劉審禮等十二總管,相王輪為涼州道行軍元帥,領契苾何力等軍,以伐吐蕃。四月戊申,至自東都。甲寅,中書侍郎李義琰同中書門下三品。戊午,如九成宮。六月癸亥,黃門侍郎高智周同中書門下三品。七月丁亥,有彗星出於東井。乙未,吐蕃寇疊州。八月庚子,避正殿,減膳,撤樂,損食粟馬,慮囚,詔文武官言事。甲子,停南北中尚、梨園、作坊,減少府雜匠。是月,青州海溢。十月乙未,至自九成宮。丙午,降封郇王素節鄱陽郡王。十一月壬申,寺赦,改元。庚寅,李敬玄為中書令。十二月戊午,來恆、薛元超為河南、河北道大使。



 二年正月乙亥,耕藉田。庚辰,京師地震。四月,太子左庶子張大安同中書門下三品。五月,吐蕃寇扶州。八月辛亥,劉仁軌為洮河軍鎮守使。十月壬辰,徙封顯為英王,更名哲。十二月乙卯,募關內、河東猛士,以伐吐蕃。是歲,西突厥及吐蕃寇安西。冬,無雪。



 三年正月丙子,李敬玄為洮河道行軍大總管,以伐吐蕃。癸未,遣使募河南、河北猛士,以伐吐蕃。四月丁亥,以旱避正殿,慮囚。戊申,大赦,改明年為通乾元年。癸丑,涇州民生子異體連心。五月壬戌,如九成宮。大雨霖。九月辛酉,至自九成宮。癸亥,張文瓘薨。丙寅,李敬玄、劉審禮及吐蕃戰於青海,敗績,審禮死之。十月丙申,停劍南、隴右歲貢。丙午,密王元曉薨。閏十一月丙申,雨木冰。壬子,來恆薨。十二月癸丑,罷通乾號。



 調露元年正月戊子,如東都。庚戌,戴至德薨。四月辛酉,郝處俊為侍中。五月丙戌,皇太子監國。戊戌,作紫桂宮。六月辛亥,大赦,改元。吏部侍郎裴行儉伐西突厥。九月壬午,行儉敗西突厥,執其可汗都支。十月,突厥溫傅、奉職二部寇邊,單于大都護府長史蕭嗣業伐之。十一月戊寅,高智周罷。甲辰,禮部尚書裴行儉為定襄道行軍大總管,以伐突厥。



 永隆元年二月癸丑,如汝州溫湯。丁巳,如少室山。乙丑,如東都。三月,裴行儉及突厥戰於黑山,敗之。四月乙丑,如紫桂宮。戊辰,黃門侍郎裴炎、崔知溫,中書侍郎王德真:同中書門下三品。五月丁酉,太白經天。七月己卯,吐蕃寇河源。辛巳,李敬玄及吐蕃戰於湟川,敗績。左武衛將軍黑齒常之為河源軍經略大使。丙申,江王元祥薨。突厥寇雲州,都督竇懷哲敗之。八月丁未,如東都。丁巳,貶李敬玄為衡州刺史。甲子,廢皇太子為庶人。乙丑,立英王哲為皇太子,大赦,改元,賜酺三日。己巳,貶張大安為普州刺史。九月甲申,王德真罷。十月壬寅,降封曹王明為零陵郡王。戊辰,至自東都。十一月壬申朔,日有食之。



 開耀元年正月乙亥,突厥寇原、慶二州。辛巳,賜京官九品以上酺三日。癸巳,裴行儉為定襄道行軍大總管,以伐突厥。己亥,減殿中、太僕馬,省諸方貢獻,免雍、岐、華、同四州二歲稅,河南、河北一年調。二月丙午,皇太子釋奠於國學。三月辛卯,郝處俊罷。五月乙酉,常州人劉龍子謀反,伏誅。丙戌,定襄道副總管曹懷舜及突厥戰於橫水,敗績。己丑,黑齒常之及吐蕃戰於良非川,敗之。六月壬子,永嘉郡王晫有罪,伏誅。七月己丑,以太平公主下嫁,赦京師。甲午,劉仁軌罷左僕射。閏月丁未,裴炎為侍中,崔知溫、薛元超守中書令。庚戌,以餌藥,皇太子監國。庚申,裴行儉及突厥戰,敗之。八月丁卯,以河南、河北大水,遣使賑乏絕,室廬壞者給復一年,溺死者贈物,人三段。九月丙申,有彗星出於天市。壬戌,裴行儉俘突厥溫傅可汗、阿史那伏念以獻。乙丑,改元,赦定襄軍及諸道緣征官吏兵募。十月丙寅朔,日有食之。十一月癸卯,徙庶人賢於巴州。



 永淳元年二月癸未,以孫重照生滿月,大赦,改元,賜酺三日。是月,突厥車薄、咽面寇邊。三月戊午,立重照為皇太孫。四月甲子朔,日有食之。丙寅,如東都,皇太子監國。辛未,裴行儉為金牙道行軍大總管,率三總管兵以伐突厥。安西副都護王方翼及車薄、咽面戰於執海,敗之。丁亥,黃門侍郎郭待舉、兵部侍郎岑長倩、秘書員外少監郭正一、吏部侍郎魏玄同與中書門下同承受進止平章事。五月乙卯,洛水溢。六月甲子,突厥骨咄祿寇邊,嵐州刺史王德茂死之。是月,大蝗,人相食。七月,作萬泉宮。己亥,作奉天宮。庚申,零陵郡王明自殺。九月,吐蕃寇柘州,驍衛郎將李孝逸伐之。十月甲子,京師地震。丙寅,黃門侍郎劉齊賢同中書門下平章事。



 弘道元年正月甲午,幸奉天宮。二月庚午,突厥寇定州,刺史霍王元軌敗之。三月庚寅,突厥寇單于都護府,司馬張行師死之。庚子,李義琰罷。丙午,有彗星出於五車。癸丑,崔知溫薨。四月己未,如東都。壬申,郭待舉、郭正一同中書門下平章事。甲申,綏州部落稽白鐵餘寇邊,右武衛將軍程務挺敗之。五月乙巳,突厥寇蔚州,刺史李思儉死之。七月甲辰,徙封輪為豫王,改名旦。薛元超罷。八月乙丑,皇太子朝於東都,皇太孫留守京師。丁卯,滹沱溢。己巳,河溢,壞河陽城。九月己丑,以太平公主子生,赦東都。十月癸亥,幸奉天宮。十一月戊戌,右武衛將軍程務挺為單于道安撫大使,以伐突厥。辛丑,皇太子監國。丁未,如東都。戊申,裴炎、劉齊賢、郭正一兼於東宮平章事。十二月丁巳,改元,大赦。是夕,皇帝崩於貞觀殿,年五十六。謚曰天皇大帝。天寶八載,改謚天皇大聖皇帝;十三載,增謚天皇大聖大弘孝皇帝。



 贊曰:《小雅》曰:「赫赫宗周,褒姒滅之。」此周幽王之詩也。是時,幽王雖亡,而太子宜臼立,是為平王。而詩人乃言滅之者,以為文、武之業於是蕩盡,東周雖在,不能復興矣。其曰滅者,甚疾之之辭也。武氏之亂,唐之宗室戕殺殆盡,其賢士大夫不免者十八九。以太宗之治,其遺德餘烈在人者未遠,而幾於遂絕,其為惡豈一褒姒之比邪?以太宗之明,昧於知子,廢立之際,不能自決,卒用昏童。高宗溺愛衣任席,不戒履霜之漸,而毒流天下,貽禍邦家。嗚呼,父子夫婦之間,可謂難哉!可不慎哉?



\end{pinyinscope}