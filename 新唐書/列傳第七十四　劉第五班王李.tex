\article{列傳第七十四 劉第五班王李}

\begin{pinyinscope}

 劉晏,字士安,曹州南華人。玄宗封泰山,晏始八歲,獻頌行在,帝奇其幼,命宰相張說試之,說曰:「國瑞也。」即授太子正字。公卿邀請旁午,號神童,名震一時。天寶中,累調夏令,未嘗督賦,而輸無逋期。舉賢良方正,補溫令,所至有惠利可紀,民皆刻石以傳。再遷侍御史。祿山亂,避地襄陽。永王璘署晏右職,固辭。移書房琯,論封建與古異,「今諸王出深宮,一旦望桓、文功,不可致。」詔拜度支郎中,兼侍御史,領江淮租庸事。晏至吳郡而璘反,乃與採訪使李希言謀拒之。希言假晏守餘杭,會戰不利,走依晏。晏為陳可守計,因發義兵堅壁。會王敗,欲轉略州縣,聞晏有備,遂自晉陵西走。終不言功。召拜彭原太守,徙隴、華二州刺史,遷河南尹。時史朝義盜東都,乃治長水。進戶部侍郎,兼御史中丞、度支鑄錢鹽鐵等使。京兆尹鄭叔清、李齊物坐殘摯罷,詔晏兼京兆尹。總大體不苛,號稱職。會司農卿嚴莊下獄,已而釋,誣劾晏漏禁中語,宰相蕭華亦忌之,貶通州刺史。



 代宗立,復為京兆尹、戶部侍郎,領度支、鹽鐵、轉運、鑄錢、租庸使。晏以戶部讓顏真卿,改國子祭酒。又以京兆讓嚴武,即拜吏部尚書、同中書門下平章事,使如故。坐與程元振善,罷為太子賓客。俄進御史大夫,領東都、河南、江淮轉運、租庸、鹽鐵、常平使。時大兵後,京師米斗千錢,禁膳不兼時,甸農挼穗以輸。晏乃自按行,浮淮、泗,達於汴,入於河。右循底柱、硤石,觀三門遺跡;至河陰、鞏、洛,見宇文愷梁公堰,廝河為通濟渠,視李傑新堤,盡得其病利。然畏為人牽制,乃移書於宰相元載,以為:「大抵運之利與害各有四:京師三輔,苦稅入之重,淮、湖粟至,可減徭賦半,為一利;東都雕破,百戶無一存,若漕路流通,則聚落邑廛漸可還定,為二利;諸將有不廷,戎虜有侵盜,聞我貢輸錯入,軍食豐衍,可以震耀夷夏,為三利;若舟車既通,百貨雜集,航海梯嶠,可追貞觀、永徽之盛,為四利。起宜陽、熊耳,虎牢、成皋五百里,見戶才千餘,居無尺椽,爨無盛煙,獸游鬼哭,而使轉車輓漕,功且難就,為一病;河、汴自寇難以來,不復穿治,崩岸滅木,所在廞淤,涉泗千里,如罔水行舟,為二病;東垣、底柱,澠池、北河之間六百里,戍邏久絕,奪攘奸宄,夾河為藪,為三病;淮陰去蒲阪,亙三千里,屯壁相望,中軍皆鼎司元侯,每言衣無纊,食半菽,輓漕所至,輒留以饋軍,非單車使者折簡書所能制,為四病。」載方內擅朝權,既得書,即盡以漕事委晏,故晏得盡其才。歲輸始至,天子大悅,遣衛士以鼓吹迓東渭橋,馳使勞曰:「卿,朕酂侯也。」凡歲致四十萬斛,自是關中雖水旱,物不翔貴矣。



 再遷吏部尚書,又兼益湖南、荊南、山南東道轉運、常平、鑄錢使,與第五琦分領天下金穀。又知吏部三銓事,推處最殿分明,下皆心習伏。元載得罪,詔晏鞫之。晏畏載黨盛,不敢獨訊,更敕李涵等五人與晏雜治。王縉得免死,晏請之也。



 常袞執政,忌晏有公望,乃言晏舊德,當師長百僚,用為左僕射,實欲奪其權。帝以計務方治,詔以僕射領使如舊。初,晏分置諸道租庸使,慎簡臺閣士專之。時經費不充,停天下攝官,獨租庸得補署,積數百人,皆新進銳敏,盡當時之選,趣督倚辦,故能成功。雖權貴幹請,欲假職仕者,晏厚以稟入奉之,然未嘗使親事,是以人人勸職。嘗言:「士有爵祿,則名重於利;吏無榮進,則利重於名。」故檢劾出納,一委士人,吏惟奉行文書而已。所任者,雖數千里外,奉教令如目前,頻伸諧戲不敢隱。惟晏能行之,它人不能也。代宗嘗命考所部官吏善惡,刺史有罪者,五品以上輒系劾,六品以下杖然後奏。



 李靈耀反,河南節帥或不奉法,擅徵賦,州縣益削。晏常以羨補乏,人不加調,而所入自如。第五琦始權鹽佐軍興,晏代之,法益密,利無遺入。初,歲收緡錢六十萬,末乃什之,計歲入千二百萬,而榷居太半,民不告勤。京師鹽暴貴,詔取三萬斛以贍關中,自揚州四旬至都,人以為神。至湖嶠荒險處,所出貨皆賤弱,不償所轉,晏悉儲淮、楚間,貿銅易薪,歲鑄緡錢十餘萬。其措置纖悉如此。諸道巡院,皆募駛足,置驛相望,四方貨殖低昂及它利害,雖甚遠,不數日即知,是能權萬貨重輕,使天下無甚貴賤而物常平,自言如見錢流地上。每朝謁,馬上以鞭算。質明視事,至夜分止,雖休澣不廢。事無閑劇,即日剖決無留。所居修行里,粗樸庳陋,飲食儉狹,室無媵婢。然任職久,勢軋宰相,要官華使多出其門。自江淮茗橘珍甘,常與本道分貢,競欲先至,雖封山斷道,以禁前發,晏厚貲致之,常冠諸府,由是媢怨益多。饋謝四方有名士無不至,其有口舌者,率以利啖之,使不得有所訾短。故議者頗言晏任數固恩。大歷時政因循,軍國皆仰晏,未嘗檢質。德宗立,言者屢請罷轉運使,晏亦固辭,不許。又加關內河東三川轉運、鹽鐵及諸道青苗使。



 始,楊炎為吏部侍郎,晏為尚書,盛氣不相下。晏治元載罪,而炎坐貶。及炎執政,銜宿怒,將為載報仇。先是,帝居東宮,代宗寵獨孤妃,而愛其子韓王。宦人劉清潭與嬖幸請立妃為後,且言王數有符異,以搖東宮。時妄言晏與謀。至是,炎見帝流涕曰:「賴祖宗神靈,先帝與陛下不為賊臣所間,不然,劉晏、黎幹搖動社稷,兇謀果矣。今干伏辜而晏在,臣位宰相,不能正其罪,法當死。」崔祐甫曰:「陛下已廓然大赦,不當究飛語,致人於罪。」硃泚、崔寧力相解釋,寧尤切至。炎怒,斥寧於外,遂罷晏使。坐新故所交簿物抗謬,貶忠州刺史,中官護送。炎必欲傅其罪,知庾準與晏素憾,乃擢為荊南節度使。準即奏晏與硃泚書,語言怨望,又搜卒,擅取官物,脅詔使,謀作亂。炎證成之。



 建中元年七月,詔中人賜晏死,年六十五。後十九日,賜死詔書乃下,且暴其罪。家屬徙嶺表,坐累者數十人,天下以為冤。時炎兼刪定使,議籍沒,眾論不可,乃止。然已命簿錄其家,唯雜書兩乘,米麥數斛,人服其廉。淄青節度使李正己表誅晏太暴,不加驗實,先誅後詔,天下駭惋,請還其妻子。不報。興元初,帝浸寤,乃許歸葬。貞元五年,遂擢晏子執經為太常博士,宗經秘書郎。執經還官,求追命,有詔贈鄭州刺史,又加司徒。



 晏歿二十年,而韓洄、元琇、裴腆、李衡、包佶、盧徵、李若初繼掌財利,皆晏所闢用,有名於時。



 晏既被誣,而舊吏推明其功。陳諫以為管、蕭之亞,著論紀其詳,大略以「開元、天寶間天下戶千萬,至德後殘於大兵,饑疫相仍,十耗其九,至晏充使,戶不二百萬。晏通計天下經費,謹察州縣災害,蠲除振救,不使流離死亡。初,州縣取富人督漕輓,謂之『船頭』;主郵遞,謂之『捉驛』;稅外橫取,謂之『白著』。人不堪命,皆去為盜賊。上元、寶應間,如袁晁、陳莊、方清、許欽等亂江淮,十餘年乃定。晏始以官船漕,而吏主驛事,罷無名之斂,正鹽官法,以裨用度。起廣德二年,盡建中元年,黜陟使實天下戶,收三百餘萬。王者愛人,不在賜與,當使之耕耘織紝,常歲平斂之,荒年蠲救之,大率歲增十之一。而晏尤能時其緩急而先後之。每州縣荒歉有端,則計官所贏,先令曰:『蠲某物,貸某戶。』民未及困,而奏報已行矣。議者或譏晏不直賑救,而多賤出以濟民者,則又不然。善治病者,不使至危憊;善救災者,勿使至賑給。故賑給少則不足活人,活人多則闕國用,國用闕則復重斂矣;又賑給近僥幸,吏下為奸,強得之多,弱得之少,雖刀鋸在前不可禁。以為二害。災沴之鄉,所乏糧耳,它產尚在,賤以出之,易其雜貨,因人之力,轉於豐處,或官自用,則國計不乏;多出菽粟,恣之糶運,散入村閭,下戶力農,不能詣市,轉相沾逮,自免阻饑,不待令驅。以為二勝。晏又以常平法,豐則貴取,饑則賤與,率諸州米嘗儲三百萬斛。豈所謂有功於國者邪!」



 琇後以尚書右丞判度支,國無橫斂而軍旅濟。為韓滉所惡,貶雷州司戶參軍。坐私入廣州,賜死。腆以兵部侍郎判度支,封聞喜縣公。衡歷戶部侍郎。



 佶字幼正,潤州延陵人。父融,集賢院學士,與賀知章、張旭、張若虛有名當時,號「吳中四士」。佶擢進士第,累官諫議大夫。坐善元載,貶嶺南。晏奏起為汴東兩稅使。晏罷,以佶充諸道鹽鐵輕貨錢物使,遷刑部侍郎,改秘書監,封丹陽郡公。



 徵,幽州人。晏薦為殿中侍御史。晏得罪,貶珍州司戶參軍。元琇判度支,薦為員外郎。琇得罪,貶秀州長史,三遷給事中。戶部侍郎竇參善之,方倚以代己,會同州刺史缺,參請用尚書左丞趙憬,德宗惡參,欲間其腹心,更用徵為之。久乃徙華州,厚結權近,冀進用。同、華地迫而貧,所獻嘗觳陋,至徵厚賦斂,有所奉入,輒加常數,人不堪其求。



 若初者,事晏為冗職,包佶稱之。歷太康令,勸刺史李芃斂羨錢,交權幸,芃厚遇之。累遷浙東觀察使。代王緯為浙西觀察、諸道鹽鐵使。時天下錢少貨輕,州縣禁錢不出境,商賈不通。若初始奏縱錢以起萬貨,詔可。而持剛檢下,吏民畏服。卒,贈禮部尚書。



 宗經終給事中、華州刺史。子濛,字仁澤。舉進士,累官度支郎中。會昌初,擢給事中。以材為宰相李德裕所知。時回鶻衰,朝廷經略河、湟,建遣濛按邊,調兵械糧餉,為宣慰靈夏以北黨項使。始議造木牛運。宣宗立,德裕得罪,濛貶朗州刺史,終大理卿。



 晏兄暹,為汾州刺史。天資疾惡,所至以方直為觀察使所畏。建中末,召為御史大夫。宰相盧杞憚其嚴,更薦前河南尹於頎代之。暹終潮州刺史。



 頎字休明,河南人。初為京兆士曹參軍,尹史翽器之。翽鎮山南東道,表為判字。翽死亂兵手,頎挺出收葬之,時稱其誼。累遷京兆尹,任機譎,為政煩碎無大體,元載暱厚之。載得罪,出鄭州刺史,徙河南尹,以佞柔,故得為大夫。三遷工部尚書,入朝,僕金吾仗下,御史劾之,以太子少師致仕,卒。



 暹孫潼,字子固。擢進士第,杜悰判度支,表為巡官,累遷祠部郎中。大中初,討黨項羌,軍食乏,宰相欲以潼為使,難其遣。潼見宰相曰:「上念邊饋,議遣使,潼畏不稱耳,安敢憚行?」遂命為供軍使。會復河、湟,調師屯守,以潼判度支河、湟供軍案。歷京兆少尹。山南有劇賊,依山為剽,宣宗怒,欲討之,宰相崔鉉曰:「此陛下赤子,迫於饑寒,弄兵山谷間,不足討,請遣使喻釋之。」詔潼馳往。澗挺身直叩其壘曰:「有詔赦爾罪。」盜皆列拜,約潼就館而降。會山南節度使封敖遣兵擊賊,潼罷歸。



 數陳邊事,擢右諫議大夫。出為朔方、靈武節度使。坐累貶鄭州刺史,改湖南觀察使。召為左散騎常侍。拜昭義節度使,徙河東,又徙西川。時李福討南詔,兵不利,潼至,填以恩信,蠻皆如約。六姓蠻持兩端,為南詔間候。有卑籠部落者請討之,潼因出兵襲擊,俘五千人。南詔大懼,自是不敢犯邊。以功加檢校尚書右僕射。卒,贈司空。



 第五琦,字禹珪,京兆長安人。少以吏幹進,頗能言強國富民術。天寶中,事韋堅。堅敗,不得調。久之,為須江丞,太守賀蘭進明才之。安祿山反,進明徙北海,奏琦為錄事參軍事。時賊已陷河間、信都,進明未戰,玄宗怒,遣使封刀趣之,曰:「不亟進兵,即斬首。」進明懼,不知所出。琦勸厚以財募勇士,出賊不意。如其計,復收所陷郡。



 肅宗駐彭原,進明遣琦奏事,既謁見,即陳:「今之急在兵,兵強弱在賦,賦所出以江淮為淵。若假臣一職,請悉東南寶貲,飛餉函、洛,惟陛下命。」帝悅,拜監察御史、句當江淮租庸使。遷司虞員外郎、河南等五道支度使。遷司金郎中,兼侍御史、諸道鹽鐵鑄錢使。鹽鐵名使,自琦始。進度支郎中,兼御史中丞。當軍興,隨事趣辦,人不益賦而用以饒,於是遷戶部侍郎、判度支,河南等道支度、轉運、租庸、鹽鐵、鑄錢、司農、太府出納、山南東西、江西、淮南館驛等使。乾元二年,進同中書門下平章事。



 初,琦請鑄乾元重寶錢,以一代十。既當國,又鑄重規,一代五十。會物價騰踴,餓饉相望,議者以為非是,詔貶忠州長史。會有告琦納金者,遣御史馳按,琦辭曰:「位宰相,可自持金邪?若付受有狀,請歸罪有司。」御史不曉,以為具服,獄上之,遂長流夷州。



 寶應初,起為朗州刺史,有異政,拜太子賓客。吐蕃盜京師,郭子儀表為糧料使,兼御史大夫、關內元帥副使。改京兆尹。俄加判度支、鑄錢、鹽鐵、轉運、常平等使。累封扶風郡公。復以戶部侍郎兼京兆尹。坐與魚朝恩善,貶括州刺史。徙饒、湖二州。復為太子賓客、東都留守。德宗素聞其才,將復用,召之。會卒,年七十一,贈太子少保。子峰、婦鄭,皆以孝著,表闕於門。



 班宏,衛州汲人。父景倩,國子祭酒,以儒名家。宏,天寶中擢進士第,調右司御胄曹參軍。高適鎮劍南,表為觀察判官。青城人以左道惑眾,謀作亂。事覺,誣引屯將規緩死,眾兇懼,宏驗治,即殺之,人心大安。郭英乂代適,表雒令,以病解。



 大歷中,擢起居舍人,四遷給事中。李寶臣死,子惟岳匿喪求節度,帝遣宏使成德喻其軍,惟岳厚獻遣,宏不納,還報稱旨,擢刑部侍郎、京官考使。右僕射崔寧署兵部侍郎劉乃為上下考,宏不從,曰:「今軍在節度,雖有尺籍伍符,省署不校也。夫上多虛美,則下趨競;上阿容,則下朋黨。」因削之。乃聞,謝曰:「敢掠一美以邀二罪乎?」進吏部侍郎。



 貞元初,仍旱蝗,賦調益急,以戶部侍郎副度支使韓滉。俄而竇參當國,代滉使。而參任大理司直時,宏已為刑部侍郎。德宗以宏熟天下計,故進宏尚書副參,且曰:「朕藉宰相重,而眾務一委卿,無庸辭。」參亦以宏素貴,私謂曰;「閱歲當歸使於公。」宏喜。後參胖自安,不念前語。宏剛愎,以參欺己,議事稍不合。揚子院,鹽鐵轉運之委藏也,宏任御史中丞徐粲主之,粲以賄聞,參議所代,宏固不可。參選諸院吏,未始訪宏,宏數條參所用吏過惡以聞,輒留中。無何,參以使勞,加吏部尚書,而封宏蕭國公。恨參以虛寵加己,銜之。每制旨有所營建,必極瑰麗,親程役,媚結權嬖以傾參。



 張滂先善於宏,薦為司農少卿。及參欲滂分掌江、淮鹽鐵,宏以滂疾惡,且以法繩粲,因謬曰:「滂強戾不可用。」滂聞,不喜。久之,參知帝遇己薄,乃讓使,然不欲宏專,問策於京兆尹薛玨,玨曰:「滂與宏交惡,而滂剛決。若分鹽鐵轉運,必能制宏。」參遂薦滂為戶部侍郎、鹽鐵轉運使,而以宏判度支,分滂關內、河東、劍南、山南西道鹽鐵轉運隸宏,以悅其意。又還江淮兩稅,置巡院官,令宏、滂共差擇。滂欲得簿最,宏不與。及署院官,更持可否不能定,處處官乏不補。滂奏言:「臣職不修,無逃死,如國家大計何?」由是有詔分掌。宏見宰相辭曰:「宏主漕,歲得江、淮米五十萬斛,前年至七十萬。今職移於人,敢請罪。」滂在側儳曰:「公所言非也。朝廷不奪公職,乃公喪官緡,縱奸吏,自取咎爾。凡為度支使,不一歲家輒鉅億,僮馬產第侈王公,非盜縣官財何以然?上既知之,故令滂分掌。今公無乃歸怨上乎?」宏不答,於是移病歸第。宰相白其狀,詔許如劉晏、韓滉故事,以東都、河南、淮南、江南、山南東道兩稅,滂主之,東渭橋以東巡院隸焉;關內、河東、劍南、山南西道宏主之。滂至揚州,乃窮劾粲,悉發其贓至鉅萬,徙死嶺表。



 宏清潔勤力,晨入官署夕而出,吏不堪其勞,而己益恭。參得罪,宏為有力。卒,年七十三,贈尚書右僕射,謚曰敬。後二年,滂亦罷為衛尉卿。



 王紹,本名純,避憲宗諱改焉。自太原徙京兆之萬年。父端,第進士,有名天寶間,與柳芳、陸據、殷寅友善。據嘗言:「端之莊,芳之辯,寅之介,可以名世。」終工部員外郎。



 紹少為顏真卿所器,字之曰德素,奏為武康尉。再佐蕭復府。包佶領租庸、鹽鐵使,署判官。時李希烈阻兵江淮,輸物留梗,乃徙餉道自潁入汴。紹及關,德宗已西狩,乃督輕貨趣間道走洋州。紹先見行在,帝勞之曰:「吾軍乏春服,朕且衣裘,奈何?」紹流涕曰:「佶遣臣貢奉,無慮五十萬,當即至。」帝曰:「道回遠,經費方急,何可望邪?」後五日繼至,由是紓難。遷倉部員外郎。是時,兵旱無年,詔戶部收闕官俸、稅茶及無名錢,以修荒政。紹由員外郎判務,遷戶部、兵部郎中,皆專領。進戶部侍郎,判度支,頃之遷尚書。德宗臨禦久,益不假借宰相,自竇參、陸贄斥罷,中書取充位,惟紹謹密,眷待殊厚。主計凡八年,每政事多所關訪,紹亦未嘗一言漏於人。



 順宗立,王叔文奪其權,拜兵部尚書,出為東都留守。元和初,檢校尚書右僕射,為武寧軍節度使,復以濠、泗二州隸其軍。自張愔後,兵驕難治,紹搜輯軍政,推誠示人,裨將安進達、唐重靖謀亂,紹以計取之,出家貲賞士,舉軍安賴。復拜兵部尚書,判戶部。卒,年七十二,贈右僕射,謚曰敬。



 李巽,字令叔,趙州贊皇人。以明經補華州參軍事,舉拔萃,授鄠尉。進累左司郎中、常州刺史,召拜給事中,出為湖南觀察使。貞元五年,徙江西。巽銳於為治,持下以法,察無遺私,吏不敢少紿。順宗立,擢兵部侍郎。杜佑表為鹽鐵、轉運副使,俄代佑。使任自劉晏後,職廢不振,賦入朘耗。巽涖職一年,較所入如晏最多之年,明年過之,又明年,增百八十萬緡。再遷吏部尚書。



 天資長於吏事,至治家,亦句檢案牘簿書如公府。史有過,秋毫無所縱,股慄脅息,常如與巽對。程異坐王叔文廢,巽特薦引之。異之計較精於巽,故巽能善職,蓋有助云。元和四年疾革,郎官省候,巽言不及病,但與商校程課功利。是夕卒,年六十三,贈尚書右僕射。



 巽為人忌刻校怨,在江西,有所憎恨輒殺之。始,竇參為相,出巽常州,促其行。及參貶郴州,巽時觀察湖南,宣武節度使劉士寧致絹數千匹於參,巽即劾參交通籓鎮,以怒德宗,遂殺參雲。



 贊曰:生人之本,食與貨而已。知所以取,人不怨;知所以予,人不乏。道御之而王,權用之而霸,古今一也。劉晏因平準法,斡山海,排商賈,制萬物低昂,常操天下贏貲,以佐軍興。雖拿兵數十年,斂不及民而用度足。唐中僨而振,晏有勞焉,可謂知取予矣。其經晏闢署者,皆用材顯,循其法,亦能富國云。



\end{pinyinscope}