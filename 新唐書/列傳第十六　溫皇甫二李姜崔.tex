\article{列傳第十六 溫皇甫二李姜崔}

\begin{pinyinscope}

 溫大雅,字彥弘,並州祁人。父君攸,北齊文林館學士,入隋為泗州司馬的構造》、《論音樂的分類》等。,見朝政不綱,謝病歸。大雅性至孝,與弟彥博、大有皆知名,薛道衡見之,嘆曰:「三人者,皆卿相才也。」初為東宮學士、長安尉,以父喪解,會天下亂,不復仕。



 高祖鎮太原,厚禮之。兵興,引為大將軍府記室參軍,主文檄。帝受禪,與竇威、陳叔達討定儀典,遷黃門侍郎,而彥博亦為中書侍郎,對管華近。帝嘗從容謂曰:「我起晉陽,為卿一門耳。」進工部侍郎、陜東道大行臺尚書。隱太子圖亂,秦王表大雅鎮洛陽須變,數陳秘畫,多所嘉納。王即位,轉禮部,封黎國公。改葬其祖,卜人占其地,曰:「弟則吉,不利於君,若何?」大雅曰:「如子言,我含笑入地矣。」歲餘卒,謚曰孝。永徽五年,贈尚書右僕射。



 彥博字大臨,通書記,警悟而辯。開皇末,對策高策,授文林郎,直內史省。隋亂,幽州總管羅藝引為司馬。藝以州降,彥博與有謀,授總管府長史,封西河郡公。召入為中書舍人,遷侍郎。高麗貢方物,高祖欲讓而不臣,彥博執不可,曰:「遼東本周箕子國,漢玄菟郡,不使北面,則四夷何所瞻仰?」帝納而止。



 突厥入寇,彥博以並州道行軍長史戰太谷,王師敗績,被執。突厥知近臣,數問唐兵多少及國虛實,彥博不肯對,囚陰山苦寒地。太宗立,突厥歸款,得還。授雍州治中,尋檢校吏部侍郎。彥博欲汰擇士類,寡術不能厭眾,訟牒滿廷,時譏其煩碎。復為中書侍郎,遷御史大夫,檢校中書侍郎事。貞觀四年,遷中書令,封虞國公。突厥降,詔議所以安邊者,彥博請如漢置降匈奴五原塞,以為捍蔽,與魏徵廷爭,徵不勝其辯,天子卒從之。其後突利可汗弟結社謀反,帝始悔云。



 彥博善辭令,每問四方風俗,臚布誥命,若成誦然;進止詳華,人皆拭目觀。高祖嘗宴近臣,遣秦王諭旨,既而顧左右曰:「何如溫彥博?」十年,遷尚書右僕射,明年卒,年六十三。



 彥博性周慎,既掌機務,謝賓客不通,進見必陳政事利害。卒後,帝嘆曰:「彥博以憂國故,耗思殫神,我見其不逮再期矣,恨不許少閑以究其壽。」家貧無正寢,殯別室,帝命有司為構寢。贈特進,謚曰恭,陪葬昭陵。



 子振、挺。振歷太子舍人,居喪以毀卒。挺尚千金公主,官延州刺史。彥博曾孫曦,尚涼國長公主。



 大有,字彥將。隋仁壽中,李綱薦之,授羽林騎尉。高祖舉兵,引為太原令。從秦王徇西河,將行,高祖曰:「士馬單少,要須經略,以君參軍事,事之濟否,卜是行也。」西河下,攝大將軍府記室,與兄大雅同掌機近,不自安,請徙它職。帝曰:「我虛心待卿,何所自疑?」武德初,累遷中書侍郎,封清河郡公。卒,贈鴻臚卿,謚曰敬。初,顏氏、溫氏在隋最盛,思魯與大雅俱事東宮,愍楚、彥博同直內史省,游秦、大有典校秘閣,顏以學業優,而溫以職位顯於唐云。



 大雅四世孫佶,字輔國,以字行。安祿山亂,往見平原太守顏真卿,助為守計。李光弼厚遇之。後居鄴,薛嵩薦之朝,授太常丞,一謝嵩即去,屏處郊野,世推其高節。



 子造。造,字簡輿,姿表瑰傑,性嗜書,然盛氣,少所降屈。不喜為吏,隱王屋山,人號其居曰「處士墅」。壽州刺史張建封聞其名,書幣招禮,造欣然曰:「可人也!」往從之。建封雖咨謀,而不敢縻以職事。及節度徐州,造謝歸下邳,慨然有高世心。建封恐失造,因妻以兄子。



 時李希烈反,攻陷城邑,天下兵鎮陰相撼,逐主帥自立,德宗患之。以劉濟方納忠於朝,密詔建封擇縱橫士往說濟,佐其必。建封強署造節度參謀,使幽州。造與濟語未訖,濟俯伏流涕曰:「僻陋不知天子神聖,大臣盡忠,願率先諸侯效死節。」造還,建封以聞,詔馳馹入奏。天子愛其才,問造家世及年,對曰:「臣五世祖大雅,外五世祖李勣,臣犬馬之齒三十有二。」帝奇之。將用為諫官,以語洩乃止。復去,隱東都。烏重胤奏致幕府。



 長慶初,以京兆司錄為太原幽鎮宣諭使,召見,辭曰:「臣,府縣吏也,不宜行,恐四方易朝廷。」穆宗曰:「朕東宮時聞劉總,比年上書請覲,使問行期,乃不報。卿為我行喻意,毋多讓。」因賜緋衣。至範陽,總橐鞬郊迎。造為開示禍福,總懼,矍然若兵在頸,繇是籍所部九州入朝。還,遷殿中侍御史。田弘正遇害,以起居舍人復宣慰鎮州行營。



 頃之,李景儉以酒得過宰相,造坐與飲,出為朗州刺史。開後鄉渠百里,溉田二千頃,民獲其利,號「右史渠」。召授侍御史,知彈奏。請復硃衣豸冠示外廡,不聽。夏州節度使李祐拜大金吾,違詔進馬,造正衙抨劾。祐曰:』吾夜入蔡州擒吳元濟,未嘗心動,今日膽落於溫御史。」遷左司郎中,知御史雜事,進中丞。



 大和二年,內昭德寺火,延禁中「野狐落」,野狐落者,宮人所居也,死者數百人。是日,宰相、兩省官、京兆尹、中尉、樞密皆集日華門,督神策兵救火所及,獨御史府不至。造自劾曰:「臺系賊,恐人緣以構奸,申警備,乃得入。臣請入三十直,崔蠡、姚合二十直,自贖。」宰相劾造不待罪於朝,而自許輕比,不可聽。有詔皆奪一月俸。



 造性剛急,人或忤己,雖貴勢,亦以氣出其上。道遇左補闕李虞,恚不避,捕從者笞辱。左拾遺舒元褒等建言:「故事,供奉官惟宰相外無屈避。造棄蔑典禮,無所畏,辱天子侍臣。凡事小而關分理者,不可失;失之,則亂所由生。遺、補雖卑,侍臣也,中丞雖高,法吏也;侍臣見陵則恭不廣,法吏自恣則法壞。聞元和、長慶時,中丞呵止不半坊,今乃至兩坊,謂之籠街。造擅自尊大,忽僭擬之嫌,請得論罪。」帝乃詔臺官、供奉官共道路,聽先後行,相值則揖。中丞傳呼不得過三百步。造彈擊無所回畏,威望隱然,發南曹偽官九十人,主史皆論死。遷尚書右丞,封祁縣子。



 興元軍亂,殺李絳,眾謂造可夷其亂,文宗亦以為能,乃授檢校右散騎常侍、山南西道節度使,許以便宜從事。帝慮其勞費,造曰:「臣計諸道戍蠻之兵方還,願得密詔受約束,用此足矣。」許之。命神策將董仲質、河中將溫德彞、郃陽將劉士和從造。而興元將衛志忠、張丕、李少直自蜀還,造喻以意,皆曰:「不敢二。」乃用八百人自從,五百人為前軍。既入,前軍呵護諸門。造至,欲大宴,視聽事,曰:「此隘狹,不足饗士。」更徙牙門。坐定,將卒羅拜,徐曰:「吾欲聞新軍去主意,可悉前,舊軍無得進。」勞問畢,就坐,酒行,從兵合,卒有覺者,欲引去,造傳言叱之,乃不敢動。即問軍中殺絳狀,志忠、丕夾階立,拔劍傳呼曰:「悉殺之!」圍兵爭奮,皆斬首,凡八百餘人。親殺絳者,醢之;號令者,殊死。取百級祭絳,三十級祭死事官王景延等,餘悉投之漢江。監軍楊叔元擁造靴祈哀,造以兵衛出之。詔流康州。叔元,始激兵亂者也,人以造不戮為恨。以功加檢校禮部尚書,賜萬縑賞其兵。



 入為兵部侍郎,以病自言,出東都留守。俄節度河陽。奏復懷州古秦渠枋口堰,以溉濟源、河內、溫、武陟四縣田五千頃。召為御史大夫。方倚以相,會疾,不能朝,改禮部尚書。卒,年七十,贈尚書右僕射。



 兄邈,弟遜。邈,長慶、大和中,累以拾遺、補闕召,不應。遜嘗為邑宰,解印綬去。



 造子璋。璋以父廕累官大理丞。陰平吏盜官物,而焚其帑,璋刺得其情,擢侍御史,賜緋衣。遷婺州刺史,以政有績,賜金紫。徙廬、宋二州刺史。宣州逐鄭薰也,崔弦調淮南兵討之,以璋為宣州刺史。事平,就拜觀察使,擢武寧節度使。銀刀軍驕橫,累將姑息,而璋政嚴明,懼之,相率逐璋,詔徙邠寧節度,歷京兆尹。璋素強幹,鉏宿弊,豪右懾服,加檢校吏部尚書。同昌公主薨,懿宗誅醫無狀者,系親屬三百餘人。璋與劉瞻極諫,貶振州司馬,嘆曰:「生不逢時,死烏足惜!」仰藥死。



 彥博裔孫廷筠,少敏悟,工為辭章,與李商隱皆有名,號「溫李」。然薄於行,無檢幅。又多作側辭艷曲,與貴胄裴諴、令狐滈等蒲飲狎暱。數舉進士不中第。思神速,多為人作文。大中末,試有司,廉視尤謹,廷筠不樂,上書千餘言,然私占授者已八人,執政鄙其為,授方山尉。徐商鎮襄陽,署巡官,不得志,去歸江東。令狐綯方鎮淮南,廷筠怨居中時不為助力,過府不肯謁。丐錢揚子院,夜醉,為邏卒擊折其齒,訴於綯。綯為劾吏,吏具道其汙行,綯兩置之。事聞京師,廷筠遍見公卿,言為吏誣染。俄而徐商執政,頗右之,欲白用。會商罷,楊收疾之,遂廢卒。本名岐,字飛卿。



 弟廷皓,咸通中,署徐州觀察使崔彥曾幕府。龐勛反,以刃脅廷皓,使為表求節度使,廷皓紿曰:「表聞天子,當為公信宿思之。」勛喜。歸與妻子決,明日復見,勛索表,倨答曰:「我豈以筆硯事汝邪?其速殺我。」勛熟視笑曰:「儒生有膽邪,吾動眾百萬,無一人操檄乎!」囚之,更使周重草表。彥曾遇害,廷皓亦死,詔贈兵部郎中。



 皇甫無逸,字仁儉,京兆萬年人。父誕,隋並州總管府司馬,漢王諒反,逼之不從,見殺。無逸在長安,聞變即號慟,人問故,對曰:「吾父生平重節義,必無茍免者。頃訃至,果然。時五等廢,煬帝嘉誕忠,特封無逸平輿侯,而贈誕柱國、弘義郡公。



 無逸歷淯陽太守,治為天下最,再遷右武衛將軍。帝幸江都,詔居守洛陽。帝被殺,乃與段達、元文都立越王侗。及王世充篡,棄母妻,斬關自歸。追騎及,無逸顧曰:「吾有死,終不能同爾為逆。」解金帶投之地,曰:「以與爾,無相困。」騎爭下取,由是獲免。



 高祖以無逸本隋勛舊,尊遇之,拜刑部尚書,封滑國公。歷陜東道行臺民部尚書,遷御史大夫。時蜀新定,吏多橫恣,人不聊,詔無逸持節巡撫,得承制除吏。既至,黜貪暴,用廉善,法令嚴明,蜀人以安。



 皇甫希仁,憸人也,誣告無逸為母故陰交世充,帝判其詐,斬希仁,遣給事中李公昌馳諭。又有告無逸交通蕭銑者,時無逸與行臺僕射竇璡不協,因表自陳,並上璡罪。有詔劉世龍、溫彥博按之,無狀,遂斬告者而黜璡。及還,帝勞曰:「比多譖毀,但以正直為佞人憎爾。」無逸頓首謝,帝曰:「卿無負,何所謝?」



 拜民部尚書,出為同州刺史,徙益州大都督府長史。所至輒閉閤不通賓客,左右無敢出入者;所須皆市易它境。嘗按部,宿民家,鐙炷盡,主人將續進,無逸抽佩刀斷帶為炷,其廉介類如此。然過自畏慎,每上表疏,讀數十猶懼未審,使者上道,追省再三乃得遣。母在長安疾篤,太宗命馳驛召還承問,憂悸不能食,道病卒。贈禮部尚書,謚曰孝。王珪駁曰:「無逸入蜀,不能與母俱,留卒京師,子道未足稱,不可謂孝。」乃更謚良。



 李襲志,字重光。其先本隴西狄道人,五世祖避地,更為金州安康人。仕隋始安郡丞。大業末,盜賊起,襲志傾私產募士,得三千人,乘城拒盜,蕭銑、林士弘屢攻之不下。聞煬帝喪,乃與士民縞素三日臨,或說曰:「公臨郡久,士大夫悅向,蠻夷畏威,雖曰隋臣,實君長也。今四海分裂,自王者非一姓,宜遂據嶺表,取百粵,豈遽不若尉佗乎?」襲志曰:「吾世隋臣,今江都雖淪,宗社尚有奉,諸君當相與戮力刷仇恥,豈怙亂圖不義哉?吾寧蹈忠死,不逆節以生,尉佗不足為吾法也。」欲斬說者,眾諫,乃止。遂固守凡二年,力窮援絕,為銑所陷,偽署工部尚書、桂州總管。



 武德初,高祖賜書,命其子玄嗣召之。襲志約嶺南酋永平郡守李光度潛圖歸國。帝復以書諭曰:「公朕之宗,不可與異姓比,宜及子弟並豫宗正屬籍。」乃銑平,嶺南六十餘州皆送款,襲志誘而致雲。趙郡王孝恭承制授桂州總管。五年來朝,進柱國,封始安郡公、江州都督。後討輔公祏,為水軍總管,轉桂州都督。襲志守桂二十八年,政尚清省,南荒便之。表請入朝,以光祿大夫、汾州刺史致仕,卒。



 弟襲譽,字茂實,通敏有識度。仕隋為冠軍府司兵。陰世師輔代王守京師也,三輔盜螘聚,襲譽請以兵據永豐倉,發粟賑窮乏,出庫物賞戰士,馳檄郡縣,共逐捕賊。世師不從。乃求出募山南兵,至漢中,高祖已定長安,召授太府少卿、安康郡公。



 伐王世充也,拜潞州總管。時突厥已和親,又通使世充,襲譽捕斬之。詔委典運,以饟東軍。擢累揚州大都督府長史、江南巡察大使,多所黜陟。揚州,江、吳大都會,俗喜商賈,不事農;襲譽為引雷陂水,築句城塘,溉田八百頃,以盡地利,民多歸本。召為太府卿。



 為人嚴愨,以威肅聞。居家儉,厚於宗親,祿稟隨多少散之。以餘資寫書,罷揚州,書遂數事載。嘗謂子孫曰:「吾性不喜財,遂至窶乏。然負京有賜田十頃,能耕之,足以食;河內千樹桑,事之可以衣;江都書,力讀可進求宦。吾歿後,能勤此,無資於人矣。」遷涼州都督,改同州刺史。坐在涼州以私憾杖殺番禾丞劉武,當死,廢為民,流泉州,卒。



 姜謨,秦州上邽人。隋大業末,為晉陽長。高祖在太原,謨前識之,謂所親曰:「隋政亂將亡,必有聖人受之。唐公負王霸資度,其必撥亂得天下。」乃深自結。及大將軍府建,引為司功參軍,從平霍邑、絳郡,兵遂度河,謨部勒一夕濟,高祖嘆其略。進平長安,除相國胄曹參軍、長道縣公。



 薛舉寇秦州,以謨山西豪望,詔安撫隴外,委以便宜。將行,請曰:「公天人之望已屬,宜膺圖緯,光有神器。謨老矣,恐先朝露,幸一見踐阼,死不恨。」高祖嘉納。乃與竇軌出散關,下河池、漢陽,遇薛舉,與戰,軌敗,召謨還朝,為員外散騎常侍。後仁杲平,擢秦州刺史。帝曰:「昔人稱衣錦故鄉,今以本州相授,所以償功。涼州荒梗,宜有以靖之。」謨至,撫邊俗以恩信,盜賊衰止。人喜曰:「不意復見太平官府。」改守隴州,以老去職。貞觀元年卒,贈岷州都督,謚曰安。



 子確。確,字行本,以字顯。貞觀中,為將作少匠,護作九成、洛陽宮及諸苑御,以幹力稱,多所賚嘗,游幸無不從,遷宣威將軍。太宗選趫才,衣五色袍,乘六閑馬,直屯營,宿衛仗內,號曰「飛騎」,每出幸,即以從,拜行本左屯衛將軍,分典之。高昌之役,為行軍副總管,出伊州,距柳穀百里,依山造攻械,增損舊法,械益精。其處有漢班超紀功碑,行本磨去古刻,更刊頌陳國威靈。遂與侯君集進平高昌,戰有功,璽書尉勞。還,為金城郡公,賜奴婢七十人,帛百五十段。帝將征高麗,行本諫未宜輕用師,不從。至蓋牟城,中流矢,卒。帝賦詩悼之,贈左衛大將軍、郕國公,謚曰襄,陪葬昭陵。子簡嗣。行本性恪敏。所居官,雖祈寒烈暑無懈容,加有巧思,凡朝之營繕,所司必諮而後行。魏徵見其倚暱,恐浸啟侈端,勸帝斥之,帝賴其強濟,不斥也。



 子柔遠,美姿容,敷奏詳辯。武后時,至左鷹揚衛將軍,攝地官尚書通事舍人、內供奉。子皎、晦。



 皎,長安中為尚衣奉御,玄宗在籓邸,皎識其有非常度,委心焉。及即位,自潤州長史召授殿中少監。出入臥內,陪燕私,詔許舍敬,坐與妃嬪連榻,間擊球鬥雞,呼之不名也。賜宮女、廄馬及它珍物,前後不勝計。帝在殿廷玩一嘉樹,皎盛贊之,帝遽令徙植其家。



 後將誅竇懷貞等,皎與密議,以功進殿中監、楚國公,食封四百戶。議者譏短皎任遇太過,帝以其籓邸舊,思有以宣布之,乃下詔曰:「殿中監、楚國公皎,往事朕於籓國,雖彭祖同書,子陵共學,不過也。朕嘗游長楊、鄠、杜間,皎於時奉侍,數謂朕曰:『相王必登天位,王且儲副。』朕叱而後止,復言於朕兄弟近戚。語聞太上皇,太上皇奏之中宗,遣嗣虢王邕等鞫問,皎一意保護,罔或貳言。宗楚客、紀處訥等請投皎炎荒,中宗特詔貶潤州長史。專以忠力戴朕,謂天且有命,故履危蹈艱而無變焉。朕既即位,又參誅奸臣,將厚以光寵,每所捴遜。造膝匪躬,舉多規益。而悠悠之談,醜正惡直,天下之人,其未及識皎之功,何見之異也?昔漢昭之任霍光,魏祖之明程昱,朕之不德,庶幾於此。且否當其悔,則必滅乃宗;泰至於亨,則所酬未補。豈流言之聽,而厚德之忘哉?茍謀始有之,圖終可也。」尋遷太常卿,監修國史。弟晦又為吏部侍郎,有權寵,宋璟以為非久安策,請抑損之。



 開元五年,下詔放歸田里,使自娛。久之,復為秘書監。十年,坐洩禁中語,為嗣濮王嶠所劾,敕中書門下究狀。嶠亦王守一姻家,中書令張嘉貞陰希其意,傅致皎獄。詔免殊死,杖之,流欽州。道病死,年五十。親厚坐謫死者數人,世以為冤。時源乾曜方侍中,不能正,為人所譏詆。帝後思皎舊勛,令遞柩還,以禮葬之,存問其家,追贈澤州刺史。後以子尚主,更贈吏部尚書,仍賜封二百戶為祠享費。



 子慶初。慶初生方卒,帝許尚主,後淪謫二十餘年。天寶初,皎甥李林甫為宰相,為帝言之,始命以官,襲楚國公。十載,尚新平公主。新平故嘗歸裴玪,玪卒,乃降慶初。主慧淑,閑文墨,帝賢之,歷肅、代朝,恩禮加重,慶初亦得幸。舊制,駙馬都尉多不拜正官,特拜慶初太常卿。會脩植建陵,詔為之使,誤毀連岡,代宗怒,下吏論不恭,賜死,建陵使史忠烈等皆誅,裴玪子仿,亦削官。主幽禁中,大歷十年薨。



 故事,太常職奉陵廟。開元末,濮陽王徹為宗正卿,有寵,始請宗正奉陵。天寶中,張垍以主婿任太常,故復舊。及慶初敗,又以陵廟歸宗正云。



 晦,起家蒲州參軍,累為高陵令,治有聲,遷長安令,人畏愛之。開元初,擢御史中丞。先是,永徽、顯慶時,御史不拜宰相,銜命使四方者,廷中揖見,後稍屈下。至晦,獨徇舊體,謂御史曰:「不如故事,且奏譴公等。」由是臺儀復振。轉太常少卿。



 時國馬乏,晦請以詔書市馬六胡州,率得馬三千,署游擊將軍,詔可。閑廄乃稍備。除黃門侍郎,辭不拜,改兵部。滿歲,為吏部侍郎,主選。曹史嘗請托為奸,前領選者周棘扈籓,檢窒內外,猶不禁。至晦,悉除之,示無防限,然處事精明,私相屬諉,罪輒得,皆以為神。始,晦革舊示簡,廷議恐必敗,既而贓賕路塞,而流品有敘,眾乃伏。皎被放,晦亦左除宗正卿。貶春州司馬,徙海州刺史,卒。



 崔善為,貝州武城人。祖顒,為魏散騎侍郎。善為巧於歷數,仕隋,調文林郎。督工徒五百營仁壽宮,總監楊素索簿閱實,善為執板暗唱,無一差謬,素大驚。自是四方有疑獄,悉令按訊,皆究其情。仁壽中,遷樓煩司戶書佐,高祖為太守,尤禮接。



 善為見隋政日紊,密勸高祖圖天下。及兵起,署大將軍府司戶參軍,封清河縣公。擢累尚書左丞,用清察稱。諸曹史惡之,以其短而傴,嘲曰:「曲如鉤,例封侯。」欲沮罷所任。帝聞,勉之曰:「昔齊末奸吏歌斛律明月,而高緯暗不察,至滅其家。朕雖不德,幸免是。」因下令購謗者,謗乃止。傅仁均撰《戊寅歷》,李淳風詆其疏,帝令善為考二家得失,多所裁正。



 貞觀初,為陜州刺史。時議,戶猥地狹者徙寬鄉,善為奏:「畿內戶眾,而丁壯悉籍府兵,若聽徙,皆在關東,虛近實遠,非經通計。」詔可。歷大理、司農二卿,坐與少卿不平,出為秦州刺史。卒,贈刑部尚書,謚曰忠。



 初,天下既定,群臣居喪者皆奪服,善為建言其敝。武德二年,始許終喪,然猶時以權迫不能免,如房玄齡、褚遂良者眾矣。



 李嗣真,字承胄,趙州柏人人。多藝數,舉明經,中之,累調許州司功參軍。賀蘭敏之修撰東臺,表嗣真直弘文館,與學士劉獻臣、徐昭皆少有名,號「三少」。高宗東封還,詔贈孔子太師,命有司為祝,司文郎中雷少潁文不稱旨,更命嗣真,成不淹頃,帝覽稱善,詔加兩階。敏之等倚恩自如,嗣真不喜,求補義烏令。敏之敗,學士多連坐,嗣真獨免。



 調露中,為始平令,風化大行。時章懷太子作《寶慶曲》,閱於太清觀,嗣真謂道人劉概、輔儼曰:「宮不召商,君臣乖也;角與徵戾,父子疑也。死聲多且哀,若國家無事,太子任其咎。」俄而太子廢,概等奏其言,擢太常丞,知五禮儀,封常山縣子。嗣真常曰:「隋樂府有《堂堂曲》,明唐再受命,比日有『側堂堂,橈堂堂』之謠,側,不正也,橈,危也。皇帝病日侵,事皆決中宮,持權與人,收之不易。宗室雖眾,居中制外,勢且不敵。諸王殆為後所蹂踐,吾見難作不久矣。」太常缺黃鐘,鑄不能成,嗣真居崇業里,疑土中有之,弗得其所。道上逢一車,有鐸聲甚厲,嗣真曰:「宮聲也。」市以歸,振於空地,若有應者,掘之得鐘,眾樂遂和。嘗引工展器於廷,後奇其風度應對,召相王府參軍閻玄靜圖之,吏部郎中楊志誠為贊,秘書郎殷仲容書,時以為寵。



 永昌初,以右御史中丞知大夫事,請周、漢為二王後,詔可。命巡撫河東,薦宋溫瑾、袁嘉祚、李日知,拔州縣職,皆至顯官。來俊臣獄方熾,嗣真上書諫,以為「昔陳平事漢祖,謀疏楚君臣,行反間,項羽遂亡。今殆有如平者謀陛下君臣,恐為社稷禍」。不納。出為潞州刺史。俊臣誣以反,流藤州,久得還。自筮死日,豫具棺斂,如言卒桂陽。有詔州縣護喪還鄉里,贈濟州刺史,謚曰昭。



 武后嘗問嗣真儲貳事,對曰:「程嬰、杵臼存趙氏孤,古人嘉之。」後悟,中宗乃安。神龍初,贈御史大夫。所撰述尤多。



 時雍州人裴知古亦善樂律,長安中,為太樂令。神龍元年正月,享太廟,樂作,知古密語萬年令元行沖曰:「金石諧婉,將有大慶,在唐室子孫乎!」是月,中宗復位。人有乘馬者,知古聞其嘶,乃曰:「馬鳴哀,主必墜死。」見新婚者,聞佩聲,曰:「終必離。」訪之,皆然。



\end{pinyinscope}