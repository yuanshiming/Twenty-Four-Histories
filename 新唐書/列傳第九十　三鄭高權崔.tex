\article{列傳第九十 三鄭高權崔}

\begin{pinyinscope}

 鄭餘慶,字居業,鄭州滎陽人,三世皆顯宦。餘慶少善屬文,擢進士第。嚴震帥山南西道述自己的哲學見解。認為水、火、木、金、土「五行」生成,奏置幕府。貞元初,還朝,擢庫部郎中,為翰林學士,以工部侍郎知吏部選。浮屠法湊以罪為民訴闕下,詔御史中丞宇文邈、刑部侍郎張彧、大理卿鄭云達為三司,與功德判官諸葛述參按。述,故史也,餘慶劾述猥賤,不宜與三司雜治,時韙其言。



 貞元十四年,拜中書侍郎、同中書門下平章事。每奏對,多傅經義。素善度支使於,凡所陳,必左右之,坐事貶;又歲旱饑,朝廷議賑禁衛十軍,為中書史漏言。疊二忤,故貶郴州司馬。



 順宗以尚書左丞召,會憲宗立,即其官復拜同中書門下平章事。時主書滑渙與宦人劉光琦相倚為奸,每宰相議,為光琦沮變者,令渙往請必得,由是四方貲餉奔委之,弟泳至官刺史。杜佑、鄭絪執政,頗姑息,而佑常行輩待,不名也。至餘慶議事,渙傲然指畫諸宰相前,餘慶叱去。未幾,罷為太子賓客。後渙以贓敗,帝浸聞叱去事,善之。改國子祭酒,累遷吏部尚書。



 醫工崔環者,自淮南小將除黃州司馬,餘慶執奏:「諸道散將無功受五品正員,開徼幸路,不可。」權者不悅,改太子少傅,兼判太常卿事。自硃泚亂,都輦數驚,太常肄樂禁用鼓,餘慶以時久平,奏復舊制。出為山南西道節度使。入拜太子少師,請老,不許。



 時數赦,官多泛階;又帝親郊,陪祠者授三品、五品,不計考;使府賓吏,以軍功借賜硃紫率十八;近臣謝、郎官出使,多所賜與;每朝會,硃紫滿廷而少衣綠者。品服太濫,人不以為貴,帝亦惡之,始詔餘慶條奏懲革。遷尚書左僕射。僕射比非其人,及餘慶以宿德進,公論浩然歸重。帝患典制不倫,謂餘慶淹該前載,乃詔為詳定使,俾參裁訂正。餘慶引韓愈、李程為副,崔郾、陳佩、楊嗣復、庾敬休為判官,凡損增儀矩,號稱詳衷。



 俄拜鳳翔尹,節度鳳翔。復為太子少師,封滎陽郡公,兼判國子祭酒事。建言:「兵興以來,學校廢,諸生離散。今天下承平,臣願率文吏月俸百取一,以資完葺。」詔可。穆宗立,加檢校司徒。卒,年七十五,贈太保,謚曰貞。帝以其貧,特給一月奉料為賵禭。



 餘慶少砥礪,行己完潔。仕四朝,其祿悉賙所親,或濟人急,而自奉粗狹。至官府,乃開肆廣大,常語人曰:「祿不及親友而侈僕妾者,吾鄙之。」大抵中外姻嫁,其禮獻皆親閱之。後生內謁,必引見,諄諄教以經義,務成就儒學。自至德後,方鎮除拜,必遣內使持幢節就第,至則多饋金帛,且以媚天子,唯恐不厚,故一使者納至數百萬緡。憲宗每命餘慶,必誡使曰:「是家貧,不可妄求取。」議者或詆其沽激,餘慶不屑也。奏議類用古言,如「仰給縣官」、「馬萬蹄」,有司不曉何等語,人訾其不適時。與從父絪家昭國坊,絪第在南,餘慶第在北,世謂「南鄭相」、「北鄭相」云。子澣。澣本名涵,避文宗故名,改焉。第進士,累遷右補闕。敢言,無所諱,憲宗謂餘慶曰:「涵,卿令子而朕直臣也,可更相賀。」遷起居舍人、考功員外郎。時刺史或迫吏下紀功愛,涵請責觀察使以杜其欺。餘慶為僕射,避除國子博士、史館脩撰。



 文宗立,入翰林為侍講學士。帝使稡擷經史為《要錄》,愛其博而精,試舉諸條擿問之,隨即酬析,無留答,因賜金紫服。累進尚書左丞,出為山南西道節度使。始,餘慶在興元創學廬,澣嗣完之,養生徒,風化大行。以戶部尚書召,未拜,卒。年六十四,贈尚書右僕射,謚曰宣。



 四子,處誨、從讜尤知名。



 處誨,字廷美,文辭秀拔。仕歷刑部侍郎、浙東觀察、宣武節度使,卒。先是,李德裕《次柳氏舊聞》,處誨謂未詳,更撰《明皇雜錄》,為時盛傳。



 從讜,字正求。及進士第,補校書郎,遷累左補闕。令狐綯、魏扶皆澣門生,數進譽之,遷中書舍人。咸通中,為吏部侍郎,銓次明允。出為河東節度使,徙宣武,以善最聞。改嶺南東道節度。先是,林邑蠻內侵,召天下兵進援,會龐勛亂,不復遣,而北兵寡弱。從讜募土豪,署其酋右職,為約束,使相捍禦,交、廣晏然。



 僖宗立,召為刑部尚書。久之,擢同中書門下平章事,進門下侍郎。沙陀都督李國昌間邊多虞,入據振武、雲朔等州,南略太谷。河東節度使康傳圭遣大將伊釗、張彥球、蘇弘軫引兵拒之,戰數負,傳圭斬軫以徇。彥球所部反,攻傳圭,殺之,劫府庫為亂。朝廷以為憂,帝欲大臣臨制,乃拜從讜檢校司徒,以宰相秩復為河東節度,兼行營招討使,詔自擇參佐。從讜即表長安令王調自副,兵部員外郎劉崇龜、司勛員外郎趙崇為節度觀察府判官,前進士劉崇魯推官,左拾遺李渥掌書記,長安尉崔澤支使,皆一時選。京師士人比太原為小朝廷,言得才多也。時承軍亂,剽奪日旁午。從讜既視事,奸無庾情,乃推捕反賊,誅其首惡。以彥球本善意,且才可任,釋不問,而付以兵,曠無餘猜,故得其死力。渠兇宿狡不敢發,發又輒得,士皆寒毛惕伏。



 會黃巢犯京師,帝駐梁、漢,詔從讜發部兵屬北面招討副使諸葛爽入討。從讜團士五千,遣將論安從爽。而李克用謂太原可乘,以沙陀兵奄入其地,壁汾東,釋言討賊,須索繁仍。從讜以餼醪犒軍,克用隃謂曰:「我且引而南,欲與公面約。」從讜登城,開勉感概,使立功報天子厚恩,克用辭窮,再拜去。然陰縱其下肆掠,以撼人心。從讜追安,使與將王蟾、高弁等踵擊,亦會振武契苾通至,與沙陀戰,沙陀大敗引還。即遣安等屯北百井,安擅還,從讜合諸將,命持安出,斬之鞠場。中和二年,朝廷赦沙陀,使擊賊自贖,兵不敢道太原,繇嵐、石並河而南,獨克用從數百騎過辭城下,從讜以名馬器幣歸之。明年,賊平,詔克用代領河東。克用使來曰:「方省親雁門,願公徐行。」從讜即日以監軍周從寓知兵馬留後,掌書記劉崇魯知觀察留後,敕克用至,按籍效之,乃行。



 黃頭軍以糧少劫其貲,從讜間走絳州,方道梗不通,數月,召拜司空,復秉政,進太傅兼侍中。從帝至興元,以疾乞骸骨,拜太子太保,還第,卒。謚文忠。



 從讜進止有禮法,性不矜滿,沈毅有謀。在汴時,以處晦歿於鎮,訖代,不奏樂牙中。識陸扆於後生,數稱譽之,扆後位宰相。張彥球者,拳摯善斷,累破虜有功,奏為行軍司馬,後署金吾將軍。初,盜流中原,沙陀強悍,而卒收其用者,蓋從讜為太原重也。時鄭畋以宰相鎮鳳翔,移檄討賊,兩人以忠義相提衡,賊尤憚之,號「二鄭」云。



 鄭珣瑜,字元伯,鄭州滎澤人。少孤,值天寶亂,退耕陸渾山,以養母,不乾州里。轉運使劉晏奏補寧陵、宋城尉,山南節度使張獻誠表南鄭丞,皆謝不應。大歷中,以諷諫主文科高第,授大理評事,調陽翟丞,以拔萃為萬年尉。崔祐甫為相,擢左補闕,出為涇原帥府判官。入拜侍御史、刑部員外郎,以母喪解。訖喪,遷吏部。貞元初,詔擇十省郎治畿、赤,珣瑜檢校本官兼奉先令。明年,進饒州刺史。入為諫議大夫,四遷吏部侍郎。



 為河南尹。未入境,會德宗生日,尹當獻馬,吏欲前取印,白珣瑜視事,且內贄。珣瑜徐曰:「未到官而遽事獻,禮歟?」不聽。性嚴重少言,未嘗以私托人,而人亦不敢謁以私。既至河南,清靜惠下,賤斂貴發以便民。方是時,韓全義將兵伐蔡,河南主饋運,珣瑜密儲之陽翟,以給官軍,百姓不知僦運勞。凡迎送敕使,皆在常處,吏密識其馬,進退不數步差也。全義與監軍別檄有所取,非詔約者,珣瑜輒掛壁不酬。至軍罷,凡數百封。有諫者曰:「軍須期會為急,公可不報?」珣瑜曰:「武士統戎,多恃以取求。茍以為罪,尹宜坐之,終不為萬人產沴也。」故下無怨讟。時謂治河南比張延賞,而重厚堅正過之。



 復以吏部侍郎召,進門下侍郎、同中書門下平章事。李實為京兆尹,剝下務進奉,珣瑜顯詰曰:「留府緡帛入有素,餘者應內度支。今進奉乃出何色邪?」具以對。實方幸,依違以免。



 順宗立,即遷吏部尚書。王叔文起州吏為翰林學士、鹽鐵副使,內交奄人,攘撓政機。韋執誼為宰相,居外奉行。叔文一日至中書見執誼,直吏曰:「方宰相會食,百官無見者。」叔文恚,叱吏,吏走入白,執誼起,就閣與叔文語。珣瑜與杜佑、高郢輟饔以待。頃之,吏白:「二公同飯矣。」珣瑜喟曰:「吾可復居此乎!」命左右取馬歸,臥家不出七日,罷為吏部尚書。亦會有疾,數月卒,年六十八,贈尚書左僕射。太常博士徐復謚文獻,兵部侍郎李巽言:「文者,經緯天地。用二謚,非《春秋》之正,請更議。」復謂:「二謚,周、漢以來有之。威烈、慎靜,周也;文終、文成,漢也。況珣瑜名臣,二謚不嫌。」巽曰:「謚一,正也,堯、舜是也。二謚,非古也,法所不載。」詔從復議。子覃。



 覃以父廕補弘文校書郎,擢累諫議大夫。憲宗取五中官為和糴使,覃奏罷之。



 穆宗立,不恤國事,數荒暱。吐蕃方強。覃與崔郾等廷對曰:「陛下新即位,宜側身勤政,而內耽宴嬉,外盤游畋。今吐蕃在邊,狙候中國,假令緩急,臣下乃不知陛下所在,不敗事乎?夫金繒所出,固民膏血,可使倡優無功濫被賜與?願節用之,以所餘備邊,毋令有司重取百姓,天下之幸也。」帝不懌,顧宰相蕭俛曰:「是皆何人?」俛曰:「諫官也。」帝意解,乃曰:「朕之闕,下能盡規,忠也。」因詔覃曰:「閣中殊不款款,後有為我言者,當見卿延英。」時閣中奏久廢,至是,士相慶。



 王承元徙鄭滑節度使,鎮人固留不出。承元請以重臣勞安其軍,詔覃為宣諭使,起居舍人王璠副之。始,鎮人慢甚,及覃傳詔,開勖大義,軍遂安,承元乃得去。



 寶歷初,擢京兆尹。文宗召為翰林侍講學士,進工部侍郎。覃於經術該深,諄篤守正,帝尤重之。李宗閔、牛僧孺知政,以覃與李德裕厚,忌其親近為助力,陽遷工部尚書,罷侍講,欲推遠之。帝雅向學,頗思覃,復召為侍講學士。德裕既相,以為御史大夫。帝嘗謂殷侑善言經,其為人鄭覃比也。宗閔猥曰:「二人誠通經,然其議論不足取。」德裕曰:「覃、侑之言,它人不欲聞,惟陛下宜聞之。」俄德裕罷,宗閔復用,覃繇戶部尚書下除秘書監。宗閔得罪,遷刑部尚書,進尚書右僕射,判國子祭酒。李訓誅,帝召覃視詔禁中,遂拜同中書門下平章事,封滎陽郡公。



 不喜文辭,病進士浮誇,建廢其科,曰:「南北朝所以不治,文採勝質厚也。士惟用才,何必文辭。」又言:「文人多佻薄。」帝曰:「純薄似賦性之異,奚特進士耶?且設是科二百年,渠可易?」乃止。帝嘗謂百司不可使一日弛惰,因指香案爐曰:「此始華好,用久則晦,不治飾,何由復新?」覃曰:「救世之敝,在先責實。比皆不攝職事,至慕王夷甫,以不及為靳。此本於治平,人人無事,安逸致然。」帝曰:「要在謹法度而已。」進門下侍郎、弘文館大學士。



 帝坐延英論詩工否,覃曰:「孔子所刪,三百篇是已,其非雅正者,烏足為天子道哉?夫《風》、大小《雅》,皆下刺上之變,非上化下為之。故王者採詩,以考風俗得失。若陳後主、隋煬帝特能詩之章解,而不知王術,故卒歸於亂。章什諓諓,願陛下不取也。」



 帝每言:「順宗事不詳實,史臣韓愈豈當時屈人邪?昔漢司馬遷《與任安書》,辭多怨懟,故《武帝本紀》多失實。」覃曰:「武帝中年大發兵事邊,生人耗瘁,府庫殫竭,遷所述非過言。」李石曰:「覃所陳,因武帝以諫,欲陛下終究盛德。」帝曰:「誠然。靡不有初,鮮克有終。」覃曰:「陛下樂觀書,然要義不過一二,陛下所道是矣。宜寢饋以之。」



 覃既名儒,故以宰相領祭酒,請太學《五經》,經置博士,祿廩比王府官。再遷太子太師。開成三年,旱,帝多出宮人,李玨入賀曰:「漢制,八月選人;晉武帝平吳,多採擇;仲尼所謂未見好德者。陛下以為無益,放之,盛德也。」覃又推贊曰:「晉以採擇之失,舉天下為左衽,宜陛下以為殷鑒。」帝善其將美。以病乞去位,有詔解太子太師,許五日一入中書,商量政事。俄罷為尚書左僕射。武宗初,李德裕復用,欲援覃共政,固辭,乃授司空,致仕,卒。



 覃清正退約,與人未嘗串狎。位相國,所居第不加飾,內無妾媵。女孫適崔皋,官裁九品衛佐,帝重其不昏權家。覃之侍講,每以厚風俗、黜朋比再三為天子言,故終為相。然疾惡多所不容,世以為太過,憚之。始,覃以經籍刓繆,博士陋淺不能正,建言:「願與鉅學鴻生共力讎刊,準漢舊事,鏤石太學,示萬世法。」詔可。覃乃表周墀、崔球、張次宗、孔溫業等是正其文,刻於石。子裔綽。



 裔綽峭立有父風,以門廕進,為李德裕所知,擢渭南尉。直弘文館,累遷諫議大夫。宣宗初,劉潼繇鄭州刺史授桂管觀察使,裔綽固爭:「潼被責未久,不宜付廉察。」帝已遣使者頒詔,追罷之。遷給事中。楊漢公為荊南節度使,坐貪沓,貶秘書監,尋拜同州刺史,裔綽與鄭公輿封還制書。帝自即位,諫臣規正無不納。至是,有為漢公地者,遂終不易。會賜宴禁中,天子擊球,至門下官,謂二人曰:「近論漢公事,類朋黨者。」裔綽曰:「同州,太宗興王地,陛下為人子孫,當慎所付。且漢公墨沒敗官,奈何以重地私之?」帝變色。翌日,貶商州刺史。時猶衣綠,因詔賜緋魚。後繇秘書監遷浙東觀察使,終太子少保。覃弟朗。



 朗,字有融,始闢柳公綽山南幕府,入遷右拾遺。開成中,擢起居郎。文宗與宰相議政,適見朗執筆螭頭下,謂曰:「向所論事,亦記之乎?朕將觀之。」朗曰:「臣執筆所書者,史也。故事,天子不觀史,昔太宗欲觀之,硃子奢曰:『史不隱善,不諱惡。自中主而下,或飾非護失,見之,則史官無以自免,且不敢直筆。』褚遂良亦稱:『史記天子言動,雖非法必書,庶幾自飭。』」帝悅,謂宰相曰:「朗援故事,不畀朕見起居注,可謂善守職者。然人君之為,善惡必記,朕恐平日言之不協治體,為將來羞,庶一見,得以自改。」朗遂上之。



 累遷諫議大夫,為侍講學士。由華州刺史入拜御史中丞、戶部侍郎。為鄂岳、浙西觀察使,進義武、宣武二節度。歷工部尚書判度支、御史大夫,復為工部尚書、同中書門下平章事。中人李敬寔排朗騶導馳去,朗以聞。宣宗詰敬寔,自言供奉官不避道,帝曰:「傳我命則絕道行可也,而私出,不避宰相邪?」即斥敬寔。右拾遺鄭言者,故在幕府,朗以諫臣與輔相爭得失,不論則廢職,奏徙它官。久之,以疾自陳,罷為太子少師。卒,贈司空。



 始,朗舉進士,有相者言:「君當貴,然不可以科第進。」俄而有司擢朗第一,既又覆實被放,相者賀曰:「安之。」已而果相。



 高郢,字公楚,其先自渤海徙衛州,遂為衛州人。九歲通《春秋》,工屬文,著《語默賦》,諸儒稱之。父伯祥為好畤尉,安祿山陷京師,將誅之,郢尚幼,解衣請代,賊義,並貸之。



 寶應初,及進士第。代宗為太後營章敬寺,郢以白衣上書諫曰:



 陛下大孝因心,與天罔極,烝烝之思,要無以加。臣謂悉力追孝,誠為有益,妨時剿人,不得無損。舍人就寺,何福之為?昔魯莊公、丹桓公廟楹而刻其桷,《春秋》書之為非禮。漢孝惠、孝景、孝宣令郡國諸侯立高祖、文、武廟,至元帝,與博士、議郎斟酌古禮,一罷之。夫廟猶不越禮而立,況寺非宗祏所安、神靈所宅乎?殫萬人之力,邀一切之報,其為不可亦明矣。


間者昆吾孔熾,薦食生人,百姓懍懍,無日不惕。遣將攘卻,亡尺寸功,隴外壤地,委諸豺狼。太宗敔難之業,傳之陛下,一夫不獲,尺土見侵,告成之時,猶恐有闕。況用武以來十三年,傷者不救,死者不收,繕卒補乘,於今未已。夫興師十萬,日費千金,計十三年,舉百萬之眾,資糧
 \gezhu{
  尸非}
 屨,取足於人,勞罷宛轉,十不一在。父子兄弟,相視無聊,延頸嗷嗷,以役王命。縱未能出禁財,贍鰥寡,猶當稍息勞敝,以噢休之。奈何戎虜未平,侵地未復,金革未戢,疲人未撫,太倉無終歲之儲,大農有榷酤之敝,欲以此時興力役哉?比八月雨不潤下,菽麥失時,黔首狼顧,憂在艱食,若遂不給,將何以救之?無寺猶可,無人其可乎?然土木之勤,功用之費,不虛府庫,將焉取之?府庫既竭,則又誅求,若人不堪命,盜賊相挺而興,戎狄乘間,以為風塵,得不為陛下深憂乎?



 臣聞聖人受命於天,以人為主,茍功濟於天,天人同和,則宗廟受福,子孫蒙慶。《傳》曰:「德教加於百姓,刑於四海,天子之孝也。」又曰:「無念爾祖,聿脩厥德,」「既受帝祉,施於孫子。」是知王者之孝,在於承順天地,嚴配宗考,恭慎德教,以臨兆民。俾四海之內,歡心助祭,延福流祚,永永無窮。未聞崇樹梵宮,雕琢金玉之為孝者。夏禹卑宮室,盡力溝洫,人到於今稱之。梁武帝窮土木,飾塔廟,人無稱焉。陛下若節用愛人,當與夏後齊美,何必勞人動眾,踵梁武遺風乎?及制作之初,支費尚淺,人貴量力,不貴必成,事貴相時,不貴必遂。陛下若回思慮,從人心,則聖德孝思,格於天地,千福萬祿,先後受之,曾是一寺較功德邪?



 書奏,未報。復上言:



 王者將有為也,將有行也,必稽於眾而順於人,則自然之福,不求而至,未然之禍,不除而絕。臣聞神人無功者,不為有功之功;聖人無名者,不為有名之名。不為有功之功,故功莫大;不為有名之名,故名莫厚。古之明王積善以致福,不費財以求福;脩德以銷禍,不勞人以攘禍。陛下之營作,臣竊惑之。若以為功,則天覆地載,陰施陽化,未曾有為也;若以為名,則至德要道,以順天下,未曾有待也;若以致福,則通於神明,光於四海,不在費財;若以攘禍,則方務厥德,罔有天災,不在勞人。今興造趣急,人徒竭作,土木並起,日課萬工,不遑食息,搒笞愁痛,盈於道路。以此望福,臣恐不然。陛下戢定多難,勵精思治,務行寬仁,以幸天下。今固違群情,徇左右過計,臣竊為陛下惜之。



 不納。



 以茂才異行高第,累擢咸陽尉。郭子儀取為朔方掌書記。子儀怒判官張曇,奏抵死,郢引捄甚力,忤子儀意,下徙猗氏丞。李懷光引佐邠寧府。懷光將還河中,郢勸不如西迎乘輿,懷光反方銳,不聽。既又欲悉兵鼓而西。時渾瑊提孤軍抗賊,群將未集,郢恐為懷光所乘,與李庸阜固止之。會懷光子琟候郢,郢因脅說曰:「君視天寶以來稱兵者,今尚誰在?且國家固有天命,人力不豫焉。今若恃眾而動,自絕於天。十室之小,必得忠信,安知三軍不有奔潰而助順者乎?」琟大懼,流汗不能語。郢因與其將呂鳴岳、張延英謀間道歸國,事洩,懷光先斬二將,然後引郢詰誚,郢抗詞無所愧隱,觀者為泣下。懷光慚,赦之。孔巢父遇害,郢撫尸而哭。懷光已誅,李晟表其忠,馬燧奏管書記。召拜主客員外郎,遷中書舍人。久之,進禮部侍郎。時四方士務朋比,更相譽薦,以動有司,徇名亡實。郢疾之,乃謝絕請謁,顓行藝。司貢部凡三歲,甄幽獨,抑浮華,流競之俗為衰。遷太常卿。



 貞元末,擢中書侍郎、同中書門下平章事。順宗立,病不能事,王叔文黨根據朝廷,帝始詔皇太子監國,而郢以刑部尚書罷。明年,為華州刺史,政尚仁靜。初,駱元光自華引軍戍良原。元光卒,軍入神策,而州仍歲餉其糧,民困輸入,累刺史憚不敢白,郢奏罷之。復召為太常卿,除御史大夫。數月,改兵部尚書,固乞骸骨,以尚書右僕射致仕。卒,年七十二,贈太子太保,謚曰貞。



 郢恭慎不與人交。常掌制誥,家無留橐,或勸盍如前人傳制集者,答曰:「王言不可藏私家。」生平不治產,有勸營之者,答曰:「祿稟雖薄,在我則有餘,田莊何所取乎?」郢之相也,與鄭珣瑜同拜。既叔文用事,珣瑜憂甚,爭不能得,乃稱疾不出,郢未有所建白,俄與珣瑜免,故議者賢珣瑜而咎郢。子定。



 贊曰:王叔文雖內連姏尹,外倚奸回,以攘天權。然是時太子已長,朝無嫌罅,若珣瑜、郢與杜佑等毅然引東宮監國,執退叔文輩,其力不難。顧循嘿茍安,所謂焉用彼相者矣。珣瑜一忿臥第,與郢、佑固位,二者亦不足相輕重雲。



 定,辯惠,七歲讀《尚書》,至《湯誓》,跪問郢曰:「奈何以臣伐君?」郢曰:「應天順人,何雲伐邪?」對曰:「用命賞於祖,不用命戮於社,是順人乎?」郢異之。小字董二,世重其早惠,以字顯。長通王氏《易》,為圖合八出,上圓下方,合則重,轉則演,七轉而六十四卦,六甲、八節備焉。仕至京兆府參軍。



 鄭絪,字文明,餘慶從父行也。幼有奇志,善屬文,所交皆天下有名士。擢進士、宏辭高第。張延賞帥劍南,奏署掌書記。入為起居郎、翰林學士,累遷中書舍人。



 德宗自興元還,置六軍統軍視六尚書,以處功臣,除制用白麻付外。又廢宣武軍,益左右神策,以監軍為中尉。竇文場恃功,陰諷宰相進擬如統軍比。絪當作制,奏言:「天子封建,或用宰相,以白麻署制,付中書、門下。今以命中尉,不識陛下特以寵文場邪?遂著為令也?」帝悟,謂文場曰:「武德、貞觀時,中人止內侍,諸衛將軍同正賜緋者無幾。自魚朝恩以來,無復舊制。朕今用爾不謂無私,若麻制宣告,天下謂爾脅我為之。」文場叩頭謝。更命中書作詔,並罷統軍用麻矣。明日,帝見絪曰:「宰相不能拒中人,得卿言乃悟。」



 順宗病,不得語,王叔文與牛美人用事,權震中外,憚廣陵王雄睿,欲危之。帝召絪草立太子詔,絪不請輒書曰:「立嫡以長。」跪白之,帝頷乃定。



 憲宗即位,拜中書侍郎、同中書門下平章事,遷門下侍郎。始,盧從史陰與王承宗連和,有詔歸潞,從史辭潞乏糧,請留軍山東。李吉甫密譖絪漏言於從史,帝怒,坐浴堂殿,召學士李絳語其故,且曰:「若何而處?」絳曰:「誠如是,罪當族。然誰以聞陛下者?」曰:「吉甫為我言。」絳曰:「絪任宰相,識名節,不當如犬彘梟獍與奸臣外通。恐吉甫勢軋內忌,造為醜辭以怒陛下。」帝良久曰:「幾誤我!」



 先是,杜黃裳方為帝夷削節度,強王室,建議裁可,不關決於絪,絪常默默。居位四年,罷為太子賓客。久乃檢校禮部尚書,出為嶺南節度使,後累遷河中節度。入為御史大夫,檢校尚書左僕射,兼太子少保。文宗太和中,年老乞骸骨,以太子太傅致仕。卒,年七十八,贈司空,謚曰宣。



 絪本以儒術進,守道寡欲,所居不為烜赫事,以篤實稱。善名理學,世以耆德推之。



 孫顥,舉進士,以起居郎尚萬壽公主,拜駙馬都尉。有器識。宣宗時,恩寵無比。終檢校禮部尚書、河南尹。



 權德輿,字載之。父皋,見《卓行傳》。德輿七歲居父喪,哭踴如成人。未冠,以文章稱諸儒間。韓洄黜陟河南,闢置幕府。復從江西觀察使李兼府為判官。杜佑、裴胄交闢之。德宗聞其材,召為太常博士,改左補闕。



 貞元八年,關東、淮南、浙西州縣大水,壞廬舍,漂殺人。德輿建言:「江、淮田一善熟,則旁資數道,故天下大計,仰於東南。今霪雨二時,農田不開,逋亡日眾。宜擇群臣明識通方者,持節勞徠,問人所疾苦,蠲其租入,與連帥守長講求所宜。賦取於人,不若藏於人之固也。」帝乃遣奚陟等四人循行慰撫。裴延齡以巧幸進,判度支,德輿上疏斥言:「延齡以常賦正額用度未盡者為羨利,以誇己功;用官錢售常平雜物,還取其直,號別貯羨錢,因以罔上;邊軍乏,不稟糧,召禍疆場,其事不細。陛下疑為流言,胡不以新利召延齡,質核本末,擇中朝臣按覆邊資。如言者不謬,則邦國之務,不宜委非其人。」疏奏,不省。



 遷起居舍人。歲中,兼知制誥,進中書舍人。當是時,帝親攬庶政,重除拜,凡命諸朝,皆手制中下。始,德輿知制誥,而徐岱給事中,高郢為舍人。居數歲,岱卒,郢知禮部,德輿獨直兩省,數旬一還舍,乃上書言:「左右掖垣,承天子誥命,奉行詳覆,各有攸司。舊制,分曹十員,以相防檢。大抵事有所壅,則吏得為非。四方聞者,或以朝廷為乏士,要重之司,不宜久廢。」帝曰:「非不知卿之勞,但擇如卿者未得其人耳。」久之,知禮部貢舉,真拜侍郎。凡三歲,甄品詳諦,所得士相繼為公卿、宰相。取明經初不限員。



 十九年,大旱,德輿因是上陳闕政曰:「陛下齋心減膳,閔惻元元,告於宗廟,禱諸天地,一物可祈,必致其禮,一士有請,必聽其言,憂人之心可謂至已。臣聞銷天災者脩政術,感人心者流惠澤,和氣洽,則祥應至矣。畿甸之內,大率赤地而無所望,轉徙之人,斃踣道路,慮種麥時,種不得下。宜詔在所裁留經用,以種貸民。今茲租賦及宿逋遠貸,一切蠲除。設不蠲除,亦無可斂之理,不如先事圖之,則恩歸於上矣。十四年夏旱,吏趣常賦,至縣令為民毆辱者,不可不察。」又言:「漕運本濟關中,若轉東都以西緣道倉廩,悉入京師,督江、淮所輸以備常數,然後約太倉一歲計,斥其餘者以糶於民,則時價不踴而蓄藏者出矣。」又言:「大歷中,一縑直錢四千,今止八百,稅入如舊,則出於民者五倍其初。四方銳於上獻,為國掊怨,廣軍實之求,而兵有虛籍,剝取多方,雖有心計巧歷,能商功利,其於割股啖口,困人均也。」又言:「比經絀放者,自謂抆拭無期,坐為匪人,以動和氣。而冬薦官逾三年未受命,衣食既空,溘然就斃,此亦窮人之一端也。近陛下洗宥絀放者,或起為二千石,其徒更相勉,知牽復可望。惟因而弘之,使人人自效。」帝頗採用之。



 憲宗元和初,歷兵部侍郎,坐累,徙太子賓客,俄還前官。時澤潞盧從史詐傲,浸不制,其父虔卒京師,而成德王承宗父死求襲,德輿諫,以為:「欲變山東,先擇昭義之帥。從史拔自軍校,偃蹇不法,今可因其喪,選守臣代之。成德習俗既久,當制以漸,許成德之請則可,許昭義則不可。」帝不聽。及王承宗叛,從史乃詭計以撓王師,兵老無功。德輿復請赦承宗,徙從史。後皆略如所料。



 會裴垍病,德輿自太常卿拜禮部尚書、同中書門下平章事。王鍔繇河中入朝,求兼宰相,李籓以為不可,德輿亦奏:「平章事非序進宜得,比方鎮帶宰相,必有大忠若勛,否則強不制者,不得已與之。今鍔無功,又非姑息時,一假此名,以開後人,不可。」帝乃止。



 董溪、於皋謨以運糧使盜軍興,流嶺南,帝悔其輕,詔中使半道殺之。德輿諫:「溪等方山東用兵,乾沒庫財,死不償責。陛下以流斥太輕,當責臣等繆誤,審正其罪,明下詔書,與眾同棄,則人人懼法。臣知已事不諍,然異時或有此比,要須有司論報,罰一勸百,孰不甘心。」帝深然之。嘗問政之寬猛孰先,對曰:「唐家承隋苛虐,以仁厚為先。太宗皇帝見《明堂圖》,始禁鞭背,列聖所循,皆尚德教。故天寶大盜竊發,俄而夷滅,蓋本朝之化,感人心之深也。」帝曰:「誠如公言。」



 德輿善辨論,開陳古今本末,以覺悟人主。為輔相,寬和不為察察名。李吉甫再秉政,帝又自用李絳參贊大機。是時,帝切於治,事鉅細悉責宰相。吉甫、絳議論不能無持異,至帝前遽言亟辯,德輿從容不敢有所輕重,坐是罷為本官。以檢校吏部尚書留守東都,進扶風郡公。于頔以子殺人,自囚,親戚莫敢過門,朝廷無為請者。德輿將行,言於帝曰:「頔之罪既貸不竟,宜因賜寬詔。」帝曰:「然,卿為吾過諭之。」復拜太常卿,徙刑部尚書。



 先是,詔許孟容、蔣乂刊匯格敕,既成,上之,留禁中;德輿請出其書,與侍郎劉伯芻參復研考,定三十篇奏上。復檢校吏部尚書,出為山南西道節度使。後二年,以病乞還,卒於道。年六十,贈尚書左僕射,謚曰文。



 德輿生三歲,知變四聲,四歲能賦詩,積思經術,無不貫綜。自始學至老,未曾一日去書不觀。嘗著論,辨漢所以亡,西京以張禹,東京以胡廣,大指有補於世。其文雅正贍縟,當時公卿侯王功德卓異者,皆所銘紀,十常七八。雖動止無外飾,其醞藉風流,自然可慕。貞元、元和間,為搢紳羽儀云。



 子璩,字大圭,元和初,擢進士。歷監察御史,有美稱。宰相李宗閔乃父門生,故薦為中書舍人。時李訓挾寵,以《周易》博士在翰林,璩與舍人高元裕、給事中鄭肅、韓佽等連章劾訓傾覆陰巧,且亂國,不宜出入禁中。不聽。及宗閔貶,璩屢表辨解,貶閬州刺史。文宗憐其母病,徙鄭州。訓誅,時人多璩明禍福大體,能世其家。



 崔群,字敦詩,貝州武城人。未冠,舉進士,陸贄主貢舉,梁肅薦其有公輔才,擢甲科,舉賢良方正,授秘書省校書郎。累遷右補闕、翰林學士、中書舍人。數陳讜言,憲宗嘉納,因詔學士:「凡奏議,待群署乃得上。」群以「禁密之言,人人當自陳,一為故事,後或有惡直醜正,則它學士不得上書矣」,固讓,見聽。惠昭太子薨,是時,遂王嫡,而澧王長,多內助。帝將建東宮,詔群為澧王作讓。群奏:「大凡己當得則讓,不當得之,烏用讓?今遂王嫡,宜為太子。」帝從其議。魏博田季安以五千縑助營開業佛祠,群以為無名之獻,不當受。有詔卻之。進戶部侍郎。



 元和十二年,以中書侍郎同中書門下平章事。李師道既誅,師古等妻子沒入掖廷,帝疑,以問群,群請釋之,並還其奴婢貲產。鹽鐵院官權長孺坐罪抵死,其母耄,丐子以養。帝奭然欲赦之,以問宰相,群對:「陛下幸憐其老,宜即遣使諭旨,若須出敕,無及矣。」於是免死。群凡啟奏,平恕如此。帝嘗語宰相:「聽受之際,不亦難乎!比詔學士集前世事,為《辨謗略》,以自儆鑒。其要云何?」群對:「無情,曲直辨之至易;有情,則欺為難審也。故孔子有眾好眾惡、浸潤膚受之說,以其難辨也。若陛下擇賢而任,待之以誠,糾之以法,則人自歸正,而不敢以欺。」帝韙其言。



 處州刺史苗積進羨錢七百萬,群以受之失信天下,請還賜其州,以紓下戶之賦。是時,皇甫鎛言利幸於帝,陰藉左右求宰相,群數言其佞邪不可用。既入對,及開元、天寶事,群因推言其極曰:「安危在出令,存亡系所任。昔玄宗少歷屯險,更民間疾苦,故初得姚崇、宋璟、盧懷慎輔以道德,蘇頲、李元紘孜孜守正,則開元為治。其後安於逸樂,遠正士,暱小人,故宇文融以言利進,李林甫、楊國忠怙寵朋邪,則天寶為亂。願陛下以開元為法,以天寶為戒,社稷之福也。」又言:「世謂祿山反,為治亂分明。臣謂罷張九齡,相林甫,則治亂固已分矣。」左右為感動。群以是諷帝,故鎛銜之。帝卒自相鎛。會群臣上帝號,鎛欲兼用「孝德」為號,群獨以為有「睿聖」,則「孝德」並見。帝聞不樂。會度支稟賜邊士不時,物多弊惡,李光顏憂甚,至欲引佩刀自決,中外皆恐。鎛奏:「邊鄙無事,乃群鼓動,欲以買直,歸怨天子。」於是罷為湖南觀察使。



 穆宗立,以吏部侍郎召之,勞曰:「我為太子,卿力也。」群曰:「此先帝意,臣何力焉?且陛下向為淮西節度使,臣起制草,其言有『能辨南陽之牘,允符東海之貴』,先帝然之,則傳付久矣。」俄拜御史大夫。未幾,檢校兵部尚書,充武寧節度使。群以其副王智興得士心,不若假以節度,不報。智興討幽、鎮還,藉兵逐群,群失守,左遷秘書監,分司東都。改華州刺史,歷宣歙池觀察使,進兵部尚書,出為荊南節度使,召拜吏部尚書。卒,年六十一,贈司空。



 贊曰:聖人不畏多難,畏無難。何哉?多難之世,人人長慮而深謀,日惕於中,猶以為未也,曰:「吾覆亡不暇,又何以安?」故能舉天下付之興,畏之也。禍難已平,上恬下嬉,施施自如曰:「賢難得,雖無賢,尚可治也;佞可去,雖存佞,不遽亂也。」視漏弗填,忽傾弗支,偃然自慰曰:「我曷以喪?」故能舉天下付之亡,不畏也。常人所畏,聖人易之;所不畏,聖人難之。觀孝明皇帝,本中主,遭變可與謀始,持成不可與共終。崔群以為相李林甫則治亂已分,其言信哉!是扁鵲所以誚桓侯也。



\end{pinyinscope}