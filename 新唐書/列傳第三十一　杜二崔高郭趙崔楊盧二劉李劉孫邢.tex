\article{列傳第三十一 杜二崔高郭趙崔楊盧二劉李劉孫邢}

\begin{pinyinscope}

 杜正倫,相州洹水人。隋世重舉秀才,天下不十人,而正倫一門三秀才,皆高第李凱爾特(HeinrichRickert,1863—1936)德國哲學家,,為世歆美。調武騎尉。太宗素知名,表直秦王府文學館。貞觀元年,魏徵薦其才,擢兵部員外郎。帝勞曰:「朕舉賢者,非朕獨私,以能益百姓也。我於宗婭故人,茍無能,終不得任。卿宜思有以稱吾舉者。」俄遷給事中,知起居注。帝嘗曰:「朕坐朝,不敢多言,必待有利於民,乃出諸口。」正倫曰:「臣職左史,陛下一言失,非止損百姓,且筆之書,千載累德。」帝悅,賜彩段二百。進累中書侍郎。與韋挺、虞世南、姚思廉論事稱旨,帝為設宴具,召四人者,謂曰:「我聞神龍可擾以馴,然頷有逆鱗,嬰者死,人君亦有之。卿屬遂犯吾鱗,裨闕失,朕其慮危亡哉!思卿至意,故舉酒以相樂也。」各賜帛有差。



 太子監國,詔正倫行左庶子,兼崇賢館學士。帝謂正倫:「吾兒幼,未有就德,我常物物戒之。今當監國,不得朝夕見,故輟卿於朝以佐太子。慎之勖之。」它日又言:「朕年十八,猶在人間,情偽無不嘗;及即位,處置有失,必待諫,乃釋然悟,況太子生深宮不及知邪?且人主不可自驕,今若詔天下,敢諫者死,將無復發言矣。故朕孜孜延進直言。卿其以是曉太子,冀裨益之。」擢中書侍郎,封南陽縣侯,仍兼太子左庶子。出入兩宮,典機密,以辦治稱。後太子稍失道,帝語正倫:「太子數私小人,卿可審喻之,教而不徙,其語我來。」故正倫顯諫無所避。」太子不從,輒道帝語督切,太子即表聞。帝責曰:「何漏洩我語?」對曰:「開示不入,故以陛下語怖之,冀當反善。」帝怒,出為穀州刺史,再貶交州都督。太子廢,坐受金帶,流馭州。久之,授郢、石二州刺史。



 顯慶元年,擢黃門侍郎,兼崇賢館學士,進同中書門下三品。又兼度支尚書,仍知政事。遷中書令,封襄陽縣公。初,正倫已通貴,李義府官尚微,及同執政,不能下。中書侍郎李友益,義府族也,晚附正倫,同摭義府釁缺。義府使人告正倫、友益交通罔上,有異計。高宗惡之,出正倫為橫州刺史,流友益峰州。正倫卒於貶。



 正倫與城南諸杜昭穆素遠,求同譜,不許,銜之。諸杜所居號杜固,世傳其地有壯氣,故世衣冠。正倫既執政,建言鑿杜固通水以利人。既鑿,川流如血,閱十日止,自是南杜稍不振。正倫工屬文,嘗與中書舍人董思恭夜直,論文章。思恭歸,謂人曰:「與杜公評文,今日覺吾文頓進。」無子,以兄子志靜為嗣。



 從子求仁、從孫咸皆顯名。



 求仁有雅才。永淳中,授監察御史,坐事為黔令。與徐敬業舉兵,為興復府左長史,死於難。



 咸擢進士第。累遷右臺監察御史。牂柯反,咸監軍出討。賊保壘自固,道荒漫,師不能進。咸乃息士,示不欲戰,陰伺之。時旱暑風熾,咸縱火,噪而前,賊眩怖相失,自騰踐死,擒其酋,遂平之。遷侍御史,出為汾州長史。開元中,為河北按察使。坐用法深,貶睦州司馬。



 崔知溫,字禮仁,許州鄢陵人。仕為左千牛,稍遷靈州司馬。境有渾、斛薩萬帳,數擾齊民,農皆釋耒習騎射以捍賊。知溫表徙河北,虜不樂遷,將軍契苾何力為言,乃止。知溫固請,疏十五報,卒徙河北,自是人得就耕。渾、斛薩至徙地,顧善水草,亦忘遷。後入朝,過州,謝曰:「初徙且怨公,今地膏腴,眾孳夥,更荷公恩。」皆再拜。



 四遷蘭州刺史。黨項羌三萬入寇,州兵寡,眾懼,莫知所出。知溫披闔不設備,羌怪之,不敢進。俄會將軍權善才率兵至,大破其眾。善才欲遂窮追取之,知溫曰:「古善戰弗逆奔,且溪穀復深,草木荒延,萬分一有變,不可悔。」善才曰:「善。」分降口五百贈知溫,辭曰:「我議公事,圖私利邪?」



 累遷尚書左丞,轉黃門侍郎,脩國史。永隆初,以秩卑,特詔同門下三品,兼脩國史。遷中書令。卒,年五十七,贈幽州大都督,謚曰忠。子泰之,開元時,為工部尚書;諤之,為將作少匠,與誅二張功,封博陵縣侯,實封戶二百,終少府監。



 兄知悌,亦至中書侍郎。與戴至德、郝處俊、李敬玄等同賜飛白書贊,而知悌、敬玄以忠勤見表。遷尚書左丞。裴行儉之破突厥,斬泥孰匐,殘落保狼山,詔知悌馳往定襄慰將士,佐行儉平遺寇,有功。終戶部尚書。



 高智周,常州晉陵人。第進士,補越王府參軍。遷費令,與丞、尉均取俸,民安其化,刻石頌美。入擢秘書郎、弘文館直學士。嘗覆弈、誦碑,無謬者。三遷蘭臺大夫。孝敬在東宮,與司文郎中賀敳、司經大夫王真儒並為侍讀,得告還鄉里嘆曰:「進不知退,取禍之道也。」即移病去。



 俄拜壽州刺史,其治尚文雅,行部,先見諸生,質經義及政得失,既乃錄獄訟,考耕餉勤墮,以為常。遷正諫大夫、黃門侍郎。儀鳳初,進同中書門下三品。遷太子左庶子。是時崔知溫、劉景先脩國史,故智周與郝處俊監蒞。久之,罷為御史大夫,與薛元超、裴炎同治章懷太子獄,無所同異,固表去位。高宗美其概,授右散騎常侍。請致仕,聽之。卒,年八十二,贈越州都督,謚曰定。



 智周始與郝處俊、來濟、孫處約共依江都石仲覽。仲覽傾產結四人驩,因請各語所期。處俊曰:「丈夫惟無仕,仕至宰相乃可。」智周、濟如之。處約曰:「得為舍人,在殿中周旋吐納可也。」仲覽使相工視之,工語仲覽曰:「高之貴,君不及見之。來早顯而末躓,高晚顯而壽。吾聞速登者易顛,徐進者少患,天道也。」後濟居吏部,處約以瀛州參軍入調,濟曰:「如志。」擬通事舍人。畢,降階勞問平生。既仲覽卒,而濟等益顯。



 智周所善義興蔣子慎,有客嘗視兩人,曰:「高公位極人臣,而嗣少弱;蔣侯宦不達,後且興。」子慎終達安尉。其子繒往見智周,智周方貴,以女妻之。生子挺,歷湖、延二州刺史。生子洌、渙,皆擢進士。洌為尚書左丞。渙,永泰初歷鴻臚卿,日本使嘗遺金帛,不納,唯取箋一番,為書以貽其副云。挺之卒,洌兄弟廬墓側,植松柏千餘。渙終禮部尚書,封汝南公。洌子煉,渙子銖,又有清白名。而高氏後無聞。



 郭正一,定州鼓城人。貞觀時,由進士署第,歷中書舍人、弘文館學士。永隆中,遷秘書少監,檢校中書侍郎,詔與郭待舉、岑長倩、魏玄同並同中書門下承受進止平章事。平章事自正一等始。永淳中,真遷中書侍郎。執政久,明習故事,文辭詔敕多出其手。



 劉審禮與吐番戰青海,大敗。高宗召群臣問所以制戎,正一曰:「吐蕃曠年梗寇,師數出,坐費糧貲。近討則喪威,深入則不能得其巢穴。今上策莫如少募兵,且明烽候,勿事侵擾,須數年之遲,力有餘,人思戰,一舉可破矣。」劉齊賢、皇甫文亮等議,亦與正一合,帝納之。



 武后專國,罷為國子祭酒,出檢校陜州刺史。與張楚金、元萬頃皆為周興所誣構,殺之,籍入其家,妻息流放。文章無存者。



 趙弘智,河南新安人,元魏車騎大將軍肅之孫。早喪母,事父篤孝。通書傳,仕隋為司隸從事。武德初,大理卿郎楚之白為詹事府主簿。太宗時,豫論譔,錄勤,繇太子舍人進黃門侍郎,兼弘文館學士。移病出為萊州刺史,稍遷太子右庶子。父事兄弘安,俸祿歸之,不敢私。弘安卒,哀慟過期,奉嫂謹甚,撫兄子慈均所生。會太子廢,免官。俄拜光州刺史。記徽初,入為陳王師。講《孝經》百福殿,於是宰相、弘文館學士、太學生皆在,弘智舉五孝,諸儒更詰辨,隨問酬悉,舌無留語。高宗喜曰:「試為我陳經之要,以輔不逮。」對曰:「『天子有爭臣七人,雖無道,不失天下。』願以此獻。」帝悅,賜絹二百、名馬一。四年,進國子祭酒,仍為學士。卒,年八十二,謚曰宣。弘安亦終國子祭酒。



 曾孫矜,舉明經,調舞陽主簿,吳少誠反,以縣歸,徙襄城主簿,賜牙緋。歷襄陽丞。客死柳州,官為斂葬。後十七年,子來章始壯,自襄陽往求其喪,不得,野哭。再閱旬,卜人秦誗為筮曰:「金食其墨,而火以貴,其墓直醜,在道之右,南有貴神,塚土是守。宜遇西人,深目而髯,乃其得實。」明日,有老人過其所,問之,得矜墓,直社北,遂歸葬弘安墓次。時人哀來章孝,皆為出涕云。



 崔敦禮,字安上。祖仲方,在隋為禮部尚書。其先,博陵著姓,魏末,徙為雍州咸陽人。敦禮涉書傳,以節義自將。武德中,官通事舍人。善辭令進止,觀者皆竦。嘗持節幽州召廬江王瑗,瑗已舉兵,執之,脅問朝廷事,敦禮不為言,太宗壯之。還,除左衛郎將,賜金幣良馬。擢中書舍人,四遷兵部侍郎。出為靈州都督。召還,拜兵部尚書。詔撫輯回紇、鐵勒部姓,會薛延陀寇邊,與李勣合兵破之,置祁連州處其餘眾。瀚海都督回紇吐迷度為下所殺,詔往綏定,立其嗣而還。敦禮通知四夷情偽,其少,慕蘇武為人,故屢使突厥,前後建明,允會事機。



 永徽四年,拜侍中,監脩國史。累封固安縣公。進中書令兼檢校太子詹事。以久疾,自言不任事奉兩宮。更拜太子少師、同中書門下三品。弟餘慶,時為定襄都督府司馬,召使侍疾。卒,年六十一。高宗為舉哀東雲龍門,賻布、秘器尤厚,贈開府儀同三司、並州大都督,謚曰昭,陪葬昭陵。餘慶位亦至兵部尚書。



 楊弘禮,字履莊,隋尚書令素弟之子。雅與玄感不心辦,嘗表其必亂。玄感誅,父岳系長安獄,煬帝使赦之,比至,岳已死。高祖即位,以素有功於隋,詔弘禮襲清河郡公,除太子通事舍人。貞觀中,累遷中書舍人。



 太宗征遼東,拜兵部侍郎。駐蹕之役,領步騎二十四軍跳出賊背,所向摧靡。帝自山下望其眾,袍仗精整,人人盡力,壯之,謂許敬宗曰:「越公兒郎,故有家風。」時宰相悉留定州輔皇太子,唯褚遂良、敬宗、弘禮掌行在機務。還,拜中書侍郎。遷司農卿。為昆丘道副大總管,破處密,殺焉耆王,降馺支部,獲龜茲、于闐王,凱旋。會帝崩,大臣疾之,下遷涇州刺史。永徽初,追論其功,遷勝州都督,改太府卿。卒,贈蘭州都督,謚曰質。



 弟弘武。弘武少修謹。永徽中,累為吏部郎中、太子中舍人。高宗東封泰山,自荊州司馬擢司戎少常伯,從帝。還,詔補授吏部五品官,遷西臺侍郎。帝嘗讓曰:「爾在戎司,授官多非其才,何邪?」弘武曰:「臣妻剛悍,此其所屬,不敢違。」以諷帝用後言也。帝笑不罪。乾封二年,同東西臺三品。弘武無它才,特謙慎自守,然居職以清簡稱。卒,贈汴州刺史,謚曰恭。



 三子:元亨、元禧、元禕。



 元禧為尚舍奉御,善醫,武后所信愛。嘗忤張易之,易之奏「素在隋有逆節,子孫不可供奉」。後乃詔「素及兄弟有子若孫不得任京官及侍衛。」貶元亨睦州刺史,元禧資州刺史,元禕梓州司馬。易之誅,復任京官,並至刺史。



 纂,字續卿,弘禮族父。大業時,第進士,為朔方郡司法書佐。坐玄感近屬,廢居蒲城。高祖度河,上謁長春宮。遷累侍御史。數上書言事,稱旨,除考功郎中。貞觀初,為長安令,賜爵長安縣男。有告女子袁妖逆者,纂按之,情不得。袁敗,太宗惡其不忠,將殺之,中書令溫彥博以過誤當宥,乃免。後為吏部侍郎,有俗才,抑文雅,進黠吏,度時舞數以自進。終戶部尚書,贈幽州都督,謚曰恭。



 纂從子昉,武后時為肅機。宇文化及子訴治先廕,昉方食,未即判,遽曰:「肅機,而未食,庸知天下有冤而求食乎?」昉怒,取牒署曰:「父弒隋主,子訴隋資,可乎?」人服其敏。終工部尚書。



 盧承慶,字子餘,幽州涿人,隋散騎侍郎思道之孫。父赤松,為河東令,與高祖雅故,聞兵興,迎見霍邑,拜行臺兵部郎中,終率更令、範陽郡公。承慶美儀矩,博學而才。少襲爵。貞觀初,為秦州參軍,入奏軍事,太宗偉其辯,擢考功員外郎。累遷民部侍郎。帝問歷代戶版,承慶敘夏、商至周、隋增損曲折,引據該詳,帝嗟賞。俄兼檢校兵部侍郎,知五品選,辭曰:「選事在尚書,臣掌之為出位。」帝不許,曰:「朕信卿,卿何不自信?」歷雍州別駕、尚書左丞。



 高宗永徽時,坐事貶簡州司馬。閱歲,改洪州長史。帝將幸汝湯泉,故拜汝州刺史。顯慶四年,以度支尚書同中書門下三品,坐調非法,免。俄拜潤州刺史。拜刑部尚書。以金紫光祿大夫致仕,卒。臨終,誡其子曰:「死生至理,猶朝有暮。吾死,斂以常服,晦朔無薦牲,葬勿卜日,器用陶漆,棺而不槨,墳高可識,碑志著官號年月,無用虛文。」贈幽州都督,謚曰定。



 初,承慶典選,校百官考,有坐漕舟溺者,承慶以「失所載,考中下」。以示其人,無慍也。更曰「非力所及,考中中」。亦不喜。承慶嘉之曰:「寵辱不驚,考中上。」其能著人善類此。



 弟承業、承泰。承業繼為雍州長史、尚書左丞,有能名。



 承泰,字齊卿,長安初,為雍州參軍。武后詔長史薛季昶擇僚吏堪御史者,季昶訪於齊卿。齊卿白長安尉盧懷慎、李休光,萬年尉李乂、崔湜,咸陽丞倪若水,盩厔尉田崇壁,新豐尉崔日用。季昶用其言,後皆為通顯巨人。及拜幽州刺史,而張守珪隸果毅,齊卿厚遇,曰:「君十年至節度使。」已而果然。喜飲酒,逾鬥不亂。寬厚樂易,士友以此親之。終太子詹事、廣陽縣公。承慶從孫藏用別有傳。



 劉祥道,字同壽,魏州觀城人。父林甫,武德時為內史舍人,典機密,以才稱。與蕭瑀等撰定律令,著《律議》萬餘言。歷中書、吏部二侍郎,賜爵樂平縣男。唐沿隋制,十一月選集,至春停,日薄事叢,有司不及研諦。林甫建請四時聽選,隨到輒擬,於是官無滯人。始,天下初定,州府及詔使以赤牒授官,至是罷,悉集吏部調,至萬員,林甫隨才銓錄,咸以為宜,論者方隋高孝基。



 祥道少襲爵,歷御史中丞。顯慶中,遷吏部黃門侍郎,知選事。既世職,乃厘補敝闕,上疏陳六事:



 一曰:今取士多且濫。入流歲千四百,多也;雜色入流,未始銓汰,濫也。故共務者,善人少,惡人多。臣謂應雜色進者,切責有司試判為四等,第一付吏部,二付兵部,三付主爵,四付司勛。若坐負當責,雖經赦,仍配三司,不者還本貫,則官不雜矣。



 二曰:內外官,一品至九品萬三千四百六十五員。大抵三十而仕,六十而退,取其中數,不三十年,存者略盡。若歲入流五百人,則三十年自相充補。況三十年外,在官猶多,不慮其少。今入流歲千四百,其倍兩之,又停選六七千人,復年別新加,其類浸廣,殆非經久之制。古者為官擇人,不聞取人多而官少也。



 三曰:永徽以來,在官者或以善政擢,論事者或以單言進,而庠序諸生未聞甄異,是獎勸之道未周也。



 四曰:唐有天下四十年,未有舉秀才者,請自六品以下至草野,審加搜訪,無令赫赫之辰,斯學遂絕。



 五曰:唐、虞三載考績,黜陟幽明。二漢用人,亦久其職。今任官率四考罷,官知秩滿,則懷去就;民知遷徙,則茍且。以去就之官,臨茍且之民,欲移風振俗,烏可得乎?請四考進階,八考聽選,以息迎新送故之弊。



 六曰:三省都事、主事、主書,比選補,皆取流外有刀筆者,雖欲參用士流,率以儔類為恥。前後相沿,遂成故事。且掖省崇峻,王言秘密,尚書政本,人物所歸,專責曹史,理有未盡,宜稍革之,以清其選。



 會中書令杜正倫亦言入流者眾,為官人敝,乃詔與祥道參議,而執政憚改作,又以勛戚子進取無他門,遂格。



 稍遷司刑太常伯。每覆大獄,必歔欷累嘆。奏決日,為再不食。詔巡察關內道,多振冤滯。兼沛王府長史。麟德元年,拜右相。祥道性審謹,居宰相,憂畏不自堪,數陳老病丐解。坐與上官儀善,罷為司禮太常伯。高宗封泰山,有司請太常卿亞獻,光祿卿終獻。祥道建言:「三代六卿重,故得佐祠。漢、魏以來,權歸臺省,九卿為常伯屬官。今封岱大禮不以八坐,用九卿,無乃徇古名忘實事乎?」帝可其議,以司徒徐王元禮亞獻,祥道終獻。禮成,進爵廣平郡公。乾封元年,以金紫光祿大夫致仕。卒,年七十一,贈幽州都督,謚曰宣。



 子齊賢,襲爵,由侍御史出為晉州司馬。帝以其方直,尊憚之。時將軍史興宗從獵苑中,言晉州出佳鷂,可捕取。帝曰:「齊賢豈捕鷂人邪?卿安得以此待之?」累遷黃門侍郎,脩國史。永淳元年,進同中書門下平章事。武后時,代裴炎為侍中,辨炎不反,後怒,左遷普州刺史,道貶吉州長史。永昌中,為酷吏所陷,系州獄,自經死,沒其家。建中三年,贈太子太保。



 齊賢三世至兩省侍郎,典選。從父應道吏部郎中,從父弟令植禮部侍郎,凡八人前後歷吏部郎中、員外,世以為罕。



 令植孫從一,擢進士宏詞第,調渭南尉。雅為常袞、盧杞所厚,薦授監察御史。普王討李希烈,表為元帥判官。德宗居奉天,超拜刑部侍郎、同中書門下平章事。從幸梁州,改中書侍郎,帝遇之善。然無它材能,容身遠罪而已。貞元初,以疾自乞,罷為戶部尚書。卒,贈太子太傅。



 李敬玄,亳州譙人。該覽群籍,尤善於禮。高宗在東宮,馬周薦其材,召入崇賢館侍讀,假中秘書讀之。為人峻整,然造請不憚寒暑。許敬宗頗薦延之。歷西臺舍人,弘文館學士。遷右肅機,檢校太子右中護。拜西臺侍郎、同東西臺三品,兼檢校司列少常伯。時員外郎張仁禕有敏才,敬玄委以曹事,仁禕為造姓歷、狀式、銓簿,鉗鍵周密,病心太勞死。敬玄因其法,衡綜有序。自永徽後,選員浸多,惟敬玄居職有能稱。性強記,雖官萬員,遇諸道,未嘗忘姓氏。有來訴者,口諭書判參舛及殿累本末無少繆,天下伏其明。杭州參軍徐太玄哀其僚張惠以贓抵死,而惠母老,乃詣獄自言與惠偕受,薄其罪,惠得不死,太玄坐免官十年。敬玄廉知之,擢為鄭州司功參軍,後至秘書少監、申王師,以德行聞。其鑒拔率若此。



 咸亨二年,轉中書侍郎。又改吏部,兼太子右庶子、同中書門下三品,監修國史。進吏部尚書。居選部久,人多附向。凡三娶皆山東舊族,又與趙李氏合譜,故臺省要職多族屬姻家。高宗知之,不能善也。儀鳳元年,拜中書令,封趙國公。



 劉仁軌西討吐蕃,有所建請,敬玄數持異,由是有隙,因奏河西鎮守非敬玄不可。敬玄辭以非將帥才,且仁軌逞憾,故強臣以不能。帝厭之,因曰:「仁軌若須朕,朕且行,卿安得辭?」乃拜洮河道大總管,兼鎮撫大使,檢校鄯州都督,統兵十八萬,代仁軌。與吐蕃將論欽陵戰青海,使劉審禮為先鋒,麈虜,敬玄按軍自如,審禮戰歿,尚首鼠不進,乃頓承風嶺,又陰溝淖,莫能前,賊屯高壓其營。偏將黑齒常之率死士夜擊賊,敬玄始得至鄯州。又戰湟川,遂大敗。數稱疾求罷歸,許之。既入見,不引謝,即還府視事。帝察實不病,貶衡州刺史。久之,遷揚州長史。卒官,贈兗州都督,謚曰文憲。撰次《禮論》及它書數十百篇。二子:思沖、守一。



 思沖,神龍初,歷工部侍郎、左羽林軍將軍,從節愍太子誅武三思,見殺,籍其家。守一郫令。孫紳別傳。



 敬玄弟元素,為武德令。刺史李文暕橫調民黃金造常滿尊以獻,官屬無敢諫,元素固爭,文暕為少損,更以私財助之。延載初,繇文昌左丞遷鳳閣侍郎、同鳳閣鸞臺平章事。為武懿宗所構,與綦連耀等同誅。神龍中,追洗其辜。



 劉德威,徐州彭城人。姿貌魁秀,有幹略。隋大業末,從裴仁基討淮賊,手劍賊酋,傳行在。後歸李密,密分麾下兵使守懷州。密降,俱入朝,授左武候將軍,封滕縣公。詔將兵擊劉武周,因判並州總管府司馬。裴寂失律,齊王元吉棄州遁,德威總留府事。賊薄城,民皆叛附賊,遂為武周所獲,使率本部徇地浩州,得自拔歸,盡上賊中虛實,高祖嘉納,改彭城縣公。未幾,檢校大理少卿,從平洛陽,有功,轉刑部侍郎,加散騎常侍,妻以平壽縣主。



 貞觀初,歷大理卿、綿州刺史。政號廉平,百姓立石頌德。尋檢校益州大都督府長史。入為大理卿。太宗問曰:「比刑網浸密,咎安在?」德威曰:「在君不在臣。下之寬猛,視主之好。律:失入者減三,失出者減五。今坐入者無辜,坐出者有罪,所以吏務深文,為自營計,非有教使然也。」帝然其言。後遷刑部尚書,檢校雍州別駕。詔至齊州按齊王祐獄,還,半道聞祐反,入據濟州。詔德威就發河南兵經略之,會母喪免。既除,為同州刺史。永徽三年,卒官,年七十一,贈禮部尚書、幽州都督,謚曰襄,陪葬獻陵。



 德威於閨門友睦,為人寬平,生平所得奉祿,以分宗親,無留藏。子審禮。



 審禮少喪母,為祖母元所養。隋末大亂,道不通,審禮尚少,自鄉里負祖母度江,轉側避地。及天下平,西入長安。元每疾病,必親煮藥,嘗而進。元曰:「兒孝通幽顯,吾一顧念,疾輒間。」貞觀中,歷左驍衛郎將。父喪免。比葬,徙跣血流,行路咨嘆。服除,當襲爵,讓其弟,不聽。見父執必感泗滂沱。事繼母尤謹,與弟延景為聞友,得祿多資之,而妻子執寒苦,晏如也。再從皆同居,合二百口,內外無間言。遷工部尚書,檢校左衛大將軍。



 儀鳳三年,吐番寇涼州,副中書令李敬玄討之。遇虜青海上,與戰,敬玄逗撓不前,審禮敗,為虜執。其子尚乘直長殆庶及延景詣闕待罪,請入賊以贖。有詔審禮徇忠以沒,非有罪,宜各還職。特詔殆庶弟易從省之。既至,而審禮卒,易從晝夜哭不止,吐番哀其志,乃還父尸,徙跣萬里,扶護以歸,見者流涕。審禮贈工部尚書,謚曰僖。



 延景,字冬日,終陜州刺史。睿宗初,以後父追贈尚書右僕射,陪葬乾陵。



 易從累遷彭州長史、任城縣男。永昌中,為酷吏周興誣構,坐死。將刑,百姓奔走,爭解衣投地,曰:「為長史祈福。」有司平直,乃十餘萬。當時號「孝義劉家」。及易從以非禍死,天下冤之。



 子升,年十餘歲流嶺表,六道使誅流人,升以信愛為首領所庇免。後易姓溫,北歸洛。景雲中,特授右武衛騎曹參軍。開元中,累遷中書舍人、太子右庶子。升能文,善草隸。



 審禮從弟延嗣,為潤州司馬。徐敬業攻潤州,延嗣與刺史固守。俄而城陷,敬業邀以降,延嗣曰:「吾世蒙恩,今城不守,所負多矣,詎能茍生為宗族羞?」敬業怒,將斬之,其黨魏思溫救止,系江都獄。敬業敗,錄忠當敘,以裴炎近親,裁遷梓州長史。轉汾州刺史。宗族至刺史者二十餘人。



 孫處約,始名道茂,汝州郟城人。貞觀中,為齊王祐記室。祐多過失,數上書切諫。王誅,帝得其書,咨嘆之,擢中書舍人。高宗即位,令杜正倫請增舍人員。帝曰:「處約一人,足辦我事。」止不除。以論譔勞,數賜段物。再遷司禮少常伯。麟德元年,以西臺侍郎同東西臺三品。為少司成,以老致仕,卒。



 子佺,延和初,為羽林將軍、幽州都督,率兵十二萬討奚李大酺,分三屯,以副將李楷洛、周以悌領之。次冷硎,楷洛與大酺戰,不勝,壯校多沒。佺氣褫,乃紿言:「天子詔我招慰奚,楷洛違詔妄戰,當斬。」遣人謝大酺。大酺曰:「審爾,願出天子賜,明不欺。」佺揪聚軍中幣萬餘匹,悉袍、帶並與之。大酺知佺詐,好語勸引還,而佺部伍離沮,奚逼之,大敗,死者數萬。佺、以悌同見獲,送默啜所殺之。



 邢文偉,滁州全椒人。與歷陽高子貢、壽春裴懷貴俱以博學聞。咸亨中,歷太子典膳丞。時孝敬罕見宮臣,文偉即減膳,上書曰:「古者太子既冠,則有司過之史、虧膳之宰。史不書過,死之;宰不徹膳,死之。皇帝簡料英俊,自庶子至司議、舍人、學士、侍讀,使佐殿下,成就聖德。比者不甚廷議,謁對稀簡,三朝之後,與內人獨居,何繇發揮天資,使浚哲文明哉?今史既闕官,宰得奉職,謹守禮經以聞。」太子答曰:「幼嗜墳典,欲研精極意,而未閑將衛,耽誦致勞。比苦風虛,奉陛下恩旨,不許強勉,加以趨侍朝夕,無自專之道,屢闕坐朝,乖廢學緒。觀尋來請,良符宿志。自非義均弼諧,渠能進此藥石?」文偉由是益知名。後右史缺,高宗謂侍臣曰:「文偉切諫吾兒,此直臣也。」遂授之。



 武后時,累遷鳳閣侍郎,兼弘文館學士。載初元年,為內史。後御明堂,詔文偉發《孝經》。後問:「天與帝異稱云何?」文偉曰:「天、帝一也。」制曰:「郊後稷以配天,祀文王於明堂以配上帝,奈何而一?」對曰:「先儒執論不同,昊天及五方總六天帝。」後曰:「帝有六,則天不同稱,固矣。」文偉不得對。後曰:「移風易俗,莫善於樂。伯牙鼓琴,鐘期聽之,知意在山水,是人能移風易俗矣。何取樂邪?」文偉曰:「聖人作樂,平人心,變風俗。末世樂壞,則為人所移。」後喜,賜帛。宗秦客以奸贓抵罪,文偉坐所善,貶珍州刺史。會它使者至,文偉內悸,自經死。



 高子貢,善《太史書》,與硃敬則善,擢明經。歷秘書省正字、弘文館直學士。不得志,因棄官去。徐敬業起兵,弟敬猷統兵五千逼和州,子貢率鄉人數百拒之,賊引去。以功擢朝散大夫,為成均助教。東莞公融嘗為和州刺史,從子貢受業。及融謀舉兵,令黃公譔見子貢,推為謀主,書疏往返,因結諸王內應。謀洩,坐死。



\end{pinyinscope}