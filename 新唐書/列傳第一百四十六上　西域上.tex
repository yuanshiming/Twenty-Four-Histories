\article{列傳第一百四十六上 西域上}

\begin{pinyinscope}

 泥婆羅,直吐蕃之西樂陵川。土多赤銅、犛牛。俗翦發逮眉,穿耳,楦以筒若角,緩至肩者為姣好。無匕箸覺的恆久可能性」即世界就是眼前的感覺和可能叫感覺,物,攫而食。其器皆用銅,其居版屋畫壁。俗不知牛耕,故少田作,習商賈。一幅布蔽身,日數盥浴。重博戲,通推步歷術。祀天神,鐫石為象,日浴之,烹羊以祭。鑄銅為錢,面文人形,背牛馬形。其君服珠、頗黎、車渠、珊瑚、虎魄垂纓,耳金鉤玉璫,佩寶伏突,御師子大床,燎香布花於堂,而大臣坐地不藉。左右持兵,數百列侍。宮中有七重樓,覆銅瓦,楹極皆大琲雜寶,四隅置銅槽,下有金龍,口激水仰注槽中。



 初,王那陵提婆之父為其叔所殺,提婆出奔,吐蕃納之,遂臣吐蕃。貞觀中,遣使者李義表到天竺,道其國,提婆大喜,延使者同觀阿耆婆珎池。池廣數十丈,水常溢沸,共傳旱潦未始耗溢。或抵以物則生煙,釜其上,少選可熟。二十一年,遣使入獻波棱、酢菜、渾提蔥。永徽時,其王尸利那連陀羅又遣使入貢。



 黨項,漢西羌別種,魏、晉後微甚。周滅宕昌、鄧至,而黨項始強。其地古析支也,東距松州,西葉護,南舂桑、迷桑等羌,北吐谷渾。處山谷崎嶇,大抵三千里。以姓別為部,一姓又分為小部落,大者萬騎,小數千,不能相統,故有細封氏、費聽氏,往利氏、頗超氏、野辭氏、房當氏、米禽氏、拓拔氏、而拓拔最強。土著,有棟宇,織犛尾、羊毛覆屋,歲一易。俗尚武,無法令、賦役。人壽多過百歲,然好為盜,更相剽奪。尤重復讎,未得所欲者,蓬首垢顏,跣足草食,殺已乃復。男女衣裘褐,被氈。畜犛牛、馬、驢、羊以食,不耕稼。地寒,五月草生,八月霜降。無文字,候草木記歲。三年一相聚,殺牛羊祭天,取麥他國以釀酒。妻其庶母、伯叔母、兄嫂、子弟婦,惟不娶同姓。老而死,子孫不哭;少死,則曰夭枉,乃悲。



 貞觀三年,南會州都督鄭元鐫諭,其酋細封步賴舉部降。太宗璽詔慰撫,步賴因入朝,宴錫特異,以其地為軌州,即授刺史。步賴請率兵討吐谷渾。其後諸酋長悉內屬,以其地為崌、奉、嚴、遠四州,即首領拜刺史。



 有拓拔赤辭者,初臣吐谷渾,慕容伏允待之厚,與結婚,諸羌已歸,獨不至。李靖擊吐谷渾,赤辭屯狼道峽抗王師。廓州刺史久且洛生欲諭降之,辭曰:「渾主以腹心待我,不知其佗,若速去,且污吾刀。」洛生怒,引輕騎破之肅遠山,斬首數百級,虜雜畜六千。帝因其勝又令約降,赤辭從子思頭潛納款,其下拓拔細豆亦降。赤辭知宗族攜沮,稍欲自歸,岷州都督劉師立復誘之,即與思頭俱內屬。以其地為懿、嵯、麟、可三十二州,以松州為都督府,擢赤辭西戎州都督,賜氏李,貢職遂不絕。於是自河首積石山而東,皆為中國地。後吐蕃浸盛,拓拔畏逼,請內徙,始詔慶州置靜邊等州處之。地乃入吐蕃,其處者皆為吐蕃役屬,更號弭藥。



 又有黑黨項者,居赤水西。其長號敦善王,慕容伏允之走也依之。及吐谷渾款附,敦善王亦納貢。居雪山者曰破丑氏。



 又有白蘭羌,吐蕃謂之丁零,左屬黨項,右與多彌接。勝兵萬人,勇戰鬥,善作兵,俗與黨項同。武德六年,使者入朝。明年,以其地為維、恭二州。貞觀六年,與契苾數十萬內屬。永徽時,特浪生羌卜樓大首領凍就率眾來屬,以其地為劍州。



 龍朔後,白蘭、舂桑及白狗羌為吐蕃所臣,籍其兵為前驅。白狗與東會州接,勝兵才千人。在西北者,天授中內附,戶凡二十萬,以其地為朝、吳、浮、歸十州,散居靈、夏間。至德末,為吐蕃所誘,使為鄉導鈔邊。俄悔悟,更來朝,願助靈州餉輓。乾元間,中國數亂,因寇邠、寧二州,肅宗詔郭子儀都統朔方、邠寧、鄜坊節度事,以鄜州刺史杜冕、邠州刺史桑如圭分二隊出討。子儀至,黨項潰去。



 上元元年,在涇、隴部落十萬眾詣鳳翔節度使崔光遠降。二年,與渾、奴剌連和,寇寶雞,殺吏民,掠財珍,焚大散關,入鳳州,殺刺史蕭心曳,節度使李鼎追擊走之。明年,又攻梁州,刺史李勉走;進寇奉天,大掠華原、同官去。詔臧希讓代勉為刺史,於是歸順、乾封、歸義、順化、和寧、和義、保善、寧定、羅雲、朝鳳凡十州部落詣希讓獻款,丐節印。詔可。



 僕固懷恩之叛,誘黨項、渾、奴剌入寇,眾數萬,掠鳳翔、盩厔。大酋鄭廷、郝德入同州,刺史韋勝走,節度使周智光破之澄城。閱月,又入同州,焚官私室廬,壁馬蘭山。郭子儀遣兵襲之,退保三堡,子儀遣慕容休明諭降廷、德。



 子儀以黨項、吐谷渾部落散處鹽、慶等州,其地與吐蕃濱近,易相脅,即表徙靜邊州都督、夏州、樂容等六府黨項于銀州之北、夏州之東、寧朔州吐谷渾住夏西,以離沮之。召靜邊州大首領左羽林大將軍拓拔朝光等五刺史入朝,厚賜齎,使還綏其部。先是,慶州有破丑氏族三、野利氏族五、把利氏族一,與吐蕃姻援,贊普悉王之,因是擾邊凡十年。子儀表工部尚書路嗣恭為朔方留後,將作少監梁進用為押黨項部落使,置行慶州。且言:「黨項陰結吐蕃為變,可遣使者招慰,芟其反謀,因令進用為慶州刺史,嚴邏以絕吐蕃往來道。」代宗然之。又表置靜邊、芳池、相興王州都督、長史,永平、旭定、清寧、寧保、忠順、靜塞、萬吉等七州都督府。於是破醜、野利、把利三部及思樂州刺史拓拔乞梅等皆入朝,宜定州刺史折磨布落、芳池州野利部並徙綏、延州。大歷末,野利禿羅都與吐蕃叛,招餘族不應,子儀擊之,斬禿羅都,而野利景庭、野利剛以其部數千人入附雞子川。六州部落,曰:野利越詩、野利龍兒、野利厥律、兒黃、野海、野窣等;居慶州者號東山部,夏州者號平夏部。永泰後稍徙石州,後為永安將阿史那思暕賦索無極,遂亡走河西。



 元和時復置宥州,護黨項。至大和中寢強,數寇掠。然器械鈍苦,畏唐兵精,則以善馬購鎧,善羊貿弓矢。鄜坊道軍糧使李石表禁商人不得以旗幟、甲胄、五兵入部落,告者,舉罪人財畀之。至開成末,種落愈繁,富賈人齎繒寶鬻羊馬,籓鎮乘其利,強市之,或不得直,部人怨,相率為亂,至靈、鹽道不通。武宗以侍御史為使招定,分三印,以邠、寧、延屬崔彥曾,鹽、夏、長澤屬李雩鄠,靈武、麟、勝屬鄭賀,皆緋衣銀魚,而功不克。



 宣宗大中四年,內掠邠、寧,詔鳳翔李業、河東李拭合節度兵討之,宰相白敏中為都統。帝出近苑,或以竹一個植舍外,見才尺許,遠且百步,帝屬二矢曰:「黨羌窮寇,仍歲暴吾鄙,今我約:射竹中則彼當自亡,不中,我且索天下兵翦之,終不以此賊遺子孫。」左右注目,帝一發竹分,矢徹諸外,左右呼萬歲。不閱月,羌果破殄,餘種竄南山。



 始,天寶末,平夏部有戰功,擢容州刺史、天柱軍使。其裔孫拓拔思恭,咸通末竊據宥州,稱刺史。黃巢入長安,與鄜州李孝昌壇而坎牲,誓討賊,僖宗賢之,以為左武衛將軍,權知夏綏銀節度事。次王橋,為巢所敗,更與鄭畋四節度盟,屯渭橋。中和二年,詔為京城西面都統、檢校司空、同中書門下平章事。俄進四面都統,權知京兆尹。賊平,兼太子太傅,封夏國公,賜姓李。嗣襄王煴之亂,詔思恭討賊,兵不出,卒。以弟思諫代為定難節度使,思孝為保大節度、鄜坊凡翟等州觀察使,並檢校司徒、同中書門下平章事。王行瑜反,以思孝為北面招討使,思諫東北面招討使。思孝亦因亂取鄜州,遂為節度使,累兼侍中。以老薦弟思敬為保大軍兵馬留後,俄為節度使。



 東女,亦曰蘇伐剌挐瞿咀羅,羌別種也,西海亦有女自王,故稱「東」別之。東與吐蕃、黨項、茂州接,西屬三波訶,北距於闐,東南屬雅州羅女蠻、白狼夷。東西行盡九日,南北行盡二十日。有八十城。以女為君,居康延川,巖險四繚,有弱水南流,縫革為船。戶四萬,勝兵萬人。王號賓就,官曰高霸黎,猶言宰相也。官在外者,率男子為之。凡號令,女官自內傳,男官受而行。王侍女數百,五日一聽政。王死,國人以金錢數萬納王族,求淑女二立之。次為小王,王死,因以為嗣,或姑死婦繼,無篡奪。所居皆重屋,王九層,國人六層。王服青毛綾裙,被青袍,袖委於地,冬羔裘,飾以文錦。為小鬟髻,耳垂璫。足曳索輶。索輶,履也。俗輕男子,女貴者咸有侍男,被發,以青塗面,惟務戰與耕而已。子從母姓。地寒宜麥,畜羊馬,出黃金。風俗大抵與天竺同。以十一月為正。巫者以十月詣山中,布糟麥,咒呼群鳥。俄有鳥來如雞狀,剖視之,有穀者歲豐,否即有災,名曰鳥卜。居喪三年,不易服,不櫛沐。貴人死,剝藏其皮,內骨甕中,糅金屑瘞之。王之葬,殉死至數十人。



 武德時,王湯滂氏始遣使入貢。高祖厚報,為突厥所掠不得通。貞觀中,使復至,太宗璽制慰撫。顯慶初,遣使高霸黎文與王子三廬來朝,授右監門中郎將。其王斂臂使大臣來請官號,武后冊拜斂臂左玉鈐衛員外將軍,賜瑞錦服。天授、開元間,王及子再來朝,詔與宰相宴曲江,封王曳夫為歸昌王、左金吾衛大將軍。後乃以男子為王。



 貞元九年,其王湯立悉與白狗君及哥鄰君董臥庭、逋租君鄧吉知、南水君薛尚悉曩、弱水君董避和、悉董君湯息贊、清遠君蘇唐磨、咄霸君董藐蓬皆詣劍南韋皋求內附。其種散居西山、弱水,雖自謂王,蓋小小部落耳。自失河、隴,悉為吐蕃羈屬,部數千戶,輒置令,歲督絲絮。至是猶上天寶所賜詔書。皋處其眾於維、霸等州,賜牛、糧,治生業。立悉等入朝,差賜官祿。於是松州羌二萬口相踵入附。立悉等官刺史,皆得世襲,然陰附吐蕃,故謂「兩面羌」。



 高昌,直京師西四千里而贏,其橫八百里,縱五百里,凡二十一城。王都交河城,漢車師前王廷也。田地城,戊己校尉所治也。勝兵萬人。土沃,麥、禾皆再熟。有草名白疊,擷花可織為布。俗辮髻垂後。



 其王曲伯雅,隋時嘗妻以戚屬宇文氏女,號華容公主。武德初,伯雅死,子文泰立,遣使來告,高祖命使者臨吊。後五年,獻狗高六寸,長尺,能曳馬銜燭,雲出拂菻,中國始有拂菻狗。



 太宗即位,獻玄狐裘,帝賜妻宇文華金奠一具,宇文亦上玉盤。凡諸國施為輒以聞。貞觀四年,文泰遂來朝,禮賜厚甚。宇文求預宗籍,有詔賜氏李,更封常樂公主。



 久之,文泰與西突厥通,凡西域朝貢道其國,咸見壅掠。伊吾嘗臣西突厥,至是內屬,文泰與葉護共擊之。帝下詔讓其反覆,召大臣冠軍阿史那矩計事,文泰不遣,使長史曲雍來謝罪。初,大業末,華民多奔突厥,及頡利敗,有逃入高昌者,有詔護送,文泰苛留之。又與西突厥乙毘設破焉耆三城,虜其人,焉耆王訴諸朝。帝遣虞部郎中李道裕問狀,復遣使謝。帝引責曰:「而主數年朝貢不入,無籓臣禮,擅置官,擬效百僚。今歲首萬君長悉來,而主不至。日我使人往,文泰猥曰:『鷹飛於天,雉竄於蒿,貓游於堂,鼠安於穴,各得其所,豈不快邪!』西域使者入貢,而主悉拘梗之。又諗薛延陀曰:『既自為可汗,與唐天子等,何事拜謁其使?』明年我當發兵虜而國,歸謂而君善自圖。」時薛延陀可汗請為軍向導,故民部尚書唐儉至延陀堅約。



 帝復下璽書示文泰禍福,促使入朝,文泰遂稱疾不至。乃拜侯君集為交河道大總管,左屯衛大將軍薛萬均、薩孤吳仁副之,契苾何力為蔥山道副大總管,武衛將軍牛進達為行軍總管,率突厥、契苾騎數萬討之。群臣諫以行萬里兵難得志,且天界絕域,雖得之,不可守。帝不聽。文泰謂左右曰:「曩吾入朝,見秦、隴北城邑蕭條,非有隋比。今伐我,兵多則糧軵不逮;若下三萬,我能制之。度磧疲鈍,以逸待勞,臥收其弊耳。」十四年,聞王師至磧口,悸駭無它計,發病死,子智盛立。



 君集奄攻田地城,契苾何力以前軍鏖戰。是夜星墜城中,明日拔其城,虜七千餘人。中郎將辛獠兒以勁騎夜逼其都。智盛以書遺君集曰:「得罪於天子者,先王也,咎深譴積,震墜厥命。智盛嗣位未幾,公其見赦。」君集曰:「能悔禍者,當面縛軍門。」智盛不答。軍進,填隍引沖車,飛石如雨,城中大震。智盛令大將曲士義居守,身與綰曹曲德俊謁軍門,請改事天子。君集諭使降,辭示屈,薛萬均勃然起曰:「當先取城,小兒何與語!」麾而進,智盛流汗伏地曰:「唯公命!」乃降。君集分兵略定,凡三州、五縣、二十二城,戶八千,口三萬,馬四千。先是,其國人謠曰:「高昌兵,如霜雪;唐家兵,如日月。日月照霜雪,幾何自殄滅。」文泰捕謠所發,不能得也。



 捷書聞,天子大悅,宴群臣,班賜策功,赦高昌所部,披其地皆州縣之,號西昌州。特進魏徵諫曰:「陛下即位,高昌最先朝謁。俄以掠商胡,遏貢獻,故王誅加焉。文泰死,罪止矣。撫其人,立其子,伐罪吊民,道也。今利其土,屯守常千人,屯士數年一易,辦裝資,離親戚,不十年隴右且空。陛下終不得高昌圭粒咫帛助中國費,所謂散有用事無用。」不納。改西昌州曰西州,更置安西都護府,歲調千兵,謫罪人以戍。黃門侍郎褚遂良諫曰:「古者先函夏,後夷狄,務廣德化,不爭荒逖。今高昌誅滅,威動四夷,然自王師始征,河西供役,飛米轉芻,十室九匱,五年未可復。今又歲遣屯戍,行李萬里,去者資裝使自營辦,賣菽粟,傾機杼,道路死亡尚不計。罪人始於犯法,終於惰業,無益於行。所遣復有亡命,官司捕逮,株蔓相牽。有如張掖、酒泉塵飛烽舉,豈得高昌一乘一卒及事乎?必發隴右、河西耳。然則河西為我腹心,高昌,他人手足也,何必耗中華,事無用?昔陛下平頡利、吐谷渾,皆為立君,蓋罪而誅之,伏而立之,百蠻所以畏威慕德也。今宜擇高昌可立者立之,召首領悉還本土,長為籓翰,中國不擾。」書聞不省。



 初,文泰以金厚餉西突厥欲谷設,約有急為表裏;使葉護屯可汗浮圖城。及君集至,懼不敢發,遂來降,以其地為庭州。焉耆請歸高昌所奪五城,留兵以守。



 君集勒石紀功,凱而旋,俘智盛君臣獻觀德殿。行飲至禮,酺三日。徙高昌豪桀於中國,智盛拜左武衛將軍、金城郡公,弟智湛右武衛中郎將、天山郡公。曲氏傳國九世,百三十四年而亡。



 智湛,麟德中以左驍衛大將軍為西州刺史,卒,贈涼州都督。有子昭,好學。有鬻異書者,母顧笥中金嘆曰:「何愛此,不使子有異聞乎?」盡持易之。昭歷司膳卿,頗能辭章。弟崇裕有武藝,永徽中為右武衛翊府中郎將,封交河郡王,邑至三千戶。終鎮軍大將軍,武後為舉哀,襚以美錦,賻賜甚厚,封爵絕。



 吐谷渾居甘松山之陽,洮水之西,南抵白蘭,地數千里。有城郭,不居也。隨水草,帳室、肉糧。其官有長史、司馬、將軍、王、公、僕射、尚書、郎中,蓋慕諸華為之。俗識文字,其王椎髻黑冒,妻錦袍織裙,金花飾首。男子服長裙繒冒,或冠驩瀍。婦人辮發縈後,綴珠貝。婚禮,富家厚納聘,貧者竊妻去。父死妻庶母,兄死妻嫂。喪有服,葬已即除。民無常稅,用不足,乃斂富室商人,足而止。凡殺人若盜馬者死,它罪贖以物。地多寒,宜麥、菽、粟、蕪菁,出小馬、犛牛、銅、鐵、丹砂。有青海者,周八九百里,中有山,須冰合,游牝馬其上,明年生駒,號龍種。嘗得波斯馬,牧於海,生驄駒,日步千里,故世稱「青海驄」。西北有流沙數百里,夏有熱風,傷行人。風將發,老駝引項鳴,埋鼻沙中,人候之,以氈蔽鼻口乃無恙。



 隋時,其王慕容伏允號步薩缽,嘗寇邊。煬帝遣鐵勒敗之,壁西平;復命觀王雄破其眾。伏允以數十騎入泥嶺,亡去,仙頭王率男女十餘萬降。置郡縣鎮戍,以長子順為質,因王之,統餘眾,俄追還。伏允客黨項,隋亂,因得復故地。



 高祖受命,順自江都還長安,於時李軌據涼州,帝乃約伏允和,令擊軌自效,當護送順。伏允喜,引兵與軌戰庫門,交綏止,即遣使請順,帝遣之。順至,號為大寧王。



 太宗時,伏允遣使者入朝,未還,即寇鄯州。帝遣使者讓,且召伏允;以疾為解,而為子求婚,驗帝意。帝召子親迎,亦稱疾。有詔止婚,遣中郎將康處真臨諭。又掠岷州,都督李道彥擊走之,執名王二,斬級七百。連歲遣名王朝。俄寇涼州,鄯州刺史李玄運表吐谷渾牧馬青海,輕兵掩之,可盡致。乃命左驍衛大將軍段志玄、左驍衛將軍梁洛仁率契苾、黨項兵擊之,未至三十里,志玄等不欲戰,壁而留。虜知之,驅牧馬走。副將李君羨率精騎尾襲懸水上,得牛羊二萬還。



 是時,伏允耄不能事,其相天柱王用事,拘天子行人鴻臚丞趙德楷。帝遣使曉敕,十返,無悛言。貞觀九年,詔李靖為西海道行軍大總管,侯君集積石道,任城王道宗鄯善道,李道彥赤水道,李大亮且末道,高甑生鹽澤道,並為行軍總管,率突厥、契苾兵擊之。黨項內屬羌及洮州羌,皆殺刺史歸伏允。夏四月,道宗破伏允於庫山,俘斬四百。伏允謀入磧疲唐兵,燒野草,故靖馬多饑。道宗曰:「柏海近河源,古未有至者。伏允西走,未知其在,方馬臒糧乏,難遠入,不如按軍鄯州,須馬壯更圖之。」君集曰:「不然。向者段志玄至鄯州,吐谷渾兵輒傅城,彼國方完,逆眾用命也。今虜大敗,斥候無在,君臣相失,我乘其困,可以得志。柏海雖遠,可鼓而至也。」靖曰:「善。」分二軍:靖與大亮、薛萬均以一軍趣北,出其右;君集、道宗以一軍趣南,出其左。靖將薩孤吳仁以輕騎戰曼都山,斬名王,獲五百級。諸將戰牛心堆、赤水源,獲虜將南昌王慕容孝俊,收雜畜數萬。君集、道宗登漢哭山,戰烏海,獲名王梁屈蔥。靖破天柱部落於赤海,收雜畜二十萬。大亮俘名王二十,雜畜五萬,次且末之西。伏允走圖倫磧,將托於闐,萬均督銳騎追亡數百里,又破之。士乏水,刺馬飲血。君集、道宗行空荒二千里,盛夏降霜,乏水草,士糜冰,馬秣雪。閱月,次星宿川,達柏海上,望積石山,覽觀河源。執失思力馳破虜車重。兩軍會於大非川、破邏真穀。



 順之質隋,為金紫光祿大夫、伏允立其弟為太子。順歸,常鞅鞅,自以失位,欲以功自結天子,乃斬天柱王,舉國降。伏允懼,引千餘騎遁磧中,眾稍亡,從者才百騎,窮無聊,即自經死。國人立順為君,稱臣內附,詔封四平郡王,號越胡呂烏甘豆可汗。帝恐未能定其國,遣李大亮率精兵鎮援。



 順久質華,國人不附,卒為下所殺,立其子燕王諾曷缽。諾曷缽幼,大臣爭權。帝詔侯君集就經紀之,始請頒歷及子弟入侍。詔封諾曷缽河源郡王,號為地也拔勒豆可汗,遣淮陽郡王道明持節冊命,賜鼓纛。諾曷缽身入謝,遂請婚,獻馬牛羊萬。比年入朝,乃以宗室女為弘化公主妻之,詔道明及右武衛將軍慕容寶持節送公主。其相宣王跋扈,謀作亂,欲襲公主,劫諾曷缽奔吐蕃。諾曷缽知之,引輕騎走鄯城,威信王以兵迎之。果毅都尉席君買率兵與威信王共討,斬其兄弟三人,國大擾。帝又詔民部尚書唐儉、中書舍人馬周持節撫慰。



 高宗立,以主故,拜駙馬都尉。又獻名馬,帝問馬種性,使者曰:「國之最良者。」帝曰:「良馬人所愛。」詔還其馬。公主表請入朝,遣左驍衛將軍鮮於匡濟迎之。十一月,及諾曷缽至京師,帝又以宗室女金城縣主妻其長子蘇度摸末,拜左領軍衛大將軍。久之,摸末死,主與次子右武衛大將軍梁漢王闥盧摸末來請婚,帝以宗室女金明縣主妻之。既而與吐蕃相攻,上書相曲直,並來請師,天子兩不許。吐谷渾大臣素和貴奔吐蕃,言其情,吐蕃出兵搗虛,破其眾黃河上。諾曷缽不支,與公主引數千帳走涼州。帝遣左武衛大將軍蘇定方為安集大使,平兩國怨。吐蕃遂有其地。



 諾曷缽請內徙。乾封初,更封青海國王。帝欲徙其部於涼州之南山,群臣議不同,帝難之。咸亨元年,乃以右威衛大將軍薛仁貴為邏娑道行軍大總管,左衛員外大將軍阿史那道真、左衛將軍郭待封副之,總兵五萬討吐蕃,且納諾曷缽於故廷。王師敗於大非川,舉吐谷渾地皆陷,諾曷缽與親近數千帳才免。三年,乃徙浩亹水南。諾曷缽以吐蕃盛,勢不抗,而鄯州地狹,又徙靈州,帝為置安樂州,即拜刺史,欲其安且樂云。



 諾曷缽死,子忠立。忠死,子宣超立,聖歷三年,拜左豹韜員外大將軍,襲故可汗號,餘部詣涼、甘、肅、瓜、沙等州降。宰相張錫與右武衛大將軍唐休璟議徙其人於秦、隴、豐、靈間,令不得畔去。涼州都督郭元振以為:「吐谷渾近秦、隴,則與監牧雜處;置豐、靈,又邇默啜;假在諸華,亦不遽移其性也。前日王孝傑自河源軍徙耽爾乙句貴置靈州,既其叛,乃入牧坊掠群馬,瘢夷州縣,是則遷中土無益之成驗。往素和貴叛去,於我無損,但失吐谷渾數十部,豈與句貴比邪?今降虜非強服,皆突矢刃,棄吐蕃而來,宜當循其情,為之制也。當甘、肅、瓜、沙降者,即其所置之。因所投而居,情易安,磔數州則勢自分。順其情,分其勢,不擾於人,可謂善奪戎心者也。歲遣鎮遏使者與宣超兄弟撫護之,無令相侵奪,生業固矣。有如叛去,無損中國。」詔可。宣超死,子曦皓立。曦皓死,子兆立。吐蕃復取安樂州,而殘部徙朔方、河東,語謬為「退渾」。



 貞元十四年,以朔方節度副使、左金吾衛大將軍慕容復為長樂都督、青海國王,襲可汗號。復死,停襲。吐谷渾自晉永嘉時有國,至龍朔三年吐蕃取其地,凡三百五十年,及此封嗣絕矣。



 焉耆國直京師西七千里而贏,橫六百里,縱四百里。東高昌,西龜茲,南尉犁,北烏孫。逗渠溉田,土宜黍、蒲陶,有魚鹽利。俗祝發氈衣。戶四千,勝兵二千,常役屬西突厥。俗尚娛遨,二月朏出野祀,四月望日游林,七月七日祀生祖,十月望日王始出游,至歲盡止。



 太原貞觀六年,其王龍突騎支始遣使來朝。自隋亂,磧路閉,故西域朝貢皆道高昌。突騎支請開大磧道以便行人,帝許之。高昌怒,大掠其邊。西突厥莫賀設與咄陸弩失畢作難,來奔,咄陸弩失畢復攻之,遣使言狀,並貢名馬。咥利失可汗立,素善焉耆,故倚為援。十二年,處月、處蜜與高昌攻陷其五城,掠千五百人,焚廬舍。侯君集討高昌,遣使與相聞,突騎支喜,引兵佐唐。高昌破,歸向所俘及城,遣使者入謝。



 西突厥臣屈利啜為弟娶突騎支女,遂相約為輔車勢,不朝貢。安西都護郭孝恪請討之。會王弟頡鼻、慄婆準葉護等三人來降,帝即命孝恪為西州道總管,率兵出銀山道,以慄婆準等為鄉導。初,焉耆所都周三十里,四面大山,海水繚其外,故恃不為虞。孝恪倍道絕水,夜傅堞,遲曙噪而登,鼓角轟哄,唐兵縱,國人擾敗,斬千餘級,執突騎支,更以慄婆準攝國事。始,帝語近臣曰:「孝恪以八月十一日詣焉耆,閱二旬可至,當以二十二日破之,使者今至矣!」俄而遽人以捷布聞。囚突騎支及妻子送洛陽,有詔赦罪。



 屈利啜以兵救焉耆,而孝恪還三日矣。屈利啜囚慄婆準,更使吐屯攝王,遣使以告。帝曰:「焉耆我所下,爾乃王之邪?」吐屯懼,不敢王。焉耆立慄婆準,而從兄薛婆阿那支自為王,號瞎干,執慄婆準獻龜茲,殺之。阿史那社爾討龜茲,阿那支奔之,壁東境抗王師,為社爾所禽,數其罪,斬以徇。立突騎支弟婆伽利為王,以其地為焉耆都督府。



 婆伽利死,國人請還前王突騎支,高宗許之,拜左衛大將軍,歸國。死,龍嫩突立。武后長安時,以其國小人寡,過使客不堪其勞,詔四鎮經略使禁止傔使私馬、無品者肉食。開元七年,龍嫩突死,焉吐拂延立。於是十姓可汗請居碎葉,安西節度使湯嘉惠表以焉耆備四鎮。詔焉耆、龜茲、疏勒、于闐征西域賈,各食其征,由北道者輪臺征之。訖天寶常朝賀。



 龜茲,一曰丘茲,一曰屈茲,東距京師七千里而贏,自焉耆西南步二百里,度小山,經大河二,又步七百里乃至。橫千里,縱六百里。土宜麻、麥、粳稻、蒲陶,出黃金。俗善歌樂,旁行書,貴浮圖法。產子以木壓首。俗斷發齊頂,惟君不翦發。姓白氏。居伊邏廬城,北倚河羯田山,亦曰白山,常有火。王以錦冒頂,錦袍、寶帶。歲朔,鬥羊馬橐它七日,觀勝負以卜歲盈耗云。蔥嶺以東俗喜淫,龜茲、于闐置女肆,徵其錢。



 高祖受禪,王蘇伐勃駃遣使入朝。會死,子蘇伐疊立,號時健莫賀俟利發。貞觀四年獻馬,太宗賜璽書,撫慰加等。後臣西突厥。郭孝恪伐焉耆,乃遣兵與焉耆影援,自是不朝貢。



 蘇伐疊死,弟訶黎布失畢立。二十一年,兩遣使朝貢,然帝怒其佐焉耆叛,議討之。是夜月食昴,詔曰:「月陰精,用刑兆也;星胡分,數且終。」乃以阿史那社爾為昆丘道行軍大總管,契苾何力副之,率安西都護郭孝恪、司農卿楊弘禮、左武衛將軍李海岸等發鐵勒十三部兵十萬討之。社爾分五軍掠其北,執焉耆王阿那支。龜茲大恐,酋長皆棄城走。社爾次磧石,去王城三百里。先遣伊州刺史韓威以千騎居前,右驍衛將軍曹繼叔次之。至多褐,與王遇,其將羯獵顛兵五萬合戰。威偽北,王見威兵少,麾而進,威退與繼叔合,還戰,大破之,追奔八十里。王嬰城,社爾將圍之,王引突騎西走,城遂拔,孝恪居守。沙州刺史蘇海政、行軍長史薛萬備以精騎窮躡六百里。王計窮,保撥換城,社爾圍之。閱月,執王及羯獵顛。其相那利夜逸,以西突厥並國人萬餘來戰,孝恪及子死之。王師擾,倉部郎中崔義起募兵戰城中,繼叔、威助擊之,斬首三千級。那利敗,裒亡散復振,還襲王師,繼叔乘之,斬八千級。那利走,或執以詣軍。社爾凡破五大城,男女數萬,遣使者諭降小城七百餘,西域震懼,西突厥、安兩國歸軍餉焉。社爾立王弟葉護王其國,勒石紀功。



 書聞,帝喜,見群臣從容曰:「夫樂有幾,朕嘗言之:土城竹馬,童兒樂也;飭金翠羅紈,婦人樂也;貿遷有無,商賈樂也;高官厚秩,士大夫樂也;戰無前敵,將帥樂也;四海寧一,帝王樂也。朕今樂矣!」遂遍觴之。初,孝恪之擊焉耆也,龜茲有浮屠善數,嘆曰:「唐家終有西域,不數年吾國亦亡。」社爾執訶黎布失畢、那利、羯獵顛獻太廟,帝受俘紫微殿。帝責謂,君臣皆頓首伏。詔赦罪,改館鴻臚寺,拜布失畢左武衛中朗將。始徙安西都護於其都,統於闐、碎葉、疏勒,號「四鎮。」



 高宗復封訶黎布失畢為龜茲王,與那利、羯獵顛還國。久之,王來朝。那利烝其妻阿史那,王不能禁,左右請殺之,由是更猜忌。使者言狀,帝並召至京師,囚那利,護遣王還。羯獵顛拒不內,遣使降賀魯,王不敢進,悒悒死。詔左屯衛大將軍楊胄發兵禽羯獵顛,窮誅部黨,以其地為龜茲都督府,更立子素稽為王,授右驍衛大將軍,為都督。是歲,徙安西都護府於其國,以故安西為西州都督府,即拜左驍衛大將軍兼安西都護曲智湛為都督。西域平。帝遣使者分行諸國風俗物產,詔許敬宗與史官撰《西域圖志》。



 上元中,素稽獻銀頗羅、名馬。天授三年,王延田跌來朝。始,儀鳳時,吐蕃攻焉耆以西,四鎮皆沒。長壽元年,武威道總管王孝傑破吐蕃,復四鎮地,置安西都護府於龜茲,以兵三萬鎮守。於是沙磧荒絕,民供貲糧苦甚,議者請棄之,武后不聽。都護以政勣稱華狄者,田揚名、郭元振、張孝嵩、杜暹云。開元七年,王白莫苾死,子多幣立,改名孝節。十八年,遣弟孝義來朝。



 自龜茲贏六百里,窬小沙磧,有跋祿迦,小國也,一曰亟墨,即漢姑墨國,橫六百里,縱三百里。風俗文字與龜茲同,言語少異。出細氈褐。西三百里度石磧至凌山,蔥嶺北原也,水東流,春夏山谷積雪。西北五百里至素葉水城,比國商胡雜居。素葉以西數十城,皆立君長,役頟屬突厥。自素葉水城至羯霜那國,衣氈褐皮氎,以繒繚。素葉城西四百里至千泉,地贏二百里,南雪山,三垂平陸,多泉池,因名之,突厥可汗歲避暑其中。群鹿飾鈴鵪,可狎也。西贏百里至呾邏私城,亦比國商胡雜居。有小城,三百,本華人,為突厥所掠,群保此,尚華語。西南贏二百里至白水城,原隰膏腴。南五十里有笯赤建國,廣千里,地沃宜稼,多蒲陶。又二百里即石國。



 疏勒,一曰佉沙,環五千里,距京師九千里而贏。多沙磧,少壤土。俗尚詭詐,生子亦夾頭取褊,其人文身碧瞳。王姓裴氏,自號「阿摩支」,居迦師城,突厥以女妻之。勝兵二千人。俗祠祅神。



 貞觀九年,遣使者獻名馬,又四年,與硃俱波、甘棠貢方物。太宗謂房玄齡等曰:「曩之一天下,克勝四夷,惟秦皇、漢武耳。朕提三尺劍定四海,遠夷率服,不減二君者。然彼末路不自保,公等宜相輔弼,毋進諛言,置朕於危亡也。」儀鳳時,吐蕃破其國。開元十六年,始遣大理正喬夢松攝鴻臚少卿,冊其君安定為疏勒王。天寶十二載,首領裴國良來朝,授折沖都尉,賜紫袍、金魚。



 硃俱波亦名硃俱槃,漢子合國也。並有西夜、蒲犁、依耐、得若四種地,直於闐西千里,蔥嶺北三百里,西距喝盤陀,北九百里屬疏勒,南三千里女國也。勝兵二千人。尚浮屠法,文字同婆羅門。



 甘棠,在海南,昆侖人也。



 喝盤陀,或曰漢陀,曰渴館檀,亦謂渴羅陀,由疏勒西南入劍末穀、不忍領六百里,其國也。距瓜州四千五百里,直硃俱波西,南距懸度山,北抵疏勒,西護密,西北判汗國也。治蔥嶺中,都城負徙多河。勝兵千人。其王本疏勒人,世相承為之。西南即頭痛山也。蔥嶺俗號極嶷山,環其國。人勁悍,貌、言如於闐。其法,殺人剽劫者死,餘得贖。賦必輸服飾,王坐人床。後魏太延中,始通中國。貞觀九年,遣使者來朝。開元中破平其國,置蔥嶺守捉,安西極邊戍也。



 于闐,或曰瞿薩旦那,亦曰渙那,曰屈丹,北狄曰于遁,諸胡曰豁旦。距京師九千七百里,瓜州贏四千里,並有漢戎廬、桿彌、渠勒、皮山五國故地。其居曰西山城,勝兵四千人。有玉河,國人夜視月光盛處必得美玉。王居繪室。俗機巧,言迂大,喜事祅神、浮屠法,然貌恭謹,相見皆跪。以木為筆,玉為印,凡得問遺書,戴於首乃發之。自漢武帝以來,中國詔書符節,其王傳以相授。人喜歌舞,工紡勣。西有沙磧,鼠大如蝟,色類金,出入群鼠為從。初無桑蠶,丐鄰國,不肯出,其王即求婚,許之。將迎,乃告曰:「國無帛,可持蠶自為衣。」女聞,置蠶帽絮中,關守不敢驗,自是始有蠶。女刻石約無殺蠶,蛾飛盡得治繭。



 王姓尉遲氏,名屋密,本臣突厥,貞觀六年,遣使者入獻。後三年,遣子入侍。阿史那社爾之平龜茲也,其王伏闍信大懼,使子獻橐它三百。長史薛萬備謂社爾曰:「公破龜茲,西域皆震恐,願假輕騎羈於闐王獻京師。」社爾許之。至於闐,陳唐威靈,勸入見天子,伏闍信乃隨使者來。會高宗立,授右衛大將軍,子葉護玷為右驍衛將軍,賜袍帶,布帛六千段,第一區,留數月遣之,請以子弟宿衛。上元初,身率子弟酋領七十人來朝。擊吐蕃有功,帝以其地為毘沙都督府,析十州,授伏闍雄都督。死,武后立其子璥。開元時獻馬、駝、豽。璥死,復立尉遲伏師戰為王。死,伏闍達嗣,並冊其妻執失為妃。死,尉遲圭嗣,妻馬為妃。圭死,子勝立。至德初,以兵赴難,因請留宿衛。乾元三年,以其弟左監門衛率葉護曜為大僕員外卿、同四鎮節度副使,權知本國事。勝自有傳。



 于闐東三百里有建德力河,七百里有精絕國;河之東有汗彌,居達德力城,亦曰拘彌城,即寧彌故城。皆小國也。



 初,德宗即位,遣內給事硃如玉之安西,求玉於於闐,得圭一,珂佩五,枕一,帶胯三百,簪四十,奩三十,釧十,杵三,瑟瑟百斤,並它寶等。及還,詐言假道回紇為所奪。久之事洩,得所市,流死恩州。



 天竺國,漢身毒國也,或曰摩伽陀,曰婆羅門。去京師九千六百里,都護治所二千八百里。居蔥嶺南,幅圓三萬里,分東、西、南、北、中五天竺,皆城邑數百。南天竺瀕海,出師子、豹、犬軍、橐它、犀、象、火齊、瑯墯、石蜜、黑鹽。北天竺距雪山,圜抱如璧,南有谷,通為國門。東天竺際海,與扶南、林邑接。西天竺與罽賓、波斯接。中天竺在四天竺之會,都城曰茶鎛和羅城,濱迦毘黎河。有別城數百,皆置長;別國數十,置王。曰舍衛;曰迦沒路,開戶皆東向;曰迦尸,或曰波羅奈,亦曰波羅那斯。其畜有稍割牛,黑色,角細,長四尺許,十日一割,不然困且死。人飲其血,或曰壽五百歲,牛壽如之。



 中天竺王姓乞利咥氏,亦曰剎利,世有其國,不篡殺。土溽熱,稻歲四熟。禾之長者沒橐它。以貝齒為貨。有金剛、旃檀、鬱金,與大秦、扶南、交趾相貿易。人富樂,無簿籍,耕王地者乃輸稅。以舐足摩踵為致禮。家有奇樂倡伎。王大臣皆服錦罽,為螺髻於頂,餘發翦使卷。男子穿耳垂當,或懸金,耳緩者為上類;徒跣,衣重白。婦人項節金、銀、珠纓絡,死者燔骸取灰,建窣堵,或委野中及河,餌鳥獸魚鱉,無喪紀。謀反者幽殺之;小罪贖錢;不孝者斷手足,劓耳鼻,徙於邊。有文字,善步歷,學《悉曇章》,妄曰梵天法。書貝多葉以記事。尚浮圖法,不殺生飲酒,國中處處指曰佛故跡也。信盟誓,傳禁咒,能致龍起雲雨。



 隋煬帝時,遣裴矩通西域諸國,獨天竺、拂菻不至為恨。武德中,國大亂,王尸羅逸多勒兵戰無前,象不弛鞍,士不釋甲,因討四天竺,皆北面臣之。會唐浮屠玄奘至其國,尸羅逸多召見曰:「而國有聖人出,作《秦王破陣樂》,試為我言其為人。」玄奘粗言太宗神武,平禍亂,四夷賓服狀。王喜,曰:「我當東面朝之。」貞觀十五年,自稱摩伽陀王,遣使者上書。帝命雲騎尉梁懷璥持節尉撫,尸羅逸多驚問國人:「自古亦有摩訶震旦使者至吾國乎?」皆曰:「無有。」戎言中國為摩訶震旦。乃出迎,膜拜受詔書,戴之頂,復遣使者隨入朝。詔衛尉丞李義表報之,大臣郊迎,傾都邑縱觀,道上焚香,尸羅逸多率群臣東面受詔書,復獻火珠、鬱金、菩提樹。



 二十二年,遣右衛率府長史王玄策使其國,以蔣師仁為副;未至,尸羅逸多死,國人亂,其臣那伏帝阿羅那順自立,發兵拒玄策。時從騎才數十,戰不勝,皆沒,遂剽諸國貢物。玄策挺身奔吐蕃西鄙,檄召鄰國兵。吐蕃以兵千人來,泥婆羅以七千騎來,玄策部分進戰茶鎛和羅城,三日破之,斬首三千級,溺水死萬人。阿羅那順委國走,合散兵復陣,師仁禽之,俘斬千計。餘眾奉王妻息阻乾陀衛江,師仁擊之,大潰,獲其妃、王子,虜男女萬二千人,雜畜三萬,降城邑五百八十所。東天竺王尸鳩摩送牛馬三萬饋軍,及弓、刀、寶纓絡。迦沒路國獻異物,並上地圖,請老子象。玄策執阿羅那順獻闕下。有司告宗廟,帝曰:「夫人耳目玩聲色,口鼻耽臭味,此敗德之原也。婆羅門不劫吾使者,寧至俘虜邪?」擢玄策朝散大夫。



 得方士那邏邇娑婆寐,自言壽二百歲,有不死術,帝改館使治丹,命兵部尚書崔敦禮護視。使者馳天下,採怪藥異石,又使者走婆羅門諸國。所謂畔茶法水者,出石臼中,有石象人守之,水有七種色,或熱或冷,能銷草木金鐵,人手入輒爛,以橐它髑髏轉注瓠中。有樹名咀賴羅,葉如梨,生窮山崖腹,前有巨虺守穴,不可到,欲取葉者,以方鏃矢射枝則落,為群鳥銜去,則又射,乃得之。其詭譎類如此。後術不驗,有詔聽還,不能去,死長安。高宗時,廬伽逸多者,東天竺烏茶人,亦以術進,拜懷化大將軍。



 乾封三年,五天竺皆來朝。開元時,中天竺遣使者三至;南天竺一,獻五色能言鳥,乞師討大食、吐蕃,丐名其軍。玄宗詔賜懷德軍。使者曰:「蕃夷惟以袍帶為寵。」帝以錦袍、金革帶、魚袋並七事賜之;北天竺一來朝。



 摩揭它,一曰摩伽陀,本中天竺屬國。環五千里,土沃宜稼穡,有異稻巨粒,號供大人米。王居拘闍揭羅布羅城,或曰俱蘇摩補羅,曰波吒厘子城,北瀕殑伽河。貞觀二十一年,始遣使者自通於天子,獻波羅樹,樹類白楊。太宗遣使取熬糖法,即詔揚州上諸蔗,拃沈如其劑,色味愈西域遠甚。高宗又遣王玄策至其國摩訶菩提祠立碑焉。後德宗自制鐘銘,賜那爛陀祠。



 又有那揭者,亦屬國也,貞觀二十年,遣使者貢方物。



 烏茶者,一曰烏伏那,亦曰烏萇,直天竺南,地廣五千里,東距勃律六百里,西罽賓四百里。山谷相屬,產金、鐵、蒲陶、鬱金。稻歲熟。人柔詐,善禁架術。國無殺刑,抵死者放之窮山。罪有疑,飲以藥,視溲清濁而決輕重。有五城,王居術瞢蘗利城,一曰瞢揭厘城,東北有達麗羅川,即烏萇舊地。貞觀十六年,其王達摩因陀訶斯遣使者獻龍腦香,璽書優答。大食與烏萇東鄙接,開元中數誘之,其王與骨咄、俱位二王不肯臣,玄宗命使者冊為王。



 章求拔國,或曰章揭拔,本西羌種。居悉立西南四山中,後徙山西,與東天竺接。衣服略相類,因附之。地袤八九百里,勝兵二千人。無城郭,好鈔暴,商旅患之。貞觀二十年,其王羅利多菩伽因悉立國遣使者入朝。玄策之討中天竺,發兵來赴,有功,由是職貢不絕。



 悉立當吐蕃西南,戶五萬,城邑多旁澗溪。男子繒束頭,衣氈褐。婦人辮發,短裙。昏姻不以財聘。其穀宜粳稻、麥、豆。死者葬於野,不封樹,喪制為黑衣,滿年而除。刑有刖、劓。常羈屬吐蕃。



 罽賓,隋漕國也,居蔥嶺南,距京師萬二千里而贏,南距舍衛三千里。王居脩鮮城,常役屬大月氏。地暑濕,人乘象,俗治浮屠法。



 武德二年,遣使貢寶帶、金鎖、水精盞、頗黎狀若酸棗。貞觀中獻名馬。太宗詔大臣曰:「朕始即位,或言天子欲耀兵,振伏四夷,惟魏徵勸我脩文德,安中夏;中夏安,遠人伏矣。今天下大安,四夷君長皆來獻,此徵力也。」遣果毅何處羅拔等厚齎賜其國,並撫尉天竺。處羅拔至罽賓,王東向稽首再拜,仍遣人導護使者至天竺。十六年,獻褥特鼠,喙尖尾赤,能食蛇,螫者嗅且尿,瘡即愈。



 國人共傳王始祖曰馨孽,至曷擷支傳十二世。顯慶三年,以其地為脩鮮都督府。龍朔初,拜其王脩鮮等十一州諸軍事、脩鮮都督。開元七年,遣使獻天文及秘方奇藥,天子冊其王為葛邏達支特勒。後烏散特勒灑年老,請以子拂菻罽婆嗣,聽之。天寶四載,冊其子勃匐準為襲罽賓及烏萇國王。乾元初使者朝貢。



\end{pinyinscope}