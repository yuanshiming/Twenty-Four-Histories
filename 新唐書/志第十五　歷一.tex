\article{志第十五 歷一}

\begin{pinyinscope}

 歷法尚矣。自堯命羲、和,歷象日月星辰,以閏月定四時成歲,其事略見於《書》。而夏、商、周以三統改正朔,為歷固已不同屬浙江)人。出身「細門孤族」,少年時受業太學,師事班彪。,而其法不傳。至漢造歷,始以八十一分為統母,其數起於黃鐘之龠,蓋其法一本於律矣。其後劉歆又以《春秋》、《易象》推合其數,蓋傅會之說也。至唐一行始專用大衍之策,則歷術又本於《易》矣。蓋歷起於數,數者,自然之用也。其用無窮而無所不通,以之於律、於《易》,皆可以合也。然其要在於候天地之氣,以知四時寒暑,而仰察天日月星之行運,以相參合而已。然四時寒暑無形而運於下,天日月星有象而見於上,二者常動而不息。一有一無,出入升降,或遲或疾,不相為謀。其久而不能無差忒者,勢使之然也。故為歷者,其始未嘗不精密,而其後多疏而不合,亦理之然也。不合,則屢變其法以求之。自堯、舜、三代以來,歷未嘗同也。



 唐終始二百九十餘年,而歷八改。初曰《戊寅元歷》,曰《麟德甲子元歷》,曰《開元大衍歷》,曰《寶應五紀歷》,曰《建中正元歷》,曰《元和觀象歷》,曰《長慶宣明歷》,曰《景福崇玄歷》而止矣。



 高祖受禪,將治新歷,東都道士傅仁均善推步之學,太史令庾儉、丞傅弈薦之。詔仁均與儉等參議,合受命歲名為《戊寅元歷》。乃列其大要,所可考驗者有七,曰:「唐以戊寅歲甲子日登極,歷元戊寅,日起甲子,如漢《太初》,一也。冬至五十餘年輒差一度,日短星昴,合於《堯典》,二也。周幽王六年十月辛卯朔,入蝕限,合於《詩》,三也。魯僖公五年壬子冬至,合《春秋命歷序》,四也。月有三大、三小,則日蝕常在朔,月蝕常在望,五也。命辰起子半,命度起虛六,符陰陽之始,六也。立遲疾定朔,則月行晦不東見,朔不西朓,七也。」高祖詔司歷起二年用之,擢仁均員外散騎侍郎。



 三年正月望及二月、八月朔,當蝕,比不效。六年,詔吏部郎中祖孝孫考其得失。孝孫使算歷博士王孝通以《甲辰歷》法詰之曰:「『日短星昴,以正仲冬。』七宿畢見,舉中宿言耳。舉中宿,則餘星可知。仁均專守昴中,執文害意,不亦謬乎?又《月令》仲冬『昏東壁中』,明昴中非為常準。若堯時星昴昏中,差至東壁,然則堯前七千餘歲,冬至昏翼中,日應在東井。井極北,去人最近,故暑;斗極南,去人最遠,故寒。寒暑易位,必不然矣。又平朔、定朔,舊有二家。三大、三小,為定朔望;一大、一小,為平朔望。日月行有遲速,相及謂之合會。晦、朔無定,由時消息。若定大小皆在朔者,合會雖定,而蔀、元、紀首三端並失。若上合履端之始,下得歸餘於終,合會有時,則《甲辰元歷》為通術矣。」仁均對曰:「宋祖沖之立歲差,隋張胄玄等因而修之。雖差數不同,各明其意。孝通未曉,乃熱南斗為冬至常星。夫日躔宿度,如垂阜傳之過,宿度既差,黃道隨而變矣。《書》云:『季秋月朔,辰弗集於房。』孔氏云:『集,合也。不合則日蝕可知。』又云:『先時者殺無赦,不及時者殺無赦。』既有先後之差,是知定朔矣。《詩》云:『十月之交,朔月辛卯。』又《春秋傳》曰:『不書朔,官失之也。』自後歷差,莫能詳正。故秦、漢以來,多非朔蝕。宋御史中丞何承天微欲見意,不能詳究,乃為散騎侍郎皮延宗等所抑。孝通之語,乃延宗舊說。治歷之本,必推上元,日月如合璧,五星如連珠,夜半甲子朔旦冬至。自此七曜散行,不復餘分普盡,總會如初。唯朔分、氣分,有可盡之理,因其可盡,即有三端。此乃紀其日數之元爾。或以為即夜半甲子朔冬至者,非也。冬至自有常數,朔名由於月起,月行遲疾匪常,三端安得即合。故必須日月相合與至同日者,乃為合朔冬至耳。」孝孫以為然,但略去尤疏闊者。



 九年,復詔大理卿崔善為與孝通等較定,善為所改凡數十條。初,仁均以武德元年為歷始,而氣、朔、遲疾、交會及五星皆有加減。至是復用上元積算。其周天度,即古赤道也。



 貞觀初,直太史李淳風又上疏論十有八事,復詔善為課二家得失,其七條改從淳風。十四年,太宗將親祀南郊,以十一月癸亥朔,甲子冬至。而淳風新術,以甲子合朔冬至,乃上言:「古歷分日,起於子半。十一月當甲子合朔冬至,故太史令傅仁均以減餘稍多,子初為朔,遂差三刻。」司歷南宮子明、太史令薛頤等言:「子初及半,日月未離。淳風之法,較春秋已來晷度薄蝕,事皆符合。」國子祭酒孔穎達等及尚書八座參議,請從淳風。又以平朔推之,則二歷皆以朔日冬至,於事彌合。且平朔行之自古,故《春秋傳》或失之前,謂晦日也。雖癸亥日月相及,明日甲子,為朔可也。從之。十八年,淳風又上言:「仁均歷有三大、三小,雲日月之蝕,必在朔望。十九年九月後,四朔頻大。」詔集諸解歷者詳之,不能定。庚子,詔用仁均平朔,訖麟德元年。



 仁均歷法祖述胄玄,稍以劉孝孫舊議參之,其大最疏於淳風。然更相出入,其有所中,淳風亦不能逾之。今所記者,善為所較也。



 《戊寅歷》上元戊寅歲至武德九年丙戌,積十六萬四千三百四十八算外。



 章歲六百七十六。亦名行分法。章閏二百四十九。章月八千三百六十一。



 月法三十八萬四千七十五。日法萬三千六。時法六千五百三度法、氣法九千四百六十四氣時法千一百八十三。



 歲分三百四十五萬六千六百七十五。歲餘二千三百一十五。周分三百四十五萬六千八百四十五半。斗分一千四百八十五半。沒分七萬六千八百一十五。沒法千一百三。



 歷日二十七,歷餘萬六千六十四。歷周七十九萬八千二百。歷法二萬八千九百六十八。餘數四萬九千六百三十五。



 章月乘年,如章歲得一,為積月。以月法乘積月,如日法得一,為朔積日;餘為小餘。



 日滿六十,去之;餘為大餘。命甲子算外,得天正平朔。加大餘二十九、小餘六千九百一,得次朔。加平朔大餘七、小餘四千九百七十六、小分四之三,為上弦。又加,得望。又加,得下弦。餘數乘年,如氣法得一,為氣積日。命日如前,得冬至。加大餘十五、小餘二千六十八、小分八之一,得次氣日。加四季之節大餘十二、小餘千六百五十四、小分四,得土王。凡節氣小餘,三之,以氣時法而一,命子半算外,各其加時。置冬至小餘,八之,減沒分,餘滿沒法為日。加冬至去朔日算,依月大小去之,日不滿月算,得沒日。餘分盡為減。加日六十九、餘七百八,得次沒。



 以平朔、弦、望入氣日算乘損益率,如十五得一,以損益盈縮數,為定盈縮分。凡不盡半法已上亦從一。以歷法乘朔積日,滿歷周去之;餘如歷法得一,為日。命日算外,得天正平朔夜半入歷日及餘。次日加一,累而裁之。若以萬四千四百八十四乘平朔小餘,如六千五百三而一,不盡,為小分,以加夜半入歷日。加之滿歷日及餘,去之,得平朔加時所入,加歷日七、餘萬一千八十四、小分三千九百九十五,命如前,得上弦。又加,得望、下弦及後朔。



 歷行分與次日相減,為行差,後多為進,後少為退。減去行分六百七十六,為差法。各置平朔、弦、望加時入歷日餘,乘所入日損益率,以損益其下積分,差法除,為定盈縮積分。置平朔、弦、望小餘,各以入氣積分盈加、縮減之,以入歷積分盈減、縮加之,滿若不足、進退日法,皆為定大小餘,命日甲子算外。以歲分乘年為積分,滿周分去之;餘如度法得一,為度。命以虛六,經斗去分,得冬至日度及分。以冬至去朔日算及分減之,得天正平朔前夜半日度及分。以小分法十四約度分為行分。凡小分滿法成行分,行分滿法成度。若注歷,又以二十六約行分。月星準此。斗分百七十七,小分七半。累加一度,得次日。以行分法乘朔、望定小餘,以九百二十九除為度分,又以十四約為行分。以加夜半度,為朔、望加時日度。定朔加時,日月同度。望則因加日度百八十二、行分四百二十六、小分十太。以夜半入歷日餘乘行差,滿歷法得一,以進加、退減歷行分,為行定分。以朔定小餘乘之,滿日法得一,為行分。以減加時月度,為朔、望夜半月度。求次日,加月行定分,累之。



 ○歲星



 率三百七十七萬五千二十三。



 終日三百九十八,行分五百九十六,小分七。



 平見,入冬至初日,減行分五千四百一十一。自後日損所減百二十分。立春初,日增所加六十分。春分,均加四日。清明畢穀雨,均加五日。立夏畢大暑,均加六日。立秋初日,加四千八十分。乃日損所加六十七分。入寒露,日增所減百一十七分。入小雪,畢大雪,均減八日。



 初見,順,日行百七十一分,日益遲一分,百一十四日行十九度二百九分。而留,二十六日。乃退,日九十七分,八十四日退十二度三十六分。又留,二十五日五百九十六分,小分七。凡五星留日有分者,以初定見日分加之。若滿行分法,去之,又增一日。乃順,初日行六十分,日益疾一分,百一十四日行十九度四百三十七分。而伏。



 ○熒惑



 率七百三十八萬一千二百二十三。



 終日七百七十九,行分六百二十六,小分三。



 平見,入冬至初日,減萬六千三百五十四分。乃日損所減五百四十五分。入大寒,日增所加四百二十六分。入雨水後,均加二十九日。立夏初日,加萬九千三百九十二分。乃日損所加二百一十三分。入立秋初,依平。入處暑,日增所減百八十四分。入小雪後,均減二十五日。



 初見,入冬至,初率二百四十一日行百六十三度。自後二日損日度各一,自百二十八日,率百七十七日行九十九度,畢百六十一日。又三日損一,盡百八十二日,率百七十日行九十二度,畢百八十八日。乃三日益一,盡二百二十七日,率百八十三日行百五度。又二日益一,盡二百四十九日,率百九十四日行百一十六度。又每日益一,盡二百一十日,率二百五十五日行百七十七度,畢三百三十七日。乃二日損一,盡大雪,復初見。入小寒後,三日去日率一。入雨水,畢立夏,均去日率二十。自後三日減所去一日,畢小暑,依平,為定日率。若入處暑,畢秋分,皆去度率六。各依冬至後日數而損益之,又依所入之氣以減之,為前疾日度率。若初行入大寒,畢大暑,皆差行,日益遲一分;其餘皆平行。若入白露,畢秋分,初遲,日行半度,四十日行二十度。即去日率四十、度率二十,別為半度之。行訖,然後求平行分,續之。以行分法乘度定率,如日定率而一,為平行分。不盡,為小分。求差行者,減日率一,又半之,加平行分,為初日行分。各盡其日度而遲。初日行三百二十六分,日益遲一分半,六十日行二十五度五分。其前疾去度六者,行三十一度五分。此遲初日加六十七分、小分六十分之三十六。



 而留,十三日。前疾去日者,分日於二留,奇從後留。乃退,日百九十二分,六十日退十七度二十八分。又留,十二日六百二十六分,小分三。



 又順。後遲,初日行二百三十八分,日益疾一分半,六十日行二十五度三十五分。此遲在立秋至秋分者,加六度,行三十一度三十五分。此遲初日加行分六十七、小分六十分之三十六。而後疾。入冬至,初率二百一十四日行百三十六度。乃每日損一,盡三十七日,率百七十七日行九十九度。又二日損一,盡五十七日,率百六十七日行八十九度,畢七十九日。又三日益一,盡百三十日,率百八十四日行百六度。又二日益一,盡百四十四日,率百九十一日行百一十三度。又每日益一,盡百九十日,率二百三十七日行百五十九度。又每日益二,盡二百日,率二百五十七日行百七十九度。又每日益一,盡二百一十日,率二百六十七日行百八十九度,畢二百五十九日。乃二日損一,畢大雪,復初。後遲加六度者,此後疾去度率六,為定。各依冬至後日數而損益之,為後疾日度率。若入立夏,畢夏至,日行半度,盡六十日,行三十度。若入小暑,畢大暑,盡四十日,行二十度皆去日度率,別為半度之。行訖,然後求平行分,續之。各盡其日度而伏。



 ○鎮星



 率三百五十七萬八千二百四十六。



 終日三百七十八,行分六十一。



 平見,入冬至初日,減四千八百一十四分。乃日增所減七十九分。入小寒,均減九日。乃每氣損所減一日。入夏至初日,均減二日。自後十日損所減一日。小暑五日外,依平。入大暑,日增所加百八十一分。入處暑,均加九日。入白露初日,加六千二分。乃日損所加百三十三分。入霜降,日增所減七十九分。



 初見,順,日行六十分,八十三日行七度二百四十八分。而留,三十八日。乃退,日四十一分,百日退六度四十四分。又留,三十七日六十一分。乃順,日行六十分,八十三日行七度二百四十八分而伏。



 ○太白



 率五百五十二萬六千二百。



 終日五百八十三,行分六百二十,小分八。



 晨見伏三百二十七日,行分六百二十,小分八。



 夕見伏二百五十六日。



 晨平見,入冬至,依平。入小寒,日增所加六十六分。入立春,畢立夏,均加三日。小滿初日,加千九百六十四分。乃日損所加六十分。入夏至,依平。入小暑,日增所減六十分。入立秋,畢立冬,均減三日。小雪初日,減千九百六十四分。乃日損所減六十六分。



 初見,乃退,日半度,十日退五度。而留,九日。乃順,遲,差行,日益疾八分,四十日行三十度。入大雪畢小滿者,依此。入芒種,十日減一度。入小暑,畢霜降,均減三度。入立冬,十日損所減一度,畢小雪。皆為定度。以行分法乘定度,四十除,為平行分。又以四乘三十九,以減平行,為初日行分。平行,日一度,十五日行十五度。入小寒,十日益日度各一。入雨水後,皆二十一日行二十一度。入春分後,十日減一。畢立夏,依平。入小滿後,六日減一。畢立秋,日度皆盡,無平行。入霜降後,四日加一。畢大雪,依平。疾,百七十日行二百四度。前順遲減度者,計所減之數,以益此度為定。而晨伏。



 夕平見,入冬至,日增所減百分。入啟蟄,畢春分,均減九日。清明初日,減五千九百八十六分。乃日損所減百分。入芒種,依平。入夏至,日增所加百分。入處暑,畢秋分,均加九日。寒露初日,加五千九百八十六分。乃日損所減百分。入大雪,依平。



 初見,順疾,百七十日行二百四度。入冬至畢立夏者,依此。入小滿,六日加一度。入夏至,畢小暑,均加五度。入大暑,三日減一度。入立秋,畢大雪,依平。從白露畢春分,皆差行,日益疾一分半。以一分半乘百六十九而半之,以加平行,為初日行分。入清明,畢於處暑,畢平行。乃平行,日一度,十五日行十五度。入冬至後,十日減日度各一。入啟蟄,畢芒種,皆九日行九度。入夏至後,五日益一。入大暑,依平。入立秋後,六日加一。畢秋分,二十五日行二十五度。入寒露,六日減一。入大雪,依平。順遲,日益遲八分,四十日行三十度。前加度者,此依數減之。又留,九日。乃退,日半度,十日退五度。而夕伏。



 ○辰星



 率百九萬六千六百八十三



 終日百一十五,行分五百九十四,小分七。



 晨見伏六十三日,行分五百九十四,小分七。



 夕見伏五十二日。



 晨平見,入冬至,均減四日。入小寒,依平。入立春後,均減三日。入雨水,畢立夏,應見不見。其在啟蟄、立夏氣內,去日十八度外、三十六度內,晨有木、火、土、金一星者,亦見。入小滿,依平。入霜降,畢立冬,均加一日。入小雪,至大雪十二日,依平。若在大雪十三日後,日增所減一日。



 初見,留,六日。順遲,日行百六十九分。入大寒,畢啟蟄,無此遲行。乃平行,日一度,十日行十度。入大寒後,二日去日度各一,畢於二十日,日度俱盡,無此平行。疾,日行一度六百九分,十日行十九度六分。前無遲行者,此疾日減二百三分,十日行十六度四分。而晨伏。



 夕平見,入冬至後,依平。入穀雨,畢芒種,均減二日。入夏至,依平。入立秋,畢霜降,應見不見。其在立秋、霜降氣內,夕有星去日如前者,亦見。入立冬,畢大雪,依平。



 初見,順疾,日行一度六百九分,十日行十九度六分。若入小暑,畢處暑,日減二百三分。乃平行,日一度,十日行十度。入大暑後,二日去日及度各一,畢於二十日,日度俱盡,無此平行。遲,日行百六十九分。若疾減二百三分者,即不須此遲行。又留,六日七分。而夕伏。



 各以星率去歲積分,餘反以減其率,餘如度法得一為日,得冬至後晨平見日及分。以冬至去朔日算及分加之,起天正,依月大小計之,命日算外,得所在日月。金、水各以晨見伏日及分加之,得夕平見。各以其星初日所加減之分,計後日損益之數以損益之。訖,乃以加減平見為定見。其加減分皆滿行分法為日。以定見去朔日及分加其朔前夜半日度,又以星初見去日度,歲星十四,太白十一,熒惑、鎮星、辰星皆十七,晨減、夕加之,得初見宿度。求次日,各加一日所行度及分。熒惑、太白有小分者,各以日率為母。其行有益疾遲者,副置一日行分,各以其差疾益、遲損,乃加之。留者因前,退則依減,伏不注度。順行出鬥,去其分;退行入斗,先加分。訖,皆以二十六約行分,為度分。



 交會法千二百七十四萬一千二百五八分。交分法六百三十七萬六百二九分。



 朔差百八萬五千四百九十四二分。望分六百九十一萬三千三百五十。交限五百八十二萬七千八百五十五八分。望差五十四萬二千七百四十七一分。



 外限六百七十六萬七百八十二九分。中限千二百三十五萬一千二十五八分。內限千二百一十九萬一千四百五十八七分。



 以朔差乘積月,滿交會法去之;餘得天正月朔入平交分。求望,以望分加之。求次月,以朔差加之。其朔望,入大雪,畢冬至,依平。入小寒,日加氣差千六百五十分。入啟蟄,畢清明,均加七萬六千一百分。自後日損所加千六百五十分。入芒種,畢夏至,依平。加之滿法,去之。若朔交入小寒畢雨水,及立夏畢小滿,值盈二時已下,皆半氣差加之。二時已上則否。如望差已下、外限已上有星伏,木、土去見十日外,火去見四十日外,金晨伏去見二十二日外,有一星者,不加氣差。入小暑後,日增所減千二百分。入白露,畢霜降,均減九萬五千八百二十五分。立冬初日,減六萬三千三百分,自後日損所減二千一百一十分。減若不足,加法,乃減之,餘為定交分。朔入交分,如交限內限已上、交分中限已下有星伏如前者,不減。不滿交分法者,為在外道;滿去之,餘為在內道。如望差已下,為去先交分,交限已上,以減交分,餘為去後交分。皆三日法約,為時數。望則月蝕,朔在內道則日蝕。雖在外道,去交近,亦蝕。在內道,去交遠,亦不蝕。



 置蝕望定小餘。入歷一日,減二百八十;若十五日,即加之;十四日,加五百五十;若二十八日,即減之;餘日皆盈加、縮減二百八十:為月蝕定餘。十二乘之,時法而一,命子半算外;不盡,得月蝕加時。約定小餘如夜漏半已下者,退日算上。



 置蝕朔定小餘。入歷一日,即減二百八十;若十五日,即加之;十四日,加五百五十;若二十八日,即減之;為定。後不入四時加減之限。其內道,春,去交四時已上入歷,盈加、縮減二百八十;夏,盈加、縮減二百八十;秋,去交十一時已下,惟盈加二百八十,已上者,盈加五百五十,縮加二百八十;冬,去交五時已下,惟盈加二百八十:皆為定餘。十二乘之,時法而一,命子半算外;不盡,為時餘,副之。仲辰半前,以副減法為差率;半後,退半辰,以法加餘,以副為差率。季辰半前,以法加副為差率;半後,退半辰,以法加餘,倍法加副,為差率。孟辰半前,三因其法,以副減之,餘為差率;半後,退半辰,以法加餘,又以法加副,乃三因其法,以副減之,為差率。又置去交時數,三已下,加三;六已下,加二;九已下,加一;九已上,依數;十二已上,從十二。若季辰半後,孟辰半前,去交六時已上者,皆從其六。六時已下,依數不加。皆乘差率,十四除,為時差。子午半後,以加時餘;卯酉半後,以減時餘;加之滿若不足,進退時法:孟謂寅、巳、申,仲謂午、卯、酉,季謂辰、未、戌。得日蝕加時。



 望去交分,冬先後交,皆去二時;春先交,秋後交,去半時;春後交,秋先交,去二時;夏則依定。不足去者,既。乃以三萬六千一百八十三為法而一,以減十五,餘為月蝕分。



 朔去交,在內道,五月朔,加時在南方,先交十三時外;六月朔,後交十三時外者,不蝕。啟蟄畢清明,先交十三時外,值縮,加時在未西;處暑畢寒露,後交十三時外,值盈,加時在巳東,皆不蝕。交在外道,先後去交一時內者,皆蝕。若二時內,及先交值盈、後交值縮二時外者,亦蝕。夏去交二時內,加時在南方者,亦蝕。若去分、至十二時內,去交六時內者,亦蝕。若去春分三日內,後交二時;秋分三日內,先交二時內者,亦蝕。諸去交三時內有星伏,土、木去見十日外,火去見四十日外,金晨伏去見二十二日外,有一星者,不蝕。各置去交分。秋分後,畢立春,均減二十二萬八百分。啟蟄初日,畢芒種,日損所減千八百一十分。夏至後,畢白露,日增所減二千四百分。以減去交分,餘為不蝕分。不足減,反相減為不蝕分。亦以減望差為定法。後交值縮者,直以望差為定法。其不蝕分,大寒畢立春,後交五時外,皆去一時。時差值減者,先交減之,後交加之。時差值加者,先交加之,後交減之。不足減者,皆既。十五乘之,定法而一,以減十五,餘為日蝕分。



 置日月蝕分,四已下,因增二;五已下,因增三;六已上,因增五;各為刻率,副之。以乘所入歷損益率,四千五十七為法而一。值盈,反其損益;值縮,依其損益。皆損益其副,為定用刻。乃六乘之,十而一,以減蝕甚辰刻,為虧初。又四乘之,十而一。以加食甚辰刻,為復滿。



\end{pinyinscope}