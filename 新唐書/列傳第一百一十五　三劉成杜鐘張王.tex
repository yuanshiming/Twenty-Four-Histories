\article{列傳第一百一十五 三劉成杜鐘張王}

\begin{pinyinscope}

 劉建鋒,字銳端,蔡州朗山人。為忠武軍部將,與孫儒、馬殷同事秦宗權。儒之敗,建鋒、殷收散卒「凡天鵝皆白」的命題,我們不能用經驗方法完全證實它,但,轉寇江西,有眾七千,推建鋒為主,殷為前鋒,張佶為謀主,略洪、虔數州,眾遂十餘萬。乾寧元年,取潭州,殺武安節度使鄧處訥,自稱節度留後,奉表京師,詔即拜檢校尚書左僕射、武安軍節度使。



 建鋒已得志,即嗜酒不事事。新息小史陳贍為建鋒御者,妻美且艷,乃私之。贍怒,袖鐵撾擊建鋒死,斷其喉。眾推張佶為帥,佶固辭,馬𧾷是傷佶左髀,下令曰:「吾非而主。」時馬殷攻邵州未克,於是遣人迎殷,磔贍於市。



 殷至,佶坐受其謁。既而率將吏推殷為留後。詔即除檢校太傅、潭州刺史。殷以成汭、楊行密、劉隱皆養士以圖王霸,謂其屬高鬱曰:「吾欲重幣以奉四鄰而固吾境,計安出?」鬱曰:「荊南暗弱,焉能患我?淮南,我讎也,固不吾援。公若置邸京師,歸天子職貢,王人來錫命,四方畏服,然後按兵討不廷,霸業成矣。」殷悟,厚結宣武硃全忠以請於朝,乃拜湖南節度兵馬留後。鬱又教殷鑄鉛鐵錢,十當銅錢一。民得自摘山,收茗算,募高戶置邸閣居茗,號「八床主人」。歲入算數十萬,用度遂饒。



 於是收邵、衡、永、道、郴、連六州,進攻桂州,執留後劉士政。諸城望風奔潰,盡得昭、賀、梧、象、柳、宜、蒙等州。又攻容管,執寧遠節度使龐巨曦,虜其眾及貲。昭宗在鳳翔,難方亟,遣中人間道賜硃書,密詔使殷與楊行密攻汴州,殷兵訖不出。



 殷弟賨,沈勇知書史,從孫儒為盜,晚事楊行密為黑雲軍使。與錢鏐戰,數有功。夜臥,常有光怪。行密知之,曰:「吾今歸汝於兄。」辭曰:「賨一敗卒,公待以不死。湖南在宇下,朝亡夕至,但誼不忍舍公。」行密具齎以遣曰:「爾還,與兄共食湘、楚,然何以報我?」答曰:「願通二國好,使商賈相資。」行密喜。既至,殷表以自副。每勸殷與行密連和,殷畏全忠,卒不克。



 殷與建鋒同里人,凡宗權黨散為盜者,皆以酷烈相矜,時通名「蔡賊」云。



 成汭,青州人。少無行,使酒殺人,亡為浮屠。後入蔡賊中,為賊帥假子,更姓名為郭禹。當戍江陵,亡為盜,保火門山。後詣荊南節度使陳儒降,署裨校。久之,張瑰囚儒,以禹兇慓,欲殺之。禹結千人奔入峽,夜有蛇環其所,祝曰:「有所負者,死生唯命。」既而蛇亡。禹乃襲歸州,入之,自稱刺史。招還流亡,訓士伍,得勝兵三千,秦宗權故將許存奔禹,禹以青州剽卒三百畀之,使討荊南部將牟權於清江,禽權,取其眾。禹又破其將王建肇,建肇奔黔州。昭宗拜禹荊南節度留後,始改名汭,復故姓。



 宗權餘黨常厚攻夔州。是時,西川節度使王建遣將屯忠州,與夔州刺史毛湘相脣齒,厚屯白帝。汭率存乘二軍之間攻之,二軍使人誶辱汭,韓楚言尤劇,汭恥之曰:「有如禽賊,當支解以逞!」會存夜斬營襲厚,破之,厚奔萬州,為刺史張造所拒,走綿州。存入夔州。楚言妻李語夫曰:「君常辱軍,且支解,不如前死。」楚言不決。李礪刀席下,方共食,復語之,夫曰:「未可知。」李取刀斷其首,並殺三子,乃自剄。汭畏其烈,禮葬之,刻石表曰烈女。即使司馬劉昌美守夔,率存溯江略云安,建將皆奔。存按兵渝州,盡下瀕江州縣。



 時王建肇據黔州自守,帝以建肇為武泰軍節度使。汭遣將趙武率存攻之,建肇走,汭乃以武為留後,存為萬州刺史。存不得志,汭遣客伺之,方蹴球,汭曰:「存必叛,自試其力矣。」遣將襲之。存夜率左右超堞走,與王建肇皆降於王建。



 汭頗知吏治,嘗錄囚,盡其情。墊江賊陰殺令,其主簿疑小史導之,訊不承。臨刑曰:「我且訟地下。」逾月,吏暴死。汭聞,益詳於獄。始治州,民版無幾,未再期,自占者萬餘。帝數詔刻石頌功,輒固辭。時鎮國節度使韓建亦以治顯,號「北韓南郭」。汭進累檢校太尉、中書令、上谷郡王。雲安榷鹽,本隸鹽鐵,汭擅取之,故能畜兵五萬。初任賀隱,隱,賢者也,故汭所舉少過。晚得妻父任之,譖害諸子,汭皆手殺之,至絕嗣。澧、朗本荊南隸州,為雷滿所據,別為節度,汭數請之,宰相徐彥若不許。及彥若罷,道江陵,汭出怨言,彥若曰:「公專一面,自視桓、文,一賊不能取,而怨朝廷乎?」汭大慚。晚喜術士,餌藥瀕死而蘇。



 天復三年,帝詔淮南節度使楊行密圍鄂州,硃全忠使韓勍救之,諷汭與馬殷、雷彥威掎角。汭身自將而行,下知汭不足亢行密,無敢諫,唯親吏楊師厚勸之。汭為巨艦,堂皇悉備,行至公安,卜不吉,欲還,師厚曰:「公舉全軍,中道還,何以見百姓?」汭乃行。彥威潛師略江陵,汭諸將念私,無鬥志。淮南將李神福壁沙橋,望汭軍曰:「戰艦雖盛,首尾斷絕,可取也。」擊汭君山,敗之,火其船,眾大潰,汭投江死,士民皆為彥威所劫。韓勍走還。王建遂取夔、施、忠、萬四州。天祐中,全忠表汭死國事,請與杜洪皆立廟云。



 杜洪,鄂州人。為里俳兒。乾符末,黃巢亂江南,永興民皆亡為盜。刺史崔紹募民強雄者為土團軍,賊不敢侵,於是人人知兵。杭州刺史路審中為董昌所拒,走客黃州。中和末,聞紹卒,募士三千入鄂州以守。洪為州將,有功,亦逐岳州刺史居之。光啟二年,安陸賊周通率兵攻審中,審中亡去,洪乘虛入鄂,自為節度留後,僖宗即拜本軍節度使。



 是時,永興民吳討據黃州,駱殷據永興,二人皆隸土團者也,故軍剽甚。洪雖得節制,而附硃全忠,絕東南貢路。乾寧初,身自將擊討,乞師淮南,楊行密遣硃延壽助之。洪引還,延壽拔黃州,俘討獻京師。駱殷棄永興走,行密取其地。洪得駱殷,倚為心腹,間取永興守之。



 全忠方圍鳳翔,昭宗遣使者東出,道武昌,洪皆殺之。時行密略光州,詔洪出兵,與忠義趙匡凝、武安馬殷襲安州。行密使李神福、劉存率舟師萬人討洪,駱殷棄永興走,縣民方詔守以待命。神福已得詔,大喜,以永興壯縣,饋餫所仰,既得,鄂半矣,遂進圍鄂州。



 洪嬰城,請救於汴,全忠率兵五萬營霍丘。行密御之,汴兵不利,引還,使別將吳章以三千兵解圍,神福迎破之。時全忠方與河東軍薄戰,故不能救洪。洪乃求助於馬殷,殷不答。洪計窮,復走全忠,全忠遣曹延祚合吳章兵萬三千救洪。淮南將劉存浚坎傅城。殷為洪謀曰:「淮兵深入,仰永興以濟,若奇兵取之,賊不戰而潰。」洪以精兵合汴人間道掩永興,三十里而舍。存以方詔、苗璘當之。汴亡卒走淮壁,言軍虛實曰:「鄆軍懦,可取,開道軍不可當也。」璘曰:「殺強則弱者撓矣。」乃自擊開道軍,敗之,禽汴士三百人,徇城下。洪軍氣沮,存使辯士臨說,洪恃汴方強,無降意。或勸存急擊援兵,則城自下,存曰:「擊之,賊入,則城固矣;若縱其遁,城可取也。」俄而汴軍走,是日城陷,執洪及曹延祚,窮斬其餘。行密見洪,責曰:「爾同逆賊弒主,與孤為仇,吾軍還,而復為賊後拒,今定何如?」洪謝曰:「不忍負硃公。」與延祚皆斬揚州市。以劉存守鄂州。行密死,馬殷遂取其地。



 鐘傳,洪州高安人。以負販自業,或勸其為盜必大顯。時王仙芝猖狂,江南大亂,眾推傳為長,乃鳩夷獠,依山為壁,至萬人,自稱高安鎮撫使。仙芝遣柳彥璋略撫州,不能守,傳入據之,言諸朝,詔即拜刺史。中和二年,逐江西觀察使高茂卿,遂有洪州。撫民危全諷間傳之去,竊州以叛,使弟仔昌據信州。僖宗擢傳江西團練使,俄拜鎮南節度使、檢校太保、中書令,爵潁川郡王,又徙南平。



 傳率兵圍撫州,天火其城,士民言雚驚,諸將請急攻之,傳曰:「乘人之險,不可。」乃祝曰:「全諷罪,無害民者。」火即止。全諷聞,謝罪聽命,以女女傳子匡時。傳以匡時為袁州刺史,擊馬殷。又以彭玕為吉州刺史。玕,健將也,傳倚以為重。



 廣明後,州縣不鄉貢,惟傳歲薦士,行鄉飲酒禮,率官屬臨觀,資以裝齎,故士不遠千里走傳府。傳少射獵,醉遇虎,與鬥,虎搏其肩,而傳亦持虎不置,會人斬虎,然後免。既貴,悔之,戒諸子曰:「士處世,尚智與謀,勿效吾暴虎也。」乃畫搏虎狀以示子孫。凡出軍攻戰,必禱佛祠,積餌餅為犀象,高數尋。晚節重斂,商人至棄其貨去。天祐三年卒。



 匡時自立為節度觀察留後。次子匡範為江州刺史,怨兄立,挈州附淮南,因言兄結汴人圖揚州。楊渥使秦裴攻匡時,圍洪州。匡時城守不出,凡三月,城陷,淮軍大掠三日止,執匡時及司馬陳象歸揚州。渥切責,匡時頓首請死,渥哀赦之,斬象於市。



 彭玕既失援,厚結馬殷,且觀虛實,使者還曰:「殷將校輯睦,未可圖也。」遂歸款。玕通《左氏春秋》,嘗募求西京《石經》,厚賜以金。揚州人至相語曰:「十金易一筆,百金償一篇,況得士乎?」故士人多往依之。



 始,危全諷聞匡時立,喜曰:「聽鐘郎為節度三年,我自取之。」及渥兵盛,不敢救,潛謀攻渥。會淮南亡將王茂章過州,請曰:「聞公欲大舉,願見諸將軍才否。」全諷搜眾十萬,邀茂章觀之,對曰:「揚州有士三等,公眾正當其下,盍更益之?」全諷不能答。後為楊氏所並。



 劉漢宏,本兗州小史,從大將擊王仙芝,劫輜重叛去。乾符末,略江陵,焚民室廬,廛無完家。於是都統王鐸遣將崔鍇降之,表為宿州刺史,漢宏恨賞薄,有望言。會浙東觀察使柳瑫得罪,乃授漢宏觀察使,代之。僖宗在蜀,貢輸踵驛而西,帝悅,寵其軍為義勝軍,即授節度使。漢宏既有七州,志侈大,輒曰:「天下方亂,卯金刀非吾尚誰哉?」鴉噪諸廷,命斫樹,或曰:「巨木不可伐。」怒曰:「吾能斬白蛇,何畏一木!」



 中和二年,遣弟漢宥率諸將攻杭州,壁西陵,為董昌所敗。復遣兵七萬瀕江而屯,昌使錢鏐宵濟襲破之。明年,漢宏屯黃嶺,發洞獠同攻昌,鏐出富陽擊諸營,多潰去。漢宏大沮,悉軍十萬,列艦西陵,謀宵濟襲昌。禱於江,有一矢墜前,惡之。俄與鏐遇,鏐俘馘五千,漢宏羸服走,或執之,紿而免。明日復戰,鏐斬其弟漢容、將辛約。時鐘季文守明州,盧約處州,蔣瑰婺州,杜雄臺州,硃褒溫州。褒兵最強,故漢宏使褒治大艦習戰,以史惠、施堅實、韓公汶將其軍。帝聞杭、越挐戰,遣中人焦居璠持節詔通好,皆不奉詔。



 光啟二年,鏐率諸將攻越,自趨導山,破公汶于曹娥埭。與褒戰,燒其艦,進屯豐山。堅實詣鏐降,漢宏率麾下六百人走臺州,鏐斬其母妻於屯。杜雄饗其軍,皆醉,執漢宏以見董昌。漢宏曰:「自古豈有不亡國邪?」昌使斬於市,叱刑者曰:「吾節度使,非庸人可殺。我嘗夢持金殺我者,必錢鏐也。」昌命鏐斬之。



 張雄,泗州漣水人。與里人馮弘鐸皆為武寧軍偏將。弘鐸為吏辱,雄為辯數,並見疑於節度使時溥。二人懼禍,乃合兵三百度江,壁白下,取蘇州據之。稍稍嘯會,戰艦千餘,兵五萬,乃自號「天成軍」。



 鎮海節度使周寶之敗,奔常州,聞高駢將徐約兵銳甚,誘之使擊雄,與之蘇州。雄匿眾海中,使別將趙暉據上元,資以舟械。寶兵散,多降暉,眾數萬。雄即以上元為西州。負其才,欲治臺城為府,旌旗衣服僭王者。



 楊行密圍揚州,畢師鐸厚齎寶幣,啖雄連和。雄率軍浮海屯東塘。是時揚州圍久,皮囊革帶食無餘,軍中殺人代糧,才千錢。聞雄至,間道挾珍走軍,以銀二斤易斗米,逮糠籺以差為直。雄軍富過所欲,即不戰去。暉數剽江道,雄擊殺之,坑其眾,自屯上元。大順初,以上元為升州,詔授雄刺史。未幾,卒。雄善馭眾,人思之,為立廟。弘鐸代為刺史。



 弘鐸善騎射,侃侃若儒者。行密已得淮南,弘鐸納好。然倚兵艦完利,謀取潤州,遣客尚公乃進說行密,行密不從。客曰:「公不見聽,未知勝幾樓船?」時行密大將田頵在宣州,陰圖弘鐸,募工治艦。工曰:「上元為舟,市木遠方,堅緻可勝數十歲。」頵曰:「我為舟於一用,不計其久,取木於境可也。」弘鐸介宣、揚間,不自安,而州數有怪。天復二年,大風發屋,巨木飛舞,州人駭曰:「州且易主。」大將馮暉等勸弘鐸悉軍南向,聲言討鐘傳,實襲頵。行密知之,遣客說止,不聽。頵逆擊於曷山,弘鐸大敗,收殘士欲入海。行密懼復振,遣人迎犒東塘,好謂曰:「兵有勝負,今眾尚強,乃自棄於海,奈何?吾府雖隘,尚可以居。若欲揚州,我且讓公。」弘鐸舉軍盡哭。行密挐飛艫,不持兵入其軍,執弘鐸手尉勉,遂以歸,表為淮南節度副使。見尚公乃曰:「頗憶為馮公求潤州否?何多尚邪?」謝曰:「臣為君,恨其未遂。」行密笑曰:「吾得君,尚何憂?」



 徐約者,曹州人。已得蘇州,有詔授刺史。錢鏐遣弟銶攻之,約驅民墨鑱其耏曰:「願戰南都。」從事或曰:「都者,國稱,杭終有國乎?」約後浸窘,與其下哭而別,入海死。鏐使沈粲守蘇州。約眾降潤州阮結,結不能定。鏐以成及討之,盡殲其眾。



 王潮,字信臣,光州固始人。五代祖曄為固始令,民愛其仁,留之,因家焉。世以貲顯。僖宗入蜀,盜興江、淮,壽春亡命王緒、劉行全合群盜據壽州。未幾,眾萬餘,自稱將軍,復取光州,劫豪傑置軍中,潮自縣史署軍正,主稟庾,士推其信。緒提二州籍附秦宗權。它日,賦不如期,宗權切責,緒懼,與行全拔眾南走,略潯陽、贛水,取汀州,自稱刺史,入漳州,皆不能有也。初以糧少,故兼道馳,約軍中曰:「以老孺從者斬!」潮與弟審邽、審知奉母以行,緒切責潮曰:「吾聞軍行有法,無不法之軍。」對曰:「人皆有母,不聞有無母之人。」緒怒,欲斬其母,三子同辭曰:「事母猶事將軍也,殺其母焉用其子?」緒赦之。會毋死,不敢哭,夜殯道左。



 時望氣者言軍中當有暴興者,緒潛視魁梧雄才,皆以事誅之,眾懼。次南安,潮語行全曰:「子美須眉,才絕眾,吾不知子死所。」而行全怪寤,亦不自安,與左右數十人伏叢翳,狙縛緒以徇。眾呼萬歲,推行全為將軍,辭曰:「我不及潮,請以為主。」潮苦讓不克,乃除地剚劍祝曰:「拜而劍三動者,我以為主。」至審知,劍躍於地,眾以為神,皆拜之。審知讓潮,自為副。緒嘆曰:「我不能殺是子,非天乎!」潮令於軍曰:「天子蒙難,今當出交、廣,入巴、蜀,以干王室。」於是悉師將行,會泉州刺史廖彥若貪暴,聞潮治軍有法,故州人奉牛酒迎潮。乃圍城,歲餘克之,殺彥若,遂有其地。



 初,黃巢將竊有福州,王師不能下,建人陳巖率眾拔之,又逐觀察使鄭鎰,自領州,詔即授刺史。久之,巖卒,其婿範暉擁兵自稱留後。巖舊將多歸潮,言暉可取,潮乃遣從弟彥復將兵,審知監之,攻福州。審知乘白馬履行陣,望者披靡,號「白馬將軍」。暉守彌年不下,潮令曰:「兵盡益兵,將盡益將,兵將盡,則吾至矣。」於是彥復急攻,暉亡入海,追斬之。建、汀二州皆舉籍聽命,潮乃盡有五州地。



 昭宗假潮福建等州圍練使,俄遷觀察使。乃作四門義學,還流亡,定賦斂,遣吏勸農,人皆安之。乾寧中,寵福州為威武軍,即拜潮節度使、檢校尚書左僕射。卒,贈司空。



 潮病,以審知權節度,讓審邽,不許。詔審知檢校刑部尚書、節度觀察留後。厚事硃全忠,全忠薦為節度使、同中書門下平章事。帝在鳳翔,賜審知硃詔,自三品皆得承制除授。天祐初,進瑯邪郡王。



 審邽,字次都。為泉州刺史,檢校司徒。喜儒術,通《書》、《春秋》。善吏治,流民還者假牛犁,興完廬舍。中原亂,公卿多來依之,振賦以財,如楊承休、鄭璘、韓偓、歸傳懿、楊贊圖、鄭戩等賴以免禍,審邽遣子延彬作招賢院以禮之。



 劉知謙,壽州上蔡人。避亂客封州,為清海牙將。節度使韋宙以兄女妻之,眾謂不可,宙曰:「若人狀貌非常,吾以子孫托之。」



 黃巢自嶺表北還,湖、湘間群盜蟻結,知謙因據封州,有詔即授刺史兼賀水鎮使,以遏梧、桂。知謙撫納流亡,愛嗇用度,養士卒。未幾,得精兵萬人,多具戰艦,境內肅然。久之,疾病,召諸子曰:「今五嶺盜賊方興,吾有精甲犀械,爾勉建功,時哉不可失也!」



 知謙卒,共推其子隱為嗣,清海軍節度使劉崇龜表為封州刺史。嗣薛王知柔代領節度,未至,而牙將盧琚叛。隱率兵奉迎知柔,直趨廣州,禽琚獻之。於是知柔以聞,昭宗拜隱本軍行軍司馬,俄遷副使。天復初,節度徐彥若死,隱自稱留後。



 虔人盧光稠者,有眾數萬,據州自為留後,又取韶州。隱與爭之,戰不勝,悉師攻虔州。光稠伏軍掉戰,隱縱驅,伏發,挺身免。天祐初,始詔隱權節度留後,乃遣使者入朝,重賂硃全忠以自固。是歲,光稠死,子延昌自稱刺史,為其下所殺,更推李圖總州事。圖死,鐘傳盡劫其眾,欲遣子匡時守之。不克,州人自立譚全播為刺史,附全忠云。



\end{pinyinscope}