\article{列傳第一百一十二 二王諸葛李孟}

\begin{pinyinscope}

 王重榮,太原祁人。父縱,太和末為河中騎將,從石雄破回鶻,終鹽州刺史。重榮以父任為列校學派·團體,與兄重盈皆以毅武冠軍,擢河中牙將,主伺察。時兩軍士干夜禁,捕而鞭之。士還,訴於中尉楊玄實,玄實怒,執重榮讓曰:「天子爪士,而籓校辱之!」答曰:「夜半執者奸盜,孰知天子爪士?」具言其狀。玄實嘆曰:「非爾明辨,孰由知之?」更諉於府,擢右署。重榮多權詭,眾所嚴憚,雖主帥莫不下之。稍遷行軍司馬。



 黃巢陷長安,分兵略蒲,節度使李都不能支,乃臣賊,然內憚重榮,表以自副。地邇京師,賊調取橫數,使者至百輩,坐傳舍,益發兵,吏不堪命。重榮脅說都曰:「我所詭謀紓難,以外援未至。今賊裒責日急,又收吾兵以困我,則亡無日矣。請絕橋,嬰城自守,不然,變生何以制之?」都曰:「吾兵寡,謀不足,絕之,禍且至,願以節假公。」遂奔行在。重榮乃悉驅出賊使斬之,因大掠居人以悅其下。天子使前京兆尹竇潏間道慰其軍,因詔代都。重榮率官屬奉迎。潏至,大饗士,倡言曰:「天子以大臣守土,誰得逐之?為我疏首惡者。」眾無敢對。重榮佩刀歷階曰:「首謀者,我也,尚誰索?」目潏,吏趣具騎,潏即奔還。重榮遂主留後。



 賊使健將硃溫以舟師下馮翊,黃鄴率眾自華陰合攻重榮。重榮感勵士眾,大戰,敗之,賊棄糧仗四十餘艘。即拜檢校工部尚書,為節度使。會忠武監軍楊復光率陳、蔡兵萬人屯武功,重榮與連和,擊賊將李詳於華州,執以徇。賊使尚讓來攻,而硃溫將勁兵居前,敗重榮兵於西關門,於是出兵夏陽,掠河中漕米數十艘。重榮選兵三萬攻溫,溫懼,悉鑿舟沉於河,遂舉同州降。復光欲斬之,重榮曰:「今招賊,一切釋罪。且溫武銳可用,殺之不祥。」表為同華節度使。有詔即副河中行營招討,賜名全忠。



 巢喪二州,怒甚,自將精兵數萬壁梁田。重榮軍華陰,復光軍渭北,掎角攻之,賊大敗,執其將趙璋,巢中流矢走。重榮兵亦死耗相當。懼巢復振,憂之,與復光計,復光曰:「我世與李克用共憂患,其人忠不顧難,死義如己。若乞師焉,事蔑不濟。」乃遣使者約連和。克用使陳景斯總兵自嵐、石赴河中,親率師從之,遂平巢,復京師。以功檢校太尉、同中書門下平章事,封瑯邪郡王。累加檢校太傅。



 中人田令孜怒重榮據鹽池之饒。於時巨盜甫定,國用大覂,諸軍無所仰,而令孜為神策軍使,建請二池領屬鹽鐵,佐軍食。重榮不許,奏言:「故事,歲輸鹽三千乘於有司,則斥所餘以贍軍。」天子遣使者諭旨,不聽。令孜徙重榮兗海節度使,以王處存代之,詔克用將兵援河中。重榮上書劾令孜離間方鎮。令孜遣邠寧硃玫進討,壁沙苑。重榮詒克用書,且言:「奉密詔,須公到,使我圖公。此令孜、硃全忠、硃玫之惑上也。」因示偽詔。克用方與全忠有隙,信之,請討全忠及玫。帝數詔和解。克用合河中兵戰沙苑,玫大敗,奔邠州。神策軍潰還京師,遂大掠。克用乘勝西,天子走鳳翔。



 俄嗣襄王煴僭位,重榮不受命,與克用謀定王室。楊復恭代令孜領神策,故與克用善,遣諫議大夫劉崇望齎詔諭天子意,兩人聽命,即獻縑十萬,願討玫自贖。崇望還,群臣皆賀。重榮遂斬煴,長安復平。然性悍酷,多殺戮,少縱舍。嘗植大木河上,內設機軸,有忤意者,輒置其上,機發皆溺。嘗辱部將常行儒,行儒怨之。光啟三年,引兵夜攻府,重榮亡出外,詰旦殺之,推立重盈。



 重盈前此已歷汾州刺史。黃巢度淮,擢陜虢觀察使,重榮據河中,三遷檢校尚書右僕射,即拜節度使。未幾,同中書門下平章事。及代重榮,留長子珙領節度事,入殺行儒,軍復安。昭宗立,進太傅、兼中書令,封瑯邪郡王。父子兄弟相繼帥守,而從子蘊亦為忠武節度使。



 乾寧二年,重盈死,軍中以其兄重簡子珂出繼重榮,故推為留後。珙與弟絳州刺史瑤爭河中,上言:「珂本家蒼頭,請選大臣鎮河中。」又與硃全忠書言之。珂急,乃遣使請婚於李克用。克用薦之天子,許嗣鎮,然猶以崔胤為河中節度使。珙復構珂於王行瑜、李茂貞,曰:「珂不受代,且晉親也,將不利於公。」行瑜等約韓建共薦珙。詔曰:「吾重已授珂矣。重榮有大功,不可廢。」行瑜怒,使其弟行約攻珂,克用遣李嗣昭援之,敗珙於猗氏,獲其將李璠。



 三鎮銜帝之卻其請也,連兵犯京師,謀廢帝、誅執政而立吉王,固請授珙河中。克用聞之怒,以師討三鎮,瑤、珙兵引去。克用拔絳州,斬瑤而屯渭北,敗行約朝邑。



 行約走京師。弟行實在左軍,共說樞密使駱全瓘,謀挾帝幸邠。右軍李繼鵬以告中尉劉景宣二人,茂貞黨也。,欲以兵劫全瓘等,請帝幸鳳翔。兩軍合噪承天門街,帝登樓諭和之,繼鵬怒,輒射帝,縱火焚門,帝率諸王及衛兵戰,繼鵬矢及帝胄,軍乃退。帝出幸定州將李筠軍,嗣延王戒丕、嗣丹王允以鹽州六都兵從帝出啟夏門,次於郊。兩軍憚鹽州兵銳,各走其軍。帝次莎城,百官繼至,士民從者亦數萬。帝欲入谷中自固,以穀有「沒唐石」,惡之,徙石門。民匿保山谷間,帝每出,或獻飴漿,帝駐馬為嘗,民皆流涕。既而遣嗣薛王知柔及劉光裕還京師。



 克用遣使者奔問行在,帝因詔克用、珂以兵趨新平,又詔涇州張鐇會克用軍以扼岐陽。克用在河中未出也,帝懼茂貞之逼,復使嗣延王戒丕以御服玉器賜之,督其西,乃壁渭北,進營渭橋。於是行瑜壁興平,茂貞壁鄠。行瑜兵數卻,茂貞懼,斬繼鵬,傳首以謝。繼鵬姓閻名珪,左神策軍拍張人,為茂貞養子云。詔削行瑜官爵,以克用為邠寧四面行營都招討使,珂為糧料使。克用遣子存貞請天子還宮。詔以騎三千戍三橋。



 帝既還,加珂檢校司空,為節度使。克用以女妻之,珂親迎太原,以李嗣昭助守河中,因攻珙,珙戰數北。珙任威虐,殺人斷首置前,而顏色泰定,下恐,不敢叛,然稍弱,無鬥志。光化二年,為部將李璠所殺,自為留後,詔代珙節度。又失眾,凡五月,為牙將硃簡所殺,挈其地入硃全忠,表授節度使、同中書門下平章事,更名友謙。



 珙殺給事中王柷等十餘人,幕府曹遭戮辱甚眾,人有罪輒刳皪以逞。貨者,故為常州刺史,避難江湖,帝聞剛鯁,以給事中召,道出陜。珙謂且柄任,厚禮之。貨鄙其武暴,不降意。既宴,盛列珍器音樂,珙請於貨曰:「僕今日得在子弟列,大賜也。」三請,貨不答。珙勃然曰:「天子召公,公不可留此。」遂罷,遣吏就道殺之,族其家,投諸河,以溺死聞。帝不能詰。珙死,贈太師。詔陜州冤死者,有司吊祭,存問其家。



 始,全忠擊楊行密不能克,諷荊、襄、青、徐等道請己為都統以討行密,帝依違未報;而珂與太原、鎮定等道亦請加行密都統,以討全忠。繇是兩罷之,全忠怨珂,不忘也。帝為劉季述所廢,珂憤見言色,屢陳討賊謀。既反正,首獻方物,帝甚倚之。而全忠以克用方強,不敢加兵。及王鎔詘服,拔定州,而克用兵折,乃謂其將張存敬曰:「珂恃太原侮慢我,爾持一繩縛之。」存敬以兵數萬度河,由含山襲,絳州刺史陶建釗、晉州刺史張漢瑜皆降,以何絪戍之,進攻珂。全忠率師繼進,即劾珂交構克用,為方鎮生事,不可赦。珂乞師太原,為絪所迮,不能進。珂急使妻遺克用書曰:「賊攻我,朝夕見俘,乞食大梁矣。」克用答曰:「道且斷,往救必俱亡,不如歸朝廷。」珂窮,遣使告李茂貞曰:「上初反正,詔籓鎮無相侵。而硃公不顧約,以攻敝邑。敝邑亡,則邠、岐非君所保,天子神器斂手付人矣。宜與華州韓公出精銳固潼關,以張兵勢。僕不武,公其惠我西偏地,以為捍守。蒲,請公自有之。關西安危,國祚長短,系公此舉也。」茂貞不答。



 珂益蹙,會橋毀,潛具舟將遁,夜諭守兵,無肯為用者。牙將劉訓叩寢門,珂疑有變,叱之,訓自袒其衣曰:「茍有它,請斷臂自明!」珂出,問計所宜,答曰:「若夜出,人將爭舟,一夫鴟張,禍系其手。如旦日,以情諗軍中,宜有樂從者,可則濟,否則召諸將行成以緩敵,徐圖所向,上策也。」珂然之。明日,登城語存敬曰:「吾於硃公有父子歡,君姑退舍,須公至,吾自聽命。」乃執太原諸將並奉節印內存敬軍,豎大幡城上,遣兄璘與諸將樊洪等見存敬。存敬解圍而戍以兵。



 全忠自洛至。全忠,王出也,始背賊事重榮,約為甥舅,德其全己,指日月曰:「我得志,凡氏王者皆事之。」至是,忘誓言,過重榮墓,偽哭而祭。次虞鄉,珂欲面縛牽羊以見,全忠報曰:「舅之恩,無日可忘。君若以亡國禮見,黃泉其謂我何?」珂出迎,握手泣下,駢轡以入。居旬日,以存敬守河中,舉珂室徙於汴。後令入覲,遣人賊之於華州。



 自重榮傳珂,凡二十年。



 諸葛爽,青州博昌人。為縣伍伯,令笞苦之,乃亡命,沈浮里中。龐勛反,入盜中為小校。勛勢蹙,率百餘人與泗州守將湯群自歸,累遷汝州防禦使。李琢討沙陀於雲州,表為北面招討副使。徙夏綏銀節度使,檢校尚書右僕射。



 黃巢犯京師,詔率代北行營兵入衛,次同州,降賊,偽署河陽節度使,代羅元杲。元杲者,本神策將,狀短陋,倚中官勢,剽財輸京師,凡鉅萬,人怨之。爽至,募州人戰,眾不從,相率迎爽,元杲奔行在。爽間道表僖宗以自明,詔拜節度使。李克用援陳許,道天井關。爽懼,不肯假道,出屯萬善。克用自河中趨汝、洛。



 爽累授京師東南面招討諸行營副都統、左先鋒使,兼中書門下平章事。硃溫為賊守同州,爽率輕兵入之,溫偃旗設伏以待,爽謂賊遁,士解甲就舍,伏發,爽悉棄鎧馬奔還。至修武,為魏博韓簡擊敗之,不敢入。簡留將趙文弁戍河陽,自攻鄆,時中和二年也。河陽人誘爽,自金、商馳,復入之,厚禮文弁及戍人,還之魏。於是爽攻新鄉,簡自鄆來,戰獲嘉西。簡陰窺關中,其下不悅,裨將樂彥禎間眾之隙,引其軍先還,故簡兵八萬自潰,相藉溺清水至不流。明年,詔爽為東南面招討使,伐秦宗權,表李罕之自副。



 爽雖興庸廝,善吏治,法令澄壹,人無愁咨。擢累檢校司空。光啟二年卒。其將劉經與澤州刺史張言共立爽子仲方為留後,為蔡賊孫儒所攻,奔於汴,儒取孟州。



 李罕之,陳州項城人。少拳捷。初為浮屠,行丐市,窮日無得者,抵缽褫祗衼去,聚眾攻剽五臺下。先是,蒲、絳民壁摩雲山避亂,群賊往攻不克,罕之以百人徑拔之,眾號「李摩雲」。隨黃巢度江,降於高駢,駢表知光州事。為秦宗權所迫,奔項城,收餘眾依諸葛爽,署懷州刺史。爽伐宗權,即表以自副。屯睢陽,無功。又表為河南尹、東都留守,使捍蔡。



 河東李克用脫上源之難,喪氣還,罕之迎謁謹甚,勞餼加等,厚相結。罕之因府為屯,會孫儒來攻,罕之不出。數月,走保黽池。東都陷,儒焚宮闕,剽居民去。爽遣將收東都,罕之逐出之,爽不能制。俄而爽死,其將劉經、張言共立爽子仲方,欲去罕之。而罕之故與郭璆有隙,擅殺璆,軍中不悅。經間眾怒,襲其壁,罕之退保乾壕,經追擊,反為所敗,乘勝入屯洛陽苑中。經戰不勝,還河陽。罕之屯鞏,將度汜,經遣張言拒河上,反與罕之合,攻經不克,屯懷州。



 孫儒逐仲方,取河陽,自稱節度使。俄而宗權敗,棄河陽走,罕之、言進收其眾,丐援河東,克用遣安金俊率兵助之,得河陽。克用表罕之為節度使、同中書門下平章事。有詔與屬籍。又表言為河南尹、東都留守。



 罕之與言甚篤,然性猜暴。是時大亂後,野無遺稈,部卒日剽人以食。又攻絳州,下之,復擊晉州,王重盈欲出汴兵救,罕之解圍還。而言善積聚,勸民力耕,儲廥稍集。罕之食乏,士仰以給,求之無涯,言不能厭,罕之拘河南官吏笞督之;又東方貢輸行在者,多為罕之邀頡。重盈反間於言,文德元年,罕之悉兵攻晉州,言夜襲河陽,俘罕之家。罕之窮,奔河東,克用復表為澤州刺史,領河陽節度使,遣李存孝、薛阿檀、安休休率師三萬攻言。城中食盡,言納孥于汴求救,全忠遣丁會、葛從周、牛存節來援,戰沅河聚。休休不利,降全忠,存孝還。全忠更以丁會為河陽節度使,言歸洛陽。



 罕之保澤州,數出鈔懷、孟、晉、絳,無休歲,人匿保山谷,出為樵汲者,罕之俘斬略盡,數百里無舍煙。克用遣罕之、存孝攻孟方立,拔磁州,方立戍將馬溉兵數萬戰琉璃陂,罕之禽溉,敗其眾。大順初,汴將李讜、鄧季筠攻罕之,罕之告急於克用,遣存孝以騎五千救之。汴士呼罕之曰:「公倚沙陀,絕大國。今太原被圍,葛司空入上黨,不旬日,沙陀無穴處矣!」存孝怒,引兵五百薄讜營,呼曰:「我,沙陀求穴者,須爾肉以飽吾軍,請肥者出鬥!」季筠引兵決戰,存孝奮槊馳,直取季筠。讜夜走,追至馬牢川,敗之。克用討王行瑜,表罕之副都統,檢校侍中。行瑜誅,封隴西郡王,檢校太尉、兼侍中。



 罕之恃功多,嘗私克用愛將蓋寓求一鎮,寓為請,克用不許,曰:「鷹鸇飽則去矣,我懼其翻覆也。」光化初,昭義節度使薛志勤卒,罕之夜襲潞,入之,自稱留後,報克用曰:「志勤死,懼它盜至,不俟命輒屯於潞。」克用遣李嗣昭先擊澤州,拘罕之家屬送太原。罕之攻沁州,執刺史、守將,送款於汴,全忠表罕之昭義節度使,命丁會援之;與嗣昭戰含口,嗣昭不利,葛從周取澤州。嗣昭又攻罕之,罕之暴得病,不能事。會代戍,全忠更以罕之節度河陽三城,卒於行,年五十八。未幾,嗣昭復取澤州,以李存璋為刺史,進收懷州,攻河陽。汴將閻寶引兵至,嗣昭還。



 始,儒去東都也,井閈不滿百室。言治數年,人安賴之,占籍至五六萬,繕池壘,作第署,城闕復完。全忠懼言異己,乃徙節天平,以韋震為河南尹。爽諸將無傳地者,言後嗣名全義。



 王敬武,青州人。隸平盧軍為偏校,事節度使安師儒。中和中,盜發齊、棣間,遣敬武擊定。已還,即逐師儒,自為留後。時王鐸方督諸道行營軍復京師,因承制授敬武平盧節度使,趣其兵使西。及京師平,進檢校太尉、同中書門下平章事。龍紀元年卒。



 子師範,年十六,自稱留後,嗣領事。昭宗自以太子少師崔安潛領節度,師範拒命。時棣州刺史張蟾迎安潛,師範遣部將盧弘攻之,弘與蟾連和。師範以金啖之,曰:「君若顧先人,使不絕其祀,君之惠也。不然,願死墳墓。」弘少之,不為備,師範伏兵迎於路,部將劉莘斬弘,遂攻棣州。蟾請救於硃全忠,全忠馳使諭解,師範拔其城,斬蟾,而安潛不敢入。



 師範喜儒學,謹孝,於法無所私。舅醉殺人,其家訴之,師範厚賂謝,訴者不置,師範曰:「法非我敢亂。」乃抵舅罪。母恚之,師範立堂下,日三四至,不得見三年,拜省戶外不敢懈。以青州父母所籍,每縣令至,具威儀入謁,令固辭,師範遣使挾坐,拜廷中乃出。或諫不可,答曰:「吾恭先世,且示子孫不忘本也。」



 全忠已並鄆州,遣兵攻師範,師範下之。會全忠圍鳳翔,昭宗詔方鎮赴難,以師範附全忠,命楊行密部將硃瑾攻青州,且欲代為平盧節度。師範聞之,哭曰:「吾為國守籓,君危不持,可乎?」乃與行密連盟。遣將張居厚、李彥威以甲槊二百輿紿為獻者,及華州,先內十輿,閽人覺,眾擐甲噪,殺全忠守將婁敬思。是時崔胤方在華,閉門拒戰,執居厚還全忠。



 劉鄩襲兗州,入之。師範亦潛兵入河南,徐、沂、鄆等十餘州同日並發。全忠使從子友寧率軍東討。是時帝還長安,故全忠並魏博軍屯齊州。王茂章方以兵二萬合師範弟師誨攻密州,破之,以張訓為刺史。進攻沂州,敗其兵,還青州,半舍而屯。友寧方攻博昌,未下,全忠督戰急,友寧驅民十萬,負木石,築山臨城中,城陷,屠老少投尸清水,遂圍登州。茂章欲啖友寧,不肯救。未幾,城破,友寧負勝攻別屯。茂章度汴軍怠,與師範合擊友寧於石樓,斬其首,傳於行密。



 全忠怒,悉軍二十萬倍道至。茂章閉營,伺軍懈,毀壁出鬥,還與諸將飲,訖,復戰。全忠望見,嘆曰:「吾有將如是,天下不足平!」於是退屯臨淄。茂章畏全忠,乃斂軍而南,使李虔裕以五百人後拒。茂章解衣寐,虔裕呼曰:「追至,將軍速去!」茂章曰:「吾共決死。」虔裕固請,茂章乃去。已而追至,虔裕一軍覆,茂章免。全忠見虔裕,欲釋之,瞋目大罵而死。張訓召諸將謀曰:「汴人至,師少,何以待之?」眾請焚城而亡,訓曰:「不然。」即封府藏,下縣門,密引兵去。汴軍見府庫完,德之,不追。



 全忠留楊師厚圍青州,敗師範兵於臨朐,執諸將,又獲其弟師克。是時,師範眾尚十餘萬,諸將請決戰,而師範以弟故,乃請降。全忠歸其弟,假師範知節度留後事,師範獻錢二十萬緡以謝軍。汴將劉重霸執棣州刺史邵播,得其書八百紙,皆教師範戰守,全忠憚而殺之。



 葛從周圍兗州,劉鄩不肯下,從周以師範命招之,乃盡出將士,開門降。從周為辦裝,使詣汴,鄩但素服乘驢而往。全忠賜冠帶,辭曰:「囚請就縶。」不許。既見,慰之,飲以酒,固辭。全忠笑曰:「取兗州,量何大邪?」擢署都押衙,在諸舊將上。諸將趨入,鄩一無讓,全忠奇之。



 歲餘,徙師範於汴,亦縞素請罪。全忠見以禮,表為河陽節度使。既受唐禪,友寧妻訴仇人於朝,乃族師範於洛陽。先是,有司坎第左,告之故。師範乃與家人宴,少長列坐,語使者曰:「死固不免,予懼坑之則昭穆失序,不可見先人地下。」酒行,以次受戮者二百人。



 孟方立,邢州人。始為澤州天井戍將,稍遷游奕使。中和元年,昭義節度使高潯擊黃巢,戰石橋,不勝,保華州,為裨將成鄰所殺,還據潞州。眾怒,方立率兵攻鄰,斬之,自稱留後,擅裂邢、洺、磁為鎮,治邢為府,號昭義軍。潞人請監軍使吳全勖知兵馬留後。時王鐸領諸道行營都統,以潞未定,墨制假方立檢校左散騎常侍、兼御史大夫,知邢州事,方立不受,囚全勖,以書請鐸,願得儒臣守潞。鐸使參謀中書舍人鄭昌圖知昭義留事,欲遂為帥。僖宗自用舊宰相王徽領節度。時天子在西,河、關雲擾,方立擅地而李克用窺潞州,徽度朝廷未能制,乃固讓昌圖。昌圖治不三月,輒去。方立更表李殷銳為刺史。謂潞險而人悍,數賊大帥為亂,欲銷懦之,乃徙治龍岡。州豪傑重遷,有懟言。會克用為河東節度使,昭義監軍祁審誨乞師,求復昭義軍。克用遣賀公雅、李筠、安金俊三部將擊潞州,為方立所破。又使李克修攻取之,殺殷銳,遂並潞州,表克修為節度留後。初,昭義有潞、邢、洺、磁四州,至是,方立自以山東三州為昭義,而朝廷亦命克修,以潞州舊軍畀之。昭義有兩節,自此始。



 克修,字崇遠,克用從父弟。精馳射,常從征伐,自左營軍使擢留後,進檢校司空。



 方立倚硃全忠為助,故克用擊邢、洺、磁無虛歲,地為鬥場,人不能稼。光啟二年,克修擊邢州取故鎮,進攻武安。方立將呂臻、馬爽戰焦岡,為克修所破,斬首萬級,執臻等,拔武安、臨洺、邯鄲、沙河。克用以安金俊為邢州刺史,招撫之。方立丐兵於王鎔,鎔以兵三萬赴之,克修還。後二年,方立督部將奚忠信兵三萬攻遼州,以金啖赫連鐸與連和。會契丹攻鐸,師失期,忠信三分其兵,鼓而行,克用伏兵於險,忠信前軍沒。既戰,大敗,執忠信,餘眾走,脫歸者才十二。龍紀元年,克用使李罕之、李存孝擊邢,攻磁、洺,方立戰琉璃陂,大敗,禽其二將,被斧鑕,徇邢壘,呼曰:「孟公速降,有能斬其首者,假三州節度使!」方立力屈,又屬州殘墮,人心恐。性剛急,持下少恩,夜自行陴,兵皆倨,告勞。自顧不可復振,乃還,引酖殺。



 從弟遷,素得士心,眾推為節度留後,請援於全忠。全忠方攻時溥,不即至,命王虔裕以精甲數百赴之,假道羅弘信,不許,乃趨間入邢州。大順元年,存孝復攻邢,遷挈邢、洺、磁三州降,執王虔裕三百人獻之,遂遷太原。表安金俊為邢、洺、磁團練使,以遷為汾州刺史。



 贊曰:以亂救亂,跋扈者能之;以亂不能救亂,險賊者能之。蓋救亂似霸,然而似之耳,故不足與共功。觀王重榮寧不信哉!破黃巢,佐李克用平京師,若有為當世者。俄而奮私隙,逼天子出奔,雖馘硃玫,僕偽襄王,謂曰「定王室」,實卑之也。身死部將手,救亂而卒於亂,重榮兩得之。不殺硃全忠,而為全忠誅,絕其嗣,宜矣。餘皆庸奴下材,無所訾責云。



\end{pinyinscope}