\article{志第二十六 五行三}

\begin{pinyinscope}

 《五行傳》曰:「簡宗廟,不禱祠,廢祭祀,逆天時,則水不潤下。」謂水失其性百川逆溢,壞鄉邑,溺人民而為災也。又曰:「聽之不聰,是謂不謀。厥咎急,厥罰常寒,厥極貧。時則有鼓妖,時則有豕禍,時則有耳痾,時則有雷電、霜、雪、雨、雹、黑眚黑祥,惟火沴水。」



 △水不潤下



 貞觀三年秋,貝、譙、鄆、泗、沂、徐、豪、蘇、隴九州水。水,太陰之氣也。若臣道顓,女謁行,夷狄強,小人道長,嚴刑以逞,下民不堪其憂,則陰類勝,其氣應而水至;其謫見於天,月及辰星與列星之司水者為之變,若七曜循中道之北,皆水祥也。四年秋,許、戴、集三州水。七年八月,山東、河南州四十大水。八年七月,山東、江淮大水。十年,關東及淮海旁州二十八大水。十一年七月癸未,黃氣際天,大雨,穀水溢,入洛陽宮,深四尺,壞左掖門,毀官寺十九;洛水漂六百餘家。九月丁亥,河溢,壞陜州之河北縣及太原倉,毀河陽中水單。十六年秋,徐、戴二州大水。十八年秋,穀、襄、豫、荊、徐、梓、忠、綿、宋、亳十州大水。十九年秋,泌、易二州水,害稼。二十一年八月,河北大水,泉州海溢,驩州水。二十二年夏,瀘、越、徐、交、渝等州水。



 永徽元年六月,新豐、渭南大雨,零口山水暴出,漂廬舍;宣、歙、饒、常等州大雨,水,溺死者數百人。秋,齊、定等州十六水。二年秋,汴、定、濮、亳等州水。四年,杭、夔、果、忠等州水。五年五月丁丑夜,大雨,麟游縣山水沖萬年宮玄武門,入寢殿,衛士有溺死者。六月河北大水,滹沱溢,損五千餘家。六年六月,商州大水。秋,冀、沂、密、兗、滑、汴、鄭、婺等州水,害稼;洛州大水,毀天津橋。十月,齊州河溢。



 顯慶元年七月,宣州涇縣山水暴出,平地四丈,溺死者二千餘人。九月,括州暴風雨,海水溢,壞安固、永嘉二縣。四年七月,連州山水暴出,漂七百餘家。



 麟德二年六月,鄜州大水,壞居人廬舍。



 總章二年六月,括州大風雨,海溢,壞永嘉、安固二縣,溺死者九千七十人;冀州大雨,水平地深一丈,壞民居萬家。



 咸亨元年五月丙戌,大雨,山水溢,溺死五千餘人。二年八月,徐州山水漂百餘家。四年七月,婺州大雨,山水暴漲,溺死五千餘人。



 上元三年八月,青州大風,海溢,漂居人五千餘家;齊、淄等七州大水。



 永隆元年九月,河南、河北大水,溺死者甚眾。二年八月,河南、河北大水,壞民居十萬餘家。



 永淳元年五月丙午,東都連日澍雨;乙卯,洛水溢,壞天津橋及中橋,漂居民千餘家。六月乙亥,京師大雨,水平地深數尺。秋,山東大雨,水,大饑。二年七月己巳,河溢,壞河陽橋。八月,恆州滹沱河及山水暴溢,害稼。



 文明元年七月,溫州大水,漂千餘家;括州溪水暴漲,溺死百餘人。



 如意元年四月,洛水溢,壞永昌橋,漂居民四百餘家。七月,洛水溢,漂居民五千餘家。八月,河溢,壞河陽縣。



 長壽二年五月,棣州河溢,壞居民二千餘家。是歲,河陽州十一水。



 萬歲通天元年八月,徐州大水,害稼。



 神功元年三月,括州水,壞民居七百餘家。是歲,河南州十九,水。



 聖歷二年七月丙辰,神都大雨,洛水壞天津橋。秋,水溢懷州,漂千餘家。三年三月辛亥,鴻州水,漂千餘家,溺死四百餘人。



 久視元年十月,洛州水。



 長安三年六月,寧州大雨,水,漂二千餘家,溺死千餘人。四年八月,瀛州水,壞民居數千家。



 神龍元年四月,雍州同官縣大雨,水,漂民居五百餘家。六月,河北州十七大水。七月甲辰,洛水溢,壞民居二千餘家。二年四月辛丑,洛水壞天津橋,溺死數百人。八月,魏州水。



 景龍三年七月,澧水溢,害稼。九月,密州水,壞民居數百家。



 開元三年,河南、河北水。四年七月丁酉,洛水溢,沉舟數百艘。五年六月甲申,瀍水溢,溺死者千餘人;鞏縣大水,壞城邑,損民居數百家;河南水,害稼。八年夏,契丹寇營州,發關中卒援之,宿澠池之缺門,營谷水上,夜半,山水暴至,萬餘人皆溺死。六月庚寅夜,穀、洛溢,入西上陽宮,宮人死者十七八,畿內諸縣田稼廬舍蕩盡,掌閑衛兵溺死千餘人,京師興道坊一夕陷為池,居民五百餘家皆沒不見。是年,鄧州三鴉口大水塞谷,或見二小兒以水相沃,須臾,有蛇大十圍,張口仰天,人或斫射之,俄而暴雷雨,漂溺數百家。十年五月辛酉,伊水溢,毀東都城東南隅,平地深六尺;河南許、仙、豫、陳、汝、唐、鄧等州大水,害稼,漂沒民居,溺死者甚眾。六月,博州,棣州河決。十二年六月,豫州大水。八月,兗州大水。十四年秋,天下州五十,水,河南、河北尤甚,河及支川皆溢,懷、衛、鄭、滑、汴、濮人或巢或舟以居,死者千計;潤州大風自東北,海濤沒瓜步。十五年五月,晉州大水。七月,鄧州大水,溺死數千人;洛水溢,入鄜城,平地丈餘,死者無算,壞同州城市及馮翊縣,漂居民二千餘家。八月,澗、谷溢,毀澠池縣。是秋,天下州六十三大水,害稼及居人廬舍,河北尤甚。十七年八月丙寅,越州大水,壞州縣城,十八年六月壬午,東都瀍水溺揚、楚等州租船,洛水壞天津、永濟二橋及民居千餘家。十九年秋,河南水,害稼。二十年秋,宋、滑、兗、鄆等州大水。二十二年秋,關輔、河南州十餘水,害稼。二十七年三月,澧、袁、江等州水。二十八年十月,河南郡十三水。二十九年七月,伊、洛及支川皆溢,害稼,毀天津橋及東西漕、上陽宮仗舍,溺死千餘人。是秋,河南、河北郡二十四水,害稼。



 天寶四載九月,河南、淮陽、睢陽、譙四郡水。十載,廣陵大風駕海潮,沈江口船數千艘。十三載九月,東都瀍、洛溢,壞十九坊。



 廣德元年九月,大雨,水平地數尺,時吐蕃寇京畿,以水自潰去。二年五月,東都大雨,洛水溢,漂二十餘坊;河南諸州水。



 大歷元年七月,洛水溢。二年秋,湖南及河東、河南、淮南、浙東西、福建等道州五十五水災。七年二月,江州江溢。十年七月,杭州海溢。十一年七月戊子,夜澍雨,京師平地水尺餘,溝渠漲溢,壞民居千餘家。十二年秋,京畿及宋、亳、滑三州大雨水,害稼,河南尤甚,平地深五尺,河溢。



 建中元年,幽、鎮、魏、博大雨,易水、滹沱橫流,自山而下,轉石折樹,水高丈餘,苗稼蕩盡。



 貞元二年六月丁酉,大風雨,京城通衢水深數尺,有溺死者。東都、河南、荊南、淮南江河溢。三年三月,東都、河南、江陵、汴揚等州大水。四年八月,灞水暴溢,殺百餘人。八年秋,自江淮及荊、襄、陳、宋至於河朔州四十餘大水,害稼,溺死二萬餘人,漂沒城郭廬舍,幽州平地水深二丈,徐、鄭、涿、薊、檀、平等州,皆深丈餘。八年六月,淮水溢,平地七尺,沒泗州城。十一年十月,朗、蜀二州江溢。十二年四月,福、建二州大水,嵐州暴雨,水深二丈。十三年七月,淮水溢於亳州。十八年春,申、光、蔡等州大水。



 永貞元年夏,朗州之熊、武五溪溢。秋,武陵、龍陽二縣江水溢,漂萬餘家。京畿長安等九縣山水害稼。



 元和元年夏,荊南及壽、幽、徐等州大水。二年六月,蔡州大雨,水平地深數尺。四年十月丁未,渭南暴水,漂民居二百餘家。六年十月,鄜坊、黔中水。七年正月,振武河溢,毀東受降城;五月,饒、撫、虔、吉、信五州暴水,虔州尤甚,平地有深至四丈者。八年五月,陳州、許州大雨,大隗山摧,水流出,溺死者千餘人。六月庚寅,大風,毀屋揚瓦,人多壓死;京師大水,城南深丈餘,入明德門,猶漸車輻。辛卯,渭水漲,絕濟。時所在百川發溢,多不由故道。滄州水潦,浸鹽山等四縣。九年秋,淮南及岳、安、宣、江、撫、袁等州大水,害稼。十一年五月,京畿大雨水,昭應尤甚;衢州山水害稼,深三丈,毀州郭,溺死百餘人。六月,密州大風雨,海溢,毀城郭;饒州浮梁、樂平二縣暴雨,水,漂沒四千餘戶;潤、常、潮、陳、許五州及京畿水,害稼。八月甲午,渭水溢,毀中橋。十二年六月乙酉,京師大雨,水,含元殿一柱傾,市中水深三尺,毀民居二千餘家;河南、河北大水,洺、邢尤甚,平地二丈;河中、江陵、幽澤潞晉隰蘇臺越州水,害稼。十三年六月辛未,淮水溢。十五年秋,洪、吉、信、滄等州水。



 長慶二年七月,河南陳、許、蔡等州大水;好畤山水漂民居三百餘家;處州大雨,水,平地深八尺,壞城邑、桑田太半。四年夏,蘇、湖二州大雨,水,太湖決溢;睦州及壽州之霍山山水暴出;鄆、曹、濮三州雨,水壞州城、民居、田稼略盡;襄、均、復、郢四州漢水溢決。秋,河南及陳、許二州水,害稼。



 寶歷元年秋,鄜、坊二州暴水;兗、海、華三州及京畿奉天等六縣水,害稼。



 大和二年夏,京畿及陳、滑二州水,害稼;河陽水,平地五尺;河決,壞棣州城;越州大風,海溢;河南鄆、曹、濮、淄、青、齊、德、兗、海等州並大水。三年四月,同官縣暴水,漂沒二百餘家;宋、亳、徐等州大水,害稼。四年夏,江水溢,沒舒州太湖、宿松、望江三縣民田數百戶;鄜坊水,漂三百餘家;浙西、浙東、宣歙、江西、鄜坊、山南東道、淮南、京畿、河南、江南、荊襄、鄂岳、湖南大水,皆害稼。五年六月,玄武江漲,高二丈,溢入梓州羅城;淮西、浙東、浙西、荊襄、岳鄂、東川大水,害稼。六年二月,蘇、湖二州大水。六月,徐州大雨,壞民居九百餘家。七年秋,浙西及揚、楚、舒、廬、壽、滁、和、宣等州大水,害稼。八年秋,江西及襄州水,害稼;蘄州湖水溢;滁州大水,溺萬餘戶。



 開成元年夏,鳳翔麟游縣暴雨,水,毀九成宮,壞民舍數百家,死者百餘人。七月,鎮州滹沱河溢,害稼。三年夏,河決,浸鄭、滑外城;陳、許、鄜、坊、鄂、曹、濮、襄、魏、博等州大水;江、漢漲溢,壞房、均、荊、襄等州民居及田產殆盡;蘇、湖、處等州水溢入城,處州平地八尺。四年秋,西川、滄景、淄青大雨,水,害稼及民廬舍,德州尤甚,平地水深八尺。五年七月,鎮州及江南水。



 會昌元年七月,江南大水,漢水壞襄、均等州民居甚眾。



 大中十二年八月,魏、博、幽、鎮、兗、鄆、滑、汴、宋、舒、壽、和、潤等州水,害稼;徐、泗等州水深五丈,漂沒數萬家。十三年夏,大水。



 咸通元年,潁州大水。四年閏六月,東都暴水,自龍門毀定鼎、長夏等門,漂溺居人。七月,東都許、汝、徐、泗等州大水,傷稼。九月,孝義山水深三丈,破武牢關金城門汜水橋,六年六月,東都大水,漂壞十二坊,溺死者甚眾。七年夏,江淮大水。秋,河南大水,害稼。十四年八月,關東、河南大水。



 乾符三年,關東大水。



 光化三年九月,浙江溢,壞民居甚眾。



 乾寧三年四月,河圮于滑州,硃全忠決其堤,因為二河,散漫千餘里。



 △常寒



 顯慶四年二月壬子,大雨雪。方春,少陽用事,而寒氣脅之,古占以為人君刑法暴濫之象。近常寒也。



 咸亨元年十月癸酉,大雪,平地三尺,人多凍死。



 儀鳳三年五月丙寅,高宗在九成宮,霖雨,大寒,兵衛有凍死者。



 開耀元年冬,大寒。



 久視元年三月,大雪。



 神龍元年三月乙酉,睦州暴寒且冰。



 開元二十九年九月丁卯,大雨雪,大木偃折。



 大歷四年六月伏日,寒。



 貞元元年正月戊戌,大風雪,寒;丙午,又大風雪,寒,民饑,多凍死者。十二年十二月,大雪甚寒,竹柏柿樹多死。占曰;「有德遭險,厥災暴寒。」十九年三月,大雪。二十年二月庚戌,始雷,大雨雹,震電,大雨雪。既雷則不當雪,陰脅陽也,如魯隱公之九年。



 元和六年十二月,大寒。八年十月,東都大寒,霜厚數寸,雀鼠多死。十二年九月己丑,雨雪,人有凍死者。十五年八月己卯,同州雨雪,害稼。



 長慶元年二月,海州海水冰,南北二百里,東望無際。



 大和六年正月,雨雪逾月,寒甚。九年十二月,京師苦寒。



 會昌三年春,寒,大雪,江左尤甚,民有凍死者。



 咸通五年冬,隰、石、汾等州大雨雪,平地深三尺。



 景福二年二月辛巳,曹州大雪,平地二尺。



 天復三年三月,浙西大雪,平地三尺餘,其氣如煙,其味苦。十二月,又大雪,江海冰。



 天祐元年九月壬戌朔,大風,寒如仲冬。是冬,浙東、浙西大雪。吳、越地氣常燠而積雪,近常寒也。



 △鼓妖



 武德三年二月丁丑,京師西南有聲如崩山。近鼓妖也。說者以為人君不聰,為眾所惑,則有聲無形,不知所從生。



 天授元年九月,檢校內史宗秦客拜日,無雲而雷震。近鼓妖也。



 貞元十三年六月丙寅,天晦,街鼓不鳴。



 中和二年十月,西北方無雲而雷。



 天復三年十月甲午,有大聲出於宣武節度使廳事。近鼓妖也。



 △魚孽



 如意中,濟源路敬淳家水碾柱將壞,易之為薪,中有占魚長尺餘,猶生。近魚孽也。



 開元四年,安南都護府江中有大蛇,首尾橫出兩岸,經日而腐,寸寸自斷。數日,江魚盡死,蔽江而下,十十五五相附著,江水臭。



 神龍中,渭水有蝦蟆大如鼎,里人聚觀,數日而失。是歲大水。



 元和十四年二月,晝,有魚長尺餘,墜於鄆州市,良久乃死。魚失水而墜於市,敗滅象也。



 開成二年三月壬申,有大魚長六丈,自海入淮,至濠州招義,民殺之。近魚孽也。



 乾符六年,汜水河魚逆流而上,至垣曲、平陸界。魚,民象,逆流而上,民不從君令也。



 光啟二年,揚州雨魚。占如元和十四年。



 △蝗



 武德六年,夏州蝗。蝗之殘民,若無功而祿者然,皆貪撓之所生。先儒以為人主失禮煩苛則旱,魚螺變為蟲蝗,故以屬魚孽。



 貞觀二年六月,京畿旱、蝗。太宗在苑中掇蝗祝之曰:「人以穀為命,百姓有過,在予一人,但當蝕我,無害百姓。」將吞之,侍臣懼帝致疾,遽以為諫。帝曰;「所冀移災朕躬,何疾之避?」遂吞之。是歲,蝗不為災。三年五月,徐州蝗。秋,德、戴、廓等州蝗。四年秋,觀、兗、遼等州蝗。二十一年秋,渠、泉二州蝗。



 永徽元年,夔、絳、雍、同等州蝗。



 永淳元年三月,京畿蝗,無麥苗。六月,雍、岐、隴等州蝗。



 長壽二年,臺、建等州蝗。



 開元三年七月,河南、河北蝗。四年夏,山東蝗,蝕稼,聲如風雨。二十五年,貝州蝗,有白鳥數千萬,群飛食之,一夕而盡,禾稼不傷。



 廣德二年秋,蝗,關輔尤甚,米斗千錢。



 興元元年秋,螟蝗自山而東際於海,晦天蔽野,草木葉皆盡。



 貞元元年夏,蝗,東自海,西盡河、隴,群飛蔽天,旬日不息,所至草木葉及畜毛靡有孑遺,餓殣枕道,民蒸蝗,曝,揚去翅足而食之。



 永貞元年秋,陳州蝗。



 元和元年夏,鎮、冀等州蝗。



 長慶三年秋,洪州螟蝗害稼八萬頃。



 開成元年夏,鎮州、河中蝗,害稼。二年六月,魏博、昭義、淄青、滄州、兗海、河南蝗。三年秋,河南、河北鎮定等州蝗,草木葉皆盡。五年夏,幽、魏、博、鄆、曹、濮、滄、齊、德、淄、青、兗、海、河陽、淮南、虢、陳、許、汝等州螟蝗害稼。占曰:「國多邪人,朝無忠臣,居位食祿,如蟲與民爭食,故比年蟲蝗。」



 會昌元年七月,關東、山南鄧唐等州蝗。



 大中八年七月,劍南東川蝗。



 咸通三年六月,淮南、河南蝗。六年八月,東都、同華陜虢等州蝗。七年夏,東都、同、華、陜、虢及京畿蝗。九年,江淮、關內及東都蝗。十年夏,陜、虢等州蝗。不絀無德,虐取於民之罰。



 乾符二年,蝗自東而西蔽天。



 光啟元年秋,蝗自東方來,群飛蔽天。二年,荊、襄蝗、米斗錢三千,人相食;淮南蝗,自西來,行而不飛,浮水緣城入揚州府署,竹樹幢節,一夕如翦,幡幟畫像,皆嚙去其首,撲不能止。旬日,自相食盡。



 △豕禍



 貞觀十七年六月,司農寺豕生子,一首八足,自頸分為二。



 貞元四年二月,京師民家有豕生子,兩首四足。首多者,上不一也。是歲,宣州大雨震雷,有物墮地如豬,手足各兩指,執赤班蛇食之。頃之,雲合不復見。近豕禍也。



 元和八年四月,長安西市有豕生子,三耳八足,自尾分為二。足多者,下不一也。



 咸通七年,徐州蕭縣民家豕出溷舞,又牡豕多將鄰里群豕而行,復自相噬嚙。



 乾符六年,越州山陰民家有豕入室內,壞器用,銜桉缶置於水次。



 廣明元年,絳州稷山縣民一豕生如人狀,無眉目耳發。占為邑有亂。



 △雷電



 貞觀十一年四月甲子,震乾元殿前槐樹。震耀,天之威怒,以象殺戮;槐,古者三公所樹也。



 證聖元年正月丁酉,雷。雷者陽聲,出非其時,臣竊君柄之象。



 長安四年五月丁亥,震雷,大風拔木,人有震死者。



 延和元年六月,河南偃師縣李材村有震電入民家,地震裂,闊丈餘,長十五里,深不可測,所裂處井廁相通,或沖塚墓,柩出平地無損。李,國姓也;震電,威刑之象;地,陰類也。



 永泰元年二月甲子夜,震霆。自是無雷,至六月甲申乃雷。



 大歷十年四月甲申,雷電,暴風拔木飄瓦,人有震死者,京畿害稼者七縣。



 建中元年九月己卯。雷。四年四月丙子,東都畿汝節度使哥舒曜攻李希烈,進軍至潁橋,大雨震電,人不能言者十三四,馬驢多死。



 貞元十四年五月己酉夏至,始雷。



 元和十一年冬,雷。



 長慶二年六月乙丑,大風震電,落太廟鴟尾,破御史臺樹。



 大和八年三月辛酉,定陵臺大雨,震,廡下地裂二十有六步。占曰;「士庶分離,大臣專恣,不救大敗。」



 會昌三年五月甲午,始雷。



 咸通四年十二月,震雷。



 乾符二年十二月,震雷,雨雹。



 乾寧四年,李茂貞遣將符道昭攻成都,至廣漢,震雷,有石隕於帳前。



 △霜



 貞觀元年秋,霜殺稼。京房《易傳》曰:「人君刑罰妄行,則天應之以隕霜。」三年,北邊霜殺稼。



 永微二年,綏、延等州霜殺稼。



 調露元年八月,邠、涇、寧、慶、原五州霜。



 證聖元年六月,睦州隕霜,殺草。吳、越地燠而盛夏隕霜,昔所未有。四年四月,延州霜,殺草。四月純陽用事,象人君當布惠於天下,而反隕霜,是無陽也。



 開元十二年八月,潞,綏等州霜殺稼。十五年,天下州十七霜殺稼。



 元和二年七月,邠、寧等州霜殺稼。九年三月丁卯,隕霜,殺桑。十四年四月,淄、青隕霜,殺惡草及荊棘,而不害嘉穀。



 寶歷元年八月,邠州霜殺稼。



 大和三年秋,京畿奉先等八縣早霜,殺稼。



 大中三年春,隕霜,殺桑。



 中和元年春,霜。秋,河東早霜,殺稼。



 △雹



 貞觀四年秋,丹、延、北永等州雹。



 顯慶二年五月,滄州大雨雹,中人有死者。



 咸亨元年四月庚午,雍州大雨雹。二年四月戊子,大雨雹,震電,大風折木,落則天門鴟尾三。先儒以為「雹者,陰脅陽也」。又曰:「人君惡聞其過,抑賢用邪,則雹與雨俱;信讒殺無罪,則雹下毀瓦、破車、殺牛馬。」



 永淳元年五月壬寅,定州大雨雹,害麥、禾及桑。



 天授二年六月庚戌,許州大雨雹。



 證聖元年二月癸卯,滑州大雨雹,殺燕雀。



 神功元年,媯、綏二州雹。



 聖歷元年六月甲午,曹州大雨雹。



 久視元年六月丁亥,曹州大雨雹。



 長安三年八月,京師大雨雹,人畜有凍死者。



 神龍元年四月壬子,雍州同官縣大雨雹,殺鳥獸。



 景龍元年四月己巳,曹州大雨雹。二年正月己卯,滄州雨雹如雞卵。



 開元八年十二月丁未,滑州大雨雹。二十二年五月戊辰,京畿渭南等六縣大風雹,傷麥。



 大歷七年五月乙酉,雨雹。



 貞元二年六月丙子,大雨雹。十七年二月丁酉,雨雹;己亥,霜;戊申夜,震霆,雨雹;庚戌,大雨雪而雹。五月戊寅,好畤縣風雹,害麥。十八年七月癸酉,大雨雹。



 元和元年,鄜、坊等州雹。十年秋,鄜、坊等州風雹,害稼。十二年夏,河南雨雹,中人有死者。十五年三月,京畿興平、醴泉等縣雹,傷麥。



 長慶四年六月庚寅,京師雨雹如彈丸。



 大和四年秋,鄜、坊等州雹。五年夏,京畿奉先、渭南等縣雨雹。



 開成二年秋,河南雹,害稼。四年七月,鄭、滑等州風雹。五年六月,濮州雨雹如拳,殺人三十六,牛馬甚眾。



 會昌元年秋,登州雨雹,文登尤甚,破瓦害稼。四年夏,雨雹如彈丸。



 乾符六年五月丁酉,宣授宰臣豆盧彖、崔沆制,殿庭氛霧四塞,及百官班賀於政事堂,雨雹如鳧卵,大風雷雨拔木。



 廣明元年四月甲申朔,汝州大雨風,拔街衢樹十二三;東都有雲起西北,大風隨之,長夏門內表道古槐樹自拔者十五六,宮殿鴟尾皆落,雨雹大如杯,鳥獸殪於川澤。



 △黑眚黑祥



 大歷二年十二月戊戌,黑氣如塵,彌漫於北方。黑氣,陰沴也。



 貞元四年七月,自陜至河陰,河水黑,流入汴,至汴州城下,一宿而復。近黑祥也。占曰;「法嚴刑酷,傷水性也。五行變節,陰陽相干,氣色繆亂,皆敗亂之象。」十四年,潤州有黑氣如堤,自海門山橫亙江中,與北固山相峙,又有白氣如虹,自金山出,與黑氣交,將旦而沒。



 大和四年正月壬寅,黑氣如帶,東西際天。



 咸通十四年七月,僖宗即位,是日,黑氣如盤,自天屬含元殿庭。



 △火沴水



 武德九年二月,蒲州河清。襄楷以為:「河,諸侯象;清,陽明之效也。」



 貞觀十四年二月,陜州、泰州河清。十六年正月,懷州河清。十七年十二月,鄭州、滑州河清。二十三年四月,靈州河清。



 永徽元年正月,濟州河清。二年十二月,衛州河清。五年六月,濟州河清十六里。



 調露二年夏,豐州河清。



 長安初,醴泉坊太平公主第井水溢流。又並州文水縣猷水竭,武氏井溢。



 神龍二年三月壬子,洛陽城東七里,地色如水,樹木車馬,歷歷見影,漸移至都,月餘乃滅。長安街中,往往見水影。昔苻堅之將死也,長安嘗有是。



 景龍四年三月庚申,京師井水溢。占曰:「君兇」。又曰:「兵將起。」



 開元二十二年八月,清夷軍黃帝祠古井湧浪。二十五年五月,淄州、棣州河清。二十九年,亳州老子祠九井涸復湧。



 乾元二年七月,嵐州合河、關河三十里清如井水,四日而變。



 寶應元年九月甲午,太州至陜州二百餘里河清,澄澈見底。



 大歷末,深州束鹿縣中有水影長七八尺,遙望見人馬往來,如在水中,及至前,則不見水。



 建中四年五月乙巳,滑州、濮州河清。



 貞元十四年閏五月乙丑,滑州河清。二十一年夏,越州鏡湖竭。是歲,朗州熊、武五溪水斗。占曰:「山崩川竭,國必亡。」又曰:「方伯力政,厥異水斗。」



 開成二年夏,旱,揚州運河竭。



 大中八年正月,陜州河清。



 咸通八年七月,泗州下邳雨湯,殺鳥雀。水沸於火,則可以傷物,近火沴水也。雨者,自上而降;鳥雀,民象。



 中和三年秋,汴水入於淮水,斗,壞船數艘。



 廣明元年夏,汝州峴陽峰龍池涸。近川竭也。



 《五行傳》曰:「皇之不極,是謂不建,厥咎眊,厥罰常陰,厥極弱。時則有射妖,時則有龍蛇之孽,時則有馬禍,時則有下人伐上之痾,時則有日月亂行,星辰逆行。」謂木金火水土沴天也。



 △常陰



 長安四年,自九月霖雨陰晦,至於神龍元年正月。



 貞元二十一年秋,連月陰霪。



 元和十五年正月庚辰至於丙申,晝常陰晦,微雨雪,夜則晴霽。占曰:「晝霧夜晴,臣志得申。」



 咸通十四年七月,靈州陰晦。



 乾符六年秋,多雲霧晦冥,自旦及禺中乃解。



 光啟元年秋,河東大雲霧。明年夏,晝陰積六十日。二年十一月,淮南陰晦雨雪,至明年二月不解。



 景福二年夏,連陰四十餘日。



 △霧



 長壽元年九月戊戌,黃霧四塞。霧者,百邪之氣,為陰冒陽,本於地而應於天;黃為土,土為中宮。



 神龍二年三月乙巳,黃霧四塞。



 景龍二年八月甲戌,黃霧昏濁不雨。二年正月丁卯,黃霧四塞。十一月甲寅,日入後,昏霧四塞,經二日乃止。占曰:「霧連日不解,其國昏亂。」



 開元五年正月戊辰,昏霧四塞。



 天寶十四載冬三月,常霧起昏暗,十步外不見人,是謂晝昏。占曰:「有破國。」



 至德二載四月,賊將武令珣圍南陽,白霧四塞。



 上元元年閏四月,大霧。占曰:「兵起。」



 貞元十年三月乙亥,黃霧四塞,日無光。



 咸通九年十一月,龐勛圍徐州,甲辰,大霧昏塞,至於丙午。



 光化四年冬,昭宗在東內,武德門內煙霧四塞,門外日色皎然。



 △虹蜺



 武德初,隋將堯君素守蒲州,有白虹下城中。



 唐隆元年六月戊子,虹蜺亙天。蜺者,斗之精。占曰:「后妃陰脅王者。」又曰:「五色迭至,照於宮殿,有兵。」



 延和元年六月,幽州都督孫佺帥兵襲奚,將入賊境,有白虹垂頭於軍門。占曰;「其下流血。」



 至德二載正月丙子,南陽夜有白虹四,上亙百餘丈。



 元和十三年十二月丙辰,有白虹闊五尺,東西亙天。



 會昌四年正月己酉,西方有白虹。



 咸通元年七月己酉朔,白虹橫亙西方。九年七月戊戌,白虹橫亙西方。



 光啟二年九月,白虹見西方。十月壬辰夜,又如之。



 天復三年三月庚申,有曲虹在日東北。



 △龍蛇孽



 貞觀八年七月,隴右大蛇屢見。蛇,女子之祥;大者,有所象也。又汾州青龍見,吐物在空中,光明如火,墮地地陷,掘之得玄金,廣尺,長七寸。



 顯慶二年五月庚寅,有五龍見於岐州之皇后泉。



 先天二年六月,京師朝堂磚下有大蛇出,長丈餘,有大蝦蟆如盤,而目赤如火,相與鬥,俄而蛇入於大樹,蝦蟆入於草。蛇、蝦蟆,皆陰類;朝堂出,非其所也。



 開元四年六月,郴州馬嶺山下有白蛇與黑蛇鬥,白蛇長六七尺,吞黑蛇,至腹,口眼血流,黑蛇長丈餘,頭穿白蛇腹出,俱死。



 天寶中,洛陽有巨蛇,高丈餘,長百尺,出芒山下,胡僧無畏見之曰:「此欲決水瀦洛城。」即以天竺法咒之,數日蛇死。十四載七月,有二龍鬥於南陽城西。《易坤》:「上六,龍戰於野。」《文言》曰:「陰疑於陽必戰。」



 至德元載八月朔,成都丈人廟有肉角蛇見。二載三月,有蛇鬥於南陽門之外,一蛇死,一蛇上城。



 建中二年夏,趙州寧晉縣沙河北,有棠樹甚茂,民祠之為神。有蛇數百千自東西來,趨北岸者聚棠樹下,為二積,留南岸者為一積,俄有徑寸龜三,繞行,積蛇盡死,而後各登其積。野人以告。蛇腹皆有瘡,若矢所中。刺史康日知圖其事,奉三龜來獻。四年九月戊寅,有龍見於汝州城壕。龍,大人象,其潛也淵,其飛也天;城壕,失其所也。



 貞元末,資州得龍丈餘,西川節度使韋皋匣而獻之,百姓縱觀,三日,為煙所薰而死。



 大和二年六月丁丑,西北有龍鬥。三年,成都門外有龍與牛斗。



 開成元年,宮中有眾蛇相與鬥。



 光化三年九月,杭州有龍鬥於浙江,水溢,壞民廬舍。占同天寶十四載。



 光啟二年冬,鄜州洛交有蛇見於縣署,復見於州署。蛇,冬則蟄,《易》曰;「龍蛇之蟄,以存身也。」



 △馬禍



 義寧二年五月戊申,有馬生角,長二寸,末有肉。角者,兵象。



 武德三年十月,王世充偽左僕射韋霽馬生角,當項。



 永隆二年,監牧馬大死,凡十八萬疋。馬者,國之武備,天去其備,國將危亡。



 文明初,新豐有馬生駒,二首同項,各有口鼻,生而死;又咸陽牝馬生石,大如升,上微有綠毛。皆馬禍也。



 開元十二年五月,太原獻異馬駒,兩肋各十六,肉尾無毛。二十五年,濮州有馬生駒,肉角。二十九年三月,滑州刺史李邕獻馬,肉鬃鱗臆,嘶不類馬,日行三百里。



 建中四年五月,滑州馬生角。



 大和九年八月,易定馬飲水,因吐珠一,以獻。



 開成元年六月,揚州民明齊家馬生角,長一寸三分。



 會昌元年四月,桂州有馬生駒,三足,能隨群於牧。



 咸通三年,郴州馬生角。十一年,沁州綿上及和川牡馬生子,皆死。京房《易傳》曰:「方伯分威,厥妖牡馬生子。



 乾符二年,河北馬生人。



 中和元年九月,長安馬生人,京房《易傳》曰:「諸侯相伐,厥妖馬生人。」一曰;「人流亡。」二年二月,蘇州嘉興馬生角。



 光啟二年夏四月,僖宗在鳳翔,馬尾皆吒蓬如篲。吒,怒象。



 文德元年,李克用獻馬二,肘膝皆有鬣,長五寸許,蹄大如七寸甌。



 △人痾



 武德四年,太原尼志覺死,十日而蘇。



 貞觀十九年,衛州人劉道安頭生肉角,隱見不常,因以惑眾,伏誅。角,兵象;肉,不可以觸者。永徽六年,淄州高苑民吳威妻、嘉州民辛道護妻皆一產四男。凡物反常則為妖,亦陰氣盛則母道壯也。



 顯慶三年,普州有人化為虎。虎,猛噬而不仁。



 儀鳳三年四月,涇州獻二小兒,連心異體。初,鶉觚縣衛士胡萬年妻吳生一男一女,其胸相連,餘各異體,乃析之,則皆死;又產,復然,俱男也,遂育之,至是四歲,以獻於朝。



 永隆元年,長安獲女魃,長尺有二寸,其狀怪異。《詩》曰:「旱魃為虐,如炎如焚。」是歲秋,不雨,至於明年正月。



 永隆二年九月,萬年縣女子劉凝靜衣白衣,從者數人,升太史令廳,問比有何災異。令執之以聞。是夜,彗星見。太史司天文、歷候,王者所以奉若天道、恭授民時者,非女子所當問。



 載初中,涪州民範端化為虎。



 神功元年一月庚子,有人走入端門,又入則天門,至通天宮,閽及仗衛不之覺。時來俊臣婢產肉塊如二升器,剖之有赤蟲,須臾化為蜂,螫人而去。



 久視二年正月,成州有大人跡見。



 長安中,郴州佐史因病化為虎,欲食其嫂,擒之,乃人也,雖未全化,而虎毛生矣。



 太極元年,狂人段萬謙潛入承天門,登太極殿,升御床,自稱天子,且言:「我李安國也,人相我年三十二當為天子。」



 開元二十三年四月,冀州獻長人李家寵,八尺有五寸。



 大歷十年二月,昭應婦人張產一男二女。



 貞元八年正月丁亥,許州人李狗兒持仗上含元殿擊欄檻,伏誅。十年四月,恆州有巨人跡見。十五年正月戊申,狂人劉忠詣銀臺,稱白起令上表,天下有火災。十七年十一月,翰林待詔戴少平死十有六日而蘇。是歲,宣州南陵縣丞李嶷死,已殯三十日而蘇。



 元和二年,商州洪崖冶役夫將化為虎,眾以水沃之,不果化。



 長慶四年三月,民徐忠信潛入浴堂門。



 寶歷二年十二月,延州人賀文妻一產四男。



 大和二年十月,狂人劉德廣入含元殿。



 咸通七年,渭州有人生角寸許。占曰:「天下有兵。」十三年四月,太原晉陽民家有嬰兒,兩頭異頸,四手聯足。此天下不一之妖。是歲,民皇甫及年十四,暴長七尺餘,長啜大嚼,三倍如初,歲餘死。



 乾符六年秋,蜀郡婦人尹生子首如豕,目在脽下。占曰:「君失道。」



 光啟元年,隰州溫泉民家有死者,既葬且半月,行人聞聲呼地下,其家發之,則復生,歲餘乃死。二年春,鳳翔郿縣女子未枌化為丈夫,旬日而死。京房《易傳》曰:「茲謂陰昌,賊人為王。」



 大順元年六月,資州兵王全義妻如孕,覺物漸下入股,至足大拇,痛甚,坼而生珠如彈丸,漸長大如杯。



 天祐二年五月,潁州汝陰民彭文妻一產三男。



 △疫



 貞觀十年,關內、河東大疫。十五年三月,澤州疫。十六年夏,穀、涇、徐、戴、虢五州疫。十七年夏,潭、濠、廬三州疫。十八年,廬、濠、巴、普、郴五州疫。二十二年,卿州大役永徽六年三月,楚州大役。



 永淳元年冬,大疫,兩京死者相枕於路。占曰:「國將有恤,則邪亂之氣先被於民,故疫。」



 景龍元年夏,自京師至山東、河北疫,死者千數。



 寶應元年,江東大疫,死者過半。



 貞元六年夏,淮南,浙西、福建道疫。



 元和元年夏,浙東大疫,死者太半。



 大和六年春,自劍南至浙西大疫。



 開成五年夏,福、建、臺、明四州疫。



 咸通十年,宣、歙、兩浙疫。



 大順二年春,淮南疫,死者十三四。



 △天鳴



 天寶十四載五月,天鳴,聲若雷。占曰:「人君有憂。」



 貞元二十一年八月,天鳴,在西北。



 中和三年三月,浙西天鳴,聲如轉磨。無雲而雨。



 元和十二年正月乙酉,星見而雨。占曰:「無雲而雨,是謂天泣。」



 △隕石



 永徽四年八月己亥,隕石於同州馮翊十八,光耀,有聲如雷。近星隕而化也。庶民惟星,在上而隕,民去其上之象。一曰:「人君為詐妄所蔽則然。」



\end{pinyinscope}