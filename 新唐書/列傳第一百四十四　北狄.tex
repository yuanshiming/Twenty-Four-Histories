\article{列傳第一百四十四 北狄}

\begin{pinyinscope}

 契丹,本東胡種,其先為匈奴所破,保鮮卑山。魏青龍中,部酋比能稍桀驁,為幽州刺史王雄所殺,眾遂微,逃潢水之南,黃龍之北。至元魏,自號曰契丹。地直京師東北五千里而贏,東距高麗,西奚,南營州,北靺鞨、室韋,阻冷陘山以自固。射獵居處無常。其君大賀氏,有勝兵四萬,析八部,臣於突厥,以為俟斤。凡調發攻戰,則諸部畢會;獵則部得自行。與奚不平,每斗不利,輒遁保鮮卑山。風俗與突厥大抵略侔。死不墓,以馬車載尸入山,置於樹顛。子孫死,父母旦夕哭;父母死則否,亦無喪期。



 武德中,其大酋孫敖曹與靺鞨長突地稽俱遣人來朝,而君長或小入寇邊。後二年,君長乃遣使者上名馬、豐貂。貞觀二年,摩會來降。突厥頡利可汗不欲外夷與唐合,乃請以梁師都易契丹。太宗曰:「契丹、突厥不同類,今已降我,尚可索邪?師都,唐編戶,盜我州部,突厥輒為助,我將禽之,誼不可易降者。」明年,摩會復入朝,賜鼓纛,由是有常貢。帝伐高麗,悉發酋長與奚首領從軍。帝還,過營州,盡召其長窟哥及老人,差賜繒採,以窟哥為左武衛將軍。



 大酋辱紇主曲據又率眾歸,即其部為玄州,拜曲據刺史,隸營州都督府。未幾,窟哥舉部內屬,乃置松漠都督府,以窟哥為使持節十州諸軍事、松漠都督,封無極男,賜氏李。以達稽部為峭落州,紇便部為彈汗州,獨活部為無逢州,芬問部為羽陵州,突便部為日連州,芮奚部為徒河州,墜斤部為萬丹州,伏部為匹黎、赤山二州,俱隸松漠府,即以辱紇主為之刺史。



 窟哥死,與奚連叛,行軍總管阿史德樞賓等執松漠都督阿卜固獻東都。窟哥有二孫:曰枯莫離,為左衛將軍、彈汗州刺史,封歸順郡王;曰盡忠,為武衛大將軍、松漠都督。而敖曹有孫曰萬榮,為歸誠州刺史。於是營州都督趙文翽驕沓,數侵侮其下,盡忠等皆怨望。萬榮本以侍子入朝,知中國險易,挾亂不疑,即共舉兵,殺文翽,盜營州反。盡忠自號無上可汗,以萬榮為將,縱兵四略,所向輒下,不重浹,眾數萬,妄言十萬,攻崇州,執討擊副使許欽寂。武后怒,詔鷹揚將軍曹仁師、金吾大將軍張玄遇、右武威大將軍李多祚、司農少卿麻仁節等二十八將擊之;以梁王武三思為榆關道安撫大使,納言姚為之副。更號萬榮曰萬斬,盡忠曰盡滅。諸將戰西硤石黃麞谷,王師敗績,玄遇、仁節皆為虜禽。進攻平州,不克。敗書聞,後乃以右武衛大將軍建安王武攸宜為清邊道大總管,擊契丹;募天下人奴有勇者,官畀主直,悉發以擊虜。萬榮銜枚夜襲檀州,清邊道副總管張九節募死士數百薄戰,萬榮敗而走山。俄而盡忠死,突厥默啜襲破其部。萬榮收散兵復振,使別將駱務整、何阿小入冀州,殺刺史陸寶積,掠數千人。



 武后聞盡忠死,更詔夏官尚書王孝傑、羽林衛將軍蘇宏暉率兵十七萬討契丹,戰東硤石,師敗,孝傑死之。萬榮席已勝,遂屠幽州。攸宜遣將討捕,不能克。乃命右金吾衛大將軍河內郡王武懿宗為神兵道大總管,右肅政臺御史大夫婁師德為清邊道大總管,右武威衛大將軍沙吒忠義為清邊中道前軍總管,兵凡二十萬擊賊。萬榮銳甚,鼓而南,殘瀛州屬縣,恣肆無所憚。於是神兵道總管楊玄基率奚軍掩其尾,契丹大敗,獲何阿小,降別將李楷固、駱務整,收仗械如積。萬榮委軍走,殘隊復合,與奚搏。奚四面攻,乃大潰,萬榮左馳。張九節為三伏伺之,萬榮窮,與家奴輕騎走潞河東,憊甚,臥林下,奴斬其首,九節傳之東都,餘眾潰。攸宜凱而還,後喜,為赦天下,改元為神功。



 契丹不能立,遂附突厥。久視元年,詔左玉鈐衛大將軍李楷固、右武威衛將軍駱務整討契丹,破之。此兩人皆虜善將,嘗犯邊,數窘官軍者也,及是有功。



 開元二年,盡忠從父弟都督失活以默啜政衰,率部落與頡利發伊健啜來歸,玄宗賜丹書鐵券。後二年,與奚長李大酺皆來,詔復置松漠府,以失活為都督,封松漠郡王,授左金吾衛大將軍。仍其府置靜析軍,以失活為經略大使,所統八部皆擢其酋為刺史。詔將軍薛泰為押蕃落使,督軍鎮撫。帝以東平王外孫楊元嗣女為永樂公主,妻失活。明年,失活死,贈特進,帝遣使吊祠,以其弟中郎將娑固襲封及所領。明年,娑固與公主來朝,宴齎有加。



 有可突於者,為靜析軍副使,悍勇得眾,娑固欲去之,未決。可突於反攻娑固,娑固奔營州。都督許欽澹以州甲五百,合奚君長李大酺兵共攻可突於。不勝,娑固、大酺皆死。欽澹懼,徙軍入榆關。可突於奉娑固從父弟鬱於為君,遣使者謝罪。有詔即拜鬱於松漠郡王,而赦可突於。鬱於來朝,授率更令,以宗室所出女慕容為燕郡公主妻之。可突於亦來朝,擢左羽林衛將軍。鬱於死,弟吐於嗣,與可突於有隙,不能定其下,攜公主來奔,封遼陽郡王,留宿衛。可突於奉盡忠弟邵固統眾,詔許襲王。天子封禪,邵固與諸蕃長皆從行在。明年,拜左羽林衛大將軍,徙王廣化郡,以宗室出女陳為東華公主,妻邵固,詔官其部酋長百餘人,邵固以子入侍。



 可突於復來,不為宰相李元紘所禮,鞅鞅去。張說曰:「彼獸心者,唯利是向。且方持國,下所附也,不假以禮,不來矣。」後三年,可突於殺邵固,立屈烈為王,脅奚眾共降突厥。公主走平廬軍。詔幽州長史、知範陽節度事趙含章擊之。遣中書舍人裴寬、給事中薛偘大募壯士,拜忠王浚河北道行軍元帥,以御史大夫李朝隱、京兆尹裴伷先副之,帥程伯獻、張文儼、宋之悌、李東蒙、趙萬功、郭英傑等八總管兵擊契丹。既又以忠王兼河東道諸軍元帥,王不行。以禮部尚書信安郡王禕持節河北道行軍副元帥,與含章出塞捕虜,大破之。可突於走,奚眾降,王以二蕃俘級告諸廟。



 明年,可突於盜邊,幽州長史薛楚玉、副總管郭英傑、吳克勤、烏知義、羅守忠率萬騎及奚擊之,戰都山下。可突於以突厥兵來,奚懼,持兩端,眾走險;知義、守忠敗,英傑、克勤死之,殺唐兵萬人。帝擢張守珪為幽州長史經略之。守珪既善將,可突於恐,陽請臣而稍趨西北倚突厥。其衙官李過折與可突於內不平,守珪使客王悔陰邀之,以兵圍可突於,過折即夜斬可突於、屈烈及支黨數十人,自歸。守珪使過折統其部,函可突於等首傳東都。拜過折北平郡王,為松漠都督。可突於殘黨擊殺過折,屠其家。一子剌乾走安東,拜左驍衛將軍。二十五年,守珪討契丹,再破之,有詔自今戰有功必告廟。



 天寶四載,契丹大酋李懷秀降,拜松漠都督,封崇順王,以宗室出女獨孤為靜樂公主妻之。是歲,殺公主叛去,範陽節度使安祿山討破之。更封其酋楷落為恭仁王,代松漠都督。祿山方幸,表討契丹以向帝意。發幽州、雲中、平廬、河東兵十餘萬,以奚為鄉導,大戰潢水南。祿山敗,死者數千。自是祿山與相侵掠未嘗解,至其反乃已。



 契丹在開元、天寶間,使朝獻者無慮二十。故事,以範陽節度為押奚、契丹使,自至德後,籓鎮擅地務自安,鄣戍斥候益謹,不生事於邊;奚、契丹亦鮮入寇,歲選酋豪數十入長安朝會,每引見,賜與有秩,其下率數百皆駐館幽州。至德、寶應時再朝獻,大歷中十三,貞元間三,元和中七,大和、開成間凡四。然天子惡其外附回鶻,不復官爵渠長。會昌二年,回鶻破,契丹酋屈戍始復內附,拜雲麾將軍、守右武衛將軍。於是幽州節度使張仲武為易回鶻所與舊印,賜唐新印,曰「奉國契丹之印」。



 咸通中,其王習爾之再遣使者入朝,部落浸強。習爾之死,族人欽德嗣。光啟時,方天下盜興,北疆多故,乃鈔奚、室韋,小小部種皆役服之,因入寇幽、薊。劉仁恭窮師逾摘星山討之,歲燎塞下草,使不得留牧,馬多死。契丹乃乞盟,獻良馬求牧地,仁恭許之。復敗約入寇,劉守光戍平州,契丹以萬騎入,守光偽與和,帳飲具於野,伏發,禽其大將。群胡慟,願納馬五千以贖,不許。欽德輸重賂求之,乃與盟,十年不敢近邊。



 欽德晚節政不競,其八部大人法常三歲代,時耶律阿保機建鼓旗為一部,不肯代,自號為王而有國,大賀氏遂亡。



 奚,亦東胡種,為匈奴所破,保烏丸山。漢曹操斬其帥蹋頓,蓋其後也。元魏時自號庫真奚,居鮮卑故地,直京師東北四千里。其地東北接契丹,西突厥,南白狼河,北霫。與突厥同俗,逐水草畜牧,居氈廬,環車為營。其君長常以五百人持兵衛牙中,餘部散山谷間,無賦入,以射獵為貲。稼多穄,已獲,窖山下。斷木為臼,瓦鼎為飦,雜寒水而食。喜戰鬥,兵有五部,部一俟斤主之。其國西抵大洛泊,距回紇牙三千里,多依土護真水。其馬善登,其羊黑。盛夏必徙保冷陘山,山直媯州西北。至隋始去「庫真」,但曰奚。



 武德中,高開道借其兵再寇幽州,長史王詵擊破之。太宗貞觀三年始來朝,閱十七歲,凡四朝貢。帝伐高麗,大酋蘇支從戰有功。不數年,其長可度者內附,帝為置饒樂都督府,拜可度者使持節六州諸軍事、饒樂都督,封樓煩縣公,賜李氏。以阿會部為弱水州,處和部為祁黎州,奧失部為洛瑰州,度稽部為太魯州,元俟折部為渴野州,各以酋領辱紇主為刺史,隸饒樂府。復置東夷都護府於營州,兼統松漠、饒樂地,置東夷校尉。



 顯慶間可度者死,奚遂叛。五年,以定襄都督阿史德樞賓、左武候將軍延陀梯真、居延州都督李含珠為冷陘道行軍總管。明年,詔尚書右丞崔餘慶持節總護定襄等三都督討之,奚懼乞降,斬其王匹帝。萬歲通天中,契丹反,奚亦叛,與突厥相表裏,號「兩蕃」。延和元年,以左羽林衛大將軍幽州都督孫佺、左驍衛將軍李楷洛、左威衛將軍周以悌帥兵十二萬,為三軍,襲擊其部。次冷陘,前軍楷洛與奚酋李大酺戰不利。佺懼,斂軍,詐大酺曰:「我奉詔來慰撫若等,而楷洛違節度輒戰,非天子意,方戮以徇。」大酺曰:「誠慰撫我,有所賜乎?」佺出軍中繒帛、袍帶與之。大酺謝,請佺還師,舉軍得脫,爭先無部伍,大酺兵躡之,遂大敗,殺傷數萬。佺、以悌皆為虜禽,送默啜害之。朝廷方多故,不暇討。



 玄宗開元二年,使奧蘇悔落丐降,封饒樂郡王,左金吾衛大將軍、饒樂都督。詔宗室出女辛為固安公主,妻大酺。明年,身入朝成昏。始復營州都督府,遣右領軍將軍李濟持節護送。大酺後與契丹可突於斗,死。弟魯蘇領其部,襲王。詔兼保塞軍經略大使。牙官塞默羯謀叛,公主置酒誘殺之,帝嘉其功,賜主累萬。會與其母相告訐得罪,更以盛安公主女韋為東光公主妻之。後三年,封魯蘇奉誠郡王,右羽林衛將軍,擢其首領無慮二百人,皆位郎將。



 久之,契丹可突於反,脅奚眾並附突厥。魯蘇不能制,奔榆關,公主奔平廬。幽州長史趙含章發清夷軍討破之,眾稍自歸。明年,信安王禕降其酋李詩鎖高等部落五千帳,以其地為歸義州,因以王詩,拜左羽林軍大將軍、本州都督,賜帛十萬,置其部幽州之偏。



 李詩死,子延寵嗣,與契丹又叛,為幽州張守珪所困。延寵降,復拜饒樂都督、懷信王,以宗室出女楊為宜芳公主妻之。延寵殺公主復叛,詔立它酋婆固為昭信王、饒樂都督,以定其部。安祿山節度範陽,詭邊功,數與鏖斗,飾俘以獻,誅其君李日越,料所俘驍壯戍雲南。終帝世,凡八朝獻,至德、大歷間十二。



 貞元四年,與室韋攻振武。後七年,幽州殘其眾六萬。德宗時,兩朝獻。元和元年,君梅落身入朝,拜檢校司空、歸誠郡王。以部酋索氏為左威衛將軍、檀薊州游弈兵馬使,沒辱孤平州游弈兵馬使,皆賜李氏。然陰結回鶻、室韋兵犯西城、振武。大抵憲宗世四朝獻。



 大和四年,復盜邊,廬龍李載義破之,執大將二百餘人,縛其帥茹羯來獻,文宗賜冠帶,授右驍衛將軍。後五年,大首領匿舍朗來朝。大中元年,北部諸山奚悉叛,廬龍張仲武禽酋渠,燒帳落二十萬,取其刺史以下面耳三百,羊牛七萬,輜貯五百乘,獻京師。咸通九年,其王突董蘇使大都督薩葛入朝。



 是後契丹方強,奚不敢亢,而舉部役屬。虜政苛,奚怨之,其酋去諸引別部內附,保媯州北山,遂為東、西奚。



 室韋,契丹別種,東胡之北邊,蓋丁零苗裔也。地據黃龍北,傍越河,直京師東北七千里,東黑水靺鞨,西突厥,南契丹,北瀕海。其國無君長,惟大酋,皆號「莫賀咄」,攝筦其部而附於突厥。小或千戶,大數千戶,濱散川谷,逐水草而處,不稅斂。每弋獵即相嘯聚,事畢去,不相臣制,故雖猛悍喜戰,而卒不能為強國。剡木為犁,人挽以耕,田獲甚褊。其氣候多寒,夏霧雨,冬霜霰。其俗,富人以五色珠垂領,婚嫁則男先傭女家三歲,而後分以產,與婦共載,鼓舞而還。夫死,不再嫁。每部共構大棚,死者置尸其上,喪期三年。土少金鐵,率資於高麗。器有角弓、楛矢,人尤善射。每溽夏,西保貣勃、次對二山。山多草木鳥獸,然苦飛蚊,則巢居以避。酋帥死,以子弟繼,無則推豪桀立之。率乘牛車,蘧蒢為室,度水則束薪為桴,或以皮為舟。馬皆草韉、繩羈靮。所居或皮蒙室,或屈木以蘧蒢覆,徙則載而行。其畜無羊少馬,有牛不用,有巨豕食之,韋其皮為服若席。其語言,靺鞨也。


分部凡二十餘:曰嶺西部、山北部、黃頭部,強部也;大如者部、小如者部、婆萵部、訥北部、駱丹部,悉處柳城東北,近者三千,遠六千里而贏;最西有烏素固部,與回紇接,當俱倫泊之西南;自泊而東有移塞沒部;稍東有塞曷支部,最強部也,居啜河之陰,亦曰燕支河;益東有和解部、烏羅護部、那禮部、嶺西部;直北曰訥比支部,北有大山,山外曰大室韋,瀕於室建河,河出俱倫,
 \gezhu{
  辶也}
 而東;河南有蒙瓦部,其北落坦部;水東合那河、忽汗河,又東貫黑水靺鞨,故靺鞨跨水有南北部,而東注于海。越河東南亦與那河合,其北有東室韋,蓋烏丸東南鄙餘人也。



 貞觀五年,始來貢豐貂,後再入朝。長壽二年叛,將軍李多祚擊定之。景龍初,復朝獻,請助討突厥。開元、天寶間,凡十朝獻,大歷中十一。貞元四年,與奚共寇振武,節度使唐朝臣方郊勞天子使者,驚而走軍,室韋執詔使,大殺掠而去。明年,使者來謝。大和中三朝獻。大中中一來。咸通時,大酋怛烈與奚皆遣使至京師,然非顯夷後,史官失傳。



 黑水靺鞨居肅慎地,亦曰挹婁,元魏時曰勿吉。直京師東北六千里,東瀕海,西屬突厥,南高麗,北室韋。離為數十部,酋各自治。其著者曰粟末部,居最南,抵太白山,亦曰徒太山,與高麗接,依粟末水以居,水源於山西,北注它漏河;稍東北曰汨咄部;又次曰安居骨部;益東曰拂涅部;居骨之西北曰黑水部;粟末之東曰白山部。部間遠者三四百里,近二百里。



 白山本臣高麗,王師取平壤,其眾多入唐,汨咄、安居骨等皆奔散,浸微無聞焉,遺人迸入渤海。唯黑水完強,分十六落,以南北稱,蓋其居最北方者也。人勁健,善步戰,常能患它部。俗編發,綴野豕牙,插雉尾為冠飾,自別於諸部。性忍悍,善射獵,無憂戚,貴壯賤老。居無室廬,負山水坎地,梁木其上,覆以土,如丘塚然。夏出隨水草,冬入處。以溺盥面,於夷狄最濁穢。死者埋之,無棺槨,殺所乘馬以祭。其酋曰大莫拂瞞咄,世相承為長。無書契。其矢石鏃,長二寸,蓋楛砮遺法。畜多豕,無牛羊。有車馬,田耦以耕,車則步推。有粟麥。土多貂鼠、白兔、白鷹。有鹽泉,氣蒸薄,鹽凝樹顛。



 武德五年,渠長阿固郎始來。太宗貞觀二年,乃臣附,所獻有常,以其地為燕州。帝伐高麗,其北部反,與高麗合。高惠真等率眾援安市,每戰,靺鞨常居前。帝破安市,執惠真,收靺鞨兵三千餘,悉坑之。



 開元十年,其酋倪屬利稽來朝,玄宗即拜勃利州刺史。於是安東都護薛泰請置黑水府,以部長為都督、刺史,朝廷為置長史監之,賜府都督姓李氏,名曰獻誠,以雲麾將軍領黑水經略使,隸幽州都督。訖帝世,朝獻者十五。大歷世凡七,貞元一來,元和中再。



 初,黑水西北又有思慕部,益北行十日得郡利部,東北行十日得窟說部,亦號屈設,稍東南行十日得莫曳皆部,又有拂涅、虞婁、越喜、鐵利等部。其地南距渤海,北、東際於海,西抵室韋,南北袤二千里,東西千里。拂涅、鐵利、虞婁、越喜時時通中國,而郡利,屈設、莫曳皆不能自通。今存其朝京師者附左方。



 拂涅,亦稱大拂涅,開元、天寶間八來,獻鯨睛、貂鼠、白兔皮;鐵利,開元中六來;越喜,七來,貞元中一來;虞婁,貞觀間再來,貞元一來。後渤海盛,靺鞨皆役屬之,不復與王會矣。



 渤海,本粟末靺鞨附高麗者,姓大氏。高麗滅,率眾保挹婁之東牟山,地直營州東二千里,南比新羅,以泥河為境,東窮海,西契丹。築城郭以居,高麗逋殘稍歸之。



 萬歲通天中,契丹盡忠殺營州都督趙翽反,有舍利乞乞仲象者,與靺鞨酋乞四比羽及高麗餘種東走,度遼水,保太白山之東北,阻奧婁河,樹壁自固。武后封乞四比羽為許國公,乞乞仲象為震國公,赦其罪。比羽不受命,後詔玉鈐衛大將軍李楷固、中郎將索仇擊斬之。是時仲象已死,其子祚榮引殘痍遁去,楷固窮躡,度天門嶺。祚榮因高麗、靺鞨兵拒楷固,楷固敗還。於是契丹附突厥,王師道絕,不克討。祚榮即並比羽之眾,恃荒遠,乃建國,自號震國王,遣使交突厥,地方五千里,戶十餘萬,勝兵數萬。頗知書契,盡得扶餘、沃沮、弁韓、朝鮮海北諸國。中宗時,使侍御史張行岌招慰,祚榮遣子入侍。睿宗先天中,遣使拜祚榮為左驍衛大將軍、渤海郡王,以所統為忽汗州,領忽汗州都督。自是始去靺鞨號,專稱渤海。



 玄宗開元七年,祚榮死,其國私謚為高王。子武藝立,斥大土宇,東北諸夷畏臣之,私改年曰仁安。帝賜典冊襲王並所領。未幾,墨水靺鞨使者入朝,帝以其地建黑水州,置長史臨總。武藝召其下謀曰:「黑水始假道於我與唐通,異時請吐屯於突厥,皆先告我,今請唐官不吾告,是必與唐腹背攻我也。」乃遣弟門藝及舅任雅相發兵擊黑水。門藝嘗質京師,知利害,謂武藝曰:「黑水請吏而我擊之,是背唐也。唐,大國,兵萬倍我,與之產怨,我且亡。昔高麗盛時,士三十萬,抗唐為敵,可謂雄強,唐兵一臨,掃地盡矣。今我眾比高麗三之一,王將違之,不可。」武藝不從。兵至境,又以書固諫。武藝怒,遣從兄壹夏代將,召門藝,將殺之。門藝懼,儳路自歸,詔拜左驍衛將軍。武藝使使暴門藝罪惡,請誅之。有詔處之安西,好報曰:「門藝窮來歸我,誼不可殺,已投之惡地。」並留使者不遣,別詔鴻臚少卿李道邃、源復諭旨。武藝知之,上書斥言:「陛下不當以妄示天下」,意必殺門藝。帝怒道邃、復漏言國事,皆左除,而陽斥門藝以報。



 後十年,武藝遣大將張文休率海賊攻登州,帝馳遣門藝發幽州兵擊之。使太僕卿金思蘭使新羅,督兵攻其南。會大寒,雪袤丈,士凍死過半,無功而還。武藝望其弟不已,募客入東都狙刺於道。門藝格之,得不死。河南捕刺客,悉殺之。



 武藝死,其國私謚武王。子欽茂立,改年大興,有詔嗣王及所領,欽茂因是赦境內。天寶末,欽茂徙上京,直舊國三百里忽汗河之東。訖帝世,朝獻者二十九。寶應元年,詔以渤海為國,欽茂王之,進檢校太尉。大歷中,二十五來,以日本舞女十一獻諸朝。貞元時,東南徙東京。欽茂死,私謚文王。子宏臨早死,族弟元義立一歲,猜虐,國人殺之。推宏臨子華璵為王,復還上京,改年中興。死,謚曰成王。



 欽茂少子嵩鄰立,改年正歷,有詔授右驍衛大將軍,嗣王。建中、貞元間凡四來。死,謚康王。子元瑜立,改年永德。死,謚定王。弟言義立,改年硃雀,並襲王如故事。死,謚僖王。弟明忠立,改年太始,立一歲死,謚簡王。從父仁秀立,改年建興,其四世祖野勃,祚榮弟也。仁秀頗能討伐海北諸部,開大境宇,有功,詔檢校司空、襲王。元和中,凡十六朝獻,長慶四,寶歷凡再。大和四年,仁秀死,謚宣王。子新德蚤死,孫彞震立,改年咸和。明年,詔襲爵。終文宗世來朝十二,會昌凡四。彞震死,弟虔晃立。死,玄錫立。咸通時,三朝獻。



 初,其王數遣諸生詣京師太學,習識古今制度,至是遂為海東盛國,地有五京、十五府、六十二州。以肅慎故地為上京,曰龍泉府,領龍、湖、渤三州。其南為中京,曰顯德府,領廬、顯、鐵、湯、榮、興六州。貃故地為東京,曰龍原府,亦曰柵城府,領慶、鹽、穆、賀四州。沃沮故地為南京,曰南海府,領沃、睛、椒三州。高麗故地為西京,曰鴨淥府,領神、桓、豐、正四州;曰長嶺府,領瑕、河二州。扶餘故地為扶餘府,常屯勁兵捍契丹,領扶、仙二州;鄚頡府領鄚、高二州。挹婁故地為定理府,領定、潘二州;安邊府領安、瓊二州。率賓故地為率賓府,領華、益、建三州。拂涅故地為東平府,領伊、蒙、沱、黑、比五州。鐵利故地為鐵利府,領廣、汾、蒲、海、義、歸六州。越喜故地為懷遠府,領達、越、懷、紀、富、美、福、邪、芝九州;安遠府領寧、郿、慕、常四州。又郢、銅、涑三州為獨奏州。涑州以其近涑沫江,蓋所謂粟末水也。龍原東南瀕海,日本道也。南海,新羅道也。鴨淥,朝貢道也。長嶺,營州道也。扶餘,契丹道也。



 俗謂王曰「可毒夫」,曰「聖王」,曰「基下」。其命為「教」。王之父曰「老王」,母「太妃」,妻「貴妃」,長子曰「副王」,諸子曰「王子」。官有宣詔省,左相、左平章事、侍中、左常侍、諫議居之。中臺省,右相、右平章事、內史、詔誥舍人居之。政堂省,大內相一人,居左右相上;左、右司政各一,居左右平章事之下,以比僕射;左、右允比二丞。左六司,忠、仁、義部各一卿,居司政下,支司爵、倉、膳部,部有郎中、員外;右六司,智、禮、信部,支司戎、計、水部,卿、郎準左:以比六官。中正臺,大中正一,比御史大夫,居司政下;少正一。又有殿中寺、宗屬寺,有大令。文籍院有監。令、監皆有少。太常、司賓、大農寺,寺有卿。司藏、司膳寺,寺有令、丞。胄子監有監長。巷伯局有常侍等官。其武員有左右猛賁、熊衛、羆衛,南左右衛,北左右衛,各大將軍一、將軍一。大抵憲象中國制度如此。以品為秩,三秩以上服紫,牙笏、金魚。五秩以上服緋,牙笏、銀魚。六秩、七秩淺緋衣,八秩綠衣,皆木笏。



 俗所貴者,曰太白山之菟,南海之昆布,柵城之豉,扶餘之鹿,鄚頡之豕,率賓之馬,顯州之布,沃州之綿,龍州之紬,位城之鐵,廬城之稻,湄沱湖之鯽。果有九都之李,樂游之梨。餘俗與高麗、契丹略等。幽州節度府與相聘問,自營、平距京師蓋八千里而遠。後朝貢至否,史家失傳,故叛附無考焉。



 贊曰:唐之德大矣!際天所覆,悉臣而屬之;薄海內外,無不州縣,遂尊天子曰「天可汗」。三王以來,未有以過之。至荒區君長,待唐璽纛乃能國;一為不賓,隨輒夷縛。故蠻琛夷寶,踵相逮於廷。極熾而衰,厥禍內移,天寶之後,區夏痍破,王官之戍,北不逾河,西止秦、邠,凌夷百年,逮於亡,顧不痛哉!故曰:治己治人,惟聖人能之。



\end{pinyinscope}