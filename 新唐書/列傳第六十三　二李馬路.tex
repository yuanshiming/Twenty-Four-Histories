\article{列傳第六十三 二李馬路}

\begin{pinyinscope}

 李嗣業,字嗣業,京兆高陵人。長七尺,膂力絕眾。開元中,從安西都護來曜討十姓蘇祿會主義觀點,並動員全黨認真從事這項工作。指出「革命的,先登捕虜,累功署昭武校尉。後應募安西,軍中初用陌刀,而嗣業尤善,每戰必為先鋒,所向摧北。馬靈察為節度,出戰必與俱。高仙芝討勃律,署嗣業及中郎將田珍為左右陌刀將。時吐蕃兵十萬屯娑勒城,據山瀕水,聯木作郛,以扼王師。仙芝潛軍夜濟信圖河,令曰:「及午破賊,不者皆死。」嗣業提步士升山,頹石四面以擊賊,又樹大旗先走險,諸將從之。虜不虞軍至,因大潰,投崖谷死者十八。鼓而驅至勃律,禽其主,平之。授右威衛將軍。從平石國及突騎施,以跳蕩先鋒加特進。虜號為「神通大將」。



 初,仙芝特以計襲取石,其子出奔,因構諸胡共怨之,以告大食,連兵攻四鎮。仙芝率兵二萬深入,為大食所敗,殘卒數千。事急,嗣業謀曰:「將軍深履賊境,後援既絕,而大食乘勝,諸胡銳於斗,我與將軍俱前死,尚誰報朝廷者?不如守白石嶺以為後計。」仙芝曰:「吾方收合餘盡,明日復戰。」嗣業曰:「事去矣,不可坐須菹醢。」即馳守白石,路既隘,步騎魚貫而前。會拔汗那還兵,輜餉塞道不可騁,嗣業懼追及,手梃鏖擊,人馬斃僕者數十百,虜駭走,仙芝乃得還。表嗣業功,進右金吾大將軍,留為疏勒鎮使。城一隅阤,屢築輒壞,嗣業祝之,有白龍見,因其處蕝祠以祭,城遂不壞,漢耿恭故井久涸,禱已,泉復出。初討勃律也,通道蔥領,有大石塞隘,以足蹶之,抵穹壑,識者以為至誠所感云。



 天寶十二載,加驃騎大將軍。入朝,賜酒玄宗前,醉起舞,帝寵之,賜彩百、金皿五十物、錢十萬,曰:「為解酲具。」



 安祿山反,肅宗追之,詔至,即引道,與諸將割臂盟曰:「所過郡縣,秋毫不可犯。」至鳳翔,上謁,帝喜曰:「今日卿至,賢於數萬眾。事之濟否,固在卿輩。」乃詔與郭子儀、僕固懷恩掎角。常為先鋒,以巨咅笞斗,賊值,類崩潰。進四鎮、伊西、北庭行軍兵馬使。廣平王收長安,嗣業統前軍,陣於香積祠北。賊酋李歸仁擁精騎薄戰,王師注矢逐之,走未及營,賊大出,掩追騎,還蹂王師,於是亂不能陣。嗣業謂子儀曰:「今日不蹈萬死取一生,則軍無類矣。」即袒持長刀,大呼出陣前,殺數十人,陣復整。步卒二千以陌刀、長柯斧堵進,所向無前。歸仁匿兵營左,覘軍勢,王分回紇銳兵擊其伏,嗣業出賊背合攻之,自日中至昃,斬首六萬級,填澗壑死幾半,賊東走,遂平長安。進收東都,嗣業戰多。乃與張鎬、魯炅、來瑱、嗣吳王祗、李奐略定諸州。兼衛尉卿,封虢國公,實封戶二百。兼懷州刺史、北庭行營節度使。



 與子儀等圍相州,師耄,諸將無功,獨嗣業被堅數奮,為諸軍冠。中流矢,臥帳中,方愈,忽聞金鼓聲,知與賊戰,大呼,創潰,血流數升卒。謚曰忠勇,贈武威郡王,給靈輿護還在所。葬日,使中人臨吊,中朝臣祖泣,塋給掃除十戶。嗣業忠毅憂國,不計居產,有宛馬十疋,前後賞賜,皆上於官以助軍云。



 子佐國,襲爵,歷丹王府長史。卒,推嗣業功,贈宋州刺史。



 馬璘,岐州扶風人。少孤,流蕩無業所。年二十,讀漢馬援傳,至「丈夫當死邊野,以馬革裹尸而歸」,慨然曰:「使吾祖勛業墜於地乎?」開元末,挾策從安西節度府,以奇勞,累遷金吾衛將軍。



 至德初,王室多難,統精甲三千,自二庭赴鳳翔。肅宗奇之,委以東討。初戰衛南,以百騎破賊五千眾。從李光弼攻洛陽,史朝義眾十萬陣北邙山,旗鎧照日,諸■疑,未敢擊。璘率部士五百,薄賊屯,出入三反,眾披靡,乘之,賊遂潰。光弼曰:「吾用兵三十年,未見以少擊眾,雄捷如馬將軍者!」遷試太常卿。



 明年,吐蕃寇邊,詔璘移軍援河西。懷恩之叛,璘引還,間關轉鬥至鳳翔,虜圍已合,節度使孫志直嬰城守。璘令士持滿外向,突入縣門,不解甲出戰,背城陣。虜潰,率輕騎追之,斬數千級,漂血丹渠。帝引見尉勞,擢兼御史大夫。



 永泰初,拜四鎮行營節度、南道和蕃使。俄檢校工部尚書,北庭行營、邠寧節度使。元日,有卒犯盜,或曰宜赦,璘曰:「赦之,則人將伺其日為盜。」遂戮之。天大旱,里巷為土龍聚巫以禱,璘曰:「旱由政不修。」即命撤之。明日雨,是歲大穰。未幾,徙涇原,權知鳳翔、隴右節度副使,四鎮、北庭如舊,復以鄭、潁二州隸之。



 大歷八年,吐蕃內寇,渾瑊戰宜祿,不利。璘設伏潘原,與瑊合擊破之,俘級數萬。進檢校尚書右僕射。明年,入朝,求宰相,以檢校左僕射知省事,進撫風郡王。十一年,卒於軍,年五十六。贈司徒,謚曰武。



 璘少學術,而武幹絕倫。遭時屯棘,以忠力奮。在涇八年,繕屯壁,為戰守具,令肅不殘,人樂為用,虜不敢犯,為中興銳將。初,涇軍乏財,帝諷李抱玉讓鄭、潁,璘因得裒積,且前後賜賚無算,家富不貲。治第京師,侈甚,其寢堂無慮費錢二十萬緡。方璘在軍,守者覆以油幔。及喪歸,都人爭入觀,假稱故吏入赴吊者日數百。德宗在東宮聞之,不喜。及即位,乃禁第舍不得逾制,詔毀璘中寢及宦人劉忠翼第。璘家懼,悉籍亭館入之官。其後賜群臣宴,多在璘山池。而子弟無行,財亦尋盡。



 李抱玉,本安興貴曾孫,世居河西,善養馬。始名重璋,閑騎射,少從軍。其為人沈毅有謀,尤忠謹,李光弼引為裨校。天寶末,玄宗以其戰河西有功,為改今名。祿山亂,守南陽,斬賊使。至德二載,上言:「世占涼州,恥與逆臣共宗。」有詔賜之姓,因徙籍京兆,舉族以李為氏。進至右羽林大將軍,知軍事,擢陳鄭潁亳節度使。史思明已破東都,兇焰勃然,鼓而行,自謂無前。光弼壁河陽拒之,使抱玉守南城。賊急攻,抱玉縱奇兵出,表裏俘殺甚眾。賊乃舍去,從光弼戰,大敗,因不能西。差功第一,封欒城縣公。代宗立,兼澤潞節度使,統相、衛、儀、邢十一州兵。以功授司空,兼兵部尚書,武威郡王。懇辭王爵,徙涼國公,進司徒。



 廣德中,吐蕃入寇,帝次陜,群盜遍南山五穀間,東距虢,西抵岐,椎剽不勝計。詔太子賓客薛景仙為南山五溪穀防禦使,引兵招捕,久不克。更詔抱玉討賊。抱玉盡得賊株柢蹊隧,分兵守諸谷,使牙將李崇客精騎四百,自桃林、虢川襲之。賊帥高玉脫身走城固,山南西道張獻誠禽以獻,悉索支黨斬之。不閱旬,五穀平。即詔抱玉權鳳翔、隴右節度,抱玉懇讓司徒,故以尚書左僕射同中書門下平章事,河西、隴右副元帥。又讓僕射,故還為兵部尚書。



 大歷二年,來朝。久之,加山南西道副元帥兼節度使,屯鷫厔。抱玉兼三節度、三副元帥,位望隆赫。乃上言:「隴坻達扶、文,綿地二千里,虜孔道不一,梁、岷重則關輔輕。願擇能臣,帥西道當一面,臣得專事關、隴。」帝多其讓,許之。抱玉在鎮十餘年,雖無破虜功,而禁暴安人,為將臣之良。卒,年七十四,贈太保,謚曰昭武。



 從父弟抱真。抱真字太玄,沈慮而斷。抱玉屬以軍事,授汾州別駕。僕固懷恩反,陷焉,挺身歸京師。代宗以懷恩倚回紇,所將朔方兵精,憂之,召抱真問狀,答曰:「郭子儀嘗領朔方軍,人多德之。懷恩欺其下曰,『子儀為朝恩所殺。』今起而用,是伐其謀,兵可不戰解也。」既而懷恩敗,如抱真策。遷殿中少監、陳鄭澤潞節度留後。既謝,因言:「百姓勞逸在牧守,願得一州以自試。」更授澤州刺史,兼澤潞節度副使。徙懷州,仍為懷澤潞觀察留後,凡八年。



 抱真策山東有變,澤、潞兵所走集,乘戰伐後,賦重人困,軍伍雕刓,乃籍戶三丁擇一,蠲其徭租,給弓矢,令閑月得曹偶習射,歲終大校,親按籍第能否賞責。比三年,皆為精兵,舉所部得成卒二萬,既不稟於官,而府庫實。乃曰:「軍可用矣。」繕甲淬兵,遂雄山東,天下稱昭義步兵為諸軍冠。久之,為澤潞節度行軍司馬。會昭義節度李承昭病,詔抱真權磁邢兵馬留後。德宗嗣位,檢校工部尚書,領昭義節度使。



 建中中,田悅反,圍邢及臨洺,詔抱真與河東馬燧合神策兵救之,敗悅於雙岡,斬其將楊朝光,又破之臨洺,遂解臨洺、邢之圍。以功檢校兵部尚書。復與悅戰洹水,走之。進圍魏,悅戰城下,大敗。進檢校尚書右僕射。會硃滔、王武俊反,救悅,抱真退保魏。帝蒼卒狩奉天,聞問,諸將皆哭,各引麾下還屯。於時,李希烈陷汴,李納反鄆,李懷光相次反河中,抱真獨以數州截然橫絕潰叛中,離沮其奸,為群盜所憚。



 興元初,檢校左僕射、同中書門下平章事,繇倪國公進義陽郡王。硃滔悉幽薊兵與回紇圍貝州,以應硃泚。而希烈既竊名號,則欲臣制諸叛,眾稍離。天子下罪己詔,並赦群盜。抱真乃遣客賈林以大義說武俊,使合從擊滔,武俊許諾,而內猶豫。抱真將自造其壁,諉軍事於司馬盧玄卿曰:「吾此行,系時安危,使遂不還,部勒以聽天子命,惟子;勵兵東向,雪吾之恥,亦唯子。」即以數騎馳入見武俊,曰:「泚、希烈爭竊帝號,滔攻貝州,此其志皆欲自肆於天下。足下既不能與競長雄,舍九葉天子而臣反虜乎?且詔書罪己,禹、湯之心也。方上暴露播越,公能自安乎?」因持武俊,涕下交頤,武俊亦感泣,左右皆泣。退臥帳中,甘寢久之。武俊感其不疑,乃益恭,指心誓天曰:「此身已許公死矣!」食訖,約為昆弟而別。旦日合戰,大破滔經城。進檢校司空,實封六百戶。貞元初,朝京師,詔還所鎮。



 抱真喜士,聞世賢者,必欲與之游,雖小善,皆卑禮厚幣數千里邀致之,至無可錄,徐徐以禮謝。會天下稍無事,乃飾臺沼以自娛。好方士,謂不死可致。有孫季長者為治丹,且曰:「服此當仙去。」抱真表署幕府。嘗語左右曰:「秦、漢君不偶此,我乃得之,後升天,不復見公等矣。」夜夢駕鶴,寤而刻寓鶴,衣羽服,習乘之。後益惑厭勝,因疾,請降官,七讓司空,還為左僕射。餌丹二萬丸,不能食,且死,醫以彘肪穀漆下之。疾少間,季長曰:「危得仙,何自棄也?」益服三千丸,卒,年六十二。



 其子殿中侍御史緘匿喪,與其屬盧會昌元仲經謀,會諸將,仲經詭抱真令曰:「吾疾不任事,令緘典軍,勉佐之。」副使李說及諸校俯首,皆嘸曰;「諾。」緘盛服出,眾拜之,悉發府庫勞軍。會昌即為抱真表,翌日,令諸將署章,請以節付緘。天子已聞抱真喪,遣使者馳入軍,詔以事屬大將王延貴。緘謾若抱真疾,請詰朝見,凡三日,緘乃出見使者,陳兵甚嚴。使者曰:「朝廷已知公薨,詔以兵屬延貴,君速歸發喪。」緘愕然,謂諸將曰:「詔不許,若何?」眾不對。乃遽以印鑰上監軍,始發喪。使者趣延貴視事,護緘赴東都,仲經逃諸外,捕殺之,會昌得不坐。始,緘遣將陳榮以書抵武俊,假其財。武俊怒曰;「吾與乃公善者,恭王命,非同惡也。今聞已亡,誰詐其子使不俟朝制邪?」囚榮而讓緘焉。詔贈抱真太保。



 路嗣恭,字懿範,京兆三原人,始名劍客,以世廕為鄴尉。席豫黜陟河朔,表為蕭關令,連徙神烏、姑臧二縣,考績為天下最。玄宗以為可嗣漢魯恭,因賜名。轉渭南令,主杜化、東陽二驛。時關畿用兵,使人系道,嗣恭儲具有素,而民不擾。後為郭子儀朔方節度留後,大將孫守亮擁重兵,驕蹇不受制,嗣恭因稱疾,守亮至,即殺之,一軍皆震。永泰三年,檢校刑部尚書,知省事。出為江西觀察使,以善治財賦稱。有賈明觀者,素事魚朝恩,朝恩誅,當坐死,宰相元載納其賂,遣效力江西,將行,居民數萬懷瓦石候擊,載諭市吏禁止,乃得去。魏少游畏載,常回容之,及嗣恭代少游,即日杖死。



 大歷八年,嶺南將哥舒晃殺節度使呂崇賁,五嶺大擾。詔嗣恭兼嶺南節度使,封冀國公。嗣恭募勇敢士八千人,以流人孟瑤、敬冕為才,擢任之。使瑤督大軍當其沖,冕率輕兵由間道出不意,遂斬晃及支黨萬餘,築尸為京觀。俚洞魁宿為惡者,皆族夷之。還為檢校兵部尚書,復知省事。嗣恭起州縣吏,以課治進至顯官,及晃事株戮舶商,沒其財數百萬私有之,代宗惡焉,故賞不酬功。德宗立,陰賕宰相楊炎,炎錄前效,更拜兵部尚書、東都留守。俄加懷鄭汝陜河陽三城節度、東都畿觀察使。卒,年七十一,贈左僕射。子應、恕。



 應,字從眾,以廕為著作郎。貞元初,出為虔州刺史,詔嗣父封。鑿贛石梗嶮以通舟道。德宗時,李泌為相,號得君。帝嘗曰:「誰於卿有恩者,朕能報之。」泌乃言:「「曩為元載所疾,謫江西,路嗣恭與載厚,臣嘗畏之。會與其子應並驅,馬嚙其脛,臣惶恐不自安,應閟不言,勉起見父。臣常愧其長者,思有以報。」帝曰:「善。」即日加應檢校屯田郎中,服金紫。累遷宣歙池觀察使,封襄陽郡王。李錡反,應發鄉兵救湖、常二州,以故錡不能拔。元和六年,以疾授左散騎常侍,卒,謚曰靖。



 恕,字體仁。從嗣恭討哥舒晃,授檢校工部員外郎,得從便宜,擢降將伊慎用之。賊平,恕功多。嗣恭節度河陽也,恕為懷州刺史,年才三十,楊炎用捍魏博,為時嗤詆。累遷鄜坊、宣歙觀察使。坐事貶吉州刺史。以右散騎常侍致仕,卒,贈洪州都督。



\end{pinyinscope}