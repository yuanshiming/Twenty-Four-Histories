\article{列傳第一百一十七 忠義中}

\begin{pinyinscope}

 顏杲卿春卿賈循隱林張巡許遠南霽雲雷萬春姚訚



 顏杲卿字昕,與真卿同五世祖,以文儒世家。父元孫,有名垂拱間,為濠州刺史。杲卿以廕調遂州司法參軍。性剛正,蒞事明濟。嘗為刺史詰讓,正色別白,不為屈。開元中,與兄春卿、弟曜卿並以書判超等,吏部侍郎席豫咨嗟推伏。再以最遷範陽戶曹參軍。安祿山聞其名,表為營田判官,假常山太守。



 祿山反,杲卿及長史袁履謙謁於道,賜杲卿紫袍,履謙緋袍,令與假子李欽湊以兵七千屯土門。杲卿指所賜衣謂履謙曰:「與公何為著此?」履謙悟,乃與真定令賈深、內丘令張通幽定謀圖賊。杲卿稱疾不視事,使子泉明往返計議,陰結太原尹王承業為應,使平盧節度副使賈循取幽州。謀洩,祿山殺循,以向潤客、牛廷玠守。杲卿陽不事事,委政履謙,潛召處士權渙、郭仲邕定策。時真卿在平原,素聞賊逆謀,陰養死士為拒守計。李憕等死,賊使段子光傳首徇諸郡,真卿斬子光,遣甥盧逖至常山約起兵,斷賊北道。杲卿大喜,以為兵掎角可挫賊西鋒。乃矯賊命召欽湊計事,欽湊夜還,杲卿辭城門不可夜開,舍之外郵;使履謙及參軍馮虔、郡豪翟萬德等數人飲勞,既醉,斬之,並殺其將潘惟慎,賊黨殲,投尸滹沱水。履謙以首示杲卿,則喜且泣。



 先是,祿山遣將高邈召兵範陽未還,杲卿使匋城尉崔安石圖之。邈至滿城,虔、萬德皆會傳舍,安石紿以置酒,邈舍馬,虔叱吏縛之。而賊將何千年自趙來,虔亦執之。日未中,送二賊。杲卿乃遣萬德、深、通幽傳欽湊首,械兩賊送京師,與泉明偕。至太原,王承業欲自以為功,厚遣泉明還,陰令壯士翟喬賊於路。喬不平,告之故,乃免。玄宗擢承業大將軍,送吏皆被賞。已而事顯,乃拜杲卿衛尉卿兼御史中丞,履謙常山太守,深司馬。即傳檄河北,言王師二十萬入土門,遣郭仲邕領百騎為先鋒,馳而南,曳柴揚塵,望者謂大軍至。日中,傳數百里。賊張獻誠方圍饒陽,棄甲走。於是趙、鉅鹿、廣平、河間並斬偽刺史,傳首常山。而樂安、博陵、上谷、文安、信都、魏、鄴諸郡皆自固。杲卿兄弟兵大振。



 祿山至陜,聞兵興,大懼。使史思明等率平盧兵度河攻常山,蔡希德自懷會師。不涉旬,賊急攻城。兵少,未及為守計,求救於河東,承業前已攘殺賊功,兵不出。杲卿晝夜戰,井竭,糧、矢盡,六日而陷,與履謙同執。賊脅使降,不應。取少子季明加刃頸上曰:「降我,當活而子。」杲卿不答。遂並盧逖殺之。杲卿至洛陽,祿山怒曰:「吾擢爾太守,何所負而反?」杲卿瞋目罵曰:「汝營州牧羊羯奴耳,竊荷恩寵,天子負汝何事,而乃反乎?我世唐臣,守忠義,恨不斬汝以謝上,從從爾反耶?」祿山不勝忿,縛之天津橋柱,節解以肉啖之,罵不絕,賊鉤斷其舌,曰:「復能罵否?」杲卿含胡而絕,年六十五。履謙既斷手足,何千年弟適在傍,咀血噴其面,賊臠之,見者垂泣。杲卿宗子近屬皆被害。杲卿已虜,諸郡復為賊守。



 張通幽以兄相賊,譖杲卿於楊國忠,故不加贈。肅宗在鳳翔,真卿表其枉,會通幽為普安太守,上皇杖殺之。李光弼、郭子儀收常山,出杲卿、履謙二家親屬數百人於獄,厚給遺,令行喪。乾元初,贈杲卿太子太保,謚曰忠節,封其妻崔清河郡夫人。初,博士裴郡以杲卿不執政,但謚曰忠,議者不平,故以二惠謚焉。逖、季明及宗子等皆贈五品官。建中中,又贈杲卿司徒。初,杲卿被殺,徇首於衢,莫敢收。有張湊者,得其發,持謁上皇。是昔見夢,帝寤,為祭。後湊歸發於其妻,妻疑之,發若動雲。後泉明購尸將葬,得刑者言,死時一足先斷,與履謙同坎瘞。指其域得之,乃葬長安鳳棲原。季明、逖同塋。



 泉明有孝節,喜振人之急。既為承業所遣,未至而常山陷,故客壽陽。史思明圍李光弼,獲泉明,裹以革,送幽州,間關得免。思明歸國,而真卿方為蒲州刺史,令泉明到河北求宗屬。始,一女及姑女並流離賊中,及是並得之,悉錢三萬贖姑女還,取貲復往,則己女復失之。履謙及父故將妻子奴隸尚三百餘人,轉徙不自存,泉明悉力贍給,分多勻薄,相扶挾度河托真卿。真卿隨所歸資送之。泉明之殯父,與履謙分柩,護還長安。履謙妻疑斂具儉狹,發視之,與杲卿等,乃號踴,待泉明如父。肅宗拜泉明郫令,政化清明,誅宿盜,人情翕然。成都尹舉其課第一,遷彭州司馬。家貧,居官廉,而孤藐相從百口,飦鬻不給,無慍嘆。居母喪,毀骨立。其行義,當世以為難。



 春卿,倜儻美姿儀,通當世務。十六舉明經、拔萃高第,調犀浦主簿。嘗送徒於州,亡其籍,至廷,口記物色,凡千人,無所差。長史陸象先異之,轉蜀尉。蘇頲代為長史,被譖系獄,為《棕櫚賦》自托,頲遽出之。魏徵遠孫瞻罪抵死,春卿為請玉真公主,得不死,時人高其節。終偃師丞。臨終,捉真卿臂曰:「爾當大吾族,顧我不得見,以諸子諉汝。」後真卿主其昏嫁。



 沈盈者,亦杲卿甥,有行義,明黃老學。解褐博野尉,與杲卿同死難,贈大理正,官其二子遙、達。」



 賈循者,京兆華原人,其先家常山。父會,有高節嘗稱疾不答闢署,里中號「一龍」。親亡,負土成墓,廬其左,手蒔松柏,時號「關中曾子」。卒,縣人私謚曰廣孝徵君。



 循有大略,禮部尚書蘇頲嘗謂今頗、牧,及為益州,表署列將。敗吐蕃於西山,三遷靜塞軍營田使。張守珪北伐,次灤河,屬凍泮,欲濟無梁。循揣廣狹為橋以濟,破虜而還,以功擢游擊將軍、榆關守捉使。地南負海,北屬長城,林良岑翳,寇所蔽伏。循調土斬木開道,賊遁去。範陽節度使李適之薦為安東副大都護。安祿山兼平盧節度,表為副,遷博陵太守。祿山欲擊奚、契丹,復奏循光祿卿自副,使知留後。九姓叛,祿山兼節度河東,而循亦兼雁門副之。母亡將葬,宅有枯桑,一夕再生,芝出北庸,人以為瑞。玄宗以循有功,詔贈其父常山太守。



 祿山反,使循守幽州,故杲卿招之,以傾賊巢穴,循許可。為向潤客等發其謀,賊縊之。建中二年,贈太尉,謚曰忠。



 從子隱林,為永平兵馬使。當入衛,屬硃泚難,率眾扈行在。德宗見隱林,偉其貌,問家世,答曰:』故範陽節度副使循,臣從父也。」帝異之,引至臥內,以手板畫地陳攻守計,即奏曰:「臣嘗夢日墜,以首承之。」帝曰:「非朕邪?」因令糾察行在,遷檢校右散騎常侍,封武威郡王。



 賊圍急,隱林與侯仲莊冒矢石死戰。已而解,從臣稱慶,隱林流涕前曰:「泚已奔,群臣大慶宗社無疆之休,然陛下資性急,不能容掩。若不悛,雖今賊亡,憂未艾也。」帝不以為忤,拜神策統軍。卒,帝思其質直,贈尚書左僕射,以實戶三百封其家。



 張巡字巡,鄧州南陽人。博通群書,曉戰陣法。氣志高邁,略細節,所交必大人長者,不與庸俗合,時人叵知也。開元末,擢進士第。時兄曉已位監察御史,皆以名稱重一時。巡由太子通事舍人出為清河令,治績最,而負節義,或以困厄歸者,傾貲振護無吝。秩滿還都。於是楊國忠方專國,權勢可炙。或勸一見,且顯用,答曰:「是方為國怪祥,朝宦不可為也。」更調真源令。土多豪猾,大吏華南金樹威恣肆,邑中語曰:「南金口,明府手。」巡下車,以法誅之,赦餘黨,莫不改行遷善。政簡約,民甚宜之。



 安祿山反,天寶十五載正月,賊酋張通晤陷宋、曹等州,譙郡太守楊萬石降賊,逼巡為長史,使西迎賊軍。巡率吏哭玄元皇帝祠,遂起兵討賊,從者千餘。初,靈昌太守嗣吳王祗受詔合河南兵拒祿山,有單父尉賈賁者,閬州刺史璇之子,率吏稱吳王兵,擊宋州。通晤走襄邑,為頓丘令盧韺所殺。賁引軍進至雍丘,巡與之合,有眾二千。是時雍丘令令狐潮舉縣附賊,遂自將東敗淮陽兵,虜其眾,反接在廷,將殺之,暫出行部。淮陽囚更解縛,起殺守者,迎賁等入。潮不得歸,巡乃屠其妻子,礫城上。祗聞,承制拜賁監察御史。潮怨賁,還攻雍丘,賁趨門,為眾躪死。巡馳騎決戰,身被創不顧,士乃奉巡主軍。間道表諸朝,騰箋祗府,祗乃舉兗以東委巡經略。



 潮以賊眾四萬薄城,人大恐。巡諭諸將曰:「賊知城中虛實,有輕我心。今出不意,可驚而潰也,乘之,勢必折。」諸將曰:「善。」巡乃分千人乘城,以數隊出,身前驅,直薄潮軍,軍卻。明日賊攻城,設百樓,巡柵城上,束芻灌膏以焚焉,賊不敢向,巡伺擊之。積六旬,大小數百戰,士帶甲食,裹瘡斗,潮遂敗走,追之,幾獲。潮怒,復率眾來。然素善巡,至城下,情語巡曰:「本朝危蹙,兵不能出關,天下事去矣。足下以羸兵守危堞,忠無所立,盍相從以茍富貴乎?」巡曰:「古者父死於君,義不報。子乃銜妻孥怨,假力於賊以相圖,吾見君頭乾通衢,為百世笑,奈何?」潮赧然去。



 當此時,王命不復通,大將六人白巡以勢不敵,且上存亡莫知,不如降。六人者,皆官開府、特進。巡陽許諾,明日堂上設天子畫像,率軍士朝,人人盡泣。巡引六將至,責以大誼,斬之。士心益勸。



 會糧乏,潮餉賊鹽米數百艘且至,巡夜壁城南,潮悉軍來拒,巡遣勇士銜枚濱河,取鹽米千斛,焚其餘而還。城中矢盡,巡縛槁為人千餘,被黑衣,夜縋城下,潮兵爭射之,久,乃槁人;還,得箭數十萬。其後復夜縋人,賊笑,不設備,乃以死士五百斫潮營,軍大亂,焚壘幕,追奔十餘里。賊慚,益兵圍之。薪水竭,巡紿潮:「欲引眾走,請退軍二舍,使我逸。」潮不知其謀,許之。遂空城四出三十里,撤屋發木而還為備。潮怒,圍復合。巡徐謂潮曰:「君須此城,歸馬三十匹,我得馬且出奔,請君取城以藉口。」潮歸馬,巡悉以給驍將,約曰:「賊至,人取一將。」明日,潮責巡,答曰:「吾欲去,將士不從,奈何?」潮怒欲戰,陣未成,三十騎突出,禽將十四,斬百餘級,收器械牛馬。潮遁還陳留,不復出。七月,潮率賊將瞿伯玉攻城,遣偽使者四人傳賊命詔巡,巡斬以徇,餘縶送祗所。圍凡四月,賊常數萬,而巡眾才千餘,每戰輒克。於是河南節度使嗣虢王巨屯彭城,假巡先鋒。



 俄而魯、東平陷賊,濟陰太守高承義舉郡叛,巨引兵東走臨淮。賊將楊朝宗謀趨寧陵,絕巡餉路。巡外失巨依,拔眾保寧陵,馬裁三百,兵三千。至睢陽,與太守許遠、城父令姚訚等合。乃遣將雷萬春、南霽雲等領兵戰寧陵北,斬賊將二十,殺萬餘人,投尸於汴,水為不流。朝宗夜去。有詔拜巡主客郎中,副河南節度使。巡籍將士有功者請於巨,巨才授折沖、果毅。巡諫曰:「宗社尚危,圍陵孤外,渠可吝賞與貲?」巨不聽。



 至德二載,祿山死,慶緒遣其下尹子琦將同羅、突厥、奚勁兵與朝宗合,凡十餘萬,攻睢陽。巡勵士固守,日中二十戰,氣不衰。遠自以材不及巡,請稟軍事而居其下,巡受不辭,遠專治軍糧戰具。前此,遠將李滔救東平,遂叛入賊,大將田秀榮潛與通。或以告遠曰:「晨出戰,以碧帽為識。」視之如言,盡覆其眾。還輒曰:「我誘之也。」請以精騎往,易錦帽。遠以告巡,巡召登城,讓之,斬首示賊。因出薄戰,子琦敗,獲車馬牛羊,悉分士,秋豪無入其家。有詔拜巡御史中丞,遠侍御史,訚吏部郎中。



 巡欲乘勝擊陳留,子琦聞,復圍城。巡語其下曰:「吾蒙上恩,賊若復來,正有死耳。諸君雖捐軀,而賞不直勛,以此痛恨!」聞者感概。乃椎牛大饗,悉軍戰。賊望兵少,大笑。巡、遠親鼓之,賊潰,追北數十里。其五月,賊刈麥,乃濟師。巡夜鳴鼓嚴隊,若將出。賊申警。俄自鼓,賊覘城上兵休,乃弛備。巡使南霽雲等開門徑抵子琦所,斬將拔旗。有大酋被甲,引拓羯千騎麾幟乘城招巡。巡陰縋勇士數十人隍中,持鉤、陌刀、強弩,約曰:「聞鼓聲而奮。」酋恃眾不為備,城上噪,伏發禽之,弩注矢外向,救兵不能前。俄而縋士復登陴,賊皆愕眙,乃按甲不出。巡欲射子琦,莫能辨,因剡蒿為矢,中者喜,謂巡矢盡,走白子琦,乃得其狀。使霽雲射,一發中左目,賊還。七月,復圍城。



 初,睢陽穀六萬斛,可支一歲,而巨發其半餫濮陽、濟陰,遠固爭,不聽。濟陰得糧即叛。至是食盡,士日賦米一勺,齕木皮、煮紙而食,才千餘人,皆臒劣不能彀,救兵不至。賊知之,以雲沖傳堞,巡出鉤銘乾拄之,使不得進,篝火焚梯。賊以鉤車、木馬進,巡輒破碎之。賊服其機,不復攻,穿壕立柵以守。巡士多餓死,存者皆痍傷氣乏。巡出愛妾曰:「諸君經年乏食,而忠義不少衰,吾恨不割肌以啖眾,寧惜一妾而坐視士饑?」乃殺以大饗,坐者皆泣。巡強令食之,遠亦殺奴僮以哺卒,至羅雀掘鼠,煮鎧弩以食。



 賊將李懷忠過城下,巡問:「君事胡幾何?」曰:「二期。」巡曰:「君祖、父官乎?」曰:「然。」君世受官,食天子粟,奈何從賊,關弓與我確?」懷忠曰:「不然,我昔為將,數死戰,竟歿賊,此殆天也。」巡曰:「自古悖逆終夷滅,一日事平,君父母妻子並誅,何忍為此?」懷忠掩涕去,俄率其黨數十人降。巡前後說降賊將甚多,皆得其死力。



 御史大夫賀蘭進明代巨節度,屯臨淮,許叔冀、尚衡次彭城,皆觀望莫肯救。巡使霽雲如叔冀請師,不應,遣布數千端。霽雲嫚罵馬上,請決死鬥,叔冀不敢應。巡復遣如臨淮告急,引精騎三十冒圍出,賊萬眾遮之,霽雲左右射,皆披靡。既見進明,進明曰:「睢陽存亡已決,兵出何益?」霽雲曰:「城或未下。如已亡,請以死謝大夫。」叔冀者,進明麾下也,房琯本以牽制進明,亦兼御史大夫,勢相埒而兵精。進明懼師出且見襲,又忌巡聲威,恐成功,初無出師意。又愛霽雲壯士,欲留之。為大饗,樂作,霽雲泣曰:「昨出睢陽時,將士不粒食已彌月。今大夫兵不出,而廣設聲樂,義不忍獨享,雖食,弗下咽。今主將之命不達,霽雲請置一指以示信,歸報中丞也。」因拔佩刀斷指,一座大驚,為出涕。卒不食去。抽矢回射佛寺浮圖,矢著磚,曰:「吾破賊還,必滅賀蘭,此矢所以志也!」至真源,李賁遺馬百匹;次寧陵,得城使廉坦兵三千,夜冒圍入。賊覺,拒之,且戰且引,兵多死,所至才千人。方大霧,巡聞戰聲,曰:「此霽雲等聲也。」乃啟門,驅賊牛數百入,將士相持泣。



 賊知外援絕,圍益急。眾議東奔,巡、遠議以睢陽江、淮保障也,若棄之,賊乘勝鼓而南,江、淮必亡。且帥饑眾行,必不達。十月癸丑,賊攻城,士病不能戰。巡西向拜曰:「孤城備竭,弗能全。臣生不報陛下,死為鬼以癘賊。」城遂陷,與遠俱執。巡眾見之,起且哭,巡曰:「安之,勿怖,死乃命也。」眾不能仰視。子琦謂巡曰:「聞公督戰,大呼輒眥裂血面,嚼齒皆碎,何至是?」答曰:「吾欲氣吞逆賊,顧力屈耳。」子琦怒,以刀抉其口,齒存者三四。巡罵曰:「我為君父死,爾附賊,乃犬彘也,安得久!」子琦服其節,將釋之。或曰:「彼守義者,烏肯為我用?且得眾心,不可留。」乃以刃脅降,巡不屈。又降霽雲,未應。巡呼曰:「南八!男兒死爾,不可為不義屈!」霽雲笑曰:「欲將有為也,公知我者,敢不死!」亦不肯降。乃與姚訚、雷萬春等三十六人遇害。巡年四十九。初,子琦議生致五人慶緒所,或曰:「用兵拒守者,巡也。」乃送遠洛陽,至偃師,亦以不屈死。巨之走臨淮,巡有姊嫁陸氏,遮王勸勿行,不納,賜百縑,弗受,為巡補縫行間,軍中號「陸家姑」,先巡被害。



 巡長七尺,須髯每怒盡張。讀書不過三復,終身不忘。為文章不立稿。守睢陽,士卒居人,一見問姓名,其後無不識。更潮及子琦,大小四百戰,斬將三百、卒十餘萬。其用兵未嘗依古法,勒大將教戰,各出其意。或問之,答曰:「古者人情敦樸,故軍有左右前後,大將居中,三軍望之以齊進退。今胡人務馳突,雲合鳥散,變態百出,故吾止使兵識將意,將識士情,上下相習,人自為戰爾。」其械甲取之於敵,未嘗自脩。每戰,不親臨行陣,有退者,巡已立其所,謂曰:「我不去此,為我決戰。」士感其誠,皆一當百。待人封鎖所疑,賞罰信,與眾共甘苦塞暑,雖廝養,必整衣見之,下爭致死力,故能以少擊眾,未嘗敗。被圍久,初殺馬食,既盡,而及婦人老弱凡食三萬口。人知將死,而莫有畔者。城破,遣民止四百而已。



 始,肅宗詔中書侍郎張鎬代進明節度河南,率浙東李希言、浙西司空襲禮、淮南高適、青州鄧景山四節度掎角救睢陽,巡亡三日而鎬至,十日而廣平王收東京。鎬命中書舍人蕭昕誄其行。時議者或謂:巡始守睢陽,眾六萬,既糧盡,不持滿按隊出再生之路,與夫食人,寧若全人?於是張澹、李紓、董南史、張建封、樊晁、硃巨川、李翰咸謂巡蔽遮江、淮,沮賊勢,天下不亡,其功也。翰等皆有名士,由是天下無異言。天子下詔,贈巡揚州大都督,遠荊州大都督,霽雲開府儀同三司、再贈揚州大都督,並寵其子孫。睢陽、雍丘賜徭稅三年。巡子亞夫拜金吾大將軍,遠子玖婺州司馬。皆立廟睢陽,歲時致祭。德宗差次至德以來將相功效尤著者,以顏杲卿、袁履謙、盧弈及巡、遠、霽雲為上。又贈姚訚潞州大都督,官一子。貞元中,復官巡它子去疾、遠子峴。贈巡妻申國夫人,賜帛百。自是訖僖宗,求忠臣後,無不及三人者。大中時,圖巡、遠、霽雲像於凌煙閣。睢陽至今祠享,號「雙廟」云。



 許遠者,右相敬宗曾孫。寬厚長者,明吏治。初客河西,章仇兼瓊闢署劍南府,欲以子妻之,固辭。兼瓊怒,以事劾貶高要尉。更赦還。會祿山反,或薦遠於玄宗,召拜睢陽太守。遠與巡同年生而長,故巡呼為兄。



 大歷中,巡子去疾上書曰:「孽胡南侵,父巡與睢陽太守遠各守一面。城陷,賊所入自遠分。尹子琦分郡部曲各一方,巡及將校三十餘皆割心剖肌,慘毒備盡,而遠與麾下無傷。巡臨命嘆曰:『嗟乎,人有可恨者!』賊曰:『公恨我乎?』答曰:『恨遠心不可得,誤國家事,若死有知,當不赦於地下。』故遠心向背,梁、宋人皆知之。使國威喪衄,巡功業墮敗,則遠於臣不共戴天,請追奪官爵,以刷冤恥。」詔下尚書省,使去疾與許峴及百官議。皆以去疾證狀最明者,城陷而遠獨生也。且遠本守睢陽,凡屠城以生致主將為功,則遠後巡死不足惑。若曰後死者與賊,其先巡死者謂巡當叛,可乎?當此時去疾尚幼,事未詳知。且艱難以來,忠烈未有先二人者,事載簡書,若日星不可妄輕重。議乃罷。然議者紛紜不齊。



 元和時,韓愈讀李翰所為巡傳,以為闕遠事非是。其言曰:「二人者,守死成名,先後異耳。二家子弟材下,不能通知其父志,使世疑遠畏死而服賊。遠誠畏死,何苦守尺寸地,食其所愛之肉,抗不降乎?且見援不至,人相食而猶守,雖其愚亦知必死矣,然遠之不畏死甚明。」又言:「城陷自所守,此與兒童之見無異。且人之將死,其臟腑必有先受病者;引繩而絕之,其絕必有處。今從而尤之,亦不達於理矣。」愈於褒貶尤慎,故著之。



 南霽雲者,魏州頓丘人。少微賤,為人操舟。祿山反,鉅野尉張沼起兵討賊,拔以為將。尚衡擊汴州賊李廷望,以為先鋒。遣至睢陽,與張巡計事。退謂人曰:「張公開心待人,真吾所事也。」遂留巡所。巡固勸歸,不去。衡齎金帛迎,霽雲謝不受,乃事巡,巡厚加禮。始被圍,築臺募萬死一生者,數日無敢應。俄有喑鳴而來者,乃霽雲也。巡對泣下。霽雲善騎射,見賊百步內乃發,無不應弦斃。



 子承嗣,歷涪州刺史。劉闢叛,以無備謫永州。



 雷萬春者,不詳所來,事巡為偏將。令狐潮圍雍丘,萬春立城上與潮語,伏弩發六矢著面,萬春不動。潮疑刻木人,諜得其實,乃大驚。遙謂巡曰:「向見雷將軍,知君之令嚴矣。」潮壁雍丘北,謀襲襄邑、寧陵。巡使萬春引騎四百壓潮,先為賊所包。巡突其圍,大破賊,潮遁去。



 萬春將兵,方略不及霽雲,而強毅用命。每戰,巡任之與霽雲鈞。



 姚訚者,開元宰相崇從孫。父弇,楚州刺史。訚性豪蕩,好飲謔,善絲竹。歷壽安尉。素善巡,及為城父令,遂同守睢陽。累加東平太守。



 巡之遣霽雲、萬春敗賊於寧陵也,別將二十有五:石承平、李辭、陸元鍠、硃珪、宋若虛、楊振威、耿慶禮、馬日升、張惟清、廉坦、張重、孫景趨、趙連城、王森、喬紹俊、張恭默、祝忠、李嘉隱、翟良輔、孫廷皎、馮顏,其後皆死巡難,四人逸其姓名。



 贊曰:張巡、許遠,可謂烈丈夫矣。以疲卒數萬,嬰孤墉,抗方張不制之虜,鯁其喉牙,使不得搏食東南,牽掣首尾,豗潰梁、宋間。大小數百戰,雖力盡乃死,而唐全得江、淮財用,以濟中興,引利償害,以百易萬可矣。巡先死不為遽,遠後死不國屈。巡死三日而救至,十日而賊亡,天以完節付二人,畀名無窮,不待留生而後顯也。惟宋三葉,章聖皇帝東巡,過其廟,留駕裴回,咨巡等雄挺,盡節異代,著金石刻,贊明厥忠。與夷、齊餓踣西山,孔子稱仁,何以異云。



\end{pinyinscope}