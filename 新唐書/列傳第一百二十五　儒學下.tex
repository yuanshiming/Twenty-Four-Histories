\article{列傳第一百二十五 儒學下}

\begin{pinyinscope}

 褚無量,字弘度,杭州鹽官人。幼授經於沈子正、曹福,刻意墳典。家濱臨平湖,有龍出學》:「致知在格物,格物而後知至。」東漢鄭玄注:「格,來,人皆走觀,無量尚幼,讀書若不聞,眾異之。尤精《禮》、司馬《史記》。擢明經第,累除國子博士,遷司業兼修文館學士。



 中宗將南郊,詔定儀典。時祝欽明、郭山惲建言皇后為亞獻,無量與太常博士唐紹、蔣欽緒固爭,以為:「郊祀,國大事,其折衷莫如《周禮》。《周禮》冬至祭天圓丘,不以地配,唯始祖為主,亦不以妣配,故後不得與。又《大宗伯》:『凡大祭祀,王後不與,則攝而薦豆籩,徹。』是後不應助祭。又內宰職『大祭祀,後祼獻則贊瑤爵。』祭天無祼,知此乃宗廟祭耳。巾車、內司服,掌後六服與五路,無後祭天之服與路,是後不助祭天也。惟漢有天地合祭,皇后參享事。末代黷神,事不經見,不可為法。」時左僕射韋巨源佐欽明,故無量議格。以母老解官。



 玄宗為太子,復拜國子司業兼侍讀,撰《翼善記》以進,厚被禮答。太子釋奠國學,令講經,建端樹義,博敏而辯,進銀青光祿大夫,錫予蕃渥。及即位,遷左散騎常侍兼國子祭酒,封舒國公。母喪解,詔州刺史薛瑩吊祭,賜物加等。廬墓左,鹿犯所植松柏,無量號訴曰:「山林不乏,忍犯吾塋樹耶?」自是群鹿馴擾,不復棖觸,無量為終身不御其肉。喪除,召復故官。以耆老,隨仗聽徐行,又為設腰輿,許乘入殿中。頻上書陳得失。



 開元五年,帝將幸東都而太廟壞,姚崇建言:「廟本苻堅故殿,不宜罷行。」無量鄙其言,以為不足聽,乃上疏曰:「王者陰盛陽微,則先祖見變。今後宮非御幸者,宜悉出之,以應變異。舉畯良,撙奢靡,輕賦,慎刑,納諫爭,察諂諛,繼絕世,則天人和會,災異訖息。」帝是崇語,車駕遂東。無量又上言:「昔虞舜之狩,秩山川,遍群神。漢孝景祠黃帝橋山,孝武祠舜九疑,高祖過魏祭信陵君墓,過趙封樂毅後,孝章祠桓譚塚。願陛下所過名山、大川、丘陵、墳衍,古帝王、賢臣在祀典者,並詔致祭。自古受命之君,必興滅繼絕,崇德報功。故存人之國,大於救人之災;立人之後,重於封人之墓。願到東都,收敘唐初逮今功臣世絕者,雖在支庶,咸得承襲。」帝納其言,即詔無量祠堯平陽,宋璟祠舜蒲阪,蘇頲祠禹安邑,在所刺史參獻。又求武德以來勛臣苗裔,紹續其封。



 初,內府舊書,自高宗時藏宮中,甲乙叢倒,無量建請繕錄補第,以廣秘籍。天子詔於東都乾元殿東廂部匯整比,無量為之使。因表聞喜尉盧僎、江夏尉陸去泰、左監門率府胄曹參軍王擇從、武陟尉徐楚璧分部讎定。衛尉設次,光祿給食。又詔秘書省、司經局、昭文、崇文二館更相檢讎,採天下遺書以益闕文。不數年,四庫完治。帝詔群臣觀書,賜無量等帛有差。無量又言:「貞觀御書皆宰相署尾,臣位卑不足以辱,請與宰相聯名跋尾。」不從。帝西還,徙書麗正殿,更以脩書學士為麗正殿直學士,比京官預朝會。復詔無量就麗正纂續前功。皇太子及四王未就學,無量以《孝經》、《論語》五通獻帝。帝曰:「朕知之矣。」乃選郗常亨、郭謙光、潘元祚等為太子、諸王侍讀。七年,太子齒胄於學,詔無量升坐講勸,百官觀禮,厚賚賜。卒,年七十五。病困語人,以麗正書未畢為恨。帝聞悼痛,詔宰相曰:「無量,朕師,今其永逝,宜用優典。」於是贈禮部尚書,謚曰文,葬事官給。所撰述百餘篇。歿後有於書殿得講《史記》、《至言》十二篇上之,帝嘆息,以絹五百匹賜其家。



 始,無量與馬懷素為侍讀,後秘書少監康子原、國子博士侯行果亦踐其選,雖賞賚亟加,而禮遇衰矣。



 陸去泰,歷左右補闕內供奉。



 王擇從,京兆人,終汜水令。



 徐楚璧,初應制舉,三登甲科,開元時為中書舍人、集賢院學士,帝屬文多令視草。終中書侍郎,東海縣子。在中書省久,是時李林甫用事,或言計議多所參助。後更名安貞。



 元澹,字行沖,以字顯,後魏常山王素蓮之後。少孤,養於外祖司農卿韋機。及長,博學,尤通故訓。及進士第,累遷通事舍人。狄仁傑器之。嘗謂仁傑曰:「下之事上,譬富家儲積以自資也,脯臘膎胰以供滋膳,參術芝桂以防疾疢。門下充旨味者多矣,願以小人備一藥石,可乎?」仁傑笑曰:「君正吾藥籠中物,不可一日無也。」



 景雲中,授太常少卿。行沖以系出拓拔,恨史無編年,乃撰《魏典》三十篇,事詳文約,學者尚之。初,魏明帝時,河西柳谷出石,有牛繼馬之象。魏收以晉元帝乃牛氏子冒司馬姓,以著石符。行沖謂昭成皇帝名犍,繼晉受命,獨此可以當之。有人破古塚得銅器,似琵琶,身正圓,人莫能辨。行沖曰:「此阮咸所作器也。」命易以木,弦之,其聲亮雅,樂家遂謂之「阮咸」。



 開元初,罷太子詹事,出為岐州刺史,兼關內按察使。自以書生,非彈治才,固辭。入為右散騎常侍、東都副留守。嗣彭王子志謙坐仇人告變,考訊自誣,株蔓數十人,行沖察其枉,列奏見原。四遷大理卿,不樂法家,固謝所居官,改左散騎常侍,封常山縣公。充使檢校集賢,再遷太子賓客、弘文館學士。先是,馬懷素撰書志,褚無量校麗正四部書,業未卒,相次物故。詔行沖並代之。玄宗自注《孝經》,詔行沖為疏,立於學官。以老罷麗正校書事。



 初,魏光乘請用魏征《類禮》列於經,帝命行沖與諸儒集義作疏,將立之學,乃引國子博士範行恭、四門助教施敬本採獲刊綴為五十篇,上於官。於是右丞相張說建言:「戴聖所錄,向已千載,與經並立,不可罷。魏孫炎始因舊書擿類相比,有如鈔綴,諸儒共非之。至征更加整次,乃為訓注,恐不可用。」帝然之,書留中不出。行沖意諸儒間己,因著論自辯,名曰《釋疑》。曰:



 客問主人:「小戴之學,康成之注,魏氏乃有刊易,二經孰優?」主人曰:「《小戴禮》行於漢末,馬融為傳,盧植合二十九篇而為之解,世所不傳。鉤黨獄起,康成於竄伏之中,理紛挐之典,雖存探究,咨謀靡所。具《鄭志》者百有餘科,章句之徒,曾不是省。王肅因之,或多攻詆。而鄭學有孫炎,雖扶鄭義,條例支分,箴石間起,增革百篇。魏氏病群言之冗脞,採眾說之精簡,刊正芟礱,書畢以聞,太宗嘉賞,錄賜儲貳。陛下纂業,宜所循襲,乃制諸儒,甄分舊義。豈悟章句之士,堅持昔言,擯壓不申,疑於知新,果於仍故?」



 客曰:「當局稱迷,傍觀必審,何所為疑而不申列?」答曰:「改易章句,是有五難:漢孔安國注《古文尚書》,族兄臧與書曰:『相如常忿俗儒淫詞冒義,欲撥亂反正而未能也。浮學守株,眾非非正,自古而然,恐此道未信,而獨智為譴。』一也。昔孔季產專古學,有孔扶者與俗浮沈,每誡產曰:『今朝廷率章句內學,君獨脩古義。古義非章句內學,危身之道也,獨善不容於世,君其殆哉!』二也。劉歆好《左氏》,欲建學官,哀帝納之,諸儒遷延不肯置對。歆移書誚讓,諸博士皆忿恨。龔勝時為光祿大夫,見歆議,乃乞骸骨。司空師丹因大發怒,詆歆改亂前志,非毀先帝所立。歆懼,出為五原太守。以君賓之學,公仲之博,猶迫同門朋黨之議,卒令子駿負謗。三也。王肅規鄭玄數千百條,鄭學馬昭詆劾肅短。詔遣博士張融按經問詰,融推處是非,而肅酬對疲於歲時。四也。王粲曰:『世稱伊、雒以東,淮、漢以北,康成一人而已。咸言先儒多闕,鄭氏道備。』粲竊嗟怪,因求所學,得《尚書注》,退思其意,意皆盡矣,所疑猶未諭焉,凡有二篇。王邵曰:『魏、晉浮華,古道湮替,歷載三百,士大夫恥為章句。唯草野生專經自許,不能博究,擇從其善,徒欲父康成,兄子慎,寧道孔聖誤,諱言鄭、服非。』然則鄭、服之外,皆讎矣。五也。夫物極則變,比及百年,當有明哲君子,恨不與吾同世者。道之行廢,必有其時者歟?何遽速近名之嫌邪?」



 俄丐致仕,十七年卒,年七十七,贈禮部尚書,謚曰獻。



 陳貞節,潁川人。開元初,為右拾遺。初,隱、章懷、懿德、節愍四太子並建陵廟,分八署,置官列吏卒,四時祠官進饗。貞節以為非是,上言:「王者制祀,以功德者猶親盡而毀,四太子廟皆別祖,無功於人,而園祠時薦,有司守衛,與列帝侔。金奏登歌,所以頌功德,《詩》曰:『鐘鼓既設,一朝饗之。』使無功而頌,不曰舞詠非度邪?周制:始祖乃稱小廟。未知四廟欲何名乎?請罷卒吏,詔祠官無領屬,以應禮典。古者別子為祖,故有大、小宗。若謂祀未可絕,宜許所後子孫奉之。」詔有司博議。駕部員外郎裴子餘曰:「四太子皆先帝塚嗣,列聖念懿屬而為之享。《春秋》書晉世子曰:『將以晉畀秦,秦將祀予。』此不祀也。又言:『神不歆非類,君祀無乃戾乎!』此有廟也。魯定公元年,立煬宮。煬,伯禽子,季氏遠祖,尚不為限,況天子篤親親以及旁期,誰不曰然?」太常博士段同曰:「四陵廟皆天子睦親繼絕也。逝者錫蘋繁,猶生者之開茅土。古封建子弟,詎皆有功?生無所議,死乃援禮停祠,人其謂何?隱於上,伯祖也,服緦;章懷,伯父也,服期;懿德、節愍,堂昆弟也,服大功。親未盡,廟不可廢。」禮部尚書鄭惟忠等二十七人亦附其言。於是四陵廟惟減吏卒半,它如舊。



 遷太常博士。玄宗奉昭成皇后祔睿宗室,又欲肅明皇后並升焉。貞節奏言:「廟必有配,一帝一後,禮之正也。昭成皇后有太姒之德,宜升配睿宗;肅明皇后既非子貴,宜在別廟。周人『奏夷則,歌小呂,以享先妣』。先妣,姜嫄也,以生后稷,故特立廟曰閟宮。晉簡文帝鄭宣皇后不配食,築宮於外,以歲時致享。肅明請準周姜嫄、晉宣後,納主別廟,時享如儀。」於是,留主儀坤廟,詔隸太廟,毋置官屬。貞節又與博士蘇獻上言:「睿宗於孝和,弟也。按賀循說,兄弟不相為後。故殷盤庚不序陽甲,而上繼先君;漢光武不嗣孝成,而上承元帝;晉懷帝繼世祖,不繼惠帝。故陽甲、孝成出為別廟。」又言:「兄弟共世,昭穆位同,則毀二廟。有天下者,從禰而上事七廟,尊者所統廣,故及遠祖。若容兄弟,則上毀祖考,天子不得全事七世矣。請以中宗為別廟,大祫則合食太祖。奉睿宗繼高宗,則祼獻永序。」詔可。乃奉中宗別廟,升睿宗為第七室。



 五年,太廟壞,天子舍神主太極殿,營新廟,素服避正寢,三百不朝,猶幸東都。伊闕男子孫平子上書曰:「乃正月太廟毀,此躋二帝之驗也。《春秋》:『君薨,卒哭而祔,祔而作主,特祀於主,烝嘗禘於廟。』今皆違之。魯文公之二年,躋僖於閔上,後太室壞,《春秋》書其災,說曰:『僖雖閔兄,嘗為之臣,臣居君上,是謂失禮,故太室壞。』且兄臣於弟,猶不可躋;弟嘗臣兄,乃可躋乎?莊公薨,閔公二年而禘,《春秋》非之。況大行夏崩,而太廟冬禘,不亦亟乎?太室尊所,若曰魯自是陵夷,墮周公之祀。太廟今壞,意者其將陵夷,墮先帝之祀乎?陛下未祭孝和,先祭太上皇,先臣後君。昔躋兄弟上,今弟先兄祭。昔太室壞,今太廟毀,與《春秋》正同,不可不察。武後篡國,孝和中興有功,今內主別祠,不得立於世,亦已薄矣。夫功不可棄,君不可下,長不可輕。且臣繼君,猶子繼父。故禹不先鯀,周不先不窋,宋、鄭不以帝乙、厲王不肖,猶尊之也,況中興邪?晉太康時,宣帝廟地陷梁折,又三年,太廟殿陷而及泉,更營之,梁又折。天之所譴,非必朽而壞也。晉不承天,故及於亂。臣謂宜遷孝和還廟,何必違禮,下同魯、晉哉?」帝異其言,詔有司復議。貞節、獻與博士馮宗質之曰:「天子七廟,三昭三穆,與太祖而七。父昭子穆,兄弟不與焉。殷自成湯至帝乙十二君,其父子世六。《易乾鑿度》曰:『殷之帝乙六世王。』則兄弟不為世矣。殷人六廟:親廟四,並湯而六。殷兄弟四君,若以為世,方上毀四室,乃無祖禰,是必不然。古者繇禰極祖,雖迭毀迭遷,而三昭穆未嘗闕也。《禮》:大宗無子,則立支子。又曰:『為人後者為之子。』無兄弟相為後者,故舍至親,取遠屬。父子曰繼,兄弟曰及,兄弟不相入廟,尚矣。借有兄弟代立承統,告享不得稱嗣子、嗣孫,乃言伯考、伯祖,何統緒乎?殷十二君,惟三祖、三宗,明兄弟自為別廟。漢世祖列七廟,而惠帝不與。文、武子孫昌衍,文為漢太宗。晉景帝亦文帝兄,景絕世,不列於廟。及告謚世祖,稱景為從祖。今謂晉武帝越崇其父,而廟毀及亡,何漢出惠帝而享世長久乎?七廟、五廟,明天子、諸侯也;父子相繼,一統也;昭穆列序,重繼也。禮,兄弟相繼,不得稱嗣子,明睿宗不父孝和,必上繼高宗者。偶室於廟,則為二穆,於禮可乎?禮所不可,而使天子旁紹伯考,棄己親正統哉?孝和中興,別建園寢,百世不毀,尚何議哉?平子猥引僖公逆祀為比,殊不知孝和升新寢,聖真方祔廟,則未嘗一日居上也。」帝語宰相召平子與博士詳論。博士護前言,合軋平子。平子援經辯數分明,獻等不能屈。蘇頲右博士,故平子坐貶都城尉。然諸儒以平子孤挺,見迮於禮官,不平。帝亦知其直,久不決,然卒不復中宗於廟。



 明年,帝將大享明堂,貞節惡武后所營,非古所謂「木不鏤、土不文」之制,乃與馮宗上言:「明堂必直丙巳,以憲房、心布政,太微上帝之所。武后始以乾元正寢占陽午地,先帝所以聽政,故毀殿作堂。撤之日,有音如雷,庶民嘩訕,以為神靈不悅。堂成,災火從之。後不脩德,俄復營構,殫用極侈,詭禳厥變,又欲嚴配上帝,神安肯臨?且密邇掖廷,人神雜擾,是謂不可放物者也。二京上都,四方是則。天子聽政,乃居便坐,無以尊示群臣。願以明堂復為乾元殿,使人識其舊,不亦愈乎?」詔所司詳議。刑部尚書王志愔等僉謂:「明堂瑰怪不法,天燼之餘,不容大享。請因舊循制,還署乾元正寢。正、至,天子御以朝會。若大享,復寓圜丘。」制曰可。貞節以壽卒。



 施敬本,潤州丹陽人。開元中,為四門助教。玄宗將封禪,詔有司講求典儀。舊制,盥手、洗爵,皆侍中主之;詔祀天神,太祝主之。敬本上言曰:「周制,大宗伯鬱人,下士二,掌祼事。漢無鬱人,用近臣。漢世侍中微甚,籍孺、閎孺等幸臣為之。後漢邵闔自侍中遷步兵校尉,秩千石,其職省起居,執虎子,蓋褻臣也。今侍中位宰相,非鬱人比。祝者薦主人意於神,非賤職也。古二君相見,卿為上儐,況天人際哉!周太祝,下大夫二,上士四。下大夫,今郎中、太常丞之比;上士,員外郎、博士之比。漢太祝令秩六百石,今太祝乃下士。以下士接天,以大臣奉天子,輕重不倫,非禮也。舊制,謁者引太尉升壇。謁者位下,升壇禮重。漢尚書御史屬,有謁者僕射一,秩六百石,銅印青綬;謁者三十五,以郎中滿歲稱給事中,未滿歲稱謁者。光祿勛屬,有謁者,掌賓贊,員七十,秩比六百石。則古謁者名秩差異等,今謁者班微,循空名,忘實事,非所以事天也。」帝詔中書令張說引敬本熟悉其議,故侍中、祝、謁者,視禮輕重,以它官攝領。



 敬本以太常博士為集賢院脩撰。逾年,遷右補闕、秘書郎,卒。



 盧履冰,幽州範陽人,元魏都官尚書義僖五世孫。開元五年,仕歷右補闕。建言:「古者父在為母期,徹靈而心喪。武后始請同父三年,非是,請如禮便。」玄宗疑之,又以舅、嫂叔服未安,並下百官議。刑部郎中田再思曰:「會禮之家比聚訟。循古不必是,而行今未必非。父在為母三年,高宗實行之,著令已久。何必乖先帝之旨,閡人子之情,愛一期服於其親,使與伯叔母、姑姊妹同?嫂叔、舅甥服,太宗實制之,閱百年無異論,不可改。」履冰因言:「上元中,父在為母三年,後雖請,未用也,逮垂拱始行之。至有祖父母在而子孫婦沒,行服再期,不可謂宜。禮,女子無專道,故曰『家無二尊』。父在為母服期,統一尊也。今不正其失,恐後世復有婦奪夫之敗,不可不察。」書留未下。履冰即極陳:「父在為母立幾筵者一期,心喪者再期,父必三年而後娶,以達子之志。夫聖人豈蔑情於所生?固有意於天下。昔武后陰儲篡謀,豫自光崇,升期齊,抗斬衰,俄而乘陵唐家,以啟釁階。孝和僅得反正,韋氏復出,■殺天子,幾亡宗社。故臣將以正夫婦之綱,非特母子間也。議者或言:『降母服,非《詩》所謂罔極者,而又與伯叔母、姑姊妹等。且齊、斬已有升降,則歲月不容異也。』此迂生鄙儒,未習先王之旨,安足議夫禮哉?罔極者,春秋祭祀,以時思之,謂君子有終身之憂,何限一期、二期服哉?聖人之於禮,必建中制,使賢不肖共成文理而後釋,彼伯叔、姑姊,烏有筵杖之制、三年心喪乎?母齊父斬,不易之道也。」左散騎常侍元行沖議曰:「古緣情制服:女天父,妻天夫,斬衰三年,情禮俱盡者,因心立極也。妻喪杖期,情禮俱殺者,遠嫌疑,尊乾道也。為嫡子三年斬衰而不去官,尊祖重嫡,崇其禮,殺其情也。孝莫大於嚴父,故父在為母免官,齊需而期,心喪三年,情已申而禮殺也,自堯、舜、周公、孔子所同。而今舍尊厭之重,虧嚴父之義,謂之禮,可乎?姨兼從母之名,以母之女黨,加以舅服,不為無禮。嫂叔不服,則遠嫌也。請據古為適。」帝弗報。是時言喪服,各以所見奮,交口紛騰。七年,乃下詔:「服紀一用古制。」自是人間父在為母服,或期而禫,禫而釋,心喪三年;或期而禫,終三年;或齊衰三年。



 後履冰以官卒。



 王仲丘,沂州瑯邪人。祖師順,仕高宗,議漕輸事有名當時,終司門郎中。仲丘開元中歷左補闕內供奉、集賢脩撰、起居舍人。



 時典章差駁,仲丘欲合《貞觀》、《顯慶》二禮,據「有其舉之,莫可廢之」之誼,即上言:「《貞觀禮》,正月上辛,祀感帝於南郊。《顯慶禮》:祀昊天上帝於圓丘以祈穀。臣謂《詩》『春夏祈穀於上帝』,《禮》『上辛祈穀於上帝』,則上帝當昊天矣。鄭玄曰:『天之五帝遞王,王者必感一以興。玭夏正月祭所生於郊,以其祖配之,因以祈穀。』感帝之祀,《貞觀》用之矣。請因祈穀之壇,遍祭五方帝。五帝者,五行之精,九穀之宗也。請二禮皆用。《貞觀禮》,雩祀五方上帝、五人帝、五官於南郊。《顯慶禮》,祀昊天上帝於圓丘。臣謂雩上帝,為百穀祈甘雨,故《月令》:『大雩帝,用盛樂。』鄭玄說:『帝,上帝也,乃天別號。祀於圓丘,尊天位也。』《顯慶》祀昊天與《月令》合,而《貞觀》嘗祀五帝矣,請二禮皆用。《貞觀禮》,季秋祀五方帝、五官於明堂。《顯慶禮》,祀昊天上帝於明堂。臣謂周郊祀后稷以配天,宗祀文王於明堂以配上帝。先儒以天為感帝,引太微五帝著之,上帝則屬之昊天。鄭玄稱《周官》旅上帝,祀五帝,各文而異禮,不容並而為一。故於《孝經》天、上帝,申之曰:『上帝亦天也。』神無二主,但異其處,以避后稷。今《顯慶》享上帝,合於《經》,然《貞觀》嘗祀五方帝矣。請二禮皆用。」詔可。



 遷禮部員外郎。卒,贈秘書少監。



 康子元,越州會稽人。仕歷獻陵令。開元初,詔中書令張說舉能治《易》、《老》、《莊》者,集賢直學士侯行果薦子元及平陽敬會真於說,說藉以聞,並賜衣幣,得侍讀。子元擢累秘書少監,會真四門博士,俄皆兼集賢侍講學士。



 玄宗將東之泰山,說引子元、行果、徐堅、韋縚商裁封禪儀。初,高宗之封,中書令許敬宗議:「周人尚臭,故前祭而燔柴。」說、堅、子元白奏:「《周官》:樂六變,天神降。是降神以樂,非緣燔也。宋、齊以來,皆先嚌福酒,乃燎。請先祭後燔,如《貞觀禮》便。」行果與趙冬曦議,以為:「先燎降神,尚矣。若祭已而燔,神無由降。」子元議挺不徙。說曰:「康子獨出蒙輪,以當一隊邪?」議未判,說請決於帝,帝詔後燔。



 乘輿自岱還,減從官,先次東都,唯子元、毋煚、韋述以學士從。久乃徙宗正少卿,以疾授秘書監,致仕。卒,贈汴州刺史。帝嘗制贊賜說、子元,命工圖其象,詔冬曦、述、煚分為傅。



 行果者,上谷人,歷國子司業,侍皇太子讀。卒,贈慶王傅。



 始,行果、會真及長樂馮朝隱同進講。朝隱能推索《老》、《莊》秘義,會真亦善《老子》,每啟篇,先薰盥乃讀。帝曰:「我欲更求善《易》者,然無賢行果云。」朝隱終太子右諭德,會真太學博士。



 趙冬曦,定州鼓城人。進士擢第,歷左拾遺。神龍初,上書曰:「古律條目千餘。隋時奸臣侮法,著律曰:『律無正條者,出罪舉重以明輕,入罪舉輕以明重。』一辭而廢條目數百。自是輕重沿愛憎,被罰者不知其然,使賈誼見之,慟哭必矣。夫法易知,則下不敢犯而遠機阱;文義深,則吏乘便而朋附盛。律、令、格、式,謂宜刊定科條,直書其事。其以準加減比附、量情及舉輕以明重、不應為之類,皆勿用。使愚夫愚婦相率而遠罪,犯者雖貴必坐。律明則人信,法一則主尊。」當時稱是。



 開元初,遷監察御史,坐事流岳州。召還復官,與秘書少監賀知章、校書郎孫季良、大理評事咸廙業入集賢院脩撰。是時,將仕郎王嗣琳、四門助教範仙廈為校勘,翰林供奉呂向、東方顥為校理。未幾,冬曦知史官事,遷考功員外郎。逾年,與季良、廙業、知章、呂向皆為直學士。冬曦俄遷中書舍人內供奉,以國子祭酒卒。



 冬曦性放達,不屑世事。兄夏日,弟和璧、安貞、居貞、頤貞、匯貞,皆擢進士第。安貞給事中,居貞吳郡採訪使,頤貞安西都護。居貞子昌,別傳。



 王嗣琳以太子校書郎罷。東方顥上書忤旨,左遷高安丞。廙業亦坐事左遷餘杭令。仙廈善講論,後為道士。



 開元集賢學士,又有尹愔、陸堅、鄭欽說、盧僎名稍著。



 尹愔,秦州天水人。父思貞,字季弱。明《春秋》,擢高第。嘗受學於國子博士王道珪,稱之曰:「吾門人多矣,尹子叵測也。」以親喪哀毀。除喪,不仕。左右史張說、尹元凱薦為國子大成。每釋奠,講辨三教,聽者皆得所未聞。遷四門助教,撰《諸經義樞》、《續史記》皆未就。夢天官、麟臺交闢,寤而會親族敘訣,二日卒,年四十。



 愔博學,尤通老子書。初為道士,玄宗尚玄言,有薦愔者,召對,喜甚,厚禮之,拜諫議大夫、集賢院學士,兼脩國史,固辭不起。有詔以道士服視事,乃就職,顓領集賢、史館圖書。開元末,卒,贈左散騎常侍。



 陸堅,河南洛陽人。初為汝州參軍,以友婿李慈伏誅,貶涪州參軍,再遷通事舍人。有詔起復,遣中官敦諭,不就。以給事中兼學士。善書。初名友悌,玄宗嘉其剛正,更賜名。從封泰山,封建安男。帝待之甚厚,圖形禁中,親制贊。以秘書監卒,年七十一,贈吏部尚書,謚曰懿。



 郭欽說,後魏濮陽太守敬叔八世孫。開元初,繇新津丞請試五經,擢第,授鞏縣尉、集賢院校理。歷右補闕內供奉。通歷術,博物。初,梁太常任昉大同四年七月於鐘山壙中得銘曰:「龜言土,蓍言水,甸服黃鐘啟靈址。瘞在三上庚,墮遇七中己。六千三百浹辰交,二九重三四百圮。」當時莫能辨者,因藏之,戒諸子曰:「世世以銘訪通人,有知之者,吾死無恨。」昉五世孫升之,隱居商洛,寫以授欽說。欽說出使,得之於長樂驛,至敷水三十里而悟曰:「卜宅者廋葬之歲月,而先識墓圮日辰。甸服,五百也,黃鐘十一也,繇大同四年卻求漢建武四年,凡五百一十一年。葬以三月十日庚寅,三上庚也。圮以七月十二日己巳,七中己也。浹辰,十二也,建武四年三月至大同四年七月,六千三百一十二月,月一交,故曰六千三百浹辰交。二九,十八也。重三,六也。建武四年三月十日,距大同四年七月十二日,十八萬六千四百日,故曰二九重三四百圮。」升之大驚,服其智。



 欽說雅為李林甫所惡,韋堅死,欽說時位殿中侍御史,常為堅判官,貶夜郎尉,卒。



 子克鈞,為都官郎中。吐蕃圍靈州,軍餉匱竭,德宗以克鈞為靈、夏二州運糧使,轉米峙塞下,守者遂安。



 盧僎,吏部尚書從願三從父也。自聞喜尉為學士,終吏部員外郎。



 兄俌,中宗時歷右補闕。默啜入寇,敗沙吒忠義,詔百官陳破賊勝策,獨俌上疏以為:「治內可以及外,賞罰明則士盡節。鳴沙之役,主將先遁,中軍猶能死戰。正法紀功,則戎行可勸。若忠義,騎將材,不可當大任。宜因古法,募人徙邊,免行役,次廬伍,明教令,賞虜獲,近戰則守家,遠戰則利貨。購辯勇,強諸蕃,以圖攻取。擇邊州刺史,搜乘積粟,謹烽燧以備守。」中宗善其言,然無施行者。俌終秘書少監。



 啖助,字叔佐,趙州人,後徙關中。淹該經術。天寶末,調臨海尉、丹陽主簿。秩滿,屏居,甘足疏糗。



 善為《春秋》,考三家短長,縫綻漏闕,號《集傳》,凡十年乃成,復攝其綱條為例統。其言孔子脩《春秋》意,以為:「夏政忠,忠之敝野;商人承之以敬,敬之敝鬼;周人承之以文,文之敝僿。救僿莫若忠。夫文者,忠之末也。設教於本,其敝且末;設教於末,敝將奈何?武王、周公承商之敝,不得已用之。周公沒,莫知所以改,故其敝甚於二代。孔子傷之曰:『虞、夏之道,寡怨於民;商、周之道,不勝其敝!』故曰:『後代雖有作者,虞帝不可及已。』蓋言唐、虞之化,難行於季世,而夏之忠,當變而致焉。故《春秋》以權輔用,以誠斷禮,而以忠道原情雲。不拘空名,不尚狷介,從宜救亂,因時黜陟。古語曰:『商變夏,周變商,春秋變周。』而公羊子亦言:『樂道堯、舜之道,以擬後聖。』是知《春秋》用二帝、三王法,以夏為本,不壹守周典明矣。」又言:「幽、厲雖衰,《雅》未為《風》。逮平王之東,人習余化,茍有善惡,當以周法正之。故斷自平王之季,以隱公為始,所以拯薄勉善,救周之敝,革禮之失也。」助愛公、穀二家,以左氏解義多謬,其書乃出於孔氏門人。且《論語》孔子所引,率前世人老彭、伯夷等,類非同時;而言「左丘明恥之,丘亦恥之」。丘明者,蓋如史佚、遲任者。又《左氏傳》、《國語》,屬綴不倫,序事乖剌,非一人所為。蓋左氏集諸國史以釋《春秋》,後人謂左氏,便傅著丘明,非也。助之鑿意多此類。



 助門人趙匡、陸質,其高弟也。助卒,年四十七。質與其子異裒錄助所為《春秋集注總例》,請匡損益,質纂會之,號《纂例》。匡者,字伯循,河東人,歷洋州刺史,質所稱為趙夫子者。



 大歷時,助、匡、質以《春秋》,施士丐以《詩》,仲子陵、袁彞、韋彤、韋荅以《禮》,蔡廣成以《易》,強蒙以《論語》,皆自名其學,而士丐、子陵最卓異。



 士丐,吳人,兼善《左氏春秋》,以二經教授。繇四門助教為博士,秩滿當去,諸生封疏乞留,凡十九年,卒於官。弟子共葬之。士丐撰《春秋傳》,未甚傳。後文宗喜經術,宰相李石因言士丐《春秋》可讀。帝曰:「朕見之矣,穿鑿之學,徒為異同,但學者如浚井,得美水而已,何必勞苦旁求,然後為得邪?」



 子陵,蜀人,好古學,舍峨眉山。舉賢良方正,擢太常博士,通後蒼、大小戴《禮》。有司請正太祖東向位,而遷獻、懿二主。子陵議藏主德明、興聖廟,其言典正。後異論紛洄,復為《通難》示諸儒,諸儒不能詘。久之,典黔中選補,乘傳過家,西人以為榮。終司門員外郎。子陵以文義自怡,及亡,其家所存,惟圖書及酒數斛而已。



 贊曰:《春秋》、《詩》、《易》、《書》,由孔子時師弟子相傳,歷暴秦,不斷如系。至漢興,刬挾書令,則儒者肆然講授,經典浸興。左氏與孔子同時,以《魯史》附《春秋》作《傳》,而公羊高、穀梁赤皆出子夏門人。三家言經,各有回舛,然猶悉本之聖人,其得與失蓋十五,義或繆誤,先儒畏聖人,不敢輒改也。啖助在唐,名治《春秋》,摭訕三家,不本所承,自用名學,憑私臆決,尊之曰「孔子意也」,趙、陸從而唱之,遂顯於時。嗚呼!孔子沒乃數千年,助所推著果其意乎?其未可必也。以未可必而必之,則固;持一己之固而倡茲世,則誣。誣與固,君子所不取。助果謂可乎?徒令後生穿鑿詭辨,詬前人,舍成說,而自為紛紛,助所階已。



 韋彤,京兆人。四世從祖方質為武后時宰相。彤名治《禮》,德宗時為太常博士。



 先此,天寶中,詔尚食朔望進食太廟,天子使中人侍祠,有司不與也。貞元十二年,帝始詔朔望食,畀宗正、太常合供。於是彤與博士裴堪議曰:「禮,宗廟朔望不祭,園寢則有之。貞觀、開元間,在禮若令,不敢變古。天寶中,始有進食事,殆王璵緣生事亡,用燕具褻饌,參瀆禮薦,不可示遠。傳曰:『祭非外至,生於心者也。』是故聖人等牲牢,布籩豆,昆蟲、草木可薦者,莫不咸在,所以享宗廟,交神明,全孝敬也。潔膳羞,八珍百品,可嗜之饌,美膬甘旨,謂之褻味,所以燕賓客,接人情,示慈惠也。是則薦與宴,聖人判為二物,不可亂也。今若熟饔而享,非以異為敬之意。且祭不欲數,亦不欲疏,感時致享,以制中也。今園寢月二祭,不為疏,廟歲五享,不為數,有司奉承,得盡其恭。若又加盛饌於朔望,是失禮之中,有司不得盡其恭也。故王者稽古,弗敢以孝思之極而溢禮,弗敢以肴品之多而剩味。願罷天寶所增,奉園寢以珍,奉宗廟以禮,兩得所宜。」帝曰:「是禮先帝裁定,遽更之,其謂朕何?徐議其可。」而朔望食卒不廢。



 會昭陵寢宮為原火延燔,而客祭瑤臺佛寺。又故宮在山上,乏水泉,作者憚勞,欲即行宮作寢,詔宰相百官議。吏部員外郎楊於陵議曰:「園寢非三代制,自秦、漢以來,附陵置寢,或遠若邇,則無聞焉。韋玄成等議園陵,於興廢初無適語。且寢宮所占,在柏城中,距陵不遠,使諸陵之寢,皆有區限,故不可徙;若止柏城,則故寢已燔,行宮已久,因以治飾,亦復何嫌?或曰:『太宗創業,寢宮不輒易。』是不然。夫陵域宅神,神本靜,今大興荒廢,囂役密邇,非幽穸所安,改之便。」彤曰:「先王建都立邑,不利則為之遷,況有故邪?今文寢災,徙而宮之,非無故也。神安於徙,因而建寢,於禮至順。又它陵皆在柏城,隨便營作,不越封兆,力省易從。」帝重改先帝制,還宮山顛。



 彤卒後,武宗會昌五年,詔京城不許群臣作私廟。宰相李德裕等引彤所議:「古制:廟必中門之外,吉兇皆告,以親而尊之,不自專也。今俾立廟京外,不能得其意於禮。宮之南九坊,三坊曰圍外,地荒左,立廟無嫌;餘六坊可禁。」詔不許,聽準古即居所立廟。



 陳京,字慶復,陳宜都王叔明五世孫。父兼,為右補闕、翰林學士。京善文辭,常袞稱之,妻以兄子。擢進士第,遷累太常博士。



 德宗在奉天,聞段秀實為賊所害,七日不朝。宰相以為「方多難時,不宜壅萬機,天下其謂何?」京曰:「丞相之言非也。夫褒大節,恤賢臣,天下所以安,況卓卓特異者乎?」帝曰:「善。」還京師,擢左補闕。帝以盧巳為饒州刺史,京與趙需、裴佶、宇文炫、盧景亮、張薦共劾:「巳輔政要位,大臣逾時月不得對,百官懍懍常若兵在頸。陛下復用之,奸賊唾掌復興。」帝不聽。京等爭尤確,帝大怒,左右闢易,諫者稍引卻。京正色曰:「需等毋遽退!」極道不可,以死請,巳遂廢。帝之立,迎訪太后,久不得,意且怠。京密白:「第遣使物色以求。」帝大悟,終代不敢置。



 初,玄宗、肅宗既附室,遷獻、懿二祖於西夾室,引太祖位東向。禮儀使於休烈議:「獻、懿屬尊於太祖,若合食,則太祖位不得正,請藏二祖神主,以太宗、中宗、睿宗、肅宗從世祖南向,高宗、玄宗從高祖北向。」禘祫不及二祖,凡十八年。建中初,代宗喪畢,當大祫。京以太常博士上言:「《春秋》之義,毀廟之主陳於太祖,未毀廟之主合食於祖,無毀廟遷主不享之言。唐家祀制與周異,周以後稷為始封祖,而毀主皆在後稷下,故太祖東向,常統其尊。司馬晉以高皇、太皇、征西四府君為別廟,大禘祫則正太祖位,無所屈。別廟祭高、太以降,所以敘親也。唐家宜別為獻、懿二祖立廟,禘祫則祭,太祖遂正東向位。德明、興聖二帝,向已有廟,則藏祔二祖為宜。」



 詔百官普議。禮儀使、太子少師顏真卿曰:「今議者有三:一謂獻、懿親遠而遷,不當祫,宜藏主西室;二謂二祖宜祫食,與太祖並昭穆,闕東向位;三謂引二祖祫禘,即太祖永不得全其始,宜以二主祔德明廟。雖然,於人神未厭也。景帝既受命始封矣,百代不遷矣,而又配天,尊無與上,至禘祫時,蹔屈昭穆以申孝尊先,實明神之意,所以教天下之孝也。況晉蔡謨等有成議,不為無據。請大祫享奉獻主東向,懿主居昭,景主居穆,重本尚順,為萬代法。夫祫,合也。有如別享德明,是乃分食,非合食也。」時議者舉然。於是還獻、懿主祫於廟,如真卿議。



 貞元七年,太常卿裴鬱上言:「商、周以卨、稷為祖,上無餘尊,故合食有序。漢受命,祖高皇帝,故太上皇不以昭穆合食。魏祖武帝,晉祖宣帝,故高皇、處士、征西等君,亦不以昭穆合食。景皇帝始封唐,唐推祖焉,而獻、懿親盡廟遷,猶居東向,非禮之祀,神所不享。願下群臣議。」於是太子左庶子李嶸等上言:「謹按晉孫欽議:『太祖以前,雖有主,禘祫所不及;其所及者,太祖後未毀已升藏於二祧者,故雖百代及之。』獻、懿在始封前,親盡主遷,上擬三代,則禘祫所不及。太祖而下,若世祖,則《春秋》所謂『陳於太祖』者。漢議罷郡國廟,丞相韋玄成議:『太上皇、孝惠親盡宜毀。太上主宜瘞於園,惠主遷高廟。』太上皇在太祖前,主瘞於園,不及禘祫,獻、懿比也。惠遷高廟,在太祖後,而及禘祫,世祖比也。魏明帝遷處士主置園邑,歲時以令丞奉薦;東晉以征西等祖遷入西除,同謂之祧,皆不及祀。故唐初下訖開元,禘祫猶虛東向位。洎立九廟,追祖獻、懿,然祝於三祖不稱臣。至德時,復作廟,遂不為弘農府君主,以祀不及也。廣德中,始以景皇帝當東向位,以獻、懿兩主親盡,罷祫而藏。顏真卿引蔡謨議,復奉獻主東向,懿昭景穆。不記謨議晉未嘗用,而唐一王法容可準乎?臣等謂嘗、禘、郊、社無二尊,瘞、毀、遷、藏,各以義斷。景皇帝已東向,一日改易,不可謂禮,宜復藏獻、懿二主於西室,以本《祭法》『遠廟為祧,去祧而壇,去壇而墠,壇、墠,有禱祭,無禱止』之義。太祖得正,無所屈。」



 吏部郎中柳冕等十二人議曰:「天子以受命之君為太祖,諸侯以始封之主為祖,故自太祖、祖以下,親盡迭毀。洎秦滅學,漢不暇禮,晉失宋因,故有連王廟之制,有虛太祖之位。且不列昭穆,非所謂有序;不建迭毀,非所謂有殺;連王廟,非所謂有別;虛太祖位,非所謂一尊。此禮所由廢也。《傳》曰:『父為士,子為天子,祭以天子,葬以士。』今獻、懿二祖,在唐未受命時,猶士也。故高祖、太宗以天子之禮祭之,而不敢奉以東向位。今而易之,無乃亂先帝序乎?周有天下,追王太王、王季以天子禮;及其祭,則親盡而毀。漢有天下,尊太上皇以天子之禮;及祭也,親盡而毀。唐家追王獻、懿二祖以天子禮;及其祭也,親盡而毀,復何所疑?《周官》有先公之祧、先王之祧。先公遷主,藏后稷之廟,其周未受命之祧乎?先王遷主,藏文、武之廟,其周已受命之祧乎?故有二祧,所以異廟也。今自獻而下,猶先公也;自景而下,猶先王也。請別廟以居二祖,則行周道,復古制,便。」



 工部郎中張薦等請自獻而降,悉入昭穆,虛東向位。司勛員外郎裴樞曰:「《禮》:『親親故尊祖,尊祖故敬宗,敬宗故收族,收族故宗廟嚴,宗廟嚴故社稷重。』太祖之上,復追尊焉,則尊祖之義乖。太廟之外,別祭廟焉,則社稷不重。漢韋玄成請瘞主於園,晉虞喜請瘞廟兩階間。喜據左氏自證曰:『先王日祭祖、考,月祀曾、高,時享及二祧,歲祫及壇墠,終禘及郊宗石室,是謂郊宗之祖。』喜請夾室中為石室以處之,是不然。何者?夾室所以居太祖下,非太祖上藏主所居。未有卑處正、尊居傍也。若建石室於園寢,安遷主,採漢、晉舊章,祫禘率一祭,庶乎《春秋》得變之正。」



 是時,京以考功員外郎又言:「興聖皇帝則獻之曾祖,懿之高祖。以曾孫祔曾高之廟,人情大順也。」京兆少尹韋武曰:「祫則大合,禘則序祧。當祫之歲,常以獻東向,率懿而後以昭穆極親親。及禘,則太祖筵於西,列眾主左右,於是太祖不為降,獻無所厭。」時諸儒以左氏「子齊聖,不先父食」,請迎獻主權東向,太祖暫還穆位。同官尉仲子陵曰:「所謂不先食者,丘明正文公逆祀。儒者安知夏後世數未足時,言禹不先鯀乎?魏、晉始祖率近,始祖上皆有遷主。引《閟宮》詩,則永閟可也。因虞主,則瘞園可也。緣遠祧,則築宮可也。以太祖實卑,則虛位可也。然永閟與瘞園,臣子所不安。若虛正位,則太祖之尊無時而申。請奉獻、懿二祖遷於德明、興聖廟為順。或曰二祖別廟,非合食。且德明、興聖二廟禘祫之年,皆有薦饗,是已分食,奚獨疑二祖乎?」



 國子四門博士韓愈質眾議,自申其說曰:「一謂獻、懿二主宜永藏夾室,臣不謂可。且禮,祫祭,毀主皆合食。今藏夾室,至祫得不食太廟乎?若二祖不豫,不謂之合矣。二謂兩主宜毀而瘞之,臣不謂可。禮,天子七廟、一壇、一墠,遷主皆藏於祧,雖百代不毀。祫則太廟享焉。魏晉以來,始有毀瘞之議,不見於經。唐家立九廟,以周制推之,獻、懿猶在壇墠,可毀瘞而不禘祫乎?三謂二祖之主宜各遷諸陵,臣不謂可。二祖享太廟二百年,一日遷之,恐眷顧依違,不即享於下國。四謂宜奉主祔興聖廟而不禘祫,臣不謂可。禮,『祭如在』。景皇帝雖太祖,於獻、懿,子孫也。今引子東向,廢父之祭,不可謂典。五謂獻、懿宜別立廟京師,臣不謂可。凡禮有降有殺,故去廟為祧,去祧為壇,去壇為墠,去墠為鬼,漸而遠者,祭益希。昔魯立煬宮,《春秋》非之,謂不當取已毀之廟、既藏之主,復築宮以祭。今議正同,故臣皆不謂可。古者殷祖玄王,周祖後稷,太祖之上,皆自為帝。又世數已遠,不復祭之,故始祖得東向也。景皇帝雖太祖,於獻、懿,子孫也。當禘祫,獻祖居東向位,景從昭若穆,是祖以孫尊,孫以祖屈,神道人情,其不相遠。又常祭眾,合祭寡,則太祖所屈少,而所伸多。與其伸孫尊,廢祖祭,不以順乎?」



 冕又上《禘祫議證》十四篇,帝詔尚書省會百官、國子儒官,明定可否。左司郎中陸淳奏:「按《禮》及諸儒議,復太祖之位,正也。太祖位正,則獻、懿二主宜有所安。今議者有四:曰藏夾室,曰置別廟,曰各遷於園,曰祔興聖廟。臣謂藏夾室,則享獻無期,非周人藏二祧之義;置別廟,論始曹魏,《禮》無傳焉,司馬晉議而不用;遷諸園,亂宗廟之制。唯祔興聖廟,禘若祫一祭,庶乎得禮。」帝依違未決也。



 十九年,將禘祭,京復奏禘祭大合祖宗,必尊太祖位,正昭穆。請詔百官議。尚書左僕射姚南仲等請奉獻、懿主祔德明、興聖廟。鴻臚卿王權、申衍之曰:「周人祖文王,宗武王,故《詩·清廟》章曰:『祀文王也。』胡不言太王、王季?則太王、王季而上,皆祔後稷,故清廟得祀文王也。太王、王季之尊,私禮也;祔後稷廟,不敢以私奪公也。古者先王遷廟主,以昭穆合藏於祖廟。獻、懿主宜祔興聖廟,則太祖東向得其尊,獻、懿主歸得其所。」是時,言祔興聖廟什七八,天子尚猶豫未刪定。至是,群臣稍顯言二祖本追崇,非有受命開國之鴻構;又權根援《詩》、《禮》明白。帝泮然,於是定遷二祖於興聖廟,凡禘祫一享。詔增廣興聖二室。會祀日薄,廟未成,張繒為室,內神主廟垣間,奉興聖、德明主居之。廟成而祔。自是景皇帝遂東向。



 京自博士獻議,彌二十年乃決,諸儒無後言。帝賜京緋衣、銀魚。昭陵寢占山上,宦侍憚輓汲乏,請更其所,宰相未能抗。京曰:「此太宗之志,其儉足以為後世法,不可改。」議者多附宦人,帝曰:「京議善。」卒不徙。帝器京,謂有宰相才,欲用之。會病狂易,自刺弗殊,又言中書舍人崔邠、御史中丞李汶訕己,帝使詰辨無狀,然猶自考功員外再遷給事中,皆兼集賢殿學士。帝疑京為忌者中傷,中人問賚相繼。後對延英,帝諭遣,京沮駭走出,罷為秘書少監,卒。



 初,帝討李希烈,財用屈,京與戶部侍郎趙贊請稅民屋架,籍賈人貲力,以率貸之。憲宗嘗問宰相李吉甫:「我在籓邸,聞德宗播遷梁、漢,久乃復,誰實召亂,為我言之。」對曰:「德宗始即位,躬行慈儉,經崔祐甫輔政,四方企望至治。祐甫歿,宰相非其人,奸佞營蠱,謂河北叛臣可以力服,甘語先入,主聽惑焉。而陳京、趙贊為帝稅屋架,貸賈緡,內怨外忿,身及大亂。咎興信宵人,剝下佐上,賴天之靈,敗不抵亡。」帝恨惋曰:「京與贊,真賊臣。」



 京無子,以從子褒嗣。褒孫伯宣,辭著作佐郎不拜。



 贊曰:德宗敝政,稅間架、借商錢、宮市為最甚。順宗為太子,欲極陳之,懲王叔文之諫而止,其畏如此。區區之臣,冒顏而關說,難哉!其饗國日淺,志不在民矣。憲宗聞暴斂之令首於賊臣,感憤太息,愛人之至也。及任程異、皇甫金尃,諫者不聽。興利之臣敗君之德甚矣!



 暢當,河東人。父璀,左散騎常侍,代宗時,與裴冕、賈至、王延昌待制集賢院,終戶部尚書。



 當進士擢第,貞元初,為太常博士。昭德皇后崩,中外服除,皇太子、諸王將服三年,詔太常議太子服。當與博士張薦、柳冕、李吉甫曰:「子為母齊衰三年,蓋通喪也;太子為皇后服,古無文。晉元皇后崩,亦疑太子服。杜預議:『古天子三年喪,既葬除服,魏亦以既葬為節。皇太子與國為體,若不變除,則東宮臣僕亦以衰麻出入殿省。』太子遂以卒哭除服。貞觀十年六月,文德皇后崩,十一月而葬,太子喪服之節,國史不書。至明年正月,以晉王為並州都督。既命官,當已除矣。今皇太子宜如魏、晉制:既葬而虞,虞而卒哭,卒哭而除,心喪三年。」宰相劉滋、齊映召問當等:「『子食於有喪者之側,未嘗飽也。』今太子以衰服侍膳至葬,可乎?令:群臣齊衰三十日公除。宜約以為服限。」乃請如宋、齊皇后為其父母服三十日除,入謁則服墨慘,還宮衰麻。右補闕穆質上疏曰:「『三年之喪,自天子達於庶人。』漢文帝以宗廟社稷之重自貶,乃以日易月,後世所不能革。太子,人臣也,不得如人君之制,母喪宜無厭降。惟晉既葬公除,議者詭辭以甘時主,不足師法。今有司之議,虧化敗俗,常情所鬱。夫政以德為本,德以孝為大。後世記禮之失,自今而始,顧不重哉!父在為母期,古禮也。國朝服之三年,臣謂三年則太重,唯行古為得禮。」德宗遣內常侍馬欽敘謂質曰:「太子有撫軍、監國、問安、侍膳之事,有司以三十日除,既葬釋服,以墨衰終,是何疑邪?」質又奏疏曰:「太子於陛下,子道也,臣道也。君臣以義,則撫軍監國,有權奪。父子問安侍膳,固無服衰之嫌,古未有服衰而廢者。舒王以下服三年,將不得問安侍膳邪?太子、舒王,皆臣子也,不宜甚異。且皇后,天下之母,其父母,士庶也,以天下之母,為士庶降服,可也。太子,臣子也,以臣子為母降,可乎?公除,非古也。入公門變服,今期喪以下慘制是也。太子晨昏侍,非公除比。墨衰奪情,事緣金革。今不監國撫軍,何抑奪邪?子之於父母,禮異而情均。太子奉君父之日遠,報母之日少,忍使失令名哉?」乃詔宰臣與有司更議,當等曰:「《禮》有公門脫齊衰,《開元禮》,皇后父母服十二月,從朝旨則十三日而除;皇太子外祖父母服五月,從朝旨則五日而除。恐喪服入侍,傷至尊之意,非特以金革奪也。太子公除,以墨慘奉朝,歸宮衰麻,酌變為制可也。」宰相乃令太常卿鄭叔則草奏:「既葬卒哭,十一月小祥,十三月大祥,十五月示覃,內謁即墨服。」復詔問質,質以為雖不能循古禮,猶愈於魏、晉之文遠甚。宰相乃言:「太子居皇后喪,至朝則抑哀承慈,實臣子至行。唯心與服,內外宜稱。今質請降詔於外,無害墨衰於內。臣謂言行於外,而服異於內,事非至誠,乖於德教。請下明詔如叔則議。」天子從之。及董晉代叔則為太常卿,帝曰:「皇太子服期,繇諫官,初非朕意。暢當等請循魏、晉故事,至論也。」



 當以果州刺史卒。



 林蘊,字復夢,泉州莆田人。父披,字茂彥,以臨汀多山鬼淫祠,民厭苦之,撰《無鬼論》。刺史樊晃奏署臨汀令,以治行遷別駕。



 蘊世通經,西川節度使韋皋闢推官。劉闢反,蘊曉以逆順,不聽。復遺書切諫,闢怒,械於獄,且殺之,將就刑,大呼曰:「『危邦不入,亂邦不居』,得死為幸矣!」闢惜其直,陰戒刑人抽劍磨其頸,以脅服之。蘊叱曰:「死即死,我項豈頑奴砥石邪?」闢知不可服,舍之,斥為唐昌尉。及闢敗,蘊名重京師。



 李吉甫、李絳、武元衡為相,蘊貽書諷以:「國家有西土,猶右臂也。今臂不附體,北彌豳郊,西極汧、隴,不數百里為外域。涇原、鳳翔、邠寧三鎮皆右臂,大籓擁旄鉞數十百人,唯李抱玉請復河、湟,命將不得其人,宜拔行伍之長,使守秦、隴。王者功成作樂,治定制禮。有權臣制樂曲,自立喪紀。舜命契:『百姓弗親,五品不遜,汝作司徒。』唐以皋、佑、鍔、季安為司徒,官不擇人。盧從史、於皋謨罪大而刑輕。農桑無百分之一,農夫一人給百口,蠶婦一人供百身,竭力於下者,饑不得食,寒不得衣。邊兵菜色,而將帥縱侈自養。中人十戶不足以給一無功之卒,百卒不足奉一驕將。」六事皆當時極敝。蘊亦韋皋所引重,嫉其專制,感憤關說。然嗜酒多忤物,宰相置不用也。



 滄景程權闢掌書記。既而權上四州版籍請吏,而軍中習熟擅地,畏內屬,挾權拒命,不得出。蘊陳君臣大誼,諭首將,人人釋然,於是權得去。蘊遷禮部員外郎。刑部侍郎劉伯芻薦之於朝,出為邵州刺史。嘗杖殺客陶玄之,投尸江中,籍其妻為倡,復坐贓,杖流儋州而卒。



 蘊辯給,嘗有姓崔者矜氏族,蘊折之曰:「崔杼弒齊君,林放問禮之本,優劣何如邪?」其人俯首不能對。



 韋公肅,隋儀同觀城公約七世孫。元和初為太常博士兼脩撰。憲宗將耕籍,詔公肅草具儀典,容家善之。太子少傅判太常卿事鄭餘慶廟有二祖妣,疑於祔祭,請諸有司。公肅議:「古諸侯一娶九女,故廟無二嫡。自秦以來有再娶,前娶後繼,皆嫡也,兩祔無嫌。晉驃騎大將軍溫嶠繼室三,疑並為夫人,以問太學博士陳舒,舒曰:『妻雖先沒,榮辱並從夫。禮祔於祖姑,祖姑有三,則各祔舅之所生。是皆夫人也。生以正禮,沒不可貶。』於是遂用舒議。且嫡繼於古有殊制,於今無異等,祔配之典,安得不同?卿士之寢祭二妻,廟享可異乎?古繼以媵妾,今以嫡妻,不宜援一娶為比,使子孫榮享不逮也。或曰:『《春秋》,魯惠公元妃孟子卒,繼室以聲子,聲子,孟侄娣也,不入惠廟。宋武公生仲子,歸於魯,生桓公而惠薨,立宮而奉之,不合於惠公,而別宮者何?追父志也。然其比奈何?』曰:晉南昌府君廟有荀、薛兩氏,景帝廟有夏侯、羊兩氏,唐家睿宗室則昭成、肅明二後,故太師顏真卿祖室有殷、柳兩氏。二夫人並祔,故事則然。」諸儒不能異。



 初,睿宗祥月,太常奏朔望弛朝,尚食進蔬具,止樂。餘日御便殿,具供奉仗。中書、門下官得侍,它非奏事毋謁。前忌與晦三日、後三日,皆不聽事。忌晦之明日,百官叩側門通慰。後遂為常。及是,公肅上言:「《禮》,忌日不樂,而無忌月。唯晉穆帝將納後,疑康帝忌月,下其議有司,於是荀納、王洽等引忌時、忌歲譏破其言。今有司承前所禁,在二十五月限,有弛朝徹樂事。喪除則禮革,王者不以私懷逾禮節,故禫禮徙月樂,漸去其情也,不容追遠,而立禮反重。今茲太常,雖郊廟,樂且停習,是謂反重以慢神也。有司悉禁中外作樂,是謂無故而徹也。願依經誼,裁正其違。」有詔中書門下召禮官、學官議,咸曰宜如公肅所請。制可。以官壽卒。



 許康佐,貞元中舉進士、宏辭,連中之。家苦貧,母老,求為知院官,人譏其不擇祿。及母喪已除,凡闢命皆不答,人乃知其為親屈,由是有名。



 遷侍御史。以中書舍人為翰林侍講學士,與王起皆為文宗寵禮。帝讀《春秋》至「閽弒吳子餘祭」,問:「閽何人邪?」康佐以中官方強,不敢對,帝嘻笑罷。後觀書蓬萊殿,召李訓問之,對曰:「古閽寺,今宦人也。君不近刑臣,以為輕死之道,孔子書之以為戒。」帝曰:「朕邇刑臣多矣,得不慮哉!」訓曰:「列聖知而不能遠,惡而不能去,陛下念之,宗廟福也。」於是內謀翦除矣。康佐知帝指,因辭疾,罷為兵部侍郎。遷禮部尚書。卒,贈吏部,謚曰懿。



 諸弟皆擢進士第,而堯佐最先進,又舉宏辭,為太子校書郎。八年,康佐繼之。堯佐位諫議大夫。



\end{pinyinscope}