\article{列傳第一百四十二上 回鶻上}

\begin{pinyinscope}

 回紇,其先匈奴也,俗多乘高輪車,元魏時亦號高車部,或曰敕勒,訛為鐵勒。其部落曰袁紇、薛延陀、契苾羽、都播、骨利幹、多覽葛、僕骨、拔野古、同羅、渾、思結、斛薛、奚結、阿跌、白霫,凡十有五種,皆散處磧北。



 袁紇者,亦曰烏護,曰烏紇,至隋曰韋紇。其人驍強,初無酋長,逐水草轉徙,善騎射,喜盜鈔,臣於突厥,突厥資其財力雄北荒。大業中,處羅可汗攻脅鐵勒部,裒責其財,既又恐其怨,則集渠豪數百悉坑之,韋紇乃並僕骨、同羅、拔野古叛去,自為俟斤,稱回紇。



 回紇姓藥羅葛氏,居薛延陀北娑陵水上,距京師七千里。眾十萬,勝兵半之。地磧鹵,畜多大足羊。有時健俟斤者,眾始推為君長。子曰菩薩,材勇有謀,嗜獵射,戰必身先,所向輒摧破,故下皆畏附,為時健所逐。時健死,部人賢菩薩,立之。母曰烏羅渾,性嚴明,能決平部事。回紇繇是浸盛。與薛延陀共攻突厥北邊,頡利遣欲谷設領騎十萬討之,菩薩身將五千騎破之馬鬣山,追北至天山,大俘其部人,聲震北方。繇是附薛延陀,相脣齒,號活頡利發,樹牙獨樂水上。



 貞觀三年,始來朝,獻方物。突厥已亡,惟回紇與薛延陀為最雄強。菩薩死,其酋胡祿俟利發吐迷度與諸部攻薛延陀,殘之,並有其地,遂南逾賀蘭山,境諸河。遣使者獻款,太宗為幸靈州,次涇陽,受其功。於是鐵勒十一部皆來言:「延陀不事大國,以自取亡,其下麕駭鳥散,不知所之。今各有分地,願歸命天子,請置唐官。」有詔張飲高會,引見渠長等,以唐官官之,凡數千人。



 明年復入朝。乃以回紇部為瀚海,多覽葛部為燕然,僕骨部為金微,拔野古部為幽陵,同羅部為龜林,思結部為盧山,皆號都督府;以渾為皋蘭州,斛薛為高闕州,阿跌為雞田州,契苾羽為榆溪州,奚結為雞鹿州,思結別部為𧾷帶林州,白霫為窴顏州;其西北結骨部為堅昆府,北骨利乾為玄闕州,東北俱羅勃為燭龍州;皆以酋領為都督、刺史、長史、司馬,即故單于臺置燕然都護府統之,六都督、七州皆隸屬,以李素立為燕然都護。其都督、刺史給玄金魚符,黃金為文,天子方招寵遠夷,作絳黃瑞錦文袍、寶刀、珍器賜之。帝坐秘殿,陳十部樂,殿前設高坫,置硃提瓶其上,潛泉浮酒,自左閣通坫趾注之瓶,轉受百斛鐐盎,回紇數千人飲畢,尚不能半。又詔文武五品官以上祖飲尚書省中。渠領共言:「生荒陋地,歸身聖化,天至尊賜官爵,與為百姓,依唐若父母然。請於回紇、突厥部治大塗,號『參天至尊道』,世為唐臣。」乃詔磧南鷿弟鳥泉之陽置過郵六十八所,具群馬、湩、肉待使客,歲內貂皮為賦。乃拜吐迷度為懷化大將軍、瀚海都督;然私自號可汗,署官吏,壹似突厥,有外宰相六、內宰相三,又有都督、將軍、司馬之號。帝更詔時健俟斤它部為祁連州,隸靈州都督,白霫它部為居延州。



 吐迷度兄子烏紇烝吐迷度之妻,遂與俱陸莫賀達干俱羅勃謀亂而歸車鼻可汗,二人者皆車鼻婿,故烏紇領騎夜劫吐迷度殺之。燕然副都護元禮臣遣使紿烏紇,許白為都督,烏紇不疑,即往謝,因斬以徇。帝恐諸部攜解,命兵部尚書崔敦禮持節臨撫,贈吐迷度左衛大將軍,賻祭備厚,擢其子婆閏左驍衛大將軍,襲父所領。俱羅勃既入朝,帝不遣。阿史那賀魯之盜北庭,婆閏以騎五萬助契苾何力等破賀魯,收北庭;又從伊麗道行軍總管任雅相等再破賀魯金牙山,遷右衛大將軍,從討高麗有功。



 婆閏死,子比慄嗣。龍朔中,以燕然都護府領回紇,更號瀚海都護府,以磧為限,大抵北諸蕃悉隸之。比慄死,子獨解支嗣。武后時,突厥默啜方強,取鐵勒故地,故回紇與契苾、思結、渾三部度磧,徙甘、涼間,然唐常取其壯騎佐赤水軍云。獨解支死,子伏帝匐立。明年,助唐攻殺默啜,於是別部移健頡利發與同羅、霫等皆來,詔置其部於大武軍北。伏帝匐死,子承宗立,涼州都督王君〓誣暴其罪,流死瀼州。當此時,回紇稍不循,族子瀚海府司馬護輸乘眾怨,共殺君〓,梗絕安西諸國朝貢道。久之,奔突厥,死。



 子骨力裴羅立。會突厥亂,天寶初,裴羅與葛邏祿自稱左右葉護,助拔悉蜜擊走烏蘇可汗。後三年,襲破拔悉蜜,斬頡跌伊施可汗,遣使上狀,自稱骨咄祿毘伽闕可汗,天子以為奉義王,南居突厥故地,徙牙烏德鞬山、昆河之間,南距西城千七百里,西城,漢高闕塞也,北盡磧口三百里,悉有九姓地。九姓者,曰藥羅葛,曰胡咄葛,曰啒羅勿,曰貊歌息訖,曰阿勿嘀,曰葛薩,曰斛嗢素,曰藥勿葛,曰奚牙勿。藥羅葛,回紇姓也,與僕骨、渾、拔、野古、同羅、思結、契苾六種相等夷,不列於數,後破有拔悉蜜、葛邏祿,總十一姓,並置都督,號十一部落。自是,戰常以二客部為先鋒。有詔拜為骨咄祿毘伽闕懷仁可汗,前殿列仗,中書令內案授冊使者,使者出門升輅,至皇城門,降乘馬,幡節導以行。凡冊可汗,率用此禮。明年,裴羅又攻殺突厥白眉可汗,遣頓啜羅達干來上功,拜裴羅左驍衛員外大將軍,斥地愈廣,東極室韋,西金山,南控大漠,盡得古匈奴地。裴羅死,子磨延啜立,號葛勒可汗,剽悍善用兵,歲遣使者入朝。



 肅宗即位,使者來請助討祿山,帝詔燉煌郡王承寀與約,而令僕固懷恩送王,因召其兵。可汗喜,以可敦妹為女,妻承寀,遣渠領來請和親,帝欲固其心,即封虜女為毘伽公主。於是可汗自將,與朔方節度使郭子儀合討同羅諸蕃,破之河上。與子儀會呼延谷,可汗恃其強,陳兵引子儀拜狼纛而後見。帝駐彭原,使者葛羅支見,恥班下,帝不欲使鞅鞅,引升殿,慰而遣。俄以大將軍多攬等造朝,及太子葉護身將四千騎來,惟所命。帝因冊毘伽公主為王妃,擢承寀宗正卿;可汗亦封承寀為葉護,給四節,令與其葉護共將。帝命廣平王見葉護,約為昆弟,葉護大喜,使首領達干等先到扶風見子儀,子儀犒飲三日。葉護辭曰:「國多難,我助討逆,何敢食!」固命,乃留。既行,日賜牛四十角、羊八百蹄、米四十斛。



 香積之戰,陣澧上,賊詭伏騎於王師左,將襲我,僕固懷恩麾回紇馳之,盡翦其伏,乃出賊背,與鎮西、北庭節度使李嗣業夾之,賊大敗,進收長安。懷恩率回紇、南蠻、大食眾繚都而南,壁滻東,進次陜西,戰新店。初,回紇至曲沃,葉護使將軍鼻施吐撥裴羅旁南山東出,搜賊伏谷中,殲之,營山陰。子儀等與賊戰,傾軍逐北,亂而卻,回紇望見,即逾西嶺,曳旗趨賊,出其後,賊反顧,遂大潰,追奔數十里,人馬相騰蹂,死者不可計,收仗械如丘。嚴莊挾安慶緒棄東京北度河,回紇大掠東都三日,奸人導之,府庫窮殫,廣平王欲止不可,而耆老以繒錦萬匹賂回紇,止不剽。葉護還京師,帝遣群臣勞之長樂,帝坐前殿,召葉護升階,席酋領於下,宴且勞之,人人賜錦繡繒器。葉護頓首言:「留兵沙苑,臣歸料馬,以收範陽,訖除殘盜。」帝曰:「為朕竭義勇,成大事,卿等力也。」詔進司空,爵忠義王,歲給絹二萬匹,使至朔方軍受賜。



 乾元元年,回紇使者多彥阿波與黑衣大食酋閣之等俱朝,爭長,有司使異門並進。又使請昏,許之。帝以幼女寧國公主下嫁,即冊磨延啜為英武威遠毘伽可汗,詔漢中郡王瑀攝御史大夫為冊命使,以宗子右司郎中巽兼御史中丞為禮會使,並以副瑀,尚書右僕射裴冕送諸境。帝餞公主,因幸咸陽,數尉勉,主泣曰:「國方多事,死不恨。」瑀至虜,而可汗胡帽赭袍坐帳中,儀衛光嚴,引瑀立帳外,問曰:「王,天可汗何屬?」瑀曰:「從昆弟也。」時中人雷靈俊立瑀上,又問:「立王上者為誰?」瑀曰:「中人也。」可汗曰:「中人奴爾,顧立郎上乎?」靈俊趨下。於是引瑀入,瑀不拜,可汗曰:「見國君,禮無不拜。」瑀曰:「天子顧可汗有功,以愛女結好。比中國與夷狄婚,皆宗室子。今寧國乃帝玉女,有德容,萬里來降,可汗天子婿,當以禮見,安踞受詔邪?」可汗慚,乃起奉詔,拜受冊。翌日,尊主為可敦。瑀所齎賜物,可汗盡與其牙下酋領。瑀還,獻馬五百匹、貂裘、白氈等。乃使王子骨啜特勒、宰相帝德等率騎三千助討賊,帝因命僕固懷恩總之。又遣大首領蓋將軍與三女子謝婚,並告破堅昆功。明年,骨啜與九節度戰相州,王師潰,帝德等奔京師,帝厚賜尉其意,乃還。俄而可汗死,國人欲以公主殉,主曰:「中國人婿死,朝夕臨,喪期三年,此終禮也。回紇萬里結昏,本慕中國,吾不可以殉。」乃止,然剺面哭,亦從其俗云。後以無子,得還。



 始葉護太子前得罪死,故次子移地健立,號牟羽可汗,其妻,僕固懷恩女也。始可汗為少子請昏,帝以妻之,至是為可敦。明年,使大臣俱錄莫賀達干等入朝,並問公主起居,使人通謁於延英殿。



 代宗即位,以史朝義未滅,復遣中人劉清潭往結好,且發其兵。比使者至,回紇已為朝義所訹,曰:「唐薦有喪,國無主,且亂,請回紇入收府庫,其富不貲。」可汗即引兵南,寶應元年八月也。清潭齎詔至其帳,可汗曰:「人言唐已亡,安得有使邪?」清潭為言:「先帝雖棄天下,廣平王已即天子位,其仁聖英武類先帝,故與葉護收二京、破安慶緒者,是與可汗素厚,且唐歲給回紇繒絹,豈忘之邪?」是時,回紇已逾三城,見州縣榛萊,烽障無守,有輕唐色。乃遣使北收單于府兵、倉庫,數以語凌靳清潭。清潭密白帝:「回紇兵十萬向塞。」朝廷震驚,遣殿中監藥子昂迎勞,且視軍,遇於太原,密識其兵裁四千,孺弱萬餘,馬四萬,與可敦偕來。帝令懷恩與回紇會。因遣使上書,請助天子討賊。回紇欲入蒲關,徑沙苑而東,子昂說曰:「自寇亂來,州縣殘虛,供億無所資,且賊在東京,若入井陘,以取邢、洺、衛、懷,收賊財帑,乃鼓而南,上策也。」不聽。子昂曰:「然則趨懷太行道,南據河陽,扼賊喉衿。」又不聽。曰:「食太原倉粟,右次陜,與澤潞、河南、懷鄭兵合。」回紇從之。



 詔以雍王為天下兵馬元帥,進子昂兼御史中丞,與右羽林衛將軍魏琚為左右廂兵馬使,中書舍人韋少華為元帥判官,御史中丞李進為行軍司馬,東會回紇。敕元帥為諸軍先鋒,與諸節度會陜州。時可汗壁陜州北,王往見之,可汗責王不蹈舞。子昂辭曰:「王,嫡皇孫,二宮在殯,禮不可以蹈舞。」回紇廷詰曰:「可汗為唐天子弟,於王,叔父行也,容有不蹈舞乎?」子昂固拒,即言:「元帥,唐太子也,將君中國,而可舞蹈見可汗哉?」回紇君臣度不能屈,即引子昂、進、少華、琚搒之百,少華、琚一夕死,王還營。官軍以王見辱,將合誅回紇,王以賊未滅止之。



 於是,懷恩與虜左殺為先驅。朝義使反間,左殺執以獻,與諸將同擊賊,戰橫水,走之,進收東都。可汗使拔賀那賀天子,獻朝義旗物。雍王還靈寶,可汗屯河陽,留三月,屯旁人困於剽辱。僕固瑒率回紇兵與朝義挐戰,蹀血二千里,梟其首,河北悉平。懷恩道相州西山崞口還屯,可汗出澤、潞,與懷恩會,道太原去。



 初,回紇至東京,放兵攘剽,人皆遁保聖善、白馬二祠浮屠避之,回紇怒,火浮屠,殺萬餘人,及是益橫,詬折官吏,至以兵夜斫含光門,入鴻臚寺。方其時,陜州節度使郭英乂留守東都,與魚朝恩及朔方軍驕肆,因回紇為暴,亦掠汝、鄭間,鄉不完廬,皆蔽紙為裳,虐於賊矣。



 帝念少華等死,故贈少華左散騎常侍,琚揚州大都督,賜一子六品官。於是冊可汗曰頡咄登裏骨啜蜜施合俱錄英義建功毘伽可汗,可敦曰娑墨光親麗華毘伽可敦,以左散騎常侍王翊使,即其牙命之,自可汗至宰相共賜實封二萬戶。又以左殺為雄朔王,右殺寧朔王,胡祿都督金河王,拔鑒將軍靜漠王,十都督皆國公。



 永泰初,懷恩反,誘回紇、吐蕃入寇。俄而懷恩死,二虜爭長,回紇首領潛詣涇陽見郭子儀,請改事。子儀率麾下叩回紇營。回紇曰:「願見令公。」子儀出旗門,回紇曰:「請釋甲。」子儀易服。酋長相顧曰:「真是公矣!」時李光進、路嗣恭介馬在側,子儀示酋長曰:「此渭北節度使某,朔方軍糧使某。」酋長下馬拜,子儀亦下見之。虜數百環視,子儀麾下亦至,子儀麾左右使卻,且命酒與飲,遺以纏頭彩三千,召可汗弟合胡祿等持手,因讓曰:「上念回紇功,報爾固厚,何負而來?今即與汝戰,何遽降也?我將獨入爾營,雖殺我,吾將士能擊汝。」酋長讋服曰:「懷恩詭我曰『唐天子南走,公見廢』,是以來。今天可汗在,公無恙,吾等願還擊吐蕃以報厚恩。然懷恩子,可敦弟也,願赦死。」於是子儀持酒,胡祿請盟而飲,子儀曰:「唐天子萬歲,回紇可汗亦萬歲,二國將相如之。有如負約,身死行陣,家屠戮。」方時,虜宰相磨咄莫賀達干、頓莫賀達干等聞言皆奪氣,酒至其所,輒曰:「無易公誓。」始,虜有二巫,言「此行必不戰,當見大人而還」;及是相顧笑曰:「巫不吾紿也。」



 朔方先鋒兵馬使白元光合回紇兵於靈臺,會雪雰嚴晦,吐蕃閉營撤備,乃縱擊之,斬首五萬級,生禽萬人,獲馬、橐它、牛、羊,收所俘唐戶五千。僕固名臣降,合胡祿都督等二百人皆來朝,賜與不可計。子儀以名臣見。名臣,懷恩兄子,銳將也。



 大歷三年,光親可敦卒,帝遣右散騎常侍蕭昕持節吊祠。明年,以懷恩幼女為崇徽公主繼室,兵部侍郎李涵持節冊拜可敦,賜繒彩二萬。是時,財用屈,稅公卿騾、橐它給行,宰相餞中渭橋。



 回紇之留京師者,曹輩掠女子於市,引騎犯含光門,皇城皆闔,詔劉清潭慰止。復出暴市物,奪長安令邵說馬,有司不敢何詰。自乾元後,益負功,每納一馬,取直四十縑,歲以數萬求售,使者相躡,留舍鴻臚,駘弱不可用,帝厚賜欲以愧之,不知也。復以萬馬來,帝不忍重煩民,為償六千。十年,回紇殺人橫道,京兆尹黎幹捕之,詔貸勿劾。又刺人東市,縛送萬年獄,首領劫取囚,殘獄吏去,都人厭苦。



 十三年,回紇襲振武,攻東陘,入寇太原。河東節度使鮑防與戰陽曲,防敗績,殘殺萬人。代州都督張光晟又戰羊虎谷,破之,虜乃去。



 德宗立,使中人告喪,且脩好。時九姓胡勸可汗入寇,可汗欲悉師向塞,見使者不為禮。宰相頓莫賀達干曰:「唐,大國,無負於我。前日入太原,取羊馬數萬,比及國,亡耗略盡。今舉國遠鬥,有如不捷,將安歸?」可汗不聽,頓莫賀怒,因擊殺之,並屠其支黨及九姓胡幾二千人,即自立為合骨咄祿毘伽可汗,使長建達干從使者入朝。建中元年,詔京兆少尹源休持節冊頓莫賀為武義成功可汗。



 始回紇至中國,常參以九姓胡,往往留京師,至千人,居貲殖產甚厚。會酋長突董、翳蜜施、大小梅錄等還國,裝橐系道,留振武三月,供擬珍豐,費不貲。軍使張光晟陰伺之,皆盛女子以橐,光晟使驛吏刺以長錐,然後知之。已而聞頓莫賀新立,多殺九姓胡人,懼不敢歸,往往亡去,突董察視嚴亟。群胡獻計於光晟,請悉斬回紇,光晟許之,即上言:「回紇非素強,助之者九胡爾。今其國亂,兵方相加,而虜利則往,財則合,無財與利,一亂不振。不以此時乘之,復歸人與幣,是謂借賊兵,資盜糧也。」乃使裨校陽不禮,突董果怒,鞭之。光晟因勒兵盡殺回紇群胡,收橐它、馬數千,繒錦十萬,且告曰:「回紇抶大將,謀取振武,謹先誅之。」部送女子還長安。帝召光晟還,以彭令方代之,遣中人與回紇使聿達干往言其端,因欲與虜絕。敕源休俟命太原。明年,乃行,因歸突董等四喪。突董,可汗諸父也。源休至,可汗令大臣具車馬出迎,其大相頡乾迦斯踞坐責休等殺突董事,休言:「彼自與張光晟鬥死,非天子命。」又曰:「使者皆負死罪,唐不自戮,何假手於我邪?」良久罷去,休等幾死。留五旬,卒不見可汗。可汗傳謂休曰:「國人皆欲爾死,我獨不然。突董等已亡,今又殺爾,猶以血濯血,徒益污。吾以水濯血,不亦善乎?為我言有司,所負馬直一百八十萬,可速償我。」遣散支將軍康赤心等隨休來朝。帝隱忍,賜以金繒。



 後三年,使使者獻方物,請和親。帝蓄前恚未平,謂宰相李泌曰:「和親待子孫圖之,朕不能已。」泌曰:「陛下豈以陜州故憾乎?」帝曰:「然。朕方天下多難,未能報,且毋議和。」泌曰:「辱少華等乃牟羽可汗也,知陛下即位必償怨,乃謀先苦邊,然兵未出,為今可汗所殺矣。今可汗初立,遣使來告,垂發不翦,待天子命。而張光晟殺突董等。雖幽止使人,然卒完歸,則為無罪矣。」帝曰:「卿言則然,顧朕不可負少華等,奈何?」泌曰:「臣謂陛下不負少華,少華負陛下。且北虜君長身赴難,陛下在籓,春秋未壯,而輕度河入其營,所謂冒豺虎之場也。為少華等計,當先定會見禮,臣猶危之,奈何孑然赴哉?臣昔為先帝行軍司馬,方葉護來,先帝祗使宴於府。及議征討,則不見也。葉護邀臣至營,帝不許,使好謂曰:『主當勞客,客返勞主邪?』東收京師,約曰:『土地、人眾歸我,玉帛、子女予回紇。』戰勝,葉護欲大掠,代宗下馬拜之,回紇乃東向洛。臣猶恨以元帥拜葉護於馬前,為左右過,然先帝曰:『王仁孝,足辦朕事。』下詔尉勉。葉護乃牟羽諸父也,牟羽之來,陛下以元子不拜於帳下,而可汗不敢少有失於陛下,則陛下未嘗屈矣。先帝拜葉護,全京城,陛下乃不拜可汗,固伸威於虜,何恨焉?然計香積、陜州事,以屈己為是乎?伸威為是乎?藉令少華等以陛下見可汗,閉壁五日,與陛下張飲,天下豈不寒心哉?而天助威神,使豺狼馴服,牟羽母捧陛下以貂裘,叱左右促命騎,躬送出營。此少華等負陛下也。假令牟羽為有罪,則今可汗已殺之,立者乃牟羽從父兄,是為有功,渠可忘之邪?且回紇可汗銘石立國門曰:『唐使來,當使知我前後功』云。今請和,必舉部南望,陛下不之答,其怨必深。願聽昏而約用開元故事,如突厥可汗稱臣,使來者不過二百,市馬不過千,不以唐人出塞,亦無不可者。」帝曰:「善。」乃許降公主,回紇亦請如約。詔咸安公主下嫁,又詔使者合闕達干見公主於麟德殿,使中謁者齎公主畫圖賜可汗。



 明年,可汗遣宰相跌都督等眾千餘,並遣其妹骨咄祿毘伽公主率大酋之妻五十人逆主,且納聘。𧾷夾跌至振武,為室韋所鈔,戰死。有詔其下七百,皆聽入朝,舍鴻臚,帝御延喜門見使者。是時,可汗上書恭甚,言:「昔為兄弟,今婿,半子也。陛下若患西戎,子請以兵除之。」又請易回紇曰回鶻,言捷鷙猶鶻然。帝欲饗回鶻公主,問禮於李泌,對曰:「肅宗於敦煌王為從祖兄,回鶻妻以女,見帝於彭原,獨拜廷下,帝呼曰『婦』而不名『嫂』也。當艱虞時,方藉其用,猶以臣之,況今日乎?」於是引回鶻公主入銀臺門,長公主三人候諸內,譯史傳導,拜必答,揖與進。帝御秘殿,長公主先入侍,回鶻公主入,拜謁已,內司賓導至長公主所,又譯史傳問,乃與俱入。至宴所,賢妃降階俟,回鶻公主拜,賢妃答拜。又拜召已,由西階升,乃坐。有賜則降拜,非帝賜則避席拜,妃、公主皆答拜。訖歸,凡再饗。帝又盡建咸安公主官屬,視王府。以嗣滕王湛然為昏禮使,右僕射關播護送,且將冊書拜可汗為汩咄祿長壽天親毘伽可汗,公主為智惠端正長壽孝順可敦。



 貞元五年,可汗死,子多邏斯立,國人號「泮官特勒」,以鴻臚卿郭鋒持節冊拜愛登里邏汩沒蜜施俱錄毘伽忠貞可汗。



 初,安西、北庭自天寶末失關、隴,朝貢道隔。伊西北庭節度使李元忠、四鎮節度留後郭昕數遣使奉表,皆不至。貞元二年,元忠等所遣假道回鶻,乃得至長安。帝進元忠為北庭大都護,昕為安西大都護。自是,道雖通,而虜求取無涘。沙陀別部六千帳,與北庭相依,亦厭虜裒索,至三葛祿、白眼突厥素臣回鶻者尤怨苦,皆密附吐蕃,故吐蕃因沙陀共寇北庭,頡乾迦斯與戰,不勝,北庭陷。於是都護楊襲古引兵奔西州。回鶻以壯卒數萬召襲古,將還取北庭,為吐蕃所擊,大敗,士死太半,迦斯奔還。襲古挈餘眾將入西州,迦斯紿曰:「弟與我俱歸,當使公還唐。」襲古至帳,殺之。葛祿又取深圖川,回鶻大恐,稍南其部落以避之。



 是歲,可汗為少可敦葉公主所毒死,可敦亦僕固懷恩之孫,懷恩子為回鶻葉護,故女號葉公主云。可汗之弟乃自立。迦斯方攻吐蕃,其大臣率國人共殺篡者,以可汗幼子阿啜嗣。迦斯還,可汗等出勞,皆俯伏言廢立狀,惟大相生死之。悉發郭鋒所賜器幣餉迦斯。可汗拜且泣曰:「今幸得繼絕,仰食於父也。」迦斯以其柔屈,乃相持哭,遂臣事之,以器幣悉給將士,無所私,其國遂安。遣達北特勒梅錄將軍來告,且聽命。詔鴻臚少卿庾鋋冊阿啜為奉誠可汗。俄以律支達干來告少寧國公主之喪。主,榮王女也。始寧國下嫁,又以媵之。寧國後歸,因留回鶻中為可敦,號「少寧國」,歷配英武、英義二可汗。至天親可汗時,始居外。其配英義生二子,皆為天親所殺。是歲,回鶻擊吐蕃、葛祿於北庭,勝之,且獻俘。明年,使藥羅葛炅來朝,炅本唐人呂氏,為可汗養子,遂從可汗姓。帝以其用事,賜齎殊優,拜檢校尚書右僕射。



 十一年,可汗死,無子,國人立其相骨咄祿為可汗,以使者來,詔秘書監張薦持節冊拜愛滕里邏羽錄沒蜜施合胡祿毘伽懷信可汗。骨咄祿本𧾷夾跌氏,少孤,為大首領所養,辯敏材武,當天親時數主兵,諸酋尊畏。至是,以藥羅葛氏世有功,不敢自名其族,而盡取可汗子孫內之朝廷。



 永貞元年,可汗死,詔鴻臚少卿孫杲臨吊,冊所嗣為滕裏野合俱錄毘伽可汗。



 元和初,再朝獻,始以摩尼至。其法日晏食,飲水茹葷,屏湩酪,可汗常與共國者也。摩尼至京師,歲往來西市,商賈頗與囊橐為奸。三年,來告咸安公主喪。主歷四可汗,居回鶻凡二十一歲。無幾,可汗亦死,憲宗使宗正少卿李孝誠冊拜愛登里羅汨蜜施合毘伽保義可汗。閱三歲,使者再朝,遣伊難珠再請昏,未報。可汗以三千騎至鷿鵜泉,於是振武以兵屯黑山,治天德城備虜。禮部尚書李絳奏言:「回鶻盛強,北邊空虛,一為風塵,則弱卒非抗敵之夫,孤城為不守之地。儻陛下懷此,增甲兵,飭城壘,中夏長策,生人大幸也。臣觀今日處置,未得其要。夫邊憂有五,請歷言之:北狄貪沒,唯利是視,比進馬規直,再歲不至,豈厭繒帛利哉?殆欲風高馬肥,而肆侵軼。故外攘內備,必煩朝廷,一可憂;兵力未完,斥侯未明,戈甲未備,城池未固,飾天德則虜必疑,虛西城則磧道無倚,二可憂;夫城保要害,攻守險易,當謀之邊將,今乃規河塞之外,裁廟堂之上,虜猝犯塞,應接失便,三可憂;自脩好以來,山川形勝,兵戍滿虛,虜皆悉之,賊掠諸州,調發在旬朔外,其系累人畜在旦夕內,比王師至則虜已歸,寇能久留,役亦轉廣,四可憂;北狄西戎,素相攻討,故邊無虞,今回鶻不市馬,若與吐蕃結約解仇,則將臣閉壁憚戰,邊人拱手受禍,五可憂。又淮西吳少陽垂死,可乘其變,諸道興發,役且十倍。臣謂宜聽其婚,使守蕃禮,所謂三利也;和親則烽燧不驚,城堞可治,盛兵以畜力,積粟以固軍,一也;既無北顧憂,可南事淮右,申令於垂盡之寇,二也;北虜恃我戚,則西戎怨愈深,內不得寧,國家坐受其安,寇掠長息,三也。今舍三利,取五憂,甚非計。或曰降主費多,臣謂不然。我三分天下賦,以一事邊。今東南大縣賦歲二十萬緡,以一縣賦為婚貲,非損寡得大乎?今惜婚費不與,假如王師北征,兵非三萬、騎五千不能捍且馳也。又如保十全之勝,一歲輒罷,其饋餉供儗,豈止一縣賦哉?」帝不聽。



\end{pinyinscope}