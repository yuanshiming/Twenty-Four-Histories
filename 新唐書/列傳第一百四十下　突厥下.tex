\article{列傳第一百四十下 突厥下}

\begin{pinyinscope}

 毘伽可汗默棘連,本謂「小殺」者,性仁友,自以立非己功,讓於闕特勒論,它要求人們從實際出發,實事求是,客觀地、全面地、歷,特勒不敢受,遂嗣位,實開元四年。以特勒為左賢王,專制其兵。初,默啜死,闕特勒盡殺其用事臣,惟暾欲谷者以女婆匐為默棘連可敦,獨免,廢歸其部。後突騎施蘇祿自為可汗,突厥部種多貳,默棘連乃召暾欲谷與謀國,年七十餘,眾尊畏之。俄而〓跌思太等自河曲歸之。始,降戶之南也,單于副都護張知運盡斂其兵,戎人怨怒;及姜晦為巡邊使,遮訴禁弓矢無以射獵為生,晦悉還之。乃共擊張知運,禽之,將送突厥;朔方行軍總管薛訥、將軍郭知運追之,眾潰,釋知運去。思太等分為二隊北走,王晙又破其左隊。



 默棘連既得降胡,欲南盜塞,暾欲谷曰:「不可,天子英武,人和歲豐,未有間,且我兵新集,不可動也。」默棘連又欲城所都,起佛、老廟,暾欲谷曰:「突厥眾不敵唐百分一,所能與抗者,隨水草射獵,居處無常,習於武事,強則進取,弱則遁伏,唐兵雖多,無所用也。若城而居,戰一敗,必為彼禽。且佛、老教人仁弱,非武強術。」默棘連當其策,即遣使者請和。帝以不情,答而不許。俄下詔伐之,乃以拔悉蜜右驍衛大將軍金山道總管處木昆執米啜、堅昆都督右武衛大將軍骨篤祿毘伽可汗、契丹都督李失活、奚都督李大酺、突厥默啜子左賢王墨特勒、左威衛將軍右賢王阿史那毘伽特勒、燕山郡王火拔石失畢等蕃漢士悉發,凡三十萬,以御史大夫、朔方道大總管王晙統之,期八年秋並集稽落水上,使拔悉蜜、奚、契丹分道掩其牙,捕默棘連。默棘連大恐,暾欲谷曰:「拔悉蜜在北庭,與二蕃相距遠,必不合。晙與張嘉貞有隙,必相執異,亦必不能來。即皆能來,我當前三日悉眾北徙,彼糧竭自去。拔悉蜜輕而好利,當先至,擊之可取也。」俄而拔悉蜜果引眾逼突厥牙,知晙等不至,乃引卻,突厥欲擊之,暾欲谷曰:「兵千里遠出,士殊死鬥,鋒不可當也。不如躡之,邀近而取之。」距北庭二百里,乃分兵由它道襲拔其城,即急擊拔悉蜜,眾走趨北庭,無所歸,悉禽之。還出赤亭,掠涼州,都督楊敬述使官屬盧公利、元澄等勒兵討捕,暾欲谷曰:「敬述若城守,當與和。如兵出,吾且決戰,必有功。」澄令於軍曰:「裸臂持滿外注。」會大寒裂膚,士手不能張弓矢,由是大敗,元澄走,敬述坐以白衣檢校涼州事,突厥遂大振,盡有默啜餘眾。



 明年,固乞和,請父事天子,許之。又連歲遣使獻方物求婚。是時天子東巡泰山,中書令張說謀益屯以備突厥,兵部郎中裴光庭曰:「封禪以告成功,若復調發,不可謂成功者。」說曰:「突厥雖請和,難以信結也。且其可汗仁而愛人,下為之用,闕特勒善戰,暾欲谷沈雄,愈老而智,李靖、世勣流也,三虜方協,知我舉國東巡,有如乘間,何以御之?」光庭即請以使召其大臣入衛,乃遣鴻臚卿袁振往諭帝意。默棘連置酒與可敦、闕特勒、暾欲谷坐帳中,謂振曰:「吐蕃,犬出也,唐與為昏;奚、契丹,我奴而役也,亦尚主;獨突厥前後請,不許,云何?」振曰:「可汗,天子子也,子而昏,可乎?」默棘連曰:「不然,二蕃皆賜姓,而得尚主,何不可云?且公主亦非帝女,我不敢有所擇,但屢請不得,為諸國笑。」振許為請,默棘連遣大臣阿史德頡利發入獻,遂從封禪。有詔四夷諸酋皆入仗佩弓矢,會兔起帝馬前,帝一發斃之,頡利發奉兔頓首賀曰:「陛下神武超絕,若天上則臣不知,人間無有也。」詔問:「饑欲食乎?」對曰:「仰觀弧矢之威,使十日不食猶為飽。」因令仗內馳射。扈封畢,厚宴賜遣之,然卒不許和親。



 自是比年遣大臣入朝,吐蕃以書約與連和鈔邊,默棘連不敢從,封上其書,天子嘉之,引使者梅錄啜宴紫宸殿,詔朔方西受降城許互市,歲賜帛數十萬。十九年,闕特勒死,使金吾將軍張去逸、都官郎中呂向奉璽詔吊祭,帝為刻辭於碑,仍立廟像,四垣圖戰陣狀,詔高手工六人往,繪寫精肖,其國以為未嘗有,默棘連視之,必悲梗。



 默棘連請昏既勤,帝許可,於是遣哥解慄必來謝,請昏期。俄為梅錄啜所毒,忍死殺梅錄啜,夷其種,乃卒。帝為發哀,詔宗正卿李佺吊祭,因立廟,詔史官李融文其碑。國人共立其子為伊然可汗。



 伊然可汗立八年,卒。凡遣使三入朝。其弟嗣立,是為苾星伽骨咄祿可汗,使右金吾衛將軍李質持冊為登利可汗。明年,遣使伊難如朝正月,獻方物,曰「禮天可汗如禮天,今新歲獻月,願以萬壽獻天子」云。可汗幼,其母婆匐與小臣飫斯達干亂,遂預政,諸部不協。登利從父分掌東西兵,號左右殺,士之精勁皆屬。可汗與母誘斬西殺,奪其兵,左殺懼,即攻登利可汗,殺之。



 左殺者,判闕特勒也,遂立毘伽可汗子,俄為骨咄葉護所殺,立其弟,旋又殺之,葉護乃自為可汗。天寶初,其大部回紇、葛邏祿、拔悉蜜並起攻葉護,殺之,尊拔悉蜜之長為頡跌伊施可汗,於是回紇、葛邏祿自為左右葉護,亦遣使者來告。國人奉判闕特勒子為烏蘇米施可汗,以其子葛臘哆為西殺。帝使使者諭令內附,烏蘇不聽,其下不與,拔悉蜜等三部共攻烏蘇米施,米施遁亡。其西葉護阿布思及葛臘哆率五千帳降,以葛臘哆為懷恩王。



 三載,拔悉蜜等殺烏蘇米施,傳首京師,獻太廟。其弟白眉特勒鶻隴匐立,是為白眉可汗。於是突厥大亂,國人推拔悉蜜酋為可汗,詔朔方節度使王忠嗣以兵乘其亂,抵薩河內山,擊其左阿波達干十一部,破之,獨其右未下,而回紇、葛邏祿殺拔悉蜜可汗,奉回紇骨力裴羅定其國,是為骨咄祿毘伽闕可汗。明年,殺白眉可汗,傳首獻。毘伽可汗妻骨咄祿婆匐可敦率眾自歸,天子御花萼樓宴群臣,賦詩美其事,封可敦為賓國夫人,歲給粉直二十萬。



 始突厥國於後魏大統時,至是滅。後或朝貢,皆舊部九姓云,其地盡入回紇。始其族分國於西者,曰西突厥。



 西突厥,其先訥都陸之孫吐務,號大葉護。長子曰土門伊利可汗,次子曰室點蜜,亦曰瑟帝米。瑟帝米之子曰達頭可汗,亦曰步迦可汗。始與東突厥分烏孫故地有之,東即突厥,西雷翥海,南疏勒,北瀚海,直京師北七千里,由焉耆西北七日行得南庭,北八日行得北庭,與都陸、弩失畢、歌邏祿、處月、處蜜、伊吾諸種雜。其風俗大抵突厥也,言語少異。



 初,東突厥木桿可汗死,舍其子大邏便,而立弟佗缽可汗。佗缽死,先令戒其子庵羅必立大邏便,國人以其母賤,不肯立,而卒立庵羅。庵羅後以讓木桿兄子攝圖,是為沙缽略可汗。而大邏便別為阿波可汗,自臣所部,沙缽略襲擊之,殺其母,阿波西走達頭。當是時,達頭為西面可汗,即授阿波兵十萬,使與東突厥戰。而阿波竟為沙缽略所禽。及啟民可汗時,達頭可汗歲以兵相加,而隋常助啟民,故達頭敗奔吐谷渾。



 始,阿波既禽,國人立鞅素特勒子,是為泥利可汗。達頭之奔,泥利亦敗,及死,其子達漫立,是為泥橛處羅可汗,政苛察多忌。大業中,從煬帝征高麗,賜號曷薩那可汗,妻以宗女,留其弟闕達度設畜牧於會寧郡,即自稱闕可汗。江都亂,曷薩那從宇文化及至黎陽,遁歸長安,高祖降榻與共坐,封歸義王,以大珠獻帝,帝不受,曰:「朕所重者王之赤心,是無用也。」闕可汗有馬三,武德元年內屬,賜號吐烏過拔闕可汗,與李軌連和。隋西戎使者曹瓊據甘州誘之,俄與瓊合,共擊軌,兵不勝,走達斗拔谷,與吐谷渾相輔車,為軌所滅。



 初,曷薩那朝隋,國人皆不欲,既被留不遣,乃共立達頭孫,號射匱可汗,建廷龜茲北之三彌山,玉門以西諸國多役屬,與東突厥亢。射匱死,其弟統葉護嗣,是為統葉護可汗。



 統葉護可汗勇而有謀,戰輒勝,因並鐵勒,下波斯、罽賓,控弦數十萬,徙廷石國北之千泉,遂霸西域諸國,悉授以頡利發,而命一吐屯監統,以督賦入。明年,射匱使使來,以曷薩那有世憾,請殺之,帝不許。群臣曰:「存一人,失一國,後且為患。」秦王曰:「不然,人來歸我,我殺之不祥。」帝又不聽。宴禁中,酒酣,至中書省,縱使者戕之,不宣也。射匱亦連年系貢條支巨卵、師子革等,帝厚申撫結,約與並力討東突厥。統葉護可汗請期,頡利大懼,乃與和,約毋相伐也。統葉護可汗來請昏,帝與群臣謀:「西突厥去我遠,緩急不可杖,可與昏乎?」封德彞曰:「計今之便,莫若遠交而近攻,請聽昏以怖北狄,待我既定,而後圖之。」帝乃許昏,詔高平王道立至其國,統葉護可汗喜,遣真珠統俟斤與道立還,獻萬釘寶鈿金帶、馬五千匹以藉約。會東突厥歲犯邊,西道梗澀,又頡利遣謂曰:「若迎唐公主,必假我道,我且留之。」統葉護可汗病之,未克昏。方負其強,不以恩結下,眾怨,多叛去,其諸父莫賀咄殺之,帝欲齎玉帛焚祭其國,會亂,不果至。



 莫賀咄立,是為屈利俟毘可汗,遣使者來獻。俟毘可汗初分統突厥為小可汗,既稱大可汗,國人不附。弩失畢部自推泥孰莫賀設為可汗,泥孰辭不受。會統葉護可汗子〓力特勒避莫賀咄亂,亡在康居,泥孰迎立之,為乙毘缽羅肆葉護可汗,與俟毘可汗分王其國,挐鬥不解,各遣使朝獻。太宗追憐曷薩那死非罪,為贈上柱國,具禮以葬。貞觀四年,俟毘可汗請昏,不許,詔曰:「突厥方亂,君臣未定,何遽昏為?各敕其部毋相侵。」由是西域諸國悉叛之,國大虛耗,眾悉附肆葉護可汗,雖俟毘之部亦稍稍去,共以兵擊俟毘,俟毘走保金山,為泥孰所殺,奉肆葉護為大可汗。



 肆葉護已立,即北討鐵勒、薛延陀,為延陀所敗。性猜愎,狹於統下。小可汗乙利者,於國最有功,肆葉護聽讒,種夷之,眾皆沮駭。又忌泥孰,陰圖殺之,泥孰亡入焉耆。未幾,沒卑達干與弩失畢部諸豪謀執廢肆葉護,葉護輕騎走康居,憂死。國人迎泥孰於焉耆,立之,是為咄陸可汗。可汗父莫賀設,本隸統葉護者,武德時來朝,太宗與之盟,約為昆弟。死而泥孰代之,或曰伽那設。既立,遣使詣闕,不敢當可汗號。帝詔鴻臚少卿劉善因持節冊號吞阿婁拔利邲咄陸可汗,賜鼓纛,段彩巨萬。泥孰遣使謝。它日,太上皇宴使者兩儀殿,謂長孫無忌曰:「今蠻夷率服,古亦有乎?」無忌上千萬歲壽,太上皇喜,以酒屬帝,帝頓首謝,亦奉觴上太上皇壽。



 咄陸可汗死,弟同俄設立,是為沙缽羅咥利失可汗,贈三遣使奉方物,遂請昏,帝慰而不俞。可汗分其國為十部,部以一人統之,人授一箭,號十設,亦曰十箭。為左、右:右五咄陸部,置五大啜,居碎葉東;右五弩失畢部,置五大俟斤,居碎葉西。其下稱一箭曰一部落,號十姓部落雲。然不為眾悅賴,其部統吐屯以兵襲之,咥利失率左右戰,統吐屯不勝去。咥利失與其弟步利設奔焉耆。阿悉吉闕俟斤與統吐屯召國人謀立欲谷設為大可汗,以咥利失為小可汗。會統吐屯被殺,欲谷設又為其俟斤所破,咥利失乃復得故地。後西部卒自立欲谷設為乙毘咄陸可汗,而與咥利失交戰,殺傷不可計,乃因伊列河約諸部:河以西受令於咄陸,其東咥利失主之。自是西突厥又分二國矣。



 咄陸可汗建廷鏃曷山西,謂之「北庭」,駁馬、結骨諸國悉附臣之,陰與咥利失部吐屯俟列發以兵攻咥利失。咥利失援窮,奔拔汗那而死。國人立其子,是為乙屈利失乙毘可汗,逾年死。弩失畢大酋迎伽那設之子畢賀咄葉護立之,是為乙毘沙缽羅葉護可汗。太宗詔左領軍將軍張大師持節冊命,賜鼓纛,建庭雖合水北,謂之「南庭」,東薄伊列河,龜茲、鄯善、且末、吐火羅、焉耆、石、史、何、穆、康等國皆隸屬。是時咄陸兵浸浸盛,與沙缽羅葉護數交戰。會二可汗使者皆來,帝敕以敦睦,令各罷兵,咄陸不肯聽,遣石國吐屯攻葉護可汗,殺之,並其國。弩失畢不服,叛去。咄陸又擊吐火羅,取之,乃入寇伊州。安西都護郭孝恪以輕騎二千,自烏骨狙擊,敗之。咄陸以處月、處蜜兵圍天山而不克,孝恪追北,拔處月俟斤之城,抵遏索山,斬千餘級,降處蜜部而歸。咄陸可汗性很傲,留使者元孝友等不遣,妄曰:「我聞唐天子才武,我今討康居,爾視我與天子等否?」遂與共攻康居,道米國,即襲破之,系虜其人,取貲口不以與下,其將泥孰啜怒,奪取之,咄陸斬以徇。泥孰啜之將胡祿屋舉兵襲咄陸可汗,多殺士,國大亂,將歸保吐火羅,大臣勸其返國,不從,率眾去,度葉水,及石國,左右亡去略盡,乃保可賀敦城。自輕出招叛亡,阿悉吉闕俟斤逆擊之,咄陸敗,襲取白水胡城以居。弩失畢不欲咄陸為可汗,遣使者至闕下,請所立。帝遣通事舍人溫無隱持璽詔與國大臣擇突厥可汗子孫賢者授之,乃立乙屈利失乙毘可汗之子,是為乙毘射匱可汗。



 乙毘射匱既立,改館使者,悉還之長安,使弩失畢將兵攻白水胡城。咄陸勒兵自城出,鳴鼓角薄斗,弩失畢不能軍,殺獲甚多。咄陸因其勝招徠舊部,皆曰:「戰千人,存一人,我猶不從也。」咄陸自知眾怨,乃走吐火羅。乙毘射匱遣使貢方物,且請昏,帝令割龜茲、于闐、疏勒、硃俱波、蔥嶺五國為聘禮,不克昏。於是阿史那賀魯反,盡得可汗部落。



 賀魯者,室點蜜可汗五世孫,曳步利設射匱特勒劫越子也。始,阿史那步真來歸國,咄陸可汗以賀魯為葉護,代步真,居多邏斯川,直西州北千五百里,統處月、處蜜、姑蘇、歌邏祿、弩失畢五姓之眾。咄陸之走吐火羅也,乙毘射匱以兵迫逐,賀魯無常居,部多散亡。有執舍地、處木昆、婆鼻三種者,以賀魯無罪,往請可汗,可汗怒,欲誅執舍地等,三種乃舉所部數千帳,與賀魯皆內屬,帝優撫之。會討龜茲,請先馳為向導,詔授昆丘道行軍總管,宴嘉壽殿,厚賜予,解衣衣之。擢累左驍衛將軍、瑤池都督,處其部於庭州莫賀城,密招攜散,廬幕益眾。



 方帝崩,即謀取西、庭二州,刺史駱弘義以聞,高宗遣通事舍人喬寶明馳撫,因令賀魯遣子咥運入宿衛。咥運中悔,劫於勢,不得去,拜右驍衛中郎將。帝遣還,咥運即勸賀魯引而西,取咄陸可汗故地,建牙於千泉,自號沙缽羅可汗,遂統咄陸、弩失畢十姓。咄陸有五啜,曰處木昆律啜、胡祿屋闕啜、攝舍提暾啜、突騎施賀邏施啜、鼠尼施處半啜。弩失畢有五俟斤,曰阿悉結闕俟斤、哥舒闕俟斤、拔塞幹暾沙缽俟斤、阿悉結泥孰俟斤、哥舒處半俟斤。而胡祿屋闕,賀魯婿也。阿悉結闕俟斤最盛強,勝兵至數十萬。以咥運為莫賀咄葉護。遂寇庭州,敗數縣,殺掠數千人去。詔左武衛大將軍梁建方、右驍衛大將軍契苾何力為弓月道行軍總管,右驍衛將軍高德逸、右武衛將軍薩孤吳仁副之,發府兵三萬,合回紇騎五萬擊之。駱弘義獻計曰:「安中國以信,馭夷狄以權,理有變通也。賀魯保一城,方寒積雪,謂唐兵必不來,宜乘此一舉滅之。遷延及春,且生變,縱不率連諸國,必遠跡遁去。且兵本誅賀魯,而處蜜、處木昆等亦各欲自免,若留不進,彼與賀魯復合矣。今雖嚴冬風勁,兵苦皸墮,又不可久留費邊糧,使賊得堅黨附、賒死期也。請寬處月、處蜜等罪,專誅賀魯,除禍務本,不可先治枝葉也。願發射脾、處月、處蜜、契苾等兵,齎一月食,急趨之,大軍住憑洛水上為之景助,此驅戎狄攻豺狼也。且戎人藉唐兵為羽翼,今胡騎出前,唐兵躡後,賀魯窮矣。」天子然其奏,詔弘義佐建方等經略之。處月硃邪孤注者,引兵附賊,據牢山,建方等攻之,眾潰,追行五百里,斬孤注,上首九千級,虜其帥六十,不如弘義所計。



 永徽四年,罷瑤池都督府,即處月置金滿州,又遣左屯衛大將軍程知節為蔥山道行軍大總管,率諸將進討。是歲,咄陸可汗死,其子真珠葉護請討賀魯自效,為賀魯所拒,不得前。明年,知節擊歌邏祿、處月,斬千級,收馬萬計。副將周智度擊處木昆城,拔之,斬馘三萬。前軍蘇定方擊賀魯別帳鼠尼施於鷹娑川,斬首虜獲馬甚眾,賊棄鎧仗彌野。會副總管王文度不肯戰,降怛篤城,取其財,屠之,知節不能制。



 顯慶初,擢定方伊麗道行軍大總管,率燕然都護任雅相、副都護蕭嗣業、左驍衛大將軍瀚海都督回紇婆閏等窮討。詔右屯衛大將軍阿史那彌射、左屯衛大將軍阿史那步真為流沙道安撫大使,分出金山道,俟斤嫩獨祿等萬餘帳迎降。定方以精騎至曳咥河西,擊處木昆,破之。賀魯舉十姓兵十萬騎來拒,定方以萬人當之,虜見兵少,以騎繞唐軍。定方令步卒據原,欑槊外注,自以騎陣於北。賀魯先擊原上軍,三犯,軍不動。定方縱騎乘之,虜大潰,追奔數十里,俘斬三萬人,殺其大酋都搭達干等二百人。明日躡北,五弩失畢皆降。五咄陸聞賀魯敗,趨南道降步真。定方命嗣業、婆閏趨邪羅斯川追虜,任雅相提降兵踵後。會大雪,軍中請須霽,定方曰:「今雰晦風冽,虜謂我不能師,掩其不虞可也,緩則遠矣,省日兼功,上策也。」於是晝夜進,收所過人畜,至雙河,與彌射、步真會,軍飽氣張,距賀魯牙二百里,陣而行,抵金牙山。賀魯眾適獵,定方兵縱破其牙,俘數萬人,獲鼓纛器械,賀魯跳度伊麗水。嗣業次千泉,彌射至伊麗,處月、處蜜諸部皆下。次雙河,賀魯先以步失達干據柵戰,彌射攻之,潰,定方追賀魯至碎葉水,盡奪其眾。賀魯、咥運將奔鼠耨設,至石國蘇咄城,馬不進,眾饑,齎寶入城,且市馬,城主伊涅達干迎之,既入,拘送石國。會彌射子元爽與嗣業兵至,取之。乃悉散諸部兵,開道置驛,收露〓,問人疾苦,賀魯所掠悉還之民,西域平。賀魯謂嗣業曰:「我,亡虜也,先帝厚我,我則背之,今天降怒罰,尚何道?且聞漢法殺人必都市,我願就死昭陵,謝罪於先帝也。」帝曰:「先帝賜賀魯二千帳主之,今罪人既得,獻昭陵其可乎?」許敬宗曰:「古者,軍凱還則飲至於廟。若諸侯,獻馘天子,未聞獻於陵。然陛下奉園寢與宗廟等,可行不疑。」於是執而獻昭陵,赦不誅。



 賀魯已滅,裂其地為州縣,以處諸部。木昆部為匐延都督府,突騎施索葛莫賀部為〓鹿都督府,突騎施阿利施部為絜山都督府,胡祿屋闕部為鹽泊都督府,攝舍提暾部為雙河都督府,鼠尼施處半部為鷹娑都督府,又置昆陵、濛池二都護府以統之。其所役屬諸國皆置州,西盡波斯,並隸安西都護府。以阿史那彌射為興昔亡可汗,兼驃騎大將軍、昆陵都護,領五咄陸部,阿史那步真為繼往絕可汗,兼驃騎大將軍、濛池都護,領五弩失畢部,各賜帛十萬,以光祿卿盧承慶持冊命之。賀魯死,詔葬頡利塚旁,紀其概於石。



 阿史那彌射,亦室點蜜可汗五世孫,世為莫賀咄葉護。貞觀中,遣使者持節立彌射為奚利邲咄陸可汗,賜鼓纛。族兄步真謀殺彌射,欲自立,彌射不能國,即舉所部處月、處蜜等入朝,拜右監門衛大將軍。而步真遂自為咄陸葉護,眾不厭,去之,亦與族人來朝,拜左屯衛大將軍。彌射從帝征高麗有功,封平壤縣伯,遷右武衛大將軍。及平賀魯,乃與步真皆為可汗,得補所部刺史以下。是歲,彌射擊真珠葉護於雙河,斬之,殺闕啜二人。



 彌射、步真無綏御材,下多怨,於是思結都曼率疏勒、硃俱波、喝〓陀三國叛,擊破於闐,詔左驍衛大將軍蘇定方討之,都曼兵保馬頭川。五年,定方傅其城,擊降之。龍朔二年,彌射、步真以兵從〓海道總管蘇海政討龜茲,步真怨彌射,且欲並其部,乃誣以謀反。海政不能察,即集軍吏計議先發誅之,因稱詔發所齎賜可汗首領,彌射以麾下至,悉收斬之。其部鼠尼施、拔塞幹叛走,海政追平之。步真死乾封時。



 咸亨二年,以西突厥部酋阿史那都支為左驍衛大將軍兼匐延都督,以安輯其眾。儀鳳中,都支自號十姓可汗,與吐蕃連和,寇安西,詔吏部侍郎裴行儉討之。行儉請毋發兵,可以計取。即詔行儉冊送波斯王子,並安撫大食,若道兩蕃者。都支果不疑,率子弟止謁,遂禽之,召執諸部渠長,降別帥李遮匐以歸,調露元年也。西姓自是益衰,其後二部人日離散。遂擢彌射子元慶為左玉鈐衛將軍,步真子步利設斛瑟羅為右玉鈐衛將軍,盡襲父所領及可汗號。元慶累拜鎮國大將軍、行左威衛大將軍。武后擅命,率諸蕃長請賜睿宗氏曰武,更號斛瑟羅曰竭忠事主可汗。長壽中,元慶坐謁皇嗣,為來俊臣所誣,要斬,流其子獻於振州。



 其明年,西突厥部立阿史那俀子為可汗,與吐蕃寇,武威道大總管王孝傑與戰冷泉、大領谷,破之;碎葉鎮守使韓思忠又破泥熟俟斤及突厥施質汗、胡祿等,因拔吐蕃泥熟沒斯城。聖歷二年,以斛瑟羅為左衛大將軍兼平西軍大總管,令撫鎮國人。是時烏質勒兵張甚,斛瑟羅不敢歸,與其部人六七萬內遷,死長安,擢子懷道為右武衛將軍。



 長安中,以阿史那獻為右驍衛大將軍,襲興昔亡可汗、安撫招慰十姓大使、北庭大都護。四年,以懷道為十姓可汗兼濛池都護。未幾,擢獻磧西節度使。十姓部落都擔叛,獻擊斬之,傳首闕下,收碎葉以西帳落三萬內屬,璽書嘉慰。葛邏祿、胡屋、鼠尼施三姓已內屬,為默啜侵掠,以獻為定遠道大總管,與北庭都護湯嘉惠等掎角。於是突騎施陰幸邊隙,故獻乞益師,身入朝,玄宗不許。詔左武衛中郎將王惠持節安慰。方冊拜突騎施都督車鼻施啜蘇祿為順國公,而突騎施已圍撥換、大石城,將取四鎮。會嘉惠拜安西副大都護,即發三姓葛邏祿兵與獻共擊之。帝將詔王惠與相經略,宰相臣璟、臣頲曰:「突騎施叛,葛邏祿攻之,此夷狄自相殘,非朝廷出也。大者傷,小者滅,皆我之利。方王惠往撫慰,不可參以兵事。」乃止。獻終以娑葛強狠不能制,亦歸死長安。



 突騎施吐火仙之敗,始以懷道子昕為十姓可汗、開府儀同三司、濛池都護,冊其妻涼國夫人李為交河公主,遣兵護送。昕至碎葉西俱蘭城,為突騎施莫賀達干所殺,交河公主與其子忠孝亡歸,授左領軍衛員外將軍,西突厥遂亡。



 突騎施烏質勒,西突厥別部也。自賀魯破滅,二部可汗皆先入侍,虜無的君。烏質勒隸斛瑟羅,為莫賀達干。斛瑟羅政殘,眾不悅,而烏質勒能撫下,有威信,諸胡順附,帳落浸盛,乃置二十都督,督兵各七千,屯碎葉西北。稍攻得碎葉,即徙其牙居之,謂碎葉川為大牙,弓月城、伊麗水為小牙,其地東鄰北突厥,西諸胡,東直西、庭州,盡並斛瑟羅地。



 聖歷二年,遣子遮弩來朝,武后厚加慰撫。神龍中,封懷德郡王。是歲,烏質勒死,其子〓鹿州都督娑葛為左驍衛大將軍,襲封爵。是時勝兵三十萬,詔十姓可汗阿史那懷道持節冊命,賜宮人四。景龍中,遣使者入謝,中宗為御前殿,列萬騎羽林二仗,引見勞賜。俄與其將厥啜忠節交怨,兵相加暴。娑葛訟忠節罪,請內之京師。忠節以千金賂宰相宗楚客等,願無入朝,請導吐蕃擊娑葛以報。楚客方專國,即以御史中丞馮嘉賓持節經制。嘉賓與忠節書疏反復,娑葛邏得之,遂殺嘉賓,使弟遮弩率兵盜塞。安西都護牛師獎與戰火燒城,師獎敗,死之,表索楚客頭以徇。大都護郭元振表娑葛狀直,當見赦,詔許,西土遂定。



 既而與遮弩分治其部,遮弩恨眾少,叛歸默啜,請為鄉導反攻其兄。默啜留遮弩,自以兵二萬擊娑葛,禽之。默啜歸語遮弩曰:「汝兄弟不相協,能盡心事我乎?」兩殺之。



 突騎施別種車鼻施啜蘇祿者,裒拾餘眾,自為可汗。蘇祿善撫循其下,部種稍合,眾至二十萬,於是復雄西域。開元五年,始來朝,授右武衛大將軍、突騎施都督,卻所獻不受。以武衛中郎將王惠持節拜蘇祿左羽林大將軍、順國公,賜錦袍、鈿帶、魚袋七事,為金方道經略大使。然詭猾,不純臣於唐,天子羈系之,進號忠順可汗。其後閱一二歲,使者納贄,帝以阿史那懷道女為交河公主妻之。是歲,突騎施鬻馬於安西,使者致公主教於都護杜暹,暹怒曰:「阿史那女敢宣教邪?」笞其使,不報。蘇祿怒,陰結吐蕃舉兵掠四鎮,圍安西城。暹方入當國,而趙頤貞代為都護,乘城久之,出戰又敗。蘇祿略人畜,發囷貯,徐聞暹已宰相,乃引去;即遣首領葉支阿布思來朝,玄宗召見,饗之。會東突厥使者亦來,與爭長曰:「突騎施國小,且突厥臣,不宜居上。」蘇祿使者曰:「宴乃為我,不可下。」遂設東西幄,而蘇祿使者西席,乃克宴。



 始,蘇祿愛治其人,性勤約,每戰有所得,盡以予下,故諸族附悅之,為盡力,又交通吐蕃、突厥,二國皆以女妻之,遂立三國女並為可敦,以數子為葉護。費日廣而無素儲,晚年愁窶不聊,故鹵獲稍留不分,下始貳矣;又病風,一支攣,不事事。於是大首領莫賀達干、都摩支二部方盛,而種人自謂娑葛後者為「黃姓」,蘇祿部為「黑姓」,更相猜讎。



 俄而莫賀達干、都摩支夜攻蘇祿,殺之。都摩支又背達干立蘇祿子吐火仙骨啜為可汗,居碎葉城,引黑姓可汗爾微特勒保怛邏斯城,共擊達干。帝使磧西節度使蓋嘉運和撫突騎施、拔汗那西方諸國。莫賀達干與嘉運率石王莫賀咄吐屯、史王斯謹提共擊蘇祿子,破之碎葉城。吐火仙棄旗走,禽之,並其弟葉護頓阿波。疏勒鎮守使夫蒙靈詧挾銳兵與拔汗那王掩怛邏斯城,斬黑姓可汗與其弟撥斯,入曳建城,收交河公主及蘇祿可敦、爾微可敦而還,又料西國散亡數萬人,悉與拔汗那王。諸國皆降。處木昆匐延闕律啜等諸部皆上書謝曰:「生於荒裔,國亂王薨,更相攻屠。賴天子遣嘉運將兵誅暴拯危,願得稽首聖顏,以部落附安西,永為外臣。」許之。明年,擢闕律啜為右驍衛大將軍,冊石王為順義王,加拜史王為特進,顯醻其功。嘉運俘吐火仙骨啜獻太廟,天子赦以為左金吾衛員外大將軍、脩義王,頓阿波為右武衛員外將軍。以阿史那懷道子昕為十姓可汗,領突騎施所部,莫賀達干怒曰:「平蘇祿,我功也。今立昕,謂何?」即誘諸落叛。詔嘉運招諭,乃率妻子及纛官首領降,遂命統其眾。後數年,復以昕為可汗,遣兵護送。昕至俱闌城,為莫賀咄所殺。莫賀咄自為可汗,安西節度使夫蒙靈詧誅斬之,以大纛官都摩支闕頡斤為三姓葉護。



 天寶元年,突騎施部更以黑姓伊裏底蜜施骨咄祿毘伽為可汗,數通使貢。十二載,黑姓部立登里伊羅蜜施為可汗,亦賜詔冊。


至德後,突騎施衰,黃、黑姓皆立可汗相攻,中國方多故,不暇治也。乾元中,黑姓可汗阿多裴羅猶能遣使者入朝。大歷後,葛邏祿盛,徙居碎葉川,二姓微,至臣役於葛祿,斛瑟羅餘部附回鶻。及其破滅,有特
 \gezhu{
  廣尨}
 勒居焉耆城,稱葉護,餘部保金莎領,眾至二十萬。



 贊曰:隋季世,虛內以攻外,生者罷道路,死者暴原野,天下盜賊共攻而亡之。當此時,四夷侵,中國微,而突厥最強,控弦者號百萬,華人之失職不逞皆往從之,槊之謀,導之入邊,故頡利自以為強大古無有也。高祖初即位,與和,因數出軍助討賊,故詭臣之,贈予不可計。虜見利而動,又與賊連和,殺掠吏民,於是掃國入寇,薄渭橋,騎壒蒙京師。太宗身勒兵,顯責而陰間之,戎始內阻。不三年,縛頡利獻北闕下,霆掃風除,其國遂墟。自《詩》、《書》以來,伐暴取亂,蔑如帝神且速也,秦漢比之,陋矣。然帝數暴師不告勞,料敵無遁情,善任將,必其功,蓋黃帝之兵也。而突厥乃以失德抗有道,浸衰當始興,雖運之盛衰屬於天,而其亡信有由矣!



\end{pinyinscope}