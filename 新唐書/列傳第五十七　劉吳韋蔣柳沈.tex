\article{列傳第五十七 劉吳韋蔣柳沈}

\begin{pinyinscope}

 劉子玄,名知幾,以玄宗諱嫌,故以字行。年十二,父藏器為授《古文尚書》,業不進,父怒,楚督之。及聞為諸兄講《春秋左氏》,冒往聽,退輒辨析所疑,嘆曰:「書如是,兒何怠!」父奇其意,許授《左氏》。逾年,遂通覽群史。與兄知柔俱以善文詞知名。擢進士第,調獲嘉主簿。



 武后證聖初,詔九品以上陳得失。子玄上書,譏「每歲一赦,或一歲再赦,小人之幸,君子之不幸」。又言:「君不虛授,臣不虛受。妄受不為忠,妄施不為惠。今群臣無功,遭遇輒遷,至都下有『車載斗量,杷椎碗脫』之諺。」又謂:「刺史非三載以上不可徙,宜課功殿,明賞罰。」後嘉其直,不能用也。時吏橫酷,淫及善人,公卿被誅死者踵相及。子玄悼士無良而甘於禍,作《思慎賦》以刺時。蘇味道、李嶠見而嘆曰:「陸機《豪士》之流乎,周身之道盡矣!」子玄與徐堅、元行沖、吳兢等善,嘗曰:「海內知我者數子耳。」



 累遷鳳閣舍人,兼修國史。中宗時,擢太子率更令。介直自守,累歲不遷。會天子西還,子玄自乞留東都。三年,或言子玄身史臣而私著述,驛召至京,領史事。遷秘書少監。時宰相韋巨源、紀處訥、楊再思、宗楚客、蕭至忠皆領監脩,子玄病長官多,意尚不一,而至忠數責論次無功,又仕偃蹇,乃奏記求罷去。因為至忠言「五不可」,曰:「古之國史,皆出一家,未聞藉功於眾。唯漢東觀集群儒,纂述無主,條章不建。今史司取士滋多,人自為荀、袁,家自為政、駿。每記一事,載一言,閣筆相視,含毫不斷,頭白可期,汗青無日:一不可。漢郡國計書上太史,副上丞相,後漢公卿所撰,先集公府,乃上蘭臺,故史官載事為廣。今史臣唯自詢採,二史不注起居,百家弗通行狀:二不可。史局深籍禁門,所以杜顏面,防請謁也。今作者如林,儻示褒貶,曾未絕口,而朝野咸知。孫盛取嫉權門,王劭見讎貴族,常人之情,不能無畏:三不可。古者史氏各有指歸,故司馬遷退處士,進奸雄;班固抑忠臣,飾主闕。今史官注記,類稟監脩,或須直辭,或當隱惡,十羊九牧,其令難行:四不可。今監者不肯指授,脩者又不遵奉,務相推避,以延歲月:五不可。」又言:「朝廷厚用其才而薄其禮。」至忠得書,悵惜不許。楚客等惡其言詆切,謂諸史官曰:「是子作書,欲致吾何地?」



 始,子玄修《武後實錄》,有所改正,而武三思等不聽。自以為見用於時而志不遂,乃著《史通》內外四十九篇,譏評今古。徐堅讀之,嘆曰:「為史氏者宜置此坐右也。」又嘗自比楊雄者四:「雄好雕蟲小伎,老而為悔;吾幼喜詩賦而壯不為,期以述者自名。雄準《易》作經,當時笑之;吾作《史通》,俗以為愚。雄著書見尤於人,作《解嘲》;吾亦作《釋蒙》。雄少為範逡、劉歆所器,及聞作經,以為必覆醬瓿;吾始以文章得譽,晚談史傳,由是減價。」其自感慨如此。



 子玄內負有所未盡,乃委國史於吳兢,別撰《劉氏家史》及《譜考》。上推漢為陸終苗裔,非堯後;彭城叢亭里諸劉,出楚孝王囂曾孫居巢侯般,不承元王。按據明審,議者高其博。嘗曰:「吾若得封,必以居巢紹司徒舊邑。」後果封居巢縣子。鄉人以其兄弟六人俱有名,號其鄉曰高陽,里曰居巢。



 累遷太子左庶子、兼崇文館學士。皇太子將釋奠國學,有司具儀:從臣著衣冠,乘馬。子玄議:「古大夫以上皆乘車,以馬為騑服。魏、晉後以牛駕車。江左尚書郎輒輕乘馬,則御史劾治。顏延年罷官,乘馬出入閭里,世稱放誕。此則乘馬宜從褻服之明驗。今陵廟巡謁、王公冊命、士庶親迎,則盛服冠履,乘輅車。他事無車,故貴賤通乘馬。比法駕所幸,侍臣皆馬上朝服。且冠履惟可配車,故博帶褒衣、革履高冠,是車中服。韈而鐙,跣而鞍,非唯不師於古,亦自取驚流俗。馬逸人顛,受嗤行路。」太子從之,因著為定令。



 開元初,遷左散騎常侍。嘗議《孝經》鄭氏學非康成注,舉十二條左證其謬,當以古文為正;《易》無子夏傳,《老子》書無河上公注,請存王弼學。宰相宋璟等不然其論,奏與諸儒質辯。博士司馬貞等阿意,共黜其言,請二家兼行,惟子夏《易傳》請罷。詔可。會子貺為太樂令,抵罪,子玄請於執政,玄宗怒,貶安州別駕。卒,年六十一。



 子玄領國史且三十年,官雖徙,職常如舊。禮部尚書鄭惟忠嘗問:「自古文士多,史才少,何耶?」對曰:「史有三長:才、學、識。世罕兼之,故史者少。夫有學無才,猶愚賈操金,不能殖貨;有才無學,猶巧匠無楩柟斧斤,弗能成室。善惡必書,使驕君賊臣知懼,此為無可加者。」時以為篤論。子玄善持論,辯據明銳,視諸儒皆出其下,朝有論著輒豫。歿後,帝詔河南就家寫《史通》,讀之稱善。追贈工部尚書,謚曰文。



 六子:貺、餗、匯、秩、迅、迥。



 貺,字惠卿。好學,多通解。子玄卒,有詔訪其後,擢起居郎。歷右拾遺內供奉。獻《續說苑》十篇,以廣漢劉向所遺,而刊落怪妄。貺嘗以《竹書紀年》序諸侯列會皆舉謚,後人追修,非當時正史。如齊人殲於遂,鄭棄其師,皆孔子新意,《師春》一篇錄卜筮事,與左氏合,知按《春秋》經傳而為也,因著《外傳》云。子滋、浹。



 滋,字公茂。通經術,喜持論。以廕歷漣水令。楊綰薦材堪諫官,累授左補闕。久之,去,養親東都。河南尹李廙奏補功曹,母喪解。服除,以司勛員外郎判南曹,勤職奉法,進至給事中。興元元年,以吏部侍郎知南選。時大盜後,旱蝗相仍,吏不能詣京師,故命滋至洪州調補,以振職聞。貞元二年,擢左散騎常侍、同中書門下平章事。為相無所設施,廉抑畏慎而已。明年罷。又明年,復為吏部侍郎,遷尚書。會御史中丞韋貞伯劾奏:「吏選不實,澄覆疏舛,吏因得為奸。」詔與侍郎杜黃裳奪階。卒,贈陜州大都督,謚曰貞。



 浹亦有學稱。生子敦儒,家東都。母病狂易,非笞掠人不能安,左右皆亡去,敦儒日侍疾,體常流血,母乃能下食,敦儒怡然不為痛隱。留守韋夏卿表其行,詔標闕於閭。元和中,權德輿復薦之,乃授左龍武軍兵曹參軍,分司東都。在母喪,毀瘠幾死。時謂劉孝子。後為起居郎,達禮好古,有祖風雲。



 餗,字鼎卿。天寶初,歷集賢院學士,兼知史官。終右補闕。父子三人更涖史官,著《史例》,頗有法。



 匯,左散騎常侍,終荊南節度使。子贊,以廕仕為鄠丞。杜鴻漸自劍南還,過鄠,廚驛豐給。楊炎薦匯名儒子,擢浙西觀察判官。炎入相,進歙州刺史,政干強濟。野媼將為虎噬,幼女呼號搏虎,俱免。觀察使韓滉表贊治有異行,加金紫,徙常州。滉輔政,分所統為三道,以贊為宣州刺史、都團練觀察使,治宣十年。贊本無學,弟以剛猛立威,官吏重足一跡。宣既富饒,即厚斂,廣貢奉以結恩。又不能訓子,皆驕傲不度,素業衰矣。卒,贈吏部尚書,謚曰敬。



 迥以剛直稱,第進士,歷殿中侍御史,佐江淮轉運使。時新更安史亂,迥饋運財賦,力於職。大歷初,為吉州刺史,治行尤異。累遷給事中。



 秩,字祚卿。開元末,歷左監門衛錄事參軍事,稍遷憲部員外郎。坐小累,下除隴西司馬。安祿山反,哥舒翰守潼關,楊國忠欲奪其兵,秩上言:「翰兵天下成敗所系,不可忽。」房琯見其書,以比劉更生。至德初,遷給事中。久之,出為閬州刺史。貶撫州長史,卒。所著《政典》、《止戈記》、《至德新議》等凡數十篇。



 迅,字捷卿。歷京兆功曹參軍事。常寢疾,房琯聞,憂不寐,曰:「捷卿有不諱,天理欺矣!」陳郡殷寅名知人,見迅嘆曰:「今黃叔度也!」劉晏每聞其論,曰:「皇王之道盡矣!」上元中,避地安康,卒。迅續《詩》、《書》、《春秋》、《禮》、《樂》五說。書成,語人曰:「天下滔滔,知我者希。」終不以示人云。



 吳兢,汴州浚儀人。少厲志,貫知經史,方直寡諧比,惟與魏元忠、硃敬則游。二人者當路,薦兢才堪論撰,詔直史館,修國史。遷右拾遺內供奉。



 神龍中,改右補闕。節閔太子難,奸臣誣構安國相王與謀,朝廷大恐。兢上言:「文明後,皇運不殊如帶。陛下龍興,恩被骨肉,相王與陛下同氣,親莫加焉。今賊臣日夜陰謀,必欲寘之極法。相王仁孝,遭荼苦哀毀,以陛下為命,而自托於手足。若信邪佞,委之於法,傷陛下之恩,失天下望。芟刈股肱,獨任胸臆,可為寒心。自昔翦伐宗支,委任異姓,未有不亡者。秦任趙高,漢任王莽,晉家自相魚肉,隋室猜忌子弟,海內麋沸,驗之覆車,安可重跡?且根朽者葉枯,源涸者游竭。子弟,國之根源,可使枯竭哉!皇家枝幹,夷芟略盡。陛下即位四年,一子弄兵被誅,一子以罪謫去,惟相王朝夕左右。『斗粟』之刺,《蒼蠅》之詩,不可不察。伏願陛下全常棣之恩,慰罔極之心,天下幸甚!」累遷起居郎,與劉子玄、徐堅等並職。



 玄宗初立,收還權綱,銳於決事,群臣畏伏。兢慮帝果而不及精,乃上疏曰:



 自古人臣不諫則國危,諫則身危。臣愚,食陛下祿,不敢避身危之禍。比見上封事者,言有可採,但賜束帛而已,未嘗蒙召見,被拔擢。其忤旨,則朝堂決杖,傳送本州,或死於流貶。由是臣下不敢進諫。古者設誹謗木,欲聞己過;今封事,謗木比也。使所言是,有益於國;使所言非,無累於朝。陛下何遽加斥逐,以杜塞直言?道路流傳,相視怪愕。夫漢高帝赦周昌桀、紂之對,晉武帝受劉毅桓、靈之譏,況陛下豁達大度,不能容此狂直耶?夫人主居尊極之位,顓生殺之權,其為威嚴峻矣。開情抱,納諫諍,下猶懼不敢盡,奈何以為罪?且上有所失,下必知之。故鄭人欲毀鄉校,而子產不聽也。陛下初即位,猶有褚無量、張廷珪、韓思復、辛替否、柳澤、袁楚客等數上疏爭時政得失。自頃上封事,往往得罪,諫者頓少。是鵲巢覆而鳳不至,理之然也。臣誠恐天下骨鯁士以讜言為戒,橈直就曲,斗方為刓,偷合茍容,不復能盡節忘身,納君於道矣。



 夫帝王之德,莫盛於納諫。故曰:「木從繩則正,後從諫則聖。」又曰:「朝有諷諫,猶發之有梳。猛虎在山林,藜藿為之不採。」忠諫之有益如此。自古上聖之君,恐不聞己過,故堯設諫鼓,禹拜昌言。不肖之主,自謂聖智,拒諫害忠,桀殺關龍逢而滅於湯,紂殺王子比干而滅於周,此其驗也。夫與治同道罔不興,與亂同道罔不亡。人將疾,必先不甘魚肉之味;國將亡,必先不甘忠諫之說。嗚呼,惟陛下深監於茲哉!隋煬帝驕矜自負,以為堯、舜莫己若,而諱亡憎諫。乃曰:「有諫我者,當時不殺,後必殺之。」大臣蘇威欲開一言,不敢發,因五月五日獻《古文尚書》,帝以為訕己,即除名。蕭瑀諫無伐遼,出為河池郡守。董純諫無幸江都,就獄賜死。自是蹇諤之士,去而不顧,外雖有變,朝臣鉗口,帝不知也。身死人手,子孫剿絕,為天下笑。太宗皇帝好悅至言,時有魏徵、王珪、虞世南、李大亮、岑文本、劉洎、馬周、褚遂良、杜正倫、高季輔,咸以切諫,引居要職。嘗謂宰相曰:「自知者為難。如文人巧工,自謂己長,若使達者、大匠詆訶商略,則蕪辭拙跡見矣。天下萬機,一人聽斷,雖甚憂勞,不能盡善。今魏徵隨事諫正,多中朕失,如明鑒照形,美惡畢見。」當是時,有上書益於政者,皆黏寢殿之壁,坐望臥觀,雖狂瞽逆意,終不以為忤。故外事必聞,刑戮幾措,禮義大行。陛下何不遵此道,與聖祖繼美乎?夫以一人之意,綜萬方之政,明有所不燭,智有所不周,上心未諭於下,下情未達於上。伏惟以虛受人,博覽兼聽,使深者不隱,遠者不塞,所謂「闢四門、明四目」也。其能直言正諫不避死亡之誅者,特加寵榮,待以不次,則失之東隅,冀得之桑榆矣。



 尋以母喪去官。服除,自陳修史有緒,家貧不能具紙筆,願得少祿以終餘功。有詔拜諫議大夫,復修史。睿宗崩,實錄留東都,詔兢馳驛取進梓宮。以父喪解,宰相張說用趙冬曦代之。終喪,為太子左庶子。



 開元十三年,帝東封太山,道中數馳射為樂。兢諫曰:「方登岱告成,不當逐狡獸,使有垂堂之危、朽株之殆。」帝納之。明年六月,大風,詔群臣陳得失。兢上疏曰:「自春以來,亢陽不雨,乃六月戊午,大風拔樹,壞居人廬舍。傳曰:『敬德不用,厥災旱。上下蔽隔,庶位逾節,陰侵於陽,則旱災應』。又曰:『政悖德隱,厥風發屋壞木。』風,陰類,大臣之象。恐陛下左右有奸臣擅權,懷謀上之心。臣聞百王之失,皆由權移於下,故曰:『人主與人權,猶倒持太阿,授之以柄。』夫天降災異,欲人主感悟,願深察天變,杜絕其萌。且陛下承天后、和帝之亂,府庫未充,冗員尚繁,戶口流散,法出多門,賕謁大行,趨競彌廣。此弊未革,實陛下庶政之闕也,臣不勝惓惓。願斥屏群小,不為慢游,出不御之女,減不急之馬,明選舉,慎刑罰,杜僥幸,存至公,雖有旱風之變,不足累聖德矣。」



 始,兢在長安、景龍間任史事,時武三思、張易之等監領,阿貴朋佞,釀澤浮辭,事多不實。兢不得志,私撰《唐書》、《唐春秋》,未就。至是,丐官筆札,冀得成書。詔兢就集賢院論次。時張說罷宰相,在家修史。大臣奏國史不容在外,詔兢等赴館撰錄。進封長垣縣男。久之,坐書事不當,貶荊州司馬,以史草自隨。蕭嵩領國史,奏遣使者就兢取書,得六十餘篇。



 累遷洪州刺史,坐累下除舒州。天寶初,入為恆王傅。雖年老衰僂甚,意猶願還史職。李林甫嫌其衰,不用。卒,年八十。



 兢敘事簡核,號良史。晚節稍疏牾。時人病其太簡。初與劉子玄撰定《武後實錄》,敘張昌宗誘張說誣證魏元忠事,頗言「說已然可,賴宋璟等邀勵苦切,故轉禍為忠,不然,皇嗣且殆。」後說為相,讀之,心不善,知兢所為,即從容謬謂曰:「劉生書魏齊公事,不少假借,奈何?」兢曰:「子玄已亡,不可受誣地下。兢實書之,其草故在。」聞者嘆其直。說屢以情蘄改,辭曰:「徇公之情,何名實錄?」卒不改。世謂今董狐云。



 韋述,弘機曾孫。家廚書二千卷,述為兒時,誦憶略遍。父景駿,景龍中為肥鄉令,述從到官。元行沖,景駿姑子也,為時儒宗,常載書數車自隨。述入其室觀書,不知寢食,行沖異之,試與語前世事,孰復詳諦,如指掌然。使屬文,受紙輒就。行沖曰:「外家之寶也。」舉進士,時述方少,儀質陋侻,考功員外郎宋之問曰:「童子何業?」述曰:「性嗜書,所撰《唐春秋》三十篇,恨未畢,它唯命。」之問曰:「本求茂才,乃得遷、固。」遂上第。



 開元初,為櫟陽尉。秘書監馬懷素奏述與諸儒即秘書續《七志》,五年而成。述好譜學,見柳沖所撰《姓族系錄》,每私寫懷之,還舍則又繕錄,故於百氏源派為詳,乃更撰《開元譜》二十篇。累除右補闕。張說既領集賢院,薦述為直學士,遷起居舍人。從封太山,奏《東封記》,有詔褒美。先是,詔修《六典》,徐堅構意歲餘,嘆曰:「吾更修七書,而《六典》歷年未有所適。」及蕭嵩引述撰定,述始摹周六官領其屬,事歸於職,規制遂定。初,令狐德棻、吳兢等撰武德以來國史,皆不能成。述因二家參以後事,遂分紀、傳,又為例一篇。嵩欲蚤就,復奏起居舍人賈登、著作佐郎李銳助述紬績。逮成,文約事詳,蕭穎士以為譙周、陳壽之流。改國子司業,充集賢學士,累遷工部侍郎,封方城縣侯。



 述典掌圖書,餘四十年,任史官二十年,淡榮利,為人純厚長者,當世宗之。接士無貴賤與均。蓄書二萬卷,皆手校定,黃墨精謹,內秘書不逮也。古草隸帖、秘書、古器圖譜無不備。安祿山亂,剽失皆盡,述獨抱國史藏南山。身陷賊,污偽官。賊平,流渝州,為刺史薛舒所困,不食死。廣德初,甥蕭直為李光弼判官,詣闕奏事稱旨。因理述「蒼卒奔逼,能存國史,賊平,盡送史官於休烈,以功補過,宜蒙恩宥。」有詔贈右散騎常侍。



 韋氏之顯者,孝友、詞學則承慶、嗣立,邃音樂有萬石,達禮儀則叔夏,史才博識有述。所著書二百餘篇行於時。弟逌、迪,學業亦亞述。與逌對為學士,與迪並禮官,搢紳高之。時趙冬曦兄弟亦各有名。張說嘗曰「韋、趙兄弟,人之杞梓」云。



 蔣乂,字德源,常州義興人,徙家河南。祖環,開元中弘文館學士。父將明,天寶末,闢河中使府。安祿山反,以計佐其帥,全並、潞等州。兩京陷,被拘,乃陽狂以免。虢王巨引致幕府,歷侍御史,擢左司郎中、國子司業、集賢殿學士。乂性銳敏,七歲時,見庾信《哀江南賦》,再讀輒誦。外祖吳兢位史官,乂幼從外家學,得其書,博覽強記。逮冠,該綜群籍,有史才,司徒楊綰尤稱之。將明在集賢,值兵興,圖籍殽舛,白宰相請引乂入院,助力整比。宰相張鎰亦奇之,署集賢小職。乂料次逾年,各以部分,得善書二萬卷。再遷王屋尉,充太常禮院修撰。貞元九年,擢右拾遺、史館修撰。德宗重其職,先召見延英,乃命之。



 張孝忠子茂宗尚義章公主,母亡,遺言丐成禮。帝念孝忠功,即日召為左衛將軍,許主下降。乂上疏,以為:「墨縗禮本緣金革,未有奪喪尚主者。繆盩典禮,違人情,不可為法。」帝令中使者諭茂宗之母之請,乂意殊堅。帝曰:「卿所言,古禮也。今俗借吉而婚不為少。」對曰:「俚室窮人子,旁無至親,乃有借吉以嫁,不聞男冒兇而娶。陛下建中詔書,郡、縣主當婚,皆使有司循典故,毋用俗儀。公主春秋少,待年不為晚,請茂宗如禮便。」帝曰:「更思之。」會太常博士韋彤、裴堪諫曰:「婚禮,主人幾筵聽命,稱事立文,謂之嘉,所以承宗廟,繼後嗣也。喪禮,創巨者日久,痛甚者愈遲,二十五月而畢,謂之兇,所以送死報終,示有節也。故夫義婦聽,父慈子孝。昔魯侯改服,晉襄墨縗,緣金革事則有權變。安有釋縗服,衣冕裳,去堊室,行親迎,以兇瀆嘉,為朝廷爽法?」疏入,帝迂其言,促行前詔,然心嘉乂有守。



 十八年,遷起居舍人,轉司勛員外,皆兼史任。帝嘗登凌煙閣,視左壁頹剝,題文漫缺,行才數字,命錄以問宰相,無能知者。遽召乂至,答曰:「此聖歷中侍臣圖贊。」帝前口以誦補,不失一字。帝嘆曰:「雖虞世南默寫《列女傳》,不是過。」會詔問神策軍建置本末,中書討求不獲,時集賢學士甚眾,悉亡以對。乃訪乂,乂條據甚詳。宰相高郢、鄭珣瑜嘆曰:「集賢有人哉!」明日,詔兼判集賢院事。父子為學士,儒者榮之。



 順宗既葬,議祧廟,有司以中宗中興之君,當百代不遷。宰相問乂,乂曰:「中宗即位,春秋已壯,而母後篡奪以移神器,賴張柬之等國祚再復,蓋曰反正,不得為中興。凡非我失之,自我復之,為中興,漢光武、晉元是也。自我失之,因人復之,晉孝惠、孝安是也。今中宗與惠、安二帝同,不可為不遷主。」有司疑曰:「五王有安社稷功,若遷中宗,則配饗永絕。」乂曰:「禘袷功臣,乃合食太廟。中宗廟雖毀,而禘祫並陳太廟,此則五王配食與初一也。」由是遷廟遂定。遷兵部郎中。與許孟容、韋貫之刪正制敕三十篇,為《開元格後敕》。李錡誅,詔宗正削一房屬籍。宰相召乂問:「一房自大功可乎?」答曰:「大功,錡之從父昆弟。其祖神通有功,配饗於廟,雖裔孫之惡,而忘其勛,不可。」「自期可乎?」曰:「期者錡昆弟。其父若幽死社稷,今以錡連坐,不可。」執政然之。故罪止錡及子息,無旁坐者。



 未幾,改秘書少監,復兼史館修撰,與獨孤鬱、韋處厚修《德宗實錄》。以勞遷右諫議大夫。裴垍罷宰相,而李吉甫惡垍,以嘗監修,故授乂太常少卿。久之,遷秘書監,累封義興縣公。卒,年七十五,贈禮部尚書,謚曰懿。



 乂在朝廷久,居史職二十年。每有大政事議論,宰相未能決,必咨訪之,乂據經義或舊章以參時事,其對允切該詳。初以是被遇,終亦忤貴近,介介不至顯官。然資質樸直,遇權臣秉政,輒數歲不遷。嘗疏裴延齡罪惡及拒王叔文,當世高之。結發志學,老而不厭,雖甚寒暑,卷不釋於前,故能通百家學,尤明前世沿革。家藏書至萬五千卷。初名武,憲宗時因進見,請曰:「陛下今日偃武修文,群臣當順承上意,請改名乂。」帝悅。時討王承宗兵方罷,乂恐天子銳於武,亦因以諷。它日,帝見侍御史唐武曰:「命名固多,何必曰武?乂既改之矣。」更曰慶。群臣乃知帝且厭兵云。乂論撰百餘篇。



 五子:人系、伸、偕知名,仙、佶皆位刺史。



 人系善屬文,得父典實。大和初,授昭應尉,直史館。明年,拜右拾遺、史館修撰,與沈傳師、鄭澣、陳夷行、李漢參撰《憲宗實錄》。轉右補闕。宋申錫被誣,文宗怒甚,人系與左常侍崔玄亮涕泣苦諍,申錫得不死。歷膳部員外、工禮兵三部郎中,皆兼史職。開成末,轉諫議大夫。宰相李德裕惡李漢,以人系友婿,出為桂管觀察使,人安其治。復坐漢貶唐州刺史。宣宗立,召為給事中、集賢殿學士判院事。轉吏部侍郎,歷興元、鳳翔節度使。懿宗初,拜兵部尚書,以弟伸位丞相,懇辭,乃檢校尚書右僕射,節度山南東道,封淮陽郡公。徙東都留守,卒。子曙,字耀之。咸通末,由進士第署鄂岳團練判官,除虞、工二部員外,改起居郎。黃巢之難,曙闔門無噍類,以是絕意仕進,隱居沈痛。中和二年,表請為道士,許之。



 伸,字大直,第進士。大中二年,以右補闕為史館修撰,轉駕部郎中,知制誥。白敏中領邠寧節度,表伸自副,加右庶子。入知戶部侍郎。九年,為翰林學士,進承旨。十年,改兵部侍郎,判戶部。



 宣宗雅信愛伸,每見必咨天下得失。伸言:「比爵賞稍易,人且偷。」帝愕然曰:「偷則亂矣。」伸曰:「否,非遽亂,但人有覬心,亂由是生。」帝嗟嘆,伸三起三留,曰:「它日不復獨對卿矣。」伸不諭。未幾,以本官同中書門下平章事。逾四月,解戶部,加中書侍郎。懿宗即位,兼刑部尚書,監修國史。咸通二年,出為河中節度使、同中書門下平章事,徙宣武。俄以太子少保分司東都。七年,用為華州刺史。再遷太子太傅,表乞骸骨,以本官致仕。卒,贈太尉。



 偕以父任,歷右拾遺、史館修撰,轉補闕、主客郎中。初,柳芳作《唐歷》,大歷以後闕而不錄,宣宗詔崔龜從、韋澳、李荀、張彥遠及偕等分年撰次,盡元和以續雲。累遷太常少卿。大中八年,與盧耽、牛叢、王渢、盧告撰次《文宗實錄》。蔣氏世禪儒,唯伸及人系子兆能以辭章取進士第,然不為文士所多。三世踵修國史,世稱良筆,咸云「蔣氏日歷」,天下多藏焉。



 柳芳,字仲敷,蒲州河東人。開元末,擢進士第,由永寧尉直史館。肅宗詔芳與韋述綴輯吳兢所次國史,會述死,芳緒成之,興高祖,訖乾元,凡百三十篇。敘天寶後事,棄取不倫,史官病之。上元中,坐事徙黔中。後歷左金吾衛騎曹參軍、史館修撰。然芳篤志論著,不少選忘厭。承寇亂史籍淪缺。芳始謫時,高力士亦貶巫州,因從力士質開元、天寶及禁中事,具識本末。時國史已送官,不可追刊,乃推衍義類,仿編年法,為《唐歷》四十篇,頗有異聞。然不立褒貶義例,為諸儒譏訕。改右司郎中、集賢殿學士,卒。



 子登、冕。



 登,字成伯。淹貫群書,年六十餘,始仕宦。元和初,為大理少卿,與許孟容等刊正敕格。以病改右散騎常侍,致仕。卒,年九十餘,贈工部尚書。



 子璟,字德輝。寶歷初,第進士、宏詞,三遷監察御史。時郊廟告祭,吏部以雜品攝上公。璟據開元、元和詔書,太尉以宰相攝事,司空、司徒以僕射、尚書、師、傅攝,餘司不及差限,請如舊制,從之。累遷吏部員外郎。文宗開成初,為翰林學士。初,芳永泰中按宗正牒,斷自武德,以昭穆系承撰《永泰新譜》二十篇。璟因召對,帝嘆《新譜》詳悉,詔璟攟摭永泰後事綴成之。復為十篇,戶部供筆札稟料。遷中書舍人。武宗立,轉禮部侍郎。璟為人寬信,好接士,稱人之長,游其門者它日皆顯於世。會昌二年,再主貢部,坐其子招賄,貶信州司馬,終郴州刺史。



 冕,字敬叔。博學富文辭,且世史官,父子並居集賢院。歷右補闕、史館修撰。坐善劉晏,貶巴州司戶參軍。還為太常博士。昭德王皇后崩,冕與張薦議皇太子宜依晉魏卒哭除服,左補闕穆質請依禮期而除,冕議見用。德宗既親郊,重慎祠事,動稽典禮。冕以吏部郎中攝太常博士,與薦及司封郎中徐岱、倉部郎中陸質修飭儀矩。帝疑郊廟每升輒去劍履及象劍尺寸、祝語輕重,冕據禮以對,本末詳明,天子嘉異。



 久之,以論議勁切,執政不善,出為婺州刺史。十三年,兼御史中丞、福建觀察使。自以久疏斥,又性躁狷,不能無恨,乃上表乞代,且推明朝覲之意,曰:「臣竊感《江漢》朝宗之誼,《鹿鳴》君臣之宴,頌聲之作,王道本始。國家自兵興,不遑議禮,方牧未朝,宴樂久缺。臣限一切之制,例無朝集,目不睹朝廷之禮,耳不聞宗廟之樂,足不踐軒墀之地,十有二年於茲矣。夫朝會,禮之本也。唐、虞之制,群後四朝,以明黜陟。商、周之盛,五歲一見,以考制度。漢法,三載上計,以會課最。聖唐稽古,天下朝集,三考一見,皆以十月上計京師,十一月禮見,會尚書省應考績事,元日陳貢棐,集於考堂,唱其考第,進賢以興善,簡不肖以黜惡。自安史亂常,始有專地;四方多故,始有不朝;戎臣恃險,或不悔過。臣忝牧圉之寄,憤不朝之臣,思一入覲,率先天下,使君臣之義,親而不疏;朝覲之禮,廢而復舉。誠恐負薪,溘先朝露,覲禮不展,臣之憂也。比聞諸將帥亡歿者眾,臣自憚何德以堪久長。鄉國,人情之不忘也;闕庭,臣子所戀也;朝覲,國家大禮也。三者,臣之大願。」表累上,其辭哀切,德宗許還。會冕奏閩中本南朝畜牧地,可息羊馬,置牧區於東越,名萬安監,又置五區於泉州,悉索部內馬驢牛羊合萬餘游畜之。不經時,死耗略盡,復調充之,民間怨苦。坐政無狀,代還。卒,贈工部尚書。



 沈既濟,蘇州吳人。經學該明。吏部侍郎楊炎雅善之,既執政,薦既濟有良史才,召拜左拾遣、史館修撰。



 初,吳兢撰國史,為《則天本紀》,次高宗下。既濟奏議,以為:「則天皇后進以強有,退非德讓,史臣追書,當稱為太后,不宜曰上。中宗雖降居籓邸,而體元繼代,本吾君也,宜稱皇帝,不宜曰廬陵王。睿宗在景龍前,天命未集,假臨大寶,於誼無名,宜曰相王,未容曰帝。且則天改周正朔,立七廟,天命革矣。今以周廁唐,列為帝紀,考於《禮經》,是謂亂名。中宗嗣位在太后前,而敘年制紀反居其下,方之躋僖公,是謂不智。昔漢高后稱制,獨有王諸呂為負漢約,無遷鼎革命事,時孝惠已歿,子非劉氏,不紀呂后,尚誰與哉?議者猶謂不可。況中宗以始年即位,季年復祚,雖尊名中奪,而天命未改,足以首事表年,何所拘閡而列為二紀?魯昭公之出,《春秋》歲書其居曰:『公在乾侯。』君在,雖失位,不敢廢也。請省《天后紀》合《中宗紀》,每歲首,必書孝和在所以統之,曰:『皇帝在房陵,太后行其事,改某制。』紀稱中宗而事述太后,名不失正,禮不違常矣。夫正名所以尊王室,書法所以觀後嗣。且太后遺制,自去帝號,及孝和上謚,開元冊命,而後之名不易。今祔陵配廟,皆以後禮,而獨承統於帝,是有司不時正,失先旨。若後姓氏名諱、才藝智略、崩葬日月,宜入皇后傳,題其篇曰《則天順聖武皇后》云。」議不行。



 德宗立,銳於治。建中二年,詔中書、門下兩省,分置待詔官三十,以見官、故官若同正、試、攝九品以上者,視品給俸,至稟餼、幹力、什器、館宇悉有差;權公錢收子,贍用度。既濟諫曰:「今日之治,患在官煩,不患員少;患不問,不患無人。兩省官自常侍、諫議、補闕、拾遺四十員,日止兩人待對,缺員二十一員未補。若謂見官不足與議,則當更選其人。若廣聰明以收淹滯,先補其缺,何事官外置官?夫置錢取息,有司之權制,非經治法。今置員三十,大抵費月不減百萬,以息準本,須二千萬得息百萬,配戶二百,又當復除其家,且得入流,所損尤甚。今關輔大病,皆言百司息錢毀室破產,積府縣,未有以革。臣計天下財賦耗斁大者唯二事:一兵資,二官俸。自它費十不當二者一。所以黎人重困,杼軸空虛。何則?四方形勢,兵未可去,資費雖廣,不獲已為之。又益以閑官冗食,其弊奈何?藉舊而置猶可,若之何加焉?」事遂寢。



 炎得罪,既濟坐貶處州司戶參軍。後入朝,位禮部員外郎。卒。撰《建中實錄》,時稱其能。



 子傳師。傳師,字子言。材行有餘,能治《春秋》,工書,有楷法。少為杜佑所器。貞元末,舉進士。時給事中許孟容、禮部侍郎權德輿樂挽轂士,號「權、許」。德輿稱之於孟容,孟容曰:「我故人子,盍不過我?」傳師往見,謝曰:「聞之丈人,脫中第,則累公舉矣,故不敢進。。」孟容曰:「如子,可使我急賢詣子,不可使子因舊見我。」遂擢第。德輿門生七十人,推為顏子。



 復登制科,授太子校書郎,以鄠尉直史館,轉右拾遺、左補闕、史館修撰,遷司門員外郎,知制誥。召入翰林為學士,改中書舍人。翰林缺承旨,次當傳師,穆宗欲面命,辭曰:「學士、院長參天子密議,次為宰相,臣自知必不能,願治人一方,為陛下長養之。」因稱疾出。帝遣中使敦召。李德裕素與善,開曉諄切,終不出。遂以本官兼史職。俄出為湖南觀察使。



 方傳師與修《憲宗實錄》,未成,監修杜元穎因建言:「張說、令狐峘在外官論次國書,今槁史殘課,請付傳師即官下成之。」詔可。



 寶歷二年,入拜尚書右丞。復出江西觀察使,徙宣州。傳師於吏治明,吏不敢罔。慎重刑法,每斷獄,召幕府平處,輕重盡合乃論決。嘗擇邸吏尹倫,遲魯不及事,官屬屢白易之,傳師曰:「始吾出長安,誡倫曰:『可闕事,不可多事。』倫如是足矣。」故所蒞以廉靖聞。入為吏部侍郎,卒,年五十九,贈尚書。



 傳師性夷粹無競,更二鎮十年,無書賄入權家。初拜官,宰相欲以姻私托幕府者,傳師固拒曰:「誠爾,願罷所授。」故其僚佐如李景讓、蕭寘、杜牧,極當時選云。治家不威嚴,閨門自化。兄弟子姓,屬無親疏,衣服飲食如一。問餉姻家故人,帑無儲錢,鬻宅以葬。



 子詢,字誠之,亦能文辭,會昌初第進士,補渭南尉。累遷中書舍人,出為浙東觀察使,除戶部侍郎,判度支。咸通四年,為昭義節度使,治尚簡易,人皆便安。奴私侍兒,詢將戮之,奴懼,結牙將為亂,夜攻詢,滅其家。贈兵部尚書、左散騎常侍。劉潼代為節度,馳至,刳奴心,祭其靈坐。



 贊曰:唐興,史官秉筆眾矣。然垂三百年,業巨事叢,簡策挐繁,其間巨盜再興,圖典焚逸,大中以後,史錄不存。雖論著之人,隨世裒掇,而疏舛殘餘,本末顛倒。故聖主賢臣,叛人佞子,善惡汩汩,有所未盡,可為永愾者矣。又舊史之文,猥釀不綱,淺則入俚,簡則及漏。寧當時儒者有所諱而不得騁耶?或因淺仍俗不足於文也?亦有待於後取當而行遠耶?何知幾以來,工訶古人而拙於用己歟!自韓愈為《順宗實錄》,議者閧然不息,卒竄定無完篇,乃知為史者亦難言之。游、夏不能措辭於《春秋》,果可信已!



\end{pinyinscope}