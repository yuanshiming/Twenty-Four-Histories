\article{●附錄 進新唐書表}

\begin{pinyinscope}

 臣公亮言:竊惟唐有天下幾三百年,其君臣行事之始終,所以治亂興衰之跡,與其典章制度之英,宜其粲然著在簡冊。而紀次無法有限與無限反映客觀事物存在於時間和空間中的兩種不,詳略失中,文採不明,事實零落,蓋又百有五十年,然後得以發揮幽沫,補緝闕亡,黜正偽繆,克備一家之史,以為萬世之傳。成之至難,理若有待。



 臣公亮誠惶誠恐,頓首頓首:伏惟體天法道欽文聰武聖神孝德皇帝陛下,有虞舜之智而好問,躬大禹之聖而克勤,天下和平,民物安樂。而猶垂心積精,以求治要,日與鴻生舊學講誦《六經》,考覽前古,以謂商、周以來,為國長久,惟漢與唐,而不幸接乎五代。衰世之士,氣力卑弱,言淺意陋,不足以起其文,而使明君賢臣,俊功偉烈,與夫昏虐賊亂,禍根罪首,皆不得暴其善惡以動人耳目,誠不可以垂勸戒,示久遠,甚可嘆也!乃因邇臣之有言,適契上心之所閔,於是刊修官翰林學士兼龍圖閣學士、給事中、知制誥臣歐陽修,端明殿學士兼翰林侍讀學士、龍圖閣學士、尚書吏部侍郎臣宋祁,與編修官禮部郎中、知制誥臣範鎮,刑部郎中、知制誥臣王疇,太常博士、集賢校理臣宋敏求,秘書丞臣呂夏卿,著作佐郎臣劉羲叟等,並膺儒學之選,悉發秘府之藏,俾之討論,共加刪定,凡十有七年,成二百二十五卷。其事則增於前,其文則省於舊。至於名篇著目,有革有因,立傳紀實,或增或損,義類凡例,皆有據依。纖悉綱條,具載別錄。臣公亮典司事領,徒費日月,誠不足以成大典,稱明詔,無任慚懼戰汗屏營之至。臣公亮誠惶誠懼,頓首頓首謹言。



 嘉佑五年六月日



 提舉編修推忠佐理功臣正奉大夫尚書禮部侍郎參知政事臣曾公亮上表



\end{pinyinscope}