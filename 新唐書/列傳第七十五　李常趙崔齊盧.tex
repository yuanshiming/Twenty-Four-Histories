\article{列傳第七十五 李常趙崔齊盧}

\begin{pinyinscope}

 李揆,字端卿,系出隴西,為冠族,去客滎陽。祖玄道與一點論相對立。指要用矛盾分析的觀點認識和處理問題,防,為文學館學士。父成裕,秘書監。揆性警敏,善文章。開元末,擢進士第,補陳留尉。獻書闕下,試中書,遷右拾遺,再轉起居郎,知宗子表疏,以考功郎中知制誥。扈狩劍南,拜中書舍人。



 乾元二年,宗室請上皇后號曰「翊聖」。肅宗問揆,對曰:「前代後妃,終則有謚,景龍不君,韋氏專恣,乃稱翊聖。今陛下動遵典禮,奈何踵其亂哉?」帝驚曰:「幾誤我家事。」遂止。後即張氏,有子數歲,欲立為太子,而帝意未決。時代宗以封成王,帝從容語揆曰:「成王長,有功,將定太子,卿意謂何?」揆曰:「陛下此言,社稷福也。」因再拜賀。帝曰:「朕計決矣。」



 俄兼禮部侍郎。揆病取士不考實,徒露搜索禁所挾,而迂學陋生,葄枕圖史,且不能自措於詞。乃大陳書廷中,進諸儒約曰:「上選士,第務得才,可盡所欲言。」由是人人稱美。未卒事,拜中書侍郎、同中書門下平章事,修國史,封姑臧縣伯。揆美風儀,善奏對,帝嘆曰:「卿門地、人物、文學皆當世第一,信朝廷羽儀乎!」故時稱三絕。於是京師多盜,至驂衢殺人,尸溝中,吏褫氣。李輔國方橫,請選羽林騎五百,備徼捕。揆曰:「漢以南、北軍相統攝,故周勃因南軍入北軍,以安劉氏。本朝置南、北衙,文武區別,更相檢伺。今以羽林代金吾,忽有非常,何以制之!」輔國議格。



 揆決事明當,然銳於進,且近名。兄楷,有時稱,滯冗官不得遷。呂諲政事出揆遠甚,以故宰相鎮荊南,治聲尤高。揆懼復用,遣吏至諲所,構抉過失,諲密訴諸朝。帝怒,貶揆袁州長史。不三日,以楷為司門員外郎。揆累年乃徙歙州刺史。



 初,苗晉卿數薦元載,揆輕載地寒,謂晉卿曰:「龍章鳳姿士不見用,麞頭鼠目子乃求官邪?」載聞,銜之。及秉政,奏揆試秘書監,江淮養疾。家百口,貧無祿,丐食取給,牧守稍厭慁,則去之,流落凡十六年。載誅,始拜睦州刺史。入為國子祭酒、禮部尚書。



 德宗幸山南,揆素為盧杞所惡,用為入蕃會盟使,拜尚書左僕射。揆辭老,恐死道路,不能達命,帝惻然。杞曰:「和戎者,當練朝廷事,非揆不可。異時年少揆者不敢辭。」揆至蕃,酋長曰:「聞唐有第一人李揆,公是否?」揆畏留,因紿之曰:「彼李揆,安肯來邪?」還。卒鳳州,年七十四,贈司空,謚曰恭。



 常袞,京兆人,天寶末,及進士第。性狷潔,不妄交游。由太子正字,累為中書舍人。文採贍蔚,長於應用,譽重一時。魚朝恩賴寵,兼判國子監。袞奏:「成均之任,當用名儒,不宜以宦臣領職。」始,回紇有戰功者,得留京師,虜性易驕,後乃創邸第、佛祠,或伏甲其間,數出中渭橋,與軍人格鬥,奪含光門魚契走城外。袞建言:「今西蕃盤桓境上,數入寇,若相連結,以乘無備,其變不細,請早圖之。」又天子誕日,諸道爭以侈麗奉獻,不則為老子、浮屠解禱事。袞以為:「漢文帝還千里馬不用,晉武帝焚雉頭裘,宋高祖碎琥珀枕,是三主者,非有聰明大聖以致治安,謹身率下而已。今諸道饋獻,皆淫侈不急,而節度使、刺史非能男耕而女織者,類出於民,是斂怨以媚上也,請皆還之。今軍旅未寧,王畿戶口十不一在,而諸祠寺寫經造像,焚幣埋玉,所以賞賚若比丘、道士、巫祝之流,歲巨萬計。陛下若以易芻粟,減貧民之賦,天下之福豈有量哉!」代宗嘉納。遷禮部侍郎。時宦者劉忠翼權震中外,涇原節度使馬璘為帝寵任,有所干請,袞皆拒卻。



 元載死,拜門下侍郎、同中書門下平章事,弘文、崇文館大學士,與楊綰同執政。綰長厚通可,而袞苛細,以清儉自賢。帝內重綰而顓任之,禮遇信愛,袞弗及也,每所恨忌。會綰卒,袞始當國。



 先是,百官俸寡狹,議增給之。時韓滉使度支,與袞皆任情輕重。滉惡國子司業張參,袞惡太子少詹事趙槊,皆少給之。太子文學為洗馬副,袞姻家任文學者,其給乃在洗馬上。其騁私崇怨類此。故事,日出內廚食賜宰相家,可十人具,袞奏罷之。又將讓堂封,它宰相不從,乃止。政事堂北門,異時宰相過舍人院咨逮政事,至袞乃塞之,以示尊大。懲元載敗,窒賣官之路,然一切以公議格之,非文詞者皆擯不用,故世謂之「濌伯」,以其濌々無賢不肖之辨云。



 袞為相,散官才朝議,而無封爵,郭子儀言於帝,遂加銀青光祿大夫,封河內郡公。德宗即位,袞奏貶崔祐甫為河南少尹。帝怒,使與祐甫換秩,再貶潮州刺史。



 建中初,楊炎輔政,起為福建觀察使。始,閩人未知學,袞至,為設鄉校,使作為文章,親加講導,與為客主鈞禮,觀游燕饗與焉,由是俗一變,歲貢士與內州等。卒於官,年五十五,贈尚書左僕射。其後閩人春秋配享袞於學官云。



 趙憬,字退翁,渭州隴西人。曾祖仁本,仕為吏部侍郎、同東西臺三品。憬志行峻潔,不自炫賈。寶應中,方營泰、建二陵,用度廣,又吐蕃盜邊,天下薦饑,憬褐衣上疏,請殺禮從儉,士林嘆美。試江夏尉,佐諸使府,進太子舍人。母喪免,有芝生壤樹。建中初,擢水部員外郎。湖南觀察使李承表憬自副。承卒,遂代之。召還,闔門不與人交。李泌薦之,對殿中,占奏明辯,通古今,德宗欽悅,拜給事中。



 貞元中,咸安公主降回紇,詔關播為使,而憬以御史中丞副之。異時使者多私齎,以市馬規利入,獨憬不然。使未還,尚書左丞缺,帝曰:「趙憬堪此。」遂以命之。考功歲終,請如至德故事課殿最,憬自言薦果州刺史韋證,以貪敗,請降考。校考使劉滋謂憬知過,更以考升。



 竇參當國,欲抑為刺史,帝不許。參罷,進中書侍郎、同中書門下平章事,與陸贄同輔政。贄於裁決少所讓,又徙憬門下侍郎,繇是不平。自以不任職,數稱疾。時杜黃裳遭奄人讒詆,穆贊、韋武、李宣、盧云等為裴延齡構擯,勢危甚,憬救護申解,皆得免。初,贄約共執退延齡,既對,贄極言其奸,帝色變,憬不為助,遂罷贄,乃始當國。



 憬精治道,常以國本在選賢、節用、薄賦斂、寬刑罰,懇懇為天子言之。又陳前世損益、當時之變,獻《審官六議》。一議相臣,曰:「中外知其賢者用之,能者任之,責材之備,為不可得。」二議庶官,曰:「臣嘗謂拔十得五,賢愚猶半。陛下曰:『何必五也,十二可矣。』故廣任用,明殿最,舉大節,略小瑕,隨能試事,用人之大要也。」三議京司闕官,曰:「今要官闕多,閑官員多。要官以材行,閑官以恩澤,是選拔少,優容眾也。宜補缺員,以育人材。」四議考課,曰:「今內庶僚,外刺史,課最尤者,擢以不次,善矣。臣謂黜陟宜責歲限,若任要重未當遷者,加爵或秩。其餘進退,宜示遲速之常。若課在中、考如限者,平轉而歷試之,即無茍且之心、滯淹之慮。」五議遺滯,曰:「陛下委宰輔舉才,不遍知也,則訪之庶僚;又不遍知也,訪之眾人,眾聲囂然,十譽之未信,一毀之可疑。臣謂宜採士論,以譽多者先用,非大故者勿棄。」六議籓府官屬,曰:「諸使闢署,務得才以重府望,能否已試,則引而置之朝,無俾久滯。」帝皆然之,下詔褒答。輔政五年,卒,年六十一。其息上卒時稿奏,帝悼惜之。贈太子太傅,謚曰貞憲。



 憬性清約,位臺宰,而第室童獲猶儒先生家也。得稟入,先建家廟,而竟不營產。其鎮湖南也,令孤峘、崔儆並為部刺史,不守法,憬以正彈治之,皆遣客暴憬失於朝。及為相,乃擢儆自大理卿為尚書右丞,峘方貶衢州別駕,引為吉州刺史,人以為賢。



 崔造,字玄宰,深州安平人。永泰中,與韓會、盧東美、張正則三人友善,居上元,好言當世事,皆自謂王佐才,故號「四夔」。



 浙西觀察使李棲筠闢為判官,累遷左司員外郎。與劉晏善,晏得罪,貶信州長史。徙建州刺史。硃泚亂,造輒馳檄比州,發所部兵二千以待命,德宗嘉之。京師平,召還,至藍田,自以舅源休與賊同逆,上疏請罪。帝以為有禮,下詔慰勉,擢給事中。



 貞元二年,以給事中同中書門下平章事。帝謂造敢言,為能立事,故不次用之。造久在江左,疾錢穀諸使罔上,或乾沒自私,乃建言:「天下兩稅,請委本道觀察使、刺史選官部送京師。諸道水陸轉運使、度支巡院、江淮轉運使,請悉停,以度支鹽鐵務還尚書省,六曹皆宰相分領。」於是齊映判兵部,李勉刑部,劉滋吏、禮二部,造戶、工二部;又以戶部侍郎元琇判諸道鹽鐵、榷酒事,吉中孚度支諸道兩稅事。而浙江東、西歲入米七十五萬石,方歲饑,更以兩稅準米百萬,豪、壽、洪、潭二十萬,責韓滉杜亞漕送東渭橋。諸道有鹽鐵處,仍置巡院。歲盡,宰相計最殿以聞。造厚元琇,故首命之。時滉方領轉運,有寵於帝,朝廷仰其須。滉持不可改,帝重違之,復以滉滉為江淮轉運使,餘如造請。是秋,江淮米大集,帝美滉功,以滉專領度支諸道鹽鐵、轉運等使。造懼,始托疾辭位,乃罷為太子右庶子,貶琇雷州司戶參軍。於是造所請悉罷,以憂愧卒,年五十一。議者謂造舉不適時,方用之乏,不能權濟大事,雖據舊典,奚能抗一切之制云。



 齊映,瀛州高陽人。舉進士,博學宏詞,中之,補河南府參軍事。滑亳節度使令狐彰署掌書記,彰疾甚,引映托後事。映因說彰納節,歸諸子京師。彰從之,即以女妻映。彰卒,軍亂,映間歸東都。



 三城使馬燧闢為判官。盧杞薦授刑部員外郎。又為鳳翔張鎰判官。映練軍事,論奏數稱旨,進行軍司馬。會德宗出奉天,鎰儒緩不知兵,部將李楚琳者,素慓悍,欲介賊為亂。映與齊抗請先事誅之,鎰不用,更示寬大,徐謂楚琳曰:「欲以君使外,若何?」楚琳恐,夜殺鎰以應賊,映雅為軍中慕賴,故得免。奔奉天,授御史中丞。



 從幸梁,道險澀,常為帝御。會馬駭突,帝恐傷映,詔舍轡,固不去,曰;「馬奔𧾷是,不過傷臣;舍之,或犯清蹕,臣雖死不中償責。」帝嘉嘆,擢給事中。映為人白皙長大,言音鴻爽,故帝常令侍左右,或前馬臚傳詔旨。進中書舍人。貞元二年,以舍人同中書門下平章事,俄改中書侍郎,封河間縣男,與崔造、劉滋並輔政。滋端重寡言,映謙不肯事否可,一顓於造。會造疾,映乃當國。



 吐蕃數入寇,關輔震騷,咸言帝欲避狄。映入諫曰:「戎狄不懲,臣之罪也。然內外恟恟,謂陛下具糗糧,欲治行。夫大幸不再,奈何不與臣等計乎?」因俯伏流涕,天子為感寤。



 後給事中袁高忤帝旨,而映以為尚書左丞、御史大夫。始,映微時,張延賞遇之善。及映相,而延賞為左僕射,數為映畫事,又為所親求官,映不答,延賞恚。既復用,即劾映非宰相器。明年,貶夔州刺史,徙衡州。久之,為桂管、江西兩觀察使。始,映罷不以罪,冀復進,乃掊斂獻貢,以中帝欲。初,諸籓銀大瓶止五尺,李兼為江西,始獻六尺瓶,至映乃八尺雲。卒,年四十八,贈禮部尚書,謚曰忠。



 盧邁,字子玄,河南河南人。性孝友。舉明經入第,補太子正字。以拔萃調河南主簿、集賢校理。公卿交薦之,擢右補闕。三遷吏部員外郎。以族屬客江介,出為滁州刺史。召還,再遷諫議大夫。數條當世病利,進給事中。俄會考課,邁以不滿歲,固辭上考,薦紳高其讓。改尚書右丞。



 將作監元亙攝祠,以私忌不聽誓,御史劾之。帝疑其罰,下尚書省議。邁曰:「按大夫士將祭於公,既視濯而父母死,猶奉祭。禮,散齊有大功喪,致齊有期喪,齊有疾病,聽還舍,不奉祭。無忌日不受誓者。雖令忌日與告,且《春秋》不以家事辭王事,今攝祭特命也,亙以常令拒特命,執非所宜。」遂抵罪。



 以本官同中書門下平章事。進中書侍郎。時陸贄、趙憬專大政,邁居中,治身循法無它過。久之,暴眩省中,輿還第。詔大臣即問,固乞骸骨,罷為太子賓客。卒,年六十,贈太子太傅。



 邁每有功、緦喪,必容稱其服,而情有加焉。叔下邽令休沐過家,邁終日與群子姓均指使,無位貌之異。再娶無子,或勸畜姬媵,對曰:「兄弟之子,猶子也,可以主後。」所得稟賜,皆賑姻舊之乏。其從父弟禋喪還洛陽,過都,邁奏請往哭之,盡哀。時執政自以宰相尊,五服皆不過從問吊,而邁獨不徇時,議者重其仁而亮云。



 贊曰:楊綰之德,陸贄之賢,而袞、憬以為憎,何哉?士固蔽於娼前,然主聽不一,故乘以為奸。昔齊桓、秦堅任管仲、王猛,興區區,霸天下,蓋不以不肖者參之。君臣相諒,果難哉!



\end{pinyinscope}