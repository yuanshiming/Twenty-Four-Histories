\article{志第七 禮樂七}

\begin{pinyinscope}

 四曰嘉禮。



 皇帝加元服。



 有司卜日,告於天地宗廟。



 前一日,尚舍設席於太極殿中楹之間,莞筵紛純,加藻席緇純,加次席黼純。有司設次,展縣,設案,陣車輦。設文官五品以上位於縣東,武官於縣西,六品以下皆於橫街之南,北上。朝集使分方於文武官當品之下,諸親位於四品、五品之下,皇宗親在東,異姓親在西。籓客分方各於朝集使六品之南,諸州使人於朝集使九品之後。又設太師、太尉位於橫街之南,道東,北面西上。典儀於縣東北,贊者二人在南,少退,俱西向。又設門外位於東西朝堂,如元日。



 其日,侍中版奏「請中嚴」。太樂令、鼓吹令帥工人入就位。有司設罍洗於阼階東南,設席於東房內,近西,張帷於東序外。殿中監陳袞服於內席,東領,緇纚、玉簪及櫛三物同箱,在服南。又設莞筵一,紛純,加藻席緇純,加次席黼純,在南。尚食實醴尊於東序外帷內,坫在尊北,實角、觶、柶各一。饌陳於尊西,籩、豆各十二;俎三,在籩、豆之北。設罍洗於尊東。袞冕、玉導置於箱。太常博士一人,立於西階下,東面。諸侍衛之官俱詣閣奉迎,典儀帥贊者及群官以次入就位。太常博士引太常卿升西階,立於西房外,當戶北向。侍中版奏「外辦」。皇帝服空頂黑介幘、絳紗袍,出自西房,即御座立。太師、太尉入就位。典儀曰:「再拜。」贊者承傳,在位者皆再拜。太師升自西階,立於東階上,東面。太尉詣阼階下罍洗,盥手,升自東階,詣東房,取纚櫛箱進,跪奠於御座西端。太師詣御座前跪奏曰:「坐。」皇帝坐。太尉當前少左,跪,脫幘置於箱,櫛畢,設纚,興,少西,東面立。太師降,盥,受冕,右執頂,左執前,升自西階,當前少左,祝曰:「令月吉日,始加元服。壽考惟祺,以介景福。」乃跪,冠,興,復西階上位。太尉前,少左,跪,設簪,結纓,興,復位。皇帝興,適東房。殿中監徹櫛纚箱以退。



 皇帝袞服出,即席南向坐。太尉詣序外帷內,盥手,洗觶,酌醴,加柶覆之,面葉,立於序內,南面。太師進,受醴,面柄,前,北向祝曰:「甘醴唯厚,嘉薦令芳。承天之休,壽考不忘。」退,降立於西階下,東面。將祝,殿中監率進饌者奉饌設於前,皇帝左執觶,右取脯,擩於棨,祭於籩、豆之間。太尉取鸑一以進,皇帝奠觶於薦西,受棨,舒左執本,右絕末以祭,上左手嚌之,授太尉。太尉加於俎,降,立於太師之南。皇帝帨手取觶,以柶祭醴,啐醴,建柶,奠觶於薦東。太師、太尉復橫街南位。典儀曰:「再拜。」贊者承傳,在位者皆再拜。太師、太尉出。侍中前,跪奏「禮畢」。皇帝興,入自東房,在位者以次出。



 皇太子加元服。



 有司豫奏司徒一人為賓,卿一人為贊冠,吏部承以戒之。



 前一日,尚舍設御幄於太極殿,有司設群官之次位,展縣,設案,陳車輿,皆如皇帝之冠。設賓受命位於橫街南道東,贊冠位於其後,少東,皆北面。又設文武官門外位於順天門外道東、西。



 其日,侍中奏「請中嚴」。群官有司皆就位。賓、贊入,立於太極門外道東,西面。黃門侍郎引主節持幡節,中書侍郎引制書案,立於樂縣東南,西面北上。侍中奏「外辦」。皇帝服通天冠、絳紗袍,乘輿出自西房,即御坐。賓、贊入就位。典儀曰:「再拜。」在位皆再拜。侍中及舍人前承制,侍中降至賓前,稱「有制」。公再拜。侍中曰:「將加冠於某之首,公其將事。」公少進,北面再拜稽首,辭曰:「臣不敏,恐不能供事,敢辭。」侍中升奏,又承制降,稱:「制旨,公其將事,無辭。」公再拜。侍中、舍人至卿前稱敕旨,卿再拜。侍中曰:「將加冠於某之首,卿宜贊冠。」卿再拜。黃門侍郎執節立於賓東北,西面。賓再拜受節,付於主節,又再拜。中書侍郎取制書立賓東北,西面,賓再拜,受制書,又再拜。典儀曰:「再拜。」贊者承傳,在位皆再拜。賓、贊出,皇帝降坐,入自東房,在位者以次出。初,賓、贊出門,以制書置於案,引以幡節,威儀鐃吹及九品以上,皆詣東宮朝堂。



 冠前一日,衛尉設賓次於重明門外道西,南向,贊冠於其西南。又設次於門內道西,以待賓、贊。又設皇太子位於閣外道東,西向。三師位於道西,三少位於其南,少退,俱東向。又設軒縣於庭,皇太子受制位於縣北,解劍席於東北,皆北面。



 冠日平明,宮臣皆朝服,其餘公服,集於重明門外朝堂。宗正卿乘車侍從,詣左春坊權停。左右二率各勒所部,屯門列仗。左庶子版奏「請中嚴」。群官有司入就位。設罍洗於東階東南。設冠席於殿上東壁下少南,西向;賓席於西階上,東向;主人席於皇太子席西南,西向;三師席於冠席北,三少席於冠席南。張帷於東序內,設褥席於帷中。又張帷於序外冠席。內直郎陳服於帷內,東領北上:袞冕,金飾象笏;遠游冠。緇布冠,服玄衣、素裳、素韡、白紗中單、青領褾紘裾,履、襪,革帶、大帶,笏。緇纚、犀簪二物同箱,在服南。櫛實於箱,又在南。莞筵四,藻席四,又在南。良紘令實側尊甒醴於序外帷內,設罍洗於尊東,實巾一,角觶、柶各一。太官令實饌豆九、籩九於尊西,俎三在豆北。袞冕,遠游三梁冠、黑介幘,緇布冠青組纓屬於冠,冠、冕各一箱。奉禮郎三人各執立於西階之西,東面北上。主人、贊冠者宗正卿為主人,庶子為贊冠者。升,詣東序帷內少北、戶東,西立。典謁引群官以次入就位。



 初,賓、贊入次,左庶子版奏「外辦」。通事舍人引三師等入就閣外道西位,東面立。皇太子空頂黑介幘、雙童髻、彩衣、紫褲褶、織成褾領、綠紳、烏皮履,乘輿以出。洗馬迎於閣門外,左庶子請降輿,洗馬引之道東位,西向立。左庶子請再拜。三師、三少答拜。乃就階東南位。三師在前,三少在後,千牛二人夾左右,其餘仗衛列於師、保之外。皇太子乃出迎賓,至阼階東,西面立。宗正卿立於門東,西面。賓立於西,東面。宗正卿再拜,賓不答拜。賓入,主人從入,立於縣東北,西面。賓入,贊冠者從,賓詣殿階間,南面。贊冠者立於賓西南,東面。節在賓東少南,西面。制案在贊冠西南,東面。賓執制,皇太子詣受制位,北面立。主節脫節衣,賓稱「有制」。皇太子再拜。宣詔曰:「有制,皇太子某,吉日元服,率由舊章,命太尉某就宮展禮。」皇太子再拜。少傅進詣賓前,受制書,以授皇太子,付於庶子。皇太子升東階,入於東序帷內,近北,南面立。賓升西階,及宗正卿各立席後。



 初,賓升,贊冠者詣罍洗,盥手,升自東階帷內,於主人冠贊之南,俱西面。主人贊冠者引皇太子出,立於席東,西面。賓贊冠者取纚、櫛二箱,坐奠於筵。皇太子進,升筵,西面坐。賓之贊冠者東面坐,脫幘置於箱,櫛畢,設纚,興,少北,南面立。執緇布冠升,賓降一等受之,右執頂,左執前,進,東向立,祝曰:「令月吉日,始加元服。棄厥幼志,慎其成德。壽考惟祺,以介景福。」乃跪,冠,興,復位。皇太子東面立,賓揖皇太子,贊冠者引適東序帷內,服玄衣素裳之服以出,立於席東,西面。賓揖皇太子升筵,西向坐。賓之贊冠者進,跪脫緇布冠置於箱,興,復位。賓降二等,受遠游冠,右執頂,左執前,進,祝曰:「吉月令辰,乃申嘉服,克敬威儀,式昭厥德。眉壽萬歲,永壽胡福。」乃跪,冠,興,復位。皇太子興,賓揖皇太子,贊冠者引適東序帷內,朝服以出,立於席東,西面。賓揖皇太子升筵坐,賓之贊冠者跪脫遠游冠,興,復位。賓降三等受冕,右執頂,左執前,進,祝曰:「以歲之正,以月之令。咸加其服,以成厥德。萬壽無疆,承天之慶。」乃跪,冠,興,復位。每冠,皆贊冠者跪設簪、結纓。



 皇太子興,賓揖皇太子適東序,服袞冕之服以出,立於席東,西面。贊冠者徹纚、櫛箱以入,又取筵入於帷內。主人贊冠者又設醴,皇太子席於室戶西,南向,下莞上藻。賓之贊冠者於東序外帷內,盥手洗觶。典膳郎酌醴,加柶覆之,面柄,授贊冠,立於序內,南面。賓揖皇太子就筵西,南面立。賓進,受醴,加柶,面柄,進,北向立,祝曰:「甘醴唯厚,嘉薦令芳。拜受祭之,以定厥祥。承天之休,壽考不忘。」皇太子拜,受觶。賓復位,東面答拜。贊冠者與進饌者奉饌設於筵前,皇太子升筵坐,左執觶,右取脯,擩於棨,祭於籩、豆之間。贊冠者取韭菹,遍擩於豆,以授皇太子,又祭於籩、豆之間。贊冠者取鸑一,以授皇太子,皇太子奠觶於薦西,興,受鸑,卻左手執本坐,繚右手絕末以祭。止,左手嚌之,興,以授贊冠者,加於俎。皇太子坐,帨手取觶,以柶祭醴三,始扱一祭,又扱再祭,加柶於觶,面葉,興,筵末坐,啐醴,建柶,興,降筵西,南面坐,奠觶,再拜,執觶,興。賓答拜。皇太子降,立於西階之東,南面。賓降,立於西階之西少南,贊冠隨降,立於賓西南,皆東面。賓少進,字之,祝曰:「禮儀既備,令月吉日。昭告厥字,君子攸宜。宜之於嘏,永受保之。奉敕字某。」皇太子再拜曰:「某雖不敏,敢不祗奉。」又再拜。洗馬引太子降阼階位,三師在南,北面,三少在北,南面立。皇太子西面再拜,三師等各再拜以出。典儀曰:「再拜。」贊者承傳,在位者皆再拜。左庶子前,稱「禮畢」。皇太子乘輿以入,侍臣從至閣,賓、贊及宗正卿出就會。



 皇子冠。



 前三日,本司帥其屬筮日、筮賓於聽事。前二日,主人至賓之門外次,東面,賓立於阼階下,西面,儐者進於左,北面,受命出,立於門東,西面,曰:「敢請事。」主人曰:「皇子某王將加冠,請某公教之。」儐者入告,賓出,立於門左,西面,再拜。主人答拜。主人曰:「皇子某王將加冠,願某公教之。」賓曰:「某不敏,恐不能共事,敢辭。」主人曰:「某猶願某公教之。」賓曰:「王重有命,其敢不從。」主人再拜而還,賓拜送。命贊冠者亦如之。



 冠之日,夙興,設洗於阼階東南,席於東房內西墉下。陳衣於席,東領北上:袞冕,遠游冠,緇布冠。緇纚、犀簪、櫛實於箱,在服南。莞筵、藻席各三,在南。設尊於房戶之外西,兩甒玄酒在西,加勺冪。設坫於尊東,置二爵於坫,加冪。豆十、籩十在服北,俎三在籩、豆之北。質明,賓、贊至於主人大門外之次,遠游三梁、緇布冠各一箱,各一人執之,待於西階之西,東面北上。設主人之席於阼階上,西面;賓席於西階上,東面;皇子席於室戶東、房戶西,南面。俱下莞上藻。主人立於阼階下,當東房,西面。諸親立於罍洗東南,西面北上。儐者立於門內道南,北面。皇子雙童髻、空頂幘、彩褲褶、錦紳、烏皮履,立於房內,南面。主人贊冠者立於房內戶東,西面。賓及贊冠者出,立於門西,贊冠者少退,俱東面北上。



 儐者受命於主人,出,立於門東,西面,曰:「敢請事。」賓曰:「皇子某王將加冠、某謹應命。」儐者入告,主人出迎賓,西面再拜,賓答拜。主人揖贊冠者,贊冠者報揖,主人又揖賓,賓報。主人入,賓、贊冠者以次入,及內門,主人揖賓,賓入,贊冠者從之。至內霤,將曲揖,賓報揖。至階,主人立於階東,西面;賓立於階西,東面。主人曰:「請公升。」賓曰:「某備將事,敢辭。」主人曰:「固請公升。」賓曰:「某敢固辭。」主人曰:「終請公升。」賓曰:「某敢終辭。」主人升自阼階,立於席東,西向;賓升自西階,立於席西,東向。贊冠者及庭,盥於洗,升自西階,入於東房,立於主人贊冠者之南,俱西面。



 主人贊冠者引皇子出,立於房戶外西,南面。賓之贊冠者取纚、櫛、簪箱,跪奠於皇子筵東端,興,席東少北,南面立。賓揖皇子,賓、主俱即座。皇子進,升席,南面坐。賓之贊冠者進筵前,北面,跪脫雙童髻置於箱,櫛畢,設纚。賓降,盥,主從降。賓東面辭曰:「願主不降。」主人曰:「公降辱,敢不從降。」賓既盥,詣西階。賓、主一揖一讓,升。主人立於席後,西面,賓立於西階上,東面。執緇布冠者升,賓降一等受之,右執頂,左執前,北面跪,冠,興,復西階上席後,東面立。皇子興,賓揖皇子適房,賓、主俱坐。皇子服青衣素裳之服,出房戶西,南面立。賓揖皇子,皇子進,立於席後,南面。賓降,盥,主人從降,辭對如初。賓跪取爵於篚,興,洗,詣西階,賓、主一揖一讓,升,坐,主人立於席後,西面。賓詣酒尊所,酌酒進皇子筵前,北向立,祝曰:「旨酒既清,嘉薦亶時。始加元服,兄弟具來。孝友時格,永乃保之。」皇子筵西拜爵,賓復西階上,東面答拜。執饌者薦籩、豆於皇子筵前。皇子升座,左執爵,右取脯,手需於棨,祭於籩、豆之間,祭酒,興,筵末坐,啐酒,執爵,興,降筵,奠爵,再拜,執爵興。賓答拜。冠者升筵,跪奠爵於薦東,興,立於筵西,南面。執饌者徹薦爵。



 賓揖皇子,皇子進,升筵,南向坐。賓之贊冠者跪脫緇布冠,置於箱。賓降二等,受遠游冠,冠之。皇子興,賓揖皇子適房,賓、主俱坐。皇子服朝服,出房戶西,南面立。賓、主俱興,賓揖皇子,皇子進,立於席後,南面。賓詣尊所,取爵酌酒,進皇子筵前,北向立,祝曰:「旨酒既湑,嘉薦伊脯。乃申其服,禮儀有序。祭此嘉爵,承天之祜。」皇子筵西拜,受爵,祭饌如初禮。賓揖皇子進,升席,南面坐。賓之贊冠者跪脫進賢冠,賓降三等,受冕,冠之。每冠,皆贊冠者設簪結纓。



 皇子興,賓揖皇子適房,服袞冕以出方房戶西,南面。賓揖皇子,進,立於席後,南面。賓詣酒尊所,取爵酌酒進皇子,祝曰:「旨酒令芳,籩豆有楚。咸加其服,肴升折俎。承天之慶,受福無疆。」皇子筵西拜,受爵。執饌者薦籩、豆,設俎於其南。皇子升筵坐,執爵,祭脯棨。贊冠者取鸑一以授皇子,皇子奠爵於薦西,興,受,坐,祭,左手嚌之,興,加於俎。皇子坐,涚手執爵,祭酒,興,筵末坐,啐酒,降筵西,南面坐,奠爵,再拜,執爵興。賓答拜。皇子升筵坐,奠爵於薦東,興。贊冠者引皇子降,立於西階之東,南面。



 初,皇子降,賓降自西階,直西序東面立。主人降自東階,直東序西面立。賓少進,字之曰:「禮儀既備,令月吉日。昭告其字,爰字孔嘉。君子攸宜,宜之於嘏。永受保之,曰孟某甫。」仲、叔、季唯其所當。皇子曰:「某雖不敏,夙夜祗奉。」賓出,主人送於內門外,主人西面請賓曰:「公辱執事,請禮從者。」賓曰:「某既得將事,敢辭。」主人曰:「敢固以請。」賓曰:「某辭不得命,敢不從?」賓就次,主人入。



 初,賓出,皇子東面見,諸親拜之,皇子答拜。皇子入見內外諸尊於別所。



 賓、主既釋服,改設席,訖,賓、贊俱出次,立於門西。主人出揖賓,賓報揖。主人先入,賓、贊從之。至階,一揖一讓,升坐,俱坐。會訖,賓立於西階上,贊冠者在北,少退,俱東面。主人立於東階上,西面。掌事者奉束帛之篚升,授主人於序端。主人執篚少進,西面立。又掌事者奉幣篚升,立於主人後。幣篚升,牽馬者牽兩馬入,陳於門內,三分庭一在南,北首西上。賓還西階上,北面再拜。主人進,立於楹間,贊冠者立於賓左,少退,俱北面再拜。主人南面,賓、贊進,立於主人之右,俱南面東上。主人授幣,賓受之,退,復位。於主人授幣,掌事者又以幣篚授贊冠者。主人還阼階上,北面拜送,賓、贊降自西階,從者訝受幣。賓當庭實,東面揖,出,牽馬者從出,從者訝受馬於門外。賓降,主人降。送賓於大門,西面再拜。



 若諸臣之嫡子三加,皆祝而冠,又祝而酌,又祝而字。庶子三加,既加,然後酌而祝之,又祝而字。其始冠皆緇布;再加皆進賢;其三加,一品之子以袞冕,二品之子以驚冕,三品之子以毳冕,四品之子以絺冕,五品之子以玄冕,六品至於九品之子以爵弁。其服從之。其即席而冠也,嫡子西面,庶子南面。其筮日,筮賓、贊,遂戒之,及其所以冠之禮,皆如親王。



\end{pinyinscope}