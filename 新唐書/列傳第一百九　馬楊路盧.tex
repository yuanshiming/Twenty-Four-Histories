\article{列傳第一百九 馬楊路盧}

\begin{pinyinscope}

 馬植,字存之,鳳州刺史勛子也。第進士,又擢制策科,補校書郎。由壽州團練副使三遷饒州刺史。開成初客觀真理,提出真理的多元論的理論,認為有用就是真理。,為安南都護。精吏事,以文雅絢飾其政,清凈不煩,洞夷便安。羈縻諸首領皆來納款,遣子弟詣府,請賦租約束。植奏以武陸縣為陸州,即柬首領為刺史。既而州部廢池珠復生。以政最,檢校左散騎常侍,徙黔中觀察使。



 會昌中,召拜光祿卿,遷大理。植自以譽望在當時諸公右,久補外,還朝不得要官,為宰相李德裕所抑,內怨望。宣宗嗣位,白敏中當國,凡德裕所不善,悉不次用之,故植以刑部侍郎領諸道鹽鐵轉運使,遷戶部,俄同中書門下平章事,進中書侍郎。



 初,左軍中尉馬元贄最為帝寵信,賜通天犀帶。而植素與元贄善,至通昭穆,元贄以賜帶遺之。它日對便殿,帝識其帶,以詰植,植震恐,具言狀,於是罷為天平軍節度使。既行,詔捕親吏下御史獄,盡得交私狀,貶常州刺史,以太子賓客分司東都。起為忠武、宣武節度使,卒。



 初,植兼集賢殿大學士,校理楊收道與三院御史遇,不肯避,朝長馮緘錄其騶僕辱之。植怒,奏言:「開元中,麗正殿賜酒,大學士張說以下十八人不知先舉者,說以學士德行相先,遂同舉酒。今緘辱收,與大學士等。請斥之。」中丞令狐綯援故事論救,宣宗釋不問。因著令「三館學士不避行臺」,自植始。臺制:「三院還臺,以一人為朝長雲。」



 楊收,字藏之,自言隋越國公素之裔,世居馮翊。父遺直,德宗時,以上書闕下,仕為濠州錄事參軍,客死姑蘇。



 收七歲而孤,處喪若成人。母長孫親授經,十三通大義。善屬文,所賦輒就,吳人號神童。里人多造門觀賦詩,至壓敗其籓。收嘲之曰:「爾非羸角者,奚用觸吾籓?」切當率類此。及壯,長六尺二寸,廣顙深頤,疏眉目,寡言笑,博學強記,至它藝無不通解。貧甚,以母奉浮屠法,自幼不食肉。約曰:「爾得進士第,乃可食。」



 涔陽耕得古鐘,高尺餘,收扣之,曰:「此姑洗角也。」既矞刂拭,有刻在兩欒,果然。嘗言:「琴通黃鐘、姑洗、無射三均,側出諸調,由羅蔦附灌木然。」時有安涚者,世稱善琴,且知音。收問:「五弦外,其二云何?」涚曰:「世謂周文、武二王所加者。」收曰:「能為《文王操》乎?」涚即以黃鐘為宮而奏之,以少商應大弦,收曰:「止!如子之言,少商,武弦也。且文世安得武聲乎?」涚大驚,因問樂意,收曰:「樂亡久矣。上古祀天地宗廟,皆不用商。周人歌大呂、舞《雲門》以俟天神,歌太蔟、舞《咸池》以俟地祇。大呂、黃鐘之合,陽聲之首。而《雲門》,黃帝樂也;《咸池》,堯樂也。不敢用黃鐘,而以太蔟次之。然則祭天者,圜鐘為宮,黃鐘為角,太蔟為徵,姑洗為羽;祭地者,函鐘為宮,太蔟為角,姑洗為徵,南呂為羽。訖不用商及二少。蓋商聲剛而二少聲下,所以取其正、裁其繁也。漢祭天則用商,而宗廟不用,謂鬼神畏商之剛。西京諸儒惑圜鐘、函鐘之說,故其自受命,郊祀、宗廟樂,唯用黃鐘一均。章帝時,太常丞鮑業始旋十二宮。夫旋宮以七聲為均,均,言韻也,古無韻字,猶言一韻聲也。始以某律為宮,某律為商,某律為角,某律為徵,某律為羽,某律少宮,某律少徵,亦曰『變』,曰『比』。一均成則五聲為之節族,此旋宮也。」乃取律次之以示



 涚。涚時七十餘,以為未始聞,而收未冠也。



 以兄假未仕,不肯舉進士。既假褫褐,乃入京師。明年,擢進士,杜悰表署淮南推官。悰領度支,又節度劍南東西川,輒隨府三遷。宰相馬植表為渭南尉、集賢校理,議補監察御史。收又以假方外遷,誼不可先,固辭。植嗟美為止。復為悰節度府判官。蜀有可縣,直巂州西南,地寬平,多水泉,可灌粳稻。或謂悰計興屯田,省轉饋以飽邊士,悰將從之,收曰:「田可致,兵不可得。且地當蠻沖,本非中國。今輟西南屯士往耕,則姚、巂兵少,賊得乘間。若調兵捍賊,則民疲士怨。假令大穰,蠻得長驅,是資賊糧,豈國計耶?」乃止。



 始,周墀罷宰相,節度東川,表其弟嚴掌書記。俄而墀卒,悰闢為觀察使判官,兄弟並在幕府。未幾,假自浙西判官擢監察御史,而收亦自西川遷,兄弟同臺,世榮其友。以詳禮學改太常博士,而嚴亦自揚州召為監察御使。收因建言:「漢制,總群官而聽曰省,分務而專治曰寺。太常,分務專治者也,所以藏天子之旗常。今旗常因車飾隸太僕,非是。」未及行,以母喪免。服除,從淮南崔鉉府為支使。還,拜侍御史。夏侯孜以宰相領度支,引判度支案。遷長安令。



 懿宗時,擢累中書舍人、翰林學士承旨,以中書侍郎同中書門下平章事。始,南蠻自大中以來,火邕州,掠交趾,調華人往屯,涉氛瘴死者十七,戰無功,蠻勢益張。收議豫章募士三萬,置鎮南軍以拒蠻。悉教蹋張,戰必注滿,蠻不能支。又峙食泛舟餉南海。天子嘉其功,進尚書右僕射,封晉陽縣男。



 既益貴,稍自盛滿,為誇侈,門吏童客倚為奸。中尉楊玄價得君,而收與之厚,收之相,玄價實左右之;乃招四方賕餉數千諉收,不能從,玄價以負己,大恚,陰加毀短。知政凡五年,罷為宣歙觀察使,不敢當兩使稟料,但受刺史俸,留公藏錢七百萬。韋保衡又劾收前用嚴譔為江西節度使,受謝百萬,及它隱盜。明年,貶端州司馬。吏具大舟以須,收不從,曰:「方謫去,可乎?」以二小舸趨官。又明年,流驩州,俄詔內養追賜死。收得詔,謝曰:「輔政無狀,固宜死。今獨一弟嚴以奉先人之祀,使者能假須臾使秉筆乎?」使者從之。收自作書謝天子,丐弟嚴死,奉先臣後。以書授使者,即仰鴆死。帝見書惻然,乃宥嚴,坐收流死者十一人。後三年,詔追雪其辜,復官爵。子鉅、鏻。



 鉅,乾寧初為翰林學士,從入洛,終散騎常侍。鏻至戶部尚書。



 收兄發,字至之。登進士,又中拔萃,累官左司郎中。宣宗追加順、憲二宗尊號,有司議改造廟主,署新謚,詔百官議。發與都官郎中盧搏以為改作主,求古無文,執不可。知禮者韙之。改太常少卿,為蘇州刺史,治以恭長慈幼為先。徙福建觀察使,又以能政聞。朝廷意有治劇才,拜嶺南節度使。承前寬弛,發操下剛嚴,軍遂怨,起為亂,囚傳舍,貶婺州刺史。



 假,字仁之,仕終常州刺史。收與昆弟護喪葬偃師,會者千人。



 嚴,字凜之,舉進士。時王起選士三十人,而楊知至、竇緘、源重、鄭樸及嚴五人皆世胄,起以聞,詔獨收嚴。累遷至工部侍郎、翰林學士。收知政,請補外,拜浙東觀察使。收貶,嚴亦斥為邵州刺史,徙吉王傅。乾符中,以兵部侍郎判度支,卒。子涉、注。



 涉,昭宗時,仕至吏部侍郎。哀帝時,進同中書門下平章事。為人端重有禮法。方賊臣陵慢,王室殘蕩,賢人多罹患。涉受命,與家人泣,語其子凝式曰:「世道方極,吾嬰網羅不能去,將重不幸,禍且累汝。」然以謙靖,終免於禍。注為翰林學士。涉已相,辭內職,為戶部侍郎。



 路巖,字魯瞻,魏州冠氏人。父群,字正夫,通經術,善屬文。性志純潔,親歿,終身不肉食。累官中書舍人、翰林學士承旨,文宗優遇之。居循循謙飭,若不在勢位者。所與交,雖褐衣之賤,待以禮,始終一節。



 巖幼惠敏過人,及進士第,父時故人在方鎮者交闢之,久乃答。懿宗咸通初,自屯田員外郎入翰林為學士,以兵部侍郎同中書門下平章事,年三十六。居位八歲,進至尚書左僕射。



 於是王政秕僻,宰相得用事。巖顧天子荒暗,且以政委己,乃通賂遺,奢肆不法。俄與韋保衡同當國,二人勢動天下,時目其黨為「牛頭阿旁」,言如鬼陰惡可畏也。既權侔則爭,故與保衡還相惡。俄罷巖為劍南西川節度使,承蠻盜邊後,巖力拊循,置定邊軍於邛州,扼大度,治故關,取壇丁子弟教擊刺,使補屯籍,由是西山八國來朝。以勞遷兼中書令,封魏國公。



 始,為相時,委事親吏邊咸。會至德令陳蟠叟奏書願請間言財利,帝召見,則曰:「臣願破邊咸家,可佐軍興。」帝問:「咸何人?」對曰:「宰相巖親吏也。」帝怒,斥蟠叟,自是人無敢言。咸乃與郭籌者相依倚為奸,巖不甚制,軍中惟邊將軍、郭司馬爾,妄給與以結士心。嘗閱武都場,咸、籌蒞之,其議事以書相示則焚之,軍中驚,以有異圖,恟恟,遂聞京師。巖坐是徙荊南節度使,道貶新州刺史,至江陵,免官,流儋州,籍入其家。巖體貌偉麗,美須髯,至江陵兩昔皆白。捕誅咸、籌等。巖至新州,詔賜死,剔取喉,上有司。或言巖嘗密請「三品以上得罪誅殛,剔取喉驗其已死」。俄而自及。



 保衡者,京兆人,字蘊用。父愨,宣宗時,終武昌軍節度使。保衡,咸通中,以右拾遺尚同昌公主,遷起居郎、駙馬都尉。主,郭淑妃所生,懿宗所愛,而妃有寵,故恩禮最異,悉宮中珍玩資予之。俄歷翰林學士承旨,以兵部侍郎同中書門下平章事,自尚主至是裁再期。又進門下侍郎、尚書右僕射。



 性浮淺,既恃恩據權,以嫌愛自肆,所悅即擢,不悅擠之。保衡舉進士王鐸第,於籍、蕭遘與同升,以嘗薄於己,皆見斥。逐楊收,傾路巖,人益畏之。主薨,而寵遇不衰。僖宗立,進司徒。俄為怨家白發陰罪,貶賀州刺史,再貶澄邁令,遂賜死。



 弟保乂,自兵部侍郎貶賓州司戶參軍。而劉瞻等坐主薨見貶者,偕復起。



 盧攜,字子升,其先本範陽,世居鄭。擢進士第,被闢浙東府。入朝為右拾遺,歷臺省,累進戶部侍郎、翰林學士承旨。乾符五年,進同中書門下平章事。俄拜中書侍郎、刑部尚書、弘文館大學士。攜姿陋而語不正,與鄭畋俱李翱甥,同位宰相,然所處議多駁。



 初,王仙芝起河南,攜表宋威、齊克讓、曾袞皆善將,為招討使。及威殺尚君長,賊熾結,益不制,乃以王鐸鎮荊南,為諸道都統。攜不悅。是時,黃巢已破廣州,勢張甚,表求天平節度使,詔宰相百官議。攜素厚高駢,屬令立功,乃固不可巢請,又欲激巢使戰而敗鐸,因授率府率。又徇駢與南詔和親,與畋爭,相恨詈,繇是罷為太子賓客,分司東都。俄為兵部尚書。會駢將張璘破賊,帝復召攜以門下侍郎同平章事。及鐸失守,以駢代之,即按關東諸將為鐸、畋所任者,悉易置。內倚田令孜,而外寄戎政於駢,與奪惟所愛惡。



 後病風足蹇,神智瞑塞,事多決於親吏楊溫、李脩,賄賂顯行。及巢破淮南,璘戰死,忠武兵亂,天下危懼,人皆咎攜,始下詔以巢為天平節度使。詔下,賊已破潼關。明日,以太子賓客罷,分司東都,是夜仰藥死。巢入京師,棺磔尸於長安市。子晏,天祐初為河南尉,柳璨殺之。



 贊曰:盧攜之敗王鐸,私高駢,賊遂卷咸、鎬而西,易若舉毛,可謂朝無人焉。唐將亡,攜為之鴟梟,宜天之假手於賊而磔其枯胔也。



\end{pinyinscope}