\article{志第十八上 歷四上}

\begin{pinyinscope}

 《開元大衍歷》演紀上元閼逢困敦之歲,距開元十二年甲子,積九千六百九十六萬一千七百四十算。



 ○一曰步中朔術



 通法三千四十。



 策實百一十一萬三百四十三。



 揲法八萬九千七百七十三。



 減法九萬一千二百。



 策餘萬五千九百四十三。



 用差萬七千一百二十四。



 掛限八萬七千一十八。



 三元之策十五,餘六百六十四,秒七。



 四象之策二十九,餘千六百一十三。



 中盈分千三百二十八,秒十四。



 朔虛分千四百二十七。



 爻數六十。



 象統二十四。



 以策實乘積算,曰中積分。盈通法得一,為積日。爻數去之,餘起甲子算外,得天正中氣。凡分為小餘,日為大餘。加三元之策,得次氣。凡率相因加者,下有餘秒,皆以類相從。而滿法迭進,用加上位。日盈爻數去之。



 以揲法去中積分,不盡曰歸餘之掛。以減中積分,為朔積分。如通法為日,去命如前,得天正經朔。加一象之日七、餘千一百六十三少,得上弦。倍之,得望。參之,得下弦。四之,是謂一揲,得後月朔。凡四分,一為少,三為太。綜中盈、朔虛分,累益歸餘之掛,每其月閏衰。凡歸餘之掛五萬六千七百六十以上,其歲有閏。因考其閏衰,滿掛限以上,其月合置閏。或以進退,皆以定朔無中氣裁焉。



 凡常氣小餘不滿通法、如中盈分之半已下者,以象統乘之,內秒分,參而伍之,以減策實;不盡,如策餘為日。命常氣初日算外,得沒日。凡經朔小餘不滿朔虛分者,以小餘減通法,餘倍參伍乘之,用減滅法;不盡,如朔虛分為日。命經朔初日算外,得滅日。



 ○二曰發斂術



 天中之策五,餘二百二十一,秒三十一;秒法七十二。



 地中之策六,餘二百六十五,秒八十六;秒法百二十。



 貞悔之策三,餘百三十二,秒百三。



 辰法七百六十。



 刻法三百四。



 各因中節命之,得初候。加天中之策,得次候。又加,得末候。因中氣命之,得公卦用事。以地中之策累加之,得次卦,若以貞悔之策加侯卦,得十有二節之初外卦用事。因四立命之,得春木、夏火、秋金、冬水用事。以貞悔之策減季月中氣,得土王用事。凡相加減而秒母不齊,當令母互乘子,乃加減之;母相乘為法。



 各以能法約其月閏衰,為日,得中氣去經朔日算。求卦、候者,各以天、地之策,累加減之。凡發斂加時,各置其小餘,以六爻乘之,如辰法而一,為半辰之數。不盡者,三約為分。分滿刻法為刻。若令滿象積為刻者,即置不盡之數,十之,十九而一,為分。命辰起子半算外。



 ○三曰步日躔術



 乾實百一十一萬三百七十九太。



 周天度三百六十五,虛分七百七十九太。



 歲差三十六太。



 以盈縮分盈減、縮加三元之策,為定氣所有日及餘。乃十二乘日,又三其小餘,辰法約而一,從之,為定氣辰數。不盡,十之,又約為分。以所入氣並後氣盈縮分,倍六爻乘之,綜兩氣辰數除之,為末率。又列二氣盈縮分,皆倍六爻乘之,各如辰數而一;以少減多,餘為氣差。至後以差加末率,分後以差減末率,為初率。倍氣差,亦倍六爻乘之,復綜兩氣辰數除,為日差。半之,以加減初末,各為定率。以日差至後以減、分後以加氣初定率,為每日盈縮分。乃馴積之,隨所入氣日加、減氣下先、後數,各其日定數。其求朓朒仿此。冬至後為陽復,在盈加之,在縮減之;夏至後為陰復,在縮加之,在盈減之。距四正前一氣,在陰陽變革之際,不可相並,皆因前末為初率。以氣差至前加之,分前減之,為末率。餘依前術,各得所求。其分不滿全數,母又每氣不同,當退法除之。以百為母,半已上,收成一。冬至、夏至偕得天地之中,無有盈、縮。餘各以氣下先後數先減、後加常氣小餘,滿若不足,進退其日,得定大小餘。凡推日月度及軌漏、交蝕,依定氣;注歷,依常氣。以減經朔、弦、望,各其所入日算。若大餘不足減,加爻數,乃減之。減所入定氣日算一,各以日差乘而半之;前少以加、前多以減氣初定率,以乘其所入定氣日算及餘秒。凡除者,先以母通全,內子,乃相乘;母相乘除之。所得以損益朓朒積,各其入朓朒定數。若非朔、望有交者,以十二乘所入日算;三其小餘,辰法除而從之;以乘損益率,如定氣辰數而一。所得以損益朓朒積,各為定數。



 南斗二十六,牛八,婺女十二,虛十,虛分七百七十九太。危十七,營室十六,東壁九,奎十六,婁十二,胃十四,昴十一,畢十七,觜觿一,參十,東井三十三,輿鬼三,柳十五,七星七,張十八,翼十八,軫十七,角十二,亢九,氐十五,房五,心五,尾十八,箕十一,為赤道度。其畢、觜觿、參、輿鬼四宿度數,與古不同,依天以儀測定,用為常數。紘帶天中,儀極攸憑,以格黃道。



 推冬至歲差所在,每距冬至前後各五度為限,初數十二,每限減一,盡九限,數終於四。當二立之際,一度少強,依平。乃距春分前、秋分後,初限起四,每限增一,盡九限,終於十二,而黃道交復。計春分後、秋分前,亦五度為限。初數十二,盡九限,數終於四。當二立之際,一度少強,依平。乃距夏至前後,初限起四,盡九限,終於十二。皆累裁之,以數乘限度,百二十而一,得度;不滿者,十二除,為分。若以十除,則大分,十二為母,命太、半、少及強、弱。命曰黃、赤道差數。二至前、後各九限,以差減赤道度,二分前、後各九限,以差加赤道度,各為黃道度。



 開元十二年,南斗二十三半,牛七半,婺女十一少,虛十,六虛之差十九太。危十七太,營室十七少,東壁九太,奎十七半,婁十二太,胃十四太,昴十一,畢十六少,觜觿一,參九少,東井三十,輿鬼二太,柳十四少,七星六太,張十八太,翼十九少,軫十八太,角十三,亢九半,氐十五太,房五,心四太,尾十七,箕十少,為黃道度,以步日行。日與五星出入,循此。求此宿度,皆有餘分,前後輩之成少、半、太,準為全度。若上考往古,下驗將來,當據歲差,每移一度,各依術算,使得當時度分,然後可以步三辰矣。



 以乾實去中積分,不盡者,盈通法為度。命起赤道虛九,宿次去之,經虛去分,至不滿宿算外,得冬至加時日度。以三元之策累加之,得次氣加時日度。



 以度餘減通法,餘以冬至日躔距度所入限數乘之,為距前分。置距度下黃、赤道差,以通法乘之,減去距前分,餘滿百二十除,為定差。不滿者,以象統乘之,復除,為秒分。乃以定差減赤道宿度,得冬至加時黃道日度。



 又置歲差,以限數乘之,滿百二十除,為秒分。不盡為小分。以加三元之策,因累裁之。命以黃道宿次,各得定氣加時日度。



 置其氣定小餘,副之。以乘其日盈、縮分,滿通法而一,盈加、縮減其副。用減其日加時度餘,得其夜半日度。因累加一策,以其日盈、縮分盈加、縮減度餘,得每日夜半日度。



 ○四曰步月離術



 轉終六百七十萬一千二百七十九。



 轉終日二十七,餘千六百八十五,秒七十九。



 轉法七十六。



 轉秒法八十。



 以秒法乘朔積分,盈轉終去之;餘復以秒法約,為入轉分;滿通法,為日。命日算外,得天正經朔加時所入。因加轉差日一、餘二千九百六十七、秒一,得次朔。以一象之策,循變相加,得弦、望。盈轉終日及餘秒者,去之。各以經朔、弦、望小餘減之,得其日夜半所入。



 各置朔、弦、望所入轉日損益率,並後率而半之,為通率。又二率相減,為率差。前多者,以入餘減通法,餘乘率差,盈通法得一,並率差而半之;前少者,半入餘,乘率差,亦以通法除之:為加時轉率。乃半之,以損益加時所入,餘為轉餘。其轉餘,應益者,減法;應損者,因餘。皆以乘率差,盈通法得一,加於通率,轉率乘之,通法約之,以朓減、朒加轉率,為定率。乃以定率損益朓肉積,為定數。其後無同率者,亦因前率。應益者,以通率為初數,半率差而減之;應損者,即為通率。其損益入餘進退日,分為二日,隨餘初末,如法求之,所得並以損益轉率。此術本出《皇極歷》,以究算術之微變。若非朔、望有交者,直以入餘乘損益率,如通法而一,以損益朓朒,為定數。



 七日、初數二千七百一,末數三百三十九。十四日、初數二千三百六十三,末數六百七十七。二十一日、初數二千二十四,末數千一十六。二十八日,初數千六百八十六,末數千三百五十四。以四象約轉終,均得六日二千七百一分。就全數約為九分日之八。各以減法,餘為末數。乃四象馴變相加,各其所當之日初、末數也。視入轉餘,如初數已下者,加減損益,因循前率;如初數以上,則反其衰,歸於後率雲。



 各置朔、弦、望大小餘,以入氣、入轉朓朒定數,朓減、朒加之,為定朔、弦、望大小餘。定朔日名與後朔同者,月大;不同者,小;無中氣者,為閏月。凡言夜半,皆起晨前子正之中。若注歷,觀弦、望定小餘,不盈晨初餘數者,退一日。其望有交、起虧在晨初已前者,亦如之。又月行九道遲疾,則有三大二小以日行盈、縮累增、損之,則容有四大三小,理數然也。若俯循常儀,當察加時早晚,隨其所近而進退之,使不過三大二小。其正月朔有交、加時正見者,消息前後一兩月,以定大小,令虧在晦、二。定朔、弦、望夜半日度,各隨所直日度及餘分命之。乃列定朔、弦、望小餘,副之。以乘其日盈、縮分,如通法而一,盈加、縮減其副。以加夜半日度,各得加時日度。



 凡合朔所交,冬在陰歷、夏在陽歷,月行青道;冬至、夏至後,青道半交在春分之宿,當黃道東。立冬、立夏後,青道半交在立春之宿,當黃道東南。至所沖之宿,亦如之。冬在陽歷、夏在陰歷,月行白道;冬至、夏至後,白道半交在秋分之宿,當黃道西。立冬、立夏後,白道半交在立秋之宿,當黃道西北。至所沖之宿,亦如之。春在陽歷、秋在陰歷,月行硃道;春分、秋分後,硃道半交在夏至之宿,當黃道南。立春、立秋後,硃道半交在立夏之宿,當黃道西南。至所沖之宿,亦如之。春在陰歷,秋在陽歷,月行黑道。春分、秋分後,黑道半交在冬至之宿,當黃道北,立春、立秋後,黑道半交在立冬之宿,當黃道東北。至所沖之宿,亦如之。四序離為八節,至陰陽之所交,皆與黃道相會,故月有九行。各視月交所入七十二候距交初中黃道日度,每五度為限,亦初數十二,每限減一,數終於四、乃一度強,依平。更從四起,每限增一,終於十二,而至半交,其去黃道六度。又自十二,每限減一,數終於四,亦一度強,依平。更從四起,每限增一,終於十二,復與日軌相會。各累計其數,以乘限度,二百四十而一,得度。不滿者,二十四除,為分,若以二十除之,則大分,以十二為母。為月行與黃道差數。距半交前後各九限,以差數為減;距正交前後各九限,以差數為加。此加減出入六度,單與黃道相較之數。若較之赤道,則隨氣遷變不常。計去冬至、夏至以來候數,乘黃道所差,十八而一,為月行與赤道差數。凡日以赤道內為陰,外為陽;月以黃道內為陰,外為陽。故月行宿度,入春分交後行陰歷、秋分交後行陽歷,皆為同名。若入春分交後行陽歷、秋分交後行陰歷,皆為異名。其在同名,以差數為加者加之,減者減之;若在異名,以差數為加者減之,減者加之。皆以增損黃道度,為九道定度。



 各以中氣去經朔日算,加其入交泛,乃以減交終,得平交入中氣日算。滿三元之策去之,餘得入後節日算。因求次交者,以交終加之,滿三元之策去之,得後平交入氣日算。



 各以氣初先後數先加、後減之,得平交入定氣日算。倍六爻乘之,三其小餘,辰法除而從之,以乘其氣損益率,如定氣辰數而一,所得以損益其氣朓朒積,為定數。



 又置平交所入定氣餘,加其日夜半入轉餘,以乘其日損益率,滿通法而一,以損益其日朓朒積,交率乘之,交數而一,為定數。乃以入氣入轉朓朒定數,朓減、朒加平交入氣餘,滿若不足,進退日算,為正交入定氣日算。其入定氣餘,副之,乘其日盈縮分,滿通法而一,以盈加、縮減其副,以加其日夜半日度,得正交加時黃道日度。以正交加時度餘減通法,餘以正交之宿距度所入限數乘之,為距前分。置距度下月道與黃道差,以通法乘之,減去距前分,餘滿二百四十除,為定差;不滿者一退為秒。以定差及秒加黃道度、餘,仍計去冬至、夏至已來候數乘定差,十八而一,所得依名同異而加減之,滿若不足,進退其度,得正交加時月離九道宿度。



 各置定朔、弦、望加時日度,從九道循次相加。凡合朔加時,月行潛在日下,與太陽同度,是謂離象。先置朔、弦、望加時黃道日度,以正交加時所在黃道宿度減之,餘以加其正交九道宿度,命起正交宿度算外,即朔、弦、望加時所當九道宿度也。其合朔加時,若非正交,則日在黃道,月在九道,各入宿度雖多少不同,考其去極,若應繩準。故云:月行潛在日下,與太陽同度。以一象之度九十一、餘九百五十四、秒二十二半為上弦,兌象。倍之,而與日沖,得望,坎象。參之,得下弦,震象。各以加其所當九道宿度,秒盈象統從餘,餘滿通法從度,得其日加時月度。綜五位成數四十,以約度餘,為分;不盡者,因為小分。



 視經朔夜半入轉,若定朔大餘有進退者,亦加、減轉日。否則因經朔為定。累加一日,得次日,各以夜半入轉餘乘列衰,如通法而一,所得以進加、退減其日轉分,為月轉定分。滿轉法,為度。



 視定朔、弦、望夜半入轉,各半列衰以減轉分。退者,定餘乘衰,以通法除,並衰而半之;進者,半餘乘衰,亦以通法除:皆加所減。乃以定餘乘之,盈通法得一,以減加時月度,為夜半月度。各以每日轉定分累加之,得次日。若以入轉定分,乘其日夜漏,倍百刻除,為晨分。以減轉定分,餘為昏分。望前以昏、望後以晨加夜半度,各得晨、昏月。



 各視每日夜半入陰陽歷交日數,以其下屈伸積,月道與黃道同名者,加之;異名者,減之。各以加、減每日辰昏黃道月度,為入宿定度及分。



 ○五曰步軌漏術



 爻統千五百二十。



 象積四百八十。



 辰八刻百六十分。



 昏、明二刻二百四十分。



 各置其氣消息衰,依定氣所有日,每以陟降率陟減、降加其分,滿百從衰,各得每日消息定衰。其距二分前後各一氣之外,陟、降不等,皆以三日為限。雨水初日,降七十八;初限,日損十二;次限,日損八;次限,日損三;次限,日損二;次限,日損後。清明初日,陟一;初限,日益一;次限,日益二;次限,日益三;次限,日益八;末限,日益十九。處暑初日,降九十九;初限,日損十九;次限,日損八;次限,日損三;次限,日損二;末限,日損一。寒露初日,陟一;初限,日益一;次限,日益二;次限,日益三;次限,日益八;末限,日益十二。各置初日陟降率,依限次損益之,為每日率。乃遞以陟減、降加氣初消息衰,各得每日定衰。



 南方戴日之下,正中無晷。自戴日之北一度,乃初數千三百七十九。自此起差,每度增一,終於二十五度,計增二十六分。又每度增二,終於四十度。又每度增六,終於四十四度,增六十八。又每度增二,終於五十度。又每度增七,終於五十五度。又每度增十九,終於六十度,增百六十。又每度增三十三,終於六十五度。又每度增三十六,終於七十度。又每度增三十九,終於七十二度,增二百六十。又度增四百四十。又度增千六十。又度增千八百六十。又度增二千八百四十。又度增四千。又度增五千三百四十。各為每度差。因累其差,以遞加初數,滿百為分,分十為寸,各為每度晷差。又累其晷差,得戴日之北每度晷數。



 各置其氣去極度,以極去戴日度五十六及分八十二半減之,得戴日之北度數。各以其消息定衰所直度之晷差,滿百為分,分十為寸,得每日晷差。乃遞以息減、消加其氣初晷數,得每日中晷常數。



 以其日處在氣定小餘,爻統減之,餘為中後分。不足減,反相減,為中前分。以其晷差乘之,如通法而一,為變差。以加、減中晷常數,冬至後,中前以差減,中後以差加;夏至後,中前以差加,中後以差減。冬至一日,有減無加;夏至一日,有加無減。得每日中晷定數。



 又置消息定衰,滿象積為刻,不滿為分。各遞以息減、消加其氣初夜半漏,得每日夜半漏定數。其全刻,以九千一百二十乘之,十九乘刻分從之,如三百而一,為晨初餘數。



 各倍夜半漏,為夜刻。以減百刻,餘為晝刻。減晝五刻以加夜,即晝為見刻,夜為沒刻。半沒刻加半辰,起子初算外,得日出辰刻。以見刻加而命之,得日入。置夜刻,五而一,得每更差刻。又五除之,得每籌差刻。以昏刻加日入辰刻,得甲夜初刻。又以更籌差加之,得五夜更籌所當辰。其夜半定漏,亦名晨初夜刻。



 又置消息定衰,滿百為度,不滿為分。各遞以息減、消加氣初去極度,各得每日去極定數。



 又置消息定衰,以萬二千三百八十六乘之,如萬六千二百七十七而一,為度差。差滿百為度。各遞以息加、消減其氣初距中度,得每日距中度定數。倍之,以減周天,為距子度。



 置其日赤道日度,加距中度,得昏中星。倍距子度,以加昏中星,得曉中星。命昏中星為甲夜中星,加每更差度,得五夜中星。



 凡九服所在,每氣初日中晷常數不齊。使每氣去極度數相減,各為其氣消息定數。因測其地二至日晷,測一至可矣,不必兼要冬夏。於其戴日之北每度晷數中,較取長短同者,以為其地戴日北度數及分。每氣各以消息定數加減之,因冬至後者,每氣以減。因夏至後者,每氣以加。得每氣戴日北度數。各因所直度分之晷數,為其地每定氣初日中晷常數。其測晷有在表南者,亦據其晷尺寸長短與戴日北每度晷數同者,因取其所直之度,去戴日北度數。反之,為去戴日南度。然後以消息定數加減之。



 二至各於其地下水漏以定當處晝夜刻數。乃相減,為冬、夏至差刻。半之,以加、減二至晝夜刻數,為定春、秋分初日晝夜刻數。乃置每氣消息定數。以當處差刻數乘之,如二至去極差度四十七分,八十而一,所得依分前、後加、減初日晝夜漏刻,各得餘定氣初日晝夜漏刻。



 置每日消息定衰,亦以差刻乘之,差度而一,所得以息減、消加其氣初漏刻,得次日。其求距中度及昏明中星日出入,皆依陽城法求之。仍以差刻乘之,差度而一,為今有之數。若置其地春、秋定日中晷常數與陽城每日晷數,較其同者,因其日夜半漏亦為其地定春、秋分初日夜半漏。求餘定氣初日,亦以消息定數依分前、後加、減刻分,春分後以減,秋分後以加。滿象積為刻。求次日,亦以消息定衰,依陽城術求之。此術究理,大體合通。然高山平川,視日不等。較其日晷,長短乃同。考其水漏,多少殊別。以茲參課,前術為審。



\end{pinyinscope}