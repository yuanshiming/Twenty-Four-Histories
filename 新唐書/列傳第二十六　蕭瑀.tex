\article{列傳第二十六 蕭瑀}

\begin{pinyinscope}

 蕭瑀,字時文,後梁明帝子也。九歲,封新安王。國除,以女兄為隋晉王妃熹等20餘人列入《道學傳》。後亦作理學的同義語。,故入長安。瑀愛經術,善屬文。性鯁急,鄙遠浮華。嘗以劉孝標《辯命論》詭悖不經,乃著論非之,以為:「人稟天地而生而謂之命,至吉兇禍福則系諸人。今一於命,非先王所以教人者。」通儒柳顧言、諸葛潁嘆曰:「是足針孝標膏肓矣!」



 晉王為太子,授右千牛。即帝位,妃為後,而瑀浸親寵,頻遷尚衣奉御、檢校左翊衛鷹揚郎將。感末疾,不呼醫,曰:「天若假吾餘年,因得為遁階矣!」後聞,責謂曰:「爾亡國後不安小官,而高為怪語,罪不測。」瑀復治疾,良已。拜內史侍郎,數言事忤旨,稍見忌。



 帝至雁門,為突厥所圍,瑀謀曰:「夷俗,可賀敦與兵馬事,況義成公主以帝女為之。若走一介使鐫喻,宜不戰而解。又眾商陛下已平突厥,方復事遼東,故怠不肯戰。願下詔赦高麗,專討突厥,則人自奮矣。」帝從之。既而主詭辭謂突厥,果解圍去。然帝素意伐遼,又銜瑀以謀擫其機,謂群臣曰:「突厥何能為,瑀乘未解時乃紿恐我!」遂出瑀為河池郡守。部有鈔賊萬人,吏不制,瑀募勇敢士擊降之,悉捐貲畜賜有功。又擊走薛舉眾數萬。



 高祖入京師,招之,挈郡自歸,授光祿大夫,封宋國公,拜民部尚書。秦王領右元帥,攻洛陽,署瑀府司馬。武德元年遷內史令,帝委以樞管,內外百務悉關決。或引升御榻,呼曰蕭郎。瑀自力孜孜,抑過繩違無所憚。上便宜,每見納用。手詔曰:「得公言,社稷所賴,朕既寶之,故賜黃金一函,公其勿辭。」



 是歲,州置七職,秦王為雍州牧,以瑀為州都督。詔嘗下中書,未即行,帝讓其稽,瑀曰:「隋季內史詔敕多違舛,百司不知所承。今朝廷初基,所以安危者系號令。比承一詔,必覆審,使先後不謬,始得下,此所以稽留也。」帝曰:「若爾,朕何憂乎?」初,瑀關內田宅悉賜勛家,至是,還給之。瑀盡以分宗族,獨留廟室奉祠。王世充平,進尚書右僕射。七年,以熒惑犯右執法,避位,不許。久之,遷左僕射。



 貞觀初,房玄齡、杜如晦新得君,事任稍分,瑀不能無少望,乘罅切詆,辭旨疏躁。太宗怒,廢於家。俄拜特進、太子少師,復為左僕射,實封六百戶。帝問瑀:「朕欲長保社稷,奈何?」瑀曰:「三代有天下所以能長久者,類封建諸侯以為籓屏。秦置守令,二世而絕。漢分王子弟,享國四百年。魏、晉廢之,亡不旋跬。此封建之有明效也。」帝納之,始議封建。坐與陳叔達忿爭御前不恭,免。歲餘,起為晉州都督。入拜太常卿,遷御史大夫,參預朝政。瑀諭議明辯,然不能容人短,意或偏駁不通,而向法深,房玄齡、魏徵、溫彥博頗裁正之,其言多黜,瑀亦不平。會玄齡等小過失,瑀即痛劾,不報,由是自失,罷為太子少傅,加特進,復為太常卿。拜河南道巡省大使。九年,復參預政事。



 帝嘗曰:「武德季,太上皇有廢立議,顧朕挾不賞之功,於昆弟弗見容,瑀於爾時不可以利怵死懼,社稷臣也。」因賜詩曰:「疾風知勁草,版蕩識誠臣。」又曰:「公守道耿介,古無以過,然善惡太明,或有時而失。」瑀頓首謝曰:「既蒙教,又許以忠亮,雖死日,猶生年也。」魏徵曰:「臣有逆眾持法,主恕之以公;孤特守節,主恕之以介。昔聞其言,乃今見之。使瑀不遇陛下,庸能自保邪?」晉王為皇太子,拜太子太保、同中書門下三品。帝曰:「三師,以德導太子者也,禮不尊,則無所取法。」乃詔:「師入謁,太子出門迎拜,師答拜;每門,讓乃入;師坐,然後坐;書前後著名,稱惶恐。」瑀素貴,但中狹。每燕見,輒言:「玄齡輩朋黨盜權,若膠固然,特未反耳。帝曰:「知臣莫若君。朕雖不明,寧頓懵臧否?」因為瑀曉解,瑀以帝有所偏信,帝積久亦不平。瑀好浮屠法,間請舍家為桑門,帝許之矣,復奏自度不能為,又足疾不入謁,帝曰:「瑀豈不得其所邪?」乃詔奪爵,下除商州刺史。未幾,復其封,加特進。卒,年七十四。遺命斂以單衣,無卜日。詔贈司空、荊州都督,陪葬昭陵。太常謚曰肅,帝以其性忌,改謚貞褊。



 子銳,尚襄城公主,為太常少卿。



 鈞,瑀從子,有才譽。永徽中,累遷諫議大夫、弘文館學士。左武候屬盧文操跳堞盜庫財,高宗以其職主乾,當自盜罪死。鈞曰:「囚罪誠死,然恐天下聞,謂陛下重貸輕法,任喜怒殺人。」帝曰:「真諫議也。」詔原死。太常工為宮人通訊遺,詔殺之,且附律。鈞言:「禁當有漸,雖附律,工不應死。」帝曰:「如姬竊符,朕以為戒,今不濫工死,然喜得忠言。」即宥工,徙遠裔。終太子率更令。



 子瓘,為渝州長史,居母喪,以毀卒。



 鈞兄子嗣業,少從煬帝後入突厥,貞觀九年歸,以其知虜曲折,詔領突厥眾。擢累鴻臚卿,兼單于都護府長史。調露中,突厥叛,嗣業與戰,敗績。高宗責曰:「我不殺薛仁貴、郭待封,故使爾至此。然爾門與我家有雅舊,故貸死。」乃流桂州。



 嵩,瓘子,貌偉秀,美須髯。始,娶會稽賀晦女,僚婿陸象先,宰相子,時為洛陽尉,已有名,士爭往交,而嵩汩汩未仕,人不之異。夏榮者善相,謂象先曰:「君後十年,貴冠人臣,然不若蕭郎位高年艾,舉門蕃熾。」時人不許。



 神龍元年,始調洺州參軍事。桓彥範為刺史,待以異禮。河北黜陟使姜師度表為判官。開元初,擢中書舍人。時崔琳、正丘、齊澣皆有名,以嵩少術學,不以輩行許也,獨姚崇稱其遠到。歷宋州刺史,遷尚書左丞。



 十四年,以兵部尚書領朔方節度使。既赴軍,有詔供帳餞定鼎門外,玄宗賦詩勞行。會吐蕃大將悉諸邏恭祿及燭龍莽布支陷瓜州,執刺史田元獻;回紇又殺涼州守將王君■,河、隴大震。帝擇堪任邊者,徙嵩河西節度使,判涼州事,封蘭陵縣子。嵩表裴寬、郭虛己、牛仙客置幕府,以建康軍使張守珪為瓜州刺史,完樹陴塢,懷保邊人。於時悉諸邏恭祿威詹諸部,吐蕃倚其健噬邊,嵩乃縱反間,示疑端,贊普果誅之。使悉末明攻瓜州,守珪拒甚力,虜引卻。會鄯州都督張志亮破賊青海西,嵩又遣副將杜賓客率強弩四千與吐蕃戰祁連城下,自晨斗迄晡,乃大潰,斬一將,虜哭震山谷。露布至,帝大悅,授嵩同中書門下三品,又官一子,恩顧第一。



 十七年,進兼中書令。自張說罷宰相,令缺四年,嵩得之,然常遙領河西節度。在公慎密,人莫見其際。子衡,尚新昌公主。嵩妻入謁,帝呼為親家,儀物貴甚。俄封徐國公。



 初,裴光庭與嵩數不協,光庭卒,帝委嵩擇相,嵩推韓休。及休同位,峭正不相假,至校曲直帝前。嵩慚,乞骸骨。帝慰之曰:「朕未厭卿,何庸去乎?」嵩伏曰:「臣待罪宰相,爵位既極,幸陛下未厭,得以乞身。有如厭臣,首領且不保,又安得自遂?」因流涕。帝為改容曰:「卿言切矣,朕未能決。弟歸,夕當有詔。」俄遣高力士詔嵩曰:「朕將爾留,而君臣誼當有始有卒者。」乃授尚書右丞相,與休皆罷。是日,荊州進黃甘,帝以紫帉包賜之。擢子華給事中。



 久之,進太子太師。而幽州節度使張守珪坐賂中人牛仙童得罪,李林甫素忌嵩,因言嵩嘗以城南墅遺仙童,貶青州刺史。尋復拜太子太師。固請老,見許。嵩退,脩蒔園區,優游自怡。家饒財,而華為工部侍郎,衡以尚主位三品,就養,年逾八十,士艷其榮。天寶八載卒,贈開府儀同三司。



 華,謹重方雅,有家法,嗣爵。天寶末,為兵部侍郎。祿山亂,陷賊,逼守魏州。郭子儀攻安慶緒於相州,華間道奉表,欲舉魏以應,為賊所執。會崔光遠得魏州,破械出之。魏人德華庇免,爭來詣光遠乞留,有詔即授刺史。思明反,子儀懼復失華,乃表崔光遠代之,而召置軍中。相州兵潰,華還朝,猶以污賊降試秘書少監。稍遷尚書右丞,擢河中晉、絳節度使。上元初,以中書侍郎同中書門下平章事。李輔國用事,求宰相,華拒之,輔國怨。會肅宗大漸,矯詔罷華為禮部尚書,引元載以代。方代宗諒暗,載助輔國,貶華為峽州司馬,卒。二子:恆、悟。



 復,字履初,衡子。生戚里,姻從豪汰,以服御輿馬相誇,復常衣垢弊,居一室,學自力,非名士夙儒不與游,以清操顯。華每嘆曰:「此子當興吾宗!」推主廕為宮門郎。廣德中,歲大饑,家百口,不自振,議鬻昭應墅。宰相王縉欲得之,使弟紘說曰:「以君才宜在左右,胡不以墅奉丞相取右職?」復曰:「鬻先人墅以濟孀單,吾何用美官,使門內餒且寒乎?」縉憾之,由是廢。數歲,乃歷歙、池二州刺史,治狀應條。遷湖南觀察使。改同州刺史,歲歉,州有京畿觀察使儲粟,復輒發以貸人,有司劾治,詔削階,停刺史。或吊之,復曰:「茍利於人,胡責之辭!」久乃拜兵部侍郎。



 普王為襄漢元帥,進復戶部尚書、統軍長史。舊制謂「行軍長史」,德宗以復父諱更之。未行,扈狩奉天。帝惡庳隘,欲西如鳳翔依張鎰。復曰:「鳳翔乃泚舊兵,今泚悖亂,當有同惡者。雖鎰,臣畏不免。」帝曰:「朕業行,留一日以驗爾言!」俄而鎰為李楚琳所害,於是拜吏部尚書、同中書門下平章事。



 復嘗言:「艱難以來,始用宦者監軍,權望太重,是曹正可委宮掖事,兵要政機,叵使參領。」帝不聽。又言:「陛下厥初清明,自楊炎、盧杞妨命穢盛德,播越及茲。今阽於危,當懲乂前敗。」因述君臣大端,即自言:「若使臣依阿偷免,不敢當宰相。」杞對上或諂諛阿匼,復厲言:「杞詞不正!」帝色眙,謂左右曰:「復慢我。」因詔復充山南、江淮、湖南、嶺南等道宣撫、安慰使。



 興元初,進門下侍郎。初,淮南陳少游左附李希烈,而張鎰判官韋皋殺邠、隴叛卒,不應楚琳。復還執政,建言:「陛下反正,功臣已貴矣,唯甄善汰惡為未明。少游位將相,首臣賊,皋名淺官下,獨挺挺抗忠。如以皋代少游,則天下嘹然知逆順之理。」帝許之。復出,中官馬欽緒揖宰相劉從一,附耳語,既而從一密諗復曰:「有詔與公議向所奏,不欲李勉、盧翰聞知。」復曰:「堯、舜有『僉曰』之言,朝廷大事尚當謀及公卿。如勉等非其人,當罷去。既曰宰相,而謀議可獨避之乎?今與公行此或可,第恐浸以生常,政由是敝。」從一以聞,帝不悅。復辭疾上政事,許之。



 弟升,尚郜國大長公主,肅宗女也。升早卒,主以奸蠱事再得罪廢,諸子悉逐丑地,女為皇太子妃,太子請離婚,帝銜曩忮,故復坐是檢校太子左庶子,廢居饒州。貞元四年卒,年五十七。



 復望閥高華,厲名節,不通狎流俗。及為相,臨事嚴方,數咈帝意,故居位亟解。然性孝友,既貶晏然,口未嘗言所累。



 復子湛。湛子置,咸通中位宰相,無顯功,史逸其傳。



 俯,字思謙,恆子。貞元中,及進士第,又以賢良方正對策異等,拜右拾遺。元和六年,召為翰林學士,凡三年,進知制誥。會張仲方以李吉甫數調發疲天下,訾其謚,憲宗怒,逐仲方,而俯坐與善,奪學士,下除太僕少卿。皇甫鎛薦為御史中丞。鎛與令狐楚皆善俯,兩人同輔政,數稱其善,故帝待俯厚。襲徐國公。穆宗立,逐鎛,議所以代者,楚薦之,授中書侍郎、同中書門下平章事,進門下侍郎。



 吐蕃寇涇州,調兵護邊,帝因問:「兵法有必勝乎?」俯曰:「兵兇器,聖人不得已用之,故武不可玩,玩則無震。夫以仁討不仁,以義討不義,先招懷,後掩襲,故有不殺厲,不禽二毛,不犯田稼,其救人如免水火,此必勝術也。若乃以小不忍輕任干戈,師曲而敵怨,非徒不勝,又將自危,是以聖王慎於兵。」帝重其言。嘗詔俯撰王承宗先銘,俯奏:「承宗比不臣,迷而後復,臣不忍稱道其先。又辭成當有餉謝,拒之,則非朝廷撫納意;受之,臣誼不當取。」帝善而止。



 令狐楚罷執政,西川節度使王播賂權幸求宰相,俯劾播纖佞不可污臺宰,帝不許。自請罷,冀有感寤,帝亦不省。俄罷為尚書左僕射,用播為鹽鐵使,後卒相,俯自謂輔政淺,固辭僕射,換吏部尚書。又避選事,徙兵部,移病求分司,不許。授太子少保,為同州刺史。復以少保分司東都。



 性簡潔,以聲利為污,疾邪太甚,孤特一概,故輕去位無所藉。文宗即位,召授少師,稱疾力不拜,乃還左僕射,許致仕。莊恪太子時,議選舊德,保輔東宮,復以少師召,輒上還制書,堅辭。即遷太子太傅,優詔褒尚。開成初,弟俶為楚州刺史,召見。帝曰:「俯先帝賢宰相,筋力未衰,可一來,爾善道朕意。」乃以詔書並絹三百因俶致之。俯終不起,以壽卒。



 母韋,賢明,治家嚴,俯雖宰相,侍左右如褐衣時。居喪哀毀。既老,家於洛,歲時賓客請謝,以為煩,乃舍濟源墅,自放山野,優游窮年。然其居位頗介謹持法,重名器,狹於用人,每除吏,常憂不稱,鮮有簡拔。



 穆宗初,兩河底定,俯與段文昌當國,謂四方無虞,遂議大平事,以為武不可黷,勸帝偃革尚文,乃密詔天下鎮兵,十之,歲限一為逃、死,不補,謂之銷兵。既而籍卒逋亡,無生業,曹聚山林間為盜賊。會硃克融、王廷湊亂燕、趙,一日悉收用之。朝廷調兵不克,乃召募市人烏合,戰輒北,遂復失河朔矣。



 贊曰:俯議銷兵,寧不野哉!當此時,河朔雖挈地還天子,而悍卒頑夫開口仰食者故在,彼皆不能自返於本業者也。又硃克融等客長安,餓且死,不得一官,而俯未有以措置,便欲去兵,使群臣失職,一日叫呼,其從如市,幽、魏相挺,復為賊淵,可謂見豪末而不察輿薪矣。宰相非其人,禍可既乎!



 仿,字思道,悟子。大和中,擢進士第。除累給事中。宣宗力治,喜直言,嘗以李璲為嶺南節度使,使者已賜節,而仿封還詔書。帝方作樂,不暇命使,遣優工趨出追之,未及璲所而還。後以封敕脫誤,法當罰,侍講學士孔溫裕曰:「給事中駁奏,為朝廷論得失,與有司奏事不類,不應罰。」詔可。



 令狐綯用李琢經略安南,琢以暴沓免,俄起為壽州團練使,仿劾奏琢無所回,時推其直。自集賢學士拜嶺南節度使。南方珍賄叢夥,不以入門。家人病,取槁梅於廚以和劑,仿知,趣市還之。



 咸通初,為左散騎常侍。懿宗怠政事,喜佛道,引桑門入禁中為禱祠事,數幸佛廬,廣施予。仿諫,以為:「天竺法割愛取滅,非帝王所尚慕。今筆梵言,口佛音,不若懲謬賞濫罰,振殃祈福。況佛者可以悟取,不可以相求。」帝雖昏縱,猶嘉嘆其言。後官數遷,拜義成軍節度使。滑州瀕河,累歲水壞西北防,仿徙其流遠去,樹堤自固,人得以安。以兵部尚書再判度支,進中書侍郎、同中書門下平章事。再遷司空、蘭陵縣侯。時天下盜起,宦人持兵柄,仿以鯁正為權近所忌。卒,年八十。



 子廩,字富侯。第進士,遷尚書郎。仿領南海,解官往侍。為人退約少合。南海多穀紙,仿敕諸子繕補殘書。廩諫曰:「州距京師且萬里,書成不可露齎,必貯以囊笥,貪者伺望,得無薏苡嫌乎?」仿曰:「善,吾思不及此。」乃止。廣明初,以諫議大夫知制誥,請厲止夜行以備賊諜,出太倉粟賤估以濟貧民。俄遷京兆尹。田令孜養子有罪亡,擊捕吏,系獄,請救踵門,廩不納,杖殺之,內外畏讋。令孜拒黃巢,以廩為糧料使,辭疾,貶賀州司戶參軍事。會襄王竊據,挈族逃河朔,鎮冀節度使王鎔厚禮之。光化中,以給事中召,不至,卒。



 遘,字得聖,置子。咸通中,擢進士第,闢節度府。入朝,拜右拾遺。與韋保衡聯第,而遘姿宇秀偉,氣孤峻,嘗慕李德裕為人。保衡才下,諸儒靳薄之,不甚齒,獨呼遘太尉,保衡憾焉。於是保衡已為相,摭遘罪,繇起居舍人斥播州司馬。道三峽,方迫畏不瞑,若有人謂曰:「公無恐,予為公呵御。」遘悅悟。俄謁白帝祠,見帝貌類向所睹,異之。未幾,保衡死,召為禮部員外郎。乾符中,累擢戶部侍郎、翰林學士承旨。



 僖宗入蜀,以兵部判度支,次綿州,拜同中書門下平章事。始,王鐸主貢舉而得遘,及是,與鐸並位。鐸年老,嘗入對踣殿中,遘掖起之。帝喜曰:「遘善事長,大臣和,予之幸也!」遘曰:「不止以長,乃鐸門生。」帝笑曰:「鐸選士,朕選宰相,卿無負我!」遘頓首謝。從還京師,累拜司空,封楚國公。



 遘負大節,以王佐自任。既當國,風採峭整,天子器之。時籓鎮多興於盜賊,橫放莫能制,權綱漼弛。支詳在徐州,引散騎常侍李損子凝吉為佐,會牙將時溥逐詳而取節度,溥為饔干所毒,不死,或讒凝吉為詳報仇者,溥怒殺之。損時在朝,溥即上言損連謀,請並誅。田令孜受溥金,劾損,付御史獄,中丞盧渥傅成其罪。御史王華嫉惡甚,表損不知狀。令孜請移神策獄,華不奉詔,奏言:「損近臣,法當死即死,獨不宜取辱於宦人手。」遘即時叩延英爭曰:「凝吉以冤就屠,已不可言。損與子音問不接且數期,安得謂同謀哉?溥恃功壞天子法,請案近臣,卑侮王室,有無將之萌。今損可無罪誅,禍且及臣輩。」帝寤,止免官。當此時,令孜持禁軍,權寵可炙,公卿無不附順,唯遘未嘗少下。



 後令孜取安邑池鹽給衛軍,王重榮固爭,乃徙重榮它鎮,不受詔。令孜以兵討之,重榮引沙陀拒王師。王師敗,逐而西,帝驚,幸鳳翔。諸節度共劾令孜生事,離間大臣。遘素惡之,與裴澈計,共召硃玫於邠。玫起邠兵五千奉迎,與沙陀等連和。令孜迫帝幸陳倉,夜出,百官不及從。玫怒令孜,並望帝不諒其心,謂遘曰:「上奔播六年,中原之人,與賊肝髓流野,得復宗廟,遺老殘民聞輿馬音,流涕相歡。上曾不念,以諸侯勤王功為敕使之寵。今奸臣為國產怨,我奉命而來,返以為脅君。群臣報國極矣,戰力殫矣,尚能垂頭塌翅求生於黃門哉!喪君有君,公其圖之。」遘曰:「上無負天下,顧為令孜掣制,每言必涕數行下。陳倉之行,又劫於兵。公誠有憂王室意,宜還籓奉表,請天子復國,策無宜此。」玫曰:「諸王才可任天下者不乏。」遘曰:「人非伊、霍,欲為禍首,未或利也。」玫退曰:「我擇一王為帝,違者斬,尚何事?」乃立嗣襄王煴,而召遘作冊,遘苦辭,玫更委鄭昌圖,滋恨遘。及還長安,使昌圖相煴,罷遘為太子太保。移疾不出。方其弟蘧為永樂令,往從之。帝還宮,宰相孔緯與遘雅隙,乃劾嘗為偽臣,即賜死其所,實光啟三年。



 遘見柄任凡五期,行完而材,逢世多故,召愎臣以濟亂,身污偽署,不得其死,人為哀之。



 定,字梅臣,瑀曾孫。以廕起家陜州參軍事、金城丞。蒞事清挺。選補黜陟使裴遵慶表為判官,還調萬年主簿。歷左右司郎中。為元載所惡,外遷袁、潤等六州刺史。大歷中,有司差天下刺史治最,定與常州蕭復、豪州張鎰為第一,而劭桑稼,均賦稅,業徠游口,在鎰、復右。遷戶部侍郎、太常卿。硃泚泚反,詭姓名為張誕,匿里中,與蔣沇不浼於賊。事平,擢太子少師。卒,年七十七,贈太子太師。



 贊曰:梁蕭氏興江左,實有功在民,厥終無大惡,以浸微而亡,故餘祉及其後裔。自瑀逮遘,凡八葉宰相,名德相望,與唐盛衰。世家之盛,古未有也。



\end{pinyinscope}