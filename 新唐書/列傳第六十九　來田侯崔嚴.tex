\article{列傳第六十九 來田侯崔嚴}

\begin{pinyinscope}

 來瑱,邠州永壽人。父曜,奮行間,開元末,持節磧西副大使、四鎮節度使新的範疇,從一個範疇到另一個範疇的發展,形成人類社會,著名西邊,終右領軍大將軍。瑱略知收,尚名節,崖然有大志。天寶初,從四鎮任劇職,累遷殿中侍御史、伊西北庭行軍司馬。詔舉智謀果決、才堪統眾者,拾遺張鎬薦瑱能斷大事,有禦侮才,擢潁川太守,充招討使。會母喪免,以孝聞。



 安祿山反,張垍薦之,興塊次,拜汝南太守。未行,改潁川。賊攻潁川,方積粟多,瑱完埤自如,手射賊,皆應弦僕。賊使降將畢思琛招之,父故將也,拜城下,泣且吊,瑱不應,前後俘殺甚眾。賊懼,目為「來嚼鐵」。以功就加防禦使、河南淮南游弈逐要招討使。徙山南東道節度使代魯炅,會嗣號王巨表炅方固守,乃還瑱故官。賊圍南陽急,瑱與魏仲犀合兵救之,不勝,人情恟懼,瑱能撫訓士,舉動安重,賊不得侵。改淮南西道節度。兩京平,封潁國公,食二百戶。



 乾元二年,徙河西。未行,王師敗於相州,詔拜陜虢節度,兼潼關防禦團練鎮守使。明年,襄州部將張維瑾等殺其使史翽,徙瑱山南東道襄、鄧、均、房、金、商、隨、郢、復十州節度使。既至,維瑾降。上元二年春,破史思明餘黨於魯山,俘賊渠,又戰汝州,獲馬、牛、橐駝,凡兩戰,斬首萬級。明年,詔瑱還,瑱安襄、漢,士亦宜其政,因諷眾留己,而外示行;至鄧,復詔歸鎮。肅宗聞其謀,惡之,呂諲、王仲昇等皆言「瑱得士心,不可以留』,乃改山南東道襄、鄧、唐、復、隨、郢六州節度。俄而仲昇與賊戰申州,為賊禽。初,仲昇被圍,而江陵呂諲病,瑱顧望不即救,及師出,仲昇已沒。行軍司馬裴奰表其狀,且言:「瑱善謀而勇,恐後難制,即除之,可一戰禽也。」帝頗謂然,遂改瑱淮西申、安、蘄、黃、光、沔兼河南陳、豫、許、鄭、汴、曹、宋、穎、泗十五州節度以寵之,陰奪其權,加奰襄、鄧等七州防禦使代瑱。瑱懼,釋言「淮西無糧,須麥收可上道」,又諷眾固留。



 代宗立,復授襄州節度、奉義軍渭北兵馬使;密詔奰圖之。奰自均州率眾浮漢下。會日入,候者白瑱,瑱與帳下謀,其副薛南陽曰:「公奉詔留鎮,而奰以兵脅代,是無名也。奰智勇非公敵,而眾心不附。彼若乘我不虞,縱火夜攻,誠可憂也。若須明,則破之必矣。」明日,奰督軍五千陣谷水北,瑱以兵迎之,呼其軍,告曰:「爾何事來?」曰:「公不受命,故中丞伐罪。」瑱曰:「詔還鎮此州。」乃以詔書示之。皆曰:「偽也。吾千里討賊,豈空歸邪?」爭射之,瑱走旗下。薛南陽曰:「請公勒兵勿戰。」乃以三百騎為奇兵,旁萬山,出其背夾擊之,其眾幾盡,奰脫身走,至申口,禽之,送京師。瑱因入朝謝罪,帝待之無疑,拜兵部尚書、同中書門下平章事,充山陵使。是時,程元振居中用事,疾瑱,乃告與巫祝言不順。會王仲昇歸,又言由瑱與賊合,故陷賊。帝積怒,遂下詔削除官爵,貶播川尉,員外置。及鄠,賜死,籍其家。瑱之死,門下客散去,掩尸於坎,校書郎殷亮獨後至,哭尸側,為備棺衾以葬。帝徐悟元振誣,以它罪流溱州。



 先是,瑱行軍司馬龐充以兵二千戍河南,至汝,聞瑱死,乃還襲襄州,別將李昭御之,走房陵。昭與薛南陽、梁崇義不相臣,崇義殺昭,帝以崇義為節度使代瑱。既而為瑱立祠,四時致饗,避瑱廳事不處,哀祈禮葬,詔可。廣德元年,追復官爵。



 裴奰者,始以廕為京兆司錄參軍。瑱鎮陜州,引為判官,移襄州,又為行軍司馬,遇之厚。及瑱私漢上,奰欲得其處,故背瑱言狀,帝倚以圖瑱。而性輕褊少謀,師興,給用無節。及敗,有詔流費州,至藍田,賜死。



 田神功,冀州南宮人。天寶末,為縣史。會天下兵興,賊署為平盧兵馬使,率眾歸朝,從李忠臣收滄、德,攻相州,拒杏園。後守陳留,戰不勝,與許叔冀降於史思明。思明使與南德信、劉從諫南略江淮,神功襲德信,斬之,從諫脫身走,乃並將其兵。詔拜鴻臚卿。襲敬釭鄆州,不克。劉展反,鄧景山引神功助討,自淄青濟淮,眾不整,入揚州,遂大掠居人貲產,發屋剔窖,殺商胡波斯數千人。俄而禽展送京師,遷淄青節度使。會侯希逸入青州,更徙兗鄆。時賊圍宋州急,李光弼奏神功往救,賊解去。又破法子營,復攻敬釭,降之。朝義聞,乃奔下博。進封信都郡王,徙河南節度、汴宋八州觀察使。



 大歷二年來朝,加檢校尚書右僕射,詔宰相百官送至省。又判左僕射,知省事,加太子太師,還軍。神功事母孝。始,嘗倨驕自如,見光弼待官屬鈞禮,乃折節謙損。既寢疾,宋之將吏為禳祈報恩。



 八年,自力入朝,卒,代宗為徹樂,贈司徒,詔其弟曹州刺史神玉知汴州留事,賻絹千匹、布五百端,百官吊喪,賜屏風茵褥,飯千桑門追福。至德後,節度使不兼宰相者,惟神功恩禮最篤。神玉終汴宋節度留後。



 侯希逸,營州人。長七尺,豐下銳上。天寶末為州裨將,守保定城。安祿山反,使中人韓朝易又傳命,希逸斬以徇。祿山又以親將徐歸道為節度使,希逸率兵與安東都護王玄志斬之,遣使上聞,詔拜玄志平盧節度使。玄志卒,副將李正己殺其子,共推希逸,有詔就拜節度使,兼御史大夫。與賊確,數有功。然孤軍無援,又為奚侵掠,乃拔其軍二萬,浮海入青州據之,平盧遂陷。肅宗因以希逸為平盧、淄青節度使。自是淄青常以平盧冠使。寶應初,與諸軍討平史朝義,加檢校工部尚書,賜實戶,圖形凌煙閣。



 希逸始得青,治軍務農有狀。後稍怠肆,好畋獵,佞佛,興廣祠廬,人苦之。夜與巫家野次,李正己因眾怨閉闔不內,遂奔滑州。召還,檢校尚書右僕射,知省事。大歷末,封淮陽郡王。建中二年,遷司空。未及拜,卒,年六十二,遺敕其子上還前後實封,贈太保。



 崔寧,本貝州安平人,後徙衛州。世儒家,而獨喜縱橫事,因落魄,客劍南,以步卒事鮮於仲通。又從李宓討雲南,無功,還成都,行軍司馬崔論悅之,薦為牙將。歷事崔圓、裴冕。冕被謗,朝廷疑之,遣使者問狀,寧部兵耳白其冤,使者以聞。寧亦還京師,留為折沖郎將。寶應初,蜀亂,山賊乘險,道不通。嚴武白寧為利州刺史,既至,賊遁去,由是知名。及武為劍南節度使,過州,心欲與俱西,而利非所屬,使寧自為計。寧曰:「節度使張獻誠見疑,難輒去。然獻誠嗜利,若厚賂之,寧可以從大夫矣。」武然之,以奇錦珍貝遺獻誠,且求寧,獻誠果喜,令自移疾去。武遂奏為漢州刺史。吐蕃引雜羌寇西山,破柘、靜等州,有詔收復。於是武遣寧將而西,既薄賊城,城皆累石,不得攻,惟東南不合者丈許,諜知之,乃為地道,再宿而拔,拓地數百里。虜眾驚相謂曰:「寧,神兵也!」及還,武大悅,裝七寶輿迎入成都,以誇於軍。



 永泰元年,武卒。行軍司馬杜濟,別將郭英干、郭嘉琳皆請英乾之兄英乂為節度使,寧與其軍亦丐大將王崇俊。奏俱至,而朝廷既用英乂矣。英乂恨之,始署事即誣殺崇俊,又遣使召寧。寧恐,托拒吐蕃,不敢還。英乂怒,因出兵,聲言助寧,實欲襲取之,即徙寧家於成都,而淫其妾媵。寧懼,益負阻。英乂乃自將討之,會天大雪,馬多凍死,士心離,遂敗歸。寧聞英乂損裁將卒稟賜,下皆恨怒,又毀玄宗冶金像,乃令軍中曰:「英乂反,輒居先帝舊宮。」乃進薄成都。英乂陣城西,使柏茂琳為前軍,英乾為左軍,嘉琳為後軍,與寧戰,茂琳等敗,軍多降寧。寧即署降將,使率兵還攻,英乂不勝,走靈池,為韓澄所殺。



 於是劍南大擾,楊子琳起瀘州,與邛州柏貞節連和討寧。明年,代宗詔宰相杜鴻漸為山西劍南邛南等道副元帥、劍南西川節度使,往平其亂。鴻漸出駱谷,或進計曰;「公不如駐閬中,數騰書陳英乂罪,嘉寧方略,因以寧所署刺史即授之,使不疑。而後與東川張獻誠及諸帥合兵擾寧,不一年,寧勢且窮,必束身歸命。」鴻漸疑未決。會寧遣使至,獻繒錦數萬,辭卑約甚,鴻漸貪其利,遂入成都,政事一委寧,日與僚屬杜亞、楊炎縱酒高會。乃表貞節為邛州刺史,子琳為瀘州刺史,以和解之。又數薦寧於朝。先是,寧與張獻誠戰,奪其旌節,不肯與,故朝廷因授寧成都尹、西山防禦使、西川節度行軍司馬。鴻漸既還朝,遂為節度使。



 大歷三年來朝。寧本名旰,至是賜名。楊子琳襲取成都,帝乃還寧於蜀。未幾,子琳敗。寧見蜀地險,饒於財,而朝廷不甚有紀,乃痛誅斂;使弟寬居京師,以賂厚謝權貴,深結元載父子,故寬驟擢御史中丞,寬兄審至給事中。寧在蜀久,兵浸強,而肆侈窮欲,將吏妻妾多為污逼,朝廷隱忍,不能詰。累加尚書左僕射。十四年,入朝,進檢校司空、同中書門下平章事,兼山陵使。俄以平章事為御史大夫,即建白擇御史當出大夫,不宜謀及宰相。因奏李衡、於結等任御史,宰相楊炎怒,寢不行。炎方詆劉晏,寧申救於帝,又素事元載,而炎亦出載門,故銜之,未忍發。



 是歲十月,南蠻與吐蕃合兵入文川、方維、邛郲,覆沒州縣,民逃匿山谷中。寧方在朝,軍無帥,德宗促寧進鎮。炎業與有嫌,恐已入蜀不可制,即說帝曰:「蜀,天下之奧壤,自寧擅制,朝廷失外府十四年矣。今寧雖來,以全師守蜀,賦稅入天子者與無地同。寧本與諸將等夷,獨因叛千百萬得位,不敢自有,以恩柔煦育,故威令不行。今雖歸之,必無功,是徒遣也;若其有功,誼不容奪。則西蜀之奧,敗固失之,勝亦非國家所有。惟陛下孰察。」帝曰:「卿策云何?」炎曰:「請無歸寧。今硃泚所部範陽勁卒戍近甸,趨與禁兵雜往,舉無不克,因是役得以親兵內其腹中,則蜀將破膽不敢動,然後換授他帥,以收其權,得千里肥饒之地,是謂因小禍受大福也。」帝曰:「善。」遂罷寧西川節度,改兼京畿觀察使、靈州大都督、單于鎮北大都護、朔方節度、鄜坊丹延州都團練觀察等使,托言重臣綏靜北陲,而每道置留後,使得自奏事,杜希全靈州,王翃振武,李建徽鄜州,及戴休顏、杜從政、呂希倩皆炎署置,使伺寧過失。寧至夏州,與希倩招黨項,降者甚眾。炎惡之,即奏希倩無綏邊才,而以神武將軍時常春代之,更拜寧尚書右僕射、知省事,司空如故。



 硃泚亂,帝出居奉天,寧後數日至,帝喜甚。寧謂所親曰:「上聰明,從善如轉規,但為盧杞所惑至此爾。」因潸然涕下。杞聞之,思有以構寧於帝。會王翃赴難時,與寧俱出延平門而西,寧數下馬趨廁,輒迂久。翃懼賊追,即呼曰:「既至此,而欲顧望乎?」杞微聞,即諷翃以聞。會泚行反間,而除柳渾為宰相,署寧中書令。時朔方掌書記康湛為盩厔尉,翃逼湛詐作寧遺泚書獻之,杞遂奏:寧初無效順心,向聞與賊盟署中書令,今果後至,復得所與賊書,反狀明甚。若兇渠外逼,奸臣內謀,則大事去矣。」因俯伏歔欷曰:「臣備位宰相,危不能持,顛不能扶,罪當死。」帝命左右扶起之,乃召寧至朝堂,雲使宣慰江淮。俄而中人引寧幕後,使二力士縊殺之,年六十一。



 初,命陸贄草制,贄索寧與泚書,將坐其事。杞復云:「書已亡。」寧死,籍其家,中外冤之。帝乃赦寧親屬,而歸其資云。貞元十二年,寧故將夏綏銀節度使韓潭請以所加禮部尚書雪寧罪,有詔聽其家收葬。始,寧入朝,留其弟寬守成都,楊子琳乘間起瀘州,以精騎數千襲據其城。寬戰力屈,寧妾任素驍果,即出家財十萬募勇士,得千人,設部隊,自將以進。子琳大懼,會糧盡,且大雨,引舟至廷,乘而去。子琳者,本瀘南賊帥,既降,詔隸劍南節度,屯瀘州,杜鴻漸表為刺史。既敗,收餘兵沿江而下,諸刺史震慄,備餼牢以饗士。過黃草峽,守捉使王守仙伏兵五百,子琳前驅至,悉禽之,遂入夔州,殺別駕張忠,城守以請罪。朝廷以其本謀近忠,故授峽州刺史,移澧州鎮遏使。後歸朝,賜名猷。



 寧季弟密,密子繪,俱以文辭稱。繪四子:蠡、黯、確、顏,皆擢進士第。



 蠡字越卿,開成中為戶部侍郎,白罷忌日百官行香,有詔褒可。歷平盧、天平軍節度使,終尚書左丞。



 子蕘,字野夫,乾符中為吏部侍郎,美文辭,談辯華給,以銓管非所長,出為陜虢觀察使。是時王仙芝亂漢上,河南群盜興,蕘簡侻不曉事,但以器韻自高,委政廝豎,不恤人疾苦。或訴旱者,指廷樹示之曰:「柯葉尚爾,何旱為?」即搒笞之,上下離心。俄為軍吏所執,髡其髯鬢。蕘再拜祈免,乃得去。渴甚,求飲於民,民飲以溺。坐失守,貶端州司馬,終左散騎常侍。



 黯,字直卿,開成初為監察御史,奏郊廟祭事不虔。文宗語宰相曰:「宗廟之禮,朕當親之。但千乘萬騎,國用不給,故使有司侍祠,然是日朕正衣冠坐以俟旦。今聞主者不虔,祭器敝惡,豈朕事神蠲潔意邪?公宜敕有司道朕斯意。」黯乃具條以聞。擢員外郎,累遷諫議大夫。



 確、顏,位皆郎中。



 嚴礪,字元明,震從祖弟也。少為浮屠法,太守見之,偉其材,表為玄武尉。震在山南,署牙將。德宗之幸,主饋餉有功。然輕躁多奸謀,以便佞自將。累為興州刺史。震卒,以礪權主留府事,遺言薦之,即拜本道節度使。詔下諫議大夫、給事中、補闕、拾遺合議,皆以為「礪資淺,士望輕,不宜授節制」,帝不從。



 礪在位,貪沓茍得,士民不勝其苦。素惡鳳州刺史馬勛,即誣奏,貶賀州司戶參軍。劉闢反,以儲備有素,檢校尚書左僕射,節度東川。擅沒吏民田宅百餘所,稅外加斂錢及芻粟數十萬。元和四年,卒,贈司空。後監察御史元稹奉使東川,劾發其贓,請加惡謚。朝廷以其死,故但追田宅奴婢還其主,稅外所斂悉蠲除云。



\end{pinyinscope}