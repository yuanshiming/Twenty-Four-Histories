\article{列傳第一百二十三 儒學上}

\begin{pinyinscope}

 高祖始受命,鉏類夷荒,天下略定,即詔有司立周公、孔子廟於國學,四時祠。求其後紀末產生於美國,20世紀初開始在資本主義各國廣泛流行。,議加爵土。國學始置生七十二員,取三品以上子、弟若孫為之;太學百四十員,取五品以上;四門學百三十員,取七品以上。郡縣三等,上郡學置生六十員,中、下以十為差;上縣學置生四十員,中、下亦以十為差。又詔宗室、功臣子孫就秘書外省,別為小學。



 太宗身橐鞬,風纚露沐,然銳情經術,即王府開文學館,召名儒十八人為學士,與議天下事。既即位,殿左置弘文館,悉引內學士番宿更休;聽朝之間,則與討古今,道前王所以成敗,或日昃夜艾,未嘗少怠。貞觀六年,詔罷周公祠,更以孔子為先聖,顏氏為先師,盡召天下惇師老德以為學官。數臨幸觀釋菜,命祭酒博士講論經義,賜以束帛。生能通一經者,得署吏。廣學舍千二百區,三學益生員,並置書、算二學,皆有博士。大抵諸生員至三千二百。自玄武屯營飛騎,皆給博士受經,能通一經者,聽入貢限。四方秀艾,挾策負素,坌集京師,文治煟然勃興。於是新羅、高昌、百濟、吐蕃、高麗等群酋長並遣子弟入學,鼓笥踵堂者,凡八千餘人。紆侈袂,曳方履,訚訚秩秩,雖三代之盛,所未聞也。帝又讎正《五經》繆闕,頒天下示學者,與諸儒稡章句為義疏,俾久其傳。因詔前代通儒梁皇偘、褚仲都、周熊安生、沈重、陳沈文阿、周弘正、張譏、隋何妥、劉炫等子孫,並加引擢。二十一年,詔「左丘明、卜子夏、公羊高、穀梁赤、伏勝、高堂生、戴聖、毛萇、孔安國、劉向、鄭眾、杜子春、馬融、盧植、鄭玄、服虔、何休、王肅、王弼、杜預、範寧二十一人,用其書,行其道,宜有以褒大之,自今並配享孔子廟廷」。於是唐三百年之盛,稱貞觀,寧不其然。



 高宗尚吏事,武後矜權變,至諸王駙馬,皆得領祭酒。初,孔穎達等始署官,發《五經》題與諸生酬問;及是,惟判祥瑞案三牒即罷。



 玄宗詔群臣及府郡舉通經士,而褚無量、馬懷素等勸講禁中,天子尊禮,不敢盡臣之。置集賢院部分典籍、乾元殿博匯群書至六萬卷,經籍大備,又稱開元焉。祿山之禍,兩京所藏,一為炎埃,官啇私楮,喪脫幾盡,章甫之徒,劫為縵胡。於是嗣帝區區救亂未之得,安暇語貞觀、開元事哉?自楊綰、鄭餘慶、鄭覃等以大儒輔政,議優學科,先經誼,黜進士,後文辭,亦弗能克也。文宗定《五經》,鑱之石,張參等是正訛文,寥寥一二可紀。由是觀之,始未嘗不成于艱難,而後敗於易也。



 嘗論之,武為救世砭劑,文其膏粱歟!亂已定,必以文治之。否者,是病損而進砭劑,其傷多矣!然則武得之,武治之,不免霸且盜,聖人反是而王。故曰武創業,文守成,百世不易之道也。若乃舉天下一之於仁義,莫若儒。儒待其人,乃能光明厥功,宰相大臣是已。至專誦習傳授、無它大事業者,則次為《儒學篇》。



 徐曠,字文遠,以字行。南齊司空孝嗣五世孫。父徹,梁秘書郎,尚元帝女安昌公主。江陵陷,俘以西,客偃師,貧不能自給。兄文林鬻書於肆,文遠日閱之,因博通《五經》,明《左氏春秋》。時耆儒沈重講太學,授業常千人,文遠從之質問,不數日辭去。或問其故,答曰:「先生所說,紙上語耳。若奧境,彼有所未見者,尚何觀?」重知其語,召與反復研辯,嗟嘆其能。性方正,舉動純重,竇威、楊玄感、李密、王世充皆從受學。



 隋開皇中,累遷太學博士,詔與漢王諒授經。會諒反,除名為民。大業初,禮部侍郎許善心薦文遠及包愷、褚徽、陸德明、魯達為學官,擢國子博士,愷等為太學博士。世稱《左氏》有文遠,《禮》有褚徽,《詩》有魯達,《易》有陸德明,皆一時冠云。文遠說經,遍舉先儒異論,分明是非,乃出新意以折衷,聽者忘勞。越王侗署國子祭酒。



 時洛陽饑,文遠自出城樵拾,為李密所得。密使文遠南向坐,備弟子禮拜之,文遠謝曰:「前日以先王之道授將軍,今將軍擁兵百萬,威振四海,猶能屈體老夫,此盛德也,安敢不盡?將軍若欲為伊、霍,繼絕扶傾,吾雖老,猶願盡力;如為莽、卓,乘危迫險,則僕耄矣,無能為也!」密頓首曰:「幸得位上公,思所以竭力,先征化及刷國恥,然後入見天子,請罪於有司,惟先生教之。」答曰:「將軍,名臣子,累世盡節,前陷玄感黨,迷未遠而復,今若終之以忠,天下之人所望於將軍者。」密頓首曰:「恭聞命。」俄而世充專制,密又問焉,對曰:「彼殘忍而意褊促,必速於亂,將軍非破之不可以朝。」密曰:「常謂先生儒者,不學軍旅,至籌大計,乃明略過人。」



 密敗,復入東都。世充給稍異等,而文遠見輒先拜。或問:「君踞見李密而下王公,何邪?」答曰:「密,君子,能受酈生之揖;世充,小人,無容故人義。相時而動可也。」世充僭號,以為國子博士。子士會奔長安,世充怒,絕其稟,文遠餓幾死,數矣。身出樵,為羅士信所獲,送京師,仍為國子博士。



 高祖幸國學觀釋奠,文遠發《春秋》題,論難鋒生,隨方占對,莫能屈。帝異之,封東莞縣男。卒,年七十四。



 孫有功,自有傳。



 陸元朗,字德明,以字行,蘇州吳人。善名理言,受學於周弘正。陳太建中,後主為太子,集名儒入講承光殿,德明始冠,與下坐。國子祭酒徐孝克敷經,倚貴縱辯,眾多下之,獨德明申答,屢奪其說,舉坐咨賞。解褐始興國左常侍。陳亡,歸鄉閈。



 隋煬帝擢秘書學士。大業間,廣召經明士,四方踵至。於是德明與魯達、孔褒共會門下省相酬難,莫能詘。遷國子助教。越王侗署為司業,入殿中授經。王世充僭號,封子玄恕為漢王,以德明為師,即其廬行束脩禮。德明恥之,服巴豆劑,殭偃東壁下。玄恕入拜床垂,德明對之遺利,不復開口,遂移病成皋。



 世充平,秦王闢為文學館學士,以經授中山王承乾,補太學博士。高祖已釋奠,召博士徐文遠、浮屠慧乘、道士劉進喜各講經,德明隨方立義,遍析其要。帝大喜曰:「三人者誠辯,然德明一舉輒蔽,可謂賢矣!」賜帛五十匹,遷國子博士,封吳縣男。卒。



 論撰甚多,傳於世。後太宗閱其書,嘉德明博辯,以布帛二百段賜其家。



 子敦信,麟德中,繇左侍極檢校右相,累封嘉興縣子,以老疾致仕,終大司成。



 曹憲,揚州江都人。仕隋為秘書學士,聚徒教授凡數百人,公卿多從之游。於小學家尤邃,自漢杜林、衛宏以後,古文亡絕,至憲復興。煬帝令與諸儒撰《桂苑珠叢》,規正文字。又注《廣雅》,學者推其該,藏於秘書。



 貞觀中,揚州長史李襲譽薦之,以弘文館學士召,不至,即家拜朝散大夫,當世榮之。太宗嘗讀書,有奇難字,輒遣使者問憲,憲具為音注,援驗詳復,帝咨尚之。卒,年百餘歲。



 憲始以梁昭明太子《文選》授諸生,而同郡魏模、公孫羅、江夏李善相繼傳授,於是其學大興。句容許淹者,自浮屠還為儒,多識廣聞,精故訓,與羅等並名家。羅官沛王府參軍事、無錫丞。模,武后時為左拾遺,子景倩亦世其學,以拾遺召,後歷度支員外郎。善,見子邕傳。



 顏師古,字籀,其先瑯邪臨沂人。祖之推,自高齊入周,終隋黃門郎,遂居關中,為京兆萬年人。父思魯,以儒學顯。武德初,為秦王府記室參軍事。



 師古少博覽,精故訓學,善屬文。仁壽中,李綱薦之,授安養尉。尚書左僕射楊素見其年弱,謂曰:「安養,劇縣。子何以治之?」師古曰:「割雞未用牛刀。」素驚其言大,後果以乾治聞。時薛道衡為襄州總管,與之推舊,佳其才,每作文章,令指摘疵短。俄失職,歸長安,不得調,窶甚,資教授為生。



 高祖入關,謁見長春宮,授朝散大夫,拜燉煌公府文學,累遷中書舍人,專典機密。師古性敏給,明練治體。方軍國務多,詔令一出其手,冊奏之工,當時未有及者。太宗即位,拜中書侍郎,封瑯邪縣男,以母喪解。服除,還官。歲餘,坐公事免。



 帝嘗嘆《五經》去聖遠,傳習浸訛,詔師古於秘書省考定,多所厘正。既成,悉詔諸儒議,於是各執所習,共非詰師古。師古輒引晉、宋舊文,隨方曉答,誼據該明,出其悟表,人人嘆服。尋加通直郎、散騎常侍。帝因頒所定書於天下,學者賴之。



 俄拜秘書少監,專刊正事,古篇奇字世所惑者,討析申熟,必暢本源。然多引後生與讎校,抑素流,先貴勢,雖商賈富室子,亦竄選中,由是素議薄之,斥為郴州刺史。未行,帝惜其才,讓曰:「卿之學,信可稱者,而事親居官,朕無聞焉。今日之行,自誰取之?念卿曩經任使,朕不忍棄,後宜自戒。」師古謝罪,復留為故官。



 師古性簡峭,視輩行傲然,罕所推接。既負其才,早見驅策,意望甚高。及是頻被譴,仕益不進,罔然喪沮,乃闔門謝賓客,巾褐裙帔,放情蕭散,為林墟之適。多藏古圖畫、器物、書帖,亦性所篤愛。與撰《五禮》成,進爵為子。又為太子承乾注班固《漢書》上之,賜物二百段、良馬一,時人謂杜征南、顏秘書為左丘明、班孟堅忠臣。



 帝將有事泰山,詔公卿博士雜定其儀,而論者爭為異端。師古奏:「臣撰定《封禪儀注書》在十一年,於時諸儒謂為適中。」於是以付有司,多從其說。遷秘書監、弘文館學士。十九年,從征遼,道病卒,年六十五,謚曰戴。



 其所注《漢書》、《急就章》大顯於時。永徽三年,子揚廷為符璽郎,表上師古所撰《匡謬正俗》八篇。



 初,思魯與妻不相宜,師古苦諫,父不聽,情有所隔,故帝及之。



 師古弟相時,字睿,亦以學聞。為天策府參軍事。貞觀中,累遷諫議大夫,有爭臣風。轉禮部侍郎。羸瘠多病。」師古死,不勝哀而卒。



 師古叔游秦,武德初,累遷廉州刺史,封臨沂縣男。時劉黑闥初平,人多強暴,比游秦至,禮讓大行,邑里歌之,高祖下璽書獎勞。終鄆州刺史。撰《漢書決疑》,師古多資取其義。



 孔穎達,字仲達,冀州衡水人。八歲就學,誦記日千餘言,暗記《三禮義宗》。及長,明服氏《春秋傳》、鄭氏《尚書》、《詩》、《禮記》、王氏《易》,善屬文,通步歷。嘗造同郡劉焯,焯名重海內,初不之禮,及請質所疑,遂大畏服。



 隋大業初,舉明經高第,授河內郡博士。煬帝召天下儒官集東都,詔國子秘書學士與論議,穎達為冠,又年最少,老師宿儒恥出其下,陰遣客刺之,匿楊玄感家得免。補太學助教。隋亂,避地虎牢。



 太宗平洛,授文學館學士,遷國子博士。貞觀初,封曲阜縣男,轉給事中。時帝新即位,穎達數以忠言進。帝問:「孔子稱『以能問於不能,以多問於寡,有若無,實若虛』,何謂也?」對曰:「此聖人教人謙耳。己雖能,仍就不能之人以咨所未能;己雖多,仍就寡少之人更資其多。內有道,外若無;中雖實,容若虛。非特匹夫,君德亦然。故《易》稱『蒙以養正』,『明夷以蒞眾』。若其據尊極之位,炫聰耀明,恃才以肆,則上下不通,君臣道乖。自古滅亡,莫不由此。」帝稱善。除國子司業,歲餘,以太子右庶子兼司業。與諸儒議歷及明堂事,多從其說。以論撰勞,加散騎常侍,爵為子。



 皇太子令穎達撰《孝經章句》,因文以盡箴諷。帝知數爭太子失,賜黃金一斤、絹百匹。久之,拜祭酒,侍講東宮。帝幸太學觀釋菜,命穎達講經,畢,上《釋奠頌》,有詔褒美。後太子稍不法,穎達爭不已,乳夫人曰:「太子既長,不宜數面折之。」對曰:「蒙國厚恩,雖死不恨。」剴切愈至。後致仕,卒,陪葬昭陵,贈太常卿,謚曰憲。



 初,穎達與顏師古、司馬才章、王恭、王琰受詔撰《五經》義訓凡百餘篇,號《義贊》,詔改為《正義》云。雖包貫異家為詳博,然其中不能無謬冗,博士馬嘉運駁正其失,至相譏詆。有詔更令裁定,功未就。永徽二年,詔中書門下與國子三館博士、弘文館學士考正之,於是尚書左僕射於志寧、右僕射張行成、侍中高季輔就加增損,書始布下。



 穎達子志,終司業。志子惠元,力學寡言,又為司業,擢累太子諭德。三世司業,時人美之。



 王恭者,滑州白馬人。少篤學,教授鄉閭,弟子數百人。貞觀初,召拜太學博士,講《三禮》,別為《義證》,甚精博。蓋文懿、文達皆當時大儒,每講遍舉先儒義,而必暢恭所說。



 馬嘉運,魏州繁水人。少為沙門,還治儒學,長論議。貞觀初,累除越王東閣祭酒。退隱白鹿山,諸方來授業至千人。十一年,召拜太學博士、弘文館學士。以孔穎達《正義》繁釀,故掎摭其疵,當世諸儒服其精。高宗為太子,引為崇賢館學士,數與洗馬秦侍講宮中,終國子博士。



 歐陽詢,字信本,潭州臨湘人。父紇,陳廣州刺史,以謀反誅。詢當從坐,匿而免。江總以故人子,私養之。貌寢侻,敏悟絕人。總教以書記,每讀輒數行同盡,遂博貫經史。仕隋,為太常博士。高祖微時,數與游,既即位,累擢給事中。



 詢初仿王羲之書,後險勁過之,因自名其體。尺牘所傳,人以為法。高麗嘗遣使求之,帝嘆曰:「彼觀其書,固謂形貌魁梧邪?」嘗行見索靖所書碑,觀之,去數步復返,及疲,乃布坐,至宿其傍,三日乃得去。其所嗜類此。貞觀初,歷太子率更令、弘文館學士,封渤海男。卒,年八十五。



 子通,儀鳳中累遷中書舍人。居母喪,詔奪哀。每入朝,徒跣及門。夜直,藉槁以寢。非公事不語,還家輒號慟。年饑,未克葬,居廬四年,不釋服。冬月,家人以氈絮潛置席下,通覺,即徹去。遷累殿中監,封渤海子。天授初,轉司禮卿,判納言事。輔政月餘,會鳳閣舍人張嘉福請以武承嗣為太子,通與岑長倩等固執,忤諸武意。及長倩下獄,坐大逆死,來俊臣並引通同謀,通雖被慘毒無異詞,俊臣代占,誅之。神龍初,追復官爵。



 通蚤孤,母徐教以父書,懼其墮,嘗遺錢使市父遺跡,通乃刻意臨仿以求售,數年,書亞於詢,父子齊名,號「大小歐陽體」。褚遂良亦以書自名,嘗問虞世南曰:「吾書何如智永?」答曰:「吾聞彼一字直五萬,君豈得此?」曰:「孰與詢?」曰:「吾聞詢不擇紙筆,皆得如志,君豈得此?」遂良曰:「然則何如?」世南曰:「君若手和筆調,固可貴尚。」遂良大喜。通晚自矜重,以貍毛為筆,覆以兔毫,管皆象犀,非是未嘗書。



 硃子奢,蘇州吳人,從鄉人顧彪授《左氏春秋》,善文辭。隋大業中,為直秘書學士。天下亂,辭疾還鄉里。後從杜伏威入朝,授國子助教。



 太宗貞觀初,高麗、百濟同伐新羅,連年兵不解。新羅告急,帝假子奢員外散騎侍郎,持節諭旨,平三國之憾。子奢有儀觀,夷人尊畏之。二國上書謝罪,贈遺甚厚。初,子奢行,帝戒曰:「海夷重學,卿為講大誼,然勿入其幣,還當以中書舍人處卿。」子奢唯唯。至其國,為發《春秋》題,納其美女。帝責違旨,而猶愛其才,以散官直國子學,累轉諫議大夫、弘文館學士。



 始,武德時,太廟享止四室,高祖崩,將祔主於廟,帝詔有司詳議。子奢建言:「漢丞相韋玄成奏立五廟,劉歆議當七,鄭玄本玄成,王肅宗歆,於是歷代廟議不能一。且天子七廟,諸侯五,降殺以兩,禮之正也。若天子與子、男同,則間無容等,非德厚游廣、德薄游狹之義。臣請依古為七廟。若親盡,則以王業所基為太祖,虛太祖室以俟無疆,迭遷乃處之。」於是尚書共奏:「自《春秋》以來,言天子七廟,諸侯五,大夫三,士二。推親親,顯尊尊,為不可易之法,請建親廟六。」詔可。乃祔弘農府君、高祖神主為六室。及帝崩,禮部尚書許敬宗議:「弘農府君廟應毀。按玄成說,毀廟主當瘞,且四海常所宗享矣,舉而瘞之,非神理所愜。晉範宣議別廟以奉毀廟之主,或言當藏天府。天府,瑞異所舍也。《禮》去祧有壇有墠,臣皆所未安。唐家宗廟,共殿異室,以右為首。若奉遷主納右夾室,而得尊處,祈之禱之未絕也。」有詔如敬宗議。然言七廟者,本之子奢。



 帝嘗詔:「起居紀錄臧否,朕欲見之以知得失,若何?」子奢曰:「陛下所舉無過事,雖見無嫌,然以此開後世史官之禍,可懼也。史官全身畏死,則悠悠千載,尚有聞乎?」



 池陽令崔文康坐事,櫟陽尉魏禮臣劾治,獄成,御史言其枉。禮臣訴御史阿黨,乞下有司雜訊,不如所言請死。鞫報禮臣不實,詔如請。子奢曰:「在律,上書不實有定罪,今抵以死,死者不可復生,雖欲自新弗可得。且天下惟知上書獲罪,欲自言者,皆懼而不敢申矣。」詔可。



 子奢為人樂易,能劇談,以經誼緣飾。每侍宴,帝令論難群臣,恩禮甚篤。卒於官。



 張士衡,瀛州樂壽人。父文慶,北齊國子助教。士衡九歲居母喪,哀慕過禮。博士劉軌思見之,為泣下,奇其操,謂文慶曰:「古不親教子,吾為君成就之。」乃授以《詩》、《禮》。又從熊安生、劉焯等受經,貫知大義。仕隋為餘杭令,以老還家。



 大業兵起,諸儒廢學。唐興,士衡復講教鄉里。幽州都督燕王靈夔以禮邀聘,北面事之。太子承乾慕風迎致,謁太宗洛陽宮,帝賜食,擢朝散大夫、崇賢館學士。



 太子以士衡齊人也,問高氏何以亡?士衡曰:「高阿那瑰之兇險,駱提婆之佞,韓長鸞之虐,皆奴隸才,是信是使,忠良外誅,骨肉內離,剝喪黎元,故周師臨郊,人莫為之用,此所以亡。」復問:「事佛營福,其應奈何?」對曰:「事佛在清靜仁恕爾,如貪婪驕虐,雖傾財事之,無損於禍。且善惡必報,若影赴形,聖人言之備矣。為君仁,為臣忠,為子孝,則福祚永;反是而殃禍至矣!」時太子以過失聞,士衡因是規之,然不能用也。太子廢,給傳罷歸鄉里,卒。



 士衡以《禮》教諸生,當時顯者:永平賈公彥、趙李玄植。



 公彥終太學博士,撰次章句甚多。子大隱,儀鳳中,為太常博士。會太常仲春告瑞太廟,高宗問禮官:「何世而然?」大隱對曰:「古者祭以首時,薦以仲月。近世元日奏瑞,則二月告廟。告者必有薦,本於始不得其時焉。」遷累中書舍人。垂拱中,博士周悰請武氏廟為七室,唐廟為五,下比諸侯。大隱奏言:「秦、漢母後稱制,未有戾古越禮者。悰損國廟數,悖大義,不可以訓。」武后不獲已,偽聽之。時皆服大隱沈正不詭從,有大臣體。終禮部侍郎。



 公彥傳業玄植,玄植又受《左氏春秋》於王德韶,受《詩》於齊威,該覽百家記書。貞觀間,為弘文館直學士。高宗時,數召見,與方士、浮屠講說。玄植以帝暗弱,頗箴切其短,帝禮之,不寤。坐事遷巴令,卒。



 張後胤,字嗣宗,蘇州昆山人。祖僧紹,梁零陵太守。父沖,陳國子博士,入隋為漢王諒並州博士。



 後胤甫冠,以學行禪其家。高祖鎮太原,引為客,以經授秦王。義寧初,為齊王文學,封新野縣公。武德中,擢員外散騎侍郎,賜宅一區。



 太宗即位,進燕王諮議,從王入朝,召見。初,帝在太原,嘗問:「隋運將終,得天下者何姓?」答曰:「公家德業,天下系心,若順天而動,自河以北,指捴可定。然後長驅關右,帝業可成。」至是自陳所言,帝曰:「是事未始忘之。」乃賜燕月池。帝從容曰:「今日弟子何如?」後胤曰:「昔孔子門人三千,達者無子男之位。臣翼贊一人,乃王天下,計臣之功,過於先聖。」帝為之笑,令群臣以《春秋》酬難。帝曰:「朕昔受大誼於君,今尚記之。」後胤頓首謝曰:「陛下乃生知,臣叨天功為己力,罪也。」帝大悅,遷燕王府司馬。出為睦州刺史,乞骸骨,帝見其強力,問欲何官,因陳謝不敢。帝曰:「朕從卿受經,卿從朕求官,何所疑?」後胤頓首,願得國子祭酒,授之。遷散騎常侍。永徽中致仕,加金紫光祿大夫,朝朔望,祿賜防閣如舊。卒,年八十三,贈禮部尚書,謚曰康,陪葬昭陵。



 孫齊丘,歷監察御史、朔方節度使,終東都留守,謚曰貞獻。子鎰,別有傳。



 蓋文達,冀州信都人。博涉前載,尤明《春秋》三家。刺史竇抗集諸生講論,於是,劉焯、劉軌思、孔穎達並以耆儒開門授業,是日悉至,而文達依經辯舉,皆諸儒意所未叩,一坐厭嘆。抗奇之,問:「安所從學?」焯曰:「若人岐嶷,出自天然,以多問寡,則焯為之師。」抗曰:「冰生於水而寒於水,其謂此邪?」



 武德中,授國子助教,為秦王文學館直學士。貞觀初,擢諫議大夫、兼弘文館學士,為蜀王師。王有罪,文達免官。拜崇賢館學士,卒。



 宗人文懿,亦以儒學稱,當時號「二蓋」。高祖於秘書省置學以教王公子,文懿為國子助教。既升席,公卿更相質問,文懿譬曉密微,遠近宗仰。終國子博士。



 谷那律,魏州昌樂人。貞觀中,累遷國子博士。淹識群書,褚遂良嘗稱為「《九經》庫」。遷諫議大夫,兼弘文館學士。從太宗出獵,遇雨沾漬,因問曰:「油衣若為而無漏邪?」那律曰:「以瓦為之,當不漏。」帝悅其直,賜帛二百段,卒。



 孫倚相,仕為秘書省正字,讎覆圖書,多所刊定。子崇義,天寶末為幽州大將,以雄敢聞。歷左金吾衛大將軍,遂客薊門。生子從政,略涉儒學,有風操。事李寶臣,歷定州刺史,封清江郡王。寶臣及張孝忠妻,其女兄弟也。



 寶臣初倚任,晚稍疏忌,從政乃闔門謝交游不事。及惟岳知節度,與田悅謀拒天子命,從政諫曰:「上神斷,絀諸侯,欲致太平。爾考與燕有切骨恨,天子致討,命帥莫先於燕。誅怨復仇,必盡力後已。前日而考誅大將百餘,子弟存者常不平,乘危相覆,誰不能爾?昔魏有洺、相之圍,王師四集,身投零陵,仰天垂泣,不知所出。賴爾考保佑,頓兵不進,而先帝寬厚,僅獲赦貸。不然,田氏尚有種乎?今悅兇獪,孰與承嗣?爾又幼富貴,不出戶庭,便欲旅拒?且人心難知,天道難欺,軍中諸將乘危投隙,自古豈少哉!今圖久安計,莫若令而兄惟誠攝留後,爾速入宿衛,則福祿可保矣。」不納。從政塞門移疾不出,惟岳所信王他奴等疑其怨望,日伺之。從政懼,乃吐血,即仰藥,五日死。曰:「吾不恨死,而痛渠覆宗矣!」後惟岳被殺於王武俊,如其揣云。



 蕭德言,字文行,陳吏部郎引子也,系出蘭陵。明《左氏春秋》。甫冠,以國子生為岳陽王賓客。陳亡,徙關中。詭浮屠服亡歸江南,州縣部送京師。仁壽中,授校書郎。貞觀時,歷著作郎、弘文館學士。



 太宗欲知前世得失,詔魏徵、虞世南、褚亮及德言裒次經史百氏帝王所以興衰者上之,帝愛其書博而要,曰:「使我稽古臨事不惑者,公等力也!」賚賜尤渥。



 德言晚節學愈苦,每開經,輒祓濯束帶危坐,妻子諫曰:「老人何終日自苦?」答曰:「對先聖之言,何復憚勞?」詔以經授晉王。時許叔牙為侍讀,同勸講。王為太子,德言又兼侍讀,而叔牙亦兼弘文館學士。德言請致仕,太宗不許,下詔敦勉。封武陽縣侯,進秘書少監,久乃得謝。



 高宗立,拜銀青光祿大夫,全給其祿,遣通事舍人即家致問。乘輿至肅章門引見,禮遇隆重。由是晉府及東宮舊臣子孫,並增秩賜金。卒,年九十七,贈太常卿,謚曰博。



 叔牙,字延基,句容人。貞觀時,遷晉王府參軍事、弘文館直學士。於《詩》、《禮》尤邃,獻《詩纂義》十篇,太子寫付司經。御史大夫高智周見之曰:「欲明《詩》者,宜先讀此。」



 子子儒,字文舉。高宗時為奉常博士。初,太尉長孫無忌等議:「祠令及禮用鄭玄六天說,圓丘祀昊天上帝,南郊太微感帝,明堂太微五帝。直據緯為說,不指蒼旻為天,而以昊天帝當北辰耀魄寶,郊、明堂當太微五帝。唐家祀圓丘,太史所上圖,昊天上帝外自有北辰。令李淳風曰:『昊天上帝位於壇,北辰、斗列第二垓。』與緯書駁異。司馬遷《天官書》,太微宮五精之神,五星所奉,有人主象,故名曰帝,猶房、心有天王象,安得盡為天乎?日月麗於天,草木麗於地,以日月為天,草木為地,昧者不信也。《周官》『兆五帝四郊』,又有『祀五帝』,皆不言天,知太微之神,非天也。《經》稱『郊祀后稷』,王肅以郊、圓丘為一,玄析而二之,曰圓丘,曰郊,非聖人意。今祠令固守玄說,與著式相違,宜有刊正。且《經》『嚴父莫大於配天』,『宗祀文王於明堂,以配上帝』。明堂之祀,天也,星不足配之矣。《月令》『孟春祈谷上帝』,《春秋》『啟蟄而郊,郊而後耕』,故郊後稷以祈農,《詩》『春夏祈穀於上帝』,皆祭天也。著之感帝,尤為不稽。請四郊迎氣祀太微五帝,郊、明堂罷六天說,止祀昊天。方丘既祭地,又祭神州北郊,皆不載經,請止一祠。」詔曰:「可。」



 乾封初,帝已封禪,復詔祀感帝、神州,以正月祭北郊。司禮少常伯郝處俊等奏言:「顯慶定禮,廢感帝祀而祈穀昊天,以高祖配。舊祀感帝、神州,以元皇帝配。今改祈穀為祀感帝,又祀神州,還以高祖配,何升降紛紛焉?虞氏禘黃帝,郊嚳;夏禘黃帝,郊鯀;殷禘嚳,郊冥;周禘嚳,郊稷。玄謂禘者,祭天圓丘;郊者,祭上帝南郊。崔靈恩說夏正郊天,王者各祭所出帝,所謂『王者禘祖之所自出,以其祖配之』。則禘遠祖,郊始祖也。今禘、郊同祖,禮無所歸。神州本祭十月,以方陰用事也。玄說三王之郊,一用夏正。靈恩謂祭神州北郊,以正月。諸儒所言,猥互不明。臣願會奉常、司成、博士普議。」於是,子儒與博士陸遵楷、張統師、權無二等共白:「北郊月不經見,漢光武正月建北郊,咸和中議北郊以正月,武德以來用十月,請循武德詔書。」明年,詔圓方二丘、明堂、感帝、神州宜奉高祖、太宗配,仍祭昊天上帝及五天帝於明堂。



 子儒,長壽中,歷天官侍郎、弘文館學士,封潁川縣男。以選事委令史句直,日偃臥不下筆,時人語曰「句直平配」。既而補授失序,傳為口實。



 德言曾孫至忠,自有傳。



 敬播,蒲州河東人。貞觀初,擢進士第。時顏師古、孔穎達撰次《隋史》,詔播詣秘書內省參纂。再遷著作佐郎,兼修國史。從太宗伐高麗,而帝名所戰山為駐蹕,播謂人曰:「鑾輿不復東矣,山所以名,蓋天意也!」其後果然。遷太子司議郎。時初置是官,尤清近,中書令馬周嘆曰:「恨資品妄高,不得歷此職!」又與令狐德棻等撰《晉書》,大抵凡例皆播所發也。



 有司建言:「謀反大逆,惟父子坐死,不及兄弟,請更議。」詔群臣大議,播曰:「兄弟雖孔懷之重,然比於父子則輕,故生有異室,死有別宗。今高官重爵,本廕唯逮子孫,而不及昆季,烏得榮隔其廕,而罪均其罰?」詔從播議。



 永徽後,仕益貴,歷諫議大夫、給事中。始,播與許敬宗撰《高祖實錄》,興創業,盡貞觀十四年。至是,又撰《太宗實錄》,訖二十三年。坐事出為越州長史,徙安州,卒。



 房玄齡嘗稱播:「陳壽之流乎!」玄齡患顏師古注《漢書》文繁,令掇其要為四十篇。是時《漢書》學大興,其章章者若劉伯莊、秦景通兄弟、劉訥言,皆名家。



 伯莊者,彭城人,為弘文館學士,遷國子博士,與許敬宗等論撰甚多,終崇賢館學士。自所著書亦百餘篇。



 子之宏,世其學。武后時,以著作郎兼修國史,終相王府司馬。睿宗立,贈秘書監。



 景通者,晉陵人。與弟俱有名,皆精《漢書》,號「大秦君」、「小秦君」。當時治《漢書》,非其授者,以為無法云。景通仕至太子洗馬、兼崇賢館學士。後復踐其官及職。



 訥言,乾封中歷都水監主簿,以《漢書》授沛王。王為太子,擢訥言洗馬兼侍讀。嘗集俳諧十五篇,為太子歡。太子廢,高宗見,怒,除名為民。復坐事流死振州。



 羅道琮,蒲州虞鄉人。慷慨尚節義。貞觀末,上書忤旨,徙嶺表。有同斥者死荊、襄間,臨終泣曰:「人生有死,獨委骨異壤邪?」道琮曰:「吾若還,終不使君獨留此。」瘞路左去。歲餘,遇赦歸,方霖潦積水,失其殯處,道琮慟諸野,波中忽若湓沸者。道琮曰:「若尸在,可再沸。」祝已,水復湧,乃得尸,負之還鄉。尋擢明經,仕至太學博士,為時名儒。



\end{pinyinscope}