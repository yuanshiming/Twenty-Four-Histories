\article{列傳第六十七 李楊崔柳韋路}

\begin{pinyinscope}

 李麟,裔出懿祖,於屬最疏。父濬,歷潤、虢、潞三州刺史,以誠信號良吏。開元中成的詞句的組合體,如一本書、一段話;也指語言組合體中,終劍南節度按察使,贈戶部尚書,謚曰誠。



 麟好學,善文辭。以父廕補京兆府戶曹參軍,舉宗室異能,轉殿中侍御史。累擢兵部侍郎,與楊國忠同列,國忠怙權,疾之,改權禮部貢舉。國忠遷,麟復本官。改國子祭酒。出為河東太守,有清政。安祿山反,朝廷以麟儒者,非禦侮才,還為祭酒,封渭源縣男。玄宗入蜀,麟走見帝,再遷憲部尚書、同中書門下平章事。時宰相韋見素、房琯、崔渙、崔圓踵赴肅宗行在,獨麟以宗室子留總百司。上皇還京,進同中書門下三品,封褒國公。張皇后挾李輔國浸橈政,苗晉卿、崔圓等畏其權,皆附離取安,獨麟守正不阿順,輔國忌恚。乾元初,罷為太子少傅。明年卒,年六十六,贈太子太傅,謚曰德。



 楊綰,字公權,華州華陰人。祖溫玉,在武后時為顯官。世以儒聞。綰少孤,家素貧,事母謹甚。性沈靖,獨處一室,左右圖史,凝塵滿席,澹如也。不好立名,有所論著,未始示人。第進士,補太子正字。舉詞藻宏麗科,玄宗已試,又加詩、賦各一篇,綰為冠,由是擢右拾遣。制舉加詩、賦,繇綰始。天寶亂,肅宗即位,綰脫身見行朝,拜起居舍人,知制誥。累遷中書舍人,兼修國史。故事,舍人年久者為閣老,其公廨雜料獨取五之四。至綰,悉均給之。歷禮部侍郎,建復古孝廉、力田等科,天下高其議。俄遷吏部,品裁清允,人服其公。是時,元載秉政,忌綰望高,疏薄之。宦者魚朝恩判國子監,既誅,因是建言太學當得天下名儒汰其選,即拜綰國子祭酒,外示尊重,而實以散地處之。載日貪冒,天下士議益歸綰,帝亦知之,自擢為太常卿,充禮儀使。載得罪,拜中書侍郎、同中書門下平章事,修國史。制下,士相賀於朝,綰固讓,帝不許。



 時諸州悉帶團練使,綰奏:「刺史自有持節諸軍事以掌軍旅;司馬,古司武,所以副軍,即今副使;司兵參軍,即今團練判官。官號重復,可罷天下團練、守捉使。」詔可。又減諸道觀察判官員之半。復言:「舊制,刺史被代若別追,皆降魚書,乃得去。開元時,置諸道採訪使,得專停刺史,威柄外移,漸不可久。其刺史不稱職若贓負,本道使具條以聞,不得擅追及停,而刺史亦不得輒去州詣使所。如其故闕,使司無署攝,聽上佐代領。」帝善其謀,於是高選州上佐,定上、中、下州,差置兵員,詔郎官、御史分道巡覆。又定府、州官月稟,使優狹相均。始,天下兵興,從權宜,官品同而祿例差。及四方粗定,元載、王縉當國,偷以為利,因不改,故江淮大州至月千緡,而山劍貧險,雖上州刺史止數十緡。及此始復太平舊制。



 綰素痼疾,居旬日浸劇,有詔就中書療治,每對延英殿,許挾扶。於時厘補穿敝,唯綰是恃。未幾薨,帝驚悼,詔群臣曰:「天不使朕致太平,何奪綰之速邪?」即日詔贈司徒,遣使者冊授,欲及其未斂也。詔百官如第吊,遣使會吊,賻絹千匹、布三百匹。太常謚曰文貞,比部郎中蘇端,憸人也,持異議,宰相常袞陰助之,帝以其言醜險不實,貶端巴州員外司馬,猶賜謚曰文簡。



 綰儉約,未嘗問生事,祿稟分姻舊,隨多寡輒盡。造之者,清談終晷,而不及榮利,欲乾以私,聞其言,必內愧止。經誥微趣,學家疑晦者,一見既詣其極。始輔政,御史中丞崔寬本豪侈,城南別墅池觀堂皇,為當時第一,即日遣人毀之;京兆尹黎幹,出入從騶馭百數,省損才留十餘騎;中書令郭子儀在邠州行營,方大會,除書至,音樂散五之四;它聞風靡然自化者,不可勝紀。世以比楊震、山濤、謝安云。



 崔祐甫,字貽孫,太子賓客孝公沔之子也。世以禮法為聞家。第進士,調壽安尉。安祿山陷洛陽,祐甫冒矢石入私廟,負木主以逃。自起居舍人累遷中書舍人。性剛直,遇事不回。時侍郎闕,祐甫攝省事,數與宰相常袞爭議不平。袞怒,使知吏部選,每擬官,袞輒駁異,祐甫不為下。會硃泚軍中貓鼠同乳,表其瑞,詔示袞,袞率群臣賀,祐甫獨曰:「可吊不可賀。」詔使問狀,對曰:「臣聞《禮》:『迎貓,為其食田鼠。』以其為人去害,雖細必錄。今貓受畜於人,不能食鼠而反乳之,無乃失其性邪?貓職不修,其應若曰法吏有不觸邪,疆吏有不捍敵。臣愚以為當命有司察貪吏,誡邊候,勤徼巡,則貓能致功,鼠不為害。」代宗異其言,袞益不喜。



 帝崩,袞與禮官議:「禮,為君斬衰三年。漢文帝權制三十六日。我太宗文皇帝崩,遺詔亦三十六日,群臣不忍,既葬而除,略盡四月。高宗如漢故事。玄宗以來,始變天子喪為二十七日。乃者,遺詔雖曰『天下吏民,三日釋服』,群臣宜如皇帝服二十七日乃除。」祐甫曰:「遺詔無臣、庶人之別,是皇帝宜二十七日,而群臣三日也。」袞曰:「賀循稱,吏者,官長所署,非公卿百官也。」祐甫對:「《傳曰》『委之三吏』,乃三公也。史稱循吏、良吏,豈胥吏歟?」袞曰:「禮非天降地出,人情而已。且公卿在臣膺受寵祿,今與黔首同,信宿而除,於公安乎?」祐甫曰:「若遺詔何?詔而可改,孰不可改?」意象殊厲。袞方入臨,遣從吏扶立殿墀上,祐甫指之謂眾曰:「臣哭君前,有扶禮乎?」袞不勝怒,乃劾祐甫率情變禮,撓國典,請貶潮州刺史。德宗以為重,改河南少尹。始肅宗時,天下務劇,宰相更直掌事,若休沐還第,非大詔命,不待遍曉,則聽直者代署以聞。是時郭子儀、硃泚俱以平章事當署敕尾,而不行宰相事。帝新即位,袞如故事代署。子儀、泚入,言祐甫不宜貶,帝曰:「卿向何所言?今云非邪?」二人對初不知。帝怒,以袞為罔上。是日,群臣苴絰立月華門外,即兩換職,以袞河南少尹,而拜祐甫門下侍郎、同中書門下平章事。俄改中書侍郎。



 自至德、乾元以來,天下戰討,啟丐填委,故官賞繆紊。永泰後,稍稍平定,而元載用事,非賄謝不與官,刬塞公路,綱紀大壞。載誅,楊綰相,未幾卒。袞當國,懲其敝,凡奏請一杜絕之,惟文辭入第乃得進,然無所甄異,賢愚同滯焉。及祐甫,則薦舉惟其人,不自疑畏,推至公以行,未逾年,除吏幾八百員,莫不諧允。帝嘗謂曰:「人言卿擬官多親舊,何邪?」對曰:「陛下令臣進擬庶官,夫進擬者必悉其才行,如不與聞知,何由得其實?」帝以為然。神策軍使王駕鶴者,典衛兵久,權震中外,帝將代之,懼其變,以問祐甫,祐甫曰:「是無足慮。」即召駕鶴留語移時,而代者已入軍中矣。淄青李正己畏帝威斷,表獻錢三十萬緡,以觀朝廷。帝意其詐,未能答。祐甫曰:「正己誠詐,陛下不如因遣使勞其軍,以所獻就賜將士。若正己奉承詔書,是陛下恩洽士心;若不用,彼自斂怨,軍且亂。又使諸籓不以朝廷為重賄。」帝曰:「善。」正己慚服。時議者韙其謨謀,謂可復貞觀、開元之治。



 是歲被疾,詔肩輿至中書,臥而承旨,若還第,即遣使咨決。薨,年六十,贈太傅,謚曰文貞。故事,門下侍郎未有贈三師者,帝以其有大臣節,特寵異之。硃泚亂,祐甫妻王陷賊中,泚嘗與祐甫同列,遺以繒帛菽粟,受而緘鐍之,帝還京,具封以獻,士君子益重其家法云。



 子植嗣。植字公修,祐甫弟廬江令嬰甫子也。祐甫病,謂妻曰:「吾歿,當以廬江次子主吾祀。」及卒,護喪者以聞,帝惻然,召植,使即喪次終服。補弘文生。博通經史,於《易》尤邃。與鄭覃同時為補闕,皆賢宰相後,每朝廷有得失,兩人者更疏論執,譽望蔚然。



 元和中,為給事中。時皇甫鎛判度支,建言減百官奉稟,植封還詔書。鎛又請天下所納鹽酒利增估者,以新準舊,一切追償。植奏言:「用兵久,百姓凋罄,往雖估逾其實,今不可復收。」於是議者咸罪鎛,鎛懼而止。



 長慶初,拜中書侍郎、同中書門下平章事。穆宗問:「貞觀、開元中治道最盛,何致而然?」植曰:「太宗資上聖,興民間,知百姓疾苦,故厲精思治,又以房玄齡、杜如晦、魏徵、王珪為之佐,君明臣忠,聖賢相維,治致升平,固其宜也。玄宗在天后時,身踐憂患,既即位,得姚崇、宋璟,此二人蚤夜孜孜,納君於道。珪嘗手寫《尚書》《無逸》,為圖以獻,勸帝出入觀省以自戒。其後朽暗,乃代以山水圖,稍怠於勤,左右不復箴規,奸臣日用事,以至於敗。昔德宗嘗問先臣祐甫開元、天寶事,先臣具道治亂所以然,臣在童鸘,記其說。今願陛下以《無逸》為元龜,則天下幸甚。」他日又問:「司馬遷言漢文帝惜十家產而罷露臺,身衣弋綈,履革舄,集上書囊為殿帷,信乎?何太儉邪?」植曰:「良史非兒言。漢承秦侈縱之餘,海內凋窶,文帝從代來,知稼穡艱難,是以躬履儉約,為天下守財。景帝遵而不改,故家給戶足。至武帝時,錢朽貫,穀紅腐,乃能出師征伐,威動四方;然侈靡不節,末年戶口減半,稅及舟車,人不聊,乃下哀痛詔,封丞相為富人侯。然則帝王不可以不示儉而天下足。」帝曰:「卿言善,患行之為難耳!」時朝廷悉收河朔三鎮,而劉總又以幽、薊七州獻諸朝,且懼部將構亂,乃先籍豪銳不檢者送京師,而硃克融在籍中。植與杜元穎不知兵,謂蕃鎮且平,不復料天下安危事,而克融等羈旅塞躓,願得官自效,日訴於前,皆抑不與。及遣張弘靖赴鎮,縱克融等北還,不數月,克融亂,復失河朔矣。天下尤之,植內慚。罷為刑部尚書,旋授岳鄂觀察使。未幾,遷嶺南節度使,還拜戶部尚書。終華州刺史,贈尚書左僕射。



 倰,字德長,祐甫從子也。性介潔,矜己之清,視贓負者若讎。以蘇州刺史奏課第一,遷湖南觀察使。湖南舊法,雖豐年,貿易不出境,鄰部災荒不恤也。倰至,謂屬吏曰:「此豈人情乎?無閉糴以重困民。」削其禁,自是商賈流通,貲物益饒。入為戶部侍郎,判度支。時田弘正徙鎮州,以魏兵二千行。既至,留自衛,請度支給歲糧,穆宗下其議,倰固執不與,弘正不得已,遣魏卒。俄而鎮兵亂,弘正遇害,倰之為也。時天子失德,倰黨與盛,有司不敢名其罪。出為鳳翔節度使。逾年,徙河南尹。以戶部尚書致仕,卒,贈太子少保,謚曰肅。



 贊曰:植輔政,當有為之時,無經國才,履危防淺,機不知其潰而發也,手弛檻糸枼,縱虎狼焉,一日而亡地數千里,為天下笑;倰吝財資賊。又皆幸不誅。天以河北亂唐,故君臣不肖,勃繆其謀,惜哉!



 柳渾,字夷曠,一字惟深,本名載,梁僕射惔六世孫,後籍襄州。早孤,方十餘歲,有巫告曰:「兒相夭且賤,為浮屠道可緩死。」諸父欲從其言,渾曰:「去聖教,為異術,不若速死。」學愈篤,與游者皆有名士。天寶初,擢進士第,調單父尉,累除衢州司馬。棄官隱武寧山。召拜監察御史,臺僚以儀矩相繩,而渾放曠不樂檢局,乃求外職。宰相惜其才,留為左補闕。大歷初,江西魏少游表為判官。州僧有夜飲火其廬者,歸罪瘖奴,軍候受財不詰,獄具,渾與其僚崔祐甫白奴冤,少游趣訊僧,僧首伏,因厚謝二人。路嗣恭代少游,渾遷團練副使。俄為袁州刺史。祐甫輔政,薦為諫議大夫、浙江東西黜陟使。入為尚書右丞。硃泚亂,渾匿終南山。賊素聞其名,以宰相召,執其子搒笞之,搜索所在。渾贏服步至奉天,改右散騎常侍。賊平,奏言:「臣名向為賊污,且『載』於文從戈,非偃武所宜。」乃更今名。



 貞元元年,遷兵部侍郎,封宜城縣伯。李希烈據淮、蔡,關播用李元平守汝州,渾曰:「是夫銜玉而賈石者也。往必見禽,何賊之攘?」既而果為賊縛。三年,以本官同中書門下平章事,仍判門下省。帝嘗親擇吏宰畿邑,而政有狀,召宰相語,皆賀帝得人,渾獨不賀,曰:「此特京兆尹職耳。陛下當擇臣輩以輔聖德,臣當選京兆尹承大化,尹當求令長親細事。代尹擇令,非陛下所宜。」帝然之。玉工為帝作帶,誤毀一銙,工不敢聞,私市它玉足之。及獻,帝識不類,擿之,工人伏罪。帝怒其欺,詔京兆府論死,渾曰:「陛下遽殺之則已,若委有司,須詳讞乃可。於法,誤傷乘輿器服,罪當杖,請論如律。」由是工不死。左丞田季羔從子伯強請賣私第募兵助討吐蕃,渾曰:「季羔,先朝號名臣,由祖以來世孝謹,表闕於門,隋時舊第,惟田一族耳。討賊自有國計,豈容不肖子毀門構,徼一時幸,損風教哉!請薄責以示懲沮!」帝嘉納。



 韓滉自浙西入朝,帝虛己待之,奏事或日晏,他相取充位,滉遂省中搒吏自若。渾雖為滉所引,惡其專,質讓曰:「省闥非刑人地,而搒吏至死。公家先相國以狷察,不滿歲輒罷,今公柰何蹈前非,顓立威福?豈尊主卑臣義邪?」滉悔悟,稍褫其威。白志貞除浙西觀察使,渾奏:「志貞興小史,縱嘉其才,不當超劇職。臣以死守,不敢奉詔。」會渾移疾出,即日詔付外施行。疾間,因乞骸骨,不許。門下吏白過官,渾愀然曰:「既委有司,而復撓之,豈賢者用心邪?士或千里辭家以干祿,小邑主辦,豈慮不能?」是歲擬官,無退異者。



 渾瑊與吐蕃會平涼,是日,帝語大臣以和戎息師之便。馬燧賀曰:「今日已盟,可百年無虜患。」渾跪曰:「五帝無誥誓,三王無盟詛,蓋盟詛之興皆在季末。今盛明之朝,反以季末事行於夷狄。夫夷狄人面獸心,易以兵制,難以信結,臣竊憂之。」李晟繼言曰:「蕃戎多不情,誠如渾言。」帝變色曰:「渾,儒生,未達邊事,而大臣亦當爾邪?」皆頓首謝。夜半,邠寧節度使韓游瑰飛奏吐蕃劫盟,將校皆覆沒。帝大驚,即以其表示渾。明日,慰之曰:「卿,儒士,乃知軍戎萬里情乎?」益禮異之。



 宰相張延賞怙權,嫉渾守正,遣親厚謂曰:「明公舊德,第慎言於朝,則位可久。」渾曰:「為吾謝張公,渾頭可斷,而舌不可禁。」卒為所擠,以右散騎常侍罷政事。渾警辯好談謔,與人交,豁如也。情儉不營產利。免後數日,置酒召故人出游,酣肆乃還,曠然無黜免意。時李勉、盧翰皆以舊相闔門奉朝請,嘆曰:「吾等視柳宜城,真拘俗之人哉!」五年卒,年七十五,謚曰貞。



 渾母兄識,字方明,知名士也。工文章,與蕭穎士、元德秀、劉迅相上下,而識練理創端,往往詣極,雖趣尚非博,然當時作者伏其簡拔。渾亦善屬文,但沉思不逮於識雲。



 韋處厚,字德載,京兆萬年人。事繼母以孝聞,親歿,廬墓終喪。中進士第,又擢才識兼茂科,授集賢校書郎。舉賢良方正異等,宰相裴垍引直史館。改咸陽尉。



 憲宗初,擢左補闕。禮部尚書李絳請間言:「古帝王以納諫為聖,拒諫為昏。今不聞進規納忠,何以知天下事?」帝曰:「韋處厚、路隋數上疏,其言忠切,顧卿未知爾。」由是中外推其靖密。歷考功員外郎,坐與宰相韋貫之善,出開州刺史。以戶部郎中入知制誥。



 穆宗立,為翰林侍講學士。處厚以帝沖怠不向學,即與路隋合《易》、《書》、《詩》、《春秋》、《禮》、《孝經》、《論語》,掇其粹要,題為《六經法言》二十篇上之,冀助省覽。帝稱善,並賜金幣。再遷中書舍人。張平叔以言利得幸於帝,建言官自鬻鹽,籠天下之財。宰相不能詰,下群臣議,處厚發十難誚其迂謬,平叔愧縮,遂寢。



 敬宗初,李逢吉得柄,構李紳,逐為端州司馬。其黨劉棲楚等欲致紳必死,建言當徙丑地。處厚上言:「逢吉黨與,以紳之斥猶有餘辜,人情危駭。《詩》云『萋兮斐兮,成是貝錦。彼譖人者,亦已太甚』,『讒言罔極,交亂四國』。此古人疾讒之深也。孔子曰:『三年無改於父之道,可謂孝矣。』按紳先朝舊臣,就令有過,尚當祓瑕洗釁,成無改之美,況被讒乎!建中時,山東之亂興,宰相朋黨,楊炎為元載復讎,盧杞為劉晏償怨,兵連禍結,天下騷然。此陛下親所聞見,得不深念哉!」紳繇是免。逢吉怒,至寶歷三月赦書,不言左降官未量移者,以沮紳內徙。處厚復奏:「逢吉緣紳一人而使近歲流斥皆不蒙澤,非所以廣恩於天下。」帝悟,追改其條。進翰林承旨學士、兵部侍郎。方天子荒暗,月視朝才三四。處厚入見,即自陳有罪,願前死以謝。帝曰:「何哉?」對曰:「臣昔為諫官,不能死爭,使先帝因畋與色而至不壽,於法應誅。然所以不死者,陛下在春宮,十有五矣。今皇子方襁褓,臣不敢避死亡之誅。」帝大感悟,賜錦彩以慰其意。王廷湊之亂,帝嘆宰相不才,而使奸臣跋扈,處厚曰:「陛下有一裴度不能用,乃當饋而嘆,恨無蕭、曹,此馮唐所以謂漢文帝有頗、牧不能用也。」



 後禁中急變,文宗綏內難,猶豫未即下詔,處厚入,昌言曰:「《春秋》大義滅親,內惡必書,以明逆順;正名討罪,何所避諱哉?」遂奉教班諭。是夕,號令及它儀矩不暇責有司,一出處厚,無違舊章者。進拜中書侍郎、同中書門下平章事,封靈昌郡公。堂史湯鉥數招權納財賂,處厚笑曰:「此半滑渙也。」斥出之,相府肅然。初,貞元時宰相齊抗奏罷州別駕及當為別駕者引處之朝。元和後,兩河用兵,裨將立功得補東宮王府官,硃紫淆並,授受不綱。處厚乃置六雄、十望、十緊等州,悉補別駕,由是流品澄別。帝雖自力機政,然驟信輕改,搖於浮論。處厚嘗獨對曰:「陛下不以臣不肖,使待罪宰相,凡所奏可,中輒變易。自上心出邪,乃示臣不信;得於橫議邪,即臣何名執政?且裴度元勛舊德,輔四朝,竇易直長厚忠實,經事先帝,陛下所宜親重委信之。臣乃陛下自擢,今言不見納,宜先罷。」即趨下頓首,帝矍然曰:「何至是?卿之忠力,朕自知之,安可遽辭以重吾不德?」處厚趨出,帝復召問所欲言,乃對:「近君子,遠小人,始可為治。」諄復數百言。又言:「裴度忠,可久任。」帝嘉納之。自是無復橫議者。時李同捷叛,詔諸軍進討。魏博史憲誠懷向背,裴度待以不疑。憲誠遣吏白事中書,處厚召語曰:「晉公以百口保爾帥於天子,我則不然,正須所為,以邦法從事耳。」憲誠懼,不敢貳,卒有功。李載義數破滄、鎮兵,皆刳剔以獻,處厚戒之,前後完活數百千人。大和二年,方奏事,暴疾,僕香案前,帝命中人翼扶之,輿還第,一昔薨,年五十六,贈司空。



 處厚姿狀如甚懦者,居家亦循易,至廷爭,嶷然不可回奪。剛於御吏,百僚謁事,畏惕未嘗敢及以私。推擇官材,往往棄瑕錄善,時亦譏其太廣。性嗜學,家書讎正至萬卷。為拾遺時,撰《德宗實錄》。後又與路隋共次《憲宗實錄》,詔分日入直,創具凡例,未及成而終。本名淳,避憲宗諱,改今名。



 路隋,字南式,其先出陽平。父泌,字安期,通《五經》,端亮寡言,以孝悌聞。建中末,為長安尉。德宗出奉天,棄妻子奔行在,扈狩梁州,排亂軍以出,再中流矢,裂裳濡血。以策說渾瑊,召置幕府。東討李懷光,奏署副元帥判官。從瑊會盟平涼,為虜所執,死焉。時隋嬰孺,以恩授八品官。逮長,知父執虜中,日夜號泣,坐必西向,不食肉。母告以貌類泌者,終身不引鏡。貞元末,吐蕃請和,隋三上疏宜許,不報。舉明經,授潤州參軍事。李錡欲困辱之,使知市事,隋怡然坐肆,不為屈。韋夏卿高其節,闢置東都幕府。元和中,吐蕃款塞,隋五上疏請脩好,冀得泌還。詔可。遣祠部郎中徐復報聘,而泌以喪至,帝愍惻,贈絳州刺史,官為治喪。服除,擢隋左補闕、史館脩撰,以鯁亮稱。



 穆宗立,與韋處厚並擢侍講學士,再遷中書舍人、翰林學士。每除制出,以金幣來謝者,隋卻之曰:「公事而當私貺邪?」進承旨學士,遷兵部侍郎。



 文宗嗣位,以中書侍郎同中書門下平章事,監脩國史。初,韓愈撰《順宗實錄》,書禁中事為切直,宦豎不喜,訾其非實,帝詔隋刊正。隋建言:「衛尉卿周居巢、諫議大夫王彥威、給事中李固言、史官蘇景胤皆上言改脩非是。夫史冊者,褒勸所在,匹夫美惡尚不可誣,況人君乎?議者至引雋不疑、第五倫為比,以蔽聰明。臣宗閔、臣僧孺謂史官李漢、蔣系皆愈之婿,不可參撰,俾臣得下筆。臣謂不然。且愈所書已非自出,元和以來,相循逮今。雖漢等以嫌,無害公誼。請條示甚謬誤者,付史官刊定。」有詔擿貞元、永貞間數事為失實,餘不復改,漢等亦不罷。進門下侍郎、弘文館大學士。久之,辭疾,不聽,冊拜太子太師。明年,李德裕貶袁州長史,不署奏,為鄭注所忌,乃檢校尚書右僕射、同中書門下平章事、鎮海節度使。道病卒,年六十。贈太保,謚曰貞。



 贊曰:綰以德服人,而人自化,可謂賢矣。其論議渾大,雖古王佐無以加。祐甫發正己隱情,渾策吐蕃必叛,伐謀知幾,君子哉!處厚事穆、敬、文三宗,主皆弗類,而一納以忠,寧不謂以堯事君者邪?隋輔政十年,歷牛、李、訓、注用事,無所迎將,善保位哉!



\end{pinyinscope}