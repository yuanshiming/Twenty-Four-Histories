\article{列傳第七十一 二李}

\begin{pinyinscope}

 李棲筠,字貞一,世為趙人。幼孤。有遠度,莊重寡言,體貌軒特。喜書「歷史」中的「布哈林」。,多所能曉,為文章,勁迅有體要。不妄交游。族子華每稱有王佐才,士多慕向。始,居汲共城山下,華固請舉進士,俄擢高第。調冠氏主簿,太守李峴視若布衣交。遷安西封常清節度府判官。常清被召,表攝監察御史,為行軍司馬。肅宗駐靈武,發安西兵,棲筠料精卒七千赴難,擢殿中侍御史。



 李峴為大夫,以三司按群臣陷賊者,表棲筠為詳理判官。推原其人所以脅污者,輕重以情,悉心助峴,故峴愛恕之,譽一旦出呂諲、崔器上。三遷吏部員外郎,判南曹。時大盜後,選簿亡舛,多偽冒,棲筠判析有條,吏氣奪,號神明。遷山南防禦觀察使。會峴去相,棲筠坐所善,除太子中允,眾不直,改河南令。



 李光弼守河陽,高其才,引為行軍司馬,兼糧料使。改絳州刺史,擢累給事中。是時,楊綰以進士不鄉舉,但試辭賦浮文,非取士之實,請置五經、秀才科。詔群臣議,棲筠與賈至、李廙以綰所言為是。進工部侍郎。關中舊仰鄭、白二渠溉田,而豪戚壅上游取磑利,且百所,奪農用十七。棲筠請皆徹毀,歲得租二百萬,民賴其入,魁然有宰相望。元載忌之,出為常州刺史。歲仍旱,編人死徙踵路,棲筠為浚渠,廝江流灌田,遂大稔。宿賊張度保陽羨西山,累年吏討不克,至是發卒捕斬,支黨皆盡,里無吠狗。乃大起學校,堂上畫《孝友傳》示諸生,為鄉飲酒禮,登歌降飲,人人知勸。以治行進銀青光祿大夫,封贊皇縣子,賜一子官。人為刻石頌德。



 蘇州豪士方清因歲兇,誘流殍為盜,積數萬,依黟、歙間,阻山自防,東南厭苦。詔李光弼分兵討平之。會平盧行軍司馬許杲恃功,擅留上元,有窺江、吳意,朝廷以創殘,重起兵,即拜棲筠浙西都團練觀察使圖之。棲筠至,張設武備,遣辯士厚齎金幣抵杲軍賞勞,使士歆愛,奪其謀。杲懼,悉眾度江,掠楚、泗而潰。以功進兼御史大夫。則又增學廬,表宿儒河南褚沖、吳何員等,超拜學官為之師,身執經問義,遠邇趨慕,至徒數百人。又奏部豪姓多徙貫京兆、河南,規脫徭科,請量產出賦,以杜奸謀。詔可。



 元載當國久,益恣橫,代宗不能堪,陰引剛鯁大臣自助,欲收綱權以黜載。會御史大夫敬括卒,即召棲筠與河南尹張延賞,擇可為大夫者。延賞先至,遂代括。會李少良、陸珽等上書劾載陰事,詔御史問狀,延賞稱疾,不敢鞫,少良、珽覆得罪死。帝殊失望,出延賞為淮南節度使,引拜棲筠為大夫。始,棲筠見帝,敷奏明辯,不阿附,帝心善之,故制麻自中以授,朝廷莫知也,中外竦眙。棲筠素方挺,無所屈。於是華原尉侯莫陳怤以優補長安尉,當參臺,棲筠物色其勞,怤色動,不能對,乃自言為徐浩、杜濟、薛邕所引,非真優也。始,浩罷嶺南節度使,以瑰貨數十萬餉載,而濟方為京兆,邕吏部侍郎,三人者,皆載所厚,棲筠並劾之。帝未決。會月蝕,帝問其故,棲筠曰:「月蝕脩刑,今罔上行私者未得,天若以儆陛下邪?」繇是怤等皆坐貶。故事,賜百官宴曲江,教坊倡顐雜侍,棲筠以任國風憲,獨不往,臺遂以為法。



 帝比比欲召相,憚載輒止。然有進用,皆密訪焉,多所補助。棲筠見帝猗違不斷,亦內憂憤,卒,年五十八,自為墓志。贈吏部尚書,謚曰文獻。



 棲筠喜獎善,而樂人攻己短,為天下士歸重,不敢有所斥,稱贊皇公云。



 子吉甫。吉甫字弘憲,以廕補左司禦率府倉曹參軍。貞元初,為太常博士,年尚少,明練典故。昭德皇后崩,自天寶後中宮虛,恤禮廢缺。吉甫草具其儀,德宗稱善。李泌、竇參器其才,厚遇之。陸贄疑有黨,出為明州長史。贄之貶忠州,宰相欲害之,起吉甫為忠州刺史,使甘心焉。既至,置怨,與結歡,人益重其量,坐是不徙者六歲。改郴、饒二州。會前刺史繼死,咸言牙城有物怪,不敢居。吉甫命菑除其署以視事,吏由是安。誅破奸盜窟穴,治稱流聞。



 憲宗立,以考功郎中召,知制誥。俄入翰林為學士,遷中書舍人。劉闢拒命,帝意討之,未決。吉甫獨請無置,宜絕朝貢以折奸謀。時李錡在浙西,厚賂貴幸,請用韓滉故事領鹽鐵,又求宣、歙。問吉甫,對曰:「昔韋皋蓄財多,故劉闢因以構亂。李錡不臣有萌,若益以鹽鐵之饒、採石之險,是趣其反也。」帝寤,乃以李巽為鹽鐵使。高崇文圍鹿頭未下,嚴礪請出並州兵,與崇文趨果、閬,以攻渝、合,吉甫以為非是,因言:「漢伐公孫述,晉伐李勢,宋伐譙縱,梁伐劉季連、蕭紀,凡五攻蜀,繇江道者四。且宣、洪、蘄、鄂強弩,號天下精兵,爭險地兵家所長,請起其兵搗三峽之虛,則賊勢必分,首尾不救,崇文懼舟師成功,人有鬥志矣。」帝從之。礪復請大臣為節度,吉甫諫曰:「崇文功且成,而又命帥,不復盡力矣。」因請以西川授崇文,而屬礪東川,益資、簡六州,使兩川得以相制。由是崇文悉力。劉闢平,吉甫謀居多。



 吐蕃遣使請尋盟,吉甫議:「德宗初,未得南詔,故與吐蕃盟。自異牟尋歸國,吐蕃不敢犯塞,誠許盟,則南詔怨望,邊隙日生。」帝辭其使。復請獻濱塞亭障南北數千里求盟,吉甫謀曰:「邊境荒岨,犬牙相吞,邊吏按圖覆視,且不能知。今吐蕃綿山跨谷,以數番紙而圖千里,起靈武,著劍門,要險之地所亡二三百所,有得地之名,而實喪之,陛下將安用此?」帝乃詔謝贊普,不納。



 張愔既得徐州,帝又欲以濠、泗二州還其軍,吉甫曰:「泗負淮,餉道所會,濠有渦口之險,前日授建封,幾失形勢。今愔乃兩廊壯士所立,雖有善意,未能制其眾。又使得淮、渦,厄東南走集,憂未艾也。」乃止。



 中書史滑渙素厚中人劉光琦,凡宰相議為光琦持異者,使渙請,常得如素,宦人傳詔,或不至中書,召渙於延英承旨,迎附群意,即為文書,宰相至有不及知者。由是通四方賂謝,弟泳,官至刺史。鄭餘慶當國,嘗一責怒,數日即罷去。吉甫請間,劾其奸,帝使簿渙家,得貲數千萬,貶死雷州。又建言:「州刺史不得擅見本道使,罷諸道歲終巡句以絕苛斂,命有司舉材堪縣令者,軍國大事以寶書易墨詔。」由是帝愈倚信。



 元和二年,杜黃裳罷宰相,乃擢吉甫中書侍郎、同中書門下平章事。吉甫連蹇外遷十餘年,究知閭里疾苦,常病方鎮強恣,至是為帝從容言:「使屬郡刺史得自為政,則風化可成。」帝然之,出郎吏十餘人為刺史。自王叔文時選任猥冒,吉甫始簿其員,人得敘進,官無留才。又度李錡必反,勸帝召之,使者三往,以病解,而多持金啗權貴,至為錡游說者。吉甫曰:「錡,庸材,而所蓄乃亡命群盜,非有鬥志,討之必克。」帝意決。復言:「昔徐州亂,嘗敗吳兵,江南畏之。若起其眾為先鋒,可以絕徐後患。韓弘在汴州,多憚其威,誠詔弘子弟率兵為掎角,則賊不戰而潰。」從之。詔下,錡眾聞徐、梁兵興,果斬錡降。以功封贊皇縣侯,徙趙國公。德宗以來,姑息蕃鎮,有終身不易地者。吉甫為相歲餘,凡易三十六鎮,殿最分明。



 裴均以尚書右僕射判度支,結黨傾執政。會皇甫湜等對策,指褭權強,用事者皆怒,帝亦不悅。均黨因宣言:「殆執政使然。」右拾遺獨孤鬱、李正辭等陳述本末,帝乃解。吉甫本善竇群、羊士諤、呂溫,薦群為御史中丞。群即奏士諤侍御史,溫知雜事。吉甫恨不先白,持之,久不決,群等銜之。俄而吉甫病,醫者夜宿其第,群捕醫者,劾吉甫交通術士。帝大駭,訊之無狀,群等皆貶。而吉甫亦固乞免,因薦裴垍自代,乃以檢校兵部尚書、兼中書侍郎、同中書門下平章事,為淮南節度使。帝為御通化門祖道,賜御餌禁方。居三歲,奏蠲逋租數百萬,築富人、固本二塘,溉田且萬頃。漕渠庳下不能居水,乃築堤閼以防不足,洩有餘,名曰平津堰。江淮旱,浙東、西尤甚,有司不為請,吉甫白以時救恤,帝驚,馳遣使分道賑貸。吉甫雖居外,每朝廷得失輒以聞。



 六年,裴垍病免,復以前官召吉甫還秉政。入對延英,凡五刻罷。帝尊任之,官而不名。吉甫疾吏員廣,繇漢至隋,未有多於今者,乃奏曰:「方今置吏不精,流品龐雜,存無事之官,食至重之稅,故生人日困,冗食日滋。又國家自天寶以來,宿兵常八十餘萬,其去為商販、度為佛老、雜入科役者,率十五以上。天下常以勞苦之人三,奉坐待衣食之人七。而內外官仰奉稟者,無慮萬員,有職局重出,名異事離者甚眾,故財日寡而受祿多,官有限而調無數。九流安得不雜?萬務安得不煩?漢初置郡不過六十,而文、景化幾三王,則郡少不必政紊,郡多不必事治。今列州三百、縣千四百,以邑設州,以鄉分縣,費廣制輕,非致化之本。願詔有司博議,州縣有可並並之,歲時入仕有可停停之,則吏寡易求,官少易治。國家之制,官一品,奉三千,職田祿米大抵不過千石。大歷時,權臣月奉至九千緡者,州刺史無大小皆千緡,宰相常袞始為裁限,至李泌量閑劇稍增之,使相通濟。然有名在職廢,奉存額去,閑劇之間,厚薄頓異,亦請一切商定。」乃詔給事中段平仲、中書舍人韋貫之、兵部侍郎許孟容、戶部侍郎李絳參閱蠲減,凡省冗官八百員,吏千四百員。又奏收都畿佛祠田、磑租入,以寬貧民。



 德宗時,義陽、義章二公主薨,詔起祠堂於墓百二十楹,費數萬計。會永昌公主薨,有司以請,帝命減義陽之半。吉甫曰:「德宗一切之恩,不可為法。昔漢章帝欲起邑屋於親陵,東平王蒼以為不可。故非禮之舉,人君所慎。請裁置墓戶,以充守奉。」帝曰:「吾固疑其冗,減之,今果然。然不欲取編民,以官戶奉墳而已。」吉甫再拜謝。帝曰:「事不安者第言之,無謂朕不能行也。」十宅諸王既不出閤,諸女嫁不時,而選尚皆繇中人,厚為財謝乃得遣。吉甫奏:「自古尚主必慎擇其人。江左悉取名士,獨近世不然。」帝乃下詔皆封縣主,令有司取門閥者配焉。



 田季安疾甚,吉甫請任薛平為義成節度使,以重兵控邢、洺,因圖上河北險要所在,帝張於浴堂門壁,每議河北事,必指吉甫曰:「朕日按圖,信如卿料矣。」劉澭舊軍屯普潤,數暴掠近縣,吉甫奏還涇原,畿民賴之。



 八年,回鶻引兵自西城、柳谷侵吐蕃,塞下傳言且入寇。吉甫曰:「回鶻能為我寇,當先絕和而後犯邊,今不足虞也。」因請起夏州至天德復驛候十一區,以通緩急;發夏州精騎五百屯經略故城,以護黨項而已。既而果邊吏妄言。六胡州在靈武部中,開元時廢之,置宥州以處降戶,寓治經略軍,居中以制戎虜,北援天德,南接夏州。至德、寶應間,廢宥州,以軍遙隸靈武,道里曠遠,故黨項孤弱,虜數擾之。吉甫始奏復宥州,乃治經略軍,以隸綏銀道,取鄜城神策屯兵九千實之。以江淮甲三十萬給太原、澤潞軍,增太原馬千匹。由是戎備完輯。



 自蜀平,帝銳意欲取淮西。方吉甫在淮南,聞吳少陽立,上下攜泮,自請徙壽州,以天子命招懷之,反間以撓其黨,會討王承宗,未及用。後田弘正以魏歸,吉甫知魏人謂田進誠才,而唐州乃蔡喉衿,請拔進誠為刺史,以臨賊境,且慰魏心。烏重胤守河陽,吉甫以汝州捍蔽東都,聯唐、許,當蔡西面,兵寡不足憚寇,而河陽乃魏博之津,弘正歸國,則為內鎮,不宜戍重兵示不信,請徙屯汝州。帝皆從之。後弘正拜檢校尚書右僕射,賜其軍錢二千萬,弘正曰:「吾未喜於移河陽軍也。」及元濟擅立,吉甫以內地無脣齒援,因時可取,不當用河朔故事,與帝意合。又請自往招元濟,茍逆志不悛,得指授群帥俘賊以獻天子。不許,固請至流涕,帝慰勉之。會暴疾卒,年五十七。帝震悼,賻外別賜縑五百恤其家,自大斂至卒哭,皆中人臨吊。吉甫圖淮西地,未及上,帝敕其子獻之。及葬,祭以少牢,贈司空。有司謚曰敬憲,度支郎中張仲方非之,帝怒,貶仲方,更賜謚曰忠懿。



 始,吉甫當國,經綜政事,眾職咸治。引薦賢士大夫,愛善無遺,褒忠臣後,以起義烈。與武元衡連位,未幾節度劍南,屢言元衡材,宜還為相。及再輔政,天下想望風採,而稍修怨,罷李籓宰相,而裴垍左遷,皆其謀也。李正辭晚相失,及與蕭俯同召為翰林學士,獨用俯而罷正辭,人莫不疑憚。帝亦知其專,乃進李絳,遂與有隙,數辯爭殿上,帝多直絳。然畏慎奉法,不忮害,顧大體。左拾遺楊歸厚嘗請對,日已旰,帝令它日見,固請不肯退。既見,極論中人許遂振之奸,又歷詆輔相,求自試,又表假郵置院具婚禮。帝怒其輕肆,欲遠斥之,李絳為言,不能得。吉甫見帝,謝引用之非,帝意釋,得以國子主簿分司東都。初,政事堂會食,有巨床,相傳徙者宰相輒罷,不敢遷,吉甫笑曰:「世俗禁忌,何足疑邪?」徹而新之。吉甫居安邑里,時號「安邑李丞相」。所論著甚多,皆行於世。前卒一歲,熒惑掩太微上相,吉甫曰:「天且殺我。」再遜位,不許。



 子德修,亦有志操,寶歷中為膳部員外郎。張仲方入為諫議大夫,德修不欲同朝,出為舒、湖、楚三州刺史。卒。



 次子德裕,自有傳。



 李庸阜,字建侯,北海太守邕之從孫。第進士,又以書判高等補秘書省正字。李懷光闢致幕府,擢累監察御史。懷光反河中,庸阜與母、妻陷焉,因紿懷光以兄病臥洛且革,母欲往視;懷光許可,戒妻子無偕行。庸阜私遣之,懷光怒,欲加罪,謝曰:「庸阜籍在軍,不得為母駕,奈何不使婦往?」懷光止不問。後與高郢刺賊虛實及所以攻取者,白諸朝,德宗手詔褒答。懷光覺,嚴兵召二人問之,庸阜詞氣不撓,三軍為感動,懷光不殺,囚之。河中平,馬燧破械致禮,表佐其府,以言不用,罷歸洛中。召為吏部員外郎。



 徐州張建封卒,兵亂,囚監軍,迫建封子愔主軍務。帝以庸阜剛敢,拜宣慰使,持節直入其軍,大會士,喻以禍福,出監軍獄中,脫桎梏,使復位,眾不敢動。愔即上表謝罪,稱兵馬留後,庸阜曰:「非詔命,安得輒稱之?」削去乃受。既還,稱旨,遷郎中。



 順宗時,進御史中丞。憲宗立,為京兆尹,進尚書右丞。元和初,京師多盜賊,復拜京兆。以檢校禮部尚書為鳳翔、隴右節度使。是鎮常兼神策行營,前此用武將,始受詔,即詣軍脩謁。庸阜以為不可,詔為去神策行營號。俄徙河東,入為刑部尚書、諸道鹽鐵轉運使。



 拜淮南節度使。王師討蔡方急,李師道謀撓沮之,庸阜以兵二萬分壁鄆境,貲餉不仰有司。是時兵興,天子憂財乏,使程異馳驛江淮,諷諸道輸貨助軍。庸阜素富強,即籍府庫留一歲儲,餘盡納於朝,諸道由是悉索以獻,繄庸阜倡之。



 先是,吐突承璀為監軍,貴寵甚,庸阜以剛嚴治,相禮憚,稍厚善。承璀歸,數稱薦之,召拜門下侍郎、同中書門下平章事。庸阜不喜由宦幸進,及出祖,樂作泣下,謂諸將曰:「吾老安外鎮,宰相豈吾任乎?」至京師,不肯視事,引疾固辭,改戶部尚書。俄檢校尚書左僕射,兼太子賓客,分司東都。以太子少傅致仕,卒,贈太子太保,謚曰肅。



 庸阜強直無私,與楊憑、穆質、許孟容、王仲舒友善,皆以氣自任。而庸阜當官,以峭法操下,所至稱治。猛決少恩,在淮南七年,其生殺禽擿,多委軍吏,而參佐束手不得與,人往往陷非法,議者亦以此少之。



 子拭,仕歷宗正卿、京兆尹、河東鳳翔節度使,以秘書監卒。



 拭子磎,字景望。大中末,擢進士,累遷戶部郎中,分司東都。劾奏內園使郝景全不法事,景全反摘磎奏犯順宗嫌名,坐奪俸。磎上言:「『因事告事,旁訟他人』者,咸通詔語也。禮,不諱嫌名;律,廟諱嫌名不坐。豈臣所引詔書而有司輒論奏?臣恐自今用格令者,委曲回避,旁緣為奸也。」詔不奪俸。



 黃巢陷洛,磎挾尚書八印走河陽,時留守劉允章為賊脅,遣人就磎索印,拒不與。允章悟,亦不臣賊。嗣襄王之亂,轉側淮南,高駢受偽命,磎苦諫,不納。入為中書舍人、翰林學士。辭職歸華陰,復以學士召。



 乾寧元年,進禮部尚書、同中書門下平章事。崔昭緯素疾磎,諷劉崇魯掠其麻哭之,言:「磎懷奸,與中人楊復恭暱款,其弟為時溥所殺,不可相天子。」翌日,下遷太子少傅。磎乃自言為崇魯誣污,書十一上不止。初,崇魯父坐受賕,仰藥死,故磎以醜語及之,議者譏其非大臣體。昭宗素所器遇,決意復用之,而李茂貞等上言深詆其非,帝不獲已,又罷為太子少師。於是茂貞及王行瑜、韓建擁兵闕下,列磎罪,殺之於都亭驛。行瑜誅,有詔復官爵,贈司徒,謚曰文。



 磎好學,家有書至萬卷,世號「李書樓」。所著文章及注解諸書傳甚多。子沇,字東濟,有俊才,亦遇害,贈禮部員外郎。



 贊曰:剛者天德,故孔子稱「剛近仁」。骨強四支,故君有忠臣,謂之骨鯁。若棲筠、庸阜二子,其剛者歟!棲筠抗權邪,不及相;庸阜得相,不願拜。非剛,疇克勝之?吉甫踐天宰,謀謨是矣,而鯁正有愧於父云。



\end{pinyinscope}