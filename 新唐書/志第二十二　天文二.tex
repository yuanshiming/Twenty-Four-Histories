\article{志第二十二 天文二}

\begin{pinyinscope}

 ○日食



 武德元年十月壬申朔,日有食之,在氐五度。占曰:「諸侯專權,則其應在所宿國;諸侯附從,則為王者事。」四年八月丙戌朔,日有食之,在翼四度。楚分也。六年十二月壬寅朔,日有食之,在南斗十九度。吳分也。九年十月丙辰朔,日有食之,在氐七度。



 貞觀元年閏三月癸丑朔,日有食之,在胃九度。九月庚戌朔,日有食之,在亢五度。胃為天倉,亢為疏廟。二年三月戊申朔,日有食之,在婁十一度。占為大臣憂。三年八月己巳朔,日有食之,在翼五度。占曰:「旱。」四年正月丁卯朔,日有食之,在營室四度。七月甲子朔,日有食之,在張十四度。占為禮失。六年正月乙卯朔,日有食之,在虛九度,虛,耗祥也。八年五月辛未朔,日有食之,在參七度。九年閏四月丙寅朔,日有食之,在畢十三度。占為邊兵。十一年三月丙戌朔,日有食之,在婁二度。占為大臣憂。十二年閏二月庚辰朔,日有食之,在奎九度。奎,武庫也。十三年八月辛未朔,日有食之,在翼十四度。翼為遠夷。十七年六月己卯朔,日有食之,在東井十六度。京師分也。十八年十月辛丑朔,日有食之,在房三度。房,將相位。二十年閏三月癸巳朔,日有食之,在胃九度。占曰:「主有疾。」二十二年八月己酉朔,日有食之,在翼五度。占曰:「旱。」



 顯慶五年六月庚午朔,日有食之,在柳五度。



 龍朔元年五月甲子晦,日有食之,在東井二十七度。皆京師分也。



 麟德二年閏三月癸酉朔,日有食之,在胃九度。占曰:「主有疾。」



 乾封二年八月己酉朔,日有食之,在翼六度。



 總章二年六月戊申朔,日有食之,在東井二十九度。



 咸亨元年六月壬寅朔,日有食之,在東井十八度。二年十一月甲午朔,日有食之,在箕九度。三年十一月戊子朔,日有食之,在尾十度。東井,京師分。箕為后妃之府,尾為後宮。五年三月辛亥朔,日有食之,在婁十三度。占為大臣憂。



 永隆元年十一月壬申朔,日有食之,在尾十六度。



 開耀元年十月丙寅朔,日有食之,在尾四度。



 永淳元年四月甲子朔,日有食之,在畢五度。十月庚申朔,日有食之,在房三度。



 垂拱二年二月辛未朔,日有食之,在營室十五度。四年六月丁亥朔,日有食之,在東井二十七度。京師分也。



 天授二年四月壬寅朔,日有食之,在昴七度。



 如意元年四月丙申朔,日有食之,在胃十一度。皆正陽之月。



 長壽二年九月丁亥朔,日有食之,在角十度。角內為天廷。



 延載元年九月壬午朔,日有食之,在軫十八度。軫為車騎。



 證聖元年二月己酉朔,日有食之,在營室五度。



 聖歷三年五月己酉朔,日有食之,在畢十五度。



 長安二年九月乙丑朔,日有食之,幾既,在角初度。三年三月壬戌朔,日有食之,在奎十度。占曰:「君不安。」九月庚寅朔,日有食之,在亢七度。



 神龍三年六月丁卯朔,日有食之,在東井二十八度。京師分也。



 景龍元年十二月乙丑朔,日有食之,在南斗二十一度。斗為丞相位。



 先天元年九月丁卯朔,日有食之,在角十度。



 開元三年七月庚辰朔,日有食之,在張四度。七度五月己丑朔,日有食之,在畢十五度。九年九月乙巳朔,日有食之,在軫十八度。十二年閏十二月丙辰朔,日有食之,在虛初度。十七年十月戊午朔,日有食之,不盡如鉤,在氐九度。二十年二月甲戌朔,日有食之,在營室十度。八月辛未朔,日有食之,在翼七度。二十一年七月乙丑朔,日有食之,在張十五度。二十二年十二月戊子朔,日有食之,在南斗二十三度。二十三年閏十一月壬午朔,日有食之,在南斗十一度。二十六年九月丙申朔,日有食之,在亢九度。二十八年三月丁亥朔,日有食之,在婁三度。



 天寶元年七月癸卯朔,日有食之,在張五度。五載五月壬子朔,日有食之,在畢十六度。十三載六月乙丑朔,日有食之,幾既,在東井十九度。京師分也。



 至德元載十月辛巳朔,日有食之,既,在氐十度。



 上元二年七月癸未朔,日有食之,既,大星皆見,在張四度。



 大歷三年三月乙巳朔,日有食之,在奎十一度。十年十月辛酉朔,日有食之,在氐十一度。宋分也。十四年七月戊辰朔,日有食之,在張四度。十二月丙寅晦,日有食之,在危十二度。貞元二年八月辛巳朔,日有食之,在軫八度。五年正月甲辰朔,日有食之,在營室六度。八年十一月壬子朔,日有食之,在尾六度。宋分也。十二年八月己未朔,日有食之,在翼十八度。占曰:「旱。」十七年五月壬戌朔,日有食之,在東井十度。



 元和三年七月辛巳朔,日有食之,在七星三度。十年八月己亥朔,日有食之,在翼十八度。十三年六月癸丑朔,日有食之,在輿鬼一度。京師分也。



 長慶二年四月辛酉朔,日有食之,在胃十三度。三年九月壬子朔,日有食之,在角十二度。



 大和八年二月壬午朔,日有食之,在奎一度。



 開成元年正月辛丑朔,日有食之,在虛三度。



 會昌三年二月庚申朔,日有食之,在東壁一度。並州分也。四年二月甲寅朔,日有食之,在營室七度。五年七月丙午朔,日有食之,在張七度。六年十二月戊辰朔,日有食之,在南斗十四度。



 大中二元五月己未朔,日有食之,在參九度。八年正月丙戌朔,日有食之,在危一度。危為玄枵,亦耗祥也。



 咸通四年七月辛卯朔,日有食之,在張十七度。



 乾符三年九月乙亥朔,日有食之,在軫十四度。四年四月壬申朔,日有食之,在畢三度。六年四月庚申朔,日有食之,既,在胃八度。



 文德元年三月戊戌朔,日有食之,在胃一度。



 天祐元年十月辛卯朔,日有食之,在心二度。三年四月癸未朔,日有食之,在胃十二度。



 凡唐著紀二百八十九年,日食九十三:朔九十,晦二,二日一。



 ○日變



 貞觀初,突厥有五日並照。二十三年三月,日赤無光。李淳風曰:「日變色,有軍急。」又曰:「其君無德,其臣亂國。」濮陽復曰:「日無光,主病。」



 咸亨元年二月壬子,日赤無光。癸丑,四方濛濛,日有濁氣,色赤如赭。



 上元二年三月丁未,日赤如赭。



 永淳元年三月,日赤如赭。



 文明元年二月辛巳,日赤如赭。



 長安四年正月壬子,日赤如赭。



 景龍三年二月庚申,日色紫赤無光。



 開元十四年十二月己未,日赤如赭。二十九年三月丙午,風霾,日無光,近晝昏也。占為上刑急,人不樂生。



 天寶三載正月庚戌,日暈五重。占曰:「是謂棄光,天下有兵。」



 肅宗上元二年二月乙酉,白虹貫日。



 大歷二年七月丙寅,日旁有青赤氣,長四丈餘。壬申,日上有赤氣,長二丈。九月乙亥至於辛丑,日旁有青赤氣。三年正月丁巳,日有黃冠、青赤珥。辛丑,亦如之。凡氣長而立者為直,橫者為格,立於日上者為冠。直為有自立者,格為戰鬥。又曰:「赤氣在日上,君有佞臣。黃為土功,青赤為憂。」



 貞元二年閏五月壬戌,日有黑暈。六年正月甲子,日赤如血。十年三月乙亥,黃霧四塞,日無光。



 元和二年十月壬午,日傍有黑氣如人形跪,手捧盤向日,盤中氣如人頭。四年閏三月,日傍有物如日。五年四月辛未,白虹貫日。十年正月辛卯,日外有物如烏。十一年正月己卯,日紫赤無光。



 長慶元年六月己丑,白虹貫日。三年二月庚戌,白虹貫日。



 寶歷元年六月甲戌,赤虹貫日。九月甲申,日赤無光。二年三月甲午,日中有黑氣如柸。辛亥,日中有黑子。四月甲寅,白虹貫日。



 太和二年二月癸亥,日無光,白霧晝昏。十二月癸亥,有黑祲,與日如斗。五年二月辛丑,白虹貫日。六年三月,有黑祲與日如斗。庚戌,日中有黑子。四月乙丑,黑氣蔽日。七年正月庚戌,白虹貫日。八年七月甲戌,白虹貫日,日有交暈。十月壬寅,白虹貫日,東西際天,上有背玦。九年二月辛卯,日月赤如血。壬辰,亦如之。



 開成元年正月辛丑朔,白虹貫日。二月己丑,亦如之。二年十一月辛巳,日中有黑子,大如雞卵,日赤如赭,晝昏至於癸未。五年正月己丑,日暈,白虹在東,如玉環貫珥。二月丙辰,日有重暈,有赤氣夾日。十二月癸卯朔,日旁有黑氣來觸。



 會昌元年十一月庚戌,日中有黑子。四年正月戊申,日無光。二月己巳,白虹貫日如玉環。



 大中十三年四月甲午,日暗無光。



 咸通六年正月,白虹貫日,中有黑氣如雞卵。七年十二月癸酉,白氣貫日,日有重暈。甲戌,亦如之。白氣,兵象也。十四年二月癸卯,白虹貫日。



 乾符元年,日中有黑子。二年,日中有若飛燕者。六年十一月丙辰朔,有兩日並出而鬥,三日乃不見。鬥者,離而復合也。



 廣明元年,日暈如虹,黃氣蔽日無光。日不可以二;虹,百殃之本也。



 中和三年三月丙午,日有青黃暈。四月丙辰,亦如之。丁巳、戊午,又如之。



 光啟三年十一月己亥,下晡,日上有黑氣。四年二月己丑,日赤如血。庚寅,改元文德。是日,風,日赤無光。



 景福元年五月,日色散如黃金。



 光化三年冬,日有虹蜺背矞彌旬,日有赤氣,自東北至於東南。



 天復元年十月,日色散如黃金。十一月,又如之。三年二月丁丑,日有赤氣,自東北至於東南。



 天祐元年二月丙寅,日中見北斗,其占重。十一月癸酉,日中,日有黃暈,旁有青赤氣二。二年正月甲申,日有黃白暈,暈上有青赤背。乙酉,亦如之,暈中生白虹,漸東,長百餘丈。二月乙巳,日有黃白暈如半環,有蒼黑雲夾日,長各六尺餘,既而雲變,狀如人如馬,乃消。舊占:「背者,叛背之象。日暈有虹者為大戰,半暈者相有謀,蒼黑,祲祥也。夾日者,賊臣制君之象。變而如人者為叛臣;如馬者為兵。」三年正月辛未,日有黃白暈,上有青赤背。二月癸巳,日有黃白暈,如半環,有青赤背。庚戌,日有黃白暈,青赤背。



 ○月變



 貞觀初,突厥有三月並見。



 儀鳳二年正月甲子朔,月見西方,是謂朓。朓則侯王其舒。



 武太后時,月過望不虧者二。



 天寶三載正月庚戌,月有紅氣如垂帶。



 肅宗元年建子月癸巳乙夜,月掩昴而暈,色白,有白氣自北貫之。昴,胡也;白氣,兵喪。建辰月丙戌,月有黃白冠,連暈,圍東井、五諸侯、兩河及輿鬼。東井,京師分也。



 大歷十年九月戊申,月暈熒惑、畢、昴、參,東及五車,暈中有黑氣,乍合乍散。十二月丙子,月出東方,上有白氣十餘道,如匹練,貫五車及畢、觜觿、參、東井、輿鬼、柳、軒轅,中夜散去。占曰:「女主兇。」白氣為兵喪,五車主庫兵,軒轅為後宮,其宿則晉分及京師也。



 元和十一年,己未旦,日已出,有虹貫月於營室。



 開成四年閏正月甲申朔,乙酉,月在營室,正偃魄質成,早也。占為臣下專恣之象。五年正月戊寅朔,甲申,月昏而中,未弦而中,早也。占同上。



 景福二年十一月,有白氣如環,貫月,穿北斗,連太微。



 天復二年十二月甲申,夜月有三暈,裹白,中赤黃,外綠。



 天祐二年二月丙申,月暈熒惑。



 ○孛彗



 武德九年二月壬午,有星孛於胃、昴間;丁亥,孛於卷舌。孛與彗皆非常惡氣所生,而災甚於彗。



 貞觀八年八月甲子,有星孛於虛、危,歷玄枵,乙亥不見。十三年三月乙丑,有星孛于畢、昴。十五年六月己酉有星孛於太微,犯郎位,七月甲戌不見。



 龍朔三年八月癸卯,有彗星於左攝提,長二尺餘,乙巳不見。攝提,建時節,大臣象。



 乾封二年四月丙辰,有彗星於東北,在五車、畢、昴間,乙亥不見。



 上元二年十二月壬午,有彗星於角、亢南,長五尺。三年七月丁亥,有彗星於東井,指北河,長三尺餘;東北行,光芒益盛,長三丈,掃中臺,指文昌。九月乙酉,不見。東井,京師分;中臺、文昌,將相位;兩河,天闕也。



 開耀元年九月丙申,有彗星於天市中,長五丈,漸小,東行至河鼓,癸丑不見。市者,貨食之所聚,以衣食生民者;一曰帝將遷都。河鼓,將軍象。



 永淳二年三月丙午,有彗星於五車北,四月辛未不見。



 文明元年七月辛未夕,有彗星於西方,長丈餘,八月甲辰不見。是謂天攙。



 光宅元年九月丁丑,有星如半月,見於西方。月,眾陰之長,星如月者陰盛之極。



 景龍元年十月壬午,有彗星於西方,十一月甲寅不見。二年七月丁酉,有星孛於胃、昴間。胡分也。三年八月壬辰,有星孛於紫宮。



 延和元年六月,有彗星自軒轅入太微,至大角滅。



 開元十八年六月甲子,有彗星於五車。癸酉,有星孛于畢、昴。二十六年三月丙子,有星孛於紫宮垣,歷北斗魁,旬餘,因云陰不見。



 乾元三年四月丁巳,有彗星於東方,在婁、胃間,色白,長四尺,東方疾行,歷昴、畢、觜觿、參、東井、輿鬼、柳、軒轅至右執法西,凡五旬餘不見。閏月辛酉朔,有彗星於西方,長數丈,至五月乃滅。婁為魯,胃、昴、畢為趙、觜觿、參為唐,東井,輿鬼為京師分,柳其半為周分。二彗仍見者,薦禍也。又婁、胃間,天倉。



 大歷元年十二月己亥,有彗星於匏瓜,長尺餘,經二旬不見,犯宦者星。五年四月己未,有彗星於五車,光芒蓬勃,長三丈。五月己卯,彗星見於北方,色白,癸未東行近八穀中星;六月癸卯近三公,己未不見。占曰:「色白者,太白所生也。」七年十二月丙寅,有長星於參下。其長亙天。長星,彗屬。參,唐星也。



 元和十年三月,有長星於太微,尾至軒轅。十二年正月戊子,有彗星於畢。



 長慶元年正月己未,有星孛于翼;丁卯,孛於太微西上將。六月,在彗星於昴,長一丈,凡十日不見。



 太和二年七月甲辰,有彗星於右攝提南,長二尺。三年十月,客星見於水位。八年九月辛亥,有彗星於太微,長丈餘,西北行,越郎位,庚申不見。



 開成二年二月丙午,有彗星於危,長七尺餘,西指南斗;戊申在危西南,芒耀愈盛;癸丑在虛;辛酉,長丈餘,西行稍南指;壬戌,在婺女,長二丈餘,廣三尺;癸亥,愈長且闊;三月甲子,在南斗;乙丑,長五丈,其末兩岐,一指氐,一掩房;丙寅,長六丈,無岐,北指,在亢七度;丁卯,西北行,東指;己巳,長八丈餘,在張;癸未,長三尺,在軒轅右不見。凡彗星晨出則西指,夕出則東指,乃常也。未有遍指四方,凌犯如此之甚者。甲申,客星出於東井下。戊子,客星別出於端門內,近屏星。四月丙午,東井下客星沒。五月癸酉,端門內客星沒。壬午,客星如孛,在南斗天龠旁。八月丁酉,有彗星於虛、危,虛、危為玄枵。枵,耗名也。三年十月乙巳,有彗星於軫魁,長二丈餘,漸長,西指。十一月乙卯,有彗星於東方,在尾、箕,東西亙天;十二月壬辰不見。四年正月癸酉,有彗星於羽林。衛分也。閏月丙午,有彗星於卷舌西北;二月己卯不見。五年二月庚申,有彗星於營室、東壁間,二十日滅。十一月戊寅,有彗星於東方。燕分也。



 會昌元年七月,有彗星於羽林、營室、東壁間也。十一月壬寅,有彗星於北落師門,在營室,入紫宮,十二月辛卯不見。並州分也。



 大中六年三月,有彗星於觜、參。參,唐星也。十一月年九月乙未,有彗星於房,長三尺。



 咸通五年五月己亥,夜漏未盡一刻,有彗星出於東北,色黃白,長三尺,在婁。徐州分也。九年正月,有彗星於婁、胃。十年八月,有彗星於大陵,東北指。占為外夷兵及水災。



 乾符四年五月,有彗星。



 光啟元年,有彗星於積水、積薪之間。二年五月丙戌,有星孛於尾、箕,歷北斗、攝提。占曰:「貴臣誅。」



 大順二年四月庚辰,有彗星於三臺,東行入太微,掃大角、天市,長十丈餘,五月甲戌不見。宦者陳匡知星,奏曰:「當有亂臣入宮。」三臺,太一三階也;太微大角,帝廷也;天市,都市也。



 景福元年五月,蚩尤旗見,初出有白彗,形如發,長二尺許,經數日,乃從中天下,如匹布,至地如蛇。六月,孫儒攻楊行密於宣州,有黑雲如山,漸下,墜於儒營上,狀如破屋,占曰:「營頭星也。」十一月,有星孛於斗、牛。占曰;「越有自立者。」十二月丙子,天攙出於西南;己卯,化為雲而沒。二年三月,天久陰,至四月乙酉夜,雲稍開,有彗星於上臺,長十餘丈,東行入太微,掃大角,入天市,經三旬有七日,益長,至二十餘丈,因云陰不見。



 乾寧元年正月,有星孛于鶉首。秦分也。又星隕於西南,有聲如雷。七月,妖星見,非彗非孛,不知其名,時人謂之妖星,或曰惡星。三年十月,有客星三,一大二小,在虛、危間,乍合乍離,相隨東行,狀如斗,經三日而二小星沒,其大星後沒。虛、危,齊分也。



 光化三年正月,客星出於中垣宦者旁,大如桃,光炎射宦者,宦者不見。



 天復元年五月,有三赤星,各有鋒芒,在南方,既而西方、北方、東方亦如之,頃之,又各增一星,凡十六星;少時,先從北滅。占曰:「濛星也,見則諸侯兵相攻。」二年正月,客星如桃,在紫宮華蓋下,漸行至御女。丁卯,有流星起文昌,抵客星,客星不動;己巳,客星在杠,守之,至明年猶不去。占曰:「將相出兵。」五月夕,有星當箕下,如炬火,炎炎上沖,人初以為燒火也,高丈餘乃隕。占曰:「機星也,下有亂。」



 天祐元年四月,有星狀如人,首赤身黑,在北斗下紫微中。占曰:「天沖也。天沖抱極泣帝前,血濁霧下天下冤。」後三日而黑風晦暝。二年四月庚子夕,西北隅有星類太白,上有光似彗,長三四丈,色如赭;辛丑夕,色如縞。或曰五車之水星也,一曰昭明星也。甲辰,有彗星於北河,貫文昌,長三丈餘,陵中臺、下臺;五月乙丑夜,自軒轅左角及天市西坦,光芒猛怒,其長亙天;丙寅雲陰,至辛未少霽,不見。兩河為天闕,在東井間,而北河,中國所經也。文昌,天之六司。天市,都市也。



 ○星變



 武德三年十月己未,有星隕於東都中,隱隱有聲。



 貞觀二年,天狗隕於夏州城中。十四年八月,有星隕於高昌城中。十六年六月甲辰,西方有流星如月,西南行三丈乃滅。占曰:「星甚大者,為人主。」十八年五月,流星出東壁,有聲如雷。占曰:「聲如雷者,怒象。」十九年四月己酉,有流星向北斗杓而滅。



 永徽三年十月,有流星貫北極。四年十月,睦州女子陳碩真反,婺州刺史崔義玄討之,有星隕於賊營。



 乾封元年正月癸酉,有星出太徽,東流,有聲如雷。



 咸亨元年十一月,西方有流星,聲如雷。



 調露元年十一月戊寅,流星入北斗魁中;乙巳,流星燭地有光,使星也。



 神龍三年三月丙辰,有流星聲如頹墻,光燭天地。



 景龍二年二月癸未,有大星隕於西南,聲如雷,野雉皆雊。



 景雲元年八月己未,有流星出五車,至上臺滅。九月甲申,有流星出中臺,至相滅。



 太極元年正月辛卯,有流星出太微,至相滅。



 延和元年六月,幽州都督孫佺討奚、契丹,出師之夕,有大星隕於營中。



 開元二年五月乙卯晦,有星西北流,或如甕,或如斗,貫北極,小者不可勝數,天星盡搖,至曙乃止。占曰:「星,民象;流者,失其所也。」《漢書》曰:「星搖者民勞。」十二年十月壬辰,流星大如桃,色赤黃,有光燭地。占曰:「色赤為將軍使。」



 天玉三載閏二月辛亥,有星如月,墜於東南,墜後有聲。



 至德二載,賊將武令珣圍南陽,四月甲辰夜中,有大星赤黃色,長數十丈,光燭地,墜賊營中。十一月壬戌,有流星大如斗,東北流,長數丈,蛇行屈曲,有碎光迸出。占曰:「是謂枉矢。」



 廣德二年六月丁卯,有妖星隕於汾州。十二月丙寅,自乙夜至曙,星流如雨。



 大歷二年九月乙丑,晝有星如一斗器,色黃,有尾長六丈餘,出南方,沒於東北。東北于中國,則幽州分也。三年九月乙亥,有星大如斗,北流,有光燭地,占為貴使。六年九月甲辰,有星西流,大如一斗器,光燭地,有尾,迸光如珠,長五丈,出婺女,入天市南垣滅。八年六月戊辰,有流星大如一升器,有尾,長三丈餘,入太微。十二月壬申,有流星大如一升器,有尾長二丈餘,出紫微入濁。十年三月戊戌,有流星出於西方,如二升器,有尾,長二丈,入濁。十二年二月辛亥,有流星如桃,尾長十丈,出匏瓜,入太微。



 建中四年八月庚申,有星隕於京師。



 興元元年六月戊午,星或什或伍而隕。



 貞元三年閏五月戊寅,枉矢墜於虛、危。十四年閏五月辛亥,有星墜於東北,光燭如晝,聲如雷。



 元和二年十二月己巳,西北有流星亙天,尾散如珠。占曰:「有貴使。」四年八月丁丑,西北有大星,東南流,聲如雷鼓。六年三月戊戌日晡,天陰寒,有漢星大如一斛器,墜於袞、鄆間,聲震數百里,野雉皆雊,所墜之上,有赤氣如立蛇,長丈餘,至夕乃滅。時占者以為日在戌,魯分也,不及十年,其野主殺而地分。九年正月有大星如半席,自下而升,有光燭地,群小星隨之。四月辛巳,有大流星,尾跡長五丈餘,光燭地,至右攝提西滅。十二年九月己亥甲夜,有流星起中天,首如甕,尾如二百斛舡,長十餘丈,聲如群鴨飛,明若火炬,過月下西流,須臾,有聲礱礱,墜地,有大聲如壞屋者三,在陳、蔡間。十四年五月己亥,有大流星出北半魁,長二丈餘,南抵軒轅而滅。占曰:「有赦,赦視星之大小。」十五年七月癸亥,有大星出鉤陳,南流至婁滅。



 長慶元年正月丙辰,有大星出狼星北,色赤,有尾跡,長三丈餘,光燭地,東北流至七星南滅。四月,有大星墜於吳,聲如飛羽。十月乙巳,有大流星出參西北,色黃,有尾跡,長六七丈,光燭地,至羽林滅。八月辛巳,東北方有大星自雲中出,色白,光燭地,前銳後大,長二丈餘,西北流入雲中滅。二年四月辛亥,有流星出天市,光燭地,隱隱有聲,至郎位滅。市者,小人所聚,郎在天廷中,主宿衛。六月丁酉,有小星隕於房、心間,戊戌亦如之,己亥亦如之。閏十月丙申,有流星大如斗,抵中臺上星。三年八月丁酉夜,有大流星如數斗器,起西北,經奎、婁,東南流,去月甚近,迸光散落,墜地有聲。四年四月,紫微中,星隕者眾。七月乙卯,有大流星出天船,犯斗魁樞星而滅。占曰:「有舟楫事。」丙子,有大流星出天將軍東北,入濁。



 寶歷元年正月乙卯,有流星出北斗樞星,光燭地,入濁。占曰:「有赦。」二年五月癸巳,西北有流星,長三丈餘,光燭地,入天市中滅。占為有誅。七月丙戌,日初入,東南有流星,向南,滅,以晷度推之,在箕、鬥間。八月丙申,有大流星出王良,長四丈餘,至北斗杓滅。王良,奉車御官也。



 大和四年六月辛未,自昏至戊夜,流星或大或小,觀者不能數。占曰:「民失其所,王者失道,綱紀廢則然。」又曰:「星在野象物,在朝象官。」七年六月戊子,自昏及曙,四方流星,大小縱橫百餘。八年六月辛巳,夜中有流星出河鼓,赤色,有尾跡,光燭地,迸如散珠,北行近天咅滅,有聲如雷。河鼓為將軍。天咅者,帝之武備。九年六月丁酉,自昏至丁夜,流星二十餘,縱橫出沒,多近天漢。



 開成二年九月丁酉,有星大如斗,長五丈,自室、壁西北流,入大角下沒,行類枉矢,中天有聲,小星數百隨之。十一月丁丑,有大星隕於興元府署寢室之上,光燭庭宇。三年五月乙丑,有大星出於柳、張,尾長五丈餘,再出再沒。四年二月己亥,丁夜至戊夜,四方中天流星小大凡二百餘,並西流,有尾跡,長二丈至五丈。八月辛未,流星出羽林,有尾跡,長八丈餘,有聲如雷。羽林,天軍也。十二月壬申,蚩尤旗見。



 會昌元年六月戊辰,自昏至戊夜,小星數十,縱橫流散。占曰;「小星,民象。」七月庚午,北方有星,光燭地,東北流經王良,有聲如雷。十一月壬寅,有大星東北流,光燭地,有聲如雷。四年八月丙午,有大星如炬火,光燭天地,自奎、婁掃西方七宿而隕。六年二月辛丑,夜中有流星赤色如桃,光燭地,有尾跡,貫紫微入濁。



 咸通六年七月乙酉,甲夜有大流星長數丈,光爍如電,群小星隨之,自南徂北。其象南方有以眾叛而之北也。九年十一月丁酉,有星出如匹練,亙空化為雲而沒,在楚分。是謂長庚,見則兵起。十三年春,有二星從天際而上,相從至中天,狀如旌旗,乃隕。九月,蚩尤旗見。



 乾符二年冬,有二星,一赤一白,大如斗,相隨東南流,燭地如月,漸大,光芒猛怒。三年,晝有星如炬火,大如五升器,出東北,徐行,隕於西北。四年七月,有大流星如盂,自虛、危,歷天市,入羽林滅。占為外兵。



 中和元年,有異星出於輿鬼,占者以為惡星。八月己丑夜,星隕如雨,或如杯碗者,交流如織,庚寅夜亦如之,至丁酉止。三年十一月夜,星隕於西北,如雨。



 光啟二年九月,有大星隕於揚州府署延和閣前,聲如雷,光炎燭地。十月壬戌,有星出於西方,色白,長一丈五尺,屈曲而隕。占曰:「長庚也,下則流血。」三年五月,秦宗權擁兵於汴州北郊,晝有大星隕於其營,聲如雷,是謂營頭。其下破軍殺將。



 乾寧元年夏,有星隕於越州,後有光,長丈餘,狀如蛇。或曰枉矢也。三年六月,天暴雨,雷電,有星大如碗,起西南,墜於東北,色如鶴練,聲如群鴨飛。占為奸謀。



 光化元年九月丙子,有大星墜於北方。三年三月丙午,有星如二十斛船,色黃,前銳後大,西南行。十一月,中天有大星自東緩流,如帶屈曲,光凝著天,食頃乃滅。是謂枉矢。



 天復三年二月,帝至自鳳翔,其明日,有大星如月,自東濁際西流,有聲如雷,尾跡橫貫中天,三夕乃滅。



 天祐元年五月戊寅乙夜,雨、晦暝,有星長二十丈,出東方,西南向,首黑、尾赤、中白,枉矢也,一曰長星。二年三月乙丑,夜中有大星出中天,如五斗器,流至西北,去地十丈許而止,上有星芒,炎如火,赤而黃,長丈五許,而蛇行,小星皆動而東南,其隕如雨,少頃沒,後有蒼白氣如竹叢,上沖天中,色瞢瞢。占曰:「亦枉矢也。」三年十二月昏,東方有星如太白,自地徐上,行極緩,至中天,如上弦月,乃曲行,頃之,分為二。占曰;「有大孽。」



\end{pinyinscope}