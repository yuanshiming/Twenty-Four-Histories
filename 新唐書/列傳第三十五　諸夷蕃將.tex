\article{列傳第三十五 諸夷蕃將}

\begin{pinyinscope}

 史大柰,本西突厥特勒也,與處羅可汗入隋,事煬帝。從伐遼,積勞為金紫光祿大夫。後分其部於樓煩。



 高祖興太原,大柰提其眾隸麾下。桑顯和戰飲馬泉,諸軍卻,大柰以勁騎數百背擊顯和,破之,軍遂振。授光祿大夫。從平長安,以多,賞帛五千匹,賜姓史。從秦王平薛舉、王世充、竇建德、劉黑闥,功殊等,積前後賜侍女三、雜彩萬段。貞觀初,擢累右武衛大將軍,檢校豐州都督,封竇國公,食封戶三百。卒,贈輔國大將軍。



 馮盎,字明達,高州良德人,本北燕馮弘裔孫。弘不能以國下魏,亡奔高麗,遣子業以三百人浮海歸晉。弘已滅,業留番禺,至孫融,事梁為羅州刺史。子寶,聘越大姓洗氏女為妻,遂為首領,授本郡太守,至盎三世矣。



 隋仁壽初,盎為宋康令,潮、成等五州獠叛,盎馳至京師,請討之。文帝詔左僕射楊素與論賊形勢,素奇之,曰:「不意蠻夷中乃生是人!」即詔盎發江、嶺兵擊賊,平之,拜漢陽太守。從煬帝伐遼東,遷左武衛大將軍。隋亡,奔還嶺表,嘯署酋領,有眾五萬。番禺、新興名賊高法澄、洗寶徹等受林士弘節度,殺官吏,盎率兵破之。寶徹兄子曰智臣,復聚兵拒戰,盎進討,兵始合,輒釋胄大呼曰:「若等識我耶?」眾委戈,袒而拜,賊遂潰,擒寶徹、智臣等,遂有番禺、蒼梧、硃崖地,自號總管。或說盎曰:「隋季崩蕩,海內震騷,唐雖應運,而風教未孚,嶺越無所系屬。公克平二十州,地數千里,名位未正,請南越王號。」盎曰:吾居越五世矣,牧伯惟我一姓,子女玉帛吾有也,人生富貴,如我希矣。常恐忝先業,尚自王哉?」



 武德五年,始以地降,高祖析為高、羅、春、白、崖、儋、林、振八州,授盎上柱國、高州總管,封越國公。拜其子智戴為春州刺史,智彧為東合州刺史。盎徙封耿。貞觀初,或告盎叛,盎舉兵拒境。太宗詔右武衛將軍藺暮發江淮甲卒將討之,魏徵諫曰:「天下初定,創夷未復,大兵之餘,疫癘方作,且王者兵不宜為蠻夷動,勝之不武,不勝為辱。且盎不及未定時略州縣,搖遠夷,今四海已平,尚何事?反未狀,當懷之以德,盎懼,必自來。」帝乃遣散騎常侍韋叔諧喻盎,盎遣智戴入侍。帝曰:「征一言,賢於十萬眾。」時暮兵已出,欲遂有功,遣副將上盎可擊狀,帝不許,罷之。



 五年,盎來朝,宴賜甚厚。俄而羅、竇諸洞獠叛,詔盎率眾二萬為諸軍先鋒。賊據險不可攻,盎持弩語左右曰:「矢盡,勝負可知矣。」發七矢斃七人,賊退走,盎縱兵乘之,斬首千餘級。帝詔智戴還慰省,賞予不可計,奴婢至萬人。盎善為治,閱簿最,擿奸伏,得民歡心。卒,贈左驍衛大將軍、荊州都督。



 子三十人,智戴知名,勇而有謀,能撫眾,得士死力,酋師皆樂屬之。嘗隨父至洛陽,統本部銳兵宿衛。煬帝弒,引其下逃歸。時盜賊多,嶺嶠路絕,智戴轉戰而前。至高源,俚帥脅為謀主,會盎至,智戴得與盎俱去。後入朝,帝勞賜加等,授衛尉少卿。聞其善兵,指雲問曰:「下有賊,今可擊乎?」對曰:「雲狀如樹,方辰在金,金利木柔,擊之勝。」帝奇其對。累遷左武衛將軍。卒,贈洪州都督。



 盎族人子猷,以豪俠聞。貞觀中,入朝,載金一舸自隨。高宗時,遣御史許瓘視其貲。瓘至洞,子猷不出迎,後率子弟數十人,擊銅鼓、蒙排,執瓘而奏其罪。帝馳遣御史楊璟驗訊。璟至,卑辭以結之,委罪於瓘。子猷喜,遺金二百兩、銀五百兩。璟不受。子猷曰:「君不取此,且留不得歸。」璟受之,還奏其狀,帝命納焉。



 阿史那社爾,突厥處羅可汗之次子。年十一,以智勇聞。拜拓設,建牙磧北,與頡利子欲谷設分統鐵勒、回紇、僕骨、同羅諸部。處羅卒,哀毀如禮。治眾十年,無課斂。或勸厚賦以自奉,答曰:「部落豐餘,於我足矣。」故首領咸愛之。頡利數用兵,社爾諫,弗納。



 貞觀元年,鐵勒、回紇、薛延陀等叛,敗欲谷設於馬獵山,社爾助擊之,弗勝。明年,將餘眾西保可汗浮圖城。會頡利滅,西突厥統葉護又死,奚利必咄陸可汗與泥孰爭國,社爾引兵襲之,得其半國,有眾十餘萬,乃自號都布可汗。謂諸部曰:「始為亂破吾國者,延陀也,今我據西方,而不平延陀,是忘先可汗,非孝也。事脫不勝,死無恨。」酋長皆曰:「我新得西方,須留撫定。今直棄之,遠擊延陀,延陀未擒,葉護子孫將復吾國。」社爾不從,選騎五萬,討延陀磧北,連兵十旬,士苦其久,稍潰去。延陀縱擊,大敗之,乃走保高昌,眾才萬人,又與西突厥不平,由是率眾內屬。



 十年入朝,授左驍衛大將軍,處其部於靈州。詔尚衡陽長公主,為駙馬都尉,典衛屯兵。十四年,以交河道行軍總管平高昌,諸將咸受賞,社爾以未奉詔,秋毫不敢取,見別詔,然後受,又所取皆老弱陳弊。太宗美其廉,賜高昌寶鈿刀、雜彩千段,詔檢校北門左屯營,封畢國公。從征遼東,中流矢,揠去復戰,所部奮厲,皆有功。還,擢兼鴻臚卿。



 二十一年,以昆丘道行軍大總管與契爾何力、郭孝恪、楊弘禮、李海岸等五將軍發鐵勒十三部及突厥騎十萬討龜茲。師次西突厥,擊處蜜、處月,敗之。入自焉耆西,兵出不意,龜茲震恐。進屯磧石,伊州刺史韓威以千騎先進,右驍衛將軍曹繼叔次之。至多褐城,其王率眾五萬拒戰。威陽卻,王悉兵逐北,威與繼叔合,殊死戰,大破之。社爾因拔都城,王輕騎遁。社爾留孝恪守,自率精騎追躡,行六百里。王據大撥換城,嬰險自固。社爾攻凡四十日,入之,擒其王,並下五大城。遣左衛郎將權祗甫徇諸酋長,示禍福,降者七十餘城,宣諭威信,莫不歡服。刻石紀功而還。因說於闐王入朝,王獻馬畜三百餉軍,西突厥、焉耆、安國皆爭犒師。孝恪之在軍,床帷器用多飾金玉,以遺社爾,社爾不受。帝聞,曰:「二將優劣,不復問人矣。」帝崩,請以身殉,衛陵寢,高宗不許。遷右衛大將軍。永徽六年卒,贈輔國大將軍、並州都督,陪葬昭陵,治塚象蔥山,謚曰元。



 子道真,歷左屯衛大將軍。咸亨初,為邏娑道副大總管,與薛仁貴討吐蕃以援吐谷渾,為論欽陵所敗,盡失其兵。詔有司問狀,免死為民。



 阿史那忠者,字義節,蘇尼失子也。資清謹。以功擢左屯衛將軍,尚宗室女定襄縣主,始詔姓獨著史。居父喪,哀慕過人。會立阿史那思摩為突厥可汗,以忠為左賢王。及出塞,不樂,見使者必泣,請入侍,許焉。封薛國公,擢右驍衛大將軍。宿衛四十八年,無纖隙,人比之金日磾卒,贈鎮軍大將軍,謚曰貞,陪葬昭陵。



 執失思力,突厥酋長也。貞觀中,護送隋蕭後入朝,授左領軍將軍。會頡利敗,太宗令思力諭降渾、斛薩部落,稍親近。帝逐兔苑中,思力諫曰:「陛下為四海父母,乃自輕,臣竊殆之。」帝異其言。後復逐鹿,思力脫巾帶固諫,帝為止。



 及討遼東,詔思力屯金山道,領突厥捍薛延陀。延陀兵十萬寇河南,思力示羸,不與確,賊深入至夏州,乃整陣擊敗之,追躡六百里。會毘伽可汗死,耀兵磧北而歸。復從江夏王道宗破延陀餘眾。與平吐谷渾。



 詔尚九江公主,拜駙馬都尉,封安國公。坐交房遺愛,高宗以其戰多,赦不誅,流巂州。主請削封邑偕往。主前卒。龍朔中,以思力為歸州刺史,卒。麟德元年,復公主封邑,贈思力勝州都督,謚曰景。



 契苾何力,鐵勒哥論易勿施莫賀可汗之孫。父葛,隋末為莫賀咄特勒,以地近吐谷渾,隘翾多癘曷,徙去熱海上。何力九歲而孤,號大俟利發。



 貞觀六年,與母率眾千餘詣沙州內屬,太宗處其部於甘、涼二州,擢何力左領軍將軍。九年,與李大亮、薛萬徹、萬均討吐谷渾於赤水川。萬均率騎先進,為賊所包,兄弟皆中創墮馬,步斗,士死十七八。何力馳壯騎,冒圍奮擊,虜披靡去。是時吐谷渾王伏允在突淪川,何力欲襲之,萬均懲前敗,以為不可。何力曰:「賊無城郭,逐薦草美水以為生,不乘其不虞,正恐鳥驚魚駭,後無以窺其巢穴。」乃閱精騎千餘,直搗其牙,斬首數千級,獲橐駝、馬、牛、羊二十餘萬,俘其妻子,伏允挺身免。有詔勞軍於大斗拔谷。萬均恥名出其下,乃排何力,引功自名。何力不勝憤,挺刀起,將殺之,諸將勸止。



 及還,帝責謂其故,何力具言萬均敗狀。帝怒,將解其官授何力。何力頓首曰:「以臣而解萬均官,恐四夷聞者,謂陛下重夷輕漢,則誣告益多。又夷狄無知,謂漢將皆然,非示遠之義。」帝重其言,乃止。有詔宿衛北門,檢校屯營事,尚臨洮縣主。十四年,為蔥山道副大總管,與討高昌,平之。



 始,何力母姑臧夫人與弟沙門在涼州,沙門為賀蘭都督。十六年,詔何力往視母。於是薛延陀毘伽可汗方強,契苾諸酋爭附之,乃脅其母、弟使從。何力驚謂其下曰:「上於爾有大恩,且遇我厚,何遽反?」皆曰:「可敦、都督去矣,尚何顧?」何力曰:「弟往侍足矣,我義許國,不可行。」眾執之,至毘伽牙下。何力箕踞,拔佩刀東向呼曰:「有唐烈士受辱賊延邪?天地日月,臨鑒吾志。」即割左耳,誓不屈。毘伽怒,欲殺之,其妻諫而止。何力被執也,或讒之帝曰:「何力入延陀如涸魚得水,其脫必遽。」帝曰:「不然。若人心如鐵石,殆不背我。」會使至言狀,帝泣下。即詔兵部侍郎崔敦禮持節許延陀尚主,因求何力,乃得還。授右驍衛大將軍。公主行有日,何力陳不可。帝曰:「天子無戲言,既許之,叵奈何?」何力曰:「禮有親迎,宣詔毘伽身到京師,或詣靈武。彼畏我,必不來,則姻不成,而憂憤不知所出,下必攜貳,不及一年,交相疑沮。毘伽素狼戾,必死,死則二子爭國。內判外攜,不戰而擒矣。」帝然之。毘伽果不敢迎,鬱邑不得志,恚而死,少子拔酌殺其庶兄突利失自立,國中亂,如其策云。



 帝征高麗,詔何力為前軍總管。次白崖城,中賊槊,創甚,帝自為傅藥。城拔,得刺何力者高突勃,騶使自殺之,辭曰:「彼為其主,冒白刃以刺臣,此義士也。犬馬猶報其養,況於人乎?」卒舍之。俄以昆丘道總管平龜茲。帝崩,欲以身殉,高宗諭止。



 永徽中,西突厥阿史那賀魯以處月、處蜜、姑蘇、歌邏祿、卑失五姓叛,寇延州,陷金嶺略蒲類,詔何力為弓月道大總管,率左武衛大將軍梁建方,統秦、成、岐、雍及燕然都護回紇兵人萬討之。處月酋硃邪孤注遂殺招慰使果毅都尉單道惠,據牢山以守。何力等分兵數道,攀F〗反而上,急攻之,賊大潰,孤注液遁。輕騎窮躡,行五百里,孤注戰死。虜渠帥六十,俘斬萬餘,牛馬雜畜七萬,取處蜜時健俟斤、合支賀等以歸。遷左驍衛大將軍,封郕國公。



 顯慶中,為沮江軍行軍大總管,與蘇定方及右驍衛大將軍劉伯英代高麗。,不克。龍朔初,復拜遼東道行軍大總管,率諸蕃三十五軍進討,帝欲自率師繼之。次鴨綠水,蓋蘇文遣男生以精兵數萬拒險,眾莫敢濟。會冰合,何力引兵噪而濟,賊驚,遂潰。斬首三萬級,餘眾降,男生脫身走。有詔斑師。



 時鐵勒九姓叛,詔何力為安撫大使。何力以輕騎五百馳入其部,虜大驚。何力喻曰:「朝家知而詿誤,遂及翻動,使我貰爾過,得自新。罪在兇渠,取之則已。」九姓大喜,共擒偽葉護及特勒等二百人以歸,何力數其罪,誅之,餘眾遂安。士卒道死者,令所在收瘞,蠲護其家。



 未幾,蓋蘇文死,男生為弟所逐,使子詣闕請降,乃拜何力為遼東道行軍大總管、安撫大使經略之,副李勣同趨高麗勣已拔新城,留何力守。時高麗兵十五萬屯遼水,引靺鞨數萬眾據南蘇城,何力奮擊,破之,斬首萬級,乘勝進拔八城。引兵還,與勣會合,攻辱夷、大行二城,克之。進拔扶餘。勣勒兵未進,何力率兵五十萬先趨平壤,勣繼進,攻凡七月,拔之,虜其王以獻。進鎮軍大將軍,行左衛大將軍,徙封涼。



 總章、儀鳳間,吐蕃滅吐谷渾,勢益張,入寇鄯、廓、河、坊等州,詔周王為洮州道、相王為涼州道行軍元帥,率何力等討之。二王不行,亦會何力卒。贈輔國大將軍、並州大都督,陪葬昭陵,謚曰毅。



 始,龍朔中,司稼少卿梁脩仁新作大明宮,植白楊於庭,示何力曰:「此木易成,不數年可庇。」何力不答,但誦「白楊多悲風,蕭蕭愁殺人」之句,脩仁驚悟,更植以桐。



 子明,字若水,孺褓授上柱國,封漁陽縣公。年十二,遷奉輦大夫。李敬玄征吐蕃,明為柏海道經略使,以戰多,進左威衛大將軍,襲封,賜錦袍、寶帶,它物蕃夥。擢嫡子三品官。再遷雞田道大總管,至烏德鞬山,誘附二萬帳。武后時,明妻及母臨洮縣主皆賜姓武。以左鷹揚衛大將軍卒,年四十六,贈涼州刺史,謚曰靖。



 明性淹厚,喜學,長辯論。子聳,襲爵。



 黑齒常之,百濟西部人。長七尺餘,驍毅有謀略。為百濟達率兼風達郡將,猶唐刺史云。蘇定方平百濟,常之以所部降。而定方囚老王,縱兵大掠,常之懼,與左右酋長十餘人遁去,嘯合逋亡,依任存山自固,不旬日,歸者三萬。定方勒兵攻之,不克,常之遂復二百餘城。龍朔中,高宗遣使招諭,乃詣劉仁軌降。累遷左領軍員外將軍、洋州刺史。



 儀鳳三年,從李敬玄、劉審禮擊吐蕃。審禮敗,敬玄欲引還,阻泥溝,兵不得出,賊屯高壓官軍。常之夜率敢死士五百人掩其營,殺掠數百人,賊酋跋地設棄軍走。帝嘆其才,擢左武衛將軍,檢校左羽林軍,賜金帛殊等。進為河源軍副使。調露中,吐蕃使贊婆等入寇,屯良非川。李敬玄之敗,常之引精騎三千夜襲其軍,斬首二千級,獲羊馬數萬,贊婆等單騎去。即拜河源道經略大使。因建言河源當賊沖,宜增兵鎮守,而運饟須廣。乃斥地置烽七十所,墾田五千頃,歲收粟斛百餘萬。由是食衍士精,戍邏有備。永隆二年,贊婆營青海,常之馳掩其屯,破之,悉燒糧廥,獲羊、馬、甲首不貲。詔書勞賜。凡蒞軍七年,吐蕃簷畏,不敢盜邊。封燕國公。



 垂拱中,突厥復犯塞,常之率兵追擊,至兩井,忽與賊遇,賊騎三千方擐甲,常之見其囂,以二百騎突之,賊皆棄甲去。其暮,賊大至,常之潛使人伐木,列炬營中,若烽燧然。會風起,賊疑救至,遂夜遁。久之,為燕然道大總管,與李多祚王九言等擊突厥骨咄祿、元珍於黃花堆,破之,追奔四十里,賊潰歸磧北。會左監門衛中郎將爨寶璧欲窮追要功,詔與常之共計,寶壁獨進,為虜所覆,舉軍沒,寶璧下吏誅,常之坐無功。會周興等誣其與右鷹揚將軍趙懷節反,捕系詔獄,投繯死。



 常之御下有恩,所乘馬為士所蜺,或請罪之。答曰:「何遽以私馬鞭官兵乎?」前後賞賜分麾下,無留貲。及死,人皆哀其枉。



 李謹行,靺鞨人。父突地稽,部酋長也。隋末,率其屬千餘內附,居營州,授金紫光祿大夫、遼西太守。武德初,奉朝貢,以其部為燕州,授總管。劉黑闥叛,突地稽身到定州,上書秦王,請節度。以戰功封耆國公,徙部居昌平。高開道以突厥兵攻幽州,突地稽邀擊,敗之。貞觀初,進右衛將軍,賜氏李,卒。



 謹行偉容貌,勇蓋軍中,累遷營州都督,家童至數千,以財自雄,夷人畏之。為積石道經略大使,論欽陵眾十萬寇湟中,候邏不知,士樵採半散。謹行聞虜至,即植旗伐鼓,開門以伺。欽陵疑有伏,不敢進。上元三年,破吐蕃於青海,璽書勞勉,封燕國公。卒,贈幽州都督,陪葬乾陵。



 泉男生,字元德,高麗蓋蘇文子也。九歲,以父任為先人。遷中里小兄,猶唐謁者也。又為中里大兄,知國政,凡辭令,皆男生主之。進中里位頭大兄。久之,為莫離支,兼三軍大將軍,加大莫離支,出按諸部。而弟男建、男產知國事,或曰:「男生惡君等逼己,將除之。」建、產未之信。又有謂男生:「將不納君。」男生遣諜往,男建捕得,即矯高藏命召,男生懼,不敢入。男建殺其子獻忠。男生走保國內城,率其眾與契丹、靺鞨兵內附,遣子獻誠訴諸朝。高宗拜獻誠右武衛將軍,賜乘輿、馬、瑞錦、寶刀,使還報。詔契苾何力率兵援之,男生乃免。授平壤道行軍大總管,兼持節安撫大使,舉哥勿、南蘇、旨巖等城以降。帝又命西臺舍人李虔繹就軍慰勞,賜袍帶、金扣七事。



 明年,召入朝,詔所過州縣傳舍作鼓吹,右羽林將軍李同以飛騎仗廷寵。遷遼東大都督、玄菟郡公,賜第京師。因詔還軍,與李勣攻平壤,使浮屠信誠內間,引高麗銳兵潛入,擒高藏。詔遣子齎手制、金皿,即遼水勞賜。還,進右衛大將軍、卞國公,賜寶器、宮侍女二、馬八十。儀鳳二年,詔安撫遼東,並置州縣,招流冗,平斂賦,罷力役,民悅其寬。卒,年四十六,帝為舉哀,贈並州大都督。喪至都,詔五品以上官哭之,謚曰襄,勒碑著功。



 男生純厚有禮,奏對敏辯,善射藝。其初至,伏斧鑕待罪,帝宥之,世以此稱焉。



 獻誠,天授中以右衛大將軍兼羽林衛。武后嘗出金幣,命宰相、南北牙群臣舉善射五輩,中者以賜。內史張光輔舉獻誠,獻誠讓右玉鈐衛大將軍薛吐摩支,摩支固辭。獻成曰:「陛下擇善射者,然皆非華人。臣恐唐官以射為恥,不如罷之。」後嘉納。來俊臣嘗求貨,獻誠不答,乃誣其謀反,縊殺之。後後知其冤,贈右羽林衛大將軍,以禮改葬。



 李多祚,其先靺鞨酋長,號「黃頭都督」,後入中國,世系湮遠。至多祚,驍勇善射,以軍功累遷右鷹揚大將軍。討黑水靺鞨,誘其渠長,置酒高會,因醉斬之,擊破其眾。室韋及孫萬榮之叛,多祚與諸將進討,以勞改右羽林大將軍,遂領北門衛兵。



 張柬之將誅二張,以多祚素感概,可動以義,乃從容謂曰:「將軍居北門幾何?」曰:「三十年矣。」「將軍擊鐘鼎食,貴重當世,非大帝恩乎?」多祚泣數行下,曰:「死且不忘!」柬之曰:「將軍知感恩,則知所以報,今在東宮乃大帝子,而嬖豎擅朝,危逼宗社。國家廢興在將軍,將軍誠有意乎?舍今日尚何在?」答曰:「茍緣王室,惟公所使。乃引天地以自誓,辭氣毅然,柬之遂定謀。以敬暉、李湛為右羽林將軍,命總禁兵,與多祚、王同皎請太子至玄武門,斬關入。及長生殿,白武后曰:「諸將誅逆臣易之、昌宗,恐漏大謀,不敢豫奏,頓首請歸死。」後病臥,顧湛曰:「我於而父子不薄,亦豫是邪?」



 中宗復位,封多祚遼陽郡王,食實戶八百,子承訓為衛尉少卿。湛遷大將軍,封趙國公,食實戶五百。帝祠太廟,特詔多祚與相王登輿夾侍。監察御史王覿謂多祚夷人,雖有功,不宜共輿輦。帝曰:「朕推以心腹,卿勿復言。」



 崔玄等得罪,多祚畏禍及,故陽厚韋氏。節愍太子誅武三思,多祚與成王千里率兵先至玄武樓下,具言所以誅三思狀,按兵不戰。宮闈令楊思勖方侍帝,即挺刀斬其婿羽林中郎將野呼利,兵因沮潰,多祚為其下所殺,二子亦見害,籍沒其家。景雲初,追復官爵,並宥家屬。



 湛者,義府最幼子,字興宗,沉厚有度。六歲,授周王府文學,累遷右散騎常侍,襲河間郡公。武後徙上陽宮,留湛宿衛。頃之,復為右散騎常侍,賜鐵券。三思惡之,貶果州刺史。歷洺、絳二州,累遷左領軍大將軍。開元十年卒,贈幽州都督。初,義府以立武后故得宰相,而湛為中興功臣,世不以其父惡為貶云。



 論弓仁,本吐蕃族也。父欽陵,世相其國。聖歷二年,弓仁以所統吐渾七千帳自歸,授左玉鈐衛將軍,封酒泉郡公。神龍三年,為朔方軍前鋒游弈使。時張仁願築三受降城,弓仁以兵出諾真水、草心山為邏衛。



 開元初,突厥九姓亂,弓仁引軍度漠,逾白檉林,收火拔部喻多真種落,降之。趯跌思太叛,戰赤柳澗,弓仁騎才五百,自新堡進,時賊四環之,眾不敵,弓仁椎牛誓士自若,再宿潰圍出,人服其壯。凡閱大小戰數百,未嘗負。賜寶玉、甲第、良田,等列莫與比。累遷左驍衛大將軍、朔方副大使。會病,玄宗遣上醫馳視。卒,年六十六,贈撥川郡王,謚曰忠。



 孫惟貞。惟貞,名瑀,以字行。志向恢大。開元末,為左武衛將軍。肅宗在靈武,以衛尉少卿募兵綏、銀,閱旬,眾數萬。從還鳳翔,遷光祿卿,為元帥前鋒討擊。戰陜州,以功進殿中監。史思明攻李光弼於河陽,周摯以兵二十萬陣城下,惟貞請銳卒數千,鑿數門出,自旦及午,苦戰破之。光弼表為開府儀同三司。光弼討史朝義,以惟貞守徐州。賊將謝欽讓據陳,乃假惟貞潁州刺史,斬賊將,降者萬人。封蕭國公,實封百戶。光弼病,表以自代。擢左領軍衛大將軍,為英武軍使,卒。



 尉遲勝,本王於闐國。天寶中,入朝,獻名玉、良馬。玄宗以宗室女妻之,授右威衛將軍、毘沙府都督。歸國,與安西節度使高仙芝擊破薩毘、播仙。累進光祿卿。



 安祿山反,勝使弟曜攝國事,身率兵五千赴難。國人固留勝,勝以少女為質而行。肅宗嘉之,拜特進,兼殿中監。廣德中,進驃騎大將軍,遣還,固請留宿衛。加開府儀同三司,封武都郡王,實封百戶。勝請授國於曜,詔可。勝既留,乃穿築池觀,厚賓客,士大夫多從之游。從德宗至興元,為右領軍將軍,歷睦王傅。貞元初,曜上言:「國中以嫡承嗣,今勝讓國,請立其子銳。」帝欲遣銳襲王。勝固辭,以「曜久行國事,人安之;銳生京華,不習其俗,不可遣」。當是時,兄弟讓國,人莫不賢之。睦府除,徙原王傅。卒,贈涼州都督。



 尚可孤,字可孤,東部鮮卑宇文之別種,世處松、漠間。天寶末,隸範陽節度使安祿山,復事史思明。上元中,自賊所歸,累授左、右威衛大將軍,封白水縣伯,為神策大將。以功試太常卿。徙封馮翊郡王,食寶戶一百五十。



 魚朝恩主衛兵,器其勇,養為子,名智德。使將兵三千,屯扶風、武功,歷十餘年,隊伍閑整。朝恩死,詔賜氏李,名嘉勛。李希烈叛,擢為招討,應援荊襄,使復本姓名,累戰有功。



 硃泚之難,召可孤,可孤率兵三千,道襄、鄧而西,屬賊兵銳,乃壁七盤。偽將仇敬忠等來寇,可孤擊卻之,遂收藍田。德宗將遷梁州,命引兵守灞上,拜神策、京畿、渭南、商州節度招討使。敬忠拒戰,可孤急擊斬之。進軍與李晟收長安,為先鋒。以功加檢校尚書右僕射,封馮翊郡王,食實戶二百。又會諸軍進討李懷光,次沙苑,卒於軍,贈司空。



 可孤性謹審沉壯,既有勛勞,未嘗自論功,御眾公嚴,晟數稱之。



 裴玢,五世祖糾,本王疏勒,武德中來朝,拜鷹揚大將軍,封天山郡公,留不去,遂籍京兆。



 玢初事金吾將軍論惟明為傔力。德宗在奉天,以功封忠義郡王。從惟明鎮鄜坊,署牙將。後節度使王棲曜卒,中軍將何朝宗夜縱火作亂,玢獨匿不出。遲明,擒朝宗以待命。有詔並軍司馬崔輅斬之,以同州刺史劉公濟領節度,擢玢為司馬。逾年,公濟卒,乃授玢節度使。元和二年,徙山南西道。



 玢為治嚴棱,畏遠權勢,不務貢奉。蔬食弊衣,居處取避風雨而已。倉庫完實,百姓安之,當世將帥未有及者。以疾辭位。入朝,不事騶仗。妻乘竹輿,二侍婢,黃碧縑服。七年卒,贈尚書左僕射,謚曰節。



 贊曰:夷狄性惇固,其能知義所在者,鷙挺不可遷,蓋巧不足而諒常有餘。觀大柰等事君,皆一其志,無有顧望,用能功績光明,為天子倚信。至渾瑊、趯跌、光顏輩,烈垂無窮,惟其諒有餘故也。瑊、光顏自有傳,今類其人著之篇。



\end{pinyinscope}