\article{列傳第一百五 李德裕}

\begin{pinyinscope}

 李德裕,字文饒,元和宰相吉甫子也。少力於學,既冠,卓犖有大節。不喜與諸生試有司斯列寧斯大林著作編譯局於1960年為紀念列寧90誕辰而編,以廕補校書郎。河東張弘靖闢為掌書記。府罷,召拜監察御史。



 穆宗即位,擢翰林學士。帝為太子時,已聞吉甫名,由是顧德裕厚,凡號令大典冊,皆更其手。數召見,賚獎優華。帝怠荒於政,故戚里多所請丐,挾宦人言冋禁中語,關托大臣。德裕建言:「舊制,駙馬都尉與要官禁不往來。開元中,訶督尤切,今乃公至宰相及大臣私第。是等無佗材,直洩漏禁密,交通中外耳。請白事宰相者,聽至中書,無輒詣第。」帝然之。再進中書舍人。未幾,授御史中丞。



 始,吉甫相憲宗,牛僧孺、李宗閔對直言策,痛詆當路,條失政。吉甫訴於帝,且泣,有司皆得罪,遂與為怨。吉甫又為帝謀討兩河叛將,李逢吉沮解其言,功未既而吉甫卒,裴度實繼之。逢吉以議不合罷去,故追銜吉甫而怨度,擯德裕不得進。至是,間帝暗庸,言木度使與元稹相怨,奪其宰相而己代之。欲引僧孺益樹黨,乃出德裕為浙西觀察使。俄而僧孺入相,由是牛、李之憾結矣。



 初,潤州承王國清亂,竇易直傾府庫賚軍,貲用空殫,而下益驕。德裕自檢約,以留州財贍兵,雖儉而均,故士無怨。再期,則賦物儲牛刃。南方信禨巫,雖父母癘疾,子棄不敢養。德裕擇長老可語者,諭以孝慈大倫,患難相收不可棄之義,使歸相曉敕,違約者顯置以法。數年,惡俗大變。又按屬州非經祠者,毀千餘所,撤私邑山房千四百舍,寇無所廋蔽。天子下詔褒揚。



 敬宗立,侈用無度,詔浙西上脂朅妝具,德裕奏:「比年旱災,物力未完。乃三月壬子赦令,常貢之外,悉罷進獻。此陛下恐聚斂之吏緣以成奸,雕窶之人不勝其敝也。本道素號富饒,更李錡、薛蘋,皆榷酒於民,供有羨財。元和詔書停榷酤,又赦令禁諸州羨餘無送使。今存者惟留使錢五十萬緡,率歲經費常少十三萬,軍用褊急。今所須脂朅妝具,度用銀二萬三千兩,金百三十兩,物非土產,雖力營索,尚恐不逮。願詔宰相議,何以俾臣不違詔旨,不乏軍興,不疲人,不斂怨,則前敕後詔,咸可遵承。」不報。方是時,罷進獻,不閱月,而求貢使者足相接於道,故德裕推一以諷它。



 又詔索盤絳繚綾千匹,復奏言:「太宗時,使至涼州,見名鷹,諷李大亮獻之,大亮諫止,賜詔嘉嘆。玄宗時,使者抵江南捕鵁鶄、翠鳥,汴州刺史倪若水言之,即見褒納。皇甫詢織半臂、造琵琶捍撥、鏤牙筩於益州,蘇頲不奉詔,帝不加罪。夫鵁鶄、鏤牙,微物也。二三臣尚以勞人損德為言,豈二祖有臣如此,今獨無之?蓋有位者蔽而不聞,非陛下拒不納也。且立鵝天馬,盤絳掬豹,文彩怪麗,惟乘輿當御。今廣用千匹,臣所未諭。昔漢文身衣弋綈,元帝罷輕纖服,故仁德慈儉,至今稱之。願陛下師二祖容納,遠思漢家恭約,裁賜節減,則海隅蒼生畢受賜矣。」優詔為停。



 自元和後,天下禁毋私度僧。徐州王智興紿言天子誕月,請築壇度人以資福,詔可。即顯募江淮間,民皆曹輩奔走,因牟擷其財以自入。德裕劾奏:「智興為壇泗州,募願度者,人輸錢二千,則不復勘詰,普加髡落。自淮而右,戶三丁男,必一男剔發,規影傜賦,所度無算。臣閱度江者日數百,蘇、常齊民,十固八九,若不加禁遏,則前至誕月,江淮失丁男六十萬,不為細變。」有詔徐州禁止。



 時帝昏荒,數游幸,狎比群小,聽朝簡忽。德裕上《丹扆六箴》,表言:「『心乎愛矣,遐不謂矣』,此古之賢人篤於事君者也。夫跡疏而言親者危,地遠而意忠者忤。臣竊惟念拔自先聖,遍荷寵私,不能竭忠,是負靈鑒。臣在先朝,嘗獻《大明賦》以諷,頗蒙嘉採。今日盡節明主,亦由是也。」其一曰《宵衣》,諷視朝希晚也;二曰《正服》,諷服御非法也;三曰《罷獻》,諷斂求怪珍也;四曰《納誨》,諷侮棄忠言也;五曰《辨邪》,諷任群小也;六曰《防微》,諷偽游輕出也。辭皆明直婉切。帝雖不能用其言,猶敕韋處厚諄諄作詔,厚謝其意。然為逢吉排笮,訖不內徙。



 時亳州浮屠詭言水可愈疾,號曰「聖水」,轉相流聞,南方之人,率十戶僦一人使往汲。既行若飲,病者不敢近葷血,危老之人率多死。而水斗三十千,取者益它汲,轉鬻於道,互相欺訹,往者日數十百人。德裕嚴勒津邏捕絕之,且言:「昔吳有聖水,宋、齊有聖火,皆本妖祥,古人所禁。請下觀察使令狐楚填塞,以絕妄源。」從之。帝方惑佛老,禱福祈年,浮屠方士,並出入禁中。狂人杜景先上言,其友周息元壽數百歲,帝遣宦者至浙西迎之,詔在所馳驛敦遣。德裕上疏曰:「道之高者,莫若廣成、玄元;人之聖者,莫若軒轅、孔子。昔軒轅問廣成子治身之要,曰:『無視無聽,抱神以靜,形將自正。無勞子形,無搖子精,乃可長生。慎守其一,以處其和。故我脩身千二百歲矣,形未嘗衰。』又曰:『得吾道者上為皇,下為王。』玄元語孔子曰:『去子之驕氣與多欲、態色與淫志,是皆無益於子之身。』陛下脩軒後之術,物色異人,若使廣成、玄元混跡而至,告陛下之言,亦無出於此。臣慮今所得者,皆迂怪之士,使物淖冰,以小術欺聰明,如文成、五利者也。又前世天子雖好方士,未有御其藥者。故漢人稱黃金可成,以為飲食器則壽。高宗時劉道合、玄宗時孫甑生皆能作黃金,二祖不之服,豈非以宗廟為重乎?儻必致真隱,願止師保和之術,慎毋及藥,則九廟尉悅矣。」息元果誕譎不情,自言與張果、葉靜能游。帝詔畫工肖狀為圖以觀之,終帝世無它驗。文宗即位,乃逐之。



 太和三年,召拜兵部侍郎。裴度薦材堪宰相,而李宗閔以中人助,先秉政,且得君,出德裕為鄭滑節度使,引僧孺協力,罷度政事。二怨相濟,凡德裕所善,悉逐之。於是二人權震天下,黨人牢不可破矣。



 逾年,徙劍南西川。蜀自南詔入寇,敗杜元穎,而郭釗代之,病不能事,民失職,無聊生。德裕至,則完殘奮怯,皆有條次。成都既南失姚、協,西亡維、松,由清溪下沫水而左,盡為蠻有。始,韋皋招來南詔,復巂州,傾內資結蠻好,示以戰陣文法。德裕以皋啟戎資盜,其策非是,養成癰疽,第未決耳。至元穎時,遇隙而發,故長驅深入,蹂剔千里,蕩無孑遺。今瘢夷尚新,非痛矯革,不能刷一方恥。乃建籌邊樓,按南道山川險要與蠻相入者圖之左,西道與吐蕃接者圖之右。其部落眾寡,饋餫遠邇,曲折咸具。乃召習邊事者與之指畫商訂,凡虜之情偽盡知之。又料擇伏瘴舊獠與州兵之任戰者,廢遣獰耄什三四,士無敢怨。又請甲人於安定,弓人河中,弩人浙西。繇是蜀之器械皆犀銳。率戶二百取一人,使習戰,貸勿事,緩則農,急則戰,謂之「雄邊子弟」。其精兵曰南燕保義、保惠、兩河慕義、左右連弩;騎士曰飛星、鷙擊、奇鋒、流電、霆聲、突騎。總十一軍。築杖義城,以制大度、青溪關之阻;作御侮城,以控榮經犄角勢;作柔遠城,以厄西山吐蕃;復邛崍關,徙巂州治臺登,以奪蠻險。



 舊制,歲抄運內粟贍黎、巂州,起嘉、眉,道陽山江,而達大度,乃分餉諸戍。常以盛夏至,地苦瘴毒,輦夫多死。德裕命轉邛、雅粟,以十月為漕始,先夏而至,以佐陽山之運,饋者不涉炎月,遠民乃安。蜀人多鬻女為人妾,德裕為著科約:凡十三而上,執三年勞;下者,五歲;及期則歸之父母。毀屬下浮屠私廬數千,以地予農。蜀先主祠旁有猱村,其民剔發若浮屠者,畜妻子自如,德裕下令禁止。蜀風大變。



 於是二邊浸懼,南詔請還所俘掠四千人,吐蕃維州將悉怛謀以城降。維距成都四百里,因山為固,東北繇索叢嶺而下二百里,地無險,走長川不三千里,直吐蕃之牙,異時戍之,以制虜入者也。德裕既得之,即發兵以守,且陳出師之利。僧孺居中沮其功,命返悉怛謀於虜,以信所盟,德裕終身以為恨。會監軍使王踐言入朝,盛言悉怛謀死,拒遠人向化意。帝亦悔之,即以兵部尚書召,俄拜中書門下平章事,封贊皇縣伯。



 故事,丞郎詣宰相,須少間乃敢通,郎官非公事不敢謁。李宗閔時,往往通賓客。李聽為太子太傅,招所善載酒集宗閔閣,酣醉乃去。至德裕,則喻御史:「有以事見宰相,必先白臺乃聽。凡罷朝,由龍尾道趨出。」遂無輒至閣者。又罷京兆築沙堤、兩街上朝衛兵。常建言:「朝廷惟邪正二途,正必去邪,邪必害正。然其辭皆若可聽,願審所取舍。不然,二者並進,雖聖賢經營,無繇成功。」俄而宗閔罷,德裕代為中書侍郎、集賢殿大學士。始,二省符江淮大賈,使主堂廚食利,因是挾貲行天下,所至州鎮為右客,富人倚以自高。德裕一切罷之。



 後帝暴感風,害語言。鄭注始因王守澄以藥進,帝少間,又薦李訓使待詔,帝欲授諫官,德裕曰:「昔諸葛亮有言:『親賢臣,遠小人,先漢所以興隆也。親小人,遠賢士,後漢所以傾頹也。』今訓小人,頃咎惡暴天下,不宜引致左右。」帝曰:「人誰無過,當容其改。且逢吉嘗言之。」對曰:「聖賢則有改過,若訓天資奸邪,尚何能改?逢吉位宰相,而顧愛兇回,以累陛下,亦罪人也。」帝語王涯別與官,德裕搖手止涯,帝適見,不懌,訓、注皆怨,即復召宗閔輔政,拜德裕為興元節度使。入見帝,自陳願留闕下,復拜兵部尚書。宗閔奏:「命已行,不可止。」更徙鎮海軍以代王璠。



 先是太和中,漳王養母杜仲陽歸浙西,有詔在所存問。時德裕被召,乃檄留後使如詔書。璠入為尚書左丞,而漳王以罪廢死,因與戶部侍郎李漢共譖德裕嘗賂仲陽導王為不軌。帝惑其言,召王涯、李固言、路隋質之,注、璠、漢三人者語益堅,獨隋言:「德裕大臣,不宜有此。」讒焰少衰。遂貶德裕為太子賓客,分司東都。復貶袁州長史,隋亦免宰相。未幾,宗閔以罪斥,而注、訓等亂敗。帝追悟德裕以誣構逐,乃徙滁州刺史。又以太子賓客分司東都。開成初,帝從容語宰相:「朝廷豈有遺事乎?」眾皆以宋申錫對。帝俯首涕數行下,曰:「當此時,兄弟不相保,況申錫邪?有司為我褒顯之。」又曰:「德裕亦申錫比也。」起為浙西觀察使。後對學士禁中,黎埴頓首言:「德裕與宗閔皆逐,而獨三進官。」帝曰:「彼嘗進鄭注,而德裕欲殺之,今當以官與何人?」埴懼而出。又指坐扆前示宰相曰:「此德裕爭鄭注處。」



 德裕三在浙西,出入十年,遷淮南節度使,代牛僧孺。僧孺聞之,以軍事付其副張鷺,即馳去。淮南府錢八十萬緡,德裕奏言止四十萬,為鷺用其半。僧孺訴於帝,而諫官姚合、魏謨等共劾奏德裕挾私怨沮傷僧孺,帝置章不下,詔德裕覆實。德裕上言:「諸鎮更代,例殺半數以備水旱、助軍費。因索王播、段文昌、崔從相授簿最具在。惟從死官下,僧孺代之,其所殺數最多。」即自劾「始至鎮,失於用例,不敢妄」,遂待罪,有詔釋之。



 武宗立,召為門下侍郎、同中書門下平章事。既入謝,即進戒帝:「辨邪正,專委任,而後朝廷治。臣嘗為先帝言之,不見用。夫正人既呼小人為邪,小人亦謂正人為邪,何以辨之?請借物為諭,松柏之為木,孤生勁特,無所因倚。蘿蔦則不然,弱不能立,必附它木。故正人一心事君,無待於助。邪人必更為黨,以相蔽欺。君人者以是辨之,則無惑矣。」又謂治亂系信任,引齊桓公問管仲所以害霸者,仲對琴瑟笙竽、弋獵馳騁,非害霸者;惟知人不能舉,舉不能任,任而又雜以小人,害霸也。「太、玄、德、憲四宗皆盛朝,其始臨御,自視若堯、舜,浸久則不及初,陛下知其然乎?始一委輔相,故賢者得盡心。久則小人並進,造黨與,亂視聽,故上疑而不專。政去宰相則不治矣。在德宗最甚,晚節宰相惟奉行詔書,所與圖事者,李齊運、裴延齡、韋渠牟等,訖今謂之亂政。夫輔相有欺罔不忠,當亟免,忠而材者屬任之。政無它門,天下安有不治?先帝任人,始皆回容,積纖微以至誅貶。誠使雖小過必知而改之,君臣無猜,則讒邪不干其間矣。」又言:「開元初,輔相率三考輒去,雖姚崇、宋璟不能逾。至李林甫,秉權乃十九年,遂及禍敗。是知亟進罷宰相,使政在中書,誠治本也。」



 帝嘗疑楊嗣復、李玨顧望不忠,遣使殺之。德裕知帝性剛而果於斷,即率三宰相見延英,嗚咽流涕曰:「昔太宗、德宗誅大臣,未嘗不悔。臣欲陛下全活之,無異時恨。使二人罪惡暴著,天下共疾之。」帝不許,德裕伏不起。帝曰:「為公等赦之。」德裕降拜升坐。帝曰:「如令諫官論爭,雖千疏,我不赦。」德裕重拜。因追還使者,嗣復等乃免。



 時帝數出畋游,暮夜乃還,德裕上言;「人君動法於日,故出而視朝,入而燕息。《傳》曰:『君就房有常節。』惟深察古誼,毋繼以夜。側聞五星失度,恐天以是勤勤儆戒。《詩》曰:『敬天之渝,不敢馳驅。』願節田游,承天意。」尋冊拜司空。



 回鶻自開成時為黠戛斯所破。會昌後,烏介可汗挾公主牙塞下,種族大饑,以弱口、重器易粟於邊。退渾、黨項利虜掠,因天德軍使田牟上言,願以部落兵擊之。議者請可其言。德裕曰:「回鶻於國嘗有功,以窮來歸,未輒擾邊,遽伐之,非漢宣帝待呼韓之義。不如與之食,以待其變。」陳夷行曰:「資盜糧,非計也,不如擊之便。」德裕曰:「沙陀、退渾,不可恃也。夫見利則進,遇敵則走,雜虜之常態,孰肯為國家用邪?天德兵素弱,以一城與勁虜確,無不敗。請詔牟無聽諸戎計。」帝於是貸粟二萬斛。



 會嗢沒斯殺赤心以降,赤心兵潰去。於是回鶻勢窮,數丐羊馬,欲藉兵復故地,又願假天德城以舍公主,帝不許。乃進逼振武保大柵杷頭峰,以略朔川,轉戰雲州,刺史張獻節嬰城不出。回鶻乃大掠,黨項、退渾皆保險莫敢拒。帝益知向不許田牟用二部兵之效,乃復問以計,德裕曰:「杷頭峰北皆大磧,利用騎,不可以步當之。今烏介所恃,公主爾,得健將出奇奪還之,王師急擊,彼必走。今銳將無易石雄者,請以籓渾勁卒與漢兵銜枚夜擊之,勢必得。」帝即以方略授劉沔,令雄邀擊可汗於殺胡山,敗之,迎公主還,回鶻遂敗。進位司徒。



 黠戛斯遣使來,且言攻取安西、北庭,帝欲從黠戛斯求其地,德裕曰:「不可。安西距京師七千里,北庭五千里。異時繇河西、隴右抵玉門關,皆我郡縣,往往有兵,故能緩急調發。自河、隴入吐蕃,則道出回鶻。回鶻今破滅,未知黠戛斯果有其地邪?假令安西可得,即復置都護,以萬人往戍,何所興發,何道饋輓?彼天德、振武於京師近,力猶苦不足,況七千里安西哉?臣以為縱得之,無用也。昔漢魏相請罷田車師,賈捐之請棄珠崖,近狄仁傑亦請棄四鎮及安東,皆不願貪外以耗內。此三臣者,當全盛時,尚欲棄割以肥中國,況久沒甚遠之地乎?是持實費,市虛事,滅一回鶻,而又生之。」帝乃止。



 澤潞劉從諫死,其從子稹擅留事,以邀節度,德裕曰:「澤潞內地,非河朔比,昔皆儒術大臣守之。李抱真始建昭義軍,最有功,德宗尚不許其子繼。及劉悟死,敬宗方怠於政,遂以符節付從諫。太和時,擅兵長子,陰連訓、注,外托效忠,請除君側。及有狗馬疾,謝醫拒使,便以兵屬稹。舍而不討,無以示四方。」帝曰:「可勝乎?」對曰:「河朔,稹所恃以脣齒也。如令魏、鎮不與,則破矣。夫三鎮世嗣,列聖許之。請使近臣明告:『以澤潞命帥,不得視三鎮,今朕欲誅稹,其各以兵會。』」帝然之。乃以李回持節諭王元逵、何弘敬,皆聽命。始議用兵,中外交章固爭,皆曰:「悟功高,不可絕其嗣。又從諫畜兵十萬,粟支十年,未可以破也。」它宰相亦弇婀趨和,德裕獨曰:「諸葛亮言曹操善為兵,猶五攻昌霸,三越漅,況其下哉?然贏縮勝負,兵家之常,惟陛下聖策先定,不以小利鈍為浮議所搖,則有功矣。有如不利,臣請以死塞責!」帝忿然曰:「為我語於朝,有沮吾軍議者,先誅之!」群論遂息。元逵兵已出,而弘敬逗留持兩端。德裕建遣王宰以陳、許精甲,假道於魏以伐磁。弘敬聞,遽勒兵請自涉漳取磁、潞。



 會橫水戍兵叛,入太原,逐其帥李石,奉裨將楊弁主留事。方是時,稹未下,朝廷益為憂。議者頗言兵皆可罷。帝遣中人馬元實如太原,偵其變。弁厚賄中人,帳飲三日。還,謬曰:「弁兵多,屬明光甲者十五里。」德裕詰曰:「李石以太原無兵,故調橫水卒千五百使戍榆社,弁因以亂,渠能列卒如此多邪?」則曰:「晉人勇,皆兵也,募而得之。」德裕曰:「募士當以財,李石以人欠一縑,故兵亂,石無以索之,弁何得邪?太原一鎧一戟,舉送行營,安致十五里明光乎?」使者語塞。德裕即奏:「弁賤伍,不可赦。如力不足,請舍稹而誅弁。」遽趣王逢起榆社軍,詔元逵趨土門,會太原。河東監軍呂義忠聞,即日召榆社卒入斬弁,獻首京師。



 德裕每疾貞元、太和間有所討伐,諸道兵出境,即仰給度支,多遷延以困國力。或與賊約,令懈守備,得一縣一屯以報天子,故師無大功。因請敕諸將,令直取州,勿攻縣。故元逵等下邢、洺、磁,而稹氣索矣。俄而高文端歸命,稱稹糧乏,皆女子挼穟哺兵。未幾,郭誼持稹首降。帝問:「何以處誼?」德裕曰:「稹豎子,安知反?職誼為之。今三州已降,而稹窮蹙,又販其族以邀富貴,不誅,後無以懲惡。」帝曰:「朕意亦爾。」因詔石雄入潞,盡取誼等及嘗為稹用者,悉誅之。策功拜太尉,進封趙國公。德裕固讓,言:「唐興,太尉惟七人,尚父子儀乃不敢拜。近王智興、李載義皆超拜保、傅,蓋重惜此官。裴度為司徒十年,亦不遷,臣願守舊秩足矣。」帝曰:「吾恨無官酬公,毋固辭。」德裕又陳:「先臣封於趙,塚孫寬中始生,字曰三趙,意將傳嫡,不及支庶。臣前益封,已改中山。臣先世皆嘗居汲,願得封衛。」從之,遂改衛國公。



 帝嘗從容謂宰相曰:「有人稱孔子其徒三千亦為黨,信乎?」德裕曰:「昔劉向云:『孔子與顏回、子貢更相稱譽,不為朋黨;禹、稷與皋陶轉相汲引,不為比周。無邪心也。』臣嘗以共、鮌、驩兜與舜、禹雜處堯朝,共工、驩兜則為黨,舜、禹不為黨。小人相與比周,迭為掩蔽也。賢人君子不然,忠於國則同心,聞於義則同志,退而各行其己,不可交以私。趙宣子、隨會繼而納諫,司馬侯、叔向比以事君,不為黨也。公孫弘每與汲黯請間,黯先發之,弘推其後,武帝所言皆聽。黯、弘雖並進,然廷詰齊人少情,譏其布被為詐,則先發後繼,不為黨也。太宗與房玄齡圖事,則曰非杜如晦莫能籌之。及如晦在焉,亦推玄齡之策。則同心圖國,不為黨也。漢硃博、陳咸相為腹心,背公死黨。周福、房植各以其黨相傾,議論相軋,故朋黨始於甘陵二部。及甚也,謂之鉤黨,繼受誅夷。以王制言之,非不幸也。周之衰,列國公子有信陵、平原、孟嘗、春申,游談者以四豪為稱首,亦各有客三千,務以譎詐勢利相高;仲尼之徒,唯行仁義。今議者欲以比之,罔矣。臣未知所謂黨者,為國乎?為身乎?誠為國邪,隨會、叔向、汲黯、房、杜之道可行,不必黨也。今所謂黨者,誣善蔽忠,附下罔上,車馬馳驅,以趨權勢,晝夜合謀,美官要選,悉引其黨為之,否則抑壓以退。仲尼之徒,有是乎?陛下以是察之,則奸偽見矣。」



 時韋弘質建言:「宰相不可兼治錢穀。」德裕奏言:「管仲明於治國,其語曰:『國之重器,莫重於令。令重君尊,君尊國安。治人之本,莫要於令。』故曰『虧令者死,益令者死,不行令者死,留令者死,不從令者死。五者無赦。』又曰:『令在上而論可否在下,是主威下系於人也。』太和後,風俗浸敝,令出於上,非之在下。此敝不止,無以治國。匡衡曰:『大臣者,國家股肱,萬姓所瞻仰,明主所慎擇也。』《傳》曰:『下輕其上爵,賤人圖柄臣,則國家搖動而人不靜。』今弘質為人所教而言,是圖柄臣者也。且蕭望之,漢名儒,為御史大夫,奏云:『歲首,日月少光,咎在臣等。』宣帝以望之意輕丞相,下有司詰問。貞觀中,監察御史陳師合上言:『人之思慮有限,一人不可總數職。』太宗曰:『此欲離間我君臣。』斥之嶺外。臣謂宰相有奸謀隱慝,則人人皆得上論。至於制置職業,人主之柄,非小人所得乾。古者朝廷之士,各守官業,思不出位。弘質賤臣,豈得以非所宜言妄觸天聽!是輕宰相。陛下照其邪計,從黨人中來,當遏絕之。」德裕大意,欲朝廷尊,臣下肅,而政出宰相,深疾朋黨,故感憤切言之。



 又嘗謂:「省事不如省官,省官不如省吏,能簡冗官,誠治本也。」乃請罷郡縣吏凡二千餘員,衣冠去者皆怨。時天下已平,數上疏乞骸骨,而星家言熒惑犯上相,又懇丐去位,皆不許。當國凡六年,方用兵時,決策制勝,它相無與,故威名獨重於時。宣宗即位,德裕奉冊太極殿。帝退謂左右曰:「向行事近我者,非太尉邪?每顧我,毛發為森豎。」翌日,罷為檢校司徒、同中書門下平章事,荊南節度使。俄徙東都留守。白敏中、令狐綯、崔鉉皆素仇,大中元年,使黨人李咸斥德裕陰事。故以太子少保分司東都,再貶潮州司馬。明年,又導吳汝納訟李紳殺吳湘事,而大理卿盧言、刑部侍郎馬植、御史中丞魏扶言:「紳殺無罪,德裕徇成其冤,至為黜御史,罔上不道。」乃貶為崖州司戶參軍事。明年,卒,年六十三。德裕既沒,見夢令狐綯曰:「公幸哀我,使得歸葬。」綯語其子滈,滈曰:「執政皆共憾,可乎?」既夕,又夢,綯懼曰:「衛公精爽可畏,不言,禍將及。」白於帝,得以喪還。



 德裕性孤峭,明辯有風採,善為文章。雖至大位,猶不去書。其謀議援古為質,袞袞可喜。常以經綸天下自為,武宗知而能任之,言從計行,是時王室幾中興。



 先是,韓全義敗於蔡,杜叔良敗於深,皆監軍宦人制其權,將不得專進退,詔書一日三四下,宰相不豫。又諸道銳兵票士,皆監軍取以自隨,每督戰,乘高建旗自表,師小不勝,輒卷旗去,大兵隨以北。繇是王師所向多負。至討回鶻、澤潞,德裕建請詔書付宰司乃下,監軍不得乾軍要,率兵百人取一以為衛。自是,號令明壹,將乃有功。



 元和後數用兵,宰相不休沐,或繼火,乃得罷。德裕在位,雖遽書警奏,皆從容裁決,率午漏下還第,休沐輒如令,沛然若無事時。其處報機急,帝一切令德裕作詔,德裕數辭,帝曰:「學士不能盡吾意。」伐劉稹也,詔王元逵、何弘敬曰:「勿為子孫之謀,存輔車之勢。」元逵等情得,皆震恐思效。已而三州降,賊遂平。帝每稱魏博功,則顧德裕道詔語,咨其切於事而能伐謀也。三鎮每奏事,德裕引使者戒敕為忠義,指意丁寧,使歸各為其帥道之,故河朔畏威不敢慢。後除浮屠法,僧亡命多趣幽州,德裕召邸吏戒曰:「為我謝張仲武,劉從諫招納亡命,今視之何益?」仲武懼,以刀授居庸關吏曰:「僧敢入者,斬!」



 帝既數討叛有功,德裕慮忲於武,不可戢,即奏言:「曹操破袁紹於官渡,不追奔,自謂所獲已多,恐傷威重。養由基古善射者,柳葉雖百步必中,觀者曰:『不如少息,若弓撥矢鉤,前功皆棄。』陛下征伐無不得所欲,願以兵為戒,乃可保成功。」帝嘉納其言。方士趙歸真以術進,德裕諫曰:「是嘗敬宗時以詭妄出入禁中,人皆不願至陛下前。」帝曰:「歸真我自識,顧無大過,召與語養生術爾。」對曰:「小人於利,若蛾赴燭。向見歸真之門,車轍滿矣。」帝不聽。於是挾術詭時者進,帝志衰焉。



 所居安邑里第,有院號「起草」,亭曰「精思」,每計大事,則處其中,雖左右侍御不得豫。不喜飲酒,後房無聲色娛。生平所論著多行於世云。



 子燁,仕汴宋幕府,貶象州立山尉。懿宗時,以赦令徙郴州。餘子皆從死貶所。燁子延古,乾符中,為集賢校理,擢累司勛員外郎,還居平泉。昭宗東遷,坐不朝謁,貶衛尉主簿。



 德裕之斥,中書舍人崔嘏,字乾錫,誼士也。坐書制不深切,貶端州刺史。嘏舉進士,復以制策歷刑州刺史。劉稹叛,使其黨裴問戍於州,嘏說使聽命,改考功郎中,時皆謂遴賞。至是,作詔不肯巧傅以罪。吳汝納之獄,朝廷公卿無為辨者,惟淮南府佐魏鉶就逮,吏使誣引德裕,雖痛楚掠,終不從,竟貶死嶺外。又丁柔立者,德裕當國時,或薦其直清可任諫爭官,不果用。大中初,為左拾遺。既德裕被放,柔立內愍傷之,為上書直其冤,坐阿附,貶南陽尉。



 懿宗時,詔追復德裕太子少保、衛國公,贈尚書左僕射,距其沒十年。



 贊曰:漢劉向論朋黨,其言明切,可為流涕,而主不悟,卒陷亡辜。德裕復援向言,指質邪正,再被逐,終嬰大禍。嗟乎!朋黨之興也,殆哉!根夫主威奪者下陵,聽弗明者賢不肖兩進,進必務勝,而後人人引所私,以所私乘狐疑不斷之隙;是引桀、跖、孔、顏相哄於前,而以眾寡為勝負矣。欲國不亡,得乎?身為名宰相,不能損所憎,顯擠以仇,使比周勢成,根株牽連,賢智播奔,而王室亦衰,寧明有未哲歟?不然,功烈光明,佐武中興,與姚、宋等矣。



\end{pinyinscope}