\article{列傳第十八 二李勣}

\begin{pinyinscope}

 李靖,字藥師,京兆三原人。姿貌魁秀,通書史。嘗謂所親曰:「丈夫遭遇,要當以功名取富貴,何至作章句儒!」其舅韓擒虎每與論兵,輒嘆曰:「可與語孫、吳者,非斯人尚誰哉!」仕隋為殿內直長,吏部尚書牛弘見之曰:「王佐才也!」左僕射楊素拊其床謂曰:「卿終當坐此!」



 大業末,為馬邑丞。高祖擊突厥,靖察有非常志,自囚上急變,傳送江都,至長安,道梗。高祖已定京師,將斬之,靖呼曰:「公起兵為天下除暴亂,欲就大事,以私怨殺誼士乎?」秦王亦為請,得釋,引為三衛。從平王世充,以功授開府。



 蕭銑據江陵,詔靖安輯,從數輕騎道金州,會蠻賊鄧世洛兵數萬屯山谷間,廬江王瑗討不勝,靖為瑗謀,擊卻之。進至峽州,阻銑兵不得前。帝謂逗留,詔都督許紹斬靖,紹為請而免。開州蠻冉肇則寇夔州,趙郡王孝恭戰未利,靖率兵八百破其屯,要險設伏,斬肇則,俘禽五千。帝謂左右曰:「使功不如使過,靖果然。」因手敕勞曰:「既往不咎,向事吾久已忘之。」靖遂陳圖銑十策。有詔拜靖行軍總管,兼攝孝恭行軍長史,軍政一委焉。



 武德四年八月,大閱兵夔州。時秋潦,濤瀨漲惡,銑以靖未能下,不設備。諸將亦請江平乃進。靖曰:「兵機事,以速為神。今士始集,銑不及知,若乘水傅壘,是震霆不及塞耳,有能倉卒召兵,無以御我,此必禽也。」孝恭從之。



 九月,舟師叩夷陵,銑將文士弘以卒數萬屯清江,孝恭欲擊之,靖曰:「不可。士弘健將,下皆勇士,今新失荊門,悉銳拒我,此救敗之師,不可當。宜駐南岸,待其氣衰乃取之。」孝恭不聽,留靖守屯,自往與戰,大敗還。賊委舟散掠,靖視其亂,縱兵擊破之,取四百餘艘,溺死者萬人。即率輕兵五千為先鋒,趨江陵,薄城而營。破其將楊君茂、鄭文秀,俘甲士四千。孝恭軍繼進,銑大懼,檄召江南兵,不及到,明日降。靖入其都,號令靜嚴,軍無私焉。或請靖籍銑將拒戰者家貲以賞軍,靖曰:「王者之兵,吊人而取有罪,彼其脅驅以來,藉以拒師,本非所情,不容以叛逆比之。今新定荊、郢,宜示寬大,以慰其心,若降而籍之,恐自荊而南,堅城劇屯,驅之死守,非計之善也。」止不籍。由是江、漢列城爭下。以功封永康縣公,檢校荊州刺史。乃度嶺至桂州,分道招慰。酋領馮盎等皆以子弟來謁,南方悉定。裁量款效,承制補官。得郡凡九十六,戶六十餘萬。詔書勞勉,授嶺南撫慰大使、檢校桂州總管。以嶺海陋遠,久不見德,非震威武、示禮義,則無以變風。即率兵南巡,所過問疾苦,延見長老,宣布天子恩意,遠近歡服。



 輔公祏據丹陽反,詔孝恭為帥,召靖入朝受方略,副孝恭東討,李世勣等七總管皆受節度。公祏遣馮惠亮以舟師三萬屯當塗,陳正通步騎二萬屯青林,自梁山連鎖以斷江道。築卻月城,延袤十餘里,為犄角。諸將議曰:「彼勁兵連柵,將不戰疲老我師。若直取丹陽,空其巢窟,惠亮等自降。」靖曰:「不然。二軍雖精,而公祏所自將亦銳卒也,既保石頭,則牢未可拔。我留不得志,退有所忌,腹背蒙患,非百全計。且惠亮、正通百戰餘賊,非怯野鬥,今方持重,特公祏立計爾。若出不意,挑攻其城,必破之。惠亮拔,公祏禽矣。」孝恭聽之。靖率黃君漢等水陸皆進,苦戰,殺傷萬餘人,惠亮等亡去。靖將輕兵至丹陽,公祏懼,眾尚多,不能戰,乃出走,禽之,江南平。置東南道行臺,以為行臺兵部尚書。賜物千段、奴婢百口、馬百匹。行臺廢,檢校揚州大都督府長史。帝嘆曰:「靖乃銑、公祏之膏肓也,古韓、白、衛、霍何以加!」



 八年,突厥寇太原,為行軍總管,以江淮兵萬人屯大谷。時諸將多敗,獨靖以完軍歸。俄權檢校安州大都督。太宗踐阼,授刑部尚書,錄功,賜實封四百戶,兼檢校中書令。突厥部種離畔,帝方圖進取,以兵部尚書為定襄道行軍總管,率勁騎三千繇馬邑趨惡陽嶺。頡利可汗大驚,曰:「兵不傾國來,靖敢提孤軍至此?」於是帳部數恐。靖縱諜者離其腹心,夜襲定襄,破之,可汗脫身遁磧口。進封代國公。帝曰:「李陵以步卒五千絕漠,然卒降匈奴,其功尚得書竹帛。靖以騎三千,蹀血虜庭,遂取定襄,古未有輩,足澡吾渭水之恥矣!」



 頡利走保鐵山,遣使者謝罪,請舉國內附。以靖為定襄道總管往迎之。又遣鴻臚卿唐儉、將軍安修仁慰撫。靖謂副將張公謹曰:「詔使到,虜必自安,若萬騎齎二十日糧,自白道襲之,必得所欲。」公謹曰:「上已與約降,行人在彼,奈何?」靖曰:「機不可失,韓信所以破齊也。如唐儉輩何足惜哉!」督兵疾進,行遇候邏,皆俘以從,去其牙七里乃覺,部眾震潰,斬萬餘級,俘男女十萬,禽其子疊羅施,殺義成公主。頡利亡去,為大同道行軍總管張寶相禽以獻。於是斥地自陰山北至大漠矣。帝因大赦天上,賜民五日酺。



 御史大夫蕭瑀劾靖持軍無律,縱士大掠,散失奇寶。帝召讓之,靖無所辯,頓首謝。帝徐曰:「隋史萬歲破達頭可汗,不賞而誅,朕不然,赦公之罪,錄公之功。」乃進左光祿大夫,賜絹千匹,增戶至五百。既而曰:「向人譖短公,朕今悟矣。」加賜帛一千匹,遷尚書右僕射。



 靖每參議,恂恂似不能言,以沈厚稱。時遣使十六道巡察風俗,以靖為畿內道大使,會足疾,懇乞骸骨。帝遣中書侍郎岑文本諭旨曰:「自古富貴而知止者蓋少,雖疾頓憊,猶力於進。公今引大體,朕深嘉之。欲成公美,為一代法,不可不聽。」乃授檢校特進,就第,賜物段千,尚乘馬二,祿賜、國官、府佐皆勿廢。若疾少間,三日一至門下中書平章政事。加賜靈壽杖。



 頃之,吐谷渾寇邊。帝謂侍臣曰:「靖能復起為帥乎?」靖往見房玄齡,曰:「吾雖老,尚堪一行。」帝喜,以為西海道行軍大總管,任城王道宗、侯君集、李大亮、李道彥、高甑生五總管兵皆屬。軍次伏俟城,吐谷渾盡火其莽,退保大非川。諸將議,春草未芽,馬弱不可戰。靖決策深入,遂逾積石山。大戰數十,多所殺獲,殘其國,國人多降,吐谷渾伏允愁蹙自經死。靖更立大寧王慕容順而還。甑生軍繇鹽澤道後期,靖簿責之。既歸而憾,與廣州長史唐奉義告靖謀反,有司按驗無狀,甑生等以誣罔論。靖乃闔門自守,賓客親戚一謝遣。改衛國公。其妻卒,詔墳制如衛、霍故事,築闕象鐵山、積石山,以旌其功,進開府儀同三司。



 帝將伐遼,召靖入,謂曰:「公南平吳,北破突厥,西定吐谷渾,惟高麗未服,亦有意乎?」對曰:「往憑天威,得效尺寸功。今疾雖衰,陛下誠不棄,病且瘳矣。」帝憫其老,不許。二十三年,病甚,帝幸其第,流涕曰:「公乃朕生平故人,於國有勞。今疾若此,為公憂之。」薨,年七十九,贈司徒、並州都督,給班劍、羽葆、鼓吹,陪葬昭陵,謚日景武。子德謇嗣,官至將作少匠,坐善太子承乾,流嶺南,以靖故徙吳郡。



 靖兄端,字藥王,以靖功襲永康公,梓州刺史。弟客師,右武衛將軍,累戰功封丹陽郡公。致仕,居昆明池南。善騎射,喜馳獵,雖老猶未衰。自京南屬山,西際澧水,鳥鵲皆識之,每出,從之翔噪,人謂之「鳥賊」。卒,年九十,贈幽州都督。



 孫令問,玄宗為臨淄王時與雅舊。及即位,以協贊功,遷殿中少監。預誅竇懷貞,封宋國公,實封五百戶。進散騎常侍,知尚食事,恩待甚渥。然未嘗輒干政,率游畋自娛,厚奉養,侈飲食,至躬視刲宰。有譏之者,答曰:「此畜豢,天所以養人,與蔬果何異,安用妄分別邪?」後坐其子與回紇部酋承宗連婚,貶撫州別駕,卒。



 靖五代孫彥芳,大和中,為鳳翔司錄參軍。家故藏高祖、太宗賜靖詔書數函,上之。一曰:「兵事節度皆付公,吾不從中治也。」一曰:「有晝夜視公疾大老嫗遣來,吾欲熟知公起居狀。」皆太宗手墨,它大略如此。文宗愛之不廢手。其舊物有佩筆,以木為管弢,刻金其上,別為環以限其間,筆尚可用也。靖破蕭銑時,所賜於闐玉帶十三胯,七方六刓,胯各附環,以金固之,所以佩物者。又有火鑒、大觿、算囊等物,常佩於帶者。天子悉留禁中。又敕摸詔本,還賜彥芳,並束帛衣服。權德輿嘗讀太宗手詔,至流涕曰:「君臣之際乃爾邪!」



 李勣,字懋功,曹州離狐人。本姓徐氏,客衛南。家富,多僮僕,積粟常數千鐘。與其父蓋皆喜施貸,所周給無親疏之間。



 隋大業末,韋城翟讓為盜,勣年十七,往從之。說曰:「公鄉壤不宜自剽殘,宋、鄭商旅之會,御河在中,舟艦相屬,往邀取之,可以自資。」讓然之。劫公私船取財,繇是兵大振。李密亡命雍丘,勣與浚儀王伯當共說讓,推密為主。以奇計破王世充。密署勣右武候大將軍、東海郡公。當是時,河南、山東大水,隋帝令饑人就食黎陽倉,吏不時發,死者日數萬。勣說密曰:「天下之亂本於饑,今若取黎陽粟以募兵,大事濟矣。」密以麾下兵五千付勣,與郝孝德等濟河,襲黎陽,守之。開倉縱食,旬日,勝兵至二十萬。宇文化及擁兵北上,密使勣守倉,周掘塹以自環。化及攻之,勣為地道出鬥,化及敗,引去。



 武德二年,密歸朝廷,其地東屬海,南至江,西直汝,北抵魏郡,勣統之,未有所屬。謂長史郭孝恪曰:「人眾土宇,皆魏公有也。吾若獻之,是利主之敗為己功,吾所羞也。」乃錄郡縣戶口以啟密,請自上之。使至,高祖訝無表,使者以意聞。帝喜曰:「純臣也。」詔授黎州總管,封萊國公。賜姓,附宗正屬籍,徙封曹,給田五十頃,甲第一區。封蓋濟陰王,固辭,改舒國公。詔勣總河南、山東兵以拒王世充。及密以謀反誅,帝遣使示密反狀。勣請收葬,詔從之。勣為密服縗絰,葬訖乃釋。



 俄為竇建德所陷,質其父,使復守黎陽。三年,自拔來歸。從秦王伐東都,戰有功。東略地至虎牢,降鄭州司兵沈悅。平建德,俘世充,乃振旅還,秦王為上將,勣為下將,皆服金甲,乘戎輅,告捷於廟。蓋亦自洺州與裴矩入朝,詔復其官。



 又從破劉黑闥、徐圓朗,累遷左監門大將軍。圓朗復反,詔勣為河南大總管,討平之。趙郡王孝恭討輔公祏也,遣勣以步卒一萬度淮,拔壽陽,攻江西賊壁,馮惠亮、陳正通相次潰,公祏平。



 太宗即位,拜並州都督,賜實封九百戶。貞觀三年,為通漠道行軍總管,出雲中,與突厥戰,走之。引兵與李靖合。因曰:「頡利若度磧,保於九姓,果不可得,我若約齎薄之,不戰縛虜矣。」靖大喜,以與己合,於是意決。靖率眾夜發,勣勒兵從之。頡利欲走磧,勣前屯磧口,不得度,由是酋長率部落五萬降於勣。詔拜光祿大夫,行並州大都督府長史。父喪解,奪哀還官,徙封英,治並州十六年,以威肅聞。帝嘗曰:「煬帝不擇人守邊,勞中國築長城以備虜。今我用勣守並,突厥不敢南,賢長城遠矣!」召為兵部尚書,未至,會薛延陀子大度設以八萬騎侵李思摩。詔勣為朔方道行軍總管,將輕騎六千,擊度設青山,斬名王一,俘口五萬。以功封一子為縣公。



 晉王為皇太子,授詹事,兼左衛率,俄同中書門下三品。帝曰:「吾兒方位東宮,公舊長史,以宮事相委,勿以資屈為嫌也。」後帝自將征高麗,以勣為遼東道行軍大總管。破蓋牟、遼東、白崖等城,從戰駐蹕山,功多,封一子為郡公。延陀部落亂,詔將二百騎發突厥兵討之,大戰烏德鞬山,破之,降其首領梯真達干,而可汁咄摩支遁入荒谷,磧北遂定。改太常卿,仍同中書門下三品,復為詹事。



 勣既忠力,帝謂可托大事。嘗暴疾,醫曰:「用須灰可治。」帝乃自翦須以和藥。及愈,入謝,頓首流血。帝曰:「吾為社稷計,何謝為!」後留宴,顧曰:「朕思屬幼孤,無易公者。公昔不遺李密,豈負朕哉?」勣感涕,因嚙指流血。俄大醉,帝親解衣覆之。帝疾,謂太子曰:「爾於勣無恩,今以事出之,我死,宜即授以僕射,彼必致死力矣!」乃授疊州都督。



 高宗立,召授檢校洛州刺史、洛陽宮留守,進開府儀同三司、同中書門下,參掌機密,遂為尚書左僕射。永徽元年,求解僕射,聽之,仍以開府儀同三司知政事。四年,冊進司空。始太宗時,勣已畫象凌煙閣,至是,帝復命圖其形,自序之。又詔得乘小馬出入東、西臺,卑官日一人迎送。



 帝欲立武昭儀為皇后,畏大臣異議,未決。李義府、許敬宗又請廢王皇后。帝召勣與長孫無忌、於志寧、褚遂良計之,勣稱疾不至。帝曰:「皇后無子。罪莫大於絕嗣,將廢之。」遂良等持不可,志寧顧望不對。帝後密訪勣,曰:「將立昭儀,而顧命之臣皆以為不可,今止矣!」答曰:「此陛下家事,無須問外人。」帝意遂定,而王後廢。詔勣、志寧奉冊立武氏。帝東封泰山,為封禪大使。嘗墜馬傷足,帝以所乘馬賜之。



 高麗莫離支男生為其弟所逐,遣子乞師。詔勣為遼東道行軍大總管,率兵二萬討之。破其國,執高藏、男建等,裂其地州縣之。詔勣獻俘昭陵,明先帝意,具軍容告於廟。進位太子太師,增食千一百戶。



 總章二年,卒,年八十六。帝曰:「勣奉上忠,事親孝,歷三朝未嘗有過,性廉慎,不立產業。今亡,當無贏貲。有司其厚賵恤之。」因泣下。舉哀光順門,七日不視朝。贈太尉、揚州大都督,謚貞武。給秘器,陪葬昭陵。起塚象陰、鐵、烏德鞬山,以旌功烈。葬日,帝與皇太子幸未央古城,哭送,百官送古城西北。



 初,勣拔黎陽倉,就食者眾,高季輔、杜正倫往客焉,及平虎牢,獲戴胄,咸引見臥內,推禮之,後皆為名臣,世以勣知人。平洛陽,得單雄信,故人也。表其材武,且言:「若貸死,必有以報,請納官爵以贖。」不許。乃號慟,割股肉啗之曰:「生死永訣,此肉同歸於土!」為收養其子焉。性友愛,其姊病,嘗自為粥而燎其須。姊戒止。答曰:「姊多疾,而勣且老,雖欲數進粥,尚幾何?」



 其用兵多籌算,料敵應變,皆契事機。聞人善,抵掌嗟嘆。及戰勝,必推功於下。得金帛,盡散之士卒,無私貯。然持法嚴,故人為之用。臨事選將,必訾相其奇厖福艾者遣之。或問故,答曰:「薄命之人,不足與成功名。」既沒,士皆為流涕。



 自屬疾,帝及皇太子賜藥即服,家欲呼醫巫,不許。諸子固以藥進,輒曰:「我山東田夫耳,位三公,年逾八十,非命乎!生死系天,寧就醫求活耶?」弟弼,始為晉州刺史。以勣疾,召為司衛卿,使省視。忽語曰:「我似少愈,可置酒相樂。」於是奏樂宴飲,列子孫於下。將罷,謂弼曰:「我即死,欲有言,恐悲哭不得盡,故一訣耳!我見房玄齡、杜如晦、高季輔皆辛苦立門戶,亦望詒後,悉為不肖子敗之。我子孫今以付汝,汝可慎察,有不厲言行、交非類者,急榜殺以聞,毋令後人笑吾,猶吾笑房、杜也。我死,布裝露車載柩,斂以常服,加朝服其中,儻死有知,庶著此奉見先帝。明器惟作五六寓馬,下帳施幔,為皁頂白紗裙,中列十偶人,它不得以從。眾妾願留養子者聽,餘出之。葬已,徙居我堂,善視小弱。茍違我言,同戮尸矣!」乃不復語。弼等遵焉。勣本二名,至高宗時,避太宗偏諱,故但名勣。後配享高宗廟廷。



 季弟感,年十五,有奇操。李密敗,陷於世充。世充令作書召勣,對曰:「兄尚節義,今巳事主,昆弟不能移也。」固不從,殺之。勣子震嗣,終桂州刺史。震子敬業、敬猷。



 敬業,少從勣征伐,有勇名。歷太僕少卿,襲英國公,為眉州刺史。嗣聖元年,坐贓,貶柳州司馬。會給事中唐之奇貶括蒼令,詹事府司直杜求仁貶黝令,長安主簿駱賓王貶臨海丞,敬猷自盩厔令坐事免,俱客揚州,失職怏怏。



 時武后既廢中宗,又立睿宗,實亦囚之。諸武擅命,唐子孫誅戮,天下憤之。敬業等乘人怨,謀起兵,先諭其黨監察御史薛璋,求使江都。及至,令雍人韋超告州長史陳敬之反,璋乃收系之。敬業即矯制殺敬之,自稱州司馬,且言奉密詔募兵,討高州叛酋。即開府庫,令參軍李宗臣釋系囚、役工數百人,授甲,斬錄事參軍孫處行以徇。乃開三府,一曰匡復府,二曰英公府,三曰揚州大都督府。自稱匡復府上將,領揚州大都督,以子奇為左長史,求仁右長史,宗臣左司馬,璋右司馬,江都令韋知止為英公府長史,賓王為藝文令,前盩厔尉魏思溫為軍師。旬日,兵十餘萬。傳檄州縣,疏武氏過惡,復廬陵王天子位。又索狀類太子賢者奉之,詭眾曰:「賢實不死。」楚州司馬李崇福率所部三縣應之。



 武后遣左玉鈐衛大將軍李孝逸兵三十萬往擊之,削其祖父官爵,毀塚藏,除屬籍,赦揚、楚民脅從者。購得敬業首,授官三品,賞帛五千;得之奇等首,官五品,帛三千。



 敬業問計於思溫,對曰:「公既以太后幽縶天子,宜身自將兵直趨洛陽。山東、韓、魏知公勤王,附者必眾,天下指日定矣!」璋曰:「不然。金陵負江,其地足以為固。且王氣尚在,宜先並常、潤為霸基,然後鼓行而北。」思溫曰:「鄭、汴、徐、亳士皆豪傑,不願武后居上,蒸麥為飯,以待我師。奈何欲守金陵,投死地乎?」敬業不從。使敬猷屯淮陰,韋超屯都梁山,自引兵擊潤州,下之。署宗臣為刺史。始回兵屯高郵,下阿溪。思溫嘆曰:「兵忌分,今敬業不知掃地度淮,率山東士先襲東都,吾知無能為也!」



 武後又使黑齒常之將江南兵為孝逸援,進擊,淮陰、都梁兵皆敗。後軍總管蘇孝祥率奇兵五千夜度擊敬業,孝祥死,兵溺者過半,孝逸軍退守石梁。有鳥群噪敬業營上,監軍御史魏真宰曰:「賊其敗乎!風順荻乾,火攻之利也。」固請戰,遂度溪擊之。敬業置陣久,士疲,皆顧望不正列,孝逸乘風縱火逼其軍,軍稍卻。敬業麾精兵居前,弱者在後,陣亂不能制,乃敗,斬七千餘級。敬業與敬猷、之奇、求仁、賓王輕騎遁江都,悉焚其圖籍,攜妻子奔潤州,潛蒜山下,將入海逃高麗,抵海陵,阻風遺山江中,其將王那相斬之,凡二十五首,傳東都,皆夷其家。中宗反正,詔還勣官封屬籍,葺完塋塚焉。



 初,敬業之叔思文為潤州刺史。敬業兵起,以使間道聞,固守逾月。城陷,敬業責曰:「廬陵王繼天下,無罪見廢,今兵以義動,何過拒邪?若太后是助,宜即姓武。」思溫等欲殺之,敬業不許。及揚、楚平,乃獨免。後遂賜武姓,歷春官尚書。或言本與敬業謀者,乃復徐氏,卒。子欽憲,開元中,仕至國子祭酒。



 贊曰:「唐興,其名將曰英、衛,皆擢罪亡之餘,遂能依乘風雲,勒功帝籍。蓋君臣之際,固有以感之,獨推期運,非也。若靖闔門稱疾,畏遠權逼,功大而主不疑,雖古哲人,何以尚茲?勣之節,見於黎陽,故太宗勤勤於托孤,誠有為也。至以老臣輔少主,會房帷易奪,天子畏大臣,依違不專,委誠取決,惟議是聽。勣乃私己畏禍,從而導之,武氏奮而唐之宗屬幾殲焉。及其孫,因民不忍,舉兵覆宗,至掘塚而暴其骨。嗚呼,不幾一言而喪邦乎?惜其不通學術,昧夫臨大節不可奪之誼,反與許、李同科,可不戒哉!世言靖精風角、鳥占、雲祲、孤虛之術,為善用兵。是不然,特以臨機果,料敵明,根於忠智而已。俗人傅著怪詭禨祥,皆不足信。故列靖所設施如此。



\end{pinyinscope}