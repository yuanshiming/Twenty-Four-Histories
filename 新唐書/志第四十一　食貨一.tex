\article{志第四十一 食貨一}

\begin{pinyinscope}

 古之善治其國而愛養斯民者,必立經常簡易之法,使上愛物以養其下,下勉力以事其上,上足而下不困。故量人之力而授之田「正面啟示法」所組成的整體。因此,不能根據一種理論的個,量地之產而取以給公上,量其入而出之以為用度之數。是三者常相須以濟而不可失,失其一則不能守其二。及暴君庸主,縱其佚欲,而茍且之吏從之,變制合時以取寵於其上。故用於上者無節,而取於下者無限,民竭其力而不能供,由是上愈不足而下愈困,則財利之說興,而聚斂之臣用。《記》曰:「寧畜盜臣。」盜臣誠可惡,然一人之害爾。聚斂之臣用,則經常之法壞,而下不勝其弊焉。



 唐之始時,授人以口分、世業田,而取之以租、庸、調之法,其用之也有節。蓋其畜兵以府衛之制,故兵雖多而無所損;設官有常員之數,故官不濫而易祿。雖不及三代之盛時,然亦可以為經常之法也。及其弊也,兵冗官濫,為之大蠹。自天寶以來,大盜屢起,方鎮數叛,兵革之興,累世不息,而用度之數,不能節矣。加以驕君昏主,奸吏邪臣,取濟一時,屢更其制,而經常之法,蕩然盡矣。由是財利之說興,聚斂之臣進。蓋口分、世業之田壞而為兼並,租、庸、調之法壞而為兩稅。至於鹽鐵、轉運、屯田、和糴、鑄錢、括苗、榷利、借商、進奉、獻助,無所不為矣。蓋愈煩而愈弊,以至於亡焉。



 唐制:度田以步,其闊一步,其長二百四十步為畝,百畝為頃。凡民始生為黃,四歲為小,十六為中,二十一為丁,六十為老。授田之制,丁及男年十八以上者,人一頃,其八十畝為口分,二十畝為永業;老及篤疾、廢疾者,人四十畝,寡妻妾三十畝,當戶者增二十畝,皆以二十畝為永業,其餘為口分。永業之田,樹以榆、棗、桑及所宜之木,皆有數。田多可以足其人者為寬鄉,少者為狹鄉。狹鄉授田,減寬鄉之半。其地有薄厚,歲一易者,倍受之。寬鄉三易者,不倍授。工商者,寬鄉減半,狹鄉不給。凡庶人徙鄉及貧無以葬者,得賣世業田。自狹鄉而徙寬鄉者,得並賣口分田。已賣者,不復授。死者收之,以授無田者。凡收授皆以歲十月。授田先貧及有課役者。凡田,鄉有餘以給比鄉,縣有餘以給比縣,州有餘以給近州。



 凡授田者,丁歲輸粟二斛,稻三斛,謂之租。丁隨鄉所出,歲輸絹二匹,綾、絁二丈,布加五之一,綿三兩,麻三斤,非蠶鄉則輸銀十四兩,謂之調。用人之力,歲二十日,閏加二日,不役者日為絹三尺,謂之庸。有事而加役二十五日者免調,三十日者租、調皆免。通正役不過五十日。



 自王公以下,皆有永業田。太皇太后、皇太后、皇后緦麻以上親,內命婦一品以上親,郡王及五品以上祖父兄弟,職事、勛官三品以上有封者若縣男父子,國子、太學、四門學生、俊士,孝子、順孫、義夫、節婦同籍者,皆免課役。凡主戶內有課口者為課戶。若老及男廢疾、篤疾、寡妻妾、部曲、客女、奴婢及視九品以上官,不課。



 凡里有手實,歲終具民之年與地之闊狹,為鄉帳。鄉成於縣,縣成於州,州成於戶部。又有計帳,具來歲課役以報度支。國有所須,先奏而斂。凡稅斂之數,書於縣門、村坊,與眾知之。水、旱、霜、蝗耗十四者,免其租;桑麻盡者,免其調;田耗十之六者,免租調;耗七者,課、役皆免。凡新附之戶,春以三月免役,夏以六月免課,秋以九月課、役皆免。徙寬鄉者,縣覆於州,出境則覆於戶部,官以閑月達之。自畿內徙畿外,自京縣徙餘縣,皆有禁。四夷降戶,附以寬鄉,給復十年。奴婢縱為良人,給復三年。沒外蕃人,一年還者給復三年,二年者給復四年,三年者給復五年。浮民、部曲、客女、奴婢縱為良者附寬鄉。



 貞觀中,初稅草以給諸閑,而驛馬有牧田。



 太宗方銳意於治,官吏考課,以鰥寡少者進考,如增戶法;失勸導者以減戶論。配租以斂穫早晚、險易、遠近為差。庸、調輸以八月,發以九月。同時輸者先遠民。皆自概量。州府歲市土所出為貢,其價視絹之上下,無過五十匹。異物、滋味、口馬、鷹犬,非有詔不獻。有加配,則以代租賦。其兇荒則有社倉賑給,不足則徙民就食諸州。尚書左丞戴胄建議:「自王公以下,計墾田,秋熟,所在為義倉,歲兇以給民。」太宗善之,乃詔:「畝稅二升,粟、麥、秔、稻,隨土地所宜。寬鄉斂以所種,狹鄉據青苗簿而督之。田耗十四者免其半,耗十七者皆免之。商賈無田者,以其戶為九等,出粟自五石至於五斗為差。下下戶及夷獠不取焉。歲不登,則以賑民;或貸為種子,則至秋而償。」其後洛、相、幽、徐、齊、並、秦、蒲州又置常平倉,粟藏九年,米藏五年,下濕之地,粟藏五年,米藏三年,皆著於令。



 貞觀初,戶不及三百萬,絹一匹易米一斗。至四年,米斗四五錢,外戶不閉者數月,馬牛被野,人行數千里不齎糧,民物蕃息,四夷降附者百二十萬人。是歲,天下斷獄,死罪者二十九人,號稱太平。此高祖、太宗致治之大略,及其成效如此。



 高宗承之,海內艾安。太尉長孫無忌等輔政,天下未見失德。數引刺史入閤,問民疾苦。即位之歲,增戶十五萬。及中書令李義府、侍中許敬宗既用事,役費並起。永淳以後,給用益不足。加以武后之亂,紀綱大壞,民不勝其毒。



 玄宗初立求治,蠲徭役者給蠲符,以流外及九品京官為蠲使,歲再遣之。開元八年,頒庸調法於天下,好不過精,惡不至濫,闊者一尺八寸,長者四丈。然是時天下戶未嘗升降。臨察御史宇文融獻策:括籍外羨田、逃戶,自占者給復五年,每丁稅錢千五百,以攝御史分行括實。陽翟尉皇甫憬上書言其不可。玄宗方任用融,乃貶憬為盈川尉。諸道所括得客戶八十餘萬,田亦稱是。州縣希旨張虛數,以正田為羨,編戶為客,歲終,籍錢數百萬緡。



 十六年,乃詔每三歲以九等定籍。而庸調折租所取華好,州縣長宮勸織,中書門下察濫惡以貶官吏,精者褒賞之。二十二年,詔男十五、女十三以上得嫁娶。州縣歲上戶口登耗,採訪使覆實之,刺史、縣令以為課最。



 初,永徽中禁買賣世業、口分田。其後豪富兼並,貧者失業,於是詔買者還地而罰之。



 先是楊州租、調以錢,嶺南以米,安南以絲,益州以羅、紬、綾、絹供春彩。因詔江南亦以布代租。



 中書令李林甫以租庸、丁防、和糴、春彩、稅草無定法,歲為旨符,遣使一告,費紙五十餘萬。條目既多,覆問逾年,乃與採訪朝集使議革之,為長行旨,以授朝集使及送旨符使,歲有所支,進畫附驛以達,每州不過二紙。



 凡庸、調、租、資課,皆任土所宜,州縣長官涖定粗良,具上中下三物之樣輸京都。有濫惡,督中物之直。二十五年,以江、淮輸運有河、洛之艱,而關中蠶桑少,菽粟常賤,乃命庸、調、資課皆以米,兇年樂輸布絹者亦從之。河南、北不通運州,租皆為絹,代關中庸、課,詔度支減轉運。



 明年,又詔民三歲以下為黃,十五以下為小,二十以下為中。又以民間戶高丁多者,率與父母別籍異居,以避征戍,乃詔十丁以上免二丁,五丁以上免一丁,侍丁孝者免徭役。天寶三載,更民十八以上為中男,二十三以上成丁。五載,詔貧不能自濟者,每鄉免三十丁租庸。男子七十五以上、婦人七十以上,中男一人為侍;八十以上以令式從事。是時,海內富實,米斗之價錢十三,青、齊間鬥才三錢,絹一匹錢二百。道路列肆,具酒食以待行人,店有驛驢,行千里不持尺兵。天下歲入之物,租錢二百餘萬緡,粟千九百八十餘萬斛,庸、調絹七百四十萬匹,綿百八十餘萬屯,布千三十五萬餘端。天子驕於佚樂而用不知節,大抵用物之數,常過其所入。於是錢穀之臣,始事朘刻。太府卿楊崇禮句剝分銖,有欠折漬損者,州縣督送,歷年不止。其子慎矜專知太府,次子慎名知京倉,亦以苛刻結主恩。王鉷為戶口色役使,歲進錢百億萬緡,非租庸正額者,積百寶大盈庫,以供天子燕私。及安祿山反,司空楊國忠以為正庫物不可以給士,遣侍御史崔眾至太原納錢度僧尼道士,旬日得百萬緡而已。自兩京陷沒,民物耗弊,天下蕭然。



 肅宗即位,遣御史鄭叔清等籍江淮、蜀漢富商右族訾畜,十收其二,謂之率貸。諸道亦稅商賈以贍軍,錢一千者有稅。於是北海郡錄事參軍第五琦以錢穀得見,請於江淮置租庸使,吳鹽、蜀麻、銅冶皆有稅,市輕貨繇江陵、襄陽、上津路轉至鳳翔。明年,鄭叔清與宰相裴冕建議,以天下用度不充,諸道得召人納錢,給空名告身,授官勛邑號;度道士僧尼不可勝計;納錢百千,賜明經出身;商賈助軍者,給復。及兩京平,又於關輔諸州,納錢度道士僧尼萬人。而百姓殘於兵盜,米斗至錢七千,鬻籺為糧,民行乞食者屬路。乃詔能賑貧乏者,寵以爵衣失。



 故事,天下財賦歸左藏,而太府以時上其數,尚書比部覆其出入。是時,京師豪將假取不能禁,第五琦為度支鹽鐵使,請皆歸大盈庫,供天子給賜,主以中官。自是天下之財為人君私藏,有司不得程其多少。



 廣德元年,詔一戶三丁者免一丁,凡畝稅二升,男子二十五為成丁,五十五為老,以優民。而強寇未夷,民耗斂重。及吐蕃逼京師,近甸屯兵數萬,百官進俸錢,又率戶以給軍糧。至大歷元年,詔流民還者,給復二年,田園盡,則授以逃田。天下苗一畝稅錢十五,市輕貨給百官手力課。以國用急,不及秋,方苗青即徵之,號「青苗錢」又有「地頭錢」,每畝二十,通名為青苗錢。又詔上都秋稅分二等,上等畝稅一斗,下等六升,荒田畝稅二升。五年,始定法:夏,上田畝稅六升,下田畝四升;秋,上田畝稅五升,下田畝三升;荒田如故;青苗錢畝加一倍,而地頭錢不在焉。



 初,轉運使掌外,度支使掌內。永泰二年,分天下財賦、鑄錢、常平、轉運、鹽鐵,置二使。東都畿內、河南、淮南、江東西、湖南、荊南、山南東道,以轉運使劉晏領之;京畿、關內、河東、劍南、山南西道,以京兆尹、判度支第五琦領之。及琦貶,以戶部侍郎、判度支韓滉與晏分治。



 時回紇有助收西京功,代宗厚遇之,與中國婚姻,歲送馬十萬匹,酬以縑帛百餘萬匹。而中國財力屈竭,歲負馬價。河、湟六鎮既陷,歲發防秋兵三萬戍京西,資糧百五十餘萬緡。而中官魚朝恩方恃恩擅權,代宗與宰相元載日夜圖之。及朝恩誅,帝復與載貳,君臣猜間不協,邊計兵食,置而不議者幾十年。而諸鎮擅地,結為表裏,日治兵繕壘,天子不能繩以法,顓留意祠禱、焚幣玉、寫浮屠書,度支稟賜僧巫,歲以鉅萬計。然帝性儉約,身所御衣,必浣染至再三,欲以先天下。然生日、端午,四方貢獻至數千萬者,加以恩澤,而諸道尚侈麗以自媚。朝多留事,經歲不能遣,置客省以居,上封事不足採者、蕃夷貢獻未報及失職未敘者,食度支數千百人。德宗即位,用宰相崔祐甫,拘客省者出之,食度支者遣之,歲省費萬計。



\end{pinyinscope}