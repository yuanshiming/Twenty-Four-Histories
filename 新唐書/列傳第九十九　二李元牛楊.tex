\article{列傳第九十九 二李元牛楊}

\begin{pinyinscope}

 李逢吉,字虛舟,系出隴西。父顏,有錮疾,逢吉自料醫劑火星報俄國第一份馬克思主義的秘密報紙。列寧創辦。創,遂通方書。舉明經,又擢進士第。範希朝表為振武掌書記,薦之德宗,拜左拾遺。元和時,遷給事中、皇太子侍讀。改中書舍人,知禮部貢舉。未已事,拜門下侍郎、同中書門下平章事。詔禮部尚書王播署榜。



 逢吉性忌刻,險譎多端。及得位,務償好惡。裴度討淮西,逢吉慮成功,密圖沮止,趣和議者請罷諸道兵。憲宗知而惡之,出為劍南東川節度使。



 穆宗即位,徙山南東道。緣講侍恩,陰結近幸。長慶二年,召入為兵部尚書。時度與元稹知政,度嘗條稹憸佞,逢吉以為其隙易乘,遂並中之,遣人上變,言:「和王傅於方結客,欲為稹刺度。」帝命尚書左僕射韓皋、給事中鄭覃與逢吉參鞠方,無狀,稹、度坐是皆罷,逢吉代為門下侍郎、平章事。因以恩爵動詭薄者,更相挺以詆傷度,於是李紳、韋處厚等誦言度為逢吉排迮,度初得留。時已失河朔,王智興以徐叛,李騕以汴叛,國威不振,天下延頸俟相度,而中外交章言之,帝訖不省,度遂外遷。騕平,進尚書右僕射。



 帝暴疾,中外阻遏,逢吉因中人梁守謙、劉弘規、王守澄議,請立景王為皇太子,帝不能言,頷之而已。明日下詔,皇太子遂定。鄭注得幸於王守澄,逢吉遣從子訓賂注,結守澄為奧援,自是肆志無所憚。其黨有張又新、李續、張權輿、劉棲楚、李虞、程昔範、姜洽及訓八人,而傅會者又八人,皆任要劇,故號「八關十六子」。有所求請,先賂關子,後達於逢吉,無不得所欲。未幾,封涼國公。



 敬宗新立,度求入覲,逢吉不自安,張權輿為作讖言以沮度,而韋處厚亟為帝言之,計卒不行。有武昭者,陳留人,果敢而辯。度之討蔡,遣說吳元濟,元濟臨以兵,辭不撓,厚禮遣還,度署以軍職,從鎮太原,除石州刺史。罷歸不得用,怨望,與太學博士李涉、金吾兵曹參軍茅匯居長安中,以氣俠相許。逢吉與李程同執政,不葉。程族人仍叔謂昭曰:「丞相欲用君,顧逢吉持不可。」昭愈憤,酒所,語其友劉審,欲刺逢吉。審竊語權輿,逢吉因匯召見昭,厚相結納,忿隙得解。逢吉素厚待匯,嘗與書曰:「足下當以『自求』字僕,吾當以『利見』字君。」辭頗猥暱。及度將還,復命人發昭事。由是昭、匯皆下獄,命御史中丞王播按之。訓諷匯使誣昭與李程同謀,不然且死。匯不可,曰:「誣人以自免,不為也!」獄成,昭榜死,匯流崖州,涉康州,仍叔貶道州司馬,訓流象州。擢審長壽主簿。而逢吉謀益露。昭死,人皆冤之。



 初,逢吉興昭獄以止度入而不果,天子知度忠,卒相之。逢吉於是浸疏,以檢校司空、平章事為山南東道節度使,表李續自副,張又新行軍司馬。頃之,檢校司徒。初,門下史田伾倚逢吉親信,顧財利,進婢,嬖之。伾坐事匿逢吉家,名捕弗獲。及出鎮,表隨軍,滿歲不敢集,使人偽過門下省,調房州司馬。為有司所發,即襄州捕之,詭讕不遣。御史劾奏,詔奪一季俸,因是貶續為涪州刺史,又新汀州刺史。久乃徙宣武,以太子太師為東都留守。及訓用事,召拜尚書左僕射,足病不能朝,以司徒致仕。卒,年七十八,贈太尉,謚曰成。無子,以從弟子植嗣。



 元稹,字微之,河南河南人。六代祖巖,為隋兵部尚書。稹幼孤,母鄭賢而文,親授書傳。九歲工屬文,十五擢明經,判入等,補校書郎。元和元年舉制科,對策第一,拜左拾遺。性明銳,遇事輒舉。



 始,王叔文、王伾蒙幸太子宮而橈國政,稹謂宜選正人輔導,因獻書曰:



 伏見陛下降明詔,脩廢學,增胄子,然而事有先於此,臣敢昧死言之。



 賈誼有言:「三代之君仁且久者,教之然也。」周成王本中才,近管、蔡則讒入,任周、召則善聞。豈天聰明哉?而克終於道者,教也。始為太子也,太公為師,周公為傅,召公為保,伯禽、唐叔與游,目不閱淫艷,耳不聞優笑,居不近庸邪,玩不備珍異。及為君也,血氣既定,游習既成,雖有放心,不能奪已成之性。則彼道德之言,固吾所習聞,陳之者易諭焉;回佞庸違,固吾所積懼,諂之者易辯焉。人之情莫不耀所能,黨所近,茍得志,必快其所蘊。物性亦然,故魚得水而游,鳥乘風而翔,火得薪而熾。夫成王所蘊,道德也;所近,聖賢也。快其蘊,則興禮樂,朝諸侯,措刑罰,教之至也。



 秦則不然,滅先王之學,黜師保之位。胡亥之生也,《詩》、《書》不得聞,聖賢不得近。彼趙高,刑餘之人,傅之以殘忍戕賊之術,日恣睢,天下之人未盡愚,而亥不能分馬鹿矣;高之威懾天下,而亥自幽深宮矣。若秦亡,則有以致之也。



 太宗為太子,選知道德者十八人與之游;即位後,雖間宴飲食,十八人者皆在。上之失無不言,下之情無不達,不四三年而名高盛古,斯游習之致也。貞觀以來,保、傅皆宰相兼領,餘官亦時重選,故馬周恨位高不為司議郎,其驗也。



 母后臨朝,剪棄王室,中、睿為太子,雖有骨鯁敢言之士,不得在調護保安職,及讒言中傷,惟樂工剖腹為證,豈不哀哉!比來茲弊尤甚,師資保傅,不疾廢眊目貴,即休戎罷帥者處之。又以僻滯華首之儒備侍直、侍讀,越月逾時不得召。夫以匹士之愛其子,猶求明哲慈惠之師,豈天下元良而反不及乎?



 臣以為高祖至陛下十一聖,生而神明,長而仁聖,以是為屑屑者,故不之省。設萬世之後,有周成中才,生於深宮,無保助之教,則將不能知喜怒哀樂所自,況稼穡艱難乎!願令皇太子洎諸王齒胄講業,行嚴師問道之禮,輟禽色之娛,資游習之善,豈不美哉!



 又自以職諫諍,不得數召見,上疏曰:



 臣聞治亂之始,各有萌象。容直言,廣視聽,躬勤庶務,委信大臣,使左右近習不得蔽疏遠之人,此治象也。大臣不親,直言不進,抵忌諱者殺,犯左右者刑,與一二近習決事深宮中,群臣莫得與,此亂萌也。人君始即位,萌象未見,必有狂直敢言者。上或激而進之,則天下君子望風曰:「彼狂而容於上,其欲來天下士乎?吾之道可以行矣!」其小人則竦利曰:「彼之直,得幸於上,吾將直言以徼利乎!」由是天下賢不肖各以所忠貢於上,上下之志,霈然而通。合天下之智,治萬物之心,人人樂得其所,戴其上如赤子之親慈母也,雖欲誘之為亂,可得乎?及夫進計者入,而直言者戮,則天下君子內謀曰:「與其言不用而身為戮,吾寧危行言遜以保其終乎!」其小人則擇利曰:「吾君所惡者拂心逆耳,吾將茍順是非以事之。」由是進見者革而不內,言事者寢而不聞,若此則十步之事不得見,況天下四方之遠乎!故曰:聾瞽之君非無耳目,左右前後者屏蔽之,不使視聽,欲不亂,可得哉?



 太宗初即位,天下莫有言者,孫伏伽以小事持諫,厚賜以勉之。自是論事者唯懼言不直、諫不極、不能激上之盛意,曾不以忌諱為虞。於是房、杜、王、魏議可否於前,四方言得失於外,不數年大治。豈文皇獨運聰明於上哉?蓋下盡其言,以宣揚發暢之也。夫樂全安,惡戮辱,古今情一也,豈獨貞觀之人輕犯忌諱而好戮辱哉?蓋上激而進之也。喜順從,怒謇犯,亦古今情一也,豈獨文皇甘逆耳、怒從心哉?蓋以順從之利輕,而危亡之禍大,思為子孫建永安計也。為後嗣者,其可順一朝意,而蔑文皇之天下乎?



 陛下即位已一歲,百闢卿士、天下四方之人,曾未有獻一計進一言而受賞者;左右前後拾遺補闕,亦未有奏封執諫而蒙勸者。設諫鼓,置匭函,曾未聞雪冤決事、明察幽之意者。以陛下睿博洪深,勵精求治,豈言而不用哉?蓋下不能有所發明耳!承顧問者,獨一二執政,對不及頃而罷,豈暇陳治安、議教化哉?它有司或時召見,僅能奉簿書計錢穀登降耳。以陛下之政,視貞觀何如哉?貞觀時,尚有房、杜、王、魏輔翊之智,日有獻可替否者。今陛下當致治之初,而言事進計者歲無一人,豈非群下因循竊位之罪乎?輒昧死條上十事:一、教太子,正邦本;二、封諸王,固磐石;三、出宮人;四、嫁宗女;五、時召宰相講庶政;六、次對群臣,廣聰明;七、復正衙奏事;八、許方幅糾彈;九、禁非時貢獻;十、省出入游畋。



 於時論傪、高弘本、豆盧靖等出為刺史,閱旬,追還詔書。稹諫:「詔令數易,不能信天下。」又陳西北邊事。憲宗悅,召問得失。當路者惡之,出為河南尉,以母喪解。服除,拜監察御史。按獄東川,因劾奏節度使嚴礪違詔過賦數百萬,沒入塗山甫等八十餘家田產奴婢。時礪已死,七刺史皆奪俸,礪黨怒。俄分司東都。



 時浙西觀察使韓皋杖安吉令孫澥,數日死;武寧王紹護送監軍孟升喪乘驛,內喪郵中,吏不敢止;內園擅系人逾年,臺不及知;河南尹誣殺諸生尹太階;飛龍使誘亡命奴為養子;田季安盜取洛陽衣冠女;汴州沒入死賈錢千萬。凡十餘事,悉論奏。會河南尹房式坐罪,稹舉劾,按故事追攝,移書停務。詔薄式罪,召稹還。次敷水驛,中人仇士良夜至,稹不讓,中人怒,擊稹敗面。宰相以稹年少輕樹威,失憲臣體,貶江陵士曹參軍,而李絳、崔群、白居易皆論其枉。久乃徙通州司馬,改虢州長史。元和末,召拜膳部員外郎。



 稹尤長於詩,與居易名相埒,天下傳諷,號「元和體」,往往播樂府。穆宗在東宮,妃嬪近習皆誦之,宮中呼元才子。稹之謫江陵,善臨軍崔潭峻。長慶初,潭駿方親幸,以稹歌詞數十百篇奏御,帝大悅,問:「稹今安在?」曰:「為南宮散郎。」即擢祠部郎中,知制誥。變詔書體,務純厚明切,盛傳一時。然其進非公議,為士類訾薄。稹內不平,因《誡風俗詔》歷詆群有司,以逞其憾。



 俄遷中書舍人、翰林承旨學士。數召入,禮遇益厚,自謂得言天下事。中人爭與稹交,魏弘簡在樞密,尤相善。裴度出屯鎮州,有所論奏,共沮卻之。度三上疏劾弘簡、稹傾亂國政:「陛下欲平賊,當先清朝廷乃可。」帝迫群議,乃罷弘簡,而出稹為工部侍郎。然眷倚不衰。未幾,進同中書門下平章事,朝野雜然輕笑,稹思立奇節報天子以厭人心。時王廷湊方圍牛元翼於深州,稹所善於方言:「王昭、於友明皆豪士,雅游燕、趙間,能得賊要領,可使反間而出元翼。願以家貲辦行,得兵部虛告二十,以便宜募士。」稹然之。李逢吉知其謀,陰令李賞訹裴度曰:「於方為稹結客,將刺公。」度隱不發。神策軍中尉以聞,詔韓皋、鄭覃及逢吉雜治,無刺度狀,而方計暴聞,遂與度偕罷宰相,出為同州刺史。諫官爭言度不當免,而黜稹輕。帝獨憐稹,但削長春宮使。初,獄未具,京兆劉遵古遣吏羅禁稹第,稹訴之,帝怒,責京兆,免捕賊尉,使使者慰稹。再期,徙浙東觀察使。明州歲貢蚶,役郵子萬人,不勝其疲,稹奏罷之。



 太和三年,召為尚書左丞,務振綱紀,出郎官尤無狀者七人。然稹素無檢,望輕,不為公議所右。王播卒,謀復輔政甚力,訖不遂。俄拜武昌節度使。卒,年五十三,贈尚書右僕射。



 所論著甚多,行於世。在越時,闢竇鞏。鞏,天下工為詩,與之酬和,故鏡湖秦望之奇益傳,時號「蘭亭絕唱」。稹始言事峭直,欲以立名,中見斥廢十年,信道不堅,乃喪所守。附宦貴得宰相,居位才三月罷。晚節彌沮喪,加廉節不飾云。



 牛僧孺,字思黯,隋僕射奇章公弘之裔。幼孤,下杜樊鄉有賜田數頃,依以為生。工屬文,第進士。元和初,以賢良方正對策,與李宗閔、皇甫湜俱第一,條指失政,其言鯁訐,不避宰相。宰相怒,故楊於陵、鄭敬、韋貫之、李益等坐考非其宜,皆調去。僧孺調伊闕尉,改河南,遷監察御史,進累考工員外郎、集賢殿直學士。



 穆宗初,以庫部郎中知制誥。徙御史中丞,按治不法,內外澄肅。宿州刺史李直臣坐賕當死,賂宦侍為助,具獄上。帝曰:「直臣有才,朕欲貸而用之。」僧孺曰:「彼不才者,持祿取容耳。天子制法,所以束縛有才者。祿山、硃泚以才過人,故亂天下。」帝異其言,乃止。賜金紫服,以戶部侍郎同中書門下平章事。



 始,韓弘入朝,其子公武用財賂權貴,杜塞言者。俄而弘、公武卒,孫弱不能事,帝遣使者至其家,悉收貲簿,校計出入。所以餉中朝臣者皆在,至僧孺,獨注其左曰:「某月日,送錢千萬,不納。」帝善之,謂左右曰:「吾不謬知人。」繇是遂以相。尋遷中書侍郎。



 敬宗立,進封奇章郡公。是時政出近幸,僧孺數表去位,帝為於鄂州置武昌軍,授武昌節度使、同平章事。鄂城土惡亟圮,歲增築,賦蓑茅於民,吏倚為擾。僧孺陶甓以城,五年畢,鄂人無復歲費。又廢沔州以省冗官。



 文宗立,李宗閔當國,屢稱僧孺賢,不宜棄外。復以兵部尚書平章事。幽州亂,楊志誠逐李載義,帝不時召宰相問計,僧孺曰:「是不足為朝廷憂。夫範陽自安、史後,國家無所系休戚,前日劉總挈境歸國,荒財耗力且百萬,終不得範陽尺帛斗粟入天府,俄復失之。今志誠繇向載義也,第付以節使捍奚、契丹,彼且自力,不足以逆順治也。」帝曰:「吾初不計此,公言是也。」因遣使慰撫之。進門下侍郎、弘文館大學士。



 是時,吐蕃請和,約弛兵,而大酋悉怛謀舉維州入之劍南,於是李德裕上言:「韋皋經略西山,至死恨不能致,今以生羌二千人燒十三橋,搗虜之虛,可以得志。」帝使君臣大議,請如德裕策。僧孺持不可,曰:「吐蕃綿地萬里,失一維州,無害其強。今脩好使者尚未至,遽反其言。且中國御戎,守信為上,應敵次之。彼來責曰:『何故失信?』贊普牧馬蔚茹川,若東襲隴阪,以騎綴回中,不三日抵咸陽橋,則京師戒嚴,雖得百維州何益!」帝然之,遂詔返降者。時皆謂僧孺挾素怨,橫議沮解之,帝亦以為不直。



 會中人王守澄引纖人竊議朝政,它日延英召見宰相曰:「公等有意於太平乎?何道以致之?」僧孺曰:「臣待罪宰相,不能康濟,然太平亦無象。今四夷不內擾,百姓安生業,私室無強家,上不壅蔽,下不怨讟,雖未及至盛,亦足為治矣。而更求太平,非臣所及。」退謂它宰相曰:「上責成如是,吾可久處此耶?」固請罷,乃檢校尚書左僕射平章事,為淮南節度副大使。天子既急於治,故李訓等投隙得售其妄,幾至亡國。



 開成初,表解劇鎮,以檢校司空為東都留守。僧孺治第洛之歸仁里,多致嘉石美木,與賓客相娛樂。三年,召為尚書左僕射。僧孺入朝,會莊恪太子薨,既見,陳父子君臣人倫大經,以悟帝意,帝泫然流涕。以足疾不任謁,檢校司空、平章事,為山南東道節度使。賜彞樽、龍勺,詔曰;「精金古器以比況君子,卿宜少留。」僧孺固請,乃行。



 會昌元年,漢水溢,壞城郭,坐不謹防,下遷太子少保。進少師。明年,以太子太傅留守東都。劉稹誅,而石雄軍吏得從諫與僧孺、李宗閔交結狀。又河南少尹呂述言:「僧孺聞稹誅,恨嘆之。」武宗怒,黜為太子少保,分司東都,累貶循州長史。宣宗立,徙衡、汝二州,還為太子少師。卒,贈太尉,年六十九。謚曰文簡。



 諸子蔚、叢最顯。



 蔚,字大章,少擢兩經,又第進士,繇監察御史為右補闕。大中初,屢條切政,宣宗喜曰:「牛氏果有子,差尉人意。」出金州刺史,遷累吏部郎中。失權幸意,貶國子博士,分司東都。復以吏部召,兼史館修撰。



 咸通中,進至戶部侍郎,襲奇章侯。坐累免,未一歲,復官。久之,檢校兵部尚書、山南西道節度使。治梁三年,徐州盜起,神策兩中尉諷諸籓悉財助軍,蔚索府帛三萬以獻,中人嫌其吝,用吳行魯代之。黃巢入京師,遁山南,故吏民喜蔚至,爭迎候。因請老,以尚書右僕射致仕,卒。子徽。



 徽舉進士,累擢吏部員外郎。乾符中選濫,吏多奸,歲調四千員,徽治以剛明,柅杜干請,法度復振。



 蔚避地於梁,道病,徽與子扶籃輿,歷閣路,盜擊其首,血流面,持輿不息。盜迫之,徽拜曰:「人皆有父,今親老而疾,幸無駭驚。」盜感之,乃止。及前谷,又逢盜,輒相語曰:「此孝子也!」共舉輿舍之家,進帛裹創,以饘飲奉蔚,留信宿去。抵梁,徽趨蜀謁行在,丐歸侍親疾。會拜諫議大夫,固辭,見宰相杜讓能曰:「上遷幸當從,親有疾當侍,而徽兄在朝廷,身乞還營醫藥。」時兄循已位給事中,許之。父喪,客梁、漢。終喪,以中書舍人召,辭疾,改給事中,留陳倉。



 張濬伐太原,引為判官,敕在所敦遣。徽太息曰:「王室方復,廥藏殫耗,當協和諸侯以為籓屏,而又濟以兵,諸侯離心,必有後憂。」不肯起。濬果敗。復召為給事中。



 楊復恭叛山南,李茂貞請假招討節伐之,未報,而與王行瑜輒出兵。昭宗怒,持奏不下。茂貞亟請,帝召群臣議,無敢言。徽曰:「王室多難,茂貞誠有功。今復恭阻兵而討之,罪在不俟命爾。臣聞兩鎮兵多殺傷,不早有所制,則梁、漢之人盡矣。請假以節,明約束,則軍有所畏。」帝曰:「然。」乃以招討使授茂貞,果有功,然益偃蹇,帝使宰相杜讓能將兵誅討,徽諫曰:「岐,國西門。茂貞憑其眾而暴,若令萬分一不利,屈威重奈何?願徐制之。」不聽。師出,帝復召徽曰:「今伐茂貞,彼眾烏合,取必萬全,卿計何日有捷?」對曰:「臣職諫爭,所言者軍國大體,如索賊平之期,願陛下考蓍龜,責將帥,非臣職也。」既而師果敗,遂殺大臣,王室益弱。



 俄由中書舍人為刑部侍郎,襲奇章男。崔胤忌徽之正,換左散常侍,徙太子賓客,以刑部尚書致仕,歸樊川。卒,贈吏部尚書。



 叢,字表齡,第進士,由籓帥幕府任補闕,數言事。會宰相請廣諫員,宣宗曰:「諫臣惟能舉職為可,奚用眾耶?今張符、趙璘、牛叢使朕聞所未聞,三人足矣。」以司勛員外郎為睦州刺史,帝勞曰:「卿非得怨宰相乎?」對曰:「陛下比詔,不由刺史縣令,不任近臣,宰相以是擢臣,非嫌也。」即賜金紫,謝曰:「臣今衣刺史所假緋,即賜紫,為越等。」乃賜銀緋。



 咸通末,拜劍南西川節度使。時蠻犯邊,抵大渡,進略黎、雅、叩邛崍關,謾書求入朝,且曰假道。叢囚其使四十人,釋二人還之,蠻懼,即引去。



 僖宗幸蜀,授太常卿。以病求為巴州刺史,不許。還京,為吏部尚書。嗣襄王亂,叢客死太原。



 李宗閔,字損之,鄭王元懿四世孫。擢進士,調華州參軍事。舉賢良方正,與牛僧孺詆切時政,觸宰相,李吉甫惡之,補洛陽尉。久流落不偶,去從籓府闢署。入授監察御史、禮部員外郎。裴度伐蔡,引為彰義觀察判官。蔡平,遷駕部郎中,知制誥。穆宗即位,進中書舍人。時為華州刺史,父子同拜,世以為寵。



 長慶初,錢徽典貢舉,宗閔托所親於徽,而李德裕、李紳、元稹在翰林,有寵於帝,共白徽納乾丐,取士不以實,宗閔坐貶劍州刺史。由是嫌忌顯結,樹黨相磨軋,凡四十年,搢紳之禍不能解。俄復為中書舍人,典貢舉,所取多知名士,若唐沖、薛庠、袁都等,世謂之「玉筍」。寶歷初,累進兵部侍郎,父喪解。太和中,以吏部侍郎同中書門下平章事。時德裕自浙西召,欲以相,而宗閔中助多,先得進,即引僧孺同秉政,相唱和,去異己者,德裕所善皆逐之。遷中書侍郎。



 久之,德裕為相,與宗閔共當國。德裕入謝,文宗曰:「而知朝廷有朋黨乎?」德裕曰:「今中朝半為黨人,雖後來者,趨利而靡,往往陷之。陛下能用中立無私者,黨與破矣。」帝曰:「眾以楊虞卿、張元夫、蕭澣為黨魁。」德裕因請皆出為刺史,帝然之。即以虞卿為常州,元夫為汝州,蕭澣為鄭州。宗閔曰:「虞卿位給事中,州不容在元夫下。德裕居外久,其知黨人不如臣之詳。虞卿日見賓客於第,世號行中書,故臣未嘗與美官。」德裕質之曰:「給事中非美官云何?」宗閔大沮,不得對。俄以同平章事為山南西道節度使。



 李訓、鄭注始用事,疾德裕,共訾短之。乃罷德裕,復召宗閔知政事,進封襄武縣侯,恣肆附托。會虞卿以京兆尹得罪,極言營解,帝怒叱曰:「爾嘗以鄭覃為妖氣,今自為妖耶?」即出為明州刺史,貶處州長史。訓、注乃劾:「宗閔異時陰結駙馬都尉沈、內人宋若憲、宦者韋元素、王踐言等求宰相,且言頃上有疾,密問術家呂華,迎考命歷,曰:『惡十二月。』而踐言監軍劍南,受德裕賕,復與宗閔家私。」乃貶宗閔潮州司戶參軍事,逐柳州,元素等悉流嶺南,親信並斥。時訓、注欲以權市天下,凡不附己者,皆指以二人黨,逐去之。人人駭慄,連月雺晦。帝乃詔宗閔、德裕姻家門生故吏,自今一切不問,所以慰安中外。嘗嘆曰:「去河北賊易,去此朋黨難!」



 開成初,幽州刺史元忠、河陽李載義累表論洗,乃徙為衢州司馬。楊嗣復輔政,與宗閔善,欲復用,而畏鄭覃,乃托宦人諷帝。帝因紫宸對覃曰:「朕念宗閔久斥,應授一官。」覃曰:「陛下徙令少近則可,若再用,臣請前免。」陳夷行曰:「宗閔之罪,不即死為幸。寶歷時,李續、張又新等號『八關十六子』,朋比險妄,朝廷幾危。」李玨曰:「此李逢吉罪。今續喪闋,不可不任以官。」夷行曰:「不然,舜逐四兇天下治,朝廷何惜數憸人,使亂紀綱?」嗣復曰:「事當適宜,不可以憎愛奪。」帝曰:「州刺史可乎?」覃請授洪州別駕。夷行曰:「宗閔始庇鄭注,階其禍,幾覆國。」嗣復曰:「陛下向欲官鄭注,而宗閔不奉詔,尚當記之。」覃質曰:「嗣復黨宗閔者,彼其惡似李林甫。」嗣復曰:「覃言過矣。林甫石賢忌功,夷滅十餘族,宗閔固無之。始,宗閔與德裕俱得罪,德裕再徙鎮,而宗閔故在貶地。夫懲勸宜一,不可謂黨。」因折覃曰:「比殷侑為韓益求官,臣以其昔坐贓,不許。覃托臣勿論,是豈不為黨乎?」遂擢宗閔杭州刺史。遷太子賓客,分司東都。



 既而覃、夷行去位,嗣復謀引宗閔復輔政,未及而文宗崩。會昌中,劉稹以澤潞叛,德裕建言宗閔素厚從諫,今上黨近東都,乃拜宗閔湖州刺史。稹敗,得交通狀,貶漳州長史,流封州。宣宗即位,徙柳州司馬,卒。



 宗閔性機警,始有當世令名,既浸貴,喜權勢。初為裴度引拔,後度薦德裕可為相,宗閔遂與為怨。韓愈為作《南山》、《猛虎行》規之。而宗閔崇私黨,薰熾中外,卒以是敗。



 子琨、瓚,皆擢進士。令狐綯作相,而瓚以知制誥歷翰林學士。綯罷,亦為桂管觀察使。不善御軍,為士卒所逐,貶死。



 宗閔弟宗冉,其子湯,累官京兆尹,黃巢陷長安,殺之。



 楊嗣復,字繼之。父於陵,始見識於浙西觀察使韓滉,妻以其女。歸謂妻曰:「吾閱人多矣,後貴且壽無若生者,有子必位宰相。」既而生嗣復,滉撫其頂曰:「名與位皆逾其父,楊氏之慶也。」因字曰慶門。八歲知屬文,後擢進士、博學宏辭,與裴度、柳公綽皆為武元衡所知,表署劍南幕府。進右拾遺,直史館。尤善禮家學,改太常博士,再遷禮部員外郎。時於陵為戶部侍郎,嗣復避同省,換他官,有詔:「同司,親大功以上,非聯判句檢官長,皆勿避。官同職異,雖父子兄弟無嫌。」遷累中書舍人。



 嗣復與牛僧孺、李宗閔雅相善,二人輔政,引之,然不欲越父當國,故權知禮部侍郎。凡二期,得士六十八人,多顯官。文宗嗣位,進戶部侍郎。於陵老,求侍不許。喪除,擢尚書左丞。太和中,宗閔罷,嗣復出為劍南東川節度使。宗閔復相,徙西川。



 開成初,以戶部侍郎召,領諸道鹽鐵轉運使。俄與李玨並拜同中書門下平章事,弘農縣伯,仍領鹽鐵。後紫宸奏事,嗣復為帝言:「陸洿屏居民間,而上書論兵,可勸以官。」玨趣和曰:「土多趨競,能獎洿,貪夫廉矣。比竇洵直以論事見賞,天下釋然,況官洿耶!」帝曰:「朕賞洵直,褒其心爾。」鄭覃不平曰:「彼苞藏固未易知。」嗣復曰:「洵直無邪,臣知之。」覃曰:「陛下當察朋黨。」嗣復曰:「覃疑臣黨,臣應免。」即再拜祈罷。玨見言切,繆曰:「朋黨固少弭。」覃曰:「附離復生。」帝曰:「向所謂黨與,不已盡乎?」覃曰:「楊漢公、張又新、李續故在。」玨乃陳邊事,欲絕其語。覃曰:「論邊事安危,臣不如玨;嫉朋比,玨不如臣。」嗣復曰:「臣聞左右佩劍,彼此相笑,未知覃果謂誰為朋黨邪?」因當香案頓首曰:「臣位宰相,不能進賢退不肖,以朋黨獲譏,非所以重朝廷。」固乞罷,帝方委以政,故尉安之。



 它日,帝問:「符讖可信乎?何從而生?」嗣復曰:「漢光武以讖決事,隋文帝亦喜之,故其書蔓天下。班彪《王命論》有所引述,特以止賊亂,非重之也。」玨曰:「治亂宜直推人事耳。」帝曰:「然。」又問:「天后時有起布衣為宰相者,果可用乎?」嗣復曰:「天後重用刑,輕用官,自為之計耳。必責能否,要待歷試乃可。」



 是時延英訪對,史官不及知。嗣復建言:「故事,正衙,起居注在前;便坐,無所紀錄。姚、趙憬皆請置時政記,不能行。臣請延英對宰相語關道德刑政者,委中書門下直日紀錄,月付史官。」它宰相議不同,止。久之,帝又問:「延英政事,孰當記之?」玨監修國史,對曰:「臣之職也。」陳夷行曰:「宰相所錄,恐掩蔽聖德,自盜美名。臣向言不欲威權在下者,此也。」玨曰:「夷行疑宰相賣威權,貨刑賞。不然,何自居位而為此言邪?臣得罷為幸。」覃曰:「陛下開成初政甚善,三年後,日不逮前。」嗣復曰:「開成初,覃、夷行當國,三年後,臣與李玨同進。臣不能悉心奉職,使政事日不逮前,臣之罪也。縱陛下不忍加誅,當自殄滅。」即叩頭請從此辭,不敢更至中書,乃趨出。帝使使者召還,曰:「覃言失,何及此邪?」覃起謝曰:「臣愚不知忌諱,近事雖善,猶未盡公。臣非專斥嗣復,而遽求去,乃不使臣言耳。」嗣復曰:「陛下月費俸稟數十萬,時新異賜必先及,將責臣輔聖功,求至治也。使不及初,豈臣當死,累陛下之德,奈何?惟陛下別求賢以自輔。」帝曰:「覃偶及之,奚執咎?」嗣復闔門不肯起,帝乃免覃、夷行相,而嗣復專天下事。



 進門下侍郎。建言:「使府官屬多,宜省。」帝曰:「無反滯才乎?」對曰:「才者自異,汰去粃滓者,菁華乃出。」帝曰:「昔蕭復乘政,難言者必言,卿其志之!」



 未幾,帝崩,中尉仇士良廢遺詔,立武宗。帝之立,非宰相意,故內薄執政臣,不加禮,自用李德裕而罷嗣復為吏部尚書,出為湖南觀察使。會誅薛季棱、劉弘逸,中人多言嘗附嗣復、玨,不利於陛下。帝剛急,即詔中使分道誅嗣復等,德裕與崔鄆、崔珙等詣延英言:「故事,大臣非惡狀明白,未有誅死者。昔太宗、玄宗、德宗三帝,皆嘗用重刑,後無不悔,願徐思其宜,使天下知盛德有所容,不欲人以為冤。」帝曰:「朕纘嗣之際,宰相何嘗比數!且玨等各有附會,若玨、季棱屬陳王,猶是先帝意。如嗣復、弘逸屬安王,乃內為楊妃謀。且其所紿書曰:『姑何不斅天後?』」德裕曰:「飛語難辨。」帝曰:「妃昔有疾,先帝許其弟入侍,得通其謀。禁中證左尤具,我不欲暴於外。使安王立,肯容我耶?」言畢戚然,乃曰:「為卿赦之!」因追使者還,貶嗣復潮州刺史。



 宣宗立,起為江州刺史。以吏部尚書召,道岳州卒,年六十六,贈尚書左僕射,謚曰孝穆。



 嗣復領貢舉時,於陵自洛入朝,乃率門生出迎,置酒第中,於陵坐堂上,嗣復與諸生坐兩序。始於陵在考功,擢浙東觀察使李師稷及第,時亦在焉。人謂楊氏上下門生,世以為美。



 嗣復五子,其顯者:授、損。



 授,字得符,於昆弟最賢。由進士第遷累戶部侍郎,以母病求為秘書監。後以刑部尚書從昭宗幸華,徙太子少保,卒,贈尚書左僕射。



 子煚,字公隱,累擢左拾遺。昭宗初立,數游宴,上疏極諫。歷戶部員外郎。崔胤招硃全忠入京師,煚挈族客湖南。終諫議大夫。



 損,字子默,繇廕補藍田尉,至殿中侍御史。家新昌里,與路巖第接。巖方為相,欲易其廄以廣第。損族仕者十餘人,議曰:「家世盛衰,系權者喜怒,不可拒。」損曰:「今尺寸土皆先人舊貲,非吾等所有,安可奉權臣邪?窮達,命也!」卒不與。巖不悅,使損按獄黔中,逾年還。三遷絳州刺史。巖罷去,召為給事,遷京兆尹。與宰相盧攜雅不葉,復除給事中。陜虢軍亂,逐觀察使崔蕘,命損代之,至則盡誅有罪者。拜平盧節度使,徙天平,未赴復留,卒官下。



 贊曰:夫口道先王語,行如市人,其名曰「盜儒」。僧孺、宗閔以方正敢言進,既當國,反奮私暱黨,排擊所憎,是時權震天下,人指曰「牛李」,非盜謂何?逢吉險邪,稹浮躁,嗣復辯給,固無足言。幸主孱昏,不底於戮,治世之罪人歟!



\end{pinyinscope}