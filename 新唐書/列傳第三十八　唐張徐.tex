\article{列傳第三十八 唐張徐}

\begin{pinyinscope}

 唐臨,字本德,京兆長安人。周內史瑾之孫。其先自北海內徙。武德初,隱太子討王世充,臨以策進說山遺書》。主要哲學著作有《周易外傳》、《尚書引義》、《讀四,太子引直典書坊,授右衛率府鎧曹參軍。太子廢,出為萬泉丞。有輕囚久系,方春,農事興,臨說令可且出囚,使就畎畝。不許。臨曰:「有所疑,丞執其罪。」令移疾,臨悉縱歸,與之約,囚如期還。



 再遷侍御史。大夫韋挺責著位不肅,明日,挺越次與江夏王道宗語,臨進曰:「王亂班。」道宗曰:「與大夫語,何至爾!」臨曰:「大夫亦亂班。」挺失色,眾皆悚伏。俄持節按獄交州,出冤系三千人。累遷大理卿。高宗嘗錄囚,臨占對無不盡,帝喜曰:「為國之要在用法,刻則人殘,寬則失有罪,惟是折中,以稱朕意。」它日復訊,餘司斷者輒紛訴不已,獨臨所訊無一言。帝問故,答曰:「唐卿斷囚不冤,所以絕意。」帝嘆曰:「為獄者固當若是。」乃自述其考曰「形如死灰,心若鐵石」云。



 永徽元年,拜御史大夫。蕭齡之嘗任廣州都督,受賕當死,詔群臣議,請論如法,詔戮於朝堂。臨建言:「群臣不知天子所以議之之意。在律有八。王族戮於隱,議親也;刑不上大夫,議貴也。今齡之貪贓狼扈,死有餘咎。陛下以異於它囚,故議之有司,又令入死,非堯、舜所以用刑者,不可為後世法。」帝然之。齡之,齊高帝五世孫,由是免死。



 臨累遷吏部尚書。初,來濟謫臺州,李義府謫普州,臨奏許禕為江南巡察使,張倫劍南巡察使。禕與濟善,而倫與義府有隙。武後常右義府,察知之,謂臨遣所私督其過,坐免官。起為潮州刺史,卒,年六十。



 臨儉薄寡欲,不好治第宅。性旁通,專務掩人過。見妻子,必正衣冠。



 兄皎,武德初,為秦王府記室,從王征討,掌書檄。貞觀中,官吏部侍郎。先是,選集四時補擬,不為限。皎請以冬初集,盡季春止,後遂為法。終益州長史,贈太常卿。



 子之奇,給事中。坐章懷太子屬徙邊。後除括蒼令,與徐敬業起兵,誅。



 臨孫紹紹,神龍時為太常博士。遷左臺侍御史、度支員外郎,常兼博士。韋庶人請妃、公主、命婦以上葬給鼓吹,詔可。紹曰:「鼓吹本軍容,黃帝戰涿鹿,以為警衛,故曲有《靈夔吼》、《雕鶚爭》、《石墜崖》、《壯士怒》之類。惟功臣詔葬,得兼用之。男子有四方功,所以加寵。雖郊祀天地,不參設,容得接閨閫哉?在令,五品官昏葬,無給鼓吹者,唯京官五品則假四品,蓋班秩在夫若子。請置前詔,用舊典。」不省。



 中宗始郊,國子祭酒祝欽明等知韋後能制天子,欲迎諂之,即奏以皇后亞獻,安樂公主終獻,又四時及列帝誕日,遣使者詣陵如事生。紹以為非禮,引正誼固爭。帝又詔武氏陵及諸武墓皆置守戶,紹謂:「昊、順二陵守戶五百,與昭陵同。在令,先世帝王陵戶二十,今雖崇奉外家,宜準附常典。又親王墓戶十,梁、魯乃追贈,不可逾真王。褒德衛卒,至逾宗廟,不可明甚,請罷之。」又言:「比群臣務厚葬,以俑人象驂眩耀相矜,下逮眾庶,流宕成俗。願按令切敕裁損,凡明器不許列衢路,惟陳墓所。昏家盛設障車,擁道為戲樂,邀貨捐貲動萬計,甚傷化紊禮,不可示天下。」事雖不從,議者美嘆。



 睿宗即位,數言政損益,再遷給事中,兼太常少卿。先天二年,玄宗講武驪山,紹以典儀坐失軍容,當斬。帝怒甚,執纛下,左右猶冀少貸,金吾將軍李邈遽傳詔斬之。時深咎邈,帝亦悔,俄詔罷邈官,擯死於家。



 張文瓘,字稚圭,貝州武城人。隋大業末,徙家魏州之昌樂。幼孤,事母、兄以孝友聞。貞觀初,第明經,補並州參軍。時李勣為長史,嘗嘆曰:「稚圭,今之管、蕭,吾所不及。」勣入朝,文瓘與屬僚二人皆餞,勣贈二人以佩刀、玉帶,而不及文瓘。文瓘以疑請,勣曰:「子無為嫌。若某,■豫少決,故贈以刀,欲其果於斷;某放誕少檢,故贈以帶,俾其守約束。若子才,無施不可,焉用贈?」因極推引。再遷水部員外郎。時兄文琮為戶部侍郎,於制,兄弟不並臺閣,出為雲陽令。累授東西臺舍人,參知政事。乾封二年,遷東臺侍郎、同東西臺三品,遂與勣同為宰相。俄知左史事。



 時高宗造蓬萊、上陽、合璧等宮,復征討四夷,京師養廄馬萬匹,帑瓘浸虛。文瓘諫曰:「王者養民,逸則富以康,勞則怨以叛。秦、漢廣事四夷,造宮室,至二世土崩,武帝末年戶口減半。夫制治於未亂,保邦於未危。人罔常懷,懷於有仁。臣願撫之,無使勞而生怨。隋監未遠,不可不察。」帝善其言,賜繒錦百段,為減廄馬數千。



 改黃門侍郎,兼太子右庶子,又兼大理卿。不旬日,斷疑獄四百,抵罪者無怨言。嘗有小疾,囚相與齋禱,願亟視事。時以執法平恕方戴胄。後拜侍中,兼太子賓客。諸囚聞其遷,皆垂泣,其得人心如此。性嚴正,未嘗回容,諸司奏議,悉心糾駁,故帝委之。或時移疾,他宰相奏事,帝必問與文瓘議未。若不者,曰:「往共籌之。」或曰:「已議。」即皆報可。



 新羅叛,帝將出兵討之。時文瓘病臥家,自力請見,曰:「吐蕃盜邊,兵屯境未解,新羅復叛,議者欲出師,二虜俱事,臣恐人不堪弊,請息兵修德,以懷異俗。」詔可。



 初,同列以堂饌豐餘,欲少損。文瓘曰:「此天子所以重樞務、待賢才也,吾等若不任職,當自引避,不宜節減,以自取名。」眾乃止。卒,年七十三,贈幽州都督,謚曰懿。以嘗事孝敬皇帝,詔陪葬恭陵。



 四子:潛,為魏州刺史;沛,同州刺史;洽,衛尉卿;涉,殿中監。父子皆至三品,時謂「萬石張家」。韋溫誅,涉為亂兵所殺。



 文琮,好自寫書,筆不釋手。子弟諫止,曰:「吾好此,不為倦。」貞觀中,為治書侍御史,遷亳州刺史。永徽初,獻《文皇帝頌》,優制褒美,拜戶部侍郎。坐房遺愛從母弟,出為建州刺史。州尚淫祀,不立社稷,文琮下教曰:「春秋二社本於農,今此州廢不立,尚何觀?比歲田畝卒荒,或未之思乎!神在於敬,可以致福。」於是始建祀場,民悅從之。卒於官。



 子錫,久視初,為鳳閣侍郎、同鳳閣鸞臺平章事,代其甥李嶠為宰相。請還廬陵王,不為張易之所右。與鄭杲俱知選,坐洩禁中語,又賕謝鉅萬,時蘇味道亦坐事,同被訊,系鳳閣,俄徙司刑三品院。錫按轡專道,神氣不懾,日膳豐鮮,無損貶。味道徒步赴逮,席地菜食。武后聞之,釋味道,將斬錫,既而流循州。神龍中,累遷工部尚書,兼修國史,東都留守。韋后臨朝,詔同中書門下三品,旬日,出為絳州刺史。累封平原郡公,卒。



 文琮從父弟文收,終太子率更令。善音律,著《新樂書》十餘篇。



 徐有功,名弘敏,避孝敬皇帝諱,以字行,國子博士文遠孫也。舉明經,累補蒲州司法參軍,襲封東莞縣男。為政仁,不忍杖罰,民服其恩,更相約曰:「犯徐參軍杖者,必斥之。」訖代不辱一人。累遷司刑丞。時武後僭位,畏唐大臣謀己。於是周興、來俊臣、丘神績、王弘義等揣識後指,置總監牧院諸獄,捕將相,俾相鉤逮,掩搦護送,楚掠凝慘。又污引天下豪人桀,馳使者即按,一切以反論。吏爭以周內窮詆相高,後輒勸以官賞,於是以急變相告言者無虛日。朝野震恐,莫敢正言,獨有功數犯顏爭枉直,後厲語折抑,有功爭益牢。時博州刺史瑯邪王沖,責息錢於貴鄉,遣家奴督斂,與尉顏餘慶相聞知,奴自市弓矢還。會沖坐逆誅,魏州人告餘慶豫沖謀,後令俊臣鞫治,以反狀聞。有司議:「餘慶更永昌赦,法當流。」侍御史魏元忠謂:「餘慶為沖督償、通書,合謀明甚,非曰支黨,請殊死,籍其家。」詔可。有功曰:「永昌赦令:『與虺貞同惡,魁首已伏誅,支黨未發者原之。』《書》曰:『殲厥渠魁』,律以『造意為首』,尋赦已伏語,則魁首無遺。餘慶赦後被言,是謂支黨。今以支為首,是以生入死。赦而復罪,不如勿赦;生而復殺,不如勿生。竊謂朝廷不當爾。」後怒曰:「何謂魁首?」答曰:「魁者,大帥;首者,元謀。」後曰:「餘慶安得不為魁首?」答曰:「若魁首者,虺貞是已。既已伏誅,餘慶今方論罪,非支黨何?」後意解,乃曰:「公更思之。」遂免死。當此時,左右及衛仗在廷陛者數百人,皆縮項不敢息,而有功氣定言詳,闉然不橈。



 有韓紀孝者,受徐敬業偽官,前已物故,推事使顧仲琰籍其家,詔已報可。有功追議曰:「律,謀反者斬。身亡即無斬法,無斬法則不得相緣。所緣之人亡,則所因之罪減。」詔從之,皆以更赦免,如此獲宥者數十百姓。



 累轉秋官郎中。鳳閣侍郎任知古、冬官尚書裴行本等七人被誣當死,後謂宰相曰:「古人以殺止殺,我今以恩止殺,就群公丐知古等,賜以再生,可乎?」俊臣、張知默固請如法,後不許。俊臣獨引行本更驗前罪。有功奏曰:「俊臣違陛下再生之賜,不可以示信。」於是悉免死。



 道州刺史李仁褒兄弟為人誣構,有功爭不能得。秋官侍郎周興劾之曰:「漢法,附下罔上者斬,面欺者亦斬。在古,析言破律者殺。有功故出反囚,罪當誅,請按之。」後不許,猶坐史官。



 俄起為左肅政臺侍御史,辭曰:「臣聞鹿走山林而命系庖廚者,勢固自然。陛下以法官用臣,臣守正行法,必坐此死矣。」後固授之。天下聞有功復進,灑然相賀。時有詔:「公坐流、私坐徒以上會赦免,逾百日不首者,復論。」有功奏曰:「陛下寬殊死罪,已發者原之,是通改過之心、自新之路。故律,告赦前事,以其罪坐之。若無告言,所犯終不自發;如告言赦前事,則與律乖。今赦前之罪,不自言者,還以法論,即恩雖布天下,而一罪不能貸,臣竊為陛下不取。」後更詔五品以上議可。



 又上疏曰:「天下員有定,比選者日多,選曹諉囑公行,囂謗滿路。唐季人多逆節,鞫訊結斷,刑慘獄嚴,革命歲久,其流弗改。事表生情,法外構理,而刻薄吏驅扇成奸。雖朝堂進表,列匭內牒,叫閽弗聽,叩鼓弗聞,使申其冤,正增其枉。誠令天官銓注有所不平、法司推斷舞法深詆、三司理匭受所上章擁塞不白者皆許臣按驗劾發,奪祿貶勞,不越月逾時,可致刑措。」後納之。



 竇孝諶妻龐為其奴怖以妖祟,教為夜解,因告以厭詛。給事中薛季昶鞫之,龐當死。子希瑊訟冤,有功明其枉。季昶劾有功黨惡逆,當棄市。有功方視事,令史泣以告。有功曰:「豈吾獨死,而諸人長不死邪?」安步去。後召詰曰:「公比斷獄多失出,何耶?」對曰:「失出,臣小過;好生,陛下大德。」後默然。龐得減死,有功免為民。



 起拜左司郎中,轉司刑少卿。與皇甫文備同按獄,誣有功縱逆黨。久之,文備坐事下獄,有功出之。或曰:「彼嘗陷君於死,今生之,何也?」對曰:「爾所言者私忿,我所守者公法,不可以私害公。」



 嘗謂所親曰:「大理,人命所系,不可阿旨詭辭,以求茍免。」故有功為獄,常持平守正,以執據冤罔,凡三坐大闢,將死,泰然不憂,赦之,亦不喜,後以此重之。所全活甚眾,酷吏為少衰,然疾之如仇矣。改司僕少卿。卒,年六十八,贈司刑卿。中宗即位,加贈越州都督,遣使就第吊祭,賜物百段,授一子官。開元初,竇希瑊等請以己官讓有功子惀,以報舊德,由是自大理司直遷恭陵令。會昌中,追謚忠正。



 初,鹿城主簿潘好禮慕有功為人,論之曰:「昔稱張釋之為廷尉,天下無冤人,今有功斷獄,亦天下無冤人。然釋之當漢文帝時,中外無事,守法而已。有功居革命之際,周興、來俊臣等掩義隱賊,崇飾惡言,以誣盛德,有功守死明道,身濱殆者數矣,此其賢於釋之明甚。」或稱有功仁恕過漢於、張。起居舍人盧若虛曰:「徐公當雷霆之震,而能全仁恕,雖千載未見其比。」五世孫商。



 贊曰:「徐有功不以唐、周貳其心,惟一於法,身蹈死以救人之死,故能處猜後、酷吏之間,以恕自將,內挫虐焰,不使天下殘於燎,可謂仁人也哉!議者謂過漢於、張,渠不信夫!



 商,字義聲,或字秋卿,客新鄭再世,因為新鄭人。幼隱中條山。擢進士第。大中時,擢累尚書左丞。宣宗詔為巡邊使,使有指,拜河中節度使。突厥殘種保特峨山,以千帳度河自歸,詔商綏定。商表處山東寬鄉,置備征軍,凡千人,襞紙為鎧,勁矢不能洞。徙節山南東道,襄多山棚,為票賊,商取材卒為捕盜將,別為屯營,寇所發,輒跡捕,捕必得,遂為精兵。江西都將反,韋宙乘傳抵山南發兵,商命部將韓季友以捕盜營士往。賊平,宙表留季友所部為綱紀。咸通初,以刑部尚書為諸道鹽鐵轉運使,封東莞縣子。四年,進同中書門下平章事,出為荊南節度使。累進太子太保,卒。



 子彥若,事僖宗為中書舍人。昭宗立,再用為御史中丞。張浚師敗太原,以彥若為戶部侍郎、同中書門下平章事。俄代李茂貞為鳳翔節度使,不得入,還為御史大夫。乾寧初,復當國,進位太保、齊國公。崔胤專政,以彥若位己右,不悅,以平章事為清海軍節度使,卒於鎮,而行軍司馬劉隱因主留務。方時多難,彥若最見信於帝,有以事自陳者,帝曰:「汝當問彥若。」其所倚任如此。



\end{pinyinscope}