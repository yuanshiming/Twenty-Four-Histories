\article{本紀第一 高祖}

\begin{pinyinscope}

 高祖神堯大聖大光孝皇帝諱淵,字叔德,姓李氏,隴西成紀人也。其七世祖皓,當晉末分三等,只有情動而處中,方為上品之情。北宋程顥主張以,據秦、涼以自王,是為涼武昭王。皓生歆,歆為沮渠蒙遜所滅。歆生重耳,魏弘農太守。重耳生熙,金門鎮將,戍於武川,因留家焉。熙生天賜,為幢主。天賜生虎,西魏時,賜姓大野氏,官至太尉,與李弼等八人佐周代魏有功,皆為柱國,號「八柱國家」。周閔帝受魏禪,虎已卒,乃追錄其功,封唐國公,謚曰襄。襄公生昺,襲封唐公,隋安州總管、柱國大將軍,卒,謚曰仁。



 仁公生高祖於長安,體有三乳,性寬仁,襲封唐公。隋文帝獨孤皇后,高祖之從母也,以故文帝與高祖相親愛。文帝相周,復高祖姓李氏,以為千牛備身,事隋譙、隴二州刺史。大業中,歷岐州刺史、滎陽樓煩二郡太守,召為殿內少監、衛尉少卿。



 煬帝征遼東,遣高祖督運糧於懷遠鎮。楊玄感將反,其兄弟從征遼者皆逃歸,高祖先覺以聞。煬帝遽班師,以高祖為弘化留守,以御玄感,詔關右諸郡兵皆受高祖節度。



 是時,隋政荒,天下大亂,煬帝多以猜忌殺戮大臣。嘗以事召高祖,高祖遇疾,不時謁。高祖有甥王氏在後宮,煬帝問之,王氏對以疾,煬帝曰:「可得死否?」高祖聞之益懼,因縱酒納賂以自晦。



 十一年,拜山西河東慰撫大使,擊龍門賊母端兒,射七十發皆中,賊敗去,而斂其尸以築京觀,盡得其箭於其尸。又擊絳州賊柴保昌,降其眾數萬人。突厥犯塞,高祖與馬邑太守王仁恭擊之。隋兵少,不敵,高祖選精騎二千為游軍,居處飲食隨水草如突厥,而射獵馳騁示以閑暇,別選善射者伏為奇兵。虜見高祖,疑不敢戰,高祖乘而擊之,突厥敗走。



 十三年,拜太原留守,擊高陽歷山飛賊甄翟兒於西河,破之。是時,煬帝南游江都,天下盜起。高祖子世民知隋必亡,陰結豪傑,招納亡命,與晉陽令劉文靜謀舉大事。計已決,而高祖未之知,欲以情告,懼不見聽。高祖留守太原,領晉陽宮監,而所善客裴寂為副監,世民陰與寂謀,寂因選晉陽宮人私侍高祖。高祖過寂飲酒,酒酣從容,寂具以大事告之,高祖大驚。寂曰:「正為宮人奉公,事發當誅,為此爾。」世民因亦入白其事,高祖初陽不許,欲執世民送官,已而許之,曰:「吾愛汝,豈忍告汝邪?」然未有以發。而所在盜賊益多,突厥數犯邊,高祖兵出無功,煬帝遣使者執高祖詣江都,高祖大懼。世民曰:「事急矣,可舉事!」已而煬帝復馳使者赦止高祖,其事遂已。



 是時,劉武周起馬邑,林士弘起豫章,劉元進起晉安,皆稱皇帝;硃粲起南陽,號楚帝;李子通起海陵,號楚王;邵江海據岐州,號新平王;薛舉起金城,號西秦霸王;郭子和起榆林,號永樂王;竇建德起河間,號長樂王;王須拔起恆、定,號漫天王;汪華起新安,杜伏威起淮南,皆號吳王;李密起鞏,號魏公;王德仁起鄴,號太公;左才相起齊郡,號博山公;羅藝據幽州,左難當據涇,馮盎據高、羅,皆號總管;梁師都據朔方,號大丞相;孟海公據曹州,號錄事;周文舉據淮陽,號柳葉軍;高開道據北平,張長遜據五原,周洮據上洛,楊士林據山南,徐圓朗據兗州,楊仲達據豫州,張善相據伊、汝,王要漢據汴州,時德睿據尉氏,李義滿據平陵,綦公順據青、萊,淳於難據文登,徐師順據任城,蔣弘度據東海,王薄據齊郡,蔣善合據鄆州,田留安據章丘,張青特據濟北,臧君相據海州,殷恭邃據舒州,周法明據永安,苗海潮據永嘉,梅知巖據宣城,鄧文進據廣州,俚酋楊世略據循、潮,冉安昌據巴東,甯長真據鬱林,其別號諸盜往往屯聚山澤。而劉武周攻汾陽宮,高祖乃集將吏告曰:「今吾為留守,而賊據離宮,縱賊不誅,罪當死。然出兵必待報,今江都隔遠,後期奈何?」將吏皆曰:「國家之利可專者,公也。」高祖曰:「善。」乃募兵,旬日間得眾一萬。



 副留守虎賁郎將王威、虎牙郎將高君雅見兵大集,疑有變,謀因禱雨晉祠以圖高祖。高祖覺之,乃陰為備。五月甲子,高祖及威、君雅視事,開陽府司馬劉政會告威、君雅反,即坐上執之。丙寅,突厥犯邊,高祖令軍中曰:「人告威、君雅召突厥,今其果然。」遂殺之以起兵。遣劉文靜使突厥,約連和。



 六月己卯,傳檄諸郡,稱義兵,開大將軍府,置三軍。以子建成為隴西公、左領軍大都督,左軍隸焉;世民為燉煌公、右領軍大都督,右軍隸焉;元吉為姑臧公,中軍隸焉。裴寂為長史,劉文靜為司馬,石艾縣長殷開山為掾,劉政會為屬,長孫順德、王長諧、劉弘基、竇琮為統軍。開倉庫賑窮乏。七月壬子,高祖杖白旗,誓眾於野,有兵三萬,以元吉為太原留守。癸丑,發太原。甲寅遣將張綸徇下離石、龍泉、文城三郡。丙辰,次靈石,營於賈胡堡。隋虎牙郎將宋老生屯於霍邑,以拒義師。丙寅,隋鷹揚府司馬李軌起武威,號大涼王。八月辛巳,敗宋老生於霍邑。丙戌,下臨汾郡。辛卯,克絳郡。癸巳,次龍門,突厥來助。隋驍衛大將軍屈突通守河東,絕津梁。壬寅,馮翊賊孫華、土門賊白玄度皆具舟以來逆。九月戊午,高祖領太尉,加置僚佐。以少牢祀河,乃濟。甲子,次長春宮。丙寅,隴西公建成、劉文靜屯永豐倉守潼關,敦煌公世民自渭北徇三輔,從父弟神通起兵於鄠,柴氏婦,高祖女也,亦起兵於司竹,皆與世民會。眉阜賊丘師利李仲文、盩厔賊何潘仁向善思、宜君賊劉炅皆來降,因略定鄠、杜。壬申,高祖次馮翊。乙亥,敦煌公世民屯阿城,隴西公建成自新豐趨霸上。丙子,高祖自下邽以西,所經隋行宮、苑御,悉罷之,出宮女還其家。十月辛巳,次長樂宮,有眾二十萬。隋留守衛文升等奉代王侑守京城,高祖遣使諭之,不報。乃圍城,下令曰:「犯隋七廟及宗室者,罪三族。」丙申,隋羅山令蕭銑自號梁公。十一月丙辰,克京城。命主符郎宋公弼收圖籍。約法十二條,殺人、劫盜、背軍、叛者死。癸亥,遙尊隋帝為太上皇,立代王為皇帝。大赦,改元義寧。甲子,高祖入京師,至朝堂,望闕而拜。隋帝授高祖假黃鉞、使持節、大都督內外諸軍事、大丞相、錄尚書事,進封唐王。以武德殿為丞相府,下教曰令,視事於虔化門。十二月癸未,隋帝贈唐襄公為景王;仁公為元王;夫人竇氏為唐國妃,謚曰穆。以建成為唐國世子;世民為唐國內史,徙封秦國公;元吉為齊國公。丞相府置長史、司錄以下官。趙郡公孝恭徇山南。甲辰,雲陽令詹俊徇巴、蜀。



 二年正月丁未,隋帝詔唐王劍履上殿,入朝不趨,贊拜不名,加前後羽葆、鼓吹。戊午,周洮降。戊辰,世子建成為左元帥,秦國公世民為右元帥,徇地東都。二月己卯,太常卿鄭元定樊、鄧,使者馬元規徇荊、襄。三月己酉,齊國公元吉為太原道行軍元帥。乙卯,世民徙封趙國公。丙辰,隋右屯衛將軍宇文化及弒太上皇於江都,立秦王浩為皇帝。吳興郡守沈法興據丹陽,自稱江南道總管。樂安人盧祖尚據光州,自稱刺史。戊辰,隋帝進唐王位相國,總百揆,備九錫,唐國置丞相等官,立四廟。四月己卯,張長遜降。辛巳,停竹使符,班銀菟符。五月乙巳,隋帝命唐王冕十有二旒,建天子旌旗,出警入蹕。甲寅,王德仁降。戊午,隋帝遜於位,以刑部尚書蕭造、司農少卿裴之隱奉皇帝璽紱於唐王,三讓乃受。



 武德元年五月甲子,即皇帝位於太極殿。命蕭造兼太尉,告於南郊,大赦,改元。賜百官、庶人爵一級,義師所過給復三年,其餘給復一年。改郡為州,太守為刺史。庚午,太白晝見。隋東都留守元文都及左武衛大將軍王世充立越王侗為皇帝。六月甲戌,趙國公世民為尚書令,裴寂為尚書右僕射、知政事,劉文靜為納言,隋民部尚書蕭瑀、丞相府司錄參軍竇威為內史令。丙子,太白晝見。己卯,追謚皇高祖曰宣簡公;皇曾祖曰懿王;皇祖曰景皇帝,廟號太祖,祖妣梁氏曰景烈皇后;皇考曰元皇帝,廟號世祖,妣獨孤氏曰元貞皇后;妃竇氏曰穆皇后。庚辰,立世子建成為皇太子,封世民為秦王,元吉齊王。癸未,薛舉寇涇州,秦王世民為西討元帥,劉文靜為司馬。太僕卿宇文明達招慰山東。乙酉,奉隋帝為酅國公,詔曰:「近世時運遷革,前代親族,莫不夷絕。歷數有歸,實惟天命;興亡之效,豈伊人力。前隋蔡王智積等子孫,皆選用之。」癸巳,禁言符瑞者。辛丑,竇威薨。黃門侍郎陳叔達判納言,將作大匠竇抗兼納言。七月壬子,劉文靜及薛舉戰於涇州,敗績。乙卯,郭子和降。庚申,廢隋離宮。八月壬申,劉文靜除名。戊寅,約功臣恕死罪。辛巳,薛舉卒。壬午,李軌降。甲申,巖州刺史王德仁殺招慰使宇文明達以反。己丑,秦王世民為西討元帥,以討薛仁杲。庚子,贈隋太常卿高熲上柱國、郯國公,上柱國賀若杞國公,司隸大夫薛道衡上開府、臨河縣公,刑部尚書宇文弼上開府、平昌縣公,左翊衛將軍董純柱國、狄道公,右驍衛將軍李金才上柱國、申國公,左光祿大夫李敏柱國、觀國公。諸遭隋枉殺而子孫被流者,皆還之。九月乙巳,慮囚。始置軍府。癸丑,改銀菟符為銅魚符。甲寅,秦州總管竇軌及薛仁杲戰,敗績。辛未,宇文化及殺秦王浩,自稱皇帝。十月壬申朔,日有食之。己卯,李密降。壬午,硃粲陷鄧州,刺史呂子臧死之。乙酉,邵江海降。己亥,盜殺商州刺史泉彥宗。辛丑,大閱。是月,竇抗罷。十一月,竇建德敗王須拔於幽州,須拔亡入於突厥。乙巳,涼王李軌反。戊申,禁獻侏儒短節、小馬庳牛、異獸奇禽者。己酉,秦王世民敗薛仁杲,執之。癸丑,行軍總管趙慈景攻蒲州,隋刺史堯君素拒戰,執慈景。癸亥,秦王世民俘薛仁杲以獻。十二月壬申,世民為太尉。丙子,蒲州人殺堯君素,立其將王行本。辛已,鄭元及硃粲戰於商州,敗之。乙酉,如周氏陂。丁亥,至自周氏陂。庚子,光祿卿李密反,伏誅。是歲,高開道陷漁陽,號燕王。



 二年正月甲子,陳叔達兼納言。詔自今正月、五月、九月不行死刑,禁屠殺。丙寅,張善相降。己巳,楊士林降。二月乙酉,初定租、庸、調法。令文武官終喪。丙戌,州置宗師一人。甲午,赦並、浩、介、石四州賈胡堡以北擊囚。閏月,竇建德陷邢州,執總管陳君賓。辛丑,竇建德殺宇文化及於聊城。硃粲降。壬寅,皇太子及秦王世民、裴寂巡於畿縣。乙巳,御史大夫段確勞硃粲於菊潭。庚戌,微行,察風俗。乙卯,以穀貴,禁關內屠酤。左屯衛將軍何潘仁及山跋張子惠戰於司竹,死之。丁巳,慮囚。庚申,驍騎將軍趙欽、王娑羅及山賊戰於盩厔,死之。丁卯,王世充隱殷州,陟州刺史李育德死之。三月甲戌,王薄降。庚辰,蔣弘度、徐師順降。丁亥,竇建德陷趙州。丁酉,李義滿降。四月,綦公順降。庚子,並州總管、齊王元吉及劉武周戰於榆次,敗績。辛丑,硃粲殺段確以反。乙巳,王世充廢越王侗,自稱皇帝。癸亥,陷伊州,執總管張善相。五月庚辰,涼州將安脩仁執李軌以降。癸未,曲赦涼、甘、瓜、鄯、肅、會、蘭、河、廓九州。六月,王世充殺越王侗。戊戌,立周公、孔子廟於國子監。庚子,竇建德陷滄州。丁未,劉武周陷介州。癸亥,裴寂為晉州道行軍總管。離石胡劉季真叛,陷石州,刺史王儉死之。七月壬申,徐圓朗降。八月丁酉,酅國公薨。甲子,竇建德陷洺州,執總管袁子幹。九月辛未,殺戶部尚書劉文靜。李子通自稱皇帝。沈法興自稱梁王。丁丑,杜伏威降。裴寂及劉武周戰於介州,敗績,右武衛大將軍姜寶誼死之。庚辰,竇建德陷相州,總管呂氏死之。辛巳,劉武周陷並州。庚寅,太白晝見。竇建德陷趙州,執總管張志昂。乙未,京師地震。梁師都寇延州,鄜州刺史梁禮死之。十月己亥,羅藝降。乙卯,如華陰,赦募士背軍者。壬戌,劉武周寇晉州,永安王孝基及工部尚書獨孤懷恩、陜州總管於筠、內史侍郎唐儉討之。甲子,祠華山。是月,夏縣人呂崇茂反。秦王世民討劉武周。十一月丙子,竇建德陷黎州,執淮安王神通、總管李世勣。十二月丙申,獵於華山。永安王孝基及劉武周戰於下邽,敗績。壬子,大風拔木。



 三年正月己巳,獵於渭濱。戊寅,王行本降。辛巳,如蒲州。癸巳,至自蒲州。二月丁酉,京師西南地有聲。庚子,如華陰。甲寅,獨孤懷恩謀反,伏誅。辛酉,檢校隰州總管劉師善謀反,伏誅。三月庚午,改納言為侍中,內史令為中書令。甲戌,中書侍郎封德彞兼中書令。乙酉,劉季真降。四月丙申,祠華山。壬寅,至自華陰。癸卯,禁關內諸州屠。甲寅,秦王世民及宋金剛戰於雀鼠谷,敗之。辛酉,王世充陷鄧州,總管雷四郎死之。壬戌,秦王世民及劉武周戰於洺州,敗之,武周亡入於突厥。克並州。五月壬午,秦王世民屠夏縣。六月丙申,赦晉、隰、潞、並四州。癸卯,詔隋帝及其宗室柩在江都者,為營窆,置陵廟,以故宮人守之。丙午,慮囚。封子元景為趙王,元晶魯王,元亨豐王。己酉,出宮女五百人,賜東征將士有功者。甲寅,顯州長史田瓚殺行臺尚書令楊士林,叛附於王世充。乙卯,瘞州縣暴骨。七月壬戌,秦王世民討王世充。甲戌,皇太子屯於蒲州,以備突厥。丙戌,梁師都導突厥、稽胡寇邊,行軍總管段德操敗之。八月庚子,慮囚。甲辰,時德睿降。九月癸酉,田瓚降。己丑,給復陜、鼎、熊、穀四州二年。十月戊申,高開道降。己酉,楊仲達降。己未,有星隕於東都。十二月己酉,瓜州刺史賀拔行威反。



 四年正月辛巳,皇太子伐稽胡。二月,竇建德陷曹州,執孟海公。己丑,車騎將軍董阿興反於隴州,伏誅。乙巳,太常少卿李仲文謀反,伏誅。丙午,慮囚。丁巳,赦代州總管府石嶺之北。三月,進封宜都郡王泰為衛王。庚申,慮囚。乙酉,竇建德陷管州,刺史郭志安死之。四月壬寅,齊王元吉及王世充戰於東都,敗績,行軍總管盧君諤死之。戊申,突厥寇並州,執漢陽郡王環、太常卿鄭元、左驍騎衛大將軍長孫順德。甲寅,封子元方為周王,元禮鄭王,元嘉宋王,元則荊王,元茂越王。丁巳,左武衛將軍王君廓敗張青特,執之。五月壬戌,秦王世民敗竇建德於虎牢,執之。乙丑,赦山東為建德所詿誤者。戊辰,王世充降。庚午,周法明降。六月庚寅,赦河南為王世充所詿誤者。戊戌,蔣善合降。庚子,營州人石世則執其總管晉文衍,叛附於靺鞨。乙卯,臧君相降。七月甲子,秦王世民俘王世充以獻。丙寅,竇建德伏誅。丁卯,大赦,給復天下一年,陜、鼎、函、虢、虞、芮、豳七州二年。甲戌,劉黑闥反於貝州。辛巳,戴州刺史孟啖鬼反,伏誅。八月丙戌朔,日有食之。丁亥,皇太子安撫北境。丁酉,劉黑闥陷鄃縣,魏州刺史權威、貝州刺史戴元祥死之。癸卯,竇厥寇代州,執行軍總管王孝基。丁未,劉黑闥陷歷亭,屯衛將軍王行敏死之。辛亥,深州人崔元遜殺其刺史裴晞,叛附於劉黑闥。兗州總管徐圓朗反。九月,盧祖尚降。乙卯,淳於難降。甲子,汪華降。是秋,夔州總管、趙郡王孝恭率十二總管兵以討蕭銑。十月己丑,秦王世民為天策上將,領司徒,齊王元吉為司空。庚寅,劉黑闥陷瀛州,執刺史盧士睿,又陷觀州。癸卯,毛州人董燈明殺其刺史趙元愷。乙巳,趙郡王孝恭敗蕭銑於荊州,執之。閏月乙卯,如稷州。己未,幸舊墅。壬戌,獵於好畤。乙丑,獵於九。丁卯,獵於仲山。戊辰,獵於清水穀,遂幸三原。辛未,如周氏陂。壬申,至自周氏陂。十一月甲申,有事於南郊。庚寅,李子通降。丙申,子通謀反,伏誅。壬寅,劉黑闥陷定州,總管李玄通死之。庚戌,杞州人周文舉殺其刺史王孝矩,叛附於黑闥。十二月乙卯,黑闥陷冀州,總管麴棱死之。甲子,左武候將軍李世勣及黑闥戰於宋州,敗績。丁卯,秦王世民、齊王元吉討黑闥。己巳,黑闥陷邢州。庚午,陷魏州,總管潘道毅死之。辛未,隱業州。壬申,徙封元嘉為徐王。



 五年正月乙酉,劉黑闥陷相州,刺史房晃死之。丙戌殷恭邃降。丁亥,濟州別駕劉伯通執其刺史竇務本,叛附於徐圓朗。庚寅,東鹽州治中王才藝殺其刺史田華,叛附於劉黑闥。丙申,相州人殺其刺史獨孤徹以其州叛附於黥闥。己酉,楊世略、劉元進降。二月,王要漢降。己巳,秦王世民克邢州。丁丑,劉黑闥陷洺水,總管羅士信死之。戊寅,汴州總管王要漢敗徐圓郎於杞州,執周文舉。三月戊戌,譚州刺名李義滿殺齊州都督王薄。丁未,秦王世民及劉黑闥戰於洺水,敗之,黑闥亡入於突厥。蔚州總管高開道反,寇易州,刺史慕容孝干死之。四月,梁州野蠶成繭。冉安昌降。己未,寧長真降。戊辰,釋流罪以下獲麥。壬申,代州總管李大恩及突厥戰,死之。戊寅,鄧文進降。五月,田留安降。庚寅,瓜州人王乾殺賀拔行威以降。乙巳,賜荊州今歲田租。六月辛亥,劉黑闥與突厥寇山東。車騎將軍元韶為瓜州道行軍總管,以備突厥。癸丑,吐谷渾寇洮、旭、疊三州,岷州總管李長卿敗之。乙卯,淮安郡王神通討徐圓朗。七月甲申,作弘義宮。甲午,淮陽郡王道玄為河北道行軍總管,討劉黑闥。貝州人董該以定州叛附於黑闥。丙申,突厥殺劉武周於白道。遷州人鄧士政反,執其刺史李敬昂。丁酉,馮盎降。八月辛亥,葬隋煬帝。甲寅,吐谷渾寇岷州,益州道行臺左僕射竇軌敗之。乙卯,突厥寇邊。庚申,皇太子出豳州道,秦王世民出秦州道,以御突厥。己巳,吐谷渾陷洮州。並州總管、襄邑郡王神符及突厥戰於汾東,敗之。戊寅,突厥陷大震關。九月癸巳,靈州總管楊師道敗之於三觀山。丙申,洪州總管宇文歆又敗之於崇岡。壬寅,定州總管雙士洛、驃騎將軍魏道仁又敗之於恆山之陽。丙午,領軍將軍安興貴之又敗之於甘州。劉黑闥陷瀛州,刺史馬匡武死之。東鹽州人馬君德以其州叛附於黑闥。十月己酉,齊王元吉討黑闥。癸丑,貝州刺史許善護及黑闥戰於鄃縣,死之。甲寅,觀州刺史劉君會叛附於黑闥。乙丑,淮陽郡王道玄及黑闥戰於下博,死之。己巳,林士弘降。十一月庚辰,劉黑闥陷滄州。甲申,皇太子討黑闥。丙申,如宜州。癸卯,獵於富平北原。十二月丙辰,獵於萬壽原。戊午,劉黑闥陷恆州,刺史王公政死之。庚申,至自萬壽原。壬申,皇太子及劉黑闥戰於魏州,敗之。甲戌,又敗之於毛州。



 六年正月己卯,黑闥將葛德威執黑闥以降。壬午,巂州人王摩娑反,驃騎將軍衛彥討之。庚寅,徐圓朗陷泗州。二月,劉黑闥伏誅。庚戌,幸溫湯。壬子,獵於驪山。甲寅,至自溫湯。丙寅,行軍總管李世勣敗徐圓朗,執之。三月,苗海潮、梅知巖、左難當降乙巳,洪州總管張善安反。四月己酉,吐蕃陷芳州。己未,以故第為通義宮,祭元皇帝、元貞皇后於舊寢。赦京城,賜從官帛。辛酉,張善安陷孫州,執總管王戎。丁卯,南州刺史龐孝泰反,陷南越州。壬申,封子元為蜀王,元慶漢王。癸酉,裴寂為尚書左僕射,蕭瑀為右僕射,封德彞為中書令,吏部尚書趙恭仁兼中書令、檢校涼州諸軍事。五月庚寅,吐谷渾、黨項寇河州,刺史盧士良敗之。癸卯,高開道以奚寇幽州,長史王說敗之。六月丁卯,突厥寇朔州,總管高滿政敗之。曲赦朔州。七月丙子,沙州別駕竇伏明反,殺其總管賀若懷廓。己亥,皇太子屯於北邊,秦王世民屯於並州,以備突厥。八月壬子,淮南道行臺左僕射輔公祏反。乙丑,趙郡王孝恭討之。九月壬辰,秦王世民為江州道行軍元帥。丙申,渝州人張大智反。十月丙午,殺廣州都督劉世讓。戊申,降死罪,流以下原之。己未,如華陰。張大智降。庚申,獵於白鹿原。壬戌,石虞侯率杜士遠殺高滿政,以朔州反。丁卯,突厥請和。十一月壬午,張善安襲殺黃州總管周法明。丁亥,如華陰。辛卯,獵於沙苑。丁酉,獵於伏龍原。十二月壬寅朔,日有食之。癸卯,張善安降。庚戌,以奉義監為龍躍宮,武功宅為慶善宮。甲寅,至自華陰。



 七年正月庚寅,鄒州人鄧同穎殺其刺史李士衡。二月丁巳,釋奠於國學。己未,漁陽部將張金樹殺高開道以降。三月戊戌,趙郡王孝恭敗輔公祐,執之。己亥,孝恭殺趙州都督闞棱。四月庚子,大赦。班新律令。給復江州道二年、揚越一年。五月丙戌,作仁智宮。六月辛丑,如仁智宮。壬戌,慶州都督楊文干反。七月己巳,突厥寇朔州,總管秦武通敗之。癸酉,慶州人殺楊文乾以降。甲午,至自仁智宮。巂州地震山崩,遏江水。閏月己未,秦王世民、齊王元吉屯於豳州,以備突厥。八月己巳,吐谷渾寇鄯州,驃騎將軍彭武傑死之。戊寅,突厥寇綏州,刺史劉大俱敗之。壬辰,突厥請和。丁酉,裴寂使於突厥。十月丁卯,如慶善宮。辛未,獵於鄠南。癸酉,幸終南山。丙子,謁樓觀老子祠。庚寅,獵於圍川。十二月丁卯,如龍躍宮。戊辰,獵於高陵。庚午,至自高陵。太子詹事裴矩檢校侍中。



 八年二月癸未,慮囚。四月甲申,如鄠,獵於甘谷。作太和宮。丙戌,至自鄠。六月甲子,如太和宮。七月丙午,至自太和宮。丁巳,秦王世民屯於蒲州,以備突厥。八月壬申,並州行軍總管張瑾及突厥戰於太谷,敗績,鄆州都督張德政死之,執行軍長史溫彥博。甲申,任城郡王道宗及突厥戰於靈州,敗之。丁亥,突厥請和。十月辛巳,如周氏陂,獵於北原。壬午,如龍躍宮。十一月辛卯,如宜州,獵於西原。裴矩罷。庚子,講武於同官。天策府司馬宇文士及權檢校侍中。辛丑,徙封元為吳王,元慶陳王。癸卯,秦王世民為中書令,齊王元吉為侍中。癸丑,獵於華池北原。十二月辛酉,至自華池。庚辰,獵於鳴犢泉。辛巳,至自鳴犢泉。



 九年正月甲寅,裴寂為司空。二月庚申,齊王元吉為司徒。壬午,有星孛於胃、昴。丁亥,孛於卷舌。三月庚寅,幸昆明池,習水戰。壬辰,至自昆明池。丙午,如周氏陂。乙卯,至自周氏陂。丁巳,突厥寇涼州,都督、長樂郡王幼良敗之。四月辛巳,廢浮屠,老子法。六月丁巳,太白經天。庚申,秦王世民殺皇太子建成、齊王元吉。大赦。復浮屠、老子法。癸亥,立秦王世民為皇太子,聽政。賜為父後者襲勛、爵,赤牒官得為真,免民逋租宿賦。己卯,太白晝見。庚辰,幽州都督、廬江郡王瑗反,伏誅。癸未,赦幽州管內為瑗所詿誤者。七月辛卯,楊恭仁罷。太子右庶子高士廉為侍中,左庶子房玄齡為中書令,蕭瑀為尚書左僕射。癸巳,宇文士及為中書令,封德彞為尚書左僕射。辛亥,太白晝見。甲寅,太白晝見。八月丙辰,突厥請和。丁巳,太白晝見。壬戌,吐谷渾請和。甲子,皇太子即皇帝位。



 貞觀三年,太上皇徙居大安宮。九年五月,崩於垂拱前殿,年七十一。謚曰大武,廟號高祖。上元元年,改謚神堯皇帝。天寶八載,謚神堯大聖皇帝;十三載,增謚神堯大聖大光孝皇帝。



 贊曰:自古受命之君,非有德不王。自夏后氏以來,始傳以世,而有賢有不肖,故其為世,數亦或短或長。論者乃謂周自後稷至於文、武,積功累仁,其來也遠,故其為世尤長。然考於《世本》,夏、商、周皆出於黃帝,夏自鯀以前,商自契至於成湯,其間寂寥無聞,與周之興異矣。而漢亦起於亭長叛亡之徒。及其興也,有天下皆數百年而後已。由是言之,天命豈易知哉!然考其終始治亂,顧其功德有厚薄與其制度紀綱所以維持者何如,而其後世,或浸以隆昌,或遽以壞亂,或漸以陵遲,或能振而復起,或遂至於不可支持,雖各因其勢,然有德則興,無德則絕,豈非所謂天命者常不顯其符,而俾有國者兢兢以自勉耶?唐在周、隋之際,世雖貴矣,然烏有所謂積功累仁之漸,而高祖之興,亦何異因時而特起者歟?雖其有治有亂,或絕或微,然其有天下年幾三百,可謂盛哉!豈非人厭隋亂而蒙德澤,繼以太宗之治,制度紀綱之法,後世有以憑藉扶持,而能永其天命歟?



\end{pinyinscope}