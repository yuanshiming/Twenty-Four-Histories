\article{列傳第九十三 韋王陸劉柳程}

\begin{pinyinscope}

 韋執誼,京兆舊族也。幼有才。及進士第,對策異等,授右拾遺。年逾冠,入翰林為學士喻老《韓非子》篇名。以博喻方法發揮老子思想。提出,便敏側媚,得幸於德宗。使豫詩歌屬和,被詔稱旨。與裴延齡、韋渠牟等寵相埒,出入備顧問。帝誕日,皇太子獻畫浮屠象,帝使執誼贊之,太子賜以帛,詔執誼到東宮謝。太子卒見無所藉言者,乃曰:「君知王叔文乎?美才也。」執誼繇是與叔文善。以母喪解。終喪,為吏部郎中,數召至禁中。補闕張正一以上書召見,所善王仲舒、韋成季、劉伯芻、裴愬、常仲孺、呂洞往賀之,或謂執誼曰:「彼將論君與叔文鉤黨事。」執誼即白成季等朋比,有所窺望。帝詔金吾伺,得相過食飲狀,悉逐出之。



 順宗立,以疾不親政,叔文用事,乃擢執誼為尚書左丞、同中書門下平章事。叔文與王稻居中竊命,欲執誼據以奉行,因用迷奪朝權。執誼既為所引,然外迫公議,欲示天下非黨與者,乃時時異論相可否,而密謝叔文曰:「不敢負約,欲共濟國家事爾。」叔文數為所梗,遂詬怒,反成仇怨。及憲宗受內禪,流叔文、伾,分北支黨,貶執誼為崖州司戶參軍。帝以宰相杜黃裳之婿,故最後貶。



 執誼已失形勢,知禍且及,雖尚在位,而臨事奄奄無氣,聞人足聲輒悸動,至於敗。始未顯時,不喜人言嶺南州縣。既為郎,嘗詣職方觀圖,至嶺南輒瞑目,命左右徹去。及為相,所坐堂有圖,不就省。既易旬,試觀之,崖州圖也,以為不祥,惡之。果貶死。



 王叔文,越州山陰人。以棋待詔。頗讀書,班班言治道。德宗詔直東宮,太子引以侍讀,因論政及宮市之弊。太子曰:「寡人見上,將極言之。」坐皆趣贊,叔文獨嘿然。既罷,太子曰:「向君無言,何哉?」叔文曰:「太子之事上,非視膳問安無與也。且陛下在位久,有如小人間之,謂殿下收厭群情,則安解乎?」太子謝曰:「非先生不聞此言!」繇是重之,宮中事咸與參訂。



 叔文淺中浮表,遂肆言不疑,曰:「某可為相,某可為將,它日幸用之。」陰結天下有名士,而士之欲速進者,率諧附之,若韋執誼、陸質、呂溫、李景儉、韓曄、韓泰、陳諫、柳宗元、劉禹錫為死友,而凌準、程異又因其黨進,出入詭秘,外莫得其端。強籓劇帥,或陰相賂遺以自結。



 順宗立,不能聽政,深居施幄坐,以牛昭容、宦人李忠言侍側,群臣奏事,從幄中可其奏。王伾密語諸黃門:「陛下素厚叔文。」即繇蘇州司功參軍拜起居郎、翰林學士。大抵叔文因伾,伾因忠言,忠言因昭容,更相依仗。伾主傳受,叔文主裁可,乃授之中書,執誼作詔文施行焉。時景儉居親喪,溫使吐蕃,惟質、泰、諫、準、畢、宗元、禹錫等倡譽之,以為伊、周、管、葛復出,心間然謂天下無人。叔文每言:「錢穀者,國大本,操其柄,可因以市士。」乃白用杜佑領度支、鹽鐵使,己副之,實專其政。不淹時,遷戶部侍郎。



 宦人俱文珍忌其權,罷叔文學士。詔出,駭悵曰:「吾當數至此議事。不然,無繇入禁中。」伾復力請,乃聽三五日一至翰林,然不得舊職矣。在省不事所職,日引其黨謀取神策兵,制天下之命。乃以宿將範希朝為西北諸鎮行營兵馬使,泰為司馬副之。於是諸將移書中尉,告且去,宦人始悟奪其權,大怒曰:「吾屬必死其手!」乃諭諸鎮,慎毋以兵屬人。希朝、泰到奉天,諸將不至,乃還。



 叔文母死,匿不發,置酒翰林,忠言、文珍等皆在,裒金以餉,因揚言曰:「天子適射兔苑中,跨鞍若飛,敢異議者斬。」又自陳:「親疾病,以身任國大事,朝夕不得侍,今當請急,宜聽。然向之悉心戮力,難易亡所避,報天子異知爾。今一去此,則百謗至,孰為吾助者?」又言:「羊士諤毀短我,我將杖殺之,而執誼懦不果。劉闢來為韋皋求三川,吾生平不識闢,便欲前執吾手,非兇人邪?掃木場將斬之,而執誼持不可。每念失此二賊,令人悵恨。」又陳領度支所以興利去害者為己勞。文珍隨語詰折,叔文不得對。左右竊語曰:「母死已腐,方留此,將何為邪?」明日,乃發喪。執誼益不用其語,乃謀起復,斬執誼與不附己者,聞者恟懼。



 廣陵王為太子,群臣皆喜,獨叔文有憂色,誦杜甫諸葛祠詩以自況,歔欷泣下。太子已監國,貶渝州司戶參軍。明年,誅死。



 王伾者,杭州人。始以書待詔翰林,入太子宮侍書。順宗立,遷左散騎常侍、待詔。伾本闒茸,貌■陋,楚語,無它大志,帝褻寵之,不如叔文任氣好言事,為帝所禮。至出處,又不及伾之無間也,叔文入止翰林,而伾至柿林院,見牛昭容等。當其黨盛,門皆若沸羹,而伾尤通天下賕謝,日月不闋。為巨櫝,裁竅以受珍,使不可出,則寢其上。



 叔文既居喪,伾日請中人及杜佑起叔文為宰相,且總北軍,不許;又請以威遠軍使同中書門下平章事,復不可。乃一日三表,皆不報。憂悸,行且臥。至夕,大呼曰:「吾疾作。」輿歸第。貶開州司馬,死其所。支黨皆逐,惟質以前死免。



 曄者,滉族子,有俊才。以司封郎中貶饒州司馬。終永州刺史。



 諫警敏,嘗覽染署歲簿,悉能言其尺寸。所治,一閱籍,終身不忘。自河中少尹貶臺州司馬,終循州刺史。



 準,字宗一,有史學。自翰林學士貶連州司馬,死於貶。



 泰,字安平,有籌畫,伾、叔文所倚重,能決大事。以戶部郎中、神策行營節度司馬貶虔州司馬。終湖州刺史。



 陸質,字伯沖。七代祖澄,仕梁為名儒。世居吳。明《春秋》,師事趙匡,匡師啖助,質盡傳二家學。陳少游鎮淮南,表在幕府,薦之朝,授左拾遺。累遷左司郎中,歷信、臺二州刺史。



 質素善韋執誼,方執誼附叔文竊威柄,用其力召為給事中。憲宗為太子,詔侍讀。質本名淳,避太子名,故改。時執誼懼太子怒己專,故以質侍東宮,陰伺意解釋左右之。質伺間有所言,太子輒怒曰:「陛下命先生為寡人講學,何可及它?」質惶懼出。



 執誼未敗時,質病甚,太子已即位,為臨問加禮。卒,門人以質能文聖人書,通於後世,私共謚曰文通先生。所著書甚多,行於世。



 劉禹錫,字夢得,自言系出中山。世為儒。擢進士第,登博學宏辭科,工文章。淮南杜佑表管書記,入為監察御史。素善韋執誼。時王叔文得幸太子,禹錫以名重一時,與之交,叔文每稱有宰相器。太子即位,朝廷大議秘策多出叔文,引禹錫及柳宗元與議禁中,所言必從。擢屯田員外郎,判度支、鹽鐵案,頗馮藉其勢,多中傷士。若武元衡不為柳宗元所喜,自御史中丞下除太子右庶子;御史竇群劾禹錫挾邪亂政,群即日罷;韓皋素貴,不肯親叔文等,斥為湖南觀察使。凡所進退,視愛怒重輕,人不敢指其名,號「二王、劉、柳」。



 憲宗立,叔文等敗,禹錫貶連州刺史,未至,斥朗州司馬。州接夜郎諸夷,風俗陋甚,家喜巫鬼,每祠,歌《竹枝》,鼓吹裴回,其聲傖佇。禹錫謂屈原居沅、湘間作《九歌》,使楚人以迎送神,乃倚其聲,作《竹枝辭》十餘篇。於是武陵夷俚悉歌之。



 始,坐叔文貶者八人,憲宗欲終斥不復,乃詔雖後更赦令不得原。然宰相哀其才且困,將澡濯用之,會程異復起領運務,乃詔禹錫等悉補遠州刺史。而元衡方執政,諫官頗言不可用,遂罷。



 禹錫久落魄,鬱鬱不自聊,其吐辭多諷托幽遠,作《問大鈞》、《謫九年》等賦數篇。又敘:「張九齡為宰相,建言放臣不宜與善地,悉徙五溪不毛處。然九齡自內職出始安,有瘴癘之嘆;罷政事守荊州,有拘囚之思。身出遐陬,一失意不能堪,矧華人士族必致丑地,然後快意哉!議者以為開元良臣,而卒無嗣,豈忮心失恕,陰責最大,雖它美莫贖邪!」欲感諷權近,而憾不釋。久之,召還。宰相欲任南省郎,而禹錫作《玄都觀看花君子》詩,語譏忿,當路者不喜,出為播州刺史。詔下,御史中丞裴度為言:「播極遠,猿狖所宅,禹錫母八十餘,不能往,當與其子死訣,恐傷陛下孝治,請稍內遷。」帝曰:「為人子者宜慎事,不貽親憂。若禹錫望它人,尤不可赦。」度不敢對,帝改容曰:「朕所言,責人子事,終不欲傷其親。」乃易連州,又徙夔州刺史。



 禹錫嘗嘆天下學校廢,乃奏記宰相曰:



 言者謂天下少士,而不知養材之道,鬱堙不揚,非天不生材也。是不耕而嘆廩庾之無餘,可乎?貞觀時,學舍千二百區,生徒三千餘,外夷遣子弟入附者五國。今室廬圮廢,生徒衰少,非學官不振,病無貲以給也。



 凡學官,春秋釋奠於先師,斯止闢雍、宮,非及天下。今州縣咸以春秋上丁有事孔子廟,其禮不應古,甚非孔子意。漢初群臣起屠販,故孝惠、高後間置原廟於郡國,逮元帝時,韋玄成遂議罷之。夫子孫尚不敢違禮饗其祖,況後學師先聖道而欲違之。《傳》曰:「祭不欲數。」又曰:「祭神如神在。」與其煩於薦饗,孰若行其教?今教頹靡,而以非禮之祀媚之,儒者所宜疾。竊觀歷代無有是事。



 武德初,詔國學立周公、孔子廟,四時祭。貞觀中,詔修孔子廟兗州。後許敬宗等奏天下州縣置三獻官,其他如立社。玄宗與儒臣議,罷釋奠牲牢,薦酒脯。時王孫林甫為宰相,不涉學,使御史中丞王敬從以明衣牲牢著為令,遂無有非之者。今夔四縣歲釋奠費十六萬,舉天下州縣歲凡費四千萬,適資三獻官飾衣裳,飴妻子,於學無補也。



 請下禮官博士議,罷天下州縣牲牢衣幣,春秋祭如開元時,籍其資半畀所隸州,使增學校,舉半歸太學,猶不下萬計,可以營學室,具器用,豐饌食,增掌故,以備使令,儒官各加稍食,州縣進士皆立程督,則貞觀之風,粲然可復。



 當時不用其言。



 由和州刺史入為主客郎中,復作《游玄都》詩,且言:「始謫十年,還京師,道士植桃,其盛若霞。又十四年過之,無復一存,唯兔葵、燕麥動搖春風耳。」以詆權近,聞者益薄其行。俄分司東都。宰相裴度兼集賢殿大學士,雅知禹錫,薦為禮部郎中、集賢直學士。度罷,出為蘇州刺史。以政最,賜金紫服。徙汝、同二州。遷太子賓客,復分司。



 禹錫恃才而廢,褊心不能無怨望,年益晏,偃蹇寡所合,乃以文章自適。素善詩,晚節尤精,與白居易酬復頗多。居易以詩自名者,嘗推為「詩豪」,又言:「其詩在處,應有神物護持。」



 會昌時,加檢校禮部尚書。卒,年七十二,贈戶部尚書。始疾病,自為《子劉子傳》,稱:「漢景帝子勝,封中山,子孫為中山人。七代祖亮,元魏冀州刺史,遷洛陽,為北部都昌人,墳墓在洛北山,後其地狹不可依,乃葬滎陽檀山原。德宗棄天下,太子立,時王叔文以善弈得通籍,因間言事,積久,眾未知。至起蘇州掾,超拜起居舍人、翰林學士,陰薦丞相杜佑為度支、鹽鐵使。翌日,自為副,貴震一時。叔文,北海人,自言猛之後,有遠祖風,東平呂溫、隴西李景儉、河東柳宗元以為信然。三子者皆予厚善,日夕過,言其能。叔文實工言治道,能以口辯移人,既得用,所施為人不以為當。太上久疾,宰臣及用事者不得對,宮掖事秘,建桓立順,功歸貴臣,由是及貶。」其自辯解大略如此。



 柳宗元,字子厚,其先蓋河東人。從曾祖奭為中書令,得罪武後,死高宗時。父鎮,天寶末遇亂,奉母隱王屋山,常間行求養,後徙於吳。肅宗平賊,鎮上書言事,擢左衛率府兵曹參軍。佐郭子儀朔方府,三遷殿中侍御史。以事觸竇參,貶夔州司馬。還,終侍御史。



 宗元少精敏絕倫,為文章卓偉精致,一時輩行推仰。第進士、博學宏辭科,授校書郎,調藍田尉。貞元十九年,為監察御史裏行。善王叔文、韋執誼,二人者奇其才。及得政,引內禁近,與計事,擢禮部員外郎,欲大進用。



 俄而叔文敗,貶邵州刺史,不半道,貶永州司馬。既竄斥,地又荒癘,因自放山澤間,其堙厄感鬱,一寓諸文,仿《離騷》數十篇,讀者咸悲惻。雅善蕭人免,詒書言情曰:



 僕向者進當臲卼不安之勢,平居閉門,口舌無數,又久興游者,岌岌而操其間。其求進而退者,皆聚為仇怨,造作粉飾,蔓延益肆。非的然昭晰、自斷於內,孰能了僕於冥冥間哉?僕當時年三十三,自御史裏行得禮部員外郎,超取顯美,欲免世之求進者怪怒媢疾,可得乎?與罪人交十年,官以是進,辱在附會。聖朝寬大,貶黜甚薄,不塞眾人之怒,謗語轉侈,囂囂嗷嗷,漸成怪人。飾智求仕者,更詈僕以悅仇人之心,日為新奇,務相悅可,自以速援引之路。僕輩坐益困辱,萬罪橫生,不知其端,悲夫!人生少六七十者,今三十七矣,長來覺日月益促,歲歲更甚,大都不過數十寒暑,無此身矣。是非榮辱,又何足道!云云不已,祗益為罪。



 居蠻夷中久,慣習炎毒,昏眊重膇,意以為常。忽遇北風晨起,薄寒中體,則肌革慘懍,毛發蕭條,瞿然注視,怵惕以為異候,意緒殆非中國人也。楚、越間聲音特異,鴂舌啅噪,今聽之恬然不怪,已與為類矣。家生小童,皆自然嘵嘵,晝夜滿耳;聞北人言,則啼呼走匿,雖病夫亦怛然駭之。出門見適州閭市井者,其十八九杖而後興。自料居此,尚復幾何,豈可更不知止,言說長短,重為一世非笑哉?讀《易·困卦》至「有言不信,尚口乃窮」,往復益喜,曰:「嗟乎!余雖家置一喙以自稱道,詬益甚耳。」用是更樂喑默,與木石為徒,不復致意。



 今天子興教化,定邪正,海內皆欣欣怡愉,而僕與四五子者,淪陷如此,豈非命歟?命乃天也,非雲云者所制,又何恨?然居治平之世,終身為頑人之類,猶有少恥,未能盡忘。儻因賊平慶賞之際,得以見白,使受天澤餘潤,雖朽枿敗腐不能生植,猶足蒸出芝菌,以為瑞物。一釋廢錮,移數縣之地,則世必曰罪稍解矣。然後收召魂魄,買土一廛為耕氓,朝夕歌謠,使成文章,庶木鐸者採取,獻之法宮,增聖唐大雅之什,雖不得位,亦不虛為太平人矣。



 又詒京兆尹許孟容曰:



 宗元早歲與負罪者親善,始奇其能,謂可以共立仁義,裨教化。過不自料,勤勤勉勵,唯以忠正信義為志,興堯、舜、孔子道,利安元元為務,不知愚陋不可以強,其素意如此也。末路厄塞臲卼,事既壅隔,很忤貴近,狂疏繆戾,蹈不測之辜。今黨與幸獲寬貸,各得善地,無公事,坐食奉祿,德至渥也。尚何敢更俟除棄廢痼,希望外之澤哉?年少氣銳,不識幾微,不知當否,但欲一心直遂,果陷刑法,皆自所求取,又何怪也?



 宗元於眾黨人中,罪狀最甚,神理降罰,又不能即死,猶對人語言,飲食自活,迷不知恥,日復一日。然亦有大故。自以得姓來二千五百年,代為塚嗣,今抱非常之罪,居夷獠之鄉,卑濕昏霧,恐一日填委溝壑,曠墜先緒,以是怛然痛恨,心骨沸熱。煢煢孤立,未有子息,荒陬中少士人女子,無與為婚,世亦不肯與罪人親暱,以是嗣續之重,不絕如縷。每春秋時饗,孑立捧奠,顧眄無後繼者,懍懍然欷歔惴惕,恐此事便已,摧心傷骨,若受鋒刃。此誠丈人所共閔惜也。先墓在城南,無異子弟為主,獨托村鄰。自譴逐來,消息存亡不一至,鄉閭主守固以益怠。晝夜哀憤,懼便毀傷松柏,芻牧不禁,以成大戾。近世禮重拜掃,今闕者四年矣。每遇寒食,則北向長號,以首頓地。想田野道路,士女遍滿,皁隸庸丐,皆得上父母丘墓;馬醫、夏畦之鬼,無不受子孫追養者。然此已息望,又何以云哉?城西有數頃田,樹果數百株,多先人手自封植,今已荒穢,恐便斬伐,無復愛惜。家有賜書三千卷,尚在善和裏舊宅,宅今三易主,書存亡不可知。皆付受所重,常系心腑,然無可為者。立身一敗,萬事瓦裂,身殘家破,為世大僇。是以當食不知辛咸節適,洗沐盥漱,動逾歲時,一搔皮膚,塵垢滿爪,誠憂恐悲傷,無所告訴,以至此也。



 自古賢人才士,秉志遵分,被謗議不能自明者,以百數。故有無兄盜嫂,娶孤女撾婦翁者。然賴當世豪傑分明辨列,卒光史冊。管仲遇盜,升為功臣;匡章被不孝名,孟子禮之。今已無古人之實為而有詬,欲望世人之明己,不可得也。直不疑買金以償同舍;劉寬下車,歸牛鄉人。此誠知疑似之不可辯,非口舌所能勝也。鄭詹束縛於晉,終以無死;鐘儀南音,卒獲返國;叔向囚虜,自期必免;範痤騎危,以生易死;蒯通據鼎耳,為齊上客;張蒼、韓信伏斧鑕,終取將相;鄒陽獄中,以書自治;賈生斥逐,復召宣室;兒寬擯厄,後至御史大夫;董仲舒、劉向下獄當誅,為漢儒宗。此皆瑰偉博辯奇壯之士,能自解脫。今以恇怯淟涊,下才末伎,又嬰痼病,雖欲慷慨攘臂,自同昔人,愈疏闊矣。



 賢者不得志於今,必取貴於後,古之著書者皆是也。宗元近欲務此,然力薄志劣,無異能解,欲秉筆覙縷,神志荒耗,前後遺忘,終不能成章。往時讀書,自以不至牴滯,今皆頑然無復省錄。讀古人一傳,數紙後,則再三伸卷,復觀姓氏,旋又廢失。假令萬一除刑部囚籍,復為士列,亦不堪當世用矣!



 伏惟興哀於無用之地,垂德於不報之所,以通家宗祀為念,有可動心者操之勿失。雖不敢望歸掃塋域,退托先人之廬,以盡餘齒,姑遂少北,益輕瘴癘,就婚娶,求胄嗣,有可付托,即冥然長辭,如得甘寢,無復恨矣!



 然眾畏其才高,懲刈復進,故無用力者。



 宗元久汩振,其為文,思益深。嘗著書一篇,號《貞符》,曰:



 臣所貶州流人吳武陵為臣言:「董仲舒對三代受命之符,誠然?非邪?」臣曰:「非也。何獨仲舒爾,司馬相如、劉向、揚雄、班彪、彪子固皆沿襲嗤嗤,推古瑞物以配受命,其言類淫巫瞽史,誑亂後代,不足以知聖人立極之本,顯至德,揚大功,甚失厥趣。臣為尚書郎時,嘗著《貞符》,言唐家正德受命於生人之意、累積厚久宜享無極之義,本末閎闊。會貶逐中輟,不克備究。」武陵即叩頭邀臣:「此大事,不宜以辱故休缺,使聖王之典不立,無以抑詭類、拔正道、表核萬代。」臣不勝奮激,即具為書。念終泯沒蠻夷,不聞於時,獨不為也。茍一明大道,施於人世,死無所憾,用是自決。臣宗元稽首拜手以聞曰:



 孰稱古初,樸蒙空侗而無爭,厥流以訛,越乃奮奪,鬥怒振動,專肆為淫威?曰:是不知道。惟人之初,總總而生,林林而群。雪霜風雨雷雹暴其外,於是乃知架巢空穴,挽草木,取皮革;饑渴牝牡之欲驅其內,於是乃噬禽獸,咀果穀。合偶而居,交焉而爭,睽焉而鬥,力大者搏,齒利者嚙,爪剛者決,群眾者軋,兵良者殺,披披藉藉,草野塗血。在後強有力者出而治之,往往為曹於險陰,用號令起,而君臣什伍之法立。德紹者嗣,道怠者奪。於是有聖人焉,曰黃帝,游其兵車,交貫乎其內,一統類,齊制量,然猶大公之道不克建。於是有聖人焉,曰堯,置州牧四岳,持而綱之,立有德有功有能者,參而維之,運臂率指,屈伸把握,莫不統率;年老,舉聖人而禪焉,大公乃克建。由是觀之,厥初罔匪極亂,而後稍可為也。而非德不樹,故仲尼敘《書》,於堯曰「克明俊德」,於舜曰「濬哲文明」,於禹曰「文命祗承於帝」,於湯曰「克寬克仁,章信兆民」,於武王曰「有道曾孫」。稽揆典誓,貞哉惟茲德,實受命之符,以奠永祀。後之祅淫囂昏好怪之徒,乃始陳大電、大虹、玄鳥、巨跡、白狼、白魚、流火之烏以為符,斯皆詭譎闊誕,其可羞也,莫知本於厥貞。



 漢用大度,克懷於有氓,登能庸賢,濯痍煦寒,以瘳以熙,茲其為符也。而其妄臣,乃下取虺蛇,上引天光,推類號休,用誇誣於無知氓,增以騶虞、神鼎,脅驅縱踴,俾東之泰山、石閭,作大號謂之「封禪」,皆《尚書》所無有。莽、述承效,卒奮驁逆。其後有賢帝曰光武,克綏天下,復承舊物,猶崇《赤伏》,以玷厥德。魏、晉而下,尨亂鉤裂,厥符不貞,邦用不靖,亦罔克久,駁乎無以議為也。



 積大亂至於隋氏,環四海以為鼎,跨九垠以為爐,爨以毒燎,煽以虐焰,其人沸湧灼爛,號呼騰蹈,莫有救止。於是大聖乃起,丕降霖雨,濬滌蕩沃,蒸為清氛,疏為泠風,人乃漻然休然,相晞以生,相持以成,相彌以寧。琢斮屠剔膏流節離之禍不作,而人乃克完平舒愉,尸其肌膚,以達於夷途。焚坼抵掎奔走轉死之害不起,而人乃克鳩類集族,歌舞悅懌,用抵於元德。徒奮袒呼,犒迎義旅,歡動六合,至於麾下。大盜豪據,阻命遏德,義威殄戮,咸墜厥緒。無劉於虐,人乃並受休嘉,去隋氏,克歸於唐,躑躅謳歌,灝灝和寧。帝庸威慄,惟人之為。敬奠厥賦,積藏於下,是謂豐國。鄉為義廩,斂發謹飭,歲丁大侵,人以有年。簡於厥刑,不殘而懲,是謂嚴威。小屬而支,大生而孥,愷悌祗敬,用底於治。凡其所欲,不謁而獲;凡其所惡,不祈而息。四夷稽服,不作兵革,不竭貨力。丕揚於後嗣,用垂於帝式,十聖濟厥治,孝仁平寬,惟祖之則。澤久而逾深,仁增而益高,人之戴唐,永永無窮。



 是故受命不於天,於其人;休符不於祥,於其仁。惟人之仁,匪祥於天。匪祥於天,茲惟貞符哉!未有喪仁而久者也,未有恃祥而壽者也。商之王以桑穀昌,以雉鴝大,宋之君以法星壽,鄭以龍衰,魯以麟弱,白雉亡漢,黃犀死莽,惡在其為符也?不勝唐德之代,光紹明濬,深鴻尨大,保人斯無疆,宜薦於郊廟,文之雅詩,祗告於德之休。帝曰諶哉!乃黜休祥之奏,究貞符之奧,思德之所未大,求仁之所未備,以極於邦治,以敬於人事。其詩曰:



 於穆敬德,黎人皇之。惟貞厥符,浩浩將之。仁函於膚,刃莫畢屠。澤於爨,灊炎以澣。勃厥兇德,乃驅乃夷。懿其休風,是煦是吹。父子熙熙,相寧以嬉。賦徹而藏,厚我糗粻。刑輕以清,我完靡傷。貽我子孫,百代是康。十聖嗣於治,仁後之子。子思孝父,易患於己。拱之戴之,神其爾宜。載揚於雅,承天之嘏。天之誠神,宜鑒於仁。神之曷依?宜仁之歸。濮金公於北,祝慄於南,幅員西東,祗一乃心。祝唐之紀,後天罔墜;祝皇之壽,與地咸久。曷徒祝之,心誠篤之。神協人同,道以告之。俾彌億萬年,不震不危。我代之延,永永毘之。仁增以崇,曷不爾思?有號於天,僉曰嗚呼,咨爾皇靈,無替厥符!



 宗元不得召,內閔悼,悔念往吝,作賦自儆曰:



 懲咎愆以本始兮,孰非余心之所求?處卑污以閔世兮,固前志之為尤。始余學而觀古兮,怪今昔之異謀。惟聰明為可考兮,追駿步而遐游。絜誠之既信直兮,仁友藹而萃之。日施陳以系縻兮,邀堯舜禹之為。上睢盱而混茫兮,下駁詭而懷私。旁羅列以交貫兮,求大中之所宜。



 曰道有象兮,而無其形。推變乘時兮,與志相迎。不及則殆兮,過則失貞。謹守而中兮,與時偕行。萬類蕓蕓兮,率由以寧。剛柔弛張兮,出入綸經。登能抑枉兮,白黑濁清。蹈乎大方兮,物莫能嬰。



 奉訏謨以植內兮,欣餘志之有獲。再明信乎策書兮,謂耿然而不惑。愚者果於自用兮,惟懼夫誠之不一。不顧慮以周圖兮,專茲道以為服。讒妒構而不戒兮,猶斷斷於所執。哀吾黨之不淑兮,遭任遇之卒迫。勢危疑而多詐兮,逢天地之否隔。欲圖退而保己兮,悼乖期乎曩昔。欲操術以致忠兮,眾呀然而互嚇。進與退吾無歸兮,甘脂潤兮鼎鑊。幸皇鑒之明宥兮,累郡印而南適。惟罪大而寵厚兮,宜夫重仍乎禍謫。既明懼乎天討兮,又幽心慄乎鬼責。惶惶乎夜寤而晝駭兮,類鹿濩秬之不息。



 凌洞庭之洋洋兮,溯湘流之沄沄。飄風擊以揚波兮,舟摧抑而回邅。日霾曀以昧幽兮,黝雲湧而上屯。暮屑窣以淫雨兮,聽嗷嗷之哀猿。眾鳥萃而啾號兮,沸洲渚以連山。漂遙逐其詎止兮,逝莫屬余之形魂。攢巒奔以紆委兮,束洶湧之崩湍。畔尺進而尋退兮,蕩洄汩乎淪漣。際窮冬而止居兮,羈累棼以縈纏。



 哀吾生之孔艱兮,循《凱風》之悲詩。罪通天而降酷兮,不亟死而生為!逾再歲之寒暑兮,猶貿貿而自持。將沈淵而隕命兮,詎蔽罪以塞禍?惟滅身而無後兮,顧前志猶未可。進路呀以劃絕兮,退伏匿又不果。為孤囚以終世兮,長拘攣而轗軻。



 曩餘志之脩蹇兮,今何為此戾也?豈貪食而盜名兮,不混同於世也。將顯身以直遂兮,眾之所宜蔽也。不擇言以危肆兮,固群禍之際也。



 禦長轅之無橈兮,行九折之峨峨。卻驚棹以橫江兮,溯凌天之騰波。幸餘死之已緩兮,完形軀之既多。茍餘齒之有懲兮,蹈前烈而不頗。死蠻夷固吾所兮,雖顯寵其焉加?配大中以為偶兮,諒天命之謂何!



 元和十年,徙柳州刺史。時劉禹錫得播州,宗元曰:「播非人所居,而禹錫親在堂,吾不忍其窮,無辭以白其大人,如不往,便為母子永決。」即具奏欲以柳州授禹錫而自往播。會大臣亦為禹錫請,因改連州。



 柳人以男女質錢,過期不贖,子本均,則沒為奴婢。宗元設方計,悉贖歸之。尤貧者,令書庸,視直足相當,還其質。已沒者,出己錢助贖。南方為進士者,走數千里從宗元游,經指授者,為文辭皆有法。世號「柳柳州」。十四年卒,年四十七。



 宗元少時嗜進,謂功業可就。既坐廢,遂不振。然其才實高,名蓋一時。韓愈評其文曰:「雄深雅健,似司馬子長,崔、蔡不足多也。」既沒,柳人懷之,托言降於州之堂,人有慢者輒死。廟於羅池,愈因碑以實之云。



 程異,字師舉,京兆長安人。居鄉以孝稱。第明經,再補鄭尉。精吏治,為叔文所引,由監察御史為鹽鐵揚子院留後。叔文敗,貶郴州司馬。



 李巽領鹽鐵,薦異心計可任,請拔擢用之,乃授侍御史,復為揚子留後。稍遷淮南等道兩稅使。異起退廢,能厲己竭節,悉矯革征利舊弊。入遷累衛尉卿、鹽鐵轉運副使。方討蔡,異使江表調財用,因行諭諸帥府,以羨贏貢。故異所至,不剝下,不加斂,經用以饒。遂兼御史大夫為鹽鐵使。元和十三年,以工部侍郎同中書門下平章事,猶領鹽鐵。異以錢穀奮而至宰相,自以非人望,久不敢當印秉筆。明年,西北軍政不治,議置巡邊使,憲宗問孰可者,乃自請行。會卒,贈尚書左僕射,謚曰恭。身歿官第,無留貲,世重其廉云。



 贊曰:叔文沾沾小人,竊天下柄,與陽虎取大弓《春秋》書為盜無以異。宗元等橈節從之,徼幸一時,貪帝病昏,抑太子之明,規權遂私。故賢者疾,不肖者媢,一僨而不復,宜哉!彼若不傅匪人,自勵材猷,不失為明卿才大夫,惜哉!



\end{pinyinscope}