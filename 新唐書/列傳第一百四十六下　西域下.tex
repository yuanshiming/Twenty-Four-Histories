\article{列傳第一百四十六下 西域下}

\begin{pinyinscope}

 康者,一曰薩末鞬,亦曰颯秣建,元魏所謂悉斤者。其南距史百五十里,西北距西曹百餘里無為而治。主張「是非有分,以法斷之;虛靜謹聽,以法為,東南屬米百里,北中曹五十里。在那密水南,大城三十,小堡三百。君姓溫,本月氏人。始居祁連北昭武城,為突厥所破,稍南依蔥嶺,即有其地。枝庶分王,曰安,曰曹,曰石,曰米,曰何,曰火尋,曰戊地,曰史,世謂「九姓」,皆氏昭武。土沃宜禾,出善馬,兵強諸國。人嗜酒,好歌舞於道。王帽氈,飾金雜寶。女子盤髻,蒙黑巾,綴金花。生兒以石蜜啖之,置膠於掌,欲長而甘言,持珤若黏云。習旁行書。善商賈,好利,丈夫年二十,去傍國,利所在無不至。以十二月為歲首,尚浮圖法,祠祅神,出機巧技。十一月鼓舞乞寒,以水交潑為樂。



 隋時,其王屈木支娶西突厥女,遂臣突厥。武德十年,始遣使來獻。貞觀五年,遂請臣。太宗曰:「朕惡取虛名,害百姓;且康臣我,緩急當同其憂。師行萬里,寧朕志邪?」卻不受。俄又遣使獻師子獸,帝珍其遠,命秘書監虞世南作賦。自是歲入貢,致金桃、銀桃,詔令植苑中。



 高宗永徽時,以其地為康居都督府,即授其王拂呼縵為都督。萬歲通天中,以大首領篤娑缽提為王。死,子泥涅師師立。死,國人立突昏為王。開元初,貢鎖子鎧、水精杯、碼瓶、駝鳥卵及越諾、硃儒、胡旋女子。其王烏勒伽與大食亟戰不勝,來乞師,天子不許。久之,請封其子咄曷為曹王,默啜為米王,詔許。烏勒伽死,遣使立咄曷,封欽化王,以其母可敦為郡夫人。



 安者,一曰布豁,又曰捕喝,元魏謂忸蜜者。東北至東安,西南至畢,皆百里所。西瀕烏滸河,治阿濫謐城,即康居小君長罽王故地。大城四十,小堡千餘。募勇健者為柘羯。柘羯,猶中國言戰士也。武德時,遣使入朝。貞觀初,獻方物,太宗厚尉其使曰:「西突厥已降,商旅可行矣。」諸胡大悅。其王訶陵迦又獻名馬,自言一姓相承二十二世云。是歲,東安國亦入獻,言子姓相承十世云。



 東安,或曰小國,曰喝汗,在那密水之陽,東距何二百里許,西南至大安四百里。治喝汗城,亦曰鷿斤。大城二十,小堡百。顯慶時,以阿濫為安息州,即以其王昭武殺為刺史;旟斤為木鹿州,以其王昭武閉息為刺史。開元十四年,其王篤薩波提遣弟阿悉爛達拂耽發黎來朝,納馬豹。後八年,獻波斯婁二,拂菻繡氍球一,鬱金香、石蜜等,其妻可敦獻柘闢大氍球二,繡氍球一,丐賜袍帶、鎧仗及可敦袿衣屬裝澤。



 東曹,或曰率都沙那,蘇對沙那,劫布呾那,蘇都識匿,凡四名。居波悉山之陰,漢貳師城地也。東北距俱戰提二百里,北至石,西至康,東北寧遠,皆四百里許,南至吐火羅五百里。有野叉城,城有巨窟,嚴以關鑰,歲再祭,人向窟立,中即煙出,先觸者死。武德中,與康同遣使入朝。其使曰:「本國以臣為健兒,聞秦王神武,欲隸麾下。」高祖大悅。



 西曹者,隋時曹也,南接史及波覽,治瑟底痕城。東北越於底城有得悉神祠,國人事之。有金具器,款其左曰:「漢時天子所賜。」武德中入朝。天寶元年,王哥邏僕羅遣使者獻方物,詔封懷德王,即上言:「祖考以來,奉天可汗,願同唐人受調發,佐天子征討。」十一載,東曹王設阿忽與安王請擊黑衣大食,玄宗尉之,不聽。



 中曹者,居西曹東,康之北。王治迦底真城。其人長大,工戰鬥。



 石,或曰柘支,曰柘折,曰赭時,漢大宛北鄙也。去京師九千里。東北距西突厥,西北波臘,南二百里所抵俱戰提,西南五百里康也。圓千餘里,右涯素葉河。王姓石,治柘折城,故康居小王窳匿城地。西南有藥殺水,入中國謂之真珠河,亦曰質河。東南有大山,生瑟瑟。俗善戰,多良馬。隋大業初,西突厥殺其王,以特勒匐職統其國。武德、貞觀間,數獻方物。顯慶三年,以瞰羯城為大宛都督府,授其王瞰土屯攝舍提於屈昭穆都督。開元初,封其君莫賀咄吐屯,有功,為石國王。二十八年,又冊順義王。明年,王伊捺吐屯屈勒上言:「今突厥已屬天可汗,惟大食為諸國患,請討之。」天子不許。天寶初,封王子那俱車鼻施為懷化王,賜鐵券。久之,安西節度使高仙芝劾其無蕃臣禮,請討之。王約降,仙芝遣使者護送至開遠門,俘以獻,斬闕下,於是西域皆怨。王子走大食乞兵,攻怛邏斯城,敗仙芝軍,自是臣大食。寶應時,遣使朝貢。



 有碎葉者,出安西西北千里所,得勃達嶺,南抵上國,北突騎施南鄙也,西南直蔥嶺贏二千里。水南流者經中國入於海,北流者經胡入於海。北三日行度雪海,春夏常雨雪。繇勃達嶺北行贏千里,得細葉川。東曰熱海,地寒不凍。西有碎葉城,天寶七載,北庭節度使王正見伐安西,毀之。川長千里,有異姓突厥兵數萬,耕者皆擐甲,相掠為奴婢。西屬怛邏斯城,石常分兵鎮之。自此抵西海矣。三月訖九月,未嘗雨,人以雪水溉田。



 石東南千餘里,有怖捍者,山四環之,地膏腴,多馬羊。西千里距堵利瑟那,東臨葉葉水,水出蔥嶺北原,色濁,西北流入大磧。無水草,望大山,尋遺胔,知所指,五百餘里即康也。



 米,或曰彌末,曰彌秣賀。北百里距康。其君治缽息德城,永徽時為大食所破。顯慶三年,以其地為南謐州,授其君昭武開拙為刺史,自是朝貢不絕。開元時,獻璧、舞筵、師子、胡旋女。十八年,大首領末野門來朝。天寶初,封其君為恭順王,母可敦郡夫人。



 何,或曰屈霜你迦,曰貴霜匿,即康居小王附墨城故地。城左有重樓,北繪中華古帝,東突厥、婆羅門,西波斯、拂菻等諸王,其君旦詣拜則退。貞觀十五年,遣使者入朝。永徽時上言:「聞唐出師西討,願輸糧於軍。」俄以其地為貴霜州,授其君昭武婆達地刺史。遣使者缽底失入謝。



 火尋,或曰貨利習彌,曰過利,居烏滸水之陽。東南六百里距戊地,西南與波斯接,西北抵突厥曷薩,乃康居小王奧鞬城故地。其君治急多颶遮城。諸胡惟其國有車牛,商賈乘以行諸國。天寶十載,君稍施芬遣使者朝,獻黑鹽。寶應時復入朝。



 史,或曰佉沙,曰羯霜那,居獨莫水南康居小王蘇薤城故地。西百五十里距那色波,北二百里屬米,南四百里吐火羅也。有鐵門山,左右巉峭,石色如鐵,為關以限二國,以金錮闔。城有神祠,每祭必千羊,用兵類先禱乃行。國有城五百。隋大業中,其君狄遮始通中國,號最強盛,築乞史城,地方數千里。貞觀十六年,君沙瑟畢獻方物。顯慶時,以其地為佉沙州,授君昭武失阿喝刺史。開元十五年,君忽必多獻舞女、文豹。後君長數死、立,然首領時時入朝。天寶中,詔改史為來威國。



 那色波,亦曰小史,蓋為史所役屬。居吐火羅故地,東厄蔥嶺,西接波剌斯,南雪山。



 循縛芻水北有呾蜜種,亦自國,東西六百里所。又東逾四種,有鑊沙者,廣三百里,長五百里,東界骨咄,接蔥嶺有十八種。南有揭職,稍大,幅員準千里,陵阜連屬,多菽麥,氣寒烈。東南抵雪山六百里,道吐火羅,又逾五種至婆羅睹邏。北逾山行六百里,得烏萇種。東北行二百里至河波羅水,水西南流,春夏涸凍。北歷十二種有婆羅吸摩補羅,最大種,綿地四千里,山周其外,土沃,產鍮、水精。北大雪山,即東女也。歷十九種得摩揭陀。又東過四種,逾大河,有迦摩縷波,皆阪險,地接西南夷,其人類蠻獠。行二月,叩蜀南邊,其東南野象群暴,故戰用象軍。又南歷三十二種有狼揭羅者,地大數千里,其君治窣菟黎濕伐羅城。西北即波剌斯,傳言廣萬里,王治蘇剌薩儻那城。土溫溽,引水為田,人富饒。出金、銀、水精。多工巧,織錦、褐、氍毹。產善馬、橐它。人服錦氎。賦稅,口出四銀錢,又以交易。西北距拂菻,西南際海島,有西女種,皆女子,多珍貨,附拂菻,拂菻君長歲遣男子配焉。俗產男不舉。又有臂、多、勢、羅四種,西北逾大山廣川,歷小城聚,行二千里即謝亹也。北五百里有弗慄恃薩儻那,地橫二千里,縱千里。其君突厥種,治護苾那城。東北大雪山,盛夏常凍,鑿冰乃可度。下有安呾羅縛者,地三千里;西北逾嶺四百里有闊悉多;西北三百里有活種,大二千里。此三種皆居吐火羅故地,臣於突厥,君亦突厥種,主鐵門南諸戎,遷徙不常。東又有七種,東南峽道險甚,無慮三百里,得俱蘭。東北山行五百里,即護密,北識匿也。南有商彌,地大二千里而贏,多蒲陶。生雌黃,鑿石乃得。東北逾山七百里,至波謎羅川,東西千里,南北百里,春夏雨雪。南有缽露種,多紫金。行五百里有朅盤陀。東行八百里出蔥嶺,又八百里至烏鎩,環千里,出白、毚、青三種玉。君長世臣朅盤陀。北徑磧,曠野五百里,得疏勒。東南五百里濟徙多水,逾大沙嶺,有斫句迦種,或曰沮渠,地千里。東逾領八百里,即於闐也,東有媲摩川。度磧行二百里,得尼壤城,在大澤中,地墊洳,蘆菼荒茂,行者鑿道趣城通於闐,而於闐以為東關。又東行入大流沙,人行無跡,故往返輒迷,聚遣骸以識道。無水草,多熱風,觸人及六畜皆迷僕。行四百里至故都邏。又六百里至故折摩馱那,古且末也。又千里至故納縛波,古樓蘭也。



 自呾蜜以下,諸種相與群聚,華人皆以國名之,故未嘗與唐通,傳記雜詭,不可得而考,然其地與諸國連屬,粗序其名云。



 寧遠者,本拔汗那,或曰鏺汗,元魏時謂破洛那。去京師八千里。居西鞬城,在真珠河之北。有大城六,小城百。人多壽。其王自魏、晉相承不絕。每元日,王及首領判二朋,朋出一人被甲斗,眾以瓦石相之,有死者止,以卜歲善惡。



 貞觀中,王契苾為西突厥瞰莫賀咄所殺,阿瑟那鼠匿奪其城。鼠匿死,子遏波之立契苾兄子阿了參為王,治呼悶城;遏波之治渴塞城。顯慶初,遏波之遣使朝貢,高宗厚慰諭。三年,以渴塞城為休循州都督,授阿了參刺史,自是歲朝貢。玄宗開元二十七年,王阿悉爛達干助平吐火仙,冊拜奉化王。天寶三載,改其國號寧遠,帝以外家姓賜其王曰竇,又封宗室女為和義公主降之。十三載,王忠節遣子薛裕朝,請留宿衛,習華禮,聽之,授左武衛將軍。其事唐最謹。



 大勃律,或曰布露。直吐蕃西,與小勃律接,西鄰北天竺、烏萇。地宜鬱金。役屬吐蕃。萬歲通天逮開元時,三遣使者朝,故冊其君蘇弗舍利支離泥為王。死,又冊蘇麟陀逸之嗣王。凡再遣大首領貢方物。



 小勃律去京師九千里而贏,東少南三千里距吐蕃贊普牙,東八百里屬烏萇,東南三百里大勃律,南五百里個失蜜,北五百里當護密之娑勒城。王居孽多城,臨娑夷水。其西山顛有大城曰迦布羅。開元初,王沒謹忙來朝,玄宗以兒子畜之,以其地為綏遠軍。國迫吐蕃,數為所困。吐蕃曰:「我非謀爾國,假道攻四鎮爾。」久之,吐蕃奪其九城,沒謹忙求救北庭,節度使張孝嵩遣疏勒副使張思禮率銳兵四千倍道往,沒謹忙因出兵,大破吐蕃,殺其眾數萬,復九城。詔冊為小勃律王;遣大首領察卓那斯摩沒勝入謝。



 沒謹忙死,子難泥立。死,兄麻來兮立。死,蘇失利之立,為吐蕃陰誘,妻以女,故西北二十餘國皆臣吐蕃,貢獻不入。安西都護三討之無功。天寶六載,詔副都護高仙芝伐之。前遣將軍席元慶馳千騎見蘇失利之曰:「請假道趨大勃律。」城中大酋五六,皆吐蕃腹心。仙芝約元慶:「吾兵到,必走山。出詔書召慰,賜繒糸採。縛酋領待我。」元慶如約。蘇失利之挾妻走,不得其處。仙芝至,斬為吐蕃者,斷娑夷橋。是暮,吐蕃至,不能救。仙芝約王降,遂平其國。於是拂菻、大食諸胡七十二國皆震恐,咸歸附。執小勃律王及妻歸京師,詔改其國號歸仁,置歸仁軍,募千人鎮之。帝赦蘇失利之不誅,授右威衛將軍,賜紫袍、黃金帶,使宿衛。



 吐火羅,或曰土豁羅,曰睹貨邏,元魏謂吐呼羅者。居蔥嶺西,烏滸河之南,古大夏地。與挹怛雜處。勝兵十萬。國土著,少女多男。北有頗黎山,其陽穴中有神馬,國人游牧牝於側,生駒輒汗血。其王號「葉護」。武德、貞觀時再入獻。



 永徽元年,獻大鳥,高七尺,色黑,足類橐駝,翅而行,日三百里,能啖鐵,俗謂駝鳥。顯慶中,以其阿緩城為月氏都督府,析小城為二十四州,授王阿史那都督。後二年,遣子來朝,俄又獻碼鐙樹,高三尺。神龍元年,王那都泥利遣弟僕羅入朝,留宿衛。開元、天寶間數獻馬、婁、異藥、乾陀婆羅二百品、紅碧玻瓈,乃冊其君骨咄祿頓達度為吐火羅葉護、挹怛王。其後,鄰胡羯師謀引吐蕃攻吐火羅,於是葉護失裡忙伽羅丐安西兵助討,帝為出師破之。乾元初,與西域九國發兵為天子討賊,肅宗詔隸朔方行營。



 挹怛國,漢大月氏之種。大月氏為烏孫所奪,西過大宛,擊大夏臣之。治藍氏城。大夏即吐火羅也。嚈噠,王姓也,後裔以姓為國,訛為挹怛,亦曰挹闐。俗類突厥,天寶中遣使朝貢。



 俱蘭,或曰俱羅弩,曰屈浪挐,與吐火羅接,環地三千里,南大雪山,北俱魯河。出金精,琢石取之。貞觀二十年,其王忽提婆遣使者來獻,書辭類浮屠語。



 劫者,居蔥嶺中,西及南距賒彌,西北挹怛也。去京師萬二千里。氣常熱,有稻、麥、粟、豆。畜羊馬。俗死棄於山。武德二年,遣使者獻寶帶、玻瓈、水精杯。



 越底延者,南三千里距天竺,西北千里至賒彌,東北五千里至瓜州,居辛頭水之北。其法不殺人,重罪流,輕罪放。無租稅。俗翦發,被錦袍,貧者白氎。自澡潔。氣溫,多稻、米、石蜜。



 謝居吐火羅西南,本曰漕矩吒,或曰漕矩,顯慶時謂訶達羅支,武後改今號。東距罽賓,東北帆延,皆四百里。南婆羅門,西波斯,北護時健。其王居鶴悉那城,地七千里,亦治阿娑你城。多鬱金、瞿草。瀵泉灌田。國中有突厥、罽賓、吐火羅種人雜居,罽賓取其子弟持兵以御大食。景雲初,遣使朝貢,後遂臣罽賓。開元八年,天子冊葛達羅支頡利發誓屈爾為王。至天寶中數朝獻。



 帆延者,或曰望衍,曰梵衍那。居斯卑莫運山之旁,西北與護時健接,東南距罽賓,西南訶達羅支,與吐火羅連境。地寒,人穴處。王治羅爛城,有大城四五。水北流入烏滸河。貞觀初,遣使者入朝。顯慶三年,以羅爛城為寫鳳都督府,縛時城為悉萬州,授王垞寫鳳州都督,管內五州諸軍事,自是朝貢不絕。



 石汗那,或曰斫汗那。自縛底野南入雪山,行四百里得帆延,東臨烏滸河。多赤豹。開元、天寶中,一再朝獻。



 識匿,或曰尸棄尼,曰瑟匿。東南直京師九千里,東五百里距蔥嶺守捉所,南三百里屬護蜜,西北五百里抵俱蜜。初治苦汗城,後散居山谷。有大谷五,酋長自為治,謂之五識匿。地二千里,無五穀。人喜攻剽,劫商賈。播蜜川四穀稍不用王號令。俗窟室。貞觀二十年,與似沒、役槃二國使者偕來朝。開元十二年,授王布遮波資金吾衛大將軍。天寶六載,王跌失伽延從討勃律戰死,擢其子都督、左武衛將軍,給祿居籓。



 似沒者,北接石。土俗與康同。



 役槃,亦與康鄰。出良馬。



 俱蜜者,治山中。在吐火羅東北,南臨黑河。其王突厥延陀種。貞觀十六年,遣使者入朝。開元中,獻胡旋舞女,其王那羅延頗言為大食暴賦,天子但尉遣而已。天寶時,王伊悉爛俟斤又獻馬。



 護蜜者,或曰達摩悉鐵帝,曰鑊侃,元魏所謂缽和者,亦吐火羅故地。東南直京師九千里而贏,橫千六百里,縱狹才四五里。王居塞迦審城,北臨烏滸河。地寒洹,堆阜曲折,沙石流漫。有豆、麥,宜木果,出善馬。人碧瞳。顯慶時以地為鳥飛州,王沙缽羅頡利發為刺史。地當四鎮入吐火羅道,故役屬吐蕃。開元八年,冊其王羅旅伊陀骨咄祿多毘勒莫賀達摩薩爾為王。十六年,與米首領米忽汗同獻方物。明年,大酋烏鶻達干復朝。王死,冊其從弟護真檀嗣王。二十九年,身入朝,宴內殿,拜左金吾衛將軍,賜紫袍、金帶。天寶初,王子頡吉匐請絕吐蕃,賜鐵券。八載,真檀來朝,請宿衛,詔可。授右武衛將軍,久乃遣。又遣首領朝貢。乾元元年,王紇設伊俱鼻施來朝,賜氏李。



 個失蜜,或曰迦濕彌邏。北距勃律五百里,環地四千里,山回繚之,它國無能攻伐。王治撥邏勿邏布邏城,西瀕彌那悉多大河。地宜稼。多雪不風。出火珠、鬱金、龍種馬。俗毛褐。世傳地本龍池,龍徙水竭,故往居之。



 開元初,遣使者朝。八年,詔冊其王真陀羅秘利為王;間獻胡藥。天木死,弟木多筆立,遣使者物理多來朝,且言:「有國以來,並臣天可汗,受調發。國有象、馬、步三種兵,臣身與中天竺王厄吐蕃五大道,禁出入,戰輒勝。有如天可汗兵至勃律者,雖眾二十萬,能輸糧以助。又國有摩訶波多磨龍池,願為天可汗營祠。」因丐王冊,鴻臚譯以聞。詔內物理多宴中殿,賜賚優備,冊木多筆為王,自是職貢有常。



 其役屬五種,亦名國。所謂呾叉始羅者,地二千里,有都城。東南餘七百里得僧訶補羅,地三千餘里,亦治都城。東南山行五百里得烏剌尸,地二千里,有都城。宜稼穡。東南限山千里即個失蜜。西南行險七百里得半笯蹉,地二千里。又得曷邏闍補羅者,其大四千里,有都城,多山阜,人驍勇。五種皆無君長雲。



 骨咄,或曰珂咄羅。廣長皆千里。王治思助建城。多良馬、赤豹。有四大鹽山,山出烏鹽。



 開元十七年,王俟斤遣子骨都施來朝。二十一年,王頡利發獻女樂,又遣大首領多博勒達干朝貢。天寶十一載,冊其王羅全節為葉護。



 蘇毘,本西羌族,為吐蕃所並,號孫波,在諸部最大。東與多彌接,西距鶻莽硤,戶三萬。天寶中,王沒陵贊欲舉國內附,為吐蕃所殺。子悉諾率首領奔隴右,節度使哥舒翰護送闕下,玄宗厚禮之。



 多彌,亦西羌族,役屬吐蕃,號難磨。濱犁牛河,土多黃金。貞觀六年,遣使者朝貢,賜遣之。



 伊吾城者,漢宜禾都尉所治。商胡雜居,勝兵千,附鐵勒。人驍悍,土良沃。隋末內屬,置伊吾郡。天下亂,復臣突厥。貞觀四年,城酋來朝。頡利滅,舉七城降,列其地為西伊州。



 師子,居西南海中,延袤二千餘里。有棱伽山,多奇寶,以寶置洲上,商舶償直輒取去。後鄰國人稍往居之。能馴養師子,因以名國。



 總章三年,遣使者來朝。天寶初,王尸羅迷迦再遣使獻大珠、鈿金、寶瓔、象齒、白氎。



 波斯,居達遏水西,距京師萬五千里而贏。東與吐火羅、康接,北鄰突厥可薩部,西南皆瀕海,西北贏四千里,拂菻也。人數十萬,其先波斯匿王,大月氏別裔,王因以姓,又為國號。治二城,有大城十餘。俗尊右下左,祠天地日月水火。祠夕,以麝揉蘇,澤耏顏鼻耳。西域諸胡受其法,以祠祅。拜必交股。俗徒跣,丈夫祝發,衣不剖襟,青白為巾帔,緣以錦。婦辮發著後。戰乘象,一象士百人,負則盡殺。斷罪不為文書,決於廷。叛者鐵灼其舌,瘡白為直,黑為曲。刑有髡、鉗、刖、劓,小罪耏,或系木於頸,以時月而置。劫盜囚終老,偷者輸銀錢。凡死,棄於山,服閱月除。氣常歊熱,地夷漫,知耕種畜牧。有鷲鳥,能啖羊。多善犬、婁、大驢。產珊瑚,高不三尺。



 隋末,西突厥葉護可汗討殘其國,殺王庫薩和,其子施利立,葉護使部帥監統。施利死,遂不肯臣。立庫薩和女為王,突厥又殺之。施利之子單羯方奔拂菻,國人迎立之,是為伊怛支。死,兄子伊嗣俟立。



 貞觀十二年,遣使者沒似半朝貢。又獻活褥蛇,狀類鼠,色正青,長九寸,能捕穴鼠。伊嗣俟不君,為大酋所逐,奔吐火羅,半道,大食擊殺之。子卑路斯入吐火羅以免。遣使者告難,高宗以遠不可師,謝遣。會大食解而去,吐火羅以兵納之。



 龍朔初,又訴為大食所侵,是時天子方遣使者到西域分置州縣,以疾陵城為波斯都督府,即拜卑路斯為都督。俄為大食所滅。雖不能國,咸亨中猶入朝,授右武衛將軍,死。始,其子泥涅師為質,調露元年,詔裴行儉將兵護還,將復王其國。以道遠,至安西碎葉,行儉還。泥涅師因客吐火羅二十年,部落益離散。景龍初,復來朝,授左威衛將軍。病死,西部獨存。開元、天寶間,遣使者十輩獻碼床、火毛繡舞筵。乾元初,從大食襲廣州,焚倉庫廬舍,浮海走。大歷時復來獻。



 又有陀拔斯單者,或曰陀拔薩憚。其國三面阻山,北瀕小海。居婆里城,世為波斯東大將。波斯滅,不肯臣大食。天寶五載,王忽魯汗遣使入朝,封為歸信王。後八年,遣子自會羅來朝,拜右武衛員外中郎將,賜紫袍、金魚,留宿衛。為黑衣大食所滅。



 貞觀後,遠小國君遣使者來朝獻,有司未嘗參考本末者,今附之左方。曰火辭彌,與波斯接。貞觀十八年,與摩羅游使者偕朝。二十一年,有健達王獻佛土菜,莖五葉,赤華紫須。龍朔元年,多福王難婆修強宜說遣使者來朝。總章元年,有末陀提王,開元五年,有習阿薩般王安殺,並遣使者朝貢。七年,訶毘施王捺塞因吐火羅大酋羅摩獻師子、五色鸚鵡。



 天寶時來朝者,曰俱爛那,曰舍摩,曰威遠,曰蘇吉利發屋蘭,曰蘇利悉單,曰建城,曰新城,曰俱位,凡八國。



 俱位,或曰商彌。治阿賒師多城,在大雪山、勃律河北。地寒,有五穀、蒲陶、若榴,冬窟室。國人常助小勃律為中國候。



 新城之國,在石東北贏百里。有弩室羯城,亦曰新城,曰小石國城,後為葛邏祿所並。



 拂菻,古大秦也,居西海上,一曰海西國。去京師四萬里,在苫西,北直突厥可薩部,西瀕海,有遲散城,東南接波斯。地方萬里,城四百,勝兵百萬。十里一亭,三亭一置。臣役小國數十,以名通者曰澤散,曰驢分。澤散直東北,不得其道里。東度海二千里至驢分國。



 重石為都城,廣八十里,東門高二十丈,扣以黃金。王宮有三襲門,皆飾異寶。中門中有金巨稱一,作金人立,其端屬十二丸,率時改一丸落。以瑟瑟為殿柱,水精、琉璃為棁,香木梁,黃金為地,象牙闔。有貴臣十二共治國。王出,一人挈囊以從,有訟書投囊中,還省枉直。國有大災異,輒廢王更立賢者。王冠如鳥翼,綴珠。衣錦繡,前無襟。坐金■榻,側有鳥如鵝,綠毛,上食有毒輒鳴。無陶瓦,屑白石塈屋,堅潤如玉。盛暑引水上,流氣為風。男子翦發、衣繡,右袒而帔,乘輜軿白蓋小車,出入建旌旗,擊鼓。婦人錦巾。家訾億萬者為上官。



 俗喜酒,嗜乾餅。多幻人,能發火於顏,手為江湖,口幡眊舉,足墮珠玉。有善醫能開腦出蟲以愈目眚。土多金、銀、夜光璧、明月珠、大貝、車渠、碼、木難、孔翠、虎魄。織水羊毛為布,曰海西布。海中有珊瑚洲,海人乘大舶,墮鐵網水底。珊瑚初生磐石上,白如菌,一歲而黃,三歲赤,枝格交錯,高三四尺。鐵發其根,系網舶上,絞而出之,失時不敢即腐。西海有市,貿易不相見,置直物旁,名鬼市。有獸名勍,大如狗,獷惡而力。北邑有羊,生土中,臍屬地,割必死,俗介馬而走,擊鼓以驚之,羔臍絕,即逐水草,不能群。



 貞觀十七年,王波多力遣使獻赤玻瓈、綠金精,下詔答賚。大食稍強,遣大將軍摩拽伐之,拂菻約和,遂臣屬。乾封至大足,再朝獻。開元七年,因吐火羅大酋獻師子、羚羊。



 自拂菻西南度磧二千里,有國曰磨鄰,曰老勃薩。其人黑而性悍。地瘴癘,無草木五穀,飼馬以槁魚,人食鶻莽。鶻莽,波斯棗也。不恥烝報,於夷狄最甚,號曰「尋」。其君臣七日一休,不出納交易,飲以窮夜。



 大食,本波斯地。男子鼻高,黑而髯。女子白皙,出輒鄣面。日五拜天神。銀帶,佩銀刀,不飲酒舉樂。有禮堂容數百人,率七日,王高坐為下說曰:「死敵者生天上,殺敵受福。」故俗勇於斗。土饒礫不可耕,獵而食肉。刻石蜜為廬如輿狀,歲獻貴人。蒲陶大者如雞卵。有千里馬,傳為龍種。



 隋大業中,有波斯國人牧於俱紛摩地那山,有獸言曰:「山西三穴,有利兵,黑石而白文,得之者王。」走視,如言。石文言當反,乃詭眾裒亡命於恆曷水,劫商旅,保西鄙自王,移黑石寶之。國人往討之,皆大敗還,於是遂強。滅波斯,破拂菻,始有粟麥倉庾。南侵婆羅門,並諸國,勝兵至四十萬。康、石皆往臣之。其地廣萬里,東距突騎施。西南屬海。



 海中有撥拔力種,無所附屬。不生五穀,食肉,刺牛血和乳飲之。俗無衣服,以羊皮自蔽。婦人明皙而麗。多象牙及阿末香,波斯賈人欲往市,必數千人納氎鑱血誓,乃交易。兵多牙角,而有弓、矢、鎧、槊,士至二十萬,數為大食所破略。



 永徽二年,大食王豃密莫末始遣使者朝貢,自言王大食氏,有國三十四年,傳二世。開元初,復遣使獻馬、鈿帶,謁見不拜,有司將劾之。中書令張說謂殊俗慕義,不可置於罪。玄宗赦之。使者又來,辭曰:「國人止拜天,見王無拜也。」有司切責,乃拜。十四年,遣使蘇黎滿獻方物,拜果毅,賜緋袍、帶。



 或曰大食族中有孤列種,世酋長,號白衣大食。種有二姓,一曰盆尼末換,二曰奚深。有摩訶末者,勇而智,眾立為王。闢地三千里,克夏臘城。傳十四世,至末換,殺兄伊疾自王,下怨其忍。有呼羅珊木鹿人並波悉林將討之,徇眾曰:「助我者,皆黑衣。」俄而眾數萬,即殺末換,求奚深種孫阿蒲羅拔為王,更號黑衣大食。蒲羅死,弟阿蒲恭拂立。至德初,遣使者朝貢。代宗取其兵平兩京。阿蒲恭拂死,子迷地立。死,弟訶論立。貞元時,與吐蕃相攻,吐蕃歲西師,故鮮盜邊。十四年,遣使者含嵯、烏雞、沙北三人朝,皆拜中郎將,賚遣之。傳言其國西南二千里山谷間,有木生花如人首,與語輒笑,則落。



 東有末祿,小國也。治城郭,多木姓,以五月為歲首,以畫缸相獻。有尋支瓜,大者十人食乃盡。蔬有顆蔥、葛藍、軍達、茇薤。



 大食之西有苫者,亦自國。北距突厥可薩部,地數千里。有五節度,勝兵萬人。土多禾。有大川,東流入亞俱羅。商賈往來相望雲。



 自大食西十五日行,得都盤,西距羅利支十五日行;南即大食,二十五日行;北勃達,一月行。



 勃達之東距大食二月行;西抵岐蘭二十日行;南都盤,北大食,皆一月行。



 岐蘭之東南二十日行,得阿沒,或曰阿昧;東南距陀拔斯十五日行;南沙蘭,一月行;北距海二日行。居你訶溫多城,宜馬羊,俗柔寬,故大食常游牧於此。



 沙蘭東距羅利支,北恆滿,皆二十日行;西即大食,二十五日行。



 羅利支東距都盤,北陀拔斯,皆十五日行;西沙蘭,二十日行;南大食,二十五日行。



 怛滿,或曰怛沒,東陀拔斯,南大食,皆一月行;北岐蘭,二十日行;西即大食,一月行。居烏滸河北平川中。獸多師子。西北與史接,以鐵關為限。



 天寶六載,都盤等六國皆遣使者入朝,乃封都盤王謀思健摩訶延曰順化王,勃達王摩俱澀斯曰守義王,阿沒王俱那胡設曰恭信王,沙蘭王卑路斯威曰順禮王,羅利支王伊思俱習曰義寧王,怛滿王謝沒曰奉順王。



 贊曰:西方之戎,古未嘗通中國,至漢始載烏孫諸國。後以名字見者浸多。唐興,以次脩貢,蓋百餘,皆冒萬里而至,亦已勤矣!然中國有報贈、冊吊、程糧、傳驛之費,東至高麗,南至真臘,西至波斯、吐蕃、堅昆,北至突厥、契丹、靺鞨,謂之「八蕃」,其外謂之「絕域」,視地遠近而給費。開元盛時,稅西域商胡以供四鎮,出北道者納賦輪臺。地廣則費倍,此盛王之鑒也。



\end{pinyinscope}