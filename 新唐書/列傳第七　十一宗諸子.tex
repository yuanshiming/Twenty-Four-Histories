\article{列傳第七 十一宗諸子}

\begin{pinyinscope}

 玄宗三十子:劉華妃生琮、第六子琬、第十二子璲,趙麗妃生瑛,元獻皇后生肅宗皇帝,錢妃生琰,皇甫德儀生瑤觀點,分別以「太虛「、「陰陽」對天作了新的說明。,劉才人生琚,武惠妃生一、第十五子敏、第十八子瑁、第二十一子琦,高婕妤生璬,郭順儀生璘,柳婕妤生玢,鐘美人生環,盧美人生瑝,閻才人生玼,王美人生珪,陳才人生珙,鄭才人生瑱,武賢儀生璿、第三十子璥;餘七子夭,母氏失傳。



 奉天皇帝琮,景雲元年,王許昌郡,與真定王同封。先天元年,進王郯,與郢王同封。開元四年,領安西大都護、安撫河東關內隴右諸蕃大使。十三年,徙王慶,與忠、棣、榮、光、儀、潁、永、壽、延、盛、濟十一王同封。十五年,與十王並領節度,不出閤。琮以涼州都督兼河西諸軍節度大使。天寶元年,改節河東。十載薨,贈太子,謚靖德。肅宗立,詔曰:「靖德太子琮,親則朕兄,睿悊聰明,朕昔踐儲極,顧誠非次,君父有命,不敢違,永言懇讓,不克如素。宜進謚奉天皇帝,妃竇為恭應皇后。」詔尚書右僕射裴冕持節改葬,群臣素服臨送達禮門,帝御門哭以過喪,墓號齊陵。無子,以太子瑛子俅嗣王。琮始名嗣直,太子嗣謙,棣王嗣真,鄂王嗣初,靖恭太子嗣玄。開元十三年,更名曰潭,曰鴻,曰洽,曰涓,曰滉。後十年改今名。



 太子瑛,始王真定,進王郢。開元三年,立為皇太子。七年,詔太子、諸王入國學行齒胄禮,太常擇日謁孔子,太子獻。詔右散騎常侍褚無量執經,群臣、學官、諸生以差賜帛。明年,瑛加元服,見太廟。十六年,詔九品官息女可配太子者,有司採閱待進止,以太常少卿薛縚女為妃。帝種麥苑中,瑛、諸王侍登,帝曰:「是將薦宗廟,故親之,亦欲若等知稼穡之難。」因分賜侍臣,曰:「《春秋》書『無麥禾』,古所甚重。比詔使者閱田畝,所對不以實,故朕自蒔以觀其成」云。初,瑛母以倡進,善歌舞,帝在潞得幸。及即位,擢妃父元禮、兄常奴皆至大官。鄂、光二王母亦帝為臨淄王時以色選。及武惠妃寵幸傾後宮,生壽王,愛與諸子絕等。而太子、二王以母失職,頗怏怏。惠妃女咸宜公主婿楊洄揣妃旨,伺太子短,嘩為醜語,惠妃訴於帝,且泣,帝大怒,召宰相議廢之。中書令張九齡諫曰:「太子、諸王日受聖訓,天下共慶。陛下享國久,子孫蕃衍,奈何一日棄三子。昔晉獻公惑嬖姬之讒,申生憂死,國乃大亂;漢武帝信江充巫蠱,禍及太子,京師蹀血;晉惠帝有賢子,賈後譖之,乃至喪亡;隋文帝聽後言,廢太子勇,遂失天下。今太子無過,二王賢。父子之道,天性也,雖有失,尚當掩之。惟陛下裁赦。」帝默然,太子得不廢。俄而九齡罷,李林甫專國,數稱壽王美以揠妃意,妃果德之。二十五年,洄復構瑛、瑤、琚與妃之兄薛銹異謀。惠妃使人詭召太子、二王,曰:「宮中有賊,請介以入。」太子從之。妃白帝曰:「太子、二王謀反,甲而來。」帝使中人視之,如言,遽召宰相林甫議,答曰:「陛下家事,非臣所宜豫。」帝意決,乃詔:「太子瑛、鄂王瑤、光王琚同惡均罪,並廢為庶人;銹賜死。」瑛、瑤、琚尋遇害,天下冤之,號「三庶人」。歲中,惠妃數見庶人為祟,因大病。夜召巫祈之,請改葬,且射行刑者瘞之,訖不解。妃死,崇亡。寶應元年,詔贈瑛皇太子,瑤等復王。瑛子五人:儼、伸、倩、俅、備。瑛之廢,帝使慶王畜儼等為子。儼封新平郡王,伸平原郡王,俅嗣慶王,備太僕卿,倩失傳。



 隸王琰,開元二年始王鄫,與鄂、鄄二王同封。後徙王棣,領太原牧、太原以北諸軍節度大使。天寶初,為武威郡都督,經略節度河西、隴右。會妃韋以過置別室,而二孺人爭寵不平,求巫者密置符琰履中以求媚。仇人告琰厭魅上,帝伺其朝,使人取履視之,信。帝怒責琰,琰頓首謝曰:「臣罪宜死,然臣與婦不相見二年,有二孺人爭長,臣恐此三人為之。」及推,果驗。然帝猶疑琰,怒未置,太子以下皆為請,乃囚於鷹狗坊,以憂薨,妃,縚之女,無子,還本宗。琰凡五十五子,得王者四人,僎王汝南郡,僑宜都,俊濟南,侒順化;僚太僕卿,俠國子祭酒,仁殿中監,僾秘書監。寶應元年,詔復琰王爵。



 鄂王瑤,既封,遙領幽州都督、河北節度大使。開元二十三年,與榮、光、儀、潁、永、壽、延、盛、濟、信、義十一王並授開府儀同三司,實封二千戶。詔詣東宮、尚書省,上日百官集送,有司供張設樂。是日,悉拜王府官屬,然未有府也,而選任冒濫,時不以為榮。



 靖恭太子琬,始王鄄,徙王榮。為京兆牧,領隴右節度大使。又詔親巡按隴右,選關內、河東飛騎五萬防盛秋。累兼單于、安北大都督。安祿山反,詔琬為征討元帥,募河、隴兵屯陜,以高仙芝副之,會薨。琬風格秀整,有素望,中外倚之。及薨,莫不為國悵恨。詔加贈謚。琬男女五十八人,得王者三人,俯王濟陰郡,偕北平,倩陳留;傆衛尉卿,僓秘書監,佩鴻臚卿。



 光王琚,開元十三年始王,與儀、潁、永、壽、延、盛、濟七王同封。俄領廣州都督。勇力善騎射,帝愛之。與鄂王同居,友睦甚,皆篤學。既廢,無嗣。初,琚名涺,儀王濰,潁王沄,永王澤,壽王清,延王洄,盛王沐,濟王溢,信王沔,義王漼,陳王沚,豐王澄,恆王潓,涼王漎,汴王滔,至二十三年,詔悉改今名。



 夏悼王一,生韶秀,以母寵,故鐘愛,命之曰一。未免懷薨,追爵及謚。時帝在東都,故葬龍門東岑,欲宮中望見雲。



 儀王璲,即封,授河南牧。薨,贈太傅。子侁王鐘陵郡,僆廣陵。



 潁王璬,喜讀書,好文辭。開元十五年,遙領安東都護。安祿山反,詔領劍南節度大使,以楊國忠為之副。帝西出,令御史大夫魏方進為置頓使,移書劍南屬郡,托璬之籓,大設儲偫。璬先即鎮,更以蜀郡長史崔圓為副。璬濟江,舟中以彩席藉步,命徹之,曰:「此可寢,奈何踐之?」璬之出遽,不及受節,司馬史賁請建大槊,蒙油囊,先驅以威道路。璬笑曰:「既為真王矣,安用假節為?」將至成都,崔圓迎拜馬前,璬不為禮,圓銜之。璬視事再逾月,人便其寬,圓奏罷居內宅。乃詔宣慰肅宗於彭原,從還京師。建中四年薨,年六十六。子伸為滎陽王,僝高邑王,伣楚國公,僔夔國公。



 懷思王敏,貌豐秀若圖畫,帝愛之。甫晬薨,追爵及謚,祔葬敬陵。



 永王璘,少失母,肅宗自養視之。長聰敏好學。貌陋甚,不能正視。既封,領荊州大都督。安祿山反,帝至扶風,詔璘即日赴鎮。俄又領山南、江西、嶺南、黔中四道節度使,以少府監竇昭為副。璘至江陵,募士得數萬,補署郎官、御史。



 時江淮租賦巨億萬,在所山委。璘生宮中,於事不通曉,見富且強,遂有窺江左意,以薛鏐、李臺卿、韋子春、劉巨鱗、蔡F駉為謀主。肅宗聞之,詔璘還覲上皇於蜀,璘不從。其子襄城王人易,剛鷙乏謀,亦樂亂,勸璘取金陵。即引舟師東下,甲士五千趨廣陵,以渾惟明、季廣琛、高仙琦為將,然未敢顯言取江左也。



 會吳郡採訪使李希言平牒璘,璘因發怒曰:「寡人上皇子,皇帝弟,地尊禮絕。今希言乃平牒抗威,落筆署字,何邪?」乃使惟明襲希言,而令廣琛趨廣陵,攻採訪使李成式。璘至當塗,希言已屯丹楊,遣將元景曜等拒戰,不勝,降於璘,江淮震動。



 明年,肅宗遣宦者啖廷瑤等與成式謀招喻之。時河北招討判官李銑在廣陵,有兵千餘,廷瑤邀銑屯揚子,成式又遣裴戎以廣陵卒三千戍伊婁埭,張旗幟,大閱士。璘與人易登陴望之,有懼色。廣琛知事不集,謂諸將曰:「與公等從王,豈欲反邪?上皇播遷,道路不通,而諸子無賢於王者。如總江淮銳兵,長驅雍、洛,大功可成。今乃不然,使吾等名絓叛逆,如後世何?」眾許諾,遂割臂盟。於是惟明奔江寧,馮季康奔白沙,廣琛以兵六千奔廣陵。璘使騎追躡之,廣琛曰:「我德王,故不忍決戰,逃命歸國耳。若逼我,且決死。」追者止,乃免。是夜,銑陣江北,夜然束葦,人執二炬,景亂水中,覘者以倍告,璘軍亦舉火應之。璘疑王師已濟,攜兒女及麾下遁去。遲明覺其紿,復入城,具舟楫,使人易驅眾趨晉陵。諜者告曰:「王走矣!」成式以兵進,先鋒至新豐,璘使人易、仙琦逆擊之。銑合勢,張左右翼,射人易中肩,軍遂敗。仙琦與璘奔鄱陽,司馬閉城拒,璘怒,焚城門入之,收庫兵,掠餘干,將南走嶺外。皇甫侁兵追及之,戰大庾嶺,璘中矢被執,侁殺之。人易為亂兵所害,仙琦逃去。璘未敗時,上皇下誥:「降為庶人,徙置房陵。」及死,侁送妻子至蜀,上皇傷悼久之。肅宗以少所自鞠,不宣其罪。謂左右曰:「皇甫侁執吾弟,不送之蜀而擅殺之,何邪?」由是不復用。薛寔等皆伏誅。子儹為餘姚王,偵莒國公,儇郕國公,伶、儀並國子祭酒。



 壽王瑁,母惠妃頻姙不育,及瑁生,寧王請養邸中,元妃自乳之,名為己子,故封比諸王最後。開元十五年,遙領益州大都督。初,帝以永王等尚幼,詔不入謁。瑁七歲,請與諸兄眾謝,拜舞有儀矩,帝異之。寧王薨,請制服以報私恩,詔可。大歷十年薨,贈太傅。子王者三人,僾王德陽郡,伓濟陽郡,偡廣陽郡,伉薛國公,傑國子祭酒。



 延王玢,母尚書右丞範之孫,帝重其名家,而玢亦仁愛有學。既封,遙領安西大都護。帝入蜀,玢凡三十六子,不忍棄,故徐進,數日,見行在所,帝怒,漢中王瑀申救得解,聽歸靈武。興元元年薨。子倬王彭城郡,侹平陽,倞魯國公,偃荊國公,優太僕卿。



 盛宣王琦,既封,領揚州大都督。帝之西,詔為廣陵大都督、淮南江東河南節度大使,以劉匯為副,李成式為副大使,琦不行。廣德二年薨,贈太傅。子償封真定王,佩武都王,俗徐國公,系許國公。



 濟王環,逸其薨年。子傃王永嘉郡,俛平樂郡。



 信王瑝,開元二十一年始王,與義、陳、豐、恆、涼、汴六王同封。子佟封新安王,倜晉陵王。



 義王玼,與信王並失薨年。子儀為舞陽王,僇高密王。



 陳王珪,二十一子,得王者三人,倫王安南郡,佗臨淮,佼安陽。



 豐王珙,已封,為左衛大將軍。帝至普安,授珙武威都督、河西隴右安西北庭節度大使,以隴西太守鄧景山為副,珙不行。廣德初,吐蕃入京師,代宗幸陜,將軍王懷忠閉苑門,以五百騎劫諸王西迎虜,遇郭子儀,懷忠曰:「上東遷,宗社無主,今僕奉諸王西奔,以系天下望。公為元帥,惟所廢置。」子儀未對。珙輒曰:「公何如?」司馬王延昌質責珙曰:「上雖蒙塵,未有失德,王為籓翰,安得狂悖之言?」子儀亦讓之,即護送行在所,帝赦不責。珙語不遜,群臣恐其亂,請除之,乃賜死。子佻為齊安王。



 恆王瑱,好方士,常服道士服。從帝幸蜀,還,代宗時薨。



 涼王璿,母高平王重規之女,宮中號小武妃者。璿薨代宗時。子仂為瀘陽郡王。



 汴哀王璥,於諸子為最少,初封才數歲,容貌秀澈,有成人風,帝愛之。開元二十三年,授右千牛衛大將軍。明年,薨。



 唐制:親王封戶八百,增至千;公主三百,長公主止六百。高宗時,沛英豫三王、太平公主武后所生,戶始逾制,垂拱中,太平至千二百戶。聖歷初,相王、太平皆三千,壽春等五王各三百。神龍初,相王、太平至五千,衛王三千,溫王二千,壽春等王皆七百,嗣雍、衡陽、臨淄、巴陵、中山王五百,安樂公主二千,長寧千五百,宣城、宜城、宣安各千,相王女為縣主,各三百。相王增至七千,安樂三千,長寧二千五百,宜城以下二千。相王、太平、長寧、安樂以七丁為限,雖水旱不蠲,以國租、庸滿之。中宗遺詔,雍、壽春王進為親王,戶千。開元後,天子敦睦兄弟,故寧王戶至五千五百,岐、薛五千,申王以外家微,戶四千,邠王千八百,帝妹戶千,中宗諸女如之,通以三丁為限。及皇子封王,戶二千,公主五百。咸宜公主以母惠妃故,封至千,自是,諸公主例千戶止。初,文德皇后崩,晉王最幼,太宗憐之,不使出閤。豫王亦以武后少子不出閤,嗣聖初,即帝位,及降封相王,乃出閤。中宗時,譙王失愛,遷外籓,溫王年十七,猶居宮中,遂立為帝。開元後,皇子幼,多居禁內,既長,詔附苑城為大宮,分院而處,號「十王宅」,所謂慶、忠、棣、鄂、榮、光、儀、潁、永、延、盛、濟等王,以十,舉全數也。中人押之,就夾城參天子起居。家令日進膳。引詞學士入授書,謂之侍讀。壽、信、義、陳、豐、恆、涼七王就封,亦居十宅。鄂、光廢死,忠王立為太子,慶、棣繼薨,唯榮、儀十四王居院,而府幕列於外坊,歲時通名起居。既又諸孫多,則於宅外更置「百孫院」。天子歲幸華清宮,又置十王、百孫院於宮側。宮人每院四百餘,百孫院亦三四十人。禁中置維城庫,以給諸王月奉。諸孫納妃、嫁女,就十王宅。太子不居東宮,處乘輿所幸別院。太子、親王、公主婚嫁並供帳於崇仁之禮院。此承平制云。



 肅宗十四子:章敬皇后生代宗皇帝,宮人孫生系,張生倓,王生佖,陳婕妤生僅,韋妃生■,張美人生侹,後宮生榮,裴昭儀生僙,段婕妤生倕,崔妃生偲,張皇后生佋,侗,後宮生僖。



 越王系,生開元時。玄宗末年,悉王太子子,故系王南陽郡。帝即位,至德二載十二月,進王趙,與彭、兗、涇、鄆、襄、杞、召、興、定九王同封。乾元二年,九節度兵潰河北,朝廷震駭,乃以李光弼代郭子儀總兵關東,而光弼請賢王為帥,於是詔系充天下兵馬元帥,而光弼以司空兼侍中、薊國公副,知節度行營事,系留京師。史思明陷洛陽,系請行,不聽。明年,徙王越。帝寢疾,皇太子監國,張皇后與中人李輔國有隙,因召太子入,謂曰:「輔國典禁軍,用事久,四方詔令皆出其口,矯天子制,逼徙聖皇,天下側目。今上疾彌留,輔國常怏怏,忌吾與汝。又程元振陰結黃門,圖不軌。若釋不誅,禍不移頃。」太子泣曰:「此二人者,陛下勛舊,而上體不豫,重以此事,得無震驚乎?願出外徐計之。」後曰:「是難與共事者!」乃召系曰:「汝能行此乎?」系許諾。即遣內謁者監段恆俊選材勇宦者二百人,授甲長生殿,以帝命召太子。元振以告輔國,乃相與勒兵凌霄門,迎太子,以難告。太子曰:「上疾亟,吾可懼死不赴乎?」元振曰:「赴則及禍。」乃以兵護太子止飛龍廄,勒兵夜入三殿,收系及恆俊等百餘人系之,幽後別殿。後及系皆為輔國所害。系三子:建王武威郡,逌興道,逾齊國公。



 承天皇帝倓,始王建寧。英毅有才略。善騎射。祿山亂,典親兵,扈車駕。度渭,百姓遮道留太子,太子使喻曰:「至尊播遷,吾可以違左右乎?」倓進說曰:「逆胡亂常,四海崩分,不因人情圖興復,雖欲從上入蜀,而散關以東非國家有。夫大孝莫若安社稷,殿下當募豪桀,趣河西,收牧馬。今防邊屯士不下十萬,而光弼、子儀全軍在河朔,與謀興復,策之上者。」廣平王亦贊之,於是議定。太子北過渭,兵仗鹽惡,士氣崩沮,日數十戰。倓以驍騎數百從,每接戰,常身先,血殷袂,不告也。太子或過時未食,倓輒涕泗不自勝,三軍皆屬目。至靈武,太子即帝位,議以倓為天下兵馬元帥,左右固請廣平王。帝曰:「廣平既塚嗣,安用元帥?」答曰:「太子從曰撫軍,守曰監國。元帥,撫軍也,莫宜於廣平王。」帝從之,更詔倓典親軍,以李輔國為府司馬。時張良娣有寵,與輔國交構,欲以動皇嗣者。倓忠謇,數為帝言之,由是為良娣、輔國所譖,妄曰:「倓恨不總兵,鬱鬱有異志。」帝惑偏語,賜倓死,俄悔悟。明年,廣平王收二京,使李泌獻捷。泌與帝雅素,從容語倓事,帝改容曰:「倓于艱難時實自有力,為細人間鬩,欲害其兄,我計社稷,割愛而為之所。」泌曰:「爾時臣在河西,知其詳。廣平於兄弟篤睦,至今言建寧,則嗚咽不自己。陛下此言得之讒口耳。」帝泣下曰:「事已爾,末耐何!」泌曰:「陛下嘗聞《黃臺瓜》乎?高宗有八子,天后所生者四人,自為行,而睿宗最幼。長曰弘,為太子,仁明孝友,後方圖臨朝,鴆鐐之,而立次子賢。賢日憂惕,每侍上,不敢有言,乃作樂章,使工歌之,欲以感悟上及後。其言曰:『種瓜黃臺下,瓜熟子離離。一摘使瓜好,再摘令瓜稀。三摘尚云可,四摘抱蔓歸。』而賢終為後所斥,死黔中。陛下今一摘矣,慎無再!」帝愕然曰:「公安得是言?」是時,廣平有大功,亦為後所構,故泌因對及之,廣平遂安。及即位,追贈倓齊王。大歷三年,有詔以倓當艱難時,首定大謀,排眾議,於中興有功,乃進謚承天皇帝,以興信公主季女張為恭順皇后,冥配焉,葬順陵,祔主奉天皇帝廟,同殿異室雲。初,李泌請加贈倓,代宗曰:「倓性忠孝,而困於讒,追帝之,若何?」答曰:「開元中,上皇兄弟皆贈太子。」帝曰:「是特祖宗友愛耳,豈若倓有功乎?」於是追帝號。遣使迎喪彭原,既至城門,喪輴不動。帝謂泌曰:「豈有恨邪?卿往祭之,以白朕意。且卿及知倓艱難定策者。」泌為挽詞二解,追述倓志,命挽士唱,泌因進輴,乃行,觀者皆為垂泣。



 衛王泌,始王西平。蚤薨,寶應元年五月,與鄆王同追封。



 彭王人堇,始王新城,進封彭。史思明陷河、洛,人心震騷,群臣請以諸王臨統方鎮兵,遙相維壓。於是詔人堇充河西節度,兗王北庭,涇王隴右,杞王陜西,興王鳳翔,並為大使。是歲人堇薨。子鎮為常山郡王。



 兗王人閑,始王潁川,進王兗。寶應元年薨。



 涇王侹,始王東陽,進王涇。興元元年薨。



 鄆王榮,始王靈昌。蚤薨,追封。



 襄王僙,至德二載始王,與杞、召、興、定四王同封。貞元七年薨。子宣為伊吾郡王,寀樂安王。宣裔孫煴。



 煴,性謹柔,材無過人者。光啟二年,田令孜逼僖宗幸興元,邠寧節度使硃玫以五千騎追乘輿不及。煴以疾不能從,玫劫之,駐鳳翔,得臺省官百餘,乃脅宰相蕭遘等率群臣盟石鼻驛,奉煴為嗣襄王,監軍國事,因還京師,即封拜官屬。初,遘執不可,於是罷遘,而玫自為侍中,號令己出。以裴澈為門下侍郎,鄭昌圖中書侍郎,皆平章事。遣柳陟等十餘人分諭天下嗣襄王所以監國意,皆得進官。玫又脅太子太師裴璩等奉箋勸進,煴五讓乃即位,改元建貞,尊僖宗為太上元皇聖帝。河中節度使王重榮率諸籓貢奉,歸者十八九,而蔡州秦宗權自僭號,惟太原李克用不從。時帝遣使喻重榮、克用,故二人聽命。樞密使楊復恭等傳檄三輔,募能斬玫者,以邠寧節度界之。其偽將王行瑜自鳳州入京師殺玫,而煴與澈、昌圖並官屬奔東渭橋。重榮紿使迎之,煴與官屬別,且泣曰:「朕見重榮,當令備所服迓公等。」至蒲,執殺之,因械澈等於獄,誅殺偽官,函煴首至行在所。煴即偽位凡九月敗。始,煴首至,群臣白帝御興元南門受之,百官稱賀。太常博士殷盈孫奏言:「禮,公族有罪,有司曰:『某之罪在大闢。』君曰:『赦之。』如是者三,走出,致刑焉,君為素服不舉者三日。今煴皇族,以不能固節,迫脅至此,宜廢為庶人,絕屬籍,葬以庶人禮。大捷之慶,須硃玫首至乃賀。」詔可。



 杞王倕,貞元十四年薨。



 召王偲,元和元年薨。



 恭懿太子佋,始封興王。上元元年薨。佋生,後方專愛,帝最憐之。後數撼儲嫡,欲以佋嗣,會薨,計塞。是夕,帝及後夢佋辭決流涕去,帝鯁悵,故冊贈皇太子。



 定王侗,寶應初薨。



 代宗二十子:睿真皇后生德宗皇帝,崔妃生邈,貞懿皇后生迥;十七王,史亡其母之氏、位。



 昭靖太子邈,好學,以賢聞。上元二年始王益昌。帝即位,寶應元年進王鄭,與韓王同封。淄青牙將李懷玉逐其帥侯希逸,詔邈為平盧淄青節度大使,以懷玉知留後。大歷初,代皇太子為天下兵馬元帥。八年薨,遂罷元帥府。



 均王遐,早薨。貞元八年追封。



 睦王述。大歷十年,田承嗣不臣,而昭靖夭,無強王,帝乃悉王諸子,領諸鎮軍,威天下。於是以述為睦王,領嶺南節度,逾郴王、渭北鄜坊節度,過韓王、汴宋節度,造忻王、昭義節度,皆為大使;連為恩王,遘鄜王,暹韶王,遇端王,遹循王,通恭王,逵原王,逸雅王,並開府儀同三司,然不出閤。



 德宗建中初,周天下訪太后所在,述於諸王最長,故拜奉迎太后使,以工部尚書喬琳副之。貞元七年薨。



 丹王逾,始王郴,建中四年,與簡王同徙封。元和十五年薨。



 恩王連,元和十二年薨。



 韓王迥,始王延慶郡,以母寵,故與鄭王先徙封。貞元十二年薨。



 簡王遘,始王鄜,徙封簡。元和四年薨。



 益王乃,大歷十四年始王。亡薨年。



 隋王迅,興元元年薨。



 荊王選,蚤薨,建中二年追王。



 蜀王溯,本名遂,大歷十四年始王,建中二年改今名。



 忻王造,元和六年薨。



 韶王暹,貞元十二年薨。



 嘉王運,貞元十七年薨。



 端王遇,貞元七年薨。



 循王遹,亡薨年。



 恭王通,亡薨年。



 原王逵,大和六年薨。



 雅王逸,貞元十五年薨。



 德宗十一子:昭德皇后生順宗皇帝,帝取昭靖太子子誼為第二子,又取順宗子謜為第六子;餘八王,史亡其母之氏、位。



 舒王誼,初名謨。帝愛其幼,取為子。大歷十四年始王舒,與通、虔、肅、資四王同封。拜開府儀同三司,詔有司給奉稍,俄以軍興罷。謨於諸王最長,帝欲試以事,故拜涇原節度大使。時尚父郭子儀病篤,帝臨軒遣謨持詔往視。謨冠遠游冠,御絳袍,乘象輅四馬,飛龍士三百,國府官皆褲褶以從。子儀手叩頭謝恩。謨宣詔已,乃易服勞問還。



 於是,李希烈反,招討使李勉戰不勝,奔宋州,朝廷大震。乃拜謨揚州大都督、荊襄江西沔鄂節度使、諸軍行營兵馬都元帥。改名誼。軍中以哥舒翰由元帥敗,而王所封同之,帝乃使徙王普。以兵部侍郎蕭復為統軍長史,湖南觀察使孔巢父為行軍左司馬,山南東道節度行軍司馬樊澤為右,刑部員外郎劉從一、侍御史韋儹為判官,兵部員外郎高參掌書記,右金吾大將軍渾瑊為中軍虞候,江西節度使嗣曹王皋為前軍兵馬使,鄂岳團練使李兼副之,山南東道節度使賈耽為中軍兵馬使,荊南節度使張伯儀為後軍兵馬使,左神武軍使王價、左衛將軍高承謙、檢校太子詹事郭曙、檢校右庶子常願為押衙。未及行,涇原兵反,誼從帝出奉天。硃泚攻城,誼晝夜傳勞諸軍不解帶。帝還京師,復故封揚州大都督如故。永貞元年薨。



 通王諶,始王,拜開府儀同三司。貞元九年,領宣武節度大使,以李萬榮為留後,二年徙河東,以李說為留後,皆不出閤。



 虔王諒,以王拜開府儀同三司。貞元二年,領蔡州節度大使,以吳少誠為留後;十年,徙節朔方靈鹽,以李欒為留後;明年,領橫海,又徙徐州,以程懷信、張愔為留後。不出閤。



 肅王詳,資秀異,帝愛之。建中二年薨,甫四歲。帝欲用浮屠說,塔而不墳,禮儀判官李岧諫非禮,乃止。詔贈揚州大都督。



 文敬太子謜,見愛於帝,命為子。貞元初,先諸王王邕。歷義武、昭義二軍節度大使,以張茂昭、王虔休為留後,不出閤。十五年薨,年十八,追贈及謚。葬日,君臣以位而哭通化門外。陵及廟置令、丞云。



 資王謙,亡薨年。



 代王諲,始王縉雲郡。蚤薨,建中二年追王。



 昭王誡,貞元二十一年始王。亡薨年。



 欽王諤,順宗即位,與珍王同封。亡薨年。



 珍王諴,大和六年薨。



 順宗二十七子:莊憲皇后生憲宗皇帝及綰,張昭訓生經,趙昭儀生結,王昭儀生總、約、緄;餘二十王,史亡母之氏、位,四王蚤薨,亡官謚。



 郯王經,本名渙。貞元四年,始王建康郡,與廣陵、洋川、臨淮、弘農、漢東、晉陵、高平、雲安、宣城、德陽、河東、洛交十二王同封。二十一年,又與均、漵、莒、密、郇、邵、宋、集、冀、和、衡、欽、會、珍、福、撫、岳、袁、桂、翼二十王皆進王。王二十九年,太和八年薨。



 均王緯,初名沔。王洋川,後進王。王三十三年,開成二年薨。



 漵王縱,初名洵。王臨淮,後進王。王三十二年,開成元年薨。



 莒王紓,初名浼。為秘書監。王弘農,後進王。王二十九年,大和八年薨。



 密王綢,初名言永。王漢東,後進王。王三年,元和二年薨。



 郇王總,初名湜。授少府監。王晉陵,後進王。王四年,元和三年薨。



 邵王約,初名漵。為國子祭酒。王高平,進王。王二年,元和元年薨。



 宋王結,初名滋。王雲安,進王。王十八年,長慶二年薨。



 集王緗,初名淮。王宣城,進王。王十八年,長慶二年薨。



 冀王絿,初名湑。為太常卿。王德陽,進王。王三十年,大和九年薨。



 和王綺,初名浥。王河東,進王。王二十八年,太和七年薨。



 衡王絢,王二十二年,寶歷二年薨。



 會王纁,王六年,元和五年薨。



 福王綰,歷魏博節度大使。咸通元年,進拜司空。王五十七年,咸通二年薨。



 珍王繕,初名況。王洛交,後進王。亡薨年。



 撫王紘,咸通初,歷司空,又進司徒、太尉。王七十三年,乾符三年薨。



 岳王緄,王二十三年,太和二年薨。



 袁王紳,王五十六年,咸通元年薨。



 桂王綸,王十年,元和九年薨。



 翼王綽,王五十八年,咸通三年薨。



 蘄王緝,王六年,咸通八年薨。



 欽王績,亡薨年。



 憲宗二十子:紀美人生寧,懿安皇后生穆宗皇帝,孝明皇后生宣宗皇帝;餘十七王,皆後宮所生,史逸其母之號、氏。



 惠昭太子寧,貞元二十一年,始王平原,與同安、彭城、高密、文安四王同封。帝即位,進王鄧,與澧、深、洋、絳四王同封。



 於是國嗣未立,李絳等建言:「聖人以天下為大器,知一人不可獨化,四海不可無本,故建太子以自副,然後人心定,宗祏安,有國不易之常道。陛下受命四年,而塚子未建,是開窺覦之端,乖慎重之義,非所以承列聖,示萬世。」帝曰:「善。」以寧為皇太子,更名宙,前以制示絳等。未幾,復初名。冊禮用孟夏,雨,不克,改用孟秋,亦雨,冬十月克行禮。明年薨,年十九。



 澧王惲,始王同安,後進王。惠昭之喪,吐突承璀議復立儲副,意屬惲,帝自以穆宗為太子。帝崩之夕,承璀死,王被殺,秘不發喪,久之以告,廢朝三日。三子:曰漢,王東陽郡;曰源,安陸;曰演,臨安。初,惲名寬,深王察,洋王寰,絳王寮,建王審,元和七年,並改今名。



 深王悰,始王彭城郡,進王深。子潭王河內,淑吳興。



 洋王忻,始王高密,進王洋。大和二年薨。子沛王潁川郡。



 絳王悟,始王文安,進王。敬宗崩,蘇佐明等矯詔以王領軍國事。王守澄等立文宗,王見殺。二子:洙王新安,滂高平。



 建王恪,元和元年始封。時淄青節度使李師古死,其弟師道丐符節,故詔恪為鄆州大都督、平盧軍淄青等州節度大使,以師道為留後,然不出閤。長慶元年薨,無嗣。



 鄜王憬,長慶元年始王,與瓊、沔、婺、茂、淄、衢、澶七王同封。開成四年薨。子溥平陽郡王。



 瓊王悅,子津河間郡王。



 沔王恂,子瀛晉陵郡王。



 婺王懌,子清新平郡王。



 茂王愔,子潓武功郡王。



 淄王心辦,開成元年薨。子浣許昌郡王,渙馮翊郡王。



 衢王詹,子涉晉平郡王。



 澶王心充,子濘雁門郡王。



 棣王惴,大中六年始王,與彭、信二王同封。咸通三年薨,無嗣。



 彭王惕,乾寧中,韓建殺之石堤穀。無嗣。



 信王憻,咸通八年薨,無嗣。



 榮王心責,咸通三年始王。廣明初,拜司空。子令平嗣王。



 凡八王,史失其薨年。



 穆宗五子:恭僖皇后生敬宗皇帝,貞獻皇后生文宗皇帝,宣懿皇后生武宗皇帝;餘二王,亡其母之氏、位。



 懷懿太子湊,少雅裕,有尋矩。長慶元年始,王漳與安王同封。文宗即位,疾王守澄顓很,引支黨橈國,謀盡誅之,密引宰相宋申錫使為計。守澄客鄭注伺知之,以告,乃謀先事殺申錫。又以王賢,有中外望,因欲株聯大臣族夷之。乃令神策虞候豆盧著上飛變,且言:「宮史晏敬則、硃訓與申錫暱吏王師文圖不軌,訓嘗言上多疾,太子幼,若兄終弟及,必漳王立。申錫陰以金幣進王,而王亦以珍服厚答。」即捕訓等系神策獄,榜掠定其辭。諫官群伏閤極言,出獄牒付外雜治。注等懼事洩,乃請下詔貶王。帝未之悟,因黜湊為巢縣公,時大和五年也。命中人持詔即賜,且慰曰:「國法當爾,無它憂!」八年薨,贈齊王。注後以罪誅,帝哀湊被讒死不自明,開成三年追贈。



 安王溶。初,楊賢妃得寵於文宗,晚稍多疾,妃陰請以王為嗣,密為自安地。帝與宰相李玨謀,玨謂不可,乃止。乃帝崩,仇士良立武宗,欲重己功,即擿溶嘗欲以為太子事,殺之。



 敬宗五子:妃郭氏生普,餘四王,亡母之氏、位。



 悼懷太子普,姿性韶悟。寶歷元年始王晉。文宗愛之若己子,嘗欲為嗣。大和二年薨,帝惻念不能已,故贈恤加焉。敬宗第二子休復,文宗開成二年封梁王,第三子執中為襄王,第四子言揚為紀王,第五子成美為陳王。執中子寀為樂平郡王。



 陳王成美。初,文宗以莊恪薨,大臣數請建東宮,開成四年,帝乃立成美為皇太子,典冊未具而帝崩,仇士良立武宗,殺之於邸。子儼王宣城郡。



 文宗二子:王德妃生永,後宮生宗儉。



 莊恪太子永,大和四年始王魯。帝以王幼,宜得賢輔,因召見傅和元亮。元亮以卒史進,有所問,不能答。帝責謂宰相:「王可教,官屬應任士大夫賢者,寧元亮比邪!」於是劇選戶部侍郎庾敬休兼王傅,太常卿鄭肅兼長史,戶部郎中李踐方兼司馬。六年,遂立為皇太子。帝承寶歷荒怠,身勤儉率天下,謂晉王生謹敏,欲引為嗣,會蚤夭,故久不議東宮事。及太子立,天下屬心焉。開成三年,詔宮臣詣崇明門謁朔望,侍讀偶日入對。太子稍事燕豫,不能壹循法,保傅戒告,■不納。又母愛弛,楊賢妃方幸,數譖之。帝它日震怒,御延英,引見群臣,詔曰:「太子多過失,不可屬天下,其議廢之。」群臣頓首言:「太子春秋盛,雖有過,尚可改。且天下本,不可輕動,惟陛下幸赦。」御史中丞狄兼暮流涕固爭,帝未決,罷。群臣又連章論救,意稍釋,詔太子還少陽院,以中人護視,誅幸暱數十人,敕侍讀竇宗直、周敬復詣院授經。然太子終不能自白其讒,而行己亦不加修也。是年暴薨,帝悔之。明年,下詔以陳王為太子,置酒殿中。有俳兒緣橦,父畏其顛,環走橦下。帝感動,謂左右曰:「朕有天下,返不能全一兒乎!」因泣下。即取坊工劉楚才等數人付京兆榜殺之,及禁中女倡十人斃永巷,皆短毀太子者。宰相楊嗣復等不及知,因言:「楚才等罪當誅,京兆殺之,不覆奏,敢以請。」翌日,詔京兆後有決死敕不覆者,亦許如故事以聞。



 蔣王宗儉,開成二年始王。亡薨年。



 武宗五子,其母氏、位皆不傳。



 杞王峻,開成五年始王;益王峴,會昌二年始王,與兗、德、昌三王同封;兗王岐;德王嶧;昌王嵯:並逸其薨年。



 宣宗十一子:元昭太后生懿宗皇帝,餘皆亡其母之氏、位。



 靖懷太子渼,會昌六年始王雍,與夔、慶二王同封。大中六年薨,有詔追冊。



 雅王涇,大中元年始王。亡薨年。



 通王滋,會昌六年始王夔,與慶王沂同封。帝初詔鄆王居十六宅,餘五王處大明宮內院,以諫議大夫鄭漳、兵部郎中李鄴為侍讀,五日一謁乾符門,為王授經。鄆王立為懿宗,乃罷。滋徙王。昭宗乾寧三年,領侍衛諸軍。是時,誅王行瑜,而李茂貞怨,以兵入覲,詔滋與諸王分統安聖、奉宸、保寧、安化軍衛京師。天子將狩太原,韓建道迎之,留次華州。建畏王等有兵,遣人上急變,告諸王欲殺建,脅帝幸河中。帝驚,召建論之,稱疾不肯入。敕滋與睦王、濟王、韶王、彭王、韓王、沂王、陳王謁建自解,建留軍中,奏言:「中外異體,臣不可以私見。」又言:「晉八王擅權,卒敗天下。請歸十六宅,悉罷所領兵。」帝不許。建以兵環行在,請誅大將李筠。帝懼,斬筠以謝。建盡逐衛兵,自是天子孤弱矣。



 初,帝使嗣延王戒丕、嗣丹王允往見李克用,二王還,建惡之;又嗣覃王嘗督軍伐茂貞,於是劾奏:「比歲兵纏近輔,諸王階其禍,使乘輿越在下籓,不得安,臣已請解其兵。今延、覃、丹三王尚陰計以危國,請誅之。」帝曰:「渠至是邪?」後三日,與劉季述矯詔以兵攻十六宅。諸王被發乘垣走,或升屋極號曰:「帝救我!」建乃將十一王並其屬至石堤穀殺之,徐以謀反聞,天下冤之。濟、韶、彭、韓、沂、陳、延、覃、丹九王,史逸其系胄云。



 慶王沂,大中十四年薨。



 濮王澤,大中二年始王。亡薨年。



 鄂王潤,大中五年始王。乾符三年薨。



 懷王洽,大中八年與昭、康二王同封。亡薨年。



 昭王汭,乾符三年薨。



 康王汶,乾符四年薨。



 廣王澭,大中十一年始王,與衛王同封。乾符四年薨。



 衛王灌,大中十四年薨。



 懿宗八子:惠安皇后生僖宗皇帝,恭憲皇后生昭宗皇帝,餘六王亡其母氏、位。



 魏王佾,咸通三年始王,與涼、蜀二王同封。



 涼王侹,乾符六年薨。



 蜀王佶。



 威王偘,咸通六年始王郢,十年徙王。



 吉王保,咸通十三年始王,與睦王同封。王於兄弟為最賢。始,僖宗崩,王最長,將立之,楊復恭獨議以昭宗嗣。乾寧元年,李茂貞等以兵入京師,謀廢帝立王,會李克用以兵逐行瑜,乃止。



 恭哀太子倚,初封睦王。為劉季述所殺,天復初追贈。



 僖宗二子,史失其母氏、位。



 建王震,中和元年始王;益王陛,光啟三年始王;並亡薨年。



 昭宗十七子:積善皇后生裕及哀皇帝,餘皆失母之氏、位。



 德王裕,大順二年始王。帝幸華州,韓建已奪諸王兵,不自安,乃請王皇子之未王者,既又殺諸王,因請立裕為皇太子,釋言於四方,時乾寧四年也。劉季述等幽帝東內,奉裕即皇帝位。季述誅,裕匿右軍,或請殺之,帝曰:「太子沖孺,賊強立之,且何罪?」詔還少陽院,復為王。硃全忠自鳳翔還,見王春秋盛,標宇軒秀,忌之,密語崔胤曰:「王既竊帝矣,大義滅親,渠可留?公任宰相,盍啟之?」胤從容言如全忠意,帝不許。它日,以語全忠,全忠曰:「此國大事,臣安敢與?此必胤賣臣也。」乃免。帝遷洛,它日謂蔣玄暉曰:「德王,朕愛子,全忠奈何欲殺之?」言已泣下,自嚙指流血。玄暉即擿語全忠,全忠恚。帝被殺,玄暉置酒邀諸王九曲池,飲酣,皆殺之,投尸水中。



 棣王祤,乾寧元年始王,與虔、沂、遂三王同封。



 虔王禊。



 沂王禋。



 遂王禕。



 景王秘,乾寧四年始王,與祁王同封。



 祁王祺。



 雅王禛,光化元年始王,與瓊王同封。



 瓊王祥。



 端王禎,天祐元年始王,與豐、和、登、嘉四王同封。



 豐王祁。



 和王福。



 登王禧。



 嘉王祜。



 潁王禔,天祐二年始王,與蔡王祐同封。



 蔡王祐。



 贊曰:唐自中葉,宗室子孫多在京師,幼者或不出閤,雖以國王之,實與匹區夫不異,故無赫赫過惡,亦不能為王室軒輊,運極不還,與唐俱殫。然則歷數短長,自有底止。彼漢七國、晉八王,不得其效,愈速禍云。



\end{pinyinscope}