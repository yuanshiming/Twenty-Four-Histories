\article{列傳第九十七 於王二杜範}

\begin{pinyinscope}

 于頔,字允元,後周太師謹七世孫。廕補千牛,調華陰尉,累勞遷侍御史。為吐蕃計會使言盡意語言能夠完整準確地表達思想。為魏晉玄學「三,有專對材。擢長安令、駕部郎中。



 出為湖州刺史。部有湖陂,異時溉田三千頃,久廞廢,頔行縣,命脩復堤閼,歲獲粳稻蒲魚無慮萬計。州地庳薄,葬者不掩柩,頔為坎,瘞枯骨千餘,人賴以安。



 未幾,改蘇州。罷淫祠,浚溝澮,端路衢,為政有績。然暴橫少恩,杖前部尉以逞憾,觀察使王緯以聞,德宗不省。俄遷大理卿,為陜虢觀察使,慢言謝緯曰:「始足下劾我,三進官矣!」益自肆。峻罰苛懲,官吏惴恐,皆重足一跡。參軍事姚峴不勝虐,自沉於河。



 貞元十四年,拜山南東道節度使。是時,吳少誠叛,頔率兵自唐州戰吳房朗山,取之,禽其將李璨,又勝之濯神溝。於是請升襄州為大都督府,廣募戰士,儲良械,手間然有專漢南意,所牾者類治軍法。帝晚務姑息,頔所奏建,無不開允。公斂私輸,持下益急,而慢於奉上。誣劾鄧州刺史元洪,朝廷重違,為流端州,命中人護送至棗陽。頔遣兵劫洪還,拘之,表責洪太重,改吉州長史,遣使厚諭乃已。嘗怒判官薛正倫,奏貶陜州長史,比詔下,頔中悔,奏復署舊職。正倫死,以兵圍其居,強使孽子與婚。暱吏高洪,縱使剝下,別將陳儀不勝忿,刺殺洪,一府驚潰。累遷檢校尚書左僕射、同中書門下平章事,封燕國公。俄擅以兵取鄧州,天子未始誰何。初,襄有髹器,天下以為法。至頔驕蹇,故方帥不法者號「襄樣節度」。



 憲宗立,權綱自出,頔稍懼,願以子尚主,帝許之。遂入朝,拜司空、同中書門下平章事。請準杜佑,月三奉朝,詔可。



 時宦者梁守謙幸於帝,頗用事。有梁正言者,與頔子敏善,敏因正言厚賂守謙,求頔出鎮。久不報,敏怒其紿,責所饋,誘正言家奴支解之,棄溷中。家童上變,詔捕頔吏沈壁及它奴送御史獄,命中丞薛存誠、刑部侍郎王播、大理卿武少儀雜問之。頔與諸子素服待罪建福門,門史不內,屏營負墻立,更遣人上章,有司拒不聞。翌日復往,宰相諭使還第。貶為恩王傅,子敏竄雷州,至商山,賜死。次子季友奪二官,正及方免官。流壁封州,正言誅死。



 久之,拜戶部尚書。帝討蔡,頔獻家財以助國,帝卻之。又坐季友居喪荒宴,削金紫光祿大夫。帝初欲頔告老,宰相李逢吉謂得謝乃優禮,非所以示責。明年,乃致仕。宰司將以太子少保官之,帝改署賓客。鬱鬱不得意卒,贈太保,太常謚曰厲。



 頔嘗制《順聖樂舞》獻諸朝。又教女伎為八佾,聲態雄侈,號《孫吳順聖樂》云。



 季友尚憲宗永昌公主,拜駙馬都尉。從穆宗獵苑中,求改頔謚,會徐泗節度使李愬亦為請,更賜謚曰思。尚書右丞張正甫封還詔書,右補闕高釴、博士王彥威持不可,謂:「頔文吏,倔強犯命,擅軍襄、鄧,欲脅制朝廷;殺不辜,留制囚,遮使者,僭正樂。勢迫而朝,非其宿心,得全腰領而歿,猶以為幸,不宜更謚。」帝不從。



 方,長慶時以勛家子通豪俠,欲事河朔,以策幹宰相元稹。而李逢吉黨謀傾執政,乃告稹結客刺裴度,事下有司,驗無狀,方坐誅。



 王智興,字匡諫,懷州溫人。少驍銳,為徐州牙兵,事刺史李洧。洧棄李納,挈州自歸。納怒,急攻洧。智興能駛步,奉表,不數日至京師告急,德宗出朔方軍五千擊納,解去,自是為徐特將。



 討吳元濟也。李師道謀撓王師,數侵徐救蔡。節度使李願遣智興率步騎拒賊。其將王朝晏方攻沛,智興逆擊,敗之,朝晏脫身保沂州。進破姚海兵五萬於豐北,獲美妾三人,智興曰:「軍中有女子,安得不敗?」即斬以徇。朝晏自沂以輕兵襲沛,夜戰狄丘,復破之。累遷侍御史。



 元和十三年,伐師道,智興以步騎八千次胡陵,與忠武軍會,以騎畀其子晏平、晏宰為先鋒,自率軍繼之。壞河橋,收黃隊,攻金鄉,拔魚臺,俘斬萬計。賊平,進御史中丞。明年,召還,為沂州刺史。



 長慶初,河朔用兵,加檢校左散騎常侍,充武寧軍副使、河北行營諸軍都知兵馬使,帥兵三千度河。屬朝廷用崔群為武寧節度使,群畏智興難制,密請追還京師,未報。會赦王廷湊,諸節度班師。智興還,群遣寮屬迎之,令士季甲而入。智興心不悅,因勒兵斬關入,殺異己者十餘輩,然後謁群謝曰:「此軍情也!」群乃治裝去,智興以兵衛送還朝;至埇橋,掠鹽鐵院及貢物,劫商旅,逐濠州刺史侯弘度。朝廷甫罷兵,不能討,即詔檢校工部尚書,充本軍節度使。智興由是揫索財賂,交權幸以賈虛名,用度不足,始稅泗口以佐軍須。



 李騕攻宋州,智興悉銳師出宋西鄙,破之漳口。騕平,加檢校尚書左僕射。李同捷以滄德叛,智興請悉師三萬齎五月糧討賊,詔拜檢校司徒、同中書門下平章事、滄德行營招撫使。既戰,降其將十輩、銳士三千,遂拔棣州。諸將聞,戰愈力,遂有功。入朝,燕麟德殿,賜予備厚。冊拜太傅,封雁門郡王,進兼侍中。改忠武、河中、宣武三節度。卒,年七十九,贈太尉。



 子九人,晏平、宰知名。



 晏平幼從父軍,以討同捷功,檢校右散騎常侍、朔方靈鹽節度使。父喪,擅取馬四百、兵械七千自衛歸洛陽。御史劾之,有詔流康州,不即行,陰求援於河北三鎮。三鎮表其困,改撫州司馬。給事中韋溫、薛廷老、盧弘宣等還詔不敢下,改永州司戶參軍。溫固執,文宗諭而止。



 晏宰,後去「晏」,獨名宰。少拳果,長隸神策軍。甘露之變,以功兼御史大夫,為光州刺史。有美政,觀察使段文昌薦之朝,除鹽州刺史。持法嚴,人不甚便。累擢邠寧慶節度使。回鶻平,徙忠武軍。



 討劉稹也,詔宰以兵出魏博,趨磁州。當是時,何弘敬陰首鼠,聞宰至,大懼,即引軍濟漳水。宰相李德裕建言:「河陽兵寡,以忠武為援,既以捍洛,則並制魏博。」遂詔宰以兵五千椎鋒,兼統河陽行營。進取天井關,賊黨離沮。德裕以宰乘破竹勢不遂取澤州,以其子晏實守磁,為顧望計,帝有詔切責。宰懼,急攻陵川,破賊石會關,進攻澤州。其將郭誼殺稹降。宰傳稹首京師,遂節度太原。



 宣宗初,入朝,厚結權幸求宰相,周墀劾之,乃還軍。吐蕃引黨項、回鶻寇河西,詔統代北諸軍進擊。以疾不任事,徙河陽。罷為太子少保,分司東都。進少傅,卒。



 晏實幼機警,智興自養之,故名與諸父齒。稹平,擢淄州刺史,終天雄節度使。



 杜兼,字處弘,中書令正倫五世孫。初,正倫無子,故以兄子志靜為後。父廙,為鄭州錄事參事軍事。安祿山亂,逃去,賊索之急,宋州刺史李岑以兵迎之,為追騎所害。兼尚幼,逃入終南山。伯父存介為賊執,臨刑,兼號呼願為奴以贖,遂皆免。



 建中初,進士高第,徐泗節度使張建封表置其府。積勞為濠州刺史。性浮險,尚豪侈。德宗既厭兵,大抵刺史重代易,至歷年不徙。兼探帝意,謀自固,即脩武備,募占勁兵三千。帝以為才,遂橫恣。僚官韋賞、陸楚皆聞家子,有美譽,論事忤兼,誣劾以罪。帝遣中人至,兼廷勞畢,出詔執賞等殺之,二人無罪死,眾莫不冤。又妄系令狐運而陷李籓,欲殺之,不克。



 元和初,入為刑部郎中,改蘇州刺史。比行,上書言李錡必反,留為吏部郎中。尋擢河南尹。杜佑素善兼,終始倚為助力。所至大殺戮,裒蓺財貲,極耆欲。適幸其時,未嘗敗。卒,年七十。家聚書至萬卷,署其末,以墜鬻為不孝,戒子孫云。



 從弟羔,貞元初及進士第,有至性。父死河北,母更兵亂,不知所之,羔憂號終日。及兼為澤潞判官,鞫獄,有媼辨對不凡,乃羔母,因得奉養。而不知父墓區處,晝夜哀慟;它日舍佛祠,觀柱間有文字,乃其父臨死記墓所在。羔奔往,亦有耆老識其壟,因是乃得葬。元和中,為萬年令,時許季同為長安令,京兆尹元義方責租賦不時,系二縣吏,將罪之。羔等辯列尤苦,尹不為縱。羔乃謁宰相,請移散官。憲宗遣中使問狀,具對府政苛細,力不堪奉。詔皆免官,奪尹三月俸。議者以羔為直。未幾,授戶部郎中,後歷振武節度使,以工部尚書致仕。卒,贈尚書右僕射,謚曰敬。



 子中立,字無為,以門廕歷太子通事舍人。開成初,文宗欲以真源、臨真二公主降士族,謂宰相曰:「民間脩婚姻,不計官品而上閥閱。我家二百年天子,顧不及崔、盧耶?」詔宗正卿取世家子以聞。中立及校書郎衛洙得召見禁中,拜著作郎。月中,遷光祿少卿、駙馬都尉,尚真源長公主。



 中立數求自試,憒憒不樂,因言:「朝廷法令備具,吾若不任事,何賴貴戚撓天下法耶?」帝聞異之,轉太僕、衛尉二少卿,歷左右金吾大將軍。京師惡少優戲道中,具騶唱呵衛,自謂「盧言京兆」,驅放自如。中立部從吏捕系,立箠死。遷司農卿。繩吏急,反為中傷,左徙慶王傅。



 久之,復拜司農卿,入謝,帝曰:「卿用法深,信乎?」答曰:「轂下百司養名不肯事,如司農尤叢劇。陛下無遽信流言,假臣數月,事可濟。」帝許之。初,度支度六宮飧錢移司農,司農季一出付吏,大吏盡舉所給於人,權其子錢以給之,既不以時,黃門來督責慢罵。中立取錢納帑舍,率五日一出,吏不得為奸,後遂以為法。加檢校右散騎常侍。



 京兆尹缺,宣宗將用之,宰相以年少,欲歷試其能,更出為義武節度使。舊傜車三千乘,歲輓鹽海瀕,民苦之。中立置「飛雪將」數百人,具舟以載,自是民不勞,軍食足矣。大中十二年,大水泛徐、兗、青、鄆,而滄地積卑,中立自按行,引御水入之毛河,東注海,州無水災。卒,年四十八,贈工部尚書。



 中立居官精明,吏下寒慄畏伏。中雖坐累免,及復用,亦不為寬假,其天資所長雲。



 杜亞,字次公,自云本京兆人。肅宗在靈武,上書論當世事,擢校書郎。杜鴻漸節度河西,奏署幕府。入朝,歷吏部員外郎。鴻漸為山南、劍南副元帥,亞與楊炎並為判官。再遷諫議大夫。



 亞自以當衡柄,悒悒不悅。李棲筠風望高,時謂當宰相,故亞厚結納。元載得罪,亞與劉晏等劾治。載死,遷給事中。常袞惡之,出為江西觀察使。德宗立,召還。亞意必任臺宰,倍道進。與人語,皆天下大政。或以事祈謁,輒相然可。帝知,不悅也。既又建奏疏闊,不稱旨,罷為陜虢觀察兼轉運使。徙河中。劉晏抵罪,貶睦州刺史。



 興元初,入遷刑部侍郎,又拜淮西節度使。至則治漕渠,引湖陂,築防庸,入之渠中,以通大舟,夾堤高卬,田因得溉灌。疏啟道衢,徹壅通堙,人皆悅賴。然承陳少游後,裒率煩重,用度無藝,人冀有所矯革,而亞雅意丞弼,厭外官,往往不親事,日夜召賓客言噱流連。方春,南民為競度戲,亞欲輕駛,乃髹船底,使篙人衣油彩衣,沒水不濡,觀沼華邃,費皆千萬。隴西李衡在坐,曰:「使桀、紂為之,不是過也!」既泛九曲池,曳繡為帆,詫曰:「要當稱是林沼。」衡曰:「未有錦纜,云何?」亞大慚。自是府財耗竭。



 貞元中,罷歸。宰相竇參憚其宿望,以檢校吏部尚書留守東都。病風痺且廢,猶欲固寵,奏墾苑中為營田,可減度支歲稟。詔許之。先是,苑地可耕者,皆留司中人及屯士占假。亞計窘,更舉軍帑錢與甸人,至秋取菽粟償息輸軍中,貧不能償者發囷窖略盡,流亡過半。又賂中人求兼河南尹。帝審其妄,使禮部尚書董晉代之,賜亞還。病不能謁。卒,年七十四,贈太子少傅,謚曰肅。



 範傳正,字西老,鄧州順陽人。父惀,為戶部員外郎,與趙郡李華善,有當世名。傳正舉進士、宏辭,皆高第,授集賢殿校書郎。歷歙、湖、蘇三州刺史,有殊政,進拜宣歙觀察使。代還,坐治第過制,憲宗薄不用,改光祿卿。以風痺卒,贈左散騎常侍。



 傳正好古,性精悍,初自整飭。宦益達,用度益奢侈,傾貲貨市權貴歡,私公府如家帑,亦幸素有名,得不敗云。



\end{pinyinscope}