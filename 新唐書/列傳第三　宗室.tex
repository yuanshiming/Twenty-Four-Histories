\article{列傳第三 宗室}

\begin{pinyinscope}

 江夏王道宗廣寧縣公道興永安王孝基涵淮陽王道玄漢長平王叔良郇國公孝協彭國公思訓新興郡王晉長樂王幼良襄武王琛河間王孝恭晦漢陽王廬江王瑗淮安王神通膠東王道彥梁郡公孝逸國貞暠說齊物復襄邑王神符從晦隴西公博義渤海王奉慈戡



 太祖八子:長延伯,次真,次世祖皇帝,次璋,次繪,次禕,次蔚,次亮。



 南陽公延伯,蚤薨,無嗣。高祖武德中,與六王同追封。



 譙王真,從太祖戰歿,無嗣。



 畢王璋,仕周為梁州刺史,與趙王祐謀殺隋文帝,不克,死。生二子:曰韶,曰孝基。韶死隋世,武德時追封東平王,生子道宗。



 江夏郡王道宗字承範。高祖即位,授左千牛備身、略陽郡公。裴寂與劉武周戰度索原,寂敗,賊逼河東,道宗年十七,從秦王討賊。王登玉壁城以望,謂道宗曰:「賊怙眾欲戰,爾計謂何?」對曰:「武周席勝,剡然鋒未可當,正宜以計摧之。且烏合之眾憚持久,若堅壁以頓其銳,須食盡氣老,可不戰禽也。」王曰:「而意與我合。」既而賊糧匱,夜引去,追戰滅之。



 出為靈州總管。時梁師都弟洛仁連突厥兵數萬傅於壘,道宗閉城守,伺隙出戰,破之。高祖謂裴寂曰:「昔魏任城王彰有卻敵功,道宗似之。」因封任城王。



 始,突厥鬱射設入居五原,道宗逐出之,震耀威武,斥地贏千里。貞觀元年,召拜鴻臚卿,遷大理。太宗方經略突厥,復授靈州都督。三年,為大同道行軍總管,助李靖破虜,親執頡利可汗,賜封六百戶,還為刑部尚書。吐谷渾寇邊,靖出昆丘道,詔與侯君集為靖副。賊聞兵且至,走嶂山數千里。諸將欲止,獨道宗請窮追,靖曰:「善。」君集未從。道宗以單師進,去大軍十日,及之。吐谷渾拒險殊死鬥,道宗陰引千騎超山乘其後,賊驚,遂大潰。徙封江夏,授鄂州刺史。久之,坐貪贓,帝聞,怒曰:「朕提四海之富,士馬若林,如使轍跡環天下,游觀不度,採絕域之玩、海表之珍,顧不得邪?特以勞民自樂,不為也。人心無藝,當以誼制之。今道宗已王,稟賜多而貪不止,顧不鄙哉!」乃免官,削封戶,以王就第。明年,召為茂州都督,未行,拜晉州刺史。遷禮部尚書。



 侯君集破高昌還,頗怨望。道宗嘗從容奏言:「君集智小言大,且為戎首。」帝問所以知必反者,對曰:「見其忌而矜功,恥為房、李下,官尚書,常鬱鬱不平。」帝曰:「君集誠有功,材無不堪,朕寧惜爵位邪?弟未及耳。不宜輕億度,使自猜危。」既而君集反,帝笑曰:「如公素揣。」



 帝將討高麗,先遣營州都督張儉輕騎度遼規形勢,儉畏,不敢深入。道宗請以百騎往,帝許之,約其還,曰:「臣請二十日行,留十日覽觀山川,得還見天子。」因秣馬束兵,旁南山入賊地,相易險,度營陣便處。將還,會高麗兵斷其路,更走間道,謁帝如期。帝曰:「賁、育之勇何以過!」賜金五十斤,絹千匹。



 乃詔與李勣為前鋒,濟遼,拔蓋牟城。會賊救至,道宗與總管張君乂領騎裁四千,虜十倍,皆欲浚溝保險須帝至,道宗曰:「賊遽來,其兵必疲,我一鼓摧之,固矣。昔耿弇不以賊遺君父,吾為前軍,當清道迎乘輿,尚何待?」勣善之。選壯騎數十,突進賊營,左右出入,勣合擊,大破之。帝至,咨美,賜奴婢四十口。乃築拒闉,攻安市城,闉毀傅城,道宗失部分,反為賊據。帝斬其果毅傅伏愛,道宗跣行請罪,帝曰:「漢武帝殺王恢,不如秦穆公赦孟明。」遂置不問。在陣傷足,帝親加砭治,賜御膳。還,以疾辭劇就閑,改太常卿。



 高宗永徽初,房遺愛以反誅,長孫無忌、褚遂良與道宗有宿怨,誣與遺愛善,流象州,道病薨,年五十四。無忌等得罪,詔復爵邑。道宗晚好學,接士大夫,不倨於貴。國初宗室,唯道宗、孝恭為最賢。子景恆,封盧國公,相州刺史。



 道宗弟道興,武德初,爵廣寧郡王,以屬疏降封縣公。貞觀九年,為交州都督,以南方瘴厲,恐不得年,頗忽忽憂悵,卒於官,贈交州都督。



 永安壯王孝基,武德初得王,歷陜州總管、鴻臚卿,以罪奪官。



 二年,劉武周寇太原,夏人呂崇茂以縣應賊。詔孝基為行軍總管攻之,工部尚書獨孤懷恩、內史侍郎唐儉、陜州總管於筠隸焉。筠請急攻城,絕外援,且當有變。時懷恩挾異計,紿說孝基曰:「夏城堅,攻之引日,宋金剛在近,內拒外強,一敗塗地。不如頓兵待秦王破賊,則夏自孤,此謂不戰而屈人也。」孝基謂然。會尉遲敬德至,與崇茂夾〓官師,遂大敗。孝基及筠等皆執於賊,謀亡歸,為賊所害。高祖為發哀,優賜其家。晉陽平,購尸不獲,招魂以葬,贈左衛大將軍及謚。



 無子,以兄子道立嗣,封高平王,後降封縣公,終陳州刺史。曾孫涵。



 涵,簡素忠謹,為宗室俊。累授贊善大夫。郭子儀表為關內鹽池判官。肅宗至平涼,未知所從。朔方留後杜鴻漸等條士馬倉廥,使涵奉箋馳謁肅宗。涵既見,敷奏明辯,肅宗悅,除左司員外郎,再遷宗正少卿。



 寶應初,河朔平,涵方母喪,奪哀持節宣慰,所至州縣,非公事未嘗言,蔬飯水飲,席地以瞑。使還,固請終制,代宗見其臒毀,許之。服除,擢給事中,遷兵部侍郎。



 硃希彩殺李懷仙,復宣慰河北,還為浙西觀察使。居五歲,入朝,拜御史大夫、京畿觀察使。德宗嗣位,以涵和易無所繩舉,除太子少傅、山陵副使。以父諱徙光祿卿。未幾,遷左散騎常侍,以尚書右僕射致仕,累封襄武縣公,卒,贈太子太保。



 子鰅,貞元初為饒州別駕。妾高以善歌入宮,鰅因御醫許泳通書,坐誅。



 雍王繪為隋夏州總管。子贄,追爵河南王,生道玄。



 淮陽壯王道玄,性謹厚,習技擊,然進止都雅。武德初,例王。年十五,從秦王擊宋金剛於介州,先登,王壯之,賞予良厚。討王世充,戰多。竇建德屯虎牢,王輕騎致賊,遣道玄伏以待,賊至,走之。戰範汜水,登南坡,貫賊陣出其背,復引還,賊皆靡,所發命中。王喜,以副騎給之。每赴敵,飛矢著身如猖,氣益厲。東都平,為洛州總管。府廢,更授刺史。俄為山東道行軍總管討劉黑闥,以多見褒。



 黑闥再亂,道玄率史萬寶戰下博,越濘馳,約萬寶繼進,萬寶素少之,不肯前,曰:「吾被詔,以王兒子名大將,而軍進退實在我。今其輕鬥,若大軍竭馳,必陷濘,莫如以王啖賊,我結陣待之,雖不利王,而利國也。」道玄遂戰歿,年十九。萬寶為賊所乘,舉軍潰,身獨免。太宗追悼曰:「自兵興,兒常從我,每見我深入輒克,故慕之。惜其少,遠圖不究,哀哉!」因流涕。贈左驍衛大將軍及謚。



 無子,以弟道明嗣王,遷左驍衛大將軍。貞觀十四年,與武衛將軍慕容寶節送弘化公主於吐谷渾,坐漏言主非帝女,奪王,終鄆州刺史。六世孫漢。



 漢字南紀,少事韓愈,通古學,屬辭雄蔚,為人剛,略類愈。愈愛重,以子妻之。擢進士第,遷累左拾遺。



 敬宗侈宮室,舶買獻沈香亭材,帝受之,漢諫曰:「以沈香為亭,何異瑤臺瓊室乎?」是時,王政謬僻,漢言切,多所救補。坐婞訐出佐興元幕府。



 文宗立,召為屯田員外郎、史館修撰。論次《憲宗實錄》,書宰相李吉甫事不假借,子德裕惡之。會李宗閔當國,擢知制誥,稍進御史中丞,吏部侍郎。初,德裕貶袁州,漢助為排擠,後德裕復輔政,漢坐宗閔黨出為汾州刺史,宗閔再逐,改州司馬。詔有司不二十年不得用。然不數歲,徙絳州長史,遂不復振。大中時,召拜宗正少卿,卒。



 始,漢為中丞,表孔溫業為御史,及漢晚見召,溫業已為中丞,每燕集,人以為榮。



 郇王禕,為隋上儀同三司。生子叔良、德良、幼良。



 長平肅王叔良,武德初,例王,鎮涇州,捍薛仁杲。仁杲內史令翟長孫以眾降。於是大饑,米斗千錢,叔良不恤士,損糧以漁利,下皆怨。仁杲知之,陽言食盡,去,遣高〓人詭降。叔良遣驃騎劉感受之,未至城,三烽發,仁杲兵自南原噪而還,大戰百里細川,感為賊執。叔良懼,悉出金勞軍,委事於長孫,乃克安。



 久之,突厥入寇,詔叔良率五將軍擊之,中流矢,道薨。贈左翊衛大將軍、靈州總管。



 子孝協嗣。



 孝協,始王範陽,俄降為郇國公、魏州刺史。麟德中,坐贓抵死,司宗卿隴西王博義等為言於高宗求貸,帝不許,遂自殺。



 弟孝斌為原州都督府長史。生子思訓,為江都令。武後多殺宗室,思訓棄官去。中宗復位,以耆舊擢宗正卿,封隴西郡公,歷益州都督府長史。開元初,進彭國公,加戶滿四百,進右武衛大將軍。卒,贈秦州都督,陪葬橋陵。思訓善畫,世所謂「李將軍山水」者。弟思誨,為揚州參軍事。子林甫,自有傳。



 新興郡王德良,少以疾不任職。薨,贈涼州都督。



 孫晉,先天中,為雍州長史,治有名,襲王。坐豫太平公主謀被誅,改氏「厲」。晉就刑,僚吏奔解,唯司功參軍李捴從王如它日,晉死,哭其尸盡哀。姚元崇嘆曰:「欒、向儔邪!」擢為尚書郎。



 長樂郡王幼良,資暴急,高祖數曉勒,不悛。有盜其馬者,輒殺之。帝怒曰:「盜信有罪,王而專殺可乎?」詔禮部尚書李綱召宗室即朝堂杖之百,乃釋。出為涼州都督,嘯不逞為左右,市里苦之。



 太宗立,或告王陰養士,交境外。詔中書令宇文士及往代,並按狀。士及繩之急,左右恐,欲劫王由間道趨長安自明,不即北奔突厥。士及露劾,帝復遣侍御史孫伏伽鞫視,無異辭,遂賜死。六世孫回,別傳。



 蔡烈王蔚為周朔州總管,生子安、哲。



 西平懷王安,仕隋為右領軍大將軍,封趙公。武德時,例王。生子琛、孝恭、瑊、〓。



 襄武郡王琛字仲寶。木訥少文。隋義寧初,封襄武郡公,與太常卿鄭元〓持女伎聘突厥始畢可汗,約和親。始畢禮之,贈遺蕃渥,遣骨吐祿特勒隨琛入獻,授刑部侍郎。武德初,始王,歷利、蒲、絳三州總管。宋金剛陷澮州,稽胡多叛,詔琛鎮隰州,政寬簡,為夷夏愛便。薨,子儉襲王,例降為公。



 河間元王孝恭,少沈敏,有識量。



 高祖已定京師,詔拜山南招尉大使,徇巴蜀,下三十餘州。進擊硃粲,破之,俘其眾,諸將曰:「粲徒食人,摯賊也,請坑之。」孝恭曰:「不然,今列城皆吾寇,若獲之則殺,後渠有降者乎?」悉縱之。繇是騰檄所至輒下。



 明年,拜信州總管,承制得拜假。當是時,蕭銑據江陵,孝恭數進策圖銑,帝嘉納。進王趙郡,以信州為夔州。乃大治舟艦,肄水戰。會李靖使江南,孝恭倚其謀,遂圖江陵,盡召巴蜀首領子弟收用之,外示引擢而內實質也。俄進荊湘道總管,統水陸十二軍發夷陵,破銑二鎮,縱戰艦放江中。諸將曰:「得舟當濟吾用,棄之反資賊,奈何?」孝恭曰:「銑之境,南際嶺,左薄洞庭,地險士眾,若城未拔而援至,我且有內外憂,舟雖多,何所用之?今銑瀕江鎮戍,見艫舠蔽江下,必謂銑已敗,不即進兵,覘候往返,以引救期,則吾既拔江陵矣。」已而救兵到巴陵,見船,疑不進。銑內外阻絕,遂降。帝悅,遷荊州大總管,詔圖破銑狀以進。



 孝恭治荊,為置屯田,立銅冶,百姓利之。遷襄州道行臺左僕射。時嶺表未平,乃分遣使者,綏輯安慰,其款附者四十有九州,朝廷號令暢南海矣。



 未幾,輔公祏反,寇壽陽,詔孝恭為行軍元帥討之。引兵趨九江,李靖、李勣〗、黃君漢、張鎮州、盧祖尚皆稟節度。將發,大饗士,杯水變為血,坐皆失色,孝恭自如,徐曰:「禍福無基,唯所召爾!顧我不負於物,無重諸君憂。公祏禍惡貫盈,今仗威靈以問罪,杯中血,乃賊臣授首之祥乎!」盡飲罷,眾心為安。公祏將馮惠亮等拒嶮邀戰,孝恭堅壁不出,遣奇兵絕餉道,賊饑,夜薄營,孝恭臥不動。明日,使羸兵扣賊壘挑之,祖尚選精騎陣以待。俄而兵卻,賊追北且囂,遇祖尚軍,薄戰,遂大敗。惠亮退保梁山,孝恭乘勝破其別鎮,賊赴水死者數千計。公祏窮,棄丹楊走,騎窮追,生禽之,江南平。璽書褒美,賜甲第一區、女樂二部、奴婢七百口、寶玩不貲。進授東南道行臺左僕射。行臺廢,更為揚州大都督。



 孝恭再破巨賊,北自淮,東包江,度嶺而南,盡統之。欲以威重誇遠俗,乃築第石頭城,陳廬徼自衛。或誣其反,召還,頗為憲司鐫詰,既無狀,赦為宗正卿。賜實封千二百戶。歷涼州都督、晉州刺史。貞觀初,為禮部尚書,改王河間。



 性奢豪,後房歌舞伎百餘,然寬恕退讓,無矜伐色,太宗用是親重之,宗室莫比也。嘗謂人曰:「吾所居頗壯麗,非吾心也。當別營一區,令粗足充事而已。吾歿後,子也才,易以守;不才,不為他人所利。」十四年,中飲暴薨,年五十。帝哭之慟,贈司空、揚州都督及謚,陪葬獻陵。



 始,隋亡,盜賊遍天下,皆太宗身自討定,謀臣驍帥並隸麾下,無特將專勛者,惟孝恭獨有方面功以自見雲。子崇義、晦。



 崇義嗣王,降封譙國公,歷蒲、同二州刺史、益州都督府長史,有威名。終宗正卿。



 晦,乾封中為營州都督,以治狀聞,璽書勞賜。遷右金吾將軍,檢校雍州長史,摧擿奸伏無留隱,吏下畏之。高宗將幸洛,詔晦居守,謂曰:「關中事一以屬公,然法令牽制,不可以成政,法令外茍可以利人者行之,毋須以聞。」故晦治有異績。武后時,遷秋官尚書。卒,贈幽州都督。初,晦第起觀閣,下臨肆區,其人候晦曰:「庶人不及以禮,然室家之私,不願外窺,今將辭公。」晦驚,遽毀徹之。子榮,奉吳王恪祀。



 濟北郡王瑊,武德中,為尚書左丞,例王。終始州刺史。



 漢陽郡王瑰,始為郡公,進王。高祖使持幣遺突厥頡利可汗言和親事。頡利始見瑰,倨甚。瑰開說,示以厚幣,乃大喜,改容加禮,因遣使隨入獻名馬。後復聘,頡利謂其下曰:「前瑰來,悔不少屈之,當使拜我。」瑰同知之,既見頡利,即長揖。頡利怒,留不遣。瑰意象自若,不為屈。虜知不可劫,卒以禮遣。



 遷左武候將軍,代孝恭為荊州都督,政務清靜。嶺外酋豪數相攻,瑰遣使諭威德,皆如約,不敢亂。後例為公。長史馮長命者,嘗為御史大夫,素貴,事多專決,瑰怒,杖之,坐免。起為宜州刺史、散騎常侍,薨。



 濟南郡王哲,為隋柱國、備身將軍,追王。



 子瑗。



 廬江郡王瑗字德圭。武德時,例王,累遷山南東道行臺右僕射。與河間王孝恭合討蕭銑,無功。更為幽州都督。瑗素懦,朝廷恐不任職,乃以右領軍將軍王君廓輔行。君廓,故盜也,其勇絕人,瑗倚之,許結婚,寄心腹。



 時隱太子有陰謀,厚結瑗。太子死,太宗令通事舍人崔敦禮召瑗,瑗懼有變。君廓內險賊,欲以計陷瑗而取己功,即謂瑗曰:「事變未可知,大王國懿親,受命守邊,擁兵十萬,而從一使者召乎?且趙郡王前已屬吏,今太子、齊王又復爾,大王勢能自保邪?」因泣。瑗信之,曰:「以命累公。」乃囚敦禮,勒兵,召北燕州刺史王詵與計事。兵曹參軍王利涉說瑗曰:「王今無詔擅發兵,則反矣。當須權結眾心。若諸刺史召之不至,將何以全?」瑗曰:「奈何?」對曰:「山東豪傑嘗為竇建德所用,今失職與編戶夷,此其思亂,若旱之望雨。王能發使,使悉復舊職,隨在所募兵,有不從,得輒誅之,則河北之地可呼吸而有。然後遣王詵外連突厥,繇太原南趨蒲、絳,大王整駕西入關,兩軍合勢,不旬月天下定矣。」瑗從之,以內外兵悉付君廓。利涉以君廓多翻覆,請以兵屬詵,瑗猶豫,君廓密知之,馳斬詵首,徇於軍曰:「李瑗與王詵反,錮敕使,擅追兵,今詵已斬,獨瑗在,無能為也。諸君從之且族滅,助我者富貴可得!」眾曰:「願討賊。」乃出敦禮於獄。瑗聞之,率左右數百被甲出。君廓呼曰:「瑗誖亂,諸君皆詿誤,若何從之以取夷戮?」眾反走。瑗駕君廓曰:「小人賣我,行自及!」即禽瑗縊之,傳首京師,廢為庶人,絕屬籍。



 鄭孝王亮,仕隋為海州刺史,追王。生子神通、神符。



 淮安靖王神通,少輕俠。隋大業末在長安。會高祖兵興,吏逮捕,亡命入南山,與豪英史萬寶、裴勣、柳崇禮等舉兵應太原,約司竹賊帥何潘仁連和,進與平陽公主兵合,徇鄠下之。自署關中道行軍總管,以萬寶為副,勣為長史,崇禮為司馬,令狐德棻為記室。從平京師,為宗正卿,典兵宿衛。王永康郡,俄徙淮安。



 武德初,拜山東安撫大使,黃門侍郎崔乾副之,進擊宇文化及於魏。化及敗走聊城,神通追北,賊糧盡願降,神通不肯受,干請納之,神通曰:「師久暴露,今賊食盡,克不旦暮,正當破之,以玉帛酬戰力。若降,吾何所藉手?」乾曰:「竇建德危至,而化及未平,我轉側兩賊間,勢必危,王又貪其玉帛,敗不日。」神通怒,囚乾軍中。



 會士及自濟北饋軍,化及復振。神通進兵薄其壘,貝州刺史趙君德先登扳堞,神通忌其功,止軍不進。君德怒,詬而還,城復堅。神通遣兵走魏州取攻具,為莘人所乘,引卻。後二日,建德拔聊城,勢遂張,山東州縣靡然歸之。神通麾下多亡,乃退保黎陽,依李世勣,俄為建德所虜。後與同安公主自賊歸。及建德滅,復授河北行臺左僕射。從平劉黑闥,遷左武衛大將軍。薨,贈司空。



 神通十一子,得王者七人,道彥、孝詧、孝同、孝慈、孝友、孝節、孝義,後皆降王。孝逸爵公。孝銳不得封,有子齊物顯。



 膠東郡王道彥,幼孝謹。初,神通避吏於鄠,被疾山谷間,累旬食盡,道彥羸服丐人間,或採野實以進;神通未食,不敢先,即有所分,辭以飽,乃藏棄以待。高祖初,封義興郡公,例得王。貞觀初,為相州都督,徙岷州,以父喪解。荷土就墳,躬蒔松柏,偃廬柴毀,雖親友不復識。太宗嗟嘆,敕侍中王珪臨諭。



 服除,復拜岷州都督。間遣入黨項諭國威靈,區落降。從李靖擊吐谷渾,詔道彥為赤水道總管。帝厚以利啖黨項,使為鄉導,其酋拓拔赤辭詣靖自言:「隋擊吐谷渾,我資其軍,而隋無信,反見仇剽。今將軍若無它,我願資糧,將復如隋乎?」諸將與歃血遣之。道彥至闊水,見無備,因掠其牛羊,諸羌怨,即引兵障野狐峽,道彥不得進,為赤辭所乘,軍大敗,死者數萬,退保松州。詔減死,謫戍邊。久之,召為媯州都督。卒,贈禮部尚書。



 初,武德五年同封者,孝詧為高密王,孝同淄川王,孝慈廣平王,孝友河間王,孝節清河王,孝義膠西王。於是唐始興,務廣支蕃鎮天下,故從昆弟子自勝衣以上,皆爵郡王。太宗即位,舉屬籍問大臣曰:「蓋王宗子於天下,可乎?」封德彞曰:「漢所封,惟帝子若親昆弟;其屬遠,非大功不王。如周郇滕、漢賈澤尚不得茆土,所以別親疏也。先朝一切封之,爵命崇而力役多,以天下為私奉,非所以示至公。」帝曰:「朕君天下以安百姓,不容勞百姓以養己之親。」於是疏屬王者皆降為公,唯嘗有功者不降。故道彥等並降封公。



 孝逸,少好學,頗屬文。始封梁郡公。高宗時,四遷益州大都督府長史。武后擅國,入為左衛將軍,親遇之。



 徐敬業稱兵,以孝逸為左玉鈐衛大將軍、揚州行軍大總管,帥師南討。至淮,而敬業已攻潤州,遣弟敬猷壁淮陰,偽將韋超據都梁山以拒孝逸,超眾憑險完屯。孝逸會諸將議曰:「賊今負山,攻則士無所用力,騎不得騁,寇救死,傷夷必眾。不如偏旅綴之,全軍趨揚州,勢不數日可破。」支度使薛克構曰:「超雖據險,然兵少,若置小敵不擊,無以示威;披眾以守,則戰有闕。舍之則後憂,不如擊之。若克超,淮陰自震,淮陰破,楚諸縣開門候官軍矣。由是以趨江都,逆首可取。」孝逸從之,登山急擊超,殺數百人,薄暝解,超夜走。進擊敬猷淮陰,破之。敬業回軍下阿溪,孝逸引兵直度,敬業大敗,遂拔揚州。以功進鎮軍大將軍,徙封吳國公,威名稜然。



 武承嗣等忌之,以讒下遷施州刺史。又使人騰惡語聞上,武后信之,以嘗有功,貸死,流儋州,薨。景雲初,贈金州大都督。



 孝同曾孫國貞。



 國貞父廣業,為劍州長史。國貞剛鯁,有吏才。乾元中,由長安令遷河南尹。史思明寇東都,李光弼壁河陽,國貞率官吏西走陜,數月,召為京兆尹。



 上元初,拜劍南節度使,召為殿中監,以戶部尚書持節朔方、鎮西、北庭、興平、陳鄭節度行營兵馬及河中節度都統處置使,治於絳。尋加晉、絳、慈、隰、沁等州觀察處置使。既至,糧乏,而所儲陳腐,民貧不忍遽斂,上書以聞。而軍中言雚謗,突將王振乘眾怨紿曰:「具畚鍤以待役事。」眾皆怒,夜燒牙門。左右奔告,請避之,國貞曰:「吾被命為將,其可棄城乎?」固請,乃逃獄中。振引眾劫取之,置食其前曰:「食是而役其力,可乎?」國貞曰:「與爾等方討賊,何事役為?正緣儲食腐儉,已請諸朝,吾何所負?」眾服其言,且引去。振曰:「都統不死,吾曹殆矣!」遂害之,並殺其二子及三大將。



 有詔以郭子儀代之。國貞清白善用法,世稱辦吏,然峻於操下,故其眾思得子儀,而振因肆其惡。及子儀至,振自謂且見德,子儀怒曰:「汝臨賊境而害主將,賊若乘虛,是無絳矣,又欲為功乎?」即斬以徇。詔贈國貞揚州大都督。



 子錡,自有傳。



 孝節曾孫暠,少孤,事母孝。始為枝江丞,荊州長史張柬之曰:「帝宗千里駒,吾得其人!」累擢衛尉少卿。居母喪,柴瘠,訖除,家人未嘗見言笑。與兄昇、弟暈相友。



 開元初,為汝州刺史,政嚴簡,有治稱。昆弟繇東都候之,輒羸服往,州人無知者,其清慎舉如此。四遷至黃門侍郎,檢校太原以北諸軍節度使。太原俗為浮屠法者,死不葬,以尸棄郊飼鳥獸,號其地曰「黃坑」。有狗數百頭,習食胔,頗為人患,吏不敢禁。暠至,遣捕群狗殺之,申厲禁條,約不再犯,遂革其風。二十一年,以工部尚書持節使吐蕃,既還,金城公主請明疆場,表石赤嶺上,盟遂堅定。還,以奉使有指,再遷吏部。



 暠,美風儀,以莊重稱,當時有宰相望。累為太子少傅、武都縣伯。卒,贈益州大都督。



 暈至太僕少卿。暈子進亦知名,好從當世賢士游,賙人之急,累擢給事中。至德初,從廣平王東征,以工部侍郎署雍王元帥府行軍司馬,為回紇鞭之幾死。遷兵部。卒,贈禮部尚書。



 孝節四世孫說,字巖甫。父遇及,天寶時為御史中丞、東畿採訪使。說以廕補率府兵曹參軍。馬燧節度太原,闢署少尹,遷汾州刺史。李自良代燧,復奏為少尹。大將張瑤得士心,嘗請告未許,而自良卒,說與監軍王定遠秘其喪,前給瑤告,以毛朝陽代之,然後告喪。詔以通王為節度大使,授說行軍司馬、節度留後。



 定遠自以有勞於說,頗橫恣,請別賜印,監軍有印自定遠始。於是擅補吏,易置諸將。彭令茵者,以久勞不服,定遠怒,殺之,埋馬矢中,其家請尸,不許,舉軍怨。說上其事,德宗以奉天扈從功,恕死免官。詔未至,定遠謀刺說,說走而免。定遠召諸將,出笥中詔書紿曰:「詔以李景略知留後,召說還。公等皆有除。」諸將欲拜,大將馬良輔呼曰:「妄言也,不可受!」定遠懼,走乾陽樓,召麾下皆不至,自投下死。說盡斬同謀者,乃安。擢說檢校禮部尚書、節度使。累封隴西縣男。



 說精於職,築天成軍,邊備積完。晚被疾,不能事,軍幾亂。卒,贈尚書右僕射。



 齊物字道用。天寶初,擢累陜州刺史。開砥柱,通漕路,發重石,下得古鐵戟若鏵然,銘曰「平陸」。上之,詔因以名縣。遷河南尹,坐與李適之善,貶竟陵太守,還,還京兆尹,太子太傅,兼宗正卿。卒,贈太子太師。性苛察少恩,喜發人私,然潔廉自喜,吏無敢欺者。忿陜尉裴冕,械而折愧之,及冕當國,除齊物太子賓客,世善冕能損怨云。



 子復。



 復字初陽,以廕仕,累為江陵司錄參軍。衛伯玉才之,表江陵令。遷少尹,厲饒、蘇二州刺史。李希烈叛,荊南節度使張伯儀數為賊窘,朝廷以復在江陵得士心,即母喪奪為少尹,充行軍司馬,佐伯儀。會伯儀罷,改容州刺史,兼本管招討使。先是,西原亂,吏獲反者沒為奴婢,長役之。復至,使訪親戚,一皆原縱。在容三年,人以賴安。轉嶺南節度使,時安南經略使高正平、張應繼卒,其佐李元度、胡懷義等因阻兵脅州縣,肆為奸贓。復至,誘懷義杖死,流元度,南裔肅然。教民作陶瓦,鐫諭蠻獠,收瓊州,置都督府,以綏定其人。召拜宗正卿。歷華州刺史。貞元十年鄭滑節度使李融卒,軍亂,以復檢校兵部尚書代融節度。復下令墾營田以稟其軍,而賦不及民,眾悅。加檢校尚書右僕射。卒,年五十九,贈司空,謚曰昭。復更方鎮,所在稱治,然頗嗜財,為世所譏。



 從父若水,為左金吾大將軍,兼通事舍人,容貌瑰偉,在朝三十年,多識舊儀,每宣勞揖贊,進止閑華,有可觀者。



 襄邑恭王神符字神符,少孤,事兄謹。高祖兵興,神符留長安,為衛文昇所囚。京師平,封安吉郡公。帝受禪,例王。遷並州總管。



 頡利可汗盜邊,神符與戰汾東,斬級五百,獲馬二千。又戰沙河,獲乙利達官,得可汗所乘馬及鎧。召為太府卿。遷揚州大都督,自丹楊度江,治隋江都故郡,揚人利之。然少威嚴,不為下所畏。累擢宗正卿,以足不良改光祿大夫,歸第,月給羊酒。太宗就第尉問,又令乘小輿入紫微殿,三衛挾輿以升。遷開府儀同三司。永徽二年薨,年七十三,贈司空、荊州都督,陪葬獻陵。



 子七人,並爵郡王,例降公。惟德懋、文暕知名。德懋,官少府監、臨川郡公。五世孫從晦。文暕,幽州都督、魏國公。垂拱中,坐累貶藤州別駕,誅。子挺、捷,捷襲封。挺曾孫程,捷曾孫石,別傳。



 從晦祖模,仕至德中為猗氏令。史思明陷洛陽,賊帥掠諸縣,模率眾拒平之。稍遷黔中觀察使。終太子賓客,贈太子太保,謚曰敬。



 從晦寶歷初及進士第,擢累太常博士。甘露之禍,御史中丞李孝本被誅,從晦以族昆弟貶郎州司戶參軍。改澶王府諮議,分司東都。忌者重發前坐,下遷亳州司馬。久乃轉吏部朗中,兼侍御史,知雜事。出為常州刺史,鎮海軍節度使。李琢表其政,賜金紫。歷京兆尹、工部侍郎、山南西道節度使。又以最就進銀青光祿大夫。卒,年六十三,贈吏部尚書。



 從晦姿質偉岸,所至以風力聞。少與崔龜從、李景讓、裴休善。獎目後進,名知人,楊收方布衣,進謁,從晦一見如雅識,即待以公輔,後果宰相。



 世祖四子:長曰澄,次湛,次洪,次高祖神堯皇帝。



 梁王澄,蚤薨,無嗣。武德初,與二王同追封。



 蜀王湛,生子博義、奉慈。



 隴西恭王博義,武德初,與奉慈例王。高宗時,擢累禮部尚書,特進。驕侈不循法度,伎妾數百,曳羅紈,甘粱肉,放於聲樂以自娛。其弟奉慈亦荒縱,皆為帝所鄙。嘗曰:「吾仇人有善且用之,況親戚乎?王等暱小人,專為不軌,先王墳典不聞學,何以為善哉?」各賜市書絹二百疋,以愧切之,然不自克也。薨,贈開府儀同三司、荊州都督。



 渤海敬王奉慈,顯慶時,為原州都督,薨。



 七世孫戡。



 戡字定臣,幼孤。年十歲所即好學,大寒,掇薪自炙。夜無然膏,默念所記。年三十,明《六經》,舉進士,就禮部試,吏唱名乃入,戡恥之。明日,徑返江東,隱陽羨裏。陽羨民有鬥爭不決,不之官而詣戡以辨。凡論著數百篇。常惡元和有元、白詩,多纖艷不逞,而世競重之。乃集詩人之類夫古者,斷為唐詩,以譏正其失雲。平盧節度使王彥威表為巡官,府遷,還洛陽,卒。



 贊曰:景、元子孫,當草昧之初,乘運而奮,方高祖攘除四方,所以宣力,皆顯顯為世豪英。至河間之功,江夏之略,可謂宗室標的者也。



 始,唐興,疏屬畢王,至太宗,稍稍降封。時天下已定,帝與名臣蕭瑀等喟然講封建事,欲與三代比隆,而魏徵、李百藥皆謂不然。征意以唐承大亂,民人雕喪,始復生業,遽起而瓜分之,故有五不可之說。百藥稱帝王自有命,歷祚之短長不緣封建。又舉春秋二百四十二年之禍,亟於哀、平、桓、靈,而詆曹元首、陸士衡之言以為繆悠。而顏師古獨議建諸侯,當少其力,與州縣雜治,以相維持。然天子由是罷不復議。



 至名儒劉秩目武氏之禍,則建論以為設爵無土,署官不職,非古之道,故權移外家,宗廟絕而更存。存之之理,在取順而難逆;絕之之原,在單弱而無所憚。至謂郡縣可以小寧,不可以久安。大抵與曹、陸相上下。而杜佑、柳宗元深探其本,據古驗今,而反復焉。



 佑之言曰:「夫為人置君,欲其蕃息則在郡縣,然而主胙常促;為君置人,不病其寡則在建國,然而主胙常永。故曰,建國利一宗,列郡利百姓。且立法未有不敝者,聖人在度其患之長短而為之。建國之制,初若磐石,然敝則鼎峙力爭,陵遲而後已,故為患也長。列郡之制,始天下一軌,敝則世崩俱潰,然而戡定者易為功,故其為患也短。」又謂:「三王以來,未見郡縣之利,非不為也,後世諸儒因泥古強為之說,非也。」



 宗元曰:「封建非聖人意,然而歷堯、舜、三王莫能去之,非不欲去之,勢不可也。秦破六國,列都會,置守宰,據天下之圖,攝制四海,此其得也。二世而亡,有由矣。暴威刑,竭人力,天下相合,劫令殺守,圜視而並起,時則有叛民,無叛吏。漢矯秦枉,剖海內,立宗子功臣,數十年間奔命扶傷不給,時則有叛國,無叛郡。唐興,制州縣,而桀黠時起,失不在州而在於兵,時則有叛將,無叛州。」以為「矯而革之,垂二百年,不在諸侯明矣」。又言「湯之興,諸侯歸者三千,資以勝夏;武王之興,會者八百,資以滅商。徇之為安,故仍以為俗,是湯、武之不得已也。不得已,非公之大也,私其力於己也。秦革之者,其為制,公之大者也;其情,私也。然而公天下之端自秦始」云。



 觀諸儒之言,誠然。然建侯置守,如質文遞救,亦不可一概責也。救土崩之難,莫如建諸侯;削尾大之勢,莫如置守宰。唐有鎮帥,古諸侯比也。故王者視所救為之,勿及於敝則善矣。若乃百藥推天命、佑言郡縣利百姓而主胙促,乃臆論也。



\end{pinyinscope}