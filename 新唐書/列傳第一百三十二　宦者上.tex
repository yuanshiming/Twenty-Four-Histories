\article{列傳第一百三十二 宦者上}

\begin{pinyinscope}

 唐制:內侍省官有內侍四,內常侍六,內謁者監、內給事各十,謁者十二,典引十八培爾(PierreBayle,1647—1706)法國哲學家,早期啟,寺伯、寺人各六。又有五局:一曰掖廷,主宮嬪簿最;二曰宮闈,扈門闌;三曰奚官,治宮中疾病死喪;四曰內僕,主供帳燈燭;五曰內府,主中藏給納。局有令,有丞,皆宦者為之。



 太宗詔內侍省不立三品官,以內侍為之長,階第四,不任以事,惟門閣守御、廷內掃除、稟食而已。武后時,稍增其人。至中宗,黃衣乃二千員,七品以上員外置千員,然衣硃紫者尚少。玄宗承平,財用富足,志大事奢,不愛惜賞賜爵位。開元、天寶中,宮嬪大率至四萬,宦官黃衣以上三千員,衣硃紫千餘人。其稱旨者輒拜三品將軍,列戟於門。其在殿頭供奉,委任華重,持節傳命,光焰殷殷動四方。所至郡縣奔走,獻遺至萬計。修功德,市禽鳥,一為之使,猶且數千緡。監軍持權,節度返出其下。於是甲舍、名園、上腴之田為中人所名者半京畿矣。肅、代庸弱,倚為捍衛,故輔國以尚父顯,元振以援立奮,朝恩以軍容重,然猶未得常主兵也。德宗懲艾泚賊,故以左右神策、天威等軍委宦者主之,置護軍中尉、中護軍,分提禁兵,是以威柄下遷,政在宦人,舉手伸縮,便有輕重。至慓士奇材,則養以為子;巨鎮強籓,則爭出我門。



 小人之情,猥險無顧藉,又日夕侍天子,狎則無威,習則不疑,故昏君蔽於所暱,英主禍生所忽。玄宗以遷崩,憲、敬以弒殞,文以憂僨,至昭而天下亡矣。禍始開元,極於天祐,兇愎參會,黨類殲滅,王室從而潰喪,譬猶灼火攻蠹,蠹盡木焚,詎不哀哉!跡其殘氣不剛,柔情易遷,褻則無上,怖則生怨,借之權則專,為禍則迫而近,緩相攻,急相一,此小人常勢也。噫!梟狐不神,天與之昏,末如亂何。故取中葉以來宦人之大者稡之篇。



 楊思勖,羅州石城人,本蘇氏,冒所養姓。少給事內侍省,從玄宗討內難,擢左監門衛將軍,帝倚為爪牙。開元初,安南蠻渠梅叔鸞叛,號黑帝,舉三十二州之眾,外結林邑、真臘、金鄰等國,據海南,眾號四十萬。思勖請行,詔募首領子弟十萬,與安南大都護光楚客繇馬援故道出不意,賊驢眙不暇謀,遂大敗,封尸為京觀而還。十二年,五溪首領覃行章亂,詔思勖為黔中招討使,率兵六萬往,執行章,斬首三萬級,以功進輔國大將軍,給祿俸、防閣。從封泰山,進驃騎大將軍,封虢國公。邕州封陵獠梁大海反,破賓、橫等州,思勖又平之,禽大海等三千人,討斬支黨皆盡。瀧州蠻陳行範自稱天子,其下何游魯號定國大將軍,馮璘南越王,破州縣四十。詔思勖發永、道、連三州兵,淮南弩士十萬,襲斬游魯、璘於陣。行範走盤遼諸洞,思勖悉眾窮追,生縛之,坑其黨六萬,獲馬金銀鉅萬計。卒,年八十餘。



 思勖鷙忍,敢殺戮,所得俘,必剝面、皦腦、褫發皮以示人,將士憚服,莫敢視,以是能立功。內給事牛仙童納張守珪賂,詔付思勖殺之。思勖縛於格,箠慘不可勝,乃探心,截手足,剔肉以食,肉盡乃得死。



 楚客者,樂安人,後歷桂州都督致仕,封松滋縣侯。



 高力士,馮盎曾孫也。聖歷初,嶺南討擊使李千里上二閹兒,曰金剛,曰力士,武后以其強悟,敕給事左右。坐累逐出之,中人高延福養為子,故冒其姓。善武三思,歲餘,復得入禁中,稟食司宮臺。既壯,長六尺五寸,謹密,善傳詔令,為宮闈丞。



 玄宗在籓,力士傾心附結,已平韋氏,乃啟屬內坊,擢內給事。先天中,以誅蕭、岑等功為右監門衛將軍,知內侍省事。於是四方奏請皆先省後進,小事即專決,雖洗沐未嘗出,眠息殿帷中,徼幸者願一見如天人然。帝曰:「力士當上,我寢乃安。」當是時,宇文融、李林甫、蓋嘉運、韋堅、楊慎矜、王鉷、楊國忠、安祿山、安思順、高仙芝等雖以才寵進,然皆厚結力士,故能踵至將相,自餘承風附會不可計,皆得所欲。中人若黎敬仁、林昭隱、尹鳳翔、韓莊、牛仙童、劉奉廷、王承恩、張道斌、李大宜、硃光輝、郭全、邊令誠等,並內供奉,或外監節度軍,脩功德,市鳥獸,皆為之使。使還,所裒獲,動巨萬計,京師甲第池園、良田美產,占者什六,寵與力士略等,然悉藉力士左右輕重乃能然。肅宗在東宮,兄事力士,它王、公主呼為翁,戚里諸家尊曰爹,帝或不名而呼將軍。



 力士幼與母麥相失,後嶺南節度使得之瀧州,迎還,不復記識,母曰:「胸有七黑子在否?」力士袒示之,如言。母出金環,曰「兒所服者」,乃相持號慟。帝為封越國夫人,而追贈其父廣州大都督。延福與妻,及力士貴時故在,侍養與麥均。金吾大將軍程伯獻約力士為兄弟,後麥亡,伯獻糸裒絰受吊。河間男子呂玄晤吏京師,女國姝,力士娶之,玄晤擢刀筆史至少卿,子弟仕皆王傅。玄晤妻死,中外贈賻送葬,自第至墓,車徒背相望不絕。



 始,李林甫、牛仙客知帝憚幸東都,而京師漕不給,乃以賦粟助漕,及用和糴法,數年,國用稍充。帝齋大同殿,力士侍,帝曰:「我不出長安且十年,海內無事,朕將吐納導引,以天下事付林甫,若何?」力士對曰:「天子順動,古制也。稅入有常,則人不告勞。今賦粟充漕,臣恐國無旬月蓄;和糴不止,則私藏竭,逐末者眾。又天下柄不可假人,威權既振,孰敢議者!」帝不悅,力士頓首自陳「心狂易,語謬當死」。帝為置酒,左右呼萬歲。由是還內宅,不復事。加累驃騎大將軍,封渤海郡公。於來廷坊建佛祠,興寧坊立道士祠,珍樓寶屋,國貲所不逮。鍾成,力士宴公卿,一扣鍾,納禮錢十萬,有佞悅者至二十扣,其少亦不減十。都北堰澧列五磑,日僦三百斛直。



 有袁思藝者,帝亦愛幸,然驕倨甚,士大夫疏畏之,而力士陰巧得人譽。帝初置內侍省監二員,秩三品,以力士、思藝為之。帝幸蜀,思藝遂臣賊,而力士從帝,進齊國公。帝聞肅宗即位,喜曰:「吾兒應天順人,改元至德,不忘孝乎,尚何憂?」力士曰:「兩京失守,生人流亡,河南漢北為戰區,天下痛心,而陛下以為何憂,臣不敢聞。」從上皇還,進開府儀同三司,實封戶五百。



 上皇徙西內,居十日,為李輔國所誣,除籍,長流巫州。力士方逃瘧功巨閣下,輔國以詔召,力士趨至閣外,遣內養授謫制,因曰:「巨當死已久,天子哀憐至今日,願一見陛下顏色,死不恨。」輔國不許。寶應元年赦還,見二帝遺詔,北向哭歐血,曰:「大行升遐,不得攀梓宮,死有餘恨。」慟而卒,年七十九。代宗以護衛先帝勞,還其官,贈揚州大都督,陪葬泰陵。



 初,太子瑛廢,武惠妃方嬖,李林甫等皆屬壽王,帝以肅宗長,意未決,居忽忽不食。力士曰:「大家不食,亦膳羞不具耶?」帝曰:「爾,我家老,揣我何為而然?」力士曰:「嗣君未定耶?推長而立,孰敢爭?」帝曰:「爾言是也。」儲位遂定。天寶中,邊將爭立功,帝嘗曰:「朕春秋高,朝廷細務付宰相,蕃夷不龔付諸將,寧不暇耶?」對曰:「臣間至閣門,見奏事者言雲南數喪師,又北兵悍且強,陛下何以制之?臣恐禍成不可禁。」其指蓋謂祿山。帝曰:「卿勿言,朕將圖之。」十三年秋,大雨,帝顧左右無人,即曰:「天方災,卿宜言之。」力士曰:「自陛下以權假宰相,法令不行,陰陽失度,天下事庸可復安?臣之鉗口,其時也。」帝不答。明年祿山反。力士善揣時事勢候相上下,雖親暱,至當覆敗,不肯為救力,故生平無顯顯大過。議者頗恨宇文融以來權利相賊,階天下之禍,雖有補益,弗相除云。



 程元振,京兆三原人。少以宦人直內侍省,遷內射生使、飛龍廄副使。張皇后謀立越王,元振見太子,發其奸,與李輔國助討難,立太子,是為代宗。拜右監門衛將軍,知內侍省事。帝以藥子昂判元帥行軍司馬,固辭,乃以命元振,封保定縣侯。再遷驃騎大將軍、邠國公,盡總禁兵。不逾歲,權震天下,在輔國右,兇決又過之,軍中呼十郎。



 王仲升者,初為淮西節度使,與襄州張維瑾部將戰申州,被執。賊平,元振薦為右羽林大將軍兼御史大夫。將軍兼大夫由仲升始。裴冕與元振忤,乃掎韓穎等罪貶施州。來瑱守襄、漢有功,元振嘗諉屬,不應,因仲升共誣殺瑱。同華節度使李懷讓被構,憂甚自殺。素惡李光弼,數媒蠍以疑之。瑱等上將,冕、光弼元勛,既誅斥,或不自省,方帥繇是攜解。



 廣德初,吐蕃、黨項內侵,詔集天下兵,無一士奔命者。虜扣便橋,帝倉黃出居陜,京師陷,賊剽府庫,焚閭弄,蕭然為空。於是太常博士、翰林待詔柳伉上疏曰:「犬戎以數萬眾犯關度隴,歷秦、渭,掠邠、涇,不血刃而入京師,謀臣不奮一言,武士不力一戰,提卒叫呼,劫宮闈,焚陵寢,此將帥叛陛下也;自朝義之滅,陛下以為智力所能,故疏元功,委近習,日引月長以成大禍,群臣在廷無一犯顏回慮者,此公卿叛陛下也;陛下始出都,百姓填然奪府庫,相殺戮,此三輔叛陛下也;自十月朔召諸道兵,盡四十日,無只輪入關者,此四方叛陛下也。內外離叛,雖一魚朝恩以陜郡戮力,陛下獨能以此守社稷乎?陛下以今日勢為安耶?危耶?若以為危,豈得高枕不為天下計?臣聞良醫療疾,當病飲藥,藥不當疾,猶無益也。陛下視今日病何繇至此乎?天下之心,乃恨陛下遠賢良,任宦豎,離間將相而幾於亡。必欲存宗廟社稷,獨斬元振首,馳告天下,悉出內使隸諸州,獨留朝恩備左右,陛下持神策兵付大臣,然後削尊號,下詔引咎,率德勵行,屏嬪妃,任將相。若曰『天下其許朕自新改過乎,宜即募士西與朝廷會;若以朕惡未悛耶,則帝王大器,敢妨聖賢,其聽天下所往。』如此而兵不至,人不感,天下不服,請赤臣族以謝。」疏聞,帝顧公議不與,乃下詔盡削元振官爵,放歸田里。帝還,元振自三原衣婦衣私入京師,舍司農卿陳景詮家,圖不軌。御史劾按,長流水奏州,景詮貶新興尉。元振行至江陵死。



 時又有駱奉先者,亦三原人,歷右驍衛大將軍,數從帝討伐,尤見幸,廣德初,監僕固懷恩軍者。奉先恃恩貪甚,懷恩不平,既而懼其譖,遂叛。事平,擢奉先軍容使,掌畿內兵,權焰熾然。永泰初,以吐蕃數驚京師,始城鄠,以奉先為使,悉毀縣外廬舍,無尺椽。累封江國公,監鳳翔軍,大歷末卒。



 魚朝恩,瀘州瀘川人。天寶末,以品官給事黃門,內陰黠,善宣納詔令。至德初,監李光進軍。京師平,為三宮檢責使,以左監門衛將軍知內侍省事。九節度圍賊相州,以朝恩為觀軍容、宣慰、處置使。觀軍容使自朝恩始。史思明攻洛陽,朝恩以神策兵屯陜。洛陽陷,思明長驅至硤石,使子朝義為游軍。肅宗詔銳兵十萬循渭而東以濟師。朝恩按兵陜東,使神策將衛伯玉與賊將康文景等戰,敗之。洛陽平,徙屯汴州,加開府儀同三司,封馮翊郡公。寶應中,還屯陜。代宗避吐蕃東幸,衛兵離散,朝恩悉軍奉迎華陰,乘輿六師乃振,帝德之,更號天下觀軍容、宣慰、處置使,專領神策軍,賞賜不涯。



 朝恩資小人,恃功岸忽無所憚。僕固瑒攻絳州,使姚良據溫,誘回紇陷河陽。朝恩遣李忠誠討瑒,以霍文場監之;王景岑討良,王希遷監之。敗瑒於萬泉,生擒良。高暉等引吐蕃入寇,遣劉德信討斬之。故朝恩因麾下數克獲,竊以自高。是時郭子儀有定天下功,居人臣第一,心媢之,乘相州敗,醜為詆譖,肅宗不內其語,然猶罷子信兵,留京師。代宗立,與程元振一口加毀,帝未及寤,子儀憂甚。俄而吐蕃陷京師,卒用其力,王室再安。故朝恩內慚,乃勸帝徙洛陽,欲遠戎狄。百僚在廷,朝恩從十餘人持兵出,曰:「虜數犯都甸,欲幸洛,云何?」宰相未對,有近臣折曰:「敕使反耶?今屯兵足以捍寇,何遽脅天子棄宗廟為?」朝恩色沮,而子儀亦謂不可,乃止。



 朝恩好引輕浮後生處門下,講《五經》大義,作文章,謂才兼文武,徼伺誤寵。



 永泰中,詔判國子監,兼鴻臚、禮賓、內飛龍、閑廄使,封鄭國公。始詣學,詔宰相、常參官、六軍將軍悉集,京兆設食,內教坊出音樂俳倡侑宴,大臣子弟二百人,硃紫雜然為附學生,列廡次。又賜錢千萬,取子錢供秩飯。每視學,從神策兵數百,京兆尹黎幹率錢勞從者,一費數十萬,而朝恩色常不足。



 凡詔會群臣計事,朝恩怙貴,誕辭折愧坐人出其上,雖元載辯強亦拱默,唯禮部郎中相里造、殿中侍御史李衎酬詰往返,未始降屈,朝恩不懌,黜衎以動造。又謀將易執政以震朝廷,乃會百官都堂,且言:「宰相者,和元氣,輯群生。今水旱不時,屯軍數十萬,饋運困竭,天子臥不安席,宰相何以輔之?不退避賢路,默默尚何賴乎?」宰相俯首,坐皆失色。造徙坐從之,因曰:「陰陽不和,五穀踴貴,皆軍容事,宰相何與哉?且軍挐不散,故天降之沴。今京師無事,六軍可相維鎮,又屯十萬,饋糧所以不足,百司無稍食,軍容為之,宰相行文書而已,何所歸罪?」朝恩拂衣去,曰:「南衙朋黨,且害我。」會釋菜,執《易》升坐,百官咸在,言《鼎》有覆餗象,以侵宰相。王縉怒,元載怡然。朝恩曰:「怒者常情,笑者不可測也。」載銜之未發。



 朝恩有賜墅,觀沼勝爽,表為佛祠,為章敬太后薦福,即後謚以名祠,許之。於是用度侈浩,公壞曲江諸館、華清宮樓榭、百司行署、將相故第,收其材佐興作,費無慮萬億。既數毀郭子儀,不見聽,乃遣盜發其先塚,子儀詭辭自解,以安眾疑。久之,讓判國子監、鴻臚禮賓等使,加內侍監,徙封韓,增實封百戶。俄兼檢校國子監。



 初,神策都虞候劉希暹魁健能騎射,最為朝恩暱信,以太僕卿封交河郡王。兵馬使王駕鶴獨謹厚,亦封徐國公。希暹諷朝恩置獄北軍,陰縱惡少年橫捕富人付吏考訊,因中以法,錄貲產入之軍,皆誣服冤死,故市人號「入地牢」。又萬年吏賈明觀倚朝恩捕搏恣行,積財鉅萬,人無敢發其奸。朝廷裁決,朝恩或不預者,輒怒曰:「天下事有不由我乎!」帝聞,不喜。養息令徽者,尚幼,為內給使,服綠,與同列爭忿,歸白朝恩。明日見帝曰:「臣之子位下,願得金紫,在班列上。」帝未答,有司已奉紫服於前,令徽稱謝。帝笑曰:「小兒章服,大稱。」滋不悅。



 元載乃用左散騎常侍崔昭尹京兆,厚以財結其黨皇甫溫、周皓。溫方屯陜,而皓射生將。自是朝恩隱謀奧語,悉為帝知。希暹覺帝指,密白朝恩,朝恩稍懼,然見帝接遇未衰,故自安而潛計不軌。帝遂倚載決除之,懼不克,載曰:「陛下第專屬臣,必濟。」朝恩入殿,嘗從武士百人自衛,皓統之,而溫握兵在外。載乃徙鳳翔尹李抱玉節度山南西道,以溫代節度鳳翔,陽重其權,寔內溫以自助。載又議析鳳翔之郿與京兆,以鄠、盩厔及鳳翔之虢、寶雞與抱玉,而以興平、武功、鳳翔之扶風天興與神策軍,朝恩利其土地,自封殖,不知為虞也。郭子儀密白:「朝恩嘗結周智光為外應,久領內兵,不早圖,變且大。」載留溫京師,未即遣,約與皓共誅朝恩。謀定,以聞,帝曰:「善圖之,勿反受禍!」方寒食,宴禁中,既罷,將還營,有詔留議事。朝恩素肥,每乘小車入宮省。帝聞車聲,危坐,載守中書省。朝恩至,帝責其異圖,朝恩自辯悖傲,皓與左右禽縊之,死年四十九,外無知者。帝隱之,下詔罷觀軍容等使,增實封戶六百,內侍監如故。外咸言「既奉詔,乃投縊」云。還尸於家,賜錢六百萬以葬。



 帝懼軍亂,進劉希暹、王駕鶴並兼御史中丞。又下詔尉曉將士,獨希暹自知同惡,言不遜,駕鶴白發之,遂賜死。而賈明觀兼得幸於載,故載奏隸江西,使立功自贖,路嗣恭搒殺之。所厚禮部尚書、禮儀使裴土淹、戶部侍郎判度支第五琦皆坐貶。



 竇文場、霍仙鳴者,始並隸東宮,事德宗,未有名。自魚朝恩死,宦人不復典兵,帝以禁衛盡委白志貞,志貞多納富人金補軍,止收其庸而身不在軍。及涇師亂,帝召近衛,無一人至者,惟文場等率宦官及親王左右從。至奉天,帝逐志貞,並左右軍付文場主之。興元初,詔監神策左廂兵馬,以王希遷監右,而馬有麟為左神策軍大將軍,軍額由此始。



 帝自山南還,兩軍復完,而帝忌宿將難制,故詔文場、仙鳴分總之,廢天威軍入左右神策。是時,竇、霍權振朝廷,諸方節度大將多出其軍,臺省要官走門下,丐援影者足相躡。衛士硃華以按摩得幸文場,參慮補置,索賕數萬緡,而籓鎮贈遺累百鉅萬,略士妻女無所憚,詔殺之於軍。其隆赫如此。



 久之,置護軍中尉、中護軍各二員,詔文場為左神策護軍中尉,仙鳴為右,焦希望為左神策中護軍,張尚進為右。中尉、護軍自文場等始。後仙鳴移病,帝賜十馬,令諸祠祈解。後稍愈,已而暴死,帝疑左右進毒,捕詰小使問狀,誅數十人,贈開府儀同三司,以內常侍第五守亮代之。文場擢累驃騎大將軍。時監察御史崔肸行囚於軍,吏為具酒食,肸欲悅媚之,故不拒。文場劾奏,詔流肸遠方。文場年老致仕卒。



 其後楊志廉、孫榮義為左右中尉,招權驕肆,與竇、霍略等。帝晚節聞民間訛語禁中事,而北軍捕太學生何竦、曹壽系訊,人情大懼,司業武少儀上書「有如罪不測,願明示四方」。俄得釋。是時宦官復盛矣。



 希望者,涇陽人,歷明威將軍,贈洪州都督。尚進,河東人,歷忠武將軍,贈開府儀同三司。志廉,弘農人,歷左監門衛大將軍;榮義,涇陽人,歷右武衛大將軍。並贈揚州大都督。



 劉貞亮,本俱氏,名文珍,冒所養宦父,故改焉。性忠強,識義理。平涼之盟,在渾瑊軍中,會虜變,被執且西,俄而得歸。出監宣武軍,自置親兵千人。貞元末,宦人領兵附順者益眾。



 會順宗立,淹痼弗能朝,惟李忠言、牛美人侍。美人以帝旨付忠言,忠言授之王叔文,叔文與柳宗元等裁定,然後下中書。然未得縱欲,遂奪神策兵以自強,即用範希朝為京西北禁軍都將,收宦者權。而忠言素懦謹,每見叔文與論事,無敢異同,唯貞亮乃與之爭。又惡朋黨熾結,因與中人劉光琦、薛文珍、尚衍、解玉、呂如全等同勸帝立廣陵王為太子監國,帝納其奏,貞亮召學士衛次公、鄭絪、李程、王涯至金鑾殿草定制詔。太子已立,盡逐叔文黨,委政大臣,議者美其忠。



 高崇文討劉闢,復為監軍。初,東川節度使李康為闢所破,囚之。崇文至,闢歸康求雪,貞亮劾以不拒賊,斬之,故以專悍見訾。遷累右衛大將軍,知內侍省事。元和八年卒,贈開府儀同三司。



 憲宗之立,貞亮為有功,然終身無所寵假。呂如全歷內侍省內常侍、翰林使,坐擅取樟材治第,送東都獄,至閿鄉自殺。又郭旻醉觸夜禁,杖殺之。五坊硃超晏、王志忠縱鷹人入民家,搒二百,奪職,由是莫不懾畏。



 吐突承璀,字仁貞,閩人也。以黃門直東宮,為掖廷局博士,察察有才。憲宗立,擢累左監門將軍、左神策護軍中尉、左街功德使,封薊國公。



 王承宗叛,承璀揣帝銳征討,因請行。帝見其果敢,自喜,謂可任,即詔承璀為行營招討處置使,以左右神策及河中、河南、浙西、宣歙兵從之。內寺伯宋惟澄、曹進玉為館驛使:自河南、陜、河陽,惟澄主之;京、華、河中至太原,進玉主之。又詔內常侍劉國珍、馬朝江分領易、定、幽、滄等州糧料使。於是諫官李庸阜、許孟容、李元素、李夷簡、呂元膺、穆質、孟簡、獨孤鬱、段平仲、白居易等眾對延英,謂古無中人位大帥,恐為四方笑。帝乃更為招討宣慰使,為御通化門慰其行。承璀御眾無它遠略,為盧從史侮狎,逾年無功,賴中詔擿使執從史,而間遣人說承宗上書待罪,乃詔班師,還為中尉。平仲劾承璀輕謀弊賦,損國威,不斬首無以謝天下。帝不獲已,罷為軍器莊宅使。尋拜左衛上將軍,知內侍省。



 會劉希光納羽林大將軍孫錢二十萬緡求方鎮,有詔賜死,跡絓承璀,故令出監淮南軍。纖人太子通事舍人李涉投匭言承璀等冤狀,於是孔戣知匭事,閱其副,不受,即表其奸,逐為峽州司倉參軍。然帝於承璀殊厚,會李絳在翰林,苦論其過,故決遣之。帝後欲還承璀,為罷絳宰相,召為內弓箭庫使,復左神策中尉。惠昭太子薨,承璀請立澧王,不從。常飾一室藏所賜詔敕,地生毛二尺,惡之,躬糞除瘞之。逾年帝崩,穆宗銜前議,殺之禁中。敬宗時,左神策中尉馬存亮論其冤,詔許子士曄收葬。宣宗時,擢士曄右神策中尉。



 是時,諸道歲進閹兒,號「私白」,閩、嶺最多,後皆任事,當時謂閩為中官區藪。咸通中,杜宣猷為觀察使,每歲時遣吏致祭其先,時號「敕使墓戶」。宣猷卒用群宦力徙宣歙觀察使。



 馬存亮,字季明,河中人。元和時,累擢左神策軍副使、左監門衛將軍,知內侍省事,進左神策中尉。軍所籍凡十餘萬,存亮料柬尤精,伍無罷士,部無冗員。



 敬宗初,染署工張韶與卜者蘇玄明善,玄明曰:「我嘗為子卜,子當御殿食,我與焉。吾聞上晝夜獵,出入無度,可圖也。」韶每輸染材入宮,衛士不呵也。乃陰結諸工百餘人,匿兵車中若輸材者,入右銀臺門,約昏夜為變。有詰其載者,韶謂謀覺,殺其人,出兵大呼成列,浴堂門閉。時帝擊球清思殿,驚,將幸右神策。或曰:「賊入宮,不知眾寡,道遠可虞,不如入左軍,近且速。」從之。初,帝常寵右軍中尉梁守謙,每游幸;兩軍角戲,帝多欲右勝,而左軍以為望。至是,存亮出迎,捧帝足泣,負而入。以五百騎往迎二太后,比至,而賊已斬關入清思殿,升御坐,盜乘輿餘膳,揖玄明偶食,且曰「如占」。玄明驚曰:「止此乎!」韶惡之,悉以寶器賜其徒,攻弓箭庫,仗士拒之,不勝。存亮遣左神策大將軍康藝全、將軍何文哲宋叔夜孟文亮,右神策大將軍康志睦、將軍李泳尚國忠,率騎兵討賊,日暮,射韶及玄明皆死。始賊入,中人倉卒繇望仙門出奔,內外不知行在。遲明,盡捕亂黨,左右軍清宮,車駕還。群臣詣延英門見天子,然至者不十一二,坐賊所入闌不禁者數十人,杖而不誅,賜存亮實封戶二百,梁守謙進開府儀同三司,它論功賞有差。存亮於一時功最高,乃推委權勢,求監淮南軍。代還,為內飛龍使。大和中,以右領軍衛上將軍致仕,封岐國公,卒贈揚州大都督。



 存亮逮事德宗,更六朝,資端畏,善訓士,始去禁衛,眾皆泣。唐世中人以忠謹稱者,唯存亮、西門季玄、嚴遵美三人而已。



 遵美父季寔,為掖廷局博士。大中時,有宮人謀弒宣宗。是夜,季寔直咸寧門下,聞變,入射殺之。明日,帝勞曰:「非爾,吾危不免。」擢北院副使,終內樞密使。



 遵美歷左軍容使,嘗嘆曰:「北司供奉官以胯衫給事,今執笏,過矣。樞密使無聽事,唯三楹舍藏書而已,今堂狀帖黃決事,此楊復恭奪宰相權之失也。」蓋疾時中官肆橫雲。後從昭宗遷鳳翔,求致仕,隱青城山,年八十餘卒。



 仇士良,字匡美,循州興寧人。順宗時得侍東宮。憲宗嗣位,再遷內給事,出監平盧、鳳翔等軍。嘗次敷水驛,與御史元稹爭舍上廳,擊傷稹。中丞王播奏御史、中使以先後至得正寢,請如舊章。帝不直稹,斥其官。元和、大和間,數任內外五坊使,秋按鷹內畿,所至邀吏供餉,暴甚寇盜。



 文宗與李訓欲殺王守澄,以士良素與守澄隙,故擢左神策軍中尉兼左街功德使,使相糜肉。已而訓謀悉逐中官,士良悟其謀,與右神策軍中尉魚弘志、大盈庫使宋守義挾帝還宮。王涯、舒元輿已就縛,士良肆脅辱,令自承反,示牒於朝。於時莫能辨其情,皆謂誠反,士良因縱兵捕,無輕重悉斃兩軍,公卿半空。事平,加特進、右驍衛大將軍,弘志右衛上將軍兼中尉,守義右領軍衛上將軍。



 李石輔政,棱棱有風岸,士良與論議數屈,深忌之,使賊刺石於親仁里,馬逸而免。石懼,辭位,士良益無憚。



 澤潞劉從諫本與訓約誅鄭注。及訓死,憤士良得志,乃上書言:「王涯等八人皆宿儒大臣,願保富貴,何苦而反。今大戮所加已不可追,而名之逆賊,含憤九泉。不然,天下義夫節士,畏禍伏身,誰肯與陛下共治耶?」即以訓所移書遣部將陳季卿以聞。季卿至,會石遇盜,京師擾,疑不敢進。從諫大怒,殺季卿,騰書於朝。又言:「臣與訓誅注,以注本宦豎所提挈,不使聞知。今四方共傳宰相欲除內官,而兩軍中尉聞,自救死,妄相殺戮,謂為反逆。有如大臣挾無將之謀,自宜執付有司,安有縱俘劫、橫尸闕下哉?陛下視不及,聽未聞也。且宦人根黨蔓延在內,臣欲面陳,恐橫遭戮害,謹修封疆,繕甲兵,為陛下腹心。如奸臣難制,誓以死清君側。」書聞,人人傳觀。士良沮恐,即進從諫檢校司徒,欲弭其言。從諫知可動,復言:「臣所陳系國大體,可聽,則宜洗宥涯等罪;不可聽,則賞不宜妄出。安有死冤不申,而生者荷祿?」固辭。累上書,暴指士良等罪。帝雖不能去,然倚其言差自強。自是鬱鬱不樂,兩軍球獵宴會絕矣。



 開成四年,苦風痺,少間,召宰相見延英,退坐思政殿,顧左右曰:「所直學士謂誰?」曰:「周墀也。」召至,帝曰:「自爾所況,朕何如主?」墀再拜曰:「臣不足以知,然天下言陛下堯、舜主也。」帝曰:「所以問,謂與周赧、漢獻孰愈?」墀惶駭曰:「陛下之德,成、康、文、景未足比,何自方二主哉?」帝曰:「赧、獻受制強臣,今朕受制家奴,自以不及遠矣!」因泣下,墀伏地流涕。後不復朝,至大漸雲。



 始,樞密使劉弘逸薛季棱、宰相李玨楊嗣復謀奉太子監國,士良與弘志議更立,玨不從,乃矯詔立潁王為皇太弟,士良以兵奉迎,而太子還為陳王。初,莊恪太子薨,楊賢妃謀引安王,不克。武宗已立,士良發其事,勸帝除之以絕人望,故王、妃皆死。士良遷驃騎大將軍,封楚國公,弘志韓國公,實封戶三百。俄而玨、嗣復罷去,弘逸、季棱誅矣。



 帝明斷,雖士良有援立功,內實嫌之,陽示尊寵。李德裕得君,士良愈恐。會昌二年,上尊號,士良宣言「宰相作赦書,減禁軍縑糧芻菽」以搖怨,語兩軍曰:「審有是,樓前可爭。」德裕以白帝,命使者諭神策軍曰:「赦令自朕意,宰相何豫?爾渠敢是?」士乃怗然。士良惶惑不自安。明年,進觀軍容使,兼統左右軍,以疾辭,罷為內侍監,知省事。固請老,詔可。尋卒,贈揚州大都督。



 士良之老,中人舉送還第,謝曰:「諸君善事天子,能聽老夫語乎?」眾唯唯。士良曰:「天子不可令閑暇,暇必觀書,見儒臣,則又納諫,智深慮遠,減玩好,省游幸,吾屬恩且薄而權輕矣。為諸君計,莫若殖財貨,盛鷹馬,日以球獵聲色蠱其心,極侈靡,使悅不知息,則必斥經術,闍外事,萬機在我,恩澤權力欲焉往哉?」眾再拜。士良殺二王、一妃、四宰相,貪酷二十餘年,亦有術自將,恩禮不衰云。死之明年,有發其家藏兵數千物,詔削官爵,籍其家。



 始,士良、弘志憤文宗與李訓謀,屢欲廢帝。崔慎由為翰林學士,直夜未半,有中使召入,至秘殿,見士良等坐堂上,帷帳周密,謂慎由曰:「上不豫已久,自即位,政令多荒闕,皇太后有制更立嗣君,學士當作詔。」慎由驚曰:「上高明之德在天下,安可輕議?慎由親族中表千人,兄弟群從且三百,何可與覆族事?雖死不承命。」士良等默然,久乃啟後戶,引至小殿,帝在焉。士良等歷階數帝過失,帝俯首。既而士良指帝曰:「不為學士,不得更坐此。」乃送慎由出,戒曰:「毋洩,禍及爾宗。」慎由記其事,藏箱枕間,時人莫知。將沒,以授其子胤,故胤惡中官,終討除之,蓋禍原於士良、弘志云。



 楊復光,閩人也,本喬氏。有武力,少養於內常侍楊玄價家,頗以節誼自奮,玄價奇之。宣宗時,玄價監鹽州軍,誣殺刺史劉皋。皋有威名者,世訟其冤。稍遷左神策軍中尉,譖去宰相楊收,權寵震時。



 復光有謀略,累監諸鎮軍。乾符初,佐平盧節度使曾元裕擊賊王仙芝,敗之。招討使宋威擊仙芝於江西,復光在軍,請判官吳彥宏約賊降,仙芝遣將尚君長自縛如約。威疾其功,密請僖宗誅之,故仙芝怨,復引兵叛。後天子寤威階禍,罷之,以兵與復光,乃進禽徐唐莒。王鐸為招討,復光仍監軍。鐸之棄荊南也,山南東道節度使劉巨容定其地,以忠武別將宋浩領荊南,泰寧將段彥謨佐之。復光父嘗監忠武軍,而浩已為大將,見復光,少之,不為禮,彥謨亦恥居浩下,遂有隙。復光曰:「胡不殺之?」彥謨引慓士擊殺浩,復光以客常滋假留後,而奏浩罪,薦彥謨為朗州刺史。詔鄭紹業為荊南節度使,以復光監忠武軍,屯鄧州,遏賊右沖。帝西幸,召紹業見行在,復光更引彥謨為荊南節度使。彥謨紿行邊,詣復光,以黃金數百兩為謝。其後忠武周岌受賊命,嘗夜宴,召復光,左右曰:「彼既附賊,必不利公,不如毋行。」復光固往,酒所語時事,復光泣曰:「丈夫所感,獨恩與義耳,彼不顧恩義,規利害,何丈夫哉!公奮匹夫封侯,乃捐十八葉天子,北面臣賊,何恩義利害昧昧耶?」岌流涕曰:「吾力不足,陽合而陰離之,故召公計。」因持杯盟曰:「有如酒!」即遣子守亮斬賊使於傳舍。秦宗權據蔡州叛,岌、復光以忠武兵三千入見之。宗權即遣部將王淑持兵萬人從。復光定荊、襄,師次鄧,淑逗遛,復光斬之,並其軍為八,以鹿宴弘、晉暉、張造、李師泰、王建、韓建等為之將,進攻南陽。賊將硃溫、何勤逆戰,大敗,遂收鄧州,追北藍橋。會母喪,班師。俄起為天下兵馬都監,總諸軍,與東面招討使王重榮並力定關中。硃溫守同州,復光遣使鐫諭,溫以所部降。方賊之強,重榮憂不知所出,謂復光曰:「臣賊邪,且負國;拒戰邪,則兵寡,柰何?」復光曰:「李克用與我世共患難,其為人,奮不顧身,比數召未即至者,由太原道不通耳,非忍禍者。若諭上意,彼宜必來。」重榮曰:「善。」白王鐸以詔使至太原,克用兵乃出。京師平,以功加開府儀同三司、同華制置使,封弘農郡公,賜號「資忠輝武匡國平難功臣」。卒河中,贈觀軍容使,謚曰忠肅。



 復光御下有恩,軍中聞其死,皆慟哭,而麾下多立功者。諸子為將帥數十人,守宗亦為忠武節度使。



 贊曰:楚鄖公辛不敢讎君而忘父冤,昭愍之世,兩軍寵遇有厚薄,而卒用存亮夷難,功莫及者。自古忠臣出於疏斥不用蓋多矣,存亮豈通記書道理之人邪,何其識君臣大誼明甚?不尸大勞,畏權處外,又愈賢矣。與夫書「龍蛇」之詩者,何其小哉!



\end{pinyinscope}