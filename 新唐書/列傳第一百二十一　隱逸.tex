\article{列傳第一百二十一 隱逸}

\begin{pinyinscope}

 古之隱者,大抵有三概:上焉者,身藏而德不晦,故自放草野,而名往從之揮古代「元氣自然論」,批駁「天人感應論」與讖緯迷信之說,,雖萬乘之貴,猶尋軌而委聘也;其次,挈治世具弗得伸,或持峭行不可屈於俗,雖有所應,其於爵祿也,泛然受,悠然辭,使人君常有所慕企,怊然如不足,其可貴也;末焉者,資槁薄,樂山林,內審其才,終不可當世取舍,故逃丘園而不返,使人常高其風而不敢加訾焉。且世未嘗無隱,有之未嘗不旌賁而先焉者,以孔子所謂「舉逸民,天下之人歸焉」。



 唐興,賢人在位眾多,其遁戢不出者,才班班可述,然皆下概者也。雖然,各保其素,非托默於語,足崖壑而志城闕也。然放利之徒,假隱自名,以詭祿仕,肩相摩於道,至號終南、嵩少為仕途捷徑,高尚之節喪焉。故裒可喜慕者類於篇。



 王績,字無功,絳州龍門人。性簡放,不喜拜揖。兄通,隋末大儒也,聚徒河、汾間,仿古作《六經》,又為《中說》以擬《論語》。不為諸儒稱道,故書不顯,惟《中說》獨傳。通知績誕縱,不嬰以家事,鄉族慶吊冠昏,不與也。與李播、呂才善。



 大業中,舉孝悌廉潔,授秘書省正字。不樂在朝,求為六合丞,以嗜酒不任事,時天下亦亂,因劾,遂解去。嘆曰:「網羅在天,吾且安之!」乃還鄉里。有田十六頃在河渚間。仲長子光者,亦隱者也,無妻子,結廬北渚,凡三十年,非其力不食。績愛其真,徙與相近。子光喑,未嘗交語,與對酌酒歡甚。績有奴婢數人,種黍,春秋釀酒,養鳧雁,蒔藥草自供。以《周易》、《老子》、《莊子》置床頭,他書罕讀也。欲見兄弟,輒度河還家。游北山東皋,著書自號東皋子。乘牛經酒肆,留或數日。



 高祖武德初,以前官待詔門下省。故事,官給酒日三升,或問:「待詔何樂邪?」答曰:「良醞可戀耳!」侍中陳叔達聞之,日給一斗,時稱「斗酒學士」。貞觀初,以疾罷。復調有司,時太樂署史焦革家善釀,績求為丞,吏部以非流不許,績固請曰:「有深意。」竟除之。革死,妻送酒不絕,歲餘,又死。績曰:「天不使我酣美酒邪?」棄官去。自是太樂丞為清職。追述革酒法為經,又採杜康、儀狄以來善酒者為譜。李淳風曰:「君,酒家南、董也。」所居東南有盤石,立杜康祠祭之,尊為師,以革配。著《醉鄉記》以次劉伶《酒德頌》。其飲至五斗不亂,人有以酒邀者,無貴賤輒往,著《五斗先生傳》。刺史崔喜悅之,請相見,答曰:「奈何坐召嚴君平邪?」卒不詣。杜之松,故人也,為刺史,請績講禮,答曰:「吾不能揖讓邦君門,談糟粕,棄醇醪也。」之松歲時贈以酒脯。初,兄凝為隋著作郎,撰《隋書》未成,死,績續餘功,亦不能成。豫知終日,命薄葬,自志其墓。



 績之仕,以醉失職,鄉人靳之,托無心子以見趣曰:「無心子居越,越王不知其大人也,拘之仕,無喜色。越國法曰:『穢行者不齒。』俄而無心子以穢行聞,王黜之,無慍色。退而適茫蕩之野,過動之邑而見機士,機士撫髀曰:『嘻!子賢者而以罪廢邪?』無心子不應。機士曰:『願見教。』曰:『子聞蜚廉氏馬乎?一者硃鬣白毳,龍骼鳳臆,驟馳如舞,終日不釋轡而以熱死;一者重頭昂尾,駝頸貉膝,𧾷是嚙善蹶,棄諸野,終年而肥。夫鳳不憎山棲,龍不羞泥蟠,君子不茍潔以罹患,不避穢而養精也。』」其自處如此。



 硃桃椎,益州成都人。澹泊絕俗,被裘曳索,人莫能測其為。長史竇軌見之,遺以衣服、鹿幘、麂靴,逼署鄉正。委之地,不肯服。更結廬山中,夏則裸,冬緝木皮葉自蔽,贈遺無所受。嘗織十芒屩置道上,見者曰:「居士屩也。」為鬻米茗易之,置其處,輒取去,終不與人接。其為屩,草柔細,環結促密,人爭躡之。高士廉為長史,備禮以請,降階與之語,不答,瞪視而出。士廉拜曰:「祭酒其使我以無事治蜀邪?」乃簡條目,薄賦斂,州大治。屢遣人存問,見輒走林草自匿云。



 孫思邈,京兆華原人。通百家說,善言老子、莊周。周洛州總管獨孤信見其少,異之,曰:「聖童也,顧器大難為用爾!」及長,居太白山。隋文帝輔政,以國子博士召,不拜。密語人曰:「後五十年有聖人出,吾且助之。」太宗初,召詣京師,年已老,而聽視聰嘹。帝嘆曰:「有道者!」欲官之,不受。顯慶中,復召見,拜諫議大夫,固辭。上元元年,稱疾還山,高宗賜良馬,假鄱陽公主邑司以居之。



 思邈於陰陽、推步、醫藥無不善,孟詵、盧照鄰等師事之。照鄰有惡疾,不可為,感而問曰:「高醫愈疾,奈何?」答曰:「天有四時五行,寒暑迭居,和為雨,怒為風,凝為雨霜,張為虹霓,天常數也。人之四支五藏,一覺一寐,吐納往來,流為榮衛,章為氣色,發為音聲,人常數也。陽用其形,陰用其精,天人所同也。失則烝生熱,否生寒,結為瘤贅,陷為癰疽,奔則喘乏,端則燋槁,發乎面,動乎形。天地亦然:五緯縮贏,孛彗飛流,其危診也;寒暑不時,其蒸否也;石立土踴,是其瘤贅;山崩土陷,是其癰疽;奔風暴雨其喘乏,川瀆竭涸其燋槁。高醫導以藥石,救以金乏劑;聖人和以至德,輔以人事。故體有可愈之疾,天有可振之災。」



 照鄰曰:「人事奈何?」曰:「心為之君,君尚恭,故欲小。《詩》曰『如臨深淵,如履薄冰』,小之謂也。膽為之將,以果決為務,故欲大。《詩》曰『赳赳武夫,公侯干城』,大之謂也。仁者靜,地之象,故欲方,《傳》曰『不為利回,不為義疚』,方之謂也。智者動,天之象,故欲圓。《易》曰『見機而作,不俟終日』,圓之謂也。」



 復問養性之要,答曰:「天有盈虛,人有屯危,不自慎,不能濟也。故養性必先知自慎也。慎以畏為本,故士無畏則簡仁義,農無畏則墮稼穡,工無畏則慢規矩,商無畏則貸不殖,子無畏則忘孝,父無畏則廢慈,臣無畏則勛不立,君無畏則亂不治。是以太上畏道,其次畏天,其次畏物,其次畏人,其次畏身。憂於身者不拘於人,畏於己者不制於彼,慎於小者不懼於大,戒於近者不侮於遠。知此則人事畢矣。」



 初,魏徵等修齊、梁、周、隋等五家史,屢咨所遺,其傳最詳。永淳初,卒,年百餘歲,遺令薄葬,不藏明器,祭去牲牢。



 孫處約嘗以諸子見,思邈曰:「俊先顯,侑晚貴,佺禍在執兵。」後皆驗。太子詹事盧齊卿之少也,思邈曰:「後五十年位方伯,吾孫為屬吏,願自愛。」時思邈之孫溥尚未生,及溥為蕭丞,而齊卿徐州刺史。



 田游巖,京兆三原人。永徽時,補太學生。罷歸,入太白山。母及妻皆有方外志,與共棲遲山水間。自蜀歷荊、楚,愛夷陵青溪,止廬其側。長史李安期表其才,召赴京師,行及汝,辭疾入箕山,居許由祠旁,自號「由東鄰」,頻召不出。



 高宗幸嵩山,遣中書侍郎薛元超就問其母,賜藥物絮帛。帝親至其門,游巖野服出拜,儀止謹樸,帝令左右扶止,謂曰:「先生比佳否?」答曰:「臣所謂泉石膏肓,煙霞痼疾者。」帝曰:「朕得君,何異漢獲四皓乎?」薛元超贊帝曰:「漢欲廢嫡立庶,故四人者為出,豈如陛下親降巖穴邪?」帝悅,因敕游巖將家屬乘傳赴都,拜崇文館學士。帝營奉天宮,游巖舊宅直宮左,詔不聽毀。天子自書榜其門,曰「隱士田游巖宅」。進太子洗馬。裴炎死,坐素厚善,放還山。蠶衣耕食,不交當世,惟與韓法昭、宋之問為方外友云。



 時又有史德義者,昆山人,居虎丘山。騎牛帶瓢,出入廛野。高宗聞其名,召至洛陽,俄稱疾歸。天授初,江南宣勞使周興薦之,復召赴都,擢朝散大夫。興死,免官歸,素譽頓衰。



 孟詵,汝州梁人。擢進士第,累遷鳳閣舍人。他日至劉禕之家,見賜金,曰:「此藥金也,燒之,火有五色氣。」試之,驗。武后聞,不悅,出為臺州司馬,頻遷春官侍郎。相王召為侍讀。拜同州刺史。神龍初,致仕,居伊陽山,治方藥。睿宗召,將用之,以老固辭,賜物百段,詔河南春秋給羊酒糜粥。尹畢構以詵有古人風,名所居為子平里。開元初,卒,年九十三。



 詵居官頗刻斂,然以治稱。其閑居嘗語人曰:「養性者,善言不可離口,善藥不可離手。」當時傳其當。



 王友貞,懷州河內人。父知敬,善書隸。武后時,仕為麟臺少監。友貞少為司經局正字。母病,醫言得人肉啖良已,友貞剔股以進,母疾愈。詔旌表其門。素好學,訓誨子弟如嚴君。口不語人過,重然諾,時以為君子。歷長水令,罷歸。中宗在東宮,召為司儀郎,不就。神龍初,以太子中舍人征,固辭疾。詔致珍饌,給全祿終身,四時送其所,州縣存問。玄宗在東宮,表以蒲車召,不至。卒,年九十九,贈銀青光祿大夫,賴縣令吊祭。



 王希夷,徐州滕人。家貧,父母喪,為人牧羊,取人庸以葬。隱嵩山,師黃頤學養生四十年。頤卒,更居兗州徂徠,與劉玄博友善。喜讀《周易》、《老子》,餌松柏葉、雜華,年七十餘,筋力柔強。刺史盧齊卿就謁問政,答曰:「『己所不欲,勿施於人』,此言足矣。」



 玄宗東巡狩,詔州縣敦勸見行在,時九十餘,帝令張說訪以政事,宦官扶入宮中,與語甚悅,拜國子博士,聽還山。敕州縣春秋致束帛酒肉,仍賜絹百、衣一稱。



 李元愷,邢州人。博學,善天步律歷,性恭慎,未嘗敢語人。宋璟嘗師之,既當國,厚遺以束帛,將薦之朝,拒不答。洺州刺史元行沖邀致之,問經義畢,贈衣服,辭曰:「吾軀不可服新麗,懼不稱以速咎也。」行沖垢衊復與之,不獲已而受。俄報身所蠶素絲,曰:「義不受無妄財也。」先是,定州崔元鑒善《禮》學,用張易之力,授朝散大夫,家居給半祿。元愷誚曰:「無功而祿,災也。」卒,年八十餘。



 衛大經,蒲州解人。卓然高行,口無二言。武后時,召之,固辭疾。素善魏夏侯乾童,聞其母卒,盛暑步往吊,或止之曰:「方夏,涉遠不如致書。」答曰:「書能盡意邪?」比至,乾童以事行,乃設席行吊禮,不訊其家而還。開元初,畢構為刺史,使縣令孔慎言就謁,辭不見。



 大經邃於《易》,人謂之「《易》聖」。豫筮死日,鑿墓自為志,如言終。



 武攸緒,則天皇后兄惟良子也。恬淡寡欲,好《易》、莊周書。少變姓名,賣卜長安市,得錢輒委去。後更授太子通事舍人,累遷揚州大都督府長史、鴻臚少卿。後革命,封安平郡王,從封中嶽,固辭官,願隱居。後疑其詐,許之,以觀所為。攸緒廬巖下如素遁者,後遣其兄攸宜敦諭,卒不起,後乃異之。盤桓龍門、少室間,冬蔽茅椒,夏居石室,所賜金銀鐺鬲、野服,王公所遺鹿裘、素障、癭杯,塵皆流積,不御也。市田潁陽,使家奴雜作,自混於民。晚年肌肉消眚,瞳有紫光,晝能見星。



 中宗初,降封巢國公,遣國子司業杜慎盈齎書以安車召,拜太子賓客。苦祈還山,詔可。安樂公主出降,又遣通事舍人李邈以璽書迎之。將至,帝敕有司即兩儀殿設位,行問道禮,詔見日山帔葛巾,不名不拜。攸緒至,更冠帶。仗入,通事舍人贊就位,攸緒趨就常班再拜,帝愕然,禮不及行,朝廷嘆息。賜予無所受,親貴來謁,道寒溫外,默無所言。及還,中書、門下、學士、朝官五品以上,並祖城東。



 俄而諸韋誅,武氏連禍,唯攸緒不及。睿宗恐其不自安,下詔慰諭,復召拜太子賓客,不就。譙王重福之亂,攸緒以誣被系,張說表置廬山,中書令姚元崇奏:「攸緒在武后時未嘗輒出,今州縣逼遣,士為驚嗟。願詔賜嵩山舊居,令州縣存問。」詔可。開元十一年卒。



 白履忠,汴州浚儀人。貫知文史,居古大梁城,時號梁丘子。景雲中,召為校書郎,棄官去。開元十年,刑部尚書王志愔薦履忠博學守操,可代褚無量、馬懷素入閣侍讀,國子祭酒楊瑒又表其賢,召赴京師。辭病老不任職,詔拜朝散大夫。乞還,手詔許游京師,徐返裏閭。履忠留數月乃去。



 吳兢,其里人也,謂曰:「子素貧,不沾斗米匹帛,雖得五品亦何益?」履忠曰:「往契丹入寇,家取排門夫,吾以讀書,縣為免。今終身高臥,寬徭役,豈易得哉!」



 盧鴻,字顥然,其先幽州範陽人,徙洛陽。博學,善書籀。廬嵩山。玄宗開元初,備禮徵再,不至。五年,詔曰:「鴻有泰一之道,中庸之德,鉤深詣微,確乎自高。詔書屢下,每輒辭托,使朕虛心引領,於今數年。雖得素履幽人之介,而失考父滋恭之誼,豈朝廷之故與生殊趣邪?將縱欲山林,往而不能返乎?禮有大倫,君臣之義不可廢也。今城闕密邇,不足為勞,有司其齎束帛之具,重宣茲旨,想有以翻然易節,副朕意焉。」



 鴻至東都,謁見不拜,宰相遣通事舍人問狀,答曰:「禮者,忠信所薄,臣敢以忠信見。」帝召升內殿,置酒。拜諫議大夫,固辭。復下制,許還山,歲給米百斛、絹五十,府縣為致其家,朝廷得失,其以狀聞。將行,賜隱居服,官營草堂,恩禮殊渥。鴻到山中,廣學廬,聚徒至五百人。及卒,帝賜萬錢。鴻所居室,自號寧極云。



 吳筠,字貞節,華州華陰人。通經誼,美文辭,舉進士不中。性高鯁,不耐沈浮於時,去居南陽倚帝山。



 天寶初,召至京師,請隸道士籍,乃入嵩山依潘師正,究其術。南游天臺,觀滄海,與有名士相娛樂,文辭傳京師。玄宗遣使召見大同殿,與語甚悅,敕待詔翰林,獻《玄綱》三篇。帝嘗問道,對曰:「深於道者,無如《老子》五千文,其餘徒喪紙札耳。」復問神仙治煉法,對曰:「此野人事,積歲月求之,非人主宜留意。」筠每開陳,皆名教世務,以微言諷天子,天子重之。群沙門嫉其見遇,而高力士素事浮屠,共短筠於帝,筠亦知天下將亂,懇求還嵩山。詔為立道館。安祿山欲稱兵,乃還茅山。而兩京陷,江、淮盜賊起,因東入會稽剡中。大歷十三年卒,弟子私謚為宗元先生。



 始,蟋嘻惡於力士而斥,故文章深詆釋氏。筠所善孔巢父、李白,歌詩略相甲乙云。



 潘師正者,貝州宗城人。少喪母,廬墓,以孝聞。事王遠知為道士,得其術,居逍遙穀。高宗幸東都,召見,問所須,對曰:「茂松清泉,臣所須也,既不乏矣。」帝尊異之,詔即其廬作崇唐觀。及營奉天宮,又敕直逍遙谷作門曰仙游,北曰尋真。時太常獻新樂,帝更名《祈仙》、《望仙》、《翹仙曲》。卒,年九十八,贈太中大夫,謚體玄先生。



 又有劉道合者,亦與師正同居嵩山,帝即所隱立太一觀,使居之。時將封太山,雨不止,帝令道合禳祝,俄而霽,乃令馳傳先行太山祈祓。得賞賜輒散貧乏,無所蓄。



 咸亨中,為帝作丹,劑成而卒。帝後營宮,遷道合墓,開其棺,見骸坼若蟬蛻者。帝聞,恨曰:「為我合丹,而自服去。」然所餘丹無它異。



 司馬承禎,字子微,洛州溫人。事潘師正,傳闢谷道引術,無不通。師正異之,曰:「我得陶隱居正一法,逮而四世矣。」因辭去,遍游名山,廬天臺不出。武后嘗召之,未幾,去。睿宗復命其兄承禕就起之。既至,引入中掖廷問其術,對曰:「為道日損,損之又損,以至於無為。夫心目所知見,每損之尚不能已,況攻異端而增智慮哉?」帝曰:「治身則爾,治國若何?」對曰:「國猶身也,故游心於淡,合氣於漠,與物自然而無私焉,而天下治。」帝嗟味曰:「廣成之言也!」錫寶琴、霞紋帔,還之。



 開元中,再被召至都,玄宗詔於王屋山置壇室以居。善篆、隸,帝命以三體寫《老子》,刊正文句。又命玉真公主及光祿卿韋縚至所居,按金籙7設祠,厚賜焉。卒,年八十九,贈銀青光祿大夫,謚貞一先生,親文其碑。



 自師正、道合與承禎等,語言詼譎似方士,叕刂之不錄,直取其隱概云。



 賀知章,字季真,越州永興人。性曠夷,善談說,與族姑子陸象先善。象先嘗謂人曰:「季真清談風流,吾一日不見,則鄙吝生矣。」



 證聖初,擢進士、超拔群類科,累遷太常博士。張說為麗正殿修書使,表知章及徐堅、趙冬曦入院,撰《六典》等書,累年無功。開元十三年,遷禮部侍郎,兼集賢院學士,一日並謝。宰相源乾曜語說曰:「賀公兩命之榮,足為光寵,然學士、侍郎孰為美?」說曰:「侍郎衣冠之選,然要為具員吏;學士懷先王之道,經緯之文,然後處之。此其為間也。」玄宗自為贊賜之。遷太子右庶子,充侍讀。



 申王薨,詔選挽郎,而知章取舍不平,廕子喧訴不能止,知章梯墻出首以決事,人皆靳之,坐徙工部。肅宗為太子,知章遷賓客,授秘書監,而左補闕薛令之兼侍讀。時東宮官積年不遷,令之書壁,望禮之薄,帝見,復題「聽自安者」。令之即棄官,徒步歸鄉里。



 知章晚節尤誕放,遨嬉里巷,自號「四明狂客」及「秘書外監」。每醉,輒屬辭,筆不停書,咸有可觀,未始刊飭。善草隸,好事者具筆研從之,意有所愜,不復拒,然紙才十數字,世傳以為寶。



 天寶初,病,夢游帝居,數日寤,乃請為道士,還鄉里,詔許之,以宅為千秋觀而居。又求周宮湖數頃為放生池,有詔賜鏡湖剡川一曲。既行,帝賜詩,皇太子百官餞送。擢其子僧子為會稽郡司馬,賜緋魚,使侍養,幼子亦聽為道士。卒,年八十六。肅宗乾元初,以雅舊,贈禮部尚書。



 令之,長溪人。肅宗亦以舊恩召,而令之已前卒。



 秦系,字公緒,越州會稽人。天寶末,避亂剡溪,北都留守薛兼訓奏為右衛率府倉曹參軍,不就。客泉州,南安有九日山,大松百餘章,俗傳東晉時所植,系結廬其上,穴石為研,注《老子》,彌年不出。刺史薛播數往見之,歲時致羊酒,而系未嘗至城門。姜公輔之謫,見系輒窮日不能去,築室與相近,忘流落之苦。公輔卒,妻子在遠,系為葬山下。張建封聞系之不可致,請就加校書郎。



 與劉長卿善,以詩相贈答。權德輿曰:「長卿自以為五言長城,系用偏師攻之,雖老益壯。」其後東度秣陵,年八十餘卒。南安人思之,為立於亭,號其山為高士峰云。



 張志和,字子同,婺州金華人。始名龜齡。父游朝,通莊、列二子書,為《象罔》、《白馬證》諸篇佐其說。母夢楓生腹上而產志和。十六擢明經,以策幹肅宗,特見賞重,命待詔翰林,授左金吾衛錄事參軍,因賜名。後坐事貶南浦尉,會赦還,以親既喪,不復仕,居江湖,自稱煙波釣徒。著《玄真子》,亦以自號。有韋詣者,為撰《內解》。志和又著《太易》十五篇,其卦三百六十五。



 兄鶴齡恐其遁世不還,為築室越州東郭,茨以生草,椽棟不施斤斧。豹席棕〓,每垂釣不設餌,志不在魚也。縣令使浚渠,執畚無忤色。嘗欲以大布制裘,嫂為躬績織,及成,衣之,雖暑不解。



 觀察使陳少游往見,為終日留,表其居曰玄真坊。以門隘,為買地大其閎,號回軒巷。先是門阻流水,無梁,少游為構之,人號大夫橋。帝嘗賜奴婢各一,志和配為夫婦,號漁童、樵青。



 陸羽常問:「孰為往來者?」對曰:「太虛為室,明月為燭,與四海諸公共處,未嘗少別也,何有往來?」顏真卿為湖州刺史值志和來謁,真卿以舟敝漏,請更之,志和曰:「願為浮家泛宅,往來苕、霅間。」辯捷類如此。



 善圖山水,酒酣,或擊鼓吹笛,舐筆輒成。嘗撰《漁歌》,憲宗圖真求其歌,不能致。李德裕稱志和「隱而有名,顯而無事,不窮不達,嚴光之比」云。



 孫述睿,越州山陰人。梁侍中休源八世孫。高祖德紹,事竇建德為中書侍郎,嘗草檄毀薄太宗,賊平,執登汜水樓,責曰:「爾以檄謗我云何?」對曰:「犬吠非其主。」帝怒曰:「賊乃主邪?」命壯士捽殞樓下。曾祖昌寓,字廣成,貞觀中對策高第,歷魏州司馬,有治狀,帝為不置刺史。為政三年,璽書褒美,進膳部郎中。祖祖舜,字奉先,為監察御史,以累下除成武令,雉馴於廷。



 述睿少與兄充符、弟克讓篤孝,已孤,偕隱嵩山。而述睿資嗜學。大歷中,劉晏薦於代宗,以太常寺協律郎召,擢累司勛員外郎、史館修撰。述睿每一遷,即至朝謝。俄而辭疾歸,以為常。



 德宗立,拜諫議大夫,命河南尹趙惠伯齎詔書束帛,備禮敦遣。既至,對別殿,賜第宅,給廄馬,兼皇太子侍讀。固辭,弗許。久乃改秘書少監,兼右庶子,復為史館修撰。述睿重次《地理志》,本末最詳。性退讓,未始忤物,雖親朋燕集,至嚴默終日,人皆畏之。與令狐峘同職,峘數抵侮,然卒不校也,時稱長者。



 貞元四年,帝念平涼之難尤惻怛,以述睿精愨而誠,故遣持祠具稱詔臨祭。又以疾乞解,久乃許,以太子賓客還鄉,賜帛五十匹、衣一襲。故事,致仕不給公馹,帝特命給焉。卒,年七十一,贈工部尚書。



 子敏行,字至之。元和初,擢進士第。岳鄂呂元膺表在節度府,元膺徙東都、河中,輒隨府遷。入拜右拾遺,四遷司勛郎中、集賢殿學士、諫議大夫。李絳遇害,事本監軍楊叔元,時無敢言,敏行上書極論之,叔元乃得罪。以名臣子,少修潔,及仕宦,能交當時豪俊,有名一時,而雅操不逮父矣。卒,年三十九,贈工部侍郎。



 陸羽,字鴻漸,一名疾,字季疵,復州竟陵人。不知所生,或言有僧得諸水濱,畜之。既長,以《易》自筮,得《蹇》之《漸》,曰:「鴻漸於陸,其羽可用為儀。」乃以陸為氏,名而字之。



 幼時,其師教以旁行書,答曰:「終鮮兄弟,而絕後嗣,得為孝乎?」師怒,使執糞除圬塓以苦之,又使牧牛三十,羽潛以竹畫牛背為字。得張衡《南都賦》,不能讀,危坐效群兒囁嚅若成誦狀,師拘之,令薙草莽。當其記文字,懵懵若有遺,過日不作,主者鞭苦,因嘆曰:「歲月往矣,奈何不知書!」嗚咽不自勝,因亡去,匿為優人,作詼諧數千言。



 天寶中,州人酺,吏署羽伶師,太守李齊物見,異之,授以書,遂廬火門山。貌侻陋,口吃而辯。聞人善,若在己,見有過者,規切至忤人。朋友燕處,意有所行輒去,人疑其多嗔。與人期,雨雪虎狼不避也。上元初,更隱苕溪,自稱桑苧翁,闔門著書。或獨行野中,誦詩擊木,裴回不得意,或慟哭而歸,故時謂今接輿也。久之,詔拜羽太子文學,徙太常寺太祝,不就職。貞元末,卒。



 羽嗜茶,著經三篇,言茶之原、之法、之具尤備,天下益知飲茶矣。時鬻茶者,至陶羽形置煬突間,祀為茶神。有常伯熊者,因羽論復廣著茶之功。御史大夫李季卿宣慰江南,次臨淮,知伯熊善煮茶,召之,伯熊執器前,季卿為再舉杯。至江南,又有薦羽者,召之,羽衣野服,挈具而入,季卿不為禮,羽愧之,更著《毀茶論》。其後尚茶成風,時回紇入朝,始驅馬市茶。



 崔覲,梁州城固人。以儒自業,身耕耨取給。老無子,乃以田宅財貲分給奴婢各為業,而身與妻隱南山,約奴婢過其舍則給酒食,夫婦嘯詠相視為娛。山南西道節度使鄭餘慶闢為參謀,敦趣就職,不曉吏事,餘慶稱長者。文宗時,左補厥王直方,其里中人也,上書論事,見便殿,訪遺逸,直方薦覲高行,詔以起居郎召,辭疾不至。



 陸龜蒙,字魯望,元方七世孫也。父賓虞,以文歷侍御史。龜蒙少高放,通《六經》大義,尤明《春秋》。舉進士,一不中,往從湖州刺史張摶游,摶歷湖、蘇二州,闢以自佐。嘗至饒州,三日無所詣。刺史蔡京率官屬就見之,龜蒙不樂,拂衣去。



 居松江甫裏,多所論撰,雖幽憂疾痛,貲無十日計,不少輟也。文成,竄稿篋中,或歷年不省,為好事者盜去。得書熟誦乃錄,讎比勤勤,硃黃不去手,所藏雖少,其精皆可傳。借人書,篇帙壞舛,必為輯褫刊正。樂聞人學,講論不倦。



 有田數百畝,屋三十楹,田苦下,雨潦則與江通,故常苦饑。身畚鍤,茠刺無休時,或譏其勞,答曰:「堯、舜霉瘠,禹胼胝。彼聖人也,吾一褐衣,敢不勤乎?」嗜茶,置園顧渚山下,歲取租茶,自判品第。張又新為《水說》七種,其二慧山泉,三虎丘井,六松江。人助其好者,雖百里為致之。初,病酒,再期乃已,其後客至,挈壺置杯不復飲。不喜與流俗交,雖造門不肯見。不乘馬,升舟設蓬席,齎束書、茶灶、筆床、釣具往來。時謂江湖散人,或號天隨子、甫裏先生,自比涪翁、漁父、江上丈人。寬以高士召,不至。李蔚、盧攜素與善,及當國,召拜左拾遺。詔方下,龜蒙卒。光化中,韋莊表龜蒙及孟郊等十人,皆贈右補闕。



 陸氏在姑蘇,其門有巨石。遠祖績嘗事吳為鬱林太守,罷歸無裝,舟輕不可越海,取石為重,人稱其廉,號「鬱林石」,世保其居云。



\end{pinyinscope}