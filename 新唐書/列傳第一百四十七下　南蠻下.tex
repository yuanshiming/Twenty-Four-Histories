\article{列傳第一百四十七下 南蠻下}

\begin{pinyinscope}

 環王,本林邑也,一曰占不勞,亦曰占婆。直交州南,海行三千里。地東西三百里而贏物《新時代》編輯,1902—1907年任《萊比錫人民報》主編。,南北千里。西距真臘霧溫山,南抵奔浪陀州。其南大浦,有五銅柱,山形若倚蓋,西重巖,東涯海,漢馬援所植也。又有西屠夷,蓋援還,留不去者,才十戶。隋末孳衍至三百,皆姓馬,俗以其寓,故號「馬留人」,與林邑分唐南境。其地冬溫,多霧雨,產虎魄、猩猩獸、結遼鳥。以二月為歲首,稻歲再熟,取檳榔沈為酒,椰葉為席。俗兇悍,果戰鬥,以麝塗身,日再塗再澡,拜謁則合爪頓顙。有文字,喜浮屠道,冶金銀像,大或十圍。呼王為陽蒲逋,王妻為陀陽阿熊,太子為阿長逋,宰相為婆漫地。王所居曰占城,別居曰齊國、曰蓬皮勢。王衣白氎,古貝斜絡臂,飾金琲為纓,鬈發,戴金華冠如章甫。妻服朝霞,古貝短裙,冠纓如王。王衛兵五千,戰乘象,藤為鎧,竹為弓矢,率象千、馬四百,分前後。不設刑,有罪者使象踐之;或送不勞山,畀自死。



 隋仁壽中,遣將軍劉芳伐之,其王範梵志挺走,以其地為三郡,置守令。道阻不得通,梵志裒遺眾,別建國邑。武德中,再遣使獻方物,高祖為設九部樂饗之。貞觀時,王頭黎獻馴象、鏐鎖、五色帶、朝霞布、火珠,與婆利、羅剎二國使者偕來。林邑其言不恭,群臣請問罪。太宗曰:「昔苻堅欲吞晉,眾百萬,一戰而亡。隋取高麗,歲調發,人與為怨,乃死匹夫手。朕敢妄議發兵邪?」赦不問。又獻五色鸚鵡、白鸚鵡,數訴寒,有詔還之。頭黎死,子鎮龍立,獻通天犀、雜寶。十九年,摩訶慢多伽獨弒鎮龍,滅其宗,範姓絕。國人立頭黎婿婆羅門為王,大臣共廢之,更立頭黎女為王。諸葛地者,頭黎之姑子,父得罪,奔真臘。女之王不能定國,大臣共迎諸葛地為王,妻以女。永徽至天寶,凡三入獻。至德後,更號環王。元和初不朝獻,安南都護張舟執其偽驩、愛州都統,斬三萬級,虜王子五十九,獲戰象、舠、鎧。



 婆利者,直環王東南,自交州泛海,歷赤土、丹丹諸國乃至。地大洲,多馬,亦號馬禮。袤長數千里。多火珠,大者如雞卵,圓白,照數尺,日中以艾藉珠,輒火出。產玳瑁、文螺;石坩,初取柔可治,既鏤刻即堅。有舍利鳥,通人言。俗黑身,硃發而拳,鷹爪獸牙,穿耳傅璫,以古貝橫一幅繚於腰。古貝,草也,緝其花為布,粗曰貝,精曰氎。俗以夜為市,自掩其面。王姓剎利邪伽,名護路那婆,世居位。繚班絲貝,綴珠為飾。坐金榻,左右持白拂、孔雀翣。出以象駕車,羽蓋珠箔,鳴金、擊鼓、吹蠡為樂。



 其東即羅剎也,與婆利同俗。隋煬帝遣常駿使赤土,遂通中國。



 赤土西南入海,得婆羅。總章二年,其王旃達缽遣使者與環王使者偕朝。



 環王南有殊柰者,泛交趾海三月乃至,與婆羅同俗。貞觀二年,使者上方物。九年,甘棠使者入朝,國居海南。十二年,僧高、武令、迦乍、鳩密四國使者朝貢。僧高直水真臘西北,與環王同俗。其後鳩密王尸利鳩摩又與富那王尸利提婆跋摩等遣使來貢。僧高等國,永徽後為真臘所並。



 盤盤,在南海曲,北距環王,限少海,與狼牙脩接,自交州海行四十日乃至。王曰楊粟圬。其民瀕水居,比木為柵,石為矢鏃。王坐金龍大榻,諸大人見王,交手抱肩以跽。其臣曰勃郎索濫,曰昆侖帝也,曰昆侖勃和,曰昆侖勃諦索甘,亦曰古龍。古龍者,昆侖聲近耳。在外曰那延,猶中國刺史也。有佛、道士祠,僧食肉,不飲酒,道士謂為貪,不食酒肉。貞觀中,再遣使朝。



 其東南有哥羅,一曰個羅,亦曰哥羅富沙羅。王姓矢利波羅,名米失缽羅。累石為城,樓闕宮室茨以草。州二十四。其兵有弓矢槊殳,以孔雀羽飾纛。每戰,以百象為一隊,一象百人,鞍若檻,四人執弓槊在中。賦率輸銀二銖。無絲糸寧,惟古貝。畜多牛少馬。非有官不束發。凡嫁娶,納檳榔為禮,多至二百盤。婦已嫁,從夫姓。樂有琵琶、橫笛、銅鈸、鐵鼓、蠡。死者焚之,取燼貯金罌沈之海。



 東南有拘蔞蜜,海行一月至。南距婆利,行十日至。東距不述,行五日至。西北距文單,行六日至。與赤土、墮和羅同俗。永徽中,獻五色鸚鵡。



 扶南,在日南之南七千里,地卑窪,與環王同俗,有城郭宮室。王姓古龍。居重觀,柵城,楉葉以覆屋。王出乘象。其人黑身、鬈發,惈行,俗不為寇盜。田一歲種,三歲獲。國出剛金,狀類紫石英,生水底石上,人沒水取之,可以刻玉,扣以羖角,乃泮。人喜鬥雞及豬。以金、珠、香為稅。治特牧城,俄為真臘所並,益南徙那弗那城。武德、貞觀時,再入朝,又獻白頭人二。



 白頭者,直扶南西,人皆素首,膚理如脂。居山穴,四面峭絕,人莫得至。與參半國接。



 真臘,一曰吉蔑,本扶南屬國。去京師二萬七百里。東距車渠,西屬驃,南瀕海,北與道明接,東北抵驩州。其王剎利伊金那,貞觀初並扶南有其地。戶皆東向,坐上東。客至,屑檳榔、龍腦、香蛤以進。不飲酒,比之淫。與妻飲房中,避尊屬。有戰象五千,良者飼以肉。世與參半、驃通好,與環王乾陀洹數相攻。自武德至聖歷,凡四來朝。神龍後分為二半:北多山阜,號陸真臘半;南際海,饒陂澤,號水真臘半。水真臘,地八百里,王居婆羅提拔城。陸真臘或曰文單,曰婆鏤,地七百里,王號「靦屈」。開元、天寶時,王子率其屬二十六來朝,拜果毅都尉。大歷中,副王婆彌及妻來朝,獻馴象十一;擢婆彌試殿中監,賜名賓漢。是時,德宗初即位,珍禽奇獸悉縱之,蠻夷所獻馴象畜苑中,元會充廷者凡三十二,悉放荊山之陽。及元和中,水真臘亦遣使入貢。



 文單西北屬國曰參半,武德八年使者來。



 道明者,亦屬國,無衣服,見衣服者共笑之。無鹽鐵,以竹弩射鳥獸自給。



 訶陵,亦曰社婆,曰闍婆,在南海中。東距婆利,西墮婆登,南瀕海,北真臘。木為城,雖大屋亦覆以栟櫚。象牙為床若席。出玳瑁、黃白金、犀、象,國最富。有穴自湧鹽。以柳花、椰子為酒,飲之輒醉,宿昔壞。有文字,知星歷。食無匕筋。有毒女,與接輒苦瘡,人死尸不腐。王居闍婆城。其祖吉延東遷於婆露伽斯城,旁小國二十八,莫不臣服。其官有三十二大夫,而大坐敢兄為最貴。山上有郎卑野州,王常登以望海。夏至立八尺表,景在表南二尺四寸。貞觀中,與墮和羅、墮婆登皆遣使者入貢,太宗以璽詔優答。墮和羅丐良馬,帝與之。至上元間,國人推女子為王,號「悉莫」,威令整肅,道不舉遺。大食君聞之,齎金一囊置其郊,行者輒避,如是三年。太子過,以足躪金,悉莫怒,將斬之,群臣固請。悉莫曰:「而罪實本於足,可斷趾。」群臣復為請,乃斬指以徇。大食聞而畏之,不敢加兵。大歷中,訶陵使者三至。元和八年,獻僧只奴四、五色鸚鵡、頻伽鳥等。憲宗拜內四門府左果毅。使者讓其弟,帝嘉美,並官之。訖大和,再朝貢。咸通中,遣使獻女樂。



 墮和羅,亦曰獨和羅,南距盤盤,北迦羅舍弗,西屬海,東真臘。自廣州行五月乃至。國多美犀,世謂墮和羅犀。有二屬國,曰曇陵、陀洹。



 曇陵在海洲中。陀洹,一曰耨陀洹,在環王西南海中,與墮和羅接,自交州行九十日乃至。王姓察失利,名婆那,字婆末。無蠶桑,有稻、麥、麻、豆。畜有白象、牛、羊、豬。俗喜樓居,謂為干欄。以白氎、朝霞布為衣。親喪,在室不食,燔尸已,則剔發浴於池,然後食。貞觀時,並遣使者再入朝,又獻婆律膏、白鸚鵡,首有十紅毛,齊於翅。因丐馬、銅鐘,帝與之。



 墮婆登在環王南,行二月乃至。東訶陵,西迷黎車,北屬海。俗與訶陵同。種稻,月一熟。有文字,以貝多葉寫之。死者實金於口,以釧貫其體,加婆律膏、龍腦眾香,積薪燔之。



 投和,在真臘南,自廣州西南海行百日乃至。王姓投和羅,名脯邪迄遙。官有朝請將軍、功曹、主簿、贊理、贊府,分領國事。分州、郡、縣三等。州有參軍,郡有金威將軍,縣有城、有局,長官得選僚屬自助。民居率樓閣,畫壁。王宿衛百人,衣朝霞,耳金鈽,金綖被頸,寶飾革履。頻盜者死,次穿耳及頰而劗其發,盜鑄者截手。無賦稅,民以地多少自輸。王以農商自業。銀作錢,類榆莢。民乘象及馬,無鞍靮,繩穿頰御之。親喪,斷發為孝,焚尸斂灰於罌,沈之水。貞觀中,遣使以黃金函內表,並獻方物。



 瞻博,或曰瞻婆。北距兢伽河。多野象群行。顯慶中,與婆岸、千支弗、舍跋若、磨臘四國並遣使者入朝。



 千支在西南海中,本南天竺屬國,亦曰半支跋,若唐言五山也,北距多摩萇。



 又有哥羅舍分、脩羅分、甘畢三國貢方物。甘畢在南海上,東距環王,王名旃陀越摩,有勝兵五千。哥羅舍分者,在南海南,東墮和羅。脩羅分者,在海北,東距真臘。其風俗大略相類,有君長,皆柵郛。二國勝兵二萬,甘畢才五千。



 又有多摩萇,東距婆鳳,西多隆,南千支弗,北訶陵。地東西一月行,南北二十五日行。其王名骨利,詭雲得大卵,剖之,獲女子,美色,以為妻。俗無姓,婚姻不別同姓。王坐常東向。勝兵二萬,有弓刀甲槊,無馬。果有波那婆、宅護遮庵摩、石榴。其國經薩廬、都訶廬、君那廬、林邑諸國,乃得交州。顯慶中貢方物。



 室利佛逝,一曰尸利佛誓。過軍徒弄山二千里,地東西千里,南北四千里而遠。有城十四,以二國分總。西曰郎婆露斯。多金、汞砂、龍腦。夏至立八尺表,影在表南二尺五寸。國多男子。有橐它,豹文而犀角,以乘且耕,名曰它牛豹。又有獸類野豕,角如山羊,名曰雩,肉味美,以饋膳。其王號「曷蜜多」。咸亨至開元間,數遣使者朝,表為邊吏侵掠,有詔廣州慰撫。又獻侏儒、僧祗女各二及歌舞。官使者為折沖,以其王為左威衛大將軍,賜紫袍、金細帶。後遣子入獻,詔宴於曲江,宰相會,冊封賓義王,授右金吾衛大將軍,還之。



 名蔑,東接真陀桓,西但游,南屬海,北波剌。其地一月行,有州三十。以十二月為歲首。王衣朝霞、氎。賦二十取一。交易皆用金準直。其人短小,兄弟共娶一妻,婦總發為角,辨夫之多少。王號「斯多題」。龍朔初,使者來貢。



 單單,在振州東南,多羅磨之西,亦有州縣。木多白檀。王姓剎利,名尸陵伽,日視事。有八大臣,號八坐。王以香塗身,冠雜寶瓔,近行乘車,遠乘象。戰必吹蠡、擊鼓。盜無輕重皆死。乾封、總章時,獻方物。



 羅越者,北距海五千里,西南哥穀羅。商賈往來所湊集,俗與墮羅缽底同。歲乘舶至廣州,州必以聞。



 驃,古硃波也,自號突羅硃,闍婆國人曰徒里拙。在永昌南二千里,去京師萬四千里。東陸真臘,西接東天竺,西南墮和羅,南屬海,北南詔。地長三千里,廣五千里,東北袤長,屬羊苴芋城。



 凡屬國十八:曰迦羅婆提,曰摩禮烏特,曰迦梨迦,曰半地,曰彌臣,曰坤朗,曰偈奴,曰羅聿,曰佛代,曰渠論,曰婆梨,曰偈陀,曰多歸,曰摩曳,餘即舍衛、瞻婆、闍婆也。



 凡鎮城九:曰道林王,曰悉利移,曰三陀,曰彌諾道立,曰突旻,曰帝偈,曰達梨謀,曰乾唐,曰末浦。



 凡部落二百九十八,以名見者三十二:曰萬公,曰充惹,曰羅君潛,曰彌綽,曰道雙,曰道甕,曰道勿,曰夜半,曰不惡奪,曰莫音,曰伽龍睒,曰阿梨吉,曰阿梨闍,曰阿梨忙,曰達磨,曰求潘,曰僧塔,曰提梨郎,曰望騰,曰擔泊,曰祿烏,曰乏毛,曰僧迦,曰提追,曰阿末邏,曰逝越,曰騰陵,曰歐咩,曰磚羅婆提,曰祿羽,曰陋蠻,曰磨地勃。



 繇彌臣至坤朗,又有小昆侖部,王名茫悉越,俗與彌臣同。繇坤朗至祿羽,有大昆侖王國,王名思利泊婆難多珊那。川原大於彌臣。繇昆侖小王所居,半日行至磨地勃柵,海行五月至佛代國。有江,支流三百六十。其王名思利些彌他。有川名思利毘離芮。土多異香。北有市,諸國估舶所湊,越海即闍婆也。十五日行,逾二大山,一曰正迷,一曰射鞮,有國,其王名思利摩訶羅闍,俗與佛代同。經多茸補邏川至闍婆,八日行至婆賄伽廬,國土熱,衢路植椰子、檳榔,仰不見日。王居以金為甓,廚覆銀瓦,爨香木,堂飾明珠。有二池,以金為堤,舟楫皆飾金寶。



 驃王姓困沒長,名摩羅惹。其相名曰摩訶思那。王出,輿以金繩床,遠則乘象。嬪史數百人。青甓為圓城,周百六十里,有十二門,四隅作浮圖,民皆居中,鉛錫為瓦,荔支為材。俗惡殺。拜以手抱臂稽顙為恭。明天文,喜佛法。有百寺,琉璃為甓,錯以金銀,丹彩紫鑛塗地,覆以錦罽,王居亦如之。民七歲祝發止寺,至二十有不達其法,復為民。衣用白氎、朝霞,以蠶帛傷生不敢衣。戴金花冠、翠冒,絡以雜珠。王宮設金銀二鐘,寇至,焚香之,以占吉兇。有巨白象,高百尺,訟者焚香跽象前,自思是非而退。有災疫,王亦焚香對象跽,自咎。無桎梏,有罪者束五竹捶背,重者五、輕三,殺人則死。土宜菽、粟、稻、梁,蔗大若脛,無麻、麥。以金銀為錢,形如半月,號登伽佗,亦曰足彈陀。無膏油,以蠟雜香代炷。與諸蠻市,以江豬、白氎、琉璃罌缶相易。婦人當頂作高髻,飾銀珠琲,衣青娑裙,披羅段;行持扇,貴家者傍至五六。近城有沙山不毛,地亦與波斯、婆羅門接,距西舍利城二十日行。西舍利者,中天竺也。南詔以兵強地接,常羈制之。



 貞元中,王雍羌聞南詔歸唐,有內附心,異牟尋遣使楊加明詣劍南西川節度使韋皋請獻夷中歌曲,且令驃國進樂人。於是皋作《南詔奉聖樂》,用正律黃鐘之均。宮、徵一變,象西南順也;角、羽終變,象戎夷革心也。舞六成,工六十四人,贊引二人,序曲二十八疊,舞「南詔奉聖樂」字。舞人十六,執羽翟,以四為列。舞「南」字,歌《聖主無為化》;舞「詔」字,歌《南詔朝天樂》;舞「奉」字,歌《海宇修文化》;舞「聖」字,歌《雨露覃無外》;舞「樂」字,歌《闢土丁零塞》。皆一章三疊而成。



 舞者初定,執羽,簫、鼓等奏散序一疊,次奏第二疊,四行,贊引以序入。將終,雷鼓作於四隅,舞者皆拜,金聲作而起,執羽稽首,以象朝覲。每拜跪,節以鉦鼓。次奏拍序一疊,舞者分左右蹈舞,每四拍,揖羽稽首,拍終,舞者拜,復奏一疊,蹈舞抃揖,以合「南」字。字成遍終,舞者北面跪歌,導以絲竹。歌已,俯伏,鉦作,復揖舞。餘字皆如之,唯「聖」字詞末皆恭揖,以明奉聖。每



 一字,曲三疊,名為五成。次急奏一疊,四十八人分行罄折,象將臣御邊也。字舞畢,舞者十六人為四列,又舞《闢四門》之舞。遽舞入遍兩疊,與鼓吹合節,進舞三,退舞三,以象三才、三統。舞終,皆稽首逡巡。又一人舞《億萬壽》之舞,歌《天南滇越俗》四章,歌舞七疊六成而終。七者,火之成數,象天子南面生成之恩。六者,坤數,象西南向化。



 凡樂三十,工百九十六人,分四部:一、龜茲部,二、大鼓部,三、胡部,四、軍樂部。龜茲部,有羯鼓、揩鼓、腰鼓、雞婁鼓、短笛、大小觱篥、拍板,皆八;長短簫、橫笛、方響、大銅鈸、貝,皆四。凡工八十八人,分四列,屬舞筵四隅,以合節鼓。大鼓部,以四為列,凡二十四,居龜茲部前。胡部,有箏、大小箜篌、五弦琵琶、笙、橫笛、短笛、拍板,皆八;大小觱篥,皆四。工七十二人,分四列,屬舞筵之隅,以導歌詠。軍樂部,金饒、金鐸,皆二;㧏鼓、金鉦,皆四。鉦、鼓,金飾蓋,垂流蘇。工十二人,服南詔服,立《壁四門》舞筵四隅,節拜合樂。又十六人,畫半臂,執㧏鼓,四人為列。舞人服南詔衣、絳裙襦、黑頭囊、金佉苴、畫皮鞾革,首飾襪額,冠金寶花鬘,襦上復加畫半臂。執羽翟舞,俯伏,以象朝拜;裙襦畫鳥獸草木,文以八彩雜華,以象庶物咸遂;羽葆四垂,以象天無不覆;正方布位,以象地無不載;分四列,以象四氣;舞為五字,以象五行;秉羽翟,以象文德;節鼓,以象號令遠布;振以鐸,明採詩之義;用龜茲等樂,以象遠夷悅服。鉦鼓則古者振旅獻捷之樂也。黃鐘,君聲,配運為土,明土德常盛。黃鐘得《乾》初九,自為其宮,則林鐘四律以正聲應之,象大君南面提天統於上,乾道明也。林鐘得《坤》初六,其位西南,西南感至化於下,坤體順也。太蔟得《乾》九二,是為人統,天地正而三才通,故次應以太蔟。三才既通,南呂復以羽聲應之。南呂,酉,西方金也;羽,北方水也。金、水悅而應乎時,以象西戎、北狄悅服。然後姑洗以角音終之。姑,故也;洗,濯也。以象南詔背吐蕃歸化,洗過日新。



 皋以五宮異用,獨唱殊音,復述《五均譜》,分金石之節奏:



 一曰黃鐘,宮之宮,軍士歌《奉聖樂》者用之。舞人服南詔衣,秉翟俯伏拜抃,合「南詔奉聖樂」五字,倡詞五,舞人乃易南方朝天之服,絳色,七節襦袖,節有青示票排衿,以象鳥翼。樂用龜茲、胡部,金鉦、㧏鼓、鐃、貝、大鼓。



 二曰太蔟,商之宮,女子歌《奉聖樂》者用之。合以管弦。若奏庭下,則獨舞一曲。樂用龜茲、鼓、笛各四部,與胡部等合作。琵琶、笙、箜篌,皆八;大小觱篥、箏、弦、五弦琵琶、長笛、短笛、方響,各四。居龜茲部前。次貝一人,大鼓十二分左右,餘皆坐奏。



 三曰姑洗,角之宮,應古律林鐘為徵宮,女子歌《奉聖樂》者用之。舞者六十四人,飾羅彩襦袖,間以八採,曳雲花履,首飾雙鳳、八卦、彩雲、花鬘,執羽為拜抃之節。以林鐘當地統,象歲功備、萬物成也。雙鳳,明律呂之和也。八卦,明還相為用也。彩雲,象氣也。花鬘,象冠也。合「奉聖樂」三字,唱詞三,表天下懷聖也。小女子字舞,則碧色襦袖,象角音主木;首飾巽卦,應姑洗之氣;以六人略後,象六合一心也。樂用龜茲、胡部,其鉦、㧏、鐃、鐸,皆覆以彩蓋,飾以花趺,上陳錦綺,垂流蘇。按《瑞圖》曰:「王者有道,則儀鳳在鼓。」故羽葆鼓棲以鳳凰,鉦棲孔雀,鐃、鐸集以翔鷺,鉦、㧏頂足又飾南方鳥獸,明澤及飛走翔伏。鉦、㧏、鐃、鐸,皆二人執擊之。貝及大鼓工伎之數,與軍士《奉聖樂》同,而加鼓、笛四部。



 四曰林鐘,徵之宮,斂拍單聲,奏《奉聖樂》,丈夫一人獨舞。樂用龜茲,鼓、笛每色四人。方響二,置龜茲部前。二隅有金鉦,中植金鐸二、貝二、鈴鈸二、大鼓十二分左右。



 五曰南呂,羽之宮,應古律黃鐘為君之宮。樂用古黃鐘方響一,大琵琶、五弦琵琶、大箜篌倍,黃鐘觱篥、小觱篥、竽、笙、塤、篪、搊箏、軋箏、黃鐘簫,笛倍。笛、節鼓、拍板等工皆一人,坐奏之。絲竹緩作,一人獨唱,歌工復通唱軍士《奉聖樂》詞。



 雍羌亦遣弟悉利移城主舒難陀獻其國樂,至成都,韋皋復譜次其聲。以其舞容、樂器異常,乃圖畫以獻。工器二十有二,其音八:金、貝、絲、竹、匏、革、牙、角。金二、貝一、絲七、竹二、匏二、革二、牙一、角二。鈴鈸四,制如龜茲部,周圓三寸,貫以韋,擊磕應節。鐵板二,長三寸五分,博二寸五分,面平,背有柄,系以韋,與鈴鈸皆飾絳紛,以花氎縷為蕊。螺貝四,大者可受一升,飾絳紛。有鳳首箜篌二:其一長二尺,腹廣七寸,鳳首及項長二尺五寸,面飾虺皮,弦一十有四,項有軫,鳳首外向;其一頂有條,軫有鼉首。箏二:其一形如鼉,長四尺,有四足,虛腹,以鼉皮飾背,面及仰肩如琴,廣七寸,腹闊八寸,尾長尺餘,卷上虛中,施關以張九弦,左右一十八柱;其一面飾彩花,傅以虺皮為別。有龍首琵琶一,如龜茲制,而項長二尺六寸餘,腹廣六寸,二龍相向為首;有軫柱各三,弦隨其數,兩軫在項,一在頸,其覆形如師子。有雲頭琵琶一,形如前,面飾虺皮,四面有牙釘,以雲為首,軫上有花象品字,三弦,覆手皆飾虺皮,刻捍撥為舞昆侖狀而彩飾之。有大匏琴二,覆以半匏,皆彩畫之,上加銅甌。以竹為琴,作虺文橫其上,長三尺餘,頭曲如拱,長二寸,以絳系腹,穿甌及匏本,可受二升。大弦應太蔟,次弦應姑洗。有獨弦匏琴,以班竹為之,不加飾,刻木為虺首;張弦無軫,以弦系頂,有四柱如龜茲琵琶,弦應太蔟。有小匏琴二,形如大匏琴,長二尺;大弦應南呂,次應應鐘。有橫笛二:一長尺餘,取其合律,去節無爪,以蠟實首,上加師子頭,以牙為之,穴六以應黃鐘商,備五音七聲;又一,管唯加象首,律度與荀勖《笛譜》同,又與清商部鐘聲合。有兩頭笛二,長二尺八寸,中隔一節,節左右開沖氣穴,兩端皆分洞體為笛量。左端應太蔟,管末三穴:一姑洗,二蕤賓,三夷則。右端應林鐘,管末三穴:一南呂,二應鐘,三大呂。下托指一穴,應清太蔟。兩洞體七穴,共備黃鐘、林鐘兩均。有大匏笙二,皆十六管,左右各八,形如鳳翼,大管長四尺八寸五分,餘管參差相次,制如笙管,形亦類鳳翼,竹為簧,穿匏達本。上古八音,皆以木漆代之,用金為簧,無匏音,唯驃國得古制。又有小匏笙二,制如大笙,律應林鐘商。有三面鼓二,形如酒缸,高二尺,首廣下銳,上博七寸,底博四寸,腹廣不過首,冒以虺皮,束三為一,碧絳約之,下當地則不冒,四面畫驃國工伎執笙鼓以為飾。有小鼓四,制如腰鼓,長五寸,首廣三寸五分,冒以虺皮,牙釘彩飾,無柄,搖之為樂節,引贊者皆執之。有牙笙,穿匏達本,漆之,上植二象牙代管,雙簧皆應姑洗。有三角笙,亦穿匏達本,漆之,上植三牛角,一簧應姑洗,餘應南呂,角銳在下,穿匏達本,柄觜皆直。有兩角笙,亦穿匏達本,上植二牛角,簧應姑洗,匏以彩飾。



 凡曲名十有二:一曰《佛印》,驃云《沒馱彌》,國人及天竺歌以事王也。二曰《言贊娑羅花》,驃云《嚨莽第》,國人以花為衣服,能凈其身也。三曰《白鴿》,驃云《荅都》,美其飛止遂情也。四曰《白鶴游》,驃云《蘇謾底哩》,謂翔則摩空,行則徐步也。五曰《斗羊勝》,膘云《來乃》。昔有人見二羊鬥海岸,強者則見,弱者入山,時人謂之「來乃」。來乃者,勝勢也。六曰《龍首獨琴》,驃云《彌思彌》,此一弦而五音備,象王一德以畜萬邦也。七曰《禪定》,驃云《掣覽詩》,謂離俗寂靜也。七曲唱舞,皆律應黃鐘商。八曰《革蔗王》,驃云《遏思略》,謂佛教民如蔗之甘,皆悅其味也。九曰《孔雀王》,驃云《桃臺》,謂毛採光華也。十曰《野鵝》,謂飛止必雙,徒侶畢會也。十一曰《宴樂》,驃云《籠聰綱摩》,謂時康宴會嘉也。十二曰《滌煩》,亦白《笙舞》,驃云《扈那》,謂時滌煩■,以此適情也。五曲律應黃鐘兩均:一黃鐘商伊越調,一林鐘商小植調。樂工皆昆侖,衣絳氎,朝霞為蔽膝,謂之瀼裓襔。兩肩加朝霞,絡腋。足臂有金寶環釧。冠金冠,左珥璫,絳貫花鬘,珥雙簪,散以毳。初奏樂,有贊者一人先導樂意,其舞容隨曲。用人或二、或六、或四、或八、至十,皆珠冒,拜首稽首以終節。其樂五譯而至,德宗授舒難陀太僕卿,遣還。開州刺史唐次述《驃國獻樂頌》以獻。大和六年,南詔掠其民三千,徙之柘東。



 兩爨蠻。自曲州、靖州西南昆川、曲軛、晉寧、喻獻、安寧距龍和城,通謂之西爨白蠻;自彌鹿、升麻二川,南至步頭,謂之東爨烏蠻。西爨自云本安邑人,七世祖晉南寧太守,中國亂,遂王蠻中。梁元帝時,南寧州刺史徐文盛召詣荊州,有爨瓚者,據其地,延袤二千餘里。土多駿馬、犀、象、明珠。既死,子震玩分統其眾。隋開皇初,遣使朝貢,命韋世沖以兵戍之,置恭州、協州、昆州。未幾叛,史萬歲擊之,至西洱河、滇池而還。震玩懼而入朝,文帝誅之,諸子沒為奴。高祖即位,以其子弘達為昆州刺史,奉父喪歸。而益州刺史段綸遣俞大施至南寧,治共範川,誘諸部皆納款貢方物。太宗遣將擊西爨,開青蛉、弄棟為縣。



 爨蠻之西,有徒莫只蠻、儉望蠻,貞觀二十三年內屬,以其地為傍、望、覽、丘、求五州,隸郎州都督府。白水蠻,地與青蛉、弄棟接,亦隸郎州。弄棟西有大勃弄、小勃弄二川蠻,其西與黃瓜、葉榆、西洱河接,其眾完富與蜀埒,無酋長,喜相讎怨。



 永徽初,大勃弄楊承顛私署將帥,寇麻州。都督任懷玉招之,不聽。高宗以左領軍將軍趙孝祖為郎州道行軍總管,與懷玉討之。至羅仵侯山,其酋禿磨蒲與大鬼主都乾以眾塞菁口,孝祖大破之。夷人尚鬼,謂主祭者為鬼主,每歲戶出一牛或一羊,就其家祭之。送鬼迎鬼必有兵,因以復仇云。孝祖按軍,多棄城,逐北至周近水。大酋儉彌於、鬼主董樸瀕水為柵,以輕騎逆戰。孝祖擊斬彌於、禿磨蒲、鬼主十餘級,會大雪,皸凍死者略盡。孝祖上言:「小勃弄、大勃弄常誘弄棟叛,今因破白水,請遂西討。」詔可。孝祖軍入,夷人皆走險。小勃弄酋長歿盛屯白旗城,率萬騎戰,敗,斬之。進至大勃弄,楊承顛嬰城守。孝祖招之,不從,麾軍進,執承顛。餘屯大者數萬、小數千,皆破降之,西南夷遂定。罷郎州都督,更置戎州都督。



 爨弘達既死,以爨歸王為南寧州都督,居石城,襲殺東爨首領蓋聘及子蓋啟,徙共範川。



 有兩爨大鬼主崇道者,與弟日進、日用居安寧城左,聞章仇兼瓊開步頭路,築安寧城,群蠻震騷,共殺築城使者。玄宗詔蒙歸義討之。師次波州,歸王及崇道兄弟千餘人泥首謝罪,赦之。俄而崇道殺日進及歸王。歸王妻阿奼,烏蠻女也,走父部,乞兵相仇,於是諸爨亂。阿奼遣使詣歸義求殺夫者,書聞,詔以其子守隅為南寧州都督,歸義以女妻之,又以一女妻崇道子輔朝。然崇道、守隅相攻討不置,阿奼訴歸義,為興師,營昆川。崇道走黎州,遂虜其族,殺輔朝,收其女,崇道俄亦被殺,諸爨稍離弱。



 閣羅鳳立,召守隅並妻歸河睒,不通中國。阿奼自主其部落,歲入朝,恩賞蕃厚。閣羅鳳遣昆川城使楊牟利以兵脅西爨,徙戶二十餘萬於永昌城。東爨以言語不通,多散依林谷,得不徙。自曲靖州、石城、升麻、昆川南北至龍和,皆殘於兵。日進等子孫居永昌城。烏蠻種復振,徙居西爨故地,與峰州為鄰。貞元中,置都督府,領羈縻州十八。



 烏蠻與南詔世昏姻,其種分七部落:一曰阿芋路,居曲州、靖州故地;二曰阿猛;三曰夔山;四曰暴蠻;五曰盧鹿蠻,二部落分保竹子嶺;六曰磨彌斂;七曰勿鄧。土多牛馬,無布帛,男子髽髻,女人被發,皆衣牛羊皮。俗尚巫鬼,無拜跪之節。其語四譯乃與中國通。大部落有大鬼主,百家則置小鬼主。



 勿鄧地方千里,有邛部六姓,一姓白蠻也,五姓烏蠻也。又有初裹五姓,皆烏蠻也,居邛部、臺登之間。婦人衣黑繒,其長曳地。又有東欽蠻二姓,皆白蠻也,居北谷。婦人衣白繒,長不過膝。又有粟蠻二姓、雷蠻三姓、夢蠻三姓,散處黎、巂、戎數州之鄙,皆隸勿鄧。勿鄧南七十里,有兩林部落,有十低三姓、阿屯三姓、虧望三姓隸焉。其南有豐琶部落,阿諾二姓隸焉。兩林地雖狹,而諸部推為長,號都大鬼主。



 勿鄧、豐琶、兩林皆謂之東蠻,天寶中,皆受封爵。及南詔陷巂州,遂羈屬吐蕃。貞元中,復通款,以勿鄧大鬼主苴嵩兼邛部團練使,封長川郡公。及死,子苴驃離幼,以苴夢沖為大鬼主,數為吐蕃侵獵。兩林都大鬼主苴那時遺韋皋書,乞兵攻吐蕃。皋遣將劉朝彩出銅山道,吳鳴鶴出清溪關道,鄧英俊出定蕃柵道,進逼臺登城。吐蕃退壁西貢川,據高為營。苴那時戰甚力,分兵大破吐蕃青海、臘城二節度軍於北谷,青海大兵馬使乞藏遮遮、臘城兵馬使悉多楊硃、節度論東柴、大將論結突梨等皆戰死,執籠官四十五人,鎧仗一萬,牛馬稱是。進拔於蔥柵。乞藏遮遮,尚結贊子也,以尸還。其下曩貢節度蘇論百餘人行哭,使一人立尸左,一人問之曰:「瘡痛乎?」曰「然。」即傅藥。曰「食乎?」曰「然。」即進膳。曰「衣乎?」曰「然。」即命裘。又問「歸乎?」曰「然。」以馬載尸而去。詔封苴那時為順政郡王,苴夢沖為懷化郡王,豐琶部落大鬼主驃傍為和義郡王,給印章、袍帶。三王皆入朝,宴麟德殿,賞賚加等,歲給其部祿鹽衣彩,黎、巂二州吏就賜之。以山阻多為盜侵,亡失所賜,皋令二州為築館,有賜,約酋長自至,授賜而遣之。然苴夢沖內附吐蕃,斷南詔使路,皋遣巂州總管蘇峞以兵三百召夢沖至琵琶川,聲其罪斬之,披其族為六部,以樣棄主之。及苴驃離長,乃命為大鬼主。驃傍年少驍敢,數出兵攻吐蕃。吐蕃間道焚其居室、部落,亡所賜印章。皋為請,復得印。



 爨蠻西有昆明蠻,一曰昆彌,以西洱河為境,即葉榆河也。距京師九千里。土歊濕,宜粳稻。人辮首、左衽,與突厥同。隨水草畜牧,夏處高山,冬入深谷。尚戰死,惡病亡,勝兵數萬。



 武德中,巂州治中吉偉使南寧,因至其國,諭使使朝貢,求內屬,發兵戍守。自是歲與牂柯使偕來。龍朔三年,矩州刺史謝法成招慰比樓等七千戶內附。總章三年,置祿州、湯望州。咸亨三年,昆明十四姓率戶二萬內附,析其地為殷州、手忽州、敦州,以安輯之。殷州居戎州西北,手忽州居西南,敦州居南,遠不過五百餘里,近三百里。其後又置盤、麻等四十一州,皆以首領為刺史。



 昆明東九百里,即牂柯國也。兵數出,侵地數千里。元和八年,上表請盡歸牂柯故地。開成元年,鬼主阿珮內屬。會昌中,封其別帥為羅殿王,世襲爵。其後又封別帥為滇王,皆牂柯蠻也。東距辰州二千四百里,其南千五百里即交州也。無城郭,土熱多霖雨,稻粟再熟。無徭役,戰乃屯聚。刻木為契,盜者倍三而償,殺人者出牛馬三十。俗與東謝同。首領亦姓謝氏,至龍羽有兵三萬。武德三年,遣使者朝,以其地為牂州,拜龍羽刺史,封夜郎郡公。其北百五十里,有別部曰充州蠻,勝兵二萬,亦來朝貢,以地為充州。



 開元中,牂柯酋長元齊死,孫嘉藝襲官,封其後,乃以趙氏為酋長。二十五年,趙君道來朝。其裔有趙國珍,天寶中戰有功。閣羅鳳叛,宰相楊國忠兼劍南節度使,以國珍有方略,授黔中都督,屢敗南詔,護五溪十餘年,天下方亂,其部獨寧。終工部尚書。貞元中,官其酋長趙主俗,亦以褒朝貢不絕。至十八年,五遣使朝。元和二年,詔黔南觀察使常以本道將為押領牂柯、昆明等使,自是數遣使,或朝正月,訖開成不絕。故事:戎夷朝貢,將至都,中官驛勞於郊,既及館,恩禮尤渥。



 西爨之南,有東謝蠻,居黔州西三百里,南距守宮獠,西連夷子,地方千里。宜五穀,為畬田,歲一易之。眾處山,巢居,汲流以飲。無賦稅,刻木為契。見貴人執鞭而拜。賞有功者以牛馬、銅鼓。犯小罪則杖,大事殺之,盜物者倍償。昏姻以牛酒為聘。女婦夫家,夫慚澀避之,旬日乃出。會聚,擊銅鼓,吹角。俗椎髻,韜以絳,垂於後。坐必蹲踞,常帶刀劍。男子服衫襖、大口褲,以帶斜馮右肩,以螺殼、虎豹、猿狖、犬羊皮為飾。有謝氏,世為酋長,部落尊畏之。其族不育女,自以姓高不可以嫁人。貞觀三年,其酋元深入朝,冠烏熊皮若注旄,以金銀絡額,被毛帔,韋行滕,著履。中書侍郎顏師古因是上言:「昔周武王時,遠國入朝,太史次為《王會篇》,今蠻夷入朝,如元深冠服不同,可寫為《王會圖》。」詔可。帝以地為應州,即拜元深刺史,隸黔州都督府。又有南謝首領謝強亦來朝,以其地為莊州,授強刺史。建中三年,大酋長檢校蠻州長史、資陽郡公宋鼎與諸謝朝賀,德宗以其國小,不許。訴於黔中觀察使王礎,以州接牂柯,願隨牂柯朝賀,礎奏:「牂、蠻二州,戶繁力強,為鄰蕃所憚,請許三年一朝。」詔從之。



 元和中,辰、漵蠻酋張伯靖嫉本道督斂苛刻,聚眾叛,侵播、費二州,黔中經略使崔能、荊南節度使嚴綬、湖南觀察使柳公綽討之,三歲不能定。伯靖上表請隸荊南,乃降。崔能內恨之,更請調荊南、湖南、桂管軍為援,約西原十洞兵皆出,可以成功。公卿議者皆以為便。宰相李吉甫曰:「伯靖挾怨而叛,壓以大兵而招之,可不戰自定。」乃命能兵毋出,獨詔嚴綬招伯靖率家屬詣江陵降,授右威衛翊府中郎將。



 東謝南有西趙蠻,東距夷子,西屬昆明,南西洱河也。山穴阻深,莫知道里。南北十八日行,東西二十三日行,戶萬餘,俗與東謝同,趙氏世為酋長。夷子渠帥姓季氏,與西趙皆南蠻別種,勝兵各萬人。自古未嘗通中國,黔州豪帥田康諷之,故貞觀中皆遣使入朝。西趙首領趙酋摩率所部萬餘戶內附,以其地為明州,授酋摩刺史。



 松外蠻尚數十百部,大者五六百戶,小者二三百。凡數十姓,趙、楊、李、董為貴族,皆擅山川,不能相君長。有城郭、文字,頗知陰陽歷數。自夜郎、滇池以西,皆莊𧾷之裔。有稻、麥、粟、豆、絲、麻、薤、蒜、桃、李。以十二月為歲首。布幅廣七寸。正月蠶生,二月熟。男子氈革為帔,女衣迤布裙衫,髻盤如髽。飯用竹筲摶而啖之,烏杯貯羹如雞彞。徒跣,有舟無車。死則坎地,殯舍左,屋之,三年乃葬,以蠡蚌封棺。父母喪,斬衰布衣不澡者四五年,近者二三年。為人所殺者,子以麻括發,墨面,衣不緝。居喪,昏嫁不廢,亦弗避同姓。婿不親迎。富室娶妻,納金銀牛羊酒,女所齎亦如之。有罪者,樹一長木,擊鼓集眾其下。強盜殺之,富者貰死,燒屋奪其田;盜者倍九而償贓。奸淫,則強族輸金銀請和而棄其妻,處女、厘婦不坐。凡相殺必報,力不能則其部助攻之。祭祀,殺牛馬,親聯畢會,助以牛酒,多至數百人。貞觀中,巂州都督劉伯英上疏:「松外諸蠻,率暫附亟叛,請擊之,西洱河天竺道可通也。」居數歲,太宗以右武候將軍梁建方發蜀十二州兵進討,酋帥雙舍拒戰,敗走,殺獲十餘。群蠻震駭,走保山谷。建方諭降者七十餘部,戶十萬九千,署首領蒙、和為縣令,餘眾感悅。



 西洱河蠻,亦曰河蠻,道繇郎州走三千里,建方遣奇兵自巂州道千五百里掩之,其帥楊盛大駭,欲遁去,使者好語約降,乃遣首領十人納款軍門,建方振旅還。二十二年,西洱河大首領楊同外、東洱河大首領楊斂、松外首領蒙羽皆入朝,授官袟。顯慶元年,西洱河大首領楊棟附顯、和蠻大首領王羅祁、郎昆梨盤四州大首領王伽沖率部落四千人歸附,入朝貢方物。其後茂州西南築安戎城,絕吐蕃通蠻之道。生羌為吐蕃鄉導,攻拔之,增兵以守,西洱河諸蠻皆臣吐蕃。開元中,首領始入朝,授刺史。會南詔蒙歸義拔大和城,乃北徙,更羈制於浪穹詔。浪穹詔已破,又徙雲南柘城。



 黎州,領羈縻奉上等州二十六。開元十七年,又領羈縻夏梁、卜貴等州三十一。南路有廓清道部落主三人,婆鹽鬼主十人。又有阿逼蠻分十四部落:一曰大龍池,二曰小龍池,三曰控,四曰苴質,五曰烏披,六曰苴賃,七曰觱篥水,八曰戎列,九曰婆狄,十曰石地,十一曰羅公,十二曰言光,十三曰離旻,十四曰里漢。



 黎、邛二州之東,又有凌蠻。西有三王蠻,蓋莋都夷白馬氏之遺種。楊、劉、郝三姓世為長,襲封王,謂之「三王」部落。疊甓而居,號舍。歲稟節度府帛三千匹,以言冋南詔,而南詔亦密賂之,覘成都虛實。每節度使至,酋長來謁,節度使多奏威惠所懷,以罔天子也。前謁必請於都押衙,且聽命,都押衙不令者,輒諷其叛,常倚三王部落求姑息,至唐末益甚。



 雅州西有通吐蕃道三:曰夏陽、曰夔松、曰始陽,皆諸蠻錯居。凡部落四十六:距州三百餘里之外有百坡、當品、嚴城、中川、鉗矣、昌逼、鉗井七部落,四百餘里之外有羅巖、當馬、三井、束鋒、名耶、鉗恭、畫重、羅林、籠羊、林波、林燒、龍逢、索古、敢川、驚川、禍眉、不燭十七部落,五百餘里之外有諾祚、三恭、布嵐、欠馬、論川、讓川、遠南、卑廬、夔龍、曜川、金川、東嘉梁、西嘉梁十三部落,六百餘里之外有椎梅、作重、禍林、金林、邏蓬五部落,皆羈縻州也。以首領襲刺史。



 巂州新安城傍有六姓蠻:一曰蒙蠻、二曰夷蠻、三曰訛蠻、四曰狼蠻,餘勿鄧及白蠻也。



 戎州管內有馴、騁、浪三州大鬼主董嘉慶,累世內附,以忠謹稱,封歸義郡王。貞元中,狼蠻亦請內附,補首領浪沙為刺史,然卒不出,劍南西川節度使韋皋檄嘉慶兼押狼蠻。又有魯望等部落,徙居戎州馬鞍山,皋以其遠邊徼,戶給米二斛、鹽五斤。北又有浪稽蠻、羅哥穀蠻。東有婆秋蠻、烏皮蠻。南有離東蠻、鍋銼蠻。西有磨些蠻,與南詔、越析相姻婭。自浪稽以下,古滇王、哀牢雜種,其地與吐蕃接。亦有姐羌,古白馬氐之裔。



 劍山當吐蕃大路,屬石門、柳強三鎮,置戍、守捉,以招討使領五部落:一曰彌羌、二曰鑠羌、三曰胡叢,其餘東欽、磨些也。又有夷望、鼓路、西望、安樂、湯谷、佛蠻、虧野、阿益、阿鶚、崟蠻、林井、阿異十二鬼主皆隸巂州。又有奉國、苴伽十一部落,春秋受賞於巂州,然挾吐蕃為輕重。每節度使至,諸部獻馬,酋長衣虎皮,餘皆紅巾束發,錦纈襖、半臂。既見,請匹錦、鬥酒,折草招父祖魂以歸鄉里。及還,裹錦植馬上而去。又有顯養、東魯諸蠻,永徽三年與胡叢皆叛。高宗以右驍衛將軍曹繼叔為巂州道行軍總管,戰斜山,拔十餘城,斬首七百,獲馬、犛牛萬五千。



 姚州境有永昌蠻,居古永昌郡地。咸亨五年叛,高宗以太子右衛副率梁積壽為姚州道行軍總管討平之。武後天授中,遣御史裴懷古招懷。至長壽時,大首領董期率部落二萬內屬。其西有撲子蠻,趫悍,以青娑羅為通身褲,善用竹弓,入林射飛鼠無不中。無食器,以蕉葉藉之。人多長大,負排持槊而鬥。又有望蠻者,用木弓短箭,鏃傅毒藥,中者立死。婦人食乳酪,肥白,跣足;青布為衫裳,聯貫珂貝珠絡之;髻垂於後,有夫者分兩髻。



 群蠻種類,多不可記。有黑齒、金齒、銀齒三種,見人以漆及鏤金銀飾齒,寢食則去之。直頂為髻,青布為通褲。有繡腳種,刻踝至腓為文。有繡面種,生逾月,涅黛於面。有雕題種,身面涅黛。有穿鼻種,以金環徑尺貫其鼻,下垂過頤。君長以絲系環,人牽乃行。其次,以二花頭金釘貫鼻下出。又有長鬃種、棟鋒種,皆額前為長髻,下過臍,行以物舉之;君長則二女在前共舉其髻乃行。



 安南有生蠻林睹符部落,大歷中置德化州,戶一萬。又以潘歸國部落置龍武州,戶千五百。詔安南節度使綏定之。貞元七年,始以驩、峰二州為都督府。酹在安南,限重海,與文單、占婆接。峰統羈縻州十八,與蜀爨蠻接。



 南平獠,東距智州,南屬渝州,西接南州,北涪州,戶四千餘。多瘴癘。山有毒草、沙虱、蝮虵。人樓居,梯而上,名為干欄。婦人橫布二幅,穿中貫其首,號曰通裙。美發髻,垂於後。竹筒三寸,斜穿其耳,貴者飾以珠璫。俗女多男少,婦人任役。昏法,女先以貨求男。貧者無以嫁,則賣為婢。男子左衽,露發,徒跣。其王姓硃氏,號劍荔王。貞觀三年,遣使內款,以其地隸渝州。有飛頭獠者,頭欲飛,周項有痕如縷,妻子共守之。及夜如病,頭忽亡,比旦還。又有烏武獠,地多瘴毒,中者不能飲藥,故自鑿齒。



 有甯氏,世為南平渠帥。陳末,以其帥猛力為寧越太守。陳亡,自以為與陳叔寶同日而生,當代為天子,乃不入朝。隋兵阻瘴,不能進。猛力死,子長真襲刺史。及討林邑,長真出兵攻其後,又率部落數千從征遼東,煬帝召為鴻臚卿,授安撫大使,遣還。又以其族人甯宣為合浦太守。隋亂,皆以地附蕭銑。長真,部越兵攻丘和於交阯者也,武德初,以寧越、鬱林之地降,自是交、愛數州始通。高祖授長真欽州都督。甯宣亦遣使請降,未報而卒。以其子純為廉州刺史,族人道明為南越州刺史。六年,長真獻大珠,昆州刺史沈遜、融州刺史歐陽世普、象州刺史秦元覽亦獻筒布,高祖以道遠勞人,皆不受。道明與高州首領馮暄、談殿據南越州反,攻姜州,甯純以兵援之。八年,長真陷封山縣,昌州刺史龐孝恭掎擊暄等走之。明年,道明為州人所殺。未幾,長真死,子據襲刺史。馮暄、談殿阻兵相掠,群臣請擊之,太宗不許,遣員外散騎常侍韋叔諧、員外散騎侍郎李公淹持節宣諭。暄等與溪洞首領皆降,南方遂定。



 大抵劍南諸獠,武德、貞觀間數寇暴州縣者不一。巴州山獠王多馨叛,梁州都督龐玉梟其首,又破餘黨符陽、白石二縣獠。其後眉州獠反,益州行臺郭行方大破之。未幾,又破洪、雅二州獠,俘男女五千口。是歲,益州獠亦反,都督竇軌請擊之,太宗報曰:「獠依山險,當附以恩信。脅之以兵威,豈為人父母意耶?」貞觀七年,東、西玉洞獠反,以右屯衛大將軍張士貴為龔州道行軍總管平之。十二年,巫州獠叛,夔州都督齊善行擊破之,俘男女三千餘口。鈞州獠叛,桂州都督張寶德討平之。明州山獠又叛,交州都督李道彥擊走之。是歲,巴、洋、集、壁四州山獠叛,攻巴州,遣右武候將軍上官懷仁破之於壁州,虜男女萬餘,明年遂平。十四年,羅、竇諸獠叛,以廣州都督黨仁弘為竇州道行軍總管擊之,虜男女七千餘人。太宗再伐高麗,為舡劍南,諸獠皆半役,雅、邛、眉三州獠不堪其擾,相率叛,詔發隴右、峽兵二萬,以茂州都督張士貴為雅州道行軍總管,與右衛將軍梁建方平之。



 高宗初,琰州獠叛,梓州都督謝萬歲、充州刺史謝法興、黔州都督李孟嘗討之。萬歲、法興入洞招慰,遇害。顯慶三年,羅、竇生獠酋領多胡桑率眾內附。上元末,納州獠叛,寇故茂、都掌二縣,殺吏民,焚廨舍,詔黔州都督發兵擊之。大歷二年,桂州山獠叛,陷州,刺史李良遁去。貞元中,嘉州綏山縣婆籠川生獠首領甫枳兄弟誘生蠻為亂,剽居人,西川節度使韋皋斬之,招其首領勇於等出降。或請增柵東凌界以守,皋不從,曰:「無戎而城,害所生也。」獠亦自是不擾境。



 戎、瀘間有葛獠,居依山谷林菁,逾數百里。俗喜叛,州縣撫視不至,必合黨數千人,持排而戰。奉酋帥為王,號曰「婆能」,出入前後植旗。大中末,昌、瀘二州刺史貪沓,以弱繒及羊強獠市,米麥一斛,得直不及半。群獠訴曰:「當為賊取死耳!」刺史召二小吏榜之曰:「皆爾屬為之,非吾過。」獠相視大笑,遂叛。立酋長始艾為王,逾梓、潼,所過焚剽。刺史劉成師誘降其黨,斬首領七十餘人。餘眾遁至東川,節度使柳仲郢諭降之。始艾稽首請罪,仲郢貰遣之。



 成都西北二千餘里有附國,蓋漢西南夷也。其東部有嘉良夷,無姓氏。地縱八百里,橫四千五百里。無城柵,居川谷,疊石為巢,高十餘丈,以高下為差,作狹戶,自內以通上。王酋帥以金飾首,胸垂金花,徑三寸。地高涼,多風少雨,宜小麥,多白雉。嘉良夷有水廣三十步,附國水廣五十步,皆南流,以韋為舡。附國南有薄緣夷,西接女國。



 三濮者,在雲南徼外千五百里。有文面濮,俗鏤面,以青涅之。赤口濮,裸身而折齒,鑱其脣使赤。黑焚濮,山居如人,以幅布為裙,貫頭而系之。丈夫衣穀皮。多白蹄牛、虎魄。龍朔中,遣使與千支弗、磨臘同朝貢。



 西原蠻,居廣、容之南,邕、桂之西。有甯氏者,相承為豪。又有黃氏,居黃橙洞,其隸也。其地西接南詔。天寶初,黃氏強,與韋氏、周氏、儂氏相脣齒,為寇害,據十餘州。韋氏、周氏恥不肯附,黃氏攻之,逐於海濱。



 至德初,首領黃乾曜、真崇鬱與陸州、武陽、硃蘭洞蠻皆叛,推武承斐、韋敬簡為帥,僭號中越王,廖殿為桂南王,莫淳為拓南王,相支為南越王,梁奉為鎮南王,羅誠為戎成王,莫潯為南海王,合眾二十萬,綿地數千里,署置官吏,攻桂管十八州。所至焚廬舍,掠士女,更四歲不能平。乾元初,遣中使慰曉諸首領,賜詔書赦其罪,約降。於是西原、環、古等州首領方子彈、甘令暉、羅承韋、張九解、宋原五百餘人請出兵討承斐等,歲中戰二百,斬黃乾曜、真鬱崇、廖殿、莫淳、梁奉、羅誠、莫潯七人。承斐等以餘眾面縛詣桂州降,盡釋其縛,差賜布帛縱之。其種落張侯、夏永與夷獠梁崇牽、覃問及西原酋長吳功曹復合兵內寇,陷道州,據城五十餘日。桂管經略使邢濟擊平之,執吳功曹等。餘眾復圍道州,刺史元結固守不能下,進攻永州,陷邵州,留數日而去。湖南團練使辛京杲遣將王國良戍武崗,嫉京杲貪暴,亦叛,有眾千人,侵掠州縣。發使招之,且服且叛。建中元年,城敘州以斷西原,國良乃降。



 貞元十年,黃洞首領黃少卿者,攻邕管,圍經略使孫公器。請發嶺南兵窮討之,德宗不許,命中人招諭。不從,俄陷欽、橫、潯、貴四州。少卿子昌沔趫勇,前後陷十三州,氣益振。乃以唐州刺史陽旻為容管招討經略使,引師掩賊,一日六七戰,皆破之,侵地悉復。元和初,邕州擒其別帥黃承慶。明年,少卿等歸款,拜歸順州刺史。弟少高為有州刺史。未幾復叛。



 又有黃少度、黃昌瓘二部,陷賓、蠻二州,據之。十一年,攻欽、橫二州,邕管經略使韋悅破走之,取賓、巒二州。是歲,復屠巖州,桂管觀察使裴行立輕其軍弱,首請發兵盡誅叛者,徼幸有功,憲宗許之。行立兵出擊,彌更二歲,妄奏斬獲二萬,罔天子為解。自是邕、容兩道殺傷疾疫死者十八以上。調費斗亡,繇行立、陽旻二人,當時莫不咎之。及安南兵亂,殺都護李象古,擢唐州刺史桂仲武為都護,逗留不敢進,貶安州刺史,以行立代之。尋召還,卒。



 長慶初,以容管經略使留後嚴公素為經略使,復上表請討黃氏。兵部侍郎韓愈建言:「黃賊皆洞獠,無城郭,依山險各治生業,急則屯聚畏死。前日邕管經略使德不能綏懷,威不能臨制,侵詐系縛,以致憾恨。夷性易動而難安,劫州縣復私讎,貪小利不為大患。自行立、陽旻建征討,生事詭賞,邕、容兩管,日以凋弊,殺傷疾患,十室九空。百姓怨嗟,如出一口;人神共嫉,二將繼死。今嚴公素非撫御之才,復尋往謬,誠恐嶺南未有寧時。昨合邕、容為一道,邕與賊限一江,若經略使居之,兵鎮所處,物力雄完,則敵人不敢輕犯;容州則隔阻已甚,以經略使居之,則邕州兵少情見,易啟蠻心。請以經略使還邕州,容置刺史,便甚。又比發南兵,遠鄉羈旅,疾疫殺傷,續添續死,每發倍難。若募邕、容千人,以給行營,糧不增而兵便習,守則有威,攻則有利。自南討損傷,嶺南人希,賊之所處,洞壘荒僻。假如盡殺其人,得其地,在國計不為有益。容貸羈縻,比之禽獸,來則捍禦,去則不追,未有虧損朝庭。願因改元大慶,普赦其罪,遣郎官、御史以天子意丁寧宣諭,必能喧叫聽命。為選材用威信者,委以經略,處理得方,宜無侵叛事。」不納。



 初,邕管既廢,人不謂宜。監察御史杜周士使安南,過邕州,刺史李元宗白狀,周士從事五管,積三十年矣,亦知其不便。嚴公素遣人盜其槁,周士憤死。公素劾元宗擅以羅陽縣還黃少度,元宗懼,引兵一百持印章依少度。穆宗遣監察御史敬僚按之。僚嘗為容州從事,與公素暱,傅致元宗罪,以母老,流驩州,眾以為不直。



 黃賊更攻邕州,陷左江鎮;攻欽州,陷千金鎮。刺史楊嶼奔石南柵,邕州刺史崔結擊破之。明年,又寇欽州,殺將吏。是歲,黃昌瓘遣其黨陳少奇二十人歸款請降,敬宗納之。



 黃氏、儂氏據州十八,經略使至,遣一人詣治所,稍不得意,輒侵掠諸州。橫州當邕江官道,嶺南節度使常以兵五百戍守,不能制。大和中,經略使董昌齡遣子蘭討平峒穴,夷其種黨,諸蠻畏服。有違命者,必嚴罰之。十八州歲輸貢賦,道路清平。其後儂洞最強,結南詔為助。懿宗與南詔約和,二洞數構敗之。邕管節度使辛讜以從事徐云虔使南詔結和,齎美貨啖二洞首領、太州刺史黃伯蘊、屯洞首領儂金意、員州首領儂金勒等與之通歡。



 員州又有首領儂金澄、儂仲武與金勒襲黃洞首領黃伯善,伯善伏兵瀼水,雞鳴,候其半濟,擊殺金澄、仲武,唯金勒遁免。後欲興兵報仇,辛讜遣人持牛酒音樂解和,並遺其母衣服。母,賢者也,讓其子曰:「節度使持物與獠母,非結好也,以汝為吾子。前日兵敗龕水,士卒略盡,不自悔,復欲動眾,兵忿者必敗,吾將囚為官老婢矣。」金勒感寤,為罷兵。



 贊曰:唐北禽頡利,西滅高昌、焉耆,東破高麗、百濟,威制夷狄,方策所未有也。交州,漢之故封,其外瀕海諸蠻,無廣土堅城可以居守,故中國兵未嘗至。及唐稍弱,西原、黃洞繼為邊害,垂百餘年。及其亡也,以南詔。《詩》曰:「惠此中國,以綏四方。」不以夷狄先諸夏也。



\end{pinyinscope}