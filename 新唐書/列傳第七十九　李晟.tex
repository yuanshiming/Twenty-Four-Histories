\article{列傳第七十九 李晟}

\begin{pinyinscope}

 李晟,字良器,洮州臨潭人。世以武力仕,然位不過裨將。晟幼孤,奉母孝。身長六尺。年十八教育家(約前372—前289)。名軻,字子輿,鄒(今山東鄒,往事河西王忠嗣,從擊吐蕃。悍酋乘城,殺傷士甚眾,忠嗣怒,募射者,晟挾一矢殪之,三軍歡奮。忠嗣撫其背曰:「萬人敵也。」鳳翔節度使高升召署列將,擊疊州叛羌於高當川,又擊連狂羌於罕山,破之。累遷左羽林大將軍。廣德初,擊黨項有功,授特進,試太常卿。



 大歷初,李抱玉署晟右軍將。吐蕃寇靈州,抱玉授以兵五千擊之,辭曰:「以眾則不足,以謀則多。」乃請千人。由大震關趨臨洮,屠定秦堡,執其帥慕容穀鐘,虜乃解靈州去。遷開府儀同三司,以右金吾衛大將軍為涇原、四鎮、北庭兵馬使。馬璘與吐蕃戰鹽倉,敗績;晟率游兵拔璘以歸,封合川郡王。璘內忌晟威略,歸之朝,為右神策都將。德宗始立,吐蕃寇劍南,方崔寧未還,蜀土大震,詔晟將神策兵救之。逾漏天,拔飛越等三城,絕大渡,斬虜千級,虜遁去。



 建中二年,魏博田悅反,晟為神策先鋒,與河東馬燧、昭義李抱真合兵攻之。斬楊朝光,晟乘冰度洺水,破悅;又戰洹水,悅大敗,遂進攻魏。加檢校左散騎常侍,兼魏府左司馬。硃滔、王武俊圍康日知於趙州也,抱真分兵二千戍邢,燧怒,欲班師。晟曰:「奉詔東討者,吾三帥也。邢、趙比壤,今賊以兵加趙,是邢有晝夜憂,李公分眾守之,不為過,公奈何遽引去!」燧悟,釋然,即造抱真壘,與交歡。晟建言:「以兵趨定州,與張孝忠合,以圖範陽,則武俊等當舍趙。」帝壯之,授御史大夫,又俾神策三將軍莫仁擢等隸之。晟自魏引而北,武俊果解去。晟留趙三日,與孝忠連兵,北略恆州。圍硃滔將鄭景濟於清苑,決水灌之。悅、武俊引兵戰白樓,孝忠兵笮,晟引步騎擊破之,清苑益急。滔、武俊大懼,悉起兵來救,圍晟軍。晟內攻景濟而外抗滔等,自正月至五月不解。會晟疾甚,不能興,軍中共計引還定州,而賊猶不敢逼。



 疾間,將復進,會帝出奉天,有詔召晟即日治嚴。而孝忠以軍介二盜間,倚晟為重,數止晟無西。晟語眾曰:「天子播越,人臣當百舍一息。義武欲止吾,吾當以子為質。」乃以憑約昏,並遺良馬。孝忠有親將謁晟,晟解玉帶遺之,使喻孝忠。乃得逾飛孤,次代州,詔迎拜神策行營節度使。進臨渭北,壁東渭橋,所過樵蘇無犯。時劉德信自扈澗敗歸,亦次渭南,軍囂無制。德信入謁晟,晟責所以敗,斬之,以數騎入壁勞其軍,無敢動。晟已並兵,則軍益振。



 於是朔方李懷光方軍咸陽,不欲晟當一面,請與晟合。有詔徙屯,乃引趨陳濤斜,與懷光聯壘。晟每與賊戰,必錦裘繡帽自表,指顧陣前。懷光望見,惡之,戒曰:「將務持重,豈宜自表襮,為賊餌哉!」晟曰:「昔在涇原,士頗相畏伏,欲令見之,奪其心爾。」懷光不悅,遷延有異志。晟使間說懷光曰:「賊據京邑,天子暴露於外,公宜速進兵。雖晟不肖,願為公先驅,死且不悔。」懷光不納。每兵至都城下,而懷光軍多鹵掠,晟軍整戢。懷光使分所獲遺之,又辭不敢受。懷光謀沮撓其軍,即奏言:「神策兵給賜比方鎮獨厚,今桀逆未平,軍不可以異。且眾以為言,臣無以解。惟陛下裁處。」懷光欲晟自削其軍,則士怨易撓。帝議諸軍與神策等,力且不贍,遣翰林學士陸贄臨詔懷光,令與晟計所宜者。懷光曰:「稟賜不均,軍何以戰!」贄數顧晟,晟曰:「公,元帥,軍政得專之。晟將一軍,唯所命,其增損費調,敢不聽?」懷光默然計塞,顧刻削稟賜事出己,乃止。



 懷光屯咸陽凡八旬,帝數促戰,以伺賊隙為言,卒不出兵,陰通硃泚,反跡浸露。晟懼為所並,上言:「當先變制備,請假裨佐趙光銑、唐良臣、張彧為洋、利、劍三州刺史,各勒兵以通蜀、漢衿喉。」未報。會吐蕃欲佐誅泚,帝議幸咸陽督戰,懷光大駭,疑帝奪其軍,圖反益急。晟與李建徽、陽惠元皆聯屯,適有使者到晟軍,晟乃令曰:「有詔徙屯。」即結陣趨東渭橋。後數日,懷光並建徽、惠元兵,惠元死之。



 是日,帝進狩梁州,駱谷道隘,儲供不豫,從官乏食,帝嘆曰:「早用晟言,三蜀之利,可坐有也。」顧渾瑊曰:「渭橋在賊腹中,兵孤絕,晟能辦勝邪?」瑊曰:「晟秉義挺忠,崒然不可奪。臣策之,必破賊。」帝乃安。自行在遣晟將張少弘口詔進晟尚書左僕射、同中書門下平章事。晟受命,拜且泣曰:「京師天下本,若皆執羈靮,誰將復之!」乃繕甲兵,治陴隍,以圖收復。



 是時,晟提孤軍橫當寇鋒,恐二盜合以軋之,則卑詞厚幣,偽致誠於懷光者。時敖廥單覂,乃使張彧假京兆少尹,多署吏,調畿內賦,不淹旬,芻米告具。乃陳兵下令曰:「國家多難,乘輿播遷,見危死節,自吾之分。公等此時不誅元兇,取富貴,非豪英也。渭橋斷賊首尾,吾欲與公戮力一心,建不世之功,可乎?」士皆雪泣曰:「惟公命。」於是駱元光以華州之眾守潼關,尚可孤以神策兵保七盤,皆受晟節度;戴休顏舉奉天,韓游瑰悉邠寧軍從晟。懷光始懼。晟乃移書顯讓之,使破賊自贖。懷光不聽,然其下益攜落,畏為晟襲,乃奔河中。其將孟涉、段威勇以兵數千自拔歸,晟皆表以要官。帝遣使者間道詔晟兼河中、晉絳慈隰節度使,又兼京畿、渭北、鄜坊、丹延節度招討使。帝欲益西幸,晟請駐梁、漢以系天下望。又進京畿、渭北、鄜坊、商華兵馬副元帥。時京兆司錄參軍李敬仲自賊中來,乃署節度府判官,以諫議大夫鄭雲逵為行軍司馬,擢張彧自副。



 神策軍及晟家皆為賊質,左右有言者,晟涕數行下,曰:「陛下安在,而欲恤家乎?」泚使晟吏王無忌婿款壁門曰:「公等家無恙。」晟怒曰:「爾乃與賊為間乎?」叱斬之。時輸縑不屬,盛夏,士有衣裘者,晟能與下同其苦,以忠誼感發士心,終無攜怨。邏士得姚令言、崔宣諜者,晟命釋縛,飯飲之,遣還,敕曰:「為我謝令言等,善為賊守,勿不忠於泚。」



 乃引兵叩都門,賊不敢出,振旅而還。明日,會諸將圖所向,眾對先拔外城,然後清宮。晟曰:「外城有里閈之隘,若設伏格戰,居人囂潰,非計也。賊重兵精甲聚苑中,今直擊之,是披其心腹,將圖走不暇。」諸將曰:「善。」乃自東渭橋移壁光泰門,以薄都城,連溝柵。而賊將張庭芝、李希倩求戰,晟顧曰:「賊不出,是吾憂也。今乃冒死來,天誘之矣。」勒吳詵等縱兵鏖擊。賊攻華師急,晟以精騎馳救,中軍噪而從,大破之,乘勝入光泰門;再戰,敗卻,殭尸相藉,餘眾走白華,賊大哭,終夜不息。翌日,將復戰。或請待西師,晟曰:「賊既敗,當乘機撲殄。茍俟西軍,是容其為計,豈吾利邪?」乃悉軍軍光泰門,使王佖、李演將騎,史萬頃將步,抵苑北。晟先夜隤苑垣為道二百步,比兵至,賊已伐木塞以拒戰。晟叱諸將曰:「安得縱賊?今先斬公矣!」萬頃懼,先登,拔柵以入,佖督騎繼之。賊崩潰,執其將段誠諫,大兵分道進,雷噪震地。令言、庭芝、希倩等殊死鬥,晟令唐良臣等步騎奔突,賊陣成輒北,十餘遇皆不勝,蹙入白華。賊伏千騎出官軍背,晟以麾下百騎自馳之,左右呼曰:「相公來!」賊驚潰,禽馘略盡。泚率殘卒萬人西走,田子奇追之,餘黨悉降。



 晟引軍屯含元外廷,舍右金吾次,令軍中曰:「五日內不得輒通家問,違者斬。」遣京兆尹李齊運部長安、萬年令,分慰居人,秋毫無所擾。別將高明曜取賊妓一,司馬伷取賊馬二,即斬以徇。坊人之遠者,宿昔乃知王師之入也。明日,孟涉屯白華,尚可孤屯望仙門,駱元光屯章敬寺,晟屯安國寺。斬賊用事者及臣賊宦豎於市,表著節不屈者,擇文武攝臺省官,以俟乘輿。條脅污於賊者,請以不死。



 露布至梁,帝感泣,群臣上壽,且言:「晟蕩夷兇憝,而市不易廛,宗廟不震,長安之人不識旗鼓,雖三代用師,不能加之。」帝曰:「天生晟,為社稷萬人,豈獨朕哉!」拜晟司徒,兼中書令,實封千戶。晟遣大將吳詵以兵三千到寶雞清道,自請迎扈,不許。帝至自梁,晟以戎服見三橋,帝駐馬勞之。晟再拜頓首,賀克殄大盜,廟朝安復,已,即跪陳:「備爪牙臣,不能指日破賊,致乘輿再狩,乃臣不任職之咎,敢請死。」伏道左,帝為掩涕,命給事中齊映起之,使就位。有詔賜第永崇里、涇陽上田、延平門之林園、女樂一列。晟入第,京兆供帳,教坊鼓吹迎導,詔將相送之。帝紀其功,自文於碑,敕皇太子書,立於東渭橋,以示後世云。又令太子錄副以賜。



 始,晟屯渭橋也,熒惑守歲,久乃退,府中皆賀曰:「熒惑退,國家之利,速用兵者昌。」晟曰:「天子暴露,人臣當力死勤難,安知天道邪?」至是,乃曰:「前士大夫勸晟出兵,非敢拒也。且人可用而不可使之知也。夫惟五緯盈縮不常,晟懼復守歲,則我軍不戰自屈矣!」皆曰:「非所及也。」



 涇州倚邊,數戕其帥,晟請治不龔命者,因以訓耕積粟實塞下,羈制西戎。帝乃拜晟鳳翔、隴右、涇原節度使,兼行營副元帥,徙王西平郡,實封千五百戶。晟請與李楚琳俱行,亦將治殺張鎰罪,帝方務安反側,不許。晟至鳳翔,亂將王斌等十餘人以次伏誅。時宦者尹元貞持節到同、華,擅入河中諭慰李懷光。晟劾元貞矯使,欲洗宥元惡,請治罪。又言:「赦懷光有五不可:河中抵京師三百里,同州制其沖,兵多則示未信,少則力不足,忽驚東偏,何以待之?一也。今赦懷光,則必以晉、絳、慈、隰還之,渾瑊、康日知又且遷徙,二也。兵力未窮,忽宥反逆,四夷聞之,謂陛下兵屈而自罷耳;今回紇拒北,吐蕃梗西,希烈僭淮、蔡,若棄強示弱,以招窺覬,三也。懷光既赦,則朔方將士悉復敘勛行賞,追還縑廩;今府庫空殫,物不酬滿,是激其叛,四也。既解河中,諸道還屯,當有賜賚,賞典不舉,怨言必起,五也。今河中米斗五百,芻稿且罄,人餓死墻壁間,其大將殺戮幾盡,圍之旬時,力窮且潰,願無養腹心疾為後憂。臣請選精兵五千,約十日糧,可以破賊。」帝方以賊委馬燧、渾瑊,故不許。



 晟至涇而田希鑒迎謁,執之,並其黨石奇等悉伏誅。表右龍武將軍李觀為涇原節度使。晟常曰:「河、隴之陷,非吐蕃能取之,皆將臣沓貪,暴其種落,不得耕稼,日益東徙,自棄之爾。且土無繒絮,人苦役擾,思唐之心,豈有既乎?」因悉家貲懷輯降附,得大酋浪息曩,表以王號。每虜使至,必召息曩於坐,衣大錦袍、金帶,誇異之,虜皆指目歆艷。吐蕃君臣大懼,相與議。尚結贊者善計,乃曰:「唐名將特李晟與馬燧、渾瑊爾,不去之,必為吾患。」即遣使委辭,因燧請和,且求盟,因盟謀執瑊以賣燧,於是結贊大興兵逾隴、岐,無所掠,陽怒曰:「召吾來,乃不牛酒犒軍。」徐引去。以是間晟。晟選兵三千,使王佖伏汧陽旁,擊其中軍,幾獲結贊。晟又遣野詩良輔等攻摧沙堡,拔之。結贊屢乞和,會晟朝京師,奏言:「戎狄無信,不可許。」宰相韓滉與晟合,因請調軍食以給西師。然天子內厭兵,疑將臣生事。亦會滉卒而張延賞當國,故與晟有隙,後雖詔講解,而陰不與也,密言晟不可久持兵,更薦劉玄佐、李抱真經略西北,俾立功以間晟。帝惑其言。



 貞元三年,帝坐宣政殿引見晟,備冊禮,進拜太尉、中書令,罷其兵。詔晟乘輅謁太廟,視事尚書省,賜良馬、錦彩千計。是歲,瑊與吐蕃盟平涼,虜劫之,瑊挺身免,詔罷燧河東,皆如結贊計云。通王府長史丁瓊者,嘗為延賞擠抑,內怨望,乃見晟曰:「以公功,乃奪兵柄,夫惟位高者難全,盍早圖之?」晟曰:「君安得不祥之言?」執以聞。



 明年,詔為晟立五廟,追賁高祖芝以下祔其主,給牲器床幄,禮官相事。它日,與馬燧見延英,帝嘉其勛,下詔曰:「昔我烈祖,乘乾坤蕩滌,掃隋季荒茀,體元禦極,作人父母。則有熊羆之士,不二心之臣,左右經綸,參翊締構,昭文德,恢武功,威不若,康不乂,用端命於上帝,付畀四方。王業既成,太階既平,乃圖厥容,列於凌煙閣,懋昭績效,表式儀形,以弗忘朝夕,永垂乎來裔。君臣之義,厚莫重焉。歲在己巳秋九月,我行西宮,瞻望崇構,見老臣遺像,顒然肅然,和敬在色。想雲龍之協期,感致業之艱難,睹往思今,取類非遠。且功與時並,才與世生,茍蘊其才,遇其時,尊主庇人,何代蔑有?在中宗時,有如桓彥範等,著輔戴之績;在玄宗時,有如劉幽求等,申弼翼之勛;在肅宗時,有如郭子儀,掃除氛祲。今顧晟等,保寧朕躬,咸宣力肆勤,光復宗祏,訂之前烈,夫豈多謝。闕而未錄,孰旌厥賢?況念功紀德,文祖所為也,在予其曷敢怠?有司宜敘先後,各圖其象於舊臣之次。」命皇太子書其文以賜晟,晟刻石於門。



 七年,以臨洮未復,請附貫萬年,詔可。九年,薨,年六十七。帝聞流涕,詔百官就第進吊。比大斂,帝手詔,誓以存保世嗣,申告柩前。冊贈太師,謚曰忠武。及葬,又禦望春門臨送,遣謁者宣詔於柩車,百官拜哭於道。憲宗元和中,詔其家與屬籍,以晟配饗德宗廟廷。僖宗狩蜀,倉部員外郎袁皓採晟功烈,為《興元聖功錄》,遍賜諸將,表勵之。



 晟性疾惡,臨下明。每治軍,必曰:「某有勞,某長於是。」雖廝養小善,必記姓名,尤惡下為朋黨者。篤分義,隆於故舊。嵐州刺史譚元澄嘗有德於晟,後貶死。晟既貴,直其枉,詔贈元澄寧州刺史,晟撫其二子,為成就之。在鳳翔,嘗曰:「魏徵以直言致太宗於堯舜上,忠臣也。我誠慕焉。」行軍司馬李叔度曰:「彼縉紳儒者事,公勛德何希是哉?」晟曰:「君失辭。晟幸得備將相,茍容身不言,豈可謂有犯無隱邪?是非唯上所擇爾。」叔度慚。故晟每進對,謇謇盡大臣節,未嘗露於外。治家以嚴,子侄非晨昏不輒見,所與言未嘗及公事。正歲,崔氏女歸寧,讓曰:「爾有家,而姑在堂,婦當治酒食,且以待賓客。」即卻之,不得進。達禮敦教類若此。與馬燧皆在朝,每宴樂恩賜,使者相銜於道。兩家日出無鐘鼓聲,則金吾以聞,少選,使者至,必曰:「今日何不舉樂?」既薨,城鹽州,復故池,以新鹽賜宰相。帝思晟,乃致鹽靈座。其眷遇終始,無與比者。



 有十五子,其聞者願、憲、愬、聽雲。



 願少謙謹。晟立功時,諸子未官,宰相以聞,即日召授太子賓客、上柱國。故事,柱國門列戟,遂父子皆賜。元和初,領夏綏銀宥節度使。政簡而嚴。部有失馬者,願署牒於道,以金購之。三日,失馬並良馬一系署下,且曰:「逸而至,不告,罪當死,謹以良馬贖。」願歸失馬,而縱其良,境內肅然。徙節武寧軍。會伐青、鄆,數有功,以久疾,用愬代之。召為刑部尚書,俄檢校尚書左僕射,節度鳳翔,自是邇聲色而政衰矣。



 長慶中,徙宣武。始,張弘靖給其軍頗厚;願至,府庫殫匱,賞賚不及弘靖時,而侈費過之。以威刑操下,用婚家竇緩典帳中兵,驕驁怠沓。牙將李臣則等因眾不忍,夜斬緩首。願聞變,不及巾,與左右數人縋而逸,奪野人乘,馳以免。其家死於兵,三子匿而免。兵既亂,因大掠,推李朅主後務,請諸朝。時責願不職,貶隋州刺史。入為左金吾衛大將軍,復拜河中、晉、絳等節度使。雖嘗以荒侈敗,不能自悛,軍政愈弛,結納權近,官貲隨賂遺輒盡。蒲人怨,且亂。會卒,贈司徒。



 憲與愬於諸子號最仁孝。長喜儒,以禮法自矜制。調太原府參軍事、醴泉尉。于頔鎮襄陽,闢署於府。時吳少誠張淮西,獨憚頔威強,時謂憲為之助。又闢魏博田弘正幕府,遷衛州刺史,以治行稱。徙絳州。絳有幻人訹民以亂,憲執誅之。河中兵本仰食於絳,而汾可輸河、渭,歲租與糴常數十萬石,故敖保山為固,民之輸者,十牛不勝一車。憲濱汾相地治新倉,當費二百萬,請留垣縣粟糶河南,以錢還糴絳粟,既免負載勞,又權其贏以完新倉,絳人賴利。入為宗正少卿,副金吾大將軍胡證為送太和公主使。還,獻《回鶻道里記》,遷太府卿。太和初,繇江西觀察使遷嶺南節度使。



 憲,勛伐家子,所歷皆以吏能顯,政績暴著。善治律令,性明恕,詳正大獄,活無罪者數百人。卒官下。



 愬,字元直,有籌略,善騎射。以廕補協律郎,遷累衛尉少卿。早喪所生,為晉國王夫人所鞠。王卒,晟以非嫡,敕諸子服緦,愬獨號慟不忍,晟乃許服縗。既練,晟薨,與憲廬墓側,德宗敦遣歸第,一夕復往,帝許之。服除,授太子右庶子。出為坊、晉二州刺史,以治異等,加金紫光祿大夫,進詹事。



 憲宗討吳元濟,唐鄧節度使高霞寓既敗,以袁滋代將,復無功。愬求自試,宰相李逢吉亦以愬可用,遂檢校左散騎常侍,為隋唐鄧節度使。愬以其軍初傷夷,士氣未完,乃不為斥候部伍。或有言者,愬曰:「賊方安袁公之寬,吾不欲使震而備我。」乃令於軍曰:「天子知愬能忍恥,故委以撫養。戰,非吾事也。」眾信而安之。乃斥倡優,未嘗嬉樂。士傷夷病疾,親為營護。蔡人以嘗敗辱霞寓等,又愬名非夙所畏者,易之,不為備。愬沈鷙,務推誠待士,故能張其卑弱而用之。賊來降,輒聽其便,或父母與孤未葬者,給粟帛遣還,勞之曰:「而亦王人也,無棄親戚。」眾願為愬死,故山川險易與賊情偽,一能曉之。



 居半歲,知士可用,乃請濟師;詔益河中、鄜坊二千騎。於是繕鎧厲兵,攻馬鞍山,下之;拔道口柵,戰嵖岈山,以取爐冶城;入白狗、汶港柵,披楚城,襲朗山,再執守將。平青陵城,禽驍將丁士良,異其才,不殺,署捉生將。士良謝曰:「吳秀琳以數千兵不可破者,陳光洽為之謀也。我能為公取之。」乃禽以獻。於是秀琳舉文城柵降。遂以其眾攻吳房,殘外垣。始出攻,吏曰:「往亡日,法當避。」愬曰:「彼謂吾不來,此可擊也。」既引還,賊以精騎尾擊。愬下馬據胡床,令軍曰:「退者斬。」眾決死戰,射殺其將,賊乃走。或勸遂取吳房,愬曰:「不可。吳房拔,則賊力專,不若留之以分其力。」



 初,秀琳降,愬單騎抵柵下與語,親釋縛,署以為將。秀琳為愬策曰:「必破賊,非李祐無與成功者。」祐,賊健將也,守興橋柵,其戰嘗易官軍。愬候祐護獲於野,遣史用誠以壯騎三百伏其旁,見羸卒若將燔聚者,祐果輕出,用誠禽而還。諸將素苦祐,請殺之,愬不聽,以為客。待間,召祐及李忠義屏人語,至夜艾。忠義,亦賊將,所謂李憲者。軍中多諫此二人不可近,愬待益厚。乃募死士三千人為突將,自教之。會雨,自五月至七月不止,軍中以為不殺祐之罰,將吏雜然不解。愬力不能獨完祐,乃持以泣曰:「天不欲平賊乎?何見奪者眾邪?」則械而送之朝,表言必殺祐,無與共誅蔡者。詔釋以還愬。愬乃令佩刀出入帳下,署六院兵馬使。六院者,隋、唐兵也,凡三千人,皆山南奇材銳士,故委祐統之。祐捧檄嗚咽,諸將乃不敢言,由是始定襲蔡之謀矣。舊令,敢舍諜者族。愬刊其令,一切撫之,故諜者反效以情,愬益悉賊虛實。



 時李光顏戰數勝,元濟悉銳卒屯洄曲以抗光顏。愬知其隙可乘,乃遣從事鄭澥見裴度告師期,於時元和十一年十月己卯。師夜起,祐以突將三千為前鋒,李忠義副之,愬率中軍三千,田進誠以下軍殿。出文城柵,令曰:「引而東。」六十里止,襲張柴,殲其戍。敕士少休,益治鞍鎧,發刃彀弓。會大雨雪,天晦,凜風偃旗裂膚,馬皆縮慄,士抱戈凍死於道十一二。張柴之東,陂澤阻奧,眾未嘗蹈也,皆謂投不測。始發,吏請所向,愬曰:「入蔡州取吳元濟!」士失色,監軍使者泣曰:「果落祐計。」然業從愬,人人不敢自為計。愬道分輕兵斷橋以絕洄曲道,又以兵絕朗山道。行七十里,夜半至懸瓠城,雪甚,城旁皆鵝鶩池,愬令擊之,以亂軍聲。賊恃吳房、朗山戍,晏然無知者。祐等坎墉先登,眾從之,殺門者,發關,留持柝傳夜自如。黎明,雪止,愬入駐元濟外宅。蔡吏驚曰:「城陷矣!」元濟尚不信,曰:「是洄曲子弟來索褚衣爾。」及聞號令曰:「常侍傳語。」始驚曰:「何常侍得在此!」率左右登牙城,田進誠兵薄之。愬計元濟且望救於董重質,乃訪其家慰安之,使無怖,以書召重質;重質以單騎白衣降,愬待以禮。進誠火南門,元濟請罪,梯而下,檻送京師。



 申、光諸屯尚二萬眾,皆降,愬不戮一人。其為賊執事帳內廚廄廝役,悉用其舊,使不疑。乃屯兵鞠場以俟裴度。至,愬以櫜鞬見,度將避之,愬曰:「此方廢上下分久矣,請因示之。」度以宰相禮受愬謁,蔡人聳觀。乃還屯文城柵。有詔進檢校尚書左僕射、山南東道節度使,封涼國公,實封戶五百,賜一子五品官。



 帝方經略隴右,故徙愬節度鳳翔。李師道反,詔愬代願帥武寧軍。旬日踐父兄兩鎮,世以為榮。董重質得罪被斥,愬請賜軍中自效,許之,乃署為牙將。愬與賊戰金鄉,破之。凡十一遇,禽其隊帥五十,俘馘萬計。淄青平,進同中書門下平章事,徙昭義節度,賜第興寧里。會田弘正守鎮州,乃以愬帥魏博。長慶初,幽、鎮亂,殺弘正,愬素服以令軍曰:「魏人富庶而通於天化者,田公力也。上以其愛人,使往治鎮。且田公撫魏七年,今鎮人不道而戕害之,是無魏也。父兄子弟食田公恩者,何以報之?」眾皆哭。又以玉帶、寶劍遺牛元翼,曰:「此劍吾先人嘗以揃大盜,吾又以平蔡奸。今鎮人逆天,公宜用此夷之也。」元翼感動,謝曰:「敢有不承而愛其死力!」乃下令軍中,勒兵以俟。會愬疾甚,不能軍,詔田布代之,以太子少保還東都。卒,年四十九,贈太尉,謚曰武。



 愬行己儉約。其昆弟賴家勛貴,飾輿馬,矜室廬,唯愬所處乃父時故院,無所增廣。始,晟克京師,市不改肆,愬平蔡,亦如之。功名之奇,近世所未有。晚雖忽於取士,與鄭注善,議者不以掩其賢。



 贊曰:愬得李祐不殺,付以兵不疑,知可以破賊也。祐受任不辭,決策入死,以愬能用其謀也。祐之才,待愬乃顯,故曰平蔡功,愬為多。



 聽,字正思,七歲以廕為協律郎,父吏少之,不甚敬,聽輒使鞭之,晟奇其才。長乃闢佐于頔府。吐突承璀討王承宗,以聽為神策行營兵馬使。既戰,斬賊驍將,憲宗壯之,詔圖狀以獻。承璀數問聽計,卒縛盧從史。遷左驍衛將軍,出為蔚州刺史。州有銅冶,自天寶後廢不治,民盜鑄不禁。聽乃開五爐,官鑄錢日五萬,人無犯者。徙安州。會觀察使柳公綽方討蔡,以聽典軍,一一咨之,聲振賊中。召為羽林將軍。



 帝討李師道,出聽楚州刺史。淮西兵綿弱,鄆人素易之。聽日整勒,士皆奮。即掩賊不虞,趨漣水,破沭陽,絕龍沮堰,遂取海州,攻朐山,降之,懷仁、東海兩城望風送款。以功兼御史大夫,夏綏銀宥節度使。又徙靈鹽。部有光祿渠,久廞廢,聽始復屯田以省轉餉,即引渠溉塞下地千頃,後賴其饒。進檢校工部尚書。



 穆宗初立,幽、鎮反,擇名臣節度太原者代裴度,使統兵北討。始聽為羽林時,有駿馬,帝在東宮,使左右諷取之,聽自以身宿衛,不敢獻。於是帝曰:「李聽往在軍中,不與朕馬,是必可任。」乃授檢校兵部尚書,充河東節度使。敬宗嗣位,改義成軍。太和初,討李同捷,而魏博將丌志沼反,擊其帥史憲誠,詔聽出援,擊殺志沼。以功封涼國公,拜一子五品官。



 王廷湊之亂,詔聽悉兵屯貝州,史憲誠懼聽因取道襲之,衷甲候諸郊。聽敕士櫜兵野次,魏人乃安。憲誠既請朝,魏人怨,詔聽兼帥魏博。聽遷延不即赴,魏遂亂,殺憲誠,共推大將何進滔乘城拒守。聽不得入,乃屯館陶。又不設備,魏人襲之,師驚潰,死失殆半,輜械盡棄之,聽晝夜馳以免。於是御史中丞溫造等劾奏魏州亂,憲誠死,職繇於聽,請論如法。天子不罪也,罷為太子少師。



 聽素以賂遺得權幸心,故多為助力。未幾,拜邠寧節度使。邠署相傳不利治垣舍,前刺史視其壞,莫敢葺。聽曰:「將出鑿兇門,何避治署邪?」亟使完新之,卒無異。改帥武寧軍。有故奴為徐將,不喜聽來,乃先殺親吏之使徐者以沮聽。聽果懼,以疾解,授太子少保。逾歲,節度鳳翔,又徙陳許。鄭注摭其過,詔以太子太保分司東都。開成初,為河中晉絳慈隰節度使。文宗嘆曰:「付之兵不疑,退處散地不怨,惟聽為可。」四年,以疾求還,復拜太子太保。卒,年六十一,贈司徒。



 聽治官苛細,急揫斂,頗極所欲,盛飾車馬服玩。或誡之,聽曰:「家聲在人,若示衰薄,恐不見忠功之效,吾欲誇而勸之也。」好方書,擇其驗者,題於帷帟墻屋皆滿。



 聽子琢,以家閥擢累義昌、平盧、鎮海三節度使,無顯功,不為士大夫稱道。數免復遷。廣明時,沙陀數盜邊,於是琢為宿將,拜檢校尚書右僕射,蔚朔等州招討、都統、行營節度使。徙河陽三城,坐逗撓,下遷刺史,卒。



 王佖者,晟之甥,武敢,閑騎射。晟在師,佖無不從。攻硃泚於光泰門,賊方銳,佖與李演鏖戰蹀血,賊數北,諸軍乘之,遂大振。以功擢神策將。擊吐蕃有功。晟視佖與子姓等,其給與過之。晟兵罷,佖亦不見用,召為左衛上將軍。元和中,拜朔方、靈鹽節度使。吐蕃欲作烏蘭橋以過師,積材河曲,朔方府常遣兵發其木,委於河,故莫能成。及佖至,虜知其寡謀,乃厚賂之而亟遂功,築月城以守。自是虜歲入為寇,朔方乘障不暇,人以咎佖。在鎮檢下亡術,猜忌多殺人。召還為右衛將軍。故事,將相除徙,皆內出制,故號「白麻」;至佖,以責罷,遂中書進制。久之,卒。



 贊曰:晟之屯東渭橋也,硃泚盜京師,李懷光反咸陽,河北三叛相王,李納猘河南,李希烈訌鄭、汳。晟無積貲輸糧,提孤軍抗群賊,身佩安危而氣不少衰者,徒以忠誼感人,故豪英樂為之死耳。至師入長安而人不知,雖三王之佐,無進其能,可謂仁義將矣!嗚呼,功能存社祏,不能見信於庸主,卒奪其兵,哀哉!雖然,功蓋天下者,惟退禍可以免。四子世似其勞,是宜有後哉。



\end{pinyinscope}