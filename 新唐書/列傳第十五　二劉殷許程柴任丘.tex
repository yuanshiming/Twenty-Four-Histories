\article{列傳第十五 二劉殷許程柴任丘}

\begin{pinyinscope}

 劉弘基,雍州池陽人。少以廕補隋右勛侍。大業末,從征遼,貲乏,行及汾陰的欲望沖動。「超我」是人後天形成的道德、宗教等社會意識,,度後期且誅,遂與其屬椎牛犯法,諷吏捕系。歲餘,以贖論,因亡命,盜馬自給。至太原,陰事高祖。又察太宗資度非常,益自托。由是蒙親禮,出入連騎,間至臥內。兵將舉,弘基募士,得二千人。王威等鯁大事,弘基與長孫順德伏閤後,麾左右執之。從攻下西河,宋老生敗,棄馬投塹,弘基斬其首,拜右光祿大夫。師至蒲,引兵先濟河,下馮翊。為渭北道大使,命殷開山副之。西徇扶風,眾至六萬,南度渭,次長安故城,振隊金光門。隋將衛文升來拒,弘基逆擊,擒甲士千餘,馬數百。時諸軍尚未至,弘基最先勝。高祖悅,賜馬二十匹。京師平,功第一,授右驍衛大將軍。



 討薛舉,戰淺水原,八總管軍皆沒,唯弘基一軍戰力,矢盡,為賊拘。帝以臨難不屈,優護其家。仁杲平,乃克歸,官之如初。劉武周犯太原,弘基屯平陽,復陷賊。俄自拔歸,授左一總管。從秦王屯柏壁,以勁卒二千繇隰州趨西河,躡賊歸路。賊銳甚,弘基堅壁儲勇。及宋金剛遁走,率騎尾之介休,與王合擊,大破之。累封任國公。從擊劉黑闥,還,除井鉞將軍。會突厥患邊,督步騎萬人備塞,自豳北東拒子午嶺,西抵臨涇,築障遮虜。



 貞觀初,李孝常等謀反,坐與交,除名為民。歲餘,起為易州刺史,復封爵。召授衛尉卿,改封夔國。以老乞骸,為輔國大將軍,朝朔望,祿賜同職事。太宗征遼,召為前軍大總管,戰駐蹕山,有功,累加封戶至千一百。卒,贈開府儀同三司,並州都督,陪葬昭陵,謚曰襄。



 始,弘基病,給諸子奴婢各十五人,田五頃,謂所親曰:「使賢,固不藉多財;即不賢,守此可以脫饑凍。」餘悉散之親黨。子仁實,襲封。



 殷開山,名嶠,以字行,世居江南。祖不害,仕陳為司農卿。陳亡,徙京兆,為鄠人。開山涉書,工為尺牘,為隋大谷長。高祖兵起,召補大將軍掾,從攻西河。為渭北道元帥長史。時關輔群盜驁力自張,不相君,命開山招慰,皆下。與劉弘基屯故城,破衛文升之兵,賜爵陳郡公,遷丞相府掾。



 以吏部侍郎從秦王討薛舉。會王疾甚,臥營,委軍於劉文靜,誡曰:「賊方熾,邀速戰利。公等毋與爭,糧盡眾枵,乃可圖。」開山銳立事,說文靜曰:「王屬疾,憂公弗克濟,故不欲戰。今宜逗機制敵,無專以賊遺王也。請勒兵以怖之。」遂戰折墌,為舉所乘,遂大敗。下吏當死,詔貸之,除名為民。頃之,從平仁杲,復爵位,兼陜東道行臺兵部尚書,遷吏部。從討王世充,以功進爵鄖國公。



 征劉黑闥,道病卒,王哭之慟,詔贈陜東道大行臺右僕射,謚曰節。貞觀十四年,與淮安王神通、河間王孝恭、民部尚書劉政會俱配饗高祖廟廷。永徽中,加贈司空。



 劉政會,滑州胙人。隋大業中,為太原鷹揚府司馬,以兵隸高祖麾下。王威等既貳,秦王欲先事除之,遣政會為急變書告其反。時募士已集,乃執威等囚之,然後舉兵,政會功也。



 大將軍府建,為戶曹參軍,遷丞相府掾。武德初,授衛尉少卿,留守太原,調輯戎政,遠近歡服。會劉武周寇並州,晉陽豪傑舉應之,政會為武周所擒,每密表賊形勢。既平,復官爵,歷光祿卿,封邢國公。貞觀初,轉洪州都督,卒。太宗手詔:「政會昔預義舉,有殊功,葬宜異等。」於是贈民部尚書,謚曰襄。後追徙渝國。



 子玄意襲爵,尚南平公主。高宗時為汝州刺史。次子奇,長壽中,為天官侍郎,薦張鷟、司馬鍠為監察御史,二人因申屠瑒以謝,奇正色曰:「舉賢本無私,何見謝?」聞者皆竦。後為酷吏陷,被誅。



 七世孫崇望,字希徒,及進士第,宣歙王凝闢轉運巡官。崔安潛帥許及劍南,崇望昆弟四人同幕府,世以為才。安潛入為吏部尚書,崇望又以員外郎主南曹,選事清辦。僖宗幸山南,王重榮怨宦豎,不肯率職,時高選使者,即河中鐫諭使自新,崇望以諫議大夫持節往。既至,陳君臣大義動之,重榮順服,請誅硃玫自效。使還,稱旨,擢翰林學士。昭宗即位,進中書侍郎、同中書門下平章事。張浚伐太原,崇望固執不可,浚果敗。代為門下侍郎、判度支。玉山都將楊守信反,夜陳兵闕下。帝列兵延喜門,命崇望守度支庫。巉旦,含光門未開,禁卒左右植立,將大掠長安中。俄聞傳呼宰相來者,門闢,崇望駐馬勞曰:「上自將在中營,公等禁軍也,不帝前殺賊取功,而茍欲剽掠成惡名乎?」士皆唯唯。至長樂門,賊望兵至,乃遁去,軍中咸呼「萬歲」。是日,京師不亂,繄其力。進尚書左僕射。硃全忠謀取徐、泗,表請以大臣代時溥,乃授崇望武寧軍節度使。溥拒命,崇望還為太常卿。會王珂、王珙爭河中,詔以崔胤為節度使。珂,李克用婿也。太原邸吏薛志勤曰:「崔公鎮河中,不若光德劉公於我公最善。」光德,崇望所居坊也。後李茂貞、王行瑜入誅執政,坐是,貶昭州司馬。行瑜誅,克用直其冤,召為吏部尚書。會王摶以吏部輔政,徙兵部。王建欲並東川,詔崇望為劍南東川節度使、同中書門下平章事。未至,建已使王宗滌知留後,崇望乃還為兵部尚書。卒,贈司空。



 兄崇龜,字子長。擢進士,仕累華要,終清海軍節度使。廣有大賈,約倡女夜集,而它盜殺女,遺刀去。賈入倡家,踐其血乃覺,乘め亡。吏跡賈捕劾,得約女狀而不殺也。崇龜方大饗軍中,悉集宰人,至日入,乃遣。陰以遺刀易一雜置之。詰朝,群宰即庖取刀,一人不去,日:「是非我刀。」問之,得其主名。往視,則亡矣。崇龜取它囚殺之,聲言賈也,陳諸市。亡宰歸,捕詰具伏。其精明類此。姻舊或乾以財,率不答,但寫《荔支圖》與之。然不能防檢其家,既沒,有鬻珠翠羽者,由是名損。



 弟崇魯,字郊文,亦第進士,擢士補闕、翰林學士,僖宗避難山南,為嗣襄王煴史館修撰,得不誅。景福中,以水部郎中知制誥。雅與崔昭緯善。帝以韋昭度、李磎輔政,而昭緯外倚邠、岐兵為援,以久其權。於是天子厚禮磎,昭緯懼見奪,共謀沮之。及磎墨麻出,崇魯輒掠麻大哭。帝問焉,崇魯曰:「今雖乏人,豈宜取憸人為宰相。磎以楊復恭、西門重遂得近職,奈何用之?前日杜讓能羞戮未刷,尚忍蹈覆轍乎?」磎由是不得相。磎亦劾奏其奸,因自陳「為山南楊守亮詆毀,不容與復恭交私」。又言:「崇望為宰相,使親吏日夕謁左軍,與復恭相親厚。絁巾慘帶,不入禁門;崇魯向殿哭,厭詛天詐,殆人之妖。且其父坐賄飲藥死。崇魯身為硃玫史官,作勸進表。在太原府使西川,見田令孜,沒階趨,廢制度自崇魯始。」其相詈訾,俚淺稽校,譬市人然。崇龜始聞哭麻,恚不食。曰:「吾兄弟未始以聲利敗名,今不幸乃生是兒。」後王行瑜、崔昭緯相繼誅,崇魯貶崖州司戶參軍。終水部員外郎。



 許紹,字嗣宗,安州安陸人。父法光,在隋為楚州刺史。元皇帝為安州總管,紹時為兒,與高祖同學,相愛也。大業末,任夷陵通守,會盜起,州境獨完,流人自占數十萬,開倉賑給。煬帝崩問至,紹率人吏三日臨,以所部遙屬越王侗。後王世充篡立,遂遣使以黔安、武陵、澧陽歸國,授峽州刺史,封安陸郡公。高祖賜書道平生舊,以加慰納。



 蕭銑將董景珍降,命紹率兵應接。以破銑功,擢其子智仁為溫州刺史。銑遣楊道生圍峽州,紹擊走之。銑將陳普環具大艦溯江,與開州賊蕭闍提略巴、蜀,紹遣智仁及婿張玄靖、掾李弘節追戰西陵,覆其兵,禽普環,悉獲戰艦。江之南有安蜀城,地直夷陵,荊門城峙其東,皆峭險處。銑以兵戍守,紹遣智仁等攻荊門,取之。制書褒美,許以便宜。紹境連王世充及銑,其下為賊剽者皆見殺,紹得敵人,獨資遣之,二邦感義,殺掠為止。進譙國公,賜帛千段。



 趙郡王孝恭等代銑,復詔督兵圖荊州。會病,卒於軍,帝為流涕。貞觀中,贈荊州都督。智仁,初以勛授封孝昌縣公,紹卒,繼守夷陵,終涼州都督。次子圉師。



 圉師有器幹,研涉藝文,擢進士第。累遷給事中、黃門侍郎、同中書門下三品。龍朔中,為左相。高宗自書詔賜遼東諸將,謂許敬宗曰:「圉師愛書,可示之。」俄坐其子獵犯人田,有辭,怒而射之,圉師掩不奏,為人告擿。帝讓曰:「宰相而暴百姓,非作威福乎?」圉師謝,且言:「作威福者,強兵重鎮,嫚天子法。臣文吏,何敢然!」帝曰:「慊無兵邪?」敬宗因是劾抵,遂免官。久之,為虔州刺史,稍遷相州,專以寬治,州人刻石頌美。部有受賕者,圉師不忍按,但賜《清白箴》,其人自愧,後修飾,更為廉士。進戶部尚書。卒,贈幽州都督,謚曰簡,陪葬恭陵。紹初爵譙國公,以子智仁自有封,故詔孫力士襲之,終洛州長史。



 子欽寂嗣封。萬歲通天元年,契丹入寇,詔為隴山軍討擊副使,戰崇州,敗,為虜所禽。方圍安東,脅令說屬城未下者。欽寂呼安東都護裴玄珪曰:「賊朝夕當滅,幸謹守!」賊怒,害之。武后下制褒美,贈蘄州刺史,謚曰忠。子輔乾,以父死難,授左監門衛中候,為海東慰勞使,使迎柩還葬。



 欽寂弟欽明,以軍功擢左玉鈐衛將軍、安西大都護、鹽山郡公。出為涼州都督。嘗輕騎按部,會突厥默啜兵奄至,被執。賊與皆至靈州,使說之降。欽明至城下,呼曰:「我乏食,有美醬乎?有粱米乎?並乞墨一枝!」時賊營四面阻水,惟一路得入。欽明欲選將簡兵,乘夜襲賊也,而城中無寤其瘦者,遂見害。兄弟死王事,世名其忠。



 程知節本名咬金,濟州東阿人。善馬槊。隋末,所在盜起,知節聚眾數百保鄉里。後事李密,而密料士八千隸四驃騎,分左右以自衛,號「內軍」,常曰:「此可當百萬。」知節領驃騎之一,恩遇隆特。王世充與密戰,知節以內騎營北邙,單雄信以外騎營偃師。世充襲雄信,密遣知節及裴行儼助之。行儼中流矢墜馬,知節馳救之,殺數人,軍闢易,乃抱行儼重騎馳。追兵以槊撞之,知節折其槊,斬追者,乃免。後密敗,為世充所獲。惡其為人,與秦叔寶來奔,授秦王府左三統軍。從破宋金剛、竇建德、王世充,並領左一馬軍總管,搴旗先登者不一,以功封宿國公。七年,隱太子譖之,出為康州刺史,白秦王曰:「大王去左右手矣,身欲久全,得乎?知節有死,不敢去!」事平,拜太子右衛率。尋遷右武衛大將軍,實封七百戶。貞觀中,歷瀘州都督、左領軍大將軍,改封盧國。顯慶二年,授蔥山道行軍大總管,以討賀魯。師次怛篤城,胡人數千出降,知節屠其城去,賀魯因遠遁。軍還,坐免。未幾,起為岐州刺史,致仕。卒,贈驃騎大將軍、益州大都督,陪葬昭陵。子處亮,尚清河公主。



 柴紹,字嗣昌,晉州臨汾人。幼矯悍,有武力,以任俠聞。補隋太子千牛備身。高祖妻以平陽公主。將起兵,紹走間道迎謁。時太子建成、齊王元吉亦自河東往,遇諸塗。建成曰:「追書急,恐吏逮捕,請依劇賊,冀自全。」紹曰:「不可。賊知君唐公子,必執以為功,徒死爾。不如疾走太原。」既入雀鼠谷,聞義兵起,謂紹有謀,乃相賀。授右領軍大都督府長史,領彀騎,發晉陽。先抵霍邑城下,覘形勢。還白:「宋老生一夫敵,我兵到必出戰,可虜也。」大師至,老生果出,紹力戰有功。從下臨汾、絳郡,隋將桑顯和來戰,紹引軍繚其背,與史大奈合攻之。顯和敗,遂平京師。進右光祿大夫,封臨汾郡公。高祖即位,拜左翊衛大將軍,累從征討,以多,進封霍國公,遷右驍衛大將軍。吐谷渾、黨項寇邊,敕紹討之,虜據高射紹軍,雨矢,士失色。紹安坐,遣人彈胡琵琶,使二女子舞。虜疑之,休射觀。紹伺其懈,以精騎從後掩擊,虜大潰,斬首五百級。貞觀二年,平梁師都,轉左衛大將軍。出為華州刺史,加鎮軍大將軍,徙譙國。既病,太宗親問之。卒,贈荊州都督,謚曰襄。二子:哲威、令武。哲威為右屯衛將軍,襲封。坐弟謀反,免死,流邵州。起為交州都督,卒。令武尚巴陵公主,遷太僕少卿、衛州刺史、襄陽郡公。與房遺愛謀反,貶嵐州刺史,自殺。公主亦賜死。



 任瑰,字瑋,廬州合淝人。父七寶,陳將忠之弟,為陳定遠太守。瑰早孤,忠撫愛甚,每曰:「吾子雖多,庸保耳。所以寄門戶者,瑰也!」年十九,試守靈溪令。遷衡州司馬,都督王勇盡以州務屬瑰。陳亡,瑰勸勇據嶺外,立陳後輔之。勇不從,以地降隋,瑰棄官去。仁壽中,調韓城尉,未幾,罷。高祖討捕於汾、晉,瑰上謁轅門,承制署河東縣戶曹。高祖之晉陽,留隱太子托之。義師起,瑰至龍門請見。高祖曰:「隋失其政,四海群沸,吾以外戚據重任,不忍坐觀其亡。晉陽,天下用武處,兵精馬強,今率之,將厭國難。公,將家子,智算練達,論吾此舉其濟乎?」瑰曰:「今主政殘酷,兵役不止,天下之人,思見拯亂,與之息肩。公天付神武,杖順而起,軍令嚴明,所下城邑,無秋豪之犯。關中起兵者跂踵而待。擁義師,迎眾欲,何不濟哉!瑰在馮翊久,悉其人情,願為一介使,入關宣布威靈,以收左輔。繇梁山濟河,直趣韓城,逼郃陽,徇朝邑。蕭造文吏,勢當自下。次招諸賊,然後鼓行而前,據永豐積粟,雖未得京師,關中固已定矣。」高祖曰:「是吾心也!」乃授銀青光祿大夫。遣陳演壽、史大奈步騎六千趣梁山,以瑰及薛獻為招慰大使。高祖謂演壽曰:「閫外事與任瑰籌之。」既而賊孫華、白玄度等果降,且具舟於河以濟師。瑰行說下韓城,與諸將進擊飲馬泉,破之。拜左光祿大夫,留戍永豐倉。高祖即位,授穀州刺史。王世充數攻新安,瑰拒破之。以功封管國公。秦王東討,瑰從至邙山,主水運餉軍。關東平,為河南安撫大使。王世辯以徐州降瑰,瑰至宋州,會徐圓朗反,副使柳浚勸退保汴,瑰笑曰:「公何怯?老將居邊久,自當有計。」俄而賊陷楚丘,將圍虞城,瑰遣崔樞、張公謹自鄢陵領諸州豪質子百餘守之。浚曰:「樞等故世充將,且諸州質子父兄皆反,奈何令保城?」瑰不答。樞至,則分質子與土人合隊,賊近,質子稍叛,樞即斬其隊帥。城中人懼曰:「是皆賊子弟,安可與守乎?」樞因聽諸隊殺質子,梟首門外。瑰陽怒曰:「去者遣招慰,何乃殺之?」退謂浚曰:「固知崔樞辦之。縣殺賊子,為怨已大,人今自為戰矣。」圓朗攻虞城,不能拔。賊平,遷徐州總管,仍為大使。輔公祏反,詔以兵自揚子津濟江討之。公祏平,拜邗州都督,遷陜州。瑰弟璨,為隱太子典膳監。太子廢,璨得罪,瑰亦左授通州都督。貞觀四年卒。瑰歷職有功,然補吏多為親故人私,至負勢賕請,瑰知,不甚禁遏,世以此譏之。瑰卒,時有司以在外對仗白奏,太宗怒曰:「昔杜如晦亡,朕不能事者數日。今瑰喪,所司不以狀言,豈朕意乎?有如朕子弟不幸死,當此奏邪!」自是大臣喪,遂不對仗奏云。



 丘和,河南洛陽人,後徙家郿。少重氣俠,閑弓馬,長乃折節自將。仕周開府儀同三司。入隋為右武衛將軍,封平城郡公,歷資、梁、蒲三州刺史,以寬惠著名。漢王諒反,使卒衣婦人衣,襲取蒲州,和挺身免,坐廢為民。宇文述有寵,和傾心附納。俄以發武陵公元胄罪,復拜代州刺史。煬帝北巡,和饋獻精腆,至朔州,而刺史楊廓無所進,帝不悅。述盛稱和美,帝用為博陵太守,詔廓就視和為式。後帝過博陵,和上食加豐,愈喜。由是所過競為珍侈獻,自和發也。然和善撫吏士,得其心。遷天水郡守,入為左御衛將軍。大業末,海南苦吏侵,數怨畔。帝以和所蒞稱淳良,而黃門侍郎裴矩亦薦之,遂拜交址太過,撫接盡情,荒憬安之。煬帝崩,而和未知。於是鴻臚卿寧長真舉鬱林附蕭銑,馮盎舉珠崖、番禺附林士弘,各遣使招和,不從。林邑西諸國,數遺和明珠、文犀、金寶,故和富埒王者。銑聞,利之,命長真以南粵蠻、俚攻交址,和遣長史高士廉率兵擊走之,郡為樹石勒其功。會隋驍果自江都來,乃審隋亡,和即陳款歸國,而嶺嶠閉岨,乃權附銑。銑平,遂得歸。詔李道裕即授和交州大總管,爵譚國公。和遣士廉奉表請入朝,詔其子師利迎之。及謁見,高祖為興,引入臥內,語平生,歡甚,奏九部樂饗之,除左武候大將軍。和時已老,以稷州其故鄉也,令為刺史以自養。尋除特進。貞觀十一年卒,年八十六,贈荊州總管,謚曰襄,陪葬獻陵。有子十五人,多至大官,而行恭為知名。



 行恭有勇,善騎射。大業末,與兄師利聚兵萬人保郿城,人多依之,群盜不敢窺境。後原州奴賊圍扶風,太守竇璡堅守。賊食盡無所掠,眾稍散歸行恭。行恭遣其酋說賊共迎高祖,乃自率五百人負糧持牛酒詣賊營。奴帥長揖,行恭手斬之,謂眾曰:「若皆豪桀也,何為事奴乎?使天下號曰奴賊。」眾皆伏,曰:「願改事公。」行恭乃率其眾,與師利迎謁秦王於渭北,拜光祿大夫。累從戰伐,功多,遷左一府驃騎,錫勞甚厚。隱太子誅,以功擢左衛將軍。貞觀中,坐與兄爭葬所生母,廢為民。從侯君集平高昌,封天水郡公,進右武候將軍。高宗立,遷大將軍、冀陜二州刺史,致仕。卒,年八十,贈荊州刺史,謚曰襄,陪葬昭陵。行恭所守嚴烈,僚吏畏之。數坐事免,太宗思其功,不逾時輒復官。初,從討王世充,戰邙山。太宗欲嘗賊虛實,與數十騎沖出陣後,多所殺傷,而限長堤,與諸騎相失,唯行恭從。賊騎追及,流矢著太宗馬,行恭回射之,發無虛鏃,賊不敢前。遂下拔箭,以己馬進太宗,步執長刀,大呼導之,斬數人,突陣而還。貞觀中,詔斫石為人馬,象拔箭狀,立昭陵闕前,以旌武功云。子神勣,見《酷吏傳》。



 贊曰:帝王之將興,其威靈氣焰有以動物悟人者,故士有一概,皆填然躍而附之,若榱椽梁柱以成大室,又負偃植,各安所施而無遺材,諸將之謂邪。然皆能禮法自完,賢矣哉!



\end{pinyinscope}