\article{列傳第八十五 徐呂孟劉楊潘崔韋}

\begin{pinyinscope}

 徐浩,字季海,越州人。擢明經,有文辭。張說稱其才,由魯山主簿薦為集賢校理一體系的環節、成分,根本不涉及實際。在社會思想方面,宣,見《喜雨》、《五色鴿賦》,咨嗟曰:「後來之英也!」進監察御史裏行。闢幽州張守珪幕府。歷河陽令,治有績。東都留守王倕表署其府。民有妄作符命者,眾不為疑,浩獨按篆詰狀,果詐為之。遷累都官郎中,為嶺南選補使,又領東都選。



 肅宗立,由襄州刺史召授中書舍人。四方詔令,多出浩手,遣辭贍速,而書法至精,帝喜之。又參太上皇誥冊,寵絕一時。授兼尚書右丞。浩建言:「故事,有司斷獄,必刑部審覆。自李林甫、楊國忠當國,專作威福,許有司就宰相府斷事,尚書以下,未省即署,乖慎恤意。請如故便。」詔可。故詳斷復自此始。進國子祭酒,為李輔國譖,貶廬州長史。



 代宗復以中書舍人召,遷工部侍郎、會稽縣公,出為嶺南節度使。召拜吏部侍郎,與薛邕分典選。浩有妾弟冒優,托之邕,擬長安尉,御史大夫李棲筠劾之,帝怒,黜邕歙州刺史,浩明州別駕。德宗初,召授彭王傅,進郡公。卒,年八十,贈太子少師,謚曰定。



 始,浩父嶠之善書,以法授浩,益工。嘗書四十二幅屏,八體皆備,草隸尤工,世狀其法曰「怒猊抉石,渴驥奔泉」云。晚節治廣及領選,頗嗜財,惑於所嬖,卒以敗。



 呂渭,字君載,河中人。父延之,終浙東節度使。渭第進士,從浙西觀察使李涵為支使,進殿中侍御史。大歷末,涵為元陵副使,渭又為判官。涵由御史大夫擢太子少傅,渭建言:「涵父名少康,當避。」宰相崔祐甫善其言,擢司門員外郎。御史共劾渭:「昔涵再任少卿,不以嫌,今謂少傅為慢官,疑渭為涵游說。」乃貶渭歙州司馬。



 貞元中,累遷禮部侍郎。始,中書省有古柳,建中末枯死,德宗自梁還,復榮茂,人以為瑞柳,渭令貢士賦之。帝聞,不以為善。又與裴延齡為姻家,擢其子操上第,會入閣,遺私謁之書於廷。出為潭州刺史。卒,贈陜州大都督。



 四子:溫、恭、儉、讓。



 溫,字和叔,一字化光,從陸質治《春秋》,梁肅為文章。貞元末,擢進士第。與韋執誼厚,因善王叔文。再遷為左拾遺。以侍御史副張薦使吐蕃,會順宗立,薦卒於虜,虜以中國有喪,留溫不遣。時叔文秉權,與游者皆貴顯,溫在絕域不得遷,常自悲。元和元年乃還,而柳宗元等皆坐叔文貶,溫獨免,進戶部員外郎。



 溫藻翰精富,一時流輩推尚。性險躁,譎詭而好利,與竇群、羊士諤相暱。群為御史中丞,薦溫知雜事,士諤為御史,宰相李吉甫持之,久不報,溫等怨。時吉甫為宦侍所抑,溫乘其間謀逐之。會吉甫病,夜召術士宿於第,即捕士掠訊,且奏吉甫陰事。憲宗駭異,既詰辯,皆妄言,將悉誅群等,吉甫苦救乃免,於是貶溫均州刺史,士諤資州。議者不厭,再貶為道州。久之,徙衡州,治有善狀。卒,年四十。



 恭,字恭叔,尚氣節,喜縱橫、孫吳術。為山南西道府掌書記,進殿中侍御史,終嶺南府判官。



 儉亦為御史。讓,太子右庶子。皆美材。



 孟簡,字幾道,德州平昌人。曾祖詵,武后時同州刺史。簡舉進士、宏辭連中,累遷倉部員外郎。王叔文任戶部,簡以不附離見疾,不敢顯黜,宰相韋執誼為徙它曹。元和中,拜諫議大夫,知匭事。韓泰、韓曄之復刺史,吐突承璀為招討使,簡皆固爭,詣延英言不可狀,以悻切出為常州刺史。州有孟瀆,久淤閼,簡治導,溉田凡四千頃,以勞賜金紫,召為給事中。



 代李遜為浙東觀察使。遜抑士族,右編人,至橫恣不檢,及簡,一反之,農估兼受其弊,時謂兩失之。以工部侍郎召還。初,使府得代,詔至,署留後即行。李翛觀察浙西,始請留故使交政。及簡還半道堂牒還之,如例,乃聽解。



 進戶部,加御史中丞。戶部有二員,判使按者居別一署,謂之「左戶」,元和後,選委華重,宰相多由此進。崔群既相,而簡代之,故簡意且柄任。及出山南東道節度使,內不樂。政頗嚴峭。時有詔置臨漢監以牧馬,命簡兼使職。簡以親吏陸翰主奏邸,關通閹侍,翰持之,數傲很,簡怒,追還,以土囊斃之。家上變,發簡奸贓,御史劾驗,得遺吐突承璀貲七百萬。左授太子賓客,分司東都,再貶吉州司馬。以赦令進睦州刺史,復徙常州,仍太子賓客分司,卒。



 簡尤工詩,聞江、淮間。尚節義,與之交者,雖歿,視恤其孤不少衰。晚路殊躁急,佞佛過甚,為時所誚。嘗與劉伯芻、歸登、蕭俯譯次梵言者。



 劉伯芻,字素芝,兵部侍郎乃之子。行修謹。淮南杜佑奏署節度府判官。府罷,召拜右補闕,遷主客員外郎。數過友家飲噱,為韋執誼陰劾,貶虔州參軍。久乃除考功員外郎。裴垍待之善,擢累給事中。李吉甫當國而垍卒,不加贈,伯芻為申理,乃贈太子少傅。或言其妻垍從母也,吉甫欲按之,求補虢州刺史。稍遷刑部侍郎、左散騎常侍。卒,贈工部尚書。伯芻風度高嚴,善談謔,而動與時適,論者少之。



 子寬夫,寶歷中為監察御史。奏言:「以王府官攝祠,位輕,非嚴恭意,請以尚書省、東宮三品若左右丞、侍郎通攝。」俄轉左補闕。陳岵注浮屠書,因供奉僧以聞,除濠州刺史。寬夫劾狀,敬宗怒謂宰相曰:「岵不繇僧得州,諫臣安受此言?」寬夫曰:「眾劾岵,獨臣草狀,應伏誅。推言所從,恐累國體。」帝讜其言,釋之。



 子允章,字蘊呂,咸通中為禮部侍郎。請諸生及進士第並謁先師,衣青衿,介幘,以還古制。改國子祭酒。又建言:「群臣輸光學錢治庠序,宰相五萬,節度使四萬,刺史萬。」詔可。後為東都留守。黃巢至,分司李磎挈尚書印走河陽,允章寄治河清。巢僭號,輒受偽官,文書盡用金統。遣取印磎所,磎不與,更悔愧,移檄近鎮起兵捍賊,磎持印還之。後廢於家。



 楊憑,字虛受,一字嗣仁,虢州弘農人。少孤,其母訓道有方。長善文辭,與弟凝、凌皆有名。大歷中,踵擢進士第,時號「三楊」。憑重交游,尚氣節然諾,與穆質、許孟容、李庸阜相友善,一時歆慕,號「楊穆許李」。



 歷事節度府,召為監察御史,不樂,輒免去。累遷太常少卿、湖南江西觀察使。性簡傲,接下脫略,人多怨之。在二鎮尤侈忲。入拜京兆尹。與御史中丞李夷簡素有隙,因劾憑江西奸贓及它不法,詔刑部尚書李庸阜、大理卿趙昌即臺參訊。於時憑治第永寧里,功役叢煩,又幽妓妾於永樂別舍,謗議頗言雚,故夷簡藉之痛擿發,欲抵以死。既置對,未得狀,即逮捕故官屬推躡,簿憑家貲。翰林學士李絳奏言:「憑所坐贓,不當同逆人法。」乃止。憲宗以憑治京兆有績,但貶臨賀尉。始,德宗時,假借方鎮,習為僭儗事,夷簡首按憑,時以為宜,而緣私怨,論者亦不與。俄徙杭州長史。以太子詹事卒。



 憑所善客徐晦者,字大章,第進士、賢良方正,擢櫟陽尉。憑得罪,姻友憚累,無往候者,獨晦至藍田慰餞。宰相權德輿謂曰:「君送臨賀誠厚,無乃為累乎?」晦曰:「方布衣時,臨賀知我,今忍遽棄邪?有如公異時為奸邪譖斥,又可爾乎?」德輿嘆其直,稱之朝。李夷簡遽表為監察御史,晦過謝,問所以舉之之由。夷簡曰:「君不負楊臨賀,肯負國乎?」後歷中書舍人,強直守正,不沈浮於時。嗜酒喪明,以禮部尚書致仕,卒。



 凝,字懋功。由協律郎三遷侍御史,為司封員外郎。坐厘正嫡媵封邑,為權幸所忌,徙吏部,稍遷右司郎中。宣武董晉表為判官,亳州刺史缺,晉以凝行州事。增墾田,決污堰,築堤防,水患訖息。時孟叔度橫縱撓軍治,而凝亦荒湎,晉卒,亂作。凝走還京師,闔門三年。拜兵部郎中,以痼疾卒。



 凌,字恭履,最善文,終侍御史。子敬之。



 敬之,字茂孝。元和初,擢進士第,平判入等,遷右衛胄曹參軍。累遷屯田、戶部二郎中。坐李宗閔黨,貶連州刺史。文宗尚儒術,以宰相鄭覃兼國子祭酒,俄以敬之代。未幾,兼太常少卿。是日,二子戎、戴登科,時號「楊家三喜」。轉大理卿,檢校工部尚書,兼祭酒,卒。



 敬之嘗為《華山賦》示韓愈,愈稱之,士林一時傳布,李德裕尤咨賞。敬之愛士類,得其文章,孜孜玩諷,人以為癖。雅愛項斯為詩,所至稱之,繇是擢上第。斯,字子遷,江東人。敬之祖客灞上,見閩人濮陽願,閱其文,大推挹,遍語公卿間。會願死,敬之為斂葬。



 潘孟陽,史亡何所人。父炎,大歷末官右庶子,為元載所惡,久不遷。載誅,進禮部侍郎,以病免。方劉晏任權,炎乃其婿,雖書疏報答,未嘗輒開,時稱有古人節。晏得罪,坐貶澧州司馬,時輿疾上道,不自言。於邵高其介,申救,不見聽。



 孟陽少以廕,俄登博學宏辭科,補渭南尉,再遷殿中侍御史。公卿多父行及外家賓客,故被慰薦,擢累兵部郎中。貞元末,王紹以恩幸進,數稱孟陽才,權知戶部侍郎。杜佑判度支,奏以自副。時憲宗新立,詔孟陽馳驛江淮視財賦,加鹽鐵轉運副使,並察諸使治否。孟陽恃奧主,又氣豪倨,從者數百人,所至會賓客,留連倡樂,招金錢,多補吏,譽望大喪。使還,罷為大理卿。其後左司郎中鄭敬宣慰江淮,帝誡曰:「朕宮中用尺寸物皆有籍,唯賑民無所計。卿是行,宜諭朕意,毋若潘孟陽殫財費酣飲游山寺而已。」



 元和三年,出為華州刺史,遷劍南東川節度使。宰相武元衡與孟陽舊,復以戶部侍郎召判度支,又兼京北五城營田使。太府王遂為西北供軍使,持營田不可,至私忿恨,更請間論列。帝怒,罷孟陽左散騎常侍。明年,復舊官。盛葺第舍,帝微行至樂游原,望見之,以問左右,孟陽懼,輟不敢治。而伎媵用度過侈汰,人多指怒之。病風痺,改左散騎常侍。卒,贈兵部尚書,謚曰康。



 初,孟陽為侍郎,年未四十,其母謂曰:「以爾之材而位丞郎,使吾憂之。」



 崔元略,博州人。父敬,貞元時終尚書左丞。元略第進士,更闢諸府,遷累殿中侍御史,以刑部郎中知御史雜事,進拜中丞。時李夷簡召為大夫,故詔元略留司東臺。改京兆少尹,行府事,數月,遷為尹。徙左散騎常侍。



 初,中丞缺,議者屬崔植,而元略謬謂植入閣不如儀,使御史彈治。及宰相以二人進,元略果得之,植恨悵。既當國,以元略為宣撫黨項使,辭疾不行。植奏:「不少責,無以示群臣。」乃出為黔南觀察使,徙鄂岳。久乃拜大理卿。



 敬宗初,還京兆尹,兼御史大夫。收貸錢萬七千緡,為御史劾奏,詔刑部郎中趙元亮、大理正元從質、侍御史溫造以三司雜治。元略素事宦人崔潭峻,頗左右之,獄具,削兼秩而已。俄授戶部侍郎,譏謗大興,諫官斥元略方劾而遷,有助力,元略自解辨,乃止。京兆劉棲楚又劾元略前造東渭橋,縱吏增估物不償直,取工徒贓二萬緡。詔奪一月俸。於是棲楚規相位,疑元略妨己路,故舉疑似衊染之。太和三年,以戶部尚書判度支,出為東都留守,改義成節度使。卒,贈尚書左僕射。子鉉。



 鉉,字臺碩,擢進士第,從李石荊南為賓佐,入拜司勛員外郎、翰林學士,遷中書舍人、學士承旨。武宗好蹴踘、角抵,鉉切諫,帝褒納之。會昌三年,拜中書侍郎、同中書門下平章事。鉉入朝,凡三歲至宰相,而石猶在江陵。澤潞平,兼戶部尚書。與李德裕不葉,罷為陜虢觀察使。宣宗初,擢河中節度使,以御史大夫召,用會昌故官輔政,進尚書左僕射,兼門下侍郎,封博陵郡公。鉉所善者鄭魯、楊紹復、段瑰、薛蒙,頗參議論。時語曰:「鄭、楊、段、薛,炙手可熱;欲得命通,魯、紹、瑰、蒙。」帝聞之,題於扆。是時,魯為刑部侍郎,鉉欲引以相,帝不許,用為河南尹。它日,帝語鉉曰:「魯去矣,事由卿否?」鉉惶懼謝罪。



 久之,出為淮南節度使,帝餞太液亭,賜詩寵之。因宣州軍亂,逐觀察使鄭薰,鉉出兵討擊,詔兼宣歙池觀察使。既平,加檢校司空,罷兼使。居九年,條教一下無復改,民以順賴。咸通初,徙山南東道、荊南二鎮,封魏國公。龐勛叛,自桂管北還,所過剽略。鉉聞,大募兵屯江、湘,邀賊歸路。賊懼,更逾嶺,自淮而北。朝廷壯其忠。卒官下。



 子沆,字內融,累遷中書舍人。韋保衡逐於琮,沆亦貶循州司戶參軍。僖宗立,召為永州刺史,復拜舍人,進禮部、吏部二侍郎。乾符五年,以戶部侍郎同中書門下平章事。昕旦告麻,大霧塞廷中,百僚就班修慶,大風雨雹,時謂不祥。俄改中書侍郎,兼工部尚書。時王景崇進兼中書令,讓其兄景儒,求易定節度。沆謂魏博、盧龍且相援,執不可。盧攜專政,而黃巢勢浸盛,沆每建裁遏,多為攜沮抑。賊陷京師,匿張直方第,遇害。



 元略弟元受、元式、元儒,皆舉進士第。



 元受以高陵尉直史館。元和時,於皋謨為河北行營糧料使,元受從之,督供饋。皋謨得罪,元受逐死嶺表。



 元式始署帥府僚佐,累官湖南觀察使。會昌中,澤潞用兵,遷河中,拜河東、義成節度使。宣宗初,以刑部尚書判度支,拜門下侍郎、同中書門下平章事,進兼戶部尚書。以疾罷。卒,贈司空,謚曰莊。



 大中時,又有宰相崔龜從,字玄告。初舉進士,復以賢良方正、拔萃,三中其科,拜右拾遺。太和初,遷太常博士。最明禮家沿革,問不虛酬。定敬宗廟室祝辭,皇帝不可云孝弟。九宮皆列星,不容為大祠。大臣薨,不於訃日輟朝,乃在數日外。因引貞觀時,任瑰卒,有司對仗奏,太宗責其不知禮;岑文本歿,是夕罷警嚴;張公謹亡,哭不避辰日;故閔悼之切,不宜過時。又言三品以上官,非經任將相密近,不宜輟朝。詔皆可其議,九宮遂為中祠。再遷至司勛郎中,知制誥,真拜中書舍人,歷戶部侍郎。大中四年,以中書侍郎同中書門下平章事。再歲,罷為宣武軍節度使,數徙鎮,卒。



 韋綬,字子章,京兆萬年人。有至性,然好不經,喪父,鑱臂血寫浮屠書。建中末,為長安尉。硃泚亂,羸服走奉天,拜華陰令。佐襄陽于頔府,數譏謔刺頔橫恣,頔不能容,薦諸朝。三遷職方郎中。



 穆宗為太子,綬入侍讀,遷諫議大夫。太子書「依」字輒去「人」,曰:「上以此可天下事,烏得全書耶?」綬白之,帝喜,即賜綬錦彩。方太子幼,綬數為俚言以悅太子,它日侍,太子為帝道之。帝怒曰:「綬當以經義輔導太子,而反語此,朕何賴焉?」外遷虔州刺史。



 穆宗立,召為尚書右丞、集賢院學士,出入禁中,怙寵甚。建白:「帝誕日,百官先詣光順門賀皇太后,然後上皇帝千萬歲壽。」詔可。久之,宰相奏古無生日稱賀者,綬議格。時大臣論啟或未決,綬居中助可否。九月九日宴群臣曲江,綬請集賢學士得別會,帝一順聽。進位禮部尚書。帝問所以振災邀福者,對曰:「宋景公以善言退法星三舍,漢文除秘祝,敕有司祭而不祈,此二君皆受自至之福,書美前史。如失德以卻災,媚神以丐助,神而有知,且因以譴也。」時帝不德,故托諷焉。



 俄以檢校戶部尚書為山南西道節度使。入辭,請門戟十二以行,又乞賜錢二百萬,官子元弼太常丞,帝以舊恩許之。綬耄而貪,不能事軍政,綱維亂弛。卒,贈尚書右僕射,帝遣中人吊其家。有司謚通丑,故吏以為言,改謬丑,不報,罷。



\end{pinyinscope}