\article{列傳第一百一十八 忠義下}

\begin{pinyinscope}

 程千里袁光廷龐堅薛願張興蔡廷玉符令奇璘劉乃孟華張伾周曾張名振石演芬吳漵高沐賈直言辛讜黃碣孫揆



 程千里,京兆萬年人。長七尺,魁岸有力。應募磧西,累官安西副都護。天寶末,兼北庭都護、安西北庭節度使。突厥首領阿布思內附,本隸朔方,賜氏李,名獻忠,度屬幽州,素與安祿山有怨,內懼,故叛還磧外,數盜邊。玄宗患之,詔千里將兵討捕。千里諭葛邏祿,陰令掎角。獻忠果以窮歸葛邏祿,縛之,並妻子帳下數千人送千里所,乃獻俘勤政樓,詔斬以徇。擢千里右金吾衛大將軍,留宿衛。



 祿山反,詔募兵河東,即拜節度副使、雲中太守,遷上黨長史。賊來攻,鏖馘多,累加開府儀同三司、禮部尚書。至德二載,賊將蔡希德圍上黨,輕騎挑戰。千里恃勇開縣門,率百騎欲直禽希德,幾得而救至,乃退。會橋壞,馬顛,為賊執,仰首敕諸騎使還,曰:「為我報諸將,可失帥,不可失城。」軍中皆為泣下,增備固守。賊不能下,乃還。囚千里至東都,安慶緒偽署特進,囚客省。慶緒敗,為嚴莊所害。後赦令數下,追褒死難者,惟千里生見執,不及雲。



 初,祿山構難,西北戍兵悉入援,故河、隴郡縣皆陷吐蕃,惟河西戍將袁光廷為伊州刺史,固守歷年,雖游說百緒,終不降,諸下同心無攜畔者。及糧竭,手殺妻子,自焚死。建中初,贈工部尚書。


龐堅,京兆涇陽人。四世祖玉,事隋為監門直閣。李密據洛口,玉以關中銳兵屬王世充擊之,百戰不衄。世充歸東都,秦王東徇洛,玉率萬騎降,高祖以隋舊臣,禮之。玉魁梧有力,明軍法,久宿衛,習知朝廷制度。帝顧諸將多不閑儀檢,故授玉領軍、武衛二大將軍,使眾觀以為模
 \gezhu{
  疒齊}
 ,出為梁州總管。巴山獠叛,玉梟其首,餘黨四奔,屬縣獠與反者州里親戚為賊游說,言不可窮躡。玉不聽,下令軍中曰:「穀熟,吾盡收以饋軍。非盡賊,吾不反。」聞者懼,相謂曰:「軍不止,吾穀盡,且餓死。」乃共入賊營,與所親相結,斬渠長以降,眾遂潰。徙越州都督。召為監門大將軍。太宗以耆厚,令主東宮兵。雖老不怠,小大之務無不親。卒,帝為廢朝,贈幽州都督、工部尚書。



 堅歷潁川太守。安祿山反,南陽節度使魯炅表堅為長史兼防禦副使,以薛願為潁川太守,共守潁川。時陳留、滎陽已陷賊,南陽被圍,而潁川當往來劇。賊將阿史那承慶悉銳攻之,傳城百里,樹木皆刊。城中士單寡,糧少,而願、堅晝夜戰,諸郡兵無援者,自正月盡十一月。賊設木鵝、沖車、飛梯薄城,矢如雨,士皆雷噪,夜半逾城入,二人不肯降。賊縛致東京,將礫解之,有說祿山曰:「義士也,彼為其主,殺之不詳。」乃縛於樹。比且死,見者哭之。



 願,汾陰人。父縚,太常卿。兄崇一,娶惠宣太子女,其女弟為太子瑛妃。瑛廢,貶願嶺外,久乃得還。



 張興者,束鹿人。長七尺,一飯至斗米,肉十斤。悍趫而辯,為饒陽裨將。祿山反,攻饒陽。興開張禍福,譬曉敵人,而嬰垮彌年,眾心遂固。滄、趙已隱,史思明引眾傳城,興擐甲持陌刀重十五斤乘城。賊將入,興一舉刀,輒數人死,賊皆氣懾。城破,思明縛之馬前,好謂曰:「將軍壯士,能屈節,當受高爵。」對曰:「昔嚴顏一巴郡將,猶不降張飛。我大郡將,安能委身逆虜?今日幸得死,然願以一言為誡。」思明曰:「雲何?」興曰:「天子遇祿山如父子,今乃反。大丈夫不能為國掃除,反為其下,何哉?」思明曰:「將軍不觀天道邪?吾上起兵二十萬,直趣洛陽,天下大定。以偏師叩函谷,守將面縛,唐亡固矣。」興曰:「桀、紂、秦、隋窮人力,舉四海與為怨,故商、周、漢、唐因得代之而有神器。皇帝無違德,祿山非數帝賢,是茍延歲月,終即禽耳。」思明怒,鋸解之。且死,罵曰:「吾能裒強死兵敗賊眾!」軍中凜然為改容。



 蔡廷玉,幽州昌平人。事安祿山,未有聞。與硃泚同里閑,少相狎近,泚為幽州節度使,秦署幕府。



 廷玉有沈略,善與人交,內外愛附。泚多所叩咨,數遣至京師。當是時,幽州兵最強,財雄,士驕悍,日思吞並,不知有上下禮法。廷玉間語泚曰:「古未有不臣而能推福及子孫者。公南聯趙、魏,北奚虜,兵我地險,然非永安計,一日趙、魏反噬,公乃沸鼎魚耳。不如奉天子,多難,可勒勛鼎彞,若何?」泚善之。廷玉陰欲耗其力,則諷泚出金幣禮士,又勸歸貢賦助天子經費,獻牛馬系道,儲廥為單。因勸泚入朝,泚將聽,諸校怒,縛廷玉辱之,廷玉無橈辭,泚不忍殺,囚歲餘出之,謂曰:「而亦悔乎?」廷玉曰:「導公為逆即悔,勉公以義何悔為?」復縶滿歲,問曰:「能省過否?不爾,且死。」對曰:「不殺我,公得名。殺我,吾得名。」泚不能屈,待如初。



 又有硃體微者,亦泚腹心。廷玉有建白,體微輒左右之,故泚愈信,桀傲稍革。廷玉遂蕆朝事。泚乃奏涿州為永泰軍,薊州靜塞軍,瀛州清夷軍,莫州唐興軍,置團練使,以支郡隸屬,盧龍軍稍削。而泚內畏弟滔逼己,滔亦勸泚入朝,乃以軍屬滔。廷玉、體微共白泚:「公入朝為功臣首,後務至重,須誠信者乃可付。滔雖大弟,多變不情,如假以兵,是嫁之禍也。」泚不聽。二人隨泚到朝,德宗為太子時,知廷玉名,及見,禮眷殊渥。泚統幽州行營為涇原鳳翔節度使,詔廷玉以大理少卿為司馬,體微為要籍。



 滔有請於泚,或不順,廷玉必折之,俾循故法。滔已破田悅,浸傲肆自用。左右有惡廷玉者,妄云:「素毀滔,欲四分燕,廷玉倡之,體微和之。」滔表言二人離間骨肉,請殺於有司。亦遺泚書云云。泚恚滔奪其軍,不從。會滔以幽州叛,帝示滔表,而泚亦白發其書,乃歸罪於二人,貶廷玉柳州司戶參軍、體微南浦尉以慰滔。滔使諜伺諸朝,曰:「上若不殺廷玉,當謫去,得東出洛,我且縛致麾下支解之。」將行,帝勞廷玉曰:「爾姑行,為國受屈,歲中當還。」遷玉至藍田驛,人白左巡使鄭詹:「商於道險,不可往。」詹追使趨潼關。廷玉告子少誠、少良曰:「我為天子不血刃下幽十一城,欲裂其壤,使不得桀,而敗於將成,天助逆邪?今吏使我出東都,此殆滔計,吾不可以辱國。」比至靈寶,自投於河。



 宰相盧杞方疾御史大夫嚴郢,欲逐之,得廷玉死狀,即抵詹死,而斥出郢。帝閔廷玉忠,歸其柩,厚賻之。李晟平硃泚,少誠等適終喪,晟表丐追贈廷玉。並官二子。而帝方招來滔,寢其奏,遂已。



 符令奇,沂州臨沂人。初為盧龍軍裨將。會幽州亂,挈子璘奔昭義,節度使薛嵩署為軍副。嵩卒,田承嗣盜其地,引令奇為右職。



 田悅拒命,馬燧敗之洹水。令奇密語璘曰:「吾閱世事多矣。自安、史干紀,無噍類。吾觀田氏覆亡無時,安用茍旦夕,系縲京師,宗族屠地?汝能委質朝廷,為唐忠臣,吾亦名揚後世矣。」璘泣曰:「悅,忍人也,近禍可畏。」答曰:「今王師四合,吾屬俎中醢。兒今行,吾死不朽;不行,吾亦死。尸疊逆地云何?」璘俯泣不能對。初,悅與李納會濮陽,因乞師,納分麾下隨之。至是,納兵歸齊,使璘以三百騎護送。璘與父嚙臂別,乃以眾降燧。璘之出,與三子同降。悅怒,引令奇切讓。令奇罵曰:「爾忘義背主,旦夕死。吾教子以順,殺身庸何悔?鈞死,愈爾遠矣!」悅怒,奮而起。令奇臨刑,色不變,年七十九,夷其家。



 燧署璘為軍副,詔拜特進,封義陽郡王。既聞父見害,號絕泣血,燧表其冤,加檢校左散騎常侍,賜晉陽第一區、祁田五十頃,贈令奇戶部尚書。



 璘字元亮。李懷光反,詔燧討之。璘介五千兵先濟河,與西師合。從燧入朝,為輔國大將軍,賜靖恭里第一區、藍田田四十頃。璘之降,母匿里中獨免,及悅死,詔迎於魏,賜宴別殿。璘居環衛十三年,卒,年六十五,贈越州都督。



 劉乃字永夷,河南伊闕人。少敬穎,暗誦《六經》,日數千言。善文詞,為時推目。天寶中擢進士第。喪父,以孝聞。服終,中書舍人宋昱知銓事,乃方調,因進書曰:《書》稱:『知人則哲,能官人則惠。』此唐虞以為難。今文部始掄材,終授位,是知人、官人,兩任其責。昔禹、稷、皋陶之聖,猶曰載採有九德,考績以九載。今有司獨委一二小宰,察言於一幅之判,觀行於一揖之內,何其易哉?夫判者,以狹詞短韻為體,是以小冶鼓眾金,雖欲為鼎鏞,不可得已。故雖有周公、尼父圖書《易象》之訓,以判責之,曾不及徐、庾;雖有至德,以喋喋取之,曾不若嗇夫。故干霄蔽日,巨樹也,求尺寸之材,必後於琢杙;龍吟虎嘯,希聲也,尚頰舌之感,必下於蛙黽。豈不悲乎!執事誠能先政事,次文學,退觀其治家,進察其臨節,則龐鴻深沈之事,亦可窺其門閾矣。」昱嘉之,補剡尉。劉晏在江西,奏使巡覆,充留後。



 大歷中,召拜司門員外郎。德宗初,進郭子儀為尚父。時冊禮廢,視詔文者不適所宜,宰相崔祐甫召乃至閤草之,少選成文,詞義典裁。俄擢給事中,權知兵部侍郎。楊炎、盧杞當國,五歲不遷。建中四年,真拜兵部侍郎。



 帝狩奉天,乃臥疾私第,硃泚遣人召之,固稱篤。復遣偽相蔣鎮慰誘,乃佯喑不答,灸無完膚。鎮再至,知不可脅,乃太息曰:「我嘗忝曹郎,不能死,寧以自辱亶腥,復欲污賢哲乎?」遂止。乃聞車駕如梁州,自投於床,搏膺呼天,不食卒,年六十。帝聞其忠,贈禮部尚書,謚曰貞惠。子伯芻,別傳。



 孟華,史失其何所人。初事李寶臣為府官屬,論議婞婞不回,同舍疾之。王武俊斬李惟岳,遣華至京師陳事,德宗問河朔利害,華對稱旨,擢檢校兵部郎中兼侍御史。



 硃滔與武俊謀解田悅之圍,帝詔華還諭,欲亂其謀。華至,讓武俊曰:「安、史未覆滅時,大夫觀其兵,自謂天下可取,今日何汩汩?且上於大夫恩甚厚,將還康中丞他州,而歸我深、趙。自古忠臣,未有不先大功而後得高官者。大夫何望於失地邪?夫藥苦口者利病,大夫後日思愚言,悔無逮!」或曰:「華入朝私奏便宜,欲傾我,故得顯職。」武俊惑之,然以華舊人,未忍奪其職,卒進援悅。華從至臨清,稱病還恆州。武俊令子察所為,乃闔門謝賓客。武俊知不足忌,無殺華意。既僭稱王,授禮部侍郎,不肯起,嘔血死。



 張伾者,本為澤潞將,守臨洺,田悅攻之,乘城固守累月,士死,糧且盡,救不至。伾悉召部將立軍門,命女出遍拜,因曰:「諸君戰良苦,吾無貲為賞,願以是女賣直,為眾士一日費。」士皆哭曰:「請死戰!」會馬燧自河東將兵擊悅城下,敗之,伾乘勝出戰,無不一當百。以功遷泗州刺史。居州十年,擢右金吾衛大將軍,未拜卒,贈尚書右僕射。



 軍中議立其子重政,母徐及兄號訴不肯從,奔告淮南節度使王鍔,乃免。詔嘉其忠,起為金吾衛大將軍,委鍔處以劇職,封徐魯國夫人。



 周曾者,本李希烈部將,與王玢、姚詹韋清志相善,號四公子。希烈反,曾密得其計,一二以告李勉。玢為許州鎮遏使。會哥舒曜拔汝州,希烈遣曾往拒。曾欲引軍據蔡,使玢為應,憺、清居中謀取希烈,密求藥毒希烈,不死。曾之行,希烈使假子十人從。次襄城,知其謀,以告。希烈使李克誠率騾軍千人劫曾殺之,而收其兵,並殺玢、憺。始,約事覺毋相引。清懼,陽說希烈曰:「今兵寡,恐不能就事,請乞師硃滔。」希烈然之。至襄邑,奪劉洽。德宗贈曾太尉,玢司徒,憺工部尚書,擢清安定郡王,實封戶二百。



 又有呂賁、康秀琳、梁興朝、賈樂卿、侯仙欽皆死希烈之難,贈賁、秀琳尚書左右僕射,興朝等皆秩尚書,遣蕭昕致祭境上。命李勉、哥舒曜訪其家子孫,詔雖三世有罪,常降一等。



 曾無後,貞元中,女及曾兄子酆爭襲封,有司奏曾首謀歸順,身死賊手,陛下錫真食,不幸絕嗣,宜令酆以五十戶奉祀,女亦封五十戶。



 張名振,李事李懷光為都將。始,懷光已立功,德宗賜鐵券,奉詔倨甚。名振到軍門大言曰:「太尉見賊不擊,使到不迎,將反邪?且安、史、僕固等今皆族滅,公欲何為?是資忠義士立功耳。」懷光召見,諭以賊強,須蓄銳俟時,誘為不反。及引軍入咸陽,又曰:「公不反,來此何邪?不急攻泚收京城,欲以賊誰遺?」懷光怒曰:「病狂人也。」使左右拉殺之。



 石演芬者,本西域胡人,事懷光至都將,尤親信,畜為假子。懷光軍三橋,將與硃泚連和。演芬使客郜成義到行在,言懷光無破賊意,請罷其總統。成義走告懷光子琟,懷光召演芬罵曰:「爾為我子,奈何欲破吾家?今日負我,宜即死。」對曰:「天子以公為股肱,公以我為腹心;公乃負天子,我何不負公?且我胡人,無異心,惟知事一人,不呼我為賊,死固吾分。」懷光使士臠食之,皆曰:「烈士也,可令快死。」以刀斷其頸。德宗聞,贈演芬兵部尚書,賜其家錢三百萬,斬成義於朔方。



 吳漵者,章敬皇后之弟。代宗立,詔贈後祖神泉為司徒,父令珪太尉,擢叔父令瑤太子家令、濮陽郡公,令瑜太子諭德、濟陽郡公,漵太子詹事、濮陽郡公,並開府儀同三司。令瑤兄弟故為縣令、郎將矣,而漵用盛王府參軍進,俄遷鴻臚少卿、金吾將軍。建中初,遷大將軍。漵循循有禮讓,無倨氣矜色,見重朝廷,時以為材當所位,不自戚屬者。



 硃泚反,盧杞、白志貞皆謂泚有功,不宜首難,得大臣一人持節尉曉,惡且悛。德宗顧左右,無敢行,漵曰:「陛下不以臣亡能,願至賊中諭天子至意。」帝大悅。漵退謂人曰:「吾知死無益而決見賊者,人臣食祿死其難,所也。方危時,安得自計?且不使陛下恨下無犯難者。」即日齎詔見泚,具道帝待以不疑者。而泚業僭逆,故留漵客省不遣,卒被害。帝悲梗甚,贈太子太保,謚曰忠,賜其家實戶二百,一子五品正員官。京師平,官庀其葬。子士矩,別傳。



 高沐者,渤海人。父馮,事宣武李靈耀,假守曹州。靈耀反,馮密遣人奏賊纖悉,有詔即拜曹州刺史。會李正已盜有曹、濮,馮不能自通朝廷,死官下。



 沐,貞元中擢進士第,以家托鄆,故李師古闢署判官。師道叛,沐率其僚郭昈、郭航、李公度引古今成敗,前後鐫說,不能入。師道所厚吏李文會、林英等乘間訴曰:「比悉心憂公家事,而為沐等所疾,公奈何舉十二州地成沐輩千載名乎?」由是疏斥沐,令守濮州。沐上書盛誇山東煮海之饒,得其地可以富國。師道謀皆露。後英奏事京師,脅邸史言沐以誠款結天子。師道怒,誅沐,而囚戶濮州,守衛苛嚴,凡十年。



 吳元濟拒命,師道引兵攻彭城,敗蕭、沛數縣而還,以緩王師。昈為繒書藏衣絮間,使郭航間道走武寧軍見李願,請奇兵三千浮海搗萊、淄,賊倚海不為備,且居皆罪人,無與守。始,昈畏事洩,署師道所信吏劉諒名以遣,願白諸朝,議者疑師道使為之,不得報。航不敢循故道,間關回遠還昈所。未幾,師道召航,昈疑事露,欲引決,航曰:「事覺,吾獨死,君無患。」航卒自殺,遂絕。及王師討師道,諸節度兵四人,而彭城兵下魚臺金鄉、李聽軍取海州若拾遺,頗用昈策。



 初,淮西平,師道勢蹙,內甚懼。李公度與大將李英曇都獻三州。使長子入侍。師道然可,俄中悔,欲殺英曇,賈直言諷師道嬖奴曰:「高沐冤氣在天,禍且至。英曇復死,是益其崇也。」乃止。逐於萊州,俄殺之。



 又有崔承寵、楊偕、陳佑、崔清皆抗節忤賊,李文會指為沐黨,沐之死,皆被囚。劉悟既平師道,捉昈臂歔欷流涕,闢置義成節度府,亦請公度為僚屬。元和十四年,贈沐吏部尚書,委馬〓備禮收葬,恤其家。



 航,萊州人,以氣聞,師道署右職,與昈世居齊。初,昈舉進士,權德輿將取之,聞其家賊中,乃罷,遂為賊聘。二人座能以忠顯。



 賈直言,河朔舊族也,史失其地。父道沖,以藝待詔。代宗時,坐事賜鴆,將死,直言紿其父曰:「當謝四方神祇。」使者少怠,輒取鴆代飲,迷而踣。明日,毒潰足而出,久乃蘇。帝憐之,減父死,俱流嶺南。直言由是鐍。



 後署師道府屬。及師道不軌,提刀負棺入諫曰:「願前死,不見城之破。」又畫縛載檻車狀而妻子系累者以獻,師道怒,囚之。劉悟既入,釋其禁,闢署義成府。後徙潞,亦隨府遷。



 監軍劉承偕與悟不平,陰與慈州刺史張汶謀縛悟送闕下,以汶代節度。事洩,悟以兵圍承偕,殺小使,赴言遽入責曰:「司空縱兵脅天子使者,是欲效李司空芽?它日復為軍中所指笑。」悟聞,感悔,匿承偕於第以免。悟每有過,必爭,故悟能以臣節光明於朝。穆宗召為諫議大夫,群情灑然稱允。而悟固留,得聽。



 始,悟子從諫貴甚,見直言輒衣紫擁笏,以兵自衛。直言諫悟曰:「郎少年,毋使襲山東熊,朝服可擅著邪?」悟死,從諫不發喪,召大將劉武德等矯悟遺言,與鄰道使共表求襲位,直言入讓曰:「父死不哭,何顏面見山東義士乎?」從諫曰:「欲反耳。」直言仰天哭曰:「爾父提十二州地歸朝廷為功臣。然以張汶故,自謂不潔淋頭,卒羞死。郎今日乃欲反邪?」從諫起抱直言項哭曰:「計窮而然。」直言曰:「君何憂無土地,今脅朝廷,正速死耳。若從武德謀,吾見劉氏為元濟矣。」從諫拜曰:「唯大夫救之。」直言乃自攝留後,使從諫居喪。初,從諫惟鄆兵二千同謀。直言既折之,軍中遂安。



 大和九年卒,贈工部尚書。



 辛讜者,太原尹云京孫也。學《詩》、《書》,能擊劍,重然諾,走人所急。初事李嶧,主錢穀。性廉勁,遇事不處文法,皆與之合。罷居揚州,年五十,不肯仕,而慨然常有濟時意。



 龐勛反,攻杜慆於泗州。讜聞之,挐舟趨泗口,貫賊柵以入。慆素聞其名,握手曰:「吾僚李延樞嘗為吾道夫子為人,何意臨教?吾無憂矣!」讜亦謂慆可共事,乃請還與妻子決,同慆生死。時賊張甚,眾皆南走,獨讜北行。讜未至,慆憂之,延樞知必來,曰:「讜至,可表為判官。」慆許諾。俄而至,慆喜曰:「圍急,飛鳥不敢過,君乃冒白刃入危城,古人所不能。」乃勸解白衣被甲。



 賊將李圓焚淮口,讜曰:「事棘矣,獨出可以求援。」乃與楊文播、李行實戊夜逾淮,坎岸登,馳三十里至洪澤,見戍將郭厚本告急。厚本許出兵,大將袁公異等曰:「賊眾我寡,不可往。」讜拔劍瞋目呼曰:「泗州陷在旦夕,公等被詔來,乃逗留不進,欲何為?大丈夫孤國恩,雖生可羞。且失泗,則淮南為寇場,君尚能獨存?吾今斷左臂殺君去。」推劍直前,厚本持之,公異等僅免。讜望泗慟哭,帳下皆流涕。厚本決許付兵五百,讜曰:「足矣!」遍問士曰:「能行乎?」皆曰:「諾。」讜僕面於地,泣以謝。眾既叩淮,有人語曰:「賊破城矣!」讜將斬之,眾為請。讜曰:「公等登舟,吾赦其死。」士遽登。已濟,慆亦出兵,表裏擊,賊大敗。讜入,人心遂固。浙西杜審權遣將翟行約赴援,壁蓮塘,慆欲遣人廷勞,諸吏憚不敢出,讜獨往犒而還。



 圍三月,救兵外敗,城益危。讜復請乞兵淮南,與壯士徐珍十人持斧夜斬賊柵出,見節度使令狐綯,復詣浙西見審權。時皆傳泗州已陷,疑讜為賊計,囚之。讜引李嶧自明。嶧時為大同防禦使,稱其忠可信。審權乃許救,合淮南兵五千,鹽粟具。方淮路梗,不得進。讜引兵決戰,斬賊六百級,乃克入,城上歡叫,心舀與下迎泣,表其功於朝,授監察御史。圍凡十月乃解,卒完一州。



 初,讜求救也,過家十餘,未嘗見妻子,得糧累二十萬。讜子及兄子客廣陵,托慆曰:「使先人不乏祀,公之惠也。」後以功第一,拜亳州刺史,徙曹、泗二州。乾符末,終嶺南節度使。



 方讜之少,耕於野,有牛斗,眾畏奔踐,讜直前,兩持其角,牛不能動,久而引觸,竟折其角。里人駭異,屠牛以飯讜。然讜臒短,才及中人。後貴,力亦少衰云。



 黃碣,閩人也。初為閩小將,喜學問,軒然有志向。同列有假其筆者,碣怒曰:「是筆它日斷大事,不可假。」後戰安南有功,高駢表其能,為漳州刺史,徙婺州,治有績。劉漢宏遣兵攻之,兵寡不可守,棄州去,客蘇州。



 董昌為威勝軍節度使,表碣自副,久乃應。及昌反,碣諫曰:「大王拔田畝,席貢輸之勤,位將相,非有勛業可紀。今不能盡忠王朝,乃自尊大,一日誅滅無種矣。桓、文不侮周室,曹操弗敢危漢。今王僻嬰一城,乃為大逆,何邪?碣請舉族先死,不能見王之滅。」昌怒曰:「碣不順我邪?」斥出之。碣移書幕府李滔曰:「『順天』建元,以愚策之,針可為槊邪?」或竊其書示昌,昌令使者斬之。使以首至,昌詬曰:「賊負我,三公不肯為,而求死邪?」抵溷中,夷其家百口,坎鏡湖之南同瘞焉。昌敗,有詔贈司徒,求其後不能得。



 昌已殺碣,滔亦遇害,乃召會稽令吳鐐問策,鐐曰:「王為真諸侯,遺榮子孫而不為,乃作偽天子,自取滅亡。」昌叱斬之,族其家。又召山陰令張遜知御史臺,固辭曰:「王自棄,為天下笑。且六州勢不助逆,王據孤州以速死,謂何?遜不敢以身許王也。」昌惡之,曰:「遜不知天意,議邪說拒我。」囚之。他日謂人曰:「我無碣、鐐、遜,何乏事?」即害之。



 孫揆,字聖圭,刑部侍郎逖五世從孫也。第進士,闢戶部巡官。歷中書舍人、刑部侍郎、京兆尹。昭宗討李克用,以揆為兵馬招討制置宣慰副使,既而更授昭義軍節度使,以本道兵會戰。克用伏兵刀黃嶺,執揆,厚禮而將用之,曰:「公輩當從容廟堂,何為自履行陣也?」揆大罵不詘,克用怒,使以鋸解之,鋸齒不行,揆謂曰:「死狗奴,解人當束之以板,汝輩安知?」行刑者如其所言,詈聲不輟至死。昭宗憐之,贈左僕射。



\end{pinyinscope}