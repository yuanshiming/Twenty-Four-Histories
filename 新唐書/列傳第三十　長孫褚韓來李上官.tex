\article{列傳第三十 長孫褚韓來李上官}

\begin{pinyinscope}

 長孫無忌,字輔機。性通悟,博涉書史。始,高祖兵度河,進謁長春宮樂、非攻;主張尚賢、尚同,反對世襲特權。認為天有意志,,授渭北道行軍典簽。從秦王征討有功,累擢比部郎中、上黨縣公。



 皇太子建成毒王,王病,舉府危駭。房玄齡謂無忌曰:「禍隙已芽,敗不旋踵矣。夫就大計者遺細行,周公所以絀管、蔡也。」遂俱入白王,請先事誅之,王未許。無忌曰:「大王以舜何如人?」王曰:「浚哲文明,為子孝,為君仁,又何議哉?」對曰:「向使浚井弗出,得為孝乎?塗廩弗下,得為仁乎?大杖避,小杖受,良有以也。」王未決。事益急,乃遣無忌陰召房玄齡、杜如晦定計。無忌與尉遲敬德、侯君集、張公謹、劉師立、公孫武達、獨孤彥雲、杜君綽、鄭仁恭、李孟嘗討難,平之。王為皇太子,授左庶子。即位,遷吏部尚書,以功第一,進封齊國公。帝以無忌皇后兄,又少相友,眷倚日厚,常出入臥內。進尚書右僕射。突厥頡利可汗已盟而政亂,諸將請遂討之。帝顧新歃血,不取為失機,取之失信,計猶豫,以問大臣。蕭瑀曰:「兼弱攻昧,討之便。」無忌曰:「今我務戢兵,待夷狄至,乃可擊。使遂弱,且不能來,我又何求?臣謂按甲存信便。」帝曰:「善。」然卒取突厥。



 或有言無忌權太盛者,帝持表示無忌曰:「我與公君臣間無少疑,使各懷所聞不言,斯則蔽矣。」因普示君臣曰:「朕子幼,無忌於我有大功,視之猶子也。疏間親、新間舊之謂不順,朕無取焉。」無忌亦自懼貴且亢,後又數言之,遂解僕射,授開府儀同三司。與房玄齡、杜如晦、尉遲敬德皆以元勛封一子郡公。進冊司空,知門下、尚書省事,無忌辭,又因高士廉口陳「以外戚位三公,嫌議者謂天子以私後家」。帝曰:「朕任官必以才,不者,雖親若襄邑王神符,不妄授;若才,雖仇如魏徵,不棄也。夫緣后兄愛暱,厚以子女玉帛,豈不得?以其兼文武兩器,朕故相之,公等孰不曰然?」無忌固讓,詔答曰:「黃帝得力牧,為五帝先;夏禹得咎繇,為三王祖;齊桓得管仲,為五伯長;朕得公,遂定天下。公其無讓!」帝又思所與共艱難,賴無忌以免,作《威鳳賦》以賜,且況其功。



 帝欲功臣並世襲刺史,貞觀十一年,乃詔有司:「朕憑明靈之祐,賢佐之力,克翦多難,清宇內。蓋時屯共資其力,世安專享其利,朕所不取。刺史,古諸侯,雖名不同,而監統一也。無忌等義貫休戚,效挺夷險,嘉庸懿績,簡在朕心。其改錫土宇,用世及之制。」乃以無忌為趙州刺史,以趙為公國;房玄齡宋州刺史,國於梁;杜如晦贈密州刺史,國於萊;李靖濮州刺史,國於衛;高士廉申州刺史,國於申;侯君集陳州刺史,國於陳;道宗鄂州刺史,王江夏;孝恭觀州刺史,王河間;尉遲敬德宣州刺史,國於鄂;李勣蘄州刺史,國於英;段志玄金州刺史,國於褒;程知節普州刺史,國於盧;劉弘基朗州刺史,國於夔;張亮澧州刺史,國於鄖。凡十有四人。餘官食邑尚不在。無忌等辭曰:「群臣披荊刺,事陛下。今四海混一,誠不願違遠左右,而使世牧外州,與遷徙等。」帝曰:「割地封功臣,欲公等後嗣長為籓翰,而薄山河之誓,反為怨望,朕亦安可強公土宇邪?」遂止。後帝幸其第,自家人姻婭勞賜皆有差。久之,進位司徒。



 太子承乾廢,帝欲立晉王,未決,坐兩儀殿,群臣已罷,獨留無忌、玄齡、勣言東宮事,因曰:「我三子一弟,未知所立,吾心亡聊。」即投床,取佩刀自向,無忌等驚,爭抱持,奪刀授晉王,而請帝所欲立。帝曰:「我欲立晉王。」無忌曰:「謹奉詔,異議者斬!」帝顧王曰:「舅許汝矣,宜即謝。」王乃拜。帝復曰:「公等與我意合,天下其謂何?」答曰:「王以仁孝聞天下久矣,固無異辭;有如不同,臣負陛下百死。」於是遂定。以無忌為太子太師、同中書門下三品,「同三品」自此始。帝又欲立吳王恪,無忌密爭止之。帝征高麗,詔攝侍中。還,辭師傅官,聽罷太子太師,遙領揚州都督。



 帝嘗從容問曰:「朕聞君聖臣直,人常苦不自知,公宜面攻朕得失。」無忌曰:「陛下神武聖文,冠卓千古,性與天道,非臣等愚所及,誠不見有所失。」帝曰:「朕冀聞過,公等乃相諛悅。朕當評公等可否以相規。」謂:「高士廉心術警悟,臨難不易節,所乏者骨鯁耳。唐儉有辭,善和解人,酒杯流行,發言可意,事朕二十年,未嘗一言國家事。楊師道性謹審,自能無過,而懦不更事,緩急非可倚。岑文本敦厚,文章、論議其所長也,謀常經遠,自當不負於物。劉洎堅正,其言有益,不輕然諾於人,能自補闕。馬周敏銳而正,評裁人物,直道而行,所任皆稱朕意。褚遂良鯁亮,有學術,竭誠親於朕,若飛鳥依人,自加憐愛。無忌應對機敏,善避嫌,求於古人,未有其比;總兵攻戰,非所善也。」



 二十三年,帝疾甚,召入臥內,帝引手捫無忌頤,無忌哭,帝感塞,不得有所言。翌日,與遂良入受詔,顧遂良曰:「我有天下,無忌力也。爾輔政,勿令讒毀者害之。」有頃,崩。方在離宮,皇太子悲慟,無忌曰:「大行以宗廟、社稷屬殿下,宜速即位。」因秘不發喪,請還宮。



 太子即位,是為高宗。進無忌太尉,檢校中書令,猶知門下、尚書二省。固辭尚書省,許之。帝欲立武昭儀為後,無忌固言不可。帝密以寶器錦帛十餘車賜之,又幸其第,擢三子皆朝散大夫,昭儀母復詣其家申請。許敬宗數勸之,無忌厲色折拒。帝後召無忌、遂良及於志寧言後無息,昭儀有子,必欲立之者。無忌已數諫,即曰:「先帝付托遂良,願陛下訪之。」遂良極道不可,帝不聽。



 後既立,以無忌受賜而不助己,銜之。敬宗揣後指,陰使洛陽人李奉節上無忌變事,與侍中辛茂將臨按,傅致反狀。帝驚曰:「將妄人構間,殆不其然。」敬宗具言:「反跡已露,陛下不忍,非社稷之福。」帝泣曰:「我家不幸,高陽公主與我同氣,往謀反,今舅復爾,使我重愧天下,奈何?」對曰:「房遺愛口乳臭,與女子反,安能就事?無忌奸雄,天下所畏伏,一旦竊發,陛下誰使御之?今即急,恐攘袂一呼,以嘯同惡,且為宗廟憂。陛下不見隋室乎?宇文化及父宰相,弟尚主,而身掌禁兵,煬帝處之不疑,然而起為戎首,遂亡隋。願陛下決之。」帝猶疑,更詔審核。明日,敬宗言無忌反明甚,請逮捕。帝泣曰:「舅果爾,我決不忍殺,後世其謂我何?」敬宗曰:「漢文帝舅薄昭,從代來有功,後坐殺人,帝惜撓法,令朝臣喪服就哭之,昭自殺,良史不以為失。今無忌忘先帝之德,舍陛下至親,乃欲移社稷、敗宗廟,豈特昭比邪?在法夷五族。臣聞當斷不斷,反受其亂。乘機亟行,緩必生變。無忌與先帝謀取天下,天下伏其智,王莽、司馬懿之流。今逆徒自承,何疑而不決?」帝終不質問。遂下詔削官爵封戶,以揚州都督一品俸置於黔州,所在發兵護送;流其子秘書監沖等於嶺外;從弟渝州刺史知仁貶翼州司馬。後數月,又詔司空勣、中書令敬宗、侍中茂將等覆按反獄。敬宗令大理正袁公瑜、御史宋之順等即黔州暴訊。無忌投繯卒,沖免死,殺族子祥,流族弟思於檀口,大抵期親皆謫徙。



 初,無忌與遂良悉心奉國,以天下安危自任,故永徽之政有貞觀風。帝亦賓禮老臣,拱己以聽。綱紀設張,此兩人維持之也。既二後廢立計不合,奸臣陰圖,帝暗於聽受,卒,以屠覆,自是政歸武氏,幾至亡國。



 上元元年,追復官爵,以孫元翼襲封。初,無忌自作墓昭陵塋中,至是許還葬。文宗開成三年,詔曰:「每覽國史至太尉無忌事,未嘗不廢卷而嘆。其以裔孫鈞為猗氏令。」



 無忌從父敞,字休明。隋煬帝為晉王,敞以庫直從畋驪山,王凌危逐鹿,諫曰:「大王冒垂堂,淫原獸,可乎?」王遂止。即位,頗見識擢。及幸江都,留守禁禦。高祖入關,率子弟謁新豐,授將作少監,出為杞州刺史。貞觀初,坐受賕免。太宗以後屬,歲私給稟,償其費。累封平原郡公。卒贈幽州都督,謚曰良,陪葬昭陵。



 從父弟操,字元節。父覽,為周大司徒、薛國公。操有學術。初,高祖闢署相國府金曹參軍。未幾,檢校虞州刺史。從秦王征討,常侍旁,與聞秘謀。徙陜州,城中無井,人勤於汲,操為釃河溜入城,百姓利安。以母喪解,長老守闕頌遺愛。服除,封樂壽縣男。為齊、揚、益三州刺史,課皆最,下詔褒揚。永徽初,以陜州刺史卒,贈吏部尚書,謚曰安,葬給鼓吹,至虞罷。



 子詮,尚新城公主。詮女兄為韓瑗妻。無忌得罪,詮流巂州,有司希旨殺之。詮有甥趙持滿者,工書,善騎射,力搏虎,走逐馬,而仁厚下士,京師無貴賤愛慕之。為涼州長史,嘗逐野馬,射之,矢洞於前,邊人畏伏。詮之貶,許敬宗懼持滿才能仇己,追至京,屬吏訊搒,色不變,曰:「身可殺,辭不可枉!」吏代為占,死獄中。



 無忌族叔順德。順德仕隋為右勛衛,征遼當行,亡命太原,素為高祖親厚。太宗將起兵,令與劉弘基募士於外,聲備賊,至數萬人,乃結隊按屯。大將軍府建,授統軍,從平霍邑、臨汾、絳郡有功。與劉文靜擊屈突通於潼關,通將奔洛陽,順德跳追桃林,執通以獻,遂定陜縣。以多進左驍衛大將軍,封薛國公。討建成餘黨,食千二百戶,賜宮女,詔宿內省。俄以受賕為有司劾發,帝曰:「順德元勛外戚,爵隆位厚至矣。若令觀古今自鑒,有以益國家者,朕當與共府庫,何至以貪冒聞乎?」因賜帛數十愧切之。大理少卿胡演曰:「順德以賂破法,不可赦,奈何又賜之?」帝曰:「使有恥者,得賜甚於戮;如不能,乃禽獸也,殺之何益?」



 李孝常謀反,坐與交,削籍為民。歲餘,帝閱功臣圖,見其像,憐之,遣宇文士及視順德,順德方頹然醉,遂召為澤州刺史,復爵邑。順德素少檢,侈放自如,至是折節為政,以嚴明稱。先時守長多通餉問,順德繩擿無所容,遂為良吏。前刺史張長貴、趙士達占部中腴田數十頃,奪之以給貧單。尋坐累還第。喪息女,感疾甚,帝薄之,謂房玄齡曰:「順德無剛氣,以兒女牽愛至大病,胡足恤?」未幾,卒,遣使吊之,贈荊州都督,謚曰襄。貞觀十三年,封邳國公。永徽中,加贈開府儀同三司。



 褚遂良,字登善,通直散騎常侍亮子。隋大業末,為薛舉通事舍人。仁杲平,授秦王府鎧曹參軍。貞觀中,累遷起居郎。博涉文史,工隸楷。太宗嘗嘆曰:「虞世南死,無與論書者!」魏徵白見遂良,帝令侍書。帝方博購王羲之故帖,天下爭獻,然莫能質真偽。遂良獨論所出,無舛冒者。



 十五年,帝將有事太山,至洛陽,星孛太微,犯郎位。遂良諫曰:「陛下撥亂反正,功超古初,方告成岱宗,而彗輒見,此天意有所未合。昔漢武帝行岱禮,優柔者數年,臣愚願加詳慮。」帝寤,詔罷封禪。



 遷諫議大夫,兼知起居事。帝曰:「卿記起居,大抵人君得觀之否?」對曰:「今之起居,古左右史也,善惡必記,戒人主不為非法,未聞天子自觀史也。」帝曰:「朕有不善,卿必記邪?」對曰:「守道不如守官,臣職載筆,君舉必書。」劉洎曰:「使遂良不記,天下之人亦記之矣。」帝曰:「朕行有三:一,監前代成敗,以為元龜,二,進善人,共成政道;三,斥遠群小,有受讒言。朕能守而勿失,亦欲史氏不能書吾惡也。」



 是時,魏王泰禮秩如嫡,群臣未敢諫。帝從容訪左右曰:「方今何事尤急?」岑文本泛言禮義為急,帝以不切,未領可。遂良曰:「今四方仰德,誰弗率者?唯太子、諸王宜有定分。」帝曰:「有是哉!朕年五十,日以衰怠,雖長子守器,而弟、支子尚五十人,心常念焉。自古宗姓無良,則傾敗相仍,公等為我簡賢者保傅之。夫事人久,情媚熟,則非意自生,其令王府官不得過四考,著為令。」帝嘗怪:「舜造漆器,禹雕其俎,諫者十餘不止,小物何必爾邪?」遂良曰:「雕琢害力農,纂繡傷女工,奢靡之始,危亡之漸也。漆器不止,必金為之,金又不止,必玉為之,故諫者救其源,不使得開。及夫橫流,則無復事矣。」帝咨美之。



 於時皇子雖幼,皆外任都督、刺史,遂良諫曰:「昔二漢以郡國參治,雜用周制。今州縣率仿秦法,而皇子孺年並任刺史,陛下誠以至親捍四方。雖然,刺史,民之師帥也,得人則下安措,失人則家勞攰。故漢宣帝曰:『與我共治,惟良二千石乎。』臣謂皇子未冠者,可且留京師,教以經學,畏仰天威,不敢犯禁,養成德器,審堪臨州,然後敦遣。昔東漢明、章諸帝,友愛子弟,雖各有國,幼者率留京師,訓飭以禮。訖其世,諸王數十百,惟二人以惡敗,自餘餐和染教,皆為善良。此前事已驗,惟陛下省察。」帝嘉納。



 太子承乾廢,魏王泰間侍,帝許立為嗣,因謂大臣曰:「泰昨自投我懷中云:『臣今日始得為陛下子,更生之日也。臣惟有一子,百年後,當殺之,傳國晉王。』朕甚憐之。」遂良曰:「陛下失言。安有為天下主而殺其愛子,授國晉王乎?陛下昔以承乾為嗣,復寵愛泰,嫡庶不明,紛紛至今。若必立泰,非別置晉王不可。」帝泣曰:「我不能。」即詔長孫無忌、房玄齡、李勣與遂良等定策立晉王為皇太子。



 時飛雉數集宮中,帝問:「是何祥也?」遂良曰:「昔秦文公時,有侲子化為雉,雌鳴陳倉,雄鳴南陽。侲子曰:『得雄者王,得雌者霸。』文公遂雄諸侯,始為寶雞祠。漢光武得其雄,起南陽,有四海。陛下本封秦,故雄雌並見,以告明德。」帝悅,曰:「人之立身,不可以無學。遂良所謂多識君子哉!」俄授太子賓客。



 薛延陀請婚,帝己納其聘,復絕之。遂良曰:「信為萬事本,百姓所歸。故文王許枯骨而不違,仲尼去食存信,貴之也。延陀,曩一俟斤耳。因天兵北討,蕩平沙塞,威加諸外,而恩結於內,以為餘寇不可以無酋長,故璽書鼓纛,立為可汗。負抱之恩,與天無極。數遣使請婚於朝,陛下既開許,為御北門受獻食。今一朝自為進退,所惜少,所失多,虧信夷狄,方生嫌恨,殆不可以訓戎兵、勵軍事也。且龍沙以北,部落牛毛,中國擊之不能盡,亦猶可北敗,芮芮興,突厥亡,延陀盛。是以古人虛外實內,懷之以德。使為惡,在夷不在華;失信,在彼不在此也。惟陛下裁幸。」不納。



 帝欲自討遼東,遂良固勸無行:「一不勝,師必再興;再興,為忿兵。兵忿者,勝負不可必。」帝然可。會李勣詆其計,帝意遂決東。遂良懼,上言:「臣請譬諸身。兩京,腹心也;四境,手足也;殊裔絕域,殆非支體所屬。高麗王陛下所立,莫離支殺之。討其逆,夷其地,固不可失,但遣一二慎將,付銳兵十萬,翔■雲輣,唾手可取。昔侯君集、李靖皆庸人爾,猶能撅高昌,纓突厥,陛下止發蹤指示,得歸功聖明。前日從陛下平天下,虓士爪臣,氣力未衰,可驅策,惟陛下所使。臣聞涉遼而左,或水潦,平地淖三尺,帶方、玄菟,海壤荒漫,決非萬乘六師所宜行。」是時,帝銳意蕩平,不見省。進黃門侍郎,參綜朝政。莫離支遣使貢金,遂良曰:「古者討殺君之罪,不受其賂。魯納郜鼎太廟,《春秋》譏之。今莫離支所貢不臣之篚,不容受。」詔可,以其使屬吏。



 帝既平高昌,歲調兵千人往屯,遂良誦諍不可,帝志取西域,寘其言不用。西突厥寇西州,帝曰:「往魏徵、褚遂良勸我立麴文泰子弟,不用其計,乃今悔之。」帝於寢宮側別置院居太子,遂良諫,以為「朋友深交者易怨,父子滯愛者多愆。宜許太子間還東宮,近師傅,專學藝,以廣懿德。」帝從其言。會父喪免,起復,拜中書令。



 帝寢疾,召遂良、長孫無忌曰:「嘆武帝寄霍光,劉備托諸葛亮,朕今委卿矣。太子仁孝,其盡誠輔之。」謂太子曰:「無忌、遂良在,而毋憂。」因命遂良草詔。高宗即位,封河南縣公,進郡公。坐事出為同州刺史。再歲,召拜吏部尚書、同中書門下三品,監修國史,兼太子賓客。進拜尚書右僕射。



 帝將立武昭儀,召長孫無忌、李勣、於志寧及遂良人。或謂無忌當先諫,遂良曰:「太尉,國元舅,有不如意,使上有棄親之譏。」又謂勣上所重,當進,曰:「不可。司空,國元勛,有不如意,使上有斥功臣之嫌。」曰:「吾奉遺詔,若不盡愚,無以下見先帝。」既入,帝曰:「罪莫大於絕嗣,皇后無子,今欲立昭儀,謂何?」遂良曰:「皇后本名家,奉事先帝。先帝疾,執陛下手語臣曰:『我兒與婦今付卿!」且德音在陛下耳,可遽忘之?皇后無它過,不可廢。」帝不悅。翌日,復言,對曰:「陛下必欲改立後者,請更擇貴姓。昭儀昔事先帝,身接帷第,今立之,奈天下耳目何?」帝羞默。遂良因致笏殿階,叩頭流血,曰:「還陛下此笏,丐歸田里。」帝大怒,命引出。武氏從幄後呼曰:「何不撲殺此獠?」無忌曰:「遂良受顧命,有罪不加刑。」會李勣議異,武氏立,乃左遷遂良潭州都督。



 顯慶二年,徙桂州,未幾,貶愛州刺史。遂良內憂禍,恐死不能自明,乃上表曰:「往者承乾廢,岑文本、劉洎奏東宮不可少曠,宜遣濮王居之,臣引義固爭。明日仗入,先帝留無忌、玄齡、勣及臣定策立陛下。當受遺詔。獨臣與無忌二人在,陛下方草土號慟,臣即奏請即位大行柩前。當時陛下手抱臣頸,臣及無忌請即還京,發哀大告,內外寧謐。臣力小任重,動貽伊戚,螻螘餘齒,乞陛下哀憐。」帝昏懦,牽於武後,訖不省。歲餘,卒,年六十三。



 後二歲,許敬宗、李義府奏長孫無忌逆謀皆遂良驅煽,乃削官爵。二子彥甫、彥沖流愛州,殺之。帝遣詔聽其家北還。神龍中,復官爵。德宗追贈太尉。文宗時,詔以遂良五世孫虔為臨汝尉。安南觀察使高駢表遂良客窆愛州,二男一孫祔。咸通九年,詔訪其後護喪歸葬陽翟雲。



 遂良曾孫璆,字伯玉,擢進士第,累拜監察御史裏行。先天中,突厥圍北庭,詔璆持節監總督諸將,破之。遷侍御史,拜禮部員外郎。而氣象凝挺,不減在臺時。



 韓瑗,字伯玉,京兆三原人。父仲良,武德初,與定律令,建言:「周律,其屬三千,秦、漢後約為五百。依古則繁,請崇寬簡,以示惟新。」於是採《開皇律》宜於時者定之。終刑部尚書、秦州都督府長史、潁川縣公。



 瑗少負節行。博學,曉吏事。貞觀中,以兵部侍郎襲爵。永徽三年,遷黃門侍郎。俄同中書門下三品,監修國史。進侍中,兼太子賓客。王後之廢,瑗雪泣言曰:「皇后乃陛下在籓時先帝所娶,今無罪輒廢,非社稷計。」不納。明日復諫曰:「王者立後,配天地,象日月。匹夫匹婦尚知相擇,況天子乎?《詩》云:『赫赫宗周,褒姒滅之。』臣讀至此,常輟卷太息,不圖本朝親見此禍。宗廟其不血食乎!」帝大怒,詔引出。褚遂良貶潭州都督,明年瑗上言:「遂良受先帝顧托,一德無二,向日論事,至誠懇切,詎肯令陛下後堯、舜而塵史冊哉?遭厚謗醜言,損陛下之明,折志士之銳。況被遷以來,再離寒暑,其責塞矣。願寬無辜,以順眾心。」帝曰:「遂良之情,朕知之矣。其孛戾好犯上,朕責之,詎有過邪?」瑗曰:「遂良,社稷臣。蒼蠅點白,傅致有罪。昔微子既去,殷以亡;張華不死,晉不及亂。陛下富有四海,安於清泰,忽驅逐舊臣,遂不省察乎?」帝愈不聽。瑗憂憤,自表歸田里,不報。



 顯慶二年,許敬宗、李義府奏「瑗以桂州授遂良,桂用武地,倚之謀不軌。」於是貶振州刺史,逾年,卒,年五十四。長孫無忌死,義府等復奏瑗與通謀,遣使即殺之;既至,瑗已死,發棺驗視乃還。追削官爵,籍其家,子孫謫廣州官奴。神龍初,武后遺詔復官爵。自瑗與遂良相繼死,內外以言為讀將二十年。帝造奉天宮,御史李善感始上疏極言,時人喜之,謂為「鳳鳴朝陽」。



 來濟,揚州江都人。父護兒,隋左翊衛大將軍。宇文化及難,闔門死之,濟幼得免。轉側流離,而篤志為文章,善議論,曉暢時務,擢進士。貞觀中,累遷通事舍人。太子承乾敗,太宗問侍臣何以處之,莫敢對。濟曰:「陛下上不失為慈父,太子得盡天年,則善。」帝納之。除考功員外郎。十八年,初置太子司議郎,高其選,而以濟為之,兼崇賢館直學士。遷中書舍人。永徽二年,拜中書侍郎,兼弘文館學士,監脩國史。俄同中書門下三品,封南陽縣男。遷中書令,檢校吏部尚書。



 帝將以武氏為後,濟諫曰:「王者立後,以承宗廟、母天下,宜擇禮義名家、幽閑令淑者,副四海之望,稱神祗之意。故文王興姒,《關睢》之化,蒙被百姓,其福如彼;成帝縱欲,以婢為後,皇統中微,其禍如此。惟陛下詳察。」初,武氏被寵,帝特號「宸妃」。濟與韓瑗諫:「妃有常員,今別立號,不可。」武氏已立,不自安。後更謾言濟等忠鯁,恐前經執奏,輒懷反仄,請加賞慰,而實銜之。帝示濟及瑗,濟等益懼。



 顯慶初,兼太子賓客,進爵為侯。帝嘗從容問馭下所宜,濟曰:「昔齊桓公出游,見老人,命之食,曰:『請遺天下食。』遺之衣,曰:『請遺天下衣。』公曰:『吾府庫有限,安得而給?」老人曰:『春不奪農時,即有食;夏不奪蠶工,即有衣。』由是言之,省徭役,馭下之宜也。」於時山東役丁,歲別數萬人,又議取庸以償雇,紛然煩擾,故濟對及之。二年,兼詹事。尋坐褚遂良事,貶臺州刺史。久之,徙庭州。龍朔二年,突厥入寇,濟總兵拒之,謂其眾曰:「吾嘗絓刑罔,蒙赦死,今當以身塞責。」遂不介胄而馳賊,沒焉,年五十三。贈楚州刺史,給靈轜還鄉。



 初,濟與高智周、郝處俊、孫處約客宣城石仲覽家,仲覽衍於財,有器識,待四人甚厚。私相與言志,處俊曰:「願宰天下。」濟及智周亦然。處約曰:「宰相或不可冀,願為通事舍人足矣。」後濟領吏部,處約始以瀛州書佐入調,濟遽注曰「如志」,遂以處約為通事舍人。後皆至公輔云。



 濟異母兄恆,上元中,為黃門侍郎、同中書門下三品,父本驍將,而恆、濟俱以學行稱,相次知政事。時虞世南子昶無才術,歷將作少匠、工部侍郎,主工作。許敬宗曰:「護兒兒作相,世南男作匠,文武豈有種邪?」



 李義琰,魏州昌樂人,其先出隴西望姓。及進士第,補太原尉。李勣為都督,僚吏憚其威,義琰獨敢廷辨曲直,勣甚禮之。徙白水令,有能名,擢司刑員外郎。義琰姿體魁秀,博學,有智識。累遷中書侍郎。上元中,進同中書門下三品,兼太子右庶子。高宗欲使武后攝國政,義琰與郝處俊固爭,事得寢。章懷太子之廢,盡赦宮臣罪,庶子薛元超等皆蹈舞,義琰獨引咎涕泣,搢紳義之。帝每顧問,必鯁切不回。宅無正寢,弟義璡為市堂材送之。義琰曰:「以吾為國相,且自愧,尚營美宇,是速吾禍,豈愛我者邪?」義璡曰:「凡仕為丞尉,且崇第舍,兄位高,安可逼下哉?」答曰:「不然。事難全遂,物不兩興。既處貴仕,又廣居宇,非有令德,必受其殃。」卒不許。後其木久腐,乃棄之。



 義琰改葬其先,使舅家移塋而兆其所。帝聞,怒曰:「是人不可使秉政。」義琰懼,以疾乞骸骨,遷銀青光祿大夫,聽致仕,乃歸田里。公卿以下悉祖餞通化門外,時人比漢疏廣。垂拱初,起為懷州刺史,自以失武后意,辭不拜,卒。



 子巢,幼豪俊,善騎射,而不治細行。義琰嘗拘之,絕其交游。後亡走闕下,獻書陳利害。拜監察御史,與李義府同按柳奭、韓瑗獄,遷殿中。上書忤旨,貶龍編主簿。



 義琰從祖弟義琛。義琛擢進士第,歷監察御史。貞觀中,文成公主貢金,遇盜於岐州,主名不立。太宗召群御史至,目義琛曰:「是人神情爽拔,可使推捕。」義琛往,數日獲賊。帝喜,為加七階。初,義琰使高麗,其王據榻召見,義琰不拜,曰:「吾,天子使,可當小國之君,奈何倨見我?」王祠屈,為加禮。及義琛再使,亦坐召之,義琛匍匐拜伏。時人由是見兄弟優劣。



 累遷刑部侍郎。為雍州長史,時關輔大饑,詔貧人就食商、鄧,義琛恐流徙不還,上疏固爭。左遷黎州都督,終岐州刺史。



 子綰,為柏人令,有仁政,縣為立祠。



 上官儀,字游韶,陜州陜人。父弘,為隋江都宮副監,大業末,為陳棱所殺。時儀幼,左右匿免,冒為沙門服。浸工文詞,涉貫墳典。貞觀初,擢進士第,召授弘文館直學士。遷秘書郎。太宗每屬文,遣儀視槁,宴私未嘗不預。轉起居郎。高宗即位,為秘書少監,進西臺同東西臺三品,時以雍州司士參軍韋絢為殿中侍御史,或疑非遷。儀曰:「此野人語耳。御史供奉赤墀下,接武夔龍,簉羽鵷鷺,豈雍州判佐比乎?」時以為清言。儀工詩,其詞綺錯婉媚。及貴顯,人多效之,謂為「上官體」。



 麟德元年,坐梁王忠事下獄死,籍其家。初,武後得志,遂牽制帝,專威福,帝不能堪;又引道士行厭勝,中人王伏勝發之。帝因大怒,將廢為庶人,召儀與議。儀曰:「皇后專恣,海內失望,宜廢之以順人心。」帝使草詔。左右奔告後,後自申訴,帝乃悔;又恐後怨恚,乃曰:「上官儀教我。」後由是深惡儀。始,忠為陳王時,儀為諮議,與王伏勝同府。至是,許敬宗構儀與忠謀大逆,後志也。自褚遂良等元老大臣相次屠履,公卿莫敢正議,獨儀納忠,禍又不旋踵,由是天下之政歸於後,而帝拱手矣。



 子庭芝,歷周王府屬,亦被殺。庭芝女,中宗時為昭容,追贈儀為中書令、秦州都督、楚國公;庭芝黃門侍郎、岐州刺史、天水郡公,以禮改葬。



 贊曰:高宗之不君,可與為治邪?內牽嬖陰,外劫讒言,以無忌之親,遂良之忠,皆顧命大臣,一旦誅斥,忍而不省。反天之剛,撓陽之明,卒使牝硃鳴辰,祚移後家,可不哀哉!天以女戎間唐而興,雖義士仁人抗之以死,決不可支。然瑗、濟、義琰、儀四子可謂知所守矣。噫,使長孫不逐江夏、害吳王,褚不譖死劉洎,其盛德可少訾乎!



\end{pinyinscope}