\article{志第二十八 地理二}

\begin{pinyinscope}

 河南道,蓋古豫、兗、青、徐之域,漢河南、弘農、潁川、汝南、陳留、沛、泰山、濟陰、濟南、東萊、齊、山陽、東海、瑯邪、北海、千乘、東郡,及梁、楚、魯、東平、城陽、淮陽、菑川、高密、泗水等國,暨平原、渤海、九江之境。洛、陜負河而北承了笛卡爾的「動物是機器」的觀點,認為「人也是機器」,,為實沈分;負河而南,虢、汝、許及新鄭之地,為鶉火分;鄭、汴、陳、蔡、潁為壽星分;宋、亳、徐、宿、鄆、曹、濮為大火分;兗、海、沂、泗為降婁分;青、淄、密、登、萊、齊、棣為玄枵分;滑為娵訾分;濠為星紀分。為府一,州二十九,縣百九十六。其名山:三崤、少室、砥柱、蒙、嶧、嵩高、泰嶽。其大川:伊、洛、汝、潁、沂、泗、淮、濟。厥賦:絹、施、綿、布。厥貢:絲布、葛、席、埏埴盎缶。



 東都,隋置,武德四年廢。貞觀六年號洛陽宮,顯慶二年曰東都,光宅元年曰神都,神龍元年復曰東都,天寶元年曰東京,上元二年罷京,肅宗元年復為東都。皇城長千八百一十七步,廣千三百七十八步,周四千九百三十步,其崇三丈七尺,曲折以象南宮垣,名曰太微城。宮城在皇城北,長千六百二十步,廣八百有五步,周四千九百二十一步,其崇四丈八尺,以象北辰籓衛,曰紫微城,武后號太初宮。上陽宮在禁苑之東,東接皇城之西南隅,上元中置,高宗之季常居以聽政。都城前直伊闕,後據中山,左瀍右澗,洛水貫其中,以象河漢。東西五千六百一十步,南北五千四百七十步,西連苑,北自東城而東二千五百四十步,周二萬五千五十步,其崇丈有八尺,武后號曰金城。



 河南府河南郡,本洛州,開元元年為府。土貢:文綾、繒、縠、絲葛、埏埴盎缶、茍杞、黃精、美果華、酸棗。戶十九萬四千七百四十六,口百一十八萬三千九十二。縣二十:有府三十九,曰武定、復梁、康城、柏林、巖邑、陽樊、王陽、永嘉、邵南、慕善、政教、鞏洛、伊陽、懷音、軹城、洛汭、郟鄏、伊川、洛泉、通谷、潁源、宜陽、金谷、王屋、成皋、夏邑、原邑、原城、鶴臺、函谷、千秋、同軌、餞濟、溫城、具茨、寶圖、鈞臺、承雲、軒轅。河南,赤。垂拱四年析河南、洛陽置永昌縣。永昌元年更河南曰合宮。長安二年省永昌。神龍元年復曰河南,二年又曰合宮,唐隆元年復故名。有洛漕新潭,大足元年開,以置租船。龍門山東抵天津,有伊水石堰,天寶十載,尹裴迥置。有瀍水,避武宗名曰吉水,宣宗立,復故名。洛陽,赤。天授三年析洛陽、永昌置來庭縣,長安二年省。神龍二年更洛陽曰永昌,唐隆元年復故名。偃師,畿。天寶七載,尹韋濟以北坡道迂,自縣東山下開新道通孝義橋。西北有故富平津、河陽故關。鞏,畿。有洛口倉。緱氐,次赤。貞觀十八年省,上元二年復置。有恭陵,有和陵,在太平山,本懊來山,天祐元年更名。東南有軒轅故關。陽城,畿。武德四年,王世充偽令王雄來降,以陽城、嵩陽、陽翟置嵩州,又析三縣地置康城縣。貞觀三年州廢,省康城。萬歲登封元年將封嵩山,改陽城曰告成。神龍元年復故名,二年復為告成。天祐二年更名陽邑。有測景臺,開元十一年,詔太史監南宮說刻石表焉。登封,畿。本嵩陽,貞觀十七年省入陽城。永淳元年營奉天宮,分陽城、緱氏復置,二年省。光宅元年復置。萬歲登封元年更名,神龍元年曰嵩陽,二年復曰登封。嵩山有中嶽祠,有少室山;有三陽宮,聖歷三年置。陸渾,畿。有鳴皋山。有漢故關。伊闕,畿。北有伊闕故關。有陸渾山,一名方山。新安,畿。義寧二年以縣置新安郡。武德元年曰穀州,以熊州之澠池隸之,並析置東垣縣。四年省東垣。貞觀元年來屬。有長石山。澠池,畿。貞觀元年徙穀州來治。西五里有紫桂宮,儀鳳二年置。調露二年曰避暑宮,永淳元年曰芳桂宮,弘道元年廢。福昌,畿。本宜陽。義寧二年以宜陽、澠池、永寧置宜陽郡,武德元年曰熊州。二年更宜陽曰福昌,因隋宮為名。四年以洛州之壽安隸之。貞觀元年州廢,以福昌、永寧二縣隸穀州。六年徙穀州來治。八年以虢州之長水隸之。顯慶二年州廢,以福昌、永寧、長水來屬。西十七里有蘭昌宮;有故隋福昌宮,顯慶三年復置。有女幾山。長水,畿。本長淵,隸弘農郡,義寧元年更名。武德元年隸虢州,貞觀八年隸穀州,顯慶二年來屬。有錫。西有高門關、松陽故關、鵜鶘故關。永寧,畿。本熊耳,義寧二年更名,隸宜陽郡。武德三年以永寧、崤置函州。八年州廢,以永寧隸熊州,崤隸陜州。西五里有崎岫營,西三十三里有蘭峰宮,皆顯慶三年置。壽安,畿。初隸穀州,貞觀七年來屬。西二十九里有連昌宮,顯慶三年置。西南四十里萬安山有興泰宮,長安四年置,並析置興泰縣,神龍元年省。有錦屏山,武后所名。密,畿。武德三年以縣置密州,並置零水、洧源二縣。四年州廢,省零水、洧源,以密隸鄭州。龍朔二年來屬。有羽山。河清,畿。本大基,武德二年置,隸懷州,八年省。咸亨四年析河南、洛陽、新安、王屋、濟源、河陽復置,並置柏崖縣,尋省柏崖。先天元年更名。會昌三年隸孟州,尋還屬,後廢。咸通中復置。有柏崖倉。潁陽,畿。本武林,載初元年析河南、伊闕、嵩陽置。開元十五年更名。西北有大谷故關。倚箔山有鍾乳,貞觀七年採。伊陽,畿。先天元年析陸渾置。有太和山。有銀、銅、錫。伊水有金。王屋。畿。武德元年更名邵伯,隸邵州。貞觀元年州廢,隸懷州。顯慶二年復故名,來屬。有王屋山。



 汝州臨汝郡,雄。本伊州襄城郡,貞觀八年更州名,天寶元年更郡名。土貢:施。戶六萬九千三百七十四,口二十七萬三千七百五十六。縣七:有府四,曰龍興、魯陽、梁川、郟城。梁,望。本承休。又有梁縣在西南四十五里。貞觀元年省梁,更承休曰梁。西南五十里有溫湯,可以熟米。又有黃女湯。高宗置溫泉頓。有石樓山、永仁山。郟城,緊。魯山,上。王世充置魯州,武德四年廢。俄以魯山、滍陽復置魯州。貞觀九年州廢,省滍陽,以魯山來屬,有堯山。有銀。有漢故關。葉,緊。本隸許州,武德四年以縣置葉州,五年州廢,隸北澧州。貞觀八年隸魯州,州廢,隸許州。開元三年,以葉、襄城及唐州之方城、豫州之西平、許州之舞陽置仙州,二十七年州廢,縣還故屬,未幾以葉來屬。大歷四年復以葉、襄城置仙州,又析置仙鳧縣,以許州之舞陽、蔡州之西平、唐州之方城隸之。五年州廢,省仙鳧,餘縣皆還故屬。有黃城山、白石山。襄城,望。武德元年以縣置汝州,並置汝墳、期城二縣。貞觀元年州廢,省汝墳、期城,以襄城隸許州,開元二十七年來屬,二十八年還隸許州,天寶七載復來屬。龍興,上。本湍陽,武德四年置,貞觀元年省。證聖元年析郟城、魯山復置,曰武興。神龍元年更名中興,尋又更名。臨汝。上。先天元年置。有清暑宮,在鳴皋山南,貞觀中置。



 右都畿採訪使,治東都城內。



 陜州陜郡,大都督府,雄。本弘農郡,義寧元年置。武德元年曰陜州。三年兼置南韓州,四年廢南韓州。天寶元年更郡名。天祐元年為興唐府,縣次畿、赤。哀帝初復故。土貢:麰麥、栝蔞、柏實。戶二萬九百五十八,口十七萬二百三十八。縣六:府十五,曰曹陽、崇樂、華望、安城、桃林、夏臺、萬歲、安戎、河北、忠孝、上陽、底柱、夏川、望陜、古亭。陜,望。有大陽故關,即茅津,一曰陜津,貞觀十一年造浮梁;有南、北利人渠,南渠,貞觀十一年太宗東幸,使武候將軍丘行恭開;有陜城宮;有廣濟渠,武德元年,陜東道大行臺金部郎中長孫操所開,引水入城,以代井汲;有太原倉;有峴山。峽石,上。本崤,義寧二年省,武德元年復置。貞觀十四年移治峽石塢,因更名。有底柱山,山有三門,河所經,太宗勒銘;有繡嶺宮,顯慶三年置;東有神雀臺,天寶二年以赤雀見置。靈寶,望。本桃林,義寧元年隸虢郡,武德元年來屬。天寶元年獲寶符於縣南古函谷關,因更名。有浢津,義寧元年置關,貞觀元年廢關,置津;有桃源宮,武德元年置。夏,望。本隸虞州,貞觀十七年隸絳州,大足元年來屬,尋還隸絳州,乾元三年復來屬。芮城,望。武德二年以芮城、河北、永樂置芮州。貞觀元年州廢,以永樂隸鼎州,芮城、河北來屬。平陸。望。本河北,隸蒲州,貞觀元年來屬。天寶元年,太守李齊物開三門以利漕運,得古刃,有篆文曰「平陸」,因更名。三門西有鹽倉,東有集津倉。有瑟瑟穴,有銀穴三十四,銅穴四十八,在覆釜、三錐、五岡、分雲等山。



 虢州弘農郡,雄。本虢郡,治盧氏。義寧元年,析隋弘農郡三縣置。貞觀八年徙治弘農。天寶元年更郡名。土貢:施、瓦硯、麝、地骨皮、梨。戶二萬八千二百四十九,口八萬八千八百四十五。縣六:有府四,曰鼎湖、全節、金門、開方。弘農,緊。本隋弘農郡,義寧元年曰鳳林,領弘農、閿鄉、湖城。武德元年曰鼎州,因鼎湖為名。貞觀八年州廢,縣皆來屬。神龍初避孝敬皇帝諱,曰恆農,開元十六年復故名。南七里有渠,貞觀元年,令元伯武引水北流入城。閿鄉,望。貞觀元年來屬。有潼關、大谷關,武德二年廢;有鳳陵關,貞觀元年廢;有軒游宮,故隋別院宮,咸亨五年更名。湖城,望。義寧元年置。乾元三年更名天平,大歷四年復舊。有故隋上陽宮,貞觀初置,咸亨元年廢。縣東故道濱河,不井汲,馬多渴死,天寶八載,館驛使、御史中丞宋渾開新路,自稠桑西由晉王斜。有熊耳山;覆釜山,一名荊山。硃陽,上。龍朔元年隸商州,萬歲通天二年隸洛州,後來屬。有鐵。玉城,上。義寧元年置。盧氏。上。武德元年置。南有硃陽關,武德八年廢。



 滑州靈昌郡,望。本東郡,天寶元年更名。土貢:方紋綾、紗、絹、席、酸棗人。戶七萬一千九百八十三,口四十二萬二千七百九。縣七:有宜義軍,大歷七年置,本永平。十四年徙屯蔡州,興元元年復還。貞元元年曰義成軍,光啟二年更名。白馬,望。衛南,緊。匡城,望。有長垣縣,貞觀八年省。韋城,望。王世充置燕州,偽刺史單宗來降,復為縣。胙城,緊。武德二年置胙州,並置南燕縣。四年州廢,省南燕,以胙城來屬。酸棗,望。本隸東梁州。武德三年析酸棗、胙城置守節縣,四年省。貞觀八年州廢,來屬。靈昌。緊。王世充置興州,世充平,廢。



 鄭州滎陽郡,雄。武德四年置,治虎牢城。貞觀七年徙治管城。土貢:絹、龍莎。戶七萬六千六百九十四,口三十六萬七千八百八十一。縣七:管城,望。武德四年以管城、中牟、原武、陽武、新鄭置管州,並置須水、清池二縣。貞觀元年州廢,省須水、清池,以管城、原武、陽武、新鄭來屬。有僕射陂,後魏孝文帝賜僕射李沖,因以為名。天寶六載更名廣仁池,禁漁採。滎陽,上。天授二年析置武泰縣,隸洛州,尋省,更滎陽曰武泰。萬歲通天元年復為滎陽,又別置武泰縣,二年省,更滎陽曰武泰。神龍元年復故名,二年來屬。滎澤,望。原武,緊。本原陵,唐初更名,復漢舊。陽武,望。本原武城,武德四年置。新鄭,望。中牟。緊。本圃田,武德三年更名,以縣置牟州。四年州廢,隸管州。貞觀元年隸汴州,龍朔二年來屬。



 潁州汝陰郡,上。本信州,武德四年置,六年更名。土貢:施、綿、糟白魚。戶三萬七百七,口二十萬二千八百九十。縣四:汝陰,緊。武德初有永安、高唐、永樂、清丘、潁陽等縣,六年省永安、高唐、永樂,貞觀元年省清丘、潁陽,皆入汝陰。南三十五里有椒陂塘,引潤水溉田二百頃,永徽中,刺史柳寶積修。潁上。上。下蔡,上。武德四年置渦州,八年州廢。西北百二十里有大崇陂,八十里有雞陂,六十里有黃陂,東北八十里有湄陂,皆隋末廢,唐復之,溉田數百頃。沈丘。中。本阜州,領沈丘、宛丘。唐初州廢,以宛丘隸陳州,沈丘來屬。後省沈丘入汝陰,神龍二年復置。



 許州潁川郡,望。土貢:絹、席、柿。戶七萬三千三百四十七,口四十八萬七千八百六十四。縣九:長社,望。本潁川,隸汴州。武德四年更名,來屬。州又領黃臺、水隱強二縣,貞觀元年省入焉。繞州郭有堤塘百八十里,節度使高瑀立以溉田。長葛,緊。有小陘山。陽翟,本畿。初隸嵩州,貞觀元年來屬,龍朔二年隸洛州,會昌三年復來屬。有具茨山。許昌,上。鄢陵,上。扶溝,望。武德四年以縣置北陳州,是年州廢,隸洧州。臨潁,上。貞觀元年省繁昌縣入焉。有講武臺,本尚書臺,馬融講書之地,顯慶二年,高宗大閱於此,更名。舞陽,上。本北舞,隸道州。貞觀元年來屬,尋廢。開元四年復置,更名。有鐵。郾城。望。武德四年以郾城、邵陵、北舞、西平置道州。貞觀元年州廢,省邵陵、西平入郾城,隸蔡州。建中二年以郾城、臨潁,陳州之溵水置溵州。貞元二年州廢,縣還故屬。元和十二年復以郾城、上蔡、西平、遂平置溵州。長慶元年州廢,縣還隸蔡州,是年,以郾城來屬。



 陳州淮陽郡,上。土貢:絹。戶六萬六千四百四十二,口四十萬二千四百八十六。縣六:有忠武軍,貞元元年置於許州。天復元年徙屯。宛丘,緊。武德元年析置新平縣,八年省。太康,緊。貞觀元年省扶樂縣入焉。項城,上。武德四年置,以項城、銅陽、南頓、溵水置沈州,並置潁東縣。貞觀元年州廢,省潁東入項城,以溵水來屬。溵水,上。建中二年隸溵州,興元二年州廢,來屬。南頓,上。武德六年省入項城。證聖元年復置,曰光武,以縣有光武祠名。景雲元年復故名。西華。上。武德元年更名箕城,貞觀元年省入宛丘。長壽元年復置,曰武城。神龍元年又曰箕城,景雲元年復故名。有鄧門廢陂,神龍中,令張餘慶復開,引潁水溉田。



 蔡州汝南郡,緊。本豫州,寶應元年更名。土貢:氏玉棋子,四窠、雲花、龜甲、雙距、溪等綾。戶八萬七千六十一,口四十六萬二百五。縣十:汝陽,緊。貞元七年析汝陽、朗山、上蔡、吳房置汝南縣,元和十三年省。朗山,上。本隸北朗州,貞觀元年隸蔡州。遂平,上。本吳房,貞觀元年省,八年復置。元和十二年更名,權隸唐州,長慶元年復來屬。上蔡,緊。新蔡,中。武德四年以新蔡、褒信、舒城置舒州。貞觀元年州廢,省舒城入沈丘。褒信,中。天祐中更曰包孚。新息,上。武德四年以縣置息州,並置淮川、長陵二縣。貞觀元年州廢,省淮川入真陽,長陵入褒信,以新息來屬。有氏玉坑,歲出貢玉。西北五十里有隋故玉梁渠,開元中,令薛務增浚,溉田三千餘頃。真陽,上。載初元年曰淮陽,神龍元年復故名。平輿,中。王世充置輿州,武德七年州廢。貞觀元年省入新蔡,天授二年復置。西平。上。武德初置,貞觀元年省。天授二年分郾城復置,尋又廢。開元四年復置。



 汴州陳留郡,雄。武德四年以鄭州之浚儀、開封,滑州之封丘置。土貢:絹。戶十萬九千八百七十六,口五十七萬七千五百七。縣六:有宣武軍,建中二年置於宋州。興元元年徙屯。浚儀,望。故縣陷李密,縣民王要漢率豪族置縣,自為令。高祖因之,復置汴州,並置小黃、新里二縣,貞觀元年省二縣。開封,望。貞觀元年省入浚儀,延和元年析浚儀、尉氏復置。有湛渠,載初元年引汴注白溝,以通曹、兗賦租。有福源池,本蓬池,天寶六載更名,禁漁採。尉氏,望。本隸潁川郡,王世充置尉州。武德四年廢,以尉氏、扶溝、焉陵置洧州,並置康陰、新汲、宛陵、歸化四縣。貞觀元年州廢,省康陰、宛陵、新汲、歸化,以扶溝、焉陵隸許州,尉氏來屬。封丘,緊。雍丘,望。本隸梁郡。武德四年,以雍丘、陳留、圉城、襄邑、外黃、濟陽置杞州。貞觀元年州廢,省濟陽、圉城、外黃,以襄邑隸宋州,雍丘、陳留來屬。陳留。緊。武德四年置。有觀省陂,貞觀十年,令劉雅決水溉田百頃。



 宋州睢陽郡,望。本梁郡,天寶元年更名。土貢:絹。戶十二萬四千二百六十八,口八十九萬七千四十一。縣十:宋城,望。襄邑,望。本隸杞州,貞觀元年來屬。寧陵,緊。下邑,上。穀熟,上。隋末縣民劉繼叔據之,武德二年置南穀州,授以刺史,四年州廢。楚丘,緊。柘城,緊。貞觀元年省入寧陵、穀熟,永淳元年復置。碭山,上。光化二年,硃全忠以碭山、虞城、單父,曹州之成武,表置輝州。三年置崇德軍。單父,緊。光化三年徙輝州來治。虞城。上。武德四年置東虞州,五年州廢。



 亳州譙郡,望。本譙州,貞觀八年更名。土貢:絹。戶八萬八千九百六十,口六十七萬五千一百二十一。縣七:譙,緊。酂,上。本隸沛郡,武德四年來屬。城父,上。王世充置成州,世充平,廢。武德三年於魯丘堡置文州,並置藥城縣。四年州廢為文城縣,七年省入城父,天祐二年更名焦夷。鹿邑,上。大業十三年,縣民田黑社盜據,號渦州。武德三年來降,復為縣。真源,望。本穀陽,乾封元年更名。戴初元年曰仙源,神龍元年復曰真源。有老子祠,天寶二年曰太清宮。又有洞霄宮,先天太后祠也。永城,上。蒙城。上。本山桑,天寶元年更名。



 徐州彭城郡,緊。土貢:雙絲綾、絹、綿紬、布、刀錯、紫石。戶六萬五千一百七十,口四十七萬八千六百七十六。縣七:彭城,望。秋丘冶有鐵。蕭,上。豐,上。沛,上。武德五年置。滕,上。宿遷,上。本宿預,隸泗州。寶應元年更名,來屬。下邳。上。武德四年以下邳、郯、良城置邳州。貞觀元年州廢,省郯、良城,以下邳隸泗州,又省泗州之淮陽入焉。元和四年來屬。



 泗州臨淮郡,上。本下邳郡,治宿預,開元二十三年徙治臨淮。天寶元年更郡名。土貢:錦、貲布。戶三萬七千五百二十六,口二十萬五千九百五十九。縣四:臨淮,緊。長安四年析徐城置。漣水,上。武德四年以縣置漣州,並置金城縣。貞觀元年州廢,省金城,以漣水來屬。總章元年隸楚州,咸亨五年復故。有新漕渠,南通淮,垂拱四年開,以通海、汧、密等州。盱眙,緊。武德四年以縣置西楚州,八年州廢,隸楚州。光宅初曰建中,後復故名。建中二年來屬,有直河,太極元年,敕使魏景清引淮水至黃土岡,以通揚州。徐城。中。



 濠州鐘離郡,上。「濠」字初作「豪」,元和三年改從「濠」。土貢:絁、綿、絲布、雲母。戶二萬一千八百六十四,口十三萬八千三百六十一。縣三:鐘離,緊。武德七年省塗山縣入焉。南有故千人塘,乾封中脩以溉田。有塗山。定遠,緊。本臨豪,武德三年更名。招義。上。本化明,武德二年析置睢陵縣,三年更化明曰招義,四年省睢陵。大業末,縣民馬簿盜據,號化州。後楊益德殺簿,自號刺史。又置濟陰縣,是年來降。貞觀元年廢化州,省濟陰。



 宿州,上。元和四年析徐州之苻離、蘄,泗州之虹置。大和三年州廢,七年復置。初治虹,後徙治苻離。土貢:絹。縣四:符離,武德四年置。貞觀元年省徐州之諸陽入焉。有西句山,一曰石城。東北九十里有隋故牌湖堤,灌田五百餘頃,顯慶中復脩。虹,中。本夏丘。武德四年以夏丘、穀陽置仁州,又析夏丘置虹及龍亢二縣。六年省夏丘。貞觀八年州廢,省龍亢,以虹隸泗州、穀陽隸北譙州。有銅。有廣濟新渠,開元二十七年,採訪使齊澣開,自虹至淮陰北十八里入淮,以便漕運,即成,湍急不可行,遂廢。蘄,上。顯慶元年省穀陽入焉。臨渙。緊。武德四年以臨渙、永城、山桑、蘄置北譙州。貞觀八年增領穀陽。十七年州廢,以臨渙、永城、山桑隸亳州,穀陽、蘄隸徐州。元和後來屬。



 鄆州東平郡,緊。本治鄆城,貞觀八年徙治須昌。土貢:絹、防風。戶八萬三千四十八,口五十萬一千五百九。縣九:須昌,望。貞觀八年省宿城縣入焉。景龍三年復置宿城縣。貞元四年曰東平,大和四年曰天平,六年省入須昌。壽張,緊。武德四年以縣置壽州,並置壽良縣。五年州廢,省壽良,以壽張來屬。有刀梁山。鄆城,緊。天祐二年曰萬安。鉅野,望。武德四年以縣置麟州。五年州廢,隸鄆州。貞觀元年省乘丘縣入焉。後隸戴州,州廢來屬。盧,緊。本濟州,武德四年析東平郡置。隋曰濟北郡,天寶元年更名濟陽郡。領盧、平陰、長清、東阿、陽谷、範六縣,又置昌城、濟北、穀城、孝感、冀丘、美政六縣。六年省美政、孝感、穀城、冀丘、昌城,八年以範隸濮州,貞觀元年省濟北,天寶十三載郡廢,以長清隸濟州,以盧、平陰、東阿、陽谷來屬。北有碻磝津故關。平陰,緊。大和六年省入盧、東阿。開成二年復置。有龍山。東阿,緊。陽谷,上。中都。上。本平陸,隸兗州。天寶元年更名。貞元十四年來屬。



 齊州濟南郡,上。本齊郡,天寶元年更名臨淄,五載又更名。土貢:絲、葛、絹、綿、防風、滑石、雲母。戶六萬二千四百八十五,口三十六萬五千九百七十二。縣六:歷城,上。有華不注山;有鐵。章丘,上。武德二年,縣民李義滿以縣來降,於平陵置譚州,並置平陵縣,以章丘、亭山、營城、臨邑隸之。八年省營城入平陵,又領臨濟、鄒平。貞觀元年州廢,以平城、亭山、章丘、臨邑、臨濟來屬,鄒平隸淄州。十七年,齊王祐反,平陵人不從,因更名全節。元和十五年省全節入歷城,省亭山入章丘。有大胡山、長白山。臨邑,上。元和十三年析德州之安德置歸化縣,隸德州。大和二年來屬,四年省入臨邑。北有鹿角故關。臨濟,上。武德元年以臨濟、鄒平、長山、高苑,滄州之蒲臺置鄒州。八年州廢,以長山、高苑、蒲臺隸淄州。長清,中。本隸濟州,貞觀十七年來屬。武德元年析置山荏縣,天寶元年曰豐齊,元和十年省。有牛山。西南有四口關,武德中廢。禹城。上。本祝阿,貞觀元年省源陽縣入焉。天寶元年更名。



 曹州濟陰郡,上。土貢:絹、綿、大蛇粟、葶歷。戶十萬三百五十二,口七十一萬六千八百四十八。縣六:濟陰,緊。武德四年析置蒙澤縣,貞觀元年,及定陶省入焉。考城,上。武德四年以縣置東梁州,五年州廢,來屬。元和十四年權隸宋州,尋復故。宛句,上。武德四年析置濟陽縣,隸杞州。貞觀元年省。乘氏,上。武德四年置晉陽縣,尋省。南華,上。本離狐,天寶元年更名。成武。緊。武德四年以成武及宋州之單父、楚丘置戴州,並置高鄉、鑿城二縣,尋省高鄉、鑿城入單父。貞觀十七年州廢,以成武來屬。光化二年,硃全忠表縣隸輝州。



 濮州濮陽郡,上。武德四年置。土貢:絹、犬。戶五萬七千七百八十二,口四十萬六百四十八。縣五:鄄城,緊。武德四年析置永定縣,八年省。北有靈津關。濮陽,緊。武德四年析置昆吾縣,八年省。範,上。武德二年以縣置範州。五年州廢,隸濟州。貞觀八年來屬。雷澤,上。武德四年析置廩城縣,八年省。臨濮。緊。武德四年析雷澤置,並置長城、安丘二縣。五年省長城、安丘。



 青州北海郡,望,土貢:仙紋綾、絲、棗、紅藍、紫草。戶七萬三千一百四十八,口四十萬二千七百四。縣七:益都,望。臨淄,緊。武德八年省時水縣入焉。千乘,緊。武德二年以千乘、博昌、壽光置乘州,並置新河縣。六年省新河。八年州廢,縣來屬。博昌,上。武德八年省樂安、安平二縣入焉。有靈山。壽光,緊。武德二年置。臨朐,上。武德五年置,八年省般陽縣入焉。北海。緊。唐初,營丘民汲嗣率鄉人拒賊,權置杞州。武德二年復為營丘縣。是年,以北海、營丘、下密置濰州;又置連永、平壽、華池、城都、東陽、寒水、訾亭、濰水、汶陽、膠東、華宛、昌安、城平十三縣,六年皆省。入年州廢,省營丘、下密入北海,來屬。長安中,令竇琰於故營丘城東北穿渠,引白浪水曲折三十里以溉田,號竇公渠。



 淄州淄川郡,上。武德元年析齊州之淄川置。土貢:防風、理石。戶四萬二千七百三十七,口二十三萬三千八百二十一。縣四:淄川,上。武德元年析置長白縣,六年省。有鐵。長山,上。高苑,上。景龍元年析置濟陽縣,元和十五年省。南有八會津。鄒平。上。武德元年置。



 登州東牟郡,中都督府。如意元年以萊州之牟平、黃、文登置。神龍三年徙治蓬萊。土貢:貲布、水蔥席、石器、文蛤、牛黃。戶二萬二千二百九十八,口十萬八千九。縣四:有平海軍,亦曰東牟守捉。蓬萊,本黃,神龍三年更名。有銀山、龍山。牟平,中。武德四年以牟平、黃置牟州。六年以登州之觀陽隸萊州。麟德元年析文登復置牟平,來屬。有之罘山。文登,武德四年置登州,以東萊郡之觀陽隸之。六年析置清陽、廓定二縣。及州廢,省清陽、廓定,以文登來屬。有成山。黃。中。先天元年析蓬萊別置。有萊山。



 萊州東萊郡,中。土貢:貲布、水蔥席、石器、文蛤、牛黃。戶二萬六千九百九十八,口十七萬一千五百一十六。縣四:有東萊守捉,亦曰「團結營」。又有蓬萊鎮兵,亦曰「挽強兵」。掖,上。貞觀元年省曲城、當利、曲臺三縣入焉。有東海祠;有鹽井二。昌陽,上。貞觀元年省盧鄉縣入焉。有銀,有鐵;東百四十里有黃銀坑,貞觀初得之。膠水,中。貞觀元年省膠東縣入焉。有鹽。即墨。中。有馬山、中祠山、女姑山;東南有堰,貞觀十年,令仇源築,以防淮涉水;有鹽。



 棣州樂安郡,上。武德四年析滄州之陽信、滳河、樂陵、厭次置。八年州廢,縣還隸滄州。貞觀十七年,復以滄州之厭次,德州之滳河、陽信置。土貢:絹。戶三萬九千一百五十,口二十三萬八千一百五十九。縣五:厭次,上。貞觀元年隸德州。滳河,中。貞觀元年隸德州。陽信,望。貞觀元年省,八年復置。蒲臺,緊。本隸淄州,貞觀六年省入高苑,七年復置。景龍元年來屬。渤海。緊。垂拱四年析蒲臺、厭次置。有鹽。



 兗州魯郡,上都督府。土貢:鏡花綾、雙距綾、絹、雲母、防風、紫石。戶八萬七千九百八十七,口五十八萬六百八。縣十:瑕丘,上。曲阜,緊。貞觀元年省,八年復置。乾封,上。本博城。武德五年以博城、梁父、贏置東泰州,並置肥城、岱二縣。貞觀元年州廢,省梁父、贏、肥城、岱入博城,來屬。乾封元年更名乾封,總章元年又曰博城,神龍元年復曰乾封。有泰山,有東嶽祠,有梁父山、亭亭山、奕奕山、雲雲山、社首山、肅然山、石閭山、蒿里山。泗水,上。鄒,上。有嶧山。任城,緊。龔丘,中。金鄉,望。武德四年以金鄉、方與置金州。五年州廢,縣隸戴州,徙戴州來治,仍析金鄉置昌邑縣。八年省昌邑。貞觀十七年,以單父、楚丘隸宋州,成武隸曹州,鉅野隸鄆州。魚臺,上。本方與,寶應元年更名。元和十四年權隸徐州,尋復故。萊蕪。中。本隸淄州,武德六年省入博城。長安四年以廢贏縣復置,元和十五年省入乾封,大和元年復置。有鐵冶十三,有銅冶十八、銅坑四;有錫;西北十五里有普濟渠,開元六年,令趙建盛開。



 海州東海郡,上。土貢:綾、楚布、紫菜。戶二萬八千五百四十九,口十八萬四千九。縣四:朐山,上。武德四年,析州境置龍沮、曲陽、利城、厚丘、新樂五縣。六年改新樂曰祝其。八年,省龍沮、曲陽入朐山,利城、祝其入懷仁,厚丘入沭陽。東二十里有永安堤,北接山,環城長七里,以捍海潮,開元十四年,刺史杜令昭築。東海,上。武德四年以縣置環州,並置青山、石城、贛榆三縣。八年州廢,省青山、石城、贛榆,以東海來屬。沭陽,中。總章元年隸泗州,咸亨五年復故。懷仁。中。



 沂州瑯邪郡,上。土貢:紫石、鐘乳。戶三萬三千五百一十。口十九萬五千七百三十七。縣五:臨沂,上。武德四年析置蘭山、臨汴、昌樂三縣,六年皆省。費,上。貞觀元年省顓臾縣入焉。丞,上。本蘭陵,武德四年以縣置鄫州,更名,別置蘭陵、鄫城二縣。貞觀元年州廢,省蘭陵、鄫城,以丞來屬。有鐵;有陂十三,畜水溉田,皆貞觀以來築。沂水,上。武德五年以沂水、新泰、莒置莒州。貞觀八年州廢,以莒隸密州,沂水、新泰來屬。有銅;有沂山、龍山;北有穆陵關。新泰。上。有蒙山。



 密州高密郡,上。土貢:貲布、海蛤、牛黃。戶二萬八千二百九十二,口十四萬六千五百二十四。縣四:諸城,上。有鹽。輔唐,上。本安丘,武德六年省郚城縣入焉。乾元二年更名。高密,上。武德三年置,六年省膠西縣入焉。莒。上。有鹽。



 右河南採訪使,治汴州。



\end{pinyinscope}