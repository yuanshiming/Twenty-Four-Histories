\article{志第三十三下 地理七下}

\begin{pinyinscope}

 ○羈縻州



 唐興,初未暇於四夷,自太宗平突厥,西北諸蕃及蠻夷稍稍內屬,即其部落列置州縣。其大者為都督府,以其首領為都督、刺史,皆得世襲。雖貢賦版籍,多不上戶部,然聲教所暨,皆邊州都督、都護所領,著於令式。今錄招降開置之目,以見其盛。其後或臣或叛,經制不一,不能詳見。突厥、回紇、黨項、吐谷渾隸關內道者,為府二十九,州九十。突厥之別部及奚、契丹、靺鞨、降胡、高麗隸河北者,為府十四,州四十六。突厥、回紇、黨項、吐谷渾之別部及龜茲、于闐、焉耆、疏勒、河西內屬諸胡、西域十六國隸隴右者,為府五十一,州百九十八。羌、蠻隸劍南者,為州二百六十一。蠻隸江南者,為州五十一,隸嶺南者,為州九十三。又有黨項州二十四,不知其隸屬。大凡府州八百五十六,號為羈縻云。



 ○關內道



 突厥州十九,府五:



 定襄都督府,貞觀四年析頡利部為二,以左部置,僑治寧朔。領州四:貞觀二十三年分諸部置州三。阿德州以阿史德部置。執失州以執失部置。蘇農州以蘇農部置。拔延州



 右隸夏州都督府



 雲中都督府,貞觀四年析頡利右部置,僑治朔方境。領州五:貞觀二十三年分諸部置州三。舍利州以舍利吐利部置。阿史那州以阿史那部置。綽州以綽部置。思壁州白登州貞觀末隸燕然都護,後復來屬。桑乾都督府,龍朔三年分定襄置,僑治朔方。領州四:貞觀二十三年分諸部置州三。鬱射州以鬱射施部置,初隸定襄,後來屬。藝失州以多地藝失部置。卑失州以卑失部置,初隸定襄,後來屬。叱略州呼延都督府,貞觀二十年置。領州三:貞觀二十三年分諸部置州三。賀魯州,以賀魯部置,初隸雲中都督,後來屬。葛邏州,以葛邏、挹怛部置,初隸雲中都督,後來屬跌州初為都督府,隸北庭,後為州,來屬。



 右隸單于都護府。



 新黎州貞觀二十三年以車鼻可汗之子羯漫陀部置。初為都督府,後為州。渾河州永徽元年,以車鼻可汗餘眾歌邏祿之烏德鞬山左廂部落置。狼山州永徽元年以歌邏祿右廂部落置,為都督府,隸雲中都護。顯慶三年為州,來屬。堅昆都督府,貞觀二十三年以沙缽羅葉護部落置。右隸安北都護府。



 回紇州十八,府九。貞觀二十二年分回紇諸部落置。



 燕然州以多濫葛部地置,初為都督府,及雞鹿、雞田、燭龍三州,隸燕然都護。開元元年來屬,僑治回樂。雞鹿州以奚結部置,僑治回樂。雞田州以阿跌部置,僑治回樂。東皋蘭州以渾部置,初為都督府,並以延陀餘眾置祁連州,後罷都督,又分東、西州,永徽三年皆廢。後復置東皋蘭州,僑治鳴沙。燭龍州貞觀二十三年析瀚海都督之掘羅勿部置,僑治溫池。燕山州僑治溫池。右隸靈州都督府。



 達渾都督府,以延陀部落置,僑治寧朔。領州五:姑衍州步訖若州嵠彈州永徽中收延陀散亡部落置。鶻州低粟州。安化州都督府僑治朔方。寧朔州都督府僑治朔方。僕固州都督府僑治朔方。右隸夏州都督府。



 榆溪州以契芯部置。寘顏州以白部置。居延州以白別部置。稽落州本高闕州,以斛薩部置。永徽元年廢高闕州,更置稽落州,後又廢,三年以阿特部復置。余吾州本玄闕州,貞觀中以骨利干部置,龍朔中更名。浚稽州仙萼州初隸瀚海都護,後來屬。瀚海都督府以回紇部置。金微都督府以僕固部置。幽陵都督府以拔野古部置。龜林都督府貞觀二年以同羅部落置。堅昆都督府以結骨部置。右隸安北都護府。



 黨項州五十一,府十五:貞觀三年,酋長細封步賴內附,其後諸姓酋長相率亦內附,皆列其地置州縣,隸松州都督府。五年又開其地置州十六,縣四十七;又以拓拔赤詞部置州三十二。乾封二年以吐蕃入寇,廢都、流、厥、調、湊、般、匐、器、邇、鍠、率、差等十二州,咸亨二年又廢蠶、黎二州。祿山之亂,河、隴陷吐蕃,乃徙黨項州所存者於靈、慶、銀、夏之境。



 清塞州歸德州僑治銀州境。蘭池都督府芳池都督府相興都督府永平都督府旭定都督府清寧都督府忠順都督府寧保都督府靜塞都督府萬吉都督府樂容州都督府,領州一:東夏州靜邊州都督府,貞觀中置,初在隴右,後僑治慶州之境。領州二十五:布州北夏州思義州思樂州昌塞州吳州天授二年置吳、朝、歸、浮等州。朝州「朝」一作「彭」。歸州「歸」,一作「陽」。浮州祐州貞觀四年置,領縣二:廓川,歸定。卑州西歸州嶂州貞觀四年置。縣四:洛平,顯川,桂川,顯平。餝州開元州歸順州本在山南之西,寶應元年詣梁州刺史內附。淳州貞觀十二年以降戶置於洮州之境,並置素恭、烏城二縣。開元中廢,後為羈糜。烏籠州恤州嵯州貞觀五年置。縣一:相雞。相雞本隸西懷州,貞觀十年來屬。蓋州本西唐州,貞觀四年置,八年更名。縣四:湘水,河唐,曲嶺,祐川。悅州回樂州烏掌州諾州貞觀五年置。縣三:諾川,德歸,籬渭。右隸靈州都督府。



 芳池州都督府,僑治懷安,皆野利氏種落。領州九:寧靜州種州玉州貞觀五年置。縣二:玉山,帶河。濮州林州尹州位州貞觀四年置。縣二:位豐,西使。長州寶州。宜定州都督府,本安定,後更名。領州七:黨州橋州貞觀六年置。烏州西戎州貞觀五年以拓拔赤詞部落置。初為都督府,後為州,來屬。野利州米州還州。安化州都督府,領州七:永和州威州旭州莫州西滄州貞觀六年置,八年更名臺州,後復故名。琮州儒州本西鹽州,貞觀五年以拓拔部置,治故後魏洪和郡之藍川縣地,八年更名。開元中廢,後為羈縻。右隸慶州都督府。



 吐谷渾州二:



 寧朔州初隸樂容都督府,代宗時來屬。右隸夏州都督府;渾州儀鳳中自涼州內附者,處於金明西境置。右隸延州都督府。



 ○河北道



 突厥州二:



 順州順義郡貞觀四年平突厥,以其部落置順、祐、化、長四州都督府於幽、靈之境;又置北開、北寧、北撫、北安等四州都督府。六年順州僑治營州南之五柳戍;又分思農部置燕然縣,僑治陽曲;分思結部置懷化縣,僑治秀容,隸順州;後皆省。祐、化、長及北開等四州亦廢,而順州僑治幽州城中。歲貢麝香。縣一:賓義。瑞州本威州,貞觀十年以烏突汗達干部落置,在營州之境。咸亨中更名。後僑治良鄉之廣陽城。縣一:來遠。右初隸營州都督府,及李盡忠陷營州,以順州隸幽州都督府,徙瑞州於宋州之境。神龍初北還,亦隸幽州都督府。



 奚州九,府一:



 鮮州武德五年析饒樂都督府置。僑治潞之古縣城。縣一:賓從。崇州武德五年析饒樂都督府之可汗部落置。貞觀三年更名北黎州,治營州之廢陽師鎮。八年復故名。後與鮮州同僑治潞之古縣城。縣一:昌黎。順化州縣一:懷遠。歸義州歸德郡總章中以新羅戶置,僑治良鄉之廣陽城。縣一:歸義。後廢。開元中,信安王禕降契丹李詩部落五千帳,以其眾復置。



 奉誠都督府,本饒樂都督府,唐初置,後廢。貞觀二十二年以內屬奚可度者部落更置,並以別帥五部置弱水等五州。開元二十三年更名。領州五:弱水州以阿會部置。祁黎州以處和部置。洛環州以奧失部置。太魯州以度稽部置。渴野州以元俟析部置。



 契丹州十七,府一:



 玄州貞觀二十年以紇主曲據部落置。僑治範陽之魯泊村。縣一:靜蕃。咸州本遼州,武德二年以內稽部落置。初治燕支城,後僑治營州城中。貞觀元年更名。後治良鄉之石窟堡。縣一:威化。昌州貞觀二年以松漠部落置,僑治營州之靜蕃戍。七年徙於三合鎮,後治安次之故常道城。縣一:龍山。師州貞觀三年以契丹、室韋部落置,僑治營州之廢陽師鎮,後僑治良鄉之東閭城。縣一:陽師。帶州貞觀十年以乙失革部落置。僑治昌平之清水店。縣一:孤竹。歸順州歸化郡本彈汗州,貞觀二十二年以內屬契丹別帥析紇便部置。開元四年更名。縣一:懷柔。沃州載初中析昌州置。萬歲通天元年沒於李盡忠,開元二年復置。後僑治薊之南回城。縣一:濱海。信州萬歲通天元年以乙失活部落置。僑治範陽境。縣一:黃龍。青山州景雲元年析玄州置。僑治範陽之水門村。縣一:青山。



 松漠都督府,貞觀二十二年以內屬契丹窟哥部置,其別帥七部分置峭落等八州。李盡忠叛後廢,開元二年復置。領州八:峭落州以達稽部置。無逢州以獨活部置。羽陵州以芬問部置。白連州以突便部置。



 徒何州以芮奚部置。萬丹州以墜斤部置。疋黎州以伏部置。赤山州以伏部分置。



 ○歸誠州



 靺鞨州三,府三:慎州武德初以涑沫、烏素固部落置。僑治良鄉之故都鄉城。縣一:逢龍。夷賓州乾符中以愁思嶺部落置,僑治良鄉之古廣陽城。縣一:來蘇。黎州載初二年析慎州置。僑治良鄉之故都鄉城。縣一:新黎。黑水州都督府開元十四年置。渤海都督府安靜都督府右初皆隸營州都督,李盡忠陷營州,乃遷玄州於徐、宋之境,威州於幽州之境,昌、師、帶、鮮、信五州於青州之境,崇、慎二州於淄、青之境,夷賓州於徐州之境,黎州於宋州之境,在河南者十州,神龍初乃使北還,二年皆隸幽州都督府。



 降胡州一:



 凜州天寶初置,僑治範陽境。右隸幽州都督府。



 高麗降戶州十四,府九太宗親征,得蓋牟城,置蓋州;得遼東城,置遼州;得白崖城,置巖州。及師還,拔蓋、遼二州之人以歸。高宗滅高麗,置都督府九,州四十二,後所存州止十四。初,顯慶五年平百濟,以其地置熊津、馬韓、東明、金連、德安五都督府,並置帶方州,麟德後廢:南蘇州蓋牟州代那州倉巖州磨米州積利州黎山州延津州木底州安市州諸北州識利州拂涅州拜漢州新城州都督府遼城州都督府哥勿州都督府衛樂州都督府舍利州都督府居素州都督府越喜州都督府去旦州都督府建安州都督府右隸安東都護府。



 ○隴右道



 突厥州三,府二十七:



 皋蘭州貞觀二十二年以阿史德特健部置,初隸燕然都護,後來屬。興昔都督府右隸涼州都督府。



 特伽州雞洛州開元中又有火拔州、葛祿州,後不復見。



 濛池都護府貞觀二十三年,以阿史那賀魯部落置瑤池都督府,永徽四年廢。顯慶二年禽賀魯,分其地,置都護府二、都督府八,其役屬諸胡皆為州。昆陵都護府匐陵都督府以處木昆部置。嗢鹿州都督府以突騎施索葛莫賀部置。潔山都督府以突騎施阿利施部置。雙河都督府以攝舍提暾部置。鷹娑都督府以鼠尼施處半部置。鹽泊州都督府以胡祿屋闕部置。陰山州都督府顯慶三年分葛邏祿三部置三府,以謀落部置。大漠州都督府以葛邏祿熾俟部置。玄池州都督府以葛邏祿踏實部置。金附州都督府析大漠州置。輪臺州都督府金滿州都督府永徽五年以處月部落置為州,隸輪臺。龍朔二年為府。咽面州都督府初,玄池、咽面為州,隸燕然,長安二年為都督府,隸北庭。鹽祿州都督府哥系州都督府孤舒州都督府西鹽州都督府東鹽州都督府叱勒州都督府迦瑟州都督府憑洛州都督府沙陀州都督府答爛州都督府右隸北庭都護府。



 回紇州三,府一:𧾷帶林州以思結別部置。金水州賀蘭州盧山都督府以思結部置。右初隸燕然都護府,總章元年隸涼州都督府。



 黨項州七十三,府一,縣一:



 馬邑州開元十七年置,在秦、成二州山谷間。寶應元年徙於成州之鹽井故城。右隸秦州都督府。



 保塞州右隸臨州都督府。



 密恭縣高宗上元三年為吐蕃所破,因廢,後復置。右隸洮州。



 叢州貞觀三年置。縣三:寧遠,臨泉,臨河。崌州貞觀元年以降戶置。縣二:江源,落稽。奉州本西仁州,貞觀元年置,八年更名。縣三:奉德,恩安,永慈。巖州本西金州,貞觀五年置,八年更名。縣三:金池,甘松,丹巖。遠州本西懷州,貞觀四年置,八年更名。縣二:羅水,小部川。麟州本西麟州,貞觀五年置,八年更名。縣七:硤川,和善,劍具,硤源,三交,利恭,東陵。可州本西義州,貞觀四年置,八年更名。縣三:義誠,清化,靜方。闊州貞觀五年置。縣二:闊源,落吳。彭州本洪州,貞觀三年置,七年更名。縣四:洪川,歸遠,臨津,歸正。直州本西集州,貞觀五年置,八年更名。縣二:集川,新川。肆州貞觀五年置。縣四:歸唐,芳叢,鹽水,磨山。序州貞觀十年置。靜州咸亨三年以內附部落置。軌州都督府貞觀二年以細封步賴部置。縣四:玉城,金原,俄徹,通川。以上有版。



 研州探那州心巳州毘州河州乾州瓊州犀州龕州陪州如州麻州霸州闌州光州至涼州曄州



 思帝州統州穀邛州達違州萬卑州慈州融洮州執州答針州稅河州吳洛州齊帝州苗州始目州悉多州



 質州兆州求易州托州志德州延避州略州索京州



 柘剛州明桑州白豆州瓚州酋和州和昔州祝州索川州拔揭州鼓州飛州索渠州目州寶劍州津州柘鐘州紀州微州以上無版。



 右初隸松州都督府,肅宗時懿、蓋、嵯、諾、嶂、祐、臺、橋、浮、寶、玉、位、儒、歸、恤及西戎、西滄、樂容、歸德等州皆內徙,餘皆沒於吐蕃;



 乾封州歸義州順化州和寧州和義州保善州寧定州羅雲州朝鳳州以上寶應元年內附。永定州永泰元年以永定等十二州部落內附,析置州十五。宜芳州餘闕。右闕。



 吐谷渾州一:合門州右隸涼州都督府。



 四鎮都督府,州三十四。咸亨元年,吐蕃陷安西,因罷四鎮,長壽二年復置。



 龜茲都督府,貞觀二十年平龜茲置。領州九。闕。



 毘沙都督府,本於闐國,貞觀二十二年內附,初置州五,高宗上元二年置府,析州為十。領州十。闕。



 焉耆都督府貞觀十八年滅焉耆置。有碎葉城,調露元年,都護王方翼築,四面十二門,為屈曲隱出伏沒之狀云。



 疏勒都督府,貞觀九年疏勒內附置。領州十五。闕。



 河西內屬諸胡,州十二,府二:



 烏壘州和墨州溫府州蔚頭州遍城州耀建州寅度州豬拔州達滿州蒲順州郢及滿州乞乍州媯塞都督府渠黎都督府



 西域府十六,州七十二龍朔元年,以隴州南由令王名遠為吐火羅道置州縣使,自於闐以西,波斯以東,凡十六國,以其王都為都督府,以其屬部為州縣。凡州八十八,縣百一十,軍、府百二十六:



 月支都督府,以吐火羅葉護阿緩城置。領州二十五:藍氏州以缽勃城置。大夏州以縛叱城置。漢樓州以俱祿犍城置。弗敵州以烏邏氈城置。沙律州以咄城置。媯水州以羯城置。盤越州以忽婆城置。忸密州以烏羅渾城置。伽倍州以摩彥城置。粟特州以阿捺臘城置。缽羅州以蘭城置。雙泉州以悉計蜜悉帝城置。祀惟州以昏磨城置。遲散州以悉蜜言城置。富樓州以乞施巘城置。丁零州以泥射城置。薄知州以析面城置。桃槐州以阿臘城置。大檀州以頰厥伊城具闕達官部落置。伏盧州以播薩城置。身毒州以乞澀職城置。西戎州以突厥施怛駃城置。篾頡州以騎失帝城置。疊仗州以發部落城置。苑湯州以拔特山城置。大汗都督府,以嚈噠部落活路城置。領州十五:附墨州以弩那城置。奄蔡州以胡路城置。依耐州以婆多楞薩達健城置。犁州以少俱部落置。榆令州以烏模言城置。安屋州以遮瑟多城置。罽陵州以數始城置。碣石州以迦沙紛遮城置。波知州以羯勞支城置。烏丹州以烏捺斯城置。諾色州以速利城置。迷蜜州以順問城置。盻頓州以乍城置。宿利州以頌施谷部落置。賀那州以汗曜部落置。條支都督府,以訶達羅支國伏寶瑟顛城置。領州九:細柳州以護聞城置。虞泉州以贊候瑟顛城置。犁蘄州以據瑟部落置。崦嵫州以遏忽部落置。巨雀州以烏離難城置。遺州以遺蘭部落置。西海州以郝薩大城置。鎮西州以活恨部落置。乾陀州以縛狼部落置。天馬都督府,以解蘇國數瞞城置。領州二:洛那州以忽論城置。束離州以達利薄紇城置。高附都督府,以骨咄施沃沙城置。領州二:五翕州以葛邏犍城置。休蜜州以烏斯城置。脩鮮都督府,以罽賓國遏紇城置。領州十:毘舍州以羅漫城置。陰米州以賤那城置。波路州以和藍城置。龍池州以遺恨城置。烏弋州以塞奔你邏斯城置。羅羅州以濫犍城置。檀特州以半制城置。烏利州以勃逸城置。漠州以鶻換城置。懸度州以布路犍城置。寫鳳都督府,以帆延國羅爛城置。領州四:嶰穀州以肩捺城置。泠淪州以俟麟城置。悉萬州以縛時伏城置。鉗敦州以未臘薩旦城置。悅般州都督府,以石汗那國艷城置。領雙靡州。以俱蘭城置。奇沙州都督府,以護時犍國遏蜜城置。領州二:沛隸州以漫山城置。大秦州以睿蜜城置。姑墨州都督府,以怛沒國怛沒城置。領慄弋州。以弩羯城置。旅獒州都督府,以烏拉喝國摩竭城置。昆墟州都督府,以多勒建國低寶那城置。至拔州都督府,以俱蜜國褚瑟城置。鳥飛州都督府,以護蜜多國摸逵城置。領缽和州。以娑勒色訶城置。王庭州都督府,以久越得犍國步師城置。



 波斯都督府,以波斯國疾陵城置。右隸安西都護府。



 ○劍南道



 諸羌州百六十八:



 西雅州貞觀五年置。縣三:新城,三泉,石龍。蛾州貞觀五年置。縣二:常平,那川。拱州顯慶元年以缽南伏浪恐部置。劍州永徽五年以大首領凍就部落置。右隸松州都督府。



 塗州武德元年以臨塗羌內附置,領臨塗、端源、婆覽三縣。貞觀元年州廢,縣亦省。二年析茂州之端源戍復置,縣三:端源,臨塗,悉鄰。炎州本西封州,貞觀五年開生羌置,八年更名。縣三:大封,慕仙,義川。徹州貞觀六年以西羌董洞貴部落置。縣三:文徹,俄耳,文進。向州貞觀五年以生羌置。縣二:貝左,向貳。冉州本西冉州,貞觀六年以徼外斂才羌地置,八年更名,九年第為冉州。縣四:冉山,磨山,玉溪,金水。穹州本西博州,貞觀五年以生羌置,八年更名。縣五:小川,徹當,璧川,當博,恭耳。笮州本西恭州,貞觀七年以白狗羌戶置,八年更名。縣三:遂都,亭勸,比思。蓬魯州永徽二年,特浪生羌董悉奉求、闢惠生羌卜簷莫等種落萬餘戶內附,又析置州三十二。姜州恕州葛州勿州鞮州占州達州浪州邠州斂州補州賴州那州舉州多州爾州射州鐸州平祭州時州箭州婆州浩州質州居州可州宕州歸化州柰州竺州卓州右隸茂州都督府。



 思亮州杜州初漢州孚川州渠川州丘盧州祐州計州龍施州月亂州浪彌州月邊州團州櫃州威川州米羌州右隸巂州都督府。



 當馬州此下二十一州,天寶前置。林波州中川州林燒州鉗矢州會野州當仁州金林州東嘉梁州西嘉梁州東石乳州西石乳州涉邛州汶東州費林州徐渠州強雞州長臂州楊常州羅巖州初隸黎州都督,後來屬。雉州椎梅州此下三十六州,開元後置。三井州束鋒州名配州鉗恭州斜恭州畫重州羅林州籠羊州龍逢州敢川州驚川州檛眉州木燭州當品州嚴城州昌磊州鉗並州作重州檛林州三恭州布嵐州欠馬州羅蓬州論川州讓川州遠南州卑廬州夔龍州耀川州金川州鹽井州涼川州夏梁州甫和州橛查州右隸雅州都督府。



 奉上州此下二十二州,開元前置。輒榮州劇川州合欽州蓬口州博盧州明川州胣皮州蓬矢州大渡州米川州木屬州河東州甫嵐州昌明州歸化州初隸巂州,後來屬。象川州叢夏州和良州和都州附樹州東川州上貴州此下二十八州,開元十七年置。滑川州比川州吉川州甫萼州比地州蒼榮州野川州邛凍州貴林州牒珍州浪彌州郎郭州上欽州時蓬州儼馬州邛川州護邛州腳川州開望州上蓬州比蓬州剝重州久護州瑤劍州明昌州護川州索古州此下三州,大和以前置。諾柞州柏坡州右隸黎州都督府。



 諸蠻州九十二:皆無城邑,椎髻皮服,惟來集於都督府,則衣冠如華人焉。



 南寧州漢夜郎地。武德元年開南中,因故同樂縣置,治味。四年置總管府。五年僑治益州,八年復治味,更名郎州。貞觀元年罷都督。開元五年復故名。天寶末沒於蠻,因廢。唐末復置州於清溪鎮,去黔州二十九日行。縣七:味,同樂,升麻,同起,新豐,隴堤,泉麻。昆州本隋置,隋亂廢。武德元年開南中,復置。土貢:牛黃。縣四:益寧,晉寧,安寧,秦臧。有滇池,在晉寧。其秦臧,則故臧漢地也。梨州本西寧州,武德七年析南寧州二縣置,貞觀八年更名。北接昆州。縣二:梁水,絳。匡州本南雲州,武德七年置,貞觀八年更名。漢永昌郡地。縣二:勃弄,匡川。州本西濮州,武德四年置,貞觀十一年更名。漢越巂郡地,南接姚州。縣四:濮水,青蛉,岐星,銅山。尹州武德四年置,北接州。縣五:馬邑,天池,鹽泉,百泉,湧泉。曾州武德四年置,西接匡州。縣五:曾,三部,神泉,龍亭,長和。鉤州本南龍州,武德七年置,貞觀十一年更名。東北接昆州。縣二:望水,唐封。裒州武德七年置。本弄棟地,南接姚州。縣二:楊彼,樂強。宗州本西宗州,武德七年置,貞觀十一年第名宗州。北接姚州。縣三:宗居,石塔,河西。微州本西利州,武德七年置,貞觀十一年更名。北接縻州。縣二:深利,十部。縻州本西豫州,武德七年置,貞觀三年更名。南接姚州。初為都督府,督縻、望、謻羅三州,後罷都督。縣二:磨豫,七部。望州貞觀末以諸蠻內附,與傍州同置,初隸郎州都督,後來屬。謻羅州盤州本西平州,武德四年置,貞觀八年更名。故興古郡地,其南交州。縣三:附唐,平夷,盤水。麻州貞觀二十二年析郎州置。英州聲州勤州傍州貞觀二十三年,諸蠻末徒莫祗、儉望二種落內附,置傍、望、求、丘、覽五州。求州丘州覽州咸州瀘慈州歸武州嚴州湯望州武德州奏龍州武鎮州本武恆,避穆宗名改。南唐州連州縣六:當為,都寧,邏游,羅龍,加平,清坎。南州析盈州置。縣三:播政,百榮,洪盧。德州析志州置。縣二:羅連,萬巖。為州析扶德州置。縣二:扶,羅僧。洛州析鏡州置。縣四:臨津,賓夷,曾城,蔥藥。移州析悅州置。縣三:移當,臨河,湯陵。悅州縣六:甘泉,青賓,臨川,悅水,夷鄰,胡璠。鏡州縣六:夷郎,賓唐,溪琳,琮連,池臨,野並。筠州縣八:鹽水,筠山,羅餘,臨居,澄瀾,臨昆,唐川,尋源。志州「志」一作「總」。縣四:浮萍,雞惟,夷賓,河西。盈州縣四:盈川,塗賽,播陵,施燕。武昌州縣七:洪武,羅虹,瑯林,夷朗,來賓,羅新,綺婆。扶德州縣三:宋水,扶德,阿陰。播朗州析鞏州置。縣三:播勝,從顏,順化。信州居州炎州馴州縣五:馴祿,天池,方陀,羅藏,播騁。騁州縣二:斛木,羅相。浪川州貞元十三年,節度使韋皋表置。縣五:郎浪,郎違,何度,郎仁,因閤。協州本隋置,隋亂廢。武德元年開南中復置。縣三:東安,西安,胡津。靖州析協州置。縣二:靖川,分協。曲州本恭州,隋置,隋亂廢。武德元年開南中復置,八年更名。故硃提郡,北接協州。縣二:硃提,唐興。硃提,本安上,武德七年更名。播陵州析盈州置。鉗州析開邊縣置。哥靈州滈州縣三:拱平,掃宮,羅谷。切騎州縣四:柳池,奏祿,縻托,通識。品州縣三:八秤,松花,牧。從州縣六:從花,昆池,武安,羅林,梯山,南寧。牛可連州縣三:牛可連,羅名,新戍。碾衛州縣三:麻金,碾衛,涪麻。右隸戎州都督府。



 於州武德四年以古滇王國民多姚姓,因置姚州都督,並置州十三。異州五陵州示由州和往州舍利州範鄧州野共州洪郎州日南州眉鄧州醿備州洛諾州右隸姚州都督府。



 納州都寧郡義鳳二年開山洞置。縣八:羅圍,播羅,施陽,都寧,羅當,羅藍,都,胡茂。先天二年與薩、晏、鞏皆降為羈縻。薩州黃池郡儀鳳二年招生獠置。縣二:黃池,播陵。晏州羅陽郡儀鳳二年招生獠置。縣七:思峨,牛可陰,新賓,扶來,思晏,哆罔,羅陽。鞏州因忠郡儀鳳二年開山洞置。縣五:哆摟,都檀,波婆,比求,播郎。奉州儀鳳二年開山洞置。縣二:牛可里,邏蓬。浙州儀鳳二年開山洞置。縣四:浙源,越賓,洛川,鱗山。順州載初二年置。縣五:曲水,順山,靈巖,來猿,龍池。思峨州天授二年置。縣二:多溪,洛溪。淯州久視元年置。縣四:新定,淯川,固城,居牢。能州大足元年置。縣四:長寧,來銀,菊池,猿山。高州縣三:牛可巴,移甫,徒西。宋州縣四:牛可龍,牛可支,宋水,盧吾。長寧州縣四:婆員,波居,青盧,羅門。定州縣二:支江,扶德。右隸瀘州都督府。



 ○江南道



 諸蠻州五十一:



 牂州武德三年以牂柯首領謝龍羽地置,四年更名柯州,後復故名。初,牂、琰、莊、充、應、矩六州皆為下州,開元中降牂、琰、莊為羈縻,天寶三載又降充、應、矩為羈縻。縣三:建安,賓化,新興。建安,本牂柯,武德二年更名。新興與州同置。琰州貞觀四年置。縣五:武侯,望江,應江,始安,東南。貞觀中又領隆昆、琰川二縣,後省。莊州本南壽州,貞觀三年以南謝蠻首領謝強地置,四年更名,十一年為都督府,景龍二年罷都督。故隋牂柯郡地。南百里有桂嶺關。縣七:石牛,南陽,輕水,多樂,樂安,石城,新安。貞觀中又領清蘭縣,後省。充州武德三年,以牂柯蠻別部置,縣七:平蠻,東停,韶明,牂柯,東陵,辰水,思王。應州貞觀三年以東謝首領謝元深地置,縣五:都尚,婆覽,應江,惣隆,羅恭。矩州武德四年置。明州貞觀中以西趙首領趙磨酋地置。蔿州勞州羲州福州犍州邦州清州峨州蠻州縣一:巴江。歐州「歐」一作「鼓」。濡州琳州縣三:多梅,古陽,多奉。鸞州令州那州暉州都州總州咸亨三年,昆明十四姓率戶二萬內屬分置。敦州咸亨三年析內屬昆明部置,縣六:武寧,溝水,古質,昆川,叢燕,孤雲。殷州咸亨三年析昆明部置,後廢。開元十五年分戎州復置,後又廢。貞元二年,節度使韋皋表復置。故南漢之境也。縣五:殷川,東公,龍原,韋川,賓川。初與敦州皆隸戎州都督,後來屬。候州晃州樊州稜州添州普寧州功州亮州茂龍州延州訓州卿州貞觀十五年置。雙城州整州懸州撫水州縣四:撫水,古勞,多蓬,京水。思源州逸州南平州勛州襲州寶州萬歲通天二年以昆明夷內附置。姜州鴻州縣五:樂鴻,思翁,都部,新庭,臨川。



 右隸黔州都督府



 ○嶺南道



 諸蠻州九十二:



 紆州縣六:東區,吉陵,賓安,南山,都邦,紆質。歸思州思順州縣五:羅遵,履博,都恩,吉南,許水。蕃州縣三:蕃水,都伊,思寮。溫泉州溫泉郡土貢:金。縣二:溫泉,洛富。述昆州土貢:桂心。縣五:夷蒙,夷水,古桂,臨山,都隴。格州右隸桂州都督府。



 椳州縣八:正平,富平,龍源,思恩,饒勉,武招,都象,歌良。歸順州本歸淳,元和初更名。思剛州侯州歸誠州倫州石西州思恩州思同州思明州縣一:顯川。萬形州萬承州上思州談州思瑯州波州員州功饒州萬德州左州思誠州曷州歸樂州青州得州七源州右隸邕州都督府。



 德化州永泰二年以林睹符部落置。縣二:德化,歸義。郎茫州永泰二年以林睹符部落分置。縣二:郎茫,古勇。龍武州大歷中以潘歸國部落置。縣二:龍丘,福宇。歸化州縣四:歸朝,洛都,落回,落巍。郡州土貢:白鑞、孔雀尾。縣二:郡口,樂安。萬泉州縣一:陸水。思農州縣三;武郎,武容,武全。為州縣三:都龍,漢會,武零。西原州縣三:羅和,古林,羅淡。林西州縣二:林西,甘橘。思廓州縣三:都寧,昆陽,羅方。武靈州縣三:文葛,甘郎,蘇物。新安州縣三:歸化,賓陽,安德。金廓州縣三:羅嘉,文龍,祿榮。提上州縣三:長賓,提頭,硃綠。甘棠州縣一:忠誠。武定州縣三:福祿,柔遠,康林。都金州縣四:溫泉,嘉陵,甘陽,都金。諒州縣二:武興,古都。武陸州開成三年,都護馬植表以武陸縣置。平原州開成四年析都金州之平原館置。縣三:龍石,平林,龍當。龍州武定州真州信州思陵州祿州中宗時有單樂縣,後省。南平州西平州門州餘州巋州金鄰州儀鳳元年置。暑州羅伏州儋陵州樊德州金龍州哥富州貞元十二年置。尚思州貞元十二年置。安德州貞元十二年置。右隸安南都護府。



 蜀爨蠻州十八貞元七年領州名逸。



 右隸峰州都督府



 唐置羈縻諸州,皆傍塞外,或寓名於夷落。而四夷之與中國通者甚眾,若將臣之所征討,敕使之所慰賜,宜有以記其所從出。天寶中,玄宗問諸蕃國遠近,鴻臚卿王忠嗣以《西域圖》對,才十數國。其後貞元宰相賈耽考方域道里之數最詳,從邊州入四夷,通譯於鴻臚者,莫不畢紀。其入四夷之路與關戍走集最要者七:一曰營州入安東道,二曰登州海行入高麗渤海道,三曰夏州塞外通大同雲中道,四曰中受降城入回鶻道,五曰安西入西域道,六曰安南通天竺道,七曰廣州通海夷道。其山川聚落,封略遠近,皆概舉其目。州縣有名而前所不錄者,或夷狄所自名云。



 營州西北百里曰松陘嶺,其西奚,其東契丹。距營州北四百里至湟水。營州東百八十里至燕郡城。又經汝羅守捉,渡遼水至安東都護府五百里。府,故漢襄平城也。東南至平壤城八百里;西南至都里海口六百里;西至建安城三百里,故中郭縣也;南至鴨淥江北泊汋城七百里,故安平縣也。自都護府東北經古蓋牟、新城,又經渤海長嶺府,千五百里至渤海王城,城臨忽汗海,其西南三十里有古肅慎城,其北經德理鎮,至南黑水靺鞨千里。



 登州東北海行,過大謝島、龜歆島、末島、烏湖島三百里。北渡烏湖海,至馬石山東之都裏鎮二百里。東傍海壖,過青泥浦、桃花浦、杏花浦、石人汪、橐駝灣、烏骨江八百里。乃南傍海壖,過烏牧島、貝江口、椒島,得新羅西北之長口鎮。又過秦王石橋、麻田島、古寺島、得物島,千里至鴨淥江唐恩浦口。乃東南陸行,七百里至新羅王城。自鴨淥江口舟行百餘里,乃小舫溯流東北三十里至泊汋口,得渤海之境。又溯流五百里,至丸都縣城,故高麗王都。又東北溯流二百里,至神州。又陸行四百里,至顯州,天寶中王所都。又正北如東六百里,至渤海王城。夏州北渡烏水,經賀麟澤、拔利干澤,過沙,次內橫刬、沃野泊、長澤、白城,百二十里至可硃渾水源。又經故陽城澤、橫刬北門、突紇利泊、石子嶺,百餘里至阿頹泉。又經大非苦鹽池,六十六里至賀蘭驛。又經庫也乾泊、彌鵝泊、榆祿渾泊,百餘里至地頹澤。又經步拙泉故城,八十八里渡烏那水,經胡洛鹽池、紇伏乾泉,四十八里度庫結沙,一曰普納沙,二十八里過橫水,五十九里至十賁故城,又十里至寧遠鎮。又涉屯根水,五十里至安樂戍,戍在河西壖,其東壖有古大同城。今大同城,故永濟柵也。北經大泊,十七里至金河。又經故後魏沃野鎮城,傍金河,過古長城,九十二里至吐俱麟川。傍水行,經破落汗山、賀悅泉,百三十一里至步越多山。又東北二十里至纈特泉。又東六十里至賀人山,山西磧口有詰特犍泊。吐俱麟川水西有城,城東南經拔厥那山,二百三十里至帝割達城。又東北至諾真水水義。又東南百八十七里,經古可汗城至咸澤。又東南經烏咄谷,二百七里至古雲中城。又西五十五里有綏遠城。皆靈、夏以北蕃落所居。



 中受降城正北如東八十里,有呼延谷,谷南口有呼延柵,谷北口有歸唐柵,車道也,入回鶻使所經。又五百里至鷿鵜泉,又十里入磧,經麚鹿山、鹿耳山、錯甲山,八百里至山燕子井。又西北經密粟山、達旦泊、野馬泊、可汗泉、橫嶺、綿泉、鏡泊,七百里至回鶻衙帳。又別道自鷿鵜泉北經公主城、眉間城、怛羅思山、赤崖、鹽泊、渾義河、爐門山、木燭嶺,千五百里亦至回鶻衙帳。東有平野,西據烏德鞬山,南依嗢昆水,北六七百里至仙娥河,河北岸有富貴城。又正北如東過雪山松樺林及諸泉泊,千五百里至骨利幹,又西十三日行至都播部落,又北六七日至堅昆部落,有牢山、劍水。又自衙帳東北渡仙娥河,二千里至室韋。骨利乾之東,室韋之西有鞠部落,亦曰祴部落。其東十五日行有俞折國,亦室韋部落。又正北十日行有大漢國,又北有骨師國。骨利幹、都播二部落北有小海,冰堅時馬行八日可度。海北多大山,其民狀貌甚偉,風俗類骨利幹,晝長而夕短。回鶻有延侄伽水,一曰延特勒泊,曰延特勒郍海。烏德鞬山左右嗢昆河、獨邏河皆屈曲東北流,至衙帳東北五百里合流。泊東北千餘里有俱倫泊,泊之四面皆室韋。



 安西西出柘厥關,渡白馬河,百八十里西入俱毘羅磧。經苦井,百二十里至俱毘羅城。又六十里至阿悉言城。又六十里至撥換城,一曰威戎城,曰姑墨州,南臨思渾河。乃西北渡撥換河、中河,距思渾河百二十里,至小石城。又二十里至於闐境之胡蘆河。又六十里至大石城,一曰於祝,曰溫肅州。又西北三十里至粟樓烽。又四十里度拔達嶺。又五十里至頓多城,烏孫所治赤山城也。又三十里渡真珠河,又西北渡乏驛嶺,五十里渡雪海,又三十里至碎卜戍,傍碎卜水五十里至熱海。又四十里至凍城,又百一十里至賀獵城,又三十里至葉支城,出穀至碎葉川口,八十里至裴羅將軍城。又西二十里至碎葉城,城北有碎葉水,水北四十里有羯丹山,十姓可汗每立君長於此。自碎葉西十里至米國城,又三十里至新城,又六十里至頓建城,又五十里至阿史不來城,又七十里至俱蘭城,又十里至稅建城,又五十里至怛羅斯城。自撥換、碎葉西南渡渾河,百八十里有濟濁館,故和平鋪也。又經故達干城,百二十里至謁者館。又六十里至據史德城,龜茲境也,一曰鬱頭州,在赤河北岸孤石山。渡赤河,經岐山,三百四十里至葭蘆館。又經達漫城,百四十里至疏勒鎮,南北西三面皆有山,城在水中。城東又有漢城,亦在灘上。赤河來自疏勒西葛羅嶺,至城西分流,合於城東北,入據史德界。自撥換南而東,經昆崗,渡赤河,又西南經神山、睢陽、咸泊,又南經疏樹,九百三十里至於闐鎮城。於闐西五十里有葦關,又西經勃野,西北渡系館河,六百二十里至郅支滿城,一曰磧南州。又西北經苦井、黃渠,三百二十里至雙渠,故羯飯館也。又西北經半城,百六十里至演渡州,又北八十里至疏勒鎮。自疏勒西南入劍末穀、青山嶺、青嶺、不忍嶺,六百里至蔥嶺守捉,故羯盤陀國,開元中置守捉,安西極邊之戍。有寧彌故城,一曰達德力城,曰汗彌國,曰拘彌城。於闐東三百九十里,有建德力河,東七百里有精絕國。於闐西南三百八十里,有皮山城,北與姑墨接。凍凌山在於闐國西南七百里。又於闐東三百里有坎城鎮,東六百里有蘭城鎮,南六百里有胡弩鎮,西二百里有固城鎮,西三百九十里有吉良鎮。於闐東距且末鎮千六百里。自焉耆西五十里過鐵門關,又二十里至於術守捉城,又二百里至榆林守捉,又五十里至龍泉守捉,又六十里至東夷僻守捉,又七十里至西夷僻守捉,又六十里至赤岸守捉,又百二十里至安西都護府。又一路自沙州壽昌縣西十里至陽關故城,又西至蒲昌海南岸千里。自蒲昌海南岸,西經七屯城,漢伊脩城也。又西八十里至石城鎮,漢樓蘭國也,亦名鄯善,在蒲昌海南三百里,康艷典為鎮使以通西域者。又西二百里至新城,亦謂之弩支城,艷典所築。又西經特勒井,渡且末河,五百里至播仙鎮,故且末城也,高宗上元中更名。又西經悉利支井、祆井、勿遮水,五百里至於闐東蘭城守捉。又西經移杜堡、彭懷堡、坎城守捉,三百里至於闐。



 安南經交趾太平,百餘里至峰州。又經南田,百三十里至恩樓縣,乃水行四十里至忠城州。又二百里至多利州,又三百里至硃貴州,又四百里至丹棠州,皆生獠也。又四百五十里至古湧步,水路距安南凡千五百五十里。又百八十里經浮動山、天井山,山上夾道皆天井,間不容跬者三十里。二日行,至湯泉州。又五十里至祿索州,又十五里至龍武州,皆爨蠻安南境也。又八十三里至儻遲頓,又經八平城,八十里至洞澡水,又經南亭,百六十里至曲江,劍南地也。又經通海鎮,百六十里渡海河、利水至絳縣。又八十里至晉寧驛,戎州地也。又八十里至柘東城,又八十里至安寧故城,又四百八十里至雲南城,又八十里至白崖城,又七十里至蒙舍城,又八十里至龍尾城,又十里至大和城,又二十五里至羊苴咩城。自羊苴咩城西至永昌故郡三百里。又西渡怒江,至諸葛亮城二百里。又南至樂城二百里。又入驃國境,經萬公等八部落,至悉利城七百里。又經突旻城至驃國千里。又自驃國西度黑山,至東天竺迦摩波國千六百里。又西北渡迦羅都河至奔那伐檀那國六百里。又西南至中天竺國東境恆河南岸羯硃嗢羅國四百里。又西至摩羯陀國六百里。一路自諸葛亮城西去騰充城二百里。又西至彌城百里。又西過山,二百里至麗水城。乃西渡麗水、龍泉水,二百里至安西城。乃西渡彌諾江水,千里至大秦婆羅門國。又西渡大嶺,三百里至東天竺北界個沒盧國。又西南千二百里,至中天竺國東北境之奔那伐檀那國,與驃國往婆羅門路合。一路自驩州東二日行,至唐林州安遠縣,南行經古羅江,二日行至環王國之檀洞江。又四日至硃崖,又經單補鎮,二日至環王國城,故漢日南郡地也。自驩州西南三日行,度霧溫嶺,又二日行至棠州日落縣,又經羅倫江及古朗洞之石蜜山,三日行至棠州文陽縣。又經漦誑澗,四日行至文單國之算臺縣,又三日行至文單外城,又一日行至內城,一曰陸真臘,其南水真臘。又南至小海,其南羅越國,又南至大海。



 廣州東南海行,二百里至屯門山,乃帆風西行,二日至九州石。又南二日至象石。又西南三日行,至占不勞山,山在環王國東二百里海中。又南二日行至陵山。又一日行,至門毒國。又一日行,至古笪國。又半日行,至奔陀浪洲。又兩日行,到軍突弄山。又五日行至海硤,蕃人謂之「質」,南北百里,北岸則羅越國,南岸則佛逝國。佛逝國東水行四五日,至訶陵國,南中洲之最大者。又西出硤,三日至葛葛僧祗國,在佛逝西北隅之別島,國人多鈔暴,乘舶者畏憚之。其北岸則個羅國。個羅西則哥穀羅國。又從葛葛僧只四五日行,至勝鄧洲。又西五日行,至婆露國。又六日行,至婆國伽藍洲。又北四日行,至師子國,其北海岸距南天竺大岸百里。又西四日行,經沒來國,南天竺之最南境。又西北經十餘小國,至婆羅門西境。又西北二日行,至拔狖國。又十日行,經天竺西境小國五,至提狖國,其國有彌蘭太河,一曰新頭河,自北渤昆國來,西流至提狖國北,入於海。又自提狖國西二十日行,經小國二十餘,至提羅盧和國,一曰羅和異國,國人於海中立華表,夜則置炬其上,使舶人夜行不迷。又西一日行,至烏剌國,乃大食國之弗利剌河,南入於海。小舟溯流二日至末羅國,大食重鎮也。又西北陸行千里,至茂門王所都縛達城。自婆羅門南境,從沒來國至烏剌國,皆緣海東岸行;其西岸之西,皆大食國,其西最南謂之三蘭國。自三蘭國正北二十日行,經小國十餘,至設國。又十日行,經小國六七,至薩伊瞿和竭國,當海西岸。又西六七日行,經小國六七,至沒巽國。又西北十日行,經小國十餘,至拔離謌磨難國。又一日行,至烏剌國,與東岸路合。西域有陀拔思單國,在疏勒西南二萬五千里,東距勃達國,西至涅滿國,皆一月行,南至羅剎支國半月行,北至海兩月行。羅剎支國東至都槃國半月行,西至沙蘭國,南至大食國皆二十日行。都槃國東至大食國半月行,南至大食國二十五日行,北至勃達國一月行。勃達國東至大食國兩月行,西北至岐蘭國二十日行,北至大食國一月行。河沒國東南至陀拔國半月行,西北至岐蘭國二十日行,南至沙蘭國一月行,北至海兩月行。岐蘭國西至大食國兩月行,南至涅滿國二十日行,北至海五日行。涅滿國西至大食國兩月行,南至大食國一月行,北至岐蘭國二十日行。沙蘭國南至大食國二十五日行,北至涅滿國二十五日行。石國東至拔汗那國百里,西南至東米國五百里。罽賓國在疏勒西南四千里,東至俱蘭城國七百里,西至大食國千里,南至婆羅門國五百里,北至吐火羅國二百里。東米國在安國西北二千里,東至碎葉國五千里,西南至石國千五百里,南至拔汗那國千五百里。史國在疏勒西二千里,東至俱蜜國千里,西至大食國二千里,南至吐火羅國二百里,西北至康國七百里。



\end{pinyinscope}