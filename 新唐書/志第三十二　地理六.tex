\article{志第三十二 地理六}

\begin{pinyinscope}

 劍南道,蓋古梁州之域,漢蜀郡、廣漢、犍為、越巂、益州、牂柯、巴郡之地,總為鶉首分。為府一,都護府一界中,與「太一」融合為一。他的思想對基督教教父哲學,尤,州三十八,縣百八十九。其名山:岷、峨、青城、鶴鳴。其大川:江、涪、雒、西漢。厥賦:絹、綿、葛、紵。厥貢:金、布、絲、葛、羅、綾、綿紬、羚角、犛尾。



 成都府蜀郡,赤。至德二載曰南京,為府,上元元年罷京。土貢:錦、單絲羅、高杼布、麻、蔗糖、梅煎、生春酒。戶十六萬九百五十,口九十二萬八千一百九十九。縣十:有府三,曰威遠、歸德、二江。有天征軍,本天威,乾元二年置,元和三年更名。成都,次赤。有江瀆祠。北十八里有萬歲池,天寶中,長史章仇兼瓊築堤,積水溉田。南百步有官源渠堤百餘里,天寶二載,令獨孤戒盈築。華陽,次赤。本蜀,貞觀十七年析成都置,乾元元年更名。新都,次畿。武德二年置。有繁陽山。犀浦,次畿。垂拱二年析成都置。新繁,次畿。雙流,次畿。廣都,次畿。龍朔二年析雙流置。郫,次畿。溫江,次畿。本萬春,武德三年置,貞觀元年更名。有新源水,開元二十三年,長史章仇兼瓊因蜀王秀故渠開,通漕西山竹木。靈池。次畿。本東陽,久視元年置,天寶元年更名。



 彭州濛陽郡,緊。垂拱二年析益州置。土貢:段羅、交梭。戶五萬五千九百二十二,口三十五萬七千三百八十七。縣四:有府二,曰天水、唐興。有威戎軍,有羊灌田、朋笮、繩橋三守捉城,有七盤、安遠、龍溪三城,有當風戍,有靜塞關。九隴,望。武德三年以九隴、綿竹、導江置濛州。貞觀二年州廢,縣皆來屬。武后時,長史劉易從決唐昌沲江,鑿川派流,合堋口垠歧水溉九隴、唐昌田,民為立祠。有葛璝山、漓沅山、陽平山。導江,望。本盤龍,武德元年以故汶山置,尋更名。貞觀中曰灌寧,開元中復為導江。有侍郎堰,其東百丈堰,引江水以溉彭、益田;龍朔中築;又有小堰,長安初築。西有蠶崖關;有岷山、玉壘山。有鎮靜軍,開元中置。有白沙守捉城。有木瓜戍、三奇戍。唐昌,望。儀鳳二年析九隴、導江、郫置。長壽二年曰周昌,神龍元年復故名。濛陽。緊。儀鳳二年析九隴、什邡、雒置。



 蜀州唐安郡,緊。垂拱二年析益州置。土貢:錦、單絲羅、花紗、紅藍、馬策。戶五萬六千五百七十七,口三十九萬六百九十四。縣四:有府三,曰金堰、廣逢、灌口。有鎮靜軍,乾符二年,節度使高駢置。晉原,望。有天倉山。青城,望。「青」故作「清」,開元十八年更。有青城山。唐安。望。本唐隆,武德元年置。長壽二年曰武隆,神龍元年復為唐隆,先天元年更名。新津。望。西南二里有遠濟堰,分四筒穿渠,溉眉州通義、彭山之田,開元二十八年,採訪使章仇兼瓊開。有稠梗山、本竹山、天社山、主簿山,有鐵。



 漢州德陽郡,上。垂拱二年析益州置。土貢:交梭,雙紃,彌牟、紵布,衫段,綾,紅藍,蜀馬。戶六萬九千五,口三十萬八千二百三。縣五:有府一,曰玉津。有威勝軍。雒,望。貞元末,刺史盧士珵立堤堰,溉田四百餘頃。德陽,緊。武德三年析雒置。有鹿頭關。什邡,望。武德二年析雒置。有李冰祠山。綿竹,緊。有庚除山、萬安山、鹿堂山。金堂。上。咸亨二年析雒、新都置。有昌利山。



 嘉州犍為郡,中。本眉山郡,天寶元年更名。土貢:麩金、紫葛、麝香。戶三萬四千二百八十九,口九萬九千五百九十一。縣八:有犍為、沐源、寺莊、牛徑、銅山、曲灘、陀和、平戎、依名、利雲、溶川、羅護、柘林、大池、雞心、龍溪、賴泥、可陽、婆籠、馬鞍、始犁、峨眉等二十二鎮兵。龍游,緊。平羌,中下。有鐵,有關。峨眉,上。有金,有鐵。夾江,上。有鐵。玉津,中。綏山,中。久視元年析置樂都縣,尋省。有綏山。羅目,中。麟德二年開生獠置,以縣置沐州。高宗上元三年州廢,縣亦省,儀鳳三年復置,來屬。有峨眉山。犍為。中。本隸戎州,高宗上元元年來屬。



 眉州通義郡,上。武德二年析嘉州置。土貢:麩金、柑、石蜜、葛粉。戶四萬三千五百二十九,口十七萬五千二百五十六。縣五:通義,緊。彭山,緊。本隆山,隸陵州。貞觀元年省入通義,二年復置,來屬。先天元年更名。有通濟大堰一,小堰十,自新津中江口引渠南下,百二十里至州西南入江,溉田千六百頃,開元中,益州長史章仇兼瓊開。有鹽,有彭女山。丹稜,上。有龍鵠山。洪雅,上。武德元年以縣置犍州,五年省南安入焉。貞觀元年州廢,來屬。開元七年置義州,並以獠戶置南安、平鄉二縣。八年州廢,省二縣,以洪雅來屬。青神。上。大和中,榮夷人張武等百餘家請田於青神,鑿山釃渠,溉田二百餘頃。



 邛州臨邛郡。上。武德元年析雅州置,顯慶二年徙治臨邛。土貢:葛、絲布、酒杓。戶四萬二千一百七,口十九萬三百二十七。縣七:有府一,曰興化。有鎮南軍,寶應元年置。臨邛,緊。有銅,有鐵。依政,上。安仁,上。武德三年析臨邛、依政置。貞觀十七年省,咸亨元年復置。大邑,上。咸亨二年析益州之晉原置。有鶴鳴山。蒲江,中下。有鹽。大和四年以蒲江、臨溪隸巂州,後皆復來屬。臨溪,中下。有鐵。火井。中下。有鎮兵,有鹽。



 簡州陽安郡,下。武德三年析益州置。土貢:麩金、葛、綿紬、柑。戶二萬三千六十六,口十四萬三千一百九。縣三:陽安,上。有銅,有鹽;有柏廟山、玉女靈山。金水,上。本金淵,武德元年更名。有銅。平泉。中。



 資州資陽郡,上。本治盤石,咸通六年徙治內江,七年復治盤石。土貢:麩金、柑。戶二萬九千六百三十五,口十萬四千七百七十五。縣八:有安定軍。盤石,中。有平岡山、崇靈山;有鹽;北七十里有百枝池,周六十里,貞觀六年,將軍薛萬徹決東使流。資陽。上。有鹽。清溪,下。本牛鞞,天寶元年更名。內江,中。有鹽。月山,下。義寧二年置。龍水,中。義寧二年置。有鹽。銀山,下。義寧二年置。丹山。中。貞觀四年置,六年省入內江,七年復置。



 巂州越巂郡,中都督府。本治越巂,至德二載沒吐蕃,貞元十三年收復。大和五年為蠻寇所破,六年徙治臺登。土貢:蜀馬、絲布、花布、麩金、麝香、刀靶。戶四萬七百二十一,口十七萬五千二百八十。縣九:有清溪關,大和中,節度使李德裕徙於中城;西南有昆明軍,其西有寧遠軍,有新安、三阜、沙野、蘇祁、保塞、羅山、西瀘、蛇勇、遏戎九城。自清溪關南經大定城百一十里至達仕城,西南經菁口百二十里至永安城,城當滇、笮要沖;又南經水口西南度木瓜嶺二百二十里至臺登城;又九十里至蘇祁縣,又南八十里至巂州,又經沙野二百六十里至羌浪驛,又經陽蓬嶺百餘里至俄準添館;陽蓬嶺北巂州境,其南南詔境。又經菁口、會川四百三十里至河子鎮城,又三十里渡瀘水,又五百四十里至姚州,又南九十里至外沴蕩館;又百里至佉龍驛,與戎州往羊苴咩城路合。貞元十四年,內侍劉希昂使南詔由此。臺登,中。武德元年隸登州,貞觀二年來屬。有九子山。越巂,中。邛部,中。蘇祁,中。西瀘,中。本可,天寶元年更名。昆明,中。武德二年置。有鹽,有鐵。和集,中。貞觀八年置。昌明,中。貞觀二十二年開松外蠻,置牢州及松外、尋聲、林開三縣,永徽三年州廢,省三縣入昌明。會川。中。本邛都,高宗上元二年徙於會川,因更名。有瀘津關。



 雅州盧山郡,下都督府。本臨邛郡,天寶元年更名。土貢:麩金、茶、石菖蒲、落雁木。戶萬八百九十二,口五萬四千一十九。縣五:有和川、始陽、靈關、安國四鎮兵,又有晏山、邊臨、統塞、集重、伐謀、制勝、龍游、尼陽八城。嚴道,中。唐初,以州境析置濛陽、長松、靈關、陽啟、嘉良、火利六縣,武德六年皆省。盧山,中。儀鳳二年置大渡縣,長安二年省。有靈關;有鹽,有銅。名山,中下。有雞棟關。百丈,中。貞觀八年置。榮經。中下。武德三年置。有邛崍山,有關。有銅。有金湯軍,乾符二年置;並置靜寇軍,故延貢地也。



 黎州洪源郡,下都督府。大足元年以雅州之漢源、飛越,巂州之陽山置。神龍三年州廢,縣還故屬。開元四年復置。土貢:升麻、椒、麝香、牛黃。戶千七百三十一,口七千六百七十。縣三:有洪源軍。有定蕃、飛越、和孤三鎮兵,又有武侯、廓清、銅山、肅寧、大定、要沖、潘倉、三碉、杖義、琉璃、和孤十一城。漢源,中。武德元年以漢源、陽山二縣置登州,九年州廢,二縣來屬。貞觀三年隸巂州,永徽三年復故。飛越,中。儀鳳二年析漢源置,並置大渡縣,隸雅州,長安二年省。神龍中隸雅州,開元三年還屬。通望。中下。本陽山,隸登州,武德元年析臺登置。州廢,隸雅州,貞觀二年來屬。天寶元年更名。



 茂州通化郡,下都督府。本汶山郡,武德元年曰會州,四年曰南會州,貞觀八年更州名,天寶元年更郡名。土貢:麩金、丹砂、麝香、狐尾、羌活、當歸、乾酪。戶二千五百一十,口萬五千二百四十二。縣四:有威戎軍。汶山,中。有龍泉山、岷山。汶川,中下。有古桃關。石泉,中下。貞觀八年置,永徽二年省北川縣入焉。有石紐山。通化。中下。



 翼州臨翼郡,下。武德元年析會州之左封、翼針置。咸亨三年僑治悉州之悉唐,上元二年還治翼針。土貢:犛牛尾、麝香、白蜜。戶七百一十一,口三千六百一十八。縣三:有峨和、白岸、都護、祚鼎四城,有合江、穀塠、三穀三守捉城;有隴東、益登、清溪、御籓、吉超五鎮兵。衛山,中下。本翼針,天寶元年更名。翼水,下。峨和。下。



 維州維川郡,下。武德七年以白狗羌戶於姜維故城置,並置金川、定廉二縣。貞觀元年以羌叛,州廢,縣亦省,二年復置。麟德二年自羈縻州為正州,儀鳳二年以羌叛,復降為羈縻州,垂拱三年復為正州。廣德元年沒吐蕃,大和五年收復,尋棄其地。大中三年首領以州內附。土貢:麝香、犛牛尾、羌活、當歸。戶二千一百四十二,口三千一百九十八。縣三:有通化軍,有乾溪、白望、暗桶、赤鼓溪、石梯、達節、鵶口、質臺、駱它九守捉城;西山南路有通耳、瓜平、乾溪、侏儒、箭上、穀口六守捉城,又有符堅城;有寧塞、姜維二鎮兵。薛城,中下。貞觀二年置,又析置鹽溪縣,永徽元年省入定廉。有鹽。通化,中下。本小封,咸亨二年以生羌戶於故金川縣地置,後更名。歸化。下。



 戎州南溪郡,中都督府。本犍為郡,治南溪,貞觀中徙治僰道。天寶元年更名。長慶中復治南溪。土貢:葛纖、荔枝煎。戶四千三百五十九,口萬六千三百七十五。縣五:有石門、龍騰、和戎、馬湖、移風、伊祿、義賓、可封、泥溪、開邊、平寇十一鎮兵;有奮戎城,乾符二年置。南溪,中。有平蓋山。僰道,中。義賓,中下。本存阜馬阜,武德二年省,三年復置。天寶元年更名,又省撫夷縣入焉。開邊,中下。貞觀四年以石門、開邊、硃提三縣置南通州,五年析置鹽泉縣以隸之。八年曰賢州,是年州廢,以石門、硃提、鹽泉置撫夷縣及開邊,隸戎州。自縣南七十里至曲州,又四百八十里至石門鎮,隋開皇五年率益、漢二州兵所開;又經鄧枕山、馬鞍渡二百二十五里至阿傍部落,又經蒙夔山百九十里至阿夔部落,又百八十里至諭官川,又經薄季川百五十里至界江山下,又經荊溪谷、水數渘池三百二十里至湯麻頓,又二百五十里至柘東城,又經安寧井三百九十里至曲水,又經石鼓二百二十里渡石門至佉龍驛,又六十里至雲南城,又八十里至白崖城,又八十里至龍尾城,又四十里至羊苴咩城。貞元十年,詔祠部郎中袁滋與內給事劉貞諒使南詔,由此。歸順。中下。聖歷二年析存阜馬阜縣地,以生獠戶置。



 姚州雲南郡,下。武德四年以漢雲南縣地置。土貢:麩金、麝香。戶三千七百。縣三:有澄川、南江二守捉城。自巂州南至西瀘,經陽蓬、鹿谷、菁口、會川四百五十里至瀘州,乃南渡瀘水,經褒州、微州三百五十里至姚州。州西距羊苴咩城三百里,東南距安南水陸二千里。姚城,下。故漢弄棟縣地。瀘南,下。本長城,垂拱元年置,天寶初更名。有蔥山。長明。下。



 松州交川郡,下都督府。武德元年以扶州之嘉誠、會州之交川置,以地產甘松名。廣德元年沒吐蕃,其後松、當、悉、靜、柘、恭、保、真、霸、乾、維、翼等為行州,以部落首領世為刺史、司馬。土貢:蠟、樸硝、麝香、狐尾、當歸、羌活。戶千七十六,口五千七百四十二。縣四:有松當軍,武后時置。嘉誠,下。交川,下。平康,下。本隸當州,垂拱元年析交川及當州之通軌、翼針置。天寶元年隸松州。鹽泉。下。



 當州江源郡,下。貞觀二十一年,以羌首領董和那蓬固守松州功,析松州之勇軌縣置,以地產當歸名。土貢:麩金、酥、麝香、當歸、羌活。戶二千一百四十六,口六千七百一十三。縣三:通軌,中下。貞觀三年置。利和,下。顯慶二年析通軌置。穀和。下。文明元年開生羌置,並置平唐縣,後省。有常舊山。



 悉州歸誠郡,下。顯慶元年以當州之左封置,並置悉唐、識臼二縣,治悉唐。咸亨元年徙治左封,儀鳳二年羌叛,僑治當州,俄徙治左封。土貢:麩金、麝香、犛牛尾、當歸、柑。戶八百一十六,口三千九百一十四。縣二:左封,中。本隸會州,武德元年隸翼州,三年省。貞觀四年復置,二十一年隸當州。歸誠。下。垂拱二年析左封置。



 靜州靜川郡,下。本南和州,儀鳳元年以悉州之悉唐置,天授二年更名。土貢:麝香、犛牛尾、當歸、羌活。戶千五百七十七,口六千六百六十九。縣三:悉唐,中。靜居,中。清道。下。



 柘州蓬山郡,下。顯慶三年開置。土貢:麝香、當歸、羌活。戶四百九十五,口二千二百二十。縣二:柘,下。喬珠。下。



 恭州恭化郡,下。開元二十四年以靜州之廣平置。土貢:麝香、當歸、升麻、羌活。戶千一百八十九,口六千二百二十三。縣三:西南有平戎軍。和集,下。本廣平,天寶元年更名。博恭,下。開元二十四年析廣平置。烈山。下。開元二十四年析廣平置。



 保州天保郡,下。本奉州雲山郡,開元二十八年以維州之定廉置。天寶八年徙治天保軍,更郡名。廣德元年沒吐蕃,乾元元年,嗣歸誠王董嘉俊以郡來歸,更州名。後又更名古州,其後復為保州。土貢:麩金、麝香、犛牛尾。戶千二百四十五,口四千五百三十六。縣四:有天保軍。定廉,下。武德七年置,永徽元年省維州之鹽溪縣入焉。歸順,下。天寶八載析定廉置。雲山,下。天寶八載析定廉置。安居。下。



 真州昭德郡,下。天寶五載析臨翼郡置。土貢:麝香、大黃。戶六百七十六,口三千一百四十七。縣四:真符,中下。天寶五載析雞川、昭德置。雞川,中下。先天元年析翼水縣地開生獠置,本隸悉州,天寶元年隸翼州。昭德,下。本識臼,顯慶元年開生獠置,隸悉州,天寶元年隸翼州。昭遠,中下。



 霸州靜戎郡,下。天寶元年招附生羌置。戶五百七十一,口千八百六十一。縣四:安信,下。牙利,中。保寧,中。歸化。中。



 乾州,下。大歷三年開西山置。縣二:招武,下。寧遠。下。



 梓州梓潼郡,下。本新城郡,天寶元年更名。土貢:紅綾、絲布、柑、蔗糖、橘皮。戶六萬一千八百二十四,口二十四萬六千六百五十二。縣九:郪,望。有鹽。射洪,上。通泉,緊。大歷二年隸遂州,後復來屬。有鹽,有鐵。玄武,上。本隸益州,武德三年來屬。有鹽。鹽亭,上。有鹽。有負戴山。飛烏,上。有鹽。永泰,中。武德四年,析鹽亭及劍州之黃安、閬州之西水置。有鹽。有女徒山。銅山,中。南可象山,西北私鎔山,皆有銅。貞觀二十三年置鑄錢官,調露元年罷,析郪、飛烏置縣。有會軍堂山。涪城。緊。本隸綿州,大歷十三年來屬。有鹽。



 遂州遂寧郡,中都督府。土貢:樗蒲綾、絲布、天門冬。戶三萬五千六百三十二,口十萬七千七百一十六。縣五:有靜戎軍。方義,望,有鹽。長江,中。有鹽。有廣山。蓬溪,中。本唐興,永淳元年析方義置。長壽二年曰武豐,神龍元年復故名。景龍二年析置唐安縣,先天二年省。天寶元年更唐興曰蓬溪。有化鹽池。青石,中。遂寧。中。景龍元年以故廣溪縣地置。



 綿州巴西郡,上。本金山郡,天寶元年更名。土貢:鏤金銀器、麩金、輕容、雙紃、綾、綿、白藕、蔗。有橘官。戶六萬五千六十六,口二十六萬三千三百五十二。縣八:巴西,望。南六里有廣濟陂,引渠溉田百餘頃,垂拱四年,長史樊思孝、令夏侯奭因故渠開;有富樂山;有金,有銀,有鐵,有鹽。昌明,緊。本昌隆,武德三年析置顯武、文義二縣。貞觀元年省文義,神龍元年更顯武曰興聖,先天元年更昌隆曰昌明,開元二年省興聖入焉。尋又析巴西、涪城、萬安地復置興聖,二十七年省,地還故屬。有北芒山;有鹽,有鐵。魏城,上。北五里有洛水堰,貞觀六年引安西水入縣,民甚利之;有鐵,有鹽。羅江,中。本萬安,天寶元年更名。北五里有茫江堰,引射水溉田入城,永徽五年,令白大信置;北十四里有楊村堰,引折腳堰水溉田,貞元二十一年,令韋德築;有白馬關;有鹽。神泉,上。北二十里有折腳堰,引水溉田,貞觀元年開;有鐵。鹽泉,中。武德三年析魏城置。有鐵。龍安,上。本金山,武德三年更名。有松嶺關,開元十八年廢;東南二十三里有雲門堰,決茶川水溉田,貞觀元年築。西昌。中。永淳元年以隋益昌縣地置。有鐵。



 劍州普安郡,上。本始州,先天二年更名。土貢:麩金、絲布、蘇薰席、葛粉。戶二萬三千五百一十,口十萬四百五十。縣八:普安,上。普城,緊。本黃安,唐末更名。永歸,中下。有停船山。梓潼,上。有亮山、神山。陰平,中。西北二里有利人渠,引馬閣水入縣溉田,龍朔三年,令劉鳳儀開,寶應中廢,後復開,景福二年又廢;有浮滄山。臨津,中上。武連,中。劍門。中下。聖歷二年析普安、永歸、陰平置。



 合州巴川郡,中。本涪陵郡,天寶元年更名。土貢:麩金、葛、桃竹箸、雙陸子、書筒、橙、牡丹、藥實。戶六萬六千八百一十四,口七萬七千二百二十。縣六:石鏡,上。有鐵,有銅梁山。新明,中。武德三年析石鏡置。漢初,中。赤水,中。巴川,中。開元二十三年析石鏡、銅梁置。有鐵。銅梁。中。長安三年置。



 龍州應靈郡,中都督府。本平武郡西龍州,義寧二年曰龍門郡,又曰西龍門郡,貞觀元年曰龍門州。初為羈縻,屬茂州,垂拱中為正州。天寶元年曰江油郡,至德二載更郡名,乾元元年更州名。土貢:麩金、酥、羚羊角、葛粉、厚樸、附子天雄、側子、烏頭。戶二千九百九十二,口四千二百二十八。縣二:江油,望。貞觀八年省平武縣入焉。有涪水關。清川。中下。本馬盤,天寶元年更名。



 普州安岳郡,中。武德二年析資州置。土貢:雙紃、葛布、柑、天門冬煎。戶二萬五千六百九十三,口七萬四千六百九十二。縣六:安岳,上。有鹽。安居,中下。大歷二年隸遂州,後復來屬。有鹽。普慈,中。樂至,中。武德三年置。有鹽。普康,中下。本隆康,先天元年更名。有鹽,崇龕。中。本隆龕,武德三年置,先天元年更名。



 渝州南平郡,下。本巴郡,天寶元年更名。土貢:葛、藥實。戶六千九百九十五,口二萬七千六百八十五。縣五:巴,中下。有鹽。江津,中下。萬壽,中下。本萬春,武德三年析江津置,五年更名。南平,中下。貞觀四年析巴縣置南平州,並置南平、清穀、周泉、昆川、和山、白溪、瀛山七縣。八年曰霸州,十三年州廢,省清穀、周泉、昆川、和山、白溪、瀛山,以南平來屬。壁山。中下。至德二載析巴、江津、萬壽置。有鹽。



 陵州仁壽郡,本隆山郡,天寶元年更名。土貢:麩金、鵝溪絹、細葛、續髓、苦藥。戶三萬四千七百二十八,口十萬一百二十八。縣五:仁壽,望。有鹽,有高城山。貴平,中。有鹽。井研,中。有井鑊山。始建,中下。有鐵。籍。上。永徽四年析貴平置。東五里有漢陽堰,武德初引漢水溉田二百頃,後廢,文明元年,令陳充復置,後又廢;有鹽。



 榮州和義郡,中。武德元年析資州置。治公井,六年徙治大牢,永徽二年徙治旭川。土貢:紬、班布、葛、利鐵、柑。戶五千六百三十九,口萬八千二十四。縣六:有威遠軍。旭川,中下。貞觀元年析大牢置。應靈,中下。本大牢,景龍二年省雲州及羅水、雲川、胡連三縣入焉。天寶元年更名。有鹽。公井,中下。武德元年置。有鹽。資官,中下。本隸嘉州,武德六年來屬。有鹽,有鐵。威遠,中下。貞觀元年析置婆日、至如二縣。二年以瀘州之隆越來屬。八年省婆日、至如、隆越入焉。有鹽。和義。中下。本隸瀘州,貞觀八年來屬。



 昌州,下都督府。乾元二年析資、瀘、普、合四州之地置,治昌元。大歷六年州、縣廢,其地各還故屬,十年復置。光啟元年徙治大足。土貢:麩金、麝香。縣四:大足,下。本合州巴川地。靜南,中。昌元,上。永川。下。本渝州壁山縣地。有鐵。



 瀘州瀘川郡,下都督府。土貢:麩金、利鐵、葛布、班布。戶萬六千五百九十四。口六萬五千七百一十一。縣五:瀘川,中。貞觀八年析置涇南縣,後省。富義,中。本富世,武德九年省來鳳縣入焉。貞觀二十三年更名。江安,中。貞觀元年以夷獠戶置思隸、思逢、施陽三縣。八年省施陽,十三年省思隸、思逢入焉。有鹽。合江,中。綿水。中。



 保寧都護府,天寶八載以劍南之索磨川置,領牂柯、吐蕃。



 右劍南採訪使,治益州。



\end{pinyinscope}