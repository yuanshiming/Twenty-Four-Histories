\article{列傳第五十三 蘇尹畢李鄭王許潘倪席齊}

\begin{pinyinscope}

 蘇珦,雍州藍田人。中明經第,調鄠尉。時李義琰為雍州長史,鄠多訟,日至長史府「矜於詁訓,摘其章句,而不能統其大義之所極」。開其後玄,恦裁決明辦,自是無訴者。義琰異之,顧聽事曰:「此公坐也,恨吾齒晚,不及見。」



 垂拱初,為監察御史。武後殺韓、魯諸王,付珦密牒按訊,珦推之無狀。或言珦助韓、魯者,後詰之,挺議無所撓,後不悅曰:「卿,大雅士,此獄不足諉卿。」即詔監軍河西。五遷右司郎中。御史王弘義附來俊臣為酷,世畏疾,莫敢觸其鋒。會督伐材於虢,笞督過程,人多死,珦按奏,弘義坐免。遷給事中,進左肅政臺御史大夫。後營大像白司馬阪,糜用億計,珦上疏切諫,見納。



 中宗將斬韋月將,珦執據時令不可以大戮,忤三思意,改右臺,俄出為岐州刺史。復為右臺大夫。會節愍太子敗,詔株索支黨。時睿宗居籓,為獄辭牽逮,珦密啟保辯,亦會宰相開陳,帝感悟,多所含貸。擢戶部尚書,封河內郡公。以檢校太子詹事致仕。卒,年八十一,贈兗州都督,謚曰文。



 子晉,數歲知為文,作《八卦論》,吏部侍郎房潁叔、秘書少監王紹宗嘆曰:「後來之王粲也。舉進士及大禮科,皆上第。先天中,為中書舍人。玄宗監國,所下制命,多晉及賈曾稿。屢獻讜言,天子嘉允。出為泗州刺史,以珦老,請解職奉養。珦卒,歷戶部侍郎,襲爵,遷吏部。時宋璟兼尚書事,晉與齊浣更典二都選,既糊名校判,而晉獨事賞拔,當時譽之。及裴光庭知尚書,有過官被卻者,就籍以硃點頭而已。晉因榜選院曰「門下點頭者更擬」,光庭以為侮己,出晉汝州刺史。遷魏州,終太子左庶子。



 始,晉與洛人張循之、仲之兄弟善,而二人以學顯。循之上書忤武後,見殺。仲之神龍中謀去武三思,為宋之愻等所發,死,晉厚撫其子漸,為營婚宦。晉卒,漸喪之若諸父云。



 尹思貞,京兆長安人。弱冠以明經第,調隆州參軍事。屬邑豪蒲氏驁肆不法,州檄思貞按之,擿其奸贓萬計,卒論死,部人稱慶,刻石嘆頌。遷明堂令,以善政聞。擢殿中少監,檢校洛州刺史。會契丹孫萬榮亂,朔方震驚,思貞循撫境內,獨無擾。武后璽書褒慰。



 長安中,遷秋官侍郎,忤張昌宗意,出為定州刺史。召授司府少卿。時卿侯知一亦厲威嚴,吏為語曰:「不畏侯卿杖,祗畏尹卿筆。」加銀青光祿大夫。其家坎地,獲古戟十二,俄而門樹戟,時人異焉。



 神龍初,擢大理卿。雍人韋月將告武三思大逆,中宗命斬之,思貞以方發生月,固奏不可,乃決杖,流嶺南。三思諷所司加法殺之,復固爭,御中大夫李承嘉助三思,而以他事劾思貞,不得謁。思貞謂承嘉曰「公為天子執法,乃擅威福,慢憲度,諛附奸臣圖不軌,今將除忠良以自恣邪?」承嘉慚怒,劾思貞,為青州刺史。或問曰:「公敏行,何與承嘉辯?」答曰:「石非能言者,而或有言。承嘉恃權而侮吾,義不辱,亦不知言何從而至。」治州有績,蠶至歲四熟,黜陟使路敬潛至部,嘆曰:「是非善政致祥乎!」表言之。



 睿宗立,召授將作大匠,封天水郡公。僕射竇懷貞護作金仙、玉真觀,廣調夫匠,思貞數有損節。懷貞讓之,答曰:「公,輔臣也,不能宣贊王化,而土木是興,以媚上害下,又聽小人譖以廷辱士,今不可事公矣。」乃拂衣去,闔門待罪。帝知之,特詔令視事。懷貞誅,拜御史大夫,累遷工部尚書。請致仕,許之。開元四年卒,年七十七,贈黃門監,謚曰簡。思貞前後為刺史十三郡,其政皆以清最聞。



 畢構,字隆擇,河南偃師人。六歲能為文。及冠,擢進士第,補金水尉,遷九隴主簿。居親喪,毀棘甚,已除,猶屏處丘園。武后召為左拾遺。神龍初,遷中書舍人。敬暉等表諸武不宜為王,構當讀表,抗聲析句,左右皆曉知。三思疾之,出為潤州刺史,政有惠愛。徙衛、同、陜三州,遷益州府長史。



 景龍末,召為左御史大夫。會平諸韋,治其黨,衣冠多坐,構詳比重輕,皆得其情。時李傑為河南尹,與構皆一時選,世謂「畢李」。封魏縣男。復為益州長史,按察劍南,振弊柅私,號為清嚴。睿宗嘉構脩罝獨行,有古人風,其治術又為諸使最,乃賜璽書、袍帶。再遷吏部尚書,並遙領益州長史,徙廣州都督。



 玄宗立,授河南尹,進戶部尚書。久之,移疾,帝手疏醫方賜之。當時以戶部為兇官,遽改太子詹事,冀其愈。會卒,贈黃門監,謚曰景。



 始,構喪繼母,而二妹襁褓,身鞠養至成人。妹為構服三年。弟栩,以太府主簿留司東都,聞疾馳歸,哀毀如大喪,雖變服未嘗笑,天下稱其友悌。終荊州司馬。



 構子炕,天寶末為廣平太守,拒安祿山,城陷,覆其家。贈戶部尚書。炕生坰,始四歲,與弟增以細弱得不殺,為賞口。河北平,宗人宏以財贖出之。後舉明經,為臨渙尉。徐州節度使張建封高炕節,聞坰篤行,表署幕府,攝符離令。後調王屋尉,以謹廉聞。喜賓客,家未嘗以有無計。及歿,無貲以治喪云。



 李傑,本名務光,相州釜陽人。後魏並州刺史寶之裔孫。少以孝友著。擢明經第,解褐齊州參軍事,遷累天官員外郎。為吏詳敏,有治譽。以採訪使行山南,時戶口逋蕩,細弱下戶為豪力所兼,傑為設科條區處,檢防亡匿,復業者十七八。神龍中,為河東巡察黜陟使,課最諸道。先天中,進陜州刺史、水陸發運使。置使自傑始。改河南尹。



 傑既精聽斷,雖行坐食飲,省治不少廢,繇是府無淹事,人吏愛之。寡婦有告其子不孝者,傑物色非是,謂婦曰:「子法當死,無悔乎?」答曰:「子無狀,寧其悔!」乃命市棺還斂之,使人跡婦出,與一道士語,頃持棺至,傑令捕道士按問,乃與婦私不得逞。傑殺道士,內於棺。河、汴之交舊有梁公埭,廢不治,南方漕弗通,傑調汴、鄭丁男復作之,不費而利。



 入代宋璟為御史大夫。尚衣奉御長孫昕素惡傑,遇於道,內恃玄宗婭婿,與所親楊仙玉共毆辱之。傑訴曰:「敗發膚,痛在身;辱衣冠,恥在國。」帝怒,詔斬昕等朝堂。左散騎常侍馬懷素建言:「陽和月,不可以殊死。」乃敕杖殺之,謝百官,降書慰傑。



 以護作橋陵,封武威縣子。初,傑引侍御史王旭為護陵判官,旭貪贓,傑將繩之,未及發,反為所構,出衢州刺史。遷揚州大都督府長史,復為御史劾免。開元六年卒,帝悼之,特贈戶部尚書。



 鄭惟忠,宋州宋城人。第進士,補井陘尉。天授中,以制舉召見廷中,武后問舉者,何所事為忠,對皆不合旨。惟忠曰:「外揚君之美,內正君之惡。」後曰:「善。」擢左司御胄曹參軍事,遷水部員外郎。後還長安,復以待制召。後曰:「非嘗於東都對忠臣者乎?朕今不忘。」遷鳳閣舍人。



 中宗立,擢黃門侍郎。時議禁嶺南酋戶不得畜兵,惟忠曰:「善為政者因其俗。且吳人所謂家鶴膝、戶犀渠,此民風也,禁之得無擾乎?」遂止。進大理卿。節愍太子敗,守衛詿誤皆流,已決,諸韋黨請悉誅之,帝欲改推。惟忠奏:「大獄始判,復改訊,恐反側者不自安,且失信天下。」有詔百司參議,卒論如前,所全貸為多。俄授御史大夫,持節賑給河北道,且許黜陟守宰。還奏稱旨,封滎陽縣男,遷太子賓客。卒,贈太子少保。



 王志愔,博州聊城人。擢進士第。中宗神龍中,為左臺侍御史,以剛鷙為治,所居人吏畏讋,呼為「皁雕」。遷大理正,嘗奏言:「法令者,人之堤防,不立則無所制。今大理多不奉法,以縱罪為仁,持文為苛,臣執刑典,恐且得謗。」遂上所著《應正論》以見志,因規帝失。大抵以《易萃》之六二曰「引吉無咎」,謂處萃之時,己獨居正,異操而聚,獨正者危,未能以遠害。惟九五應之,乃履正迎吉,由己居下位而中正是托,期於上應之,不括囊以守祿也。又言:「刑賞二柄,惟人主操之。故曰:『以力役法者,百姓也;以死守法者,有司也;以道變法者,君上也。』魏游肇為廷尉,帝私敕肇有所降恕,肇執不從,曰:『陛下自能恕之,豈可令臣曲筆也。』」又言:「為國當以嚴致平,非以寬致平。嚴者,非凝網重罰,在人不易犯而防難越也。故舍銜策於奔𧾷是,則王良不能御駻;停藥石於膚腠,則俞附不能攻疾。」又言:「漢武帝甥昭平君殺人,以公主子,廷尉上請,帝垂涕曰:『法令者,先帝之所造也,用親故誣先帝法,吾何面目入高廟乎?』卒可其奏。隋文帝子秦王俊為並州總管,以奢縱免官。楊素曰:『王,陛下愛子,請赦之。』帝曰:『法不可違,若如公意,我乃五兒之父,非兆人之父,何不別制天子子律乎?』故天子操法有不變之義。」凡數千言,帝嘉之。



 景雲初,以左御史中丞遷大理少卿。時詔用漢故事,設刺史監郡,於天下劇州置都督,選素威重者授之。遂拜志愔齊州都督,事中格,復授齊州刺史、河南道按察使。徙汴州,封北海縣男。太極元年,兼御史中丞內供奉,實封百戶。出為魏州刺史,改揚州長史。所至破碎奸猾,令行禁信,境內肅然。



 開元九年,帝幸東都,詔留守京師。京兆人權梁山妄稱襄王子,與左右屯營官謀反,自稱光帝,夜犯長樂門,入宮城,將殺志愔,志愔逾垣走,而屯營兵悔,更斬梁山等自歸,志愔慚悸卒。



 許景先,常州義興人。曾祖緒,武德時以佐命功,歷左散騎常侍,封真定公,遂家洛陽。景先由進士第釋褐夏陽尉。神龍初,東都造服慈閣,景先獻賦,李迥秀見其文,畏嘆曰:「是宜付太史!」擢左拾遺,以論事切直,外補滑州司士參軍。舉手筆俊拔、茂才異等連中,進揚州兵曹參軍。還為左補闕。宋璟、蘇頲擇殿中侍御史,久不補,以授景先,時議僉愜。抨按不避近強。與齊浣、王丘、韓休、張九齡更知制誥,以雅厚稱。張說曰:「許舍人之文,雖乏峻峰激流,然詞旨豐美,得中和之氣。」



 開元十年,伊、汝溢,壞廬舍甚眾,景先見侍中源乾曜曰:「災眚所降,王者宜修德應之,因遣大臣存問失職,罪己引咎,以答天譴。公在元弼,庸可默乎?」乾曜悟,遽白玄宗,遣陸象先持節賑贍。



 十三年,帝自擇刺史,景先由吏部侍郎為刺史治虢州,大理卿源光裕鄭州,兵部侍郎寇泚宋州,禮部侍郎鄭溫琦邠州,大理少卿袁仁敬杭州,鴻臚少卿崔志廉襄州,衛尉少卿李升期邢州,太僕少卿鄭放定州,國子司業蔣挺湖州,左衛將軍裴觀滄州,衛率崔誠遂州,凡十一人。治行,詔宰相、諸王、御史以上祖道洛濱,盛具,奏太常樂,帛舫水嬉,命高力士賜詩,帝親書,且給筆紙令自賦,賚絹三千遣之。後徙岐州,入為吏部侍郎,卒。


潘好禮,貝州宗城人。第明經,累遷上蔡令,治在最,擢監察御史。坐小累,下除芮城令,拜侍御史,徙岐王府司馬。居後母喪,詔奪服,固辭不出。開元初,為邠王府長史。王為滑州刺史,好禮兼府司馬、知州事。王御下不能肅,有詔好禮檢督王家,至過失皆上聞。王每游觀,好禮必諫諭禁切。農月,王出獵,家奴羅
 \gezhu{
  辶列}
 ,好禮遮道諫,王初不許,乃臥馬下呼曰:「今農在田,王何得非時暴禾稼,以損下人?要先踐殺司馬,然後聽所為!」王慚,為還。



 遷豫州刺史。勤力於治,清廉無所私,然喜察細事,下厭其苛。子請舉明經,好禮曰:「經不明,不可妄進。」乃自試之,不能通,怒笞之,械而徇於門。復以公累,徙溫州別駕,卒。



 好禮博學,能論議,節行修整,一意無所傾附。未嘗自列階勛,居室服用粗茍至終身,世謂近名。



 倪若水,字子泉,恆州槁城人。擢進士第,累遷右臺監察御史。黜陟劍南道,繩舉嚴允,課第一。開元初,為中書舍人、尚書右丞,出為汴州刺史,政清凈。增修孔子廟,興州縣學廬,勸生徒,身為教誨,風化興行。



 玄宗遣中人捕帟鶄、溪沴南方,若水上言:「農方田,婦方蠶,以此時捕奇禽怪羽為園御之玩,自江、嶺而南,達京師,水舟陸齎,所飼魚蟲、稻粱,道路之言,不以賤人貴鳥望陛下邪?」帝手詔褒答,悉放所玩,謫使人過取罪,而賜若水帛四十段。



 時天下久平,朝廷尊榮,人皆重內任,雖自冗官擢方面,皆自謂下遷。班景倩自揚州採訪使入為大理少卿,過州,若水餞於郊,顧左右曰:「班公是行若登仙,吾恨不得為騶僕。」未幾,入為戶部侍郎,復拜右丞,卒。



 席豫,字建侯,襄州襄陽人。後周昌州刺史固七世孫,後徙河南。長安中,舉學兼流略、詞擅文場科,擢上第,時年十六,以父喪罷。復舉手筆俊拔科,中之。補襄邑尉,奏事闕下,會節愍太子難,安樂公主請為皇太女,豫曰:「昔梅福上書譏後族,彼何人哉!」乃上疏請立皇太子,語深切,人為寒懼。太平公主聞其名,將表為諫官,豫恥污詖謁,遁去。俄舉賢良方正異等,為陽翟尉。



 開元初,觀察使薦豫賢,遷監察御史,出為樂壽令。前令以親喪解,而豫母病,訴諸朝,改懷州司倉參軍。復舉超拔群類科。會母喪去。服除,授大理丞,遷考功員外郎,進絀清明。為中書舍人,與韓休、許景先、徐安貞、孫逖名相甲乙。出鄭州刺史。韓休輔政,舉代己,入拜吏部侍郎。玄宗曰:「卿前日考功職詳事允,故有今授。」豫典選六年,拔寒遠士多至臺閣,當時推知人,號席公云。天寶六載,進禮部尚書,累封襄陽縣子。凡四以使者按行江南、江東、淮南、河北。南方俗死不葬,暴骨中野,豫教以埋斂,明列科防,俗為之改。



 豫清直亡欲,當官不為勢權所撼。性謹畏,與子弟、屬吏書,不作草字。或曰:「此細事耳,何留慮?」答曰:「細不謹,況大事邪?」及疾篤,遺令:「三日斂,斂已即葬,勿久留以黷公私;貲不足,可賣居宅以終事。」卒,年六十九,贈江陵大都督,謚曰文。



 帝嘗登朝元閣賦詩,群臣屬和,帝以豫詩最工,詔曰:「詩人之冠冕也。」



 弟晉,亦以文名當時。



 齊澣,字洗心,定州義豐人。少開敏,年十四,見特進李嶠,嶠稱有王佐才。



 中宗在廬陵,澣上言請抑諸武,迎太子東宮,不報。及太子還,武后召水幹宴同明殿,諭曰:「朕母子如初,卿豫有力焉,方不次待爾。」澣辭母老不忍遠離,賞而罷。聖歷初,及進士第,以拔萃調蒲州司法參軍。有父子連坐論死者,澣曰:「條落則本枯,奈何俱死?」議貸其父,太守不聽,固爭,卒原。景雲初,姚崇取為監察御史。凡劾奏,常先風教,號善職。睿宗將祠太廟,刑部尚書裴談攝太尉,先告。澣奏:「孝享攝事,稽首而拜,恭明神也,而談慢媟不恭。」並劾談「神昏形滓,挾邪以罔上。神龍時,事武三思,陷敬暉,沒其家以獲進。妻外淫,男女不得姓氏。夫告神慢,事主不忠,家不治,有是三罪,不可不置之法。」談由是下除汾州刺史。



 開元初,姚崇復相,用為給事中、中書舍人。論駁及誥詔皆援準古誼,朝廷大政必咨之,時號「解事舍人」。數諷崇年老宜避位。時宋璟在廣州,因勸崇舉自代,崇用其謀。璟為相,它日問曰:「吾不敢冀房、杜,比爾日諸公云何?」澣曰:「不如。」璟請故,答曰:「前時近郊戶三百以為困,今不百戶,是以知之。」馬懷素等緒次四庫書,表澣為副,改秘書少監。



 出為汴州刺史,地當舟車湊集,事浩繁,前刺史數不稱職,唯倪若水與澣以清毅聞,吏民頌美。玄宗封太山,歷汴、宋、許,車騎數萬,王公妃主四夷君長馬、橐駝亦數萬,所頓彌數十里。澣列長棚,帟幕聯亙,上食凡千輿,納筦鑰,身進膳,帝以為知禮,喜甚,為留三日,賜帛二千匹。澣以淮至徐城險急,鑿渠十八里,入青水,人便其漕。



 中書令張說擇丞轄,以王丘為左,澣為右。李元紘、杜暹當國,表宋璟為吏部尚書,澣及蘇晉為侍郎,世謂臺選。嘗奏事,帝指政事堂曰:「非卿尚誰居者。」



 是時,開府王毛仲寵甚,與龍武將軍葛福順相婚嫁,毛仲請奏無不從。澣乘間曰:「福順典兵馬,與毛仲為婚家,小人寵極則奸生,不預圖,且有後患。高力士小心謹畏,加宦人可備禁中驅使,腹心所委,何必毛仲哉?」又言:「君不密失臣,臣不密失身,惟陛下密此言。」帝嘉納,且勞曰:「卿第出,我徐計其宜。」會大理丞麻察坐事,出為興州別駕,澣往餞,因道諫語。察素奸佻,遽言狀。帝怒,召澣入殿中曰:「卿向疑朕不密,而反告察,謂何?且察輕躁無行,常游太平門者,詎不知邪?」澣免冠頓首謝,貶高州良德丞,察再貶皇化尉,其黨齊敷、郭稟皆流放。



 久之,澣徙索盧丞、郴州長史、濠常二州刺史。遷潤州,州北距瓜步沙尾,紆匯六十里,舟多敗溺。澣徙漕路繇京口埭,治伊婁渠以達揚子,歲無覆舟,減運錢數十萬。又立伊婁埭,官徵其入;招還流人五百戶,置明州以安輯之。復徙汴州。



 澣中失勢,益悵恨,素操浸衰。更倚力士助,得為兩道採訪使,興利以中天子意,裒貨財遺謝貴幸。納劉戒女為妾,不答其妻。李林甫惡其行,欲擠而廢之。會其幕府坐贓,事連澣,詔矜澣老,放歸田里。天寶初,召為太子少詹事,留司東都。嚴挺之亦為林甫所廢,與澣家居,杖屨經過不缺日,林甫畏之,乃用澣為平陽太守,離其謀。更以黃老清靜為治,卒,年七十二。肅宗時,錄林甫所陷者,皆褒洗,故澣贈禮部尚書。



 澣嘗稱陳希烈、宋遙、苗晉卿、韋述之才,後皆大顯。



 麻察者,河東人,由明經第五遷殿中侍御史。魏元忠子升死節愍太子難,而元忠系大理,升妻鄭父遠,嘗納錢五百萬,以女易官。武後重元忠舊臣,欲榮其姻對,授遠河內令,子洛州參軍。元忠下獄,遣人絕婚,許之。明日,嫁其女。察劾遠敗風教,請錮終身,遠遂廢。當時謂察為公,而終以憸險斥云。



 澣孫抗。抗字遐舉,少值天寶亂,奉母夫人隱會稽。壽州刺史張鎰闢署幕府。抗吏事閑敏,有文雅,從鎰鎮江西。及以宰相領鳳翔,奏署監察御史。李楚琳亂,奔奉天,授侍御史,遷戶部員外郎。蕭復引為江淮宣慰判官。德宗自梁、洋還,財用大屈,鹽鐵使元琇薦抗材,改倉部郎中,斡鹽利。俄為水陸運副使,護漕江淮,給京師。歷諫議大夫,坐小累,為處州刺史。歷蘇州,徙潭州觀察使,召為給事中,遷河南尹,進太常卿,以中書侍郎同中書門下平章事。



 抗無遠謀大略,雖用心至精,末乃滋彰苛刻。以病乞身,罷為太子賓客。卒,年六十五,贈戶部尚書,謚曰成。



 初,吏部歲考書言,以它官第上下,中書、門下遣官覆實,以為常。抗以尚書、侍郎皆大臣選,今更覆核,非任人勿疑之道。禮部侍郎試貢士,其姻舊悉試考功,謂之「別頭」,皆奏罷之。又省州別駕、田曹司田官、判司雙曹者,減中書吏員。此其稍近治者云。



\end{pinyinscope}