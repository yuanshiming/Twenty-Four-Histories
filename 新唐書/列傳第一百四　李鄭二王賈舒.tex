\article{列傳第一百四 李鄭二王賈舒}

\begin{pinyinscope}

 李訓,字子垂,始名仲言,字子訓,故宰相揆族孫。質狀魁梧述商鞅變法的主張和過程,闡述其政治、哲學思想,對秦國,敏於辯論,多大言,自標置。擢進士第,補太學助教,闢河陽節度府。從父逢吉為宰相,以仲言陰險善謀事,厚暱之。坐武昭獄,流象州。文宗嗣位,更赦還,以母喪居東都。鄭注佐昭義府,仲言慨然曰:「當世操權力者皆齪齪,吾聞注好士,有中助,可與共事。」因往見注,相得甚歡。時逢吉方留守,怏怏不樂,思復用,知與注善,付金幣百萬,使西至京師厚結注。注喜,介之謁王守澄。守澄善遇之,即以注術、仲言經義並薦於帝。



 仲言持詭辯,激卬可聽,善鉤揣人主意,又以身儒者,海內望族,既見識擢,志望不淺。始,宋申錫謀誅守澄不克,死。宦尹益橫,帝愈憤恥。而憲祖之弒,罪人未得,雖外假借,內不堪,欲夷絕其類,顧在位臣持祿取容,無仗節死難者。注陰知帝指,屢建密計,引仲言葉力。帝外托講勸,又皆以守澄進,故與之謀則其黨不疑。仲言尚縗粗,帝使衣戎服,號「王山人」,與注出入禁中。服除,起為四門助教,賜緋袍、銀魚,時太和八年也。其十月,遷《周易》博士,兼翰林侍講學士。入院,詔法曲弟子二十人侑宴,示優寵。於是給事中鄭肅、韓佽、諫議大夫李珝、郭承嘏、中書舍人高元裕、權璩等共劾仲言憸人,天下共知,不宜在左右。帝不聽。仲言數進講,至閹寺,必感憤申重,以激帝心。帝見其言縱橫,謂果可任,遂不疑,而待遇莫與比,因改名訓。帝猶慮宦人猜忌,乃疏《易》五義示群臣,有能異訓意者賞,欲天下知以師臣待訓。



 明年秋七月,進翰林學士、兵部郎中,知制誥,居中倚重,實行宰相事。宦人陳弘志時監襄陽軍,訓啟帝召還,至青泥驛,遣使者杖殺之。復以計白罷守澄觀軍容使,賜鴆死。又逐西川監軍楊承和、淮南韋元素、河東王踐言於嶺外,已行,皆賜死。而崔潭峻前物故,詔剖棺鞭尸。元和逆黨幾盡。



 訓本挾奇進,及大權在己,銳意去惡,故與帝言天下事,無不如所欲。與注相朋比,務報恩復仇,素忌李德裕、宗閔之寵,乃因楊虞卿獄,指為黨人,嘗所惡者,悉陷黨中,遷貶無闋日,班列幾空,中外震畏。帝為下詔開諭,群情稍安。不逾月,以禮部侍郎同中書門下平章事,賜金紫服,仍詔三日一至翰林,以終《易》義。



 訓起流人,一歲至宰相,謂遭時,其志可行。欲先誅宦豎,乃復河、湟,攘夷狄,歸河朔諸鎮。意果而謀淺,天子以為然。俄賜第勝業里,賞賚旁午。每進見,它宰相備位,天子傾意,宦官衛兵皆心習憚迎拜。天下險怪士徼取富貴,皆憑以為資。訓時時進賢才偉望,以悅士心,人皆惑之。嘗建言天下浮屠避傜賦,耗國衣食,請行業不如令者還為民。既執政,自白罷,因以市恩。



 始,注先顯,訓藉以進,及勢相埒,賴寵爭功,不兩立。然方事未集,乃出注使鎮鳳翔,外為助援,內實猜克,待逞,且殺之。擢所厚善分總兵柄,於是王璠為太原節度使,郭行餘為邠寧節度使,羅立言權京兆尹,韓約金吾將軍,李孝本權御史中丞。陰許璠、行餘多募士及金吾臺府卒,劫以為用。



 十一月壬戌,帝御紫宸殿,約奏甘露降金吾左仗樹,群臣賀。訓、元輿奏言:「甘露近在禁中,陛下宜親往以承天祉。」許之。即輦如含元殿,詔宰相群臣往視。還,訓奏言:「非甘露。」帝曰:「豈約妄邪?」顧中尉仇士良、魚志弘等驗之,訓因欲閉止諸宦人,使無逸者。時璠、行餘皆辭赴鎮,兵列丹鳳門外,彀而待,訓傳呼曰:「兩鎮軍入受詔旨!」聞者趨入,邠寧軍不至,璠懼,弗能前,獨行餘拜殿下。宦人至仗所,約流汗不能舉首,士良等怪之曰:「將軍何為爾?」會風動廡幕,見執兵者,士良等驚,走出。閽者將闔扉,為宦侍叱爭,不及閉。訓急連呼金吾兵曰:「衛乘輿者,人賜錢百千!」於是有隨訓入者。宦人曰:「急矣,上當還內!」即扶輦決罘罳下殿趨,訓攀輦曰:「陛下不可去!」士良曰:「李訓反!」帝曰:「訓不反。」士良手搏訓而躓,訓壓之,將引刀靴中,救至,士良免。立言、孝本領眾四百東西來,上殿與金吾士縱擊,宦官死者數十人。訓持輦愈急,至宣政門,宦人郗志榮揕訓,僕之,輦入東上閣,即閉,宮中呼萬歲。元輿雖知謀,不以告涯,曰:「上將開延英邪?」而群臣見宰相問故。會士良遣神策副使劉泰倫、陳君奕等率衛士五百挺兵出,所值輒殺。涯等惶遽易服步出。殺諸司史六七百人,復分兵屯諸宮門,捕訓黨千餘人,斬四方館,流血成渠。宦豎知訓事連天子,相與怨嘖,帝懼,偽不語,故宦人得肆志殺戮。俄而元輿、涯皆為兵所執。涯實不知謀,士良榜笞急,乃自署反狀。詔出衛騎千餘,馳咸陽、奉天捕亡者,大索都城,分掩涯、訓等第,兵遂大掠,入黎埴、羅讓、渾金歲、胡證等家及賈耽廟,貲產一空。兩省印、簿書輒持去,秘館圖籍,蕩然無餘者。



 明日,召群臣朝,至建福門,從者不得入,光範門尚閉,列兵誰何,乃繇金吾右仗至宣政衙,兵皆露持。是時無宰相、御史中丞,久之,閣門使馬元贄啟宣政扉傳詔,張仲方可京兆尹,而吏皆前死,群臣不能班。帝初未知涯等被系,猶遲其不朝,既而士良白涯與訓謀逆,將立鄭注。遽召僕射令狐楚、鄭覃、兵部尚書王源中、吏部侍郎李虞仲等至,帝對悲憤,因付涯訊牒曰:「果涯書邪?」楚曰:「然!涯誠有謀,罪應死。」



 是日,京師兵剽劫未止,民乘亂,往往復私怨,相戕擊,人死甚眾。帝遣楊鎮、靳遂良等屯兵大衢,鼓而儆之,兵乃止。帝逼宦官,於是下詔暴訓、涯等罪。孝本易綠誇,猶金帶,以帽障面,奔鄭注,至咸陽,追騎及之。餗匿民間,羸服乘驢自歸。璠聚河東兵環第自衛,弘志使偏將攻之,呼曰:「王涯等得罪,起尚書為相。」璠喜,啟關納之。既行,知見紿,泣曰:「李訓累我。」俄行餘、立言皆得。自涯十餘族並奴婢悉系左右軍。璠見涯,恚曰:「公何見引?」涯曰:「君昔漏宋丞相謀於守澄,今焉逃死?」



 訓既敗,被綠衣,詭言黜官,走終南山,依浮屠宗密。宗密欲匿之,其徒不可,乃奔鳳翔,為盩厔將所執,械而東。訓恐為宦人酷辱,祈監者曰:「得我者有賞,不如持首去。」乃斬之,傳其首,餘黨悉禽。



 後一日,兩神策兵將涯等赴郊廟,過兩市,皆腰斬梟首以徇。餗臨刑憤叱,獨元輿曰:「晁錯、張華尚不免,豈特吾屬哉?」約最後捕得,責以反狀,不服,斬之。殺訓弟仲褒、元皋。始,元皋以屬疏自解,得去,士良訊奴,言事前一昔宿訓第,遣人追斬之。訓死,士良捕宗密,將殺之,怡然曰:「與訓游久,浮屠法遇困則救,死固其分。」乃釋之。是時暴尸旁午,有詔棄都外,男女孩嬰相雜廁。淹旬,許京兆府瘞斂,作二大塚,葬道左右。



 它日,帝頗思訓,數為李石、鄭覃稱其才。而宦豎益熾,帝末以制,居常忽忽不懌,每游燕,雖倡樂雜沓,未嘗歡,顏慘不展,往往瞋目獨語,或裴回眺望,賦詩以見情,自是感疢,至棄天下云。



 鄭注,絳州翼城人。世微賤,以方伎游江湖間。元和末,至襄陽,依節度使李愬。為愬煮黃金餌之,浸親遇,署衙推,從至徐州,稍參處軍政。注多藝,詭譎陰狡,億探人廋隱,輒中所欲。為愬籌事,未嘗不用,挾邪市權,舉軍患之。監軍王守澄白愬,愬曰:「然彼奇士也,將軍試與語。」守澄始拒不納,既坐,機辯橫生,鉤得其意,守澄大驚,引至後堂,語終夕,恨相見晚。謝愬曰:「誠如公言。」即署巡官。



 守澄入總樞密,與俱至京師,厚加贍恤,日夜為守澄計議,因陰通賂遺。初士纖巧者附離,後要官貴人亦趨往。既陷宋申錫,搢紳側目。金吾將孟文亮鎮邠寧,取為司馬,不肯行,御史中丞宇文鼎劾奏,乃上道,過奉天輒還。御史復言注奸狀,請付有司治罪。始,王涯用注力再輔政,又憚守澄,遏其奏。更擢通王府司馬、右神策判官,士議言雚駭。劉從諫惡其人,欲因斥去之,即表副昭義節度。至府不旬月,文宗暴眩,守澄復薦注,即日召入,對浴堂門,賜賚至渥。是夜,彗出東方,長三尺,芒耀怒急。俄進太僕卿,兼御史大夫。



 注資貪沓,既藉權寵,專鬻官射利,貲積鉅萬,不知止。起第善和里,通永巷,飛廡復壁,聚京師輕薄子、方鎮將吏,以煽聲焰。間入神策,與守澄語必終日,或夜艾乃罷。險人躁夫有所干謝,日走門。李訓既附注進,於是兩人權震天下矣。尋擢工部尚書、翰林侍講學士,時訓已在禁中,日日議論帝前,相倡和,謀鉏翦中官,自謂功在晷刻,帝惑之。乘是進退士大夫,撓骫朝法,賢不肖淆亂,以為弛張當然。眾策其必亂。



 帝問富人術,以榷茶對。其法欲置茶官,籍民圃而給其直,工自擷暴,則利悉之官。帝始詔王涯為榷茶使。又言秦、雍災,當興役厭之。帝嘗詠杜甫《曲江辭》,有「宮殿千門」語,意天寶時環江有觀榭宮室,聞注言,即詔兩神策治曲江、昆明,作紫雲樓、採霞亭,詔公卿得列舍堤上。



 注本姓魚,冒為鄭,故當時號「魚鄭」。及用事,人廋謂曰「水族」。貌寢陋,不能遠視,常衣粗裘,外示質素。始,李愬病痿,注治之有狀,守澄神其術,故中人皆暱愛。



 俄檢校尚書左僕射、鳳翔隴右節度使,詔月入奏事。請寮屬於訓,訓與舒元輿謀終殺注,慮其豪俊為助,更擇臺閣長厚者,以錢可復為副,李敬彞為司馬,盧簡能、蕭傑為判官,盧弘茂為掌書記。舊制,節度使受命,戎服詣兵部謁,後浸廢,注請復之,而王璠、郭行餘皆踵為常。是日,度支、京兆等供帳。入辭,帝賜通天犀帶。出都門,旗幹折,注惡之。



 先是,守澄死,以十一月葬滻水,注奏言:「守澄,國勞舊,願身護喪。」因群宦者臨送,欲以鎮兵悉禽誅之。訓畏注專其功,乃先五日舉事。注率五百騎至,扶風令韓遼知其謀,奔武功。注聞訓敗,乃還。其屬魏弘節勸注殺監軍張仲清及大將賈克中等十餘人,注驚撓不暇聽。仲清與前少尹陸暢用其將李叔和策,訪注計事,斬其首,兵皆潰去。注妻兄魏逢尤佻險,贊注為奸,數顧賕,為率更令、鳳翔少尹。遣逢至京師與訓約,被誅。可復等及親卒千餘人皆族矣。擢仲清內常侍,遼咸陽令,叔和檢校太子賓客,賜錢千萬,暢鳳翔行軍司馬。



 梟注首光宅坊,三日瘞之,群臣皆賀,乃夷其家。初,未獲注,京師戒嚴,涇原、鄜坊節度使王茂元、蕭弘皆勒兵備非常。及是人相慶。籍其貲,得絹百萬匹,它物稱是。注敗前,菌生所服帶上,褚中藥化為蠅數萬飛去。



 可復,徽子也,為禮部郎中。簡能者,簡辭弟,駕部員外郎。傑者,俛弟也,主客員外郎。弘茂,右拾遺。可復將死,女年十四,為祈免,女曰:「殺我父,何面目以生!」抱可復求死,亦斬之。弘茂妻蕭,臨刑詬曰:「我太后妹,奴輩可來殺!」兵皆斂手,乃免。弘節勇而多謀,始在鄜坊趙儋節度府,為注所闢。敬彞為路隋所闢,隋卒,客江淮,以未赴免,因擢兵部員外郎,終衢州刺史。



 王涯,字廣津,其先本太原人,魏廣陽侯冏之裔。祖祚,武后時諫罷萬象神宮知名;開元時,以大理司直馳傳決獄,所至仁平。父晃,歷左補闕、溫州刺史。



 涯博學,工屬文。往見梁肅,肅異其才,薦於陸贄。擢進士,又舉宏辭,再調藍田尉。久之,以左拾遺為翰林學士,進起居舍人。元和初,會其甥皇甫湜以賢良方正對策異等,忤宰相,涯坐不避嫌,罷學士,再貶虢州司馬,徙為袁州刺史。憲宗思之,以兵部員外郎召,知制誥,再為翰林學士,累遷工部侍郎,封清源縣男。



 涯文有雅思,永貞、元和間,訓誥溫麗,多所稿定。帝以其孤進自樹立,數訪逮,以私居遠,或召不時至,詔假光宅里官第,諸學士莫敢望。俄拜中書侍郎、同中書門下平章事,坐循默不稱職罷。再遷吏部侍郎。



 穆宗立,出為劍南東川節度使。時吐蕃寇邊,西北騷然,又略雅州,涯調兵拒之。上言:「蜀有兩道直搗賊腹,一繇龍川清川以抵松州,一繇綿州威蕃柵抵棲雞城,皆虜險要地。臣願不愛金帛,使信臣持節與北虜約曰:『能發兵深入者,殺某人,取某地,受某賞。』開懷以示之,所以要約諄熟異它日者,則匈奴之銳可出,西戎之力衰矣。」帝不報。



 長慶三年,入為御史大夫,遷戶部尚書、鹽鐵轉運使。寶歷時,復出領山南西道節度使。文宗嗣位,召拜太常卿,以吏部尚書代王播,復總鹽鐵,政益刻急。歲中,進尚書右僕射、代郡公。而御史中丞宇文鼎以涯兼使職,恥為之屈,奏:「僕射視事日,四品以上官不宜獨拜。」涯怒,即建言:「與其廢禮,不如審官,請避位以存舊典。」帝難之,詔尚書省雜議。工部侍郎李固言謂:「《禮》:君於士不答拜,非其臣則答,不臣人之臣也;大夫於其臣,雖賤必答拜,避正君也;大夫於獻不親,君有賜不面拜,為君之答己也。古者列國君猶與大夫答拜,所以尊事天子,別嫌明微也。議者謂『僕射代尚書令,禮當重。凡百司州縣皆有副貳,缺則攝總,至著定之禮,則不可越,僕射由是也』。按令,凡文武三品拜一品,四品拜二品。《開元禮》,京兆河南牧、州刺史、縣令上日,丞以下答拜。此禮令相戾,不可獨據。」又言:「受冊官始上,無不答拜者,而僕射亦受冊,禮不得異。雖相承為故事,然人情難安者,安得弗改?請如禮便。」帝不能決,涯竟用舊儀。



 自李師道平,三道十二州皆有銅鐵官,歲取冶賦百萬,觀察使擅有之,不入公上。涯始建白:「如建中元年九月戊辰詔書,收隸天子鹽鐵。」詔可。久之,以本官同中書門下平章事,合度支、鹽鐵為一使,兼領之。乃奏罷京畿榷酒錢以悅眾。俄檢校司空,兼門下侍郎。罷度支,真拜司空。始變茶法,益其稅以濟用度,下益困,而鄭注亦議榷茶,天子命涯為使,心知不可,不敢爭。李訓敗,乃及禍。初,民怨茶禁苛急,涯就誅,皆群詬詈,抵以瓦礫。



 涯質狀頎省,長上短下,動舉詳華。性嗇儉,不畜妓妾,惡卜祝及它方伎。別墅有佳木流泉,居常書史自怡,使客賀若夷鼓琴娛賓。文宗惡俗侈靡,詔涯懲革。涯條上其制,凡衣服室宇,使略如古,貴戚皆不便,謗訕囂然,議遂格。然涯年過七十,嗜權固位,偷合訓等,不能絜去就,以至覆宗。是時,十一族貲貨悉為兵掠,而涯居永寧里,乃楊憑故第,財貯鉅萬,取之彌日不盡。家書多與秘府侔,前世名書畫,嘗以厚貨鉤致,或私以官,鑿垣納之,重復秘固,若不可窺者。至是為人破垣剔取奩軸金玉,而棄其書畫於道。籍田宅入於官。



 子孟堅為工部郎中、集賢殿學士,仲翔太常博士,季琰校書郎,皆死。仲翔始匿侍御史裴鐇家,鐇執以赴軍,仲翔曰:「業不見容,當自求生,奈何反相噬邪?」聞者哀之。後令狐楚見帝,從容言:「向與臣並列者,既族滅矣,而露胔不藏,深可悼痛。」帝惻然,詔京兆尹薛元賞葬涯等十一人,各賜襲衣。仇士良使盜竊發其塚,投骨渭水。涯女為竇紃妻,以痼病免,家人紿告涯當貶,忽夢涯自提首告曰:「族滅矣,惟若存,歲時無忘我。」女驚號墮地,乃以實告。涯從弟沐,客江南,困窮來京師謁涯,二歲乃得見,許以祿仕,難作,亦死。



 昭宗天復初,大赦,明涯、訓之冤,追復爵位,官其後裔。



 賈餗,字子美,河南人。少孤,客江淮間。從父全觀察浙東,餗往依之,全尤器異,收恤良厚。舉進士高第,聲稱籍甚。又策賢良方正異等,授渭南尉、集賢校理。擢累考功員外郎,知制誥。餗美文辭,開敏有斷,然褊急,氣陵輩行。李渤為諫議大夫,惡其人,為宰相言之,而李逢吉、竇易直愛餗才,得不斥。



 穆宗崩,告哀江、浙,道拜常州刺史。舊制,兩省官出使,得硃衣吏前導。餗赴州,猶用之,觀察使李德裕敕吏還,怏怏為憾。入為太常少卿,復知制誥,歷禮部侍郎,凡三典貢舉,得士七十五人,多名卿宰相。再遷京兆尹、兼御史大夫、姑臧縣男。太和九年上巳,詔百官會曲江。故事,尹自門步入,揖御史。食束自矜大,不徹扇蓋,騎而入。御史楊儉、蘇特固爭,餗曰:「黃面兒敢爾!」儉曰:「公為御史,能嘿嘿耶?」大夫溫造以聞。坐奪俸,不勝恚,求出為浙西觀察使。未行,拜中書侍郎、同中書門下平章事。俄為集賢殿大學士、監修國史。既得位,會李宗閔得罪,而指儉、特為黨,斥去之。



 少與沈傳師善,傳師前死,嘗夢云:「君可休矣!」餗寤而祭諸寢,復夢曰:「事已爾,叵奈何!」劉蕡以賢良方正對策,指中人為禍亂根本,而餗與馮宿、龐嚴為考官,畏避不敢聞,竟罹其禍。



 餗本中立,不肯身犯顏排奸幸以及誅,與王涯實不知謀,人冤之。



 舒元輿,婺州東陽人。地寒,不與士齒。始學,即警悟。去客江夏,節度使郗士美異其秀特,數延譽。



 元和中,舉進士,見有司鉤校苛切,既試尚書,雖水炭脂炬餐具,皆人自將,吏一倡名乃得入,列棘圍,席坐廡下,因上書言:「古貢士未有輕於此者,且宰相公卿繇此出,夫宰相公卿非賢不在選,而有司以隸人待之,誠非所以下賢意。羅棘遮截疑其奸,又非所以求忠直也。詩賦微藝,斷離經傳,非所以觀人文化成也。臣恐賢者遠辱自引去,而不肖者為陛下用也。今貢珠貝金玉,有司承以棐笥皮幣。何輕賢者,重金玉邪?」又言:「取士不宜限數,今有司多者三十,少止二十,假令歲有百元凱,而曰吾格取二十,謂求賢可乎?歲有才德才數人,而曰必取二十,謬進者乃過半,謂合令格可乎?」



 俄擢高第,調鄠尉,有能名。裴度表掌興元書記,文檄豪健,一時推許。拜監察御史,劾按深害無所縱。再遷刑部員外郎。



 元輿自負才有過人者,銳進取。太和五年,獻文闕下,不得報。上書自言:「馬周、張嘉貞代人作奏,起逆旅,卒為名臣。今臣備位於朝,自陳文章,凡五晦朔不一報,竊自謂才不後周、嘉貞,而無因入,又不露所縕,是終無振發時也。漢主父偃、徐樂、嚴安以布衣上書,朝奏暮召,而臣所上八萬言,其文鍛煉精粹,出入今古數千百年,披剔剖抉,有可以輔教化者未始遺,拔犀之角,擢象之齒,豈主父等可比哉?盛時難逢,竊自愛惜。」文宗得書,高其自激卬,出示宰相,李宗閔以浮躁誕肆不可用,改著作郎,分司東都。



 時李訓居喪,尤與元輿善。及訓用事,再遷左司郎中。御史大夫李固言表知雜事。固言輔政,權知御史中丞。會帝錄囚,元輿奏辨明審,不三月即真,兼刑部侍郎。專附鄭注,注所惡,舉繩逐之。月中,以本官同中書門下平章事。詭謀謬算,日與訓比,敗天下事,二人為之也。然加禮舊臣,外釣人譽。先時,裴度、令狐楚、鄭覃皆為當路所軋,致閑處。至是,悉還高秩。



 元輿為《牡丹賦》一篇,時稱其工。死後,帝觀牡丹,憑殿闌誦賦,為泣下。



 弟元褒、元肱、元迥,皆第進士。元褒又擢賢良方正,終司封員外郎。餘及誅。



 王璠,字魯玉。元和初舉進士、宏辭,皆中,遷累監察御史。儀宇峻整,著稱於時。以起居舍人副鄭覃宣慰鎮州。長慶末,擢職方郎中,知制誥。



 時李逢吉秉政,特厚璠,驟拜御史中丞。璠挾所恃,頗橫恣,道直左僕射李絳,交騎不避。絳上言:「左右僕射,師長庶官,開元時,名左右丞相,雖去機務,然猶總百司,署位不著姓。上日班見百官,而中丞、御史在廷。元和中,伊慎為僕射,太常博士韋謙以慎位緣恩進,削其禮,至僕射就臺見中丞,或立廷中,中丞乃至。憲度倒置,不可為法。」逢吉憚絳正,遏其事不奏,但罷璠為工部侍郎,而絳亦用太子少師分司東都,議者不直之。初,璠按武昭獄,意逢吉德己,及罷中丞,乃失望。



 久之,出為河南尹。時內廄小兒頗擾民,璠殺其尤暴者,遠近畏伏。入為尚書右丞,再遷京兆尹。自李諒後,政條隳斁,奸豪浸不戢,璠頗修舉,政有名。



 鄭注奸狀始露,宰相宋申錫、御史中丞宇文鼎密與璠議除之,璠反以告王守澄,而注由是傾心於璠。進左丞,判太常卿事。出為浙西觀察使。李訓得幸,璠於逢吉舊故,故薦之,復召為左丞,拜戶部尚書,判度支,封祁縣男。李宗閔得罪,璠亦其黨,見注求解,乃免。訓將誅宦人,乃授河東節度使,已而敗。



 璠子遐休,直弘文館,所善學士令狐定及劉軻、劉軿、仲無頗、柳喜集其所,皆被縛。定等自解辯,得釋。遐休誅。璠鑿潤州外隍,得石刻曰:「山有石,石有玉,玉有瑕。」術家謂璠祖名崟,生礎,礎生璠,盡遐休,蓋其應云。



 郭行餘者,元和時擢進士。河陽烏重胤表掌書記。重胤葬其先,使志塚,辭不為,重胤怒,即解去。擢累京兆少尹。嘗值尹劉棲楚,不肯避,棲楚捕導從系之。自言宰相裴度,頗為諭止。行餘移書曰:「京兆府在漢時有尹,有都尉,有丞,皆詔自除,後循而不改。開元時,諸王為牧,故尹為長史,司馬即都尉、丞耳。今尹總牧務,少尹副焉,未聞道路間有下車望塵避者,故事猶在。」棲楚不能答。



 遷楚、汝二州刺史、大理卿,擢邠寧節度使。李訓在東都,與行餘善,故用之。



 韓約,朗州武陵人,本名重華。志勇決,略涉書,有吏乾。歷兩池榷鹽使、虔州刺史。交趾叛,領安南都護。再遷太府卿。太和九年,代崔鄯為左金吾衛大將軍,居四日,起事。約繇錢穀進,更安南富饒地,聚貲尤多。



 羅立言者,宣州人。貞元末擢進士,魏博田弘正表佐其府。改陽武令,以治劇遷河陰。立言始築城郭,地所當者,皆富豪大賈所占,下令使自築其處,吏籍其闊狹,號於眾曰:「有不如約,為我更完!」民憚其嚴,數旬畢。民無田者,不知有役。設鎖絕汴流,奸盜屏息。河南尹丁公著上狀,加朝散大夫。然倨下傲上,出具弓矢呵道,宴賓客列倡優如大府,人皆惡之,以是稀遷,然自放不衰。



 改度支河陰留後,坐平糴非實,沒萬九千緡,鹽鐵使惜其干,止奏削兼侍御史。繇廬州刺史召為司農少卿,以財事鄭注,亦與李訓厚善。訓以京兆多吏卒,擢為少尹,知府事,以就其謀。



 李孝本,宗室子。元和時第進士,累遷刑部郎中。依訓得進,於是御史中丞舒元輿引知雜事。元輿入相,擢權知中丞事。



 顧師邕,字睦之,少連子。性恬約,喜書,寡游合。第進士。累遷監察御史。李訓薦為水部員外郎、翰林學士。訓遣宦官田全操、劉行深、周元稹、薛士干、似先義逸、劉英誗按邊,既行,命師邕為詔,賜六道殺之,會訓敗,不果。師邕流崖州,至藍田,賜死。



 李貞素,嗣道王實子。性和裕,衣服喜鮮明。漢陽公主妻以季女。累遷宗正少卿,由將作監改左金吾衛將軍。韓約之詐,貞素知之。流儋州,至商山,賜死。



 贊曰:李訓浮躁寡謀,鄭注斬斬小人,王涯暗沓,舒元輿險而輕,邀幸天功,寧不殆哉!李德裕嘗言天下有常勢,北軍是也。訓因王守澄以進,此時出入北軍,若以上意說諸將,易如靡風,而反以臺、府抱關游徼抗中人以搏精兵,其死宜哉!文宗與宰相李石、李固言、鄭覃稱:「訓稟五常性,服人倫之教,不如公等,然天下奇才,公等弗及也。」德裕曰:「訓曾不得齒徒隸,尚才之云!」世以德裕言為然。《傳》曰:「國將亡,天與之亂人。」若訓等持腐株支大廈之顛,天下為寒心豎毛,文宗偃然倚之成功,卒為閹謁所乘,天果厭唐德哉!



\end{pinyinscope}