\article{志第二十九 地理三}

\begin{pinyinscope}

 河東道,蓋古冀州之域,漢河東、太原、上黨、西河、雁門、代郡及鉅鹿、常山、趙國、廣平國之地。河中、絳、晉、慈、隰、石、太原、汾、忻、潞、澤、沁、遼為實沈分,代、雲、朔、蔚、武、新、嵐、憲為大梁分。為府二,州十九在自己的發展中不斷以新的經驗、新的知識豐富起來。,縣百一十。其名山:雷首、介、霍、五臺。其大川:汾、沁、丹、潞。厥賦:布、繭。厥貢:布、席、豹尾、熊郭、雕羽。



 河中府河東郡,赤。本蒲州,上輔。義寧元年治桑泉,武德三年徙治河東。開元八年置中都,為府;是年罷都,復為州。乾元三年復為府。土貢:氈、簹扇、龍骨、棗、鳳棲梨。戶七萬八百,口四十六萬九千二百一十三。縣十三:有府三十三,曰興樂、德義、胡壁、龍亭、清源、永和、陶城、霍山、瀵水、首陽、壽貴、歸仁、長渠、虞城、通閏、寶鼎、鹽海、歸淳、大陽、永安、奉信、永興、右威、汾陰、甘泉、平川、安保、石門、綏化、壇道、安邑、崇義、六軍。又有耀德軍,乾元二年置,廣德二年廢。河東,次赤。有芳醞監、汲河以釀,武德三年置,貞觀十年廢;南有風陵關,聖歷元年置;有歷山。河西,次赤。開元八年析河東置,尋省。乾元三年更同州之朝邑曰河西,來屬。大歷五年復還同州,析朝邑、河東別置。有蒲津關,一名蒲阪。開元十三年鑄八牛,牛有一人策之,牛下有山,皆鐵也,夾岸以維浮梁。十五年自朝邑徙河瀆祠於此。臨晉,次畿。本桑泉,武德三年析置溫泉縣,九年省。天寶十三載更名。解,次畿。本虞鄉,武德元年更名。貞觀十七年省,以地入虞鄉,二十二年復置。有鹽池,又有女鹽池;有紫泉監,乾元元年置;有銅穴十二。猗氏,次畿,有孤山。虞鄉,次畿。武德元年別置。貞觀二十二年省,以地入解。天授二年復置。北十五里有涑水渠,貞觀十七年,刺史薛萬徹開,自聞喜引涑水下入臨晉。永樂,次畿。武德元年置,本隸芮州,州廢,隸鼎州,貞觀八年來屬,後又隸虢州,神龍元年復故。有雷首山。安邑,次畿。義寧元年以安邑、虞鄉、夏置安邑郡。武德元年曰虞州,又析置桐鄉縣。三年析安邑置興樂縣。貞觀元年省。十七年州廢,省桐鄉入聞喜,以安邑、解來屬。至德二載更安邑曰虞邑,乾元元年隸陜州,大歷四年復故名,元和三年來屬。有龍池宮,開元八年置;有鹽池,與解為兩池,大歷十二年生乳鹽,賜名寶應靈慶池;有銀監。寶鼎,次畿。本汾陰。義寧元年以汾陰、龍門置汾陰郡,武德元年曰泰州,州廢來屬。開元十年獲寶鼎,更名。有後土祠。襄陵,緊。本隸晉州,元和十四年隸絳州,大和元年來屬。稷山,上。本隸絳州,唐末來屬。有稷山。萬泉,上。本隸泰州,武德三年析稷山、安邑、猗氏、汾陰、龍門置,州廢隸絳州,大順二年來屬。有介山。龍門。次畿。武德二年徙泰州來治,五年析置萬春縣。貞觀十七年州廢,省萬春入龍門,隸絳州。元和初來屬。有龍門關;有高祖廟,貞觀中置;北三十里有瓜谷山堰,貞觀十年築;東南二十三里有十石壚渠,二十三年,縣令長孫恕鑿,溉田良沃,畝收十石;西二十一里有馬鞍塢渠,亦恕所鑿;有龍門倉,開元二年置。



 晉州平陽郡,望。本臨汾郡,義寧二年更名。土貢:蠟燭。有平陽院礬官。戶六萬四千八百三十六,口四十二萬九千二百二十一。縣八:有府十五,曰神山、平陽、豐寧、冀城、安信、萬安、益昌、英臺、岳陽、仁壽、高陽、臨汾、晉安、白澗、高華、仁德。臨汾,望。東北十里有高梁堰,武德中引高梁水溉田,入百金泊。貞觀十三年為水所壞。永徽二年,刺史李寬自東二十五里夏柴堰引潏水溉田,令陶善鼎復治百金泊,亦引潏水溉田。乾封二年堰壞,乃西引晉水;有姑射山。洪洞,望。本楊,義寧二年更名。武德元年析洪洞、臨汾置西河縣,貞觀十七年省入臨汾。神山,中。本浮山,武德二年析襄陵置。東南有羊角山。四年以老子祠更名。岳陽,中。東有府城關;有鐵。霍邑,上。義寧元年以霍邑、趙城、汾西、靈石置霍山郡。武德元年曰呂州。貞觀十七年州廢,以靈石隸汾州,霍邑、趙城、汾西來屬。有西北鎮霍山祠。趙城,上。義寧元年析霍邑置。汾西,中。有鐵。冀氏。中。



 絳州絳郡,雄。土貢:自縠、粱米、梨、墨、蠟燭、防風。戶八萬二千二百四,口五十一萬七千三百三十一。縣七:有府三十三,曰新田、太平、正平、武城、長社、大鄉、垣城、涑川、絳川、蓋松、鳳亭、延光、平原、高涼、神泉、桐鄉,萬泉、翼城、皮氏、董澤、零原、石池、延福、永康、景山、周陽、夏臺、古亭、崇樂、絳邑、長平、武陽、蒲邑。正平,望。西有武平故關。太平,緊。有太平關,貞觀七年置。曲沃,望。東北三十五里有新絳渠,永徽元年,令崔翳引古堆水溉田百餘頃;南十三里山有銅。翼城,望。義寧元年以翼城、絳置翼城郡,並置小鄉縣。武德元年曰澮州,二年曰北澮州,四年州廢,縣皆來屬,九年省小鄉入翼城。天祐二年更曰澮川。有銅源、翔皋錢坊二;有澮高山,有銅,有鐵。絳,望。有鐵。聞喜,望。武德元年置。有銅冶。東南三十五里有沙渠,儀鳳二年,詔引中條山水於南坡下,西流經十六里,溉涑陰田。垣。上。義寧元年以垣、王屋置邵原郡,又置清廉、亳城二縣。武德元年曰邵州。二年置長泉縣,是年,以長泉隸懷州,後省。五年省亳城入垣。貞觀元年州廢,省清廉入垣,來屬。龍朔三年隸洛州,長安二年復舊;貞元三年隸陜州,元和三年復舊。



 慈州文城郡,下。本汾州,武德五年曰南汾州,貞觀八年更名。土貢:白蜜、蠟燭。戶萬一千六百一十六,口六萬二千四百八十六。縣五:有府三,曰仵城、吉昌、平昌。吉昌,中。有鐵。文城,中。天祐中更曰屈邑。有孟門山、石鼓山。昌寧,中。有鐵。呂香,中。本平昌,義寧元年析仵城置,貞觀元年更名。仵城。中。有雞山。



 隰州大寧郡,下。本龍泉郡,天寶元年更名。土貢:胡女布、蜜、蠟燭。戶萬九千四百五十五,口十三萬四千四百二十。縣六:有府六,曰隰川、大義、孝敬、修善、玉城、屈產。隰川,中。蒲,中。武德二年以縣置昌州,並置仵城、常安、昌原三縣。貞觀元年州廢,省昌原、仵城、常安,以蒲來屬。西南有常安原。大寧,中。本仵城,武德二年更名,是年置中州,並置大義、白龍二縣。貞觀元年州廢,省大義、白龍,以大寧來屬。有孔山。西有馬鬥關。永和,中。武德二年置東和州,六年析置樓山縣。貞觀二年州廢,省樓山,以永和來屬。西北有永和關。石樓,中。武德二年以縣置西德州,並置長壽、臨河二縣。貞觀元年州廢,省長壽、臨河,以石樓隸東和州,州廢來屬。北有上平津。溫泉。中。武德三年置北溫州,並置新城、高唐二縣。貞觀元年州廢,省新城、高唐,以溫泉來屬。有鐵。



 北都,天授元年置,神龍元年罷,開元十一年復置,天寶元年曰北京,上元二年罷,肅宗元年復為北都。晉陽宮在都之西北,宮城周二千五百二十步,崇四丈八尺。都城左汾右晉,潛丘在中,長四千三百二十一步,廣三千一百二十二步,周萬五千一百五十三步,其崇四丈。汾東曰東城,貞觀十一年長史李勣築。兩城之間有中城,武后時築,以合東城。宮南有大明城,故宮城也。宮城東有起義堂。倉城中有受瑞壇。唐初高祖使子元吉留守,獲瑞石,有文曰「李淵萬吉」,築壇,祠以少牢。



 太原府太原郡,本並州,開元十一年為府。土貢:銅鏡、鐵鏡、馬鞍、梨、蒲萄酒及煎玉粉屑、龍骨、柏實人、黃石鉚、甘草、人、礬石、石。戶十二萬八千九百五,口七十七萬八千二百七十八。縣十三:有府十八,曰興政、復化、寧靜、洞渦、五泉、昌寧、志節、汾陽、靜智、信童、晉原、聞陽、清定、豐川、竹馬、攘胡、西胡、文谷。城中有天兵軍,開元十一年廢。太原,赤。井苦不可飲,貞觀中,長史李勣架汾引晉水入東城,以甘民食,謂之晉渠。晉陽,赤。有號令堂,高祖誓義師於此。西北十五里有講武臺、飛閣,顯慶五年築。有龍山。太谷,畿。武德三年以太谷、祁置太州,六年州廢,二縣來屬。東南八十里馬嶺上有長城,自平城至於魯口三百里,貞觀元年廢。祁,畿。文水,畿。武德三年隸汾州,六年來屬,七年又隸汾州,貞觀元年復舊,天授元年更名武興,神龍元年復故名。西北二十里有柵城渠,貞觀三年,民相率引文谷水,溉田數百頃;西十里有常渠,武德二年,汾州刺史蕭顗引文水南流入汾州;東北五十里有甘泉渠,二十五里有蕩沙渠,二十里有靈長渠、有千畝渠,俱引文谷水,傳溉田數千頃,皆開元二年令戴謙所鑿。榆次,畿。盂,畿。武德三年以盂、受陽置受州,貞觀元年省並州之烏河縣入焉。有銅、有鐵。東北有白馬故關。壽陽,畿。本受陽。武德六年徙受州來治,又以遼州之石艾、樂平隸之。貞觀八年州廢,縣皆來屬,十一年更名。有方山。樂平,畿。廣陽,畿。本石艾,天寶元年更名。東有井陘故關,東北有盤石故關、葦澤故關。清源,畿。武德元年置。交城,畿。先天二年析置靈川縣,開元二年省。有鐵。陽曲。畿。本陽直。武德三年析置汾陽縣,七年省陽直,更汾陽曰陽曲,仍析置羅陰縣。貞觀元年省,六年以蘇農部落置燕然縣,隸順州,八年僑治陽曲,十七年省。有赤塘關、天門關。



 汾州西河郡,望。本浩州,武德三年更名。士貢:鞍、面氈、龍須席、石膏、消石。戶五萬九千四百五十,口三十二萬二百三十。縣五:有府十二,曰嘉善、六壁、崇德、華夏、靈扶、五柳、京陵、介休、賈胡、寧固、開遠、清勝。西河,望。本隰城,肅宗上元元年更名。孝義,望。本永安,貞觀元年更名,有隱泉山。介休,望。義寧元年以介休、平遙置介休郡,武德元年曰介州,貞觀元年州廢,以二縣來屬。有雀鼠谷,有介山。平遙,望。靈石。上。有賈胡堡,宋金剛拒唐兵,高祖所次。西南有陰地關,又有長寧關。



 沁州陽城郡,下。本義寧郡,義寧元年置,天寶元年更郡名。土貢:龍須席、弦麻。戶六千三百八,口三萬四千九百六十三。縣三:有府二,曰延雙、安樂。沁源,中。武德二年析置招遠縣,三年省。有柴店關。和川,中。義寧元年析沁源置。綿上。中。有鐵。



 遼州樂平郡,下。武德三年析並州之樂平、遼山、平城、石艾置,六年徙治遼山,八年曰箕州。先天元年避玄宗名曰儀州。中和三年復曰遼州。土貢:人、蠟。戶九千八百八十二,口五萬四千五百八十。縣四:有府三,曰遼城、清穀、龍城。遼山,中。榆社,中。本隸太原郡,義寧元年析上黨之鄉置。武德元年隸韓州。三年以縣及並州之平城置榆州,又析置偃武縣。六年州廢,省偃武,以榆社、平城來屬。平城,中。和順。中。武德三年析置義興縣,六年省。



 嵐州樓煩郡,下。本東會州,武德六年更名。土貢:熊革郭、麝香。戶萬六千七百四十八,口八萬四千六。縣四:有府一,曰嵐山。有守捉兵。宜芳,中。本嵐城,武德四年更名,析置豐閏、合會二縣,五年省豐閏,六年省合會。靜樂,中。武德四年置管州,仍析置汾陽、六度二縣。五年曰北管州。六年州廢,省汾陽、六度,以靜樂來屬。有天池祠;有管涔山;北有樓煩關,有隋故汾陽宮。合河,中。本臨泉。武德三年曰臨津,四年隸東會州,九年省太和縣入焉。貞觀元年更名,三年復置,大和八年又省。北有合河關,東有蔚汾關。嵐谷。中。長安三年析宜芳置,神龍二年省,開元十二年復置。有岢嵐軍,永淳二年以岢嵐鎮為柵,長安三年為軍,景龍中,張仁亶徙其軍於朔方,留者號岢嵐守捉,隸大同。



 憲州,下。本樓煩監牧,嵐州刺史領之。貞元十五年別置監牧使。龍紀元年,李克用表置州,領縣三:樓煩,下。玄池,下。有鐵。天池。下。有雁門關。



 石州昌化郡,下。本離石郡,天寶元年更名。土貢:胡女布、龍須席、蜜、蠟燭、荑。戶萬四千二百九十四,口六萬六千九百三十五。縣五:有府二,曰離石、昌化。離石,中。平夷,中。有孝文山。定胡,中。武德三年置西定州。貞觀二年州廢,來屬,又析置孟門縣,七年省。西有孟門關。臨泉,中。本太和。武德三年更名,置北和州,別析置太和縣,四年以太和隸東會州。貞觀三年州廢,以臨泉來屬。方山。中。武德二年以縣置方州,三年州廢來屬。



 忻州定襄郡,下。本新興郡,義寧元年以樓煩郡之秀容置。土貢:麝香、豹尾。戶萬四千八百六,口八萬二千三十二。縣二:有府四,曰秀容、高城、漳源、定襄。有守捉兵。秀容。上。貞觀五年以思結部落於縣境置懷化縣,隸順州。十二年以懷化隸代州,後省。有系舟山,有鐵。定襄。上。武德四年析秀容置。有石嶺關。



 代州雁門郡,中都督府。土貢:蜜、青碌彩、麝香、豹尾、白雕羽。戶二萬一千二百八十,口十萬三百五十。縣五:有府三,曰五臺、東冶、雁門。有守捉兵。其北有大同軍,本大武軍,調露二年曰神武軍,天授二年曰平狄軍,大足元年復更名。其西有天安軍,天寶十二載置;又有代北軍,永泰元年置。雁門,上。有東陘關、西陘關。五臺,中。柏谷有銀,有銅,有鐵;有五臺山。繁畤,中。崞,中。有石門關。唐林。中。本武延,證聖元年析五臺、崞置,唐隆元年更名。



 雲州雲中郡,下都督府。貞觀十四年自朔州北定襄城徙治定襄縣。永淳元年為默啜所破,徙其民於朔州。開元十八年復置。土貢:犛牛尾、雕羽。戶三千一百六十九,口七千九百三十。縣一:有雲中、樓煩二守捉;城東有牛皮關。雲中。中。本馬邑郡云內之恆安鎮。武德六年置北恆州,七年廢。貞觀十四年復置,曰定襄縣。永淳元年廢。開元十八年復置,更名。有陰山道、青坡道,皆出兵路。



 朔州馬邑郡,下。本治善陽,建中中,節度使馬燧徙治馬邑,後復故治。土貢:白雕羽、豹尾、甘草。戶五千四百九十三,口二萬四千五百三十三。縣二:善陽,中。武德四年省常寧縣入焉。馬邑。中。開元五年析善陽於大同軍城置。



 蔚州興唐郡,下。本安邊郡。隋雁門郡之靈丘、上谷郡之飛狐縣地。唐初沒突厥。武德六年置州,並置靈丘、飛狐二縣,僑治陽曲。七年僑治繁畤。八年僑治秀容故北恆州城。貞觀五年破突厥,復故地,還治靈丘。開元初徙治安邊。至德二載更郡名,復故治。土貢:熊郭、豹尾、松實。戶五千五十二,口二萬九百五十八。縣三:東北有橫野軍。乾元元年徙天成軍合之,而廢橫野軍,西有清塞軍,本清塞守捉城,貞元十五年置。靈丘,中。有直谷關;其北有孔嶺關,有太安鎮。飛狐,中。初僑治易州之遂城,遙隸蔚州,貞觀五年復故地。有三河銅冶,有錢官。興唐。中。本安邊,開元十二年置,治橫野軍,至德二載更名。



 武州。闕。領縣一:文德。



 新州。闕。領縣四:永興,礬山,龍門,懷安。



 潞州上黨郡,大都督府。土貢:貲布、人、石蜜、墨。戶六萬八千三百九十一,口三十八萬八千六百六十一。縣十:有府一,曰戡黎。上黨,望。有啟聖宮,本飛龍,玄宗故第,開元十一年置,後又更名。有瑞閣,有五龍山、馬駒山。壺關,上。武德四年析上黨置。長子,緊。屯留,上。有三嵕山。潞城,上。天祐二年更曰潞子。襄垣,上。武德元年以襄垣、黎城、涉、銅鞮、鄉置韓州,貞觀十七年州廢,縣皆來屬。東有井谷故關。黎城,上。天祐二年更曰黎亭。有銅山;東有壺口故關。涉,中。有鐵。銅鞮,上。武德三年析置甲水縣,隸韓州,九年省。永徽六年隸沁州。顯慶四年來屬。武鄉。中。本鄉,武後更名武鄉,神龍元年復故名,尋又曰武鄉。北有昂車關。



 澤州高平郡,上。本長平郡,治濩澤,武德八年徙治端氏,貞觀元年徙治晉城,天寶元年更郡名。土貢:人、石英、野雞。戶二萬七千八百二十二,口十五萬七千九十。縣六:有府五,曰丹川、永固、安平、沁水、白澗。晉城,上。本丹川,武德元年置建州。三年析丹川置晉城以隸之。六年州廢,隸蓋州,徙蓋州來治。九年省丹川、蓋城入晉城。貞觀元年州廢,以晉城、高平、陵川來屬。天祐二年更曰丹川。南有天井關,一名太行關。端氏,中。有隗山。陵川,中。陽城,中。本濩澤,天寶元年更名,天祐二年更曰濩澤。有銅,有錫,有鐵。沁水,中。高平。上。本隋長平郡,武德元年曰蓋州,領高平、丹川、陵川三縣,並析置蓋城縣以隸之。有泫水,一曰丹水,貞元元年,令明濟引入城,號甘泉;有省冤谷,本殺穀,玄宗幸潞州,過之,因更名;北有長平關。



 右河東採訪使,治蒲州。



 河北道,蓋古幽、冀二州之境,漢河內、魏、渤海、清河、平原、常山、上谷、涿、漁陽、右北平、遼西、真定、中山、信都、河間、廣陽等郡國,又參有東郡、河東、上黨、鉅鹿之地。孟、懷、澶、衛及魏、博、相之南境為娵訾分,邢、洺、惠、貝、冀、深、趙、鎮、定及魏、博、相之北境為大梁分,滄、景、德為玄枵分,瀛、莫、幽、易、涿、平、媯、檀、薊、營、安東為析木津分。為州二十九,都護府一,縣百七十四。其名山:林慮、白鹿、封龍、井陘、碣石、常嶽。其大川:漳、淇、呼陀。厥賦:絲、絹、綿。厥貢:羅、綾、紬、紗、鳳翮葦席。



 孟州,望。建中二年,以河南府之河陽、河清、濟源、溫租賦入河陽三城使,又以汜水租賦益之。會昌三年遂以五縣為州。土貢:黃魚差。縣五:有河陽軍,建中四年置。河陽,望。武德四年,析懷州之河陽、集城、溫於河陽宮置盟州。八年州廢,省集城入河陽,溫隸懷州。顯慶二年隸洛州。有河陽關。有回洛故城。有池,永徽四年引濟水漲之,開元中以畜黃魚。汜水,望。本隸鄭州,武德四年析置成皋縣,貞觀元年省,顯慶二年隸洛州,垂拱四年曰廣武,神龍元年復故名。有虎牢關,東南有成皋故關,西南有旋門故關;有牛口渚;西一里伏龜山有昭武廟,會昌五年置。河陰,望。開元二十二年析汜水、滎澤、武陟置,隸河南府,領河陰倉,會昌三年來屬。有梁公堰,在河、汴間,開元二年,河南尹李傑因故渠浚之,以便漕運。溫,望。武德四年,隋令周仲隱以縣去王世充來降,置平州,名縣城曰李城;是年州廢,隸懷州。顯慶二年隸洛州。濟源。望。武德二年,王世充將丁伯德以縣來降,置西濟州,又析置溴陽、蒸川、邵原三縣。四年州廢,省溴陽、蒸川、邵原,以濟源隸懷州。貞觀元年省懷州之軹縣入焉。顯慶二年隸洛州。有枋口堰,大和五年,節度使溫造浚古渠,溉濟源、河內、溫、武陟田五千頃;有濟瀆祠、北海祠;西有故軹關。



 懷州河內郡,雄。武德二年沒王世充,僑治濟源之柏崖城。四年,世充平,還舊治。土貢:「平紗、平紬、枳殼、茶、牛膝。戶五萬五千三百四十九,口三十一萬八千一百二十六。縣五:有府二,曰丹水、吳澤。河內,望。武德三年析置太行、忠義、紫陵三縣,析河陽置穀旦縣。四年皆省。有太行山;有丹水,開元十一年更名懷水。武德,望。本安昌,武德二年更名,是年,置北義州。四年州廢,來屬。北百里有大斛故關在太行山。獲嘉,望。武德四年以獲嘉、武陟、脩武、新鄉、共城置殷州,並置博望縣。貞觀元年州廢,以獲嘉、武陟、脩武來屬,新鄉、共城、博望隸衛州。武陟,望。貞觀元年省懷縣入焉。脩武。緊。武德二年,河內民李厚德以濁鹿城來降,置陟州,並置脩武縣。四年徙縣治故脩武,更脩武曰武陟,別置脩武縣;是年州廢,隸殷州。西北二十里有新河,自六真山下合黃丹泉水南流入吳澤陂,大中年,令杜某開。



 魏州魏郡,大都督府,雄。本武陽郡,龍朔二年更名冀州,咸亨三年復曰魏州,天寶元年更郡名。土貢:花紬、綿紬、平紬、施、絹、紫草。戶十五萬一千五百九十六,口百一十萬九千八百七十三。縣十四:貴鄉,望。有西渠,開元二十八年,刺史盧暉徙永濟渠,自石灰窠引流至城西,注魏橋,以通江、淮之貨。元城,望。貞觀十七年省入貴鄉,聖歷二年復置。魏,望。武德四年置漳陰縣,貞觀元年省入焉。館陶,望。武德五年,以館陶、冠氏及博州之堂邑,貝州之臨清、清水置毛州,並析臨清置沙丘縣。貞觀元年州廢,省清水入冠氏,省沙丘入臨清,餘縣皆還故屬。冠氏,望。莘,上。武德五年以莘、臨黃、武陽、博州之武水置莘州,貞觀元年州廢,縣還故屬。朝城,緊。本武陽,貞觀十七年省入臨黃、莘。永昌元年復置,曰武聖。開元七年更名。元和中隸澶州,後復來屬。天祐三年更曰武陽,又以武陽、莘河外地入鄆州。昌樂,望。武德五年置,貞觀十八年省繁水入焉。臨河,上。武德二年隸黎州,貞觀十七年省澶水縣入焉。澶水,本澶淵,避高祖名更。州廢,隸相州,天祐三年來屬。洹水,上。本隸相州,天祐三年來屬。成安,上。本隸相州,天祐二年更名斥丘,三年來屬。內黃,緊。本隸相州,武德四年析置繁陽縣,隸黎州,貞觀元年省,天祐三年來屬。宗城,望。本隸貝州,武德四年,以宗城、經城及冀州之南宮、斌強置宗州,析經城置府城縣。九年州廢,省府城入經城,省斌強入清河,餘縣皆還故屬。天祐三年曰廣宗,是歲來屬。永濟。上。本隸貝州,大歷七年,田承嗣析魏州之臨清置。天祐三年來屬。



 博州博平郡,上。武德四年以魏州之聊城、武水、堂邑、高唐置。土貢:綾、平紬。戶五萬二千六百三十一,口四十萬八千二百五十二。縣六:卿城,緊。武德四年析置茌平縣,又析魏州之華置華亭縣。貞觀元年皆省。天祐三年更曰聊邑,又以聊邑、博平、高唐、武水之河外地入鄆州。東南有四口故關。博平,上。武德三年析置靈泉縣,四年省。貞觀十七年省博平入聊城,天授二年復置。武水,上。清平,上。武德四年置。堂邑,上。高唐。上。長壽二年曰崇武,神龍元年復故名。



 相州鄴郡,望。本魏郡,天寶元年更名。土貢:紗、絹、隔布、鳳翮席、花口瓢、知母、胡粉。戶十萬一千一百四十二,口五十九萬一百九十六。縣六:有昭義軍,大歷元年置。安陽,緊。武德四年省零泉縣,五年省相縣入焉。西二十里有高平渠,刺史李景引安陽水東流溉田,入廣潤陂,咸亨三年開。鄴,緊。南五里有金鳳渠,引天平渠下流溉田,咸亨三年開;有鐵。湯陰,上。本蕩陰。武德四年析安陽置蕩源縣,隸衛州,六年來屬。貞觀元年更蕩源曰湯陰。林慮,上。武德二年以縣置巖州,五年州廢,來屬。有鐵,有林慮山。堯城,上。天祐三年更曰永定。北四十五里有萬金渠,引漳水入故齊都領渠以溉田,咸亨三年開。臨漳。上。南有菊花渠,自鄴引天平渠水溉田,屈曲經三十里;又北三十里有利物渠,自滏陽下入成安,並取天平渠水以溉田。皆咸亨四年令李仁綽開。



 衛州汲郡,望。本治衛,貞觀元年徙治汲。土貢:綾、絹、綿、胡粉。戶四萬八千五十六,口二十八萬四千六百三十。縣五:汲,緊。武德元年以汲、新鄉置義州。四年州廢,以汲來屬,新鄉隸殷州。衛,緊。貞觀十七年省清淇縣入焉。長安三年復置清淇縣。神龍元年又省。御水有石堰一,貞觀十七年築;有蘇門山。共城。上。武德元年以縣置共州,並析置凡城縣。四年州廢,省凡城,以共城隸殷州。六年省博望縣入焉。有白鹿山。新鄉,望。東北有故臨清關,東南有故延津關。黎陽。上。武德二年以縣置黎州,尋沒竇建德。四年,建德平,復以黎陽、臨河、內黃、澶水,魏州觀城、頓丘,相州之蕩源置;是年,以頓丘、觀城還隸魏州,蕩源還隸相州。貞觀十七年州廢,省澶水,以黎陽來屬,內黃、臨河隸相州。有白馬津,一名黎陽關;有大岯山,一名黎陽山;有新河,元和八年,觀察使田弘正及鄭滑節度使薛平開,長十四里,闊六十步,深丈有七尺,決河注故道,滑州遂無水患。



 貝州清河郡,望。本治清河,武德六年徙治歷亭,八年復故治。土貢:絹、氈、覆鞍氈。戶十萬一十五,口八十三萬四千七百五十七。縣八:清河,緊。清陽,緊。武德四年析置夏津縣,九年省。武城,上。經城,望。西南四十里有張甲河,神龍三年,姜師度因故瀆開。臨清,望。大歷七年隸瀛州,貞元末來屬。漳南,上。歷亭,上。夏津。上。本鄃,天寶元年更名。



 澶州,上。武德四年析黎州之澶水,魏州之頓丘、觀城置。貞觀元年州廢,縣還故屬。大歷七年,田承嗣表以魏州之頓丘、臨黃復置。土貢:角弓、鳳翮席、胡粉。縣四:頓丘,望。清豐,上。大歷七年析頓丘、昌樂置,以孝子張清豐名。觀城,緊。貞觀十七年省入昌樂、臨黃,大歷七年復置。臨黃。緊。東南有盧津關,一名高陵津。



 邢州鉅鹿郡,上。本襄國郡,天寶元年更名。土貢:絲布、磁器、刀、文石。戶七萬一百八十九,口三十八萬二千七百九十八。縣八:龍岡,上。武德元年析龍岡、內丘置青山縣,開成五年省入焉。沙河,上。武德元年置溫州,四年州廢來屬。有鐵。南和,緊。武德元年置和州,四年州廢來屬。鉅鹿,上。武德元年置起州,並析置白起縣。四年州廢,省白起,以鉅鹿隸趙州。貞觀元年來屬。有大陸澤;有咸泉,煮而成鹽。平鄉,上。武德元年置封州,四年州廢,來屬。貞元中,刺史元誼徙漳水,自州東二十里出,至鉅鹿北十里入故河。任,中。武德四年置。堯山,上。本柏仁。武德元年置東龍州,四年州廢,隸趙州,五年來屬。天寶元年更名。內丘。上。武德四年隸趙州,五年來屬。有鐵。



 洺州廣平郡,望。本武安郡,天寶元年更名。土貢:施、綿、紬、油衣。戶九萬一千六百六十六,口六十八萬三千二百八十。縣六:永年,望。平恩,上。臨洺,緊。武德元年以臨洺、武安、肥鄉、邯鄲置紫州,四年州廢,縣皆隸磁州,六年以臨洺、肥鄉來屬。狗山有太宗故壘,討劉黑闥於此。雞澤,上。武德四年置。有普樂縣,武德初置,後陷竇建德,遂廢。有漳、水名南堤二,沙河南堤一,永徽五年築。肥鄉,上。州又領清漳、池水二縣。會昌三年省清漳入肥鄉,池水入曲周。曲周。上。武德四年置。



 惠州,上。本礠州,武德元年以相州之滏陽、臨水、成安置。貞觀元年州廢,滏陽、成安還隸相州。永泰元年,昭義節度使薛嵩表復以相州之滏陽,洺州之邯鄲、武安置。天祐三年以「礠」「慈」聲一,更名。土貢:紗、礠石。縣四:滏陽,望。邯鄲,上。貞觀元年隸洺州。武安,上。武德六年隸洺州。有錫。昭義。上。本臨水,武德六年省,永泰元年復置,更名。有鐵。



 鎮州常山郡,大都督府。本恆州恆山郡,治石邑,義寧元年析隋高陽郡置。武德四年徙治真定。天寶元年更郡名。十五載曰平山,尋復為恆山。元和十五年避穆宗名更。土貢:孔雀羅、瓜子羅、春羅、梨。戶五萬四千六百三十三,口三十四萬二千一百三十四。縣十一:有恆陽軍,開元中置。真定,望。武德六年析置恆山縣,貞觀元年省。載初元年曰中山,神龍元年復故名。槁城,緊。義寧元年置鉅鹿郡,並析置柏肆、新豐、宜安三縣,武德元年曰廉州。四年,以趙州之鼓城、定州之毋極、冀州之鹿城隸之,省柏肆、新豐、宜安入槁城。貞觀元年州廢,以鹿城隸深州,鼓城、毋極隸定州,槁城來屬。天祐二年更曰槁平。石邑,緊。九門,上。義寧元年置九門郡,並析置新市、信義二縣。武德元年曰觀州,五年州廢,省信義、新市,以九門來屬。行唐,中。武德四年置玉城縣,五年省滋陽縣入焉。長壽二年曰章武,神龍元年復故名。大歷三年以縣置泜州,又以靈壽及定州之恆陽隸之。九年州廢,縣還故屬。井陘,中。義寧元年置井陘郡,又析置葦澤縣。武德元年曰井州,後又領鹿泉及房山、蒲吾、靈壽。貞觀元年省蒲吾入房山,鹿泉、葦澤入井陘。十七年州廢,縣皆來屬。有鐵;有離隔山。平山,中。本房山。義寧元年置房山郡,又置蒲吾縣。武德元年曰岳州,四年州廢,縣皆隸井州。天寶十五載更名。有鐵,有白馬關,有房山。獲鹿,中。本鹿泉,天寶十五載更名。有故井陘關,一名土門關。東北十里有大唐渠,自平山至石邑,引太白渠溉田;有禮教渠,總章二年,自石邑西北引太白渠東流入真定界以溉田;天寶二年,又自石邑引大唐渠東南流四十三里入太白渠;有抱犢山。靈壽,中。義寧元年以縣置燕州,武德四年州廢,隸井州。鼓城,中。本隸定州,大歷三年來屬。欒城。中。本隸趙州,大歷二年來屬,天祐二年更名欒氏。



 冀州信都郡,上。本治信都,武德六年徙治下博,貞觀元年復故治,龍朔二年更名魏州,咸亨三年復故名。土貢:絹、綿。戶十一萬三千八百八十五,口八十三萬五百二十。縣九:信都,望。天祐二年更曰堯都。東二里有葛榮陂,貞觀十一年,刺史李興公開,引趙照渠水以注之。南宮,望。西五十九里有濁漳堤,顯慶元年築;有通利渠,延載元年開。堂陽,上。西南三十里有渠,自鉅鹿入縣境,下入南宮,景龍元年開。西十里有漳水堤,開元六年築。棗強,上。武邑,上。武德四年析置昌亭縣,貞觀元年省。北三十里有衡漳右堤,顯慶元年築。衡水,上。南一里有羊令渠,載初中,令羊元珪引漳水北流,貫城注隍。阜城,望。天祐二年更曰漢阜。蓚,上。本隸德州,永泰元年來屬。武強。望。貞觀元年隸深州,州廢來屬。後復隸深州,開元二年來屬。永泰元年復隸深州,唐末來屬。



 深州饒陽郡,上。武德四年以定州之安平、瀛州之饒陽置,尋徙治饒陽。貞觀十七年州廢,縣還故屬。先天二年,以瀛州之饒陽,冀州之鹿城、下博、武強,定州之安平復置。土貢:絹。戶萬八千八百二十五,口三十四萬六千四百七十二。縣七:陸澤,上。先天二年析饒陽、鹿城置。饒陽,望。武德四年析置無蔞縣,貞觀元年省。束鹿,上。本鹿城,天寶十五載更名。安平,上。博野,望。本隸蒲州。武德五年以博野、清苑、定州之義豐置蠡州,八年州廢,縣還故屬,九年復以博野、清苑置。貞觀元年州廢,以博野、清苑隸瀛州。永泰中以博野來屬。元和十年復隸瀛州,後又來屬。樂壽,緊。本隸瀛州,大歷中來屬,元和十年復隸瀛州,後又來屬。下博。上。本隸冀州,貞觀元年來屬。州廢,還隸冀州。後又來屬。開元二年隸冀州。永泰元年復來屬。有永寧軍,貞元十年置。



 趙州趙郡,望。武德初治柏鄉,四年徙治平棘,五年更名欒城,貞觀初復故名。土貢:絹。戶六萬三千四百五十四,口三十九萬五千二百三十八。縣八:平棘,上。東二里有廣潤陂,引太白渠以注之,東南二十里有畢泓,皆永徽五年令弓志元開,以畜洩水利。寧晉,緊。本癭陶,天寶元年更名。地旱鹵。西南有新渠,上元中,令程處默引洨水入城以溉田,經十餘里,地用豐潤,民食乃甘。昭慶,望。本大陸,武德四年曰象城,天寶元年更名。西南二十里有建初陵、啟運陵,二陵共塋。城下有澧水渠,儀鳳三年,令李玄開,以溉田通漕。柏鄉,上。西有千金渠、萬金堰,開元中,令王佐所浚築,以疏積潦。高邑,中。臨城,中。本房子,天寶元年更名,天祐二年更曰房子。贊皇,中。元氏。上。有靈山、封龍山。



 滄州景城郡,上。本渤海郡,治清池,武德元年徙治饒安,六年徙治胡蘇,貞觀元年復治清池。土貢:絲布、柳箱、葦簟、糖蟹、鱧鮬。戶十二萬四千二十四,口八十二萬五千七百五。縣七:西南有橫海軍,開元十四年置,天寶後廢,大歷元年復置。清池,緊。西北五十五里有永濟堤二,永徽二年築;西四十五里有明溝河堤二,西五十里有李彪澱東堤及徒駭河西堤,皆三年築;西四十里有衡漳堤二,顯慶元年築;西北六十里有衡漳東堤,開元十年築;東南二十里有渠,注毛氏河,東南七十里有渠,注漳,並引浮水,皆刺史姜師度開;西南五十七里有無棣河,東南十五里有陽通河,皆開元十六年開;南十五里有浮河堤、陽通河堤,又南三十里有永濟北堤,亦是年築。有甘泉二,十年,令毛某母老,苦水咸無以養,縣舍穿地,泉湧而甘,民謂之毛公井;有鹽。鹽山,緊。武德四年置東鹽州,五年,以景州之清池並析鹽山置浮水縣以隸之。貞觀元年州廢,省浮水,以清池、鹽山來屬。有鹽。長蘆,上。本隸瀛州。武德四年,以長蘆、平舒、魯城及滄州之清池置景州。貞觀元年州廢,以平舒還隸瀛州,長蘆、魯城來屬。樂陵,上。本隸棣州,武德八年來屬,大和二年又隸棣州,尋復來屬。饒安,上。武德四年析置鬲津縣,貞觀元年省入樂陵。無棣,上。貞觀元年省入陽信,八年復置,大和二年隸棣州,尋來屬。有無棣溝通海,隋末廢,永徽元年,刺史薛大鼎開。乾符。上。本魯城,乾符元年生野稻水穀二千餘頃,燕、魏饑民就食之,因更名。



 景州,上。貞元三年析滄州之弓高、東光、臨津置。長慶元年州廢,縣還滄州,二年復以弓高、東光、臨津、南皮、景城置。大和四年,州又廢,縣還滄州。景福元年復置。土貢:葦簟。縣四:弓高,上。本隸德州,武德四年,以弓高及胡蘇、東光,冀州之阜城、、安陵、觀津置觀州,並析東光置安陵縣,析置觀津縣。六年以胡蘇隸滄州。貞觀元年省觀津,復以胡蘇隸觀州。十七年州廢,以弓高東光、胡蘇隸滄州,、安陵隸德州,阜城還隸冀州。東光,上。南二十里有靳河,自安陵入浮河,開元中開。臨津,上。本胡蘇,天寶元年更名。南皮。上。古毛河,自臨津經縣入清池,開元十年開。有唐昌軍,貞元二十一年置。



 德州平原郡,上。土貢:絹、綾。戶八萬三千二百一十一,口六十五萬九千八百五十五。縣六:安德,緊。長河,上。東南有張公故關。平原,上。大和二年隸齊州,三年來屬。平昌,上。貞觀十七年省般縣入焉,大和二年隸齊州,三年來屬。有馬頰河,久視元年開,號「新河」。將陵,望。安陵,望。景福元年隸景州,尋復來屬。



 定州博陵郡,上。本高陽郡,天寶元年更名。土貢:羅、紬、細綾、瑞綾、兩窠綾、獨窠綾、二包綾、熟線綾。戶七萬八千九十,口四十九萬六千六百七十六。縣十:有義武軍,建中四年置。西有北平軍,開元中置。安喜,緊。本鮮虞,武德四年更名。義豐,緊。萬歲通天二年以拒契丹更名立節,神龍元年復故名。北平,上。萬歲通天二年以拒契丹更名徇忠,神龍元年復故名。西北有安陽故關。望都,上。武德四年置。曲陽,上。本恆陽,元和十五年更名,是年,又更恆岳曰鎮岳,有嶽祠。陘邑,中。本隋昌,武德四年曰唐昌,天寶元年更名。唐,上。有銅,有鐵;西北有八度故關、倒馬故關,北有委粟故關。新樂,中。東南二十里有木刀溝,有民木刀居溝傍,因名之。無極,上。「無」本作「毋」,萬歲通天二年更。有無極山。景福二年,節度使王處存以縣及深澤表置祁州。深澤。中。



 易州上谷郡,上。土貢:紬、綿、墨。戶四萬四千二百三十,口二十五萬八千七百七十九。縣六:有府九,曰遂城、安義、脩武、德行、新安、古亭、武遂、長樂、龍水。有高陽軍。易,上。容城,上。本遒。武德五年,以容城及幽州之固安、歸義置北義州。貞觀元年州廢,縣還故屬。聖歷二年以拒契丹更名全忠,神龍三年復故名,天寶元年又更名。遂城,上。淶水,上。滿城,中。本永樂,天寶元年更名。有郎山。有永清軍,貞元十五年置。五回。中下。開元二十三年析易置,並置樓亭、板城二縣。天寶後省。



 幽州範陽郡,大都督府。本涿郡,天寶元年更名。土貢:綾、綿、絹、角弓、人、慄。戶六萬七千二百四十三,口三十七萬一千三百一十二。縣九:有府十四,曰呂平、涿城、德聞、潞城、樂上、清化、洪源、良鄉、開福、政和、停驂、柘河、良杜、咸寧。城內有經略軍,又有納降軍,本納降守捉城,故丁零川也。西南有安塞軍,有赫連城。有宗王、乾澗、殄寇三鎮城,召堆、車坊、蒿城、河旁四戍。薊,望。天寶元年析置廣寧縣,三載省。有鐵;有故隋臨朔宮。幽都,望。本薊縣地。隋於營州之境汝羅故城置遼西郡,以處粟末靺鞨降人。武德元年曰燕州,領縣三:遼西、瀘河、懷遠。土貢:豹尾。是年,省滬河。六年自營州遷於幽州城中,以首領世襲刺史。貞觀元年省懷遠。開元二十五年徙治幽州北桃谷山。天寶元年曰歸德郡。戶二千四十五,口萬一千六百三。建中二年為硃滔所滅,因廢為縣。廣平,上。天寶元年析薊置,三載省,至德後復置。潞,上。武德二年自無終徙漁陽郡於此,置玄州,領潞、漁陽,並置臨溝縣。貞觀元年州廢,省臨溝、無終,以潞、漁陽來屬。武清,上。本雍奴,天寶元年更名。永清,緊。本武隆,如意元年析安次置,景雲元年曰會昌,天寶元年更名。安次,上。良鄉,望。聖歷元年曰固節,神龍元年復故名,有大防山。昌平。望。北十五里有軍都陘;西北三十五里有納款關,即居庸故關,亦謂之軍都關;其北有防御軍,古夏陽川也;有狼山。



 涿州,上。大歷四年,節度使硃希彩表析幽州之範陽、歸義、固安置。縣五:範陽,望。本涿,武德七年更名。歸義,上。武德五年置,貞觀元年省,八年復置。景雲二年隸鄚州,是年,還隸幽州。固安,上。新昌,上。大歷四年析固安置。新城。上。大和六年以故督亢地置。



 瀛州河間郡,上。土貢:絹。戶九萬八千一十八,口六十六萬三千一百七十一。縣五:河間,望。武德五年置武垣縣,貞觀元年省入焉。西北百里有長豐渠,二十一年,刺史硃潭開。又西南五里有長豐渠,開元二十五年,刺史盧暉自東城、平舒引滹沱東入淇通漕,溉田五百餘頃。高陽,上。武德四年以高陽、鄚、博野、清苑置滿州。五年以博野、清苑隸蠡州。貞觀元年州廢,以鄚、高陽來屬。平舒,上。束城,上。景城。上。本隸滄州,武德四年來屬,貞觀元年隸滄州,大歷七年復舊。後隸景州,尋又來屬。



 莫州文安郡,上。本鄚州,景雲二年,以瀛州之鄚、任丘、文安、清苑、唐興,幽州之歸義置。開元十三年以「鄚」「鄭」文相類,更名。土貢:絹、綿。戶五萬三千四百九十三,口三十三萬九千九百七十二。縣六:有唐興軍,開元十四年置;北又有渤海軍。莫,緊。本鄚,開元十三年更。有九十九澱。清苑,上。文安,上。貞觀元年省豐利縣入焉。任丘,上。武德五年分鄚置。有通利渠,開元四年,令魚思賢開,以洩陂澱,自縣南五里至城西北入滱,得地二百餘頃。長豐,中。本利豐,開元十年析文安、任丘置,是年更名。唐興。上。本武昌,如意元年析河間置。長安四年隸易州,是年,還隸瀛州。神龍元年更名。



 平州北平郡,下。初治臨渝,武德元年徙治盧龍。土貢:熊郭、蔓荊實、人。戶三千一百一十三,口二萬五千八十六。縣三:有府一,曰盧龍。有盧龍軍,天寶二載置;又有柳城軍,永泰元年置;有溫溝、白望、西狹石、東狹石、綠疇、米磚、長楊、黃花、紫蒙、白狼、昌黎、遼西等十二戍,愛川、周夔二鎮城;東北有明垤關、鶻湖城、牛毛城。盧龍,中。本肥如,武德二年更名,又置撫寧縣,七年省。石城,中。本臨渝,武德七年省,貞觀十五年復置,萬歲通天二年更名。有臨渝關,一名臨閭關;有大海關。有碣石山;有溫昌鎮。馬城。中。古海陽城也,開元二十八年置,以通水運。東北有千金冶;城東有茂鄉鎮城。



 媯州媯川郡,上。本北燕州,武德七年平高開道,以幽州之懷戎置。貞觀八年更名。土貢:樺皮、胡祿、甲榆、矢、麝香。戶二千二百六十三,口萬一千五百八十四。縣一:有府二,曰密雲、白檀。有清夷軍,垂拱中置;有堆北、白陽度、雲治、廣邊四鎮兵;有橫河、柴城二戍。有陽門城;有永定、窯子子二關。又有懷柔軍,在媯、蔚二州之境。懷戎。上。天寶中析置媯川縣,尋省。媯水貫中。北九十里有長城,開元中張說築;東南五十里有居庸塞,東連盧龍、碣石,西屬太行、常山,實天下之險;有鐵門關。西有寧武軍;又北有廣邊軍,故白雲城也。



 檀州密雲郡,本安樂郡,天寶元年更名。土貢:人、麝香。戶六千六十四,口三萬二百四十六。縣二:有威武軍,萬歲通天元年置,本漁陽,開元十九年更名;又有鎮遠軍,故黑城川也。有三叉城、橫山城、米城;有大王、北來、保要、鹿固、赤城、邀虜、石子七鎮;有臨河、黃崖二戍。密雲,中。有隗山。燕樂。中。東北百八十五里有東軍、北口二守捉。北口,長城口也。又北八百里有吐護真河,奚王衙帳也。



 薊州漁陽郡,下。開元十八年析幽州置。土貢:白膠。戶五千三百一十七,口萬八千五百二十一。縣三:有府二,曰漁陽、臨渠。南二百里有靜塞軍,本障塞軍,開元十九年更名;又有雄武軍,故廣漢川也;東北九十里有洪水守捉,又東北三十里有鹽城守捉,又東北渡灤河有古盧龍鎮,又有鬥陘鎮;自古盧龍北經九荊嶺、受米城、張洪隘度石嶺至奚王帳六百里;又東北行傍吐護真河五百里至奚、契丹衙帳;又北百里至室韋帳。漁陽,中。神龍元年隸營州,開元四年還隸幽州。有平虜渠傍海穿漕,以避海難,又其北漲水為溝,以拒契丹,皆神龍中滄州刺史姜師度開。三河,中。開元四年析潞置。北十二里有渠河塘。西北六十里有孤山陂,溉田三千頃。玉田。中。本無終,武德二年置,貞觀元年省,乾封二年復置,萬歲通天元年更名,神龍元年隸營州,開元四年還隸幽州,八年隸營州,十一年又隸幽州。有壕門、米亭、三穀、礓石、方公、白楊等七戍。



 營州柳城郡,上都督府。本遼西郡,萬歲通天元年為契丹所陷,聖歷二年僑治漁陽,開元五年又還治柳城,天寶元年更名。土貢:人、麝香、豹尾、皮骨。戶九百九十七,口三千七百八十九。縣一:有平盧軍,開元初置;東有鎮安軍,本燕郡守捉城,貞元二年為軍城;西四百八十里有渝關守捉城;又有汝羅、懷遠、巫閭、襄平四守捉城。柳城。中。西北接奚,北接契丹。有東北鎮醫巫閭山祠,又東有碣石山。



 安東,上都護府。總章元年,李勣平高麗國,得城百七十六,分其地為都督府九,州四十二,縣一百,置安東都護府於平壤城以統之,用其酋渠為都督、刺史、縣令。上元三年徙遼東郡故城,儀鳳二年又徙新城。聖歷元年更名安東都督府,神龍元年復故名。開元二年徙於平州,天寶二年又徙於遼西故郡城。至德後廢。土貢:人。有安東守捉。有懷遠軍,天寶二載置;又有保定軍。



 右河北採訪使,治魏州。



\end{pinyinscope}