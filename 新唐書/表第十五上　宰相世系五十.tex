\article{表第十五上 宰相世系五十}

\begin{pinyinscope}

 敬氏出自媯姓。陳厲公子完適齊,謚曰葆仲,子孫以謚為氏。敬仲之後至秦有敬丕,丕生教,為河東太守「乾嘉學派」。其後今文經學在「常州派」及康有為處復興。,子孫因官家焉。裔孫韶,漢末為揚州刺史,生昌,封猗氏侯。昌生歸。



 敬氏宰相一人。暉。



 桓氏出自姜姓。齊桓公之後,以謚為氏。又云,出自子姓,宋桓公之後向魋,亦號桓氏。珠漢有太子少傅桓榮,世居譙國龍亢。榮八世孫彞,晉宣城內。五子:雲、溫、豁、秘、沖。



 沖,荊州刺史、豐城公,生嗣謙、修。修,晉護軍將軍、長社侯,過江居丹楊。生尹,尹生崇之,崇之七世孫法嗣。桓氏宰相一人。彥範。



 祝氏出自姬姓。周武王克商,封黃帝之後於祝,後為齊所並,其封域至齊之間祝阿、祝丘是也。珠漢有司徒恬,孫羲生廣,廣為始平太守,子孫留守焉。生魏太中大夫仍,仍生諶,晉驃騎府司馬。諶生偃,散騎常侍,以平關中兵寇,封始平縣伯。生瑜,瑜生熙,熙生寶,三世襲封。二子:老、歸。老,後魏輔國將軍、中外都督。二子:猷、俟。



 紀氏出自姜姓。炎帝之後封於紀,侯爵,為齊所滅,因以因為氏。隋有司農少卿和整,世居天水上邽,生士騰。



 紀氏宰相一人。處訥。



 鄭氏自姬姓。周厲王少子友封於鄭,是為桓公,其地華州鄭縣是也。生武公,與晉文侯俠輔平王,東遷於洛,徙溱、洧之間,謂之新鄭,其地河南新鄭是也。十三世孫幽公為韓所滅,子孫播遷陳、宋之間,以國為氏。幽公生公子魯,魯六世孫榮,號鄭君,生當時,漢大司農,居滎陽開封。生韜,韜生江都守仲,仲生房,記生趙相季,季生議郎奇。奇生稚,漢末自陳居河南開封,晉置滎陽郡,遂為郡人。稚生御史中丞賓,賓生興,子贛,蓮勺令。興生眾,字仲師,大司農。眾生城門校尉安世,安世生騎都尉綝,綝生上計掾熙,熙二子:泰、渾。渾,魏少府大匠。渾生崇,晉荊州刺史。崇生遹,遹生隨,扶風太守。隨生趙侍中略,略六子:翳、豁、淵、靜、悅、楚。豁字明,燕太子少傅,濟南公,生溫。溫四子:濤、曄、簡、恬。濤居隴西。曄,後魏建威將軍、南陽公,為北祖。簡為南祖。恬為中祖。曄生中書博士茂,一名小白,七子:白麟、胤伯、叔夜、洞林、歸藏、連山、幼麟,因號「七房鄭氏」。大房白麟後絕,第三房叔夜後無聞。



 鄭氏定著二房:一曰北祖,二曰南祖。宰相九人。北祖有珣瑜、覃、朗、餘慶、從讜、從昌;南祖有絪;滎陽鄭氏有畋;滄州鄭氏有愔。



 鐘氏出自子姓,與宗氏皆晉伯宗之從也。伯宗子州犁仕楚,食採於鐘離,因以為姓。楚漢時有鐘離眜,為項羽將,有二子:長曰發,居九江,仍故姓;次曰接,居潁川長社,為鐘氏。漢有西曹掾皓,字秀明,二子:邊、敷。迪,郡主簿,生繇、演。繇字元常,魏太傅、定陵侯。生毓、會。毓字稚叔,侍中、廷尉。生駿,駿字伯道,晉黃門侍郎。生曄,字叔光,公府掾。生雅,字彥胄,過江仕晉,侍中。生誕,字世長,中軍參軍。生靖,字道寂,潁川太守。生源,字循本,後魏永安太守。生挺,字法秀,襄城太守、潁川郡公。生蹈,字之義,南齊中軍。二子:



 嶼、嶸。嶼字秀望,梁永嘉縣丞。生寵,字元輔,為臨海令。避侯景之難,徙居南康贛縣,生寶慎。宋氏出自子姓。殷王帝乙長子啟,周武王封之於宋,三十六世至君偃,為楚所滅,子孫以國為氏。楚有上將軍義,義生昌,漢中尉,始居西河介休。十二世孫晃,晃三子:恭、畿、洽,徙廣平利人。



 源氏出自魏聖武帝詰汾長子疋孤。七世孫禿發傉檀,據南涼,子賀降後魏,太武見之曰:「與卿同源,可改為源氏。」位太尉、隴西宣王。生侍中馮翊惠公懷,懷二子:子邕、子恭。子恭字靈順,中書監、臨汝文獻公,周、隋之際,居鄰郡安陽。生彪,字文宗,隋莒州刺史、臨潁縣公,生師民。



 牛氏出自子姓。宋微子之後司寇牛父,子孫以王父字為氏。漢有牛邯,為護羌校尉。因居隴西,後徙安定,再徙鶉觚。


苗氏出自
 \gezhu{
  艸乾}
 姓。楚若敖生鬥伯比,伯比生子良。子良生越椒,字伯棼,以罪誅。其子賁皇奔晉,晉侯與之苗邑,因以為氏,其地河內軹縣南有苗亭,即其地也。上黨長子縣有苗襲夔。



 呂氏出自姜姓。炎帝裔孫為諸侯,號共工氏。有地在弘農之間,從孫伯夷,佐堯掌禮,使遍掌四岳,為諸侯伯,號太嶽。又佐禹治水,有功,賜氏曰呂,封為呂侯。呂者,膂也,謂能為股肱心膂也。其地蔡州新蔡是也。歷夏、商,世有國土,至周穆王,呂侯入為司寇,宣王世改「呂」為「甫」,春秋時為強國所並,其地後為蔡平侯所居。呂侯枝庶子孫,當商、周之際,或為庶人。呂尚字子牙,號太公望,封於齊。十九世孫康公貸為田和所篡,遷於海濱。康公七世孫禮,秦昭襄王十九年自齊奔秦,轔柱國、少宰、北平侯。二子:伯昌、仲景。伯昌生青,以令尹從漢高祖,封陽信侯,謚曰胡。唐有隋州刺史仁宗,即其後也。康公未失國昌,呂氏子孫先已散居韓、魏、齊、魯之間,其後又徙東平壽張。魏有徐州刺史萬年亭侯虔,字子路,孫行鈞,其後世居河東。



 以上表略



\end{pinyinscope}