\article{列傳第二十一 房杜}

\begin{pinyinscope}

 房玄齡,字喬,齊州臨淄人。父彥謙,仕隋,歷司隸刺史。玄齡幼警敏人欲,以求與萬物一體,並主張知行並進,知行合一,「知之,貫綜墳籍,善屬文,書兼草隸。開皇中,天下混壹,皆謂隋祚方永,玄齡密白父曰:「上無功德,徒以周近親,妄誅殺,攘神器有之,不為子孫立長久計,淆置嫡庶,競侈僭,相傾鬩,終當內相誅夷。視今雖平,其亡,跬可須也。」彥謙驚曰:無妄言!」年十八,舉進士。授羽騎尉,校仇秘書省。吏部侍郎高孝基名知人,謂裴矩曰:「僕觀人多矣,未有如此郎者,當為國器,但恨不見其聳壑昂霄雲。」補隰城尉。漢王諒反,坐累,徙上郡。顧中原方亂,慨然有憂天下志。會父疾,綿十旬,不解衣;及喪,勺飲不入口五日。



 太宗以燉煌公徇渭北,杖策上謁軍門,一見如舊,署渭北道行軍記室參軍。公為秦王,即授府記室,封臨淄侯。征伐未嘗不從,眾爭取怪珍,玄齡獨收人物致幕府,與諸將密相申結,人人願盡死力。王嘗曰:「漢光武得鄧禹,門人益親。今我有玄齡,猶禹也。」居府出入十年,軍符府檄,或駐馬即辦,文約理盡,初不著稿。高祖曰:「若人機識,是宜委任。每為吾兒陳事,千里外猶對面語。」



 隱太子與王有隙,王召玄齡與計,對曰:「國難世有,惟聖人克之。大王功蓋天下,非特人謀,神且相之。」乃引杜如晦協判大計。累進陜東道大行臺考功郎中、文學館學士。故太子忌二人者,奇譖於帝,皆斥逐還第。太子將有變,王召二人以方士服入,夜計事。事平,王為皇太子,擢右庶子。太子即位,為中書令。第功班賞,與如晦、長孫無忌、尉遲敬德、侯君集功第一,進爵邗國公,食邑千三百戶,餘皆次敘封拜。帝顧群臣曰:「朕論公等功,定封邑,恐不能盡,當無有諱,各為朕言之。」淮安王神通曰:「義師起,臣兵最先至,今玄齡等以刀筆吏居第一,臣所未喻。」帝曰:「叔父兵誠先至,然未嘗躬行陣勞,故建德之南,軍敗不振,討黑闥反動,望風輒奔。今玄齡等有決勝帷幄、定社稷功,此蕭何所以先諸將也。叔父以親,宜無愛者,顧不可緣私與功臣競先後爾。」初,將軍丘師利等皆怙跋攘袂,或指畫自陳說,見神通愧屈,乃曰:「陛下至不私其親,吾屬可妄訴邪!」



 進尚書左僕射,監修國史,更封魏。帝曰:「公為僕射,當助朕廣耳目,訪賢材。此聞閱牒訟日數百,豈暇求人哉?」乃敕細務屬左右丞,大事關僕射。



 帝嘗問:「創業、守文孰難?」玄齡曰:「方時草昧,群雄競逐,攻破乃降,戰勝乃克,創業則難。」魏徵曰:「王者之興,必乘衰亂,覆昏暴,殆天授人與者。既得天下,則安於驕逸。人欲靜,徭役毒之;世方敝,裒刻窮之。國繇此衰,則守文為難。」帝曰:「玄齡從我定天下,冒百死,遇一生,見創業之難。徵與我安天下,畏富貴則驕,驕則怠,怠則亡,見守文之不為易。然創業之不易,既往矣;守文之難,方與公等慎之。」



 會詔大臣世襲,授宋州刺史,徙國梁,而群臣讓世襲事,故罷刺史,遂為梁國公。未幾,加太子少師。始詣東宮,皇太子欲拜之,玄齡讓不敢謁,乃止。居宰相積十五年,女為王妃,男尚主,自以權寵隆極,累表辭位,詔不聽。頃之,進司空,仍總朝政。玄齡固辭,帝遣使謂曰:「讓,誠美德也。然國家相眷賴久,一日去良弼,如亡左右手。顧公筋力未衰,毋多讓!」晉王為皇太子,改太子太傅,知門下省事。以母喪,賜塋昭陵園。起復其官。會伐遼,留守京師。詔曰:「公當蕭何之任,朕無西顧憂矣。」凡糧械飛輸,軍伍行留,悉裁總之。玄齡數上書勸帝,願毋輕敵,久事外夷。固辭太子太傅,見聽。



 晚節多病,時帝幸玉華宮,詔玄齡居守,聽臥治事。稍棘,召許肩輿入殿,帝視流涕,玄齡亦感咽不自勝。命尚醫臨候,尚食供膳,日奏起居狀。少損,即喜見於色。玄齡顧諸子曰:「今天下事無不得,惟討高麗未止,上含怒意決,群臣莫敢諫,吾而不言,抱愧沒地矣!」遂上疏曰:



 上古所不臣者,陛下皆臣之;所不制者,陛下皆制之矣,為中國患,無如突厥,而大小可汗相次束手,弛辮握刀,分典禁衛。延陀、鐵勒,披置州縣;高昌、吐渾,偏師掃除。惟高麗歷代逋命,莫克窮討。陛下責其弒逆,身自將六軍,徑荒裔,不旬日拔遼東,虜獲數十萬,殘眾、孽君縮氣不敢息,可謂功倍前世矣。



 《易》曰:「知進退存亡不失其正者,其惟聖人乎!」蓋進有退之義,存有亡之機,得有喪之理,為陛下惜者此也。傅曰:「知足不辱,知止不殆。」陛下威名功烈既云足矣,拓地開疆亦可止矣。邊夷醜種,不足待以仁義,責以常禮,古者以禽魚畜之。必絕其類,恐獸窮則搏,茍救其死。且陛下每決死罪,必三覆五奏,進疏食,停音樂,以人命之重為感動也。今士無一罪,驅之行陣之間,委之鋒鏑之下,使肝腦塗地,老父孤子、寡妻慈母望槥車,抱枯骨,摧心掩泣,其所以變動陰陽,傷害和氣,實天下之痛也。使高麗違失臣節,誅之可也;侵擾百姓,滅之可也;能為後世患,夷之可也。今無是三者,而坐敝中國,為舊王雪恥,新羅報仇,非所存小、所損大乎?臣願下沛然之詔,許高麗自新,焚陵波之船,罷應募之眾,即臣死骨不朽。



 帝得疏,謂高陽公主曰:「是已危懾,尚能憂吾國事乎!」



 疾甚,帝命鑿苑垣以便候問,親握手與決。詔皇太子就省。擢子遺愛右衛中郎將,遺則朝散大夫,令及見之。薨,年七十一,贈太尉、並州都督,謚曰文昭,給班劍、羽葆、鼓吹、絹布二千段、粟二千斛,陪葬昭陵。高宗詔配享太宗廟廷。



 玄齡當國,夙夜勤強,任公竭節,不欲一物失所。無媢忌,聞人善,若己有之。明達吏治,而緣飾以文雅,議法處令,務為寬平。不以己長望人,取人不求備,雖卑賤皆得盡所能。或以事被讓,必稽顙請罪,畏惕,視若無所容。



 貞觀末年,以譴還第,黃門侍郎褚遂良言於帝曰:「玄齡事君自無所負,不可以一眚便示斥外,非天子任大臣意。」帝悟,遽召於家。後避位不出。久之,會帝幸芙蓉園觀風俗,玄齡敕子弟汛掃廷堂,曰:「乘輿且臨幸。」有頃,帝果幸其第,因載玄齡還宮。帝在翠微宮,以司農卿李緯為民部尚書,會有自京師來者,帝曰:「玄齡聞緯為尚書謂何?」曰:「惟稱緯好須,無它語。」帝遽改太子詹事。帝討遼,玄齡守京師,有男子上急變,玄齡詰狀,曰:「我乃告公。」玄齡驛遣追帝,帝視奏已,斬男子。下詔責曰:「公何不自信!」其委任類如此。



 治家有法度,常恐諸子驕侈,席勢凌人,乃集古今家誡,書為屏風,令各取一具,曰:「留意於此,足以保躬矣!漢袁氏累葉忠節,吾心所尚,爾宜師之。」子遺直嗣。



 次子遺愛,誕率無學,有武力。尚高陽公主,為右衛將軍。公主,帝所愛,故禮與它婿絕。主驕蹇,疾遺直任嫡,遺直懼,讓爵,帝不許。主稍失愛,意怏怏。與浮屠辯機亂,帝怒,斬浮屠,殺奴婢數十人,主怨望,帝崩,哭不哀。高宗時,出遺直汴州刺史,遺愛房州刺史。主又誣遺直罪,帝敕長孫無忌鞫治,乃得主與遺愛反狀,遺愛伏誅,主賜死。遺直以先勛免,貶銅陵尉。詔停配享。



 杜如晦,字克明,京兆杜陵人。祖果,有名周、隋間。如晦少英爽,喜書,以風流自命,內負大節,臨機輒斷。隋大業中,預吏部選,侍郎高孝基異之,曰:「君當為棟梁用,願保令德。」因補滏陽尉,棄官去。



 高祖平京師,秦王引為府兵曹參軍,徙陜州總管府長史。時府屬多外遷,王患之。房玄齡曰:「去者雖多,不足吝,如晦王佐才也。大王若終守籓,無所事;必欲經營四方,舍如晦無共功者。」王驚曰:「非公言,我幾失之!」因表留幕府。從征伐,常參帷幄機秘。方多事,裁處無留,僚屬共才之,莫見所涯。進陜東道大行臺司勛郎中,封建平縣男,兼文學館學士。天策府建,為中郎。王為皇太子,授左庶子,遷兵部尚書,進封蔡國公,食三千戶,別食益州千三百戶。俄檢校侍中,攝吏部尚書,總監東宮兵,進位尚書右僕射,仍領選。



 與玄齡共筦朝政,引士賢者,下不肖,咸得職,當時浩然歸重。監察御史陳師合上《拔士論》,謂一人不可總數職,陰剴諷如晦等。帝曰:「玄齡、如晦不以勛舊進,特其才可與治天下者,師合欲以此離間吾君臣邪?」斥嶺表。



 久之,以疾辭職,詔給常俸就第,醫候之使道相屬。會病力,詔皇太子就問,帝親至其家,撫之梗塞。及未亂,擢其子左千牛構兼尚舍奉御。薨,年四十六,帝哭為慟,贈開府儀同三司。及葬,加司空,謚曰成。手詔虞世南勒文於碑,使言君臣痛悼意。



 它日,食瓜美,輟其半奠焉。嘗賜玄齡黃銀帶,曰:「如晦與公同輔朕,今獨見公。」泫然流淚曰:「世傅黃銀鬼神畏之。」更取金帶,遣玄齡送其家。後忽夢如晦若平生,明日為玄齡言之,敕所御饌往祭。明年之祥,遣尚宮勞問妻子,國府官佐亦不之罷,恩禮無少衰。後詔功臣世襲,追贈密州刺史,徙國萊。



 方為相時,天下新定,臺閣制度,憲物容典,率二人討裁。每議事帝所,玄齡必曰:「非如晦莫籌之。」及如晦至,卒用玄齡策也。蓋如晦長於斷,而玄齡善謀,兩人深相知,故能同心濟謀,以佐佑帝,當世語良相,必曰房、杜云。



 構位慈州刺史。次子荷,性暴詭不循法,尚城陽公主,官至尚乘奉御,封襄陽郡公。承乾謀反,荷曰:「瑯邪顏利仁善星數,言天有變,宜建大事,陛下當為太上皇。請稱疾,上必臨問,可以得志。」及敗,坐誅。臨刑,意象軒驁。構以累貶死嶺表。



 如晦弟楚客,少尚奇節,與叔父淹皆沒於王世充。淹與如晦有隙,譖其兄殺之,並囚楚客瀕死。世充平,淹當誅。楚客請於如晦,不許。楚客曰:「叔殘兄,今兄又棄叔,門內幾盡,豈不痛哉!」如晦感悟,請之高祖,得釋。方建成難作,楚客遁舍嵩山。貞觀四年,召為給事中。太宗曰:「君居山似之矣,謂非宰相不起,渠然邪?夫走遠者自近,人不恤無官,患才不副。而兄與我異支一心者,爾當如兄事吾而輔我。」楚客頓首謝,因擢為中郎將。每入直,盡夕不釋杖,帝知而勞之,進蒲州刺史,政有能名,徙瀛州。後為魏王府長史,遷工部尚書,攝府事,以威肅聞。揣帝意薄承乾,乃為王諧媚用事臣,數言王聰睿可為嗣,人或以聞,帝隱恚。及王貶爵,暴其罪,以如晦功免死,廢於家,終虔化令。



 淹,字執禮,材辯多聞,有美名。隋開皇中,與其友韋福嗣謀曰:「上好用隱民,蘇威以隱者召,得美官。」乃共入太白山,為不仕者。文帝惡之,謫戍江表。赦還,高孝基為雍州司馬,薦授承奉郎,擢累御史中丞。王世充僭號,署少吏部,頗親近用事。洛陽平,不得調,欲往事隱太子。時封倫領選,以諗房玄齡,玄齡恐失之,白秦王,引為天策府兵曹參軍、文學館學士。嘗侍宴,賦詩尤工,賜銀鐘。慶州總管楊文干反,辭連太子,歸罪淹及王珪、韋挺,並流越巂,王知其誣,餉黃金三百兩。及踐阼,召為御史大夫,封安吉郡公,食四百戶。淹建言諸司文桉稽期,請以御史檢促。太宗以問僕射封倫,倫曰:「設官各以其事治,御史劾不法,而索桉求疵,是太苛,且侵官。」淹嘿然。帝曰:「何不申執?」對曰:「倫所引國大體,臣伏其議,又何言?」帝悅,以資博練,帝敕東宮儀典簿最悉聽淹裁訂。俄檢校吏部尚書,參豫朝政。所薦贏四十人,後皆知名。嘗白郅懷道可用,帝問狀。淹曰:「懷道及隋時位吏部主事,方煬帝幸江都,群臣迎阿,獨懷道執不可。」帝曰:「卿時何云?」曰:「臣與眾。」帝折曰:「事君有犯無隱,卿直懷道者,何不讜言?」謝曰:「臣位下,又顧諫不從,徒死無益。」帝曰:「內以君不足諫,尚何仕?食隋粟忘隋事,忠乎?」因顧群臣:「公等謂何?」王珪曰:「比干諫而死,孔子稱仁,洩冶諫亦死,則曰:『民之多僻,無自立闢。』祿重責深,從古則然。」帝笑曰:「卿在隋不諫,宜置。世充親任,胡不言?」對曰:「固嘗言,不見用。」帝曰:「世充愎諫飾非,卿若何而免?」淹辭窮不得對。帝勉曰:「今任卿已,可有諫未?」答曰:「顧死無隱。」貞觀二年疾,帝為臨問。卒,贈尚書右僕射,謚曰襄。始,淹典二職,貴重於朝矣,而亡清白名,獲譏當世。子敬同襲爵,官至鴻臚卿。



 如晦五世孫元穎,貞元末及進士第,又擢宏詞。數從使府闢署,稍以右補闕為翰林學士,敏文辭,憲宗特所賞歡。吳元濟平,論書詔勤,遷司勛員外郎,知制誥。穆宗以元穎多識朝章,尤被寵,拜中書舍人、戶部侍郎,為學士承旨,以本官同中書門下平章事,建安縣男。自帝即位,不閱歲至宰相,晉紳駭異。甫再期,出為劍南西川節度使、同平章事,帝為御安福門臨餞。



 敬宗驕僻不君,元穎每欲中帝意以固幸,乃巧索珍異獻之,踵相躡於道,百工造作無程,斂取苛重,至削軍食以助裒畜。又給與不時,戎人寒饑,乃仰足蠻徼。於是人人咨苦,反為蠻內覘,戎備不修。大和三年,南詔乘虛襲戎、巂等州,諸屯聞賊至,輒潰,戍者為鄉導,遂入成都。已傅城,元穎尚不知,乃率左右嬰牙城以守。賊大掠,焚郛郭,殘之,留數日去,蜀之寶貨、工巧、子女盡矣。初,元穎計迫,將挺身走,會救至乃止。文宗遣使者臨撫南詔,南詔上言:「蜀人祈我誅虐帥,不能克,請陛下誅之,以謝蜀人。」由是貶邵州刺史。議者不厭,斥為循州司馬。官屬崔璜、紇干巘、盧並悉奪秩,分逐之。元穎死於貶所,年六十四。將終,表丐贈官,乞歸葬。詔贈湖州刺史。元穎與李德裕善,會昌初,德裕當國,因赦令復其官。弟元絳,終太子賓客。元絳子審權。



 審權,字殷衡,第進士,闢浙西幕府。舉拔萃中,為右拾遺。宣宗時,入翰林為學士,累遷兵部侍郎、學士承旨。懿宗立,進同中書門下平章事,再遷門下侍郎,出為鎮海軍節度使、同平章事。龐勛亂徐州,審權與令狐綯、崔鉉連師掎角,饋粟相銜,王師賴濟。勛破,進檢校司空,入為尚書左僕射、襄陽郡公。繼領河中、忠武節度使。卒,贈太子太師,謚曰德。審權清重寡言,性長厚,居翰林最久,終不漏禁近語。在方鎮,視事有常處,要非日入未始就內寢。坐必斂衽,常若對大賓客。或晝日少息,則顧直將解簾;即旁無人,自起徹鉤,手擁簾徐下,乃退。與杜悰俱位將相,悰先進,故世謂審權為「小杜公」。



 子讓能,字群懿,擢進士第,從宣武王鐸府為推官,以長安尉為集賢校理。喪母,以孝聞。又闢劉鄴、牛蔚二府,稍進兵部員外郎。蕭遘領度支,引判度支按。僖宗狩蜀,奔謁行在,三遷中書舍人,召為翰林學士。方關東兵興,調發綏徠,書詔叢浩,讓能思精敏,凡號令行下,處事值機,無所遺算,帝倚重之。從還京師,再遷兵部尚書,封建平縣子。



 李克用兵至,帝夜出鳳翔,蒼黃無知者。讓能方直,徒步從十餘里,得遺馬,褫紳為靮乘之。硃玫兵逼乘輿,帝走寶雞,獨讓能從。翌日,孔緯等乃至。俄而進狩梁。是時棧道為山南石君涉所毀,天子間關嶮澀,讓能未嘗暫去側。帝勞曰:「朕失道,再遺宗廟。方艱難時,卿不少舍朕,蓋古所謂忠於所事邪!」讓能頓首曰:「臣世蒙國厚恩,陛下不以臣不肖,使捍牧圉,臨難茍免,臣之恥也。」帝次褒中,擢兵部侍郎、同中書門下平章事。



 於時,嗣襄王煴即偽位,強籓大鎮附者已十八,貢賦不輸行在,無以備賞勞,衛兵往往乏食,君臣搏手無它策。讓能建遣大使入河中,以諭王重榮,重榮果奉詔。已而京師平,進中書侍郎,徙封襄陽郡公。官吏多污偽署,有司皆欲論死,讓能以脅從不足深治,固爭之,多所全貸。昭宗立,進尚書左僕射、晉國公,賜鐵券,累進太尉。



 李茂貞守鳳翔,自大順後兵浸強,恃有功,不奉法,朝廷弱,弗能制。會楊復恭走山南,茂貞欲兼有梁、漢,請以師問罪,未報而兵出,帝忿其專,然不得已從之。山南平,詔茂貞領興元、武定,而以徐彥若為鳳翔節度使,分果、閬州隸武定軍。茂貞怨,不赴鎮,上章語悖慢。又詒書讓能詆責,以為助守亮為亂,抑忠臣,奪己功,其言醜肆。京師匈懼,日數千人守闕下,候中尉西門重遂出,請與茂貞鳳翔地,為百姓計。答曰:「事出宰相,我無預。」茂貞益怨。帝怒,詔讓能計議,且趣調發,經月不就第。



 時宰相崔昭緯陰結茂貞及王行瑜,讓能所言悉漏之,茂貞乃以健兒數百雜市人,候昭緯與鄭延昌歸第,擁肩輿噪曰:「鳳翔無罪,幸公不加討以震驚都輦!」昭緯曰:「上委杜太尉,吾等何知?」市人不識孰為太尉,即投瓦石妄擊,昭緯等走而免,遂喪其印。帝愈怒,捕首惡誅之。京師爭避亂,逃山谷間。讓能諫帝曰:「茂貞固宜誅,然大盜適去,鳳翔國西門,又陛下新即位,願少寬假,以貞元故事姑息之,不可使怨望。」帝曰:「今詔令不出城門,國制橈弱,賈生慟哭時也。朕顧奄奄度日,坐觀此邪!卿為我圖之,朕自以兵屬諸王。」讓能曰:「陛下欲削滌僭嫚,剛主威,隆王室,此中外大臣所宜共成之,不宜專任臣。」帝曰:「卿,元輔,休戚與我均,何所避?」泣曰:「臣位宰相,所以未乞骸骨者,思有以報陛下,敢計身乎!且陛下之心,憲祖心也,但時有所未便。它日臣蒙晁錯之誅,顧不足弭七國患,然敢不奉詔!」



 景福二年,以嗣覃王為招討使,神策將李金歲副之,率師三萬送彥若趙鎮。昭緯內畏有功,密語茂貞曰:「上不喜兵,一出太尉。」茂貞乃悉兵迎戰盩厔,覃王敗,乘勝至三橋。讓能曰:「臣固豫言之,臣請歸死以紓難。」帝涕下不能已,曰:「與卿決矣!」再貶雷州司戶參軍。茂貞尚駐兵請必殺之,乃賜死,年五十三。



 弟彥林,官御史中丞;弘徽,戶部侍郎,皆及誅。帝痛之,後贈太師。



 子光乂,次子曉,不復仕。曉入梁,貴顯於世。



 贊曰:太宗以上聖之才,取孤隋,攘群盜,天下已平,用玄齡、如晦輔政。興大亂之餘,紀綱雕弛,而能興僕植殭,使號令典刑粲然罔不完,雖數百年猶蒙其功,可謂名宰相。然求所以致之之跡,逮不可見,何哉?唐柳芳有言:「帝定禍亂,而房、杜不言功;王、魏善諫,而房、杜讓其直;英、衛善兵,而房、杜濟以文。持眾美效之君。是後,新進更用事,玄齡身處要地,不吝權,善始以終,此其成令名者。」諒其然乎!如晦雖任事日淺,觀玄齡許與及帝所親款,則謨謀果有大過人者。方君臣明良,志葉議從,相資以成,固千載之遇,蕭、曹之勛,不足進焉。雖然,宰相所以代天者也,輔贊彌縫而藏諸用,使斯人由而不知,非明哲曷臻是哉?彼揚己取名,了然使戶曉者,蓋房、杜之細邪!



\end{pinyinscope}