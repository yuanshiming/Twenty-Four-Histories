\article{列傳第七十八 段顏}

\begin{pinyinscope}

 段秀實,字成公,本姑臧人。曾祖師濬,仕為隴州刺史,留不歸者、無神論者和空想共產主義者。曾任鄉村神甫。主張物質,更為汧陽人。秀實六歲,母疾病,不勺飲至七日,病間乃肯食,時號「孝童」。及長,沈厚能斷,慨然有濟世意。舉明經,其友易之,秀實曰:「搜章擿句,不足以立功。」乃棄去。



 天寶四載,從安西節度使馬靈詧討護密有功,授安西府別將。靈詧罷,又事高仙芝。仙芝討大食,圍怛邏斯城。會虜救至,仙芝兵卻,士相失。秀實夜聞副將李嗣業聲,識之,因責曰:「憚敵而奔,非勇也;免己陷眾,非仁也。」嗣業慚,乃與秀實收散卒,復成軍,還安西,請秀實為判官。遷隴州大推府果毅。後從封常清討大勃律,次賀薩勞城,與虜戰,勝之。常清逐北,秀實曰:「賊出羸師,餌我也,請大索。」悉得其廋伏,虜師唧。改綏德府折沖都尉。



 肅宗在靈武,詔嗣業以安西兵五千走行在。節度使梁宰欲逗留觀變,嗣業陰然可。秀實責謂曰:「天子方急,臣下乃欲晏然,公常自稱大丈夫,今誠兒女耳。」嗣業因固請宰,遂東師,以秀實為副。嗣業為節度使,而秀實方居父喪,表起為義王友,充節度判官。安慶緒奔鄴,嗣業與諸將圍之,以輜重委河內,署秀實兼懷州長史,知州事,兼留後。時師老財覂,秀實督饋系道,募士市馬以助軍。諸軍戰愁思岡,嗣業中流矢卒,眾推荔非元禮代將其軍。秀實聞之,即遺白孝德書,使發卒護喪送河內,親與將吏迎諸境,傾私財葬之。元禮高其義,奏擢試光祿少卿。俄而元禮為麾下所殺,將佐多死,惟秀實以恩信為士卒所服,皆羅拜不敢害,更推白孝德為節度使。秀實凡佐三府,益知名。



 時吐蕃襲京師,代宗幸陜,勸孝德即日鼓行入援。孝德徙邠寧,署支度營田副使。於是邠寧乏食,乃請屯奉天,仰給畿內。時公廩竭,縣吏不知所出,皆逃去,軍輒散剽,孝德不能制。秀實曰:「使我為軍候,豈至是邪?」司馬王稷言之,遂知奉天行營事。號令嚴壹,軍中畏戢。兵還,孝德薦為涇州刺史,封張掖郡王。



 時郭子儀為副元帥,居蒲,子晞以檢校尚書領行營節度使,屯邠州。士放縱不法,邠人之嗜惡者,納賄竄名伍中,因肆志,吏不得問。白晝群行丐頡於市,有不嗛,輒擊傷市人,椎釜鬲甕盎盈道,至撞害孕婦。孝德不敢劾,秀實自州以狀白府,願計事,至則曰:「天子以生人付公治,公見人被暴害,恬然,且大亂,若何?」孝德曰:「願奉教。」因請曰:「秀實不忍人無寇暴死,亂天子邊事。公誠以為都虞候,能為公已亂。」孝德即檄署付軍。俄而晞士十七人入市取酒,刺酒翁,壞釀器,秀實列卒取之,斷首置槊上,植市門外。一營大噪,盡甲,孝德恐,召秀實曰:「奈何?」秀實曰:「請辭於軍。」乃解佩刀,選老鐍一人持馬,至晞門下。甲者出,秀實笑且入,曰:「殺一老卒,何甲也!吾戴頭來矣。」甲為愕眙。因曉之曰:「尚書固負若屬邪,副元帥固負若屬邪?奈何欲以亂敗郭氏!」晞出,秀實曰:「副元帥功塞天地,當務始終。今尚書恣卒為暴,使亂天子邊,欲誰歸罪?罪且及副元帥。今邠惡子弟以貨竄名軍籍中,殺害人,藉藉如是,幾日不大亂?亂由尚書出。人皆曰:尚書以副元帥故不戢士。然則郭氏功名,其與存者有幾!」晞再拜曰:「公幸教,晞願奉軍以從。」即叱左右皆解甲,令曰:「敢喧者死!」秀實曰:「吾未晡食,請設具。」已食,曰:「吾疾作,願宿門下。」遂臥軍中。晞大駭,戒候卒擊柝衛之。旦,與俱至孝德所,謝不能。邠由是安。



 初,秀實為營田官。涇大將焦令諶取人田自占,給與農,約熟歸其半。是歲大旱,農告無入,令諶曰:「我知入,不知旱也。」責之急,農無以償,往訴秀實。秀實署牒免之,因使人遜諭令諶。令諶怒,召農責曰:「我畏段秀實邪?」以牒置背上,大杖擊二十,輿致廷中。秀實泣曰:「乃我困汝。」即自裂裳裹瘡注藥,賣己馬以代償。淮西將尹少榮頗剛鯁,入罵令諶曰:「汝誠人乎!涇州野如赭,人饑死,而爾必得穀,擊無罪者。段公,仁信大人,惟一馬,賣而市穀入汝,汝取之不恥?凡為人傲天災、犯大人、擊無罪者,尚不愧奴隸邪!」令諶聞,大愧流汗,曰:「吾終不可以見段公。」一夕,自恨死。



 馬璘代孝德,每所咨逮。璘處決不當,固爭之,不從不止。始,璘城涇州,秀實為留後,以勞加御史中丞。大歷三年,遂徙涇州。是軍自四鎮、北庭赴難,征伐數有功,既驟徙,相與出怨言。別將王童之謀作亂,約曰:「聞警鼓而縱。」秀實知之,召鼓人,陽怒失節,戒曰:「每籌盡當報。」因延數刻,盡四鼓而曙。明日,復有告者曰:「夜焚稿積,約救火則亂。」秀實嚴警備。夜中果火發,令軍中曰:「敢救者斬!」童之居外,請入,不許。明日,捕之,並其黨八人斬以徇,曰:「後徙者族!」軍遂遷涇州。於時,食無久儲,郛無居人,朝廷患之,詔璘領鄭、潁二州以佐軍,命秀實為留後。軍不乏資,二州以治。璘嘉其績,奏為行軍司馬,兼都知兵馬使。



 吐蕃寇邊,戰鹽倉,師不利。璘為虜隔,未能還,都將引潰兵先入,秀實讓曰:「兵法:失將,麾下斬。公等忘死,而欲安其家邪!」乃悉城中士,使銳將統之,依東原列奇兵,示賊將戰。虜望之,不敢逼。俄而璘得歸。



 久之,璘有疾,請秀實攝節度副使。秀實按甲備變,璘卒,命願將馬頔主喪,李漢惠主賓客,家人位於堂,宗族位於廷,賓將位於牙內,尉吏士卒位於營次,非其親,不得居喪側。朝夕臨,三日止。有族談離立者,皆捕囚之。都虞候史廷幹、裨將崔珍、張景華欲謀亂,秀實送廷幹京師,徙珍、景華於外,一軍遂安。



 即拜四鎮北庭行軍、涇原鄭潁節度使。數年,吐蕃不敢犯塞。又按格令,官使二料取其一,非公會不舉樂飲酒;室無妓媵,無贏財;賓佐至,議軍政,不及私。十三年來朝,對蓬萊殿,代宗問所以安邊者,畫地以對,件別條陳。帝悅,慰賚良渥,又賜第一區,實封百戶。還之鎮。德宗立,加檢校禮部尚書。建中初,宰相楊炎追元載議,欲城原州,詔中使問狀,秀實言:「方春不可興土功,請須農隙。」炎謂沮己,遂召為司農卿。



 硃泚反,以秀實失兵,必恨憤,且素有人望,使騎往迎。秀實與子弟訣而入,泚喜曰:「公來,吾事成矣。」秀實曰:「將士東征,宴賜不豐,有司過耳,人主何與知?公本以忠義聞天下,今變起倉卒,當諭眾以禍福,掃清宮室,迎乘輿,公之職也。」泚默然。秀實知不可,乃陽與合,陰結將軍劉海賓、姚令言、都虞候何明禮,欲圖泚。三人者,皆秀實素所厚。會源休教泚偽迎天子,遣將韓旻領銳師三千疾馳奉天。秀實以為宗社之危不容喘,乃遣人諭大吏岐靈岳竊取令言印,不獲,乃倒用司農印追其兵。旻至駱驛,得符還。秀實謂海賓曰:「旻之來,吾等無遺類。我當直搏殺賊,不然則死。」乃約事急為繼,而令明禮應於外。翌日,泚召秀實計事,源休、姚令言、李忠臣、李子平皆在坐。秀實戎服與休並語,至僭位,勃然起,執休腕,奪其象笏,奮而前,唾泚面大罵曰:「狂賊!可磔萬段,我豈從汝反邪!」遂擊之。泚舉臂捍笏,中顙,流血衊面,匍匐走。賊眾未敢動,而海賓等無至者。秀實大呼曰:「我不同反,胡不殺我!」遂遇害,年六十五。海賓、明禮、靈岳等皆繼為賊害。帝在奉天,恨用秀實不極才,垂涕悔悵。



 初,秀實自涇州被召,戒其家曰:「若過岐,硃泚必致贈遺,慎毋納。」至岐,泚固致大綾三百,家人拒,不遂。至都,秀實怒曰:「吾終不以污吾第。」以置司農治堂之梁間。吏後以告泚,泚取視,其封帕完新。



 秀實嘗以禁兵寡弱,不足備非常,言於帝曰:「古者天子曰萬乘,諸侯曰千乘,大夫曰百乘,蓋以大制小,以十制一。今外有不廷之虜,內有梗命之臣,而禁兵寡少,卒有患難,何以待之?且猛虎所以百獸畏者,為爪牙也;若去之,則犬彘馬牛,皆能為敵。」帝不用。及涇卒亂,召神策六軍,無一人至者,世多其謀。



 興元元年,詔贈太尉,謚曰忠烈。賜封戶五百,莊、第各一區;長子三品,諸子五品,並正員官。帝還都,又詔致祭,旌其門閭,親銘其碑云。太和中,子伯倫始立廟,有詔給鹵簿,賜度支綾絹五百,以少牢致祭。



 伯倫累官福建觀察使,終太僕卿。時宰相李石請文宗加賻襚,鄭覃曰:「自古殺身利社稷,未有如秀實者。」帝惻然,為罷朝,可其請。



 孫嶷、文楚、珂知名。



 嶷自鄭滑節度使入為右金吾衛大將軍,封西平郡公。甘露之變,嶷當誅,裴度奏忠臣後,宜免死,貶循州司馬。



 文楚,咸通末為雲州防禦使。時李國昌鎮振武,國昌子克用欲得雲中,引兵攻之,殺於鬥雞臺下,沙陀之亂自此始。



 珂,僖宗時居潁州。黃巢圍潁,刺史欲以城降,珂募少年拒戰,眾裹糧請從,賊遂潰,拜州司馬。



 劉海賓者,彭城人,以義俠聞。為涇原兵馬將,與秀實友善。累戰功,兼御史中丞。劉文喜據涇州叛,海賓與其子光國紿以奏請。及入對,因言奸慝可誅狀。既還,光國手斬文喜獻闕下,拜左驍衛大將軍,封五原郡王;海賓樂平郡王,贈太子太保,實封百戶。



 顏真卿,字清臣,秘書監師古五世從孫。少孤,母殷躬加訓導。既長,博學工辭章,事親孝。



 開元中,舉進士,又擢制科。調醴泉尉。再遷監察御史,使河、隴。時五原有冤獄久不決,天且旱,真卿辨獄而雨,郡人呼「御史雨」。復使河東,劾奏朔方令鄭延祚母死不葬三十年,有詔終身不齒,聞者聳然。遷殿中侍御史。時御史吉溫以私怨構中丞宋渾,謫賀州,真卿曰:「奈何以一時忿,欲危宋璟後乎?」宰相楊國忠惡之,諷中丞蔣冽奏為東都採訪判官,再轉武部員外郎。國忠終欲去之,乃出為平原太守。



 安祿山逆狀牙孽,真卿度必反,陽托霖雨,增陴浚隍,料才壯,儲廥廩。日與賓客泛舟飲酒,以紓祿山之疑。果以為書生,不虞也。祿山反,河朔盡陷,獨平原城守具備,使司兵參軍李平馳奏。玄宗始聞亂,嘆曰:「河北二十四郡,無一忠臣邪?」及平至,帝大喜,謂左右曰:「朕不識真卿何如人,所為乃若此!」



 時平原有靜塞兵三千,乃益募士,得萬人,遣錄事參軍李擇交統之,以刁萬歲、和琳、徐浩、馬相如、高抗朗等為將,分總部伍。大饗士城西門,慷慨泣下,眾感勵。饒陽太守盧全誠、濟南太守李隨、清河長史王懷忠、景城司馬李韋、鄴郡太守王燾各以眾歸,有詔北海太守賀蘭進明率精銳五千濟河為助。賊破東都,遣段子光傳李憕、盧奕、蔣清首徇河北,真卿畏眾懼,紿諸將曰:「吾素識心登等,其首皆非是。」乃斬子光,藏三首。它日,結芻續體,斂而祭,為位哭之。



 是時,從父兄杲卿為常山太守,斬賊將李欽湊等,清土門。十七郡同日自歸,推真卿為盟主,兵二十萬,絕燕、趙。詔即拜戶部侍郎,佐李光弼討賊。真卿以李暉自副,而用李銑、賈載、沈震為判官。俄加河北招討採訪使。



 清河太守使郡人李崿來乞師,崿曰:「聞公首奮裾唱大順,河朔恃公為金城。清河,西鄰也,有江淮租布備北軍,號『天下北庫』。計其積,足以三平原之有,士卒可以二平原之眾。公因而撫有,以為腹心,它城運之如臂之指耳。」真卿為出兵六千,謂曰:「吾兵已出,子將何以教我?」崿曰:「朝家使程千里統眾十萬,自太行而東,將出郭口,限賊不得前。公若先伐魏郡,斬賊守袁知泰,以勁兵披郭口,出官師使討鄴、幽陵,平原、清河合十萬眾徇洛陽,分犀銳制其沖。公堅壁勿與戰,不數十日,賊必潰,相圖死。」真卿然之。乃檄清河等郡,遣大將李擇交、副將範冬馥、和琳、徐浩與清河、博平士五千屯堂邑。袁知泰遣將白嗣深、乙舒蒙等兵二萬拒戰,賊敗,斬首萬級,知泰走汲郡。



 史思明圍饒陽,遣游奕兵絕平原救軍,真卿懼不敵,以書招賀蘭進明,以河北招討使讓之。進明敗於信都。會平盧將劉正臣以漁陽歸,真卿欲堅其意,遣賈載越海遺軍資十餘萬,以子頗為質。頗甫十歲,軍中固請留之,不從。



 肅宗已即位靈武,真卿數遣使以蠟丸裹書陳事。拜工部尚書兼御史大夫,復為河北招討使。時軍費困竭,李崿勸真卿收景城鹽,使諸郡相輸,用度遂不乏。第五琦方參進明軍,後得其法以行,軍用饒雄。



 祿山乘虛遣思明、尹子奇急攻河北,諸郡復陷,獨平原、博平、清河固守。然人心危,不復振。真卿謀於眾曰:「賊銳甚,不可抗。若委命辱國,非計也。不如徑赴行在,朝廷若誅敗軍罪,吾死不恨。」至德元載十月,棄郡度河,間關至鳳翔謁帝,詔授憲部尚書,遷御史大夫。



 方朝廷草昧不暇給,而真卿繩治如平日。武部侍郎崔漪、諫議大夫李何忌皆被劾斥降。廣平王總兵二十萬平長安,辭日,當闕不敢乘,趨出■枑乃乘。王府都虞候管崇嗣先王而騎,真卿劾之。帝還奏,慰答曰:「朕子每出,諄諄教戒,故不敢失。崇嗣老而鐍,卿姑容之。」百官肅然。兩京復,帝遺左司郎中李選告宗廟,祝署「嗣皇帝」,真卿謂禮儀使崔器曰:「上皇在蜀,可乎?」器遽奏改之,帝以為達識。又建言:「《春秋》,新宮災,魯成公三日哭。今太廟為賊毀,請築壇於野,皇帝東向哭,然後遣使。」不從。宰相厭其言,出為馮翊太守。轉蒲州刺史,封丹陽縣子。為御史唐旻誣劾,貶饒州刺史。



 乾元二年,拜浙西節度使。劉展將反,真卿豫飭戰備,都統李峘以為生事,非短真卿,因召為刑部侍郎。展卒舉兵度淮,而峘奔江西。



 李輔國遷上皇西宮,真卿率百官問起居,輔國惡之,貶蓬州長史。代宗立,起為利州刺史,不拜,再遷吏部侍郎。除荊南節度使,未行,改尚書右丞。



 帝自陜還,真卿請先謁陵廟而即宮,宰相元載以為迂,真卿怒曰:「用舍在公,言者何罪?然朝廷事豈堪公再破壞邪!」載銜之。俄以檢校刑部尚書為朔方行營宣慰使,未行,留知省事,更封魯郡公。時載多引私黨,畏群臣論奏,乃紿帝曰:「群臣奏事,多挾讒毀。請每論事,皆先白長官,長官以白宰相,宰相詳可否以聞。」真卿上疏曰:



 諸司長官者,達官也,皆得專達於天子。郎官、御史,陛下腹心耳目之臣也,故出使天下,事無細大得失,皆俾訪察,還以聞。此古明四目、達四聰也。今陛下欲自屏耳目,使不聰明,則天下何望焉?《詩》曰:「營營青蠅,止於棘;讒言罔極,交亂四國。」以其能變白為黑,變黑為白也。詩人疾之,故曰:「取彼讒人,投畀豺虎;豺虎不食,投畀有北。」昔夏之伯明、楚之無極、漢之江充,皆讒人也,陛下惡之,宜矣。胡不回神省察?其言虛誣,則讒人也,宜誅殛之;其言不誣,則正人也,宜獎勵之。舍此不為,使眾人謂陛下不能省察而倦聽覽,以是為辭,臣竊惜之。



 昔太宗勤勞庶政,其《司門式》曰:「無門籍者有急奏,令監司與仗家引對,不得關礙。」防擁蔽也。置立仗馬二,須乘者聽。此其平治天下也。天寶後,李林甫得君,群臣不先咨宰相輒奏事者,托以他故中傷之,猶不敢明約百司,使先關白。時閹人袁思藝日宣詔至中書,天子動靜必告林甫,林甫得以先意奏請,帝驚喜若神,故權寵日甚,道路以目。上意不下宣,下情不上達,此權臣蔽主,不遵太宗之法也。陵夷至於今,天下之敝皆萃陛下,其所從來漸矣。自艱難之初,百姓尚未凋竭,太平之治猶可致,而李輔國當權,宰相用事,遞為姑息。開三司,誅反側,使餘賊潰將北走黨項,裒嘯不逞,更相驚恐,思明危懼,相挻而反,東都陷沒,先帝由是憂勤損壽。臣每思之,痛貫心骨。



 今天下瘡痏未平,干戈日滋,陛下豈得不博聞讜言以廣視聽,而塞絕忠諫乎?陛下在陜時,奏事者不限貴賤,群臣以為太宗之治可跂而待。且君子難進易退,朝廷開不諱之路,猶恐不言,況懷厭怠。令宰相宣進止,御史臺作條目,不得直進,從此人不奏事矣。陛下聞見,止於數人耳目。天下之士,方鉗口結舌,陛下便謂無事可論,豈知懼而不敢進,即林甫、國忠復起矣。臣謂今日之事,曠古未有,雖林甫、國忠猶不敢公為之。陛下不早覺悟,漸成孤立,後悔無及矣。



 於是中人等騰布中外。後攝事太廟,言祭器不飭,載以為誹謗,貶峽州別駕。改吉州司馬,遷撫、湖二州刺史。載誅,楊綰薦之,擢刑部尚書,進吏部。帝崩,以為禮儀使。因奏列聖謚繁,請從初議為定,袁傪固排之,罷不報。時喪亂後,典法湮放,真卿雖博識今古,屢建議釐正,為權臣沮抑,多中格云。



 楊炎當國,以直不容,換太子少師,然猶領使。及盧杞,益不喜,改太子太師,並使罷之,數遣人問方鎮所便,將出之。真卿往見杞,辭曰:「先中丞傳首平原,面流血,吾不敢以衣拭,親舌舐之,公忍不見容乎!」杞矍然下拜,而銜恨切骨。



 李希烈陷汝州,杞乃建遣真卿:「四方所信,若往諭之,可不勞師而定。」詔可,公卿皆失色。李勉以為失一元老,貽朝廷羞,密表固留。至河南,河南尹鄭叔則以希烈反狀明,勸不行,答曰:「君命可避乎?」既見希烈,宣詔旨,希烈養子千餘拔刃爭進,諸將皆慢罵,將食之,真卿色不變。希烈以身捍,麾其眾退,乃就館。逼使上疏雪己,真卿不從。乃詐遣真卿兄子峴與從吏數輩繼請,德宗不報。真卿每與諸子書,但戒嚴奉家廟,恤諸孤,訖無它語。希烈遣李元平說之,真卿叱曰:「爾受國委任,不能致命,顧吾無兵戮汝,尚說我邪?」希烈大會其黨,召真卿,使倡優斥侮朝廷。真卿怒曰:「公,人臣,奈何如是?」拂衣去。希烈大慚。時硃滔、王武俊、田悅、李納使者皆在坐,謂希烈曰:「聞太師名德久矣,公欲建大號而太師至,求宰相孰先太師者?」真卿叱曰:「若等聞顏常山否?吾兄也。祿山反,首舉義師,後雖被執,詬賊不絕於口。吾年且八十,官太師,吾守吾節,死而後已,豈受若等脅邪!」諸賊失色。



 希烈乃拘真卿,守以甲士,掘方丈坎於廷,傳將坑之,真卿見希烈曰:「死生分矣,何多為!」張伯儀敗,希烈令齎旌節首級示真卿,真卿慟哭投地。會其黨周曾、康秀林等謀襲希烈,奉真卿為帥。事洩,曾死,乃拘送真卿蔡州。真卿度必死,乃作遺表、墓志、祭文,指寢室西壁下曰:「此吾殯所也。」希烈僭稱帝,使問儀式,對曰:「老夫耄矣,曾掌國禮,所記諸侯朝覲耳!」



 興元後,王師復振,賊慮變,遣將辛景臻、安華至其所,積薪於廷曰:「不能屈節,當焚死。」真卿起赴火,景臻等遽止之。希烈弟希倩坐硃泚誅,希烈因發怒,使閹奴等害真卿,曰:「有詔。」真卿再拜。奴曰:「宜賜卿死。」曰:「老臣無狀,罪當死,然使人何日長安來?」奴曰:「從大梁來。」罵曰:「乃逆賊耳,何詔雲!」遂縊殺之,年七十六。嗣曹王皋聞之,泣下,三軍皆慟,因表其大節。淮、蔡平,子頵、碩護喪還,帝廢朝五日,贈司徒,謚文忠,賻布帛米粟加等。



 真卿立朝正色,剛而有禮,非公言直道,不萌於心。天下不以姓名稱,而獨曰魯公。如李正己、田神功、董秦、侯希逸、王玄志等,皆真卿始招起之,後皆有功。善正、草書,筆力遒婉,世寶傳之。貞元六年,赦書授頵五品正員官。開成初,又以曾孫弘式為同州參軍。



 贊曰:唐人柳宗元稱:「世言段太尉,大抵以為武人,一時奮不慮死以取名,非也。太尉為人姁姁,常低首拱手行步,言氣卑弱,未嘗以色待物;人視之,儒者也。遇不可,必達其志,決非偶然者。」宗元不妄許人,諒其然邪,非孔子所謂仁者必有勇乎?當祿山反,哮噬無前,魯公獨以烏合嬰其鋒,功雖不成,其志有足稱者。晚節偃蹇,為奸臣所擠,見殞賊手。毅然之氣,折而不沮,可謂忠矣。詳觀二子行事,當時亦不能盡信於君,及臨大節,蹈之無貳色,何耶?彼忠臣誼士,寧以未見信望於人,要返諸己得其正,而後慊於中而行之也。嗚呼,雖千五百歲,其英烈言言,如嚴霜烈日,可畏而仰哉!



\end{pinyinscope}