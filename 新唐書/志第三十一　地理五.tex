\article{志第三十一 地理五}

\begin{pinyinscope}

 淮南道,蓋古揚州之域,漢九江、廬江、江夏等郡,廣陵、六安國及南陽、汝南、臨淮之境。揚、楚、滁、和、廬、壽、舒為星紀分,安、黃、申、光、蘄為鶉尾分。為州十二「仰以觀於天文,俯以察於地理。」唐孔穎達疏:「天有懸象而,縣五十三。其名山:灊、天柱、羅、塗、八公。其大川:滁、肥、巢湖。厥賦:絁、絹、綿、布。厥貢:絲、布、紵、葛。



 揚州廣陵郡,大都督府。本南兗州江都郡,武德七年曰邗州,以邗溝為名,九年更置揚州,天寶元年更郡名。土貢:金、銀、銅器、青銅鏡、綿、蕃客袍錦、被錦、半臂錦、獨窠綾、殿額莞席、水兕甲、黃[QXDV]米、烏節米、魚臍、魚魚誇、糖蟹、蜜姜、藕、鐵精、空青、白芒、兔絲、蛇粟、括姜粉。有丹楊監、廣陵監錢官二。戶七萬七千一百五,口四十六萬七千八百五十七。縣七:有府四,曰江平、新林、方山,邗江。江都,望。東十一里有雷塘,貞觀十八年,長史李襲譽引渠,又築勾城塘,以溉田八百頃;有愛敬陂水門,貞元四年,節度使杜亞自江都西循蜀岡之右,引陂趨城隅以通漕,溉夾陂田;寶歷二年,漕渠淺,輸不及期,鹽鐵使王播自七里港引渠東注官河,以便漕運;有銅。江陽,望。貞觀十八年析江都置。有康令祠,咸通中大旱,令以身禱雨赴水死,天即大雨,民為立祠。六合,緊。武德七年析置石梁縣,以石梁、六合二縣置方州。貞觀元年州廢,省石梁,以六合來屬。有銅,有鐵。海陵,望。武德三年更名吳陵,以縣置吳州。七年州廢,復故名,來屬。景龍二年析置海安縣,開元十年省。有鹽官。高郵。上。有堤塘,溉田數千頃,元和中,節度使李吉甫築。揚子,望。永淳元年析江都置。天長。望。本千秋,天寶元年析江都、六合、高郵置,七載更名。有銅。



 楚州淮陰郡,緊。本江都郡之山陽、安宜縣地,臧君相據之,號東楚州。武德四年,君相降,因之,八年更名。土貢:貲布、紵布。戶二萬六千六十二,口十五萬三千。縣四:山陽,上。有常豐堰,大歷中,黜陟使李承置以溉田。鹽城,上。本故漢鹽瀆縣地。隋末,盜韋徹據其地,置射州及射陽、安樂、新安三縣。武德四年來歸,因之。七年州廢,省射陽、安樂、新安,置鹽城縣。有鹽亭百二十三,有監。寶應,望。本安宜。武德四年以縣置倉州,七年州廢,來屬。上元三年以獲定國寶更名。西南八十里有白水塘、羨塘,證聖中開,置屯田;西南四十里有徐州涇、青州涇,西南五十里有大府涇,長慶中興白水塘屯田,發青、徐、揚州之民以鑿之,大府即揚州;北四里有竹子涇,亦長慶中開。淮陰。中。武德七年省,乾封二年析山陽復置。南九十五里有棠梨涇,長慶二年開。



 滁州永陽郡,上。武德三年析揚州置。土貢:貲布、絲布、紵、綀、麻。有銅坑二。戶二萬六千四百八十六,口十五萬二千三百七十四。縣三:清流,上。全椒,緊。永陽。上。景龍三年析清流置。



 和州歷陽郡,上。土貢:紵布。戶二萬四千七百九十四,口十二萬二千一十三。縣三:有府一,曰新川。歷陽,上。有禱應山,本白石山,有棲隱山,本梅山,皆天寶六載更名。烏江,上。東南二里有韋游溝,引江至郭十五里,溉田五百頃,開元中,丞韋尹開,貞元十六年,令游重彥又治之,民享其利,以姓名溝。含山。上。武德六年析歷陽之故龍亢縣地置,八年省,長安四年復置,更名武壽,神龍元年復故名。



 壽州壽春郡,中都督府。本淮南郡,天寶元年更名。土貢:絲布、施、茶、生石斛。戶三萬五千五百八十一,口十八萬七千五百八十七。縣五:壽春,上。有八公山。安豐,緊。武德七年省小黃、肥陵二縣入焉。東北十里有永樂渠,溉高原田,廣德二年宰相元載置,大歷十三年廢。盛唐,上。本霍山。武德四年以霍山、應城、灊城三縣置霍州。貞觀元年州廢,省應城、灊城,以霍山來屬。神功元年曰武昌,神龍元年復故名,開元二十七年更名。有開化縣,武德四年置;有潛縣,五年置:貞觀中皆省。霍丘,緊。武德四年以松滋、霍丘二縣置蓼州,七年州廢,省松滋,以霍丘來屬。神功元年曰武昌。景雲元年復故名。霍山。上。天寶初析盛唐別置。有大別山、霍山。



 廬州廬江郡,上。土貢:花紗、交梭絲布、茶、蠟、酥、鹿脯、生石斛。戶四萬三千三百二十三,口二十萬五千三百九十六。縣五:合肥,緊。慎,緊。巢,上。本襄安。武德三年置巢州,並析置開成、扶陽二縣。七年州廢,省開成、扶陽,以巢來屬。東南四十里有故東關。廬江,緊。有擩山、白茅山,有銅。舒城。上。開元二十三年析合肥、廬江置。



 舒州同安郡,上。至德二載更名盛唐郡,後復故名。土貢:紵布、酒器、鐵器、石斛、蠟。戶三萬五千三百五十三,口十八萬六千三百九十八。縣五:懷寧,上。武德五年析置皖城、安樂、梅城、皖陽四縣,是年省安樂,七年省皖城、梅城、皖陽。有皖山。宿松,上。武德四年以縣置嚴州,七年以望江隸之,八年州廢,縣皆來屬。有嚴恭山。望江,中。武德四年以縣置高州,尋更名智州。七年州廢,以望江隸嚴州。太湖,上。武德四年析置青城、荊陽二縣,七年省青城入荊陽,八年省荊陽入太湖。桐城。緊。本同安,至德二載更名。自開元中徙治山城,地多猛虎、毒虺,元和八年,令韓震焚薙草木,其害遂除。



 光州弋陽郡,上。本治光山,太極元年徙治定城。土貢:葛布、石斛。戶三萬一千四百七十三,口十九萬八千五百八十。縣五:定城,上。武德三年置弦州,貞觀元年州廢,來屬。光山,上。南有木陵故關。西南八里有雨施陂,永徽四年,刺史裴大覺積水以溉田百餘頃。仙居,上。本樂安。武德三年析置宋安縣,以宋安置穀州。貞觀元年州廢,省宋安。天寶元年更名。殷城,中。武德元年置義州,貞觀元年州廢,來屬。西有定城故關。固始。上。



 蘄州蘄春郡,上。土貢:白紵,簟,鹿毛筆,茶,白花蛇、烏蛇脯。戶二萬六千八百九,口十八萬六千八百四十九。縣四:蘄春,上。武德四年省蘄水縣入焉。有鼓吹山。黃梅,上。武德四年置,以縣置南晉州,析置義豐、長吉、塘陽、新蔡四縣。八年州廢,省義豐、長吉、塘陽、新蔡,以黃梅來屬。廣濟,中。本永寧,武德四年析蘄春置,天寶元年更名。有鐵。蘄水。上。本浠水。武德四年更名蘭溪,省羅田縣入焉。天寶元年又更名。有鐵。



 安州安陸郡,中都督府。土貢:青紵布、糟筍瓜。戶二萬二千二百二十一,口十七萬一千二百二。縣六:有府一,曰義安。安陸,上。雲夢,中。有神山。孝昌,中。武德四年以縣置澴州,並置澴陽縣。八年州廢,省澴陽,以孝昌來屬。寶應二年隸沔州,後復來屬。元和三年省入雲夢。咸通中復置。應城,中。本應陽,武德四年更名。元和三年省入雲夢,大和二年復置。天祐二年復曰應陽。吉陽,中。元和三年省入應山,後復置。有白兆山。應山。中。武德四年以縣置應州,並析置禮山縣。八年州廢,省禮山,以應山來屬。有故黃峴、武陽、百雁、平靖四關。



 黃州齊安郡,下。本永安郡,天寶元年更名。土貢:白紵布、貲布、連翹、松蘿、虻蟲。戶萬五千五百一十二,口九萬六千三百六十八。縣三:黃岡,上。武德三年省木蘭縣入焉;又析置堡城縣,七年省。有木蘭山。黃陂,中。武德三年以縣置南司州,七年州廢,來屬。北有大後關,有白沙關。麻城。中。武德三年以縣置亭州,又析置陽城縣。八年州廢,省陽城,以麻城來屬。元和三年省入黃岡,建中三年復置。西北有木陵關,在木陵山上;東北有陰山關。



 申州義陽郡,中。土貢:緋葛、紵布、貲布、茶、虻蟲。戶二萬五千八百六十四,口十四萬七千七百五十六。縣三:義陽,上。南有故平靖關。鐘山,上。羅山。上。武德四年以縣置南羅州,八年州廢,來屬。



 右淮南採訪使,治揚州。



 江南道,蓋古揚州南境,漢丹楊、會稽、豫章、廬江、零陵、桂陽等郡,長沙國及牂柯、江夏、南郡地。潤、升、常、蘇、湖、杭、睦、越、明、衢、處、婺、溫、臺、宣、歙、池、洪、江、饒、虔、吉、袁、信、撫、福、建、泉、汀、漳為星紀分,岳、鄂、潭、衡、永、道、郴、邵、黔、辰、錦、施、敘、獎、夷、播、思、費、南、溪、溱為鶉尾分。為州五十一,縣二百四十七。其名山:衡、廬、茅、蔣、天目、天臺、會稽、四明、括蒼、縉雲、金華、大庾、武夷。其大川:湘、贛、沅、澧、浙江、洞庭、彭蠡、太湖。厥賦:麻、紵。厥貢:金、銀、紗、綾、蕉、葛、綿、綀、鮫革、藤紙、丹沙。



 潤州丹楊郡,望。武德三年以江都郡之延陵縣地置,取潤浦為州名。土貢:衫羅,水紋、方紋、魚口、繡葉、花紋等綾,火麻布,竹根,黃粟,伏牛山銅器,鱘,鮓。戶十萬二千二十三,口六十六萬二千七百六。縣四:有丹楊軍,乾元二年置,元和二年廢。丹徒,望。本延陵縣地,武德三年置。開元二十二年,刺史齊浣以州北隔江,舟行繞瓜步,回遠六十里,多風濤,乃於京口埭下直趨渡江二十里,開伊婁河二十五里,渡揚子,立埭,歲利百億,舟不漂溺。有勾驪山。丹楊,望。本曲阿。武德二年以縣置雲州。五年曰簡州,以縣南有簡瀆取名。八年州廢,來屬。天寶元年更名。有練塘,周八十里,永泰中,刺史韋損因廢塘復置,以溉丹楊、金壇、延陵之田,民刻石頌之。金壇,緊。本曲阿縣地。隋末,土人保聚,因為金山縣。隋亡,沈法興又置瑯王耶縣,李子通以瑯琊置茅州,以金山隸之。賊平,因之,後隸蔣州。武德八年省入延陵。垂拱四年復置,來屬,更名。東南三十里有南、北謝塘,武德二年,刺史謝元超因故塘復置以溉田。延陵。緊。故治丹徒,武德三年別置,隸茅州,後隸蔣州,九年來屬。有茅山。



 升州江寧郡,至德二載以潤州之江寧縣置,上元二年廢,光啟三年復以上元、句容、溧水、溧陽四縣置。土貢:筆、甘棠。縣四:有江寧軍,乾元二年置;有石頭鎮兵;有下蜀、淮山二戍。上元,望。本江寧,隸潤州。武德三年以江寧、溧水二縣置揚州,析置丹楊、溧陽、安業三縣,更江寧曰歸化。七年平輔公祏,更名蔣州。八年,復為揚州,又以延陵、句容隸之,省安業入歸化,更歸化曰金陵。九年州廢,都督徙治江都,更名金陵曰白下,以白下、延陵、句容隸潤州,丹楊、溧水、溧陽隸宣州。貞觀九年更白下曰江寧,肅宗上元二年又更名。有銅,有鐵;有蔣山。句容,望。武德三年以句容、延陵二縣置茅州,七年州廢,隸蔣州,九年隸潤州。乾元元年來屬。西南三十里有絳巖湖,麟德中,令楊延嘉因梁故堤置,後廢,大歷十二年,令王昕復置,周百里為塘,立二斗門以節旱,開田萬頃。絳巖,故赤山,天寶中更名。有銅,有礬。溧水,上。乾元元年隸升州,州廢,還隸宣州。有銅。溧陽。緊。上元元年隸升州,州廢,還隸宣州。有湖山。有銅,有鐵。



 常州晉陵郡,望。本昆陵郡,天寶元年更名。土貢:紬、絹布、紵、紅紫綿巾、緊紗、兔褐、皁布、大小香秔、龍鳳席、紫筍茶、署預。戶十萬二千六百三十三,口六十九萬六百七十三。縣五:晉陵,望。武進,望。武德三年以故蘭陵縣地置,貞觀八年省入晉陵,垂拱二年復置。西四十里有孟瀆,引江水南注通漕,溉田四千頃,元和八年,刺史孟簡因故渠開。江陰,望。武德三年以縣置暨州,並析置暨陽、利城二縣。九年州廢,省暨陽、利城,以江陰來屬。義興,緊。武德七年以縣置南興州,並析置臨津、陽羨二縣。八年州廢,省陽羨、臨津,以義興來屬。有張公山。無錫。望。南五里有泰伯瀆,東連蠡湖,亦元和八年孟簡所開。



 蘇州吳郡,雄。土貢:絲葛,絲綿,八蠶絲,緋綾,布,白角簟,草席、奚,大小香秔、柑、橘、藕、鯔皮、魬、昔、鴨胞、肚魚、魚子、白石脂、蛇粟。戶七萬六千四百二十一,口六十三萬二千六百五十。縣七:有長洲軍,乾元二年置,大歷十二年廢。吳,望。有包山,有銅。長洲,望。萬歲通天元年析吳置。嘉興,望。武德七年置,八年省入吳,貞觀八年復置。有鹽官。昆山,望。常熟,緊。海鹽,緊。貞觀元年省,景雲二年復置。有古涇三百,長慶中令李諤開,以御水旱;又西北六十里有漢塘,大和七年開;有故縣山。華亭。上。天寶十載析嘉興置。



 湖州吳興郡,上。武德四年,以吳郡之烏程縣置。土貢:御服、烏眼綾、折皁布、綿紬、布、紵、糯米、黃、紫筍茶、木瓜、杭子、乳柑、蜜、金沙泉。戶七萬三千三百六,口四十七萬七千六百九十八。縣五:烏程,望。東百二十三里有官池,元和中刺史範傳正開。東南二十五里有陵波塘,寶歷中刺史崔玄亮開。北二里有蒲帆塘,刺史楊漢公開,開而得蒲帆,因名。有卞山;有太湖,占湖、宣、常、蘇四州境。武康,上。李子通置安州,又曰武州。武德四年平子通,因之,七年州廢,縣隸湖州。有封山,有銅。長城,望。大業末沈法興置長州。武德四年更置綏州,因古綏安縣名之,又更名雉州,並置原鄉縣。七年州廢,省原鄉,以長城來屬。有西湖,溉田三千頃,其後堙廢,貞元十三年,刺史于頔復之,人賴其利。顧山有茶,以供貢;有銅。安吉,緊。義寧二年沈法興置。武德四年賊平,因之,以縣隸桃州。七年,省入長城。麟德元年復置。北三十里有邸閣池,北十七里有石鼓堰,引天目山水溉田百頃,皆聖歷初令鉗耳知命置;有銅,有錫。德清。上。本武源,天授二年析武康置,景雲二年曰臨溪,天寶元年更名。



 杭州餘杭郡,上。土貢:白編綾、緋綾、藤紙、木瓜、橘、蜜姜、乾姜、芑、牛膝。有臨平監、新亭監鹽官二。戶八萬六千二百五十八,口五十八萬五千九百六十三。縣八:有餘杭軍,乾元二年置;有鎮海軍,建中二年置於潤州,元和六年廢,大和九年復置,景福二年徙屯;又有烏山戍。錢塘,望。南五里有沙河塘,咸通二年刺史崔彥曾開;有皋亭山。鹽官,緊。武德四年隸東武州,七年省入錢塘,貞觀四年復置。有鹽官。有捍海塘堤,長百二十四里,開元元年重築。餘杭,望。南五里有上湖,西二里有下湖,寶歷中,令歸珧因漢令陳渾故跡置;北三里有北湖,亦珧所開,溉田千餘頃。珧又築甬道,通西北大路,高廣徑直百餘里,行旅無山水之患。有銅。富陽,緊。北十四里有陽陂湖,貞觀十二年令郝某開;南六十步有堤,登封元年令李浚時築,東自海,西至於莧浦,以捍水患,貞元七年,令鄭早又增脩之;王洲有橘,以供貢。於潛,緊。武德七年以縣置潛州,並置臨水縣。八年州廢,省臨水,以於潛來屬。南三十里有紫溪水溉田,貞元十八年令杜泳開,又鑿渠三十里,以通舟楫;有天目山。臨安,緊。垂拱四年析餘杭、於潛地以故臨水城置。有石鏡山。新城,上。武德七年省入富陽,永淳元年復置。北五里有官塘,堰水溉田;有九澳,永淳元年開。唐山。中。垂拱二年析於潛置紫溪縣。萬歲通天元年曰武隆,其年復為紫溪,又析紫溪別置武隆縣。聖歷三年省武隆入紫溪,長安四年復置。神龍元年更武隆為唐山。大歷二年皆省。長慶初復置唐山。



 睦州新定郡,上。本遂安郡,治雉山。武德七年曰東睦州,八年復舊名。萬歲通天二年徙治建德。天寶元年更郡名。土貢:文綾、簟、白石英、銀花、細茶。有銅坑二。戶五萬四千九百六十一,口三十八萬二千五百六十三。縣六:有三河戍。建德,上。武德四年置,七年省入桐廬、雉山。永淳二年復置。有銅。清溪,上。本雉山,文明元年曰新安,開元二十年曰還淳,永貞元年更名。壽昌,上。永昌元年析雉山置,載初元年省,神龍元年復置。桐廬,緊。武德四年以桐廬、分水、建德置嚴州。七年州廢,以桐廬來屬。分水,上。武德七年省入桐廬,如意元年復置,更名武盛,神龍元年復故名。寶應二年析置昭德縣,大歷六年省。遂安。上。石英山有白石英,以供貢;有銅。



 越州會稽郡,中都督府。土貢:寶花、花紋等羅,白編、交梭、十樣花紋等綾,輕容、生穀、花紗,吳絹,丹沙,石蜜,橘,葛粉,瓷器,紙,筆。有蘭亭監鹽官。戶九萬二百七十九,口五十二萬九千五百八十九。縣七:有府一,曰浦陽。有義勝軍、靜海軍,寶應元年置。大歷二年廢靜海軍,元和六年廢義勝軍。中和二年復置義勝軍,乾寧三年曰鎮東。會稽,望。有南鎮會稽山,有祠。東北四十里有防海塘,自上虞江抵山陰百餘里,以畜水溉田,開元十年令李俊之增脩,大歷十年觀察使皇甫溫、大和六年令李左次又增脩之;有錫。山陰,緊。武德七年析會稽置,八年省,垂拱二年復置,大歷二年省,七年復置,元和七年省,十年復置。北三十里有越王山堰,貞元元年,觀察使皇甫政鑿山以畜洩水利,又東北二十里作硃儲斗門;北五里有新河,西北十里有運道塘,皆元和十年觀察使孟簡開;西北四十六里有新逕斗門,大和七年觀察使陸亙置;有鐵。諸暨,望。有銀冶。東二里有湖塘,天寶中令郭密之築,溉田二十餘頃。餘姚,緊。武德四年析故句章縣置,以縣置姚州,七年州廢,來屬。有風山、四明山。剡,望。武德四年以縣置嵊州,並析置剡城縣,八年州廢,省剡城,以剡來屬。蕭山,緊。本永興,儀鳳二年置,天寶元年更名。上虞。上。貞元中析會稽置。西北二十七里有任嶼湖,寶歷二年令金堯恭置,溉田二百頃;北二十里有黎湖,亦堯恭所置。



 明州餘姚郡,上。開元二十六年,採訪使齊浣奏以越州之鄮縣置,以境有四明山為名。土貢:吳綾、交梭綾、海味、署預、附子。戶四萬二千二百七,口二十萬七千三十二。縣四:有上亭戍。鄮,上。武德四年析故句章縣置鄞州,八年州廢,更置鄮縣,隸越州。開元二十六年析置翁山縣,大歷六年省。有鹽。南二里有小江湖,溉田八百頃,開元中令王元緯置,民立祠祀之;東二十五里有西湖,溉田五百頃,天寶二年令陸南金開廣之;西十二里有廣德湖,溉田四百頃,貞元九年,刺史任侗因故跡增脩;西南四十里有仲夏堰,溉田數千頃,大和六年刺史於季友築。奉化,上。開元二十六年析鄮置。有銅。慈溪,上。開元二十六年析鄮置。象山。中。本隸臺州,神龍元年析寧海及貿阜置,廣德二年來屬。



 衢州信安郡,上。武德四年析婺州之信安縣置,六年沒輔公祐,因廢州,垂拱二年析婺州之信安、龍丘、常山復置。土貢:綿紙、竹扇。戶六萬八千四百七十二,口四十四萬四百一十一。縣四:西安,望。本信安,武德四年析置定陽縣,六年省,咸通中更信安曰西安。東五十五里有神塘,開元五年,因風雷摧山,偃澗成塘,溉田二百頃。有銀。龍丘,緊。本太末,武德四年置,以縣置穀州,並置白石縣,八年州廢,省太末、白石入信安。貞觀八年析信安、金華復置,更名龍丘,隸婺州。如意元年析置盈川縣。證聖二年置武安縣,後省武安。元和七年省盈川入信安。有岑山。須江,上。武德四年析信安置,八年省,永昌元年復置。常山。上。咸亨五年析信安置,隸婺州,垂拱二年來屬,乾元元年隸信州,後復故。



 處州縉雲郡,上。本括州永嘉郡,天寶元年更郡名,大歷十四年更州名。土貢:綿、蠟、黃連。戶四萬二千九百三十六,口二十五萬八千二百四十八。縣六:麗水,上。本括蒼,武德八年省麗水縣入焉,大歷十四年更名。有銅,出豫章、孝義二山;東十里有惡溪,多水怪,宣宗時刺史段成式有善政,水怪潛去,民謂之好溪;有括蒼山。松陽,上。武德中以縣置松州,八年州廢,來屬。有銀,出馬鞍山。縉雲,上。聖歷元年析括蒼及婺州之永康置。有縉雲山。青田,中。景雲二年析括蒼置。遂昌,上。武德八年省入松陽,景雲二年復置。龍泉。中。乾元二年析遂昌、松陽置。



 婺州東陽郡,上。土貢:綿、葛、紵布、藤紙、漆、赤松澗米、香粳、葛粉、黃連。戶十四萬四千八十六,口七十萬七千一百五十二。縣七:金華,望。武德八年省長山縣入焉。垂拱四年曰金山,神龍元年復故名。有百沙山、金華山;有銅。義烏,緊。本烏傷,武德四年以縣置綢州,因綢巖為名,並析置華川縣。七年州廢,省華川入烏傷,更名,來屬。永康,望。本縉雲,武德四年置麗州,八年州廢,更名,來屬。東陽,望。垂拱二年析義烏置。有歌山。蘭溪,緊。咸亨五年析金華置。有望雲山、大家山。武成,上。本武義,天授二年析永康置,更名,天祐中復曰武義。浦陽。上。天寶十三載析義烏、蘭溪及杭州之富陽置。



 溫州永嘉郡,上。高宗上元元年析括州之永嘉、安固置。土貢:布、柑、橘、蔗、蛟革。有永嘉監鹽官。戶四萬二千八百一十四,口二十四萬一千六百九十。縣四:永嘉,上。武德五年以縣置東嘉州,並析置永寧、安固、橫陽、樂成四縣。貞觀元年州廢,省橫陽、永寧,以永嘉、安固隸括州。安固,上。有銅。橫陽,上。大足元年析安固復置。樂成。上。武德七年省入永嘉,載初元年復置。



 臺州臨海郡,上。本海州,武德四年以永嘉郡之臨海置。土貢:金漆、乳柑、乾姜、甲香、蛟革、飛生鳥。戶八萬三千八百六十八,口四十八萬九千一十五,縣五:臨海,望。武德四年析置章安縣,八年省。有鐵。唐興,上。本始豐,武德四年析臨海置,八年省,貞觀八年復置,高宗上元二年更名。有土墻山、鼻山、天臺山。黃巖,上。本永寧,高宗上元二年析臨海置,天授元年更名。有鐵,有鹽。樂安,上。武德四年析臨海置,八年省,高宗上元二年復置。寧海。上。武德四年析臨海置,七年省入章安,永昌元年復置。有鐵,有鹽。



 福州長樂郡,中都督府。本泉州建安郡治,武德六年別置,景雲二年曰閩州,開元十三年更州名,天寶元年更郡名。土貢:蕉布、海蛤、文扇、茶、橄欖。戶三萬四千八十四,口七萬五千八百七十六。縣十:有經略軍,有寧海軍,至德二載置,元和六年廢。閩,望。東五里有海堤,大和三年令李茸築。先是,每六月潮水咸鹵,禾苗多死,堤成,渚溪水殖稻,其地三百戶皆良田。候官,緊。武德六年置,八年省,長安二年析閩復置,元和三年省,五年復置。有鹽官。西南七里有洪塘浦,自石江而東,經甓瀆至柳橋,以通舟楫,貞元十一年觀察使王翃開。長樂,上。本新寧,武德六年析閩置,尋更名。元和三年省入福唐,五年復置。有鹽。東十里有海堤,大和七年令李茸築,立十斗門以御潮,旱則渚水,雨則洩水,遂成良田。福唐,上。本萬安,聖歷二年析長樂置,天寶元年更名。有鐵。連江,上。本溫麻,武德六年析閩置,尋更名。有鹽。東北十八里有材塘,貞觀元年築。長溪,中下。武德六年置,尋省入連江,長安二年復置。有鹽。古田,中下。永泰二年析候官、尤溪置。梅溪,中。貞觀元年析候官置。永泰,中。咸通二年析連江及閩置。尤溪。中下。開元二十九年開山洞置。有銀,有銅,有鐵。



 建州建安郡,上。武德四年置。土貢:蕉、花練、竹綀。戶二萬二千七百七十,口十四萬二千七百七十四。縣五:建安,上。有銀,有銅。邵武,中下。本隸撫州,武德四年析置綏城縣,隸建州,七年以邵武來屬。貞觀三年省綏城入焉。有銅,有鐵。浦城,緊。本吳興,武德四年更名唐興,後廢入建安,載初元年復置,天授二年曰武寧,神龍元年復曰唐興,天寶元年更名。建陽,上。武德四年置,八年省入建安,垂拱四年復置。有武夷山。將樂。中下。武德五年析邵武置,隸撫州,七年省,垂拱四年析邵武及故綏城縣地復置。元和三年省,五年復置。金泉有金,又有銀、有鐵。



 泉州清源郡,上。本武榮州,聖歷二年析泉州之南安、莆田、龍溪置,治南安,後治晉江。三年,州廢,縣還隸泉州。久視元年復置。景雲二年更名。土貢:綿、絲、蕉、葛。戶二萬三千八百六,口十六萬二百九十五。縣四:自州正東海行二日至高華嶼,又二日至濆嶼,又一日至流求國。晉江,上。開元八年析南安置。北一里有晉江,開元二十九年,別駕趙頤貞鑿溝通舟楫至城下;東一里有尚書塘,溉田三百餘頃,貞元五年刺史趙昌置,名常稔塘,後昌為尚書,民思之,因更名;西南一里有天水塘,灌田百八十頃,大和二年刺史趙棨開;有鹽。南安,緊。武德五年以縣置豐州,並析置莆田縣,貞觀元年州廢,二縣來屬。有鹽,有鐵。莆田,上。武德五年析南安置。西一里有諸泉塘,南五里有瀝潯塘,西南二里有永豐塘,南二十里有橫塘,東北四十里有頡洋塘,東南二十里有國清塘,溉田總千二百頃,並貞觀中置;北七里有延壽陂,溉田四百餘頃,建中年置。仙游。中。本清源,聖歷二年析莆田置,天寶元年更名。



 汀州臨汀郡,下。開元二十四年開福、撫二州山洞置,治雜羅,大歷四年徙治白石,皆長汀縣地。土貢:蠟燭。戶四千六百八十,口萬三千七百二。縣三:長汀,中下。有銅,有鐵。寧化,中下。本黃連,天寶元年更名。有銀,有鐵。沙。中下。本隸建州,武德四年置,後省入建安,永徽六年復置,大歷十二年來屬。有銅,有鐵。



 漳州漳浦郡,下。垂拱二年析福州西南境置,以南有漳水為名,並置漳浦、懷恩二縣,初治漳浦,開元四年徙治李澳川,乾元二年徙治龍溪。土貢:甲香、鮫革。戶五千八百四十六,口萬七千九百四十。縣三:龍溪,中下。本隸泉州,後隸武榮州,開元二十九年來屬。龍巖,中下。開元二十四年置,隸汀州,大厲十二年來屬。漳浦。中下。開元二十九年省懷恩縣入焉。有梁山。



 右東道採訪使,治蘇州。



 宣州宣城郡,望。土貢:銀、銅器、綺、白糸寧、絲頭紅毯、兔褐、簟、紙、筆、署預、黃連、碌青。有鉛坑一。戶十二萬一千二百四,口八十八萬四千九百八十五。縣八:有採石軍,乾元二年置,元和六年廢。宣城,望。武德三年析置懷安縣,六年省。東十六里有德政陂,引渠溉田二百頃,大歷二年觀察使陳少游置;有敬亭山。當塗,緊。武德三年以縣置南豫州,八年州廢,來屬。貞觀元年省丹楊縣入焉。乾元元年隸升州,上元二年復來屬。有神武山。有採石戍。有銅,有鐵。涇,緊。武德三年以縣置南徐州,尋更名猷州,並置南陽、安吳二縣。八年州廢,省南陽、安吳,以涇來屬。廣德,緊。本綏安。武德三年以縣置桃州,並置桐陳、懷德二縣。七年州廢,省桐陳、懷德,以綏安來屬。至德二載更名。有橫山。南陵,望。武德四年隸池州,州廢來屬。後析置義安縣,又廢義安為銅官冶。利國山有銅,有鐵;鳳凰山有銀。有大農陂,溉田千頃,元和四年,寧國令範某因廢陂置,為石堰三百步,水所及者六十里;有永豐陂,在青弋江中,咸通五年置。有鵲頭鎮兵。有梅根、宛陵二監錢官。太平,上。天寶十一載析當塗、涇置,大歷中省,永泰中復置。寧國,緊。武德三年析宣城置,六年省,天寶三載析宣城、當塗復置。有銀。旌德。上。寶應二年析太平置。



 歙州新安郡,上。土貢:白紵、簟、紙、黃連。戶三萬八千三百二十,口二十四萬九千一百九。縣六:歙,緊。東南十二里有呂公灘,本車輪灘,湍悍善覆舟,刺史呂季重以俸募工鑿之,遂成安流。有主簿山。休寧,上。永泰元年,盜方清陷州,州民拒賊,保於山險,二年賊平,因析置歸德縣,大歷四年省。黟,上。績溪,中下。本北野,永徽五年析歙置,後更名。有銀,有鉛。婺源,上。開元二十八年析休寧置。祈門。中下。永泰二年平方清,因其壘析黟及饒州之浮梁置。西四十里有武陵嶺,元和中令路旻鑿石為盤道。西南十三里有閶門灘,善覆舟,旻開斗門以平其隘,號路公溪,後斗門廢。咸通三年,令陳甘節以俸募民穴石積木為橫梁,因山派渠,餘波入於乾溪,舟行乃安。



 池州,上。武德四年以宣州之秋浦、南陵二縣置,貞觀元年州廢,縣還隸宣州,永泰元年復析宣州之秋浦、青陽,饒州之至德置。土貢:紙、鐵。有鉛坑一。縣四:秋浦,緊。有烏石山,廣德初盜陳莊、方清所據。有銀,有銅。青陽,上。天寶元年析涇、南陵、秋浦置。有銅,有銀。至德,中。至德二載析鄱陽、秋浦置,隸潯陽郡,乾元元年隸饒州。石埭。中。永泰二年析青陽、秋浦置。



 洪州豫章郡,上都督府。土貢:葛、絲布、梅煎、乳柑。有銅坑一。戶五萬五千五百三十。口三十五萬三千二百三十一。縣七:有南昌軍,乾元二年置,元和六年廢。南昌,望。本豫章。武德五年析置鐘陵縣,又置南昌縣,以南昌置孫州,八年州廢,又省南昌、鐘陵。寶應元年更豫章曰鐘陵。貞元中又更名。縣南有東湖,元和三年,刺史韋丹開南塘斗門以節江水,開陂塘以溉田。豐城,上。天祐中曰吳皋。高安,望。本建城,武德五年更名,以縣置靖州,又置望蔡、華陽、宜豐、陽樂四縣。七年曰米州,又更名筠州。八年州廢,省華陽、望蔡、宜豐、陽樂,以高安來屬。有米山。建昌,緊。武德五年置南昌州,又析置龍安、永脩、新吳三縣。八年州廢,省永脩、龍安、新吳,以建昌來屬。南一里有捍水堤,會昌六年攝令何易於築;西二里又有堤,咸通二年令孫永築。新吳,上。永淳二年析建昌復置。武寧,上。長安四年析建昌置,景雲元年曰豫寧,寶應元年復故名。分寧。上。貞元十五年析武寧置。



 江州潯陽郡,上。本九江郡,天寶元年更名。土貢:葛、紙、碌、生石斛。戶萬九千二十五,口十萬五千七百四十四。縣三:有湖口、湓城二戍。潯陽,緊。本湓城,武德四年更名,又別析置湓城縣,五年析湓城置楚城縣,八年省湓城,貞觀八年省楚城。南有甘棠湖,長慶二年刺史李渤築,立斗門以蓄洩水勢;東有秋水堤,大和三年刺史韋珩築,西有斷洪堤,會昌二年刺史張又新築,以窒水害;有銀,有銅;有廬山;有彭蠡湖,一名宮亭湖。彭澤,上。武德五年置浩州,又析置都昌、樂城二縣。八年州廢,省樂城,以彭澤、都昌隸江州。有銅。都昌。上。南一里有陳令塘,咸通元年令陳可夫築,以阻潦水。



 鄂州江夏郡,緊。土貢:銀、碌、貲布。有鳳山監錢官。戶萬九千一百九十,口八萬四千五百六十三。縣七:有武昌軍,元和元年置。江夏,望。有鐵。永興,緊。有銅,有鐵。北有長樂堰,貞元十三年築。武昌,緊。有樊山,有銀,有銅,有鐵。浦圻,上。唐年,上。天寶二年開山洞置。漢陽,中。本沔州漢陽郡,武德四年以沔陽郡之漢陽、汊川二縣置。寶應二年以安州之孝昌隸之。建中二年州廢,四年復置。元和三年省孝昌。寶歷三年州又廢,二縣來屬。汊川。中。武德四年析漢陽置。



 岳州巴陵郡,中。本巴州,武德六年更名。土貢:紵布、鱉甲。戶萬一千七百四十,口五萬二百九十八。縣五:巴陵,上。有鐵。有洞庭山,在洞庭湖中。華容,上。垂拱二年更名容城,神龍元年復故名。橋江,中。本沅江,乾寧中更名。湘陰,中下。武德八年省羅縣入焉。昌江。中下。神龍三年析湘陰置。



 饒州鄱陽郡,上。土貢:麩金、銀、簟、茶。有永平監錢官。有銅坑三。戶四萬八百九十九,口二十四萬四千三百五十。縣四:鄱陽,上。武德五年析置廣晉縣,隸浩州,八年州廢,省縣入焉。縣東有邵父堤,東北三里有李公堤,建中元年刺史李復築,以捍江水。東北四里有馬塘,北六里有土湖,皆刺史馬植築。餘干,上。武德四年置玉亭、長城二縣,七年省玉亭入長城,八年省長城入餘干。有神山。樂平,上。武德四年置,九年省,後復置。有金,有銀,有銅,有鐵。浮梁。上。本新平,武德四年析鄱陽置,八年省,開元四年復置,曰新昌,天寶元年更名。



 虔州南康郡,上。土貢:絲布、紵布、竹綀、石蜜、梅、桂子、斑竹。戶三萬七千六百四十七,口二十七萬五千四百一十。縣七:有猶口鎮兵,有百丈戍。贛,上。虔化,上。有梅嶺山。南康,上。有錫。有大庾山。雩都,上。有金,天祐元年置瑞金監;有君山,有般固山。信豐,上。本南安,永淳元年析南康置,天寶元年更名。大庾,中。神龍元年析南康置。有鉛、錫。有橫浦關。安遠。中。貞元四年析雩都置。有鐵,有錫。



 吉州廬陵郡,上。土貢:絲葛、紵布、陟厘、斑竹。戶三萬七千七百五十二,口三十三萬七千三十二。縣五:廬陵,緊。太和,上。武德五年置南平州,並置永新、廣興、東昌三縣。八年州廢,省永新、廣興、東昌入太和,來屬。有王山。安福,上。武德五年以縣置穎州,七年州廢,來屬。新淦,上。永新。上。顯慶二年析太和置。



 袁州宜春郡,上。土貢:白紵。有銅坑一。戶二萬七千九十三,口十四萬四千九十六。縣三:宜春,上。有宜春泉,醞酒入貢;西南十里有李渠,引仰山水入城,刺史李將順鑿;有鐵。蘋鄉,上。新喻。上。本作「渝」,天寶後相承作「喻」。



 信州,上。乾元元年析饒州之弋陽,衢州之常山、玉山及建、撫之地置。土貢:葛粉。有玉山監錢官。有銅坑一,鉛坑一。縣四:上饒,緊。武德四年置,隸饒州,七年省入弋陽,乾元元年復置,並置永豐縣,元和七年省永豐入焉。有金,有銅,有鐵,有鉛。弋陽,上。有銀。貴溪,中。永泰元年析弋陽置。玉山。上。證聖二年析常山、須江及弋陽置。有銀。



 撫州臨川郡,上。土貢:金絲布、葛、竹箭、硃橘。戶三萬六百五,口十七萬六千三百九十四。縣四:臨川,上。有金,有銀。南城,上。武德五年析置永城、東興二縣,七年省。崇仁,上。武德五年析置宜黃縣,八年省。南豐。上。景雲二年析南城置,先天二年省,開元八年復置。



 潭州長沙郡,中都督府。土貢:絲葛、絲布、木瓜。戶三萬二千二百七十二,口十九萬二千六百五十七。縣六:有府一,曰長沙。有淥口、花石二戍,有橋口鎮兵。長沙,望。有金。湘潭,緊。本隸衡州,元和後來屬。有衡山。湘鄉,緊。武德四年析衡山置。益陽,上。武德四年析置新康縣,七年省。永泰元年,都督翟灌自望浮驛開新道,經浮丘至湘鄉。醴陵,中。武德四年析長沙置。有王喬山。瀏陽。中。景龍二年析長沙置。



 衡州衡陽郡,上。本衡山郡,天寶元年更名。土貢:麩金、綿紙。戶三萬三千六百八十八,口十九萬九千二百二十八。縣六:有戎分、洞口、平陽三戍。衡陽,緊。本臨烝,武德四年置,七年省重安、新城二縣入焉。開元二十年更名。有西母山。衡山,上。本隸潭州,神龍三年來屬。有南岳衡山祠。常寧,中下。本新寧,天寶元年更名。攸,中。武德四年置南雲州,又析置茶陵、安樂、陰山、新興、建寧五縣。貞觀元年州廢,省茶陵、安樂、陰山、新興、建寧,以攸來屬。茶陵,中。聖歷元年析攸因故縣復置。耒陽。上。本耒陰,武德四年更名。



 永州零陵郡,中。土貢:葛、笴、零陵香、石蜜、石燕。戶二萬七千四百九十四,口十七萬六千一百六十八。縣四:有麻田鎮兵,有雷石、盧洪二戍。零陵,上。祁陽,上。武德四年析零陵置,貞觀元年省,四年復置。有鐵。湘源,上。有金,有鐵。灌陽。中。蕭銑析湘源置,武德七年省,上元二年復置。



 道州江華郡,中。本營州,武德四年以零陵郡之營道、永陽二縣置,五年曰南營州,貞觀八年更名,十七年,州廢入永州,上元二年復置。土貢:白紵、零陵香、犀角。戶二萬二千五百五十一,口十三萬九千六十三。縣五:弘道,上。本營道,天寶元年更名。延唐,上。本梁興,蕭銑析營道置,銑平,更名唐興,長壽二年曰武盛,神龍元年復曰唐興,天寶元年又更名。有鐵。江華,中。武德四年析賀州之馮乘縣置,文明元年曰雲溪,神龍元年更名。有錫。永明,中。本永陽,貞觀八年省入營道,天授二年復置,天寶元年更名。有銀,有鐵。大歷。中。大歷二年析延唐置。



 郴州桂陽郡,上。土貢:赤錢、糸寧布、絲布。有桂陽監錢官。戶三萬三千一百七十五。縣八:郴,上。有馬嶺山。義章,中下。蕭銑析郴置,武德七年省,八年復置。有銀,有銅,有鉛。平陽,上。資興,上。本晉興,貞觀八年省,咸亨三年復置,更名。高亭,中下。本安陵,開元十三年析郴置,天寶元年更名。義昌,中下。臨武,中下。如意元年曰隆武,神龍元年復故名。藍山。上。本南平,咸亨二年置,天寶元年更名。



 邵州邵陽郡,下。本南梁州,武德四年析潭州之邵陽置,並置邵陵、建興二縣,貞觀十年更名。土貢:銀、犀角。戶萬七千七十三,口七萬一千六百四十四。縣二:邵陽,上。武德七年省邵陵縣入焉。有文斤山。武岡,中。本武攸,武德四年更名,七年省建興縣入焉。



 右西道採訪使,治洪州。



 黔州黔中郡,下都督府。本黔安郡,天寶元年更名。土貢:犀角、光明丹沙、蠟。戶四千二百七十,口二萬四千二百四。縣六:彭水,上。武德元年析置都上、石城二縣,二年又析置盈隆、洪杜、相永、萬資四縣。貞觀四年以相永、萬資置費州,都上置夷州,十年以夷州之高富來屬,十一年以高富隸夷州。有鹽。黔江,中下。本石城,天寶元年更名。洪杜,中下。洋水,中下。本盈隆,先天元年曰盈川,天寶元年更名。信寧,中下。本信安,武德二年更名,隸義州,貞觀十一年州廢,來屬。都濡。中下。貞觀二十年析盈隆置。



 辰州盧溪郡,中都督府。本沅陵郡,天寶元年更名。土貢:光明丹沙、犀角、黃連、黃牙。戶四千二百四十一,口二萬八千五百五十四。縣五:沅陵,上。盧溪,中下。武德三年析沅陵置。有武山。漵浦,上。武德五年析辰溪置。麻陽,中下。武德三年析沅陵、辰溪置。垂拱四年析置龍門縣,尋省。有丹穴。辰溪。中。



 錦州盧陽郡,下。垂拱二年以辰州麻陽縣地及開山洞置。土貢:光明丹砂、犀角。戶二千八百七十二,口萬四千三百七十四。縣五:盧陽,中下。招諭,中下。渭陽,中下。常豐,中下。本萬安,天寶元年更名。洛浦。中下。本隸溪州,天授二年析辰州之大鄉置,長安四年來屬。



 施州清化郡,下。本清江郡,天寶元年更名。土貢:麩金、犀角、黃連、蠟、藥實。戶三千七百二,口萬六千四百四十四。縣二:清江,中下。義寧元年置開夷縣,武德元年省入焉。建始。中下。義寧二年置葉州,貞觀八年州廢,來屬。



 敘州潭陽郡,下。本巫州,貞觀八年以辰州之龍標縣置,天授二年曰沅州,開元十三年以「沅」「原」聲相近,復為巫州,大歷五年更名。土貢:麩金、犀角。戶五千三百六十八,口二萬二千七百三十八。縣三:龍標,上。武德七年置,貞觀八年析置夜郎、朗溪、思微三縣,九年省思微。朗溪,中下。潭陽。中下。先天二年析龍標置。



 獎州龍溪郡,下。本舞州,長安四年以沅州之夜郎、渭溪二縣置,開元十三年以「舞」「武」聲相近,更名鶴州,二十年曰業州,大歷五年又更名。土貢:麩金、犀角、蠟。戶千六百七十二,口七千二百八十四。縣三:峨山,中下。本夜郎,天寶元年更名。渭溪,中下。天授二年析夜郎置。梓姜。中下。本隸充州,天寶三載廢為羈縻州,以縣來屬。



 夷州義泉郡,下。本隋明陽郡地,武德四年以思州之寧夷縣置,貞觀元年州廢,四年復以黔州之都上縣開南蠻置,十一年徙治綏陽。土貢:犀角、蠟燭。戶千二百八十四,口七千一十三。縣五:綏陽,中下。有綏陽山。都上,中下。義泉,中下。本隸明陽郡。武德二年以信安、義泉、綏陽三縣置義州,並置都牢、洋川二縣,五年曰智州。貞觀四年省都牢。五年,以廢虧阜州之樂安、宜林、芙蓉、瑘川四縣隸之,後又領廢夷州之綏養。十一年曰牢州,徙治義泉。十六年州廢,省綏養、樂安、宜林,以綏陽、義泉、洋川來屬,芙容、瑘川隸播州。洋川,中下。寧夷。中下。武德四年,析置夜郎、神泉、豐樂、綏養、雞翁、伏遠、明陽、高富、思義、丹川、宣慈、慈岳十二縣。六年省雞翁。及州廢,省夜郎、神泉、豐樂,以寧夷、伏遠、明陽、高富、思義、丹川隸務州,宣慈、慈岳隸涪州,綏養隸智州。貞觀六年復置雞翁縣,來屬。十一年又以高富來屬。永徽後省雞翁、高富。開元二十五年復以寧夷來屬。



 播州播川郡,下。本郎州,貞觀九年以隋牂柯郡之牂柯縣置,十一年廢,十三年復置,更名。土貢:斑竹。戶四百九十,口二千一百六十八。縣三:遵義,中下。本恭水,貞觀元年以牂柯地置,並置高山、貢山、柯盈、邪施、釋燕五縣。及郎州廢,縣亦省。十三年復置州,亦復置縣。十四年,更恭水曰羅蒙,高山曰舍月,貢山曰湖江,柯盈曰帶水,邪施曰羅為,釋燕曰胡刀。十六年更羅蒙曰遵義。顯慶五年省舍月、湖江、羅為。芙蓉,中下。貞觀五年置,隸虧阜州,十一年並瑘川,隸牢州。開元二十六年省瑘川、胡刀入焉。帶水。中下。



 思州寧夷郡,下。本務州,武德四年以隋巴東郡之務川、扶陽置,貞觀四年更名。土貢:蠟。戶千五百九十九,口萬二千二十一。縣三:務川,中下。武德元年置。貞觀元年,以廢夷州之寧夷、伏遠、思義、明陽、高富、丹川及廢思州之丹陽、城樂、感化、思王、多田隸務州,尋省思義、明陽、丹川,二年省丹陽,八年省感化,十年以高富隸黔州,十一年省伏遠。思王,中下。武德三年置。思邛。中下。開元四年開生獠置。



 費州涪川郡,下。貞觀四年析思州之涪川、扶陽,開南蠻置。土貢:蠟。戶四百二十九,口二千六百九。縣四:涪川,中下。武德四年析務川置。貞觀四年以黔州之相永、萬資隸費州,十一年省。扶陽,中下。多田,中下。武德四年置,隸思州,貞觀元年隸務州,八年來屬。城樂。中下。武德四年招慰生獠置,隸思州,貞觀元年隸務州,八年來屬。



 南州南川郡,下。武德二年開南蠻置,三年更名僰州,四年復故名。土貢:斑布。戶四百四十三,口二千四十三。縣二:南川,中下。本隆陽,武德二年置,並置扶化、隆巫、丹溪、靈水四縣。貞觀十一年省扶化、隆巫、靈水。先天元年更隆陽曰南川。三溪。中下。貞觀五年置,七年又置當山、嵐山、歸德、汶溪四縣,八年皆省。



 溪州靈溪郡,下。天授二年析辰州置。土貢:丹沙、犀角、茶牙。戶二千一百八十四,口萬五千二百八十二。縣二:大鄉,上。三亭。中下。貞觀九年析大鄉置。有大酉山。



 溱州溱溪郡,下。貞觀十六年開山洞置。土貢:文龜、斑布、丹沙。戶八百七十九,口五千四十五。縣五:榮懿,中下。貞觀十六年置,並置扶歡、樂來二縣。咸亨元年省樂來。扶歡,中下。夜郎,中下。貞觀十六年開山洞置珍州,並置夜郎、麗皋、樂源三縣,後為夜郎郡。元和三年州廢,縣皆來屬。麗皋,中下。樂源。中下。



 右黔中採訪使,治黔州。



\end{pinyinscope}