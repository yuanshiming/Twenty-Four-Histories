\article{志第三十五 選舉志下}

\begin{pinyinscope}

 凡選有文、武,文選吏部主之,武選兵部主之,皆為三銓,尚書、侍郎分主之。



 凡官員有數,而署置過者有罰,知而聽者有罰,規取者有罰。每歲五月,頒格於州縣,選人應格,則本屬或故任取選解,列其罷免、善惡之狀,以十月會於省,過其時者不敘。其以時至者,乃考其功過。同流者,五五為聯,京官五人保之,一人識之。刑家之子、工賈異類及假名承偽、隱冒升降者有罰。文書粟錯,隱幸者駁放之;非隱幸則不。



 凡擇人之法有四:一曰身,體貌豐偉;二曰言,言辭辯正;三曰書,楷法遒美;四曰判,文理優長。四事皆可取,則先德行;德均以才,才均以勞。得者為留,不得者為放。五品以上不試,上其名中書門下;六品以下始集而試,觀其書、判。已試而銓,察其身、言;已銓而注,詢其便利而擬;已注而唱,不厭者得反通其辭,三唱而不厭,聽冬集。厭者為甲,上於僕射,乃上門下省,給事中讀之,黃門侍郎省之,侍中審之,然後以聞。主者受旨而奉行焉,謂之「奏受」。視品及流外,則判補。皆給以符,謂之「告身」。凡官已受成,皆廷謝。



 凡試判登科謂之「入等」,甚拙者謂之「藍縷」。選未滿而試文三篇,謂之「宏辭」;試判三條,謂之「拔萃」。中者即授官。



 凡出身,嗣王、郡王,從四品下;親王諸子封郡公者,從五品上;國公,正六品上;郡公,正六品下;縣公,從六品上;侯,正七品上;伯,正七品下;子,從七品上;男,從七品下;皇帝緦麻以上親、皇太後期親,正六品上;皇太后大功、皇後期親,從六品上;皇帝袒免、皇太后小功緦麻、皇后大功親,正七品上;皇后小功緦麻、皇太子妃期親,從七品上。外戚,皆以服屬降二階敘。娶郡主者,正六品上;娶縣主者,正七品上;郡主子,從七品上;縣主子,從八品上。



 凡用廕,一品子,正七品上;二品子,正七品下;三品子,從七品上;從三品子,從七品下;正四品子,正八品上;從四品子,正八品下;正五品子,從八品上;從五品及國公子,從八品下。凡品子任雜掌及王公以下親事、帳內勞滿而選者,七品以上子,從九品上敘。其任流外而應入流內,敘品卑者,亦如之。九品以上及勛官五品以上子,從九品下敘。三品以上廕曾孫,五品以上廕孫。孫降子一等,曾孫降孫一等。贈官降正官一等,死事者與正官同。郡、縣公子,神從五品孫。縣男以上子,降一等。勛官二品子,又降一等。二王後孫,視正三品。



 凡秀才,上上第,正八品上;上中第,正八品下;上下第,從八品上;中上第,從八品下。明經,上上第,從八品下;上中第,正九品上;上下第,正九品下;中上第,從九品下。進士、明法,甲第,從九品上;乙第,從九品下。弘文、崇文館生及第,亦如之。應入五品者,以聞。書、算學生,從九品下敘。



 凡弘文、崇文生,皇緦麻以上親,皇太后、皇后大功以上親,一家聽二人選。職事二品以上、散官一品、中書門下正三品同三品、六尚書等子孫並侄,功臣身食實封者子孫,一廕聽二人選。京官職事正三品、同中書門下平章事、供奉官三品子孫,京官職事從三品、中書黃門侍郎並供奉三品官、帶四品五品散官子,一廕一人。



 凡勛官選者,上柱國,正六品敘;六品而下,遞降一階。驍騎尉、武騎尉,從九品上敘。



 凡居官必四考,四考中中,進年勞一階敘。每一考,中上進一階,上下二階,上中以上及計考應至五品以上奏而別敘。六品以下遷改不更選及守五品以上官,年勞歲一敘,給記階牒。考多者,準考累加。



 凡醫術,不過尚藥奉御。陰陽、卜筮、圖畫、工巧、造食、音聲及天文,不過本色局、署令。鴻臚譯語,不過典客署令。凡千牛備身、備身左右,五考送兵部試,有文者送吏部。凡齋郎,太廟以五品以上子孫及六品職事並清官子為之,六考而滿;郊社以六品職事官子為之,八考而滿。皆讀兩經粗通,限年十五以上、二十以下,擇儀狀端正無疾者。



 武選,凡納課品子,歲取文武六品以下、勛官三品以下五品以上子,年十八以上,每州為解上兵部,納課十三歲而試,第一等送吏部,第二等留本司,第三等納資二歲,第四等納資三歲;納已,復試,量文武授散官。若考滿不試,免當年資;遭喪免資。無故不輸資及有犯者,放還之。凡捉錢品子,無違負滿二百日,本屬以簿附朝集使,上於考功、兵部。滿十歲,量文武授散官。其視品國官府佐應停者,依品子納課,十歲而試,凡一歲為一選。自一選至十二選,視官品高下以定其數,因其功過而增損之。



 初,武德中,天下兵革新定,士不求祿,官不充員。有司移符州縣,課人赴調,遠方或賜衣續食,猶辭不行。至則授用,無所黜退。不數年,求者浸多,亦頗加簡汰。



 貞觀二年,侍郎劉林甫言:「隋制以十一月為選始,至春乃畢。今選者眾,請四時注擬。」十九年,馬周以四時選為勞,乃復以十一月選,至三月畢。



 太宗嘗謂攝吏部尚書杜如晦曰:「今專以言辭刀筆取人,而不悉其行,至後敗職,雖刑戮之,而民已敝矣。」乃欲放古,令諸州闢召。會功臣行世封,乃止。它日復顧侍臣曰:「致治之術,在於得賢。今公等不知人,朕又不能遍識,日月其逝,而人遠矣。吾將使人自舉,可乎?」而魏徵以為長澆競,又止。



 初,銓法簡而任重。高宗總章二年,司列少常伯裴行儉始設長名榜,引銓注法,復定州縣升降為八等,其三京、五府、都護、都督府,悉有差次,量官資授之。其後李敬玄為少常伯,委事於員外郎張仁禕,仁禕又造姓歷,改狀樣、銓歷等程式,而銓總之法密矣。然是時仕者眾,庸愚咸集,有偽主符告而矯為官者,有接承它名而參調者,有遠人無親而置保者。試之日,冒名代進,或旁坐假手,或借人外助,多非其實。雖繁設等級、遞差選限、增譴犯之科、開糾告之令以遏之,然猶不能禁。大率十人競一官,餘多委積不可遣,有司患之,謀為黜落之計,以僻書隱學為判目,無復求人之意。而吏求貨賄,出入升降。至武后時,天官侍郎魏玄同深嫉之,因請復古闢署之法,不報。



 初,試選人皆糊名,令學士考判,武后以為非委任之方,罷之。而其務收人心,士無賢不肖,多所進獎。長安二年,舉人授拾遺、補闕、御史、著作佐郎、大理評事、衛佐凡百餘人。明年,引見風俗使,舉人悉授試官,高者至鳳閣舍人、給事中,次員外郎、御史、補闕、拾遺、校書郎。試官之起,自此始。時李嶠為尚書,又置員外郎二千餘員,悉用勢家親戚,給俸祿,使厘務,至與正官爭事相毆者。又有檢校、敕攝、判知之官。神龍二年,嶠復為中書令,始悔之,乃停員外官厘務。



 中宗時,韋後及太平、安樂公主等用事,於側門降墨敕斜封授官,號「斜封官」,凡數千員。內外盈溢,無聽事以居,當時謂之「三無坐處」,言宰相、御史及員外郎也。又以鄭愔為侍郎,大納貨賂,選人留者甚眾,至逆用三年員闕,而綱紀大潰。韋氏敗,始以宋璟為吏部尚書,李乂、盧從願為侍郎,姚元之為兵部尚書,陸象先、盧懷慎為侍郎,悉奏罷斜封官,量闕留人,雖資高考深,非才實者不取。初,尚書銓掌七品以上選,侍郎銓掌八品以下選。至是,通其品而掌焉。未幾,璟、元之等罷,殿中侍御史崔涖、太子中允薛昭希太平公主意,上言:「罷斜封官,人失其所,而怨積於下,必有非常之變。」乃下詔盡復斜封別敕官。



 玄宗即位,厲精為治。左拾遺內供奉張九齡上疏言:「縣令、刺史,陛下所與共理,尤親於民者也。今京官出外,乃反以為斥逐,非少重其選不可。」又曰:「古者或遙聞闢召,或一見任之,是以士脩名行,而流品不雜。今吏部始造簿書,以備遺忘,而反求精於案牘,不急人才,何異遺劍中流,而刻舟以記。」於是下詔擇京官有善政者補刺史,歲十月,按察使校殿最,自第一至第五,校考使及戶部長官總核之,以為升降。凡官,不歷州縣不擬臺省。已而悉集新除縣令宣政殿,親臨問以治人之策,而擢其高第者。又詔員外郎、御史諸供奉官,皆進名敕授,而兵、吏部各以員外郎一人判南曹,由是銓司之任輕矣。其後戶部侍郎宇文融又建議置十銓,乃以禮部尚書蘇頲等分主之。太子左庶子吳兢諫曰:「《易》稱『君子思不出其位』,言不侵官也。今以頲等分掌吏部選,而天子親臨決之,尚書、侍郎皆不聞參,議者以為萬乘之君,下行選事。」帝悟,遂復以三銓還有司。



 開元十八年,侍中裴光庭兼吏部尚書,始作循資格,而賢愚一概,必與格合,乃得銓授,限年躡級,不得逾越。於是久淹不收者皆便之,謂之「聖書」。及光庭卒,中書令蕭嵩以為非求材之方,奏罷之。乃下詔曰:「凡人年三十而出身,四十乃得從事,更造格以分寸為差,若循新格,則六十未離一尉。自今選人才業優異有操行及遠郡下寮名跡稍著者,吏部隨材甄擢之。」



 初,諸司官兼知政事者,至日午後乃還本司視事。兵部、吏部尚書侍郎知政事者,亦還本司分闕注唱。開元以來,宰相位望漸崇,雖尚書知政事,亦於中書決本司事以自便。而左、右相兼兵部、吏部尚書者,不自銓總。又故事,必三銓、三注、三唱而後擬官,季春始畢,乃過門下省。楊國忠以右相兼文部尚書,建議選人視官資、書判、狀跡、功優,宜對眾定留放。乃先遣吏密定員闕,一日會左相及諸司長官於都堂注唱,以誇神速。由是門下過官、三銓注官之制皆廢,侍郎主試判而已。



 肅、代以後兵興,天下多故,官員益濫,而銓法無可道者。至德宗時,試太常寺協律郎沈既濟極言其敝曰:



 近世爵祿失之者久,其失非他,四太而已:入仕之門太多,世胄之家太優,祿利之資太厚,督責之令太薄。臣以為當輕其祿利,重其督責。夫古今選用之法,九流常敘,有三科而已,曰德也,才也,勞也;而今選曹,皆不及焉。且吏部甲令,雖曰度德居任,量才授職,計勞升敘,然考校之法,皆在書判簿歷、言辭俯仰之間,侍郎非通神,不可得而知。則安行徐言,非德也;空文善書,非才也;累資積考,非勞也。茍執不失,猶乖得人,況眾流茫茫,耳目有不足者乎?蓋非鑒之不明,非擇之不精,法使然也。王者觀變以制法,察時而立政。按前代選用,皆州、府察舉,至於齊、隋,署置多由請托。故當時議者,以為與其率私,不若自舉;與其外濫,不若內收。是以罷州府之權,而歸於吏部。此矯時懲弊之權法,非經國不刊之常典。今吏部之法蹙矣,不可以坐守刓弊。臣請五品以上及群司長官、宰臣進敘,吏部、兵部得參議焉;六品以下或僚佐之屬,聽州、府闢用。則銓擇之任,委於四方;結奏之成,歸於二部。必先擇牧守,然後授其權。高者先署而後聞,卑者聽版而不命。其牧守、將帥,或選用非公,則吏部、兵部得察而舉之。聖主明目達聰,逖聽遐視,罪其私冒。不慎舉者,小加譴黜,大正刑典。責成授任,誰敢不勉?夫如是,則接名偽命之徒,菲才薄行之人,貪叨賄貨,懦弱奸宄,下詔之日,隨聲而廢,通大數,十去八九矣。如是,人少而員寬,事核而官審,賢者不獎而自進,不肖者不抑而自退。或曰:『開元、天寶中,不易吏部之法,而天下砥平,何必外闢,方臻於理?』臣以為不然。夫選舉者,經邦之一端,雖制之有美惡,而行之由法令。是以州郡察舉,在兩漢則理,在魏、齊則亂。吏部選集,在神龍、景龍則紊,在開元、天寶則理。當其時久承升平,御以法術,慶賞不軼,威刑必齊,由是而理,匪用吏部而臻此也。向以此時用闢召之法,則理不益久乎?」



 天子雖嘉其言,而重於改作,訖不能用。



 初,吏部歲常集人,其後三數歲一集,選人猥至,文簿紛雜,吏因得以為奸利,士至蹉跌,或十年不得官,而闕員亦累歲不補。陸贄為相,乃懲其弊,命吏部據內外員三分之,計闕集人,歲以為常。是時,河西、隴右沒於虜,河南、河北不上計,吏員大率減天寶三之一,而入流者加一,故士人二年居官,十年待選,而考限遷除之法浸壞。憲宗時,宰相李吉甫定考遷之格,諸州刺史、次赤府少尹、次赤令、諸陵令、五府司馬、上州以上上佐、東宮官詹事諭德以下、王府官四品以上皆五考。侍御史十三月,殿中侍御史十八月,監察御史二十五月。三省官、諸道敕補、檢校五品以上及臺省官皆三考,餘官四考,文武官四品以下五考。凡遷,尚書省四品以上、文武官三品以上皆先奏。



 唐取人之路蓋多矣,方其盛時,著於令者,納課品子萬人,諸館及州縣學六萬三千七十人,太史歷生三十六人,天文生百五十人,太醫藥童、針咒諸生二百一十一人,太卜卜筮三十人,千牛備身八十人,備身左右二百五十六人,進馬十六人,齋郎八百六十二人,諸衛三衛監門直長三萬九千四百六十二人,諸屯主、副千九百八人,諸折沖府錄事、府、史一千七百八十二人,校尉三千五百六十四人,執仗、執乘每府三十二人,親事、帳內萬人,集賢院御書手百人,史館典書、楷書四十一人,尚藥童三十人,諸臺、省、寺、監、軍、衛、坊、府之胥史六千餘人。凡此者,皆入官之門戶,而諸司主錄已成官及州縣佐史未敘者,不在焉。



 至於銓選,其制不一,凡流外,兵部、禮部舉人,郎官得自主之,謂之「小選」。太宗時,以歲旱穀貴,東人選者集於洛州,謂之「東選」。高宗上元二年,以嶺南五管、黔中都督府得即任土人,而官或非其才,乃遣郎官、御史為選補使,謂之「南選」。其後江南、淮南、福建大抵因歲水旱,皆遣選補使即選其人。而廢置不常,選法又不著,故不復詳焉。



\end{pinyinscope}