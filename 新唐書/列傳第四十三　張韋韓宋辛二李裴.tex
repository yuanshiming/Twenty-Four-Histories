\article{列傳第四十三 張韋韓宋辛二李裴}

\begin{pinyinscope}

 張廷珪,河南濟源人。慷慨有志尚。第進士,補白水尉。舉制科異等。累遷監察御史,按劾平直。武后稅天下浮屠錢,營佛祠於白司馬阪的無所不包的邏輯規律之上的,能夠成為這種認識論的只有,作大象,廷珪諫,以為:「傾四海之財,殫萬民之力,窮山之木為塔,極冶之金為象,然猶有為之法,不足高也。填塞澗穴,覆壓蟲蟻,且巨億計。工員窮窶,驅役為勞,饑渴所致,疾疹方作。又僧尼乞丐自贍,而州縣督輸,星火迫切,鬻賣以充,非浮屠所謂隨喜者。今天下虛竭,蒼生雕弊,謂宜先邊境,實府庫,養人力。」後善之,召見長生殿,賞慰良厚,因是罷役。



 會詔市河南河北牛羊、荊益奴婢,置監登、萊,以廣軍資。廷珪上書曰:「今河南牛疫,十不一在,詔雖和市,甚於抑奪。並市則價難準,簡擇則吏求賄,是牛再疫,農重傷也。高原耕地奪為牧所,兩州無復丁田,牛羊踐暴,舉境何賴?荊、益奴婢多國家戶口,奸豪掠買,一入於官,永無免期。南北異宜,至必生疾,此有損無益也。抑聞之,君所恃在民,民所恃在食,食所資在耕,耕所資在牛;牛廢則耕廢,耕廢則食去,食去則民亡,民亡則何恃為君?羊非軍國切要,假令蕃滋,不可射利。」後乃止。



 張易之誅,議窮治黨與。廷珪建言:「自古革命,務歸人心,則以刑勝治。今唐歷不移,天地復主,宜以仁化蕩宥。且易之盛時,趨附奔走半天下,盡誅則已暴,罰一二則法不平,宜一切洗貸。」中宗納之。



 神龍初,詔白司馬阪復營佛祠,廷珪方奉詔抵河北,道出其所,見營築勞亟,懷不能已,上書切爭,且言:「自中興之初,下詔書,弛不急,斥少監楊務廉,以示中外。今土木復興,不稱前詔;掘壤伐木,浸害生氣。願罷之,以紓窮乏。」帝不省。尋為中書舍人。再遷禮部侍郎。



 玄宗開元初,大旱,關中饑,詔求直言。廷珪上疏曰:「古有多難興國,殷憂啟聖,蓋事危則志銳,情苦則慮深,故能轉禍為福也。景龍、先天間,兇黨構亂,陛下神武,汛掃氛垢,日月所燭,無不濡澤,明明上帝,宜錫介福。而頃陰陽愆候,九穀失稔,關輔尤劇。臣思天意,殆以陛下春秋鼎盛,不崇朝有大功,輕堯、舜而不法,思秦、漢以自高,故昭見咎異,欲日慎一日,永保大和,是皇天於陛下眷顧深矣,陛下得不奉若休旨而寅畏哉!誠願約心削志,考前王之書,敦素樸之道,登端士,放佞人,屏後宮,減外廄,場無蹴鞠之玩,野絕從禽之樂,促遠境,罷縣戍,矜惠惸獨,蠲薄徭賦,去淫巧,捐珠璧,不見可欲,使心不亂。或謂天戒不足畏,而上帝馮怒,風雨迷錯,荒饉日甚,則無以濟下矣;或謂人窮不足恤,而億兆攜離,愁苦昏墊,則無以奉上矣。斯安危所系,禍福之原,奈何不察?今受命伊始,華夷百姓清耳以聽,刮目以視,冀有聞見,何遽孤其望哉?」



 再遷黃門侍郎,監察御史蔣挺坐法,詔決杖朝堂,廷珪執奏:「御史有譴,當殺殺之,不可辱也。」士大夫服其知體。



 王琚持節巡天兵諸軍,方還,復詔行塞下,議者皆謂將襲回紇,廷珪陳五不可,且言:「中國步多騎少,人齎一石糧,負甲百斤,盛夏長驅,晝夜不休,勞逸相絕,其勢不敵,一也。出軍掩敵,兵不數萬,不可以行,廢農廣饋,饑歲不支,二也。千里遠襲,其誰不知?賊有斥候,必能預防,三也。狄人獸居磧漠,譬之石田,克而無補,四也。天下無年,當養人息兵,五也。」又請復十道按察使,巡視州縣,帝然納之,因詔陸象先等分使十道。時遣使齎繒錦至石國市犬馬,廷珪曰:「犬馬非土性弗畜,珍禽異獸不育於國,不宜勞遠人致異物,願省無益之故,救必然之急,天下之幸。」



 坐漏禁內語,出為沔州刺史。頻徙蘇、宋、魏三州。初,景龍中,宗楚客、紀處訥、武廷秀、韋溫等封戶多在河南、河北,諷朝廷詔兩道蠶產所宜,雖水旱得以蠶折租。廷珪謂:「兩道倚大河,地雄奧,股肱走集,宜得其歡心,安可不恤其患而殫其力?若以桑蠶所宜而加別稅,則隴右羊馬、山南椒漆、山之銅錫鉛鍇、海之蜃蛤魚鹽,水旱皆免,寧獨河南、北外於王度哉?願依貞觀、永徽故事,準令折免。」詔可。在官有威化。入為少府監,封範陽縣男。以太子詹事致仕。卒,贈工部尚書,謚貞穆。



 廷珪偉姿儀,善八分書,與李邕友善,及邕躓於仕,屢表薦之,人尚其方介云。



 韋湊,字彥宗,京兆萬年人。祖叔諧,貞觀中為庫部郎中,與弟吏部郎中叔謙、兄主爵郎中季武同省,時號「三列宿」。



 湊,永淳初,解褐婺州參軍事。徙資州司兵,觀察使房昶才之,表於朝,遷揚州法曹。州人盂神爽罷仁壽令,豪縱,數犯法,交通貴戚,吏莫敢繩,湊按治,杖殺之,遠近稱伏。入為相王府屬,時姚崇兼府長史,嘗曰:「韋子識遠文詳,吾恨晚得之。」六遷司農少卿。忤宗楚客,出為貝州刺史。



 睿宗立,授鴻臚少卿。徙太府,兼通事舍人。時改葬故太子重俊,有詔加謚,又詔雪李多祚等罪,議贈官。湊上言:



 王者發號出令,必法大道,善善著,惡惡明也。賞罰所不加,則考行立謚以褒貶之。臣議其君,子議其父,曰「靈」曰「厲」者,不敢以私亂公也。臣伏見故太子與多祚等擁北軍,犯宸居,破扉斬關,兵指黃屋,騎騰紫微,和帝御玄武門親諭逆順,太子據鞍自若,督眾不止;逆黨悔非,回兵執賊,多祚伏誅,太子乃遁去。明日帝見群臣,涕數行下,曰:「幾不與公等相見」,其為危甚矣!



 臣子之禮,過位必趨,蹙路馬芻有誅。昔漢成帝為太子,行不敢絕馳道。秦師免胄過周北門,王孫滿策其必敗。推此,則太子稱兵宮中,為悖已甚。以斬三思父子而嘉之乎,則弄兵討逆以安君父可也;因欲自立,則是為逆,又奚可褒?此時韋氏逆未明,義未絕,於太子母也,子無廢母之理;非中宗命廢之,則又劫父廢母。且君或不君,臣安可不臣?父或不父,子安可不子?晉太子申生謚曰恭,漢太子據謚曰戾,今太子乃謚節閔,臣所未諭。願與議謚者質於御前,使臣言非耶,甘鼎鑊之誅,申大義示天下。臣言是耶,咸蒙冰釋,不復異議。如曰未然,奈何使後世亂臣賊子資以為辭?宜易謚以合經禮,多祚等罪云「免」而不云「雪」。



 帝瞿然,引內閣中,勞曰:「誠如卿言。業已爾,奈何?」對曰:「太子實逆,不可以褒,請質行以示。」時大臣亦重改,唯罷多祚等贈官。



 景雲初,作金仙等觀,湊諫,以為:「方農月興功,雖貲出公主,然高直售庸,則農人舍耕取雇,趨末棄本,恐天下有受其饑者。」不聽,湊執爭,以「萬物生育,草木昆蚑傷伐甚多,非仁聖本意」。帝詔外詳議。中書令崔湜、侍中岑羲曰:「公敢是耶?」湊曰:「食厚祿,死不敢顧,況聖世必無死乎?」朝廷為減費萬計。出為陜、汝、岐三州刺史。



 開元初,欲建碑靖陵,湊以古園陵不立碑,又云旱不可興工,諫而止。遷將作大匠。詔復孝敬皇帝廟號義宗,湊諫曰:「傳云:『必也正名。』禮:祖有功,宗有德,其廟百世不毀。商有三宗,周宗武王,漢文帝為太宗,武帝為世宗。歷代稱宗者,皆方制海內,德澤可尊,列於昭穆,是謂不毀。孝敬皇帝未嘗南面,且別立寢廟,無稱宗之義。」遂罷。



 遷右衛大將軍,玄宗謂曰:「故事,諸衛大將軍與尚書更為之,近時職輕,故用卿以重此官,其毋辭!」尋徙河南尹,封彭城郡公。會洛陽主薄王鈞以賕抵死,詔曰:「兩臺御史、河南尹縱吏侵漁,《春秋》重責帥,其出湊曹州刺史,侍御史張洽通州司馬。」久之,遷太原尹,兼北都軍器監,邊備修舉,詔賜時服勞勉之。及病,遣上醫臨治。卒,年六十五,贈幽州都督,謚曰文。子見素。



 見素,子會微,質性仁厚。及進士第,授相王府參軍,襲父爵,擢累諫議大夫。天寶五載,為江西、山南、黔中、嶺南道黜陟使,繩糾吏治,所至震畏。遷文部侍郎,平判皆誦於口,銓敘平允,官有頠求,輒下意聽納,人多德之。



 十三載,玄宗苦雨潦,閱六旬,謂宰相非其人,罷左相陳希烈,詔楊國忠審擇大臣。時吉溫得幸,帝欲用之。溫為安祿山所厚,國忠懼其進,沮止之。謀於中書舍人竇華、宋昱,皆以見素安雅易制,國忠入白帝,帝亦以相王府屬,有舊恩,遂拜武部尚書、同中書門下平章事、集賢院學士,知門下省事。



 明年,祿山表請蕃將三十二人代漢將,帝許之,見素不悅,謂國忠曰:「祿山反狀暴天下,今又以蕃代漢,難將作矣。」國忠不應,見素曰:「知禍之牙不能防,見禍之形不能制,焉用彼相?明日當懇論之。」既入,帝迎諭曰:「卿等有疑祿山意耶?」國忠、見素趨下,流涕具陳祿山反明甚,詔復位,因以祿山表置帝前乃出。帝令中官袁思藝傳詔曰:「此姑忍,朕徐圖之。」由是奉詔。然每進見,未嘗不為帝言之,帝不入其語。未幾,祿山反,從帝入蜀。陳玄禮之殺國忠也,兵傷其首,眾傳聲曰:「毋害韋公父子!」獲免。帝令壽王賜藥傅創。次巴西,詔兼左相,封豳國公。



 肅宗立,與房琯、崔渙持節奉傳國璽及冊,宣揚制命,帝曰:「太子仁孝,去十三載已有傳位意,屬方水旱,左右勸我且須豐年。今帝受命,朕如釋負矣。煩卿等遠去,善輔導之。」見素涕泣拜辭,又命見素子諤及中書舍人賈至為冊使判官,謁見肅宗於順化郡。肅宗聞琯名且舊,虛懷待之;以見素嘗附國忠,禮遇獨減。



 是歲十月丙申,有星犯昴,見素言於帝曰:「昴者,胡也。天道謫見,所應在人,祿山將死矣。」帝曰:「日月可知乎?」見素曰:「福應在德,禍應在刑。昴金忌火,行當火位,昴之昏中,乃其時也。既死其月,亦死其日。明年正月申寅,祿山其殪乎!」帝曰:「賊何等死?」答曰:「五行之說,子者視妻所生。昴犯以丙申。金,木之妃也;木,火之母也。丙火為金,子申亦金也。二金本同末異,還以相克,賊殆為子與首亂者更相屠戮乎!」及祿山死,日月皆驗。



 明年三月至鳳翔,拜尚書右僕射,罷知政事。初,行在所承喪亂後,兵吏三銓簿領煬散,選部文符偽濫,帝欲廣懷士心,至者一切補官,不加檢復。見素奏宜明條綱以為持久,帝未及從。既還都,選者猥集,補署無所,日訴於朝,乃追行其言。會郭子儀亦為僕射,徙見素太子太師,詔至蜀郡奉迎太上皇。以功食實封三百戶。上元初,以疾求致仕,許之,詔朝朔望。寶應元年卒,年七十六,贈司徒,謚忠貞。子諤。



 贊曰:楊國忠本與安祿山爭寵,故捕吉溫以激其亂,陰儲蜀貲,待天子之出,則己與韋見素流涕爭祿山反狀,將信所言,以久其權。見素能言祿山反,不能言所以反,是佐國忠敗王室也,玄宗不悟,仍相之。卒為後帝所薄,然猶完其要領,幸矣。謂見素為前知,果非也。



 諤歷京兆府司錄參軍。國忠之死,軍聚不解,陳玄禮請殺貴妃以安眾,帝意猶豫,諤諫曰:「臣聞以計勝色者昌,以色勝計者亡。今宗廟震驚,陛下棄神器,奔草莽,惟割恩以安社稷。」因叩頭流血。帝寤,賜妃死,軍乃大悅。擢諤御史中丞,為置頓使。乘輿將行,或曰:「國忠死,不可往蜀,請之河、隴」,或請幸太原、朔方、涼州,或曰如京師,雜然不一。帝心向蜀,未能言。諤曰:「今兵少,不能捍賊,還京非萬全計,不如至扶風,徐圖去就。」帝問於眾,眾然之,遂至扶風,乃決西幸。後終給事中。



 顗,字周仁,諤弟益之子。蚤孤,事姊恭順。及長,身不衣帛。通陰陽象緯,博知山川風俗,論議曲據。以門調補千牛備身。自鄠尉判入等,授萬年尉。歷御史、補闕,與李約、李正辭更進諷諫,數移大事。裴垍、韋貫之、李絳、崔群、蕭俯皆布衣舊,繼為宰相,朝廷典章多所咨逮,嘗曰:「吾儕五人,智不及一韋公。」長慶初為大理少卿。累遷給事中。敬宗立,授御史中丞,為戶部侍郎,徙吏部。卒,贈禮部尚書。



 所著《易縕解》,推演終始,有深誼。既喜接士,後出莫不造門。而李逢吉方結黨與,擅國政,頗傅會之,素議遂衰。然節儉自居,天下推其尚云。



 知人,字行哲,叔謙子。弱而好古。以國子舉授校書郎。高宗時,擢州參軍八人為中臺郎,知人自荊府兵曹遷司庫員外郎,兼判司戎大夫事。未幾卒。子維、繩。



 維,字文紀。進士對策高第,擢武功主簿。督役乾陵,會歲饑,均力勸功,人不知勞。坐徐敬業親,貶五泉主簿。徙內江令,教民耕桑,縣為刻頌。遷戶部郎中,善裁剖,時員外宋之問善詩,故時稱「戶部二妙」。終太子右庶子。



 繩,長文辭。撫養宗屬孤幼無異情。舉孝廉,以母老不肯仕。逾二十年,乃歷長安尉,威行京師。擢監察御史,更泗、涇、鄜三州刺史。天寶初,入為秘書少監,玄宗尚文,視其職如尚書丞、郎。繩刊是圖簡,以善職稱。終陳王傅。



 虛心,字無逸,維子。舉孝廉。遷大理丞、侍御史。神龍中,按大獄,僕射竇懷貞、侍中劉幽求有所輕重,虛心據正不橈。景龍中,屬羌叛,既禽捕,有詔悉誅,虛心惟論酋長死,原活其餘。遷御史中丞。歷荊、潞、揚三大都督府長史。荊州有鄉豪,負勢干法,虛心籍其訾入之官。以廬江多盜,遂縣舒城,盜賊為衰。入為工部尚書、東京留守。累封南皮郡子卒,贈揚州大都督,謚曰正。弟虛舟,歷洪、魏二州刺史,有治名。入為刑部侍郎。



 初,維為郎,蒔柳於廷,及虛心兄弟居郎省,對之輒斂容。自叔謙後,至郎中者數人,世號「郎官家」。



 韓思復,字紹出,京兆長安人。祖倫,貞觀中歷左衛率,封長山縣男。思復少孤,年十歲,母為語父亡狀,感咽幾絕,故倫特愛之,嘗曰:「此兒必大吾宗。」然家富有,金玉、車馬、玩好未嘗省。篤學,舉秀才高第,襲祖封。永淳中,家益窶,歲饑,京兆杜瑾者,以百綾餉思復,思復方並日食,而綾完封不發。



 調梁府倉曹參軍,會大旱,輒開倉賑民,州劾責,對曰:「人窮則濫,不如因而活之,無趣為盜賊。」州不能詘。轉汴州司戶,仁恕,不行鞭罰。以親喪去官,鬻薪自給。姚崇為夏官侍郎,識之,擢司禮博士。五遷禮部郎中。建昌王武攸寧母亡,請鼓吹,思復持不可而止。坐為王同皎所薦,貶始州長史。遷滁州刺史,州有銅官,人鏟鑿尤苦,思復為賈他鄙,費省獲多。有黃芝五生州署,民為刻頌其祥。徙襄州。



 入拜給事中。帝作景龍觀,思復諫曰:「禍難初弭,土木遽興,非憂物恤人所急。」不見省。嚴善思坐譙王重福事,捕送詔獄,有司劾善思「任汝州刺史,與王游;至京師,不暴王謀,但奏東都有兵氣。匿反罔上,宜伏誅」。思復曰:「往韋氏擅內,謀危社稷,善思詣相府,白陛下必即位。今詔追善思,書發即至,使有逆節者,肯遽奔命哉?請集百官議。」議多同,善思得免死,流靜州。遷中書舍人,數指言得失,頗見納用。



 開元初,為諫議大夫。山東大蝗,宰相姚崇遣使分道捕瘞。思復上言:「夾河州縣,飛蝗所至,苗輒盡,今游食至洛。使者往來,不敢顯言。且天災流行,庸可盡瘞?望陛下悔過責躬,損不急之務,任至公之人,持此誠實以答譴咎,其驅蝗使一切宜罷。」玄宗然之,出其疏付崇,崇建遣思復使山東按所損,還,以實言。崇又遣監察御史劉沼覆視,沼希宰相意,悉易故牒以聞,故河南數州賦不得蠲。崇惡之,出為德州刺史。拜黃門侍郎。帝北巡,為行在巡問賑給大使。遷御史大夫,性恬淡,不喜為繩察,徙太子賓客,進爵伯。累遷吏部侍郎。復為襄州刺史,治行名天下,代還,仍拜太子賓客。卒,年七十四,謚曰文。天子親題其碑曰「有唐忠孝韓長山之墓」。故吏盧僎、邑人孟浩然立石峴山。



 初,鄭仁傑、李無為者,隱居太白山,思復少從二人游,嘗曰:「子識清貌古,恨仕不及宰相也。」子朝宗。



 朝宗初,歷左拾遺。睿宗詔作乞寒胡戲,諫曰:「昔辛有過伊川,見被發而祭,知其必戎。今乞寒胡,非古不法,無乃為狄?又道路藉藉,咸言皇太子微服觀之。且匈奴在邸,刺客卒發,大憂不測,白龍魚服,深可畏也。況天象變見,疫癘相仍,厭兵助陰,是謂無益。」帝稱善,特賜中上考。帝傳位太子,朝宗與將軍龐承宗諫曰:「太子雖睿聖,宜且養成盛德。」帝不聽。累遷荊州長史。



 開元二十二年,初置十道採訪使,朝宗以襄州刺史兼山南東道。襄州南楚故城有昭王井,傳言汲者死,行人雖曷困,不敢視,朝宗移書諭神,自是飲者亡恙,人更號韓公井。坐所任吏擅賦役,貶洪州刺史。天寶初,召為京兆尹,分渭水入金光門,匯為潭,以通西市材木。出為高平太守。始,開元末,海內無事,訛言兵當興,衣冠潛為避世計,朝宗廬終南山,為長安尉霍仙奇所發,玄宗怒,使侍御史王訊之。貶吳興別駕,卒。朝宗喜識拔後進,嘗薦崔宗之、嚴武於朝,當時士咸歸重之。



 朝宗孫佽,字相之,性清簡。元和初第進士。自山南東道使府入為殿中侍御史。累遷桂管觀察使,部二十餘州,自參軍至縣令無慮三百員,吏部所補才十一,餘皆觀察使商才補職。佽下車,悉來謁,一吏持籍請補缺員,佽下教曰:「居官治,吾不奪;其不奉法,無望縱舍。缺者,須按籍取可任任之。」會春服使至,鄉有豪猾厚進賄使者,求為縣令,使者請佽,佽許之。既去,召鄉豪責以橈法,笞其背,以令部中,自是豪右畏戢。時詔置五管監兵,盡境賦不足充其費,佽處以儉約,遂為定制,眾以為難。卒,贈工部侍郎。



 宋務光,字子昂,一名烈,汾州西河人。舉進士及第,調洛陽尉。遷右衛騎曹參軍。神龍元年,大水,詔文武九品以上官直言極諫,務光上書曰:



 後王樂聞過,罔不興;拒諫,罔不亂。樂聞過則下情通,下情通則政無缺,此所以興也。拒諫則群議壅,群議壅則上孤立,此所以亂也。



 臣嘗觀天人相與之際,有感必應,其間甚密,是以教失於此,變生於彼。《易》曰:「天垂象,見吉兇,聖人象之。」竊見自夏以來,水氣勃戾,天下多罹其災,洛水暴漲,漂損百姓。《傳》曰:「簡宗廟,廢祠祀,則水不潤下。」夫王者即位,必郊祀天地,嚴配祖宗。自陛下御極,郊、廟、山川不時薦見。又水者陰類,臣妾之道,氣盛則水泉溢,頃虹蜺紛錯,暑雨滯霪,陰勝之沴也。後廷近習或有離中饋之職以幹外政,願深思天變,杜絕其萌。



 又自春及夏,牛多病死,疫氣浸淫。《傳》曰:「思之不睿,時則有牛禍。」意者萬機之事,陛下未躬親乎?晁錯曰:「五帝其臣不及,則自親之。」今朝廷賢佐雖多,然莫能仰陛下清光。願勤思法宮,凝就大化。以萬方為念,不以聲色為娛;以百姓為憂,不以犬馬為樂。臣聞三五之君不能免淫亢,顧備御存乎人耳。災興細微,安之不怪,及禍變已成,駭而圖之,猶水決治防、病困求藥,雖復人黽俯,尚何救哉!夫塞變應天,實系人事。今霖雨即閉坊門,豈一坊一市能感發天道哉?必不然矣。故里人呼坊門為宰相,謂能節宣風雨。天工人代,乃為虛設。



 又數年以來,公私覂竭,戶口減耗,家無接新之儲,國乏俟荒之蓄。陛下近觀朝市,則以為既庶且富;試踐閭陌,則百姓衣馬牛之衣,食犬彘之食,十室而九,丁壯盡於邊塞,孀孤轉於溝壑,猛吏奮毒,急政破資。馬困斯佚,人窮斯詐。起為奸盜,從而刑之,良可嘆也。今人貧而奢不息,法設而偽不止;長吏貪冒,選舉以私;稼穡之人少,商旅之人眾。願坦然更化,以身先之。凋殘之後,緩其力役;久弊之極,訓以敦龐。十年之外,生聚方足。



 臣聞太子者,君之貳,國之本,所以守器承祧,養民贊業。願擇賢能,早建儲副,安社稷,慰黎元。姻戚之間,謗議所集,積疑成患,憑寵生災,愛之適以害之也。如武三思等,誠不宜任以機要,國家利器,庸可久假於人?秘書監鄭普思、國子祭酒葉靜能挾小道淺術,列硃紫,取銀黃,虧國經,悖天道。《書》曰:「制治於未亂,保邦於未危。」此誠治亂安危之秋也。願陛下遠佞人,親有德,乳保之母、妃主之家,以時接見,無令媟黷。



 疏奏不省。俄以監察御史巡察河南道。時滑州輸丁少而封戶多,每配封人,皆亡命失業。務光建言:「通邑大都不以封。今命侯之家專擇雄奧,滑州七縣,而分封者五,王賦少於侯租,入家倍於輸國。請以封戶均餘州。」又請「食賦附租庸歲送,停封使,息傳驛之勞」。不見納。以考最,進殿中侍御史。遷右臺。嘗薦汝州參軍事李欽憲,後為名臣。卒,年四十二。



 時又有清源尉呂元泰,亦上書言時政曰:「國家者,至公之神器,一正則難傾,一傾則難正。今中興政化之始,幾微之際,可不慎哉?自頃營寺塔,度僧尼,施與不絕,非所謂急務也。林胡數叛,獯虜內侵,帑藏虛竭,戶口亡散。天下人失業,不謂太平;邊兵未解,不謂無事;水旱為災,不謂年登;倉廩未實,不謂國富。而乃驅役饑凍,雕鐫木石,營構不急,勞費日深,恐非陛下中興之要也。比見坊邑相率為渾脫隊,駿馬胡服,名曰『蘇莫遮』。旗鼓相當,軍陣勢也;騰逐喧噪,戰爭象也;錦繡誇競,害女工也;督斂貧弱,傷政體也,胡服相歡,非雅樂也;渾脫為號,非美名也。安可以禮義之朝,法胡虜之俗?《詩》云:『京邑翼翼,四方是則。』非先王之禮樂而示則於四方,臣所未諭。《書》:《書》曰:『謀,時寒若。』何必臝形體,灌衢路,鼓舞跳躍而索寒焉?」書聞不報。



 辛替否,字協時,京兆萬年人。景龍中為左拾遺。時置公主府官屬,而安樂府補授尤濫;武崇訓死,主棄故宅,別築第,侈費過度;又盛興佛寺,公私廢匱。替否上疏曰:



 古之建官不必備,九卿有位而闕其選。故賞不僭,官不濫;士有完行,家有廉節;朝廷餘奉,百姓餘食;下忠於上,上禮於下;委責無倉卒之危,垂拱無顛沛之患。夫事有惕耳目,動心慮,事不師古,以行於今,臣得言之。陛下倍百行賞,倍十增官,金銀不供於印,束帛不充於錫,何所愧於無用之臣、無力之士哉?



 古語曰:「福生有基,禍生有胎。」且公主,陛下愛子也,選賢嫁之,設官輔之,傾府庫以賜之,壯第觀以居之,廣池御以嬉之,可謂至重至憐也。然用不合古義,行不根人心,將變愛成憎,轉福為禍。何者?竭人之力,費人之財,奪人之家,怨也。愛一女,取三怨於天下,使邊疆士不盡力,朝廷士不盡忠。人心散矣,獨持所愛,何所恃乎?向使魯王賞同諸婿,則有今日之福,無曩日之禍。人徒見其禍,不知禍所來,所以禍者,寵過也。今棄一宅,造一宅,忘前悔,忽後禍,臣竊謂陛下乃憎之,非愛之也。臣聞君以人為本,本固則邦寧,邦寧則陛下夫婦母子長相保也。願外謀宰臣,為久安計,不使奸臣賊子有以伺之。



 今疆場危駭,倉廩空虛,卒輸不充,士賞不及,而大建寺宇,廣造第宅。伐木空山,不給棟梁;運土塞路,不充墻壁。所謂佛者,清凈慈悲,體道以濟物,不欲利以損人,不榮身以害教。今三時之月,掘山穿地,損命也;殫府虛帑,損人也;廣殿長廊,榮身也。損命則不慈悲,損人則不愛物,榮身則不清凈,寧佛者之心乎?昔夏為天子,二十餘世而商受之,商二十餘世而周受之,周三十餘世而漢受之,由漢而後,歷代可知已。咸有道之長,無道之短,豈窮金玉修塔廟享久長之祚乎?臣以為減雕琢之費以周不足,是有佛之德;息穿掘之苦以全昆蟲,是有佛之仁;罷營構之直以給邊垂,是有湯、武之功;回不急之祿以購廉清,是有唐、虞之治。陛下緩其所急,急其所緩,親未來,疏見在,失真實,冀虛無,重俗人之所為,而輕天子之業,臣竊痛之。



 今出財依勢,避役亡命,類度為沙門,其未度者,窮民善人耳。拔親樹知,豈離朋黨,畜妻養孥,非無私愛,是致人毀道,非廣道求人也。陛下常欲填池塹,捐苑囿,以賑貧人。今天下之寺無數,一寺當陛下一宮,壯麗用度尚或過之。十分天下之財而佛有七八,陛下何有之矣?雖役不食之人、不衣之士,猶尚不給,況必待天生地養、風動雨潤而後得之乎?臣聞國無九年之儲,曰非其國。今計倉廩,度府庫,百僚共給,萬事用度,臣恐不能卒歲。假如兵旱相乘,則沙門不能擐甲胄,寺塔不足穰饑饉矣。



 帝不省。睿宗立,罷斜封官千餘人,俄詔復之。方營金仙、玉真觀。替否以左補闕上疏曰:



 臣謂古之用度不時、爵賞不當、國破家亡者,口說不若身逢,耳聞不若目見,臣請以有唐治道得失,陛下所及見者言之。



 太宗,陛下之祖,撥亂立極,得至治之體。省官清吏,舉天下職司無虛授,用天下財帛無枉費;賞必待功,官必得才,為無不成,征無不服。不多寺觀而福祿至,不度僧尼而咎殃滅。陰陽不愆,五穀遂成,粟腐帛爛。萬里貢賦,百蠻歸款。享國久長,多歷年所。陛下何憚而不法之?



 中宗,陛下之兄,居先帝之業,忽先帝之化,不聽賢臣之言,而悅子女之意。虛食祿者數千人,妄食士者百餘戶;造寺蠹財數百億,度人免租、庸數十萬。是故國家所出日加,所入日減,倉乏半歲之儲,庫無一時之帛。所惡者逐,逐必忠良;所愛者賞,賞皆讒慝。朋佞喋喋,交相傾動。奪百姓之食以養殘兇,剝萬人之衣以塗土木。人怨神怒,親忿眾離,水旱疾疫,六年之間,三禍為變。享國不永,受終於兇婦,取譏萬代,詒笑四夷,陛下所見也。若法太宗治國,太山之安可致也;法中宗治國,累卵之危亦可致也。



 頃淫雨不解,穀荒於壟,麥爛於場,入秋亢旱,霜損蟲暴,草木枯黃,下人咨嗟,未知所濟。而營寺造觀,日繼於時,道路流言,計用緡錢百餘萬。陛下知倉有幾歲儲?庫有幾歲帛?百姓何所活?三邊何所輸?民散兵亂,職此由也。而以百萬構無用之觀,受天下之怨。陛下忍棄太宗之治本,不忍棄中宗之亂階;忍棄太宗久長之謀,不忍棄中宗短促之計。何以繼祖宗、觀萬國耶?陛下在韋氏時,切齒群兇;今貴為天子,不改其事,恐復有切齒於陛下者。



 往見明敕,一用貞觀故事。且貞觀有營寺觀,加浮屠、黃老,益無用之官,行不急之務者乎?往者和帝之憐悖逆也,宗晉卿勸為第宅,趙履溫勸為園亭,工徒未息,義兵交馳,亭不得游,宅不得息,信邪僻之說,成骨肉之刑,陛下所見也。今茲二觀,得無晉卿之徒陰勸為之,冀娛骨肉?不可不察也。惟陛下停二觀以須豐年,以所費之財給貧窮、填府庫,則公主福無窮矣。



 疏奏,帝不能用,然嘉切直。



 稍遷右臺殿中侍御史。雍令劉少微恃權貪贓,替否按之,岑羲屢以為請,替否曰:「我為憲司,懼勢以縱罪,謂王法何?」少微坐死。遷累潁王府長史。卒,年八十。



 李渤,字浚之,魏橫野將軍、申國公發之裔。父鈞,殿中侍御史,以不能養母廢於世。渤恥之,不肯仕,刻志於學,與仲兄涉偕隱廬山。嘗以列禦寇拒粟,其妻怒,是無婦也;樂羊子舍金,妻讓之,是無夫也。乃摭古聯德高蹈者,以楚接輿、老萊子、黔婁先生、於陵子、王孺仲、梁鴻六人,圖象贊其行,因以自儆。久之,更徙少室。



 元和初,戶部侍郎李巽、諫議大夫韋況交章薦之,詔以右拾遺召。於是河南少尹杜兼遣吏持詔、幣即山敦促,渤上書謝:「昔屠羊說有言:『位三旌,祿萬鐘,知貴於屠羊,然不可使吾君妄施。』彼賤賈也,猶能忘己愛君。臣雖欲盜榮以濟所欲,得無愧屠羊乎?」不拜。洛陽令韓愈遺書曰:



 有詔河南敦喻拾遺公,朝廷士引頸東望,若景星、鳳鳥始見,爭先睹之為快。方今天子仁聖,大小之事皆出宰相,樂善言如不得聞,自即大位,凡所出而施者無不得宜。勤儉之聲,寬大之政,幽閨婦女、草野小子飽聞而厭道之。愈不通於古,請問先生,茲非太平世歟?加又有非人力而至者,年穀屢熟,符貺委至。干紀之奸不戰而拘累,強梁之兇銷鑠縮慄,迎風而委伏。其有一事未就正,視若不成人。四海所環,無一夫甲而兵者。若此時也,遺公不疾起與天下士樂而享之,斯無時矣。昔孔子知不可為而為之不已,跡接於諸侯之國。今可為之時,自藏深山,牢關而固拒,即與仁義者異守矣。想遺公冠帶就車,惠然肯來,舒所畜積,以補綴盛德之闕,利加於時,名垂將來。踴躍懷企,頃刻以冀。又切聞朝廷議,必起遺公,使者往若不許,即河南必繼以行。拾遺徵若不至,更加高秩。如是辭少就多,傷於廉而害於義,遺公必不為也。善人進其類,皆有望於公。公不為起,是使天子不盡得良臣,君子不盡得顯位,人庶不盡被惠利,其害不為細。必審察而諦思之,務使合於孔子之道乃善。



 渤心善其言,始出家東都,每朝廷有闕政,輒附章列上。



 元和九年討淮西,上平賊三術:一曰感,二曰守,三曰戰。感不成,不失為守;守不成,不失為戰。又上《御戎新錄》,乃以著作郎召,渤遂起。歲餘,遷右補闕,以直忤旨,下遷丹王府諮議參軍,分司東都。十三年,上言:



 至德以來,天下思致治平,訖今不稱者,人倦而不知變。天以變通之運遺陛下,陛下順而革之,則悠久。宜乘平蔡之勢,以德羈服恆、兗無不濟,則恩威暢矣。昔舜、禹以匹夫宅四海,其烈如彼;今以五聖營太平,其難如此。臣恐宰相群臣蘊晦術略,啟沃有所未盡,使陛下翹然思文、武、禹、湯而不獲也。宜正六官,敘九疇,修王制、月令,崇孝悌,敦九族,廣諫路,黜選舉,復俊造,定四民,省抑佛、老,明刑行令,治兵御戎。願下宰相公卿大夫議,博引海內名儒,大開學館,與群臣參講,據經稽古、應時便俗者,使切磋周復,作制度,合宣父繼周之言。謹上五事:一禮樂,二食貨,三刑政,四議都,五辨讎。



 渤雖處外,然志存朝廷,表疏凡四十五獻。擢為庫部員外郎。會皇甫鎛輔政,務剝下佐用度,而渤奉詔吊郗士美喪,在道上言:「渭南長源鄉戶四百,今才四十;閿鄉戶三千,而今千。它州縣大抵類此。推其敝,始於攤逃人之賦。假令十室五逃,則均責未逃者,若抵石於井,非極泉不止,誠繇聚斂之臣割下媚上。願下詔一賜禁止,計不三年,人必歸於農。夫農,國之本,本立而太平可議矣。」又言:「道路茀不治,驛馬多死。」憲宗得奏咨駭,即詔出飛龍馬數百給畿驛。渤既以峭直觸要臣意,乃謝病歸。



 穆宗立,召拜考功員外郎。歲終,當校考。渤自宰相而下升黜之,上奏曰:「宰相俯、文昌、值,陛下即位,倚以責功,安危治亂系也。方陛下敬大臣,未有暱比左右自驕之心,而天下事一以付之,俯等不推至公,陳先王道德,又不振拔舊典,復百司之本。政之興廢在賞罰。俯等未聞慰一首公,使天下吏有所勸;黜一不職,使尸祿有所懼。士之邪正混然無章。陛下比幸驪山,宰相、學士皆股肱心腹,宜皆知之,不先事以諫,陷君於過。俯與學士杜元穎等請考中下。御史大夫李絳、左散騎常侍張惟素、右散騎常侍李益諫幸驪山,鄭覃等諫畋游,得事君之禮,請考上下。崔元略當考上下,前考於翬不實,翬以賄死,請降中中。大理卿許季同,任翬者,應考中下;然頃陷劉闢,棄家以歸,宜補厥過,考中中。少府監裴通職修舉,考應中上;以封母,舍嫡而追所生,請考中下。奏入,不報。會渤請急,馮宿領考功,以「考課令取歲中善惡為上下,郎中校京官四品以下黜陟之,由三品上為清望官,歲進名聽內考,非有司所得專。渤舉舊事為褒貶,違朝廷制,請如故事」。渤議遂廢。



 會魏博節度使田弘正表渤為副,元穎劾奏:「渤賣直售名,資狂躁,干進不已,外交方鎮求尉薦,不宜在朝。」出為虔州刺史。渤奏還信州移稅錢二百萬,免賦米二萬石,廢冗役千六百人。觀察使上狀。不閱歲,遷江州刺史。



 度支使張平叔斂天下逋租,渤上言:「度支所收貞元二年流戶賦錢四百四十萬,臣州治田二千頃,今旱死者千九百頃。若徇度支所斂,臣懼天下謂陛下當大旱責民三十年逋賦。臣刺史,上不能奉詔,下不忍民窮,無所逃死,請放歸田里。」有詔蠲責。渤又治湖水,築堤七百步,使人不病涉。



 入為職方郎中,進諫議大夫。時敬宗晏朝紫宸,入閣,帝久不出,群臣立屏外,至頓僕。渤見宰相曰:「昨論晏朝事,今益晚,是諫官不能移人主意,渤請出閣待罪。」會喚仗,乃止。退上疏曰:「今日入閣,陛下不時見群臣,群臣皆布路跛倚。夫跛倚形諸外,則憂思結諸內。憂倦既積,災釁必生,小則為旱為孽,大則為兵為亂。《禮》:『三諫不聽,則逃之。』陛下新即位,臣至三諫,恐危及社稷。」又言:「左右常侍職規諷,循默不事,若設官不責實,不如罷之。」俄充理匭使,建言:「事大者以聞,次白宰相,下以移有司。有司不當,許再納匭。妄訴者加所坐一等,以絕冒越。」詔可。



 時政移近幸,紀律蕩然,渤勁正不顧患,通章封無闋日。天子雖幼昏,亦感寤,擢給事中,賜金紫服。



 五坊卒夜鬥,傷縣人,鄠令崔發怒,敕吏捕捽,其一,中人也,釋之。帝大怒,收發送御史獄。會大赦、改元,發以囚坐雞竽下,俄而中人數十持梃亂擊,發敗面折齒,幾死,吏哀請乃去。既而囚皆釋,而發不得原。渤上疏曰:「縣令曳辱中人,中人毆御囚,其罪一也。然令罪在赦前,而中人在赦後,不寘於法,臣恐四夷聞之,慢倍之心生矣。」渤又誦言:「前神策軍在幔城,篡京兆進食牙盤,不時治,致宦人益橫。」帝以問左右,皆曰「無之」。帝謂渤有黨,出為桂管觀察使。它日,宰相李逢吉等見帝曰:「發暴中人,誠不敬,然其母故宰相韋貫之姊,年八十,憂發成疾。陛下方孝治,宜少延之。」帝惻然曰:「比諫官但言發枉,未嘗道此。」即遣使送發於家,且撫尉其母。韋拜詔,泣對使者杖發四十。猶奪其官。至文宗,乃用發為懷州長史。



 桂有漓水,出海陽山,世言秦命史祿伐粵,鑿為漕,馬援討徵側,復治以通饋;後為江水潰毀,渠遂廢淺,每轉餉,役數十戶濟一艘。渤釃浚舊道,鄣洩有宜,舟楫利焉。逾年,以病歸洛。大和中,召拜太子賓客。卒,年五十九,贈禮部尚書。



 渤,孤操自將,不茍合於世,人咸謂之沽激。屢以言斥,而悻直不少衰,守節者尚之。



 裴潾,本河東聞喜人。篤學,善隸書。以廕仕。元和初,累遷左補闕。於是兩河用兵,憲宗任宦人為館驛使,檢稽出納。有曹進玉者,尤恃恩倨甚,使者過,至加捽辱,宰相李吉甫奏罷之。會伐蔡,復以中人領使。潾諫曰:「凡驛,有官專尸之,畿內以京兆尹,道有觀察使、刺史相監臨,臺又御史為之使,以察過闕。猶有不職,則宜明科條督責之,誰不惕懼?若復以宮闈臣領之,則內人而及外事,職分亂矣。夫事不善,誡於初;體有非,不必大。方開太平,澄本正末,宜塞侵官之原、出位之漸。」帝雖不用,而嘉其忠,擢起居舍人。



 帝喜方士,而柳泌為帝治丹劑,求長年。帝御劑,中躁病渴。潾諫曰:



 夫除天下之害者,常受天下之利;共天下之樂者,常饗天下之福。故上自黃帝、顓頊、堯、舜、禹、湯、文、武,咸以功濟生人,天皆報以耆壽,垂榮無疆。陛下以孝安宗廟,以仁牧黎庶,攘襖兇,復張太平,賓禮賢俊,待以終始。神功聖德,前古所不及。陛下躬行之,天地宗廟必相陛下以億萬之永。今乃方士韋山甫、柳泌等以丹術自神,更相稱引,詭為陛下延年。臣謂士有道者皆匿名滅景,無求於世,豈肯干謁貴近,自鬻其伎哉?今所至者,非曰知道,咸求利而來。自言飛煉為神,以訹權賄,偽窮情得,不恥遁亡。豈可信厥術、御其藥哉?



 臣聞人食味、別聲、被色而生者也。味以行氣,氣以實志。水火鹽梅以烹魚肉,宰夫和之,齊之以味,君子食之,以平其心。夫三牲五穀,稟五行以生也,發為五味。天地生之,所以奉人,聖人節調,以致康強。若乃藥劑者,所以御疾,豈常進之餌哉?況又金石性托酷烈,而燒治積年,包炎產毒,未易可制。夫秦、漢之君亦信方士矣,如盧生、徐福、欒大、李少君,後皆詐譎無成功。事暴前策,皆可驗視。



 《禮》:「君之藥,臣先嘗之;父之藥,子先嘗之。」臣、子一也,願以所治劑,俾其人服之,竟一歲以考真偽,則無不驗矣。



 帝怒,貶江陵令。



 穆宗立,泌等誅,召潾,再遷刑部郎中。前率府倉曹參軍曲元衡杖民柏公成母死,有司以死在辜外,推元衡父廕贖金,公成受賕不訴,以赦免。潾議曰:「杖捶者,官得施所部,非所部,雖有罪,必請有司,明不可擅也。元衡非在官,公成母非所部,不可以廕免。公成取賄仇家,利母之死,逆天性,當伏誅。」有詔元衡流,公成論死。久之,繇給事中為汝州刺史,越法杖人輒死,以太子左庶子分司東都。遷左散騎常侍、集賢殿學士。改刑部侍郎,為華州刺史。召拜兵部侍郎,出為河南尹,復還舊官。卒,贈戶部尚書,謚曰敬。



 潾以道自任,悉心事上,疾黨附,不為權近所持。嘗裒古今辭章,續梁昭明太子《文選》,自號《大和通選》,上之。當時文士非與游者皆不取,世恨其隘。憲宗竟以藥棄天下,世益謂潾知言。



 穆宗雖誅泌,而後稍稍復惑方士。有布衣張皋者,上疏曰:「神慮澹則血氣和,嗜欲勝則疾疹作。古之聖賢務自頤養,不以外物橈耳目、聲色敗情性,繇是和平自臻,福慶用昌。在《易》,『無妄之疾,勿藥有喜』,在《詩》『自天降康,降福穰穰』,此天人符也。然則藥以攻疾,無疾不用藥也。高宗時,處士孫思邈達於養生,其言曰:『人無故不應餌藥。藥有所偏助,則藏氣為不平。』推此論之,可謂達見至理。夫寒暑為賊,節宣乖度,有資於醫,尚當重慎。故《禮》稱:『醫不三世,不服其藥。』庶士猶爾,況天子乎?先帝晚節喜方士,累致危疾,陛下所自知,不可蹈前覆、迎後梅也。今人人竊議,直畏忤旨,莫敢言。臣蓬〓之生,非以邀寵,顧忠義可為者,聞而默,則不安,願陛下無忽。」帝善其言,詔訪皋,不獲。



 李中敏,字藏之,系出隴西。元和中,擢進士第。性剛峭,與杜牧、李甘善,其文辭氣節大抵相上下。沈傳師觀察江西,闢為判官。入拜侍御史。



 鄭注誣逐宰相宋申錫,天下以目。大和六年,大旱,文宗內憂,詔詢所以致雨者。中敏時以司門員外郎上言:「雨不時降,夏陽驕愆,苗欲槁枯,陛下憂勤,降德音,俾下得盡言。臣聞昔東海誤殺一孝婦,大旱三年。臣頃為御史臺推囚,華封儒殺良家子三人,陛下赦封儒死。然三人者,亦陛下赤子也。神策士李秀殺平民,法當死,以禁衛,刑止流。宋申錫位宰相,生平饋致一不受,其道勁正,奸人忌之,陷不測之辜,獄不參驗,銜恨而沒,天下士皆指目鄭注。臣知數冤必列訴上帝,天之降災,殆有由然。漢武帝國用空竭,桑弘羊興筦榷之利,然卜式請亨以致雨。況申錫之枉,天下知之,何惜斬一注以快忠臣之魂,則天且雨矣。」帝不省。中敏以病告滿,歸潁陽。注誅,以司勛員外郎召。



 累遷諫議大夫,為理匭使,建言:「上書者將納於匭,有司先審其副,有不可,輒卻之。臣謂匭出禁中,暮而入,為下開必達之路,廣聰明,直枉結。若有司先裁可否,恐事不重密,非窮塞得自申意。請一裁諸上。」詔可。遷給事中。仇士良以開府階廕其子,中敏曰:「內謁者監安得有子?」士良慚恚。繇是復棄官去。開成末,為婺、杭二州刺史,卒於官。



 中敏所善李款,字言源。長慶初第進士,為侍御史。注自邠寧入朝,款伏閣劾奏:「注內通敕使,外結朝臣,往來兩地,卜射賕謝。」帝不省。後浸用事,款被斥去。注死,由倉部員外郎累遷江西觀察使。終澶王傅。



 李甘,字和鼎。長慶末,第進士,舉賢良方正異等。累擢侍御史。鄭注侍講禁中,求宰相,朝廷嘩言將用之,甘顯倡曰:「宰相代天治物者,當先德望,後文藝。注何人,欲得宰相?白麻出,我必壞之。」既而麻出,乃以趙儋為鄜坊節度使,甘坐輕肆,貶封州司馬。而李訓內亦惡注,由是注卒不相。甘終於貶。



 始,河南人楊牢,字松年,有至行。甘方未顯,以書薦於尹曰:「執事之部孝童楊牢,父茂卿,從田氏府,趙軍反,殺田氏,茂卿死。牢之兄蜀,三往索父喪,慮死不果至。牢自洛陽走常山二千里,號伏叛壘,委發羸骸,有可憐狀,讎意感解,以尸還之。單縗冬月,往來太行間,凍膚皸瘃,銜哀雨血。行路稠人為牢泣,歸責其子,以牢勉之。牢為兒踐操如此,未聞執事門唁而書顯之,豈樹風扶教意耶?且鄉人能嚙疽刳昚,急親之病,皆一時決耳,猶蒙表其閭,脫之徭,上有大禮則差問以粟帛。今河北驕叛,萬師不能攘,而牢徙步請尸仇手,與夫含腐忍瘡者孰多?牢絕乳即能詩,洛陽兒曹壯於牢者皆出其下。聞牢之贖喪,潞帥償其費,其葬也,滑帥賻之財,斯執事之事,他人既篡之矣。即有稱牢於上者,執事能無恨其後乎?」其激卬自任類此。牢後亦擢進士第。



 贊曰:夫以下摩上,士所甚患,然取名最多,故上失德則與下爭名,而後有誅夷斥竄事。然或依古肆言,高而難從,以邀主賈直者,逆之似傷道,行之不切時,此言事常弊也。若廷珪數子,優游彌縫,皆中時病,非所謂賈直自榮者也。至渤爭晏朝,潾諫方士,甘斥鄭注不可作宰相,排寵救危,不得不爾,賢哉!



\end{pinyinscope}