\article{志第四十 兵}

\begin{pinyinscope}

 古之有天下國家者,其興亡治亂,未始不以德,而自戰國、秦、漢以來,鮮不以兵。夫兵豈非重事哉!然其因時制變張取實予名。反對以名取實。莊子認為:「名者實之賓。」後,以茍利趨便,至於無所不為,而考其法制,雖可用於一時,而不足施於後世者多矣,惟唐立府兵之制,頗有足稱焉。



 蓋古者兵法起於井田,自周衰,王制壞而不復;至於府兵,始一寓之於農,其居處、教養、畜材、待事、動作、休息,皆有節目,雖不能盡合古法,蓋得其大意焉,此高祖、太宗之所以盛也。至其後世,子孫驕弱,不能謹守,屢變其制。夫置兵所以止亂,及其弊也,適足為亂;又其甚也,至困天下以養亂,而遂至於亡焉。



 蓋唐有天下二百餘年,而兵之大勢三變:其始盛時有府兵,府兵後廢而為廣騎,彍騎又廢,而方鎮之兵盛矣。及其末也,強臣悍將兵布天下,而天子亦自置兵於京師,曰禁軍。其後天子弱,方鎮強,而唐遂以亡滅者,措置之勢使然也。若乃將卒、營陣、車旗、器械、征防、守衛,凡兵之事不可以悉記,記其廢置、得失、終始、治亂、興滅之跡,以為後世戒云。



 府兵之制,起自西魏、後周,而備於隋,唐興因之。隋制十二衛,曰翊衛,曰驍騎衛,曰武衛,曰屯衛,曰御衛,曰候衛,為左右,皆有將軍以分統諸府之兵。府有郎將、副郎將、坊主、團主,以相統治。又有驃騎、車騎二府,皆有將軍。後更驃騎曰鷹揚郎將,車騎曰副郎將。別置折沖、果毅。



 自高祖初起,開大將軍府,以建成為左領大都督,領左三軍,敦煌公為右領大都督,領右三軍,元吉統中軍。發自太原,有兵三萬人。及諸起義以相屬與降群盜,得兵二十萬。武德初,始置軍府,以驃騎、車騎兩將軍府領之。析關中為十二道,曰萬年道、長安道、富平道、醴泉道、同州道、華州道、寧州道、岐州道、豳州道、西麟州道、涇州道、宜州道,皆置府。三年,更以萬年道為參旗軍,長安道為鼓旗軍,富平道為玄戈軍,醴泉道為井鉞軍,同州道為羽林軍,華州道為騎官軍,寧州道為折威軍,岐州道為平道軍,豳州道為招搖軍,西麟州道為苑游軍,涇州道為天紀軍,宜州道為天節軍;軍置將、副各一人,以督耕戰,以車騎府統之。六年,以天下既定,遂廢十二軍,改驃騎曰統軍,車騎曰別將。居歲餘,十二軍復,而軍置將軍一人,軍有坊,置主一人,以檢察戶口,勸課農桑。



 太宗貞觀十年,更號統軍為折沖都尉,別將為果毅都尉,諸府總曰折沖府。凡天下十道,置府六百三十四,皆有名號,而關內二百六十有一,皆以隸諸衛。凡府三等:兵千二百人為上,千人為中,八百人為下。府置折沖都尉一人,左右果毅都尉各一人,長史、兵曹、別將各一人,校尉六人。士以三百人為團,團有校尉;五十人為隊,隊有正;十人為火,火有長。火備六馱馬。凡火具烏布幕、鐵馬盂、布槽、鍤、钁、鑿、碓、筐、斧、鉗、鋸皆一,甲床二,鎌二;隊具火金贊一,胸馬繩一,首羈、足絆皆三;人具弓一,矢三十,胡祿、橫刀、礪石、大觿、氈帽、氈裝、行藤皆一,麥飯九斗,米二斗,皆自備,並其介胄、戎具藏於庫。有所征行,則視其入而出給之。其番上宿衛者,惟給弓矢、橫刀而已。



 凡民年二十為兵,六十而免。其能騎而射者為越騎,其餘為步兵、武騎、排手、步射。



 每歲季冬,折沖都尉率五校兵馬之在府者,置左右二校尉,位相距百步。每校為步隊十,騎隊一,皆卷槊幡,展刃旗,散立以俟。角手吹大角一通,諸校皆斂人騎為隊;二通,偃旗槊,解幡;三通,旗槊舉。左右校擊鼓,二校之人合噪而進。右校擊鉦,隊少卻,左校進逐至右校立所;左校擊鉦,少卻,右校進逐至左校立所;右校復擊鉦,隊還,左校復薄戰;皆擊鉦,隊各還。大角復鳴一通,皆卷幡、攝矢、弛弓、匣刃;二通,旗槊舉,隊皆進;三通,左右校皆引還。是日也,因縱獵,獲各入其人。其隸於衛也,左、右衛皆領六十府,諸衛領五十至四十,其餘以隸東宮六率。



 凡發府兵,皆下符契,州刺史與折沖勘契乃發。若全府發,則折沖都尉以下皆行;不盡,則果毅行;少則別將行。當給馬者,官予其直市之,每匹予錢二萬五千。刺史、折沖、果毅歲閱不任戰事者鬻之,以其錢更市,不足則一府共足之。



 凡當宿衛者番上,兵部以遠近給番,五百里為五番,千里七番,一千五百里八番,二千里十番,外為十二番,皆一月上。若簡留直衛者,五百里為七番,千里八番,二千里十番,外為十二番,亦月上。



 先天二年誥曰:「往者分建府衛,計戶充兵,裁足周事,二十一入募,六十一出軍,多憚勞以規避匿。今宜取年二十五以上,五十而免。屢征鎮者,十年免之。」雖有其言,而事不克行。玄宗開元六年,始詔折沖府兵每六歲一簡。自高宗、武后時,天下久不用兵,府兵之法浸壞,番役更代多不以時,衛士稍稍亡匿,至是益耗散,宿衛不能給。宰相張說乃請一切募士宿衛。十一年,取京兆、蒲、同、岐、華府兵及白丁,而益以潞州長從兵,共十二萬,號「長從宿衛」,歲二番,命尚書左丞蕭嵩與州吏共選之。明年,更號曰「彍騎」。又詔:「諸州府馬闕,官私共補之。今兵貧難致,乃給以監牧馬。」然自是諸府士益多不補,折沖將又積歲不得遷,士人皆恥為之。



 十三年,始以彍騎分隸十二衛,總十二萬,為六番,每衛萬人。京兆彍騎六萬六千,華州六千,同州九千,蒲州萬二千三百,絳州三千六百,晉州千五百,岐州六千,河南府三千,陜、虢、汝、鄭、懷、汴六州各六百,內弩手六千。其制:皆擇下戶白丁、宗丁、品子強壯五尺七寸以上,不足則兼以戶八等五尺以上,皆免征鎮、賦役,為四籍,兵部及州、縣、衛分掌之。十人為火,五火為團,皆有首長。又擇材勇者為番頭,頗習弩射。又有習林軍飛騎,亦習弩。凡伏遠弩自能施張,縱矢三百步,四發而二中;擘張弩二百三十步,四發而二中;角弓弩二百步,四發而三中;單弓弩百六十步,四發而二中:皆為及第。諸軍皆近營為堋,士有便習者,教試之,及第者有賞。



 自天寶以後,彍騎之法又稍變廢,士皆失拊循。八載,折沖諸府至無兵可交,李林甫遂請停上下魚書。其後徒有兵額、官吏,而戎器、馱馬、鍋幕、糗糧並廢矣,故時府人目番上宿衛者曰侍官,言侍衛天子;至是,衛佐悉以假人為童奴,京師人恥之,至相罵辱必曰侍官。而六軍宿衛皆市人,富者販繒彩、食粱肉,壯者為角牴、拔河、翹木、扛鐵之戲,及祿山反,皆不能受甲矣。



 初,府兵之置,居無事時耕於野,其番上者,宿衛京師而已。若四方有事,則命將以出,事解輒罷,兵散於府,將歸於朝。故士不失業,而將帥無握兵之重,所以防微漸、絕禍亂之萌也。及府兵法壞而方鎮盛,武夫悍將雖無事時,據要險,專方面,既有其土地,又有其人民,又有其甲兵,又有其財賦,以布列天下。然則方鎮不得不強,京師不得不弱,故曰措置之勢使然者,以此也。



 夫所謂方鎮者,節度使之兵也。原其始,起於邊將之屯防者。唐初,兵之戍邊者,大曰軍,小曰守捉,曰城,曰鎮,而總之者曰道:若盧龍軍一,東軍等守捉十一,曰平盧道;橫海、北平、高陽、經略、安塞、納降、唐興、渤海、懷柔、威武、鎮遠、靜塞、雄武、鎮安、懷遠、保定軍十六,曰範陽道;天兵、大同、天安、橫野軍四,岢嵐等守捉五,曰河東道;朔方經略、豐安、定遠、新昌、天柱、宥州經略、橫塞、天德、天安軍九,三受降、豐寧、保寧、烏延等六城,新泉守捉一,曰關內道;赤水、大斗、白亭、豆盧、墨離、建康、寧寇、玉門、伊吾、天山軍十,烏城等守捉十四,曰河西道;瀚海、清海、靜塞軍三,沙缽等守捉十,曰北庭道;保大軍一,鷹娑都督一,蘭城等守捉八,曰安西道;鎮西、天成、振威、安人、綏戎、河源、白水、天威、榆林、臨洮、莫門、神策、寧邊、威勝、金天、武寧、曜武、積石軍十八,平夷、綏和、合川守捉三,曰隴右道;威戎、安夷、昆明、寧遠、洪源、通化、松當、平戎、天保、威遠軍十,羊灌田等守捉十五,新安等城三十二,犍為等鎮三十八,曰劍南道;嶺南、安南、桂管、邕管、容管經略、清海軍六,曰嶺南道;福州經略軍一,曰江南道;平海軍一,東牟、東萊守捉二,蓬萊鎮一,曰河南道。此自武德至天寶以前邊防之制。其軍、城、鎮、守捉皆有使,而道有大將一人,曰大總管,已而更曰大都督。至太宗時,行軍征討曰大總管,在其本道曰大都督。自高宗永徽以後,都督帶使持節者,始謂之節度使,猶猶未以名官。景雲二年,以賀拔延嗣為涼州都督、河西節度使。自此而後,接乎開元,朔方、隴右、河東、河西諸鎮,皆置節度使。



 及範陽節度使安祿山反,犯京師,天子之兵弱,不能抗,遂陷兩京。肅宗起靈武,而諸鎮之兵共起誅賊。其後祿山子慶緒及史思明父子繼起,中國大亂,肅宗命李光弼等討之,號「九節度之師」。久之,大盜既滅,而武夫戰卒以功起行陣,列為侯王者,皆除節度使。由是方鎮相望於內地,大者連州十餘,小者猶兼三四。故兵驕則逐帥,帥強則叛上。或父死子握其兵而不肯代;或取舍由於士卒,往往自擇將吏,號為「留後」,以邀命於朝。天子顧力不能制,則忍恥含垢,因而撫之,謂之姑息之政。蓋姑息起於兵驕,兵驕由由方鎮,姑息愈甚,而兵將愈俱驕。由是號令自出,以相侵擊,虜其將帥,並其土地,天子熟視不知所為,反為和解之,莫肯聽命。



 始時為朝廷患者,號「河朔三鎮」。及其末,硃全忠以梁兵、李克用以晉兵更犯京師,而李茂貞、韓建近據岐、華,妄一喜怒,兵已至於國門,天子為殺大臣、罪己悔過,然後去。及昭宗用崔胤召梁兵以誅宦官,劫天子奔岐,梁兵圍之逾年。當此之時,天下之兵無復勤王者。向之所謂三鎮者,徒能始禍而已。其他大鎮,南則吳、浙、荊、湖、閩、廣,西則岐、蜀,北則燕、晉,而梁盜據其中,自國門以外,皆分裂於方鎮矣。



 故兵之始重於外也,土地、民賦非天子有;既其盛也,號令、徵代非其有;又其甚也,至無尺土,而不能庇其妻子宗族,遂以亡滅。語曰:「兵猶火也,弗戢將自焚。」夫惡危亂而欲安全者,庸君常主之能知,至於措置之失,則所謂困天下以養亂也。唐之置兵,既外柄以授人,而末大本小,方區區自為捍衛之計,可不哀哉!



 夫所謂天子禁軍者,南、北衙兵也。南衙,諸衛兵是也;北衙者,禁軍也。



 初,高祖以義兵起太原,已定天下,悉罷遣歸,其願留宿衛者三萬人。高祖以渭北白渠旁民棄腴田分給之,號「元從禁軍」。後老不任事,以其子弟代,謂之「父子軍」。及貞觀初,太宗擇善射者百人,為二番於北門長上,曰「百騎」。以從田獵。又置北衙七營,選材力驍壯,月以一營番上。十二年,始置左右屯營於玄武門,領以諸衛將軍,號「飛騎」,其法:取戶二等以上、長六尺闊壯者,試弓馬四次上、翹關舉五、負米五斛行三十步者。復擇馬射為百騎,衣五色袍,乘六閑駁馬,虎皮韉,為游幸翊衛。



 高宗龍朔二年,始取府兵越騎、步射置左右羽林軍,大朝會則執仗以衛階陛,行幸則夾馳道為內仗。武後改百騎曰「千騎」。中宗又改千騎曰「萬騎」,分左、右營。及玄宗以萬騎平韋氏,改為左右龍武軍,皆用唐元功臣子弟,制若宿衛兵。是時,良家子避征戍者,亦皆納資隸軍,分日更上如羽林。開元十二年,詔左右羽林軍、飛騎闕,取京旁州府士,以戶部印印其臂,為二籍,羽林、兵部分掌之。末年,禁兵浸耗,及祿山反,天子西駕,禁軍從者裁千人,肅宗赴靈武,士不滿百,及即位,稍復調補北軍。至德二載,置左右神武軍,補元從、扈從官子弟,不足則取它色,帶品者同四軍,亦曰「神武天騎」,制如羽林。總曰「北衙六軍」。又擇便騎射者置衙前射生手千人,亦曰「供奉射生官」,又曰「殿前射生」,分左、右廂,總號曰「左右英武軍」。乾元元年,李輔國用事,請選羽林騎士五百人邀巡。李揆曰:「漢以南、北軍相制,故周勃以北軍安劉氏。朝廷置南、北衙,文武區列,以相察伺。今用羽林代金吾警,忽有非常,何以制之?」遂罷。



 上元中,以北衙軍使衛伯玉為神策軍節度使,鎮陜州,中使魚朝恩為觀軍容使,監其軍。初,哥舒翰破吐蕃臨洮西之磨環川,即其地置神策軍,以成如璆為軍使。及祿山反,如璆以伯玉將兵千人赴難,伯玉與朝恩皆屯於陜。時邊土陷蹙,神策故地淪沒,即詔伯玉所部兵,號「神策軍」,以伯玉為節度使,與陜州節度使郭英乂皆鎮陜。其後伯玉罷,以英乂兼神策軍節度。英乂入為僕射,軍遂統於觀軍容使。



 代宗即位,以射生軍入禁中清難,皆賜名「寶應功臣」,故射生軍又號「寶應軍」。廣德元年,代宗避吐蕃幸陜,朝恩舉在陜兵與神策軍迎扈,悉號「神策軍」。天子幸其營。及京師平,朝恩遂以軍歸禁中,自將之,然尚未與北軍齒也。永泰元年,吐蕃復入寇,朝恩又以神策軍屯苑中,自是浸盛,分為左、右廂,勢居北軍右,遂為天子禁軍,非它軍比。朝恩乃以觀軍容宣慰處置使知神策軍兵馬使。大歷四年,請以京兆之好畤,鳳翔之麟游、普潤,皆隸神策軍。明年,復以興平、武功、扶風、天興隸之,朝廷不能遏。又用愛將劉希暹為神策虞候,主不法,遂置北軍獄,募坊市不逞,誣捕大姓,沒產為賞,至有選舉旅寓而挾厚貲多橫死者。朝恩得罪死,以希暹代為神策軍使。是歲,希暹復得罪,以朝恩舊校王駕鶴代將。十數歲,德宗即位,以白志貞代之。是時,神策兵雖處內,而多以裨將將兵征伐,往往有功。



 及李希烈反,河北盜且起,數出禁軍征伐,神策之士多鬥死者。建中四年下詔募兵,以志貞為使,搜補峻切。郭子儀之婿端王傅吳仲孺殖貲累巨萬,以國家有急不自安,請以子率奴馬從軍。德宗喜甚,為官其子五品。志貞乃請節度、都團練、觀察使與世嘗任者家,皆出子弟馬奴裝鎧助征,授官如仲孺子。於是豪富者緣為幸,而貧者苦之。神策兵既發殆盡,志貞陰以市人補之,名隸籍而身居市肆。及涇卒潰變,皆戢伏不出,帝遂出奔。初,段秀實見禁兵寡弱,不足備非常,上疏曰:「天子萬乘,諸侯千,大夫百,蓋以大制小,十制一也,尊君卑臣強幹弱支之道。今外有不廷之虜,內有梗命之臣,而禁兵不精,其數削少,後有猝故,何以待之?猛虎所以百獸畏者,爪牙也,爪牙廢,則孤豚特犬悉能為敵。願少留意。」至是方以秀實言為然。



 及志貞等流貶,神策都虞候李晟與其軍之它將,皆自飛狐道西兵赴難,遂為神策行營節度,屯渭北,軍遂振。貞元二年,改神策左右廂為左右神策軍,特置監句當左右神策軍,以寵中官,而益置大將軍以下。又改殿前射生左右廂曰殿前左右射生軍,亦置大將軍以下。三年,詔射生、神策、六軍將士,府縣以事辦治,先奏乃移軍,勿輒逮捕。京兆尹鄭叔則建言:「京劇輕猾所聚,慝作不常,俟奏報,將失罪人,請非昏田,皆以時捕。」乃可之。俄改殿前左右射生軍曰左右神威軍,置監左右神威軍使。左右神策軍皆加將軍二員,左右龍武軍加將軍一員,以待諸道大將有功者。



 自肅宗以後,北軍增置威武、長興等軍,名類頗多,而廢置不一。惟羽林、龍武、神武、神策、神威最盛,總曰左右十軍矣。其後京畿之西,多以神策軍鎮之,皆有屯營。軍司之人,散處甸內,皆恃勢凌暴,民間苦之。自德宗幸梁還,以神策兵有勞,皆號「興元元從奉天定難功臣」,恕死罪。中書、御史府、兵部乃不能歲比其籍,京兆又不敢總舉名實。三輔人假比於軍,一牒至十數。長安奸人多寓占兩軍,身不宿衛,以錢代行,謂之納課戶。益肆為暴,吏稍禁之,輒先得罪,故當時京尹、赤令皆為之斂屈。十年,京兆尹楊於陵請置挾名敕,五丁許二丁居軍,餘差以條限,繇是豪強少畏。



 十二年,以監句當左神策軍、左監門衛大將軍、知內侍省事竇文場為左神策軍護軍中尉,監句當右神策軍、右監門衛將軍、知內侍省事霍仙鳴為右神策軍護軍中尉,監右神威軍使、內侍兼內謁者臨張尚進為右神威軍中護軍,監左神威軍使、內侍兼內謁者監焦希望為左神威軍中護軍。護軍中尉、中護軍皆古官,帝既以禁衛假宦官,又以此寵之。十四年,又詔左右神策置統軍,以崇親衛,如六軍。時邊兵衣饟多不贍,而戍卒屯防,藥茗蔬醬之給最厚。諸將務為詭辭,請遙隸神策軍,稟賜遂贏舊三倍,繇是塞上往往稱神策行營,皆內統於中人矣,其軍乃至十五萬。



 故事,京城諸司、諸使、府、縣,皆季以御史巡囚。後以北軍地密,未嘗至。十九年,監察御史崔薳不知近事,遂入右神策,中尉奏之,帝怒,杖薳四十,流崖州。



 順宗即位,王叔文用事,欲取神策兵柄,乃用故將範希朝為左右神策、京西諸城鎮行營兵馬節度使,以奪宦者權,而不克。元和二年,省神武軍。明年,又廢左右神威軍,合為一,曰「天威軍」。八年,廢天威軍,以其兵騎分隸左右神策軍。及僖宗幸蜀,田令孜募神策新軍為五十四都,離為十軍,令孜自為左右神策十軍兼十二衛觀軍容使,以左右神策大將軍為左右神策諸都指揮使,諸都又領以都將,亦曰「都頭」。



 景福二年,昭宗以籓臣跋扈、天子孤弱,議以宗室典禁兵。及伐李茂貞,乃用嗣覃王允為京西招討使,神策諸都指揮使李金歲副之,悉發五十四軍屯興平,已而兵自潰。茂貞逼京師,昭宗為斬神策中尉西門重遂、李周言童,乃引去。乾寧元年,王行瑜、韓建及茂貞連兵犯闕,天子又殺宰相韋昭度、李磎,乃去。太原李克用以其兵伐行瑜等,同州節度使王行實入迫神策中尉駱全瓘、劉景宣請天子幸邠州,全瓘、景宣及子繼晟與行實縱火東市,帝御承天門,敕諸王率禁軍捍之。捧日都頭李筠以其軍衛樓下,茂貞將閻圭攻筠,矢及樓扉,帝乃與親王、公主幸筠軍,扈蹕都頭李君實亦以兵至,侍帝出幸莎城、石門。詔嗣薛王知柔入長安收禁軍、清宮室,月餘乃還。又詔諸王閱親軍,收拾神策亡散,得數萬。益置安聖、捧宸、保寧、安化軍,曰「殿後四軍」,嗣覃王允與嗣延王戒丕將之。三年,茂貞再犯闕,嗣覃王戰敗,昭宗幸華州。明年,韓建畏諸王有兵,請皆歸十六宅,留殿後兵三十人,為控鶴排馬官,隸飛龍坊,餘悉散之,且列甲圍行宮,於是四軍二萬餘人皆罷。又請誅都頭李筠,帝恐,為斬於大雲橋。俄遂殺十一王。



 及還長安,左右神策軍復稍置之,以六千人為定。是歲,左右神策中尉劉季述、王仲先以其兵千人廢帝,幽之。季述等誅。已而昭宗召硃全忠兵入誅宦官,宦官覺,劫天子幸鳳翔。全忠圍之歲餘,天子乃誅中尉韓全誨、張弘彥等二十餘人,以解梁兵,乃還長安。於是悉誅宦官,而神策左右軍繇此廢矣。諸司悉歸尚書省郎官,兩軍兵皆隸六軍者,而以崔胤判六軍十二衛事。六軍者,左右龍武、神武、羽林,其名存而已。自是軍司以宰相領。



 及全忠歸,留步騎萬人屯故兩軍,以子友倫為左右軍宿衛都指揮使,禁衛皆汴卒。崔胤乃奏:「六軍名存而兵亡,非所以壯京師。軍皆置步軍四將,騎軍一將。步將皆兵二百五十人,騎將皆百人,總六千六百人。番上如故事。」乃令六軍諸衛副使京兆尹鄭元規立格募兵於市,而全忠陰以汴人應之。胤死,以宰相裴樞判左三軍,獨孤損判右三軍,向所募士悉散去。全忠亦兼判左右六軍十二衛。及東遷,唯小黃門打球供奉十數人、內園小兒五百人從。至谷水,又盡屠之,易以汴人,於是天子無一人之衛。昭宗遇弒,唐乃亡。



 馬者,兵之用也;監牧,所以蕃馬也,其制起於近世。唐之初起,得突厥馬二千匹,又得隋馬三千於赤岸澤,徙之隴右,監牧之制始於此。其官領以太僕,其屬有牧監、副監。監有丞,有主簿、直司、團官、牧尉、排馬、牧長、群頭,有正,有副。凡群置長一人,十五長置尉一人,歲課功,進排馬。又有掌閑,調馬習上。又以尚乘掌天子之御。左右六閑:一曰飛黃,二曰吉良,三曰龍媒,四曰騊餘,五曰駃騠,六曰天苑。總十有二閑為二廄,一曰祥驎,二曰鳳苑,以系飼之。其後禁中又增置飛龍廄。



 初,用太僕少卿張萬歲領群牧。自貞觀至麟德四十年間,馬七十萬六千,置八坊岐、豳、涇、寧間,地廣千里:一曰保樂,二曰甘露,三曰南普閏,四曰北普閏,五曰岐陽,六曰太平,七曰宜祿,八曰安定。八坊之田,千二百三十頃,募民耕之,以給芻秣。八坊之馬為四十八監,而馬多地狹不能容,又析八監列布河曲豐曠之野。凡馬五千為上監,三千為中監,餘為下監。監皆有左、右,因地為之名。方其時,天下以一縑易一馬。萬歲掌馬久,恩信行於隴右。



 後以太僕少卿鮮於匡俗檢校隴右牧監。儀鳳中,以太僕少卿李思文檢校隴右諸牧監使,監牧有使自是始。後又有群牧都使,有閑廄使,使皆置副,有判官。又立四使:南使十五,西使十六,北使七,東使九。諸坊若涇川、亭川、闕水、洛、赤城,南使統之;清泉、溫泉,西使統之;烏氏,北使統之;木硤、萬福,東使統之。它皆失傅。其後益置八監於鹽州、三監於嵐州。鹽州使八,統白馬等坊;嵐州使三,統樓煩、玄池、天池之監。



 凡征伐而發牧馬,先盡強壯,不足則取其次。錄色、歲、膚第印記、主名送軍,以帳馱之,數上於省。



 自萬歲失職,馬政頗廢,永隆中,夏州牧馬之死失者十八萬四千九百九十。景雲二年,詔群牧歲出高品,御史按察之。開元初,國馬益耗,太常少卿姜晦乃請以空名告身市馬於六胡州,率三十匹仇一游擊將軍。命王毛仲領內外閑廄。九年又詔:「天下之有馬者,州縣皆先以郵遞軍旅之役,定戶復緣以升之。百姓畏苦,乃多不畜馬,故騎射之士減曩時。自今諸州民勿限有無廕,能家畜十馬以上,免帖驛郵遞征行,定戶無以馬為貲。」毛仲既領閑廄,馬稍稍復,始二十四萬,至十三年乃四十三萬。其後突厥款塞,玄宗厚撫之,歲許朔方軍西受降城為互市,以金帛市馬,於河東、朔方、隴右牧之。既雜胡種,馬乃益壯。



 天寶後,諸軍戰馬動以萬計。王侯、將相、外戚牛駝羊馬之牧布諸道,百倍於縣官,皆以封邑號名為印自別;將校亦備私馬。議謂秦、漢以來,唐馬最盛,天子又銳志武事,遂弱西北蕃。十一載,詔二京旁五百里勿置私牧。十三載,隴右群牧都使奏:馬牛駝羊總六十萬五千六百,而馬三十二萬五千七百。



 安祿山以內外閑廄都使兼知樓煩監,陰選勝甲馬歸範陽,故其兵力傾天下而卒反。肅宗收兵至彭原,率官吏馬抵平涼,搜監牧及私群,得馬數萬,軍遂振。至鳳翔,又詔公卿百寮以後乘助軍。其後邊無重兵,吐蕃乘隙陷隴右,苑牧畜馬皆沒矣。乾元後,回紇恃功,歲入馬取繒,馬皆病弱不可用。永泰元年,代宗欲親擊虜,魚朝恩乃請大搜城中百官、士庶馬輸官,曰「團練馬」。下制禁馬出城者,已而復罷。德宗建中元年,市關輔馬三萬實內廄。貞元三年,吐蕃、羌、渾犯塞,詔禁大馬出潼、蒲、武關者。元和十一年伐蔡,命中使以絹二萬市馬河曲。其始置四十八監也,據隴西、金城、平涼、天水,員廣千里,繇京度隴,置八坊為會計都領,其間善水草、腴田皆隸之。後監牧使與坊皆廢,故地存者一歸閑廄,旋以給貧民及軍吏,間又賜佛寺、道館幾千頃。十二年,閑廄使張茂宗舉故事,盡收岐陽坊地,民失業者甚眾。十三年,以蔡州牧地為龍陂監。十四年,置臨漢監於襄州,牧馬三千二百,費田四百頃。穆宗即位,岐人叩闕訟茂宗所奪田,事下御史按治,悉予民。大和七年,度支鹽鐵使言:「銀州水甘草豐,請詔刺史劉源市馬三千,河西置銀川監,以源為使。」襄陽節度使裴度奏停臨漢監。開成二年,劉源奏:「銀川馬已七千,若水草乏,則徙牧綏州境。今綏南二百里,四隅險絕,寇路不能通,以數十人守要,畜牧無它患。」乃以隸銀川監。



 其後闕,不復可紀。



\end{pinyinscope}