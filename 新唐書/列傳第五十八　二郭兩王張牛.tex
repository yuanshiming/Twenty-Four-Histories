\article{列傳第五十八 二郭兩王張牛}

\begin{pinyinscope}

 郭虔瓘,齊州歷城人。開元初,錄軍閥,遷累右衛驍將軍,兼北庭都護、金山道副大總管。明年的精神性的存在,哲學的任務是為宗教神學作論證。有兩個,突厥默啜子同俄特勒圍北庭,虔瓘飭壘自守。同俄單騎馳城下,勇士狙道左突斬之。虜亡酋長,相率丐降,請悉軍中所資贖同俄死,聞已斬,舉軍慟哭去。虔瓘以功授冠軍大將軍、安西副大都護,封潞國公。建募關中兵萬人擊餘寇,遂前功,有詔募士給公乘,在所續食。將作大匠韋湊上言:「漢徙豪族以實關中,今畿輔戶口逋耗,異時戎虜入盜,丁壯悉行,不宜更募驍勇,以空京甸,資荒服。萬人所過,遞馱熟饔,亙六千里,州縣安所供億?秦、隴以西,多沙磧,少居人,若何而濟?縱有克獲,其補幾何?儻稽天誅,則諉大事。」不省。既而虔瓘果不見虜,還,遷涼州刺史、河西節度大使,進右威衛大將軍。四年,奏家奴八人有戰功,求為游擊將軍,宰相劾其恃功亂綱紀,不可聽,罷之。



 陜王為安西都護,詔虔瓘為副。虔瓘與安撫招慰十姓可汗使阿史那獻數持異,交訴諸朝。玄宗遣左衛中郎將王惠齎詔書諭解曰:「朕聞師克在和,不在眾,以虔瓘、獻宿將,當舍嫌窒隙,戮力國家。自開西鎮,列諸軍,戍有定區,軍有常額,卿等所統,蕃漢雜之,在乎善用,何必加募?或云突騎施圍石城,獻所致也;葛邏祿稱兵,虔瓘所沮也。大將不協,小人以逞,何功可圖?昔相如能詘廉頗,寇恂不吝賈復,宜各曠然,終承朕命。今賜帛二千段及他珍器,俾諒朕意。」虔瓘奉詔。久之,卒軍中。以張孝嵩為安西副都護。



 孝嵩,偉姿貌,及進士第,而慷慨好兵。在安西勸田訓士,府庫盈饒。徙太原尹,卒。以黃門侍郎杜暹代。



 郭知運,字逢時,瓜州晉昌人。長七尺,猿臂虎口,以格鬥功累補秦州三度府果毅。從郭虔瓘破突厥有功,加右驍衛將軍,封介休縣公。



 吐蕃將坌達延、乞力徐寇渭源,盜牧馬,詔知運與薛訥、王晙等相掎角,敗之。進階冠軍大將軍,兼臨洮軍使,封太原郡公,賜賚萬計。徙隴右諸軍節度大使、鄯州都督。突厥降戶阿悉爛、𧾷夾跌思泰率眾叛,執單于副都護張知運,詔以朔方兵追擊,至黑山呼延穀敗之,虜棄伏走,取副都護還。詔知運兼隴右經略使,營柳城。開元五年,大破吐蕃,獻俘京師。明年,復出,將輕兵丙夜至九曲,獲精甲、名馬、犛牛甚眾。既獻獲,詔分賜文武五品以上清官及朝集使三品者。進兼鴻臚卿,攝御史中丞。六州胡康待賓反,率王晙討平之。拜左武衛大將軍,授一子官,賜金帛。九年,卒於軍,年五十五,贈涼州都督。



 知運屯西方,戎夷畏憚,與王君■功名略等,時號「王郭」。帝詔中書令張說紀其功於墓碑。上元中,配饗太公廟。永泰初,謚曰威。子英傑、英乂。



 英傑,字孟武,為左衛將軍、幽州副總管。開元二十三年,長史薛楚玉遣英傑與裨將吳克勤、烏知義、羅守忠帥萬騎及奚眾討契丹,屯榆關。契丹酋長可突於拒戰都山下,奚眾貳,官軍不利,知義、守忠引麾下遁去,英傑、克勤力戰死。其下尚六千人,殊死戰,虜示以英傑首,終不屈,師遂殲。



 英乂,字元武,以武勇有名河、隴間,累遷諸衛員外將軍。哥舒翰見之曰:「是當代吾節制者。」祿山亂,拜秦州都督、隴右採訪使。賊將高嵩擁兵入汧、隴,英乂偽勞之,且具饗,既而伏兵發,盡虜其眾。至德二年,加隴右節度使。召還,改羽林軍大將軍,掌衛兵。以喪去職。



 史思明陷洛陽,謀掠陳、蔡,詔英乂統淮南節度兵。賊叩陜、虢,又改陜西節度、潼關防禦使。進御史大夫,兼神策軍節度使。代宗即位,以檢校戶部尚書兼大夫。雍王率諸將討賊洛陽,留英乂殿於陜。東都平,權知留守,無檢御才,其麾下與朔方、回紇遂大掠都城及鄭、汝,環千里無居人。



 以功實封三百戶,召拜尚書右僕射,封定襄郡王。日驕蹇,為侈汰。陰事宰相元載以久其權。未幾,嚴武死成都,乃拜劍南節度使。自以有內主,故肆志無所憚。初,玄宗在蜀時舊宮為道士祠,冶金作帝象,盡繪乘輿侍衛,每尹至,先拜祠,後視事。英乂愛其地勝選,輒壞繪像自居之,眾始不平。又教女伎乘驢擊球,鈿鞍寶勒及它服用,日無慮數萬費,以資倡樂,未嘗問民間事,為政苛暴,人以目相謂。怨崔寧不己同也,出兵襲寧,不克。寧因人之怨,率麾下五千直搗成都。英乂拒戰,眾皆反戈內攻,乃奔簡州,次靈池,普州刺史韓澄斬首送寧,遂屠其家。



 王君■,字威明,瓜州常樂人。初事郭知運為別奏,累功至右衛副率。知運卒,代為河西隴右節度使、右羽林軍將軍,判涼州都督事。



 開元十四年,吐蕃酋悉諾邏寇大斗拔谷,君■間其怠,率秦州都督張景順乘冰度青海襲破之。以功遷大將軍,封晉昌縣伯;拜其父壽為少府監,聽不事。君■凱旋,玄宗宴君■及妻夏於廣達樓,賜金帛,夏亦自以戰功封武威郡夫人。俄而吐蕃陷瓜州,執刺史田元獻及壽,殺居人,取資糧,進攻玉門軍,使人靳君■曰:「將軍常自以忠勇,今不一進戰,奈何?」君■登陴西向哭,兵不敢出。



 初,涼州有回紇、契苾、思結、渾四部,世為酋長,君■微時,數往來,為所輕。及節度河西,回紇等頗鞅鞅,恥為下。君■怒,數督過之。既怨望,潛遣人至東都言狀。君■間驛奏四部有叛謀,帝使中人即訊,回紇不能自直。於是瀚海大都督回紇承宗流瀼州,渾大得流吉州,賀蘭都督契苾承明流藤州,盧山都督思結歸國流瓊州,而承宗黨瀚海州司馬護輸等益不平,思有以復怨。會吐蕃使間道走突厥,君■率騎到肅州掩取之,還至甘州,護輸狙兵發,奪君■節,殺左右親吏,剖其心,曰:「是始謀者。」君■引帳下力戰,兵盡乃死。輸欲以尸奔吐蕃,追兵至,乃棄尸去。帝痛惜之,贈特進、荊州大都督。以喪還京師,官護其葬。詔張說刻文墓碑,帝自書以寵之。



 始,吐蕃寇瓜州,分遣莽布支攻常樂,令賈師順乘城守。俄而瓜州陷,悉諾邏並兵攻之。數日,虜眾有姻家在城中,使夜見師順曰:「州已失守,虜悉眾來,孤城渠可久,不早降以全噍類乎?」師順曰:「吾受天子命守此,義不可下賊。」數日,又說師順曰:「明府不降,吾眾且還,宜有以贈我。」師順請脫士卒衣襦。悉諾邏知無有,乃夜徹營去,毀瓜州城。師順開門收器械,復完守備。吐蕃果使精騎還襲,見有備,乃去。以功遷鄯州都督、隴右節度使。師順,岐州人,終左領軍將軍。



 張守珪,陜州河北人。姿乾瑰壯,慷慨尚節義,善騎射。以平樂府別將從郭虔瓘守北庭。突厥侵輪臺,遣守珪往援,中道逢賊,苦戰,斬首千餘級,禽頡斤一人。開元初,虜復攻北庭,守珪從儳道奏事京師,因上書言利害,請引兵出蒲昌、輪臺夾擊賊。再遷幽州良杜府果毅。時盧齊卿為刺史,器之,引與共榻坐,謂曰:「不十年,子當節度是州,為國重將,願以子孫托,可僚屬相期邪?」稍遷建康軍使。



 王君■死,河西震懼,詔以守珪為瓜州刺史、墨離軍使,督餘眾完故城。版築方立,虜奄至,眾失色。守珪曰:「創痍之餘,詎可矢石相確,須權以勝之。」遂置酒城上,會諸將作樂。虜疑有備,不敢攻,引去,守珪縱兵擊敗之。於是修復位署,招流冗使復業。有詔以瓜州為都督府,即詔守珪為都督。州地沙脊不可蓺,常瀦雪水溉田。是時,渠堨為虜毀,材木無所出。守珪密禱於神,一昔水暴至,大木數千章塞流下,因取之,修復堰防,耕者如舊,州人神之,刻石紀事。遷鄯州刺史、隴右節度使。徙幽州長史、河北節度副大使。俄加採訪處置等使。



 契丹、奚連年梗邊,牙官可突於,胡有謀者,前長史趙含章、薛楚玉等不能制,守珪至,每戰輒勝,虜遂大敗。帝喜,詔有司告九廟。契丹酋屈剌及突於恐懼,乃遣使詐降。守珪得其情,遣右衛騎曹王悔詣部計事,屈刺無降意,徙帳稍西北,密引突厥眾將殺悔以叛。契丹別帥李過折與突於爭權不葉,悔因間誘之,夜斬屈剌及突於,盡滅其黨,以眾降。守珪次紫蒙川,大閱軍實,賞將士,傳屈刺、突於首於東都。



 二十三年,入見天子,會藉田畢,即酺燕為守珪飲至,帝賦詩寵之。加拜輔國大將軍、右羽林大將軍,賜金彩,授二子官,詔立碑紀功。



 久之,復討契丹餘黨於捺祿山,鹵獲不訾。會裨將趙堪、白真陀羅等強使平盧軍使烏知義度湟水邀叛奚,且蹂其稼,知義辭不往,真陀羅矯詔脅之。知義與虜斗,不勝,還,守珪匿其敗,但上克獲狀。事頗洩,帝遣謁者牛仙童按實,守珪逼真陀羅自殺,厚賂使者,還奏如狀。後仙童以贓敗,事逮守珪,以功貶括州刺史,疽發背死。



 子獻誠。獻誠,天寶末,陷安祿山,授偽署。後事史思明,將兵數萬守汴州。東都平,史朝義走還汴,獻誠不內,籍所統兵以州降,詔即拜汴州刺史,封南陽郡公。改寶應軍左廂兵馬使,更封鄧國公。既來朝,代宗禮賜尤渥。擢山南西道節度使,討南山劇賊高玉,禽之。俄兼劍南東川節度。時崔旰殺郭英乂,獻誠率眾戰梓州,大敗。大歷三年,以疾歸京師。舉其弟獻恭自代。以檢校戶部尚書知省事,病甚,固乞辭位,卒。始,獻誠喜功名,為政寬裕,有機略,隨方制變,而簡廉不逮於父。



 從弟獻恭,數有軍功,以右羽林軍代為節度使。大歷末,破吐蕃於岷州。久之,拜東都留守,累遷檢校吏部尚書。德宗欲徙盧杞為饒州刺史,給事中袁高上還詔書,苦爭。獻恭見帝曰:「高所奏宜聽。」帝不答。復前曰:「高乃陛下良臣,當優異之。」上遂不徙杞。世咨其不撓。



 子煦,積閥亦至夏州節度使。元和八年,振武軍逐節度使李進賢,屠其家及判官嚴澈。憲宗怒,詔煦以本軍進討,許以便宜,賜縑三萬為軍資,河東王鍔遣兵五千為援。煦入,捕亂卒蘇國珍等數百人,誅之。卒,贈太子太保。



 獻誠從弟獻甫,以軍功試光祿卿、殿中監,從河中節度使賈耽討梁崇義有勞。德宗西幸,又從渾瑊討硃泚,戰多,累遷至金吾將軍、檢校工部尚書。李懷光叛,吐蕃盜邊,獻甫領禁兵戍咸陽累年,兵農悅安。貞元四年,代韓游瑰領邠寧節度使。邠寧軍素驕,憚獻甫嚴,因游瑰去,遂縱掠,邀範希朝為帥。都將楊朝晟誅首亂者,獻甫乃得入。於是斷山浚塹,選嚴要地築烽堡。請復鹽州及洪門、洛原鎮屯兵,詔可。獻甫遣兵馬使魏茪逐吐蕃,築鹽、夏二城,虜眾畏,不敢入寇。十二年,加檢校尚書左僕射。卒,贈司空。



 王忠嗣,華州鄭人。父海賓,太子右衛率、豐安軍使。開元二年,吐蕃寇隴右,詔隴右防禦使薛訥率杜賓客、郭知運、王晙、安思順御之。以海賓為先鋒,戰武階,追北至壕口,殺其眾。進戰長城堡,諸將媢其功,按兵顧望,海賓戰死,大軍乘之,斬賊萬七千級,獲馬七萬、牛羊四十萬。玄宗憐其忠,贈左金吾大將軍。忠嗣時年九歲,始名訓,授尚輦奉御。入見帝,伏地號泣,帝撫之曰:「此去病孤也,須壯而將之。」更賜今名,養禁中。肅宗為忠王,帝使與游。及長,雄毅寡言,有武略,上與論兵,應對蜂起,帝器之,曰:「後日爾為良將。」試守代州別駕,大猾閉門自斂,不敢干法。數以輕騎出塞,忠王言於帝曰:「忠嗣敢鬥,恐亡之。」由是召還。



 信安王禕在河東,蕭嵩出河西,數引為麾下。帝以其年少,有復讎志,詔不得特將。嵩入朝,忠嗣曰:「從公三年,無以歸報天子。」乃請精銳數百襲虜。會贊普大酋閱武鬱標川,其下欲還,忠嗣不從,提刀略陣,斬數千人,獲羊馬萬計。嵩上其功,帝大悅。累遷左威衛將軍、代北都督,封清源縣男。與皇甫惟明輕重不得,構忠嗣罪,貶東陽府左果毅。


河西節度使杜希望欲取吐蕃新羅城,有言忠嗣才者,希望以聞,詔追赴河西,進拔其城。忠嗣錄多,授左威衛郎將,專知兵馬。俄吐蕃大出,欲取當新城,晨壓官軍陣,眾不敵,舉軍皆恐。忠嗣單馬進,左右馳突,獨殺數百人,賊眾囂相蹂,軍
 \gezhu{
  廣多}
 翼掩之,虜大敗。拜左金吾衛將軍,領河東節度副使、大同軍使,尋為節度使。二十九年,節度朔方,兼靈州都督。



 天寶元年,北討奚怒皆,戰桑乾河,三遇三克,耀武漠北,高會而還。時突厥新有難,忠嗣進軍磧口經略之。烏蘇米施可汗請降,忠嗣以其方強,特文降耳,乃營木刺、蘭山,諜虛實。因上平戎十八策,縱反間於拔悉密與葛邏祿、回紇三部,攻多羅斯城,涉昆水,斬米施可汗,築大同、靜邊二城,徙清塞、橫野軍實之,並受降、振武為一城,自是虜不敢盜塞。徙河東節度使,進封縣公。



 忠嗣本負勇敢,及為將,乃能持重安邊,不生事,嘗曰:「平世為將,撫眾而已。吾不欲竭中國力以幸功名。」故訓練士馬,隨缺繕補。有漆弓百五十斤,每弢之,示無所用。軍中士氣盛,日夜思戰,忠嗣縱詭間,伺虜隙,時時出奇兵襲敵,所向無不克,故士亦樂為用。軍每出,召屬長付以兵,使授士卒,雖弓矢亦志姓名其上。軍還,遣弦亡鏃,皆按名第罪。以是部下人自觀,器甲充物。自朔方至雲中袤數千里,據要險築城堡,斥地甚遠。自張仁亶後四十餘年,忠嗣繼其功。



 俄為河西、隴右節度使,權朔方、河東節度,佩四將印,勁兵重地,控制萬里,近世未有也。又授一子五品官。後數出戰青海、積石,虜輒奔破。又討吐谷渾於墨離,平其國。乃固讓朔方、河東二節度,許之。



 帝方事石堡城,詔問攻取計,忠嗣奏言:「吐蕃舉國守之,若頓兵堅城下,費士數萬,然後可圖,恐所得不讎所失,請厲兵馬,待釁取之。」帝意不快。而李林甫尤忌其功,日鉤摭過咎。會董延光建言請下石堡,詔忠嗣分兵應接,忠嗣不得已為出軍,而士無賞格,延光不悅。河西兵馬使李光弼入說曰:「大夫愛惜士卒,有拒延光心,雖名受詔,實奪其謀。然大夫已付萬眾,而不立重賞,何以賈士勇?且大夫惜數萬段賜,以啟讒口,有如不捷,歸罪大夫,大夫先受禍矣。」忠嗣曰:「吾固審得一城不足制敵,失之未害於國。吾忍以數萬人命易一官哉!明日見責,不失一金吾、羽林將軍,歸宿衛;不者,黔中上佐耳。」光弼謝曰:「大夫乃行古人事,光弼又何言!」趨而出。延光過期不克,果訴忠嗣沮兵。又安祿山城雄武,扼飛狐塞,謀亂,請忠嗣助役,因欲留其兵;忠嗣先期至,不見祿山而還。數上言祿山且亂,林甫益惡之,陰使人誣告「忠嗣嘗養宮中,云吾欲奉太子」。帝怒,召入付三司訊驗,罪應死。哥舒翰方有寵,白上,請以官爵贖忠嗣罪,帝意解,貶漢陽太守。久之,徙漢東郡,卒,年四十五。後翰引兵攻石堡,拔之,死亡略盡,如忠嗣言,故當世號為名將。



 初,在朔方,至互市,輒高償馬直,諸蕃爭來市,故蕃馬浸少,唐軍精。及鎮河、隴,又請徙朔方、河東九千騎以實軍。迄天寶末,益滋息。寶應元年,追贈兵部尚書。



 贊曰:以忠嗣之才,戰必破,攻必克,策石堡之得不當所亡,高馬直以空虜資,論祿山亂有萌,可謂深謀矣。然不能自免於讒,卒死放地。自古忠賢,工謀於國則拙於身,多矣,可勝吒哉!



 牛仙客,涇州鶉觚人。初為縣小史,令傅文靜器之,會為隴右營田使,引與計事,積功遷洮州司馬。河西節度使王君■召為判官。君■死,仙客獨得免。蕭嵩代節度,復委以軍政。仙客清勤不懈,接士大夫以信。及嵩還執政,因薦之。稍遷太僕少卿,判涼州別駕,知節度留後事,俄為節度使。開元二十四年,代信安王禕為朔方行軍大總管。



 始在河西,嗇事省用,倉庫積鉅萬,器械犀銳。崔希逸代之,即以聞。帝令刑部員外郎張利馳傳覆視,如狀。帝悅,將用為尚書,宰相張九齡持不可,乃封隴西郡公,實封戶二百。李林甫探知帝旨,稱其材。會九齡罷,故以工部尚書、同中書門下三品,知門下事,遙領河東節度副大使。



 為相謹身無它,與時沉浮,唯唯恭願。前後錫與,緘庋不敢用。百司諮決,無所處可,輒曰:「如令式。」帝既用仙客,知時議不歸,乘間以問高力士,力士曰:「仙客本胥史,非宰相器。」帝忿然曰:「朕且用康厓!」蓋恚言也。有為厓言者,厓以為實,喜甚。久之,封豳國公,加左相。卒,贈尚書右丞相,謚曰貞簡。



\end{pinyinscope}