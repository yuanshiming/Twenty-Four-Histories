\article{列傳第二十三 王薛馬韋}

\begin{pinyinscope}

 王珪,字叔玠。祖僧辯,梁太尉、尚書令。父顗,北齊樂陵郡太守。世居郿。性沉澹,志量隱正施行仁政。語出《尚書·洪範》:「無偏無黨,王道蕩蕩。」孟,恬於所遇,交不茍合。隋開皇十三年,召入秘書內省,讎定群書,為太常治禮郎。季父頗,通儒有鑒裁,尤所器許。頗坐漢王諒反,誅,珪亡命南山十餘年。高祖入關,李綱薦署世子府諮議參軍事。建成為皇太子,授中舍人,遷中允,禮遇良厚。太子與秦王有隙,帝責珪不能輔導,流巂州。太子已誅,太宗召為諫議大夫。帝嘗曰:「正主御邪臣,不可以致治;正臣事邪主,亦不可以致治。唯君臣同德,則海內安。朕雖不明,幸諸公數相諫正,庶致天下於平。」珪進曰:「古者,天子有爭臣七人,諫不用,則相繼以死。今陛下開聖德,收採芻言,臣願竭狂瞽,佐萬分一。」帝可,乃詔諫官隨中書、門下及三品官入閣。珪推誠納善,每存規益,帝益任之。封永寧縣男、黃門侍郎,遷侍中。



 它日進見,有美人侍帝側,本廬江王瑗姬也。帝指之曰:「廬江不道,賊其夫而納其室,何有不亡乎?」珪避席曰:「陛下以廬江為是邪?非邪?」帝曰:「殺人而取妻,乃問朕是非,何也?」對曰:「臣聞齊桓公之郭,問父老曰:『郭何故亡?』曰:『以其善善而惡惡也。』公曰:『若子之言,乃賢君也,何至於亡?』父老曰:『不然,郭君善善不能用,惡惡不能去,所以亡。』今陛下知廬江之亡,其姬尚在,竊謂陛下以為是。審知其非,所謂知惡而不去也。」帝嗟美其言。



 帝使太常少卿祖孝孫以樂律授宮中音家,伎不進,數被讓。珪與溫彥博同進曰:「孝孫,修謹士,陛下使教女樂,又責譙之,天下其以士為輕乎!」帝怒曰:「卿皆我腹心,乃附下罔上,為人游說邪?」彥博懼,謝罪,珪不謝,曰:「臣本事前宮,罪當死,陛下矜其性命,引置樞密,責以忠效。今疑臣以私,是陛下負臣,臣不負陛下。」帝默然慚,遂罷。明日,語房玄齡曰:「昔武王不用夷、齊,宣王殺杜伯,自古帝王納諫固難矣。朕夙夜庶幾於前聖,昨責珪等,痛自悔,公等勿懲是不進諫也!」



 時珪與玄齡、李靖、溫彥博、戴胄、魏徵同輔政。帝以珪善人物,且知言,因謂曰:「卿標鑒通晤,為朕言玄齡等材,且自謂孰與諸子賢?」對曰:「孜孜奉國,知無不為,臣不如玄齡;兼資文武,出將入相,臣不如靖;敷奏詳明,出納惟允,臣不如彥博;濟繁治劇,眾務必舉,臣不如胄;以諫諍為心,恥君不及堯、舜,臣不如征。至激濁揚清,疾惡好善,臣於數子有一日之長。」帝稱善。而玄齡等亦以為盡己所長,謂之確論。



 進封郡公。坐漏禁近語,左除同州刺史。帝念名臣,俄召拜禮部尚書兼魏王泰師。王見之,為先拜,珪亦以師自居。王問珪何以為忠孝,珪〗曰:「陛下,王之君,事思盡忠;陛下,王之父,事思盡孝。忠孝可以立身,可以成名。」王曰:「忠孝既聞命矣,願聞所習。」珪曰:「漢東平王蒼稱『為善最樂』,願王志之。」帝聞,喜曰:「兒可以無過矣!」



 子敬直,尚南平公主。是時,諸主下嫁,以帝女貴,未嘗行見舅姑禮。珪曰:「主上循法度,吾當受公主謁見。豈為身榮,將以成國家之美。」於是,與夫人坐堂上,主執盥饋乃退。其後公主降,有舅姑者,備婦禮,本於珪。



 十三年,病。帝遣公主就第省視,復遣民部尚書唐儉增損藥膳。卒,年六十九。帝素服哭別次,詔魏王率百官臨哭。贈吏部尚書,謚曰懿。



 珪少孤且貧,人或饋遺,初無讓。及貴,厚報之,雖已亡,必酬贍其家。性不苛察,臨官務舉綱維,去甚不可者,至僕妾亦不見喜慍。奉寡嫂,家事咨而後行。教撫孤侄,雖其子不過也。宗族匱乏,周恤之,薄於自奉。獨不作家廟,四時祭於寢,為有司所劾,帝為立廟愧之,不罪也。世以珪儉不中禮,少之。始,隱居時,與房玄齡、杜如晦善,母李嘗曰:「而必貴,然未知所與游者何如人,而試與偕來。」會玄齡等過其家,李窺大驚,敕具酒食,歡盡日,喜曰:「二客公輔才,汝貴不疑。」敬直封南城縣男,後坐交皇太子承乾,徙嶺外。



 珪孫燾、旭。燾,性至孝,為徐州司馬。母有疾,彌年不廢帶,視絮湯劑。數從高醫游,遂窮其術,因以所學作書,號《外臺秘要》,討繹精明,世寶焉。歷給事中、鄴郡太守,治聞於時。旭,見《酷吏傳》。



 薛收,字伯褒。蒲州汾陰人。隋內史侍郎道衡子也,出繼從父孺。年十二,能屬文。以父不得死於隋,不肯仕。郡舉秀才,不應。聞高祖興,遁入首陽山,將應義舉。通守堯君素覺之,迎置其母城中,收不得去。及君素東連王世充,遂挺身歸國。房玄齡亟言之秦王,王召見,問方略。所對合旨,授府主簿,判陜東大行臺金部郎中。是時方討世充,軍事繁綜,收為書檄露布,或馬上占辭,該敏如素構,初不竄定。竇建德來援,諸將爭言斂軍以觀賊形勢,收獨曰:「不然。世充據東都,府庫盈衍,其兵皆江淮選卒,正苦乏食爾,是以求戰不得,為我所持。今建德身總眾以來,必飛轂轉糧,更相資哺。兩賊連固,則伊、洛間勝負未可歲月定也。不若勒諸將嚴兵締壘,浚其溝防,戒毋出兵。大王親督精銳據成皋,厲兵按甲,邀建德路。彼以疲老,當吾堂堂之鋒,一戰必舉。不旬日,二賊可縛致麾下矣。」王曰:「善。」遂禽建德,降世充。



 王入觀隋宮室,且嘆煬帝無道,殫人力以事誇侈。收進曰:「峻宇雕墻,殷辛以亡;土階茅茨,唐堯以昌。始皇興阿房而秦禍速,文帝罷露臺而漢祚永。後主曾不是察,奢虐是矜,死一夫之手,為後世笑,何此之能保哉?」王重其言。俄授天策府記室參軍。從平劉黑闥,封汾陰縣男。嘗上書諫王止畋獵,王答曰:「覽所陳,知成我者,卿也。明珠兼乘,未若一言,今賜黃金四十鋌。」



 武德七年,寢疾。王遣使臨問,相望於道。命輿疾至府,親舉袂撫之,論敘生平,感激涕泗。卒,年三十三。王哭之慟,與其從兄子元敬書曰:「吾與伯褒共軍旅間,何嘗不驅馳經略,款曲襟抱,豈期一朝成千古也。且家素貧而子幼,善撫安之,以慰吾懷。」因遣使吊祭,贈帛三百段。其後圖學士像,嘆其早死不得與。既即位,語房玄齡曰:「收若在,朕當以中書令處之。」又嘗夢收如平生,賜其家粟、帛。貞觀七年,贈定州刺史。永徽中,又贈太常卿,陪葬昭陵。



 子元超,九歲襲爵。及長,好學,善屬文。尚巢王女和靜縣主,累授太子舍人。高宗即位,遷給事中,數上書陳當世得失,帝嘉納。轉中書舍人、弘文館學士。省中有盤石,道衡為侍郎時,常據以草制,元超每見,輒泫然流涕。以母喪解,奪服授黃門侍郎、檢校太子左庶子。所薦豪俊士,若任希古、高智周、郭正一、王義方、孟利貞、鄭祖玄、鄧玄挺、崔融等,皆以才自名於時。累拜東臺侍郎。李義府流巂州,舊制,流人不得乘馬,元超為請,坐貶簡州刺史。歲餘,又坐與上官儀文章款密,流巂州。上元初,赦還,拜正諫大夫。三年,遷中書侍郎、同中書門下三品。



 帝校獵溫泉,諸蕃酋長得持弓矢從。元超奏:「夷狄野心,而使挾兵在圍中,非所宜。」帝納可。嘗宴諸王,召元超與,從容謂曰:「任卿中書,寧藉多人哉!」俄拜中書令兼左庶子。帝幸東都,留輔太子監國,手敕曰:「朕留卿,若失一臂。顧太子未習庶務,關中事,卿悉專之。」時太子射獵,詔得入禁禦,故太子稍怠政事。元超諫曰:「內苑之地,繚叢薄,冒翳薈,絕磴險途。殿下截輕禽,逐狡兔,銜橛之變,詎無可虞?又戶奴多反逆餘族,或夷狄遺醜,使兇謀竊發,將何以御哉?夫為人子者,不登高,不臨深,謂其近危辱也。天皇所賜書戒丁寧,惟殿下罷馳射之勞,留情墳典,豈不美歟!」帝知之,遣使厚賜慰其意,召太子還東都。帝疾劇,政出武后。因陽喑,乞骸骨。加金紫光祿大夫。卒,年六十二,贈光祿大夫、秦州都督,陪葬乾陵。子曜,聖歷中,附會張易之,官正諫大夫。



 元敬,隋選部郎邁之子,與收及收族兄德音齊名,世稱「河東三鳳」。收為長離雛,德音為鸑鷟,元敬年最少,為鵷雛。武德中,為秘書郎、天策府參軍,直記室、文學館學士。是時,收與房、杜處心腹之寄,更相結附。元敬謹畏,未嘗申款曲。如晦嘆曰:「小記室不可得而親,不可得而疏!」秦王為皇太子,除舍人。於是軍國之務總於東宮,而元敬掌文翰,號稱職。卒於官。



 稷,字嗣通,道衡曾孫。擢進士第。累遷禮部郎中、中書舍人。與從祖兄曜更踐兩省,俱以辭章自名。景龍末,為諫議大夫、昭文館學士。初,貞觀、永徽間,虞世南、褚遂良以書顓家,後莫能繼。稷外祖魏徵家多藏虞、褚書,故銳精臨仿,結體遒麗,遂以書名天下。畫又絕品。睿宗在籓,喜之,以其子伯陽尚仙源公主。及踐阼,遷太常少卿,封晉國公,實封三百戶。會鐘紹京為中書令,稷諷使讓,因入言於帝曰:「紹京本胥史,無素才望,今特以勛進,師長百僚,恐非朝廷具瞻之美。」帝然之,遂許紹京讓,改戶部尚書。翌日,遷稷黃門侍郎,參知機務。與崔日用數爭事帝前,罷為左散騎常侍。歷太子少保、禮部尚書。帝以翊贊功,每召入宮中與決事,恩絕群臣。竇懷貞誅,稷以知本謀,賜死萬年獄,年六十五。



 伯陽為駙馬都尉、安邑郡公,別食實封四百戶。稷死,坐貶晉州員外別駕,又流嶺表,自殺。伯陽子談,尚玄宗恆山公主,拜駙馬都尉、光祿員外卿。



 馬周,字賓王,博州茌平人。少孤,家窶狹。嗜學,善《詩》、《春秋》。資曠邁,鄉人以無細謹,薄之。武德中,補州助教,不治事。刺史達奚恕數咎讓,周乃去,客密州。趙仁本高其才,厚以裝,使入關。留客汴,為浚儀令崔賢所辱,遂感激而西,舍新豐,逆旅主人不之顧,周命酒一斗八升,悠然獨酌,眾異之。至長安,舍中郎將常何家。



 貞觀五年,詔百官言得失。何武人,不涉學,周為條二十餘事,皆當世所切。太宗怪問何,何曰:「此非臣所能,家客馬周教臣言之。客,忠孝人也。」帝即召之,間未至,遣使者四輩敦趣。及謁見,與語,帝大悅,詔直門下省。明年,拜監察御史,奉使稱職。帝以何得人,賜帛三百段。周上疏曰:



 臣每讀前史,見賢者忠孝事,未嘗不廢卷長想,思履其跡。臣不幸早失父母,犬馬之養,已無所施;顧來事可為者,惟忠義而已。是以徒步二千里,歸於陛下。陛下不以臣愚,擢臣不次。竊自惟念無以論報,輒竭區區,惟陛下所擇。



 臣伏見大安宮在宮城右,墻宇門闕,方紫極為卑小。東宮,皇太子居之,而在內;大安,至尊居之,反在外。太上皇雖志清儉,愛惜人力,陛下不敢違,而蕃夷朝見,四方觀聽,有不足焉。臣願營雉堞門觀,務從高顯,以稱萬方之望,則大孝昭矣。



 臣伏讀明詔,以二月幸九成宮。竊惟太上皇春秋高,陛下宜朝夕視膳。今所幸宮去京三百里而遠,非能旦發暮至也。萬有一太上皇思感,欲即見陛下,何以逮之?今茲本為避暑行也,太上皇留熱處,而陛下走涼處,溫清之道,臣所未安。然詔書既下,業不中止,願示還期,以開眾惑。



 臣伏見詔宗室功臣悉就籓國,遂貽子孫,世守其政。竊惟陛下之意,誠愛之重之,欲其裔緒承守,與國無疆也。臣謂必如詔書者,陛下宜思所以安存之,富貴之,何必使世官也?且堯、舜之父,有硃、均之子。若令有不肖子襲封嗣職,兆庶被殃,國家蒙患。正欲絕之,則子文之治猶在也;正欲存之,則欒黶之惡已暴也。必曰與其毒害於見存之人,寧割恩於已亡之臣,則向所謂愛之重之者,適所以傷之也。臣謂宜賦以茅土,疇以戶邑,必有材行,隨器而授。雖幹翮非強,亦可以免累。漢光武不任功臣以吏事,所以終全其世者,良得其術也。願陛下深思其事,使得奉大恩,而子孫終其福祿也。



 臣聞聖人之化天下,莫不以孝為本,故曰:「孝莫大於嚴父,嚴父莫大於配天」,「國之大事,在祀與戎」,孔子亦言「吾不與祭,如不祭」,是聖人之重祭祀也。自陛下踐祚,宗廟之享,未嘗親事。竊惟聖情,以乘輿一出,所費無蓺,故忍孝思,以便百姓。而一代史官,不書皇帝入廟,將何以貽厥孫謀、示來葉邪?臣知大孝誠不在俎豆之間,然聖人訓人,必以己先之,示不忘本也。



 臣聞致化之道,在求賢審官。孔子曰:「惟名與器,不可以假人。」是言慎舉之為重也。臣伏見王長通、白明達本樂工輿皁雜類;韋般提、斛斯正無他材,獨解調馬。雖術逾等夷,可厚賜金帛以富其家。今超授高爵,與外廷朝會,騶豎倡子,鳴玉曳履,臣竊恥之。若朝命不可追改,尚宜不使在列,與士大夫為伍。



 帝善其言,除侍御中。又言:



 臣歷觀夏、商、周、漢之有天下,傳祚相繼,多者八百餘年,少者猶四五百年,皆積德累業,恩結於人,豈無僻王,賴先哲以免。自魏、晉逮周、隋,多者五六十年,少者三二十年而亡。良由創業之君不務仁化,當時僅能自守,後無遺德可思,故傳嗣之主,其政少衰,一夫大呼,天下土崩矣。今陛下雖以大功定天下,而積德日淺,固當隆禹、湯、文、武之道,使恩有餘地,為子孫立萬世之基,豈特持當年而已。然自古明王聖主,雖因人設教,而大要節儉於身,恩加於人,故其下愛之如父母,仰之如日月,畏之如雷霆,卜祚遐長,而禍亂不作也。今百姓承喪亂之後,比於隋時才十分一,而徭役相望,兄去弟還,往來遠者五六千里,春秋冬夏,略無休時。陛下雖詔減省,而有司不得廢作,徒行文書,役之如故。四五年來,百姓頗嗟怨,以為陛下不存養之。堯之茅茨土階,禹之惡衣菲食,臣知不可復行於今。漢文帝惜百金之費而罷露臺,集上書囊以為殿帷,所幸慎夫人衣不曳地;景帝亦以錦繡纂組妨害女功,特詔除之,所以百姓安樂。至孝武帝雖窮奢極侈,承文、景遺德,故人心不搖。向使高祖之後即值武帝,天下必不能全。此時代差近,事跡可見。今京師及益州諸處,營造供奉器物,並諸王妃主服飾,皆過靡麗。臣聞昧旦丕顯,後世猶怠,作法於治,其弊猶亂。陛下少處人間,知百姓辛苦,前代成敗,目所親見,尚猶如此,而皇太子生長深宮,不更外事,即萬歲後,聖慮之所當憂也。



 臣竊尋自古黎庶怨叛,聚為盜賊,其國無不即滅,人主雖悔,未有重能安全者。凡脩政教,當脩之於可脩之時。若事變一起而後悔之,無益也。故人主每見前代之亡,則知其政教之所由喪,而不知其身之失。故紂笑桀之亡,而幽、厲笑紂之亡,隋煬帝又笑齊、魏之失國也。今之視煬帝,猶煬帝之視齊、魏也。



 往貞觀初,率土荒儉,一匹絹才易斗米,而天下帖然者,百姓知陛下憂憐之,故人人自安無謗讟也。五六年來,頻歲豐稔,一匹絹易粟十餘斛,而百姓咸怨,以為陛下不憂憐之。何則?今營為者,多不急之務故也。自古以來,國之興亡,不由積畜多少,在百姓苦樂也。且以近事驗之,隋貯洛口倉而李密因之,積布帛東都而王世充據之,西京府庫亦為國家之用。向使洛口、東都無粟帛,王世充、李密未能必聚大眾。但貯積者,固有國之常,要當人有餘力而後收之,豈人勞而強斂之以資寇邪?



 夫儉以息人,貞觀初,陛下己躬為之,今行之不難也。為之一日,則天下知之,式歌且舞矣。若人既勞,而周之不息,萬一中國水旱,而邊方有風塵之警,狂狡竊發,非徒旰食晏寢而已。古語云:「動人以行不以言,應天以實不以文。」以陛下之明,誠欲厲精為政,不煩遠採上古,但及貞觀初,則天下幸甚。



 昔賈誼謂漢文帝云「可痛哭及長嘆息者」,言:當韓信王楚、彭越王梁、英布王淮南之時,使文帝即天子位,必不能安。又言:「賴諸王年少,傅相制之,長大之後,必生禍亂。」後世皆以誼言為是。臣竊觀今諸將功臣,陛下所與定天下,無威略振主如韓、彭者;而諸王年並幼少,縱其長大,陛下之日,必無他心,然則萬代之後,不可不慮。漢、晉以來,亂天下者,何嘗不在諸王。皆由樹置失宜,不豫為節制,以至滅亡。人主豈不知其然,溺於私愛爾。故前車既覆,而後車不改轍也。今天下百姓尚少,而諸王已多,其寵遇過厚者,臣愚慮之,非特恃恩驕矜也。昔魏武帝寵陳思王,文帝即位,防守禁閉同獄囚焉。何則?先帝加恩太多,故嗣主疑而畏之也。此武帝寵陳思王,適所以苦之也。且帝子身食大國,何患不富,而歲別優賜,曾無限極。里語曰:「貧不學儉,富不學奢。」言自然也。今大聖創業,豈唯處置見子弟而已,當制長久之法,使萬代奉行。



 臣聞天下者以人為本。必也使百姓安樂,在刺史、縣令爾。縣令既眾,不可皆賢,但州得良刺史可矣。天下刺史得人,陛下端拱巖廊之上,夫復何為?古者郡守、縣令皆選賢德,欲有所用,必先試以臨人,或由二千石高第入為宰相。今獨重內官,縣令、刺史頗輕其選。又刺史多武夫勛人,或京官不稱職始出補外;折沖果毅身力強者入為中郎將,其次乃補邊州。而以德行才術擢者,十不能一。所以百姓未安,殆在於此。



 疏奏,帝稱善。擢拜給事中,轉中書舍人。



 周善敷奏,機辯明銳,動中事會,裁處周密,時譽歸之。帝每曰:「我暫不見周即思之。」岑文本謂所親曰:「馬君論事,會文切理,無一言可損益,聽之纚纚,令人忘倦。蘇、張、終、賈正應此耳。然鳶肩火色,騰上必速,恐不能久。」俄遷治書侍御史,兼知諫議大夫,檢校晉王府長史。王為皇太子,拜中書侍郎,兼太子右庶子。十八年,遷中書令,猶兼庶子。時置太子司議郎,帝高其除。周嘆曰:「恨吾資品妄高,不得歷此官。」帝征遼,留輔太子定州。及還,攝吏部尚書,進銀青光祿大夫。帝嘗以飛白書賜周曰:「鸞鳳沖霄,必假羽翼;股肱之寄,要在忠力。」



 周病消渴連年,帝幸翠微宮,求勝地為構第,每詔尚書食具膳,上醫使者視護,躬為調藥,太子問疾。疾甚,周取所上章奏悉焚之,曰:「管、晏暴君之過,取身後名,吾不為也!」二十二年卒,年四十八,贈幽州都督,陪葬昭陵。



 初,帝遇周厚,周頗自負。為御史時,遣人以圖購宅,眾以其興書生,素無貲,皆竊笑。它日,白有佳宅,直二百萬,周遽以聞,詔有司給直,並賜奴婢什物,由是人乃悟。周每行郡縣,食必進雞,小吏訟之。帝曰:「我禁御史食肉,恐州縣廣費,食雞尚何與?」榜吏斥之。及領選,猶廢浚儀令。先是,京師晨暮傳呼以警眾,後置鼓代之,俗曰「冬冬鼓」;品官舊服止黃紫,於是三品服紫,四品五品硃,六品七品綠,八品九品青;城門入由左,出由右;飛驛以達警急;納居人地租;宿衛大小番直;截驛馬尾;城門、衛舍、守捉士,月散配諸縣,各取一,以防其過;皆周建白。自周亡,帝思之甚,將假方士術求見其儀形。高宗即位,追贈尚書右僕射、高唐縣公。垂拱中,配享高宗廟庭。



 子載,咸亨中為司列少常伯,與裴行儉分掌選事,言吏部者稱裴、馬焉。終雍州長史。



 贊曰:周之遇太宗,顧不異哉!由一介草茅言天下事,若素宦於朝、明習憲章者,非王佐才,疇以及茲?其自視與築巖、釣渭亦何以異!跡夫帝銳於立事,而周所建皆切一時,以明佐聖,故君宰間不膠漆而固,恨相得晚,宜矣。然周才不逮傅說、呂望,使後世未有述焉,惜乎!



 韋挺,京兆萬年人。父沖,仕隋為民部尚書。挺少與隱太子善,高祖平京師,署隴西公府祭酒。累遷太子左衛驃騎,檢校左衛率。太子遇之厚,宮臣無與比。武德七年,帝避暑仁智宮。或言太子與宮臣謀逆,又慶州刺史楊文乾坐大逆誅,辭連東宮,帝專責宮臣,由是挺與杜淹、王珪等皆流越巂。未幾,召拜主爵郎中。貞觀初,王珪數薦之,遷尚書右丞。歷吏部、黃門侍郎,拜御史大夫、扶陽縣男。太宗謂挺曰:「卿之任大夫,獨朕意,左右無為卿地者!」挺曰:「臣駑下,不足以辱高位,且非勛非舊,而在籓邸故僚上,願後臣以勸立功者。」不聽。是時承隋大亂,風俗薄惡,人不知教。挺上疏曰:「父母之恩,昊天罔極;創巨之痛,終身何已。今衣冠士族,辰日不哭,謂為重喪,親賓來吊,輒不臨舉。又閭里細人,每有重喪,不即發問,先造邑社,待營辦具,乃始發哀。至假車乘,雇棺郭,以榮送葬。既葬,鄰伍會集,相與酣醉,名曰出孝。夫婦之道,王化所基,故有三日不息燭、不舉樂之感。今昏嫁之初,雜奏絲竹,以窮宴歡。官司習俗,弗為條禁。望一切懲革,申明禮憲。」俄復為黃門侍郎,兼魏王泰府事。時泰有寵,太子多過失,帝密欲廢立,語杜正倫,正倫以漏言貶。帝謂挺曰:「不忍復置卿於法。」改太常卿。



 初,挺為大夫時,馬周為監察御史,挺不甚禮。及周為中書令,帝欲湔拭用之,周言挺佷於自用,非宰相器,遂止。帝將討遼東,擇主餉運者。周言挺才任粗使,帝謂然。挺父故為營州總管,嘗經略高麗,故札藏家,挺上之。帝悅曰:「自幽距遼二千里無州縣,吾軍靡所仰食,卿為朕圖之。茍吾軍用不乏,是公之功。其自擇文武官四品十人為子使,取幽、易、平三州銳士若馬各三百以從。」即詔河北列州皆取挺節度,許以便宜。帝親解貂裘及中廄馬賜之。挺遣燕州司馬王安德行渠,作漕艫轉糧,自桑乾水抵盧思臺,行八百里,渠塞不可通。挺以方苦寒,未可進,遂下米臺側,廥之,待凍泮乃運以為解。即上言:「度王師至,食且足。」帝不悅曰:「兵寧拙速,無工遲。我明年師出,挺乃度它歲運,何哉?」即詔繁畤令韋懷質馳按。懷質還劾:「挺在幽州,日置酒,弗憂職,不前視渠長利,即造船行粟,綿八百里,乃悟非是,欲進則不得,還且水涸。六師所須,恐不如陛下之素。」帝怒,遣將少監李道裕代之。敕治書侍御史唐臨馳傳,械挺赴洛陽,廢為民,使白衣從。



 帝破蓋牟城,詔挺將兵鎮守,示復用。城與賊新城接,日夜轉鬥無休時。挺以失職,內不平,作書謝所善公孫常。常,善數者也,以他事系,投繯死。索橐中得挺書,言所屯危蹙,意怨望,貶象州刺史。歲餘卒,年五十八。



 子待價、萬石。



 待價,初為左千牛備身,永徽中,江夏王道宗得罪,待價以婿貶盧龍府果毅。時將軍辛文陵招慰高麗,次吐護真水,為虜所襲,待價與中郎將薛仁貴率所部兵殺之,文陵亦苦戰,遂免。待價重創,矢著左足,隱不言,卒以疾免。起為蘭州刺史。吐蕃盜邊,高宗以沛王賢為涼州大都督,而待價為司馬。俄遷肅州刺史,以功召拜右武衛將軍。儀鳳三年,吐蕃復入寇,以待價檢校涼州都督,兼知鎮守兵馬事。召還,封扶陽侯。武后臨朝,攝司空,護營乾陵,改天官尚書、同鳳閣鸞臺三品。待價起武力,典選無銓總才,故朝野共蚩薄之。俄為燕然道行軍大總管,御突厥。逾年還,拜文昌右相、同鳳閣鸞臺三品。不自安,累表辭職,不聽。且請盡力行陣,許之,於是拜安息道行軍大總管,督三十六總管以討吐蕃,進爵公。軍至寅識迦河,與吐蕃合戰,勝負略相當。會其副閻溫古逗留,又天大寒,待價不善撫御,師人多死,餉道乏,乃旋師頓高昌。後大怒,斬溫古,流待價繡州,卒。



 曾孫武。武少孤。年十一,廕補右千牛,累遷長安丞。德宗幸梁州,委妻子奔行在,除殿中侍御史。戶部侍郎元琇為水陸轉運使,表武以倉部員外郎充判官。謀不用,杜門數月而琇敗。轉刑部員外郎。是時,帝以反正告郊廟,大兵後,典章茍完,執事者時時咨武。武酌宜約用,得禮之衷,群司奉焉。後為絳州刺史,鑿汾水灌田萬三千餘頃,璽書勞勉。憲宗時,入為京兆尹,護治豐陵,未成,卒,贈吏部尚書。



 萬石,頗涉學,善音律。上元中,遷累太常少卿。當時郊廟燕會樂曲,皆萬石與太史令姚元辯增損之,號任職。始,萬石奏「太樂博士弟子遭喪者,先無它業,請以卒哭追集」。侍御史劉思立劾奏萬石曰:「移風易俗,莫善於樂;睦親化人,莫善於孝。所以三年之禮,天下通喪。今遣音聲人釋服為樂,帶絰治音,豈以小人不能執禮,遂欲約為非法?萬石官太常,首紊風化,請付吏論罪。」高宗方委任萬石,罷其奏。後知吏部選事,卒於官。



 贊曰:王者用人非難,盡其才之為難。觀太宗之責任也,謀斯從,言斯聽,才斯奮,洞然不疑,故人臣未始遺力,天子高拱操成功,致太平矣。始皆奮亡命布衣,嬪然列置上袞。薛收雖早夭,帝本以中書令待之。御臣之方,顧不善哉!挺晚節流落,蓋有致而然。



\end{pinyinscope}