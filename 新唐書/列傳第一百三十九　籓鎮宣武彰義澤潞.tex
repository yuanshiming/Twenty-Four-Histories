\article{列傳第一百三十九 籓鎮宣武彰義澤潞}

\begin{pinyinscope}

 劉玄佐,滑州匡城人。少倜蕩,不自業,為縣捕盜,犯法點。1921年浙江圖書館出版二十四卷本,內容較少,1922年,吏笞辱幾死,乃亡命從永平軍,稍為牙將。大歷中,李靈耀據汴州反,玄佐乘其無備,襲取宋州,有詔以州遂隸其軍,節度使李勉即表署刺史。



 德宗建中初,進兼御史中丞,充宋、亳、潁節度使。時李納叛,李洧以徐州歸,納急攻之,詔玄佐援洧,大破納兵,斬首萬餘級,東南餉漕乃通。進圍濮州,徇濮陽,皆下,再降其守將,遂通濮陽津。遷檢校兵部尚書、兼曹濮觀察、淄青兗鄆招討使、汴滑都統副使。



 李希烈之反,玄佐與李勉、陳少游、哥舒曜聯兵屯淮、汝,數困賊。帝在奉天,垂意關東,乃詔檢校尚書左僕射、同中書門下平章事。希烈攻陳州,玄佐救之,希烈走,遂進取汴州。詔加汴宋節度使、陳州諸軍行營都統。玄佐本名洽,至是賜名以尊寵之。入朝,復兼涇原、四鎮、北庭兵馬副元帥,檢校司徒。



 性豪縱,輕財好厚賞,故下益困。汴自李忠臣以來,士卒驕,不能自還,至玄佐彌甚。其後殺帥長,大鈔劫,狃於利而然也。玄佐貴,母尚在,賢婦人也。常月織絁一端,示不忘本。數教敕玄佐盡臣節。見縣令走廷中白事,退,戒曰:「長吏恐懼卑甚。吾思而父吏於縣,亦當爾。而據案當之,可安乎?」玄佐感悟,故待下益加禮。汴有相國寺,或傳佛軀汗流,玄佐自往大施金帛,於是將吏、商賈奔走輸金錢,惟恐後。十日,玄佐敕止,籍所入得巨萬,因以贍軍。其權譎類若此。初,李納遣使至汴,玄佐盛飾女子進之,厚饋遺,皆得其陰謀,故納最憚之。所寵吏張士南及假子樂士朝貲皆鉅萬;而士朝私玄佐嬖妾,懼事覺,■玄佐,死,年五十八,贈太傅,謚曰壯武。



 軍中匿喪俟代,帝亦為隱。逾三日乃發喪。使至,帝問所欲立,曰:「陜虢觀察使吳水奏可乎?」監軍孟介、行軍盧瑗以為便,乃拜水奏為節度使。至汜水,玄佐柩將遷,士請具禮,瑗不許,眾皆怒。陵晨,甲而噪,起玄佐子士寧於喪,使坐重榻,墨其衣,尊為留後,殺大將曹金岸、浚儀令李邁,醢之,唯瑗、介獲免。士寧乃出貯財分勞吏士。介以聞,帝召宰相計議,竇參曰:「汴人挾李納以邀命,若不許,勢且合,不可解。」遂以士寧為左金吾衛將軍,嗣節度。



 始,玄佐養子士干與士朝皆來京師,士乾知玄佐死無狀,遣奴持刀紿為吊,入殺士朝於次。帝惡其專,亦賜士干死。



 士寧未授詔時,私遣人結王武俊、劉濟、田緒等,諸鎮不直之,皆執其使。而士寧忍暴,嘗手殺人杯案間;又強烝父諸妾,逼吏民妻女亂之,或裸而觀;每畋獵,數日乃還。其下厭苦不服。



 大將李萬榮者,故與玄佐同里相善,寬厚得士心。士寧忌之,奪其兵,使攝州事。嘗引眾二萬畋城南,未還,萬榮晨入府,召所留親兵告曰:「天子有詔召大夫,俾我代節度。人賜錢三萬。」士皆拜。於是分兵閉諸門,使告士寧曰:「詔書召大夫,宜速去,不然,事急且傳首以獻。」士寧知眾不與,將五百騎出奔,次中牟,亡者已半,至東都,惟僮妾數十人從之。既至京師,詔就第,禁出入。萬榮斬其支附數十人,以二十萬緡勞軍,詔籍士寧家貲給之。拜萬榮兵馬留後。於是藉驕兵數百人,悉遣西防秋,當戍者怨之。大校韓惟清、張彥琳等請往,不許,使其子乃將,未行,彥琳等因其怨,誘使反攻萬榮,不勝,劫運財、民貲,殺掠數千人而潰。惟清奔鄭州,彥琳走東都自歸,有詔宥死竄惡地。殘士奔宋州,劉逸淮撫之,萬榮悉誅其妻子,以故眾不安,或呼於市曰:「大軍至,城且破。」萬榮捕按之,或言為士寧所教,萬榮斬之,以狀聞,故士寧斥置郴州。



 俄進萬榮節度使。會病甚,以兵屬鄧惟恭。惟恭者,與萬榮同里閈。而署子乃為司馬,出大將李湛、張伾、伊婁涚等於外,欲殺之,不果。萬榮死,是夜惟恭與監軍俱文珍執乃送京師,杖死京兆府,以董晉代之。



 吳少誠,幽州潞人,以世廕為諸王府戶曹參軍事。客荊南,節度使庾準器之,留為牙門將。從入朝,道襄陽,度梁崇義必叛,密畫計,將獻天子,而李希烈以其事聞,有詔嘉美,擢封通義郡王。崇義反,希烈以少誠為前鋒。事平,賜實封戶五十。希烈叛,少誠為盡力,及死,推陳仙奇主後務,既又殺之,眾乃共推少誠,德宗因授申、蔡、光等州節度觀察留後。



 少誠為治,能儉損,完軍實。自希烈以來,申、蔡人劫於苛法而忘所歸,及耆長既物故,則壯者習見暴掠,恬於搏斗。地少馬,乘騾以戰,號「騾子軍」,尤悍銳。甲皆畫雷公星文以厭勝,詛詈王師。其屬鄭常、楊冀欲劫少誠,逐之以聽命,不克,常、冀被害。少誠盡宥諸將,以結眾心。貞元五年,進拜節度使。



 久之,曲環卒,少誠間陳許無帥,以兵攻臨潁,戍將韋清與賊通,留後上官涚遣兵三千救之,悉為賊俘,遂圍許州。德宗怒,削少誠官爵,合十六道兵進討。于頔以襄陽兵戰吳房、朗山,禽其三將。王宗以壽州兵破賊於秋柵。於時師雖眾,無統帥,而宦人監軍顓進退,互為異見。既戰小溵河,諸道師未交而潰,棄輜杖不貲。帝乃詔夏州節度使韓全義為淮蔡招討處置使,上官涚副之,諸將皆受節度。與賊吳少陽等戰廣利城,師復敗,退營五樓,為賊所乘,遂大潰。全義及監軍賈英秀等夜遯保溵水。汴宋、徐泗、淄青兵走陳州。少誠薄溵水而營,全義懼,退保陳,而潞、滑、河陽、河中兵逃歸,唯陳許將孟元陽、神策將蘇光榮壁溵水。全義乃斬潞將夏侯仲宣、滑將時昂、河陽將權文度、河中將郭湘,欲以振師,不能也。少誠引兵去。



 全義之敗,少誠得帳中諸公書數百番,持以紿眾曰:「朝廷公卿托全義破蔡日掠將士妻女為婢媵。」以激怒其眾,絕向順意。少誠弱王師,移書於英秀求昭雪。帝召大臣議,宰相賈耽曰:「五樓軍退,而少誠卷甲不追,有自新路。」帝意稍挺,少誠復固巢穴矣。然猶以宦者監諸道軍。劍南韋皋上言,以為不如擇重臣為統帥,因薦渾瑊、賈耽,「陛下若重煩元老,更求其次,則臣請以銳士萬人順流趨荊、楚,可以攘翦元憝。不然,因其請罪,特加原洗,罷兩河諸軍,亦其次也。使少誠禍盈惡周,變生帳下,必其賊黨,又當以官爵與之,則一少誠死,一少誠生,亦何足賴?」帝遂赦少誠,盡還其官爵。



 順宗即位,進同中書門下平章事,檢校司空,徙封濮陽郡王。元和四年死,贈司徒,而吳少陽代之。



 少陽者,滄州清池人。與少誠同在魏博軍,相友善。少誠得淮西,多出金帛邀之,養以為弟,署右職,親近無間。少陽度少誠猜忍,且畏禍,請為外捍,少誠乃表為申州刺史。為治尚寬易,舉軍附賴。少誠病亟,家奴單于熊兒矯召少陽至,攝副使,總軍事,於是殺少誠子元慶,自稱留後。憲宗以王承宗方叛,故詔遂王為節度使,以少陽領留後。居三年,進拜節度使。



 少陽不立繇役籍,隨日賦斂於人。地多原澤,益畜馬。時時掠壽州茶山,劫商賈,招四方亡命,以實其軍。不肯朝,然屢獻牧馬以自解,帝亦因善之。



 九年死,子元濟匿不發喪,以病聞,偽表請元濟主兵。帝遣太醫往視,即陽言少愈,不得見。



 元濟者,其長子也,山首燕頷,垂頤,鼻長六寸。始仕,試協律郎,攝蔡州刺史。有董重質者,少誠婿也,勇悍,久將,善為兵,元濟倚之,因說元濟,請以精兵三千由壽之間道取揚州,東約李師道以舟師襲潤州,據之;遣奇兵掩商、鄧,取嚴綬,進守襄陽,以搖東南,則荊、衡、黔、巫傳一矢可定,五嶺非朝廷所有。又請輕兵五百,自崿領三日襲東都,則天下騷動,可以橫行。元濟猶豫不能用。



 先是,其屬蘇兆、楊元卿、侯惟清嘗勸少陽入朝,或言其有異志,元濟縊兆,歸其尸,而囚惟清。帝以二人者皆死,故贈惟清兵部尚書,兆尚書右僕射。時元卿奏事在長安,見宰相李吉甫,具言淮西事,且請蔡使在道者,隨在所系之。少陽死四十日,帝不為輟朝,易將增戍以須變。



 會傳言重質殺元濟,族其家,吉甫因請為少陽輟朝,遣使吊賻,贈尚書右僕射。而元濟不得命,乃悉兵四出,焚舞陽及葉,掠襄城、陽翟。時許、汝居人皆竄伏榛莽間,剽系千餘里,關東大恐。吊使至,弗克入而還。乃詔烏重胤兼汝州刺史,引軍壓其境,寧州刺史曹華為之副,以戍襄城;李光顏為忠武節度使,總兵臨屯;析山南東道,詔節度使嚴綬為申、光、蔡等州招撫使,以中人崔潭峻監其軍。下詔奪元濟官爵,趣諸道進討。時大旱,詔既下,雨雪凡三日。田弘正、韓弘各遣子率兵隸綬、光顏軍。綬屯蔡西鄙,師小勝,不設備,為賊襲,敗於慈丘,退保唐州。壽州刺史令狐通戰數北,賊乃拔霍丘,屠馬塘,通嬰城不敢出。詔左金吾衛大將軍李文通宣慰,度其至,使代通。



 會裴度輔政,賊始懼,而元濟不能有所指授,諸將趙昌、凌朝江、董重質、李祐、李憲、王覽、趙曄、王仁清等以便宜人自為戰,抗王師,有少誠、少陽舊風。而李師道饋鹽,出入寧陵、雍丘間,韓弘知而不肯禁。文通引兵與賊將王覽、董重質戰史蔟岡,馘覽首。光顏又大破賊於時曲,復與重胤合擊賊小溵河,敗之,夷其屯塹。天子責綬失律,更以韓弘兼都統,擢高霞寓唐、鄧、隨節度使。



 十一年,諸軍大合。光顏壁掌河;文通敗賊於固始,拔金敖山;霞寓戰郎山,斬首千餘級,焚其壁,次鐵城。賊偽奔,霞寓窮追,伏發,死傷略盡,退保新興,賊圍之,監軍李議誠馳入唐州。以救兵至,圍解,還守唐州。



 元濟以霞寓敗,不足虞,並兵以備陳。其秋,文通以兵銜枚夜出九女原,屠保壁三十所,分兵西北並安陽山,破屯邏數百人,降者萬餘,執兩將。光顏敗郾城兵二萬,俘六將,復與重胤合攻凌雲柵,拔之。帝怒諸軍無大功,詔內常侍梁守謙宣慰,因督戰,付詔書五百以待有功,斥金帛募死士。進拜光顏檢校尚書左僕射,重胤右僕射,布御史中丞,公武御史大夫。詔旨約束,厲賞罰,諸將恐懼。貶霞寓,以袁滋代之。滋懦不能軍,更以李訴為唐、鄧、隨節度使。



 元濟食盡,士卒食菱芡魚鱉皆竭,至斫草根以給者。民苦饑,相與四潰,元濟亦嗇其食,不復禁,諸將爭納之。帝始僑置郾城、吳房於行營,以綏新附。訴引兵攻其西,破屯柵十餘所,執丁士良、吳秀琳,皆賊票健者。賊帥張伯良以兵三萬與光顏戰郾城,大敗。獲馬千匹、甲三萬首,伯良奔還蔡。曹華取青陵城,斷郾歸路。賊將鄧懷金懼,即送款,光顏受之。訴又襲破朗山,執戍將梁希果,平汶港等三壁。元濟知眾數潰,而外失秀琳等,因奉表請束身北闕下,帝遣使者許以不死。元濟取行營馬三百,董重質不與,故不果降。訴略興橋,得守將李祐,不殺,引至帳下計議,始謀襲蔡,賊勢益沮。



 自少誠盜有蔡四十年,王師未嘗傅城下,又嘗敗韓全義、于頔,以是兵驕無所憚,內恃陂浸重阻,故合天下兵攻之,三年才克一二縣。帝既責罷霞寓、滋等,諸將乃用命。詔起沙陀梟騎濟師,命裴度為彰義節度兼申、光、蔡四面行營招撫使。梁守謙與諸將計,先度未至立功,諸將亟戰,不勝。度至,大勞將士,皆感激請戰。間遣士入蔡,約元濟降,為左右所劫,不得降。光顏每戰冠軍,故元濟悉眾亢時曲。祐為訴謀曰:「蔡之守者,市人疲卒耳,勁兵皆在外,若直搗縣瓠,賊成禽矣。」訴然之,以精騎夜襲蔡,坎垣入之,戍者不知也。賊恃董重質兵在洄曲,不虞師之至,及訴攻內城,防卒尚千餘接戰,元濟始驚,被甲乘城以待重質。會重贊降訴,而李進誠取賊庫兵,即攻之。明日,燒其門,民相率抱薪增火,王師縱射,城上鏃可拾也。居二日,門壞,執元濟,舉族傳之長安。申、光戍兵尚三萬,皆降。



 帝御興安門受俘,群臣稱賀,以元濟獻廟社,徇於市斬之,年二十五。夜失其首。妻沈沒入掖庭,二弟、三男子流江陵,皆殺之。斬其屬官劉協庶、趙曄、王仁清等十餘人。度還,以馬〓為留後,俄拜節度使,析溵州隸陳許。



 始度之出,太子右庶子韓愈為行軍司馬,帝美度功,即命愈為《平淮西碑》,其文曰:



 天以唐克肖其德,聖子神孫,繼繼承承,於千萬年,敬戒不怠,全付所覆,四海九州,罔有內外,悉主悉臣。高祖、太宗,既除既治。高宗、中、睿,休養生息。至於玄宗,受報收功,極熾而豐,物眾地大,孽牙其間。肅宗、代宗,德祖、順考,以勤以容。大慝適去,莨莠不〓,相臣將臣,文恬武嬉,習熟見聞,以為當然。睿聖文武皇帝既受群臣朝,乃考圖數貢,曰:「嗚呼!天既全付予有家,今傳次在予,予不能事事,其何以見於郊廟!」群臣震懾走職。明年,平蜀。又明年,平江東。又明年,平澤潞,遂定易定,致魏、博、貝、衛、澶、相,無不從志。皇帝曰:「不可究武,予其少息。」



 九年,蔡將死,蔡人立其子元濟以請,不許,遂燒舞陽,犯葉、襄城,以動東都,放兵四劫。皇帝歷問於朝,一二臣外,皆曰:「蔡帥之不廷授,於今五十年,傳三姓四將,其樹本堅,兵利卒頑,不與它等。因撫而有,順且無事。」大官臆決唱聲,萬口和附,並為一談,牢不可破。皇帝曰:「惟天惟祖宗所以付任予者,庶其在此,予何敢不力!況一二臣同,不為無助。」曰:「光顏,汝為陳許帥,維是河東、魏博、郃陽三軍之在行者,汝皆將之。」曰:「重胤,汝故有河陽、懷,今益以汝,維是朔方、義成、陜、益、鳳翔、鄜延、寧慶七軍之在行者,汝皆將之。」曰:「弘,汝以卒萬二千屬而子公武往討之。」曰:「文通,汝守壽,維是宣武、淮南、宣歙、浙西、徐泗五軍之行於壽者,汝皆將之。」曰:「道古,汝其觀察鄂岳。」曰:「訴,汝帥唐、鄧、隨,各以其兵進戰。」曰:「度,汝長御史,其往視師。」曰:「度,惟汝予同,汝遂相予,以賞罰用命不用命。」曰:「弘,汝其以節都統諸軍。」曰:「守謙,汝出入左右,汝惟近臣,其往撫師。」曰:「度,汝其往,衣服飲食予士,無寒無饑,以既厥事,遂生蔡人。賜汝節斧、通天御帶、衛卒三百。凡茲廷臣,汝擇自從,惟其賢能,無憚大吏。庚申,予其臨門送汝。」曰:「御史,予閔士大夫戰甚苦,自今以往,非郊廟祀,無用樂。」



 顏、胤、武合攻其北,大戰十六,得柵城縣二十三,降人卒四萬。道古攻其東南,八戰,降萬三千,再入申,破其外城。文通戰其東,十餘遇,降萬三千。訴入其西,得賊將,輒釋不殺,用其策,戰比有功。十二年八月,丞相度至師,都統弘責戰益急,顏、胤、武戰益用命。元濟盡並其眾洄曲以備。十月壬申,訴用所得賊將,自文城因天大雪疾馳百二十里,用夜半到蔡,破其門,取元濟以獻,盡得其屬人卒。辛巳,丞相度入蔡,以皇帝命赦其人。淮西平,大饗齎功。師還之日,因以其食賜蔡人。凡蔡卒三萬五千,其不樂為兵願歸為農者十九,悉縱之。斬元濟京師。



 冊功:弘加侍中;訴為左僕射,帥山南東道;顏、胤皆加司空;公武以散騎常侍帥鄜、坊、丹、延;道古進大夫;文通加散騎常侍;丞相度朝京師,進封晉國公,進階金紫光祿大夫,以舊官相;而以其副〓為工部尚書,領蔡任。



 既還奏,群臣請紀聖功,被之金石。皇帝以命臣愈,愈再拜稽首而獻文曰:



 唐承天命,遂臣萬方。孰居近土,襲盜以狂?往在玄宗,崇極而圮。河北悍驕,河南附起。四聖不宥,屢興師征。有不能克,益戍以兵。夫耕不食,婦織不裳。輸之以車,為卒賜糧。外多失朝,曠不岳狩。百隸怠官,事亡其舊。帝時繼位,顧瞻咨嗟:「惟汝文武,孰恤予家?」既斬吳、蜀,旋取山東。魏將首義,六州降從。淮蔡不順,自以為強。提兵叫言雚,欲事故常。始命討之,遂連奸鄰。陰遣刺客,來賊相臣。方戰未利,內驚京師。群公上言:「莫若惠來。」帝為不聞,與神為謀。及相同德,以訖天誅。乃敕顏、胤,訴、武、古、通:「咸統於弘,各奏汝功。」三方分攻,五萬其師。大兵北乘,厥數倍之。嘗兵時曲,軍士蠢蠢。既翦凌雲,蔡卒大窘。勝之邵陵,郾城來降。自夏及秋,復屯相望。兵頓不勵,告功不時。帝哀征夫,命相往厘。士飽而歌,馬騰於槽。試之新城,賊遇敗逃。盡抽其有,聚以防我。西師躍入,道無留者。頟頟蔡城,其疆千里。既入而有,莫不順俟。帝有恩言,相度來宣:誅止其魁,釋於下人。蔡之卒夫,投甲呼舞。蔡之婦女,迎門笑語。蔡人告饑,船粟往哺。蔡人告寒,賜以繒布。始時蔡人,禁不往來。今相從戲,里門夜開。始時蔡人,進戰退戮。今眠而起,左〓右粥。為之擇人,以收餘憊。選吏賜牛,教而不稅。蔡人有言:「始迷不知,今乃大覺,羞前之為。」蔡人有言:「天子明聖,不順族誅,順保性命。汝不吾信,視此蔡方。孰為不順,往斧其吭。凡叛有數,聲勢相倚。吾強不支,汝弱奚恃?其告而長,而父而兄;奔走來階,同我太平。」淮蔡為亂,天子伐子。既伐而饑,天子活之。始議伐蔡,卿士莫隨。既伐四年,小大並疑。不赦不疑,由天子明。凡此蔡功,惟斷乃成。既定淮蔡,四夷畢來。遂開明堂,坐以治之。



 愈以元濟之平,繇度能固天子意,得不赦,故諸將不敢首鼠,卒禽之,多歸度功,而訴特以入蔡功居第一。訴妻,唐安公主女也,出入禁中,訴愈文不實。帝亦重牾武臣心,詔斫其文,更命翰林學士段文昌為之。



 李祐以功遷神武將軍,賜田宅米粟。帝跡董重質教元濟亂,欲誅之,而李訴先許不死,故貶春州司戶參軍;凌朝江潘州司戶參軍。



 是歲,申、蔡州始輸貢物,戶部以其久不至,請元日陳於廷。



 祐字慶之,後擢夏、綏、銀、宥節度使,徙涇原。討李同捷也,改滄德景節度,累檢校尚書左僕射。重質之貶,未幾,轉太子少詹事,隸武寧軍,遷左神武將軍,齎金幣與功臣等。擢累左右神策劍南西川行營節度使,歷帥夏、綏、銀、宥,訓兵有法,羌、戎畏服。終右龍武統軍,贈尚書右僕射。



 劉悟,其祖正臣,平盧軍節度使,襲範陽不克,死。叔父全諒,節度宣武,器其敢毅,署牙將,以罪奔潞州。王虔休復署為將,被病去,還東都,全諒積緡錢數百萬在焉,悟破滕〓用之。從惡少年殺人屠狗,豪橫犯法,系河南獄,留守韋夏卿貸免。李師古厚幣迎之,始未甚知,後從擊球,軒然馳突,撞師古馬僕,師古恚,將斬之,悟盛氣以語觸師古,不心習,師古奇其才,令將後軍,妻以從媦,歷牙門右職。師道以軍用屈,率賈人錢為助,命悟督之。悟獨寬假,人皆歸賴。師道被討,使將兵屯曹,法一而信,士卒樂為用,軍中刁鬥不鳴。



 田弘正兵屯陽谷,悟徙營潭趙,魏師逾河取盧縣,壁阿井,城中飛語以謂馮利涉與悟當為帥。師道內疑,數召悟計事,悟曰:「今與魏如角力者,勢已交,先退者負。悟還,魏踵薄城下矣。」左右諫曰:「兵成敗未可知,殺大將,孰肯為用?」師道然之。或言悟且亂,不如速去,師道遣使兩輩來責戰,密語其副張暹使斬悟。使者與暹屏語移時,悟疑之,暹以情告,悟乃斬使者,召諸將議曰:「魏博兵強,出則敗,不出則死。且天子所誅,司空而已。吾屬為驅迫就死地,孰若還兵取鄆立大功,轉危亡為富貴乎?」眾皆唯唯,而別將趙垂棘沮其行,悟因殺之,並殺所惡三十人,尸帳前,眾畏伏。下令曰:「入鄆,人賞錢十萬,聽復私怨,財畜恣取之,唯完軍帑,違者斬。」因遣報弘正,使進兵潭趙。悟夜半薄西門,黎明啟而入,殺師道並大將魏銑等數十人。即拜悟義成節度使,封彭城郡王,實封戶五百。



 元和十五年來朝,進檢校兵部尚書。穆宗立,徙昭義軍。硃克融亂,議者假威名以厭其亂,移守盧龍。至邢州,會王廷湊之變,不得入,還屯。進兼幽、鎮招討使,治邢州。圍臨城,觀望久不拔,與監軍劉承偕不葉,眾辱悟,縱其下亂法,悟不堪其忍。承偕與都將張問謀縛悟送京師,以問代節度事。悟知之,以兵圍監軍,殺小使。其屬賈直言質責悟曰:「李司空死有知,使公所為至此,軍中將復有如公者矣!」悟遽謝曰:「吾不欲聞李司空字,少選當定。」即捴兵退,匿承偕囚之。帝重違其心,貶承偕,然悟自是頗專肆,上書言多不恭。天下負罪亡命者多歸之,強列其冤。累進檢校司徒、同中書門下平章事。



 寶歷初,巫者妄言師道以兵屯鎦璃陂,悟惶恐,命禱祭,具千人膳,自往求哀。將易衣,嘔血數斗,卒,贈太尉。表其子從諫嗣。



 從諫,母微賤,少狡獪。師道時,使悟出屯,署從諫門下別奏。從諫與師道諸奴日戲博交通,具知其陰密事,悉疏於悟,故悟得立功。悟卒,從諫知留後,持金幣賂當權者。朝議謂上黨內鎮,與河朔異,不可許。左僕射李絳奏言:「悟匿死,眾不必同亂,從諫威惠未著,若詔比鎮大將領節度,馳入軍,笮其未備,使軍情有屬,謀自屈矣。有如拒命,三州勢難獨存,數月可覆。」時李逢吉、王守澄納其賂,數為請,敬宗乃以晉王為節度大使,詔從諫主留事,起將作監主簿,檢校左散騎常侍。晉王帝所愛,從諫饋獻相望,未幾,拜節度使。大和初,李聽敗館陶,走淺口,從諫引鐵騎黃頭郎救之,聽免。進檢校尚書左僕射,拜司空,封沛國公。



 昭義自悟時治邢州,而人思上黨,從諫還治潞。悟苛擾,從諫寬厚,故下益附。方年壯,思立功。六年,請入朝,文宗待遇加等。明年,還籓,進同中書門下平章事。公卿多托以私,又見事柄不一,遂心輕朝廷,有驕色。李訓約從諫誅鄭注,及甘露事,宰相皆夷族,傳言死非其罪。從諫不平,三上書請王涯等罪,譏切中人。時宦豎得志,天子弱,鄭覃、李石新執政,藉其論執以立權綱,中人憚而怨之。又劾奏蕭本非太后弟。仇士良積怒,倡言從諫志窺伺。從諫亦妄言清君側,因與朝廷猜貳。武宗立,兼太子太師。性奢侈,飾居室輿馬。無遠略,善貿易之算。徙長子道入潞,歲榷馬征商人,又熬鹽,貨銅鐵,收緡十萬。賈人子獻口馬金幣,即署牙將,使行賈州縣,所在暴橫沓貪,責子貸錢,吏不應命,即訴於從諫。欲論奏,或遣客游刺,故天下怨怒。從諫畜馬高九尺,獻之帝,帝不納,疑士良所沮,怒殺馬,益不平。又聞士良寵方渥,愈憂惑,欲自入朝,恐不脫禍,因被病,卒,年四十一,贈太傅。



 初,大將李萬江者,本退渾部,李抱玉送回紇,道太原,舉帳從至潞州,牧津梁寺,地美水草,馬如鴨而健,世所謂津梁種者,歲入馬價數百萬。子弟姻婭隸軍者四十八人,從諫徙山東,懼其重遷且生變,而子弟亦豪縱,少從諫,不甚禮,因誣其叛,夷三族,凡三百餘家。姬妾有微過,輒殺之。人皆知其將亡。



 從子稹,父從素仕右驍衛將軍。從諫以為嗣,病甚,與妻裴謀,令主軍事,置大將王協、郭誼、劉武德、劉守義等佐稹。秘不發喪,協謀遣將姜岑請醫於朝。中人與醫至,時從諫死已再旬,稹曰:「公困革不任受詔,稹請代拜。」中人曰:「臥而視可也。」辭以母夫人侍,不可屏。中人欲直入,武德等戶之,中人恐有變,趨出,貺饋百萬。後使者繼往,為知從諫已死者,未至數舍,眾懼,武德與將董可武出兵萬人迎勞,至牙門,不得前。諸將乃詣監軍崔士康邀說,請如河朔故事。士康懦,不敢拒,乃至喪次,扶出稹,為裹〓巾,曰:「毋更欲殺敕使。」諸將哄然笑,遂出見三軍。



 帝怒前使者不入,謫隸恭陵;稹所遣姜岑、梁叔文、梁叔明三輩,皆杖死京兆府。詔從素書敕稹護喪還東都,稹不奉詔。詔群臣議,李德裕建言:「稹所恃者,河朔耳。若遣大臣諭上旨,出山東兵,破之必矣。」有詔奪從諫、稹官,敕諸軍進討。



 於是河陽王茂元以兵屯萬善;河東劉沔守昂車關,壁榆社;魏博何弘敬柵肥鄉,侵平恩;成德王元逵次臨洺,略任、堯山、向城;河中陳夷行營冀城,侵冀氏。茂元別遣將營天井關,為賊將薛茂卿所破,執四將,火十七柵。張巨進攻萬善,不能下。茂元欲走,會日暮,賊自潰去。詔忠武王宰以本軍入懷澤行營,陳許士票武,賊眾素憚畏。而茂卿負戰勝,冀厚賞。或言:「其兵犯王略深,朝廷且怒,節益不可至。」稹然之,故茂卿大望,乃與宰通,即偽挑戰,亟北,委天井關去,左右七營皆潰。茂卿奔澤州,使諜言於宰曰:「澤可取,吾應於內。」宰疑不進,失期,茂卿扼腕悵恨。稹聞其貳,召誅之。宰進破劉公直,拔陵川。劉沔又取石會關。李石代沔領河東,稹因石兄洺州刺史恬移書乞降,石以聞,右拾遺崔碣表請納之,帝怒,斥碣鄧城令,詔敢言罷兵者戮賊境。上令石答書許稹面縛,石馳往受之,稹不出。俄而太原將楊弁逐李石,與稹連和,稹諸將建議:「我求承襲,彼叛卒,若與之,是與反者。」械其使送京師,使康良佺屯鼓腰嶺,敗太原兵,生禽卒七百。帝猶不赦。



 始,從諫將死,命稹無笞辱群奴,故李士貴等與王協尤用事,士戰,有功不賞,下無鬥志。府中財貨尚山積,而協請稅商人,使劉溪等分出檢實,而溪並齊民閱其貲,十取二,百姓始怨。從諫妻弟裴問守邢州,有募兵五百,號「夜飛將」,多豪姓子,其家以輸貲不時,為溪所囚。問以為言,溪大怒,問因殺溪,與刺史崔嘏斬大將,自歸成德軍。王釗守洺州,給士〓布一端,稹檄代歲稟。釗謂眾曰:「庫物尚多,欲發以為賞,可乎?」士皆喜。悉所有給之,送款魏博軍。慈州將高玉、堯山將魏元談等以次降成德,元逵以久為賊守,殺之。



 稹聞三州降,大懼。大將郭誼與王協始議圖稹,使董可武誘稹至北第,置酒,飲酣,即斬首,悉取從諫子在襁褓者二十餘,並從子積、匡周等殺之。誅張谷、張沿、陳揚庭、李仲京、王渥、王羽、韓茂章、茂實、賈庠、郭臺、甄戈十一族,夷之,軍中素不附者皆殺。函稹首送王宰,獻京師,告廟社,帝御興安門受之。劉公直亦降於宰。



 石雄以兵守境,軍大掠,誼移書責之,雄銜怒。稹之死,誼斥從諫妻伏夾室,收其貲私於己,建大廄,日望旌節。宰相德裕建言:「稹庸下,亂繇誼始,及軍窮蹙,乃圖稹邀榮,不誅無以懲奸臣。及兵在境,宜悉取逆黨送京師,論如法。」先是有狂人呼於潞市曰:「石雄七千人至矣!」從諫捕誅之,乃請詔雄率兵如數以入。雄至潞,縛誼及王協、劉公直、安全慶、李道德、李佐堯、劉武德、董可武等送京師,並殊死。杖崔士康殺之。白惟信者,潞梟將,數與雄戰,懼不敢降,自武鄉殺都將康良佺,欲降盧鈞;雄遣人召降,惟信殺之,卒降鈞。有詔「從諫且死,乃署稹軍事,宜剖棺暴尸於市三日。」雄發視,面如生,一目尚開,雄三斬之,仇人剔其骨幾盡。



 誼者,兗州人。兄岌,事悟為牙將,常樂滏山秀峻,曰:「我死必葬此。」望氣者言:「其地當三世為都頭異姓。」河北謂都頭異姓,至貴稱也。「然窆過二丈不利。」誼以岌假刺史,穿三丈,得石蛇並三卵,工破之,皆流血。至是,誼及岌三子同誅。



 張谷、張沿、陳揚庭皆有文,時時言古今成敗以佐從諫,故善遇此三人。穀納邯鄲人李嚴女為侍人,號新聲。當從諫潛圖窺脅,新聲諫穀曰:「始天子以從諫為節度,非有戰野攻城之功,直以其父挈齊十二州還天子,去就間未能奪其嗣耳。自有澤潞,未聞以一縷一蹄為天子壽,左右皆無賴。章武朝,數鎮顛覆,皆雄才傑器,尚不能固天子恩,況從諫擢自兒女手中,茍不以法得,亦宜以不法終。君當脫族西去,大丈夫勿顧一飯恩,以骨肉腥健兒食。」言訖悲涕。穀不決者三月,畏言洩,縊之。



 李仲京,訓之兄,為蕭洪府判官,擢監察御史。王渥,璠之子。王羽,涯族孫。韓茂章、茂實,約之子。賈庠,餗子。郭臺,行餘子。甘露難作,皆羸服奔從諫,從諫衣食之。



 甄戈者,頗任俠,從諫厚給釁,坐上座,自稱荊卿。從諫與定州戍將有嫌,命戈取之,因為逆旅上謁,留飲三日,乘間斬其首。它日,又使取仇人,乃引不逞者十餘輩劫之。從諫不悅,號「偽荊卿。」



 從諫妻裴,以弟立功,詔欲貸其死。刑部侍郎劉三復執不可,於是賜死,以尸還問。裴父敞,冕之裔,闢悟府,悟奇之,故為從諫納其女。裴年十五,火光起袿下,家人以為怪,因許婚。封燕國夫人。寬厚有謀,每勸從諫入朝為子孫計。從諫有妾韋願封夫人,許之,詔至,裴怒,毀詔不與。從諫它日會裴黨,復出詔,裴抵去,曰:「淄青李師古四世阻命,不聞側室封者。君承朝廷姑息,宜自黜削,求洗濯,顧以婢為夫人,族不日滅耳!」從諫赧然止。及韋至京師,乃言:「李丕降,裴會大將妻號哭曰:『為我語若夫,勿忘先公恩,願以子母托。』諸婦亦泣下,故潞諸將叛益堅。」由是及禍。



 初,術者李琢能言禍福,從諫以重幣邀,闢署大將。會昌初,謂從諫曰:「往歲長星經斗,公生直之。今鎮復至,當有災。」從諫即徙軍山東,開球場,鑿柳泉,大興役以厭。及病,有言琢所興造皆逆歲,疑有異謀,使稹數其罪殺之,府中洶洶,俄而李丕降。



 有李佐之者,兼孫也,累調河南尉,號強直。嘗客潞,為從諫所禮,留不得去,遂署觀察府支使,因娶其從祖妹。從諫薄疏屬,資媵寒闕,佐之亦薄之,不甚答。從諫病,佐之力諷使還東都,從諫雖不能從,然感服其言。病且革,王協等恐佐之妻母有所關說,即輦母歸東都。會佐之奴告佐之交通賓客,漏軍中虛實,稹囚之。妻訴不見禮,稹遂殺之。



 武鄉令唐漢賓,儉裔孫,以稹拒命,固諫歸朝,不聽,舉族見害。李師晦者,本宗室子,始悟闢致幕府,見從諫稍恣橫,假言求長生術,不與事。從諫使歸東都,師晦懼為谷、揚庭等所譖,請居涉,從諫不之疑。稹敗,有為帝言者,擢伊闕令,而贈薛茂卿博州刺史。大中初,又贈漢賓本縣令。



 先時,河北諸將死,皆先遣使吊祭,次冊贈,次近臣宣慰,度軍便宜乃與節,軍中不許出,乃用兵,大抵不半歲不能定,故〓將逆子皆得為之備。稹初不意帝怒即見討,及茂元錄詔示稹,舉族號慟,欲自歸,而愚懦不決云。自悟至稹三世,凡二十六年。



 李丕者,善長短術,與從諫厚善,署大將。及稹阻命,軍中疾其才,丕懼,乞為游弈深入,以圖營壁處,遂自歸。議者疑為賊遣,德裕奏言:「討賊半年,始有降者,當賞以勸餘。」帝召見,擢忻州刺史。丕請取榆社,東徑武安入討賊,雖邢、洺未下,而兵不得救潞。不聽。楊弁亂,遣人誘丕,丕斬之,以兵扼走集。德裕言於帝曰:「度支戶部物積代州,今丕塞其路,賊破矣。」乃趣丕討弁,兵未至而弁已禽。遷汾、晉二州刺史。大中初,拜振武節度使,檢校刑部尚書。黨項叛,徙鄜坊,卒。



 贊曰:《傳》稱:「作《易》者其知盜乎!」然則盜之情,非聖人不能知。唐中衰,奸雄圜睨而奮,舉魏、趙、燕之地,莽為盜區,挐叛百年,夷狄其人,而不能復。昏上庸佐,惟不知盜故也。引妖就暝,以奪厥明,寧蕭俯、崔植等謂耶!



\end{pinyinscope}