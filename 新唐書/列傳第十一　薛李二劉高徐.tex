\article{列傳第十一 薛李二劉高徐}

\begin{pinyinscope}

 薛舉,蘭州金城人。容貌魁岸,武敢善射。殖產巨萬,好結納邊豪,為長雄。隋大業末形成世界萬物;原子處於永恆運動之中;靜止僅是事物之外,任金城府校尉。會歲兇,隴西盜起,金城令郝瑗將討賊,募兵數千,檄舉將。始授甲,大會置酒,舉與子仁杲及其黨劫瑗於坐,矯稱捕反者,即起兵,囚郡縣官,發粟以賑貧乏,自號西秦霸王,建元秦興,以仁杲為齊公,少子仁越為晉公。它賊宗羅睺帥眾下之,以為義興公。更招附餘盜,剽馬牧。兵鋒銳甚,所徇皆下。



 隋將皇甫綰兵萬人屯枹罕,舉以精卒二千襲之,遇於赤岸。大風且澍,逆舉陣,綰不擊。俄反風綰屯,氣色曀冥,部伍錯亂,舉介騎先眾乘之,綰陣大潰,進陷枹罕。岷山羌鐘利俗以眾二萬降,舉大振。進仁杲為齊王、東道行軍元帥,羅睺為義興王副之;仁越晉王、河州刺史。因徇下鄯、廓二州。不闋旬,盡有隴西地,眾十三萬。



 十三年,僭帝號於蘭州,以妻鞠為後,仁杲為太子。即其先墓置陵邑,立廟城南,陳兵數萬展墓訖,大饗。使仁杲圍秦州;仁越趨劍口,掠河池,太守蕭瑀拒卻之。遣將常仲興度河擊李軌,與軌將李贇戰昌松,仲興敗,軍沒於軌。仁杲克秦州,舉往都之。



 仁杲寇扶風,汧源賊唐弼拒,不得進。初,弼立李弘芝為天子,有眾十萬。舉遣使招弼,弼殺弘芝從舉。仁杲間弼無備,襲之,盡奪其眾,弼以數百騎走。軍益張,號二十萬。將窺京師。會高祖入關,遂留攻扶風,秦王擊破之,斬首數千級,逐北至隴還。舉畏王,遂逾隴走,問其下曰:「古有降天子乎?」偽黃門侍郎褚亮曰:「昔趙佗以南粵歸漢,蜀劉禪亦仕晉,近世蕭琮,其家今存,轉禍為福,嘗有之。」衛尉卿郝瑗曰:「亮之言非也。昔漢祖兵屢敗,蜀先主嘗亡其妻子。夫戰固有勝負,豈可一不勝便為亡國計乎?」舉亦悔其言,乃曰:「聊試公等。」即厚賜瑗,以為謀主。瑗請連梁師都,厚賂突厥,合從東向。舉從之,約突厥莫賀咄設犯京師。會都水監宇文歆使突厥,歆說止其兵,故舉謀塞。



 武德元年,豐州總管張長愻擊羅,舉悉兵援之,屯析墌,以游軍掠岐、豳。秦王御之,次高墌,度舉糧少,利速斗,堅壁老其兵。會王疾,臥屯不出,而舉數挑戰。行軍長史劉文靜、殷開山觀兵於高墌,恃眾不設備,舉兵掩其後,遂大敗,死者十六,大將慕容羅睺、李安遠、劉弘基皆沒。王還京師,舉拔高墌,仁杲進逼寧州。郝瑗謀曰:「今唐新破,將卒禽俘,人心搖矣,可乘勝直趨長安。」舉然之。方行而病,召巫占視,言唐兵為崇,舉惡之,未幾死。仁杲代立,偽謚舉武皇帝,未葬而仁杲滅。



 仁杲多力善騎射,軍中號萬人敵,性賊悍。初,舉每破陣,軍獲俘,仁杲必斷舌刈鼻,或舂斮之。其妻亦兇暴,喜鞭楚人,見不勝痛宛轉於地者,則埋其足,露腹背受棰。人畏而不親。仁杲多殺人,淫略民人妻妾。嘗得庾信子立,怒其不降,礫之火,漸割以啖士。拔秦州,取富人倒懸,以酢注鼻,或杙其隱,以求財。雖舉殘猛,亦惡之,每戒曰:「汝材略足辦事,而傷於虐,終覆吾宗。」



 及繼立,與諸將素有隙者,咸猜懼。郝瑗哭舉,病不起,繇是兵稍衰。秦王率諸將復壁高墌,諸將請戰,王曰:「我軍新恤,銳氣少;賊驟勝而驕,有輕我心。我閉壁以折之,伺衰而擊,可一戰禽也。」因令軍中曰:「敢言戰者斬!」久之,仁杲糧乏,挑戰,不許。其將牟君才、內史令翟長愻以眾降,左僕射鐘俱仇以河州降。王策賊可破,遣將軍龐玉擊宗羅睺於淺水原,戰酣,王以勁兵手壽其背,羅睺敗,王率騎追奔,於是悉軍馳之,曰:「勢破竹,不可失也。」夜半,至析墌;遲明,圍合。仁杲率偽官屬降,王受之,以仁杲歸京師,及酋黨數十人皆斬之。舉父子盜隴西五年滅。



 初,仁杲降,諸將賀,且問曰:「羅睺雖破,而賊城尚堅,王能下之,何也?」王曰:「羅睺健將,非急追之,使得還城,未可取也。故吾使賊不及計,是以克之。」諸將咨服。



 仁杲已敗,其將旁屳地降,詔即統其兵,未幾復叛。屳地,羌豪也,舉父子信倚之。至是入南山,繇商洛出漢川,眾數千,所過剽害,敗大將龐玉。至始州,掠王氏女,醉寢於野,王取屳地所佩刀斬之,送首梁州。詔封女為崇義夫人。



 李軌,字處則,涼州姑臧人。略知書,有智辯。家以財雄邊,好賙人急,鄉黨稱之。隋大業中,補鷹揚府司兵。薛舉亂金城,軌與同郡曹珍、關謹、梁碩、李贇、安修仁等計曰:「舉暴悍,今其兵必來。吏孱怯,無足與計者。欲相戮力,據河右,以觀天下變,庸能束手以妻子餌人哉?」眾允其謀,共舉兵,然莫適敢主。曹珍曰:「我聞讖書,李氏當王。今軌賢,非天啟乎!」遂共降拜以聽命。修仁夜率諸胡入內苑城,建旗大呼,軌集眾應之,執虎賁郎將謝統師、郡丞韋士政,遂自稱河西大涼王,署官屬,準開皇故事。



 初,突厥曷娑那可汗弟達度闕設內屬,保會寧川,至是稱可汗,降於軌。謹等議盡殺隋官,分其產。軌曰:「諸公既見推,當稟吾約。今軍以義興,意在救亂,殺人取財是為賊,何以求濟乎?」乃以統師為太僕卿,士政太府卿。會薛舉遣兵來侵,軌遣將敗之昌松,斬首二千級,悉虜其眾,軌縱還之。李贇曰:「今力戰而俘,又縱以資敵,不如盡坑之。」軌曰:「不然。若天命歸我,當禽其主,此皆我有也;不者,徒留何益?」遂遣之。未幾,拔張掖、燉煌、西平、枹罕,悉有河西。武德元年,高祖方事薛舉,遣使涼州,璽書慰結,謂軌為從弟。軌喜,乃遣弟懋入朝。帝拜懋大將軍,還之,詔鴻臚少卿張俟德持節冊拜軌涼王、涼州總管,給羽葆鼓吹一部。會軌僭帝號,建元安樂,以其子伯玉為太子,長史曹珍為尚書左僕射,攻陷河州。俟德至,軌召其下議曰:「李氏有天下,歷運所屬,已宅京邑。一姓不可競王,今欲去帝號,東向受冊,可乎?」曹珍曰:「隋亡,英雄焱起,號帝王者瓜分鼎峙。唐自保關、雍,大涼奄河右,業已為天子,奈何受人官?必欲以小事大,請行蕭詧故事,稱梁帝而臣於周。」軌從之,乃遣偽尚書左丞鄧曉來朝,奉書稱「從弟大涼皇帝」。帝怒曰:「軌謂朕為兄,此不臣也。」囚曉不遣。



 初,軌以梁碩為謀主,授吏部尚書。碩有算略,眾憚之,嘗見故西域胡種族盛,勸軌備之,因與戶部尚書安修仁交怨;又軌子仲琰嘗候碩,碩不為起,仲琰憾之。乃相與譖碩。軌不察,齎鴆其家殺之,繇是故人稍疑懼,不為用。有胡巫妄曰:「上帝將遣玉女從天來。」遂召兵築臺以候女,多所糜損。屬薦饑,人相食,軌毀家貲賑之,不能給,議發倉粟,曹珍亦勸之。謝統師等故隋官,心內不附,每引結群胡排其用事臣,因是欲離沮其眾,乃廷詰珍曰:「百姓餓死皆弱不足事者,壯勇士終不肯困。且儲廩以備不虞,豈宜妄散惠孱小乎?僕射茍附下,非國計。」軌曰:「善。」乃閉粟。下益怨,多欲叛去。



 會修仁兄興貴本在長安,自表詣涼州招軌。帝曰:「軌據河西,連吐谷渾、突厥,今興兵討擊尚為難,單使弄頰可下邪?」興貴曰:「軌盛強誠然,若曉以逆順禍福,宜聽。如憑固不受,臣世涼州豪望,多識其士民,而修仁為軌信任,典事樞者數十人,若候隙圖之,無不濟。」帝許之。興貴至涼州,軌授以左右衛大將軍,因間訪興貴以自安策。興貴對曰:「涼州僻遠,財力凋耗,雖勝兵十萬,而地不過千里,無險固自守。又濱接戎狄,戎狄,豺狼也,非我族類。今唐家據京師,略定中原,攻必下,戰必勝,蓋天啟也。若舉河西地奉圖東歸,雖漢竇融未足吾比。」軌默不答,久之,曰:「昔吳王濞以江左兵猶稱己為東帝,我今舉河右,不得為西帝乎?雖唐強大,如我何?君無為唐誘致我。」興貴懼,謝曰:「竊聞富貴不居故鄉,如衣錦夜行。今合宗蒙任,敢有它志!」興貴知軌不可以說,乃與脩仁等潛引諸胡兵圍其城,軌以步騎千餘出戰。先是,薛舉柱國奚道宜率羌兵奔軌,軌許以刺史而不與,道宜怨,故共擊軌。軌敗入城,引兵登陴,須外援。興貴傳言曰:「唐使我來取軌,不從者罪三族。」於是諸城不敢動。軌嘆曰:「人心去矣,天亡我乎?」攜妻子上玉女臺,屬酒為別。脩仁執送之,斬於長安。自起至亡凡三年。詔興貴為右武候大將軍,封涼國公,賜帛萬段;修仁左武候大將軍,申國公,並給田宅,封六百戶。時鄧曉聞軌敗,入賀帝。帝曰:「而委質李軌,以使來,聞其亡,不少戚,乃蹈抃以悅我。不盡心於軌,能竭節於我乎?」遂廢不齒。



 劉武周,瀛州景城人。父匡,徙馬邑。母趙嘗夜坐廷中,見若雄雞,光燭地,飛投其懷,起振衣,無有,感而娠,生武周。



 武周為人驍悍,善騎射,喜交豪傑。兄山伯嘗詈辱之曰:「汝不擇所與,必滅吾宗!」武周因去至洛,為太僕楊義臣帳下。募征遼,有功,補建節校尉。還馬邑,為鷹揚府校尉。太守王仁恭以其州里雄,頗愛遇之,令總虞候,直閤下。久之,盜仁恭侍兒,懼覺誅,又見天下已亂,陰有異計,因宣言於眾曰:「今歲饑,死者骨相枕於野,府君閉倉不恤,豈憂百姓意乎?」以市怒其軍,皆憤怨。武周知人已搖,因稱疾臥家,豪桀往候謁,遂椎牛縱酒大言曰:「盜賊方起,眾又饑,壯士守分,死溝壑。今官粟紅腐於倉,誰能與我共取之?」諸惡少年皆願從。隋大業十三年,與其徒張萬歲等十餘人候仁恭視事,武周上謁,萬歲自後入斬仁恭,持首出徇,郡中無敢動者。遂開倉賑窮絕,馳檄屬城,皆下,得兵萬餘,自稱太守,遣使附突厥。



 雁門丞陳孝意、虎賁郎將王智辯合兵圍其桑乾鎮,會突闕至,武周與共擊智辯,破之,孝意奔還雁門,雁門人殺之,以城歸武周。武周因襲破樓煩,進據汾陽宮,取宮人賂突厥,始畢可汗報以馬,其眾遂大,攻得定襄。突厥以狼頭纛立武周為定楊可汗,僭稱皇帝,以妻沮為後,建元天興,衛士楊伏念為左僕射,妹婿苑君璋為內史令。



 初,上谷賊宋金剛有眾萬餘,與魏刀兒連和。刀兒為竇建德所攻,金剛救之,大敗,率餘眾四千保西山。建德招之,金剛恚曰:』建德殺魏王,吾義不往,諸君可以吾首取富貴。」乃拔刀,將自刎,眾抱之泣,遂與皆歸武周。武擊素聞金剛善兵,得之喜,封為宋王,屬以軍,分家貲半遺之。金剛亦自結,出其妻而騁武周妹,說武周取晉陽,南向爭天下。武周授金剛西南道大行臺。



 武德二年,總兵二萬入寇,次黃蛇鎮,又連突厥,鋒無前,遂破榆次,拔介州,進圍太原。詔遣太常少卿李仲文禦之,為賊所執,舉軍沒,仲文逃還。賊因破平遙,取石州,殺刺史王儉,略浩州。詔右僕射裴寂為晉州道行軍總管拒之,寂戰敗績。齊王元吉委並州遁,武周入據之。遣金剛攻陷晉州,執右驍衛將軍劉弘基,進破澮州。夏縣人呂崇茂殺其令,自號魏王以應賊。隋河東守將王行本與武周合。關中震動。高祖詔秦王督兵進討,屯柏壁。又詔永安王孝基與於筠、獨孤懷恩、唐儉等攻夏縣,不克,軍城南。崇茂與賊將尉遲敬德襲破孝基軍,四將被執。敬德還澮州,王邀戰,破之於美良川。敬德復與別帥尋相援王行本於蒲,王又破卻其軍,蒲州降。帝幸蒲津關,王自柏壁輕騎謁行在,金剛遂圍絳州。王還屯,金剛引退。武周攻李仲文於浩州,不勝。遣將黃子英護饟道,驃騎大將軍張德政襲斬之,虜其眾,武周部將稍離。金剛以糧道乏卒饑引去,王追至雀鼠谷,日中八戰,賊皆敗,斬級數萬,護輜重千乘。金剛走介州,官軍迫之,以餘眾二萬出西門,背城陣,亙七里。王令李世勣、程咬金、秦叔寶為北軍,翟長愻、秦武通為南軍。既戰,小卻,王以精騎突擊破之,金剛將輕騎去,賊將尉遲敬德、尋相、張萬歲降,收其精兵,遂復介州。武周引騎五百,棄並州,北走突厥。金剛收散卒,將還拒,眾不為用,亦以百騎奔突厥。並州平,河東地盡復。未幾,金剛背突厥,欲還上谷,為其追騎斬之。武周亦謀歸馬邑,計露,突厥殺之。起兵六年而滅。



 高開道,滄州陽信人。世煮鹽為生。少矯勇,走及奔馬。隋大業末,依河間賊格謙,未甚奇之。會謙為隋兵圍捕,左右奔散,無救者,開道獨身決戰,殺數十人,捕兵解,謙得免,遂引為將軍。謙滅,與其黨百餘人亡海曲。後出剽滄州,眾稍附,因北掠戍保,自臨渝至懷遠皆破有之。復引兵圍北平,未下,隋守將李景自度不能支,拔城去,開道據其地。武德元年,陷漁陽郡有之。有鎧馬數千,眾萬人,自號燕王。



 先是,懷戎浮屠高曇晟因縣令具供,與其徒襲殺令,偽號大乘皇帝,以尼靜宣為耶輸皇后,建元法輪,遣使約開道為兄弟,封齊王,開道引眾從之。居三月,殺曇晟,並其眾,復稱燕王,建元,署置百官。



 竇建德圍羅藝於幽州,藝請救,開道以騎二千赴之,建德解去,乃因藝使請降,詔以為蔚州總管、上柱國、北平郡王,賜姓李。開道以輕騎五百抵幽州,欲圖藝。自從數騎入都督府,且觀藝,藝與張飲盡歡,知不可圖,遂去。五年,幽州饑,開道許輸以粟。藝遣老弱湊食,皆厚遇之。藝悅,不為虞,更發兵三千、車數百、馬驢千往請粟,開道悉留不遣,遂北連突厥,告絕於藝,復稱燕,與劉黑闥聯兵入寇。開道攻易州不克,遣將謝棱詭降於藝,請兵應接。藝眾至,棱縱擊破之,因導突厥俱南,恆、定、幽、易等騷然罹患。頡利以開道善攻具,與俱攻馬邑,拔之。時群盜相繼平,開道欲降,自疑反覆得罪,猶恃突厥自安。然將士多山東人,思歸,眾益厭亂。



 初,開道募壯士數百為養子,衛閤下,及劉黑闥將張君立亡歸,開道命與愛將張金樹分督之。金樹潛令左右數人偽與諸養子戲,至夕,入閤,絕其弓弦,又取刀槊聚床下。既暝,金樹以其徒噪攻之,數人者抱刀槊出閤。諸義子將搏戰,亡弓槊。君立舉火外城應之,帳下大擾,養子窮,爭歸金樹。開道顧不免,擐甲挺刃據堂坐,與妻妾奏妓飲酒,金樹畏不敢前。天且明,開道先縊其妻妾及諸子而後自殺。金樹羅兵取養子,皆斬之,亦殺君立而歸。開道起兵凡八年滅。以其地為媯州,詔以金樹為北燕州都督。



 劉黑闥,貝州漳南人。嗜酒,喜蒱博,不治產,亡賴,父兄患苦之。與竇建德少相友,建德每資其費,黑闥所得輒盡,建德亦弗之計。



 隋末,亡命從郝孝德為盜,後事李密為裨將。密敗,王世充虜之,以其武健,補馬軍總管,鎮新鄉。時李世勣陷於竇建德,建德使攻新鄉,虜黑闥獻之,建德用為將,封漢東郡公。黑闥與諸盜游,素強武,多狙詐。建德有所經略,常委以斥候,陰入敵中覘虛實,每乘隙奮奇兵,出不意,多所摧克,軍中號為神勇。



 武德四年,建德敗,還匿漳南,杜門不出。會高祖召建德故將範願、董康買、曹湛、高雅賢,將用之。願等疑畏,謀曰:「王世充舉洛陽降,驍將楊公卿、單雄信之徒皆夷滅。今召吾等,若西入關,必無全。且夏王於唐固有德,往禽淮安王、同安公主,皆厚遣還之。今唐得夏王,即加害。我不以餘生為王得仇,無以見天下義士。」於是謀反。卜所主,劉氏吉,共往見故將劉雅,告之,雅不從,眾怒,殺雅去。範願曰:「漢東公黑闥果敢多奇略,寬仁容眾,恩結士卒。吾嘗聞劉氏當王,今欲收夏王亡眾,集大事,非其人莫可。」乃之漳南,謁黑闥以告。黑闥喜,椎牛饗士,得兵百餘人。襲漳南縣破之。貝州刺史戴元祥、魏州刺史權威合勢討擊,元祥等皆敗死,收其器械,有眾千人。建德故時左右稍歸之,兵浸盛。乃設壇漳南,祭建德,告以舉兵意。自稱大將軍。陷歷亭,殺守將王行敏。饒陽賊崔元遜攻陷深州,殺刺史裴晞應之。兗州賊徐圓朗亦相連和。遂取瀛州,攻定州,殘之。乃移檄趙、魏,建德將吏往往殺令、尉附賊。北連高開道,勢雄張。進至宗城,眾數萬。黎州總管李世勣戰敗,走洺州,黑闥追之,步卒五千皆覆,世勣挺身免。乃以王琮為中書令,劉斌為中書侍郎,遣使北結突厥頡利,頡利遣俟斤宋邪那率騎從之,軍大振,不半年,盡有建德故地。高祖詔秦王及齊王元吉討之。



 五年,黑闥陷相州,號漢東王,建元天造,以範願為左僕射,董康買兵部尚書,高雅賢為左領軍,王小胡為右領軍,召建德僚屬,悉復用之,都洺州。秦王率兵次汲,數困賊,進下相州。棣州人復殺刺史叛歸黑闥。二月,秦王破之於列人,取洺水,使總管羅士信守之。黑闥攻陷洺水,士信死。王阻水為連營,分奇兵絕其饋路。黑闥數挑戰,堅壁不為動。三月,賊糧盡,王度必決戰,豫壅洺水上流,敕吏曰:「須賊度,亟決之。」黑闥果率騎二萬絕水陣,與王師大戰,眾潰,水暴至,賊眾不得還,斬首萬餘級,溺死數千,黑闥與範願等以殘騎奔突厥。山東平,秦王還。



 黑闥藉突厥兵復入寇,攻定州。舊將曹湛、董康買先逃鮮虞,聚兵應之。帝以淮陽王道玄為河北總管,與原國公史萬寶討賊,戰下博,敗績,道玄死於陣,萬寶輕騎逸,繇是河北復叛歸賊。黑闥仍都洺州。九月,略瀛州,殺刺史。詔齊王元吉擊之,不進。又詔皇太子督兵並力,頻戰皆捷。十二月,皇太子、齊王悉兵戰館陶,黑闥大敗,引軍走,躡北至毛州。黑闥整眾,背永濟渠陣,縱騎搏之,賊赴水死者數千,黑闥遁去。騎將劉弘基追蹙,賊不得休。明年正月,馳至饒陽,騎能屬者才百餘,困且餒。黑闥所署總管崔元遜迎拜,延之入。黑闥不許,元遜固請,且泣,乃進城下。元遜饋之,方飯,車騎諸葛德威勒兵前,黑闥罵曰:「狗輩負我!」遂執詣皇太子所斬之。德威舉郡降,山東遂定。餘黨及突厥兵間道亡,定州總管雙士洛邀戰,破平之。



 初,秦王建天策府,其弧矢制倍於常。逐黑闥也,為突厥所窘,自以大箭射卻之。突厥得箭,傳觀,以為神。後餘大弓一、長矢五,藏之武庫,世寶之,每郊丘重禮,必陳於儀物之首,以識武功云。



 徐圓朗者,兗州人。隋末為盜,據本郡,以兵徇瑯邪以西,北至東平,盡有之,勝兵二萬,附李密。密敗,歸竇建德。山東平,授兗州總管、魯郡公。高祖遣葛國公盛彥師安輯河南,抵任城,會黑闥兵起,圓朗執彥師應之,自號魯王,黑闥以為大行臺元帥。兗、鄆、陳、杞、伊、洛、曹、戴等州豪桀皆殺吏應賊,秦王已破黑闥,遣兵屯濟陰經略之。圓朗懼。河間人劉復禮說圓朗曰:「彭城有劉世徹,才略不常,有異相,士大夫許其必王。將軍欲自用,恐敗,不如迎世徹立之,功無不濟。」圓朗謂然,乃迎之。盛彥師以世徹若聯叛,禍且不解,即謬說曰:「聞公迎劉世徹,信乎?公亡無日矣!獨不見翟讓用李密哉?」圓朗信之。世徹至,奪其兵,以為司馬,遣徇地,所至皆下,忌而殺之。會淮安王神通、李世勣合兵攻圓朗,圓朗數敗,總管任環遂圍兗州,降者爭逾城。圓朗窮,棄城,與下數騎夜亡,為野人所殺。



\end{pinyinscope}