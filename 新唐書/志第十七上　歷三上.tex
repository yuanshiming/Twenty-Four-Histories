\article{志第十七上 歷三上}

\begin{pinyinscope}

 開元九年,《麟德歷》署日蝕比不效,詔僧一行作新歷,推大衍數立術以應之,較經史所書氣朔、日名、宿度可考者皆合。十五年列寧主義問題斯大林在1924—1939年間的重要論文、報,草成而一行卒,詔特進張說與歷官陳玄景等次為《歷術》七篇、《略例》一篇、《歷議》十篇,玄宗顧訪者則稱制旨。明年,說表上之,起十七年頒於有司。時善算瞿壇譔者,怨不得預改歷事,二十一年,與玄景奏:「《大衍》寫《九執歷》,其術未盡。」太子右司禦率南宮說亦非之。詔侍御史李麟、太史令桓執圭較靈臺候簿,《大衍》十得七、八,《麟德》才三、四,九執一、二焉。乃罪說等,而是否決。



 自《太初》至《麟德》,歷有二十三家,與天雖近而未密也。至一行,密矣,其倚數立法固無以易也。後世雖有改作者,皆依仿而已,故詳錄之。《略例》,所以明述作本旨也;《歷議》,所以考古今得失也。其說皆足以為將來折衷。略其大要,著於篇者十有二。



 其一《歷本議》曰:



 《易》:「天數五,地數五,五位相得而各有合,所以成變化而行鬼神也。」天數始於一,地數始於二,合二始以位剛柔。天數終於九,地數終於十,合二終以紀閏餘。天數中於五,地數中於六,合二中以通律歷。天有五音,所以司日也。地有六律,所以司辰也。參伍相周,究於六十,聖人以此見天地之心也。自五以降,為五行生數;自六以往,為五材成數。錯而乘之,以生數衍成位。一、六而退極,五、十而增極;一、六為爻位之統,五、十為大衍之母。成數乘生數,其算六百,為天中之積。生數乘成數,其算亦六百,為地中之積。合千有二百,以五十約之,則四象周六爻也;二十四約之,則太極包四十九用也。綜成數,約中積,皆十五。綜生數,約中積,皆四十。兼而為天地之數,以五位取之,復得二中之合矣。蓍數之變,九、六各一,乾坤之象也。七、八各三,六子之象也。故爻數通乎六十,策數行乎二百四十。是以大衍為天地之樞,如環之無端,蓋律歷之大紀也。



 夫數象微於三、四,而章於七、八。卦有三微,策有四象,故二微之合,在始中之際焉。蓍以七備,卦以八周,故二章之合,而在中終之際焉。中極居五六間,由闢闔之交,而在章微之際者,人神之極也。天地中積,千有二百,揲之以四,為爻率三百;以十位乘之,而二章之積三千;以五材乘八象,為二微之積四十。兼章微之積,則氣朔之分母也。以三極參之,倍六位除之,凡七百六十,是謂辰法,而齊於代軌。以十位乘之,倍大衍除之,凡三百四,是謂刻法,而齊於德運。半氣朔之母,千五百二十,得天地出符之數,因而三之,凡四千五百六十,當七精返初之會也。《易》始於三微而生一象,四象成而後八卦章。三變皆剛,太陽之象。三變皆柔,太陰之象。一剛二柔,少陽之象。一柔二剛,少陰之象。少陽之剛,有始、有壯、有究。少陰之柔,有始、有壯、有究。兼三才而兩之,神明動乎其中。故四十九象,而大業之用周矣。數之德圓,故紀之以三而變於七。象之德方,故紀之以四而變於八。



 人在天地中,以閱盈虛之變,則閏餘之初,而氣朔所虛也。以終合通大衍之母,虧其地十,凡九百四十為通數。終合除之,得中率四十九,餘十九分之九,終歲之弦,而鬥分復初之朔也。地於終極之際,虧十而從天,所以遠疑陽之戰也。夫十九分之九,盈九而虛十也。乾盈九,隱乎龍戰之中,故不見其首。坤虛十,以導潛龍之氣,故不見其成。周日之朔分,周歲之閏分,與一章之弦,一蔀之月,皆合於九百四十,蓋取諸中率也。



 一策之分十九,而章法生;一揲之分七十六,而蔀法生。一蔀之日二萬七千七百五十七,以通數約之,凡二十九日餘四百九十九,而日月相交於朔,此六爻之紀也。以卦當歲,以爻當月,以策當日,凡三十二歲而小終,二百八十五小終而與卦運大終,二百八十五,則參伍二終之合也。數象既合,而遁行之變在乎其間矣。



 所謂遁行者,以爻率乘朔餘,為十四萬九千七百,以四十九用、二十四象虛之,復以爻率約之,為四百九十八、微分七十五太半,則章微之中率也。二十四象,象有四十九蓍,凡千一百七十六。故虛遁之數七十三,半氣朔之母,以三極乘參伍,以兩儀乘二十四變,因而並之,得千六百一十三,為朔餘。四揲氣朔之母,以八氣九精遁其十七,得七百四十三,為氣餘。歲八萬九千七百七十三而氣朔會,是謂章率。歲二億七千二百九十萬九百二十而無小餘,合於夜半,是謂蔀率。歲百六十三億七千四百五十九萬五千二百而大餘與歲建俱終,是謂元率。此不易之道也。



 策以紀日,象以紀月。故乾坤之策三百六十,為日度之準。乾坤之用四十九象,為月弦之檢。日之一度,不盈全策;月之一弦,不盈全用。故策餘萬五千九百四十三,則十有二中所盈也。用差萬七千一百二十四,則十有二朔所虛也。綜盈虛之數,五歲而再閏。中節相距,皆當三五;弦望相距,皆當二七。升絳之應,發斂之候,皆紀之以策而從日者也。表里之行,朓朒之變,皆紀之以用而從月者也。



 積算曰演紀,日法曰通法,月氣曰中朔,朔實曰揲法,歲分曰策實,周天曰乾實,餘分曰虛分。氣策曰三元,一元之策,則天一遁行也。月策曰四象,一象之策,則朔、弦、望相距也。五行用事,曰發斂。候策曰天中,卦策曰地中,半卦曰貞悔。旬周曰爻數,小分母曰象統。日行曰躔,其差曰盈縮,積盈縮曰先後。古者平朔,月朝見曰朒,夕見曰朓。今以日之所盈縮、月之所遲疾損益之,或進退其日,以為定朔。舒亟之度,乃數使然,躔離相錯,偕以損益,故同謂之朓朒。月行曰離,遲疾曰轉度,母曰轉法。遲疾有衰,其變者勢也。月逶迤馴屈,行不中道,進退遲速,不率其常。過中則為速,不及中則為遲。積遲謂之屈,積速謂之伸。陽,執中以出令,故曰先後;陰,含章以聽命,故曰屈伸。日不及中則損之,過則益之。月不及中則益之,過則損之,尊卑之用睽,而及中之志同。觀晷景之進退,知軌道之升降。軌與晷名舛而義合,其差則水漏之所從也。總名曰軌漏。中晷長短謂之陟降。景長則夜短,景短則夜長。積其陟降,謂之消息。游交曰交會,交而周曰交終。交終不及朔,謂之朔差。交中不及望,謂之望差。日道表曰陽歷,其里曰陰歷。五星見伏周,謂之終率。以分從日謂之終日,其差為進退。



 其二《中氣議》曰:



 歷氣始於冬至,稽其實,蓋取諸晷景。《春秋傳》僖公五年正月辛亥朔,日南至。以《周歷》推之,入壬子蔀第四章,以辛亥一分合朔冬至,《殷歷》則壬子蔀首也。昭公二十年二月己丑朔,日南至。魯史失閏,至不在正。左氏記之,以懲司歷之罪。《周歷》得己丑二分,《殷歷》得庚寅一分。《殷歷》南至常在十月晦,則中氣後天也。《周歷》蝕朔差《經》或二日,則合朔先天也。《傳》所據者《周歷》也,《緯》所據者《殷歷》也。氣合於《傳》,朔合於《緯》,斯得之矣。《戊寅歷》月氣專合於《緯》,《麟德歷》專合於《傳》,偏取之,故兩失之。又《命歷序》以為孔子修《春秋》用《殷歷》,使其數可傳於後。考其蝕朔不與《殷歷》合,及開元十二年,朔差五日矣,氣差八日矣。上不合於《經》,下不足以傳於後代,蓋哀、平間治甲寅元歷者托之,非古也。又漢太史令張壽王說黃帝《調歷》以非《太初》。有司劾:「官有黃帝《調歷》不與壽王同,壽王所治乃《殷歷》也。」漢自中興以來,圖讖漏洩,而《考靈曜》、《命歷序》皆有甲寅元,其所起在《四分歷》庚申元後百一十四歲。延光初中謁者亶誦、靈帝時五官郎中馮光等,皆請用之,卒不施行。《緯》所載壬子冬至,則其遺術也。《魯歷》南至又先《周歷》四分日之三,而朔後九百四十分日之五十一,故僖公五年辛亥為十二月晦,壬子為正月朔。又推日蝕密於《殷歷》,其以閏餘一為章首,亦取合於當時也。



 開元十二年十一月,陽城測景,以癸未極長,較其前後所差,則夜半前尚有餘分。新歷大餘十九,加時九十九刻,而《皇極》、《戊寅》、《麟德歷》皆得甲申,以《玄始歷》氣分二千四百四十三為率,推而上之,則失《春秋》辛亥,是減分太多也。以《皇極歷》氣分二千四百四十五為率,推而上之,雖合《春秋》,而失元嘉十九年乙巳冬至及開皇五年甲戌冬至、七年癸未夏至;若用《麟德歷》率二千四百四十七,又失《春秋》己丑,是減分太少也。故新歷以二千四百四十四為率,而舊所失者皆中矣。



 漢會稽東部尉劉洪以《四分》疏闊,由斗分多,更以五百八十九為紀法,百四十五為斗分,減餘太甚,是以不及四十年而加時漸覺先天。韓翊、楊偉、劉智等皆稍損益,更造新術,而皆依讖緯「三百歲改憲」之文,考《經》之合朔多中,較《傳》之南至則否。《玄始歷》以為十九年七閏,皆有餘分,是以中氣漸差。據渾天,二分為東西之中,而晷景不等;二至為南北之極,而進退不齊。此古人所未達也。更因劉洪紀法,增十一年以為章歲,而減閏餘十九分之一。春秋後五十四年,歲在甲寅,直應鐘章首,與《景初歷》閏餘皆盡。雖減章閏,然中氣加時尚差,故未合於《春秋》。其鬥分幾得中矣。



 後代歷家,皆因循《玄始》,而損益或過差。大抵古歷未減鬥分,其率自二千五百以上。《乾象》至於《元嘉歷》,未減閏餘,其率自二千四百六十以上。《玄始》、《大明》至《麟德歷》皆減分破章,其率自二千四百二十九以上。較前代史官注記,惟元嘉十三年十一月甲戌景長,《皇極》、《麟德》、《開元歷》皆得癸酉,蓋日度變常爾。祖沖之既失甲戌冬至,以為加時太早,增小餘以附會之。而十二年戊辰景辰,得己巳;十七年甲午景長,得乙未;十八年己亥景長,得庚子。合一失三,其失愈多。劉孝孫、張胄玄因之,小餘益強,又以十六年己丑景長為庚寅矣。治歷者糾合眾同,以稽其所異,茍獨異焉,則失行可知。今曲就其一,而少者失三,多者失五,是舍常數而從失行也。周建德六年,以壬辰景長,而《麟德》、《開元歷》皆得癸巳。開皇七年,以癸未景短,而《麟德》、《開元歷》皆得壬午。先後相戾,不可葉也,皆日行盈縮使然。



 凡歷術在於常數,而不在於變行。既葉中行之率,則可以兩齊先後之變矣。《麟德》已前,實錄所記,乃依時歷書之,非候景所得。又比年候景,長短不均,由加時有早晏,行度有盈縮也。



 自春秋以來,至開元十二年,冬、夏至凡三十一事,《戊寅歷》得十六,《麟德歷》得二十三,《開元歷》得二十四。



 其三《合朔議》曰:



 日月合度謂之朔。無所取之,取之蝕也。《春秋》日蝕有甲乙者三十四。《殷歷》、《魯歷》先一日者十三,後一日者三;《周歷》先一日者二十二,先二日者九。其偽可知矣。



 莊公三十年九月庚午朔,襄公二十一年九月庚戌朔,定公五年三月辛亥朔,當以盈縮、遲速為定朔。《殷歷》雖合,適然耳,非正也。僖公五年正月辛亥朔,十二月丙子朔,十四年三月己丑朔;文公元年五月辛酉朔,十一年三月甲申晦;襄公十九年五月壬辰晦;昭公元年十二月甲辰朔,二十年二月己丑朔,二十三年正月壬寅朔、七月戊辰晦:皆與《周歷》合。其所記多周、齊、晉事,蓋周王所頒,齊、晉用之。僖公十五年九月己卯晦,十六年正月戊申朔;成公十六年六月甲午晦;襄公十八年十月丙寅晦,十一月丁卯朔,二十六年三月甲寅朔,二十七年六月丁未朔:與《殷歷》、《魯歷》合。此非合蝕,故仲尼因循時史,而所記多宋、魯事,與齊、晉不同可知矣。



 昭公十二年十月壬申朔,原輿人逐原伯絞,與《魯歷》、《周歷》皆差一日,此丘明即其所聞書之也。僖公二十二年十一月己巳朔,宋、楚戰於泓。《周》、《殷》、《魯歷》皆先一日,楚人所赴也。昭公二十年六月丁巳晦,衛侯與北宮喜盟;七月戊午朔,遂盟國人。三歷皆先二日,衛人所赴也。此則列國之歷不可以一術齊矣。而《長歷》日子不在其月,則改易閏餘,欲以求合。故閏月相距,近則十餘月,遠或七十餘月,此杜預所甚繆也。夫合朔先天,則《經》書日蝕以糾之。中氣後天,則《傳》書南至以明之。其在晦、二日,則原乎定朔以得之。列國之歷或殊,則稽於六家之術以知之。此四者,皆治歷之大端,而預所未曉故也。



 新歷本《春秋》日蝕、古史交會加時及史官候簿所詳,稽其進退之中,以立常率。然後以日躔、月離、先後、屈伸之變,偕損益之。故經朔雖得其中,而躔離或失其正;若躔離各得其度,而經朔或失其中,則參求累代,必有差矣。三者迭相為經,若權衡相持,使千有五百年間朔必在晝,望必在夜,其加時又合,則三術之交,自然各當其正,此最微者也。若乾度盈虛,與時消息,告譴於經數之表,變常於潛遁之中,則聖人且猶不質,非籌歷之所能及矣。



 昔人考天事,多不知定朔。假蝕在二日,而常朔之晨,月見東方;食在晦日,則常朔之夕,月見西方。理數然也。而或以為朓朒變行,或以為歷術疏闊,遇常朔朝見則增朔餘,夕見則減朔餘,此紀歷所以屢遷也。漢編、李梵等又以晦猶月見,欲令蔀首先大。賈逵曰:「《春秋》書朔、晦者,朔必有朔,晦必有晦,晦、朔必在其月前也。先大,則一月再朔,後月無朔,是朔不可必也。、梵等欲諧偶十六日、月朓昏、晦當滅而已。又晦與合朔同時,不得異日。」考逵等所言,蓋知之矣。晦朔之交,始終相際,則光盡明生之限,度數宜均。故合於子正,則晦日之朝,猶朔日之夕也,是以月皆不見;若合於午正,則晦日之晨,猶二日之昏也,是以月或皆見。若陰陽遲速,軌漏加時不同,舉其中數率,去日十三度以上而月見,乃其常也。且晦日之光未盡也,如二日之明已生也。一以為是,一以為非。又常朔進退,則定朔之晦、二也。或以為變,或以為常。是未通於四三交質之論也。



 綜近代諸歷,以百萬為率齊之,其所差,少或一分,多至十數失一分。考《春秋》才差一刻,而百數年間不足成朓朒之異。施行未幾,旋復疏闊,由未知躔離經朔相求耳。李業興、甄鸞等欲求天驗,輒加減月分,遷革不已,朓朒相戾,又未知昏明之限與定朔故也。楊偉採《乾象》為遲疾陰陽歷,雖知加時後天,蝕不在朔,而未能有以更之也。



 何承天欲以盈縮定朔望小餘。錢樂之以為:「推交會時刻雖審,而月頻三大二小。日蝕不唯在朔,亦有在晦、二者。」皮延宗又以為:「紀首合朔,大小餘當盡,若每月定之,則紀首位盈,當退一日,便應以故歲之晦為新紀之首。立法之制,如為不便。」承天乃止。虞廣刂曰:「所謂朔在會合,茍躔次既同,何患於頻大也?日月相離,何患於頻小也?」《春秋》日蝕不書朔者八,《公羊》曰:「二日也。」《穀梁》曰:「晦也。」《左氏》曰:「官失之也。」。劉孝孫推俱得朔日,以丘明為是,乃與劉焯皆議定朔,為有司所抑不得行。傅仁均始為定朔,而曰「晦不東見,朔不西朓」,以為昏晦當滅,亦、梵之論。淳風因循《皇極》,《皇極》密於《麟德》,以朔餘乘三千四十,乃一萬除之,就全數得千六百一十三。又以九百四十乘之,以三千四十而一,得四百九十八秒七十五太強,是為《四分》餘率。



 劉洪以古歷斗分太強,久當後天,乃先正鬥分,而後求朔法,故朔餘之母煩矣。韓翊以《乾象》朔分太弱,久當先天,乃先考朔分,而後覆求度法,故度餘之母煩矣。何承天反覆相求,使氣朔之母合簡易之率,而星數不得同元矣。李業興、宋景業、甄鸞、張賓欲使六甲之首眾術同元,而氣朔餘分,其細甚矣。《麟德歷》有總法,《開元歷》有通法,故積歲如月分之數,而後閏餘偕盡。



 考漢元光已來史官注記,日蝕有加時者凡三十七事,《麟德歷》得五,《開元歷》得二十二。



 其四《沒滅略例》曰:



 古者以中氣所盈之日為沒,沒分偕盡者為滅;《開元歷》以中分所盈為沒,朔分所虛為滅。綜終歲沒分,謂之策餘;終歲滅分,謂之用差。皆歸於揲易再手力而後掛也。



 其五《卦候議》曰:



 七十二候,原於周公《時訓》。《月令》雖頗有增益,然先後之次則同。自後魏始載於歷,乃依《易軌》所傳,不合經義。今改從古。



 其六《卦議》曰:



 十二月卦出於《孟氏章句》,其說《易》本於氣,而後以人事明之。京氏又以卦爻配期之日,坎、離、震、兌,其用事自分、至之首,皆得八十分日之七十三。頤、晉、井、大畜,皆五日十四分,餘皆六日七分,止於占災眚與吉兇善敗之事。至於觀陰陽之變,則錯亂而不明。自《乾象歷》以降,皆因京氏。惟《天保歷》依《易通統軌圖》。自八十有二節、五卦、初爻,相次用事,及上爻而與中氣偕終,非京氏本旨及《七略》所傳。按郎顗所傳,卦皆六日七分,不以初爻相次用事,齊歷謬矣。又京氏減七十三分,為四正之候,其說不經,欲附會《緯》文《七日來復》而已。



 夫陽精道消,靜而無跡,不過極其正數,至七而通矣。七者,陽之正也,安在益其小餘,令七日而後雷動地中乎?當據孟氏,自冬至初,中孚用事,一月之策,九六、七八,是為三十。而卦以地六,候以天五,五六相乘,消息一變,十有二變而歲復初。坎、震、離、兌,二十四氣,次主一爻,其初則二至、二分也。坎以陰包陽,故自北正,微陽動於下,升而未達,極於二月,凝涸之氣消,坎運終焉。春分出於震,始據萬物之元,為主於內,則群陰化而從之,極於南正,而豐大之變窮,震功究焉。離以陽包陰,故自南正,微陰生於地下,積而未章,至於八月,文明之質衰,離運終焉。仲秋陰形於兌,始循萬物之末,為主於內,群陽降而承之,極於北正,而天澤之施窮,兌功究焉。故陽七之靜始於坎,陽九之動始於震,陰八之靜始於離,陰六之動始於兌。故四象之變,皆兼六爻,而中節之應備矣。《易》爻當日,十有二中,直全卦之初;十有二節,直全卦之中。齊歷又以節在貞,氣在悔,非是。



 其七《日度議》曰:



 古歷,日有常度,天周為歲終,故系星度於節氣。其說似是而非,故久而益差。虞喜覺之,使天為天,歲為歲,乃立差以追其變,使五十年退一度。何承天以為太過,乃倍其年,而反不及。《皇極》取二家中數為七十五年,蓋近之矣。考古史及日官候簿,以通法之三十九分太為一歲之差。自帝堯演紀之端,在虛一度。及今開元甲子,卻三十六度,而乾策復初矣。日在虛一,則鳥、火、昴、虛皆以仲月昏中,合於《堯典》。



 劉炫依《大明歷》四十五年差一度,則冬至在虛、危,而夏至火已過中矣。梁武帝據虞廣刂歷,百八十六年差一度,則唐、虞之際,日在斗、牛間,而冬至昴尚未中。以為皆承閏後節前,月卻使然。而此經終始一歲之事,不容頓有四閏,故淳風因為之說曰:「若冬至昴中,則夏至秋分星火、星虛,皆在未正之西。若以夏至火中,秋分虛中,則冬至昴在巳正之東。互有盈縮,不足以為歲差證。」是又不然。今以四象分天,北正玄枵中,虛九度;東正大火中,房二度;南正鶉火中,七星七度;西正大梁中,昴七度。總晝夜刻以約周天,命距中星,則春分南正中天,秋分北正中天。冬至之昏,西正在午東十八度;夏至之昏,東正在午西十八度:軌漏使然也。冬至,日在虛一度,則春分昏張一度中;秋分虛九度中;冬至胃二度中,昴距星直午正之東十二度;夏至尾十一度中,心後星直午正之西十二度。四序進退,不逾午正間。而淳風以為不葉,非也。又王孝通云:「如歲差自昴至壁,則堯前七千餘載,冬至,日應在東井。井極北,故暑;斗極南,故寒。寒暑易位,必不然矣。」所謂歲差者,日與黃道俱差也。假冬至日躔大火之中,則春分黃道交於虛九,而南至之軌更出房、心外,距赤道亦二十四度。設在東井,差亦如之。若日在東井,猶去極最近,表景最短,則是分、至常居其所。黃道不遷,日行不退,又安得謂之歲差乎?孝通及淳風以為冬至日在斗十三度,昏東壁中,昴在巽維之左,向明之位,非無星也。水星昏正可以為仲之候,何必援昴於始覿之際,以惑民之視聽哉!



 夏后氏四百三十二年,日卻差五度。太康十二年戊子歲冬至,應在女十一度。



 《書》曰:「乃季秋月朔,辰弗集於房。」劉炫曰:「房,所舍之次也。集,會也。會,合也。不合則日蝕可知。或以房為房星,知不然者,且日之所在正可推而知之。君子慎疑,寧當以日在之宿為文?近代善歷者,推仲康時九月合朔,已在房星北矣。」按,古文「集」與「輯」義同。日月嘉會,而陰陽輯睦,則陽不疚乎位,以常其明,陰亦含章示沖,以隱其形。若變而相傷,則不輯矣。房者辰之所次,星者所次之名,其揆一也。又《春秋傳》「辰在斗柄」、「天策焞焞」、「降婁之初」、「辰尾之末」,君子言之,不以為繆,何獨慎疑於房星哉?新歷仲康五年癸巳歲九月庚戌朔,日蝕在房二度。炫以《五子之歌》,仲康當是其一,肇位四海,復脩大禹之典,其五年,羲、和失職,則王命徂征。虞廣刂以為仲康元年,非也。



 《國語》單子曰:「辰角見而雨畢,天根見而水涸,本見而草木節解,駟見而隕霜,火見而清風戒寒。」韋昭以為夏后氏之令,周人所因。推夏后氏之初,秋分後五日,日在氏十三度,龍角盡見,時雨可以畢矣。又先寒露三日,天根朝覿,《時訓》「爰始收潦」,而《月令》亦云「水涸」。後寒露十日,日在尾八度而本見,又五日而駟見。故隕霜則蟄蟲墐戶。鄭康成據當時所見,謂天根朝見,在季秋之末,以《月令》為謬。韋昭以仲秋水始涸,天根見乃竭。皆非是。霜降六日,日在尾末,火星初見,營室昏中,於是始脩城郭、宮室。故《時儆》曰:「營室之中,土功其始。火之初見,期於司理。」《麟德歷》霜降後五日,火伏。小雪後十日,晨見。至大雪而後定星中,日旦南至,冰壯地坼。又非土功之始也。



 《夏歷》十二次,立春,日在東壁三度,於《太初》星距壁一度太也。



 《顓頊歷》上元甲寅歲正月甲寅晨初合朔立春,七曜皆直艮維之首。蓋重黎受職於顓頊,九黎亂德,二官咸廢,帝堯復其子孫,命掌天地四時,以及虞、夏。故本其所由生,命曰《顓頊》,其實《夏歷》也。湯作《殷歷》,更以十一月甲子合朔冬至為上元。周人因之,距羲、和千祀,昏明中星率差半次。夏時直月節者,皆當十有二中,故因循夏令。其後呂不韋得之,以為秦法,更考中星,斷取近距,以乙卯歲正月己巳合朔立春為上元。《洪範傳》曰:「歷記始於顓頊上元太始閼蒙攝提格之歲,畢陬之月,朔日己巳立春,七曜俱在營室五度。」是也。秦《顓頊歷》元起乙卯,漢《太初歷》元起丁丑,推而上之,皆不值甲寅,猶以日月五緯復得上元本星度,故命曰閼蒙攝提格之歲,而實非甲寅。



 《夏歷》章蔀紀首,皆在立春,故其課中星、揆斗建與閏餘之所盈縮,皆以十有二節為損益之中。而《殷》、《周》、《漢歷》,章蔀紀首皆直冬至,故其名察發斂,亦以中氣為主。此其異也。



 《夏小正》雖頗疏簡失傳,乃羲、和遺跡。何承天循大戴之說,復用夏時,更以正月甲子夜半合朔雨水為上元,進乖《夏歷》,退非周正,故近代推《月令》、《小正》者,皆不與古合。《開元歷》推夏時立春,日在營室之末,昏東井二度中。古歷以參右肩為距,方當南正。故《小正》曰:「正月初昏,斗杓懸在下。」魁枕參首,所以著參中也。季春,在昴十一度半,去參距星十八度,故曰:「三月,參則伏。」立夏,日在井四度,昏角中。南門右星入角距西五度,其左星入角距東六度,故曰:「四月初昏,南門正。昴則見。」五月節,日在輿鬼一度半。參去日道最遠,以渾儀度之,參體始見,其肩股猶在濁中。房星正中。故曰:「五月,參則見。初昏,大火中。」「八月,參中則曙」,失傳也。辰伏則參見,非中也。「十月初昏,南門見」,亦失傳也。定星方中,則南門伏,非昏見也。



 商六百二十八年,日卻差八度。太甲二年壬午歲冬至,應在女六度。



 《國語》曰:「武王伐商,歲在鶉火,月在天駟,日在析木之津,辰在斗柄,星在天黿。」舊說歲在己卯,推其朏魄,乃文王崩,武王成君之歲也。其明年,武王即位,新歷孟春定朔丙辰,於商為二月,故《周書》曰:「維王元祀二月丙辰朔,武王訪於周公。」《竹書》:「十一年庚寅,周始伐商。」而《管子》及《家語》以為十二年,蓋通成君之歲也。先儒以文王受命九年而崩;至十年,武王觀兵盟津;十三年,復伐商。推元祀二月丙辰朔,距伐商日月,不為相距四年。所說非是。武王十年,夏正十月戊子,周師始起。於歲差日在箕十度,則析木津也。晨初,月在房四度。於《易》,雷乘乾曰大壯,房、心象焉。心為乾精,而房,升陽之駟也。房與歲星實相經緯,以屬靈威仰之神,後稷感之以生。故《國語》曰:「月之所在,辰馬農祥,我祖後稷之所經緯也。」又三日得周正月庚寅朔,日月會南斗一度。故曰「辰在斗柄」。壬辰,辰星夕見,在南斗二十度。其明日,武王自宗周次於師所。凡月朔而未見曰「死魄」,夕而成光則謂之「朏」。朏或以二日,或以三日,故《武成》曰:「維一月壬辰,旁死魄。翌日癸巳,王朝步自周,於征伐商。」是時辰星與周師俱進,由建星之末,歷牽牛、須女,涉顓頊之虛。戊午,師度盟津,而辰星伏於天黿。辰星,汁光紀之精,所以告顓頊而終水行之運,且木帝之所繇生也。故《國語》曰:「星與日辰之位皆在北維,顓頊之所建也,帝嚳受之。我周氏出自天黿;及析木,有建星、牽牛焉,則我皇妣太姜之侄、伯陵之後逢公之所憑神也。」是歲,歲星始及鶉火。其明年,周始革命。歲又退行,旅於鶉首,而後進及鳥帑,所以反復其道,經綸周室。鶉火直軒轅之虛,以爰稼穡,稷星系焉,而成周之大萃也。鶉首當山河之右,太王以興,後稷封焉,而宗周之所宅也。歲星與房實相經緯,而相距七舍;木與水代終,而相及七月。故《國語》曰;「歲之所在,則我有周之分也。自鶉及駟七列,南北之揆七月。其二月戊子朔,哉生明,王自克商還,至於酆,於周為四月。新歷推定望甲辰,而乙巳旁之。故《武成》曰:「維四月,既旁生魄,粵六月庚戌,武王燎於周廟。」《麟德歷》,周師始起,歲在降婁,月宿天根,日躔心而合辰在尾,水星伏於星紀,不及天黿。又《周書》,革命六年而武王崩。《管子》、《家語》以為七年,蓋通克商之歲也。



 周公攝政七年二月甲戌朔,己丑望,後六日乙未。三月定朔甲辰,三日丙午。故《召誥》曰:「惟二月既望,越六日乙未,王朝步自周,至於酆」,「三月,惟丙午朏,越三日戊申,太保朝至於洛。」其明年,成王正位。三十年四月乙酉朔甲子,哉生魄。故《書》曰:「惟四月,才生魄。」甲子,作《顧命》。康王十二年,歲在乙酉,六月戊辰朔,三日庚午。故《畢命》曰:「惟十有二年,六月庚午朏。越三日壬申,王以成周之眾命畢公。」自伐紂及此,五十六年,朏魄日名,上下無不合。而《三統歷》以己卯為克商之歲,非也。夫有效於古者,宜合於今。《三統歷》自太初至開元,朔後天三日。推而上之,以至周初,先天,失之蓋益甚焉。是以知合於歆者,必非克商之歲。



 自宗周訖春秋之季,日卻差八度。康王十一年甲申歲冬至,應在牽牛六度。



 《周歷》十二次,星紀初,南斗十四度,於《太初》星距鬥十七度少也。



 古歷分率簡易,歲久輒差。達歷數者隨時遷革,以合其變。故三代之興,皆揆測天行,考正星次,為一代之制。正朔既革,而服色從之。及繼體守文,疇人代嗣,則謹循先王舊制焉。



 《國語》曰:「農祥晨正,日月底於天廟,土乃脈發。先時九日,太史告稷曰,自今至於初吉,陽氣俱蒸,土膏其動。弗震不渝,脈其滿眚,穀乃不殖。」周初,先立春九日,日至營室。古歷距中九十一度,是日晨初,大火正中,故曰「農祥晨正,日月底於天廟」也。於《易》象,升氣究而臨受之,自冬至後七日,乾精始復。及大寒,地統之中,陽洽於萬物根柢,而與萌芽俱升,木在地中之象,升氣已達,則當推而大之,故受之以臨。於消息,龍德在田,得地道之和澤,而動於地中,升陽憤盈,土氣震發,故曰:「自今至於初吉,陽氣俱蒸,土膏其動。」又先立春三日,而小過用事,陽好節止於內,動作於外,矯而過正,然後返求中焉。是以及於艮維,則山澤通氣,陽精闢戶,甲坼之萌見,而莩穀之際離,故曰:「不震不渝,脈其滿眚,穀乃不殖。」君子之道,必擬之而後言,豈人意度而已哉!韋昭以為日及天廟,在立春之初,非也。於《麟德歷》則又後立春十五日矣。



 《春秋》「桓公五年,秋,大雩」。《傳》曰:「書不時也。凡祀,啟蟄而郊,龍見而雩。」《周歷》立夏日在觜觿二度。於軌漏,昏角一度中,蒼龍畢見。然則當在建巳之初,周禮也。至春秋時,日已潛退五度,節前月卻,猶在建辰。《月令》以為五月者,《呂氏》以《顓頊歷》芒種亢中,則龍以立夏昏見,不知有歲差,故雩祭失時。然則唐禮當以建巳之初,農祥始見而雩。若據《麟德歷》,以小滿後十三日,則龍角過中,為不時矣。《傳》曰:「凡土功,龍見而畢務,戒事。火見而致用,水昏正而栽,日至而畢。」十六年冬,城向。十有一月,衛侯朔出奔齊。「冬,城向,書時也。」以歲差推之,周初霜降,日在心五度,角、亢晨見。立冬,火見營室中。後七日,水星昏正,可以興板幹。故祖沖之以為定之方中,直營室八度。是歲九月六日霜降,二十一日立冬。十月之前,水星昏正,故《傳》以為得時。杜氏據晉歷,小雪後定星乃中,季秋城向,似為太早,因曰:功役之事,皆總指天象,不與言歷數同。引《詩》云「定之方中」,乃未正中之辭,非是。《麟德歷》,立冬後二十五日火見,至大雪後營室乃中。而《春秋》九月書時,不已早乎。大雪,周之孟春,陽氣靜復,以繕城隍,治宮室,是謂發天地之房,方於立春斷獄,所失多矣。然則唐制宜以玄枵中天興土功。



 僖公五年,晉侯伐虢。卜偃曰:「克之。童謠云:丙之辰,龍尾伏辰,袀服振振,取虢之旂,鶉之賁賁,天策焞焞,火中成軍。』其九月十月之交乎!丙子旦,日在尾,月在策,鶉火中,必是時。」策,入尾十二度。新歷是歲十月丙子定朔,日月合尾十四度於黃道。古歷日在尾,而月在策,故曰「龍尾伏辰」,於古距張中而曙,直鶉火之末,始將西降,故曰「賁賁」。



 昭公七年四月甲辰朔,日蝕。士文伯曰:「去衛地,如魯地。於是有災,魯實受之。」新歷是歲二月甲辰朔入常,雨水後七日,在奎十度。周度為降婁之始,則魯、衛之交也。自周初至是,已退七度,故入雨水。七日方及降婁,雖日度潛移,而周禮未改,其配神主祭之宿,宜書於建國之初。淳風駁《戊寅歷》曰:「《漢志》降婁初在奎五度,今歷日蝕在降婁之中,依無歲差法,食於兩次之交。」是又不然。議者曉十有二次之所由生,然後可以明其得失。且劉歆等所定辰次,非能有以睹陰陽之賾,而得於鬼神,各據當時中節星度耳。歆以《太初歷》冬至日在牽牛前五度,故降婁直東壁八度。李業興《正光歷》,冬至在牽牛前十二度,故降婁退至東壁三度。及祖沖之後,以為日度漸差,則當據列宿四正之中以定辰次,不復系於中節。淳風以冬至常在斗十三度,則當以東壁二度為降婁之初,安得守漢歷以駁仁均耶?又《三統歷》昭公二十年,己丑,日南至,與《麟德》及《開元歷》同。然則入雨水後七日,亦入降婁七度,非魯、衛之交也。三十一年十二月辛亥朔,日蝕。史墨曰:「日月在辰尾,庚午之日,日始有謫。」《開元歷》是歲十月辛亥朔,入常立冬。五日,日在尾十三度,於古距辰尾之初。《麟德歷》日在心三度於黃道,退直於房矣。



 哀公十二年冬十有二月,螽。《開元歷》推置閏當在十一年春,至十二年冬,失閏已久。是歲九月己亥朔,先寒露三日,於定氣,日在亢五度,去心近一次。火星明大,尚未當伏。至霜降五日,始潛日下。乃《月令》「蟄蟲咸俯」,則火辰未伏,當在霜降前。雖節氣極晚,不得十月昏見。故仲尼曰:「丘聞之,火伏而後蟄者畢。今火猶西流,司歷過也。」方夏后氏之初,八月辰伏,九月內火,及霜降之後,火已朝覿東方,距春秋之季千五百餘年,乃云「火伏而後蟄者畢。」向使冬至常居其所,則仲尼不得以西流未伏,明是九月之初也。自春秋至今又千五百歲,《麟德歷》以霜降後五日,日在氐八度,房、心初伏,定增二日,以月蝕沖校之,猶差三度。閏餘稍多,則建亥之始,火猶見西方。向使宿度不移,則仲尼不得以西流未伏,明非十月之候也。自羲、和已來,火辰見伏,三睹厥變。然則丘明之記,欲令後之作者參求微象,以探仲尼之旨。是歲失閏浸久,季秋中氣後天三日,比及明年仲冬,又得一閏。寤仲尼之言,補正時歷,而十二月猶可以螽。至哀公十四年五月庚申朔,日蝕。以《開元歷》考之,則日蝕前又增一閏,《魯歷》正矣。《長歷》自哀公十年六月,迄十四年二月,才置一閏,非是。



 戰國及秦,日卻退三度。始皇十七年辛未歲冬至,應在斗二十二度。秦歷上元正月己巳朔,晨初立春,日、月、五星俱起營室五度。蔀首日名皆直四孟。假朔退十五日,則閏在正月前。朔進十五日,則閏在正月後。是以十有二節,皆在盈縮之中,而晨昏宿度隨之。以《顓頊歷》依《月令》自十有二節推之,與不韋所記合。而潁子嚴之倫謂《月令》晨昏距宿,當在中氣,致雩祭太晚,自乖左氏之文,而杜預又據《春秋》,以《月令》為否。皆非是。梁《大同歷》夏后氏之初,冬至日在牽牛初,以為《明堂》、《月令》乃夏時之記,據中氣推之不合,更以中節之間為正,乃稍相符。不知進在節初,自然契合。自秦初及今,又且千歲,節初之宿,皆當中氣。淳風因為說曰:「今孟春中氣,日在營室,昏明中星,與《月令》不殊。」按秦歷立春,日在營室五度。《麟德歷》以啟蟄之日乃至營室,其昏明中宿十有二建,以為不差,妄矣。



 古歷,冬至昏明中星去日九十二度,春分、秋分百度,夏至百一十八度,率一氣差三度,九日差一刻。



 秦歷十二次,立春在營室五度,於《太初》星距危十六度少也。昏,畢八度中,《月令》參中,謂肩股也。晨,心八度中,《月令》尾中,於《太初》星距尾也。仲春昏,東井十四度中,《月令》弧中,弧星入東井十八度。晨,南斗二度中,《月令》建星中,於《太初》星距西建也。《甄耀度》及《魯歷》,南方有狼、弧,無東井、鬼,北方有建星,無南斗,井、斗度長,弧、建度短,故以正昏明云。



 古歷星度及漢落下閎等所測,其星距遠近不同,然二十八宿之體不異。古以牽牛上星為距,《太初》改用中星,入古歷牽牛太半度,於氣法當三十二分日之二十一。故《洪範傳》冬至日在牽牛一度,減《太初》星距二十一分,直南斗二十六度十九分也。《顓頊歷》立春起營室五度,冬至在牽年一度少。《洪範傳》冬至所起無餘分,故立春在營室四度太。祖沖之自營室五度,以《太初》星距命之,因云秦歷冬至,日在牽牛六度。虞廣刂等襲沖之之誤,為之說云:「夏時冬至,日在斗末,以歲差考之,牽牛六度乃《顓頊》之代。漢時雖覺其差,頓移五度,故冬至還在牛初。」按《洪範》古今星距,僅差四分之三,皆起牽牛一度。廣刂等所說,亦非是。魯宣公十五年,丁卯歲,《顓頊歷》第十三蔀首與《麟德歷》俱以丁巳平旦立春。至始皇三十三年丁亥,凡三百八十歲,得《顓頊歷》壬申蔀首。是歲,秦歷以壬申寅初立春,而《開元歷》與《麟德歷》俱以庚午平旦,差二日,日當在南斗二十二度。古歷後天二日,又增二度。然則秦歷冬至,定在午前二度。氣後天二日,日不及天二度,微而難覺,故《呂氏》循用之。



 及漢興,張蒼等亦以《顓頊歷》比五家疏闊中最近密。今考月蝕沖,則開元冬至,上及牛初正差一次。淳風以為古術疏舛,雖弦望、昏明差天十五度而猶不知。又引《呂氏春秋》,黃帝以仲春乙卯日在奎,始奏十二鐘,命之曰《咸池》。至今三千餘年,而春分亦在奎,反謂秦歷與今不異。按不韋所記,以其《月令》孟春在奎,謂黃帝之時亦在奎,猶淳風歷冬至斗十三度,因謂黃帝時亦在建星耳。經籍所載,合於歲差者,淳風皆不取,而專取於《呂氏春秋》。若謂十二紀可以為正,則立春在營室五度,固當不易,安得頓移使當啟蟄之節?此又其所不思也。



 漢四百二十六年,日卻差五度。景帝中元三年甲午歲冬至,應在斗二十一度。



 太初元年,《三統歷》及《周歷》皆以十一月夜半合朔冬至,日月俱起牽牛一度。古歷與近代密率相較,二百年氣差一日,三百年朔差一日。推而上之,久益先天;引而下之,久益後天。僖公五年,《周歷》正月辛亥朔,餘四分之一,南至。以歲差推之,日在牽牛初。至宣公十一年癸亥,《周歷》與《麟德歷》俱以庚戌日中冬至,而月朔尚先《麟德歷》十五辰。至昭公二十年己卯,《周歷》以正月己丑朔日中南至,《麟德歷》以己丑平旦冬至。哀公十一年丁巳,《周歷》入己酉蔀首,《麟德歷》以戊申禺中冬至。惠王四十三年己丑,《周歷》入丁卯蔀首,《麟德歷》以乙丑日昳冬至。呂后八年辛酉,《周歷》入乙酉蔀首,《麟德歷》以壬午黃昏冬至;其十二月甲申,人定合朔。太初元年,《周歷》以甲子夜半合朔冬至,《麟德歷》以辛酉禺中冬至,十二月癸亥晡時合朔。氣差三十二辰,朔差四辰。此疏密之大較也。



 僖公五年,《周歷》、漢歷、唐歷皆以辛亥南至。後五百五十餘歲,至太初元年,《周歷》、漢歷皆得甲子夜半冬至,唐歷皆以辛酉,則漢歷後天三日矣。祖沖之、張胄玄促上章歲至太初元年,沖之以癸亥雞鳴冬至,而胄玄以癸亥日出。欲令合於甲子,而適與《魯歷》相會。自此推僖公五年,《魯歷》以庚戌冬至,而二家皆以甲寅。且僖公登觀臺以望而書云物,出於表晷天驗,非時史人意度。乖丘明正時之意,以就劉歆之失。今考麟德元年甲子,唐歷皆以甲子冬至,而《周歷》、漢歷皆以庚午。然則自太初下至麟德差四日,自太初上及僖公差三日,不足疑也。



 以歲差考太初元年辛酉冬至加時,日在斗二十三度。漢歷,氣後天三日,而日先天三度,所差尚少。故落下閎等雖候昏明中星,步日所在,猶未覺其差。然《洪範》、《太初》所揆,冬至昏奎八度中,夏至昏氐十三度中,依漢歷,冬至日在牽牛初太半度,以昏距中命之,奎十一度中;夏至,房一度中。此皆閎等所測,自差三度,則劉向等殆已知《太初》冬至不及天三度矣。



 及永平中,治歷者考行事,史官注日,常不及《太初歷》五度。然諸儒守讖緯,以為當在牛初,故賈逵等議:「石氏星距,黃道規牽牛初直斗二十度,於赤道二十一度也。《尚書》《考靈耀》斗二十二度,無餘分。冬至,日在牽牛初,無牽年所起文。編等據今日所去牽牛中星五度,於斗二十一度四分一,與《考靈耀》相近。」遂更歷從斗二十一度起。然古歷以斗魁首為距,至牽牛為二十二度,未聞移牽牛六度以就《太初》星距也。逵等以末學僻於所傳,而昧天象,故以權誣之,而後聽從他術,以為日在牛初者,由此遂黜。



 今歲差,引而退之,則辛酉冬至,日在斗二十度,合於密率,而有驗於今;推而進之,則甲子冬至,日在斗二十四度,昏奎八度中,而有證於古。其虛退之度,又適及牽牛之初。而沖之雖促減氣分,冀符漢歷,猶差六度,未及於天。而《麟德歷》冬至不移,則昏中向差半次。淳風以為太初元年得本星度,日月合璧,俱起建星。賈逵考歷,亦云古歷冬至皆起建星。兩漢冬至,日皆後天,故其宿度多在斗末。今以儀測,建星在斗十三四度間,自古冬至無差,審矣。



 按古之六術,並同《四分》。《四分》之法,久則後天。推古歷之作,皆在漢初,卻較《春秋》,朔並先天,則非三代之前明矣。



 古歷,南斗至牽牛上星二十一度,入《太初》星距四度,上直西建之初。故六家或以南斗命度,或以建星命度。方周、漢之交,日已潛退,其襲《春秋》舊歷者,則以為在牽牛之首;其考當時之驗者,則以為入建星度中。然氣朔前後不逾一日,故漢歷冬至,當在斗末。以為建星上得《太初》本星度,此其明據也。《四分》法雖疏,而先賢謹於天事,其遷革之意,俱有效於當時,故太史公等觀二十八宿疏密,立晷儀,下漏刻,以稽晦朔、分至、躔離、弦望,其赤道遺法,後世無以非之。故雜候清臺,《太初》最密。若當時日在建星,已直斗十三度,則壽王《調歷》宜允得其中,豈容頓差一氣而未知其謬?不能觀乎時變,而欲厚誣古人也。



 後百餘歲,至永平十一年,以《麟德歷》較之,氣當後天二日半,朔當後天半日。是歲《四分歷》得辛酉蔀首,已減《太初歷》四分日之三,定後天二日太半。《開元歷》以戊午禺中冬至,日在斗十八度半弱,潛退至牛前八度。進至辛酉夜半,日在斗二十一度半弱。《續漢志》云:「元和二年冬至,日在斗二十一度四分之一。」是也。



 祖沖之曰:「《四分歷》立冬景長一丈,立春九尺六寸,冬至南極日晷最長。二氣去至日數既同,則中景應等。而相差四寸,此冬至後天之驗也。二氣中景,日差九分半弱,進退調均,略無盈縮。各退二日十二刻,則景皆九尺八寸。以此推冬至後天亦二日十二刻矣。」東漢晷漏定於永元十四年,則《四分》法施行後十五歲也。



 二十四氣加時,進退不等,其去午正極遠者,四十九刻有餘。日中之晷,頗有盈縮,故治歷者皆就其中率,以午正言之。而《開元歷》所推氣及日度,皆直子半之始。其未及日中,尚五十刻。因加二日十二刻,正得二日太半。與沖之所算及破章二百年間輒差一日之數,皆合。



 自漢時辛酉冬至,以後天之數減之,則合於今歷歲差斗十八度。自今歷戊午冬至,以後天之數加之,則合於賈逵所測斗二十一度。反復僉同。而淳風冬至常在斗十三度,豈當時知不及牽牛五度,而不知過建星八度耶?



 晉武帝太始三年丁亥歲冬至,日當在斗十六度。晉用魏《景初歷》,其冬至亦在斗二十一度少。太元九年,姜岌更造《三紀術》,退在斗十七度。曰:「古歷斗分強,故不可施於今;《乾象》斗分細,故不可通於古。《景初》雖得其中,而日之所在,乃差四度,合朔虧盈,皆不及其次。假月在東井一度蝕,以日檢之,乃在參六度。」岌以月蝕沖知日度,由是躔次遂正,為後代治歷者宗。



 宋文帝時,何承天上《元嘉歷》,曰:「《四分》、《景初歷》,冬至同在斗二十一度,臣以月蝕檢之,則今應在斗十七度。又土圭測二至,晷差三日有餘,則天之南至,日在斗十三四度矣。」事下太史考驗,如承天所上。以《開元歷》考元嘉十年冬至,日在斗十四度,與承天所測合。



 大明八年,祖沖之上《大明歷》,冬至在斗十一度,《開元歷》應在斗十三度。梁天監八年,沖之子員外散騎侍郎恆之上其家術。詔太史令將作大匠道秀等較之,上距大明又五十年,日度益差。其明年,閏月十六日,月蝕,在虛十度,日應在張四度。承天歷在張六度,沖之歷在張二度。



 大同九年,虞廣刂等議:「姜岌、何承天俱以月蝕沖步日所在。承天雖移岌三度,然其冬至亦上岌三日。承天在斗十三四度,而岌在斗十七度。其實非移。祖沖之謂為實差,以推今冬至,日在斗九度,用求中星不合。自岌至今,將二百年,而冬至在斗十二度。然日之所在難知,驗以中星,則漏刻不定。漢世課昏明中星,為法已淺。今候夜半中星,以求日沖,近於得密。而水有清濁,壺有增減,或積塵所擁,故漏有遲疾。臣等頻夜候中星,而前後相差或至三度。大略冬至遠不過鬥十四度,近不出十度。」又以九年三月十五日夜半,月在房四度蝕。九月十五日夜半,月在昴三度蝕。以其沖計,冬至皆在斗十二度。自姜岌、何承天所測,下及大同,日已卻差二度。而淳風以為晉、宋以來三百餘歲,以月蝕沖考之,固在斗十三四度間,非矣。



 劉孝孫《甲子元歷》,推太初冬至在牽牛初,下及晉太元、宋元嘉皆在斗十七度。開皇十四年,在斗十三度。而劉焯歷仁壽四年冬至,日在黃道鬥十度,於赤道鬥十一度也。其後孝孫改從焯法,而仁壽四年冬至,日亦在斗十度。焯卒後,胄玄以其前歷上元起虛五度,推漢太初,猶不及牽牛,乃更起虛七度,故太初在斗二十三度,永平在斗二十一度,並與今歷合。而仁壽四年,冬至在斗十三度,以驗近事,又不逮其前歷矣。《戊寅歷》,太初元年辛酉冬至,進及甲子,日在牽牛三度。永平十一年,得戊午冬至,進及辛酉,在斗二十六度。至元嘉,中氣上景初三日,而冬至猶在斗十七度。欲以求合,反更失之。又曲循孝孫之論,而不知孝孫已變從《皇極》,故為淳風等所駁。歲差之術,由此不行。



 以太史注記月蝕沖考日度,麟德元年九月庚申,月蝕在婁十度。至開元四年六月庚申,月蝕在牛六度。較《麟德歷》率差三度,則今冬至定在赤道鬥十度。



 又《皇極歷》,歲差皆自黃道命之,其每歲周分,常當南至之軌,與赤道相較,所減尤多。計黃道差三十六度,赤道差四十餘度,雖每歲遁之,不足為過。然立法之體,宜盡其原,是以《開元歷》皆自赤道推之,乃以今有術從變黃道。



\end{pinyinscope}