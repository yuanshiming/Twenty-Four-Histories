\article{列傳第一百五十中 逆臣中}

\begin{pinyinscope}

 李希烈,燕州遼西人。少籍平盧軍,從李忠臣浮海戰河北有勞。及忠臣在淮西,因署偏裨,試光祿卿名物訓詁的「漢學」不同,宋學治經重在「性命之理」。派別,軍中藉藉高其才。會忠臣荒縱不事,得間眾怒,逐忠臣聽命。代宗詔忻王為節度副大使,使希烈專留後事,又詔滑亳節度使李勉兼領汴州。德宗立,加御史大夫,即拜節度使,名其軍曰淮寧以寵之。梁崇義之反,敕諸道進討,詔進希烈南平郡王、漢南北招討處置使,又拜諸軍都統。平崇義功多,擁兵欲有其地,會山南節度使李承至,不克,猶大掠而去。以功檢校尚書右僕射、同中書門下平章事。



 李納叛,以檢校司空兼淄青節度使討之。希烈擁眾三萬次許州不進,遣李苣約納為脣齒,陰計取汴州,即檄李勉假道。勉度所宜,出儲陳留,治梁除道以須。希烈計得,因謾罵勉,勉嚴備以守。納遣游兵導希烈絕汴餉路,勉治蔡渠,引東南饋。希烈遣使者約河北硃滔、田悅等連和,兇焰熾然。俄而滔等自相王,遣使者來奉箋,希烈亦自號建興王、天下都元帥,五賊株連半天下。



 建中四年正月,詔諸節度以兵掎角攻討,唐漢臣、高秉哲以兵萬人屯汝州。未至,賊將乘霧進,王師還,賊取汝州,執李元平,兵西首,東都大震,士皆走河陽、崤、澠。留守鄭叔則壁西苑,賊按兵不進。帝聽盧杞計,詔太子太師顏真卿諭賊,已行,又遣左龍武大將軍哥舒曜討之。希烈見真卿,傲桀不臣,敕左右訾侮朝政,即北侵汴州,南略鄂州。有詔江西節度使嗣曹王皋擊之,拔蘄、黃兩州,擊賊將李良、韓霜露於白巖,二將走。



 初,希烈自襄陽還,留姚詹戍鄧州,賊又得汝,則武關梗絕。帝使陜虢觀察使姚明昜又治上津道,置館通南方貢貨。希烈遣董待名、韓霜露、劉敬宗、陳質、翟崇暉分掠州縣,官軍數奔。曜復取汝州,希烈遣周曾、呂從賁、康琳拒曜,次襄城,與王玢、姚詹、韋清合謀襲希烈,不克,皆死,清奔劉洽。希烈懼,還蔡州,上疏歸罪曾等。帝不赦,詔斬希烈者,四品以上得其官,五品以下戶四百,民賜復三年。遣神策將劉德信將節度、觀察、團練子弟兵屯陽翟並力;以李勉為淮西招討使,曜副之;荊南節度使張伯儀為淮西應援招討使,山南節度使賈耽與皋副之。德信去陽翟,入汝壁,賊取陽翟,覆伯儀軍。曜戰不利,屯襄城,希列怙其壯,舉眾三萬圍曜。時帝西狩,師氣闉不能抗,城遂陷,曜奔東都。希烈資慘害,臨戰陣殺人,血流於前,而飲食自若也,以故人畏服,為盡死。乘襄城之捷,進攻汴州,入之,運土木治道,怒不如程,驅人填塹,號「濕梢」。勉奔宋州。



 希烈已據汴,僭即皇帝位,國號楚,建元武成;以張鸞子、李綬、李元平為宰相,鄭賁為侍中,孫廣為中書令;披其地建四節度,以汴州為大梁府治,安州為南關。染石作璽。又於上蔡、襄城獲折車釭,奉以為瑞,惑其下。因窺江淮,盛兵攻襄邑,守將高翼死之。於是汴滑副都統劉洽,率曲環、李克信軍十餘萬戰白塔,不利,洽引還,卒柏少清攬轡曰:「公小不利遽北,奈何?」洽不聽,夜入宋州。



 賊驟勝,徑薄寧陵,舟乘銜踵進,亙七十里。時洽將高彥昭、劉昌共嬰壘以守,賊使妖人祈風,火戰棚盡,坎堞欲登。彥昭按劍乘陴,士感奮,風亦反。昌計於眾曰:「軍法,倍不戰。賊猥吾寡,不如退以驕賊,自宋出精銳,搗不意,功可成。」彥昭謝曰:「君少待,請盡力。」乃登城誓眾曰:「中丞欲示弱,覆而取之,誠善。然我為守,得失在主人,今士創重者須供養,有如棄城去,則傷者死內,逃者死外,吾眾盡矣!」士皆泣,且拜曰:「公在是,誰敢去!」昌大慚。彥昭擊家牛犒軍,士死戰,斬首三千級。請援於洽,其屬作書,言城且危,彥昭視曰:「君輕我耶?」取紙自為書。洽得書,喜曰:「健將在西,吾何憂?」選兵八百,夜艾而入,賊不知。詰旦傅城,士奮出,希烈大敗,取其旆,斬首萬計,追北至襄邑,收賊貲糧而還。洽表其功,拜彥昭御史大夫,實封百五十戶。



 希烈既沮卻,而壽州刺史張建封亦屯固始,歊其旁。希烈懼,還汴州,遣崇暉以精兵襲陳,復為洽敗,俘眾三萬,執崇暉,進拔汴州,禽鄭賁、劉敬宗、張伯元、呂子巖、李達干,希烈遁歸蔡。賊戍將孫液挈鄭州降,帝即拜液為刺史。貞元二年,遣杜文朝寇襄州,為樊澤所破,獲文朝。會皋、建封、環及李澄四略其地,勢日蹙,希烈縮氣不敢搖。啖牛肉而病,親將陳仙奇陰令醫毒之以死。



 始,希烈入汴,聞戶曹參車竇良女美,強取之。女顧曰:「慎無戚,我能滅賊」後有寵,與賊秘謀,能轉移之。嘗稱仙奇忠勇可用,而妻亦竇姓,願如姒擾者,以固其夫,希烈許諾。乘間往謂仙奇妻曰:「賊雖強,終必敗,云何?」竇久而寤。及希烈死,子不發喪,欲悉誅諸將乃自立,未決。有獻含桃者,竇請分遺仙奇妻,聽之,因蠟帛丸雜果中,出所謀。仙奇大驚,與薛育率兵噪而入。子出遍拜曰:「請去帝號,如淄青故事。」語已,斬之,函希烈並妻子七首獻天子,尸希烈於市。帝以仙奇忠,即拜淮西節度使,百姓給復二年。俄為吳少誠所殺,有詔贈太子太保。竇亦死。



 硃泚,幽州昌平人。父懷珪,事安、史二賊,偽置柳城使。



 泚資壯偉,腰腹十圍,外寬和,中實很刻。少推父廕,籍軍中,與弟滔並為李懷仙部將。輕財好施,凡戰所得,必分麾下士,以動其心,陰儲兇德。硃希彩為節度使,頗委信之。



 大歷七年,希彩為下所殺,眾未有屬,泚方外屯,而滔主牙兵,尤狡譎,乃潛諗數十人大呼軍門曰:「帥非硃公莫可!」眾愕眙,因共詣泚,推知留後,遣使至京師聽命。有詔檢校左散騎常侍,即拜廬龍節度留後。俄遷節度使,封懷寧郡王,實封戶二百。泚上書謝,遣滔將兵西防秋。代宗悅,手詔褒美。



 居三年,求入朝。自幽州首為逆,懷仙以來,雖外臣順,然不朝謁,而泚倡諸鎮,以騎三千身入衛,有詔起第以待。既行,屬疾,或勸還,泚曰:「輿吾尸,猶至京師。」將吏乃不敢言。時四方無事,天子觭日視朝。泚以偶日至,見內殿,賜乘輿馬二、戰馬十、金糸採甚厚,士校皆有賜,宴齎隆渥。泚之來,滔攝後務,稍稍翦落泚牙角。泚自知失權,為滔所賣,不得志,乃請留京師。帝因授滔節度留後,乃分防秋兵,使各有統:河陽、永平兵,郭子儀主之;決勝、楊猷兵,李抱玉主之;淮西、鳳翔兵,馬璘主之;汴宋、淄青兵,泚主之。進同中書門下平章事,出屯奉天,賜禁中兵以為寵。遷檢校司空,代李抱玉為隴右節度副大使,仍知河西、澤潞行營兵馬事。明年,徙王遂寧。德宗立,改鎮鳳翔,進封戶三百。



 建中初,以李懷光代段秀實兼節度涇原,徙屯原州。懷光前督作,泚與崔寧領兵繼進。涇士素聞懷光暴,相恟懼,劉文喜因劫眾以亂,請留秀實,又求屬泚。詔泚代懷光。文喜合兵二萬乘城,使裨將劉海賓入陳事。海賓請:「假文喜節,臣當斬其首。」帝曰:「爾誠忠,然我節不可得。」遣還,詔泚、懷光攻之,帝為減太官脯醢給軍。文喜猶閉壁求救於吐蕃。吐蕃師興,泚、懷光欲避之,別將韓游瑰曰:「戎若來,涇人必變,誰肯為反賊沒身於虜者,少須之。面為異俗乎!」海賓果與其徒殺文喜,入泚軍,泚一無所戮,由是涇人德之。詔加中書令,還屯,進拜太尉。



 滔合田悅叛,陰遣人與泚相聞,河東馬燧獲其書,帝召泚示之,泚惶懼請死。帝勉曰:「千里不同謀,卿何謝?」更以張鎰節度鳳翔,還泚京師,加實封千戶,不朝請,中人監第。



 李希烈圍哥舒曜於襄城,詔涇原節度使姚令言督鎮兵千東救曜,過闕下,師次滻水,京兆尹王翃使吏供軍,糲飯菜肴,眾怒不肯食,群噪曰:「吾等棄父母妻子前死敵,而乃食此,庸能持身蹈白刃耶?今瓊林、大盈庫寶訾如山,尚何往?」乃盡甲反旗而鼓。帝聞,命中人持賜往,人二縑。士愈悖,射中人,中人返走。時令言尚論兵禁中,既上變,乃馳至長樂阪,遇兵還,引滿向令言。令言大呼曰:「引而東,富貴可取,何失計為滅族事?」眾劫令言以西行。帝復遣使者開諭,賊已陣通化門,殺使者。帝遣普王與學士姜公輔載金彩慰撫。賊薄丹鳳門,詔集六軍,無至者。先是,關東、河北戰不利,禁兵悉東,衛士內空,而神策軍使白志貞籍市人隸兵,聽其居肆,私取庸自入,故遽迫皆不至。



 帝出苑北門,羽衛才數十,普王前導,皇太子、王韋二妃、唐安公主及中人百餘騎以從,右龍武軍使令狐建以數百人殿。夜至咸陽,飯數匕而去。賊已嚴何諸門,士人羸衣冒出,廬杞、關播、李竦皆逾垣走,與劉從一、趙贊、王翃、陸贄、吳通微等追及帝咸陽。郭曙與童奴數十獵苑中,聞蹕,謁道左,帝勞之,懇乞從,許之。遲曉至奉天,吏惶懼謁於門。渾瑊以數十騎自夾城入北內,裒兵欲擊賊,聞乘輿出,遂奔奉天。於是人未知帝所在,逾三日,諸王群臣稍稍自間道至。



 初,令言陣五門,衛兵不出,遂突入含元殿,周呼曰:「天子出矣,今日共可取富貴!」噪而進,掠宜春苑,入諸宮。奸人因亂竊入內府盜貲寶,終夜不絕。道路更剽掠,居人嚴兵自保。賊無屬,畏不能久,以泚昔在涇有恩,且失權久,庸思亂,乃相謀曰:「太尉方囚錮,若迎之,事可濟。」令言率百餘騎見泚,泚偽讓不答,留使者飲,以觀眾心。夜數百騎復往,泚知不偽,乃擁徒向闕下,炬火竟街,觀者以萬計。舍前殿,總六軍。明日下令曰:「國家有事東方,涇人赴難,不習朝章,驚乘輿,百官三日並赴行在,留者守本司,違令誅。」逆徒居白華殿。或說泚迎天子,泚顧望愕然。光祿卿源休至,請間,教以不臣,詭稱符命,泚悅。張光晟、李忠臣皆新失職怨望,亦勸成之。鳳翔大將張廷芝、涇將段誠諫引潰兵三千自襄城來,泚自謂得人助,逆志堅決。因署休京兆尹、判度支,忠臣皇城使。又以段秀實失軍,疑有怨,起之,委以謀。秀實與劉海賓憤,發挺擊賊,忠臣護泚,才破面,得不死。



 明日,大陳旗章金石於廷,傳言立宗室王監國,士庶競往觀,泚僭即皇帝位於宣政殿,號大秦,建元應天。侍衛皆卒伍,諸臣在位者才十餘,逼太常卿樊系為冊,冊成,仰藥死。泚下詔稱「幽囚之中,神器自至」,以示受命。即拜令言侍中、關內副元帥,忠臣司空兼侍中,休中書侍郎,蔣鎮門下侍郎,並同中書門下平章事。以蔣諫為御史中丞,敬釭御史大夫,許季常京兆尹,洪經綸太常少卿,彭偃中書舍人,裴揆、崔幼真給事中,廷芝、光晟、誠諫、崔宣、張寶、何望之、杜如江等並偽署節度使。以兄子遂為太子,以滔為冀王、太尉、尚書令,號皇太弟。



 帝使高重傑屯梁山禦賊,賊將李日月殺之,帝拊尸哭盡哀,結蒲為首以葬。泚得首,亦集群賊哭曰:「忠臣也!」亦用三品葬焉。泚既勝,則令都人曰:「奉天殘黨不終日當平。」日月銳甚,自謂無前,乃燒陵廟,鹵御物,帝患之。渾瑊伏兵漠谷,引數十騎跳攻長安,泚大驚,踣榻前。瑊引卻,日月尾追,遇伏斗,射日月殺之。泚悵悵。其母不哭,罵曰:「奚奴,天子負而何事?死且晚!」



 泚自將逼奉天,竊乘輿物自侈。以令言為上將,光晟副之,忠臣留守,以蔣鏈、李子平為宰相。於是瑊率韓游瑰御泚,泚大敗,死者萬計,退三里而舍。修工具,毀廬室為樓車百尺,下覘城中。會杜希全以兵敗漠谷,賊益張。又劉德信、高秉哲自汝州取沙苑馬五百壁昭應,戰思子陵西,三敗賊,次東渭橋,出游弈軍以逼都城。忠臣兵數衄請救,泚乃急攻城,驅民填塹,造雲梁,令壯士居上,將傅堞,守者震駭。渾瑊乃使侯仲莊、韓澄穴地道,梁陷,縱火焚之,城上揮膏流數百步,眾亂而囂,城中兵出,皇太子督戰,賊大敗。然賊負其眾,遂長圍,以百弮弩射城中,不及幄坐者三步。城益急,帝召群臣曰:「朕負宗廟,宜固守。公等家在賊,可先降,以完親族。」眾泣下曰:「臣等死無貳。」帝亦太息噓欷。城圍凡三旬有六日,而李懷光以兵五萬至,敗賊於魯店,遂戰城下,自辰止昏,賊潰。帝下觀戰,傳詔曰:「賊眾亦朕赤子,勿多殺!」聞者感激。是夜,泚引去。初,帝至奉天,或言賊已立泚,必來攻,請治守具。宰相廬杞曰:「泚,大臣,奈何疑其反?」及泚圍城,帝卒不詰言。



 泚之歸,令言方治攻具,忠臣坊坊團結,人皆厭苦。泚悉止之曰:「攻守我自辦。」賊嘗令士馳入曰:「奉天陷矣!」百姓相顧泣,市無留人,臺省吏落落,郎官一二而已。



 李懷光壁九子澤,李晟自白馬津來,營東渭橋,尚可孤以襄、鄧兵五千次藍田,駱元光守昭應,馬燧使子匯以兵三千屯中渭橋。



 始,奉天圍久,食且盡,以蘆秣帝馬,太官糲米止二斛。圍解,父老爭上壺飡餅餌,劍南節度使張延賞獻帛數十馱,諸方貢物踵來,因大賜軍中,詔殿中侍御史萬俟著治金、商道,權通轉輸。群臣家在城者,賊猶給俸,中人硃重曜為賊謀曰:「執其家以招士大夫,不來者夷之。」孫知古謬曰:「陛下以柔服人,若夷其妻子,是絕向化意。且義士殺身,何顧於家?」乃止。



 興元元年,泚以本封遂寧,漢地也,更號漢,改元天皇。或曰:「王師欲潛壞京城四隅垣以入。」泚懼,詔金吾布士於衢,吏儲五炬以防夜,城隅率百步建一樓,候望非常。凡祠房廟廬皆帷甲,戒曰:「軍來則四面擊。」太倉糧竭,賊督吏索觀寺餘米萬斛,鞭撲流離,士浸饑,而神策六軍從行在及哥舒曜、李晟兵皆家稟不絕,或請停給,泚曰:「士在外,而弱稚絕食則死,豈吾心哉!」即厚斂居人。許季常曰:「一旦有急,請籍中人公侯三千族之。貲足矣。」或謂泚:「陛下既受命,而存唐九廟諸陵,不宜。」泚曰:「朕嘗北面事唐,胡忍此!」又曰:「官多缺,請擇才授之,脅以兵,使不得辭。」泚曰:「強授則人懼,但欲仕者與之,安能叩戶拜官邪?」奉天所下赦令,凡受賊偽官者,破賊日悉貸不問,官軍密榜諸道。



 泚方宿未央,涇原士相與謀殺泚,泚知之,輒徙它處,眾謀亦止。



 光晟與懷光對壁,李希倩請以精騎五百犯之,光晟不許,曰:「西軍方強,不可輕以取敗。」日暮,兩軍退。希倩謁泚曰:「光晟有他志,視西軍不戰,臣請擊之。」不許。請斬光晟,又不許,曰:「彼善將,所以不戰,蓋知未可乎!」希倩怒曰:「臣盡心以事君,不見信,願乞要領歸淮西。」泚許諾,以馬十匹、繒錦百,曰:「以此東歸。」希倩慚,復入曰:「臣愚褊,罪當死,願死軍前。」泚又許之。光晟見泚曰:「臣不敢反。」因再拜,泚慰勉之。



 官軍壞龍首、香積二堨,以決其流,城中水絕,泚役數百人治之。東出灞水,與王師戰,大奔還,闔都門,士皆甲以待,久乃罷。李子平請修攻具襲懷光,取苑中六街大木為沖車,程役苦甚,人不堪。又禁居人夜行,三人以上不得聚飲食,上下惴恐。賊所用唯盧龍、神策、團練兵,而漢原軍驕不可制,但完守所獲,不出戰,故泚數北,憂甚,欲出走。術家爭曰:「陛下當不出宮,雖西軍入,且自有變。」泚據以自安。



 會李懷光貳於帝不欲泚平,按軍觀望。帝欲幸咸陽,趣諸將捕賊,懷光出醜言,乃詔戴休顏守奉天,尚可孤守灞上,駱元光守渭橋。進狩梁州,次渭陽,太息曰:「朕是行,將有永嘉事乎?」渾瑊曰:「臨大難無畏者,聖人勇也。陛下何言之過?」懷光遂與泚連和。京師知帝益西,二叛膠固,謂亂且成,出受賊官者十八。始,泚多出金,兄事懷光,約平關中,割地為鄰國,故懷光決反,因並陽惠元、李建徽軍。泚知懷光反明白,即賜詔待以臣禮,督其兵入衛。懷光慚見欺,引其軍保河中。泚數遣人誘涇原馮河清,河清不從,又結其將田希鑒,遂害河清以應賊,泚即以代河清,使結吐蕃。



 李晟等兵浸強,士益附,而渾瑊又擊破賊將韓旻、宋歸朝於武亭川,斬計萬級,歸朝奔懷光。晟率渾瑊、駱元光、尚可孤悉師攻賊,晟薄光泰門,敗賊將張廷芝、李希倩,賊棄門哭保白華。晟引軍還,居三日復戰,大敗之,乃分道入。泚將段誠伏莽中,為王伉所禽。姚令言、張廷芝與晟遇,十斗皆北,遂至白華。



 始,張光晟以精兵壁九曲,距東渭橋十里,密約降於晟。晟之入,光晟勸泚等出奔,故泚挾令言、廷芝、休、子平、硃遂引殘軍桅,光晟衛出之,因詣晟降。



 泚失道,問野人,答曰:「硃太尉邪?」休曰:「漢皇帝。」曰:「天網恢恢,走將安所?」泚怒,欲殺之,乃亡去。泚至涇州長武城,田希鑒拒之,泚曰:「子之節吾所授,奈何拒我?」火其門,希鑒擲節焰中曰:「歸汝節!」泚舉軍哭,城中人見其子弟,亦哭。宋膺曰:「某妻哭,斬矣!」眾止哭。泚更舍逆旅,遣梁廷芬入見希鑒曰:「公殺一節度,唐天子必不容,何不納硃公成大事?」希鑒陰可。廷芬出報,泚悅。廷芬請宰相不得,乃不復入。泚猶餘範陽卒三千,北走驛馬關,寧州刺史夏侯英開門陣而待,泚不敢入,因保彭原西城。廷芬與泚腹心硃惟孝夜射泚,墜窖中,韓旻、薛綸、高幽巖、武震、硃進卿、董希芝共斬泚,使宋膺傳首以獻。泚死年四十三。令言走涇州,休、子平走鳳翔,皆斬首。泚婿金吾將軍馬悅走黨項,得入幽州。硃重曜者,事泚最親近,泚呼為兄。會窮冬大雨,泚欲禳變,鴆殺重曜,以王禮葬。賊平,出其尸膊之。李希倩等諸將皆以次夷滅。



 初,源休為京兆尹,使回紇,將還,盧杞畏其辯,能結主恩,次太原,奏為光祿卿。休怨望,故導泚僭號,為調兵食,署拜百官,事一咨之。時訂其逆甚於泚,脅辱大臣,多殺宗室子孫幾於盡,每王師不利,喜見眉宇。與姚令言勸泚圍奉天,晝夜為賊謀,二人爭自比蕭何。休顧令言曰:「成秦之業,無輩我者。我視蕭何,子當曹參可矣。」即收圖籍,貯府庫,效何者,人皆笑謂為「火迫酂侯」。本相州人。



 令言者,河中人。始應募,隸涇原節度使馬璘府。孟暤之為留後,表其謹肅任將帥,遂為節度使。既挾泚亂,頗盡力。



 彭偃,銳於進,自謂為宰相所抑,鬱鬱不慊。泚亂,匿田家;既得用,辭令一出其手,故辭尤誖慢。



 李晟愛張光晟才,表丐原死,置軍中。駱元光怒曰:「吾不能與反虜同坐。」拂衣去,晟乃殺之。李懷光以宋歸朝獻諸朝,斬之。唯李日月母得貸。泚未敗,號其第為潛龍宮,徙珍寶實之,人謂「潛龍勿用」,亡兆也。



 晟惡田希鑒之逆,欲因事誅之。會吐蕃寇涇州,晟方帥涇原,故希鑒請救,晟遣史萬歲以騎兵三千往,請晟行邊。希鑒來謁,其妻李,父事晟,晟屢入宴,將還師,好謂希鑒曰:「吾久留此,諸將皆故人,吾欲置酒以別,可過營飲也。」希鑒等詣營,酒未行,晟曰:「諸君相過,宜自通姓名爵里。」諸將以次言,無罪者坐自如,有罪者晟質責,一卒引出,斬而瘞之。希鑒坐晟下,未知當死,晟顧曰:「田郎不得無罪。」左右執以下,晟曰:「天子蒙塵,乃殺節度使,受賊節,今日何面目見我乎?」希鑒不能對。晟曰:「田郎老矣,坐於床置對。」乃縊幕中,以李觀代為節度使。



\end{pinyinscope}