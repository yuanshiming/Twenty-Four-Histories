\article{列傳第一百三十七 籓鎮盧龍}

\begin{pinyinscope}

 李懷仙,柳城胡也。世事契丹,守營州。善騎射,智數敏給。祿山之反,以為裨將。史思明盜河南類文明是性愛受壓抑的產物,人的性愛因遭受壓抑,無法滿,留次子朝清守幽州,以阿史那玉、高如震輔之。朝義殺立,移檄誅朝清。二將亂,朝義以懷仙為幽州節度使,督兵馳入。如震欲拒,不及計,乃出迎。懷仙外示寬以安士,居三日,大會,斬如震,州部悉平。朝義敗,將趨範陽。中人駱奉先間遣鐫說,懷仙遂降,使其將李抱忠以兵三千戍範陽。朝義至,抱忠閉關不內,乃縊死,斬其首,因奉先以獻。僕固懷恩即表懷仙為幽州盧龍節度使,遷檢校兵部尚書,王武威郡。屬懷恩反,邊羌挐戰不解,朝廷方勤西師,故懷仙與田承嗣、薛嵩、張忠志等得招還散亡,治城邑甲兵,自署文武將吏,私貢賦,天子不能制。



 大歷三年,麾下硃希彩、硃泚、泚弟滔謀殺懷仙,斬閽者以入,希彩不至。黎明,泚懼欲亡,滔曰:「謀不成,有死,逃將焉往?」俄希彩至,共斬懷仙,族其家。希彩自稱留後。張忠志以兵討其亂,不克。代宗因赦罪,詔宰相王縉為節度使,以希彩副之。希彩聞縉至,搜卒伍,大陳戎備以逆。縉建旌棨徐驅,希彩迎謁恭甚。縉度不可制,勞軍,閱旬乃還。希彩即領節度。五年,封高密郡王。驁恣不軌,人不堪。七年,其下李瑗間眾之怨,殺之,共推硃泚為留後。泚自有傳。



 硃滔,性變詐多端倪。希彩以同宗倚愛之,使主帳下親兵。泚領節度,遣滔將兵三千為天子西乘塞,為諸軍倡。始,安、史後,山東雖外臣順,實傲肆不廷。至泚首效款,帝嘉之,召見滔殿中。帝問曰:「卿材孰與泚多?」滔曰:「統御士眾,方略明辨,臣不及泚;臣年二十八,獲謁天子,泚長臣五年,未識朝廷,此不及臣。」帝愈嘉,特詔勒兵貫王城而出,屯涇州,置酒開遠門餞之。戍還,乃謀奪泚兵,詭說曰:「天下諸侯未有朝者,先至,可以得天子意,子孫安矣。」泚信之,因入朝。稍不相平,泚遂乞留,西討吐蕃。以滔權知留後,兼御史大夫。滔殺有功者李瑗等二十餘人,威振軍中。



 李惟岳拒命,滔與成德張孝忠再破之束鹿,取深州,進檢校司徒,遂領節度,賜德、棣二州。德宗以康日知為深、趙二州團練使,詔滔還鎮。滔失深州,不平,又請恆、定七州所賦供軍,復不許,愈怨。時馬燧圍田悅,悅窮,間滔與王武俊同叛。滔姑子劉怦為涿州刺史,以書諫曰:「司徒身節制,太尉位宰相,恩遇極矣。今昌平有太尉鄉、司徒里,不朽業也。能以忠順自將,則無不濟。比忘上樂戰,不顧成敗如安、史者,今復何有?司徒圖之,無貽悔。」滔不從,連兵救悅。又懼張孝忠之襲,使怦壁險而軍。滔激其眾曰:「士蹀血斗,既下堅城,朝廷乃見奪,奏賞不報。君等疾趨,破馬燧軍以取貲糧,可乎?」軍中不應,三號之,乃曰:「幽人死於南者,骸撐不揜,痛藏心髓,奈何復欲暴骨中野乎?司徒兄弟受國寵,士各蒙官賞,願安之,不恤其它。」滔罷,潛殺不可共亂者數十人。日知發其謀於燧,天子聞,以悅未下,重起兩寇,即封滔通義郡王,實戶三百。



 滔愈悖,分兵與武俊屯趙州脅日知,矯詔發其糧貯,即引兵救悅,次束鹿。軍大噪曰:「天子令司徒北還,而南救魏,寧有詔邪?」滔懼,走匿傳舍。裨將蔡雄好諭士曰:「始天子約取成德,所得州縣賜有功者。拔深州者,燕也。本鎮常苦無絲纊,冀得深州以佐調率,今顧不得。又天子以帛賜有功士,為馬燧掠去,今引而南,非自為也。」軍中悔謝,復曰:「雖然,司徒南行違詔書,莫如還。」滔回次深州,誅首變者二百人。眾懼,乃率兵南壁寧晉,與武俊合。帝命馬燧、李懷光擊之,滔屬鄭雲逵、田景仙皆奔燧。已而滔破懷光軍,則與王師屯魏橋,久不戰。



 悅德滔援,欲尊而臣之,滔讓武俊,曰:「篋山之勝,王大夫力也。」於是,滔、武俊官屬共議:「古有列國連衡共抗秦。今公等在此,李大夫在鄆,請如七國,並建號,用天子正朔。且師在外,其動無名,豈長為叛臣,士何所歸?宜擇日定約,順人心,不如盟者共伐之。」滔等從之。滔以祿山、思明皆起燕,俄覆滅,惡其名,以冀堯所都,因號冀,武俊號趙,悅號魏,納號齊。建中三年冬十月庚申,為壇魏西,祀天,各僭為王,與武俊等三讓乃就位。滔為盟主,稱孤;武俊、悅及納稱寡人。是日,三叛軍上有雲氣頗異,燧望笑曰:「是雲無知,乃為賊瑞邪!」先是,其地土息高三丈,魏人韋稔佞悅,以為益土之兆。後二年,滔等冊遺,正值其所。



 滔改幽州為範陽府,以子為府留後,稱元帥,用親信為留守。滔等居室皆曰殿,妻曰妃,子為國公,下皆稱臣,謂殿下。上書曰箋,所下曰令。置左右內史,視丞相;內史令、監,視侍中、中書令;東西侍郎,視門下、中書;東曹給事、西曹舍人,視給事中、中書舍人;司議大夫,視諫議大夫;六官省,視尚書;東、西曹僕射,視左右僕射;御史臺曰執憲,置大夫至監察御史,驅使要籍官曰承令;左右將軍曰虎牙、豹略;軍使曰鷹揚、龍驤。以劉怦為範陽府留守,柳良器、李子千為左右內史,滔兄瓊瑰、陸慶為東、西曹僕射,楊霽、馬寔、寇瞻、楊榮國為司文、司武、司禮、司刑侍郎,李士真、樊播為執憲大夫、中丞。其餘以次補署。聘處士張遂、王道為司諫。



 燧遣李晟將兵至易、定,率張茂昭攻涿、莫,以絕滔援。明年,圍清苑,滔將鄭景濟固守。滔使馬寔將兵萬人,與武俊拒燧,自以兵萬餘救清苑,絕晟糧道。兵至定州,晟不知,夜引兵還。滔疑有伏,不敢逼,遽保瀛州。而孝忠、晟合兵千人城萊水,滔驍將烏薩戒以兵七百襲殺城卒數百,晟不出。景濟望滔軍立幟為應。滔進軍薄晟營,晟戰不利,城中兵亦出,晟大敗,奔易州。茂昭走滿城。滔已破晟,則回屯河間不進。武俊使宋端趣讓,滔怒曰:「孤亟戰且病,就醫藥,而王已復云云。孤南救魏,棄兄背君如脫屣。王必相疑,亦聽所為!」端還,武俊謂寔曰:「寡人望王速來指縱,決勝負,復何惡?王異日並天下,寡人得六七城,為節度足矣。」寔遣具道所以然,武俊亦遣使謝滔,滔悅,亦報謝。然武俊內銜之,滋不懌,與田悅潛謀絕滔。



 及泚反,燧等皆班師,武俊、寔亦還。悅、武俊遣使至河間,賀泚即位。武俊詭請寔共攻康日知於趙州,謀覆其軍,不克。實歸,武俊餞之,厚贈遺。泚遣人密召滔,使趨洛陽。滔發書,西向再拜,移檄諸道曰:「今發突騎四十萬走洛陽,與皇帝會上陽宮。」使王郅說悅連和俱西。滔素強調斂,武俊等不能堪。又令各以兵五千從攻洛,欲僭稱帝,乘輿、法從及赦令皆具。



 初,回紇以女妻奚王,大歷末,奚亂,殺王,女逃歸,道平盧,滔以錦繡張道,待其至,請為婚,女悅,許焉。既而遣使修婿禮於回紇,回紇喜,報以名馬重寶。及僭相王,與武俊、悅、納納四金鑰於回紇,曰:「四國願聽命於可汗,謹上金鑰,啟閉出納,唯所命。」至是,乞師焉。回紇以二千騎從,而武俊亦先乞師,以斷懷光餉路,未至,而王師還。回紇過幽州,滔使說其酋達干曰:「若能同度河而南,玉帛子女不貲,計可得也。」達干許諾,滔啖以金帛,約曰:「五十里舍,以須悅軍。」滔兵五萬,車千乘,騎二萬,士私屬萬餘,虜兵三千,馬、橐它倍之;過武俊境,武俊勞之,牛酒芻米皆具。然悅已用武俊謀,不肯出,儲峙於野以待。滔至貝州,悅刺史邢曹俊上謁滔,即歸閉城守,滔疑之,次永濟。武俊陰遣客反間滔曰:「悅有憾,須公南,以兵斷公歸路,宜少備。」滔聞怒,入永濟,執悅吏掠訊,不得其情,殺之。使回紇大掠,南及澶、衛,系執老幼無遺者。悅大恐,闔城自保。滔遣將楊布略定館陶,屯平恩,置官吏。



 滔整軍北還,使馬寔屯冠氏,聞悅死,遂攻魏州,圍貝州。於是,武俊、李抱真合軍擊滔。滔急召寔至貝州,步馬乏頓。明日,輒約戰,寔請休士三日,蔡雄、達干等畏武俊堅壁難圖,請戰。楊布曰:「大王將取東都,逢小敵即怯,何以長驅天下邪?」術士尹少伯亦言必勝。既戰,為二軍所乘,大敗,大將硃良祐、李進皆被執,委杖如丘,滔奔入德州。恨少伯、雄、布之謬,殺之。俄而京師平,滔已敗,不能軍,走還幽州,上書待罪。有詔武俊、抱真開示大信,若誠心審固者,當洗釁錄勛,與更始。



 初,滔以劉怦忠力,使留守,及敗,疑圖己,仿徨不敢入。怦聞其至,搜兵繕鎧,夾道陳二十里迎謁,望滔哭,滔遂入府。氣沮索,日邑邑,被病,政事一委怦。貞元元年死,年四十二,贈司徒。



 劉怦,幽州昌平人。少為範陽裨將,以親老疾宜侍,輒去職。李懷仙為節度使,檄召不應。硃滔時,積功至雄武軍使,廣墾田,節用度,以辦治稱。稍遷涿州刺史。滔之討田承嗣,表知府事,和裕得眾心。李寶臣以兵劫滔於瓦橋,滔走,寶臣乘勝欲襲幽州,怦設方略,勒兵完守,寶臣不敢謀,擢御史中丞。滔敗歸,終不貳,益治兵,人嘉怦忠於所奉。及滔死,軍中盡推怦,乃總軍事。俄詔為節度副大使、彭城郡公。居鎮才三月死,年五十九,贈兵部尚書,謚曰恭。子濟。



 濟,字濟。游學京師,第進士,歷莫州刺史。怦病,詔濟假州事。及怦卒,嗣節度,累遷檢校尚書右僕射、同中書門下平章事。奚數侵邊,濟擊走之,窮追千餘里,至青都山,斬首二萬級。其後又掠檀、薊北鄙,濟率軍會室韋,破之。



 王承宗叛,濟合諸將曰:「天子知我怨趙,必命我伐之,趙且大備我,奈何?」裨將譚忠欲激濟伐承宗,疾言曰:「天子不使我伐趙,趙亦不備燕。」濟怒,系之。使視趙,果不設備。數日,詔書許濟無出師。濟釋忠,謝而問之,忠曰:「昭義盧從史外親燕,內實忌之;外絕趙,內實與之。此為趙畫曰:『燕倚趙自固,雖甚怨,必不殘趙,故不足虞也。』趙既不備燕,從史則告天子曰:『燕、趙,宿怨也,今趙見伐而不備燕,是燕反與趙。』此所以知天子不使君伐趙,趙亦不備燕。」濟曰:「計安出?」曰:「今天子誅承宗,而燕無一卒濟易水者,正使潞人賣恩於趙,販忠於上,是君貯忠誼心,而染私趙之名,卒不見德於趙,惡聲徒嘈嘈於天下。」濟然之,以兵七萬先諸軍,斬首數千級,又拔饒陽,屯瀛州。進攻安平,久不拔,濟命次子總以兵八千先登,日中拔其城。會赦承宗,進中書令。



 濟之出,以長子緄攝留務,總為行營都知兵馬使。濟病甚,總與左右張、成國寶及帳內親近謀殺濟,乃使人詐從京師來,曰:「朝廷以公前屯瀛州逗留,詔副大使代節度。」明日,復使人曰:「詔節至太原矣。」又使人走呼曰:「過代矣。」舉軍驚。濟憤且怒,不知所為,誅主兵大將數十人及素與緄厚善者,亟追緄,以玨已兄皋代留事。濟自朝至中昃不食,渴索酏漿,總使吏唐弘實寘毒,濟飲而死,年五十四。緄至涿州,總矯濟命殺之。乃發喪,贈太師,謚曰莊武。



 總性陰賊,尤險譎,已毒父,即領軍政,朝廷不知其奸,故詔嗣節度,封楚國公,進累檢校司空。承宗再拒命,總遣兵取武強,按軍兩端,以私饋齎。憲宗知之,外示崇寵,進同中書門下平章事。及吳元濟、李師道平,承宗憂死,田弘正入鎮州,總失支助,大恐,謀自安。又數見父兄為崇,乃衣食浮屠數百人,晝夜祈禳,而總憩祠場則暫安,或居臥內,輒驚不能寐。晚年益慘悸,請剔發,衣浮屠服,欲祓除之。



 譚忠復說總曰:「天地之數,合必離,離必合。河北與天下離六十年,數窮必合。往硃泚、希烈自立,趙、冀、齊、魏稱王,郡國弄兵,低目相視,可謂危矣,然卒於無事。元和以來,劉闢、李錡、田季安、盧從史、齊、蔡之強,或首於都市,或身為逐客,皆君自見。今兵駸駸北來,趙人已獻德、棣十二城,助魏破齊,唯燕無一日勞,後世得無事乎?為君憂之。」總泣且謝,因上疏願奉朝請,且欲割所治為三;以幽、涿、營為一府,請張弘靖治之;瀛、莫為一府,盧士玫治之;平、薊、媯、檀為一府,薛平治之。盡籍宿將薦諸朝。



 會穆宗沖逸,宰相崔植、杜元穎無遠謀,欲寵弘靖,重其權,故全付總地,唯分瀛、莫置觀察使。拜總檢校司徒兼侍中、天平節度使。又賜浮屠服,號大覺,榜其第為佛祠,遣使者以節、印偕來。時總已自髡祝,讓節、印,遂衣浮屠服。行及定州,卒。



 始,總請代,獻馬萬五千匹,群臣或疑其詐,帝獨納之,使給事中薛存慶宣慰,給所部復一歲,緡錢百萬勞軍,高年惸獨不能自存者,官吏就問,賜粟帛。總遂與忠俱行,軍中世懷其惠,擁留不得進。總殺首謀者十人,以節付張皋,夜間道去,遲明,軍中乃知。



 詔贈太尉。子礎及弟約至長安者十一人,皆擢州刺史。忠護總喪至,亦卒。忠,絳人,喜兵,善謀事,蓋健男子云。



 硃克融,滔孫也。以偏校事劉總。總將入朝,慮後有變,籍其軍材勇與黠暴不制者,悉薦之朝,冀厚與爵位,使北方歆艷,無甘亂心,克融在遣。方是時,執政非其人,既見總納地,謂天下曠然無復事。克融等留京師,久之不得調,數詣宰相求自試,皆不聽,羸色敗服,饑寒無所貣丐,內怨忿。會張弘靖赴鎮,因悉遣還。



 俄幽州亂,囚弘靖。時克融父洄,號有智譎,以疾廢臥家,眾往請為帥。洄辭老且病,因推克融領軍務。詔以劉悟為節度使馳往,俄而瀛、冀皆附克融,悟不得入。克融縱兵掠易州,敗兩縣;寇蔚州,易州刺史柳公濟戰白石嶺,斬三千級;轉寇定州,節度使陳楚破其兵二萬。會鎮州反,殺田弘正,議者謂二賊均逆,而克融全弘靖不敢害,可悉兵先誅趙,赦燕。朝廷度幽薊未可復取,乃拜克融檢校左散騎常侍,為幽州盧龍節度使,長慶元年也。



 明年,陷弓高,攻下博,與王廷水奏共圍深州。裴度以檄譙諭,克融乃還,因進檢校工部尚書,表獻馬萬匹、羊十萬,請直賞軍。敬宗初,遷檢校司空,賜邊屯時服,克融以帛疏惡,囚詔使楊文端以聞。又上言:「聞陛下東幸雒,願率匠丁五千助營宮室,迎乘輿,且請帛三十萬,備一歲費。」帝怒,用裴度謀,忍不問,以好言答之,屈其謀,進爵吳興郡王。



 是年,軍亂,殺克融及其子延齡,詔贈司徒。次子延嗣立,領留後,為大將李載義殺而代之,並族其家。



 李載義,自稱恆山愍王之後。性矜蕩,好與豪傑游,力挽強搏斗。劉濟在幽州,高其能,引補帳下,從征伐,積多為牙中兵馬使。硃克融死,子延嗣叛命,殘用其人。載義因眾不忍,殺之,暴其罪於朝。敬宗即授檢校戶部尚書、盧龍軍節度使,封武威郡王。



 初,張弘靖之囚,幕府多見害,妻子留不遣。及是,載義悉護送京師,雖僮廝畢行。俄而李同捷據滄、景,邀襲封,載義請討賊自效,文宗嘉之,進檢校尚書右僕射。斬級數有功,賊平,詔同中書門下平章事,賜白玉帶,示殊禮。



 大和四年,為兵馬使楊志誠所逐,奔易州,即上言:「自破滄州賊,屢請朝不許,今願將妻子身入見。」帝令使者抵太原尉迎,賜袍笏裝器;又以其嘗有功,且意恭順,乃冊拜太保,仍平章事。俄為山南西道節度使。徙河東。


始,回鶻使者歲入朝,所過暴慢,吏不敢何禁,但嚴兵自守。虜忸習,益謷悍,至鞭候人,剽突市區。時大酋李暢者,曉華人語,尤兇黠。既就館,橫須索,抶
 \gezhu{
  疒只}
 郵人。載義召暢語曰:「可汗以舅甥故,使將軍朝貢,誼不容將軍暴也。天子厚饔餼以禮客,有不謹,吏皆論死。若將軍所部不戢,而奪攘自如,我必殺所犯者,將軍其少戒。」因悉罷所防兵,以兩卒護闔。暢嚴憚之,訖無犯者。進兼侍中。會吏下請立碑紀功,詔李程為之辭,未有字。帝詔曰:「《周書》『凡厥正人,既富方谷。』卿宜當之,以方谷為字。」其寵待如此。開成二年卒,年五十,贈太尉。



 初,載義母葬範陽,為楊志誠掘發。後志誠被逐,道太原,載義奏請剔其心,償母怨,不許。又欲殺之,官屬苦救乃免,然盡戕其妻息士卒,其天資驕暴雲,帝屈法弗劾也。



 志誠者,事載義為牙將。載義宴天子使者鞠場,志誠與其黨噪而起,載義走,因自為都知兵馬使。文宗更以嘉王領節度,用志誠為留後。俄檢校工部尚書,擢節度副大使。逾年,進檢校吏部。詔下,邸吏白宰相曰:「軍中不識朝廷儀,惟知尚書改僕射為進秩。今一府盛服以待天子命,如復為尚書,則舉軍慚,使者勢不得出。」既志誠果怨望,軍有謾言,囚中人魏寶義及它使焦奉鸞、尹士恭,而遣部將王文穎入謝,讓還所命。帝復賜之,文穎不肯受,輒去。帝忍不責,乃遣使進檢校尚書右僕射。



 八年,為下所逐,推部將史元忠總留後。志誠在鎮,密制天子袞冕,其被服皆擬乘輿。元忠表而暴於朝,詔御史按治,斥嶺南,至商州,誅之,而以通王領節度,授元忠留後。明年,檢校工部尚書,為副大使。會昌初,為偏將陳行泰所殺。行泰邀節制,未報。次將張絳殺行泰,起求帥軍,武宗自用張仲武代之。



 張仲武,範陽人。通《左氏春秋》。會昌初,為雄武軍使。行泰殺元忠,宰相李德裕計:河朔請帥,皆報下太速,故軍得以安,若少須下,且有變。帝許之,未報,果為絳所殺,復誘其軍以請,亦置未報。是時,回鶻為黠戛斯所破,烏介可汗托天德塞上,而仲武遣其屬吳仲舒入朝,請以本軍擊回鶻。德裕因問北方事,仲舒曰:「行泰、絳皆游客,人心不附。仲武,舊將張光朝子,年五十餘,通書,習戎事,性忠義,願歸款朝廷舊矣。」德裕曰:「即以為帥,軍得無復亂乎?」答曰:「仲武得士心,受命必有逐絳者。」德裕入白帝曰:「行泰等邀節不可許,仲武求自效,用之有名,軍且無辭。」乃擢兵馬留後,而詔撫王領節度。詔下,絳果為軍中所逐,即拜仲武副大使、檢校工部尚書、蘭陵郡公。會回鶻特勒那頡啜擁赤心部七千帳逼漁陽,仲武使其弟仲至與別將游奉寰等率銳兵三萬破之,獲馬、牛、橐它、旗纛不勝計,遣吏獻狀,進檢校兵部尚書。



 始,回鶻常有酋長監奚、契丹以督歲貢,因言冋刺中國。仲武使裨將石公緒等厚結二部,執諜者八百餘人殺之。回鶻欲入五原,掠保塞雜虜,乃先以宣門將軍四十七人詭好結歡,仲武賂其下,盡得所謀,因逗留不遣,使失師期,回鶻人馬多病死者,由是不敢犯五原塞。烏介失勢,往依康居,盡徙餘種,寄黑車子部。回鶻遂衰,名王貴種相繼降,捕幾千人。仲武表請立石以紀聖功,帝詔德裕為銘,揭碑盧龍,以告後世。大中初,又破奚北部及山奚,俘獲雜畜不貲。擢累檢校司徒、同中書門下平章事。卒,謚曰莊。



 子直方,以右金吾將軍襲節度留後,俄進副大使。舉動多不法,畏下變起,乃托出畋奔京師。軍中以張允伸總後務。直方至,宣宗遣使者郊勞,授金吾大將軍,以其族大,給檢校工部尚書俸。久之,進檢校尚書右僕射。性暴率,坐以小罪笞殺金吾使,改右羽林統軍。好馳獵,往往設罝罘於道。當宿衛不時入,下遷驍衛將軍。奴婢細過輒殺,積其罪,貶思州司戶參軍。母驚曰:「尚有尊於我子邪?」久乃復授羽林統軍。縱部下為盜,復貶康州司馬。後居東都,弋獵愈甚,洛陽飛鳥皆識之,見必群噪。乾符中,累進左驍衛大將軍。時鄭畋輔政,頗言:「仲武會昌時功第一,今直方百口不自存,每內燕,以衣敝惡,辭不赴。陛下錄功念舊,宜少優假。」詔還檢校右僕射,進左金吾衛大將軍。



 黃巢犯京師,直方迎灞上,既而納亡命,謀劫巢報天子,公卿多依之。賊覺,屠其族。



 張允伸字逢昌,範陽人。世為軍校。直方出奔,以都知兵馬使為眾立為留後,天子報可。未幾,檢校散騎常侍,為節度使,累進檢校司徒、兼太傅、同中書門下平章事,封燕國公。



 龐勛以徐州反,上書欲遣弟允皋領兵討賊,不許。上米五十萬斛、鹽二萬斛佐用度,詔嘉美,賜玉帶、寶器、紈錦,進兼侍中。咸通十二年,以疾甚,上節、印,便醫藥,詔聽許,以子簡會為副大使。卒,年八十八,贈太尉,謚曰忠烈。



 允伸性勤儉,下所安賴,未嘗有邊鄙虞。子十四人。簡會入朝,昆弟多至大將軍、刺史、郡佐者,而軍中推張公素為留後。



 公素,範陽人。以列將事允伸,擢累平州刺史。允伸卒,以兵來會喪,軍士素附其威望,簡會知不可制,即出奔。詔公素為節度使,進同中書門下平章事。性暴歷,眸子多白,燕人號「白眼相公」。為李茂勛所襲,奔京師,貶復州司戶參軍。



 李茂勛,本回鶻阿布思之裔。張仲武時,與其侯王皆降。資沈勇,善馳射,仲武器之,任以將兵,常乘邊積功,賜姓及名。陳貢言者,燕健將,為納降軍使,軍中素信服,茂勛襲殺之,因舉兵,紿稱貢言反。公素迎擊不利,走,茂勛入府,眾始悟,因推主州務,以聞,詔即拜節度使。俄以病自上,詔進尚書右僕射致仕。表子可舉代,遂領留後,進為節度使,擢累檢校太尉。



 中和末,太原李克用始強大,與定州王處存厚相結,可舉惡其窺山東為己患,乃遣使約吐渾都督赫連鐸、鎮州王鎔聯和,揚言易、定本燕、趙屬,得其地,且參有之。即遣軍司馬韓玄紹擊沙陀藥兒嶺,斬首七千級,殺其將硃耶盡忠等,收牛、馬、器鎧數萬。又戰雄武軍,殺獲萬人。鐸又破沙陀於蔚州,詔以鐸為雲州刺史,進可舉檢校侍中。乃遣票將李全忠率眾六萬圍易州。鎔以兵攻無極,處存求援太原,克用自將赴之,鎮人懼,退保新城,克用急攻之,鎔引去,追破之九門。易久未下,盧龍將劉仁恭穴地以入,得其城,士卒有驕色;處存以輕兵三千蒙羊皮,夜布之野,以精騎伏它道,全忠軍望為群羊,爭趨之,處存伏騎發,大敗之,復取易州。全忠遁還,盡失芻糧仗鎧,懼得罪,乃裒餘眾反攻幽州,可舉度不支,引其族登樓自燔死。



 李全忠,範陽人。仕為棣州司馬。有蘆生其室,一尺三節,怪之,以問別駕張建,建曰:「蘆,茅類,生於澤,公茅土兆也。傳節者其三世乎?」罷歸,事可舉為牙將。可舉死,眾推為留後。光啟元年,拜節度使,未幾卒。



 子匡威嗣,領留後,進為使。性豪爽,恃燕、薊勁兵處,軒然有雄天下意。與赫連鐸共攻太原,爭雲、代。李克用使安金俊攻鐸,匡威救鐸,戰蔚州,射金俊殺之,乃共表請討沙陀,而硃全忠亦上言願協力,故張浚因請用兵矣。浚敗,克用攻雲州,以騎將薛阿檀為前鋒,設伏河上。鐸以精騎追阿檀,抵河而伏起,乃大敗,禽其將賈塞兒,遂圍雲州,塹而守,分兵出井陘,屯常山,大掠深、趙。匡威以步騎萬餘援王鎔,克用還,因急攻鐸。會食盡,鐸棄州奔匡威。克用取雲州,表石善友為刺史。鐸本吐谷渾部酋也,開成中,其父率種人三千帳自歸,守雲州十五年。至是,失其地。



 景福初,鎔誘太原將李存孝降之,克用怒,伐鎔。鎔來求救,匡威遣將赴之,克用去。明年,兵復出井陘,匡威自將援鎔,將行,置酒大會。其弟兵馬留後、檢校司徒匡籌妻張,國艷,匡威酒酣,報之,弟怒,匡威軍次博野,乃據城自為留後。天子即授檢校太保,為節度使。匡威麾下多去,屏營無所歸,留深州,遣其屬李抱貞上書願入朝。時京師數寇難,人人危懼,傳言金頭王且來,皆亡竄山谷。抱貞還,而鎔已迎館於鎮。匡威引抱貞登城西大悲浮屠,顧望流涕,美其山川,乃共圖鎔。陽為鎔繕甲,治城塹,施授方略,陰施予,以傾士心。鎮軍忠於王氏,皆惡之。匡威親忌日,鎔過慰。匡威士衷甲劫鎔入牙城,戰不勝,鎮人斬匡威以徇。匡籌表訴諸朝,檄暴鎔罪,攻樂壽、武強以報。



 始,匡籌之奪也,燕人不以為義。劉仁恭出奔太原,克用倚其謀,下武、媯二州,敗匡籌於居庸關。李存審與戰,匡籌又敗,挈其族奔京師,次景城,滄州節度使盧彥威殺之,掠入車馬僮妓。妻方乳,不能進,仁恭獲之,納於克用為嬖夫人。始,匡威見逐,嘆曰:「兄失弟得,皆吾之宗,無所悔,然其材恐不足以守。」果亡,而幽州地歸克用,以仁恭為帥。



 劉仁恭,深州人。父晟,客範陽,為李可舉新興鎮將,故仁恭事軍中。從李全忠攻易州,號「窟頭」,稍遷裨校。為人豪縱,多智數,有大志,嘗自言:「夢大幡出指端,年四十九,當秉旄節。」李匡威惡之,補景城令。



 會瀛州亂,殺守吏,仁恭募士千人定其亂。匡威復使將兵,戍蔚州,逾期未代,士皆怨。會匡籌奪地,故戍卒擁仁恭趨幽州,匡籌逆戰,敗之,遂以族奔太原。李克用待之甚厚,賜田宅,拜壽陽鎮將。數以策干克用,請步騎一萬東取幽州,且為導。克用攻匡籌,匡籌遁去。仁恭與苻存審入城,封府庫以待。克用悅,留仁恭守之,以親信分典其兵。



 乾寧二年,克用擊王行瑜,表仁恭為檢校司空、盧龍軍節度使。明年,克用攻魏州,召盧龍兵,仁恭以契丹解。又明年,克用復興其兵救硃瑄,仁恭不答,使者數十往,卒不出。克用以書讓之,仁恭乃慢罵,執其使,盡囚太原士之在燕者。復以厚利誘克用麾下士,多亡歸之。克用怒,自將往擊,不勝,師喪過半。仁恭獻馘於硃全忠,全忠表同中書門下平章事。



 既與克用絕,則益募兵。光化初,使其子守文襲滄州,節度使盧彥威棄城走,遂有滄、景、德三州地,用守文為節度留後,請命於朝。昭宗怒,不與。會中人至,仁恭謾謂曰:「旄節吾自可為,要假長安本色耳,何見拒邪?」由是兵益張,顯圖河北。悉幽、滄步騎十萬,聲言三十萬,南徇魏、鎮。次貝州,屠之,清水為不流。



 羅紹威求救於硃全忠,全忠使李思安、葛從周赴之,屯內黃。仁恭負強,下令曰:「思安懦,當先破之,乃取魏。」守文與單可及精甲五萬,循清水上。思安設伏,自引兵逆戰,偽不勝。守文躡北至內黃,思安整兵還擊守文,伏發,斬可及,獨守文挺逸,眾無還者。從周興邢、洺兵與魏將賀德倫等出館陶門,夜擊仁恭,破八屯。仁恭走,自魏抵長河數百里,尸蔽道。鎮人邀敗之東境。仁恭遂衰。



 三年,葛從周攻滄州,仁恭壁乾寧。從周潛軍戰老鴉堤,仁恭敗,退壁瓦橋,卑辭歸窮於克用求救,克用為侵邢、洺。俄而全忠取瀛、莫,克用使周德威出飛狐。天祐三年,全忠自將攻滄州,壁長蘆。仁恭悉發男子十五以上為兵,涅其面曰「定霸都」,士人則涅於臂曰:「一心事主」,盧龍閭里為空,得眾二十萬,屯瓦橋。全忠環滄築而溝之,內外援絕,人相食。仁恭求戰,不許,復從克用乞師,使百輩往,乃許。仁恭以兵三萬合攻潞州,降全忠將丁會,滄州圍乃解。



 是時,中原方多故,仁恭得倚燕強且遠,無所憚,意自滿。從方士王若訥學長年,築館大安山,掠子女充之。又招浮屠,與講法。以堇土為錢,斂真錢,穴山藏之,殺匠滅口。禁南方茶,自擷山為茶,號山曰大恩,以邀利。



 子守光烝嬖妾,事覺,仁恭謫之。李思安來攻,屯石子河。仁恭居大安山,城中無備。守光引兵出戰,思安去,因回攻大安,虜仁恭,囚別室,殺左右婢媵,遂有盧龍。



 贊曰:硃滔脅其兄泚入朝,及引兵東向,稱帝以自尊,名雖助泚,志可知矣。至克融再得幽州,硃氏無遺種,其禍與泚鈞,而族夷有先後為間也。



\end{pinyinscope}