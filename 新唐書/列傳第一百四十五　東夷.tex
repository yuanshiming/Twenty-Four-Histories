\article{列傳第一百四十五 東夷}

\begin{pinyinscope}

 高麗,本扶餘別種也。地東跨海距新羅,南亦跨海距百濟,西北度遼水與營州接,北靺鞨。其君居平壤城說它「明理一而分殊」。並說:「萬物皆是一理」,「一物之理,亦謂長安城,漢樂浪郡也,去京師五千里而贏,隨山屈繚為郛,南涯浿水,王築宮其左。又有國內城、漢城,號別都。水有大遼、少遼:大遼出靺鞨西南山,南歷安市城;少遼出遼山西,亦南流,有梁水出塞外,西行與之合。有馬訾水出靺鞨之白山,色若鴨頭,號鴨淥水,歷國內城西,與鹽難水合,又西南至安市,入於海。而平壤在鴨淥東南,以巨艫濟人,因恃以為塹。



 官凡十二級:曰大對廬,或曰吐捽;曰鬱折,主圖簿者;曰太大使者;曰帛衣頭大兄,所謂帛衣者,先人也,秉國政,三歲一易,善職則否,凡代日,有不服則相攻,王為閉宮守,勝者聽為之;曰大使者;曰大兄;曰上位使者;曰諸兄;曰小使者;曰過節;曰先人;曰古鄒大加。其州縣六十。大城置傉薩一,比都督;餘城置處閭近支,亦號道使,比刺史。有參佐,分干。有大模達,比衛將軍;末客,比中郎將。



 分五部:曰內部,即漢桂婁部也,亦號黃部;曰北部,即絕奴部也,或號後部;曰東部,即順奴部也,或號左部;曰南部,即灌奴部也,亦號前部;曰西部,即消奴部也。



 王服五採,以白羅制冠,革帶皆金扣。大臣青羅冠,次絳羅,珥兩鳥羽,金銀雜扣,衫筒袖,褲大口,白韋帶,黃革履。庶人衣褐,戴弁。女子首巾幗。俗喜弈、投壺、蹴鞠。食用籩、豆、簠、簋、罍、洗。居依山谷,以草茨屋,惟王宮、官府、佛廬以瓦。窶民盛冬作長坑,煴火以取暖。其治,峭法以繩下,故少犯。叛者叢炬灼體,乃斬之,籍入其家。降、敗、殺人及剽劫者斬,盜者十倍取償,殺牛馬者沒為奴婢,故道不掇遺。婚娶不用幣,有受者恥之。服父母喪三年,兄弟逾月除。俗多淫祠,禮靈星及日、箕子、可汗等神。國左有大穴曰神隧,每十月,王皆自祭。人喜學,至窮里廝家,亦相矜勉,衢側悉構嚴屋,號局堂,子弟未婚者曹處,誦經習射。



 隋末,其王高元死,異母弟建武嗣。武德初,再遣使入朝。高祖下書脩好,約高麗人在中國者護送,中國人在高麗者敕遣還。於是建武悉搜亡命歸有司,且萬人。後三年,遣使者拜為上柱國、遼東郡王、高麗王。命道士以像法往,為講《老子》。建武大悅,率國人共聽之,日數千人。帝謂左右曰:「名實須相副。高麗雖臣於隋,而終拒煬帝,何臣之為?朕務安人,何必受其臣?」裴矩、溫彥博諫曰:「遼東本箕子國,魏晉時故封內,不可不臣。中國與夷狄,猶太陽於列星,不可以降。」乃止。明年,新羅、百濟上書,言建武閉道,使不得朝,且數侵入。有詔散騎侍郎硃子奢持節諭和,建武謝罪,乃請與二國平。太宗已禽突厥頡利,建武遣使者賀,並上封域圖。帝詔廣州司馬長孫師臨瘞隋士戰胔,毀高麗所立京觀。建武懼,乃築長城千里,東北首扶餘,西南屬之海。久之,遣太子桓權入朝獻方物,帝厚賜齎,詔使者陳大德持節答勞,且觀亹。大德入其國,厚餉官守,悉得其纖曲。見華人流客者,為道親戚存亡,人人垂涕,故所至士女夾道觀。建武盛陳兵見使者。大德還奏,帝悅。大德又言:「聞高昌滅,其大對廬三至館,有加禮焉。」帝曰:「高麗地止四郡,我發卒數萬攻遼東,諸城必救,我以舟師自東萊帆海趨平壤,固易。然天下甫平,不欲勞人耳。」



 有蓋蘇文者,或號蓋金,姓泉氏,自雲生水中以惑眾。性忍暴。父為東部大人、大對廬,死。蓋蘇文當嗣,國人惡之,不得立。頓首謝眾,請攝職,有不可,雖廢無悔。眾哀之,遂嗣位。殘兇不道,諸大臣與建武議誅之,蓋蘇文覺,悉召諸部,紿云大閱兵,列饌具請大臣臨視,賓至盡殺之,凡百餘人。馳入宮殺建武,殘其尸投諸溝。更立建武弟之子藏為王,自為莫離支,專國,猶唐兵部尚書、中書令職云。貌魁秀,美須髯,冠服皆飾以金,佩五刀,左右莫敢仰視。使貴人伏諸地,踐以升馬。出入陳兵,長呼禁切,行人畏竄,至投坑谷。



 帝聞建武為下所殺,惻然遣使者持節吊祭。或勸帝可遂討之,帝不欲因喪伐罪,乃拜藏為遼東郡王、高麗王。帝曰:「蓋蘇文殺君攘國,朕取之易耳,不願勞人,若何?」司空房玄齡曰:「陛下士勇而力有餘,戢不用,所謂『止戈為武』者。」司徒長孫無忌曰:「高麗無一介告難,宜賜書安尉之,隱其患,撫其存,彼當聽命。」帝曰:「善。」



 會新羅遣使者上書言:「高麗、百濟聯和,將見討。謹歸命天子。」帝問:「若何而免?」使者曰:「計窮矣,惟陛下哀憐!」帝曰:「我以偏兵率契丹、靺鞨入遼東,而國可紓一歲,一策也。我以絳袍丹幟數千賜而國,至,建以陣,二國見,謂我師至,必走,二策也。百濟恃海,不修戎械,我以舟師數萬襲之;而國女君,故為鄰侮,我以宗室主而國,待安則自守之,三策也。使者計孰取?」使者不能對。於是遣司農丞相裏玄獎以璽書讓高麗,且使止勿攻。使未至,而蓋蘇文已取新羅二城。玄獎諭帝旨,答曰:「往隋見侵,新羅乘邅奪我地五百里,今非盡反地,兵不止。」玄獎曰:「往事烏足論邪?遼東故中國郡縣,天子且不取,高麗焉得違詔?」不從。玄獎還奏,帝曰:「莫離支殺君,虐用其下如擭阱,怨痛溢道,我出師無名哉?」諫議大夫褚遂良曰:「陛下之兵度遼而克固善,萬分一不得逞,且再用師,再用師,安危不可億。」兵部尚書李勣曰:「不然。曩薛延陀盜邊,陛下欲追擊,魏徵苦諫而止。向若擊之,一馬不生返。後復畔擾,至今為恨。」帝曰:「誠然。但一慮之失而尤之,後誰為我計者?」新羅數請援,乃下吳船四百柁輸糧,詔營州都督張儉等發幽、營兵及契丹、奚、靺鞨等出討。會遼溢,師還。莫離支懼,遣使者內金,帝不納。使者又言:「莫離支遣官五十入宿衛。」帝怒責使者曰:「而等委質高武,而不伏節死義,又為逆子謀,不可赦。」悉下之獄。



 於是帝欲自將討之,召長安耆老勞曰:「遼東故中國地,而莫離支賊殺其主,朕將自行經略之,故與父老約:子若孫從我行者,我能拊循之,毋庸恤也。」即厚賜布粟。群臣皆勸帝毋行,帝曰:「吾知之矣,去本而就末,舍高以取下,釋近而之遠,三者為不祥,伐高麗是也。然蓋蘇文弒君,又戮大臣以逞,一國之人延頸待救,議者顧未亮耳。」於是北輸粟營州,東儲粟古大人城。帝幸洛陽,乃以張亮為平壤道行軍大總管,常何、左難當副之,冉仁德、劉英行、張文乾、龐孝泰、程名振為總管,帥江、吳、京、洛募兵凡四萬,吳艘五百,泛海趨平壤。以李勣為遼東道行軍大總管,江夏王道宗副之,張士貴、張儉、執失思力、契苾何力、阿史那彌射、姜德本、曲智盛、吳黑闥為行軍總管隸之,帥騎士六萬趨遼東。詔曰:「朕所過,營頓毋飭,食毋豐怪,水可涉者勿作橋梁,行在非近州縣不得令學生、耆老迎謁。朕昔提戈撥亂,無盈月儲,猶所響風靡。今幸家給人足,只恐勞於轉餉,故驅牛羊以飼軍。且朕必勝有五:以我大擊彼小,以我順討彼逆,以我安乘彼亂,以我逸敵彼勞,以我悅當彼怨,渠憂不克邪!」又發契丹、奚、新羅、百濟諸君長兵悉會。



 十九年二月,帝自洛陽次定州,謂左右曰:「今天下大定,唯遼東未賓,後嗣因士馬盛強,謀臣導以征討,喪亂方始,朕故自取之,不遺後世憂也。」帝坐城門,過兵,人人撫慰,疾病者親視之,敕州縣治療,士大悅。長孫無忌白奏:「天下符魚悉從,而宮官止十人,天下以為輕神器。」帝曰:「士度遼十萬,皆去家室。朕以十人從,尚恧其多,公止勿言!」帝身屬橐房,結兩箙於鞍。四月,勣績濟遼水,高麗皆嬰城守。帝大饗士,帳幽州之南,詔長孫無忌誓師,乃引而東。



 勣攻蓋牟城,拔之,得戶二萬,糧十萬石,以其地為蓋州。程名振攻沙卑城,夜入其西,城潰,虜其口八千,游兵鴨淥上。勣遂圍遼東城。帝次遼澤,詔瘞隋戰士露骼。高麗發新城、國內城騎四萬救遼東。道宗率張乂君逆戰,君乂卻。道宗以騎馳之,虜兵闢易,奪其梁,收散卒,高以望,見高麗陣囂,急擊破之,斬首千餘級,誅君乂以徇。帝度遼水,徹杠彴,堅士心。營馬首山,身到城下,見士填塹,分負之,重者馬上持之』群臣震懼,爭挾塊以進。城有硃蒙祠,祠有鎖甲、銛矛,妄言前燕世天所降。方圍急,飾美女以婦神,誣言硃蒙悅,城必完。勣列拋車,飛大石過三百步,所當輒潰,虜積木為樓,結絙罔,不能拒。以沖車撞陴屋,碎之。時百濟上金旐鎧,又以玄金為山五文鎧,士被以從。帝與勣會,甲光炫日。會南風急,士縱火焚西南,熛延城中,屋幾盡,人死於燎者萬餘。眾登陴,虜蒙盾以拒,士舉長矛舂之,藺石如雨,城遂潰,獲勝兵萬,戶四萬,糧五十萬石。以其地為遼州。初,帝自太子所屬行在,舍置一烽,約下遼東舉烽,是日傳燎入塞。



 進攻白崖城,城負山有厓水,險甚。帝壁西北。虜酋孫伐音陰丐降,然城中不能一。帝賜幟曰:「若降,建於堞以信。」俄而舉幟,城人皆以唐兵登矣,乃降。初,伐音中悔,帝怒,約以虜口畀諸將。及是,李勣曰:「士奮而先,貪虜獲也。今城危拔,不可許降以孤士心。」帝曰:「將軍言是也。然縱兵殺戮,略人妻孥,朕不忍。將軍麾下有功者,朕能以庫物賞之,庶因將軍贖一城乎。」獲男女凡萬、兵二千。以其地為巖州,拜伐音為刺史。莫離支以加尸人七百戍蓋牟,勣俘之。請自效,帝曰:「而家加尸,乃為我戰,將盡戮矣。夷一姓求一人力,不可。」稟而縱之。



 次安市。於是高麗北部傉薩高延壽、南部傉薩高惠真引兵及靺鞨眾十五萬來援。帝曰:「彼若勒兵連安市而壁,據高山,取城中粟食之,縱靺鞨略吾牛馬,攻之不可下,此上策也。拔城夜去,中策也。與吾爭鋒,則禽矣。」有大對廬為延壽計曰:「吾聞中國亂,豪雄並奮,秦王神武,敵無堅,戰無前,遂定天下,南面而帝,北狄、西戎罔不臣。今掃地而來,謀臣重將皆在,其鋒不可校。今莫若頓兵曠日,陰遣奇兵絕其餉道,不旬月糧盡,欲戰不得,歸則無路,乃可取也。」延壽不從,引軍距安市四十里而屯。帝曰:「虜墮吾策中矣。」命左衛大將軍阿史那社爾以突厥千騎嘗之,虜常以靺鞨銳兵居前,社爾兵接而北。延壽曰:「唐易與耳。」進一舍,倚麓而陣。帝詔延壽曰:「我以爾有強臣賊殺其主,來問罪,即交戰,非我意。」延壽謂然,按甲俟。帝夜召諸將,使李勣率步騎萬五千陣西嶺當賊,長孫無忌、牛進達精兵萬人出虜背狹谷,帝以騎四千偃幟趨虜北山上,令諸軍曰:「聞鼓聲而縱。」張幄朝堂,曰:「明日日中,納降虜於此。」是夜,流星墮延壽營。旦日,虜視勣軍少,即戰。帝望無忌軍塵上,命鼓角作,兵幟四合,虜惶惑,將分兵御之,眾已囂。勣以步槊擊敗之,無忌乘其後,帝自山馳下,虜大亂,斬首二萬級。延壽收餘眾負山自固,無忌、勣合圍之,徹川梁,斷歸路。帝按轡觀虜營壘曰:「高麗傾國來,一麾而破,天贊我也。」下馬再拜,謝水兄於天。延壽等度勢窮,即舉眾降。入轅門,膝而前,拜手請命。帝曰:「後敢與天子戰乎?」惶汗不得對。帝料酋長三千五百人,悉官之,許內徙,餘眾三萬縱還之,誅靺鞨三千餘人,獲馬牛十萬,明光鎧萬領。高麗震駭,後黃、銀二城自拔去,數百里無舍煙。乃驛報太子,並賜諸臣書曰:「朕自將若此,云何?」因號所幸山為駐蹕山,圖破陣狀,勒石紀功。拜延壽鴻臚卿,惠真司農卿。候騎獲覘人,帝解其縛,自言不食且三日,命飼之,賜以屩,遣曰:「歸語莫離支,若須軍中進退,可遣人至吾所。」帝每營不作塹壘,謹斥候而已,而士運糧,雖單騎,虜不敢鈔。



 帝與勣議所攻,帝曰:「吾聞安市地險而眾悍,莫離支擊不能下,因與之。建安恃險絕,粟多而士少,若出其不意攻之,不相救矣。建安得,則安市在吾腹中。」勣曰:「不然。積糧遼東,而西擊建安,賊將梗我歸路,不如先攻安市。」帝曰:「善。」遂攻之,未能下。延壽、惠真謀曰:「烏骨城傉薩已耄,朝攻而夕可下。烏骨拔,則平壤舉矣。」群臣亦以張亮軍在沙城,召之一昔至,若取烏骨,度鴨淥,迫其腹心,計之善者。無忌曰:「天子行師不徼幸。安市眾十萬在吾後,不如先破之,乃驅而南,萬全勢也。」乃止。城中見帝旌麾,輒乘陴噪,帝怒。勣請破日男子盡誅。虜聞,故死戰。江夏王道宗築距闉攻東南,虜增陴以守。勣攻其西,撞車所壞,隨輒串柵為樓。帝聞城中雞彘聲,曰:「圍久,突無黔煙。今雞彘鳴,必殺以饗士,虜且夜出。」詔嚴兵。丙夜,虜數百人縋而下,悉禽之。道宗以樹枚裹土積之,距闉成,迫城不數丈,果毅都尉傅伏愛守之,自高而排其城,城且頹,伏愛私去所部,虜兵得自頹城出,據而塹斷之,積火縈盾固守。帝怒,斬伏愛,敕諸將擊之,三日不克。



 有詔班師,拔遼、蓋二州之人以歸。兵過城下,城中屏息偃旗,酋長登城再拜,帝嘉其守,賜絹百匹。遼州粟尚十萬斛,士取不能盡。帝至渤錯水,阻淖,八十里車騎不通。長孫無忌、楊師道等率萬人斬樵築道,聯車為梁,帝負薪馬上助役。十月,兵畢度,雪甚,詔屬燎以待濟。始行,士十萬,馬萬匹;逮還,物故裁千餘,馬死十八。船師七萬,物故亦數百。詔集戰骸葬柳城,祭以太牢,帝臨哭,從臣皆流涕。帝總飛騎入臨渝關,皇太子迎道左。初,帝與太子別,御褐袍,曰:「俟見爾乃更。」袍歷二時弗易,至穿穴。群臣請更服,帝曰:「士皆敝衣,吾可新服邪?」及是,太子進潔衣,乃御。遼降口萬四千,當沒為奴婢,前集幽州,將分賞士。帝以父子夫婦離析,詔有司以布帛贖之,原為民,列拜歡舞,三日不息。延壽既降,以憂死,獨惠真至長安。



 明年春,藏遣使者上方物,且謝罪;獻二姝口,帝敕還之,謂使者曰:「色者人所重,然愍其去親戚以傷乃心,我不取也。」初,師還,帝以弓服賜蓋蘇文,受之,不遣使者謝,於是下詔削棄朝貢。



 又明年三月,詔左武衛大將軍牛進達為青丘道行軍大總管,右武衛將軍李海岸副之,自萊州度海;李勣為遼東道行軍大總管,右武衛將軍孫貳朗、右屯衛大將軍鄭仁泰副之,率營州都督兵,繇新城道以進。次南蘇、木底,虜兵戰不勝,焚其郛。七月,進達等取石城,進攻積利城,斬級數千,乃皆還。藏遣子莫離支高任武來朝,因謝罪。



 二十二年,詔右武衛大將軍薛萬徹為青丘道行軍大總管,右衛將軍裴行方副之,自海道入。部將古神感與虜戰曷山,虜潰;虜乘暝襲我舟,伏兵破之。萬徹度鴨淥,次泊灼城,拒四十里而舍。虜懼,皆棄邑居去。大酋所夫孫拒戰,萬徹擊斬之,遂圍城,破其援兵三萬,乃還。帝與長孫無忌計曰:「高麗困吾師之入,戶亡耗,田歲不收,蓋蘇文築城增陴,下饑臥死溝壑,不勝敝矣。明年以三十萬眾,公為大總管,一舉可滅也。」乃詔劍南大治船,蜀人願輸財江南,計直作舟,舟取縑千二百。巴、蜀大騷,邛、眉、雅三州獠皆反,發隴西、峽內兵二萬擊定之。始,帝決取虜,故詔陜州刺史孫伏伽、萊州刺史李道裕儲糧械於三山浦、烏胡島,越州都督治大艎偶舫以待。會帝崩,乃皆罷。藏遣使者奉慰。



 永徽五年,藏以靺鞨兵攻契丹,戰新城。大風,矢皆還激,為契丹所乘,大敗。契丹火野復戰,人死相藉,積尸而塚之。遣使者告捷,高宗為露布於朝。六年,新羅訴高麗、靺鞨奪三十六城,惟天子哀救。有詔營州都督程名振、左衛中郎將蘇定方率師討之。至新城,敗高麗兵,火外郛及墟落,引還。顯慶三年,復遣名振率薛仁貴攻之,未能克。後二年,天子已平百濟,乃以左驍衛大將軍契苾力何、右武衛大將軍蘇定方、左驍衛將軍劉伯英率諸將出浿江、遼東、平壤道討之。龍朔元年,大募兵,拜置諸將,天子欲自行,蔚州刺史李君球建言:「高麗小醜,何至傾中國事之?有如高麗既滅,必發兵以守,少發則威不振,多發人不安,是天下疲於轉戍。臣謂徵之未如勿征,滅之未如勿滅。」亦會武後苦邀,帝乃止。八月,定方破虜兵於浿江,奪馬邑山,遂圍平壤。明年,龐孝泰以嶺南兵壁蛇水,蓋蘇文攻之,舉軍沒;定方解而歸。



 乾封元年,藏遣子男福從天子封泰山,還而蓋蘇文死,子男生代為莫離支,有弟男建、男產相怨。男生據國內城,遣子獻誠入朝求救,蓋蘇文弟凈土亦請割地降。乃詔契苾何力為遼東道安撫大使,左金吾衛將軍龐同善、營州都督高偘為行軍總管,左武衛將軍薛仁貴、左監門將軍李謹行殿而行。九月,同善破高麗兵,男生率師來會。詔拜男生特進、遼東大都督兼平壤道安撫大使,封玄菟郡公。又以李勣為遼東道行軍大總管兼安撫大使,與契苾何力、龐同善並力。詔獨孤卿云由鴨淥道,郭待封積利道,劉仁願畢列道,金待問海谷道,並為行軍總管,受勣節度;轉燕、趙食廥遼東。明年正月,勣引道次新城,合諸將謀曰:「新城,賊西鄙,不先圖,餘城未易下。」遂壁西南山臨城,城人縛戍酋出降。勣進拔城十有六。郭待封以舟師濟海,趨平壤。三年二月,勣率仁貴拔扶餘城,它城三十皆納款。同善、偘守新城,男建遣兵襲之,仁貴救偘,戰金山,不勝。高麗鼓而進,銳甚。仁貴橫擊,大破之,斬首五萬級,拔南蘇、木底、蒼巖三城,引兵略地,與勣會。侍御史賈言忠計事還,帝問軍中云何。對曰:「必克。昔先帝問罪,所以不得志者,虜未有邅也。諺曰:『軍無媒,中道回』。今男生兄弟鬩很,為我鄉導,虜之情偽,我盡知之,將忠士力,臣故曰必克。且高麗祕記曰:『不及九百年,當有八十大將滅之。』高氏自漢有國,今九百年,勣年八十矣。虜仍薦饑,人相掠賣,地震裂,狼狐入城,分穴於門,人心危駭,是行不再舉矣。」



 男建以兵五萬襲扶餘,勣破之薩賀水上,斬首五千級,俘口三萬,器械牛馬稱之。進拔大行城。劉仁願與勣會,後期,召還當誅,赦流姚州。契苾何力會勣軍於鴨淥,拔辱夷城,悉師圍平壤。九月,藏遣男產率首領百人樹素幡降,且請入朝,勣以禮見。而男建猶固守,出戰數北。大將浮屠信誠遣諜約內應。五日,闔啟,兵噪而入,火其門,鬱焰四興,男建窘急,自刺不殊。執藏、男建等,收凡五部百七十六城,戶六十九萬。詔勣便道獻俘昭陵,凱而還。十二月,帝坐含元殿,引見勣等,數俘於廷。以藏素脅制,赦為司平太常伯,男產司宰少卿;投男建黔州,百濟王扶餘隆嶺外;以獻誠為司衛卿,信誠為銀青光祿大夫,男生右衛大將軍,何力行左衛大將軍,勣兼太子太師,仁貴威衛大將軍。剖其地為都督府者九,州四十二,縣百。復置安東都護府,擢酋豪有功者授都督、刺史、令,與華官參治。仁貴為都護,總兵鎮之。是歲郊祭,以高麗平,謝成於天。



 總章二年,徙高麗民三萬於江淮、山南。大長鉗牟岑率眾反,立藏外孫安舜為主。詔高偘東州道,李謹行燕山道,並為行軍總管討之,遣司平太常伯楊昉綏納亡餘。舜殺鉗牟岑走新羅。偘徙都護府治遼東州,破叛兵於安市,又敗之泉山,俘新羅援兵二千。李謹行破之於發廬河,再戰,俘馘萬計。於是平壤痍殘不能軍,相率奔新羅,凡四年乃平。始,謹行留妻劉守伐奴城,虜攻之,劉擐甲勒兵守,賊引去。帝嘉之,封燕郡夫人。



 儀鳳二年,授藏遼東都督,封朝鮮郡王,還遼東以安餘民,先編僑內州者皆原遣,徙安東都護府於新城。藏與靺鞨謀反,未及發,召還放邛州,廝其人於河南、隴右,弱窶者留安東。藏以永淳初死,贈衛尉卿,葬頡利墓左,樹碑其阡。舊城往往入新羅,遺人散奔突厥、靺鞨,由是高氏君長皆絕。垂拱中,以藏孫寶元為朝鮮郡王。聖歷初,進左鷹揚衛大將軍,更封忠誠國王,使統安東舊部,不行。明年,以藏子德武為安東都督,後稍自國。至元和末,遣使者獻樂工云。


百濟,扶餘別種也。直京師東六千里而贏,濱海之陽,西界越州,南倭,北高麗,皆逾海乃至,其東,新羅也。王居東、西二城,官有內臣佐平者宣納號令,內頭佐平主帑聚,內法佐平主禮,衛士佐平典衛兵,朝廷佐平主獄,兵官佐平掌外兵。有六方,方統十郡。大姓有八:沙氏,燕氏,[B125]氏,解氏,貞氏,國氏,木氏,
 \gezhu{
  艸曰}
 氏。其法:反逆者誅,籍其家;殺人者,輸奴婢三贖罪;吏受賕及盜,三倍償,錮終身。俗與高麗同。有三島,生黃漆,六月刺取沈,色若金。王服大袖紫袍,青錦褲,素皮帶,烏革履,烏羅冠飾以金花。群臣絳衣,飾冠以銀花。禁民衣絳紫。有文籍,紀時月如華人。



 武德四年,王扶餘璋始遣使獻果下馬,自是數朝貢。高祖冊為帶方郡王、百濟王。後五年,獻明光鎧,且訟高麗梗貢道。太宗貞觀初,詔使者平其怨。又與新羅世仇,數相侵,帝賜璽書曰:「新羅,朕蕃臣,王之鄰國。聞數相侵暴,朕已詔高麗、新羅申和,王宜忘前怨,識朕本懷。」璋奉表謝,然兵亦不止。再遣使朝,上鐵甲雕斧,帝優勞之,賜帛段三千。十五年,璋死,使者素服奉表曰:「君外臣百濟王扶餘璋卒。」帝為舉哀玄武門,贈光祿大夫,賻賜甚厚。命祠部郎中鄭文表冊其子義慈為柱國,紹王。



 義慈事親孝,與兄弟友,時號「海東曾子」。明年,與高麗連和伐新羅,取四十餘城,發兵守之。又謀取棠項城,絕貢道。新羅告急,帝遣司農丞相裏玄獎齎詔書諭解。聞帝新討高麗,乃間取新羅七城;久之,又奪十餘城,因不朝貢。高宗立,乃遣使者來,帝詔義慈曰:「海東三國,開基舊矣,地固犬牙。比者隙爭侵校無寧歲,新羅高城重鎮皆為王並,歸窮於朕,丐王歸地。昔齊桓一諸侯,尚存亡國,況朕萬方主,可不恤其危邪?王所兼城宜還之,新羅所俘亦畀還王。不如詔者,任王決戰,朕將發契丹諸國,度遼深入。王可思之,無後悔!」



 永徽六年,新羅訴百濟、高麗、靺鞨取北境三十城。顯慶五年,乃詔左衛大將軍蘇定方為神丘道行軍大總管,率左衛將軍劉伯英、右武衛將軍馮士貴、左驍衛將軍龐孝泰發新羅兵討之,自城山濟海。百濟守熊津口,定方縱擊,虜大敗。王師乘潮帆以進,趨真都城一舍止。虜悉眾拒,復破之,斬首萬餘級,拔其城。義慈挾太子隆走北鄙,定方圍之。次子泰自立為王,率眾固守,義慈孫文思曰:「王、太子固在,叔乃自王,若唐兵解去,如我父子何?」與左右縋而出,民皆從之,泰不能止。定方令士超堞立幟,泰開門降,定方執義慈、隆及小王孝演、酋長五十八人送京師,平其國五部、三十七郡、二百城,戶七十六萬。乃析置熊津、馬韓、東明、金漣、德安五都督府,擢酋渠長治之。命郎將劉仁願守百濟城,左衛郎將王文度為熊津都督。九月,定方以所俘見,詔釋不誅。義慈病死,贈衛尉卿,許舊臣赴臨,詔葬孫皓、陳叔寶墓左,授隆司稼卿。文度濟海卒,以劉仁軌代之。



 璋從子福信嘗將兵,乃與浮屠道琛據周留城反,迎故王子扶餘豐於倭,立為王。西部皆應,引兵圍仁願。龍朔元年,仁軌發新羅兵往救,道琛立二壁熊津江,仁軌與新羅兵夾擊之,奔入壁,爭梁墮溺者萬人,新羅兵還。道琛保任孝城,自稱領軍將軍,福信稱霜岑將軍,告仁軌曰:「聞唐與新羅約,破百濟,無老孺皆殺之;畀以國。我與受死,不若戰。」仁軌遣使齎書答說,道琛倨甚,館使者於外,嫚報曰:「使人官小,我,國大將,禮不當見。」徒遣之。仁軌以眾少,乃休軍養威,請合新羅圖之。福信俄殺道琛,並其兵,豐不能制。二年七月,仁願等破之熊津,拔支羅城,夜薄真峴,比明入之,斬首八百級,新羅餉道乃開。仁願請濟師,詔右威衛將軍孫仁師為熊津道行軍總管,發齊兵七千往。福信顓國,謀殺豐;豐率親信斬福信,與高麗、倭連和。仁願已得齊兵,士氣振,乃與新羅王金法敏率步騎,而遣劉仁軌率舟師,自熊津江偕進,趨周留城。豐眾屯白江口,四遇皆克,火四百艘;豐走不知所在。偽王子扶餘忠勝、忠志率殘眾及倭人請命,諸城皆復。仁願勒軍還,留仁軌代守。



 帝以扶餘隆為熊津都督,俾歸國,平新羅故憾,招還遺人。麟德二年,與新羅王會熊津城,刑白馬以盟。仁軌為盟辭曰:「往百濟先王,罔顧逆順,不敦鄰,不睦親,與高麗、倭共侵削新羅,破邑屠城。天子憐百姓無辜,命行人脩好,先王負險恃遐,侮慢弗恭。皇赫斯怒,是伐是夷。但興亡繼絕,王者通制,故立前太子隆為熊津都督,守其祭祀,附杖新羅,長為與國,結好除怨,恭天子命,永為籓服。右威衛將軍魯城縣公仁願,親臨厥盟,有貳其德,興兵動眾,明神監之,百殃是降,子孫不育,社稷無守,世世毋敢犯。」乃作金書鐵契,藏新羅廟中。



 仁願等還,隆畏眾攜散,亦歸京師。儀鳳時,進帶方郡王,遣歸籓。是時,新羅強,隆不敢入舊國,寄治高麗死。武後又以其孫敬襲王,而其地已為新羅、渤海靺鞨所分,百濟遂絕。



 新羅,弁韓苗裔也。居漢樂浪地,橫千里,縱三千里,東拒長人,東南日本,西百濟,南瀕海,北高麗。而王居金城,環八里所,衛兵三千人。謂城為侵牟羅,邑在內曰喙評,外曰邑勒。有喙評六,邑勒五十二。朝服尚白,好祠山神。八月望日,大宴齎官吏,射。其建官,以親屬為上,其族名第一骨、第二骨以自別。兄弟女、姑、姨、從姊妹,皆聘為妻。王族為第一骨,妻亦其族,生子皆為第一骨,不娶第二骨女,雖娶,常為妾媵。官有宰相、侍中、司農卿、太府令,凡十有七等,第二骨得為之。事必與眾議,號「和白」,一人異則罷。宰相家不絕祿,奴僮三千人,甲兵牛馬豬稱之。畜牧海中山,須食乃射。息穀米於人,償不滿,庸為奴婢。王姓金,貴人姓樸,民無氏有名。食用柳杯若銅、瓦。元日相慶,是日拜日月神。男子褐褲。婦長襦,見人必跪,則以手據地為恭。不粉黛,率美發以繚首,以珠彩飾之。男子翦發鬻,冒以黑巾。市皆婦女貿販。冬則作灶堂中,夏以食置冰上。畜無羊,少驢、■,多馬。馬雖高大,不善行。



 長人者,人類長三丈,鋸牙鉤爪,黑毛覆身,不火食,噬禽獸,或搏人以食;得婦人,以治衣服。其國連山數十里,有峽,固以鐵闔,號關門,新羅常屯弩士數千守之。



 初,百濟伐高麗,來請救,悉兵往破之,自是相攻不置。後獲百濟王殺之,滋結怨。武德四年,王真平遣使者入朝,高祖詔通直散騎侍郎庾文素持節答齎。後三年,拜柱國,封樂浪郡王、新羅王。



 貞觀五年,獻女樂二。太宗曰:「比林邑獻鸚鵡,言思鄉,丐還,況於人乎?」付使者歸之。是歲,真平死,無子,立女善德為王,大臣乙祭柄國。詔贈真平左光祿大夫,賻物段二百。九年,遣使者冊善德襲父封,國人號聖祖皇姑。十七年,為高麗、百濟所攻,使者來乞師,亦會帝親伐高麗,詔率兵以披虜勢。善德使兵五萬入高麗南鄙,拔水口城以聞。二十一年,善德死,贈光祿大夫,而妹真德襲王。明年,遣子文王及弟伊贊子春秋來朝,拜文王左武衛將軍,春秋特進。因請改章服,從中國制,內出珍服賜之。又詣國學觀釋奠、講論,帝賜所制《晉書》。辭歸,敕三品以上郊餞。



 高宗永徽元年,攻百濟,破之,遣春秋子法敏入朝。真德織錦為頌以獻,曰:「巨唐開洪業,巍巍皇猷昌。止戈成大定,興文繼百王。統天崇雨施,治物體含章。深仁諧日月,撫運邁時康。幡旗既赫赫,鉦鼓何鍠鍠。外夷違命者,翦覆被天殃。淳風凝幽顯,遐邇競呈祥。四時和玉燭,七耀巡萬方。維嶽降宰輔,維帝任忠良。三五成一德,昭我唐家唐。」帝美其意,擢法敏太府卿。



 五年,真德死,帝為舉哀,贈開府儀同三司,賜彩段三百,命太常丞張文收持節吊祭,以春秋襲王。明年,百濟、高麗、靺鞨共伐取其三十城。使者來請救,帝命蘇定方討之,以春秋為隅夷道行軍總管,遂平百濟。龍朔元年,死,法敏襲王。以其國為雞林州大都督府,授法敏都督。



 咸亨五年,納高麗叛眾,略百濟地守之。帝怒,詔削官爵,以其弟右驍衛員外大將軍、臨海郡公仁問為新羅王,自京師歸國。詔劉仁軌為雞林道大總管,衛尉卿李弼、右領軍大將軍李謹行副之,發兵窮討。上元二年二月,仁軌破其眾於七重城,以靺鞨兵浮海略南境,斬獲甚眾。詔李謹行為安東鎮撫大使,屯買肖城,三戰,虜皆北。法敏遣使入朝謝罪,貢篚相望,仁問乃還,辭王,詔復法敏官爵。然多取百濟地,遂抵高麗南境矣。置尚、良、康、熊、全、武、漢、朔、溟九州,州有都督,統郡十或二十,郡有大守,縣有小守。開耀元年,死,子政明襲王。遣使者朝,丐唐禮及它文辭,武后賜《吉兇禮》並文詞五十篇。死,子理洪襲王。死,弟興光襲王。



 玄宗開元中,數入朝,獻果下馬、朝霞紬、魚牙紬、海豹皮。又獻二女,帝曰:「女皆王姑姊妹,違本俗,別所親,朕不忍留。」厚賜還之。又遣子弟入太學學經術。帝間賜興光瑞文錦、五色羅、紫繡紋袍、金銀精器。興光亦上異狗馬、黃金、美髢諸物。初,渤海靺鞨掠登州,興光擊走之,帝進興光寧海軍大使,使攻靺鞨。二十五年死,帝尤悼之,贈太子太保,命邢以鴻臚少卿吊祭。子承慶襲王,詔曰:「新羅號君子國,知《詩》、《書》。以卿惇儒,故持節往,宜演經誼,使知大國之盛。」又以國人善棋,詔率府兵曹參軍楊季鷹為副。國高弈皆出其下,於是厚遺使者金寶。俄冊其妻樸為妃。承慶死,詔使者臨吊,以其弟憲英嗣王。帝在蜀,遣使溯江至成都朝正月。



 大歷初,憲英死,子乾運立,甫丱,遣金隱居入朝待命。詔倉部郎中歸崇敬往吊,監察御史陸珽、顧愔為副冊授之,並母金為太妃。會其宰相爭權相攻,國大亂,三歲乃定。於是,歲朝獻。建中四年死,無子,國人共立宰相金良相嗣。貞元元年,遣戶部郎中蓋塤持節命之。是年死,立良相從父弟敬信襲王。十四年,死,無子,立嫡孫俊邕。明年,遣司封郎中韋丹持冊,未至,俊邕死,丹還。子重興立,永貞元年,詔兵部郎中元季方冊命。後三年,使者金力奇來謝,且言:「往歲冊故主俊邕為主,母申太妃,妻叔妃,而俊邕不幸,冊今留省中,臣請授以歸。」又為其宰相金彥升、金仲恭、王之弟蘇金添明丐門戟,詔皆可。凡再朝貢。七年死,彥升立,來告喪,命職方員外郎崔廷吊,且命新王,以妻貞為妃。長慶、寶歷間,再遣使者來朝,留宿衛。彥升死,子景徽立。大和五年,以太子左諭德源寂冊吊如儀。開成初,遣子義琮謝,願留衛,見聽,明年遣之。五年,鴻臚寺籍質子及學生歲滿者一百五人,皆還之。



 有張保皋、鄭年者,皆善鬥戰,工用槍。年復能沒海,履其地五十里不噎,角其勇健,保皋不及也。年以兄呼保皋,保皋以齒,年以藝,常不相下。自其國皆來為武寧軍小將。後保皋歸新羅,謁其王曰:「遍中國以新羅人為奴婢,願得鎮清海,使賊不得掠人西去。」清海,海路之要也。王與保皋萬人守之。自大和後,海上無鬻新羅人者。保皋既貴於其國,年饑寒客漣水,一日謂戍主馮元規曰:「我欲東歸,乞食於張保皋。」元規曰:「若與保皋所負何如?奈何取死其手?」年曰:「饑寒死,不如兵死快,況死故鄉邪!」年遂去。至,謁保皋,飲之極歡。飲未卒,聞大臣殺其王,國亂無主。保皋分兵五千人與年,持年泣曰:「非子不能平禍難。」年至其國,誅反者,立王以報。王遂召保皋為相,以年代守清海。會昌後,朝貢不復至。



 贊曰:杜牧稱:「安思順為朔方節度時,郭汾陽、李臨淮俱為牙門都將,二人不相能,雖同盤飲食,常睇相視,不交一言。及汾陽代思順,臨淮欲亡去,計未決。旬日,詔臨淮分汾陽半兵東出趙、魏,臨淮入請曰:『一死固甘,乞免妻子。』汾陽趨下,持手上堂,曰:『今國亂主遷,非公不能東伐,豈懷私忿時邪?』及別,執手泣涕,相勉以忠義,訖平劇盜,實二公之力。知其心不叛,知其心,難也;忿必見短,知其材,益難也。此保皋與汾陽之賢等耳。年投保皋必曰:『彼貴我賤,我降下之,不宜以舊忿殺我。』保皋果不殺,人之常情也。臨淮請死於汾陽,亦人之常情也。保皋任年,事出於己,年且寒饑,易為感動。汾陽、臨淮,平生亢立,臨淮之命,出於天子。榷於保皋,汾陽為優。此乃聖賢遲疑成敗之際也。世稱周、邵為百代之師,周公擁孺子而邵公疑之,以周公之聖,邵公之賢,少事文王,老佐武王,能平天下,周公之心,邵公且不知之。茍有仁義之心,不資以明,雖邵公尚爾,況其下哉!」嗟乎,不以怨毒相槊,而先國家之憂,晉有祁奚,唐有汾陽、保皋,孰謂夷無人哉!



 日本,古倭奴也。去京師萬四千里,直新羅東南,在海中,島而居,東西五月行,南北三月行。國無城郛,聯木為柵落,以草茨屋。左右小島五十餘,皆自名國,而臣附之。置本率一人,檢察諸部。其俗多女少男,有文字,尚浮屠法。其官十有二等。其王姓阿每氏,自言初主號天御中主,至彥瀲,凡三十二世,皆以「尊」為號,居築紫城。彥瀲子神武立,更以「天皇」為號,徙治大和州。次曰綏靖,次安寧,次懿德,次孝昭,次天安,次孝靈,次孝元,次開化,次崇神,次垂仁,次景行,次成務,次仲哀。仲哀死,以開化曾孫女神功為王。次應神,次仁德,次履中,次反正,次允恭,次安康,次雄略,次清寧,次顯宗,次仁賢,次武烈,次繼體,次安閑,次宣化,次欽明。欽明之十一年,直梁承聖元年。次海達。次用明,亦曰目多利思比孤,直隋開皇末,始與中國通。次崇峻。崇峻死,欽明之孫女雄古立。次舒明,次皇極。其俗椎髻,無冠帶,跣以行,幅巾蔽後,貴者冒錦;婦人衣純色裙,長腰襦,結發於後。至煬帝,賜其民錦線冠,飾以金玉,文布為衣,左右佩銀■長八寸,以多少明貴賤。



 太宗貞觀五年,遣使者入朝。帝矜其遠,詔有司毋拘歲貢。遣新州刺史高仁表往諭,與王爭禮不平,不肯宣天子命而還。久之,更附新羅使者上書。



 永徽初,其王孝德即位,改元曰白雉,獻虎魄大如斗,碼硇若五升器。時新羅為高麗、百濟所暴,高宗賜璽書,令出兵援新羅。未幾孝德死,其子天豐財立。死,子天智立。明年,使者與蝦蛦人偕朝。蝦蛦亦居海島中,其使者須長四尺許,珥箭於首,令人戴瓠立數十步,射無不中。天智死,子天武立。死,子總持立。咸亨元年,遣使賀平高麗。後稍習夏音,惡倭名,更號日本。使者自言,國近日所出,以為名。或云日本乃小國,為倭所並,故冒其號。使者不以情,故疑焉。又妄誇其國都方數千里,南、西盡海,東、北限大山,其外即毛人云。



 長安元年,其王文武立,改元曰太寶,遣朝臣真人粟田貢方物。朝臣真人者,猶唐尚書也。冠進德冠,頂有華■四披,紫袍帛帶。真人好學,能屬文,進止有容。武後宴之麟德殿,授司膳卿,還之。文武死,子阿用立。死,子聖武立,改元曰白龜。開元初,粟田復朝,請從諸儒受經。詔四門助教趙玄默即鴻臚寺為師,獻大幅布為贄,悉賞物貿書以歸。其副朝臣仲滿慕華不肯去,易姓名曰朝衡,歷左補闕,儀王友,多所該識,久乃還。聖武死,女孝明立,改元曰天平勝寶。天寶十二載,朝衡復入朝。上元中,擢左散騎常侍、安南都護。新羅梗海道,更繇明、越州朝貢。孝明死,大炊立。死,以聖武女高野姬為王。死,白壁立。建中元年,使者真人興能獻方物。真人,蓋因官而氏者也。興能善書,其紙似繭而澤,人莫識。貞元末,其王曰桓武,遣使者朝。其學子橘免勢、浮屠空海願留肄業,歷二十餘年。使者高階真人來請免勢等俱還,詔可。次諾樂立,次嵯峨,次浮和,次仁明。仁明直開成四年,復入貢。次文德,次清和,次陽成。次光孝,直光啟元年。



 其東海嶼中又有邪古、波邪、多尼三小王,北距新羅,西北百濟,西南直越州,有絲絮、怪珍雲。



 流鬼去京師萬五千里,直黑水靺鞨東北,少海之北,三面皆阻海,其北莫知所窮。人依嶼散居,多沮澤,有魚鹽之利。地蚤寒,多霜雪,以木廣六寸、長七尺系其上,以踐冰,逐走獸。土多狗,以皮為裘。俗被發。粟似莠而小,無蔬蓏它穀。勝兵萬人。南與莫曳靺鞨鄰,東南航海十五日行,乃至。貞觀十四年,其王遣子可也餘莫貂皮更三譯來朝。授騎都尉,遣之。



 龍朔初,有儋羅者,其王儒李都羅遣使入朝,國居新羅武州南島上,俗樸陋,衣大豕皮,夏居革屋,冬窟室。地生五穀,耕不知用牛,以鐵齒杷土。初附百濟。麟德中,酋長來朝,從帝至太山。後附新羅。



 開元十一年,又有達末婁、達妒二部首領朝貢。達末婁自言北扶餘之裔,高麗滅其國,遺人度那河,因居之,或曰他漏河,東北流入黑水。達姤,室韋種也,在那河陰,凍末河之東,西接黃頭室韋,東北距達末婁云。



\end{pinyinscope}