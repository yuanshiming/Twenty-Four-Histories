\article{列傳第一百八 畢崔劉陸鄭硃韓}

\begin{pinyinscope}

 畢諴,字存之,黃門監構從孫。構弟栩,生凌,凌生勻然申明自己為無產階級服務;強調理論對於實踐的依賴關系。,世失官,為鹽估。勻生諴,蚤孤。夜然薪讀書,母恤其疲,奪火使寐,不肯息,遂通經史,工辭章。性端愨,不妄與人交。



 太和中,舉進士、書判拔萃,連中。闢忠武杜悰幕府。悰領度支,表為巡官,又從闢淮南,入拜侍御史。李德裕始與悰同輔政,不協,故出悰劍南東川節度使。故吏惟諴餞訊如平日,德裕忌之,出為慈州刺史。累官駕部員外郎、倉部郎中。故事,要家勢人,以倉、駕二曹為辱,諴沛然如處美官,無異言。宰相知之,以職方郎中兼侍御史知雜事,召入翰林為學士。



 黨項擾河西,宣宗嘗召訪邊事,諴援質古今,條破羌狀甚悉,帝悅曰:「吾將擇能帥者,孰謂頗、牧在吾禁署,卿為朕行乎。」諴唯唯,即拜刑部侍郎,出為邠寧節度、河西供軍安撫使。諴到軍,遣吏懷諭,羌人皆順向。時戍兵常苦調饟乏,諴募士置屯田,歲收穀三十萬斛,以省度支經費,詔書嘉美。俄徙昭義,又遷河東。河東尤近胡,復脩杷頭七十烽,謹候虜,寇不敢入。



 懿宗立,遷宣武節度使,召為戶部尚書,判度支。未幾,以禮部尚書同中書門下平章事。再期,固稱疾,改兵部尚書,罷。旋兼平章事節度河中。卒,年六十二。



 諴於吏術尤所長,既貴,所得祿奉,養護宗屬之乏,無間然。始,諴被知於宣宗,嘗許以相。令狐綯忌之,自邠寧凡三徙,不得還。諴思有以結綯,至太原,求麗姝盛飾使獻。綯曰:「太原於我無分,今以是餌,將破吾族矣。」不受。使者留於邸,諴亦放之。太醫李玄伯者,帝所喜,以錢七十萬聘之,夫婦日自進食,得其歡心,乃進之帝,嬖幸冠後宮。玄伯又治丹劑以進,帝餌之,疽生於背。懿宗立,收玄伯及方士王岳、虞芝等,俱誅死。



 崔彥昭,字思文,其先清河人。淹貫儒術,擢進士第。數應帥鎮闢奏,於吏治精明,所至課最。累進戶部侍郎。繇河陽節度使徙河東。先是,沙陀諸部多犯法,彥昭撫循有威惠,三年,境內大治,耆老叩闕願留,詔可。僖宗立,授兵部侍郎、諸道鹽鐵轉運使。俄同中書門下平章事,仍判度支。初,楊收、路巖、韋保衡皆坐朋比賄賂得罪死,蕭仿秉政,矯革之,而彥昭協力,故百職修舉,察不至苛。不六月,遷門下侍郎。帝因下詔暴收等過惡,申勵丁寧,以成其美。



 彥昭雖宰相,退朝侍母膳,與家人齒,順色柔聲,在左右無違,士人多其孝。與王凝外昆弟也。凝大中初先顯,而彥昭未仕,嘗見凝,凝倨不冠帶,嫚言曰:「不若從明經舉。」彥昭為憾。至是,凝為兵部侍郎。母聞彥昭相,敕婢多制屨襪,曰:「王氏妹必與子皆逐,吾將共行。」彥昭聞之,泣且拜,不敢為怨。而凝竟免。



 伶人李可及為懿宗所寵,橫甚,彥昭奏逐,死嶺南。累拜兼尚書右僕射,以疾去位,授太子太傅,卒。



 劉鄴,字漢籓,潤州句容人。父三復,以善文章知名。少孤,母病廢,三復丐粟以養。李德裕為浙西觀察使,奇其文,表為掌書記。德裕三領浙西及劍南、淮南,未嘗不從。會昌時,位宰相,擢三復刑部侍郎、弘文館學士。



 鄴六七歲能屬辭,德裕憐之,使與其子共師學。德裕既斥,鄴無所依,去客江湖間。陜虢高元裕表署推官,高少逸又闢鎮國幕府。咸通初,擢左拾遺,召為翰林學士,賜進士第。歷中書舍人,遷承旨。鄴傷德裕以朋黨抱誣死海上,令狐綯久當國,更數赦,不為還官爵。至懿宗立,綯去位,鄴乃申直其冤,復官爵,世高其義。進戶部侍郎、諸道鹽鐵轉運使。以禮部尚書同中書門下平章事,判度支。僖宗嗣位,再遷尚書左僕射。



 初,韋保衡、路巖與鄴同秉政,為跡親。俄而蕭仿、崔彥昭得相,罷鄴為淮南節度使、同平章事。黃巢方熾,詔高駢代之,徙節度鳳翔,固辭,還左僕射。帝西狩,追乘輿不及,與崔沆、豆盧彖匿將軍張直方家,賊捕急,三人不肯臣,俱見殺。



 豆盧彖者,字希真,河南人。仕歷翰林學士、戶部侍郎,與崔沆皆拜同中書門下平章事。是日,宣告於廷,大風雷雨拔樹。未幾,及禍。初,咸通中,有治歷者工言禍福,或問:「比宰相多不至四五,謂何?」答曰:「紫微方災,然其人又將不免。」後楊收、韋保衡、路巖、盧攜、劉鄴、於琮、彖與沆,皆不得終云。



 陸扆,字祥文,宰相贄族孫。客於陜,遂為陜人。光啟二年,從僖宗幸山南,擢進士第,累進翰林學士、中書舍人。扆工屬辭,敏速若注射然,一時書命,同僚自以為不及,昭宗優遇之。帝嘗作賦,詔學士皆和,獨扆最先就。帝覽之,嘆曰:「貞元時,陸贄、吳通玄兄弟善內廷文書,後無繼者,今朕得之。」始,得舉進士時,方遷幸,而六月榜出。至是,每甚暑,它學士輒戲曰:「造榜天也。」以譏扆進非其時。累為尚書左丞,封嘉興縣男。徙戶部侍郎、同中書門下平章事。故事,自三省得宰相,有光署錢,留為宴資,學士院未始有。至扆,送光院錢五十萬,以榮近司。進中書侍郎,判戶部。



 嗣覃王以兵伐鳳翔,扆諫曰:「國步方安,不宜加兵近輔,必為它盜所乘,無益也。且親王而屬軍事,必有後害。」帝顧軍興,責扆沮撓,貶峽州刺史。師果敗。久之,授工部尚書。從天子自華州還,以兵部尚書復當國,封吳郡公。



 天復初,帝密語韓偓曰:「陸扆、裴贄孰忠於我?」偓曰:「扆等皆宰相,安有它腸?」帝曰:「外言扆不喜我復位,元日易服奔啟夏門,信不?」偓曰:「孰為陛下言此?」曰:「崔胤、令狐渙。」偓曰:「設扆如是,亦不足責。且陛下反正,扆素不知謀,忽聞兵起,欲出奔耳。陛下責其不死難則可,以為不喜,乃讒言也。」帝遂悟。累兼戶部尚書。



 帝至自鳳翔,大赦天下,諸道皆賜詔,獨不及李茂貞。扆曰:「國西,鳳翔為最近,跡其罪固不可赦。然尚修職貢,朝廷未之絕,無宜於詔書有以異也。」始,崔胤罷相,扆代之。胤內怨望,及是議以為陰有黨附,貶沂王傅,分司東都。胤死,復授吏部尚書,從遷洛。柳璨始附硃全忠,謀去朝廷衣冠有望者,貶扆濮州司戶參軍,殺之白馬驛,年五十九。扆初名允迪,後改云。



 鄭綮,字蘊武。及進士第,歷監察御史,擢累左司郎中。因窶甚,丐補廬州刺史。黃巢掠淮南,綮移檄請無犯州境,巢笑,為斂兵,州獨完。僖宗嘉之,賜緋魚。歲滿去,贏錢千緡藏州庫。後它盜至,終不犯鄭使君錢。及楊行密為刺史,送都還綮。王徽為御史大夫,以兵部郎中表知雜事,遷給事中。杜弘徽任中書舍人,綮以其兄讓能輔政,不宜處禁要,上還制書,不報,輒移病去。召為右散騎常侍,往往條摘失政,眾言雚傳之,宰相怒,改國子祭酒,議者不直,復還常侍。大順後,王政微,綮每以詩謠托諷,中人有誦之天子前者。昭宗意其有所蘊未盡,因有司上班簿,遂署其側曰:「可禮部侍郎、同中書門下平章事。」綮本善詩,其語多俳諧,故使落調,世共號「鄭五歇後體」。至是,省史走其家上謁,綮笑曰;「諸君誤矣,人皆不識字,宰相亦不及我。」史言不妄。俄聞制詔下,嘆曰:「萬一然,笑殺天下人!」既視事,宗戚詣慶,搔首曰:「歇後鄭五作宰相,事可知矣。」固讓,不聽。立朝偘然,無復故態。自以不為人所瞻望,才三月,以疾乞骸,拜太子少保致仕,卒。



 硃樸,襄州襄陽人。以三史舉,繇荊門令進京兆府司錄參軍,改著作郎。乾寧初,太府少卿李元實欲取中外九品以上官兩月俸助軍興,樸上疏執不可而止。



 擢國子《毛詩》博士。上書言當世事,議遷都曰:「古王者不常厥居,皆觀天地興衰,隨時制事。關中,隋家所都,我實因之,凡三百歲,文物資貨,奢侈僭偽皆極焉。廣明巨盜陷覆宮闕,局署帑藏,里閈井肆,所存十二,比幸石門、華陰,十二之中又亡八九,高祖、太宗之制蕩然矣。夫襄、鄧之西,夷溫數百里,其東,漢輿、鳳林為之關,南,菊潭環屈而流屬於漢,西有上洛重山之險,北有白崖聯絡,乃形勝之地,沃衍之墟。若廣浚漕渠,運天下之財,可使大集。自古中興之君,去已衰之衰,就未王而王。今南陽,漢光武雖起而未王也。臣視山河壯麗處多,故都已盛而衰,難可興已;江南土薄水淺,人心囂浮輕巧,不可以都;河北土厚水深,人心強愎狠戾,不可以都。惟襄、鄧實惟中原,人心質良,去秦咫尺,而有上洛為之限,永無夷狄侵軼之虞,此建都之極選也。」不報。



 樸為人木強,無它能。方是時,天子失政,思用特起士,任之以中興,而樸所善方士許巖士得幸,出入禁中,言樸有經濟才,又水部郎中何迎亦表其賢,帝召與語,擢左諫議大夫、同中書門下平章事。以素無聞,人人大驚,俄判戶部,進中書侍郎。帝益治兵,所處可一委樸。樸移檄四方,令近者出甲士,資饋饟,遠者以羨餘上。後數月,巖士為韓建所殺,樸罷為秘書監,三貶郴州司戶參軍,卒。與樸皆相者孫渥。



 孫偓,字龍光。父景商,為天平軍節度使。偓第進士,歷顯官,以戶部侍郎同中書門下平章事,遷門下,為鳳翔四面行營都統。俄兼禮部尚書、行營節度諸軍都統招討處置等使。始,家第堂柱生槐枝,期而茂,既而偓秉政,封樂安縣侯。與樸皆貶衡州司馬,卒。



 偓性通簡,不矯飭,嘗曰:「士茍有行,不必以己長形彼短、己清彰彼濁。」每對客,奴童相詬曳僕諸前,不之責,曰:「若持怒心,即自撓矣。」



 兄儲,歷天雄節度使,終兵部尚書。



 韓偓,字致光,京兆萬年人。擢進士第,佐河中幕府。召拜左拾遺,以疾解。後遷累左諫議大夫。宰相崔胤判度支,表以自副。王溥薦為翰林學士,遷中書舍人。偓嘗與胤定策誅劉季述,昭宗反正,為功臣。帝疾宦人驕橫,欲盡去之。偓曰:「陛下誅季述時,餘皆赦不問,今又誅之,誰不懼死?含垢隱忍,須後可也。天子威柄,今散在方面,若上下同心,攝領權綱,猶冀天下可治。宦人忠厚可任者,假以恩幸,使自翦其黨,蔑有不濟。今食度支者乃八千人,公私牽屬不減二萬,雖誅六七巨魁,未見有益,適固其逆心耳。」帝前膝曰:「此一事終始屬卿。」



 中書舍人令狐渙任機巧,帝嘗欲以當國,俄又悔曰:「渙作宰相或誤國,朕當先用卿。」辭曰:「渙再世宰相,練故事,陛下業已許之。若許渙可改,許臣獨不可移乎?」帝曰:「我未嘗面命,亦何憚?」偓因薦御史大夫趙崇勁正雅重,可以準繩中外。帝知偓,崇門生也,嘆其能讓。初,李繼昭等以功皆進同中書門下平章事,時謂「三使相」,後稍稍更附韓全誨、周敬容,皆忌胤。胤聞,召鳳翔李茂貞入朝,使留族子繼筠宿衛。偓聞,以為不可,胤不納。偓又語令狐渙,渙曰:「吾屬不惜宰相邪?無衛軍則為閹豎所圖矣。」偓曰:「不然。無兵則家與國安,有兵則家與國不可保。」胤聞,憂,未知所出。李彥弼見帝倨甚,帝不平,偓請逐之,赦其黨許自新,則狂謀自破,帝不用。彥弼譖偓及渙漏禁省語,不可與圖政,帝怒,曰:「卿有官屬,日夕議事,奈何不欲我見學士邪?」繼昭等飲殿中自如,帝怒,偓曰:「三使相有功,不如厚與金帛官爵,毋使豫政事。今宰相不得顓決事,繼昭輩所奏必聽。它日遽改,則人人生怨。初以衛兵檢中人,今敕使、衛兵為一,臣竊寒心,願詔茂貞還其衛軍。不然,兩鎮兵斗闕下,朝廷危矣。」及胤召硃全忠討全誨,汴兵將至,偓勸胤督茂貞還衛卒。又勸表暴內臣罪,因誅全誨等;若茂貞不如詔,即許全忠入朝。未及用,而全誨等已劫帝西幸。



 偓夜追及鄠,見帝慟哭。至鳳翔,遷兵部侍郎,進承旨。



 宰相韋貽範母喪,詔還位,偓當草制,上言:「貽範處喪未數月,遽使視事,傷孝子心。今中書事,一相可辦。陛下誠惜貽範才,俟變縗而召可也。何必使出峨冠廟堂,入泣血柩側,毀瘠則廢務,勤恪則忘哀,此非人情可處也。」學士使馬從皓逼偓求草,偓曰:「腕可斷,麻不可草!」從皓曰:「君求死邪?」偓曰:「吾職內署,可默默乎?」明日,百官至,而麻不出,宦侍合噪。茂貞入見帝曰:「命宰相而學士不草麻,非反邪?」艴然出。姚洎聞曰:「使我當直,亦繼以死。」既而帝畏茂貞,卒詔貽範還相,洎代草麻。自是宦黨怒偓甚。從皓讓偓曰:「南司輕北司甚,君乃崔胤、王溥所薦,今日北司雖殺之可也。兩軍樞密,以君周歲無奉入,吾等議救接,君知之乎?」偓不敢對。



 茂貞疑帝間出依全忠,以兵衛行在。帝行武德殿前,因至尚食局,會學士獨在,宮人招偓,偓至,再拜哭曰:「崔胤甚健,全忠軍必濟。」帝喜,偓曰:「願陛下還宮,無為人知。」帝賜以面豆而去。全誨誅,宮人多坐死。帝欲盡去餘黨,偓曰:「禮,人臣無將,將必誅,宮婢負恩不可赦。然不三十年不能成人,盡誅則傷仁。願去尤者,自內安外,以靜群心。」帝曰:「善。」崔胤請以輝王為元帥,帝問偓:「它日累吾兒否?」偓曰:「陛下在東內時,天陰雺,王聞烏聲曰:『上與後幽困,烏雀聲亦悲。』陛下聞之惻然,有是否?」帝曰:「然。是兒天生忠孝,與人異。」意遂決。偓議附胤類如此。



 帝反正,勵精政事,偓處可機密,率與帝意合,欲相者三四,讓不敢當。蘇檢復引同輔政,遂固辭。初,偓侍宴,與京兆鄭元規、威遠使陳班並席,辭曰:「學士不與外班接。」主席者固請,乃坐。既元規、班至,終絕席。全忠、胤臨陛宣事,坐者皆去席,偓不動,曰:「侍宴無輒立,二公將以我為知禮。」全忠怒偓薄己,悻然出。有譖偓喜侵侮有位,胤亦與偓貳。會逐王溥、陸扆,帝以王贊、趙崇為相,胤執贊、崇非宰相器,帝不得已而罷。贊、崇皆偓所薦為宰相者。全忠見帝,斥偓罪,帝數顧胤,胤不為解。全忠至中書,欲召偓殺之。鄭元規曰:「偓位侍郎、學士承旨,公無遽。」全忠乃止,貶濮州司馬。帝執其手流涕曰:「我左右無人矣。」再貶榮懿尉,徙鄧州司馬。天祐二年,復召為學士,還故官。偓不敢入朝,挈其族南依王審知而卒。



 兄儀,字羽光,亦以翰林學士為御史中丞。偓貶之明年,帝宴文思球場,全忠入,百官坐廡下,全忠怒,貶儀棣州司馬,侍御史歸藹登州司戶參軍。



 贊曰:懿、僖以來,王道日失厥序,腐尹塞朝,賢人遁逃,四方豪英,各附所合而奮。天子塊然,所與者,惟佞愎庸奴,乃欲鄣橫流、支已顛,寧不殆哉!觀綮、樸輩不次而用,捭豚臑,拒貙牙,趣亡而已。一韓偓不能容,況賢者乎?



\end{pinyinscope}