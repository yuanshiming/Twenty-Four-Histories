\article{列傳第六十二 郭子儀}

\begin{pinyinscope}

 郭子儀,字子儀,華州鄭人。長七尺二寸。以武舉異等補左衛長史,累遷單于副都護、振遠軍使。天寶八載,木剌山始築橫塞軍及安北都護府現象學的本質現象學哲學用語。指直接呈現在意識或現,詔即軍為使。俄苦地偏不可耕,徙築永清,號天德軍,又以使兼九原太守。



 十四載,安祿山反,詔子儀為衛尉卿、靈武郡太守,充朔方節度使,率本軍東討。子儀收靜邊軍,斬賊將周萬頃,擊高秀巖河曲,敗之,遂收雲中、馬邑,開東陘。加御史大夫。賊陷常山,河北郡縣皆沒。會李光弼攻賊常山,拔之,子儀引軍下井陘,與光弼合,破賊史思明眾數萬,平幰城。南攻趙郡,禽賊四千,縱之,斬偽守郭獻璆,還常山。思明以眾數萬尾軍,及行唐,子儀選騎五百更出挑之。三日,賊引去,乘之,又破於沙河,遂趨常陽以守。祿山益出精兵佐思明。子儀曰:「彼恃加兵,必易我;易我,心不固,戰則克矣。」與戰未決,戮一步將以徇,士殊死鬥,遂破之,斬首二千級,俘五百人,獲馬如之。於是晝揚兵,夜搗壘,賊不得息,氣益老。乃與光弼、僕固懷恩、渾釋之、陳回光等擊賊嘉山,斬首四萬級,獲人馬萬計。思明跳奔博陵。於是河北諸郡往往斬賊守,迎王師。方北圖範陽,會哥舒翰敗,天子入蜀,太子即位靈武,詔班師。子儀與光弼率步騎五萬赴行在。時朝廷草昧,眾單寡,軍容缺然,及是國威大振。拜子儀兵部尚書、同中書門下平章事,仍總節度。肅宗大閱六軍,鼓而南,至彭原。宰相房琯自請討賊,次陳濤,師敗,眾略盡,故帝唯倚朔方軍為根本。賊將阿史那從禮以同羅、僕骨騎五千,誘河曲九府、六胡州部落數萬迫行在。子儀以回紇首領葛邏支擊之,執獲數萬,牛羊不可勝計,河曲平。



 至德二載,攻賊崔乾祐於潼關,乾祐敗,退保蒲津。會永樂尉趙復、河東司戶參軍韓旻、司士徐景及宗室子鋒在城中,謀為內應,子儀攻蒲,復等斬陴者,披闔內軍。乾祐走安邑,安邑偽納之,兵半入,縣門發,乾祐得脫身走。賊安守忠壁永豐倉,子儀遣子旰與戰,多殺至萬級,旰死於陣。進收倉。於是關、陜始通。詔還鳳翔,進司空,充關內、河東副元帥。率師趨長安,次潏水上。賊守忠等軍清渠左。大戰,王師不利,委仗奔。子儀收潰卒保武功,待罪於朝,乃授尚書左僕射。俄從元帥廣平王率蕃、漢兵十五萬收長安。李嗣業為前軍,元帥為中軍,子儀副之,王思禮為後軍,陣香積寺之北,距灃水,臨大川,彌亙一舍。賊李歸仁領勁騎薄戰,官軍囂,嗣業以長刀突出,斬賊數十騎,乃定。回紇以奇兵繚賊背,夾攻之,斬首六萬級,生禽二萬,賊帥張通儒夜亡陜郡。翌日,王入京師,老幼夾道呼曰:「不圖今日復見官軍!」王休士三日,遂東。安慶緒聞王師至,遣嚴莊悉眾十萬屯陜,助通儒,旌幟鉦鼓徑百餘里。師至新店,賊已陣,出輕騎,子儀遣二隊逐之,又至,倍以往,皆不及賊營輒反。最後,賊以二百騎掩軍,未戰走,子儀悉軍追,橫貫其營。賊張兩翼包之,官軍卻。嗣業率回紇從後擊,塵且坌,飛矢射賊,賊驚曰:「回紇至矣!」遂大敗,殭尸相屬於道。嚴莊等走洛陽,挾慶緒度河保相州,遂收東都。於是河東、河西、河南州縣悉平。以功加司徒,封代國公,食邑千戶。入朝,帝遣具軍容迎灞上,勞之曰:「國家再造,卿力也。」子儀頓首陳謝。有詔還東都,經略北討。



 乾元元年,破賊河上,執安守忠以獻,遂朝京師。詔百官迎於長樂驛,帝禦望春樓待之。進中書令。帝即詔大舉九節度師討慶緒,以子儀、光弼皆元功,難相臨攝,第用魚朝恩為觀軍容宣慰使,而不立帥。



 子儀自杏園濟河,圍衛州。慶緒分其眾為三軍。將戰,子儀選善射三千士伏壁內,誡曰:「須吾卻,賊必乘壘,若等噪而射。」既戰,偽遁,賊薄營,伏發,注射如雨。賊震駭,王師整而奮,斬首四萬級,獲鎧胄數十萬,執安慶和,收衛州。又戰愁思岡,破之。連營進圍相州,引漳水灌城,漫二時,不能破。城中糧盡,人相食。慶緒求救於史思明,思明自魏來,李光弼、王思禮、許叔冀、魯炅前軍遇之,戰鄴南,夷負相當,炅中流矢。子儀督後軍,未及戰。會大風拔木,遂晦,跬步不能相物色,於是王師南潰,賊亦走,輜械滿野。諸節度引還。子儀以朔方軍保河陽,斷航橋。時王師眾而無統,進退相顧望,責功不專,是以及於敗。有詔留守東都,俄改東畿、山南東道、河南諸道行營元帥。魚朝恩素疾其功,因是媒譖之,故帝召子儀還,更以趙王為天下兵馬元帥,李光弼副之,代子儀領朔方兵。子儀雖失軍,無少望,乃心朝廷。思明再陷河、洛,西戎逼擾京輔,天子旰食,乃授邠寧、鄜坊兩節度使,仍留京師。議者謂子儀有社稷功,而孽寇首鼠,乃置散地,非所宜。帝亦悟。



 上元初,詔為諸道兵馬都統,以管崇嗣副之,率英武、威遠兵及河西、河東鎮兵,繇邠寧、朔方、大同、橫野軍以趨範陽。詔下,為朝恩沮解。明年,光弼敗邙山,失河陽。又明年,河中亂,殺李國貞,太原戕鄧景山。朝廷憂二軍與賊合,而少年新將望輕不可用,遂以子儀為朔方、河中、北庭、潞儀澤沁等州節度行營,兼興平、定國副元帥,進封汾陽郡王,屯絳州。時帝已不豫,群臣莫有見者,子儀請曰:「老牙受命,將死於外,不見陛下,目不瞑。」帝引至臥內,謂曰:「河東事一以委卿。」子儀嗚咽流涕。賜御馬、銀器、雜彩,別賜絹布九萬。子儀至屯,誅首惡王元振等數十人,太原辛云京亦治害景山者,諸鎮皆惕息。



 代宗立,程元振自謂於帝有功,忌宿將難制,離構百計。因罷子儀副元帥,加實戶七百,為肅宗山陵使。子儀懼讒且成,盡裒代宗所賜詔敕千餘篇上之,因自明。詔曰:「朕不德,詒大臣憂,朕甚自愧,自今公毋有疑。」初,帝與子儀平兩京,同天下憂患,至是悔悟,眷禮彌重。



 時史朝義尚盜洛,帝欲使副雍王,率師東討,為朝恩、元振交訾之,乃止。會梁崇義據襄州叛,僕固懷恩屯汾州,陰召回紇、吐蕃寇河西,殘涇州,犯奉天、武功,遽拜子儀為關內副元帥,鎮咸陽。初,子儀自相州罷歸京師,部曲離散,逮承詔,麾下才數十騎,驅民馬補行隊。至咸陽,虜已過渭水,並南山而東,天子跳幸陜。子儀聞,流涕,董行營還京師。遇射生將王獻忠以彀騎叛,劫諸王欲奔虜,子儀讓之,取諸王送行在。乃率騎南收兵,得武關防卒及亡士數千,軍浸完。會六軍將張知節迎子儀洛南,大閱兵,屯商州,威震關中。乃遣知節率烏崇福、羽林將長孫全緒為前鋒,營韓公堆,擊鼓歡山,張旗幟,夜叢萬炬,以疑賊。初,光祿卿殷仲卿募兵藍田,以勁騎先官軍為游弈,直度滻,民紿虜曰:「郭令公來。」虜懼。會故將軍王甫結俠少,夜鼓硃雀街,呼曰:「王師至!」吐蕃夜潰。於是遣大將李忠義屯苑中,渭北節度使王仲升守朝堂,子儀以中軍繼之。射生將王撫自署京兆尹,亂京城,子儀斬以徇。破賊書聞,帝以子儀為京城留守。



 自變生倉卒,賴子儀復安,故天下皆咎程元振,群臣數論奏。元振懼,乃說帝都洛陽,帝可其計。子儀奏曰:



 雍州古稱天府,右隴、蜀,左崤、函,襟馮終南、太華之險,背負清渭、濁河之固,地方數千里,帶甲十餘萬,兵強士勇,真用武之國,秦、漢所以成帝業也。後或處而泰、去而亡者不一姓,故高祖先入關定天下,太宗以來居洛陽者亦鮮。先帝興朔方,誅慶緒,陛下席西土,戮朝義,雖天道助順,亦地勢則然。比吐蕃馮陵而不能抗者,臣能言其略。夫六軍皆市井人,竄虛名,逃實賦,一日驅以就戰,有百奔無一前;又宦豎掩迷,庶政荒奪,遂令陛下徬徨暴露,越在陜服。斯委任失人,豈秦地非良哉!今道路流言,不識信否,咸謂且都洛陽。洛陽自大盜以來,焚埃略盡,百曹榛荒,寰服不滿千戶,井邑如墟,豺狼群嗥;東薄鄭、汴,南界徐,北綿懷、衛及相,千里蕭條,亭舍不煙,何以奉萬乘牲餼、供百官次舍哉?且地狹厄,裁數百里,險不足防,適為鬥場。陛下意者不以京畿新罹剽蹂,國用不足乎?昔衛為狄滅,文公廬於曹,衣大布之衣,冠大帛之冠,卒復舊邦,況赫赫天子,躬儉節用,寧為一諸侯下哉?臣願陛下斥素餐,去冗食,抑閹寺,任直臣,薄征弛役,恤隱撫鰥,委宰相以簡賢任能,付臣以訓兵禦侮,則中興之功,日月可冀。惟時邁亟還,見宗廟,謁園陵,再造王家,以幸天下。



 帝得奏,泣謂左右曰:「子儀固社稷臣也,朕西決矣。」乘輿還,子儀頓首請罪,帝勞曰:「用卿晚,故至此。」乃賜鐵券,圖形凌煙閣。



 僕固懷恩縱兵掠並、汾屬縣,帝患之,以子儀兼河東副元帥、河中節度使,鎮河中。懷恩子瑒屯榆次,為帳下張惟岳所殺,傳首京師,持其眾歸子儀。懷恩懼,委其母走靈州。廣德二年,進太尉,兼領北道邠寧、涇原、河西通和吐蕃及朔方招撫觀察使。辭太尉不拜。懷恩誘吐蕃、回紇、黨項數十萬入寇,朝廷大恐,詔子儀屯奉天。帝問計所出,對曰:「無能為也。懷恩本臣偏將,雖剽果,然素失士心。今能為亂者,訹思歸之人,劫與俱來,且皆臣故部曲,素以恩信結之,彼忍以刃相向乎?」帝曰:「善。」虜寇邠州,先驅至奉天,諸將請擊之。子儀曰:「客深入,利速戰。彼下素德我,吾緩之,當自攜貳。」因下令:「敢言戰者斬!」堅壁待之,賊果遁。



 子儀至自涇陽,恩賚崇縟,進拜尚書令,懇辭,不聽。詔趣詣省視事,百官往慶,敕射生五百騎執戟寵衛。子儀確讓,且言:「太宗嘗踐此官,故累聖曠不置員,皇太子為雍王,定關東,乃得授,渠可猥私老臣,隳大典?且用兵以來,僭賞者多,至身兼數官,冒進亡恥。今兇丑略平,乃作法審官之時,宜從老臣始。」帝不獲已,許之,具所以讓付史官。因賜美人六人,從者自副,車服帷帟咸具。



 永泰元年,詔都統河南道節度行營,復鎮河中。懷恩盡說吐蕃、回紇、常項、羌、渾、奴剌等三十萬,掠涇、邠,躪鳳翔,入醴泉、奉天,京師大震。於是帝命李忠臣屯渭橋,李光進屯雲陽,馬璘、郝廷玉屯便橋,駱奉先、李日越屯厔盩,李抱玉屯鳳翔,周智光屯同州,杜冕屯坊州,天子自將屯苑中。急召子儀屯涇陽,軍才萬人。比到,虜騎圍已合,乃使李國臣、高升、魏楚玉、陳回光、硃元琮各當一面,身自率鎧騎二千出入陣中。回紇怪問,:「是謂誰?」報曰:「郭令公。」驚曰:「令公存乎?懷恩言天可汗棄天下,令公即世,中國無主,故我從以來。公今存,天可汗存乎?」報曰:「天子萬壽。」回紇悟曰:「彼欺我乎!」子儀使諭虜曰:「昔回紇涉萬里,戡大憝,助復二京,我與若等休戚同之。今乃棄舊好,助叛臣,一何愚!彼背主棄親,於回紇何有?」回紇曰:「本謂公云亡,不然,何以至此。今誠存,我得見乎?」子儀將出,左右諫:「戎狄野心不可信。」子儀曰:「虜眾數十倍,今力不敵,吾將示以至誠。」左右請以騎五百從,又不聽。即傳呼曰:「令公來!」虜皆持滿待。子儀以數十騎出,免胄見其大酋曰:「諸君同艱難久矣,何忽亡忠誼而至是邪?」回紇舍兵下馬拜曰:「果吾父也。」子儀即召與飲,遺錦彩結歡,誓好如初。因曰:「吐蕃本吾舅甥國,無負而來,棄親也。馬牛被數百里,公等若倒戈乘之,若俯取一芥,是謂天賜,不可失。且逐戎得利,與我繼好,不兩善乎?」會懷恩暴死,群虜無所統一,遂許諾。吐蕃疑之,夜引去。子儀遣將白元光合回紇眾追躡,大軍繼之,破吐蕃十萬於靈臺西原,斬級五萬,俘萬人,盡得所掠士女牛羊馬橐駝不勝計。遂自涇陽來朝,加實封二百戶,還河中。



 大歷元年,華州節度使周智光謀叛,帝間道以蠟書賜子儀,令悉軍討之。同、華將吏聞軍起,殺智光,傳首闕下。二年,吐蕃寇涇州,詔移屯涇陽。邀戰於靈州,敗之,斬首二萬級。明年,還河中。吐蕃復寇靈武,詔率師五萬屯奉天,白元光破虜於靈武。議者以吐蕃數為盜,馬璘孤軍在邠不能支,乃以子儀兼邠寧慶節度使,屯邠州,徙璘為涇原節度使。回紇赤心請市馬萬匹,有司以財乏,止市千匹。子儀曰:「回紇有大功,宜答其意,中原須馬,臣請內一歲奉,佐馬直。」詔不聽,人許其忠。



 九年,入朝,對延英,帝與語吐蕃方強,慷慨至流涕。退,上書曰:



 朔方,國北門,西御犬戎,北虞獫狁,五城相去三千里。開元、天寶中,戰士十萬,馬三萬匹,僅支一隅。自先帝受命靈武,戰士從陛下征討無寧歲。頃以懷恩亂,痍傷雕耗,亡三分之二,比天寶中止十之一。今吐蕃兼吞河、隴,雜羌、渾之眾,歲深入畿郊,勢逾十倍,與之角勝,豈易得邪?屬者虜來,稱四節度,將別萬人,人兼數馬。臣所統士不當賊四之一,馬不當賊百之二,外畏內懼,將何以安?臣惟陛下制勝,力非不足,但簡練不至,進退未一,時淹師老,地廣勢分。願於諸道料精卒滿五萬者,列屯北邊,則制勝可必。竊惟河南、河北、江淮大鎮數萬,小者數千,殫屈稟給,未始搜擇。臣請追赴關中,勒步隊,示金鼓,則攻必破,守必全,長久之策也。



 又自陳衰老,乞骸骨。詔曰:「朕終始倚賴,未可以去位。」不許。



 德宗嗣位,詔還朝,攝塚宰,充山陵使,賜號「尚父」,進位太尉、中書令,增實封通前二千戶,給糧千五百人,芻馬二百匹,盡罷所領使及帥。建中二年,疾病,帝遣舒王到第傳詔省問,子儀不能興,叩頭謝恩。薨,年八十五。帝悼痛,廢朝五日。詔群臣往吊,隨喪所須,皆取於官。贈太師。陪葬建陵。及葬,帝御安福門,哭過其喪,百官陪位流涕。賜謚曰忠武,配饗代宗廟廷。著令,一品墳崇丈八尺,詔特增丈,以表元功。



 子儀事上誠,御下恕,賞罰必信。遭幸臣程元振、魚朝恩短毀,方時多虞,握兵處外,然詔至,即日就道,無纖介顧望,故讒間不行。破吐蕃靈州,而朝恩使人發其父墓,盜未得。子儀自涇陽來朝,中外懼有變,及入見,帝唁之,即號泣曰:「臣久主兵,不能禁士殘人之墓,人今發先臣墓,此天譴,非人患也。」朝恩又嘗約子儀修具,元載使人告以軍容將不利公。其下衷甲願從,子儀不聽,但以家僮十數往。朝恩曰:「何車騎之寡?」告以所聞。朝恩泣曰:「非公長者,得無致疑乎?」田承嗣傲狠不軌,子儀嘗遣使至魏,承嗣西望拜,指其膝謂使者曰:「茲膝不屈於人久矣,今為公拜。」李靈耀據汴州,公私財賦一皆遏絕,子儀封幣道其境,莫敢留,令持兵衛送。麾下宿將數十,皆王侯貴重,子儀頤指進退,若部曲然。幕府六十餘人,後皆為將相顯官,其取士得才類如此。與李光弼齊名,而寬厚得人過之。子儀歲入官俸無慮二十四萬緡。宅居親仁里四分之一,中通永巷,家人三千相出入,不知其居。前後賜良田、美器、名園、甲館不勝紀。代宗不名,呼為大臣。以身為天下安危者二十年,校中書令考二十四。八子七婿,皆貴顯朝廷。諸孫數十,不能盡識,至問安,但頷之而已。富貴壽考,哀榮終始,人臣之道無缺焉。



 子曜、旰、晞、昢、晤、曖、曙、映,而四子以才顯。



 曜,性沉靜,資貌瑰傑。累從節度府闢署,破虜有功,為開陽府果毅都尉。至德初,推子儀功,授衛尉卿,累進太子詹事、太原郡公。子儀專征伐,曜留治家事,少長無閑言。諸弟或飾池館,盛車服,曜獨以樸簡自處。子儀罷兵,遷太子少保,昆弟六人,共制拜官。子儀薨,以遺命簿上四朝所賜名馬珍物,德宗復賜之,乃悉散諸弟。居喪以禮,疾甚,或勸茹蔥薤,終不屬口。後盧杞秉政,忌勛族,子儀婿太僕卿趙縱、少府少監李洞清、光祿卿王宰皆以次得罪。奸人幸其危,多論奪田宅奴婢,曜大恐,獨宰相張鎰力保護。德宗稍聞之,詔有司曰:「尚父子儀有大勛力,保乂王家,嘗誓山河,琢金石,許宥十世。前日其家市田宅奴婢,而無賴者以尚父歿,妄論奪之,自今有司毋得受。」建中三年,卒,贈太子太傅,謚曰孝。初,曜襲代國公,食二千戶。貞元初,詔減半以封晞、曖、映、曙,人二百五十戶。未幾,復詔四人各減五十戶,封曜子鋒、晤子鐇各百戶云。



 晞,善騎射,從征伐有功,復兩京,戰最力,出奇兵破賊,累進鴻臚卿。河中軍亂,子儀召首惡誅之,其支黨猶反仄,晞選親兵晝夜警,以備非常,奸人不得發。以功拜殿中監。吐蕃、回紇入寇,加御史中丞,領朔方軍援邠州,與馬璘合軍擊虜,破之。虜復來,陣涇水北,子儀遣晞率徒兵五千、騎五百襲虜。晞以兵寡不進,須暮,賊半濟,乃擊,斬首五千級。加御史大夫,子儀固讓,乃止。居父喪,值硃泚亂,南走山谷。賊舁致之,欲污以官,佯暗不答;賊露兵脅之,不動。數以城中事貽書李晟。既而奔奉天。天子還,改太子賓客。子鋼,從朔方杜希全幕府。希全檄為豐州刺史,晞憐其弱不任事,丐罷。德宗遣使者召鋼,鋼疑得罪,挺身走吐蕃,不納。希全執送京師,賜死。晞坐免,尋復太子賓客。累封趙國公。卒,贈兵部尚書。孫承嘏。



 承嘏,字復卿,幼秀異,通《五經》。元和中,及進士第,累遷起居舍人。居母喪,以孝聞。太和六年,為諫議大夫,言政事得失。文宗以鄭注為太僕卿,承嘏極論其非,注頗懼。進給事中。俄出為華州刺史,給事中盧載還詔書,且言:「承嘏數封駁稱職,宜在禁闥。」帝曰:「朕謂久次,欲優其稍入耳。」乃復留給事中。時江淮旱,用度不支,詔宰相分領度支、戶部。承嘏言:「宰相調和陰陽,安黎庶。若使閱視簿書,校緡帛,非所宜。」帝順納。遷刑部侍郎。帝嘗稱其儒素,無貴驕氣,不類勛家。每進對,恩接備厚。方大任用,會卒。家無餘貲,親友為辦喪祭。贈吏部尚書。



 曖,字曖,以太常主簿尚昇平公主。曖年與公主侔,十餘歲許昏。拜駙馬都尉,試殿中監,封清源縣侯,寵冠戚里。大歷末,檢校左散騎常侍。建中時,主坐事,留禁中。硃泚亂,逼署曖官,辭以居喪被疾。既而與公主奔奉天。德宗嘉之,釋主罪,進曖金紫光祿大夫,賜實封五十戶。尋遷太常卿。貞元三年,襲代國公。卒,年四十八,贈尚書左僕射,初,曖女為廣陵郡王妃。王即位,是為憲宗。妃生穆宗。穆宗立,尊妃為皇太后,贈曖太傅。四子:鑄、釗、鏦、銛。鑄襲封。



 釗,長七尺,方口豐下。代宗朝,以外孫為奉禮郎。累官至左金吾大將軍,改檢校工部尚書,為邠寧節度使,入為司農卿。憲宗寢疾,宦豎或妄議廢立者。穆宗問計於釗,答曰:「殿下為太子,當旦夕視膳,何外慮乎?」時稱得元舅體。穆宗即位,檢校戶部尚書兼司農卿。俄為河陽三城節度使。徙河中尹,領晉絳慈隰節度。敬宗立,召拜兵部尚書,又帥劍南東川。太和中,南蠻寇蜀,取成都外郛,杜元穎不能御,詔釗兼領西川節度。未行,蠻眾已略梓州。州兵寡,不可用。釗貽書譙蠻首帟巔以侵叛意。帟巔曰:「元穎不自守,數侵吾圉,我以是報。」乃與修好,約無相犯。天子嘉之,即拜西川節度使。以疾請代,為太常卿,卒,贈司徒。子仲文、仲恭、仲詞。開成二年,詔仲文襲太原郡公。給事中盧弘宣奏:「劍妻沈,公主女,代宗皇帝外孫,其子仲詞尚饒陽公主。仲文冒嫡不應襲。使仲文承嫡,則沈當黜,且仲詞亦不得尚主。」乃詔仲詞檢校殿中少監、駙馬都尉,襲封。而仲文以太皇太后故,置不問。仲恭歷詹事府丞,亦尚金堂公主。



 鏦,字利用,尚德陽郡主。詔裴延齡為主營第長興里。順宗立,主進封漢陽公主,擢鏦檢校國子祭酒、駙馬都尉。自景龍後,外戚多為檢校官,不治事。宰相薦其才,不當以外戚廢,乃拜右金吾將軍,封太原郡公。恭遜折節,不以富貴加人。性周畏,不立赫赫名。有諫於上,退必毀稿,家人子弟無知者。別墅在都南,尤勝塏,穆宗嘗幸之,置酒極歡。改太子詹事,充閑廄宮苑使。卒,贈尚書左僕射。



 銛,性和易,累為殿中監,尚西河公主。鏦卒,代為太子詹事、宮苑閑廄使。長慶三年,暴卒。太后遣使按問發疾狀,久乃解。初,西河主降沈氏,生一子,銛無嗣,以沈氏子嗣。



 曙,代宗朝累官司農卿。德宗幸奉天,曙方領家兵獵苑北,聞蹕至,伏謁道左,遂從乘輿入駱谷。霖雨塗潦,衛兵或異語。帝召謂曰:「朕不德而苦公等,宜執朕送硃泚,以謝天下。」諸將皆感泣曰:「願死生從陛下。」時曙與功臣子李昇、韋清、令狐建、李彥輔被甲請見,言曰:「南行路險,且虞奸變。臣等世蒙恩,今相誓,願更挾帝馬。」許之。帝還,曙、清擢金吾大將軍,餘並為禁軍將軍。曙終祁國公。



 子儀母弟幼明,性謹願無過,拙於武,喜賓客。以子儀故,終少府監,贈太子太傅。



 子昕,肅宗末為四鎮留後。關、隴陷,不得歸,朝廷但命官遙領其使。建中二年,昕始與伊西、北庭節度使曹令忠遣使入朝。德宗詔曰:「四鎮、二庭,統西夏五十七蕃十姓部落,國朝以來,相與率職。自關、隴失守,王命阻絕,忠義之徒,泣血固守,奉遵朝法,此皆侯伯守將交修共治之效,朕甚嘉之。令忠可北廷大都護、四鎮節度留後,賜氏李,更名元忠。昕可安西大都護、四鎮節度使。諸將吏超七資敘官」云。



 贊曰:天寶末,盜發幽陵,外阻內訌。子儀自朔方提孤軍,轉戰逐北,誼不還顧。當是時,天子西走,唐祚若贅斿,而能輔太子,再造王室。及大難略平,遭讒甚,詭奪兵柄,然朝聞命,夕引道,無纖介自嫌。及被圍涇陽,單騎見虜,壓以至誠,猜忍沮謀。雖唐命方永,亦由忠貫日月,神明扶持者哉!及光弼等畏偪不終,而子儀完名高節,爛然獨著,福祿永終,雖齊桓、晉文比之為褊。唐史臣裴垍稱:「權傾天下而朝不忌,功蓋一世而上不疑,侈窮人欲而議者不之貶。」嗚呼!垍誠知言。其子孫多以功名顯,蓋盛德後云。



\end{pinyinscope}