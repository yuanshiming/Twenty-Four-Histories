\article{本紀第十 昭宗 哀帝}

\begin{pinyinscope}

 昭宗聖穆景文孝皇帝諱曄,懿宗第七子也。母曰恭憲皇太后王氏。始封壽王。乾符三年,領幽州盧龍軍節度使。僖宗遇亂再出奔,壽王握兵侍左右,尤見倚信。



 文德元年三月,僖宗疾大漸,群臣以吉王長,且欲立之。觀軍容使楊復恭率兵迎壽王,立為皇太弟,改名敏。乙巳,即皇帝位於柩前。四月戊辰,孫儒陷揚州,自稱淮南節度使,楊行密奔於廬州。庚午,追尊母為皇太后。韋昭度為中書令,孔緯為司空。乙亥,張全義陷孟州,李罕之奔於河東。成汭陷江陵,自稱留後。辛卯,硃全忠及秦宗權戰於蔡州,敗之。五月壬寅,趙德諲以襄州降,以德諲為忠義軍節度使、蔡州四面行營副都統。六月,閬州防禦使王建陷漢州,執刺史張頊,遂寇成都。韋昭度罷為劍南西川節度副大使,兼兩川招撫制置使。十月,陳敬瑄反。辛卯,葬惠聖恭定孝皇帝於靖陵。十一月丙申,秦宗權陷許州,執忠武軍節度使王縕。辛酉,奉國軍將申叢執秦宗權。十二月丁亥,韋昭度為行營招討使,及永平軍節度使王建討陳敬瑄。山南西道節度使楊守厚陷夔州。



 龍紀元年正月癸巳,大赦,改元。翰林學士承旨、兵部侍郎劉崇望同中書門下平章事。壬子,宣武軍將郭璠殺奉國軍留後申叢,自稱留後。二月戊辰,硃全忠俘秦宗權以獻。己丑,宗權伏誅。三月,孔緯為司徒,杜讓能為司空。丙申,錢升陷蘇州,逐刺史徐約。六月,李克用寇邢州。昭義軍節度使孟方立卒,其弟遷自稱留後。楊行密陷宣州,宣歙觀察使趙鍠死之。廬州刺史蔡儔叛附於孫儒。八月甲戌,孟遷叛附於李克用。十月,平盧軍節度使王敬武卒,其子師範自稱留後,陷棣州,刺史張蟾死之。宣歙觀察使楊行密陷常州,刺史杜陵死之。錢升陷潤州。十一月丁未,朝獻於太清宮。戊申,朝享於太廟。己酉,有事於南郊,大赦。十二月,孫儒陷常、潤二州。戊午,孫緯為太保,杜讓能為司徒。壬申,眉州刺史山行章叛附於王建。



 大順元年正月戊子,群臣上尊號曰聖文睿德光武弘孝皇帝,大赦,改元。壬寅,簡州將杜有遷執其刺史員虔嵩,叛附於王建。二月己未,資州將侯元綽執其刺史楊戡,叛附於建。三月戊申,昭義軍節度使李克修卒,其弟克恭自稱留後。四月丙辰,宿州將張筠逐其刺史張紹光。丙寅,嘉州刺史硃實叛附於王建。丙子,戎州將文武堅執其刺史謝承恩,叛附於建。五月,張浚為河東行營都招討宣慰使,京兆尹孫揆副之;幽州盧龍軍節度使李匡威為北面招討使,雲州防禦使赫連鐸副之;硃全忠為南面招討使,王熔為東面招討使,以討李克用。壬寅,昭義軍將安居受殺其節度使李克恭,叛附於硃全忠。癸丑,劍南東川節度使顧彥朗卒,其弟彥暉自稱留後。六月辛酉,雅州將謝從本殺其刺史張承簡,叛附於王建。辛未,硃全忠為河東東面行營招討使。是月,河東將安知建以邢、洺、滋三州叛附於全忠。七月,楊行密陷潤州。戊申,李克用執昭義軍節度使孫揆。八月,錢升殺蘇州刺史杜孺休。楊行密陷蘇州。海南節度使孫儒陷潤州。庚午,硃全忠為中書令。九月,李克用陷潞州。楊行密陷潤、常二州。閏月,孫儒陷常州。壬戌,邛州將任可知殺其刺史毛湘。十月癸未,蜀州刺史李行周叛附於王建。李克用陷邢、洺、滋三州。十一月丁卯,李匡威陷蔚州。是月,張浚及李克用戰於陰地,敗績。孫儒陷蘇州。十二月,李克用陷晉州。



 二年正月庚申,孔緯、張浚罷。翰林學士承旨、兵部侍郎崔昭緯,御史中丞徐彥若為戶部侍郎:同中書門下平章事。甘露鎮使陳可言陷常州。錢升陷蘇州。二月乙巳,赦陳敬瑄。丁未,詔王建罷兵,不受命。是春,淮南大饑。四月庚辰,有彗星入於太微。甲申,大赦,避正殿,減膳,徹樂。賜兩軍金帛,贖所略男女還其家。民年八十以上及疾不能自存者,長吏存恤。訪武德功臣子孫。癸卯,王建寇成都。五月,孫儒陷和、滁二州。六月,楊行密陷和、滁二州。丙午,封子祐為德王。七月,李克用陷雲州,防禦使赫連鐸奔於退渾。孫儒焚揚州以逃。八月庚子,王建陷成都,執劍南西川節度使陳敬瑄,自稱留後。十月壬午,硃全忠陷宿州。十一月己未,曹州將郭銖殺其刺史郭詞,叛附於全忠。辛未,全忠陷壽州。



 景福元年正月己未,硃全忠陷孟州,逐河陽節度使趙克裕。丙寅,大赦,改元。二月,劉崇望罷。錢升陷蘇州。甲申,硃全忠寇鄆州,天平軍節度使硃宣敗之。三月,戶部尚書鄭延昌為中書侍郎、同中書門下平章事。乙巳,楊行密陷楚州,執刺史劉瓚;又陷常州,刺史陳可言死之。丙辰,武定軍節度使楊守忠、龍劍節度使楊守貞會楊守厚兵寇梓州。丙寅,福建觀察使陳巖卒,護閩都將範暉自稱留後。庚午,泉州刺史王潮寇福州。四月辛巳,杜讓能為太尉。六月戊寅,楊行密陷揚州。己巳,鳳翔隴右節度使李茂貞陷鳳州,感義軍節度使滿存奔於興元,遂陷興、洋二州。八月壬申,寇興元,楊守亮、滿存奔於閬州。丙戌,降京畿、關輔囚罪,免淮南、浙西、宣州逋負。十月,蔡儔以廬州叛附於硃全忠,河東將李存孝以邢州叛附於全忠。十一月,有星孛於斗、牛。辛丑,武寧軍將張燧、張諫以濠、泗二州叛附於硃全忠。乙巳,硃友裕陷濮州,執刺史邵儒。孫儒將王壇陷婺州,刺史蔣瑰奔於越州。是歲,明州刺史鐘文季卒,其將黃晟自稱刺史。



 二年正月,徐彥若罷為鳳翔隴右節度使,李茂貞為山南西道節度使。茂貞不受命。二月,楊行密陷常州。三月辛酉,幽州盧龍軍兵馬留後李匡籌逐其兄匡威,自稱節度留後。四月乙亥,王建殺陳敬瑄及劍南西川監軍田令孜。乙酉,有彗星入於太微。丁亥,王熔殺李匡威。戊子,硃全忠陷徐州,武寧軍節度使時溥死之。五月庚子,王潮陷福州,範暉死之,潮自稱留後。七月,楊行密陷廬州,蔡儔死之。八月丙申,嗣覃王嗣周為京西路招討使,神策大將軍李鐵副之,以討李茂貞。庚子,升州刺史張雄卒,其將馮弘鐸自稱刺史。是月,楊行密陷歙州。九月壬午,嗣覃王嗣周及李茂貞戰於興平,敗績。甲申,茂貞犯京師。乙酉,茂貞殺觀軍容使西門重遂、內樞密使李周言童、段詡。貶杜讓能為梧州刺史。壬辰,東都留守、檢校司徒韋昭度為司徒,御史中丞崔胤為戶部侍郎:同中書門下平章事。是月,升州刺史馮弘鐸叛附於楊行密。十月乙未,殺杜讓能及戶部侍郎杜弘徽。楊行密陷舒州。十二月,韋昭度為太傅。邵州刺史鄧處訥陷潭州,欽化軍節度使周岳死之,處訥自稱留後。是歲,建州刺史徐歸範、汀州刺史鐘全慕叛附於王潮。



 乾寧元年正月,有星孛于鶉首。乙丑,大赦,改元。李茂貞以兵來朝。二月,右散騎常侍鄭綮為禮部侍郎、同中書門下平章事。彰義軍節度使張鈞卒,其兄金番自稱留後。三月甲申,李克用寇邢州,執李存孝殺之。五月丙子,王建陷彭州,威戎軍節度使楊晟死之。是月,鄭延昌罷。孫儒將劉建鋒、馬殷陷潭州,武安軍節度使鄧處訥死之,建鋒自稱留後。武岡指揮使蔣勛陷邵州。六月,大同軍防禦使赫連鐸及李克用戰於雲州,死之。戊午,翰林學士承旨、禮部尚書李磎同中書門下平章事。庚申,磎罷。御史大夫徐彥若為中書侍郎、同中書門下平章事。七月,以雨霖避正殿,減膳。鄭綮罷。李茂貞陷閬州。八月,楊守亮伏誅。癸巳,減京畿、興元、洋金商州賦役。九月庚申,李克用陷潞州,昭義軍節度使康君立死之。十月丁酉,封子祤為棣王,禊虔王,禋沂王,禕遂王。十一月,李克用陷武州。十二月,陷新州。甲寅,幽州盧龍軍節度使李匡籌奔於滄州,義昌軍節度使盧彥威殺之。丙辰,李克用陷幽州。是冬,楊行密陷黃州,執刺史吳討。



 二年正月己巳,給事中陸希聲為戶部侍郎、同中書門下平章事。壬申,護國軍節度使王重盈卒,其子珂自稱留後。二月乙未,太子太傅李磎為戶部侍郎、同中書門下平章事。三月,崔胤、李磎罷。戶部侍郎、判戶部王搏為中書侍郎、同中書門下平章事。楊行密陷濠州,執刺史張燧。庚午,河東地震。四月,蘇州大雨雪。陸希聲、韋昭度罷。泰寧軍節度使硃瑾及硃全忠戰於高梧,敗績,其將安福慶死之。楊行密陷壽州,執刺史江從勖。五月甲子,靜難軍節度使王行瑜、鎮國軍節度使韓建及李茂貞犯京師,殺太保致仕韋昭度、太子少師李磎。是月,李克用陷絳州,刺史王瑤死之。六月庚寅,鎮海軍節度使錢升為浙江東道招討使。癸巳,吏部尚書孔緯為司空,兼門下侍郎、同中書門下平章事。七月丙辰,李克用以兵屯於河中。戊午,匡國軍節度使王行約奔於京師。庚申,左右神策軍護軍中尉駱全瓘劉景宣、指揮使王行實李繼鵬反。行在莎城。嗣薛王知柔權知中書事。壬戌,李克用陷同州。甲子,次石門。前護國軍節度使崔胤為中書侍郎、同中書門下平章事。八月戊戌,李克用為邠寧四面行營招討使,保大軍節度使李思孝為北面招討使,定難軍節度使李思諫為東北面招討使,彰義軍節度使張鐇為西面招討使。辛丑,李克用為邠寧四面行營都統。李繼鵬伏誅。赦李茂貞。辛亥,至自石門。壬子,崔昭緯罷。九月丙辰,徐彥若為司空。癸亥,孔緯薨。前昭義軍節度使李罕之為邠寧四面行營副都統。十月,京兆尹孫偓為戶部侍郎、同中書門下平章事。丙戌,李克用及王行瑜戰於梨園,敗之。庚寅,王行約焚寧州以逃。義武軍節度使王處存卒,其子郜自稱留後。十一月丁巳,李克用及王行瑜戰於龍泉,敗之。辛酉,衢州刺史陳儒卒,其弟岌自稱刺史。丁卯,王行瑜伏誅。壬申,齊州刺史硃瓊叛附於硃全忠。丁丑,王建陷利州,刺史李繼顒死之。十二月癸未,赦京師,復大順以來削奪官爵非其罪者。甲申,閬州防禦使李繼雍、蓬州刺史費存、渠州刺史陳璠叛附於王建。丙申,建寇梓州。戊辰,通州刺史李彥昭叛附於建。是歲,安州防禦使宣晟陷桂州,靜江軍節度使周元靜部將劉士政死之,晟自稱知軍府事。



 三年正月癸丑,王建陷龍州,刺史田昉死之。閏月丁亥,果州刺史周雄叛附於建。四月壬子,武安軍亂,殺其節度使劉建峰,其將馬殷自稱留後。五月癸未,楊行密陷蘇州,執刺史成及;陷光州,刺史劉存死之。庚寅,成汭陷黔州,武泰軍節度使王建肇奔於成都。乙未,董昌伏誅。是月,蘄州刺史馮行章叛附於楊行密。六月庚戌,李茂貞犯京師,嗣延王戒丕御之。丙寅,及茂貞戰於婁館,敗績。七月癸巳,行在渭北。甲午,韓建來朝,次華州。乙巳,崔胤罷。丙午,翰林學士承旨、尚書左丞陸扆為戶部侍郎、同中書門下平章事。八月甲寅,王摶罷。乙丑,國子《毛詩》博士硃樸為左諫議大夫、同中書門下平章事。九月乙未,武安軍節度使崔胤為中書侍郎,翰林學士承旨、戶部侍郎崔遠:同中書門下平章事。丁酉,貶陸扆為峽州刺史。十月,李克用及羅弘信戰於白龍潭,敗之。壬子,孫偓持節鳳翔四面行營節度、諸軍都統、招討、處置使。戊午,威勝軍節度使王摶為吏部尚書、同中書門下平章事。十一月戊子,忠國軍節度使李師悅卒,其子繼徽自稱留後。



 四年正月乙酉,韓建以兵圍行宮,殺扈蹕都將李筠。丙申,硃全忠陷鄆州,天平軍節度使硃宣死之。己亥,孫偓罷都統。二月,硃全忠寇兗州,泰寧軍節度使硃瑾奔於淮南,其子用貞以兗州叛附於全忠。全忠陷沂、海、密三州。保義軍節度使王珙寇河中。韓建殺太子詹事馬道殷、將作監許巖士。楊行密為江南諸道行營都統。癸丑,王建陷瀘州,刺史馬敬儒死之。己未,立德王裕為皇太子,太赦,饗於行廟。辛未,王建陷渝州。乙亥,孫偓、硃樸罷。五月壬午,硃全忠陷黃州,刺史矍璋死之。六月,貶王建為南州刺史。以李茂貞為劍南西川節度使,嗣覃王嗣周為鳳翔隴右節度使,茂貞不受命,嗣周及茂貞戰於奉天,敗績。八月,韓建殺通王滋、沂王禋、韶王、彭王、嗣韓王、嗣陳王、嗣覃王嗣周、嗣延王戒丕、嗣丹王允。九月,錢升陷湖州,忠國軍節度使李繼徽奔於淮南。彰義軍節度使張璉為鳳翔西北行營招討使,靜難軍節度使李思諫為鳳翔四面行營副都統,以討李茂貞。十月壬子,遂州刺史侯紹叛附於王建。乙卯,合州刺史王仁威叛附於建。庚申,建陷梓州,劍南東川節度使顧彥暉死之。甲子,封子秘為景王,祚輝王,祺祁王。十一月癸酉,楊行密及硃全忠戰於消口,敗之。丙子,錢升陷臺州。十二月丁未,威武軍節度使王潮卒,其弟審知自稱留後。



 光化元年正月,徐彥若為司徒。二月,赦李茂貞。三月,幽州盧龍軍節度使劉仁恭之子守文陷滄州,義昌軍節度使盧彥威奔於汴州。四月丙寅,立淑妃何氏為皇后。五月己巳,大赦。辛未,硃全忠陷洺州,刺史邢善益死之;又陷邢州。壬午,陷磁州,刺史袁奉韜死之。是月,馬殷陷邵、衡、永三州,刺史蔣勛、楊師遠、唐旻死之。七月丙申,硃全忠陷唐州,又陷隋州,執刺史趙匡璘。八月戊午,陷鄧州,執刺史國湘。壬戌,至自華州。甲子,大赦,改元。九月丙子,有星隕於北方。甲申,錢升陷蘇州。十月,魏博節度使羅弘信卒,其子紹威自稱留後。己亥,硃全忠陷安州,刺史武瑜死之。十一月,衢州刺史陳岌叛附於楊行密。甲寅,封子禛為雅王,祥瓊王。十二月癸未,李罕之陷潞州,自稱節度留後。李克用陷澤州。



 二年正月乙未,給復綿、劍二州二年。丁未,崔胤罷。兵部尚書陸扆同中書門下平章事。是月,李罕之陷沁州。劉仁恭陷貝州。二月甲子,硃全忠陷蔡州,奉國軍節度使崔洪奔於淮南。三月丁巳,全忠陷澤州。六月丁丑,保義軍亂,殺其節度使王珙,其將李璠叛附於全忠。七月壬辰,海州戍將陳漢賓以其州叛附於楊行密。馬殷陷道州,刺史蔡結死之。八月,李克用陷澤、潞、懷三州。十一月,徐彥若為太保,王摶為司空。馬殷陷郴、連二州,刺史陳彥謙、魯景仁死之。辛丑,保義軍將硃簡殺其節度使李璠,叛附於硃全忠。



 三年四月辛未,皇后及皇太子享於太廟。六月丁卯,清海軍節度使崔胤為尚書左僕射,兼門下侍郎、同中書門下平章事。王摶罷。己巳,殺之。七月,浙江溢。八月庚辰,李克用陷洺州,執刺史硃紹宗。九月,硃全忠陷洺州。錢升陷婺州,刺史王壇奔於宣州。衢州刺史陳岌叛附於錢升。乙巳,徐彥若罷。丙午,崔遠罷。戊申,刑部尚書裴贄為中書侍郎、同中書門下平章事。甲寅,硃全忠陷瀛州。十月丙辰,陷景州,執刺史劉仁霸。辛酉,陷莫州。辛巳,陷祁州,刺史楊約死之。甲申陷定州,義武軍節度使王郜奔於太原。十一月己丑,左右神策軍中尉劉季述、王仲先、內樞密使王彥範、薛齊偓作亂,皇帝居於少陽院。辛卯,季述以皇太子裕為皇帝。丁未,太白晝見。十二月,劉季述殺睦王倚。是歲,馬殷陷桂、宜、巖、柳、象五州。睦州刺史陳晟卒,其弟詢自稱刺史。



 天復元年正月乙酉,左神策軍將孫德昭、董彥弼、周承誨以兵討亂,皇帝復於位。劉季述、薛齊偓伏誅,降封皇太子裕為德王。戊申,硃全忠陷絳州。壬子,崔胤為司空。硃全忠陷晉州。二月甲寅,以旱避正殿,減膳。戊辰,硃全忠陷河中,執護國軍節度使王珂。辛未,封全忠為梁王。是月,翰林學士、戶部侍郎王溥為中書侍郎,吏部侍郎裴樞為戶部侍郎:同中書門下平章事。三月辛亥,昭義軍節度使孟遷叛附於硃全忠。四月壬子,全忠陷沁、澤二州。丁巳,儀州刺史張鄂叛附於全忠。甲戌,享於太廟。丙子,大赦,改元。武德、貞觀配饗功臣主祭子孫敘進之,介公、酅公後予一子九品正員官。免光化以來畿內逋負。五月,李茂貞來朝。六月,李克用陷隰、慈二州。十月戊戌,硃全忠犯京師。十一月己酉,陷同州。壬子,如鳳翔。丁巳,硃全忠陷華州,鎮國軍節度使韓建叛附於全忠。辛酉,兵部侍郎盧光啟權句當中書事。癸亥,李茂貞及硃全忠戰於武功,敗績。丁卯,盧光啟為右諫議大夫,參知機務。戊辰,硃全忠犯鳳翔。辛未,陷邠州,靜難軍節度使李繼徽叛附於全忠。甲戌,崔胤、裴樞罷。十二月,鐘傳陷吉州。是歲,清海軍節度使徐彥若卒,行軍司馬劉隱自稱留後。武貞軍節度使雷蒲卒,其子彥威自稱留後。



 二年正月丁卯,給事中韋貽範為工部侍郎、同中書門下平章事。丙子,給事中嚴龜為汴、岐和協使。二月己亥,盜發簡陵。王建陷利州,昭武軍節度使李繼忠奔於鳳翔。三月庚戌,晝晦。癸丑,硃全忠陷汾州。乙卯,浙西大雨雪。戊午,硃全忠陷慈、隰二州。丁卯,李克用陷汾、慈、隰三州。四月,盧光啟罷。丙申,溫州刺史硃褒卒,其兄敖自稱刺史。楊行密陷升州。五月丙午,李茂貞及硃全忠戰於武功,敗績。庚午,韋貽範罷。六月丙子,中書舍人蘇檢為工部侍郎、同中書門下平章事。丙戌,硃全忠陷鳳州。七月甲辰,陷成州。乙巳,陷隴州。八月己亥,韋貽範起復。辛丑,王建陷興元,山南西道節度使王萬弘叛附於建。九月戊申,李茂貞及硃全忠戰於槐林,敗績。武定軍節度使拓拔思恭叛附於王建。十月癸酉,楊行密為東面諸道行營都統,及湖南節度使馬殷討硃全忠。王建陷興州。十一月癸卯,保大軍節度使李茂勛以兵援鳳翔。丙辰,韋貽範薨。十二月癸巳,溫州將丁章逐其刺史硃敖。己亥,硃全忠陷鄜州,保大軍節度使李茂勛叛附於全忠。是歲,盧光稠陷韶州。岳州刺史鄧進思卒,其弟進忠自稱刺史。



 三年正月丙午,平盧軍節度使王師範取兗州。戊申,殺左右神策軍護軍中尉韓全誨張彥弘、內樞密使袁易簡周敬容。辛亥,翰林學士姚洎為汴、岐和協使。壬子,工部尚書崔胤為司空,兼門下侍郎、同中書門下平章事。甲子,幸硃全忠軍。己巳,至自鳳翔,哭於太廟,大赦。庚午,崔胤及硃全忠殺中官七百餘人。辛未,胤判六軍十二衛事。丁章伏誅。二月,雨土。甲戌,貶陸扆為沂王傅,分司東都。丙子,王溥罷。硃全忠殺蘇檢、吏部侍郎盧光啟。戊寅,降京畿、河中鳳翔興德府、同邠鄜三州死罪以下。己卯,輝王祚為諸道兵馬都元帥;庚辰,硃全忠為太尉、中書令副之。崔胤為司徒。乙未,清海軍節度使裴樞為門下侍郎、同中書門下平章事。三月,硃全忠陷青州。楊行密陷密州,刺史劉康乂死之。酉月己卯,硃全忠判元帥府事。五月壬子,荊南節度使成汭及楊行密戰於君山,死之。武貞軍節度使雷彥威之弟彥恭陷江陵。六月乙亥,硃全忠陷登州。九月,楊行密殺奉國軍節度使硃延壽。辛亥,硃全忠陷棣州,刺史邵播死之;陷密州。戊午,平盧軍節度使王師範叛附於全忠。十月,忠義軍將趙匡明陷江陵,自稱留後。王建陷忠、萬、施三州。甲戌,陷夔州。丁丑,平盧軍將劉鄩以兗州叛附於硃全忠。十二月,裴贄罷。楊行密陷宣州,寧國軍節度使田頵死之。辛巳,禮部尚書獨孤損為兵部侍郎、同中書門下平章事。丙申,硃全忠殺尚書左僕射致仕張浚。



 天祐元年正月乙巳,崔胤罷。裴樞判左三軍事,獨孤損判右三軍事。兵部尚書崔遠為中書侍郎,翰林學士、右拾遺柳璨為右諫議大夫:同中書門下平章事。己酉,硃全忠殺太子少傅崔胤及京兆尹鄭元規、威遠軍使陳班。戊午,全忠遷唐都於洛陽。二月丙寅,日中見北斗。戊寅,次陜州。硃全忠來朝。甲申,封子禎為端王,祁豐王,福和王,禧登王,祐嘉王。三月丁未,硃全忠兼判左右神策及六軍諸衛事。閏四月壬寅,次穀水。硃全忠來朝。甲辰,至自西都。享於太廟。大風,雨土。乙巳,大赦,改元。六月,靜難軍節度使楊崇本會李克用、王建兵以討硃全忠。七月乙丑,全忠以兵屯於河中。八月壬寅,全忠以左右龍武統軍硃友恭、氏叔琮、樞密使蔣玄暉兵犯宮門;是夕,皇帝崩,年三十八。明年,起居郎蘇楷請更謚「恭靈莊閔」,廟號襄宗。至後唐同光初,復故號謚雲。



 昭宣光烈孝皇帝諱祝,昭宗第九子也。母曰皇太后何氏。始封輝王。硃全忠已弒昭宗,矯詔立為皇太子,監軍國事。



 天祐元年八月丙午,即皇帝位於柩前。衢州刺史陳璋、睦州刺史陳詢叛附於楊行密。九月庚午,尊皇后為皇太后。十月辛卯朔,日有食之。癸巳,硃全忠來朝。甲午,全忠殺硃友恭、氏叔琮。十一月,全忠陷光州。是歲,虔州刺史盧光稠卒,衙將李圖自稱知州事。



 二年正月,盧約陷溫州。楊行密殺平盧軍節試使安仁義。丁丑,盜焚乾陵下宮。二月,楊行密陷鄂州,武昌軍節度使杜洪死之。戊戌,硃全忠殺德王裕及棣王祤、虔王禊、遂王禕、景王秘、祁王祺、瓊王祥。己酉,葬聖穆景文孝皇帝於和陵。三月甲子,裴樞罷。戊寅,獨孤損罷。禮部侍郎張文蔚同中書門下平章事。甲申,崔遠罷。吏部侍郎楊涉同中書門下平章事。四月乙未,以旱避正殿,減膳。庚子,有彗星出於酉北;甲辰,出於北河。辛亥,降京畿死罪以下,給復山陵役者一年。五月,王建陷金州,戎昭軍節度使馮行襲奔於均州。六月,行襲克金州。楊行密陷婺州,執刺史沈夏。戊子,硃全忠殺裴樞及靜海軍節度使獨孤損、左僕射崔遠、吏部尚書陸扆、工部尚書王溥、司空致仕裴贄、檢校司空兼太子太保致仕趙崇、兵部侍郎王贊。七月,卜郊。岳州刺史鄧進忠叛附於馬殷。九月甲子,硃全忠陷襄州,忠義軍節度使趙匡凝奔於淮南。丙寅,封弟裕為潁王,祐蔡王。硃全忠陷江陵,留後趙匡明奔於成都。乙酉,改卜郊。十月丙戌,硃全忠為諸道兵馬元帥。十一月庚午,三卜郊。庚辰,淮南節度使楊行密卒,以其子渥為淮南節度副大使、東面諸道行營都統。辛巳,硃全忠為相國,總百揆,封魏王。十二月乙未,全忠為天下兵馬元帥,殺蔣玄暉及豐德庫使應頊、尚食使硃建武。癸卯柳璨為司空。戊申,硃全忠殺皇太后。辛亥,罷郊。癸丑,貶柳璨為登州刺史。甲寅,殺璨及太常卿張廷範。



 三年正月壬戌,淮南將王茂章以宣、歙二州叛附於錢升。二月,楊渥陷岳州。癸巳,王建陷歸州。四月癸未朔,日有食之。鎮南軍節度使鐘傳卒,其子匡時自稱留後。六月,錢升陷衢、睦二州,刺史陳璋、陳詢奔於淮南。七月,楊渥陷饒州。八月癸未,硃全忠陷相州。九月,楊渥陷洪州,執鐘匡時。乙亥,匡國軍節度使劉知俊陷坊州,執刺史劉彥暉。十月辛巳,楊崇本會鳳翔、涇原、鄜延、秦隴兵以討硃全忠,戰於美原、敗績。十一月,忠國軍節度使高彥卒,其子澧自稱留後。閏十二月戊辰,李克用陷潞州,昭義軍節度使丁會叛附於克用。乙亥,震電,雨雪。



 四年三月,劉守光囚其父仁恭,自稱幽州盧龍軍節度使。四月戊午,錢升陷溫州。甲子,皇帝遜於位,徙於曹州,號濟陰王。梁開平二年二月遇殺,年十七,謚曰哀帝。後唐明宗追謚昭宣光烈孝皇帝,陵曰溫陵。



 贊曰:自古亡國,未必皆愚庸暴虐之君也。其禍亂之來有漸積,及其大勢巳去,適丁斯時,故雖有智勇,有不能為者矣,可謂真不幸也,昭宗是已。昭宗為人明雋,初亦有志於興復,而外患已成,內無賢佐,頗亦慨然思得非常之材,而用匪其人,徒以益亂。自唐之亡也,其遺毒餘酷,更五代五十餘年,至於天下分裂,大壞極亂而後止。跡其禍亂,其漸積豈一朝一夕哉!



\end{pinyinscope}