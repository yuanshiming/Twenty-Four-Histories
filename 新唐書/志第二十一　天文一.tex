\article{志第二十一 天文一}

\begin{pinyinscope}

 昔者,堯命羲、和,出納日月,考星中以正四時。至舜,則曰「在璿璣玉衡,以齊七政」而已。雖二典質略,存其大法,亦由古者天人之際,推候占測,為術猶簡。至於後世,其法漸密者。必積眾人之智,然後能極其精微哉。蓋自三代以來詳矣。詩人所記,婚禮、土功必候天星。而《春秋》書日食、星變,《傳》載諸國所占次舍、伏見、逆順。至於《周禮》測景求中、分星辨國、妖祥察候,皆可推考,而獨無所謂璿璣玉衡者,豈其不用於三代耶?抑其法制遂亡,而不可復得耶?不然,二物者,莫知其為何器也。至漢以後,表測景晷,以正地中,分列境界,上當星次,皆略依古。而又作儀以候天地,而渾天、周髀、宣夜之說,至於星經、歷法,皆出於數術之學。唐興,太史李淳風、浮圖一行,尤稱精博,後世未能過也。故採其要說,以著於篇。至於天象變見所以譴告人君者,皆有司所宜謹記也。



 貞觀初,淳風上言:「舜在璿璣玉衡,以齊七政,則渾天儀也。《周禮》,土圭正日景以求地中,有以見日行黃道之驗也。暨於周末,此器乃亡。漢落下閎作渾儀,其後賈逵、張衡等亦各有之,而推驗七曜,並循赤道。按冬至極南,夏至極北,而赤道常定於中,國無南北之異。蓋渾儀無黃道久矣。」太宗異其說,因詔為之。至七年儀成。表裏三重,下據準基,狀如十字,末樹鰲足,以張四表。一曰六合儀,有天經雙規、金渾緯規、金常規,相結於四極之內。列二十八宿、十日、十二辰、經緯三百六十五度。二曰三辰儀,圓徑八尺,有璿璣規、月游規,列宿距度,七曜所行,轉於六合之內。三曰四游儀,玄樞為軸,以連結玉衡游筩而貫約矩規。又玄極北樹北辰,南矩地軸,傍轉於內。玉衡在玄樞之間,而南北游,仰以觀天之辰宿,下以識器之晷度。皆用銅。帝稱善,置於凝暉閣,用之測候。閣在禁中,其後遂亡。



 開元九年,一行受詔,改治新歷,欲知黃道進退,而太史無黃道儀,率府兵曹參軍梁令瓚以木為游儀,一行是之,乃奏:「黃道游儀,古有其術而無其器,昔人潛思,皆未能得。今令瓚所為,日道月交,皆自然契合,於推步尤要,請更鑄以銅鐵。」十一年儀成。一行又曰:「靈臺鐵儀,後魏斛蘭所作,規制樸略,度刻不均,赤道不動,乃如膠柱。以考月行,遲速多差,多或至十七度,少不減十度,不足以稽天象、授人時。李淳風黃道儀,以玉衡旋規,別帶日道,傍列二百四十九交,以攜月游,法頗難,術遂寢廢。臣更造游儀,使黃道運行,以追列舍之變,因二分之中,以立黃道,交於奎、軫之間,二至陟降,各二十四度。黃道內施白道月環,用究陰陽朓,朒,動合天運。簡而易從,可以制器垂象,永傳不朽。」於是玄宗嘉之,自為之銘。



 又詔一行與令瓚等更鑄渾天銅儀,圓天之象,具列宿赤道及周天度數。注水激輪,令其自轉,一晝夜而天運周。外絡二輪,綴以日月,令得運行。每天西旋一周,日東行一度,月行十三度十九分度之七,二十九轉有餘而日月會,三百六十五轉而日周天。以木櫃為地平,令儀半在地下,晦明朔望遲速有準。立木人二於地平上:其一前置鼓以候刻,至一刻則自擊之;其一前置鐘以候辰,至一辰亦自撞之。皆於櫃中各施輪軸,鉤鍵關鎖,交錯相持。置於武成殿前,以示百官。無幾而銅鐵漸澀,不能自轉,遂藏於集賢院。



 其黃道游儀,以古尺四分為度。旋樞雙環,其表一丈四尺六寸一分,縱八分,厚三分,直徑四尺五寸九分,古所謂旋儀也。南北科兩極,上下循規各三十四度。表裏畫周天度,其一面加之銀釘。使東西運轉,如渾天游旋。中旋樞軸,至兩極首內,孔徑大兩度半,長與旋環徑齊。玉衡望筩,長四尺五寸八分,廣一寸二分,厚一寸,孔徑六分。衡旋於軸中,旋運持正,用窺七曜及列星之闊狹。外方內圓,孔徑一度半,周日輪也。陽經雙環,表一丈七尺三寸,里一丈四尺六寸四分,廣四寸,厚四分,直徑五尺四寸四分,置於子午。左右用八柱,八柱相固。亦表裏畫周天度,其一面加之銀釘。半出地上,半入地下。雙間挾樞軸及玉衡望筩旋環於中也。陰緯單環,外內廣厚周徑,皆準陽經,與陽經相銜各半,內外俱齊。面平,上為天,下為地。橫周陽環,謂之陰渾也。平上為兩界,內外為周天百刻。天頂單環,表一丈七尺三寸,縱廣八尺,厚三分,直徑五尺四寸四分。直中國人頂之上,東西當卯酉之中,稍南,使見日出入。令與陽經、陰緯相固,如鳥殼之裹黃。南去赤道三十六度,去黃道十二度,去北極五十五度,去南北平各九十一度強。赤道單環,表一丈四尺五寸九分,橫八分,厚三分,直徑四尺五寸八分。赤道者,當天之中,二十八宿之位也。雙規運動,度穿一穴。古者,秋分日在角五度,今在軫十三度;冬至日在牽牛初,今在斗十度。隨穴退交,不復差繆。傍在卯酉之南,上去天頂三十六度,而橫置之。黃道單環,表一丈五尺四寸一分,橫八分,厚四分,直徑四尺八寸四分。日之所行,故名橫道。太陽陟降,積歲有差。月及五星,亦隨日度出入。古無其器,規制不知準的,斟酌為率,疏闊尤甚。今設此環,置於赤道環內,仍開合使運轉,出入四十八度,而極畫兩方,東西列周天度數,南北列百刻,可使見日知時。上列三百六十策,與用卦相準。度穿一穴,與赤道相交。白道月環,表一丈五尺一寸五分,橫八分,厚三分,直徑四尺七寸六分。用行有迂曲遲速,與日行緩急相及。古亦無其器,今設於黃道環內,使就黃道為交合,出入六度,以測每夜月離,上畫周天度數,度穿一穴,擬移交會。皆用鋼鐵。游儀,四柱為龍,其崇四尺七寸,水槽及山崇一尺七寸半,槽長六尺九寸,高、廣皆四寸,池深一寸,廣一寸半。龍能興雲雨,故以飾柱。柱在四維。龍下有山、雲,俱在水平槽上。皆用銅。



 其所測宿度與古異者:舊經,角距星去極九十一度,亢八十九度,氐九十四度,房百八度,心百八度,尾百二十度,箕百一十八度,南斗百一十六度,牽牛百六度,須女百度,虛百四度,危九十七度,營室八十五度,東壁八十六度,奎七十六度,婁八十度,胃、昴七十四度,畢七十八度,觜觿、八十四度,參九十四度,東井七十度,輿鬼六十八度,柳七十七度,七星九十一度,張九十七度,翼九十七度,軫九十八度。今測,角九十三度半,亢九十一度半,氐九十八度,房百一十度半,心百一十度,尾百二十四度,箕百二十度,南斗百一十九度,牽牛百四度,須女百一度,虛百一度,危九十七度,營室八十三度,東壁八十四度,奎七十三度,婁七十七度,胃、昴七十二度,畢七十六度,觜觿八十二度,參九十三度,東井六十八度,輿鬼六十八度,柳八十度半,七星九十三度半,張百度,翼百三度,軫百度。



 又舊經,角距星正當赤道,黃道在其南;今測,角在赤道南二度半,則黃道復經角中,與天象合。虛北星舊圖入虛,今測在須女九度。危北星舊圖入危,今測在虛六度半。又奎誤距以西大星,故壁損二度,奎增二度;今復距西南大星,即奎、壁各得本度。畢、赤道十六度,黃道亦十六度。觜觿,赤道二度,黃道三度。二宿俱當黃道斜虛,畢尚與赤道度同,觜觿總二度,黃道損加一度,蓋其誤也。今測畢十七度半,觜觿半度。又柳誤距以第四星,今復用第三星。張中央四星為硃鳥膆,外二星為翼,北距以翼而不距以膺,故張增二度半,七星減二度半;今復以膺為距,則七星、張各得本度。



 其他星:舊經,文昌二星在輿鬼,四星在東井。北斗樞在七星一度,璿在張二度,機在翼二度,權在翼八度,衡在軫八度,開陽在角七度,杓在亢四度。天關在黃道南四度,天尊、天槨在黃道北,天江、天高、狗國、外屏、雲雨、虛梁在黃道外,天囷、土公吏在赤道外,上臺在東井,中臺在七星,建星在黃道北半度,天苑在昴、畢,王良在壁,外屏在觜觿,雷電在赤道外五度,霹靂在赤道外四度,八魁在營室,長垣、羅堰當黃道。今測,文昌四星在柳,一星在輿鬼,一星在東井。北斗樞在張十三度,璿在張十二度半,機在翼十三度,權在翼十七度太,衡在軫十度半,開陽在角四度少,杓在角十二度少。天關、天尊、天槨、天江、天高、狗國、外屏,皆當黃道。雲雨在黃道內七度,虛梁在黃道內四度,天囷當赤道,土公吏在赤道內六度,上臺在柳,中臺在張,建星在黃道北四度半,天苑在胃、昴,王良四星在奎,一星在壁,外屏在畢,雷電在赤道內二度,霹靂四星在赤道內,一星在外,八魁五星在壁,四星在營室,長垣在黃道北五度,羅堰在黃道北。



 黃道,春分與赤道交於奎五度太;秋分交於軫十四度少;冬至在斗十度,去赤道南二十四度;夏至在井十三度少,去赤道北二十四度。其赤道帶天之中,以分列宿之度。黃道斜運,以明日月之行。乃立八節、九限,校二道差數,著之歷經。



 蓋天之說,李淳風以為天地中高而四頹,日月相隱蔽,以為晝夜。繞北極常見者謂之上規,南極常隱者謂之下規,赤道橫絡者謂之中規。及一行考月行出入黃道,為圖三十六,究九道之增損,而蓋天之狀見矣。



 削篾為度,徑一分,其厚半之,長與圖等,穴其正中,植針為樞,令可環運。自中樞之外,均刻百四十七度。全度之末,旋為外規。規外太半度,再旋為重規。以均賦周天度分。又距極樞九十一度少半,旋為赤道帶天之紘。距極三十五度旋為內規。



 乃步冬至日躔所在,以正辰次之中,以立宿距。按渾儀所測,甘、石、巫咸眾星明者,皆以篾,橫考入宿距,縱考去極度,而後圖之。其赤道外眾星疏密之狀,與仰視小殊者,由渾儀去南極漸近,其度益狹;而蓋圖漸遠,其度益廣使然。若考其去極入宿度數,移之於渾天則一也。又赤道內外,其廣狹不均,若就二至出入赤道二十四度,以規度之,則二分所交不得其正;自二分黃赤道交,以規度之,則二至距極度數不得其正;當求赤道分、至之中,均刻為七十二限,據每黃道差數,以篾度量而識之,然後規為黃道,則周天咸得其正矣。又考黃道二分二至之中,均刻為七十二候,定陰陽歷二交所在,依月去黃道度,率差一候,亦以篾度量而識之,然後規為月道,則周天咸得其正矣。



 中晷之法。初,淳風造歷,定二十四氣中晷,與祖沖之短長頗異,然未知其孰是。及一行作《大衍歷》,詔太史測天下之晷,求其土中,以為定數。其議曰:



 《周禮·大司徒》:「以土圭之法測土深。日至之景,尺有五寸,謂之地中。」鄭氏以為「日景於地,千里而差一寸。尺有五寸者,南戴日下萬五千里,地與星辰四游升降於三萬里內,是以半之,得地中,今潁川陽城是也」。宋元嘉中,南征林邑,五月立表望之,日在表北,交州影在表南三寸,林邑九寸一分。交州去洛,水陸之路九千里,蓋山川回折使之然,以表考其弦當五千乎。開元十二年,測交州,夏至,在表南三寸三分,與元嘉所測略同。使者大相元太言:「交州望極,才高二十餘度。八月海中望老人星下列星粲然,明大者甚眾,古所未識,乃渾天家以為常沒地中者也。大率去南極二十度已上之星則見。」又鐵勒、回紇在薛延陀之北,去京師六千九百里,其北又有骨利幹,居澣海之北,北距大海,晝長而夜短,既夜,天如曛不暝,夕胹羊髀才熟而曙,蓋近日出沒之所。太史監南宮說擇河南平地,設水準繩墨植表而以引度之,自滑臺始白馬,夏至之晷,尺五寸七分。又南百九十八里百七十九步,得浚儀嶽臺,晷尺五寸三分。又南百六十七里二百八十一步,得扶溝,晷尺四寸四分。又南百六十里百一十步,至上蔡武津,晷尺三寸六分半。大率五百二十六里二百七十步,晷差二寸餘。而舊說王畿千里,影差一寸,妄矣。



 今以句股校陽城中晷,夏至尺四寸七分八厘,冬至丈二尺七寸一分半,定春秋分五尺四寸三分,以覆矩斜視,極出地三十四度十分度之四。自滑臺表視之,極高三十五度三分,冬至丈三尺,定春秋分五尺五寸六分。自浚儀表視之,極高三十四度八分,冬至丈二尺八寸五分,定春秋分五尺五寸。知扶溝表視之,極高三十四度三分,冬至丈二尺五寸五分,定春秋分五尺三寸七分。上蔡武津表視之,極高三十三度八分,冬至丈二尺三寸八分,定春秋分五尺二寸八分。其北極去地,雖秒分微有盈縮,難以目校,大率三百五十一里八十步,而極差一度。極之遠近異,則黃道軌景固隨而變矣。自此為率推之,比歲武陵晷,夏至七寸七分,冬至丈五寸三分,春秋分四尺三寸七分半,以圖測之,定氣四尺四寸七分,按圖斜視,極高二十九度半,差陽城五度三分。蔚州橫野軍夏至二尺二寸九分,冬至丈五尺八寸九分,春秋分六尺四寸四分半,以圖測之,定氣六尺六寸二分半。按圖斜視,極高四十度,差陽城五度三分。凡南北之差十度半,其徑三千六百八十里九十步。自陽城至武陵,千八百二十六里七十六步;自陽城至橫野,千八百六十一里二百十四步。夏至晷差尺五寸三分;自陽城至武陵,差七寸三分;自陽城至橫野,差八寸。冬至晷差五尺三寸六分,自陽城至武陵差二尺一寸八分;自陽城至橫野,差三尺一寸八分。率夏至與南方差少,冬至與北方差多。



 又以圖校安南,日在天頂北二度四分,極高二十度四分。冬至晷七尺九寸四分,定春秋分二尺九寸三分,夏至在表南三寸三分,差陽城十四度三分,其徑五千二十三里。至林邑,日在天頂北六度六分強,極高十七度四分,周圓三十五度,常見不隱。冬至晷六尺九寸,定春秋分二尺八寸五分,夏至在表南五寸七分,其徑六千一百一十二里。若令距陽城而北,至鐵勒之地,亦差十七度四分,與林邑正等,則五月日在天頂南二十七度四分,極高五十二度,周圓百四度,常見不隱。北至晷四尺一寸三分,南至晷二丈九尺二寸六分,定春秋分晷五尺八寸七分。其沒地才十五餘度,夕沒亥西,晨出醜東,校其里數,已在回紇之北,又南距洛陽九千八百一十五里,則極長之晝,其夕常明。然則骨利幹猶在其南矣。



 吳中常侍王蕃考先儒所傳,以戴日下萬五千里為句股,斜射陽城,考周徑之率以揆天度,當千四百六里二十四步有餘。今測日晷,距陽城五千里,已在戴日之南,則一度之廣皆三分減二,南北極相去八萬里,其徑五萬里。宇宙之廣,豈若是乎?然則蕃之術,以蠡測海者也。



 古人所以恃句股術,謂其有證於近事。顧未知目視不能及遠,遠則微差,其差不已,遂與術錯。譬游於太湖,廣袤不盈百里,見日月朝夕出入湖中;及其浮於巨海,不知幾千萬里,猶見日月朝夕出入其中矣。若於朝夕之際,俱設重差而望之,必將大小之同術,無以分矣。橫既有之,縱亦宜然。



 又若樹兩表,南北相距十里,其崇皆數十里,置大炬於南表之端,而植八尺之木於其下,則當無影。試從南表之下,仰望北表之端,必將積微分之差,漸與南表參合。表首參合,則置炬於其上,亦當無影矣。又置大炬於北表之端,而植八尺之木於其下,則當無影。試從北表之下,仰望南表之端,又將積微分之差,漸與北表參合。表首參合,則置炬於其上,亦當無影矣。復於二表間更植八尺之木,仰而望之,則表首環屈相合。若置火炬於兩表之端,皆當無影矣。夫數十里之高與十里之廣,然猶斜射之影與仰望不殊。今欲憑晷差以指遠近高下,尚不可知,而況稽周天裏步於不測之中,又可必乎?十三年,南至,岱宗禮畢,自上傳呼萬歲,聲聞於下。時山下夜漏未盡,自日觀東望,日已漸高。據歷法,晨初迨日出差二刻半,然則山上所差凡三刻餘。其冬至夜刻同立春之後,春分夜刻同立夏之後。自岳趾升泰壇僅二十里,而晝夜之差一節。設使因二十里之崇以立句股術,固不知其所以然,況八尺之表乎!



 原古人所以步圭影之意,將以節宣和氣,轉相物宜,不在於辰次之周徑。其所以重歷數之意,將欲恭授人時,欽若乾象,不在於渾、蓋之是非。若乃述無稽之法於視聽之所不及,則君子當闕疑而不議也。而或者各守所傳之器以術天體,謂渾元可任數而測,大象可運算而窺。終以六家之說,迭為矛盾,誠以為蓋天邪?則南方之度漸狹;果以為渾天邪?則北方之極浸高。此二者,又渾、蓋之家盡智畢議,未能有以通其說也。則王仲任、葛稚川之徒,區區於異同之辨,何益人倫之化哉。凡晷差,冬夏不同,南北亦異,先儒一以里數齊之,遂失其實。今更為《覆矩圖》,南自丹穴,北暨幽都,每極移一度,輒累其差,可以稽日食之多少,定晝夜之長短,而天下之晷,皆協其數矣。



 昭宗時,太子少詹事邊岡,脩歷術,服其精粹,以為不刊之數也。



 初,貞觀中,淳風撰《法象志》,因《漢書》十二次度數,始以唐之州縣配焉。而一行以為天下山河之象存乎兩戒。北戒自三危、積石,負終南地絡之陰,東及太華,逾河,並雷首、厎柱、王屋、太行,北抵常山之右,乃東循塞坦,至濊貊、朝鮮,是謂北紀,所以限戎狄也;南戒自岷山、嶓塚,負地絡之陽,東及太華,連益山、熊耳、外方、桐柏,自上洛南逾江、漢,攜武當、荊山,至於衡陽,乃東循嶺徼,達東甌、閩中,是謂南紀,所以限蠻夷也。故《星傳》謂北戒為「胡門」,南戒為「越門」。



 河源自北紀之首,循雍州北徼,達華陰,而與地絡相會,並行而東,至太行之曲,分而東流,與涇、謂、濟瀆相為表裏,謂之「北河」。江源自南紀之首,循梁州南徼,達華陽,而與地絡相會,並行而東,及荊山之陽,分而東流,與漢水、淮瀆相為表裏,謂之「南河」。故於天象,則弘農分陜為兩河之會,五服諸侯在焉。自陜而西為秦、涼,北紀山河之曲為晉、代,南紀山河之曲為巴、蜀,皆負險用武之國也。自陜而東,三川、中岳為成周;西距外方、大伾,北至於濟,南至於淮,東達鉅野,為宋、鄭、陳、蔡;河內及濟水之陽為鄁、衛;漢東濱淮水之陰為申、隨。皆四戰用文之國也。北紀之東,至北河之北,為邢、趙。南紀之東,至南河之南,為荊、楚。自北河下流,南距岱山為三齊,夾右碣石為北燕。自南河下流,北距岱山為鄒、魯,南涉江、淮為吳、越。皆負海之國,貸殖之所阜也。自河源循塞垣北,東及海,為戎狄。自江源循嶺徼南,東及海,為蠻越。觀兩河之象。與雲漢之所始終,而分野可知矣。



 於《易》,五月一陰生,而云漢潛萌於天稷之下,進及井、鉞間,得坤維之氣,陰始達於地上,而云漢上升,始交於列宿,七緯之氣通矣。東井據百川上流,故鶉首為秦、蜀墟,得兩戒山河之首。雲漢達坤維右而漸升,始居列宿上,觜觿、參、伐皆直天關表而在河陰,故實沈下流得大梁,距河稍遠,涉陰亦深。故其分野,自漳濱卻負恆山,居北紀眾山之東南,外接髦頭地,皆河外陰國也。十月陰氣進逾乾維,始上達於天,雲漢至營室、東壁間,升氣悉究,與內規相接。故自南正達於西正,得雲漢升氣,為山河上流;自北正達於東正,得雲漢降氣,為山河下流。陬訾在雲漢升降中,居水行正位,故其分野當中州河、濟間。且王良、閣道由紫垣絕漢抵營室,上帝離宮也,內接成周、河內,皆豕韋分。十一月一陽生,而云漢漸降,退及艮維,始下接於地,至斗、建間,復與列舍氣通,於《易》,天地始交,泰象也。逾析木津,陰氣益降,進及大辰,升陽之氣究,而云漢沈潛於東正之中,故《易》,雷出地曰豫,龍出泉為解,皆房、心象也。星紀得雲漢下流,百川歸焉,析木為雲漢末派,山河極焉。故其分野,自南河下流,窮南紀之曲,東南負海,為星紀;自北河末派,窮北紀之曲,東北負海,為析木。負海者,以其雲漢之陰也。唯陬訾內接紫宮,在王畿河、濟間。降婁、玄枵與山河首尾相遠,鄰顓頊之墟,故為中州負海之國也。其地當南河之北、北河之南,界以岱宗,至於東海。自鶉首逾河,戒東曰鶉火,得重離正位,軒轅之祇在焉。其分野,自河、華之交,東接祝融之墟,北負河,南及漢,蓋寒燠之所均也。自析木紀天漢而南,曰大火,得明堂升氣,天市之都在焉。其分野,自鉅野岱宗,西至陳留,北負河、濟,南及淮,皆和氣之所布也。陽氣自明堂漸升,達於龍角,曰壽星。龍角謂之天關,於《易》,氣以陽決陰,夬象也。升陽進逾天關。得純乾之位,故鶉尾直建巳之月,內列太微,為天廷。其分野,自南河以負海,亦純陽地也。壽星在天關內,故其分野,在益、亳西南,淮水之陰,北連太室之東,自陽城際之,亦巽維地也。



 夫云漢自坤抵艮為地紀,北斗自乾攜巽為天綱,其分野與帝車相直,皆五帝墟也。究咸池之政而在乾維內者,降婁也,故為少昊之墟。葉北宮之政而在乾維外者,陬訾也,故為顓頊之墟。成攝提之政而在巽維內者,壽星也,故為太昊之墟。布太微之政,而在巽維外者,鶉尾也,故為列山氏之墟。得四海中承太階之政者,軒轅也,故為有熊氏之墟。木、金得天地之微氣,其神治於季月;水、火得天地之章氣,其神治於孟月。故章道存乎至,微道存乎終,皆陰陽變化之際也。若微者沈潛而不及,章者高明而過亢,皆非上帝之居也。



 斗杓謂之外廷,陽精之所布也。斗魁謂之會府,陽精之所復也。杓以治外,故鶉尾為南方負海之國。魁以治內,故陬訾為中州四戰之國。其餘列舍,在雲漢之陰者八,為負海之國。在雲漢之陽者四,為四戰之國。降婁、玄枵以負東海,春神主於岱宗,歲星位焉。星紀、鶉尾以負南海,其神主於衡山,熒惑位焉。鶉首、實沈以負西海,其神主於華山,太白位焉。大梁、析木以負北海,其神主於恆山,辰星位焉。鶉火、大火、壽星、豕韋為中州,其神主於嵩丘,鎮星位焉。



 近代諸儒言星土者,或以州,或以國。虞、夏、秦、漢,郡國廢置不同。周之興也,王畿千里,及其衰也,僅得河南七縣。今又天下一統,而直以鶉火為周分,則疆場舛矣。七國之初,天下地形雌韓而雄魏,魏地西距高陵,盡河東、河內,北固漳、鄴,東分梁、宋,至於汝南,韓據全鄭之地,南盡潁川、南陽、西達虢略,距函谷,固宜陽,北連上地,皆綿亙數州,相錯如繡。考云漢山河之象,多者或至十餘宿。其後魏徙大梁,則西河合於東井;秦拔宜陽,而上黨入於輿鬼。方戰國未滅時,星家之言,屢有明效。今則同在畿甸之中矣。而或者猶據《漢書地理志》推之,是守甘、石遺術,而不知變通之數也。



 又古之辰次與節氣相系,各據當時歷數,與歲差遷徙不同。今更以七宿之中分四象中位,自上元之首,以度數紀之,而著其分野,其州縣雖改隸不同,但據山河以分爾。



 須女、虛、危,玄枵也。初,須女五度,餘二千三百七十四,秒四少。中,虛九度。終,危十二度。其分野,自濟北東逾濟水,涉平陰,至於山莊,循岱岳眾山之陰,東南及高密,又東盡萊夷之地,得漢北海、千乘、淄川,濟南、濟郡及平原、渤海、九河故道之南,濱於碣石。古齊、紀、祝、淳于、萊、譚、寒及斟尋、有過、有鬲、蒲姑氏之國,其地得陬訾之下流,自濟東達於河外,故其象著為天津,絕雲漢之陽。凡司人之星與群臣之錄,皆主虛、危,故岱宗為十二諸侯受命府。又下流得婺女,當九河末派,比於星紀,與吳、越同占。



 營室、東壁,陬訾也。初,危十三度,餘二千九百二十六,秒一太。中,營室十二度。終,奎一度。自王屋、太行而東,得漢河內,至北紀之東隅,北負漳、鄴,東及館陶、聊城。又自河、濟之交,涉滎波,濱濟水而東,得東郡之地,古邶、庸阜、衛、凡、胙、邗、雍、共、微、觀、南燕、昆吾、豕韋之國。自閣道、王良至東壁,在豕韋,為上流。當河內及漳、鄴之南,得山河之會,為離宮。又循河、濟而東接玄枵為營室之分。



 奎、數,降婁也。初,奎二度,餘千二百一十七,秒十七少。中,婁一度。終,胃三度。自蛇丘、肥成,南屆鉅野,東達梁父,循岱岳眾山之陽,以負東海。又濱泗水,經方與、沛、留、彭城,東至於呂梁,乃東南抵淮,並淮水而東,盡徐夷之地,得漢東平、魯國、瑯邪、東海、泗水、城陽,古魯、薛、邾、莒、小邾、徐、郯、鄫、鄅、邳、邿、任、宿、須句、顓臾、牟、遂、鑄夷、介、根牟及大庭氏之國。奎為大澤,在陬訾下流,當鉅野之東陽,至於淮、泗。婁、胃之墟,東北負山,蓋中國膏腴地,百穀之所阜也。胃得馬牧之氣,與冀之北土同占。



 胃、昴、畢,大梁也。初,胃四度,餘二千五百四十九,秒八太。中,昴六度。終,畢九度。自魏郡濁漳之北,得漢趙國、廣平、鉅鹿、常山,東及清河、信都,北據中山、真定,全趙之分。又北逾眾山,盡代郡、雁門、雲中、定襄之地與北方群狄之國。北紀之東陽,表裏山河,以蕃屏中國,為畢分。循北河之表,西盡塞垣,皆髦頭故地,為昴分。冀之北土,馬牧之所蕃庶,故天苑之象存焉。



 觜觿、參、伐,實沈也。初,畢十度,餘八百四十一,秒四之一。中,參七度。終,東井十一度。自漢之河東及上黨、太原,盡西河之地,古晉、魏、虞、唐、耿、楊、霍、冀、黎、郇與西河戎狄之國。西河之濱,所以設險限秦、晉,故其地上應天闕。其南曲之陰,在晉地,眾山之陽;南曲之陽,在秦地,眾山之陰。陰陽之氣並,故與東井通。河東永樂、芮城、河北縣及河曲豐、勝、夏州,皆東井之分。參、伐為戎索,為武政,當河東,盡大夏之墟。上黨次居下流,與趙、魏接,為觜觿之分。



 東井、輿鬼,鶉首也。初,東井十二度,餘二千一百七十二,秒十五太。中,東井二十七度。終,柳六度。自漢三輔及北地、上郡、安定,西自隴坻至河右,西南盡巴、蜀、漢中之地,及西南夷犍為、越雋、益州郡,極南河之表,東至牂柯,古秦、梁、豳、芮、豐、畢、駘杠、有扈、密須、庸、蜀、羌、髳之國。東井居兩河之陰,自山河上流,當地絡之西北。輿鬼居兩河之陽,自漢中東盡華陽,與鶉火相接,當地絡之東南。鶉首之外,雲漢潛流而未達,故狼星在江、河上源之西,弧矢、犬、雞皆徼外之備也。西羌、吐蕃、吐谷渾及西南徼外夷人,皆占狼星。



 柳、七星、張,鶉火也。初,柳七度,餘四百六十四,秒七少。中,七星七度。終,張十四度。北自滎澤、滎陽,並京、索,暨山南,得新鄭、密縣,至外方東隅,斜至方城,抵桐柏,北自宛、葉,南暨漢東,盡漢南陽之地。又自雒邑負北河之南,西及函谷,逾南紀,達武當、漢水之陰,盡弘農郡,以淮源、桐柏、東陽為限,而申州屬壽星,古成周、虢、鄭、管、鄶、東虢、密、滑、焦、唐、隨、申、鄧及祝融氏之都。新鄭為軒轅、祝融之墟,其東鄙則入壽星。柳。在輿鬼東,又接漢源,當益、洛之陽,接南河上流。七星系軒轅,得土行正位,中嶽象也,河南之分。張,直南陽,漢東,與鶉尾同占。



 翼、軫,鶉尾也。初,張十五度,餘千七百九十五,秒二十二太。中,翼十二度。終,軫九度。自房陵、白帝而東,盡漢之南郡、江夏,東達廬江南部,濱彭蠡之西,得長沙、武陵,又逾南紀,盡鬱林、合浦之地,自沅、湘上流,西達黔安之左,皆全楚之分。自富、昭、象、龔、繡、容、白、廉州已西,亦鶉尾之墟。古荊楚、鄖、鄀、羅、權、巴、夔與南方蠻貊之國。翼與硃張同象,當南河之北,軫在天關之外,當南河之南,其中一星主長沙,逾嶺徼而南,為東甌、青丘之分。安南諸州在雲漢上源之東陽,宜屬鶉火。而柳、七星、張皆當中州,不得連負海之地,故麗於鶉尾。



 角、亢,壽星也。初,軫十度,餘八十七,秒十四少。中,角八度。終,氐一度。自原武、管城,濱河、濟之南,東至封丘、陳留,盡陳、蔡、汝南之地,逾淮源,至於弋陽,西涉南陽郡至於桐柏,又東北抵嵩之東陽,中國地絡在南北河之間,首自西傾,極於陪尾,故隨、申,光皆豫州之分,宜屬鶉火,古陳、蔡、許、息、江、黃、道、柏、沈、賴、蓼、須頓、胡、防、弦、厲之國。氐涉壽星,當洛邑眾山之東,與亳土相接,次南直潁水之間,曰太昊之墟,為亢分。又南涉淮氣連鶉尾,在成周之東陽,為角分。



 氐、房、心,大火也。初,氐二度,餘千四百一十九,秒五太。中,房二度。終,尾六度。自雍丘、襄邑、小黃而東,循濟陰,界於齊、魯,右泗水,達於呂梁,乃東南接太昊之墟,盡漢濟陰、山陽、楚國、豐、沛之地,古宋、曹、郕、滕、茅、郜、蕭、葛、向城、逼陽、申父之國。商、亳負北河,陽氣之所升也,為心分、豐、沛負南河,陽氣之所布也,為房分。其下流與尾同占,西接陳、鄭為氐分。



 尾、箕,析木津也。初,尾七度,餘二千七百五十,秒二十一少,中,箕五度,終,南斗八度。自渤海、九河之北,得漢河間、涿郡、廣陽及上谷、漁陽、右北平、遼西、遼東、樂浪、玄菟,古北燕、孤竹、無終、九夷之國。尾得雲漢之末派,龜、魚麗焉,當九河之下流,濱於渤碣,皆北紀之所窮也。箕與南斗相近,為遼水之陽,盡朝鮮三韓之地,在吳、越東。



 南斗、牽牛,星紀也。初,南斗九度,餘千四十二,秒十二太。中,南斗二十四度。終,女四度。自廬江、九江,負淮水,南盡臨淮、廣陵,至於東海,又逾南河,得漢丹楊、會稽、豫章,西濱彭蠡,南涉越門,迄蒼梧、南海,逾嶺表,自韶、廣以西,珠崖以東,為星紀之分也。古吳、越、群舒、廬、桐、六、蓼及東南百越之國。南斗在雲漢下流,當淮、海間,為吳分。牽牛去南河浸遠,自豫章迄會稽,南逾嶺徼,為越分。島夷蠻貊之人,聲教所不暨,皆系於狗國云。



\end{pinyinscope}