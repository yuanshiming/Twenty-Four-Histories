\article{列傳第八十二 陸贄}

\begin{pinyinscope}

 陸贄,字敬輿,蘇州嘉興人。十八第進士,中博學宏辭。調鄭尉,罷歸。壽州刺史張鎰有重名賦,贄往見,語三日,奇之,請為忘年交。既行,餉錢百萬,曰:「請為母夫人一日費。」贄不納,止受茶一串,曰:「敢不承公之賜?」以書判拔萃補渭南尉。



 德宗立,遣黜陟使庾何等十一人行天下。贄說使者,請以五術省風俗,八計聽吏治,三科登雋義,四賦經財實,六德保罷瘵,五要簡官事。五術曰:「聽謠誦審其哀樂,納市賈觀其好惡,訊簿書考其爭訟,覽車服等其儉奢,省作業察其趣舍。」八計曰:「視戶口豐耗以稽撫字,視墾田贏縮以稽本末,視賦役薄厚以稽廉冒,視案籍煩簡以稽聽斷,視囚系盈虛以稽決滯,視奸盜有無以稽禁禦,視選舉眾寡以稽風化,視學校興廢以稽教導。」三科曰:「茂異,賢良,幹蠱。」四賦曰:「閱稼以奠稅,度產以衰征,料丁壯以計庸,占商賈以均利。」六德曰:「敬老,慈幼,救疾,恤孤,賑貧窮,任失業。」五要曰:「廢兵之冗食,蠲法之撓人,省官之不急,去物之無用,罷事之非要。」時皆韙其言。遷監察御史。



 帝在東宮,已聞其名矣,召為翰林學士。會馬燧討賊河北,久不決,請濟師;李希烈寇襄城。詔問策安出,贄言:



 勞於服遠,莫若脩近;多方以救失,莫若改行。今幽、燕、恆、魏之勢緩而禍輕,汝、洛、滎、汴之勢急而禍重。田悅覆敗之餘,無復遠略,王武俊有勇無謀,硃滔多疑少決,互相制劫,急則合力,退則背憎,不能有越軼之患,此謂緩也。希烈果於奔噬,忍於傷殘,據蔡、許富全之地,而益以鄧、襄虜獲之實,東寇則饟道阻,北窺則都邑震,此謂急也。代、朔、邠、靈自昔之精騎,上黨、盟津今之選師,舉而委之山東,將多而勢分,兵廣而財屈,則屯戍失於太繁也。李勉,文吏也,而當汴必爭地;哥舒曜之眾,烏合也,捍襄城方銳之賊。本非素習,首鼠莫前,則守御失於不足也。今若還李芃河陽以援東都,李懷光解襄城之圍,專以太原、澤、潞兵抗山東,則梁、宋安。



 又言:



 立國之權,在審輕重,本大而末小,所以能固。故治天下者,若身使臂,臂使指,小大適稱而不悖。王畿者,四方之本也;京邑者,王畿之本也。其勢當京邑如身,王畿如臂,而四方如指,此天子大權也。是以前世轉天下租稅,徙郡縣豪傑,以實京師。太宗列置府兵八百所,而關中五百,舉天下不敵關中,則居重馭輕之意也。方世承平久,武備微,故祿山乘外重之勢,一舉而覆兩京。然猶諸牧有馬,州縣有糧,肅宗得以中興。乾元後,外虞踵發,悉師東討,故吐蕃乘虛,而先帝莫與為御,是失馭輕之權也。既自陜還,懲艾前事,稍益禁衛,故關中有朔方、涇原、隴右之兵以捍西戎,河東有太原之兵以制北虜。今朔方、太原眾已屯山東,而神策六軍悉戍關外,將不能盡敵,則請濟師。陛下為之輟邊軍,缺環衛,竭內廄之馬、武庫之兵,占將家子以益師,賦私畜以增騎。又告乏財,則為算室廬,貸商人,設諸榷之科,日日以甚。萬有一如硃滔、李希烈負固邊壘,竊發都甸者,何以備之?



 夫關中,王業根本在焉。豪傑之在關中者,與籍於營衛不殊;車乘之在關中者,與列於廄牧不殊;財用之在關中者,與貯於帑藏不殊。一朝有急,可取也。陛下幸聽臣計,使芃還軍援洛,懷光救襄城,希烈必走。請神策軍及將家子占而東者追還之,凡京師稅間架、榷酒、抽貫、貸商、點召之令,一切停之,則端本整棼之術。



 帝不納。後涇師急變,贄言皆效。



 從狩奉天,機務填總,遠近調發,奏請報下,書詔日數百,贄初若不經思,逮成,皆周盡事情,衍繹孰復,人人可曉。旁吏承寫不給,它學士筆閣不得下,而贄沛然有餘。



 始,帝倉卒變故,每自克責。贄曰:「陛下引咎,堯、舜意也。然致寇者乃群臣罪。」贄意指盧杞等。帝護杞,因曰:「卿不忍歸過朕,有是言哉。然自古興衰,其亦有天命乎?今之厄運,恐不在人也。」贄退而上書曰:



 自安史之亂,朝廷因循涵養,而諸方自擅壤地,未嘗會朝。陛下將一區宇,乃命將興師,以討四方。一人征行,十室資奉;居者疲饋轉,行者苦鋒鏑;去留騷然,而閭里不寧矣。聚兵日眾,供費日博,常賦不給,乃議蹙限而加斂焉;加斂既殫,乃別配之;別配不足,於是榷算之科設,率貸之法興。禁防滋章,吏不堪命;農桑廢於追呼,膏血竭於笞捶;兆庶嗷然,而郡邑不寧矣。邊陲之戍以保封疆,禁衛之旅以備巡警,邦之大防也。陛下悉而東征,邊備空屈,又搜私牧、責將家以出兵籍馬。夫私牧者,元勛貴戚之門也;將家者,統帥岳牧之後也;其復除征徭舊矣。今奪其畜牧,事其子孫,丐假以給資裝,破產以營卒乘,元臣貴位,孰不解體?方且稅侯王之廬,算裨販之緡,貴不見優,近不見異,群情囂然而關畿不寧矣。



 陛下又謂百度弛廢,則持義以掩恩,任法以成治,斷失於太速,察傷於太精。斷速則寡恕於人,而疑似不容辨也;察精則多猜於物,而億度未必然也。寡恕而下懼禍,故反側之釁生;多猜而下妨嫌,故茍且之患作。由是叛亂繼產,忿讟並興,非常之虞,惟人主獨不聞。兇卒鼓行,白晝犯闕;重門無結草之御,環衛無誰何之人。陛下雖有股肱之臣,耳目之佐,見危不能竭誠,臨難不能效死,是則群臣之罪也。



 陛下方以興衰諉之天命,亦過矣。《書》曰:「天視自我民視,天聽自我民聽。」則天所視聽,皆因於人,非人事外自有天命也。紂之辭曰:「我生不有命在天?」此舍人事推天命,必不可之理也。《易》曰:「自天祐之。」仲尼以謂:「祐者助也。天之所助者順也,人之所助者信也。履信思乎順,是以祐之。」《易》論天人祐助之際,必先履行,而吉兇之報象焉。此天命在人,蓋昭昭矣。人事治而天降亂,未之有也;人事亂而天降康,亦未之有也。尚恐有可疑者,請以近事信之。



 自比兵興,物力耗竭。人心驚疑如風濤然,洶洶靡定,族謀聚議,謂必有變。則京師之人,固非悉通占術、曉天命也,則致寇之由,豈運當然?夫治或生亂,亂或資治;有以無難而亡,多難而興。治或生亂者,恃治而不修也;亂或資治者,遭亂而能治也;無難而失者,忽萬幾之重,而忘憂畏也;多難而興者,涉庶事之艱,而知敕慎也。今生亂失序之事不可追矣,其資治興邦之業,在刻勵而謹修之。當至危之機,得其道則興,失則廢,其間不容復有所悔也,惟勤思而熟計之。舍己以從眾,違欲以遵道,遠憸佞,親忠直,推至誠,去逆詐,斯道甚易知,甚易行,不耗神,不劬力,第約之於心耳。何憂乎亂人,何畏乎厄運,何患乎不寧哉?



 帝又問贄事切於今者,贄勸帝:「群臣參日,使極言得失。若以軍務對者,見不以時,聽納無倦。兼天下之智以為聰明。」帝曰:「朕豈不推誠!然顧上封者,惟譏斥人短長,類非忠直。往謂君臣一體,故推信不疑,至憸人賣為威福。今茲之禍,推誠之敝也。又諫者不密,要須歸曲於朕,以自取名。朕嗣位,見言事多矣,大抵雷同道聽,加質則窮。故頃不詔次對,豈曰倦哉!」贄因是極諫曰:



 昔人有因噎而廢食者,又有懼溺而自沈者,其為防患,不亦過哉!願陛下鑒之,毋以小虞而妨大道也。臣聞人之所助在信,信之所本在誠。一不誠,心莫之保;一不信,言莫之行。故聖人重焉。傳曰:「誠者,物之終始,不誠無物。」物者事也,言不誠即無所事矣。匹夫不誠,無復有事,況王者賴人之誠以自固,而可不誠於人乎?陛下所謂誠信以致害者,臣竊非之。孔子曰:「可與言而不與之言,失人;不可與言而與之言,失言。智者不失人,亦不失言。」陛下可審其言而不可不信,可慎其所與而不可不誠。所謂民者,至愚而神。夫蚩蚩之倫,或昏或鄙,此似於愚也。然上之得失靡不辨,好惡靡不知,所秘靡不傳,所為靡不效。馭以智則詐,示以疑則偷;接不以禮則其徇義輕,撫不以情則其效忠薄。上行則下從之,上施則下報之,若景附形,若響應聲。故曰:「惟天下至誠,為能盡其性。」不盡於己而責盡於人,不誠於前而望誠於後,必紿而不信矣。今方鎮有不誠於國,陛下興師伐之;臣有不信於上,陛下下令誅之。有司奉命而不敢赦者,以陛下所有責彼所無也。故誠與信不可斯須去己。願陛下慎守而力行之,恐非所以為悔也。



 《傳》曰:「人誰無過?過而能改,善莫大焉。」仲虺歌成湯之德曰:「改過不吝。」吉甫美宣王之功曰:「袞職有闕,仲山甫補之。」夫成湯聖君也,仲虺聖輔也,以聖輔贊聖君,不稱其無過,稱其改過;周宣中興賢王也,吉甫文武賢臣也,歌誦其主,不美其無闕,而美其補闕。則聖賢之意,貴於改過,較然甚明。蓋過差者,上智下愚所不免,惟智者能改而之善,愚者恥而之非也。中古以降,其臣尚諛,其君亦自聖,掩盛德,行小道,乃有入則造膝,出則詭辭,奸由此滋,善由此沮,天子意由此惑,爭臣罪由此生,媚道行而害斯甚矣。太宗有文武仁義之德、治致太平之功,可謂盛矣,然而人到於今以從諫改過為稱首。是知諫而能從,過而能改,帝王之大烈也。陛下謂諫官論事,引善自予,歸過於上者,信非其美,然於盛德,未有虧焉。納而不違,傳之適足增美;拒而違之,又安能禁之勿傳?不宜以此梗進言之路也。



 聖人不忽細微,不侮鰥寡;奓言無驗不必用,質言當理不必違;遜於志不必然,逆於心不必否;異於人不必是,同於眾不必非;辭拙而效迂者不必愚,言甘而利重者不必智。考之以實,惟善所在,則可以盡天下之心矣。夫人情蔽於所信,沮於所疑,忽於所輕,溺於所欲。信偏則聽言不盡其實,故有過當之言;疑甚則雖實不聽其言,故有失實之聽。輕其人則遺可重之事,欲其事則存可棄之人。茍縱所私,不考其實,則是失天下之心矣。故常情之所輕,聖人之所重,不必慕高而好異也。



 陛下又以雷同道說,加質則窮。臣謂陛下雖窮其辭而未窮其理,能服其口而未服其心。且下之情莫不願達於上,上之情莫不求知於下。然而下常苦上之難達,上常苦下之難知。若是者何?九弊不去也。所謂九弊者,上有六,下有三:好勝人,恥聞過,騁辯給,炫聰明,厲威嚴,恣強愎,上之弊也;諂諛、顧望、畏懦,下之弊也。好勝而恥過,必甘佞辭,忌直言,則諂諛者進,而忠實之語不聞矣。騁辯而炫明,必折人以言,虞人以詐,則顧望者自便,而切摩之益不盡矣。厲威而恣愎,必不能降情接物,引咎在己,則畏懦者至,而情理之說不申矣。人之難知,堯、舜所病,胡可以一酬一詰,而謂盡其能哉?夫欲治天下,而不務得人心,則固不治矣;務得人心,而不勤接下,則心固不得矣;務接下而不辨君子小人,則下固不可接矣;務辨君子小人,而惡直嗜諛,則君子小人固不可辨矣。趨和求媚,人之甚利存焉;犯顏冒禍,人之甚害存焉。居上者易其言而以美利利之,猶懼忠告之不暨,況疏隔而猜忌者乎?



 是時,賊未平,帝欲明年遂改元,而術家爭言數鐘百六,宜有所變,示天下復始。帝乃議更益大號。贄曰:「今乘輿播越,大憝未去,此人情向背、天意去就之隙。陛下宜痛自貶勵,不宜益美名以累謙德。」帝曰:「卿言固善,然要當小有變革,為朕計之。」贄奏言:「古之人君,德合於天曰『皇』,合於地曰『帝』,合於人曰『王』,父天母地以養人治物得其宜者曰『天子』,皆大名也。三代而上,所稱象其德,不敢有加焉。至秦乃兼曰『皇帝』,流及後世昏僻之君,始有聖劉、天元之號。故人主重輕,不在稱謂,視德何如耳。若以時屯當有變革,不若引咎降名,以祗天戒。且矯舊失,至明也;損虛飾,大知也。寧與加冗號以受實患哉?」帝從之。



 會興元赦令方具,帝以稿付贄,使商討其詳。贄知帝執德不固,困則思治,泰則易驕,欲激之使強其意,即建言:「履非常之危者,不可以常道安;解非常之紛者,不可以常令諭。陛下窮用兵甲,竭取財賦,變生京師,盜據宮闥。今假王者四兇,僭帝者二豎,其他顧瞻懷貳,不可悉數。而欲紓多難,收群心,惟在赦令而已。動人以言,所感已淺;言又不切,人誰肯懷?故誠不至者物不感,損不極者益不臻。夫悔過不得不深,引咎不得不盡,招延不可不廣,潤澤不可不弘,使天下聞之,廓然一變,人人得其所欲,安有不服哉?其須改革科條,已別封上。臣聞知過非難,改之難;言善非難,行之難。《易》曰:『聖人感人心而天下和平。』夫感者,誠發於心而形於事,事或未諭,故宣之於言,言必顧心,心必副事,三者相合,乃可求感。惟陛下先斷厥志,以施其辭,度可行者而宣之,不可者措之。無茍於言,以重取悔。」帝納之。



 始,帝播遷,府藏委棄,衛兵無褚衣。至是,天下貢奉稍至,乃於行在夾廡署瓊林、大盈二庫,別藏貢物。贄諫,以為:「瓊林、大盈於古無傳。舊老皆言:開元時貴臣飾巧以求媚,建言郡邑賦稅,當委有司以制經用,其貢獻悉歸天子私有之。蕩心侈欲,亦終以餌寇。今師旅方殷,瘡痛呻吟之聲未息,遽以珍貢私別庫,恐群下有所觖望,請悉出以賜有功。令後納貢必歸之有司,先給軍賞,瑰怪纖麗無得以供。是乃散小儲成大儲,捐小寶固大寶也。」帝悟,即撤其署。



 李懷光有異志,欲怒其軍使叛,即上言:「兵稟薄,與神策不等,難以戰。」李晟密言其變,因請移屯。帝遣贄見懷光議事。贄還奏:「懷光寇奔不追,師老不用,群帥欲進,輒沮止其謀。此必反,宜有以制之。」因勸帝許晟移軍。初,贄與懷光語及晟,懷光妄詫曰:「吾無所藉晟。」贄即美其強雄,使不得翻覆。至是,請下詔書如其意者,且無辭歸短於朝。又建:「遣李建徽、陽惠元與晟並屯東渭橋,托言晟兵寡不足支賊,俾為掎角。懷光雖不欲遣,且辭窮,無以沮解。」帝猶豫曰:「晟移屯,懷光固怏怏,若又遣建徽等俱東,彼且為辭。少須之。」晟已徙營,不閱旬,懷光果奪兩節度兵。建徽挺身免,惠元死之。行在震驚,遂徙幸梁。



 道有獻瓜果者,帝嘉其意,欲授以試官。贄曰:「爵位,天下公器,不可輕也。」帝曰:「試官虛名,且已與宰相議矣,卿其無嫌。」贄奏:「信賞必罰,霸王之資也;輕爵褻刑,衰亂之漸也。非功而獲爵則輕,非罪而肆刑則褻。天寶之季,嬖幸傾國,爵以情授,賞以寵加,綱紀始壞矣。羯胡乘之,遂亂中夏。財賦不足以供賜,而職官之賞興焉;職員不足以容功,而散、試之號行焉。今所病者爵輕也,設法貴之,猶恐不重,若又自棄,將何勸焉?陛下謂試官為虛名,豈思之未熟邪?夫立國惟義與權,誘人惟名與利。名近虛,於教為重;利近實,於德為輕。凡所以裁是非,立法制,則存乎其義;參虛實,揣輕重,則存乎其權。專實利而不濟之以虛,則物有匱耗而不給矣;專虛名而不副之以實,則情有誕謾而不趨矣。故錫貨財,列稟秩,以彰實也;差品列,異服章,以飾虛也。居上者達其變,相須以為表裏,則為國之權得矣。按甲令,有職事官、有散官、有勛官、有爵號。其賦事受奉者,惟職事一官,以敘才能,以位勛德,所謂施實利而寓虛名也;勛、散、爵號,止於服色、資廕,以馭崇貴,以甄功勞,所謂假虛名佐實利者也。今員外、試官與勛、散、爵號同,然而突銛鋒、排禍難者以是酬之可謂重矣。今獻瓜一器、果一盛則受之,彼忘軀命者有以相謂矣,曰:『吾之軀命乃同瓜果。』瓜果,草木也。若草木然,人何勸哉?夫田父野人必欲得其歡心,厚賜之可也。」



 俄以勞遷諫議大夫,仍為學士。時鳳翔節度使李楚琳殺張鎰得位,雖數貢奉,議者頗言其挾兩端,有所狙伺然。帝亦不能容,其使至,皆不得召,欲以渾瑊代之。贄諫曰:「楚琳之罪舊矣,今議者乃始紛紜,不亦晚哉?且勤王之師在畿內者,急宣亟告,景刻不可差。商嶺既回遠,而駱穀又為賊所扼,通王命者唯褒斜爾。若復阻,則諸鎮之向背者,我勝則來,賊勝遂往,此焉幾會,不容差跌。使楚琳逞憾,敢為猖狂,南塞要沖,東與賊合,則我咽喉梗而心膂分矣,豈不病哉!今顧望兩端,是乃天誘其衷,通歸塗,濟大業也。」帝釋然,盡召見其使,優詔勞安之。



 帝欲以內外從官普號「定難元從功臣」。贄曰:「宮官具寮,恪居奔走,勞則有之,何功之云?難則嘗之,何定之云?今與奮命者齒,恐沮戰士之心,結勛臣之憤。」帝乃止。



 京師已平,帝欲召渾瑊訪奔亡內人,給裝使赴行在。贄諫曰:「大難始平,而百役疲瘵之氓、重傷殘廢之卒,皆忍死扶疾,想聞德音。蓋事有先後,義有輕重,重者宜先,輕者宜後。昔武王克殷,有未下車而為之者,有下車而為之者。當今所務,謂宜以大臣馳傳,迎復神主,脩飭郊丘,展禋享之禮,申告謝之意;恤死義,犒有功,崇進忠直,優問耆耄;定反側,寬脅從,官失職,復廢業,是皆宜先不可後也。葺宮室,治服玩,耳目之娛,巾櫛之侍,是皆宜後不可先也。且內人當離潰之後,或為將士所私。昔人掩絕纓、飲盜馬者,豈忘其愛邪?知為君之體然也。天下固多褻人,何必獨此?」帝不復下詔,猶遣使諭瑊資遣。



 初,劉從一、姜公輔等材下不逮贄遠甚,徒以單言暫謀偶有合,由下位建臺宰。而贄孤立一意,為左右權幸沮短,又言事無所回諱,陰失帝意,久之不得宰相。還京,但為中書舍人。母韋猶在江東,帝遣中人迎還京師。俄以喪解官,客東都。諸方賵遺一不取,惟韋皋以布衣交,先以聞,故所致輒稱詔受之。又詔中人護父柩至自吳會,葬洛陽。服除,以權知兵部侍郎復召為學士。入謝,伏地鯁泣,帝為興,改容慰撫。眷遇彌渥,天下屬以為相,而竇參素不平,忌之。贄亦數言參罪失。貞元七年,罷學士,以兵部侍郎知貢舉。明年,參黜,乃以中書侍郎同中書門下平章事。



 帝始任楊炎、盧杞,引樹私黨,排忠良,天下怨疾。貞元後,懲艾其失,雖置宰相,至除用庶官,反覆參詰乃得下。及贄秉政,始請臺閣長官得自薦其屬,有不職,坐舉者。帝初許之,或言諸司所引皆親黨,招賂遺,無實才,帝復詔宰相自擇。贄奏言:「齊桓公問管仲害霸,對曰:『得賢不能任,害霸也;任賢不能固,害霸也;固始而不終,害霸也;與賢人謀事,而小人議之,害霸也。』所謂小人者,非悉懷險詖以覆邦家也,蓋趨向狹促,以沮議為出眾,自異為不群,趣小利,昧遠圖,效小信,傷大道爾。所謂臺省長官,僕射、尚書、丞、郎、御史大夫、中丞是也。陛下擇輔相多出其中,行實不能頓殊也。今乃謂不能進一二屬吏,豈後位宰相則可擇天下材乎?夫求才者貴廣,考課者貴精。往武後收人心,務拔擢,非徒人得薦士,亦許自舉其才,豈不易哉?然而課責嚴,進退速,故當世稱知人之明,累朝賴多士之用。陛下賞鑒獨任,難於公舉,有登延之路,無練核之方。武后以易得人,陛下以精失士。今擇宰相以重於庶品,選長官以愈於下流。及宰相獻言,長吏薦士,則又納橫議,廢始謀,是任以重者輕其言,待以輕者重其事也。」帝雖嘉之,然卒停薦士詔。



 舊制,吏部選以歲集。乾元後,天下兵興,率三年一調,吏員稽壅,則案牒叢淆,偽冒蒙真,吏緣以為奸,廢置無綱,至十年不被調者,缺員或累歲不補。贄乃請以內外員三分之,每歲計闕集人,檢柅吏奸,天下便之。



 當是時,賈耽、盧邁、趙憬同輔政,凡有司關白,三人者更相顧不肯判。贄又請如故事,旬一人秉筆,所咨輒判。



 又以西北邊歲調河南、江淮兵,謂之「防秋」,士不素練,戰數敗,將統制不一,亡以應敵。乃上陳其弊曰:



 自祿山構亂,肅宗始撤邊備,以靖中邦,借外威,寧內難,於是吐蕃乘釁,回紇矜功,中國不振,四十餘年。率傷耗之民,竭力以事,西輸賄繒,北償馬資,尚不足滿其意。於是調斂四方,以屯疆陲,又不能遏其侵。故小入則驅略,深入則戒嚴。於時議安邊者,皆務所難,忽所易,勉所短,略所長,行之而要不精,圖之而功靡就。



 夫勢有難易,事有先後。力大而敵脆,則先所難,是謂奪人之心也;力寡而敵堅,則先所易,是謂觀釁而動也。今財匱於中,人勞未瘳,而欲發師徒以犯獵寇境,復其侵疆,攻其堅城,前有勝負未必之虞,後有饋運不繼之患。萬一橈敗,適所以啟戎心,挫國威也。以此安邊,可謂不量勢而務所難矣。天之授有分,地之產有宜,是以五方之俗,長短各殊。勉所短而敵長者殆,用所長而乘短者強。且以水草為居,討獵為生,便於馳突,不恥敗亡,此戎狄所長,中國之短也。而欲益兵搜乘,爭驅角力,交鋒原野之上,決命尋常之間,以此御寇,可謂勉所短而校其長矣。務所難,勉所短,勞費百倍,終無成功,雖果成之,不挫則廢。誠以越天授,違地產,虧時勢,以反物宜者也。胡不守所易,用所長乎?



 若乃擇將吏,脩紀律,訓齊師徒;耀德以佐威,能邇以示遐;禁侵暴以彰吾信,抑攻取以昭吾仁;彼求和則善之而勿與盟,彼為寇則備之而不報復。此當今所易也。賤力貴智,好生惡殺;輕利重人,忍小全大;安其居而動,俟其時後行。脩封疆,守要害,蹊塹隧,列屯營,謹禁防,明斥候,務農足食,非萬全不謀,非百克不鬥;寇小至則遏其入,寇大至則邀其歸,據險以乘之,多方以誤之,使其勇無所加,眾無所用,掠則靡獲,攻則不能,進有腹背支敵之虞,退有首尾不相救之患。是謂乘其弊,不戰而屈人兵。此中國之長也。我之所長,戎狄之短也;我之所易,戎狄之難也。以長制短,則用力寡而見功多;以易敵難,則財不匱而事速成。舍此不務而反為所乘,斯謂倒持戈矛,以鐏授寇者也。今皆務之矣,尚且守封未固,寇戎未懲者何邪?病在謀無定用,眾無適從;任者不必才,才者不必任;聞不必實,實不必聞;所信不必誠,所誠不必信;行不必當,當不必行。



 又有六失焉。夫兵有攻討,有鎮守。權以紓難,暫以應機,事有便宜,謀有奇詭,不恤常制,不徇眾情,死生進退,唯將所命,攻討之兵也。人情者,利焉則勸,習焉則安,保親戚而後樂生,顧家業而後忘死,可以治術馭,不可以法制驅,鎮守之兵也。王者欲備封疆,御戎狄,則選鎮守之兵以置之。古之善選置者,必辨其土宜,察其技能,知其好惡。用其力,不違其性;齊其俗,不易其宜;引其善,不責其所不能;禁其非,不處其所不欲。類其部伍,安其家室,然後能使之樂其居,定其志。以惠則感而不驕,以威則肅而不怨。靡督課而自用,馳禁防而不攜。故守則固,戰則強。其術無它,便於人而已。今遠調屯士,以戍邊陲,邀所不能,強所不欲,廣其數不考於用,責其力不察其情,斯可為羽衛之儀,而無益備御之實也。何者?窮邊之地,千里蕭條,寒風裂膚,豺狼為鄰,晝則荷戈以耕,夜則倚烽以覘,有剽害之慮,無休暇之娛,非生其域、習其風,幼而視焉,長而安焉,則不能寧居而狎其敵也。關東百物阜殷,士忲溫飽,比諸邊隅,不翅天地。聞絕塞荒陬,則辛酸動容;聆強蕃勁虜,則懾駭褫情。又使去親族,舍園廬,甘所辛酸,抗所懾駭,將冀為用,不亦疏乎?又有休代之期,無統制之善,資奉姑息,譬如驕子,進不邀以成功,退不處以嚴憲,屈指計歸,張頤待飼,師一挫傷,則乘其危橈,布路東潰。平居殫資儲以奉浮冗,臨難棄城鎮以搖疆場。其弊豈特無益哉?謫徙之人,本以增戶實邊,立功自贖。既無良之人,而思亂幸災又甚於戍卒,適有防衛之煩,而無立功之益。雖前代行之,固非可遵者也。帥臣身不臨邊,而以偏師戍守。大抵士之犀銳,悉選以自奉,委疲羸者以守要沖,寇至而不支,則劫執芟蹂,恣所欲得,比都府聞之,虜已旋返。治兵若此,斯可謂措置乖方。一失也。



 賞以存勸,罰以示懲,以懋有庸,以威不恪。故賞罰之於馭眾,譬輗軏所以行車,銜勒所以服馬也。今將之號令不能行之軍,國之典刑不能施之將,上下遵養,以茍歲時。欲褒一有功,慮無功者怨,嫌疑而不賞;欲責一有罪,畏同惡者竦,隱忍而不誅。故忘身效節者抵噪於眾,僨軍緩救者畜奸不畏,褒貶稱毀,紛然相亂。公者直己不求諸人,則罹困厄;奸者行私茍媚於眾,則取優崇。此義士勇夫所以痛心解體也。又如遇敵而守不固,陳謀而功不成。責將帥,將帥曰資糧不足;責有司,有司曰須給無乏;更相為解,而朝廷含糊,未嘗究詰。故抱直者吞聲,罔上者不慚。馭眾若此,可謂課責虧度。二失也。



 以課責之虧,措置之乖,將不得竭其才,卒不得盡其力,屯集雖眾,無施戰陣,虜常橫行,以謂境無人焉。吏習其常,惟曰兵少不敵,朝廷莫之省,則又調發益師,無裨於備御,而有弊於供億。閭井日耗,斂求日繁,傾家析產,榷鹽稅酒,無慮所入半以事邊。制用若此,可謂財匱於兵眾矣。三失也。



 今四夷最強盛者,莫如吐蕃。舉吐蕃眾,未當中國十數大郡,而內虞外備與中國不殊,所以能寇邊者無幾。又器不犀利,甲不精完,材不趨敏。動則中國慹其眾不敢抗,靜則憚其強不敢侵,何哉?良以我之節制多,而彼之統帥一也。且節制多,則人心不一;人心不一,則號令不行;號令不行,則進退難必;進退難必,則疾徐失宜;疾徐失宜,則機會不及;機會不及,則氣勢自衰。斯乃勇廢為尪,眾失為弱。開元、天寶時,制西北二蕃,則朔方、河西、隴右三節度而已,尚慮權分,或詔兼領之。中興未遑外討,則僑四鎮隸安定,以隴右附扶風,所當二蕃,則朔方、涇原、隴右、河東四節度而已,以關東戍卒屬之。雖任未得人,而措置之法存焉。自賊泚亂以誘涇原,懷光反以污朔方,則分朔方為三節度,其鎮軍且四十,皆特詔任之,各有中人監軍,咸得相抗。既無軍法臨下,莫能稟屬,邊書告急,方使關白用兵,是謂從容拯溺,揖讓救焚矣。兵以氣若勢為用者也,氣聚則盛,散則消;勢合則威,析則弱。今之邊戍,勢弱氣消。建軍若此,可謂力分於將多矣。四失也。



 治戎之要,在均齊而已。故軍法無貴賤之差、多少之異,所以同其志、盡其力也。被邊長鎮之兵,皆百戰傷夷,角所能則習,度所處則危,考服役則勞,察臨敵則勇,然衣稟止於當身,又為家室所分,居常凍餒。而關東戍士,歲月更代,怯於應敵,懈於服勞,然衣稟優厚,繼以茶藥,資以蔬醬。豐寡相縣,勢則遠甚。又有以邊軍詭為奏請遙隸神策者,稟賜之饒,有三倍之益。此士類所以忿恨,經費所以褊匱。夫事業未異,給養頓殊,人情所不甘也。不為戎首,已可嘉者,況使協力同心,以攘寇難,臣知有所不能焉。養士若此,可謂怨生於不均矣。五失也。



 凡任將帥,必先考察行能,然後指所授之方、所委之要,令自揣可否,以見要領。須某甲兵,藉某參屬,用若干步騎,計若干資糧,何所列屯,何時成功,觀其言,校其實。若曰不足取,當艱之於初,不宜詒悔於後也;若曰可任,則當要之於終,不宜掣肘於內也。故疑者不使,使者不疑。勞神於拔選,端拱於委任,然後核否臧,信賞罰,受賞者不為濫,當罰者不敢辭,付授專則茍且之心息矣。是以古之遣將者,君推轂而命之,又賜鈇鉞,故軍容不入國,國容不入軍,機宜不以遠決,號令不以兩從。今陛下命帥,先求易制者,多其部使力分,輕其任使心弱。由是分閫責成之義廢,死綏任咎之志衰。一則聽命,二則聽命,止取承順可矣,若有意乎靖難則不可。兩疆相接,兩軍相持,事機所急,罅不留息,況千里之遠,九重之深,陳述之難明,聽覽之不專,欲事無遺策,雖聖亦有所不能焉。守戍者以寡不敢抗,分鎮者以無詔不敢救,逗留之頃,寇已奔逼。牧馬屯牛,鞠椎剽矣;嗇夫樵婦,罄俘囚矣。假令詔至發兵,更相顧望,莫敢遮礙,敗者減百為一,獲者衍百為千。帥守以總制在朝,不恤於罪;陛下以權出己,不究厥情。用帥若此,可謂機失於遙制矣。六失也。



 臣愚謂宜罷四方之防秋者,以其數析而三之:其一,責本道節度,募壯士願屯邊者徙焉;其一,則第以本道衣稟,責關內、河東募用蕃、夏子弟願傅軍者給焉;其一,以所輸資糧給應募者,以安其業。詔度支市牛,召工就諸屯繕完器具。至者家給牛一,耕耨水火之器畢具,一歲給二口糧,賜種子,勸之播蒔。須一年,則使自給,有餘粟者,縣官倍價以售。既息調發之煩,又無幸免之弊,出則人自為戰,處則家自為耕。與夫暫屯遽罷,豈同日論哉!然後建文武大臣一人為隴右元帥,自涇、隴、鳳翔薄長武城,盡山南西道,凡節度府之兵皆屬焉。又詔一人為朔方元帥,由鄜坊、邠寧揵靈夏,凡節度府之兵屬焉。又詔一人為河東元帥,舉河東,極振武,節度府之兵屬焉。各以臨邊要州為治所,所部州若府,遴柬良吏為刺史,外奉軍興,內課農桑,慎守中國所長,謹行當今所易,則八利可致,六失可去矣。



 帝愛重其言,不從也。



 班宏判度支,卒官,贄薦李巽,帝漫許之,而自用裴延齡。贄言:「延齡僻戾躁妄,不可用。」不聽。俄而延齡奸佞得君,天下仇惡,無敢言。贄上書苦諫,帝不懌,竟以太子賓客罷。贄本畏慎,未嘗通賓客。延齡揣帝意薄,讒短百緒,帝遂發怒,欲誅贄,賴陽城等交章論辨,乃貶忠州別駕。後稍思之,會薛延為刺史,諭旨慰勞。韋皋數上表請贄代領劍南,帝猶銜之,不肯與。順宗立,召還。詔未至,卒,年五十二。贈兵部尚書,謚曰宣。



 始,贄入翰林,年尚少,以材幸,天子常以輩行呼而不名。在奉天,朝夕進見,然小心精潔,未嘗有過,由是帝親倚,至解衣衣之,同類莫敢望。雖外有宰相主大議,而贄常居中參裁可否,時號「內相」。嘗為帝言:「今盜遍天下,宜痛自咎悔,以感人心。昔成湯罪己以興,楚昭王出奔,以一言善復國。陛下誠不吝改過,以言謝天下,使臣持筆亡所忌,庶叛者革心。」帝從之。故奉天所下制書,雖武人悍卒無不感動流涕。後李抱真入朝,為帝言:「陛下在奉天、山南時,赦令至山東,士卒聞者皆感泣思奮。臣是時知賊不足平。」議者謂興元戡難功,雖爪牙宣力,蓋贄有助焉。狩山南也,道險澀,與從官相失,夜召贄不得,帝驚且泣,詔軍中得贄者賞千金。久之,上謁,帝喜見顏間,自太子以下皆賀。及輔政,不敢自顧重,事有可否必言之,所言皆剴拂帝短,懇到深切。或規其太過者,對曰:「吾上不負天子,下不負所學,皇它恤乎?」既放荒遠,常闔戶,人不識其面。又避謗不著書,地苦瘴癘,只為《今古集驗方》五十篇示鄉人云。



 贊曰:德宗之不亡,顧不幸哉!在危難時聽贄謀,及已平,追仇盡言,怫然以讒幸逐,猶棄梗。至延齡輩,則寵任磐桓,不移如山,昏佞之相濟也。世言贄白罷翰林,以為與吳通玄兄弟爭寵,竇參之死,贄漏其言,非也。夫君子小人不兩進,邪諂得君則正士危,何可訾耶?觀贄論諫數十百篇,譏陳時病,皆本仁義,可為後世法,炳炳如丹,帝所用才十一。唐祚不競,惜哉!



\end{pinyinscope}