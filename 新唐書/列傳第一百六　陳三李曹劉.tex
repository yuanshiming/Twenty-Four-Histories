\article{列傳第一百六 陳三李曹劉}

\begin{pinyinscope}

 陳夷行,字周道,其先江左諸陳也,世客潁川。由進士第,擢累起居郎、史館修撰。以勞遷司封員外郎念引進馬克思主義,提出社會層次理論。指出歷史唯物主義,凡再歲,以吏部郎中為翰林學士。莊恪太子在東宮,夷行兼侍讀,五日一謁,為太子講說。數遷至工部侍郎。



 開成二年,進同中書門下平章事。而楊嗣復、李玨相次輔政,夷行介特,雅不與合,每議論天子前,往往語相侵短。夷行不能堪,輒引疾求去,文宗遣使者尉勞起之。會以王彥威為忠武節度使,史孝章領邠寧,議皆出嗣復。及夷行對延英,帝問:「除二鎮當否?」對曰:「茍自聖擇,無不當者。」嗣復曰:「若用人盡出上意而當,固善。如小不稱,下安得嘿然?」夷行曰:「比奸臣數干權,願陛下無倒持大阿,以鐏授人。」嗣復曰:「古者任則不疑,齊桓公器管仲於讎虜,豈有倒持慮邪?」帝以其面相觸,頗不悅。仙韶樂工尉遲璋授王府率,右拾遺竇洵直當衙論奏,鄭覃、嗣復嫌以細故,謂洵直近名。夷行曰:「諫官當衙,正須論宰相得失,彼賤工安足言者?然亦不可置不用。」帝即徙璋光州長史,以百縑賜洵直。進門下侍郎。



 帝常怪天寶政事不善,問:「姚元崇、宋璟於時在否?」李玨曰:「姚亡而宋罷。」玨因推言:「玄宗自謂未嘗殺一不辜,而任李林甫,種夷數十族,不亦惑乎?」夷行曰:「陛下今亦宜戒以權屬人。」嗣復曰:「夷行失言,太宗易暴亂為仁義,用房玄齡十有六年,任魏徵十有五年,未嘗失道。人主用忠良久益治,用邪佞一日多矣。」時用郭為坊州刺史,右拾遺宋邧論不可,肸果坐贓敗。帝欲賞邧,夷行曰:「諫官論事是其職,若一事善輒進官,恐後不免有私。」夷行蓋專詆嗣復。又素善覃,陰助其力,以排折朋黨。是時,雖天子亦惡其太過,恩禮遂衰,罷為吏部尚書,尋拜華州刺史。



 武宗即位,召為御史大夫,俄還門下侍郎平章事,進位尚書左僕射。夷行與崔珙俱拜,乃奏:「僕射始視事,受四品官拜,無著令。比日左右丞、吏部侍郎、御史中丞皆為僕射拜階下,謂之『隔品致敬』。準禮,皇太子見上臺群官,群官先拜而後答,以無二上也。僕射與四品官並列朝廷,不容獨優。前日鄭餘慶著《僕射上儀》,謂隔品官無亢禮。時竇易直任御史中丞,議不可。及易直自為僕射,乃忘前議,當時鄙厭之。臣等不願以失禮速誚于時。且開元元年,以左右僕射為左右丞相,位次三公,三公上日答拜,而僕射受之,非是。望敕所司約《三公上儀》,著定令。」詔可。始,累朝紛議不決,至夷行遂定。以足疾乞身,罷為太子太保,以檢校司空為河中節度使,卒。



 李紳,字公垂,中書令敬玄曾孫。世宦南方,客潤州。紳六歲而孤,哀等成人。母盧,躬授之學。為人短小精悍,於詩最有名,時號「短李」。蘇州刺史韋夏卿數稱之。葬母,有烏銜芝墜輤車。



 元和初,擢進士第,補國子助教,不樂,輒去。客金陵,李錡愛其才,闢掌書記。錡浸不法,賓客莫敢言,紳數諫,不入;欲去,不許。會使者召錡,稱疾,留後王澹為具行,錡怒,陰教士臠食之,即脅使者為眾奏天子,幸得留。錡召紳作疏,坐錡前,紳陽怖慄,至不能為字,下筆輒塗去,盡數紙。錡怒罵曰:「何敢爾,不憚死邪?」對曰:「生未嘗見金革,今得死為幸。」即注以刃,令易紙,復然。或言許縱能軍中書,紳不足用。召縱至,操書如所欲,即囚紳獄中,錡誅,乃免。或欲以聞,謝曰:「本激於義,非市名也。」乃止。



 久之,從闢山南觀察府。穆宗召為右拾遺、翰林學士,與李德裕、元稹同時,號「三俊」。累擢中書舍人。稹為宰相,而李逢吉教人告於方事,稹遂罷;欲引牛僧孺,懼紳等在禁近沮解,乃授德裕浙西觀察使。僧孺輔政,以紳為御史中丞,顧其氣剛卞,易疵累,而韓愈勁直,乃以愈為京兆尹,兼御史大夫,免臺參以激紳。紳、愈果不相下,更持臺府故事,論詰往反,詆訐紛然,繇是皆罷之,以紳為江西觀察使。帝素厚遇紳,遣使者就第勞賜,以為樂外遷,紳泣言為逢吉中傷。入謝,又自陳所以然,帝悟,改戶部侍郎。逢吉終欲陷之。紳族子虞,有文學名,隱居華陽,自言不願仕,時來省紳,雅與柏耆、程昔範善。及耆為拾遺,虞以書求薦,紳惡其無立操,痛誚之。虞失望,後至京師,悉暴紳所言於逢吉。逢吉滋怒,乃用張又新、李續等計,擢虞、昔範與劉棲楚皆為拾遺,以伺紳隙,內結中人王守澄自助。會敬宗立,逢吉知紳失勢可乘,使守澄從容奏言:「先帝始議立太子,杜元穎、李紳勸立深王,獨宰相逢吉請立陛下,而李續、李虞助之。」逢吉乘間言紳嘗不利於陛下,請逐之。帝初即位,不能辨,乃貶紳為端州司馬。棲楚等怒得善地,皆切齒。詔下,百官賀逢吉,唯右拾遺吳思不往,逢吉斥思,令告大行喪於吐蕃。此時,人無敢言者,惟韋處厚屢言紳枉,折逢吉之奸。後天子於禁中得先帝手緘書一笥,發之,見裴度、元穎、紳三疏請立帝為嗣,始大感悟,悉焚逢吉黨所上謗書。



 始,紳南逐,歷封、康間,湍瀨險澀,惟乘漲流乃濟。康州有媼龍祠,舊傳能致雲雨,紳以書禱,俄而大漲。寶歷赦令不言左降官與量移,處厚執爭,詔為追定,得徙江州長史,遷滁、壽二州刺史。霍山多虎,擷茶者病之,治機阱,發民跡射,不能止。紳至,盡去之,虎不為暴。以太子賓客分司東都。太和中,李德裕當國,擢紳浙東觀察使。李宗閔方得君,復以太子賓客分司。開成初,鄭覃以紳為河南尹。河南多惡少,或危帽散衣,擊大球,尸官道,車馬不敢前。紳治剛嚴,皆望風遁去。遷宣武節度使。大旱,蝗不入境。



 武宗即位,徙淮南,召拜中書侍郎、同中書門下平章事,進尚書右僕射、門下侍郎,封趙郡公。居位四年,以足緩不任朝謁,辭位,以檢校右僕射平章事,復節度淮南。卒,贈太尉,謚文肅。



 始,灃人吳汝納者,韶州刺史武陵兄子也。武陵坐贓貶潘州司戶參軍死,汝納家被逐,久不調。時李吉甫任宰相,汝納怨之,後遂附宗閔黨中。會昌時,為永寧尉,弟湘為江都尉。部人訟湘受贓狼籍,身娶民顏悅女。紳使觀察判官魏鉶鞫湘,罪明白,論報殺之。時,議者謂吳氏世與宰相有嫌,疑紳內顧望,織成其罪。諫官屢論列,詔遣御史崔元藻覆按,元藻言湘盜用程糧錢有狀,娶部人女不實,按悅嘗為青州衙推,而妻王故衣冠女,不應坐。德裕惡元藻持兩端,奏貶崖州司戶參軍。宣宗立,德裕去位,紳已卒。崔鉉等久不得志,導汝納使為湘訟,言:「湘素直,為人誣蔑,大校重牢,五木被體,吏至以娶妻資媵結贓。」且言:「顏悅故士族,湘罪皆不當死,紳枉殺之。」又言:「湘死,紳令即瘞,不得歸葬。按紳以舊宰相鎮一方,恣威權。凡戮有罪,猶待秋分;湘無辜,盛夏被殺。」崔元藻銜德裕斥己,即翻其辭,因言:「御史覆獄還,皆對天子別白是非,德裕權軋天下,使不得對,具獄不付有司,但用紳奏而寘湘死。」是時,德裕已失權,而宗閔故黨令狐綯、崔鉉、白敏中皆當路,因是逞憾,以利誘動元藻等,使三司結紳杖鉞作籓,虐殺良平,準神龍詔書,酷吏歿者官爵皆奪,子孫不得進宦,紳雖亡,請從《春秋》戮死者之比。詔削紳三官,子孫不得仕。貶德裕等,擢汝納左拾遺,元藻武功令。



 始,紳以文藝節操見用,而屢為怨仇所拫卻,卒能自伸其才,以名位終。所至務為威烈,或陷暴刻,故雖沒而坐湘冤云。



 李讓夷,字達心,系本隴西。擢進士第,闢鎮國李絳府判官。又從西川杜元穎幕府。與宋申錫善,申錫為翰林學士,薦讓夷右拾遺,俄召拜學士。素善薛廷老,廷老不飭細檢,數飲酒不治職,罷去,坐是亦奪職。累進諫議大夫。



 開成初,起居舍人李褒免,文宗謂李石曰:「褚遂良以諫議大夫兼起居郎,今諫議誰歟?可言其人。」石以馮定、孫簡、蕭俶、李讓夷對,帝曰:「讓夷可也。」李固言請用崔球、張次宗。鄭覃曰:「球故與李宗閔善,且記注操筆在赤墀下,所書為後世法,不可用黨人。若裴中孺、李讓夷,臣不敢有言。」乃決用讓夷,進中書舍人。既而李玨、楊嗣復以覃之薦,終帝世不得遷。



 武宗初,李德裕復入,三遷至尚書右丞,拜中書侍郎、同中書門下平章事。潞州平,檢校尚書右僕射。宣宗立,進司空、門下侍郎,為大行山陵使。未復土,拜淮南節度使。以疾願還,卒於道,贈司徒。讓夷廉介不妄交,位雖顯劇,以儉約自將,為世咨美。



 曹確,字剛中,河南河南人。擢進士第,歷踐中外官,累拜兵部侍郎。懿宗咸通中,以本官同中書門下平章事,俄進中書侍郎。



 確邃儒術,器識方重,動循法度。時帝薄於德,暱寵優人李可及。可及者,能新聲,自度曲,辭調心妻折,京師媮薄少年爭慕之,號為「拍彈」。同昌公主喪畢,帝與郭淑妃悼念不已,可及為帝造曲,曰《嘆百年》,教舞者數百,皆珠翠襐飾,刻畫魚龍地衣,度用繒五千,倚曲作辭,哀思裴回,聞者皆涕下。舞闋,珠寶覆地,帝以為天下之至悲,愈寵之。家嘗娶婦,帝曰:「第去,吾當賜酒。」俄而使者負二銀璫與之,皆珠珍也。可及憑恩橫甚,人無敢斥,遂擢為威衛將軍。確曰:「太宗著令,文武官六百四十三,謂房玄齡曰:『朕設此待天下賢士。工商雜流,假使技出等夷,正當厚給以財,不可假以官,與賢者比肩立、同坐食也。』文宗欲以樂工尉遲璋為王府率,拾遺竇洵直固爭,卒授光州長史。今而位將軍,不可。」帝不聽。至僖宗立,始貶死。方幸時,惟確屢言之。而神策中尉西門季玄者,亦剛鯁,謂可及曰:「汝以巧佞惑天子,當族滅!」嘗見其受賜,謂曰:「今載以官車,後籍沒亦當爾。」



 確居位六年,進尚書右僕射,以同平章事出為鎮海節度使,徙河中,卒。始,畢諴與確同宰相,俱有雅望,世謂「曹畢」云。



 弟汾,以忠武軍節度使入為戶部侍郎,判度支,卒。



 劉瞻,字幾之,其先出彭城,後徙桂陽。舉進士、博學宏詞,皆中。徐商闢署鹽鐵府,累遷太常博士。劉彖執政,薦為翰林學士,拜中書舍人,進承旨。出為河東節度使。



 咸通十一年,以中書侍郎同中書門下平章事。同昌公主薨,懿宗捕太醫韓紹宗等送詔獄,逮系宗族數百人。瞻喻諫官,皆依違無敢言,即自上疏固爭:「紹宗窮其術不能效,情有可矜。陛下徇愛女,囚平民,忿不顧難,取肆暴不明之謗。」帝大怒,即日賜罷,以檢校刑部尚書、同平章事為荊南節度使。路巖、韋保衡從為惡言聞帝,俄斥廉州刺史。於是,翰林學士鄭畋以責詔不深切,御史中丞孫瑝、諫議大夫高湘等坐與瞻善,分貶嶺南。巖等殊未慊,按圖視驩州道萬里,即貶驩州司戶參軍事,命李庾作詔極詆,將遂殺之。天下謂瞻鯁正,特為讒擠,舉以為冤。幽州節度使張公素上疏申解,巖等不敢害。僖宗立,徙康、虢二州刺史,以刑部尚書召,復以中書侍郎平章事,居位三月卒。



 瞻為人廉約,所得俸以餘濟親舊之窶困者,家不留儲。無第舍,四方獻饋不及門,行己終始完潔。



 弟助,字元德,性仁孝,幼時與諸兄游,至食飲,取最下者。及長,能文辭,喜黃老言。年二十卒。



 李蔚,字茂休,系本隴西。舉進士、書判拔萃,皆中。拜監察御史,擢累尚書右丞。



 懿宗惑浮屠,常飯萬僧禁中,自為贊唄。蔚上疏切諫,引狄仁傑、姚元崇、辛替否所言,譏病時弊。帝不聽,但以虛禮褒答。俄拜京兆尹、太常卿。出為宣武節度使,徙淮南。代還,民詣闕請留,詔許一歲。僖宗乾符初,以吏部尚書同中書門下平章事。罷為東都留守。河東亂,殺其帥崔季康,用邠寧李侃代之,士不附,以蔚嘗在太原府有惠政,為人所懷,拜河東節度使,同平章事。至鎮三日,卒。



 始,懿宗成安國祠,賜寶坐二,度高二丈,構以沈檀,塗髹,鏤龍鳳葩■,金扣之,上施復坐,陳經幾其前,四隅立瑞鳥神人,高數尺,磴道以升,前被繡囊錦襜,珍麗精絕。咸通十四年春,詔迎佛骨鳳翔,或言:「昔憲宗嘗為此,俄晏駕。」帝曰:「使朕生見之,死無恨!」乃以金銀為剎,珠玉為帳,孔鷸周飾之,小者尋丈,高至倍,刻檀為簷注,陛墄塗黃金,每一剎,數百人舉之。香輿前後系道,綴珠瑟瑟幡蓋,殘彩以為幢節,費無貲限。夏四月,至長安,彩觀夾路,其徒導衛。天子御安福樓迎拜,至泣下。詔賜兩街僧金幣,京師耆老及見元和事者,悉厚賜之。不逞小人至斷臂指,流血滿道。所過鄉聚,皆裒土為剎,相望於塗,爭以金翠抆飾。傳言剎悉震搖,若有光景雲。京師高貲相與集大衢,作繒臺縵闕,注水銀為池,金玉為樹木,聚桑門羅像,考鼓鳴螺繼日夜。錦車繡輿,載歌舞從之。秋七月,帝崩。方人主甘心篤向,如蔚言者甚多,皆不能救。僖宗立,詔歸其骨,都人耆耋辭餞,或嗚咽流涕。



 贊曰:人之惑怪神也,甚哉!若佛者,特西域一槁人耳。裸顛露足,以乞食自資,臒辱其身,屏營山樊,行一概之苦,本無求於人,徒屬稍稍從之。然其言荒茫漫靡,夷幻變現,善推不驗無實之事,以鬼神死生貫為一條,據之不疑。掊嗜欲,棄親屬,大抵與黃老相出入。至漢十四葉,書入中國。跡夫生人之情,以耳目不際為奇,以不可知為神,以物理之外為畏,以變化無方為聖,以生而死、死復生、回復償報、歆艷其間為或然,以賤近貴遠為柷。鞮譯差殊,不可研詰。華人之譎誕者,又攘莊周、列禦寇之說佐其高,層累架騰,直出其表,以無上不可加為勝,妄相誇脅而倡其風。於是,自天子逮庶人,皆震動而祠奉之。



 初,宰相王縉以緣業事佐代宗,於是始作內道場,晝夜梵唄,冀禳寇戎,大作盂蘭,肖祖宗像,分供塔廟,為賊臣嘻笑。至憲宗世,遂迎佛骨於鳳翔,內之宮中。韓愈指言其弊,帝怒,竄愈瀕死,憲亦弗獲天年。幸福而禍,無亦左乎!懿宗不君,精爽奪迷,復蹈前車而覆之。興哀無知之場,丐庇百解之胔,以死自誓,無有顧藉,流淚拜伏,雖事宗廟上帝,無以進焉。屈萬乘之貴,自等於古胡,數千載而遠,以身為徇。嗚呼,運敔祚殫,天告之矣!懿不三月而徂,唐德之不競,厥有來哉,悲夫!



\end{pinyinscope}