\article{列傳第二十 高竇}

\begin{pinyinscope}

 高儉,字士廉,以字顯,齊清河王岳之孫,父勵樂安王面證明上帝的存在,並把世界描繪成遞相依屬的等級結構。斷,入隋為洮州刺史。士廉敏惠有度量,狀貌若畫,觀書一見輒誦,敏於占對。隋司隸大夫薛道衡、起居舍人崔祖浚皆宿臣顯重,與為忘年友,繇是有名。自以齊宗室,不欲廣交,屏居終南山下。吏部侍郎高孝基勸之仕,仁壽中,舉文才甲科,補治禮郎。斛斯政奔高麗,坐與善,貶為硃鳶主簿,以母老不可居瘴癘地,乃留妻鮮於奉養而行。會世大亂,京師阻絕,交趾太守丘和署司法書佐。時欽州俚帥寧長真以兵侵交趾,和懼,欲出迎,士廉曰:「長真兵雖多,縣軍遠客,勢不得久,城中勝兵尚可戰,奈何受制於人?」和因命為行軍司馬,逆擊破之。



 高祖遣使徇嶺南,武德五年與和來降,於是秦王領雍州牧,薦士廉為治中,親重之。隱太子與王隙已熾,乃與長孫無忌密計計定,是日率吏卒釋囚授甲,趨芳林門助戰。王為皇太子,授右庶子。進侍中,封義興郡公。坐匿王珪奏不時上,左授安州都督。



 進益州大都督府長史。蜀人畏鬼而惡疾,雖父母病皆委去,望舍投餌哺之,昆弟不相假財。士廉為設條教,辯告督勵,風俗翕然為變。又引諸生講授經藝,學校復興。秦時李冰導汶江水灌田,瀕水者頃千金,民相侵冒。士廉附故渠廝引旁出,以廣溉道,人以富饒。



 入為吏部尚書,進封許國公。雅負裁鑒,又詳氏譜,所署州,人地無不當者。高祖崩,攝司空,營山陵;加特進,遷尚書右僕射。士廉三世居此官,世榮其貴。



 太宗幸洛陽,太子監國,命攝少師。手詔曰:「端拱三川,不憂關中者,以屬卿也。」久之,請致仕,聽解僕射,加開府儀同三司、同中書門下三品,知政事。帝伐高麗,皇太子監國駐定州,又攝太傅,同掌機務。太子令曰:「寡人資公訓道,而比聽政,據桉對公,情所未安,所司宜別設桉奉太傅。」士廉固辭。



 還至並州,有疾,帝即所舍問之。貞觀二十一年疾甚,帝幸其第,為流涕,卒年七十一。又欲臨吊,房玄齡以帝餌金石,諫不宜近喪。帝曰:「朕有舊故姻戚之重,君臣之分,卿置勿言。」即從數百騎出。長孫無忌伏馬前,陳士廉遺言,乞不臨喪,帝猶不許,無忌至流涕,乃還入東苑,南向哭。詔贈司徒、並州都督,謚曰文獻,陪葬昭陵。方寒食,敕尚宮以食四舉往祭,帝自為文。喪出橫橋,又登城西北樓望哭以過喪。高宗即位,加贈太尉,配享太宗廟廷。



 士廉進止詳華,凡有獻納,搢紳皆屬以目。奏議未嘗不焚稿,家人無見者。士廉少識太宗非常人,以所出女歸之,是為文德皇后。及遺令墓不得它藏,惟置衣一襲與平生所好書示先王典訓可用終始者。



 初,太宗嘗以山東士人尚閥閱,後雖衰,子孫猶負世望,嫁娶必多取貲,故人謂之賣昏。由是詔士廉與韋挺、岑文本、令狐德棻責天下譜諜,參考史傳,檢正真偽,進忠賢,退悖惡,先宗室,後外戚,退新門,進舊望,右膏粱,左寒畯,合二百九十三姓,千六百五十一家,為九等,號曰《氏族志》,而崔幹仍居第一。帝曰:「我於崔、盧、李、鄭無嫌,顧其世衰,不復冠冕,猶恃舊地以取貲,不肖子偃然自高,販鬻松檟,不解人間何為貴之?齊據河北,梁、陳在江南,雖有人物,偏方下國,無可貴者,故以崔、盧、王、謝為重。今謀士勞臣以忠孝學藝從我定天下者,何容納貨舊門,向聲背實,買昏為榮耶?太上有立德,其次有立功,其次有立言,其次有爵為公、卿、大夫,世世不絕,此謂之門戶。今皆反是,豈不惑邪?朕以今日冠冕為等級高下。」遂以崔乾為第三姓,班其書天下。



 高宗時,許敬宗以不敘武後世,又李義府恥其家無名,更以孔志約、楊仁卿、史玄道、呂才等十二人刊定之,裁廣類例,合二百三十五姓,二千二百八十七家,帝自敘所以然。以四后姓、酅公、介公及三公、太子三師、開府儀同三司、尚書僕射為第一姓,文武二品及知政事三品為第二姓,各以品位高下敘之,凡九等,取身及昆弟子孫,餘屬不入,改為《姓氏錄》。當時軍功入五品者,皆升譜限,搢紳恥焉,目為「勛格」。義府奏悉索《氏族志》燒之。又詔後魏隴西李寶,太原王瓊,滎陽鄭溫,範陽盧子遷、盧澤、盧輔,清河崔宗伯、崔元孫,前燕博陵崔懿,晉趙郡李楷,凡七姓十家,不得自為昏;三品以上納幣不得過三百匹,四品五品二百,六品七品百,悉為歸裝,夫氏禁受陪門財。先是,後魏太和中,定四海望族,以寶等為冠。其後矜尚門地,故《氏族志》一切降之。王妃、主婿皆取當世勛貴名臣家,未嘗尚山東舊族。後房玄齡、魏徵、李勣復與昏,故望不減,然每姓第其房望,雖一姓中,高下縣隔。李義府為子求昏不得,始奏禁焉。其後天下衰宗落譜,昭穆所不齒者,皆稱「禁昏家」,益自貴,凡男女皆潛相聘娶,天子不能禁,世以為敝云。士廉六子,履行、審行、真行有名。



 履行居母喪毀甚,太宗諭使強食。尚東陽公主,襲爵。繇戶部尚書為益州大都督府長史,政有名。坐長孫無忌,左授洪州都督,改永州刺史。



 真行至左衛將軍。其子岐連章懷太子事,詔令自誡切,真行以佩刀刺殺之,斷首棄道上,高宗鄙其為,貶睦州刺史。



 審行自戶部侍郎貶渝州刺史。



 士廉五世孫重,字文明,以明經中第,李巽表鹽鐵轉運巡官,善職,凡十年,進累司門郎中。



 敬宗慎置侍講學士,重以簡厚惇正,與崔郾偕選,再擢國子祭酒。文宗好《左氏春秋》,命分列國各為書,成四十篇。與鄭覃刊定《九經》於石。出為鄂岳觀察使,以美政被褒。久之,拜太子賓客,分司東都。卒,贈太子少保。



 贊曰:古者受姓受氏以旌有功,是時人皆土著,故名宗望姓,舉郡國自表,而譜系興焉,所以推敘昭穆,使百代不得相亂也。遭晉播遷,胡醜亂華,百宗蕩析,士去墳墓,子孫猶挾系錄,以示所承,而閥閥顯者,至賣昏求財,汨喪廉恥。唐初流弊仍甚,天子屢抑不為衰。至中葉,風教又薄,譜錄都廢,公靡常產之拘,士亡舊德之傳,言李悉出隴西,言劉悉出彭城,悠悠世詐,訖無考按,冠冕皁隸,混為一區,可太息哉!



 竇威,字文蔚,岐州平陸人。父熾,在周為上柱國,入隋為太傅,太穆皇后,其從兄弟女也。



 威沈邃有器局,貫覽群言,家世貴,子弟皆喜武力,獨威尚文,諸兄詆為書癡。內史令李德林舉秀異,授秘書郎,當遷不肯調者十年,故其學益博。而諸兄以軍功位通顯矣,薄威職閑冗,更謂曰:「昔仲尼積學成聖,猶棲遲不偶,汝尚何求耶?」威笑不答。蜀王秀闢為記室,威以秀多不法,謝疾去。秀廢,府屬皆得罪,威獨免。大業中,累遷內史舍人,數諫忤旨,轉考功郎中,後坐事免。



 高祖入關,召補大丞相府司錄參軍。方天下亂,禮典湮缺,威多識朝廷故事,乃裁定制度。帝語裴寂曰:「威,今之叔孫通也。」武德元年,授內史令。每論政事得失,必陳古為諭,帝益親矚,嘗引入臥內,謂曰:「昔周有八柱國,吾與公家是也。今我為天子,而公為內史令,事固有不等耶?」威懼,頓首謝曰:「臣家在漢,再為外戚。至元魏,有三皇后。今陛下龍興,臣復以姻戚進,夙夜懼不克任。帝笑曰:「公以三后族誇我邪!關東人與崔、盧婚者,猶自矜大,公世為帝戚,不亦貴乎。」



 後寢疾,帝臨問,及卒,哭之慟。贈同州刺史,追封延安郡公,謚曰靖。威性儉素,家不樹產,比喪,無餘貲,遺令薄葬。詔皇太子、百官臨送。



 兄子軌,字士則。父恭,仕周為雍州牧、酂國公。軌性剛果有威,大業中,為資陽郡東曹掾,去官歸。高祖起兵,軌募眾千餘人迎謁長春宮。帝大悅,賜良馬十匹,使略地渭南,下永豐倉,收兵五千,從平京師。封贊皇縣公,為大丞相諮議參軍。



 稽胡賊五萬掠宜春,詔軌討之。次黃欽山,遇賊乘高叢射,眾為卻。軌斬部將十四人,更拔其次代之,身擁數百騎殿,令曰:「聞鼓不進者斬。」既鼓,士爭赴賊,賊射不勝,大破之,斬首千級,獲男女二萬。擢太子詹事。赤排羌與薛舉叛將鐘俱仇寇漢中,拜秦州總管,討賊連戰有功,餘黨悉降。復酂國舊封,遷益州道行臺左僕射。黨項引吐谷渾寇松州,詔軌與扶州刺史蔣善合援之。善合先期至,敗之鉗川。軌進軍臨洮,擊左封,走其眾。度羌必為患,始屯田松州。詔率所部兵從秦王討王世充。明年,還蜀。



 軌既貴,益嚴酷,然能自勤苦,每出師臨敵,身未嘗解甲,其下有不用命即誅,至小過亦鞭棰流血,人見者皆重足股心慄,由是蜀盜悉平。初,以其甥為腹心,嘗夜出,呼不時至,斬之。又戒家奴毋出外,忽遣奴取漿公廚,既而悔焉,曰:「要當借汝頭以明法。」命斬奴,奴稱冤,監刑者疑不時決,軌並斬之。後入朝,賜坐御樓,容不肅,又坐對詔,帝怒曰:「公入蜀,車騎、驃騎從者二十人,公斬誅略盡,我隴種車騎,尚不足給公。」因系詔獄。俄釋之,還鎮益州。



 軌與行臺尚書韋雲起、郭行方素不協,及隱太子誅,詔至,軌內詔懷中,雲起問詔安在,軌不肯示,因執殺之。行方懼,奔京師,得免。是歲,行臺廢,授益州都督,加食邑戶六百。



 貞觀元年,召授右衛大將軍,出為洛州都督。周洛間,因隋亂,人不土著,軌下令諸縣,有游手末作者按之,由是威信大行,民皆趨本。卒,贈並州都督。子奉節,尚永嘉公主,歷左衛將軍、秦州都督。



 軌弟琮,有武幹。大業末,犯法亡命太原,依高祖。與秦王有憾,不自安。王方收天下豪英,降禮接之,與出入臥內,琮意乃釋。大將軍府建,引為統軍。從平西河,破霍邑。授金紫光祿大夫,封扶風郡公。從劉文靜擊屈突通於潼關,敗其將桑顯和,通遁去,琮以輕騎追獲於稠桑。進兵下陜縣,拔太原倉。遷左領軍大將軍,賜物五百段。隋河陽都尉獨孤武潛謀歸款,命琮總萬騎,自柏崖迎之,逗留不進,武見殺,坐除名。武德初,為右屯衛大將軍。時將圖洛陽,詔琮留守陜,護饟道。王世充將羅士信數以兵鈔絕,琮使人說降之。東都平,檢校晉州總管。從隱太子平劉黑闥,以功封譙國公,賜黃金五十斤。卒,贈左衛大將軍,謚曰敬。永徽五年,加贈特進。



 威從兄子抗,字道生。父榮定,為隋洺州總管、陳國公,謚曰懿。母,隋文帝姊安成公主也。抗美容儀,性通率,涉見圖史。以帝甥蚤貴,入太學,釋褐千牛備身、儀同三司。侍父疾,束帶五旬不弛;居喪,哀臒過常。襲爵,累轉梁州刺史。將之官,文帝幸其第,酣宴如家人禮。母卒,數號絕。詔宮人節哭。歲餘,為岐州刺史,轉幽州總管,所至以寬惠聞。漢王諒反,煬帝疑抗為應,遣李子雄馳往代之。子雄因誣抗得諒書不奏,按鞫無狀,然坐是遂廢。



 抗與高祖少相狎,及楊玄感反,抗謂高祖曰:「玄感為我先耳,李氏名在圖錄,天所啟也。」高祖曰:「為禍始不祥,公無妄言。」煬帝遣抗出靈武,逴護長城,聞高祖已定京師,喜曰:「此吾家婿,豁達有大度,真撥亂主也。」因歸長安。高祖見之喜,握手曰:「李氏果王,何如?」因置酒為樂,授將作大匠兼納言,尋罷為左武候大將軍。



 帝聽朝,或引升御坐,既退,入臥內,從容談笑,極平生歡,以兄呼之,宮中稱為舅,或留宿禁省,侍燕豫,然未嘗干朝廷事。後從秦王平薛舉,功第一;又從征王世充。東都平,冊勛於廟者九人,抗與從弟軌與焉。賜女樂一部,珍幣不貲。卒,贈司徒,謚曰密。子衍、靜、誕,衍襲爵。



 靜字元休,在隋佐親衛,以父得罪煬帝,久不之進。高祖入京師,擢並州大總管府長史。時突厥數為邊患,糧道不屬,靜表請屯田太原,以省饋運。議者以流亡未復,不宜重困,於是召入與裴寂、蕭瑀、封倫廷議,寂等不能屈,帝從之,歲收粟十萬斛。詔檢校並州大總管。又請斷石嶺以為鄣塞,制突厥之入。太宗即位,授司農卿,封信都縣男。趙元楷為少卿,靜鄙其聚斂,因會官屬大言曰:「如煬帝奢侈,竭四海自奉,司農須公矣。今天子躬節儉,屈一人安兆庶,惡用公哉?」元楷大慚。改夏州都督。突厥攜貳,諸將出征者過靜,靜為陳虜中虛實,諸將由是大克獲。又間其部落,鬱射所部鬱孤尼等九俟斤皆內附。帝嘉之,賜馬百匹、羊千口。及禽頡利,詔處其眾河南。靜上書曰:「夷狄窮則搏噬,飽則群聚,不可以刑法繩、仁義教也。衣食仰給,不恃耕桑。今損有為之民,資無知之虜,得之無益於治,失之不害於化。況首丘未忘,則一旦變生,犯我王略矣。不如因其破亡,假以賢王一號,妻之宗女,披其土地部落,使權弱勢分,易為羈制,則世為籓臣矣。」帝雖不從,然嘉其忠,優詔答曰:「北方之務,悉以相委,以卿為寧朔大使,朕無北顧憂矣。」再遷民部尚書。卒,謚曰肅。子逵,尚遂安公主,襲爵。



 誕,隋末起家朝請郎。義寧初,闢丞相府祭酒,封安豐郡公,尚襄陽公主。從秦王征薛舉,為元帥府司馬。累遷太常卿。高祖諸子幼,未出宮者十餘王,國司家事,皆誕主之。出為梁州都督。貞觀初,召授右領軍大將軍,進莘國公,為宗正卿。太宗與語,昏謬失對。乃下詔曰:「誕比衰耗,不能事,朕知而任之,是謂不明。且為官擇人者治,為人擇官者亂。其以光祿大夫罷就第。」卒,贈工部尚書、荊州刺史,謚曰安。



 抗弟璡,字之推,性沈厚。隋大業末,為扶風太守。唐兵起,以郡歸,歷民部尚書。從秦王平薛仁杲,賜錦袍。尋鎮益州,時蜀盜賊多,皆討平之。與皇甫無逸不協,數相訴毀,因請入朝,至半道,詔還之。璡內憂恐。會使者至,璡引宴臥內,厚餉遺。無逸以聞,坐免官。未幾,授秘書監,封鄧國公。貞觀初,遷將作大匠,詔脩洛陽宮,鑿池起山,務極侈浮,費不勝算。太宗怒,詔毀之,免其官。以酆王納璡女為妃,復位。卒,贈禮部尚書,謚曰安。璡有巧思,工書。武德中,與太常少卿祖孝孫受詔定雅樂,是正鐘律云。



 威從孫德玄,隋大業中,起家國學生。祖照,尚周文帝義陽公主,封鉅鹿郡公。父彥,襲爵,終隋西平太守。兄德明,師事陳留王孝逸,通知文史。漢王諒反,遣將綦良攻黎州。德明年十八,募士五千,號令嚴整,倍道擊賊,破之。以功擢累齊王府屬。坐事免。高祖兵叩長安,而宗室孝基、神符、道宗及竇誕、趙慈景等並系獄,隋將衛文升、陰世師欲殺之,德明諫曰:「罪不在此,殺之無傷於彼,祗取怨焉,不如縱之。」乃止。長安平,謁高祖,終不自言,時稱長者。拜考功郎中。從秦王擊王世充。封顯武男,歷常、愛二州刺史,卒。德玄始為高祖丞相府千牛,歷太宗時不甚顯,高宗以舊臣,自殿中少監為御史大夫,歲中遷司元太常伯。時帝又以源直心為奉常正卿,劉祥道為司刑太常伯,上官儀為西臺侍極,郝處俊為太子左中護,凡十餘人,皆帝自擇,以示宰相李勣等,皆頓首謝。麟德初,進檢校左相,勤職約己,天子嘗臨朝,咨其清素,加以賜賚。居位數年,贊圖封禪事,與李勣皆為使。帝次濮陽,問古謂帝丘,德玄不能對,許敬宗具道其然,帝稱善。敬宗自矜於人,德玄知,不為忤,眾服其量。禮成,進爵二級。以弟德遠未及爵,願分封,詔可,故德玄封鉅鹿男,德遠樂安男。德玄迎時取合,未嘗有過,然無它補益。卒,年六十九,贈光祿大夫,幽州都督,謚曰恭。



 贊曰:高、竇雖緣外戚姻家,然自以才猷結天子,廁跡名臣,垂榮無窮,時有遇合,故見諸事業。古來賢豪,不遭與運,埋光鏟採,與草木俱腐者,可勝吒哉!竇宗自魏訖唐,支胄扶疏數百年,所馮厚矣。



\end{pinyinscope}