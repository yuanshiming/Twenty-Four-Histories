\article{列傳第十九 侯張薛}

\begin{pinyinscope}

 侯君集,豳州三水人。以材雄稱。少事秦王幕府,從征討有功,擢累左虞候、車騎將軍,封全椒縣子。預誅隱太子尤力。王即位展過程中的主體。「絕對精神」的發展經歷了邏輯、自然、精,拜左衛將軍,進封潞國公,邑千戶。貞觀四年,遷兵部尚書,俄檢校吏部尚書,參議朝政。



 李靖討吐谷渾,以君集為積石道行軍總管。師次鄯州,議所向。君集曰:「王師已至,而賊不走險,天贊我也。若以精兵掩不備,彼不我虞,必有大利。若遁岨山谷,克之實難。」靖然其計,簡銳士,約齎深入,追及其眾於庫山,大戰,破之,進會大非川,平其國。



 會詔世封功臣,授陳州刺史,更封陳;群臣不願封,進吏部尚書。君集本以行伍奮,不知學;後貴,益自喜,好書。及典選,分明課最,有譽於時。



 吐蕃圍松州,授當彌道行軍大總管以擊之。高昌不臣,拜交河道行軍大總管出討。王曲文泰笑曰:「唐去我七千里,磧鹵二千里無水草,冬風裂肌,夏風如焚,行賈至者百之一,安能致大兵乎?使能頓吾城下一再旬,食盡當潰,吾且系而虜之。」君集次磧口,而文泰死,子智盛襲位。進營柳谷,候騎言國方葬死君,諸將請襲之。君集曰:「不可,天子以高昌驕慢,使吾龔行天罰,今襲人於墟墓間,非問罪也。」於是鼓而前。賊嬰城自守,遣諭之,不下。乃刊木塞塹,引撞車毀其堞,飛石如雨,所向無敢當,因拔其城,俘男女七千,進圍都城,初,文泰與西突厥欲谷設約,有急相援。及是,欲谷設益懼,西走,智盛失援,乃降。高昌平,君集刻石紀功還。



 初,君集配沒罪人不以聞,又私取珍寶、婦女,將士因亦盜入,不能制。及還京師,有司劾之,詔君集詣獄簿對。中書侍郎岑文本諫曰:「高昌之罪,議者以其遐遠,欲置度外,唯陛下奮獨見之明,授決勝之略,君集得指期平殄。今推勞將帥,從征之人悉蒙重賞,未逾數日,更以屬吏,天下聞之,謂陛下錄過遺功,無以勸後。且古之出師,克敵有重賞,不勝蒙顯戮。當其有功也,雖貪財縱欲,尚蒙爵邑;其無功也,雖勤躬潔己,不免鈇鉞。故曰:『記人之功,忘人之過,宜為君者也。』昔李廣利貪不愛卒,陳湯盜所收康居財物,二主皆赦其罪,封侯賜金。夫將帥之臣,廉慎少而貪沒多。軍法曰:『使智,使勇,使貪,使愚。故智者樂立其功,勇者好行其志,貪者邀趨其利,愚者不計其死。』是以前聖使人,必收所長而棄所短。陛下宜申宥君集,俾復朝列,以勸有功。」帝寤,釋不問。



 君集自恃有功,以它罪被系,居怏怏不平。會張亮出洛州都督,君集謬激說曰:「何為見排?」亮曰:「公排我,尚誰咎?」君集曰:「我平一國還,觸天子嗔,何能排君?」因攘袂曰:「鬱鬱不可活,能反乎?當與公反。」亮密以聞。帝曰:「卿與君集皆功臣,今獨相語而無左驗,奈何?」秘不發,待君集如初。皇太子承乾數有過,慮廢,知君集犯望,因其婿賀蘭楚石為千牛,私引君集入,問自安計。君集舉手謂曰:「此手當為殿下用之。」又遣楚石語承乾曰:「魏王得愛,陛下若有詔召,願毋輕入。」承乾納之。然君集常畏謀洩,忽忽不自安,或中夕驚吒,妻怪之,曰:「公,國大臣,何為爾?若有所負,宜自歸,首領尚可全。」不從。



 承乾事覺,捕君集下獄。楚石告狀,帝自臨問,曰:「我不欲令刀筆吏辱公。君集辭窮不能對。帝語群臣曰:「君集於國有功,朕不忍置諸法,將丐其命,公卿其許我乎?」君臣皆曰:「君集罪大逆不道,請論如法。」帝乃謂曰:「與公訣矣,今而後,徒見公遺像已!」因泣下,遂斬之,籍其家。君集臨刑色不變,謂監吏曰:「我豈反者乎?蹉跌至此。然嘗為將,破二國,若言之陛下,丐一子以守祭祀。」帝聞,原其妻及一子,徙嶺表。



 始,帝命李靖教君集兵法,既而奏:「靖且反,兵之隱微,不以示臣。」帝以讓靖,靖曰:「方中原無事,臣之所教,足以制四夷,而求盡臣術,此君集欲反耳。靖為右僕射,君集為兵部尚書,同還省,君集馬過門數步乃覺,靖語人曰:「君集其有異慮乎?」後果如言。



 張亮,鄭州滎陽人。起畎畝,志趣奇譎,雖外敦厚而內不情。隋大業末,李密略地滎、汴,亮從之,未甚甄識。時軍中有謀叛去者,亮輒以告,密愛其誠,乃署驃騎將軍,隸李勣。勣以黎陽歸,亮頗佐佑之,擢鄭州刺史。會王世充取鄭,亮提孤軍不敢入,亡命共城山。俄檢校定州別駕。勣討劉黑闥,使亮守相州,賊方盛,棄城遁。



 房玄齡以亮沈果有謀,白秦王,引為車騎將軍。隱太子將作難,命亮統左右千人之洛陽,陰結山東豪傑以備變。齊王告亮反,高祖以屬吏詰訊,終無所言,乃得釋。王即位,除右衛將軍,封長平郡公。累遷御史大夫,進封鄅國公,食益州戶五百。歷豳夏汭鄜三州都督、相州長史,徙鄖國。召拜工部尚書。亮為政多伺察,發F〗縹隱微,示神明,抑強恤弱,所至有績。拜太子詹事,出為洛州都督。侯君集已誅,以刑部尚書參預朝政。



 時茂州俚童張仲文自稱天子,有司論斥乘輿有罪當死,攝刑部尚書韋挺奏:「童乃妖言,無死坐。」帝怒曰:「爾作威福於下,而歸虐朕耶!」挺失據趨出。亮為挺直之,帝曰:「公欲取剛正名乎?」亮不謝,帝寤曰:「寧屈我,以申公之請。」童免死。



 帝將伐高麗,亮頻諫,不納,因自請行,詔為平壤道行軍大總管。引兵自東萊浮海,襲破沙卑城,進至建安,營壁未立,賊奄至,亮不知所為,踞胡床直視無所言,眾謂其勇,得自安。於是副將張金樹鼓於軍,士奮擊,因破賊。及從帝還,至並州,乃得罪。



 初,亮棄故妻,更娶李氏。李妒悍,私通歌兒,養為子,名慎幾。亮子顗數諫止,亮不納。李好左道,交通巫覡,橈政事。亮為相州,假子公孫節以讖有「弓長之主當別都」,亮自以相舊都,「弓長」其姓,陰有怪謀。術家程公穎者,亮素與厚,陰謂曰:「君前言陛下真天下主,何其神邪!」公穎內曉,即稱亮臥若龍,當大貴。亮曰:「國家殆必亂,吾臂龍鱗奮矣,慎幾且大貴。」公孫常者,節兄也,亮謂曰:「吾有妾,相者云必為諸王姬。」常曰:「我兄子大品言,有神告公名在讖書。」亮悅。會陜人常德發其謀,並言亮養假子五百。帝使馬周案之,亮讕辭曰:「囚等畏死,見誣耳。」因自陳佐命舊臣。帝曰:「亮養子五百將何為?正欲反耳。」詔百官議,皆言亮當誅。帝遣長孫無忌、房玄齡就獄謂曰:「法者,天下平,與公共為之。公不自修,乃至此,將奈何?」於是斬西市,籍其家。



 薛萬均,本燉煌人,後徙京兆咸陽。父世雄,大業末為涿郡太守,萬均與弟萬徹因客幽州,以材武為羅藝所厚善。與藝歸款,高祖授萬均上柱國、永安郡公。



 竇建德帥眾十萬寇範陽,藝迎拒之。萬均曰:「眾寡不敵,宜以計勝。」即教藝羸兵阻水以誘之,萬均自以精騎百匿城左。建德師度水,邀半度擊之,大敗其眾。明年,建德以二十萬騎來攻,兵已緣堞,萬均與萬徹率死士百人出地道,掩擊其背,眾驚潰去。秦王平劉黑闥,引萬均為右二護軍,北門長上。



 柴紹之討梁師都也,以萬均為副,萬徹亦從。距朔方數十里,突厥兵驟至,王師卻,萬均兄弟橫擊之,斬其驍將,虜陣歡,乘之,俘殺相藉。突厥走,遂圍師都。諸將以城險未可下,萬均曰:「城中氣死,鼓不能聲,破亡兆也。」既而賊果斬師都降。拜左屯衛將軍。



 俄為沃沮道行軍副總管,從李靖討吐谷渾。軍次青海,萬均、萬徹各以百騎行前,卒與虜遇,萬均單騎馳突,無敢當者。還語諸將曰:「賊易與。」復馳進擊,斬數千級,勇蓋三軍。追奔至積石山,大風折旗,萬均曰:「虜且來!」乃勒兵。俄而虜至,萬均直前斬其將,眾遂潰,追至圖倫磧乃還,與靖會青海。璽書勉勞,遷本衛大將軍。又副侯君集擊高昌,曲智盛堅守未下,萬均麾軍進,智盛懼,乃降。進潞國公。



 會有訴萬均與高昌女子亂,太宗欲窮治。魏徵曰:君使臣以禮,若所訴實,罪且輕,虛則所失重矣。」詔勿治。後帝幸芙蓉園,坐清宮不謹下獄,憂憤卒。帝驚悼,為舉哀,詔陪葬昭陵。後嘗賜群臣膜皮,及萬徹而誤呼萬均,愴然曰:「萬均朕勛舊,忽口其名,豈死者有知,冀此賜乎?」因命取焚之,舉坐感嘆。弟萬徹、萬淑、萬備。



 萬徹與萬均歸高祖,授車騎將軍、武安縣公,事隱太子。太子誅,萬徹督宮兵戰武門,噪而趨秦府,眾失色;乃示以太子首,然後去,與數十騎亡之南山。秦王數使貸諭,乃出謝。王以其忠於所事,不之罪也。從李靖討突厥頡利可汗,以功授統軍,進爵郡公。歷右衛將軍、蒲州刺史。副李勣擊薛延陀,與虜戰磧南,率數百騎為先鋒,繞擊陣後。虜顧見,遂潰,斬首三千級,獲馬萬五千,封一子為縣侯。改左衛將軍,尚丹陽公主,加駙馬都尉。遷代州都督、右武衛大將軍。太宗嘗曰:「當今名將,唯李勣、江夏王道宗、萬徹而已。勣、道宗雖不能大勝,亦未嘗大敗;至萬徹,非大勝即大敗矣。」貞觀二十二年,以青丘道行軍總管帥師三萬伐高麗,次鴨淥水,以奇兵襲大行城,與高麗步騎萬餘戰,斬虜將所夫孫。虜皆震恐,遂傅泊汋城。虜眾三萬來援,擊走之,拔其城。萬徹在軍中,任氣不能下人,或有上書言狀者,帝愛其功,直加讓勖而已,即為焚書。副將裴行方亦言其怨望。李勣曰:「萬徹位大將軍,親主婿,而內懷不平,罪當誅。」因詔除籍徙邊,會赦,還。高宗永徽二年,授寧州刺史。入朝,與房遺愛暱甚,因曰:「我雖病足,坐置京師,諸輩猶不敢動。」遺愛曰:「若國有變,當與公共輔荊王。」謀洩下獄,誅。臨刑曰:「萬徹大健兒,留為國效死,安得坐遺愛殺之!」遂解衣顧監刑者曰:「亟斬我!」斬之不殊,叱曰:「胡不力!」三斬乃絕。



 萬淑亦以戰功顯。歷右領軍將軍、梁郡公、暢武道行軍總管。



 萬備有至行,居母喪,廬墓前,太宗詔表異其門。以尚輦奉御從伐高麗。李勣圍白巖,虜遣兵萬餘來援,將軍契苾何力以八百騎苦戰,中槊創甚,為賊所窘,萬備單馬進救,何力獲免。仕至左衛將軍。



 在武德、貞觀時,又有盛彥師、盧祖尚、劉世讓、劉蘭、李君羨等,頗以功力顯,而皆不終,附於左。



 盛彥師者,宋州虞城人。少任俠。隋大業末,為澄城長。高祖兵至汾陰,彥師率賓客上謁,授行軍總管,從平京師,與史萬寶鎮宜陽。李密叛,謀出山南,萬寶懼,謂彥師曰:「密,驍賊也,以王伯當輔之,挾思東歸之士,非計出萬全不為也,殆不可當。」彥師笑曰:「請以數千兵為公梟其首。」萬寶問計,答曰:「兵詭道也,難豫言。」即引眾逾洛水,入熊耳山,命士持滿夾道,伏短兵溪谷間,令曰:「賊半度乃擊。」所部皆笑曰:「賊趨洛州,何為備此?」彥師曰:「密聲言入洛,其實走襄城就張善相,我據其要,必禽之。」密果至,彥師橫擊,首尾不相救,遂斬密及伯當。以功封葛國公,授武衛將軍,鎮熊州。



 討王世充也,彥師與萬寶軍伊闕,絕山南路。世充平,為宋州總管。始,彥師入關,世充以陳寶遇為宋州刺史,待其家不以禮。及是,彥師因事殺之,又殺平生所惡數十家,州人震駭,皆重足立。



 徐圓朗反,詔為安撫大使,戰敗,為賊所執。圓朗待之厚,命作書招其弟,使舉虞城叛。彥師為書曰:「吾奉使無狀,為賊禽,誓死報國。若宜善侍毋,勿以我為念。」圓朗笑曰:「將軍,壯士也。」置之。武德六年,圓朗平,彥師得還。高祖以罪誅之。



 盧祖尚,字季良,光州樂安人。家饒財,好施,以俠聞。隋大業末,募壯士捕盜,時年十九,善御眾,所向有功,盜畏,不入境。宇文化及之亂,據州稱刺史,歃血誓眾,士皆感泣。越王侗立,遣使歸地,因署本州總管,封沈國公。



 王世充僭位,以州歸高祖,授刺史,封弋陽郡公。從趙郡王孝恭討輔公祏,為前軍總管,下宣、歙,進擊賊帥馮惠亮、陳正通,破之。歷蔣州刺史、壽州都督、瀛州刺史,有能名。



 貞觀二年,交州都督以賄敗,太宗方擇人任之,咸以祖尚才備文武,可用也。召見內殿,謂曰:「交州去朝廷遠,前都督不稱職,公為我行,無以道遠辭也。」祖尚頓首奉詔,既而托疾自解,帝遣杜如晦等諭意曰:「匹夫不負然諾,公既許朕矣,豈得悔?三年當召,不食吾言。」對曰:「嶺南瘴癘,而臣不能飲,當無還理。」遂固辭。帝怒曰:「我使人不從,何以為天下!」命斬朝堂。既而悔之,詔復其官。



 劉世讓,字元欽,京兆醴泉人。仕隋為徵仕郎。高祖入長安,以湋川歸,授通議大夫。時唐弼餘黨寇扶風,世讓自請安輯,許之,得其眾數千,因授安定道行軍總管,率兵二萬拒薛舉,戰不勝,與弟寶皆沒於賊。舉令至城下,紿說使降。世讓陽許之,至則告守者曰:「賊兵極於此矣,善自固!」舉重其節,不加害。秦王方屯高墌,世讓密遣寶間走王,言賊虛實。高祖悅,賜其家帛千匹。舉平,授彭州刺史。俄領陜東道行軍總管,從永安王孝基討呂崇茂於夏縣,軍敗,為賊所囚。聞獨孤懷恩有逆謀,唐儉語世讓曰:「懷恩謀行,則國難未息,可亡歸,白發之。」世讓逃還,高祖方濟河幸懷恩營,驚曰:「世讓之來,天也!」因封為弘農郡公,賜田百畝、錢百萬。母喪免,起為檢校並州總管。竇建德之援王世充也,世讓率萬騎出黃沙嶺,襲洛州。會突厥入寇,又詔以兵屯雁門,世讓馳騎八百赴之,而可汗軍大至,乃保武州。可汗與高開道、苑君璋合眾攻之,城數壞;輒立柵完拒。鄭元先使可汗,可汗使來說,世讓叱曰:「丈夫奈何為夷狄作說客邪?」久之,虜引去。元還,具道其忠,賜良馬、金帶。襄邑王神符鎮並州,世讓數以氣凌之,坐是削籍徙康州。未幾,召授廣州總管。帝問以備邊策,答曰:「突厥數南寇者,恃有馬邑為地耳。如使勇將屯崞城,厚儲金帛以招降者,數出奇兵略城下,踐禾稼,不逾歲,馬邑可圖也。」帝曰:「非公無可任者。」乃使馳驛經略,於是世讓至馬邑。高滿政以地來降,突厥患之,縱反間,云:「世讓與可汗為亂。」帝不之察,因誅之,籍其家。貞觀初,突厥降者言世讓無逆謀,乃原其妻子。



 劉蘭字文鬱,青州北海人。仕隋鄱陽郡書佐。涉圖史,能言成敗事。性陰狡,以天下將亂,見北海完富,潛介賊破其鄉,取子女玉帛。淮安王神通安撫山東,率宗黨歸順。貞觀初,梁師都未平,蘭上書陳方略,太宗以為夏州都督府司馬。師都以突厥兵頓城下,蘭僕旗息鼓,賊疑不敢迫,夜引去。蘭追擊,破之,遂進軍夏州。師都平,遷豐州刺史,召為右領軍衛將軍。十一年,為夏州都督長史。時突厥攜貳,鬱射設阿史那摸末率屬帳居河南,蘭縱反間離之,頡利果疑。摸末懼,來降,頡利急追,蘭逆拒,卻其眾。封平原郡公,俄檢校代州都督。初,長社許絢解讖記,謂蘭曰:「天下有長年者,咸言劉將軍當為天下主。」蘭子昭又曰:「讖言海北出天子,吾家北海也。」會鄠縣尉游文芝以罪系獄當死,因發其謀,蘭及黨與皆伏誅。



 李君羨,洛州武安人。初事李密,後為王世充驃騎。惡世充為人,率其屬歸高祖,授上輕車都尉。秦王引置左右,從破宋金剛於介休,加驃騎將軍,賜以宮人、繒帛。從討王世充,為馬軍副總管。世充子玄應自武牢轉糧入洛,君羨俘其軍,玄應走。從破竇建德、劉黑闥,所向必先登摧其鋒,累授左衛府中郎將。突厥至渭橋,君羨與尉遲敬德擊破之。太宗曰:「使皆如君羨者,虜何足憂!」改左武候中郎將,封武連縣公,北門長上。在仗讀書不休,帝嘉勞。歷蘭州都督、左監門衛將軍。先是,貞觀初,太白數晝見,太史占曰:「女主昌。」又謠言「當有女武王者」。會內宴,為酒令,各言小字,君羨自陳曰「五娘子」。帝愕然,因笑曰:「何物女子,乃此健邪!」又君羨官邑屬縣皆「武」也,忌之。未幾,出為華州刺史。會御史劾奏君羨與狂人為妖言,謀不軌,下詔誅之。天授中,家屬詣闕訴冤,武后亦欲自詫,詔復其官爵,以禮改葬。



 贊曰:侯君集位將相私謁太子,張亮養子五百人,薛萬徹與狂豎謀,皆死有餘責,又何咎哉?以太宗之明德,蔽於謠讖,濫君羨之誅,徒使孽後引以自神,顧不哀哉!



\end{pinyinscope}