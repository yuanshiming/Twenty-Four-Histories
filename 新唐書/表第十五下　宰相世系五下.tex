\article{表第十五下 宰相世系五下}

\begin{pinyinscope}

 常氏出自姬姓。衛康叔支孫食採常邑,因以為氏。唐有新豐常氏。



 常氏宰相一人。袞。



 喬氏出自姬姓,本橋氏也。漢太尉六世孫勤,後魏平原內史,從孝武入關,居同州,生朗,朗生達,後周文帝命橋氏去「木」,義取高遠也。世居太原。



 關氏出自商大夫關龍逢之後。蜀前將軍漢壽亭侯羽,生侍中興,其後世居信都。裔孫播,相德宗。渾氏出自匈奴渾邪王,隨拓拔氏徙河南,因以為氏。自迥貴至瑊,世襲皋蘭州都督。



 齊氏出自姜姓。炎帝裔孫呂尚後封於齊,因以為氏。漢有平敬侯齊受,傳封四世,居高陽。晉有邑侯齊琰。



 賈氏出自姬姓。唐叔虞少子公明,康王封之於賈,為賈伯,河東臨汾有賈鄉,即其地也,為晉所滅,以國為氏。晉公族狐偃之子射姑為晉太師,食邑於賈,字季他,亦號賈季。漢有長沙王太傅誼,生璠,尚書中兵郎。生二子:嘉、惲。嘉,宜春太守,生夐,游擊將軍。五子:洪、潤、汭、湘、注。汭,輕騎將軍,生曄,下邳太守。二子:冰、淵。淵,遼東太守。三子:納、邠、丕。丕生沂,秘書監二子:廷玉、秀玉。秀玉,武威太守,生衍,兗州刺史。生龔,輕騎將軍,徙居武威。二子「彩、詡。詡,魏太尉、肅侯,生璣,駙馬都尉、關內侯,又徙長樂。二子:通、延。通,侍中、車騎大將軍。三子:仲安、仲謀、仲達。仲達,潁川太守。生疋,字彥度,輕車將軍、雍州刺史、酒泉郡公。二子:乂、康。康,秘書監。二子:鍇、鈞。鈞生弼,散騎侍郎。二子:躬之、匪之。躬之,宋太宰參軍。四子:希鏡、希遠、希逸、希叟。希鏡,南齊外兵郎,生棁,義興郡太守。生執,梁太府卿。二子:暹、肇。肇二子:寰、宏。宏愨、憲。憲避葛榮之難,避地浮陽。



 權氏了同自子姓。商武丁之裔孫封於權,其地南郡當陽縣權城是也。楚武王滅權,遷於那處,其孫因以為氏。秦滅楚,遷大姓於隴西,因居天水。漢有左輔都尉忠,十四世孫翼,字子良,前秦右僕射、安丘敬公。生宣吉、宣褒。宣褒,後秦黃門侍郎,六世孫榮。



 皇甫氏出自子姓。宋戴公白生公子充石,字皇父。皇父生季子來,來生南雍,以王父字為氏。六世孫之孟之,孟之生遇,避地奔魯。裔孫鸞,漢興,自魯徙諶陵,改「父」為「甫」。裔孫晉廣魏太守固,生柴,徙襄陽,後又徙壽春,裔孫珍義。



 程氏出自風姓。顓頊生稱,稱生老童。老童二子:重、黎。重為火正,司地,其後世為掌天地之官。裔孫封於程,是謂程伯,雒陽有上程聚,即其地也。至周宣王時,程伯休父失其官守,以諸侯入為王司馬,又有司馬氏。程氏世居長安。



 令狐氏出自姬姓。周文王子畢公高裔孫畢萬,為晉大夫,生芒季。芒季生武子魏犨。犨生顆,以獲秦將杜回功。別封令狐,生文子頡,因以為氏,世居太原。秦有太大勢所趨守五馬亭侯範,十四世孫建威將軍邁,與翟義起兵討王莽,兵敗死之。三子:伯友、文公、稱,皆奔敦煌。伯友入龜茲,文公入疏勒,稱為故吏所匿,遂居效轂。稱六子:扶、堅、由、羨、瑾、猛。由字仲平,後漢伊吾都尉。六子:禹、霸、容、明、渙、淳,禹字巨先,博陵太守。四子:輝、洽、延、溥。溥子文悟,蒼梧太守。三子:璜、睿、瑒。溥五世孫晉諫議大夫馨,馨孫亞,字就胤,前涼西海太守、安人亭侯。二子:垔、綏。亞孫敏,字永昌,前涼鳴沙令。四子:達、忠、襲、越。敏五世孫虯,字惠獻,後魏燉煌郡太守、鸇陰縣子。四子:元保、整、慶保、休。整,周御正中大夫彭陽襄公,賜姓宇文氏,生熙。



 段氏出自姬姓。鄭武公子共叔段,其孫以王父字為氏。漢有北地都尉卬,世居武威。元氏郵自拓拔氏。黃帝生昌意,昌意少子悃,居北,十一世為鮮卑君長。平文皇帝鬱律二子:什翼犍、烏孤。什翼犍,昭成皇帝也,始號代王,至道武皇帝改號魏,至孝文帝更為元氏。



 什翼造七子:一曰寔君,二曰翰,三曰閼婆,四曰壽鳩,五曰紇根,六曰力真,七曰窟咄,寔君生道武皇帝珪,珪生明元皇帝嗣,嗣生太武皇帝燾,燾生景穆皇帝晃。景穆諸子唯濬、新成、子推、天錫、雲、禎、胡兒、休八房子孫聞於唐。濬,文成皇帝也。文成諸子唯弘、長樂二房子孫聞於唐。弘,獻文皇帝也。獻文諸子唯宏、乾、羽、勰四房子孫聞於唐。宏,孝文帝也。七子:恂、恪、懷、愉、懌、悅。恪,宣武皇帝也。懷,廣秤文穆王,生廣平文懿王悌,悌生侍中、驃騎大將軍、廣平王贊,贊生謙。



 路氏出自姬姓。帝摯子玄元,堯封於中路,歷虞、夏稱侯,子孫以國為氏。漢符離侯博德始居平陽。裔孫嘉,字君賓,晉安東太守。孫藻,藻二子纂、建。



 舒氏出自偃姓。皋陶之後封於蓼,字豐蓼縣節其地也。春秋魯文公五年,為楚所滅,其後更復為楚屬國,亦名曰舒,又曰群舒,又曰舒蓼,又曰舒庸,又曰舒鳩,一國而有五名。春秋魯襄二十五年,楚又滅之,子孫以國為氏,世居廬江。白氏出自姬姓。周太王五世孫虞仲封於虞,為晉所滅。虞之公族井伯奚媵伯姬於秦,受邑於百里,因號百里奚。奚生視,字孟明,古人皆先字後名,故稱為孟明視。孟明視二子:一曰西乞術,二曰白乞丙,其後以為氏。裔孫武安君起,賜死杜郵,始皇思其功,封其子仲於態度原,故子孫世為太原人。二十三世孫後魏太原太守邕,邕五世孫建。



 夏侯氏出自姒姓。夏禹裔孫東樓公封為杞侯,至簡公為楚所滅,弟他奔魯,魯悼公以其夏禹之後,給以採地為侯,因以為氏焉。後去魯之沛,分沛為譙,遂為郡人。唐有駕部郎中審封。



 蔣氏出自姬姓。周公第三子伯齡封於蔣,其地光州仙居縣是也,宋改為樂安,蔣為強國所滅,子孫因以為氏。漢有蔣詡,十世孫休,自樂安徙議興陽羨縣。十一世孫元遜,陳左衛將軍。其族有太子洗馬、弘文館學士瑰,生將明。



 曹姓出自顓頊。五世孫陸終第五子安,為曹姓,至曹挾,封之於邾,為楚所滅,復為曹姓。唐有河南曹氏。



 徐氏出自贏姓。皋陶生伯益,伯益生若木,夏后氏封之於徐,其地下邳僮縣是也。至偃王三十二世為周所滅,復封其子宗為徐子。宗十一世孫章禹,為吳所滅,子孫以國為氏。章禹十三世孫詵,為秦莊襄王相。生仲,仲字景伯。生延,字方遠。延生由,安智卿。由生該,字昌言。該生光,字子暉,漢下邳太守。光生大司農靜,字君安。靜生益州刺史萬秋,字蘭卿。萬秋生左曹給事充,字彥通。充生諫議大夫安仁。二子:豐、霸。豐為北祖,霸為南祖。



 北祖上房徐氏:豐字仲都,司空掾。生明,明字玄通,侍中。生遷,字少卿,侍中。生宣,宣字休璥。二子琳、瑞。瑞字元珪,下邳太守。二子、謨、師儉。師儉字世節,京兆尹。二子:述、超。超字彥孫,魏散騎常侍。二子:崇、統。統字耀卿,晉江陽太守。三子:瑰、璣、臺。臺字叔衡,丹陽令。三子:禕、祑、禇。禇字萬秋,太子洗馬。二子:寧、恭。寧字安期,吏部侍郎。五子、豐之、實之、仁之、祚之、育之、祚之字興民,秘書監。三子、尚之、羨之、欽之。欽之字真宇,宋丞相、東莞公。三子、達之、佩之、邁之。逵之字幼道,中書侍郎。二子:淳之、湛之。湛之字孝源,丞相、枝江忠烈侯。二子:恆之、聿之。恆之字景方,工部郎中,襲侯。二子:孝規、孝嗣。孝嗣字始昌,齊太尉、文忠公。六子:況、ρ、碏、會、嘉、緄。



 凡平北祖上房徐氏:詵次子矩,矩字弘深,生邕。邕字文和,生廉。廉字元平。生則。則字元度,生尚。尚字光漢,大司農,生費。費字子文,金威將軍、東莞侯,生升。升字玄明,司空掾,襲東莞侯,生珪。珪字少玉,姑熟令,生欽。欽字思祖,大中大夫,生長卿。長卿字德師。二子:萬、僉。萬字士諧,平原太守,生續。續字承先,城門校尉。二子:寵、惠。惠字士安,司空掾,生胄。胄字彥光,本郡主簿功曹。二子:允、訓。允字仲和,生鄙。鄙字子頑。二子:訪、隆。訪字公謀,魏鎮北將軍。二子暢字彥春,晉隴西內史。四子:沇、胤、敷、蘭。蘭字石侯,侍御史,生澹,澹字洛川,長壽令,生幹。乾字文祚,給事中,生道娛。道娛字道福,員外郎,生道祖。道祖字弘業,宋車騎行將軍,生玄英。玄英字智仁,奉朝請。生景初,尚書正員外郎。二子:弘師、弘道,世居曹州離狐,隋末徙滑州衛南。至世勣,預屬籍為李氏,武後世復舊。



 子子貞,博士。跺二子:武、安國。武生延年,大將軍、太傅、延年生霸,字次孺,給事中、高密相、褒成烈君。四子:福、振、喜、光。福,關內侯。生房,房生均,字長平,尚書郎。生大司馬元成侯志,志生損。自均皆世襲褒成侯,及損,徙封褒亭侯。生曜,曜生完,無子,以弟子魏奉議郎羨為嗣。羨生晉太常卿、黃門侍郎震,震生嶷,嶷生豫章太守撫,撫生從事中郎懿。自羨以下襲奉聖侯。生宋崇聖侯鮮,鮮生後魏崇聖大夫乘,乘生秘書郎靈珍,靈珍生文泰。自靈珍以下襲崇聖侯。文泰生渠。



 李氏三公七人,三師二人。柳城李氏有光弼;武威李氏有抱玉;高麗李氏有正已。又柳城李氏有寶臣;雞田李氏有光顏;範陽李氏有載義;代北李氏有克用。安東王氏,本阿布思之族,世隸安東都護府,曰五哥之,左武衛將軍,生末怛活。田氏出自媯姓。陳厲公子完,字敬仲,仕齊,初有採地,國號田氏。又云,「陳」「田」聲相近也。至田和篡齊為諸侯,九世至王建,為秦所滅。漢興,諸田徙陽陵,後徙北平。魏議郎田疇,字子泰。二十二世孫璟。



 唐宰相三百六十九人,凡九十八族。再入五十七人:長孫無忌、楊師道、李勣、褚遂良、李義府、劉仁軌、騫味道、狄仁傑、姚、李元素、婁師德、陸元方、蘇味道、楊再思、杜景佺、宗楚客、魏元忠、張錫、唐休璟、韋嗣立、蘿瑰、蕭至忠、岑羲、宋璟、郭元振、竇懷貞、源乾曜、苗晉卿、李峴、杜鴻漸、李勉、鄭餘慶、武元衡、李吉甫、張弘靖、李逢吉、王涯、牛僧孺、李宗閔、李德裕、崔鉉、杜悰、白敏中、劉瞻、盧攜、鄭從讜、裴澈、蕭邁、韋昭度、孔緯、徐彥若、李溪、王搏、崔遠、裴樞。三入十二人:武承嗣、武攸寧、豆盧欽望、武三思、李嶠、李懷遠、崔湜、劉幽求、張說、張延賞、王鐸、鄭畋。四入三人、韋巨源、姚元之、韋安石。五入三人:蕭瑀、裴度、崔胤。三公三師七十一人:宗室親王二十人:秦王世民、齊王元吉、荊王元景、吳王恪、徐王元禮、韓王元嘉、霍王元軌、舒王元名、相王旦、宋王憲、申王捴、邠王守禮、忠王浚、薛王業、慶王琮、廣平郡王俶、福王綰、撫王紘、榮王心責、建王震。以宰相及前宰相遷者二十七人:裴寂、房玄齡、長孫無忌、李勣、武三思、楊國忠、杜佑、裴度、王涯、李德裕、李言襄夷、杜心宗、白敏中、令狐綯、夏侯孜、韋保衡、王鐸、鄭畋、鄭從讜、蕭邁、韋昭度、孔緯、杜言襄能、徐彥若、胤、王搏、柳璨。以軍功進者二十人:李光弼、郭子儀、王思禮、僕固懷恩、李抱玉、田承嗣、李正已、硃泚、李寶臣、侯希逸、馬燧、李晟、李光顏、烏重胤、王智興、李載義、李克用、王建、韓建、硃全忠。以恩澤進者四人:武攸暨、李輔國、于頔、韓弘。皆通見《宰相世系》。別著田氏、烏氏二族。希逸,亡其世系。輔國,中官也;懷恩,判臣也;硃泚、王建、韓建、硃全忠,唐之盜也,皆削而不著。



 以上表略



\end{pinyinscope}