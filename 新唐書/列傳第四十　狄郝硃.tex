\article{列傳第四十 狄郝硃}

\begin{pinyinscope}

 狄仁傑,字懷英,並州太原人。為兒時,門人有被害者,吏就詰雅斯貝爾斯(KarlJaspers,1883—1969)德國哲學家、精,眾爭辨對,仁傑誦書不置,吏讓之,答曰:「黃卷中方與聖賢對,何暇偶俗吏語耶?」舉明經,調汴州參軍。為吏誣訴,黜陟使閻立本召訊,異其才,謝曰:「仲尼稱觀過知仁,君可謂滄海遺珠矣。」薦授並州法曹參軍。親在河陽,仁傑登太行山,反顧,見白雲孤飛,謂左右曰:「吾親舍其下。」瞻悵久之,雲移乃得去。同府參軍鄭崇質母老且疾,當使絕域。仁傑謂曰:「君可貽親萬里憂乎?」詣長史蘭仁基請代行。仁基咨美其誼,時方與司馬李孝廉不平,相敕曰:「吾等可少愧矣!」則相待如初,每曰:「狄公之賢,北斗以南,一人而已。」



 稍遷大理丞,歲中斷久獄萬七千人,時稱平恕。左威衛大將軍權善才、右監門中郎將範懷義坐誤斧昭陵柏,罪當免,高宗詔誅之。仁傑奏不應死,帝怒曰:「是使我為不孝子,必殺之。」仁傑曰:「漢有盜高廟玉環,文帝欲當之族,張釋之廷諍曰:『假令取長陵一抔土,何以加其法?』於是罪止棄市。陛下之法在象魏,固有差等。犯不至死而致之死,何哉?今誤伐一柏,殺二臣,後世謂陛下為何如主?」帝意解,遂免死。數日,授侍御史。左司郎中王本立怙寵自肆,仁傑劾奏其惡,有詔原之。仁傑曰:「朝廷借乏賢,如本立者不鮮。陛下惜有罪,虧成法,奈何?臣願先斥,為群臣戒。」本立抵罪。繇是朝廷肅然。使岐州,亡卒數百剽行人,道不通。官捕系盜黨窮訊,而餘曹紛紛不能制。仁傑曰:「是其計窮,且為患。」乃明開首原格,出系者,稟而縱之,使相曉,皆自縛歸。帝嘆其達權宜。



 遷度支郎中。帝幸汾陽宮,為知頓使。並州長史李沖玄以道出石女祠,俗言盛服過者,致風雷之變,更發卒數萬改馳道。仁傑曰:「天子之行,風伯清塵,雨師灑道,何石女避邪?」止其役。帝壯之,曰:「真丈夫哉!」出為寧州刺史,撫和戎落,得其歡心,郡人勒碑以頌。入拜冬官侍郎、持節江南巡撫使。吳、楚俗多淫祠,仁傑一禁止,凡毀千七百房,止留夏禹、吳太伯、季札、伍員四祠而已。



 轉文昌右丞,出豫州刺史。時越王兵敗,支黨餘二千人論死。仁傑釋其械,密疏曰:「臣欲有所陳,似為逆人申理;不言,且累陛下欽恤意。表成復毀,自不能定。然此皆非本惡,詿誤至此。」有詔悉謫戍邊。囚出寧州,父老迎勞曰:「狄使君活汝耶!」因相與哭碑下。囚齋三日乃去。至流所,亦為立碑。初,宰相張光輔討越王。軍中恃功,多暴索,仁傑拒之。光輔怒曰:「州將輕元帥邪?」仁傑曰:「亂河南者一越王,公董士三十萬以平亂,縱使暴橫,使無辜之人咸墜塗炭,是一越王死,百越王生也。且王師之至,民歸順以萬計,自縋而下,四面成蹊。奈何縱邀賞之人殺降以為功,冤痛徹天?如得上方斬馬劍加君頸,雖死不恨!」光輔還,奏仁傑不遜,左授復州刺史。徙洛州司馬。



 天授二年,以地官侍郎同鳳閣鸞臺平章事。武后謂曰:「卿在汝南有善政,然有譖卿者,欲知之乎?」謝曰:「陛下以為過,臣當改之;以為無過,臣之幸也。譖者乃不願知。」後嘆其長者。時太學生謁急,後亦報可。仁傑曰:「人君惟生殺柄不以假人,至簿書期會,宜責有司。尚書省決事,左、右丞不句杖,左、右丞相不判徒,況天子乎?學徒取告,丞、簿職耳,若為報可,則胄子數千,凡幾詔耶?為定令示之而已。」後納其言。



 會為來俊臣所構,捕送制獄。於時,訊反者一問即臣,聽減死。俊臣引仁傑置對,答曰:「有周革命,我乃唐臣,反固實。」俊臣乃挺系。其屬王德壽以情謂曰:「我意求少遷,公為我引楊執柔為黨,公且免死。」仁傑嘆曰;「皇天后土,使仁傑為此乎!」即以首觸柱,血流沫面。德壽懼而謝。守者浸弛,即丐筆書帛,置褚衣中,好謂吏曰;「方暑,請付家徹絮。」仁傑子光遠得書上變,後遣使案視。俊臣命仁傑冠帶見使者,私令德壽作謝死表,附使以聞。後乃召見仁傑,謂曰:「承反何耶?」對曰:「不承反,死笞掠矣。」示其表,曰:「無之。」後知代署,因免死。武承嗣屢請誅之,後曰:「命已行,不可返。」時同被誣者鳳閣侍郎任知古等七族悉得貸。御史霍獻可以首叩殿陛苦爭,欲必殺仁傑等,乃貶仁傑彭澤令,邑人為置生祠。



 萬歲通天中,契丹陷冀州,河北震動,擢仁傑為魏州刺史。前刺史懼賊至,驅民保城,修守具。仁傑至,曰:「賊在遠,何自疲民?萬一虜來,吾自辦之,何預若輩?」悉縱就田。虜聞,亦引去,民愛仰之,復為立祠。俄轉幽州都督,賜紫袍、龜帶,後自制金字十二於袍,以旌其忠。



 召拜鸞臺侍郎,復同鳳閣鸞臺平章事。時發兵戍疏勒四鎮,百姓怨苦。仁傑諫曰:



 天生四夷,皆在先王封域之外。東距滄海,西隔流沙,北橫大漠,南阻五嶺,天所以限中外也。自典籍所紀,聲教所暨,三代不能至者,國家既已兼之。詩人矜薄伐於太原,化行於江、漢,前代之遐裔,而我之域中,過夏、商遠矣。今乃用武荒外,邀功絕域,竭府庫之實,以爭磽確不毛之地,得其人不足以增賦,獲其土不可以耕織。茍求冠帶遠夷,不務固本安人,此秦皇、漢武之所行也。傳曰:「與覆車同軌者未嘗安。」此言雖小,可以喻大。



 臣伏見國家師旅歲出,調度之費狃以浸廣,右戍四鎮,左屯安東,杼軸空匱,轉輸不絕,行役既久,怨曠者多。上不是恤,則政不行;政不行,則害氣作;害氣作,則蟲螟生,水旱起矣。方今關東薦饑,蜀漢流亡,江、淮而南,賦斂不息。人不復本,則相率為盜,本根一搖,憂患非淺。所以然者,皆貪功方外,耗竭中國也。昔漢元帝納賈捐之之謀而罷珠崖,宣帝用魏相之策而棄車師田。貞觀中,克平九姓,冊拜李思摩為可汗,使統諸部,夷狄叛則伐,降則撫,得推亡固存之義,無遠戍勞人之役。今阿史那斛瑟羅,皆陰山貴種,代雄沙漠,若委之四鎮,以統諸蕃,建為可汗,遣御寇患,則國家有繼絕之美,無轉輸之苦。損四鎮,肥中國,罷安東,實遼西,省軍費於遠方,並甲兵於要塞,恆、代之鎮重,而邊州之備豐矣。



 且王者外寧,容有內危。陛下姑敕邊兵謹守備,以逸待勞,則戰士力倍;以主御客,則我得其便;堅壁清野,寇無所得。自然深入有顛躓之慮,淺入無虜獲之益。不數年,二虜不討而服矣。



 又請廢安東,復高姓為君長,省江南轉餉以息民,不見納。



 張易之嘗從容問自安計,仁傑曰:「惟勸迎廬陵王可以免禍。」會後欲以武三思為太子,以問宰相,眾莫敢對。仁傑曰:「臣觀天人未厭唐德。比匈奴犯邊,陛下使梁王三思募勇士於市,逾月不及千人。廬陵王代之,不浹日,輒五萬。今欲繼統,非廬陵王莫可。」後怒,罷議。久之,召謂曰:「朕數夢雙陸不勝,何也?」於是,仁傑與王方慶俱在,二人同辭對曰:「雙陸不勝,無子也。天其意者以儆陛下乎!且太子,天下本,本一搖,天下危矣。文皇帝身蹈鋒鏑,勤勞而有天下,傳之子孫。先帝寢疾,詔陛下監國。陛下掩神器而取之,十有餘年,又欲以三思為後。且姑侄與母子孰親?陛下立廬陵王,則千秋萬歲後常享宗廟;三思立,廟不祔姑。」後感悟,即日遣徐彥伯迎廬陵王於房州。王至,後匿王帳中,召見仁傑語廬陵事。仁傑敷請切至,涕下不能止。後乃使王出,曰:「還爾太子!」仁傑降拜頓首,曰:「太子歸,未有知者,人言紛紛,何所信?」後然之。更令太子舍龍門。具禮迎還,中外大悅。初,吉頊、李昭德數請還太子,而後意不回,唯仁傑每以母子天性為言,後雖忮忍,不能無感,故卒復唐嗣。



 尋拜納言,兼右肅政御史大夫。突厥入趙、定,殺掠甚眾,詔仁傑為河北道行軍元帥,假以便宜。突厥盡殺所得男女萬計,由五回道去,仁傑追不能逮。更拜河北安撫大使。時民多脅從於賊,賊已去,懼誅,逃匿。仁傑上疏曰:「議者以為虜入寇,始明人之逆順,或迫脅,或願從,或受偽官,或為招慰。誠以山東之人重氣,一往死不為悔。比緣軍興,調發煩重,傷破家產,剔屋賣田,人不為售。又官吏侵漁,州縣科役,督趣鞭笞,情危事迫,不循禮義,投跡犬羊,以圖賒死,此君子所愧,而小人之常。民猶水也,壅則為淵,疏則為川,通塞隨流,豈有常性。昔董卓之亂,神器播越,卓已誅禽,部曲無赦,故事窮變生,流毒京室。此由恩不溥洽,失在機先。今負罪之伍,潛竄山澤,赦之則出,不赦則狂。山東群盜,緣茲聚結。故臣以為邊鄙暫警不足憂,中土不寧可為慮也。夫持大國者不可以小治,事廣者不可以細分。人主所務,弗檢常法。願曲赦河北,一不問罪。」詔可。



 還,除內史。後幸三陽宮,王公皆從,獨賜仁傑第一區,眷禮卓異,時無輩者。是時李楷固、駱務整討契丹,克之,獻俘含樞殿,後大悅。二人者,本契丹李盡忠部將,盡忠入寇,楷固等數挫王師,後降,有司請論如法。仁傑稱其驍勇可任,若貸死,必感恩納節,可以責功。至是凱旋,後舉酒屬仁傑,賞其知人。授楷固左玉鈐衛大將軍、燕國公,賜姓武;務整右武威衛將軍。



 後將造浮屠大像,度費數百萬,官不能足,更詔天下僧日施一錢助之。仁傑諫曰:「工不役鬼,必在役人;物不天降,終由地出。不損百姓,且將何求?今邊垂未寧,宜寬征鎮之傜,省不急之務。就令顧作,以濟窮人,既失農時,是為棄本。且無官助,理不得成。既費官財,又竭人力,一方有難,何以救之?」後由是罷役。



 聖歷三年卒,年七十一。贈文昌右相,謚曰文惠。仁傑所薦進,若張柬之、桓彥範、敬暉、姚崇等,皆為中興名臣。始居母喪,有白鵲馴擾之祥。中宗即位,追贈司空。睿宗又封梁國公。子光嗣、景暉。



 光嗣,聖歷初,為司府丞。武后詔宰相各舉尚書郎一人,仁傑薦光嗣,由是拜地官員外郎,以稱職聞。後曰:「祁奚內舉,果得人。」歷淄、許、貝三州刺史。母喪,奪為太府少卿,固讓,睿宗嘉其誠,許之。累遷揚州長史,以罪貶歙州別駕,卒。



 景暉,官魏州司功參軍,貪暴為虐,民苦之,因共毀其父生祠,不復奉。至元和中,田弘正鎮魏博,始奏葺之,血食不絕。族孫兼謨。



 兼謨字汝諧,及進士第。闢襄陽使府,剛正有祖風。令狐楚執政,薦授左拾遺,數上書言事。歷刑部郎中、蘄鄧鄭三州刺史。歲旱饑,發粟賑濟,民人不流徙。改蘇州,以治最,擢給事中。左藏史盜度支縑帛,文宗以經赦詔勿治,兼謨封還詔書,帝問之,對曰:「典史犯贓,不可免。」帝曰:「朕已赦其長官,吏亦宜宥,與其失信,寧失罪人。」既而曰:「後或事有不可,勿以還詔為憚。」遷御史中丞。帝曰:「御史臺朝廷綱紀,一臺正,則朝廷治,朝廷正,則天下治。畏忌顧望,則職業廢矣。卿,梁公後,當嗣家聲,不可不慎。」兼謨頓首謝。江西觀察使吳士矩加給其軍,擅用上供錢數十萬。兼謨劾奏:「觀察使為陛下守土,宣國詔條,知臨戎賞士,州有定數,而與奪由己,貽弊一方,為諸道觖望,請付有司治罪。」士矩繇是貶蔡州別駕。歷兵部侍郎、河東節度使。還為尚書左丞。武宗子峴封益王,命兼謨為傅。俄領天平節度使,辭疾,以秘書監歸洛陽,遷東都留守,卒。



 郝處俊,安州安陸人。父相貴,因隋亂,與婦翁許紹據峽州,歸國,拜滁州刺史,封甑山縣公。處俊甫十歲而孤,故吏歸千縑賵之,已能讓不受。及長,好學,嗜《漢書》,崖略暗誦。貞觀中,第進士,解褐著作佐郎,襲父爵。兄弟友睦,事諸舅謹甚。再轉滕王友,恥為王府屬,棄官去。久之,召拜太子司議郎,累遷吏部侍郎。高麗叛,詔李勣為浿江道大總管,處俊副之。師入虜境,未陣,賊遽至,舉軍危駭。處俊方據胡床,體胖,安餐乾Я不顧,密畀料精銳擊之,虜卻,眾壯其謀。



 入拜東臺侍郎。時浮屠盧伽逸多治丹,曰:「可以續年。」高宗欲遂餌之,處俊諫曰:「脩短固有命,異方之劑,安得輕服哉?昔先帝詔浮屠那羅邇娑寐案其方書為秘劑,取靈■怪石,歷歲乃能就。先帝餌之,俄而大漸,上醫不知所為。群臣請顯戮其人,議者以為取笑夷狄,故法不得行。前鑒不遠,惟陛下深察。」帝納其言,第拜盧伽逸多為懷化大將軍,進處俊同東西臺三品。



 咸亨初,幸東都,皇太子監國,諸宰相皆留,而處俊獨從。帝嘗曰:「王者無外,何為守禦?而重門擊柝,庸待不虞邪?我嘗疑秦法為寬,荊軻匹夫耳,匕首竊發,群臣皆荷戟侍,莫敢拒,豈非習慢使然?」處俊對曰:「此乃法急耳。秦法,輒升殿者,夷三族。人皆懼族,安有敢拒邪?魏曹操著令曰;『京城有變,九卿各守其府。』後嚴才亂,與徒數十人攻左掖門,操登銅爵臺望之,無敢救者。時王脩為奉常,聞變,召車騎未至,領官屬步至宮門。操曰;『彼來者,必王脩乎!』此由脩察變識幾,故冒法赴難。向若拘常,則遂成禍矣。故王者設法不可急,亦不可慢。《詩》曰『不懈於位,人之攸塈』,仁也;『式遏寇虐,無俾作慝』,刑也。《書》曰『高明柔克,沈潛剛克』,中道也。」帝曰:「善。」



 轉中書侍郎,監脩國史。初,顯慶中,令狐德棻、劉胤之撰國史,其後許敬宗復加緒次。帝恨敬宗所紀失實,更命宰相刊正,且曰:「朕昔從幸未央宮,闢仗既過,有橫刀伏草中者,先帝斂轡卻,謂朕曰;『事發,當死者數十人,汝可命出之。』史臣惟敘此為實。」處俊曰:「先帝仁恩溥博,類非一。臣之弟處傑被擇供奉,時有三衛誤拂禦衣者,懼甚。先帝曰:『左右無御史,我不汝罪。』」帝曰:「此史臣應載。」處俊乃表左史李仁實欲刪整偽辭,會仁實死而止。



 上元初,帝觀酺翔鸞閣,時赤縣與太常音技分東西朋,帝詔雍王賢主東,周王顯主西,因以角勝,處俊曰:「禮所以示童子無誑者,恐其欺詐之心生也。二王春秋少,意操未定,乃公朋造黨使相誇,彼俳兒優子,言辭無度,爭負勝,相譏誚,非所以導仁義,示雍和也。」帝遽止,嘆曰:「處俊遠識,非眾臣所逮。」遷中書令,兼太子賓客,檢校兵部尚書。



 帝多疾,欲遜位武後,處俊諫曰:「天子治陽道,後治陰德,然則帝與後猶日之與月,陽之與陰,各有所主,不相奪也。若失其序,上謫見於天,下降災諸人。昔魏文帝著令,帝崩,不許皇后臨朝。今陛下奈何欲身傳位天後乎?天下者,高祖、太宗之天下,非陛下之天下,正應謹守宗廟,傳之子孫,不宜持國與人,以喪厥家。」中書侍郎李義琰曰:「處俊言可從,惟陛下不疑。」事遂沮。又兼太子左庶子,拜侍中,罷為太子少保。開耀元年卒,年七十五。贈開府儀同三司、荊州大都督。帝哀嘆其忠,舉哀光順門,祭以少牢,賻絹布八百段、米粟八百石,詔百官赴哭,官庀葬事。子北叟固辭,未聽。裴炎為白帝曰:「處俊阽死,諉臣曰;『生無益於國,死無煩費,凡詔賜,願一罷之。』」帝聞惻然,答其意,止賻物而已。



 處俊資約素,土木形骸,然臨事敢言,自秉政,在帝前議論諄諄,必傅經義,凡所規獻,得大臣體。武後雖忌之,以其操履無玷,不能害。與舅許圉師同里,俱宦達;鄉人田氏、彭氏以高貲顯。故江、淮間為語曰:「貴如郝、許,富如田、彭。」



 孫象賢,垂拱中,為太子通事舍人,後素銜處俊,故因事誅之。臨刑,極罵乃死,後怒,令離磔其尸,斫夷祖、父棺塚。自是訖後世,將刑人,必先以木丸窒口雲。



 硃敬則,字少連,亳州永城人。以孝義世被旌顯,一門六闕相望。敬則志尚恢博,好學,重節義然諾,善與人交,振其急難,不責報於人。與左史江融、左僕射魏元忠善。咸亨中,高宗聞其名,召見,異之,為中書令李敬玄所毀,故授洹水尉。久之,除右補闕。



 初,武后稱制,天下頗流言,遂開告密羅織之路,興大獄,誅將相大臣。至是,已革命,事益寧。敬則諫曰:



 臣聞李斯之相秦也,行申、商之法,重刑名之家,杜私門;張公室;棄無用之費,損不急之官;惜日愛功,亟戰疾耕。既庶而富,遂屠諸侯。此救弊之術也。故曰:「刻薄可施子進趨,變詐可陳於攻戰。」天下已平,故可易之以寬簡,潤之以淳和。秦乃不然,淫虐滋甚,往而不反,卒至土崩。此不知變之禍也。



 陸賈、叔孫通事漢祖,當滎陽、成皋間,糧餉窮,智勇困,未嘗敢開一說,效一奇,唯進豪猾貪暴之人。及區宇適定,乃陳《詩》、《書》,說禮、樂,開王道。高帝忿然曰:「吾以馬上得之,安事《詩》、《書》?」對曰:「馬上得之,可馬上治之乎?」帝默然。於是賈著《新語》,通定禮儀。此知變之善也。向若高帝斥二子,置《詩》、《書》,重攻戰,尊首級,則復道爭功,拔劍擊柱,晷漏之不保,何十二帝二百年乎?故曰:仁義者,聖人之蘧廬;禮者,先王之陳跡。祠祝畢,芻狗捐;淳精流,糟粕棄。仁義尚爾,況其輕乎?



 國家自文明以來,天地草昧,內則流言,外則構難。故不設鉤距,無以順人;不切刑罰,無以息暴。於是置神器,開告端,故能不出房闈,而天下晏然易主矣。臣聞急趨者無善跡,促柱者無和聲;拯溺不規行,療饑不鼎食。即向時秘策,今之芻狗也。願鑒秦、漢之失,考時事之宜,毀蘧廬,遺糟粕;下寬大之令,流曠蕩之澤,去萋斐之角牙,頓奸險之芒刃,塞羅織之妄源,掃朋黨之險跡,曠然使天下更始,豈不樂哉!



 後善其言。遷正諫大夫,兼修國史。乃請高史官選,以求名才。侍中韋安石嘗閱其稿史,嘆曰:「董狐何以加!世人不知史官權重宰相,宰相但能制生人,史官兼制生死,古之聖君賢臣所以畏懼者也。」時賦斂繁重,民多蕩析,後數召入禁中訪失得,進同鳳閣鸞臺平章事。張易之構魏元忠、張說,欲誅之,無敢言者。敬則獨奏曰:「元忠、說秉心忠一,而所坐無名,殺之失天下望。」乃得不死。



 以老疾還政事,俄改成均祭酒、冬官侍郎。易之等集名儒撰《三教珠英》,又繪武三思、李嶠,蘇味道、李迥秀、王紹宗等十八人像以為圖,欲引敬則,固辭不與,世潔其為人。出為鄭州刺史,遂致仕。侍御史冉祖雍誣奏與王同皎善,貶涪州刺史。既明其非罪,改廬州。代還,無淮南一物,所乘止一馬,子曹步從以歸。卒年七十五。



 敬則與三從昆弟居四十年,貲產無異。及執政,每以用人為先,細務不省也。嶺表蠻叛,以裴懷古有文武才,用為桂州都督,蠻服其威惠,相率降。薦魏知古為鳳閣舍人,張思敬為右史,皆稱職。初,二張權寵盛,敬則密謂敬暉曰:「公若假太子令,舉北軍誅易之兄弟,兩飛騎力耳。」暉卒用其策。始崔實、仲長統、王朗、曹冏論封建,指秦為失,敬則以為秦、漢世禮義陵遲,不可復用周制封諸侯,著論明之,儒者以為知言。



 睿宗嗣位,嘗曰:「神龍以來,忠於本朝者,李多祚、王同晈、韋月將、燕欽融並褒復矣,尚有遺者耶?」劉幽求曰:「硃敬則忠正義烈,天下所推,往為宗楚客、冉祖雍等所誣,謫守刺史。長安中,嘗語臣曰:『相王必受命,當悉心事之。』及韋氏乾紀,臣遂見危赴難。雖天誘其衷,亦敬則啟之。」於是追贈秘書監,謚曰元。



 敬則兄仁軌,字德容,隱居養親。常誨子弟曰:「終身讓路,不枉百步;終身讓畔,不失一段。」有赤烏、白鵲棲所居樹,按察使趙承恩表其異。及卒,郭山惲、員半千、魏知古共謚為孝友先生。



 贊曰:武后乘唐中衰,操殺生柄,劫制天下而攘神器。仁傑蒙恥奮忠,以權大謀,引張柬之等,卒復唐室,功蓋一時,人不及知。故唐呂溫頌之曰:「取日虞淵,洗光咸池。潛授五龍,夾之以飛。」世以為名言。方高宗舉天下將以禪後,處俊固爭,不使妻乘夫,陰反陽,至奸人銜怨,仇胔以逞。蓋所謂誼形於主耶。敬則一諫,而羅織之獄衰,時而後言者歟!



\end{pinyinscope}