\article{列傳第二 後妃下}

\begin{pinyinscope}

 張皇後章敬吳太后貞懿獨孤皇后睿真沈太后昭德王皇后韋賢妃莊憲王皇后懿安郭太后孝明鄭太后恭僖王太后貞獻蕭太后宣懿韋太后尚宮宋若昭郭貴妃王賢妃元昭晁太后惠安王太后郭淑妃恭憲王太后何皇后



 肅宗廢後庶人張氏,鄧州向城人,家徙新豐。祖母竇,昭成皇后女弟也。玄宗幼失昭成,母視姨,鞠愛篤備。帝即位,封鄧國夫人,親寵無比。五息子,曰去惑、去疑、去奢、去逸、去盈,皆顯官。去盈尚常芬公主。去逸生後。



 肅宗為忠王時,納韋元珪女為孺人。既建太子,以孺人為妃,後為良娣。妃兄堅為李林甫構死,太子懼,請與妃絕,毀服幽禁中。安祿山反,陷於賊,至德中薨。



 始,妃既絕,良娣得專侍太子,慧中而辯,能迎意傅合。玄宗西幸,娣與太子從,度渭,民鄣道乞留復長安,太子不聽。中人李輔國密啟,娣又贊其謀,遂定計北趣靈武。時軍衛單寡,夕次,娣必寢前,太子曰:「暮夜可虞,且捍賊非婦人事,宜少戒。」對曰:「方多事,若倉卒,妾自當之,殿下可徐為計。」駐靈武,產子三日,起縫戰士衣,太子敕止,對曰:「今豈自養時邪?」乾元初,冊拜淑妃,贈其父尚書左僕射,姊妹皆封號,弟清、潛尚大寧、延和二郡主。遂立為皇后,詔內外命婦悉朝光順門。



 後能牢寵,稍稍豫政事,與李輔國相助,多以私謁橈權。親蠶苑中,群命婦相禮,儀物甚盛。二年,群臣上帝尊號,後亦諷群臣尊己號「翊聖」,帝問李揆,揆爭不可。會月蝕,帝以咎在後宮,乃止。又與輔國謀徙上皇西內。端午日,帝召見山人李唐,帝方擁幼女,顧唐曰:「我念之,無怪也。」唐曰:「太上皇今日亦當念陛下。」帝泫然涕下,而內制於後,卒不敢謁西宮。帝不豫,後自箴血寫佛書以示誠。



 初,建寧王倓數短後於帝,上皇在蜀,以七寶鞍賜后,而李泌請分以賞戰士,倓助泌請,故後怨,卒被譖死。繇是太子深畏,事後謹甚。後猶欲危之,然以子〓早世而侗幼,故太子得無患。寶應元年,帝大漸,後與內官硃輝光等謀立越王系,而李輔國、程元振以兵衛太子,幽後別殿。代宗已立,群臣白帝請廢為庶人,殺之。清、潛與舅竇履信皆流放,支黨伏誅。



 肅宗章敬皇后吳氏,濮州濮陽人。父令珪,以郫丞坐事死,故後幼入掖廷。



 肅宗在東宮,宰相李林甫陰構不測,太子內憂,鬢發班禿。後入謁,玄宗見不悅,因幸其宮,顧廷宇不汛掃,樂器塵蠹,左右無嬪侍,帝愀然謂高力士曰:「兒居處乃爾,將軍叵使我知乎?」詔選京兆良家子五人虞侍太子,力士曰;「京兆料擇,人得以藉口,不如取掖廷衣冠子,可乎?」詔可。得三人,而後在中,因蒙幸。忽寢厭不寤,太子問之,辭曰:「夢神降我,介而劍,決我脅以入,殆不能堪。」燭至,其文尚隱然。生代宗,為嫡皇孫。生之三日,帝臨澡之。孫體攣弱,負姆嫌陋,更取他宮兒以進,帝視之不樂,姆叩頭言非是。帝曰:「非爾所知,趣取兒來!」於是見嫡孫,帝大喜,向日視之,曰:「福過其父。」帝還,盡留內樂宴具,顧力士曰:「可與太子飲,一日見三天子,樂哉!」



 後性謙柔,太子禮之甚渥,年十八薨。代宗即位,群臣請以後祔肅宗廟,乃追尊為皇后,上謚,合葬建陵。啟故窆,貌澤若生,衣皆赭色,見者嘆異,謂有聖子之符云。



 代宗貞懿皇后獨孤氏,失其何所人。父穎,左威衛錄事參軍。



 天寶中,帝為廣平王,時貴妃楊氏外家貴冠戚里,秘書少監崔峋妻韓國夫人以其女女皇孫為妃。妃生子偲,所謂召王者。妃倚母家,頗騎媢。諸楊誅,禮浸薄,及薨,後以姝艷進,居常專夜。王即位,冊貴妃,生韓王回、華陽公主。



 大歷十年薨,追號為皇后,上謚。帝悼思不已,故殯內殿,累年不外葬。後三年,始詔於都左治陵,欲朝夕望見之。補闕姚南仲諫而止,乃葬莊陵。詔宰相常袞為哀冊,帝於後厚,故送終華廣,務稱其情,袞極道淒婉,以中帝意。又詔群臣為挽辭,帝擇其尤悲者令歌之。



 初,後愛遇第一,官其宗叔卓少府監,兄良佐太子中允。



 代宗睿真皇後沈氏,吳興人。開元末,以良家子入東宮,太子以賜廣平王,實生德宗。



 天寶亂,賊囚後東都掖廷。王入洛,復留宮中。時方北討,未及歸長安,而河南為史思明所沒,遂失後所在。代宗立,以德宗為皇太子,詔訪後在亡,不能得。



 德宗即位,乃先下詔贈後曾祖士衡太保,祖介福太傅,父易直太師,弟易良司空,易直子震太尉。一日封拜百二十七人,詔制皆錦翠池飾,以廄馬負載賜其家。易良妻崔入謁,帝易服,召王、韋美人出拜,詔崔勿答。



 建中元年,乃具冊前上皇太后尊號,帝供張含元殿,具袞冕,出自左序,立東方,群臣在位,帝再拜奉冊,欷歔感咽,左右皆泣。於是中書舍人高參上議:「漢文帝即位,遣薄昭迎太后於代。今宜用漢故事,令有司擇日分遣諸沈行州縣物色咨訪,以述宣皇帝孝思意,冀上天降休,靈命允答。須審知皇太后行在,然後遣大臣備法駕奉迎。」帝乃以睦王述為奉迎使,工部尚書喬琳副之,昇平公主侍起居,使者分行天下。



 故中官高力士女頗能言禁中事,與女官李真一嘗從後游。李見高,疑問之,含糊不堅,而年狀差似後。又後嘗削脯哺帝,傷左指,高亦嘗剖瓜傷指。是時宮中無識後者。於是迎還上陽宮,馳以聞。帝喜,群臣皆賀。力士子知非是,具言其情,詔貸之。帝謂左右:「吾寧受百罔,冀一得真。」於是自謂太后者數矣,及索驗左,皆辭窮,終帝世無聞焉。貞元七年,詔贈外高祖琳為司徒,封徐國公,為立五廟,以琳為始祖,詔族子房為金吾將軍,奉其祀。



 憲宗即位,有司建言:「皇太后沈氏厭代二十有七年,大行皇帝至孝,哀思罔極,建中時,發明詔,遣使者奉迎,凡舟車所至罔不逮,歲推月遷,參訪理絕。請因大行皇帝啟殯,詔群臣為皇太后發哀肅章內殿,中人奉廞衣置幄坐,宮中朝夕上食,告天地宗廟,上太皇太后謚冊,作神主祔代宗廟,備法駕,奉褘衣,納於元陵祠至。」詔曰「可」。



 德宗昭德皇后王氏,本仕家,失其譜系。帝為魯王時納為嬪,生順宗,尤見寵禮。既即位,冊號淑妃,贈其父遇揚州大都督,子姓姻出悉得官。



 貞元三年,妃久疾,帝念之,遂立為皇后。冊禮方訖而後崩,群臣大臨三日,帝七日釋服。將葬,後母郕國鄭夫人請設奠,有詔祭物無用寓,欲祭聽之。於是宗室王、大臣李晟渾瑊等皆祭,自發塗日日奠,終發引乃止。葬靖陵,置令丞如它陵臺。立廟,奏《坤元之舞》。敕宰相張延賞、柳渾等制樂曲,帝嫌文不工;李紓上謚冊曰「大行皇后」,帝又謂不典。並詔翰林學士吳通玄改撰,冊曰「咨後王氏」。然議者謂岑文本所上文德皇后冊言「皇后長孫氏」為得禮。永貞元年,改祔崇陵。



 德宗賢妃韋氏,戚裏舊族也。祖濯,尚定安公主。初為良娣,德宗貞元四年,冊拜賢妃。宮壺事無不聽,而性敏淑,言動皆有繩矩,帝寵重之,後宮莫不師其行。帝崩,自表留奉崇陵園。元和四年薨。



 順宗莊憲皇后王氏,瑯邪人。祖難得,有功名於世。代宗時,後以良家選入宮,為才人。順宗在籓,帝以才人幼,故賜之,為王孺人,是生憲宗。王在東宮,冊為良娣。後性仁順,宮中化其德,莫不柔雍。順宗即位,疾已綿頓,後侍醫藥不少怠。將立後,會病棘而止。憲宗內禪,尊為太上皇后。元和元年,乃上尊號曰皇太后。



 後謹畏,深抑外家,無豪絲假貸,訓厲內職,有古后妃風。十一年崩,年五十四。遺令曰:「皇太后敬問具位。萬物之理,必歸於有極,未亡人嬰霜露疾,日以衰頓,幸終天年,得奉陵寢,志願獲矣,其何所哀。易月之典,古今所共。皇帝宜三日聽政,服二十七日釋。天下吏民,令到臨三日止。宮中非朝暮臨,無輒哭。無禁昏嫁、祠祀、飲食酒肉。已釋服,聽舉樂。侍醫無加罪。陪祔如舊制。」有司上謚,葬豐陵。



 憲宗懿女皇后郭氏,汾陽王子儀之孫。父曖,尚昇平公主,實生後。憲宗為廣陵王,娉以為妃。順宗以其家有大功烈,而母素貴,故禮之異諸婦,是生穆宗。元和元年,進冊貴妃。八年,群臣三請立為後,帝以歲子午忌,又是時后廷多嬖艷,恐後得尊位,鉗掣不得肆,故章報聞罷。



 穆宗嗣位,上尊號皇太后,贈曖太尉,母齊國大長公主,擢兄釗刑部尚書,鏦〗金吾大將軍。後移御興慶宮,凡朔望三朝,帝率百官詣宮門為壽。或歲時慶問燕饗,後宮戚里內外婦,車騎駢壅,環佩之聲滿宮。帝亦豪矜,朝夕供御,務華衍侈大稱後意。後嘗幸驪山,登覽裴回,詔景王督禁甲從,帝自到昭應奉迎,留帳飲數日還。帝崩,中人有為後謀稱制者,後怒曰:「吾效武氏邪?今太子雖幼,尚可選重德為輔,吾何與外事哉?」



 敬宗立,號太皇太后。寶歷倉卒,後召江王嗣皇帝位,是為文宗。文宗性謹孝,事後有禮,凡羞果鮮珍及四方奇奉,必先獻宗廟、三宮,而後御之。



 武宗喜畋游,角武抃,擇五坊小兒得出入禁中。它日問後起居,從容請曰:「如何可為盛天子?」後曰:「諫巨章疏宜審覽,度可用用之,有不可,以詢宰相。毋拒直言,勿納偏言,以忠良為腹心,此盛天子也。」帝再拜,還索諫章閱之,往往道游獵事,自是畋幸稀,小兒武抃等不復橫賜矣。



 宣宗立,於後,諸子也,而母鄭,故侍兒,有曩怨。帝奉養禮稍薄,後鬱鬱不聊,與一二侍人登勤政樓,將自隕,左右共持之。帝聞不喜,是夕後暴崩。有司上尊謚,葬景陵外園。太常官王暤請後合葬景陵,以主祔憲宗室,帝不悅,令宰相白敏中讓之。暤曰:「後乃憲宗東宮元妃,事順宗為婦,歷五朝母天下,不容有異論。」敏中亦怒,周墀又責謂,暤終不橈,墀曰:「皋信孤直。」俄貶暤句容令。懿宗咸通中,暤還為禮官,申抗前論,乃詔後主祔於廟。



 憲宗孝明皇後鄭氏,丹楊人,或言本爾硃氏。元和初,李錡反,有相者言後當生天子。錡聞,納為侍人。錡誅,沒入掖廷,侍懿安後。憲宗幸之,生宣宗。宣宗為光王,後為王太妃。及即位,尊為皇太后。太后不肯別處,故帝奉養大明宮,朝夕躬省候焉。懿宗立,尊後為太皇太后。咸通三年,帝奉後宴三殿,命翰林學士侍立結綺樓下。六年崩,移仗西內,上謚冊,葬景陵旁園。



 穆宗恭僖皇后王氏,越州人,本仕家子。幼得侍帝東宮,生敬宗。長慶時,冊為妃。敬宗立,上尊號為皇太后,贈後父紹卿司空,母張追封趙國夫人。文宗時,稱寶歷太后。大和五年,宰相建白以太皇太后與寶歷太后稱號未辨,前代詔令不敢斥言,皆以宮為稱,今寶歷太后居義安殿,宜曰義安太后。詔可。會昌五年崩,有司上謚,葬光陵東園。



 穆宗貞獻皇后蕭氏,閩人也。穆宗為建安王,後得侍,生文宗。文宗立,上尊號曰皇太后。



 初,後去家入長安,不復知家存亡,惟記有弟,帝為訪之。俄有男子蕭洪因后姊婿呂璋白見之,太后謂得真弟,悲不自勝。帝拜洪金吾將軍,出為河陽三城節度使,稍徙坊。始,節度自神策出者,舉軍為辨裝,因三倍取償。洪所代未及償而死,軍中並責償於洪,洪不許,左軍中尉仇士良憾之。會閩有男子蕭本又稱太后弟,士良以聞,自坊召洪下獄按治,洪乃代人,詔流驩州,不半道,賜死。擢本贊善大夫,寵贈三世,帝以為真,不淹旬,賜累鉅萬。然太后真弟庸軟莫能自達,本紿得其家系,士良主之,遂聽不疑。歷衛尉卿、金吾將軍。會福建觀察使唐扶上言,泉州男子蕭弘自言太后弟,御史臺參治非是,昭義劉從諫又為言,請與本辨,有詔三司高元裕、孫簡、崔郇雜問,乃皆妄。本流愛州,弘儋州,而太后終不獲弟。



 初,大和中,懿安太后居興慶宮,寶歷太后居義安殿,後居大內,號「三宮太后」。帝每五日問安及歲時慶謁,率繇復道至南內,群臣及命婦詣宮門候起居。有司獻四時新物送三宮,亦稱賜,帝曰:「上三宮,何可言賜?」遽索筆滅「賜」為「奉」。開成中,正月望夜,帝御咸泰殿,大然鐙作樂,迎三宮太后,奉觴進壽,禮如家人,諸王、公主皆得侍。



 武宗時,徙積慶殿,又號積慶太后。大中元年崩,上今謚。



 穆宗宣懿皇后韋氏,失其先世。穆宗為太子,後得侍,生武宗。長慶時,冊為妃。



 武宗立,妃已亡,追冊為皇太后,上尊謚,又封後二女弟為夫人。有司奏:「太后陵宜別制號。」帝乃名所葬園曰福陵。既又問宰相:「葬從光陵與但祔廟孰安?」奏言:「神道安於靜,光陵因山為固,且二十年,不可更穿。福陵崇築已有所,當遂就。臣等請奉主祔穆宗廟便。」帝乃下詔:「朕因誕日展禮於太皇太后,謂朕曰:『天子之孝,莫大於承續。』今穆宗皇帝虛合享之位,而宣懿太後實生嗣君,當以祔廟。」繇是奉後合食穆宗室。



 尚宮宋若昭,貝州清陽人,世以儒聞。父廷芬,能辭章,生五女,皆警慧,善屬文。長若莘,次若昭、若倫、若憲、若荀。莘、昭文尤高。皆性素潔,鄙薰澤靚妝,不願歸人,欲以學名家,家亦不欲與寒鄉凡裔為姻對,聽其學。若莘誨諸妹如嚴師,著《女論語》十篇,大抵準《論語》,以韋宣文君代孔子,曹大家等為顏、冉,推明婦道所宜。若昭又為傳申釋之。



 貞元中,昭義節度使李抱真表其才,德宗召入禁中,試文章,並問經史大誼,帝咨美,悉留宮中。帝能詩,每與侍臣賡和,五人者皆預,凡進御,未嘗不蒙賞。又高其風操,不以妾侍命之,呼學士。擢其父饒州司馬、習藝館內教,賜第一區,加穀帛。



 元和末,若莘卒,贈河內郡君。自貞元七年,秘禁圖籍,詔若莘總領,穆宗以若昭尤通練,拜尚宮,嗣若莘所職。歷憲、穆、敬三朝,皆呼先生,後妃與諸王、主率以師禮見。寶歷初卒,贈梁國夫人,以鹵簿葬。



 若憲代司秘書,文宗尚學,以若憲善屬辭,粹論議,尤禮之。大和中,李訓、鄭注用事,惡宰相李宗閔,譖言因駙馬都尉沈〓厚賂若憲求執政。帝怒,幽若憲外第,賜死,家屬徙嶺南。訓、注敗,帝悟其讒,追恨之。



 若倫、若荀早卒。廷芬男獨愚不可教,為民終身。



 敬宗貴妃郭氏,右威衛將軍義之子,失義何所人。長慶時,後以容選入太子宮。太子即位,為才人,生晉王普。帝以早得子,又淑麗冠後廷,故寵異之。逾年,為貴妃,贈義禮部尚書,兄環少府少監,賜大第。文宗立,愛晉王若己子,待妃禮不衰。亡其薨年。



 武宗賢妃王氏,邯鄲人,失其世。年十三,善歌舞,得入宮中。穆宗以賜潁王。性機悟。開成末,王嗣帝位,妃陰為助畫,故進號才人,遂有寵。狀纖頎,頗類帝。每畋苑中,才人必從,袍而騎,校服光侈,略同至尊,相與馳出入,觀者莫知孰為帝也。帝欲立為後,宰相李德裕曰:「才人無子,且家不素顯,恐詒天下議。」乃止。



 帝稍惑方士說,欲餌藥長年,後寢不豫。才人每謂親近曰:「陛下日燎丹,言我取不死。膚澤消槁,吾獨憂之。」俄而疾侵,才人侍左右,帝熟視曰:「吾氣奄奄,情慮耗盡,顧與汝辭。」答曰:「陛下大福未艾,安語不祥?」帝曰:「脫如我言,奈何?」對曰:「陛下萬歲後,妾得以殉。」帝不復言。及大漸,才人悉取所常貯散遺宮中,審帝已崩,即自經幄下。當時嬪媛雖常妒才人專上者,返皆義才人,為之感動。宣宗即位,嘉其節,贈賢妃,葬端陵之柏城。



 宣宗元昭皇后晁氏,不詳其世。少入邸,最見寵答。及即位,以為美人。大中中薨,贈昭容,詔翰林學士蕭寘銘其窆,具載生鄆王、萬壽公主。後夔、昭等五王居內院,而鄆獨出閣。及即位,是為懿宗。外頗疑帝非長。寘出銘辭以示外廷,乃解。帝追冊昭容為皇太后,上尊謚,詔後二等以上親悉官之,配主宣宗廟,自建陵曰慶陵,置宮寢。



 懿宗惠安皇后王氏,亦失所來。咸通中,冊號貴妃,生普王。七年薨。十四年,王即位,是為僖宗。追尊皇太后,冊上謚號,祔主懿宗廟,即其園為壽陵。後屬緦以上,帝悉官之。



 懿宗淑妃郭氏,幼入鄆王邸。宣宗在位,春秋高,惡人言立太子事。王以嫡長居外宮,心常憂惴。妃護侍左右,慰安起居,終得無恙。生女未能言,忽曰:「得活。」王驚異之。及即位,以妃為美人,進拜淑妃。



 女為同昌公主,下嫁韋保衡。保衡處內宅,妃以主故,出入娛飲不禁,是時嘩言與保衡亂,莫得其端。僖宗立,保衡緣它罪為人所發,且污舊謗,卒貶死。妃猶處禁中。黃巢之難,天子出蜀倉卒,妃不及從,遂流落閭里,不知所終。



 懿宗恭憲皇后王氏,其出至微。咸通中,列後廷,得幸,生壽王而卒。王立,是為昭宗,追號皇太后,上謚,祔主懿宗室,即故葬號安陵,召后弟〓官之。



 景福初,〓位任浸重,帝亦以外家倚之,為中尉楊復恭所媢,表為黔南節度使。〓之鎮,道吉柏江,復恭密喻楊守亮覆其家。



 昭宗皇后何氏,梓州人,系族不顯。帝為壽王,後得侍,婉麗多智,恩答厚甚。既即位,號淑妃。從狩華州,詔冊為皇后。



 光化三年,帝獵夜歸,後遣德王還邸,遇劉季述,留王紫廷院。明日,季述等挾王陳兵召百官,脅帝內禪。後恐賊臣加害天子,即取璽授季述,與帝同幽東宮。賊平,反正。



 天復中,從帝駐鳳翔,李茂貞請帝勞軍,不得已,後從禦南樓。會硃全忠逼帝東遷,後謂帝曰:「此後大家夫婦委身賊手矣!」涕數行下。帝奔播既屢,威柄盡喪,左右皆悍逆庸奴,後侍膳服,無須臾去側。至洛,帝憂,忽忽與後相視無死所。已而遇弒。



 哀帝即位,尊為皇太后,宮中不敢哭,徙居積善宮,號積善太后。帝將禪天下,後亦遇害。初,蔣玄暉為全忠邀九錫,入喻,後度不免,見玄暉垂泣祈哀,以母子托命。宣徽使趙殷衡譖於全忠曰:「玄暉等銘石像瘞積善宮,將復唐。」全忠怒,遂遣縊後,以醜名加之,廢為庶人。



\end{pinyinscope}