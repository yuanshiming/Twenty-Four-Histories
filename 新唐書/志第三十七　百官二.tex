\article{志第三十七 百官二}

\begin{pinyinscope}

 ○門下省



 侍中二人,正二品。掌出納帝命,相禮儀。凡國家之務,與中書令參總,而顓判省事。下之通上,其制有六:一曰奏鈔,以支度國用、授六品以下官、斷流以下罪及除免官用之;二曰奏彈;三曰露布;四曰議;五曰表;六曰狀。自露布以上乃審,其餘覆奏,畫制可而授尚書省。行幸,則負寶以從,版奏中嚴、外辦;還宮,則請降輅、解嚴。皇帝齋,則請就齋室;將奠,則奉玉、幣;盥,則奉匜、取盤,酌罍水,贊洗;酌泛齊,受虛爵,進福酒,皆左右其儀。饗宗廟,進瓚而贊酌鬱酒;既祼,贊酌醴齊。籍田,則奉耒。四夷朝見,則承詔勞問。臨軒命使冊皇后、皇太子,則承詔降宣命。慰問、聘召,則涖封題。發驛遣使,則給魚符。凡官爵廢置、刑政損益,授之史官;既書,復涖其記注。職事官六品以下進擬,則審其稱否而進退之。武德元年改侍內曰納言,三年曰侍中。龍朔二年改門下省曰東臺,侍中曰左相,武後光宅元年曰納言,垂拱元年改門下省曰鸞臺。開元元年曰黃門省,侍中曰監,天寶元年曰左相。



 門下侍郎二人,正三品。掌貳侍中之職。大祭祀則從;盥則奉巾,既帨,奠巾;奉匏爵贊獻。元日、冬至,奏天下祥瑞。侍中闕,則涖封符券、給傳驛。龍朔二年改黃門侍郎曰東臺侍郎,武后垂拱元年曰鸞臺侍郎,天寶元年曰門下侍郎,乾元元年曰黃門侍郎,太歷二年復舊。



 左散騎常侍二人,正三品下。掌規諷過失,侍從顧問。隋廢散騎常侍。貞觀元年復置,十七年為職事官。顯慶二年,分左右,隸門下、中書省,皆金蟬、珥貂,左散騎與侍中為左貂,右散騎與中書令為右貂,謂之八貂。龍朔二年曰侍極。



 左諫議大夫四人,正四品下。掌諫諭得失,侍從贊相。武后垂拱二年,有魚保宗者,上書請置匭以受四方之書,乃鑄銅匭四,塗以方色,列於朝堂:青匭曰「延恩」在東,告養人勸農之事者投之,丹匭曰「招諫」,在南,論時政得失者投之;白匭曰「申冤」,在西,陳抑屈者投之;黑匭曰「通玄」,在北,告天文、秘謀者投之。以諫議大夫、補闕、拾遺一人充使,知匭事;御史中丞、侍御史一人,為理匭使。其後同為一匭。天寶九載,玄宗以「匭」聲近「鬼」,改理匭使為獻納使,至德元年復舊。寶應元年,命中書門下擇正直清白官一人知匭,以給事中、中書舍人為理匭使。建中二年,以御史中丞為理匭使,諫議大夫一人為知匭使;投匭者,使先驗副本。開成三年,知匭使李中敏以為非所以廣聰明而慮幽枉也,乃奏罷驗副封。武德元年置諫議大夫,龍朔二年曰正諫大夫,貞元四年分左右。



 給事中四人,正五品上。掌侍左右,分判省事,察弘文館繕寫讎校之課。凡百司奏抄,侍中既審,則駁正違失。詔敕不便者,塗竄而奏還,謂之「塗歸」。季終,奏駁正之目。凡大事,覆奏;小事,署而頒之。三司詳決失中,則裁其輕重。發驛遣使,則與侍郎審其事宜。六品以下奏擬,則校功狀殿最、行藝,非其人,則白侍中而更焉。與御史、中書舍人聽天下冤滯而申理之。



 門下省有錄事四人,從七品上;主事四人,從八品下。有令史二十二人,書令史四十三人,甲庫令史十三人,能書一人,傳制二人,亭長六人,掌固十四人,脩補制敕匠五人,裝潢一人。起居郎領令史三人,贊者六人。武德三年,改給事郎曰給事中。



 左補闕六人,從七品上;左拾遺六人,從八品上。掌供奉諷諫,大事廷議,小則上封事。武后垂拱元年,置補闕、拾遺,左右各二員。



 起居郎二人,從六品上。掌錄天子起居法度。天子御正殿,則郎居左,舍人居右。有命,俯陛以聽,退而書之,季終以授史官。貞觀初,以給事中、諫議大夫兼知起居注,或知起居事。每仗下,議政事,起居郎一人執筆記錄於前,史官隨之。其後,復置起居舍人,分侍左右,秉筆隨宰相入殿;若仗在紫宸內閤,則夾香案分立殿下,直第二螭首,和墨濡筆,皆即坳處,時號螭頭。高宗臨朝不決事,有司所奏,唯辭見而已。許敬宗、李義府為相,奏請多畏人之知也,命起居郎、舍人對仗承旨,仗下,與百官皆出,不復聞機務矣。長壽中,宰相姚建議:仗下後,宰相一人,錄軍國政要,為時政紀,月送史館。然率推美讓善,事非其實,未幾亦罷。而起居郎猶因制敕,稍稍筆削,以廣國史之闕。起居舍人本記言之職,唯編詔書,不及它事。開元初,復詔脩史官非供奉者,皆隨仗而入,位於起居郎、舍人之次。及李林甫專權,又廢。大和九年,詔入閣日,起居郎、舍人具紙筆立螭頭下,復貞觀故事。有令史三人,贊者六人。貞觀三年置起居郎,廢舍人。龍朔二年曰左史,天授元年亦如之。



 典儀二人,從九品下。掌贊唱及殿中版位之次,侍中版奏中嚴、外辦,亦贊焉。隋謁者臺有典儀,武德五年復置,隸門下省。



 城門郎四人,從六品上。掌京城、皇城、官殿諸門開闔之節,奉管鑰而出納之。開則先外後內,闔則先內後外;啟閉有時,不以時則詣閤覆奏。有令史二人,書令史二人。武德五年,置門僕八百人,番上送管鑰。



 符寶郎四人,從六品上。掌天子八寶及國之符節。有事則請於內,既事則奉而藏之。大朝會,則奉寶進於御座;行幸,則奉以從焉。大事出符,則藏其左而班其右,以合中外之契,兼以敕書;小事則降符函封,使合而行之。凡命將、遣使,皆請旌、節,旌以顓賞,節以顓殺。有令史二人,書令史三人,主寶二人,主符四人,主節四人。武後延載元年,改符璽郎曰符寶郎;開元元年,亦曰符寶郎。



 ○弘文館



 學士,掌詳正圖籍,教授生徒;朝廷制度沿革、禮儀輕重,皆參議焉。武德四年,置修文館於門下省;九年,改曰弘文館。貞觀元年,詔京官職事五品已上子嗜書者二十四人,隸館習書,出禁中書法以授之。其後又置講經博士。儀鳳中,置詳正學士,校理圖籍。武德後,五品以上曰學士,六品已下曰直學士,又有文學直館,皆它官領之。武后垂拱後,以宰相兼領館務,號館主;給事中一人判館事。神龍元年,改弘文館曰昭文館,以避孝敬皇帝之名;二年曰脩文館。景龍二年,置大學士四人,以象四時;學士八人,以象八節;直學士十二人,以象十二時。景雲中,減其員數,復為昭文館。開元七年曰弘文館,置校書郎,又有校理、讎校錯誤等官。長慶三年,與詳正學士、講經博士皆罷,顓以五品以上曰學士,六品以下曰直學士,未登朝為直館。



 校書郎二人,從九品上。掌校理典籍、刊正錯謬。凡學生教授、考試,如國子之制。有學生三十八人,令史二人,楷書十二人,供進筆二人,典書二人,拓書手三人,筆匠三人,熟紙裝潢匠八人,亭長二人,掌固四人。



 ○中書省



 中書令二人,正二品。掌佐天子執大政,而總判省事。凡王言之制有七:一曰冊書,立皇后、皇太子,封諸王,臨軒冊命則用之;二曰制書,大賞罰、赦宥慮囚、大除授則用之;三曰慰勞制書,褒勉贊勞則用之;四曰發敕,廢置州縣、增減官吏、發兵、除免官爵、授六品以上官則用之;五曰敕旨,百官奏請施行則用之;六曰論事敕書,戒約臣下則用之;七曰敕牒,隨事承制,不易於舊則用之。皆宣署申覆,然後行焉。大祭祀,則相禮;親征纂嚴,則戒飭百官;臨軒冊命,則讀冊;若命於朝,則宣授而已。冊太子,則授璽綬。凡制詔文章獻納,以授記事之官。武德三年,改內書省曰中書省,內書令曰中書令。龍朔元年,改中書省曰西臺,中書令曰右相。光宅元年,改中書省曰鳳閣,中書令曰內史。開元元年,改中書省曰紫微省,中書令曰紫微令。天寶元年曰右相,至大歷五年,紫微侍郎乃復為中書侍郎。



 侍郎二人,正三品。掌貳令之職,朝廷大政參議焉。臨軒冊命,為使,則持冊書授之。四夷來朝,則受其表疏而奏之;獻贄幣,則受以付有司。



 舍人六人,正五品上。掌侍進奏,參議表章。凡詔旨制敕、璽書冊命,皆起草進畫;既下,則署行。其禁有四:一曰漏洩,二曰稽緩,三曰違失,四曰忘誤。制敕既行,有誤則奏改之。大朝會,諸方起居,則受其表狀;大捷、祥瑞,百寮表賀亦如之。冊命大臣,則使持節讀冊命;將帥有功及大賓客,則勞問。與給事中及御史三司鞫冤滯。百司奏議考課,皆預裁焉。以久次者一人為閣老,判本省雜事;又一人知制誥,顓進畫,給食於政事堂;其餘分署制敕。以六員分押尚書六曹,佐宰相判案,同署乃奏,唯樞密遷授不預。姚崇為紫微令,奏:大事,舍人為商量狀,與本狀皆下紫微令,判二狀之是否,然後乃奏。開元初,以它官掌詔敕策命,謂之「兼知制誥」。肅宗即位,又以它官知中書舍人事。兵興,急於權便,政去臺閣,決遣顓出宰相,自是舍人不復押六曹之奏。會昌末,宰相李德裕建議:臺閣常務、州縣奏請,復以舍人平處可否。先是,知制誥率用前行正郎,宣宗時,選尚書郎為之。



 主書四人,從七品上。主事四人,從八品下。有令史二十五人,書令史五十人,能書四人,蕃書譯語十人,乘驛二十人,傳制十人,亭長十八人,掌固二十四人,裝制敕匠一人,脩補制敕匠五十人,掌函、掌案各二十人。



 右散騎常侍二人,右諫議大夫四人,右補闕六人,右拾遺六人,掌如門下省。



 起居舍人二人,從六品上。掌脩記言之史,錄制誥德音,如記事之制,季終以授國史。有楷書手四人,典二人。



 通事舍人十六人,從六品上。掌朝見引納、殿庭通奏。凡近臣入侍、文武就列,則導其進退,而贊其拜起、出入之節。蠻夷納貢,皆受而進之。軍出,則受命勞遣;既行,則每月存問將士之家,視其疾苦;凱還,則郊迓。有令史十人,典謁十人,亭長十八人,掌固二十四人。武德四年,廢謁者臺,改通事謁者曰通事舍人。



 ○集賢殿書院



 學士、直學士、侍讀學士、脩撰官,掌刊緝經籍。凡圖書遺逸、賢才隱滯,則承旨以求之。謀慮可施於時,著述可行於世者,考其學術以聞。凡承旨撰集文章、校理經籍,月終則進課於內,歲終則考最於外。開元五年,乾元殿寫四部書,置乾元院使,有刊正官四人,以一人判事;押院中使一人,掌出入宣奏,領中官監守院門;知書官八人,分掌四庫書。六年,乾元院更號麗正脩書院,置使及檢校官,改脩書官為麗正殿直學士。八年,加文學直,又加脩撰、校理、刊正、校勘官。十一年,置麗正院脩書學士;光順門外,亦置書院。十二年,東都明福門外亦置麗正書院。十三年,改麗正脩書院為集賢殿書院,五品以上為學士,六品以下為直學士,宰相一人為學士知院事,常侍一人為副知院事,又置判院一人、押院中使一人。玄宗嘗選耆儒,日一人侍讀,以質史籍疑義,至是,置集賢院侍講學士、侍讀直學士。其後,又增脩撰官、校理官、待制官、留院官、知檢討官、文學直之員;募能書者為書直及寫御書人,其後亦以前資、常選、三衛、散官五品以上子孫為之;又置畫直。至十九年,以書直、畫直、拓書有官者為直院。至德二年,置大學士。貞元初,置編錄官;四年,罷大學士;八年,罷校理,置校書四人、正字二人。元和二年,復置集賢校理,罷校書、正字;四年,集賢御書院學士、直學士皆用五品,如開元故事,以學士一人年高者判院事,非登朝官者為校理,餘皆罷。初,太宗即位,命京官五品以上,更宿中書、門下兩省,以備訪問。永徽中,命弘文館學士一人,日待制於武德殿西門。文明元年,詔京官五品以上清官,日一人待制於章善、明福門。先天末,又命朝集使六品以上二人,隨仗待制。永泰時,勛臣罷節制,無職事,皆待制於集賢門,凡十三人。崔祐甫為相,建議文官一品以上更直待制。其後著令,正衙待制官日二人。



 校書四人,正九品下。正字二人,從九品上。有中使一人,孔目官一人,專知御書檢討八人,知書官八人,書直、寫御書手九十人,畫直六人,裝書直十四人,造筆直四人,拓書六人,典四人。



 ○史館



 修撰四人,掌修國史。貞觀三年,置史館於門下省,以他官兼領,或卑位有才者亦以直館稱,以宰相涖脩撰;又於中書省置秘書內省,脩五代史。開元二十年,李林甫以宰相監脩國史,建議以為中書切密之地,史官記事隸門下省,疏遠。於是諫議大夫、史館脩撰尹愔奏徙於中書省。天寶後,他官兼史職者曰史館脩撰,初入為直館。元和六年,宰相裴垍建議:登朝官領史職者為修撰,以官高一人判館事;未登朝官皆為直館。大中八年,廢史館直館二員,增脩撰四人,分掌四季。有令史二人,楷書十二人,寫國史楷書十八人,楷書手二十五人,典書二人,亭長二人,掌固四人,熟紙匠六人。



 ○秘書省



 監一人,從三品;少監二人,從四品上;丞一人,從五品上。監掌經籍圖書之事,領著作局,少監為之貳。武德四年,改少令曰少監。龍朔二年,改秘書省曰蘭臺,監曰太史,少監曰侍郎,丞曰大夫,秘書郎曰蘭臺郎。武后垂拱元年,秘書省曰麟臺;太極元年曰秘書省。有典書四人,楷書十人,令史四人,書令史九人,亭長六人,掌固八人,熟紙匠十人,裝潢匠十人,筆匠六人。



 秘書郎三人,從六品上。掌四部圖籍。以甲乙丙丁為部,皆有三本,一曰正,二曰副,三曰貯。凡課寫功程,皆分判。



 校書郎十人,正九品上;正字四人,正九品下。掌讎校典籍,刊正文章。



 ○著作局



 郎二人,從五品上;著作佐郎二人,從六品上;校書郎二人,正九品上;正字二人,正九品下。著作郎掌撰碑志、祝文、祭文,與佐郎分判局事。武德四年,改著作曹曰局。龍朔二年,曰司文局;郎曰郎中,佐郎曰司文郎。有楷書五人,書令史一人,書吏二人,掌固四人。



 ○司天臺



 監一人,正三品;少監二人,正四品上;丞一人,正六品上;主簿二人,正七品上;主事一人,正八品下。監掌察天文,稽歷數。凡日月星辰、風雲氣色之異,率其屬而占。有通玄院,以藝學召至京師者居之。凡天文圖書、器物,非其任不得與焉。每季錄祥眚送門下、中書省,紀於起居注,歲終上送史館。歲頒歷於天下。武德四年,改太史監曰太史局,隸秘書省;七年,廢監候。龍朔二年,改太史局曰秘書閣局,令曰秘書閣郎中。武後光宅元年,改太史局曰渾天監,不隸麟臺;俄改曰渾儀監,置副監及丞、主簿,改司辰師曰司辰。長安二年,渾儀監復曰太史局,廢副監及丞,隸麟臺如故,改天文博士曰靈臺郎,歷博士曰保章正。景龍二年,改太史局曰太史監,不隸秘書省,復置丞。景雲元年,又為局,隸秘書省,逾月為監,歲中復為局;二年,改曰渾儀監。開元二年,復曰太史監,改令為監,置少監。十四年,太史監復為局,以監為令,而廢少監。天寶元年,太史局復為監,自是不隸秘書省。乾元元年,曰司天臺。藝術人韓潁、劉烜建議改令為監,置通玄院及主簿,置五官監候及五官禮生十五人,掌布諸壇神位,五官楷書手五人,掌寫御書。有令史五人,天文觀生九十人,天文生五十人,歷生五十五人。初,有天文博士二人,正八品下;歷博士一人,從八品上;司辰師五人,正九品下,裝書歷生五人。掌候天文,掌教習天文氣色,掌寫御歷,後皆省。



 春官、夏官、秋官、冬官、中官正,各一人,正五品上;副正各一人,正六品上。掌司四時,各司其方之變異。冠加一星珠,以應五緯;衣從其方色。元日、冬至、朔望朝會及大禮,各奏方事,而服以朝見。乾元三年,置五官正及副正。



 五官保章正二人,從七品上;五官監候三人,正八品下;五官司歷二人,從八品上。掌歷法及測景分至表準。



 五官靈臺郎各一人,正七品下。掌候天文之變。五官挈壺正二人,正八品上;五官司辰八人,正九品上;漏刻博士六人,從九品下。掌知漏刻。凡孔壺為漏,浮箭為刻,以考中星昏明,更以擊鼓為節,點以擊鐘為節。武后長安二年,置挈壺正。乾元元年,與靈臺郎、保章正、司歷、司辰,皆加五官之名。有漏刻生四十人,典鐘、典鼓三百五十人。初,有刻漏視品、刻漏典事,掌知刻漏、檢校刻漏,後皆省。



 ○殿中省



 監一人,從三品;少監二人,從四品上;丞二人,從五品上。監掌天子服御之事。其屬有六局,曰尚食、尚藥、尚衣、尚乘、尚舍、尚輦。少監為之貳。凡聽朝,率屬執繖扇列於左右;大朝會、祭祀,則進爵;行幸,則侍奉仗內、驂乘,百司皆納印而藏之,大事聽焉,有行從百司之印。左右仗廄:左曰奔星,右曰內駒。兩仗內又有六廄:一曰左飛,二曰右飛,三曰左萬,四曰右萬,五曰東南內,六曰西南內。園苑有官馬坊,每歲河隴群牧進其良者以供御。六閑馬,以殿中監及尚乘主之。武後萬歲通天元年,置仗內六閑:一曰飛龍,二曰祥麟,三曰鳳苑,四曰鵷鸞,五曰吉良,六曰六群,亦號六廄。以殿中丞檢校仗內閑廄,以中官為內飛龍使。聖歷中,置閑廄使,以殿中監承恩遇者為之,分領殿中、太僕之事,而專掌輿輦牛馬。自是,宴游供奉,殿中監皆不豫。開元初,閑廄馬至萬餘匹,駱駝、巨象皆養焉。以駝、馬隸閑廄,而尚乘局名存而已。閑廄使押五坊,以供時狩:一曰雕坊,二曰鶻坊,三曰鷂坊,四曰鷹坊,五曰狗坊。侍御尚醫二人,正六品上;主事二人,從九品上。武德元年,改殿內省曰殿中省。龍朔二年,曰中御府,監曰大監,丞曰大夫。有令史四人,書令史十二人,左右仗、千牛各十人,掌固、亭長各八人。舊有天藏府,開元二十三年省。



 進馬五人,正七品上。掌大陳設,戎服執鞭,居立仗馬之左,視馬進退。天寶八載,罷南衙立仗馬,因省進馬;十二載復置,乾元後又省,大歷十四年復。



 △尚食局



 奉御二人,正五品下;直長五人,正七品上。諸奉御、直長,品皆如之。食醫八人,正九品下。奉御掌儲供,直長為之貳。進御必辨時禁,先嘗之;饗百官賓客,則與光祿視品秩而供;凡諸陵月享,視膳乃獻。龍朔二年,改尚食局曰奉膳局,諸局奉御皆曰大夫。有書令史二人,書吏五人,主食十六人,主膳八百四十人,掌固八人。



 △尚藥局



 奉御二人,直長二人。掌和御藥、診視。凡藥供御,中書、門下長官及諸衛上將軍各一人,與監、奉御涖之。藥成,醫佐以上先嘗,疏本方,具歲月日,涖者署奏;餌日,奉御先嘗,殿中監次之,皇太子又次之,然後進御。太常每季閱送上藥,而還其朽腐者。左右羽林軍,給藥;飛騎、萬騎病者,頒焉。龍朔二年,改尚藥局曰奉醫局。有按摩師四人,咒禁師四人,書令史二人,書吏四人,直官十人,主藥十二人,藥童三十人,合口脂匠二人,掌固四人。



 侍御醫四人,從六品上。掌供奉診候。



 司醫五人,正八品下;醫佐十人,正九品下。掌分療眾疾。皆貞觀中置。



 △尚衣局



 奉御二人,直長四人,掌供冕服、幾案。祭祀,則奉鎮圭於監,而進於天子;大朝會,設案。龍朔二年,改尚衣局曰奉冕局。有書令史三人,書吏四人,主衣十六人,掌固四人。



 △尚舍局



 奉御二人,直長六人,掌殿庭祭祀張設、湯沐、燈燭、汛掃。行幸,則設三部帳幕,有古帳、大帳、次帳、小次帳、小帳凡五等,各三部;其外,則蔽以排城。大朝會,設黼扆,施躡席,薰爐。朔望,設幄而已。龍朔二年,改尚舍局曰奉扆局。有書令史三人,書吏七人,掌固十人,幕士八十人。舊有給使百二十人,掌供御湯沐、燈燭、雜使,貞觀中省。



 尚乘局奉御二人,直長十人,掌內外閑廄之馬。左右六閑:一曰飛黃,二曰吉良,三曰龍媒,四曰騊駼,五曰駃騠,六曰天苑。凡外牧歲進良馬,印以三花、「飛」「鳳」之字。飛龍廄日以八馬列宮門之外,號南衙立仗馬,仗下,乃退。大陳設,則居樂縣之北,與象相次。龍朔二年,改尚乘局曰奉駕局。有書令史六人,書吏十四人,直官二十人,習馭五百人,掌閑五千人,典事五人,獸醫七十人,掌固四人。習馭,掌調六閑之馬;掌閑,掌飼六閑之馬,治其乘具鞍轡;典事,掌六閑芻粟。太宗置司廩,司庫;高宗置習馭、獸醫。



 司廩、司庫各一人,正九品下。掌六閑槁秸出納。奉乘十八人,正九品下。掌飼習御馬。



 △尚輦局



 奉御二人;直長三人;尚輦二人,正九品下。掌輿輦、繖扇,大朝會則陳於庭,大祭祀則陳於廟,皆繖二、翰一、扇一百五十有六,既事而藏之。常朝則去扇,左右留者三。龍朔二年,改尚輦局曰奉輿局。有書令史二人,書吏四人,七輦主輦各六人,掌扇六十人,掌翰三十人,掌輦四十二人,奉輿十五人,掌固六人。掌扇、掌翰,掌執繖扇、紙筆硯雜供奉之事;掌輦,掌率主輦以供其事。高宗置掌翰。



 ○內侍省



 監二人,從三品;少監二人,內侍四人,皆從四品上。監掌內侍奉,宣制令。其屬六局,曰掖庭、宮闈、奚官、內僕、內府、內坊。少監、內侍為之貳。皇后親蠶,則升壇執儀;大駕出入,為夾引。武德四年,改長秋監曰內侍監,內承奉曰內常侍,內承直曰內給事。龍朔二年,改監為省。武后垂拱元年,曰司宮臺。天寶十三載,置內侍監,改內侍曰少監;尋更置內侍。有高品一千六百九十六人,品官白身二千九百三十二人,令史八人,書令史十六人。



 內常侍六人,正五品下,通判省事。



 內給事十人,從五品下。掌承旨勞問,分判省事。凡元日、冬至,百官賀皇后,則出入宣傳;宮人衣服費用,則具品秩,計其多少,春秋宣送於中書。主事二人,從九品下。



 內謁者監十人,正六品下。掌儀法、宣奏、承敕令及外命婦名帳。凡諸親命婦朝會者,籍其數上內侍省;命婦下車,則導至朝堂奏聞。唐廢內謁者局,置內典引十八人,掌諸親命婦朝參,出入導引。有內亭長六人,掌固八人。



 內謁者十二人,從八品下。掌諸親命婦朝集班位,分涖諸門。



 內寺伯六人,正七品下。掌糾察宮內不法,歲儺則涖出入。



 寺人六人,從七品下。掌皇后出入執御刀冗從。



 △掖庭局



 令二人,從七品下;丞三人,從八品下。掌宮人簿帳、女工。凡宮人名籍,司其除附;公桑養蠶,會其課業;供奉物皆取焉。婦人以罪配沒,工縫巧者隸之,無技能者隸司農。諸司營作須女功者,取於戶婢。有書令史四人,書吏八人,計史二人,典事十人,掌固四人。計史掌料功程。



 宮教博士二人,從九品下。掌教習宮人書、算、眾藝。初,內文學館隸中書省,以儒學者一人為學士,掌教宮人。武後如意元年,改曰習藝館,又改曰萬林內教坊,尋復舊。有內教博士十八人,經學五人,史、子、集綴文三人,楷書二人,《莊老》、太一、篆書、律令、吟詠、飛白書、算、棋各一人。開元末,館廢,以內教博士以下隸內侍省,中官為之。



 監作四人,從九品下。掌監涖雜作,典工役。



 △宮闈局



 令二人,從七品下;丞二人,從八品下。掌侍宮闈,出入管鑰。凡享太廟,皇后神主出入,則帥其屬輿之。總小給使學生之籍,給以糧稟。有書令史三人,書吏六人,內閽史二十人,內掌扇十六人,內給使無常員,小給使學生五十人,掌固四人。凡無官品者,號曰內給使,掌諸門進物之歷;內閽史,掌承傳諸門,出納管鑰;內掌扇,掌中宮繖扇。



 △奚官局



 令二人,正八品下;丞二人,正九品下。掌奚隸、工役、宮官之品。宮人病,則供醫藥;死,給衣服,各視其品。陪陵而葬者,將作給匠戶,衛士營塚,三品葬給百人,四品八十人,五品六十人,六品、七品十人,八品、九品七人;無品者,斂以松棺五釘,葬以犢車,給三人。皆監門校尉、直長涖之。內命婦五品以上無親戚者,以近塚同姓中男一人主祭於墓;無同姓者,春、秋祠以少牢。有書令史三人,書吏六人,典事、藥童、掌固各四人。



 △內僕局



 令二人,正八品下;丞二人,正九品下。掌中宮車乘。皇后出,則令居左、丞居右,夾引。有書令史二人,書吏四人,駕士百四十人,典事八人,掌固八人。駕士掌習御車輿、雜畜。



 ○內府局



 令二人,正八品下;丞二人,正九品下。掌中藏寶貨給納之數,及供燈燭、湯沐、張設。凡朝會,五品已上及有功將士、蕃酋辭還,皆賜於庭。有書令史二人,書吏、典史,掌固各四人,典事六人。



 △太子內坊局



 令二人,從五品下;丞二人,從七品下。掌東宮閤內及宮人糧稟。坊事五人,從八品下。初,內坊隸東宮。開元二十七年,隸內侍省,為局,改典內曰令,置丞。坊事及導客舍人六人,掌序導賓客;閤帥六人,掌帥閽人、內給使以供其事;內閽人八人,掌承諸門出入管鑰,內繖扇、燈燭;內廄尉二人,掌車乘。有錄事一人,令史三人,書令史五人,典事二人,駕士三十人,亭長、掌固各一人。



 典直四人,正九品下。掌宮內儀式導引,通傳勞問,糾劾非違,察出納。



 ○內官



 貴妃、惠妃、麗妃、華妃各一人,正一品。掌佐皇後論婦禮於內,無所不統。唐因隋制,有貴妃、淑妃、德妃、賢妃各一人,為夫人,正一品;昭儀、昭容、昭媛、脩儀、脩容、脩媛、充儀、充容、充媛各一人,為九嬪,正二品;婕妤九人,正三品;美人四人,正四品;才人五人,正五品;寶林二十七人,正六品;御女二十七人,正七品;採女二十七人,正八品。六尚,亦曰諸尚書,正三品;二十四司,亦曰諸司事,正四品;二十四典,亦曰諸典事,正六品;二十四掌,亦曰諸掌事。龍朔二年,置贊德二人,正一品;宣儀四人,正二品;承閏五人,正四品;承旨五人,正五品;衛仙六人,正六品;供奉八人,正七品;侍櫛二十人,正八品;侍巾三十人,正九品。咸亨復舊。開元中,玄宗以後妃四星,一為後,有後而復置四妃,非典法,乃置惠妃、麗妃、華妃,以代三夫人;又置六儀、美人、才人,增尚宮、尚儀、尚服三局。諸司諸典,自六品至九品而止。其後復置貴妃。



 淑儀、德儀、賢儀、順儀、婉儀、芳儀各一人,正二品。掌教九御四德,率其屬以贊後禮。



 美人四人,正三品,掌率女官脩祭祀、賓客之事。才人七人,正四品,掌敘燕寢,理絲枲,以獻歲功。



 ○宮官



 △尚宮局



 尚宮二人,正五品。六尚皆如之。掌導引中宮,總司記、司言、司簿、司闈。凡六尚事物出納文籍,皆涖其印署。有女史六人,掌執文書。



 司記二人,正六品;二十四司皆如之。掌宮內文簿入出,錄為抄目,審付行焉。牒狀無違,然後加印。典記佐之。典記二人,正七品;二十四典皆如之。掌記二人,正八品;二十四掌皆如之。



 司言、典言各二人,掌承敕宣付,別鈔以授司閽傳外。掌言二人,掌宣傳,外司附奏受事者,奏聞;承敕處分,則錄所奏為案記。有女史四人。



 司簿、典簿、掌簿各二人,掌女史以上名簿。稟賜,則品別條錄為等。有女史六人。



 司闈六人,掌諸閤管鑰。典闈、掌闈各六人,掌分涖啟閉。有女史四人。



 △尚儀局



 尚儀二人,掌禮儀起居。總司籍、司樂、司賓、司贊。



 司籍、典籍、掌籍各二人,掌供御經籍。分四部,部別為目,以時暴涼。教學則簿記課業,供奉幾案、紙筆,皆預偫焉。有女史十人。



 司樂、典樂、掌樂各四人,掌宮縣及諸樂陳布之儀,涖其閱習。有女史二人。



 司賓、典賓、掌賓各二人,掌賓客朝見,受名以聞。宴會,則具品數以授尚食;有賜物,與尚功涖給。有女史二人。



 司贊、典贊、掌贊各二人,掌賓客朝見、宴食,贊相導引。會日,引客立於殿庭,司言宣敕坐,然後引即席。酒至,起再拜;食至,亦起。皆相其儀。



 彤史二人,正六品。有女史二人。



 △尚服局



 尚服二人,掌供服用採章之數,總司寶、司衣、司飾、司仗。



 司寶二人,掌神寶、受命寶、六寶及符契,皆識其行用,記以文簿。典寶、掌寶各二人,凡出付皆旬別案記,還則硃書注入。有女史四人。



 司衣、典衣、掌衣各二人,掌宮內御服、首飾整比,以時進奉。有女史四人。



 司飾、典飾、掌飾各二人,掌湯沐、巾櫛。凡供進,識其寒溫之節。有女史二人。



 司仗、典仗、掌仗各二人,掌仗衛之器。凡立儀衛,尚服率司仗等供其事。有女史二人。



 △尚食局



 尚食二人,掌供膳羞品齊。總司膳、司醞、司藥、司饎。凡進食,先嘗。



 司膳二人,掌烹煎及膳羞、米面、薪炭。凡供奉口味,皆種別封印。典膳、掌膳各四人,掌調和御食,溫、涼、寒、熱,以時供進則嘗之。有女史四人。



 司醞、典醞、掌醞各二人,掌酒醴酏飲,以時進御。有女史二人。



 司藥、典藥、掌藥各二人,掌醫方。凡藥外進者,簿案種別。有女史四人。



 司饎、典饎、掌饎各二人,掌給宮人餼食、薪炭,皆有等級,受付則旬別案記。有女史四人。



 △尚寢局



 尚寢二人,掌燕見進御之次敘,總司設、司輿、司苑、司燈。



 司設、典設、掌設各二人,掌床帷茵席鋪設,久故者以狀聞。凡汛掃之事,典設以下分視。有女史四人。



 司輿、典輿、掌輿各二人,掌輿輦、繖扇、文物、羽旄,以時暴涼。典輿以下分察。有女史二人。



 司苑、典苑、掌苑各二人,掌園苑蒔植蔬果。典苑以下分察之。果熟,進御。有女史二人。



 司燈、典燈、掌燈各二人,掌門閤燈燭。晝漏盡一刻,典燈以下分察。有女史二人。



 △尚功局



 尚功二人,掌女功之程,總司制、司珍、司彩、司計。



 司制、典制、掌制各二人,掌供御衣服裁縫。有女史二人。



 司珍、典珍、掌珍各二人,掌珠珍、錢貨。有女史六人。



 司彩、典彩、掌彩各二人,掌綿彩、縑帛、絲枲。有賜用,則旬別案記。有女史二人。



 司計、典計、掌計各二人,給衣服、飲食、薪炭。有女史二人。



 宮正一人,正五品;司正二人,正六品;典正二人,正七品。宮正掌戒令、糾禁、謫罰之事。宮人不供職者,司正以牒取裁,小事決罰,大事秦聞。有女史四人。阿監、副監,視七品。



 ○太子內官



 良娣二人,正三品;良媛六人,正四品;承徽十人,正五品;昭訓十六人,正七品;奉儀二十四人,正九品。



 司閨二人,從六品;三司皆如之。掌導引妃及宮人名簿,總掌正、掌書、掌筵。



 掌正三人,從八品,九掌皆如之。掌文書出入、管鑰、糾察推罰。有女史三人。



 掌書三人,掌符契、經籍、宣傳、啟奏、教學、稟賜、紙筆。有女史三人。



 掌筵三人,掌幄帟、床褥、幾案、輿繖、汛掃、鋪設。



 司則二人,掌禮儀參見,總掌嚴、掌縫、掌藏。



 掌嚴三人,掌首飾、衣服、巾櫛、膏沐、服玩、仗衛。有女史三人。



 掌縫三人,掌裁紉、織績。有女史三人。



 掌藏三人,掌財貨、珠寶、縑彩。



 司饌二人,掌進食先嘗,總掌食、掌醫、掌園。有女史四人。



 掌食三人,掌膳羞、酒醴、燈燭、薪炭、器皿。有女史四人。



 掌醫三人,掌方藥、優樂。有女史二人。



 掌園三人,掌種植蔬果。有女史二人。



\end{pinyinscope}