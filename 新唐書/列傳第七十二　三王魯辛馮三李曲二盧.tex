\article{列傳第七十二 三王魯辛馮三李曲二盧}

\begin{pinyinscope}

 王思禮,高麗人,入居營州。父為朔方軍將。思禮習戰鬥,從王忠嗣至河西,與哥舒翰同籍麾下。翰為隴右節度使所謂產婆術,即通過雙方一問一答,反復詰難,從個別求得,思禮與中郎將周佖事翰,以功授右衛將軍、關西兵馬使。從討九曲,後期當斬,臨刑,翰釋之,思禮徐曰:「死固分也,何復貸為?」諸將壯之。天寶十三載,吐谷渾蘇毘王款附,詔翰至磨環川應接,思禮墜馬,蹇甚。翰謂監軍李文宜曰:「思禮跛足,尚欲何之?」俄加金城郡太守。



 安祿山反,翰為元帥,奏思禮赴軍,玄宗曰:「河、隴精銳,悉在潼關,吐蕃有釁,唯倚思禮耳。」翰固請,乃兼太常卿,充元帥府馬軍都將,翰委以軍事。密勸翰表誅楊國忠,翰不應;復請以三十騎劫至潼關殺之,翰曰:「此乃吾反,何與祿山事?」



 潼關失守,思禮與呂崇賁、李承光同走行在,肅宗責不堅守,引至纛下將斬之。宰相房琯諫,以為可收後效,遂獨斬承光,赦思禮等。尋副房琯戰便橋,不利,更為關內行營節度、河西隴右伊西行營兵馬使,守武功。賊安守忠來戰,思禮退保扶風。賊分兵略大和關,去鳳翔五十里,李光進戰未利,行在戒嚴,從官潛出其孥,帝使左右巡御史虞候識其姓名,眾稍稍止。命郭子儀以朔方兵擊之。會崔光遠行軍司馬王伯倫、判官李椿以兵二千屯扶風。聞賊已西,欲乘虛襲京師,徑至高陵。賊引軍還擊椿等,椿已至中渭橋,殺守者千人,進攻苑門。伯倫戰死,椿被執。先是,賊餘眾留武功,既傳官軍入京師,乃燒營遁,自是賊不敢西。



 長安平,思禮先入清宮;收東京,戰數有功。遷兵部尚書,封霍國公,食實戶五百。尋兼潞、沁等州節度。乾元元年,總關中、潞州行營兵三萬、騎八千,與子儀圍賊相州,軍潰,惟李光弼、思禮完軍還。尋破史思明別將萬餘眾於直千嶺。光弼徙河陽,代為河東節度副大使。上元元年,加司空。自武德以來,三公不居宰輔,唯思禮而已。二年,薨,贈太尉,謚曰武烈。



 思禮善守計,短攻戰。然持法嚴整,士不敢犯。在太原,器甲完精,儲粟至百萬斛云。



 魯炅,幽人。長七尺餘,略通書史。以廕補左羽林長上。隴右節度使哥舒翰引為別奏。顏真卿嘗使隴右,謂翰曰:「君興郎將,總節制,亦嘗得人乎?」炅時立階下,翰指曰:「是當為節度使。」從破石堡城,收河曲,遷左武衛將軍。後復以破吐蕃跳蕩功,除右領軍大將軍。



 安祿山反,拜上洛太守,將行,於帝前畫攻守勢,遷南陽太守,兼守捉防禦使,封金鄉公。尋為山南節度使,以嶺南、黔中、山南東道子弟五萬屯滍水南。賊將武令珣、畢思琛等擊之,眾欲戰,炅不可。賊右趨,乘風縱火,鬱氣奔營,士不可止,負扉走,賊矢如雨,炅與中人薛道挺身走,舉眾沒賊。時嶺南節度使何履光、黔中節度使趙國珍、襄陽節度使徐浩未至,其子弟半在軍,挾金為資糧,至是與械偕棄與山等,賊資以富。



 炅揪散兵保南陽。潼關失守,賊使哥舒翰招下,不從,使武令珣攻之。令珣死,田承嗣繼往。潁川來瑱、襄陽魏仲犀合兵援炅。仲犀弟孟馴兵至明府橋,望賊走。炅城中食盡,米斗五十千,一鼠四百,餓者相枕藉。朝廷遣使者曹日昇宣慰,加炅特進、太僕卿,不得入。日昇請單騎致命,仲犀不可。會顏真卿自河北至,謂曰:「使者不顧死,致天子命,設為賊獲,是亡一使者;脫能入城,則萬心固矣。」中官馮廷環亦曰:「將軍必入,我請以兩騎助。」仲犀益騎凡十輩。賊望見,知皆銳兵,不敢擊,遂入致命,人心益固。日昇復以騎趨襄陽,領兵千,由音聲道運糧餉炅,故炅得與賊相持逾三月。炅被圍凡一年,晝夜戰,人至相食,卒無救。



 至德二載五月,乃率眾突圍走襄陽。承嗣尾擊,炅殊死戰二日,斬獲甚眾,賊引去。俄拜御史大夫、襄鄧十州節度使。亦會二京平,賊走河北。時襄、漢數百里,鄉聚蕩然,舉無樵煙。初,賊欲剽亂江湖,賴炅適扼其沖,故南夏以完。策勛封岐國公,實封二百戶。



 乾元元年,又加淮西節度、鄧州刺史。與九節度圍安慶緒於相州,炅領淮西、襄陽兩鎮步卒萬人、騎三百。明年,與史思明戰安陽,王師不利,炅中流矢,輒奔,諸節度潰去,所過剽奪,而炅軍尤甚。有詔來瑱節度淮西,徙炅鄭陳亳節度使。至新鄭,聞郭子儀整軍屯谷水,李光弼還太原,炅羞惴,仰藥死,年五十七。



 王難得,沂州臨沂人。父思敬,少隸軍,試太子賓客。難得健於武,工騎射。天寶初,為河源軍使。吐蕃贊普子郎支都者,恃趫敏,乘名馬,寶鈿鞍,略陣挑戰,甚閑暇,無敢校者。難得怒,挾矛駷馬馳,支都不暇斗,直斬其首。玄宗壯其果,召見,令殿前乘馬挾矛作刺賊狀,大悅,賜錦袍、金帶。累授金吾將軍。從哥舒翰擊吐蕃,至積石,虜吐谷渾王子悉弄參及悉頰藏而還。復收五橋,拔樹惇城,進白水軍使。收九曲,加特進。



 肅宗在靈武,軍賞乏,難得上家貲助軍,試衛尉卿。俄領興平軍及鳳翔兵馬使,收京師。方戰,麾下士失馬,難得馳救,矢著眉,披膚鄣目,乃拔箭斷膚,殊死前鬥,血衊面不已,帝嘉之。從郭子儀攻相州。累封瑯邪郡公,為英武軍使。寶應二年,卒,贈潞州大都督。



 子子顏,少從父征討,檢校衛尉卿,生莊憲太后。元和元年,憲宗朝太后南宮,乃褒贈思敬為司徒,難得太尉,子顏太師。唯子顏子用及封。



 用字師柔。拜太子詹事,才三月,封太原郡公,掌廄苑。累遷檢校左散騎常侍,兼右金吾大將軍。謙畏無過。卒,贈工部尚書。



 辛云京,蘭州金城人,客籍京兆,世為將家。云京有膽決,以禽生斬馘常冠軍,積功遷特進、太常卿。史思明屯相州,云京以銳兵四千襲滏陽,追破其眾,至浪井。錄多,授開府儀同三司,加代州都督、鎮北兵馬使。



 太原軍亂,帝惡鄧景山繩下無漸,以云京性沉毅,故授太原尹,進封金城郡王。云京治謹於法,下有犯,雖絲毫比不肯貸,及賞功亦如之,故軍中畏而信。回紇恃舊勛,每入朝,所在暴鈔,至太原,云京以戎狄待之,虜畏不敢惕息。數年,太原大治。加檢校尚書右僕射、同中書門下平章事。



 大歷三年,檢校左僕射。卒,年五十五,代宗為發哀流涕,贈太尉,謚曰忠獻。它日,郭子儀、元載見上,語及云京,帝必泫然。及葬,命中使吊祠,時將相祭者至七十餘幄,喪車移晷乃得去。德宗時,第至德以來將相,云京為次。



 從弟京杲,字京杲。信安王禕節度朔方,京杲與弟旻以策干說,禕評咨加異。後從李光弼出井陘,督趫蕩先驅,戰嘉山尤力,肅宗異之,召見曰:「黥、彭、關、張之流乎!」累遷鴻臚卿,召為英武軍使。代宗立,封肅國公,遷左金吾衛大將軍,進晉昌郡王,歷湖南觀察使,後為工部尚書致仕。硃泚盜京師,以老病不能從,西向慟而卒,贈太子少保。



 旻亦從光弼定恆、趙,後署太原三城使。史思明屯相,軍及滏陽,旻逆擊走之。東都陷,退守河陽,卒於屯。



 雲京曾孫讜,別傳。



 馮河清,京兆人。始隸郭子儀軍,以戰多拜左衛大將軍。後從涇原節度使馬璘,充兵馬使,數以偏師與吐蕃遇,多效級,名聞軍中。



 建中時,節度使姚令言率兵討關東,以河清知留後,幕府殿中侍御史姚況領州;而行師過闕,有急變,德宗走奉天。河清、況聞問,召諸將計事,東向哭,相勵以忠,意象軒毅,眾義其為,無敢異言,即發儲鎧完仗百餘乘獻行在。初,帝之出,六軍倉卒無良兵,士氣沮。及河清輸械至,被堅勒兵,軍聲大振。即拜河清涇原節度使、安定郡王,況行軍司馬。硃泚數遣諜人訹之,河清輒斬以徇。



 興元元年,渾瑊以吐蕃兵敗賊韓旻等,涇人妄傳吐蕃有功,將以叛卒孥與貲歸之,眾大恐,且言:「不殺馮公,吾等無類矣。」田希鑒遂害河清,況挺身還鄉里。



 京師平,贈河清尚書左僕射,拜況太子中舍人。況性簡退,未嘗言功,屬歲兇,奉稍不自給,以饑死。河清再贈太子少傅。



 李芃,字茂初,趙州人。解褐上邽主簿。嚴武為京兆尹,薦補長安尉。李勉觀察江西,表署判官。



 永泰初,宣饒劇賊方清、陳莊西絕江,劫商旅為亂,支黨槃結。芃請以秋浦置州,扼衿要,使不得合從。勉是其計,奏以宣之秋浦青陽、饒之至德置池州。即詔芃行州事。後魏少游代勉,表署都團練副使,攝江州刺史。以母喪解。勉之節度永平,復闢幕府。會李靈耀反,署芃兼亳州防禦使,護陳、潁饟道,便軍興。



 德宗立,授河陽三城鎮遏使。糧貲善者,必先以給士,士悅之。達練事宜,嚴備常若有敵。未幾,拜節度使,以東畿汜水等五縣隸屬。與馬燧等破田悅洹水上,以功檢校兵部尚書,實封百戶。進圍悅,悅將符璘以騎五百降,芃大開壁門納之。



 興元初,檢校尚書右僕射。以疾將請老,謂所親曰:「歲方旱蝗,上厭征伐,天下城壘堅,戈鋋利,然務以力勝,其可盡乎?救敝者莫若德,方鎮之臣宜先退讓,死權錮祿,吾敢哉!言而不踐,非吾志也。」固求罷,歸東都。卒,年六十四,贈太子太保。



 李叔明,字晉,閬州新政人。本鮮于氏,世為右族。兄仲通,字向,天寶末為京兆尹、劍南節度使。兄弟皆涉學,輕財務施。叔明擢明經,為楊國忠劍南判官。乾元中,除司勛員外郎,副漢中王瑀使回紇,回紇遇瑀慢,叔明讓曰:「大國通好,使賢王持節。可汗,唐之婿,恃功而倨,可乎?」可汗為加禮。復命,遷司門郎中。



 東都平,拜洛陽令,招徠遺民,號能吏。擢商州刺史、上津轉運使。遷京兆尹,長安歌曰:「前尹赫赫,具瞻允若;後尹熙熙,具瞻允斯。」久之,以疾辭,除太子右庶子。崔旰擾成都,出為卬州刺史。旰入朝,即拜東川節度使、遂州刺史,徙治梓州。



 大歷末,或言叔明本嚴氏,少孤,養外家,冒鮮于姓,請還宗。詔可。叔明初不知,意醜之,表乞宗姓,列屬籍,代宗從之。



 建中初,吐蕃襲火井,掠龍州,陷扶、文、遠三州。叔明分五將邀擊,走之,以功加檢校戶部尚書。梁崇義阻命,詔引兵下峽,戰荊門,敗其眾,襄州平,遷檢校尚書左僕射。德宗幸興元,出家貲助軍,悉衣幣獻宮掖,加太子太傅,封薊國公。初,東川承兵盜,鄉邑雕破,叔明治之二十年,撫接有方,華裔遂安。後朝京師,以病足,賜錦輦,令宦士肩舁以見,拜尚書右僕射。乞骸骨,改太子太傅致仕。貞元三年,卒,謚曰襄。始,叔明與仲通俱尹京兆,及兼秩御史中丞,並節制劍南,又與子昇俱兼大夫,蜀人推為盛門。



 叔明素惡道、佛之弊,上言曰:「佛,空寂無為者也;道,清虛寡欲者也。今迷其內而飾其外,使農夫工女墮業以避役,故農桑不勸,兵賦日屈,國用軍儲為斁耗。臣請本道定寺為三等,觀為二等,上寺留僧二十一,上觀道士十四,每等降殺以七,皆擇有行者,餘還為民。」德宗善之,以為不止本道,可為天下法,乃下尚書省雜議。於是都官員外郎彭偃曰:「王者之政,變人心為上,因人心次之,不變不因為下。今道士有名亡實,俗鮮歸重,於亂政輕;僧尼帑穢,皆天下不逞,茍避征役,於亂人甚。今叔明之請雖善,然未能變人心,亦非因人心者。夫天生蒸人,必將有職;游閑浮食,王制所禁。故賢者受爵祿,不肖者出租稅,古常道也。今僧、道士不耕而食,不織而衣,一僧衣食,歲無慮三萬,五夫所不能致。舉一僧以計天下,其費不貲。臣謂僧、道士年未滿五十,可令歲輸絹四,尼及女官輸絹二,雜役與民同之;過五十者免。凡人年五十,嗜欲已衰,況有戒法以檢其性情哉!」刑部員外郎裴伯言曰:「衣者,蠶桑也;食者,耕農也;男女者,繼祖之重也。而二教悉禁,國家著令,又從而助之,是以夷狄不經法反制中夏禮義之俗也。傳曰:『女子十四有為人母之道,四十九絕生育之理;男子十六有為人父之道,六十四絕陽化之理。』臣請僧、道士一切限年六十四以上,尼、女官四十九以上,許終身在道,餘悉還為編人,官為計口授地,收廢寺觀以為廬舍。」議雖上,罷之。



 子昇,以少卿從德宗梁州。叔明嚴敕以死報,故昇有功,擢禁軍將軍。貞元初,遷太子詹事。坐郜國公主,貶羅州別駕。



 叔明素豪侈,在蜀殖財,廣第舍田產。歿數年,子孫驕縱,貲產皆盡。世言多藏者以叔明為鑒云。



 曲環,陜州安邑人,客隴右。少喜兵法,資勇敢,善騎射。天寶中,從哥舒翰討吐蕃,拔石堡,取黃河九曲洪濟等城,授果毅別將。安祿山反,從魯炅守鄧州,與賊武令珣戰尤力,加左清道率。從李抱玉屯河陽。又自將兵守澤州,破賊銳將安曉,拜羽林將軍。與諸將討史朝義,平河北,累轉金吾大將軍。



 大歷中,戍隴州,數破吐蕃,以功兼太常卿。德宗初,虜寇劍南,詔環以邠、隴兵五千馳救,收七盤城、威武軍、維茂等州,虜破走,威名大振,加太子賓客,賜名馬。豫討涇州劉文喜,遷開府儀同三司,封晉昌郡王,邠隴兵馬使。時李納逼徐州,環與劉玄佐救之,敗其眾,功最。建中三年,擢邠隴行營節度使。



 李希烈陷汴州,環守寧陵,戰陳州,斬賊三萬五千級,禽其將翟崇暉,進檢校工部尚書,兼陳州刺史。希烈平,改陳許節度,賜封三百戶。二州比為寇沖,民苦剽鹵,客他縣。環勤身節用,寬賦斂,簡條教,不三歲,歸者繦系。訓農治兵,穀食豐衍。轉檢校尚書左僕射。貞元十五年,卒,年七十四,贈司空。



 王虔休,字君佐,汝州梁人。少涉學,有材武,以信義為鄉黨畏慕。大歷中,刺史李深署為裨將。澤潞李抱真聞其名,厚以幣招之,授兵馬使。抱真討河北,戰雙罔、臨洺,虔休以多擢步軍都虞候,封同昌郡王,實封五十戶。抱真卒,元仲經等謀樹其子緘,一軍思亂,虔休正色語眾曰:「軍,王軍;州,王土也。帥亡當稟天子,何雲云有妄謀?」眾服其言,得不亂。德宗嘉之,以邕王為昭義節度大使,擢虔休潞州左司馬,領留後。本名延貴,至是賜名。號令撫循,軍中大治。



 初,抱真之喪,軍司馬元誼據洺州叛,虔休遣將李廷芝討之,戰長橋,斬級數百;次雞澤,又破之。守戍皆奔魏博,即決水灌城,將壞,遣掌書記盧頊入見誼,陳利害。誼請朝,即以頊為洺州別駕,使守洺。誼出,亦奔魏。



 治潞二歲,遷昭義節度使,檢校工部尚書。始,屬城州縣守宰多署它職,不親政,故治茍簡。虔休悉增俸稟,遣就部,人以妥安。卒,年六十三,贈尚書左僕射,謚曰敬。



 虔休性恪敏,節用度,既沒,所部帑廩皆可支數歲。嘗得太常樂家劉玠撰《繼天誕聖樂》,因帝誕日以獻。其樂,以宮為均,示五聲有君也;以土為德,本五運在中也;奏二十五疊,取二十四氣而成一歲;奏十六節,象元、凱登庸於朝雲。後《中和樂》本於此。



 子麗成等十人,並補太學生。



 盧群,字載初,系出範陽。少學於垂山,淮南陳少游聞其名,奏署幕府,已而薦諸朝。李希烈反,以監察御史為江西行營糧料使。嗣曹王皋節度江西,奏為判官。皋徙荊襄,皆從其府,以勁正聞。入為侍御史。郭子儀家與嬖人張昆弟訟財不平,又言嬖人宅匿珍寶。德宗促按之。群奏言:「子儀有大勛德,今所訟皆其家事,且嬖人宅,子儀昔畀之,非子弟所宜言,請赦勿問。」從之。人謂群識大體。



 累遷兵部郎中。淮西吳少誠擅決司洧水溉田,使者止之,不奉詔。命群臨詰,少誠曰:「是於人有利。」群曰:「臣道貴順,恭恪所以為順也。專命廢順,雖利何有?且怠於事上者,固不能責其下矣。」少誠聽命。群又為陳古今成敗事,逆順禍福皆有效,所以感動之,少誠竦然。既置酒,與賦詩,又歌以慰之。少誠感悅,不敢桀。以奉使稱旨,遷檢校秘書監、鄭滑節度行軍司馬。姚南仲入朝,即以群代節度。群嘗客於鄭,質良田以耕。至是則出券貸直,以田歸其人。卒,年五十九,贈工部尚書。



 李元素,字大樸,邢國公密裔孫,仕為御史。東都留守杜亞惡大將令狐運,會盜劫輸絹於洛北,運適與其下畋近郊,亞疑而訊之。幕府穆員、張弘靖按鞫無狀,亞怒,更以愛將武金掠服之,死者甚眾。亞請斥運丑土,詔監察御史楊寧覆驗,事皆不讎。亞怒,劾寧罔上,寧抵罪。又自以不失盜為功,因必其怒,傅致而周內之,若不可翻者。德宗信不疑,宰相難之。詔元素與刑部員外郎崔從質、大理司直盧士瞻馳按,亞迎,以獄告。元素徐察其冤,悉縱所囚以還。亞大驚,復劾元素失有罪。比元素還,帝已怒,奏獄未畢,帝曰:「出。」元素曰:「臣言有所未盡。」帝曰:「第去。」元素曰:「臣以御史按獄,知冤不得盡辭,是無容復見陛下。」帝意解,即道運冤狀。帝感寤曰:「非卿,孰能辨之?」然運猶以擅捕人得罪,流歸州,死於貶。武金流建州。後歲餘,齊抗得真盜,繇是天下重之。



 遷給事中。後美官缺,咸冀元素得其處。會鄭滑節度使盧群卒,拜元素檢校工部尚書節度其軍,治有異績。元和初,召為御史大夫。大夫,自貞元後難其人不補,而元素以夙望召拜,中外企聽風採。既而一不建為,容容持祿,內望作宰相。久之不見用,則謝賓客曰:「無以官散外我。」見屬吏輒先拜,人人失望。李錡反,拜浙西節度使。數月還,為國子祭酒,進戶部尚書、判度支。



 元素少孤,奉長姊謹悌,及沒,悲鯁成疾,因辭職屏居。其妻,石泉公王方慶之孫。前妻子皆不肖,而元素溺姬侍,王不見答。元素久疾,益昏惑,遂出之。王訴諸朝,詔免元素官,且令畀王貲五百萬。卒,贈陜州大都督。



 盧士玫者,山東人。以文儒進,端厚無競。為吏部員外郎,善於職。再遷知京兆尹。劉總入朝,與士玫故內姻,乃請析瀛、鄚兩州,用士玫為觀察使。詔可。俄而幽州亂,硃克融襲之,朝廷欲重其任,就加節度使。士玫空家貲助軍,然部卒多家幽州,陰導克融入,故士玫闔府皆見囚幽州。天子赦克融,得還。以太子賓客分司東都,徐虢州刺史,復為賓客。卒,贈工部尚書。



\end{pinyinscope}