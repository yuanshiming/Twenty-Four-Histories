\article{列傳第三十四 崔楊竇宗祝王}

\begin{pinyinscope}

 崔義玄,貝州武城人。隋大業亂,往見李密,密不用。河內賊黃君漢為密守柏崖,義玄見群鼠度河,槊刃有華文,曰:「此王敦亡兆也。」因說君漢以城歸,乃拜君漢懷州刺史、行軍總管,以義玄為司馬。王世充將高毘寇河內,義玄擊走之,多下屯堡。君漢以所掠子女金帛分之,拒不受。以功封清丘縣公。太宗討世充,數用其謀。東都平,轉隰州都督府長史。貞觀初,歷左司郎中,兼韓王府長史,與王友孟神慶志趣不同,而俱以介直任。



 永徽中,累遷婺州刺史。時睦州女子陳碩真舉兵反。始,碩真自言仙去,與鄉鄰辭訣,或告其詐,已而捕得,詔釋不問。於是姻家章叔胤妄言碩真自天還,化為男子,能役使鬼物,轉相熒惑,用是能幻眾。自稱文佳皇帝,以叔胤為僕射,破睦州,攻歙,殘之,分遣其黨圍婺州。義玄發兵拒之,其徒爭言碩真有神靈,犯其兵輒滅宗,眾兇懼不肯用。司功參軍崔玄籍曰:「仗順起兵,猶無成;此乃妖人,勢不持久。」義玄乃署玄籍先鋒,而自統眾繼之。至下淮戍,擒其諜數十人。有星墜賊營,義玄曰:「賊必亡。」詰朝奮擊,左右有以盾鄣者,義玄曰:「刺史而有避邪,誰肯死?」敕去之。由是眾為用,斬首數百級,降其眾萬餘。賊平,拜御史大夫。



 義玄有章句學,先儒疑繆,或音故不通者,輒採諸家,條分節解,能是正之。高宗詔與博士討論《五經》義。



 武氏為皇后,義玄贊帝決,又以後旨按長孫無忌等誅之。終蒲州刺史,年七十一。贈幽州都督,謚曰貞。後持政,贈揚州大都督,賜其家實封戶二百。



 子神基襲爵。神基,長壽中,為司賓卿、同鳳閣鸞臺平章事。為酷吏所構,流嶺南。中宗初,稍用為大理卿。



 弟神慶,舉明經,武后時,累遷萊州刺史。入朝,待制億歲殿,奏事稱旨。後以歷官有佳政,且其父於己有功,擢拜並州長史,謂曰:「並州,朕鄉里,宿兵多,前長史皆尚書為之,今授卿,宜知所以委重者。」乃親為按行圖,謀日而遣。神慶始至,有詔改錢幣法,州縣布下,俄而物價踴昂,百賈驚擾,神慶質其非於朝,果豪猾妄為之。後喜,下制褒美。初,州隔汾為東、西二城,神慶跨水聯堞,合而一之,省防禦兵歲數千。神基既下獄,馳赴都告變,得召見,後出具獄示之,神慶為申理,得減死,然用是貶歙州司馬。



 長安中,累轉禮部侍郎,數上疏陳時政。轉太子右庶子,封魏縣子。是時,突厥使者入見,皇太子應朝,有司移文東宮召太子。神慶諫曰:「五品以上佩龜者,蓋防徵召之詐,內出龜以合之,況太子乎?古者召太子用玉契,此誠重慎防萌之意,不可不察。凡慮事於未萌之前,故長無悔吝之咎。今太子與陛下異宮,非朝朔望而別喚者,請降墨敕玉契。」詔可。尋詔與詹事祝欽明更日侍讀東宮。歷司刑卿,劾張昌宗獄,頗闊略不盡。神龍初,昌宗伏誅,坐流欽州,卒。五王得罪,緣昌宗被流者皆詔原雪,贈神慶幽州都督。



 神慶子琳,明政事,開元中,與高仲舒同為中書舍人。侍中宋璟親禮之,每所訪逮,嘗曰:「古事問仲舒,今事問琳,尚何疑?」累遷太子少保。天寶二年卒,秘書監潘肅聞之,泫然曰:「古遺愛也!」琳長子儼,諫議大夫。



 其群從數十人,自興寧里謁大明宮,冠蓋騶哄相望。每歲時宴於家,以一榻置笏,猶重積其上。琳與弟太子詹事珪、光祿卿瑤俱列棨戟,世號「三戟崔家」。開元、天寶間,中外宗屬無緦麻喪。初,玄宗每命相,皆先書其名,一日書琳等名,覆以金甌,會太子入,帝謂曰:「此宰相名,若自意之,誰乎?即中,且賜酒。」太子曰:「非崔琳、盧從願乎?」帝曰:「然。」賜太子酒。時兩人有宰相望,帝欲相之數矣,以族大,恐附離者眾,卒不用。



 楊再思,鄭州原武人,第明經,為人佞而智。初,調玄武尉,使至京師,舍逆旅,有盜竊其衣囊,再思遇之,盜窘謝。再思曰:「而苦貧,故至此。囊中檄無所事,幸留,它物可持去。」」初不為人言,但假貸以還。累遷天官員外郎,歷左肅政御史中丞。延載初,擢鸞臺侍郎、同鳳閣鸞臺平章事,加兼左肅政御史大夫,封鄭縣侯,遷內史。



 居宰相十餘年,阿匼取容,無所薦達。人主所不喜,毀之;所善,譽之。畏慎足恭,未嘗忤物。或曰:「公位尊,何自屈折?」答曰:「世路孔艱,直者先禍。不爾,豈全吾軀?」於時水沴,閉坊門以禳。再思入朝,有車陷於濘,叱牛不前,恚曰:「癡宰相不能和陰陽,而閉坊門,遣我艱於行!」再思遣吏謂曰:「汝牛自弱,不得獨責宰相。」



 張昌宗坐事,司刑少卿桓彥範劾免其官,昌宗訴諸朝,武后意申釋之,問宰相:「昌宗於國有功乎?」再思曰:「昌宗為陛下治丹,餌而愈,此為有功。」後悅,昌宗還官。自是天下貴彥範,賤再思。左補闕戴令言賦「兩腳狐」以譏之,再思怒,謫令言為長社令,士愈蚩噪。



 易之兄司禮少卿同休,請公卿宴其寺,酒酣,戲曰:「公面似高麗。」再思欣然,翦穀綴巾上,反披紫袍,為高麗舞,舉動合節,滿坐鄙笑。昌宗以姿貌人幸,再思每曰:「人言六郎似蓮華,非也;正謂蓮華似六郎耳。」其巧諛無恥類如此。俄檢校右庶子。



 中宗立,拜戶部尚書、同中書門下三品、京師留守,封弘農郡公,加兼揚州長史,檢校中書令。改侍中,鄭國公,賜實封戶三百,為順天皇后奉冊使。武三思誣陷王同晈,再思與李嶠、韋巨源按獄,希意抵同晈死,眾以為冤。復拜中書令,監修國史。遷尚書右僕射,仍同三品。卒,贈特進、並州大都督,陪葬乾陵,謚曰恭。



 弟季昭,中茂才第,為殿中侍御史。武后誅駙馬都尉薛紹,紹兄顗為齊州刺史,命季昭按之,不得反狀,後怒,放於沙州。赦還,為懷州司馬。



 竇懷貞,字從一,左相德玄子。少詭激,衣服羸儉,不為輿馬豪侈事。仕累清河令,有治狀。後遷越州都督、揚州長史。



 神龍中,進左御史大夫兼檢校雍州長史。會歲除,中宗夜宴近臣,謂曰:「聞卿喪妻,今欲繼室可乎?」懷貞唯唯。俄而禁中寶扇鄣衛,有衣翟衣出者,已乃韋后乳媼王,所謂莒國夫人者,故蠻婢也。懷貞納之不辭。又避後先諱,而以字稱。世謂媼婿為阿赩,懷貞每謁見奏請,輒自署「皇后阿赩」,而人或謂為「國赩」,軒然不訴,以自媚於後。時政令多門,赤尉由墨制授御史者眾,或戲曰:「尉入臺多,而縣辦否?」對曰:「辦於異日。」問其故,答曰:「佳吏在,僥幸去,故辦。」聞者皆笑。又附宗楚客、安樂公主等以取貴位,為素議所斥,名稱盡矣。韋後敗,斬妻獻其首,貶濠州司馬,再徙益州長史,乃復故名。



 景雲初,以殿中監召,閱月遷左御史大夫、同中書門下平章事,封中山縣公。再遷侍中。方太平公主干政,懷貞傾己附離,日視事退,輒詣主第,刺取所欲。睿宗為金仙、玉真二公主營觀,費鉅萬,諫者交疏不止,唯懷貞勸成之,躬護役作。族弟維鍌諫曰:「公位上袞,當思獻可替否輔天子,而計校瓦木,雜廁工匠間,使海內何所瞻仰乎?」不答,督繕益急。時語曰:「前作後國赩,後為主邑丞。」言事公主如邑官屬也。在位半歲,無所事,帝引見承天門,切責之。俄與李日知、郭元振、張說皆罷。為左御史大夫。於時,歲犯左執法,術家又言懷貞且有禍,大懼,表請為安國寺奴,不許。逾年,復同中書門下三品,兼太子詹事,監修國史。又以尚書右僕射兼御史大夫,軍國重事宜共平章。玄宗受內禪,進左僕射,封魏國公。與太平公主謀逆,既敗,投水死,追戮其尸,改姓毒氏。然生平所得俸祿,悉散親族無留蓄,敗時,家惟粗米數石而已。



 性諂詐,善諧結權貴,宦者用事,尤所畏奉,或見無須者,誤為之禮。監察御史魏傳弓嫉中人輔信義,欲劾奏其奸,懷貞曰:「是安樂所信任者,奈何繩之?」傳弓曰:「王綱壞矣,正坐此屬。今日殺之,明日誅,無所悔!」懷貞猶固止之。傳弓者,鉅鹿人,忠謇士也,終司農丞。



 懷貞從子兢,字思慎,舉明經,為英王府參軍、尚乘直長。調郪令,修郵舍道路,設冠婚喪紀法,百姓德之。



 宗楚客,字叔敖,其先南陽人。曾祖丕,後梁南弘農太守,梁亡入隋,居河東之汾陰,故為蒲州人。父岌,仕魏王泰府,與謝偃等撰《括地志》。



 楚客,武后從姊子,長六尺八寸,明皙美須髯。及進士第,累遷戶部侍郎。兄秦客,垂拱中,勸武后革命,進為內史,而弟晉卿典羽林兵。後兄弟並坐奸贓流嶺外。歲餘,秦客死,而楚客等還。俄檢校夏官侍郎、同鳳閣鸞臺平章事。與武懿宗不協,會賜將作材營第,僭侈過度,為懿宗所劾,自文昌左丞貶播州司馬,晉卿流峰州。稍為豫州長史,遷少府少監、岐陜二州刺史。久之,復以夏官侍郎同鳳閣鸞臺平章事。坐聘邵王妓,貶原州都督。



 神龍初,為太僕卿、郢國公。武三思引為兵部尚書,以晉卿為將作大匠。節愍太子敗,逃於鄠,被殺,殊其首祭三思等柩,楚客請之也。俄同中書門下三品。韋后、安樂公主親賴之,與紀處訥為黨,世號「宗紀」。



 景龍二年,詔突厥娑葛為金河郡王,而其部闕啜忠節賂楚客等罷之,娑葛怨,將兵患邊。監察御史崔琬廷奏:「楚客、處訥專威福,有無君心,納境外交,為國取怨;晉卿專徇贓私,驕恣跋扈。並請收付獄,三司推鞫。」故事,大臣為御史對仗彈劾,必趨出,立朝堂待罪。楚客乃厲色大言:「性忠鯁,為琬誣詆。」中宗不能窮也,詔琬與楚客、處訥約兄弟兩解之,故世謂帝為「和事天子」。尋遷中書令。韋氏敗,與晉卿同誅。



 楚客性明達。武后時,降突厥沓實力吐敦者,部落在平夏。會邊書至,言吐敦反,楚客為兵部員外郎,後召問方略,對曰:「吐敦者,臣昔與之言,其為人忠義和厚,且國家與有恩必不反。其兄之子默子者,狡悍,與吐敦不和,今言叛,疑默子為之,然無能為。」俄而夏州表默子劫部落北奔,為州兵及吐敦所擒。後張仁亶請築三城,議者或不同,獨楚客言:「萬世利也。」然冒於權利,嘗諷右補闕趙廷禧陳符命以媚帝,曰:「唐有天下,當百世繼周,陛下承母禪,周、唐一統,其符兆有八:天皇再以陛下為周王,是在唐興周,則天立陛下為皇太子,是在周興唐,一也;天后立文王廟,二也;唐同泰《洛水圖》云:『永昌帝業』,三也;讖曰:『百代不移宗』,四也;孔子曰:『百世繼周』,五也;《桑條韋歌》,應二聖在位九十八年,而子孫相承九十八世,六也;乃二月慶雲五色,天應以和,七也;去六月九日,內出瑞蒜,八也。起則天為一世,聖朝為二世,後子孫相承九十八,其數正滿百世,唐之歷乃三千餘年。」帝大喜,擢延禧諫議大夫。識者以楚客等欺神誣君,且有大咎。又嘗密語其黨曰:「始,吾在卑位,尤愛宰相;及居之,又思天子,南面一日足矣。」雖外附韋氏,而內畜逆謀,故卒以敗。



 晉卿髭貌雄偉,聲如鐘。雖不學,然性倜儻。垂拱後,武後任之,宮苑、閑廄、內外眾作無不總。開中嶽,造明堂,鑄九鼎,有力焉。



 紀處訥者,秦州上邽人。為人魁岸,髭長數尺。其妻武三思婦之姊,縱使通三思,繇是款暱,進為太府卿。神龍元年夏,大旱,穀價騰踴,中宗召問所以救人者。三思知之,陰諷太史迦葉志忠奏「是夜攝提入太微,近帝坐,此天子與大臣接,有納忠之符」。帝信之,下詔褒美,賜處訥衣一副、彩六十段。與楚客並同三品,進侍中。後伏誅。



 祝欽明字文思,京兆始平人。父綝,字叔良,少通經,頗著書質諸家疑異;門人張後胤既顯宦,薦於朝,詔對策高第,終無極尉。



 欽明擢明經,為東臺典儀。永淳、天授間,又中英才傑出、業奧《六經》等科,拜著作郎,為太子率更令。中宗在東宮,欽明兼侍讀,授太子經,兼弘文館學士。中宗復位,擢國子祭酒、同中書門下三品。進禮部尚書,封魯國公,食實封戶三百。桓彥範、崔玄、袁恕巳,敬暉等皆從受《周官》大義,朝廷尊之。以匿親忌日,為御史中丞蕭至忠所劾,貶申州刺史。入為國子祭酒。



 景龍三年,天子將郊,欽明與國子司業郭山惲陰迎韋後意,謬立議曰:



 《周官》天神曰祀,地祗曰祭,宗廟曰享。《大宗伯》曰:「祀大神,祭大祗,享大鬼,王有故不預,則攝而薦。追師掌後首服,以待祭祀。內司服掌後六服,祭祀則供。又九嬪,凡大祭祀,後裸獻則贊瑤爵。然則後當助天子祀天神、祭地只。鄭玄稱:闕狄,後助王祭群小祀服。小祀尚助,況天地哉?闕狄之上,禕、示俞、狄,三服皆以助祭,知禕衣助大祀也。王之祭服二:曰先王兗冕,先公冕。故後助祭,亦以禕衣祭先王,示俞狄祭先公。不言助祭天地,舉此以明彼,反三隅也。《春秋外傳》:禘郊,天子親射其牛,王後親春其粢。」世婦詔後之禮事,不專主宗廟。《祭統》曰:「祭也者,必夫婦親之,所以備內外之官。」哀公問孔子曰:「冕而親迎,不已重乎?」答曰:「合二姓之好,以繼先聖之後,以為天地宗廟社稷主,君何謂已重焉?」則知後宜助祭。臣請因經誼,制儀典。



 帝雖不睿,猶疑之,召禮官質問。於是太常博士唐紹、蔣欽緒對:「欽明所引,皆宗廟禮,非祭天地者。周、隋而上,無皇后助祭事。」帝令宰相參訂,紹、欽緒又引博士彭景直共議曰:



 《周官》所云祀、祭、享,皆互言。《典瑞》:「兩圭以祀地。」《司幾筵》:「設祀先王昨席。」《內宗》:「掌宗廟祭祀。傳曰:「聖人為能饗帝。」「春秋祭祀,以時思之。」此祀天稱享,享廟稱祭也。禮家凡稱大祭祀,不獨主天。《爵人》:「大祭祀,與量人受舉斝之卒爵。」祭天不祼,則九嬪贊瑤爵,容廟稱大祭祀也。欽明據《大宗伯》之職,以謂後有祭天地之禮。按經:「凡祀大神、祭大祗、享大鬼,帥執事而卜宿,視滌濯,涖玉鬯,省牲鑊,奉玉盥,制大號。若王不與祭祀,則攝位。」自凡而推,兼言王祭天地宗廟也。下言:「凡大祭祀,王後不與,則攝而薦。」直王後祭廟一凡耳。若當助祭天地,應不列重凡。且內宗、外宗所掌,皆佐王后廟薦,無佐祭天地語。有如助祭天地,誰當贊佐者?是則攝薦為宗廟明甚。內司服掌後祭服,無祭天服。禮家說曰:「後不助祭天地五岳,故無具服。」又言:「後有五輅,以重翟從祭先王先公,以厭翟從饗諸侯,以安車朝夕見王,以翟車採桑,以輦車游宴。」按此,後無祭天車明甚。然後助王祭天地,古無聞焉。



 時左僕射韋巨源助後掎掣帝,奪政事,即傳欽明議,帝果用其言,以皇后為亞獻。取大臣李嶠等女為齋娘,奉豆籩。禮成,詔齋娘有夫者悉進官。



 初,後屬婚,上食禁中,帝與群臣宴,欽明自言能《八風舞》,帝許之。欽明體肥醜,據地搖頭睆目,左右顧眄,帝大笑。吏部侍郎盧藏用嘆曰:「是舉《五經》掃地矣!」景雲初,侍御史倪若水劾奏:「欽明、山惲等腐儒無行,以諂佞亂常改作,百王所傳,一朝惰放。今聖德中興,不宜使小人在朝,請斥遠之,以肅具臣。」乃貶欽明饒州刺史,山惲括州刺史。欽明於《五經》為該淹,自見坐不孝免,無以澡祓,乃阿附韋氏,圖再用,又坐是見逐,諸儒共羞之。後徙洪州都督,入為崇文館學士,卒。



 贊曰:「欽明以經授中宗,為朝大儒,乃詭聖僻說,引艷妻郊見上帝,腥德播聞,享胙不終。蓋與少正卯順非而澤,莊周以詩書破塚者同科。獨保腰領死家簀,寧不幸邪!後之托儒為奸者,可少戒云。



 山惲者,河東人。善治《禮》。景龍中,累遷國子司業。帝暱宴近臣及修文學士,詔遍為伎。工部尚書張錫為《談容娘舞》,將作大匠宗晉卿為《渾脫舞》,左衛將軍張洽為《黃麞舞》,給事中李行言歌《賀車西河曲》,餘臣各有所陳,皆鄙黷;而出惲奏:「我所習,惟知誦詩。」乃誦《鹿鳴》、《蟋蟀》二篇,未畢,中書令李嶠以其近規諷,止之。帝嘉其直,下詔褒咨,賜服一稱。其後與欽明僻論阿世,不能終其守。久之,復拜國子司業。



 王璵者,方慶六世孫,少為禮家學。玄宗在位久,推崇老子道,好神仙事,廣修祠祭,靡神不祈。璵上言,請築壇東郊祀青帝,天子入其言,擢太常博士、侍御史,為祠祭使。璵專以祠解中帝意,有所禳祓,大抵類巫覡。漢以來葬喪皆有瘞錢,後世里俗稍以紙寓錢為鬼事,至是璵乃用之。



 肅宗立,累遷太常卿,又以祠禱見寵。乾元三年,拜蒲同絳等州節度使,俄以中書侍郎同中書門下平章事。時大兵後,天下願治,璵望輕,無它才,不為士議諧可,既驟得政,中外悵駭。乃奏置太一壇,勸帝身見九宮祠。帝由是專意,它議不能奪。帝嘗不豫,太卜建言祟在山川。璵遣女巫乘傳分禱天下名山大川,巫皆盛服,中人護領,所至乾托州縣,賂遺狼藉。時有一巫美而蠱,以惡少年數十自隨,尤憸狡不法。馳入黃州,刺史左震晨至館請事,門鐍不啟。震怒,破鐍入,取巫斬廷下,悉誅所從少年,籍其贓得十餘萬,因遣還中人。既以聞,璵不能詰,帝亦不加罪。明年,罷璵為刑部尚書,又出為淮南節度使,猶兼祠祭使,徙浙東。召入,再遷太子少師。卒,贈開府儀同三司,謚曰簡懷。



 始,璵托鬼神致位將相,當時以左道進者紛紛出焉。李國禎者,以術士顯,廣德初,建言「唐家仙系,宜崇表福區,招致神靈,請度昭應南山作天華上宮、露臺、大地婆父祠,並三皇、道君、太古天皇、中古伏羲、女媧等各為堂皇,給百戶掃除」。又即義扶穀故湫祠龍,置房宇。有詔從之,乃除地課工,方歲饑,人不堪命。昭應令梁鎮上疏切諫,以為有七不可:「天地之神,推之尊極者,掃地可祭,精意可享。今廢先王之典,為人祈福,福未至而人已困。又違神虐人,何從而致福邪?宗廟月無三祭,此不宜然。婆父之鄙語,不經見,若為地建祖廟,上天必貽向背之責。夫湫者,龍所托耳,今湫竭已久,龍安所存?不宜崇去龍之穴,破生人之產。若三皇、五帝、道君等,兩京及所都各有宮廟,春秋彞饗,此復營造,是謂瀆神。夫休咎豐兇本於五事,不在山川百神明矣。」即劾國禎等「動眾則得人,興工則獲利,祭祀則受胙,主執則市權,營罔天聽,負抱粢糈,道路相望,無時而息,人神胥怨,災孽並至。臣昨受命,有所安輯,陛下許以權宜,今所興造臣謹以便宜悉停」。帝從之。鎮忼慨有名士也,仕至司門郎中。璵曾孫摶,別傳。



\end{pinyinscope}