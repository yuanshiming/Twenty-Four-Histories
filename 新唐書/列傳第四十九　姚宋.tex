\article{列傳第四十九 姚宋}

\begin{pinyinscope}

 姚崇,字元之,陜州硤石人。父懿,字善懿,貞觀中對立①矛盾雙方相互排斥、相互鬥爭的性質。是無條件,為巂州都督,贈幽州大都督,謚文獻。



 崇少倜儻,尚氣節,長乃好學。仕為孝敬挽郎,舉下筆成章,授濮州司倉參軍,五遷夏官郎中。契丹擾河北,兵檄叢進,崇奏決若流,武後賢之,即拜侍郎。後嘗語左右:「往周興、來俊臣等數治詔獄,朝臣相逮引,一切承反。朕意其枉,更畏近臣臨問,皆得其手牒不冤,朕無所疑,即可其奏。自俊臣等誅,遂無反者,然則向論死得無冤邪?」崇曰:「自垂拱後,被告者類自誣。當是時,以告言為功,故天下號曰『羅織』,甚於漢之鉤黨。雖陛下使近臣覆訊,彼尚不自保,敢一搖手以悖酷吏意哉!且被問不承,則重罹其慘,如張虔勖、李安靜等皆是也。今賴天之靈,發寤陛下,兇豎殲夷,朝廷乂安,臣以一門百口保內外官無復反者。陛下以告牒置弗推,後若反有端,臣請坐知而不告。」後悅曰:「前宰相務順可,陷我為淫刑主,聞公之言,乃得朕心。」賜銀千兩。



 聖歷三年,進同鳳閣鸞臺平章事。遷鳳閣侍郎,俄兼相王府長史,以母老納政歸侍,乃詔以相王府長史侍疾,月餘,復兼夏官尚書、同鳳閣鸞臺三品。崇建言:「臣事相王,而夏官本兵,臣非惜死,恐不益王。」乃詔改春官。張易之私有請於崇,崇不納,易之譖於後,降司僕卿,猶同鳳閣鸞臺三品。出為靈武道大總管。



 張柬之等謀誅二張,崇適自屯所還,遂參計議。以功封梁縣侯,實封二百戶。後遷上陽宮,中宗率百官起居,王公更相慶,崇獨流涕。柬之等曰:「今豈涕泣時邪?恐公禍由此始。」崇曰:「比與討逆,不足以語功,然事天后久,違舊主而泣,人臣終節也,由此獲罪甘心焉。」俄為亳州刺史。後五王被害,而崇獨免。歷宋、常、越、許四州。睿宗立,拜兵部尚書、同中書門下三品,進中書令。



 玄宗在東宮,太平公主干政,宋王成器等分典閑廄、禁兵。崇與宋璟建請主就東都,出諸王為刺史,以壹人心。帝以謂主,主怒。太子懼,上疏以崇等槊間王室,請加罪,貶為申州刺史。移徐、潞二州,遷揚州長史。政條簡肅,人為紀德於碑。徙同州刺史。



 先天二年,玄宗講武新豐。故事,天子行幸,牧守在三百里者,得詣行在。時帝亦密召崇,崇至,帝方獵渭濱,即召見,帝曰:「公知獵乎?」對曰:「少所習也。臣年二十,居廣成澤,以呼鷹逐獸為樂。張憬藏謂臣當位王佐,無自棄,故折節讀書,遂待罪將相。然少為獵師,老而猶能。」帝悅,與俱馳逐,緩速如旨,帝歡甚。既罷,乃咨天下事,袞袞不知倦。帝曰:「卿宜遂相朕。」崇知帝大度,銳於治,乃先設事以堅帝意,即陽不謝,帝怪之。崇因跪奏:「臣願以十事聞,陛下度不可行,臣敢辭。」帝曰:「試為朕言之。」崇曰:「垂拱以來,以峻法繩下;臣願政先仁恕,可乎?朝廷覆師青海,未有牽復之悔;臣願不倖邊功,可乎?比來壬佞冒觸憲網,皆得以寵自解;臣願法行自近,可乎?後氏臨朝,喉舌之任出閹人之口;臣願宦豎不與政,可乎?戚里貢獻以自媚於上,公卿方鎮浸亦為之;臣願租賦外一絕之,可乎?外戚貴主更相用事,班序荒雜;臣請戚屬不任臺省,可乎?先朝褻狎大臣,虧君臣之嚴;臣願陛下接之以禮,可乎?燕欽融、韋月將以忠被罪,自是諍臣沮折;臣願群臣皆得批逆鱗,犯忌諱,可乎?武后造福先寺,上皇造金仙、玉真二觀,費鉅百萬;臣請絕道佛營造,可乎?漢以祿、莽、閻、梁亂天下,國家為甚;臣願推此鑒戒為萬代法,可乎?」帝曰:「朕能行之。」崇乃頓首謝。翌日,拜兵部尚書、同中書門下三品。封梁國公。遷紫微令。固辭實封,乃停舊食,賜新封百戶。



 中宗時,近戚奏度僧尼,溫戶強丁因避賦役。至是,崇建言:「佛不在外,悟之於心。行事利益,使蒼生安穩,是謂佛理。烏用奸人以汨真教?」帝善之,詔天下汰僧偽濫,發而農者餘萬二千人。



 崇嘗於帝前序次郎吏,帝左右顧,不主其語。崇懼,再三言之,卒不答,崇趨出。內侍高力士曰:「陛下新即位,宜與大臣裁可否。今崇亟言,陛下不應,非虛懷納誨者。」帝曰:「我任崇以政,大事吾當與決,至用郎吏,崇顧不能而重煩我邪?」崇聞乃安。由是進賢退不肖而天下治。



 開元四年,山東大蝗,民祭且拜,坐視食苗不敢捕。崇奏:「《詩》云:『秉彼蟊賊,付畀炎火。』漢光武詔曰:『勉順時政,勸督農桑。去彼螟域,以及蟊賊。』此除蝗誼也。且蝗畏人易驅,又田皆有主,使自救其地,必不憚勸。請夜設火,坎其旁,且焚且瘞,蝗乃可盡。古有討除不勝者,特人不用命耳。」乃出御史為捕蝗使,分道殺蝗。汴州刺史倪若水上言:「除天災者當以德,昔劉聰除蝗不克而害愈甚。」拒御史不應命。崇移書誚之曰:「聰偽主,德不勝祆,今祆不勝德。古者良守,蝗避其境,謂修德可免,彼將無德致然乎?今坐視食苗,忍而不救,因以無年,刺史其謂何?」若水懼,乃縱捕,得蝗十四萬石。時議者喧嘩,帝疑,復以問崇,對曰:「庸儒泥文不知變。事固有違經而合道,反道而適權者。昔魏世山東蝗,小忍不除,至人相食;後奏有蝗,草木皆盡,牛馬至相啖毛。今飛蝗所在充滿,加復蕃息,且河南、河北家無宿藏,一不獲則流離,安危系之。且討蝗縱不能盡,不愈於養以遺患乎?」帝然之。黃門監盧懷慎曰:「凡天災,安可以人力制也!且殺蟲多,必戾和氣。願公思之。」崇曰:「昔楚王吞蛭而厥疾瘳,叔敖斷虵福乃降。今蝗幸可驅,若縱之,穀且盡,如百姓何?殺蟲救人,禍歸於崇,不以諉公也!」蝗害訖息。


於是,帝方躬萬機,朝夕詢逮,它宰相畏帝威決,皆謙憚,唯獨崇佐裁決,故得專任。崇第賒僻,因近舍客廬。會懷慎卒,崇病
 \gezhu{
  疒占}
 移告,凡大政事,帝必令源乾曜就咨焉。乾曜所奏善,帝則曰:「是必崇畫之。」有不合,則曰:「胡不問崇?」乾曜謝其未也,乃已。帝欲崇自近,詔徙寓四方館,日遣問食飲起居,高醫、尚食踵道。崇以館局華大,不敢居。帝使語崇曰:「恨不處禁中,此何避?」久之,紫微史趙誨受夷人賕,當死。崇素親倚,署奏營減,帝不悅。時曲赦京師,惟誨不原。崇惶懼,上還宰政,引宋璟代,乃以開府儀同三司罷政事。



 帝將幸東都,而太廟屋自壞,帝問宰相,宋璟、蘇頲同對曰:「三年之喪未終,不可以行幸。壞壓之變,天所以示教戒,陛下宜停東巡,修德以答至譴。」帝以問崇,對曰:「臣聞隋取苻堅故殿以營廟,而唐因之。且山有朽壞乃崩,況木積年而木自當蠹乎。但壞與行會,不緣行而壞。且陛下以關中無年,輪餉告勞,因以幸東都,所以為人不為己也。百司已戒,供擬既具,請車駕如行期。舊廟難復完,盡奉神主舍太極殿?更作新廟,申誠奉,大孝之德也。」帝曰:「卿言正契朕意。」賜絹二百匹,詔所司如崇言,天子遂東。因詔五日一參,入閣供奉。



 八年,授太子少保,以疾不拜。明年卒,年七十二。贈揚州大都督,謚曰文獻。十七年,追贈太子太保。



 崇析貲產,令諸子各有定分。治令曰:



 比見達宦之裔多貧困,至銖尺是競,無論曲直,均受絜,詆。田宅水磑既共有之,至相推倚以頓廢。陸賈、石苞,古達者也,亦先有定分,以絕後爭。



 昔楊震、趙明、盧植、張奐咸以薄葬,知真識去身,貴速朽耳。夫厚葬之家流於俗,以奢靡為孝,令死者戮尸暴骸,可不痛哉!死者無知,自同糞土,豈煩奢葬;使其有知,神不在柩,何用破貲徇侈乎?吾亡,斂以常服,四時衣各一稱。性不喜冠衣,毋以入墓。紫衣玉帶,足便於體。



 今之佛經,羅什所譯,姚興與之對翻,而興命不延,國亦隨滅。梁武帝身為寺奴,齊胡太后以六宮入道,皆亡國殄家。近孝和皇帝發使贖生,太平公主、武三思等度人造寺,身嬰夷戮,為天下笑。五帝之時,父不喪子,兄不哭弟,致仁壽,無兇短也。下逮三王,國祚延久,其臣則彭祖、老聃皆得長齡,此時無佛,豈抄經鑄像力邪?緣死喪造經像,以為追福。夫死者生之常,古所不免,彼經與像何所施為?兒曹慎不得為此!崇尤長吏道,處決無淹思。三為宰相,常兼兵部,故屯戊斥候、士馬儲械,無不諳記。玄宗初立,賓禮大臣故老,雅尊遇崇,每見便殿,必為之興,去輒臨軒以送,它相莫如也。時承權戚干政之後,綱紀大壞,先天末,宰相至十七人,臺省要職不可數。崇常先有司罷冗職,修制度,擇百官各當其材,請無廣釋道,無數移吏。繇是天子責成於下,而權歸於上矣。



 然資權譎。始為同州,張說以素憾,諷趙彥昭劾崇。及當國,說懼,潛詣岐王申款。崇它日朝,眾趨出,崇曳踵為有疾狀,帝召問之,對曰:「臣損足。」曰:「無甚痛乎?」曰:「臣心有憂,痛不在足。」問以故,曰:「岐王陛下愛弟,張說輔臣,而密乘車出入王家,恐為所誤,故憂之。」於是出說相州。魏知古,崇所引,及同列,稍輕之,出攝吏部尚書,知東都選,知古憾焉。時崇二子在洛,通賓客饋遺,憑舊請托。知古歸,悉以聞。他日,帝召崇曰:「卿子才乎,皆安在?」崇揣知帝意,曰:「臣二子分司東都,其為人多欲而寡慎,是必嘗以事干魏知古。」帝始以崇私其子,或為隱,微以言動之。及聞,乃大喜,問:「安從得之?」對曰:「知古,臣所薦也,臣子必謂其見德而請之。」帝於是愛崇不私而薄知古,欲斥之。崇曰:「臣子無狀,橈陛下法,而逐知古,外必謂陛下私臣。」乃止,然卒罷為工部尚書。



 崇始名元崇,以與突厥叱剌同名,武后時以字行;至開元世,避帝號,更以今名。三子:彞、異、弈,皆至卿、刺史。



 弈少修謹。始,崇欲使不越官次而習知吏道,故自右千牛進至太子舍人,皆平遷。開元中,有事五陵,有司以鷹犬從,弈曰:「非禮也。」奏罷之。請治劇,為睢陽太守,召授太僕卿。後為尚書右丞。子閎,居右相牛仙客幕府。仙客病甚,閎強使薦弈及盧奐為宰相,仙客妻以聞,閎坐死,弈貶永陽太守,卒。



 曾孫合、勖。合,元和中進士及第,調武功尉,善詩,世號姚武功者。遷監察御史,累轉給事中。奉先、馮翊二縣民訴牛羊使奪其田,詔美原主簿硃儔覆按,猥以田歸使,合劾發其私,以地還民。歷陜虢觀察使,終秘書監。



 勖字斯勤。長慶初擢進士第,數為使府表闢,進監察御史,佐鹽鐵使務。累遷諫議大夫,更湖、常二州刺史。為宰相李德裕厚善。及德裕為令狐綯等譖逐,擿索支黨,無敢通勞問;既海上,家無資,病無湯劑,勖數饋餉候問,不傅時為厚薄。終夔王傅。自作壽藏於萬安山南原崇塋之旁,署兆曰「寂居穴」,墳曰「復真堂」,中叕刂土為床曰「化臺」,而刻石告後世。



 宋璟,邢州南和人。七世祖弁為元魏吏部尚書。璟耿介有大節,好學,工文辭,舉進士中第。調上黨尉,為監察御史,遷鳳閣舍人。居官鯁正,武後高其才。張易之誣御史大夫魏元忠有不臣語,引張說為驗,將廷辯,說惶遽,璟謂說曰:「名義至重,不可陷正人以求茍免。緣此受謫,芬香多矣。若不測者,吾且叩閣救,將與子偕死。」說感其言,以實對,元忠免死。



 璟後遷左臺御史中丞,會飛書告張昌宗引相工觀吉兇者,璟請窮治,後曰:「易之等已自言於朕。」璟曰:「謀反無容以首原,請下吏明國法。易之等貴寵,臣言之且有禍,然激於義,雖死不悔。」後不懌,姚遽傳詔令出,璟曰:「今親奉德音,不煩宰相擅宣王命。」後意解,許收易之等就獄。俄詔原之,敕二張詣璟謝,璟不見,曰:「公事公言之,若私見,法無私也。」顧左右嘆曰:「吾悔不先碎豎子首而令亂國經。」嘗宴朝堂,二張列卿三品,璟階六品,居下坐。易之諂事璟,虛位揖曰:「公第一人,何下坐?」璟曰:「才劣品卑,卿謂第一何邪?」是時朝廷以易之等內寵,不名其官,呼易之「五郎」,昌宗「六郎」。鄭善果謂璟曰:「公奈何謂五郎為卿?」璟曰:「以官正當為卿。君非其家奴,何郎之云?」會有喪,告滿入朝,公卿以次謁,通禮意。易之等後至,促步前,璟舉笏卻揖唯唯。故積怨,常欲中傷,後知之,得免。然以數忤旨,詔按獄揚州,璟奏:「按州縣,才監察御史職耳。」又詔按幽州都督屈突仲翔,辭曰:「御史中丞非大事不出使。仲翔罪止贓,今使臣往,此必有危臣者。」既而詔副李嶠使隴、蜀,璟復言:「隴右無變,臣以中丞副李嶠,非朝廷故事。」終辭。易之初冀璟出則劾奏誅之,計不行,乃伺璟〗家婚禮,將遣客刺殺之。有告璟者,璟乘庳車舍他所,刺不得發。俄二張死,乃免。



 神龍初,為吏部侍郎。中宗嘉其直,令兼諫議大夫、內供奉,仗下與言得失。遷黃門侍郎。武三思怙烝寵,數有請於璟。璟厲答曰:「今復子明闢,王宜以侯就第,安得尚干朝政,獨不見產、祿事乎?」後韋月將告三思亂宮掖,三思諷有司論大逆不道,帝詔殊死,璟請付獄按罪,帝怒,岸巾出側門,謂璟曰:「朕謂已誅矣,尚何請?」璟曰:「人言後私三思,陛下不問即斬之,臣恐有竊議者,請按而後刑。」帝愈怒。璟曰:「請先誅臣,不然,終不奉詔。」帝乃流月將嶺南。會還京師,詔璟權檢校並州長史,未行,又檢校貝州刺史。時河北水,歲大饑,三思使斂封租,璟拒不與,故為所擠。歷杭、相二州,政清毅,吏下無敢犯者。遷洛州長史。



 睿宗立,以吏部尚書、同中書門下三品。玄宗在東宮,兼右庶子。先是崔湜、鄭愔典選,為戚近干奪,至迎用二歲闕,猶不能給,更置比冬選,流品淆並,璟與侍郎李乂、盧從願澄革之,銓總平允。



 太平公主不利東宮,嘗駐輦光範門,伺執政以諷。璟曰:「太子有大功,宗朝社稷主也,安得異議?」乃與姚崇白奏出公主諸王於外,帝不能用。貶楚州刺史,歷兗冀魏三州、河北按察使,進幽州都督,以國子祭酒留守東都,遷雍州長史。



 玄宗開元初,以雍州為京兆府,復為尹。進御史大夫,坐小累為睦州刺史,徙廣州都督。廣人以竹茅茨屋,多火。璟教之陶瓦築堵,列邸肆,越俗始知棟宇利而無患災。召拜刑部尚書。四年,遷吏部兼侍中。



 帝幸東都,次崤谷,馳道隘,稽擁車騎,帝命黜河南尹李朝隱、知頓使王怡等官。璟曰:「陛下富春秋,今始巡守,以道不治而罪二臣,繇此相飭,後有受其蔽者。」帝遽命舍之。璟謝曰:「陛下向以怒責之,以臣言免之,是過歸於上而恩在下。姑聽待罪於朝,然後詔還其職,進退得矣。」帝善之。累封廣平郡公。廣人為璟立遺愛頌,璟上言:「頌所以傳德載功也。臣之治不足紀,廣人以臣當國,故為溢辭,徒成諂諛者。欲厘正之,請自臣始。」有詔許停。



 帝嘗命璟與蘇頲制皇子名與公主號,遂差次所封,且詔別擇一美稱及佳邑封上。璟奏言:「七子均養,詩人所稱。今若同等別封,或母寵子愛,恐傷跂鳩之平。昔袁盎引卻慎夫人席,文帝納之,夫人亦不為嫌,以其得長久計也。臣不敢別封。」帝嘆重其賢。



 皇后父王仁籞卒,將葬,用昭成皇后家竇孝諶故事,墳高五丈一尺。璟等請如著令,帝已然可,明日,復詔如孝諶者。璟還詔曰:「儉,德之恭;侈,惡之大也。僭禮厚葬,前世所誡,故古墓而不墳。人子於哀迷則未遑以禮自制,故聖人制齊、斬、緦、免,衣衾棺郭,各有度數。雖有賢者,斷其私懷。眾皆務奢,獨能以儉,所謂至德要道者。中宮若謂孝諶逾制,初無非者,一切之令固不足以法。貞觀時嫁長樂公主,魏徵謂不可加長公主,太宗欣納,而文德皇后降使厚謝。韋庶人追王其父,擅作,酆陵,而禍不旋踵。國家知人情無窮,故為制度,不因人以搖動,不變法以愛憎。比來人間競務靡葬,今以後父重戚,不憂乏用,高塚大寢,不畏無人,百事官給,一朝可就,而區區屢聞者,欲成朝廷之政、中宮之美爾。儻中宮情不可奪,請準令一品陪陵,墳四丈,差合所宜。」帝曰:「朕常欲正身紀綱天下,於後容有私邪?然人所難言,公等乃能之。」即可其奏。又遣使賚彩絹四百匹。



 會日食,帝素服俟變,錄囚多所貸遣,賑恤災患,罷不急之務。璟曰:「陛下降德音,恤人隱,末宥輕系,惟流、死不免,此古所以慎赦也。恐議者直以月蝕修刑,日蝕修德,或言分野之變,冀有揣合。臣以謂君子道長,小人道銷。止女謁,放讒夫,此所謂修德也。囹圄不擾,兵甲不瀆,官不苛治,軍不輕進,此所謂修刑也。陛下常以為念,雖有虧食,將轉而為福,又何患乎?且君子恥言浮於行,願勸天以誠,無事空文。」帝嘉納。後以開府儀同三司罷政事。



 京兆人權梁山謀逆,敕河南尹王怡馳傳往按。牢械充滿,久未決,乃命璟為京留守,覆其獄。初,梁山詭稱婚集,多假貸,吏欲並坐貸人。璟曰:「婚禮借索大同,而狂謀率然,非所防億。使知而不假,是與為反。貸者弗知,何罪之云?」平縱數百人。



 十二年,東巡泰山,璟復為留守。帝將發,謂曰:「卿,國元老,別方歷時,宜有嘉謀以遺朕。」璟因一二極言。手制答曰:「所進當書之坐右,出入觀省,以誡終身。」賜賚優渥,進兼吏部尚書。十七年。為尚書右丞相,而張說為左丞相,源乾曜為太子少傅,同日拜。有詔太官設饌,太常奏樂,會百官尚書省東堂。帝賦三傑詩,自寫以賜。二十年,請致仕,許之,仍賜全祿。退居洛。乘輿東幸,璟謁道左。詔榮王勞問,別遣使賜藥餌。二十五年卒,年七十五,贈太尉,謚文貞。



 璟風度凝遠,人莫涯其量。始,自廣州入朝,帝遣內侍楊思勖驛迓之。未嘗交一言。思勖自以將軍貴幸,訴之帝,帝益嗟重。璟為宰相,務清政刑,使官人皆任職。聖歷後,突厥默啜負其強,數窺邊,侵九姓拔曳固,負勝輕出,為其狙擊斬之,入蕃使郝靈人傳其首京師。靈佺自謂還必厚見賞。璟顧天子方少,恐後干寵蹈利者誇威武,為國生事,故抑之,逾年,才授右武衛郎將,靈佺恚憤不食死。張嘉貞後為相,閱堂案,見其危言切議,未嘗不失聲嘆息。六子:升、尚、渾、恕、華、衡。



 升,太僕少卿。尚,漢東太守。渾,與李林甫善,歷諫議大夫、平原太守、御史中丞、東京採訪使。在平原,暴斂求進,至重取民一年庸、租。使東畿,薛稷甥女鄭寡而美,渾使南尉楊朝宗聘而己納之,薦朝宗為赤尉。恕,以都官郎中為劍南採訪判官,數貪縱不法,陰養刺客。天寶中,渾、恕、尚並以贓敗,渾流高要,恕流海康,尚貶臨海長史。華、衡亦皆坐貪得罪。廣德中,渾起為太子諭德。物議穢薄之,流死江嶺。昆弟皆荒飲俳嬉,而衡最險悖,廣平之風衰焉。



 贊曰:姚崇以十事要說天子而後輔政,顧不偉哉,而舊史不傳。觀開元初皆已施行,信不誣已。宋璟剛正又過於崇,玄宗素所尊憚,常屈意聽納。故唐史臣稱崇善應變以成天下之務,璟善守文以持天下之正。二人道不同,同歸於治,此天所以佐唐使中興也。嗚呼!崇勸天子不求邊功,璟不肯賞邊臣,而天寶之亂,卒蹈其害,可謂先見矣。然唐三百年,輔弼者不為少,獨前稱房、杜,後稱姚、宋,何哉?君臣之遇合,蓋難矣夫!



\end{pinyinscope}