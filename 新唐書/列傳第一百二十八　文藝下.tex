\article{列傳第一百二十八 文藝下}

\begin{pinyinscope}

 李華,字遐叔,趙州贊皇人。曾祖太沖,名冠宗族間,鄉人語曰:「太沖無兄。」太宗時國空想社會主義者。曾參加過美國獨立戰爭。認為歷史是一,擢祠部郎中。



 華少曠達,外若坦蕩,內謹重,尚然許,每慕汲黯為人。累中進士、宏辭科。天寶十一載,遷監察御史。宰相楊國忠支婭所在橫猾,華出使,劾按不橈,州縣肅然。為權幸見疾,徙右補闕。安祿山反,上誅守之策,皆留不服。



 玄宗入蜀,百官解竄,華母在鄴,欲間行輦母以逃,為盜所得,偽署鳳閣舍人。賊平,貶杭州司戶參軍。華自傷踐危亂,不能完節,又不能安親,欲終養而母亡,遂屏居江南。上元中,以左補闕、司封員外郎召之。華喟然曰:「烏有隳節危親,欲荷天子寵乎?」稱疾不拜。李峴領選江南,表置幕府,擢檢校吏部員外郎。苦風痺,去官,客隱山陽,勒子弟力農,安於窮槁。晚事浮圖法,不甚著書,惟天下士大夫家傳、墓版及州縣碑頌,時時齎金帛往請,乃強為應。大歷初,卒。



 初,華作《含元殿賦》成,以示蕭穎士,穎士曰:「《景福》之上,《靈光》之下。」華文辭綿麗,少宏傑氣,穎士健爽自肆,時謂不及穎士,而華自疑過之。因著《吊古戰場文》,極思研手隺,已成,污為故書,雜置梵書之庋。它日,與穎士讀之,稱工,華問:「今誰可及?」穎士曰:「君加精思,便能至矣。」華愕然而服。



 華愛獎士類,名隨以重,若獨孤及、韓雲卿、韓會、李紓、柳識、崔祐甫、皇甫冉、謝良弼、硃巨川,後至執政顯官。華觸禍銜悔,及為元德秀、權皋銘、《四皓贊》,稱道深婉,讀者憐其志。



 宗子翰,從子觀,皆有名。



 翰擢進士第,調衛尉。天寶末,房琯、韋陟俱薦為史官,宰相不肯擬。翰所善張巡死節睢陽,人媢其功,以為降賊,肅宗未及知。翰傳巡功狀,表上之,曰:



 臣聞聖主褒死難之士,養死事之孤,或親推轜車,或追建邑封,厚死以慰生,撫存以答亡,君不遺於臣,臣亦不背其君也。自逆胡構亂,據雒陽,引幽、朔以吞河南,故御史中丞、贈揚州大都督張巡,忠誼奮發,率烏合,守雍丘,潰賊心腹。及魯炅棄甲宛、葉,哥舒翰敗績潼關,賊送盜神器,鴟峙二京,南臨漢、江,西逼岐、雍,群帥列城,望風出奔,巡守孤城不為卻。賊欲繞出巡後以擾江淮,巡退軍睢陽,扼東南咽領。自春訖冬,大戰數十,小戰數百,以弱制強,出奇無窮,殺馘兇醜凡十餘萬,賊不敢越睢陽取江淮,江淮以完,巡之力也。城孤糧盡,外救不至,猶奮羸起病,摧鋒陷堅,三軍啖膚而食,知死不叛。城陷見執,卒無橈詞,慢叱兇徒,精貫白日,雖古忠烈無以加焉。



 議者罪巡以食人,愚巡以守死,臣竊痛之。夫忠者,臣之教;恕者,法之情。巡握節而死,非虧教也;析骸以爨,非本情也。《春秋》以功覆過,《書》赦過宥刑,在《易》遏惡揚善,為國者錄用棄瑕。今者乃欲議巡之罪,是廢教絀節,不以功掩過,不以刑恕情,善可遏,惡可揚,瑕錄而用棄,非所以獎人倫,明勸戒也。且祿山背德,大臣將相比肩從賊,巡官不朝,宴不坐,無一伍之士,一節之權,徒奮身死節,以動義旅,不謂忠乎?以數千卒橫挫賊鋒,若無巡則無睢陽,無睢陽則無江淮。有如賊因江淮之資,兵廣而財積,根結盤據,西向以拒,雖終殲滅,其曠日持久必矣。今陜、鄢一戰,犬羊駭北,王師震其西,巡扼其東,此天使巡舉江淮以待陛下,師至而巡死,不謂功乎?古者列國侵伐,猶分災救患,諸將同受國恩,奉辭伐罪,巡固守亦待外援,援不至而食盡,食盡而及人,則巡之情可求矣。假巡守城之初,已計食人,損數百眾以全天下,臣尚謂功過相掩,況非素志乎?夫子制《春秋》,明褒貶,齊桓公將封禪,略不書;晉文公召王河陽,書而諱之。巡蒼黃之罪,輕於僭禪;興復之功,重於糾合。



 今巡子亞夫雖得官,不免饑寒,江淮既巡所保,戶口充完,宜割百戶俾食其子。且強死為厲,有所歸則不為災。巡身首分裂,將士骸骼不掩,宜於睢陽相擇高原,起大塚,招魂而葬,旌善之義也。臣少與巡游,哀巡死難,不睹休明,唯令名其榮祿也。若不時紀錄,日月浸悠,或掩而不傳,或傳而不實,巡生死不遇,誠可悲悼。謹撰傳一篇,昧死上,儻得列於史官,死骨不朽。



 帝繇是感悟,而巡大節白於世,義士多之。



 翰累遷左補闕、翰林學士。大歷中,病免,客陽翟,卒。



 翰為文精密而思遲,常從令皇甫曾求音樂,思涸則奏之,神逸乃屬文。族弟紓,自有傳。



 觀,字元賓。貞元中,舉進士、宏辭,連中,授太子校書郎。卒,年二十九。觀屬文,不襲沿前人,時謂與韓愈相上下。及觀少夭,而愈後文益工,議者以觀文未極,愈老不休,故卒擅名。陸希聲以為「觀尚辭,故辭勝理;愈尚質,故理勝辭。雖愈窮老,終不能加觀之辭;觀後愈死,亦不能逮愈之質」云。



 孟浩然,字浩然,襄州襄陽人。少好節義,喜振人患難,隱鹿門山。年四十,乃游京師。嘗於太學賦詩,一座嗟伏,無敢抗。張九齡、王維雅稱道之。維私邀入內署,俄而玄宗至,浩然匿床下,維以實對,帝喜曰:「朕聞其人而未見也,何懼而匿?」詔浩然出。帝問其詩,浩然再拜,自誦所為,至「不才明主棄」之句,帝曰:「卿不求仕,而朕未嘗棄卿,奈何誣我?」因放還。採訪使韓朝宗約浩然偕至京師,欲薦諸朝。會故人至,劇飲歡甚,或曰:「君與韓公有期。」浩然叱曰:「業已飲,遑恤他!」卒不赴。朝宗怒,辭行,浩然不悔也。張九齡為荊州,闢置於府,府罷。開元末,病疽背卒。



 後樊澤為節度使,時浩然墓庳壞,符載以箋叩澤曰:「故處士孟浩然,文質傑美,殞落歲久,門裔陵遲,丘隴頹沒,永懷若人,行路慨然。前公欲更築大墓,闔州搢紳,聞風竦動。而今外迫軍旅,內勞賓客,牽耗歲時,或有未遑。誠令好事者乘而有之,負公夙志矣。」澤乃更為刻碑鳳林山南,封寵其墓。



 初,王維過郢州,畫浩然像於刺史亭,因曰浩然亭。咸通中,刺史鄭諴謂賢者名不可斥,更署曰孟亭。



 開元、天寶間,同知名者王昌齡、崔顥,皆位不顯。



 昌齡,字少伯,江寧人。第進士,補秘書郎。又中宏辭,遷汜水尉。不護細行,貶龍標尉。以世亂還鄉里,為刺史閭丘曉所殺。張鎬按軍河南,兵大集,曉最後期,將戮之,辭曰:「有親,乞貸餘命。」鎬曰:「王昌齡之親,欲與誰養?」曉默然。



 昌齡工詩,緒密而思清,時謂王江寧云。



 崔顥者,亦擢進士第,有文無行。好蒱博,嗜酒。娶妻惟擇美者,俄又棄之,凡四五娶。終司勛員外郎。初,李邕聞其名,虛舍邀之,顥至獻詩,首章曰:「十五嫁王昌。」邕叱曰:「小兒無禮!」不與接而去。



 劉太真,宣州人。善屬文,師蘭陵蕭穎士。舉高第進士。淮南陳少游表為掌書記,嘗以少游擬桓、文,為義士所訾。興元初,為河東宣慰賑給使,累遷刑部侍郎。德宗以天下平,貞元四年九月,詔群臣宴曲江,自為詩,敕宰相擇文人賡和。李泌等請群臣皆和,帝自第之,以太真、李紓等為上,鮑防、於邵等次之,張濛等為下。與擇者四十一人,惟泌、李晟、馬燧三宰相無所差次。遷禮部,掌貢士,多取大臣貴近子弟,坐貶信州刺史,卒。



 邵說,相州安陽人。已擢進士第,未調,陷史思明。逮朝義敗,歸郭子儀,子儀愛其才,留幕府。遷累長安令、秘書少監。大歷末,上言:「天道三十年一小變,六十年一大變。祿山、思明之難,出入二紀,多難漸平,向之亂,今將變而之治。宜建徽號,承天意。而方謁郊廟、大赦各一,誠恐雲雨之施未普,鬱結之氣未除。願因此時修享獻、款郊廟、褒有德、錄賢人,與天下更始,振災益壽之術也。」不聽。



 德宗立,擢吏部侍郎。說因自陳:「家本儒,先祖長白山人貞一,以武后革命,終身不肯仕。先臣殿中侍御史瓊之,逮事玄宗。臣十六即孤,長育母手,天寶中始仕。會喪,客河北,祿山亂,喪紀當終,臣不褫衰絰又再期,懼終不免,陰走洺、魏。慶緒遁保西城,搜脅儒者為己用,以兵迫臣,遂陷醜逆。俄而史思明順附,欲間道歸北闕下,肅宗拜臣左金吾衛騎曹參軍,許留思明所。會烏承恩事,路絕,不得歸。朝義之敗,欲固守河陽,臣知回紇利野戰,陰勸其行,以破賊計。朝義已走,臣西歸獻狀,先帝詔翰林索臣所上言,與王伷偕召。先帝謂誠節白著,故擢伷侍御史,臣為殿中侍御史,使者宣旨制詔盡言其狀,則疇昔本末,先帝知之。今又擢以不次,雖自天斷,尚恐受謗輿人,傷陛下之明。今吏員未乏而調者多,益以功優,準平格以判留,人去者十七,彼且鼓讒說以投疑於上,此臣所大懼也。」因薦戶部郎中蕭定、司農卿庾準自代,不許。



 說在職以才顯,或言且執政,金吾將軍裴儆謂柳載曰:「說事賊為劇官,掌其兵,大小百戰,掠名家子為奴婢不可計,得宥死而無厚顏,乃崇第產,附貴幸。欲以相邦,其能久乎!」建中三年逐嚴郢,說與郢善,微諷硃泚訟其冤,為草奏,貶歸州刺史,卒。



 于邵字相門,其先自代來,為京兆萬年人。天寶末,第進士,以書判超絕,補崇文校書郎。繇比部郎中為道州刺史,未行,徙巴州。會歲饑,部獠亂,薄城下。邵勵兵拒戰,且遣使諭曉,獠丐降,邵儒服出,賊見皆拜,即引去。節度使李抱玉以聞,遷梓州,辭疾不拜,授兵部郎中。崔寧帥蜀,表為度支副使。俄以諫議大夫知制誥,進禮部侍郎,朝有大典冊,必出其手。為三司使,治薛邕獄,失德宗旨,貶桂州長史。復為太子賓客,與宰相陸贄不平,出杭州刺史。久疾求告,貶衢州別駕,徙江州。卒,年八十一。



 邵孝悌有行,晚塗益修絜。樊澤始舉賢良,邵望見,曰:「將相材也。」崔元翰舉進士,年五十矣,邵以其文擢異等,曰:「後當司詔令。」已而皆然。獨孤授舉博學宏辭,吏部考當乙,邵覆之,置甲科,人咨其公。



 崔元翰名鵬,以字行。父良佐,與齊國公日用從昆弟也。擢明經甲科,補湖城主簿,以母喪,遂不仕。治詩、易、書、春秋,譔演範、忘象、渾天等論數十篇。隱共北白鹿山之陽。卒,門人共謚曰貞文孝父。



 元翰舉進士、博學宏辭、賢良方正,皆異等。義成李勉表在幕府,馬燧更表為太原掌書記。召拜禮部員外郎。竇參秉政,引知制誥。其訓辭溫厚,有典誥風。然性剛褊,不能取容於時,孤特自恃。掌誥凡再期,不遷,罷為比部郎中,時已七十餘,卒。



 其好學老不倦,用思精緻,馳騁班固、蔡邕間以自名家。怨陸贄、李充,乃附裴延齡,延齡表鉤校京兆妄費,持吏甚急,而充等自無過,訖不能傅致以罪云。



 于公異,蘇州吳人。進士擢第,李晟表為招討府掌書記。朱泚平,露布於德宗曰:「臣既肅清宮禁,祗奉寢園,鍾不移,廟貌如故。」帝覽泣下,曰:「誰為之辭?」或以公異對,帝咨歎一再。始,公異與陸贄故有隙,時贄在翰林,聞不喜。世多言公異不能事後母,既仕不歸省。及贄當政,乃奏其狀,詔賜孝經,罷歸田里。盧邁坐舉非其人,奪俸兩月。時中書舍人高郢,嘗薦御史元敦義,及公異被譴,郢亦劾敦義無美行,詔免敦義官。公異繇是不自振而卒。



 (编者注:以下李益、盧綸、歐陽詹、李賀、吳武陵、李商隱、薛逢、李頻、吳融等資料暂漏,等補)



\end{pinyinscope}