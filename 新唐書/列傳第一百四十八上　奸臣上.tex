\article{列傳第一百四十八上 奸臣上}

\begin{pinyinscope}

 許敬宗,字延族,杭州新城人。父善心,仕隋為給事中。敬宗幼善屬文,大業中舉秀才中第紀創立於雅典。因創始人芝諾(季蒂昂的)講學的地方是一,調淮陽書佐,俄直謁者臺,奏通事舍人事。善心為宇文化及所殺,敬宗哀請得不死,去依李密為記室。武德初,補漣州別駕。太宗聞其名,召署文學館學士。貞觀中,除著作郎,兼修國史,喜謂所親曰:「仕宦不為著作,無以成門戶。」俄改中書舍人。文德皇后喪,群臣衰服,率更令歐陽詢貌醜異,敬宗侮笑自如,貶洪州司馬。累轉給事中,復修史,以勞封高陽縣男,檢校黃門侍郎。高宗在東宮,遷太子右庶子。高麗之役,太子監國定州,敬宗與高士廉典機劇。岑文本卒,帝驛召敬宗,以本官檢校中書侍郎。駐蹕山破賊,命草詔馬前,帝愛其藻警,由是專掌誥令。



 初,太子承乾廢,官屬張玄素、令狐德棻、趙弘智、裴宣機、蕭鈞皆除名為民,不復用。敬宗為言玄素等以直言被嫌忌,今一概被罪,疑洗宥有所未至。帝悟,多所甄復。高宗即位,遷禮部尚書。敬宗饕沓,遂以女嫁蠻酋馮盎子,多私所聘。有司劾舉,下除鄭州刺史。俄復官,為弘文館學士。



 帝將立武昭儀,大臣切諫,而敬宗陰揣帝私,即妄言曰:「田舍子剩獲十斛麥,尚欲更故婦。天子富有四海,立一後,謂之不可,何哉?」帝意遂定。王後廢,敬宗請削後家官爵,廢太子忠而立代王,遂兼太子賓客。帝得所欲,故詔敬宗待詔武德殿西闥。頃拜侍中,監修國史,爵郡公。



 帝嘗幸故長安城,按蹕裴回,視古區處,問侍臣:「秦、漢以來幾君都此?」敬宗曰:「秦居咸陽,漢惠帝始城之。其後苻堅、姚萇、宇文周居之。」帝復問:「漢武開昆明池實何年?」對曰:「元狩三年,將伐昆明,實為此池以肄戰。」帝乃詔與弘文學士討古宮室故區,具條以聞。進中書令,仍守侍中。敬宗於立後有助力,知後鉗戾,能固主以久己權,乃陰連後謀逐韓瑗、來濟、褚遂良,殺梁王、長孫無忌、上官儀,朝廷重足事之,威寵熾灼,當時莫與比。改右相,辭疾,拜太子少師、同東西臺三品。年老,不任趨步,特詔與司空李勣朝朔日,聽乘小馬至內省。



 帝東封泰山,以敬宗領使。次濮陽,帝問竇德玄:「此謂帝丘,何也?」德玄不對。敬宗儳曰:「臣能知之。昔帝顓頊始居此地,以王天下。其後夏後相因之,為寒浞所滅。後緡方侲,逃出自竇,在此地也。後昆吾氏因之,而為夏伯。昆吾既衰,湯滅之。其頌曰:『韋、顧既伐,昆吾、夏桀』是也。至春秋時,衛成公自楚丘徙居之,《左氏》稱『相奪予享』,以舊地也。由顓頊所居,故曰帝丘。臣聞有德者啟其國土,失道者則喪其疆宇。自古大都美國,居者不一姓,故有國家者不可不慎也。」帝曰:「《書》稱『浮於濟、漯』,今濟與漯斷不相屬,何故而然?」對曰:「夏禹道沇水東流為濟,入於河。今自漯至溫而入河,水自此洑地過河而南,出為滎;又洑而至曹、濮,散出於地,合而東,汶水自南入之,所謂『泆為滎,東出於陶丘北,又東會於汶』是也。古者五行皆有官,水官不失職,則能辨味與色。潛而出,合而更分,皆能識之。」帝曰:「天下洪流巨谷,不載祀典,濟甚細而在四瀆,何哉?」對曰:「瀆之言獨也。不因餘水,獨能赴海者也。且天有五星,運而為四時;地有五岳,流而為四瀆;人有五事,用而為四支。五,陽數也;四,陰數也,有奇偶、陰陽焉。陽者光曜,陰者晦昧,故辰隱而難見。濟潛流屢絕,狀雖微細,獨而尊也。」帝曰:「善。」敬宗退,矜曰:「大臣不可無學,向德玄不能對,吾恥之。」德玄聞之,不屑曰:「人各有能。不強所不知,吾所能也。」李勣曰:「敬宗多聞,美矣;竇之不強,不亦善乎?」



 初,《高祖、太宗實錄》,敬播所撰,信而詳。及敬宗身為國史,竄改不平,專出己私。始虞世基與善心同遭賊害,封德彞常曰:「昔吾見世基死,世南匍匐請代;善心死,敬宗蹈舞求生。」世為口實,敬宗銜憤。至立《德彞傳》,盛誣以惡。敬宗子娶尉遲敬德女孫,而女嫁錢九隴子。九隴,本高祖隸奴也,為虛立門閥功狀,至與劉文靜等同傳。太宗賜長孫無忌《威鳳賦》,敬宗猥稱賜敬德。蠻酋龐孝泰率兵從討高麗,賊笑其懦,襲破之。敬宗受其金,乃稱「屢破賊,唐將言驍勇者唯蘇定方與孝泰,曹繼叔、劉伯英出其下遠甚」。然知貞觀後,論次諸書,自晉盡隋,及《東殿新書》、《西域圖志》、《姓氏錄》、《新禮》等數十種皆敬宗總知之,賞賚不勝紀。



 敬宗營第舍華僭,至造連樓,使諸妓走馬其上,縱酒奏樂自娛。嬖其婢,因以繼室,假姓虞。子昂烝之,敬宗怒黜虞,奏斥昂嶺外,久乃表還。



 咸亨初,以特進致仕,仍朝朔望,續其俸祿。卒,年八十一。帝為舉哀,詔百官哭其第,冊贈開府儀同三司、揚州大都督,陪葬昭陵。太常博士袁思古議:「敬宗棄子荒徼,女嫁蠻落,謚曰繆。」其孫彥伯訴思古有嫌,詔更議。博士王福畤曰:「何曾忠而孝,以食日萬錢謚繆醜,況敬宗忠孝兩棄,飲食男女之累過之。」執不改。有詔尚書省雜議,更謚曰恭。



 彥伯,昂子也,頗有文。敬宗晚年不復下筆,凡大典冊悉彥伯為之。嘗戲昂曰:「吾兒不及若兒。」答曰:「渠父不如昂父。」後又納婢譖,奏流彥伯嶺表,遇赦還,累官太子舍人。既與思古有憾,欲邀擊諸路,思古曰:「吾為先子報仇耳。」彥伯慚而止。



 垂拱中,詔敬宗配饗高宗廟廷。



 李義府,瀛州饒陽人。其祖嘗為射洪丞,因客永泰。貞觀中,李大亮巡察劍南,表義府才,對策中第,補門下省典儀。劉洎、馬周更薦之,太宗召見,轉監察御史,詔侍晉王。王為太子,除舍人、崇賢館直學士,與司議郎來濟俱以文翰顯,時稱「來李」。獻《承華箴》,末云:「佞諛有類,邪巧多方。其萌不絕,其害必彰。」義府方諂事太子,而文致若讜直者,太子表之,優詔賜帛。



 高宗立,遷中書舍人,兼修國史,進弘文館學士。為長孫無忌所惡,奏斥壁州司馬。詔未下,義府問計於舍人王德儉。德儉者,許敬宗甥,癭而智,善揣事,因曰:「武昭儀方有寵,上欲立為後,畏宰相議,未有以發之。君能建白,轉禍於福也。」義府即代德儉直夜,叩閣上表,請廢後立昭儀。帝悅,召見與語,賜珠一斗,停司馬詔書,留復侍。武後已立,義府與敬宗、德儉及御史大夫崔義玄、中丞袁公瑜、大理正侯善業相推轂,濟其奸,誅棄骨鯁大臣,故後得肆志攘取威柄,天子斂衽矣。



 義府貌柔恭,與人言,嬉怡微笑,而陰賊褊忌著於心,凡忤意者,皆中傷之,時號義府「笑中刀」。又以柔而害物,號曰「人貓」。



 永徽六年,拜中書侍郎、同中書門下三品,封廣平縣男,又兼太子右庶子,爵為侯。洛州女子淳于以奸系大理,義府聞其美,屬丞畢正義出之,納以為妾。卿段竇玄以狀聞。詔給事中劉仁軌、侍御史張倫鞫治。義府且窮,逼正義縊獄中以絕始謀。侍御史王義方廷劾,義府不引咎,三叱之,然後趨出。義方極陳其惡,帝陰德義府,故貸不問,為抑義方,逐之。未幾進中書令,檢校御史大夫,加太子賓客,更封河間郡公,詔造私第。諸子雖褓負皆補清官。



 初,杜正倫為黃門侍郎,義府才典儀。及同輔政,正倫恃先進不相下,密與中書侍郎李友益圖去義府,反為所誣,交訟帝前。帝兩黜之,正倫為橫州刺史,義府普州刺史,流友益峰州。明年,召為吏部尚書、同中書門下三品。母喪免,奪喪為司列太常伯、同東西臺三品。更葬其先永康陵側,役縣人牛車輸土築墳,助役者凡七縣,高陵令不勝勞而死。公卿爭賵遺。葬日,詔御史節哭。送車從騎相銜,帷帟奠帳自灞橋屬三原七十里不絕,轜輴芻偶,僭侈不法,人臣送葬之盛無與比者。殷王出閣,又兼府長史,稍遷右相。



 義府已貴,乃言系出趙郡,與諸李敘昭穆,嗜進者往往尊為父兄行。給事中李崇德引與同譜,既謫普州,亟削去,義府銜之,及復當國,傅致其罪,使自殺於獄。貞觀中,高士廉、韋挺、岑文本、令狐德棻修《氏族志》,凡升降,天下允其議,於是州藏副本以為長式。時許敬宗以不載武后本望,義府亦恥先世不見敘,更奏刪正。委孔志約、楊仁卿、史玄道、呂才等定其書,以仕唐官至五品皆升士流。於是兵卒以軍功進者,悉入書限,更號《姓氏錄》,縉紳共嗤靳之,號曰「勛格」。義府奏悉收前志燒絕之。自魏太和中定望族,七姓子孫迭為婚姻,後雖益衰,猶相誇尚。義府為子求婚不得,遂奏一切禁止。



 既主選,無品鑒才,而溪壑之欲,惟賄是利,不復銓判,人人咨訕。又母、妻、諸子賣官市獄,門如沸湯。自永徽後,御史多制授,吏部雖有調注,至門下覆不留。義府乃自注御史、員外、通事舍人,有司不敢卻。帝嘗從容戒義府曰:「聞卿兒子女婿橈法多過失,朕為卿掩覆,可少勖之。」義府內倚後,揣群臣無敢白其罪者,不虞帝之知,乃勃然變色,徐曰:「誰為陛下道此?」帝曰:「何用問我所從得邪!」義府謷然不謝,徐引去,帝由是不悅。



 會術者杜元紀望義府第有獄氣,曰:「發積錢二千萬,可以厭勝。」義府信之,裒索殊急。居母喪,朔望給告,即羸服與元紀出野,馮高窺覘災眚,眾疑其有異謀。又遣子津召長孫延,謂曰:「吾為子得一官。」居五日,延拜司津監,索謝錢七十萬。右金吾倉曹參軍楊行穎白其贓,詔司刑太常伯劉祥道與三司雜訊,李勣監按,有狀,詔除名,流巂州,子率府長史洽、千牛備身洋及婿少府主簿柳元貞並流廷州,司議郎津流振州,朝野至相賀。三子及婿尤兇肆,既敗,人以為誅「四兇」。或作《河間道元帥劉祥道破銅山大賊李義府露布》,榜於衢。乾封元年大赦,獨流人不許還,義府憤恚死,年五十三。自其斥,天下憂且復用,比死,內外乃安。



 上元初,赦妻子還洛陽。如意中,贈義府揚州大都督,崔義玄益州大都督,王德儉、袁公瑜魏、相二州刺史,各賜實封。睿宗立,詔停。少子湛,見《李多祚傳》。



 傅游藝,衛州汲人。載初初,由合宮主簿再遷左補闕。武后奪政,即上書詭說符瑞,勸後當革姓以明受命。後悅,擢給事中。閱三月,進同鳳閣鸞臺平章事,即拜鸞臺侍郎。後乃黜唐稱周,廢唐宗廟,自稱皇帝,賜游藝姓武氏,以兄神童為冬官尚書。游藝嘗夢登湛露殿,既寤,以語所親。有告其謀反者,下獄自殺,以五品禮葬之。



 初,游藝探后旨,誣殺宗室,復請發六道使,後卒用其言。萬國俊等既出,天下被其酷。游藝起一歲,賜袍自青及紫,人號「四時仕宦」。然歲中即敗,前古少其比云。



 李林甫,長平肅王叔良曾孫。初為千牛直長,舅姜晈愛之。開元初,遷太子中允。源乾曜執政,與晈為姻家,而乾曜子為林甫求司門郎中,乾曜素薄之,曰:「郎官應得才望,哥奴豈郎中材邪?」哥奴,林甫小字也。即授以諭德,累擢國子司業。宇文融為御史中丞,引與同列,稍歷刑、吏部侍郎。初,吏部置長名榜,定留放。寧王私謁十人,林甫曰:「願絀一人以示公。」遂榜其一,曰:「坐王所囑,放冬集。」



 時武惠妃寵傾後宮,子壽王、盛王尤愛。林甫因中人白妃,願護壽王為萬歲計,妃德之。侍中裴光庭夫人,武三思女,嘗私林甫,而高力士本出三思家。及光庭卒,武請力士以林甫代為相。力士未敢發,而帝因蕭嵩言,自用韓休。方具詔,武擿語林甫,使為休請。休既相,重德林甫,而與嵩有隙,乃薦林甫有宰相才,妃陰助之,即拜黃門侍郎。尋為禮部尚書、同中書門下三品,再進兵部尚書。



 皇太子、鄂王、光王被譖,帝欲廢之。張九齡切諫,帝不悅。林甫惘然,私語中人曰:「天子家事,外人何與邪?」二十四年,帝在東都,欲還長安。裴耀卿等建言:「農人場圃未畢,須冬可還。」林甫陽蹇,獨在後。帝問故,對曰:「臣非疾也,願奏事。二都本帝王東西宮,車駕往幸,何所待時?假令妨農,獨赦所過租賦可也。」帝大悅,即駕而西。始九齡繇文學進,守正持重,而林甫特以便佞,故得大任,每嫉九齡,陰害之。帝欲進朔方節度使牛仙客實封,九齡謂林甫:「封賞待名臣大功,邊將一上最,可遽議?要與公固爭。」林甫然許。及進見,九齡極論,而林甫抑嘿,退又漏其言。仙客明日見帝,泣且辭。帝滋欲賞仙客,九齡持不可。林甫為人言:「天子用人,何不可者?」帝聞,善林甫不專也。由是益疏薄九齡,俄與耀卿俱罷政事,專任林甫,相仙客矣。初,三宰相就位,二人磬折趨,而林甫在中,軒驁無少讓,喜津津出眉宇間。觀者竊言:「一雕挾兩兔。」少選,詔書出,耀卿、九齡以左右丞相罷,林甫嘻笑曰:「尚左右丞相邪?」目恚而送乃止,公卿為戰慄。於是林甫進兼中書令。帝卒用其言,殺三子,天下冤之。大理卿徐嶠妄言:「大理獄殺氣盛,鳥雀不敢棲。今刑部斷死,歲才五十八,而烏鵲巢獄戶,幾至刑措。」群臣賀帝,而帝推功大臣,封林甫晉國公,仙客豳國公。



 及帝將立太子,林甫探帝意,數稱道壽王,語秘不傳,而帝意自屬忠王,壽王不得立。太子既定,林甫恨謀不行,且畏禍,乃陽善韋堅。堅,太子妃兄也。使任要職,將覆其家,以搖東宮。乃構堅獄,而太子絕妃自明,林甫計黜。杜良娣之父有鄰與婿柳勣不相中,勣浮險,欲助林甫,乃上有鄰變事,捕送詔獄賜死。逮引裴敦復、李邕等,皆林甫素忌惡者,株連殺之。太子亦出良娣為庶人。未幾,擿濟陽別駕魏林,使誣河西節度使王忠嗣欲擁兵佐太子。帝不信,然忠嗣猶斥去。林甫數曰:「太子宜知謀。」帝曰:「吾兒在內,安得與外人相聞?此妄耳!」林甫數危太子,未得志,一日從容曰:「古者立儲君必先賢德,非有大勛力於宗稷,則莫若元子。」帝久之曰:「慶王往年獵,為豽傷面甚。」答曰:「破面不愈於破國乎?」帝頗惑,曰:「朕徐思之。」然太子自以謹孝聞,內外無槊言,故飛語不得入,帝無所發其猜。



 林甫善刺上意,時帝春秋高,聽斷稍怠,厭繩檢,重接對大臣,及得林甫,任之不疑。林甫善養君欲,自是帝深居燕適,沈蠱衽席,主德衰矣。林甫每奏請,必先餉遺左右,審伺微旨,以固恩信,至饔夫御婢皆所款厚,故天子動靜必具得之。性陰密,忍誅殺,不見喜怒。面柔令,初若可親,既崖阱深阻,卒不可得也。公卿不由其門而進,必被罪徙;附離者,雖小人且為引重。同時相若九齡、李適之皆遭逐;至楊慎矜、張瑄、盧幼臨、柳升等緣坐數百人,並相繼誅。以王鉷、吉溫、羅希奭為爪牙,數興大獄,衣冠為累息。適之子霅嘗盛具召賓客,畏林甫,乃終日無一人往者。林甫有堂如偃月,號月堂。每欲排構大臣,即居之,思所以中傷者。若喜而出,即其家碎矣。子岫為將作監,見權勢熏灼,惕然懼,常從游後園,見輦重者,跪涕曰:「大人居位久,枳棘滿前,一旦禍至,欲比若人可得乎?」林甫不樂曰:「勢已然,可奈何?」



 時帝詔天下士有一藝者得詣闕就選,林甫恐士對詔或斥己,即建言:「士皆草茅,未知禁忌,徒以狂言亂聖聽,請悉委尚書省長官試問。」使御史中丞監總,而無一中程者。林甫因賀上,以為野無留才。俄兼隴右、河西節度使。改右相,罷節度,加累開府儀同三司,實封戶三百。



 咸寧太守趙奉璋得林甫隱惡二十條,將言之,林甫諷御史捕系奉璋,劾妖言,抵死;著作郎韋子春坐厚善貶。帝嘗大陳樂勤政樓,既罷,兵部侍郎盧絢按轡絕道去,帝愛其愬藉,稱美之。明日林甫召絢子曰:「尊府素望,上欲任以交、廣,若憚行,且當請老。」絢懼,從之。因出為華州刺史,俄授太子員外詹事,絢繇是廢。於時有以材譽聞者,林甫護前,皆能得於天子抑遠之,故在位恩寵莫比。凡御府所貢遠方珍鮮,使者傳賜相望。帝食有所甘美,必賜之。嘗詔百僚閱歲貢於尚書省,既而舉貢物悉賜林甫,輦致其家。從幸華清宮,給御馬、武士百人、女樂二部。薛王別墅勝麗甲京師,以賜林甫,它邸第、田園、水磑皆便好上腴。車馬衣服侈靡,尤好聲伎。侍姬盈房,男女五十人。故事,宰相皆元功盛德,不務權威,出入騎從簡寡,士庶不甚引避。林甫自見結怨者眾,憂刺客竊發,其出入,廣騶騎,先驅百步,傳呼何衛,金吾為清道,公卿闢易趨走。所居重關復壁,絡版甃石,一夕再徙,家人亦莫知也。或帝不朝,群司要官悉走其門,臺省為空。左相陳希烈雖坐府,卒無人入謁。



 林甫無學術,發言陋鄙,聞者竊笑。善苑咸、郭慎微,使主書記。然練文法,其用人非諂附者一以格令持之,故小小綱目不甚亂,而人憚其威權。久之,又兼安西大都護、朔方節度使。俄兼單于副大都護,以朔方副使李獻忠反,讓還節度。



 始厚王鉷,為盡力。及鉷敗,詔宰相治狀,林甫大懼,不敢面鉷,獄具署名,亦無所申救。因以楊國忠代為御史大夫。林甫薄國忠材孱,無所畏,又以貴妃故善之。及是權益盛,貴震天下,始交惡若仇敵。然國忠方兼劍南節度使,而南蠻入寇,林甫因建遣之鎮,欲離間之。國忠入辭,帝曰:「處置且訖,亟還,指日待卿。」林甫聞之憂懣。是時已屬疾,稍侵。會帝幸溫湯,詔以馬輿從,御醫珍膳繼至,詔旨存問,中官護起居。病劇,巫者視疾云:「見天子當少間。」帝欲視之,左右諫止。乃詔林甫出廷中,帝登降聖閣,舉絳巾招之。林甫不能興,左右代拜。俄而國忠至自蜀,謁林甫床下,垂涕托後事,因不食卒。諸子護還京發喪,贈太尉、揚州大都督。



 林甫居相位凡十九年,固寵市權,蔽欺天子耳目,諫官皆持祿養資,無敢正言者。補闕杜璡再上書言政事,斥為下邽令。因以語動其餘曰:「明主在上,群臣將順不暇,亦何所論?君等獨不見立仗馬乎,終日無聲,而飫三品芻豆;一鳴,則黜之矣。後雖欲不鳴,得乎?」由是諫爭路絕。



 貞觀以來,任蕃將者如阿史那社爾、契苾何力皆以忠力奮,然猶不為上將,皆大臣總制之,故上有餘權以制於下。先天、開元中,大臣若薛訥、郭元振、張嘉貞、王晙、張說、蕭嵩、杜暹、李適之等,自節度使入相天子。林甫疾儒臣以方略積邊勞,且大任,欲杜其本,以久己權,即說帝曰:「以陛下雄材,國家富強,而夷狄未滅者,繇文吏為將,憚矢石,不身先。不如用蕃將,彼生而雄,養馬上,長行陣,天性然也。若陛下感而用之,使必死,夷狄不足圖也。」帝然之,因以安思順代林甫領節度,而擢安祿山、高仙芝、哥舒翰等專為大將。林甫利其虜也,無入相之資,故祿山得專三道勁兵,處十四年不徙,天子安林甫策,不疑也,卒稱兵蕩覆天下,王室遂微。



 初,林甫夢人皙而髯,將逼己。寤而物色,得裴寬類所夢,曰:「寬欲代我。」因李適之黨逐之。其後易國忠代林甫,貌類寬云。國忠素銜林甫,及未葬,陰諷祿山暴其短。祿山使阿布思降將入朝,告林甫與思約為父子,有異謀。事下有司,其婿楊齊宣懼,妄言林甫厭祝上,國忠劾其奸。帝怒,詔林甫淫祀厭勝,結叛虜,圖危宗社,悉奪官爵,斫棺剔取含珠金紫,更以小槥,用庶人禮葬之;諸子司儲郎中儒、太常少卿嶼及岫等悉徙嶺南、黔中,各給奴婢三人,籍其家;諸婿若張博濟、鄭平、杜位、元捴,屬子復道、光,皆貶官。



 博濟亦憸薄自肆。為戶部郎中,部有考堂,天下歲會計處,博濟廢為員外郎中聽事,壯偉華敞,供擬豐侈至千品;別取都水監地為考堂,擅費諸州籍帳錢不貲,有司不敢言。



 帝之幸蜀也,給事中裴士淹以辯學得幸。時肅宗在鳳翔,每命宰相,輒啟聞。及房琯為將,帝曰:「此非破賊才也。若姚元崇在,賊不足滅。」至宋璟,曰:「彼賣直以取名耳。」因歷評十餘人,皆當。至林甫,曰:「是子妒賢疾能,舉無比者。」士淹因曰:「陛下誠知之,何任之久邪?」帝默不應。



 至德中,兩京平,大赦,唯祿山支黨及林甫、楊國忠、王鉷子孫不原。天寶時,嘗鏤玉為玄元皇帝及玄宗、肅宗像於太清宮,復琢林甫、陳希烈像列左右序。代宗時,或言:「林甫陰險,嘗不利先帝,宗廟幾危,奈何留像至今?」有詔瘞宮中。廣明初,盧攜為太清宮使,發地得其像,輦送京兆毀之云。



 陳希烈者,宋州人。博學,尤深黃老,工文章。開元中,帝儲思經義,自褚無量、元行沖卒,而希烈與康子元、馮朝隱進講禁中,其應答詔問,敷盡微隱,皆希烈為之章句。累遷中書舍人,十九年為集賢院學士,進工部侍郎,知院事。帝有所撰述,希烈必助成之。遷門下侍郎。



 天寶元年,有神降丹鳳門,以為老子告錫靈符,希烈因是上言:「臣侍演《南華真經》至七篇,陛下顧曰:『此言養生,朕既悟其術,而《德充符》詎無非常應哉?』臣稽首對:『陛下德充於內,符應於外,必有絕瑞表之。』今靈符降錫,與帝意合,宜示史官,著顯祥,摛照無窮。」其媮佞類如此。俄兼崇玄館大學士,封臨潁侯。



 林甫顓朝,茍用可專制者,引與共政。以希烈柔易,且帝眷之厚,乃薦之。五載,進同中書門下平章事,遷左丞相兼兵部尚書,許國公,又兼秘書省圖書使,寵與林甫侔。林甫居位久,其陰詭雖足自固,亦希烈左右焉。楊國忠執政,素忌之,希烈引避,國忠即薦韋見素代相,罷為太子太師。希烈失職,內忽忽無所賴。及祿山盜京師,遂與達奚珣等偕相賊。後論罪當斬,肅宗以上皇素所遇,賜死於家。



\end{pinyinscope}