\article{志第四十五 食貨五}

\begin{pinyinscope}

 武德元年,文武官給祿,頗減隋制,一品七百石,從一品六百石可知論主張世界可以認識的哲學學說。唯物主義者和徹,二品五百石,從二品四百六十石,三品四百石,從三品三百六十石,四品三百石,從四品二百六十石,五品二百石,從五品百六十石,六品百石,從六品九十石,七品八十石,從七品七十石,八品六十石,從八品五十石,九品四十石,從九品三十石,皆以歲給之。外官則否。



 一品有職分田十二頃,二品十頃,三品九頃,四品七頃,五品六頃,六品四頃,七品三頃,五十畝,八品二頃五十畝,九品二頃,皆給百里內之地。諸州都督、都護、親王府官二品十二頃,三品十頃,四品八頃,五品七頃,六品五頃,七品四頃,八品三頃,九品二頃五十畝。鎮戍、關津、岳瀆官五品五頃,六品三頃五十畝,七品三頃,八品二頃,九品一頃五十畝。三衛中郎將、上府折沖都尉六頃,中府五頃五十畝,下府及郎將五頃;上府果毅都尉四頃,中府三頃五十畝,下府三頃;上府長史、別將三頃,中府、下府二頃五十畝;親王府典軍五頃五十畝,副典軍四頃;千牛備身左右、太子千牛備身三頃;折沖上府兵曹二頃,中府、下府一頃五十畝。外軍校尉一頃二十畝,旅帥一頃,隊正、副八十畝。



 親王以下又有永業田百頃,職事官一品六十頃,郡王、職事官從一品五十頃,國公、職事官從二品三十五頃,縣公、職事官三品二十五頃,職事官從三品二十頃,侯、職事官四品十二頃,子、職事官五品八頃,男、職事官從五品五頃,六品、七品二頃五十畝,八品、九品二頃。上柱國三十頃,柱國二十五頃,上護軍二十頃,護軍十五頃,上輕車都尉十頃,輕車都尉七頃,上騎都尉六頃,騎都尉四頃,驍騎、飛騎尉八十畝,雲騎、武騎尉六十畝。散官五品以上給同職事官。五品以上受田寬鄉,六品以下受於本鄉。解免者追田,除名者受口分之田,襲爵者不別給。流內九品以上口分田終其身,六十以上停私乃收。



 凡給田而無地者,畝給粟二斗。



 京司及州縣皆有公廨田,供公私之費。其後以用度不足,京官有俸賜而巳。諸司置公廨本錢,以番官貿易取息,計員多少為月料。



 貞觀初,百官得上考者,給祿一季。未幾,又詔得上下考給祿一年,出使者稟其家,新至官者計日給糧。中書舍人高季輔言:「外官卑品貧匱,宜給祿養親。」自後以地租春秋給京官,歲凡五十萬一千五百餘斛。外官降京官一等,一品以五十石為一等,二品、三品以三十石為一等,四品、五品以二十石為一等,六品、七品以五石為一等,八品、九品以二石五斗為一等。無粟則以鹽為祿。



 十一年,以職田侵漁百姓,詔給逃還貧戶,視職田多少,每畝給粟二升,謂之「地子」。是歲,以水旱復罷之。



 十二年,罷諸司公廨本錢,以天下上戶七千人為胥士,視防閤制而收其課,計官多少而給之。十五年,復置公廨本錢,以諸司令史主之,號「捉錢令史」。每司九人,補於吏部,所主才五萬錢以下,市肆販易,月納息錢四千,歲滿受官。諫議大夫褚遂良上疏:「京七十餘司,更一二載,捉錢令史六百餘人受職。太學高第,諸州進士,拔十取五,猶有犯禁罹法者,況廛肆之人,茍得無恥,不可使其居職。」太宗乃罷捉錢令史,復詔給百官俸。



 十八年,以京兆府、岐、同、華、邠、坊州隙地陂澤可墾者,復給京官職田。



 二十二年,置京諸司公廨本錢,捉以令史、府史、胥士。永徽元年,廢之,以天下租腳直為京官俸料。其後又薄斂一歲稅,以高戶主之,月收息給俸。尋顓以稅錢給之,歲總十五萬二千七百三十緡。



 一品月俸八千,食料一千八百,雜用一千二百。二品月俸六千五百,食料一千五百,雜用一千。三品月俸五千一百,雜用九百。四品月俸三千五百,食料、雜用七百。五品月俸三千,食料、雜用六百。六品月俸二千,食料、雜用四百。七品月俸一千七百五十,食料、雜用三百五十。八品月俸一千三百,食料三百,雜用二百五十。九品月俸一千五十,食料二百五十,雜用二百。行署月俸一百四十,食料三十。



 職事官又有防閤、庶僕:一品防閤九十六人,二品七十二人,三品四十八人,四品三十二人,五品二十四人;六品庶僕十五人,七品四人,八品三人,九品二人。公主有邑士八十人,郡主六十人,縣主四十人。外官以州、府、縣上下中為差,少尹、長史、司馬及丞減長官之半,參軍、博士減判司三之二,主簿、縣尉減丞三之二,錄事、市令以參軍職田為輕重,京縣錄事以縣尉職田為輕重。羈縻州官,給以土物。關監官,給以年支輕貨。折沖府官則有仗身:上府折沖都尉六人,果毅四人,長史、別將三人,兵曹二人,中、下府各減一人,皆十五日而代。開府儀同三司、特進、光祿大夫同職事官,公廨、雜用不給。員外官、檢校、判、試、知給祿料食糧之半,散官、勛官、衛官減四之一,致仕五品以上給半祿,解官充侍亦如之。四夷宿衛同京官。



 天下置公廨本錢,以典史主之,收贏十之七,以供佐史以下不賦粟者常食,餘為百官俸料。京兆、河南府錢三百八十萬,太原及四大都督府二百七十五萬,中都督府、上州二百四十二萬,下都督、中州一百五十四萬,下州八十八萬;京兆、河南府京縣一百四十三萬,太原府京縣九十一萬三千,京兆、河南府畿縣八十二萬五千,太原府畿縣、諸州上縣七十七萬,中縣五十五萬,中下縣、下縣三十八萬五千;折沖上府二十萬,中府減四之一,下府十萬。



 麟德二年,給文官五品以上仗身,以掌閑、幕士為之。咸亨元年,與職事官皆罷。乾封元年,京文武官視職事品給防閤、庶僕。



 百官俸出於租調,運送之費甚廣。公廨出舉,典史有徹垣墉、鬻田宅以免責者。又以雜職供薪炭,納直倍於正丁。儀鳳三年,王公以下率口出錢,以充百官俸食防閤、庶僕、邑士、仗身、封戶。



 調露元年,職事五品以上復給仗身。光宅元年,以京官八品、九品俸薄,詔八品歲給庶僕三人,九品二人。文武職事三品以上給親事、帳內。以六品、七品子為親事,以八品、九品子為帳內,歲納錢千五百,謂之「品子課錢」。三師、三公、開府儀同三司百三十人;嗣王、郡王百八人;上柱國領二品以上職事九十五人,領三品職事六十九人;柱國領二品以上職事七十三人,領三品職事五十五人;護軍領二品以上職事六十二人,領三品職事三十六人。二品以下又有白直、執衣:二品白直四十人,三品三十二人,四品二十四人,五品十六人,六品十人,七品七人,八品五人,九品四人;二品執衣十八人,三品十五人,四品十三人,五品九人,六品、七品各六人,八品、九品各三人。皆中男為之。防閤、庶僕,皆滿歲而代。外官五品以上亦有執衣。都護府不治州事亦有仗身:都護四人,副都護、長史、司馬三人,諸曹參軍事二人,上鎮將四人,中下鎮將、上鎮副三人,中、下鎮副各二人,鎮倉曹、關令丞、戍主副各一人,皆取於防人衛士,十五日而代。宿衛官三品以上仗身三人,五品以上二人,六品以下及散官五品以上各一人,取於番上衛士,役而不收課。親王出籓者,府佐史、典軍、副典軍有事力人,數如白直。諸司、諸使有守當及廳子,以兵及勛官為之。白直、執衣以下分三番,周歲而代,供役不逾境。後皆納課:仗身錢六百四十,防閤、庶僕、白直錢二千五百,執衣錢一千。其後親事、帳內亦納課如品子之數。



 州縣典史捉公廨本錢者,收利十之七。富戶幸免徭役,貧者破產甚眾。秘書少監崔沔請計戶均出,每丁加升尺,所增蓋少;流亡漸復,倉庫充實,然後取於正賦,罷新加者。



 開元十年,中書舍人張嘉貞又陳其不便,遂罷天下公廨本錢,復稅戶以給百官;籍內外職田,賦逃還戶及貧民;罷職事五品以上仗身。



 十八年,復給京官職田。州縣籍一歲稅錢為本,以高戶捉之,月收贏以給外官。復置天下公廨本錢,收贏十之六。十九年,初置職田頃畝簿,租價無過六斗,地不毛者畝給二斗。



 二十四年,令百官防閤、庶僕俸食雜用以月給之,總稱月俸:一品錢三萬一千,二品二萬四千,三品萬七千,四品萬一千五百六十七,五品九千二百,六品五千三百,七品四千一百,八品二千四百七十五,九品千九百一十七。祿米則歲再給之:一品七百斛,從一品六百斛,二品五百斛,從二品四百六十斛,三品四百斛,從三品三百六十斛,四品三百斛,從四品二百五十斛,五品二百斛,從五品百六十斛,六品百斛,自此十斛為率,至從七品七十斛,八品六十七斛,自此五斛為率,至從九品五十二斛。外官降一等。



 先是州縣無防人者,籍十八以上中男及殘疾以守城門及倉庫門,謂之門夫。番上不至者,閑月督課,為錢百七十,忙月二百。至是以門夫資課給州縣官。



 二十九年,以京畿地狹,計丁給田猶不足,於是分諸司官在都者,給職田於都畿,以京師地給貧民。是時河南、北職田兼稅桑,有詔公廨、職田有桑者,毋督絲課。



 天寶初,給員外郎料,天下白直歲役丁十萬,有詔罷之,計數加稅以供用,人皆以為便。



 自開元後,置使甚眾,每使各給雜錢。宰相楊國忠身兼數官,堂封外月給錢百萬。幽州平盧節度使安祿山、隴右節度使哥舒翰兼使所給,亦不下百萬。



 十二載,國忠以兩京百官職田送租勞民,請五十里外輸於縣倉,斗納直二錢,百里外納直三錢,使百官就請於縣,然縣吏欺盜蓋多,而閑司有不能自直者。十四載,兩京九品以上月給俸加十之二,同正員加十之一。兵興,權臣增領諸使,月給厚俸,比開元制祿數倍。



 至德初,以用物不足,內外官不給料錢,郡府縣官給半祿及白直、品子課。乾元元年,亦給外官半料及職田,京官給手力課而已。上元元年,復令京官職田以時輸送,受加耗者以枉法贓論。其後籍以為軍糧矣。永泰末,取州縣官及折沖府官職田苗子三之一,市輕貨以賑京官。



 大歷元年,斂天下青苗錢,得錢四百九十萬緡,輸大盈庫,封太府左、右藏,鐍而不發者累歲。二年,復給京兆府及畿縣官職田,以三之一供軍饟。增稅青苗錢,一畝至三十。權臣月俸有至九十萬者,刺史亦至十萬。楊綰、常袞為相,增京官正員官及諸道觀察使、都團練使、副使以下料錢。初,檢校官同中書門下平章事者,月給錢十二萬。至是戶部侍郎判度支韓滉請同正官,從高而給之。文官一千八百五十四員,武官九百四十二員,月俸二十六萬緡,而增給者居三之一。



 先是,州縣職田、公廨田,每歲六月以白簿上尚書省覆實;至十月輸送,則有黃籍,歲一易之。後不復簿上,唯授祖清望要官,而職卑者稽留不付,黃籍亦不復更矣。德宗即位,詔黃籍與白簿皆上有司。



 建中三年,復減百官料錢以助軍。李泌為相,又增百官及畿內官月俸,復置手力資課,歲給錢六十一萬六千餘緡,文官千八百九十二員,武官八百九十六員。左右衛上將軍以下又有六雜給:一曰糧米,二曰鹽,三曰私馬,四曰手力,五曰隨身,六曰春冬服。私馬則有芻豆,手力則有資錢,隨身則有糧米、鹽,春冬服則有布、絹、絁、紬、綿,射生、神策軍大將軍以下增以鞋,比大歷制祿又厚矣。州縣官有手力雜給錢,然俸最薄者也。李泌以度支有兩稅錢,鹽鐵使有筦榷錢,可以擬經費,中外給用,每貫墊二十,號「戶部除陌錢」。復有闕官俸料、職田錢,積戶部,號「戶部別貯錢」。御史中丞專掌之,皆以給京官,歲費不及五十五萬緡。京兆和糴,度支給諸軍冬衣,亦往往取之。減王公以下永業田:郡王、職事官從一品田五十頃,國公、職事官正二品田四十頃,郡公、職事官從二品田三十頃,縣公、職事官正四品田十四頃,職事官從四品田十一頃。尚郡主檢校四品京官者月給料錢三十萬,祿百二十石。尚縣主檢校五品京官者料錢二十萬,祿百石。



 自李泌增百官俸,當時以為不可朘削矣,然有名存而職廢、額去而俸在者。宰相李吉甫建議減之,遂為常法。



 於時祠祭、蕃夷賜宴、別設,皆長安、萬年人吏主辦,二縣置本錢,配納質積戶收息以供費。諸使捉錢者,給牒免徭役,有罪府縣不敢劾治。民間有不取本錢,立虛契,子孫相承為之。嘗有毆人破首,詣閑廄使納利錢受牒貸罪。御史中丞柳公綽奏諸使捉錢戶,府縣得捕役,給牒者毀之,自是不得錢者不納利矣。議者以兩省、尚書省、御史臺總樞機,正百寮,而倍稱息利,非馭官之體。



 元和九年,戶部除陌錢每緡增墊五錢,四時給諸司諸使之餐,置驅使官督之,御史一人核其侵漁,起明年正月,收息五之一,號「元和十年新收置公廨本錢」。



 初,捉錢者私增公廨本,以防耗失,而富人乘以為奸,可督者私之,外以逋官錢迫蹙閭里,民不堪其擾。御史中丞崔從奏增錢者不得逾官本。其後兩省捉錢,官給牒逐利江淮之間,鬻茶鹽以橈法。十三年,以職田多少不均,每司收草粟以多少為差。其後宰相李玨、楊嗣復奏堂廚食利錢擾民煩碎,於是罷堂廚捉錢官,置庫量入計費。



 唐世百官俸錢,會昌後不復增減,今著其數:太師、太傅、太保,錢二百萬。太尉、司徒、司空,百六十萬。侍中,百五十萬。中書令,門下中書侍郎,左右僕射,太子太師、太保、太傅,百四十萬。尚書,御史大夫,太子少師、少保、少傅,百萬。節度使,三十萬。都防禦使、副使,監軍,十五萬。觀察使十萬。左右丞,侍郎,散騎常侍,諫議大夫,給事中,中書舍人,秘書、殿中、內侍監,御史中丞,太常、宗正、大理、司農、太府、鴻臚、太僕、光祿、衛尉卿,國子祭酒,將作、少府監,太子賓客、詹事,諸府尹,大都督府長史,都團練使、副使,上州刺史,八萬。太常、宗正少卿,太子左右庶子,節度副使,刺史知軍事,七萬。六軍統軍,諸府少尹,少監,少卿,國子司業,少詹事,六萬五千。左右衛、金吾衛上將軍,六軍大將軍,六萬。左右驍衛、武衛、威衛、領軍衛、監門衛、千牛衛上將軍,上州別駕,五萬五千。郎中,司天監,太子左右諭德、家令寺、僕寺、率更寺令,親王傅,別敕判官,觀察、團練判官掌書記,上州長史、司馬,五萬。左右衛、金吾衛大將軍,懷化大將軍,諸府、大都督司錄參軍事,鴘赤縣令,四萬五千。員外郎,起居郎,通事舍人,起居舍人,著作郎,內常侍,侍御史,殿中侍御史,太常、宗正、殿中、秘書丞,大理正,國子博士,京都宮苑總監監,都水使者,太子中舍、中允,王府長史,歸德將軍,節度推官、支使,防禦判官,上州錄事參軍事,畿縣、上縣令,四萬。懷化中郎將,三萬七千。左右驍衛、武衛、威衛、領軍衛、監門衛、千牛衛、殿前左右射生軍、神策軍大將軍,左右衛、金吾衛將軍,三萬六千。補闕,殿中侍御史,諸府、大都督府判官,赤縣丞,三萬五千。懷化郎將,三萬二千。拾遺,司天少監,六局奉御,內常侍,監察御史,御史臺主簿,太常博士,陵署令,大理司直,中書主書,門下錄事,太子贊善、典內、洗馬、司議郎,王府司馬,驍衛、武衛、威衛、領軍衛、監門衛、六軍、射生、神策軍將軍,歸德中郎將,觀察防禦團練推官巡官,鴘赤縣丞,兩赤縣主簿、尉,上州功曹參軍以下,上縣丞,三萬。城門郎,秘書郎,著作佐郎,六局直長,十六衛、六軍、諸府、十率府長史,懷化司階,畿縣丞,鴘赤縣主簿、尉,二萬五千。歸德司階,二萬三千。五官正,太常寺協律郎,陵署丞,諸寺監主簿,國子、太學、廣文助教,都水監丞,詹事府司直,太子通事舍人、文學、三寺丞、五局郎,王府諮議參軍、友,畿縣上縣主簿尉,二萬。懷化中候,萬八千。十六衛六軍十率府率、副率、中郎、中郎將,萬七千三百五十。歸德中候,萬七千。四門助教,十六衛佐,秘書省、崇文、弘文館校書郎、正字,太常寺奉禮郎、太祝,郊社、太樂、鼓吹署令,四門助教,京都宮苑總監副監,九成宮總監監、主事,十六衛、六軍衛佐,尚書省都事,萬六千。十六衛、六軍中候,太子內率府千牛,六千一百七十四。內寺伯,懷化司戈,諸府大都督府參軍事、文學、博士、錄事,上州參軍事、博士,萬五千。歸德司戈,萬四千。十六衛、六軍、十率府左右郎將,親王府典軍、副典軍,萬三千八百。司戈、內率府備身、僕寺進馬,三千七百一十二。符寶郎,內謁者監,九寺諸監,詹事府丞,太醫署令,太學、廣文、四門博士,中書門下主事,太子文學、侍醫,諸府、都督府醫博士、法直,兩赤縣錄事,上州錄事,市令,萬三千。懷化執戟長上,萬一千。門下省典儀,侍御醫,司天臺丞,都水監主簿,率府衛佐,諸司主事、御史臺主事,萬二千。司醫,太醫署丞,歸德執戟長上,一萬。醫佐,大理寺評事,太常宗正寺詹事府主簿、寺監,內侍省司天臺左右春坊詹事府錄事、主事,八千。司階,千牛備身左右,七千九百九十。京都園苑四面監監,兩京諸市、中尚、武庫、武成王廟署令,王府掾、屬、主簿、記室、錄事參軍事,七千。司天臺主簿、靈臺郎、保章正,上局署令,七品陵廟令,京都宮苑總監丞,司竹、溫泉監監,太子內坊丞,王府功曹以下參軍事,親王國令,公主邑司令,六千。奚官、內僕、內府局令,司竹、溫泉副監,五千。書、算、律學博士,內謁者,中局署令,上局署丞,五官挈壺正,京都園苑四面監、九成宮總監副監,醫、針博士,醫監,陵廟令,司竹、溫泉監丞,太子藥藏局丞,王府參軍事,王國大農,公主邑司丞,四千。獄丞,國子監直講,掌客,司儀,中局署丞,監膳,監作,監事,食醫,尚輦,進馬,奉乘,主乘,典乘,司庫,司廩,十六衛、十率府錄事,親、勛、翊府兵曹參軍事,司天臺司辰、司歷、監候,內坊典直,宮教博士,樂正,醫正,卜正,按摩、咒禁、卜博士,針、醫、卜、書、算助教,陵廟、太樂、鼓吹署丞,京都園苑四面監、九成宮總監丞,諸總監主簿,太子典膳、內直、典設、宮門局丞,三寺主簿,親王國尉、丞,三千。十六衛、六軍、十率府執戟、長上、左右中郎將二千八百五十。



\end{pinyinscope}