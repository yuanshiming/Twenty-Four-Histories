\article{列傳第一百一十 鄭二王韋張}

\begin{pinyinscope}

 鄭畋,字臺文,系出滎陽。父亞,字子佐。爽邁有文,舉進士、賢良方正、書判拔萃真假。邏輯實證主義者在運用這一原則時,又分為強形式的,三中其科。李德裕為翰林學士,高其才,及守浙西,闢署幕府。擢監察御史,李回任中丞,薦為刑部郎中知雜事,拜給事中。德裕罷宰相,出為桂管觀察使,坐吳湘獄不能直冤,貶循州刺史,死於官。



 畋舉進士,時年甚少,有司上第籍,武宗疑,索所試自省,乃可。奏為宣武推官,以書判拔萃擢渭南尉。父喪免。宣宗時,白敏中、令狐綯繼當國,皆怨德裕,其賓客並廢斥,故畋不調幾十年,外更帥鎮幕府。綯去位,始為虞部員外郎。右丞鄭薰誣畋罪,不可任郎官,出之。久乃入為刑部員外郎。劉瞻為宰相,薦授戶部郎中,入翰林為學士,俄知制誥。會討徐州賊龐勛,書詔紛委,畋思不淹晷,成文粲然,無不切機要,當時推之。勛平,以戶部侍郎進學士承旨。瞻以諫迕懿宗,賜罷,畋草制書多褒言,韋保衡等怨之,以為附下罔上,貶梧州刺史。僖宗立,內徙郴、絳二州,以右散騎常侍召還。故事,兩省轉對延英,獨常侍不與。畋建言宜備顧問,詔可,遂著於令。以兵部侍郎進同中書門下平章事。故時,宰相騶哄聯數坊,呵止行人。畋敕導者止百步,禁百官僕史不得擅至宰相府。交、廣、邕南兵,舊取嶺北五道米往餉之,船多敗沒。畋請以嶺南鹽鐵委廣州節度使韋荷,歲煮海取鹽直四十萬緡,市虔、吉米以贍安南,罷荊、洪等漕役,軍食遂饒。後以王師甫為嶺南供軍副使,師甫請兼總兵,而歲加獻錢二十萬緡。畋曰:「荷且有功,而師甫以利啖朝廷,謀奪其兵,不可。」罷之。再遷門下侍郎,封滎陽郡侯。以星變求去位,不許。



 乾符六年,黃巢勢浸盛,據安南,騰書求天平節度使。帝令群臣議,咸請假節以紓難。畋欲因授嶺南節度使,而盧攜方倚高駢,使立功,乃曰:「駢才略無雙,淮南天下勁兵,又諸道之師方至,蕞爾賊,奈何舍之,令四方解體邪?」畋曰:「不然。巢之亂本於饑,其眾以利合,故能興江、淮,根蔓天下。國家久平,士忘戰,所在閉壘不敢出。如以恩釋罪,使及歲豐,其下思歸,眾一離,巢即機上肉耳,法謂不戰而屈人兵也。今不伐以謀,而怖以兵,恐天下憂未艾也。」僕射於琮言:「南海以寶產富天下,如與賊,國藏竭矣。」天子內亦屬駢,乃然攜議。畋曰:「安危屬吾等,而公倚淮南用兵,吾不知所稅駕。」會駢奏:「南蠻方強,請如西戎,以公主下嫁。」攜又議從之。畋以為損國威靈,不可,即抗論,至相詬嫚。攜怒,拂衣去,裾衊於硯,因抵之。帝以大臣爭口語,無以示百官,乃俱罷,以畋為太子賓客,分司東都。俄召拜吏部尚書。



 明年,為鳳翔隴西節度使,募銳兵五百,號「疾雷將」;境中盜不敢發,發輒得。會巢陷東都,遣兵戍京師,以家財勞行,妻自糸任戎衣給戰士。帝出梁、洋,畋上謁斜谷,泣曰:「將相誤國,臣請死以懲無狀。」帝勞遣之,且曰:「公謹扼賊沖,無令得西向。」畋曰:「方艱虞時,事有機急,不可中覆,請便宜從事,臣當以死報國。」帝曰:「利社稷,無不可。」畋還,搜士卒,繕器械,浚城隍,使於梁者道相屬。俄而賊使至,諸將皆欲附賊,畋開諭不可,即悉出金帛,請得脫身去,復不聽。而使以偽赦令示軍中乃去。明日,詔使至,畋召監軍袁敬柔以逆順曉諸將,乃聽命,刺血以盟。畋遣子凝績從帝,有詔進同中書門下平章事。賊將又至,畋斬於軍,餘黨數百人皆捕誅之。遷檢校尚書右僕射、西面行營都統。軍中承制除拜。乃以前靈武節度使唐弘夫為行軍司馬。



 中和元年,賊將王璠率眾三萬來攻,畋使弘夫設伏以待。璠內輕畋儒柔,縱步騎鼓而前,畋以銳卒數千當賊,疏陣而多旗幟,乘高伐鼓,賊不測眾寡,陣未整,伏發,眾皆囂。日暮,軍四合,鏖戰龍尾坡,殺賊二萬級,積尸數十里,多獲鎧仗,璠遁去,禽璠子斬之,威動京師。時諸鎮兵在寰內尚數萬,無所歸,畋招來之,厚加慰結。乃與涇原程宗楚、秦州仇公遇、鄜延李孝恭、夏州拓拔思恭約盟,傳檄天下。時王命不出劍門,四方謂王室微,不能復興;及畋檄至,遠近咸聳,各治兵思立功,奔問行在。巢大懼,不敢西謀。當此時,微畋,天子幾殆。帝聞捷曰:「朕知畋不盡,儒者之勇乃爾!」



 弘夫取咸陽,以桴濟兵渭水。賊伏甲偽走,弘夫與宗楚乘勝入都門,為賊所覆。畋數敕無輕進,二人不聽,果敗。以鄜、夏兵屯東渭橋。再進司空、兼門下侍郎、京城四面行營都統,賜御袍犀帶。拜而不賀。



 行軍司馬李昌言者屯興平,遣麾下求為南面都統,輒引兵趨府。畋不意見襲,登城好語曰:「吾方入朝,公能戢兵愛人,為國滅賊乎?能,則守此矣。」遂委軍去。昌言自為留後,衛畋出境。既半道,內慚負,即辭疾。詔授太子少傅,分司東都,便醫於興元。



 明年,召至行在,以王鐸將兵,復拜畋司空、門下侍郎、平章事,軍務一以咨決。興州戍將孫鄴坐贓抵死,畋奏言:「方關輔失守,鄴護褒斜有功,請免死。陳秋兒保嵯峨山拒賊,農不廢耕,請以檢校散騎常侍隸奉天軍。」制皆可。舊制,使府校書郎以上,滿三歲遷;監察御史裏行至大夫、常侍,滿三十月遷。雖節度兼宰相,亦不敢越。自軍興,有歲內數遷者,畋以為不可,請:「行營節度,繇里行至大夫,許滿二十月遷;校書郎以上,滿二歲乃奏。非軍興者如故事。」從之。



 時田令孜恃權有所干請,畋不應。陳敬瑄欲以官品居宰相上,畋曰:「外宰相安得論品乎?」卒不肯處其下。令孜、敬瑄內常銜之。賊平,帝將還,而李昌言自以襲畋而奪之鎮,今畋當國,內不喜,故三人相結,而遣客上畋過咎。帝得其情,不許。畋乃引疾去位,入見帝曰:「乘輿東還,繇大散關幸鳳翔,供張頓峙,一委昌言,乃可安。臣若以宰相從,彼且猜阻,非所以靖反側也。請以散官養疾。或群臣有疑,願出臣章示之,使知天子於臣無纖芥者。」帝以其誠,乃授檢校司徒、太子太保,罷政事。以凝績為壁州刺史,留養。徙龍州,卒,年六十三,贈太尉。後帝思畋忠力,又贈太傅。凝績數歲亦卒。始,李茂貞以博野裨將戍奉天,畋召隸麾下,委以游邏,厚禮之。茂貞感其飾擢,及畋還葬鄭,表為請謚曰文昭。天復初,與李思恭配饗僖宗廟廷,又贈宗楚、弘夫官。



 畋為人仁恕,姿採如峙玉。凡與布衣交,至貴無少易。鄭縠者,薰子也。方畋秉政,擢為給事中,至侍郎。其損怨類如此。巢之難,先諸軍破賊,雖功不終,而還相天子,坐籌帷幄,終能復國云。



 王鐸,字昭範。宰相播昆弟子也。會昌初,擢進士第,累遷右補闕、集賢殿直學士。白敏中闢署西川幕府。咸通後,仕浸顯,歷中書舍人、禮部侍郎。所取多才實士,為世稱挹。拜御史中丞,以戶部侍郎判度支。十二年,繇禮部尚書進同中書門下平章事,加門下侍郎、尚書左僕射,超拜司徒。韋保衡緣恩幸輔政,始由鐸得進士,故謹事之。雖竊政權,將大斥不附者,病鐸持其事,不得肆,搢紳賴焉。鐸亦上疏祈解,乃以檢校左僕射出為宣武節度使。



 僖宗初,以左僕射召。始,鐸當國,練制度,智慮周密,時論推允。會河南盜起,天下跂鐸入輔,又鄭畋數言其賢,復拜門下侍郎、平章事。乾符六年,賊破江陵,宋威無功,諸將觀望不進,天下大震。朝廷議置統帥,鐸因請自率諸將督群盜。帝即以鐸為侍中、荊南節度使、諸道行營都統,封晉國公。綏納流冗,益募軍,完器鎧,武備張設。李系者,西平王晟諸孫。敏辯善言兵,然中無有。鐸信之,舉為將,分精兵使守湖南。俄而賊舍廣州,鼓而北,系望風未戰輒潰,鐸退營襄陽。於是以高駢代之,貶太子賓客,分司東都。



 未幾,召拜太子少師,從天子入蜀,拜司徒、門下侍郎、平章事,加侍中。復以太子太保平章事。是時,誅討大計悉屬駢,駢內幸多難,數偃蹇,而外逗撓。鐸感慨王室,每入對,必噫嗚流涕,固請行。時中和二年也。乃以檢校司徒、中書令為義成節度使,諸道行營都統,判延資、戶部、租庸等使。於是表崔安潛自副,鄭昌圖、裴贄、裴樞、王摶等在幕府,以周岌、王重榮、諸葛爽、康實、安師儒、時溥六節度為將佐,而中尉西門思恭為監軍,率衛兵洎梁、蜀師三萬壁盩厔,移檄天下。先是,諸將雖環賊,莫肯先。及鐸檄至,號令殷然,士氣皆起,急欲破賊,故巢戰數蹙。宦人田令孜策賊必破,欲使功出於己,乃構鐸於帝,罷為檢校司徒,以義成節度還屯。鐸功危就,而讒見奪,然卒因其勢困賊。後數月,復京師,策勛居關東諸鎮第一。四年,徙義昌節度使。



 鐸世貴,出入裘馬鮮明,妾侍且眾。過魏,樂彥禎子從訓心利之。李山甫者,數舉進士被黜,依魏幕府,內樂禍,且怨中朝大臣,導從訓以詭謀,使伏兵高雞泊劫之,鐸及家屬吏佐三百餘人皆遇害。朝廷微弱,不能治其冤,天下痛之。



 弟鐐,累官汝州刺史。乾符中,王仙芝來攻,鐐拒之,自督勇士與別將董漢勛守南、北門。城陷,漢勛力戰死,鐐貶韶州司馬。終太子賓客。



 王徽,字昭文,京兆人。第進士,授校書郎。沈詢判度支,徐商領鹽鐵,皆闢署使府。始,宣宗詔宰相選可尚主者,或以徽聞。徽本澹聲利,聞不喜,往見宰相劉彖曰:「徽年過四十,又多病,不應在選。」彖為言,乃罷。從令狐綯署宣武、淮南掌書記,召授右拾遺。書二十餘上,言無回忌,公議浩然歸重。徐商罷政事,守江陵,心欲表徽在幕府,恐其不樂外,忍不言。徽自往曰:「公知徽,安得不從?」商大喜,表為殿中侍御史,署節度府判官。御史中丞高湜薦知雜事,進考功員外郎。故事,考簿以硃注上下為殿最,歲久易漫,吏輒竄易為奸。徽始用墨,遂絕妄欺。擢翰林學士。



 廣明元年,盧攜罷宰相,以徽為戶部侍郎、同中書門下平章事。是日,黃巢入關,僖宗西狩,冒夜出。徽與崔沆、豆盧彖、僕射於琮詰朝乃知,追帝不及,墮崖樾間,為賊所執,迫還,將污以官。徽陽喑不答,以刃環脅,卒不動。賊令歸第,使醫護視。久之,守者懈,乃奔河中,裂縑書章,遣人間走蜀。詔拜兵部尚書、京城四面宣慰催陣使。



 昭義高潯與賊戰石橋,敗績。其將劉廣擅還,據潞州。別將孟方立殺廣,因取邢、洺、磁三州貳於己。昭義所隸,唯澤一州。帝以兵部侍郎鄭昌圖權守潞,士心多附方立,昌圖不能制。朝議以大臣鎮撫,即授徽檢校尚書左僕射、同中書門下平章事,領昭義節度使。是時,李克用亦爭澤、潞,徽商朝廷力未能以兵抗之,奉表固辭,詔可。更為諸道租庸供軍使。因說行營都監楊復光,請赦沙陀罪,令赴難。其夏,沙陀會諸軍,遂平京師,徽助為多,遷右僕射。



 大亂之後,宮觀焚殘,園陵皆發掘,鞠為丘莽,乘輿未有東意,詔徽充大明宮留守、京畿安撫制置脩奉使。徽外調兵食,內撫綏流亡,逾年,稍稍完聚,興復殿寢,裁制有宜,即奉表請帝東還。又進檢校司空、御史大夫,仍權京兆尹。宦要家爭遣人治第,侵冒齊民,訟訴滿前,徽不屈勢幸,一平以法,繇是為帝左右所憎,以其黨薛杞為少尹,輕其權。杞方居喪,徽奏止不使到府。眾忿,共譖罷徽,令赴行在。俄授太子少師。徽遂移疾河中,滿百日免。帝還京師,復申前授,稱疾不任奉謁。宰相疾其怨望,貶集州刺史。會帝避沙陀,出次寶雞。帝念徽無罪,拜吏部尚書,封瑯邪郡侯。未行而嗣襄王煴作亂,帝進次漢中。煴逼召徽,以尪廢自言。及煴僭號,迫群臣作誓牒,徽托手弱,卒不肯署。煴平,帝至鳳翔,召徽為御史大夫,固辭足痺,復拜太子少師。



 昭宗立,見便殿,進對詳洽,帝顧宰相曰:「徽神氣尚強,可用。」乃復授吏部尚書。是時,銓選失序,吏肆為奸,補調重復不可檢。徽為手籍,一驗實之,遂無奸滯。進右僕射。大順元年卒,贈司空,謚曰貞。



 譜言其先本魏諸公子,秦滅魏,至漢徙關中霸陵,以其故王家,為王氏。十世祖羆,仕周為同州刺史,死葬咸陽鳳政原,子孫因家杜陵。曾祖擇從,昆弟四人,曰易從、朋從、言從,皆擢進士第。至鳳閣舍人者三人,故號「鳳閣王氏」。自是訖大中時,登進士者十八人,位臺省牧守者三十餘人。徽有雅望,拜宰相一日而京師亂,故其設施無可道者。



 韋昭度,字正紀,京兆人。擢進士第,踐歷華近,累遷中書舍人。僖宗西狩,以兵部侍郎、翰林學士承旨從。未幾,同中書門下平章事。還京,授司空。再狩山南,還次鳳翔。李昌符亂興倉卒,昭度質家族於禁軍,誓共討賊,士感動,乃平昌符。遷太保,兼侍中。昭宗即位,守中書令,封岐國公。



 閬州刺史王建攻陳敬瑄於成都,以昭度為西川節度使。敬瑄不內,詔東川顧彥朗與建合兵以討,拜昭度兼行營招撫使。乃建幢節行城下,諭其眾曰:「毋久閉壘。」敬瑄遣人詈曰:「鐵券,先帝所命,若何違之?」淹半歲,始拔漢州。建紿昭度曰:「公暴師遠出,事蠻夷地,方山東兵連禍結,朝廷不能治,腹心疾也,宜亟還定之。敬瑄小醜,當責建等可辦。」昭度信之,請還。未半道,建以重兵守劍門,急攻成都。囚敬瑄,自稱留後。罷昭度為東都留守。



 杜讓能既被害,以司徒、門下侍郎復為平章事,進太傅。王行瑜求為尚書令,昭度建言:「太宗由是即位,後人臣無復拜者。郭子儀有大功,嘗授之,固辭免,況行瑜乎?」乃更號尚父。行瑜怨。會用李磎輔政,而崔昭緯密語行瑜曰:「前公已為尚書令,昭度持不可。今又引磎葉力,此奸人務立黨與,惑上聽,恐事復有如杜太尉時。」行瑜乃與李茂貞數上書譏詆朝政。昭度懼,稱疾,罷為太傅,致仕。行瑜、茂貞、韓建聯兵至闕下,言昭度伐蜀失謀,請貶之。未及報,而行瑜收昭度於都亭驛殺之。天子不得已,下詔暴其罪。行瑜誅,乃追復官爵,許其家收葬,贈太尉。



 張浚,字禹川,本河間人。性通脫無檢,泛知書史,喜高論,士友擯薄之。不得志,乃羸服屏居金鳳山,學從橫術,以捭闔乾時。樞密使楊復恭遇之,以處士薦為太常博士,進度支員外郎。黃巢之亂,稱疾,挾其母走商山。僖宗西出,衛士食不給,漢陰令李康獻糗餌數百馱,士皆厭給。帝異之,曰:「爾乃及是乎?」對曰:「臣安知為此,張浚教臣也。」乃急召浚至行在,再進諫議大夫。宰相王鐸任行營都統,奏署都統判官。



 時王敬武在平盧,軍最強,累召不肯應。浚往說之,而敬武已臣賊,不迎使者。浚責之曰:「公為天子守籓,今使者齎詔至,不北面俯伏而敢侮慢,公乃未識君臣大分,何以長吏民哉?」敬武愕眙愧謝。浚宣詔已,士按兵默默。浚召將佐至鞠場,倡言:「忠義之士當審利害。黃巢,販鹽虜耳。舍天子而臣之,何利邪?今諸侯勤王者踵相接,公等據一州以觀成敗,後賊平,將安往?誠能此時共誅大盜,迎天子,功名富貴可反手而取。吾憐公等舍安而蹈危也。」諸將雜然曰:「諫議語是!」敬武即引軍從浚西。擢浚為會軍使。賊平,以戶部侍郎判度支。後再狩山南,拜同中書門下平章事,仍判度支。



 浚始繇復恭進,復恭中失權,更依田令孜,故復恭銜之。及為中尉,數被離間。昭宗即位,復恭恃援立功,專任事,帝稍不平。當時多言浚有方略,善處大計,乃復見委信,嘗問致治之要,對曰:「在強兵。兵強,天下服矣。」天子繇是甘心於武功。後與論古今事,浚輒曰:「漢、晉之遠無可道,陛下春秋鼎富,天資英特,內逼宦臣,外迫強臣,故不能安。此臣所以痛心而泣血也。」



 是時,硃全忠威振關東,而安居受殺李克恭,以潞州歸全忠。全忠乃與幽州李匡威、雲州赫連鐸上言:「先帝幸梁,繇李克用與硃玫連和,請舉兵誅之,願帥兵為掎角。」帝詔文武四品以上議,皆言:「王室未寧,雖得太原,猶非所有。」浚固爭:「先帝時,身播屯亂,蓋克用、全忠不相下也。請因其弱討之,斷兩雄勢。」帝曰:「平巢,克用功第一。今乘危伐之,天下其謂我何?」久不決。孔緯曰:「浚言萬世之利,陛下所顧一時事爾。臣見師度河,賊必破。今軍中費尚足支數年,幸聽勿疑。」既浚、緯相倡和,帝乃決出師,詔浚為河東行營兵馬招討制置使,京兆尹孫揆為昭義節度使副之,韓建為供軍使;以全忠、匡威、鐸並為招討使,樞密使駱全諲為行營都監,以汴甲三千為帳下;發五十二軍,邠、寧、鄜、夏雜虜合五萬。帝置酒安喜樓臨餞,浚飲酣,泣下曰:「陛下逼於賊,臣願以死除之。」復恭聞不懌,率中尉等餞長樂阪,以酒屬浚,浚不肯舉。是役也,浚外幸成功,而內制復恭,故銜之。



 先是,汴、華、邠、岐兵絕河會平陽。汴將硃崇節已戍潞,浚慮汴人遂據有之,乃令揆分兵趨潞,以中人韓歸範持節護送至軍。會太原將李存孝方攻潞,揆至長子,為存孝所禽,汴人亦棄城去。浚次陰地關,諸軍壁平陽。存孝擊之,皆大北,委仗械去。浚斂眾夜遁,比明,軍失太半。存孝進掠晉、絳、慈、隰,其鋒甚盛。浚間道出王屋,奔河清,桴而濟,麾下略盡。全諲飲藥死,建遁去。克用上書請罪,其辭悖慢,因韓歸範以聞。朝廷震動,即日下詔罷浚為武昌軍節度使,三貶繡州司戶參軍。全忠為申請,詔聽使便。浚乃至藍田依韓建。及韋昭度死,復用緯為宰相,故浚亦拜兵部尚書,領天下租庸使。將復用,克用上言:「若朝以浚為相,暮請以兵見。」乃止。



 乾寧中,罷使,拜尚書右僕射。上疏乞骸骨,遷左僕射致仕,居洛長水墅。雖自屏處,然朝廷得失,時時言之。劉季述亂,浚徒步入洛,泣諭張全義,並致書諸籓,請謀王室之難。王師範起兵青州,欲取浚為謀主,不克。全忠脅帝東遷,浚聞曰:「乘輿卜洛,則大事去矣。」蓋知其將篡也。全忠畏浚構它鎮兵,使全義遣牙將如盜者夜圍墅殺之,屠其家百餘人,實天復二年十二月。



 始,浚素厚永寧史葉彥,彥知其謀,以告浚子格。浚度不免,父子相持泣曰:「留則俱死,不如去以存吾嗣。」格拜而辭,彥率士三十人送之,溯漢入蜀,後事王建。少子播,間道走淮南,依楊行密。時行密得承制除拜,播請每除吏,必紫極宮玄宗像前致制誥於案,乃出之,示不忘朝廷,且欲雪家冤而不克。終廣陵。



 贊曰:唐之季,嗣君暗庸,天穢其德久矣。纖人柄朝,靡謀不乖。如畋、鐸皆社稷之才,當大過之世,為天下倡。扶支王室,幾致中興。俄而為逆豎亂宦所乘,功業無所成就。浚以亂止亂,悖繆厥心,悲夫!



\end{pinyinscope}