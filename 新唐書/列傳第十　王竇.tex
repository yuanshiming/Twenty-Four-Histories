\article{列傳第十 王竇}

\begin{pinyinscope}

 王世充字行滿。祖西域胡,號支頹耨,後徙新豐,死,其妻與霸城人王粲為庶妻。頹耨子收從之子總結名辨思潮的發展成果,把名辯分為「名、辭、辨」三,冒粲姓,仕隋,歷懷、汴二州長史。生世充,豺聲卷發,忌刻深阻。涉書傳,喜兵法,通龜策、推步。以廕為左翊衛,遷御府直長、兵部員外郎。從楊素北伐,為幽州長史。



 大業初,為民部侍郎,善占對,習法,敢舞文上下。人或辨駁,世充以口舌緣飾,眾知其非,亦不能屈也。出為江都贊治,遷郡丞。煬帝數南幸,世充善伺帝顏色,阿意順旨。性機巧,飾臺沼、陰奏遠方珍物以媚帝,帝愛暱之,拜江都通守,兼知宮監事。



 世充觀隋政方亂,而江左浮剽易動,乃陰結豪桀,有系獄者,皆橈法貸減,以樹私恩。楊玄感反,吳人硃燮、晉陵人管崇起江南應之,兵十餘萬。隋將吐萬緒、魚俱羅討之不克,世充以偏將募江都萬人,頻擊破之。每捷必歸功於下。虜獲盡推與士卒,故人爭為效,由是功最多。



 大業十年,齊賊孟讓轉寇諸郡,至盱眙,世充拒之,保都梁山,列五壁不戰,羸兵以示弱。讓笑曰:「世充文法吏,安知兵?吾今生縛之,鼓行下江都矣!」時百姓皆入保,野無所掠,讓眾餧,又苦五壁閉道不得南,即分兵圍之。世充數戰,陽不利,走壁;讓益驕,數日,稍分其下南略,裁留兵足圍壁。世充知賊懈,夜夷灶撤幕,為方陣外向,毀垣而出,奮擊,大破之,讓以數十騎去,斬首萬級,虜十餘萬人。煬帝以世充有將帥略,復委捕諸盜,所向輒定。會突厥圍帝雁門,世充悉發江都兵赴難,詐為可喜事以邀聲譽。在軍蓬首垢面,日夜悲泣,不釋甲,臥必席槁。帝以為忠,愈屬信之。



 厭次賊格謙兵十餘萬屯豆子,太僕卿楊義臣殺謙,世充討其餘黨,夷之。進擊賊盧明月於南陽,俘系數萬。還,帝自持酒為勞。



 世充啟帝:「江淮良家女願備後廷,無繇進。」帝喜,令閱端麗者,以庫貲為聘,費不可校,署計簿云「敕別用」,有司不敢聞。具舟送東都宮,會道路剽奪,使者苦之,或沈舟亡去,世充屏不奏。



 李密逼東都,詔世充為將軍,以兵屯洛口。大小百餘戰,無大勝負。詔即拜右翊衛將軍,趣破賊。十四年,世充引軍與密戰洛南,有氣若城壓其營,世充大敗,眾幾盡,走保河陽。自系獄,請罪於越王侗,侗以書慰勉,賜金帛安之,召還洛,裒亡散得萬人,屯含嘉城,畏縮不敢出。



 會江都殺逆,群臣奉侗為帝,以世充為吏部尚書,封鄭國公。宇文化及擁兵北還,侗聽內史令元文都、盧楚等謀,以重官畀李密,使討賊,若化及破而密兵亦疲,乘其弊,可得志。乃遣使以太尉、尚書令即軍中拜密,趣兵北討。密稱臣奉制,引後從化及黎陽,戰勝來告,眾大悅;世充獨謂其下曰:「文都等刀筆才,必為密禽,且我軍與賊戰,多殺其父子兄弟,一旦為之下,吾屬無類矣!」以此言激眾,文都等聞,大懼。



 侗欲以文都為御史大夫,世充不許,曰:「嘗與公等約,左右僕射、尚書令、御史大夫,留待勛舊。今各欲得,則流競開矣,何以共守?」文都憾焉,潛與楚謀,因世充入殿伏甲殺之。納言段達庸怯,畏不果,馳告世充。世充夜以兵襲含嘉門,圍宮城。右武衛大將軍皇甫無逸等遣將費曜、田闍拒戰太陽門,曜敗,世充入之,無逸以單騎遁,收楚殺之。時紫微宮尚閉世充扣門,紿侗曰:「元文都等欲執陛下降李密,臣不反,誅反者耳。」段達執文都送世充,殺之。世充悉遣腹心代衛士,然後入謝曰:「文都、楚無狀,規相屠戮,臣急為此,非敢它。」侗與之盟,進拜尚書左僕射,總督內外諸軍事。乃去含嘉城,居尚書省,專宰朝政,以其兄世惲為內史令,居禁中,子弟皆將兵。分官吏為十頭,以主軍政。



 未幾,李密破化及,還屯金墉,勁兵良馬多死。世充欲擊之,恐士心未一,乃謀以鬼動眾,令德陽門衛張永通言夢人謂己曰:「我,周公,能以兵助討密。」世充白侗,立祠洛旁,使巫宣言:「周公令急擊密,有大功;不然,兵且疫。」世充下皆楚人,信妖,遂請戰。乃簡精卒二萬、騎二千,跨洛水為三橋以度兵。密軍偃師北山,新破敵,有輕世充心,不設壁壘。世充夜遣二百騎蔽山伏,因秣馬蓐食,遲明薄之,密陣未成,伏兵上北原,乘高馳下,壓其營,縱焚廬落,密眾大潰,降其將張童仁、陳智略,進拔偃師。初,密得世充兄世偉及子玄應於化及軍,囚之,至是皆歸。世充兵次洛口,密長史邴元真、司馬鄭虔象以城降,悉收美人、寶貨而還。密以數十騎跳奔。



 於是,世充自為太尉、尚書令,加黃門印綠綟綬,以尚書省為府,置官屬。乃設三榜於府外,其一求文學堪濟世務者,其一武幹絕眾、推鋒陷陣者,其一能治冤抑不申者。繇是上書陳事日數百,皆慰勞省接,雖吏卒,必飾詞誘納。而世充素詭妄,不能仇其語,士大夫遂貳。初,殺文都,欲詭眾取信,乃請事侗母劉太后為假子,至是加號聖感太后。散騎常侍崔德本曰:「此王莽文母何異乎?」後食侗前,得嘔疾,疑見毒,遂不復朝。以將張績、董浚衛宮城。



 武德二年,矯侗詔假黃鉞,相國總百揆,封鄭王,授九錫,冕十有二旒,建天子旌旗,金根車,駕六馬,備五時副車、旄頭雲罕,舞八佾,設宮縣,出入警蹕。術士桓法嗣自言能決讖,乃上《孔子閉房記》,畫男子持一竿驅羊狀,因說世充曰:「隋,楊姓也;於文,『乾一』為『王』,王處羊後,大王代隋之符。」又陳莊周《人間世》、《德充符》二篇曰:「上下篇與大王名協,明受符命,德被人間,為天子也。」世充喜曰:「天命也!」拜受之。以法嗣為諫議大夫。又羅取飛鳥,書符命於帛,系鳥頸縱之,有彈捕得鳥而獻者亦拜官。諷百官勸進。時納言蘇威老就第,世充以威隋大臣,有素望,每表必署威名。使段達等脅侗曰:「天命不常,今鄭王功德甚盛,請揖讓,用堯、舜故事。」侗怒曰:「天下者,高祖天下,若隋德未究,此言不可發。必天命遂改,尚何禪?公非先帝舊臣乎?朕何賴?」達等流涕。世充又詐曰:「天下未定,須鎮以長君,待天下安,則復子明闢。」



 四月,矯侗策禪位,幽侗於含涼殿,猶三讓。遣諸將以兵清宮,世充襲戎服,法駕,導鼓吹入宮,每歷一門,從者必呼。至東上閤,更兗冕,即正殿僭位。建元開明,國號鄭。乃封兄世衡為秦王,世偉楚王,世惲齊王,諸族屬以次封拜,以子玄應為皇太子,玄恕為漢王。世充每聽朝決政,誨喻言語諄復百緒,以示勤篤,百司奏事者聽受為疲。出則輕騎,無警蹕,游歷衢肆,行者但止立,徐謂百姓曰:「故時天子居九重,在下之情無繇察。世充非貪位者,本救時耳。正若一州刺史,事皆親覽,當與士人共議之。恐門衛有禁,無以盡通,今止順天門外置座聽事。」又詔西朝堂聽冤訴,東朝堂延諫者。繇是章牘真委,觀省不暇,後亦不能復出。



 五月,裴仁基與其子行儼及宇文儒童、崔德本等謀劫世充,復立侗,不克,夷三族。六月,鴆殺侗,以絕眾望。世充率眾東徇地至滑,以兵臨黎陽。時黎陽為竇建德守,故建德亦破世充殷州,以報其役。



 三年,下書大赦,築練兵臺於伊闕。守將羅士信、豆盧達稍稍歸國,世充顧下多背己,乃峻誅暴禁以威之。戶一人逃,家無少長皆坐,父子、兄弟、夫婦許相告免。令伍伍相保,一家叛,舉伍誅。樵牧出入皆為限,公私不聊生。遣臺省官督十二郡營田,行者自謂仙去。以宮城為大獄,意所猜惡,必收系其人,內家屬宮中。或命將,亦質其孥乃遣。既而囚質且萬口,食不足,餓死者日數十。



 七月,高祖詔秦王率兵攻之,至新安,屯保多下,敗世充於慈澗城。八月,王陳兵青城宮,世充悉精兵來拒,隔澗言曰:「隋失其國,天下分崩,長安、洛陽各有分地,吾常自守,不敢西顧。熊、穀二州在度內,不取,敦鄰好也。今王遠涉吾地,越三崤,饋糧千里,勤師遠出,將何求?」王曰:「四海之人皆承唐正朔,獨公迷不復。東都士民來請師,陛下重違,我是以來。公若降,富貴可保;必拒我,勉之,無多言!」世充約割地,不許。潁州總管田瓚請舉山南二十五郡歸。九月,王君廓進拔軒轅,徇地至管城,河南州縣以次降定。始竇建德與世充隙,至是建德遣使結好,並陳赴援意。世充遣兄子琬、內史令長孫安世報,且乞師。



 四年二月,青城宮守將以宮降,王進保之。世充引兵出方諸門,臨谷水以戰,王陣北邙,令屈突通步士五千逾水擊之。兵接,王以騎決戰,世充排兵殊死鬥,自辰及午乃潰,俘斬八千人。王傅城,塹而守之。世充糧且盡,人相食,至以水汨泥去礫,取浮土糅米屑為餅。民病腫股弱,相藉倚道上,其尚書郎盧君業、郭子高等皆餓死。御史大夫鄭頲丐為浮屠,世充惡其言,殺之。然氣竭,但嬰城須建德之救。



 五月,王禽建德,並獲王琬、長孫安世,俘示東都城下,且遣安世入言敗狀。世充惶惑,將突圍出保襄、漢,謀於諸將,皆不答,遂率將吏降軍門。王受之,以屬吏,陳兵入城,發府庫賚將士。其黃門侍郎薛德音以移檄嫚逆,崔弘丹造弩多傷士,前誅之;又收段達、楊汪、孟孝義、單雄信、楊公卿、郭士衡、郭什柱、董浚、張童仁、硃粲、王德仁等斬洛渚上。以世充歸長安,高祖數其罪,世充曰:「計臣罪不容誅,但秦王許臣以不死。」乃赦為庶人,與其族徙於蜀。將行,為羽林將軍獨孤修德所殺。初,修德父機嘗仕越王侗,世充既篡,謀歸唐,為所屠者也。高祖免修德官。子玄應,兄世偉,在道謀反,伏誅。世充篡,凡三年滅。



 竇建德,貝州漳南人。世為農,自言漢景帝太后父安成侯充之苗裔。材力絕人,少重然許,喜俠節。鄉人喪親,貧無以葬,建德方耕,聞之太息,遽解牛與給喪事,鄉黨異之。盜夜劫其家,建德立戶下,盜入,擊三人死,餘不敢進。請其尸,建德曰:「可投繩系取之。」盜投繩,建德乃自縻,使盜曳出,躍起捉刀,復殺數人,繇是益知名。為里長,犯法亡,會赦歸。久之,父卒,里中送葬千餘人,所贈予皆讓不受。



 隋大業七年,募兵伐遼東,建德補隊長。方如軍,會邑人孫安祖盜羊,為縣令捕劾笞辱,安祖刺殺令,亡抵建德,建德陰舍之。時山東饑,群盜起,乃謀曰:「往文皇帝時,天下盛強,發百萬眾伐遼東,猶為所敗。今水潦為災,民力刓敝,主上不是恤,而親駕臨遼。且往歲西征,十不一返,今創夷未平,又重發兵,人情危駭,易以搖動。丈夫不死,常建功於世,渠為亡命虜乎!我聞高雞泊廣袤數百里,葭薍〗阻奧,可以違難;承間竊出,椎埋掠奪,足以自資。因得聚豪傑,且觀時變,以就大計。」安祖然之。建德為招亡兵及民無產者數百,使安祖率之,入高雞為盜,安祖號「摸羊公」。



 時鄃人張金稱亦結眾萬餘,依河渚間,蓚人高士達兵千餘屯清河鄙上。諸盜往來漳南者多剽殺人,焚鄉聚,獨不入建德閭,郡縣意建德與賊通,捕族其家。建德至河間,聞家屠滅,即率麾下二百人亡歸士達。士達自稱東海公,以建德為司兵。安祖為金稱所殺,其下數千人歸建德,眾益盛,至萬人,猶保高雞泊。然傾身接物,其執苦與士卒均,由是能致人死力。



 十二年,涿郡通守郭絢率兵萬人討士達,士達自以智略不及建德,乃推為軍司馬,以兵屬焉。建德既統眾,思用奇厭伏群盜,乃請士達守輜重,自以精兵七千迎絢,詐為亡狀。士達取所虜,陽言建德妻子,殺之。建德遺絢書約降,請前驅執賊自效。絢信之,引兵從建德至長河界,欲與盟,兵懈不設備。建德襲殺其軍數千人,獲馬千匹,絢以數十騎去,追斬於平原,獻首士達,威振山東。



 隋遣太僕卿楊義臣討破張金稱於清河,殘黨畏誅,復屯嘯歸建德。義臣乘勝欲遂入高雞泊,窮劃根穴。建德謂士達曰:「隋善將獨義臣耳,新破金稱,其鋒不可當。宜引兵避之,彼欲戰不得,軍老食乏,乘之可有功。」士達不納。留建德守壁,身將兵逆戰,置酒享士。建德聞,曰:「東海公未捷,遽自矜大,禍至不日矣。隋兵勝,必長驅而來,吾不能獨支。」乃留眾保壁,帥銳士據險待。後五日,義臣斬士達於陣,追北薄壘,守兵潰。建德不能軍,以百餘騎走饒陽,饒陽無備,因取之。義臣已殺士達,謂餘黨不足憂,引去。故建德得還平原,收士達士死胔葬焉。為士達發喪,軍皆縞素。招潰卒,得數千人,軍復振,自稱將軍。初,他盜得隋官及士人必殺之,唯建德恩遇甚備,引故饒陽長宋正本為客,尊任之,參決軍議。隋郡縣吏多以地歸之,勢益張,兵至十餘萬。上谷賊王須拔自號「漫天王」,以兵略幽州,戰死。其下魏刀兒號「歷山飛」,壁深澤,眾十萬。建德以計襲取之,並有其地。



 十三年正月,築壇場於河間樂壽,自立為長樂王。



 十四年五月,更號夏王,建元丁丑,署官屬,分治郡縣。



 七月,隋右翊衛將軍薛世雄督兵三萬討之,屯河間七里井,建德以勁兵伏旁澤中,悉拔諸城偽遁。世雄以為畏,稍弛備,建德率敢死士千人襲之。會大霧晝冥,跬不可視,隋軍驚,遂潰,相騰藉,死者如丘,世雄引數百騎亡去。盡得其眾,獲河間丞王琮,勞遣之。琮復嬰城,建德進攻未下,而河間食盡,聞煬帝遇殺,琮率吏發喪,乘城大臨,建德遣使入吊,琮因請降。建德為退舍,飭饌具。琮率郡屬素服面縛軍門,建德親釋徽纆,與言隋之亡,琮伏哭極哀,建德亦為泣。麾下或言:「河間久拒守,多殺士,今力窮而下,請烹之。」建德曰:「琮,誼士也,吾方旌擢以勵事君者。且往為盜,可妄殺人,今將安百姓,定天下,而害忠臣乎?」即令其軍曰:「與琮隙者敢輒搖,罪三族!」乃授琮瀛州刺史。



 始都樂壽,號金城宮,備百官,準開皇故事。冬至,大會僚吏,有五大鳥集其宮,群鳥從之。又宗城人獻玄圭一,景城丞孔德紹曰:「昔天以是授禹,今瑞與之侔,國宜稱夏。」建德然之。改元五鳳,以德紹為內史侍郎。



 武德元年,宇文化及至魏縣,建德謂其納言宋正本及德紹曰:「吾,隋民也;隋,吾君也。今化及殺之,大逆不道,乃吾仇,欲為天下誅之,何如?」正本等曰:「大王奮布衣,起漳南,隋之列城莫不爭附者,以能杖順扶義、安四方也。化及為隋姻里,倚之不疑,今戕君而移其國,仇不共天,請鼓行執其罪。」建德善之。即引兵討化及,連戰破之。化及保聊城,乃縱撞車機石,四面乘城,拔之。建德入,先謁蕭皇后,語稱臣。執宇文智及、楊士覽、元武達、許弘仁、孟景等,召隋文武官共臨斬之,梟首轅門;囚化及並其子,載以檻車,至大陸縣斬之。



 建德性約素,不喜食肉,飯脫粟加蔬具,妻曹未嘗衣紈綺。及為王,妾侍裁十數。每下城破敵,貲寶並散賚將士。至是,得隋宮人尚千數,悉放去;其文武、驍果尚萬餘,各聽所之。乃以誅化及報越王侗,侗封之夏王,遂號大夏。以隋黃門侍郎裴矩為尚書右僕射,兵部侍郎崔君肅為侍中,少府令何稠為工部尚書,餘隨才署職,委以政事。有願往關中及東都者,恣聽不留,仍給道里費,以兵護出於境。



 二年,陷邢、趙、滄三州。復陷冀州,執刺史曲棱,赦之,復以為刺史。八月,陷洺州,虜刺史袁子干,遂遷都焉,更號萬春宮。使人如灌津祠先墓,置守塚三十家。又遣使朝侗,因與王世充結歡,北聘突厥,士馬益精雄。俄而世充廢侗,乃絕之。始建天子旌旗,出入警蹕,書稱詔。追謚隋煬帝為閔帝,以齊王暕子政道為鄖公。義成公主在突厥,遣使迎蕭後,建德自將千餘騎送之,並獻化及首。



 未幾,連突厥侵相州,刺史呂氏死之。進攻衛州,執河北大使淮安王神通、同安長公主、黎陽守將李世勣,釋之。復使世勣守黎陽,館王、公主,饋以客禮。滑州刺史王軌為奴所殺,奴以首奔建德,建德曰:「奴殺主,大逆。納之不可不賞,賞逆則廢教,將焉用為?」命斬奴而返軌首,滑人德之,遂降,齊、濟二州亦降。兗賊徐圓朗聞風送款。



 三年,世勣自拔歸國,吏白建德誅其父,建德曰:「臣勣,唐臣,不忘其主,忠也。父何罪?」釋不問。高祖遣使修好,建德即以公主等歸京師。嘗執趙州刺史張志昂、邢州刺史陳君賓、大使張道源等,將殺之,國之祭酒凌敬諫曰:「夫犬吠非其主,彼悉力堅守,以窮就禽,伏節士也。今殺之,無以勸。」建德怒曰:「我傅其城,猶不下,勞費士旅,何可赦?」敬曰:「王之大將高士興抗羅藝於易南,兵未交,士興即降,王以為可乎?」建德悟,即釋之。然其大將王伏寶數持兵,功略在諸帥上,或讒其反,建德殺之。伏寶臨死呼曰:「我無罪,王何信讒,自刈左右手乎?」後戰數不利。



 九月,建德自帥師圍幽州,為羅藝所敗,藝乘勝襲其營,建德陣營中,填塹而出,敗藝眾,進薄其城,不能拔,乃還。濟陰賊孟海公兵三萬,據周橋城以掠河南,建德自擊之。會秦王伐東都,其中書舍人劉斌獻說曰:「唐據關內,鄭王河南,夏有冀方,此鼎足相持勢也。今唐悉兵臨鄭,出入二年,鄭人日蹙。二國兵不解,唐強鄭弱,勢必舉鄭,鄭滅則大夏有齒寒之憂。為大王計,莫若援鄭,使鄭抗其內,我攻其外,唐之兵必卻,唐卻而鄭完,然後徐觀其變。鄭若可圖,因而取之,並二國兵,乘唐師老,長驅而西,關中可遂有也。」建德曰:「善。」乃遣使聘世充,與連和,會世充亦自乞師,即令其臣李大師、魏處繪來朝,請解鄭圍,秦王留之不答。



 四年,建德克周橋,虜海公,留其將範願戍之。悉發海公、徐圓朗之眾,並兵號三十萬救世充,至滑州,世充行臺僕射韓弘開城納之。建德進逼元、梁、管三州,皆陷,遂屯滎陽。運糧溯河西上,舟相屬不絕。壁成皋東原,築營板渚。遣使與世充約期,又遺秦王以書。



 三月,王進據虎牢。翌日,以騎五百覘建德營,設伏道側,獨以數騎去賊營三里,覺,賊出騎追之,王漸卻,誘至伏所,卒起奮擊。賊騎驚,引去,追斬三百級,獲其將殷秋、石瓚,乃報建德以書。建德失二將,又聞唐兵精,得書猶豫,頓六十日不敢西。



 時世充弟世辯為徐州行臺,亦遣將郭士衡、兵數千人從建德,王遣王君廓以輕騎抄其饟,執賊大將張清特。建德懼,人情攜駭,其諸將又新破海公,掠獲盈給,日夜思歸。凌敬說建德曰:「今唐以重兵圍東都,守虎牢,我若悉兵濟河,取懷州河陽,以重將戍之,然後鳴鼓建旗,逾太行,入上黨,傳檄旁郡,進壺口以駭蒲津,收河東地,此上策也。且有三利:乘虛手壽境,師有萬全,一也;拓土得眾,二也;鄭圍自解,三也。」建德將從之,而王琬、長孫安世日請兵西,每言必流涕,又陰齎金玉啗諸將,以撓其謀。眾乃曰:「凌敬書生,豈知戰?」建德乃謝曰:「今士心銳,天贊我也,師將大捷。方用眾議,不得如公言。」敬固爭,建德怒,命扶出。其妻諫曰:「祭酒計甚善,王盍用之?夫自滏口道乘唐之虛,連營漸進以取山北,因招突厥西抄關中,唐必還師自救,鄭難紓矣。今頓兵虎牢下,徒自苦,恐無功。」建德曰:「此非女子所知。且鄭朝暮待吾來,既許之,豈可見難而退,且示天下不信。」



 五月,建德自板渚出為陣,西薄汜南,屬鵲山,亙二十里,鼓而前。郭士衡為游兵。秦王登虎牢城望其軍,按甲不戰,曰:「賊起山東,未嘗見大敵,今度險士囂,令不肅也;逼城而陣,有輕我心。待其饑,破之果矣。」日中,建德士皆坐列,渴爭飲,意益怠。王麾軍先登,騎怒,塵大漲,乃率史大奈、秦叔寶纏麾幟,弛出賊陣後,建德軍顧而驚,遂大潰。建德被重創,竄牛口谷。車騎將軍白士讓、楊武威獲之,傳而西,斬長安市,年四十九。初,其軍有謠曰:「豆入牛口,勢不得久。」至是果敗。



 建德妻與其左僕射齊善行以騎數百遁還洺州。餘黨欲立其養子為主,善行曰:「夏王奄定河朔,號為威強,今一出不復,非天命有歸哉?不如委心請命,無為塗炭生民也!」遂分府庫散給將士,令各解去。善行乃與右僕射裴矩、行臺曹旦率官屬及建德妻奉山東地並傳國八璽來降。建德起兵至滅凡六年。



 贊曰:煬帝失德,天醜其為,生人辜,群盜乘之,如胃毛而奮。其劇者,若李密因黎陽,蕭銑始江陵,竇建德連河北,王世充舉東都,皆磨牙搖毒以相噬螫。其間亦假仁義,禮賢才,因之擅王僭帝,所謂盜亦有道者。本夫孽氣腥焰,所以亡隋,觸唐明德,折北不支,禍極兇殫,乃就殲夷,宜哉!



\end{pinyinscope}