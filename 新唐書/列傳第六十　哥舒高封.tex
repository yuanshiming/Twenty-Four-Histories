\article{列傳第六十 哥舒高封}

\begin{pinyinscope}

 哥舒翰,其先蓋突騎施酋長哥舒部之裔。父道元,為安西都護將軍、赤水軍使,故仍世居安西。翰少補效轂府果毅,家富於財形而下之器也,生物之具也」(《朱文公文集》)。王夫之主張,任俠重然諾,縱蒱酒長安市。年四十餘,遭父喪,不歸。不為長安尉所禮,慨然發憤,游河西,事節度使王倕。倕攻新城,使翰經略,稍知名。又事王忠嗣,署衙將。翰能讀《左氏春秋》、《漢書》,通大義。疏財,多施予,故士歸心。為大斗軍副使,佐安思順,不相下。忠嗣更使討吐蕃,副將倨見,翰怒,立殺之,麾下為股抃。遷左衛郎將。



 吐蕃盜邊,與翰遇苦拔海。吐蕃枝其軍為三行,從山差池下,翰持半段槍迎擊,所向輒披靡,名蓋軍中。擢授右武衛將軍,副隴右節度,為河源軍使。先是,吐蕃候積石軍麥熟,歲來取,莫能禁。翰乃使王難得、楊景暉設伏東南谷。吐蕃以五千騎入塞,放馬褫甲,將就田,翰自城中馳至鏖斗,虜駭走,追北,伏起,悉殺之,只馬無還者。翰嘗逐虜,馬驚,陷於河,吐蕃三將欲刺翰,翰大呼,皆擁矛不敢動,救兵至,追殺之。翰有奴曰左車,年十六,以膂力聞。翰工用槍,追及賊,擬槍於肩,叱之,賊反顧,翰刺其喉,剔而騰之,高五尺許,乃墮,左車即下馬斬其首,以為常。



 會忠嗣被罪,帝召翰入朝,部將請齎金帛以救忠嗣,翰但齎樸裝,曰:「使吾計從,奚取於是?不行,用此足矣。」翰至,帝虛心待,與語,異之,拜鴻臚卿,為隴右節度副大使。翰已謝,即極言忠嗣之枉。帝起入禁中,翰叩頭從帝,且泣。帝寤,為末貸其罪,忠嗣不及誅。朝廷稱其義。



 逾年,築神威軍青海上,吐蕃攻破之。更築於龍駒島,有白龍見,因號應龍城。翰相其川原宜畜牧,謫罪人二千戍之,由是吐蕃不敢近青海。天寶八載,詔翰以朔方、河東群牧兵十萬攻吐蕃石堡城。數日未克,翰怒,捽其將高秀巖、張守瑜,將斬之。秀巖請三日期,如期而下。遂以赤嶺為西塞,開屯田,備軍實。加特進,賜賚彌渥。十一載,加開府儀同三司。



 翰素與安祿山、安思順不平,帝每欲和解之。會三人俱來朝,帝使驃騎大將軍高力士宴城東,翰等皆集。詔尚食生擊鹿,取血瀹腸為熱洛河以賜之。翰母,於闐王女也。祿山謂翰曰:「我父胡,母突厥;公父突厥,母胡。族類本同,安得不親愛?」翰曰:「諺言『狐向窟嗥,不祥』,以忘本也。兄既見愛,敢不盡心。」祿山以翰譏其胡,怒罵曰:「突厥敢爾!」翰欲應之,力士目翰,翰托醉去。



 久之,進封涼國公,兼河西節度使。攻破吐蕃洪濟、大莫門等城,收黃河九曲,以其地置洮陽郡,築神策、宛秀二軍。進封西平郡王,賜音樂、田園,又賜一子五品官,裨將賞拜有差。宰相楊國忠惡祿山,白發其反狀,故厚結翰。俄進太子少保。翰耆酒,極聲色,因風痺,體不仁。既疾廢,遂還京師,闔門不朝請。



 十四載,祿山反,封常清以王師敗。帝乃召見翰,拜太子先鋒兵馬元帥,以田良丘為軍司馬,蕭昕為判官,王思禮、鉗耳大福、李承光、高元蕩、蘇法鼎、管崇嗣為屬將,火拔歸仁、李武定、渾萼、契苾寧以本部隸麾下,凡河、隴、朔方、奴刺等十二部兵二十萬守潼關。師始東,先驅牙旗觸門,墮注旄,幹折,眾惡之。天子御勤政樓臨送,詔翰以軍行,過門毋下,百官郊餞,旌旗亙二百里。翰惶恐,數以疾自言,帝不聽。然病痼不能事,以軍政委良丘,使王思禮主騎,李承光主步。三人爭長,政令無所統一,眾攜弛,無鬥意。明年,進拜尚書左僕射、同中書門下平章事。祿山遣子慶緒攻關,翰擊走之。



 始,安思順度祿山必反,嘗為帝言,得不坐。翰既惡祿山,又怨思順。及是,知重兵在己,有所論請,天子重違,因偽為賊書遺思順者,使關邏禽以獻。翰因疏七罪,請誅之。有詔思順及弟元貞皆賜死,徙放其家。國忠始懼。或說翰曰:「祿山本以誅國忠故稱兵,今若留卒三萬守關,悉精銳度滻水誅君側,此漢挫七國計也。」思禮亦勸翰。翰猶豫未發,謀頗露。國忠大駭,入見帝曰:「兵法,安不忘危。大兵在潼關而無後殿,萬有一不利,京師危矣。」即募牧兒三千人,日夜訓練,以劍南列將分統之。又募萬人屯灞上,使腹心杜乾運為帥。翰疑圖己,表請乾運兵隸節下,因詭召乾運計事者,至軍,即斬首梟牙門,並其軍。國忠愈恐,謂其子曰:「吾無死所矣!」然翰亦不自安,又謀久不決。數奏言:「祿山雖竊據河朔,不得人心,請持重以敝之,待其離隙,可不血刃而禽。」賊將崔乾祐守陜郡,僕旗鼓,羸師以誘戰。覘者曰:「賊無備,可圖也。」帝信之,詔翰進討。翰報曰:「祿山習用兵,今始為逆,不能無備,是陰計誘我。賊遠來,利在速戰。王師堅守,毋輕出關,計之上也。且四方兵未集,宜觀事勢,不必速。」



 當是時,祿山雖盜河、洛,所過殘殺,人人怨之,淹時月不能進尺寸地。又郭子儀、李光弼兵益進,取常山十數郡。祿山始悔反矣,將還幽州以自固。而國忠計迫,謬說帝趣翰出潼關復陜、洛。時子儀、光弼遙計曰:「翰病且耄,賊素知之,諸軍烏合不足戰。今賊悉銳兵南破宛、洛,而以餘眾守幽州,吾直搗之,覆其巢窟,質叛族以招逆徒,祿山之首可致。若師出潼關,變生京師,天下怠矣。」乃極言請翰固關無出軍。而帝入國忠之言,使使者趣戰,項背相望也。翰窘不知所出。六月,引而東,慟哭出關,次靈寶西原,與乾祐戰。由關門七十里,道險隘,其南薄山,北阻河,賊以數千人先伏險。翰浮舟中流以觀軍,謂乾祐兵寡,易之,促士卒進,道岨無行列。賊乘高頹石下擊,殺士甚眾。翰與良丘登北阜,以軍三萬夾河鳴鼓,思禮等以精卒居前,餘軍十萬次之。乾祐為陣,十十五五,或卻或進,而陌刀五千列陣後。王師視其陣無法,指觀嗤笑,曰:「禽賊乃會食。」



 及戰,乾祐旗少偃,如欲遁者,王師懈,不為備。伏忽起薄戰,皆奮死鬥。翰以氈蒙馬車,畫龍虎,飾金銀爪目,將駭賊,掎戈矢逐北。賊負薪塞路,順風火其車,熛焱熾突,騰煙如夜,士不復相辨,自相鬥殺,尸血狼籍,久乃悟。又棄甲奔山谷及陷河死者十一二。有糧艘百餘,軍爭濟,艘輒沉,至縛矛盾乘以度,喧叫振天地。賊乘之,奔潰略盡。始,斗門有三塹,廣二丈,深一丈,士馬奔籥相壓迮,少選塹平,後至者踐之以入。



 既敗,翰引數百騎絕河還營,羸兵裁八千,至潼津,收散卒復守關。乾祐進攻,於是火拔歸仁等紿翰出關,翰曰:「何邪?」曰:「公以二十萬眾,一日覆沒,持是安歸?公不見高仙芝等事乎?」翰曰:「吾寧效仙芝死,汝舍我。」歸仁不從,執以降賊,械送洛陽,京師震動,由是天子西幸。祿山見翰責曰:「汝常易我,今何如?」翰俯伏謝罪曰:「陛下撥亂主。今天下未平,李光弼在土門,來瑱在河南,魯炅在南陽,臣為陛下以尺書招之,三面可平。」祿山悅,即署司空、同中書門下平章事。執火拔歸仁,曰:「背主忘義,吾不爾容。」斬之。翰以書招諸將,諸將皆讓翰不死節。祿山知事不可就,囚之。東京平,安慶緒以翰度河。及敗,乃殺之。



 翰為人嚴,少恩。軍行未嘗恤士饑寒,有啗民椹者,痛笞辱之。監軍李大宜在軍中,不治事,與將士樗蒱、飲酒、彈箜篌琵琶為樂,而士米籺不饜。帝令中人袁思藝勞師,士皆訴衣服穿空,帝即斥御服餘者,制袍十萬以賜其軍,翰藏庫中,及敗,封鐍如故。



 先是,有客梁慎初遺翰書,請壁勿戰以屈賊,翰善之,奏為左武衛胄曹參軍,留幕府。及翰與國忠貳,慎初曰:「難將作矣。」乃遁去。翰失守,華陰、馮翊、上洛郡官吏皆潰。帝遣劍南將劉光庭等將新募兵萬餘人往助翰,未至而翰被縛云。其後贈太尉,謚曰武愍。



 子曜,字子明。八歲,玄宗召見華清宮,擢尚輦奉御。累遷光祿卿。以翰陷賊,哀憤號慟,故吏裴冕、杜鴻漸等見之嘆息。李光弼討河北,曜請行,拜鴻臚卿,為光弼副。降安太清、救宋州有功,改殿中監,襲封,為東都鎮守兵馬使。德宗立,召為左龍武大將軍。李希烈陷汝州,以周晃為偽刺史。詔拜曜東都、汝州行營節度使,將鳳翔、邠寧、涇原、奉天、好畦兵萬人討希烈。帝召見,問曰:「卿治兵孰與父賢?」對曰:「先臣,安敢比。但斬長蛇,殪封豕,然後待罪私室,臣之願也。」帝曰:「爾父在開元時,朝廷無西憂;今朕得卿,亦不東慮。」及行,帝祖通化門。是日,牙幹折。時以翰出師已如此,而斬持旗者,卒以敗,今曜復爾,人憂之。曜擊賊,收汝州,禽晃以獻,斬其將二人。希烈退保許州。詔城襄城,曜以疲人版築不如按甲持重以挫之,帝不許。有詔督戰。曜進次潁橋,雷震軍中七馬斃,曜懼,還屯襄城。希烈遣眾萬人縱火攻柵,殪人於塹以薄壘,曜苦戰破之。居數月,希烈自率兵三萬圍曜,築甬道屬城,矢集如雨。帝遣神策將劉德信以兵三千援之,又詔河南都統李勉出兵相掎角。勉以「希烈在外,許守兵少,乘虛襲之,希烈自解」,乃遣部將與德信趨許,未至,有詔切讓,使班師。德信等惶惑還,軍無斥候,至扈澗,為賊設伏詭擊,死者殆半,器械輜重皆亡。德信走汝州。勉恐東都危,使將李堅華以兵四千往守,賊梗道,不得入。汴兵沮,襄城圍益急。帝乃詔普王以荊、襄、江西、鄂、沔之師討蔡州,詔涇原節度使姚令言救襄城。未行,京師亂,帝幸奉天。襄城陷,曜走洛陽。會母喪,奪為東都畿、汝節度使。遷河南尹。曜拙於統御,而銳殺戮,士畏而不懷。貞元元年,部將叛,夜焚河南門,曜挺身免。帝以汴州刺史薛玨代之,召入為鴻臚卿。終右驍衛上將軍,贈幽州大都督。子七人,俱以儒聞。峘,茂才高第,有節概。崿、嵫、屺皆明經擢第。



 高仙芝,高麗人。父舍雞,初以將軍隸河西軍,為四鎮校將。仙芝年二十餘,從至安西,以父功補游擊將軍。數年,父子並班。仙芝美姿質,善騎射,父猶以其儒緩憂之。初事節度使田仁琬、蓋嘉運等,不甚知名。後事夫蒙靈察,乃善遇之。開元末,表為安西副都護、四鎮都知兵馬使。



 小勃律,其王為吐蕃所誘,妻以女,故西北二十餘國皆羈屬吐蕃。自仁琬以來三討之,皆無功。天寶六載,詔仙芝以步騎一萬出討。是時步兵皆有私馬自隨,仙芝乃自安西過撥換城,入握瑟德,經疏勒,登蔥嶺,涉播密川,遂頓特勒滿川,行凡百日。特勒滿川,即五識匿國也。仙芝乃分軍為三,使疏勒趙崇玼自北谷道、撥換賈崇瓘祐自赤佛道、仙芝與監軍邊令誠自護蜜俱入,約會連雲堡。堡有兵千餘。城南因山為柵,兵九千守之。城下據婆勒川。會川漲,不得度,仙芝殺牲祭川,命士人齎三日備集水涯,士不甚信。既涉,旗不沾,韉不濡。兵已成列,仙芝喜,告令誠曰:「向吾方涉,賊擊我,我無類矣。今既濟而陣,天以賊賜我也。」遂登山挑戰,日未中,破之。拔其城,斬五千級,生禽千人,馬千餘匹,衣資器甲數萬計。仙芝欲遂深入,令誠懼,不肯行。仙芝留羸弱三千使守,遂引師行。三日,過坦駒嶺,嶺峻絕,下四十里。仙芝恐士憚險不敢進,乃潛遣二十騎,衣阿弩越胡服來迎,先語部校曰:「阿弩越胡來迎,我無慮矣。」既至,士不肯下,曰:「公驅我何去?」會二十人至,曰:「阿努越胡來迎,已數娑夷橋矣。」仙芝即陽喜,令士盡下。娑夷河,弱水也。既行三日,越胡來迎。明日,至阿弩越城。遣將軍席元慶以精騎一千先往,謂小勃律王曰:「不窺若城,吾假道趨大勃律耳。」城中大酋領皆吐蕃腹心,仙芝密令元慶曰:「若酋領逃者,弟出詔書呼之,賜以繒彩,至,皆縛以待我。。」元慶如言。仙芝至,悉斬之。王及妻逃山穴,不可得,仙芝招喻,乃出降,因平其國。急遣元慶斷娑夷橋,其暮,吐蕃至,不克度。橋長度一箭所及者,功一歲乃成。八月,仙芝以小勃律王及妻自赤佛道還連雲堡,與令誠俱班師。於是拂菻、大食諸胡七十二國皆震懾降附。



 仙芝遣判官王庭芬奏捷京師。軍至河西,靈察怒,不迎勞。既見,罵曰:「高麗奴,於闐使爾何從得之?」仙芝懼,且謝曰:「中丞力也。」又曰:「焉耆鎮守使、安西副都護、都知兵馬使,皆何從得之?」答曰:「亦中丞力也。」靈察曰:「審若此,捷書不待我而敢即奏,何邪?奴當斬,顧新立功,故貸爾。」仙芝不知所為。令誠密言狀於朝,且曰:「仙芝立功而以憂死,後孰為朝廷用者?」帝乃擢仙芝鴻臚卿、假御史中丞,代靈察為四鎮節度使,而詔靈察還,靈察懼。仙芝朝夕見,輒趨走,靈察益慚。副都護程千里、衙將畢思琛、行官王滔康懷順陳奉忠等皆嘗譖仙芝於靈察者。既視事,呼千里嫚罵曰:「公面雖男兒,而心似婦女,何邪?」謂琛曰:「爾奪吾城東千石種田,憶之乎?」對曰:「公見賜者。」仙芝曰:「爾時吾畏汝威,豈憐汝而賜邪?」又召滔,欲捽辱。良久,皆釋,曰:「吾不恨矣。」由是舉軍安之。俄加左金吾衛大將軍,與一子五品官。



 九載,討石國,其王車鼻施約降,仙芝為俘獻闕下,斬之,由是西域不服。其王子走大食,乞兵攻仙芝於怛邏斯城,以直其冤。仙芝為人貪,破石,獲瑟瑟十餘斛、黃金五六橐駝、良馬寶玉甚眾,家貲累鉅萬。然亦不甚愛惜,人有求輒與,不問幾何。尋除武威太守,代安思順為河西節度使,群胡固留思順,更拜右羽林軍大將軍,封密雲郡公。祿山反,榮王為元帥,仙芝副之,領飛騎、彍騎及朔方等兵,出禁財募關輔士五萬,繼封常清東討。帝御勤政樓,引榮王受命,宴仙芝以下。帝又幸望春亭勞遣,詔監門將軍邊令誠監軍。次陜郡,而常清敗還。仙芝急,乃開太原倉,悉以所有賜士卒,焚其餘,引兵趨潼關。會賊至,甲仗資糧委於道,彌數百里。既至關,勒兵繕守具,士氣稍稍復振。賊攻關不得入,乃引還。



 初,令誠數私於仙芝,仙芝不應,因言其逗撓狀以激帝,且云:「常清以賊搖眾,而仙芝棄陜地數百里,朘盜稟賜。」帝大怒,使令誠即軍中斬之。令誠已斬常清,陳尸於蘧祼。仙芝自外至,令誠以陌刀百人自從,曰:』大夫亦有命。」仙芝遽下,曰:「我退,罪也,死不敢辭。然以我為盜頡資糧,誣也。」謂令誠曰:「上天下地,三軍皆在,君豈不知?」又顧麾下曰:「我募若輩,本欲破賊取重賞,而賊勢方銳,故遷延至此,亦以固關也。我有罪,若輩可言;不爾,當呼枉。」軍中咸呼曰:「枉!」其聲殷地。仙芝視常清尸曰:「公,我所引拔,又代吾為節度,今與公同死,豈命歟!」遂就死。



 封常清,蒲州猗氏人。外祖教之讀書,多所該究。然孤貧,年過三十,未有名。夫蒙靈察為四鎮節度使,以高仙芝為都知兵馬使。嘗出軍,奏傔從三十餘人,衣褷鮮明,常清慨然投牒請豫。常清素瘠,又腳跛,仙芝陋其貌,不納。明日復至,仙芝謝曰:「傔已足,何庸復來?」常清怒曰:「我慕公義,願事鞭靮,故無媒自前,公何見拒深乎?以貌取士,恐失之子羽。公其念之。」仙芝猶未納,乃日候門下,仙芝不得已,竄名傔中。



 會達奚諸部叛,自黑山西趣碎葉,有詔邀擊。靈察使仙芝以二千騎追躡。達奚行遠,人馬疲,禽馘略盡。常清於幕下潛作捷布,具記井泉次舍、克賊形勢謀略,條最明審。仙芝取讀之,皆意所欲出,乃大駭,即用之。軍還,靈察迎勞,仙芝已去奴襪帶刀,而判官劉眺、獨孤峻爭問:「向捷布誰作者?公幕下安得此人?」答曰:「吾傔封常清也。」眺等驚,進揖常清坐,與語,異之,遂知名。以功授疊州戍主,仍為判官。仙芝破小勃律,代靈察為安西節度使,常清以從戰有勞,擢慶王府錄事參軍事,為節度判官。仙芝征討,常知後務。常清才而果,胸無疑事。仙芝委家事於郎將鄭德詮,其乳母子也,威動軍中。常清嘗自外還,諸將前謁。德詮見常清始貴,易之,走馬突常清騶士去。常清命左右引德詮至廷中,門輒閉,因離席曰:「吾起細微,中丞公過聽,以主留事,郎將安得無禮?」因叱曰:「須暫假郎將死,以肅吾軍。」因杖死,以面僕地曳出之。仙芝妻及乳母哭門外救請,不能得,遽以狀白仙芝,仙芝驚,及見常清,憚其公,不敢讓。常清亦不謝。會大將有罪,又殺二人,軍中莫不股慄。仙芝節度河西,復請為判官。久之,擢安西副大都護、安西四鎮節度副大使,知節度事。未幾,改北庭都護,持節伊西節度使。常清性勤儉,耐勞苦,出軍乘騾,私廄裁二馬,賞罰分明。



 天寶末入朝,而安祿山反,帝引見,問何策以討賊。常清見帝憂,因大言曰:「天下太平久,人不知戰。然事有逆順,勢有奇變,臣請馳至東京,悉府庫募驍勇,挑馬箠度河,計日取逆胡首以獻闕下。」天子壯之。明日,以常清為範陽節度副大使,乘驛赴東京。常清募兵得六萬人,然皆市井庸保,乃部分旗幟,斷河陽橋以守。賊移書平原,令太守顏真卿以兵七千防河。真卿馳使司兵參軍事李平入奏。常清取平表發視,即倚帳作書遺真卿,勸堅守,且傳購祿山檄數十函與之,真卿得,以分曉諸郡。祿山度河,陷滎陽,入罌子穀,先驅至葵園。常清使驍騎拒之,殺拓羯數十百人。賊大軍至,常清不能御,退入上東門,戰不利。賊鼓而進,劫官吏。再戰於都亭驛,又不勝,引兵守宣仁門,復敗。乃自提象門出,伐大木塞道以殿,至谷水,西奔陜。語高仙芝曰:「賊銳甚,難與爭鋒。潼關無兵,一夫奔突則京師危,不如急守潼關。」仙芝從之。



 敗書聞,帝削常清官,使白衣隸仙芝軍效力。仙芝使衣黑衣監左右部軍。及邊令誠以詔書至,示之,常清曰:「吾所以不死者,恐污國家節,受戮賊手。今死乃甘心。」



 始,常清敗,徑入關,欲見上陳討賊事。至渭南,有詔赴潼關。常清憂懼,為表以謝,且言:「自東京陷,三遣使表論成敗,不得對。」又言:「臣死後,望陛下無輕此賊,則社稷安。」至是臨刑,以表授令誠而死。人多哀之。



 贊曰:祿山裒百斗驍虜,乘天下忘戰,主德耄勤,故提戈內噪,人情崩潰。常清乃驅市人數萬以嬰賊鋒,一戰不勝,即奪爵土。欲入關見天子論成敗事,使者三輩上書,皆不報,回斬於軍。仙芝棄陜守關,遏賊西勢,以喪地被誅。玄宗雖為左右蒙瞽,然荒奪其明亦甚矣。卒使叛將得借口,執翰以降賊。嗚呼,非天熟其惡,使亂四海,舉黔首而殘之邪!彼二將奚誅焉?



\end{pinyinscope}