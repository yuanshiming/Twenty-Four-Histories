\article{列傳第一百二十六 文藝上}

\begin{pinyinscope}

 唐有天下三百年,文章無慮三變。高祖、太宗,大難始夷,沿江左餘風,絺句繪章說。主張「性生於氣」,認為人有生氣則性存,無生氣則性滅;,揣合低卬,故王、楊為之伯。玄宗好經術,群臣稍厭雕彖,索理致,崇雅黜浮,氣益雄渾,則燕、許擅其宗。是時,唐興已百年,諸儒爭自名家。大歷、正元間,美才輩出,擩嚌道真,涵泳聖涯,於是韓愈倡之,柳宗元、李翱、皇甫湜等和之,排逐百家,法度森嚴,抵轢晉、魏,上軋漢、周,唐之文完然為一王法,此其極也。若侍從酬奉則李嶠、宋之問、沈佺期、王維,制冊則常袞、楊炎、陸贄、權德輿、王仲舒、李德裕,言詩則杜甫、李白、元稹、白居易、劉禹錫,譎怪則李賀、杜牧、李商隱,皆卓然以所長為一世冠,其可尚已。



 然嘗言之,夫子之門以文學為下科,何哉?蓋天之付與,於君子小人無常分,惟能者得之,故號一藝。自中智以還,恃以取敗者有之,朋奸飾偽者有之,怨望訕國者有之。若君子則不然,自能以功業行實光明於時,亦不一於立言而垂不腐,有如不得試,固且闡繹優游,異不及排,怨不及誹,而不忘納君於善,故可貴也。今但取以文自名者為《文藝篇》,若韋應物、沈亞之、閻防、祖詠、薛能、鄭谷等,其類尚多,皆班班有文在人間,史家逸其行事,故弗得而述云。



 袁朗,其先雍州長安人。父樞,仕陳為尚書左僕射。朗在陳為秘書郎,江總尤器之。後主聞其才,詔為《月賦》一篇,灑然無留思,後主曰:「謝莊不得獨美於前矣。」復詔為《芝草》、《嘉蓮》二頌,嘆賞尤厚。累遷太子洗馬、德教殿學士。陳亡入隋,歷尚書儀曹郎。



 武德初,隱太子與秦王、齊王相傾,爭致名臣以自助。太子有詹事李綱、竇軌、庶子裴矩、鄭善果、友賀德仁、洗馬魏徵、中舍人王珪、舍人徐師謨、率更令歐陽詢、典膳監任璨、直典書坊唐臨、隴西公府祭酒韋挺、記室參軍事庾抱、左領大都督府長史唐憲;秦王有友於志寧、記室參軍事房玄齡、虞世南、顏思魯、諮議參軍事竇綸、蕭景、兵曹杜如晦、鎧曹褚遂良、士曹戴胄、閻立德、參軍事薛元敬、蔡允恭、主簿薛收、李道玄、典簽蘇干、文學姚思廉、褚亮、敦煌公府文學顏師古、右元帥府司馬蕭瑀、行軍元帥府長史屈突通、司馬竇誕、天策府長史唐儉、司馬封倫、軍諮祭酒蘇世長、兵曹參軍事杜淹、倉曹李守素、參軍事顏相時;齊王有記室參軍事榮九思、戶曹武士逸、典簽裴宣儼,朗為文學。從父弟承序亦有名,王召為文學館學士。朗累封汝南縣男,再轉給事中。卒,太宗為廢朝一日,謂高士廉曰:「朗任淺而性謹厚,使人悼惜。」詔給喪費,存問其家。



 朗遠祖滂,為漢司徒。自滂至朗凡十二世,其間位司徒、司空者四世,淑、顗、察皆死宋難,昂著節齊、梁。時朗自以中外人物為海內冠,雖瑯邪王氏踵為公卿,特以累朝佐命有功,鄙不為伍。



 朗孫誼,神功中為蘇州刺史。司馬張沛者,侍中文瓘子,嘗白誼曰:「州得一長史,隴西李亶,天下甲門也。」誼曰:「夫門戶者,歷世名節為天下所高,老夫是也。山東人尚婚媾,求祿利耳,至見危受命,則無人焉,何足尚邪?」沛大慚。



 承序為齊王元吉府學士,府廢,補建昌令。治尚慈簡,吏民懷德。高宗之為晉王也,太宗崇選僚屬,問梁、陳名臣子弟誰可者。岑文本曰:「昔陳亡,百司奔散,有袁憲者,朝服立後主傍,白刃不避也。王世充篡隋,群臣表勸進,而憲子給事中承家稱疾不肯署。今其少子承序,風操清亮,無愧先烈。」帝乃召拜晉王友、兼侍讀,加弘文館學士,卒。



 朗從祖弟利貞,陳中書令敬孫,高宗時為太常博士、周王侍讀。及王立為太子,百官上禮,帝欲大會群臣、命婦合宴宣政殿,設九部伎、散樂。利貞上疏諫,以為:「前殿路門,非命婦宴會、倡優進御之所,請徙命婦別殿,九部伎從左右門入,罷散樂不進。」帝納之。既會,帝傳詔利貞曰:「卿奕葉忠鯁,能抗疏規朕之失,不厚賜無以勸能者。」乃賜物百段。擢祠部員外郎,卒。中宗立,以舊恩追贈秘書少監。



 賀德仁,越州山陰人。父朗,終陳散騎常侍。德仁與從兄德基師事周弘正,以文辭稱,人為語曰:「學行可師賀德基,文質彬彬賀德仁。」兄弟八人,時比漢荀氏,太守鄱陽王伯山改所居甘滂里為高陽雲。



 始,德仁在陳,為吳興王友。入隋,楊素薦其材,授豫章王記室,王遇之厚;徙封齊,復為府屬。王廢,官吏抵罪,而德仁以忠謹獲貰,補河東司法參軍。素與隱太子善,高祖起兵,太子封隴西公,以德仁為友,庾抱為記室。俄並遷中舍人。以年耆不更吏職,徙洗馬,與蕭德言、陳子良皆為東宮學士。貞觀初,遷趙王友,卒。



 從子紀、敱亦博學。高宗時,紀為太子洗馬,豫修五禮,敱率更令、兼太子侍讀,皆為崇賢館學士。



 抱者,陳御史中丞眾孫。開皇中,為延州參軍。入調吏部,尚書牛弘給筆札,令自序,援筆而成。為元德太子學士,會嫡皇孫生,大宴,坐中獻頌,太子嗟賞。及在隴西府,文檄皆出其手。



 蔡允恭,荊州江陵人,後梁左民尚書大業子。美姿容,工為詩。仕隋,歷起居舍人。煬帝有所賦,必令諷誦。遣教宮人,允恭恥之,數稱疾。授內史舍人,俾入宮,因辭,繇是疏斥。帝遇弒,經事宇文化及、竇建德,歸國為秦王府參軍、文學館學士。貞觀初,除太子洗馬,卒,著《後梁春秋》。



 謝偃,衛州衛人,本姓直勒氏,祖孝政,仕北齊為散騎常侍,改姓謝。偃在隋為散從正員郎。貞觀初,應詔對策高第,歷高陵主簿。太宗幸東都,方谷、洛壞洛陽宮,詔求直言,偃上書陳得失,帝稱善,引為弘文館直學士,遷魏王府功曹。嘗為《塵》、《影賦》二篇,帝美其文,召見,欲偃作賦。先為序一篇,頗言天下乂安、功德茂盛意,授偃使賦。偃緣帝指,名篇曰《述聖》,帝悅,賜帛數十。



 初,帝即位,直中書省張蘊古上《大寶箴》,諷帝以民畏而未懷,其辭挺切,擢大理丞。偃又獻《惟皇誠德賦》,其序大略言:「治忘亂,安忘危,逸忘勞,得忘失,四者人主莫不然。桀以瑤臺為麗,而不悟南巢之禍;殷辛以象箸為華,而不知牧野之敗。是以聖人處宮室則思前王所以亡,朝萬國則思己所以尊,巡府庫則思今所以得,視功臣則思其輔佐之始,見名將則思用力之初,如此則人無易心,天下何患乎不化哉?旦行之堯、舜,暮失之桀、紂,豈異人哉?」其賦蓋規帝成功而自處至難云。又撰《玉諜真紀》以勸封禪。時李百藥工詩,而偃善賦,時人稱「李詩謝賦」。府廢,終湘潭令。



 蘊古,洹水人。敏書傳,曉世務,文擅當時。後坐事誅。



 崔信明,青州益都人。高祖光伯,仕後魏為七兵尚書。信明之生,五月五日日方中,有異雀鳴集庭樹,太史令史良為占曰:「五月為火,火主《離》,《離》為文,日中,文之盛也,雀五色而鳴,此兒將以文顯。然雀類微,位殆不高邪。」及長,強記,美文章。鄉人高孝基嘗語人曰:「崔生才富,為一時冠,但恨位不到耳。」隋大業中,為堯城令。竇建德僭號,而信明族弟敬素者,為賊鴻臚卿,自謂得意,語信明曰:「夏王英武,有舉天下心,士女襁負而至不可數。兄不以此時立功立事,豈所謂見幾不俟終日乎?」答曰:「昔申胥海隅釣師,能固其節。爾欲吾屈身賊中求斗筲邪?」遂逾城去,隱太行山。貞觀六年,有詔即家拜興勢丞。遷秦川令,卒。



 信明蹇亢,以門望自負,嘗矜其文,謂過李百藥,議者不許。揚州錄事參軍鄭世翼者,亦驁倨,數恌輕忤物,遇信明江中,謂曰:「聞公有『楓落吳江冷』,願見其餘。」信明欣然多出眾篇,世翼覽未終,曰:「所見不逮所聞!」投諸水,引舟去。



 世翼,鄭州滎陽人,周儀同大將軍敬德孫。貞觀時,坐怨謗流死巂州。撰《交游傳》,行於世。



 信明子冬日,武后時位黃門侍郎,為酷吏誣死。



 劉延祐,徐州彭城人。伯父胤之,少志學,與孫萬壽、李百藥相友善。武德中,杜淹薦為信都令,有惠政。永徽初,以著作郎、弘文館學士與令狐德棻、陽仁卿等撰次國史並實錄,以勞封陽城縣男。終楚州刺史。



 延祐擢進士,補渭南尉,有吏能,治第一。李勣戒之曰:「子春秋少而有美名,宜稍自抑,無為出人上。」延祐欽納。後檢校司賓少卿,封薛縣男。



 徐敬業敗,詔延祐持節到軍。時吏議敬業所署五品官殊死,六品流,延祐謂誣脅可察以情,乃論授五品官當流,六品以下除名,全宥甚眾。拜箕州刺史,轉安南都護。舊俚戶歲半租,延祐責全入,眾始怨,謀亂。延祐誅其渠李嗣仙,而餘黨丁建等遂叛,合眾圍安南府。城中兵少不支,嬰壘待援。廣州大族馮子猷幸立功,按兵不出,延祐遇害。桂州司馬曹玄靜進兵討建,斬之。



 延祐從弟藏器,高宗時為侍御史。衛尉卿尉遲寶琳脅人為妾,藏器劾還之,寶琳私請帝止其還,凡再劾再止。藏器曰:「法為天下縣衡,萬民所共,陛下用舍繇情,法何所施?今寶琳私請,陛下從之;臣公劾,陛下亦從之。今日從,明日改,下何所遵?彼匹夫匹婦猶憚失信,況天子乎!」帝乃詔可,然內銜之,不悅也。稍遷比部員外郎。監察御史魏元忠稱其賢,帝欲擢任為吏部侍郎,魏玄同沮曰:「彼守道不篤者,安用之?」遂出為宋州司馬,卒。



 子知柔,性簡靜,美風儀。居親喪,廬墓側,詔築闕表之。歷國子司業,累遷工部尚書。開元六年,河南大水,詔知柔馳驛察民疾苦及吏善惡,所表陳州刺史韋嗣立、汝州刺史崔日用、兗州刺史韋元珪、符離令綦毋頊等,止二十七人有治狀。久之,遷太子賓客,封彭城縣侯。致仕,給全祿終身。遺令薄葬,祖載服用皆自處其費。贈太子少保,謚曰文。弟知幾,別有傳。



 張昌齡,冀州南宮人。與兄昌宗皆以文自名,州欲舉秀才,昌齡以科廢久,固讓。更舉進士,與王公治齊名,皆為考功員外郎王師旦所絀。太宗問其故,答曰:「昌齡等華而少實,其文浮靡,非令器也。取之則後生勸慕,亂陛下風雅。」帝然之。



 貞觀末,翠微宮成,獻頌闕下,召見,試《息兵詔》,少選成文。帝大悅,戒之曰:「昔禰衡、潘岳矜己慠物,不得死。卿才不減二人,宜鑒於前,副朕所求。」乃敕於通事舍人裡供奉。俄為昆山道記室,《平龜茲露布》為士所稱。賀蘭敏之奏豫北門修撰,卒。



 昌宗官至太子舍人、修文館學士。撰《古文紀年新傳》數十篇。



 崔行功,恆州井陘人。祖謙之,仕北齊,終鉅鹿太守,徙占鹿泉。少好學,唐儉愛其才,妻以女,因倩作文奏。高宗時,累轉吏部郎中,以善占奏,常兼通事舍人內供奉。坐事貶游安令,又召為司文郎中,與蘭臺侍郎李懷儼並主朝廷大典冊。



 初,太宗命秘書監魏徵寫四部群書,將藏內府,置讎正三十員、書工百員。徵徙職,又詔虞世南、顏師古踵領,功不就。顯慶中,罷讎正員,聽書工寫於家,送官取直,使散官隨番刊正。至是詔東臺侍郎趙仁本、舍人張文瓘及行功、懷儼相次充使檢校,置詳正學士代散官。以勞遷蘭臺侍郎,卒。



 孫銑,尚定安公主,為太府卿。初,主降王同皎,後降銑,主卒,皎子繇請與父合葬。給事中夏侯銛駁奏「主與王氏絕,喪當還崔」,詔可。銛猶出為瀘州都督。



 行功兄子玄、別有傳。



 杜審言,字必簡,襄州襄陽人,晉征南將軍預遠裔。擢進士,為隰城尉。恃才高,以傲世見疾。蘇味道為天官侍郎,審言集判,出謂人曰:「味道必死。」人驚問故,答曰:「彼見吾判,且羞死。」又嘗語人曰:「吾文章當得屈、宋作衙官,吾筆當得王羲之北面。」其矜誕類此。



 累遷洛陽丞,坐事貶吉州司戶參軍。司馬周季重、司戶郭若訥構其罪,系獄,將殺之。季重等酒酣,審言子並年十三,袖刃刺季重於坐,左右殺並。季重將死,曰:「審言有孝子,吾不知,若訥故誤我。」審言免官,還東都。蘇頲傷並孝烈,志其墓,劉允濟祭以文。



 後武后召審言,將用之,問曰:「卿喜否?」審言蹈舞謝,後令賦《歡喜詩》,嘆重其文,授著作佐郎,遷膳部員外郎。神龍初,坐交通張易之,流峰州。入為國子監主簿、修文館直學士,卒。大學士李嶠等奏請加贈,詔贈著作郎。



 初,審言病甚,宋之問、武平一等省候何如,答曰「甚為造化小兒相苦,尚何言?然吾在,久壓公等,今且死,固大慰,但恨不見替人」云。少與李嶠、崔融、蘇味道為文章四友,世號「崔、李、蘇、杜」。融之亡,審言為服緦云。



 從祖兄易簡,九歲能屬文,長博學,為岑文本所器。擢進士,補渭南尉。咸亨初,歷殿中侍御史。嘗道遇吏部尚書李敬玄,不避,敬玄恨,召為考功員外郎屈之。而侍郎裴行儉與敬玄不平,故易簡上書言敬玄罪,敬玄曰:「襄陽兒輕薄乃爾。」因奏易簡險躁,高宗怒,貶開州司馬。



 審言生子閑,閑生甫。



 甫,字子美,少貧不自振,客吳越、齊趙間。李邕奇其材,先往見之。舉進士不中第,困長安。



 天寶十三載,玄宗朝獻太清宮,饗廟及郊,甫奏賦三篇。帝奇之,使待制集賢院,命宰相試文章,擢河西尉,不拜,改右衛率府胄曹參軍。數上賦頌,因高自稱道,且言:「先臣恕、預以來,承儒守官十一世,迨審言,以文章顯中宗時。臣賴緒業,自七歲屬辭,且四十年,然衣不蓋體,常寄食於人,竊恐轉死溝壑,伏惟天子哀憐之。若令執先臣故事,拔泥塗之久辱,則臣之述作雖不足鼓吹《六經》,至沈鬱頓挫,隨時敏給,揚雄、枚皋可企及也。有臣如此,陛下其忍棄之?」



 會祿山亂,天子入蜀,甫避走三川。肅宗立,自鄜州羸服欲奔行在,為賊所得。至德二年,亡走鳳翔上謁,拜右拾遺。與房琯為布衣交,琯時敗陳濤斜,又以客董廷蘭,罷宰相。甫上疏言:「罪細,不宜免大臣。」帝怒,詔三司親問。宰相張鎬曰:「甫若抵罪,絕言者路。」帝乃解。甫謝,且稱:「琯宰相子,少自樹立為醇儒,有大臣體,時論許琯才堪公輔,陛下果委而相之。觀其深念主憂,義形於色,然性失於簡。酷嗜鼓琴,廷蘭托琯門下,貧疾昏老,依倚為非,琯愛惜人情,一至玷污。臣嘆其功名未就,志氣挫衄,覬陛下棄細錄大,所以冒死稱述,涉近訐激,違忤聖心。陛下赦臣百死,再賜骸骨,天下之幸,非臣獨蒙。」然帝自是不甚省錄。



 時所在寇奪,甫家寓鄜,彌年艱窶,孺弱至餓死,因許甫自往省視。從還京師,出為華州司功參軍。關輔饑,輒棄官去,客秦州,負薪採橡慄自給。流落劍南,結廬成都西郭。召補京兆功曹參軍,不至。會嚴武節度劍南東、西川,往依焉。武再帥劍南,表為參謀,檢校工部員外郎。武以世舊,待甫甚善,親至其家。甫見之,或時不巾,而性褊躁傲誕,嘗醉登武床,瞪視曰:「嚴挺之乃有此兒!」武亦暴猛,外若不為忤,中銜之。一日欲殺甫及梓州刺史章彞,集吏於門。武將出,冠鉤於簾三,左右白其母,奔救得止,獨殺彞。武卒,崔旰等亂,甫往來梓、夔間。



 大歷中,出瞿唐,下江陵,溯沅、湘以登衡山,因客耒陽。游岳祠,大水遽至,涉旬不得食,縣令具舟迎之,乃得還。令嘗饋牛炙白酒,大醉,一昔卒,年五十九。



 甫曠放不自檢,好論天下大事,高而不切。少與李白齊名,時號「李杜」。嘗從白及高適過汴州,酒酣登吹臺,慷慨懷古,人莫測也。數嘗寇亂,挺節無所污,為歌詩,傷時橈弱,情不忘君,人憐其忠云。



 贊曰:唐興,詩人承陳、隋風流,浮靡相矜。至宋之問、沈佺期等,研揣聲音,浮切不差,而號「律詩」,競相襲沿。逮開元間,稍裁以雅正,然恃華者質反,好麗者壯違,人得一概,皆自名所長。至甫,渾涵汪茫,千匯萬狀,兼古今而有之,它人不足,甫乃厭餘,殘膏賸馥,沾丐後人多矣。故元稹謂:「詩人以來,未有如子美者。」甫又善陳時事,律切精深,至千言不少衰,世號「詩史」。昌黎韓愈於文章慎許可,至歌詩,獨推曰:「李、杜文章在,光焰萬丈長。」誠可信云。



 王勃,字子安,絳州龍門人。六歲善文辭,九歲得顏師古注《漢書》讀之,作《指瑕》以擿其失。麟德初,劉祥道巡行關內,勃上書自陳,祥道表於朝,對策高第。年未及冠,授朝散郎,數獻頌闕下。沛王聞其名,召署府修撰,論次《平臺秘略》。書成,王愛重之。是時,諸王鬥雞,勃戲為文檄英王雞,高宗怒曰:「是且交構。」斥出府。



 勃既廢,客劍南。嘗登葛憒山曠望,慨然思諸葛亮之功,賦詩見情。聞虢州多藥草,求補參軍。倚才陵藉,為僚吏共嫉。官奴曹達抵罪,匿勃所,懼事洩,輒殺之。事覺當誅,會赦除名。父福畤,繇雍州司功參軍坐勃故左遷交址令。勃往省,度海溺水,痵而卒,年二十九。



 初,道出鐘陵,九月九日都督大宴滕王閣,宿命其婿作序以誇客,因出紙筆遍請客,莫敢當,至勃,沆然不辭。都督怒,起更衣,遣吏伺其文輒報。一再報,語益奇,乃矍然曰:「天才也!」請遂成文,極歡罷。勃屬文,初不精思,先磨墨數升,則酣飲,引被覆面臥,及寤,援筆成篇,不易一字,時人謂勃為「腹稿」。尤喜著書。



 初,祖通,隋末居白牛溪教授,門人甚眾。嘗起漢、魏盡晉作書百二十篇,以續古《尚書》,後亡其序,有錄無書者十篇,勃補完缺逸,定著二十五篇。嘗謂人子不可不知醫,時長安曹元有秘術,勃從之游,盡得其要。嘗讀《易》,夜夢若有告者曰:「《易》有太極,子勉思之。」寤而作《易發揮》數篇,至《晉卦》,會病止。又謂:「王者乘土王,世五十,數盡千年;乘金王,世四十九,數九百年;乘水王,世二十,數六百年;乘木王,世三十,數八百年;乘火王,世二十,數七百年。天地之常也。自黃帝至漢,五運適周,土復歸唐,唐應繼周、漢,不可承周、隋短祚。」乃斥魏、晉以降非真主正統,皆五行沴氣。遂作《唐家千歲歷》。



 武后時,李嗣真請以周、漢為二王後,而廢周、隋,中宗復用周、隋。天寶中,太平久,上言者多以詭異進,有崔昌者採勃舊說,上《五行應運歷》,請承周、漢,廢周、隋為閏,右相李林甫亦贊佑之。集公卿議可否,集賢學士衛包、起居舍人閻伯璵上表曰:「都堂集議之夕,四星聚於尾,天意昭然矣。」於是玄宗下詔以唐承漢,黜隋以前帝王,廢介、酅公,尊周、漢為二王後,以商為三恪,京城起周武王、漢高祖廟。授崔昌太子贊善大夫,衛包司虞員外郎。楊國忠為右相,自稱隋宗,建議復用魏為三恪,周、隋為二王後,酅、介二公復舊封,貶崔昌烏雷尉,衛包夜郎尉,閻伯璵涪川尉。



 勃兄劇,弟助,皆第進士。



 劇,長壽中為鳳閣舍人,壽春等五王出閣,有司具儀,忘載冊文,群臣已在,乃寤其闕,宰相失色。劇召五吏執筆,分占其辭,粲然皆畢,人人嗟服。尋加弘文館學士,兼知天官侍郎。始,裴行儉典選,見劇與蘇味道,曰:「二子者,皆銓衡才。」至是語驗。劇素善劉思禮,用為箕州刺史,與綦連耀謀反,劇與兄涇州刺史勔及助皆坐誅。神龍初,詔復官。



 助,字子功,七歲喪母哀號,鄰里為泣。居父憂,毀骨立。服除,為監察御史裏行。



 初,勔、劇、勃皆著才名,故杜易簡稱「三珠樹」,其後助、劼又以文顯。劼早卒。福畤少子勸亦有文。福畤嘗詫韓思彥,思彥戲曰:「武子有馬癖,君有譽兒癖,王家癖何多耶?」使助出其文,思彥曰:「生子若是,可誇也。」



 勃與楊炯、盧照鄰、駱賓王皆以文章齊名,天下稱「王、楊、盧、駱」四傑。炯嘗曰:「吾愧在盧前,恥居王後。」議者謂然。



 炯,華陰人。舉神童,授校書郎。永隆二年,皇太子已釋奠,表豪俊充崇文館學士,中書侍郎薛元超薦炯及鄭祖玄、鄧玄挺、崔融等,詔可。遷詹事司直。俄坐從父弟神讓與徐敬業亂,出為梓州司法參軍。遷盈川令,張說以箴贈行,戒其苛。至官,果以嚴酷稱,吏稍忤意,搒殺之,不為人所多。卒官下,中宗時贈著作郎。



 照鄰,字升之,範陽人。十歲從曹憲、王義方授《蒼》、《雅》。調鄧王府典簽,王愛重,謂人曰:「此吾之相如。」調新都尉,病去官,居太白山,得方士玄明膏餌之,會父喪,號嘔,丹輒出,由是疾益甚。客東龍門山,布衣藜羹,裴瑾之、韋方質、範履冰等時時供衣藥。疾甚,足攣,一手又廢,乃去具茨山下,買園數十畝,疏潁水周舍,復豫為墓,偃臥其中。照鄰自以當高宗時尚吏,己獨儒;武後尚法,己獨黃老;後封嵩山,屢聘賢士,己已廢。著《五悲文》以自明。病既久,與親屬訣,自沈潁水。



 賓王,義烏人。七歲能賦詩。初為道王府屬,嘗使自言所能,賓王不答。歷武功主簿。裴行儉為洮州總管,表掌書奏,不應,調長安主簿。武后時,數上疏言事。下除臨海丞,鞅鞅不得志,棄官去。徐敬業亂,署賓王為府屬,為敬業傳檄天下,斥武後罪。後讀,但嘻笑,至「一手不之土未幹,六尺之孤安在」,矍然曰:「誰為之?」或以賓王對,後曰:「宰相安得失此人!」敬業敗,賓王亡命,不知所之。中宗昌,詔求其文,得數百篇。



 它日,崔融與張說評勃等曰:「勃文章宏放,非常人所及,炯、照鄰可以企之。」說曰:「不然。盈川文如縣河,酌之不竭,優於盧而不減王。恥居後,信然;愧在前,謙也。」



 開元中,說與徐堅論近世文章,說曰:「李嶠、崔融、薛稷、宋之問之文如良金美玉,無施不可。富嘉謨如孤峰絕岸,壁立萬仞,濃雲鬱興,震雷俱發,誠可畏也;若施於廊廟,駭矣。閻朝隱如麗服靚妝,燕歌趙舞,觀者忘疲,若類之《風》、《雅》,則罪人矣。」堅問:「今世奈何?」說曰:「韓休之文如大羹玄酒,有典則,薄滋味。許景先如豐肌膩理,雖穠華可愛,而乏風骨。張九齡如輕縑素練,實濟時用,而窘邊幅。王翰如瓊杯玉斝,雖爛然可珍,而多玷缺。」堅謂篤論云。



 元萬頃,後魏京兆王子推裔。祖白澤,武德中,仕至梁、利十一州都督,封新安公。萬頃起家為通事舍人。



 從李勣征高麗,管書記。勣命別將郭待封以舟師赴平壤,馮師本載糧繼之,不及期。欲報勣,而恐為諜所得,萬頃為作離合詩遺勣。勣怒曰:「軍機切遽,何用詩為?」欲斬待封,萬頃言狀,乃免。又使萬頃草檄讓高麗,而譏其不知守鴨淥之險,莫離支報曰:「謹聞命。」徙兵固守,軍不得入。高宗聞之,投萬頃嶺外。



 會赦還,為著作郎。武后諷帝召諸儒論撰禁中,萬頃與周王府戶曹參軍範履冰、苗神客、太子舍人周思茂、右史胡楚賓與選,凡撰《列女傳》、《臣軌》、《百僚新戒》、《樂書》等九千餘篇。至朝廷疑議表疏皆密使參處,以分宰相權,故時謂「北門學士」。思茂、履冰、神客供奉左右,或二十餘年。



 萬頃敏文辭,然放達不治細檢,無儒者風。武后時,累遷鳳閣侍郎,坐誅。



 履冰者,河內人。垂拱中,歷鸞臺天官二侍郎、春官尚書、同鳳閣鸞臺平章事,兼修國史。載初初,坐舉逆人被殺。



 神客,東光人,終著作郎。



 思茂,漳南人,與弟思鈞早知名。累遷麟臺少監、崇文館學士。垂拱中,下獄死。



 楚賓,秋浦人。屬文敏甚,必酒中,然後下筆。高宗命作文,常以金銀杯畾酒飲之,文成輒賜焉。家居率沈飲,無留賄,費盡復入,得賜而出,類為常。性重慎,未嘗語禁中事,人及其醉問之,亦熟視不答。尋兼崇賢直學士,卒。



 萬頃孫正,修名節,擢明經高第,授監門衛兵曹參軍。舅孫逖與譚物理,嘆己不逮。肅宗初,吏部尚書崔寓典選,正以書判第一召詣京師,以父詢倩老,辭疾免。河南節度使崔光遠表置其府。史思明陷河、洛,輦父匿山中,賊以名購,正度事急,謂弟曰:「賊祿不可養親,彼利吾名,難免矣,然不污身而死,吾猶生也。」賊既得,誘以高位,瞋目固拒,兄弟皆遇害,父聞,仰藥死,路人為哭。事平,詔錄伏節十一姓,而正為冠。贈秘書少監,以其子義方為華州參軍。



 義方,歷京兆府司錄,韋夏卿、李實繼為尹,事必咨之。歷虢商二州刺史、福建觀察使。中官吐突承璀,閩人也,義方用其親屬為右職。李吉甫再當國,陰欲承璀奧助,即召義方為京兆尹。李絳惡其黨,出為鄜坊觀察使,一切辨治,然苛刻,人多怨之。卒,贈左散騎常侍。



 弟季方,舉明經,調楚丘尉,歷殿中侍御史。兵部尚書王紹表為度支員外郎,遷金、膳二部郎中,號能職。王叔文用事,憚季方不為用,以兵部郎中使新羅。新羅聞中國喪,不時遣,供饋乏,季方正色責之,閉戶絕食待死,夷人悔謝,結歡乃還。卒,年五十一,贈同州刺史。



\end{pinyinscope}