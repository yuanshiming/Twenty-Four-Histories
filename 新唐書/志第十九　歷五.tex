\article{志第十九 歷五}

\begin{pinyinscope}

 寶應元年六月望戊夜,月蝕三之一。官歷加時在日出後,有交,不署蝕。代宗以《至德歷》不與天合,詔司天臺官屬郭獻之等嘉派先導學者。山西太原人,五世祖始遷江蘇淮安。應試不,復用《麟德》元紀,更立歲差,增損遲疾、交會及五星差數,以寫《大衍》舊術。上元七曜,起赤道虛四度。帝為制序,題曰《五紀歷》。



 其與《大衍》小異者九事,曰:仲夏之朔,若月行極疾,合於亥正,朔不進,則朔之晨,月見東方矣。依《大衍》,戌初進初朔,則朔之夕,月見西方矣。當視定朔小餘不滿《五紀》通法,如晨初餘數減十刻已下者,進以明日為朔。一也。以三萬二千一百六十乘夜半定漏刻,六十七乘刻分從之,二千四百而一,為晨初餘數。二也。陽歷去交分,交前加一辰,交後減一辰,餘百八十三已下者,日亦蝕。三也。月蝕有差,以望日所入定數,視月道同名者,交前為加,交後為減;異名者,交前為減,交後為加,各以加減去交分。又交前減一辰,交後加一辰,餘如三百三十八已下者,既。已上,以減望差,八十約之,得蝕分。四也。日蝕有差,以朔日所入定數,十五而一,以減百四,餘為定法。以蝕差減去交分。又交前減兩辰,餘為陰歷蝕。其不足減者,反減蝕差。在交後減兩辰,交前加三辰,餘為類同陽歷蝕。又自小滿畢小暑,加時距午正八刻外者,皆減一辰;三刻內者,皆加一辰。自大寒畢立春,交前五辰外,自大暑畢立冬,交後五辰外,又減一辰。不足減者,既。加、減訖,各如定法而一,以減十五,餘為蝕分。其陽歷蝕者,置去交分,以蝕差加之。交前加一辰,交後減一辰。所得,以減望差,餘如百四約之,得為蝕分。五也。所蝕分,日以十八乘之,月以二十乘之,皆十五而一,為泛用刻,不復因加。六也。日蝕定用刻在辰正前者,以十分之四為虧初刻,六為復末刻。未正後者,六為虧初刻,四為復末刻。不復相半。七也。五星乘數、除數,諸變皆通用之,不復變行異數。入進退歷,皆用度中率。八也。以定合初日與前疾初日、後疾初日與合前伏初日先後定數,各同名者,相消為差;異名者,相從為並。皆四而一。所得滿辰法,各為日。乃以前日盈減、縮加其合後伏日變率,亦以後日盈加、縮減合前伏日變率。太白、辰星夕變,則返加減留退。二退度變率,若差於中率者,倍所差之數,曰伏差,以加減前疾日度變率。熒惑均加減前疾兩變日度變率。歲星、熒惑、鎮星前留日變率,若差於中率者,以所差之數為度,加減前遲日變率。皆多於中率之數者,加之;少於中率者,減之。後留日變率,若差於中率者,以所差之數為日,以加減後遲日變率及加減二退度變率。又以伏差加減後疾日度變率。多於中率之數者,減之;少於中率者,加之。其熒惑均加減疾遲兩變日度變率。歲星、鎮星無遲,即加減前後順行日度變率。太白晨夕退行度變率,若差於中率者,亦倍所差之數為度,加減本疾度變率。夕合前、後伏,雖亦退行,不取加、減。二留日變率,若差於中率者,以所差之數為度,加減本遲度變率。皆多於中率之數加之,少於中率減之。其辰星二留日變率,若差於中率者,以所差之數為度,各加、減本遲度變率。疾行度變率,若差於中率者,以所差之數為日,各加、減留日變率。亦多於中率之數者,加之;少於中率者,減之。其留日變率,若少不足減者,侵減遲日變率。加減訖,皆為日度定率。九也。



 《大衍》以四象考五星進退,或時弗葉。獻之加減頗異,而偶與天合。於是頒用,訖建中四年。



 《寶應五紀歷》演紀上元甲子,距寶應元年壬寅,積二十六萬九千九百七十八算。



 《五紀》通法千三百四十。



 策實四十八萬九千四百二十八。



 揲法三萬九千五百七十一。



 策餘七千二十八。



 用差七千五百四十八。



 掛限三萬八千三百五十七。



 三元之策十五,餘二百九十二,秒五;秒母六。以象統為母者,以四因之。



 四象之策二十九,餘七百一十一。



 一象之策七,餘五百一十二太。



 天中之策五,餘九十七,秒十五;秒母十八。



 地中之策六,餘百一十九,秒四;秒母三十。



 貞悔之策三,餘五十八,秒十七。



 辰法三百三十五。



 刻法百三十四。



 乾實四十八萬九千四百四十二,秒七十。



 周天度三百六十五,虛分三百四十二,秒七十。



 歲差十四,秒七十。



 秒法百。



 定氣所有日及餘,以辰計之,曰辰數,與《大衍》同。



 六虛之差七,秒七十。



 轉終分百三十六萬六千一百五十六。



 轉終日二十七,餘七百四十三,秒五。



 秒法三十七。



 轉法六十七。約轉分為度,曰逡程。積逡程,曰轉積度。



 七日初,千一百九十一。末,百四十九。十四日初,千四十二。末,二百九十八。



 二十一日初,八百九十二。末,四百四十八。二十八日初,七百四十三。末,五百九十七。



 半紀六百七十。



 象積四百八十。



 辰刻八刻,分百六十。



 昏明刻各二刻,分二百四十。



 交終三億六千四百六十四萬三千七百六十七。



 交終日二十七,餘二百八十四,秒三千七百六十七。



 交中日十三,餘八百一十二,秒千八百八十三半。



 朔差日二,餘四百二十六,秒六千二百三十三。



 望差日一,餘二百一十三,秒三千一百一十六半。



 望數日十四,餘千二十五,秒五千。



 交限日十二,餘五百九十八,秒八千七百六十七。



 交率六十一。



 交數七百七十七。凡春分後陰歷交後,秋分後陽歷交後,為月道同名。餘皆為異名。



 辰分百一十三。



 秒法一萬。



 去交度乘數十一,除數千一百六十五。



 太陰損益差:冬至、夏至,益十九,積七十六;小寒、小暑,益十六,積九十五;大寒、大暑,益十四,積百一十一;立春、立秋,益十二,積百二十五;雨水、處暑,益十,積百三十七;驚贄、白露,益七,積百四十七;春分、秋分,損七,積百五十四。清明、寒露,損十,積百四十七;穀雨、霜降,損十二,積百三十七;立夏、立秋,損十四,積百二十五;小滿、小雪,損十七,積百一十一;芒種、大雪,損十九,積九十五;依定氣求朓朓術入之,得其望日所入定數。



 太陽每日蝕差:月在陰歷,自秋分後、春分前,皆以四百五十七為蝕差;入春分後,日損五分;入夏至初日,損不盡者七;乃自後日益五分。月在陽歷,自春分後、秋分前,亦以四百五十七為蝕差;入秋分後,日損五分,入冬至初日,損不盡者七;乃自後日益五分。各得朔日所入定數。



 ○歲星



 終率五十三萬四千四百八十二,秒三十六。



 終日三百九十八,餘千一百六十二,秒三十六。



 變差十四,秒八十八。



 象算九十一,餘百五,秒十八。



 爻算十五,餘七十三,秒四十六,微分三十二。



 乘數五。



 除數四。



 熒惑。



 終率百四萬五千八十八,秒八十三。



 終日七百七十九,餘千二百二十八,秒八十三。



 變差三十二,秒五十七。



 象算九十一,餘百六,秒二十八,微分五十四。



 爻算十五,餘七十三,秒五十四,微分七十三。



 乘數百二十七。



 除數三十。



 ○鎮星



 終率五十萬六千六百二十三,秒二十九。



 終日三百七十八,餘百三,秒二十九。



 變差九,秒八十七。



 象算九十一,餘百四,秒八十六,微分六十六。



 爻算十五,餘七十三,秒三十一,微分十一。



 乘數十二。



 除數十一。



 ○太白



 終率七十八萬二千四百四十九,秒九。



 終日五百八十三,餘千二百二十九,秒九。



 中合二百九十二,餘千二百八十四,秒五十九,微分七十二。



 變差四十九,秒七十二。



 象算九十一,餘百七,秒三十五,微分七十二。



 爻算十五,餘七十三,秒七十二,微分六十。



 乘數十五。



 除數二。



 ○辰星



 終率十五萬五千二百七十八,秒六十六。



 終日百一十五,餘千一百七十八,秒六十六。



 中合五十七,餘千二百五十九,秒三十三。



 變差五十,秒八十五。



 象算九十一,餘百七,秒四十二,微分七十八。



 爻算十五,餘七十三,秒七十三,微分七十七。



 秒法百。



 微分法九十六。



 德宗時,《五紀歷》氣朔加時稍後天,推測星度與《大衍》差率頗異。詔司天徐承嗣與夏官正楊景風等,雜《麟德》、《大衍》之旨治新歷。上元七曜,起赤道虛四度。建中四年歷成,名曰《正元》。其氣朔、發斂、日躔、月離、軌漏、交會,悉如《五紀》法。惟發斂加時無辰法,皆以象統乘小餘,通法而一,為半辰數。餘五因之,六刻法除之,得刻。不盡,六而一,為刻分。其軌漏夜半刻分以刻法準象積取其數用之,以刻法通夜半定漏刻,內分,二十而一,為晨初餘數。月蝕去交分,如二百七十九已下者,既。已上,以減望差,六十六約之,為蝕分。日蝕差亦十五約之,以減八十五,餘為定法。又加減去交分訖,以減望差,八十五約之,得蝕分。日法不同也。其五星寫《麟德歷》舊術,因冬至後夜半平合日算,加合後伏日及餘,即平見日算。金、水先得夕見;其滿晨見伏日及餘秒去之,餘為晨平見。求入常氣,以取定見而推之。《麟德歷》之啟蟄,《正元歷》之雨水;《麟德歷》之雨水,《正元歷》之驚蟄也。《麟德歷》熒惑前、後疾變度率,初行入氣差行,日益遲、疾一分,《正元歷》則二分,亦度母不同也。詔起五年正月行新歷。會硃泚之亂,改元興元。自是頒用,訖元和元年。



 《建中正元歷》演紀上元甲子,距建中五年甲子,歲積四十萬二千九百算外。



 《正元》通法千九十五。



 策實三十九萬九千九百四十三。



 揲法三萬三千三百三十六。



 章閏萬一千九百一十一。



 策餘五千七百四十三。



 用差六千一百六十八。



 掛限三萬一千三百四十三。



 三元之策十五,餘二百三十九,秒七。



 四象之策二十九,餘五百八十一。



 一象之策七,餘四百一十九。



 中盈分四百七十八,秒一十四。



 朔虛分五百一十四。



 象統二十四。



 象位六。



 天中之策五,餘七十九,秒五十五;秒母七十二。



 地中之策六,餘九十五,秒四十三;秒母六十。



 貞悔之策三,餘四十七,秒五十一半。



 刻法二百一十九。六刻法千三百一十四。



 乾實三十九萬九千九百五十五,秒二。



 周天度三百六十五,虛分二百八十,秒二。



 歲差十二,秒二。



 秒母百。



 定氣辰數同《大衍》。



 六虛之差六,秒二十。



 轉終分三億一百七十二萬一百三十二。



 轉終日二十七,餘六百七,秒百三十二。



 入轉秒法一萬。



 轉法二百一十九。約轉分為度,曰逡程。積逡程,曰轉積度。



 七日:初九百七十三,末百二十二。



 十四日:初八百五十一,末二百四十四。



 二十一日:初七百二十九,末三百六十六。



 二十八日:初六百七,末四百八十八。



 辰刻八刻,分七十三。



 刻法二百一十九。



 昏明刻各二刻,分百九半。



 交終分二億九千七百九十七萬三千八百一十五。



 交終日二十七,餘二百三十二,秒三千八百一十五。



 交中日十三,餘六百六十三,秒六千九百七半。



 朔差日二,餘三百四十八,秒六千一百八十五。



 望差日一,餘百七十四,秒三千九十二半。



 望數日十四,餘八百三十八。



 交限日十二,餘四百八十九,秒三千八百一十五。



 交率六十一。



 交數七百七十七。



 交辰法九十一少。



 秒法一萬。



 去交度乘數十一,除數九百四十五。



 太陰損益差:冬至、夏至,益十六,積六十二。小寒、小暑,益十三,積七十八。大寒、大暑,益十一,積九十一。立春、立秋,益十,積百二。雨水、處暑,益八,積百一十二。驚蟄、白露,益六,積百二十。春分、秋分,損六,積百二十六。清明、寒露,損八,積百二十。穀雨、霜降,損十,積百一十二。立夏、立冬,損十一,積百二。小滿、小雪,損十三,積九十一。芒種、大雪,損十六,積七十八。以損益依入定氣求朓朒術入之,各得其望日所入定數。



 太陽每日蝕差:月在陰歷,自秋分後、春分前,皆以三百七十三為蝕差,入春分後,日損四分;入夏至初日,損不盡者六;乃自後日益四分。月在陽歷,自春分後、秋分前,亦以三百七十三為蝕差;入秋分後,日損四分;入冬至初日,損不盡者六;乃自後日益四分:各得朔日所入定數。



 ○歲星



 終率四十三萬六千七百六十,秒四。



 終日三百九十八,餘九百五十,秒四。



 合後伏日十七,餘千二十三。



 ○熒惑



 終率八十五萬四千七,秒七十九。



 終日七百七十九,餘千二,秒七十九。



 合後伏日七十一,餘千四十九。



 ○鎮星



 終率四十一萬三千九百九十四,秒六十三。



 終日三百七十八,餘八十四,秒六十三。



 合後伏日十八,餘五百九十。



 ○太白



 終率六十三萬九千三百八十九,秒二十八。



 終日五百八十三,餘四,秒二十八。



 晨合後伏日四十一,餘九百一十五。



 夕見伏日二百五十六,餘五百二,秒一十四。



 晨見伏日三百二十七,餘五百二,秒一十四。



 ○辰星



 終率十二萬六千八百八十八,秒四半。



 終日百一十五,餘九百六十三,秒四半。



 晨合後伏日十六,餘千四十。



 夕見伏日五十二,餘四百八十一,秒五十二少。



 晨見伏日六十三,餘四百八十一,秒五十二少。



 秒法一百。



 五星平見加減差。



 ○歲星



 初見,去日十四度,見。入冬至,畢小寒,均減六日。自入大寒後,日損百九分半。入春分初日,依平。自後日加百四十五分半。入立夏,畢小滿,均加六日。自入芒種後,日損百四十五分。入夏至,畢立秋,均加四日。自入處暑後,日損二百九十一分半。入白露初日,依平。自後日減八十七分。入小雪,畢大雪,均減六日。



 ○熒惑



 初見,去日十七度,見。入冬至初日,減二十七日。自後日損九百八十五分半。入大寒初日,依平。自後日加六百五十七分。入驚蟄,畢穀雨,均加二十七日。自入立夏後,日損三百二十三分。入立秋,依平。自入處暑後,日減三百二十三分。入小雪,畢大雪,均減二十七日。



 ○鎮星



 初見,去日十七度,見。入冬至初日,減四日。自後日益百四十五分半。入大寒,畢春分,均減八日。自入清明後,日損九十六分。入小暑初日,依平。自後日加百四十五分半。入白露初日,加八日。自後日損二百九十一分。入秋分,均加四日。自入寒露後,日損九十六分。入小雪初日,依平。自後日減百四十五分半。



 ○太白



 初見,去日十一度。夕見:入冬至初日,依平。自後日減百六十三分。入雨水,畢春分,均減九日。自入清明後,日減百六十三分。入芒種,依平。自入夏至,日加百六十三分。入處暑,畢秋分,均加九日。自入寒露後,日損百六十三分。入大雪,依平。晨見:入冬至,依平。入小寒後,日加百九分半。入立春,畢立夏,均加三日。入小滿後,日損百九分半。入夏至,依平。入小暑後,日減百九分半。入立秋,畢立冬,均減三日。入小雪後,日損百九分半。



 ○辰星



 初見,去日十七度。夕見:入冬至,畢清明,依平。入穀雨,畢芒種,均減二日。入夏至,畢大暑,依平。入立秋,畢霜降,應見不見。其在立秋及霜降二氣之內者,去日十八度外,三十六度內,有水、火、土、金一星已上者,見。入立冬,畢大雪,依平。晨見;入冬至,均減四日。入小寒,畢雨水,均減三日。其在雨水氣內,去日度如前,晨無水、火、土、金一星已上者,不見。入驚蟄,畢立夏,應見不見。其在立夏氣內,去日度如前,晨有水、火、土、金一星已上者,亦見。入小滿,畢寒露,依平。入霜降,畢立冬,均加一日。入小雪,畢大雪,依平。



 ○五星變行加減差日度率



 △歲星



 前順:差行。百一十四日,行十八度九百七十一分。先疾,二日益遲三分。



 前留:二十六日。



 前退:差行。四十二日,退六度。先遲,日益疾二分。



 後退:差行。四十二日,退六度。先疾,日益遲二分。



 後留:二十五日。



 後順:差行。百一十四日,行十八度九百七十一分。先遲,二日益疾三分。日盡而夕伏。



 △熒惑



 前疾:入冬至初日,二百四十三日行百六十五度。自後三日損日度各二。小寒初日,二百三十三日行百五十五度。自後二日損日度各一。穀雨四日,依平。畢小滿九日,百七十八日行百度。自九日後,三日損日度各一。夏至初日,依平。畢六日,百七十一日行九十三度。自六日後,每三日益日度各一。立秋初日,百八十四日行百六度。自後每日益日度各一。白露初日,二百一十四日行百三十六度。自後五日益日度各六。秋分初日,二百三十二日行百五十四度。自後每日益日度各一。寒露初日,二百四十七日行百六十九度。自後五日益日度各三。霜降五日,依平。畢立冬十三日,二百五十九日行百八十一度。自入十三日後,二日損日度各一。



 前遲:差行。入冬至,六十日行二十五度;先疾,日益遲三分。自入小寒後,三日損日度各一。大寒初日,五十五日行二十度。自後三日益日度各一。立春初日,畢清明,平,六十日行二十五度。自入穀雨,每氣損度一。立夏初日,畢小滿,平,六十日行二十三度。自入芒種後,每氣益一度。夏至初日,平。畢處暑,六十日行二十五度。自入白露後,三日損度一。秋分初日,六十日行二十度。自後每日益日一,三日益度二。寒露初日,七十五日行三十度。自後每日損日一,三日損度一。霜降初日,六十日行二十五度。自後二日損度一。立冬一日,平。畢氣末,六十日行十七度。自小雪後,五日益度一。大雪初日,六十日行二十度。自後三日益度一。



 前留:十三日前疾減一日率者,以其差分益此留及遲日率。前疾加日率者,以其差分減此留及後遲日率。



 退行:入冬至初日,六十三日行二十二度。自後四日益度一。小寒一日,六十三日行二十六度。自入小寒一日後,三日半損度一。立春三日,平。畢雨水,六十三日退十七度。自入驚蟄後,二日益日度各一,驚蟄八日,平。畢氣末,六十七日退二十一度。自入春分後,一日損日度各一。春分四日,平。畢芒種,六十三日退十七度。自入夏至後,每六日損日度各一。大暑初日,平。畢氣末,五十八日退十二度。立秋初日,平。畢氣末,五十七日退十一度。自入白露後,二日益日度各一。白露十二日,平。畢秋分,六十三日退十七度。自入寒露後,三日益日度各一。寒露九日,平。畢氣末,六十六日退二十度。自入霜降後,二日損日度各一。霜降六日,平。畢氣末,六十三日退十七度。自入立冬後,三日益日度各一。立冬十二日,平。畢氣末,六十七日退二十一度。自入小雪後,二日損日度各一。小雪八日,平。畢氣末,六十三日退十七度。自入大雪後,三日益度一。



 後留:冬至初日,十三日。大寒初日,平。畢氣末,二十五日。自入立春後,二日半損一日。驚蟄初日,十三日。自後三日益日一。清明初日,三十三日。自後每日損日一。清明十日,平。畢處暑,十三日。自入白露後,二日損日一。秋分十一日,無留。自入秋分十一日後,日益日一。霜降初日,十九日。立冬畢大雪,十三日。



 後遲:差行。六十日行二十五度。先遲,日益疾三分,前疾加度者,此遲依數減之為定。若不加度者,此遲入秋分至立冬減三度,入立冬到冬至減五度,後留定日十三日者,以所朒數加此遲日率。



 後疾:冬至初日,二百一十日行百三十二度。自後每日損日度各一。大寒八日,百七十二日行九十四度。自入大寒八日後,二日損日度各一。雨水,平。畢氣末,百六十一日行八十三度。自入驚蟄後,三日益日度各一。穀雨三日,百七十七日行九十九度。自三日後,每日益日度各一。芒種十四日,平。畢夏至十日,二百三十三日行百五十五度。自十日後,每日益日度各一。小暑五日,二百五十三日行百七十五度。自後每日益日度各一。大暑初日,平。畢處暑,二百六十三日行百八十五度。自入白露後,二日損日度各一。秋分一日,二百五十五日行百七十七度。自一日後,每三日損日度各一。大雪初日,二百五日行百二十七度。自後三日益日度各一。



 △鎮星



 前順:差行。八十三日,行七度四百七十四分。先疾,三日益遲二分。



 前留:三十七日。



 前退:差行。五十一日,退三度。先遲,二日益疾一分。



 後退:差行。五十一日,退三度。先疾,二日益遲一分。



 後留:三十六日。



 後順:差行。八十三日,行七度四百七十四分。先遲,三日益疾二分。



 △太白



 夕見:入冬至,畢立夏,立秋畢大雪,百七十二日行二百六度。自入小滿後,十日益度一,為定初。入白露,畢春分,差行;先疾,日益遲二分。自餘平行。夏至畢小暑,百七十二日行二百九度。自入大暑後,五日損一度,畢氣末。



 夕平行:冬至及大暑、大雪各畢氣末,十三日行十三度。自入冬至後,十日損一,畢立春。入立秋,六日益一,畢秋分。雨水畢芒種,七日行七度。自入夏至後,五日益一,畢小暑。寒露初日,二十三日行二十三度。自後六日損一,畢小雪。



 夕遲:差行。四十二日行三十度。先疾,日益遲十三分。前加度過二百六度者,準數損此度。



 夕留:七日。



 夕退:十日,退五度。日盡而夕伏。



 晨退:十日,退五度。



 晨留:七日。



 晨遲:差行。冬至畢立夏,大雪畢氣末,四十二日行三十度;先遲,日益疾十三分。自小滿後,率十日損一度,畢芒種。夏至畢寒露,四十二日行二十七度;差依前。自入霜降後,每氣益一度,畢小雪。



 晨平行:冬至畢氣末,立夏畢氣末,十三日行十三度。自小寒後,六日益日度各一,畢雨水。入小滿後,七日損日度各一,畢立秋。驚蟄初日,二十三日行二十三度。自後六日損日度各一,畢穀雨。處暑畢寒露,無此平行。自入霜降後,五日益日度各一,畢大雪。



 晨疾:百七十二日,行二百六度。前遲行損度不滿三十者,此疾依數益之。處暑畢寒露,差行;先遲,日益疾二分。自餘平行。日盡而晨伏。



 △辰星



 夕見疾:十二日,行二十一度十分。大暑畢處暑,十二日,行十七度十六分。



 夕平:七日,行七度。自入大暑後,二日損度各一。入立秋,無此平行。



 夕遲:六日,行二度七分。前疾行十七度者,無此遲行。



 夕伏留:五日。日盡而夕伏。



 晨見留:五日。



 晨遲:六日,行二度七分。自入大寒,畢雨水,無此遲行。



 晨平行:七日,行七度。入大寒後,二日損日度各一。入立春,無此平行。



 晨疾;十二日,行二十一度十分。前無遲行者,十二日,行十七度十六分。日盡而晨伏。



\end{pinyinscope}