\article{列傳第一百四十九下 叛臣下}

\begin{pinyinscope}

 李忠臣,本董秦也,幽州薊人。少籍軍,以材力奮,事節度使薛楚玉、張守邦、安祿山等集》五卷,《伊川先生文集》八卷,《伊川易傳》四卷,《經,甄勞至折沖郎將。平盧軍先鋒使劉正臣殺偽節度呂知晦,擢秦兵馬使,攻長楊,戰獨山,襲榆關、北平,殺賊將申子貢、榮先欽,執周釗送京師。從正臣赴難,復敗李歸仁、李咸、白秀芝等。潼關失守,秦整軍北還。奚王阿篤孤初引眾與正臣合,已而紿約皆攻範陽,至後城,夜乘間襲秦。秦接戰,敗之,追奔至溫泉山,禽首領阿布離,斬以釁鼓。至德二載,節度使王玄志使秦率兵三千自雍奴桴葦絕海,擊賊將石帝廷、烏承洽,轉戰累日,拔魯城、河間、景城,收糧貲以實軍。又與田神功下平原、樂安,禽偽刺史以獻。於是防河招討使李銑承制假秦德州刺史。



 史思明自歸,河南節度使張鎬督秦軍合諸將平河南州縣,與裨將陽惠元破安慶緒將王福德於舒舍,肅宗下詔褒諭,令屯濮州,又徙韋城。從郭子儀圍相州,軍潰,秦至滎陽,破賊將敬釭,取糧艘二百柁以餉汴軍。未幾,授濮州刺史,屯杏園渡。許叔冀以汴下史思明,秦力屈,亦降。思明撫背曰:「始吾有左手,得公今完矣!」與俱寇河陽,秦夜挈五百人冒圍歸李光弼。詔加殿中監,封戶二百,召至京師,賜今氏名,給良馬、甲第。時陜西、神策兩節度使郭英乂、衛伯玉屯陜,故以忠臣為兩軍兵馬使,戰永寧、莎柵,與賊將李感義等數十遇,皆破之。淮西節度使王仲升為賊執,以忠臣為汝、仙、蔡六州節度使,兼安州。合諸軍平東都,進御史大夫。



 回紇可汗既歸,留其下安恪、石帝廷居河陽守貲廥,因是招亡命為盜,道路畏澀。詔忠臣討定之。吐蕃犯京師,天子追兵,秦方宴鞠場,使者至,即整師引道。諸將白:「須良日。」忠臣怒曰:「君父在難,方擇日救患乎?」時召兵無先忠臣至者。代宗嘉之,加本道觀察使,賚與倍等。



 周智光為帳下所殺,忠臣提兵入華州,所過大掠,自赤水距潼關二百里無居人。大歷五年,加蔡州刺史。陜虢李國清為下所逐,掠府庫,國清遍拜諸將乃免。會忠臣入朝,次陜,詔訊於眾。眾懼忠臣,不敢搖,即圍棘,約士投所掠物圍中,一日盡獲。



 討李靈耀也,戰西梁固,敗之。復與馬燧軍合,敗賊於汴州。田悅以援兵三萬屯汴郛,忠臣勒裨將李重倩夜率百騎襲之,貫其營而還,殺數十百人。悅間道走,靈耀開城亡去,軍遂潰。以忠臣為汴州刺史,加檢校司空、同中書門下平章事,封西平郡王。



 忠臣資婪沓嗜色,將士婦女逼與亂,所至人苦之。以女弟妻張惠光,用為牙將,恃勢殘克。或白忠臣,不之信。又以惠光子居牙下,愈橫肆。十四年,大將李希烈因眾怒,與少將丁皓、賈子華等共斬惠光父子,以兵脅逐忠臣。跳奔京師,帝素寵之,不責也。復授檢校司空、同中書門下平章事,奉朝請。



 德宗立,散騎常侍張涉以贓得罪,帝怒不赦。涉故侍讀東宮者,忠臣曰:「陛下貴為天子,先生以乏財觸法,非過也。」帝意解,免涉歸田里。湖南觀察使辛京杲私怒部曲,殺之,有司劾當死。忠臣曰:「京杲應死久矣!」帝問故,對曰:「京杲諸父戰某所死,兄弟戰某所死,渠從行獨得存,以故知之。」帝淒然悟,釋之,下除王傅。



 忠臣戇直不通書。帝嘗謂:「卿耳大,真貴兆。」對曰:「臣聞驢耳大,龍耳小。」帝喜其野而誠。然既失兵,怫鬱不顧藉。硃泚反,偽署司空兼侍中。泚攻奉天,以忠臣居守。泚敗,系有司,與其子俱斬。



 喬琳,並州太原人。少孤苦志學,擢進士第。性誕蕩無禮檢。郭子儀表為朔方府掌書記。與聯舍畢曜相掉訐,貶巴州司戶參軍。歷果、綿、遂、懷四州刺史,治寬簡,不親事。嘗謂錄事參軍任紹業曰:「子綱紀一州,能劾刺史乎?」紹業出條所失示之,驚曰:「能知吾失,御史材也。」



 琳素善蒲人張涉。涉以國子博士侍太子讀,太子即位,召訪政事,不淹日,詔入翰林,遷散騎常侍。薦琳任宰相,乃拜御史大夫、同中書門下平章事。天下矍然駭之。琳年高且聵,每進封失次,所言不厭帝旨,在位閱八旬,以工部尚書罷。帝由是亦疏涉。



 琳從幸奉天,再遷太子少師;進幸梁州,次盩厔,詭言馬殆不進。帝素以舊老禮之,給乘輿馬,辭病力。帝賜所執策曰:「勉為良圖,與卿別矣!」不數日,祝髯發舍仙游佛廬。泚聞,遣數十騎取之,署吏部尚書,令姻家源休衣以朝服,食以肉,琳亦不辭。士有訴官非便者,琳曰:「子謂此選便乎?」及收京師,李晟憫其老,表貰死。帝曰:「琳,故宰相,失節背義,不可赦。」臨刑嘆曰:「我以七月七日生,以此日死,非命耶?」



 時又有蔣鎮者,洌子也,與兄鏈俱以文辭顯。擢賢良方正科,累轉諫議大夫。大歷中,淫雨壞河中鹽池,味苦惡。韓滉判度支,慮減常賦,妄言池生瑞鹽,王德之美祥。代宗疑不然,命鎮馳驛按視。鎮內欲結滉,故實其事,表置祠房,號池曰「寶應靈慶」云。再進工部侍郎。妹婿源溥者,休弟也,故鎮與休交。泚叛,竄於鄠,傷足不能進。泚先得鏈,而鎮左右逃歸,語所在,源休聞,白泚,以二百騎求得之。知不免,懷刃將自刺,鏈止之。復謀出奔,懦不決。中朝臣遁伏者,休多所誅殺,賴鎮救原十五。初,洌與弟渙在安史時皆污偽官,鏈兄弟復屈節於賊云。



 高駢,字千里,南平郡王崇文孫也。家世禁衛,幼頗修飭,折節為文學,與諸儒交,硜硜譚治道,兩軍中人更稱譽之。事硃叔明為司馬。有二雕並飛,駢曰:「我且貴,當中之。」一發貫二雕焉,眾大驚,號「落雕侍御」。後歷右神策軍都虞候。黨項叛,率禁兵萬人戍長武。是時諸將無功,唯駢數用奇,殺獲甚多。懿宗嘉之,徙屯秦州,即拜刺史兼防禦使。取河、渭二州,略定鳳林關,降虜萬餘人。



 咸通中,帝將復安南,拜駢為都護,召還京師,見靈臺殿。於是容管經略使張茵不討賊,更以茵兵授駢。駢過江,約監軍李維周繼進。維周擁眾壁海門,駢次峰州,大破南詔蠻,收所獲贍軍。維周忌之,匿捷書不奏。朝廷不知駢問百餘日,詔問狀。維周劾駢玩敵不進,更命右武衛將軍王晏權往代駢。俄而駢拔安南,斬蠻帥段酋遷,降附諸洞二萬計。晏權方挾維周發海門,檄駢北歸。而駢遣王惠贊傳酋遷首京師,見艟艫甚盛,乃晏權等,惠贊懼奪其書,匿島中,間關至京師。天子覽書,御宣政殿,群臣皆賀,大赦天下。進駢檢校刑部尚書,仍鎮安南,以都護府為靜海軍,授駢節度,兼諸道行營招討使。始築安南城。由安南至廣州,江漕梗險,多巨石,駢募工毚刂治,由是舟濟安行,儲餉畢給。又使者歲至,乃鑿道五所,置兵護送。其徑青石者,或傳馬援所不能治。既攻之,有震碎其石,乃得通,因名道曰「天威」云。加檢校尚書右僕射。



 駢之戰,其從孫潯常先鋒冒矢石以勸士。駢徙節天平,薦潯自代,詔拜交州節度使。僖宗立,即其軍加同中書門下平章事。



 南詔寇巂州,掠成都,徙駢劍南西川節度,乘傳詣軍。及劍門,下令開城,縱民出入。左右諫:「寇在近,脫大掠,不可悔。」駢曰:「屬吾在安南破賊三十萬,驃信聞我至,尚敢邪!」當是時,蠻攻雅州,壁廬山,聞駢至,亟解去。駢即移檄驃信,勒兵從之。驃信大懼,送質子入朝,約不敢寇。



 蜀有突將,分左右二廂,廂有虞候,詰火督盜賊,有兵馬虞候,主調發。駢罷其一,各置一虞候。又以蜀兵孱,詔蠻新定,人未安業,罷突將月稟並餐錢,約曰:「府庫完,當如舊。」又團練兵戰者,厚其衣稟;不團練者,但掌文書、倉庫,衣稟減焉。駢曰:「皆王卒,命均之。」戰士大望。於時天平、昭義、義成戍軍合蜀兵凡六萬。駢之自將出屯也,突將亂,乘門以入,駢匿於圊,求不得。天平軍聞變,其校張桀以士五百格戰,不勝。監軍慰撫之,皆曰:「州雖更蠻亂,戶口尚完,府庫方實,公削軍稟以自養,不堪其虐,故亂。」監軍懼,講解之。取役夫數百,名叛卒,藉斬其首,乃定。駢徐出,以金帛厚賞士,開府庫悉還其衣稟。然密籍所給姓名,夜遣牙將擊殺之,夷其族,雖孕者不貰,投尸於江。有一婦方踞而乳子,將就刑,媼傷之,疑其畏死,謂曰:「以子丐我,一詣曹司也。」婦蹶起曰:「我知之,且飽吾子,不可使以饑就戮也。」見刑者拜曰:「渠有節度使奪戰士食,一日忿怒,淫刑以逞,國家法令何有也?我死當訴於天,使此賊闔門如今日冤也!」逮死,神色晏然。蜀人聞者為垂泣。駢復錄突將戍還者,丸名貯器中,意不懌,則探之,或十或五,授將李敬全斬決。親吏王殷說駢曰:「突將在行者,初不知謀,公當赦之。」駢悅,投丸池中,人乃安。



 蜀之土惡,成都城歲壞,駢易以磚甓,陴堞完新,負城丘陵悉墾平之,以便農桑。訖功,筮之得《大畜》。駢曰:「畜者,養也。濟以剛健篤實,輝光日新,吉孰大焉!文宜去下存上。」因名大玄城。進檢校司徒,封燕國公,徙荊南節度。



 梁纘者,本以昭義兵西戍,駢表隸麾下。王仙芝之敗,殘黨過江,帝以駢治鄆威化大行,且仙芝黨皆鄆人,故授駢鎮海節度使。駢遣將張潾與纘分兵窮討,降其驍帥畢師鐸數十人,賊走嶺表。帝美其功,加諸道行營都統、鹽鐵轉運等使。又詔駢料官軍義營鄉團,歸其老弱傷夷,裁制軍食;刺史以下小罪輒罰,大罪以聞。賊更推黃巢南陷廣州,駢建遣潾以兵五千屯郴扼賊西路,留後王重任以兵八千並海進援循、潮,自將萬人繇大庾擊賊廣州,且請起荊南王鐸兵三萬壁桂、永,以邕管兵五千壁端州,則賊無遺類。帝納其策,而駢卒不行。



 俄徙淮南節度副大使。駢繕完城壘,募軍及土客,得銳士七萬,乃傳檄召天下兵共討賊,威震一時,天子倚以為重。廣明初,潾破賊大雲倉,詐降巢。巢不意其襲,遂大奔,引殘黨壁上饒,然眾亡幾。會疫癘起,人死亡,潾進擊之,巢大懼,以金啖潾,騰書於駢,丐歸命。駢信之,許為求節度。當此時,昭義、武寧、義武兵數萬赴淮南,駢欲專己功,即奏賊已破,不須大兵。有詔班師。巢知兵罷,即絕駢請戰,擊殺潾,乘勝度江攻天長。



 始,巢在廣州,求天平節度,宰相廬攜善駢,以有討賊功,不肯赦巢,與鄭畋爭於朝,故巢怨不得節度。而駢聞議不一,亦不平,至是欲縱賊以聳朝廷,然後立功。畢師鐸諫曰:「朝廷所恃,誰易於公?制賊要害,莫先淮南。今不據要津以滅賊,使得北度,必亂中原。」駢矍然,下令將出師。嬖將呂用之畏師鐸有功,諫曰:「公勛業極矣,賊未殄,朝廷且有口語。況賊平,挾震主之威,安所稅駕?不如觀釁求福,為不朽資也。」駢入其計,托疾未可以出屯,嚴兵保境。巢據滁、和,去廣陵才數百里,乃求援陳許。



 巢逼揚州,眾十五萬。駢將曹全晸以兵五千戰不利,壁泗州以待援,駢兵終不出。賊北趨河洛,天子遣使者促駢討賊,冠蓋相望也。俄而兩京陷,天子猶冀駢立功,眷寄未衰,詔刺史若諸將有功,自監察御史至常侍,許墨制除授。尋進檢校太尉,東面都統,京西、京北神策軍諸道兵馬等使。會二雉雊署寢,占者曰:「軍府將空。」駢惡之,悉兵出營東塘,舟二千艘,戈鎧完銳,日討金鼓,以侈士志。與浙西節度使周寶檄,欲連和而西,寶大喜。有謂寶:「彼欲並江東為孫策三分計。」寶未之信。俄而駢請寶至軍議事,寶怒,辭疾不出,釁隙遂構。駢屯東塘百日,托以寶及浙東劉漢宏將為不利,乃還,以應其變。



 帝知駢無出兵意,天下益殆。乃以王鐸代為都統,以崔安潛副之。詔韋昭度領諸道鹽鐵轉運使,加駢侍中,增實戶一百,封渤海郡王。駢失兵柄利權,攘袂大詬,即上書謾言不恭,詆鐸乃敗軍將,而安潛狼貪,有如橈敗,詒千古之悔。又引更始刮席、子嬰軹道事以激帝。帝怒,下詔切責。當此時,王室微,不絕如帶。駢都統三年,無尺寸功,幸國顛沛,大料兵,陰圖割據,一旦失勢,威望頓盡,故肆為醜悖,脅邀天子,冀復故權。而吳人顧云以文辭緣澤其奸,偃然無所忌畏。又請帝南幸江淮。會平賊,駢聞,縮氣悵恨,部下多叛去,鬱鬱無聊,乃篤意求神仙,以軍事屬用之。



 用之者,鄱陽人,世為商儈,往來廣陵,得諸賈之驩。既孤,依舅家,盜私其室,亡命九華山,事方士牛弘徽,得役鬼術,賣藥廣陵市。始詣駢親將俞公楚,驗其術,因得見駢,署幕府,稍補右職。用之既少賤,具知閭里利病、吏得失,頗班班言政事,以將左道,駢愈器之。乃廣樹朋黨,刺知駢動息,持金帛還結左右,日為誕妄以動駢。又薦狂人諸葛殷、張守一為長年方,並署牙將。初,殷將見,用之紿曰:「上帝以公為人臣,慮機事薙廢,使神人來備羽翼,且當以職縻之。」明日,殷以褐衣見,辯詐無窮,駢大驚,號「葛將軍」。其陰狡過用之遠甚。有大賈居第華壯,殷求之不得,謂駢曰:「城中且有妖,當築壇禳卻之。」因指賈居。駢敕吏即日驅徙,殷入居之。



 駢造迎仙等樓,皆廣高八十尺,飾以金珠璖玉,侍女衣羽衣,新聲度曲,以擬鈞天,薰齋其上,祈與仙接。用之自謂與仙真通,對駢叱吒風雨,或望空顧揖再拜,語言俚近,左右或竊議,輒殺之,後無敢出口者。蕭勝納賄用之,求鹽城監,駢不肯。用之曰:「仙人言鹽城有寶劍,須真人取之,唯勝可往。」駢許諾。數月,勝獻銅匕首,用之曰:「此北帝所佩也,得之者兵不敢犯。」駢寶秘之,常持以坐起。用之憚其術窮且見詰,乃刻青石手板為龍蛇隱起,文曰:「帝賜駢。」使人潛植機上,駢得之大喜。為寓鵠廷中,設機關,觸人則飛動,駢衣羽服,乘之作仙去狀。用之懼有擿其奸者,乃曰:「仙人當下,但患學者真氣虧沮耳。」駢始棄人間事,絕妾媵,雖將吏不得見。客至,先遣薰濯,詣方士祓除,謂之解穢,少選即引去。自是內外無敢言者,惟梁纘屢為駢言,駢不聽。纘懼,解所領兵,駢還其軍於昭義,纘不復事矣。



 用之既自任,淫刑重賦,人人思亂。乃擢廢吏百餘,號「察子」,厚稟食,令居衢哄間,凡民私鬩隱語莫不知,道路箝口。誅所惡者數百族。又募卒二萬,為左、右「鏌邪軍」,與守一分總,置官屬如駢府。用之每出入,騶御至千人;建大第,軍胥營署皆備。建百尺樓,托云占星,實窺伺城中之有變者。左右姬侍百餘,皆娟秀光麗,善歌舞,巾鷖束帶以侍。月二十宴,其費仰於民,不足,至苛留度支運物。誘人上變,則許入貲產贖罪。俞公楚數規戒其失,不聽。姚歸禮謀殺之,弗克。用之因譖二人於駢,使以驍雄兵三千督盜於外,密使兵襲之,舉師殲焉。駢從子澞密疏用之罪,諫駢曰:「不除之,高氏且無種。」駢怒,命左右扶出,以狀授用之。用之誣澞貣貰不能滿,故妄言。因出澞筆驗之,駢敕吏禁澞出入。俄署舒州刺史,未幾為下所逐,用之構之也。駢使人殺澞。



 嗣襄王煴之亂,駢上書勸進,偽假駢中書令、諸道兵馬都統、江淮鹽鐵轉運使,以用之為嶺南節度使。駢久觖望,至是大喜,貢賦不絕。用之始開府置官屬,禮與駢均矣。以鄭杞、董僅、吳邁為腹心,駢之親信皆偪使附己,政事未嘗關決駢。駢內悔,欲收其權,不能也。用之問計於杞、僅,謀請駢齋於其第,密縊之,紿為升天,事不克。



 光啟三年,蔡賊孫儒兵略定遠,聲言涉淮。壽州刺史張翱奔告駢,命畢師鐸率騎三百戍高郵。師鐸者,故仙芝黨,以善騎射稱。駢敗巢於浙西,用其力,故寵待絕等。用之厚啖以利,欲其諧附,然不肯情。師鐸有妾美,用之請見,不可,狙其出,觀焉,怒而棄之;內忿懼,為子結婚於高郵將張神劍,陰倚為援。硃全忠方攻秦宗權,駢慮其奔突,使師鐸率兵逾都梁山,不見賊還。師鐸見駢府宿將多以讒死,憂甚。用之益加禮,師鐸愈恐,謀於神劍。神劍不然其言,而猜嫌日結。用之亦慮其變,內欲除之,亟請罷屯。其母密擿師鐸使去,曰:「毋顧家室。」師鐸憂,未知所出。而駢子怒用之專恣,覬師鐸與諸將發其奸,遣使謂師鐸曰:「用之欲因此行圖君,既授書神劍矣,君其備之!」師鐸驚,軍中稍稍傳言。諸將介而見,請殺神劍,並其軍,驅市人以濟亂。師鐸曰:「不可,我若重擾百姓,復一用之也。鄭漢璋素與我善,兵精士強,以用之用事,常不平。今若告之謀,彼必喜,則事濟矣。」眾然之。神劍未知,方椎牛釃酒,且將犒師。師鐸潛師夜出,士皆絳繒抹首,且行且掠。漢璋聞,以麾下出迎,師鐸諗以計,大喜。留其妻守淮口,帥兵及亡命數千至高郵,見神劍,詰其變,神劍辭不知。師鐸語稍侵,神劍瞋目曰:「大夫何晚計!彼一妖人,前假嶺南節,不肯行,志圖淮海,令君既奪魄,彼一日得志,吾能握刀頭北面事之邪!吾前未量君意,故不出口,尚何疑?」漢璋喜,取酒割臂血而盟,推師鐸為大丞相,作誓告神,乃移檄州縣,以誅呂用之、張守一、諸葛殷為名。神劍以高郵兵諸校倪詳、褾並以天長子弟會,唐宏為先鋒,駱玄真主騎,趙簡主徒,王朗為殿,得勝兵三千。將發,神劍中悔,繆曰:「公兵雖精,然城堅,旬日不下則糧乏,眾心搖矣。神劍請按軍高郵,為公聲援而督糧道。」師鐸曰:「民稟尚多,何患資儲?城中攜離無鬥志,何事聲援?君意不行,孰敢違?」漢璋內忌神劍,恐不為己下,勸許其計,約城破玉帛子女共之。



 其四月,兵傅城,營其下。城中駭亂,用之分兵守,且自督戰。令曰:「斬一級,賞金一餅。」士多山東人,堅悍頗用命。師鐸懼,退舍自固。用之稍堙塞諸門。駢登延和閣,聞囂甚,左右告之故,大驚,召用之問狀,徐曰:「師鐸眾思歸,為門衛所軋,隨已處置,不爾,煩玄女一符耳!」駢曰:「吾覺爾之誕多矣,善自為之,勿使吾為周寶也!」時寶已為下所逐出奔雲。用之慚,不復有言。師鐸見城未下,頗懼,求救於宣州秦彥,約事平迎以代駢。



 駢數責用之曰:「始吾以心腹任君,君御下無方,卒誤我。今百姓饑饉,不可虐用,當遣大將齎吾書諭之,使罷兵。」用之疑諸將不為用,以其黨許戡奉書往。始師鐸意駢令宿將勞軍,因得口陳用之罪。及戡至,大怒曰:「梁纘、韓問安在?若何庸來!」即斬之。乃系書射城內,用之不發,即火之。它日以甲士百人入謁,駢驚匿內寢,少選乃出,叱曰:「得非反邪?」命左右驅出。用之至南門,舉策曰:「吾不復入是矣!」始與駢貳。



 師鐸壁揚子,發民廬舍治攻具。用之大索居人馬及丁壯,驍將以長刀擁脅乘城,晝夜不得息。又疑為間,數易區處,家有饁餉,皆相失,至饑死者相枕藉。駢召大將古鍔齎師鐸母書及其子出諭,師鐸遣子還曰:「不敢負恩,朝斬兇人,夕還屯,願以妻子為質。」駢恐用之屠其家,乃收置署中。會秦彥遣秦稠率兵與師鐸合,攻益急,守陴者夜焚南柵以應於外,師鐸入,守將張全乃戰死,用之距三橋,殺傷相當。駢從子傑率牙兵將執用之以畀師鐸,左鏌邪兵復斷其後,用之懼,乃出奔。



 駢召梁纘謝曰:「初不用子計以及此,庸何追?」授以兵,使保子城。遲明師鐸縱火大掠,駢乃命徹備,改服須其入。師鐸見延和閣,駢待之如賓,即署師鐸節度副使,漢璋、神劍以次授署,秦稠封府庫以待,師鐸去丞相號。於時何衛未謹,駢愛將申及說駢曰:「逆人兵少弛,願奉公夜出,發諸鎮兵,還刷大恥,賊不足平也。若不決,則及將不得侍公。」因泣下。駢恇怯不能用其策,及乃匿去。



 師鐸誅用之支黨數十,使孫約迎秦彥。彥者,徐州人,本名立,隸伍籍。乾符中,以盜系獄且死,夢言虖曰:「秦彥,而從我去!」寤而視械破,因得亡命,即名彥。聚徒百人,殺下邳令,取其貲,入黃巢黨中。既敗,與許勍降駢,累表和州刺史。中和初,宣歙觀察使竇潏病,彥襲而代之。師鐸之召彥也,或計曰:「足下向誅妖人,故下樂從。今軍府已安,宜還政高公,足下身典兵,權在掌握,四鄰聞之,不失大義,諸將未敢謀也。若令彥為帥,兵非足下有也。且秦稠封府庫,勢已相疑。足下如厚德彥,宜以金玉子女報之,勿聽度江。假足下能下彥,楊行密夕聞而朝必至。」師鐸不決,以告漢璋。漢璋曰:「善。」



 師鐸出駢,囚南第。稠麾下求無厭,燒貢奉樓數十楹,取珍寶。始駢自乾符以來,貢獻不入天子,貲貨山積,私置郊祀、元會供帳什器,殫極功巧,至是為亂兵所剽略盡。師鐸徙駢東第。禽諸葛殷,腰下得金數斤,百姓交唾,拔須發無遺,再縊乃絕,仇家迍其目云,市人投瓦礫擊尸,俄而成塚。駢出金遺守者,師鐸知之,加兵苛督,復入囚署中,子弟十餘人同幽之。顧雲入見,駢猶自若曰:「吾復居此,天時人事必有在。」意師鐸復推立之。



 用之既出,以兵攻淮口未下,鄭漢璋擊之,遂奔天長。初,用之詐為駢書,召兵於廬、壽,城陷,而楊行密兵萬人次天長,用之自歸。



 張神劍求賂於師鐸,辭以彥未至。神劍怒,與別將高霸將攻師鐸。彥之來,召池州刺史趙鍠守宣,自將入揚州,稱節度使,以師鐸為行軍司馬,居用之第,不得在牙中。師鐸怏怏失志。行密與神劍等連和,自江北至槐家橋,柵壘相聯。彥登城望之,色沮,乃授鄭漢璋、唐宏等兵屯門,樵蘇道絕,食且乏。稠及師鐸以勁卒八千出戰,大敗,稠死之,士奔溺死者十八。彥大出金求救於張雄,雄引兵至東塘,得金,不戰去。彥使師鐸率兵二萬陣城下,漢璋為前鋒,宏次之,駱玄真、樊約又次之,師鐸、王朗以騎為左右翼。既成列,久之,行密乃出,委輜重於壁,以贏兵守之,伏精卒數千其旁。行密先犯玄真,短兵接,偽北,師鐸諸軍奔其壁,爭取金玉貲糧。伏噪而出,行密引輕兵躡其尾,俘殺旁午,橫尸十里。師鐸等奔還,玄真戰死。師鐸雅倚玄真驍敢能拒敵,既失之,惋沮彌日,不復議出戰矣。



 駢久囚拘,供億窘狹,群奴徹延和閣闌楯為薪,煮革帶以食。駢召幕府盧涚曰:「予粗立功,比求清凈,非與此世爭利害,今而及此,神道何望邪?」涕下不能已。師鐸既敗,慮駢內應。有女巫王奉仙謂師鐸曰:「揚州災,有大人死,可以厭。」彥曰:「非高公邪?」命左右陳賞等往殺之。侍者白有賊,駢曰:「此必秦彥來。」正色須之。眾入,駢罵曰:「軍事有監軍及諸將在,何遽爾?」眾闢易,有奮而擊駢者,曳廷下數之曰:「公負天子恩,陷人塗炭,罪多矣,尚何云?」駢未暇答,仰首如有所伺,即斬之。左右奴客遁歸行密,行密舉軍縞素,大臨而祭,獨用之縗服哭三日。



 彥屢敗,軍氣摧喪,與師鐸抱膝相視無它略,更問奉仙,賞罰輕重皆自出。彥遣漢璋擊神劍,破之。神劍奔高郵,漢璋欲窮追,會大雨還。行密以城尚堅,師且老,議解去。用之裨將晨伏兵四壕,伺守者休代,引而登,殺數十人於門,以招外兵。守軍亦厭苦,皆委兵潰。師鐸與其家及彥奔東塘,人爭出,相騰藉死,壕塹幾滿,王朗踣而殞。行密既入,殺梁纘於牙門,以不死高氏難。韓問聞之,赴井死。居人臒心叕奄奄,兵不忍加暴,反斥餘糧救之。



 彥、師鐸與唐宏、倪詳焚白砂,將度江,會秦宗權使孫儒引兵三萬襲揚州,次天長,彥等與之合,還攻行密,取行密輜重牛羊數千計。儒以食乏,乃屠高郵,據之。張神劍奔還,行密授之館,而高郵戍兵七百潰而來,行密責有謀,悉擊殺之,因殺神劍。用之始詐行密曰:「廡下有瘞金五千斤,事平願備一日乏。」行密掘地無埋金,但得銅人三尺,身桎梏,釘刺其口,刻駢名於背,蓋用蠱厭駢也。行密責其罪,並張守一斬於三橋,妻子皆死,著其罪於路。



 儒攻城未得志,慮彥、師鐸有異謀,稍並其兵。唐宏度不免,即告儒曰:「師鐸密遣人至汴。」儒大恐。明日,召彥、師鐸、漢璋會軍中,彥、師鐸先至,壯士捽之至儒所,儒質彥反駢罪,斬之。至師鐸,呼曰:「丈夫成則王,敗則虜,君何多責為?吾嘗將數萬兵,不死常人手,得公之劍,瞑目矣!」儒罵曰:「庸賊欲污我手邪!」趣斬之。漢璋至,奮臂擊殺數人,乃死,身首糜散。儒使宏主騎兵,厚賜之。文德元年,儒諜知行密糧乏,自高郵襲之。行密拔其眾還廬州,儒遂據揚州。



 駢之死,裹以故氈,與子弟七人一坎而瘞。行密擢駢孫愈為副使,令主喪事,未克葬,愈暴死,至是故吏鄺師虔收葬之。



 揚州雄富冠天下,自師鐸、行密、儒迭攻迭守,焚市落,剽民人,兵饑相仍,其地遂空。



 硃玫,邠州人。少以材武為州戍將。黃巢盜長安,有王玫者為偽節度使,方調兵,玫陽事之,乘間斬王玫,以留後讓李重古,約合兵討巢。廣明二年,玫襲賊,戰開遠門,槍洞咽,不死。以多擢晉州刺史,進邠寧節度使,合涇、原、岐、隴兵八萬屯興平,號定國砦。戰澇上,敗走邠,詔益靈、鹽軍,拜河南都統。引兵屯中橋,列五壁,進西北面都統。賊平,授同中書門下平章事,封吳興侯。



 田令孜議討王重榮,以兵屬玫,合鄜、延、靈、夏軍三萬保沙苑。重榮上疏乞誅玫、令孜。既戰,玫輒北,因縱軍還掠。僖宗蒼黃幸鳳翔避其鋒。玫反與重榮、李克用連和,請誅令孜。宰相蕭遘密召玫迎帝,玫趨鳳翔,令孜劫乘輿走陳倉,遂至興元。玫追不及,劫嗣襄王煴,奉為帝。玫自號大丞相,專決萬機。



 始與李昌符共謀挾煴,至是反為讎,昌符乃自歸天子,人心浸離。及王行瑜敗於大唐峰,懼歸且見殺,又聞購能得玫者以邠寧節度畀之,行瑜謂其下曰:「今敗歸必以無功死,若斬玫,與北軍迎天子,取富貴,可乎?」眾曰:「諾。」即勒兵倍道趨長安。玫居孔緯第,方據幾署事,聞兵入,趣召行瑜叱曰:「公擅歸,反邪?」行瑜厲聲曰:「我非反者,將得君首為邠寧節度耳!」玫遽起,左右斬之,殺其徒數百。諸軍遂大亂,燒京師。時盛寒,吏民被剽敚,殭死尸相藉。即傳首興元,帝為受俘馘。宦者偽樞密使王能著等皆坐誅。



 王行瑜,邠州人。少隸軍,從硃玫為列校,討黃巢數有功。煴即位,授行瑜天平節度使,令率兵守大散關,為李鋋所破,即奉款行在,還取玫首以獻,擢邠寧節度使。



 景福元年,與李茂貞、韓建及弟同州節度使行實請討楊守亮於山南,且言不敢仰度支費,止請假茂貞招討一節。宦官難之,昭宗亦顧茂貞等得山南則益橫,不許。行瑜等因擅興軍擊取之。



 後茂貞拒覃王,殺宰相,行瑜參有力,得賜鐵券。稍憑兵跋扈,求為尚書令,宰相韋昭度執不可,但加號尚父,行瑜望甚。會河中王重榮喪,李克用請以其子珂嗣節度,而行瑜、建、茂貞請授王珙,因各以兵陳闕下,欲廢天子,不克,即殺昭度、李磎,留弟行約宿衛。克用悉兵度河問行瑜等罪,行實棄同州趨長安,與行約謀劫乘輿,又不克,皆奔邠州。行瑜屯梨園,克用與戰,破行實等軍,執其母及行瑜子,俘大校。帝下詔削行瑜官爵。行瑜以銳卒五千營龍泉,茂貞壁其西。克用夜發精騎擾餉道,岐軍走,行瑜歸邠州,嬰城守,厚賂克用求自歸。克用軍環其城,行瑜窮,登城哭語克用曰:「我無罪,昨殺大臣,脅天子,岐人也。行實止宿衛,而有司妄以劫遷罪歸之,今公討亂者,當問茂貞,願得束身歸,聽命天子。」克用曰:「尚父何自卑?吾被命討三賊,公其一也。如歸國者,當從中決,老夫敢專之邪?」行瑜度不免,悉族奔慶州,為麾下斬於路,傳首京師,帝御延喜門納之,於是乾寧二年也。其屬二百人,克用獻於朝。



 始,行瑜亂,宗正卿李涪盛陳其忠,必悔過。至是帝怒,放死嶺南。



 陳敬瑄,田令孜兄也。少賤,為餅師,得隸左神策軍。令孜為護軍中尉,敬瑄緣藉擢左金吾衛將軍、檢校尚書右僕射、西川節度使。性畏慎,善撫士。



 黃巢亂,僖宗幸奉天,敬瑄夜召監軍梁處厚,號慟奉表迎帝,繕治行宮。令孜亦倡西幸,敬瑄以兵三千護乘輿。冗從內苑小兒先至,敬瑄知素暴橫,遣邏士伺之。諸兒連臂歡咋行宮中,士捕系之,言虖曰:「我事天子者!」敬瑄殺五十人,尸諸衢,由是道路不嘩。帝次綿州,敬瑄謁於道,進酒,帝三舉觴,進檢校左僕射、同中書門下平章事。時雲南叛,請遣使與和親,乃聽命。敬瑄奉行在百官諸吏無敢乏,帝欲命判度支,固讓,再加檢校司徒兼侍中,封梁國公。以弟敬珣為閬州刺史。討定邛州首望阡能、涪州叛校韓秀升,再進兼中書令,封潁川郡王,實封四百戶,賜一歲上輸錢及上都田宅邸磑各十區,鐵券恕十死。巢平,進潁川王,增實戶二百。車駕東,敬瑄供億豐餘,又進檢校太師。



 俄而令孜得罪,敬瑄被流端州。會昭宗立,敬瑄拒詔,帝召為左龍武統軍,以宰相韋昭度代領節度。使者至,敬瑄使百姓遮道剺耳訴己功,且言鐵券恕死。使者馳還。令孜勸敬瑄募黃頭軍為自守計。



 時王建盜據閬、利,故令孜召建。建至綿州,發兵拒之,激建攻諸州,以限朝廷。或言:「建鴟視狼顧,惟利是賴,公何用之?」不聽。建詒顧彥朗書曰:「十軍阿父召我,欲依太師丐一大州。」即寄孥梓州,身引兵入鹿頭關。敬瑄不納,漢州刺史張頊逆戰,敗,建入漢州。成都嚴守,建走城下遙謝令孜曰:「父召我,及門而拒我,尚誰容?」與諸將斷發再拜辭曰:「今作賊矣!」因請兵於彥朗,攻成都,殘掠州縣。彥朗亦畏建,表請大臣代敬瑄。建自請討敬瑄贖罪,詔立永平軍,授建節度使,以昭度為行營招討使,山南西道節度使楊守亮副之,彥朗為行軍司馬。有詔暴敬瑄殺孟昭圖罪,削官爵。昭度使建屯學射山,敬瑄迎戰不克,又戰蠶厓,大敗。



 龍紀元年,昭度至軍中,持節諭人,約開門。守陴者詬曰:「鐵券在,安得違先帝意!」今孜籍城中戶一人乘城,夜循行,晝浚濠伐薪。敬瑄屯彌牟、德陽,樹二壁拒建。使富人自占貲多少,布巨梃,搒不實者,不三日輸錢如市。建、昭度傅城而壘,簡州刺史張造攻笮橋,大敗,死之。



 大順元年,建稍擊降諸州。邛州刺史毛湘本令孜孔目官,謂其下曰:「吾不忍負軍容,以頭見建可也。」乃沐浴以須,吏斬其首降。敬瑄戰浣花,不勝,明日復戰,將士皆為建俘。城中謀降者,令孜支解之以怖眾。會大疫,死人相藉。



 明年三月,詔還敬瑄官爵,召昭度還,諭建罷兵,建不奉詔。帝更以建為西川行營招討制置使。建知敬瑄可禽,欲遂有蜀地,即脅說昭度曰:「公以數萬眾討賊,糧數不屬,關東諸節度相吞噬,朝廷危若贅斿,與其勞師遠方,不如先中國,公宜還為天子謀之。」昭度未決。會吏盜減諸軍稟食,建怒其眾曰:「招討吏之謀也。」縱士執之,醢食於軍。昭度大駭,是日授建符節,跳馳出劍門。建絕棧梯,東道不通。因急擊敬瑄,分親騎為十團,所當輒披靡,烽塹相望幾百里,縱諜入城,以搖眾心。建好謂軍中曰:「成都號『花錦城』,玉帛子女,諸兒可自取。」謂票將韓武等:「城破,吾與公遞為節度使一日。」下聞之,戰愈力。圍凡三歲,城中糧盡,以筒容米,率寸鬻錢二百。敬瑄出家貲給民,募士出剽麥,收其半。民亦夜至建壘市鹽,不可禁,吏請殺之。敬瑄曰:「民饑無以恤,使求生可也。」人至相暴以相啖,敬瑄不能止,乃行斬、劈二法,亦不為戢。敬瑄自將出犀浦,列二營邀建。建軍偽遁,遇伏,敬瑄敗,建破斜橋、昝街二屯。明日戰,又破一壁,降其將。建屯七里亭,敬瑄攻之。建將張武馳入城,戰子城下,守陴皆噪,不能克。張勍破浣花營,敬瑄諸將或死或降且盡。凡五十戰,敬瑄皆北,乃上表以病丐還京師。令孜素服至建軍。建入自西門,以張勍為斬斫使,建徇於軍曰:「與而等累年鬥死,今日如志。若橫恣有犯者,吾能全之;即為勍所斬,吾不得救也!」軍中肅然。囚敬瑄、令孜,建自稱留後,表於朝。詔以建為西川節度副大使,知節度事。



 建以敬瑄居新津,食其租賦,累表請誅,不報。景福二年,陰令左右告敬瑄、令孜養死士,約楊晟等反,於是斬敬瑄於家。初,敬瑄知不免,嘗置藥於帶,至就刑,視帶,藥已亡矣。自是建盡有兩川、黔中地。



 李巨川,字下己,逢吉從曾孫。乾符中舉進士。方天下崩騷,乃去京師,河中王重榮闢為掌書記。重榮討黃巢,書檄奏請日紛沓,須報趣發,皆屬巨川。神安思敏,言輒中理,鄰籓皆驚。會賊走出關,收京師,人言巨川有助力。重榮死於亂,貶為興元參軍,節度使楊守亮喜曰:「天以生遺我邪!」復管記室。守亮為韓建所禽,巨川械以從,題木葉遺建祈哀。建動容,因釋縛,置幕府。昭宗幸華,建患一州供億不能濟,使巨川傳檄天下,督轉餉。



 初,帝在石門,數遣嗣延王、通王將親軍,大選安聖、奉宸、保寧、安化四軍,又置殿後軍,合士二萬。建惡衛兵強,不利己,與巨川謀,即上飛變,告八王欲脅帝幸河中,因請囚十六宅,選嚴師傅督教,盡散麾下兵。書再上,帝不得已,詔可。又廢殿後軍,且言「無示天下不廣」。詔留三十人為控鶴排馬官,隸飛龍坊。自是天子爪牙盡矣。建初懼帝不聽,以兵環宮,請誅定州行營將李筠。帝懼,斬筠,兵乃解。又言:「七國災漢,八王亂晉,永王帥江左謀不軌,吐蕃、硃玫亂,首立宗支搖人望。今王室多故,渠可使諸王將命四方,惑征鎮?」於是詔諸王奉使者,悉赴行在。巨川日夜導建不臣,乃請立德王為皇太子,文掩其惡。帝還京,拜諫議大夫。



 光化初,硃全忠陷河中,將攻潼關,建懼,使巨川往詣軍納款,因言當世利害。全忠屬官敬翔以文翰事左右,疑巨川用則全忠待己或衰,乃詭說曰:「巨川誠奇才,顧不利主人,若何?」是日,全忠殺之。



\end{pinyinscope}