\article{本紀第四 則天皇后 中宗}

\begin{pinyinscope}

 則天順聖皇后武氏諱曌,並州文水人也。父士彠,官至工部尚書、荊州都督,封應國公。



 後年十四,太宗聞其有色,選為才人。太宗崩,後削發為比丘尼,居於感業寺。高宗幸感業寺,見而悅之,復召入宮。久之,立為昭儀,進號宸妃。永徽六年,高宗廢皇后王氏,立宸妃為皇后。



 高宗自顯慶後,多苦風疾,百司奏事,時時令後決之,常稱旨,由是參豫國政。後既專寵與政,乃數上書言天下利害,務收人心,而高宗春秋高,苦疾,後益用事,遂不能制。高宗悔,陰欲廢之,而謀洩不果。上元元年,高宗號天皇,皇后亦號天後,天下之人謂之「二聖」。



 弘道元年十二月,高宗崩,遺詔皇太子即皇帝位,軍國大務不決者,兼取天後進止。甲子,皇太子即皇帝位,尊後為皇太后,臨朝稱制。大赦,賜九品以下勛官一級。庚午,韓王元嘉為太尉,霍王元軌為司徒,舒王元名為司空。甲戌,劉仁軌為尚書左僕射,裴炎為中書令,劉齊賢為侍中:同中書門下三品。戊寅,郭待舉、魏玄同、岑長倩同中書門下三品。癸未,郭正一罷。



 光宅元年正月癸未,改元嗣聖。癸巳,左散騎常侍韋弘敏為太府卿、同中書門下三品。二月戊午,廢皇帝為廬陵王,幽之。己未,立豫王旦為皇帝,妃劉氏為皇后,立永平郡王成器為皇太子。大赦,改元為文明。賜文武官五品以上爵一等、九品以上勛兩轉。老人版授官,賜粟帛。職官五品以上舉所知一人。皇太后仍臨朝稱制。庚申,廢皇太孫重照為庶人,殺庶人賢於巴州。甲子,皇帝率群臣上尊號於武成殿。丁卯,冊皇帝。丁丑,太常卿王德真為侍中,中書侍郎劉禕之同中書門下三品。庚辰,贈玉清觀道士太中大夫王遠知金紫光祿大夫。三月丁亥,徙封上金為畢王,素節葛王。四月丁巳,滕王元嬰薨。辛酉,徙封上金為澤王,素節許王。癸酉,遷廬陵王於房州;丁丑,又遷於均州。五月癸巳,以大喪禁射獵。閏月甲子,禮部尚書武承嗣為太常卿、同中書門下三品。七月戊午,廣州昆侖殺其都督路元睿。乙丑,突厥寇朔州,左武衛大將軍程務挺敗之。辛未,有彗星出於西方。八月庚寅,葬天皇大帝於乾陵。丙午,武承嗣罷。九月甲寅,大赦,改元。旗幟尚白,易內外官服青者以碧,大易官名,改東都為神都,追尊老子母為先天太后。丙辰,左威衛大將軍程務挺為單于道安撫大使,以備突厥。己巳,追尊武氏五代祖克己為魯國公,妣裴氏為魯國夫人;高祖居常為太尉、北平郡王,妣劉氏為王妃;曾祖儉為太尉、金城郡王,妣宋氏為王妃;祖華為太尉、太原郡王,妣趙氏為王妃;考士皞為太師、魏王,妣楊氏為王妃。丁丑,柳州司馬李敬業舉兵於揚州以討亂。貶韋弘敏為汾州刺史。十月癸未,楚州司馬李崇福以山陽、安宜、鹽城三縣歸於敬業。甲申,左玉鈐衛大將軍梁郡公孝逸為揚州道行軍大總管,左金吾衛大將軍李知十為副,率兵三十萬以拒李敬業。丁亥,左肅政臺御史大夫騫味道檢校內史、同鳳閣鸞臺三品,鳳閣舍人李景諶同鳳閣鸞臺平章事。壬辰,李敬業克潤州。丙申,殺裴炎。追謚五代祖魯國公曰靖,高祖北平郡王曰恭肅,曾祖金城郡王曰義康,祖太原郡王曰安成,考魏王曰忠孝。丁酉,曲赦揚、楚二州。復敬業姓徐氏。貶劉齊賢為辰州刺史。李景諶罷。右史沈君諒、著作郎崔察為正諫大夫、同鳳閣鸞臺平章事。十一月辛亥,左鷹揚衛大將軍黑齒常之為江南道行軍大總管。庚申,右監門衛將軍蘇孝祥及徐敬業戰於阿谿,死之。乙丑,徐敬業將王那相殺敬業降。丁卯,郭待舉罷。鸞臺侍郎韋方質為鳳閣侍郎、同鳳閣鸞臺平章事。十二月戊子,遣御史察風俗。癸卯,殺程務挺。



 垂拱元年正月丁未,大赦,改元。庚戌,騫味道守內史。戊辰,劉仁軌薨。二月乙巳,春官尚書武承嗣、秋官尚書裴居道、右肅政臺御史大夫韋思謙同鳳閣鸞臺三品。突厥寇邊,左玉鈐衛中郎將淳于處平為陽曲道行軍總管以擊之。沈君諒罷。三月,崔察罷。丙辰,遷廬陵王於房州。辛酉,武承嗣罷。辛未,頒《垂拱格》。四月丙子,貶騫味道為青州刺史。癸未,淳于處平及突厥戰於忻州,敗績。五月丙午,裴居道為納言。丁未,流王德真於象州。己酉,冬官尚書蘇良嗣守納言。封皇帝子成義為恆王。壬戌,以旱慮囚。壬申,韋方質同鳳閣鸞臺三品。六月,天官尚書韋待價同鳳閣鸞臺三品。九月丁卯,揚州地生毛。十一月癸卯,韋待價為燕然道行軍大總管,以擊突厥。



 二年正月辛酉,大赦,賜酺三日,內外官勛一轉。二月辛未朔,日有蝕之。三月戊申,作銅匭。四月庚辰,岑長倩為內史。五月丙午,裴居道為內史。六月辛未,蘇良嗣同鳳閣鸞臺三品。己卯,韋思謙守納言。十月己巳,有山出於新豐縣,改新豐為慶山,赦囚,給復一年,賜酺三日。十二月,免並州百姓庸、調,終其身。是冬,無雪。



 三年閏正月丁卯,封皇帝子隆基為楚王,隆範衛王,隆業趙王。二月己亥,以旱避正殿,減膳。丙辰,突厥寇昌平,黑齒常之擊之。三月乙丑,韋思謙罷。四月辛丑,追號孝敬皇帝妃裴氏曰哀皇后,葬於恭陵。癸丑,以旱慮囚,命京官九品以上言事。壬戌,裴居道為納言。五月丙寅,夏官侍郎張光輔為鳳閣侍郎、同鳳閣鸞臺平章事。庚午,殺劉禕之。七月丁卯,冀州雌雞化為雄。乙亥,京師地震,雨金於廣州。八月壬子,魏玄同兼檢校納言,交趾人李嗣仙殺安南都護劉延祐,據交州,桂州司馬曹玄靜敗之。是月,突厥寇朔州,燕然道行軍大總管黑齒常之敗之。九月巳卯,虢州人楊初成自稱郎將,募州人迎廬陵王於房州,不果,見殺。十月庚子,右監門衛中郎將爨寶璧及突厥戰,敗績。十二月壬辰,韋待價為安息道行軍大總管,安西大都護閻溫古副之,以擊吐蕃。是歲,大饑。



 四年正月甲子,增七廟,立高祖、太宗、高宗廟於神都。庚午,毀乾元殿,作明堂。三月壬戌,殺麟臺少監周思茂。四月戊戌,殺太子通事舍人郝象賢。五月庚申,得「寶圖」於洛水。乙亥,加尊號為聖母神皇。六月丁亥朔,日有食之。得瑞石於汜水。七月丁巳,大赦,改「寶圖」為「天授聖圖」,洛水為永昌洛水,封其神為顯聖侯,加特進,禁漁釣。改嵩山為神岳,封其神為天中王、太師、使持節、大都督。賜酺五日。戊午,京師地震。八月戊戌,神都地震。丙午,博州刺史瑯邪郡王沖舉兵以討亂,遣左金吾衛大將軍丘神勣拒之。戊申,沖死之。庚戌,越王貞舉兵於豫州以討亂。辛亥,曲赦博州。九月丙辰,左豹韜衛大將軍曲崇裕為中軍大總管,岑長倩為後軍大總管,以拒越王貞;張光輔為諸軍節度。削越王貞及瑯邪郡王沖屬籍,改其姓為虺氏。貞死之。丙寅,赦豫州。殺韓王元嘉、魯王靈夔、範陽郡王靄、黃國公詵、東莞郡公融及常樂公主,皆改其姓為虺氏。丁卯,左肅政臺御史大夫騫味道、夏官侍郎王本立同鳳閣鸞臺平章事。十月辛亥,大風拔木。十一月辛酉,殺濟州刺名薛顗及其弟駙馬都尉紹。十二月乙酉,殺霍王元軌、江都郡王緒及殿中監裴承光。大殺唐宗室,流其幼者於嶺南。己亥,殺騫味道。己酉,拜洛受圖。辛亥,改明堂為萬象神宮,大赦。



 永昌元年正月乙卯,享於萬象神宮,大赦,改元,賜酺七日。丁巳,舒王元名為司徒。戊午,布政於萬象神宮,頒九條以訓百官。己未,朗州雌雞化為雄。二月丁酉,尊考太師魏忠孝王曰周忠孝太皇。置崇先府官。戊戌,追謚妣楊氏曰周忠孝太后;太原郡王曰周安成王,妃趙氏為王妃;金城郡王曰魏義康王,妣宋氏為王妃;北平郡王曰趙肅恭王,妃劉氏為王妃;五代祖魯國公曰太原靖王,夫人裴氏為王妃。三月甲子,張光輔守納言。癸酉,天官尚書武承嗣為納言,張光輔守內史。四月甲辰,殺汝南郡王瑋、鄱陽郡公諲、廣漢郡公謐、汶山郡公蓁、零陵郡王俊、廣都郡公,徙其家於巂州。己酉,殺天官侍郎鄧玄挺。五月丙辰,韋待價及吐蕃戰於寅識迦河,敗績。己巳,白馬寺僧薛懷義為新平道行軍大總管,以擊突厥。七月丁巳,流紀王慎於巴州,改其姓為虺氏。丙子,流韋待價於繡州,殺閻溫古。戊寅,王本立同鳳閣鸞臺三品。八月癸未,薛懷義為新平道中軍大總管,以擊突厥。甲申,殺張光輔、洛州司馬弓嗣業、洛陽令弓嗣明、陜州參軍弓嗣古、流人徐敬真。乙未,松州雌雞化為雄。辛丑,殺陜州刺史郭正一。丁未,殺相州刺史弓志元、蒲州刺史弓彭祖、尚方監王令基。九月庚戌,殺恆山郡王承乾之子厥。閏月甲午,殺魏玄同、夏官侍郎崔察。戊申,殺彭州長史劉易從。十月癸丑,殺涼州都督李光誼。丁巳,殺陜州刺史劉延景。戊午,殺右武威衛大將軍黑齒常之、右鷹揚衛將軍趙懷節。己未,殺嗣鄭王璥。丁卯,春官尚書範履冰、鳳閣侍郎邢文偉同鳳閣鸞臺平章事。



 天授元年正月庚辰,大赦,改元曰載初,以十一月為正月,十二月為臘月,來歲正月為一月。以周、漢之後為二王後,封爵、禹、湯之裔為三恪,周、隋同列國,封其嗣。乙未,除唐宗室屬籍。臘月丙寅,殺劉齊賢。一月戊子,王本立罷。邢文偉為內史,岑長倩、武承嗣同鳳閣鸞臺三品,鳳閣侍郎武攸寧為納言。甲午,流韋方質於儋州。二月丁卯,殺地官尚書王本立。三月乙酉,以旱減膳。丁亥,蘇良嗣薨。五月戊子,殺範履冰。己亥,殺梁郡公孝逸。六月戊申,殺汴州刺史柳明肅。七月辛巳,流舒王元名於和州。頒《大雲經》於天下。壬午,殺豫章郡王鳷。丁亥,殺澤王上金、許王素節。甲午,赦永昌縣。癸卯,殺太常丞蘇踐言。八月辛亥,殺許王素節之子璟、曾江縣令白令言。甲寅,殺裴居道。壬戌,殺將軍阿史那惠、右司郎中喬知之。癸亥,殺尚書右丞張行廉、太州刺史杜儒童。甲子,殺流人張楚金。戊辰,殺流人元萬頃、苗神客。辛未,殺南安郡王穎、鄅國公昭及諸宗室李直、李敞、李然、李勛、李策、李越、李黯、李玄、李英、李志業、李知言、李玄貞。九月乙亥,殺鉅鹿郡公晃、麟臺郎裴望及其弟司膳丞璉。壬午,改國號周。大赦,改元,賜酺七日。乙酉,加尊號曰聖神皇帝,降皇帝為皇嗣,賜姓武氏,皇太子為皇孫。丙戌,立武氏七廟於神都。追尊周文王曰始祖文皇帝,妣姒氏曰文定皇后;四十代祖平王少子武曰睿祖康皇帝,妣姜氏曰康惠皇后;太原靖王曰嚴祖成皇帝,妣曰成莊皇后;趙肅恭王曰肅祖章敬皇帝,妣曰章敬皇后;魏義康王曰烈祖昭安皇帝,妣曰昭安皇后;周安成王曰顯祖文穆皇帝,妣曰文穆皇后;忠孝太皇曰太祖孝明高皇帝,妣曰孝明高皇后。追封伯父及兄弟之子為王,堂兄為郡王,諸姑姊為長公主,堂姊妹為郡主。司賓卿史務滋守納言,鳳閣侍郎宗秦客檢校內史,給事中傅游藝為鸞臺侍郎、同鳳閣鸞臺平章事。十月丁巳,給復並州武興縣百姓,子孫相承如漢豐、沛。甲子,貶宗秦客為遵化尉。丁卯,殺流人韋方質。己巳,殺許王素節之子瑛、琪、琬、瓚、瑒、瑗、琛、唐臣。辛未,貶邢文偉為珍州刺史。置大雲寺。封周公為褒德王,孔子為隆道公。改唐太廟為享德廟,以武氏七廟為太廟。



 二年正月甲戌,改置社稷,旗幟尚赤。戊寅,殺雅州刺史劉行實及其弟渠州刺史行瑜、尚衣奉御行感、兄子左鷹揚衛將軍虔通。戊子,武承嗣為文昌左相。庚寅,賜酺。乙未,殺丘神勣、左豹韜衛將軍衛蒲山。庚子,殺史務滋。臘月己未,始用周臘。四月壬寅朔,日有蝕之。丙午,大赦。五月丁亥,大風折木。岑長倩為武威道行軍大總管,以擊吐蕃。六月庚戌,左肅政臺御史大夫格輔元為地官尚書,鸞臺侍郎樂思晦,鳳閣侍郎任知古:同鳳閣鸞臺平章事。七月庚午,徙關內七州戶以實神都。八月戊申,武攸寧罷。夏官尚書歐陽通為司禮卿兼判納言事。庚申,殺右玉鈐衛大將軍張虔勖。九月乙亥,殺岐州刺史雲弘嗣。壬辰,殺傅游藝。癸巳,左羽林衛大將軍武攸寧守納言,冬官侍郎裴行本,洛州司馬狄仁傑為地官侍郎:同鳳閣鸞臺平章事。十月己酉,殺岑長倩、歐陽通、格輔元。壬戌,殺樂思晦、左衛將軍李安靜。



 長壽元年正月戊辰,夏官尚書楊執柔同鳳閣鸞臺平章事。庚午,貶任知古為江夏令,狄仁傑彭澤令。流裴行本於嶺南。乙亥,殺右衛大將軍泉獻誠。庚辰,司刑卿李游道為冬官尚書、同鳳閣鸞臺平章事。二月戊午,秋官尚書袁智弘同鳳閣鸞臺平章事。四月丙申朔,日有食之。大赦,改元如意。五月,洛水溢。七月,又溢。八月甲戌,河溢,壞河陽縣。戊寅,武承嗣、武攸寧、楊執柔罷;秋官侍郎崔元綜為鸞臺侍郎,夏官侍郎李昭德為鳳閣侍郎,權檢校天官侍郎姚為文昌左丞,檢校地官侍郎李元素為文昌右丞,營繕大匠王璿為夏官尚書,司賓卿崔神基:同鳳閣鸞臺平章事。九月戊戌,大霧。庚子,大赦,改元。改用九月社,賜酺七月。癸卯,以並州為北都。癸丑,流李游道、袁智弘、王璿、崔神基、李元素於嶺南。十月丙戌,武威道行軍總管王孝傑敗吐蕃,克四鎮。



 二年臘月癸亥,殺皇嗣妃劉氏、德妃竇氏。丁卯,降封皇孫成器為壽春郡王,恆王成義衡陽郡王,楚王隆基臨淄郡王,衛王隆範巴陵郡王,越王隆業彭城郡王。一月庚子,夏官侍郎婁師德同鳳閣鸞臺平章事。甲寅,殺尚方監裴匪躬、內常侍範雲仙。三月己卯,殺左衛員外大將軍阿史那元慶、白潤府果毅薛大信。五月乙未,殺冬官尚書蘇乾、相州刺史來同敏。癸丑,河溢棣州。九月丁亥朔,日有食之。乙未,加號金輪聖神皇帝,大赦,賜酺七日,作七寶。庚子,追尊烈祖昭安皇帝曰渾元昭安皇帝,顯祖文穆皇帝曰立極文穆皇帝,太祖孝明高皇帝曰無上孝明高皇帝。辛丑,姚罷。文昌右丞韋巨源同鳳閣鸞臺平章事,秋官侍郎陸元方為鸞臺侍郎、同鳳閣鸞臺平章事,司賓卿豆盧欽望守內史。



 延載元年臘月甲戌,突厥默啜寇靈州。右鷹揚衛大將軍李多祚敗之。一月甲午,婁師德為河源、積石、懷遠等軍營田大使。二月庚午,薛懷義為伐逆道行軍大總管,領十八將軍以擊默啜。乙亥,以旱慮囚。己卯,武威道大總管王孝傑及吐蕃戰於冷泉,敗之。三月甲申,鳳閣舍人蘇味道為鳳閣侍郎、同鳳閣鸞臺平章事,李昭德檢校內史。薛懷義為朔方道行軍大總管,擊默啜。昭德為朔方道行軍長史,味道為司馬。四月壬戌,常州地震。五月甲午,加號越古金輪聖神皇帝,大赦,改元,賜酺七日。七月癸未,嵩岳山人武什方為正諫大夫、同鳳閣鸞臺平章事。八月,什方罷。戊辰,王孝傑為瀚海道行軍總管。己巳,司賓少卿姚守納言;左肅政臺御史大夫楊再思為鸞臺侍郎,洛州司馬杜景佺檢校鳳閣侍郎:同鳳閣鸞臺平章事。戊寅,流崔元綜於振州。九月壬午朔,日有食之。壬寅,貶李昭德為南賓尉。十月壬申,文昌右丞李元素為鳳閣侍郎,右肅政臺御史中丞周允元檢校鳳閣侍郎:同鳳閣鸞臺平章事。嶺南獠寇邊,容州都督張玄遇為桂、永等州經略大使。癸酉,雨木冰。



 天冊萬歲元年正月辛巳,加號慈氏越古金輪聖神皇帝,改元證聖。大赦,賜酺三日。戊子,貶豆盧欽望為趙州刺史,韋巨源鄜州刺史,杜景佺溱州刺史,蘇味道集州刺史,陸元方綏州刺史。丙申,萬象神宮火。丙午,王孝傑為朔方行軍總管,以擊突厥。二月己酉朔,日有食之。壬子,殺薛懷義。甲子,罷「慈氏越古」號。三月丙辰,周允元薨。四月戊寅,建大周萬國頌德天樞。七月辛酉,吐蕃寇臨洮,王孝傑為肅邊道行軍大總管以擊之。九月甲寅,祀南郊。加號天冊金輪大聖皇帝。大赦,改元,賜酺九日。以崇先廟為崇尊廟。



 萬歲通天元年臘月甲戌,如神嶽。甲申,封於神嶽。改元曰萬歲登封。大赦,免今歲租稅,賜酺十日。丁亥,禪於少室山。己丑,給復洛州二年,登封、告成縣三年。癸巳,復於神都。一月甲寅,婁師德為肅邊道行軍副總管,以擊吐蕃。己巳,改崇尊廟為太廟。二月辛巳,尊神嶽天中王為神嶽天中黃帝,天靈妃為天中黃後。三月壬寅,王孝傑、婁師德及吐蕃戰於素羅汗山,敗績。丁巳,復作明堂,改曰通天宮。大赦,改元,賜酺七日。四月癸酉,檢校夏官侍郎孫元亨同鳳閣鸞臺平章事。庚子,貶婁師德為原州都督府司馬。五月壬子,契丹首領松漠都督李盡忠、歸誠州刺史孫萬榮陷營州,殺都督趙文翽。乙丑,左鷹揚衛將軍曹仁師、右金吾衛大將軍張玄遇、左威衛大將軍李多祚、司農少卿麻仁節等擊之。七月辛亥,春官尚書武三思為榆關道安撫大使,納言姚為副,以備契丹。八月丁酉,張玄遇、曹仁師、麻仁節等及契丹戰於黃麞,敗績,執玄遇、仁節。九月庚子,同州刺史武攸宜為清邊道行軍大總管,以擊契丹。丁巳,吐蕃寇涼州,都督許欽明死之。庚申,並州長史王方慶為鸞臺侍郎,殿中監李道廣:同鳳閣鸞臺平章事。十月辛卯,契丹寇冀州,刺史陸寶積死之。甲午,慮囚。



 神功元年正月壬戌,殺李元素、孫元亨、洛州錄事參軍綦連耀、箕州刺史劉思禮、知天官侍郎事石抱忠劉奇、給事中周譒、鳳閣舍人王抃、前涇州刺史王勔、太子司議郎路敬淳、司門員外郎劉順之、右司員外郎宇文全志、來庭縣主簿柳璆。癸亥,突厥默啜寇勝州,平狄軍副使安道買敗之。甲子,婁師德守鳳閣侍郎、同鳳閣鸞臺平章事。二月乙巳,慮囚。三月庚子,王孝傑及孫萬斬戰於東硤石谷,敗績,孝傑死之。戊申,赦河南、北。四月戊辰,置九鼎於通天宮。癸酉,前益州大都督府長史王及善為內史。癸未,右金吾衛大將軍武懿宗為神兵道行軍大總管,及右豹韜衛將軍何迦密以擊契丹。五月癸卯,婁師德為清邊道行軍副大總管,右武衛將軍沙吒忠義為清邊中道前軍總管,以擊契丹。六月丁卯,殺監察御史李昭德、司僕少卿來俊臣。己卯,尚方少監宗楚客同鳳閣鸞臺平章事。戊子,特進武承嗣、春官尚書武三思同鳳閣鸞臺三品。辛卯,婁師德安撫河北。七月丁酉,武承嗣、武三思罷。八月丙戌,姚罷。九月壬寅,大赦,改元,賜酺七日。庚戌,婁師德守納言。十月甲子,給復徇忠、立節二縣一年。閏月甲寅,檢校司刑卿、幽州都督狄仁傑為鸞臺侍郎,司刑卿杜景佺為鳳閣侍郎:同鳳閣鸞臺平章事。



 聖歷元年正月甲子,大赦,改元,賜酺九日。丙寅,宗楚客罷。丁亥,李道廣罷。三月己巳,召廬陵王於房州。戊子,廬陵王至自房州。四月庚寅,赦神都及河北。辛丑,婁師德為隴右諸軍大使,檢校河西營田事。五月庚午,禁屠。六月乙卯,大風拔木。七月辛未,杜景佺罷。八月,突厥寇邊。戊子,左豹韜衛將軍閻知微降於突厥,寇邊。甲午,王方慶罷。庚子,春官尚書武三思檢校內史,狄仁傑兼納言。司屬卿武重規為天兵中道大總管,沙吒忠義為天兵西道前軍總管,幽州都督張仁亶為天兵東道總管,左羽林衛大將軍李多祚、右羽林衛大將軍閻敬容為天兵道西後軍總管,以擊突厥。癸丑,突厥寇蔚州。乙卯,寇定州,刺史孫彥高死之。九月甲子,夏官尚書武攸寧同鳳閣鸞臺三品。戊辰,突厥寇趙州,長史唐波若降於突厥,刺史高睿死之。突厥寇相州,沙吒忠義為河北道前軍總管,將軍陽基副之,李多祚為後軍總管,大將軍富福信為奇兵總管,以御之。壬申,立廬陵王顯為皇太子,大赦,賜酺五日。甲戌,皇太子為河北道行軍元帥,以擊突厥。戊寅,狄仁傑為河北道行軍副元帥、檢校納言。辛巳,試天官侍郎蘇味道為鳳閣侍郎、同鳳閣鸞臺平章事。十月癸卯,狄仁傑為河北道安撫大使。夏官侍郎姚元崇、麟臺少監李嶠同鳳閣鸞臺平章事。族閻知微。



 二年正月壬戌,封皇嗣旦為相王。臘月戊子,左肅政臺御史中丞吉頊為天官侍郎,檢校右肅政臺御史中丞魏元忠為鳳閣侍郎:同鳳閣鸞臺平章事。辛亥,賜皇太子姓武氏,大赦。一月庚申,武攸寧罷。二月己丑,如緱氏。辛卯,如嵩陽。丁酉,復於神都。三月甲戌,以隋、唐為二王後。婁師德為納言。四月壬辰,魏元忠檢校並州大都督府長史、天兵軍大總管,婁師德副之,以備突厥。辛丑,類師德為隴右諸軍大使。甲辰,慮囚。七月丙辰,神都大雨,洛水溢。八月庚子,王及善為文昌左相、同鳳閣鸞臺平章事,太子宮尹豆盧欽望為文昌右相、同鳳閣鸞臺三品。楊再思罷。丁未,試天官侍郎陸元方為鸞臺侍郎、同鳳閣鸞臺平章事。婁師德薨。戊申,武三思為內史。九月乙亥,如福昌縣,曲赦。戊寅,復於神都。庚辰,王及善薨。是秋,黃河溢。十月丁亥,吐蕃首領贊婆來。



 久視元年正月戊午,貶吉頊為琰川尉。壬申,武三思罷。臘月辛巳,封皇太子之子重潤為邵王。庚寅,陸元方罷司禮卿。阿史那斛瑟羅為平西軍大總管。丁酉,狄仁傑為內史。庚子,文昌左相韋巨源為納言。乙巳,如嵩山。一月丁卯,如汝州溫湯。戊寅,復於神都。作三陽宮。二月乙未,豆盧欽望罷。三月癸丑,夏官尚書唐奉一為天兵中軍大總管,以備突厥。四月戊申,如三陽宮。五月己酉朔,日有食之。癸丑,大赦,改元,罷「天冊金輪大聖」號,賜酺五日,給復告成縣一年。閏七月戊寅,復於神都。己丑,天官侍郎張錫為鳳閣侍郎、同鳳閣鸞臺平章事。李嶠罷。丁酉,吐蕃寇涼州,隴右諸軍州大使唐休璟敗之於洪源谷。八月庚戌,魏元忠為隴右諸軍州大總管,以擊吐蕃。庚申,斂天下僧錢作大像。九月辛丑,狄仁傑薨。十月辛亥,魏元忠為蕭關道行軍大總管,以備突厥。甲寅,復唐正月,大赦。丁巳,韋巨源罷。文昌右丞韋安石為鸞臺侍郎、同鳳閣鸞臺平章事。丁卯,如新安隴澗山,曲赦。壬申,復於神都。十二月甲寅,突厥寇隴右。



 長安元年正月丁丑,改元大足。二月己酉,鸞臺侍郎李懷遠同鳳閣鸞臺平章事。三月丙申,流張錫于循州。四月丙午,大赦癸丑,姚元崇檢校並州以北諸軍州兵馬。五月乙亥,如三陽宮。丁丑,魏元忠為靈武道行軍大總管,以備突厥。丙申,天官侍郎顧琮同鳳閣鸞臺平章事。六月庚申,夏官侍郎李迥秀同鳳閣鸞臺平章事。辛未,赦告成縣。七月甲戌,復於神都。乙亥,揚、楚、常、潤、蘇五州地震。壬午,蘇味道按察幽、平等州兵馬。甲申,李懷遠罷。九月壬申,殺邵王重潤及永泰郡主、主婿武延基。十月壬寅,如京師。辛酉,大赦,改元。給復關內三年,賜酺三日。丙寅,魏元忠同鳳閣鸞臺三品。十一月壬申,武三思罷。戊寅,改含元宮為大明宮。



 二年正月,突厥寇鹽州。三月丙戌,李迥秀安置山東軍馬,檢校武騎兵。庚寅,突厥寇並州,雍州長史薛季昶持節山東防禦大使以備之。七月甲午,突厥寇代州。八月辛亥,劍南六州地震。九月乙丑朔,日有食之。壬申,突厥寇忻州。己卯,吐蕃請和。十月甲辰,顧琮薨。戊申,吐蕃寇悉州,茂州都督陳大慈敗之。甲寅,姚元崇同鳳閣鸞臺平章事,蘇味道、韋安石、李迥秀同鳳閣鸞臺三品。十一月甲子,相王旦為司徒。戊子,祀南郊,大赦,賜酺三日。十二月甲午,魏元忠為安東道安撫使。



 三年三月壬戌朔,日有食之。四月庚子,相王旦罷。吐蕃來求婚。乙巳,以旱避正殿。閏月庚午,成均祭酒李嶠同鳳閣鸞臺平章事。己卯,李嶠知納言事。七月壬寅,正諫大夫硃敬則同鳳閣鸞臺平章事。庚戌,檢校涼州都督唐休璟為夏官尚書、同鳳閣鸞臺平章事。八月乙酉,京師大雨雹。九月庚寅朔,日有蝕之。丁酉,貶魏元忠為高要尉。十月丙寅,如神都。十二月丙戌,天下置關三十。



 四年正月丁未,作興泰宮。壬子,天官侍郎韋嗣立為鳳閣侍郎、同鳳閣鸞臺三品。二月癸亥,貶李迥秀為廬州刺史。壬申,硃敬則罷。三月丁亥,進封皇孫平恩郡王重福為譙王。己亥,夏官侍郎宗楚客同鳳閣鸞臺平章事。貶蘇味道為坊州刺史。四月壬戌,韋安石知納言事,李嶠知內史事。丙子,如興泰宮,赦壽安縣,給復一年。五月丁亥,大風拔木。六月辛酉,姚元之罷。乙丑,天官侍郎崔玄為鸞臺侍郎、同鳳閣鸞臺平章事。丁丑,李嶠同鳳閣鸞臺三品。壬午,相王府長史姚元之兼知夏官尚書、同鳳閣鸞臺三品。七月丙戌,左肅政臺御史大夫楊再思守內史。甲午,復於神都。貶宗楚客為原州都督。八月庚申,唐休璟兼幽營二州都督、安東都護。九月壬子,姚元之為靈武道行軍大總管。十月辛酉,元之為靈武道安撫大使。甲戌,判秋官侍郎張柬之同鳳閣鸞臺平章事。壬午,懷州長史房融為正諫大夫、同鳳閣鸞臺平章事。十一月丁亥,天官侍郎韋承慶行鳳閣侍郎、同鳳閣鸞臺平章事。李嶠罷。十二月丙辰,韋嗣立罷。



 五年正月壬午,大赦。庚寅,禁屠。癸卯,張柬之、崔玄及左羽林衛將軍敬暉、檢校左羽林衛將軍桓彥範、司刑少卿袁恕己、左羽林衛將軍李湛薛思行趙承恩、右羽林衛將軍楊元琰、左羽林衛大將軍李多祚、職方郎中崔泰之、廟部員外郎硃敬則、司刑評事冀仲甫、檢校司農少卿兼知總監翟世言、內直郎王同皎率左右羽林兵以討亂;麟臺監張易之、春官侍郎張昌宗、汴州刺史張昌期、司禮少卿張同休、通事舍人張景雄伏誅。丙午,皇帝復於位。丁未,徙後於上陽宮。戊申,上後號曰則天大聖皇帝。十一月,崩,謚曰大聖則天皇后。唐隆元年,改為天後;景雲元年,改為大聖天後;延和元年,改為天后聖帝,未幾,改為聖後;開元四年,改為則天皇后;天寶八載,加謚則天順聖皇后。



 中宗大和大聖大昭孝皇帝諱顯,高宗第七子也。母曰則天順聖皇后武氏。高宗崩,以皇太子即皇帝位,而皇太后臨朝稱制。嗣聖元年正月,廢居於均州,又選於房州。聖歷二年,復為皇太子。太后老且病。



 神龍元年正月,張柬之等以羽林兵討亂。甲辰,皇太子監國,大赦,改元。丙午,復於位,大赦,賜文武官階、爵,民酺五日,免今歲租賦,給復房州三年,放宮女三千人。相王旦為安國相王、太尉、同鳳閣鸞臺三品。庚戌,張柬之、袁恕己同鳳閣鸞臺三品,崔玄守內史,敬暉為納言,相彥範守納言。二月甲寅,復國號唐。貶韋承慶為高要尉,流房融於高州。楊再思同中書門下三品。姚元之罷。甲子,皇后韋氏復於位,大赦,賜酺三日,復宗室死於周者官爵。丙寅,太子賓客武三思為司空、同中書門下三品。貶譙王重福為濮州刺史。丁卯,右散騎常侍、附馬都尉武攸暨為司徒。辛未,安國相王旦罷。甲戌,太子少詹事祝欽明同中書門下三品。韋安石罷。進封子義興郡王重俊為衛王,北海郡王重茂溫王。丁丑,武三思、武攸暨罷。三月甲申,詔文明後破家者昭洗之,還其子孫廕。己丑,袁恕己守中書令。四月辛亥,桓彥範為侍中,袁恕己為中書令。丁卯,高要尉魏元忠為衛尉卿、同中書門下平章事。辛未,敬暉為侍中。甲戌,魏元忠、崔玄,刑部尚書韋安石為吏部尚書,太子右庶子李懷遠為左散騎常侍,涼州都督唐休璟為輔國大將軍:同中書門下三品。乙亥,張柬之為中書令。五月壬午,遷武氏神主於崇恩廟。乙酉,立太廟、社稷於東都。戊子,復周、隋二王後。壬辰,進封兄成紀郡王千里為成王。甲午,敬暉、桓彥範、張柬之、袁恕己、崔韋玄罷。韋安石兼檢校中書令,魏元忠兼侍中。甲辰,唐休璟為尚書左僕射,特進豆盧欽望為尚書右僕射:同中書門下三品。六月壬子,左驍衛大將軍裴思諒為靈武道行軍大總管,以備突厥。癸亥,韋安石為中書令,魏元忠為侍中,楊再思檢校中書令,豆盧欽望平章軍國重事。七月辛巳,太子賓客韋巨源同中書門下三品。甲辰,洛水溢。八月戊申,給復河南、洛陽二縣一年。壬戌,追冊妃趙氏為皇后。乙亥,祔孝敬皇帝於東都太廟。皇后見於廟。丁丑,幸洛城南門,觀斗象。九月壬午,祀天地於明堂。大赦,賜文武官勛、爵,民為父後者古爵一級,酺三日。癸巳,韋巨源罷。十月癸亥,幸龍門。乙丑,獵於新安。辛未,魏元忠為中書令,楊再思為侍中。十一月戊寅,上尊號曰應天皇帝,皇后曰順天皇后。壬午,及皇后享於太廟,大赦,賜文武官階、勛、爵,民酺三日。己丑,幸洛城南門,觀潑寒胡戲。壬寅,皇太后崩,廢崇恩廟。



 二年正月戊戌,吏部尚書李嶠同中書門下三品,中書侍郎於惟謙同中書門下平章事。閏月丙午,公主開府置官屬。二月乙未,禮部尚書韋巨源為刑部尚書、同中書門下三品。丙申,遣十道巡察使。三月甲辰,韋安石罷。戶部尚書蘇環守侍中。戊申,唐休璟罷。庚戌,殺光祿卿、附馬都尉王同皎。是月,置員外官。四月己丑,李懷遠罷。己亥,雨毛於鄮縣。辛丑,洛水溢。五月庚申,葬則天大聖皇后。六月戊寅,貶敬暉為崖州司馬,桓彥範瀧州司馬,袁恕己竇州司馬,崔玄韋白州司馬,張柬之新州司馬。七月戊申,立衛王重俊為皇太子。丙寅,魏元忠為尚書右僕射、兼中書令,李嶠守中書令。辛未,左散騎常侍致仕李懷遠同中書門下三品。流敬暉於嘉州,桓彥範於瀼州,袁恕己於環州,崔玄韋於古州,張柬之於瀧州。八月丙子,貶祝欽明為申州刺史。九月戊午,李懷遠薨。十月癸巳,蘇環為侍中。戊戌,至自東都。十一月乙巳,大赦,賜行從官勛一轉。十二月己卯,靈武軍大總管沙吒忠義及突厥戰於鳴沙,敗績。丙戌,以突厥寇邊、京師旱、河北水,減膳,罷土木工。蘇環存撫河北。丙申,魏元忠為尚書左僕射。



 景龍元年正月丙辰,以旱慮囚。二月丙戌,復武氏廟、陵,置令、丞、守戶如昭陵。甲午,褒德廟、榮先陵置令、丞。四月庚寅,赦雍州。五月戊戌,右屯衛大將軍張仁亶為朔方道行軍大總管,以備突厥。丙午,假鴻臚卿臧思言使於突厥,死之。以旱避正殿,減膳。六月丁卯朔,日有食之。庚午,雨土於陜州。戊子,吐蕃及姚州蠻寇邊,姚雋道討擊使唐九徵敗之。七月辛丑,皇太子以羽林千騎兵誅武三思,不克,死之。癸卯,大赦。壬戌,李嶠為中書令。八月丙戌,上尊號曰應天神龍皇帝,皇后曰順天翊聖皇后。魏元忠罷。九月丁酉,吏部侍郎蕭至忠為黃門侍郎,兵部尚書宗楚客,左衛將軍兼太府卿紀處訥:同中書門下三品。於惟謙罷。庚子,大赦,改元。賜文武官階、勛、爵。辛亥,楊再思為中書令,韋巨源、紀處訥為侍中。蘇環罷。十月戊寅,殺習藝館內教蘇安恆。壬午,有彗星出於西方。十二月乙丑朔,日有食之。丁丑,雨土。



 二年二月癸未,有星隕於西南。庚寅,大赦,進五品以上母、妻封號二等,無妻者授其女,婦人八十以上版授郡、縣、鄉君。七月癸巳,朔方道行軍大總管張仁亶同中書門下三品。丁酉,有星孛於胃、昴。十一月庚申,西突厥寇邊,御史中丞馮嘉賓使於突厥,死之。己卯,大赦,賜酺三日。癸未,安西都護牛師獎及西突厥戰於火燒城,死之。是歲,皇后、妃、主、昭容賣官,行墨敕斜封。



 三年二月己丑,及皇后幸玄武門,觀宮女拔河,為宮市以嬉。壬寅,韋巨源為尚書左僕射,楊再思為右僕射:同中書門下三品。壬子,及皇后幸太常寺。三月戊午,宗楚客為中書令,蕭至忠守侍中,太府卿韋嗣立守兵部尚書、同中書門下三品。中書侍郎兼檢校吏部侍郎崔湜,守兵部侍郎趙彥昭為中書侍郎:同中書門下平章事。戊寅,禮部尚書韋溫為太子少保、同中書門下三品。太常少卿鄭愔守吏部侍郎、同中書門下平章事。五月丙戌,貶崔湜為瀼州刺史,鄭愔江州司馬。六月癸巳,太白晝見。庚子,以旱避正殿,減膳,撤樂。詔括天下圖籍。壬寅,慮囚。癸卯,楊再思薨。七月丙辰,西突厥娑葛降。辛酉,許婦人非緣夫、子封者廕其子孫。癸亥,慮囚。庚辰,澧水溢。八月乙酉,李嶠同中書門下三品,特進韋安石為侍中。壬辰,有星孛於紫宮。九月戊辰,吏部尚書蘇環為尚書左僕射、同中書門下三品。十一月乙丑,有事於南郊,以皇后為亞獻,大赦,賜文武官階、爵,入品者減考,免關內今歲賦,賜酺三日。甲戌,豆盧欽望薨。十二月壬辰,前宋國公致仕唐休璟為太子少師、同中書門下三品。甲午,如新豐溫湯。甲辰,赦新豐,給復一年,賜從官勛一轉。乙巳,至自新豐。



 四年正月丙寅,及皇后微行以觀燈,遂幸蕭至忠第。丁卯,微行以觀燈,幸韋安石、長寧公主第。己卯,如始平。二月壬午,赦咸陽、始平,給復一年。癸未,至自始平。庚戌,及後、妃、公主觀三品以上拔河。三月,以河源九曲予吐蕃。庚申,雨木冰,井溢。五月辛酉,封嗣虢王邕為汴王。丁卯,殺許州司兵參軍燕欽融。丁丑,剡縣地震。六月,皇后及安樂公主、散騎常侍馬秦客反。壬午,皇帝崩,年五十五,謚曰孝和皇帝天寶十三年,加謚大和大聖大昭孝皇帝。



 贊曰:昔者孔子作《春秋》而亂臣賊子懼,其於殺君篡國之主,皆不黜絕之,豈以其盜而有之者,莫大之罪也,不沒其實,所以著其大惡而不隱歟?自司馬遷、班固皆作《高后紀》,呂氏雖非篡漢,而盜執其國政,遂不敢沒其實,豈其得聖人之意歟?抑亦偶合於《春秋》之法也。唐之舊史因之,列武后於本紀,蓋其所從來遠矣。夫吉兇之於人,猶影響也,而為善者得吉常多,其不幸而罹於兇者有矣;為惡者未始不及於兇,其幸而免者亦時有焉。而小人之慮,遂以為天道難知,為善未必福,而為惡未必禍也。武后之惡,不及於大戮,所謂幸免者也。至中宗韋氏,則禍不旋踵矣。然其親遭母後之難,而躬自蹈之,所謂下愚之不移者歟!



\end{pinyinscope}