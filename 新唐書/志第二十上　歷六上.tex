\article{志第二十上 歷六上}

\begin{pinyinscope}

 憲宗即位,司天徐昴上新歷,名曰《觀象》。起元和二年用之,然無蔀章之數。至於察斂啟閉之候,循用舊法文獻編輯委員會編輯。收入毛澤東1921—1965年的著作68,測驗不合。至穆宗立,以為累世纘緒,必更歷紀,乃詔日官改撰歷術,名曰《宣明》。上元七曜,起赤道虛九度。其氣朔、發斂、日躔、月離,皆因《大衍》舊術;晷漏、交會,則稍增損之;更立新數,以步五星,其大略謂:



 通法曰統法。策實曰章歲。揲法曰章月。掛限曰閏限。三元之策曰中節。四象之策曰合策。一象之策曰象準。策餘曰通餘。爻數曰紀法。通紀法為分,曰旬周。章歲乘年,曰通積分。地中之策曰候策。天中之策曰卦策。以貞悔之策減中節,曰辰數。以加季月之節,即土用事日。凡小餘滿辰法,為辰數;滿刻法,為刻。乾實曰象數。秒法三百。以乘統法,曰分統。



 凡步七曜入宿度,皆以刻法為度母。凡刻法乘盈縮分,如定氣而一,曰氣中率。與後氣中率相減,為合差。以定氣乘合差,並後定氣以除,為中差。加、減氣率,為初、末率。倍中差,百乘之,以定氣除,為日差。半之,以加、減初、末,各為定率。以日差累加、減之,為每日盈縮分。凡百乘氣下先後數,先減、後加常氣,為定氣限數。乘歲差千四百四十,為秒分。以加中節,因冬至黃道日度,累而裁之,得每定氣初日度。



 入轉曰歷。凡入歷,如歷中已下為進;已上,去之,為退。凡定朔小餘,秋分後,四分之三已上,進一日。春分後,昏明小餘差春分初日者,五而一,以減四分之三。定朔小餘如此數已上者,進一日。或有交,應見虧初,則否。定弦望小餘,不滿昏明小餘者,退一日。或有交,應見虧初者,亦如之。凡正交,以平交入歷朓朒定數,朓減、朒加平交入定氣餘,滿若不足,進退日算,為正交入定氣,不復以交率乘、交數除,及不加減平交入氣朓朒也。



 凡推月度,以歷分乘夜半定全漏,如刻法而一,為晨分;以減歷分,為昏分。又以定朔、弦、望小餘乘歷分,統法除之,以減晨分,餘為前;不足,反相減,餘為後。乃前加、後減加時月度,為晨昏月度。以所入加時日度減後歷加時日度,餘加上弦之度及餘,以所入日前減、後加,又以後歷前加、後減,各為定程。乃累計距後歷每日歷度及分,以減定程,為盈;不足,反相減,為縮。以距後歷日數均其差,盈減、縮加每日歷分,為歷定分。累以加朔、弦、望晨昏月度,為每日辰昏月度,不復加減屈伸也。



 爻統曰中統。象積曰刻法。消息曰屈伸。以屈伸準盈縮分,求每日所入,日定衰。五乘之,二十四除之,曰漏差。屈加、伸減氣初夜半漏,得每日夜半定漏。刻法通為分,曰昏明小餘。二十一乘屈伸定數,二十五而一,為黃道屈伸差。乃屈減、伸加氣初去極度分,得每日去極度分。以萬二千三百八十六乘黃道屈伸差,萬六千二百七十七而一,為每日度差。屈減、伸加氣初距中度分,得每日距中度數。凡屈伸準消息於中晷,曰定數;於漏刻,曰漏差;於去極,曰屈伸差;於距中度,曰度差。



 交終曰終率。朔差曰交朔。望數日交望,交限曰前準。望差曰後準。凡月行入四象陰陽度有分者,十乘之,七而一,為度分。不盡,十五乘之,七除,為大分。不盡又除,為小分。乃以一象之度九十除之,兼除度差分百一十三、大分七、小分一少,然後以次象除之。



 凡日蝕,以定朔日出入辰刻距午正刻數,約百四十七,為時差。視定朔小餘如半法已下,以減半法,為初率;已上,減去半法,餘為末率。以乘時差,如刻法而一,初率以減,末率倍之,以加定朔小餘,為蝕定餘。月蝕,以定望小餘為蝕定餘。



 凡日蝕,有氣差,有刻差,有加差。二至之初,氣差二千三百五十。距二至前後,每日損二十六,至二分而空。以日出沒辰刻距午正刻數,約其朔日氣差,以乘食甚距午正刻數。所得以減氣差,為定數。春分後,陰歷加之,陽歷減之;秋分後,陰歷減之,陽歷加之。



 二至初日,無刻差。自後每日益差分二、小分十。起立春至立夏、起立秋至立冬,皆以九十四分有半為刻差。自後日損差分二、小分十,至二至之初損盡。以朔日刻差乘食甚距午正刻數,為刻差定數。冬至後食甚在午正前,夏至後食甚在午正後,陰歷以減,陽歷以加;冬至後食甚在午正後,夏至後食甚在午正前,陰歷以加,陽歷以減。



 又立冬初日後,每氣增差十七。至冬至初日,得五十一。自後,每氣損十七,終於大寒,損盡。若蝕甚在午正後,則每刻累益其差,陰歷以減,陽歷以加。應加減差,同名相從,異名相銷,各為蝕差。以加減去交分,為定分。月在陰歷,不足減,反減蝕差,交前減之,餘為陽歷交後定分;交後減之,餘為陽歷交前定分:皆不蝕。陽歷不足減,亦反減蝕差。交前減之,餘為陰歷交後定分;交後減之,餘為陰歷交前定分:皆蝕。



 凡去交定分,如陽歷蝕限已下,為陽歷蝕。以陽歷定法約,為蝕分。已上者,以陽歷蝕限減之,餘為陰歷蝕。以陰歷定法約之,以減十五,餘為蝕分。



 凡月蝕去交分,二千一百四十七已下,皆既。已上者,以減後準,餘如定法五百六約,為蝕分。凡月蝕既,泛用刻二十。如去交分千四百三十五已下,因增半刻。七百一十二已下,又增半刻。凡日月帶蝕出沒,各以定法通蝕分,半定用刻約之,以乘見刻。多於半定用刻,出為進,沒為退;少於半定用刻,出為退,沒為進:各如定法而一,為見蝕之大分。朔晝、望夜皆為見刻。其九服蝕差,則不復考詳。



 五星終率曰周率。因平合加中伏,得平見。金、水加夕,得晨;加晨,得夕。又以變差乘年,滿象數去之;不盡為變交。三百約為分,統法而一,以減平見。三十六乘平見秒,十二乘變交秒,同以三千六百為母。餘如交率已下,星在陽歷;已上,去之,為入陰歷。各以變策除,為變數,命初變算外;不盡為入其變度數及餘。自此百約餘分,母同刻法。以所入變下數,加減平見,為常見。金星晨見,先計自夕見,盡夕退,應加減先後差。同名相從,異名相銷。與晨常見加減差,異名相銷,同名相從。依加減晨平見為常見。



 凡常見計入定氣,求先從定數,各以差率乘之,差數而一,為定差。晨見先減、後加,夕見先加、後減常見,為定見。以常見與定見加減數,加減平見入變度數及餘秒,為定見初變所入。以所行度順加、退減之,即次變所入。以所入變下差數加減日度變率。其水星常見與定見加減數,同名相從,異名相銷,反其加減。夕見差加疾行日率者,倍其差,加度率。又分其差,以加遲留日率。晨見亦分其差,以加遲留日率,以所差之數,加疾行日率,亦倍其差,加疾行度率。夕見差減疾行日率者,倍其差,減度率。又以其差減留日,不足減,侵減遲日。晨見差減留日,不足減者,侵減遲日,亦以其差減疾行日率,倍其差,以減度率。前變初日與後變末日先後數,同名相銷,異名相從,為先後定數。各以差率乘之,差數而一,為日差。金星用後變差率、差數。以先後定數減之,為度差。金星夕伏,又日差減先後定數,為度差。晨伏以先後定數加日差,為度差。水星夕伏,以先後定數為日差。倍之,為度差。乃以日度差,積盈者以減、積縮者以加末變日度率。金、水晨伏,反用其差。又倍退行差,差率乘之,差數而一,為日差。以退差減之,為度差。金星夕伏,以日差減退差,為度差。晨伏以退差加日差,為度差。以退行日度差應加者減末變日度率。晨伏反用其差。各加減變訖,為日度定率。



 他亦皆準《大衍歷》法。其分秒不同,則各據本歷母法云。



 起長慶二年,用《宣明歷》。自敬宗至於僖宗,皆遵用之。雖朝廷多故,不暇討論,然《大衍歷》後,法制簡易,合望密近,無能出其右者。訖景福元年。



 《觀象歷》,今有司無傳者。



 《長慶宣明歷》演紀上元甲子,至長慶二年壬寅,積七百七萬一百三十八算外。



 《宣明》統法八千四百。



 章歲三百六萬八千五十五。



 章月二十四萬八千五十七。



 通餘四萬四千五十五。



 章閏九萬一千三百七十一。



 閏限二十四萬四百四十三,秒六。



 中節十五,餘千八百三十五,秒五。



 合策二十九,餘四千四百五十七。



 象準七,餘三千二百一十四少。



 中盈分三千六百七十一,秒二。



 朔虛分三千九百四十三。



 旬周五十萬四千。



 紀法六十。



 秒法八。



 候數五,餘六百一十一,秒七。



 卦位六,餘七百三十四,秒二。



 辰數十二,餘千四百六十八,秒四。



 刻法八十四。



 象數九億二千四十四萬六千一百九十九。



 周天三百六十五度。



 虛分二千一百五十三,秒二百九十九。



 歲差二萬九千六百九十九。



 分統二百五十二萬。



 秒母三百。



 二十四定氣,皆百乘其氣盈縮分,盈減、縮加中節,為定氣所有日及餘、秒。



 六虛之差五十三,秒二百九十九。



 歷周二十三萬一千四百五十八,秒十九。



 歷周日二十七,餘四千六百五十八,秒十九。



 歷中日十三,餘六千五百二十九,秒九半。



 周差日一,餘八千一百九十八,秒八十一。



 秒母一百。



 七日:初數,七千四百六十五;末數,九百三十五。



 十四日:初數,六千五百二十九;末數,千八百七十一。



 上弦:九十一度,餘二千六百三十八,秒百四十九太。



 望:百八十二度,餘五千二百七十六,秒二百九十九半。



 下弦:二百七十三度,餘七千九百一十五,秒百四十九半。



 秒母三百。以刻法約歷分為度,積之為積度。



 中統四千二百。



 辰刻八刻,分二十八。



 昏、明刻各二刻,分四十二。



 刻法八十四。度母同刻法。



 距極度五十六,餘八十二分半。



 北極出地三十四度,餘四十七分半。



 終率二十二萬八千五百八十二,秒六千五百一十二。



 終日二十七,餘千七百八十二,秒六千五百一十二。



 中日十三,餘五千九十一,秒三千二百五十六。



 交朔日二,餘二千六百七十四,秒三千四百八十八。



 交望日十四,餘六千四百二十八,秒五千。



 前準日十二,餘三千七百五十四,秒千五百一十二。



 後準日一,餘千三百三十七,秒千七百四十四。



 陰歷蝕限六千六十。



 陽歷蝕限二千六百四十。



 陰歷定法四百四。



 陽歷定法百七十六。



 交率二百二。



 交數二千五百七十三。



 秒法一萬。



 去交度乘數十一,除數七千三百三。



 ○歲星



 周率三百三十五萬五百四十,秒八十三。



 周策二百九十八,餘七千三百四十,秒八十三。



 中伏日十六,餘七千八百七十,秒四十一半。



 變差九十八,秒三十二。



 交率百八十二,餘五十二,秒二十七。



 變策十五,餘十八,秒三十五。



 差率五。



 差數四。



 ○熒惑



 周率六百五十五萬一千三百九十五,秒二十六。



 周策七百七十九,餘七千七百九十五,秒二十六。



 中伏日七十,餘八千九十七,秒六十二。



 變差三千五,秒一。



 交率百八十二,餘五十二,秒三十二。



 變策十五,餘十八,秒三十六。



 差率三十九。



 差數十。



 ○鎮星



 周率三百一十七萬五千八百七十九,秒七十九。



 周策三百七十八,餘六百七十九,秒七十九。



 中伏日十八,餘四千五百三十九,秒八十九半。



 變差二百七十七,秒九十二。



 交率百八十二,餘五十二,秒二十七。



 變策十五,餘十八,秒三十五。



 差率十。



 差數九。



 ○太白



 周率四百九十萬四千八百四十五,秒八十五。



 周策五百八十三,餘七千六百四十五,秒八十五。



 夕見伏日二百五十六。



 夕見伏行二百四十四度。



 晨見伏日三百二十七,餘七千六百四十五,秒八十五。



 晨見伏行三百四十九,餘七千六百四十五,秒八十五。



 中伏日四十一,餘八千二十二,秒九十二半。



 變差千二百三十六,秒十二。



 交率百八十二,餘五十二,秒二十九。



 變策十五,餘十八,秒三十五。



 夕見差率三十一。



 差數十。



 晨見差率二。



 差數三。



 ○辰星



 周率九十七萬三千三百九十,秒二十五。



 周策百一十五,餘七千三百九十,秒二十五。



 夕見伏日五十二。



 夕見伏行十八度。



 晨見伏日六十三,餘七千三百九十,秒二十五。



 晨見伏行九十七度,餘七千三百九十,秒二十五。



 中伏日十八,餘七千八百九十五,秒十二半。



 變差三千二百一,餘十,秒六十七。



 交率百八十二,餘五十二,秒三十二。



 變策十五,餘十八,秒三十六。



 差率、差數空。秒法百。



 小分法三千六百。



 五星平見加減歷



\end{pinyinscope}