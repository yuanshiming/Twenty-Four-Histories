\article{列傳第一百三十三 宦者下}

\begin{pinyinscope}

 李輔國,本名靜忠,以閹奴為閑廄小兒。貌儜陋,略通書計。事高力士,年四十餘王文成公全書即《陽明全書》。,使主廄中簿最。王鉷為使,以典禾豆,能檢擿耗欺,馬以故肥,薦之皇太子,得侍東宮。



 陳玄禮等誅楊國忠,輔國豫謀,又勸太子分中軍趨朔方,收河、隴兵,圖興復。太子至靈武,愈親近,勸遂即位系天下心。擢家令,判元帥府行軍司馬。肅宗稍稍任以肱膂事,更名護國,又改今名。凡四方章奏、軍符、禁寶一委之。輔國能隨事齪齪謹密,取人主親信,而內深賊未敢肆。不啖葷,時時為浮屠詭行,人以為柔良,不忌也。帝還京師,拜殿中監,閑廄、五坊、宮苑、營田、栽接總監使,兼隴右群牧、京畿鑄錢、長春宮等使,少府、殿中二監,封成國公,實封戶五百。宰相群臣欲不時見天子,皆因輔國以請,乃得可。常止銀臺門決事。置察事聽兒數十人,吏雖有秋豪過,無不得,得輒推訊。州縣獄訟,三司製劾,有所捕逮流降,皆私判臆處,因稱制敕,然未始聞上也。詔書下,輔國署已乃施行,群臣無敢議。出則介士三百人為衛。貴幸至不敢斥官,呼五郎。李揆當國,以子姓事之,號「五父」。帝為娶元擢女為妻,擢以故為梁州長史,弟兄皆位臺省。



 李峴輔政,叩頭言:「且亂國。」於是詔敕不由中書出者,峴必審覆,輔國不悅。



 時太上皇居興慶宮,帝自復道來起居,太上皇亦間至大明宮,或相逢道中。帝命陳玄禮、高力士、王承恩、魏悅、玉真公主常在太上皇左右,梨園弟子日奏聲伎為娛樂。輔國素微賤,雖暴貴,力士等猶不為禮,怨之,欲立奇功自固。初,太上皇每置酒長慶樓,南俯大道,因裴回觀覽,或父老過之,皆拜舞乃去。上元中,劍南奏事吏過樓下,因上謁,太上皇賜之酒,詔公主及如仙媛主之,又召郭英乂、王銑等飲,賚予頗厚。輔國因妄言於帝曰:「太上皇居近市,交通外人,玄禮、力士等將不利陛下,六軍功臣反側不自安,願徙太上皇入禁中。」帝不寤。先時,興慶宮有馬三百,輔國矯詔取之,裁留十馬。太上皇謂力士曰:「吾兒用輔國謀,不得終孝矣。」會帝屬疾,輔國即詐言皇帝請太上皇按行宮中,至睿武門,射生官五百遮道,太上皇驚,幾墜馬,問何為者,輔國以甲騎數十馳奏曰:「陛下以興慶宮湫陋,奉迎乘輿還宮中。」力士厲聲曰:「五十年太平天子,輔國欲何事?」叱使下馬,輔國失轡,罵力士曰:「翁不解事!」斬一從者。力士呼曰:「太上皇問將士各好在否!」將士納刀虖萬歲,皆再拜。力士復曰:「輔國可御太上皇馬!」輔國靴而走,與力士對執轡還西內,居甘露殿,侍衛才數十,皆尪老。太上皇執力士手曰:「微將軍,朕且為兵死鬼。」左右皆流涕。又曰:「興慶,吾王地,數以讓皇帝,帝不受。今之徙,自吾志也。」俄而流承恩播州,魏悅水奏州,如仙媛歸州,公主居玉真觀;更料後宮聲樂百餘,更侍太上皇,備灑掃;詔萬安、咸宜二公主視服膳。自是太上皇怏怏不豫,至棄天下。輔國以功遷兵部尚書。南省視事,使武士戎裝夾道,陳跳丸舞劍,百騎前驅,御府設食,太常備樂,宰相群臣畢會。既得志,乃厭然驕觖,求宰相,帝重違曰:「卿勛力何任不可,但群望未一,如何?」輔國遂諷宰相裴冕使聯表薦己。帝密擿蕭華使喻止冕。



 張皇后數疾其顓,帝寢疾,太子監國,後召太子,將誅輔國及程元振,太子不從,更召越王、兗王圖之。元振告輔國,即伏兵凌霄門,迎太子,伺變,是夜捕二王及中人硃輝光、馬英俊等囚之,而殺後它殿。



 代宗立,輔國等以定策功,愈跋扈,至謂帝曰:「大家弟坐宮中,外事聽老奴處決。」帝矍然欲翦除,而憚其握兵,因尊為尚父,事無大小率關白,群臣出入皆先詣輔國,輔國頗自安。又冊進司空兼中書令,實封戶八百。未幾,以左武衛大將軍彭體盈代為閑廄、嫩牧、苑內、營田、五坊等使,以右武衛大將軍藥子昂代判元帥行軍司馬,賜輔國大第於外。中外聞其失勢,舉相賀。輔國始惘然憂,不知所出,表乞解官。有詔進封博陸郡王,仍為司空、尚父,許朝朔望。輔國欲入中書作謝表,閽者不內,曰:「尚父罷宰相,不可入。」輔國氣塞,久乃曰:「老奴死罪,事郎君不了,請地下事先帝矣!」帝優辭諭遣。



 有韓穎、劉烜善步星,乾元中待詔翰林,穎位司天監,烜起居舍人,與輔國暱甚。輔國領中書,穎進秘書監,烜中書舍人,裴冕引為山陵使判官,輔國罷,俱流嶺南,賜死。



 自輔國徙太上皇,天下疾之,帝在東宮積不平。既嗣位,不欲顯戮,遣俠者夜刺殺之,年五十九,抵其首溷中,殊右臂,告泰陵。然猶秘其事,刻木代首以葬,贈太傅,謚曰醜。後梓州刺史杜濟以武人為牙門將,自言刺輔國者。



 王守澄者,史亡所來。元和中監徐州軍,召還。方憲宗喜方士說,詔天下求其人,宰相皇甫鎛、左金吾將軍李道古等白見楊仁晝、浮屠大通。仁晝更姓名曰柳泌,大通自言壽百五十歲,有不死藥,並待詔翰林。虢人田元佐言有秘方,能化瓦礫為黃金,詔除虢令,與董景珍、李元戢皆介泌、大通薦於天子,天子惑其說。泌以金石進帝餌之,躁甚,數暴怒,恚責左右,踵得罪,禁中累息,帝自是不豫。十五年,罷元會,群臣危恐,會義成劉悟來朝,賜對麟德殿,悟出曰:「上體平矣。」內外乃安。是夜,守澄與內常侍陳弘志弒帝於中和殿,緣所餌,以暴崩告天下,乃與梁守謙、韋元素等定冊立穆宗。俄知樞密事。



 文宗嗣位,守澄有助力,進拜驃騎大將軍。帝疾元和逆罪久不討,故以宋申錫為宰相,謀因事除之,不克,更因其黨鄭注、李訓乘其罅,於是流楊承和於驩州,韋元素象州。遣中人劉忠諒追殺元素於武昌,承和次公安賜死。訓乃脅守澄以軍容使就第,使內養齎■賜死,事秘,時無知者,贈揚州大都督。其弟守涓自徐州監軍召還,死於中牟。



 劉克明,亦亡所來,得幸敬宗。敬宗善擊球,於是陶元皓、靳遂良、趙士則、李公定、石定寬以球工得見便殿,內籍宣徽院或教坊,然皆出神策隸卒或裏閭惡少年,帝與狎息殿中為戲樂。四方聞之,爭以趫勇進於帝。嘗閱角牴三殿,有碎首斷臂,流血廷中,帝歡甚,厚賜之,夜分罷。所親近既皆兇不逞,又小過必責辱,自是怨望。帝夜艾自捕狐貍為樂,謂之「打夜狐」,中人許遂振、李少端、魚志弘侍從不及,皆削秩。帝獵夜還,與克明、田務澄、許文端、石定寬、蘇佐明、王嘉憲、閻惟直等二十有八人群飲,既酣,帝更衣,燭忽滅,克明與佐明、定寬弒帝更衣室,矯詔召翰林學士路隋作詔書,命絳王領軍國事。明日,下遺詔,絳王即位。克明等恃功,將易置左右,自引支黨顓兵柄。於時,樞密使王守澄楊承和、中尉梁守謙魏從簡與宰相裴度共迎江王,發左、右神策及六軍飛龍兵討之,克明投井死,出其尸戮之。務澄等皆斬首以徇,籍入家貲,又殺其黨數十人。



 始,克明謀逆,母禁不許。文宗立,嘉母忠,賜錢千緡、絹五百匹,給婢二人。



 田令孜,字仲則,蜀人也,本陳氏。咸通時,歷小馬坊使。僖宗即位,擢令孜左神策軍中尉,是時西門匡範位右中尉,世號「東軍」、「西軍」。



 帝沖騃,喜鬥鵝走馬,數幸六王宅、興慶池與諸王斗鵝,一鵝至五十錢。與內園小兒尤暱狎,倚寵暴橫。始,帝為王時,與令孜同臥起,至是以其知書能處事,又帝資狂昏,故政事一委之,呼為「父」。而荒酣無檢,發左藏、齊天諸庫金幣,賜伎子歌兒者日巨萬,國用耗盡。令孜語內園小兒尹希復、王士成等,勸帝籍京師兩市蕃旅、華商寶貨舉送內庫,使者監閟櫃坊茶閣,有來訴者皆杖死京兆府。



 令孜知帝不足憚,則販鬻官爵,除拜不待旨,假賜緋紫不以聞。百度崩弛,內外垢玩。既所在盜起,上下相掩匿,帝不及知。是時賢人無在者,惟佞鄙沓貪相與備員,偷安噤默而已。左拾遺侯昌蒙不勝憤,指言豎尹用權亂天下,疏入,賜死內侍省。



 宰相盧攜素事令孜,每建白,必阿邑倡和。初,黃巢求廣州,願罷兵,攜欲寵高駢,使有功,不聽賊。因又易置關東諸節度,賊乘之,陷東都。令孜急,歸罪攜,奉帝西幸,步出金光門,至咸陽沙野,軍十餘騎呼曰:「巢為陛下除奸臣,乘輿今西,秦中父老何望?願還宮。」令孜叱之,以羽林騎馳斬,即以羽林白馬載帝,晝夜馳,舍駱谷。時陳敬瑄方節度西川,令孜兄也,故請帝幸蜀。有詔以令孜為十軍十二衛觀軍容制置左右神策護駕使。至成都,進左金吾衛上將軍,兼判四衛事,封晉國公。帝見蜀狹陋,稍鬱鬱,日與嬪侍博飲,時時攘袂北望,怊然流涕。令孜伺間開釋,呼萬歲,帝為怡悅,因盛稱鄭畋、王鐸、程宗楚、李鋌、敬瑄方並力,賊不足虞。帝曰:「善。」



 初,成都募陳許兵三千,服黃帽,名「黃頭軍」,以捍蠻。帝至,大勞將士,扈從者已賜,而不及黃頭軍,皆竊怨令孜。令孜置酒會諸將,以黃金樽行酒,即賜之。黃頭將郭琪不肯飲,曰:「軍容能易偏惠,均眾士,誠大願也。」令孜目曰:「君有功邪?」答曰:「戰黨項,薄契丹,數十戰,此琪之功。」令孜嘻,怒曰:「知之。」密以■注酒中,琪飲已,馳歸,殺一婢,吮血得解。因夜燒營,剽城邑,敬瑄討敗之,奔廣都,遂走高駢所。帝聞變,與令孜保東城自守,群臣不得見。左拾遺孟昭圖請對,不召,因上疏極陳:「君與臣一體相成,安則同寧,危則共難。昔日西幸,不告南司,故宰相、御史中丞、京兆尹悉碎於賊,唯兩軍中尉以扈乘輿得全。今百官之在者,率冒重險出百死者也。昨昔黃頭亂,火照前殿,陛下惟與令孜閉城自守,不召宰相,不謀群臣,欲入不得,求對不許。且天下者,高祖、太宗之天下,非北司之天下;陛下固九州天子,非北司之天子。北司豈悉忠於南司?廷臣豈無用於敕使?文宗時,宮中災,左右巡使不到,皆被顯責,安有天子播越,而宰相無所豫,群司百官棄若路人?已事誠不足諫,而來者冀可追也。」疏入,令孜匿不奏,矯詔貶昭圖嘉州司戶參軍,使人沈於蟆頤津。初,昭圖知正言必見害,謂家隸曰:「大盜未殄,宦豎離間君臣,吾以諫為官,不可坐觀覆亡,疏入必死,而能收吾骸乎?」隸許諾,卒葬其尸。朝廷痛之。



 賊平,令孜以王鐸為儒臣且無功,而首謀召沙陀者,楊復光也,欲歸重北司,故罷鐸都統,以復光功第一。又忌復光且逼己,故薄其賞。自謂帷幄決勝,系王室輕重,出入倨甚。會復光死,大喜,即罷復恭樞密使。中人曹知愨者,富家子,頗沈鷙。賊在長安,知愨以清、濁二穀之人倚山為屯,不屈賊。陰教士卒變衣服、言語與賊類者,夜入長安攻賊營,賊大懼。帝聞,賜金紫,擢內常侍。聞帝將還,因大言:「我且擁眾大散關下,閱群臣可歸者納之。」令孜謂然,密令王行瑜以邠州兵度嵯峨山,襲殺其眾。由是益自肆,禁制天子不得有所主斷。帝以其專,語左右輒流涕。



 復光部將鹿晏弘、王建等,以八都眾二萬取金、洋等州,進攻興元,節度使牛頊奔龍州,晏弘自為留後,以建及張造、韓建等為部刺史。帝還,懼見討,引兵走許州。王建率義勇四軍迎帝西縣,復以建及韓建等主之,號「隨駕五都。」令孜以復光故,才授諸衛將軍,皆養為子。別募神策新軍,以千人為都,凡五十四都,分左右為十軍統之。又遣親信覘諸鎮,不附己者以罪除徙。



 養子匡祐宣慰河中,王重榮厚為禮,基祐傲甚,舉軍怒,重榮因數令孜罪,責其無禮,監軍和解乃去。匡祐還,訴令孜,且勸圖之。令孜白以兩鹽池歸鹽鐵使,即自兼兩池榷鹽使。重榮不奉詔,表暴令孜十罪。令孜自將討重榮,率邠寧硃玫、鳳翔李昌符,合鄜、延、靈、夏等兵凡三萬,壁沙苑。重榮說太原李克用連和,克用上書請誅令孜、玫,帝和之,不從。大戰沙苑,王師敗。玫走還邠州,與昌符皆恥為令孜用,還與重榮合。神策兵潰還,略所過皆盡。克用逼京師,令孜計窮,乃焚坊市,劫帝夜啟開遠門出奔。自賊破長安,火宮室、舍廬十七,後京兆王徽葺復粗完,至是令孜唱曰:「王重榮反。」命火宮城,唯昭陽、蓬萊三宮僅存。王建以義勇四軍扈帝,夜亂牢水,遂次陳倉。克用還河中,玫畏克用且偪,與重榮連章請誅令孜,而駐鳳翔。令孜請帝幸興元,帝不從,令孜以兵入寢,逼帝夜出,郡臣無知者,宰相蕭遘等皆不及從。玫勸興元節度使石君涉焚閣道,絕帝西意。遘惡令孜劫質天子,生方鎮之難,使玫進迎乘輿。玫引兵追行在,敗興鳳楊晟軍,帝次梁、洋,稍引而南,玫兵及中營,左右被剽戮者不勝計。令孜懼人圖己,蒙面以行。使王建長劍五百清道,囊傳國璽授之。次大散關,道險澀,帝危及難數矣。分軍守靈壁,亢追兵。玫長驅躡帝,帝以閣道毀,走它道,困甚,枕王建膝且寐,覺而飯,僅能至興元。玫、重榮表誅令孜,安尉群臣。詔以令孜為劍南監軍使,留不去。重榮請幸河中,令孜沮而止。宰相遘率群臣在鳳翔者表令孜顓國煽禍,惑小人計,交亂群帥,請誅之。帝不及省,且詔重榮餉糧十五萬斛給行在,重榮以令孜在,不奉命。玫乃奉嗣襄王煴即偽位。玫敗,帝乃得還京師。



 始,帝入蜀,諸王徒步以從,壽王至斜谷不能進,令孜驅使前,王謝足且拘,得馬可濟。令孜怒抶王,強之行,王恥之。及帝病,中外屬壽王,令孜入候帝曰:「陛下記臣否?」帝直視不能語。令孜自署劍南監軍使,閱拱宸奉鑾軍自衛,晝夜馳入成都,固表解官求醫藥,詔可。俄削官爵,長流儋州,然猶依敬瑄不行。



 王即位,是為昭宗。楊復恭代為觀軍容使,出王建為壁州刺史。建取利州,自署防禦使,因略定閬、邛、蜀、黎、雅等州,詔即置永平軍,拜建節度使。令孜謀與建連衡亢朝廷,且曰「吾子也」,書召之。建喜,將至,復卻之。建怒,進圍成都。令孜登城謝建曰:「老夫久相厚,何見困?」答曰:「父子恩,何敢忘!顧父自絕朝廷,茍改圖,則父子如初。」令孜曰:「吾欲面計事。」建然許,令孜夜負印節授建,明日入成都,囚令孜碧雞坊。始,右神策統軍宋文通為諸軍所疾,令孜因事召見,欲殺之。既見,乃欣然更養為子,名彥賓,即李茂貞也,故獨上書雪其罪,詔為湖南監軍。凡二歲,與敬瑄同日死。臨刑,裂帛為絙,授行刑者曰:「吾嘗位十軍容,殺我庸有禮!」因教縊人法,既死,而色不變。乾寧中,詔復官爵。



 楊復恭,字子恪,本林氏子,楊復光從兄也。宦父玄翼,咸通中領樞密,世為權家。復恭略涉學術,監諸鎮兵。龐勛亂,戰有功,自河陽監軍入拜宣徽使,擢樞密使。黃巢盜京師,令孜顓威福,斫喪天下,中外莫敢亢,惟復恭屢與爭得失,令孜怒,下遷飛龍使,復恭乃臥疾藍田。僖宗出居興元,復為樞密使,制置經略,多更其手。車駕還,遂代令孜為左神策中尉、六軍十二衛觀軍容使,封魏國公,實戶八百,賜號「忠貞啟聖定國功臣」。



 帝崩,定冊立昭宗,賜鐵券,加金吾上將軍,稍攘取朝政。帝嘗曰:「朕不德,爾援立我矣,當減省侈長示天下。我見故事,尚衣上御服日一襲,太常新曲日一解,今可禁止。」復恭頓首稱善。帝遂問游幸費,對曰:「聞懿宗以來,每行幸無慮用錢十萬,金帛五車,十部樂工五百,犢車、紅網硃網畫香車百乘,諸衛士三千。凡曲江、溫湯若畋獵曰大行從,宮中、苑中曰小行從。」帝乃詔類減半。



 於是宰相韋昭度、張浚、杜讓能等為帝言大中故事,抑宦官不假借,帝亦稍厭復恭橫恣。王瑰者,恭憲太后弟,求節度使,帝問復恭,對曰:「產、祿頃漢,三思危唐,後族不可封拜。陛下誠愛瑰,任以它職可也,不宜假節外籓,恐負勢顓地不可制。」帝乃止。瑰聞,怒甚,至禁中見復恭詬辱之,遂居中任事。復恭不欲分己權,白為黔南節度使,道興元,而兄子守亮方領節度,陰勒利州刺史覆瑰舟於江,宗屬賓客皆死,以舟自敗聞。帝知復恭謀,繇是深銜之。



 復恭以諸子為州刺史,號「外宅郎君」;又養子六百人,監諸道軍。天下威勢,舉歸其門。守立為天威軍使,本胡弘立也,勇武冠軍,人畏之。帝欲斥復恭,懼為亂,乃好謂曰:「卿家胡子安在?吾欲令衛殿內。」復恭以守立見帝,賜姓李,名順節,使掌六軍管鑰,光寵甚。既勢鈞,遂與復恭爭恨相中傷,暴發其私。



 復恭常肩輿抵太極殿。宰相對延英,論叛臣事,孔緯曰:「陛下左右有將反者。」帝矍然。緯指復恭。復恭曰:「臣豈負陛下者?」緯曰:「復恭,陛下家奴,而肩輿至前殿。廣樹不逞皆姓楊,非反邪?」復恭曰:「欲收士心輔天子。」帝曰:「誠欲收士心,胡不假李姓乎?」復恭無以對。會緯出守江陵,乃使人劫之長樂坡,斬其旌節,貲貯皆盡,緯僅免。



 復恭子守貞為龍劍節度使,守忠洋州節度使,皆自擅貢賦,上書訕薄朝政。大順二年,罷復恭兵,出為鳳翔監軍,不肯行,因丐致仕,詔可,遷上將軍,賜幾杖。使者還,遣腹心殺使者於道,遁居商山。俄入居昭化坊第,第近玉山營,而子守信為軍使,數省候出入。或告父子且謀亂,時順節遙領鎮海軍節度使、同中書門下平章事,詔與神策軍使李守節率衛兵攻復恭,治殺使者罪,帝御延喜樓須之。家人拒戰,守信亦率兵至昌化里,陣以待。會日入,復恭與守信舉族出奔,遂走興元。



 順節已斥復恭,則橫暴,出入以兵從,兩軍中尉劉景宣、西門重遂察其意非常,以狀聞。有詔召順節,輒以甲士三百入,至銀臺門,何止之,景宣引順節坐殿廡,部將嗣光審出斬之,從者大噪,出延喜門,剽永寧里,盡夕止。賈德晟與順節皆為天威軍使,順節誅,頗嗟憤,重遂亦奏誅之。於是鳳翔李茂貞、邠州王行瑜、華州韓建、同州王行約、秦州李茂莊同劾守亮納叛臣,請出兵討罪,軍餉不仰度支。茂貞請假山南招討使。宦尹惜類執不可,帝亦謂茂貞得山南必難制,詔兩解之。茂貞劾復恭自謂隋諸孫,以恭帝禪唐,故名復恭,逆狀明白,且請削守亮官爵。遂擅與行瑜出討,自號興元節度使,詒宰相書,慢悖不臣。帝為下詔,令茂貞、行瑜討之。景福元年,破其城,復恭、守亮、守信奔閬州,茂貞以子繼密守興元。詔吏部尚書徐彥若為鳳翔節度使,而以茂貞帥興元,不拜,請繼密為留後。帝不得已,授以節度使,自是茂貞始強大。



 復恭與守亮等自閬州將北奔太原,趨商山,至乾元,為韓建邏士所禽,即斬復恭、守信,檻車送守亮京師,梟首長安市。茂貞上復恭與守亮書曰:「承天門者,隋家舊業也,兒但積粟訓兵,何進奉為?吾披荊榛立天子,既得位,乃廢定策國老,奈負心門生何!」門生,謂天子也,其不臣類此。假子彥博奔太原收葬其尸,李克用為申雪,詔復官爵。



 劉季述者,本微單,稍顯於僖、昭間,擢累樞密使。楊復恭之斥,帝以西門重遂為右神策軍中尉、觀軍容使。時李茂貞得興元,愈跋扈不軌,宰相杜讓能與內樞密使李周言童及重遂謀誅之,乃興師,以嗣覃王戒丕為京西招討使,神策大將軍李金歲副之。茂貞引兵迎壁盩厔,薄興平,王師潰。遂逼臨皋以陣,暴言讓能等罪,京師震恐,帝坐安福門,斬重遂、周言童以謝茂貞,更以駱全瓘、劉景宣代為兩中尉。乾寧二年,茂貞與王行瑜、韓建以兵入朝,李克用率師討茂貞,次渭北。同州節度使王行實奔京師,謂景宣等曰:「沙陀十萬至矣,請奉天子出幸避其鋒。」景宣方與茂貞睦,故全瓘與鳳翔衛將閻圭共脅帝狩岐,王行實及景宣子繼晟縱火剽東市,帝登承天門,矢著樓闔。帝懼,暮出莎城,士民從者數十萬。至谷口,人曷死十三,夜為盜掠,哭聲殷山。徙駐石門。茂貞恐,乃殺全瓘、景宣及圭自解。天子還京師,以景務脩、宋道弼代之,俄專國。宰相崔胤惡之,徐彥若、王摶懼禍不解,稍抑胤以和北軍。胤怒,劾摶黨宦豎,不忠,罷去,俄賜死;流道弼驩州,務脩愛州,並死灞橋;逐彥若於南海。乃以季述、王仲先為左右中尉,疾胤尤甚。



 時帝嗜酒,怒責左右不常,季述等愈自危。先是,王子病,季述引內醫工車讓、謝筠,久不出,季述等共白帝,宮中不可妄處人。帝不納,詔著籍不禁。由是疑帝與有謀,乃外約硃全忠為兄弟,遣從子希正與汴邸官程巖謀廢帝。會全忠遣天平節度副使李振上計京師,巖因曰:「主上嚴急,內外惴恐,左軍中尉欲廢昏立明,若何?」振曰:「百歲奴事三歲郎主,常也。亂國不義,廢君不祥,非吾敢聞。」希正大沮。帝夜獵苑中,醉殺侍女三人,明日午漏上,門不啟。季述見胤曰:「宮中殆不測。」與仲先率王彥範、薛齊偓、李師虔、徐彥回總衛士千人毀關入,謀所立,未決。是夜,宮監竊取太子以入,季述等因矯皇后令曰:「車讓、謝筠勸上殺人,禳塞災咎,皆大不道。兩軍軍容知之,今立皇太子,以主社稷。」黎明,陳兵廷中,謂宰相曰:「上所為如此,非社稷主,今當以太子見群臣。」即召百官署奏,胤不得對。季述衛皇太子至紫廷院,左右軍及十道邸官俞潭、程巖等詣思玄門請對,士皆呼萬歲。入思政殿,遇者輒殺。帝方坐乞巧樓,見兵入,驚墮於床,將走,季述、仲先持帝坐,以所持釦杖畫地責帝曰:「某日某事爾不從我,罪一也。」至數十未止。皇后出,遍拜曰:「護宅家,勿使怖,若有罪,惟軍容議。」季述出百官奏,曰:「陛下瞀,倦於勤,願奉太子監國,陛下自頤東宮。」帝曰:「昨與而等飲甚樂,何至是?」後曰:「陛下如軍容語。」宮監掖帝出思政殿,後倡言曰:「軍容一心輔持,請上養疾。」帝亦曰:「朕久疾,令太子監國。」巖等皆呼萬歲。後以傳國寶授季述,就帝輦,左右十餘人,入囚少陽院。季述液金以完鐍,師虔以兵守。太子即位於武德殿,帝號太上皇,皇后為太上皇后,大赦天下,東宮官屬三品賜爵一級,四品以下一階,天下為父後者爵一級,群臣加爵秩厚賜,欲媚附上下。改東宮為問安宮。季述等皆先誅戮以立威,夜鞭笞,晝出尸十輦,凡有寵於帝,悉榜殺之。殺帝弟睦王。師虔尤苛察,左右出入搜索,天子動靜輒白季述。帝衣晝服夜浣,食自竇進,下至筆紙銅鐵,疑作詔書兵器,皆不與。方寒,公主嬪御無衾纊,哀聞外廷。



 胤告難於硃全忠,使以兵除君側,全忠封胤書與季述曰:「彼翻覆,宜圖之。」季述以責胤,胤曰:「奸人偽書,從古有之,必以為罪,請誅不及族。」季述易之,乃與盟。胤謝全忠曰:「左軍與胤盟,不相害,然僕歸心於公,並送二侍兒。」全忠得書,恚曰:「季述使我為兩面人。」自是始離。季述子希度至汴,言廢立本計,又遣李奉本齎示太上皇誥,全忠狐疑不決。李振入見曰:「豎刁、伊戾之亂,以資霸者。今閹奴幽劫天子,公不討,無以令諸侯。」乃囚希度、奉本,遣振至京師與胤謀。是時季述欲盡誅百官,乃弒帝,挾太子令天下。都將孫德昭、董從實盜沒錢五千緡,仲先眾辱之,督其償,株連甚眾。胤間其不逞,曰:「能殺兩中尉,迎太上皇,而立大功,何小罪足羞!」又遣客密告德昭,割帶內蜜丸通意。德昭邀別將周承誨,期十二月晦,伏士安福門待旦。仲先乘肩輿造朝,德昭等劫之,斬東宮門外,叩少陽院呼曰:「逆賊斬矣。」帝疑未信,皇后曰:「可獻賊首。」德昭擲仲先頭以進,宮人毀扉,出禦長樂門,群臣稱賀。承誨馳入左軍,執季述、彥範至樓前,胤先戒京兆尹鄭元規集萬人持大梃,帝詰季述未已,萬梃皆進,二人同死梃下,遂尸之。兩軍支黨死者數十人。中官奉太子遁入左軍,收傳國璽。齊偓死井中,出其尸斬之。全忠檻送巖京師,斬於市。季述等夷三族。以德昭檢校太保、靜海軍節度使,從實檢校司徒、容管節度使,並同中書門下平章事,賜氏李,曰繼昭,曰彥弼。承誨亦檢校司徒、邕管節度使,視宰相秩。皆號「扶傾濟難忠烈功臣」,圖形凌煙閣,留宿衛凡十日乃休,竭內庫珍寶賜之。當時號「三使相」,人臣無比。



 初,延英宰相奏事,帝平可否,樞密使立侍,得與聞,及出,或矯上旨謂未然,數改易橈權。至是,詔如大中故事,對延英,兩中尉先降,樞密使候旨殿西,宰相奏事已畢,案前受事。師虔請於屏風後錄宰相所奏,帝以侵官,不許,下詔與徐彥回同誅。



 韓全誨、張彥弘者,皆不知所來,並監鳳翔軍。全誨入為內樞密使。劉季述之誅,崔胤、陸扆見武德殿右廡,胤曰:「自中人典兵,王室愈亂,臣請主神策左軍,以扆主右,則四方籓臣不敢謀。」昭宗意不決。李茂貞語人曰:「崔胤奪軍權未及手,志滅籓鎮矣。」帝聞,召李繼昭等問以胤所請奈何,對曰:「臣世世在軍,不聞書生主衛兵。且罪人已得,持軍還北司便。」帝謂胤曰:「議者不同,勿庸主軍。」乃以全誨為左神策中尉,彥弘為右,皆拜驃騎大將軍,袁易簡、周敬容為樞密使。胤怒,約京兆鄭元規遣人狙殺之,不克。全誨等知胤必除己乃已,因諷茂貞留選士四千宿衛,以李繼筠、繼徽總之。胤亦諷硃全忠內兵三千居南司,以婁敬思領之。韓偓聞岐、汴交戍,數諫止胤,胤曰:「兵不肯去耳。」偓曰:「初何為召邪?」胤不對。議者知京師不復安矣。



 全誨、彥弘及彥弼合勢恣暴,中官倚以自驕,帝不平,有斥逐者,皆不肯行,胤固請盡誅之。全誨、彥弘見帝祈哀,帝知左右漏言,始詔囊封奏事。宦人更求麗姝知書者數十人,侍帝為內言冋,由是胤計多露。



 始,張浚判度支,楊復恭以軍貲乏,奏假鹽曲一歲入以濟用度,遂不復還。至胤,乃白度支財盡,無以稟百官,請如舊制。全誨擿李繼筠訴軍中匱甚,請割三司隸神策。帝不能卻,詔罷胤領鹽鐵,胤銜之。



 全誨等懼帝誅己,與繼誨、彥弼、繼筠交通謀亂。帝問令狐渙,渙請召胤及全誨等宴內殿和解之。韓偓謂:「不如顯斥一二柄臣,許餘人自新,妄謀必息。不然皆自疑,禍且速,雖和解之,兇焰益肆。」帝乃止。是時全忠並河中,胤為急詔令入朝,又詒書曰:「上反正,公之力,而鳳翔入朝,引功自歸。今若後至,必先見討。」全忠得詔,還汴,悉師討全誨。帝以為忠,又欲其與茂貞同功,即詔並力。令胤詒二鎮書,示帝意。全忠取同州,汴兵凡七萬,威震關中。全誨等泣奏曰:「全忠且至,欲脅陛下幸關東,將謀傳禪。臣不忍見高祖天下移他姓,願至鳳翔,合義兵討元惡。」帝未許,方在乞巧樓,全誨急,即火其下,帝降樓,乃決西幸。彥弼等以帝未即駕,愈誖,宮中禁索苛亟,帝與後相視泣,宮人私逃出都,民崩沸,或奔開化坊依胤第自固,閈無留家。鳳翔軍與左神策兵陣大衢,長樂門外若丘墟然。於是日南至,百官不朝,帝坐思政殿。時彥弼先入鳳翔,全誨逼帝出,惟皇后、諸王數百騎為衛,帝繡袍、塗金帽,以右神策軍從,實天復元年十一月壬子。全誨等遂火宮城,繼誨、彥弼欲劫百官從天子,李德昭等按兵衛之,乃得免。茂貞以帝居盩厔。



 全忠取華州,下令自釋曰:「吾被詔及得宰相書令入朝,既至,皆偽也。逆臣全誨震驚天子,脅乘輿出遷,暴露草莽,吾當入對言狀。」時公卿皆在長安,數日不聞朝廷敕畫。胤使王溥見全忠曰:「上猶在盩厔,公宜亟進。」群臣盧知猷等奏記全忠,請西迎天子,答曰:「進則似脅君,退則負國,然敢不勉?」胤率百官迎全忠灞橋,入舍長安一昔而西。



 茂貞聞全忠至,以帝入鳳翔,從臣才三四人。全忠遣楊達、裴鑄入鳳翔,奉表天子。汴部將康懷英襲破李繼昭於武功,禽馘六千級。全誨懼,請救於李克用。克用遺全忠書,勸執崔胤,洗海內謗,全忠不答,進屯鳳翔東偏。茂貞登城隃語曰:「天子厭災於此,讒人誤公來,公當入覲。」全忠曰:「宦官脅驚乘輿,吾以兵問罪,迎上東還。王非同謀者,尚何所言?」明日,圍鳳翔,茂貞不出。帝遣中人詔全忠班師,不奉詔。使者再往,全忠聽命,引兵攻邠州,李繼徽嬰城三日乃降。質其妻,復使繼徽守,回壁三原。胤與鄭元規至三原,邀說全忠。全忠亦自聞茂貞將戰,徙營渭北,據高原,戰不勝。全忠夜入盩厔,拔藍田,復屯三原。



 時李克用攻慈、隰,救鳳翔,全忠還河中。克用部將李嗣昭戰數不利,全忠取晉、汾二州,嗣昭遁還河東。全忠曰:「此茂貞所倚,今敗矣,何能久乎?」胤復說全忠曰:「宦豎謀擁帝入蜀。」且泣。全忠執其手,乃定計迎天子。會硃友寧敗岐兵於莫父,居人皆入保。全忠以精甲五萬與茂貞決戰,岐兵敗,僕尸萬餘,茂貞帳下八百人就縛,乃嬰城,自夏訖冬,兵連不能解,勝敗略相償。援軍十餘壁,數為全忠擾襲,不得進,城中日困。全忠由是取鳳、鄜、坊、成、隴等州,間劫鈔以佐軍餉,故能不乏。茂貞疑帝與全忠有密約,增甲士守宮殿。



 初,帝至鳳翔,有鴉數萬棲殿樹,謂之神鴉。俄而鴉不來,人以為恐。全誨等小人既勢窘,更相怨疾,不復遠慮。時財用窶短,帝輟所御膳賜全誨等,三讓,帝曰:「難得時欲同味耳。」茂貞食鮓美,帝曰:「此後池魚。」茂貞曰:「臣養魚以候天子。」聞者皆駭。



 於是全忠軍攻東城,焚橋鏖戰,部將李繼寵出降,茂貞懼,密圖誅中官以紓難。先遺書曰:「禍亂之生,全誨首之。變興倉卒,故迎天子至此。且公未至,懼它盜馮陵。公既志輔社稷,請奉乘輿還宮,僕願以敝賦從。」全忠然許,然軍稍薄城,大言虖者三,岐軍皆投塹,無鬥意。帝召茂貞、全誨、彥弼及宰相蘇檢、李繼岌、繼忠議,和已決,中官復沮罷。它日,帝召茂貞等曰:「十六宅諸王日奏餒死者十三,王、公主、夫人皆間日食,今又將竭,奈何?」皆不敢對。有衛士十餘人叩左銀臺門,遮全誨罵曰:「破一州,餓死者十萬,徒以軍容數人耳!」全誨詣茂貞叩頭訴,茂貞謝曰:「士伍亦何知?」復訴於帝,帝不許。李繼昭見全誨曰:「昔楊軍容破楊守亮一族,今驃騎復破吾族乎?」罵之,乃出降。宦豎數傳援軍至,皆相賀,百姓笑曰:「紿我乎!」



 是時,全忠合四鎮兵十餘萬,營壘相屬,晝夜攻。外兵詬守者曰:「劫天子賊」,守者亦詬外曰「奪天子賊」。諸鎮見崔胤檄,皆狐疑不出師,唯青州節度使王師範取兗州,襲華州,李克用攻晉州以為援。全忠懼,圍益急。全誨等素譎險,常為全忠、胤所憚,乃請先殺之,以迎天子。帝既惡宦人脅遷,而茂貞又其黨,全忠雖外示順,終悖逆,皆不可倚。欲狩襄、漢,依趙匡凝,然不得去,乃定計歸全忠,以紓近禍。



 三年正月,茂貞請遣使諭全忠軍,詔崔構挾中人郭遵誨往,既行,又命宮人寵顏馳見全忠,諭密旨,乃以蔣玄暉入衛。二日,茂貞獨見,至日旰,全誨、彥弘恨甚,逮食,不能捉匕,自見勢去,計無所用,垂頭喪氣。帝召韓偓見東橫門,執手涕泗。帝曰:「今先去四大惡,餘以次誅矣。」於是內養八輩候廷中授命,每二輩以衛士十人取一首,俄而全誨、彥弘、易簡、敬容皆死。即詔第五可範為左軍都尉,王知古、揚虔朗為樞密使,知古領上院,虔朗領下院。繼筠、繼誨、彥弼皆伏誅,茂貞取其輜重。是夜,誅內諸司使韋處廷等二十二人,悉以首內布囊,詔蔣玄暉、學士薛貽矩送全忠,曰:「是皆不肯使乘輿東者,既斬之矣。」全忠大喜,遍告軍中,以姚洎為岐、汴通和使。全忠詒茂貞書曰:「宦者乘陴詈不已,曰『稟王旨』,是乎?」茂貞懼,復誅小使李繼彞等十人,於是開壘門。全忠猶攻北壘,帝遣寵顏賜御巾箱寶器,使罷兵,又捕殺中官七十人,全忠亦使京兆誅黨與百餘人。



 天子入全忠軍,全忠泥首素服,待罪客省,傳呼徹三仗,有詔釋全忠罪,使朝服見。全忠伏地泣曰:「老臣位將相,勤王無狀,使陛下及此,臣之罪也。」帝亦嗚咽,命韓偓起之,解玉帶以賜,召之食。帝顧衛兵,或有憤發者,因履系解,目全忠:「為吾系之。」全忠跪結履,汗浹於背,而左右莫敢動。是夜,帝三召,皆辭,硃友倫以兵衛帝。



 李克用引軍去,帝還京師。胤、全忠議,盡誅第五可範等八百餘人於內侍省,哀號之聲聞於路,留單弱數十人,備宮中灑掃。胤以鎮人性謹厚,即詔王鎔擇五十人為敕使,內諸司宦官主領者皆罷。於是追諸道監軍,所在賜死,其財產籍入之。詔以中官脅遷狀及全忠迎乘輿本末告方鎮,罷監軍院,咸視國初故事,以三十人為員,衣黃衣,不得養子。內諸司皆歸省若寺,兩軍內外八鎮兵悉屬六軍。全忠還汴州,帝以第五可範等無辜,頗悼之,為文以祭。自是宣傳詔命,皆以宮人。



 始,劉季述專廢立,中人皆與聞。帝反正,誅季述及薛齊偓數族而已,餘貸不問;又悔之,後稍稍誅夷,群宦浸不安。時帝懲幽辱,能勵心庶政,數召見群臣問治道,有志中興,而全誨、胤爭權,外召強臣,劫本朝以相吞嚙,卒用關東軍窮討暴誅,君側雖清,而全忠勢遂張,帝卒弒死,唐室以亡,其禍本於全誨、彥弘云。



 贊曰:袁紹誅常侍以逞,而曹操移漢;崔丞相血軍容甘心焉,而硃溫篡唐。大抵假威柄於外,以內攘奸人,則大臣專,王室卑矣。漢、唐相去五百歲,產亂取亡猶蹈一轍,非天所廢,而人謀洄刺乃然邪!



\end{pinyinscope}