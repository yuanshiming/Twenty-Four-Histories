\article{列傳第七十三 令狐張康李劉田王牛史}

\begin{pinyinscope}

 令狐彰,字伯陽,京兆富平人,其先自燉煌內徙。父濞,為世善吏。始心術《管子》篇名。分為上下篇。戰國時稷下學士著。一,尉範陽,通民家女,生彰。罷歸,留彰母所。既長,志膽沈果,知書傳大義,射命中。從安祿山,署左衛郎將。與張通儒入長安,又署左街使。二京平,走河朔。史思明署博、滑二州刺史,屯滑臺。時中人楊萬定監滑州軍,彰欲以節自顯,募沒人夜度河,悉籍士馬州縣獻款,因萬定以聞。肅宗大悅,下書慰勞。彰移壁杏園渡,思明疑之,遣薛岌以兵劫彰。彰諭眾以大誼,皆感附死力,遂破岌兵,潰圍出,以麾下數百入朝,賜甲第、帷帳、什器,拜滑亳、魏博節度使。河朔平,加兼御史大夫,封霍國公,檢校尚書右僕射。



 始,滑當寇沖,城邑墟榛,彰躬訓吏下,檢軍力農,法令嚴,無敢犯者。田疇大闢,庫委豐餘,歲時貢賦如期。時吐蕃盜邊,召防秋兵,彰遣士三千,自齎糧,所過無秋毫犯,供儗讓不受,時韙其能。然猜阻忮忍,忤者輒死。怒潁州刺史李岵,遣姚奭代之,戒曰:「不時代,殺之。」岵知其謀,因殺奭,死者百餘人,奔汴州,上書自言,彰亦劾之。河南尹張延賞畏彰,留岵使,故彰書先聞,斥岵夷州,殺之。與魚朝恩有隙,及用事,彰不敢入朝。



 會母喪,失明,卒。方疾甚,敕子建、通、運歸東都私第,悉上軍府兵仗財用簿最,表吏部尚書劉晏、工部尚書李勉堪大事,請以自代。代宗得表咨悼,下詔褒美其門閭,贈太傅。



 建累官右龍武軍使。德宗幸奉天,建方肄士射,遂以四百人從,且殿。擢行在中軍鼓角使、左神武軍大將軍。其妻,成德節度使李寶臣女也,建將棄之,誣與門下客郭士倫通,榜殺士倫而逐其妻,士倫母痛憤卒。寶臣請劾按,無狀。建會赦免。帝取常膳錢五十萬葬士倫母子,並恤其家。俄起建為右領軍大將軍。復坐專殺,以勛被貸。坐妄自陳,貶施州別駕,卒,贈右領軍大將軍,又加贈揚州大都督。



 憲宗時,宰相李吉甫奏言:「彰將死,籍上土地兵甲,遣諸子還第,彰同時河朔諸鎮,傳子孫,熏灼數代,唯彰忠義奮發,而長子建坐事,幼子運無辜,皆竄死,今通幸存,惟陛下用之。」因授贊善大夫。時討蔡,故連徙壽州團練使。聞吉甫卒,不自安。每戰,虛張首級,敗則掩不奏。露布上,宰相武元衡卻之。後為賊攻,焚廥聚,破屯柵,通大懼,重塹不敢出。詔金吾大將軍李文通宣慰,將至,遂代之。貶昭州司戶參軍事。久乃召為右衛將軍,給事中崔植還其制,帝使喻植,以彰有功,不忍棄其嗣,制乃下。終左衛大將軍。



 運為東都留守將,為杜亞所陷,流死歸州。



 張孝忠,字孝忠,本奚種,世為乙失活酋長。父謐,開元中提眾納款,授鴻臚卿。孝忠始名阿勞,以勇聞,燕、趙間共推張阿勞、王沒諾干,二人齊名。沒諾干,王武俊也。孝忠魁偉,長六尺,性寬裕,事親孝。天寶末,以善射供奉仗內。安祿山奏為偏將,破九姓突厥,以功擢漳源府折沖。祿山、史思明陷河、洛,常為賊前鋒。朝義敗,乃自歸,授左領軍將軍,以兵屬李寶臣。累加左金吾衛將軍,賜今名。寶臣以其沈毅謹詳,遂為姻家,易州諸屯委以統制,十餘年,威惠流聞。田承嗣寇冀州,寶臣付兵四千,使出上谷,屯貝丘。承嗣見其軍整嚴,嘆曰:「阿勞在焉,冀未可圖也。」即焚營去。寶臣與硃滔戰瓦橋,奏孝忠為易州刺史,分精騎七千,當幽州。擢太子賓客,封符陽郡王。



 寶臣晚節稍忌刻,殺大將李獻誠等而召孝忠,孝忠不往,復使其弟孝節召之。孝忠復命曰:「諸將無狀,連頸受戮。吾懼禍,不敢往,亦不敢叛,猶公不覲天子也。」孝節泣曰:「即歸,且僇死。」孝忠曰:「偕往則並命,吾留,無患也。」果不敢殺。



 然寶臣素善孝忠,及病不能語,以手指北而死。子惟岳擅立,詔硃滔以幽州兵討之。滔忌孝忠善戰,慮師出為己患,使判官蔡雄往說曰:「惟岳孺子,弄兵拒命,吾奉詔伐罪,公乃宿將,安用助逆而不自求福也?今昭義、河東軍已破田悅,而淮西軍下襄陽,梁崇義尸出井中,斬漢江上者五千人,河南軍計日北首,趙、魏滅亡可見。公誠去逆蹈順,倡先歸國,可以建不世功。」孝忠然之,遣將程華報滔連和,遣易州錄事參軍事董稹入朝。德宗嘉之,擢孝忠檢校工部尚書、成德軍節度使,令與滔並力。孝忠子弟在恆州者皆死。孝忠重德滔,為子茂和聘其女,締約益堅。



 敗惟岳於束鹿,滔欲乘勝襲恆州,孝忠乃引軍西北,壁義豐。滔疑之,孝忠將佐諫曰:「尚書推赤心於硃司徒,可謂至矣。今逆賊已潰,元功不終,後且悔之。」孝忠曰:「本求破賊,賊已破矣,而恆州多宿將,迫之則死鬥,緩之則改圖。且滔言大而識淺,可以慮始,難與守成。故吾堅壁於此,以待賊之滅耳。」滔亦止屯束鹿。月餘,王武俊果斬惟岳以獻。已而定州刺史楊政義以州降孝忠,遂有易、定。時三分成德地,詔定州置軍,名義武,以孝忠為節度、易定滄等州觀察使。



 後滔與武俊叛,復遣蔡雄說之,答曰:「吾既為唐臣,而天性樸強,業已效忠,不復助惡矣。吾與武俊少相狎,然其心喜反覆,不可信。幸謝司徒,志鄙言。」滔復啖以金帛,皆不受。易、定介二鎮間,乃浚溝壘,脩器械,感厲將士,乘城固守。滔悉兵攻之,帝詔李晟、竇文場率師援孝忠,滔解去,遂全其軍。孝忠因與晟結婚。天子出奉天,孝忠遣將楊榮國以銳卒六百佐晟赴難,收京師。興元初,詔同中書門下平章事。



 貞元二年,河北蝗,民餓死如積,孝忠與其下同粗淡,日膳裁豆而已,人服其儉,推為賢將。明年,檢校司空。詔其子茂宗尚義章公主,孝忠遣妻入朝,執親迎禮,賞賚甚厚。五年,為將佐所惑,以兵襲尉州,入之,奉詔還鎮。有司劾擅興,削司空。六年,還其官。卒,年六十二,追封上谷郡王,贈太師,謚曰貞武。子茂昭、茂宗、茂和。



 茂宗,擢累光祿少卿、左衛將軍。元和中,歷閑廄使。初,至德時,西戎陷隴右,故隴右監及七廄皆廢,而閑廄私其地入,寶應初,始以其地給貧民。茂宗恃恩,奏悉收其賦,又奏取麟游岐陽牧地三百餘頃,民訴諸朝,詔監察御史孫革按行,還奏不可。茂宗負左右助,誣革所奏不實,復遣侍御史範傳式覆實,乃悉奪其田。長慶初,岐人列訴,下御史,盡以其地還民。寶歷初,遷兗海節度使。終左龍武統軍。



 茂和,歷左武衛將軍。裴度討蔡,奏為都押衙。茂和數以膽勇求自試,謂度無功,辭不行。度請斬之以令軍,憲宗曰:「予以其家忠且孝,為卿遠斥。」後終諸衛將軍。



 茂昭本名升雲,德宗時賜今名,字豐明。少沈毅,頗通書傳。孝忠時,累擢檢校工部尚書。孝忠卒,帝拜邕王諒為義武軍節度大使,以茂昭為留後,封延德郡王。後二年,為節度使。弟昇璘薄王武俊為人,座上嫚罵,武俊怒,襲義豐、安喜、無極,掠萬餘人,茂昭嬰城,遣人厚謝,乃止。久之,入朝,為帝從容言河朔事,帝竦聽,曰:「恨見卿晚!」召宴麟德殿,賜良馬、甲第、器幣優具,詔其子克禮尚晉康郡主。帝方倚之經置北方,會崩,故茂昭每入臨,輒哀不自勝。



 順宗立,進同中書門下平章事,復遣之鎮,賜女樂二人,固辭,車至第門,茂昭引詔使辭曰:「天子女樂,非臣下所宜見。昔汾陽、咸寧、西平、北平皆有大功,故當是賜。今下臣述職以朝,奈何濫賞?後日有立功之臣,陛下何以加之?」復賜安仁里第,亦讓不受。憲宗元和二年,請朝,五奏乃聽。願留,不許,加兼太子太保。



 既還,王承宗叛,詔河東、河中、振武、義武合軍為恆州北道招討,茂昭治廩廄,列亭候,平易道路,以待西軍。承宗以騎二萬逾木刀溝與王師薄戰,茂昭躬擐甲為前鋒,令其子克讓、從子克儉與諸軍分左右翼繞賊,大敗之,承宗幾危。會有詔班師,加檢校太尉,兼太子太傅。乃請舉宗還朝,表數上,帝乃許。北鎮遣客間說,皆不納。詔左庶子任迪簡為行軍司馬,乘驛往代。茂昭奉兩州符節、管鑰、圖籍歸之。先敕妻子上道,戒曰:「吾使而曹出易,庶後世不為污俗所染。」未半道,迎拜兼中書令,充河中晉絳慈隰節度使。至京師,雙日開延英,對五刻罷。又表遷墳墓於京兆,許之。明年,疽發於首卒,年五十,冊贈太師,謚曰獻武。帝思其忠,擢諸子皆要職,歲給絹二千匹。



 少子克勤,開成中歷左武衛大將軍。有詔賜一子五品官,克勤以息幼,推與其甥,吏部員外郎裴夷直劾曰:「在勤骫有司法,引庇它族,開後日賣爵之端,不可許。」詔聽,遂著於令。



 夷直字禮卿,亦婞亮,第進士,歷右拾遺,累進中書舍人。武宗立,夷直視冊牒,不肯署,乃出為杭州刺史,斥驩州司戶參軍。宣宗初內徙,復拜江、華等州刺史。終散騎常侍。



 陳楚者,茂昭甥也,字材卿,定州人。有武干,事茂昭,歷牙將,常統精卒從征伐。茂昭入朝,擢諸衛大將軍,封普寧郡王。元和末,義武節度使渾鎬喪師,定州亂,拜楚為節度使,馳傳赴軍。及郊,無迎者,左右勸無入,楚曰:「定軍不來迎以試我。今不入,正墮計中。」乃冒雪行四十里,夜入其州,然軍樣部伍,皆楚舊也,由是眾心乃定。徙河陽三城,入為左羽林統軍,檢校司空。卒,年六十一,贈司空。子君奕,亦至鳳翔節度使。



 康日知,靈州人。祖植,當開元時,縛康待賓,平六胡州,玄宗召見,擢左武衛大將軍,封天山縣男。日知少事李惟岳,擢累趙州刺史。惟岳叛,日知與別駕李濯及部將百人啐牲血共盟,固州自歸。惟岳怒,遣先鋒兵馬使王武俊攻之,日知使客謝武俊曰:「賊孱甚,安足共安危哉?吾城固士和,雖引歲未可下,且賊所恃者田悅耳,悅兵血衊邢,壕可浮,不能殘半堞,況吾城之完乎?」又紿為臺檢示曰:「使者齎詔喻中丞,中丞奈何負天子,從小兒跳梁哉?」武俊悟,引兵還,斬惟岳以獻。德宗美其謀,擢為深趙觀察使,賜實封戶二百。



 會武俊拒命,遣將張鐘葵攻趙州,日知破之,上俘京師。興元元年,以深趙益成德,徙日知奉誠軍節度使,又徙晉絳,加累檢校尚書左僕射,封會稽郡王。貞元初卒,贈太子太師。



 子志睦,字得眾。資趫偉,工馳射。隸右神策軍,遷累大將軍。討張韶,以多兼御史大夫,進平盧軍節度使。李同捷反,放兵略千乘,志睦挫其銳,不得逞,遂下蒲臺,盡奪其械。加檢校尚書左僕射。徙涇原,封會稽郡公。卒,年五十七,贈司空。



 子承訓,字敬辭。推門功進累左神武軍將軍。宣宗擢為天德軍防禦使,軍中馬乏,虜來戰,數負,承訓罷冗費,市馬益軍,軍乃奮張。始,黨項破射雕軍洛源鎮,悉俘其人,聞承訓威政,皆還俘不敢謷。詔檢校工部尚書,封會稽縣男,擢義武節度。



 會南詔破安南,詔徙嶺西南道,城邕州,合容管經略使隸之,遂統諸軍行營兵馬。南詔深入,承訓分兵六道出以掩蠻,戰不利,士死十八,唯天平卒二千還屯,闔軍震。於是節度副使李行素完城不出,南詔圍之四日,或請夜出兵襲蠻,承訓意索,不聽。天平裨將陰募勇兒三百,夜縋燒蠻屯,斬首五百,南詔恐,明日解而去。承訓謬言大破賊,告於朝,群臣皆賀,加檢校尚書右僕射,籍子弟姻暱冒賞,而士不及,怨言嚾流。嶺南東道節度使韋宙白狀宰相,承訓慚,移疾,授右武衛大將軍,分司東都。



 咸通中,南詔復盜邊。武寧兵七百戍桂州,六歲不得代,列校許佶、趙可立因眾怒殺都將,詣監軍使丐糧鎧北還,不許,即擅斧庫,劫戰械,推糧料判官龐勛為長,勒眾上道。懿宗遣中人張敬思部送,詔本道觀察使崔彥曾慰安之。次潭州,監軍詭奪其兵,勛畏必誅,篡舟循江下,益裒兵,招亡命,收銀刀亡卒艚匿之。及徐城,謀曰:「吾等叩城大呼,眾必應,前日賞緡五十萬可得也。」眾喜。牙健趙武等欲亡,勛斬首送彥曾曰:「此搖亂者。」彥曾不能詰。勛怨都押衙尹戡、教練使杜璋、兵馬使徐行儉,又使白彥曾曰:「士負罪,不敢釋甲,請為二屯。」且白退戡等。府屬溫廷皓謂彥曾曰:「勛擅委戍,一可殺。專戕大將,二可殺。私置兵,三可殺。士不子弟即父兄,振袂而唱,內外必應,銀刀亡命,復在其中,四可殺。請分兩營,脅去三將,五可殺。」彥曾謂然。乃祃纛黃堂前,選兵三千授都虞候元密。屯任山,須勛至劫取之,遣邏子贏服覘賊。比暮,勛至,捕覘者,知其謀,即嶠偶人,剚虛幟,而詭路襲符離。密久乃寤,回屯城南。勛與宿將喬翔戰睢河,翔大敗,攝太守焦璐遁去。勛入據州,自稱兵馬留後。



 初,璐決汴水,絕勛北道,水未至,勛度,比密兵攻宿,水大至,涉而傅城,不克攻。勛劫百艘運糧趨泗州,留婦弱持掫。翌日,密覺,追之,士未食。賊伏兵於舟而陣汴上,軍見密皆走。密追躡,伏發,夾攻之,密敗,眾殲。遂入徐州,囚彥曾及官屬,殺尹戡等。又徇下邳、漣水、宿遷、臨淮、蘄、虹諸縣,皆下。遣偽將屯柳子,屯豐,屯滕,屯沛,屯蕭,以張其軍,乃露章求節度使。有周重者,隱濠、泗間,號有謀,勛迎為上客,問策所出,因教勛:「赦囚徒,據揚州,北收兗、鄆,西舉汴、宋,東掠青、齊,拓境大河,食敖倉,可以持久。」勛無雄才,不納。偽將劉行及攻濠州,執刺史盧望回,自稱刺史。帝遣中人康道隱宣慰徐州,勛郊迎,旗鎧矛戟亙三十里,使騎鳴鼙角,聲動山谷。置酒球球場,引道隱閱其眾,紿為賊來降六十人,妄戮平民,上首級誇勝。」道隱還,固求節度。即殘魚臺、金鄉、易山、單父十餘縣,斬官吏,出金帛募兵,游民多從之。



 帝乃拜承訓檢校尚書右僕射、義成軍節度使、徐泗行營都招討使,以神武大將軍王晏權為武寧軍節度使、北面行營招討使,羽林將軍戴可師為南面行營招討使,率魏博、鄜延、義武、鳳翔、沙陀、吐渾兵二十萬討之。



 勛好鬼道,有言漢高祖廟夜閱兵,人馬流汗,勛日往請命。巫言球場有隱龍,得之可戰勝,勛大役徒鑿地,不能得。賊將李圓、劉佶攻泗,歐宗、丁從實分徇舒、廬、壽、沂、海諸道。兵屯海州,度賊至,作機橋,維以長絙,賊半度,絙絕,半溺死,度者不得戰,殲之。賊別取和州,破沐陽、下蔡、烏江、巢諸縣,揚州大恐,民悉度江。淮南節度使令狐綯移書陳禍福,許助求節度,勛按甲聽命。淮南合宣、潤兵戍都梁山。勛夜度淮,崿曙薄壘,賊將劉行立、王弘立與勛合,敗淮南將李湘,屯淮口,劫盱眙。帝又詔將軍宋威與淮南並力。



 承訓屯新興,賊挑戰,時諸道兵未集,承訓帳下才萬人,退壁宋州。勛益驕。光、蔡鉅賊陷滁州,殺刺史高錫望應勛。戴可師引兵三萬奪淮口,圍勛都梁山下,降其眾。可師恃勝不戒,弘立以兵襲之,可師不克陣而潰,士溺淮死,逸者數百人,賊取可師首傳徐州。詔以馬士舉為淮南節度使、南面行營諸軍都統,馳傳入揚州。士舉曰:「城堅士多,賊何能為?」眾稍安。始,帝以晏權故智興子,節度武寧,欲以怖賊。及是,返為賊困,不敢戰,乃更以隴州刺史曹翔為兗海節度、北面都統招討使,屯滕、沛,魏博將薛尤屯蕭、豐。



 賊首孟敬文欲絕勛自立,陰刻鑒為文曰「天口雲云,錫爾將軍」,夜瘞之野,耕者得之以獻,眾駭異,乃齋三日授之。勛知其謀,使人襲殺之。



 於是承訓屯柳子右,夾汴築壘,連屬一舍。勛籍城中兵,止三千,劫民授甲,皆穿窟穴遁去。王弘立度睢,圍新興、鹿塘。承訓縱沙陀騎躪之,弘立走,士赴水死,自鹿塘屬襄城,伏尸五十里,數首二萬,獲器鎧不貲。承訓攻柳子,姚周度水戰,又敗,乘風火賊,周提餘卒去,沙陀躡之,及芳亭,死者枕藉,斬劉豐,而周以十騎走宿州,守將斬之。勛懼,乃害崔彥曾等,謂其下曰:「上不許我節度,與諸君真反矣。」大索兵,得三萬。許佶、趙可立勸勛稱「天冊將軍」,勛謁漢高祖廟受命,以其父舉直為大司馬,守徐州。或曰:「方大事,不可私於父,失上下序。」舉直乃拜於廷,勛坐受之。引兵救豐,刻木作婦人,衣絳被發,軍過,斫而火之,乃行。勛夜入城,外不知。勛出銳軍擊援屯,魏博軍知勛自將,驚而潰。賊以所得送徐州以誇下。曹翔退保兗州。勛欲乘勝攻承訓,或曰:「今北兵敗,西軍搖,不足虞也。方蠶月,宜息眾力農,至秋士馬強,決可以取勝。」舉直曰:「時不重得,願將軍無縱敵。」勛曰:「然。」時承訓方攻臨渙,聞勛計,追還兵仗以待。勛軍皆市人,囂而狂,未陣即奔,相蹈藉死者四萬。勛釋甲服垢襦脫,收夷痕士三千以歸,遣張行實屯第城。



 馬士舉救泗州,賊解去,進攻賊濠州。是時,又詔黔中觀察使秦匡謀討賊,下招義、鐘離、定遠。勛遣吳迥屯北津援濠,士舉銳兵度淮,盡碎其營。初,勛之遁,懼眾不軍,妄言有神呼野中曰:「天符下,國兵休。」勛使下相語,符未降,故敗北津。



 帝恨魏博軍不勝,以宋威為西北面招討使,率兵三萬屯蕭、豐,約勛:「降者當赦之。」始,宿鄙人劉洪者,被黃袍,白馬,使人封檄叩觀察府曰:「我當王徐。」崔彥曾斬之,遺黨匿山谷,欲附勛,承訓喻降之。王師破臨渙,斬萬級,收襄城、留武、小睢諸壁。曹翔下滕,賊將以蘄、沛降,賊李直奔入徐州。翔又破豐、徐城、下邳,賊益蹙。



 勛以張玄稔守宿州,張儒、劉景助之,自稱統軍,列壁相望。承訓拔第城,張行實奔宿州,承訓遂圍宿州。行實教勛:「官軍盡銳於此,西鄙虛單,將軍直搗宋、亳,出不意,宿圍自解。」勛喜,引而西,使舉直、許佶守徐。承訓攻敗,十遇皆勝。遣辯士以威動玄稔。玄稔,賊重將也,以帛書射城外,約誅勛自歸,使張皋獻期。俄與二將會柳溪,伏士於旁,玄稔馳騎呼曰:「龐勛首已梟僕射寨矣!」伏興,斬劉景、張儒。玄稔率諸將肉袒見承訓,自陳陷賊不早奮,久暴王師,願禽賊贖死。承訓許之。復請詐為潰軍劫符離。符離不知,內之,已入,即斬守將,得兵萬人,北攻徐州。許佶等不敢出。玄稔環城,彥曾故吏路審中啟白門內玄稔兵,許佶等啟北門走,玄稔身追之,士大崩,皆赴水死,斬舉直、許佶、李直等,收叛卒親族悉夷之。



 勛聞徐已拔,氣喪,無顧賴,眾尚二萬,自石山而西,所在焚掠。承訓悉兵八萬逐北,沙陀將硃耶赤衷急追至宋州,勛焚南城,為刺史鄭處沖所破,將南趨亳,承訓兵循渙而東,賊走蘄縣,官兵斷橋,不及濟,承訓乃縱擊之,斬首萬級,餘皆溺死。閱三日,得勛尸。斬其子於京師。吳迥守濠州,糧盡食人,驅女孺運薪塞隍,並填之,整旅而行,馬士舉斬以獻。勛之始得徐州,貲儲蕩然,乃四出剽取,男子十五以上皆執兵,舒鉏鉤為兵,號「霍錐」,破十餘州,凡二歲滅。



 詔擢張玄稔右驍衛大將軍,承訓遷檢校左僕射、同中書門下平章事,徙節河東。於是宰相路巖、韋保衡劾承訓討賊逗撓,貪虜獲,不時上功。貶蜀王傅,分司東都。再貶恩州司馬。僖宗立,授左千牛衛大將軍。卒,年六十六。



 子傳業,嘗從父征伐,終鄜坊節度使。



 李洧者,淄青節度使正己從父兄也。始,署徐州刺史。建中初,正己卒,子納叛,攻宋州,洧挈州自歸,加兼御史大夫,封潮陽郡王,實封戶二百,充招諭使。初,洧遣巡官崔程入朝,且白宰相:「徐州不足獨抗賊,得海、沂為節度,可與成功。洧素與二州刺史有約,且不肯為賊守。」程先咨張鎰,而盧杞怒不先白,故洧請中格。及納攻徐,劉玄佐與諸將擊退之。既賊方張,乃加洧徐海沂密觀察使。時海、密為賊守,不受命,洧未有以取之。遷檢校戶部尚書。會疽發背,少間,肩輿過市,市人叫歡,洧驚,疽潰卒,贈尚書左僕射。以洧將高承宗代之。



 其弟淡,險人也,恥居下,陰約納攻徐為內應,並說滕將翟濟,濟執以聞。擢濟沂州刺史。召淡入京師,以洧赦不罪。



 劉澭,盧龍節度使怦之次子,濟母弟也。涉書史,有材武,好施愛士,能得人死力。始事硃滔,常陳君臣大分,裁抑其兇。及怦得幽州,不三月病且死,澭侍湯液未嘗離,輒以父命召濟於莫州,濟嗣總軍事,故德澭之讓,以為瀛州刺史,有如不諱,許代己。



 久之,濟自用其子為副大使,澭不能無恨,因請以所部為天子戍隴,悉發其兵千五百馳歸京師,無一卒敢違令者。德宗甚寵之,拜秦州刺史,屯普潤。軍中不設音樂。士卒病,親存問所欲,不幸死,哭之。



 憲宗立,方士羅令則詣澭營,妄言廢立以動澭,命系之,辭曰:「吾之黨甚眾,公無囚我,約大行梓宮發兵,無不濟。」澭械送闕下,殺之。錄功,號其軍曰保義。蕃戎畏懾,不敢入寇。常愾然有復河湟志,屢為朝廷言之,未見省。封累彭城郡公。及病,籍士馬求代。既還,卒於道,年四十九,贈尚書右僕射,謚曰景。



 田弘正,字安道。父廷玠,尚儒學,不樂軍旅,與承嗣為從昆弟,仕為平舒丞,遷樂壽、清池、束城、河間四縣令,以治稱。遷滄州刺史。李寶臣、硃滔與承嗣不協,合兵圍滄州,廷玠固守連年,食雖盡無叛者。朝廷嘉其節,徙相州。承嗣盜磁、相,廷玠無所回染。及悅代立,忌廷玠之正,召為節度副使。廷玠至,讓悅曰:「而承伯父緒業,當守朝廷法度以保富貴,何苦與恆、鄆為叛臣?自兵興來,叛天子能完宗族者誰邪?而志不悛,盍殺我,無令我見田氏血污人刀也!」遂稱疾不出。悅過謝之,杜門不納,憤而卒。



 弘正幼通兵法,善騎射,承嗣愛之,以為必興吾宗,名之曰興。季安時,為衙內兵馬使、同節度副使,封沂國公。季安侈汰,銳殺罰,弘正從容規切,軍中賴之,翕然歸重。季安內忌,出為臨清鎮將,欲因罪誅之。弘正陽痺痼,臥家不出,乃免。季安死,子懷諫襲節度,召還舊職。



 懷諫委政於家奴蔣士則,措置不平,眾怒,咸曰:「兵馬使吾帥也。」牙兵即詣其家迎之,弘正拒不納,眾嘩於門,弘正出,眾拜之,脅還府,弘正頓於地,度不免,即令於軍曰:「爾屬不以吾不肖,使主軍,今與公等約,能聽命否?」皆曰:「惟公命。」因曰:「吾欲守天子法,舉六州版籍請吏於朝,茍天子未命,敢有請吾旗節者死,殺人及掠人者死。」皆曰:「諾。」遂到府,殺士則及支黨十餘人。於是圖魏、博、相、衛、貝、澶之地,籍其入以獻,不敢署僚屬,而待王官。



 先時,諸將出屯,質妻子,里民不得相往來。弘正悉除其禁,聽民通饋謝慶吊。服玩僭侈者,即日徹毀之。承嗣時,正寢華顯,弘正避不敢居,更就採訪使堂皇聽事。幽、恆、鄆、蔡大懼,遣客鐫說鉤染,弘正皆拒遣之。憲宗美其誠,詔檢校工部尚書,充魏博節度使。又遣司封郎中知制誥裴度宣慰,賚其軍錢百五十萬緡,六州民給復一年,赦見囚,存問高年、煢獨、廢疾不能自存者。度明辯,具陳朝廷厚意,弘正不覺自失,乃深相結納,奉上益謹。復請度遍行其部,宣示天子恩詔。因令節度僉謀布衣崔歡奉表陳謝,且言:「天寶以來,山東奧壤,化為戎墟,官封世襲,刑賞自出,國家含垢,垂六十年。臣若假天之齡,奉陛下宸算,冀道揚太和,洗濯偽風,然後退歸丘園,避賢者路,死不恨。」制詔褒答,且賜今名,錫與踵塗。



 天子討蔡,弘正遣子布以兵三千進戰,數有功。李師道疑其襲己,不敢顯助蔡,故元濟失援,王師得致誅焉。王承宗叛,詔弘正以全師壓境,破其眾南宮,承宗懼,歸窮於弘正,弘正表諸朝,遂獻德、棣二州以謝,納二子為質。



 俄而李師道拒命,詔弘正與宣武等五節度兵進討。弘正自揚劉度河,距鄆四十里堅壁;師道大將劉悟率精兵屯河東。戰陽谷,再遇再北,斬萬餘級,賊勢蹙。悟乃反兵,斬師道首,詣弘正降,取十有二州以獻。初,悟既平賊,大張飲軍中,凡三日,設角抵戲,引魏博使至廷以為歡,悟盱衡攘臂助其決,坐中皆憚悟勇。客有白弘正者,弘正曰:「鄆士疲於戰,瘡者未起,悟當恤亡吊乏,尉士大夫心,奈何取快目前邪?吾奉詔按軍,伺悟去就,今知其無能為也。」既而詔悟為義成軍節度使,狼狽上道,時稱知悟之明。



 以功加弘正檢校司徒、同中書門下平章事。是歲來朝,對麟德殿,眷勞殊等;引見僚佐將校二百餘人,皆有班賜;進兼侍中,實封戶三百;擢其兄融為太子賓客、東都留司。弘正數上表固請留闕下,帝勞曰:「昨韓弘以疾辭不就軍,朕既從之矣,今卿復爾,我不應違。但魏人樂卿之政,四鄰畏卿之威,為朕長城,又安用辭?」弘正遂還。常欲變山東承襲舊風,故悉遣子姓仕朝廷,帝皆擢任之,硃紫滿門,榮冠當時。



 穆宗立,王承元以成德軍請帥,帝詔弘正兼中書令,為節度使。弘正以新與鎮人戰,有父兄怨,取魏兵二千自衛,入其軍。時天子賜錢一百萬緡,不時至,軍有怨言,弘正親加撫喻乃安。仍請留魏兵為紀綱,以持眾心,度支崔倰吝其稟,沮卻之。長慶元年七月,歸衛卒於魏,是月軍亂,並家屬將吏三百餘人皆遇害,年五十八。帝聞震悼,冊贈太尉,謚曰忠愍。



 弘正幼孤,事融甚謹,軍中嘗分曹習射,弘正注矢聯中,融退,抶怒之,故當季安猜暴時能自全。及為軍中推迫,融不悅曰:「爾竟不自晦,取禍之道也。」朝廷知其友愛,詔拜相州刺史,賜金紫,不欲其相遠也。



 弘正性忠孝,好功名,起樓聚書萬餘卷,通《春秋左氏》,與賓屬講論終日,客為著《沂公史例》行於世。弘正之禍也,其判官劉茂復獨免,士相戒曰:「是人議事盡忠,遇吾等信,敢干其家者共殺之。」



 弘正子布、群、牟。



 布字敦禮,幼機悟。弘正戍臨清,布知季安且危,密白父,請以眾歸朝,弘正奇之。及得魏,使布總親兵。王師誅蔡,以軍隸嚴綬,屯唐州。帝以布大臣子,或有罪,且撓法,弘正請以董畹代,而士卒愛布願留,帝乃止。凡十八戰,破凌雲柵,下郾城,以功授御史中丞。裴度輕出觀兵沱口,賊將董重質以奇兵掩擊,布伏騎數百突出薄之,諸軍繼至,賊驚引還。蔡平,入為左金吾衛將軍。諫官嘗論事帝前,同列將麾卻之,布止曰:「使天子容直臣,毋輕進。」弘正徙成德,以布為河陽節度使,父子同日受命。時韓弘與子公武亦皆領節度,而天下以忠義多田氏。布所至,必省冗將,募戰卒,寬賦勸穡,人皆安之。長慶初,徙涇原。



 弘正遇害,魏博節度使李愬病不能軍,公卿議以魏強而鎮弱,且魏人素德弘正,以布之賢而世其官,可以成功。穆宗遽召布,解縗拜檢校工部尚書、魏博節度使,乘傳以行。布號泣固辭,不聽;乃出伎樂,與妻子賓客決曰:「吾不還矣!」未至魏三十里,跣行被發,號哭而入,居堊室,屏節旄。凡將士老者,兄事之。祿奉月百萬,一不入私門,又發家錢十餘萬緡頒士卒。以牙將史憲誠出麾下可任,乃委以精銳。時中人屢趣戰,而度支饋餉不繼,布輒以六州租賦給軍。引兵三萬進屯南宮,破賊二壘。



 於是硃克融據幽州,與王廷湊脣齒。河朔三鎮舊連衡,桀驁自私,而憲誠蓄異志,陰欲乘釁,又魏軍驕,憚格戰,會大雪,師寒糧乏,軍中謗曰:「它日用兵團,粒米盡仰朝廷。今六州刮肉與鎮、冀角死生,雖尚書瘠己肥國,魏人何罪?」憲誠得間,因以搖亂。會有詔分布軍合李光顏救深州,兵怒,不肯東,眾遂潰,皆歸憲誠,唯中軍不動。布以中軍還魏。明日,會諸將議事,眾嘩曰:「公能行河朔舊事,則生死從公,不然,不可以戰。」布度眾且亂,嘆曰:「功無成矣!」即為書謝帝曰:「臣觀眾意,終且負國。臣無功,不敢忘死。願速救元翼,毋使忠臣義士塗炭於河朔。」哭授其從事李石訖,乃入,至幾筵,引刀刺心曰:「上以謝君父,下以示三軍。」言訖而絕,年三十八,贈尚書右僕射,謚曰孝。子金歲,宣宗時歷銀州刺史,坐以私鎧易邊馬論死,宰相崔鉉奏布死節於國,可貸金歲以勸忠烈,故貶為州司馬。



 群,會昌中歷蔡州刺史,坐贓且抵死,兄肇聞之,不食卒。宰相李德裕奏:「漢河間人尹次、潁川人史玉坐殺人當死,次兄初、玉母渾詣官請代,因縊物故,於時皆赦其死。」於是武宗詔減死一等。



 牟寬厚明吏治,為神策大將軍。開成初,鹽州刺史王宰失羌人之和,詔牟代之。累遷鄜坊節度使,再徙天平,三為武寧,一為靈武軍,官至檢校尚書左僕射,卒。諸子皆有方面功,以忠義為當世所高。



 王承元者,承宗弟也。有沉謀。年十六,勸承宗亟引兵共討李師道,承宗少之,不用,然軍中往往指目之。承宗死,未發喪,大將謀取帥它姓。參謀崔燧與諸校計,以祖母涼國夫人李命承元嗣。承元泣且拜,不受,諸將牢請,承元曰:「上使中貴人監軍,盍先請?」監軍至,又如命,乃謝曰:「諸君不忘王氏以及孺子,茍有令,其從我乎?」眾曰:「惟所命。」乃視事牙闔之偏,約左右不得稱留後,事一關參佐,密表請帥於朝。穆宗詔起居舍人柏耆宣慰。授承元檢校工部尚書、義成軍節度使。北鎮以兩河故事脅誘,承元不納,諸將皆悔。耆至,士哭於軍,承元令曰:「諸君不欲我去,意固善。雖然,格天子詔,我獲罪奈何?前李師道有詔赦死,欲舉族西,諸將止弗遣,他日乃共殺之。今君等幸置我,無與師道比。」乃遍拜諸將,諸將語塞。承元即出家貲盡賜之,斬不從命者十輩,軍乃定。於是諫議大夫鄭覃宣慰,賜其軍錢百萬緡,赦囚徒,問孤獨、廢疾不能自存者粟帛有差。



 承元去鎮,左右裒器幣自隨,承元使空褚毋留。入朝,昆弟拜刺史者四人,位於朝者四十人。祖母入見,帝命中宮禮賚異等。徙承元鄜坊丹延節度。俄徙鳳翔。鳳翔右袤涇、原,地平少巖險,吐蕃數入盜。承元據勝地為鄣,置守兵千,詔號臨汧城。府郛左百賈州聚,異時為虜剽奪,至燎烽相警,承元版堞繚之,人乃告安。以勞封岐國公。太和初,祖母喪,詔曰:「武俊當橫流時,拯定奔潰,功在史官。今李不幸,贈恤宜加厚。」且給儀仗以葬。



 五年,徙節平盧、淄青。始,鹽禁未嘗行兩河,承元請歸有司,由是兗、鄆諸鎮皆奉法。承元資仁裕,所至愛利。卒,年三十三,贈司徒。



 牛元翼,趙州人。材果而謀。王承宗時倚其計為強雄,與傅良弼二人冠諸將。王廷湊叛,穆宗以元翼在成德,名出廷湊遠甚,自深州刺史擢為深冀節度使,以攜其軍。廷湊怒,遣部將王位以銳兵攻元翼,不勝,乃合硃克融共圍之。詔進元翼成德軍節度使,以宣武兵五百進援,元翼固守。長慶二年,詔赦廷湊罪,徙元翼山南東道,以深州賜廷湊,使中人促元翼南。廷湊恨之,已受詔,兵不解。招討使裴度詒書誚讓,克融解而歸,廷湊退舍。詔並加檢校工部尚書,兩悅之。淹月,元翼率十餘騎冒圍跳德、棣,朝京師。廷湊入,盡殺元翼親將臧平等百八十人。元翼見延英,賚問優縟,命中人楊再昌取其家,並迎田弘正喪。廷湊辭以弘正殯亡在所,元翼家須秋遣。魏博節度使史憲誠遣其弟入趙,四返,說廷湊曰:「田公非得罪於趙,尸尚何惜?元翼去深州,乃一孤將,何利其家?」廷湊乃歸弘正喪於京師。元翼聞平等死,憤恚卒,悉還所賜於朝,廷湊遂夷其家。



 良弼字安道,清河人。以射冠軍中。初,瀛之博野、樂壽,介範陽、成德間,每兵交,先薄二城,故常為劇屯。德宗以王武俊破硃滔功,皆隸成德,故以良弼守樂壽,李寰守博野。廷湊之叛,兩賊交誘之,而堅壁為國固守。有詔以樂壽為左神策行營,拜良弼為都知兵馬使;寰所領士隸右神策,號忻州營,亦以寰為都知兵馬使。賜第京師。俄以良弼為沂州刺史。良弼率眾出,戰力,乃得去。寰引兵三千趨忻州,廷湊邀之,寰斬三百級,追者不敢前。天子以良弼、寰忠有狀,乃更賜奴婢服馬。召良弼為左神策軍將軍。寶歷初,擢夏綏銀節度使。異時蕃帳亡命來者,必償馬乃與,良弼至,皆執付其部,酋種歡懷。終橫海節度使。寰擢累保義軍節度使。



 王智興討李同捷未克,而烏重胤卒,謂寰可共立功,請諸朝,乃授橫海節度使。師所過暴鈔,至屯,按軍不進,遂身入朝,盛陳賊勢,請濟師,欲大調發。群臣議寰兵太重,且盜滄、景,未決而棣州平。寰內愧不自安,願留京師,遂罷保義軍、忻州營,更授夏綏宥節度使,卒。



 寰再易鎮,治無可言者。然廷湊之亂,聯軍十五萬無成功,賊鋒不可嬰,而樂壽、博野截然峙中者累歲,梗其吞暴,議者以為難。敬宗世,寰圖其事上之。



 史孝章,字得仁,資修謹。父憲誠,以戰力奮,賓客用挽強擊劍相矜,孝章獨退讓如諸生,稱道皆《詩》、《書》。魏博節度使李訴閱大將子弟籍於軍,孝章願以文署職,愬奇之,檄試都督府參軍。



 憲誠得魏,遷士曹參軍。孝章見父數奸命,內非之,承間諫曰:「大河之北號富強,然而挺亂取地,天下指河朔若夷狄然。今大人身封侯,家富不貲,非痛洗溉,竭節事上,恐吾踵不旋禍且至。」因涕下沾衿。父粗武,不盡聽。文宗賢之,擢孝章節度副使,累遷檢校左散騎常侍。父欲助李同捷,孝章切爭,憲誠稍憚其義。又勸出師討同捷自明,帝益嘉之,進檢校工部尚書。及兵出,父敕孝章統之。入朝,勞予蕃厚。憲誠亦上書求覲,帝知非憲誠意,特緣孝章悟發,故分相、衛、澶而授孝章節度使。未至,魏人亂,父卒死於軍。帝念史氏禍而恤孝章,故奪喪拜右金吾衛將軍。徙節鄜坊,進檢校戶部尚書。久之,自邠寧以病丐還,卒於行,年三十九,贈尚書右僕射。孝章本名唐,後改今名。



 憲誠弟憲忠,字元貞,少為魏牙門將。田弘正討齊、蔡,常為先鋒,閱三十戰,中流矢,酣鬥不解,由是著名。憲誠表為貝州刺史。魏亂,奔京師,加累檢校右散騎常侍、隴州刺史。增亭鄣,徙客館於外,戎諜無所伺。



 會昌中,築三原城,吐蕃因之數犯邊。拜憲忠涇原節度使以怖其侵,吐蕃遣使來請墮城,且願以嘗殺使者之人置塞上。憲忠使謝曰:「前吾未城。爾犯我地,安得禁吾城?爾知殺吾使為負,宜先取罪人謝我,將無所不得。今與爾約,前節度使事一置之。」吐蕃情得而服。憲忠疏涇于隍,積緡錢十萬、粟百萬斛,戍人宜之。會黨項羌內寇,又徙朔方,有詔馳驛赴屯,憲忠辭曰:「羌不得其心,故不自安。今亟往,知吾為備,鬥益健,請徐行。」許之。乃移書與羌人,示要約。羌人乃皆喜,奉酒湩迎道。



 大中初,突厥擾河東,鈔漕米行賈,徙節振武軍。於是故帥荒沓,使游弈兵覘戎有良馬牛,強取之,歸直十一,戎人怒,因興盜掠。憲忠廉儉,少所欲,嘗曰:「吾居河朔,去此三千里,乃乘五健馬。今守邊,發吾餘奉,不患無馬,何忍豪市哉?」故所至莫不懷德。累封北海縣子,檢校尚書左僕射,兼金吾大將軍。以病自丐,改左龍武統軍。卒,年七十一,贈司空。



\end{pinyinscope}