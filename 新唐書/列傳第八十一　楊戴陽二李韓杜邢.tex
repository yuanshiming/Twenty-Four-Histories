\article{列傳第八十一 楊戴陽二李韓杜邢}

\begin{pinyinscope}

 楊朝晟,字叔明,夏州朔方人。興行間,以先鋒功授甘泉府果毅。建中初,從李懷光討劉文喜涇州但在闡述時,有不少錯誤和混亂,如誇大了人類知識的相對,斬獲多,加驃騎大將軍。李納寇徐州,從唐朝臣往討,常冠軍。懷光赴難奉天,屬朝晟兵千人下咸陽,賜實封百五十戶。



 懷光反,韓游瑰退保邠、寧,賊黨張昕守邠州,大索軍實,多募士,欲潛歸之。朝晟父懷賓為游瑰將,夜以數十騎斬昕及同謀者。游瑰遣懷賓告行在,德宗勞問,授兼御史中丞。朝晟泣見懷光曰:「父立功於國,子當誅,不可以主兵。」懷光縶之。及諸軍圍河中,游瑰營長春宮,而懷賓戰甚力。懷光平,帝原朝晟,因為游瑰都虞候。父子皆開府、賓客、御史中丞,軍中以為榮。



 吐蕃犯邊,游瑰自將守寧州,而御士寬,軍驕。及張獻甫來代,軍遂亂,朝晟逃於郊。眾脅監軍,請以範希朝為節度使。希朝時已在京師。明日,朝晟出,紿眾曰:「予來賀所請之當也。」眾稍定。朝晟結諸將謀誅首惡者。居三日,紿遣人自邠來,曰:「前請報罷,張公已舍邠矣,反者皆當死,吾不願盡誅也,第取首惡者。」眾所言雚指,斬二百餘人,獻甫遂入於軍。帝以希朝為節度副使,而朝晟加御史大夫。



 貞元九年,城鹽州,發卒護境,朝晟屯木波堡。會獻甫卒,有詔代為邠寧節度使。朝晟請城方渠,合道,木波以遏吐蕃路。詔問:「須兵幾何?」報曰:「部兵可辦。」帝問:「前日城五原,興師七萬,今何易邪?」對曰:「鹽州之役,虜先知之。今薄戎而城,虜料王師不十萬,勢難輕入。若發部兵,十日至塞下,未三旬城畢,積芻聚糧,留卒守之,寇至不可拔,萊野翦夷,虜且走,此萬全計也。若大發兵,閱月乃至,虜亦來,來必戰,戰則不暇城矣。」帝納其策。師次方渠,水乏。有青蛇降險下走,視其跡,水從而流,朝晟使築防環之,遂為渟淵,士飲仰足,圖其事以聞。有詔置祠,命泉曰應聖。已城,吐蕃悉眾至,度不能害,乃引去。復城馬嶺而歸,開地三百里。十七年,卒於屯。



 戴休顏,字休顏,夏州人。家世尚武,志膽不常。郭子儀引為大將,諭平黨項羌,以安河曲。試太常卿,封濟陰郡公,進封咸寧郡王,兼朔方節度副使。城邠州功最,遷鹽州刺史。硃泚反,率兵三千晝夜馳,奔問行在,德宗嘉之,賜實戶二百。與渾瑊、杜希全、韓游瑰等捍禦有勞。帝進狩梁、洋,留守奉天。李懷光屯咸陽,使人誘之,休顏斬其使,勒兵自守。懷光眙駭,自涇陽夜走。遷檢校工部尚書、奉天行營節度使。合渾瑊兵破泚偏師,斬首三千級,追至中渭橋。京師平,又與瑊率兵趨岐陽,邀泚殘黨。加檢校尚書右僕射,進戶四百。從乘輿至京師,賜女樂、甲第,拜左龍武軍統軍。卒,贈揚州大都督。



 弟休璿,歷開府儀同三司,封東陽郡王;休晏,歷輔國大將軍,封彭城郡公。俱以將略稱。



 陽惠元,平州人。以趫勇奮,事平盧軍。從田神功、李忠臣浮海入青州。詔以兵隸神策,為京西兵馬使,鎮奉天。



 德宗初立,稍繩諸節度跋扈者。於是李正己屯曹州,田悅增河上兵,河南大擾。詔移兵萬二千戍關東,帝禦望春樓誓師,因勞遣諸將。酒至神策,將士不敢飲。帝問故,惠元曰:「初發奉天,臣之帥張巨濟與眾約:『是役也,不立功,毋飲酒!』臣不敢食其言。」既行,有饋於道,惟惠元軍瓶罍不發。帝咨嘆不已,璽書慰勞。俄以兵三千會諸將擊田悅,戰御河,奪三橋,惠元功多。以兵屬李懷光。



 及硃泚反,自河朔赴難,解奉天圍,加檢校工部尚書,攝貝州刺史。詔惠元與神策行營節度使李晟、鄜坊節度使李建徽及懷光聯營便橋。晟知懷光且叛,移屯東渭橋。翰林學士陸贄諫帝曰;「四將接壘,晟等兵寡位下,為懷光所易,勢不兩完。晟既慮變,請與惠元東徙,則建徽孤立。宜因晟行,合兩軍皆往,以備賊為解,趣裝進道,則懷光計無所施。」帝不從,使神策將李升往伺。還奏:「懷光反明甚。」是夕,奪二軍,惠元、建徽走奉天,懷光遣將冉宗馳騎追及於好畤。惠元被發呼天,血流出眥,袒裼戰而死。二子晟、暠匿井中,皆及害。建徽獨免。詔贈惠元尚書左僕射,晟殿中監,暠邠州刺史。



 少子旻,字公素。惠元之死,被八創,墮別井,或救得免。歷邢州刺史。盧從史既縛,潞軍潰,有驍卒五千,從史嘗以子視者,奔於旻,旻閉城不內。眾皆哭曰:「奴失帥,今公有完城,又度支錢百萬在府,少賜之,為表天子求旌節。」旻開諭禍福遣之,眾感悟,遂還軍。憲宗嘉之,遷易州刺史。



 王師討吳元濟,以唐州刺史提兵深入二百里,薄申州,拔外郛,殘其垣。以功加御史中丞。容州西原蠻反,授本州經略招討使,擊定之。進御史大夫,合邕、容兩管為一道。卒,贈左散騎常侍。



 李元諒,安息人。本安氏,少為宦官駱奉先養息,冒姓駱,名元光。美須髯,鷙敢有謀。以宿衛積勞試太子詹事。李懷讓節度鎮國,署奏以自副。居軍十年,士心憚服。



 德宗出奉天,賊遣將何望之襲華州,於是刺史董晉棄城走。望之欲聚兵以絕東道,元諒自潼關引兵徑薄其城,拔之。時兵興倉卒,裹罽為鎧,剡蒿為矢,募兵數日至萬餘,軍氣乃振。賊來攻,輒卻。時尚可孤守藍田,元諒屯昭應,王權壁中渭橋,賊兵不能逾渭南。未幾,遷鎮國軍節度使,封武康郡王。先是,詔發豳、隴兵東討李希烈。師方出關,泚使劉忠孝召還;至華陰,華陰尉李夷簡說驛官捕之,追及關,元諒斬以徇,所召兵不得入,由是華州獨完。俄詔元諒與李晟收京師,次滻西。元諒先奮鏖賊,敗之,進屯苑東,晟使壞苑垣入。泚連戰皆北,遂大潰,京師平。讓功於晟,退壁近郊。加檢校尚書左僕射,實封戶五百,賜甲第、女樂、一子六品官。



 李懷光反,與馬燧、渾瑊討之。其將徐廷光素易元諒,數嫚罵,為優胡戲,斥侮其祖。又使約降,曰:「我降漢將耳。」及馬燧至,降於燧。元諒見韓游瑰曰:「彼詬吾祖,今日斬之,子助我乎?」許諾。既而遇諸道,即數其罪,叱左右斬之,詣燧謝。燧大怒,將殺元諒,游瑰見曰:「殺一偏裨尚爾,即殺一節度,法宜如何?」燧默然。元諒請輸錢百萬勞軍自贖,瑊亦為請,燧赦之。帝以專殺,恐有司劾治,前詔勿論。



 貞元三年,吐蕃請盟,詔以軍從瑊會平涼,元諒軍潘原、游瑰軍洛口以為援。元諒曰:「潘原去平涼七十里,虜詐不情,如有急,何以赴?請與公連屯。」瑊以違詔,不聽。瑊壁盟所二十里,元諒密徙營次之。既會,元諒望雲物曰:「不詳,虜必有變!」傳令約部伍出陣。俄而虜劫盟,瑊奔還,元諒兵成列出,而涇原節度使李觀亦以精兵五千伏險,與元諒相表裏,虜騎乃解。元諒遣車重先,而與瑊振旅徐還,時以為有古良將風。是會也,微元諒、觀二人,瑊且不免。帝嘉嘆,賜善馬金幣良厚,因賜姓及名。



 更節度隴右,治良原。良原隍堞湮圮,旁皆平林薦草,虜入寇,常牧馬休徒於此。元諒培高浚淵,身執苦與士卒均,菑翳榛莽,闢美田數十里,勸士墾藝,歲入粟菽數十萬斛,什具畢給。又築連弩臺,遠烽偵,為守備,進據勢勝,列新壁。虜至無所掠,戰又輒北,由是涇、隴以安,西戎憚之。卒,年六十二,贈司空,謚曰莊威。



 李觀,其先自趙郡徙洛陽,故為洛陽人。少沈厚寡言。以策干朔方節度使郭子儀,子儀遣佐坊州刺史吳人由為防遏使。以親喪解。吐蕃內寇,代宗幸陜,觀隱盩厔,率鄉里子姓千人守黑水,虜不敢侵。嶺南節度使楊慎微奏為偏將,徐浩、李勉代節度,常倚以軍政,數捕平劇賊。遷大將,試殿中監,召為右龍武將軍。



 涇師叛,觀適番上,即領兵千餘扈德宗奉天。詔盡察諸軍,整飭誰邏,增募五千人,鼙■歡豎,士氣益振。賜封戶二百,授二子八品官。從至梁州。帝還,詔總後軍。擢四鎮、北廷行軍涇原節度使。在屯四年,訓部伍,儲藏饒衍。平涼之盟,吐蕃不得志。是年,觀入朝,前一日就道,虜至期出精騎狙擊,不及,去。以少府監檢校工部尚書。卒,贈太子少傅。



 韓游瑰,靈州靈武人,始為郭子儀裨將。安祿山反,使阿史那從禮將同羅、突厥五千騎偽降於朔方,出塞門,誘河曲九蕃府、六胡叛,部落凡五十萬。子儀使游瑰率辛京杲擊破之,九蕃府還附。累進邠寧節度留後。



 奉天之狩,兵未集,游瑰與慶州刺史論惟明以兵三千來赴,自乾陵北趨醴泉,未至,有詔引軍屯便橋。次泥泉,與泚兵值,游瑰欲還奉天,監軍翟文秀曰:「吾壁於此,賊敢逾我而西,可夾攻取之。今入奉天,賊亦隨至,是引賊迫天子也。」游瑰曰:「不然,我寡賊眾。彼分以亢我,餘眾猶能鼓而西也,不如先入衛天子。且奉天無強卒,安得夾攻?吾士乏且寒,賊以利誘之,眾且潰。」遂還奉天。泚兵躡攻之,戰不利;泚兵奪門,游瑰殊死戰,乃解。泚大治戰棚、雲橋,士皆懼,游瑰曰:「賊取佛祠干木為攻具,可以火之。」既而賊大噪攻南雉,游瑰曰:「是分吾力也。」趨北雉,遣將郭詢、郭廷玉以銳士三百傅滿直出,火其棚,投薪於中,風返,棚皆燼,賊氣沮。故諸將推游瑰赴難功第一。帝以衛軍無職局,軍置統軍一員,以游瑰、惟明、賈隱林處之。



 李懷光叛,誘游瑰為變,游瑰白發其書。帝曰:「卿可謂忠義矣!」對曰:「臣安知忠義?但懷光誤臣,使震驚乘輿,後持臣自解。」帝嘉其誠,從問:「計欲安出?」對曰:「懷光總諸府兵,怙以為亂。今邠有張昕,靈武有寧薍景璿,河中有呂鳴岳,振武有杜從政,潼關有李朝臣,渭北有竇覦,皆守將也。陛下以其眾與地授之,罷懷光權,而尊以元功,諸將仰首,各聽其帥,彼安能以亂?」帝曰:「罷懷光權而泚益張,若何?」對曰:「陛下約士以不次之賞,今貢賦方至,發而酬之,其守自固。邠有萬精甲,臣得將之,可以誅賊。四方杖義而起,賊不足慮。」帝美其言。



 會懷光誘復至,渾瑊得書,稍嚴卒以警。游瑰不知,發怒,嫚罵瑊。帝疑有變,即日幸梁州,游瑰使子從帝。懷光檄假游瑰邠州刺史,欲因張昕殺之。游瑰既失兵,不知所圖。有客劉南金說曰:「邠有留甲,可以立功,殆天假也!」游瑰悟,誘舊部兵八百馳入邠,說昕曰:「懷光自蹈禍機,公今可取富貴,無共污不義也。我願以麾下為公先驅。」昕不聽。游瑰移疾不出,陰結其將高固等。昕欲殺游瑰,戒左右衷甲入。昕小史李岌潛白游瑰,伏甲先起,高固等應之,斬昕首以聞。時懷光子玫在邠,游瑰衛出之,曰:「殺之只以怒敵,至必遽,不如舍之。」玫至涇陽,懷光遂走蒲州。



 游瑰屯七盤,受李晟節度。詔拜邠寧節度使,遂會渾瑊於奉天,與瑊、戴休顏分扼京西要險。李晟入長安,游瑰破泚兵咸陽。泚走涇州,游瑰使諭涇將楊澄,澄拒不納,泚遂敗。京師平,遷檢校尚書左僕射,實封戶四百。帝至自興元,游瑰及瑊、休顏從,而李晟、尚可孤、李元諒奉迎,論功與瑊等皆第一。游瑰還屯邠寧。懷光寇同州,瑊、元諒敗於乾坑。詔游瑰率兵並力,敗賊眾五千於屯。遂會瑊、馬燧圍蒲城。師次焦籬堡,守將尉珪降。懷光見勢單蹙,乃縊死。



 貞元二年,吐蕃入涇、隴、邠、寧,游瑰追至安化,虜營合水北。游瑰策曰:「賊行無人地,必怠,可襲取之。」使將史履澄夜領兵五百入其營,斬數百級,取馬五千。遲明,虜以兵尾擊,游瑰羅幟自衛,鼙鼓四發,虜驚潰去。是歲,復圍鹽州,刺史杜彥光約與之城,吐蕃許之,又取銀、夏、麟等州。游瑰請收鹽州以斷戎人走集:「虜入漢,食禾菽,方春而病,此天亡時也。」有詔李元諒、韓全義率師一萬,會游瑰收鹽州。吐蕃請修清水盟,以歸侵地,馬燧為之請。詔問游瑰,答曰:「西戎弱則請盟,強則入寇。今侵地益深而乞盟,詐我也!」帝不從。會盟平涼,詔游瑰以軍屯洛口。盟之日,游瑰以勁騎五百待非常,令曰:「即有變,急趨柏泉以分虜勢。」瑊被劫,馳以免,虜見兵出,即解去。後吐蕃寇大回原,游瑰方壁長武,即選騎八百迎擊,自引兵繼之。監軍以為戎不可易,答曰:「賊攻豐義,今游騎先破,則彼大眾不敢前,豐義全矣!」戰南原,敗之,吐蕃夜遁。



 會子欽緒以射生將衛京師,與妖人李廣弘謀反,謀洩,奔邠州,中人捕斬,以狀示游瑰。游瑰懼,求歸死京師,帝不許。又執欽緒二息送京師,帝亦原之。未幾,入朝,素服聽命,有詔復位,勞遇如故。



 游瑰盛言城豐義以遏虜侵。帝悅,趣還軍。初,游瑰之朝,眾謂且得罪,故齎送殊薄。既還,舉軍不自安。大將範希朝善兵,游瑰畏其逼,欲誅之,希朝奔鳳翔,帝聞,召入宿衛。游瑰遣兵築豐義,才二板而潰,寧卒數百大掠,游瑰不能禁。詔用張獻甫代之。游瑰畏亂,委軍輕出,還京師,拜右龍武統軍。卒,謚曰襄。



 廣弘者,自言宗室子。始為浮屠,妄曰:「我嘗見岳、瀆神,當作天子,可復冠。」男子董昌舍廣弘於資敬寺,召相工唐郛視之,教郛告人曰:「廣弘且大貴。」乃誘欽緒、神策將魏循、李傪、越州參軍事劉昉等作亂。昉家數具酒大會廣弘所,陰相署置。又妄曰:「神戒我十月十日趣舉。」約欽緒夜擊鼓,噪凌霄門,焚飛龍廄,循等以神策兵迎廣弘,事捷,大剽三日。循、傪上變,乃禽廣弘及支黨鞫仗內,付三司訊實,皆殊死。廣弘臨刑,色自如。由是禁人不得入觀、祠。



 杜希全,京兆醴泉人。以裨將隸郭子儀,積功勞至朔方節度使。軍令整嚴,士畏其威。奉天之狩,希全與鄜坊節度使李建徽、鹽州刺史戴休顏、夏州刺史時常春引兵赴難。次漠谷,為賊邀擊,乘高縱石下之,強弩雜發。德宗使援之,不克,還保邠州。賊平,遷檢校尚書左僕射、靈鹽豐夏節度使,封餘姚郡王。將即屯,獻《體要》八章,砭切政病。帝嘉納,賜《君臣箴》一篇。



 尋兼夏綏銀節度都統,建言:「鹽州據要會,為塞保鄣。自平涼背盟,城陷於虜,於是靈武勢縣,鄜坊單逼,為邊深患,請復城鹽州。」乃詔希全及朔方、邠寧、銀夏、鄜坊、振武及神策行營諸節度合選士三萬五千屯鹽州,又敕涇原、劍南、山南軍深入吐蕃,牽撓其力,使不得犯塞。執築凡六千人,閱二旬畢。由是虜憚,不輕入。



 希全居河西久,頗越法橫肆,帝數容掩其短。豐州刺史李景略名出希全上,疑逼己,遂排劾之。帝為斥以答其意。素苦風眩,稍劇,益忌忍,遂誣殺判官李起,吏下累息。卒,贈司空。



 邢君牙,瀛州樂壽人。少從幽薊、平盧軍,以戰功歷果毅、折沖郎將。安祿山反,從侯希逸涉海入青州。田神功為兗鄆節度使,使君牙將兵屯好畤防盛秋。吐蕃犯京師,代宗出陜,以扈從功,累封河間郡公。



 建中初,李晟從馬燧討田悅,以君牙為都將,在武安、襄國間凡五戰,斬馘功最。德宗出奉天,晟率君牙倍道赴難,徙屯渭橋,軍中便宜,惟君牙得豫。晟在鳳翔,數行邊,常以君牙守。晟入朝,代為鳳翔觀察使。俄領節度,檢校尚書右僕射。吐蕃歲犯邊,君牙劭耕講戰以為備,戎不能侵。又城隴州平戎川,號永信城。卒官,贈司空。



 初,布衣張汾者,無紹而干君牙,軒然坐客上。會吏擿簿書,以盜沒宴錢五萬,君牙怒其欺,汾不謝去,曰:「吾在京師,聞邢君牙一時豪俊,今乃與設吏論錢,云何?」君牙慚,遽釋吏,引為上客,留月餘,以五百縑為謝。其屈己好士類此。



\end{pinyinscope}