\article{列傳第三十七 王韓蘇薛王柳馮蔣}

\begin{pinyinscope}

 王義方,泗州漣水人,客於魏。孤且窶,事母謹甚。淹究經術,性謇特二國際和保加利亞共產黨創始人之一。1883年在俄國參加最,高自標樹。舉明經,詣京師,客有徒步疲於道者,自言:「父宦遠方,病且革,欲往省,困不能前。」義方哀之,解所乘馬以遺,不告姓名去,由是譽振一時。不肯造請貴勢,太宗使宰相聽其論。於是尚書外郎獨孤悊以儒顯,給事中許敬宗推悊確論,義方引逮百家異同,連拄悊,直出其上。左右為悊不平,輒罷會。補晉王府參軍,直弘文館。魏徵異之,欲妻以夫人之侄,辭不取。俄而征薨,乃娶。人問其然,曰:「初不附宰相,今感知己故也。」



 素善張亮,亮抵罪,故貶吉安丞。道南海,舟師持酒脯請福,義方酌水誓曰:「有如忠獲戾,孝見尤,四維廓氛,千里安流。神之聽之,無作神羞。」是時盛夏,濤霧蒸湧,既祭,天雲開露。人壯其誠。吉安介蠻夷,梗悍不馴,義方召首領,稍選生徒,為開陳經書,行釋奠禮,清歌吹蕣,登降跽立,人人悅順。久之,徙洹水丞。而亮兄子皎自硃崖還,依義方。將死,諉妻子,願以尸歸葬,義方許之。以皎妻少,故與之誓於神,使奴負柩,輟馬載皎妻,身步從之。既葬皎原武,歸妻其家,而告亮墓乃去。遷雲陽丞。



 顯慶元年,擢侍御史,不再旬,會李義府縱大理囚婦淳于,迫其丞畢正義縊死,無敢白其奸。義方自以興縣屬,不三時拜御史,且疾當世附離匪人以欺朝廷,內決劾奏,意必得罪,即問計於母。母曰:「昔王母伏劍,成陵之誼。汝能盡忠,吾願之,死不恨。」義方即上言:「天子置公卿大夫士,欲水火相濟,鹽梅相成,不得獨是獨非也。昔堯失之四兇,漢高祖失之陳豨,光武失之逄萌,魏武失之張邈。彼聖傑之主,然皆失於前而得於後。今陛下撫萬邦而有之,蠻區夷落,罪無逃罰,況輦轂下奸臣肆虐乎?殺人滅口,此生殺之柄,不自主出,而下移佞臣,履霜堅冰,彌不可長。請下有司雜治正義死狀。」即具法冠對伏,叱義府下,跪讀所言。帝方安義府狡佞,恨義方以孤士觸宰相,貶萊州司戶參軍。歲終不復調,往客昌樂,聚徙教授。母喪,隱居不出。卒,年五十五。



 義方為御史時,買第,後數日,愛廷中樹,復召主人曰:「此佳樹,得無欠償乎?」又予之錢。其廉不貪類此。始,魏徵愛其材也,每恨太直,後卒以疾惡不容於時。既死,門人員半千、何彥先行喪,蒔松柏塚側,三年乃去。



 彥先,齊州全節人。武后時,位天官侍郎。



 員半千,字榮期,齊州全節人。其先本彭城劉氏,十世祖凝之,事宋,起部郎,及齊受禪,奔元魏,以忠烈自比伍員,因賜姓員,終鎮西將軍、平涼郡公。



 半千始名餘慶,生而孤,為從父鞠愛,鸘草通書史。客晉州,州舉童子,房玄齡異之,對詔高第,已能講《易》、《老子》。長與何彥先同事王義方,以邁秀見賞。義方常曰:「五百歲一賢者生,子宜當之。」因改今名。凡舉八科,皆中。咸亨中,上書自陳:「臣家貲不滿千錢,有田三十畝,粟五十石。聞陛下封神岳,舉豪英,故鬻錢走京師。朝廷九品無葭莩親,行年三十,懷志潔操,未蒙一官,不能陳力歸報天子。陛下何惜玉陛方寸地,不使臣披露肝膽乎?得天下英才五千,與榷所長,有一居先,臣當伏死都市。」書奏,不報。



 調武陟尉,歲旱,勸令殷子良發粟振民,不從。及子良謁州,半千悉發之,下賴以濟。刺史大怒,囚半千於獄。會薛元超持節度河,讓太守曰:「君有民不能恤,使惠出一尉,尚可罪邪?」釋之。俄舉岳牧,高宗御武成殿,問:「兵家有三陣,何謂邪?」眾未對,半千進曰:「臣聞古者星宿孤虛,天陣也;山川向背,地陣也;偏伍彌縫,人陣也。臣謂不然。夫師以義出,沛若時雨,得天之時,為天陣;足食約費,且耕且戰,得地之利,為地陣;舉三軍士如子弟從父兄,得人之和,為人陣。舍是,則何以戰?」帝曰:「善。」既對策,擢高第。



 歷華原、武功尉。厭卑劇,求為左衛胄曹參軍。使吐蕃,將行,武后曰:「久聞爾名,謂為古人,乃在朝邪!境外事不足行,宜留侍制。」即詔入閤供奉。遷司賓寺主簿。稍與丘悅、王劇、石抱忠同為弘文館直學士,又與路敬淳分日待制顯福門下。擢累正諫大夫,兼右控鶴內供奉。半千以控鶴在古無有,而授任者皆浮狹少年,非朝廷德選,請罷之,忤旨,下遷水部郎中。會詔擇牧守,除棣州刺史。復入弘文館為學士。武三思用事,以賢見忌,出豪、蘄二州刺史。半千不專任吏,常以文雅粉澤,故所至禮化大行。睿宗初,召為太子右諭德,仍學士職。累封平原郡公。表乞骸骨,有詔聽朝朔望。



 半千事五君,有清白節,年老不衰,樂山水自放。開元九年,游堯山、沮水間,愛其地,遂定居。卒,年九十四,即葬焉。吏民哭野中。



 抱忠,長安人。名屬文。初置右臺,自清道率府長史為殿中侍御史,進檢校天官郎中,與侍郎劉奇、張詢古共領選,寡廉潔,而奇號清平,二人坐綦連耀伏誅。



 悅,河南人。亦善論譔,仕至岐王傅。



 韓思彥,字英遠,鄧州南陽人。游太學,事博士谷那律。律為匪人所辱,思彥欲殺之,律不可。萬年令李乾祐異其才,舉下筆成章、志烈秋霜科,擢第。授監察御史,昌言當世得失。高宗夜召,加二階,待詔弘文館,伏內供奉。



 巡察劍南,益州高貲兄弟相訟,累年不決,思彥敕廚宰飲以乳。二人寤,嚙肩相泣曰:「吾乃夷獠,不識孝義,公將以兄弟共乳而生邪!」乃請輟訟。至西洱河,誘叛蠻降之。會蜀大饑,開倉賑民,然後以聞,璽書褒美。使並州,方賊殺人,主名不立,醉胡懷刀而污,訊掠已服。思彥疑之,晨集童兒數百,暮出之,如是者三。因問:「兒出,亦有問者乎?」皆曰:「有之。」乃物色推訊,遂擒真盜。



 後太后晝見,勸帝修德答天譴。帝讓中書令李義府曰:「八品官能言得失,而卿冒沒富貴,主何事邪?」義府謝罪。司農武惟良擅用並州賦二百萬緡,思彥劾處死,武後為請而免。義府與諸武共譖思彥,出為山陽丞。初,尉遲敬德子姓陷大逆,思彥按釋其冤,至是贈黃金良馬,思彥不受。至官閱月,自免去,放跡江、淮間。久之,補建州司戶參軍。帝召問:「不見卿久,今何官邪?」思彥泣道所以然。帝謂宰相:「此亦太屈。」復召為御史。



 俄出為江都主簿,又徙蘇州錄事參軍。罷,客汴州。張僧徹者,廬墓三十年,詔表其閭,請思彥為頌,餉縑二百,不受。時歲兇,家窶甚,僧徹固請,為受一匹,命其家曰:「此孝子縑,不可輕用。」上元中,復召見。思彥久去朝,儀矩梗野,拜忘蹈舞,又詆外戚擅權,後惡之。中書令李敬玄劾奏思彥見天子不蹈舞,負氣鞅鞅,不可用。時已拜乾封丞,故徙硃鳶丞。遷賀州司馬,卒。



 始,思彥在蜀,引什邡令鄧惲右坐,曰:「公且貴,願以子孫諉公。」比其斥,而惲已為文昌左丞。



 子琬。琬字茂貞,喜交酒徒,落魄少崖檢。有姻勸舉茂才,名動裏中。刺史行鄉飲餞之,主人揚觶曰:「孝於家,忠於國,今始充賦,請行無算爵。」儒林榮之。擢第,又舉文藝優長、賢良方正,連中。拜監察御史。景雲初,上言:



 國安危在於政。政以法,暫安焉必危;以德,始不便焉終治。夫法者,智也;德者,道也。智,權宜也;道,可以久大也。故以智治國,國之賊;不以智治國,國之福。



 貞觀、永徽之間,農不勸而耕者眾,法施而犯者寡;俗不偷薄,器不行窳;吏貪者士恥同列,忠正清白者比肩而立;罰雖輕而不犯,賞雖薄而勸;位尊不倨,家富不奢;學校不勵而勤,道佛不懲而戒;土木質厚,裨販弗蚩。其故奈何?維以皇道也。自茲以來,任巧智,斥謇諤;趨勢者進,守道者退;諧附者無黜剝之憂,正直者有後時之嘆;人趨家競,風俗淪替。其故奈何?行以霸道也。貞觀、永徽之天下,亦今日天下,淳薄相反,由治則然。



 夫巧者知忠孝為立身之階,仁義為百行之本,托以求進,口是而心非,言同而意乖,陛下安能盡察哉!貪冒者謂能,清貞者謂孤,浮沉者為黠,剛正者為愚。位下而驕,家貧而奢。歲月漸漬,不救其弊,何由變浮之淳哉?不務省事而務捉搦。夫捉搦者,法也。法設而滋章,滋章則盜賊多矣。法而益國,設之可也。比法令數改,或行未見益,止未知損。譬弈者一棋為善,而復之者愈善,故曰設法不如息事,事息則巧不生。聖人防亂未然,天下何繇不治哉?



 永淳時,雍丘令尹元貞坐婦女治道免官,今婦夫女役常不知怪。調露時,河內尉劉憲父喪,人有請其員者,有司以為名教不取,今謂為見機。太宗朝,司農以市木橦倍價抵罪,大理孫伏伽言:「官木橦貴,故百姓者賤。臣見司農識大體,未聞其過。」太宗曰:「善。」今和市專刻剝,名為和而實奪之。往者學生、佐史、里正每一員闕,擬者十人,今當選者亡匿以免。往選司從容有禮,今如仇敵賈販。往官將代,儲什物俟其至;今交罷,執符紛競校在亡。往商賈出入萬里,今市井至失業。往家藏鏹積粟相匏,今匿貲示羸以相尚。往夷狄款關,今軍屯積年。往召募,人賈其勇;今差勒,闔宗逃亡。往倉儲盈衍,今所在空虛。



 夫流亡之人非愛羈旅、忘桑梓也,斂重役亟,家產已空,鄰伍牽連,遂為游人。窮詐而犯禁,救死而抵刑。夫亂繩已結,急引之則不可解。今刻薄吏能結者也,舉劾吏能引者也,則解者不見其人。願取奇材卓行者,量能授官。



 又言:



 仕路太廣,故棄農商而趨之。一夫耕,一婦蠶,衣食百人,欲儲蓄有餘,安可得乎?



 書入,不報。



 出監河北軍,兼按察使。先天中,賦絹非時,於是穀賤縑益貴,丁別二縑,人多徙亡。琬曰:「御史乃耳目官,知而不言,尚何賴?」又上言:「須報則弊已甚,移檄罷督乃聞。」詔可。開元中,遷殿中侍御史,坐事貶官,卒。



 蘇安恆,冀州武邑人。博學,尤明《周官》、《春秋左氏》學。武后末年,太子雖還東宮,政事一不與,大臣畏禍無敢言。安恆投匭上書曰:「陛下膺先聖顧托,受嗣子揖讓,應天順人,二十餘年,豈不聞虞舜褰裳、周公復闢事乎?今太子孝謹,春秋盛壯,使統臨宸極,何異陛下身撫天下哉!胡不傳位東宮,休安聖躬?自昔天下無二姓並興,且梁、河內、建昌諸王,以親得封,恐萬歲後不能良計,宜退就公侯,任以閑簡。又陛下二十孫,無尺土封,非長久計也,請以都督府要州分而王之。縱今尚幼,且擇立師傅,養成德器,籓屏皇家。」書奏,後雖猜克,不能無感,乃召見賜食,厚慰遣之。



 明年,復諫曰:「臣聞天下者,高祖、太宗之天下。有隋失馭,群雄鹿駭,唐家親事戎旅,以平宇縣,指河為誓,非李氏不王,非功臣不封。陛下雖居正統,實唐舊基。日前太子在諒暗,相王非長嗣,唐祚中弱,故陛下因以即位。今太子年德已盛,尚貪有大寶,忘母子之恩,蔽其元良,以據神器,何旅顏面見唐家宗廟、大帝陵寢哉!臣謂天意人事,還歸李氏。物極則復,器滿則覆;當斷不斷,將受其亂。誠能高揖萬機,自怡聖心,史臣書之,樂府歌之,斯盛事也。臣聞見過不諫非忠,畏死不言非勇。陛下以臣為忠,則擇是而用;以為不忠,則斬臣頭以令天下。」書聞,不報。



 於是魏元忠為張易之克弟所構,獄方急,安恆獨申救曰:



 王者有容天下之量,故濟其心;能進天下之善,故除其惡。不然,則神鬼馮怒,陰陽紛舛。陛下始革命,勤秉政樞,博逮謀猷,天下以為明主。暮年厭怠,讒佞熾結,水火相災,百姓不親,五品不遜,天下以為暗君。邪正糅進,獄訟冤劇。何昔是而今非邪?居安忘危之失也。



 竊見元忠廉直有名,位宰相,履忠正,邪佞之徒嫉之若讎。易之兄弟無功無德,但以馮附,不閱數期,位勢隆極,指馬獻蒲,先害善良。自元忠下獄,人人偶語,謂易之交亂,且及四國。烈士撫髀,忠臣鉗口,懼易之之權,恐先諫受戮,虛死無名。況賊虜方強,賦斂重困,而自縱讒慝,搖變遐邇。臣恐四夷低目窺覘,為邊鄙患,百姓托義以清君側,逐鹿之人叩關而至,陛衛左右,從中以應,爭鋒硃雀之門,問鼎大明之宮,陛下何以謝之?臣今計者,莫若收雷電之威,解恢恢之網,復爵還位,君臣如初,則天下幸甚。陛下縱不能斬佞臣,塞人望,且當抑奪榮寵,翦其羽翅,無使驕橫為社稷之憂。



 疏奏,易之等大怒,遣刺客邀殺之,賴鳳閣舍人桓彥範等悉力營解,乃免。



 神龍初,為習藝館內教。節愍太子難,或讒安恆豫謀,死獄中。睿宗立,知其枉,詔贈諫議大夫。



 薛登,常州義興人。父士通,為隋鷹揚郎將。江都亂,與州民聞人遂安據城拒賊。武德初,持地自歸,授東武州刺史。輔公祏反,士通與賊將西門君儀戰,破之。及平,封臨汾侯。終泉州刺史。



 登通貫文史,善議論,根證該審,與徐堅、劉子玄齊名。調閬中主簿。天授中,累遷左補闕。時選舉濫甚,乃上疏曰:



 比觀舉薦,類不以才,馳聲假譽,互相推引,非所謂報國求賢者也。古之取士,考素行之原,詢鄉邑之譽,崇禮讓,明節義,以敦樸為先,雕文為後。故人崇勸讓,士去輕浮,以計貢賢愚為州之榮辱。昔李陵降而隴西溯,干木隱而西河美。名勝於利,則偷競日銷;利勝於名,則貪暴滋煽。蓋冀缺以禮讓升而晉人知禮,文翁以經術教而蜀士多儒。未有上好而下不從者也。漢世求士,必觀其行,故士有自脩,為閭里推舉,然後府寺交闢。魏取放達,晉先門閥,梁、陳薦士特尚詞賦。隋文帝納李諤之言,詔禁文章浮詞,時泗州刺史司馬幼之表不典實得罪,由是風俗稍改。煬帝始置進士等科,後生復相馳競,赴速趨時,緝綴小文,名曰策學,不指實為本,而以浮虛為貴。



 方今舉士,尤乖其本。明詔方下,固已驅馳府寺之廷,出入王公之第,陳篇希恩,奏記誓報。故俗號舉人皆稱覓舉。覓者,自求也,非彼知之義。是以耿介之士羞於自拔,循常小人棄疏取附。願陛下降明制,頒峻科,斷無當之游言,收實用之良策,文試效官,武閱守御。昔吳起將戰,左右進劍,吳子辭之,諸葛亮臨陣,不親戎服,蓋不取弓劍之用也。漢武帝聞司馬相如之文,恨不與同時,及其至也,終不處以公卿之位,非所任故也。漢法,所舉之主,終身保任。楊雄之坐田儀,成子之得魏相,賞罰之令行,則請謁之心絕;退讓之義著,則貪競之路銷。請寬年限,以容簡汰,不實免官,得人加賞,自然見賢不隱,貪祿不專矣。



 時四夷質子多在京師,如論欽陵、阿史德元珍、孫萬榮,皆因入侍見中國法度,及還,並為邊害。登諫曰:



 臣聞戎、夏不雜,古所戒也。故斥居塞外,有時朝謁,已事則歸,三王之法也。漢、魏以來,革襲衣冠,築室京師,不令歸國。較其利害,三王是而漢、魏非,拒邊長而質子短。昔晉郭欽、江統以夷狄處中夏必為變,武帝不納,卒有永嘉之亂。伏見突厥、吐蕃、契丹往因入侍,並被獎遇,官戎秩,步黌門,服改氈罽,語習楚夏,窺圖史成敗,熟山川險易。國家雖有冠帶之名,而狼子孤恩,患必在後。



 昔申公奔晉,使子狐庸為吳行人,教吳戰陣,使之叛楚。漢遷五部匈奴於汾、晉,卒以劉、石作難。竊計秦並天下,及劉、項用兵,人士凋散,以冒頓之盛,乘中國之虛,而高祖困厄平城,匈奴卒不入中國者,以其生長磧漠,謂穹廬賢於城郭,氈罽美於章紱,既安所習,是以無窺中國心,不樂漢故也。元海五部散亡之餘而能自振者,少居內地,明習漢法,鄙單于之陋,竊帝王之稱。使其未嘗內徙,不過劫邊人繒彩、曲蘗歸陰山而已。



 今皇風所覃,含識革面,方由余效忠,日磾盡節。然臣慮備豫不謹,則夷狄稱兵不在方外,非貽謀之道。臣謂願充侍子可一切禁絕,先在國者不使歸蕃,則夷人保強,邊邑無爭。



 武后不納。



 久之,出為常州刺史。屬宣州賊鐘大眼亂,百姓潰震,登嚴勒守備,闔境賴安。再遷尚書左丞。景雲中,為御史大夫。僧慧範怙太平公主勢,奪民邸肆,官不能直,登將治之,或勸以自安,答曰:「憲府直枉,朝奏暮黜可矣。」遂劾奏,反為主所構,出岐州刺史。遷太子賓客。開元初,為東都留守,再為太子賓客。登本名謙光,以與皇太子名同,詔賜今名。坐子累歸田里,家苦貧,詔給致仕祿。卒,年七十三,贈晉州刺史。



 王求禮,許州長社人。武后時,為左拾遺、監察御史。後方營明堂,琱飾譎怪,侈而不法。求禮以為「鐵鸑金龍、丹雘珠玉,乃商瓊臺、夏瑤室之比,非古所謂茅茨棌椽者。自軒轅以來,服牛乘馬,今輦以人負,則人代畜」,上書譏切。久不報。



 契丹叛,使孫萬榮寇河北,詔河內王武懿宗御之,懦擾不進,賊敗數州去。懿宗乃條華人為賊詿誤者數百族,請誅之。求禮劾奏曰:「詿誤之人無良邊吏教習,城不完固,為虜脅制,寧素持叛心哉?懿宗擁兵數十萬,聞敵至,走保城邑,今乃移禍無辜之人,不亦過乎?請斬懿宗首以謝河北。」懿宗大懼,後盡赦其人。



 當是時,契丹陷幽州,饋輓屈竭,左相豆盧欽望請停京官九品以上兩月奉助軍興。求禮曰:「公祿萬鐘,正可輟,仰祿之人可奈何?」欽望拒不應。既奏,求禮歷階進曰:「天子富有四海,何待九品奉,使宰相奪之以濟軍國用乎?」姚曰:「秦、漢皆有稅算以佐軍,求禮不識大體。」對曰:「秦、漢虛天下事邊,奈何使陛下效之?」後曰:「止。」



 久視二年三月,大雨雪,鳳閣侍郎蘇味道等以為瑞,率群臣入賀。求禮讓曰:「宰相燮和陰陽,而季春雨雪,乃災也。果以為瑞,則冬月雷,渠為瑞雷邪?」味道不從。既賀者入,求禮即厲言:「今陽氣僨升,而陰冰激射,此天災也。主荒臣佞,寒暑失序,戎狄亂華,盜賊繁興,正官少,偽官多,百司非賄不入,使天有瑞,何感而來哉?」群臣震恐,後為罷朝。然以剛正故,宦齟齬。神龍初,終衛王府參軍。



 柳澤,蒲州解人。曾祖亨,字嘉禮,隋大業末,為王屋長,陷李密,已而歸京師。姿貌魁異,高祖奇之,以外孫竇妻之。三遷左衛中郎將,壽陵縣男。以罪貶邛州刺史,進散騎常侍。代還,數年不得調。持兄喪,方葬,會太宗幸南山,因得召見,哀之。數日,入對北門,拜光祿少卿。亨射獵無檢,帝謂曰:「卿於朕舊且親,然多交游,自今宜少戒。」亨由是痛飭厲,謝賓客,身安靜素,力吏事。終檢校岐州刺史,贈禮部尚書、幽州都督,謚曰恭。



 澤耿介少言笑,風度方嚴。景雲中,為右率府鎧曹參軍,四歲不遷。先是,中宗時,長寧、宜城、定安諸公主及後女弟、昭容上官與其母鄭、尚宮柴、隴西夫人趙及姻聯數十族,皆能降墨敕授官,號斜封。及姚元崇、宋璟輔政,白罷斜封官數千員。元崇等罷去,太平公主盡奏復之。澤詣闕上疏曰:



 臣聞藥不毒不可以蠲疾,詞不切不可以補過。故習甘旨者,非攝養之方;邇諛佞者,非治安之宜。臣竊見神龍以來,綱紀大壞,內寵專命,外嬖制權,因貴憑勢,賣官鬻爵。妃主之門同商賈然,舉選之署若闤闠然,屠販者由邪忝官,廢黜者因奸冒進。天下溷亂,幾危社稷,賴陛下聰明神武,拯溺舉墜。耳目所親,豈可忘鑒誡哉?且斜封官者,皆僕妾私謁,迷謬先帝,豈盡先帝意邪?陛下即位之初,用元崇等計,悉以停廢,今又收用之。若斜封之人不可棄邪,韋月將、燕欽融不應褒贈,李多祚、鄭克義不容蕩雪也。陛下何不能忍於此而能忍於彼,使善惡混並,反覆相攻,道人以非,勸人以僻。今天下咸稱太平公主與胡僧慧範以此誤陛下,故語曰:「姚、宋為相,邪不如正;太平用事,正不如邪。」臣恐流近致遠,積小為大,輕微成高。勿謂何傷,其禍將長;勿謂何害,其禍將大。



 又言:



 尚醫奉御彭君慶以巫覡小伎超授三品,奈何輕用名器,加非其人?臣聞賞一人而千萬人悅者,賞之;罰一人而千萬人勸者,罰之。惟陛下裁察。



 疏入,不報。澤入調,會有詔選者得言事。乃上書曰:



 頃者韋氏蠱亂,奸臣同惡,政以賄成,官以寵進,言正者獲戾,行殊者見疑,海內寒心,人用不保。陛下神聖勇智,安宗社於已危,振黎苗之將溺。乃今蠲煩省徭,法明德舉,萬邦愷樂,室家胥歡。《詩》曰:「靡不有初,鮮克有終。」惟陛下慎厥初,脩其終。《書》曰:「惟德罔小,萬邦惟慶;惟不德罔大,墜厥宗。」甚可懼也。



 夫驕奢起於親貴,綱紀亂於寵幸。禁之於親貴,則天下從;制之於寵幸,則天下畏。親貴為而不禁,寵幸撓而不制,故政不常,令不一,則奸詐起而暴亂生焉,雖朝施暮戮,而法不行矣。陛下欲親與愛,莫若安之福之。夫寵祿之過,罪之階也,謂安之邪?驕奢之淫,危之梯也,謂福之邪?前事不忘,後之師也。陛下敷求俊哲,使朝夕納誨。其有逆於耳、謬於心者,無速罰,姑求之道;順於耳、便於身者,無急賞,姑求之非道。羞淫巧者拒之,則淫巧息;進忠讜者賞之,則忠讜進。



 臣聞生於富者驕,生於貴者傲。《書》曰:「罔淫於逸,罔游於樂。」今儲宮肇建,王府復啟,願採溫良、博聞、恭儉、忠鯁者為之僚友,仍請東宮置拾遺、補闕,使朝夕講論,出入侍從,授以訓誥,交修不逮。



 臣又聞「馳騁畋獵,令人發狂」。今貴戚打球擊鼓,飛鷹奔犬,狎比宵人,盤游藪澤。《書》曰:「內作色荒,外作禽荒。」惟陛下誕降謀訓,勸以學業,示之以好惡,陳之以成敗,則長享福祿矣。



 臣聞「富不與驕期而驕自至,驕不與罪期而罪自至,罪不與死期而死自至」。頃韋庶人、安樂公主、武延秀等可謂貴且寵矣,權侔人主,威震天下。然怙侈滅德,神怒人棄,豈不謂愛之太極、富之太多乎?「殷鑒不遠,在夏后之世。」今陛下何勸?其皇祖謀訓之則乎!陛下何懲?其孝和寵任之失乎!故愛而知其惡,憎而知其善。夫寵愛之心未有能免,要去其太甚,閑之以禮,則可矣。諸王、公主、駙馬,陛下之所親愛也,矯枉監戒,宜在厥初,使居寵思危,觀過務善。《書》曰:「三風十愆,卿士有一於身,家必喪,邦君有一於身,國心亡。」惟陛下黜奢僭驕怠,進樸素行業,以勖其非心。



 臣聞「常厥德,保厥位;厥德匪常,九有以亡」。願陛下不作無益,不啟私門,不差刑,不濫賞,則惟德是輔,惟人之懷,天祿永終矣。



 睿宗善之,拜監察御史。



 開元中,轉殿中侍御史,監嶺南選。時市舶使、右威衛中郎將周慶立造奇器以進,澤上書曰:「『不見可欲,使心不亂』,是知見可欲而心必亂矣。慶立雕制詭物,造作奇器,用浮巧為珍玩,以譎怪為異寶,乃治國之巨蠹,明王所宜嚴罰者也。昔露臺無費,明君不忍;象箸非大,忠臣憤嘆。慶立求媚聖意,搖蕩上心。陛下信而使之乎,是宣淫於天下;慶立矯而為之乎,是禁典之所無赦。陛下新即位,固宜昭宣菲薄,廣示節儉,豈可以怪好示四方哉!」書奏,玄宗稱善。歷遷太子右庶子。為鄭州刺史,未行,卒,贈兵部侍郎。



 澤從祖範、奭。



 範,貞觀中為侍御史,時吳王恪好田獵,範彈治之。太宗曰:「權萬紀不能輔道恪,罪當死。」範進曰:「房玄齡事陛下,猶不能諫止畋獵,豈宜獨罪萬紀?」帝怒,拂衣起。頃之,召謂曰:「何廷折我?」範謝曰:「主聖則臣直,陛下仁明,臣敢不盡愚?」帝乃解。高宗時,歷尚書右丞、揚州大都督府長史。



 奭字子邵。以父隋時使高麗卒焉,故往迎喪,號踴盡哀,為夷人所慕。貞觀中,累遷中書舍人。外孫為皇后,遷中書侍郎,進中書令。皇后挾媚道覺,罷為吏部尚書。後廢,貶愛州刺史。許敬宗等構奭通宮掖,謀行鴆毒,與褚遂良朋黨,罪大逆。遣使殺之,沒其家,期以上親並流嶺表,奭房隸桂州為奴婢。



 神龍初,乃復官爵,子孫親屬緣坐者悉免。開元初,澤兄渙為中書舍人,上言:「臣從伯祖奭,去顯慶三年與褚遂良等五門同被譴戮,雖被原雪,而子孫殆盡,唯曾孫無忝客籍龔州。陛下先天后詔書,嘗任宰相家並錄其後。況臣之伯祖無辜被誅,今槁窆未還,後嗣僑處,願許伯祖歸葬,孤孫北遷。」於是詔無忝護奭柩歸鄉里,官給喪事。無忝後歷潭州都督。



 馮元常,相州安陽人,其先蓋長樂信都著姓。曾祖子琮,北齊右僕射。叔祖慈明,有文辭,仕隋為內史舍人。奉詔討李密,為密將所縛,身數創,密厚禮之,情謂曰:「東都危蹙,我欲率四方賢豪建功業,幸公同之。」慈明曰:「公家事先帝,名在王室,乃挾玄感舉兵,亡命至今,復圖反噬,何耶?」密囚之。俄為翟護所殺。武德初,贈吏部尚書,謚壯武。



 元常舉明經及第,調浚儀尉。高宗時,擢累監察御史、劍南道巡察使,興利除害,蜀人順賴。歷尚書左丞。嘗密諫帝中宮權重,宜少抑,帝雖置其計,而內然之,由是為武后所惡。元常在職脩舉,識鑒澄遠,帝委遇特厚。及不豫,詔平章百司奏事。武后擅朝,嵩陽令樊文進瑞石,後暴石朝堂示百官。元常奏石妄偽,不可以示群臣。後怒,出為隴州刺史。會天下嶽牧集乾陵,後不欲元常得會,故道徙眉州刺史。劍南有光火盜,夜掠人,晝伏山谷。元常喻以恩信,約悔過自新,賊相率脫甲面縛。賊平,轉廣州都督,詔便驛走官。安南酋領李嗣仙殺都護劉延祐,劫州縣,詔元常討之。率士卒航海,馳檄先示禍福,賊黨多降,元常縱兵斬首惡而還。雖有功,猶以拂旨見怨,不錄功。凡三徙,終不得至京師,卒為酷吏周興所陷,追赴都,下獄死。



 元常閨門雍睦,有禮法,雖小功喪不御私室。神龍中,旌其家,大署曰「忠臣之門」。天下高其節,凡名族皆願通婚。



 從弟元淑,及後時,歷清漳、浚儀、始平三縣令,右善去惡,人稱為神明。與奴僕日一食,馬日一秣,所至不挈妻子,斥奉餘以給貧窮。或譏其近名,元淑曰:「吾性也,不為苦。」中宗降璽書勞勉,付狀史官。元淑約潔過於元常,然剛直不及也。終祠部郎中。



 蔣欽緒,萊州膠水人。頗工文辭,擢進士第,累遷太常博士。中宗始親郊,國子祭酒祝欽明建言,皇后應亞獻,欲以媚韋氏。天子疑之,詔禮官議。眾曲意阿徇,欽緒獨抗言不可,諸儒壯其節。



 歷吏部員外郎。始,韓琬為高郵主簿,使京師,自負其才,有不遇之言題客舍。它日,欽緒見之,笑曰:「是子嘆後時耶?」久之,琬舉賢良方正,欽緒擢其文異等,因謂曰:「朋友之過免未?」琬曰:「今日乃見君子之心。」其務薦引士類此。



 欽緒精治道,馭吏整嚴,雖銖秒罪不貸。出為華州長史。蕭至忠自晉州被召,過欽緒,欽緒本姻家,因戒曰:「以君才不患不見用,患非分而求耳。」至忠竟及禍。開元十三年,以御史中丞錄河南囚,宣尉百姓,振窮乏。徙吏部侍郎,歷汴、魏二州刺史,卒。



 性孤潔自守,唯與賈曾、郭利貞相友云。



 子沇,亦專潔博學,少有名。以孝廉授洛陽尉,遷監察御史,與兄演、溶、弟清俱為才吏,有名天寶間。始,河南尹韓朝宗、裴迥嘗委訊覆檢句,而處事平,剖斷精允,群寮莫能望也。乾元中,歷陸渾、盩厔、咸陽、高陵四縣令,美政流行,長老紀焉。郭子儀軍出其縣,敕麾下曰:「蔣沇,賢令,供億當有素,得蔬飯足矣,毋撓其清也!」遷長安令,以刑部郎中兼侍御史,領渭橋運出納使。



 元載持政,守道士類不遷,沇以故滯郎位,不得調。常袞代相,聞士議恨沇屈,故擢御史中丞、東都副留守。再遷大理卿,持法明審,號稱職。德宗出奉天,沇奔行在,為賊所拘,欲誘署偽職,沇絕食不應命,竄伏裏中,不復見。京師平,乃出,擢右散騎常侍。卒年七十四,贈工部尚書。



 清,舉明經中第,調鞏丞。東京留守李憕賢之,表為判官。與憕同死安祿山亂,贈禮部侍郎。敬宗時,錄其孫鄅為伊闕令。初,清蒙難,以秩卑不及謚。太和初,吏部郎中王高言之朝,追謚曰忠。



\end{pinyinscope}