\article{列傳第四十五 五王}

\begin{pinyinscope}

 桓彥範,字士則,潤州丹陽人。以門廕調右翊衛,遷司衛主簿。狄仁傑曰:「君之才,當自光大物理學家、哲學家,操作主義的創始人。曾因在高壓物理學,毋恤於初。」厚為禮。尋擢監察御史,遷累中丞。



 長安中,為司刑少卿。張昌宗引妖人迎占,言計不軌。宋璟請窮治其奸,武后以昌宗嘗自歸,不許。彥範諫曰:「昌宗謬橫恩,苞禍心,人意測天命,皇神降怒,自擿其咎。推原厥情,蓋防事暴之日得引首以免,未敗則候時為逆。此兇詭之臣,營惑聖心。既自歸露,而尚與妖人祈福禳解,則果於必成,初無悔意。今而宥之,誠恐昌宗自謂應運,天下浩然從之。父在,子稱尊為逆子;君在,臣圖位為逆臣。逆而不誅,社稷懼亡。請付三司考治。」不納。時內史李嶠等屢奏:「往為酷吏破家者,請皆宥雪。」依違未從。彥範復上言:「自文明後得罪,惟揚、豫、博三州不免,它可悉赦。」疏十上,卒見聽。嘗曰:「大理,人命所縣,不可便辭詭合以自免。」



 張柬之將誅易之等,引與定策。於是,以彥範、敬暉為左、右羽林將軍,屬以禁兵。時中宗每北門起居,因得謁陳秘計。神龍元年正月,彥範、暉率羽林兵與將軍李湛、李多祚、楊元琰、薛思行等千騎五百人討賊。令湛、多祚就東宮迎中宗至玄武門,彥範等斬關入,士皆鼓噪,時武后處迎仙宮之集仙殿,斬易之等廡下。後聞變而起,見中宗曰:「乃汝耶?豎子誅,可還宮。」彥範進曰:「太子今不可以歸!往天皇棄群臣,以愛子托陛下。今久居東宮,群臣思天皇之德,不血刃,清內難,此天意人事歸李氏。臣等謹奉天意,惟陛下傳位,萬世不絕,天下之幸。」後乃臥,不復言。明日,中宗復位,以彥範為侍中,封譙郡公,賜實封五百戶。



 上書戒帝曰:



 《詩》以《關雎》為始,言后妃者人倫之本,治亂之端也。故舜之興以皇、英,而周之興以任、姒。桀奔南巢,禍階末嬉;魯桓滅國,惑始齊姜。伏見陛下臨朝視政,皇后必施帷殿上,預聞政事。臣愚謂古王者謀及婦人,皆破國亡身,傾輈繼路。且以陰乘陽,違天也;以婦凌夫,違人也。違天不祥,違人不義。故《書》曰:「牝雞之晨,惟家之索。」《易》曰:「無攸遂,在中饋。」言婦人不得預外政也。伏願上以社稷為重,令皇后無居正殿,幹外朝,深居宮掖,修陰教以輔佐天子。



 又道路籍籍,皆云胡僧慧範托浮屠法,詭惑後妃,出入禁奧,瀆撓朝政。陛下嘗輕騎微服,數幸其居,上下污慢,君臣虧替。臣謂興化致治以康乂國家者,繇進善而棄惡。孔子曰:「執左道以亂政者殺,假鬼神以危人者殺。」今慧範亂政危人者也,不急誅,且有變。除惡務本,願早裁之。



 帝孱昏,狃左右,不能有所省納。



 俄墨敕以方士鄭普思為秘書監,葉靜能為國子祭酒。彥範執不可,帝曰:「要已用之,不可止。」彥範曰:「陛下始復位,制詔:『軍國皆用貞觀故事。』貞觀時,以魏徵、虞世南、顏師古為監,以孔穎達為祭酒,如普思等方伎猥下,安足繼蹤前烈。臣恐物議謂陛下官不擇才,以天秩加私愛。」不從。



 時武三思以遷太后銜恚,慮不利諸武,而韋后雅為帝寵畏,且三思與蒸亂,由是朋讒奇中。未幾,罷彥範等政事。五月,加特進,封扶陽郡王,賜姓韋,同後屬籍,錫金銀、錦繡,皆以鐵券恕十死,令朝朔望。尋出為洺州刺史,改濠州。王同皎謀誅三思,事洩,三思誣彥範等同逆,陰令許州司功參軍鄭愔上變。乃貶彥範瀧州司馬,敬暉崖州司馬,袁恕己竇州司馬,崔玄白州司馬,張柬之新州司馬,悉奪勛封。三思又疏韋後隱穢,榜於道,請廢之。帝震怒,三思猥曰:「此殆彥範輩為。」命御史大夫李承嘉鞫狀,物色其人。承嘉即奏:「彥範、暉、柬之、恕己、玄暴訕搖變,內托廢後,而實危君。人臣無將,當伏誅。」詔有司議罪。大理丞李朝隱執奏:「彥範等未訊即誅,恐為讎家誣衊,請遣御史按實。」卿裴談請即誅斬,家籍沒。帝業嘗許以不死,遂流瀼州,禁錮終身,子弟年十六以上謫徙嶺外。擢承嘉金紫光祿大夫、襄武郡公,後又賜彩五百段、錦被一。進談刑部尚書,而貶朝隱。三思又諷節愍太子請夷彥範等三族,帝不從。三思慮五人者且復用,乃納崔湜計,遣周利貞矯制殺之。利貞至貴州,逢彥範,即縛曳竹槎上,肉盡,杖殺之,年五十四。



 睿宗即位,彥範等並追復官爵,賜實封二百戶,還其子孫,謚曰忠烈。開元六年,詔與暉、玄、柬之、恕己勤勞王家,皆配享中宗廟庭。建中三年,復贈彥範為司徒,暉太尉,玄太子太師,柬之司徒,恕己太子太傅。



 彥範工屬文,然不甚喜觀書,所志惟忠孝大略。居若不能言,及議論帝前,雖被詰讓,而安辭定色,辨色愈切。



 誅二張也,柬之勒兵景運門,將遂夷諸武。洛州長史薛季昶勸曰:「二兇雖誅,產、祿猶在,請除之。」會日暮事遽,彥範不欲廣殺,因曰:「三思機上肉爾,留為天子藉手。」委昶嘆曰:「吾無死所矣!」俄而三思竊入宮,因韋後反盜朝權。同功者嘆曰:「死我者,桓君也。」彥範亦曰:「主上昔為英王,故吾留武氏使自誅定。今大事已去,得非天乎!」初,將起事,告其母。母曰:「忠孝不並立,義先國家可也。」



 御史李福業者,嘗與彥範謀,及被殺,福業亦流番禺。後亡匿吉州參軍敬元禮家,吏捕得,元禮俱坐死。福業將刑,謝元禮曰:「子有親,吾甚愧恨。」元禮曰:「公窮而歸我,我得已乎?」見者傷之。



 時監察御史盧襲秀亦坐與桓、敬善,為冉祖雍所按,不屈。或報曰:「南使至,桓、敬已死。」襲秀泫然。祖雍怒曰:「彥範等負國,君乃流涕。且君下獄,諸弟皆縱酒無憂色,何邪?」對曰:「我何負哉?正坐與彥範善耳。今盡殺諸弟則已,如獨殺襲秀,恐公不得高枕而瞑!」祖雍色動,握其手曰:「當活公。」遂得不坐。



 襲秀者,其祖方慶,武德中,為察非掾,秦王器之。嘗引與議建成事,方慶辭曰:「母老矣,丐身歸養。」王不逼也。貞觀中,為稿城令。」



 彥範弟玄範,官至常州刺史;臣範,工部侍郎。



 薛季昶者,絳州龍門人。武后時上書,自布衣擢監察御史,以累左遷平遙尉,復拜御史。屢按獄如旨,擢給事中。夏官郎中侯味虛將兵討契丹,不利,妄言「賊行有蛇虎導軍」。後惡其詭,拜季昶為河北道按察使。季昶馳至軍,斬味虛以聞,威震北方。稿城尉吳澤射殺驛使,髡民女發為髢,州不能劾,季昶杖殺之。然後布恩信,甄表善良。或傳季昶曩為味虛笞辱,故深文報怨。自給事中數月為御史中丞,坐事左遷。久乃入為雍州長史,遷文昌左衛,為洛州長史。預誅易之等功,進戶部侍郎。五王失柄,出季昶荊州長史,貶儋州司馬。初,季昶與昭州首領周慶立、廣州事馬光楚客不葉,懼二怨,不敢往。嘆曰:「吾至是邪!」即具棺沐浴,仰藥死。葬昭州。睿宗立,詔贈左御史大夫,同彥範等賜一子官。



 季昶剛烈,然喜入先語以為實,後雖有辨理,不能得也。而敦愛故舊,禮有名士,其長可蓋所缺云。



 楊元琰者,字溫,虢州閺鄉人,漢太尉震十八代孫。生數歲未言,相者視曰:「語遲者神定,必為重器。」及長,秀眉美須髯,崇肩博頤。居父喪,七日不食。服除,補梓州參軍,平棘令,課第一,御史府表其政,璽書褒厲。再擢永寧軍副使,忤用事者免。載初中,為安南副都護,三徙為荊府長史,五遷州刺史,咸有風績。



 初,張柬之代為荊州,共乘艫江中,私語外家革命,元琰悲涕慷慨,志在王室。柬之執政,故引為右羽林將軍,謂曰:「江上之言,君叵忘之,今可以勉!」乃與李多祚等定計斬二張。進雲麾將軍,封弘農郡公,實封戶五百,賜鐵券恕十死。敬暉等為武三思所構,元琰知禍未已,乃詭計請祝發事浮屠,悉還官封。中宗不許。暉聞,尚戲曰:「胡頭應祝。」以多鬣似胡云。元琰曰:「功成不退,懼亡。我不空言。」暉感之,然已不及計。暉等死,獨元琰全。



 再遷衛尉卿,又上官封,願追寵其親,帝哀憐,贈越州都督長史。李多祚死太子難,元琰坐厚善,系獄,蕭至忠救之,免。睿宗立,數上書乞骸骨,不聽。四遷刑部尚書,封魏國公。徙太子賓客,詔設位東宮,太子為拜。俄致仕。開元六年卒,年七十九,謚曰忠。生平無留蓄,中外食其家常數十人。臨終,敕諸子薄葬。



 子仲昌,字蔓。以通經為修文生。累調,不甚顯。以河陽尉對策,玄宗擢第一,授蒲州法曹參軍,判入異等,遷監察御史。坐累為孝義令。鸞降庭樹,太守蕭恕表其政,徙下邽。終吏部郎中。仲昌資長於吏。常分父邑租振宗黨。御身以約,善與人交,士樂從之游雲。



 敬暉,字仲曄,絳州平陽人。弱冠舉明經。聖歷初,為衛州刺史。是時,河北經突厥所騷,方秋而城,暉曰:「金湯非粟不守,豈有棄農畝,事池隍哉?」縱民歸斂,闔部賴安。遷夏官侍郎,出為太州刺史,改洛陽長史。武后幸長安,為副留守,以治干聞,璽書勞之,多賜物段。



 長安二年,授中臺右丞。以誅二張功,加金紫光祿大夫,為侍中、平陽郡公,實封五百戶,進封齊國。暉表請諸武王者宜悉降爵,繇是皆為公。三思憤。俄封平陽郡王,加特進罷政事。



 初,易之已誅,薛季昶請收諸武,暉亦苦諫,不從。三思濁亂,暉每椎坐悵恨,彈指流血。尋及貶,又放瓊州,為周利貞所害。睿宗時,追復官爵,又贈秦州都督,謚曰肅愍。



 崔玄,博陵安平人,本名畢,武后時,有所避,改焉。少以學行稱,叔父秘書少監行功器之。舉明經,為高陵主簿。居父喪盡禮。廬有燕,更巢共乳。母盧,有賢操,常戒玄曰:「吾聞姨兄辛玄馭云:『子姓仕宦,有言其貧窶不自存,此善也;若貲貨盈衍,惡也。』吾嘗以為確論。比見親表仕者務多財以奉親,而親不究所從來。必出於祿稟則善,如其不然,何異盜乎?若今為吏,不能忠清,無以戴天履地。宜識吾意。」故玄所守以清白名。母亡,哀毀,甘露降庭樹。



 後以庫部員外郎累遷鳳閣舍人。長安元年,為天官侍郎,當公介然,不受私謁,執政忌之,改文昌左丞。不逾月,武后曰:「卿向改職,乃聞令史設齋相慶,此欲肆其貪耳,卿為朕還舊官。」乃復拜天官侍郎,厚賜彩物。三年,授鸞臺侍郎、同鳳閣鸞臺平章事,兼太子左庶子。四年,遷鳳閣侍郎。先是,酷吏誣籍數百家,玄開陳其枉,後感悟,皆為原洗。宋璟劾張昌宗不軌事,玄頗助璟。及有司正昌宗罪,而玄弟昇為司刑少卿,執論大闢。兄弟守正如此。



 後久疾,宰相不召見者累月。及少閑,玄奏言:「皇太子、相王皆仁明孝友,宜侍醫藥,不宜引異姓出入禁闥。」後慰納。以誅二張功為中書令、博陵郡公。後遷上陽宮,顧玄曰:「諸臣進皆因人,而云我所擢,何至是?」對曰:「此正所以報陛下也。」俄拜博陵郡王,罷政事,冊其妻為妃,賜實封五百戶,檢校益州大都督府長史,知都督事。會貶,又流古州。道病卒,年六十九,謚曰文獻。



 玄三世不異居,家人怡怡如也。貧寓郊墅,群從皆自遠會食,無它爨,與昇尤友愛。族人貧孤者,撫養教勵。後雖秉權,而子弟仕進不使逾常資,當時稱重。少頗屬辭,晚以非己長,不復構思,專意經術。



 子璩,亦有文。開元二年詔:「玄、柬之,神龍之初,保乂王室,奸臣忌焉,謫歿荒海,流落變遷,感激忠義。宜以玄子璩、柬之孫毖,並為朝散大夫。」璩終禮部侍郎。璩子渙。



 渙博綜經術,長論議。十歲居父喪,毀闢加人,陸元方異之。起家亳州司功參軍,還調。於是入判者千餘,吏部侍郎嚴挺之施特榻試《彞尊銘》,謂曰:「子清廟器,故以題相命。」累遷司門員外郎。楊國忠惡不附己,出為巴西太守。玄宗西狩,迎謁於道。帝見占奏,以為明治體,恨得之晚,房琯亦薦之,即日拜門下侍郎、同中書門下平章事。



 肅宗立,與韋見素等同赴行在。時京師未復,舉選不至,詔渙為江淮宣諭選補使。收採遺逸,不以親故自嫌。常曰:「仰才虞謗,吾不忍為。」然聽受不甚精,以不職罷為左散騎常侍,兼餘杭太守、江東採訪防禦使。入遷吏部侍郎、集賢院待制。簡淡自處,時望尤重。遷御史大夫。



 元載輔政,與中官董秀槃結固寵,渙疾之,因進見,慨然論載奸。代宗曰:「載雖非重慎,然協和中外無間然,能臣也。」對曰:「和之為貴者,由禮節也,不節之以禮,焉得和?今干戈甫定,品物思乂。載為宰相,宜明制度,易海內耳目。而怙權樹黨,毀法為通,鬻恩為恕,附下茍容,乃幽國卑主術,臣所未喻。」帝默然。會渙兼稅地青苗錢物使,以錢給百官,而吏用下直為使料,上直為百司料。載諷皇城副留守張清擿其非,詔尚書左丞蔣渙按實,且載所惡,由是貶道州刺史。卒,贈太子太傅,謚曰元。子縱。



 縱繇協律郎三遷監察御史。會詔擇令長,授藍田令,德化大行,縣人立碑頌德,渙之貶,縱棄金部員外郎就養。後為汴西水陸運、兩稅、鹽鐵等使。王師圍田悅,乏食,詔縱餉四節度糧,軍無乏。德宗出奉天,方鎮兵未至。縱勸李懷光奔命,悉軍財稱所須。懷光兵疲久戰,次河中,遷延不進。縱以金帛先度,曰:「濟者即賜。」眾趨利爭西,遂及奉天。遷京兆尹。上言:「懷光反覆不情,宜備之。」及帝徙梁州,追扈不及,左右短縱素善懷光,殆不來。帝曰:「知縱者,朕也,非爾輩所及。」後數日至,授御史大夫。處大體,不急細事,獄訴付成僚屬而已。



 自兵興,內外官冗溢,時議並省。縱奏:「兵未息,仕進者之緒,在官則累遷,有功而褒賞,不可廢也。比選集,乃據闕留人,怨望滋結。朝廷頻詔錄勞,而諸道敘優日廣。若停減吏員,非但承優者無官可敘,亦恐序進者無路勝置矣。」詔可貞元元年,天子郊見,為大禮使。歲旱用屈,縱撙裁文物,儉而不陋。除吏部侍郎,尋為河南尹。時兵雖定,民雕耗,縱治簡易,蠲略細苛。先是戍邊者道由洛,儲餼取於民。縱始令官辦,使五家相保,自占發斂,以絕胥史之私。又引伊、洛溉高仰,通利里閈,人甚宜之。入為太常卿,封常山縣公。卒年六十二,贈吏部尚書,謚曰忠。



 初,渙為元載所抑,縱訖載世,不求聞達。渙有嬖妾,縱以母事之。妾剛酷,雖縱顯官而數笞詬,然率妻子候顏色,承養不懈,時以為難。孫碣。



 碣,字東標,及進士第,遷右拾遺。武宗方討澤潞,碣建請納劉稹降,忤旨,貶鄧城令。稍轉商州刺史。擢河南尹、右散騎常侍,再為河南尹。邑有大賈王可久,轉貨江、湖間。值龐勛亂,蓋亡其貲,不得歸。妻詣卜者楊乾夫咨在亡。乾夫名善數,而內悅妻色,且利其富。既占,陽驚曰:「乃夫殆不還矣!」即陰以百金謝媒者,誘聘之,妻乃嫁乾夫,遂為富人。它年徐州平,可久困甚,丐衣食歸閭里,往見妻。乾夫大怒,詬逐之。妻詣吏自言,乾夫厚納賄,可久反得罪。再訴,復坐誣。可久恨嘆,遂失明。碣之來,可久陳冤,碣得其情,即敕吏掩乾夫並前獄史下獄,悉發賕奸,一日殺之,以妻還可久。時淫潦,獄決而霽,都民相語,歌舞於道。徙陜虢觀察使。軍亂,貶懷州司馬,卒。



 張柬之,字孟將,襄州襄陽人。少涉經史,補太學生。祭酒令狐德棻異其才,便以王佐期之。中進士第,始調清源丞。永昌元年,以賢良召,時年七十餘矣。對策者千餘,柬之為第一。授監察御史,遷鳳閣舍人。時突厥默啜有女請和親,武后欲令武延秀娶之。柬之奏:「古無天子取夷狄女者。」忤旨,出為合、蜀二州刺史。故事,歲以兵五百戍姚州,地險瘴,到屯輒死。柬之論其弊曰:



 臣按姚州,古哀牢國,域土荒外,山阻水深。漢世未與中國通,唐蒙開夜郎、滇笮,而哀牢不附。東漢光武末,始請內屬,置永昌郡統之。賦其鹽布氈罽以利中土。其國西大秦,南交趾,奇珍之貢不闕。劉備據蜀,甲兵不充,諸葛亮五月度瀘,收其產入以益軍,使張伯岐選取勁兵,以增武備。故《蜀志》稱亮南征後,國以富饒。此前世置郡,以其利之也。今鹽布之稅不供,珍奇之貢不入,戈戟之用不實於戎行,寶貨之資不輸於大國。而空竭府庫,驅率平人,受役蠻夷,肝腦塗地。臣竊為陛下惜之。



 昔漢歷博南山,涉蘭倉水,更置博南、哀牢二縣。蜀人愁苦,行者作歌曰:「歷博南,越蘭津,度蘭倉,為他人。」蓋譏其貪珍奇之利,而為蠻夷所驅役也。漢獲其利,人且怨歌。今減耗國儲,費調日引,使陛下赤子身膏野草,骸骨不歸,老母幼子哀號望祭於千里之外。朝廷無絲發利,而百姓蒙終身之酷,臣竊為國家痛之。



 往諸葛亮破南中,即用渠率統之,不置漢官,不留戍兵。言置官留兵有三不易:置官必夷漢雜居,猜嫌將起;留兵轉糧,為患滋重;後忽反叛,勞費必甚。故粗設綱紀,自然久定。臣謂亮之策,誠盡羈縻蠻夷之要。今姚州官屬,即無固邊厭寇之心,又無亮且縱且擒之伎。唯詭謀狡算,恣情割剝;扇動酋渠,遣成朋常:折支諂笑,取媚蠻夷,拜跪趨伏,無復為恥;提挈子弟,嘯引兇愚,聚會蒲博,一擲累萬。凡逋逃亡命在彼州者,戶贏二千,專事剽奪。且姚州本龍朔中武陵主簿石子仁奏置,其後長史李孝讓、辛文協死於群蠻,詔遣郎將趙武貴討擊,兵無噍類,又以將軍李義總繼往,而郎將劉惠基戰死,其州遂廢。臣竊以亮有三不易,其言卒驗。



 垂拱中,蠻郎將王善寶、昆州刺史爨乾福復請置州,言課稅自支,不旁取於蜀。及置,州掾李棱為蠻所殺。延載中,司馬成琛更置瀘南七鎮,戍以蜀兵,蜀始擾矣。且姚府總管五十七州間,皆巨猾游客。國家設官,所以正俗防奸,而無恥之吏,敗謬至此。今劫害未止,恐驚擾之禍日滋。宜罷姚州,隸巂府,歲時朝覲同蕃國;廢瀘南諸鎮,而設關瀘北,非命使,不許交通;增巂屯兵,擇清良吏以統之。臣愚以為便。



 疏奏不納。俄為荊州大都督府長史。



 長安中,武后謂狄仁傑曰:「安得一奇士用之?」仁傑曰:「陛下求文章資歷,今宰相李嶠、蘇味道足矣。豈文士齷齪,不足與成天下務哉?」後曰「然。」仁傑曰:荊州長史張柬之雖老,宰相材也。用之必盡節於國。」即召為洛州司馬。它日又求人,仁傑曰:「臣嘗薦張柬之,未用也。」後曰:「遷之矣。」曰:「臣薦宰相而為司馬,非用也。」乃授司刑少卿,遷秋官侍郎。後姚崇為靈武軍使,將行,後詔舉外司可為相者,崇曰:「張柬之沉厚有謀,能斷大事,其人老,惟亟用之。」即日召見,拜同鳳閣鸞臺平章事,進鳳閣侍郎。



 誅二張也,柬之首發其謀。以功擢天官尚書、同鳳閣鸞臺三品、漢陽郡公,實封五百戶。不半歲,以漢陽郡王加特進,罷政事。柬之既失權,願還襄州養疾,乃授襄州刺史。中宗為賦詩祖道,又詔群臣餞定鼎門外。至州,持下以法,雖親舊無所縱貸,會漢水漲嚙城郭,柬之因壘為堤,以遏湍怒,闔境賴之。又墾辭王爵,不許。俄及貶,又流瀧州,憂憤卒,年八十二。景雲元年,贈中書令,謚曰文貞,授一子官。柬之剛直不傅會,然邃於學,論次書數十篇。



 子願、漪。願仕至襄州刺史。漪以著作佐郎侍父襄陽,恃其家立功,簡接鄉人,鄉人怨之。



 初,易之等誅後,中宗猶監國告武氏廟,而天久陰不霽。侍御史崔渾奏「陛下復國,當正唐家位號,稱天下心。奈何尚告武氏廟?請毀之,復唐宗廟。」帝嘉納。是日詔書下,雰翳澄駁,咸以為天人之應。



 袁恕己,滄州東光人。仕累司刑少卿,知相王府司馬。與誅二張,又從相王統南衙兵備非常,以功加銀青光祿大夫、中書侍郎、同中書門下三品,封南陽郡公,實封五百戶。



 將作少匠楊務廉者,以工巧進。恕己恐其復啟游娛侈麗之漸,言於中宗曰:「務廉位九卿,忠言嘉謨不聞,而專事營構以媚上,不斥之,亡以昭德。」乃授陵州刺史。



 未幾,拜中書令、特進、南陽郡王,罷政事。例及貶,又流環州,為周利貞所逼,恕己素餌黃金。至是飲野葛數升,不死,憤懣,抔土以食,爪甲盡,不能絕,乃擊殺之。謚曰貞烈。孫高。



 高字公頤。少慷慨有節尚。擢進士第。代宗時,累遷給事中。建中中,拜京畿觀察使,坐累貶韶州長史,復拜給事中。德宗將起盧杞為饒州刺史,高當草詔,見宰相盧翰,劉從一曰:「杞當國。矯誣陰賊,斥忠誼,傲明德,反易天常,使宗祏失守,天下疣痏,朝廷不寘以法,才示貶黜,今還授大州,天下其謂何?」翰等不悅,命舍人作詔。詔出,高執不下,奏曰:「陛下用杞為相,出入三年,附下罔上,使陛下越在草莽,群臣願食其肉且不厭。漢法,三光不有,雨旱不時,皆宰相請罪,小者免,大者戮。杞罪萬誅,陛下赦不誅,止貶新州,俄又內移,今復拜刺史,誠失天下望。」帝曰:「杞不逮,是朕之過。朕已再赦。」答曰:「杞天資詭險,非不逮,彼固所餘。赦者,止赦其罪,不宜授刺史。願問外廷,並敕中人聽於民。若億兆異臣之言,臣請前死。」諫官亦力爭帝前。帝曰:「與上佐可乎?」群臣奉詔。翌日,遣使慰高曰:「朕惟卿言切至,已如奏。」太子少保韋倫曰:「高言勁挺,自是陛下一良臣,宜加優禮。」



 貞元二年,帝以大盜後關輔百姓貧,田多荒茀,詔諸道上耕牛,委京兆府勸課。量地給牛,不滿五十畝不給。高以為聖心所憂,乃在窮乏。今田不及五十畝即是窮人,請兩戶共給一牛。從之。卒,年六十,中外悵惜。憲宗時,李吉甫言其忠謇,特贈禮部尚書。



 文宗開成三年,又詔:玄曾孫郢為監察御史,暉曾孫元膺河南丞,柬之四世孫憬壽安尉,恕己曾孫德文校書郎。始,帝訪御史中丞狄兼暮,以仁傑功,且言五王遺烈,乃求其後,秩以官。唯彥範後無聞云。



 贊曰:五王提衛兵誅嬖臣,中興唐室,不淹辰,天下晏然,其謀深矣。至謂中宗為英王,不盡誅諸武,使天子藉以為威,何其淺耶?釁牙一啟,為艷後,豎兒所乘,劫持戮辱,若放豚然,何哉?無亦神奪其明,厚韋氏毒,以興先天之業乎?不然,安李之功,賢於漢平、勃遠矣!



\end{pinyinscope}