\article{列傳第五十 蘇張}

\begin{pinyinscope}

 蘇瑰,字昌容,雍州武功人,隋尚書僕射威之曾孫。擢進士第,補恆州參軍。居母喪士一方面保存了傳統文化,同時也受傳統的束縛,有是古非,哀毀加人,左庶子張大安表舉孝悌,擢豫王府錄事參軍,歷朗、歙二州刺史。



 時來俊臣貶州參軍,人懼復用,多致書請瑰,瑰叱其使曰:「吾忝州牧,高下自有體,能過待小人乎?」遂不發書。俊臣未至追還,恨之。由是連外徙,不得入。久之,轉揚州大都督府長史。州據都會,多名珍怪產,前長史張潛、於辯機貲取鉅萬,瑰單身襆被自將。徙同州刺史。



 歲旱,兵當番上者不能赴。瑰奏:「宿衛不可闕,宜月賜增半糧,俾相給足,則不闕番。又宜卻進獻,罷營造不急者。」不見省。時十道使括天下亡戶,初不立籍,人畏搜括,即流入比縣旁州,更相庾蔽。瑰請罷十道使,專責州縣,豫立簿注,天下同日閱正,盡一月止,使柅奸匿,歲一括實,檢制租調,以免勞弊。武后鑄浮屠,立廟塔,役無虛歲。瑰以為「縻損浩廣,雖不出國用,要自民產日殫。百姓不足,君孰與足?天下僧尼濫偽相半,請並寺,著僧常員數,缺則補。」後善其言。



 神龍初,入為尚書右丞,封懷縣男。瑰明曉法令,多識臺省舊章,一朝格式,皆所刪正。再遷戶部尚書,拜侍中,留守京師。



 中宗復政,鄭普思以妖幻位秘書員外監,支黨遍岐、隴間,相煽訹為亂。瑰捕系普思窮訊,普思妻以左道得幸韋後,出入禁中,有詔勿治。瑰廷爭不可,帝猶依違。司直範獻忠,瑰使按普思者,進曰:「瑰為大臣,不能前誅逆豎而報天子,罪大矣,臣請先斬瑰。」於是,僕射魏元忠頓首曰:「瑰長者,用刑不枉,普思法當死。」帝不得已,流普思於儋州,餘黨論死。累拜尚書右僕射、同中書門下三品,進封許國公。



 帝南郊,國子祭酒祝欽明建白皇后為亞獻,安樂公主為終獻。瑰以為非禮,帝前折愧之。帝昏懦,不能從。時大臣初拜官,獻食天子,名曰「燒尾」,瑰獨不進。及侍宴,宗晉卿嘲之,帝默然。瑰自解於帝曰:「宰相燮和陰陽,代天治物。今粒食踴貴,百姓不足,衛兵至三日不食,臣誠不稱職,不敢燒尾。」帝崩,遺詔皇太后臨朝,相王以太尉輔政。後召宰相韋安石、韋巨源、蕭至忠、宗楚客、紀處訥、韋溫、李嶠、韋嗣立、唐休璟、趙彥昭洎瑰議禁中。楚客猥曰:「太后臨朝,相王有不通問之嫌,不宜輔政。」瑰正色曰:「遺制乃先帝意,安得輒改?」楚客等恕,卒削相王輔政事,瑰稱疾不朝。是月,韋氏敗,睿宗即位,進左僕射。



 景雲元年,老病,罷為太子少傅。卒,年七十二,贈司空荊州大都督,謚曰文貞。皇太子別次發哀。遺令薄葬,布車一乘。



 瑰治州考課常最,為宰相,陳當世病利甚多。韋溫始為汴洲司倉參軍,以賕被杖,及用事,憚瑰正,卒不敢傷。開元二年,賜其家實封百戶,長子頲固辭,乃擢中子乂左補闕。六年,詔與劉幽求配享睿宗廟廷。文宗大和中,錄舊德,官其四代孫翔。



 瑰諸子,頲、詵顯。



 頲,字廷碩,弱敏悟,一覽至千言,輒覆誦。第進士,調烏程尉。武后封嵩高,舉賢良方正異等,除左司禦率府胄曹參軍。吏部侍郎馬載曰:「古稱一日千里,蘇生是已。」再遷監察御史。長安中,詔覆來俊臣等冤獄,頲驗發其誣,多從洗宥。遷給事中、修文館學士,拜中書舍人。時瑰同中書門下三品,父子同在禁筦,朝廷榮之。



 玄宗平內難,書詔填委,獨頲在太極後筦,口所占授,功狀百緒,輕重無所差。書史白曰:「丐公徐之,不然,手腕脫矣。」中書令李嶠曰:「舍人思若湧泉,吾所不及。」遷太常少卿,仍知制誥。遭父喪,起為工部侍郎,辭不拜,終制乃就職。帝問宰相:「有自工部侍郎得中書侍郎乎?」對曰:「陛下任賢惟所命,何資之計?」乃詔以頲為中書侍郎。帝勞曰:「方美官缺,每欲用卿,然宰相議遂無及者,朕為卿恨。陸象先歿,紫微侍郎未嘗補,朕思其人無易卿者。」頲頓首謝。明日加知制誥,給政事食,給食自頲始。時李軿對掌書命,帝曰:「前世李嶠、蘇味道文擅當時,號「蘇李」。今朕得頲及軿,何愧前人哉!」俄襲封許國公。



 吐蕃盜邊,諸將數敗,虜益張,秣騎內侵。帝怒,欲自將兵討之。頲諫曰:「古稱荒服,取荒忽之義,非常奉職貢也。故來則拒,去則勿逐,以禽獸畜之,羈縻御之。譬若獵然,羽毛不入服用,體肉不登郊廟,則王者不射也。況萬乘之重,與犬羊蚊虻語負勝哉?遠夷左衽,不足以辱天子,亦可見矣。雖然,兵法先聲後實,陛下姑班親征之詔,而敕虓將謀夫投會濟師,則吐蕃不日崩破,亦無待躬致天討也。臣謂岐、隴凋弊積年,若千乘萬騎,供億不涯,誠恐徭役內興,寇掠外虞,斯人不堪,一也。戎虜之性,驟往倏來,敗不恥奔,勝不讓成。若大軍一臨邊,怖震鳥散,彼出多方,我受其誤,二也。太上皇聞陛下身對寇場,不能無憂,烝烝之思,何以自安?三也。漢蒯成侯諫高帝曰:「上嘗自勞,豈謂無人使哉?」高帝以為愛我。今將相大臣,豈無為陛下宣力者,何親行之遽邪?」不省。



 復上言:「王者之師,有征無戰,籓貢或闕,王命征之,於是乎治兵其郊,獲辭而止,非謂按甲自臨。敵人畏之莫敢戰也。古天子無親將,惟黃帝五十二戰,當未平之時。自阪泉功成,則修身閑居,無為無事。陛下撥定禍亂,方當深視高居,制禮作樂,禪梁父,登空同,何至厭天居,衽金革,為一日之敵?今吐蕃遣渠領干犯國令,軍吏一不勝,而陛下屈至尊為之敵,雖朝鼎夕砧,猶未可以誇四夷,安足勞聖躬哉?虜之入,唯盜羊馬,發窖裭衣,未嘗殺略邊人,其罪易原也。臣恐虜情狼顧,牽連北狄,聞六師之行,入幽、並,犯靈、夏,南動京師,太上皇一致憂勞,是陛下以天下之安,不能寧其親也。臣固曰,居中制勝,策之上者。若夫擇良將,募重而約嚴,違律必誅,殺敵必賞,多出金以購酋長,虜亡無日矣。願稍遷延,以須西音。」亦會薛訥大破吐蕃,俘獲不貲,由是帝止不行。



 時詔立靖陵碑,命頲為之詞,辭曰:「前世帝後不志碑,事弗稽古,謂之不法。審當可者,祖宗諸陵,一須營立,後嗣謂何?」帝不納其言。



 開元四年,進同紫微黃門平章事,修國史,與宋璟同當國。璟剛正,多所裁決,頲能推其長。在帝前敷奏,璟有未及,或少屈,頲輒助成之,有不會意,頲更申璟所執,故帝未嘗不從,二人相得歡甚。璟嘗曰:「吾與蘇氏父子同為宰相,僕射長厚,自是國器;若獻可替否,事至即斷,盡公不顧私,則今丞相為過之。」



 八年,罷為禮部尚書。俄檢校益州大都督長史,按察節度劍南諸州。時蜀彫攰,人流亡,詔頲收劍南山澤鹽鐵自贍。頲尚簡靜,重興力役,即募戌人,輸雇直,開井置爐,量入計出,分所贏市穀,以廣見糧。時前司馬皇甫恂使蜀,檄取庫錢市錦半臂、琵琶捍撥、玲瓏鞭,頲不肯予,因上言:「遣使銜命,先取不急,非陛下以山澤贍軍費意。」或謂頲:「公在遠,叵得忤上意。」頲曰:「不然。明主不以私愛奪至公,吾可以遠近廢忠臣節邪?」巂州蠻苴院與吐蕃連謀入寇,獲諜者,吏請討之,頲不聽,移書還其諜曰:「毋得爾。」苴院羞悔,不敢侵邊。



 從封泰山,詔頌朝覲壇,世咨其文。還,分主十銓事。卒,年五十八。帝猶視朝,起居舍人韋述上疏曰:「貞觀、永徽時,大臣薨,輒置朝舉哀,成終始恩,上有旌賢錄舊之德,下有生榮死哀之美。昔晉知悼子卒,平公宴樂,杜蕢一言而悟,《春秋》載之。故禮部尚書頲累葉輔弼,奉事軒陛二十餘年,今奄忽不還,邦人痛嗟。惟帷盡之舊,股肱之戚,宜即廢朝,明君臣之誼。」帝曰:「固朕意也。」即日帳次哭洛城南門,不朝。詔贈右丞相,謚曰文憲。葬日,帝游咸宜宮,將獵,聞之,曰:「頲且葬,我忍自娛哉!」半道而還。



 頲性廉儉,奉稟悉推散諸弟親族,儲無長貲。自景龍後,與張說以文章顯,稱望略等,故時號「燕許大手筆」。帝愛其文,曰:「卿所為詔令,別錄副本,署臣某撰,朕當留中。」後遂為故事。其後李德裕著論曰「近世詔誥,惟頲敘事外自為文章」云。



 詵,字廷言,舉賢良方正高第,補汾陰尉,遷秘書詳正學士,累轉給事中,時頲為紫微侍郎,固辭。帝曰:「古有內舉不避親者乎?」對曰:「晉祁奚是也。」帝曰:「若然,朕自用詵,卿言非公也。」頃之,出徐州刺史,治有跡。卒,贈吏部侍郎。



 詵子震,以廕補千牛。十餘歲,強學有成人風。頲曰:「吾家有子。」累遷殿中侍御史、長安令。安祿山隱京師,震與尹崔光遠殺開遠門吏,棄家出奔。會肅宗興師靈武,震晝夜馳及行在,帝嘉之,拜御史中丞,遷文部侍郎。廣平王為元帥,崇擇賓佐,以震為糧料使。二京平,封岐陽縣公,改河南尹。九節度兵敗相州,震與留守崔圓奔襄、鄧,貶濟王府長史。起為絳州刺史,進戶部侍郎,判度支,為泰陵、建陵鹵簿使,以勞封岐國公,拜太常卿。代宗將幸東都,復以震為河南尹,未行,卒,贈禮部尚書。



 乾,瑰從父兄也。父勖,字慎行,武德中,為秦王諮議、典簽、文學館學士,尚南康公主,拜駙馬都尉。遷魏王泰府司馬,博學有美名,泰重之。勸開館引文學士,著書名家。歷吏部侍郎、太子左庶子,卒。乾擢明經,授徐王府記室參軍,王好畋,每諫止之。垂拱中,遷魏刺史。河朔饑,前刺史苛暴,百姓流徙,乾檢吏督奸,勸課農桑,由是流冗盡復,以治稱。拜右羽林軍將軍,遷冬官尚書。來俊臣素忌之,誣干與瑯邪王沖通書,系獄,發憤卒。



 張說,字道濟,或字說之,其先自範陽徙河南,更為洛陽人。永昌中,武后策賢良方正,詔吏部尚書李景諶糊名較覆,說所對第一,後署乙等,授太子校書郎,遷左補闕。



 後嘗問:「諸儒言氏族皆本炎、黃之裔,則上古乃無百姓乎?若為朕言之。」說曰:「古未有姓,若夷狄然。自炎帝之姜、黃帝之姬,始因所生地而為之姓。其後天下建德,因生以賜姓,黃帝二十五子,而得姓者十四。德同者姓同,德異者姓殊。其後或以官,或以國,或以王父之字,始為賜族,久乃為姓。降唐、虞,抵戰國,姓族漸廣。周衰,列國既滅,其民各以舊國為之氏,下及兩漢,人皆有姓。故姓之以國者,韓、陳、許、鄭、魯、衛、趙、魏為多。」後曰:「善。」



 久視中,後逭暑三陽宮,汔秋未還。說上疏曰:



 宮距洛城百六十里,有伊水之隔,崿阪之峻,過夏涉秋,水潦方積,道壞山險,不通轉運,河廣無梁,咫尺千里,扈從兵馬,日費資饟。太倉、武庫,並在都邑,紅粟、利器,蘊若山丘,奈何去宗朝之上都,安山谷之僻處?是猶倒持劍戟,示人樽柄,臣竊為陛下不取。夫禍變之生,在人所忽,故曰:「安樂必戒,無行所悔。」不可一也。宮城褊小,萬方輻湊,填郛溢郭,並鍤無所。排斥居人,蓬宿草次,風雨暴至,不知庇托,孤惸老病,流轉衢巷。陛下作人父母,將若之何?不可二也。池亭奇巧,蕩誘上心。削巒起觀,堨流漲海,俯貫地脈,仰出雲路,易山川之氣,奪農桑之土。延木石,運斧斤,山谷連聲,春夏不輟。勸陛下作此者,豈正人邪?《詩》云:「人亦勞止,迄可小康。」不可三也。御苑東西二十里,外無墻垣扃禁,內有榛業谿谷,猛毅所伏,暴慝所憑。陛下往往輕行,警蹕不肅,歷蒙密,乘險巇,卒有逸獸狂夫,驚犯左右,豈不殆哉?《易》曰:「思患豫防。」願為萬姓持重。不可四也。



 今北有胡寇覷邊,南有夷獠騷徼,關西小旱,耕稼是憂,安東近平,輸漕方始。臣願及時旋軫,深居上京,息人以展農,修德以來遠,罷不急之役,省無用之費。澄心澹懷,惟億萬年,蒼蒼群生,莫不幸甚。臣度芻議,十不從一,何者?沮盤游之娛,間林沚之玩,規遠圖,替近適,要後利,棄前歡,未沃明主之心,已捩貴臣之意。然不愛死者,懼言責不職耳。



 後不省。



 擢鳳閣舍人。張易之誣陷魏元忠也,援說為助。說廷對「元忠無不順言」,忤后旨,流欽州。中宗立,召為兵部員外郎,累遷工部、兵部二侍郎,以母喪免。既期,詔起為黃門侍郎,固請終制,祈陳哀到。時禮俗衰薄,士以奪服為榮,而說獨以禮終,天下高之。除喪,復為兵部,兼修文館學士。



 睿宗即位,擢中書侍郎兼雍州長史。譙王重福死,東都支黨數百人,獄久不決,詔說往按,一昔而罪人得,乃誅張靈均、鄭愔,餘詿誤悉原。帝嘉其不枉直,不漏惡,慰勞之。玄宗為太子,說與褚無量侍讀,尤見親禮。逾年,進同中書門下平章事,監修國史。



 景雲二年,帝謂侍臣曰:「術家言五日內有急兵入宮,為我備之。」左右莫對。說進曰:「此讒人謀動東宮耳,陛下若以太子監國,則名分定,奸膽破,蜚禍塞矣。」帝悟,下制如說言。明年,皇太子即皇帝位,太平公主引蕭至忠、崔湜等為宰相,以說不附己,授尚書左丞,罷政事,為東都留守。說知太平等懷逆,乃因使以佩刀獻玄宗,請先決策,帝納之。至忠等已誅,召為中書令,封燕國公,實封二百戶。



 始,武后末年,為潑寒胡戲,中宗嘗乘樓從觀。至是,因四夷來朝,復為之。說上疏曰:「韓宣適魯,見周禮而嘆,孔子會齊,數倡優之罪。列國如此,況天朝乎?今四夷請和,使者入謁,當按以禮樂,示以兵威,雖曰戎夷,不可輕也。焉知無駒支之辯,由余之賢哉?且乞寒潑胡,未聞典故,裸體跳足,汨泥揮水,盛德何觀焉?恐非干羽柔遠,樽俎折沖之道。」納之,自是遂絕。



 素與姚元崇不平,罷為相州刺史、河北道按察使。坐累徙岳州,停實封。說既失執政意,內自懼。雅與蘇瑰善,時瑰子頲為宰相,因作《五君詠》獻頲,其一紀瑰也,候瑰忌日致之。頲覽詩嗚咽,未幾,見帝陳說忠謇有勛,不宜棄外,遂遷荊州長史。



 俄以右羽林將軍檢校幽州都督,入朝以戎服見。帝大喜,授檢校並州長史,兼天兵軍大使,修國史,敕齎稿即軍中論譔。朔方軍大使王晙誅河曲降虜阿布思也,九姓同羅、拔野固等皆疑懼。說持節從輕騎二十,直詣其部,宿帳下,召見酋豪慰安之。副使李憲以虜難信,不宜涉不測。說報曰:「吾肉非黃羊,不畏其食;血非野馬,不畏其刺。士當見危致命,亦吾效死秋也。」由是九姓遂安。晙後討蘭池叛胡康待賓,詔說相聞經略。時黨項羌亦連兵攻銀城,說將步騎萬人出合河關掩擊,破之,追北駱駝堰。羌、胡自相猜,夜鬥,待賓遁入鐵建山,餘眾奔潰。說招納黨項,使復故處。副使史獻請盡誅之,說不從,奏置麟州以安羌眾。



 召拜兵部尚書、同中書門下三品,讓宋璟、陸象先,不許。明年,詔為朔方節度大使,親行五城,督士馬。時慶州方渠降胡康願子反,自為可汗,掠牧馬,西涉河出塞。說進討,至木槃山禽之,俘獲三千。乃議徙河曲六州殘胡五萬於唐、鄧仙、豫間,空河南朔方地。以功賜實封三百戶。故時,邊鎮兵贏六十萬,說以時平無所事,請罷二十萬還農。天子以為疑,說曰:「邊兵雖廣,諸將自衛、營私爾,所以制敵,不在眾也。以陛下之明,四夷畏威,不慮減兵而招寇,臣請以闔門百口為保。」帝乃可。時衛兵貧弱,番休者亡命略盡,說建請一切募勇強士,優其科條,簡色役。不旬日,得勝兵十三萬,分補諸衛,以強京師,後所謂「廣騎」者也。



 帝自東都將還京,因幸並州。說見帝曰:「太原王業所基,陛下巡幸,振耀威武,以申永思。繇河東入京師,有漢武脽上祠,此禮廢闊,歷代莫舉,願為三農祈彀,誠四海之福。」帝納其言,過祠后土乃還。進中書令。



 說又倡封禪議,受詔與諸儒草儀,多所裁正。帝召說與禮官學士置酒集仙殿,曰:「朕今與賢者樂於此,當遂為集賢殿。」乃下制改麗正書院為集賢殿書院,而授說院學士,知院事。東封還,為尚書右丞相兼中書令。詔說撰《封禪壇頌》,刻之泰山,以誇成功。初,源乾曜不欲封禪,說固請,乃不相平。及升山,執事官當從者,說皆引所厚超階入五品,從兵唯加勛而不賜,眾怨其專。



 宇文融先獻策,括天下游戶及籍外田,署十道勸農使,分行郡縣。說畏其擾,數沮格之。至是,融請吏部置十銓,與蘇釐等分治選事,有所論請,說頗抑之,於是銓綜失敘。融恨恚,乃與崔隱甫、李林甫共劾奏說「引術士王慶則夜祠禱解,而奏表其閭;引僧道岸窺言冋時事,冒署右職;所親吏張觀、範堯臣依據說勢,市權招賂,擅給太原九姓羊錢千萬。」其言醜慘。帝怒,詔乾曜、隱甫、刑部尚書韋抗即尚書省鞫之,發金吾兵圍其第。說兄左庶子光詣朝堂刑耳列冤,帝遣高力士往視,見說蓬首垢面,席槁,家人以瓦器饋脫粟鹽疏,為自罰憂懼者。力士還奏,且言:「說往納忠,於國有功。」帝憮然,乃停說中書令,誅慶則等,坐者猶十餘人。說既罷政事,在集賢院專脩國史。又乞停右丞相,不許。然每軍國大務,帝輒訪焉。隱甫等恐說復用,巧文詆毀,素忿說者又著《疾邪篇》,帝聞,因令致仕。



 始為相時,帝欲事吐蕃,說密請講和以休息鄣塞,帝曰:「朕待王君■計之。」說出告源乾曜曰:「君■好兵以求利,彼入,吾言不用矣。」後君■破吐蕃於青海西,說策其且敗,因上巂州斗羊於帝,以申諷諭,曰:「使羊能言,必將曰『斗而不解,立有死者』。所賴至仁無殘,量力取歡焉。」帝識其意,納之,賜彩千匹。後瓜州失守,君■死。



 十七年,復為右丞相,遷左丞相。上日,敕所司供帳設樂,內出醪饌,帝為賦詩。俄授開府儀同三司。十八年卒,年六十四,為停正會,贈太師,謚曰文貞,群臣駁異未決,帝為制碑,謚如太常,繇是定。



 說敦氣節,立然諾,喜推藉後進,於君臣朋友大義甚篤。帝在東宮,所與秘謀密計甚眾,後卒為宗臣。朝廷大述作多出其手,帝好文辭,有所為必使視草。善用人之長,多引天下知名士,以佐佑王化,粉澤典章,成一王法。天子尊尚經術,開館置學士,脩太宗之政,皆說倡之。為文屬思精壯,長於碑志,世所不逮。既謫岳州,而詩益淒婉,人謂得江山助云。常典集賢圖書之任,間雖致仕一歲,亦修史於家。



 始,帝欲授說大學士,辭曰:「學士本無大稱,中宗崇寵大臣,乃有之,臣不敢以為稱。」固辭乃免。後宴集賢院,故事,官重者先飲,說曰:「吾聞儒以道相高,不以官閥為先後。太宗時修史十九人,長孫無忌以元舅,每宴不肯先舉爵。長安中,與修《珠英》,當時學士亦不以品秩為限。」於是引觴同飲,時伏其有體。中書舍人陸堅以學士或非其人,而供擬太厚,無益國家者,議白罷之。說聞曰:「古帝王功成,則有奢滿之失,或興池觀,或尚聲色。今陛下崇儒向道,躬自講論,詳延豪俊,則麗正乃天子禮樂之司,所費細而所益者大。陸生之言,蓋未達邪。」帝知,遂薄堅。



 說嘗自為其父碑,帝為書其額曰:「嗚呼,積善之墓。」說歿後,帝使就家錄其文,行於世。開元後,宰相不以姓著者,曰燕公云。大歷中,詔配享玄宗廟廷。子均、垍、埱。



 均亦能文。自太子通事舍人累遷主爵郎中、中書舍人。開元十七年,說授左丞相,校京官考,注均考曰:「父教子忠,古之善訓,王言帝載,尤難以任。庸以嫌疑,而撓紀綱?考上下。」當時亦不以為私。後襲燕國公,累遷兵部侍郎,以累貶饒、蘇二州刺史。久之,復為兵部侍郎。



 自以己才當輔相,為李林甫所抑,林甫卒,倚陳希烈,冀得其處。既而楊國忠用事,希烈罷,而均為刑部尚書。坐垍,貶建安太守。還,授大理卿,居常觖望不平。祿山盜國,為偽中書令。肅宗反正,兄弟皆論死。房琯聞之,驚曰:「張氏滅矣。」乃見苗晉卿,營解之。帝亦顧說有舊,詔免死,流合浦。建宮初,贈太子少傅。子濛,事德宗,為中書舍人。



 垍尚寧親公主。時說居中秉政,均為舍人,諸父光為銀青光祿大夫,榮盛冠時。玄宗眷垍厚,即禁中置內宅,侍為文章,珍賜不可數。均供奉翰林,而垍以所賜誇均,均曰:「此婦翁遺婿,非天子賜學士也。」垍嘗為帝贊禮,舉止都雅,帝悅之。因幸內宅,顧垍曰:「希烈辭宰相,孰可代者?垍錯愕,未得對。帝曰:「無易吾婿。」垍頓首謝。會貴妃聞,以語國忠,國忠惡之,及希烈罷,薦韋見素代之,垍始怨上。



 天寶十三載,祿山入朝,以破奚、契丹功,求平章事,國忠曰:「祿山有軍功,然不識字,與之,恐四夷輕漢。」乃止。及還範陽,詔高力士餞滻坡,力士歸曰:「祿山內鬱鬱,若知欲相而不行者。」帝以語國忠,國忠曰:「所告者必張垍。」帝怒,盡逐其兄弟,以均守建安,而垍為盧溪郡司馬,埱自給事中為宜春郡司馬。歲中,還,垍為太常卿。



 帝西狩至咸陽,唯韋見素、楊國忠、魏方進從。帝謂力士曰:「若計朝臣當孰至者?」力士曰:「張垍兄弟世以恩戚貴,其當即來。房琯有宰相望,而陛下久不用,又為祿山所器,此不來矣。」帝曰:「未可知也。」後琯至,召見流涕。帝撫勞,且問:「均、垍安在?」琯曰:『臣之西,亦嘗過其家,將與偕來。均曰:「馬不善馳,後當繼行。』然臣觀之,恐不能從陛下矣。」帝嗟悵,顧力士曰:「吾豈欲誣人哉?均等自謂才器亡雙,恨不大用,吾向欲始全之,今非若所料也。」垍遂與希烈皆相祿山,垍死賊中。



 贊曰:說於玄宗最有德,及太平用事,納忠惓惓,又圖封禪,發明典章,開元文物彬彬,說力居多。中為奸人排擯,幾不免,自古功名始終亦幾希,何獨說哉!至子以利遽敗其家。若瑰、頲再世稱賢宰相,盛矣!



\end{pinyinscope}