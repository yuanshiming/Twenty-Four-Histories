\article{列傳第一百五十下 逆臣下}

\begin{pinyinscope}

 黃巢,曹州冤句人。世鬻鹽,富於貲。善擊劍騎射,稍通書記,辯給銳利武器。,喜養亡命。



 咸通末,仍歲饑,盜興河南。乾符二年,濮名賊王仙芝亂長垣,有眾三千,殘曹、濮二州,俘萬人,勢遂張。仙芝妄號大將軍,檄諸道,言吏貪沓,賦重,賞罰不平。宰相恥之,僖宗不知也。其票帥尚君長、柴存、畢師鐸、曹師雄、柳彥璋、劉漢宏、李重霸等十餘輩,所在肆掠。而巢喜亂,即與群從八人,募眾得數千人以應仙芝,轉寇河南十五州,眾遂數萬。



 帝使平廬節度使宋威與其副曹全晸數擊賊,敗之,拜諸道行營招討使,給衛兵三千、騎五百,詔河南諸鎮皆受節度,以左散騎常侍曾元裕副焉。仙芝略沂州,威敗賊城下,仙芝亡去。威因奏大渠死,擅縱麾下兵還青州,君臣皆入賀。居三日,州縣奏賊故在。時兵始休,有詔復遣,士皆忿,思亂。賊間之,趣郟城,不十日破八縣。帝憂迫近東都,督諸道兵檢遏,於是鳳翔、邠寧、涇原兵守陜、潼關,元裕守東都,義成、昭義以兵衛宮。



 仙芝去攻汝州,殺其將,刺史走,東都大震,百官脫身出奔。賊破陽武,圍鄭州,不克,蟻聚鄧、汝間。關以東州縣,大抵皆畏賊,嬰城守,故賊放兵四略,殘郢、復二州,所過焚剽,生人幾盡。官軍急追,則遺貲布路,士爭取之,率逗橈不前。賊轉入申、光,殘隋州,執刺史,據安州自如,分奇兵圍舒,擊廬、壽、光等州。



 時威老且暗,不任軍,陰與元裕謀曰:「昔龐勛滅,康承訓即得罪。吾屬雖成功,其免禍乎?不如留賊,不幸為天子,我不失作功臣。」故躡賊一舍,完軍顧望。帝亦知之,更以陳許節度使崔安潛為行營都統,以前鴻臚卿李琢代威,右威衛上將軍張自勉代元裕。



 賊出入蘄、黃,蘄州刺史裴渥為賊求官,約罷兵。仙芝與巢等詣渥飲。未幾,詔拜仙芝左神策軍押衙,遣中人慰撫。仙芝喜,巢恨賞不及己,詢曰:「君降,獨得官,五千眾且奈何?丐我兵,無留。」因擊仙芝,傷首。仙芝憚眾怒,即不受命,劫州兵,渥、中人亡去。賊分其眾:尚君長入陳、蔡;巢北掠齊、魯,眾萬人,入鄆州,殺節度使薛崇,進陷軍州,遂至數萬,繇潁、蔡保嵖岈山。



 是時柳彥璋又取江州,執刺史陶祥。巢引兵復與仙芝合,圍宋州。會自勉救兵至,斬賊二千級,仙芝解而南,度漢,攻荊南。於是節度使楊知溫嬰城守,賊縱火焚樓堞,知溫不出,有詔以高駢代之。駢以蜀兵萬五千齎Я糧,期三十日至,而城已陷,知溫走,賊不能守。於是詔左武衛將軍劉秉仁為江州刺史,勒兵乘單舟入賊柵,賊大駭,相率迎降,遂斬彥璋。



 巢攻和州,未克。仙芝自圍洪州,取之,使徐唐莒守。進破朗、岳,遂圍潭州,觀察使崔瑾拒卻之。乃向浙西,擾宣、潤,不能得所欲,身留江西,趣別部還入河南。



 帝詔崔安潛歸忠武,復起宋威、曾元裕,以招討使還之,而楊復光監軍。復光遣其屬吳彥宏以詔諭賊,仙芝乃遣蔡溫球、楚彥威、尚君長來降,欲詣闕請罪,又遺威書求節度。威陽許之,上言「與君長戰,禽之」。復光固言其降。命侍御史與中人馳驛即訊,不能明。卒斬君長等於狗脊嶺。仙芝怒,還攻洪州,入其郛。威自將往救,敗仙芝於黃梅,斬賊五萬級,獲仙芝,傳首京師。



 當此時,巢方圍亳州未下,君長弟讓率仙芝潰黨歸巢,推巢為王,號「沖天大將軍」,署拜官屬,驅河南、山南之民十餘萬掠淮南,建元王霸。



 曾元裕敗賊於申州,死者萬人。帝以威殺尚君長非是,且討賊無功,詔還青州,以元裕為招討使,張自勉為副。巢破考城,取濮州,元裕軍荊、襄,援兵阻,更拜自勉東北面行營招討使,督諸軍急捕。巢方掠襄邑、雍丘,詔滑州節度使李嶧壁原武。巢寇葉、陽翟,欲窺東都。會左神武大將軍劉景仁以兵五千援東都,河陽節度使鄭延休兵三千壁河陰。巢兵在江西者,為鎮海節度使高駢所破;寇新鄭、郟、襄城、陽翟者,為崔安潛逐走;在浙西者,為節度使裴璩斬二長,死者甚眾。巢大沮畏,乃詣天平軍乞降,詔授巢右衛將軍。巢度籓鎮不一,未足制己,即叛去,轉寇浙東,執觀察使崔璆。於是高駢遣將張潾、梁纘攻賊,破之。賊收眾逾江西,破虔、吉、饒、信等州,因刊山開道七百里,直趨建州。



 初,軍中謠曰:「逢儒則肉,師必覆。」巢入閩,俘民紿稱儒者,皆釋,時六年三月也。儳路圍福州,觀察使韋岫戰不勝,棄城遁,賊入之,焚室廬,殺人如蓺。過崇文館校書郎黃璞家,令曰:「此儒者,滅炬弗焚。」又求處士周樸,得之,謂曰:「能從我乎?」答曰:「我尚不仕天子,安能從賊?」巢怒斬樸。是時閩地諸州皆沒,有詔高駢為諸道行營都統以拒賊。



 巢陷桂管,進寇廣州,詒節度使李迢書,求表為天平節度,又脅崔璆言於朝,宰相鄭畋欲許之,盧攜、田令孜執不可。巢又丐安南都護、廣州節度使。書聞,右僕射於琮議:「南海市舶利不貲,賊得益富,而國用屈。」乃拜巢率府率。巢見詔大詬,急攻廣州,執李迢,自號「義軍都統」,露表告將入關,因詆宦豎柄朝,垢蠹紀綱,指諸臣與中人賂遺交構狀,銓貢失才,禁刺史殖財產,縣令犯贓者族,皆當時極敝。



 天子既懲宋威失計,罷之,而宰相王鐸請自行,乃拜鐸荊南節度使、南面行營招討都統,率諸道兵進討。鐸屯江陵,表泰寧節度使李系為招討副使、湖南觀察使,以先鋒屯潭州,兩屯烽驛相望。會賊中大疫,眾死什四,遂引北還。自桂編大桴,沿湘下衡、永,破潭州,李系走朗州,兵十餘萬闉焉,投胔蔽江。進逼江陵,號五十萬。鐸兵寡,即乘城。先此,劉漢宏已略地,焚廬廥,人皆竄山谷。俄而系敗問至,鐸棄城走襄陽,官軍乘亂縱掠,會雨雪,人多死溝壑。



 其十月,巢據荊南,脅李迢草表報天子。迢曰:「吾腕可斷,表不可為。」巢怒,殺之。欲進躡鐸,會江西招討使曹全晸與山南東道節度使劉巨容壁荊門,使沙陀以五百騎釕轡藻韉望賊陣縱而遁,賊以為怯。明日,諸將乘以戰,而馬識沙陀語,呼之輒奔還,莫能禁。官兵伏於林,鬥而北,賊急追,伏發,大敗之,執賊渠十二輩。巢懼,度江東走,師促之,俘什八,鐸招漢宏降之。或勸巨容窮追,答曰:「國家多負人,危難不吝賞,事平則得罪,不如留賊冀後福。」止不追,故巢得復整,攻鄂州,入之。全晸將度江,會有詔以段彥枌代其使,乃止。



 巢畏襲,轉掠江西,再入饒、信、杭州,眾至二十萬。攻臨安,戍將董昌兵寡,不敢戰,伏數十騎莽中,賊至,伏弩射殺賊將,下皆走。昌進屯八百里,見舍媼曰:「有追至,告以臨安兵屯八百里矣。」賊駭曰:「向數騎能困我,況軍八百里乎?」乃還,殘宣、歙等十五州。



 廣明元年,淮南高駢遣將張潾度江敗王重霸,降之。巢數卻,乃保饒州,眾多疫,別部常宏以眾數萬降,所在戮死。諸軍屢奏破賊,皆不實,朝廷信之,稍自安。巢得計,破殺張潾,陷睦、婺二州,又取宣州。而漢宏殘眾復奮,寇宋州,掠申、光,來與巢合,濟採石,侵揚州。高駢按兵不出。詔兗海節度使齊克讓屯汝州,拜全晸天平節度使兼東面副都統。賊方守滁、和,全晸以天平兵敗於淮上。宰相豆盧彖計:「救師未至,請假巢天平節度使,使無得西,以精兵戍宣武,塞汝、鄭路,賊首可致矣。」盧攜執不可,請「召諸道兵壁泗上,以宣武節度統之,則巢且還寇東南,徘徊山浙,救死而已」。詔可。前此已詔天下兵屯溵水,禁賊北走。於是徐兵三千道許,其帥薛能館徐眾城中,許人驚謂見襲,部將周岌自溵水還,殺能,自稱留後。徐軍聞亂,列將時溥亦引歸,囚其帥支詳。兗海齊克讓懼下叛,引軍還兗州,溵水屯皆散。



 巢聞,悉眾度淮,妄稱「率土大將軍」,整眾不剽掠,所過惟取丁壯益兵。李罕之犯申、光、潁、宋、徐、兗等州,吏皆亡。巢自將攻汝州,欲薄東都。當是時,天子沖弱,怖而流淚,宰相更共建言,悉神策並關內諸節度兵十五萬守潼關。田令孜請自將而東,然內震擾,前說帝以幸蜀事。帝自幸神策軍,擢左軍騎將張承範為先鋒,右軍步將王師會督糧道,以飛龍使楊復恭副令孜。於是募兵京師,得數千人。



 當是時,巢已陷東都,留守劉允章以百官迎賊。巢入,勞問而已,裏閭晏然。帝餞令孜章信門,齎遺豐優。然衛兵皆長安高貲,世籍兩軍,得稟賜,侈服怒馬以詫權豪,初不知戰,聞料選,皆哭於家,陰出貲雇販區病坊以備行陣,不能持兵,觀者寒毛以心慄。承範以強弩三千防關,辭曰:「祿山率兵五萬陷東都,今賊眾六十萬,過祿山遠甚,恐不足守。」帝不許。賊進取陜、虢,檄關戍曰:「吾道淮南,逐高駢如鼠走穴,爾無拒我!」神策兵過華,裹三日糧,不能飽,無鬥志。



 十二月,巢攻關,齊克讓以其軍戰關外,賊少卻。俄而巢至,師大呼,川穀皆震,時士饑甚,潛燒克讓營,克讓走入關。承範出金諭軍中曰:「諸君勉報國,救且至。」士感泣,拒戰。賊見師不繼,急攻關,王師矢盡,飛石以射,巢驅氏內塹,火關樓皆盡。始,關左有大谷,禁行人,號「禁穀」。賊至,令孜屯關,而忘穀之可入。尚讓引眾趨谷,承範惶遽,使師會以勁弩八百邀之,比至,而賊已入。明日,夾攻關,王師潰。師會欲自殺,承範曰:「吾二人死,孰當辨者?不如見天子以實聞,死未晚。」乃羸服逃。始,博野、鳳翔軍過渭橋,見募軍服鮮燠,怒曰:「是等何功,遽然至是!」更為賊鄉導,前賊歸,焚西市。帝類郊祈哀。會承範至,具言不守狀。帝黜宰相盧攜。方朝,而傳言賊至,百官奔,令孜以神策兵五百奉帝趨咸陽,惟福、穆、潭、壽四王與妃御一二從,中人西門匡範統右軍以殿。



 巢以尚讓為平唐大將軍,蓋洪、費全古副之。賊眾皆被發錦衣,大抵輜重自東都抵京師,千里相屬。金吾大將軍張直方與群臣迎賊灞上。巢乘黃金輿,衛者皆繡袍、華幘,其黨乘銅輿以從,騎士凡數十萬先後之。陷京師,入自春明門,升太極殿,宮女數千迎拜,稱黃王。巢喜曰:「殆天意歟!」巢舍田令孜第。賊見窮民,抵金帛與之。尚讓即妄曉人曰:「黃王非如唐家不惜而輩,各安毋恐。」甫數日,因大掠,縛棰居人索財,號「淘物」。富家皆跣而驅,賊酋閱甲第以處,爭取人妻女亂之,捕得官吏悉斬之,火廬舍不可貲,宗室侯王屠之無類矣。



 巢齋太清宮,卜日舍含元殿,僭即位,號大齊。求袞冕不得,繪弋綈為之;無金石樂,擊大鼓數百,列長劍大刀為衛。大赦,建元為金統。王官三品以上停,四品以下還之。因自陳符命,取「廣明」字,判其文曰:「唐去醜口而著黃,明黃當代唐;又黃為土,金所生,蓋天啟」云。其徒上巢號承天應運啟聖睿文宣武皇帝,以妻曹為皇后,以尚讓、趙璋、崔璆、楊希古為宰相,鄭漢璋御史中丞,李儔、黃諤、尚儒為尚書,方特諫議大夫,皮日休、沈雲翔、裴渥翰林學士,孟楷、蓋洪尚書左右僕射兼軍容使,費傳古樞密使,張直方檢校左僕射,馬祥右散騎常侍,王璠京兆尹,許建、米實、劉瑭、硃溫、張全、彭攢、李逵等為諸將軍游弈使,其餘以次封拜。取趫偉五百人號「功臣」,以林言為之使,比控鶴府。下令軍中禁妄殺人,悉輸兵於官。然其下本盜賊,皆不從。召王官,無有至者,乃大索裏閭,豆盧彖、崔沆等匿永寧里張直方家。直方者,素豪桀,故士多依之。或告賊納亡命者,巢攻之,夷其家,彖、沆及大臣劉鄴、裴諗、趙濛、李溥、李湯死者百餘人。將作監鄭綦、郎官鄭系舉族縊。



 是時,乘輿次興元,詔促諸道兵收京師,遂至成都。巢使硃溫攻鄧州,陷之,以擾荊、襄。遣林言、尚讓寇鳳翔,為鄭畋將宋文通所破,不得前。畋乃傳檄召天下兵,於是詔涇原節度使程宗楚為諸軍行營副都統,前朔方節度使唐弘夫為行營司馬。數攻賊,斬萬級。邠將硃玫陽為賊將王玫裒兵,俄而殺玫,引軍入於王師。弘夫進屯渭北,河中王重榮營沙苑,易定王處存次渭橋,鄜延李孝昌、夏州拓拔思恭壁武功。弘夫拔咸陽,伐渭水,破尚讓軍,乘勝入京師。巢竊出,至石井。宗楚入自延秋門,弘夫傅城舍,都人共噪曰:「王師至!」處存選銳卒五千以白自志,綯夜入殺賊,都人傳言巢已走,邠、涇軍爭入京師,諸軍亦解甲休,競掠貨財子女,市少年亦冒作綯,肆為剽。



 巢伏野,使覘城中弛備,則遣孟楷率賊數百掩邠、涇軍,都人猶謂王師,歡迎之。時軍士得珍賄,不勝載,聞賊至,重負不能走,是以甚敗。賊執弘夫害之,處存走營。始,王璠破奉天,引眾數千隨弘夫,及諸將敗,獨一軍戰尤力。巢復入京師,怒民迎王師,縱擊殺八萬人,備流於路可涉也,謂之「洗城」。諸軍退保武功,於是中和二年二月也。



 其五月,昭義高潯攻華州,王重榮與並力,克之。硃玫以涇、岐、麟、夏兵八萬營興平,巢亦遣王璠營黑水,玫戰未能勝。鄭畋將竇玫夜率士燔都門,殺邏卒,賊震懼。於時畿民柵山谷自保,不得耕,米斗錢三十千,屑樹皮以食,有執柵民鬻賊以為糧,人獲數十萬錢。士人或賣餅自業,舉奔河中。李孝昌、拓拔思恭徙壁東渭橋,收水北壘。



 數月,賊帥硃溫、尚讓涉渭敗孝昌等軍。高潯擊賊李詳,不勝,賊復取華州,巢即授華州刺史,以溫為同州刺史。賊又襲孝昌,二軍引去。賊破陳敬瑄兵,走南山。齊克儉營興平,為賊所圍,決河灌之,不克。有題尚書省戶譏賊且亡,尚讓怒,殺吏,輒剔目懸之,誅郎官門闌卒凡數千人,百司逃,無在者。



 天子更以王鐸為諸道行營都統,崔安潛副之,周岌、王重榮為左右司馬,諸葛爽、康實為左右先鋒,平師儒為後軍,時溥督漕賦,王處存、拓拔思恭為京畿都統,處存直左,孝章在北,思恭直右。西門思恭為鐸都監,楊復光監行營,中書舍人盧胤徵為克復制置副使。於是鐸以山南、劍南軍營靈感祠,硃玫以岐、夏軍營興平,重榮、處存營渭北,復光以壽、滄、荊南軍合岌營武功,孝章合拓拔思恭營渭橋,程宗楚營京右。



 硃溫以兵三千掠丹、延南鄙,趨同州,刺史米逢出奔,溫據州以守。六月,尚讓寇河中,使硃溫攻西關,敗諸葛爽,破重榮數千騎於河上,爽閉關不出,讓遂拔郃陽,攻宜君壘,大雨雪盈尺,兵死什三。七月,賊攻鳳翔,敗節度李昌言於澇水,又遣強武攻武功、槐里,涇、邠兵卻,獨鳳翔兵固壁。拓拔思恭以銳士萬八千赴難,逗留不進。河中糧艘三十道夏陽,硃溫使兵奪艘,重榮以甲士三萬救之,溫懼,鑿沉其舟,兵遂圍溫。溫數困,又度巢勢蹙且敗,而孟楷方專國,溫丐師,楷沮不報,即斬賊大將馬恭,降重榮。帝進拓拔思恭為京四面都統,敕硃玫軍馬嵬。溫既降,重榮遇之厚,故李詳亦獻款,賊覺,斬之於赤水,更以黃思鄴為刺史。



 十月,鐸浚壕於興平,左抵馬嵬,使將薛韜董之,由馬嵬、武功入斜谷,以通盩厔,列屯十四,使將梁璩主之,置關於沮水、七盤、三溪、木皮嶺,以遮秦、隴。京左行營都統東方逵禽賊銳將李公迪,破堡三十。華卒逐黃思鄴,巢以王遇為刺史,遇降河中。



 明年正月,王鐸使雁門節度使李克用破賊於渭南,承制拜東北行營都統。會鐸與安潛皆罷,克用獨引軍自嵐、石出夏陽,屯沙苑,破黃揆軍,遂營乾坑。二月,合河中、易定、忠武等兵擊巢。巢命王璠、林言軍居左,趙璋、尚讓軍居右,眾凡十萬,與王師大戰梁田陂。賊敗,執俘數萬,殭胔三十里,斂為京觀。璠與黃揆襲華州,據之,遇亡去。克用掘塹環州,分騎屯渭北,命薛志勤、康君立夜襲京師,火廥聚,俘賊而還。



 巢戰數不利,軍食竭,下不用命,陰有遁謀,即發兵三萬扼藍田道,使尚讓援華州。克用率重榮迎戰零口,破之,遂拔其城,揆引眾出走。涇原節度使張鈞說蕃、渾與盟,共討賊。是時,諸鎮兵四面至。四月,克用遣部將楊守宗率河中將白志遷、忠武將龐從等最先進,擊賊渭橋,三戰,賊三北。於是諸節度兵皆奮,無敢後,入自光泰門。克用身決戰,呼聲動天,賊崩潰,逐北至望春,入升陽殿闥。巢夜奔,眾猶十五萬,聲趨徐州,出藍田,入商山,委輜重珍貲於道,諸軍爭取之,不復追,故賊得整軍去。



 自祿山陷長安,宮闕完雄,吐蕃所燔,唯衢弄廬舍;硃泚亂定百餘年,治繕神麗如開元時。至巢敗,方鎮兵互入虜掠,火大內,惟含元殿獨存,火所不及者,止西內、南內及光啟宮而已。楊復光獻捷行在,帝詔陳許、延州、鳳翔、博野軍合東西神策二萬人屯京師,命大明宮留守王徽衛諸門,撫定居人。詔尚書右僕射裴璩修復宮省,購輦輅、仗衛、舊章、秘籍。豫敗巢者:神策將橫沖軍使楊守亮、躡雲都將高周彞、忠順都將胡真、天德將顧彥朗七十人。



 巢已東,使孟楷攻蔡州。節度使秦宗權迎戰,大敗,即臣賊,與連和。楷擊陳州,敗死,巢自圍之,略鄧、許、孟、洛,東入徐、兗數十州。人大饑,倚死墻塹,賊俘以食,日數千人,乃辦列百巨碓,糜骨皮於臼,並啖之。時硃全忠為宣武節度使,與周岌、時溥帥師救陳,趙犨亦乞兵太原。巢遣宗權攻許州,未克。於是糧竭,木皮草根皆盡。



 四年二月,李克用率山西兵由陜濟河而東,會關東諸鎮壁汝州。全忠擊賊瓦子堡,斬萬餘級,諸軍破尚讓於太康,亦萬級,獲械鎧馬羊萬計,又敗黃鄴於西華,鄴夜遁。巢大恐,居三日,軍中相驚,棄壁走,巢退營故陽里。其五月,大雨震電,川溪皆暴溢,賊壘盡壞,眾潰,巢解而去。全忠進戍尉氏,克用追巢,全忠還汴州。



 巢取尉氏,攻中牟,兵度水半,克用擊之,賊多溺死。巢引殘眾走封丘,克用追敗之,還營鄭州。巢涉汴北引,夜復大雨,賊驚潰,克用聞之,急擊巢河瀕。巢度河攻汴州,全忠拒守,克用救之,斬賊驍將李周、楊景彪等。巢夜走胙城,入冤句。克用悉軍窮躡,賊將李讜、楊能、霍存、葛從周、張歸霸、張歸厚往降全忠,而尚讓以萬人歸時溥。巢愈猜忿,屢殺大將,引眾奔兗州。克用追至曹,巢兄弟拒戰,不勝,走兗、鄆間,獲男女牛馬萬餘、乘輿器服等,禽巢愛子。克用軍晝夜馳,糧盡不能得巢,乃還。巢眾僅千人,走保太山。



 六月,時溥遣將陳景瑜與尚讓追戰狼虎谷,巢計蹙,謂林言曰:「我欲討國奸臣,洗滌朝廷,事成不退,亦誤矣。若取吾首獻天子,可得富貴,毋為他人利。」言,巢出也,不忍。巢乃自刎,不殊,言因斬之,及兄存、弟鄴、揆、欽、秉、萬通、思厚,並殺其妻子,悉函首,將詣溥。而太原博野軍殺言,與巢首俱上溥,獻於行在,詔以首獻於廟。徐州小史李師悅得巢偽符璽,上之,拜湖州刺史。



 巢從子浩眾七千,為盜江湖間,自號「浪蕩軍」。天復初,欲據湖南,陷瀏陽,殺略甚眾。湘陰強家鄧進思率壯士伏山中,擊殺浩。



 贊曰:廣明元年,巢始盜京師,自陳「唐去醜口而著黃,明黃且代唐也。」鳴呼,其言妖歟!後巢死,秦宗權始張,株亂遍天下,硃溫卒攘神器有之,大氐皆巢黨也。寧天托諸人告亡於下乎!



 秦宗權,蔡州上蔡人,為許牙將。巢涉淮,節度使薛能遣宗權搜兵淮西,而許軍亂,殺能。宗權外示赴難,因逐刺史,據蔡以叛。周岌代能領節度,即授以州,有兵萬人,乃遣將從諸軍敗賊於汝州。楊復光言之朝,擢防禦使,寵其軍曰奉國,即為本軍節度使,進檢校司空。



 巢走出關,宗權與連和,遂圍陳州,樹壁相望,擾敓梁、宋間。巢死,宗權張甚,嘯會逋殘,有吞噬四海意。乃遣弟宗言寇荊南;秦誥出山南,攻襄州,陷之,進破東都,圍陜州;使秦彥寇淮、肥;秦賢略江南;宗衡亂岳、鄂。賊渠率票慘,所至屠老孺,焚屋廬,城府窮為荊萊,自關中薄青、齊,南繚荊、郢,北亙衛、滑,皆麕駭雉伏,至千里無舍煙。惟趙犨保陳,硃全忠保汴,僅自完而已。然無霸王計,惟亂是恃,兵出未始轉糧,指鄉聚曰:「啖其人,可飽吾眾。」官軍追躡,獲鹽尸數十車。



 僖宗假硃全忠都統節以討賊。秦賢略宋及曹,全忠好書約和,賢遣張調請分地,自汴以南歸之蔡,全忠陰許,而賢引兵濟汴,肆燔劫無孑餘。全忠大怒,斬調而還,曰:「我出十將,必破此賊。」進與賊戰,殺獲甚眾。宗權急攻許,節度使鹿晏弘乞師於全忠,師未及出,已破晏弘,進攻鄭州,取之。擊河橋,遂守河陽,放兵侵汴西鄙、北鄙。



 全忠壁酸棗,戰不克。宗權屯邊村,使秦賢營雙丘,侵板橋,盧瑭引兵進屯萬勝,夾汴而柵,將梁以濟師。全忠詭擊殺瑭,宗權悉軍十五萬列三十六屯,逼汴。全忠懼,求救於兗、鄆,而硃瑾、硃宣皆身自將同拒賊。五月,全忠閉城大會,鼓聞於郊無置聲,陰啟北門擊賊壘,士嘩,趨中營,兗、鄆整兵合擊,大敗之。宗權忿,過鄭,焚郛舍,驅民入淮西,全忠遂有鄭、許、河陽、東都。



 於是合諸鎮兵會上蔡,分為五軍入其地。宗權召孫儒,儒不應。宗權素壁上蔡以扼險要,全忠拔其壁,遂圍蔡州,傅城而壘,以羸兵誘賊。賊出,全忠盡斬之。宗權退守中州,未能下,全忠使大將胡元琮圍之,身還汴。宗權間許無備,襲取其州,執守將元琮,引兵復收許。



 宗權還,為愛將申叢所囚,折一足以待命。全忠署叢節度留後,叢中悔,夷其族。宗權至汴,全忠以禮迎勞,且曰:「公昔陷許,能戢兵賜盟,戮力勤王,烏有今日乎?」宗權曰:「英雄不兩立,天亡僕以資公也。」謷然無懼色。全忠以檻車上送京師,兩神策兵縻護。昭宗御延喜樓受俘,京兆尹曳以組練,徇兩市,引頸視車外,呼曰:「宗權豈反者耶?顧輸忠不效耳。」觀者大笑。與妻趙俱斬獨柳下。宗權以中和三年叛,居六年而誅。



 董昌,杭州臨安人。始籍土團軍,以功擢累石鏡鎮將。中和三年,刺史路審中臨州,昌率兵拒,不得入,即自領州事。鎮海節度使周寶不能制,因表為刺史。昌已破劉漢宏,兵益強,進義勝軍節度使、檢校尚書右僕射。僖宗始還京師,昌取越民裴氏藏書獻之,補秘書之亡,授兼諸道採訪圖籍使。



 始,為治廉平,人頗安之。當是時,天下貢輸不入,獨昌賦外獻常參倍,旬一道,以五百人為率,人給一刀,後期即誅。朝廷賴其入,故累拜檢校太尉、同中書門下平章事,爵隴西郡王。視詔書訖,字償一縑,歸當制官。而小人意足,浸自侈大,托神以詭眾。始立生祠,刳香木為軀,內金玉紈素為肺府,冕而坐,妻媵侍別帳,百倡鼓吹於前,屬兵列護門䍦。屬州為土馬獻祠下,列牲牢祈請,或紿言土馬若嘶且汗,皆受賞。昌自言:「有饗者,我必醉。」蝗集祠旁,使人捕沈鏡湖,告曰:「不為災。」客有言:「嘗游吳隱之祠,止一偶人。」昌聞,怒曰:「我非吳隱之比!」支解客祠前。



 始,罷榷鹽以悅人,豐衣食,後稍峭法,笞至千百,或小過輒夷族,血流刑場,地為之赤。有五千餘姓當族,昌曰:「能孝於我,貸而死。」皆曰:「諾。」昌厚養之,號「感恩都」,刻其臂為誓,親族至號泣相別者。凡民訟,不視獄,但與擲博齒,不勝者死。用人亦取勝者。



 昌得郡王,吒曰:「朝廷負我,吾奉金帛不貲,何惜越王不吾與?吾當自取之!」下厭其虐,乃勸為帝。近縣舉狂畐言虖請,昌令曰:「時至,我當應天順人。」其屬吳繇、秦昌裕、盧勤、硃瓚、董庠、李暢、薛遼與妖人應智、王溫、巫韓媼皆贊之。昌益兵城四縣自防。山陰老人偽獻謠曰:「欲知天子名,日從日上生。」昌喜,賜百縑,免稅征。命方士硃思遠築壇祠天,詭言天符夜降,碧楮硃文不可識。昌曰:「讖言『兔上金床』,我生於卯,明年歲旅其次,二月朔之明日,皆卯也,我以其時當即位。」客倪德儒曰:「咸通末,《越中秘記》言:『有羅平鳥,主越禍福。』中和時,鳥見吳、越,四目而三足,其鳴曰『羅平天冊』,民祀以攘難。今大王署名,文與鳥類。」即圖以示昌,昌大喜。



 乾寧二年,即偽位,國號大越羅平,建元曰天冊,自稱「聖人」,鑄銀印方四寸,文曰「順天治國之印」。又出細民所上銅鉛石印十床及它鳥獸龜蛇陳於廷,指曰「天瑞」。其下制詔,皆自署名,或曰帝王無押詔,昌曰:「不親署,何由知我為天子?」即榜南門曰天冊樓。先是,州寢有赤光,長十餘丈;虺長尺餘,金色,見思道亭。昌署寢曰明光殿,亭曰黃龍殿,以自神。以次拜置百官,監軍與官屬皆西北向慟哭,乃北面臣昌。或請署近侍,昌曰:「吾假處此位,安得如宮禁?」不許。下書屬州曰:「以某日權即位,然昌荷天子恩,死不敢負國。」



 初,官屬不徇昌旨者,節度副使黃碣、山陰令張遜皆誅死。鎮海節度使錢鏐書讓昌曰:「開府領節度,終身富貴,不能守,閉城作天子,滅親族,亦何賴?願王改圖。」昌不聽,燜悉兵三萬攻之,望城再拜曰:「大王位將相,乃不臣。能改過,請諭還諸軍。」昌懼,獻鏐錢二百萬緡犒軍,執應智、王溫、韓媼、吳繇、秦昌裕送於鏐,且待罪。燜乃還,表於朝,以為昌不可赦,復討之,傅城而壘。昌又執硃思遠、王守真、盧勤送鏐軍求解。昭宗遣中人李重密勞師,除昌官爵,授鏐浙東道招討使。昌乃求援於淮南楊行密,行密遣將臺濛圍蘇州,安仁義、田頵攻杭州,以救昌。鏐將顧全武等數敗昌軍,昌將多降,遂進圍越州。



 候人言外師強,輒斬以徇;紿告鏐兵老,皆賞。昌身閱兵五雲門,出金帛傾鏐眾。全武等益奮,昌軍大潰,遽還,去偽號,曰:「越人勸我作天子,固無益,今復為節度使。」全武四面攻,未克,會臺濛取蘇州,鏐召全武還,全武曰:「賊根本在甌、越,今失一州而緩賊,不可。」攻益急。城中以口率錢,雖簪珥皆輸軍。昌從子真得士心,昌信讒殺之,眾始不用命。又減戰糧欲犒外軍,下愈怨,反攻昌,昌保子城。鏐將駱團入見,紿言:「奉詔迎公居臨安。」昌信之,全武執昌還,及西江,斬之,投尸於江,傳首京師,夷其族。於是斬偽大臣李邈、蔣瑰等百餘人,發昌先墓,火之。昌敗,猶積糧三百萬斛,金幣大抵五百餘帑,而兵不及萬人。鏐遂為鎮海、鎮東兩軍節度雲。



 贊曰:唐亡,諸盜皆生於大中之朝,太宗之遺德餘澤去民也久矣,而賢臣斥死,庸懦在位,厚賦深刑,天下愁苦。方是時也,天將去唐,諸盜並出,歷五姓,兵未嘗少解,至宋然後天下復安。漢之亡也,天下大亂,至晉然後稍定;晉之亡也,天下大亂,至唐然後復安。治少而亂多者,古今之勢,盛王業業以求治,可少忽哉!



\end{pinyinscope}