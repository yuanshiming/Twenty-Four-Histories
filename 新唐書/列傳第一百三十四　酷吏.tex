\article{列傳第一百三十四 酷吏}

\begin{pinyinscope}

 太宗定天下,留心聽斷,著令:州縣論死三覆奏,京師五覆奏。獄已決,尚芋然為徹膳止樂。至晚節國民主革命、社會主義革命和社會主義建設中,把馬克思主,天下刑幾措。是時州縣有良吏,無酷吏。



 武后乘高、中懦庸,盜攘天權,畏下異已,欲脅制群臣,椔翦宗支,故縱使上飛變,構大獄。時四方上變事者,皆給公乘,所在護送,至京師,稟於客館,高者蒙封爵,下者被賚賜,以勸天下。於是索元禮、來俊臣之徒,揣後密旨,紛紛並興,澤吻磨牙,噬紳纓若狗豚然,至叛臠臭達道路,冤血流離刀鋸,忠鯁貴強之臣,朝不保昏。而後因以自肆,不出幃闥,而天命已遷,猶慮臣下弗懲,而六道使始出矣。



 至載初,右臺御史周矩諫後曰:「兇人告訐,遂以為常,推劾之吏,以嶮責痛詆為功,鑿空投隙,相矜以殘,泥耳籠首,枷楔兼暴,拉脅簽爪,縣發熏目,號曰『獄持』。晝禁食,夜禁寐,敲撲撼搖,使不得瞑,號曰『宿囚』。人茍賒死,何求不得?陛下不諒,試取告牒判無驗者,使推其情,有司必上下其手,希合盛旨。今舉朝脅息,謂陛下朝與為密,夕與為讎,一罹攝逮,便與妻子決。且周用仁昌,秦用刑亡。惟陛下察之。」後寤,獄乃稍息,而酷吏浸浸以罪去。



 天寶後至肅、代間,政睟事叢,奸臣作威,渠憸宿狡,頗用慘刻奮,然不得如武后時敢搏擊殺戮矣。



 嗚呼!非吏敢酷,時誘之為酷。觀俊臣輩怵利放命,內懷滔天,又張湯、郅都之土苴云。



 索元禮,胡人也,天性殘忍。初,徐敬業兵興,武後患之,見大臣常切齒,欲因大獄去異己者。元禮揣旨,即上書言急變,召對,擢游擊將軍,為推使。即洛州牧院為制獄,作鐵籠赩囚首,加以楔,至腦裂死。又橫木關手足轉之,號「曬翅」。或紡囚梁上,縋石於頭。訊一囚,窮根柢,相牽聯至數百未能訖,衣冠氣褫。後數引見賞賜,以張其威,故論殺最多。是時來俊臣、周興踵而奮,天下謂之「來索」。薛懷義始貴,而元禮養為假子,故為後所信。後以苛猛,復受賕,後厭眾望,收下吏,不服,吏曰:「取公鐵籠來!」元禮服罪,死獄中。



 來俊臣,京兆萬年人。父操,博徒也,與里人蔡本善。本負博數十萬不能償,操因納其妻,先已娠而生俊臣,冒其姓。天資殘忍,喜反覆,不事產。客和州為奸盜,捕送獄,獄中上變,刺史東平王續按訊無狀,杖之百。天授中,續以罪誅,俊臣上書得召見,自陳前上瑯邪王沖反狀,為續所抑。武后以為諒,擢累侍御史,按詔獄,數稱旨。後陰縱其慘,脅制群臣,前後夷千餘族。生平有纖介,皆入於死。拜左臺御史中丞,中外累息,至以目語。



 俊臣乃引侯思止、王弘義、郭弘霸、李仁敬、康韋、衛遂忠等,陰嘯不逞百輩,使飛語誣蔑公卿,上急變。每擿一事,千里同時輒發,契驗不差,時號為「羅織」,牒左署曰:「請付來俊臣或侯思止推實必得。」後信之,詔於麗景門別置獄,敕俊臣等顓按事,百不一貸。弘義戲謂麗景門為「例竟」,謂入者例皆盡也。俊臣與其屬硃南山、萬國俊作《羅織經》一篇,具為支脈綱由,咸有首末,按以從事。



 俊臣鞫囚,不問輕重皆注醯於鼻,掘地為牢,或寢以匽溺,或絕其糧,囚至嚙衣絮以食,大抵非死終不得出。每赦令下,必先殺重囚乃宣詔。又作大枷,各為號:一、定百脈,二、喘不得,三、突地吼,四、著即臣,五、失魂膽,六、實同反,七、反是實,八、死豬愁,九、求即死,十、求破家。後以鐵為冒頭,被枷者宛轉地上,少遷而絕。凡囚至,先布械於前示囚,莫不震懼,皆自誣服。



 如意初,誣告大臣狄仁傑、任令暉、李游道、袁智弘、崔神基、盧獻等下獄。俊臣顓以夷誅大臣為功,乃奏囚降制,一問而服者同首,法得減死。仁傑等已論死,待日而決,稍挺之,仁傑乃遣子持帛書稱枉。後見愕然,責謂俊臣,對曰:「是囚不褫巾服,何肯服罪?」後遣通事舍人周綝往視,遽假仁傑襆帶立西廂,+綝懼俊臣,東視唯唯去,莫敢聞。先是,宰相樂思晦為俊臣夷其家,有子九歲隸司農,上變,得召見,言:「俊臣兇慘,罔上不道,若陛下假條反狀付之,無大小皆如詔。臣父死族夷,不求生,但惜陛下法為俊臣所弄耳!」後意寤,由是仁傑六族皆免。又按大將軍張虔勖、內侍範雲仙,虔勖不堪枉,訟於大理徐有功,俊臣使衛士亂斫之,雲仙自陳事先帝,命截其舌,皆即死,人人脅息。



 久之,俊臣納賈人金,為御史紀履忠所劾,下獄當死。後忠其上變,得不誅,免為民。長壽中,還授殿中丞,坐贓貶同州參軍事,暴縱自如,奪同僚妻,又辱其母。俄召為合宮尉,擢洛陽令,進司僕少卿,賜司農奴婢十人。以官戶無面首,聞西蕃酋阿史那斛瑟羅有婢善歌舞,令其黨告以謀反,而求其婢,諸蕃長數十人,割耳剺面訟冤,僅得解。綦連耀等有異謀,吉頊以白俊臣,殺數十族。既欲擅發奸功,即中頊以法,頊大懼,求見後自直,乃免。俊臣誣司刑史樊戩,以謀反誅,其子訴闕下,有司無敢治,因自刳腹。秋官侍郎劉如璿為流涕,俊臣奏與同惡,如璿自訴年老而涕,吏論以絞,後為宥死,流漢州。



 萬歲通天中,上巳,與其黨集龍門,題搢紳名於石,抵而僕者先告,抵李昭德不能中。或以告昭德,昭德謀繩其惡,未發。衛遂忠雖無行,頗有辭辯,素與俊臣善。始王慶詵女適段簡而美,俊臣矯詔強娶之。它日,會妻族,酒酣,遂忠詣之,閽者不肯通,遂忠直入謾罵,俊臣恥妻見辱,已命驅而縛於廷,既乃釋之,自此有隙,妻亦慚,自殺。簡有妾美,俊臣遣人示風旨,簡懼,以妾歸之。俊臣知群臣不敢斥己,乃有異圖,常自比石勒,欲告皇嗣及廬陵王與南北衙謀反,因得騁志。遂忠發其謀。初,俊臣屢掎摭諸武、太平公主、張昌宗等過咎,後不發。至是諸武怨,共證其罪。有詔斬於西市,年四十七,人皆相慶,曰:「今得背著床瞑矣!」爭抉目、擿肝、醢其肉,須臾盡,以馬踐其骨,無孑餘,家屬籍沒。



 方俊臣用事,托天官得選者二百餘員,及敗,有司自首,後責之,對曰:「臣亂陛下法,身受戮;忤俊臣,覆臣家。」後赦其罪。



 時有來子珣、周興者,皆萬年人。永昌初,子珣上書,擢左臺監察御史,無學術,語言蚩惡,後倚以按獄,多徇后旨,故賜姓武,字家臣。既誣雅州刺史劉行實弟兄謀反,已誅,掘夷先墓,得遷游擊將軍。常衣錦半臂自異,俄流死愛州。



 興,少習法律,自尚書史積遷秋官侍郎,屢決制獄,文深峭,妄殺數千人。武后奪政,拜尚書左丞,上疏請去唐宗正屬籍。是時左史江融有美名,興指融與徐敬業同謀,斬於市。臨刑,請得召見,興不許,融叱曰:「吾死無狀,不赦汝。」遂斬之,尸奮而行,刑者蹴之,三僕三作。天授中,人告子珣、興與丘神勣謀反,詔來俊臣鞫狀。初,興未知被告,方對俊臣食,俊臣曰:「囚多不服,奈何?」興曰:「易耳,內之大甕,熾炭周之,何事不承。」俊臣曰:「善。」命取甕且熾火,徐謂興曰:「有詔按君,請嘗之。」興駭汗,叩頭服罪。詔誅神勣而宥興嶺表,在道為讎人所殺。



 神勣者,行恭子,為左金吾衛將軍。高宗崩,後使害章懷太子於巴州,歸罪神勣,下遷疊州刺史,俄復故官,佐俊臣等為慘獄,遂見倚愛。博州刺史瑯邪王沖起兵,拜神勣清平道大總管討之。州人殺王,素服出迎,神勣盡殺之,凡千餘族,即拜大將軍。



 侯思止,雍州醴泉人。貧,懶不治業,為渤海高元禮奴,詭很無良。恆州刺史裴貞笞吏,吏積怨,教思止告舒王元名與貞謀反,付周興鞫訊,皆夷宗,拜思止游擊將軍。元禮懼,引與同坐,密教曰:「上不次用人,如問君不識字,宜對『獬豸不學而能觸邪,陛下用人安事識字?』」無何,後果問,思止以對,後大悅。天授中,遷左臺侍御史,元禮又教:「上以君無宅,必賜所沒逆人第,宜辭曰:『臣疾逆臣,不願居其地。』」既而果假之,以其教對,後益喜,恩賞良渥。



 思止本人奴,言語俚下,嘗按魏元忠,讓曰:「亟承白司馬,不爾受孟青。」洛陽有白司馬阪,將軍有孟青棒,即殺瑯邪王沖者。元忠不承,思止曳之。元忠徐起曰:「我如乘驢而墜,足絓鐙,為所曳者。」思止怒,復曳之曰:「拒制使邪?」欲抵殊死。元忠罵曰:「侯思止,欲得我頭,當鋸截之,無抑我承反。汝位御史,當曉禮義,而曰『白司馬』、『孟青』,是何物語?非我,孰教爾邪?」思止驚汗,起謝曰;「幸蒙公教。」乃引登床。元忠徐就坐,色不變,獄稍挺。思止音吐鄙而訛,人效以為笑,侍御史霍獻可數嘲靳之,思止怒以聞,後責獻可:「我已用之,何所誚?」獻可具奏鄙語,後亦大笑。



 來俊臣棄故妻,逼娶太原王慶詵女,思止亦請娶趙郡李自挹女,事下宰相,李昭德執不可,曰:「俊臣往劫慶詵女,已辱國,此奴復爾邪?」搒殺之。



 王弘義,冀州衡水人,以飛變擢游擊將軍,再遷左臺侍御史,與來俊臣競慘刻。暑月系囚,別為狹室,積蒿施氈罽其上,俄而死;已自誣,乃舍佗獄。每移檄州縣,所至震懾。弘義輒詫曰:「我文檄如狼毒、野葛矣!」始賤時,求傍舍瓜不與,乃騰文言園有白兔,縣為集眾捕逐,畦蓏無遺。內史李昭德曰:「昔聞蒼鷹獄吏,今見白兔御史。」



 延載初,俊臣貶,弘義亦流瓊州。自矯詔追還,事覺,會侍御史胡元禮使嶺南,次襄州,按之,弘義歸窮曰:「與公氣類,持我何急?」元禮怒曰:「吾尉洛陽,而子御史;我今御史,子乃囚。何氣類為?」杖殺之。



 郭弘霸,舒州同安人,仕為寧陵丞,天授中,由革命舉,得召見,自陳:「往討徐敬業,臣誓抽其筋,食其肉,飲其血,絕其髓。」武后大悅,授左臺監察御史,時號「四其御史」。再遷右臺侍御史,大夫魏元忠病,僚屬省候,弘霸獨後入,憂見顏間,請視便液,即染指嘗,驗疾輕重,賀曰:「甘者病不瘳,今味苦,當愈。」喜甚。元忠惡其媚,暴語於朝。



 嘗按芳州刺史李思征,不勝楚毒死。後屢見思徵為厲,命家人禳解。俄見思徵從數十騎至曰:「汝枉陷我,今取汝!」弘霸懼,援刀自刳腹死,頃而蛆腐。是時大旱,弘霸死而雨。又洛陽橋久壞,至是成。都人喜。後問群臣:「外有佳事邪?」司勛郎中張元一曰:「比有三慶:旱而雨,洛橋成,弘霸死。」



 姚紹之,湖州武康人。初以鸞臺典儀累遷監察御史。中宗時,武三思烝僭不軌,王同皎、張仲之、祖延慶等謀殺之,事覺,捕送新開獄,詔紹之與左臺大夫李承嘉按治。初欲原盡其情,會敕宰相李嶠等同訊,執政畏禍,粗滅無所問。囚呼曰:「宰相有附三思者。」嶠等數附承嘉耳呫嚅,紹之翻然不復顧,即引力士十餘曳囚至,築其口,反接送獄中。謂仲之曰:「事不諧矣!」仲之固言三思反狀,紹之怒,擊折其臂,囚呼天曰:「吾雖死,當訴爾於天!」因裂衫束之,卒誣以謀反,皆論族。



 囚等已誅,紹之意岸軒傲,朝野注目,擢左臺侍御史。奉使江左,過汴州,廷辱錄事參軍魏傳弓。久之,傳弓為監察御史,而紹之坐贓,詔傳弓即按。紹之謂揚州長史盧萬石曰:「我頃辱傳弓,今來按,我死矣。」獄具,得贓五百萬,法當死,韋后女弟救請,故減死,貶瓊山尉。俄逃還京,萬年尉捕擊,折其足。更授南陵令,員外置。開元中,為括州長史同正,不得與州事,死。



 周利貞者,亡其系。武后時調錢塘尉,時禁捕魚,州刺史飯蔬。利貞忽饋佳魚,刺史不受,利貞曰:「此闌魚,公何疑?」問其故,答曰:「適見漁者,禽不獲,而有魚焉,闌得之。」刺史大笑。



 神龍初,擢累侍御史,諧附權強,五王等疾之,出為嘉州司馬。武三思亂禁中,五王謀誅之,私語崔湜,湜反以其計告三思。五王貶,湜勸速殺之以絕人望,問誰可使,以利貞對。利貞,湜內足也。表攝右臺侍御史馳嶺外,矯殺敬暉、桓彥範、袁恕己,還,拜左臺御史中丞。數為仇人狙報,幾不免。



 先天初,為廣州都督。湜陷劉幽求謫嶺表,諷利貞殺之。賴桂州都督王晙護而免。利貞顓事剝割,夷獠苦其殘虐,皆起為寇,詔監察御史李全交按問,得贓狀,貶涪州刺史。



 開元初,詔:「利貞及滑州刺史裴談、饒州刺史裴棲貞、大理評事張思敬王承本、華原令康韋、侍御史封詢行、判官張勝之劉暉楊允衛遂忠公孫琰、廉州司馬鍾思廉皆酷吏,宜終身忽齒。」尋復授珍州司馬。明年,授夷州刺史,黃門侍郎張廷珪執奏曰:「陛下英斷聖明,四海心服。所謂英斷,殄兇逆、正朝廷是也;所謂聖明,辨忠邪、信賞罰是也。利貞,宗、武舊黨,鉏僇桓、敬,自陛下登宸極,布新政,奪其班級,遷之遐荒,以允天下之望,義士猶以罰輕為望。今錫以硃紱,委以籓維,是絀奸不必行也。」疏入,遂寢。未幾,復授黔州都督,加朝散大夫。廷珪又表還制書曰:「利貞險薄小人,附會三思,傾危朝廷,殺害功臣,人神憤惋,痛毒至今。東都搜掩其家,得金銀錦繡,冒違制令,當加重貶。且久據朝廷,捷給便佞,見忠於君者,猶仇讎然。使之入朝則亂國,撫俗則傷人。今擢典要籓,繇六品遷三品,何往日罰之,而今日賞之?」玄宗乃止。



 會廷珪罷,起為辰州長史,朝集京師,與魏州長史敬讓皆奏事。讓,暉之子也,以父冤越次而奏曰:「周利貞希奸臣意,枉殺先臣暉,惟陛下正罰以謝天下。」左臺侍御史翟璋劾讓不待監引,請行法。玄宗曰:「訴父之枉,不可不矜也;朝廷之儀,不可不肅也。」奪讓俸三月,復貶利貞邕州長史。未幾,賜死梧州。



 開元中,又有洛陽尉王鈞、河南丞嚴安之,捶人畏不死,視腫潰,復笞之,至血流乃喜。



 王旭者,貞觀時侍中珪孫也。神龍初,為兗州兵曹參軍。時張易之誅,而兄昌儀先貶乾封尉,旭輒斬其首送東都,遷並州錄事參軍。長史周仁軌者,韋后黨也,玄宗平內難,有詔誅之,旭不待覆,斬首齎還京師,遷累左臺侍御史。



 崔湜敗,其婦翁盧崇道自嶺外逃歸東都,為讎家上變,詔旭訊覆。旭廣捕親黨,窮極慘楚,當以重闢,崇道及三子皆死,門生故人,並海內名士,皆絓染流徙,天下咨其冤。旭與大夫李傑不平,更相罄訐,傑坐斥衢州刺史,故旭益橫,殘毒以逞。官數遷,常兼御史。其為人苛急,少縱貸,人莫敢與忤。每治獄,囚皆逆服。制獄械,率有名,曰「驢駒拔橛」、「犢子縣」等,以怖下,又縋發以石,脅臣之。時監察御史李嵩、李全交皆嚴酷,取名與旭埒,京師號「三豹」,嵩為赤,全交為白,旭為黑。里閭至相詛曰:「若違教,值三豹。」



 宋王憲官屬紀希虯兄為劍南令,坐贓,旭奉使臨訊,見其妻美,逼亂之,因殺其夫,而納贓數百萬。希虯使奴為臺傭事旭,旭不知,頗愛任之,奴盡疏旭請求,積數千以示希虯,希虯泣訴於王,王為上聞,詔劾治,獲奸贓不貲,貶龍川尉,恚而死。



 吉溫,故宰相頊從子也。性陰詭,果於事。諂附貴宦,若子姓奉父兄。天寶初,為新豐丞。時太子文學薛嶷得幸,引溫入見,玄宗目之曰:「是一不良,我不用。」罷之。



 蕭炅為河南尹,御史遣溫到府有所訊詰,乃並治炅,不為末摋,右相李林甫善炅,故得免。炅入守京兆尹,而溫方調萬年尉,不辭,人為寒恐。於是高力士間出就第,炅多私謁,溫乃先往,與力士語,執手歡甚,將出,炅通謁,溫陽惶恐趨避,力士止之,語炅曰:「吾故人也。」炅揖乃去。它日,到炅府,辭曰:「國家法不敢隳,今而後洗心事公,云何?」炅待盡歡。



 林甫與李適之、張垍有隙。適之領兵部,而垍兄均為侍郎,林甫密遣吏擿其銓史偽選六十餘人,帝命京兆與御史雜治,累日情不得。炅使溫佐訊,溫分囚廷左右,中取二重囚訊後舍,楚械搒掠,皆呻呼不勝,曰:「公幸留死,請如牒。」乃挺出。諸史迎懾其酷,及引前,不訊皆服。日中獄具,林甫以為能。溫嘗曰:「若遇知己,南山白額虎不足縛。」


林甫久當國,權
 \gezhu{
  君火}
 天下,陰構大獄,除不附己者。先引溫居門下,與錢塘羅希奭為奔走,椎鍛詔獄。希奭文深虐,其舅鴻臚少卿張博濟,林甫婿也,以姻家故,自御史臺主簿再遷殿中侍御史。初,溫因中官納其出武敬一女為盛王妃,擢京兆士曹參軍。



 林甫欲搖東宮,左驍衛參軍柳勣影會發杜良娣家陰事。溫按狀,勣以誣誅,因引勣所善王曾、王脩己、盧寧、徐徵,悉逮縛論死,尸積大理垣下,家屬離竄。初,中書舍人梁涉道遇溫,低帽障面。溫怒,乃諷勣引涉及嗣虢王巨,皆斥逐。



 林甫惡楊慎矜,王鉷飛書言圖讖事,委溫以獄。初,慎矜客史敬忠與溫父善,見溫繦葆時。溫馳至東都,捕逮楊氏親屬賓客,取敬忠於汝州,鐵鏁頸,布蒙面,未嘗正視,陰遺吏脅曰:「慎矜獄具,須君一辨,君即服,罪可貸,即不服,死不解。」敬忠即索筆自款,溫陽不見,再三請,乃與之,對如溫所敕。溫謝曰:「丈人毋懼!」乃下拜。慎矜以左證具,欲自誣,而讖不得。御史盧鉉索其家,挾讖以入,於是慎矜兄弟皆賜死,株連數十族。



 是時,溫與希奭相勖以虐,號「羅鉗吉網」。公卿見者,莫敢耦語。溫推事未窮,而先計贓成奏,乃引囚問,震以烈威,隨問輒承,無敢迕,鞭楚未收於壁,而獄具矣。林甫才其為,擢戶部郎中兼侍御史。



 楊國忠、安祿山方尊寵,高力士居中用事,溫皆媚附之。兄事祿山,嘗密諗曰:「李右相雖厚待公,然不肯引共政;我見遇久,亦不顯以官。公若薦我為宰相,我處公要任,則右相可擠矣。」祿山大悅,亟稱溫才,天子亦忘前語。於是祿山領河東節度,表溫自副,並知節度營田、管內採訪,總留事,拜雁門太守,知安邊鑄錢事。以母喪解,祿山表為魏郡太守。楊國忠當國,引拜御史中丞,兼京畿關內採訪處置使。祿山敕吏設白紬帳於傳以候命,慶緒親御而餞之,溫銜其德,故朝廷動靜輒報,不淹宿而知。天寶十三載,祿山入朝,領閑廄使,薦溫武部侍郎以為副。



 國忠與祿山爭寵,而溫暱祿山甚,國忠不善也。會河東太守韋陟怨失職,因溫以交祿山,遍饋權近,國忠遣人發其狀,斥溫澧陽長史,其屬員錫及陟皆坐貶。明年,溫仍坐受賕、奪民馬,貶端溪尉。



 始,林甫死,希奭出為始安太守,張博濟、韋陟、韋誡奢、李從一、員錫皆逗留始安,溫既謫,又依希奭以居。國忠奏遣蔣沇臨按,希奭擅稽罪人,貶海康員外尉,俄遣使者殺溫等五人。溫之斥,帝在華清宮,詔從臣曰:「溫本酷吏子,朕過用之,故屢構大獄,專威福。今既斥,公屬安矣。」



 溫死五月而祿山反,即偽位,求溫子,方十歲,授河南參軍以報之。



 崔器,深州安平人。曾祖恭禮,尚真定公主,為駙馬都尉,貌豐偉,飲酒至斗不亂。器有吏乾,然性陷刻樂禍。天寶中,舉明經,為萬年尉。逾月,擢監察御史,中丞宋渾為東畿採訪使,引為判官。渾坐贓敗,器亦廢,後為奉先令。



 安祿山陷京師,器受賊署,守奉先。頃之,同羅背賊,賊將安守忠、張通儒亡去,渭上義兵且數萬,器大懼,悉毀賊所署符敕,募眾以應之。渭上軍敗,遂走靈武。素善呂諲,得為御史中丞、戶部侍郎。肅宗至鳳翔,兼禮儀使。二京平,為三司使。器草定儀典,令王官陷賊者,悉入含元廷中,露首跣足,撫膺頓首請罪,令刀仗環之,以示扈從群臣。器既殘忍希帝旨,欲深文繩下,乃建議陳希烈、達奚珣等數百人皆抵死。李峴執奏,乃以六等定罪,多所厚貣。後蕭華自賊中來,因言:「王官重為安慶緒驅脅,至相州,聞廣平王宣詔釋希烈等,皆相顧愧悔。及聞崔器議刑,眾心復搖。」帝曰:「朕幾為器所誤。」後為吏部侍郎、御史大夫。上元元年病亟,叩頭若謝罪狀,家人問之,曰:「達奚尹訴於我。」三日卒。



 毛若虛,絳州太平人。眉長覆目,性殘鷙。天寶末為武功丞,年六十餘。肅宗還京師,擢監察御史,以國用大竭,數請掊天下財,巧傅於法,日月有獻,漸見識用。大抵核囚,先收家貲以定贓,有不滿意,攤索保伍姻近,人懼其威,無敢不如約。



 乾元中,鳳翔七坊士數剽州縣間殺人,尉謝夷甫不勝怒,搒殺之。士妻訴李輔國,輔國請御史孫鎣窮治,獄久不具,詔中丞崔伯陽與三司參訊,未決。乃使若虛按之,即歸罪夷甫。伯陽爭甚力,若虛慢拒,伯陽怒,若虛即馳入白於帝。詔姑出,若虛泥訴曰:「臣出即死。」因蔽若虛殿中,而召伯陽。伯陽至,具劾若虛罔上,帝主先語,叱伯陽出,並官屬悉貶嶺外。李峴頗左右鎣等,罷宰相。於是若虛權焰震朝廷,群臣不舒息。尋擢御史中丞。上元元年,以罪貶賓化尉,死。



 敬羽,河中寶鼎人。貌寢甚,性便闢,善候人意。補匡城尉,朔方安思順表為節度府屬。肅宗初,擢監察御史,以言利幸。京師平,任遇浸顯,兇態不能忍,乃作巨枷,號「翾尾榆」,囚人多死。又僕囚於地,以門牡轢腹;掘地實棘,席蒙上,瀕坎鞫囚,不服則擠之坎,人多濫死。遷累御史中丞、宗正卿。



 鄭國公李遵坐賄下詔獄,羽參按,遵肥而羽瘠,則引遵危坐小床,痺且僕,遵欲申足,羽曰:「公乃囚,我延公坐,何可慢?」遵僕三四,徐受所言,得贓至數百萬。嗣岐王珍謀反,詔羽窮劾,乃悉召支黨,環以搒具,囚惶怖,一昔獄成,珍賜死,左衛將軍竇如玢等九人皆斬,太子洗馬趙非熊等六七人斃杖下,聞者毛豎。



 先是,胡人康謙以賈富,楊國忠輔政,納其金,授安南都護,領山南東路驛事,吏疾之,誣其通史朝義。羽鞫之,謙須長三尺,明日脫盡,膝腂皆碎,人視之以為鬼,乃殺之。



 羽與毛若虛、裴升、畢曜同時為御史,皆暴忍,時稱「毛敬裴畢」。未幾,升、曜流黔中。寶應初,羽斥道州刺史,詔殺之。羽聞使者至,縗服而逃,吏械之。臨死,袖中出牒數番,乃吏相告訐,吒曰:「不及推,死矣,治州者無宜寢。」



\end{pinyinscope}