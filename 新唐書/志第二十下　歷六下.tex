\article{志第二十下 歷六下}

\begin{pinyinscope}

 昭宗時,《宣明歷》施行已久,數亦漸差,詔太子少詹事邊岡與司天少監胡秀林、均州司馬王墀改治新歷,然術一出于岡。岡用算巧陸王心學即南宋陸九淵和明王守仁兩大學說的合稱。陸,能馳騁反覆於乘除間。由是簡捷、超徑、等接之術興,而經制、遠大、衰序之法廢矣。雖籌策便易,然皆冥於本原。其上元七曜,起赤道虛四度。景福元年,歷成,賜名《崇玄》。氣朔、發斂、盈縮、朓朒、定朔弦望、九道月度、交會、入蝕限去交前後,皆《大衍》之舊。餘雖不同,亦殊塗而至者。大略謂:



 策實曰歲實。揲法曰朔實。三元之策曰氣策。四象之策曰平會。一象之策曰弦策。掛限曰閏限。爻數曰紀法。策餘曰歲餘。天中之策曰候策。地中之策曰卦策。貞悔之策曰土王策。辰法,半辰法也。乾實曰周天分。盈縮、朒朓,皆用常氣。盈縮分曰升降。先後曰盈縮。



 凡升降、損益,皆進一等,倍象統乘之,除法而一,為平行率。與後率相減,為差。半之,以加減平行率,為初、末率。倍差,進一等,以象統乘之,除法而一,為日差。以加減初、末為定。以日差累加減,為每日分。凡小餘,皆萬乘之,通法除,為約餘,則以萬為法。又以百約之,為大分,則以百為法。



 凡冬至赤道日度及約餘,以減其宿全度,乃累加次宿,皆為距後積度。滿限九十一度三十一分三十七小分,去之。餘半已下,為初;已上,以減限,為末。皆百四十四乘之。退一等,以減千三百一十五。所得以乘初、末度分,為差。又通初、末度分,與四千五百六十六先相減、後相乘,千六百九十除之,以減差,為定差。再退為分。至後以減、分後以加距後積度,為黃道積度。宿次相減,即其度也。以冬至赤道日度及約餘,依前求定差以減之,為黃道日度。



 凡歲差,十一乘之,又以所求氣數乘之,三千八百八十八而一,以加前氣中積;又以盈縮分盈加、縮減之,命以冬至宿度,即其氣初加時宿度。



 其定朔小餘,如日法四十分之二十九已上,以定朔小餘減日法,餘如晨初餘數已下,進一日。



 岡又作徑術求黃道月度。以蔀率去積年,為蔀周;不盡,為蔀餘。以歲餘乘蔀餘,副之。二因蔀周,三十七除之,以減副。百一十九約蔀餘,以加副。滿周天去之。餘,四因之為分,度母而一為度,即冬至加時平行月。



 又以冬至約餘距午前後分,二百五十四乘之,萬約為分,滿度母為度;午前以加、午後以減加時月,為午中月。自此計日平行十三度十九分度之七。自冬至距定朔,累以平行減之,為定朔午中月。求次朔及弦望,各計日以平行加之。其分以度母除,為約分。



 又四十七除蔀餘,為率差。不盡,以乘七日三分半,副之。九因率差,退一等,為分,以減副。又百約冬至加時距午分,午前加之、午後減之、滿轉周去之,即冬至午中入轉。以冬至距朔日減之,即定朔午中入轉。求次朔及弦望,計日加之。



 各以所入日下損益率乘轉餘,百而一,以損益盈縮積,為定差。以盈加、縮減午中月,為定月。以月行定分乘其日晨昏距午分,萬約為分,滿百為度,以減午中定月,為晨月;加之,為昏月。以朔昏月減上弦昏月,以上弦昏月減望昏月,以望晨月減下弦晨月,以下弦晨月減後朔晨月,各為定程。以相距日均,為平行度分。與次程相減,為差。以加、減平行,為初、末日定行。後少,加為初,減為末;後多,減為初,加為末。減相距日一,均差,為日差。累損、益初日,為每日定行。後多,累益之;後少,累減之。因朔弦望晨昏月,累加之,得每日晨昏月。



 ○晷漏



 各計其日中入二至加時已來日數及餘。如初限已下,為後;已上,以減二至限,餘為前,副之。各以乘數乘之,用減初、末差。所得再乘其副,滿百萬為尺,不滿為寸、為分。夏至後,則退一等。皆命曰晷差。冬至前後,以減冬至中晷:夏至前後,以加夏至中晷;為每日陽城中晷。與次日相減,後多曰息,後少曰消。以冬夏至午前、後約分乘之,萬而一,午前息減、消加;午後息加、消減中晷:為定數也。凡冬至初日,有減無加;夏至初日,有加無減。



 又計二至加時已來至其日昏後夜半日數及餘。冬至後為息,夏至後為消。如一象已下,為初;已上,反減二至限,餘為末。令自相乘,進二位,以消息法除為分,副之。與五百分先相減,後相乘,千八百而一,以加副,為消息數。以象積乘之,百約為分,再退為度。春分後以加六十七度四十分,秋分後以減百一十五度二十分,即各其日黃道去極。與一象相減,則赤道內外也。以消息數,春分後加千七百五十二,秋分後以減二千七百四十八,即各其日晷漏母也。以減五千,為晨昏距子分。



 置晷漏母,千四百六十一乘,而再半之。百約,為距午度。以減半周天,餘為距中度。百三十五乘晷漏母,百約為分,得晨初餘數。凡晷漏,百為刻。不滿,以象積乘之,百約為分,得夜半定漏。



 九服中晷,各於其地立表候之。在陽城北,冬至前候晷景與陽城冬至同者,為差日之始;在陽城南,夏至前候晷景與陽城夏至同者,為差日之始。自差日之始,至二至日,為距差日數也。在至前者,計距前已來日數;至後者,計入至後已來日數。反減距差日,餘為距後日準。求初、末限晷差,各冬至前後以加、夏至前後以減冬夏至陽城中晷,得其地其日中晷。若不足減,減去夏至陽城中晷,即其日南倒中晷也。自餘之日,各計冬夏至後所求日數。減去距冬夏至差日,餘準初、末限入之。又九服所在,各於其地置水漏,以定二至夜刻,為漏率。以漏率乘每日晷漏母,各以陽城二至晷漏母除之,得其地每日晷漏母。



 ○交會



 以四百一乘朔望加時入交常日及約餘,三十除,為度;不滿退除為分,得定朔望入交定積度分。以減周天,命起朔望加時黃道日躔,即交所在宿次。



 凡入交定積度,如半交已下,為在陽歷;已上,減去半交,餘為入陰歷。以定朔望約餘乘轉分,萬約為分,滿百為度;以減入陰、陽歷積度,為定朔望夜半所入。



 如一象已下,為在少象;已上者,反減半交,餘為入老象。皆七十三乘之,退一等。用減千三百二十四,餘以乘老、少象度及餘,再退為分,副之。在少象三十度已下,老象六十一度已上,皆與九十一度先相減、後相乘,五十六除,為差。若少象三十度已上,反減九十一度,及老象六十度已下,皆自相乘,百五除,為差。皆以減副,百約為度,即朔望夜半月去黃道度分。



 凡定朔約餘距午前、後分,與五千先相減、後相乘,三萬除之;午前以減,午後倍之,以加約餘,為日蝕定餘。定望約餘,即為月蝕定餘。晨初餘數已下者,皆四百乘之,以晨初餘數除之,所得以加定望約餘,為或蝕小餘。各以象統乘之,萬約,為半辰之數。餘滿二千四百為刻。不盡退除,為刻分,即其辰刻日蝕有差。



 置其朔距天正中氣積度,以減三百六十五度半,餘以千乘,滿三百六十五度半除為分,曰限心。加二百五十分,為限首。減二百五十分,為限尾。滿若不足,加減一千,退蝕定餘一等。與限首、尾相近者,相減,餘為限內外分。其蝕定餘多於限首、少於限尾者,為外。少於限首、多於限尾者,為內。在限內者,令限內分自乘,百七十九而一,以減六百三十,餘為陰歷蝕差。限外者,置限外分與五百先相減、後相乘,四百四十六而一,為陰歷蝕差。又限內分亦與五百先相減,後相乘,三百一十三半而一,為陽歷蝕差。在限內者,以陽歷蝕差加陰歷蝕差,為既前法。以減千四百八十,餘為既後法。在限外者,以六百一十分為既前法,八百八十分為既後法,其去交度分,在限外陰歷者,以陰歷差減之。不足減者,不蝕。又限外無陽歷。交在限內陰歷者,以陽歷蝕差加之。若在限內陽歷者,以去交度分反減陽歷蝕差。若不足反減者,不蝕。皆為去交定分。如既前法已下者,為既前分;已上者,以減千四百八十,餘為既後分;皆進一位。各以既前、後法除,為蝕分。在既後者,其虧復陰歷也;既前者,陽歷也。



 凡朔望月行定分,日以九百乘,月以千乘,如千三百三十七而一,日以減千八百,月以減二千,餘為泛用刻分。凡月蝕泛用刻,在陽歷以三十四乘,在陰歷以四十一乘,百約,為月蝕既限。以減千四百八十,餘為月蝕定法。其去交度分,如既限已下者,既。已上者,以減千四百八十,餘進一位,以定法約,為蝕分。其蝕五分已下者,為或食;已上為的蝕。



 凡日月食分,泛用刻乘之,千而一,為定用刻。不盡,退除為刻分。既者,以泛為定。各以減蝕甚約餘,為虧初。加之,為復滿。凡蝕甚與晨昏分相近,如定用刻已下者,因相減,餘以乘蝕分,滿定用刻而一,所得以減蝕分,得帶蝕分。



 五星變差曰歲差。陰陽進退差曰盈縮。爻算曰畫度。畫有十二,亦爻數也。推冬至加後時平合日算,曰平合中積。副之,曰平合中星。歲差減中星,曰入歷。有餘者,皆約之。因平合以諸變常積日加中積,常積度加中星、入歷,各其變中積、中星、入歷也。



 凡入歷盈限已下,為盈。已上,去之,為縮。各如畫度分而一,命畫數算外。不滿,以畫下損益乘之,畫度分除之。以損益盈縮積,為定差。盈加、縮減中積,為定積。準求所入氣及月日,加冬至大餘及約餘,為其變大小餘。以命日辰,則變行所在也。亦以盈加、縮減中星,應用躔差。視定積如半交已下,為在盈;已上,去之。為在縮。所得,令半交度先相減、後相乘,三千四百三十五除,為度。不盡退除為分者,亦盈加、縮減之。



 其變異術者,從其術,各為定星。命起冬至黃道日躔,得其變行加時所在宿度也。凡辰星依歷變置算,乃視晨見、晨順在冬至後,夕見、夕順在夏至後,計中積去二至九十一日半已下,令自乘;已上,以減百八十二日半,亦自乘。五百而一,為日。以加晨夕見中積、中星,減晨夕順中積、中星,各為應見不見中積、中星也。凡盈縮定差,熒惑晨見變六十一乘之、五十四除之,乃為定差。太白、辰星再合,則半其差。其在夕見、晨疾二變,則盈減、縮加。凡歲、鎮、熒惑留退,皆用前遲入歷定差。又各視前遲定星,以變下減度減之。餘半交已下,為盈;已上,去之,為縮。又視之,七十三已下三因之。已上減半交,餘二因之,為差。歲、鎮二星退一等,熒惑全用之。在後退,又倍其差。後留,三之。皆滿百為度。以盈加、縮減中積,又以前遲定差盈加、縮減,乃為留退定積。其前後退中星,則以差縮加、盈減,又以前遲定差盈加、縮減,乃為退行定星。



 凡諸變定星迭相減,為日度率。熒惑遲日盈六十、度盈二十四者,所盈日度加疾變日度,為定率。太白退日率,百乘之、二百一十二除之,為留日。以減退日率,為定率。辰星退順日率一等,為留日。以減順日率,為定率。以日均度,為平行。又與後變平行相減,為差。半之,視後多少,以加減平行,為初、末日行分。以初日行分乘其變小餘,萬而一,順減、退加其變加時宿度,為夜半宿度。又減日率一,均差,為日差。視後多少,累損益初日,為每日行分。因夜半宿度,累加減之。得每日所至。



 五星差行,衰殺不倫,皆以諸變類會消息署之。



 起二年頒用,至唐終。



 《景福崇玄歷》演紀上元甲子,距景福元年壬子,歲積五千三百九十四萬七千三百八算外。



 《崇玄》通法萬三千五百。



 歲實四百九十三萬八百一。



 氣策十五,餘二千九百五十,秒一。



 朔實三十九萬八千六百六十三。



 平會二十九,餘七千一百六十三。



 望策十四,餘萬三百三十一半。



 弦策七,餘五千一百六十五太。



 朔虛分六千三百三十七。中盈分五千九百,秒二。



 歲餘七萬八百一。



 閏限三十八萬六千四百二十五,秒二十三。



 象位六。象統二十四。



 候策五,餘九百八十三,秒二十五;秒母七十二。



 卦策六,餘千一百八十,秒一;秒母六十。



 土王策三,餘五百九十,秒一;秒母百二十。



 辰數五百六十二半。



 刻法百三十五。



 周天分四百九十三萬九百六十一,秒二十四。



 歲差百六十,秒二十四。



 周天三百六十五度,虛分三千四百六十一,秒二十四。



 約虛分二千五百六十三,秒八十八。



 除法七千三百五。



 秒母一百。



 二十四氣中積,自冬至,每氣以氣策及約餘累之。



 轉周分三十七萬一千九百八十六,秒九十七。



 轉終日二十七,餘七千四百八十六,秒九十七。



 朔差日一,餘萬三千一百七十六,秒三。



 度母一百。每日累轉分為轉積度。



 秒母一百。



 七日:初數萬一千九百九十六太,末數千五百三少。



 十四日:初數萬四百九十三半,末數三千六半。



 二十一日:初數八千九百九十少,末數四千五百九太。



 二十八日:初數七千四百八十七。



 蔀率九千三十六。



 歲餘六百三十九。



 周天分千七百三十五。



 周天三百六十五度五分。



 度母十九。



 月行定分同轉分。



 平行積度,日累十三度七分。



 轉周二十七日,五十五分半。



 七日:初八十八分,小分八十七半;末十一分,小分十二半。



 十四日:初七十七分太;末二十二分少。



 二十一日:初六十六分,小分六十二半;末三十三分,小分三十七半。



 二十八日:初五十五分半。



 入轉日母一百。



 二至限百八十二日,六十二分,小分二十二分半。



 消息法千六百六十七半。



 一象九十一度三千一百三十一分。



 辰法八刻百六十分。



 昏、明二刻二百四十分。



 象積四百八十。



 冬至前後限五十九日;差二千一百九十五分;乘數十五。



 夏至前後限百二十三日六十二分,小分二十二半;差四千八百八十分;乘數四。



 陽城冬至晷丈二尺七寸一分半。



 夏至晷尺四寸七分,小分八十。



 交終分三十六萬七千三百六十四,秒九千六百七十三。



 交終日二十七,餘二千八百六十四,秒九千六百七十三;約餘二千一百二十二。



 交中日十三,餘八千一百八十二,秒四千八百三十六半;約餘六千六百十一。



 朔差日二,餘四千二百九十八,秒三百二十七;約餘三千一百八十四。



 望策日十四,餘萬二百三十一,秒五千;約餘七千六百五十;四。



 交限日十二,餘六千三十三,秒四千六百七十三;約餘四千五百六十九。



 望差日一,餘二千一百四十九,秒百六十三半;約餘千五百九十二。



 交率二百六十二。



 交數三千三百五十。



 交終三百六十三度七十三分,小分六十四。



 轉終三百七十四度二十八分。



 半交百八十一度八十六分,小分八十二。



 一象九十度,九十三分,小分四十一。



 去交度乘數十一,除數八千六百三十二。



 秒母一萬。



 ○歲星



 終率五百三十八萬四千九百六十二,秒十一。



 平合日三百九十八,餘萬一千九百六十二,秒十一;約餘八千八百六十一。



 盈限二百五度。



 盈畫十七度八分,秒三十三。



 縮限百六十度二十五分,秒六十三太。



 縮畫十三度三十五分,秒四十七。



 歲差百三十三,秒九十二半。



 ○熒惑



 終率千五十二萬八千九百一十六,秒九十一。



 平合日七百七十九,餘萬二千四百一十六,秒九十一;約餘九千一百九十八。



 盈限百九十六度八十分。



 盈畫十六度四十分。



 縮限百六十八度四十五分,秒六十三太。



 縮畫十四度三分,秒八十。



 歲差百三十三,秒四十六。



 ○鎮星



 終率五百一十萬四千八十四,秒五十四。



 平合日三百七十八,餘千八十四,秒五十四;約餘八百三。



 盈限百八十二度六十二分,秒六十三太。



 盈畫十五度二十二分。



 縮限百八十二度六十三分。



 縮畫十五度二十二分。



 歲差百三十二,秒九十四。



 ○太白



 終率七百八十八萬二千六百四十八,秒七十六。



 平合日五百八十三,餘萬二千一百四十八,秒七十六;約餘八千九百九十九。



 再合日二百九十一,餘萬二千八百二十四,秒三十八;約餘九千五百。



 盈限百九十七度十六分。



 盈畫十六度四十三分。



 縮限百六十八度九分,秒六十三太。



 縮畫十四度,秒八十。



 歲差百三十四,秒三十六。



 ○辰星



 終率百五十六萬四千三百七十八,秒九十七。



 平合日百一十五,餘萬一千八百七十八,秒九十七;約餘八千八百。



 再合日五十七,餘萬二千六百八十九,秒四十八半;約餘九千四百。



 盈限百八十二度六十三分。



 盈畫十五度二十二分。



 縮限百八十二度六十二分,秒六十三太。



 縮畫十五度二十一分,秒八十九。



 歲差百三十三,秒六十四。



 ○五星入變歷表略



\end{pinyinscope}