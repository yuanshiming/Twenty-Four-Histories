\article{列傳第八十四 鮑李蕭薛樊王吳鄭陸盧柳崔}

\begin{pinyinscope}

 鮑防,字子慎,襄州襄陽人。少孤窶,強志於學,善辭章。及進士第月。未完稿。編入《列寧全集》第30卷。本文闡述從資本主,歷署節度府僚屬。入為職方員外郎。薛兼訓帥太原,被病,代宗授防少尹、節度行軍司馬,召見,慰遣之。俄知留後,兼太原尹、節度使。人樂其治,詔圖形別殿。入為御史大夫,歷福建、江西觀察使,召拜左散騎常侍。從德宗奉天,進禮部侍郎,封東海郡公。



 貞元元年,策賢良方正,得穆質、裴復、柳公綽、歸登、崔邠、韋純、魏弘簡、熊執易等,世美防知人。時比歲旱,策問陰陽祲沴,質對:「漢故事,免三公,卜式請烹弘羊。」指當時輔政者。右司郎中獨孤愐欲下質,防不許,曰:「使上聞所未聞,不亦善乎?」卒置質高第,帝見策嘉揖。



 初,防與知雜御史竇參遇,導騎不引避,參謫其僕。及為相,防尹京兆,迫使致仕,授工部尚書。防吒曰:「吾與蕭昕子齒,而同昕老,坐宰相餘忿邪!」不得志卒,年六十九,贈太子少保,謚曰宣。防於詩尤工,有所感發,以譏切世敝,當時稱之。與中書舍人謝良弼友善,時號「鮑謝」云。



 李自良,兗州泗水人。天寶亂,往從兗鄆節度使能元皓。以戰多,累授右衛率。從袁傪討賊袁晁,積閥至試殿中監,事浙東薛兼訓節度府。兼訓徙太原,又為牙將。鮑防代總節度事,會回紇入寇,防遣大將焦伯瑜等擊之。自良曰:「寇遠來,難與爭鋒。請築二壘搤歸路,堅壁勿出,求戰不許,師老而墮,其勢易乘。」防不聽。伯瑜戰百井,大敗。由是知名。



 馬燧代防,表為軍候。自良為人勤且有謀,燧倚信之。從討田悅還,攻李懷光河中,數履鋒陷陣,功在諸將右。貞元三年,燧來朝,德宗罷燧兵,以自良代之。自良以事燧久,不敢當,議者多其讓,乃授右龍武大將軍。入謝,帝終以河東近胡,謂曰:「卿於進退寧不有禮?然守北門無易卿者,勉為朕行。」乃以檢校工部尚書充河東節度使。居治九年,舉不愆法,簡儉易循,民不知有軍,上下諧附。卒於官,贈尚書左僕射。



 蕭昕,字中明,梁鄱陽王恢七世孫,世居河南。再中博學宏辭科,調壽安尉,累遷左補闕。哥舒翰為副元帥拒安祿山,闢掌書記,翰敗,儳道走蜀。肅宗立,奉誥冊見行在。歷中書舍人、禮部侍郎。代宗狩陜,昕由武關從帝,擢國子祭酒。建請崇太學以樹教本,帝寤其言,詔群臣有籍於朝及神策六軍子弟隸業者,聽補生員。



 大歷中,持節吊回紇。回紇恃功,廷讓昕曰:「乃中國亂,非我無以平,奈何市馬不時歸我直?」眾失色。昕徐曰:「國家龕定寇難,功雖絲毫不遺賞,況鄰國乎?僕固懷恩,我之叛臣,爾與連禍,又引吐蕃暴我郊甸。天舍其衷,吐蕃敗北,回紇悔懼,叩顙乞和。非天子恤舊功,則只馬不得出塞下,孰為失信者?」回紇大慚,因厚禮昕,遣使者約和。轉工部尚書,封晉陵侯。德宗出奉天,昕年八十餘,步出城。賊求之急,獨竄山谷間,僅至奉天。遷太子少傅,爵郡公,兼禮部尚書,知貢舉。久之,以太子少師致仕。卒,年九十三,贈揚州大都督,謚曰懿。



 昕始薦張鎬、來瑱,在禮部擢杜黃裳、高郢、裴垍。其後鎬興布衣,不數年位將相,瑱為將有威名,黃裳等繼輔政,並為名宰云。



 薛播,河中寶鼎人。曾祖文思,官中書舍人。播早孤,伯母林通經史,善屬文,躬授經諸子及播兄弟,故開元、天寶間,播兄弟七人皆擢進士第,為衣冠光韙。累授殿中侍御史,遷武功、萬年令。溫敏而裕,與人交有常,李棲筠、常袞、崔祐甫並器之。祐甫輔政,拜中書舍人,出為汝州刺史。坐小累,貶泉州,再遷至河南尹。以禮部侍郎卒,贈本曹尚書。



 子公達,擢進士第。佐鳳翔軍。會帥不文,嘗集射,設的高數十尺,令曰:「中者酬錦與金。」一軍莫能中。公達執弓矢揖曰:「請為公歡。」射三發連中,眾大呼笑。帥不喜,乃自免去。復佐河陽軍。以國子助教居東都卒。



 樊澤,字安時,河中人。少孤,依外家客河朔。相衛節度使薛嵩表為堯山令。舉賢良方正,次潼關,雨淖,困不能前。有熊執易者,同舍逆旅,哀之,輟所乘馬,傾褚以濟,自罷所舉。是歲,澤上第,楊炎善之,擢左補闕。



 澤有武力,喜兵法,議者謂有將帥器。嘗召對延英,德宗嘆其論兵「與我意合」。累遷山南東道司馬,就拜節度使。每射獵,諸將憚其材武。數與李希烈確,禽票將張嘉瑜、杜文朝、梁悛之等,賊氣沮縮,遂取唐、隋二州。貞元三年,為荊南節度使。會山南東道嗣曹王皋卒,軍亂,剽居人。以澤威惠著襄、漢間,復徙山南東道,加檢校尚書右僕射。十四年卒,年五十七,贈司空,謚曰成。訃至,帝為撤宴廢朝。



 子宗師,字紹述。始為國子主簿,元和三年,擢軍謀宏遠科,授著作佐郎。歷金部郎中、綿州刺史。徙絳州,治有跡。進諫議大夫,未拜卒。始,宗師家饒於財,悉散施姻舊賓客,妻子告不給,宗師笑不答。然力學多通解,著《春秋傳》、《魁紀公》、《樊子》凡百餘篇,別集尚多。韓愈稱宗師論議平正有經據,嘗薦其材云。



 王緯,字文卿,並州太原人。父之咸,為長安尉,與弟之賁、之奐皆有文。緯舉明經,以書判入等,歷長安尉。大歷中,與李泌俱為路嗣恭江西觀察判官。泌見惡於元載,嗣恭希意欲殺之,緯護解,僅免。泌執政,奏於己有私恩,德宗許為泌報,故進緯給事中。浙西觀察使缺,泌擬緯,帝曰:「是朕為君報德者乎?黃門要地,獨不留議事耶?」對曰:「浙西賦入尤劇,緯清而忠,能惠養民,故請遣之。」制可。初,州縣有韓滉時罰錢未入者十八萬緡,府史請裒為進奉,緯上疏願蠲以紓民,詔聽之。貞元十年,加御史大夫兼諸道鹽鐵轉運使。裴延齡以諸道負錢四百萬緡獻為羨錢,以圖寵,緯奏「此諸州經費」,大忤延齡意,改檢校工部尚書。卒,年七十一,贈太子少保。



 緯居官以清白稱,然好用刻深吏督察其下,條約苛碎,人不聊云。



 吳湊,章敬皇后弟也。由布衣與兄漵一日賜官封皆等,而湊畏太盛,乞解太子詹事,換檢校賓客兼家令。進累左金吾衛大將軍。



 湊才敏銳,而謙畏自將,帝數顧訪,尤見委信。是時,令狐彰、田神功等繼沒,其下乘喪挾兵,輒偃蹇搖亂。湊持節至汴、滑,委悉慰說,裁所欲為奏,各盡其情,亦度朝廷可行者,故軍中驩附。帝才其為,重之。元載當國久,愎狀日肆,帝陰欲誅,未發也,顧左右無可與計,即召湊圖之。俄而收載賜死。於是王縉、楊炎、王昂、韓會、包佶等皆當坐,湊建言:「法有首從,從不應死,一用極刑,虧德傷仁。」縉等繇是得減死。丁后母喪解職。既除,拜右衛將軍。



 德宗初,出為福建觀察使,政勤清,美譽四騰。與宰相竇參有憾,參數加短毀,又言湊風痺不良趨走,帝召還,驗其疾,非是,繇是不直參。擢湊陜虢觀察使,代李翼。翼,參黨也。宣武劉玄佐死,以湊檢校兵部尚書領節度使馳代。未至,汴軍亂,立玄佐子士寧。帝欲遣兵內湊,而參請授士寧以沮湊,還為右金吾衛大將軍。



 貞元十四年夏,大旱,穀貴,人流亡,帝以過京兆尹韓皋,罷之。即召湊代皋,已謝,督視事,明日詔乃下。湊為人強力劬儉,瞿瞿未嘗擾民,上下愛向。京師苦宮市強估取物,而有司附媚中官,率阿從無敢爭。湊見便殿,因言:「中人所市,不便宵民,徒紛紛流議。宮中所須,責臣可辦。若不欲外吏與聞禁中事,宜料中官高年謹信者為宮市令,平賈和售,以息眾喧。」又言:「掌閑、彍騎、飛龍、內園、芙蓉園、禁兵諸司雜供役手,資課太繁,宜有蠲省。」帝輒順可。初,府中易湊貴戚子,不便簿領,每有疑獄,時其將出,則遮湊取決,幸倉卒得容欺。湊叩鞍一視,凡指擿,盡中其弊,初無留思,眾畏服,不意湊精裁遣如此。僚史非大過不榜責,召至廷,詰服原去,其下傳相訓勖,舉無稽事。



 文敬太子、義章公主仍薨,帝悼念,厚葬之,車土治墳,農事廢。湊候帝間徐言,極爭不避。或勸論事宜簡約,不爾,為上厭苦。湊曰:「上明睿,憂勞四海,不以愛所鐘而疲民以逞也。顧左右鉗噤自安耳,若反復啟寤,幸一聽之,則民受賜為不少。撟舌阿旨固善,有如窮民上訴,叵云罪何?」以能進兼兵部尚書。



 及屬病,門不內醫巫,不嘗藥,家人泣請。對曰:「吾以庸謹起田畝,位三品,顯仕四十年,年七十,尚何求?自古外戚令終者可數,吾得以天年歸侍先人地下,足矣!」帝知之,詔侍醫敦進湯劑,不獲已,一飲之。卒,年七十一,贈尚書右僕射,謚曰成。



 先是,街樾稀殘,有司蒔榆其空,湊曰:「榆非人所廕玩。」悉易以槐,及槐成而湊已亡,行人指樹懷之。唐興,後族退居奉朝請者,猶以事失職,而湊任中外,未嘗以罪過罷,為世外戚表云。



 漵子士矩。士矩文學蚤就,喜與豪英游,故人人助為談說。開成初,為江西觀察使,饗宴侈縱,一日費凡十數萬。初至,庫錢二十七萬緡,晚年才九萬,軍用單匱,無所仰。事聞,中外共申解,得以親議,文宗弗窮治也,貶蔡州別駕。諫官執處其罪,不納。於是御史中丞狄兼謨建言:「陛下擢任士矩,非私也;士矩負陛下而治之,亦非私也。請遣御史至江西即訊,使杜江淮它鎮循習意。」帝聽,乃流端州。



 鄭權,汴州開封人。擢進士第,佐涇原節度劉昌府。昌被病入朝,度其軍必亂,以權寬厚容眾,檄主後務。昌去,軍果亂,權挺身冒刃,明諭逆順,殺首亂者,一軍畏伏。德宗方厭兵,籓屯校佐得士心者,皆就命之,權自試參軍拜行軍司馬。擢累河南尹,進拜山南東道節度使,徙領德棣滄景軍。時討李師道,權身將兵出屯,奏置歸化縣,綏納降附。滄州刺史李宗奭數違命,權劾奏,詔追之,宗奭以州兵留己自解。憲宗更以烏重胤代權,滄人懼,共逐宗奭還京師,有詔斬以徇,徙權節度邠寧。或訟宗奭為權所誣,左遷原王傅。改右金吾衛大將軍。



 穆宗立,以左散騎常侍持節為回鶻告哀使,以足疾辭,不許,肩舁就道。權識詣魁然,有閎辯。與可汗爭曲直,持議明壯,虜禮異之。使還,三遷工部尚書。用度豪侈,乃結權幸求鎮守,於是檢校尚書右僕射、嶺南節度使。多裒貲珍,使吏輸送,凡帝左右助力者皆有納焉,人笑之。卒於官。



 陸亙,字景山,蘇州吳人。元和三年,策制科中第,補萬年丞。再遷太常博士。禮史孟真練容典,博士降色訪逮,史倚以倨橫。會將冊皇太子,草儀,真參議偃蹇,亙榜逐之,胥曹失色。遷累戶部郎中、太常少卿。歷兗蔡虢蘇四州刺史、浙東觀察使,徙宣歙。太和八年卒,年七十一,贈禮部尚書。



 亙文明嚴重,所到以善政稱。初為兗州,對延英,具陳:「節度分兵屯屬州,刺史不能制,故易亂。」帝因詔屯士得隸刺史。溫州瀕海,經賊亂,奪官吏半祿代民租,後相沿,更以為奸,亙還官全稟,繩贓罪,吏畏而賴之。



 盧坦,字保衡,河南洛陽人。仕為河南尉。時杜黃裳為尹,召坦立堂下,曰:「某家子與惡人游,破產,盍察之?」坦曰:「凡居官廉,雖大臣無厚畜,其能積財者必剝下以致之。如子孫善守,是天富不道之家,不若恣其不道,以歸於人。」黃裳驚其言,自是遇加厚。



 李復為鄭滑節度使,表為判官。監軍薛盈珍數干政,坦每據理拒之。有善笛者,大將等悅之,詣復請為重職。坦笑曰:「大將久在軍,積勞亟遷,乃及右職。奈何自薄,欲與吹笛少年同列邪?」諸將慚,遽出就坦謝。復病甚,盈珍以甲士五百內牙中,封府庫,舉軍大恐。坦勸止之,軍乃安。復卒,詔姚南仲代之。盈珍以南仲本書生,易之,曰:「是將材邪?」坦私謂人曰:「姚大夫外柔中剛,監軍若侵之,必不受。我留,恐及禍。」乃從復喪歸東都,為壽安令。盈珍果與南仲不相中,幕府多黜死者。



 河南賦限已窮,縣人訴機織未就,坦詣府請申十日。不聽。坦諭縣人第輸,勿顧限,違之不過罰令俸爾。由是知名。累為刑部郎中,兼侍御史知雜事。赤縣尉為臺所按,京兆尹密救之,帝遣中人就釋。坦白中丞,請中覆,中人走以聞,帝曰:「吾固宜先命有司。」遂下詔,乃釋。數月遷中丞。



 初,諸道長吏罷還者,取本道錢為進奉,帝因赦令一切禁止,而山南節度使柳晟、浙西觀察使閻濟美格詔輸獻,坦劾奏,晟、濟美白衣待罪。帝諭坦曰:「二人所獻皆家財,朕已許原,不可失信。」坦曰:「所以布大信者,赦令也。今二臣違詔,陛下奈何以小信失大信乎!」帝曰:「朕既受之,奈何?」坦曰:「出歸有司,以明陛下之德。」帝納之。李錡誅,有司將毀其祖墓,坦上疏諫止。裴均為僕射,將居諫議、常侍上,坦引故事及姚南仲舊比。均曰:「南仲何人?」曰:「守正而不交權幸者。」均怒,遂罷為左庶子。



 數月,拜宣歙池觀察使。初,劉闢婿蘇強坐誅。強兄弘,宦晉州,自免去,人莫敢用者。坦奏「弘有才行,其弟從闢時,距三千里,宜不通謀,今坐廢,非用人意」,因請署判官,帝曰:「使強不誅,尚錄其材,況彼兄耶?」時江淮旱,穀踴貴,或請抑其價。坦曰:「所部地狹,穀來他州,若直賤,穀不至矣,不如任之。」既而商以米坌至,乃多貸兵食出諸市,估遂平。



 再遷戶部侍郎,判度支。或告泗州刺史薛謇為代北水運時,畜異馬,不以獻,事下度支。坦遣吏驗,未反,帝遲之,更遣中人劉泰昕往。坦曰:「事付有司,而又遣宦官,豈有司不足信乎?」三奏,帝乃止。表韓重華為代北水運使,開廢田,列壁二十,益兵三千人,歲收粟二十萬石。



 河毀西受降城,宰相李吉甫議徙天德。坦以為:「城當磧口,得制北狄之要,美水豐草,邊鄣所利。若避河流,不過退徙數里,奈何徇一時省費,墮萬世策邪?天德故城,地壤墝瘠,北倚山,去河遠,烽候無所統接,虜騎唐突,勢不容知,是無故而蹙地二百里,故曰非便。」城使周懷義亦以為言。吉甫不悅,出坦為東川節度。後數月,懷義憂死,燕重旰代之,遂徙天德。師人怨,殺重旰,覆其家。



 初,坦與宰相李絳議多協,絳藉為己助,及坦出半歲而絳罷。治東川,盡蠲山澤鹽井榷率之籍。吳少誠之誅,詔以兵二千屯安州,坦每朔望使人問其父母妻子,視疾病醫藥,故士皆感慰,無逃還者。惟請收軍吏閏月糧助行營,為人所非。元和十二年卒,年六十九,贈禮部尚書。



 舊制,官、階、勛俱三品始聽立戟,後雖轉四品官,非貶削者戟不奪。坦為戶部侍郎時,階朝議大夫,勛護軍,以嘗任宣州刺史三品,請立戟,許之。時鄭餘慶淹練舊章,以為非是。為憲司劾正,詔罰一月俸,奪戟。自貞元以來,立戟十八家不應令,並追正之。



 閻濟美者,第進士,有長者名。貞元末,繇婺州刺史為福建觀察使,徙浙西。為治簡易,居鎮未嘗增常賦。罷浙西也,方在道,見詔而貢獻無所還,故帝為言之。尋出華州刺史,入為秘書監,以工部尚書致仕。卒,謚曰溫。



 柳晟,河中解人。六世祖敏,仕後周為太子太保。父潭,尚和政公主,官太僕卿。晟年十二,居父喪,為身孝。代宗養宮中,使與太子諸王受學於吳大瓘並子通玄,率十日輒上所學。既長,詔大瓘等即家教授。拜檢校太常卿。



 德宗立,晟親信用事。硃泚反,從帝至奉天,自請入京師說賊黨以攜沮之,帝壯其志,得遣。泚將右將軍郭常、左將軍張光晟皆晟雅故,晟出密詔,陳禍福逆順,常奉詔受命,約自拔歸。要籍硃既昌告其謀,泚捕系晟及常外獄,晟夜半坎垣毀械而亡,斷發為浮屠,間歸奉天,帝見,為流涕。乘輿還京師,擢原王府長史。吳通玄得罪,晟上書理其辜,其弟止曰:「天子方怒,無詒悔!」不聽。凡三上書,帝意解,通玄得減死。



 晟累遷將作少監,以護作崇陵,封河東縣子,授山南西道節度使。府兵討劉闢還,未扣城,復詔戍梓州,軍曹怒,脅監軍謀變。晟聞,疾驅入勞士卒,既而問曰:「若等何為成功?」曰:「誅驕不受命者。」晟曰:「若知劉闢得罪天子而誅之,奈何復欲使後人誅若等耶?」士皆免胄拜,從所徙。入為將作監。使回鶻,奉冊立可汗,逆謂曰:「屬聞可汗無禮自大,去信自疆。夫禮信不能為,何足奉中國乎?」可汗諸貴人愕然駭,皆跪伏成禮。還為左金吾衛大將軍,爵為公。卒,年六十九,詔從官臨吊,贈太子少保。



 晟敏於辯,下士樂施。唯自興元入朝,貢獻不如詔,為御史中丞盧坦所劾,憲宗以其賢,置弗暴雲。



 崔戎,字可大,玄韋從孫也。舉明經,補太子校書郎。判入等,調藍田主簿。闢淮南李庸阜府。衛次公代庸阜,憲宗稱戎才,故次公倚成於職。裴度節度太原,署參謀。時王承宗以鎮叛,度請戎往諭,承宗至泣下,乃聽命。入為殿中侍御史,擢累諫議大夫。



 雲南蠻亂成都,詔戎持節劍南為宣撫使。奏罷稅外姜芋錢;當賦錢者率三之,以其一準繒布,優其估以與民;綏招流亡。凡廢若置,公私莫不便之。還拜給事中。出為華州刺史。吏以故事置錢萬緡為刺史私用,戎不取。及去,召吏曰:「籍所置錢享軍,吾重矯激以誇後人也。」徙兗海沂密觀察使,民擁留於道不得行,乃休傳舍,民至抱持取其靴。時詔使尚在,民泣詣使,請白天子丐戎還,使許諾。戎恚責其下,眾曰:「留公而天子怒,不過斬吾二三老人,則公不去矣。」戎夜單騎亡去,民追不及乃止。至兗州,鉏滅奸吏十餘輩,民大喜。歲餘卒,年五十五,贈禮部尚書。



 子雍,字順中,由起居郎出為和州刺史。龐勛以兵劫烏江,雍不能抗,遣人持牛酒勞之,密表其狀。民不知,訴諸朝,宰相路巖素不平,因是傅其罪,賜死宣州。



\end{pinyinscope}