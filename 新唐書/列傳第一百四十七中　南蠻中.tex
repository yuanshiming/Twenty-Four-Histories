\article{列傳第一百四十七中 南蠻中}

\begin{pinyinscope}

 元和三年,異牟尋死,詔太常卿武少儀持節吊祭。子尋閣勸立,或謂夢湊,自稱「膘信」自1922年起,大部分時間都在維也納大學開設「歸納科學的,夷語君也。改賜元和印章。明年死,子勸龍晟立,淫肆不道,上下怨疾。十一年,為弄棟節度王嵯巔所殺,立其弟勸利。詔少府少監李銑為冊立吊祭使。勸利德嵯巔,賜氏蒙,封「大容」,蠻謂兄為「容」。長慶三年,始賜印。是歲死,弟豐祐立。豐祐趫敢,善用其下,慕中國,不肯連父名。穆宗使京兆少尹韋審規持節臨冊。豐祐遣洪成酋、趙龍些、楊定奇入謝天子。



 於是,西川節度使杜元穎治無狀,障候弛沓相蒙,時大和三年也。嵯巔乃悉眾掩邛、戎、巂三州,陷之。入成都,止西郛十日,慰賚居人,市不擾肆。將還,乃掠子女、工技數萬引而南,人懼自殺者不勝計。救兵逐,嵯巔身自殿,至大度河,謂華人曰:「此吾南境,爾去國,當哭。」眾號慟,赴水死者十三。南詔自是工文織,與中國埒。明年,上表請罪。比年使者來朝,開成、會昌間再至。



 大中時,李琢為安南經略使,苛墨自私,以斗鹽易一牛。夷人不堪,結南詔將段酋遷陷安南都護府,號「白衣沒命軍」。南詔發硃弩佉苴三千助守。然朝貢猶歲至,從者多。杜悰自西川入朝,表無多內蠻傔,豐祐怒,即慢言索質子。會宣宗崩,使者告哀。是時,豐祐亦死,坦綽酋龍立,恚朝廷不吊恤;又詔書乃賜故王,以草具進使者而遣。遂僭稱皇帝,建元建極,自號大禮國。懿宗以其名近玄宗嫌諱,絕朝貢。乃陷播州。安南都護李鄠屯武州,咸通元年,為蠻所攻,棄州走。天子斥鄠,以王寬代之。明年,攻邕管,經略使李弘源兵少不能拒,奔巒州。南詔亦引去。詔殿中監段文楚為經略使,數改條約,眾不悅,以胡懷玉代之。南詔知邊人困甚,剽掠無有,不入寇。杜悰當國,為帝謀,遣使者吊祭示恩信,並詔驃信以名嫌,冊命未可舉,必易名乃得封。帝乃命左司郎中孟穆持節往,會南詔陷巂州,穆不行。



 安南桃林人者,居林西原,七綰洞首領李由獨主之,歲歲戍邊。李琢之在安南也,奏罷防冬兵六千人,謂由獨可當一隊,遏蠻之入。蠻酋以女妻由獨子,七綰洞舉附蠻,王寬不能制。三年,以湖南觀察使蔡襲代之,發諸道兵二萬屯守,南詔怛畏不敢出。



 會詔左庶子蔡京經制嶺南,忌襲功,有所欲,沮壞之,乃言:「南方自無虞,武夫幸功,多聚兵耗饋運,請還戍兵惜財用。」襲執不可,願留五千兵,累表不報。即極陳南詔伺隙久,有十必死狀。朝廷昏肆,不省也。京還奏,得意甚,復詔為宣慰安撫使。即建析廣州為嶺南東道,邕州為西道,以龔、象、藤、巖為隸州。乃拜京西道節度使。京褊忮貪克,峻條令,為砲熏刳斮法,下愁毒,為軍中所逐,走藤州,矯制作攻討使印,召鄉兵比道軍攻邕州,不克,眾潰,貶死崖州。以桂管觀察使鄭愚代節度。



 南詔攻交州,進略安南,襲請救,發湖、荊、桂兵五千屯邕州。嶺南韋宙奏:「南詔必襲邕管,不先防近而圖遠,恐搗虛絕糧道,且深入。」乃詔襲按軍海門,詔鄭愚分兵御之。襲請濟師,以山南東道兵千人赴之。南詔酋將楊思僭、麻光高以兵六千薄城而屯。四年正月,攻益急,襲錄異牟尋盟言系矢上射入其營,不答。俄而城陷,襲闔宗死者七十人,幕府樊綽取襲印走度江。荊南兵入東郛苦戰,斬南詔二千級。是夜,蠻遂屠城。有詔諸軍保嶺南,更以秦州經略使高駢為安南都護。帝見輸發頻,罷游幸,不奏樂,宰相杜悰以為非是,止之。



 南詔稍逼邕州,鄭愚自陳非將帥才,願更擇人。會康承訓自義成來朝,乃授嶺南西道節度使,發荊、襄、洪、鄂兵萬人從之。承訓辭兵寡,乃大興諸道兵五萬往。六月,置行交州於海門,進為都護府,調山東兵萬人益戍,以容管經略使張茵鎮之。因命經略安南,茵逗留不敢進。安南之陷,將吏遺人多客伏溪洞,詔所在招還救恤之,免安南賦入二年。



 韋宙請分兵屯容、藤披蠻勢。五年,南詔回掠巂州以搖西南。西川節度使蕭鄴率屬蠻鬼主邀南詔大度河,敗之。明年,復來攻。會刺史喻士珍貪獪,陰掠兩林東蠻口縛賣之,以易蠻金,故開門降。南詔盡殺戍卒,而士珍遂臣於蠻。安南久屯,兩河銳士死瘴毒者十七,宰相楊收議罷北軍,以江西為鎮南軍,募強弩二萬建節度,且地便近,易調發。詔可。夏侯孜亦以張茵懦,不足事,悉以兵授高駢。駢以選士五千度江,敗林邑兵於邕州,擊南詔龍州屯,蠻酋燒貲畜走。酋龍遣楊緝思助酋遷共守安南,以範脆些為安南都統,趙諾眉為扶邪都統。七年六月,駢次交州,戰數勝,士酣鬥,斬其將張詮。李溠龍舉眾萬人降,拔波風三壁。緝思出戰,敗,還走城。士乘之,超堞入,斬酋遷、脆些、諾眉,上首三萬級,安南平。



 初,酋龍遣清平官董成等十九人詣成都,節度使李福將廷見之,成辭曰:「皇帝奉天命改正朔,請以敵國禮見。」福不許。導譯五返,日旰士倦,議不決。福怒,命武士捽辱之,械系於館。俄而劉潼代福節度,即挺其系,表縱還。有詔召成等至京師,見別殿,賜物良厚,慰遣還國。



 明年,酋龍使楊酋慶等來謝釋囚。初,李師望建言:「成都經手忽蠻事,曠日不能決,請析邛、蜀、嘉、眉、黎、雅、巂七州為定邊軍,建節度制機事,近且速。」天子謂然,即詔師望為節度使,治邛州。邛距成都才五舍,巂州最南,去邛乃千里,緩急首尾不相副,而師望利專制,諱不言。裒積無厭,私賄以百萬計。又欲激蠻怒,幸有功,乃殺酋慶等。既而戍士怒,將醢師望以逞,會召還,以竇滂代之。滂沓冒尤不法,誅責苛纖甚師望。時蠻役未興,而定邊已困。



 酋龍怨殺其使,十年,乃入寇。以軍綴青溪關,密引眾伐木開道,徑雪岥,盛夏,卒凍死者二千人。出沐源,窺嘉州,破屬蠻,遂次沐源。滂遣兗海兵五百往戰,一軍覆。酋龍乃身自將,督眾五萬侵巂州,攻青溪關。屯將杜再榮絕大度河走,諸屯皆退保北涯。蠻攻黎州,詭服漢衣,濟江襲犍為,破之。裴回陵、榮間,焚廬舍,掠糧畜。薄嘉州,刺史楊忞與南詔夾江而軍,士攢射,蠻不得進,陰自上游濟,背擊王師,殺忠武將顏慶師,忞走,嘉州陷。明年正月,攻杜再榮,滂自勒兵戰。酋龍遣使者十輩請和,滂信之,語未半,蠻桴爭岸,噪而進。滂不知所為,將自殺,武寧將苗全緒止之,殊死戰,蠻稍卻,滂乃遁,全緒殿而行。黎州陷,人走匿山谷,蠻掠金帛不勝負。入自邛崍關,圍雅州,遂擊邛州。是冬,滂棄州,壁導江,儲貲峙械皆亡矣。



 酋龍進攻成都,次眉州,坦綽杜元忠日夜教酋龍取全蜀。於是西川節度使廬耽遣其副王偃、中人張思廣約和,蠻強之使南面拜,然卒不見酋龍而還。蠻次新津,耽復遣副譚奉祀好言申約,蠻留之。耽畏援軍未集,即飛請天子降大使通好,以紓其深入。懿宗馳遣太僕卿支詳為和蠻使。



 蠻本無謀,不能乘機會鼓行亟驅,但蚍結蠅營,忸鹵剽小利,處處留屯,故蜀孺老得扶攜悉入成都。闍里皆滿,戶所占地不得過一床,雨則冒箕盎自庇。城中井為竭,則共飲摩訶池,至爭捽溺死者,或欻沙取滴飲之。死不能具棺,即共坎瘞。故瀘州刺史楊慶復為耽治攻具、藺石,置牢城兵,八將主之,樹笓格,夜列炬照城,守具雄新。又選悍士三千,號「突將」,為長刀、巨撾斧,分左右番休,日隸於軍,士心侈欲鬥。而酋龍自雙流徐行,內欲報董成之辱,因紿耽請上介至軍議事。耽遣節度副使柳槃往見杜元忠議和,元忠妄言:「帝見耽,請具車蓋葆翣。」槃未能決,還。蠻以三百騎負幄幕來,大言曰:「供帳隋蜀王聽事,為驃信行在。」耽不許,乃馳去。



 蠻稍前,傅外郛。於是游弈使王晝督援兵三千屯毘橋;竇滂亦以其軍自導江來,將與大軍掎角,然戰不甚力,小不勝即保廣漢。自以失定邊,覬成都陷,得薄其罪。會有詔斥徙,軍遂無功。



 耽部將李自孝者,與刺史喻士珍善。士珍臣蠻,自孝陰與賊通,乃說耽城下蒔葦稻,瀦水頹城,舉府不之覺。蠻攻城,自孝守陴,樹麾以自表。麾所指,蠻輒攻之,為下所覺,耽殺自孝以徇。


城左有民樓肆,蠻俯射城中,耽募勇士燒之,器械俱盡。二月,蠻以雲梁、鵝車四面攻,士叫呼,鵝車未至,陴者以巨索鉤系,投膏炬,車焚,箱間蠻卒盡死。耽遣李、張察率突將戰城下,俘斬二千級。蠻徹民鄣落為蓬籠如車
 \gezhu{
  厶大}
 ,下設枕木,推而前,不及城丈,匿蠻其內以穴墉。楊忞以坰貯糞沈潑蠻,蠻不能處;注以鐵液,蓬籠皆火。然南詔負眾,益治器械,斧兵晝夜有聲,將擊錦樓,眾失色。耽遣將出,三面苦戰,蠻引卻。蠻利夜晦,輒薄城,聞呼嘯,眾齊奮。城上施鐵籠千炬,賊來不得隱,屯夫終夜哄,蠻不能侵。



 支詳遣諜與約好,且謂耽毋多殺以速蠻和。是時,傳言救師至,城中合噪開門,士爭出迎軍,南詔搏戰不解。日入,判官程克裕以北門兵二千乘之,蠻乃走。耽猶遺之書,謝不得已交兵,且請和。士脫鎧迎支詳,詳陳所齎,植二旗,署曰「賜雲南幣物」。謂蠻使者曰:「天子詔雲南和解,而兵薄成都,奈何?請退舍撤警以修好。」或勸詳:「蠻多詐,毋入死地。」詳不行。蠻復圍成都,夜穿西北隅,犁旦乃覺,即頹茭火於壖,蠻皆死穴中。以鐵絙曳雲輣僕之,燎作,少選盡,益固守。



 是時,帝遣東川節度使顏慶復為大度河制置、劍南應接使,兵次新都,博野將曾元裕敗蠻兵,斬二千級。南詔騎數萬晨壓官軍以騁,大將宋威以忠武兵戰,斬首五千,獲馬四百尾。南詔退屯星宿山,威進戍沱江。酋龍遣酋望至支詳所請和,詳曰:「今列城固守,北軍望功,歸語而主,審自度。」耽遣銳將趣蠻壁燒攻具,殺二千人,為南詔所躡,卻而潰。蠻聞鳳翔、山南軍且來,乃迎戰毘橋,不勝,趨沱江,為伏士所擊,又敗。城中出突將,夜火蠻營,酋龍、坦綽身督戰。後三日,王師奪升遷梁,蠻大敗,夜燒亭傳,乘火所向,雨矢射王師。威疏軍行,向矢所發叢射之。兩軍不能決,各解去。酋龍知不敵,夜徹營南奔,至雙流,江無梁,計窮,將赴水死,或止之曰:「今北軍與成都兵合,若來追,我無類矣。不如偽和以紓急;不然,死未晚。」乃來請。三日梁成而濟,即斷梁,按隊緩驅。黎州刺史嚴師本收散卒保邛州,酋龍懼,圍二日去。蠻俘華民,必劓耳鼻已,縱之,既而居人刻木為耳鼻者什八。



 慶復之來,眾以其弟慶師死於蠻,必甘心。及成都不破,以己功輕,乃按軍廣溪,縱殘寇,人人切齒。初,成都無隍塹,乃教耽浚隍,廣三丈,作戰棚於埤,列左右屯營,營別五區。區卒五十,蒔皁莢夾壕,後三年合拱。又為大■連弩。自是南詔憚之。



 酋龍年少嗜殺戮,親戚異己者皆斬,兵出無寧歲,諸國更讎忿,屢覆眾,國耗虛。蜀之役,男子十五以下悉發,婦耕以餉軍。



 十四年,坦綽復寇蜀,絙舟大度河以濟,為刺史黃景復擊卻之。眾循河而南,夜桴上流兵,夾攻瀕水諸屯,景復敗,走還黎州。蠻躡追,為景復所敗。會蠻踵來,還攻大度河,僕旗息鼓,請曰:「坦綽欲上書天子白冤事。」戍兵信之,不戰。橋成而濟,黎州陷。遂攻雅州,擊定邊軍,卒潰入邛州。成都大震,人亡入玉壘關,士乘城。坦綽遣使者王保城等四十人齎驃信書遺節度使牛叢,欲假道入朝,請憩蜀王故殿。叢欲許之,楊慶諫曰:「蠻無信,彼禮屈辭甘,詐我也。請斬其使,留二人還書。」叢因責之曰:「詔王之祖,六詔最小夷也。天子錄其勤,合六詔為一,俾附庸成都,名之以國,許子弟入太學,使習華風,今乃自絕王命。且雀蛇犬馬,猶能報德,王乃不如蟲鳥乎?比成都以武備未修,故令爾突我疆埸。然毘橋、沱江之敗,積胔附城,不四年復來。今吾有十萬眾,舍其半未用。以千人為軍。十軍為部,驍將主之。凡部有強弩二百,鎛斧輔之;勁弓二百,越銀刀輔之;長戈二百,掇刀輔之;短矛二百,連錘輔之。又軍四面,面有鐵騎五百。悉收芻薪、米粟、牛馬、犬豕,清野待爾。吾又能以旁騎略爾樵採。我日出以一部與爾戰,部別二番,日中而代;日昃一部至,以夜屯,月明則戰,黑則休,夜半而代。凡我兵五日一殺敵,爾乃晝夜戰,不十日,懵且死矣。州縣繕甲厲兵,掎角相從,皆蠻之深讎,雖女子能齽齘薄賊,況強夫烈士哉!爾祖嘗奴事西蕃,為爾仇家,今顧臣之,何恩讎之戾邪?蜀王故殿,先世之寶宮,非邊夷所宜舍,神怒人憤,驃信且死!」叢猶火郊民室廬觀閣,嚴兵為固守計。坦綽至新津而還,回寇黔中,經略使秦匡謀懼,奔荊南。會僖宗立,遣金吾將軍韓重持節往使。俄攻黎州,景復擊走之。乾符元年,劫略巂、雅間,破黎州,入邛崍關,掠成都,成都閉三日,蠻乃去。



 詔徙天平軍高駢領西川節度使,乃奏:「蠻小醜,勢易制。而蜀道險,館餉窮覂。今左神策所發長武、河東兵多,用度繁廣。且彼皆扼制羌戎,不可以弛備。」詔乃罷長武等兵。駢至不淹月,閱精騎五千,逐蠻至大度河,奪鎧馬,執酋長五十斬之,收邛崍關,復取黎州,南詔遁還。駢召景復責大度河之敗,斬以徇。戍望星、清溪等關。南詔懼,遣使者詣駢結好,而踵出兵寇邊,駢斬其使。初,安南經略判官杜驤為蠻所俘,其妻,宗室女也,故酋龍使奉書丐和。駢答曰:「我且將百萬眾至龍尾城問爾罪。」酋龍大震。自南詔叛,天子數遣使至其境,酋龍不肯拜,使者遂絕。駢以其俗尚浮屠法,故遣浮屠景仙攝使往,酋龍與其下迎謁且拜,乃定盟而還。遣清平官酋望趙宗政、質子三十入朝乞盟,請為兄弟若舅甥。詔拜景仙鴻臚卿、檢校左散騎常侍。駢結吐蕃尚延心、嗢末魯耨月等為間,築戎州馬湖、沐源川、大度河三城,列屯拒險,料壯卒為平夷軍,南詔氣奪。酋龍恚,發疽死,偽謚景莊皇帝。子法嗣,改元貞明、承智、大同,自號大封人。



 法年少,好畋獵酣逸,衣絳紫錦罽,鏤金帶。國事顓決大臣。乾符四年,遣陀西段羌寶詣邕州節度使辛讜請修好,詔使者答報。未幾,寇西川,駢奏請與和親,右諫議大夫柳韜、吏部侍郎崔澹醜其事,上言:「遠蠻畔逆,乃因浮屠誘致,入議和親,垂笑後世。駢職上將,謀乘謬,不可從。」遂寢。蠻使者再入朝議和親,而駢徙荊南,持前請不置。宰相鄭畋、廬攜爭不決,皆賜罷。



 辛讜遣幕府徐云虔攝使者往覘。到善闡府,見騎數十,曳長矛,擁絳服少年,硃繒約發。典客伽陀酋孫慶曰:「此驃信也。」問天子起居,下馬揖客,取使者佩刀視之,自解左右鈕以示。乃除地剚三丈版,命左右馳射。每一人射,法束馬逐以為樂,數十發止。引客就幄,侲子捧瓶盂,四女子侍樂飲,夜乃罷。又遣問客《春秋》大義,送使者還。



 是時,駢徙節鎮海,劾澹等沮議,帝蒙弱不能曉,下詔尉解。西川節度使崔安潛上言:「蠻蓄鳥獸心,不識禮義,安可以賤隸尚貴主,失國家大體?澹等議可用。臣請募義征子,率十戶一保,願發山東銳兵六千戍諸州,比五年,蠻可為奴。」久之,帝手詔問安潛和親事,答曰:「雲南姚州譬一縣,中國何資於彼而遣重使,加厚禮?彼且妄謂朝廷畏怯無能為,脫有它請,陛下何以待之?且天宗近屬,不可下小蠻夷。臣比移書,不言舅甥,黜所僭也。有如蠻使者不復至,當遣諜人伺其隙,可以得志。」



 南詔知蜀強,故襲安南,陷之,都護曾袞奔邕府,戍兵潰。會西川節度使陳敬瑄申和親議,時廬攜復輔政,與豆廬彖皆厚駢,乃譎說帝曰:「陛下初即位,遣韓重使南詔,將官屬留蜀期年,費不貲,蠻不肯迎。及駢節度西川,招嗢末,繕甲訓兵,蠻夷震動,遣趙宗政入獻,見天子,附驃信再拜;雲虔之使,驃信答拜。其於禮不為少。宣宗皇帝收三州七關,平江、嶺以南,至大中十四年,內庫貲積如山,戶部延資充滿,故宰相敏中領西川,庫錢至三百萬緡,諸道亦然。咸通以來,蠻始叛命,再入安南、邕管,一破黔州,四盜西川,遂圍廬耽,召兵東方,戍海門,天下騷動,十有五年,賦輸不內京師者過半,中藏空虛,士死瘴厲,燎骨傳灰,人不念家,亡命為盜,可為痛心!前年留宗政等,南方無虞,及遣還,彼猶冀望。蒙法立三年,比兵不出要防,其蓄力以間我虞。今朝廷府庫匱,甲兵少,牛叢有北兵七萬,首尾奔沖不能救,況安南客戍單寡,涉冬寇禍可虞。誠命使者臨報,縱未稱臣,且伐其謀,外以縻服蠻夷,內得蜀休息也。」帝謂然,乃以宗室女為安化長公主許婚。拜嗣曹王龜年宗正少卿,為雲南使,大理司直徐云虔副之;內常侍劉光裕為雲南內使,霍承錫副之。及還,具言驃信誠款,以為敬瑄功,故進檢校司空,賜一子官。



 法遣宰相趙隆眉、楊奇混、段義宗朝行在,迎公主。高駢自揚州上言:「三人者,南詔心腹也,宜止而鴆之,蠻可圖也。」帝從之。隆眉等皆死,自是謀臣盡矣,蠻益衰。中和元年,復遣使者來迎主,獻珍怪氈罽百床,帝以方議公主車服為解。後二年,又遣布燮楊奇朋友肱來迎,詔檢校國子祭酒張譙為禮會五禮使,徐云虔副之,宗正少卿嗣虢王約為婚使。未行,而黃巢平,帝東還,乃歸其使。



 法死,偽謚聖明文武皇帝。子舜化立,建元中興。遣使款黎州修好,昭宗不答。後中國亂,不復通。



 先是,有時傍、矣川羅識二族,通號「八詔」。時傍母,歸義女也。其女復妻閣羅鳳。初,咩羅皮之敗,時傍入居厓川州,誘上浪千餘,勢稍張,為閣羅所猜,徙置白厓城。後與矣川羅識詣神川都督求自立為詔,謀洩被殺。矣川羅識奔神川,都督送之羅些城。



 蒙巂詔,最大。其王巂輔首死,無子,弟佉陽照立。佉陽照死,子照原立,喪明,子原羅質南詔。歸義欲並國,故歸其子原羅,眾果立之。居數月,使人殺照原,逐原羅,遂有其地。



 越析詔,或謂磨些詔,居故越析州,西距曩蔥山一日行。貞元中,有豪酋張尋求烝其王波沖妻,因殺波沖。劍南節度使召尋求至姚州,殺之。部落無長,以地歸南詔。



 波沖兄了于贈持王所寶鐸鞘東北度瀘,邑於龍佉河,才百里,號雙舍。使部酋楊墮居河東北。歸義樹壁侵於贈,不克。閣羅鳳自請往擊楊墮,破之,於贈投瀘死。得鐸鞘,故王出軍必雙執之。



 浪穹詔,其王豐時死,子羅鐸立。羅鐸死,子鐸羅望立,為浪穹州刺史,與南詔戰,不勝,挈其部保劍川,更稱劍浪。死,子望偏立。望偏死,子偏羅矣立。偏羅矣死,子羅君立。貞元中,南詔擊破劍川,虜羅君,徙永昌。凡浪穹,邆睒、施浪,皛謂之浪人,亦稱「三浪」。



 邆睒詔,其王豐咩,初據邆睒,為御史李知古所殺。子咩羅皮自為邆川州刺史,治大厘城,歸義襲敗之,復入邆睒,與浪穹、施浪合拒歸義。既戰,大敗,歸義奪邆睒,咩羅皮走保野共川。死,子皮羅鄧立。皮羅鄧死,子鄧羅顛立。鄧羅顛死,子顛文托立。南詔破劍川,虜之。徙永昌。



 施浪詔,其王施望欠居矣苴和城。有施各皮者,亦八詔之裔,據石和城。閣羅鳳攻虜之,而施望欠孤立,故與咩羅皮合攻歸義,不勝。歸義以兵脅降其部,施望欠以族走永昌,獻其女遺南詔丐和,歸義許之,度蘭江死。弟望千走吐蕃,吐蕃立為詔,納之劍川,眾數萬。望千死,子千旁羅顛立。南詔破劍川,千旁羅顛走瀘北。三浪悉滅,唯千旁羅顛及矣川羅識子孫在吐蕃。



 贊曰:唐之治不能過兩漢,而地廣於三代,勞民費財,禍所繇生。晉獻公殺嫡,賊二公子,號為暗君。明皇一日殺三庶人,昏蔽甚矣。鳴呼!父子不相信,而遠治閣羅鳳之罪,士死十萬,當時冤之。懿宗任相不明,籓鎮屢畔,南詔內侮,屯戍思亂,龐勛乘之,倡戈橫行。雖兇渠殲夷,兵連不解,唐遂以亡。《易》曰:「喪牛於易。」有國者知戒西北之虞,而不知患生於無備。漢亡於董卓,而兵兆於冀州;唐亡於黃巢,而禍基於桂林。《易》之意深矣!



\end{pinyinscope}