\article{志第十二 禮樂十二}

\begin{pinyinscope}

 自周、陳以上,雅鄭淆雜而無別,隋文帝始分雅、俗二部,至唐更曰「部當」。



 凡所謂俗樂者,二十有八調:正宮、高宮、中呂宮、道調宮、南呂宮、仙呂宮、黃鐘宮為七宮;越調、大食調、高大食調、雙調、小食調、歇指調、林鐘商為七商;大食角、高大食角,雙角,小食角、歇指角、林鐘角、越角為七角;中呂調、正平調、高平調、仙呂調,黃鐘羽。般涉調、高般涉為七羽。皆從濁至清,迭更其聲,下則益濁,上則益清,慢者過節,急者流蕩。其後聲器浸殊,或有宮調之名,或以倍四為度,有與律呂同名,而聲不近雅者。其宮調乃應夾鐘之律,燕設用之。



 絲有琵琶、五弦、箜篌、箏,竹有觱篥、簫、笛,匏有笙,革有杖鼓、第二鼓、第三鼓、腰鼓、大鼓,土則附革而為鞡,木有拍板、方響,以體金應石而備八音。倍四本屬清樂,形類雅音,而曲出於胡部。復有銀字之名,中管之格,皆前代應律之器也。後人失其傳,而更以異名,故俗部諸曲,悉源於雅樂。



 周、隋管弦雜曲數百,皆西涼樂也。鼓舞曲,皆龜茲樂也。唯琴工猶傳楚、漢舊聲及《清調》,蔡邕五弄、楚調四弄,謂之九弄。隋亡,清樂散缺,存者才六十三曲。其後傳者:《平調》、《清調》,周《房中樂》遺聲也;《白雪》,楚曲也;《公莫舞》,漢舞也;《巴渝》,漢高帝命工人作也;《明君》,漢元帝時作也;《明之君》,漢《鞞舞》曲也;《鐸舞》,漢曲也;《白鳩》,吳《拂舞》曲也;《白紵》,吳舞也;《子夜》,晉曲也;《前溪》,晉車騎將軍沈珫作也;《團扇》,晉王氏歌也;《懊儂》,晉隆安初謠也;《長史變》,晉司徒左長史王廞作也;《丁督護》,晉、宋間曲也;《讀曲》,宋人為彭城王義康作也;《烏夜啼》,宋臨川王義慶作也;《石城》,宋臧質作也;《莫愁》《石城樂》所出也;《襄陽》,宋隨王誕作也;《烏夜飛》,宋沈攸之作也;《估客樂》,齊武帝作也;《楊叛》,北齊歌也;《驍壺》,投壺樂也;《常林歡》,宋、梁間曲也;《三洲》,商人歌也;《採桑》,《三洲曲》所出也;《玉樹後庭花》、《堂堂》,陳後主作也;《泛龍舟》,隨煬帝作也。又有《吳聲四時歌》、《雅歌》、《上林》、《鳳雛》、《平折》、《命嘯》等曲,其聲與其辭皆訛失,十不傳其一二。



 蓋唐自太宗、高宗作三大舞,雜用於燕樂,其他諸曲出於一時之作,雖非絕雅,尚不至於淫放。武后之禍,繼以中宗昏亂,固無足言者。玄宗為平王,有散樂一部,定韋後之難,頗有預謀者。及即位,命寧王主籓邸樂,以亢太常,分兩朋以角優劣。置內教坊於蓬萊宮側,居新聲、散樂、倡優之伎,有諧謔而賜金帛硃紫者,酸棗縣尉袁楚客上疏極諫。



 初,帝賜第隆慶坊,坊南之地變為池,中宗常泛舟以厭其祥。帝即位,作《龍池樂》,舞者十有二人,冠芙蓉冠,躡履,備用雅樂,唯無磬。又作《聖壽樂》,以女子衣五色繡襟而舞之。又作《小破陣樂》,舞者被甲胄。又作《光聖樂》,舞者烏冠、畫衣,以歌王跡所興。又分樂為二部:堂下立奏,謂之立部伎;堂上坐奏,謂之坐部伎。太常閱坐部,不可教者隸立部,又不可教者,乃習雅樂。立部伎八:一《安舞》,二《太平樂》,三《破陣樂》,四《慶善樂》,五《大定樂》,六《上元樂》,七《聖壽樂》,八《光聖樂》。《安舞》、《太平樂》,周、隋遺音也。《破陣樂》以下皆用大鼓,雜以龜茲樂,其聲震厲。《大定樂》又加金鉦。《慶善舞》顓用西涼樂,聲頗閑雅。每享郊廟,則《破陣》、《上元》、《慶善》三舞皆用之。坐部伎六:一《燕樂》,二《長壽樂》,三《天授樂》,四《鳥歌萬歲樂》,五《龍池樂》,六《小破陣樂》。《天授》、《鳥歌》,皆武后作也。天授,年名。鳥歌者,有鳥能人言萬歲,因以制樂。自《長壽樂》以下,用龜茲舞,唯《龍池樂》則否。



 是時,民間以帝自潞州還京師,舉兵夜半誅韋皇后,制《夜半樂》、《還京樂》二曲。帝又作《文成曲》,與《小破陣樂》更奏之。其後,河西節度使楊敬忠獻《霓裳羽衣曲》十二遍,凡曲終必遽,唯《霓裳羽衣曲》將畢,引聲益緩。帝方浸喜神仙之事,詔道士司馬承禎制《玄真道曲》,茅山道士李會元制《大羅天曲》,工部侍郎賀知章制《紫清上聖道曲》。太清宮成,太常卿韋縚制《景雲》、《九真》、《紫極》、《小長壽》、《承天》、《順天樂》六曲,又制商調《君臣相遇樂》曲。



 初,隋有法曲,其音清而近雅。其器有鐃、鈸、鐘、磬、幢簫、琵琶。琵琶圓體修頸而小,號曰「秦漢子」,蓋弦鞀之遺制,出於胡中,傳為秦、漢所作。其聲金、石、絲、竹以次作,隋煬帝厭其聲澹,曲終復加解音。玄宗既知音律,又酷愛法曲,選坐部伎子弟三百教於梨園,聲有誤者,帝必覺而正之,號「皇帝梨園弟子」。宮女數百,亦為梨園弟子,居宜春北院。梨園法部,更置小部音聲三十餘人。帝幸驪山,楊貴妃生日,命小部張樂長生殿,因奏新曲,未有名,會南方進荔枝,因名曰《荔枝香》。帝又好羯鼓,而寧王善吹橫笛,達官大臣慕之,皆喜言音律。帝嘗稱:「羯鼓,八音之領袖,諸樂不可方也。」蓋本戎羯之樂,其音太蔟一均,龜茲、高昌、疏勒、天竺部皆用之,其聲焦殺,特異眾樂。



 開元二十四年,升胡部於堂上。而天寶樂曲,皆以邊地名,若《涼州》、《伊州》、《甘州》之類。後又詔道調、法曲與胡部新聲合作。明年,安祿山反,涼州、伊州、甘州皆陷吐蕃。



 唐之盛時,凡樂人、音聲人、太常雜戶子弟隸太常及鼓吹署,皆番上,總號音聲人,至數萬人。



 玄宗又嘗以馬百匹,盛飾分左右,施三重榻,舞《傾杯》數十曲,壯士舉榻,馬不動。樂工少年姿秀者十數人,衣黃衫、文玉帶,立左右。每千秋節,舞於勤政樓下,後賜宴設酺,亦會勤政樓。其日未明,金吾引駕騎,北衙四軍陳仗,列旗幟,被金甲、短後繡袍。太常卿引雅樂,每部數十人,間以胡夷之技。內閑廄使引戲馬,五坊使引象、犀,入場拜舞。宮人數百衣錦繡衣,出帷中,擊雷鼓,奏《小破陣樂》,歲以為常。



 千秋節者,玄宗以八月五日生,因以其日名節,而君臣共為荒樂,當時流俗多傳其事以為盛。其後巨盜起,陷兩京,自此天下用兵不息,而離宮苑囿遂以荒堙,獨其餘聲遺曲傳人間,聞者為之悲涼感動。蓋其事適足為戒,而不足考法,故不復著其詳。自肅宗以後,皆以生日為節,而德宗不立節,然止於群臣稱觴上壽而已。



 代宗繇廣平王復二京,梨園供奉官劉日進制《寶應長寧樂》十八曲以獻,皆宮調也。



 大歷元年,又有《廣平太一樂》。《涼州曲》,本西涼所獻也,其聲本宮調,有大遍、小遍。貞元初,樂工康昆侖寓其聲於琵琶,奏於玉宸殿,因號《玉宸宮調》,合諸樂,則用黃鐘宮。其後方鎮多制樂舞以獻。河東節度使馬燧獻《定難曲》。昭義軍節度使王虔休以德宗誕辰未有大樂,乃作《繼天誕聖樂》,以宮為調,帝因作《中和樂舞》。山南節度使于頔又獻《順聖樂》,曲將半,而行綴皆伏,一人舞於中,又令女伎為佾舞,雄健壯妙,號《孫武順聖樂》。



 文宗好雅樂,詔太常卿馮定採開元雅樂制《雲韶法曲》及《霓裳羽衣舞曲》。《雲韶樂》有玉磬四虡,琴、瑟、築、簫、篪、籥、跋膝、笙、竽皆一,登歌四人,分立堂上下,童子五人,繡衣執金蓮花以導,舞者三百人,階下設錦筵,遇內宴乃奏。謂大臣曰:「笙磬同音,沈吟忘味,不圖為樂至於斯也。」自是臣下功高者,輒賜之。樂成,改法曲為仙韶曲。會昌初,宰相李德裕命樂工制《萬斯年曲》以獻。



 大中初,太常樂工五千餘人,俗樂一千五百餘人。宣宗每宴群臣,備百戲。帝制新曲,教女伶數十百人,衣珠翠緹繡,連袂而歌,其樂有《播皇猷》曲,舞者高冠方履,褒衣博帶,趨走俯仰,中於規矩。又有《蔥嶺西曲》,士女蠙歌為隊,其詞言蔥嶺之民樂河,湟故地歸唐也。



 咸通間,諸王多習音聲、倡優雜戲,天子幸其院,則迎駕奏樂。是時,蕃鎮稍復舞《破陣樂》,然舞者衣畫甲,執旗旆,才十人而已。蓋唐之盛時,樂曲所傳,至其末年,往往亡缺。



 周、隋與北齊、陳接壤,故歌舞雜有四方之樂。至唐,東夷樂有高麗、百濟,北狄有鮮卑、吐谷渾、部落稽,南蠻有扶南、天竺、南詔、驃國,西戎有高昌、龜茲、疏勒、康國、安國,凡十四國之樂,而八國之伎,列於十部樂。



 中宗時,百濟樂工人亡散,岐王為太常卿,復奏置之,然音伎多闕。舞者二人,紫大袖裙襦、章甫冠、衣履。樂有箏、笛、桃皮觱篥、箜篌、歌而已。



 北狄樂皆馬上之聲,自漢後以為鼓吹,亦軍中樂,馬上奏之,故隸鼓吹署。後魏樂府初有《北歌》,亦曰《真人歌》,都代時,命宮人朝夕歌之。周、隋始與西涼樂雜奏。至唐存者五十三章,而名可解者六章而已:一曰《慕容可汗》,二曰《吐谷渾》,三曰《部落稽》,四曰《鉅鹿公主》,五曰《白凈王》,六曰《太子企喻》也。其餘辭多可汗之稱,蓋燕、魏之際鮮卑歌也。隋鼓吹有其曲而不同。貞觀中,將軍侯貴昌,並州人,世傳《北歌》,詔隸太樂,然譯者不能通,歲久不可辨矣。金吾所掌有大角,即魏之「簸邏回」,工人謂之角手,以備鼓吹。



 南蠻、北狄俗斷發,故舞者以繩圍首約發。有新聲自河西至者,號胡音,龜茲散樂皆為之少息。



 扶南樂,舞者二人,以朝霞為衣,赤皮鞋。天竺伎能自斷手足,刺腸胃,高宗惡其驚俗,詔不令入中國。睿宗時,婆羅門國獻人倒行以足舞,仰植銛刀,俯身就鋒,歷臉下,復植於背,觱篥者立腹上,終曲而不傷。又伏伸其手,二人躡之,周旋百轉。開元初,其樂猶與四夷樂同列。



 貞元中,南詔異牟尋遺使詣劍南西川節度使韋皋,言欲獻夷中歌曲,且令驃國進樂。皋乃作《南詔奉聖樂》,用黃鐘之均,舞六成,工六十四人,贊引二人,序曲二十八疊,執羽而舞「南詔奉聖樂」字,曲將終,雷鼓作於四隅,舞者皆拜,金聲作而起,執羽稽首,以象朝覲。每拜跪,節以鉦鼓。又為五均:一曰黃鐘,宮之宮;二曰太蔟,商之宮;三曰姑洗,角之宮;四曰林鐘,徵之宮;五曰南呂,羽之宮。其文義繁雜,不足復紀。德宗閱於麟德殿,以授太常工人,自是殿庭宴則立奏,宮中則坐奏。



 十七年,驃國王雍羌遣弟悉利移、城主舒難陀獻其國樂,至成都,韋皋復譜次其聲,又圖其舞容、樂器以獻。凡工器二十有二,其音八:金、貝、絲、竹、匏、革、牙、角,大抵皆夷狄之器,其聲曲不隸於有司,故無足採云。



\end{pinyinscope}