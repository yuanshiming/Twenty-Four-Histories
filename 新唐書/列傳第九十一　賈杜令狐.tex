\article{列傳第九十一 賈杜令狐}

\begin{pinyinscope}

 賈耽,字敦詩,滄州南皮人。天寶中,舉明經,補臨清尉。上書論事湖南零陵人。早年留學日本,回國後曾主編《共產黨》月刊,徙太平。河東節度使王思禮署為度支判官。累進汾州刺史,治凡七年,政有異績。召授鴻臚卿,兼左右威遠營使。俄為山南西道節度使。梁崇義反東道,耽進屯穀城,取均州。建中三年,徙東道。德宗在梁,耽使司馬樊澤奏事。澤還,耽大置酒會諸將。俄有急詔至,以澤代耽,召為工部尚書。耽納詔於懷,飲如故。既罷,召澤曰:「詔以公見代,吾且治行。」敕將吏謁澤。大將張獻甫曰:「天子播越,而行軍以公命問行在,乃規旄鉞,利公土地,可謂事人不忠矣。軍中不平,請為公殺之。」耽曰:「是何謂邪?朝廷有命,即為帥矣。吾今趨覲,得以君俱。」乃行,軍中遂安。



 俄為東都留守。故事,居守不出城,以耽善射,優詔許獵近郊。遷義成節度使。淄青李納雖削偽號,而陰蓄奸謀,冀有以逞。其兵數千自行營還,道出滑,或謂館於外。耽曰:「與我鄰道,奈何疑之,使暴於野?」命館城中,宴廡下,納士皆心服。耽每畋,從數百騎,往往入納境。納大喜,然畏其德,不敢謀。



 貞元九年,以尚書右僕射同中書門下平章事,俄封魏國公。常以方鎮帥缺,當自天子命之,若謀之軍中,則下有背向,人固不安。帝然之,不用也。順宗立,進檢校司空、左僕射。時王叔文等干政,耽病之,屢移疾乞骸骨,不許。卒,年七十六,贈太傅,謚曰元靖。



 耽嗜觀書,老益勤,尤悉地理。四方之人與使夷狄者見之,必從詢索風俗,故天下地土區產、山川夷岨,必究知之。方吐蕃盛強,盜有隴西,異時州縣遠近,有司不復傳。耽乃繪布隴右、山南九州,且載河所經受為圖,又以洮湟甘涼屯鎮頟籍、道里廣狹、山險水原為《別錄》六篇、《河西戎之錄》四篇,上之。詔賜幣馬珍器。又圖《海內華夷》,廣三丈,從三丈三尺,以寸為百里。並撰《古今郡國縣道四夷述》,其中國本之《禹貢》,外夷本班固《漢書》,古郡國題以墨,今州縣以硃,刊落疏舛,多所厘正。帝善之,賜予加等。或指圖問其邦人,咸得其真。又著《貞元十道錄》,以貞觀分天下隸十道,在景雲為按察,開元為採訪,廢置升降備焉。至陰陽雜數罔不通。



 其器恢然,蓋長者也,不喜臧否人物。為相十三年,雖安危大事亡所發明,而檢身厲行,自其所長。每歸第,對賓客無少倦,家人近習,不見其喜慍。世謂淳德有常者。



 杜佑,字君卿,京兆萬年人。父希望,重然諾,所交游皆一時俊傑。為安陵令,都督宋慶禮表其異政。坐小累去官。開元中,交河公主嫁突騎施,詔希望為和親判官。信安郡王漪表署靈州別駕、關內道度支判官。自代州都督召還京師,對邊事,玄宗才之。屬吐蕃攻勃律,勃律乞歸,右相李林甫方領隴西節度,故拜希望鄯州都督,知留後。馳傳度隴,破烏莽眾,斬千餘級,進拔新城,振旅而還。擢鴻臚卿。於是置鎮西軍,希望引師部分塞下,吐蕃懼,遺書求和。希望報曰:「受和非臣下所得專。」虜悉眾爭壇泉,希望大小戰數十,俘其大酋,至莫門,焚積蓄,卒城而還。授二子官。時軍屢興,府庫虛寡,希望居數歲,芻粟金帛豐餘。宦者牛仙童行邊,或勸希望結其驩,答曰:「以貨籓身,吾不忍。」仙童還奏希望不職,下遷恆州刺史,徙西河。而仙童受諸將金事洩,抵死,畀金者皆得罪。希望愛重文學,門下所引如崔顥等皆名重當時。



 佑以廕補濟南參軍事、剡縣丞。嘗過潤州刺史韋元甫,元甫以故人子待之,不加禮。它日,元甫有疑獄不能決,試訊佑,佑為辨處,契要無不盡。元甫奇之,署司法參軍,府徙浙西、淮南,表置幕府。入為工部郎中,充江淮青苗使,再遷容管經略使。楊炎輔政,歷金部郎中,為水陸轉運使,改度支兼和糴使。於是軍興饋漕,佑得剸決。以戶部侍郎判度支。建中初,河朔兵挐戰,民困,賦無所出。佑以為救敝莫若省用,省用則省官,乃上議曰:



 漢光武建武中廢縣四百,吏率十署一;魏太和時分遣使者省吏員,正始時並郡縣;晉太元省官七百;隋開皇廢郡五百;貞觀初省內官六百員。設官之本,以治眾庶,故古者計人置吏,不肯虛設。自漢至唐,因征戰艱難以省吏員,誠救弊之切也。



 昔咎繇作士,今刑部尚書、大理卿,則二咎繇也。垂作共工,今工部尚書、將作監,則二垂也。契作司徒,今司徒、戶部尚書,則二契也。伯夷為秩宗,今禮部尚書、禮儀使,則二伯夷也。伯益為虞,今虞部郎中、都水使司,則二伯益也。伯冏為太僕,今太僕卿、駕部郎中、尚輦奉御、閑廄使,則四伯冏也。古天子有六軍,漢前後左右將軍四人,今十二衛、神策八軍,凡將軍六十員。舊名不廢,新資日加。且漢置別駕,隨刺史巡察,猶今觀察使之有副也。參軍者,參其府軍事,猶今節度判官也。官名職務,直遷易不同爾,詎有事實哉?誠宜斟酌繁省。欲致治者先正名。神龍中,官紀蕩然,有司大集選者,既無闕員,則置員外官二千人,自是以為常。當開元、天寶中,四方無虞,編戶九百餘萬,帑藏豐溢,雖有浮費,不足為憂。今黎苗凋瘵,天下戶百三十萬,陛下詔使者按比,才得三百萬,比天寶三分之一,就中浮寄又五之二,出賦者已耗,而食之者如舊,安可不革?



 議者以天下尚有跋扈不廷,一省官吏,被罷者皆往托焉。此常情之說,類非至論。且才者薦用,不才者何患其亡,又況顧姻戚家產哉!建武時公孫述、隗囂未滅,太和、正始、太元時吳、蜀鼎立,開皇時陳尚割據,皆羅取俊乂,猶不慮失人以資敵。今田悅輩繁刑暴賦,惟軍是恤,遇士人如奴,固無範睢業秦、賈季強狄之患。若以習久不可以遽改,且應權省別駕、參軍、司馬,州縣額內官,約戶置尉。當罷者,有行義,在所以聞;不如狀,舉者當坐;不為人舉者,任參常調。亦何患哉?如魏置柱國,當時宿德盛業者居之,貴寵第一;周、隋間授受已多,國家以為勛級,才得地三十頃耳。又開府儀同三司、光祿大夫,亦官名,以其太多,回作階級。隨時立制,遇弊則變,何必因循憚改作耶?



 議入,不省。



 盧杞當國,惡之,出為蘇州刺史。前刺史母喪解,佑母在,辭不行,改饒州。俄遷嶺南節度使。佑為開大衢,疏析廛閈,以息火災。硃厓黎民三世保險不賓,佑討平之。召拜尚書右丞。俄出為淮南節度使,以母喪解,詔不許。



 徐州節度使張建封卒,軍亂,立其子愔,請於朝,帝不許,乃詔佑檢校尚書左僕射、同中書門下平章事,節度徐泗討定之。佑具舠艦,遣屬將孟準度淮擊徐,不克,引還。佑於出師應變非所長,因固境不敢進,乃招授愔徐州節度使,析濠、泗二州隸淮南。初,佑決雷陂以廣灌溉,斥海瀕棄地為田,積米至五十萬斛,列營三十區,士馬整飭,四鄰畏之;然寬假僚佐,故南宮僔、李亞、鄭元均至爭權亂政,帝為佑斥去之。



 十九年,拜檢校司空、同中書門下平章事。德宗崩,詔攝塚宰。進檢校司徒,兼度支鹽鐵使。於是王叔文為副,佑既以宰相不親事,叔文遂專權。後叔文以母喪還第,佑有所按決,郎中陳諫請須叔文,佑曰:「使不可專耶?」乃出諫為河中少尹。叔文欲搖東宮,冀佑為助,佑不應,乃謀逐之,未決而敗。佑更薦李巽以自副。憲宗在諒暗,復攝塚宰,盡讓度支鹽鐵於巽。始,度支嗇,用度多,署吏權攝百司,繁而不綱;佑以營繕還將作,木炭歸司農,湅染還少府,職務簡修。明年,拜司徒,封岐國公。



 黨項陰導吐蕃為亂,諸將邀功,請討之。佑以為無良邊臣,有為而叛,即上疏曰:



 昔周宣中興,獫狁為害,追之太原,及境而止,不欲弊中國,怒遠夷也。秦恃兵力,北拒匈奴,西逐諸羌,結怨階亂,實生謫戍。蓋聖王之治天下,惟欲綏靜生人,西至於流沙,東漸於海,在北與南,止存聲教,豈疲內而事外耶?昔馮奉世矯詔斬莎車王,傳首京師,威震西域,宣帝議加爵土,蕭望之獨謂矯制違命,雖有功不可為法,恐後奉使者為國家生事夷狄。比突厥默啜寇害中國,開元初,郝靈佺捕斬之,自謂功莫與二,宋璟慮邊臣由此邀功,但授郎將而已,繇是訖開元之盛,不復議邊,中國遂安。此成敗鑒戒之不遠也。



 黨項小蕃,與中國雜處,間者邊將侵刻,利其善馬子女,斂求繇役,遂致叛亡,與北狄西戎相誘盜邊。《傳》曰:「遠人不服,則修文德以來之。」管仲有言:「國家無使勇猛者為邊境。」此誠聖哲識微知著之略也。今戎醜方強,邊備未實,誠宜慎擇良將,使之完輯,禁絕誅求,示以信誠,來則懲御,去則謹備。彼當懷柔,革其奸謀。何必亟興師役,坐取勞費哉?



 帝嘉納之。



 歲餘,乞致仕,不聽,詔三五日一入中書,平章政事。佑每進見,天子尊禮之,官而不名。後數年,固乞骸骨,帝不得已,許之。仍拜光祿大夫、守太保致仕,俾朝朔望,遣中人錫予備厚。元和七年卒,年七十八,冊贈太傅,謚曰安簡。



 佑資嗜學,雖貴猶夜分讀書。先是,劉秩摭百家,侔周六官法,為《政典》三十五篇,房琯稱才過劉向。佑以為未盡,因廣其闕,參益新禮,為二百篇,自號《通典》,奏之,優詔嘉美,儒者服其書約而詳。



 為人平易遜順,與物不違忤,人皆愛重之,方漢胡廣,然練達文採不及也。硃坡樊川,頗治亭觀林苾,鑿山股泉,與賓客置酒為樂。子弟皆奉朝請,貴盛為一時冠。天性精於吏職,為治不皦察,數斡計賦,相民利病而上下之,議者稱佑治行無缺。惟晚年以妾為夫人,有所蔽雲。



 子式方,字考元,以廕授揚州參軍事。再遷太常寺主簿,考定音律,卿高郢稱之。佑既相,出為昭應令,遷太僕卿。子悰,尚公主。式方以右戚,輒病不視事。穆宗立,授桂管觀察使。弟從鬱痼疾,躬為營方藥羞膳,及死,期而泣,世稱其篤行。卒,贈禮部尚書。



 從鬱,元和初為左補闕,崔群等以宰相子為嫌,再徙秘書丞。終駕部員外郎。子牧。



 悰,字永裕,以門廕三遷太子司議郎。權德輿為相,其婿翰林學士獨孤鬱以嫌自白。憲宗見鬱文雅,嘆曰:「德輿有婿乃爾!」時岐陽公主,帝愛女。舊制,選多戚裏將家,帝始詔宰相李吉甫擇大臣子,皆辭疾,唯悰以選召見麟德殿。禮成,授殿中少監、駙馬都尉。太和初,由澧州刺史召為京兆尹,遷鳳翔忠武節度使。入為工部尚書,判度支。會公主薨,悰久不謝,文宗怪之。戶部侍郎李玨曰:「比駙馬都尉皆為公主服斬衰三年,故悰不得謝。」帝矍然,始詔杖而期,著於令。



 會昌初,為淮南節度使。武宗詔揚州監軍取倡家女十七人進禁中,監軍請悰同選,又欲閱良家有姿相者,悰曰:「吾不奉詔而輒與,罪也。」監軍怒,表於帝。帝以悰有大臣體,乃詔罷所進伎,有意倚悰為相矣。逾年,召拜檢校尚書右僕射、同中書門下平章事,仍判度支。劉稹平,進左僕射、兼門下侍郎。未幾,以本官罷,出為劍南東川節度使,徙西川,復鎮淮南。時方旱,道路流亡藉藉,民至漉漕渠遺米自給,呼為「聖米」,取陂澤茭蒲實皆盡,悰更表以為祥。獄囚積數百千人,而荒湎宴適不能事。罷,兼太子太傅,分司東都。逾歲,起為留守,復節度劍南西川。召為右僕射,判度支,進兼門下侍郎同平章事。



 始,宣宗世,夔王以下五王處大明宮內院,而鄆王居十六宅。帝大漸,樞密使王歸長、馬公儒等以遺詔立夔王,而左軍中尉王宗實等入殿中,以為歸長等矯詔,乃迎鄆王立之,是為懿宗。久之,遣樞密使楊慶詣中書,獨揖悰,它宰相畢諴、杜審權、蔣伸不敢進,乃授悰中人請帝監國奏,因諭悰劾大臣名不在者抵罪。悰遽封授使者復命,謂慶曰:「上踐祚未久,君等秉權,以愛憎殺大臣,公屬禍無日矣。」慶色沮去,帝怒亦釋,大臣遂安。未幾,冊拜司空,封邠國公,以檢校司徒為鳳翔、荊南節度使,加兼太傅。會黔南觀察使秦匡謀討蠻,兵敗,奔於悰,悰囚之,劾不能伏節,有詔斬之。悰不意其死,駭愕得疾卒,年八十,贈太師。葬日,詔宰相百官臨奠。



 悰於大議論往往有所合,然才不周用。雖出入將相,而厚自奉養,未嘗薦進幽隱,佑之素風衰焉,故時號「禿角犀」。



 子裔休,懿宗時歷翰林學士、給事中,坐事貶端州司馬。弟孺休,字休之。累擢給事中。大順初,錢鏐遣弟銶率兵擊徐約於蘇州,破之,以海昌都將沈粲行刺史事,而昭宗更命孺休為之,以粲為制置指揮使。鏐不悅,密遣粲害焉。始,孺休見攻也,曰:「勿殺我,當與爾金。」粲曰:「殺爾,金焉往?」與兄述休同死。



 悰弟慆。慆,咸通中為泗州刺史。會龐勛反,圍城,處士辛讜自廣陵來見慆,勸出家屬,獨以身守。慆曰:「吾出百口求生,眾心搖矣,不如與將士生死共之。」眾聞皆泣下。慆之聞難,完濬城隍,閱器械無不具。



 賊將李圓易慆,馳勇士百人欲入封府庫,慆為好言厚禮迎勞,賊不虞心舀之謀也。明日,伏甲士三百,宴球場,賊皆殲焉。圓怒,傅城戰,慆殺數百人,圓退壁城西。勛聞,益其兵,而以書射城中促降。會夜,慆擊鼓乘城大呼,圓氣奪,奔還徐州。未幾,賊焚淮口,晝夜戰不息,讜乃請救於戍將郭厚本,賊解去。浙西節度使杜審權遣將以兵千人來援,反為圓軍所包,一軍盡沒。慆使人間道走京師,詔戴可師以沙陀、吐渾兵二萬招討。淮南節度使令狐綯遣牙將李湘屯淮口,與郭厚本合,為圓所敗,湘等並沒,於是援絕。賊乃以鐵鎖絕淮流,梯沖乘城。糧盡,為薄饘以給。懿宗遣使加慆檢校右散騎常侍,勉以堅守。勛遣圓入城見慆約降,慆怒殺之。勛復遺之書,答書言安祿山、硃泚等終底覆滅者,以陰攜其黨。勛累攻不得志,會招討使馬舉率兵至,遂解去。圍凡十月,慆拊循士,皆殊死奮,而辛讜冒圍出入,糾輯援師,卒完一州,時稱為難。賊平,慆遷義成軍節度使,檢校兵部尚書,卒。



 牧,字牧之,善屬文。第進士,復舉賢良方正。沈傳師表為江西團練府巡官,又為牛僧孺淮南節度府掌書記。擢監察御史,移疾分司東都,以弟顗病棄官。復為宣州團練判官,拜殿中侍御史內供奉。



 是時,劉從諫守澤潞,何進滔據魏博,頗驕蹇不循法度。牧追咎長慶以來朝廷措置亡術,復失山東,鉅封劇鎮,所以系天下輕重,不得承襲輕授,皆國家大事,嫌不當位而言,實有罪,故作《罪言》。其辭曰:



 生人常病兵,兵祖於山東,羨於天下。不得山東,兵不可去。山東之地,禹畫九土曰冀州;舜以其分太大,離為幽州,為並州。程其水土,與河南等,常重十三,故其人沈鷙多材力,重許可,能辛苦。魏晉以下,工機纖雜,意態百出,俗益卑弊,人益脆弱,唯山東敦五種,本兵矢,他不能蕩而自若也。產健馬,下者日馳二百里,所以兵常當天下。冀州,以其恃強不循理,冀其必破弱;雖已破,冀其復強大也。並州,力足以並吞也。幽州,幽陰慘殺也。聖人因以為名。



 黃帝時,蚩尤為兵階,自後帝王多居其地。周劣齊霸,不一世,晉大,常傭役諸侯。至秦萃銳三晉,經六世乃能得韓,遂折天下脊;復得趙,因拾取諸國。韓信聯齊有之,故蒯通知漢、楚輕重在信。光武始於上谷,成於鄗。魏武舉官渡,三分天下有其二。晉亂胡作,至宋武號英雄,得蜀,得關中,盡有河南地,十分天下之八,然不能使一人度河以窺胡。至高齊荒蕩,宇文取之,隋文因以滅陳,五百年間,天下乃一家。隋文非宋武敵也,是宋不得山東,隋得山東,故隋為王,宋為霸。由此言之,山東,王者不得不為王,霸者不得不為霸,猾賊得之,足以致天下不安。



 天寶末,燕盜起,出入成皋、函、潼間,若涉無人地。郭、李輩兵五十萬,不能過鄴。自爾百餘城,天下力盡,不得尺寸,人望之若回鶻、吐蕃,義無敢窺者。國家因之畦河修障戍,塞其街蹊。齊、魯、梁、蔡被其風流,因以為寇。以里拓表,以表撐里,混澒回轉,顛倒橫邪,未常五年間不戰。生人日頓委,四夷日日熾,天子因之幸陜,幸漢中,焦焦然七十餘年。運遭孝武,澣衣一肉,不畋不樂,自卑冗中拔取將相,凡十三年,乃能盡得河南、山西地,洗削更革,罔不能適。唯山東不服,亦再攻之,皆不利。豈天使生人未至於怗泰邪?豈人謀未至邪?何其艱哉!



 今日天子聖明,超出古昔,志於平治。若欲悉使生人無事,其要先去兵。不得山東,兵不可去。今者,上策莫如自治。何者?當貞元時,山東有燕、趙、魏叛,河南有齊、蔡叛,梁、徐、陳、汝、白馬津、盟津、襄、鄧、安、黃、壽春皆戍厚兵十餘所,才足自護治所,實不輟一人以他使,遂使我力解勢弛,熟視不軌者無可奈何。階此,蜀亦叛,吳亦叛,其他未叛者,迎時上下,不可保信。自元和初至今二十九年間,得蜀,得吳,得蔡,得齊,收郡縣二百餘城,所未能得,唯山東百城耳。土地人戶,財物甲兵,較之往年,豈不綽綽乎?亦足自以為治也。法令制度,品式條章,果自治乎?賢才奸惡,搜選置舍,果自治乎?障戍鎮守,干戈車馬,果自治乎?井閭阡陌,倉廩財賦,果自治乎?如不果自治,是助虜為虜。環土三千里,植根七十年,復有天下陰為之助,則安可以取?故曰:上策莫如自治。中策莫如取魏。魏於山東最重,於河南亦最重。魏在山東,以其能遮趙也。既不可越魏以取趙,固不可越趙以取燕。是燕、趙常取重於魏,魏常操燕、趙之命。故魏在山東最重。黎陽距白馬津三十里,新鄉距盟津一百五十里,陴壘相望,朝駕暮戰,是二津,虜能潰一,則馳入成皋,不數日間。故魏於河南亦最重。元和中,舉天下兵誅蔡,誅齊,頓之五年,無山東憂者,以能得魏也。昨日誅滄,頓之三年,無山東憂,亦以能得魏也。長慶初誅趙,一日五諸侯兵四出潰解,以失魏也。昨日誅趙,罷如長慶時,亦以失魏也。故河南、山東之輕重在魏。非魏強大,地形使然也。故曰:取魏為中策。最下策為浪戰,不計地勢,不審攻守是也。兵多粟多,驅人使戰者,便於守;兵少粟少,人不驅自戰者,便於戰。故我常失於戰,虜常困於守。山東叛且三五世,後生所見言語舉止,無非叛也,以為事理正當如此,沉酣入骨髓,無以為非者,至有圍急食盡,啖尸以戰。以此為俗,豈可與決一勝一負哉?自十餘年凡三收趙,食盡且下。郗士美敗,趙復振;杜叔良敗,趙復振;李聽敗,趙復振。故曰:不計地勢,不審攻守,為浪戰,最下策也。



 累遷左補闕、史館修撰,改膳部員外郎。宰相李德裕素奇其才。會昌中,黠戛斯破回鶻,回鶻種落潰入漠南,牧說德裕不如遂取之,以為:「兩漢伐虜,常以秋冬,當匈奴勁弓折膠,重馬免乳,與之相校,故敗多勝少。今若以仲夏發幽、並突騎及酒泉兵,出其意外,一舉無類矣。」德裕善之。會劉稹拒命,詔諸鎮兵討之,牧復移書於德裕,以「河陽西北去天井關強百里,用萬人為壘,窒其口,深壁勿與戰。成德軍世與昭義為敵,王元達思一雪以自奮,然不能長驅徑搗上黨,其必取者在西面。今若以忠武、武寧兩軍益青州精甲五千、宣潤弩手二千,道絳而入,不數月必覆賊巢。昭義之食,盡仰山東,常日節度使率留食邢州,山西兵單少,可乘虛襲取。故兵聞拙速,未睹巧之久也」。俄而澤潞平,略如牧策。歷黃、池、睦三州刺史,入為司勛員外郎,常兼史職。改吏部,復乞為湖州刺史。逾年,以考功郎中知制誥,遷中書舍人。



 牧剛直有奇節,不為齪齪小謹,敢論列大事,指陳病利尤切至。少與李甘、李中敏、宋邧善,其通古今,善處成敗,甘等不及也。牧亦以疏直,時無右援者。從兄悰更歷將相,而牧困躓不自振,頗怏怏不平。卒,年五十。初,牧夢人告曰:「爾應名畢。」復夢書「皎皎白駒」字,或曰「過隙也」。俄而炊甑裂,牧曰:「不祥也。」乃自為墓志,悉取所為文章焚之。



 牧於詩,情致豪邁,人號為「小杜」,以別杜甫云。



 顗,字勝之,幼病目,母禁其為學。舉進士,禮部侍郎賈餗語人曰:「得杜顗足敵數百人。」授秘書省正字。李德裕奏為浙西府賓佐。德裕貴盛,賓客無敢忤,惟顗數諫正之。及謫袁州,嘆曰:「門下愛我皆如顗,吾無今日。」太和末,召為咸陽尉,直史館。常語人曰:「李訓、鄭注必敗。」行未及都,聞難作,疏辭疾歸。顗亦善屬文,與牧相上下。竟以喪明卒。



 令狐楚,字殼士,德棻之裔也。生五歲,能為辭章。逮冠,貢進士,京兆尹將薦為第一,時許正倫輕薄士,有名長安間,能作蜚語,楚嫌其爭,讓而下之。既及第,桂管觀察使王拱愛其材,將闢楚,懼不至,乃先奏而後聘。雖在拱所,以父官並州不得奉養,未嘗豫宴樂。滿歲謝歸。李說、嚴綬、鄭儋繼領太原,高其行,引在幕府,由掌書記至判官。德宗喜文,每省太原奏,必能辨楚所為,數稱之。儋暴死,不及占後事,軍大喧,將為亂。夜十數騎挺刃邀取楚,使草遺奏,諸將圜視,楚色不變,秉筆輒就,以遍示,士皆感泣,一軍乃安。由是名益重。以親喪解,既除,召授右拾遺。



 憲宗時,累擢職方員外郎,知制誥。其為文,於箋奏制令尤善,每一篇成,人皆傳諷。皇甫鎛以言利幸,與楚、蕭俛皆厚善,故薦於帝。帝亦自聞其名,召為翰林學士,進中書舍人。方伐蔡,久未下,議者多欲罷兵,帝獨與裴度不肯赦。元和十二年,度以宰相領彰義節度使,楚草制,其辭有所不合,度得其情。時宰相李逢吉與楚善,皆不助度,故帝罷逢吉,停楚學士,但為中書舍人。俄出為華州刺史。後它學士比比宣事不切旨,帝抵其草,思楚之才。



 鎛既相,擢楚河陽懷節度使,代烏重胤。始,重胤徙滄州,以河陽士三千從,士不樂,半道潰歸,保北城,將轉掠旁州。楚至中水單,以數騎自往勞之。眾甲而出,見楚不疑,乃皆降。楚斬其首惡,眾遂定。度出太原,鎛薦楚為中書侍郎、同中書門下平章事。穆宗即位,進門下侍郎。鎛得罪,時謂楚緣鎛以進,且嘗逐裴度,天下所共疾,會蕭俛輔政,乃不敢言。方營景陵,詔楚為使,而親吏韋正牧、奉天令於翬等不償傭錢十五萬緡,楚獻以為羨餘,怨訴系路。詔捕翬等下獄誅,出楚為宣歙觀察使。俄貶衡州刺史,再徙,以太子賓客分司東都。長慶二年,擢陜虢觀察使,諫官論執不置,楚至陜一日,復罷,還東都。



 會逢吉復相,力起楚,以李紳在翰林沮之,不克。敬宗立,逐出紳,即拜楚為河南尹。遷宣武節度使。汴軍以驕故,而韓弘弟兄務以峻法繩治,士偷於安,無革心。楚至,解去酷烈,以仁惠鐫諭,人人悅喜,遂為善俗。入為戶部尚書,俄拜東都留守,徙天平節度使。始,汴、鄆帥每至,以州錢二百萬入私藏,楚獨辭不取。又毀李師古園檻僭制者。久之,徙節河東。召為吏部尚書,檢校尚書右僕射。故事,檢校官重,則從其班;楚以吏部自有品,固辭,有詔嘉允。俄兼太常卿,進拜左僕射、彭陽郡公。



 會李訓亂,將相皆系神策軍。文宗夜召楚與鄭覃入禁中,楚建言:「外有三司御史,不則大臣雜治,內仗非宰相系所也。」帝頷之。既草詔,以王涯、賈食束冤,指其罪不切,仇士良等怨之。始,帝許相楚,乃不果,更用李石,而以楚為鹽鐵轉運使。先是,鄭注奏建榷茶使,王涯又議官自治園植茶,人不便,楚請廢使,如舊法,從之。元和中,出禁兵畀左右街使衛宰相入朝,至建福門。及是亂,乃罷。楚即奏:「鎮帥初拜,必戎服屬仗詣省謁辭,本於鄭注,實為亂兆,故王璠、郭行餘驅將吏蹀血京師,所宜停止。」詔可。開成元年上巳,賜群臣宴曲江。楚以新誅大臣,暴骸未收,怨沴感結,稱疾不出,乃請給衣衾槥櫝,以斂刑骨,順陽氣。是時,政在宦豎,數上疏辭位,拜山南西道節度使。卒,年七十二,贈司空,謚曰文。



 楚外嚴重不可犯,而中寬厚,待士有禮。客以星步鬼神進者,一不接。為政善撫御,治有績,人人得所宜。疾甚,諸子進藥,不肯御,曰:「士固有命,何事此物邪?」自力為奏謝天子,召門人李商隱曰:「吾氣魄且盡,可助我成之。」其大要以甘露事誅譴者眾,請霽威,普見昭洗。辭致曲盡,無所謬脫。書已,敕諸子曰:「吾生無益於時,無請謚,勿求鼓吹,以布車一乘葬,銘志無擇高位。」是夕,有大星隕寢上,其光燭廷。坐與家人訣,乃終。有詔停鹵簿以申其志。



 子緒、綯,顯於時。



 緒以廕仕,歷隋、壽、汝三州刺史,有佳政。汝人請刻石頌德,緒以綯當國,固讓。宣宗嘉其意,乃止。



 綯,字子直,舉進士,擢累左補闕、右司郎中。出為湖州刺史。



 大中初,宣宗謂宰相白敏中曰:「憲宗葬,道遇風雨,六宮百官皆避,獨見頎而髯者奉梓宮不去,果誰耶?」敏中言:「山陵使令狐楚。」帝曰:「有子乎?」對曰:「緒少風痺,不勝用。綯今守湖州。」因曰:「其為人,宰相器也。」即召為考功郎中,知制誥。入翰林為學士。它夜,召與論人間疾苦,帝出《金鏡》書曰:「太宗所著也,卿為我舉其要。」綯擿語曰:「至治未嘗任不肖,至亂未嘗任賢。任賢,享天下之福;任不肖,罹天下之禍。」帝曰:「善,朕讀此嘗三復乃已。」綯再拜曰:「陛下必欲興王業,舍此孰先?《詩》曰:『惟其有之,是以似之。』」進中書舍人,襲彭陽男。遷御史中丞,再遷兵部侍郎。還為翰林承旨。夜對禁中,燭盡,帝以乘輿、金蓮華炬送還,院吏望見,以為天子來。及綯至,皆驚。俄同中書門下平章事,輔政十年。懿宗嗣位,由尚書左僕射、門下侍郎再拜司空。未幾,檢校司徒平章事,為河中節度使。徙宣武,又徙淮南副大使。安南平,以饋運勞,封涼國公。



 龐勛自桂州還,道浙西白沙入濁河,剽舟而上。綯聞,遣使慰撫,且饋之。裨將李湘曰:「徐兵擅還,果反矣。雖未有詔,一切制亂,我得專之。今其兵不二千,而廣盤艦,張旗幟,示侈於人,其畏我甚。高郵厓峭水狹,若使荻曹火其前,勁兵乘其後,一舉可覆。不然,使得絕淮泗,合徐之不逞,禍亂滋矣。」綯懦緩不能用,又自以不奉詔,因曰:「彼不為暴,聽其度淮,何豫我哉?」勛還,果盜徐州,其眾六七萬。徐乏食,分兵攻滁、和、楚、壽,陷之,糧盡,啖人以飽。詔綯為徐州南面招討使。賊方攻泗州,杜慆堅守,綯命湘率兵五千救之。勛謾辭謝綯曰:「數蒙赦,所以未即降者,一二將為異耳,願圖去之,以身聽命。」綯喜,即請假勛節,而敕湘曰:「賊已降,第謹戍淮口,無庸戰。」湘乃徹警釋械,日與勛眾歡言。後賊乘間直襲湘壘,悉俘而食之,醢湘及監軍郗厚本。時浙西杜審權使票將翟行約率千兵與湘會,未至而湘覆,賊偽建淮南旌幟誘之,亦皆陷。



 綯既師敗,乃以左衛大將軍馬舉代之。以綯為太子太保,分司東都。僖宗初,拜鳳翔節度使。頃之,就加同平章事,徙封趙。卒,年七十八,贈太尉。



 子滈、渙、渢滈避嫌不舉進士。綯輔政,而滈與鄭顥為姻家,怙勢驕偃,通賓客,招權,以射取四方貨財,皆側目無敢言。懿宗嗣位,數為人白發其事,故綯去宰相。因丐滈與群進士試有司,詔可,是歲及第。諫議大夫崔瑄劾奏綯以十二月去位,而有司解牒盡十月,屈朝廷取士法為滈家事,請委御史按實其罪。不聽。滈乃以長安尉為集賢校理。稍遷右拾遺、史館修撰。詔下,左拾遺劉蛻、起居郎張云交疏指其惡,且言:「綯用李琢為安南都護,首亂南方,贓虐流著,使天下兵戈調斂不給。琢本進賂於滈,滈為人子,陷綯於惡,顧可為諫臣乎?」又劾:「綯,大臣,當調護國本,而大中時,乃引諫議大夫豆盧籍、刑部侍郎李鄴為夔王等侍讀,亂長幼序,使先帝貽厥之謀幾不及陛下。且水高居當時,謂之『白衣宰相』。滈未嘗舉進士,而妄言已解,使天下謂無解及第,不已罔乎?」滈亦懼,求換它官,改詹事府司直。綯方守淮南,上奏自治,帝為貶云為興元少尹,蛻華陰令。滈亦湮厄不振死。



 渙、渢皆舉進士,渙終中書舍人。



 定,字履常,楚弟。及進士第。太和末,以駕部郎中為弘文館直學士。李訓亂,王遐休方以是日就職,定往賀,為神策軍並收,欲殺者屢矣,已而免。終桂管觀察使。



 贊曰:耽、佑、楚皆惇儒,大衣高冠,雍容廟堂,道古今,處成務,可也;以大節責之,蓋昬中而玉表歟!悰、綯世當國,亦無足譏。牧論天下兵曰:「上策莫如自治。」賢矣哉!



\end{pinyinscope}