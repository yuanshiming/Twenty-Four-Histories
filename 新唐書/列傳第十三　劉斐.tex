\article{列傳第十三 劉斐}

\begin{pinyinscope}

 劉文靜字肇仁,自言系出彭城,世居京兆武功。父韶,仕隋戰死,贈上儀同三司。文靜以死難子式,藝術家是高度擴張自我、表現自我的人。主要著作有,襲儀同。侗儻有器略。大業末,為晉陽令,與晉陽宮監裴寂善。寂夜見邏堞傳烽,吒曰:「天下方亂,吾將安舍?」文靜笑曰:「如君言,豪英所資也。吾二人者可終■賤乎?」



 高祖為唐公,鎮太原,文靜察有大志,深自結。既又見秦王,謂寂曰:「唐公子,非常人也,豁達神武,漢高祖、魏太祖之徒歟!殆天啟之也。」寂未謂然。文靜俄坐李密姻屬系獄,秦王顧它無可與計者,私入視之。文靜喜,挑言曰:「喪亂方剡,非湯、武、高、光不能定。」王曰:「安知無其人哉?今過此,非兒女子姁姁相憂者。世道將革,直欲共大計,試為我言之。」文靜曰:「上南幸,兵填河、洛,盜賊蜉結,大連州縣,小阻山澤,以萬數,須真主取而用之。誠能投天會機,奮示藝大呼,則四海不足定也。今汾、晉避盜者皆在,文靜素知其豪傑,一朝號召,十萬眾可得也。加公府兵數萬,一下令,誰不願從?鼓而入關,以震天下,王業成矣。」王笑曰:「君言正與我意合。」乃陰部署賓客。



 將發,恐唐公不從,文靜謀因裴寂開說,於是介寂以交王,遂得進議。及突厥敗高君雅兵,唐公被劾,王遣文靜、寂共說曰:「公據嫌疑之地,勢不圖全。今部將敗,方以罪見收,事急矣,尚不為計乎?晉陽兵精馬強,宮庫饒豐,大事可舉也。今關中空虛,代王弱,賢豪並興,未有適歸,願公引兵西,誅暴除亂。乃受單使囚乎?」唐公私可,會得釋而止。



 王教文靜偽為詔「發太原、西河、雁門、馬邑男子年二十至五十悉為兵,期歲盡集涿郡以伐遼。」繇是人心愁擾,益思亂。文靜謂寂曰:「公聞先發制人,後發制於人乎?唐公名載圖讖,聞天下,尚可怗怗以待禍哉?」又脅寂曰:「公為監,以宮人侍客,公死何憾,奈何累唐公?」寂懼,乃勸起兵。秦王即委文靜、長孫順德等募士,聲討劉武周。文靜與寂作符敕,發宮監庫物佐軍興。會王威、高君雅猜貳,文靜與劉政會為急變書,詣留守告二人反,候唐公與威、君雅視事,文靜進曰:「有密牒言反者。」公目威等省牒,政會不肯,曰:「所告乃副留守,唯唐公得觀。」公驚曰:「詎有是乎?」讀已,語威曰:「人告公等,信乎?」君雅詬曰:「反人欲殺我耳。文靜叱左右執之,由是舉兵。



 唐公乃開大將軍府,以文靜為司馬。文靜勸改旗幟,彰特興,又請與突厥連和,唐公從之。遣文靜使始畢可汗,始畢曰:「唐公兵何事而起?」文靜曰:「先帝廢塚嗣以授後主,故大亂。唐公,國近戚,懼毀王室,起兵黜不當立者。願與突厥共定京師,金幣、子女盡以歸可汗。」始畢大喜,即遣二千騎隨文靜至,又獻馬千匹。公喜曰:「非君何以致之?」尋拒屈突通於潼關,與其將桑顯和苦鬥,死者數千。文靜度顯和軍怠,以奇兵從後掩之,顯和敗績。通兵尚數萬,欲引而東,文靜命將追執之,徇新安以西,皆下。轉大丞相府司馬,進光祿大夫、魯國公。



 唐公踐天子位,擢納言。時多引貴臣共榻,文靜諫曰:「今率土莫不臣,而延見群下,言尚稱名。帝坐嚴尊,屈與臣子均席,此王導所謂太陽俯同萬物者也。」帝曰:「我雖應天受命,宿昔之好何可忘?公其無嫌。」薛舉寇涇州,以元帥府長史與司馬殷開山出戰,大敗,奔還京師,坐除名。與討仁杲,平之,復爵邑,授民部尚書、陜東道行臺左僕射。從秦王鎮長春宮。



 文靜自以材能過裴寂遠甚,又屢有軍功,而寂獨用故舊恩居其上,意不平。每論政多戾駁,遂有隙。嘗與弟散騎常侍文起飲酣,有怨言,拔刀擊柱曰:「當斬寂!」會家數有怪,文起憂,召巫夜被發銜刀為禳厭。文靜妾失愛,告其兄上變,遂下吏。帝遣裴寂、蕭瑀訊狀,對曰:「昔在大將軍府,司馬與長史略等。今寂已僕射,居甲第,寵賚不貲。臣官賞等眾人,家無贏,誠不能無少望。」帝曰:「文靜此言,反明甚。」李綱、蕭瑀明其不反;秦王亦以文靜首決非常計,事成乃告寂,今任遇弗等,故怨望,非敢反,宜賜全宥。帝素疏忌之,寂又言:「文靜多權詭,而性猜險,忿不顧難,醜言怪節已暴驗,今天下未靖,恐為後憂。」帝遂殺之,年五十二。文起亦死,籍其家。文靜臨刑,撫膺曰:「高鳥盡,良弓藏,果不妄。」貞觀三年,追復官爵,以子樹義襲魯國公,詔尚主。然怨父不得死,謀反,誅。



 裴寂,字玄真,蒲州桑泉人。幼孤,兄鞠之。年十四,補郡主簿。及長,偉容貌,涉知書傳。隋開皇中,調左親衛。家貧,徙步走京師,過華山祠,祈神自卜,夜夢老人謂曰:「君年逾四十當貴。」



 大業中,為齊州司戶參軍,歷侍御史,晉陽宮副監。唐公雅與厚,及留守太原,契分愈密,至蒲酒通晝夜。秦王與劉文靜方建大計,未敢白公,以寂最厚善,乃同私錢數百萬餉龍山令高斌廉,俾與寂博,陽不勝,寂得進多,大喜,日滋暱。太宗以情告之,許諾。寂嘗以宮人侍唐公,恐事發誅,閑飲酣,乃白秦王將舉兵狀,因言:「今盜遍天下,城闔外即戰場,雖徇小節,猶不脫死。若舉義師,不特免禍,且就大功。」唐公然所計。兵起,寂進宮女五百,米九百萬斛,雜彩五萬段,鎧四十萬首。



 大將軍府建,為長史。下臨汾,封聞喜縣公。至河東,屈突通未下,而三輔豪傑多歸者。唐公欲先取京師,恐通掎其後,猶豫未決,寂說曰:「今通據蒲關,未下而西,我腹背支敵,敗之符也。不若破通而後趨京師。」秦王曰:「不然。兵尚權,權利於速。今乘機度河以奪其心。且關中群盜處處屯結,疑力相杖,易以招懷,撫而有之,眾附兵強,何向不克。通自守賊耳,庸能患我?一失其機,勝負未可計也。」唐公兩從之,留兵圍蒲,而遣秦王入關。長安平,賜寂田千頃、甲第一區,物四萬段,遷大丞相府長史,進魏國公,邑三百戶。



 隋帝禪位,公固讓,寂開陳符命以勸,又督太常具儀、撰日。唐公即位,曰:「使我至此者,公也。」拜尚書右僕射,賜服玩不貲,詔尚食日給御膳,視朝必引與同坐,入閤則延臥內,言無不從,呼為裴監,不名也,貴震當世。



 武德二年,劉武周寇太原,守將數困,寂請行,授晉州道行軍總管討賊,以便宜決事。賊將宋金剛據介州,寂屯度索原,賊埭水上流,寂徙屯,為賊所搏,兵大潰,死亡略盡。寂晝夜馳抵平陽,鎮戍皆沒。上書謝罪,高祖薄其過,下詔慰諭,俾留撫河東。寂無它才,惟飛檄郡縣,促入屯壘相保贅,焚積聚,人益惴駭思亂。夏人呂崇茂殺其令,反,為賊守,寂攻之,復為所敗。召還,帝責讓良久,以屬吏,俄釋之,遇待如初。



 帝每巡幸,必委以居守。麟州刺史韋雲起告寂反,按訊無狀,帝謂曰:「朕有天下,公推轂成之也,容有貳哉?所以訊吏,欲天下人信公不反耳。」詔三貴妃齎玉食寶器宴其家,經宿去。帝嘗從容誇語曰:「前王多興細微,間關行陣而後成功。我家隴西舊族,世姻婭帝室,一呼倡義,不三月有天下,公復華胄,職宦光顯,非劉季亭長、蕭曹刀筆吏比也。我與公無愧焉。」四年,改鑄錢,賜一爐得自鑄。又聘其女為趙王元景妃。遷左僕射。帝置酒含章殿,歡甚,寂頓首曰:「始陛下發太原,約天下已定,許上印綬。今四海妥安,願賜骸骨歸田里。」帝泣下曰:「未也,要當相與老爾。公為宗臣,我為太上皇,逍遙晚歲,不亦善乎!」九年,冊拜司空,遣尚書員外郎日一人直第。貞觀初,太宗親郊,命寂與長孫無忌升金輅,寂辭,帝曰:「公有佐命勛,無忌宣力王室,非二人誰可參乘者?」遂同載歸。



 浮屠法雅坐妖言,辭連寂,坐免官,削封邑半,歸故郡。寂請留京師,帝讓曰:「公勛不稱位,徙以恩澤居第一。武德之政,間或弛紊,職公為之。今歸掃墳墓,尚何辭?」寂遂歸。未幾,汾陰狂男子謂寂奴曰:「公有天分。」監奴白寂,寂惶懼不敢聞,遣監奴殺所言者。奴盜寂封邑錢百萬,寂捕急,遂上變。帝怒曰:「寂有死罪四:為三公,與妖人游,一也;既免官,乃恚稱國家之興皆其所謀,二也;匿妖人言不奏,三也;專殺以滅口,四也。我戮之非無辭。」議者多請貸,乃放靜州。會山羌反,或言劫寂為主。帝曰:「國家於寂有恩,必不爾。」既而寂率家僮破賊。帝念寂功,詔入朝,會卒,年六十。贈相州刺史、工部尚書、河東郡公。子律師嗣,尚臨海長公主,終汴州刺史。律師子承先,武后時為殿中監,酷吏殺之。



 始,高祖論太原首功,詔尚書令秦王、尚書左僕射裴寂、納言劉文靜恕二死;左驍衛大將軍長孫順德、右驍衛大將軍劉弘基、右屯衛大將軍竇琮、左翊衛大將軍柴紹、內史侍郎唐儉、吏部侍郎殷開山、鴻臚卿劉世龍、衛尉少卿劉政會、都水監趙文恪、庫部郎中武士〓驃騎將軍張平高、李思行、李高遷、左屯衛府長史許世緒等十四人恕一死。



 武德九年十月,太宗又定功臣封戶,時文靜已死,乃自寂而下差功大小第之,總四十三人。寂戶千五百,長孫無忌、王君廓、尉遲敬德、房玄齡、杜如晦戶千三百,長孫順德、柴紹、羅藝、趙郡王孝恭戶千二百,侯君集、張公謹、劉師立戶千,李勣、劉弘基戶九百,高士廉、宇文士及、秦叔寶、程知節戶七百,安興貴、安修仁、唐儉、竇軌、屈突通、蕭瑀、封德彞、劉義節戶六百,錢九隴、樊興、公孫武達、李孟嘗、段志玄、龐卿惲、張亮、李藥師、杜淹、元仲文戶四百,張長遜、張平高、李安遠、李子和、秦行師、馬三寶戶三百。寂等三十人已見於傳。自趙文恪等十八人功不甚顯,然參附義始事,班班見當世。今次第其名,總出左方云。



 趙文恪,並州人。為隋鷹揚府司馬。義兵起,授右三統軍。武德二年,擢都水監,封新興郡公。時中國經大亂,馬耗,會突厥講和,詔文恪至並州,與齊王誘市邊馬以備軍。劉武周寇太原,屬城盡沒,李仲文守浩州,兵力孤絕,齊王使文恪率步騎千餘助守。會太原陷,遂棄城遁,詔下獄死。



 李思行,趙州人,避仇太原。唐公將起,使覘言冋長安,還,具論機策,以贊大議授左三統軍。從破霍邑,平京師,擢累嘉州刺史、樂安郡公。卒,贈洪州都督,謚曰襄。



 李高遷,岐州人,客太原,唐公引致左右。執高君雅等有功,以右三統軍從下霍邑,圍長安,力戰。遷左武衛大將軍、江夏郡公、檢校西麟州刺史。突厥寇馬邑,高滿政請救,詔高遷督兵助守。賊盛,乃夜斬關走,所將皆沒,坐除名徙邊。後歷資州刺史,卒,贈涼州都督。



 姜寶誼,秦州上邽人。父遠,仁周為秦州刺史、朝邑縣公。寶誼游太學,受書,業不進,去為左翊衛,以積勞遷鷹揚郎將,領府兵,從高祖督盜太原。及起兵,授左統軍,下西河、霍邑,以多,爵累永安縣公,歷右武衛大將軍。劉武周使黃子英數盜雀鼠谷,帝遣寶誼擊之。賊輕甲挑師,戰接而三遁,逐之,伏發,寶誼為賊執,俄亡歸。與裴寂拒宋金剛,戰汾州,兵合,寂棄軍走,寶誼復為所禽。帝聞為泣下曰:「彼烈士,必不下賊,死矣!「賜其家物千段,米三百斛。果謀還,被害。且死,西向大呼曰:「臣無狀,負陛下。」賊平,詔迎其柩,贈左衛大將軍、幽州總管,謚曰剛。子協,字壽,善篆籀。歷燕然都護、夏州都督,封成紀縣侯,謚曰威。



 許世緒,並州人。隋鷹揚府司馬。知隋將亡,請唐公曰:「天輔德,人與能,乘機不發,後必蹈悔。隋政不綱,天下搖亂,公姓名已著謠籙,今攬五郡之兵,據四戰之沖,茍無奇計,禍不反踵。若收取英俊,為天下倡,帝王業也。」公奇之,顧倚親密。兵起,授右一府司馬。累除蔡州刺史、真定郡公,卒。弟洛仁,亦從起晉陽,錄功至冠軍大將軍。卒,贈代州都督,謚曰勇,陪葬昭陵。



 劉師立,宋州虞城人。始事王世充為親將,洛陽平,當誅,秦王壯其才,釋不死,引為左親衛。建成之釁,師立參奉密議,後與尉遲敬德、龐卿惲、李孟嘗等九人錄功拜左衛率。遷左驍衛將軍、襄武郡公,賜絹五千匹。有告師立姓在符讖欲反者,太宗謂曰:「人言卿將反,果乎?」師立對曰:「臣為隋官,不過六品,材駑下,不敢希富貴。今遭非常之會,位將軍,顧巳極矣,何敢反?」帝笑曰:「朕知妄耳!」賜束帛,召入臥內慰勉。羅藝反,京師震駭,詔師立檢校右武候大將軍,勒兵備非常。藝平,有司劾黨與,師立坐與善,除名。尋以籓邸舊,檢校岐州都督。上書請討吐谷渾,未報,即遣使間諭部落,多降附者,列其地為開、橋二州。又黨項酋拓拔赤辭先附吐谷渾,倚險自守,亦遣說下之,詔赤辭為西戎州都督。師立以母喪解,岐人表留,遂不得赴哀。時河西黨項破丑氏嘗苦邊,又阻新附,師立討之。軍未至,破醜懼,遁去,師立窮追之,抵恤於真山而還。又戰吐谷渾於小莫門川,破之。轉始州刺史,卒,謚曰肅。



 劉義節,並州人。隋大業末,補晉陽鄉長,富於財。裴寂薦之唐公,又與王威、高君雅游,然於唐公為最厚。兵將起,威、君雅疑之,義節刺知其情,得先事禽威等。從平京師,為鴻臚卿。時傾府庫為軍賞,帑財大乏。義節曰:「今京師屯兵多,樵貴帛賤,若伐街苑樹為薪,以易布帛,歲數十萬可致。」又請軸舒藏內見繒,取羨尺,補雜費,得十餘萬段,調度遂給。遷太府,封葛國公。義節本名世龍,或言世龍子名鳳昌,父子非人臣兆,高祖不聽,更賜今名。貞觀初,轉少府監,坐貴入賈人珠及故出署丞罪,廢為民,徙嶺南,終欽州別駕。從子思禮,武后時為箕州刺史。少學相人於張憬藏,憬藏謂思禮歷刺史,位至太師。萬歲通天二年,授箕州,益喜,以為太師位尊,非佐命不可得,乃結洛州錄事參軍綦連耀謀反,謂耀曰;「君體有龍氣如大帝。」耀亦曰:「公金刀,當輔我。」陰約君臣。思禮因以術眩眾,見者必曰:「當三品」,使嗜進者充望,然後云:「綦連耀且受命,公等皆因之。事敗,武懿宗按之,陰弛思禮禁,使多逮引。思禮冀自脫,悉引素相忤者,將刑猶不寤,與眾人斬於市。其知名者,如李元素、孫元亨、石抱忠、王抃、抃兄勔、路敬淳等三十餘族,竄逐千餘人。



 錢九隴,字永業,湖州長城人。父文強,為吳明徹裨將,與明徹俱敗彭城。入隋,以罪沒為奴,故九隴事唐公。善騎射,常備左右。兵起,以功授金紫光祿大夫。從戰薛仁杲、劉武周,擢累為右武衛將軍。從平洛陽,佐皇太子建成討劉黑闥魏州,力戰破賊,以功最封郇國公,以本官為苑游將軍。貞觀初,為眉州刺史,改巢國。卒,贈左武衛大將軍、潭州都督,謚曰勇,陪葬獻陵。



 樊興,安州人。以罪為奴。從唐公平長安,授左監門將軍。從秦王積戰多,封營國公,數賜黃金雜物。後坐事削爵。貞觀六年,陵州獠反,命討之,為左驍衛將軍。又從李靖擊吐谷渾,為赤水道行軍總管。後軍期,士多死,亡失器仗,以勛減死。後為左監門大將軍、襄城郡公。太宗征遼,以興忠謹,副房玄齡留守京師,檢校右武候將軍。卒,贈左武候大將軍、洪州都督,陪葬獻陵。



 公孫武達,京兆櫟陽人。以豪俠稱,為隋驍果。兵興,武達至長春宮上謁。從秦王討劉武周,苦戰功多,累遷秦府右三軍驃騎,封清水縣公。貞觀初,為肅州刺史。突厥騎數千、輜重萬餘入寇,謀南趨吐谷渾,武達以精兵二千人與戰,虜稍卻,復殊死鬥,薄之張掖河,潛命上流度兵,虜已半濟,乃兩岸夾擊,斬溺略盡。璽書勞之,遷左監門將軍。鹽州突厥叛,詔武達趨靈州,追及賊,賊方度河,乃據南涯陣,武達擊之,斬其帥可邏拔扈,進封東萊郡公。終右武衛大將軍,贈荊州都督,陪葬昭陵,謚曰壯。



 龐卿惲,並州人。從討隱太子有功,拜右驍衛將軍、邾國公。卒,追改濮國。子同善,右金吾大將軍。同善子承宗,開元初,仕至太子賓客。



 張長遜,京兆櫟陽人。精馳射,在隋為里長。以平陳功,擢上開府,累遷五原郡通守。遭亂,附突厥,突厥號為割利特勒。義兵起,以郡降,即拜五原太守、安化郡公,徙封範陽。時梁師都、薛舉請突厥兵南度河,長遜矯作詔與莫賀咄設,以伐其謀,會唐使亦至,突厥兵不出。武德元年,詔右武候驃騎將軍高世靜聘始畢可汗,至豐州而始畢死,詔留金幣不遣。突厥怒,引兵南至河。長遜遣世靜出塞勞之,且若專致賻賜者,虜引還。授總管,改楊國公。及討薛舉,不待命輒引兵會,賜錦袍金甲。或譖長遜居豐久,恐與突厥為脣齒,乃請入朝,授右武候將軍,徙息國公,加賜宮人、彩千段。屬有疾,高祖親問之。後竇軌率巴、蜀兵擊王世充,以長遜檢校益州行臺左僕射。歷遂、夔二總管,政以惠稱。貞觀十一年卒。



 張平高,綏州人。為隋鷹揚府校尉,戍太原,遂預謀議。從唐公平京城,累授左領軍將軍,封蕭國公。貞觀初,為丹州刺史,坐事,以右光祿大夫還第。卒,追封羅國,贈潭州都督。



 李安遠,夏州人。父徹,隋上柱國、雲州刺史。世為將家,以財雄。安遠少無檢,與博徙游,至破產。晚乃折節向書,從士大夫,茍勝己,必傾心交之。襲爵城陽公。與王珪最善,珪坐王頗得罪,當流,安遠為營護免。後補正平令。兵起,攻絳州,安遠與通守陳叔達嬰城拒。唐公素與安遠善,及拔絳,撫慰其家,引與同食,授右翊衛統軍、正平縣公。後從破屈突通,進上柱國、右武衛大將軍。數從秦王征討,積功,累封至廣德郡公。奉使吐谷渾,安遠與約和,吐谷渾乃請為互市,邊場利之。隱太子將亂,陰使誘動,安遠介無貳志,秦王益親重。貞觀初,嘗命統邏騎都下,督盜賊。歷潞州都督、懷州刺史,皆以干用顯,然急刻少恩,由是損名。卒,贈涼州都督,謚曰安,追封遂安郡公。



 馬三寶,性敏獪。事柴紹,為家僮。紹尚平陽公主,高祖兵起,紹間道走太原。三寶奉公主遁司竹園,說賊何潘仁與連和。潘仁入謁,以百兵為主衛。三寶自稱總管,撫接群盜,兵至數萬。唐公濟河,授三寶左光祿大夫。秦王至竹林宮,三寶以兵詣軍門謁,遂從平京師,拜太子監門率。別擊叛胡劉拔真於北山,破之。從平薛仁杲。與柴紹擊吐谷渾於岷州,先鋒陷陣,斬名王,俘執數千,以功封新興縣男。後高祖幸司竹園,顧謂曰:「汝興兵處邪?衛青大不惡。」貞觀初,拜左驍衛大將軍,進爵為公,卒謚曰忠。



 李孟嘗,趙州人。終右威衛大將軍、漢東郡公。



 元仲文,洛州人。終右監門將軍、河南縣公。



 秦行師,並州人。終右監門將軍、清水郡公。



 贊曰:應龍之翔,雲霧滃然而從,震風薄怒,萬空不約而號,物有自然相動耳。觀二子非有踔越之姿,當高祖受命,赫然利見於世,故能或翼或從,尸天之功云。文靜數履軍陷陣,以才自進,而寂專用串暱顯。外者易乘,邇者難疏,故文靜先被躁望誅,寂後坐訞言斥,誠異夫蕭何、曹參雲!



\end{pinyinscope}