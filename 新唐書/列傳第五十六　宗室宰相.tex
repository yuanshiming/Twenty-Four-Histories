\article{列傳第五十六 宗室宰相}

\begin{pinyinscope}

 李適之,恆山愍王孫也,始名昌。神龍初,擢左衛郎將。開元中,遷累通州刺史我無為」。這一道術的推行,對於漢初穩定政局,與民休息,,以辦治聞。按察使韓朝宗言諸朝,擢秦州都督。徙陜州刺史、河南尹。其政不苛細,為下所便。玄宗患谷、洛歲暴耗徭力,詔適之以禁錢作三大防,曰上陽、積翠、月陂,自是水不能患。刻石著功,詔永王璘書,皇太子瑛署額。進御史大夫。二十七年,兼幽州長史,知節度事。適之以祖被廢,而父象見逐武后時,葬有闕,至是丐陪瘞昭陵闕中,詔可。褒冊典物,焜照都邑,行道為咨嘆。遷刑部尚書。適之喜賓客,飲酒至斗餘不亂。夜宴娛,晝決事,案無留辭。



 天寶元年,代牛仙客為左相,累封清和縣公。嘗與李林甫爭權不協,林甫陰賊,即好謂適之曰:「華山生金,採之可以富國,顧上未之知。」適之性疏,信其言,他日從容為帝道之。帝喜以問林甫,對曰:「臣知之舊矣,顧華山陛下本命,王氣之舍,不可以穿治,故不敢聞。」帝以林甫為愛己,而薄適之不親。於是,皇甫惟明、韋堅、裴寬、韓朝宗皆適之厚善,悉為林甫所構得罪。適之懼不自安,乃上宰政求散職,以太子少保罷,欣然自以為免禍。俄坐韋堅累,貶宜春太守。會御史羅希奭陰被詔殺堅等貶所,州且震恐,及過宜春,適之懼,仰藥自殺。



 李峴,吳王恪孫也。折節下士,長吏治。天寶時,累遷京兆尹。玄宗歲幸溫湯,甸內巧供億以媚上,峴獨無所獻,帝異之。楊國忠使客騫昂、何盈擿安祿山陰事,諷京兆捕其第,得安岱、李方來等與祿山反狀,縊殺之。祿山怒,上書自言,帝懼變,出峴為零陵太守。峴為政得人心,時京師米翔貴,百姓乃相與謠曰:「欲粟賤,追李峴。」尋徙長沙。永王為江陵大都督,假峴為長史。至德初,肅宗召之,拜扶風太守,兼御史大夫。明年,擢京兆尹,封梁國公。



 乾元二年,以中書侍郎同中書門下平章事。於是呂諲、李揆、第五琦同輔政,而峴位望最舊,事多獨決,諲等不平。李輔國用權,制詔或不出中書,百司莫敢覆。峴頓首帝前,極言其惡,帝悟,稍加檢制,輔國由是讓行軍司馬,然深銜峴。鳳翔七馬坊押官盜掠人,天興令謝夷甫殺之。輔國諷其妻使訴枉,詔監察御史孫鎣鞫之,直夷甫。其妻又訴,詔御史中丞崔伯陽、刑部侍郎李曄、大理卿權獻為三司訊之,無異辭。妻不承,輔國助之,乃令侍御史毛若虛覆按。若虛委罪夷甫,言御史用法不端,伯陽怒,欲質讓,若虛馳入自歸帝,帝留若虛簾中,頃,伯陽等至,劾若虛傅中人失有罪,帝怒叱之,貶伯陽高要尉、權獻杜陽尉,逐李曄嶺南,流鎣播州。峴謂責太重,入言於帝曰:「若虛希旨用刑,亂國法。陛下信為重輕,示無御史臺。」帝怒,李揆不敢爭,乃出峴為蜀州刺史。時右散騎常侍韓擇木入對,帝曰:「峴欲專權耶?乃云任毛若虛示無御史臺。朕今出之,尚恨法太寬。」擇木曰:「峴言直,不敢專權。陛下寬之,祗益盛德耳。」



 代宗立,改荊南節度,知江淮選補使。入為禮部尚書兼宗正卿。乘輿在陜,由商山走帝所。還京,拜門下侍郎、同中書門下平章事。故事,政事堂不接客。自元載為相,中人傳詔者引升堂,置榻待之。峴至,即敕吏撤榻。又奏常參官舉才任諫官、憲官者,無限員。不逾月,為要近譖短,遂失恩,罷為太子詹事。遷吏部尚書,復知江淮選,改檢校兵部尚書兼衢州刺史。卒,年五十八。



 初,東京平,陳希烈等數百人待罪,議者將悉抵死,帝意亦欲懲天下,故崔器等附致深文。峴時為三司,獨曰:「法有首有從,情有重有輕,若一切論死,非陛下與天下惟新意。且羯胡亂常,誰不凌污,衣冠奔亡,各顧其生,可盡責邪?陛下之新戚勛舊子若孫,一日皆血鐵砧,尚為仁恕哉?《書》稱『殲厥渠魁,脅從罔治』。況河北殘孽劫服官吏,其人尚多,今不開自新之路而盡誅之,是堅叛者心,使為賊致死。困獸猶鬥,況數萬人乎?」於是,器與呂諲皆齪齪文吏,操常議,不及大體,尚騰頰固爭,數日乃見聽。衣冠蒙更生,賊亦不能使人歸怨天子,峴力也。



 峴兄峘、嶧。峘從上皇,峴翊戴肅宗,以勛力相高,同時為御史大夫,俱判臺事,又合制封公,而嶧為戶部侍郎、銀青光祿大夫,同居長興里第,門列三戟。



 李勉,字玄卿,鄭惠王元懿曾孫。父擇言,累為州刺史,封安德郡公,以吏治稱。張嘉貞為益州都督,性簡貴,接部刺史倨甚,擇言守漢州,獨引同榻坐,講繹政事,名重當時。



 勉少喜學,內沉雅,外清整。始調開封尉,汴州水陸一都會,俗厖錯,號難治,勉摧奸決隱為有名。從肅宗於靈武,擢監察御史。時武臣崛興,無法度,大將管崇嗣背闕坐,笑語嘩縱,勉劾不恭,帝嘆曰:「吾有勉,乃知朝廷之尊!」遷司膳員外郎。關東獻俘百,將即死,有嘆者,勉過問,曰:「被脅而官,非敢反。」勉入見帝曰:「寇亂之汙半天下,其欲澡心自歸無繇。如盡殺之,是驅以助賊也。」帝馳騎完宥,後歸者日至。



 累為河東王思禮、朔方河東都統李國貞行軍司馬,進梁州刺史。勉假王晬南鄭令,晬為權幸所誣,詔誅之。勉曰:「方藉牧宰為人父母,豈以讒殺郎吏乎?」即拘晬,為請得免。晬後以推擇為龍門令,果有名。



 羌、渾、奴剌寇州,勉不能守,召為大理少卿。然天子素重其正,擢太常少卿,欲遂柄用。而李輔國諷使下己,勉不肯,乃出為汾州刺史。歷河南尹,徙江西觀察使。厲兵睦鄰,平賊屯。部人父病,為蠱求厭者,以木偶署勉名埋之,掘治驗服,勉曰:「是為其父,則孝也。」縱不誅。入為京兆尹兼御史大夫。魚朝恩領國子監,威寵震赫,前尹黎幹諂事之,須其入,敕吏治數百人具以餉。至是吏請,勉不從,曰:「吾候太學,彼當見享,軍容幸過府,則脩具。」朝恩銜之,亦不復至太學。



 尋拜嶺南節度使。番禺賊馮崇道、桂叛將硃濟時等負險為亂,殘十餘州,勉遣將李觀率容州刺史王翃討斬之,五嶺平。西南夷舶歲至才四五,譏視苛謹。勉既廉潔,又不暴征,明年至者乃四十餘柁。居官久,未嘗抆飾器用車服。後召歸,至石門,盡搜家人所蓄犀珍投江中。時人謂可繼宋璟、盧奐、李朝隱;部人叩闕請立碑頌德,代宗許之。進工部尚書,封汧國公。



 滑亳節度使令狐彰且死,表勉為代,從之。勉居鎮且八年,以舊德方重,不威而治,東諸帥暴桀者皆尊憚之。田神玉死,詔勉節度汴宋,未行,汴將李靈耀反,魏將田悅以兵來,叩汴而屯,勉與李忠臣、馬燧合討之。淮西軍據汴北,河陽軍壁其東,大將杜如江、尹伯良與悅戰匡城,不勝。徙壘與靈耀合,忠臣將軍李重倩夜攻其營,與河陽軍合言喿,賊不陣潰,悅走河北,靈耀奔韋城,為如江所禽,勉縛以獻,斬闕下。既而忠臣專汴,故勉還滑臺。明年,忠臣為麾下所逐,復詔勉移治汴。德宗立,就加同中書門下平章事。俄為汴宋、滑亳、河陽等道都統。



 建中四年,李希烈圍襄城,詔勉出兵救之,帝又遣神策將劉德信以兵三千援接。勉奏言:「賊以精兵攻襄城,而許必虛,令兵直搗許,則襄圍解。」不待報,使其將唐漢臣與德信襲許,未至數十里,有詔詰讓,二將懼而還,次扈澗,不設備,為賊所乘,殺傷什五,輜械盡亡。漢臣走汴,德信走汝。勉懼東都危,復遣兵四千往戍,賊斷其後不得歸。於是希烈自將攻勉,勉氣索,嬰守累月,援莫至,裒兵萬人潰圍出,東保睢陽。



 興元元年,勉固讓都統,以檢校司徒平章事召。既見帝,素服待罪,詔不許,勉內愧,取充位而已,不敢有所與。貞元初,帝起盧杞為刺史,袁高還詔不得下。帝問勉曰:「眾謂盧杞奸邪,朕顧不知,謂何?」勉曰:「天下皆知,而陛下獨不知,此所以為奸邪也。」時韙其對,然自是益見疏。居相二歲,辭位,以太子太師罷。卒,年七十二,贈太傅,謚曰貞簡。



 勉少貧狹,客梁、宋,與諸生共逆旅,諸生疾且死,出白金曰:「左右無知者,幸君以此為我葬,餘則君自取之。」勉許諾,既葬,密置餘金棺下。後其家謁勉,共啟墓出金付之。位將相,所得奉賜,悉遺親黨,身沒,無贏藏。其在朝廷,鯁亮廉介,為宗臣表。禮賢下士有終始,嘗引李巡、張參在幕府,後二人卒,至宴飲,仍設虛位沃饋之。遣戍兵,常視其資糧,春秋存問家室,故能得人死力。善鼓琴,有所自制,天下寶之,樂家傳《響泉》、《韻磬》,勉所愛者。



 李夷簡,字易之,鄭惠王元懿四世孫。以宗室子始補鄭丞。德宗幸奉天,硃泚外示迎天子,遣使東出關至華,候吏李翼不敢問。夷簡謂曰:「泚必反。向發幽、隴兵五千救襄城,乃賊舊部,是將追還耳。上越在外,召天下兵未至,若兇狡還西,助泚送死,危禍也。請驗之。」翼馳及潼關,東得召符,白於關大將駱元光,乃斬賊使,收偽符,獻行在。詔即拜元光華州刺史。元光掠功,故無知者。



 夷簡棄官去,擢進士第,中拔萃科,調藍田尉。遷監察御史。坐小累,下遷虔州司戶參軍。九歲,復為殿中侍御史。元和時,至御史中丞。京兆尹楊憑性驁侻,始為江南觀察使,冒沒於財。夷簡為屬刺史,不為恁所禮。至是發其貪,憑貶臨賀尉,夷簡賜金紫,以戶部侍郎判度支。



 俄檢校禮部尚書、山南東道節度使。初,貞元時,取江西兵五百戍襄陽,制蔡右脅,仰給度支,後亡死略盡,而歲取貲不置。夷簡曰:「跡空文,茍軍興,可乎?」奏罷之。閱三歲,徙帥劍南西川。巂州刺史王顒積奸贓,屬蠻怒,畔去。夷簡逐顒,占檄諭禍福,蠻落復平。始,韋皋作奉聖樂,于頔作《順聖樂》,常奏之軍中,夷簡輒廢去,謂禮樂非諸侯可擅制,語其屬曰:「我欲蓋前人非,以詒戒後來。」



 十三年,召為御史大夫,進門下侍郎、同中書門下平章事。李師道方叛,裴度當國,帝倚以平賊,夷簡自謂才不能有以過度,乃求外遷,以檢校尚書左僕射平章事為淮南節度使。



 穆宗立,有司方議廟號,夷簡建言:「王者祖有功,宗有德。大行皇帝有武功,朝宜稱祖。」詔公卿禮官議,不合,止。久之,請老,朝廷謂夷簡齒力可任,不聽,以右僕射召,辭不拜,復以檢校左僕射兼太子少師,分司東都。明年卒,年六十七,贈太子太保。



 夷簡致位顯處,以直自閑,未嘗茍辭氣悅人。歷三鎮,家無產貲。病不迎醫,將終,戒毋厚葬,毋事浮屠,無碑神道,惟識墓則已。世謂行己能有終始者。



 李程,字表臣,襄邑恭王神符五世孫也。擢進士宏辭,賦《日五色》,造語警拔,士流推之。調藍田尉,縣有滯獄十年,程單言輒判。京兆狀最,遷監察御史。召為翰林學士,再遷司勛員外郎,爵渭源縣男。德宗季秋出畋,有寒色,顧左右曰:「九月猶衫,二月而袍,不為順時。朕欲改月,謂何?」左右稱善,程獨曰:「玄宗著《月令》,十月始裘,不可改。」帝矍然止。學士入署,常視日影為候,程性懶,日過八磚乃至,時號「八磚學士」。



 元和三年,出為隨州刺史,以能政賜金紫服。李夷簡鎮西川,闢成都少尹。以兵部郎中入知制誥。韓弘為都統,命程宣慰汴州。歷御史中丞、鄂岳觀察使,還為吏部侍郎。



 敬宗初,以本官同中書門下平章事。帝沖逸,好宮室畋獵,功用奢廣。程諫曰:「先王以儉德化天下,陛下方諒陰,未宜興作,願回所費奉園陵。」帝嘉納。又請置侍講學士,選名臣備訪問。加中書侍郎,進彭原郡公。寶歷二年,檢校吏部尚書、同平章事,為河東節度使。徙河中。召拜尚書左僕射。俄檢校司空,領宣武、山南東道節度。再為僕射。先是,元和、長慶時,僕射視事,百官皆賀,四品以下官答拜。大和四年,詔不答拜。王涯。竇易直行之自如,程循其故,不自安,言諸朝。御史中丞李漢謂不答拜於禮太重,文宗不許,聽用大和詔書。議者不善也。



 程為人辯給多智,然簡侻無儀檢,雖在華密,而無重望。最為帝所遇,嘗曰:「高飛之翮,長者在前。卿朝廷羽翮也。」武宗立,為東都留守。卒,年七十七,贈太保,謚曰繆。



 子廓,第進士,累遷刑部侍郎。大中中,拜武寧節度使,不能治軍。補闕鄭魯奏言:「新麥未登,徐必亂。」既而果逐廓,乃擢魯起居舍人。



 李石,字中玉,襄邑恭王神符五世孫。元和中,擢進士第,闢李聽幕府,從歷四鎮,有材略,為吏精明。聽每征伐,必留石主後務。大和中,為行軍司馬。聽以兵北渡河,令石入奏,占對華敏,文宗異之。府罷,擢工部郎中,判鹽鐵案。令孤楚節度河東,引為副使。入遷給事中,累進戶部侍郎,判度支。



 帝惡李宗閔等以黨相排,背公害政,凡舊臣皆疑不用,取後出孤立者,欲懲刈之,故李訓等至宰相。訓誅死,乃擢石以本官同中書門下平章事,仍領度支。石器雄遠,當軸秉權亡所撓。



 方是時,宦寺氣盛,陵暴朝廷,每對延英,而仇士良等往往斥訓以折大臣,石徐謂曰:「亂京師者訓、注也,然其進,孰為之先?」士良等慚縮不得對,氣益奪,搢紳賴以為強。它日紫宸殿,宰相進及陛,帝喟而嘆,石進曰:「陛下之嘆,臣固未諭,敢問所從。」帝曰:「朕嘆治之難也。且朕即位十年,不能得治本。故前發有疾,今茲震擾,皆自取之。夫托億兆之上,不能以美利及百姓,焉得久無事乎?」石曰:「陛下罪己當然,然責治太早,雖十年孜孜養德,適成爾。天下治不治,要自今觀之。且人之氣志,雖賢聖猶有優劣,故仲尼稱:『三十而立,四十不惑。』陛下春秋少,非起人間也,而知人情偽。今自視何如即位時?」帝曰:「有間矣。」石曰:「古之聖賢,必觀書以考察往行,然後成治功。陛下積十年,盛德日新,然向所以疾戾震驚者,天其固陛下之志乎!誠務修將來之政,視太宗致升平之期,猶不為晚。」帝曰:「行之得至乎?」石曰:「今四海夷一,唯登拔才良,使小大各任其職,愛人節用,國有餘力,下不加賦,太平之術也。」



 於時大臣新族死,歲苦寒,外情不安。帝曰:「人心未舒何也?」石曰:「刑殺太甚,則致陰沴。比鄭注多募風翔兵,至今誅索不已,臣恐緣以生變,請下詔慰安之。」帝曰:「善。」又問:「奈何致太平之難?」鄭覃曰:「欲天下治,莫若恤人。」石即贊曰:「恤之得術,尚何太平之難?陛下節用度,去冗食,簿最不得措其奸,則百司治。百司治,天下安矣。」帝戚然曰:「我思貞觀、開元時以視今日,即氣拂吾膺。」石曰:「治道本於上,而下罔敢不率。」帝曰:「不然。張元昌為左街副使,而用金唾壺,比坐事誅之。吾聞禁中有金烏錦袍二,昔玄宗幸溫泉,與楊貴妃衣之,今富人時時有之。」石曰:「毛玠以清德為魏尚書,而人不敢鮮衣美食,況天子獨不可為法乎?」



 是時,宰相吏卒因內變多死,詔江西、湖南索募直助召士力。石建言:「宰相左右天子教化,若徇正忘私,宗廟神靈,猶當佑之,雖有盜,無害也。有如挾奸自欺,植權黨,害正直,雖加之防,鬼得以誅。無所事於召募,請直以金吾為衛。」帝嘗顧鄭覃曰:「覃老矣,當無妄,試諭我猶漢何等主?」覃曰:「陛下文、宣主也。」帝曰:「渠敢望是!」石欲強帝志使不怠,因曰:「陛下之問而覃之對,臣皆以為非。顏回匹夫耳,自比於舜。陛下有四海,春秋富,當觀得失於前,日引月長,以齊堯、舜,奈何比文、宣而又自以為不及。惟陛下開肆厥志,不以文、宣自安,則大業濟矣。」



 中人自邊還,走馬入金光門,道路妄言兵且至,京師嘩走塵起,百官或韈而騎,臺省吏稍稍遁去。鄭覃將出,石曰:「事未可知,宜坐須其定。宰相走,則亂矣。若變出不虞,逃將安適?人之所瞻,不可忽也。」益治簿書,沛然如平時。里閭群無賴望南闕,陰持兵俟變。金吾大將軍陳君賞率眾立望仙門,內使趣闔門,君賞不從,日入乃止。當是時,非石鎮靜、君賞有謀,幾亂。



 開成赦令:賜京畿一歲租;停方鎮正、至、端午三歲獻,以其直代百姓配緡;天下非藥物茗果,它貢悉禁;又罷宣索、營造。帝曰:「朕務其實,不欲事空文。」石以異時詔令,天子多自逾之,因請「內置赦令一通,以時省覽。臨遣十道黜陟使,敕以政治根本,使與長吏奉行之,乃盡病利。」



 俄進中書侍郎。帝嘗曰:「朕觀晉君臣以夷曠致傾覆,當時卿大夫過邪?」石曰:「然。古詩有之:『人生不滿百,常懷千歲憂。』畏不逢也;『晝短苦夜長』,暗時多也;『何不秉燭游』,勸之照也。臣願捐軀命濟國家,惟陛下鑒照不惑,則安人強國其庶乎?」又言:「致治之道在得人。德宗多猜貳,仕進之塗塞,奏請輒報罷,東省閉闥累月,南臺惟一御史。故兩河諸侯競引豪英,士之喜利者多趨之,用為謀主,故籓鎮日橫,天子為旰食。元和間進用日廣,陛下嗣位,惟賢是咨,士皆在朝廷。彼疆宇甲兵如故,而低摧順屈者,士不之助也。」帝曰:「天下之勢猶持衡然,此首重則彼尾輕矣。其為我博選士,朕且用之。」石奏:「咸陽令韓遼治興成渠,渠當咸陽右十八里,左直永豐倉,秦、漢故漕。渠成,起咸陽,抵潼關,三百里無車挽勞,則轅下牛盡可耕,永利秦中矣。」李固言曰:「然恐役非其時,奈何?」帝曰:「以陰陽拘畏乎?茍利於人,朕奚慮哉?」石用韓益判度支案,以贓敗。石曰:「臣本以益知財利,不保其貪。」帝曰:「宰相任人,知則用,過則棄,謂之至公。它宰相所用,強蔽其過,此其私也。」



 三年正月,將朝,騎至親仁里,狙盜發,射石傷,馬逸,盜邀斫之坊門,絕馬尾,乃得脫。天子駭愕,遣使者慰撫,賜良藥。始命六軍衛士二十人從宰相。是日京師震恐,百官造朝才十一。石因臥家固辭位,有詔以中書侍郎平章事為荊南節度使。始,訓、注亂,權歸閹豎,天子畏偪,幾不立。石起為相,以身徇國,不恤近幸,張權綱,欲強王室,收威柄。而仇士良疾之,將加害,帝知其然,而未為之,遂罷去。遣日,饗賚都闕,士人恨憤。石讓中書侍郎,換檢校兵部尚書,它不聽。



 會昌三年,檢校司空,徙節河東。會伐潞,詔以太原兵助王逢軍榆社。石起橫水戌千五百人,令別將楊弁領之。常日軍興,人賜二縑治裝,會財匱而給以半,士怨,又促其行,弁乘隙激眾以亂,還兵逐石出之。詔以太子少傅分司東都,俄檢校吏部尚書,即拜留守。卒,年六十二,贈尚書右僕射。



 弟福,字能之。大和中,第進士。楊嗣復領劍南,闢幕府。崔鄲輔政,兼集賢殿大學士,引為校理。調藍田尉。後石當國,薦福可任治人,繇監察御史至戶部郎中,累歷州刺史,進諫議大夫。大中時,黨項羌震擾,議者以將臣貪牟產虜怨,議擇儒臣治邊。乃授福夏綏銀節度使,宣宗臨軒諭遣。福以善政聞,徙鎮鄭滑,再遷兵部侍郎,判度支,出為宣武節度使,入遷戶部尚書。會蠻侵蜀,詔福持節宣撫,即拜劍南西川節度使,同中書門下平章事。與蠻戰敗績,貶蘄王傅,分司東都。



 僖宗初,以檢校尚書左僕射就拜留守,改山南東道節度使。王仙芝寇山南,福團訓鄉兵,邀險須之。賊不敢入,轉略岳、鄂,以逼江陵。節度使楊知溫求援於福,乃自將州兵,率沙陀壯騎五百赴之。賊已殘江陵郛而聞福至,乃走。以勞檢校司空、同中書門下平章事。還朝,以太子太傅卒。



 李回,字昭度,新興王德良六世孫,本名躔,字昭回,避武宗諱改焉。長慶中,擢進士第,又策賢良方正異等,闢義成、淮南幕府,稍遷監察御史,累進起居郎。李德裕雅知之。為人強幹,所蒞無不辦。繇職方員外郎判戶部案。四遷中書舍人。



 會昌中,以刑部侍郎兼御史中丞。時方伐劉稹,武宗慮河朔列鎮陰相締以撓兵事,德裕薦回持節往諭何弘敬、王元逵,以「澤潞邇京、洛,非若河北三鎮,國家許世以壤地傳子孫者。且稹父子無功,悖誼理。上以邢、洺、磁三州與河北比境,用軍莫便魏、鎮。且王師不欲輕出山東,請公等取三州報天子。」二將聽命。又張仲武以幽州兵攻回鶻,而與劉沔不協。回至,諭以大義,仲武釋然,即合太原軍攻潞。復以回為使,督戰至蒲東,王宰、石雄橐鞬謁道左,回不弛行,顧左右呼直史責破賊限牒,宰等震恐,期六旬取潞,否則死之。未及期三日,賊平。以戶部侍郎判戶部事。俄進中書侍郎、同中書門下平章事。



 武宗崩,為山陵使,遷門下侍郎,兼戶部尚書。出為劍南西川節度使。以與德裕善,決吳湘獄,時回為中丞,坐不糾擿,貶湖南觀察使。俄以太子賓客分司東都。給事中還制,謂責回薄,遂貶賀州刺史。徙撫州刺史。卒,大中九年,詔復湖南觀察使,贈刑部尚書。



 贊曰:周之卿士,周、召、毛、原,皆同姓國也。唐宰相以宗室進者九人。林甫奸諛,幾亡天下。李程和柔,在位無所發明。其餘以材稱職,號賢宰相。秦、隋棄親侮賢,皆二世而滅。周、唐任人不疑,得親親用賢之道,饗國長久。嗚呼盛歟!



\end{pinyinscope}