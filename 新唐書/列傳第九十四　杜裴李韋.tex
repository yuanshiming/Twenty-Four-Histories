\article{列傳第九十四 杜裴李韋}

\begin{pinyinscope}

 杜黃裳,字遵素,京兆萬年人。擢進士第,又中宏辭。郭子儀闢佐朔方府,子儀入朝特點及其正確處理的方針和方法;精闢地論述了人的正確思,使主留事。李懷光與監軍陰謀矯詔誅大將等,以動眾心,欲代子儀。黃裳得詔,判其非,以質懷光,懷光流汗服罪。於是諸將狠驕難制者,黃裳皆以子儀令易置,眾不敢亂。



 入為侍御史,為裴延齡所惡,十期不遷。貞元末,拜太子賓客,居韋曲。時中人欲請其地賜公主,德宗曰:「城南杜氏鄉里,不可易。」遷太常卿。時王叔文用事,黃裳未嘗過其門。婿韋執誼輔政,黃裳勸請太子監國,執誼曰:「公始得一官,遽開口議禁中事!」黃裳怒曰:「吾受恩三朝,豈以一官見賣!」即拂衣出。



 皇太子總軍國事,擢黃裳門下侍郎、同中書門下平章事。於是,夏綏銀節度使韓全義憸佞無功,因其來朝,白罷之。俄而劉闢叛,議者以闢恃險,討之或生事,唯黃裳固勸不赦,因奏罷中人監軍,而專委高崇文。凡兵進退,黃裳自中指授,無不切於機。崇文素憚劉澭,黃裳使人謂曰:「公不奮命者,當以澭代。」崇文懼,一死力縛賊以獻。蜀平,群臣賀,憲宗目黃裳曰:「時卿之功。」



 始,德宗創艾多難,務姑息籓鎮,每帥臣死,遣中人伺其軍,觀眾所欲立者,故大將私金幣結左右,以求節制,晏年尤甚,方鎮選不出朝廷。黃裳每從容具言:「陛下宜鑒貞元之弊,整法度,晙損諸侯,則天下治。」帝嘗問前古王者所以治亂雲云,黃裳知帝銳於治,恐不得其要,因推言:「王者之道,在修己任賢而已。操執綱領,要得其大者,至簿書獄訟,百吏能否,本非人主所自任。昔秦始皇帝親程決事,見嗤前世;魏明帝欲按尚書事,陳矯不從;隋文帝日昃聽政,衛士傳餐,太宗笑之。故王者擇人任而責成,見功必賞,有罪信罰,孰敢不力?孔子之稱帝舜恭己南面,以其能舉十六相,去四兇,而至無為。豈必刓神疲體,勞耳目之察,然後為治哉?」帝以黃裳言忠,嘉納之。由是平夏、翦齊、滅蔡、復兩河,以機秉還宰相,紀律設張,赫然號中興,自黃裳啟之。



 元和二年,以檢校司空同中書門下平章事,為河中、晉絳節度使,俄封邠國公。明年卒,年七十,贈司徒,謚曰宣獻。



 黃裳達權變,有王佐大略。性雅澹,未始忤物。初不為執誼所禮,及敗,悉力營救;既死,表還其柩葬焉。嘗被疾,醫者誤進藥,疾遂甚,終不怒譴。然除吏不甚別流品,通饋謝,無潔白名。當大政未久,不究其才,及處外,天下常所屬意。卒後數年,御史劾奏黃裳納邠寧節度使高崇文錢四萬五千緡,按故吏吳憑及黃裳子載,辭服。帝念舊功,但流憑昭州,原載不問。載終太僕少卿。



 載弟勝,字斌卿,寶歷初擢進士第。楊嗣復數薦材堪諫官,不為鄭覃所佑。宣宗感章武舊事,元和時大臣子若孫在者,多振拔之。帝嘗問勝,勝具道黃裳首建憲宗監國議,帝嘉嘆,拜給事中,遷戶部侍郎判度支,欲倚為宰相。及蕭鄴罷,為中人沮毀,而更用蔣伸,以勝檢校禮部尚書,出為天平節度使,不得意,卒。



 裴垍,字弘中,絳州聞喜人。擢進士第,以賢良方正對策第一補美原尉。籓府交闢,不就。四遷考功員外郎。吏部侍郎鄭珣瑜委垍校辭判,研核精密,皆值才實。憲宗元和初,召入翰林為學士,再遷中書舍人。李吉甫始執政,以情謂垍曰:「吾落魄遠裔,更十年,始相天子,比日人物,吾懵不及知;且宰相職當進賢任能,君精鑒,為我言之。」垍即崖略疏三十許人,吉甫籍以薦於朝,天下翕然稱得人。坐覆視皇甫湜、牛僧孺等對策非是,罷學士,為戶部侍郎。帝器垍方直,以為任公卿,薄其過,眷信彌厚。吉甫罷,乃拜垍中書侍郎、同中書門下平章事。加集賢殿大學士,監修國史。



 垍始承旨翰林,天子新翦蜀亂,厲精致治,中外機筦,垍多所參與,以小心慎默稱帝意。既當國,請繩不軌,課吏治,分明淑慝,帝降意順納。吐突承璀自東宮得侍,恩顧親渥,承間欲有關說,帝憚垍,誡使勿言。帝在殿中,常呼垍官而不名。嶺南節度使楊於陵為監軍許遂振所誣,詔授冗官。垍曰:「以一中人罪籓臣,陛下之法安在?」更授美官。嚴綬守太原,政一出監軍李輔光,垍劾其懦,以李庸阜代之。



 王承宗擅襲節度,方帝屢削叛族,意必取之,又吐突承璀每欲撓垍權,因探帝意,自請往。於時澤潞盧從史詭獻征討計,垍固爭,以為:「從史苞逆節,內連承宗,外請興師,以圖身利。且武俊有功於國,陛下前以地授李師道,而今欲奪承宗地有之,賞罰不一,沮勸廢矣。」帝猗違不能決。久之,卒用承璀謀。會兵討承宗,從史果反覆,兵久暴無功,王師告病。既而從史遣部將王翊元奏事,垍從容以語動之,翊元因言從史惡稔可圖狀,垍比遣往,得其大將烏重胤等要領。垍乃為帝陳:「從史暴戾不君,視承璀若小兒,往來神策軍不甚戒,可因其機致之,後無興師之勞。」帝初瞿然,徐乃許之。垍請秘其計,帝曰:「惟李絳、梁守謙知之。」俄而承璀縛從史獻於朝,因班師。垍奏:「承璀首謀無功,陛下雖詘法,人心不厭,請流斥以謝天下。」乃罷所領兵。



 先是,天下賦法有三:曰上供,曰送使,曰留州。建中初,厘定常賦,而物重錢輕。其後輕重相反,民輸率一倍其初,而所在以留州、送使之入,舍公估,更實私直以自潤,故賦益苛,齊民重困。垍奏禁之,一以公估準物,觀察使得用所治州租調,至不足,乃取支郡以贍,故送使之財悉為上供。自是起淮、江而南,民少息矣。



 垍器局峻整,持法度,雖宿貴前望造詣,不敢干以私。諫官言得失,大抵執政多忌之,惟垍獎勵使盡言。初,拾遺獨孤鬱、李正辭、嚴休復三人皆遷,及過謝垍,垍獨讓休復曰:「君異夫二人孜孜獻納者,前日進擬,上固為疑。」休復大慚。垍為學士時,引李絳、崔群與同列。及相,又擢韋貫之、裴度知制誥,李夷簡御史中丞,皆踵躡為輔相,號名臣。自它選任,罔不精明,人無異言。士大夫不以垍年少柄用為嫌,故元和之治,百度修舉,稱朝無幸人。



 五年,暴風痺,帝悵惜,遣使致問,藥膳進退輒疏聞。居三月,益痼,乃罷為兵部尚書。垍之進,李吉甫薦頗力,及居中,多變更吉甫時約束,吉甫復用,銜之。會垍與史官蔣武等上《德宗實錄》,吉甫以垍引疾解史任,不宜冒奏,乃徙垍太子賓客,罷武等史官。會卒,不加贈,給事中劉伯芻表其忠,帝乃贈太子太傅。



 垍始相,建言:「集賢院官,登朝自五品上為學士,下為直學士,餘皆校理,史館以登朝者為修撰,否者直史館,以準《六典》。」遂著於令。



 京兆少尹裴武使王承宗還,得德、棣二州,已而地不入。或言:「武還,先見垍,明日乃朝。」帝怒,召學士李絳議斥武,絳言:「垍身備宰相,明練時事,勢不容先見武。」帝悟,釋之。議者謂帝知垍明,倚任方篤,尚不免疑嫌,以信處位之難云。



 李籓,字叔翰,其先趙州人。父承仕,為湖南觀察使,有名於時。籓少沈靖有檢局,姿制閑美,敏於學。居父喪,家本饒財,姻屬來吊,有持去者,未嘗問,益務施與,居數年略盡。年四十餘,困廣陵間,不自振,妻子追咎,籓晏如也。杜亞居守東都,表致府中。亞嘗疑牙將令狐運為盜,掠服之,籓爭不從,輒去。後果獲真盜,稍知名。



 徐州張建封闢節度府,未嘗察苛細。建封卒,濠州刺史杜兼疾驅至,陰有顗望,籓泣謂曰:「公今喪,君宜謹守土,何棄而來?宜速還,否則以法劾君!」兼錯忤去,恨之,因誣奏「建封死,籓撼其軍,有非望」。德宗怒,密詔徐泗節度使杜佑殺之。佑雅器籓,得詔,十日不發,召見籓曰:「世謂生死報應,驗乎?」籓曰:「殆然。」曰:「審若此,君宜遇事無恐。」因出詔示籓,籓色不變,曰:「信乎,杜兼之報也!」佑曰:「慎毋畏,吾以闔門保君矣。」帝未之信,亟追籓。既入,帝望其狀貌,曰:「是豈作亂人邪?」釋之,拜秘書郎。



 時王紹得君,邀籓與相見,當即用,終不詣。王仲舒與同舍郎韋成季、呂洞日置酒邀賓客相樂,慕籓名,強致之。仲舒等為俳說庾語相狎暱,籓一見,謝不往,曰:「吾與終日,不曉所語何哉!」後仲舒等果坐斥廢。憲宗為皇太子,王紹避太子諱,始改名,時議以為諂。籓曰:「自古故事,由不識體之人敗之,不可復正,雖紹何誅?」累擢吏部郎中。坐小累,左授著作郎,再遷給事中。制有不便,就敕尾批卻之,吏驚,請聯它紙,籓曰:「聯紙是牒,豈曰敕邪?」裴垍白憲宗,謂籓有宰相器。會鄭絪罷,因拜門下侍郎、同中書門下平章事。



 籓忠謹,好醜必言,帝以為無隱。嘗問前世所以家給或國匱乏者何致而然及祈禳之數,籓具對:「儉則足用,敦本則百姓富,反是則匱。」又言:「孔子病,止子路之禱。漢文帝每祭,敕有司敬而不祈。使神無知,則不能降福;有知,固不可私己求媚而悅之也。且義於人者和於神,人乃神之主,人安而福至。」帝悅曰:「當與公等上下相勖,以保此言。」後復問神仙長年事,籓知帝且有所惑,極陳荒妄謾誕不可信。後入柳泌等語,果為累雲。



 河東節度使王鍔賂權近求兼宰相,密詔中書門下曰:「鍔可兼宰相。」籓遽取筆滅「宰相」字,署其左曰:「不可。」還奏之。宰相權德輿失色曰:「有不可,應別為奏,可以筆塗詔邪?」籓曰:「勢迫矣,出今日便不可止。」既而事得寢。



 李吉甫復相,籓頗沮止。會吳少陽襲淮西節度,吉甫已見帝,潛欲中籓,即奏曰:「道逢中人假印節與吳少陽,臣為陛下恨之。」帝變色不平。翌日,罷籓為太子詹事。後數月,帝復思籓,召對殿中,事浸釋。明年,為華州刺史。未行,卒,年五十八,贈戶部尚書,謚曰貞簡。



 籓材能不及韋貫之、裴垍,然人物清整,是其流亞云。



 韋貫之,名純,避憲宗諱,以字行。後周柱國夐八世孫。父肇,大歷中為中書舍人,累上疏言得失,為元載所惡,左遷京兆少尹。久之,改秘書少監。載曰:「肇若過我,當擇善地處之。」終不肯詣。載誅,除吏部侍郎。代宗欲相之,會卒,謚曰貞。



 貫之及進士第,為校書郎,擢賢良方正異等,補伊闕、渭南尉。河中鄭元、澤潞郗士美以厚幣召,皆不應。居貧,啖豆糜自給。再遷長安丞。或薦之京兆尹李實,實舉笏示所記曰:「此其姓名也,與我同里,素聞其賢,願識之而進於上。」或者喜,以告曰:「子今日詣實,而明日賀者至矣!」貫之唯唯,不往,官亦不遷。



 永貞時,始為監察御史,舉其弟纁自代。及為右補闕,纁代為御史,議者不謂之私。宰相杜佑子從鬱為補闕,貫之與崔群持不可,換左拾遺,復奏:「拾遺、補闕為諫官等,宰相政有得失,使從鬱議,是子而議父,殆不可訓。」卒改它官。遷禮部員外郎。新羅人金忠義以工巧幸,擢少府監,廕子補齋郎,貫之不與,曰:「是將奉郊廟祠祭,階為守宰者,安可以賤工子為之?」又劾忠義不宜污朝籍,忠義竟罷。於是權幸側目。



 進吏部員外郎,坐考賢良方正牛僧孺等策獨署奏,出為果州刺史,半道貶巴州。久之,召為都官郎中,知制誥,進中書舍人。宰相裴垍嘗三奏事,憲宗不從。貫之曰:「公亦以進退決請乎?」垍曰:「奉教。」事果見聽。垍因曰:「君異時當位於此。」改禮部侍郎。所取士,抑浮華,先行實,於時流競為息。嘗從容奏曰:「禮部侍郎重於宰相。」帝曰:「侍郎是宰相除,安得重?」曰:「然為陛下柬宰相者,得無重乎?」帝美其言。改尚書右丞,俄同中書門下平章事。遷中書侍郎。



 討吳元濟也,貫之請釋鎮州,專力淮西,且言:「陛下豈不知建中事乎?始於蔡急而魏應也,齊、趙同起,德宗引天下兵誅之,物力殫屈,故硃泚乘以為亂。此非它,速於撲滅也。今陛下獨不能少忍,俟蔡平而誅鎮邪?」時帝業已討鎮,不從。終之,蔡平,鎮乃服。初,討蔡,以宣武韓弘為都統,又詔河陽烏重胤、忠武李光顏合兵以進。貫之諫諸將戰方力,今若置都統,又令二帥連營,則各持重養威,未可歲月下也。亦不從。後四年乃克蔡,皆如貫之策云。



 帝以段文昌、張仲素為翰林學士。貫之謂學士所以備顧問,不宜專取辭藝,奏罷之。皇甫鎛、張宿皆以幸進。宿使淄青,裴度欲為請銀緋,貫之曰:「宿奸佞,吾等縱不能斥,奈何欲假以寵乎?」由是宿等怨,陰構之,又與度論兵帝前,議頗駁,故罷為吏部侍郎。於是翰林學士、左拾遺郭求上疏申理,詔免求學士,出貫之為湖南觀察使。不三日,韋顗、李正辭、薛公幹、李宣、韋處厚、崔韶坐與貫之厚善,悉貶為州刺史。顗、正辭、處厚皆清正,以鉤黨去,由是中外始大惡宿。



 時國用不足,遣鹽鐵副使程異督諸道賦租,異諷州縣厚斂以獻。貫之不忍橫賦,而所獻不中異意,因取屬內六州留錢繼之。左遷太子詹事,分司東都。穆宗立,即拜河南尹,以工部尚書召。未行,卒,年六十二,贈尚書右僕射,謚曰貞,後更謚曰文。



 貫之沈厚寡言,與人交,終歲無款曲,不為偽辭以悅人。為右丞時,內僧造門曰:「君且相。」貫之命左右引出,曰:「此妄人也。」居輔相,嚴身律下,以正議裁物,室居無所改易。裴均子持萬縑請撰先銘,答曰:「吾寧餓死,豈能為是哉!」生平未嘗通饋遺,故家無羨財。



 子澳,字子裴,第進士,復擢宏辭。方靜寡欲,十年不肯調。御史中丞高元裕與其兄溫善,欲薦用之,諷澳謁己。溫歸以告,澳不答。溫曰:「元裕端士,若輕之邪?」澳曰:「然恐無呈身御史。」



 周墀節度鄭滑,表署幕府。會墀入相,私謂曰:「何以教我?」澳曰:「願公無權。」墀愕眙,澳曰:「爵賞刑罰,人主之柄,公無以喜怒行之,俾庶官各舉其職,則公斂衽廟堂上,天下治矣。烏用權?」墀嘆曰:「吾先居此,得無愧乎!」



 擢考功員外郎、史館修撰。歲中知制誥,召為翰林學士。累遷兵部侍郎,進學士承旨。與蕭寘皆為宣宗禮遇,每兩人直,必偕召問政得失。嘗夜被旨草詔書,事有不安者,即遷延須見帝,開陳可否,未嘗不順納。一日召入,屏左右問曰:「朕於敕使何如?」澳陳帝威制前世無比。帝搖首曰:「未也。策安出?」澳倉卒答曰:「若謀之外廷,則太和事可用追鑒,不若就擇可任者與計事。」帝曰:「朕固行之矣。自黃至綠,自綠至緋,猶可,衣紫即合為一矣。」澳愧汗不能對,乃罷。改京兆尹。



 帝舅鄭光主墅吏豪肆,積年不輸官賦,澳逮系之。它日延英,帝問其故。澳具道奸狀,且言必寘以法。帝曰:「可貸否?」答曰:「陛下自內署擢臣尹京邑,安可使畫一法獨行於貧下乎?」帝入白太后曰:「是不可犯。」後為輸租,乃免。由是豪右斂跡。



 會戶部闕判使,帝以問澳,澳三不對。帝曰:「任卿可乎?」曰:「臣老矣,力疲氣耗,煩劇非所任者。」帝默不樂。出謂其甥柳玼曰:「吾本不為宰相知,上便委以使務,脫謂吾他岐而得,卒無以自白。今時事浸惡,皆吾輩貪爵位致然。」未幾,授河陽節度使。入辭,帝曰:「卿自便而遠我,非我去卿。」



 懿宗立,徙平盧軍,入為吏部侍郎,復出為邠寧節度使。宰相杜審權素不悅澳,坐吏部時史盜簿書為奸,貶秘書監,分司東都。就遷河南尹,辭疾不拜,丐歸樊川。逾年,以吏部侍郎召,不起。卒,贈戶部尚書,謚曰貞。



 澳在河陽累年,宣宗遣使至魏博,道出澳所,帝以簿紙手作詔賜澳曰:「密飭裝,秋當見卿。」蓋將以為相也。因問輔養術,澳具言金石非可御,方士怪妄,宜斥遠之。其八月,帝崩,不果相。為學士時,帝嘗曰:「朕每遣方鎮刺史,欲各悉州郡風俗者,卿為朕撰一書。」澳乃取十道四方志,手加紬次,題為《處分語》。後鄧州刺史薛弘宗中謝,帝敕戒州事,人人驚服。



 綬,貫之兄。舉孝廉,又貢進士,禮部侍郎潘炎將以為舉首,綬以其友楊凝親老,故讓之,不對策輒去,凝遂及第。後擢明經,闢東都幕府。



 德宗時,以左補闕為翰林學士,密政多所參逮。帝嘗幸其院,韋妃從,會綬方寢,學士鄭絪欲馳告之,帝不許,時大寒,以妃蜀示頡袍覆而去,其待遇若此。每入直,逾月不得休。以母老,屢丐解職,每請,帝輒不悅。出入八年,而性謹畏甚。晚乃感心疾,罷還第,不極於用。九月九日,帝為《黃菊歌》,顧左右曰:「安可不示韋綬!」即遣使持往,綬遽奉和,附使進。帝曰:「為文不已,豈頤養邪?」敕自今勿復爾。終左散騎常侍。



 弟纁,有精識,為士林器許,兄弟皆名重當時。



 綬子溫。溫,字弘育。方七歲,日誦書數千言。十一,舉兩經及第,以拔萃高等補咸陽尉。父愕然,疑假權謁進,召而試諸廷,文就無留意,喜曰:「兒無愧矣!」入為監察御史,以臺制苛嚴,不可以省養,不拜。換著作郎,既謝,輒解歸。侍親疾,調適湯劑,彌二十年,衣不弛帶。既居喪,毀瘠不支。服除,李逢吉闢置宣武府。頻遷右補闕。宰相宋申錫被構,罪不測,溫倡曰:「丞相操履有初,不宜反,乃奸人陷之。吾等豈避雷霆,使上蒙霧咎邪!」率同舍伏閣切爭,由是益知名。



 太和五年,太廟室漏罅,詔宗正、將作營治,不時畢,文宗怒,責卿李銳、監王堪,奪其稟,自敕中人葺之。溫諫:「吏舉其職,國以治;事歸於正,法以修。夫設制度,立官司,度經費,則宗廟最重也。比詔下閱月,有司弛墯不力,正可黜慢官,懲不恪,擇可任者繕完之,則吏舉職,事歸正矣。今慢吏奪稟,而易以中人,是許百司公廢職,以宗廟之重,為陛下所私,臣竊惜之。請還將作,則官修業矣。」帝乃罷宦人。會群臣請上尊號,溫固諫:「今河南水,江淮旱歉,京師雪積五尺,老稚凍僕,此非崇飾虛名時。」帝順納,乃謝群臣。改侍御史。



 李德裕入輔,擢禮部員外郎。或言雅為牛僧孺厚,德裕曰:「是子堅正,可以私廢乎?」鄭注節度鳳翔,表為副,溫曰:「拒則遠黜,從之禍不測,吾焉能為注起邪?」注誅,由考功員外郎拜諫議大夫。未幾,為翰林學士。先是,綬在禁廷,積憂畏病廢,故誡溫不得任近職,至是固辭。帝怒曰:「寧綬治命邪?」禮部侍郎崔蠡曰:「溫用亂命,益所以為孝。」帝意釋,換知制誥。引疾徙太常少卿。宰相李固言薦溫給事中,帝曰:「溫素避事,肯為我論駁乎?須太子長,以為賓客。」久之,卒為給事中。



 初,兼莊恪太子侍讀,晨詣宮,日中見太子,諫曰:「殿下盛年,宜雞鳴蚤作,問安天子,如文王故事。」太子不悅。辭侍讀,見聽。王晏平罷靈武節度使,以馬及鎧仗自隨,貶康州司戶參軍,厚賂貴近,浹日,改撫州司馬,樂工尉遲璋授光州長史,溫悉封上詔書。太子得罪,詔諭群臣,溫曰:「陛下訓之不早,非獨太子罪。」時頗直其言。遷尚書右丞。鹽鐵推官姚勖按大獄,帝以為能,擢職方員外郎,將趨省,溫使戶止,即上言:「郎官清選,不可賞能吏。」帝命中人諭送,溫執議不移,詔改勖檢校禮部郎中。帝問故於楊嗣復,對曰:「勖,名臣後,治行無疵。若吏材幹而不入清選,佗日孰肯當劇事者?此衰晉風,不可以法。」帝素重溫,出為陜虢觀察使。民當輸租而麥未熟,吏白督之,溫曰:「使民貨田中穗以供賦,可乎?」為緩期而賦辦。



 武宗立,擢吏部侍郎。李德裕欲引同輔政,溫苦言李漢可釋,德裕悵然,出宣歙觀察使。池民訟刺史,劾無狀,榜殺之,威行部中。既疾,召親屬,賦綬詩「在室愧屋漏」,因泣下曰:「今知沒身不負斯誡矣!」卒,年五十八,贈工部尚書,謚曰孝。



 溫性剛峻,人望見無敢戲慢者。與楊嗣復、李玨善,嘗勸與李德裕平故憾,二人不從,及皆謫,溫嘆曰:「用吾言,孰至是邪!」一女,歸薛蒙。女工屬文,續曹大家《女訓》,行於世。溫少合,所善惟蕭祐。



 祐者,字祐之,夷澹君子也。少貧窶,隱居,以孝養聞。司農卿李實督官租,祐居喪,未及輸,召至,將責之。會有賜與,倩祐為奏,實稱善,即薦於朝。終制,以處士拜左拾遺。累遷諫議大夫,終桂州觀察使,贈右散騎常侍。精畫及書,自鐘、王、蕭、張以來,皆能識其真。謷然不以塵事自蒙,故溫號「山林友」云。



 贊曰:杜黃裳善謀,裴垍能持法,李籓鯁挺,韋貫之忠實,皆足穆天縡,經國體,撥衰奮王,菑攘四方。憲宗中興,寧不謂得人而致然邪?昔子貢孔堂高第而貨殖,韓安國漢名宰而資貪,黃裳亦以受餉見疵,至於忠烈嶢然,則不可掩已。



\end{pinyinscope}