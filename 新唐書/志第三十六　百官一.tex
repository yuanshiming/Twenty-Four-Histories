\article{志第三十六 百官一}

\begin{pinyinscope}

 唐之官制,其名號祿秩雖因時增損,而大抵皆沿隋故。其官司之別,曰省、曰臺、曰寺、曰監、曰衛、曰府,各統其屬文藝批評家。他們在政治觀點上,主張通過農民革命、消滅,以分職定位。其辯貴賤、敘勞能,則有品、有爵、有勛、有階,以時考核而升降之,所以任群材、治百事。其為法則精而密,其施於事則簡而易行,所以然者,由職有常守,而位有常員也。方唐之盛時,其制如此。蓋其始未嘗不欲立制度、明紀綱為萬世法,而常至於交侵紛亂者,由其時君不能慎守,而徇一切之茍且,故其事愈繁而官益冗,至失其職業而卒不能復。



 初,太宗省內外官,定制為七百三十員,曰:「吾以此待天下賢材,足矣。」然是時已有員外置,其後又有特置,同正員。至於檢校、兼、守、判、知之類,皆非本制。又有置使之名,或因事而置,事已則罷,或遂置而不廢。其名類繁多,莫能遍舉。自中世已後,盜起兵興,又有軍功之官,遂不勝其濫矣。故採其綱目條理可為後法,及事雖非正後世遵用因仍而不能改者,著於篇。



 宰相之職,佐天子總百官、治萬事,其任重矣。然自漢以來,位號不同,而唐世宰相,名尤不正。初,唐因隋制,以三省之長中書令、侍中、尚書令共議國政,此宰相職也。其後,以太宗嘗為尚書令,臣下避不敢居其職,由是僕射為尚書省長官,與侍中、中書令號為宰相,其品位既崇,不欲輕以授人,故常以他官居宰相職,而假以他名。自太宗時,杜淹以吏部尚書參議朝政,魏徵以秘書監參預朝政,其後或曰「參議得失」、「參知政事」之類,其名非一,皆宰相職也。貞觀八年,僕射李靖以疾辭位,詔疾小瘳,三兩日一至中書門下平章事,而「平章事」之名蓋起於此。其後,李勣以太子詹事同中書門下三品,謂同侍中、中書令也,而「同三品」之名蓋起於此。然二名不專用,而佗官居職者猶假佗名如故。自高宗已後,為宰相者必加「同中書門下三品」,雖品高者亦然;惟三公、三師、中書令則否。其後改易官名,而張文瓘以東臺侍郎同東西臺三品,「同三品」入銜,自文瓘始。永淳元年,以黃門侍郎郭待舉、兵部侍郎岑長倩等同中書門下平章事,「平章事」入銜,自待舉等始。自是以後,終唐之世不能改。



 初,三省長官議事於門下省之政事堂,其後,裴炎自侍中遷中書令,乃徙政事堂於中書省。開元中,張說為相,又改政事堂號「中書門下」,列五房於其後:一曰吏房,二曰樞機房,三曰兵房,四曰戶房,五曰刑禮房,分曹以主眾務焉。



 宰相事無不統,故不以一職名官,自開元以後,常以領他職,實欲重其事,而反輕宰相之體。故時方用兵,則為節度使;時崇儒學,則為大學士;時急財用,則為鹽鐵轉運使,又其甚則為延資庫使。至於國史、太清宮之類,其名頗多,皆不足取法,故不著其詳。



 學士之職,本以文學言語被顧問,出入侍從,因得參謀議、納諫諍,其禮尤寵;而翰林院者,待詔之所也。



 唐制,乘輿所在,必有文詞、經學之士,下至卜、醫、伎術之流,皆直於別院,以備宴見;而文書詔令,則中書舍人掌之。自太宗時,名儒學士時時召以草制,然猶未有名號;乾封以後,始號「北門學士」。玄宗初,置「翰林待詔」,以張說、陸堅、張九齡等為之,堂四方表疏批答、應和文章;既而又以中書務劇,文書多壅滯,乃選文學之士,號「翰林供奉」,與集賢院學士分掌制詔書敕。開元二十六年,又改翰林供奉為學士,別置學士院,專掌內命。凡拜免將相、號令征伐,皆用白麻。其後,選用益重,而禮遇益親,至號為「內相」,又以為天子私人。凡充其職者無定員,自諸曹尚書下至校書郎,皆得與選。入院一歲,則遷知制誥,未知制誥者不作文書,班次各以其官,內宴則居宰相之下,一品之上。憲宗時,又置「學士承旨」。唐之學士,弘文、集賢分隸中書、門下省,而翰林學士獨無所屬,故附列於此云。



 ○三師三公



 太師、太傅、太保各一人,是為三師;太尉、司徒、司空各一人,是為三公。皆正一品。三師,天子所師法,無所總職,非其人則闕。三公,佐天子理陰陽、平邦國,無所不統。親王拜者不親事,祭祀闕則攝。隋廢三師,貞觀十一年復置,與三公皆不設官屬。



 ○尚書省



 尚書令一人,正二品,掌典領百官。其屬有六尚書:一曰吏部,二曰戶部,三曰禮部,四曰兵部,五曰刑部,六曰工部。六尚書:兵部、吏部為前行,刑部、戶部為中行,工部、禮部為後行;行總四司,以本行為頭司,餘為子司。庶務皆會決焉。凡上之逮下,其制有六:一曰制,二曰敕,三曰冊,天子用之;四曰令,皇太子用之;五曰教,親王、公主用之;六曰符,省下於州,州下於縣,縣下於鄉。下之達上,其制有六:一曰表,二曰狀,三曰箋,四曰啟,五曰辭,六曰牒。諸司相質,其制有三:一曰關,二曰刺,三曰移。凡授內外百司之事,皆印其發日為程,一曰受,二曰報。諸州計奏達京師,以事大小多少為之節。凡符、移、關、牒,必遣於都省乃下。天下大事不決者,皆上尚書省。凡制敕計奏之數、省符宣告之節,以歲終為斷。龍朔二年,改尚書省曰中臺,廢尚書令;尚書曰太常伯,侍郎曰少常伯。光宅元年,改尚書省曰文昌臺,俄曰文昌都省。垂拱元年曰都臺,長安三年曰中臺。



 左右僕射各一人,從二品,掌統理六官,為令之貳,令闕則總省事,劾御史糾不當者。龍朔二年,改左右僕射曰左右匡政,光宅元年曰文昌左右相;開元元年曰左、右丞相;天寶元年復。



 左丞一人,正四品上;右丞一人,正四品下。掌辯六官之儀,糾正省內,劾御史舉不當者。吏部、戶部、禮部,左丞總焉;兵部、刑部、工部,右丞總焉。郎中各一人,從五品上;員外郎各一人,從六品上。掌付諸司之務,舉稽違,署符目,知宿直,為丞之貳。以都事受事發辰、察稽失、監印、給紙筆;以主事、令史、書令史署覆文案,出符目;以亭長啟閉、傳禁約;以掌固守當倉庫及陳設。諸司皆如之。隋尚書省諸司郎及承務郎各一人,而廢左右司。武德三年,改諸司郎為郎中,承務郎為員外郎。貞觀元年,復置左右司郎中。龍朔元年,改左右丞曰左右肅機,郎中曰左右承務,諸司郎中曰大夫。永昌元年,復置員外郎。神龍元年省,明年復置。初有馹驛百人,掌乘傳送符,後廢。



 都事各六人,從七品上;主事各六人,從八品下。吏部考功、禮部主書皆如之。諸司主事,從九品上。有令史各十八人,書令史各三十六人,亭長各六人,掌固各十四人。



 ○吏部



 尚書一人,正三品;侍郎二人,正四品上;郎中二人,正五品上;員外郎二人,從六品上。掌文選、勛封、考課之政。以三銓之法官天下之材,以身、言、書、判、德行、才用、勞效較其優劣而定其留放,為之注擬。五品以上,以名上而聽制授;六品以下,量資而任之。其屬有四:一曰吏部,二曰司封,三曰司勛,四曰考功。



 吏部郎中,掌文官階品、朝集、祿賜,給其告身、假使,一人掌選補流外官。員外郎二人,從六品上,一人判南曹。皆為尚書、侍郎之貳。凡文官九品,有正、有從,自正四品以下,有上、下,為三十等。凡文散階二十九:從一品曰開府儀同三司,正二品曰特進,從二品曰光祿大夫,正三品曰金紫光祿大夫,從三品曰銀青光祿大夫,正四品上曰正議大夫,正四品下曰通議大夫,從四品上曰太中大夫,從四品下曰中大夫,正五品上曰中散大夫,正五品下曰朝議大夫,從五品上曰朝請大夫,從五品下曰朝散大夫,正六品上曰朝議郎,正六品下曰承議郎,從六品上曰奉議郎,從六品下曰通直郎,正七品上曰朝請郎,正七品下曰宣德郎,從七品上曰朝散郎,從七品下曰宣義郎,正八品上曰給事郎,正八品下曰徵事郎,從八品上曰承奉郎,從八品下曰承務郎,正九品上曰儒林郎,正九品下曰登仕郎,從九品上曰文林郎,從九品下曰將仕郎。自四品,皆番上於吏部;不上者,歲輸資錢,三品以上六百,六品以下一千,水、旱、蟲、霜減半資。有文藝樂京上者,每州七人;六十不樂簡選者,罷輸。勛官亦如之。以征鎮功得護軍以上者,納資減三之一。凡流外九品,取其書、計、時務,其校試、銓注,與流內略同,謂之小選。



 吏部主事四人,司封主事二人,司勛主事四人,考功主事三人。武德五年改選部曰吏部,七年省侍郎。貞觀二年復置。龍朔元年改吏部曰司列,主爵曰司封,考功曰司績。武後光宅元年改吏部曰天官。垂拱元年改主爵曰司封。天寶十一載改吏部曰文部,至德二載復舊。有吏部令史三十人,書令史六十人;制書令史十四人;甲庫令史十一人,亭長八人,掌固十二人;司封令史四人,書令史九人,掌固四人;司勛令史三十三人,書令史六十七人,掌固四人;考功令史十五人,書令史三十人,掌固四人。



 司封郎中一人,從五品上;員外郎一人,從六品上;諸郎中、員外郎品皆如之。掌封命、朝會、賜予之級。凡爵九等:一曰王,食邑萬戶,正一品;二曰嗣王、郡王,食邑五千戶,從一品;三曰國公,食邑三千戶,從一品;四曰開國郡公,食邑二千戶,正二品;五曰開國縣公,食邑千五百戶,從二品;六曰開國縣侯,食邑千戶,從三品;七曰開國縣伯,食邑七百戶,正四品上;八曰開國縣子,食邑五百戶,正五品上;九曰開國縣男,食邑三百戶,從五品上。皇兄弟、皇子,皆封國為親王;皇太子子,為郡王;親王之子,承嫡者為嗣王,諸子為郡公,以恩進者封郡王;襲郡王、嗣王者,封國公。皇姑為大長公主,正一品;姊妹為長公主,女為公主,皆視一品;皇太子女為郡主,從一品;親王女為縣主,從二品。凡王、公十五以上,預朝集,宗親女婦、諸王長女月二參。內命婦,一品母為正四品郡君,二品母為從四品郡君,三品、四品母為正五品縣君。凡諸王、公主、外戚之家,卜、祝、占、相不入門。王妃、公主、郡縣主嫠居有子者,不再嫁。凡外命婦有六;王、嗣王、郡王之母、妻為妃,文武官一品、國公之母、妻為國夫人,三品以上母、妻為郡夫人,四品母、妻為郡君,五品母、妻為縣君,勛官四品有封者母、妻為鄉君。凡外命婦朝參,視夫、子之品。諸蕃三品以上母、妻授封以制。流外技術官,不封母、妻。親王,孺人二人,視正五品,媵十人,視從六品;二品,媵八人,視正七品;國公及三品,媵六人,視從七品;四品,媵四人,視正八品;五品,媵三人,視從八品。凡置媵,上其數,補以告身。散官三品以上,皆置媵。凡封戶,三丁以上為率,歲租三之一入於朝庭。食實封者,得真戶,分食諸州。皇后、諸王、公主食邑,皆有課戶。名山、大川、畿內之地,皆不以封。



 司勛郎中一人,員外郎二人,掌官吏勛級。凡十有二轉為上柱國,視正二品;十有一轉為柱國,視從二品;十轉為上護軍,視正三品;九轉為護軍,視從三品;八轉為上輕車都尉,視正四品;七轉為輕車都尉,視從四品;六轉為上騎都尉,視正五品;五轉為騎都尉,視從五品;四轉為驍騎尉,視正六品;三轉為飛騎尉,視從六品;二轉為雲騎尉,視正七品;一轉為武騎尉,視從七品。凡以功授者,覆實然後奏擬,戰功則計殺獲之數。堅城苦戰,功第一者,三轉。出少擊多,曰上陣;兵數相當,曰中陣;出多擊少,曰下陣;矢石未交,陷堅突眾,敵因而敗者,曰跳蕩。殺獲十之四,曰上獲;十之二,曰中獲;十之一,曰下獲。凡酬功之等:見任、前資、常選,曰上資;文武散官、衛官、勛官五品以上,曰次資;五品以上子孫,上柱國、柱國子,勛官六品以下,曰下資;白丁、衛士,曰無資。跳蕩人,上資加二階,次資、下資、無資以次降。凡上陣:上獲五轉,中獲四轉,下獲三轉,第二、第三等遞降焉。中陣之上獲視上陣之中獲,中獲視上陣之下獲,下獲兩轉。下陣之上獲視中陣之中獲,中獲視中陣之下獲,下獲一轉。破蠻、獠,上陣上獲,比兩番降二轉。凡勛官九百人,無職任者,番上於兵部,視遠近為十二番,以強幹者為番頭,留宿衛者為番,月上。外州分五番,主城門、倉庫,執刀。上柱國以下番上四年,驍騎尉以下番上五年,簡於兵部,授散官;不第者,五品以上復番上四年,六品以下五年,簡如初;再不中者,十二年則番上六年,八年則番上四年。勛至上柱國有餘,則授周以上親,無者賜物。太常音聲人,得五品以上勛,非征討功不除簿。諸州授勛人,歲第勛之高下,三月一報戶部,有蠲免必驗。



 考功郎中、員外郎各一人,掌文武百官功過、善惡之考法及其行狀。若死而傳於史官、謚於太常,則以其行狀質其當不;其欲銘於碑者,則會百官議其宜述者以聞,報其家。其考法,凡百司之長,歲較其屬功過,差以九等,大合眾而讀之。流內之官,敘以四善:一曰德義有聞,二曰清慎明著,三曰公平可稱,四曰恪勤匪懈。善狀之外有二十七最:一曰獻可替否,拾遺補闕,為近侍之最;二曰銓衡人物,擢盡才良,為選司之最;三曰揚清激濁,褒貶必當,為考校之最;四曰禮制儀式,動合經典,為禮官之最;五曰音律克諧,不失節奏,為樂官之最;六曰決斷不滯,與奪合理,為判事之最;七曰部統有方,警守無失,為宿衛之最;八曰兵士調習,戎裝充備,為督領之最;九曰推鞫得情,處斷平允,為法官之最;十曰讎校精審,明於刊定,為校正之最;十一曰承旨敷奏,吐納明敏,為宣納之最;十二曰訓導有方,生徒克業,為學官之最。十三曰賞罰嚴明,攻戰必勝,為軍將之最;十四曰禮義興行,肅清所部,為政教之最;十五曰詳錄典正,詞理兼舉,為文史之最;十六曰訪察精審,彈舉必當,為糾正之最;十七曰明於勘覆,稽失無隱,為句檢之最;十八曰職事脩理,供承強濟,為監掌之最;十九曰功課皆充,丁匠無怨,為役使之最;二十曰耕耨以時,收獲成課,為屯官之最;二十一曰謹於蓋藏,明於出納,為倉庫之最;二十二曰推步盈虛,究理精密,為歷官之最;二十三曰占候醫卜,效驗多者,為方術之最;二十四曰檢察有方,行旅無壅,為關津之最;二十五曰市廛弗擾,奸濫不行,為市司之最;二十六曰牧養肥碩,蕃息孳多,為牧官之最;二十七曰邊境清肅,城隍脩理,為鎮防之最。一最四善為上上,一最三善為上中,一最二善為上下;無最而有二善為中上,無最而有一善為中中,職事粗理,善最不聞,為中下;愛憎任情,處斷乖理,為下上;背公向私,職務廢闕,為下中;居官飾詐,貪濁有狀,為下下。凡定考,皆集於尚書省,唱第然後奏。親王及中書、門下、京官三品以上、都督、刺史、都護、節度、觀察使,則奏功過狀,以核考行之上下。每歲,尚書省諸司具州牧、刺史、縣令殊功異行,災蝗祥瑞,戶口賦役增減,盜賊多少,皆上於考司。監領之官,以能撫養役使者為功;有耗亡者,以十分為率,一分為一殿。博士、助教,計講授多少為差。親、勛、翊衛,以行能功過為三等,親、勛、翊衛備身,東宮親、勛、翊衛備身,王府執仗親事、執乘親事及親勛翊衛主帥、校尉、直長、品子、雜任、飛騎,皆上、中、下考,有二上第者,加階。番考別為簿,以侍郎顓掌之。流外官,以行能功過為四等:清謹勤公為上,執事無私為中,不勤其職為下,貪濁有狀為下下。凡考,中上以上,每進一等,加祿一季;中中,守本祿;中下以下,每退一等,奪祿一季。中品以下,四考皆中中者,進一階;一中上考,復進一階;一上下考,進二階;計當進而參有下考者,以一中上覆一中下,以一上下覆二中下。上中以上,雖有下考,從上第。有下下考者,解任。凡制敕不便,有執奏者,進其考。貞觀初,歲定京官望高者二人,分校京官、外官考,給事中、中書舍人各一人涖之,號監中外官考使。考功郎中判京官考,員外郎判外官考。其後屢置監考、校考、知考使。故事,考簿硃書,吏緣為奸;咸通十四年,始以墨。



 ○戶部



 尚書一人,正三品;侍郎二人,正四品下。掌天下土地、人民、錢穀之政、貢賦之差。其屬有四:一曰戶部,二曰度支,三曰金部,四曰倉部。



 戶部郎中、員外郎,掌戶口、土田、賦役、貢獻、蠲免、優復、姻婚、繼嗣之事,以男女之黃、小、中、丁、老為之帳籍,以永業、口分、園宅均其土田,以租、庸、調斂其物,以九等定天下之戶,以為尚書、侍郎之貳。其後以諸行郎官判錢穀,而戶部、度支郎官失其職矣。會昌二年著令:以本行郎官,分判錢穀。



 戶部巡官二人,主事四人;度支主事二人;金部主事三人;倉部主事三人。高宗即位,改民部曰戶部。龍朔三年,改戶部曰司元,度支曰司度,金部曰司珍,倉部曰司庾。光宅元年,改戶部曰地官。天寶十一載,改金部曰司金,倉部曰司儲。有戶部令史十七人,書令史三十四人,計史一人,亭長六人,掌固十人;度支令史十六人,書令史三十三人,計史一人,掌固四人;金部主事三人,令史十人,書令史二十一人,計史一人,掌固四人;倉部令史十二人,書令史二十三人,計史一人,掌固四人。



 度支郎中、員外郎各一人,掌天下租賦、物產豐約之宜、水陸道塗之利,歲計所出而支調之,以近及遠,與中書門下議定乃奏。



 金部郎中、員外郎各一人,掌天下庫藏出納、權衡度量之數,兩京市、互市、和市、宮市交易之事,百官、軍鎮、蕃客之賜,及給宮人、王妃、官奴婢衣服。



 倉部郎中、員外郎各一人,掌天下軍儲,出納租稅、祿糧、倉廩之事。以木契百,合諸司出給之數,以義倉、常平倉備兇年,平谷價。



 ○禮部



 尚書一人,正三品;侍郎一人,正四品下。掌禮儀、祭享、貢舉之政。其屬有四:一曰禮部,二曰祠部,三曰膳部,四曰主客。



 禮部郎中、員外郎,掌禮樂、學校、衣冠、符印、表疏、圖書、冊命、祥瑞、鋪設,及百官、宮人喪葬贈賻之數,為尚書、侍郎之貳。五禮之儀:一曰吉禮,二曰賓禮,三曰軍禮,四曰嘉禮,五曰兇禮。凡齊衰心喪以上奪情從職,及周喪未練、大功未葬,皆不預宴;大功以上喪,受冊涖官,鼓吹從而不作,戎事則否。凡朝,晚入、失儀,御史錄名奪俸,三奪者奏彈。凡出蕃冊授、吊贈者,給衣冠。皇帝巡幸,兩京文武官職事五品以上,月朔以表參起居;近州刺史,遣使一參;留守,月遣使起居;北都,則四時遣使起居。諸司奏大事者,前期三日具狀,長官躬署,對仗伏奏,仗下,中書門下涖讀。河南、太原府父老,每歲上表願駕幸,遣使以聞。駕在都,則京兆府亦如之。凡景雲、慶雲為大瑞,其名物六十有四;白狼、赤兔為上端,其名物三十有人;蒼烏、硃雁為中瑞,其名物三十有二;嘉禾、芝草、木連理為下瑞,其名物十四。大瑞,則百官詣闕奉賀;餘瑞,歲終員外郎以聞,有司告廟。凡喪,三品以上稱薨,五品以上稱卒,自六品達於庶人稱死。皇親三等以上喪,舉哀,有司帳具給食;諸蕃首領喪,則主客、鴻臚月奏。



 禮部主事二人,祠部主事二人,膳部主事二人,主客主事二人。武德三年,改儀曹郎曰禮部郎中,司籓郎曰主客郎中。龍朔二年,改禮部曰司禮,祠部曰司禋,膳部曰司膳,光宅元年,改禮部曰春官。有禮部令史五人,書令史十一人,亭長六人,掌固八人;祠部令史六人,書令史十三人,掌固四人;主客令史四人,書令史九人,掌固四人。



 祠部郎中、員外郎各一人,掌祠祀、享祭、天文、漏刻、國忌、廟諱、卜筮、醫藥、僧尼之事。珠玉珍寶供祭者,不求於市。駕部、比部歲會牲之死亡,輸皮於太府。郊祭酒醴、脯醢、黍稷、果實,所司長官封署以供。兩京及磧西諸州火祆,歲再祀,而禁民祈祭。凡巡幸,路次名山、大川、聖帝明王名臣墓,州縣以官告祭。二王後享廟,則給牲牢、祭器,而完其帷帟、幾案,主客以四時省問。凡國忌廢務日,內教、太常停習樂,兩京文武五品以上及清官七品以上,行香於寺觀。凡名醫子弟試療病,長官涖覆,三年有驗者以名聞。



 膳部郎中、員外郎各一人,掌陵廟之牲豆酒膳。諸司供奉口味,躬鐍其輿乃遣,進胙亦如之。非大禮、大慶不獻食,不進口味。凡羊,至廚而乳者釋之長生。大齋日,尚食進蔬食,釋所殺羊為長生供奉。凡獻食、進口味,不殺犢。尚食有猝須別索,必奏覆,月終而會之。凡尚食進食,以種取而別嘗之。殿中省主膳上食於諸陵,以番上下,四時遣食醫、主食各一人涖之。



 主客郎中、員外郎各一人,掌二王後、諸蕃朝見之事。二王後子孫視正三品,酅公歲賜絹三百,米粟亦如之,介公減三之一。殊俗入朝者,始至之州給牒,覆其人數,謂之邊牒。蕃州都督、刺史朝集日,視品給以衣冠、褲褶。乘傳者日四驛,乘驛者六驛。供客食料,以四時輸鴻臚,季終句會之。客初至及辭設會,第一等視三品,第二等視四品,第三等視五品,蕃望非高者,視散官而減半,參日設食。路由大海者,給祈羊豕皆一。西南蕃使還者,給入海程糧,西北諸蕃,則給度磧程糧。蕃客請宿衛者,奏狀貌年齒。突厥使置市坊,有貿易,錄奏,為質其輕重,太府丞一人涖之。蕃王首領死,子孫襲初授官,兄弟子降一品,兄弟子代攝者,嫡年十五還以政。使絕域者還,上聞見及風俗之宜、供饋贈貺之數。



 ○兵部



 尚書一人,正三品;侍郎二人,正四品下。掌武選、地圖、車馬、甲械之政。其屬有四:一曰兵部,二曰職方,三曰駕部,四曰庫部。凡將出征,告廟,授斧鉞;軍不從令,大將專決,還日,具上其罪。凡發兵,降敕書於尚書,尚書下文符。放十人,發十馬,軍器出十,皆不待敕。衛士番直,發一人以上,必覆奏。諸蕃首領至,則備威儀郊導。凡俘馘,酬以絹,入鈔之俘,歸於司農。



 郎中一人判帳及武官階品、衛府眾寡、校考、給告身之事;一人判簿及軍戎調遣之名數,朝集、祿賜、假告之常。員外郎一人掌貢舉、雜請;一人判南曹,歲選解狀,則核簿書、資歷、考課。皆為尚書、侍郎之貳。武散階四十有五;從一品曰驃騎大將軍;正二品曰輔國大將軍;從二品曰鎮軍大將軍;正三品上曰冠軍大將軍、懷化大將軍;正三品下曰懷化將軍;從三品上曰雲麾將軍、歸德大將軍;從三品下曰歸德將軍;正四品上曰忠武將軍;正四品下曰壯武將軍、懷化中郎將;從四品上曰宣威將軍;從四品下曰明威將軍、歸德中郎將;正五品上曰定遠將軍;正五品下曰寧遠將軍、懷化郎將;從五品上曰游騎將軍;從五品下曰游擊將軍、歸德郎將;正六品上曰昭武校尉;正六品下曰昭武副尉、懷化司階;從六品上曰振威校尉;從六品下曰振威副尉、歸德司階;正七品上曰致果校尉;正七品下曰致果副尉、懷化中候;從七品上曰翊麾校尉;從七品下曰翊麾副尉、歸德中候;正八品上曰宣節校尉;正八品下曰宣節副尉、懷化司戈;從八品上曰禦侮校尉;從八品下曰禦侮副尉、歸德司戈;正九品上曰仁勇校尉;正九品下曰仁勇副尉、懷化執戟長上;從九品上曰陪戎校尉;從九品下曰陪戎副尉、歸德執戟長上。自四品以上,皆番上於兵部,以遠近為八番,三月一上;三千里外者免番,輸資如文散官,唯追集乃上。六品以下,尚書省送符。懷化大將軍、歸德大將軍,配諸衛上下;餘直諸衛為十二番,皆月上。忠武將軍以下、游擊將軍以上,每番,閱強毅者直諸衛;番滿,有將略者以名聞。



 兵部主事四人,職方主事二人,駕部主事二人,庫部主事二人。龍朔二年,改兵部曰司戎,職方曰司城,駕部曰司輿,庫部曰司庫。光宅元年,改兵部曰夏官,天寶十一載曰武部,駕部曰司駕。有兵部令史三十人,書令史六十人,制書令史十三人,甲庫令史十二人,亭長八人,掌固十二人;職方令史四人,書令史九人,掌固四人;駕部令史十人,書令史二十四人,掌固四人;庫部令史七人,書令史十五人,掌固四人。



 職方郎中、員外郎各一人,掌地圖、城隍、鎮戍、烽候、防人道路之遠近及四夷歸化之事。凡圖經,非州縣增廢,五年乃脩,歲與版籍偕上。凡蕃客至,鴻臚訊其國山川、風土,為圖奏之,副上於職方;殊俗入朝者,圖其容狀、衣服以聞。



 駕部郎中、員外郎各一人,掌輿輦、車乘、傳驛、廄牧馬牛雜畜之籍。凡給馬者,一品八匹,二品六匹,三品五匹,四品、五品四匹,六品三匹,七品以下二匹;給傳乘者,一品十馬,二品九馬,三品八馬,四品、五品四馬,六品、七品二馬,八品、九品一馬;三品以上敕召者給四馬,五品三馬,六品以上有差。凡驛馬,給地四頃,蒔以苜蓿。凡三十里有驛,驛有長,舉天下四方之所達,為驛千六百三十九;阻險無水草鎮戍者,視路要隙置官馬。水驛有舟。凡傳驛馬驢,每歲上其死損、肥瘠之數。



 庫部郎中、員外郎各一人,掌戎器、鹵簿儀仗。元日冬至陳設、祠祀、喪葬,辨其名數而供焉。凡戎器,色別而異處,以衛尉幕士暴涼之。京衛旗畫蹲獸、立禽,行幸則給飛走旗。凡諸衛儀仗,以御史涖其庋掌;武庫器仗,則兵部長官涖其脩完。京官五品以上征行者,假甲、纛、旗、幡、槊;諸衛,給弓;千牛,給甲。



 ○刑部



 尚書一人,正三品;侍郎一人,正四品下。掌律令、刑法、徒隸、按覆讞禁之政。其屬有四:一曰刑部,二曰都官,三曰比部,四曰司門。



 刑部郎中、員外郎,掌律法,按覆大理及天下奏讞,為尚書、侍郎之貳。凡刑法之書有四:一曰律,二曰令,三曰格,四曰式。凡鞫大獄,以尚書侍郎與御史中丞、大理卿為三司使。凡國有大赦,集囚徒於闕下以聽。



 刑部主事四人,都官主事二人,比部主事四人,司門主事二人。龍朔二年,改刑部曰司刑,都官曰司僕,比部曰司計,司門曰司關。光宅元年,改刑部曰秋官。天寶十一載,改刑部曰司憲,比部曰司計。有刑部令史十九人,書令史三十八人,亭長六人,掌固十人;都官令史九人,書令史十二人,掌固四人;比部令史十四人,書令史二十七人,計史一人,掌固四人;司門令史六人,書令史十三人,掌固四人。



 都官郎中、員外郎各一人,掌俘隸簿錄,給衣糧醫藥,而理其訴免。凡反逆相坐,沒其家配官曹,長役為官奴婢。一免者,一歲三番役。再免為雜戶,亦曰官戶,二歲五番役。每番皆一月。三免為良人。六十以上及廢疾者,為官戶;七十為良人。每歲孟春上其籍,自黃口以上印臂,仲冬送於都官,條其生息而按比之。樂工、獸醫、騙馬、調馬、群頭、栽接之人皆取焉。附貫州縣者,按比如平民,不番上,歲督丁資,為錢一千五百;丁婢、中男,五輸其一;侍丁、殘疾半輸。凡居作者,差以三等:四歲以上為小;十一以上為中;二十以上為丁。丁奴,三當二役;中奴、丁婢,二當一役;中婢,三當一役。



 比部郎中、員外郎各一人,掌句會內外賦斂、經費、俸祿、公廨、勛賜、贓贖、徒役課程、逋欠之物,及軍資、械器、和糴、屯收所入。京師倉庫,三月一比,諸司、諸使、京都,四時句會於尚書省,以後季句前季;諸州,則歲終總句焉。



 司門郎中、員外郎各一人,掌門關出入之籍及闌遺之物。凡著籍,月一易之。流內,記官爵、姓名;流外,記年齒、貌狀。非遷解不除。凡有召者,降墨敕,勘銅魚、木契然後入。監門校尉巡日送平安。凡奏事,遣官送之,晝題時刻,夜題更籌。命婦諸親朝參者,內侍監校尉涖索。凡葦畚車,不入宮門。闌遺之物,揭於門外,榜以物色,期年沒官。天下關二十六,有上、中、下之差,度者,本司給過所;出塞逾月者,給行牒;獵手所過,給長籍,三月一易。蕃客往來,閱其裝重,入一關者,餘關不譏。



 ○工部



 尚書一人,正三品;侍郎一人,正四品下。掌山澤、屯田、工匠、諸司公廨紙筆墨之事。其屬有四:一曰工部,二曰屯田,三曰虞部,四曰水部。



 工部郎中、員外郎各一人,掌城池土木之工役程式,為尚書、侍郎之貳。凡京都營繕,皆下少府、將作共其用,役千功者先奏。凡工匠,以州縣為團,五人為火,五火置長一人。四月至七月為長功,二月、三月、八月、九月為中功,十月至正月為短功。雇者,日為絹三尺,內中尚巧匠,無作則納資。凡津梁道路,治以九月。



 工部主事三人,屯田主事二人,虞部主事二人,水部主事二人。武德三年,改起部曰工部,龍朔二年,曰司平,屯田曰司田,虞部曰司虞,水部曰司川。光宅元年,改工部曰冬官。天寶十一載,改虞部曰司虞,水部曰司水。工部有令史十二人,書令史二十一人,計史一人,亭長六人,掌固八人;屯田令史七人,書令史十二人,計史一人,掌固四人;虞部令史四人,書令史九人,掌固四人;水部令史四人,書令史九人,掌固四人。



 屯田郎中、員外郎各一人,掌天下屯田及京文武職田、諸司公廨田,以品給焉。



 虞部郎中、員外郎各一人,掌京都衢閧、苑囿、山澤草木及百官蕃客時蔬薪炭供頓、畋獵之事。每歲春,以戶小兒、戶婢仗內蒔種溉灌,冬則謹其蒙覆。凡郊祠神壇、五嶽名山,樵採、芻牧皆有禁,距遺三十步外得耕種,春夏不伐木。京兆、河南府三百里內,正月、五月、九月禁弋獵。山澤有寶可供用者,以聞。



 水部郎中、員外郎各一人,掌津濟、船艫、渠梁、堤堰、溝洫、漁捕、運漕、碾磑之事。凡坑陷、井穴皆有標。京畿有渠長、斗門長。諸州堤堰,刺史、縣令以時檢行,而涖其決築。有埭,則以下戶分牽,禁爭利者。



\end{pinyinscope}