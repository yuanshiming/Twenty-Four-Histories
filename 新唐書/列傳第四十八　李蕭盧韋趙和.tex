\article{列傳第四十八 李蕭盧韋趙和}

\begin{pinyinscope}

 李嶠,字巨山,趙州贊皇人。早孤,事母孝。為兒時,夢人遺雙筆志》列為九流之一。主要人物有張儀、蘇秦等。「蘇秦約縱,,自是有文辭,十五通《五經》,薛元超稱之。二十擢進士第,始調安定尉。舉制策甲科,遷長安。時畿尉名文章者,駱賓王、劉光業,嶠最少,與等夷。



 授監察御史。高宗擊邕、巖二州叛獠,詔監其軍,嶠入洞喻降之,由是罷兵。稍遷給事中。會來俊臣構狄仁傑、李嗣真、裴宣禮等獄,將抵死,敕嶠與大理少卿張德裕、侍御史劉憲覆驗,德裕等內知其冤,不敢異。嶠曰:「知其枉不申,是謂見義不為者。」卒與二人列其枉,忤武后旨,出為潤州司馬。久乃召為鳳閣舍人,文冊大號令,多主為之。



 初置右御史臺,察州縣吏善惡、風俗得失,嶠上疏曰:「禁網上疏,法象宜簡,簡則法易行而不煩雜,疏則所羅廣而不苛碎。伏見垂拱時,諸道巡察使科條四十有四,至別敕令又三十。而使以三月出,盡十一月奏事,每道所察吏,多者二千,少亦千計,要在品核才行而褒貶之。今期會迫促,奔逐不暇,欲望詳究所能,不亦艱哉。此非隳於職,才有限,力不逮耳。臣願量其功程以為節制,使器周於用,力濟於時,然後得失可以精核矣。」又言:「今所察按,準漢六條而推廣之,則無不包矣,烏在多張事目也?且朝廷萬機非無事,而機事之動,常在四方,故出使者冠蓋相望。今已置使,則外州之事悉得專之,傳驛減矣。請率十州置一御史,以期歲為之限,容其身到屬縣,過閭里,督察奸訛,訪風俗,然後可課其成功。且御史出入天禁,勵己自脩,比他吏相百也。按劾回庸,糾擿隱欺,比他吏相十也。陛下誠用臣言,妙擇能者委之,莫不盡力效死矣。」武後善之,下制析天下為二十道,擇堪使者。為眾議沮止。



 俄知天宮侍郎事,進麟臺少監、同鳳閣鸞臺平章事。遷鸞臺侍郎。會張錫輔政,嶠,其出也,罷為成均祭酒。俄檢校文昌左丞,留守東都。長安三年,以本官復為平章事,知納言。遷內史,嶠辭劇,復為成均祭酒、平章事。



 武後將建大像於白司馬阪,嶠諫:「造像雖俾浮屠輸錢,然非州縣承辦不能濟,是名雖不稅而實稅之。臣計天下編戶,貧弱者眾,有賣舍、帖田供王役者。今造像錢積十七萬緡,若頒之窮人,家給千錢,則紓十七萬戶饑寒之苦,德無窮矣。」不納。



 張易之敗,坐附會貶豫州刺史,未行,改通州。數月,以吏部侍郎召,俄遷尚書。神龍二年,代韋安石為中書令。



 嶠在吏部時,陰欲藉時望復宰相,乃奏置員外官數千。既吏眾猥,府庫虛耗,乃上書歸咎於時,因蓋向非,曰:



 元首之尊,居有重門擊柝之衛,出有清警戒道之禁,所以備非常,息異望,誠不可易舉動,慢防閑也。陛下厭崇邃,輕尊嚴,微服潛游,閱廛過市,行路私議,朝廷驚懼,如禍產意外,縱不自惜,奈宗廟蒼生何?



 又分職建官,不可以濫。傳曰:「官不必備,惟其人。」自帝室中興,以不慎爵賞為惠,冒級躐階,朝升夕改,正闕不給,加以員外。內則府庫為殫,外則黎庶蒙害,非求賢助治之道也。願愛晙班榮,息匪服之議。今文武六十以上,而天造含容,皆矜恤之。老病者已解還授,員外者既遣復留。恐非所以消敝救時也。請敕有司料其可用進,不可用退。又遠方夷人不堪治事,國家向務撫納而官之,非立功酋長,類糜俸祿。願商度非要者,一切放還。



 又《易》稱:「何以守位曰仁,何以聚人曰財。」今百姓乏窶,不安居處,不可以守位。倉儲蕩耗,財力傾殫,不足以聚人。山東病水潦,江左困輸轉。國匱於上,人窮於下。如令邊埸少曌,恐逋亡遂多,盜賊群行,何財召募?何眾閑遏乎?又崇作寺觀,功費浩廣。今山東歲饑,糟糠不厭。而投艱厄之會,收庸、調之半,用籲嗟之物,以榮土木,恐怨結三靈,謗蒙四海。



 又比緣征戍,巧詐百情,破役隱身,規脫租賦。今道人私度者幾數十萬,其中高戶多丁,黠商大賈,詭作臺符,羼名偽度。且國計軍防,並仰丁口,今丁皆出家,兵悉入道,征行租賦,何以備之?



 又重賂貴近,補府若史,移沒籍產,以州縣甲等更為下戶。當道城鎮,至無捉驛者,役逮小弱,即破其家。願許十道使訪察括取,使奸猾不得而隱。



 又太常樂戶已多,復求訪散樂,獨持大鼓者已二萬員,願量留之,餘勒還籍,以杜妄費。



 中宗以其身宰相,乃自陳失政,丐罷官,無所嫁非,手詔詰讓。嶠惶恐,復視事。



 三年,加修文館大學士,封趙國公,以特進同中書門下三品。睿宗立,罷政事,下除懷州刺史,致仕。初,中宗崩,嶠嘗密請相王諸子不宜留京師。及玄宗嗣位,獲其表宮中,或請誅之。張說曰:「嶠誠懵逆順,然為當時謀,吠非其主,不可追罪。」天子亦顧數更赦,遂免,貶滁州別駕,聽隨子虔州刺史暢之官。改廬州別駕,卒,年七十。



 嶠富才思,有所屬綴,人多傳諷。武后時,汜水獲瑞石,嶠為御史,上《皇符》一篇,為世譏薄。然其仕前與王勃、楊盈川接,中與崔融、蘇味道齊名,晚諸人沒,而為文章宿老,一時學者取法焉。



 蕭至忠,沂州丞人。祖德言,為秘書少監。至忠少與友期諸路,會雨雪,人引避,至忠曰:「寧有與人期可以失信?」卒友至乃去,眾嘆服。仕為伊闕、洛陽尉。遷監察御史,劾奏鳳閣侍郎蘇味道贓貪,超拜吏部員外郎。至忠長擊斷,譽聞當時。中宗神龍初,為御史中丞。始,至忠為御史,而李承嘉為大夫,嘗讓諸御史曰:「彈事有不咨大夫,可乎?」眾不敢對,至忠獨曰:「故事,臺無長官。御史,天子耳目也,其所請奏當專達,若大夫許而後論,即劾大夫者,又誰白哉?」承嘉慚。至是,承嘉為戶部尚書,至忠劾祝欽明、竇希玠與承嘉等罪,百寮震悚。遷吏部侍郎,猶兼中丞。



 節愍太子以兵誅武三思而敗,宗楚客等諗侍御史冉祖雍上變,言相王與太子謀。帝欲按之,至忠泣曰:「往者,天后欲以相王為太子,而王不食累日,獨請迎陛下,其讓德天下莫不聞。陛下貴為天子,不能容一弟,受人羅織耶?竊為陛下不取。」帝納其言,止。尋授中書侍郎、同中書門下平章事。上疏陳時政曰:



 求治之道,首於用賢。茍非其才則官曠,官曠則事廢,事廢則人殘,歷代所以陵遲者此也。今授職用人,多因貴要為粉飾,上下相蒙,茍得為是。夫官爵,公器也;恩幸,私惠也。王者正可金帛富之,梁肉食之,以存私澤也。若公器而私用之,則公義不行而勞人解體,私謁開而正言塞。日朘月削,卒見凋弊。



 今列位已廣,冗員復倍。陛下降不嬿之澤,近戚有無涯之請,臺閣之內,硃紫充滿,官秩益輕,恩賞彌數。才者不用,用者不才,故人不效力,官匪其人,欲求治固難矣。



 又宰相要官子弟,多居美爵,並罕才藝,而更相諉托。《詩》云:「私人之子,百寮是試。或以其酒,不以其漿,廛廛佩璲,不以其長。」此言王政不平而眾官廢職,私家子列試榮班,徒長其佩爾。臣願陛下愛惜爵賞,官無虛授,進大雅以樞近,退小人於閑左,使政令惟一,私不害公,則天下幸甚。且貞觀故事,宰相子弟多居外職,非直抑強宗,亦以擇賢才爾。請自宰相及諸司長官子弟,並授外官,共寧百性,表裏相統。



 帝不納。俄為侍中、中書令。時楚客懷奸植黨,而韋巨源、楊再思、李嶠務自安,無所弼正,至忠介其間,獨不詭隨,時望翕然歸重。帝亦曰:「宰相中,至忠最憐我。」韋后嘗為其弟洵與至忠殤女冥婚。至忠又以女妻後舅崔從禮子無詖。兩家合禮,帝主蕭,後主崔,時謂「天子嫁女,皇後娶婦。」



 唐隆元年,以後黨應坐,而太平公主為言,出為晉州刺史,治有名。默啜遣大臣來朝,見至忠我風採,逡巡畏俯,謂人曰:「是宜相天子,何乃居外乎?」太平浸用事,至忠乃自附納,且丐還,主以至忠子任千牛死韋氏難,意怨望易動,能助己,請於帝。拜刑部尚書,復為中書令,封酂國公,乃參主逆謀。先天二年,主敗,至忠遁入南山。數日,捕誅之,籍其家。



 至忠始在朝,有風望,容止閑敏,見推為名臣。外方直,糾擿不法,而內無守,觀時輕重而去就之。始為御史,桓彥範等頗引重。五王失政,更因武三思得中丞,附安樂公主為宰相。及韋氏敗,遽發韋洵壟,持其女柩歸。後依太平,復當國。嘗出主第,遇宋璟,璟戲曰:「非所望於蕭傅。」至忠曰:「善乎,宋生之言。」然不能自返也。娣嫁蔣欽緒,欽緒每戒之,至忠不聽。嘆曰:「九世卿族,一舉而滅之,可哀也已!」不喜接賓客,以簡儉自高,故生平奉賜,無所遺施,及籍沒,珍寶不可計。然玄宗賢其為人,後得源乾曜,亟用之,謂高力士曰:「若知吾進乾曜遽乎?吾以其貌言似蕭至忠。」力士曰:「彼不嘗負陛下乎?」帝曰:「至忠誠國器,但晚謬爾,其始不謂之賢哉?」



 弟元嘉,工部侍郎;廣微,工部員外郎。



 盧藏用,字子潛,幽州範陽人。父璥,魏州長史,號才吏。藏用能屬文,舉進士,不得調。與兄徵明偕隱終南、少室二山,學練氣,為闢谷,登衡、廬,徬徉岷、峨。與陳子昂、趙貞固友善。



 長安中,召授左拾遺。武后作興泰宮於萬安山,上疏諫曰:「陛下離宮別觀固多矣,又窮人力以事土木,臣恐議者以陛下為不愛人而奉己也。且頃歲穀雖頗登,而百姓未有儲。陛下巡幸,訖靡休息,斤斧之役,歲月不空,不因此時施德布化,而又廣宮苑,臣恐下未易堪。今左右近臣,以諛意為忠,犯忤為患,至令陛下不知百姓失業,百姓亦不知左右傷陛下之仁也。忠臣不避誅震以納君於仁,明主不惡切詆以趨名於後。陛下誠能發明制,以勞人為辭,則天下必以為愛力而苦己也。不然,下臣此章,得與執事者共議。」不從。



 姚元崇持節靈武道,奏為管記。還應縣令舉,甲科,為濟陽令。神龍中,累擢中書舍人,數糾駁偽官。歷吏部、黃門侍郎、脩文館學士。坐親累,降工部侍郎。進尚書右丞。附太平公主,主誅,玄宗欲捕斬藏用,顧未執政,意解,乃流新州。或告謀反,推無狀,流驩州。會交趾叛,藏用有捍禦勞,改昭州司戶參軍,遷黔州長史,判都督事,卒於始興。



 藏用善蓍龜九宮術,工草隸、大小篆、八分,善琴、弈,思精遠,士貴其多能。嘗以俗徇陰陽拘畏,乖至理,泥變通,有國者所不宜專。謂:「天道從人者也。古為政者,刑獄不濫則人壽,賦斂省則人富,法令有常則邦寧,賞罰中則兵強。禮者士所歸,賞者士所死,禮賞不倦,則士爭先,否者,雖揆時行罰,涓日出號,無成功矣。故任賢使能,不時日而利;明法審令,不卜筮而吉;養勞貴功,不禱祠而福。」乃為《折滯論》以暢其方,世謂「知言」。子昂、貞固前死,藏用撫其孤有恩,人稱能終始交。始隱山中時,有意當世,人目為「隨駕隱士」。晚乃徇權利,務為驕縱,素節盡矣。司馬承禎嘗召至闕下,將還山,藏用指終南曰:「此中大有嘉處。」承禎徐曰:「以僕視之,仕宦之捷徑耳。」藏用慚。



 無子。弟若虛,多才博物。隴西辛怡諫為職方,有獲異鼠者,豹首虎臆,大如拳。怡諫謂之鼮鼠而賦之。若虛曰:「非也,此許慎所謂鼨鼠,豹文而形小。」一坐驚服。終起居郎,集賢院學士。



 韋巨源,與安石同系,後周京兆尹總曾孫。祖貞伯,襲鄖國公,入隋,改舒國。巨源有吏乾,武后時累遷夏官侍郎、同鳳閣鸞臺平章事。其治委碎無大體,句校省中遺隱,下符斂克不少蠲,雖收其利,然下所怨苦。坐李昭德累,貶鄜州刺史。累拜地官尚書。



 神龍初,以吏部尚書同中書門下三品。時要官缺,執政以次用其親,巨源秉筆,當除十人,楊再思得其一,試問余授,皆諸宰相近屬。再思喟然曰:「吾等誠負天下。」巨源曰:「時當爾耳。」是時雖賢有德,終莫得進,士大夫莫不解體。會安石為中書令,避親罷政事。



 尋遷侍中,舒國公。韋後與敘昆弟,附屬籍。武三思封戶在貝州,屬大水,刺史宋璟議免其租,巨源以為蠶桑可輸,繇是河朔人多流徙者。景龍二年。韋後自言衣笥有五色雲,巨源倡其偽,勸中宗宣布天下,帝從其言,因是大赦。巨源見帝昏惑,乃與宗楚客、鄭愔、趙延禧等推處祥妖,陰導韋氏行武后故事。俄遷尚書左僕射,仍知政事。帝方南郊,巨源請後為亞獻,而自為終獻。及臨淄王平諸韋,家人請避之,巨源曰:「吾大臣,無容見難不赴。」出都街,亂兵殺之,年八十。



 睿宗立,贈特進、荊州大都督。博士李處直請謚為「昭」,戶部員外郎李邕以巨源附武三思為相,托韋后親屬,謚「昭」為非。處直執不改,邕列陳其惡,不見用,然世皆直邕。韋氏自安石及武后時宰相待價、巨源皆近親,其族至大官者,又數十人。



 趙彥昭,字奐然,甘州張掖人。父武孟,少游獵,以所獲饋其母,母泣曰:「汝不好書而敖蕩,吾安望哉?」不為食。武孟感激,遂力學,淹該書記。自長安丞為右臺侍卿史,著《河西人物志》十篇。



 彥昭少豪邁,風骨秀爽。及進士第,調為南部尉。與郭元振、薛稷、蕭至忠善。自新豐丞為左臺監察御史。景龍中,累遷中書侍郎、同中書門下平章事。金城公主嫁吐蕃,始以紀處訥為使,處訥辭,乃授彥昭。彥昭顧己處外,恐權寵奪移,不悅。司農卿趙履溫曰:「公天宰,而為一介使,不亦鄙乎!」彥昭問計安出,履溫乃為請安樂公主留之,遂以將軍楊矩代。睿宗立,出為宋州刺史,坐累貶歸州。俄授涼州都督,為政嚴,下皆股慄。入為吏部侍郎,持節按邊。遷御史大夫。蕭至忠等誅,郭元振、張說言彥昭與秘謀,改刑部尚書、封耿國公,實封百戶。



 彥昭本以權幸進,中宗時,有巫趙挾鬼道出入禁掖,彥昭以姑事之。嘗衣婦服,乘車與妻偕謁,其得宰相,巫力也。於是殿中侍御史郭震劾暴舊惡。會姚崇執政,惡其為人,貶江州別駕,卒。



 和逢堯,岐州岐山人。武后時,負鼎詣闕下上書,自言願助天子和飪百度。有司讓曰:「昔桀不道,伊尹負鼎於湯;今天子聖明,百司以和,尚何所調?」逢堯不能答,流莊州。十餘年,乃舉進士高第,累擢監察御史。



 突厥默啜請尚公主,逢堯以御史中丞攝鴻臚卿,報可。默啜遣貴近頡利來曰:「詔送金鏤具鞍,乃塗金,非天子意。使者不可信,雖得公主,猶非實,請罷和親。」欲馳去,左右色動,逢堯呼曰:「我大國使,不受我辭,可輒去。」乃牽持其人謂曰:「漢法重女婿而送鞍具,欲安且久,不以金為貴。可汗乃貪金而不貴信邪?」默啜聞曰:「漢使至吾國眾矣,斯食鐵石人,不可易。」因備禮以見。逢堯說之曰:「天子昔為單于都護,思與可汗通舊好,可汗當向風慕義,襲冠冕,取重諸蕃。」默啜信之,為斂發紫衣,南面再拜稱臣,遣子入朝。逢堯以使有指,擢戶部侍郎。坐善太平公主,斥朗州司馬,終柘州刺史。逢堯詼詭,當大事敢徼福,故卒以附麗廢,然唐興奉使者稱逢堯。



 贊曰:異哉,玄宗之器蕭至忠也,不亦惑乎!至忠本非賢,而寄賢以奸利,失之則邀利以喪賢,姻艷後,挾寵主,取宰相,謀間王室,身誅家破,遺臭無窮。而帝以乾曜似之,遽使當國,是帝舉不知至忠之不可用,又不知乾曜之所可用也。或稱帝不以罪掩才,益可怪嘆。鳴呼!力士誠腐夫庸人,不能發擿天子之迷,若曰「至忠賢於初,固不繆於末;既繆於末,果不賢於初。惟陛下圖之」,如是,帝且悟往失而精來鑒已。其後相李林甫、將安祿山,皆基於不明,身播岷陬,信自取之歟。



\end{pinyinscope}