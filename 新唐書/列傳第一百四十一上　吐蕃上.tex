\article{列傳第一百四十一上 吐蕃上}

\begin{pinyinscope}

 吐蕃本西羌屬,蓋百有五十種,散處河、湟、江、岷間,有發羌、唐旄等,然未始與中國通。居析支水西。祖曰鶻提勃悉野犬儒主義見「犬儒學派」。,健武多智,稍並諸羌,據其地。蕃、發聲近,故其子孫曰吐蕃,而姓勃窣野。或曰南涼禿發利鹿孤之後。二子,曰樊尼,曰傉檀。傉檀嗣,為乞佛熾盤所滅。樊尼挈殘部臣沮渠蒙遜,以為臨松太守。蒙遜滅,樊尼率兵西濟河,逾積石,遂撫有群羌云。



 其俗謂強雄曰贊,丈夫曰普,故號君長曰贊普,贊普妻曰末蒙。其官有大相曰論茝,副相曰論茝扈莽,各一人,亦號大論、小論;都護一人,曰悉編掣逋;又有內大相曰曩論掣逋,亦曰論莽熱,副相曰曩論覓零逋,小相曰曩論充,各一人;又有整事大相曰喻寒波掣逋,副整事曰喻寒覓零逋,小整事曰喻寒波充:皆任國事,總號曰尚論掣逋突瞿。地直京師西八千里,距鄯善五百里,勝兵數十萬。國多霆、電、風、雹,積雪,盛夏如中國春時,山谷常冰。地有寒癘,中人輒痞促而不害。其贊普居跋布川,或邏娑川,有城郭廬舍不肯處,聯毳帳以居,號大拂廬,容數百人。其衛候嚴,而牙甚隘。部人處小拂廬,多老壽至百餘歲者。衣率氈韋,以赭塗面為好。婦人辮發而縈之。其器屈木而韋底,或氈為般,凝面為碗,實羹酪並食之,手捧酒漿以飲。其官之章飾,最上瑟瑟,金次之,金塗銀又次之,銀次之,最下至銅止,差大小,綴臂前以辨貴賤。屋皆平上,高至數丈。其稼有小麥、青稞麥、蕎麥、〓豆。其獸,犛牛、名馬、犬、羊、彘,天鼠之皮可為裘,獨峰駝日馳千里。其寶,金、銀、錫、銅。其死,葬為塚,塈塗之。其吏治,無文字,結繩齒木為約。其刑,雖小罪必抉目,或刖、劓,以皮為鞭抶之,從喜怒,無常算。其獄,窟地深數丈,內囚於中,二三歲乃出。其宴大賓客,必驅耗牛,使客自射,乃敢饋。其俗,重鬼右巫,事原羝為大神。喜浮屠法,習咒詛,國之政事,必以桑門參決。多佩弓刀。飲酒不得及亂。婦人無及政。貴壯賤弱,母拜子,子倨父,出入前少而後老。重兵死,以累世戰沒為甲門,敗懦者垂狐尾於首示辱,不得列於人。拜必手據地為犬號,再揖身止。居父母喪,斷發、黛面、墨衣,既葬而吉。其舉兵,以七寸金箭為契。百里一驛,有急兵,驛人臆前加銀鶻,甚急,鶻益多。告寇舉烽。其畜牧,逐水草無常所。其鎧胃精良,衣之周身,竅兩目,勁弓利刃不能甚傷。其兵法嚴,而師無饋糧,以鹵獲為資。每戰,前隊盡死,後隊乃進。其四時,以麥熟為歲首。其戲,棋、六博。其樂,吹螺、擊鼓。其君臣自為友,五六人曰共命。君死,皆自殺以殉,所服玩乘馬皆瘞,起大屋塚顛,樹眾木為祠所。贊普與其臣歲一小盟,用羊、犬、猴為牲;三歲一大盟,夜肴諸壇,用人、馬、牛、閭為牲。凡牲必折足裂腸陳於前,使巫告神曰:「渝盟者有如牲。」



 其後有君長曰瘕悉董摩,董摩生佗土度,佗土生揭利失若,揭利生勃弄若,勃弄生詎素若,詎素生論贊索,論贊生棄宗弄贊,亦名棄蘇農,亦號弗夜氏。其為人慷慨才雄,常驅野馬、〓牛,馳刺之以為樂,西域諸國共臣之。



 太宗貞觀八年,始遣使者來朝,帝遣行人馮德遐下書臨撫。弄贊聞突厥、吐谷渾並得尚公主,乃遣使齎幣求昏,帝不許。使者還,妄語曰:「天子遇我厚,幾得公主,會吐谷渾王入朝,遂不許,殆有以間我乎?」弄贊怒,率羊同共擊吐谷渾,吐谷渾不能亢,走青海之陰,盡取其貲畜。又攻黨項、白蘭羌,破之。勒兵二十萬入寇松州,命使者貢金甲,且言迎公主,謂左右曰:「公主不至,我且深入。」都督韓威輕出覘賊,反為所敗,屬羌大擾,皆叛以應賊。乃詔吏部尚書侯君集為行軍大總管,出當彌道,右領軍大將軍執失思力出白蘭道,右武衛大將軍牛進達出闊水道,右領軍將軍劉蘭出洮河道,並為行軍總管,率步騎五萬進討。進達自松州夜鏖其營,斬首千級。



 初東寇也,連歲不解,其大臣請返國,不聽,自殺者八人。至是弄贊始懼,引而去,以使者來謝罪,固請昏,許之。遣大論薛祿東贊獻黃金五千兩,它寶稱是,以為聘。



 十五年,妻以宗女文成公主,詔江夏王道宗持節護送,築館河源王之國。弄贊率兵次柏海親迎,見道宗,執婿禮恭甚,見中國服飾之美,縮縮愧沮。歸國,自以其先未有昏帝女者,乃為公主築一城以誇後世,遂立宮室以居。公主惡國人赭面,弄贊下令國中禁之。自褫氈罽,襲紈綃,為華風。遣諸豪子弟入國學,習《詩》、《書》。又請儒者典書疏。



 帝伐遼還,使祿東贊上書曰:「陛下平定四方,日月所照,並臣治之。高麗恃遠,弗率於禮,天子自將度遼,隳城陷陣,指日凱旋,雖雁飛於天,無是之速。夫鵝猶雁也,臣謹冶黃金為鵝以獻。」其高七尺,中實酒三斛。二十二年,右衛率府長史王玄策使西域,為中天竺所鈔,弄贊發精兵從玄策討破之,來獻俘。



 高宗即位,擢駙馬都尉、西海郡王。弄贊以書詒長孫無忌曰:「天子初即位,下有不忠者,願勒兵赴國共討之。」並獻金琲十五種以薦昭陵。進封賨王,賜餉蕃渥。又請蠶種、酒人與碾磑等諸工,詔許。永徽初,死,遣使者吊祠。無子,立其孫,幼不事,故祿東贊相其國。



 顯慶三年,獻金盎、金頗羅等,復請昏。未幾,吐谷渾內附,祿東贊怨忿,率銳兵擊之,而吐谷渾大臣素和貴奔吐蕃,槊以虛實,故吐蕃能破其國。慕容諾曷缽與弘化公主引殘落走涼州,詔涼州都督鄭仁泰為青海道行軍大總管,率將軍獨孤卿雲等屯涼、鄯,左武候大將軍蘇定方為安集大使,為諸將節度,以定其亂。吐蕃使論仲琮入朝,表吐谷渾罪,帝遣使者譙讓,乃使來請與吐谷渾平憾,求赤水地牧馬,不許。會祿東贊死。



 東贊不知書而性明毅,用兵有節制,吐蕃倚之,遂為強國。始入朝,占對合旨,太宗擢拜右衛大將軍,以瑯邪公主外孫妻之。祿東贊自言:「先臣為聘婦,不敢奉詔。且贊普未謁公主,陪臣敢辭!」帝異其言,然欲懷以恩,不聽也。有子曰欽陵、曰贊婆、曰悉多於、曰勃論。祿東贊死,而兄弟並當國。自是歲入邊,盡破有諸羌羈縻十二州。



 總章中,議徙吐谷渾部於涼州旁南山。帝刈吐蕃之入,召宰相姜恪閻立本、將軍契苾何力等議先擊吐蕃。立本曰:「民饑未可以師。」何力曰:「吐蕃介在西極,臣恐師到,獸竄山伏,捕討無所得,至春復侵吐谷渾。臣請勿救,使疑吾力困而驕之,一舉可滅也。」恪曰:「不然,吐谷渾方衰,吐蕃負勝,以衰氣拒勝兵,戰必不亢,不救則滅。臣謂王師亟助之,使國幸存,後且徐圖可也。」議不決,亦不克徙。



 咸亨元年,入殘羈縻十八州,率於闐取龜茲撥換城,於是安西四鎮並廢。詔右威衛大將軍薛仁貴為邏娑道行軍大總管,左衛員外大將軍阿史那道真、左衛將軍郭待封自副,出討吐蕃,並護吐谷渾還國。師凡十餘萬,至大非川,為欽陵所拒,王師敗績,遂滅吐谷渾而盡有其地。詔司戎太常伯、同東西臺三品姜恪為涼州道行軍大總管出討,會恪卒,班師。



 吐蕃遣大臣仲琮入朝。仲琮少游太學,頗知書。帝召見問曰:「贊普孰與其祖賢?」對曰:「勇果善斷不逮也,然勤以治國,下無敢欺,令主也。且吐蕃居寒露之野,物產寡薄,烏海之陰,盛夏積雪,暑毼冬裘。隨水草以牧,寒則城處,施廬帳。器用不當中國萬分一。但上下一力,議事自下,因人所利而行,是能久而強也。」帝曰:「吐谷渾與吐蕃本甥舅國,素和貴叛其主,吐蕃任之,奪其土地。薛仁貴等往定慕容氏,又伏擊之,而寇我涼州,何邪?」仲琮頓首曰:「臣奉命來獻,它非所聞。」帝韙其答。然以仲琮非用事臣,故殺其禮。



 上元二年,遣大臣論吐渾彌來請和,且求與吐谷渾脩好,帝不聽。明年,攻鄯、廓、河、芳四州,殺略吏及馬牛萬計。乃詔周王顯為洮州道行軍元帥,率工部尚書劉審禮等十二總管,以相王輪為涼州道行軍元帥,率左衛大將軍契苾何力、鴻臚卿蕭嗣業等軍討之。二王不克行。吐蕃進攻疊州,破密恭、丹嶺二縣,又攻扶州,敗守將。乃高選尚書左僕射劉仁軌為洮河鎮守使,久之,無功。



 吐蕃與西突厥連兵攻安西,復命中書令李敬玄為洮河道行軍大總管、西河鎮撫大使、鄯州都督,代仁軌。下詔募猛士,毋限籍役痕負,帝自臨遣。又敕益州長史李孝逸、巂州都督拓王奉益發劍南、山南士。先戰龍支,吐蕃敗。敬玄率劉審禮擊吐蕃青海上,審禮戰沒。敬玄頓承風嶺,礙險不得縱,吐蕃壓王師屯,左領軍將軍黑齒常之率死士五百,夜斧其營,虜驚,自相轥藉而死者甚眾,乃引去。敬玄僅脫。



 帝既儒仁無遠略,見諸將數敗,乃博咨近臣,求所以御之之術。帝曰:「朕未始擐甲履軍,往者滅高麗、百濟,比歲用師,中國騷然,朕至今悔之。今吐蕃內侵,盍為我謀?」中書舍人劉禕之等具對,須家給人足可擊也。或言賊險黠不可與和,或言營田嚴守便。惟中書侍郎薛元超謂:「縱敵生患,不如料兵擊之。」帝顧黃門侍郎來恆曰:「自李勣亡,遂無善將。」恆即言:「向洮河兵足以制敵,但諸將不用命,故無功。」帝殊不悟,因罷議。



 儀鳳四年,贊普死,子器弩悉弄立,欽陵復擅政,使大臣來告喪,帝遣使者往會葬。明年,贊婆、素和貴率兵三萬攻河源,屯良非川,敬玄與戰湟川,敗績。左武衛將軍黑齒常之以精騎三千夜搗其營,贊婆懼,引去。遂擢常之為河源軍經略大使。乃嚴烽邏,開屯田,虜謀稍折。



 初,劍南度茂州之西築安戎城,以迮其鄙。俄為生羌導虜取之以守,因並西洱河諸蠻,盡臣羊同、黨項諸羌。其地東與松、茂、巂接,南極婆羅門,西取四鎮,北抵突厥,幅圓餘萬里,漢、魏諸戎所無也。



 永隆元年,文成公主薨,遣使者吊祠,又歸我陳行焉之喪。初,行焉使虜,論欽陵欲拜己,臨以兵,不為屈,留之十年。及是喪還,贈睦州刺史。贊婆復入良非川,常之擊走之。



 武后時,與蠻夷同朝賀。永昌元年,詔文昌右相韋待價為安息道大總管,安西大都護閻溫古副之,以討吐蕃,兵逗留,坐死、徙。明年,復詔文昌右相岑長倩為武威道行軍大總管討之,兵半道罷。



 又明年,大首領曷蘇率貴川部與黨項種三十萬降,後以右玉鈐衛將軍張玄遇為安撫使,率兵二萬迎之,次大度水,吐蕃禽曷蘇去。而它酋昝插又率羌、蠻八千自來,玄遇即其部置葉州,用昝插為刺史,刻石大度山以紀功。



 是歲,又詔右鷹揚衛將軍王孝傑為武威道行軍總管,率西州都督唐休璟、左武衛大將軍阿史那忠節擊吐蕃,大破其眾,復取四鎮,更置安西都護府於龜茲,以兵鎮守。議者請廢四鎮勿有也,右史崔融獻議曰:「戎狄為中國患尚矣,五帝、三王所不臣。漢以百萬眾困平城,其後武帝赫然發憤,甘心四夷,張騫始通西域,列四郡,據兩關,斷匈奴右臂,稍稍度河、湟,築令居,以絕南羌。於是鄣候亭燧出長城數千里,傾府庫,殫士馬,行人使者歲月不絕,至作皮幣,算緡法,稅舟車,榷酒酤。夫豈不懷,為長久計然也!匈奴於是孤特遠竄,遂開西域,置使者領護。光武中興,皆復內屬,至於延光,三絕三通。太宗文皇帝踐漢舊跡,並南山抵蔥嶺,剖裂府鎮,煙火相望,吐蕃不敢內侮。高宗時,有司無狀,棄四鎮不能有,而吐蕃遂張,入焉耆之西,長鼓右驅,逾高昌,歷車師,鈔常樂,絕莫賀延磧,以臨燉煌。今孝傑一舉而取四鎮,還先帝舊封,若又棄之,是自毀成功而破完策也。夫四鎮無守,胡兵必臨西域,西域震則威憺南羌,南羌連衡,河西必危。且莫賀延磧袤二千里,無水草,若北接虜,唐兵不可度而北,則伊西、北庭、安西諸蕃悉亡。」議乃格。



 於是首領勃論贊與突厥偽可汗阿史那俀子南侵,與孝傑戰冷泉,敗走。碎葉鎮守使韓思忠破泥熟沒斯城。證聖元年,欽陵、贊婆攻濫洮,孝傑以肅邊道大總管戰素羅汗山,虜敗還。又攻涼州,殺都督。遣使者請和,約罷四鎮兵,求分十姓地。武后詔通泉尉郭元振往使,道與欽陵遇。元振曰:「東贊事朝廷,誓好無窮,今猥自絕,歲擾邊,父通之,子絕之,孝乎?父事之,子叛之,忠乎?」欽陵曰:「然!然天子許和,得罷二國戍,使十姓突厥、四鎮各建君長,俾其國自守若何?」元振曰:「唐以十姓、四鎮撫西土,為列國主,道非有它,且諸部與吐蕃異,久為唐編人矣。」欽陵曰:「使者意我規削諸部為唐邊患邪?我若貪土地財賦,彼青海、湟川近矣,今舍不爭何哉?突厥諸部磧漠廣莽,去中國遠甚,安有爭地萬里外邪?且四夷唐皆臣並之,雖海外地際,靡不磨滅,吐蕃適獨在者,徒以兄弟小心,得相保耳。十姓五咄陸近安西,於吐蕃遠,俟斤距我裁一磧,騎士騰突,不易旬至,是以為憂也。烏海、黃河,關源阻奧,多癘毒,唐必不能入;則弱甲孱將易以為蕃患,故我欲得之,非規諸部也。甘、涼距積石道二千里,其廣不數百,狹才百里,我若出張掖、玉門,使大國春不耕,秋不獲,不五六年,可斷其右。今棄不為,亦無虞於我矣。青海之役,黃仁素約和,邊守不戒,崔知辯徑俟斤掠我牛羊萬計,是以求之。」使使者固請,元振固言不可許,後從之。



 欽陵專國久,常居中制事,諸弟皆領方面兵,而贊婆專東境幾三十年,為邊患。兄弟皆才略沈雄,眾憚之。器弩悉弄既長,欲自得國,漸不平,乃與大臣論巖等圖去之。欽陵方提兵居外,贊普托言獵,即勒兵執其親黨二千餘人殺之,發使者召欽陵、贊婆,欽陵不受命,贊普自討之。未戰,欽陵兵潰,乃自殺,左右殉而死者百餘人。



 贊婆以所部及兄子莽布支等款塞,遣羽林飛騎迎勞,擢贊婆特進、輔國大將軍、歸德郡王,莽布支左羽林大將軍、安國公,皆賜鐵券,禮尉良厚。贊婆即領部兵戍河源,死,贈安西大都護。又遣左肅政臺御史大夫魏元忠為隴右諸軍大總管,率隴右諸軍大使唐休璟出討。方虜攻涼州,休璟擊之,斬首二千級。於是論彌薩來朝請和。贊普自將萬騎攻悉州,都督陳大慈四戰皆克。明年,乃獻馬、黃金求昏。而虜南屬帳皆叛,贊普自討,死於軍。諸子爭立,國人立棄隸〓贊為贊普,始七歲,使者來告喪,且求盟。又使大臣悉董熱固求昏,未報。會監察御史李知古建討姚州蠻,削吐蕃向導,詔發劍南募士擊之。蠻酋以情輸虜,殺知古,尸以祭天,進攻蜀漢。詔靈武監軍右臺御史唐九徵為姚巂道討擊使,率兵擊之。虜以鐵絙梁漾、水鼻二水,通西洱蠻,築城戍之。九徵毀絙夷城,建鐵柱於滇池以勒功。



 中宗景龍二年,還其昏使。或言彼來逆公主,且習聞華言,宜勿遣,帝以中國當以信結夷狄,不許。明年,吐蕃更遣使者納貢,祖母可敦又遣宗俄請昏。帝以雍王守禮女為金城公主妻之,吐蕃遣尚贊咄名悉臘等逆公主。帝念主幼,賜錦繒別數萬,雜伎諸工悉從,給龜茲樂。詔左衛大將軍楊矩持節送。帝為幸始平,帳飲,引群臣及虜使者宴,酒所帝悲涕噓欷,為赦始平縣,罪死皆免,賜民繇賦一年,改縣為金城,鄉曰鳳池,里曰愴別。公主至吐蕃,自築城以居。拜矩鄯州都督。吐蕃外雖和而陰銜怒,即厚餉矩,請河西九曲為公主湯沐,矩表與其地。九曲者,水甘草良,宜畜牧,近與唐接。自是虜益張雄,易入寇。



 玄宗開元二年,其相坌達延上書宰相,請載盟文,定境於河源,丐左散騎常侍解琬涖盟。帝令姚崇等報書,命琬持神龍誓往。吐蕃亦遣尚欽藏、御史名悉臘獻載辭。未及定,坌達延將兵十萬寇臨洮,入攻蘭、渭,掠監馬。楊矩懼,自殺。有詔薛訥為隴右防禦使,與王晙等並力擊。帝怒,下詔自將討之。會晙等戰武階,斬首萬七千,獲馬羊無慮二十萬。又戰長子,豐安軍使王海賓戰死。乘之,虜大敗,眾奔突不能去,相枕藉死,洮水為不流。帝乃罷行。詔紫微舍人倪若水臨按軍實戰功,且吊祭戰亡士,敕州縣並瘞吐蕃露胔。



 宰相建言:「吐蕃本以河為境,以公主故,乃橋河築城,置獨山、九曲二軍,距積石二百里。今既負約,請毀橋,復守河如約。」詔可。遣左驍衛郎將尉遲瑰使吐蕃,慰安公主。然小小入犯邊無閑歲,於是郭知運、王君■相繼節度隴右、河西,一以捍之。吐蕃遣宗俄因子到洮水祭戰死士,且請和。然恃盛強,求與天子敵國,語悖傲。使者至臨洮,詔不內。金城公主上書求聽脩好,且言贊普君臣欲與天子共署誓刻。吐蕃又遣使者上書言:「孝和皇帝嘗賜盟,是時唐宰相豆盧欽望、魏元忠、李嶠、紀處訥等凡二十二人及吐蕃君臣同誓。孝和皇帝崩,太上皇嗣位,脩睦如舊。然唐宰相在誓刻者皆歿,今宰相不及前約,故須再盟。比使論乞力等前後七輩往,未蒙開許,且張玄表、李知古將兵侵暴甥國,故違誓而戰。今舅許湔貸前惡,歸於大和,甥既堅定,然不重盟為未信,要待新誓也。甥自總國事,不牽於下,欲使百姓久安。舅雖及和,而意不專,於言何益?」又言:「舅責乞力徐集兵,且兵以新故相代,非集也。往者疆埸自白水皆為閑壤,昨郭將軍屯兵而城之,故甥亦城。假令二國和,以迎送;有如不通,因以守境。又疑與突厥骨咄祿善者,舊與通聘,即日舅甥如初,不與交矣。因奉寶瓶、杯以獻。」帝謂昔已和親,有成言,尋前盟可矣,不許復誓。禮其使而遣,且厚賜贊普,自是歲朝貢不犯邊。



 十年,攻小勃律國,其王沒謹忙詒書北庭節度使張孝嵩曰:「勃律,唐西門。失之,則西方諸國皆墯吐蕃,都護圖之。」孝嵩聽許,遣疏勒副使張思禮以步騎四千晝夜馳,與謹忙兵夾擊吐蕃,死者數萬,多取鎧仗、馬羊,復九城故地。始勃律王來朝,父事帝。還國,置綏遠軍以捍吐蕃,故歲常戰。吐蕃每曰:「我非利若國,我假道攻四鎮爾。」及是,累歲不出兵。


於是隴右節度使王君
 \gezhu{
  大}
 請深入取償。十二年,破吐蕃,獻俘。後二年,悉諾邏兵入大斗拔谷,遂攻甘州,火鄉聚。王君
 \gezhu{
  大}
 勒兵避其銳,不戰。會大雪,吐蕃皸凍如積,乃逾積石軍趨西道以歸。君
 \gezhu{
  大}
 豫遣諜出塞,燒野草皆盡,悉諾邏頓大非川,無所牧,馬死過半。君
 \gezhu{
  大}
 率秦州都督張景順約齎窮躡,出青海西,方冰合,師乘而度。於時虜已逾大非山,留輜重疲弱濱海,君
 \gezhu{
  大}
 縱兵俘以旋。時中書令張說以吐蕃出入數十年,勝負略相當,甘、涼、河、鄯之人奉調發困甚,願聽其和。帝方寵君
 \gezhu{
  大}
 ,不聽。未幾,悉諾邏恭祿、燭龍莽布支入陷瓜州,毀其城,執刺史田元獻及君
 \gezhu{
  大}
 父,遂攻玉門軍,圍常樂,不能拔,回寇安西,副都護趙頤貞擊卻之。會君
 \gezhu{
  大}
 為回紇所殺,功不遂。帝乃用蕭嵩為河西節度使,左金吾將軍張守珪瓜州刺史,復城之。嵩縱反間,殺悉諾邏恭祿。明年,大將悉末朗攻瓜州,守珪擊走之;鄯州都督張志亮又戰青海西,破大莫門城,焚橐它橋;隴右節度使杜賓客以強弩四千射虜,破之祁連城下,斬副將一,上級五千首。虜敗,慟而走山。又明年,守珪率伊、沙等州兵破虜大同軍;又信安王禕出隴西,拔石堡城,即之置振武軍,獻俘於廟。帝以書賜將軍裴旻曰:「敢有掩戰功不及賞者,士自陳,將吏皆斬。戰有逗留,舉隊如軍法。能禽其王者,授大將軍。」於是士益奮。



 吐蕃令曩骨委書塞下,言:「論莽熱、論泣熱皆萬人將,以贊普命,謝都督刺史:二國有舅甥好,昨彌不弄羌、黨項交構二國,故失歡,此不聽,唐亦不應聽。」都督遣腹心吏與曩骨還議盟事。曩骨,猶千牛官也。於是忠王友皇甫惟明並言約和便。帝曰:「贊普向上書悖慢,朕必滅之,毋議和!」惟明曰:「昔贊普幼,是必邊將好功之人為之,以激怒陛下。且二國交惡必興師,師興則隱盜財利,詐功級,希陛下過賞以甘心焉。今河西、隴右貲耗力窮,陛下幸詔金城公主許贊普約,以紓邊患,息民之上策也。」帝採其言,敕惟明及中人張元方往聘,以書賜公主。惟明見贊普言天子意,贊普大喜,因悉出貞觀以來書詔示惟明,厚饋獻。使名悉臘隨使者入朝,奉表言:「甥,先帝舅顯親也。曩為張玄表、李知古交鬥,遂成大釁。甥以文成、金城公主,敢失禮乎?特以沖幼,枉為邊將讒亂。如蒙澄亮,死且萬足,千萬歲不敢先負盟。」且獻怪寶。使者至,帝御前殿,列羽林仗內之。悉臘略通華文,既宴與語,禮甚厚,賜紫服、金魚。悉臘受服辭魚,曰:「國無是,不敢當。」帝遣御史大夫崔琳報聘。



 吐蕃又請交馬於赤嶺,互市於甘松嶺。宰相裴光庭曰:「甘松中國阻,不如許赤嶺。」乃聽以赤嶺為界,表以大碑,刻約其上。又請《五經》,敕秘書寫賜,並遣工部尚書李〓往聘,賜物萬計。吐蕃遣使謝,且言:「唐、吐蕃皆大國,今約和為久長計,恐邊吏有妄意者,請以使人對相曉敕,令昭然具知。」帝又令金吾將軍李佺監赤嶺樹碑,詔張守珪與將軍李行禕、吐蕃使者莽布支分諭劍南、河西州縣曰:「自今二國和好,無相侵暴。」乃使悉諾渤海納貢,並以幣器遍遺執政。明年,上寶器數百具,制冶詭殊,詔置提象門示群臣。



 其後吐蕃西擊勃律,勃律告急,帝諭令罷兵,不聽,卒殘其國。於是崔希逸為河西節度使,鎮涼州,故時疆畔皆樹壁守捉,希逸謂虜戍將乞力徐曰:「兩國約好,而守備不廢,云何?請皆罷,以便人。」乞力徐曰:「公忠誠,無不可,恐朝廷未皆信,脫掩吾不備,其可悔?」希逸固邀,乃許。即共刑白犬盟,而後悉徹障壁,虜畜牧被野。



 明年,傔史孫誨奏事,妄言「虜無備,可取也」。帝採之,詔內豎趙惠琮共往按狀。小人欲徼幸,至涼州,因共矯詔,詔希逸發兵襲破吐蕃青海上,斬獲不貲,乞力徐遁走。吐蕃恚,不朝。二十六年,大入河西,希逸拒破之。鄯州都督杜希望又拔新城,更號威戎軍。希逸顧失信,悒悒悵恨,召拜河南尹。既而與惠琮俱見犬崇,疑而死,誨亦及它誅。



 蕭炅代為河西節度留後,杜希望隴右節度留後,王昱劍南節度使,分道經略,碎赤嶺碑。希望發鄯州兵奪虜河橋,並河築鹽泉城,號鎮西軍,破吐蕃兵三萬。昱以劍南兵入攻安戎城,築二少壘左右之,兵次蓬婆嶺,輸劍南粟餉軍。吐蕃悉銳來救,昱大敗,少壘皆沒,士死凡數萬。昱貪妄,非將選,故敗,貶死高要。明年,吐蕃攻白草、安人軍,詔臨洮、朔方分援。虜絕臨洮道,白水軍使高柬于拒守,虜引去。炅遣將追尾,有雲出軍上,白兔舞,大破吐蕃。昱之敗,以張宥代節度劍南,以章仇兼瓊為益州司馬。宥,文吏,不知兵,委事兼瓊。兼瓊因得入奏,天子果其議,拔兼瓊代宥節度。兼瓊諜誘吐蕃安戎城主為應,導官軍入,盡殺虜戍,以監察御史許遠守之。吐蕃圍安戎,絕水泉,會石裂泉湧,虜驚引去。復攻維州,不得志。詔乃改安戎曰平戎云。



 是歲,金城公主薨。明年,為發哀,吐蕃使者朝,因請和,不許。虜乃悉眾四十萬攻承風堡,抵河源軍,西入長寧橋、安仁軍,渾崖烽騎將臧希液以銳兵五千破之。吐蕃又襲廓州,敗一縣,屠吏人。攻振武軍石堡城,蓋嘉運不能守。



 天寶元年,隴右節度使皇甫惟明破虜大嶺軍;戰青海,破莽布支,斬首三萬級。明年,破洪濟城,戰石堡,不克,副將諸葛誗死之。又明年,惟明破虜,獻俘京師。帝以哥舒翰節度隴右,翰攻拔石堡,更號神武軍。又禽其相兀論樣郭。



 十載,安西節度使高仙芝俘大酋以獻。是時,吐蕃與蠻閣羅鳳聯兵攻瀘南,劍南節度使楊國忠方以奸罔上,自言:「破蠻眾六萬於雲南,拔故洪州等三城,獻俘口。」哥舒翰破洪濟、大莫門諸城,收九曲故地,列郡縣,實天寶十二載。於是置神策軍於臨洮西、澆河郡於積石西、及宛秀軍以實河曲。後二年,蘇毘子悉諾邏來降,封懷義王,賜李氏。蘇毘,強部也。是歲,贊普乞黎蘇籠臘贊死,子挲悉籠臘贊嗣,遣使者脩好,詔京兆少尹崔光遠持節齎冊吊祠。還而安祿山亂,哥舒翰悉河、隴兵東守潼關,而諸將各以所鎮兵討難,始號行營,邊候空虛,故吐蕃得乘隙暴掠。



 至德初,取巂州及威武等諸城,入屯石堡。其明年,使使來請討賊且脩好。肅宗遣給事中南巨川報聘。然歲內侵,取廓、霸、岷等州及河源、莫門軍。使數來請和,帝雖審其譎,姑務紓患,乃詔宰相郭子儀、蕭華、裴遵慶等與盟。



 寶應元年,陷臨洮,取秦、成、渭等州。明年,使散騎常侍李之芳、太子左庶子崔倫往聘,吐蕃留不遣。破西山合水城。明年,入大震關,取蘭、河、鄯、洮等州,於是隴右地盡亡。進圍涇州,入之,降刺史高暉。又破邠州,入奉天,副元帥郭子儀御之。吐蕃以吐谷渾、黨項兵二十萬東略武功,渭北行營將呂日將戰盩厔西,破之。又戰終南,日將走。代宗幸陜,子儀退趨商州。高暉導虜入長安,立廣武王承宏為帝,改元,擅作赦令,署官吏。衣冠皆南奔荊、襄,或逋棲山谷,亂兵因相攘鈔,道路梗閉。光祿卿殷仲卿率千人壁藍田,選二百騎度滻,或紿虜曰:「郭令公軍且來!」吐蕃大震。會少將王甫與惡少年伐鼓噪苑中,虜驚,夜引去。子儀入長安,高暉東奔至潼關,守將李日越殺之。吐蕃留京師十五日乃走,天子還京。



 吐蕃退圍鳳翔,節度使孫志直拒守,鎮西節度使馬璘以千騎戰卻之,吐蕃屯原、會、成、渭間,自如也。是歲,南入松、維、保等州及雲山新籠城。明年,還使人李之芳等。劍南嚴武破吐蕃南鄙兵七萬,拔當狗城。會僕固懷恩反,自靈武遣其將範志誠、任敷合吐蕃、吐谷渾兵攻邠州,白孝德、郭晞嬰壘守,乃入居奉天西。子儀入奉天,按軍不戰。郭晞以銳士夜搗其營,斬首數千級,奪馬五百,取四將,吐蕃引去。是時嚴武拔鹽川,又戰西山,取其眾八萬。虜圍涼州,河西節度使楊志烈不能守,跳保甘州,而涼州亡。



 永泰元年,吐蕃請和,詔宰相元載、杜鴻漸與虜使者同盟。懷恩不得志,導虜與回紇、黨項羌、渾、奴刺犯邊,吐蕃大酋尚結息、贊摩、尚悉東贊等眾二十萬至醴泉、奉天,邠將白孝德不能亢,任敷以兵略鳳翔、盩厔,於是京師戒嚴。朔方兵馬使渾日時、孫守亮屯奉天,詔子儀以河中兵屯涇陽,李忠臣屯東渭橋,李光進屯雲陽,馬璘、郝廷玉屯便橋,駱奉先、李日越屯盩厔,李抱玉屯鳳翔,周智光屯同州,杜冕屯坊州,天子自率六軍屯於苑。吐蕃逼奉天,日進以單騎馳之,士二百踵進,左右擊刺,射皆應弦僕,虜大驚闢易。日進挾虜一將躍出,舉軍望而噪,士還,無一矢著身者。明日,虜薄城,日進發機石勁弩,故兵多死。凡三日,虜斂軍入壁,日進知虜曲折,即夜斫其營,斬千餘級,生禽五百。又戰馬嵬,凡七日,破賊萬人,斬首五千,獲馬、橐它、幟械甚眾。帝欲自討賊,下詔大搜馬,京師始置團練,都人震擾,鑿垣亡去者十八,詔中人戶都門,不能止。吐蕃游騎四百略武功,鎮西節度使馬璘使健士五十擊之,殲,士氣益奮。虜徙營九〓之陰,掠醴泉居人數萬,焚室廬,田皆赤地。周智光與虜戰澄城,破之。吐蕃至邠北,復與回紇合,還攻奉天,抵馬嵬。任敷以兵五千掠白水,殘同州。於是城中渭橋、鄠以屯兵。



 會懷恩死,虜謀無主,遂與回紇爭長。回紇怒,詣子儀請擊吐蕃自效,子儀許之,使白元光合兵攻吐蕃於靈臺西,大破之,降僕固名臣,帝乃班師。



\end{pinyinscope}