\article{志第十八下 歷四下}

\begin{pinyinscope}

 ○六曰步交會術



 終數八億二千七百二十五萬一千三百二十二。



 交終日二十七,餘六百四十五,秒千三百二十二。



 中日十三,餘千八百四十二,秒五千六百六十一。



 朔差日二,餘九百六十七,秒八千六百七十八。



 望差日一,餘四百八十三,秒九千三百三十九。



 望數日十四,餘二千三百二十六,秒五千。



 交限日十二,餘千三百五十八,秒六千三百二十二。



 交率三百四十三。



 交數四千三百六十九。



 交秒法一萬。



 以交數去朔積分;不盡,以秒法乘之,盈交數又去之;餘如秒法而一,為入交分。滿通法為日,命日算外,得天正經朔時加入交泛日及餘。因加朔差,得次朔。以望數加朔,得望。若以經朔望小餘減之,各得夜半所入。累加一日,得次日。加之滿交終,去之。各以其日入氣朓朒定數,朓減、朒加入交泛,為入交常日及餘。又以交率乘其日入轉朓朒定數,如交數而一,而朓減、朒加入交常,為入交定日及餘。各如中日已下者,為月入陽歷;已上者,去之,餘為月入陰歷。



 ○陰陽歷



 以其爻加減率與後爻加減率相減,為前差。又以後爻率與次後爻率相減,為後差。二差相減,為中差。置所在爻並後爻加減率,半中差以加而半之,十五而一,為爻末率,因為後爻初率。每以本爻初、末率相減,為爻差。十五而一,為度差。半之,以加減初率,少象減之,老象加之。為定初率。每以度差累加減之,少象以差減,老象以差加。各得每歲加減定分。乃循積其分,滿百二十為度,各為月去黃道數及分。其四象初爻無初率,上爻無末率,皆倍本爻加減率,十五而一。所得各以初、末率減之,皆互得其率。



 各置夜半入轉,以夜半入交定日及餘減之,不足減,加轉終。餘為定交初日夜半入轉。乃以定交初日與其日夜半入餘,各乘其日轉定分,如通法而一,為分。滿轉法,為度。各以加其日轉積度分,乃相減,所餘為其日夜半月行入陰陽度數。轉求次日,以轉定分加之。以一象之度九十除之,若以少象除之,則兼除差度一、度分百六、大分十三、小分十四。訖,然後以次象除之。所得以少陽、老陽、少陰、老陰為次,起少陽算外,得所入象度數及分。先以三十乘陰陽度分,十九而一,為度分。不盡,以十五乘、十九除,為大分。不盡者,又乘、又除,為小分。然後以象度及分除之。乃以一爻之度十五除之,所得入爻度數及分。其月行入少象初爻之內及老象上爻之中,皆沾黃道。當朔望,則有虧蝕。



 凡入交定如望差已下,交限已上,為入蝕限;望入蝕限,則月蝕。朔入蝕限,月在陰歷,則日蝕。如望差已下,為交後。交限已上,以減交中,餘為交前。置交前、後定日及餘,通之,為去交前、後定分。十一乘之,二千六百四十三除,為去交度數。不盡,以通法乘之,復除為餘。大抵去交十三度已上,雖入蝕限,為涉交數微,光景相接,或不見蝕。望去交分七百七十九已下者,皆既。已上者,以定交分減望差,餘以百八十三約之,命以十五為限,得月蝕之大分。



 月在陰歷,初起東南,甚於正南,復於西南;月在陽歷,初起東北,甚於正北,復於西北。其蝕十二分已上者,起於正東,復於正西。此據午正而論之。餘各隨方面所在,準此取正。



 凡月蝕之大分五已下,因增三。十已下,因增四。十已上,因增五。其去交定分五百二十已下,又增半。二百六十已下,又增半。各為泛用刻率。



 以所入氣並後氣增損差,倍六爻乘之,綜兩氣辰數除之,為氣末率。又列二氣增損差,皆倍六爻乘之,各如辰數而一;少減多,餘為氣差。加減末率冬至後以差減,夏至後以差加。為初率。倍氣差,綜兩氣辰數除,為日差。半之,加減初、末,為定率。以差累加、減氣初定率,冬至後以差加,夏至後以差減。為每日增損差。乃循積之,隨所入氣日增損氣下差積,各其日定數。其二至之前一氣,皆後無同差,不可相並,各因前末為初率。以氣差冬至前減、夏至前加,為末率。



 陰歷蝕差千二百七十五,蝕限三千五百二十四,或限三千六百五十九。陽歷蝕限百三十五,或限九百七十四。以蝕朔所入氣日下差積,陰歷減之,陽歷加之,各為蝕定差及定限。朔在陰歷,去交定分滿蝕定差已上者,為陰歷蝕。不滿者,雖在陰歷,皆類同陽歷蝕。其去交定分滿定限已下者,的蝕。或限已下者,或蝕。



 陰歷蝕者,置去交定分,以蝕定差減之,餘百四已下者,皆蝕既。已上者,以百四減之。餘以百四十三約之。其入或限者,以百五十二約之。半已下,為半弱。半已上,為半強。以減十五,餘為日蝕之大分。其同陽歷蝕者,其去交定分少於蝕定差六十已下者,皆蝕既。已上者,以陽歷蝕定限加去交分,以九十約之。其陽歷蝕者,置去交定分,亦以九十約之。入或限者,以百四十三約之。皆半已下,為半弱。半已上,為半強。命之,以十五為限,得日蝕之大分。



 月在陰歷,初起西北,甚於正北,復於東北。月在陽歷,初起西南,甚於正南,復於東南。其蝕十二分已上,皆起於正西,復於正東。



 凡日蝕之大分,皆因增二。其陰歷去交定分多於蝕定差七十已上者,又增;三十五已下者,又增半。其同陽歷去交定分少於蝕定差二十已下者,又增半;四已下者,又增少。各為泛用刻率。



 置去交定分,以交率乘之,二十乘交數除之;其月道與黃道同名者,以加朔望定小餘:異名者,以減朔、望定小餘:為蝕定餘。如求發斂加時術入之,得蝕甚辰刻。各置泛用刻率,副之。以乘其日入轉損益率,如通法而一。所得應朒者,依其損益;應朓者,損加、益減其副:為定用刻數。半之,以減蝕甚辰刻,為虧初;以加蝕甚辰刻,為復末。其月蝕,置定用刻數,以其日每更差刻除,為更數。不盡,以每籌差刻除,為籌數。綜之為定用更籌。乃累計日入後至蝕甚辰刻,置之,以昏刻加日入辰刻減之,餘以更籌差刻除之。所得命以初更籌算外,得蝕甚更籌。半定用更籌減之,為虧初;加之,為復末。按天竺俱摩羅所傳斷日蝕法,日躔鬱車宮者,的蝕。其餘據日所在宮,火星在前三及後五之宮,並伏在日下,則不蝕。若五星皆見,又水在陰歷及三星已上同聚一宿,則亦不蝕。凡星與日別宮或別宿則易斷,若同宿則難。天竺所云十二宮,即中國之十二次。鬱車宮者,降婁之次也。



 九服之地,蝕差不同。先測其地二至及定春秋分中晷長短,與陽城每日中晷常數較取同者,各因其日蝕差為其地二至及定春秋分蝕差。以夏至差減春分差,以春分差減冬至,各為率。並二率,半之,六而一,為夏率。二率相減,六而一,為總差。置總差,六而一,為氣差。半氣差,以加夏率,又以總差減之,為冬率。冬率即冬至率。每以氣差加之,各為每氣定率。乃循積其率,以減冬至蝕差,各得每氣初日蝕差。求每日,如陽城法求之。若戴日之南,當計所在地,皆反用之。



 ○七曰步五星術



 △歲星



 終率百二十一萬二千五百七十九,秒六。



 終日三百九十八,餘二千六百五十九,秒六。



 變差三十四,秒十四。



 象算九十一,餘二百三十八,秒五十七,微分十二。



 爻算十五,餘百六十六,秒四十二,微分八十二。



 △熒惹



 終率二百三十七萬一千三,秒八十六。



 終日七百七十九,餘二千八百四十三,秒八十六。



 變差三十二,秒二。



 象算九十一,餘二百三十八,秒四十三,微分八十四。



 爻算十五,餘百六十六,秒四十,微分六十二。



 △鎮星



 終率百一十四萬九千三百九十九,秒九十八。



 終日三百七十八,餘二百七十九,秒九十八。



 變差二十二,秒九十二。



 象算九十一,餘二百三十七,秒八十七。



 爻算十五,餘百六十六,秒三十一,微分十六。



 △太白



 終率百七十七萬五千三十,秒十二。



 終日五百八十三,餘二千七百一十一,秒十二。



 中合日二百九十一,餘二千八百七十五,秒六。



 變差三十,秒五十三。



 象算九十一,餘二百三十八,秒三十四,微分五十四。



 爻算十五,餘百六十六,秒三十九,微分九。



 △辰星



 終率三十五萬二千二百七十九,秒七十二。



 終日百一十五,餘二千六百七十九,秒七十二。



 中合日五十七,餘二千八百五十九,秒八十六。



 變差百三十六,秒七十八。



 象算九十一,餘二百四十四,秒九十八,微分六十。



 爻算十五,餘百六十七,秒四十九,微分七十四。



 辰法七百六十。



 秒法一百。



 微分法九十六。



 置中積分,以冬至小餘減之,各以其星終率去之,不盡者,返以減終率;餘滿通法為日,得冬至夜半後平合日算。各以其星變差乘積算,滿乾實去之;餘滿通法,為日。以減平合日算,得入歷算數。皆四約其餘,同於辰法。及以一象之算除之,以少陽、老陽、少陰、老陰為次,起少陽算外。餘以一爻之算除之;所得命起其象初爻算外,得外入爻算數。



 ○五星爻象歷



 以所入爻與後爻損益率相減,為前差;又以後爻與次後爻損益率相減,為後差;二差相減,為中差。置所入爻並後爻損益率,半中差以加之,九之,二百七十四而一,為爻末率,因為後爻初率。皆因前爻末率,以為後爻初率。初、末之率相減,為爻差。倍爻差,九之,二百七十四而一,為算差。半之,加減初、末,各為定率。以算差累加、減爻初定率,少象以差減,老象以差加。為每算損益率。循累其率,隨所入爻損益其下進退積,各得其算定數。其四象初爻無初率,上爻無末率,皆置本爻損益率四而九之,二百七十四得一,各以初、末率減之,皆互得其率。



 各置其星平合所入爻之算差,半之,以減其入算損益率。損者,以所入餘乘差,辰法除,並差而半之;益者,半入餘,乘差,亦辰法除:皆中所減之率。乃以入餘乘之,辰法而一。所得以損益其算下進退,各為平合所入定數。



 置進退定數,金星則倍置之。各以合下乘數乘之,除數除之。所得滿辰法為日,以進加、退減平合日算,先以四約平合餘,然後加減。為常合日算。



 置常合日先後定數,四而一,以先減、後加常合日算,得定合日算。又四約盈縮分,以定合餘乘之,滿辰法而一。所得以盈加、縮減其定餘,加其日夜半日度,為定合加時星度。



 又置定合日算,以冬至大小餘加之,天正經朔大小餘減之。其至朔小餘,皆先以四約之。若大餘不足減,又以爻數加之,乃減之。餘滿四象之策除,為月數。不盡者,為入朔日算。命月起天正、日起經朔算外,得定合月、日。視定朔與經朔有進退者,亦進減、退加一日為定。



 置常合及定合應加減定數,同名相從,異名相消;乃以加減其平合入爻算,滿若不足,進退爻算,得定合所入。乃以合後諸變歷度累加之,去命如前,得次變初日所入。如平合求進退定數,乃以乘數乘之,除數除之,各為進退變率。



 五星變行日中率、度中率、差行損益率、歷度乘數、除數



 ○歲星



 合後伏:十七日三百三十二分,行三度三百三十二分。先遲,二日益疾九分。歷,一度三百五十七分。乘數三百五十,除數二百八十一。



 前順:百一十二日,行十八度六百五十六分。先疾,五日益遲六分。歷,九度三百三十七分。乘數三百五十,除數二百八十一。



 前留:二十七日。歷,二度二百二十分。乘數二百六十七,除數二百二十一。



 前退:四十三日,退五度三百六十九分。先遲,六日益疾十一分。歷,三度四百七十五分。乘數四百七十,除數四百三。



 後退:四十三日,退五度三百六十九分。先遲,六日益遲十一分。歷,三度四百七十五分。乘數五百一十,除數四百六十七。



 後留:二十七日。歷,三度二百一十分。乘數二百七十,除數二百二十二。



 後順:百一十二日,行十八度六十五分。先遲,五日益疾六分。歷,九度三百三十七分。乘數二百六十七,除數二百二十七。



 合前伏:十七日三百三十二分,行三度三百三十二分。先疾,二日益遲九分。歷,一度三百五十八分。乘數三百五十,除數二百八十一。



 ○熒惑



 合後伏:七十一日七百三十五分,行五十四度七百三十五分。先疾,五日益遲七分。歷,三十八度二百一分。乘數百二十七,除數三十。



 前疾:二百一十四日,行百三十六度。先疾,九日益遲四分。歷,百一十三度五百九十六分。乘數百二十七,除數三十。



 前遲:六十日,行二十五度。先疾,日益遲四分。歷,三十一度六百八十五分。乘數二百三,除數五十四。



 前留:十三日,歷,六度六百九十三分。乘數二百三,除數五十四。



 前退:三十一日,退八度四百七十三分。先遲,六日益疾五分。歷,十六度三百六十七分。乘數二百三,除數四十八。



 後退:三十一日,退八度四百七十三分。先疾,六日益遲五分。歷,十六度三百六十七分。乘數二百三,除數四十八。



 後留:十三日。歷,六度六百九十三分。乘數二百三,除數四十八。



 後遲:六十日,行二十五度。先遲,日益疾四分。歷,三十一度六百八十五分。乘數二百三,除數五十四。



 後疾:二百一十四日,行百三十六度。先遲,九日益疾四分。歷,百一十三度五百九十六分。乘數二百三,除數五十四。



 合前伏:七十一日七百三十六分,行五十四度七百三十六分。先遲,五日益疾七分。歷,三十八度二百一分。乘數百二十七,除數三十。



 ○鎮星



 合後伏:十八日四百一十五分,行一度四百一十五分。先遲,二日益疾九分。歷,四百八十分。乘數十二,除數十一。



 前順:八十三日,行七度二百四十一分。先疾,六日益遲五分。歷,二度六百二十三分。乘數十二,除數十一。



 前留:三十七日三百八十分。歷,一度二百八分。乘數十,除數九。



 前退:五十日,退二度三百三十四分。先遲,七日益疾一分。歷,一度五百三十一分。乘數二十,除數十七。



 後退:五十日,退二度三百三十四分,先疾,七日益遲一分。歷,一度五百三十一分。乘數五,除數四。



 後留:三十七日三百八十分。歷,一度二百八分。乘數二十,除數一十七。



 後順:八十三日,行七度二百四十一分。先遲,六日益疾五分。歷,二度六百二十三分。乘數十,除數九。



 合前伏:十八日四百一十五分,行一度四百一十五分。先疾,二日益遲九分。歷,四百八十分。乘數十二,除數十一。



 ○太白



 晨合後伏:四十一日七百一十九分,行五十二度七百一十九分。先遲,三日益疾十六分。歷,四十一度七百一十九分。乘數七百九十七,除數二百九。



 夕疾行:百七十一日,行二百六度。先疾,五日益遲九分。歷,百七十一度乘數七百九十七,除數二百九。



 夕平行:十二日,行十二度。歷,十二度。乘數五百一十五,除數百五十六。



 夕遲行:四十二日,行三十一度,先疾,日益遲十分。歷,四十二度。乘數五百一十五,除數百三十七。



 夕留:八日。歷,八度。乘數五百一十五,除數九十二。



 夕退:十日,退五度。先遲,日益疾九分。歷,十度。乘數五百一十五,除數八十六。



 夕合前伏:六日,退五度。先疾,日益遲十五分。歷,六度。乘數五百一十五,除數八十四。



 夕合後伏:六日,退五度。先遲,日益疾十五分。歷,六度。乘數五百一十五,除數八十三。



 晨退:十日,退五度。先疾,日益遲九分。歷,十度。乘數五百一十五,除數八十四。



 晨留:八日,歷八度。乘數五百一十五,除數八十六。



 晨遲行:四十二日,行三十一度。先遲,日益疾十分。歷,四十二度。乘數五百一十五,除數九十二。



 晨平行:十二日,行十二度。歷,十二度。乘數五百一十五,除數百三十七。



 晨疾行:百七十一日,行二百六度。先遲,五日益疾九分。歷,百七十一度。乘數五百一十五,除數百五十六。



 晨合前伏:四十一日七百一十九分,行五十二度七百一十九分。先疾,三日益遲十六分。歷,四十一度七百一十九分。乘數七百九十七,除數二百九。



 ○辰星



 晨合後伏:十六日七百一十五分,行三十三度七百一十五分。先遲,日益疾二十二分。歷,十六度七百一十五分。乘數二百八十六,除數二百八十七。



 夕疾行:十二日,行十七度。先疾,日益遲五十分。歷,十二度。乘數二百八十六,除數二百八十七。



 夕平行:九日,行九度。歷,九度。乘數四百九十五,除數百九十四。



 夕遲行:六日,行四度。先疾,日益遲七十六分。歷,六度。乘數四百九十六,除數百九十五。



 夕留:三日。歷,三度。乘數四百九十七,除數百九十六。



 夕合前伏:十一日,退六度。先遲,日益疾三十一分。歷,十一度。乘數四百九十八,除數百九十七。



 夕合後伏:十一日,退六度。先疾,日益遲三十一分。歷,十一度。乘數五百,除數百九十八。



 晨留:三日。歷,三度。乘數四百九十八,除數百九十八。



 晨遲行:六日,行四度。先遲,日益疾七十六分。歷,六度。乘數四百九十七,除數百九十六。



 晨平行:九日,行九度。歷,九度。乘數四百九十六,除數百九十五。



 晨疾行,十二日,行十七度。先遲,日益疾五十分。歷,十二度。乘數四百九十二,除數百九十四。



 晨合前伏:十六日七百一十五分,行三十三度七百一十五分。先疾,日益遲二十二分。歷,十六度七百一十五分。乘數二百八十六,除數二百八十七。



 各置其本進退變率與後變率。同名者,相消為差。在進前少,在退前多,各以差為加;在進前多,在退前少,各以差為減。異名者,相從為並。前退後進,各以並為加;前進後退,各以並為減。逆行度率則反之。皆以差及並,加、減日度中率,各為日度變率。其水星疾行,直以差、並加、減度中率,為變率。其日直因中率為變率,勿加、減也。



 以定合日與前疾初日、後疾初日與合前伏初日先後定數,各以同名者相消為差,異名者相從為並。皆四而一。所得滿辰法,各為日度。乃以前日度盈加、縮減其合後伏度之變率及合前伏、前疾日之變率,亦以後日度盈減、縮加其後疾日之變率及合前伏、前疾度之變率。金水夕合,反其加減。留退亦然。其二留日之變率,若差於中率者,即以所差之數為度,各加、減本遲度之變率。謂以所多於中率之數加之,少於中率之數減之。已下加、減準此。退行度之變率,若差於中率者,即倍所差之數,各加、減本疾度之變率。其土、木二星,既無遲、疾,即加、減前、後順行度之變率。其水星疾行度之變率,若差於中率者,即以所差之數為日,各加、減留日變率。其留日變率若少不足減者,即侵減遲日變率;若多於中率者,亦以所多之數為日,以加留日變率。各加、減變率訖,皆為日度定率。其日定率有分者,前後輩之。輩,配也,以少分配多分,滿全為日。有餘轉配其諸變率。不加減者,皆依變率為定率。



 置其星定合餘,以減辰法;餘以其星初日行分乘之,辰法而一,以加定合加時度,得定合後夜半星度及餘。自此各依其星計日行度,所至皆從夜半為始。各以一日所行度分順加、退減之。其行有小分者,各滿其法從行分。伏不注度,留者因前,退則依減。順行出虛,去六虛之差。退行入虛,先加此差。六虛之差,亦四而一,乃用加減。訖,皆以轉法約行分,為度分,得每日所至。日度定率,或加或減,益疾益遲,每日漸差,不可預定。今且略據日度中率,商量置之。其定率既有盈縮,即差數合隨而增損,當先檢括諸變定率與中率相較近者因用其差,求其初、末之日行分為主。自餘諸變,因此消息,加、減其差,各求初、末行分。循環比較,使際會參合,衰殺相循。其金、水皆以平行為主,前後諸變,準此求之。其合前伏,雖有日度定率,因加至合而與後算不葉者,皆從後算為定。其初見伏之度,去日不等,各以日度與星辰相較。木去日十四度,金十一度,火、土、水各十七度皆見。各減一度,皆伏。其木、火、土三星,前順之初,後順之末,及金、水疾行、留、退初、末,皆是見、伏之初日,注歷消息定之。金、水及日、月度,皆不注分。



 置日定率減一,以所差分乘之,為實。以所差日乘定日率,為法。實如法而一,為行分,得每日差。以辰法通度定率,從其分,如日定率而一,為平行度分。減日定率一,以所差分乘之,二而一,為差率。以加、減平行分,益疾者,以差率減平行為初日,加平行為末日;益遲者,以差率加平行為初日,減平行為末日。得初、末日所行度及分。其差不全而與日相合者,先置日定率減一,以所差分乘之,為實。倍所差日,為法。實如法而一,為行分。不盡者,因為小分。然後為差率。



 置初日行分,益遲者,以每日差累減之;益疾者,以每日差累加之:得次日所行度分。其每日差及初日行,皆有小分。母既不同,當令同之,乃用加、減。



 其先定日數而求度者,減所求日一,以每日差乘之,二而一。所得以加、減初日行分,益遲減之,益疾加之。以所求日乘之,如辰法而一,為度。不盡者,為行分,得從初日至所求日積度及分。



 若先定度數而返求日者,以辰法乘所求行度。有分者,從之。八之,如每日差而一,為積。倍初日行分,以每日差加、減之,益遲者加之,益疾者減之。如每日差而一,為率。令自乘,以積加、減之。益遲者以積減之,益疾者以積加之。開方除之,所得以率加、減之。益遲者以率加之,益疾者以率減之。乃半之,得所求日數。開方除者,置所開之數為實。借一算於實之下,名曰下法。步之,超一位。置商於上方,副商於下法之上,名曰方法。命上商以除實。畢,倍方法一折,下法再折。乃置後商於下法之上,名曰隅法。副隅並方。命後商以除實。畢,隅從方法折下,就除如前開之。



 五星前變,入陽爻,為黃道北;入陰爻,為黃道南。後變,入陽爻,為黃道南;入陰爻,為黃道北。其金、水二星,以夕為前變,晨為後變。各計其變行,起初日入爻之算,盡老象上爻未算之數。不滿變行度常率者,因置其數以變行日定率乘之,如變行度常率而一,為日。其入變日數與此日數已下者,星在道南北依本所入陰陽爻為定。過此日數之外者,南北返之。



 《九執歷》者,出於西域。開元六年,詔太史監瞿壇悉達譯之。斷取近距,以開元二年二月朔為歷首。度法六十。月有二十九日,餘七百三分日之三百七十三。歷首有朔虛分百二十六。周天三百六十度,無餘分。日去沒分九百分度之十三。二月為時,六時為歲。三十度為相,十二相而周天。望前曰白博義;望後曰黑博義。其算皆以字書,不用籌策。其術繁碎,或幸而中,不可以為法。名數詭異,初莫之辯也。陳玄景等持以惑當時,謂一行寫其術未盡,妄矣。



\end{pinyinscope}