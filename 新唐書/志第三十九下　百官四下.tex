\article{志第三十九下 百官四下}

\begin{pinyinscope}

 ○王府官



 傅一人,從三品。掌輔正過失。諮議參軍事一人,正五品上。掌訏謀議事。友一人,從五品下。掌侍游處,規諷道義。侍讀,無定員。文學一人,從六品上。掌校典籍,侍從文章。東、西閤祭酒各一人,從七品上。掌禮賢良、導賓客。自祭酒以下為王官。武德中,置師一人、常侍二人、侍郎四人,皆掌表啟書疏,贊相禮儀;舍人四人,掌通傳引納;謁者二人、舍人二人、諮議參軍事、友,皆正五品下;文學、祭酒,皆正六品下。高宗、中宗時,相王府長史以宰相兼之,魏、雍、衛王府以尚書兼之,徐、韓二王為刺史,府官同外官,資望愈下。永淳以前,王未出閤則不開府。天授二年,置皇孫府官。玄宗諸子多不出閤,王官益輕而員亦減矣。景雲二年,改師曰傅,開元二年廢,尋復置,廢常侍、侍郎、謁者、舍人。開成元年,改諸王侍讀曰奉諸王講讀,大中初復舊。



 長史一人,從四品上;司馬一人,從四品下。皆掌統府僚、紀綱職務。掾一人,掌通判功曹、倉曹、戶曹事,屬一人,皆正六品上,掌通判兵曹、騎曹、法曹、士曹事。主簿一人,掌覆省書教,記室參軍事二人,掌表啟書疏,錄事參軍事一人,皆從六品上,掌付事、句稽,省署鈔目。錄事一人,從九品下。功曹參軍事掌文官簿書、考課、陳設,倉曹參軍事掌祿稟、廚膳、出內、市易、畋漁、芻槁,戶曹參軍事掌封戶、僮僕、弋獵、過所,兵曹參軍事掌武官簿書、考課、儀衛、假使,騎曹參軍事掌廄牧、騎乘、文物、器械,法曹參軍事掌按訊、決刑,士曹參軍事掌土功、公廨,自功曹以下各一人,正七品上。參軍事二人,正八品下;行參軍事四人,從八品上。皆掌出使雜檢校。典簽二人,從八品下,掌宣傳書教。武德中,改功曹以下書佐、法曹行書佐、土曹佐皆曰參軍事,長兼行書佐曰行參軍,廢城局參軍事。又有鎧曹參軍事二人,掌儀衛兵仗;田曹參軍事一人,掌公廨、職田、弋獵;水曹參軍事二人,掌舟船、漁捕、芻草。皆正七品下。家吏二人,百司問事謁者一人,正七品下。司閤一人,正九品下。貞觀中,廢鎧曹、田曹、水曹。武后時,家吏以下皆廢。主簿、記室有史二人;錄事、功曹、倉曹、兵曹、騎曹、法曹、士曹,各府一人、史二人;戶曹府、史各二人。自典簽以上為府官,郡王、嗣王不置長史。



 ○親事府



 典軍二人,正五品上;副典軍二人,從五品上。皆掌校尉以下守衛、陪從,兼知鞍馬。校尉五人,從六品上;旅帥,從七品下;隊正,從八品下,隊副,從九品下。皆掌領親事、帳內陪從。自旅帥以下,視親事多少乃置。



 ○帳內府



 典軍二人,正五品上;副典軍二人,從五品上。自校尉以下,員、品如親事府。初,典軍以武官及流外為之,領執仗、帳內等。秦王、齊王府置左右六護軍府、左右親軍府、左右帳內府。左一、右一護軍府,護軍各一人,副護軍各二人,長史、錄事參軍事,倉曹、兵曹、鎧曹參軍事,各一人,統軍各五人,別將各一人。左二、右二護軍府,左三、右三護軍府,減統軍三人,別將六人。左右親軍府,統軍各一人,長史各一人,錄事參軍事,兵曹、鎧曹參軍事,左別將,右別將,各一人。帳內府職員與護軍府同。又有庫直,隸親事府;驅咥直,隸帳內府。選材勇為之。貞觀中,庫直以下皆廢。親事府有府一人,史二人;執仗親事十六人,執弓仗;執乘親事十六人,掌供騎乘;親事三百三十人。帳內府有府一人,史一人,帳內六百六十七人。



 ○親王國



 令一人,從七品下;大農一人,從八品下。掌判國司。尉一人,正九品下;丞一人,從九品下。學官長、丞各一人,掌教授內人;食官長、丞各一人,掌營膳食;廄牧長、丞各二人,掌畜牧;典府長、丞各二人,掌府內雜事。長皆正九品下,丞皆從九品下。有典衛八人,掌守衛、陪從。舍人四人,錄事一人,府四人,史八人。



 ○公主邑司



 令一人,從七品下;丞一人,從八品下。掌公主財貨、稟積、田園。主簿一人,正九品下;錄事一人,從九品下。督封租、主家財貨出入。有史八人,謁者二人,舍人二人,家史二人。



 ○外官



 天下兵馬元帥、副元帥,都統、副都統,行軍長史,行軍司馬、行軍左司馬、行軍右司馬,判官,掌書記,行軍參謀,前軍兵馬使、中軍兵馬使、後軍兵馬使,中軍都虞候,各一人。元帥、都統、招討使,掌征伐,兵罷則省。都統總諸道兵馬,不賜旌節。高祖起兵,置左右領軍大都督,各總三軍。及定京師,置左右元帥、太原道行軍元帥、西討元帥,皆親王領之。天寶末,置天下兵馬元帥,都統朔方、河東、河北、平盧節度使。招討、都統之名始於此。大歷八年,罷天下兵馬元帥。建中四年,以李希烈反,置諸軍行營兵馬都元帥;興元元年,置副都統。會昌中,置靈、夏六道元帥。黃巢之難,置諸道行營都都統。天復三年,置諸道兵馬元帥,尋復改曰天下兵馬元帥。



 行軍司馬,掌弼戎政。居則習搜狩,有役則申戰守之法,器械、糧備、軍籍、賜予皆專焉。武德元年,改贊治曰治中。太宗即位,曰司馬,下州亦置焉。顯慶二年,置洛州司馬。武后大足元年,東都、北都,雍、荊、揚、益州,置左右司馬。神龍二年省。太極元年,雍、洛四大都督府增司馬一人,亦分左右。



 掌書記,掌朝覲、聘問、慰薦、祭祀、祈祝之文與號令升絀之事。行軍參謀,關豫軍中機密。景龍元年,置掌書記。開元十二年,罷行軍參謀,尋復置。



 節度使、副大使知節度事、行軍司馬、副使、判官、支使、掌書記、推官、巡官、衙推各一人,同節度副使十人,館驛巡官四人,府院法直官、要籍、逐要親事各一人,隨軍四人。節度使封郡王,則有奏記一人;兼觀察使,又有判官、支使、推官、巡官、衙推各一人;又兼安撫使,則有副使、判官各一人;兼支度、營田、招討、經略使,則有副使、判官各一人;支度使復有遣運判官、巡官各一人。



 節度使掌總軍旅,顓誅殺。初授,具帑抹兵仗詣兵部辭見,觀察使亦如之。辭日,賜雙旌雙節。行則建節、樹六纛,中官祖送,次一驛輒上聞。入境,州縣築節樓,迎以鼓角,衙仗居前,旌幢居中,大將鳴珂,金鉦鼓角居後,州縣齎印迎於道左。視事之日,設禮案,高尺有二寸,方八尺。判三案:節度使判宰相,觀察使判節度使,團練使判觀察使。三日洗印,視其刓缺。歲以八月考其治否;銷兵為上考,足食為中考,邊功為下考;觀察使以豐稔為上考,省刑為中考,辦稅為下考;團練使以安民為上考,懲奸為中考,得情為下考;防禦使以無虞為上考,清苦為中考,政成為下考;經略使以計度為上考,集事為中考,脩造為下考。罷秩則交廳,以節度使印自隨,留觀察使、營田等印,以郎官主之。鎖節樓、節堂,以節院使主之,祭奠以時。入朝未見,不入私第。京兆、河南牧,大都督,大都護,皆親王遙領。兩府之政,以尹主之;大都督府之政,以長史主之;大都護府之政,以副大都護主之,副大都護則兼王府長史。其後有持節為節度、副大使知節度事者,正節度也。諸王拜節度大使者,皆留京師。



 觀察使、副使、支使、判官、掌書記、推官、巡官、衙推、隨軍、要籍、進奏官,各一人。



 團練使、副使、判官、推官、巡官、衙推,各一人。



 防禦使、副使、判官、推官、巡官,各一人。



 觀察處置使,掌察所部善惡,舉大綱。凡奏請,皆屬於州。貞觀初,遣大使十三人巡省天下諸州,水旱則遣使,有巡察、安撫、存撫之名。神龍二年,以五品以上二十人為十道巡察使,按舉州縣,再周而代。景雲二年,置都督二十四人,察刺史以下善惡,置司舉從事二人,秩比侍御史。揚、益、並、荊四州為大都督,汴、兗、魏、冀、蒲、綿、秦、洪、潤、越十州為中都督,皆正三品;齊、鄜、涇、襄、安、潭、遂、通、梁、夔十州為下都督,從三品。當時以為權重難制,罷之,唯四大都督府如故。置十道按察使,道各一人。開元二年,曰十道按察採訪處置使,至四年罷,八年復置十道按察使,秋、冬巡視州縣,十年又罷。十七年復置十道、京都、兩畿按察使,二十年曰採訪處置使,分十五道,天寶末,又兼黜陟使,乾元元年,改曰觀察處置使。



 西都、東都、北都牧各一人,從二品;西都、東都、北都、鳳翔、成都、河中、江陵、興元、興德府尹各一人,從三品:掌宣德化,歲巡屬縣,觀風俗、錄囚、恤鰥寡。親王典州,則歲以上佐巡縣。武德元年,雍州置牧一人,以親王為之,然常以別駕領州事。永徽中,改尹曰長史。初,太宗伐高麗,置京城留守,其後車駕不在京都,則置留守,以右金吾大將軍為副留守;開元元年,改京兆、河南府長史復為尹,通判府務,牧缺則行其事;十一年,太原府亦置尹及少尹,以尹為留守,少尹為副留守:謂之三都留守。三都大都督府有典獄十八人,問事十二人,白直二十四人;典獄以防守囚系,問事以行罰。中府、上州,典獄十四人,問事八人,白直二十人;下府、中州,典獄十二人,問事六人,白直十六人;下州,典獄八人,問事四人,白直十六人。自三都以下,皆有執刀十五人。



 少尹二人,從四品下。掌貳府州之事,歲終則更次入計。



 司錄參軍事二人,正七品上。錄事四人,從九品上。功曹、倉曹、戶曹、田曹、兵曹、法曹、士曹參軍事各二人,皆正七品下。參軍事六人,正八品下。六府錄事參軍事以下減一人。錄事參軍事,掌正違失,蒞符印。武德初,改州主簿曰錄事參軍事,開元元年,改曰司錄。有史十人。大都督府有史四人,中府有史三人,下府、都護府、上州、中州、下州各有史二人。



 功曹司功參軍事,掌考課、假使、祭祀、禮樂、學校、表疏、書啟、祿食、祥異、醫藥、卜筮、陳設、喪葬。武德初,司功、司倉、司戶、司兵、司法、司士書佐皆為司功等參軍事,有府四人、史十人。大都督府有府三人、史六人;中府有府二人、史三人;下府有府一人、史三人。大都護府有府一人、史二人。上府有府、史各二人。上州有佐二人、史五人;中州減史二人。



 倉曹司倉參軍事,掌租調、公廨、庖廚、倉庫、市肆。有府五人,史十三人。大都督府有府四人,史六人。中府、下府各有府三人,史五人。都護府有府、史各二人。上州有佐二人,史五人;中州、下州減史二人。



 戶曹司戶參軍事,掌戶籍、計帳、道路、過所、蠲符、雜徭、逋負、良賤、芻槁、逆旅、婚姻、田訟、旌別孝悌。有府八人,史十六人,帳史二人,知籍,按帳目捉錢。大都督府有府四人,史七人,帳史二人;中府有府三人,史五人,帳史一人;下府有府二人,史五人,帳史一人。上州有佐四人,史六人,帳史一人;中州有佐三人,史五人,帳史一人;下州有佐二人,史四人,帳史一人。都護府有府、史各二人,帳史一人。



 田曹司田參軍事,掌園宅、口分、永業及廕田。景龍三年,初置司田參軍事,唐隆元年省,上元二年復置。有府四人,史十人。大都督府有府二人,史六人;中府有府、史各二人;下府有府一人,史二人。上州有佐二人,史五人;中州、下州減史二人。



 兵曹司兵參軍事,掌武官選、兵甲、器仗、門禁、管鑰、軍防、烽候、傳驛、畋獵。有府六人,史十四人。大都督府有府四人,史八人;中府有府三人,史六人;下府有府二人,史五人。都護府有府三人,史四人。上州有佐二人,史五人;中州減史二人。



 法曹司法參軍事,掌鞠獄麗法、督盜賊、知贓賄沒入。有府六人,史十四人。大都督府有府三人,史八人;中府有府三人,史六人;下府有府二人,史五人。上州有佐四人,史七人;中州有佐一人,史四人;下州有佐一人,史三人。



 士曹司士參軍事,掌津梁、舟車、舍宅、工藝。有府五人,史十一人。大都督府有府四人,史八人;中府、下府有府三人,史六人。上州有佐二人,史五人;中州有佐一人,史四人。



 參軍事掌出使、贊導。武德初,改行書佐曰行參軍,尋又改曰參軍事。初有亟使十五人,後省。



 文學一人,從八品上。掌以五經授諸生。縣則州補,州則授於吏部。然無職事,衣冠恥之。武德初,置經學博士、助教、學生。德宗即位,改博士曰文學。元和六年,廢中州、下州文學。京兆等三府,助教二人,學生八十人。大都督府、上州,各助教一人;中都督府,學生五十人;下府、下州,各四十人。



 醫學博士一人,從九品上。掌療民疾。貞觀三年,置醫學,有醫藥博士及學生。開元元年,改醫藥博士為醫學博士,諸州置助教,寫《本草》、《百一集驗方》藏之。未幾,醫學博士、學生皆省,僻州少醫藥者如故。二十七年,復置醫學生,掌州境巡療。永泰元年,復置醫學博士。三都、都督府、上州、中州各有助教一人。三都學生二十人,都督府、上州二十人,中州、下州十人。



 ○大都督府



 都督一人,從二品;長史一人,從三品;司馬二人,從四品下;錄事參軍事一人,正七品上;錄事二人,從九品上;功曹參軍事、倉曹參軍事、戶曹參軍事、田曹參軍事、兵曹參軍事、法曹參軍事、士曹參軍事各一人,正七品下;參軍事五人,正八品下;市令一人,從九品上;文學一人,正八品下;醫學博士一人,從八品上。



 ○中都督府



 都督一人,正三品;別駕一人,正四品下;長史一人,正五品上;司馬一人,正五品下;錄事參軍事一人,正七品下;錄事二人,從九品上;功曹參軍事、倉曹參軍事、戶曹參軍事、田曹參軍事、兵曹參軍事、法曹參軍事、士曹參軍事各一人,從七品上;參軍事四人,從八品上;市令一人,從九品上;文學一人,從八品上;醫學博士一人,正九品上。



 ○下都督府



 都督一人,從三品;別駕一人,從四品下;長史一人,從五品上;司馬一人,從五品下;錄事參軍事一人,從七品上;錄事二人,從九品上;功曹參軍事、倉曹參軍事、戶曹參軍事、田曹參軍事、兵曹參軍事、法曹參軍事、士曹參軍事各一人,從七品下;參軍事三人,從八品下;文學一人,從八品下;醫學博士一人,正九品上。



 都督掌督諸州兵馬、甲械、城隍、鎮戍、糧稟,總判府事。武德初,邊要之地置總管以統軍,加號使持節,蓋漢刺史之任。有行臺,有大行臺。其員有尚書省令一人,正二品,掌管內兵民,總判省事。有僕射一人,從二品,掌貳令事。自左右丞以下,諸司郎中略如京省。又有食貨監一人,丞二人,掌膳羞、財物、賓客、帳具、音樂、醫藥;有農圃監一人,丞四人,掌倉廩、園圃、薪炭、芻槁、運漕;有武器監一人,丞二人,掌兵械、廄牧;有百工監一人,丞四人,掌舟車、營作。監皆正八品下,丞正九品下。七年,改總管曰都督,總十州者為大都督。貞觀二年,去「大」字,凡都督府有刺史以下如故,然大都督又兼刺史,而不檢校州事。其後都督加使持節,則為將,諸將亦通以都督稱,唯朔方猶稱大總管。邊州別置經略使,沃衍有屯田之州,則置營田使。武后聖歷元年,以夏州都督領鹽州防禦使。及安祿山反,諸郡當賊沖者,皆置防禦守捉使。乾元元年,置團練守捉使、都團練守捉使,大者領州十餘,小者二三州。代宗即位,廢防禦使,唯山南西道如故。元載秉政,思結人心,刺史皆得兼團練守捉使。楊綰為相,罷團練守捉使,唯澧、朗、峽、興、鳳如故。建中後,行營亦置節度使、防禦使、都團練使。大率節度、觀察、防禦、團練使,皆兼所治州刺史。都督府則領長史,都護府則領都護,或亦別置都護。都督府有掾,有屬,有記室參軍事,有典簽,武德中省。



 市令一人,從九品上。掌交易,禁奸非,通判市事。貞觀十七年廢市令。垂拱元年復置。都督府、三都、諸州,各有市丞一人,佐一人,史二人,帥三人,分行檢察;倉督二人,顓蒞出納;史二人。下州省丞。



 ○大都護府



 大都護一人,從二品;副大都護二人,從三品;副都護二人,正四品上;長史一人,正五品上;司馬一人,正五品下;錄事參軍事一人,正七品上;錄事二人,從九品上;功曹參軍事、倉曹參軍事、戶曹參軍事、兵曹參軍事、法曹參軍事各一人,正七品下;參軍事三人,正八品下。



 ○上都護府



 都護一人,正三品;副都護二人,從四品上;長史一人,正五品上;司馬一人,正五品下;錄事參軍事一人,正七品下;功曹參軍事、倉曹參軍事、戶曹參軍事、兵曹參軍事各一人,從七品上;參軍事三人,從八品上。



 都護掌統諸蕃,撫慰、征討、敘功、罰過,總判府事。



 ○上州



 刺史一人,從三品,職同牧尹;別駕一人,從四品下。武德元年,改太守曰刺史,加使持節,丞曰別駕。十年,改雍州別駕曰長史。高宗即位,改別駕皆為長史。上元二年,諸州復置別駕,以諸王子為之。永隆元年省,永淳元年復置。景雲二年,始參用庶姓。天寶元年,改刺史曰太守。八載,諸郡廢別駕,下郡置長史一員。上元二年,諸州復置別駕。德宗時,復省。元和、長慶之際,兩河用兵,裨將有功者補東宮王府官,久次當進及受代居京師者,常數十人,訴宰相以求官;文宗世,宰相韋處厚建議,復置兩輔、六雄、十望、十緊州別駕。



 長史一人,從五品上;司馬一人,從五品下;錄事參軍事一人,從七品上;錄事二人,從九品下;司功參軍事一人、司倉參軍事一人、司戶參軍事二人、司田參軍事一人、司兵參軍事一人、司法參軍事二人、司士參軍事一人,皆從七品下;參軍事四人,從八品下;市令一人,從九品上;丞一人,從九品下;文學一人,從八品下;醫學博士一人,從九品下。



 ○中州



 刺史一人,正四品下;錄事參軍事一人,正八品上;錄事一人,從九品上;司功參軍事、司倉參軍事、司戶參軍事、司田參軍事、司兵參軍事、司法參軍事、司士參軍事各一人,正八品下;參軍事三人,正九品下;醫學博士一人,從九品下。



 ○下州



 刺史一人,正四品下;別駕一人,從五品上;司馬一人,從六品上;錄事參軍事一人,從八品上;錄事一人,從九品下;司倉參軍事、司戶參軍事、司田參軍事、司法參軍事各一人,從八品下;參軍事二人,從九品下;醫學博士一人,從九品下。



 諸軍各置使一人,五千人以上有副使一人,萬人以上有營田副使一人。軍皆有倉、兵、胄三曹參軍事。刺史領使,則置副使、推官、衙官、州衙推、軍衙推。



 ○京縣



 令各一人,正五品上;丞二人,從七品上;主簿二人,從八品上;錄事二人,從九品下;尉六人,從八品下。



 ○畿縣



 令各一人,正六品上;丞一人,正八品下;主簿一人,正九品上;尉二人,正九品下。



 ○上縣



 令一人,從六品上;丞一人,從八品下;主簿一人,正九品下;尉二人,從九品上。



 ○中縣



 令一人,正七品上;丞一人,從八品下;主簿一人,從九品上;尉一人,從九品下。



 ○中下縣



 令一人,從七品上;丞一人,正九品上;主簿一人,從九品上;尉一人,從九品下。



 ○下縣



 令一人,從七品下;丞一人,正九品下;主簿一人,從九品上;尉一人,從九品下。



 縣令掌導風化,察冤滯,聽獄訟。凡民田收授,縣令給之。每歲季冬,行鄉飲酒禮。籍帳、傳驛、倉庫、盜賊、堤道,雖有專官,皆通知。縣丞為之貳,縣尉分判眾曹,收率課調。武德元年,改書佐曰縣尉,尋改曰正。諸縣置主簿,以流外為之。京縣、上縣,丞皆一人;畿縣、上縣,正皆四人。七年,改縣正復曰尉。貞觀初,諸縣置錄事。開元,上縣萬戶、中縣四千戶以上,增尉一人。京兆、河南府諸縣,戶三千以上置市令一人,戶一萬以上置義倉督三人。其後畿縣戶不及四千,亦置尉二人,萬戶增一人。凡縣有司功佐、司倉佐、司戶佐、司兵佐、司法佐、司士佐、典獄、門事等,畿縣減司兵,上縣有司戶、司法而已。凡縣皆有經學博士、助教各一人,京縣學生五十人,畿縣四十人,中縣以下各二十五人。



 上鎮,將一人,正六品下;鎮副二人,正七品下;倉曹參軍事、兵曹參軍事各一人,從八品下。中鎮,將一人,正七品上;鎮副一人,從七品上;兵曹參軍事一人,正九品下。下鎮,將一人,正七品下;鎮副一人,從七品下;兵曹參軍事一人,從九品下。每鎮又有使一人、副使一人。凡軍鎮,二萬人以上置司馬一人,正六品上;增倉曹、兵曹參軍事各一人,從七品下。不及二萬者,司馬從六品上,倉曹、兵曹參軍事正八品上。



 上戍,主一人,正八品下;戍副一人,從八品下。中戍,主一人,從八品下。下戍,主一人,正九品下。



 鎮將、鎮副,戍主、戍副,掌捍防守御。凡上鎮二十,中鎮九十,下鎮一百三十五;上戍十一,中戍八十六,下戍二百四十五。倉曹參軍事,掌儀式、倉庫、飲膳、醫藥、付事、句稽、省署鈔目、監印、給紙筆、市易、公廨。中鎮則兵曹兼掌。兵曹參軍事,掌防人名帳、戎器、管鑰、馬驢、土木、謫罰之事。上鎮:有錄事一人,史一人,倉曹佐一人、史二人,兵曹佐、史各二人,倉督一人、史二人;中鎮:錄事一人,兵曹佐一人、史四人,倉督一人、史二人;下鎮:錄事一人,兵曹佐一人、史二人,倉督一人、史一人。凡軍鎮,五百人有押官一人,千人有子總管一人,五千人又有府三人、史四人。上戍:佐一人、史二人;中戍:史二人;下戍:史一人。唐廢戍子,每防人五百人為上鎮,三百人為中鎮,不及者為下鎮;五十人為上戍,三十人為中戍,不及者為下戍。開元十五年,朔方五城各置田曹參軍事一人,品同諸軍判司,專蒞營田。永泰後,諸鎮官頗增減開元之舊。



 五岳、四瀆,令各一人,正九品上,掌祭祀。有祝史三人,齎郎各三十人。



 上關:令一人,從八品下;丞二人,正九品下。中關:令一人,正九品下;丞一人,從九品下。下關:令一人,亦從九品下。掌禁末游,察奸慝。凡行人車馬出入,據過所為往來之節。凡關二十有六,京四面關有驛道者為上關,元驛道者為中關,餘為下關。丞掌付事、句稽、監印、省署鈔目,通判關事。上關:錄事一人,府二人,史四人,典事六人;中關:錄事一人,府二人,史二人,典事四人;下關:府一人,史、典事各二人。典事掌巡雉及雜當。初,諸關置都尉,亦有它官奉敕監者。上津置尉一人,掌舟梁之事;府一人,史二人,津長四人。下津尉一人,府一人,史二人,津長二人。永徽中,廢津尉,上關置津吏八人。永泰元年,中關置津吏六人,下關四人,無津者不置。



\end{pinyinscope}