\article{列傳第九十五 二高伊硃二劉範二王孟趙李任張}

\begin{pinyinscope}

 高崇文,字崇文。其先自渤海徙幽州,七世不異居,開元中,再表其閭。崇文性樸重寡言的純形式,因果性是知性的先天範疇(範疇共十二,因果性,少籍平盧軍。貞元中,從韓全義鎮長武城,治軍有聲。累官金吾將軍。吐蕃三萬寇寧州,崇文率兵三千往救,戰佛堂原,大破之,封渤海郡王。全義入朝,留知行營節度後務,遷長武城都知兵馬使。



 劉闢反,宰相杜黃裳薦其才,詔檢校工部尚書、左神策行營節度使,俾統左右神策、麟游奉天諸屯兵討闢。時顯功宿將,人人自謂當選,及詔出,皆大驚。始,崇文選兵五千,常若寇至。至是,卯漏受命,辰已出師,器良械完,無一不具。過興元,士有折逆旅匕箸者,即斬以徇。乃西自閬中出,卻劍門兵,解梓潼之圍,賊將邢泚退守梓州。詔拜崇文東川節度使。初,闢陷東川,執節度使李康不殺也;至是,歸康以丐雪,崇文數康失守罪,斬之。鹿頭山南距成都百五十里,扼二川之要,闢城之,旁連八屯,以拒東兵。崇文始破賊二萬於城下,會雨,不克攻。明日,戰萬勝堆,堆直鹿頭左,使驍將高霞寓鼓之,士扳緣上,矢石如雨,募死士奪而有之,盡殺戍者,焚其柵,下瞰鹿頭城,人可頭數。凡八戰皆捷,賊心始搖。大將阿跌光顏與崇文約,後期,懼罪,請深入自贖,乃軍鹿頭西,斷賊糧道。賊大震,其將李文悅以兵三千自歸,仇良輔舉鹿頭城二萬眾降,執闢子方叔、婿蘇強。遂趣成都,餘兵皆面縛送款。闢走,追禽之,檻送京師。



 入成都也,師屯大達,市井不移,珍貨如山,無秋毫之犯。邢泚已降而貳,斬於軍,衣冠脅污者詣牙請命,崇文為條上全活之。進檢校司空、西川節度副大使,南平郡王,實封三百戶,刻石紀功於鹿頭山。



 崇文不通書,厭案牘諮判以為繁,且蜀優富無所事,請捍邊自力,乃詔同中書門下平章事、邠寧慶節度使,為京西諸軍都統。崇文恃功而侈,舉蜀帑藏百工之巧者皆自隨,又不曉朝廷儀,憚於覲謁,有詔聽便道之屯。居邠三年,戎備整修。卒,年六十四,贈司徒,謚曰威武。會昌六年,詔配享憲宗廟。



 子承簡,少事忠武軍,後更隸神策。以崇文平蜀功,除嘉王傅。裴度徵蔡,奏署牙將。蔡平,詔析上蔡、郾城、遂平、西平四縣為洲殷州,拜承簡刺史,治郾城。始開屯田,列防庸,瀕溵綿地二百里無復水敗,皆為腴田。先是,賊築武宮以誇戰勞,承簡夷其丘,庀家財以葬。葺儒宮,備俎豆,歲時行禮。野有荍實,民得以食。將吏立石頌功。遷邢州刺史,觀察府責賦尤急,承簡代下戶數百輸租。



 遷宋州。會宣武將李朅反,遣使責財於宋,承簡囚之,前後數輩輒系獄,一日並出斬於牙門,威震部中。朅悉兵攻之。宋有三城,南城陷,承簡保北兩城,數與賊確。會徐州救至,朅為李質所執,兵遂潰。拜兗海沂密節度使。遷義成軍,檢校尚書左僕射。入拜右金吾衛大將軍,復節度邠寧。先是,虜多以盛秋犯邊,承簡請屯寧州以制其侵。屬疾還朝,道卒,贈司空,謚曰敬。



 崇文孫駢,自有傳。



 伊慎,字寡悔,兗州人。通《春秋》、《戰國策》、天官、五行書,用善射為折沖都尉。喪母,將合葬而不知父墓,晝夜哭,夢若有導者;既發之,舊志可按也,乃得葬。



 江西路嗣恭討哥舒晃,以慎為先鋒。疾戰破賊,斬首三千級,下韶州。戰把江口,水湍駛,乃為桴,寘薪焉,乘風縱火,賊焚且溺不可計,與諸將追斬晃泔溪。授連州長史,知團練副使。三遷江州別駕。



 討梁崇義也,慎以江西牙兵屬李希烈,希烈署漢南北兵馬使,不受,獨率所部破崇義於蠻水,效俘三萬。襄、漢平,功多。希烈愛其材,數饋遺,欲縻止之,卒以計免。明年,希烈果反。嗣曹王皋至鐘陵,得而壯之,拔為大將。希烈恐為皋所任,遺以七屬甲,詐為慎書,行反間。帝遣使即軍中斬之,皋表列其誣,未報。賊溯江徇地,皋授慎兵,勞而遣,與賊大戰,破之,收黃梅,次長平,殺賊將,斬級千餘,拔蔡山尤力。遂下蘄州,即拜刺史,封南充郡王。



 天子在梁州,包佶轉東南財糧次蘄口,賊遣驍將杜少誠以兵萬人遏江道,不得西。慎選士七千,列三屯相望,偃旗以待。少誠分圍之,未合,慎自中屯鼓之,諸屯悉出奮擊,賊亂,少誠走,斬別將許少華,封其尸為京塚,漕無留艱。進圍安州,希烈之甥劉戒虛以兵八千來援,慎逆擊於應山,禽之,示城下,州開門降。以功為安州刺史,實封百戶。改隋州。戰厲鄉,斬首五千級,喻降李惠登,即薦惠登為刺史。拜慎安、黃州節度使。



 吳少誠反,詔領步騎五千兼統荊南、湖南、江西兵,當一面,遇賊於三州港,營義陽,戰於申,斬首數千,加檢校刑部尚書。貞元末,詔安、黃為奉義軍,即為奉義節度。



 憲宗即位,以兵付其子宥,身入朝,拜尚書右僕射,改金吾衛大將軍。以錢三千萬賂宦人求帥河中,事暴,帝沒其半贓,貶右衛將軍。明年,念舊勞,復檢校右僕射兼右衛上將軍。卒,贈太子太保,謚曰壯繆。乾符中,盜發其墓,賜絹二百修瘞云。



 硃忠亮,字仁輔,汴州浚儀人。舉明經不中,往事昭義節度使薛嵩為裨將,屯普潤,開田峙糧,以功擢太子賓客。



 硃泚亂,率麾下四十騎至奉天,封東陽郡王,為「定難功臣」。扈狩梁州,為賊鈔獲,系長安獄。賊平,李晟釋之,奏隸本軍,累遷定平軍使。憲宗立,加御史大夫。涇州將楊琦謀拒詔為亂,方集諸校計事,屋壞,琦壓死,乃授忠亮涇原四鎮節度使。本名士明,至是賜今名。



 隱核軍籍,得竄名者三千人,歲收乾沒十萬緡。吏白耄卒不任戰者可罷,答曰:「古於老馬不棄,況戰士乎?」聞者莫不感奮。涇俗舊多賣子,忠亮以財贖免者前後數百。築潘原城有勞,改封丹陽。卒,贈尚書右僕射,謚曰靈。



 劉昌裔,字光後,太原陽曲人。幼重遲不好戲,常若有所思度。及壯,策說邊將不售,去入蜀。楊惠琳亂,昌裔說之。惠琳順命,拜瀘州刺史,署昌裔州佐。惠琳死,客河朔間。曲環方攻濮州,表為判官。為環檄李納,剴曉大誼,環上其稿,德宗異之。環領陳許軍,又從府遷。累進營田副使。



 環卒,上官涚知後務,吳少誠引兵薄城,涚欲遁去,昌裔止曰:「受詔而守,死其職也。況士馬完奮,足支賊。若堅壁不戰七日,賊氣必衰,我以全制之可也。」涚許諾。賊攻堞壞,不得修。昌裔密造飛棚聯柵,即募突將千人鑿城以出,擊賊走之。比還,柵已立,守陴遂安。兵馬使安國寧謀應賊,昌裔以計斬之;召其麾下千人為饗,人賞二縑,乃伏兵於道,令「持縑者斬」,一不能脫,賊聞解去。以功擢涚陳許節度使,昌裔陳州刺史。



 韓全義敗於溵水,引軍走陳,求入保,昌裔登陴揖曰:「天子命君討蔡,何為來陳?且賊不敢至我城下,君其舍外無恐。」明日,從十餘騎持牛酒抵全義營勞軍,全義不自意,迎拜嘆服。改陳許行軍司馬。涚卒,軍中推昌裔,有詔檢校工部尚書,代節度。命境上吏不得犯蔡人,少誠吏有來犯者,捕得縛送,使自治之。少誠慚其軍,亦禁境上暴掠者。封彭城郡公。



 元和八年,大水壞廬舍,溺居人,以檢校尚書左僕射兼左龍武統軍召還京師。始,憲宗惡昌裔自立,欲召之而重生變,宰相李吉甫曰:「陛下乘人心愁苦可召也。」遂以韓皋代之。至長樂驛,知帝意,因稱風眩臥第。歲中卒,贈潞州大都督,謚曰威。



 範希朝,字致君,河中虞鄉人。初從邠寧軍為別將,事節度使韓游瑰。德宗在奉天,以戰守功累兼御史中丞。治軍整毅,游瑰畏其才,將伺隙殺之,希朝懼,奔鳳翔。帝聞,召寘左神策軍。貞元四年,以游瑰政無狀,使代之。希朝曰:「始偪而來,終代其任,非所以防覬覦、安反仄也。」固讓左金吾衛將軍張獻甫。軍中憚獻甫嚴,以兵脅監軍使請於帝,必得希朝乃止。詔拜寧州刺史、邠寧節度副使,俾佐獻甫。



 俄遷振武節度使。部有黨項、室韋雜居,暴掠放肆,日入慝作,謂之「刮城門」。希朝度要害置屯保,斥邏嚴密,鄙民以安。至小竊取亦殺無赦,虜人憚伏,相謂曰:「是必張光晟紿姓名來也!」邊州每長帥至,必效橐它駿馬,雖甚廉者猶受之,以結其歡。希朝一不納。積十四年,虜保塞不敢橫。初,單于城池不樹,希朝命蒔柳,數歲成林。



 貞元末,請朝。時諸鎮不以事自述職者,希朝而已。帝悅,拜右金吾衛大將軍。王叔文用事,謂其易制,用為右神策統軍,充左右神策京西諸城鎮行營節度使,屯奉天,以韓泰為副,因欲使泰代之。會不能得神策軍而罷。憲宗立,檢校尚書左僕射,復為右金吾衛大將軍。俄檢校司空,出為朔方靈鹽節度使。遷河東,率師討王承宗,敗之木刀溝,然老病,不能有大功。還朝,改左龍武統軍,以太子太保致仕。卒,贈太子太師,謚忠武,改曰宣武。



 希朝號當世善將,或比之趙充國。在朔方時,招突厥別部沙陀千落眾萬餘有之,其後用沙陀戰者,所至有功。



 王鍔,字昆吾,自言太原人。始隸湖南團練府為裨將。楊炎道潭,與語,異其才。嗣曹王皋為團練使,俾鍔誘降武岡叛將王國良,以功擢邵州刺史。



 皋之節度江西也,李希烈南侵,皋與鍔兵三千,使屯潯陽,而皋全軍臨九江,襲蘄州,遂以眾濟。表鍔江州刺史兼御史中丞,充都虞候。鍔小心,善刺軍中情偽,事無細大,皋悉知之。因推以腹心,雖家人燕居或預焉。皋攻安州,使伊慎盛兵圍之,而遣鍔入城中約降,使殺不從者。翌日城開,慎以賊降乃己功,不下鍔,鍔稱疾避之。



 皋為荊南節度使,欲署府少尹,而上佐鄙其人,乃復檄都虞候。從皋朝京師,皋奏鍔文用雖不足,而它可試。德宗擢為鴻臚少卿。先是,天寶末,西域朝貢酋長及安西、北庭校吏歲集京師者數千人,隴右既陷,不得歸,皆仰稟鴻臚禮賓,月四萬緡,凡四十年,名田養子孫如編民。至是,鍔悉籍名王以下無慮四千人,畜馬二千,奏皆停給。宰相李泌盡以隸左右神策軍,以酋長署牙將,歲省五十萬緡。帝嘉其公,擢容管經略使。凡八年,溪落安之。



 遷嶺南節度使。廣人與蠻雜處,地征薄,多牟利於市,鍔租其廛,榷所入與常賦埒,以為時進,裒其餘悉自入。諸蕃舶至,盡有其稅,於是財蓄不貲,日十餘艘載皆犀象珠琲,與商賈雜出於境。數年,京師權家無不富鍔之財。召為刑部尚書。淮南節度使杜佑數請代,乃以鍔檢校兵部尚書為佑副,厚事佑以悅之,坐必就司馬聽事,不數日,遂代佑。久之,入拜尚書左僕射,又檢校司徒,為河中節度使。



 進兼太子太傅,徙河東。河東自範希朝討鎮無功,兵才三萬,騎六百,府庫殘耗。鍔能補完嗇費,未幾,兵至五萬,騎五千,財用豐餘。會回鶻並摩尼師入朝,鍔欲示威武傾駭之,乃悉軍迎,廷列五十里,旗幟光鮮,戈鎧犀密。回鶻恐,不敢仰視,鍔偃然受其禮。帝聞嘉之,即除檢校司空、同中書門下平章事。鍔自見居財多,且懼謗,納錢二千萬。李絳奏言:「鍔雖有勞,然僉望不屬,恐天下議以為宰相可市而取。」帝曰:「鍔當太原殘破後,成雄富之治。官爵所以待功,功之不圖,何以為勸?王播所獻數萬萬,亦可以平章政事乎?」不聽。卒,贈太尉,謚曰魏。



 鍔初附太原王翃為從子,以婚閥自高。翃子弟亦藉鍔多得官。又常讀《春秋》,自稱儒者,士頗笑之。善任數持下,在淮南時,嘗得無名書,內靴中,俄取它書焚之,人信其無名者,異日因小罪,並以所告窮驗,示眾以神明。性纖嗇,有所程作,雖碎瑣無所遺。官曹簾壞,吏將易之,鍔取壞者付船坊以針箬。每燕饗,輒錄其餘,賣之以收利。故鍔家錢遍天下。



 子稷,歷鴻臚少卿。鍔在籓,稷常留京師,視勢高下輕重以納貲焉。嘗請籍坊以廣第舍,作復垣洞穴,實金錢其中。鍔卒,奴告稷更遺占,沒所獻,裴度為言,乃論殺奴。長慶二年,用稷為德州刺史,悉金寶、媵侍以行。節度使李全略利其貨,因軍亂殺稷,納其女為媵。



 開成中,滄州節度使劉約奏稷子叔泰生五歲,值全略亂,為郡人匿養,得不死。送叔泰京師,文宗憫焉,詔授九品官,使奉鍔祀。



 孟元陽,史失其何所人。起陳許軍中,以嚴整稱。曲環領節度使,時已為大將,使董作西華屯。盛夏,屩而立於塗,役休乃就舍,故田輒歲稔,而軍食常足。環卒,吳少誠來寇,元陽嬰城守,圍甚急,然終不能傅城。韓全義敗五樓,列將多私去,獨元陽與神策將蘇元策、宣州將王幹以所部屯溵水,破賊二千,詔拜陳州刺史。憲宗立,遷河陽節度使。五年,盧從史敗,檢校尚書右僕射,徙帥昭義軍。入為右羽林統軍,封趙國公。改右金吾大將軍,復拜統軍。卒,贈揚州大都督。



 王棲曜,濮州濮陽人。安祿山反,尚衡裒義兵討賊,署牙將,徇兗、鄆諸縣下之,進牙前總管。賊將邢超然守曹州,乘城指顧,棲曜曰:「彼可取也。」一矢殞之,遂拔曹州。累授試金吾衛將軍。



 袁晁亂浙東,御史中丞袁傪討之,表為偏將。與賊戰,日十餘遇,生禽晁,收州縣十六。授常州別駕、浙西都知兵馬使。時江介未定,詔內常侍馬日新以汴滑軍五千鎮之。中人暴橫,賊蕭廷蘭乘眾怨逐日新,劫其眾。棲曜方游弈近郊,賊脅取之,與圍蘇州。棲曜乘賊怠,挺身登城,率城中兵出戰,賊眾大敗,遷試金吾大將軍。



 李靈曜反汴州,浙西觀察使李涵使提兵四千為河南掎角,有功。李希烈陷汴州也,乘勝東略,次寧陵,將襲宋州。浙西節度使韓滉使棲曜以強弩三千涉水,夜入寧陵,希烈不之知。晨朝,矢集帳前,驚曰:「江淮弩士入矣!」遂不敢東。



 貞元初,拜左龍武大將軍,出為鄜坊節度使。十九年,卒,贈尚書右僕射,謚曰成。



 棲曜性謹厚,善騎射。始將兵時,涉寇境,遇游騎環合,乃規百步,立表而射,每射破的,虜相顧懼,引去。



 子茂元,少好學。德宗時上書自薦,擢試校書郎,改太子贊善大夫。呂元膺留守東都,署防禦判官。淄青留邸卒謀亂,元膺率兵圍之,士無敢先者,茂元取一人斬之,眾乃進,賊遂出奔。累遷嶺南節度使,蠻落安之。



 家積財,交煽權貴。鄭注用事,遷涇原節度使。注敗,悉出家貲餉兩軍,得不誅,封濮陽郡侯。召為將作監,領陳許節度使,又徙河陽。討劉真也,李德裕以茂元兵寡,詔王宰領陳許合義成兵援之,以河陰所貯兵械、內庫甲弓矢陌刀賜之。會病,以宰兼河陽行營攻討使。卒,贈司徒,謚曰威。



 劉昌,字公明,汴州開封人。善騎射。天寶末,從河南防禦使張介然討安祿山,授易州遂城府左果毅。史朝義兵圍宋州,城中食盡且降。昌說刺史李岑曰:「李光弼在河陽,江淮足兵,勢必來援。今廩麴尚多,若屑以食,可支二十日,則救至。」岑聽之。昌乃被鎧登城,以忠義諭賊,賊畏不敢攻。俄而光弼援兵至,賊夜潰。光弼聞其謀,召置軍中,將用之。會光弼卒,還為宋州牙門將。



 李靈曜以汴州反,刺史李僧惠欲應之,昌請見,陳逆順計,且泣。僧惠悟,即馳奏請自將討賊。故靈曜失助,不得逞。汴州平,李忠臣疾僧惠,攻殺之,昌遁去。



 劉玄佐領宣武節度使,擢昌左廂兵馬使。李納反,以偏師收考城,充行營諸軍馬步都虞候。玄佐攻濮州,以昌攝刺史。李希烈取汴,玄佐別將高翼提精卒守襄邑,城陷,翼赴水死,江淮大震。昌以兵三千守寧陵,希烈眾五萬攻之,昌掘塹以遏地道,相拒凡四十餘日,賊數敗,乃解圍去。更攻陳州,昌從玄佐以浙西兵三萬救之。西去陳五十里,昌薄其軍,大戰破之,禽賊將翟曜,希烈奔還蔡州。加檢校工部尚書,累實封二百戶。



 貞元三年入朝,詔以宣武兵八千北出五原。士卒有逗留沮事者,斬三百人乃行,舉軍心習伏。尋授京西行營節度使。歲餘,改四鎮、北庭行營兼涇原節度。七年,城平涼,開地二百里,扼彈箏峽。又西築保定,捍青石嶺,凡七城二堡,旬日就。以功檢校尚書右僕射,累封南川郡王。十四年,歸化堡軍亂,逐大將張國誠,詔昌經略。昌入堡,誅數百人,復使國誠統之。昌在邊凡十五年,身率士墾田,三年而軍有羨食,兵械銳新,邊障妥寧。及感疾,詔赴京師。未行,卒,年六十五,贈司空。



 初城平涼,當劫盟後,將士骸骨不藏,昌始命瘞之。夕夢若詣昌厚謝者,昌具以聞。德宗下詔哀痛,出衣數百稱,官為賽具,斂以棺槥,分建二塚,大將曰旌義塚,士曰懷忠塚,葬淺水原,詔翰林學士為銘識其所。昌盛陳兵衛,具牢醴,率諸將素服臨之,邊兵莫不感泣。



 子士涇,尚雲安公主,拜駙馬都尉,累遷少卿。家積財,內結權近。善胡琴,故得幸於貴人。後遷太僕卿,給事中韋弘景等封還制書,以士涇交通近幸,不當居九卿。憲宗曰:「昌有功於邊,士涇又尚主,官少卿已十餘年,制書宜下。」弘景等乃奉詔。



 贊曰:唐杜牧稱:「寧陵之圍解,劉玄佐召昌問曰:『君以孤城,用一當十,何以能守?』昌泣曰:『始昌令:守陴內顧者斬。昌孤甥張俊守西北,未嘗內顧,捽下斬之。士有死志,故能守。』因伏地流涕,玄佐亦泣曰:『國家將富貴汝。』」史臣謂不然,且勒兵乘城與賊抗,所賴賞罰耳。今無罪而斬其甥,士心且離,不祥莫大焉。寧好事者傅此以益其美?非昌志也。牧以為張巡、許遠陷睢陽,其名傳,昌全寧陵而事不得暴於世,寧牧未之思邪?



 趙昌,字洪祚,天水人。始為昭義李承昭節度府屬,累遷虔州刺史。安南酋獠杜英翰叛,都護高正平以憂死,拜昌安南都護,夷落向化毋敢桀。居十年,足疾,請還朝,以兵部郎中裴泰代之,入為國子祭酒。未幾,州將逐泰,德宗召昌問狀,時年逾七十,占對精明,帝奇之,復拜安南都護。詔書至,人相賀,叛兵即定。



 憲宗初立,檢校戶部尚書,遷嶺南節度使。降輯陬荒,以勞徙節荊南。召入,再遷工部尚書、兼大理卿。出為華州刺史。對麟德殿,趨拜強駃,帝訪其所以頤養。遷太子少保。卒,年八十五,贈揚州大都督,謚曰成。



 李景略,幽州良鄉人。父承悅,檀州刺史、密雲軍使。景略以廕補幽州府功曹參軍。大歷末,客河中,闔門讀書。



 李懷光為朔方節度使,署巡官。五原將張光殺其妻,以貲市獄,前後不能決,景略核實,論殺之。既而有若女厲者進謝廷中,如光妻云。遷大理司直。懷光屯咸陽,將襲東渭橋,召幕府計議。景略曰:「殺硃泚,還軍諸道,杖策詣行在,此轉禍為福也。」不聽。既出軍門,慟哭曰:「豈意此軍乃陷不義乎!」遂遁歸。



 靈武節度使杜希全表置於府,累轉侍御史、豐州刺史。豐州當回紇通道,前刺史軟柔,每虜使至,與抗禮。時梅錄將軍入朝,景略欲折之,因郊勞,前遣人謂曰:「可汗新沒,欲吊使者。」乃坐高壟待之。梅錄俯僂前哭,景略即撫之曰:「可汗棄代,助爾號慕。」於是虜容氣沮索,不敢抗,以父行呼景略。自此回紇使至者,皆拜於廷,威名顯聞。希全忌之,誣奏,貶袁州司馬。



 希全死,遷左羽林將軍,對德宗延英殿,論奏衎衎,有大臣風。會河東節度使李說病,以景略為太原少尹、行軍司馬。時方鎮既重,故少召還者,惟不幸則司馬代之。自說有疾,人心固屬景略矣。會梅錄復入朝,說大會,虜人爭坐,說不敢遏,景略叱之,梅錄識其聲,驚拜曰:「非李豐州邪?」遂就坐。將吏相顧嚴憚,說愈不平,賂中尉竇文場謀毀去之。



 歲餘,塞下傳言回紇將南寇,文場方侍帝傍,即言豐州當得良將,且舉景略,乃拜豐州刺史、天德軍西受降城都防禦使。窮塞苦寒,地脊鹵,邊戶勞悴。景略至,節用約己,與士同甘蓼,鑿咸應、永清二渠,溉田數百頃,儲稟器械畢具,威令肅然,聲雄北疆,回紇畏之。卒於屯,年五十五。天下惜用景略才有所未盡。贈工部尚書。



 任迪簡,京兆萬年人。擢進士第。天德李景略表佐其軍,嘗宴客,而行酒者誤進醢,景略用法嚴,迪簡不忍其死,飲為釂,徐以它辭請易之,歸衉血,不以聞,軍中悅其長者。景略卒,舉軍請為帥,監軍使拘迪簡,不聽,眾大呼,破戶出之。德宗遣使者察變,具得所以然,乃授豐州刺史、天德軍使。由殿中侍御史授兼大夫、散騎常侍。入為太常少卿、太子左庶子。



 張茂昭以易定歸,擢迪簡行軍司馬代之。大將楊伯玉據牙不納,眾殺之;別將張佐元復叛,迪簡斬以徇,乃入,以檢校工部尚書為節度使。承茂昭奢縱後,公私屈覂,欲饗士,無所給,至與下同糲食,身居戟戶。逾月,軍中感其公,請安臥內,迪簡乃許。三年,上下完充。以疾入,除工部侍郎。不能朝,改太子賓客。卒,贈刑部尚書,謚曰襄。



 張萬福,魏州元城人。三世明經,止縣令、州佐。萬福以儒業不顯,乃學騎射,從王斛斯以別校征遼東,有功。



 李峘伐劉展,署為部將,效首萬級。累攝壽州刺史、舒廬壽都團練使。州送租賦詣都,至潁,為盜所奪。萬福領輕兵尾襲,賊倉卒不得戰,悉禽之,盡得所亡,並先掠人妻女、財畜萬計,還其家,不能自致者,給船車以遣。真拜刺史,兼淮南節度副使。而節度崔圓忌之,失刺史,改鴻臚卿,使將千人鎮壽州,不以為恨。時許杲以平盧行軍司馬將卒三千駐濠州,陰窺淮南。圓使萬福攝濠州刺史。杲聞,即移戍當塗。賊陳莊陷舒州,圓又令攝舒州刺史,督淮南盜賊,窮破株黨。



 大歷三年,召見。代宗曰:「欲一識卿面,且將以許杲累卿。」萬福辭謝,因前曰:「陛下以一許杲召臣,如河北諸將叛,欲屬何人?」帝笑曰:「姑為我了杲事,且當大用。」乃拜和州刺史兼行營防禦使,督盜淮南。萬福至州,杲懼,徙屯上元,過楚州,大掠,節度使韋元甫使萬福追討。未至,杲為其將康自勸所逐,自勸循淮鈔而東,萬福倍道追殺之,免者十三,盡還所剽於民。元甫將厚賞士,萬福曰:「官健坐仰衣食,無所事,今一小煩之,不足過賞,請用三之一。」帝下詔褒美,賜具衣、宮錦十雙。



 久之,詔以本鎮兵千五百人防秋京西。萬福詣揚州還所領兵。會元甫死,諸將願得萬福為帥,監軍使邀請之,對曰:「我非幸人,勿以此待我。」遂去。以利州刺史鎮咸陽,且留宿衛。



 李正己反,屯兵埇橋,江淮漕船積千餘不敢逾渦口。德宗乃以萬福為濠州刺史,召謂曰:「先帝改爾名正者,所以褒也。朕謂江淮草木亦知爾威名,若從所改,恐賊不曉是卿也。」復賜舊名。萬福因馳至渦口,駐馬於岸,悉發漕船相銜進,賊兵倚岸熟視不敢動。改泗州刺史。魏州饑,父子相賣,萬福曰:「魏州吾鄉里,安忍其困?」令兄子將米百車餉之,贖魏人自賣者,給資遣之。



 為杜亞所忌,召拜右金吾將軍。及見,帝驚曰:「亞乃言爾昏耄,何邪?」詔圖形凌煙閣,數賜與,並敕度支籍口畜給其費。陽城等詣延英門論裴延齡事,伏閣不去,帝震怒,左右懼不測。萬福大言曰:「國有直臣,天下無慮矣。吾年八十,與見盛事。」遍揖城等勞之,天下益重其名。以工部尚書致仕,卒,年九十。



 萬福自始終祿食七十年,未嘗一日言病。蒞凡九州,皆有惠愛。初,在泗州,遇李希烈反,陳少游悉以部刺史妻子質揚州,萬福獨不遣。謂使者:「為我白公,妻老且醜,不足慁公意。」卒不行,人稱其直。



 高固,不知何許人,或言四世祖侃,永徽中為北庭安撫使,禽車鼻可汗,以功為安東都護。



 固生微賤,為家所賣,轉為渾瑊童奴,字黃芩。性敏惠,有旅力,善騎射,能讀《左氏春秋》。瑊愛養之,以齊有高固,因以名,以乳媼女女固。從瑊屯朔方。德宗在奉天,固仍從瑊,賊突入東壅門,固引銳士長刀殺賊數十人,曳車塞闔,賊不能入。封渤海郡王。



 李懷光反,使邠寧留後張昕將兵萬人先趣河中,固在行,乃伺間入帳下,斬昕首以徇,拜檢校右散騎常侍、前軍兵馬使。貞元十七年,邠寧節度使楊朝晟卒,詔將並邠寧、朔方為一軍,議以李朝寀為節度,劉南金副之,以詢邠軍,咸曰:「如詔。」數日復劫固為帥,固曰:「然能聽吾言。乃可。」眾唯唯。固徇曰:「毋殺人,毋肆掠!」三軍皆順悅。帝亦念固功,乃拜邠寧節度使。固本宿將,且寬厚,人皆安之。然久在散位,數為儕類輕笑。及受命,眾多懼,固一釋不問。



 憲宗時,檢校尚書右僕射,入為右羽林統軍。卒,贈陜州大都督。



 郝玭,不記其鄉里。貞元中為臨涇鎮將,嘗從數百騎出野,還,說節度使馬璘曰:「臨涇扼洛口,其川饒衍,利畜牧。其西走戎道,曠數百里皆流沙,無水草。願城之,為休養便地。」玭出,或謂璘曰:「玭言信然。雖然,公所以蒙恩大幸,以邊防未固也。上心日夜念此,故厚於公。今若用玭言,則邊已安,尚何事為?」璘遂不聽。



 及段佑代節度,玭又說曰:「天寶時,天下以兵為防,獨西戎耳。而塞至京師且萬里。自祿山反,西陲盡亡,寰內為邊郡。每虜入寇,驅井閭父子與馬牛,焚積聚,殘室廬,邊人耗盡。今若築臨涇以折虜勢,便甚。」佑唯許,請於朝。卒詔城臨涇,為行原州,以玭為刺史,戍之。自是虜不敢過臨涇。



 玭在邊積三十年,每討賊,不持糗糧,取之於敵。獲虜必刳剔而歸其尸,虜大畏,道其名以怖啼兒。遷檢校左散騎常侍、涇原行營節度使,封保定郡王。贊普常等故身鑄金象,令於國曰:「得生玭者,以金玭償之。」朝廷畏失名將,徙為慶州刺史,卒。



 佑,本郭子儀牙將,從征伐有功。貞元末,為涇原節度使,虜畏憚之。終右神策大將軍。



 史敬奉者,靈州人。事朔方軍為牙將。元和中,吐蕃數犯塞,十四年,敬奉白節度使杜叔良,請兵三千,齎一月糧,深入虜地,分賊勢。叔良以二千兵予之,行十餘日,不聞問,皆謂已歿。敬奉乃由間道繞出虜後,部落奔駭,因大破之,驅其餘眾於瓠蘆河,獲馬牛雜畜迨萬數。賜實封五十戶。



 敬奉[B126]陋,類不勝衣,其走逐奔馬,挾鞍勒以上,而後羈帶之,矛矢在手,前無強敵。甥侄部曲二百人,每出輒分其隊為四五,隨水草,數日不相知,及相遇,已皆有獲。與鳳翔將野詩良輔及郝玭皆以名雄邊。



 良輔者,後為隴州刺史。朝廷遣使至吐蕃,虜輒言:「唐家稱和好豈妄邪!不爾,安得任良輔為隴州刺史?」



\end{pinyinscope}