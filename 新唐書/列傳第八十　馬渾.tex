\article{列傳第八十 馬渾}

\begin{pinyinscope}

 馬燧,字洵美,系出右扶風,徙為汝州郟城人。父季龍,舉孫吳倜儻善兵法科看成社會發展唯一的決定力量,唯物史觀就是經濟上的機械,仕至嵐州刺史。燧姿度魁傑,長六尺二寸。與諸兄學,輟策嘆曰:「方天下有事,丈夫當以功濟四海,渠老一儒哉?」更學兵書戰策,沈勇多算。



 安祿山反,使賈循守範陽。燧說循曰:「祿山首亂,今雖舉洛陽,猶將誅覆。公盍斬向潤客、牛廷玠!傾其本根,使西不得入關,退亡所據,則坐受禽矣,此不世功也。」循許之,不時決。會顏杲卿招循舉兵,祿山遣韓朝陽召循計事,因縊殺之。燧走西山,間道歸平原。平原不守,復走魏。



 寶應中,澤潞節度使李抱玉署為趙城尉。時回紇還國,恃功恣睢,所過皆剽傷,州縣供餼不稱,輒殺人。抱玉將饋勞,賓介無敢往,燧自請典辦具。乃先賂其酋與約,得其旗章為信,犯令者得殺之。燧又取死囚給役左右,小違令輒戮死,虜大駭,至出境,無敢暴者。抱玉才之。因進說曰:「屬與回紇接,且得其情。觀僕固懷恩樹黨自重,裂河北以授李懷仙、張忠志、薛嵩、田承嗣等,其子瑒佻勇不義,將必窺太原,公當備之。」既而懷恩與太原將謀舉其城,辛云京覺之,不克。嵩自相、衛歸懷恩糧,以絕河津。抱玉令燧說嵩,嵩告絕於懷恩。即署燧左武衛兵曹參軍。



 累進至鄭州刺史。勸督農力,歲一稅,人以為便。徙懷州。時師旅後,歲大旱,田茀不及耕。燧務勤教化,止橫調。將吏有親者,必造之,厚為禮。瘞暴胔,止煩苛。是秋,穭生於境,人賴以濟。抱玉守鳳翔,表燧隴州刺史。西山直吐蕃,其上有通道,虜常所出入者。燧聚石種樹障之,設二門為譙櫓,八日而畢,虜不能暴。從抱玉入朝,代宗雅聞其才,召見,授商州刺史,兼水陸轉運使。



 大歷中,河陽兵逐其將常休明,詔燧檢校左散騎常侍,為三城使。汴將李靈耀反,帝務息人,即授以汴宋節度留後,靈耀不拜,引魏博田承嗣為援。詔燧與淮西李忠臣討之。師次鄭,靈耀多張旗幟以犯王師,忠臣之兵潰而西。燧軍頓熒澤,鄭人震駭。忠臣將遂歸,燧止之,益治軍,忠臣乃還收亡卒,復振。忠臣行汴南,燧行汴北,敗賊於西梁固。靈耀以銳卒八千,號「餓狼軍」,燧獨戰破之,進至浚儀。是時河陽兵冠諸軍,田悅帥眾二萬助靈耀,破永平將杜如江等,乘勝距汴一舍而屯。忠臣合諸軍戰不利,燧為奇兵擊之,悅單騎遁,汴州平。



 燧知忠臣暴傲,讓其功,出舍板橋。忠臣入汴,果因會擊殺宋州刺史李僧惠。燧還河陽。秋大雨,河溢,軍吏請具舟以避,燧曰:「使城中盡魚而獨完其家,吾不忍。」既而水不為害。



 遷河東節度留後,進節度使。太原承鮑防之敗,兵力衰單,燧募廝役,得數千人,悉補騎士,教之戰,數月成精卒。造鎧必短長三制,稱士所衣,以便進趨。為戰車,冒以狻猊象,列戟於後,行以載兵,止則為陣,遇險則制沖冒。器用完銳。居一年,闢廣場,羅兵三萬以肄,威震北方。建中二年,朝京師,遷檢校兵部尚書,封豳國公,還軍。



 初,田悅新有魏博,恐下未附,即輸款朝廷,燧建言悅必反。既而悅果圍邢州,身攻臨洺,築重城絕內外援。邢將李洪、臨洺將張伾固守。詔燧以步騎二萬與昭義李抱真、神策兵馬使李晟合軍救之。燧出郭口,未過險,移書抵悅,示之好。悅以燧畏己,大喜。既次邯鄲,悅使至,燧皆斬之,遣兵破其支軍,射殺賊將成炫之。悅聞,使大將楊朝光以兵萬人據雙岡,築東西二柵以御燧。燧率軍營二壘間。是夜,東壘遁,燧進營狗明山,取棄壘置輜重。悅計曰:「朝光堅柵,且萬人,雖燧能攻,未可以數日下,且殺傷必眾,則吾已拔臨洺,饗士以戰,必勝術也。」即分恆州兵五千助朝光。燧令大將李自良等以騎兵守雙岡,戒曰:「令悅得過者斬!」燧乃推火車焚朝光柵,自晨及晡,急擊,大破之,斬朝光,禽其將盧子昌,獲首五千,執八百人。居五日,進軍臨洺。悅悉軍戰,燧自以銳士當之,凡百餘返,士皆決死,悅大敗,斬首萬級,俘系千餘,館穀三十萬斛,邢圍亦解。以功遷尚書右僕射。初,將戰,燧約眾,勝則以家貲賞。至是,殫私財賜麾下。德宗嘉之,詔出度支錢五千萬償其財。進兼魏博招討使。



 李納、李惟岳合兵萬三千人救悅,悅裒散兵二萬壁洹水,淄青軍其左,恆冀軍其右。燧進屯鄴,請益兵。詔河陽李芃以兵會,次於漳。悅遣將王光進以兵守漳之長橋,築月壘扼軍路。燧於下流以鐵鏁維車數百絕河,載土囊遏水而後度。悅知燧食乏,深壁不戰。燧令士齎十日糧,進營倉口,與悅夾洹而軍,造三橋逾洹,日挑戰。悅不出,陰伏萬人,將以掩燧。燧令諸軍夜半食,先雞鳴時鳴鼓角,而潛師並洹趨魏州,令曰:「聞賊至,止為陣。」留百騎持火,待軍畢發,匿其旁,須悅眾度,即焚橋。燧行十餘里,悅率李納等兵逾橋,乘風縱火,噪而前。燧乃令士無動,命除榛莽廣百步為場,募勇士五千人陣而待。比悅至,火止,氣少衰,燧縱兵擊之,悅敗走橋,橋已焚,眾赴水死者不可計,斬首二萬級,殺賊將孫晉卿、安墨啜,虜三千人,尸相駘藉三十里,淄青兵幾殲。悅夜走魏州,其將拒不納,比明,追不至,悅乃得入。



 抱真、芃問曰:「糧少而深入,何也?」燧曰:「糧少戰利速,兵善於致人。今悅與淄青、恆三軍為首尾,欲不戰以老我師。若分擊左右,未可必破,悅且來助,是腹背支敵也。法有攻其必救,故趨魏以破之。」皆曰:「善。」



 悅嬰城自守。於是李再春以博州、悅兄昂以洛州、王光進以長橋皆降。悅使符璘、李瑤衛還淄青殘兵,璘等亦降。魏導禦溝貫城,燧塞其上游,魏人恐。悅遣許士則、侯臧間行告窮於硃滔、王武俊,會二人者怨望,乃連和。悅恃燕、趙方至,即出兵背城陣,燧復與諸軍破之。進同中書門下平章事、北平郡王、魏州大都督長史。



 滔、武俊聯兵五萬傅魏。會帝遣李懷光以朔方軍萬五千助燧。懷光勇於斗,未休士,即與滔等戰,不利。悅決水灌軍,燧兵亦屈,退保魏縣。滔等瀕河為壘。會涇師亂,帝幸奉天,燧還軍太原。



 初,李抱真欲殺懷州刺史楊鉥,鉥奔燧,燧奏其非罪,乃免。抱真怒。及共解邢圍,獲軍糧,燧自有之,以餘給抱真軍,抱真益怒。洹之捷,軍進薄魏,悅以突騎犯燧營,李芃救之,抱真勒兵不出。燧將攻魏,取攻具於抱真營,並請雜兩軍平其功,抱真不聽,請獨當一面,繇是逗遛。帝數遣使講解。武俊略趙地,抱真分麾下二千人戍邢,燧怒謂:「抱真以兵還守其地,我能獨戰死邪?」將引還,李晟和之,乃復與抱真善。及田昂降,燧請以洺州隸抱真,而用昭義副使盧玄卿為刺史,兼魏博招討副使。李晟兵前獨隸抱真,抱真亦請兼隸於燧,以示協一。然議者咎燧私忿交惡,卒不成大功。



 至太原,遣軍司馬王權以兵五千走奉天,又遣子匯與諸將子壁中渭橋,帝已幸梁,乃還。時天下方騷,北邊數有警,燧念晉陽王業所基,宜固險以示敵。乃引晉水架汾而屬之城,瀦為東隍,省守陴萬人。又釃汾環城,樹以固堤。詔兼保寧軍節度使。



 帝還京,李懷光反河中,詔燧為河東保寧、奉誠軍行營副元帥,與渾瑊、駱元光合兵討之。時賊黨要廷珍守晉、毛朝易又守隰、鄭抗守慈,燧移檄鐫諭,皆以州降,因拜燧晉絳慈隰節度使。



 武俊之圍趙也,康日知不支,將棄趙,燧請詔武俊擊硃滔,授以深、趙,以日知為晉慈隰節度使。及三州降,燧固讓日知,且言因降受節,恐後有功者踵以為利,帝嘉許。籍府庫兵仗以授日知,日知大喜過望。燧乃率步騎三萬次於絳,略定諸縣,降其將馮萬興、任象玉,遂圍絳,拔外郛,守將夜棄城去,降四千人。遣李自良定六縣,降其將辛兟,收卒五千。裨將穀秀違令掠士女,斬以徇。與賊戰寶鼎,射殺賊將徐伯文,斬首萬級,獲馬五百。



 於時天下蝗,兵艱食,物貨翔踴,中朝臣多請宥懷光者,帝未決。燧以「懷光逆計久,反覆不可信。河中近甸,舍之屈威靈,無以示天下,」乃舍軍入朝,為天子自言之:「且得三十日糧,足平河中。」許之。乃與瑊、元光、韓游瑰之兵合。



 賊將徐廷光守長春宮城。燧度長春不下,則懷光固守,久攻所傷必眾,乃挺身至城下見廷光。廷光憚燧威,拜城上。燧顧其心已屈,徐曰:「我自朝廷來,可西向受命。」廷光再拜。燧曰:「公等朔方士,自祿山以來,功高天下,奈何棄之為族滅計?若從吾言,非止免禍,富貴可遂也。」未對,燧曰:「爾以吾為欺邪?今不遠數步,可射我。」披而示之心。廷光感泣,一軍皆流涕,即率眾降。燧以數騎入其城,眾大呼曰:「吾等更為王人矣!」渾瑊亦自以為不及也,嘆曰:「嘗疑馬公能窘田悅,今觀其制敵,固有過人者,吾不逮遠矣!」



 進營焦籬堡,堡將降,餘戍望風遁去。燧濟河,兵八萬陣城下。是日,賊將牛名俊斬懷光降,眾猶萬六千。誅其黨閻晏、孟寶、張清、吳冏等,它脅附悉赦之。不閱月,河中平。遷光祿大夫,兼侍中,賜一子五品官。還太原,帝賜《宸扆》、《臺衡》二銘,以言君臣相成之美。勒石起義堂,帝榜其顏以寵之。



 貞元二年,吐蕃尚結贊破鹽、夏二州,守之,自屯鳴沙。及春,畜產死,糧乏。詔燧為綏銀麟勝招討使,與駱元光、韓游瑰等會師擊虜。燧次石州。結贊懼,乞盟,帝不許。乃遣將論頰熱甘辭請於燧,且重幣申勤勤。明年,燧還太原,與論頰熱俱朝,盛言宜許以盟,天子然之。燧之朝,結贊遽引去。帝詔渾瑊與盟平涼,虜劫瑊,僅得免。吐蕃歸燧之兄子弇,曰:「河曲之屯,春草未生,吾馬饑,公若度河,我無種矣。賴公許和,今釋弇以報。」帝聞,悔怒,奪其兵,拜司徒,兼侍中,賜妓樂,奉朝請而已。與李晟皆圖象凌煙閣。後病足,不任謁。九年十月,自力朝延英,詔毋拜。時晟已卒,帝顧燧曰:「尚記與太尉晟俱來邪?今乃獨見公。」因悲涕。燧亦疾而僕,帝親掖之,詔左右扶去,送至陛,燧頓首泣謝。固乞骸,讓侍中,不許。卒,年七十,贈太傅,謚曰莊武。子匯、暢。



 暢少以廕至鴻臚少卿。建中中,燧討賊山東,暢留京師。於是大旱,朝廷議括商旅緡錢,多亡命入南山為盜。暢客單超俊、李雲端等竊議,以為事且危。暢是其言,遣奴諫燧班師。燧怒,執奴以聞,使兄炫拘暢請罪。帝方倚燧,貸不問,但誅其客,敕炫賜暢杖三十,然亦罷括商人令。燧沒後,以貲甲天下,暢亦善殖財,家益豐。晚為豪幸牟侵,又匯妻訟析產。貞元末,神策中尉楊志廉諷使納田產。至順宗時,復賜之。中官往往逼取,暢畏不敢吝,以至困窮。終少府監,贈工部尚書。諸子無室廬自托。奉誠園亭觀,即其安邑里舊第云,故當世視暢以厚畜為戒。有司謚曰縱。



 子繼祖,生四歲以門功為太子舍人,五遷至殿中少監。



 燧兄炫,字弱翁。少以儒學聞,隱蘇門山,不應闢召。至德中,李光弼鎮太原,始署掌書記,常參軍謀,光弼器焉。刑部郎中田神功帥宣武,署節度判官,授連、潤二州刺史,以清白顯。燧為司徒,授刑部侍郎,辭疾,以兵部尚書致仕,卒。



 渾瑊,本鐵勒九姓之渾部也。世為皋蘭都督。父釋之,有才武,從朔方軍,積戰多,遷累開府儀同三司、試太常卿、寧朔郡王。廣德中與吐蕃戰沒。



 瑊年十一,善騎射,隨釋之防秋,朔方節度使張齊丘戲曰:「與乳媼俱來邪?」是歲立跳蕩功。後二年,從破賀魯部,拔石堡城、龍駒島,其勇常冠軍。署折沖果毅。節度使安思順授瑊偏師,入葛祿部,略特羅斯山,破阿布思,與諸軍城永清及天安軍。遷中郎將。



 祿山反,從李光弼定河北,射賊驍將李立節,貫其左肩,死之。肅宗即位,瑊以兵趨行在。至天德,與虜軍遇,敗之。從郭子儀復兩京,討安慶緒,勝之新鄉,擢武鋒軍使。從僕固懷恩平史朝義,大小數十戰,功最,改太常卿,實封二百戶。懷恩反,瑊以所部歸子儀,會釋之喪,起復朔方行營兵馬使。從子儀擊吐蕃邠州,留屯邠。虜復入,至奉天,瑊戰漠谷,有功,遷太子賓客,屯奉天。周智光反,子儀令瑊以步騎萬人下同州。智光平,以邠寧隸朔方軍,瑊屯宜祿。



 大歷七年,吐蕃盜塞深入,瑊會涇原節度使馬璘討之。次黃菩原,瑊引眾據險,設槍壘自營,遏賊奔突。舊將史抗等內輕瑊,顧左右去槍,叱騎馳賊。既還,虜躡而入,遂大敗,死者十八。子儀召諸將曰:「朔方軍高天下,今敗於虜,奈何?」瑊曰:「願再戰。」乃馳朝那,與鹽州刺史李國臣趨秦原。吐蕃引去,瑊邀擊破之,悉奪所掠而還。自是歲防長武城盛秋,領邠州刺史。吐蕃入方渠、懷安,瑊擊走之。



 子儀入朝,留知邠寧慶兵馬後務。回紇侵太原,破鮑防軍。拜瑊都知兵馬使,自石嶺關而南,督諸軍掎角,虜引去。進兼單于副都護、振武軍使。子儀為太尉,德宗析所部為三節度,以瑊兼單于大都護,振武、東受降城、鎮北大都護府、綏銀麟勝州節度副大使。未幾,崔寧領朔方,故召為左金吾衛大將軍。建中中,李希烈詐為瑊書,若同亂者,帝識其諜,用不疑,更賜良馬、錦幣。普王為荊襄元帥討希烈也,以瑊為中軍都虞候。



 帝狩奉天,瑊率家人子弟以從,授行在都虞候、京畿渭北節度使。硃泚兵薄城,戰譙門,晨至日中不解。或以芻車至,瑊曳車塞門,焚以戰,賊乃解。泚治攻具,矢石四集如雨,晝夜不息,凡浹日,鑿塹圜城。城中死者可藉,人心危惴,或夜縋出掇蔬本供御,帝與瑊相泣。泚方據乾陵下瞰城,翠翟紅袍,左右宦人趨走,宴賜拜舞,又縱慢辭戲斥天子,以為勝在景刻。使騎環馳,責大臣不識天命。造雲梁,廣數十丈,施大輪,濡氈及革冒之,周布水囊為鄣,指城東北;構木廬,蒙革周置之,運薪土其下,將塞隍。帝召瑊,授以詔書千餘,自御史大夫、實封五百戶而下,募突將死士當賊,賜瑊筆,使量功署詔,不足則署衣以授,因曰:「朕與公訣矣,令馬承倩往,有急可奏。」瑊俯伏嗚咽,帝撫而遣之。瑊前與防城使侯仲莊揣云梁所道,掘大隧,積馬矢及薪然之。賊乘風推梁以進,載數千人。王師乘城者皆凍餒,甲弊兵盬,瑊但以忠義感率使當賊,人憂不支,群臣號天以禱。瑊中矢,自揠去,被血而戰愈厲。雲梁及隧而陷,風返悉焚,賊皆死,舉城歡噪。是日詔授瑊二子官,乃第賞將校。泚攻城益急,會李懷光奔難,賊乃去。進行在都知兵馬使,實封五百戶。



 乘輿進狩山南,瑊以諸軍衛入谷口,懷光追騎至,後軍擊卻之。遷檢校尚書左僕射、同中書門下平章事,兼靈鹽豐夏定遠西城天德軍節度、朔方邠寧振武道永平軍奉天行營副元帥。帝臨軒授鉞,用漢拜韓信故事,制曰:「寇賊干紀,授爾節鉞,以戡多難,往欽哉!」瑊頓首曰:「敢不畢力以對揚天子休命?」乃率諸軍趨京師。



 賊韓旻拒武功,瑊率吐蕃論莽羅兵破之武亭川,斬首萬級,遂屯奉天,以抗西面。李晟自東渭橋破賊,瑊與韓游瑰、戴休顏以西軍收咸陽,進屯延秋門。泚平,論功,以瑊兼侍中,實封戶八百。天子還宮,授河中絳慈隰節度使、河中同陜虢行營副元帥,繇樓煩郡王徙咸寧;賜大寧里甲第,女樂五人,將相送歸第,與李晟鈞禮。俄加朔方行營副元帥,與馬燧同討李懷光。懷光平,檢校司空,任一子五品官。還屯河中。



 吐蕃相尚結贊陷鹽、夏,陰窺京師,而畏瑊與李晟、馬燧,欲以計勝之。乃詭辭重禮,請燧講好,燧苦贊,帝乃詔約盟平涼川,以瑊為會盟使。為結贊所劫,副使崔漢衡以下皆陷,惟瑊得免。自奉天入朝,羸服待罪,詔釋之。會吐蕃復入盜,使瑊鎮奉天。虜罷,還河中。貞元四年,虜入涇、邠,授邠寧慶副元帥。進檢校司徒,兼中書令。十五年卒,年六十四。群臣奉慰延英,贈太師,謚曰忠武。喪車至自鎮,帝復廢朝。



 瑊好書,通《春秋》、《漢書》。嘗慕《司馬遷自敘》,著《行紀》一篇,其辭一不矜大。天性忠謹,功高而志益下,歲時貢奉,必躬閱視。每有賜予,下拜跽受,常若在帝前,世方之金日磾,故帝終始信待。貞元後,天子常恐籓侯生事,稍桀驁則姑息之,惟瑊有所奏論不盡從可,輒私喜曰:「上不疑我。」故治蒲十六年,常持軍,猜間不能入。君子賢之。本名日進,稍顯改焉。五子,鎬、金歲為達官。



 鎬謙謹,喜交士大夫,歷鄧、唐二州刺史,有政譽。元和中,延州沙陀部苦邊吏貪,震擾不安。李絳建言,宜選才職稱者為刺史。乃任鎬延州。會討王承宗,而義武節度使任迪簡病不能軍,以鎬將家可用,乃遷檢校右散騎常侍、義武軍節度副使,俄代迪簡為使。治兵頗有法,然短於計略,不持重。鎮、定二軍間不百里,鎬引兵壓鎮境而屯,距賊三十里,鼓角聲相聞。賊始亦畏,見鎬無斥候,乃潛師入定境,焚廥蓄,屠鄉聚,鎬軍遂搖。亦會中人督戰,乃出薄賊,大敗而還。詔以陳楚代之。時師饑凍,聞鎬方罷,遂亂,劫鎬之家,至裸辱。楚聞,馳入城,乃定。令軍中斂所剽歸鎬,以兵衛出之。貶韶州刺史。後代州刺史韓重華奏收鎬供軍金幣十餘萬,乃復貶循州。卒,贈工部尚書。



 金歲以廕補諸衛參軍,累擬至豐州刺史。坐贓七百萬,文宗以勛臣子,貶袁州司馬。還為袁王傅,至太子詹事。訓、注亂,或言金歲匿賈餗,為百騎所捕,苦辨乃免,然家為兵剽皆盡。文宗憐之,授少府監,遷殿中。宰相以瑊之裔,擢刺史,帝曰:「是豈可以牧民?念其父功,富之可也。」宰相言金歲嘗治郡有績,從之,拜壽州刺史。終諸衛大將軍。



 贊曰:唐史臣稱燧沈雄忠力,常先計後戰。每戰,親令於眾,無不感概用命,鬥必決死,未嘗折北,名蓋一時。然力能得田悅而不取,虜不可信而決信之,故河北三盜卒不臣,平涼大臣奔辱,燧之罪也。雖然,燧賢者也,天下以為可責故責之,不以功掩罪,亦不可以罪廢功。瑊親與結贊盟,不能料虜詐,但以如詔為恭,殆有猛志而無英才乎?李晟謂虜不可與盟,則燧、瑊固出晟下遠甚。功名大小,信其然乎!



\end{pinyinscope}