\article{志第六 禮樂六}

\begin{pinyinscope}

 二曰賓禮,以待四夷之君長與其使者。



 蕃國主來朝,遣使者迎勞。前一日,守宮設次於館門之外道右,南向。其日,使者就次,蕃主服其國服,立於東階下,西面。使者朝服出次,立於門西,東面;從者執束帛立於其南。有司出門,西面曰:「敢請事。」使者曰:「奉制勞某主。」稱其國名。有司入告,蕃主迎於門外之東,西面再拜,俱入。使者先升,立於西階上,執束帛者從升,立於其北,俱東向。蕃主乃升,立於東階上,西面。使者執幣曰:「有制。」蕃主將下拜,使者曰:「有後制,無下拜。」蕃主旋,北面再拜稽首。使者宣制,蕃主進受命,退,復位,以幣授左右,又再拜稽首。使者降,出立於門外之西,東面。蕃主送於門之外,西,止使者,揖以俱入,讓升,蕃主先升東階上,西面;使者升西階上,東面。蕃主以土物儐使者,使者再拜受。蕃主再拜送物,使者降,出,蕃主從出門外,皆如初。蕃主再拜送使者,還。蕃主入,鴻臚迎引詣朝堂,依方北面立,所司奏聞,舍人承敕出,稱「有敕」。蕃主再拜。宣勞,又再拜。乃就館。



 皇帝遣使戒蕃主見日,如勞禮。宣制曰:「某日,某主見。」蕃主拜,稽首。使者降,出,蕃主送。



 蕃主奉見。前一日,尚舍奉御設御幄於太極殿,南向;蕃主坐於西南,東向。守宮設次,太樂令展宮縣,設舉麾位於上下。鼓吹令設十二案,乘黃令陳車輅,尚輦奉御陳輿輦。典儀設蕃主立位於縣南道西,北面;蕃國諸官之位於其後,重行,北面西上,典儀位於縣之東北,贊者二人在南,差退,俱西面。諸衛各勒部,屯門列黃麾仗。所司迎引蕃主至承天門外就次。本司入奏,鈒戟近仗皆入。典儀帥贊者先入,就位。侍中版奏「請中嚴」。諸侍衛之官及符寶郎詣閣奉迎,蕃主及其屬各立於閣外西廂,東面。侍中版奏「外辦」。皇帝服通天冠、絳紗袍,乘輿以出。舍人引蕃主入門,《舒和》之樂作。典儀曰:「再拜。」蕃主再拜稽首。侍中承制降,詣蕃主西北,東面曰:「有制。」蕃主再拜稽首,乃宣制,又再拜稽首。侍中還奏,承制降勞,敕升座。蕃主再拜稽首,升座。侍中承制勞問,蕃主俯伏避席,將下拜,侍中承制曰:「無下拜。」蕃主復位,拜而對。侍中還奏,承制勞還館。蕃主降,復縣南位,再拜稽首。其官屬勞以舍人,與其主俱出。侍中奏「禮畢」。皇帝興。若蕃國遣使奉表幣,其勞及戒見皆如蕃國主。庭實陳於客前,中書侍郎受表置於案,至西階以表升。有司各率其屬受其幣焉。



 其宴蕃國主及其使,皆如見禮。皇帝已即御坐,蕃主入,其有獻物陳於其前。侍中承制降敕,蕃主升座。蕃主再拜奉贄,曰:「某國蕃臣某敢獻壤奠。」侍中升奏,承旨曰:「朕其受之。」侍中降於蕃主東北,西面,稱《有制》。蕃主再拜,乃宣制。又再拜以贄授侍中,以授有司。有司受其餘幣,俱以東。舍人承旨降敕就座,蕃國諸官俱再拜。應升殿者自西階,其不升殿者分別立於廊下席後。典儀曰:「就坐。」階下贊者承傳。皆就座。太樂令引歌者及琴瑟至階,脫履,升坐,其笙管者,就階間北面立。尚食奉御進酒,至階,典儀曰:「酒至,興。」階下贊者承傳,皆俯伏,興,立。殿中監及階省酒,尚食奉御進酒,皇帝舉酒,良醞令行酒。典儀曰:「再拜。」階下贊者承傳,皆再拜,受觶。皇帝初舉酒,登歌作《昭和》三終。尚食奉御受虛觶,奠於坫。酒三行,尚食奉御進食,典儀曰:「食至,興。」階下贊者承傳,皆興,立。殿中監及階省案,尚食奉御品嘗食,以次進,太官令行蕃主以下食案。典儀曰:「就坐。」階下贊者承傳,皆就坐。皇帝乃飯,蕃主以下皆飯。徹案,又行酒,遂設庶羞。二舞以次入,作。食畢,蕃主以下復位於縣南,皆再拜。若有筐篚,舍人前承旨降宣敕,蕃主以下又再拜,乃出。



 其三曰軍禮。



 皇帝親征。纂嚴。前期一日,有司設御幄於太極殿,南向。文武群官次於殿庭東西,每等異位,重行北向。乘黃令陳革輅以下車旗於庭。其日未明,諸衛勒所部,列黃麾仗。平明,侍臣、將帥、從行之官皆平巾幘、褲褶。留守之官公服,就次。上水五刻,侍中版奏「請中嚴」。鈒戟近仗列於庭。三刻,群官就位,諸侍臣詣閣奉迎。侍中版奏「外辦」。皇帝服武弁,御輿以出,即御座。典儀曰:「再拜。」在位者皆再拜。中書令承旨敕百寮群官出,侍中跪奏「禮畢。」皇帝入自東房,侍臣從至閣。



 乃示類於昊天上帝。前一日,皇帝清齋於太極殿,諸豫告之官、侍臣、軍將與在位者皆清齋一日。其日,皇帝服武弁,乘革輅,備大駕,至於壇所。其牲二及玉幣皆以蒼。尊以太尊、山罍各二,其獻一。皇帝已飲福,諸軍將升自東階,立於神座前,北向西上,飲福受胙。將軍之次在外壝南門之外道東,西向北上。其即事之位在縣南,北面,每等異位,重行西上。其奠玉帛、進熟、飲福、望燎,皆如南郊。



 其宜於社,造於廟,皆各如其禮而一獻。軍將飲福於太稷,廟則皇考之室。



 其凱旋,則陳俘馘於廟南門之外,軍實陳於其後。



 其解嚴,皇帝服通天冠、絳紗袍,君臣再拜以退,而無所詔。其餘皆如纂嚴。



 若祃於所征之地,則為壝再重,以熊席祀軒轅氏。兵部建兩旗於外壝南門之外,陳甲胄、弓矢於神位之側,植槊於其後。尊以牛羲、象、山罍各二,饌以特牲。皇帝服武弁,群臣戎服,三獻。其接神者皆如常祀,瘞而不燎。其軍將之位如示類。



 其犮於國門,右校委土於國門外為犮,又為瘞於神位西北,太祝布神位於犮前,南向。太官令帥宰人刳羊。郊社之屬設尊、罍、篚、冪於神左,俱右向;置幣於尊所。皇帝將至,太祝立於罍、洗東南,西向再拜,取幣進,跪奠於神。進饌者薦脯棨,加羊於犮西首。太祝盥手洗爵,酌酒進,跪奠於神,興,少退,北向立,讀祝。太祝再拜。少頃,帥齋郎奉幣、爵、酒饌,宰人舉羊肆解之,太祝並載,埋於臽。執尊者徹罍、篚、席、駕至,權停。太祝以爵酌酒,授太僕卿,左並轡,右受酒,祭兩軹及軌前,乃飲,授爵,駕轢犮而行。



 其所過山川,遣官告,以一獻。若遣將出征,則皆有司行事。



 賊平而宣露布。其日,守宮量設群官次。露布至,兵部侍郎奉以奏聞,承制集文武群官、客使於東朝堂,各服其服。奉禮設版位於其前,近南,文東武西,重行北向。又設客使之位。設中書令位於群官之北,南面。吏部、兵部贊群官、客使,謁者引就位。中書令受露布置於案。令史二人絳公服。對舉之以從。中書令出,就南面位,持桉者立於西南,東面。中書令取露布,稱「有制」。群官、客使皆再拜。遂宣之,又再拜,舞蹈,又再拜。兵部尚書進受露布,退復位,兵部侍郎前受之。中書令入,群官、客使各還次。



 仲冬之月,講武於都外。



 前期十有一日,所司奏請講武。兵部承詔,遂命將帥簡軍士,除地為場,方一千二百步,四出為和門。又為步、騎六軍營域,左右廂各為三軍,北上。中間相去三百步,立五表,表間五十步,為二軍進止之節。別墠地於北廂,南向。前三日,尚舍奉御設大次於墠。前一日,講武將帥及士卒集於墠所,建旗為和門,如方色。都墠之中及四角皆建五採牙旗、旗鼓甲仗。大將以下,各有統帥。大將被甲乘馬,教習士眾。少者在前,長者在後。其還,則反之。長者持弓矢,短者持戈矛,力者持旌,勇者持鉦、鼓、刀、楯為前行,持槊者次之,弓箭者為後。使其習見旌旗、金鼓之節。旗臥則跪,旗舉則起。



 講武之日,未明十刻而嚴,五刻而甲,步軍為直陣以俟,大將立旗鼓之下。六軍各鼓十二、鉦一、大角四。未明七刻,鼓一嚴,侍中奏「開宮殿門及城門」。五刻,再嚴,侍中版奏「請中嚴」。文武官應從者俱先置,文武官皆公服,所司為小駕。二刻,三嚴,諸衛各督其隊與鈒戟以次入,陳於殿庭。皇帝乘革輅至單所,兵部尚書介胄乘馬奉引,入自北門,至兩步軍之北,南向。黃門侍郎請降輅。乃入大次。兵部尚書停於東廂,西向。領軍減小駕,騎士立於都墠之四周,侍臣左右立於大次之前,北上。九品以上皆公服,東、西在侍臣之外十步所,重行北上。諸州使人及蕃客先集於北門外,東方、南方立於道東,西方、北方立於道西,北上。駕將至,奉禮曰:「再拜」。在位者皆再拜。皇帝入次,謁者引諸州使人,鴻臚引蕃客,東方、南方立於大次東北,西方、北方立於西北,觀者立於都墠騎士仗外四周,然後講武。



 吹大角三通,中軍將各以鞞令鼓,二軍俱擊鼓。三鼓,有司偃旗,步士皆跪。諸帥果毅以上,各集於其中軍。左廂中軍大將立於旗鼓之東,西面,諸軍將立於其南;右廂中軍大將立於旗鼓之西,東面,諸軍將立於其南。北面,以聽大將誓。左右三軍各長史二人,振鐸分循,諸果毅各以誓詞告其所部。遂聲鼓,有司舉旗,士眾皆起行,及表,擊鉦,乃止。又擊三鼓,有司偃旗,士眾皆跪。又擊鼓,有司舉旗,士眾皆起,驟及表,乃止。東軍一鼓,舉青旗為直陣;西軍亦鼓,舉白旗為方陣以應。次西軍鼓,舉赤旗為銳陣;東軍亦鼓,舉黑旗為曲陣以應。次東軍鼓,舉黃旗為圓陣;西軍亦鼓,舉青旗為直陣以應。次西軍鼓,舉白旗為方陣;東軍亦鼓,舉赤旗為銳陣以應。次東軍鼓,舉黑旗為曲陣;西軍亦鼓,舉黃旗為圓陣以應。



 凡陣,先舉者為客,後舉者為主。每變陣,二軍各選刀、楯五十人挑戰,第一、第二挑戰迭為勇怯之狀,第三挑戰為敵均之勢,第四、第五挑戰為勝敗之形。每將變陣,先鼓而直陣,然後變從餘陣之法。既已,兩軍俱為直陣。又擊三鼓,有司偃旗,士眾皆跪。又聲鼓舉旗,士眾皆起,騎馳、徒走,左右軍俱至中表,相擬擊而還。每退至一行表,跪起如前,遂復其初。侍中跪奏「請觀騎軍」,承制曰:「可。」二軍騎軍皆如步軍之法,每陣各八騎挑戰,五陣畢,大擊鼓而前,盤馬相擬擊而罷。遂振旅。侍中跪奏稱:「侍中臣某言,禮畢。」乃還。



 皇帝狩田之禮,亦以仲冬。



 前期,兵部集眾庶脩田法,虞部表所田之野,建旗於其後。前一日,諸針帥士集於旗下。質明,弊旗,後至者罰。兵部申田令,遂圍田。其兩翼之將皆建旗。及夜,布圍,闕其南面。駕至田所,皇帝鼓行入圍,鼓吹令以鼓六十陳於皇帝東南,西向;六十陳於西南,東向。皆乘馬,各備簫角。諸將皆鼓行圍。乃設驅逆之騎。皇帝乘馬南向,有司斂大綏以從。諸公、王以下皆乘馬,帶弓矢,陳於前後。所司之屬又斂小綏以從。乃驅獸出前。初,一驅過,有司整飭弓矢以前。再驅過,有司奉進弓矢。三驅過,皇帝乃從禽左而射之。每驅必三獸以上。皇帝發,抗大綏,然後公、王發,抗小綏。驅逆之騎止,然後百姓獵。



 凡射獸,自左而射之,達於右腢為上射,達右耳本為次射,左髀達於右泬為下射。群獸相從不盡殺,已被射者不重射。不射其面,不翦其毛。凡出表者不逐之。田將止,虞部建旗於田內,乃雷擊駕鼓及諸將之鼓,士從躁呼。諸得禽獻旗下,致其左耳。大獸公之,小獸私之。其上者供宗廟,次者供賓客,下者充皰廚。乃命有司饁獸於四郊,以獸告至於廟社。



 射。



 前一日,太樂令設宮縣之樂,鼓吹令設十二案於射殿之庭,東面縣在東階東,西面縣在西階西。南北二縣及登歌廣開中央,避射位。張熊侯去殿九十步,設乏於侯西十步、北十步。設五楅庭前,少西。布侍射者位於西階前,東面北上。布司馬位於侍射位之南,東面。布獲者位乏東,東面。布侍射者射位於殿階下,當前少西,橫布,南面。侍射者弓矢俟於西門外。陳賞物於東階下,少東。置罰豐於西階下,少西。設罰尊於西階,南北以殿深。設篚於尊西,南肆,實爵加冪。



 其日質明,皇帝服武弁,文武官俱公服,典謁引入見,樂作,如元會之儀。酒二遍,侍中一人奏稱:「有司謹具,請射。」侍中一人前承制,退稱:「制曰可。」王、公以下皆降。文官立於東階下,西面北上。武官立於西階下。於射乏後,東面北上。持鈒沄群立於兩邊,千牛備身二人奉御弓及矢立於東階上,西面,執弓者在北。又設坫於執弓者之前,又置御決、拾笥於其上。獲者持旌自乏南行,當侯東,行至侯,負侯北面立。侍射者出西門外,取弓矢,兩手奉弓,搢乘矢帶,入,立於殿下射位西,東面。司馬奉弓自西階升,當西楹前,南面,揮弓,命獲者以旌去侯西行十步,北行至乏止。司馬降自西階,復位。千牛中郎一人奉決、拾以笥,千牛將軍奉弓,千牛郎將奉矢,進,立於御榻東少南,西向。郎將跪奠笥於御榻前,少東。遂拂以巾,取決,興。贊設決。又跪取拾,興,贊設拾。以笥退,奠於坫。千牛將軍北面張弓,以袂順左右隈,上再下一,弓左右隈,謂弓面上下。以衣袂摩拭上面再,下面一。西面,左執付、右執簫以進。千牛郎將以巾拂矢進,一一供御。欲射,協律郎舉麾,先奏鼓吹,及奏樂《騶虞》五節,御及射,第一矢與第六節相應,第二矢與第七節相應,以至九節。協律郎偃麾,樂止。千牛將軍以矢行奏,中曰「獲」,下曰「留」,上曰「揚」,左曰「左方」,右曰「右方」。留,謂矢短不及侯;揚,謂矢過侯;左、右,謂矢偏不正。千牛將軍於御座東,西面受弓,退,付千牛於東階上。千牛郎將以笥受決、拾,奠於坫。



 侍射者進,升射席北面立,左旋,東面張弓,南面挾矢。協律郎舉麾,乃作樂,不作鼓吹。樂奏《貍首》三節,然後發矢。若侍射者多,則齊發。第一發與第四節相應,第二發與第五節相應,以至七節。協律郎偃麾,樂止。弓右旋,東西弛弓,如面立,乃退,復西階下立。司馬升自西階,自西楹前,南面,揮弓,命取矢。取矢者以禦矢付千牛於東階下,侍射者釋弓於位,庭前北面東上。有司奏請賞罰,侍中稱:「制曰可。」有司立楅之西,東面,監唱射矢。取矢者各唱中者姓名。中者立於東階下,西面北上;不中者立於西階下,東面北上。俱再拜。有司於東階下以付賞物。酌者於罰尊西,東面,跪,奠爵於豐上。不中者進豐南,北面跪,取爵,立飲,卒爵,奠豐下。酌者北面跪,取虛爵酌奠,不中者以次繼飲,皆如初。典謁引王公以下及侍射者,皆庭前北面相對為首,再拜訖,引出。持鈒隊復位。皇帝入,奏樂,警蹕。有司以弓矢出中門外,侍射者出。若特射無侍射之人,則不設楅,不陳賞罰。若燕游小射,則常服,不陳樂縣,不行會禮。



 合朔伐鼓。



 其日前二刻,郊社令及門僕赤幘絳衣,守四門,令巡門監察。鼓吹令平巾幘、褲褶,帥工人以方色執麾旒,分置四門屋下,設龍蛇鼓於右。東門者立於北塾,南面;南門者立於東塾,西面;西門者立於南塾,北面;北門者立於西塾,東面。隊正一人平巾幘、褲褶,執刀,帥衛士五人執五兵立於鼓外,矛在東,戟在南,斧、鉞在西,槊在北。郊社令立於社壇四隅,以硃絲繩縈之。太史一人赤幘、赤衣,立於社壇北,向日觀變。黃麾次之;龍鼓一次之,在北;弓一、矢四次之。諸兵鼓立候變。日有變,史官曰:「祥有變。」工人舉麾,龍鼓發聲如雷。史官曰:「止。」乃止。



 其日,皇帝素服,避正殿,百官廢務,自府史以上皆素服,各於其廳事之前,重行,每等異位,向日立。明復而止。



 貞元三年八月,日有食之,有司將伐鼓,德宗不許。太常卿董晉言:「伐鼓所以責陰而助陽也,請聽有司依經伐鼓。」不報,由是其禮遂廢。



 大儺之禮。



 選人年十二以上、十六以下為侲子,假面,赤布褲褶。二十四人為一隊,六人為列。執事十二人,赤幘、赤衣,麻鞭。工人二十二人,其一人方相氏,假面,黃金四目,蒙熊皮,黑衣、硃裳,右執楯;其一人為唱帥,假面,皮衣,執棒;鼓、角各十,合為一隊。隊別鼓吹令一人、太卜令一人,各監所部;巫師二人。以逐惡鬼於禁中。有司預備每門雄雞及酒,擬於宮城正門、皇城諸門磔攘,設祭。太祝一人,齋郎三人,右校為瘞臽,各於皇城中門外之右。前一日之夕,儺者赴集所,具其器服以待事。



 其日未明,諸衛依時刻勒所部,屯門列仗,近仗入陳於階。鼓吹令帥儺者各集於宮門外。內侍詣皇帝所御殿前奏「侲子備,請逐疫」。出,命寺伯六人,分引儺者於長樂門、永安門以入,至左右上閣,鼓噪以進。方相氏執戈揚楯唱,侲子和,曰:「甲作食兇,胇胃食虎,雄伯食魅,騰簡食不祥,攬諸食咎,伯奇食夢,強梁、祖明共食磔死寄生,委隋食觀,錯斷食巨,窮奇、騰根共食蠱,凡使一十二神追惡兇,赫汝軀,拉汝幹,節解汝肉,抽汝肺腸,汝不急去,後者為糧。」周呼訖,前後鼓噪而出,諸隊各趨順天門以出,分詣諸城門,出郭而止。



 儺者將出,祝布神席,當中門地南向。出訖,宰手、齋郎牲匈磔之神席之西,藉以席,北首。齋郎酌清酒,太祝受,奠之。祝史持版於座右,跪讀祝文曰:「維某年歲次月朔日,天子遣太祝臣姓名昭告於太陰之神。」興,尊版於席,乃舉牲並酒瘞於臽。



\end{pinyinscope}