\article{列傳第一 後妃上}

\begin{pinyinscope}

 太穆竇皇后文德長孫皇后徐賢妃王皇后則天武皇后和思趙皇后韋皇后上官昭容肅明劉皇后昭成竇皇后王皇后貞順武皇后元獻楊皇后楊貴妃



 唐制:皇后而下,有貴妃、淑妃、德妃、賢妃,是為夫人。昭儀、昭容、昭媛、修儀、修容、修媛、充儀、充容、充媛,是為九嬪。婕妤、美人、才人各九,合二十七,是代世婦。寶林、御女、採女各二十七,合八十一,是代御妻。自餘六尚,分典乘輿服御,皆有員次。後世改復不常。開元時,以後下復有四妃非是,乃置惠、麗、華三妃,六儀,四美人,七才人,而尚宮、尚儀、尚服各二,參合前號,大抵踵《周官》相損益云,然則尚矣。



 禮本夫婦,《詩》始後妃,治亂因之,興亡系焉。盛德之君,帷薄嚴奧,里謁不忓於朝,外言不內諸閫,《關雎》之風行,彤史之化修,故淑範懿行,更為內助。若夫艷嬖之興,常在中主。第裯既交,則情與愛遷;顏辭媚熟,則事為私奪。乘易昏之明,牽不斷之柔,險言似忠,故受而不詰,醜行已效,反狃而為好。左右附之,憸壬惎之,狡謀鉗其悟先,哀誓楗於寵初,天下之事已去,而恬不自覺,此武、韋所以遂篡弒而喪王室也。至於楊氏未死,玄亂厥謀;張後制中,肅幾斂衽。籲,可嘆哉!中葉以降,時多故矣,外有攻討之勤,內寡嬿溺之私,群閹朋進,外戚勢分,後妃無大善惡,取充職位而已,故列著於篇。



 高祖太穆順聖皇后竇氏,京兆平陵人。父毅,在周為上柱國,尚武帝姊襄陽長公主,入隋為定州總管、神武公。



 後生,發垂過頸,三歲與身等。讀《女誡》、《列女》等傳,一過輒不忘。武帝愛之,養宮中,異它甥。時突厥女為後,無寵,後密諫曰:「吾國未靖,虜且強,願抑情撫接,以取合從,則江南、關東不吾梗。」武帝嘉納。及崩,哀毀同所生。聞隋高祖受禪,自投床下,曰:「恨我非男子,不能救舅家禍。」毅遽掩其口,曰:「毋妄言,赤吾族!」常謂主曰:「此女有奇相,且識不凡,何可妄與人?」因畫二孔雀屏間,請昏者使射二矢,陰約中目則許之。射者閱數十,皆不合。高祖最後射,中各一目,遂歸於帝。



 始,元貞太后羸老有疾,而性素嚴,諸姒娣皆畏,莫敢侍。後事之,獨怡謹盡孝,或淹月不釋衣履。工為篇章規誡,文有雅體。又善書,與高祖書相雜,人不辨也。崩於涿郡,年四十五。



 帝在煬帝時,多畜善馬,後見曰:「上性樂此,盍以獻?徒留之速罪,無益也。」不聽,頃裡坐譴。帝後見隋政亂,多妄誅殛,乃為自安計,數奏鷹犬異駒,煬帝果喜,擢位將軍。因泣謂諸子曰:「早用而母言,得此久矣!」帝有天下,詔即所葬園為壽安陵,謚曰穆。及祔獻陵,尊為太穆皇后。



 始,太宗生,有二龍之符,後於諸子中愛視最篤。後即位,過慶善宮,覽觀梗欷,顧侍臣曰:「朕生於此,今母後永違,育我之德不可報。」因號慟,左右皆流涕。乃享后於正寢。它日幸九成宮,夢後若平生,既悟,潸然不自勝。明日,詔有司大發倉賑貧瘠,以為後報焉。上元中,益謚太穆神皇后。



 太宗文德順聖皇后長孫氏,河南洛陽人。其先魏拓拔氏,後為宗室長,因號長孫。高祖稚,大丞相、馮翊王。曾祖裕,平原公。祖兕,左將軍。父晟,字季,涉書史,趫〗鷙曉兵,仕隋為右驍衛將軍。



 後喜圖傳,視古善惡以自鑒,矜尚禮法。晟兄熾,為周通道館學士。嘗聞太穆勸撫突厥女,心志之。每語晟曰:「此明睿人,必有奇子,不可以不圖昏。」故晟以女太宗。後歸寧,舅高士廉妾見大馬二丈立後舍外,懼,占之,遇《坤》之《泰》。卜者曰:「《坤》順承天,載物無疆;馬,地類也;之《泰》,是天地交而萬物通也,又以輔相天地之宜。繇協《歸妹》,婦人事也。女處尊位,履中而居順,後妃象也。」時隱太子釁鬩已構,後內盡孝事高祖,謹承諸妃,消釋嫌猜。及帝授甲宮中,後親慰勉,士皆感奮。尋為皇太子妃,俄為皇后。



 性約素,服御取給則止。益觀書,雖容櫛不少廢。與帝言,或及天下事,辭曰:「牝雞司晨,家之窮也,可乎?」帝固要之,訖不對。後廷有被罪者,必助帝怒請繩治,俟意解,徐為開治,終不令有冤;下嬪生豫章公主而死,後視如所生;媵侍疾病,輟所御飲藥資之。下懷其仁。兄無忌,於帝本布衣交,以佐命為元功,出入臥內,帝將引以輔政,後固謂不可,乘間曰:「妾托體紫宮,尊貴已極,不願私親更據權於朝。漢之呂、霍,可以為誡。」帝不聽,自用無忌為尚書僕射。後密諭令牢讓,帝不獲已,乃聽,後喜見顏間。異母兄安業無行,父喪,逐後、無忌還外家。後貴,未嘗以為言。擢位將軍。後與李孝常等謀反,將誅,後叩頭曰:「安業罪死無赦。然向遇妾不以慈,戶知之;今論如法,人必謂妾釋憾於兄,無乃為帝累乎!」遂得減流越巂。太子承乾乳媼請增東宮什器,後曰:「太子患無德與名,器何請為?」



 從幸九成宮,方屬疾,會柴紹等急變聞,帝甲而起,後輿疾以從,宮司諫止,後曰:「上震驚,吾可自安?」疾稍亟,太子欲請大赦,汎度道人,祓塞災會。後曰:「死生有命,非人力所支。若修福可延,吾不為惡;使善無效,我尚何求?且赦令,國大事,佛、老異方教耳,皆上所不為,豈宜以吾亂天下法!」太子不敢奏,以告房玄齡,玄齡以聞,帝嗟美。而群臣請遂赦,帝既許,後固爭止。及大漸,與帝決,時玄齡小譴就第,後曰:「玄齡久事陛下,預奇計秘謀,非大故,願勿置也。妾家以恩澤進,無德而祿,易以取禍,無屬樞柄,以外戚奉朝請足矣。妾生無益於時,死不可以厚葬,願因山為壟,無起墳,無用棺槨,器以瓦木,約費送終,是妾不見忘也。」又請帝納忠容諫,勿受讒,省游畋作役,死無恨。崩,年三十六。



 後嘗採古婦人事著《女則》十篇,又為論斥漢之馬後不能檢抑外家,使與政事,乃戒其車馬之侈,此謂開本源,恤末事。常誡守者:「吾以自檢,故書無條理,勿令至尊見之。」及崩,宮司以聞,帝為之慟,示近臣曰:「後此書可用垂後,我豈不通天命而割情乎!顧內失吾良佐,哀不可已已!」謚曰文德,葬昭陵,因九嵕山,以成後志。帝自著表序始末,揭陵左。上元中,益謚文德聖皇后。



 太宗賢妃徐惠,湖州長城人。生五月能言,四歲通《論語》、《詩》,八歲自曉屬文。父孝德,嘗試使擬《離騷》為《小山篇》曰:「仰幽巖而流盼,撫桂枝以凝想。將千齡兮此遇,荃何為兮獨往?」孝德大驚,知不可掩,於是所論著遂盛傳。太宗聞之,召為才人。手未嘗廢卷,而辭致贍蔚,文無淹思。帝益禮顧,擢孝德水部員外郎,惠再遷充容。



 貞觀末,數調兵討定四夷,稍稍治宮室,百姓勞怨。惠上疏極諫,且言:「東戍遼海,西討昆丘,士馬罷耗,漕餉漂沒。捐有盡之農,趨無窮之壑;圖未獲之眾,喪已成之軍。故地廣者,非常安之術也;人勞者,為易亂之符也。」又言:「翠微、玉華等宮,雖因山藉水,無築構之苦,而工力和僦,不謂無煩。有道之君,以逸逸人;無道之君,以樂樂身。」又言:「伎巧為喪國斧斤,珠玉為蕩心鳩毒,侈麗纖美,不可以不遏。志驕於業泰,體逸於時安。」其剴切精詣,大略如此。帝善其言,優賜之。帝崩,哀慕成疾,不肯進藥,曰:「帝遇我厚,得先狗馬侍園寢,吾志也。」復為詩、連珠以見意。永徽元年卒,年二十四,贈賢妃,陪葬昭陵石室。



 惠之弟齊聃,齊聃子堅,皆以學聞,女弟為高宗婕妤,亦有文藻,世以擬漢班氏。



 高宗廢後王氏,並州祁人,魏尚書左僕射思政之孫。從祖母同安長公主以後婉淑,白太宗以為晉王妃。王居東宮,妃亦進冊,擢父仁祐陳州刺史。帝即位,立為皇后。仁祐以特進封魏國公;母柳,本國夫人。仁祐卒,贈司空。



 初,蕭良娣有寵,而武才人貞觀末以先帝宮人召為昭儀,俄與後、良娣爭寵,更相毀短。而昭儀詭險,即誣后與母挾媚道蠱上,帝信之,解魏國夫人門籍,罷後舅柳奭中書令。李義府等陰佐昭儀,以偏言怒帝,遂下詔廢后、良娣皆為庶人,囚宮中。後母兄、良娣宗族悉流嶺南。許敬宗又奏:「仁祐無他功,以宮掖故,超列三事,今庶人謀亂宗社,罪宜夷宗,仁祐應斫棺,陛下不窮其誅,家止流竄,仁祐不宜引庇廕宥逆子孫。」有詔盡奪仁祐官爵。而後及良娣俄為武后所殺,改後姓為「蟒」,良娣為「梟」。



 初,帝念後,間行至囚所,見門禁錮嚴,進飲食竇中,惻然傷之,呼曰:「皇后、良娣無恙乎?今安在?」二人同辭曰:「妾等以罪棄為婢,安得尊稱耶?」流淚嗚咽。又曰:「陛下幸念疇日,使妾死更生,復見日月,乞署此為『回心院』。」帝曰:「朕即有處置。」武後知之,促詔杖二人百,剔其手足,反接投釀甕中,曰:「令二嫗骨醉!」數日死,殊其尸。初,詔旨到,後再拜曰:「陛下萬年!昭儀承恩,死吾分也。」至良娣,罵曰:「武氏狐媚,翻覆至此!我後為貓,使武氏為鼠,吾當扼其喉以報。」後聞,詔六宮毋畜貓。武後頻見二人被發瀝血為厲,惡之,以巫祝解謝,即徙蓬萊宮,厲復見,故多駐東都。中宗即位,皆復其姓。



 高宗則天順聖皇后武氏,並州文水人。父士獲,見《外戚傳》。文德皇后崩,久之,太宗聞士獲女美,召為才人,方十四。母楊,慟泣與訣,後獨自如,曰:「見天子庸知非福,何兒女悲乎?」母韙其意,止泣。既見帝,賜號武媚。及帝崩,與嬪御皆為比丘尼。高宗為太子時,入侍,悅之。王後後久無子,蕭淑妃方幸,後陰不悅。它日,帝過佛廬,才人見且泣,帝感動。後廉知狀,引內後宮,以撓妃寵。



 才人有權數,詭變不窮。始,下辭降體事後,後喜,數譽於帝,故進為昭儀。一旦顧幸在蕭後,寢與後不協。後性簡重,不曲事上下,而母柳見內人尚宮無浮禮,故昭儀伺後所薄,必款結之,得賜予,盡以分遺。由是後及妃所為必得,得輒以聞,然未有以中也。昭儀生女,後就顧弄,去,昭儀潛斃兒衾下,伺帝至,陽為歡言,發衾視兒,死矣。又驚問左右,皆曰:「後適來。」昭儀即悲涕,帝不能察,怒曰:「後殺吾女,往與妃相讒媢,今又爾邪!」由是昭儀得入其訾,後無以自解,而帝愈信愛,始有廢後意。久之,欲進號「宸妃」,侍中韓瑗、中書令來濟言:「妃嬪有數,今別立號,不可。」昭儀乃誣后與母厭勝,帝挾前憾,實其言,將遂廢之。長孫無忌、褚遂良、韓瑗及濟瀕死固爭,帝猶豫;而中書舍人李義府、衛尉卿許敬宗素險側,狙勢即表請昭儀為後,帝意決,下詔廢後。詔李勣、於志寧奉璽綬進昭儀為皇后,命群臣及四夷酋長朝後肅義門,內外命婦入謁。朝皇后自此始。



 後見宗廟,再贈士獲至司徒,爵周國公,謚忠孝,配食高祖廟。母楊,再封代國夫人,家食魏千戶。後乃制《外戚誡》獻諸朝,解釋譏言喿。於是逐無忌、遂良,踵死徙,寵煽赫然。後城宇深,痛柔屈不恥,以就大事,帝謂能奉己,故扳公議立之。已得志,即盜威福,施施無憚避,帝亦儒昏,舉能鉗勒,使不得專,久稍不平。麟德初,後召方士郭行真入禁中為蠱祝,宦人王伏勝發之,帝怒,因是召西臺侍郎上官儀,儀指言後專恣,失海內望,不可承宗廟,與帝意合,乃趣使草詔廢之。左右馳告,後遽從帝自訴,帝羞縮,待之如初,猶意其恚,且曰:「是皆上官儀教我!」後諷許敬宗構儀,殺之。



 初,元舅大臣怫旨,不閱歲屠覆,道路目語,及儀見誅,則政婦房帷,天子拱手矣。群臣朝、四方奏章,皆曰「二聖」。每視朝,殿中垂簾,帝與後偶坐,生殺賞罰惟所命。當其忍斷,雖甚愛,不少隱也。帝晚益病風不支,天下事一付後。後乃更為太平文治事,大集諸儒內禁殿,撰定《列女傳》、《臣軌》、《百僚新誡》、《樂書》等,大氐千餘篇。因令學士密裁可奏議,分宰相權。



 始,士獲娶相里氏,生子元慶、元爽。又娶楊氏,生三女:伯嫁賀蘭越石,蚤寡,封韓國夫人;仲即後;季嫁郭孝慎,前死。楊以後故,寵日盛,徙封榮國。始,兄子惟良、懷運與元慶等遇楊及後禮薄,後銜不置。及是,元慶為宗正少卿,元爽少府少監,惟良司衛少卿,懷運淄州刺史。它日,夫人置酒,酣,謂惟良曰:「若等記疇日事乎?今謂何?」對曰:「幸以功臣子位朝廷,晚緣戚屬進,憂而不榮也。」夫人怒,諷後偽為退讓,請惟良等外遷,無示天下私。繇是,惟良為始州刺史;元慶,龍州;元爽,濠州,俄坐事死振州。元慶至州,憂死。韓國出入禁中,一女國姝,帝皆寵之。韓國卒,女封魏國夫人,欲以備嬪職,難於後,未決。後內忌甚,會封泰山,惟良、懷運以岳牧來集,從還京師,後毒殺魏國,歸罪惟良等,盡殺之,氏曰「蝮」,以韓國子敏之奉士〓祀。初,魏國卒,敏之入吊,帝為慟,敏之哭不對。後曰:「兒疑我!」惡之。俄貶死。楊氏徙酂、衛二國,咸亨元年卒,追封魯國,謚忠烈,詔文武九品以上及五等親與外命婦赴吊,以王禮葬咸陽,給班劍、葆杖、鼓吹。時天下旱,後偽表求避位,不許。俄又贈士獲太尉兼太子太師、太原郡王,魯國忠烈夫人為妃。



 上元元年,進號天後,建言十二事:一、勸農桑,薄賦徭;二、給復三輔地;三、息兵,以道德化天下;四、南北中尚禁浮巧;五、省功費力役;六、廣言路;七、杜讒口;八、王公以降皆習《老子》;九、父在為母服齊衰三年;十、上元前勛官已給告身者無追核;十一、京官八品以上益稟入;十二、百官任事久,材高位下者得進階申滯。帝皆下詔略施行之。



 蕭妃女義陽、宣城公主幽掖廷,幾四十不嫁,太子弘言於帝,後怒,酖殺弘。帝將下詔遜位於後,宰相郝處俊固諫,乃止。後欲外示寬裕,劫人心使歸已,即奏言:「今群臣納半俸、百姓計口錢以贍邊兵,恐四方妄商虛實,請一罷之。」詔可。



 儀鳳三年,群臣、蕃夷長朝後於光順門。即並州建太原郡王廟。帝頭眩不能視,侍醫張文仲、秦鳴鶴曰:「風上逆,砭頭血可愈。」後內幸帝殆,得自專,怒曰:「是可斬,帝體寧刺血處邪?」醫頓首請命。帝曰:「醫議疾,烏可罪?且吾眩不可堪,聽為之!」醫一再刺,帝曰:「吾目明矣!」言未畢,後簾中再拜謝,曰:「天賜我師!」身負繒寶以賜。



 帝崩,中宗即位,天后稱皇太后,遺詔軍國大務聽參決。嗣聖元年,太后廢帝為廬陵王,自臨朝,以睿宗即帝位。後坐武成殿,帝率群臣上號冊。越三日,太后臨軒,命禮部尚書攝太尉武承嗣、太常卿攝司空王德真冊嗣皇帝。自是太后常御紫宸殿,施慘紫帳臨朝。追贈五世祖後魏散騎常侍克己為魯國公,妣裴即其國為夫人;高祖齊殷州司馬居常為太尉、北平郡王,妣劉為王妃;曾祖永昌王諮議參軍、贈齊州刺史儉為太尉、金城郡王,妣宋為王妃;祖隋東郡丞、贈並州刺史、大都督華為太尉、太原郡王,妣趙為王妃。皆置園邑,戶五十。考為太師、魏王,加實戶滿五千,妣為王妃,王園邑守戶百。時睿宗雖立,實囚之,而諸武擅命。又謚魯國公曰靖,裴為靖夫人;北平郡王曰恭肅,金城郡王曰義康,太原郡王曰安成,妃從夫謚。太后遣冊武成殿使者告五世廟室。



 於是柳州司馬李敬業、括蒼令唐之奇、臨海丞駱賓王疾太后脅逐天子,不勝憤,乃募兵殺揚州大都督府長史陳敬之,據州欲迎廬陵王,眾至十萬。楚州司馬李崇福連和。盱眙人劉行舉嬰城不肯從,敬業攻之,不克。太后拜行舉游擊將軍,擢其弟行實楚州刺史。敬業南度江取潤州,殺刺史李思文,曲阿令尹元貞拒戰死。太后詔左玉鈐衛大將軍李孝逸為揚州道行軍大總管,率兵三十萬討之,戰於高郵,前鋒左豹韜果毅成三朗為唐之奇所殺。又以左鷹揚衛大將軍黑齒常之為江南道行軍大總管,並力。敬業興三月敗,傳首東都,三州平。



 始,武承嗣請太后立七廟,中書令裴炎沮止,及敬業之興,下炎獄,殺之,並殺左威衛大將軍程務挺。太后方怫恚,一日,召群臣廷讓曰:「朕於天下無負,若等知之乎?」群臣唯唯。太后曰:「朕輔先帝逾三十年,憂勞天下。爵位富貴,朕所與也;天下安佚,朕所養也。先帝棄群臣,以社稷為托,朕不敢愛身,而知愛人。今為戎首者皆將相,何見負之遽?且受遺老臣伉扈難制有若裴炎乎?世將種能合亡命若徐敬業乎?宿將善戰若程務挺乎?彼皆人豪,不利於朕,朕能戮之。公等才有過彼,蚤為之。不然,謹以事朕,無詒天下笑。」群臣頓首,不敢仰視,曰:「惟陛下命。」



 久之,下詔陽若復闢者。睿宗揣非情,固請臨朝,制可。乃冶銅匭為一室,署東曰「延恩」,受干賞自言;南曰「招諫」,受時政失得;西曰「申冤」,受抑枉所欲言;北曰「通玄」,受讖步秘策。詔中書門下一官典領。



 太后不惜爵位,以籠四方豪桀自為助,雖妄男子,言有所合,輒不次官之,至不稱職,尋亦廢誅不少縱,務取實材真賢。又畏天下有謀反逆者,詔許上變,在所給輕傳,供五品食,送京師,即日召見,厚餌爵賞歆動之。凡言變,吏不得何詰,雖耘夫蕘子必親延見,稟之客館。敢稽若不送者,以所告罪之。故上變者遍天下,人人屏息,無敢議。



 新豐有山因震突出,太后以為美祥,赦其縣,更名慶山。荊人俞文俊上言:「人不和,疣贅生;地不和,堆阜出。今陛下以女主處陽位,山變為災,非慶也。」太后怒,投嶺外。



 詔毀乾元殿為明堂,以浮屠薛懷義為使督作。懷義,鄠人,本馮氏,名小寶,偉岸淫毒,佯狂洛陽市,千金公主嬖之。主上言:「小寶可入侍。」後召與私,悅之。欲掩跡,得通籍出入,使祝發為浮屠,拜白馬寺主。詔與太平公主婿薛紹通昭穆,紹父事之。給廄馬,中官為騶侍,雖承嗣、三思皆尊事惟謹。至是護作,士數萬,巨木率一章千人乃能引。又度明堂後為天堂,鴻麗巖奧次之。堂成,拜左威衛大將軍、梁國公。



 始作崇先廟於西京,享武氏。承嗣偽款洛水石,導使為帝,遣雍人唐同泰獻之,後號為「寶圖」,擢同泰游擊將軍。於是汜人又上瑞石,太后乃郊上帝謝況,自號聖母神皇,作神皇璽,改寶圖曰「天授聖圖」,號洛水曰永昌水,圖所曰聖圖泉,勒石洛壇左曰「天授聖圖之表」,改汜水曰廣武。時柄去王室,大臣重將皆撓不得逞,宗室孤外無寄足地。於是,韓王元嘉等謀舉兵唱天下,迎還中宗。瑯邪王沖、越王貞先發,諸王倉卒無應者,遂敗。元嘉與魯王靈夔等皆自殺,餘悉坐誅,諸王牽連死滅殆盡,子孫雖嬰褓亦投嶺南。太后身拜洛受圖,天子率太子、群臣、蠻夷以次列,大陳珍禽、奇獸、貢物、鹵簿壇下,禮成去。



 永昌元年,享萬象神宮,改服袞冕,裯大圭,執鎮圭,睿宗亞獻,太子終獻。合祭天地,五方帝、百神從,以高祖、太宗、高宗配,引魏王士護從配。班九條,訓百官。遂大饗群臣。號士護周忠孝太皇,楊忠孝太后。以文水墓為章德陵,咸陽墓為明義陵。太原安成王為周安成王,金城郡王為魏義康王,北平郡王為趙肅恭王,魯國公為太原靖王。



 載初中,又享萬象神宮,以太穆、文德二皇后配皇地祗,引周忠孝太后從配。作曌、ь、〓、щ、囝、○、ъ、ы、〓、ш、ч、ю十有二文。太后自名曌。改詔書為制書。以周、漢為二王後,虞、夏、殷後為三恪,除唐屬籍。拜薛懷義輔國大將軍,封鄂國公,令與群浮屠作《大雲經》,言神皇受命事。春官尚書李思文詭言:「《周書·武成》為篇,辭有『垂拱天下治』,為受命之符。」後喜,皆班示天下,稍圖革命。然畏人心不肯附,乃陰忍鷙害,肆斬殺怖天下。內縱酷吏周興、來俊臣等數十人為爪吻,有不慊若素疑憚者,必危法中之。宗姓侯王及它骨鯁臣將相駢頸就鈇,血丹狴戶,家不能自保。太后操奩具坐重幃,而國命移矣。



 御史傅游藝率關內父老請革命,改帝氏為武。又脅群臣固請,妄言鳳集上陽宮,赤雀見朝堂。天子不自安,亦請氏武,示一尊。太后知威柄在己,因大赦天下,改國號周,自稱聖神皇帝,旗幟尚赤,以皇帝為皇嗣。立武氏七廟於神都。尊周文王為文皇帝,號始祖,妣姒曰文定皇后;武王為康皇帝,號睿祖,妣姜曰康惠皇后;太原靖王為成皇帝,號嚴祖,妣曰成莊皇后;趙肅恭王為章敬皇帝,號肅祖,妣曰章敬皇后;魏義康王為昭安皇帝,號烈祖,妣曰昭安皇后;祖周安成王為文穆皇帝,號顯祖,妣曰文穆皇后;考忠孝太皇為孝明高皇帝,號太祖,妣曰孝明高皇后。罷唐廟為享德廟,四時祠高祖以下三室,餘廢不享。至日,祀上帝萬象神宮,以始祖及考妣配,以百神從祀。盡王諸武。詔並州文水縣為武興,比漢豐、沛,百姓世給復。以始祖塚為德陵,睿祖為喬陵,嚴祖為節陵,肅祖為簡陵,烈祖為靖陵,顯祖為永陵,章德陵為昊陵,明義陵為順陵。



 太皇雖春秋高,善自塗澤,雖左右不悟其衰。俄而二齒生,下詔改元為長壽。明年,享神宮,自制大樂,舞工用九百人,以武承嗣為亞獻,三思為終獻。帝之為皇嗣,公卿往往見之,會尚方監裴匪躬、左衛大將軍阿史那元慶、白澗府果毅薛大信、監門衛大將軍範雲仙潛謁帝,皆腰斬都市,自是公卿不復上謁。



 有上封事言嶺南流人謀反者,太后遣攝右臺監察御史萬國俊就按,得實即論決。國俊至廣州,盡召流人,矯詔賜自盡,皆號哭不服,國俊驅之水曲,使不得逃,一日戮三百餘人。乃誣奏流人怨望,請悉除之。於是太后遣右衛翊府兵曹參軍劉光業、司刑評事王德壽、苑南面監丞鮑思恭、尚輦直長王大貞、右武衛兵曹參軍屈貞筠,皆攝監察御史,分往劍南、黔中、安南等六道訊鞫,而擢國俊左臺侍御史。光業等亦希功於上,惟恐殺人之少。光業殺者九百人,德壽殺七百人,其餘亦不減五百人。太后久乃知其冤,詔六道使所殺者還其家。國俊等亦相踵而死,皆見有物為厲云。



 太后又自加號金輪聖神皇帝,置七寶於廷:曰金輪寶,曰白象寶,曰女寶,曰馬寶,曰珠寶,曰主兵臣寶,曰主藏臣寶,率大朝會則陳之。又尊其顯祖為立極文穆皇帝,太祖為無上孝明皇帝。延載二年,武三思率蕃夷諸酋及耆老請作天樞,紀太后功德,以黜唐興周,制可。使納言姚護作。乃大裒銅鐵合冶之,署曰「大周萬國頌德天樞」,置端門外。其制若柱,度高一百五尺,八面,面別五尺,冶鐵象山為之趾,負以銅龍,石鑱怪獸環之。柱顛為雲蓋,出大珠,高丈,圍三之。作四蛟,度丈二尺,以承珠。其趾山周百七十尺,度二丈。無慮用銅鐵二百萬斤。乃悉鏤群臣、蕃酋名氏其上。



 薛懷義寵稍衰,而御醫沈南璆進,懷義大望,因火明堂,太后羞之,掩不發。懷義愈很恣怏怏。乃密詔太平公主擇健婦縛之殿中,命建昌王武攸寧、將作大匠宗晉卿率壯士擊殺之,以畚車載尸還白馬寺。懷義負幸暱,氣蓋一時,出百官上,其徒多犯法。御史馮思勖劾其奸,懷義怒,遇諸道,命左右歐之,幾死,弗敢言。默啜犯塞,拜新平、伐逆、朔方道大總管,提十八將軍兵擊胡,宰相李昭德、蘇味道至為之長史、司馬。後厭入禁中,陰募力少年千人為浮屠,有逆謀。侍御史周矩劾狀請治驗,太后曰:「第出,朕將使詣獄。」矩坐臺,少選,懷義怒馬造廷,直往坐大榻上,矩召吏受辭,懷義即乘馬去。矩以聞,太后曰:「是道人素狂,不足治,力少年聽窮劾。」矩悉投放醜裔。懷義構矩,俄免官。



 太后祀天南郊,以文王、武王、士獲與唐高祖並配。太后加號天冊金輪聖神皇帝。遂封嵩山,禪少室,冊山之神為帝,配為後。封壇南有大槲,赦日置雞其杪,賜號「金雞樹」。自制《升中述志》,刻石示後。改明堂為通天宮,鑄九州鼎,各位其方,列廷中。又斂天下黃金作大儀鐘,不克。久之,以崇先廟為崇尊廟,禮視太廟,旋復崇尊廟為太廟。



 自懷義死,張易之、昌宗得幸,乃置控鶴府,有監,有丞及主簿、錄事等,監三品,以易之為之。太后自見諸武王非天下意,前此中宗自房州還,復為皇太子,恐百歲後為唐宗室躪藉無死所,即引諸武及相王、太平公主誓明堂,告天地,為鐵券使藏史館。改昊陵署為攀龍臺。久視初,以控鶴監為天驥府,又改奉宸府,罷監為令,以左右控鶴為奉宸大夫,易之復為令。



 神龍元年,太后有疾,久不平,居迎仙院。宰相張柬之與崔玄韋等建策,請中宗以兵入誅易之、昌宗,於是羽林將軍李多祚等帥兵自玄武門入,斬二張於院左。太后聞變而起,桓彥範進請傳位,太后返臥,不復語。中宗於是復即位。徙太后上陽宮,帝率百官詣觀風殿問起居,後率十日一詣宮,俄朝朔、望。廢奉宸府官,選東都武氏廟於崇尊廟,更號崇恩,復唐宗廟。諸武王者咸降爵。是歲,後崩,年八十一。遺制稱則天大聖皇太后,去帝號。謚曰則天大聖後,祔乾陵。



 會武三思蒸韋庶人,復用事。於是大旱,祈陵輒雨。三思訹帝詔崇恩廟祠如太廟,齋郎用五品子。博士楊孚言:「太廟諸郎取七品子,今崇恩取五品,不可。」帝曰:「太廟如崇恩可乎?」孚曰:「崇恩太廟之私,以臣準君則僭,以君準臣則惑。」乃止。及韋、武黨誅,詔則天大聖皇后復號天後,廢崇恩廟及陵。景雲元年,號大聖天後。太平公主奸政,請復二陵官,又尊后曰天后聖帝,俄號聖後。太平誅,詔黜周孝明皇帝號,復為太原郡王,後為妃,罷昊、順等陵。開元四年,追號則天皇后。太常卿姜晈建言:「則天皇后配高宗廟,主題天后聖帝,非是,請易題為則天皇后武氏。」制可。



 中宗和思順聖皇后趙氏,京兆長安人。祖綽,武德中,戰有功,終右領軍將軍。父瑰,尚高祖常樂公主。



 帝為英王,聘後為妃。高宗於公主恩尤隆。武后不喜,乃幽妃內侍省。環自定州刺史、駙馬都尉貶括州,絕主朝謁,隨瑰之官。妃既囚,扃鍵牢謹,日給飼料。衛者候其突煙數日不出,披戶視之,死腐矣,瑰以壽州刺史與主預越王事,死。神龍元年,追謚妃曰恭皇后,贈瑰左衛大將軍。中宗崩,蕆陵事,韋庶人不臣,不得祔,有司加上尊謚,以後祔定陵。



 中宗庶人韋氏,京兆萬年人。祖弘表,貞觀中曹王府典軍。



 帝在東宮,後被選為妃。嗣聖初,立為皇后。俄與帝處房陵,每使至,帝輒恐,欲自殺。後止曰:「禍福何常,早晚等死耳,無遽!」及帝復即位,後居中宮。



 是時,上官昭容與政事,方敬暉等將盡誅諸武,武三思懼,乃因昭容入請,得幸於後,卒謀暉等誅之。初,帝幽廢,與後約:「一朝見天日,不相制。」至是與三思升御床博戲,帝從旁典籌,不為忤。三思諷群臣上後號為順天皇后。乃親謁宗廟,贈父玄貞上洛郡王。左拾遺賈虛已建言:「非李氏王者,盟書共棄之。今復國未幾,遽私後家,且先朝禍鑒未遠,甚可懼也。如今皇后固辭,使天下知後宮謙讓,不亦善乎?」不聽。神龍三年,節愍太子舉兵敗。宗楚客率群臣請加號「翊聖」,詔可。禁中謬傳有五色雲起後衣笥,帝圖以示諸朝,因大赦天下,賜百官母、妻封號。太史迦葉志忠表上《桑條歌》十二篇,言後當受命,曰:「昔高祖時,天下歌《桃李》;太宗時,歌《秦王破陣》;高宗歌《堂堂》;天後世,歌《武媚娘》;皇帝受命,歌《英王石州》;後今受命,歌《桑條韋》,蓋後妃之德專蠶桑,共宗廟事也。」乃賜志忠第一區,彩七百段。太常少卿鄭愔因之被樂府。楚客又諷補闕趙延禧離釋《桑條》為九十八代,帝大喜,擢延禧諫議大夫。



 於是昭容以武氏事動後。即表增出母服;民以二十三為丁,限五十九免;五品而上母、妻不繇夫、子封者,喪得用鼓吹。數改制度,陰儲人望。稍寵樹親屬,封拜之。昭容與母及尚宮賀婁等多受金錢。封巫趙隴西夫人,出入禁中,勢與上官埒。繇是墨敕斜封出矣。三年,帝親郊,引後亞獻。明年,正月望夜,帝與後微服過市,徬徉觀覽,縱宮女出游,皆淫奔不還。國子祭酒葉靜能善禁架,常侍馬秦客高醫,光祿少卿楊均善烹調,綿引入後廷。均、秦客蒸於後,嘗喪免,不歷旬輒起。



 帝遇弒,議者裯咎秦客及安樂公主。後大懼,引所親議計,乃以刑部尚書裴談、工部尚書張錫輔政,留守東都,詔將軍趙承福、薛簡以兵五百衛譙王重福,與兄溫定策,立溫王重茂為皇太子,列府兵五萬分二營屯京師,然後發喪。太子即位,是為殤帝。皇太后臨朝,溫總內外兵,檢護宮省。族弟濯、播,宗子捷、璿,璿〗甥高崇及武延秀,分領左右屯營、羽林、飛騎、萬騎。京師大恐,傳言且革命。播、璿入軍中,鞭督萬騎欲立威,士怨不為用。俄而臨淄王引兵夜披玄武門入羽林,殺璿、播、崇於寢,斧關叩太極殿,後遁入飛騎營,為亂兵所殺。斬延秀、安樂公主。分捕諸韋、諸武與其支黨,悉誅之,梟後及安樂首東市。翌日,追貶為庶人,葬以一品禮。



 上官昭容者,名婉兒,西臺侍郎儀之孫。父廷芝,與儀死武后時。母鄭,太常少卿休遠之姊。



 婉兒始生,與母配掖廷。天性韶警,善文章。年十四,武后召見,有所制作,若素構。自通天以來,內掌詔命,掞麗可觀。嘗忤旨當誅,後惜其才,止黥而不殺也。然群臣奏議及天下事皆與之。



 帝即位,大被信任,進拜昭容,封鄭沛國夫人。婉兒通武三思,故詔書推右武氏,抑唐家,節愍太子不平。及舉兵,叩肅章門索婉兒,婉兒曰:「我死,當次索皇后、大家矣!」以激怒帝,帝與後挾婉兒登玄武門避之。會太子敗,乃免。婉兒勸帝侈大書館,增學士員,引大臣名儒充選。數賜宴賦詩,群臣賡和,婉兒常代帝及後、長寧安樂二主,眾篇並作,而採麗益新。又差第群臣所賦,賜金爵,故朝廷靡然成風。當時屬辭者,大抵雖浮靡,然所得皆有可觀,婉兒力也。鄭卒,謚節義夫人。婉兒請降秩行服,詔起為婕妤,俄還昭容。帝即婉兒居穿沼築巖,窮飾勝趣,即引侍臣宴其所。是時,左右內職皆聽出外,不何止。婉兒與近嬖至皆營外宅,邪人穢夫爭候門下,肆狎暱,因以求劇職要官。與崔湜亂,遂引知政事。湜開商山道,未半,因帝遺制,虛列其功,加甄賞。韋後之敗,斬闕下。



 初,鄭方妊,夢巨人畀大稱曰:「持此稱量天下。」婉兒生逾月,母戲曰:「稱量者豈爾邪?」輒啞然應。後內秉機政,符其夢云。景雲中,追復昭容,謚惠文。始,從母子王昱為拾遺,昱戒曰:「上往囚房陵,武氏得志矣,卒而中興,天命所在,不可幸也。三思雖乘釁,天下知必敗,今昭容上所信,而附之,且滅族!」鄭以責婉兒,不從。節愍誅三思,果索之,始憂懼。及草遺制,即引相王輔政。臨淄王兵起,被收。婉兒以詔草示劉幽求,幽求言之王,王不許,遂誅。開元初,裒次其文章,詔張說題篇。



 睿宗肅明順聖皇後劉氏,祖德威,自有傳。儀鳳中,帝在籓,納為孺人,俄為妃。生寧王、壽昌代國二公主。帝即位,為皇后。會帝降號皇嗣,復為妃。長壽二年,為戶婢誣與竇德妃挾蠱道祝詛武後,並殺之宮中,葬秘莫知。景雲元年,追謚肅明皇后。



 睿宗昭成順聖皇后竇氏,曾祖抗,父孝諶,自有傳。



 後婉淑,尤循禮則。帝為相王,納為孺人;即位,進德妃。生玄宗及金仙、玉真二公主。與肅明同追謚,並招魂葬東都之南,肅明曰惠陵,後曰靖陵,立別廟曰儀坤以享雲。帝崩,追稱皇太后,與肅明祔橋陵。後以子貴,故先祔睿宗室。肅明以開元二十年乃得祔廟。



 初,太常加謚后曰「大昭成」。或言:「法宜引『聖真』冠謚,而曰『大昭成』,非也。以單言配之,應曰『聖昭』若『睿成』;以復言配之,應曰『大聖昭成』、『聖真昭成』。」又引太穆皇后始謚穆,及高祖崩,合帝謚曰太穆,追增太穆神皇后;文德皇后始謚文德,及太宗崩,合謚文德聖皇后。又援範曄著漢光烈等為比。太常謂:「曄以帝號標後謚,是史家記事體,婦人非必與夫同也。入廟稱後,系夫;在朝稱太,系子。『文母』,生號也;『文王』,既沒謚也。周公豈以夫從婦乎?漢法不可以為據。」制曰「可」。天寶八載制詔,自太穆而下六皇后,並增上「順聖」二謚雲。



 玄宗皇后王氏,同州下邽人。梁冀州刺史神念之裔孫。帝為臨淄王,聘為妃。將清內難,預大計。先天元年,立為皇后。久無子,而武妃稍有寵,後不平,顯詆之。然撫下素有恩,終無肯譖短者。帝密欲廢後,以語姜晈。晈漏言,即死。後兄守一懼,為求厭勝,浮屠明悟教祭北斗,取霹靂木刻天地文及帝諱合佩之,曰:「後有子,與則天比。」開元十二年,事覺,帝自臨劾有狀,乃制詔有司:「皇后天命不祐,華而不實,有無將之心,不可以承宗廟、母儀天下,其廢為庶人。」賜守一死。



 始,後以愛弛,不自安。承間泣曰:「陛下獨不念阿忠脫紫半臂易鬥面,為生日湯餅邪?」帝憫然動容。阿忠,後呼其父仁皎云。繇是久乃廢。當時王諲作《翠羽帳賦》諷帝。未幾卒,以一品禮葬。後宮思慕之,帝亦悔。寶應元年,追復後號。



 玄宗貞順皇后武氏,恆安王攸止女,幼入宮。帝即位,寢得幸。時王皇后廢,故進冊惠妃,其禮秩比皇后。



 初,帝在潞,趙麗妃以倡幸,有容止,善歌舞。開元初,父兄皆美官。及妃進,麗妃恩亦弛,以十四年卒,謚曰和。生太子瑛。而皇甫德儀生鄂王,劉才人生光王,皆籓邸之舊,後愛薄,而妃乃專寵。封所生母楊鄭國夫人,弟忠國子祭酒,信秘書監。將遂立皇后,御史潘好禮上疏曰:「《禮》,父母仇,不共天。《春秋》,子不復仇,不子也。陛下欲以武氏為後,何以見天下士!妃再從叔三思也,從父延秀也,皆干紀亂常,天下共疾。夫惡木垂廕,志士不息;盜泉飛溢,廉夫不飲。匹夫匹婦尚相擇,況天子乎?願慎選華族,稱神祇之心。《春秋》:宋人夏父之會,無以妾為夫人;齊桓公誓葵丘曰:『無以妾為妻。』此聖人明嫡庶之分。分定,則窺競之心息矣。今人間咸言右丞相張說欲取立後功圖復相,今太子非惠妃所生,而妃有子,若一儷宸極,則儲位將不安。古人所以諫其漸者,有以也!」遂不果立。



 妃生子必秀嶷,凡二王、一主,皆不育。及生壽王,帝命寧王養外邸。又生盛王、咸宜太華二公主。後李林甫以壽王母愛,希妃意陷太子、鄂光二王,皆廢死。會妃薨,年四十餘,贈皇后及謚,葬敬陵。



 玄宗元獻皇后楊氏,華州華陰人。曾祖士達,為隋納言。天授中,以武后母黨,追封士達為鄭王,父知慶太尉。



 帝在東宮,後以景雲初入宮為良媛。時太平公主忌帝,而宮中左右持兩端,纖悉必聞。媛方娠,帝不自安,密語侍讀張說曰:「用事者不欲吾多子,奈何?」命說挾劑以入,帝於曲室自煮之。夢若有介而戈者環鼎三,而三煮盡覆。以告說,說曰:「天命也!」乃止。生男,是為肅宗。



 帝即位,為貴嬪。其姊,節愍太子妃也。初,肅宗生,卜云:「不宜養。」乃命王皇后舉之。後無子,撫肅宗如所生。後又生寧親公主,乃薨。說以舊恩,故子自得尚寧親。肅宗即位,至德二載,太上皇自蜀誥有司「其議尊稱」,遂上冊謚。寶應末,祔泰陵。



 玄宗貴妃楊氏,隋梁郡通守汪四世孫。徙籍蒲州,遂為永樂人。幼孤,養叔父家。始為壽王妃。開元二十四年,武惠妃薨,後廷無當帝意者。或言妃姿質天挺,宜充掖廷,遂召內禁中,異之,即為自出妃意者,丐籍女官,號「太真」,更為壽王聘韋昭訓女,而太真得幸。善歌舞,邃曉音律,且智算警穎,迎意輒悟。帝大悅,遂專房宴,宮中號「娘子」,儀體與皇后等。



 天寶初,進冊貴妃。追贈父玄琰太尉、齊國公。擢叔玄珪光祿卿,宗兄銛鴻臚卿,錡侍御史,尚太華公主。主,惠妃所生,最見寵遇。而釗亦浸顯。釗,國忠也。三姊皆美劭,帝呼為姨,封韓、虢、秦三國,為夫人,出入宮掖,恩寵聲焰震天下。每命婦入班,持盈公主等皆讓不敢就位。臺省、州縣奉請托,奔走期會過詔敕。四方獻餉結納,門若市然。建平、信成二公主以與妃家忤,至追內封物,駙馬都尉獨孤明失官。



 它日,妃以譴還銛第,比中仄,帝尚不御食,笞怒左右。高力士欲驗帝意,乃白以殿中供帳、司農酒餼百餘車送妃所,帝即以御膳分賜。力士知帝旨,是夕,請召妃還,下鑰安興坊門馳入。妃見帝,伏地謝,帝釋然,撫尉良渥。明日,諸姨上食,樂作,帝驟賜左右不可貲。由是愈見寵,賜諸姨錢歲百萬為脂粉費。銛以上柱國門列戟,與銛、國忠、諸姨五家第舍聯亙,擬憲宮禁,率一堂費緡千萬。見它第有勝者,輒壞復造,務以環侈相誇詡,土木工不息。帝所得奇珍及貢獻分賜之,使者相銜於道,五家如一。



 妃每從游幸,乘馬則力士授轡策。凡充錦繡官及冶彖金玉者,大抵千人,奉須索,奇服秘玩,變化若神。四方爭為怪珍入貢,動駭耳目。於是嶺南節度使張九章、廣陵長史王翼以所獻最,進九章銀青階,擢翼戶部侍郎,天下風靡。妃嗜荔支,必欲生致之,乃置騎傳送,走數千里,味未變已至京師。



 天寶九載,妃復得譴還外第,國忠謀於吉溫。溫因見帝曰:「婦人過忤當死,然何惜宮中一席廣為鈇金質地,更使外辱乎?」帝感動,輟食,詔中人張韜光賜之。妃因韜光謝帝曰:「妾有罪當萬誅,然膚發外皆上所賜,今且死,無以報。」引刀斷一繚發奏之,曰:「以此留訣。」帝見駭惋,遽召入,禮遇如初。因又幸秦國及國忠第,賜兩家鉅萬。



 國忠既遙領劍南,每十月,帝幸華清宮,五宅車騎皆從,家別為隊,隊一色,俄五家隊合,爛若萬花,川谷成錦繡,國忠導以劍南旗節。遺鈿墮舄,瑟瑟璣琲,狼藉於道,香聞數十里。十載正月望夜,妃家與廣寧主僮騎爭闤門,鞭挺言雚競,主墮馬,僅得去。主見帝泣,乃詔殺楊氏奴,貶駙馬都尉程昌裔官。國忠之輔政,其息昢尚萬春公主,暄尚延和郡主;弟鑒尚承榮郡主。又詔為玄琰立家廟,帝自書其碑。銛、秦國早死,故韓、虢與國忠貴最久。而虢國素與國忠亂,頗為人知,不恥也。每入謁,並驅道中,從監、侍姆百餘騎,炬蜜如盡,靚妝盈里,不施幃障,時人謂為「雄狐」。諸王子孫凡婚聘,必先因韓、虢以請,輒皆遂,至數百千金以謝。



 初,安祿山有邊功,帝寵之,詔與諸姨約為兄弟,而祿山母事妃,來朝,必宴餞結歡。祿山反,以誅國忠為名,且指言妃及諸姨罪。帝欲以皇太子撫軍,因禪位,諸楊大懼,哭於廷。國忠入白妃,妃銜塊請死,帝意沮,乃止。及西幸至馬嵬,陳玄禮等以天下計誅國忠,已死,軍不解。帝遣力士問故,曰:「禍本尚在!」帝不得已,與妃訣,引而去,縊路祠下,裹尸以紫茵,瘞道側,年三十八。



 帝至自蜀,道過其所,使祭之,且詔改葬。禮部侍郎李揆曰:「龍武將士以國忠負上速亂,為天下殺之。今葬妃,恐反仄自疑。」帝乃止。密遣中使者具棺槨它葬焉。啟瘞,故香囊猶在,中人以獻,帝視之,淒感流涕,命工貌妃於別殿,朝夕往,必為鯁欷。



 馬嵬之難,虢國與國忠妻裴柔等奔陳倉,縣令率吏追之,意以為賊,棄馬走林。虢國先殺其二子,柔曰:「丐我死!」即並其女刺殺之,乃自剄,不殊,吏載置於獄,問曰:「國家乎?賊乎?」吏曰:「互有之。」乃死,瘞陳倉東郭外。



 贊曰:或稱武、韋亂唐同一轍,武持久,韋亟滅,何哉?議者謂否。武后自高宗時挾天子威福,脅制四海,雖逐嗣帝,改國號,然賞罰己出,不假借群臣,僭於上而治於下,故能終天年,阽亂而不亡。韋氏乘夫,淫蒸於朝,斜封四出,政放不一,既鴆殺帝,引睿宗輔政,權去手不自知,戚地已疏,人心相挻,玄宗藉其事以撼豪英,故取若掇遺,不旋踵宗族夷丹,勢奪而事淺也。然二后遺後王戒,顧不厚哉!



\end{pinyinscope}