\article{志第九 禮樂九}

\begin{pinyinscope}

 皇帝元正、冬至受群臣朝賀而會。



 前一日,尚舍設御幄於太極殿,有司設群官客使等次於東西朝堂,展縣,置桉,陳車輿,又設解劍席於縣西北橫街之南。文官三品以上位於橫街之南,道東;褒聖侯位於三品之下,介公、酅公位於道西;武官三品以上位於介公之西,少南;文官四品、五品位於縣東,六品以下位於橫街之南。又設諸州朝集使位:都督、刺史三品以上位於文、武官三品之東、西,四品以下分方位於文、武官當品之下。諸州使人又於朝集使之下,諸親於四品、五品之南。設諸蕃方客位:三等以上,東方、南方在東方朝集使之東,西方、北方在西方朝集使之西,每國異位重行,北面;四等以下,分方位於朝集使六品之下。又設門外位:文官於東朝堂,介公、酅公在西朝堂之前,武官在介公之南,少退,每等異位重行;諸親位於文、武官四品、五品之南;諸州朝集使,東方、南方在宗親之南,使人分方於朝集使之下;諸方客,東方、南方在東方朝集使之南,西方、北方在西方朝集使之南,每國異位重行。



 其日,將士填諸街,勒所部列黃麾大仗屯門及陳於殿庭,群官就次。侍中版奏「請中嚴」。諸侍衛之官詣閣奉迎,吏部兵部主客戶部贊群官、客使俱出次,通事舍人各引就朝堂前位,引四品以下及諸親、客等應先置者入就位。侍中版奏「外辦」。皇帝服袞冕,冬至則服通天冠、絳紗袍,御輿出自西房,即御座南向坐。符寶郎奉寶置於前,公、王以下及諸客使等以次入就位。典儀曰:「再拜。」贊者承傳,在位者皆再拜。上公一人詣西階席,脫舄,跪,解劍置於席,升,當御座前,北面跪賀,稱:「某官臣某言:元正首祚,景福惟新,伏惟開元神武皇帝陛下與天同休。」冬至云:「天正長至,伏惟陛下如日之升。」乃降階詣席,跪,佩劍,俯伏,興,納舄,復位。在位者皆再拜。侍中前承詔,降,詣群官東北,西面,稱「有制」。在位者皆再拜。宣制曰:「履新之慶,與公等同之。」冬至云:「履長。」在位者皆再拜,舞蹈,三稱萬歲,又再拜。



 初,群官將朝,中書侍郎以諸州鎮表別為一桉,俟於右延明門外,給事中以祥瑞桉俟於左延明門外,侍郎、給事中俱就侍臣班。初入,戶部以諸州貢物陳於太極門東東、西廟,禮部以諸蕃貢物可執者,蕃客執入就位,其餘陳於朝堂前。上公將入門,中書侍郎、給事中皆降,各引其桉入,詣東、西階下立。上公將升賀,中書令、黃門侍郎俱降,各立,取所奏之文以次升。上公已賀,中書令前跪奏諸方表,黃門侍郎又進跪奏祥瑞,俱降,置所奏之文於桉。侍郎與給事中引桉退至東、西階前,桉出。



 初,侍中已宣制,朝集使及蕃客皆再拜。戶部尚書進詣階間跪奏,稱:「戶部尚書臣某言:諸州貢物請付所司。」侍中前承制,退,稱:「制曰可。」禮部尚書以次進詣階間,跪奏,稱「禮部尚書臣某言:諸蕃貢物請付所司。」侍中前承制,退,稱:「制曰可。」太府帥其屬受諸州及諸蕃貢物出歸仁、納義門,執物者隨之。典儀曰:「再拜。」通事舍人以次引北面位者出。侍中前,跪奏稱:「侍中臣某言:禮畢。」皇帝降座,御輿入自東房,侍臣從至閣。引東、西面位者以次出,蕃客先出。



 冬至,不奏祥瑞,無諸方表。其會,則太樂令設登歌於殿上,二舞入,立於縣南。尚舍設群官升殿者座:文官三品以上又御座東南,西向;介公、酅公在御座西南,東向;武官三品以上又於其後;朝集使、都督、刺史,蕃客三等以上,座如立位。設不升殿者座各於其位。又設群官解劍席於縣之西北,橫街之南。尚食設壽尊於殿上東序之端,西向;設坫於尊南,加爵一。太官令設升殿者酒尊於東、西廂,近北;設在庭群官酒尊各於其座之南。皆有坫、冪,俱障以帷。吏部兵部戶部主客贊群官、客使俱出次,通事舍人引就朝堂前位,又引非升殿者次入就位。侍中版奏「外辦」。皇帝改服通天冠、絳紗袍,御輿出自西房,即御座。典儀一人升就東階上,通事舍人引公、王以下及諸客使以次入就位。侍中進,當御座前北面跪奏,稱:「侍中臣某言:請延諸公、王等升。」又侍中稱:「制曰可。」侍中詣東階上,西面,稱:「制延公、王等升殿上。」典儀承傳,階下贊者又承傳,在位者皆再拜。應升殿者詣東、西階,至解劍席,脫舄,解劍,升。上公一人升階,少東,西面,立於座後。光祿卿進詣階間,跪奏稱:「臣某言:請賜群臣上壽。」侍中稱:「制曰可。」光祿卿退,升詣酒尊所,西向立。上公酒尊所,北面。尚食酌酒一爵授上公,上公受爵,進前,北面授殿中監,殿中監受爵,進,置御前,上公退,北面跪稱:「某官臣某等稽首言:元正首祚冬至云:「天正長至。」,臣某等不勝大慶,謹上千秋萬歲壽。」再拜,在位者皆再拜,立於席後。侍中前承制,退稱:「敬舉公等之觴。」在位者又再拜。殿中監取爵奉進,皇帝舉酒,在位者皆舞蹈,三稱萬歲。皇帝舉酒訖,殿中監進,受虛爵,以授尚食,尚食受爵於坫。



 初,殿中監受虛爵,殿上典儀唱:「再拜。」階下贊者承傳,在位者皆再拜。上公就座後立,殿上典儀唱:「就座。」階下贊者承傳,俱就座。歌者琴瑟升坐,笙管立階間。尚食進,酒至階,殿上典儀唱:「酒至,興。」階下贊者承傳,坐者皆俯伏,起,立於席後。殿中監到階省酒,尚食奉酒進,皇帝舉酒。太官令又行群官酒,酒至,殿上典儀唱:「再拜。」階下贊者承傳,在位者皆再拜,搢笏受觶。殿上典儀唱:「就座。」階下贊者承傳,皆就座。皇帝舉酒,尚食進,受虛爵,復於坫。觴行三周,尚食進御食,食至階,殿上典儀唱:「食至,興。」階下贊者承傳,坐者皆起,立座後。殿中監到階省桉,尚食品嘗食訖,以次進置御前。太官令又行群安桉,設食訖,殿上典儀唱:「就座。」階下贊者承傳,皆就座。皇帝乃飯,上下俱飯。御食畢,仍行酒,遂設庶羞,二舞作。若賜酒,侍中承詔詣東階上,西面稱:「賜酒。」殿上典儀承傳,階下贊者又承傳,坐者皆起,再拜,立,受觶,就席坐飲,立,授虛爵,又再拜,就座。酒行十二遍。



 會畢,殿上典儀唱:「可起。」階下贊者承傳,上下皆起,降階,佩劍,納舄,復位。位於殿庭者,仍立於席後。典儀曰:「再拜。」贊者承傳,在位者皆再拜。若有賜物,侍中前承制,降,詣群官東北,西面,稱:「有制。」在位者皆再拜。侍中宣制,又再拜,以次出。侍中前,跪奏稱:「侍中臣某言:禮畢。」皇帝興,御輿入自東房,東、西面位者以次出。



 皇帝若服翼善冠、褲褶,則京官褲褶,朝集使公服。設九部樂,則去樂縣,無警蹕。太樂令帥九部伎立於左、右延明門外,群官初唱萬歲,太樂令即引九部伎聲作而入,各就座,以次作。



 臨軒冊皇太子。



 有司卜日,告於天地宗廟。



 前一日,尚舍設御幄於大極殿,有司設太子次於東朝堂之北,西向。又設版位於大橫街之南,展縣,設桉,陳車輿,及文武群官、朝集、蕃客之次位,皆如加元服之日。



 其日,前二刻,宮官服其器服,諸衛率各勒所部陳於庭。左庶子奏「請中嚴」。侍衛之官奉迎,僕進金路,內率一人執刀。贊善奏「發引」。令侍臣上馬,庶子承令。其餘略如皇帝出宮之禮。皇太子遠游冠、絳紗袍,三師導,三少從,鳴鐃而行。降路入次,亦如鑾駕。



 其日,列黃麾大仗,侍中請「中嚴」。有司與群官皆入就位。三師、三少導從,皇太子立於殿門外之東,西向。黃門侍郎以冊、寶綬桉立於殿內道北,西面,中書侍郎立桉後。侍中乃奏「外辦」。皇帝服袞冕,出自西房,即御座。皇太子入就位。典儀曰:「再拜。」皇太子再拜。又曰:「再拜。」在位者皆再拜。中書令降,立於皇太子東北,西向。中書侍郎一人引冊、一人引寶綬桉立於其東,西面,以冊授之。中書令曰:「有制。」皇太子再拜,中書令跪讀冊,皇太子再拜受冊,左庶子受之。侍郎以璽綬授中書令,皇太子進受,以授左庶子。皇太子再拜,在位者皆再拜。侍中奏「禮畢」。皇帝入自東房,在位者以次出。



 皇帝御明堂讀時令。



 孟春,禮部尚書先讀令三日奏讀月令,承以宣告。



 前三日,尚舍設大次於東門外道北,南向;守宮設文、武侍臣次於其後之左、右;設群官次於璧水東之門外,文官在北,武官在南,俱西上。



 前一日,設御座於青陽左個,東向。三品以上及諸司長官座於堂上:文官座於御座東北,南向;武官座於御座之東,北向。俱重行西上。設刑部郎中讀令座於御座東南,北向,有桉。設文官解劍席於丑陛之左,武官於卯陛之右,皆內向。太樂令展宮縣於青陽左個之庭,設舉麾位於堂上寅階之西,北向;其一位於樂縣東北,南向。典儀設三品以上及應升坐者位於縣東,文左武右,俱重行西向。非升坐者文官四品、五品位於縣北,六品以下於其東,絕位,俱南向;武官四品、五品於縣南,六品以下於其東,俱北向。皆重行西上。設典儀位於縣之西北,贊者二人在東,差退,俱南向。奉禮設門外位各於次前,俱每等異位,重行相向,西上。



 其日,陳小駕,皇帝服青紗袍,佩蒼玉,乘金路出宮,至於大次。文、武五品以上從駕之官皆就門外位,太樂令、工人、協律郎、典儀帥贊者皆先入,群官非升坐者次入,就位。刑部郎中以月令置於桉,覆以帕,立於武官五品東南,郎中立於桉後,北面。侍中版奏「外辦」。皇帝輿入自青龍門,升自寅階,即座。符寶郎置寶於前。典儀升,立於左個東北,南向。公、王以下入就西面位。典儀曰:「再拜。」贊者承傳,在位者皆再拜。侍中前,跪奏稱:「侍中臣某言:請延公、王等升。」又侍中稱:「制曰可。」侍中詣左個東北,南向稱:「詔延公、王等升。」典儀傳,贊者承傳,在位者皆再拜。西面位者各詣其階,解劍,脫舄,升,立於座後。刑部郎中引桉進,立於卯階下。侍中跪奏「請讀月令」。又侍中稱:「制曰可。」刑部郎中再拜,解劍,俯,脫舄,取令,升自卯階,詣席南,北向跪,置令於桉,立於席後。堂上典儀唱:「就座。」公、王以下及刑部郎中並就座。刑部郎中讀令,每句一絕,使言聲可了。讀訖,堂上典儀唱:「可起。」王、公以下皆起。刑部郎中以令置於桉,與群官佩劍,納舄,復於位。典儀曰;「再拜。」在位者皆再拜。西面位者出。侍中跪奏稱:「侍中臣某言:禮畢。」皇帝降座,御輿出之便次,南、北面位者以次出。



 自仲春以後,每月各居其位,皆冠通天,服、玉之色如其時。若四時之孟月及季夏土王讀五時令於明堂,亦如之。



 皇帝親養三老五更於太學。



 所司先奏三師、三公致仕者,用其德行及年高者一人為三老,次一人為五更,五品以上致仕者為國老,六品以下致仕者為庶老。尚食具牢饌。



 前三日,尚舍設大次於學堂之後,隨地之宜。設三老、五更次於南門外之西,群老又於其後,皆東向。文官於門外之東,武官在群老之西,重行,東西向,皆北上。



 前一日,設御座於堂上東序,西向,莞筵藻席。三老座於西楹之東,近北,南向;五更座於西階上,東向;國老三人座於三老西階,不屬焉。皆莞筵藻席。眾國老座於堂下西階之西,東面北上,皆蒲筵緇布純,加莞席。太樂令展宮縣於庭,設登歌於堂上,如元會。典儀設文、武官五品以上位於縣東、西,六品以下在其南,皆重行,西向北上,蕃客位於其南;諸州使人位於九品之後;學生分位於文、武官之後。設門外位如設次。又設尊於東楹之西,北向,左玄酒,右坫以置爵。



 其日,鑾駕將至,先置之官就門外位,學生俱青衿服,入就位。鑾駕至太學門,回輅南向,侍中跪奏「請降輅」。降,入大次。文、武五品以上從駕之官皆就門外位,太樂令、工人、二舞入,群官、客使以次入。



 初,鑾駕出宮,量時刻,遣使迎三老、五更於其第,三老、五更俱服進賢冠,乘安車,前後導從。其國老、庶老則有司預戒之。



 鑾駕既至太學,三老、五更及群老等俱赴集,群老各服其服。太常少卿贊三老、五更俱出次,引立於學堂南門外之西,東面北上;奉禮贊群老出次,立於三老、五更之後;太常博士引太常卿升,立於學堂北戶之內,當戶北面。侍中版奏「外辦」。皇帝出戶,殿中監進大珪,皇帝執大珪,降,迎三老於門內之東,西面立。侍臣從立於皇帝之後,太常卿與博士退立於左。三老、五更皆杖,各二人夾扶左右,太常少卿引導,敦史執筆以從。三老、五更於門西,東面北上,奉禮引群老隨入,立於其後。太常卿前奏「請再拜」。皇帝再拜,三老、五更去杖,攝齊答拜。皇帝揖進,三老在前,五更從,仍杖,夾扶至階,皇帝揖升,俱就座後立。皇帝西面再拜三老,三老南面答拜,皇帝又西向肅拜五更,五更答肅拜,俱坐。三公授幾,九卿正履。殿中監、尚食奉御進珍羞及黍、稷等,皇帝省之,遂設於三老前。皇帝詣三老座前,執醬而饋,乃詣酒尊所取爵,侍中贊酌酒,皇帝進,執爵而酳。尚食奉御以次進珍羞酒食於五更前,國老、庶老等皆坐,又設酒食於前,皆食。皇帝即座。三老乃論五孝六順、典訓大綱,格言宣於上,惠音被於下。皇帝乃虛躬請受,敦史執筆錄善言善行。禮畢,三老以下降筵,太常卿引皇帝從以降階,逡巡立於階前。三老、五更出,皇帝升,立於階上,三老、五更出門。侍中前奏「禮畢」。皇帝降,還大次。三老、五更升安車,導從而還,群官及學生等以次出。明日,三老主旨闕表謝。



 州貢明經、秀才、進士身孝悌旌表門閭者,行鄉飲酒之禮,皆刺史為主人。先召鄉致仕有德者謀之,賢者為賓,其次為介,又其次為眾賓,與之行禮,而賓舉之。主人戒賓,立於大門外之西,東面;賓立於東階下,西面。將命者立於賓之左,北面,受命出,立於門外之東,西面,曰:「敢請事。」主人曰:「某日行鄉飲酒之禮,請吾子臨之。」將命者入告,賓出,立於門東,西面拜辱,主人答拜。主人曰:「吾子學優行高,應茲觀國,某日展禮,請吾子臨之。」賓曰:「某固陋,恐辱命,敢辭。」主人曰:「某謀於父師,莫若吾子賢,敢固以請。」賓曰:「夫子申命之,某敢不敬須。」主人再拜,賓答拜,主人退,賓拜送。其戒介亦如之,辭曰;「某日行鄉飲酒之禮,請吾子貳之。」



 其日質明,設賓席於楹間,近北,南向;主人席於阼階上,西向;介席於西階上,東向;眾賓席三於賓席之西,南向;皆不屬。又設堂下眾賓席於西階西南,東面北上。設兩壺於賓席之東,少北,玄酒在西,加勺冪。置篚於壺南,東肆,實以爵觶。設贊者位於東階東,西向北上。賓、介及眾賓至,位於大門外之右,東面北上。主人迎賓於門外之左,西面拜賓,賓答拜;又西南面拜介,介答拜;又西南面揖眾賓,眾賓報揖。主人又揖賓,賓報揖。主人先入門而右,西面。賓入門而左,東面。介及眾賓序入,立於賓西南,東面北上。眾賓非三賓者皆北面東上。



 主人將進揖,當階揖,賓皆報揖。及階,主人曰:「請吾子升。」賓曰:「某敢辭。」主人曰:「固請吾子升。」賓曰:「某敢固辭。」主人曰:「終請吾子升。」賓曰:「某敢終辭。」主人升自阼階,賓升自西階,當楣,北面立。執尊者徹冪。主人適篚,跪取爵,興,適尊實之,進賓席前,西北面獻賓。賓西階上北面拜。主人少退,賓進於席前,受爵,退,復西階上,北面立。主人退於阼階上,北面拜,送爵。賓少退,贊者薦脯、醢於賓席前。賓自西方升席,南面立。贊者設折俎,賓跪,左執爵,右取脯,擩於醢,祭於籩、豆之間,遂祭酒,啐酒,興,降席東,適西階上,北面跪,卒爵,執爵興,適尊實之,進主人席前,東面酢主人。主人於阼階上北面拜,賓少退。主人進受爵,退阼階上,北面立。賓退,復西階上,北面拜,送爵。贊者薦脯、醢於主人席前,主人由席東自北方升席,贊者設折俎,主人跪,左執爵,右祭脯,擩於醢,祭於籩、豆之間,遂祭酒,啐酒,興,自南方降席,復阼階上,北面跪,卒爵,執爵興,跪奠爵於東序端,興,適篚,跪取觶實之以酬,復阼階上,北面跪,奠觶,遂拜,執觶興。賓西階上答拜。主人跪酒祭,遂飲,卒觶,執觶興,適尊實之,進賓席前,北面。賓拜,主人少退。賓既拜,主人跪奠觶於薦西,興,復阼階上位。賓遂進席前,北面跪,取觶,興,復西階上位。主人北面拜送。賓進席前,北面跪,奠觶於薦東,興,復西階上位。主人北面揖,降,立阼階下,西面。賓降,立於階西,東面。



 主人進延介,揖之,介報揖。至介,一讓升,主人升阼階,介升西階,當楣,北面立。主人詣東序端,跪取爵,興,適尊實之。進於介席前,西南面獻介。介西階上北面拜,主人少退,介進,北面受爵,退,復位。主人於介右北面拜送爵,介少退,主人立於西階之東。贊者薦脯、醢於介席前,介進自北方,升席,贊者設折俎,介跪,左執爵,右祭脯、醢,遂祭酒,執爵興,自南方降席,北面跪,卒爵,執爵興,介授主人爵,主人適尊實之,酢於西階上,立於介右,北面跪,奠爵,遂拜,執爵興。介答拜。主人跪祭,遂飲,卒爵,執爵興,進,跪奠爵於西楹南,還阼階上,揖降。介降,立於賓南。



 主人於阼階前西南揖眾賓,遂升,適西楹南,跪取爵,興,適尊實之,進於西階上,南面獻眾賓之長,升西階上,北面拜,受爵。主人於眾賓長之右,北面拜送。贊者薦脯、醢於其席前,眾賓之長升席,跪,左執爵,右祭脯、醢,祭酒,執爵,興,退於西階上,立飲訖,授主人爵,降,復位。主人又適尊實之,進於西階上,南面獻眾賓之次者,如獻眾賓之長。又次一人升,飲,亦如之。主人適尊實酒,進於西階上,南面獻堂下眾賓。每一人升,受爵,跪祭,立飲,贊者遍薦脯、醢於其位。主人受爵、尊於篚。主人與賓一揖一讓升,賓、介、眾賓序升,即席。



 設工人席於堂廉西階之東,北面東上。工四人,先二瑟,後二歌。工持瑟升自階,就位坐。工鼓《鹿鳴》,卒歌。笙入,立於堂下,北面,奏《南陔》。乃間歌,歌《南有嘉魚》,笙《崇丘》;乃合樂《周南》《關雎》、《召南》《鵲巢》。



 司正升自西階司正謂主人贊禮者,禮樂之正。既成,將留賓,為有懈墮,立司正以監之。跪取觶於篚,興,適尊實之,降自西階,詣階間,左還,北面跪,奠觶,拱手少立,跪,取觶,遂飲,卒觶,奠,再拜。賓降席,跪取觶於篚,適尊實之,詣阼階上,北面酬主人。主人降席,進,立於賓東,賓跪奠觶,遂拜,執觶興,主人答拜,賓立飲,卒觶,適尊實之,阼階上東南授主人,主人再拜,賓少退,主人受觶,賓於主人之西,北面拜送,賓揖,復席。主人進西階上,北面酬介,介降席,自南方進,立於主人之西,北面。主人跪奠觶,遂拜,執觶興,介答拜。主人立飲,卒觶,適尊實之,進西階上,西面立,介拜,主人少退,介受觶,主人於介東,北面送,主人揖,復席。



 司正升自西階,近西,北面立,相旅曰:「某子受酬。」受酬者降席,自西方進,北面立於介右。司正退,立於序端,東面,避受酬者。介跪奠觶,遂拜,執觶興,某子答拜。介立飲,卒觶,適尊實之,進西階上,西南面授某子,某子受觶,介立於某子之左,北面,揖,復席。司正曰:「某子受酬。」受酬者降席,自西方立於某子之左,北面,某子跪奠觶,遂拜,執觶興,受酬者答拜。某子立飲,卒觶,適尊實之,進西階上,西南面授之,受酬者受觶,某子立於酬者之右,揖,復席。次一人及堂下眾賓受酬亦如之。卒受酬者以觶跪奠於篚,興,復階下位。司正適阼階上,東面請命於主人,主人曰:「請坐於賓。」司正回,北面告於賓曰:「請賓坐。」賓曰:「唯命。」賓、主各就席坐。若賓、主公服者,則降脫履,主人先左,賓先右。司正降,復位。乃羞肉胾、醢,賓、主燕飲,行無算爵,無算樂,主人之贊者皆興焉。已燕,賓、主俱興,賓以下降自西階,主人降自東階。賓以下出,立於門外之西,東面北上,主人送於門外之東,西面再拜,賓、介逡巡而退。



 季冬之月正齒位,則縣令為主人,鄉之老人年六十以上有德望者一人為賓,次一人為介,又其次為三賓,又其次為眾賓。年六十者三豆,七十者四豆、八十者五豆,九十者及主人皆六豆。賓、主燕飲,則司正北面請賓坐,賓、主各就席立。司正適篚,跪取觶,興,實之,進,立於楹間,北面,乃揚觶而戒之以忠孝之本。賓、主以下皆再拜。司正跪奠觶,再拜,跪取觶飲,卒觶,興,賓、主以下皆坐。司正適篚,跪奠觶,興,降復位,乃行無算爵。其大抵皆如鄉飲酒禮。



\end{pinyinscope}