\article{本紀第九 懿宗 僖宗}

\begin{pinyinscope}

 懿宗昭聖恭惠孝皇帝諱漼,宣宗長子也。母曰元昭皇太后晁氏。始封鄆王。宣宗愛夔王滋,欲立為皇太子,而鄆王長,故久不決。



 大中十三年八月,宣宗疾大漸,以夔王屬內樞密使王歸長、馬公儒、宣徽南院使王居方等。而左神策護軍中尉王宗實、副使丌元實矯詔立鄆王為皇太子。癸巳,即皇帝位於柩前。王宗實殺王歸長、馬公儒、王居方。庚子,始聽政。癸卯,令狐綯為司空。尊皇太后曰太皇太后。九月庚申,追尊母為皇太后。十月辛卯,大赦。賜文武官階、勛、爵,耆老粟帛。十一月戊午,蕭鄴罷。十二月甲申,翰林學士承旨、兵部侍郎杜審權同中書門下平章事。丁酉,令狐綯罷。荊南節度使白敏中為司徒,兼門下侍郎、同中書門下平章事。是歲,雲南蠻陷播州。



 咸通元年正月,浙東人仇甫反,安南經略使王式為浙江東道觀察使以討之。二月丙申,葬聖武獻文孝皇帝於貞陵。五月,京師地震。袁王紳薨。七月,封叔心丐為信王。八月,衛王灌薨。己卯,仇甫伏誅。九月戊申,白敏中為中書令。十月,安南都護李鄠克播州。己亥,夏侯孜罷。戶部尚書、判度支畢諴為禮部尚書、同中書門下平章事。閏月乙亥,朝獻於太清宮。十一月丙子,朝享於太廟。丁丑,有事於南郊,大赦,改元。是月,慶王沂薨。十二月戊申,雲南蠻寇安南。癸亥,福王綰為司空。



 二年二月,白敏中罷。尚書左僕射、判度支杜悰兼門下侍郎、同中書門下平章事。福王綰薨。六月,鹽州刺史王寬為安南經略招討使。八月,雲南蠻寇邕州。九月,寇巂州。



 三年正月庚午,群臣上尊號曰睿文明聖孝德皇帝。大赦。是月,蔣伸罷。二月庚子,杜悰為司空。是月,隸王惴薨。湖南觀察使蔡襲為安南經略招討使。三月戊寅,歸義軍節度使張義潮克涼州。七月,武寧軍亂,逐其節度使溫璋。劍南酉川節度使夏侯孜為尚書左僕射,兼門下侍郎、同中書門下平章事。九月,嶺南西道軍亂,逐其節度使蔡京。十月丙申,封子佾為魏王,侹涼王,佶蜀王。杜悰為司徒。十一月,封叔祖緝為蘄王,叔心責榮王。雲南蠻寇安南。丙寅,降囚罪,免徐州秋稅。十二月,翼王繟薨。



 四年正月戊辰,朝獻於太清宮。己巳,朝享於太廟。庚午,有事於南郊,大赦。雲南蠻陷安南,蔡襲死之。庚辰,撫王紘為司空。二月,拜十六陵。秦州經略使高駢為安南經略招討使。四月,畢諴罷。五月己巳,翰林學士承旨、兵部侍郎楊收同中書門下平章事。戊子,杜審權罷。閏六月,杜悰罷。兵部侍郎、判度支曹確同中書門下平章事。七月辛卯朔,日有食之。免安南戶稅、丁錢二歲,弛廉州珠池禁。八月,夔王滋薨。十二月乙酉,昭義軍亂,殺其節度使沈詢。



 五年正月丙午,雲南蠻寇雋州。三月,寇邕州。四月,兵部侍郎、判戶部蕭寘同中書門下平章事。五月丁酉,瘞邕、巂州死事者。己亥,有彗星出於婁。八月丁卯,夏侯孜為司空。十月,貞陵隧隱。十一月戊戌,寘夏侯孜罷。壬寅,翰林學士承旨、兵部侍郎路巖同中書門下平章事。



 六年三月,蕭寘薨。四月,劍南東川節度使高璩為兵部侍郎、同中書門下平章事。五月,高駢及雲南蠻戰於邕州,敗之。六月,高璩薨。御史大夫徐商為兵部侍郎、同中書門下平章事。七月,封子侃為郢王。十二月,晉、絳二州地震。壬子,太皇太后崩。



 七年二月戊申,免河南府、同華陜虢四州一歲稅,湖南及桂邕容三管、岳州夏秋稅之半。三月,成德軍節度使王昭懿卒,其兄子景崇自稱留後。閏月,吐蕃寇邠、寧。五月甲辰,葬孝明太皇太后於景陵之園。六月,魏博節度使何弘敬卒,其子全皞自稱留後。八月辛卯,晝晦。十月壬申,楊收罷。是月,高駢克安南。十一月辛亥,大赦,免咸通三年以前逋負,賜文武官階、勛、爵。



 八年正月丁未,河中府、晉絳二州地震。五月丙辰,以不豫降囚罪,出宮人五百,縱神策、五坊、飛龍鷹鷂,禁延慶、端午節獻女口。七月,雨湯於下邳。壬寅,蘄王緝薨。乙巳,懷州民亂,逐其刺史劉仁規。甲子,兵部侍郎、諸道鹽鐵轉運使於琮同中書門下平章事。十一月辛丑,疾愈,避正殿,賜民年七十而痼疾及軍士戰傷者帛。十二月,信王心丐薨。



 九年正月,有彗星出於婁、胃。七月,武寧軍節度糧料判官龐勛反於桂州。十月庚午,陷宿州。丁丑,陷徐州,觀察使崔彥曾死之。十一月,陷濠州,刺史盧望回死之。右金吾衛大將軍康承訓為徐泗行營兵馬都招討使,神武大將軍王晏權為北面招討使,羽林將軍戴可師為南面招討使。十二月,龐勛陷和、滁二州,滁州刺史高錫望死之。壬申,戴可師及龐勛戰於都梁山,死之。是月,前天雄軍節度使馬舉為南面招討使,泰寧軍節度使曹翔為北面招討使。



 十年二月,殺歡州流人楊收。三月,徙封侃為威王。四月,殺鎮南軍節度使嚴撰。康承訓及龐勛戰於柳子,敗之。六月,神策軍將軍宋威為西北面招討使。戊戌,以蝗旱理囚。癸卯,徐商罷。翰林學士承旨、戶部侍郎劉瞻同中書門下平章事。八月,有彗星出於大陵。九月癸酉,龐勛伏誅。十月戊戌,免徐、宿、濠、泗四州三歲稅役。十二月壬子,雲南蠻寇嘉州。



 十一年正月甲寅,群臣上尊號曰睿文英武明德至仁大聖廣孝皇帝。大赦。雲南蠻寇黎、雅二州,及成都。二月甲申,劍南西川節度副使王建立及雲南蠻戰於城北,死之。甲午,劍南東川節度使顏慶復及雲南蠻戰於新都,敗之。三月,曹確罷。四月丙午,翰林學士承旨、兵部侍郎韋保衡同中書門下平章事。八月,殺醫待詔韓宗紹。魏博軍亂,殺其節度使何全皞,其將韓君雄自稱留後。九月丙辰,劉瞻罷。十一月辛亥,禮部尚書、判度支王鐸同中書門下平章事。



 十二年四月癸卯,路巖罷。五月庚申,理囚。十月,兵部侍郎、諸道鹽鐵轉運使劉鄴為禮部尚書、同中書門下平章事。



 十三年二月丁巳,於琮罷。刑部侍郎、判戶部趙隱為戶部侍郎、同中書門下平章事。幽州盧龍軍節度使張允伸卒,其子簡會自稱留後。三月癸酉,平州刺史張公素逐簡會,自稱留後。四月庚子,浙江東西道地震。封子保為吉王,傑壽王,倚睦王。五月乙亥,殺國子司業韋殷裕。十一月,王鐸為司徒,韋保衡為司空。



 十四年正月,沙陀寇代北。三月,迎佛骨於鳳翔。癸巳,雨土。四月,並州民產子二頭四手。壬寅,大赦。六月,不豫。王鐸罷。七月辛巳,皇帝崩於咸寧殿,年四十一。



 僖宗惠聖恭定孝皇帝諱儇,懿宗第五子也。母曰惠安皇太后王氏。始封普王,名儼。



 咸通十四年七月,懿宗疾大漸,左右神策護軍中尉劉行深、韓文約立普王為皇太子。辛巳,即皇帝位於柩前。八月癸巳,始聽政。丁未,追尊母為皇太后。乙卯,韋保衡為司徒。九月,貶保衡為賀州刺史。十月乙未,尚書左僕射蕭仿為中書侍郎、同中書門下平章事。十二月,震電。癸卯,大赦,免水旱州縣租賦,罷貢鷹鶻。雲南蠻寇黎州。



 乾符元年二月甲午,葬昭聖恭惠孝皇帝於簡陵。癸丑,降死罪以下。趙隱罷。華州刺史裴坦為中書侍郎、同中書門下平章事。四月辛卯,以旱理囚。五月乙未,裴坦薨。刑部尚書劉瞻為中書侍郎、同中書門下平章事。八月辛未,瞻薨。兵部侍郎、判度支崔彥昭為中書侍郎、同中書門下平章事。十月,劉鄴罷。吏部侍郎鄭畋為兵部侍郎,翰林學士承旨、戶部侍郎盧攜:同中書門下平章事。十一月庚寅,改元。群臣上尊號曰聖神聰睿仁哲明孝皇帝。是月,蕭仿為司空。魏博節度使韓允中卒,其子簡自稱留後。十二月,黨項、回鶻寇天德軍。雲南蠻寇黎、雅二州,河西、河東、山南東道、東川兵伐雲南。



 二年正月己丑,朝獻於太清宮。庚寅,朝享於太廟。辛卯,有事於南郊,大赦。賜文武官階、勛、爵,文宣王及二王后、三恪一子官。雲南蠻請和。四月庚辰,太白晝見。浙西突陳將王郢反。五月,右龍武軍大將軍宋皓討之。蕭仿薨。六月,濮州賊王仙芝、尚君長陷曹、濮二州,河南諸鎮兵討之。吏部尚書李蔚為中書侍郎、同中書門下平章事。幽州將李茂勛逐其節度使張公素,自稱留後。七月,以蝗避正殿,減膳。十一月,震電。



 三年二月丙子,以旱降死罪以下。三月,葬暴骸。平盧軍節度使宋威為指揮諸道招討草賊使,檢校左散騎常侍曾元裕副之。募能捕賊三百人者,官以將軍。幽州盧龍軍節度使李茂勛立其子可舉為留後。五月庚子,以旱理囚,免浙東西一歲稅。昭王汭薨。六月乙丑,雄州地震。撫王紘為太尉。七月辛巳,雄州地震。鎮海軍節度使裴璩及王郢戰,敗之。鄂王潤薨。九月乙亥朔,日有食之,避正殿。丙子,王仙芝陷汝州,執刺史王鐐。十一月,陷郢、復二州。十二月,京師地震。王仙芝陷申、光、盧、壽、通、舒六州。忠武軍節度使崔安潛為諸道行營都統,宮苑使李琢為諸軍行營招討草賊使,右威衛上將軍張自勉副之。是冬,無雪。



 四年正月丁丑,降死罪以下二等,流人死者聽收葬。崔彥昭為司空。二月,王仙芝陷鄂州。閏月,崔彥昭罷。昭義軍亂,逐其節度使高湜。宣武軍節度使王鐸檢校司徒,兼門下侍郎、同中書門下平章事。三月,宛句賊黃巢陷鄆、沂二州,天平軍節度使薛崇死之。四月壬申朔,日有食之。是月,陜州軍亂,逐其觀察使崔碣。江西賊柳彥璋陷江州,執其刺史陶祥。高安制置使鐘傳陷撫州。五月,有彗星,避正殿,減膳。六月,王鐸為司徒。庚寅,雄州地震。八月,黃巢陷隋州,執刺史崔休徵。九月,沙陀寇雲、朔二州。鹽州軍亂,逐其刺史王承顏。十月,河中軍亂,逐其節度使劉侔。十一月,尚君長降,宋威殺之。十二月,安南戍兵亂,逐桂管觀察使李瓚。江州刺史劉秉仁及柳彥璋戰,敗之。



 五年正月丁酉,王仙芝陷江陵外郛。壬寅,曾元裕及王仙芝戰於申州,敗之。元裕為諸道行營招討草賊使,張自勉副之。宋威罷招討使。二月癸酉,雲中守捉使李克用殺大同軍防禦使段文楚。己卯,克用寇遮虜軍。是月,王仙芝伏誅,其將王重隱陷饒州,刺史顏標死之。江西賊徐唐莒陷洪州。三月,黃巢隱濮州,寇河南。崔安潛罷都統。張自勉為東西面行營招討使。湖南軍亂,逐其觀察使崔瑾。四月,饒州將彭令璋克饒州,自稱刺史,徐唐莒伏誅。五月丁酉,鄭畋、盧攜罷。翰林學士承旨、戶部侍郎豆盧彖為兵部侍郎,吏部侍郎崔沆為戶部侍郎:同中書門下平章事。是日,雨雹,大風拔木。八月,大同軍節度使李國昌陷岢嵐軍。黃巢陷杭州。九月,李蔚罷。吏部尚書鄭從讜為中書侍郎、同中書門下平章事。黃巢陷越州,執觀察使崔琢。鎮海軍將張潾克越州。十月,昭義軍節度使李鈞、幽州盧龍軍節度使李可舉討李國昌。十一月丁未,河東宣慰使崔季康為河東節度、代北行營招討使。十二月甲戌,黃巢陷福州。庚辰,崔季康、李鈞及李克用戰於洪谷,敗績。是歲,天平軍節度使張裼卒,衙將崔君裕自知州事。



 六年正月,鎮海軍節度使高駢為諸道行營兵馬都統。魏王佾薨。二月,京師地震,藍田山裂,出水。河東軍亂,殺其節度使崔季康。四月庚申朔,日有食之。涼王侹薨。王鐸為荊南節度使、南面行營招討都統。五月,泰寧軍節度使李系為湖南觀察使,副之。黃巢陷廣州,執嶺南東道節度使李迢,陷安南。八月甲子,東都留守李蔚為河東節度、代北行營招討使。閏十月,黃巢陷潭、澧三州,澧州刺史李絢死之。十一月丙辰,兩日並出而鬥,戊午,河東節度使康傳圭為代北行營招討使。辛酉,黃巢陷江陵,殺李迢。丁丑,山南東道節度使劉巨容及黃巢戰於荊門,敗之。十二月壬辰,克江陵。是月,貶王鐸為太子賓客,分司東都。兵部尚書盧攜為門下侍郎、同中書門下平章事。是歲,淄州刺史曹全晸克鄆州,殺崔君裕。黃巢隱鄂、宣、歙、池四州。朗州賊周岳陷衡州,逐其刺史徐顥。荊南將雷滿陷朗州,刺史崔翥死之。石門蠻向瑰陷澧州,權知州事呂自牧死之。桂陽賊陳彥謙陷郴州,刺史董岳死之。廣明元年正月乙卯,改元。免嶺南、荊湖、河中、河東稅賦十之四。戊寅,荊南監軍楊復光、泰寧軍將段彥謨殺其守將宋浩,以常滋為節度留後。淮南將張潾及黃巢戰於大雲倉,敗之。二月丙戌,李國昌寇忻、代二州。戊戌,河東軍亂,殺其節度使康傳圭。壬子,鄭從讜罷為河東節度使、代北行營招討使。三月辛未,以旱避正殿,減膳。四月甲申,京師、東都、汝州雨雹,大風拔木。丁酉,太府卿李琢為蔚、朔招討都統。壬寅,張潾克饒州。五月,汝州防禦使諸葛爽為蔚、朔招討副使。泰寧軍將劉漢宏反。張潾及黃巢戰於信州,死之。六月,巢陷睦、婺、宣三州。江華賊蔡結陷道州。宿州賊魯景仁陷連州。七月,黃巢陷滁、和二州。辛酉,天平軍節度使曹全晸為東面副都統。辛未,劉漢宏降。李可舉及李國昌戰於藥兒嶺,敗之。八月辛卯,昭義軍亂,殺其節度使李鈞。癸卯,榮王心責為司空。是月,心責薨。九月,忠武軍將周岌殺其節度使薛能。牙將秦宗權自稱權知蔡州事。十月,黃巢陷申州。十一月,河中都虞候王重榮逐其節度使李都。黃巢陷汝州。壬戌,幸左神策軍閱武。護軍中尉田令孜為諸道兵馬都指揮制置招討使,忠武軍監軍楊復光副之。丁卯,東都留守劉允章叛附於黃巢。壬申,巢陷虢州。田令孜為汝、洛、晉、絳、同、華都統。十二月壬午,黃巢陷潼關。甲申,貶盧攜為太子賓客,分司東都。翰林學士承旨、尚書左丞王徽為戶部侍郎,翰林學士、戶部侍郎裴澈為工部侍郎:同中書門下平章事。行在咸陽。丙戌,左金吾衛大將軍張直方率武官叛附於黃巢。巢陷京師。辛卯,次鳳翔。丙申,河陽節度使諸葛爽叛附於黃巢。丁酉,次興元。庚子,廣德公主、豆盧琢、崔沆、尚書左僕射劉鄴、右僕射於琮、太子少師裴諗、御史中丞趙濛、刑部侍郎李溥,京兆尹李湯死於黃巢。是歲,雨血於靖陵。



 中和元年正月壬子,如成都。壬申,兵部侍郎、判度支蕭遘為工部侍郎、同中書門下平章事。丁丑,次成都。二月己卯,赦劍南三川。太子少師王鐸為司徒,兼門下侍郎、同中書門下平章事。淮南節度使高駢為京城四面都統。邠寧節度使李存禮討黃巢。鳳翔節度使鄭畋及巢戰於龍尾坡,敗之。邠寧將王玫陷邠州。戊戌,清平鎮使陳晟執睦州刺史韋諸,自稱刺史。三月辛亥,黃巢陷鄧州,執刺史趙戎。辛酉,鄭畋為京城西面行營都統。甲子,畋及涇原節度使程宗楚、天雄軍經略使仇公遇盟於鳳翔。是月,王徽罷。諸葛爽以河陽降。四月戊寅,王玫伏誅。程宗楚、朔方軍節度使唐弘夫及黃巢戰於咸陽,敗之。壬午,巢遯乾灞上。丁亥,復入於京師,弘夫、宗楚死之。是月,赦李國昌及其子克用以討黃巢。五月丙辰,克用寇太原,振武軍節度使契苾璋敗之。辛酉,大風,雨土。是月,劉巨容為南面行營招討使。楊復光克鄧州。六月,鄮賊鐘季文陷明州。辛卯,邠寧節度副使硃玫及黃巢戰於興平,敗績。戊戌,鄭畋為司空,兼門下侍郎、同中書門下平章事、京城四面行營都統。丙午,李克用陷忻、代二州。七月丁巳,大赦,改元。庚申,翰林學士承旨、兵部侍郎韋昭度同中書門下平章事。丙寅,神策軍將郭琪反,伏誅。辛未,田令孜殺左拾遺孟昭圖。義武軍節度使王處存為東南面行營招討使。八月,感化軍將時溥逐其節度使支詳,自稱留後。昭義軍節度使高潯及黃巢戰於石橋,敗績,其將成麟殺潯,入於潞州。己丑,眾星隕於成都。九月丙午,鄜延節度使李孝章、夏綏銀節度使拓拔思恭及黃巢戰於東渭橋,敗績。臨海賊杜雄陷臺州。辛酉,封子震為建王。己已,昭義軍戍將孟方立殺成麟,自稱留後。永嘉賦硃褒陷溫州。是秋,河東霜殺禾。十月,鳳翔行軍司馬李昌言逐其節度使鄭畋。十一月,李昌言為鳳翔節度行營招討使。鄭畋、裴澈罷。遂昌賊盧約陷處州。十二月,安南戍將閔頊逐湖南觀察使李裕,自稱留後。是歲,霍丘鎮使王緒陷壽、光二州。



 二年正月辛亥,王鐸為諸道行營都都統,承制封拜,太子少師崔安潛副之。高駢罷都統。辛未,王處存為京城東面都統,李孝章為北面都統,拓拔思恭為南面都統。二月甲戌,黃巢陷同州。己卯,太子少傅分司東都鄭畋為司空,兼門下侍郎、同中書門下平章事。丙戌,李昌言為京城西面都統,邠寧節度使硃玫為河南都統、諸穀防遏使。三月,邛州蠻阡能叛,西川部將楊行遷討之。李克用隱蔚州。六月,硃玫為京城西北面行營都統。楊行遷及阡能戰於乾溪,敗績。己亥,荊南監軍硃敬玫殺其節度使段彥謨,少尹李燧自稱留後。七月,保大軍節度留後東方逵為京城東面行營招討使。撫州刺史鐘傳陷洪州,江西觀察使高茂卿奔於江州。八月丁巳,東方逵為京城東北面行營都統,拓拔思恭為京城四面都統。魏博節度使韓簡陷孟州。九月丙戌,黃巢將硃溫以同州降。己亥,溫為右金吾衛大將軍、河中行營招討副使。是月,太原桃李實。嶺南西道軍亂,逐其節度使張從訓。平盧軍將王敬武逐其節度使安師儒,自稱留後。十月,嵐州刺史湯群以沙陀反。韓簡寇鄆州,天平軍節度使曹全晸死之,部將崔用自稱留後。諸葛爽陷孟州。十一月,荊南軍亂,衙將陳儒自稱留後丙子,湯群伏誅。是歲,關中大饑。南城賊危全諷陷撫州,危仔倡陷信州。廬州將楊行密逐其刺史郎幼復。和州刺史泰彥逐宣歙觀察使竇潏。



 三年正月,雁門節度使李克用為京城東北面行營都統。乙亥,王鐸罷。二月,魏博軍亂,殺其節度使韓簡,其將樂彥禎自稱留後。己未,建王震為太保。三月。天有聲於浙西。壬申,李克用及黃巢戰於零口,敗之。四月甲辰,又敗之於渭橋。丙午,復京師。五月,鄭畋為司徒,東都留守、檢校司空鄭從讜為司空:同中書門下平章事。淮南將張瑰陷復州。奉國軍節度使秦宗權叛附於黃巢。七月,宣武軍節度副大使硃全忠為東北面都招討使。鄭畋罷。兵部尚書、判度支裴澈同中書門下平章事。八月,黃巢、秦宗權寇陳州。淮南將韓師德陷岳州。九月,武寧軍節度使時溥為東面兵馬都統。是秋,晉州地震。十月,全椒賊許勍陷滁州。李克用陷潞州,刺史李殷銳死之。十一月壬申,劍南西川行軍司馬高仁厚及阡能戰於邛州,敗之。十二月,忠武軍將鹿晏弘逐興元節度使牛勖,自稱留後。是歲,天平軍將曹存實克鄆州。石鏡鎮將董昌逐杭州刺史路審中。



 四年正月,婺州將王鎮執其刺史黃碣,叛附於董昌。二月,鎮伏誅。浦陽將蔣瑰陷婺州。舒州賊吳迥逐其刺史高水戰。三月甲子,劍南東川節度副大使楊師立反,西川節度使陳敬瑄為西川、東川、山南西道都指揮招討使。前杭州刺史路審中陷鄂州。五月辛酉,硃全忠及黃巢戰,敗之。辛未,河東節度使李克用及巢戰於宛句,敗之。癸酉,高仁厚為劍南東川節度使以討楊師立。壬午,福建團練副使陳巖逐其觀察使鄭鎰,自稱觀察使。六月乙卯,赦劍南三川。瘞京畿骸骨。七月辛酉,楊師立伏誅。壬午,黃巢伏誅。九月,山南西道節度使鹿晏弘反。十月,蕭遘為司空。十一月,鹿晏弘陷許州,殺西度使周岌,自稱留後。十二月甲午,荊南行軍司馬張瑰逐其節度使陳儒,自稱留後。盜殺義昌軍節度使王鐸。是歲,關中大饑。濮州刺史硃宣逐天平軍節度使曹存實,自稱留後。武昌軍將杜洪陷岳州。



 光啟元年正月庚辰,荊南軍將成汭陷歸州。是月,王緒陷汀、漳二州。南康賊盧光稠陷虔州。三月丁卯,至自成都。己巳,大赦,改元。時溥為蔡州四面行營兵馬都統。蕭遘為司徒,韋昭度為司空。四月,吳迥伏誅。秦宗權陷襄州,山南東道節度使劉巨容奔於成都。武當賊馮行襲陷均州,逐其刺史呂燁。五月,群臣上尊號曰至德光烈皇帝。六月,幽州盧龍軍亂,殺其節度使李可舉,其將李全忠自稱留後。壬戌,秦宗權陷東都。七月,義昌軍亂,逐其節度使楊全玫,衙將盧彥威自稱留後。八月,光州賊王潮執王緒。甲寅,殺右補闕常浚。樂彥楨殺洺州刺史馬爽。九月,河中節度使王重榮反,邠寧節度使硃玫討之。十月癸丑,硃全忠及秦宗權戰於雙丘,敗績。十一月,河東節度使李克用叛附於王重榮,重榮及克用寇同州,刺史郭璋死之。十二月癸酉,硃玫及王重榮、李克用戰於沙苑,敗績。乙亥,克用犯京師。丙子,如鳳翔。



 二年正月辛巳,鎮海軍將張鬱陷常州。戊子,如興元。癸巳,硃玫叛,寇鳳翔。二月,鄭從讜為太傅。三月壬午,山南西道節度使石君涉奔於鳳翔。遂州刺史鄭君雄陷漢州。丙申,次興元。戊戌,御史大夫孔緯、翰林學士承旨、兵部尚書杜讓能為兵部侍郎、同中書門下平章事。是春,成都地震,鳳翔女子化為丈夫。四月乙卯,硃玫以嗣襄王煴入於京師。五月丙戌,有星孛於箕、尾。武寧軍將丁從實陷常州,逐其刺史張鬱。六月,淮西將黃皓殺欽化軍節度使閔頊。衡州刺史周岳陷潭州,自稱節度使。七月,秦宗權陷許州,忠武軍節度使鹿晏弘死之。八月,王潮陷泉州,刺史廖彥若死之。幽州盧龍軍節度使李全忠卒,其子匡威自稱留後。九月,有星隕於揚州,戊寅,靜難軍將王行瑜陷興、鳳二州。十月丙午,嗣襄王煴自立為皇帝,尊皇帝為太上元皇聖帝。硃全忠陷滑州,執義成軍節度使安師儒。丙辰,杭州刺史董昌攻越州,浙東觀察使劉漢宏奔於臺州。是月,河陽節度使諸葛爽卒,其子仲方自稱留後。神策行營先鋒使滿存克興、鳳二州。感義軍節度使楊晟陷文州。武寧軍將張雄陷蘇州。十一月庚子,秦宗權鄭州。十二月,魏州地震。丙午,臺州刺史杜雄執劉漢宏,降於董昌。昌自稱浙東觀察使。丙辰,硃玫伏誅。丁巳。煴伏誅。秦宗權陷孟州,諸葛仲方奔於汴州。是歲,天平軍將硃瑾逐泰寧軍節度使齊克讓,自稱留後。湘陰賊鄧進思陷岳州。杜洪陷鄂州,自稱武昌軍節度留後。



 三年三月癸未,蕭遘、裴澈、兵部侍郎鄭昌圖有罪伏誅。壬辰,如鳳翔。鄭從讜罷。韋昭度為司徒。癸巳,鎮海軍將劉浩逐其節度使周寶,度支催勘使薛朗自稱知府事。四月甲辰,六合鎮遏使徐約陷蘇州,逐其刺史張雄。甲子,淮南兵馬使畢師鐸陷揚州,執其節度使高駢。是月,維州山崩。五月甲戌,宣歙觀察使秦彥入於揚州。癸未,秦宗權陷鄭州。六月,陷孟州,河陽將李罕之人於孟州,張全義入於東都。己酉,鳳翔節度使李昌符反。庚戌,犯大安門,不克,奔於隴州。壬子,武定軍節度使李茂貞為隴州招討使。丁巳,護國軍將常行儒殺其節度使王重榮,其兄重盈自稱留後。壬戌,亳州將謝殷逐其刺史宋袞。七月丁亥,降死罪以下,貞觀、開元、建中、興元功臣後予一子九品正員官,減常膳三之一,賜民九十以上粟帛。七月,李昌符伏誅。八月,韋昭度為太保。壬寅,謝殷伏誅。硃全忠陷亳州。壬子,陷曹州,刺史丘弘禮死之。九月,戶部侍郎、判度支張浚為兵部侍郎、同中書門下平章事。秦彥殺高駢。十月丁未,硃全忠陷濮州。甲寅,封子升為益王。杭州刺史錢升〗陷常州。丁卯,升殺周寶。是月,秦宗權將孫儒寇揚州。十一月壬申,廬州刺史楊行密陷揚州,秦彥、畢師鐸奔於孫儒。十二月癸巳,淮西將趙德諲陷江陵,荊南節度使張瑰死之。硃全忠為東南面招討使。饒州刺史陳儒陷衢州。上蔡賊馮敬章陷蘄州。



 文德元年正月甲寅,孫儒殺秦彥、畢師鐸。癸亥,硃全忠為蔡州四面行營都統。丙寅,薛朗伏誅。錢升陷潤州。二月乙亥,不豫。乙丑,至自鳳翔。庚寅,竭於太廟。大赦,改元。是月,魏博軍亂,殺其節度使樂彥禎,其將羅弘信自稱權知留後。三月戊戌朔,日有食之,既。壬寅,疾大漸,立壽王傑為皇太弟,知軍國事。癸卯,皇帝崩於武德殿,年二十七。



 贊曰:唐自穆宗以來八世,而為宦官所立者七君。然則唐之衰亡,豈止方鎮之患?蓋朝廷天下之本也,人君者朝廷之本也,始即位者人君之本也。其本始不正,欲以正天下,其可得乎?懿、僖當唐政之始衰,而以昏庸相繼;乾符之際,歲大旱蝗,民悉盜起,其亂遂不可復支,蓋亦天人之會歟!



\end{pinyinscope}