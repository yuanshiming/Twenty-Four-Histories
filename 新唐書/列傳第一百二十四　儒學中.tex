\article{列傳第一百二十四 儒學中}

\begin{pinyinscope}

 郎餘令,定州新樂人。祖穎,字楚之,與兄蔚之俱有名。隋大業中,為尚書民曹朗中央毛澤東主席著作編輯出版委員會編輯。收入毛澤東1949,蔚之位左丞。煬帝語稱「二郎」。武德時,楚之以大理卿封常山郡公,與李綱、陳叔達定律令。持節諭山東,為竇建德所獲,脅以白刃,終不屈。賊平,以老乞身,謚曰平。



 餘令博於學,擢進士第,授霍王元軌府參軍事。從父知年,亦為王友。元軌每曰:「郎家二賢皆入府,不意培塿而松柏為林也。」徙幽州錄事參軍。有為浮屠者,積薪自焚,長史裴煚率官屬將觀焉,餘令曰:「人好生惡死,情也。彼違蔑教義,反其所欲,公當察之,毋輕往。」煚試廉按,果得其奸。



 孝敬在東宮,餘令以梁元帝有《孝德傳》,更撰《後傳》數十篇獻太子,太子嗟重。改著作佐郎,卒。



 兄餘慶,為吏清而刻於法。高宗時,為萬年令,道無掇遺。累遷御史中丞,務謙謹下人,引御史坐與議論。吏部侍郎楊思玄倨貴,視選者不以禮,餘慶劾免其官。久之,出為蘇州刺史。坐累下遷交州都督。



 驩州司馬裴敬敷與餘慶雅故,以事笞餘慶婢父,婢方嬖,譖敬敷死獄中。又裒貨無藝,民詣闕訴之,使者十輩臨按,餘慶謾讕,不能得其情。最後,廣州都督陳善弘按之,餘慶自恃在朝廷久,明法令,輕善弘,不置對。善弘怒曰:「舞文弄法,吾不及君;今日以天子命治君,吾力有餘矣。」欲搒械之,餘慶懼,服罪。高宗詔放瓊州。會赦當還,朝廷惡其暴,徙春州。



 始,餘慶治萬年,父知運嫌其酷,將杖之,餘慶避免。父嘆曰:「國家用之矣,吾尚奈何!」及為御史中丞,復嘆曰:「郎氏危矣!」以憂死。餘慶卒以貪殘廢。



 徐齊聃,字將道,湖州長城人,世客馮翊。梁慈源侯整四世孫。八歲能文,太宗召試,賜所佩金削刀。舉弘文生,調曹王府參軍。高宗時,為潞王府文學、崇文館學士,侍皇太子講,修書於芳林門。時姑為帝婕妤,嫌以恩進,故求出為桃林令。召為沛王侍讀,再遷司議郎,皆不就。累進西臺舍人。



 咸亨初,詔突厥酋長子弟得事東宮,齊聃上書諫,以為:「氈裘冒頓之裔,解辮削衽,使在左右,非所謂『恭慎威儀,以近有德』、『任官惟賢才,左右惟其人』之義。」又長孫無忌以讒死,家廟毀頓,齊聃言於帝曰:「齊獻公,陛下外祖,雖後嗣有罪,不宜毀及先廟。今周忠孝公廟反崇飾逾制,恐非所以示海內。」帝寤,有詔復獻公官,以無忌孫延主其祀。



 齊聃善文誥,帝愛之,令侍皇太子及諸王屬文,以職樞劇,許間日一至。坐漏禁中事,貶蘄州司馬。又流欽州。卒,年四十四。睿宗時,贈禮部尚書。子堅。



 堅,字元固,幼有敏性。沛王聞其名,召見,授紙為賦,異之。十四而孤,及壯,寬厚長者。舉秀才及第,為汾州參軍事,遷萬年主簿。



 天授三年,上言:「書有五聽,令有三覆,慮失情也。比犯大逆,詔使者勘當,得實輒決。人命至重,萬有一不實,欲訴無由,以就赤族,豈不痛哉!此不足檢下之奸亂,適長使人威福耳。臣請如令覆奏,則死者無恨。又古者罰不逮嗣,故卻芮亂國而缺升諸朝,嵇康蒙戮而紹死於難,則於它親不復致疑。今選部廣責逆人親屬,至無服者尚數十條。且詔書『與逆同堂親不任京畿,緦麻親不得侍衛』,臣請如詔書外,一切不禁,以申曠蕩。」



 聖歷中,東都留守楊再思、王方慶共引為判官。方慶善《禮》學,嘗就質疑晦,堅為申釋,常得所未聞。屬文典厚,再思每目為鳳閣舍人樣。與徐彥伯、劉知幾、張說與修《三教珠英》,時張昌宗、李嶠總領,彌年不下筆,堅與說專意撰綜,條匯粗立,諸儒因之,乃成書。累遷給事中,封慈源縣子。



 中宗怒韋月將,欲即斬之,堅奏盛夏生長,請須秋乃決,時申救者亦眾,得以搒死。俄以禮部侍郎為修文館學士。



 睿宗即位,授太子左庶子兼崇文館學士,修史,進東海郡公,遷黃門侍郎。時監察御史李知古兵擊姚州渳河蠻,降之,又請築城,使輸賦徭。堅議:「蠻夷羈縻以屬,不宜與中國同法,恐勞師遠伐,益不償損。」不聽,詔知古發劍南兵築城堡,列州縣。知古因是欲誅其豪酋,入子女為奴婢,蠻懼,殺知古,相率潰叛,姚、巂路閉不通者數年。



 初,太平公主用事,武攸暨屢邀請堅,堅不許。又以妻岑羲女弟,固辭機密,轉太子詹事,曰:「吾非求高,逃禍耳。」羲敗,不染於惡,出為絳州刺史。數外徙,久乃遷秘書監、左散騎常侍。



 玄宗改麗正書院為集賢院,以堅充學士,副張說知院事。帝大酺集賢,幔舍在百司上,說令揭大榜以侈其寵,堅見,遽命撤之,曰:「君子烏取多尚人!」從上泰山,以參定儀典,加光祿大夫。堅於典故多所諳識,凡七當撰次高選。卒,年七十餘,帝悼惜,遣使就吊,贈太子少保,謚曰文。



 齊聃姑為太宗充容,仲為高宗婕妤,皆明圖史,議者以堅父子如漢班氏。



 子嶠,字巨山。開元中為駕部員外郎、集賢院直學士,遷中書舍人、內供奉、河南尹。封慈源縣公。父子相次為學士,自祖及孫,三世為中書舍人。



 沈伯儀,湖州吳興人。武后時,為太子右諭德。



 初,太常少卿韋萬石議明堂大享事,上言:「鄭玄說祀五天帝,王肅謂祀五行帝。《貞觀禮》從玄,至《顯慶禮》祀昊天上帝,乾封詔書祀五天帝兼祀昊天,上元詔書從《貞觀禮》,儀鳳初詔祀事一用周制。今應何樂?」高宗乃詔尚書省集諸儒議,未能定。於是大享參用《貞觀》、《顯慶》二禮。



 垂拱元年,成均助教孔玄義奏:「嚴父莫大配天,天於萬物為最大,推父偶天,孝之大,尊之極也。《易》稱『先王作樂崇德,殷薦之上帝,以配祖、考』。上帝,天也。昊天之祭,宜祖、考並配,請以太宗、高宗配上帝於圓丘,神堯皇帝配感帝南郊。《祭法》:『祖文王,宗武王。』祖,始也;宗,尊也。一名而有二義。《經》稱『宗祀文王』,文王當祖而云宗,包武王以言也。知明堂以祖、考配,與二經合。」伯儀曰:「有虞氏禘黃帝而郊嚳,祖顓頊而宗堯;夏后氏禘黃帝而郊鯀,祖顓頊而宗禹;殷人禘嚳而郊冥,祖契而宗湯;周人禘嚳而郊稷,祖文王而宗武王。鄭玄曰:『禘、郊、祖、宗,皆配食也。祭昊天圓丘曰禘,祭上帝南郊曰郊,祭五帝、五神明堂曰祖、宗。』此為最詳。虞夏退顓頊郊嚳,殷舍契郊冥,去取違舛,惟周得禮之序,至明堂始兩配焉。文王上配五帝,武王下配五神,別父子也。《經》曰:『嚴父莫大於配天。』又曰:『宗祀文王於明堂,以配上帝。』不言嚴武王以配天,則武王雖在明堂,未齊於配,雖同祭而終為一主也。緯曰:『後稷為天地主,文王為五帝宗。』若一神而兩祭之,則薦獻數瀆,此神無二主也。貞觀、永徽禮實專配,由顯慶後始兼尊焉。今請以高祖配圓丘、方澤,太宗配南北郊,高宗配五天帝。」鳳閣舍人元萬頃、範履冰等議:「今禮昊天上帝等五祀,咸奉高祖、太宗兼配,以申孝也。《詩昊天》章『二後受之』,《易》『薦上帝,配祖、考』,有兼配義。高祖、太宗既先配五祀,當如舊。請奉高宗歷配焉。」自是郊、丘,三帝並配雲。



 伯儀歷國子祭酒、修文館學士,卒。



 路敬淳,貝州臨清人。父文逸,遇隋季大亂,闔門死於盜。文逸遁免,流離辛苦,自傷家多難,閉口不食,行者哀其窮,強飲食之,更負以行,乃得脫。貞觀末,官申州司馬。



 敬淳少力學,足不履門。居親喪,倚廬不出者三年。服除,號慟入門,形容臒毀,妻不之識。後擢進士第。天授中,再遷太子司議郎兼修國史、崇賢館學士。數受詔纂輯慶恤儀典,武后稱之。尤明姓系,自魏、晉以降,推本其來,皆有條序,著《姓略》、《衣冠系錄》等百餘篇。後坐綦連耀交通,下獄死。神龍初,贈秘書少監。



 弟敬潛,少與敬淳齊名,歷懷州錄事參軍,亦坐耀事系獄,免死。後為遂安令。先是,令多死,敬潛欲辭,妻曰:「君不死獄而得全,非生死有命邪?」從之。到官,有梟嘯其屏,鼠數十走於前,左右驅之,擁杖而號,敬潛不為懼。久之,遷衛令,位中書舍人。



 唐初,姓譜學唯敬淳名家。其後柳沖、韋述、蕭穎士、孔至各有撰次,然皆本之路氏。



 王元感,濮州鄄城人。擢明經高第,調博城丞。紀王慎為兗州都督,厚加禮,敕其子東平王續往受業。天授中,稍遷左衛率府錄事,兼直弘文館。武后時,已郊,遂享明堂,封嵩山,詔與韋叔夏等草儀具,眾推其練洽。轉四門博士,仍直弘文館。



 年雖老,讀書不廢夜。所撰《書糾謬》、《春秋振滯》、《禮繩愆》等凡數十百篇,長安時上之,丐官筆楮寫藏秘書。有詔兩館學士、成均博士議可否。祝欽明、郭山惲、李憲等本章句家,見元感詆先儒同異,不懌,數沮詰其言,元感緣罅申釋,竟不詘。魏知古見其書,嘆曰:「《五經》指南也。」而徐堅、劉知幾、張思敬等惜其異聞,每為助理,聯疏薦之,遂下詔褒美,以為儒宗。拜太子司議郎兼崇賢館學士。中宗以東宮官屬,加朝散大夫,卒。



 元感初著論三年之喪以三十有六月,譏詆諸儒。鳳閣舍人張柬之破其說曰:「三年之喪,二十五月,由古則然。《春秋》僖公三十三年十二月『乙巳,公薨』,文公二年冬『公子遂如齊納幣』。左氏曰:『禮也。』杜預謂:『僖喪終是年十一月,納幣在十二月。故謂之禮。』《公羊傳》:『納幣不書,此何以書?譏。何以譏?三年之內不圖婚。』何休曰:『僖以十二月薨,未終二十五月,故譏云。』杜預推歷乙巳乃在十一月,《經》書十二月為誤。文公元年四月,葬僖公。《傳》曰:『緩。』夫諸侯之葬五月,若十二月薨,五月不得雲緩,則十一月明甚。然二家所競,乃一月,非一歲,則二十五月,其一驗也。《書》稱成湯既沒,太甲元年曰:『惟元祀,十有二月,伊尹祀於先王,奉嗣王祇見厥祖。』孔安國曰:『湯以元年十一月崩。』此則明年祥,又明年大祥,故下言『惟三祀,十有二月朔,尹以冕服,奉嗣王歸於亳』。是十一月服除而冕。《顧命》:『四月哉生魄,王不懌。翌日乙丑,王崩。丁卯,命作冊度。越七日癸酉,伯相命士須材。』則成王崩至康王麻冕黼裳凡十日,康王始見廟。明湯崩在十一月。比殯訖,以十二月祗見其祖。《顧命》見廟訖『諸侯出廟門俟』,《伊訓》言『祗見厥祖,侯甸群後咸在』,則崩及見廟,周因於殷也,非元年前復有一歲,此二十五月之二驗。《禮》:『三年之喪,二十五月而畢,哀痛未盡,然而以是為斷者,送死有已,服生有節。』又曰:『期而小祥,食菜果;又期而大祥,有醯醬;中月而禫,食酒肉。』又曰:『再期之喪,三年;期之喪,二年;九月、七月之喪,三時;五月之喪,二時;三月之喪,一時。』此二十五月之三驗。《儀禮》:『期而小祥,又期而大祥,中月而禫,是月也,吉祭。』此二十五月之四驗。《書》、《春秋》、《禮》皆周公、尼父所定,敢問此可為法否?昔鄭玄以中月而禫者,內容一月,自喪至禫,凡二十七月。今既用之,而二十五月初無疑論。大抵子於親喪,有終身之痛,創巨者日久,痛深者愈遲,何歲月而止乎?故練而慨然,悲慕未盡,而踴擗之情差未;祥而廓然,哀傷已除,而孤藐之懷更劇。此情之所致,寧外飾哉?故先王立其中制,使情文兩稱,是以祥則縞帶素紕,禫則無不佩。夫去衰麻,襲錦縠,行道之人皆不忍,直為節之以禮,叵如之何。故仲由不能過制為姊服,孔鯉不能過期哭母,彼詎不懷?畏名教之嚴也。」當世謂柬之言不詭聖人,而元感論遂廢。



 王紹宗,字承烈,梁左民尚書銓曾孫。系本瑯邪,徙江都云。少貧俠,嗜學,工草隸,客居僧坊,寫書取庸自給,凡三十年。庸足給一月即止,不取贏,人雖厚償,輒拒不受。



 徐敬業起兵,聞其行,以幣劫之,稱疾篤。復令唐之奇強遣,不肯赴,敬業怒,將殺之,之奇曰:「彼人望也,殺之沮士心,不可。」由是免。事平,大總管李孝逸表其節,武后召赴東都,謁殿中,褒慰良厚,擢太子文學。累進秘書少監,使侍皇太子。紹宗雅修飾,當時公卿莫不慕悅其風,張易之兄弟亦頗結納。易之誅,坐廢,卒於家。



 嘗與人書曰:「鄙夫書無工者,特由水墨之積習耳。常精心率意、虛神靜思以取之。吳中陸大夫常以餘比虞君,以不臨寫故也。聞虞被中畫腹,與余正同。」虞,即世南也。



 紹宗兄玄宗,隱嵩山,號太和先生,傳黃老術。



 彭景直,瀛州河間人。中宗景龍末,為太常博士。時獻、昭、乾三陵皆日祭,景直上言:



 在禮,陵不日祭,宗廟有月祭,故王者設廟、祧、壇、



 墠,為親疏多少之殺。立七廟、一壇、一墠。曰考廟,曰王考廟,曰皇考廟,曰顯考廟,皆月祭。遠廟為祧,享嘗乃止。去祧為壇,去壇為墠,有禱祭之,無禱乃止。譙周曰:「天子始祖、高祖、曾祖、祖、考之廟,皆朔加薦,以象生時朔食,號月祭,二祧廟不月祭。」則古無日祭者。今諸陵朔、望進食,近古之殷事;諸節進食,近古之薦新。鄭玄曰:「殷事,月之朔、半,薦新奠也。」於《儀禮》,朔、半日,猶常日朝夕也,既大祥,即四時焉,此其祭皆在廟云。近世始以朔、望諸節祭陵寢,唯四時及臘,五享於廟。尋經質禮,無日祭於陵之文。漢時,京師自高祖下至宣帝,與太上皇、悼皇考陵旁立廟。園各有寢、便殿,故日祭諸寢,月祭諸便殿。貢禹以禮節煩數,白元帝願罷郡、國廟。丞相韋玄成等後因議七廟外寢園皆無復修。議者亦以祭不欲數,宜復古四時祭於廟。劉歆引《春秋外傳》曰:「祖、禰日祭,曾、高月祀,二祧時享,壇、墠歲貢。」魏、晉以降,不祭墓。唐家擇古作法,臣謂宜罷諸陵日祭,如禮便。



 帝不從,因下詔:「有司言諸陵不當日進食。夫禮以人情為之沿革,何專古而泥所聞?乾陵宜朝晡進奠,昭、獻陵日一進,或所司乏於費,可減朕常膳為之。」



 帝崩,葬定陵,有司議以和思皇后祔葬,後為武后所殺,不得其喪所,將以招魂合諸梓宮,景直曰:「招魂古無傳,不可。請如橋山藏衣冠故事,納後禕衣,復寢宮,舉衣魂輅,告以太牢,內之方中,奉帝梓棺右,覆以夷衾。」眾當其言,制曰:「可。」景直後歷禮部郎中卒。



 盧粲,幽州範陽人,後魏侍中陽烏五世孫。祖彥卿,亦善著書。粲始冠,擢進士第。神龍中,累遷給事中。時節愍太子立,韋後疾之,諷中宗以衛府封物給東宮,粲駁奏:「太子匕鬯主,歲時服用,宜取於百司。《周禮》:諸用財器,『歲終則會,唯王及太子不會』。今乃與諸王等夷,非所謂憲章古昔者。」詔可。



 武崇訓死,詔墓視陵制,粲曰:「凡王、公主墓,無稱陵者,唯永泰公主事出特制,非後人所援比。崇訓塋兆,請視諸王。」詔曰:「安樂公主與永泰不異,崇訓於主當同穴,為陵不疑。」粲固執,以「陵之稱,本施尊極,雖崇訓之親,不及雍王,雍墓不稱陵,崇訓緣主而得假是名哉?」詔可。主大怒,出粲陳州刺史。粲曰:「茍所論得行,雖遠何憚!」開元初,為秘書少監。



 其從父行嘉,仕為雍王記室,亦以學聞。



 粲累封固安縣侯,終邠王傅,謚曰景。



 尹知章,絳州翼城人。少雖學,未甚通解,忽夢人持巨鑿破其心,內若劑焉,驚悟,志思開徹,遂遍明《六經》。諸生嘗講授者,更北面受大義。



 長安中,擢定王府文學。遷太常博士。中宗時,或建言以涼武昭王為七廟始祖,知章議:「武昭遠世,非王業所因。」乃止。出為陸渾令,坐事,輒棄官去。時散騎常侍解琬亦罷歸,與知章覃思經術,舉欣欣然。張說表諸朝,擢禮部員外郎,轉國子博士。馬懷素緒定秘書,奏知章是正文字。



 每休沐,講授未始輟。於《易》、《老》、《莊》書尤縣解。弟子貧者,賙給之。性和厚,人不見有喜慍。未嘗問產業,其子欲廣市樵米為歲中計,知章曰:「如而計,則貧人何以取資?且吾烏應奪民利邪?」卒官。所注傳頗多,行於時。門人孫季良等頌其德,刻著東都國子監門外。



 季良,偃師人,一名翌,仕歷左拾遺、集賢院直學士。



 張齊賢,陜州陜人。聖歷初,為太常奉禮郎。



 武后詔百官議告朔於明堂,講時令,布政事,京官九品以上、四方朝集使皆列於廷。太常博士闢閭仁住曰:「經無天子月告朔。唯《玉藻》:『天子聽朔南門之外。』《周太宰》:『正月之吉,布政於邦國都鄙。』干寶曰:『建子月告朔日也。』此《玉藻》聽朔同誼。今元日讀時令,合古聽朔事。獨鄭玄以秦制《月令》有五帝五官,因言『聽朔必以特牲告時帝及神,以文王、武王配。』其言非是。《月令》曰『其帝太昊,其神句芒』,謂宣令告人,使奉時務業,月皆有令,故云,非天子月朔以配帝祭也。告朔者,諸侯禮也,《春秋》:『既視朔,遂登臺。』玄又說人君月告朔於廟,其祭為朝享。魯自文公始不視朔,明非天子所行。玄謂告帝即人帝,神即重、黎、五官,不言天子拜祭。臣請罷告朔、月祭,以應古禮。」齊賢不韙其說,質曰:「穀梁氏稱『閏月,天子不告朔』,它月故告朔矣。左氏言魯『不告閏朔,為棄時政』,則諸侯雖閏告朔矣。《周太史》『頒朔於邦國』,《玉藻》『閏月,王居門』,是天子雖閏亦告朔。二家去聖不遠,載天子、諸侯告朔事,顯顯弗繆。今議者乃以《太宰》正月之吉,布治邦國,而言天子元日一告朔,殊失其旨。一歲之元,六官自布所職之典。干寶謂吉為朔,故世人謬吉為告,據繆失經,不得為法。議者又引左氏說,專在諸侯,不知《玉藻》與左說正同,而獨於天子言歲首一告,何去取之恣也!又謂時帝,五人帝也。玄於時帝包天人,故以文、武作配,是並告兩五帝為不疑。諸侯受朔天子,藏於廟。天子受朔於天,宜在明堂,故告時帝,配祖考。議者曰:『天子月告祭頒朔,則諸侯安得藏之?故太宰歲首布一歲事,太史頒之也。』是不然。《周太史》『頒朔邦國』,是總頒十二朔於諸侯;天子猶月告者,頒官府都鄙也。內外異言之也。禮不可罷。」鳳閣侍郎王方慶又推言:「明堂,布政之宮,所以明天氣,統萬物也。漢儒以明堂、太廟為一,宗祀其祖,而配上帝。取宗祀曰清廟,正室為太室,向陽為明堂,建學為太學,圜水為闢雍,異名同事,古之制也。天子以正月上辛總受十二月政於南郊,還藏於祖廟,月取一政,班之明堂。諸侯則受於天子,藏之祖廟,月取一政,行之於國。王者以其禮告廟,謂之告朔;視月之政,謂之視朔。《玉藻》:『玄冕而朝日東門之外,聽朔南門之外。』鄭玄說:『明堂在國陽,就其時之堂而聽朔焉。卒事,宿路寢。』今元日通天宮受朝,有司遂讀時令、布政,古之禮也。舊說天子歲入明堂者十八:大享,一;月告朔,十二;四時迎氣,四;巡狩之歲,一。今議者唯許歲首一入,不亦隘乎?陛下幸建明堂,遵用告朔事,若月一聽,則近於煩,每孟月視朔,惟制定其禮,臣下不敢專。」成均博士吳楊吾等共言:「秦滅學,告朔廢。今用四孟月、季夏,至明堂告五時帝堂上,請兼如齊賢、方慶議。」不數歲,禮亦廢。



 久之,齊賢遷博士。時東都置太社,禮部尚書祝欽明問禮官博士:「周家田主用所宜木,今社主石,奈何?」齊賢與太常少卿韋叔夏、國子司業郭山惲、尹知章等議:「《春秋》:『君以軍行,祓社釁鼓,祝奉以從。』故曰:『不用命,戮於社。』社稷主用石,以可奉而行也。崔靈恩曰:『社主用石,以地產最實歟!』《呂氏春秋》言『殷人社用石』。後魏天平中,遷太社石主,其來尚矣。周之田主用所宜木,其民間之社歟!非太社也。」於是舊主長尺有六寸,方尺七寸,問博士云何,齊賢等議:「社主之制,禮無傳。天子親征,載以行,則非過重。《禮》:『社祭土,主陰氣。』《韓詩外傳》:『天子太社方五丈,諸侯半之。』五,土數。社主宜長五尺,以準數五;方二尺,以準陰偶;剡其上,以象物生;方其下,以象地體;埋半土中,本末均也。請度以古尺」云。又問:「社稷壇隨四方用色,而中不數尺,冒黃土,謂何?」齊賢等曰:「天子太社,度廣五丈,分四方,上冒黃土,象王者覆被四方,然則當以黃土覆壇上。舊壇上不數尺,覆被之狹,乖於古。」於是以方色飾壇四面及陛,而黃土全覆上焉。祭牲皆太牢。其後改先農曰「帝社」,又立「帝稷」,皆齊賢等參定。



 中宗即位,因武後東都廟改為唐廟,議滿七室,以涼武昭王為始祖。齊賢上議:「《禮》,天子七廟,尊始封君曰太祖,百代不遷,始祖無聞焉。殷自玄王至湯,周后稷至武王,皆出太祖後,合食有序。景皇帝始封唐,實為太祖,以世數近,故尚在昭穆。今乃上引武昭王為始祖,異乎殷、周之本卨、稷也。卨、稷興胙,景皇帝是也。昭王國不世傳,後嗣失守。景帝實始封唐,子孫是承。若近舍唐,遠引涼,不見其可。且魏不祖曹參,晉不祖司馬卬,宋不祖楚元王,齊、梁不祖蕭何,陳、隋不祖胡公、楊震,今謂昭王為祖,可乎?漢以周郊後稷,議欲郊堯,杜林以為周興自後稷,漢業特起,功不緣堯,卒不果郊。武德初定,去昭王尤近,不托祖者,不可故也。今而立之,非祖宗意。景皇失位,神弗臨享,殆非詒厥孫謀者。」博士劉承慶、尹知章又言:「受命之君,王跡有淺深,代系有遠邇。祖以功,昭穆以親。有功者不遷,親盡者毀。今不宜以廟數未備,引當遷之主於昭穆上,茍充七室也。景皇帝既號太祖,以世淺猶在六室位,則室未當有七,非天子廟不當七也。大帝神主既祔,宣皇帝當遷。宣非始祖,又無宗號,親盡而遷,不可復立。請仍為六室。」詔宰相詳裁。於是祝欽明等上言:「博士等三百人為兩說:齊賢等不祖武昭王,劉承慶等請遷宣皇帝。臣等欲皆可其奏。」詔可。俄以孝敬皇帝為義宗,列於廟為七室。西京太廟亦如之。



 齊賢遷累諫議大夫,卒。



 柳沖,蒲州虞鄉人,隋饒州刺史莊曾孫。父楚賢,大業中為河北縣長。高祖兵興,堯君素據郡固守,楚賢說曰:「隋之亡,天下共知。唐公名在圖籙,動以誠信,豪英景赴,天所贊也。君子見幾而作,俟終日邪?」君素不從,楚賢潛行自歸,授侍御史。貞觀中,持節冊拜突厥,辭其遺不受。歷交、桂二州都督、杭州刺史,皆有名。



 沖好學,多所研總。天授初,為司府寺主簿,詔遣安撫淮南,使有指,封河東縣男。中宗景龍中,遷左散騎常侍,修國史。



 初,太宗命諸儒撰《氏族志》,甄差群姓。其後門胄興替不常,沖請改修其書,帝詔魏元忠、張錫、蕭至忠、岑羲、崔湜、徐堅、劉憲、吳兢及沖共取德、功、時望、國籍之家,等而次之。夷蕃酋長襲冠帶者,析著別品。會元忠等繼物故,至先天時,復詔沖及堅、兢與魏知古、陸象先、劉子玄等討綴,書乃成,號《姓系錄》。歷太子賓客、宋王師、昭文館學士,以老致仕。開元初,詔沖與薛南金復加刊竄,乃定。



 後柳芳著論甚詳,今刪其要,著之左方。芳之言曰:



 氏族者,古史官所記也。昔周小史定系世,辯昭穆,故古有《世本》,錄黃帝以來至春秋時諸侯、卿、大夫名號繼統。左丘明傳《春秋》,亦言:「天子建德,因生以賜姓,胙之土,命之氏;諸侯以字為氏,以謚為族。」昔堯賜伯禹姓曰姒,氏曰有夏;伯尼姓曰姜,氏曰有呂。下及三代,官有世功,則有官族,邑亦如之。後世或氏於國,則齊、魯、秦、吳;氏於謚,則文、武、成、宣;氏於官,則司馬、司徒;氏於爵,則王孫、公孫;氏於字,則孟孫、叔孫;氏於居,則東門、北郭;氏於志,則三烏、五鹿;氏於事,則巫、乙、匠、陶。於是受姓命氏,粲然眾矣。



 秦既滅學,公侯子孫失其本系。漢興,司馬遷父子乃約《世本》修《史記》,因周譜明世家,乃知姓氏之所由出,虞、夏、商、周、昆吾、大彭、豕韋、齊桓、晉文皆同祖也。更王迭霸,多者千祀,少者數十代。先王之封既絕,後嗣蒙其福,猶為強家。



 漢高帝興徒步,有天下,命官以賢,詔爵以功,誓曰:「非劉氏王、無功侯者,天下共誅之。」先王公卿之胄,才則用,不才棄之,不辨士與庶族,然則始尚官矣。然猶徙山東豪傑以實京師,齊諸田,楚屈、景,皆右姓也。其後進拔豪英,論而錄之,蓋七相、五公之所由興也。



 魏氏立九品,置中正,尊世胄,卑寒士,權歸右姓已。其州大中正、主簿,郡中正、功曹,皆取著姓士族為之,以定門胄,品藻人物。晉、宋因之,始尚姓已。然其別貴賤,分士庶,不可易也。於時有司選舉,必稽譜籍,而考其真偽。故官有世胄,譜有世官,賈氏、王氏譜學出焉。由是有譜局,令史職皆具。過江則為「僑姓」,王、謝、袁、蕭為大;東南則為「吳姓」,硃、張、顧、陸為大;山東則為「郡姓」,王、崔、盧、李、鄭為大;關中亦號「郡姓」,韋、裴、柳、薛、楊、杜首之;代北則為「虜姓」,元、長孫、宇文、于、陸、源、竇首之。「虜姓」者,魏孝文帝遷洛,有八氏十姓,三十六族九十二姓。八氏十姓,出於帝宗屬,或諸國從魏者;三十六族九十二姓,世為部落大人;並號河南洛陽人。「郡姓」者,以中國士人差第閥閱為之制,凡三世有三公者曰「膏粱」,有令、僕者曰「華腴」,尚書、領、護而上者為「甲姓」,九卿若方伯者為「乙姓」,散騎常侍、太中大夫者為「丙姓」,吏部正員郎為「丁姓」。凡得入者,謂之「四姓」。又詔代人諸胄,初無族姓,其穆、陸、奚、于,下吏部勿充猥官,得視「四姓」。北齊因仍,舉秀才、州主簿、郡功曹,非「四姓」不在選。故江左定氏族,凡郡上姓第一,則為右姓;太和以郡四姓為右姓;齊浮屠曇剛《類例》凡甲門為右姓;周建德氏族以四海通望為右姓;隋開皇氏族以上品、茂姓則為右姓;唐《貞觀氏族志》凡第一等則為右姓;路氏著《姓略》,以盛門為右姓;柳沖《姓族系錄》凡四海望族則為右姓。不通歷代之說,不可與言譜也。今流俗獨以崔、盧、李、鄭為四姓,加太原王氏號五姓,蓋不經也。



 夫文之弊,至於尚官;官之弊,至於尚姓;姓之弊,至於尚詐。隋承其弊,不知其所以弊,乃反古道,罷鄉舉,離地著,尊執事之吏。於是乎土無鄉里,里無衣冠,人無廉恥,士族亂而庶人僭矣。故善言譜者,系之地望而不惑,質之姓氏而無疑,綴之婚姻而有別。山東之人質,故尚婚婭,其信可與也;江左之人文,故尚人物,其智可與也;關中之人雄,故尚冠冕,其達可與也;代北之人武,故尚貴戚,其泰可與也。及其弊,則尚婚婭者先外族、後本宗,尚人物者進庶孽、退嫡長,尚冠冕者略伉儷、慕榮華,尚貴戚者徇勢利、亡禮教。四者俱弊,則失其所尚矣。



 人無所守,則士族削;士族削,則國從而衰。管仲曰:「為國之道,利出一孔者王,二孔者強,三孔者弱,四孔者亡。」故冠婚者,人道大倫。周、漢之官人,齊其政,一其門,使下知禁,此出一孔也,故王;魏、晉官人,尊中正,立九品,鄉有異政,家有競心,此出二孔也,故強;江左、代北諸姓,紛亂不一,其要無歸,此出三孔也,故弱;隋氏官人,以吏道治天下,人之行,不本鄉黨,政煩於上,人亂於下,此出四孔也,故亡。唐承隋亂,宜救之以忠,忠厚則鄉黨之行修;鄉黨之行修,則人物之道長;人物之道長,則冠冕之緒崇;冠冕之緒崇,則教化之風美;乃可與古參矣。



 晉太元中,散騎常侍河東賈弼撰《姓氏簿狀》,十八州百十六郡,合七百一十二篇,甄析士庶無所遺。宋王弘、劉湛好其書。弘每日對千客,可不犯一人諱。湛為選曹,撰《百家譜》以助銓序,文傷寡省,王儉又廣之,王僧孺演益為十八篇,東南諸族自為一篇,不入百家數。弼傳子匪之,匪之傳子希鏡,希鏡撰《姓氏要狀》十五篇,尤所諳究。希鏡傳子執,執更作《姓氏英賢》一百篇,又著《百家譜》,廣兩王所記。執傳其孫冠,冠撰《梁國親皇太子序親簿》四篇。王氏之學,本於賈氏。



 唐興,言譜者以路敬淳為宗,柳沖、韋述次之。李守素亦明姓氏,時謂「肉譜」者。後有李公淹、蕭穎士、殷寅、孔至,為世所稱。



 初,漢有鄧氏《官譜》,應劭有《氏族》一篇,王符《潛夫論》亦有《姓氏》一篇,宋何承天有《姓苑》二篇。譜學大抵具此。魏太和時,詔諸郡中正,各列本土姓族次第為舉選格,名曰「方司格」,人到於今稱之。



 馬懷素,字惟白,潤州丹徒人。客江都,師事李善,貧無資,晝樵,夜輒然以讀書,遂博通經史。擢進士第,又中文學優贍科,補郿尉。積勞,遷左臺監察御史。長安中,大夫魏元忠為張易之構謫嶺表,太僕崔貞心貞、東宮率獨孤禕之祖道,易之怒,使人上急變,告貞心貞等與元忠謀反。武后詔懷素按之,使者促迫,懷素執不從,曰:「貞心貞餞流人當得罪,以為謀反,則非。昔彭越以逆誅,欒布奏事尸下,漢不坐罪。今元忠罪非越比,不宜坐餞闊之人。且陛下操生殺柄,欲加之罪,自當處決聖心。既付臣按狀,惟知守陛下法爾。」後意解,貞心貞等乃免。宰相李迥秀藉易之勢,斂賕諉法,懷素劾罷之。轉禮部員外郎。以十道使黜陟江西,處決平恕。遷考功,核取實才,權貴謁請不能阿撓。擢中書舍人內供奉,為修文館直學士。



 開元初,為戶部侍郎,封常山縣公,進兼昭文館學士。篤學,手未嘗廢卷。謙恭慎畏,推為長者。玄宗詔與褚無量同為侍讀,更日番入。既叩閣,肩輿以進;或行在遠,聽乘馬。宮中每宴見,帝自送迎以師臣禮。有詔句校秘書。是時,文籍盈漫,皆炱朽蟫斷,簽啇紛舛。懷素建白:「願下紫微、黃門,召宿學巨儒就校繆缺。」又言:「自齊以前舊籍,王儉《七志》已詳。請採近書篇目及前志遺者,續儉《志》以藏秘府。」詔可。即拜懷素秘書監。乃詔國子博士尹知章、四門助教王直、直國子監趙玄默,陸渾丞吳綽、桑泉尉韋述、扶風丞馬利征、湖州司功參軍劉彥直、臨汝丞宋辭玉、恭陵丞陸紹伯、新鄭尉李子釗、杭州參軍殷踐猷、梓潼尉解崇質、四門直講余欽、進士王愜、劉仲丘、右威衛參軍侯行果、邢州司戶參軍袁暉、海州錄事參軍晁良、右率府胄曹參軍毋煚、滎陽主簿王灣、太常寺太祝鄭良金等分部撰次,踐猷從弟秘書丞承業、武陟尉徐楚璧是正文字。懷素奏秘書少監盧俌、崔沔為修圖書副使,秘書郎田可封、康子元為判官。然懷素不善著述,未能有所緒別。會卒,帝舉哀洛陽南城門,贈潤州刺史,謚曰文,給輿還鄉里,喪事官辦。



 懷素卒後,詔秘書官並號修書學士,草定四部,人人意自出,無所統一,逾年不成。有司疲於供擬,太僕卿王毛仲奏罷內料。又詔右常侍褚無量、大理卿元行沖考絀不應選者,無量等奏:「修撰有條,宜得大儒綜治。」詔委行沖。乃令煚、述、欽總緝部分,踐猷、愜治經,述、欽治史,煚、彥直治子,灣、仲丘治集。八年,《四錄》成,上之。學士無賞擢者。



 行沖知麗正院,又奏紹伯、利征、彥直、踐猷、行果、子釗、直、煚、述、灣、玄默、欽、良金與朝邑丞馮朝隱、冠氏尉權寅獻、秘書省校書郎孟曉、揚州兵曹參軍韓覃、王嗣琳,福昌令張悱、進士崔藏之入校麗正書。由是秘書省罷撰緝,而學士皆在麗正矣。



 愜、仲丘老病還鄉里。紹伯卒於官。直終岐王府記室參軍事。玄默集賢直學士。利征,出為山茌令,儒緩無治術,免官,終於家。子釗坐保任非人,終德州長史。欽至太學博士、集賢院學士。灣,洛陽尉。良金,右補闕、京兆府倉曹參軍事。寅獻,臨淮太守。曉,左補闕。覃,萊州別駕,坐誣告刺史,流遠方。藏之,膳部員外郎,明年,以將仕郎梁令瓚文學直書院,後以右率府兵曹參軍而罷,終恆王府司馬。秘書省校書郎源幼良代利征,後以協律郎罷。



 殷踐猷,字伯起,陳給事中不害五世從孫。博學,尤通氏族、歷數、醫方。與賀知章、陸象先、韋述最善,知章嘗號為「五總龜」,謂龜千年五聚,問無不知也。初為杭州參軍,舉文儒異等科,授秘書省學士,用曹州司法參軍,兼麗正殿學士。以叔父喪,哀慟歐血而卒,年四十八。



 少子寅,舉宏辭,為太子校書,出為永寧尉。吏侮謾甚,寅怒殺之,貶澄城丞。病且死,以母蕭老,不忍決。及斂,其子亮斷指剪發置棺中,自誓事祖母如寅在。其後侍蕭疾,不脫衣者數年,有白燕巢其楣。後終給事中、杭州刺史。



 踐猷弟季友,歷秘書郎,善畫。



 從父仲容,終冬官郎中,有重名。子承業,以謹樸稱,歷太子左諭德、右威衛將軍。



 族子成己,晉州長史。初,母顏叔父吏部郎中敬仲為酷吏所陷,率二妹割耳訴冤,敬仲得減死。及成己生,而左耳缺云。



 孔若思,越州山陰人,陳吏部尚書奐四世孫。祖紹安,與兄紹新蚤知名。陳亡,客居鄠,勵志於學。外兄虞世南曰:「本朝淪覆,吾分湮滅,有弟若此,知不亡矣。」紹安與孫萬壽皆以文辭稱,時謂「孫孔」。隋大業末,為監察御史。高祖討賊河東,紹安與夏侯端同監軍,禮遇尤密。帝受禪,端先歸,拜秘書監。已而紹安間道走長安,帝悅,擢內史舍人,賜宅一區、良馬二匹。



 若思早孤,其母躬訓教,長以博學聞。有遺以褚遂良書者,納一卷焉。其人曰:「是書貴千金,何取之廉?」答曰:「審爾,此為多矣。」更還其半。擢明經,歷庫部郎中,常曰:「仕宦至郎中足矣。」座右置止水一石,明自足意。



 中宗初,敬暉、桓彥範當國,以若思多識古今,凡大政事,必咨質後行。三遷禮部侍郎,出為衛州刺史。故事,以宗室為州別駕,見刺史,驁放不肯致恭。若思劾奏別駕李道欽,請訊狀。有詔別駕見刺史致恭,自若思始。以清白擢銀青光祿大夫,賜絹百匹,累封梁郡公。開元七年卒,謚曰惠。



 從父禎,第進士,歷監察御史,門無賓謁,時譏其介。高宗時,再遷絳州刺史,封武昌縣子,謚曰溫。



 子季詡,字季和。永昌初,擢制科,授秘書郎。陳子昂常稱其神清韻遠,可比衛玠。終左補闕。



 若思子至,字惟微。歷著作郎,明氏族學,與韋述、蕭穎士、柳沖齊名。撰《百家類例》,以張說等為近世新族,叕刂去之。說子垍方有寵,怒曰:「天下族姓,何豫若事,而妄紛紛邪?」垍弟素善至,以實告。初,書成,示韋述,述謂可傳。及聞垍語,懼,欲更增損,述曰:「止!丈夫奮筆成一家書,奈何因人動搖?有死不可改。」遂罷。時述及穎士、沖皆撰《類例》,而至書稱工。



\end{pinyinscope}