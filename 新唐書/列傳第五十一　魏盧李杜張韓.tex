\article{列傳第五十一 魏盧李杜張韓}

\begin{pinyinscope}

 魏知古,深州陸澤人。方直有雅才,擢進士第。以著作郎修國史,累遷衛尉少卿,檢校相王府司馬。神龍初物質的客觀存在,推翻了以往唯物主義原理。得出了「物質,為吏部侍郎,以母喪解。服除,為晉州刺史。睿宗立,以故屬拜黃門侍郎,兼修國史。



 會造金仙、玉真觀,雖盛夏,工程嚴促,知古諫曰:「臣聞『古之君人,必時視人之所勤,人勤於食則百事廢』。故曰『不作無益害有益』。又曰『罔咈百姓以從己之欲』。《禮》:『季夏之月,樹木方盛,無有斬伐,不可以興土功。』此皆興化立治、為政養人之本也。今為公主造觀,將以樹功祈福,而地皆百姓所宅,卒然迫逼,令其轉徙,扶老攜幼,剔椽發瓦,呼嗟道路。乖人事,違天時,起無用之作,崇不急之務,群心震搖,眾口藉藉。陛下為人父母,欲何以安之?且國有簡冊,君舉必記,言動之微,可不慎歟!願下明詔,順人欲,除功役,收之桑榆,其失不遠。」不納。復諫曰:「自陛下戡翦兇逆,保定大器,蒼生顒顒,以謂朝有新政。今風教頹替日益甚,府藏空屈,人力勞敝,營作無涯,吏員浸增,諸司試補、員外、檢校官已贏二千,太符之帛為殫,太倉之米不支。臣前請停金仙、玉真,訖亦未止。今前水後旱,五穀不立,繇茲向春,必甚饑饉,陛下欲何方以賑之?又突厥於中國為患自久,其人非可以禮義誠信約也。雖遣使請婚,恐豺狼之心,弱則順伏,強則驕逆,月滿騎肥,乘中國饑虛,講親際會,窺犯亭鄣,復何以防之?」帝嘉其直,以左散騎常侍同中書門下三品。玄宗在春宮,又兼左庶子。



 先天元年,為侍中。從獵渭川,獻詩以諷,手制褒答,並賜物五十段。明年,封梁國公。竇懷貞等詭謀亂國,知古密發其奸,懷貞誅,賜封二百戶,物五百段。玄宗恨前賞薄,手敕更加百戶,旌其著節。是冬,詔知東都吏部選事,以稱職聞,優詔賜衣一副。自是恩意尤渥,由黃門監改紫微令。與姚元崇不協,除工部尚書,罷政事。開元三年卒,年六十九。宋璟聞而嘆曰:「叔向古遺直,子產古遺愛,兼之者其魏公乎!」贈幽州都督,謚曰忠。



 所薦洹水令呂太一、蒲州司功參軍齊浣、右內率騎曹參軍柳澤、密尉宋遙、左補闕袁暉、右補闕封希顏、伊闕尉陳希烈,後皆有聞於時。



 文宗大和二年,求其曾孫處訥,授湘陽尉,與魏徵、裴冕後擢任之。



 盧懷慎,滑州人,蓋範陽著姓。祖悊,仕為靈昌令,遂為縣人。懷真在童卯已不凡,父友監察御史韓思彥嘆曰:「此兒器不可量!」及長,第進士,歷監察御史。神龍中,遷侍御史。中宗謁武後上陽宮,後詔帝十日一朝。懷慎諫曰:「昔漢高帝受命,五日一朝太公於櫟陽宮,以起布衣登皇極,子有天下,尊歸於父,故行此耳。今陛下守文繼統,何所取法?況應天去提象〓才二里所,騎不得成列,車不得方軌,於此屢出,愚人萬有一犯屬車之塵,雖罪之何及。臣愚謂宜遵內朝以奉溫清,無煩出入。」不省。



 遷右御史臺中丞。上疏陳時政曰:



 臣聞「善人為邦百年,可以勝殘去殺」。孔子稱:「茍用我者,期月而已,三年有成。」故《書》:「三載考績,三考黜陟幽明。」昔子產相鄭,更法令,布刑書,一年人怨,思殺之,三年人德而歌之。子產,賢者也,其為政尚累年而後成,況常材乎?比州牧、上佐、兩畿令或一二歲,或三五月即遷,曾不論以課最,使未遷者傾耳以聽,企踵以望,冒進亡廉,亦何暇為陛下宣風恤人哉?禮義不能興,戶口益以流,倉庫愈匱,百姓日敞,職為此耳。人知吏之不久,不率其教;吏知遷之不遙,不究其力。媮處爵位,以養資望,雖明主有勤勞天下之志,然僥幸路啟,上下相蒙,寧盡至公乎?此國病也。賈誼所謂蹠盩,乃小小者耳。此而不革,雖和、緩將不能為。漢宣帝綜核名寶,興治致化,黃霸良二千石也,加秩賜金,就旌其能,終不肯遷。故古之為吏,至長子孫。臣請都督、刺史、上佐、畿令任未四考,不得遷。若治有尤異,或加賜車裘祿秩,降使臨問,璽書慰勉,須公卿闕,則擢之以勵能者。其不職或貪暴,免歸田里,以明賞罰之信。



 昔唐、虞稽古,建官惟百。夏、商官倍,亦克用軿。此省官也。故曰「官不必備,惟其才」,「無曠庶官,天工人其代之。」此擇人也。今京諸司員外官數十倍,近古未有。謂不必備,則為有餘,求其代工,乃多不厘務,而奉稟之費,歲巨億萬,徙竭府藏,豈致治意哉」今民力敞極,河、渭廣漕,不給京師,公私耗損,邊隅未靜。儻炎成沴,租稅減入,疆場有警,賑救無年,何以濟之?「毋輕人事,惟艱;毋安闕位,惟危。」此慎微也。原員外之官,皆一時良幹,擢以才不申其用,尊以名不任其力,自昔用人,豈其然歟?臣請才堪牧宰上佐,並以遷授,使宣力四方,責以治狀。有老病若不任職者,一廢省之,使賢不肖確然殊貫,此切務也。



 夫冒於寵賂,侮於鰥寡,為政之蠹也。竊見內外官有賕餉狼藉,劓剝蒸人,雖坐流黜,俄而遷復,還為牧宰,任以江、淮、嶺、磧,粗示懲貶,內懷自棄,徇貨掊貲,訖無悛心。明主之於萬物,平分而無偏施,以罪吏牧遐方,是謂惠奸而遺遠。遠州陬邑,何負聖化,而獨受其惡政乎?邊徼之地,夷夏雜處,憑險擾而難安;官非其才,則黎庶流亡,起為盜賊。由此言之,不可用凡才,況猾吏乎?臣請以贓論廢者,削跡不數十年,不賜收齒。《書》曰「旌別淑慝」,即其誼也。



 疏奏,不報。



 遷黃門侍郎、漁陽縣伯。與魏知古分領東都選。開元元年,進同紫微黃門平章事。三年,改黃門監。薛王舅王仙童暴百姓,憲司按得其罪,業為申列,有詔紫微,黃門覆實。懷慎與姚崇執奏「仙童罪狀明甚,若御史可疑,則它人何可信?」由是獄決。懷慎自以才不及崇,故事皆推而不專,時譏為「伴食宰相」。又兼吏部尚書,以疾乞骸骨,許之。卒,贈荊州大都督,謚曰文成。遺言薦宋璟、李傑、李朝隱、盧從願,帝悼嘆之。



 懷慎清儉不營產,服器無金玉文綺之飾,雖貴而妻子猶寒饑,所得祿賜,於故人親戚無所計惜,隨散輒盡。赴東都掌選,奉身之具,止一布囊。既屬疾,宋璟、盧從願候之,見敞簀單藉,門不施箔。會風雨至,舉席自障。日晏設食,蒸豆兩器、菜數桮而已。臨別,執二人手曰:「上求治切,然享國久,稍倦於勤,將有憸人乘間而進矣。公第志之!」及治喪,家無留儲。帝時將幸東都,四門博士張晏上言:「懷慎忠清,以直道始終,不加優錫,無以勸善。」乃下制賜其家物百段,米粟二百斛。帝後還京,因校獵、杜間,望懷慎家,環堵庳陋,家人若有所營者,馳使問焉,還白懷慎大祥,帝即以縑帛賜之,為罷獵。經其墓,碑表未立,停蹕臨視,泫然流涕,詔官為立碑,令中書侍郎蘇頲為之文,帝自書。



 子奐、弈。



 奐,早修整,為吏有清白稱。歷御史中丞,出為陜州刺史。開元二十四年,帝西還,次陜,嘉其美政,題贊於聽事曰:「專城之重,分陜之雄,亦既利物,內存匪躬,斯為國寶,不墜家風。」尋召為兵部侍郎。天寶初,為南海太守。南海兼水陸都會,物產瑰怪,前守劉巨鱗、彭杲皆以贓敗,故以奐代之。污吏斂手,中人之市舶者亦不敢干其法,遠俗為安。時謂自開元後四十年,治廣有清節者,宋璟、李朝隱、奐三人而已。終尚書右丞。弈見《忠義傳》。



 李元紘,字大綱,其先滑州人,後世占京兆萬年,本姓丙氏。曾祖粲,仕隋為屯衛大將軍,煬帝使督京師之西二十四郡盜賊,善撫循,能得士心。高祖與之厚,及兵入關,以眾歸,授宗正卿、應國公,賜姓李。後為左監門大將軍,以其老,聽乘馬按視宮禁。年八十餘卒,謚曰明。祖寬,高宗時為太常卿、隴西公。父道廣,武后時為汴州刺史,有善政。突厥、契丹寇河北,議發河南兵擊之,百姓震擾,道廣悉心撫定,人無離散。遷殿中監、同鳳閣鸞臺平章事,封金城侯。卒,贈秦州都督,謚曰成。



 元紘,早修謹,仕為雍州司戶參軍。時太平公主勢震天下,百司順望風指,嘗與民競碾磑,元紘還之民。長史竇懷貞大驚,趣改之,元紘大署判後曰:「南山可移,判不可搖也。」改好畤令,遷潤州司馬,以辦治得名。開元初,為萬年令,賦役稱平,擢京兆少尹。詔決三輔渠,時王、主、權家皆旁渠立磑,瀦堨爭利,元紘敕吏盡毀之,分溉渠下田,民賴其恩。三遷吏部侍郎。會戶部楊瑒、白知慎坐支調失宜,貶刺史,帝求可代者,公卿多薦元紘。帝欲擢為尚書,宰相以資薄,乃為戶部侍郎。條陳利害及政得失,帝才之,謂可丞輔,賜衣一稱、絹二百匹。明年,遂拜中書侍郎、同中書門下平章事,封清水縣男。



 元紘當國,務峻涯檢,抑奔競,誇進者憚之。五月五日,宴武成殿,賜群臣襲衣,特以紫服、金魚錫元紘及蕭嵩,群臣無與比。是時,廢京司職田,議者欲置屯田。元紘曰:「軍國不同,中外異制,若人閑無役,地棄不墾,以閑手耕棄地,省饋運,實軍糧,於是有屯田,其為益尚矣。今百官所廢職田不一縣,弗可聚也;百姓私田皆力自耕,不可取也。若置屯,即當公私相易,調發丁夫。調役則業廢於家,免庸則賦闕於國,內地為屯,古未有也。恐得不補失,徒為煩費。」遂止。初,左庶子吳兢為史官,譔《唐書》及《春秋》,未成,以喪解,後上書請畢其功,詔許就集賢院成書;張說致仕,詔在家修史。元紘因言:「國史記人君善惡、王政損益,褒貶所系,前聖尤重。今國大典,分散不一,且太宗別置史館禁中,所以秘嚴之也。請勒說以書就館,參會譔錄。」詔可。



 後與杜暹不協,數辨爭帝前,帝不懌,皆罷之,以元紘為曹州刺史,徙蒲州,引疾去。後以戶部尚書致仕,復起為太子詹事。卒,贈太子少傅,謚曰文忠。



 元紘再世宰相,有清節,其當國累年,未嘗改治第宅,僮馬敝弱,得封物賙給親族。宋璟嘗嘆曰:「李公引宋遙之美,黜劉晃之貪,為國相,家無留儲,雖季文子之德,何以加之!」



 杜暹,濮州濮陽人。父承志,武后時為監察御史。懷州刺史李文暕為人所告,詔承志推驗,無實。文暕,宗室近屬也,卒得罪,承志貶為方義令,遷天官員外郎。見羅織獄興,移疾去,卒於家。



 自高祖至暹,五世同居。暹尤恭謹,事繼母孝。擢明經第,補婺州參軍,秩滿歸,吏以紙萬番贐之,暹為受百番,眾嘆曰:「昔清吏受一大錢,何異哉?」為鄭尉,復以清節顯。華州司馬楊孚,公挺人也,每咨重暹。會孚遷大理正,暹適以累當坐,孚曰:「使若人得罪,眾安勸乎?」以狀言執政,繇是擢為大理評事。



 開元四年,以監察御史覆屯磧西。會安西副都護郭虔瓘與西突厥可汗阿史那獻、鎮守使劉遐慶更相訟,詔暹即按。入突騎施帳,究索左驗。虜以金遺暹,暹固辭,左右曰:「公使絕域,不可失戎心。」乃受焉,陰埋幕下。已出境,乃移文畀取之。突厥大驚,度磧追,不及,去。遷給事中,以母喪解。會安西都護張孝嵩遷太原尹,或言暹往使安西,虜伏其清,今猶慕思,乃奪服拜黃門侍郎兼安西副大都護。明年,於闐王尉遲朒約突厥諸國叛,暹覺其謀,發兵討斬之,支黨悉誅,更立君長,於闐遂安。以功加光祿大夫。守邊四年,撫戎練士,能自勤勵,為夷夏所樂。



 十四年,召同中書門下平章事,遣中使往迎。謁見,賜絹二百、馬一匹、第一區。與李元紘輕重不得,罷為荊州都督長史,歷魏州刺史、太原尹。帝幸北都,進戶部尚書,許扈蹕。還,復東幸,以暹為京留守。暹率當番衛士繕三宮城,浚池,督役不少懈。帝聞嘉之,數賜書褒勞,進禮部尚書,封魏縣侯。



 二十八年卒,贈尚書右丞相,遣使護喪,禁中出絹三百匹賜之,太常謚曰貞肅。右司員外郎劉同升等以暹行忠孝,謚有未盡,博士裴總謂暹往以墨衰受命安西,雖勤勞於國,不得盡孝。其子列訴,帝更敕有司考定,卒謚貞孝。



 暹友愛,撫異母弟昱甚厚。其為人少學術,故當朝議論,時時失淺薄。然能以公清勤約自將,亹癖為之,自弱冠誓不通親友獻遺,以終身。既卒,尚書省及故吏致賻,其子孝友一不受,以行暹素志云。



 暹族子鴻漸。鴻漸字之巽。父鵬舉,與盧藏用隱白鹿山,以母疾,與崔沔同授醫蘭陵蕭亮,遂窮其術。歷右拾遺。玄宗東行河,因游畋,上賦以風。終安州刺史。



 鴻漸第進士,解褐延王府參軍,安思順表為朔方判官。祿山亂,皇太子按軍平涼,未知所適,議出蕭關趣豐安。鴻漸與六城水運使魏少游、節度判官崔漪、支度判官盧簡金、關內鹽池判官李涵謀曰:「胡羯亂常,二京覆沒,太子治兵平涼,然散地難恃也。今朔方制勝之會,若奉迎太子,西詔河、隴,北結回紇,回紇固與國,收其勁騎,與大兵合,鼓而南,雪社稷之恥,不亦易乎!」即具上兵馬招輯之勢,且錄軍資、器械、儲廥凡最,使涵詣平涼見太子,太子大悅。會裴冕至自河西,亦勸之朔方。而鴻漸與漪至白草頓迎謁,說曰:「朔方天下勁兵,靈州用武地。今回紇請和,吐蕃結附,天下列城堅守,以待王命。縱為賊據,日夜望官軍,以圖收復。殿下治兵長驅,逆胡不足滅也。」太子喜曰:「靈武我之關中,卿乃吾蕭何也。」



 既至靈武,鴻漸即與冕等勸即皇帝位,以系中外望。六請,見聽。鴻漸明習朝章,採舊儀,設壇遺城南,先一日草其儀上之。太子曰:「聖皇在遠,寇逆方結,宜罷壇場,它如奏。」太子即位,是為肅宗,授鴻漸兵部郎中,知中書舍人事。俄為武部侍郎,遷河西節度使。兩京平,又節度荊南。乾元二年,襄州大將康楚元等反,刺史王政脫身走,楚元偽稱南楚霸王,因襲荊州。鴻漸棄城遁,人皆南奔,爭舟溺死者甚眾。澧、朗、復、郢等州聞鴻漸出,皆竄伏山谷。俄而商州刺史韋倫平其亂。



 久之,乃召鴻漸為尚書右丞、太常卿,充禮儀使。泰、建二陵制度皆鴻漸綜正,以優,封衛國公。又建言:「《周官》:『兇荒殺禮。』今承大亂,民人夷殘,其婚葬鹵簿,非於國有大功及二等以上親皆不許給。」詔可。



 代宗廣德二年,以兵部侍郎同中書門下平章事。尋進中書侍郎。崔旰殺郭英軿據成都,邛州牙將柏貞節、滬州牙將楊子琳、劍州牙將李昌膋以兵討旰,蜀、劍大亂。命鴻漸以宰相兼成都尹、山南西道劍南東川副元帥、劍南西川節度副大使往鎮撫之。鴻漸性畏怯,無它遠略,而晚節溺浮圖道,畏殺戮。及逾劍門,懲艾張獻誠敗,憚旰雄武,先許以不死。既見,禮遇之,不敢加譙責,反委以政,日與從事杜亞、楊炎縱酒高會,因薦旰為成都尹,而授貞節邛州刺史,子琳滬州刺史,各罷兵。乃請入朝,許之。及見帝,盛言旰威略可任,宜為留後。獻寶器五床、羅錦十五床,麝臍五石。復輔政。議者疾其長亂。進門下侍郎。大歷三年,兼東都留守、河南淮西山南東道副元帥,辭疾不行。又讓山南、劍南副元帥,聽之。四年,疾甚,辭宰相,罷三日卒,年六十一,贈太尉,謚曰文憲。



 鴻漸自蜀還,食千僧,以為有報,搢紳效之。病甚,令僧剔項發,遺命依浮圖葬,不為封樹。



 張九齡,字子壽,韶州曲江人。七歲知屬文,十三以書乾廣州刺史王方慶,方慶嘆曰:「是必致遠。」會張說謫嶺南,一見厚遇之。居父喪,哀毀,庭中木連理。擢進士,始調校書郎,以道侔伊呂科策高第,為左拾遺。時玄宗即位,未郊見,九齡建言:



 天,百神之君,王者所由受命也。自古繼統之主,必有郊配,盡敬天命,報所受也。不以德澤未洽,年穀未登,而闕其禮。昔者周公郊祀后稷以配天,謂成王幼沖,周公居攝,猶用其禮,明不可廢也。漢丞相匡衡曰:「帝王之事,莫重乎郊祀。」董仲舒亦言:「不郊而祭山川,失祭之序,逆於禮,故《春秋》非之。」臣謂衡、仲舒古之知禮。皆以郊之祭所宜先也。陛下紹休聖緒,於今五載,而未行大報,考之於經,義或未通。今百穀嘉生,鳥獸咸若,夷狄內附,兵革用弭,乃怠於事天,恐不可以訓。願以迎日之至,升紫壇,陳採席,定天位,則聖典無遺矣。



 又言:



 乖政之氣,發為水旱。天道雖遠,其應甚邇。昔東海枉殺孝婦,天旱久之。一吏不明,匹婦非命,則天昭其冤。況六合元元之眾,縣命於縣令,宅生於刺史,陛下所與共治,尤親於人者乎!若非其任,水旱之繇,豈唯一婦而已。今刺史,京輔雄望之郡,猶少擇之,江、淮、隴、蜀、三河大府之外,稍非其人。繇京官出者,或身有累,或政無狀,用牧守之任。為斥逐之地。或因附會以忝高位,及勢衰,謂之不稱京職,出以為州。武夫、流外,積資而得,不計於才。刺史乃爾,縣令尚可言哉?氓庶,國家之本,務本之職,乃為好進者所輕,承弊之民,遭不肖所擾,聖化從此銷鬱,繇不選親人以成其敝也。古者刺史入為三公,郎官出宰百里。今朝廷士入而不出,其於計私,甚自得也。京師衣冠所聚,身名所出,從容附會,不勤而成,是大利在於內,而不在於外也。智能之士,欲利之心,安肯復出為刺史、縣令哉?國家賴智能以治,而常無親人者,陛下不革以法故也。臣愚謂欲治之本,莫若重守令,守令既重,則能者可行。宜遂科定其資:凡不歷都督、刺史,雖有高第,不得任侍郎、列卿;不歷縣令,雖有善政,不得任臺郎、給、舍;都督、守、令雖遠者,使無十年任外。如不為此而救其失,恐天下猶未治也。



 又古之選士,惟取稱職,是以士修素行,而不為徼幸,奸偽自止,流品不雜。今天下不必治於上古,而事務日倍於前,誠以不正其本而設巧於末也。所謂末者,吏部條章,舉贏千百。刀筆之人,溺於文墨;巧史猾徒,緣奸而奮。臣以謂始造簿書,備遺忘耳,今反求精於案牘,而忽於人才,是所謂遺劍中流,契丹以記者也。凡稱吏部能者,則曰自尉與主簿,繇主簿與丞,此執文而知官次者也,乃不論其賢不肖,豈不謬哉!夫吏部尚書、侍郎,以賢而授者也,豈不能知人?如知之難,拔十得五,斯可矣。今膠以格條,據資配職,為官擇人,初無此意,故時人有平配之誚,官曹無得賢之實。



 臣謂選部之法,敝於不變。今若刺史、縣令精核其人,則管內歲當選者,使考才行,可入流品,然後送臺,又加擇焉,以所用眾寡為州縣殿最,則州縣慎所舉,可官之才多,吏部因其成,無庸人之繁矣。今歲選乃萬計,京師米物為耗,豈多士哉?盡冒濫抵此爾。方以一詩一判,定其是非,適使賢人遺逸,此明代之闕政也。天下雖廣,朝廷雖眾,必使毀譽相亂,聽受不明,事則已矣。如知其賢能,各有品第,每一官缺,不以次用之,豈不可乎?如諸司要官,以下等叨進,是議無高卑,唯得與不爾。故清議不立,而名節不修,善士守志而後時,中人進求而易操也。朝廷能以令名進人,士亦以修名獲利,利之出,眾之趨也。不如此,則小者得於茍求,一變而至阿私;大者許以分義,再變而成朋黨矣。故於用人不可不第其高下,高下有次,則不可以妄干,天下之士必刻意修飾,而刑政自清,此興衰之大端也。



 俄遷左補闕。九齡自才鑒,吏部試拔萃與舉者,常與右拾遺趙冬曦考次,號稱詳平。改司勛員外郎。時張說為宰相,親重之,與通譜系,常曰:「後出詞人之冠也。」遷中書舍人內供奉,封曲江男,進中書舍人。會帝封泰山,說多引兩省錄事主書及所親攝官升山,超階至五品。九齡當草詔,謂說曰:「官爵者,天下公器,先德望,後勞舊。今登封告成,千載之絕典,而清流隔於殊恩,胥史乃濫章韍,恐制出,四方失望。方進草,尚可以改,公宜審計。」說曰:「事已決矣,悠悠之言不足慮。」既而果得謗。御史中丞宇文融方事田法,有所關奏,說輒建議違之。融積不平,九齡為言,說不聽。俄為融等痛詆,幾不免,九齡亦改太常少卿,出為冀州刺史。以母不肯去鄉里,故表換洪州都督。徙桂州,兼嶺南按察選補使。



 始說知集賢院,嘗薦九齡可備顧問。說卒,天子思其言,召為秘書少監、集賢院學士,知院事。會賜渤海詔,而書命無足為者,乃召九齡為之,被詔輒成。遷工部侍郎,知制誥。數乞歸養,詔不許,以其弟九皋、九章為嶺南刺史,歲時聽給驛省家。遷中書侍郎,以母喪解,毀不勝哀,有紫芝產坐側,白鳩、白雀巢家樹。是歲,奪哀拜中書侍郎、同中書門下平章事。固辭,不許。明年,遷中書令。始議河南開水屯,兼河南稻田使。上言廢循資格,復置十道採訪使。



 李林甫無學術,見九齡文雅,為帝知,內忌之。會範陽節度使張守珪以斬可突幹功,帝欲以為侍中。九齡曰:「宰相代天治物,有其人然後授,不可以賞功。國家之敗,由官邪也。」帝曰:「假其名若何?」對曰:「名器不可假也。有如平東北二虜,陛下何以加之?」遂止。又將以涼州都督牛仙客為尚書,九齡執曰:「不可。尚書,古納言,唐家多用舊相,不然,歷內外貴任,妙有德望者為之。仙客,河、湟一使典耳,使班常伯,天下其謂何?」又欲賜實封,九齡曰:「漢法非有功不封,唐遵漢法,太宗之制也。邊將積穀帛,繕器械,適所職耳。陛下必賞之,金帛可也,獨不宜裂地以封。」帝怒曰:「豈以仙客寒士嫌之邪?卿固素有門閱哉?」九齡頓首曰:「臣荒陬孤生,陛下過聽,以文學用臣。仙客擢胥史,目不知書。韓信,淮陰一壯夫,羞絳、灌等列。陛下必用仙客,臣實恥之。」帝不悅。翌日,林甫進曰:「仙客,宰相材也,乃不堪尚書邪?九齡文吏,拘古義,失大體。」帝由是決用仙客不疑。九齡既戾帝旨,固內懼,恐遂為林甫所危,因帝賜白羽扇,乃獻賦自況,其末曰:「茍效用之得所,雖殺身而何忌?」又曰:「縱秋氣之移奪,終感恩於篋中。」帝雖優答,然卒以尚書右丞相罷政事,而用仙客。自是朝廷士大夫持祿養恩矣。嘗薦長安尉周子諒為監察御史,子諒劾奏仙客,其語援讖書。帝怒,杖子諒於朝堂,流瀼州,死於道。九齡坐舉非其人,貶荊州長史。雖以直道黜,不戚戚嬰望,惟文史自娛,朝廷許其勝流。久之,封始興縣伯,請還展墓,病卒,年六十八,贈荊州大都督,謚曰文獻。



 九齡體弱,有愬藉。故事,公卿皆搢笏於帶,而後乘馬。九齡獨常使人持之,因設笏囊,自九齡始。後帝每用人,必曰:「風度能若九齡乎?」初,千秋節,公、王並獻寶監,九齡上「事鑒」十章,號《千秋金鑒錄》,以伸諷諭。與嚴挺之、袁仁敬、梁升卿、盧怡善,世稱其交能終始者。及為相,諤諤有大臣節。當是時,帝在位久,稍怠於政,故九齡議論必極言得失,所推引皆正人。武惠妃謀陷太子瑛,九齡執不可。妃密遣宦奴牛貴兒告之曰:「廢必有興,公為援,宰相可長處。」九齡叱曰:「房幄安有外言哉!」遽奏之,帝為動色,故卒九齡相而太子無患。安祿山初以範陽偏校入奏,氣驕蹇,九齡謂裴光庭曰:「亂幽州者,此胡雛也。」及討奚、契丹敗,張守珪執如京師,九齡署其狀曰:「穰苴出師而誅莊賈,孫武習戰猶戮宮嬪,守珪法行於軍,祿山不容免死。」帝不許,赦之。九齡曰:「祿山狼子野心,有逆相,宜即事誅之,以絕後患。」帝曰:「卿無以王衍知石勒而害忠良。」卒不用。帝後在蜀,思其忠,為泣下,且遣使祭於韶州,厚幣恤其家。開元後,天下稱曰曲江公而不名云。建中元年,德宗賢其風烈,復贈司徒。



 子拯,居父喪,有節行,後為伊闕令。會祿山盜河、洛,陷焉。而終不受偽官。賊平,擢太子贊善大夫。



 九齡弟九皋,亦有名,終嶺南節度使。其曾孫仲方。



 仲方,生歧秀,父友高郢見,異之,曰:「是兒必為國器,使吾得位,將振起之。」貞元中,擢進士、宏辭,為集賢校理,以母喪免。會郢拜御史大夫,表為御史。進累倉部員外郎。



 會呂溫等以劾奏宰相李吉甫不實,坐斥去,仲方以溫黨,補金州刺史。宦人奪民田,仲方三疏申理,卒與民直。入為度支郎中。吉甫卒,太常謚恭懿,博士尉遲汾清謚敬憲,仲方挾前怨未已,因上議曰:「古之謚,考大節,略細行,善善惡惡,一言而足。按吉甫雖多才多藝,而側媚取容,疊致臺袞,寡信易謀,事無成功。且兵兇器,不可從我始,至以伐罪,則邀必成功。今內有賊輔臣之盜,外有懷毒蠆之臣,師徒暴野,農不得在畝,婦不得在桑,耗賦殫畜,尸殭血流,胔骼成嶽,毒痡之痛,訴天無辜,階禍之發,實始吉甫。」又言:「吉甫平易柔寬,名不配行。請俟蔡平,然後議之。」憲宗方用兵,疾其言醜訐,貶為遂州司馬。稍進河南少尹、鄭州刺史。



 敬宗立,李程輔政,引為諫議大夫。帝時詔王播造競渡舟三十艘,度用半歲運費。仲方見延英,論諍堅苦,帝為減三之二。又詔幸華清宮,仲方曰:「萬乘之行,必具葆衛,易則失威重。」不從,猶見慰勞。鄠令崔發以辱黃門系獄,逢赦不見宥。仲方曰:「恩被天下,流昆蟲,而不行御前乎?」發繇是不死。大和初,出為福建觀察使。召還,進至左散騎常侍。李德裕秉政,以太子賓客分司東都。德裕罷,復拜常侍。



 李訓之變,大臣或誅或系。翌日,群臣謁宣政,牙闔不啟。群臣錯立朝堂,無史卒贊候,久乃半扉啟,使者傳召仲方曰:「有詔,可京兆尹。」然後門闢,喚仗。於時族夷將相,顱足旁午,仲方皆密使識其尸。俄許收葬,故胔骸不相亂。已而禁軍橫,多撓政,仲方勢笮,不能有所繩劾。宰相鄭覃更以薛元賞代之,出為華州刺史。召入,授秘書監。人頗言覃助德裕,擯仲方不用。覃乃擬丞、郎以聞。文宗曰:「侍郎,朝廷華選。彼牧守無狀,不可得。」但封曲江縣伯。卒,七十二,贈禮部尚書,謚曰成。仲方確正有風節,既駁吉甫謚,世不直其言,卒不至顯。既歿,人多傷之。



 始,高祖仕隋時,太宗方幼而病,為刻玉像於熒陽佛祠以祈年,久而刓晦,仲方在鄭,敕吏治護,鏤石以聞,傳於時。



 韓休,京兆長安人。父大智,洛州司功參軍,其兄大敏,仕武後為鳳閣舍人。梁州都督李行褒為部人告變,詔大敏鞫治。或曰:「行褒諸李近屬,後意欲去之,無列其冤,恐累公。」大敏曰:「豈顧身枉人以死乎?」至則驗出之。後怒,遣御史覆按,卒殺行褒,而大敏賜死於家。



 休工文辭,舉賢良。玄宗在東宮,令條對國政,與校書郎趙冬曦並中乙科,擢左補闕,判主爵員外郎。進至禮部侍郎,知制誥。出為虢州刺史。虢於東、西京為近州,乘輿所至,常稅廄芻,休請均賦它郡。中書令張說曰:「免虢而與它州,此守臣為私惠耳。」休復執論,吏白恐忤宰相意,休曰:「刺史幸知民之敝而不救,豈為政哉?雖得罪,所甘心焉。」訖如休請。以母喪解,服除,為工部侍郎,知制誥。遷尚書右丞。侍中裴光庭卒,帝敕蕭嵩舉所以代者,嵩稱休志行,遂拜黃門侍郎、同中書門下平章事。



 休直方不務進趨,既為相,天下翕然宜之。萬年尉李美玉有罪,帝將放嶺南。休曰:「尉小官,犯非大惡。今朝廷有大奸,請得先治。金吾大將軍程伯獻恃恩而貪,室宅輿馬僣法度,臣請先伯獻,后美玉。」帝不許,休固爭曰:「罪細且不容,巨猾乃置不問,陛下不出伯獻,臣不敢奉詔。」帝不能奪。大率堅正類此。初,嵩以休柔易,故薦之。休臨事或折正嵩,嵩不能平。宋璟聞之曰:「不意休能爾,仁者之勇也。」嵩寬博多可,休峭鯁,時政所得失,言之未嘗不盡。帝嘗獵苑中,或大張樂,稍過差,必視左右曰:「韓休知否?」已而疏輒至。嘗引鑒,默不樂。左右曰:「自韓休入朝,陛下無一日歡,何自戚戚,不逐去之?」帝曰:「吾雖瘠,天下肥矣。且蕭嵩每啟事,必順旨,我退而思天下,不安寢。韓休敷陳治道,多訐直,我退而思天下,寢必安。吾用休,社稷計耳。」後以工部尚書罷。遷太子少師,封宜陽縣子。卒,年六十八,贈揚州大都督,謚曰文忠。寶應元年,贈太子太師。



 子浩、洽、洪、汯、滉、渾、洄,皆有學尚。



 浩,萬年主簿,坐籍王鉷家貲有隱入,為尹鮮於仲通所劾,流循州。洪為司庫員外郎,與汯皆以累貶。洪後為華州長史。渾,大理司直。安祿山盜京師,皆陷賊,賊逼以官,浩與洪、汯、滉、渾出奔,將走行在,浩、洪、渾及洪四子復為賊禽殺之。洪善與人交,有節義,藉甚於時,見者為流涕。肅宗以大臣子能死難,詔贈浩吏部郎中,洪太常卿,渾太常少卿。汯上元中終諫議大夫。洽,終殿中侍御史。



 滉,字太沖,以廕補左威衛騎曹參軍。至德初,避地山南,採訪使李承昭表為通川郡長史,改彭王府諮議參軍。初,汯知制誥,當草王璵詔,無借言,銜之。及當國,滉兄弟皆斥冗官。璵罷,乃擢殿中侍御史,三遷吏部員外郎。性強直,明吏事,蒞南曹五年,簿最詳致。再遷給事中,知兵部選。時盜殺富平令韋當,賊隸北軍,魚朝恩私其兇,奏原死,滉執處,卒伏辜。遷右丞。知吏部選,以戶部侍郎判度支。



 自至德軍興,所在賦稅無藝,帑司給輸乾隱。滉檢制吏下及四方輸將,犯者痛根以法。會歲數稔,兵革少息,故儲積穀帛稍豐實。然覆治案牘,深文鉤剝,人亦咨怨。大歷十二年秋,大雨害稼什八,京兆尹黎幹言狀,滉恐有所蠲貸,固表不實。代宗命御史行視,實損田三萬餘頃。始,渭南令劉藻附滉,言部田無害,御史趙計按驗如藻言,帝又遣御史硃敖覆實,害田三千頃。帝怒曰:「縣令,所以養民,而田損不問,豈恤隱意邪?」貶南浦員外尉,計亦斥為豐州司戶員外參軍。方是時,潦敗河中鹽池,滉奏池產瑞鹽。帝疑,遣諫議大夫蔣鎮廉狀,鎮畏滉,還乃賀帝,且請置祠,詔號寶應靈慶池。



 德宗立,惡滉掊刻,徙太常卿。議者不厭,乃出為晉州刺史。未幾,遷浙江東、西觀察使,尋檢校禮部尚書為鎮海軍節度使。綏輯百姓,均租、調,不逾年,境內稱治。帝在奉天,淮、汴震騷,滉訓士卒,分兵戌河南。既狩梁州,又獻縑十萬匹,請以鎮兵三萬助討賊,有詔嘉勞,進檢校尚書右僕射,封南陽郡公。李希烈陷汴州,滉遣裨將王棲耀、李長榮、柏良器以勁卒萬人進計,次睢陽,而賊已攻寧陵,棲耀等破走之,漕路無梗,完靖東南,滉功多。



 時里胥有罪,輒殺無貸,人怪之。滉曰:「袁晁本一鞭背史,禽賊有負,聚其類以反,此輩皆鄉縣豪黠,不如殺之,用年少者,惜身保家不為惡。」又以賊非牛酒不嘯結,乃禁屠牛,以絕其謀。婺州屬縣有犯令者,誅及鄰伍,坐死數十百人。又遣官分察境內,罪涉疑似必誅,一判輒數十人,下皆愁怖。



 聞京都未平,乃閉關梁,禁牛馬出境,築石頭五城,自京口至玉山。毀上元道、佛祠四十區,修捴壁,起建業、抵京峴,樓雉相望。以為朝廷有永嘉南走事,置館第數十於石頭城,穿井皆百尺。命偏將丘涔督役,日數千人,涔虐用其眾,朝令夕辦,先世丘壟皆發夷。造樓艦三千柁,以舟師由海門大閱,至申浦乃還。追李長榮等歸,以親吏盧復為宣州刺史,增營壘,教習長兵,毀鐘鑄軍器。陳少游在揚州,以甲士三千臨江大閱;滉亦總兵臨金山,與少游會,以金繒相餉酬。然滉握強兵,遷延不赴難,而調發糧帛以濟朝廷者繦屬,當時實賴之。李晟方屯渭北,滉運米饋之,船置十弩以相警捍,賊不能剽。始,漕船臨江,滉顧僚吏曰:「天子蒙塵,臣下之恥也。」乃自舉一囊,將佐爭負之。



 貞元元年,加檢校左僕射、同中書門下平章事、江淮轉運使,封鄭國公。以繕治石頭城,人頗言有窺望意,雖帝亦惑之。會李泌間關辯數,帝意乃解。二年,更封晉。是歲入朝。滉既宿齒先達,頗簡倨,接新進用事,不能滿其意,眾怨之。獻羨錢五百餘萬緡,詔加度支諸道轉運、鹽鐵等使。



 右丞元琇判度支也,以關輔旱,請運江南租米西給京師。帝委滉專督之,而琇畏其剛愎難共事,請自江至揚子,滉主之;揚子以北,自主之。滉由是銜琇。會琇以京師錢重貨輕,發江東鹽監院錢四十萬緡入關。滉紿奏「運錢至京師,率費萬致千,不可從。」帝責謂琇,琇曰:「千錢其重與斗米均,費三百可致。」帝以諭滉,滉執不可。至是,誣劾琇饋米與淄青李納、河中李懷光。帝怒,不復究驗,貶琇雷州司戶參軍。左丞董晉白宰相劉滋、齊映曰:「昨關輔用兵,方蝗旱,琇不增一賦,而軍興皆濟,可謂勞臣。今被謫無名,刑濫人懼,假令權臣逞志,公胡不請三司鞫之?」滋、映不能用。給事中袁高抗疏申執,滉指為黨與,寢不報。



 劉玄佐不朝,帝密詔滉諷之。及過汴,玄佐素憚滉,修屬吏禮。滉辭不敢當,因結為兄弟,入拜其母,置酒設女樂。酒行,滉曰:「宜早見天子,不可使夫人白首與新婦子孫填宮掖也。」玄佐泣悟。滉以錢二十萬緡為玄佐辦裝,又以綾二十萬犒軍。玄佐入朝,滉薦可任邊事。時兩河罷兵,滉上言:「吐籓盜河、湟久,近歲浸弱,而西近大食,北捍回鶻,東抗南詔,分軍外戰,兵在河、隴者不過五六萬,若朝廷命將,以十萬眾城涼、鄯、洮、渭,各置兵二萬為守御,臣請以本道財賦饋軍,給三年費,然後營田積粟,且耕且戰,則河、隴之地可翹足而復。」帝善其言,因訪玄佐,玄佐請行。會滉病甚,張延賞奏減州縣冗官,收祿俸,募戰士西討。玄佐慮延賞靳削資儲,辭犬戎未釁,不可輕進,因稱疾。帝遣中人勞問,臥受命。延賞知不可用,乃止。滉尋卒,年六十五,贈太傅,謚曰忠肅。



 滉雖宰相子,性節儉,衣裘茵衽,十年一易。甚暑不執扇,居處陋薄,取庇風雨。門當列戟,以父時第門不忍壞,乃不請。堂先無挾廡,弟洄稍增補之,滉見即徹去,曰:「先君容焉,吾等奉之,常恐失墜。若摧圮,繕之則已,安敢改作以傷儉德?」居重位,清潔疾惡,不為家人資產。自始仕至將相,乘五馬,無不終櫪下。好鼓琴,書得張旭筆法,畫與宗人幹相埒。嘗自言:「不能定筆,不可論書畫。」以非急務,故自晦,不傳於人。善治《易》、《春秋》,著《通例》及《天文事序議》各一篇。初判度支,李晟以裨將白軍事,滉待之加禮,使其子拜之,厚遺器幣鞍馬。後晟終立大功。滉幼時已有美名,所與游皆天下豪俊。晚節益苛慘,故論者疑其飾情希進,既得志,則強肆,蓋自其性云。子群、皋。



 群終國子司業。皋字仲聞,資質重厚,有大臣器。由雲陽尉策賢良方正異等,拜右拾遺。累遷考功員外郎。父喪,德宗遣使吊問,俾論譔滉行事,號泣承命,立草數千言以進,帝嘉之。服除,宰相擬考功郎中,帝為加知制誥。遷中書舍人、御史中丞、兵部侍郎,號稱職。俄拜京兆尹。奏署鄭鋒為倉曹參軍。鋒苛斂吏,乃說皋悉索府中雜錢,折糴粟麥三十萬石獻於帝,皋悅之,奏為興平令。貞元十四年,大旱,民請蠲租賦,皋府帑已空,內憂恐,奏不敢實。會中人出入,百姓遮道訴之,事聞,貶撫州員外司馬。未幾,改杭州刺史,入拜尚書右丞。王叔文用事,皋嫉之,謂人曰:「吾不能事新貴。」從弟曄以告叔文,叔文怒,出為鄂岳蘄〗沔觀察使。叔文敗,即拜節度,徙鎮海,入為戶部尚書,歷東都留守、忠武軍節度使。大抵以簡儉治,所至有績。召拜吏部尚書,兼太子少傅。莊憲太后崩,充大明宮留守。穆宗以舊傅恩,加檢校尚書右僕射,俄為真。又進左僕射。長慶四年,復為東都留守,卒於道,年七十九,贈太子太保,謚曰貞。



 皋貌類父,既孤,不復視鑒。生知音律,常曰:「長年後不願聽樂,以門內事多逆知之。」聞鼓琴,至《止息》,嘆曰:「美哉!嵇康之為是曲,其當晉、魏之際乎。其音主商,商為秋,秋者天將搖落肅殺,其歲之晏乎。晉乘金運,商又金聲,此所以知魏方季而晉將代也。緩其商糸玄,與宮同音,臣奪君之義,知司馬氏之將篡也。王陵、毋丘儉、文欽、諸葛誕繼為揚州都督,咸有興復之謀,皆為司馬懿父子所殺。康以揚州故廣陵地,陵等皆魏大臣,故名其曲曰《廣陵散》,言魏散亡自廣陵始。『止息』者,晉雖暴興,終止息於此。其哀憤、躁蹙、憯痛、迫脅之音,盡於是矣。永嘉之亂,其兆乎!康避晉、魏之禍,托以鬼神,以俟後世知音云。」



 洄字幼深,廕補弘文生,滿歲,參調吏部侍郎,達奚珣以地望抑之。除章懷太子陵令,無慍容。安祿山亂,家七人遇害,洄避難江南,蔬食不聽樂。乾元中,授睦州別駕,劉晏表為屯田員外郎,知揚子留後。召拜諫議大夫,與補闕李翰數上章言得失,擢知制誥。坐與元載善,貶邵州司戶參軍。德宗即位,起為淮南黜陟使,復為諫議大夫。



 晏被罪,天下錢穀歸尚書省,而省司廢久,無綱紀,莫總其任,乃擢洄戶部侍郎,判度支。洄上言:「江、淮七監,歲鑄錢四萬五千緡輸京師,工用運轉,每緡度二千,是本倍於子。今商州紅崖冶產銅,而洛源監久廢,請鑿山取銅,即治舊監,置十爐鑄之,歲得錢七萬二千緡,度費每緡九百,則得可浮本矣。江、淮七監,請皆罷。」又言:「天下銅鐵冶,乃山澤利,當歸王者,請悉隸鹽鐵使。」從之。復罷省胥史冗食二千人,積米長安、萬年二縣各數十萬石,視年豐耗而發斂焉,故人不艱食。



 洄與楊炎善,炎得罪,不自安。無何,皋上疏理炎罪,帝意洄教之,貶蜀州刺史。興元元年,入為兵部侍郎,轉京兆尹。貞元十年,終國子祭酒,贈戶部尚書。



 贊曰:人之立事,無不銳始而工於初,至其半則稍怠,卒而漫澶不振也。觀玄宗開元時,厲精求治,元老魁舊,動所尊憚,故姚元崇、宋璟言聽計行,力不難而功已成。及太平久,左右大臣皆帝自識擢,狎而易之,志滿意驕,而張九齡爭愈切,言益不聽。夫志滿則忽其所謀,意驕則樂軟熟、憎鯁切,較力雖多,課所效不及姚、宋遠矣。終之胡雛亂華,身播邊陬,非曰天運,亦人事有致而然。若知古等皆宰相選,使當天寶時,庸能有救哉!



\end{pinyinscope}