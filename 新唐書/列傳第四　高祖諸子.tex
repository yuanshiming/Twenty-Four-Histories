\article{列傳第四 高祖諸子}

\begin{pinyinscope}

 隱太子建成衛王玄霸巢王元吉楚王智云荊王元景漢王元昌酆王元亨周王元方徐王元禮韓王元嘉黃公言喿彭王元則鄭王元懿霍王元軌虢王鳳道王元慶鄧王元裕舒王元名魯王靈夔江王元祥密王元曉滕王元嬰



 高祖二十二子:竇皇后生建成、太宗皇帝、玄霸、元吉,萬貴妃生智云,莫嬪生元景,孫嬪生元昌,尹德妃生元亨,張氏生元方,郭婕妤生元禮,宇文昭儀生元嘉及第十九子靈夔,王才人生元則,張寶林生元懿,張美人生元軌,楊美人生鳳,劉婕妤生元慶,崔嬪生元裕,小楊嬪生元名,楊嬪生元祥,魯才人生元曉,柳寶林生元嬰。



 隱太子建成小字毘沙門。資簡弛,不治常檢,荒色嗜酒,畋獵無度,所從皆博徒大俠。



 隋末,高祖被詔捕賊汾、晉間,留建成護家,居河東。高祖已起兵,密召與元吉赴太原,隋人購之急,從間道至,授左領軍大都督,封隴西郡公。引兵略定西河,從平京師。唐國建,為世子,開府置官屬。又遷撫軍大將軍,為東討元帥,將萬人徇洛陽,授尚書令。



 高祖受禪,立為皇太子。詔率將軍桑顯和擊司竹群盜,平之。涼州人安興貴殺李軌,以眾降,詔趣原州應接。建成素驕,不恤士,雖甚署,晝夜馳獵,眾不堪其勞,亡者過半。帝欲其習事,乃敕非軍國大務聽裁決之。又以李綱、鄭善果為宮官,參謀議。稽胡劉屳成寇邊,詔建成進討,破之鄠州,斬虜千計,引渠長悉官之,使還招群胡。屳成與它大帥降,建成畏其眾,紿欲城州縣者,使降胡操築,陰勒兵殺六千人,屳成奔梁師都。嘗循行北邊,遇賊四百出降,悉馘其耳縱之。



 中允王珪、洗馬魏徵以帝初興,建成不知謀,而秦王數平劇寇,功冠天下,英豪歸之,陰許立為皇太子,勢危甚。會劉黑闥亂河北,珪等進說曰:「殿下特以嫡長居東宮,非有功德為人所稱道。今黑闥痍叛殘孽,眾不盈萬,利兵鏖之,唾手可決,請往討,因結山東英俊心,自封殖。」建成遂請行。黑闥敗洺水,建成問徵曰:「山東其定乎?」對曰:「黑闥雖敗,殺傷太甚,其魁黨皆縣名處死,妻子系虜,欲降無繇,雖有赦令,獲者必戮,不大蕩宥,恐殘賊嘯結,民未可安。」既而黑闥復振,廬江王瑗棄洺州,山東亂。命齊王元吉討之,有詔降者赦罪,眾不信。建成至,獲俘皆撫遣之,百姓欣悅。賊懼,夜奔,兵追戰。黑闥眾猶盛,乃縱囚使相告曰:「褫而甲還鄉里,若妻子獲者,既已釋矣。」眾乃散,或縛其渠長降,遂禽黑闥。



 帝晚多內寵,張婕妤、尹德妃最幸,親戚分事宮府。建成與元吉通謀,內結妃御以自固。當是時,海內未定,秦王數將兵在外,諸妃希所見。及洛陽平,帝遣諸妃馳閱後宮,見府庫服玩,皆私有求索,為兄弟請官。秦王已封帑簿,及官爵非有功不得,妃媛曹怨之。會為陜東道行臺,有詔屬內得專處決。王以美田給淮安王神通,而張婕妤為父丐之,帝手詔賜田,詔至,神通已得前,不肯與。婕妤妄曰:「詔賜妾父田,而王奪與人。」帝怒,召秦王讓曰:「我詔令不如爾教邪?」他日,謂裴寂曰:「兒久典兵,為儒生所誤,非復我昔日子。」秦府屬杜如晦騎過尹妃父門,恚其傲,率家童捽毆,折一指。父懼,即使妃前訴秦王左右暴其父,帝不察,大怒,詰王曰:「兒左右乃凌我妃家,況百姓乎?」王自辨曉,訖不置,繇是見疏。帝召諸王燕,秦王感母之不及有天下也,偶獨泣,帝顧不樂,妃媛因得中傷之,為建成游說曰:「海內無事,陛下春秋高,當自娛,秦王輒悲泣,正為嗔忌妾屬耳。使陛下萬歲後,王得志,妾屬無遺類。東宮慈愛,必能全養。」乃皆悲不自勝。帝惻然,遂無易太子意。



 突厥入寇,帝議遷都,秦王苦諫止。建成見帝曰:「秦王欲外禦寇,沮遷都議,以久其兵,而謀篡奪。」帝浸不悅。



 初,帝令秦王居西宮承乾殿,元吉居武德殿,與上臺、東宮晝夜往來,皆攜弓刀,相遇如家人禮。由是皇太子令、秦齊二王教與詔敕雜行,內外懼,莫知所從。建成等私募四方驍勇及長安惡少年二千人為宮甲,屯左右長林門,號「長林兵」。又令左虞候率可達志募幽州突厥兵三百內宮中,將攻西宮。或告於帝,帝召建成責謂,乃流志巂州。



 華陰楊文幹素兇詖,建成暱之,使為慶州總管,遣募兵送京師,欲為變。時帝幸仁智宮,秦王、元吉從,建成謂元吉曰:「秦王且遍見諸妃,彼金寶多,有以賂遺之也。吾安得箕踞受禍?安危之計決今日。」元吉曰:「善。」乃命郎將尒硃煥、校尉橋公山齎甲遺文乾,趣興兵。煥等懼,至豳鄉白反狀,寧州人杜鳳亦上變。帝遣司農卿宇文穎驛召文乾,元吉陰結穎,使告文乾,文乾遽率兵反。帝以建成首謀,未忍治,即詔捕王珪、魏徵及左衛率韋挺、舍人徐師〓、左衛車騎馮世立,欲殺之以薄太子罪。乃手詔召建成,建成懼,不敢往。師〓勸遂舉兵,詹事主簿趙弘智諫建成捐車服,輕往謝罪。乃詣行在所,未至,屏官屬,徑入謁,叩頭請死,投身於地,不能起。帝怒,夜囚幕中,使兵衛守。會文乾陷寧州,帝驚,以宮近賊,夜率衛士南趣,山行十餘里,明乃還宮。召秦王問計,對曰:「文乾豎子耳,官司當即禽之,就使假刻漏之久,正須遣一將可辦。」帝曰:「事連建成,恐應者眾。爾自行,還,吾以爾為太子,使建成王蜀,蜀地狹,不足為變,若不能事汝,取之易也。」秦王率眾趣寧州,文乾為其下所殺,以其首降,執宇文穎送京師。秦王之行,元吉及內嬖更為建成請,封德彞亦陰說帝,由是意解,復詔建成居守,但責兄弟不相容,而謫王珪、韋挺、天策兵曹參軍杜淹於遠方。然怨猜日結。



 建成等召秦王夜宴,毒酒而進之,王暴疾,〓血數升,淮安王扶掖還宮。帝問疾,因敕建成:「秦王不能酒,毋夜聚。」又謂秦王曰:「吾起晉陽,平天下,皆爾力,將定東宮,爾亟讓,故成而美志。又太子立多歷年,吾重奪之。觀而兄弟終不相下,同在京師,忿鬩且深。爾還洛陽行臺,自陜以東悉主之,建天子旌旗,如梁孝王故事。」王泣曰:「非所願也,不可遠膝下。」帝曰:「陸賈,漢臣也,猶遞過諸子,況我天下主,東西兩宮,思汝即往,何所悲邪?」王將行,建成等謀曰:「秦王得土地甲兵,必為患;留之京師,一匹夫耳。」因密使人說帝,言「秦王左右皆山東人,聞還洛,皆灑然喜,觀其意,不復來矣」。事果寢。



 俄而突厥寇邊,太子薦元吉北討,欲因其兵作亂。長孫無忌、房玄齡、杜如晦、尉遲敬德、侯君集等勸秦王先圖之。王乃密奏建成等與後宮亂,因曰:「臣無負兄弟,今乃欲殺臣,是為世充、建德復仇。使臣死,雖地下,愧見諸賊。」帝大驚,報曰:「旦日當窮治,而必早參。」張婕妤馳語建成,乃召元吉謀,曰:「請勒宮甲,托疾不朝。」建成曰:「善,然不共入朝,事何繇知?」遲明,乘馬至玄武門,秦王先至,以勇士九人自衛。時帝已召裴寂、蕭瑀、陳叔達、封德彞、宇文士及、竇誕、顏師古等入。建成、元吉至臨湖殿,覺變,遽反走,秦王隨呼之,元吉引弓欲射,不能彀者三。秦王射建成即死,元吉中矢走,敬德追殺之。俄而東宮、齊府兵三千攻玄武門,閉不得入。接戰久之,矢及殿屋。王左右數百騎至,合擊之,眾遂潰。帝謂裴寂等曰:「事今奈何?」蕭瑀、陳叔達曰:「臣聞內外無限,父子不親,失而弗斷,反蒙其亂。建成、元吉自草昧以來,未始與謀,既立,又無功德,疑貳相濟,為蕭墻憂。秦王功蓋天下,內外歸心,立為太子,付軍國大務,陛下釋重負矣。」帝曰:「此吾志也!」乃召秦王至,尉撫之曰:「朕幾有投杼之惑。」秦王號泣不能止。



 建成死年三十八。長子承宗為太原王,早卒;承道安陸王,承德河東王,承訓武安王,承明汝南王,承義巨鹿王,皆坐誅。詔除建成、元吉屬籍。其黨疑懼,更相告,廬江王瑗遂反。乃下詔建成、元吉、瑗支黨不得相告訐,由是遂安。太宗立,追封建成為息王,謚曰隱,以禮改葬,詔東宮舊臣皆會,帝於宜秋門哭之,以子福為後。十六年,追今贈。



 宇文穎者,代人。自李密所來降,為農圃監,封化政郡公。性貪昏,與元吉厚善,故豫文乾謀。事敗,帝責曰:「朕以文幹叛,故遣卿,乃同逆邪?」穎無以對,斬之。



 衛懷王玄霸字大德。幼辯惠。隋大業十年薨,年十六,無子。武德元年,追王及謚,又贈秦州總管、司空。以太宗子泰為宜都王,奉其祀,葬芷陽。泰徙封越,更以宗室西平王瓊子保定嗣。薨,無子,國除。



 巢刺王元吉小字三胡。高祖兵已西,留守太原,封姑臧郡公,進齊國,總十五郡諸軍事,加鎮北將軍、太原道行軍元帥。帝受禪,進王齊,為並州總管。



 初,元吉生,太穆皇后惡其貌,不舉,侍媼陳善意私乳之。及長,猜鷙好兵,居邊久,益驕侈。常令奴客、諸妾數百人被甲習戰,相擊刺,死傷甚眾。後元吉中創,善意止之,元吉恚,命壯士拉死,私謚慈訓夫人。



 劉武周略汾、晉,詔遣右衛將軍宇文歆助守。元吉喜鷹狗,出常載罝罔三十車,曰:「我寧三日不食,不可一日不獵。」夜潛出淫民家,府門不閉。歆驟諫,不納,乃顯表於帝曰:「王數出與竇誕縱獵,蹂民田,縱左右攘奪,畜產為盡。每射於道,觀人避矢以為樂。百姓怨毒。不可與共守。」有詔召還。元吉密諷民詣闕請,乃得歸。武周以五千騎屯黃蛇嶺,元吉使將軍張達以步卒百人嘗寇,達辭兵少,強之,至則盡沒。達怒,導武周陷榆次。元吉保祁,賊急攻之,遁還並州,賊張甚。元吉紿司馬劉德威曰:「公以老弱守,吾率銳士拒賊。」因齎寶物、攜妻妾夜出,委軍奔京師,並州陷。帝怒,自是嘗令從秦王征討,不復顓軍矣。



 尋授侍中、襄州道行臺尚書令、稷州刺史。秦王圍東都,竇建德來援,王以精騎逆戰,留元吉、屈突通守,而世充易之,輒出兵,元吉設伏劫之,斬首八百級,禽其將。東都平,拜司空,賜袞冕服、鼓吹二部、班劍二十人、黃金二千斤,與太子、秦王得三爐鑄錢。累進司徒,兼侍中、並州大都督。



 時秦王有功,而太子不為中外所屬,元吉喜亂,欲並圖之。乃構於太子曰:「秦王功業日隆,為上所愛,殿下雖為太子,位不安,不早計,還踵受禍矣,請為殿下殺之。」太子不忍,元吉數諷不已,許之。於是邀結宮掖,厚賂中書令封德彞,使為游說,帝遂疏秦王,愛太子。元吉乃多匿亡命壯士,厚賜之,使為用。元吉記室參軍榮九思為詩刺之曰:「丹青飾成慶,玉帛禮專諸。」元吉見之,弗悟也。其典簽裴宣儼免官,往事秦府,元吉疑事洩,鴆殺之。自是人莫敢言。秦王嘗從帝幸元吉第,伏護軍宇文寶寢內,將以刺王,太子固止之,元吉慍曰:「為兄計,於我何害?」



 突厥鬱射設入圍烏城,建成薦元吉北討,乃多引秦王府驍將秦叔寶、尉遲敬德、程知節、段志玄與行,又籍秦府精兵益麾下。帝知之,不能禁。元吉承間密請害秦王,帝曰:「是有定四海功,殺之無名。」元吉曰:「王昔平東都,顧望不即西,散金帛樹私惠,豈非反邪?」帝不應。太子與元吉謀:「兵行,吾與秦王至昆明池,伏壯士拉之,以暴卒聞,上無不信。然後說帝付吾國,吾以爾為皇太弟,而盡擊殺叔寶等。」率更令王晊密以謀告秦王,王召僚屬謀,皆曰:「元吉戾很,使得志,且不能事其兄。往者護軍薛寶以元吉字合之,其文成『唐』,元吉喜曰:『但除秦王,取東宮如反掌耳!』為亂未克,已復傾奪,大王不蚤正之,社稷非復唐有。」秦王由是定計。



 死年二十四。子承業為梁郡王,承鸞漁陽王,承獎普安王,承裕江夏王,承度義陽王,並伏誅。貞觀初,改葬,追爵海陵郡王及謚。後改封巢,以曹王明嗣。



 楚哀王智雲初名稚詮。善射,工書、弈。隋大業末,從建成寓河東。高祖初,建成走太原,吏捕智雲送長安,為陰世師所害,年十四。武德元年,追王及謚。



 母萬貴妃,性恭順,為帝所禮,宮中事一一咨決。



 三年,以太宗子寬為嗣,又贈涼州總管、司徒。寬早薨,國除。貞觀二年,復以濟南公世都子靈龜嗣,歷魏州刺史,為政威嚴,盜賊不發;鑿永濟渠,通新市,百姓利之。薨,子福嗣,降為公。卒,子承況嗣,神龍中為右羽林將軍,同節愍太子死於難。



 荊王元景,武德三年始王趙,與魯、酆二王同封。貞觀初,累遷雍州牧。十年,徙封荊。



 明年,詔荊州都督荊王元景、梁州都督漢王元昌、徐州都督徐王元禮、潞州都督韓王元嘉、遂州都督彭王元則、鄭州刺史鄭王元懿、絳州刺史霍王元軌、虢州刺史虢王鳳、豫州刺史道王元慶、鄧州刺史鄧王元裕、壽州刺史舒王元名、幽州都督燕王靈夔、蘇州刺史許王元祥、安州都督吳王恪、相州都督魏王泰、齊州都督齊王祐、益州都督蜀王愔、襄州刺史蔣王惲、揚州都督越王貞、並州都督晉王治、秦州都督紀王慎所任刺史並功臣令世世襲。會長孫無忌等固讓,遂廢不行。徙鄠州。永徽初,進位司徒,賜實封至千五百戶。



 房遺愛謀反,坐子則與往還系獄。時吳王亦抵罪,高宗謂大臣曰:「朕欲從公丐叔及兄死。」兵部侍郎崔敦禮曰:「陛下雖申恩,不可詘天下法。」遂賜死。久之,追封沈黎王,以渤海王奉慈子長沙嗣,降為侯。神龍初,復王爵,以孫逖嗣。薨,無子,國除。



 漢王元昌,初王魯,累遷梁州都督,後徙封漢。有勇力,善騎射。數觸軌憲,太宗手詔誨督,乃怨望,附太子承乾,通饋謝。來朝京師,宿東宮,嘗有醜語;又見帝側有宮人善琵琶,乃曰:「事成幸賜我。」承乾許之,割臂血盟。事敗,帝弗忍誅,欲免死,高士廉、李勣等固爭不奉詔,乃賜死,國除。



 酆悼王元亨,貞觀二年,授金州刺史,之籓,太宗憐其幼,思之,數遣使為勞問,賜金盞以娛樂之。六年薨,無子,國除。



 周王元方,武德四年始王,與鄭、宋、荊、滕四王同封。貞觀三年薨,無子,國除。



 徐康王元禮性恭畏,善騎射。始王鄭,即授鄭州刺史。後徙王徐,遷徐州都督。為絳州刺史,有治名,璽書勞勉,實封至千戶。永徽中,加司徒,兼潞州刺史。薨,贈太尉、冀州大都督,陪葬獻陵。



 三子,茂為淮南王,餘爵公。



 茂險薄無行。初,元禮疾,姬趙有美色,茂逼蒸之,元禮切責,茂恚,屏侍衛藥膳,曰:「為王五十年足矣,何服藥為?」以不食薨。茂嗣。上元中,事洩,流死振州。



 神龍初,以茂子璀嗣,開元中,為宗正員外卿。薨,子延年嗣。拔汗那王入朝,延年將以女嫁之,為右相李林甫劾奏,貶文安郡別駕,終餘杭司馬,國除。永泰初,延年婿黔中觀察使趙國珍言諸朝,詔以其子諷嗣王。



 韓王元嘉字元嘉。始王宋,後改王徐,為潞州刺史。母昭儀,宇文述女也,寵於高祖,既即位,欲立為後,固辭。元嘉以母寵故,特為帝愛,後出諸子無及者。在潞時,年十五,聞太妃病,涕泣不食。居喪毀甚,太宗數尉勉。少好學,藏書至萬卷,皆以古文字參定同異。與弟靈夔友愛,燕見終日如布衣禮,閨門修整,當世稱之。



 貞觀九年,更封韓,遷滑州都督。高宗末,為澤州刺史。武後得政,進授太尉,徙定州刺史,以霍王元軌為司徒,舒王元名為司空,滕王元嬰開府儀同三司,魯王靈夔太子太師,越王貞太子太傅,紀王慎太子太保,外示尊寵,而內將圖之。



 垂拱中,元嘉徙絳州刺史,與子譔及越王子沖糾合宗室同舉兵,未發。會武后詔宗室朝明堂,元嘉遣使告諸王曰:「大享後,太后必盡誅諸王,不如先事起。不然,李氏無種矣。」乃為中宗詔,督諸王發兵。沖即以兵五千攻濟州,而諸王倉卒兵不至,遂敗。元嘉至京師,謀洩,後逼令自殺,年七十。詔改氏元嘉、魯王、越王為「虺」。



 元嘉六子。訓,潁川王,蚤卒。誼,武陵王。諶,上黨公。譔,黃公,工為辭章,孟利貞嘗稱其文曰:「劉鄰之、周思茂不過也。」出為通州刺史,辭疾歸,且謀慮越王也。諶通音律,歷杭州別駕,與譔俱死。時籍沒者眾,惟沖、譔家書為多,皆文句詳正,秘府所不及。神龍初,追復元嘉爵士,以第五子訥嗣。薨,子叔璩嗣,歷國子司業。薨,子煒嗣。建中中,改王鄆。後懿宗以鄆王即位,復改嗣韓王云。



 彭思王元則字彞。初王荊,出為婺州刺史。貞觀十年徙王,為遂州都督,以冠服奢僭免。久之,為澧州刺史,更折節厲行。薨,贈司徒、荊州大都督,陪葬獻陵。高宗登望春宮,過其喪,哭之慟。



 無子,以霍王子絢嗣,龍朔中,封南昌王。薨,子志暕嗣,開元中,為宗正卿。



 鄭惠王元懿,始王滕,貞觀中,出為兗州刺史,徙王,厲鄭、潞、絳三州刺史,實封千戶。喜經術,數斷大獄,務寬平,高宗嘉之,璽詔褒錫。薨,贈司徒、荊州大都督,陪葬獻陵。



 十子,長子璥嗣王,為鄂州刺史。薨,子希言嗣,開元中,為右金吾大將軍,再為太子詹事。弟察言,生二子,曰自仙、〓。自仙為楚州別駕,生夷簡。〓為陳留公,生宗閔。璥弟琳,安德郡公,生擇言,擇言生勉。勉、宗閔、夷簡皆位宰相,別有傳,時稱小鄭王後,亦曰惠鄭王後,以別鄭王亮云。



 霍王元軌,武德六年始王蜀,與豳、漢二王同封,後徙吳。多材藝,高祖愛之。



 太宗嘗問群臣曰:「朕子弟孰賢?」魏徵曰:「臣愚不盡知其能,唯吳王數與臣言,未嘗不自失。」帝曰:「朕亦器之,然卿以為前代孰比?」對曰:「經學文雅,漢河間、東平也。至孝行,曾、閔不能過。」帝由是遇益厚。詔納徵女為妃。嘗從獵,遇群豕,帝使射之,筈不虛彀,豕為盡。帝撫其背曰:「爾藝過人,顧今無所施。方天下未定,得若豈不用乎?」



 貞觀七年,為壽州刺史。高祖崩,去官,毀瘠甚,服除,遂菜食布衣終身,至忌日,輒累晝不食。十年,徙王,歷絳、徐、定三州刺史,實封至千戶。所至閉閣讀書,以吏事委長史、司馬。謙慎未嘗與物忤。數引見處士劉玄平,為布衣交。或問王所長於玄平,答曰:「無長。」問者不解,玄平曰:「人有短,所以見長。若王無所不備,吾何以稱之。」



 突厥寇定州,元軌令開城門,偃旗幟,虜疑,不敢入,夜遁。州人李嘉運潛結賊,詔窮誅支黨,元軌以寇近且強,人心危,但殺嘉運,餘無所詰,因自劾。帝喜曰:「朕固悔之。非王之明,幾失定州矣。」



 王文操者,與賊戰,敗,二子鳳、賢更以身蔽父,得全,二子死。縣抑不為言,元軌廉知之,遣使員祭,上其事。詔贈鳳、賢朝散大夫,旌禮其閭。



 元軌每朝,數上疏陳得失,多所裨正,帝尊重之,有大事,常密驛咨逮。帝崩,與侍中劉齊賢同知山陵事。元軌淹練故事,齊賢嘆曰:「是非吾等及已!」嘗遣國令督封租,令請貿易取贏,答曰:「汝當正吾矣,反訹吾以利邪?」不納。進司徒,出為襄、青二州刺史。越王敗,坐嘗通謀,徙黔州,檻車載至陳倉,薨。



 六子,緒為江都王,純安定王,餘皆爵為公。緒有名譽,為金州刺史,誅。神龍初,並復官爵,以緒孫暉嗣王,開元中,為左千牛員外將軍。



 虢莊王鳳字季成。始王豳,為鄧州刺史。俄徙王,歷虢、豫、青三州刺史,實封千戶。喜畋游,遇官屬尤嫚。使奴蒙虎皮,怖其參軍陸英俊幾死,因大笑為樂。薨,贈司徒、揚州大都督,陪葬獻陵。



 七子,長子翼嗣,為平陽王。薨,子寓嗣。寓無子,爵不傳。次子茂融,以勇聞,垂拱中為申州刺史。黃公譔與越王謀舉兵,倚以為助。時詔諸王公赴東都,茂融私問所親高子貢,子貢報曰:「來必死。」乃稱疾不朝,以俟兵期。及得越王書,倉卒不能應,僚屬勸白其書,擢太子右贊善大夫,俄為黨屬所引,誅。



 中宗更以鳳孫邕嗣王,娶韋后妹,累遷秘書監,知隴右三使仗內諸廄。徙王汴。未幾,韋氏敗,邕殺其妻,送首於朝,議者鄙之。削爵,貶沁州刺史,不事。後復爵,還戶二百,累還衛尉卿。薨,子巨嗣。



 巨剛銳果決,略通書史,好屬辭。天寶五載,出為西河太守。坐資給柳勣支黨,貶義陽司馬。明年,御史中丞楊慎矜得罪,其附離史敬忠與巨善,又坐免官,錮置南賓郡。召拜夷陵太守。



 安祿山陷東京,玄宗方擇將帥,張〓言巨有謀,可屬大事。召至京師,楊國忠忌之,謂人曰:「小兒詎可使對天子?」逾月不得見。帝知之,召入禁中,對合旨,帝大悅,敕宰相與語,久不得罷,國忠怠,謂巨曰:「比來人多口打賊,君不爾乎?」巨曰:「誰為相公手打賊者乎?」乃授陳留、譙郡太守,攝御史大夫、河南節度使。明日謝,帝驚曰:「何攝為?」即詔兼御史大夫。巨奏:「方艱難時,賊多詐,有如陛下召臣,何以取信?」乃析契授之。



 俄兼統嶺南何履光、黔中趙國珍、南陽魯炅三節度使事。時炅戰數屈,詔貶為果毅,以來瑱代之。巨奏:「炅若能存孤城,功足補過,則何以處之?」帝曰:「卿隨所處置。」巨至內鄉,賊將畢思琛解圍走,遂趣南陽,貶炅白衣從軍,其暮,稱詔復職。



 京師平,拜留守,兼御史大夫。明年,拜太子少師,兼河南尹、東畿採訪使。征乘牛之出入市者,斥所得佐用度,然稍自盜沒。其妃即張皇后從女弟,內不睦。巨選府縣官備使令,妃亦引蒲博少年分黨招貨賄,橈政事。宗正卿李遵素私張,發巨贓事,貶遂州刺史。會段子璋反,道遂州,巨倉卒不知所出,即迎謁,為子璋所殺。



 子則之,嗜學,年五十餘,尚執經太學,嗣曹王皋薦之。貞元二年,繇睦王府長史遷左金吾衛大將軍。坐與從甥竇申善,貶昭州司馬。



 道孝王元慶,始王漢,後徙陳,出為趙州刺史。貞觀十年,徙王,授豫州刺史,累實封千戶。時諸王奉給薄於帝子,至數寠乏,大臣莫敢言。十八年,黃門侍郎褚遂良為太宗從容言之,不能行。高宗時,歷滑州刺史,以治績聞,數蒙褒賜。遷徐、沁、衛三州刺史。事母謹,及喪,請躬修墳墓,詔不聽。薨,贈司徒、益州都督,陪葬獻陵。



 九子,誘為嗣,王臨淮,為澧州刺史,坐贓削爵。更以次子詢之子微嗣,終宗正卿。子煉嗣,廣德中,亦至宗正卿。



 鄧康王元裕,貞觀五年始王鄶,十一年徙王。始王及徙,皆與譙、魏、許、密四王同封。累實封至千二百戶。



 好學,善談名理,與典簽盧照鄰為布衣交。五為州刺史,遷兗州都督。薨,贈司徒、益州大都督,陪葬獻陵。無子,以江王子廣平公炅嗣。薨,子孝先嗣,開元中,為冠軍大將軍。



 舒王元名,始王譙,後徙王。高祖之在大安宮,太宗晨夕使尚宮問起居,元名才十歲,保媼言:「尚宮有品當拜。」元名曰:「此帝侍婢耳,何拜為?」太宗壯之,曰:「真吾弟也!」及長,矜嚴疏財,未嘗問家人生業。歷五州刺史,實封至千戶。



 子豫章王亶,洛江州,有美政。高宗以元名善訓子,手詔褒美。又欲授元名大州,辭曰:「臣忝屬籍,豈以州郡為仕進資邪?」治石州二十年,數游山林,有高蹈意。垂拱中,徙鄭州,境接東畿,諸王貴戚為刺史者縱家人暴百姓,元名至,一革之,為治廉威。進加司空。



 武后時,亶為丘神勣所構,系詔獄死,元名坐遷利州,尋被殺。神龍初,詔復官爵,贈司徒。時少子鄅國公昭已卒,乃以亶子津嗣,開元中,為左威衛將軍。薨,子萬嗣。薨,子藻嗣。



 魯王靈夔,篤學,善草隸,通音律。初王魏,後王燕,為幽州都督。已而徙王,實封至千戶。頻歷五州刺史,遷太子太師。垂拱元年,徙相州,坐與越王謀起兵,流振州,自殺。



 子詵,為清河王,早夭。藹為範陽王,知越王必敗,白發其謀,得不誅。歷右散騎常侍,為酷吏所害。神龍初,悉追復王爵,以藹子道堅嗣。



 道堅方嚴有禮法,閨門肅如也。七為州刺史,遷國子祭酒。開元中,選授汴州刺史、河南道採訪使。州據水陸都會,前後刺史多瀆利,唯道堅以清毅稱。入為宗正卿。薨,贈禮部尚書。子宇嗣,從玄宗至蜀,為右金吾將軍。寶應初,皇太子子封魯王,更封宇為嗣鄒王。弟道邃封戴國公,恭默自守,以修山東婚姻故事,數任清職,終尚書右丞。



 江安王元祥,始王許,後徙王,四為州刺史,實封至千戶。性庸遴,所至營財產無厭。時滕、蔣、虢三王皆貪暴,得其府官者惡之不願行,故時語曰:「寧向儋、崖、振、白,不事江、滕、蔣、虢。」元祥魁大,帶十圍,食兼數人。韓、虢、魏亦鴻偉,然不逮也。薨,贈司徒、並州大都督,陪葬獻陵。



 七子,〓為永嘉王,有禽獸行,誅死;皎為武陽王,餘皆爵公,武后時,多及誅。皎子叢,以幼流死嶺表,葬南安,人號其塚為「天孫墓」。中宗立,以從子欽嗣王,又以皎封絕,更取弟子繼宗嗣,既而以郡王不襲,降澧國公。



 密貞王元曉,貞觀中為虢州刺史,實封至千戶。徙澤州。薨,贈司徒、揚州都督,陪葬獻陵。



 子穎嗣,為南安王。薨,子勖嗣,早薨。神龍初,以穎弟亮養子曇嗣。開元五年,更詔元曉再從孫東莞郡公徹嗣,徙封濮陽郡王,歷宗正卿、金紫光祿大夫。



 滕王元嬰,貞觀十三年始王,實封千戶。為金州刺史,驕縱失度。在太宗喪,集官屬燕飲歌舞,狎暱廝養;巡省部內,從民借狗求罝,所過為害;以丸彈人,觀其走避則樂;城門夜開,不復有節。高宗以書切責曰:「朕以王至親,不忍致於法,今署下上考,冀愧王心。」



 久之,遷洪州都督。官屬妻美者,紿為妃召,逼私之。嘗為典簽崔簡妻鄭嫚罵,以履抵元嬰面血流,乃免。元嬰慚,歷旬不視事。後坐法削戶及親事帳內之半,謫置滁州。起授壽州刺史,徙隆州,復不循法。隸事參軍事裴聿諫正其失,元嬰捽辱之。聿入計具奏,帝遷聿六品上階。帝嘗賜諸王彩五百,以元嬰及蔣王貪黷,但下書曰:「滕叔、蔣弟不須賜,給麻二車,助為錢緡。」二王大慚。武后時,進拜開府儀同三司、梁州都督。薨,贈司徒、冀州都督,陪葬獻陵。



 子十八人,長子修琦嗣,為長樂王,餘爵公。垂拱中,六人死詔獄。神龍初,更以少子修信子涉嗣,開元中,授左驍衛將軍。薨,子湛然嗣,從玄宗至蜀,擢左金吾將軍。



\end{pinyinscope}