\article{志第四十三 食貨三}

\begin{pinyinscope}

 唐都長安,而關中號稱沃野,然其土地狹,所出不足以給京師、備水旱,故常轉漕東南之粟。高祖、太宗之時級專政的性質及其任務;批判和繼承的關系;群眾、階級、政,用物有節而易贍,水陸漕運,歲不過二十萬石,故漕事簡。自高宗已後,歲益增多,而功利繁興,民亦罹其弊矣。



 初,江淮漕租米至東都輸含嘉倉,以車或馱陸運至陜。而水行來遠,多風波覆溺之患,其失常十七八,故其率一斛得八斗為成勞。而陸運至陜,才三百里,率兩斛計傭錢千。民送租者,皆有水陸之直,而河有三門底柱之險。顯慶元年,苑西監褚朗議鑿三門山為梁,可通陸運。乃發卒六千鑿之,功不成。其後,將作大匠楊務廉又鑿為棧,以輓漕舟。輓夫系二鈲於胸,而繩多絕,輓夫輒墜死,則以逃亡報,因系其父母妻子,人以為苦。



 開元十八年,宣州刺史裴耀卿朝集京師,玄宗訪以漕事,耀卿條上便宜曰:「江南戶口多,而無徵防之役。然送租、庸、調物,以歲二月至楊州入斗門,四月已後,始渡淮入汴,常苦水淺,六七月乃至河口,而河水方漲,須八九月水落始得上河入洛,而漕路多梗,船檣阻隘。江南之人不習河事,轉雇河師水手,重為勞費。其得行日少,阻滯日多。今漢、隋漕路,瀕河倉稟,遺跡可尋。可於河口置武牢倉,鞏縣置洛口倉,使江南之舟不入黃河,黃河之舟不入洛口。而河陽、柏崖、太原、永豐、渭南諸倉,節級轉運,水通則舟行,水淺則寓於倉以待,則舟無停留,而物不耗失。此甚利也。」玄宗初不省。二十一年,耀卿為京兆尹,京師雨水,穀踴貴。玄宗將幸東都,復問耀卿漕事,耀卿因請「罷陜陸運,而置倉河口,使江南漕舟至河口者,輸粟於倉而去,縣官雇舟以分入河、洛。置倉三門東西,漕舟輸其東倉,而陸運以輸西倉,復以舟漕,以避三門之水險。」玄宗以為然。乃於河陰置河陰倉,河清置柏崖倉;三門東置集津倉,西置鹽倉;鑿山十八里以陸運。自江、淮漕者,皆輸河陰倉,自河陰西至太原倉,謂之北運,自太原倉浮渭以實關中。玄宗大悅,拜耀卿為黃門侍郎、同中書門下平章事,兼江淮都轉運使,以鄭州刺史崔希逸、河南少尹蕭炅為副使,益漕晉、絳、魏、濮、邢、貝、濟、博之租輸諸倉,轉而入渭。凡三歲,漕七百萬石,省陸運傭錢三十萬緡。



 是時,民久不罹兵革,物力豐富,朝廷用度亦廣,不計道里之費,而民之輸送所出水陸之直,增以「函腳」、「營窖」之名,民間傳言用鬥錢運斗米,其縻耗如此。及耀卿罷相,北運頗艱,米歲至京師才百萬石。二十五年,遂罷北運。而崔希逸為河南陜運使,歲運百八十萬石。其後以太倉積粟有餘,歲減漕數十萬石。



 二十九年,陜郡太守李齊物鑿砥柱為門以通漕,開其山顛為輓路,燒石沃醯而鑿之。然棄石入河,激水益湍怒,舟不能入新門,候其水漲,以人輓舟而上。天子疑之,遣宦者按視,齊物厚賂使者,還言便。齊物入為鴻臚卿,以長安令韋堅代之,兼水陸運使。堅治漢、隋運渠,起關門,抵長安,通山東租賦。乃絕灞、滻,並渭而東,至永豐倉與渭合。又於長樂坡瀕苑墻鑿潭於望春樓下,以聚漕舟。堅因使諸舟各揭其郡名,陳其土地所產寶貨諸奇物於袱上。先時民間唱俚歌曰「得體紇那邪」。其後得寶符於桃林,於是陜縣尉崔成甫更《得體歌》為《得寶弘農野》。堅命舟人為吳、楚服,大笠、廣袖、芒屩以歌之。成甫又廣之為歌辭十闋,自衣缺後綠衣、錦半臂、紅抹額,立第一船為號頭以唱,集兩縣婦女百餘人,鮮服靚妝,鳴鼓吹笛以和之。眾艘以次輳樓下,天子望見大悅,賜其潭名曰廣運潭。是歲,漕山東粟四百萬石。自裴耀卿言漕事,進用者常兼轉運之職,而韋堅為最。



 初,耀卿興漕路,請罷陸運,而不果廢。自景雲中,陸運北路分八遞,雇民車牛以載。開元初,河南尹李傑為水陸運使,運米歲二百五十萬石,而八遞用車千八百乘。耀卿罷久之,河南尹裴迥以八遞傷牛,乃為交場兩遞,濱水處為宿場,分官總之,自龍門東山抵天津橋為石堰以遏水。其後大盜起,而天下匱矣。



 肅宗末年,史朝義兵分出宋州,淮運於是阻絕,租庸鹽鐵溯漢江而上。河南尹劉晏為戶部侍郎,兼句當度支、轉運、鹽鐵、鑄錢使,江淮粟帛,繇襄、漢越商於以輸京師。



 及代宗出陜州,關中空窘,於是盛轉輸以給用。廣德二年,廢句當度支使,以劉晏顓領東都、河南、淮西、江南東西轉運、租庸、鑄錢、鹽鐵,轉輸至上都,度支所領諸道租庸觀察使,凡漕事亦皆決於晏。晏即鹽利顧傭分吏督之,隨江、汴、河、渭所宜。故時轉運船繇潤州陸運至揚子,斗米費錢十九,晏命囊米而載以舟,減錢十五;繇揚州距河陰,斗米費錢百二十,晏為歇皇支江船二千艘,每船受千斛,十船為綱,每綱三百人,篙工五十,自揚州遣將部送至河陰,上三門,號「上門填闕船」,米斗減錢九十。調巴、蜀、襄、漢麻枲竹筱為綯挽舟,以朽索腐材代薪,物無棄者。未十年,人人習河險。江船不入汴,汴船不入河,河船不入渭;江南之運積揚州,汴河之運積河陰,河船之運積渭口,渭船之運入太倉。歲轉粟百一十萬石,無升斗溺者。輕貨自揚子至汴州,每馱費錢二千二百,減九百,歲省十餘萬緡。又分官吏主丹楊湖,禁引溉,自是河漕不涸。大歷八年,以關內豐穰,減漕十萬石,度支和糴以優農。晏自天寶末掌出納,監歲運,知左右藏,主財穀三十餘年矣。及楊炎為相,以舊惡罷晏,罷運使復歸度支,凡江淮漕米,以庫部郎中崔河圖主之。



 及田悅、李惟岳、李納、梁崇義拒命,舉天下兵討之,諸軍仰給京師。而李納、田悅兵守渦口,梁崇義搤襄、鄧,南北漕引皆絕,京師大恐。江淮水陸轉運使杜佑以秦、漢運路出浚儀十里入琵琶溝,絕蔡河,至陳州而合,自隋鑿汴河,官漕不通,若導流培岸,功用甚寡;疏雞鳴岡首尾,可以通舟,陸行才四十里,則江、湖、黔中、嶺南、蜀、漢之粟可方舟而下,繇白沙趣東關,歷潁、蔡,涉汴抵東都,無濁河溯淮之阻,減故道二千餘里。會李納將李洧以徐州歸命,淮路通而止。戶部侍郎趙贊又以錢貨出淮迂緩,分置汴州東西水陸運兩稅鹽鐵使,以度支總大綱。



 貞元初,關輔宿兵,米斗千錢,太倉供天子六宮之膳不及十日,禁中不能釀酒,以飛龍駝負永豐倉米給禁軍,陸運牛死殆盡。德宗以給事中崔造敢言,為能立事,用為相。造以江、吳素嫉錢穀諸使顓利罔上,乃奏諸道觀察使、刺史選官部送兩稅至京師,廢諸道水陸轉運使及度支巡院、江淮轉運使,以度支、鹽鐵歸尚書省,宰相分判六尚書事。以戶部侍郎元琇判諸道鹽鐵、榷酒,侍郎吉中孚判度支諸道兩稅。增江淮之運,浙江東、西歲運米七十五萬石,復以兩稅易米百萬石,江西、湖南、鄂岳、福建、嶺南米亦百二十萬石,詔浙江東、西節度使韓滉,淮南節度使杜亞運至東、西渭橋倉。諸道有鹽鐵處,復置巡院。歲終宰相計課最。崔造厚元琇,而韓滉方領轉運,奏國漕不可改。帝亦雅器滉,復以為江淮轉運使。元琇嫉其剛,不可共事,因有隙。琇稱疾罷,而滉為度支、諸道鹽鐵、轉運使,於是崔造亦罷。滉遂劾琇常餫米淄青、河中,而李納、懷光倚以構叛,貶琇雷州司戶參軍,尋賜死。



 是時,汴宋節度使春夏遣官監汴水,察盜灌溉者。歲漕經底柱,覆者幾半。河中有山號「米堆」,運舟入三門,雇平陸人為門匠,執標指麾,一舟百日乃能上。諺曰:「古無門匠墓。」謂皆溺死也。陜虢觀察使李泌益鑿集津倉山西逕為運道,屬於三門倉,治上路以回空車,費錢五萬緡。下路減半;又為入渭船,方五板,輸東渭橋太倉米至凡百三十萬石,遂罷南路陸運。其後諸道鹽鐵、轉運使張滂復置江淮巡院。及浙西觀察使李錡領使,江淮堰埭隸浙西者,增私路小堰之稅,以副使潘孟陽主上都留後。李巽為諸道轉運、鹽鐵使,以堰埭歸鹽鐵使,罷其增置者。自劉晏後,江淮米至渭橋浸減矣,至巽乃復如晏之多。



 初,揚州疏太子港、陳登塘,凡三十四陂,以益漕河,輒復堙塞。淮南節度使杜亞乃浚渠蜀岡,疏句城湖、愛敬陂,起堤貫城,以通大舟。河益庳,水下走淮,夏則舟不得前。節度使李吉甫築平津堰,以洩有餘,防不足,漕流遂通。然漕益少,江淮米至渭橋者才二十萬斛。諸道鹽鐵、轉運使盧坦糴以備一歲之費,省冗職八十員。自江以南,補署皆剸屬院監,而漕米亡耗於路頗多。刑部侍郎王播代坦,建議米至渭橋五百石亡五十石者死。其後判度支皇甫鎛議萬斛亡三百斛者償之,千七百斛者流塞下,過者死;盜十斛者流,三十斛者死。而覆船敗輓,至者不得十之四五。部吏舟人相挾為奸,榜笞號苦之聲聞於道路,禁錮連歲,赦下而獄死者不可勝數。其後貸死刑,流天德五城,人不畏法,運米至者十亡七八。鹽鐵、轉運使柳公綽請如王播議加重刑。太和初,歲旱河涸,掊沙而進,米多耗,抵死甚眾,不待覆奏。



 秦、漢時故漕興成堰,東達永豐倉,咸陽縣令韓遼請疏之,自咸陽抵潼關三百里,可以罷車輓之勞。宰相李固言以為非時,文宗曰:「茍利於人,陰陽拘忌,非朕所顧也。」議遂決。堰成,罷輓車之牛以供農耕,關中賴其利。



 故事,州縣官充綱,送輕貨四萬,書上考。開成初,為長定綱,州擇清強官送兩稅,至十萬遷一官,往來十年者授縣令。江淮錢積河陰,轉輸歲費十七萬餘緡,行綱多以盜抵死。判度支王彥威置縣遞群畜萬三千三百乘,使路傍民養以取傭,日役一驛,省費甚博。而宰相亦以長定綱命官不以材,江淮大州,歲授官者十餘人,乃罷長定綱,送五萬者書上考,七萬者減一選,五十萬減三選而已。及戶部侍郎裴休為使,以河瀕縣令董漕事,自江達渭,運米四十萬石。居三歲,米至渭橋百二十萬石。



 凡漕達於京師而足國用者,大略如此。其他州、縣、方鎮,漕以自資,或兵所征行,轉運以給一時之用者,皆不足紀。



 唐開軍府以捍要沖,因隙地置營田,天下屯總九百九十二。司農寺每屯三十頃,州、鎮諸軍每屯五十頃。水陸腴瘠、播殖地宜與其功庸煩省、收率之多少,皆決於尚書省。苑內屯以善農者為屯官、屯副,御史巡行蒞輸。上地五十畝,瘠地二十畝,稻田八十畝,則給牛一。諸屯以地良薄與歲之豐兇為三等,具民田歲穫多少,取中熟為率。有警,則以兵若夫千人助收。隸司農者,歲三月,卿、少卿循行,治不法者。凡屯田收多者,褒進之。歲以仲春籍來歲頃畝、州府軍鎮之遠近,上兵部,度便宜遣之。開元二十五年,詔屯官敘功以歲豐兇為上下。鎮戍地可耕者,人給十畝以供糧。方春,屯官巡行,謫作不時者。天下屯田收穀百九十餘萬斛。



 初,度支歲市糧於北都,以贍振武、天德、靈武、鹽、夏之軍,費錢五六十萬緡,溯河舟溺甚眾。建中初,宰相楊炎請置屯田於豐州,發關輔民鑿陵陽渠以增溉。京兆尹嚴郢嘗從事朔方,知其利害,以為不便,疏奏不報。郢又奏:「五城舊屯,其數至廣,以開渠之糧貸諸城,約以冬輸;又以開渠功直布帛先給田者,據估轉穀。如此則關輔免調發,五城田闢,比之浚渠利十倍也。」時楊炎方用事,郢議不用,而陵陽渠亦不成。然振武、天德良田,廣袤千里。



 元和中,振武軍饑,宰相李絳請開營田,可省度支漕運及絕和糴欺隱。憲宗稱善,乃以韓重華為振武、京西營田、和糴、水運使,起代北,墾田三百頃,出贓罪吏九百餘人,給以耒耜、耕牛,假種糧,使償所負粟,二歲大熟。因募人為十五屯,每屯百三十人,人耕百畝,就高為堡,東起振武,西逾雲州,極於中受降城,凡六百餘里,列柵二十,墾田三千八百餘頃,歲收粟二十萬石,省度支錢二千餘萬緡。重華入朝,奏請益開田五千頃,法用人七千,可以盡給五城。會李絳已罷,後宰相持其議而止。憲宗末,天下營田皆雇民或借庸以耕,又以瘠地易上地,民間苦之。穆宗即位,詔還所易地,而耕以官兵。耕官地者,給三之一以終身。靈武、邠寧,土廣肥而民不知耕。大和末,王起奏立營田。後黨項大擾河西,邠寧節度使畢諴亦募士開營田,歲收三十萬斛,省度支錢數百萬緡。



 貞觀、開元後,邊土西舉高昌、龜茲、焉耆、小勃律,北抵薛延陀故地,緣邊數十州戍重兵,營田及地租不足以供軍,於是初有和糴。牛仙客為相,有彭果者獻策廣關輔之糴,京師糧稟益羨,自是玄宗不復幸東都。天寶中,歲以錢六十萬緡賦諸道和糴,鬥增三錢,每歲短遞輸京倉者百餘萬斛。米賤則少府加估而糴,貴則賤價而糶。



 貞元初,吐蕃劫盟,召諸道兵十七萬戍邊。關中為吐蕃蹂躪者二十年矣,北至河曲,人戶無幾,諸道戍兵月給粟十七萬斛,皆糴於關中。宰相陸贄以「關中穀賤,請和糴,可至百餘萬斛。計諸縣船車至太倉,穀價四十有餘,米價七十,則一年和糴之數當轉運之二年,一斗轉運之資當和糴之五斗。減轉運以實邊,存轉運以備時要。江淮米至河陰者罷八十萬斛,河陰米至太原倉者罷五十萬,太原米至東渭橋者罷二十萬。以所減米糶江淮水菑州縣,斗減時五十以救乏。京城東渭橋之糴,鬥增時三十以利農。以江淮糶米及減運直市絹帛送上都。」帝乃命度支增估糴粟三十三萬斛,然不能盡用贄議。憲宗即位之初,有司以歲豐熟,請畿內和糴。當時府、縣配戶督限,有稽違則迫蹙鞭撻,甚於稅賦,號為和糴,其實害民。



\end{pinyinscope}