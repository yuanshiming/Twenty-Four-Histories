\article{志第十三下 儀衛下}

\begin{pinyinscope}

 太皇太后、皇太后、皇后出,尚儀版奏「請中嚴」。尚服率司仗布侍衛,司賓列內命婦於庭,西向北上,六尚以下詣室奉迎事實的描寫和記錄,它們都是由人的主觀感覺所構成的,並,尚服負寶,內僕進車於閣外,尚儀版奏「外辦」。馭者執轡,太皇太后乘輿以出,華蓋,侍衛,警蹕,內命婦從。



 出門,太皇太后升車,從官皆乘馬,內命婦、宮人以次從。



 清游隊,旗一,執者一人,佩橫刀,引、夾皆二人,佩弓箭、橫刀,騎。次金吾衛折沖都尉一人,佩橫刀、弓箭;領騎四十,亦佩橫刀,夾折沖;執槊二十人,持弩四人,佩弓箭十六人,持槊、刀二人。次虞候佽飛二十八人,騎,佩弓箭、橫刀,夾道分左右,以屬黃麾仗。



 次內僕令一人在左,丞一人在右,各書令史二人騎從。次黃麾一,執者一人,夾道二人,皆騎。次左右廂黃麾仗,廂皆三行,行百人。第一短戟,五色氅,執者黃地白花綦襖、冒;第二戈,五色氅,執者赤地黃花綦襖、冒;第三鍠,五色幡,執者青地赤花綦襖、冒。左右衛、左右威衛、左右武衛、左右驍衛、左右領軍衛各三行,行二十人,每衛以主帥六人主之,皆豹文袍、冒,執鍮石裝長刀,騎,唯左右領軍衛減三人。每衛果毅都尉一人,被繡袍,各一人從;左右領軍衛有絳引幡,引前者三,掩後者三。



 次內謁者監四人,給事二人,內常侍二人,內侍少監二人,騎,分左右,皆有內給使一人從。次內給使百二十人,平巾幘、大口褲、緋裲襠,分左右,屬於宮人車。次偏扇、團扇、方扇皆二十四,宮人執之,衣彩大袖裙襦、彩衣、革帶、履,分左右。次香蹬一,內給使四人輿之,居重翟車前。



 次重翟車,駕四馬,駕士二十四人。次行障六,次坐障三,皆左右夾車,宮人執之,服同執扇。次內寺伯二人,領寺人六人,執御刀,服如內給使,夾重翟車。次腰輿一,執者八人,團雉尾扇二,夾輿。次大繖四。次雉尾扇八,左右橫行,為二重。次錦花蓋二,單行。次小雉尾扇、硃畫團扇皆十二,橫行。次錦曲蓋二十,橫行,為二重。次錦六柱八,分左右。自腰輿以下,皆內給使執之。



 次宮人車。次降麾二,分左右。次後黃麾一,執者一人,夾二人,皆騎。次供奉宮人,在黃麾後。



 次厭翟車、翟車、安車,皆駕四馬,駕士各二十四人;四望車,駕士二十二人;金根車,駕牛,駕士十二人。



 次左右廂衙門各二,每門二人執,四人夾,皆赤綦襖,黃袍、冒,騎。



 次左右領軍衛,廂皆一百五十人,執殳,赤地黃花綦襖、冒,前屬於黃麾仗,後盡鹵簿;廂各主帥四人主之,皆黃袍、冒,執鍮石裝長刀,騎。折沖都尉二人,檢校殳仗,皆一人騎從。次衙門一,盡鹵簿後殳仗內正道,每門監門校尉二人主之,執銀裝長刀;廂各有校尉一人,騎,佩銀橫刀,往來檢校。御馬減大駕之半。



 太皇太后將還,三嚴,內典引引外命婦出次,就位;司賓引內命婦出次,序立大次之前。既外辦,馭者執轡。太皇太后乘輿出次,華蓋、警蹕、侍衛如初。內命婦以下乘車以從。車駕入,內典引引外命婦退,駕至正殿門外,車駕南問,尚儀前奏「請降車」。將士還。



 皇太子出,則鹵簿陳於重明門外。其日三刻,宮臣皆集於次,左庶子版奏「請中嚴」。典謁引宮臣就位,侍衛官服其器服,左庶子負璽詣閣奉迎,僕進車若輦於西閣外,南向,內率一人執刀立車前,北向,中允一人立侍臣之前,贊者二人立中允之前。前二刻,諸衛之官詣閣奉迎,宮臣應從者各出次,立於門外,文東武西,重行北向北上。



 左庶子版奏「外辦」,僕升正位執轡,皇太子乘輿而出,內率前執轡,皇太子升車,僕立授綏,左庶子以下夾侍。中允奏:「請發」,車動,贊者夾引而出,內率夾車而趨,出重明門,中允奏「請停車,侍臣上馬」。左庶子前承令,退稱:「令曰諾」。中允退稱:「侍臣上馬。」贊者承傳,侍臣皆騎。中允奏「請車右升」。左庶子前承令,退稱:「令曰諾」。內率升訖,中允奏「請發」。車動,鼓吹振作,太傅乘車訓導,少傅乘車訓從。出延喜門,家令先導,次率更令、詹事、太保、太傅、太師,皆軺車,備鹵薄。



 次清游隊,旗一,執者一人,佩橫刀,引、夾皆二人,亦佩弓箭、橫刀,騎。次清道率府折沖都尉一人,佩弓箭、橫刀,領騎三十,亦佩橫刀,十八人執槊,九人挾弓箭,三人持弩,各二人騎從。次左右清道率、府率各一人,騎,佩橫刀、弓箭,領清道直蕩及檢校清游隊各二人,執槊騎從。次外清道直蕩二十四人,騎,佩弓箭、橫刀,夾道。



 次龍旗六,各一人騎執,佩橫刀,戎服大袍,橫行正道,每旗前後二人騎,為二重,前引後護,皆佩弓箭、橫刀,戎服大袍。次副竿二,分左右,各一人騎執。次細引六重,皆騎,佩橫刀,每重二人,自龍旗後屬於細仗,槊、弓箭相間,廂各果毅都尉一人主之。



 次率更丞一人,府、史二人騎從,領鼓吹。次誕馬十,分左右,執者各二人。次廄牧令一人居左,丞一人居右,各府、史二人騎從。



 次左右翊府郎將二人,主班劍。次左右翊衛二十四人,執班劍,分左右。次通事舍人四人、司直二人、文學四人、洗馬二人,司議郎二人居左,太子舍人二人居右,中允二人居左,中舍人二人居右,左右諭德二人,左右庶子四人,騎,分左右,皆一人從。次左右衛率府副率二人步從。



 次親、勛、翊衛,廂各中郎將、郎將一人,皆領儀刀六行:第一親衛二十三人,第二親衛二十五人,皆執金銅裝儀刀,纁硃綬紛;第三勛衛二十七人,第四勛衛二十九人,皆執銀裝儀刀,綠綟紛;第五翊衛三十一人,第六翊衛三十三人,皆執鍮石裝儀刀,紫黃綬紛。自第一行有曲折三人陪後門,每行加一人,至第六行八人。次三衛十八人,騎,分左右夾路。



 次金路,駕四馬,駕士二十三人,僕寺僕馭,左右率府率二人執儀刀陪乘。次左右衛率府率二人,夾路,各一人從,居供奉官後。次左右內率府率二人,副率二人,領細刀、弓箭,皆一人從。次千牛,騎,執細刀、弓箭。次三衛儀刀仗,後開衙門。次左右監門率府直長各六人,執鍮石儀刀,騎,監後門。次左右衛率府,廂各翊衛二隊,皆騎,在執儀刀行外;壓角隊各三十人,騎,佩橫刀,一人執旗,二人引,二人夾,十五人執槊,二人佩弓箭,三人佩弩,隊各郎將一人主之。



 次繖,二人執,雉尾扇四,夾繖。次腰輿一,執者八人,團雉尾扇二,小方雉尾扇八,以夾腰輿,內直郎二人主之,各令史二人騎從。次誕馬十,分左右,馭者各二人。次典乘二人,各府、史二人騎從。次左右司禦率府校尉二人騎從,佩鍮石裝儀刀,領團扇、曲蓋。次硃漆團扇六,紫曲蓋六,各橫行。次諸司供奉。次左右清道率府校尉二人,騎,佩鍮石裝儀刀,主大角。



 次副路,駕四馬,駕士二十二人;軺車,駕一馬,駕士十四人;四望車,駕一馬,駕士十人。



 次左右廂步隊十六,每隊果毅都尉一人,領騎二十八,戎服大袍,佩橫刀,一人執旗,二人引,二人夾,二十五人佩弓箭,前隊持槊,與佩弓箭隊以次相間。次左右司禦率府副率各一人,騎,檢校步隊,二人執槊騎從。



 次儀仗,左右廂各六色,每色九行,行六人,赤綦襖、冒,行滕、鞋襪。第一戟,赤氅,六人;第二弓箭,六人;第三儀鋋,毦,六人;第四刀楯,六人;第五儀鍠,五色幡,六人;第六油戟,六人。次前仗首,左右廂各六色,每色三行,行六人,左右司禦率府二人,果毅都尉各一人,主帥各六人主之;次左右廂各六色,每色三行,行六人,左右衛率府副率二人,果毅都尉各一人,主帥各六人主之。左右司禦率府主帥各六人,騎,護後,率及副率各一人步從。廂有絳引幡十二,引前者六,引後者六。廂各有獨揭鼓六重,重二人,居儀仗外、殳仗內,皆赤綦襖、冒,行滕、鞋襪。左右司禦率府四重,左右衛率府二重。



 次左右廂皆百五十人,左右司禦率府各八十六人,左右衛率府各六十四人,赤綦襖、冒,主殳,分前後,居步隊外、馬隊內。各司禦率府果毅都尉一人主之,各一人騎從。廂各主帥七人,左右司禦率府各四人,左右衛率府各三人,騎,分前後。



 次左右廂馬隊,廂各十隊,隊有主帥以下三十一人,戎服大袍,佩橫刀,騎。隊有旗一,執者一人,引、夾各二人,皆佩弓箭,十六人持槊,七人佩弓箭,三人佩弩。第一,左右清道率府果毅都尉二人主之;第二、第三、第四,左右司禦率府果毅都尉二人主之;第五、第六、第七,左右衛率府果毅都尉主之;第八、第九、第十,左右司禦率府果毅都尉二人主之;皆戎服大袍,佩弓箭、橫刀。



 次後拒隊,旗一,執者佩橫刀,引、夾路各二人,佩弓箭、橫刀。次清道率府果毅都尉一人,領四十騎,佩橫刀;凡執槊二十人,佩弓箭十六人,佩弩四人,騎從。次後拒隊,前當正道殳仗內,有衙門。次左右廂各有衙門三:第一,當左右司禦率府步隊後,左右衛率府步隊前;第二,當左右衛率府步隊後,左右司禦率府儀仗前;第三,當左右司禦率府儀仗後,左右衛率府步隊前。每門二人執,四人夾,皆騎,赤綦襖,黃袍、冒。門有監門率府直長二人檢校,左右監門率府副率各二人檢校諸門,各一人騎從。次左右清道率府、副率各二人,檢校仗內不法,各一人騎從。次少師、少傅、少保,正道乘路,備鹵簿,文武以次從。



 皇太子所至,回車南向,左庶子跪奏「請降路」。



 還宮。一嚴,轉仗衛於還塗;再嚴,左庶子版奏「請中嚴」;三嚴,僕進車,左庶子版奏「外辦」。皇太子乘輿出門外,降輿,乘車,左庶子請車右升,侍臣皆騎。車動,至重明門,宮官下馬,皇太子乘車而入,太傅、少傅還。皇太子至殿前,車南向,左庶子奏「請降」。皇太子乘輿而入,侍臣從至閣,左庶子版奏「解嚴」。



 若常行、常朝,無馬隊、鼓吹、金路、四望車、家令、率更令、詹事、太保、太師、少保、少師,又減隊仗三之一,清道、儀刀、誕馬皆減半,乘軺車而已。二傅乘犢車,導從十人,太傅加清道二人。



 皇太子妃鹵簿:清道率府校尉六人,騎,分左右,為三重,佩橫刀、弓箭。次青衣十人,分左右。次導客舍人四人,內給使六十人,皆分左右,後屬內人車。次偏扇、團扇、方扇各十八,分左右,宮人執者間彩衣、革帶。次行障四,坐障二,宮人執以夾車。次典內二人,騎,分左右。次厭翟車,駕三馬,駕士十四人。次閣帥二人,領內給使十八人,夾車。次六柱二,內給使執之。次供奉內人,乘犢車。次繖一,雉尾扇二,團扇四,曲蓋二,皆分左右,各內給使執之。次戟九十,執者絳綦襖、冒,分左右。



 親王鹵簿:有清道六人為三重,武弁、硃衣、革帶。次幰弩一,執者平巾幘、緋褲褶,騎。次青衣十二人,平巾青幘、青布褲褶,執青布仗袋,分左右。次車輻十二,分左右。車輻,棒也,夾車而行,故曰車輻。執者服如幰弩。次戟九十,執者絳綦襖、冒,分左右。次絳引幡六,分左右,橫行,以引刀、楯、弓、箭、槊。次內第一行廂,執刀楯,絳綦襖、冒。第二行廂,執弓矢,戎服。第三行廂,執槊,戎服大袍。廂各四十人。次節一,夾槊一,各一人騎執,平巾幘、大口褲、緋衫。次告止幡四,傳教幡四,信幡八。凡幡皆絳為之,署官號,篆以黃,飾以鳥翅,取其疾也,金塗鉤,竿長一丈一尺,執者服如夾槊,分左右。次儀鋋二,儀鍠六,油戟十八,儀槊十,細槊十,執者皆絳綦襖、冒。次儀刀十八,執者服如夾槊,分左右。次誕馬八,馭者服如夾槊,分左右。次府佐六人,平巾幘、大口褲、緋裲襠,騎,持刀夾引。次象路一,駕四馬,佐二人立侍:一人武弁、硃衣、革帶,居左;一人緋裲襠、大口褲,持刀居右。駕士十八人,服如夾槊。次繖一,雉尾扇二。次硃漆團扇四,曲蓋二,執者皆絳綦襖、冒,分左右。次僚佐,本服陪從。次麾、幢各一,左麾右幢。次大角、鼓吹。



 一品鹵簿:有清道四人為二重,幰弩一騎。青衣十人,車輻十人,戟九十,絳引幡六,刀、楯、弓、箭、槊皆八十,節二,大槊二,告止幡、傳教幡皆二,信幡六,誕馬六,儀刀十六,府佐四人夾行。革路一,駕四馬,駕士十六人。繖一,硃漆團扇四,曲蓋二,僚佐本服陪從,麾、幢、大角、鐃吹皆備。



 自二品至四品,青衣、車輻每品減二人。二品,刀、楯、弓、箭、戟、槊各減二十。三品以下,每品減十而已。二品,信幡四,誕馬四,儀刀十四,革路駕士十四人。三品亦如之,儀刀十,革路駕士十二人。四品、五品,信幡二,誕馬二,儀刀八,木路駕士十人。



 自二品至四品,皆有清道二人,硃漆團扇二,曲蓋一,幰弩一騎,幡竿長丈,繖一,節一,夾槊二。



 萬年縣令亦有清道二人,幰弩一騎,青衣、車輻皆二人,戟三十,告止幡、傳教幡、信幡皆二,竿長九尺,誕馬二,軺車,一馬,駕士六人,繖、硃漆團扇、曲蓋皆一。非導駕及餘四等縣初上者,減幰弩、車輻、曲蓋,其戟亦減十。



 內命婦、夫人鹵簿:青衣六人,偏扇、團扇皆十六,執者間彩裙襦、彩裳、革帶,行障三,坐障二,厭翟車,駕二馬,馭人十,內給使十六人夾車,從車六乘,繖、雉尾扇皆一,團扇二,內給使執之,戟六十。外命婦一品亦如之,厭翟車馭人減二,有從人十六人。非公主、王妃則乘白銅飾犢車,駕牛,馭人四,無雉尾扇。



 嬪,青衣四人,偏扇、團扇、方扇十四,行障二,坐障一,翠車,馭人八,內給使十四人,夾車四乘,戟四十。外命婦二品亦如之,乘白銅飾犢車,青通幰,硃裹,從人十四人。



 婕妤、美人、才人,青衣二人,偏扇、團扇、方扇十,行障二,坐障一,安車,駕二馬,馭人八,內給使十人,從車二乘,戟二十。太子良娣、良媛、承徽、外命婦三品亦如之,白銅飾犢車,從人十人。



 外命婦四品,青衣二人,偏扇、團扇、方扇皆八,行障、坐障皆一,白銅飾犢車,馭人四,從人八。餘同三品,唯無戟。



 自夫人以下皆清道二人,繖一,又有團扇二。



 大駕鹵簿鼓吹,分前後二部。鼓吹令二人,府、史二人騎從,分左右。



 前部:扛鼓十二,夾金鉦十二,大鼓、長鳴皆百二十,鐃鼓十二,歌、簫、笳次之;大橫吹百二十,節鼓二,笛、簫、觱篥、茄、桃皮觱篥次之;扛鼓、夾金鉦皆十二,小鼓、中鳴皆百二十,羽葆鼓十二,歌、簫、笳次之。至相風輿,有扛鼓一,金鉦一,鼓左鉦右。至黃麾,有左右金吾衛果毅都尉二人主大角百二十,橫行十重;鼓吹丞二人,典事二人騎從。



 次後部鼓吹:羽葆鼓十二,歌、簫、笳次之;鐃鼓十二,歌、簫、笳次之;小橫吹百二十,笛、蕭、觱篥、笳、桃皮觱篥次之。凡歌、簫、笳工各二十四人,主帥四人,笛、簫、觱篥、笳、桃皮觱篥工各二十四人。



 法駕,減太常卿、司徒、兵部尚書、白鷺車、闢惡車、大輦、五副路、安車、四望車,又減屬車四,清游隊、持鈒沄、玄武隊皆減四之一,鼓吹減三之一。



 小駕,又減卿史大夫、指南車、記里鼓車、鸞旗車、皮軒車、象革木三路、耕根車、羊車、黃鉞車、豹尾車、屬車、小輦、小輿,諸隊及鼓吹減大駕之半。



 凡鼓吹五部:一鼓吹,二羽葆,三鐃吹,四大橫吹,五小橫吹,總七十五曲。



 鼓吹部有扛鼓、大鼓、金鉦小鼓、長鳴、中鳴。扛鼓十曲:一《警雷震》,二《猛獸駭》,三《鷙鳥擊》,四《龍媒蹀》,五《靈夔吼》,六《雕鶚爭》,七《壯士怒》,八《熊羆吼》,九《石墜崖》,十《波蕩壑》。大鼓十五曲,嚴用三曲:一《元驎合邏》,二《元驎他固夜》、三《元驎跋至慮》。警用十二曲:一《元咳大至游》,二《阿列乾》,三《破達析利純》,四《賀羽真》,五《鳴都路跋》,六《他勃鳴路跋》,七《相雷析追》,八《元咳赤賴》,九《赤咳赤賴》,十《吐咳乞物真》,十一《貪大訐》,十二《賀粟胡真》。小鼓九曲:一《漁陽》,二《雞子》,三《警鼓》,四《三鳴》,五《合節》,六《覆參》,七《步鼓》,八《南陽會星》,九《單搖》。皆以為嚴、警,其一上馬用之。長鳴一曲三聲:一《龍吟聲》,二《彪吼聲》,三《河聲》。中鳴一曲三聲:一《蕩聲》,二《牙聲》,三《送聲》。



 羽葆部十八曲:一《太和》,二《休和》,三《七德》,四《騶虞》,五《基王化》,六《纂唐風》,七《厭炎精》,八《肇皇運》,九《躍龍飛》,十《殄馬邑》,十一《興晉陽》,十二《濟渭險》,十三《應聖期》,十四《御宸極》,十五《寧兆庶》,十六《服遐荒》,十七《龍池》,十八《破陣樂》。



 鐃吹部七曲:一《破陣樂》,二《上車》,三《行車》,四《向城》,五《平安》,六《歡樂》,七《太平》。



 大橫吹部有節鼓二十四曲:一《悲風》,二《游弦》,三《間弦明君》,四《吳明君》,五《古明君》,六《長樂聲》,七《五調聲》,作《烏夜啼》,九《望鄉》,十《跨鞍》,十一《間君》,十二《瑟調》,十三《止息》,十四《天女怨》,十五《楚客》,十六《楚妃嘆》,十七《霜鴻引》。十八《楚歌》,十九《胡笳聲》,二十《辭漢》,二十一《對月》,二十二《胡笳明君》,二十三《湘妃怨》,二十四《沈湘》。



 小橫吹部有角、笛、簫、笳、觱篥、桃皮觱篥六種,曲名失傳。



 伶工謂夜警為嚴。凡大駕嚴,夜警十二曲,中警三曲,五更嚴三遍。天子謁郊廟,夜五鼓過半,奏四嚴;車駕至橋,復奏一嚴。元和初,禮儀使高郢建議罷之。



 歷代獻捷必有凱歌,太宗平東都,破宋金剛,執賀魯,克高麗,皆備軍容,凱歌入京都,然其禮儀不傳。太和初,有司奏:「命將征討,有大功,獻俘馘,則神策兵衛於門外,如獻俘儀。凱樂用鐃吹二部,笛、觱篥、簫、笳、鐃鼓,皆工二人,歌工二十四人,乘馬執樂,阿列如鹵簿。鼓吹令、丞前導,分行俘馘之前。將入都門,鼓吹振作,奏《破陣樂》、《應聖期》、《賀朝歡》、《君臣同慶樂》等四曲。至太社、太廟門外,陳而不作。吉獻禮畢,樂作。至御樓前,陳兵仗於旌門外二十步,樂工步行,兵部尚書介胃執鉞,於旌門中路前導,協律郎二人執麾,門外分導,太常卿跪請奏凱樂。樂闋,太常卿跪奏『樂畢』。兵部尚書、太常卿退,樂工立於旌門外,引俘馘入獻,及稱賀,俘囚出,乃退。」



\end{pinyinscope}