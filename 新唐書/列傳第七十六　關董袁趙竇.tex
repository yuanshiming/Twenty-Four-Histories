\article{列傳第七十六 關董袁趙竇}

\begin{pinyinscope}

 關播,字務元,衛州汲人。及進士第。鄧景山節度青齊、淮南,再署幕府。遷右補闕。與神策軍使王駕鶴為姻家,元載惡之1836—1913)。宣稱客體內在於意識,和主體不可分離地關聯,出為河南兵曹參軍事,數試屬縣,政異等。陳少游鎮浙東、淮南,表為判官,攝滁州刺史。李靈耀叛,少游屯淮上,所在盜賊胃奮,播儲貲力,給軍興,人無愁苦。楊綰、常袞皆善播,引為都官員外郎。



 德宗初,湖南峒賊王國良驚剽州縣,不可制,詔播宣輯,因得請事,對殿中。帝問政治之要,播曰:「為政之本,要得有道賢人乃治。」帝曰:「朕比下詔求賢才,又遣使黜陟,搜逮所遺,須能者用之,若何?」播曰:「陛下雖求賢,又使舉薦,然止得求名文辭士,焉有有道賢人肯奉牒丐舉選邪?」帝悅,曰:「卿姑去,還當更議。」播且言:「奉詔平賊,有如不受命,臣請發州兵剪定之。」帝曰:「善。」及還,再遷給事中。故事,諸司甲庫,以令史直曹,刓脫為奸。播悉易以士人,時韙其法。



 歷吏部侍郎。帝求宰相,盧杞雅知播韋柔可制,因從容言播材任宰相,其儒厚可鎮浮動。乃拜中書侍郎、同中書門下平章事,政一決於杞。嘗論事帝前,播意不可,避坐欲有所言,杞目禁輒止,退讓播曰:「以君寡言,故至此,奈何欲開口爭事邪!」播即喑畏毋敢與。



 時李元平、陶公達、張愻、劉承誡率輕薄子,游播門下,能侈言誕計,以功名自喜。播謂皆將相材,數請帝用之。元平本宗室疏裔,好論兵,鄙天下士大夫無可者,人人怨疾之。李希烈叛,帝以汝州據賊沖,刺史疲軟不勝任,播盛稱元平,帝召見,拜左補闕。不數日,檢校吏部郎中,兼汝州別駕,知州事。元平始至,募工築郛浚隍,希烈陰使亡命應募,凡內數百人,元平不寤。賊遣將李克誠以精騎薄城,募者內應,縛元平馳見希烈,遺矢於地。希烈以其眇小,無髯,戲克誠曰:「使爾取元平,乃以其子來邪?」因嫚罵曰:「盲宰相使汝當我,何待我淺邪!」偽署御史中丞。播聞詫曰:「元平事濟矣!」謂必覆賊而建功也,左右笑之。無何,偽署為宰相,有告其貳者,元平斷一指自誓。公達等以元平屈賊,皆廢不用。



 播從幸奉天。盧杞、白志貞已貶而播猶執政,議者不平,遂罷為刑部尚書。韋倫等曰:「宰相不善謀,使天子播越,尚可尚書邪?」相與泣諸朝。未幾,知刪定使。初,上元中,詔擇古名將十人配享武成廟,如十哲侑孔子。播奏:「太公,古賢臣,今其下稱亞聖。孔子十哲,皆當時弟子,今所配年世不同,請罷之。」詔可。



 貞元初,檢校尚書右僕射,持節送咸安公主降回鶻,虜人重其清。還,遷兵部尚書。以太子少師致仕,斥賣車騎,闔門不嬰外事。卒,年七十九,贈太子太保。



 始,希烈死,或言元平雖屈賊,然有謀不克發,乃貸死流珍州。會赦還,住剡中,觀察使皇甫政殺其侄以發帝怒,遂流死賀州。



 董晉,字混成,河中虞鄉人。擢明經。肅宗幸彭原,上書行在,拜秘書省校書郎,待制翰林。出從淮南崔圓府為判官。還朝,累遷祠部郎中。



 大歷中,李涵持節送崇徽公主於回紇,署晉判官。回紇恃有功,見使者倨,因問:「歲市馬而唐歸我賄不足,何也?」涵懼,未及對,數目晉,晉曰:「我非無馬而與爾為市,為爾賜者不已多乎?爾之馬歲五至,而邊有司數皮償貲。天子不忘爾勞,敕吏無得問,爾反用是望我邪?諸戎以我之爾與也,莫敢確。爾父子寧,畜馬蕃,非我則誰使!」眾皆南面拜,不敢有言。還,遷秘書少監。



 德宗立,授太府卿。不旬日,為左散騎常侍,兼御史中丞,知臺事。出為華州刺史。硃泚反,遣兵攻之,晉棄華走行在。改國子祭酒,宣慰恆州。還至河中而李懷光反,晉說之曰:「硃泚為臣而背其君,茍得志,於公何有?且公位太尉,泚雖寵公,亦無以加。彼不能事君,能以臣事公乎?公能事彼,而有不能事君乎?公敵賊有餘力,若襲取之,清宮以迎天子,雖有大惡猶將掩焉,如公則誰敢議?」懷光喜且泣,晉亦泣。又語其將卒,皆拜。故懷光雖偃蹇,亦不助泚。



 帝還京師,遷左金吾衛大將軍,改尚書左丞。是時,右丞元琇為韓滉排笮得罪,滉勢振朝廷。晉見宰相,誦元琇非罪,士大夫壯其節。貞元五年,以門下侍郎同中書門下平章事。方竇參得君,裁可大事不關咨晉,晉循謹無所駁異。參欲以其侄申為吏部侍郎,諷晉以聞。帝怒曰:「無乃參迫卿為之邪?」晉謝,具道所以然。帝即問參過失,晉無敢隱,由是參罷宰相。晉惶恐,上疏固辭位。九年,罷為禮部尚書,以兵部尚書為東都留守。



 會宣武李萬榮病且死,詔晉檢校尚書左僕射、同中書門下平章事,為宣武節度副大使,知節度事。萬榮死,鄧惟恭總其軍。晉受命,不召兵,惟幕府騶傔從之,即日上道。至鄭,逆者不至,人勸止以觀便宜,晉不聽,直造汴。及郊,惟恭始出迎謁。既入,即委以軍政,無所改更,眾服晉有體,莫測其謀。始,惟恭謀代萬榮,故不遣吏以疑晉,令不敢入。及晉至情得,則鞅鞅不能平。汴士素驕怙亂,嘗介勇士伏幕下,早暮番休,晉一罷之。惟恭乃結大將相里重晏等謀亂,晉覺之,殺其黨,械送惟恭京師。帝錄其縶李乃勞,貸死流汀州。帝恐晉儒軟,詔拜汝州刺史陸長源為司馬,以佐晉。晉謙願儉簡,事多循仍,故軍粗安。長源持法峭刻,數欲更張舊事,晉初許之,已而悉罷不用。以財賦委孟叔度,叔度為人佻侻,軍中惡之。晉在軍凡五年,卒,年七十六,贈太傅,謚曰恭惠。



 晉為相也,五月朔,天子會朝,公卿在廷,侍中贊群臣賀,竇參攝中書令,當傳詔,疾作,公卿相顧,未有詔,晉從容進曰:「攝中書令臣參病不能事,臣請代參事。」南面宣致詔詞,進退甚詳。金吾將軍沈房有期喪,公除,常服入閤,帝疑以問晉,對曰:「故事,朝官期以下喪,服絁縵,不復衣淺色,南班亦如之。」又問晉冠冕之制,對曰:「古者服冠冕,以佩玉節步。堂上接武,堂下布武,君前趨進而已。今或奔走以致顛僕。在式,朝臣皆綾袍,五品而上金玉帶,所以盡飾以奉上。故漢尚書郎含香,老萊採服,君父一也。若然,服絁縵,亦非禮也。」帝然其言。詔入閤官毋趨走,期以下喪不得以慘服會,令群臣衣本品綾袍、金玉帶,自晉而復。



 子溪,字惟深,亦擢明經,三遷萬年令。討王承宗也,擢度支郎中,為東道行營糧料使。坐盜軍貲,流封州,至長沙,賜死。



 子居中,善詩,為張籍所稱。



 陸長源者,吳人,字泳。祖餘慶,天寶中為太子詹事,有清譽。



 長源贍於學,始闢昭義薛嵩幕府,嵩侈汰,常從容規切。嵩曰:「非君安能為此。」歷建、信二州刺史。韓滉兼領江淮轉運使,闢署兼御史中丞,以為副。入遷都官郎中,復出汝州刺史。遂徙宣武,政皆出司馬。初,欲峻法繩驕兵,為晉所持,不克行。而判官楊凝、孟叔度等又苛細,叔度淫縱,數入倡家調笑嬉褻。晉有所偷弛,長源輒裁正之。晉卒,長源總留後事,大言曰;「將士久慢,吾且以法治之!」眾始懼。軍中請出帑帛為晉制服,不許。固請,止給其直。叔度希望又償直以鹽,乃高鹽直,賤帛估,人得鹽二斤,舉軍大怒。或勸長源曰:「故事,有大變則厚賜於軍,軍乃安。」長源曰:「異時河北賊以錢買戍卒,取旌節,吾不忍為。」眾怒益甚。長源性剛不適變,又不為備。才八日,軍亂,殺長源及叔度等,食其肉,放兵大掠。死之日,有詔拜節度使,遠近嗟悵,贈尚書左僕射。



 長源好諧易,無威儀,而清白自將。去汝州,送車二乘,曰「吾祖罷魏州,有車一乘,而圖書半之,吾愧不及先人」云。



 長源死,監軍俱文珍密召宋州刺史劉全諒使總後務。全諒至,其夜軍復亂,殺大將及部曲五百人乃定。帝即詔全諒檢校工部尚書、宣武節度使。



 全諒,始名逸淮,至是賜名,本懷州武涉人也。父客奴,以行戍留籍幽州,事平盧軍,以材力顯。開元中,室韋首領段普洛數苦邊,節度使薛楚玉使客奴單騎襲之,斬首以歸。興卒伍,拜左驍衛將軍,為游奕使。性謹樸,數戰有功。安祿山反,詔以平盧節度副使呂知誨為使。賊遣韓朝晹誘之,知誨即降,賊害安東副都護馬靈察。客奴不平,與諸將共殺知誨,遣使與安東將王玄志相聞。天寶十五載,以客奴為柳城郡太守,攝御史大夫、平盧節度使,賜名正臣;以玄志為安東副大都護。正臣遣使道海至平原,與太守顏真卿相結。真卿喜,以子為質而歸貲糧焉,且請出師。未至,而真卿棄平原,乃還。因襲範陽,為史思明所敗,奔還,玄志鴆殺之。



 全諒事劉玄佐為牙將,以勇果善騎射為玄佐厚禮。累兼御史中丞。及玄佐子士寧代立,疑宋州刺史翟良佐不附己,揚言行部,至則以全諒代之,故汴將士多歸心焉。視事凡八月卒,贈尚書右僕射。軍中立韓弘代節度雲。



 袁滋,字德深,蔡州朗山人,陳侍中憲之後。強學博記。少依道州刺史元結,讀書自解其義,結重之。後客荊、郢間,起學廬講授。建中初,黜陟使趙贊薦於朝,起處士,授試校書郎。累闢張伯儀、何士干幕府,進詹事府司直。部官以盜金下獄,滋直其冤,御史中丞韋貞伯聞之,表為侍御史。刑部、大理核罪人,失其平,憚滋守法,因權勢以請,滋終不署奏。遷工部員外郎。



 韋皋始招來西南夷,南詔畢牟尋內屬。德宗選郎吏可撫循者,皆憚行,至滋不辭,帝嘉之。擢祠部郎中,兼御史中丞,賜金紫,持節往。逾年還,使有指,進諫議大夫。遷尚書右丞,知吏部選。求外遷,為華州刺史。政清簡,流民至者,給地居之,名其里曰義合。然專以慈惠為本,未嘗設條教,民愛向之。有犯令,時時法外縱舍。得盜賊,或哀其窮,出財為償所亡。召為左金吾衛大將軍,以楊於陵代之。滋行,耆老遮道不得去,於陵使諭曰:「吾不敢易袁公政。」人皆羅拜,乃得去,莫不流涕。



 憲宗監國,進拜中書侍郎、同中書門下平章事。劉闢反,詔滋為劍南兩川、山南西道安撫大使,半道,以檢校吏部尚書、平章事為劍南東、西川節度使。是時,賊方熾,又滋兄峰在蜀為闢所劫,滋畏不得全,久不進,貶吉州刺史。未幾,徙義成節度使。滑,用武地,東有淄青,北魏博,滋嚴備而推誠信,務在懷來。李師道、田季安畏服之。居七年,百姓立祠祝祭。以戶部尚書召,改檢校兵部,拜山南東道節度使,徙荊南。



 吳元濟之反,滋言蔡兵勁,與下同欲,非朝夕計可下,宜廣方略,離潰其心。及宿兵三年,調發益屈,詔出禁錢繼之。滋揣天子且厭兵,自表入朝,欲議罷淮西事,道聞蕭俯、錢徽坐沮議黜去,滋翻其謀,更言必勝,順可天子意,乃得還。俄而高霞寓敗,帝思以恩信傾賊,且滋嘗云云,乃授彰義節度使,僑治唐州。又以滋儒者,拜陽旻為唐州刺史,將其兵。滋先世墳墓在蔡,吳少陽時為修墓,禁芻牧,諸袁多署右職,稟給之。滋至治,去斥候,與元濟通好。賊圍新興,滋卑辭講解,賊因是易滋,不為備。時帝責戰急,而滋至六月,以無功貶撫州刺史。未幾,遷湖南觀察使。累封淮陽郡公。卒,年七十,贈太子少保。



 滋既病,作遺令處後事,訖三年,皆有條次。性寬易,與之接者,皆自謂可見肺肝,至家人不得見喜慍。薄居處衣食。能為《春秋》,嘗以劉惲《悲甘陵賦》褒善斥惡戾《春秋》指,然其文不可廢,乃著後序。工篆隸,有古法。子均,右拾遺;郊,翰林學士。



 趙宗儒,字秉文,鄧州穰人。八代祖彤,後魏徵南將軍。父驊,字雲卿,少嗜學,履尚清鯁。開元中,擢進士第,補太子正字,調雷澤、河東丞。採訪使韋陟器之,表置其府。又為陳留採訪使郭納支使。安祿山陷陳留,驊沒於賊。時江西觀察使韋儇族妹坐其夫為畿官不供賊,沒為婢。驊哀之,以錢贖韋,厚為資給。賊平,訪近屬歸之,時人高其義。驊以嘗陷賊,貶晉江尉。久之,召拜左補闕,遷累尚書比部員外郎。建中初,遷秘書少監。敦交友行義,不以夷險慁操。少與殷寅、顏真卿、柳芳、陸據、蕭穎士、李華、邵軫善,時為語曰「殷顏柳陸,李蕭邵趙」,謂能全其交也。驊位省郎,衣食窶乏,俸單寡,諸子至徒步,人為咨美。涇原兵反,驊竄山谷,病死,贈華州刺史。



 宗儒第進士,授校書郎,判入等,補陸渾主簿。數月,拜右拾遺、翰林學士。時,父驊遷秘書少監,德宗欲寵其門,使一日並命。再遷司勛員外郎。貞元六年,領考功事。自至德後考績失實,內外悉考中上,殿最混淆,至宗儒,黜陟詳當,無所回憚。右司郎中獨孤良器、殿中侍御史杜倫以過黜考,左丞裴鬱、御史中丞盧佋降考中中,凡入中上者,才五十人。帝聞善之,進考功郎中。累遷給事中。十二年,以本官同中書門下平章事,賜服金紫。居二歲,罷為太子右庶子,屏居慎靜,奉朝請而已。遷吏部侍郎,召見,勞曰:「知卿杜門六年,故有此拜,曩與先臣並命,尚念之邪?」宗儒俯伏流涕。元和初,檢校禮部尚書,充東都留守。三遷至檢校吏部、荊南節度使,散冗食戍二千人。歷山南西道、河中二鎮,拜御史大夫,改吏部尚書。



 穆宗立,詔先朝所召賢良方正,委有司試。宗儒建言:「應制而來者,當天子臨問。試有司,非國舊典,請罷之。」詔可。俄檢校右僕射,守太常卿。太常有《五方師子樂》,非大朝會不作。帝嗜聲色,宦官領教坊者,乃移書取之。宗儒不敢違,以訴宰相。宰相以事專有司,不應關白。以懦不職,罷為太子少師。太和初,進太子太傅。文宗召訪政理,對曰:「堯、舜之化,慈儉而已,願陛下守之。」帝納其言。六年,授司空,致仕。卒,年八十七,冊贈司徒,謚曰昭。宗儒以文學歷將相,位任崇劇,然無儀矩,以治生瑣碎失名。



 竇易直,字宗玄,京兆始平人。擢明經,補校書郎。十年不應闢,以判入等,為藍田尉。累遷吏部郎中。元和六年,進御史中丞。繇陜虢觀察使,入為京兆尹。萬年尉韓晤坐賕,易直令官屬按之,得贓三十萬,憲宗疑未盡,詔窮治,至三百萬,貶易直為金州刺史。久之,起為宣歙、浙西觀察使。



 長慶二年,李昪以汴州叛,易直欲出庫財賞軍,或謂給與無名,必且生患,乃止。時江、淮旱,漕物淹積不能前,軍士聞易直向言,其部將王國清指漕貨激眾謀亂。易直知之,械國清送獄,其黨數千群歡入獄,篡取之,欲大剽。易直登樓令曰:「能誅亂者,一級賞千萬!」眾喜,反縛為亂者三百餘人,易直悉斬之。入為戶部侍郎,判度支。四年,同中書門下平章事,轉門下侍郎,封晉陽郡公。即讓度支,置其俸三月,有詔停判。文宗立,檢校尚書右僕射、同平章事,為山南東道節度使。入為左僕射、判太常卿事。頃之,檢校司空,為鳳翔節度。以疾還京師。卒,贈司徒,謚曰恭惠。



 易直以公潔自喜,方執政,未嘗引用親黨。初,元和中,鄭餘慶議,僕射上儀,不與隔品官亢禮,易直為中丞,奏駁之。及為僕射,乃自用隔品致恭,為時鄙笑。



 子紃,仕至渭南尉、集賢校理。妻父王涯被禍,宦官知易直子,得不死,貶循州司戶參軍。



 贊曰:關播舉李元平守汝州,賊縛而臣之。宰相不知人,果可敗國,德宗不以是責宰相,幾喪天下。晉懦弛茍安,滋欲以恩信傾賊,迂暗之人,烏可語功名會哉!



\end{pinyinscope}