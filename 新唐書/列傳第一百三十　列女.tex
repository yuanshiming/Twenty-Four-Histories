\article{列傳第一百三十 列女}

\begin{pinyinscope}

 李德武妻裴淑英楊慶妻王房玄齡妻盧獨孤師仁姆王蘭英楊三安妻李樊會仁母敬衛孝女無忌鄭義宗妻盧劉寂妻夏侯碎金於敏直妻張楚王靈龜妃上官楊紹宗妻王賈孝女李氏妻王阿足攀彥琛妻魏李母汴女李崔繪妻盧賢貞節婦李符鳳妻玉英高叡妻秦王琳妻韋盧惟清妻徐饒娥竇伯女仲女盧甫妻李鄒待徵妻薄金節婦高愍女楊烈婦賈直言妻董李孝女妙法李湍妻董昌齡丹楊王孝女和子段居貞妻謝楊含妻蕭韋雍妻蕭衡方厚妻程鄭孝女李廷節妻崔殷保晦妻封絢竇烈婦李拯妻盧山陽女趙周迪妻硃延壽妻王



 女子之行,於親也孝,婦也節,母也義而慈,止矣。中古以前,書所載後、妃、夫人事,天下化之。後彤史職廢,婦訓、姆則不及於家,故賢女可紀者千載間寥寥相望。唐興,風化陶淬且數百年,而聞家令姓窈窕淑女,至臨大難,守禮節,白刃不能移,與哲人烈士爭不朽名,寒如霜雪,亦可貴矣。今採獲尤顯行者著之篇,以緒正父父、子子、夫夫、婦婦之懿云。



 李德武妻裴,字淑英,安邑公矩之女,以孝聞鄉黨。德武在隋,坐事徙嶺南,時嫁方逾歲,矩表離婚。德武謂裴曰:「我方貶,無還理,君必儷它族,於此長決矣。」答曰:「夫,天也,可背乎?願死無它。」欲割耳誓,保姆持不許。夫姻媦,歲時塑望裴致禮惟謹。居不御薰澤。讀《列女傳》,見述不更嫁者,謂人曰:「不踐二廷,婦人之常,何異而載之書?」後十年,德武未還,矩決嫁之,斷發不食,矩知不能奪,聽之。德武更娶汆硃氏,遇赦還,中道聞其完節,乃遣後妻,為夫婦如初。



 楊慶妻王者,世充足之女。慶以河間王子為郇王,守滎陽,陷於世充,故世充妻之,用為管州刺史。太宗攻洛陽,慶謀與王歸唐,謝曰:「鄭以我奉箕帚者,綴公之心,今負恩背義,自為身謀,可若何?至長安,則公家婢耳,願送我還東都。」慶不聽,王謂左右曰:「唐勝則鄭滅,鄭安則吾夫死,若是,生何益?」乃飲藥死。慶入朝,官宜州刺史。



 房玄齡妻盧,失其世。玄齡微時,病且死,諉曰:「吾病革,君年少,不可寡居,善事後人。」盧泣人帳中,剔一目示玄齡,明無它。會玄齡良愈,禮之終身。



 王蘭英者,獨狐師仁之姆。師仁父武都謀歸唐,王世充殺之。師仁始三歲,免死禁錮,蘭英請髡鉗得保養,許之。時喪亂,餓死者藉藉,游丐道路以食師仁,身啖土飲水。後詐為採新,竊師仁歸京師。高祖嘉其義,詔封蘭英永壽鄉君。



 楊三安妻李,京兆高陵人。舅姑亡,三安又死,子幼,孤窶,畫田夜紡,凡三年,葬舅姑及夫兄弟凡七喪,遠近嗟涕。太宗聞而異之,賜帛三百段,遣州縣存問,免其徭役。



 樊會仁母敬,蒲州河東人,字象子。笄而生會仁。夫死,事舅姑祥順。家以其少,俗嫁之,潛約婚於里人,至期,陽為母病,使歸視。敬至,知見紹,乃外為不知者,私謂會仁曰:「吾孀處不死者,以母老兒幼,今舅將奪吾志,汝雲何?」會仁泣,敬曰:「兒毋啼!」乃伺隙遁去,家追及半道,以死自守,乃罷。會仁未冠卒,時敬母又終,既葬,謂所親曰:「母死子亡,何生為!」不食數日死,聞者憐之。



 衛孝女,絳州夏人,字無忌。父為鄉人衛長則所殺,無忌甫六歲,無兄弟,母改嫁。逮長,志報父仇。會從父大延客,長則在坐,無忌抵以甓,殺之。詣吏稱父冤已報,請就刑。巡察使褚遂良以聞,太宗免其罪,給驛徙雍州,賜田宅。州縣以禮嫁之。



 鄭義宗妻盧者,範陽士族也。涉書史,事舅姑恭順。夜有盜持兵劫其家,人皆匿竄,惟姑不能去,盧冒刃立姑側,為賊捽捶幾死。賊去,人問何為不懼,答曰:「人所以異鳥獸者,以其有仁義也。今憐裏急難尚相赴,況姑可委棄邪?若百有一危,我不得獨生。」姑曰:「歲寒然後知松柏後凋,吾乃今見婦之心。」



 劉寂妻夏侯,滑州胙城人,字碎金。父長雲為鹽城丞,喪明。時劉已生二女矣,求與劉絕,歸侍父疾。又事後母以孝稱。五年父亡,毀不勝喪,被發徙跣,身負土作塚,廬其左,寒不綿、日一食者三年。詔賜物二十段、粟十石,表異門閭。後其女居母喪,亦如母行,官又賜粟帛,表其門。



 於敏直妻張者,皖城公儉女也。生三歲,每父母病,已能晝夜省侍,顏色如成人。及長,愈恭順仁孝。儉病篤,聞之,號泣幾絕。儉死,一慟遂卒。高宗懿其行,賜物百段,以狀屬史官。



 楚王靈龜妃上官者,下邽士族也。靈龜出繼哀王後,而舅姑在,妃朝夕侍奉,謹甚,凡珍美,非經獻不先嘗。靈龜卒,將葬,前妃無近族,議者欲不舉,妃曰:「逝者有知,魂可無托乎?」乃備禮合葬。聞者嘉嘆。喪除,兄弟共諭:「妃少,又無子,可不有行。」泣曰:「丈夫以義,婦人以節,我未能殉溝壑,尚可御妝澤、祭他胙乎?」將自劓刵,眾遂不敢強。



 楊紹宗妻王,華州華陰人。在褓而母亡,繼母鞠愛。父征遼歿,繼母又卒,王年十五,乃舉二母柩而立父象,招魂以葬,廬墓左。永徽中,詔:「楊氏婦在隋時,父歿遼西,能招魂克葬。至祖父母塋隧,親服板築,哀感行路。」因賜物段並粟,以闕表門。



 賈孝女,濮州鄄城人。年十五,父為族人玄基所殺。孝女弟強仁尚幼,孝女不肯嫁,躲撫育之。強仁能自樹立,教伺玄基殺之,取其心告父墓。強仁詣縣言狀,有司論死。孝女詣闕請代弟死,高宗閔嘆,詔並免之,內徙洛陽。



 李氏妻王阿足,深州鹿城人。早孤,無兄弟。歸李氏數歲,夫死無子,以嫠姊高年無供養,乃不忍嫁。畫耕夜織,能辦生事,餘二十年,姊乃亡,葬送如禮。鄉人服其義,爭遣女妻往師其風訓。壽終於家。



 樊彥琛妻魏者,揚州人。彥琛病,魏曰:「公病且篤,不忍公獨死。」彥琛曰:「死生,常道也。幸養諸孤使成立,相從而死,非吾取也。」彥琛卒,值徐敬業難,陷兵中。聞其知音,令鼓箏,魏曰:「夫亡不死,而逼我管弦,禍由我發。」引刀斬其指。軍伍欲強妻之,固拒不從,乃妨擬頸曰:「從我者不死。」魏厲聲曰:「狗盜,乃欲辱人,速死,吾志也!」乃見害,聞者傷之。



 李畬母者,失其氏。有淵識。畬為監察御史,得稟米,量之三斛而贏,問於史,曰:「御史米,不概也。」又問車庸有幾,曰:「御史不償也。」母怒,敕歸餘米,償其庸,因切責畬。畬劾倉官,自言狀,諸御史聞之,有慚色。



 汴女李者,年八歲父亡,殯於堂十年,朝夕臨。及笄,母欲嫁之。斷發,丐終養。居母喪,哀號過人,自庀葬具,州里送葬千餘人。廬於墓,蓬頭,跣而負土,以完園塋,蒔松數百。武后時,按察使薛季昶表之,詔樹闕門閭。



 崔繪妻盧者,鸞臺侍郎獻之女。獻有美名。繪喪,盧年少,家欲嫁之,盧稱疾不許。女兄適工部侍郎李思沖,早亡。思沖方顯重,表求繼室,詔許,家內外姻皆然可。思沖歸幣三百輿,盧不可,曰:「吾豈再辱於人乎?寧沒身為婢。」是夕,出自竇,糞穢〓面,還崔舍,斷發自誓。思沖以聞,武后不奪也,詔為浮屠庀以終。



 堅貞節婦李者,年十七,嫁為鄭廉妻。未逾年,廉死,常布衣蔬食。夜忽夢男子求為妻,初不許,後數數夢之。李自疑容貌未衰醜所召也,即截發,麻衣,不薰飾,垢面塵膚,自是不復夢。刺史白大威欽其操,號堅貞節婦,表旌門闕,名所居曰節婦里。



 符鳳妻某氏,字玉英,尤姝美。鳳以罪徙儋州,至南海,為獠賊所殺,脅玉英私之,對曰:「一婦人不足事眾男子,請推一長者。」賊然之。乃請更衣,有頃,盛服立於舟,罵曰:「受賊辱,不如死!」自沉於海。



 高叡妻秦。叡為趙州刺史,為默啜所攻。州陷,叡仰藥不死,至默啜所,示以竇帶異袍,曰:「降我,賜爾官;不降,且死。」叡視秦,秦曰:「君受天子恩,當以死報,賊一品官安足榮?」自是皆瞑目不語。默啜知不可屈,乃殺之。



 王琳妻韋者,士族也。琳為眉州司功參軍,俗僭侈盛飾,韋不知有簪珥。訓二子堅、冰有法,後皆名聞。琳卒時,韋年二十五,家欲強嫁之,韋固拒,至不聽音樂,處一室,或終日不食。卒年七十五,著《女訓》行於世。



 盧惟清妻徐,淄州人,世客陳留。惟清仕歷校書郎。徐女兄之夫李宜得以罪斥,惟清坐僚姻,貶播川尉。徐還鄉里,糲食,斥鉛膏,採絺不御。會大赦,徐間關迎惟清,至荊州,聞惟清死,二髯奴將劫徐歸下江,徐知之,數其罪,奴不敢逼,劫其貲去。徐倍道行至播川,足繭流血,得惟清戶,以喪還,閱歲至洛陽。既葬,以無子,終服還陳留。汴州刺史齊瀚高其節,頌而詩之。



 饒娥字瓊真,饒州樂平人。生小家,勤織紝,頗自修整。父勣,漁於江,遇風濤,舟覆,尸不出。娥年十四,哭水上,不食三日死。俄大震電,水蟲多死,父尸浮出,鄉人異之,歸賵具禮,葬父及娥鄱水之陰。縣令魏仲光碣其墓。建中初,黜陟使鄭淑則表旌其閭,河東柳宗元為立碑云。



 竇伯女、仲女,京兆奉天人。永泰中,遇賊行剽,二女自匿山谷,賊跡而得之,將逼以私。行臨大谷,伯曰:「我豈受污於賊!」乃自投下,賊大駭。俄而仲亦躍而墜。京兆尹第五琦表其烈行,詔旌門閭,免其家徭役,官為庀葬。



 盧甫妻李,秦州成紀人。父瀾,永泰初為蘄令。梁、宋兵興,瀾諭降劇賊數千人。刺史曹升襲賊,敗之。賊疑瀾賣己,執瀾及其弟渤,兄弟爭相代死,李見父被殷,亦請代父,遂皆遇害。



 又有王泛妻裴者,亦俘賊中,欲污之,罵曰:「吾,衣冠子,豈愛生受污邪!」賊臨以兵,罵不止,乃支解焉。



 宣慰使李季卿聞狀,詔贈李者昌縣君、裴河東縣君,瀾、渤並贈官。



 鄒待徵妻薄者,從侍征官江陰。袁晁亂,薄為賊所掠,將污之,不從。語家媼使報待徵曰:「我義不辱。」即死於水。賊去,得其尸。義聲動江南,聞人李華作《哀節婦賦》。



 金節婦者,安南賊帥陶齊亮之母也。常以忠義誨齊亮,頑不受,遂絕之。自田而食,紡而衣,州里矜法焉。大歷初,詔賜兩丁侍養,本道使四時存問終身。



 高愍女名妹妹,父彥昭事李正己。及納拒命,質其妻子,使守濮陽。建中二年,挈城歸河南都統劉玄佐,納屠其家。時女七歲,母李憐其幼,請免死為婢,許之。女不肯,曰:「母兄綿不免,何賴而生?」母兄將被刑,遍拜四方。女問故,答曰:「神可祈也。」女曰:「我家以忠義誅,神尚何知而拜之!」問父在所,西向哭,再拜就死。德宗駭嘆,詔太常謚曰愍。諸儒爭為之誄。



 彥昭從玄佐救寧陵,復汴州,累功授潁州刺史。朝廷錄其忠,居州二十年不徙,卒贈陜州都督。



 楊烈婦者,李侃妻也。建中末,李希烈陷汴,謀襲陳州。侃為項城令,希烈分兵數千略定諸縣,侃以城小賊銳,欲逃去,婦曰:「寇至當守,力不足,則死焉。君而逃,尚誰守?」侃曰:「兵少財乏,若何?」婦曰:「縣不守,則地賊地也,倉廩府庫皆其積也,百姓皆其戰士也於國家何有?請重賞募死士,尚可濟。」侃乃召吏民入廷中曰:「令誠若主也,然滿歲則去,非如吏民生此土也,墳墓存焉,宜相與死守,忍失身北面奉賊乎?」眾泣,許諾。乃徇曰:「以瓦石擊賊者,賞千錢;以刀矢殺賊者,萬錢。」得數百人。侃率以乘城,婦身自釁以享眾。報賊曰:「項城父老義不下賊,得吾城不足為威,宜亟去;徒失利,無益也。」賊大笑。侃中流矢,還家,婦責曰:「君不在,人誰肯固?死於外,猶愈於狀也。」侃遽登城。會賊將中矢死,遂引去,縣卒完。詔遷侃太平令。



 先是萬歲通天初,契丹寇平州,鄒保英為刺史,城且陷,妻奚率家僮女丁乘城,不下賊,詔封誠節夫人。默啜攻飛狐,縣令古玄應妻高能固守,虜引去,詔封徇忠縣君。史思明之叛,衛州女子侯、滑州女子唐、青州女子王,相與歃血赴行營討賊,滑濮節度使許叔冀表其忠,皆補果毅。雖敢決不忘於國,然不如楊烈婦慨慷知君臣大義云。



 賈直言妻董。直言坐事,貶嶺南,以妻少,乃訣曰:「生死不可期,吾去,可亟嫁,無須也。」董不答,引繩束發,封以帛,使直言署,曰:「非君手不解。」直言貶二十年乃還,署帛宛然。及湯沐,發墮無餘。



 李孝女者,名妙法,瀛州博野人。安祿山亂,被劫徙它州。聞父亡,欲間道奔喪,一子不忍去,割一乳留以行。既至,父已葬,號踴請開父墓以視,宗族不許。復持刀刺心,乃為開。見棺,舌去塵,發治拭之。結廬墓左,手植松柏,有異鳥至。後,母病,或不食飲,女終日未嘗視匕箸,及亡,刺血書於母臂而葬,廬墓終身。



 李湍妻某氏。湍籍吳元濟軍,元和中,自拔歸鳥重胤,妻為賊縛而臠食之,將死,猶號湍曰:「善事鳥僕射!」觀者嘆泣。重胤請以其事屬史官,詔可。



 董昌齡母楊,世居蔡。昌齡更事吳少陽,至元濟時,為吳房令。母常密戒曰:「逆順成敗,兒可圖之。」昌齡未決,徒郾城,楊復曰:「逆賊欺天,神所不福。當逆降,無以我累。兒為忠臣,吾死不慊。」會王師逼郾城,昌齡乃降。憲宗喜,即拜郾城令兼監察御史,昌齡謝曰:「母之訓也,臣何能!」帝嗟嘆。元濟囚楊,欲殺者屢矣。蔡平而母在,陳許節度李遜表之,封北平郡太君。



 王孝女,徐州人,字和子。元和中,父兄皆防秋屯涇州,葉蕃寇邊,並戰死。和子年十七,單身被發徒跣蓑裳抵涇屯,日丐貸,護二喪還,葬於鄉,植松柏,翦發壞容,廬墓所。節度使王智興白狀,詔旌其門。



 段居貞妻謝,字小娥,洪州豫章人。居貞本歷陽俠少年,重氣決,娶歲餘,與謝父同賈江湖上,並為盜所殺。小娥赴江流,傷腦折足,人救以免。轉側丐食至上元,夢父及夫告所殺主名,離析其文為十二言,持問內外姻,莫能曉。隴西李公佐隱占得其意,曰:「殺若父者必申蘭,若天必申春,試以是求之。」小娥泣謝。諸申,乃名盜亡命者也。小娥詭服為男子,與傭保雜。物色歲餘,得蘭於江州,春於獨樹浦。蘭與春,從兄弟也。小娥托傭蘭家,日以護信自效,蘭〓倚之,雖包苴無不委。小娥見所盜段、謝服用故在,益知所夢不疑。出入二箕,伺其便。它日蘭盡集群偷釃酒,蘭與春醉,臥廬。小娥閉戶,拔佩刀斬蘭首,因大呼捕賊。鄉人墻救,禽春,得贓千萬,其黨數十。小娥悉疏其人上之官,皆抵死,乃始自言狀。刺史張錫嘉其烈,白觀察使,使不為請。還豫章,人爭聘之,不許。祝發事浮屠道,垢衣糲飯終身。



 楊含妻蕭,父歷,為撫州長史,以官卒,母亦亡。蕭年十六,與謂皆韶淑,毀貌,載二喪還鄉里,貧不能給舟庸,次宣州戰鳥山,舟子委柩去。蕭結廬水濱,與婢穿壙納棺成墳,蒔松柏,朝夕臨,有馴鳥、縞兔、菌芝之祥。長老等為立舍,歲時進粟縑。喪滿不釋蓑,人高其行。或請昏,女曰:「我弱不能北還,君誠為我致二柩葬故里,請事君子。」於是,含以高安尉罷歸,聘之,且請如素。蕭以親未葬,許其載,辭其採。已葬,乃釋服而歸楊云。



 韋雍妻蕭。張弘靖鎮幽州也,表雍在幕府。硃克融亂,雍被劫。蕭聞難,與雍皆出,左右格之,不退。雍臨刃,蕭呼曰:「我茍生無益,願今日死君前。」刑者斷其臂,乃殺雍。蕭意象晏然,觀者哀嘆。是夕死。大和中,楊志誠表其烈,詔贈蘭陵縣君。



 雍字和叔,擢進士第。



 衡方厚妻程。大和中,方厚為邕州錄事參軍。招討使董昌齡治無狀,方厚數爭事,昌齡怒,將執付吏,辭以疾,不免,即以死告,臥棺中。昌齡知之,使闔棺甚牢。方厚閉久,以爪攫棺,爪盡乃絕。程懼並死,不敢哭。昌齡恬不疑,厚遣其喪。程徒行至闕下,叩右銀臺門,自刵陳冤,下御史鞫治有實,昌齡乃得罪。文宗詔封程武昌縣君,賜一子九品正官員。



 鄭孝女,兗州瑕丘人。父神佐,為官兵,戰死慶州。時母已亡,又無兄弟,女時年二十四,即翦發毀服,身護喪還鄉里,與母合葬。廬墓下,手樹松柏成林。初,許適牙兵李玄慶,至是,謝不嫁。大中中,兗州節度使蕭俶狀於朝,有詔旌表其閭。



 李廷節妻崔。乾符中,廷節為郟城尉。王仙芝攻汝州,廷節被執。賊見崔妹美,將妻之,詬曰:「我,士人妻,死亡有命,柰何受賊污?」賊怒,刳其心食之。



 殷保晦妻封,敖孫也,名絢,字景文。能文章、草隸。保晦歷校書郎。黃巢入長安,共匿蘭陵里。明日,保晦逃。賊悅封色,欲取之,固拒。賊誘說萬詞,不答。賊怒,勃然曰:「從則生,不然,正膏我劍!」封罵曰:「我,公卿子,守正而死,猶生也,終不辱逆賊手!」遂遇害。保晦歸,左右曰:「夫人死矣!」保晦號而絕。



 竇烈婦者,河南人,朝邑令華某妻。初,同州軍亂,逐節度使李瑭走河中,令匿望仙里,不知所舍乃仇家也。夜半盜入,捽令首,欲殺之,竇泣蔽捍,苦持賊袂,至中刀不解,令得脫走不死,賊亦去。京兆聞之,歸酒帛醫藥,幾死而愈。



 李拯妻盧者,美姿,能屬文。拯字昌時,咸通末擢進士,遷累考功郎中。黃巢亂,避地平陽,僖宗召為翰林學士。帝出寶雞,陷於嗣襄王煴。煴敗,拯死,盧伏尸哭。王行瑜兵逼之,不從,脅以刃,斷一臂死。



 山陽女趙者,父盜鹽,當論死,女詣官訴曰:「迫饑而盜,救死爾,情有可原,能原之邪?否則請俱死。」有司義之,許減父死。女曰:「身今為官所賜,願毀服依浮屠法以報。」節截耳自信,侍父疾,卒不嫁。



 周迪妻某氏。迪善賈,往來廣陵。會畢師鐸亂,人相掠賣以食。迪饑將絕,妻曰:「今欲歸,不兩全。君親在,不可並死,願見賣以濟君行。」迪不忍,妻固與詣肆,售得數千錢以奉。迪至城門,守者誰何,疑其紿,與迪至肆問狀,見妻首已在枅矣。迪里餘體歸葬之。



 硃延壽妻王者,當楊行密時,延壽事行密為壽州刺史,惡行密不臣,與寧國節度使田頵謀絕之以歸唐。事洩,行密以計召延壽,欲與揚州,延壽信之。將行,王曰:「今若得揚州,成宿志,具興衰在時,非系家也,然願日一介為驗。」許之。及為行密所殺,介不至,王曰:「事敗矣。」即部家僕,授兵器。方闔扉而捕騎至,遂出私帑施民,發百燎焚牙居,呼天曰:「我誓不為仇人辱!」赴火死。



\end{pinyinscope}