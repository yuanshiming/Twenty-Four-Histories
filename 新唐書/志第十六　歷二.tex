\article{志第十六 歷二}

\begin{pinyinscope}

 高宗時,《戊寅歷》益疏,淳風作《甲子元歷》以獻。詔太史起麟德二年頒用,謂之《麟德歷》。古歷有章、蔀,有元、紀性的、獨立的精神實體的存在,並由此確立了上帝和物質實,有日分、度分,參差不齊,淳風為總法千三百四十以一之。損益中晷術以考日至,為木渾圖以測黃道,餘因劉焯《皇極歷》法,增損所宜。當時以為密,與太史令瞿壇羅所上《經緯歷》參行。



 弘道元年十二月甲寅朔,壬午晦。八月,詔二年元日用甲申,故進以癸未晦焉。



 永昌元年十一月,改元載初,用周正,以十二月為臘月,建寅月為一月。神功二年,司歷以臘為閏,而前歲之晦,月見東方,太后詔以正月為閏十月。是歲,甲子南至,改元聖歷。命瞿壇羅作《光宅歷》,將用之。三年,罷作《光宅歷》,復行夏時,終開元十六年。



 《麟德歷》麟德元年甲子,距上元積二十六萬九千八百八十算。



 總法千三百四十。



 期實四十八萬九千四。



 常朔實三萬九千五百七十一。加三百六十二曰盈朔實,減三百五十一曰肉朔實。



 辰率三百三十五。



 以期實乘積算,為期總。如總法得一,為日。六十去之,命甲子算外,得冬至。累加日十五、小餘二百九十二、小分六之五,得次氣。六乘小餘,辰率而一,命子半算外,各其加時。



 以常朔實去期總,不滿為閏餘。以閏餘減期總,為總實,如總法得一,為日。以減冬至,得天正常朔。又以常朔小餘並閏餘,以減期總,為總實。因常朔加日二十九、小餘七百一十一,得次朔。因朔加日七、小餘五百一十二太,得上弦。又加,得望及下弦。



 進綱十六。秋分後。



 退紀十七。春分後。



 各以其氣率並後氣率而半之,十二乘之,綱紀除之,為末率。二率相減,餘以十二乘之,綱紀除,為總差。又以十二乘總差,綱紀除之,為別差。以總差前少以減末率,前多以加末率,為初率。累以別差,前少以加初率,前多以減初率,為每日躔差及先後率。乃循積而損益之,各得其日定氣消息與盈朒積。其後無同率,因前末為初率;前少者加總差,前多者以總差減之,為末率。餘依術入之。



 各以氣下消息積,息減、消加常氣,為定氣。各以定氣大小餘減所近朔望大小餘,十二通其日,以辰率約其餘,相從為辰總。其氣前多以乘末率,前少以乘初率,十二而一,為總率。前多者,以辰總減綱紀,以乘十二,綱紀而一,以加總率,辰總乘之,二十四除之;前少者,辰總再乘別差,二百八十八除之:皆加總率。乃以先加、後減其氣盈朒積為定。以定積盈加、朒減常朔弦望,得盈朒大小餘。



 變周四十四萬三千七十七。



 變日二十七,餘七百四十三,變奇一。



 變奇法十二。



 月程法六十七。



 以奇法乘總實,滿變周,去之;不滿者,奇法而一,為變分。盈總法從日,得天正常朔夜卒入變。加常朔小餘,為經辰所入。因朔加七日、餘五百一十二、奇九,得上弦。轉加,得望、下弦及次朔。加之滿變日及餘,去之。又以所入盈朒定積,盈加、朒減之,得朔、弦、望盈朒經辰所入。



 以離程與次相減,得進退差;後多為進,後少為退,等為平。各列朔、弦、望盈朒經辰所入日增減率,並後率而半之,為通率。又二率相減,為率差。增者以入變歷日餘減總法,餘乘率差,總法而一,並率差而半之;減者半入餘乘率差,亦總法而一:皆加通率。以乘入餘,總法除,為經辰變率。半之,以速減、遲加入餘,為轉餘。增者以減總法,減者因餘:皆乘率差,總法而一;以加通率,變法乘之,總法除之,以速減、遲加變率,為定率。乃以定率增減遲速積為定。其後無同率,亦因前率。應增者,以通率為初數,半率差而減之,應損者,即為通率。其歷率損益入餘進退日者,分為二日,隨餘初末,如法求之,所得並以加減變率為定。



 七日:初,千一百九十一;末,百四十九。十四日:初,千四十二;末,二百九十八。二十一日:初,八百九十二;末,四百四十八。二十八日:初,七百四十三;末,五百九十七。各視入餘初數,已下為初,已上以初數減之,餘為末。



 各以入變遲速定數,速減、遲加朔、弦、望盈朒小餘;滿若不足,進退其日。加其常日者為盈,減其常日者為朒。各為定大小餘,命日如前。乃前朔、後朔迭相推校,盈朒之課,據實為準;損不侵朒,益不過盈。



 定朔日名與次朔同者大,不同者小,無中氣者為閏月。其元日有交、加時應見者,消息前後一兩月,以定大小,令虧在晦、二,弦、望亦隨消息。月朔盈朒之極,不過頻三。其或過者,觀定小餘近夜半者量之。



 黃道:南斗,二十四度三百二十八分。牛,七度。婺女,十一度。虛,十度。危,十六度。營室,十八度。東壁,十度。奎,十七度。婁,十三度。胃,十五度。昴,十一度。畢,十六度。觜觿,二度。參,九度。東井,三十度。輿鬼,四度。柳,十四度。七星,七度。張,十七度。翼,十九度。軫,十八度。角,十三度。亢,十度。氐,十六度。房,五度。心,五度。尾,十八度。箕,十度。



 冬至之初日,躔定在南斗十二度。每加十五度二百九十二分、小分五,依宿度去之,各得定氣加時日度。



 各以初日躔差乘定氣小餘,總法而一,進加、退減小餘,為分;以減加時度,為氣初夜半度。乃日加一度,以躔差進加、退減之,得次日。以定朔弦望小餘副之;以乘躔差,總法而一,進加、退減其副,各加夜半日躔,為加時宿度。



 合朔度,即月離也。上弦,加度九十一度、分四百一十七。望,加度百八十二度、分八百三十四。下弦,加度二百七十三度、分千二百五十一。訖,半其分,降一等,以同程法,得加時月離。因天正常朔夜半所入變日及餘,定朔有進退日者,亦進退一日,為定朔夜半所入。累加一日,得次日。



 各以夜半入變餘乘進退差,總法而一,進加、退減離程,為定程。以定朔弦望小餘乘之,總法而一,以減加時月離,為夜半月離。求次日,程法約定程,累加之。若以定程乘夜刻,二百除,為晨分。以減定程,為昏分。其夜半月離,朔後加昏為昏度,望後加晨為晨度。其注歷,五乘弦望小餘,程法而一,為刻。不滿晨前刻者,退命算上。



 辰刻八,分二十四。



 刻分法七十二。



 置其氣屈伸率,各以發斂差損益之,為每日屈伸率。差滿十,從分;分滿十,為率。各累計其率為刻分。百八十乘之,十一乘綱紀除之,為刻差。各半之,以伸減、屈加晨前刻分,為每日晨前定刻。倍之,為夜刻。以減一百,為晝刻。以三十四約刻差,為分;分滿十,為度。以伸減、屈加氣初黃道去極,得每日。以晝刻乘期實,二百乘,總法除,為昏中度。以減三百六十五度三百二十八分,餘為旦中度。各以加日躔,得昏旦中星,赤道計之。其赤道同《太初》星距。



 游交終率千九十三萬九千三百一十三。奇率三百。



 約終三萬六千四百六十四,奇百十三。交中萬八千二百三十二,奇五十六半。交終日二十七,餘二百八十四,奇百一十三。交中日十三,餘八百一十二,奇五十六半。



 虧朔三千一百六,奇百八十七。實望萬九千七百八十五,奇百五十。



 後準千五百五十三,奇九十三半。前準萬六千六百七十八,奇二百六十三。



 置總實,以奇率乘之,滿終率去之;不滿,以奇率約,為入交分。加天正常朔小餘,得朔泛交分。求次朔,以虧朔加之。因朔求望,以實望加之。各以朔望入氣盈朒定積,盈加、朒減之;又六十乘遲速定數,七百七十七除,為限數;以速減、遲加,為定交分。其朔,月在日道里者,以所入限數減遲速定數,餘以速減、遲加其定交分。而出日道表者,為變交分。不出表者,依定交分。其變交分三時半內者,依術消息,以定蝕不。交中已下者,為月在外道;已上者,去之,餘為月在內道。其分如後準已下,為交後分;前準已上者,反減交中,餘為交前分。望則月蝕,朔在內道則日蝕。百一十二約前後分,為去交時。置定朔小餘,副之。辰率約之,以艮、巽、坤、乾為次,命算外。其餘,半法已下為初;已上者,去之,為末。初則因餘,末則減法,各為差率。月在內道者,益去交時十而三除之。以乘差率,十四而一,為差。其朔,在二分前後一氣內,即以差為定;近冬至以去寒露、雨水,近夏至以去清明、白露氣數倍之,又三除去交時增之;近冬至艮巽以加、坤乾以減,近夏至艮巽以減、坤乾以加其差,為定差。艮、巽加副,坤、乾減副。月在外道者,三除去交時數,以乘差率,十四而一,為差。艮、坤以減副,巽、乾以加副,為食定小餘。望即因定望小餘,即所在辰;近朝夕者,以日出沒刻校前後十二刻半內候之。



 月在外道,朔不應蝕。夏至初日,以二百四十八為初準。去交前後分如初準已下、加時在午正前後七刻內者,蝕。朔去夏至前後,每一日損初準二分,皆畢於九十四日,為每日變準。交分如變準已下、加時如前者,亦蝕。又以末準六十減初準及變準,餘以十八約之,為刻準,以並午正前後七刻內數,為時準。加時準內交分,如末準已下,亦蝕。又置末準,每一刻加十八,為差準。加時刻去午前後如刻準已下、交分如差準已下者,亦蝕。自秋分至春分,去交如末準已下、加時巳、午、未者,亦蝕。



 月在內道,朔應蝕。若在夏至初日,以千三百七十三為初準。去交如初準已上、加時在午正前後十八刻內者,或不蝕。夏至前後每日益初準一分半,皆畢於九十四日,為每日變準。以初準減變準,餘十而一,為刻準。以減午正前後十八刻,餘為時準。其去交在變準已上、加時在準內,或不蝕。



 望去交前後定分:冬,減二百二十四;夏,減五十四;春,交後減百,交前減二百;秋,交後減二百,交前減百。不足減者,蝕既。有餘者,以減後準,百四而一,得月蝕分。



 朔交,月在內道,入冬至畢定雨水,及秋分畢大雪,皆以五百五十八為蝕差。入春分,日損六分,畢芒種。以蝕差減去交分;不足減者,反減蝕差,為不蝕分。其不蝕分,自小滿畢小暑,加時在午正前後七刻外者,畢減一時;三刻內者,加一時。大寒畢立春交前五時外、大暑畢立冬交後五時外者,皆減一時;五時內者,加一時。諸加時蝕差應減者,交後減之,交前加之;應加者,交後加之,交前減之。不足減者,皆既;加減入不蝕限者,或不蝕。月在外道,冬至初日,無蝕差。自後日益六分,畢於雨水。入春分,畢白露,皆以五百二十二為差。入秋分,日損六分,畢大雪。以差加去交分,為蝕分。以減後準,餘為不蝕分。十五約蝕差,以減百四,為定法。其不蝕分,如定法得一,以減十五,餘得日蝕分。



 ○歲星



 總率五十三萬四千四百八十三,奇四十五。



 伏分二萬四千三十一,奇七十二半。



 終日三百九十八,餘千一百六十三,奇四十五。



 平見,入冬至,畢小寒,均減六日。入大寒,日損六十七分。入春分,依平。乃日加八十九分,入立夏,畢小滿,均加六日。入芒種,日損八十九分。入夏至,畢立秋,均加四日。入處暑,日損百七十八分。入白露,依平。自後日減五十二分。入小雪,畢大雪,均減六日。



 初順,百一十四日行十八度五百九分,日益遲一分。前留,二十六日。旋退,四十二日,退六度十二分,日益疾二分。又退,四十二日,退六度十二分,日益遲二分。後留,二十五日。後順,百一十四日行十八度五百九分,日益疾一分。日盡而夕伏。



 ○熒惑



 總率百四萬五千八十,奇六十。



 伏分九萬七千九十,奇三十。



 終日七百七十九,餘千二百二十,奇六十。



 平見,入冬至,減二十七日。自後日損六百三分。入大寒,日加四百二分。入雨水,畢穀雨,均加二十七日。入立夏,日損百九十八分。入立秋,依平。入處暑,日減百九十八分。入小雪,畢大雪,均減二十七日。



 初順,入冬至,率二百四十三日行百六十五度。乃三日損日度各二。小寒初日,率二百三十三日行百五十五度,乃二日損一。入穀雨四日,平,畢小滿九日。率百七十八日行百度,乃三日損一。夏至初日,平,畢六日,率百七十一日行九十三度,乃三日益一。入立秋初日,百八十四日行百六度,乃每日益一。入白露初日,率二百一十四日行百三十六度;乃五日益六,入秋分初日,率二百三十二日行百五十四度,又每日益一。入寒露初日,率二百四十七日行百六十九度。乃五日益三。入霜降五日,平,畢立冬十三日,率二百五十九日行百八十一度,乃二日損日一。入冬至,復初。



 各依所入常氣,平者依率,餘皆計日損益,為前疾日度定率。其前遲及留退,入氣有損益日度者,計日損益,皆準此法。疾行日率,入大寒,六日損一;入春分,畢立夏,均減十日;入小滿,三日損所減一;畢芒種,依平;入立秋,三日益一;入白露,畢秋分,均加十日;入寒露,一日半損所加一;畢氣盡,依平,為變日率。疾行度率,入大寒畢啟蟄,立夏畢夏至,大暑畢氣盡,霜降畢小雪,皆加四度;清明畢穀雨,加二度,為變度率。



 初行入處暑,減日率六十,度率三十;入白露,畢秋分,減日率四十四。度率二十二:皆為初遲半度之行。盡此日、度,乃求所減之餘日、度率,續之,為疾。初行入大寒畢大暑,差行,日益遲一分。其前遲、後遲,日率既有增損,而益遲、益疾,差分皆檢括前疾末日行分,為前遲初日行分。以前遲平行分減之,餘為前遲總差。後疾初日行分,為後遲末日行分,以後遲初日行分減之,餘為後遲總差。相減,為前後別日差分,其不滿者皆調為小分。遲疾之際,行分衰殺不倫者,依此。



 前遲,入冬至,率六十日行二十五度;先疾,日益遲二分。入小寒,三日損一。大寒初日,率五十五日行二十度,乃三日益一。立春初日,平,畢清明,率六十日行二十五度。入穀雨,每氣別減一度。立夏初日,平,畢小滿,率六十日行二十二度。入芒種,每氣別益一度。夏至初日,平,畢處暑,率六十日行二十五度,入白露,三日損一。秋分初日,率六十日行二十五度。乃每日益日一,三日益度二。寒露初日,率七十五日行三十度,乃每日損日一,三日損度一。霜降初日,率六十日行二十五度,乃二日損一度。入立冬一日,平,畢氣盡,率六十日行十七度。入小雪,五日益一度。大雪初日,率六十日行二十度,乃三日益一度。入冬至,復初。



 前留,十三日。前疾減日率一者,以其數分益此留及後遲日率。前疾加日率者,以其數分減此留及後遲日率。旋退,西行。入冬至初日,率六十三日退二十二度,乃四日益度一。小寒一日,率六十三日退二十六度,乃三日半損度一。立春三日,平,畢驚蟄,率六十三日退十七度,乃二日益日、度各一。雨水八日,平,畢氣盡,率六十七日退二十一度。入春分,每氣損日、度各一。大暑初日,平,畢氣盡,率五十八日退十二度。立秋初日,平,畢氣盡,率五十七日退十一度,乃二日益日一。寒露九日,平,畢氣盡,率六十六日退二十度,乃二日損一。霜降六日,平,畢氣盡,率六十三日退十七度,乃三日益一。立冬十一日,平,畢氣盡,率六十七日退二十一度,乃二日損一。入冬至,復初。



 後留,冬至初,留十三日,乃二日半益一。大寒初日,平,畢氣盡,留二十五日,乃二日半損一。雨水初日,留十三日,乃三日益一。清明初日,留二十三日,乃日損一。清明十日,平,畢處暑,留十三日,乃二日損一。秋分十一日,無留,乃每日益一。霜降初日,留十九日,乃三日損一。立冬畢大雪,留十三日。



 後遲,順,六十日行二十五度,日益疾二分。前疾加度者,此遲依數減之,為定度。前疾無加度者,此遲入秋分至立冬減三度,入冬至減五度。後留定日肉十三日者,以所朒日數加此遲日率。



 後疾,冬至初日,率二百一十日行百三十二度,乃每日損一。大寒八日,率百七十二日行九十四度,乃二日損一。啟蟄,平,畢氣盡,率百六十一日行八十三度,乃二日益一。芒種十四日,平,畢夏至,率二百三十三日行百五十五度,乃每日益一。大暑初日,平,畢處暑,率二百六十三日行百八十五度,乃二日損一。秋分一日,率二百五十五日行百七十七度,乃一日半損一。大雪初日,率二百五日行百二十七度,乃三日益一。入冬至,復初。



 其入常氣日度之率有損益者,計日損益,為後疾定日率度。疾行日率,其前遲定日朒六十、及退行定日朒六十三者,皆以所朒日數加疾行定日率;前遲定日盈六十、退行定日盈六十三、後留定日盈十三者,皆以所盈日數減此疾定日率;各為變日率。疾行度率,其前遲定度朒二十五、退行定度盈十七、後遲入秋分到冬至減度者,皆以所盈朒度數加此疾定率;前遲定度盈二十五、及退行定度朒十七者,皆以所盈朒度數減此疾定度率:各為變度率。



 初行入春分畢穀雨,差行,日益疾一分。初行入立夏畢夏至,日行十度,六十六日行三十三度。小暑畢大暑,五十日行二十五度。立秋畢氣盡,二十日行十度。減率續行,並同前,盡日度而夕伏。



 ○鎮星



 總率五十萬六千六百二十三,奇二十九。



 伏分二萬二千八百三十一,奇六十四半。



 終日三百七十八,餘一百三,奇二十九。



 平見,入冬至,初減四日。乃日益八十九分。入大寒,畢春分,均減八日。入清明,日損五十九分。入小暑初,依平。自後日加八十九分。入白露初,加八日。自後日損百七十八分。入秋分,均加四日。入寒露,日損五十九分。入小雪初日,依平,乃日減八十九分。



 初順,八十三日行七度二百九十分,日益遲半分,前留,三十七日。旋退,五十一日退二度四百九十一分,日益疾少半。又退,五十一日退二度四百九十一分,日益遲少半。後留,三十七日。後順,八十三日,行七度二百九十分,日益疾半分。日盡而夕伏。



 ○太白



 總率七十八萬四千四百四十九,奇九。



 伏分五萬六千二百二十四,奇五十四半。



 終日五百八十三,餘千二百二十九,奇九。



 夕見伏日二百五十六。



 晨見伏日三百二十七,餘千二百二十九,奇九。



 夕平見,入冬至,初依平,乃日減百分。入啟蟄,畢春分,均減九日。入清明,日損百分,入芒種,依平。入夏至,日加百分。入處暑,畢秋分,均加九日。入寒露,日損百分。入大雪,依平。



 夕順,入冬至畢立夏,入立秋畢大雪,率百七十二日行二百六度。入小滿後,十日益一度,為定度。入白露,畢春分,差行,益遲二分,自餘平行。夏至畢小暑,率百七十二日行二百九度。入大暑,五日損一度,畢氣盡。平行,入冬至,大暑畢氣盡,率十三日行十三度。入冬至,十日損一,畢立春。入立秋,十日益一,畢秋分。啟蟄畢芒種,七日行七度。入夏至後,五日益一,畢於小暑。寒露初日,率二十三日行二十二度,乃六日損一,畢小雪。順遲,四十二日,行三十度,日益遲八分。前疾加過二百六度者,準數損此度。夕留,七日。夕退,十日退五度。日盡而夕伏。



 晨平見,入冬至,依平。入小寒,日加六十七分。入立春,畢立夏,均加三日。入小滿,日損六十七分。入夏至,依平。入小暑,日減六十七分。入立秋,畢立冬,均減三日。入小雪,日損六十七分。



 晨退,十日退五度。晨留,七日。順遲,冬至畢立夏,大雪畢氣盡,率四十二日行三十度,日益疾八分。入小滿,率十日損一度,畢芒種。夏至畢寒露,率四十二日行二十七度。入霜降,每氣益一度,畢小雪。平行,冬至畢氣盡,立夏畢氣盡,十三日行十三度。入小寒後,六日益日、度各一,畢啟蟄。小滿後,七日損日、度各一,畢立秋。雨水初日,率二十三日行二十三度。自後六日損日、度各一,畢穀雨。處暑畢寒露,無平行。入霜降後,五日益日、度各一,畢大雪。疾行,百七十二日,行二百六度。前遲行損度不滿三十度者,此疾依數益之。處暑畢寒露,差行,日益疾一分。自餘平行。日盡而晨伏。



 ○辰星



 總率十五萬五千二百七十八,奇六十六。



 伏分二萬二千六百九十九,奇三十三。



 終日百一十五,餘千一百七十八,奇六十六。



 夕見伏日五十二。



 晨見伏日六十三,餘千一百七十八,奇六十六。



 夕平見,入冬至,畢清明,依平。入穀雨,畢芒種,均減二日。入夏至,畢大暑,依平。入立秋,畢霜降,應見不見。其在立秋、霜降氣內,夕去日十八度外、三十六度內有木、火、土、金星者,亦見。入立冬,畢大雪,依平。



 順疾,十二日行二十一度六分,日行一度五百三分。大暑畢處暑,十二日行十七度二分,日行一度二百八十分。平行,七日行七度。入大暑後,二日損日、度各一。入立秋,無此平行。順遲,六日行二度四分,日行二百二十四分。前疾行十七度者,無此遲行。夕留,五日。日盡而夕伏。



 晨平見,入冬至,均減四日。入小寒,畢大寒,依平。入立春,畢啟蟄,均減三日。其在啟蟄氣內,去日度如前,晨無木、火、土、金星者,不見。入雨水,畢立夏,應見不見。其在立夏氣內,去日度如前,晨有木、火、土、金星者,亦見。入小滿,畢寒露,依平。入霜降,畢立冬,均加一日。入小雪,畢大雪,依平。



 晨見,留,五日。順遲,六日行二度四分,日行二百二十四分。入大寒,畢驚蟄,無此遲行。平行,七日行七度。入大寒後,二日損日、度各一。入立春,無此平行。順疾,行十二日行二十一度六分,日行一度五百三分。前無遲行者,十二日行十七度一十分,日行一度二百八十分。日盡而晨伏。



 各以伏分減總實,以總率去之;不盡,反以減總率,如總法,為日。天正定朔與常朔有進退者,亦進減、退加一日。乃隨次月大小去之,命日算外,得平見所在。各半見餘以同半總。太白、辰星以夕見伏日加之,得晨平見。各依所入常氣加減日及應計日損益者,以損益所加減;訖,餘以加減平見,為常見。又以常見日消息定數之半,息減、消加常見,為定見日及分。



 置定見夜半日躔,半其分,以其日躔差乘定見餘,總法而一,進加、退減之,乃以其星初見去日度,歲星十四,太白十一,熒惑、鎮星、辰星十七,晨減、夕加,得初見定辰所在宿度。其初見消息定數,亦半之,以息加、消減其星初見行留日率。其歲星、鎮星不須加減。其加減不滿日者,與見通之,過半從日,乃依行星日度率,求初日行分。



 置定見餘,以減半總,各以初日行分乘之,半總而一,順加、逆減星初見定辰所在度分,得星見後夜半宿度。以所行度分,順加、逆減之。其差行益疾益遲者,副置初日行分,各以其差遲損、疾加之,留者因前,逆則依減,以程法約行分為度分,得每日所至。



 求行分者,皆以半總乘定度率,有分者從之。日率除,為平行度分。置定日率,減一,以所差分乘之,二而一,為差率。以疾減、遲加平行,為初日所行度及分。



 中宗反正,太史丞南宮說以《麟德歷》上元,五星有入氣加減,非合璧連珠之正,以神龍元年歲次乙巳,故治《乙巳元歷》。推而上之,積四十一萬四千三百六十算,得十一月甲子朔夜半冬至,七曜起牽牛之初。其術有黃道而無赤道,推五星先步定合,加伏日以求定見。他與淳風術同。所異者,惟平合加減差。既成,而睿宗即位,罷之。



\end{pinyinscope}