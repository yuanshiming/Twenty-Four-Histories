\article{列傳第二十二 魏徵}

\begin{pinyinscope}

 魏徵,字玄成,魏州曲城人。少孤,落魄,棄貲產不營矯虛誕之弊。文辭精富,為世名論。」東晉孫盛《老聃非下賢,有大志,通貫書術。隋亂,詭為道士。武陽郡丞元寶藏舉兵應李密,以徵典書檄。密得寶藏書,輒稱善,既聞徵所為,促召之。徵進十策說密,不能用。王世充攻洛口,徵見長史鄭頲曰:「魏公雖驟勝,而驍將銳士死傷略盡;又府無見財,戰勝不賞。此二者不可以戰。若浚池峭壘,曠日持久,賊糧盡且去,我追擊之,取勝之道也。」頲曰:「老儒常語耳!」徵不謝去。



 後從密來京師,久之未知名。自請安輯山東,乃擢秘書丞,馳驛至黎陽。時李勣尚為密守,徵與書曰:「始魏公起叛徒,振臂大呼,眾數十萬,威之所被半天下,然而一敗不振,卒歸唐者,固知天命有所歸也。今君處必爭之地,不早自圖,則大事去矣!」勣得書,遂定計歸,而大發粟饋淮安王之軍。



 會竇建德陷黎陽,獲徵,偽拜起居舍人。建德敗,與裴矩走入關,隱太子引為洗馬。徵見秦王功高,陰勸太子早為計。太子敗,王責謂曰:「爾鬩吾兄弟,奈何?」答曰:「太子蚤從徵言,不死今日之禍。」王器其直,無恨意。



 即位,拜諫議大夫,封鉅鹿縣男。當是時,河北州縣素事隱、巢者不自安,往往曹伏思亂。徵白太宗曰:「不示至公,禍不可解。」帝曰:「爾行安喻河北。」道遇太子千牛李志安、齊王護軍李思行傳送京師,徵與其副謀曰:「屬有詔,宮府舊人普原之。今復執送志安等,誰不自疑者?吾屬雖往,人不信。」即貸而後聞。使還,帝悅,日益親,或引至臥內,訪天下事。徵亦自以不世遇,乃展盡底蘊無所隱,凡二百餘奏,無不剴切當帝心者。由是拜尚書右丞,兼諫議大夫。



 左右有毀徵阿黨親戚者,帝使溫彥博按訊,非是。彥博曰:「徵為人臣,不能著形跡,遠嫌疑,而被飛謗,是宜責也。」帝謂彥博行讓徵。徵見帝,謝曰:「臣聞君臣同心,是謂一體,豈有置至公,事形跡?若上下共由茲路,邦之興喪未可知也。」帝矍然,曰:「吾悟之矣!」徵頓首曰:「願陛下俾臣為良臣,毋俾臣為忠臣。」帝曰:「忠、良異乎?」曰:「良臣,稷、契、咎陶也;忠臣,龍逢、比干也。良臣,身荷美名,君都顯號,子孫傅承,流祚無疆;忠臣,己嬰禍誅,君陷昏惡,喪國夷家,只取空名。此其異也。」帝曰:「善。」因問:「為君者何道而明,何失而暗?」徵曰:「君所以明,兼聽也;所以暗,偏信也。堯、舜氏闢四門,明四目,達四聰。雖有共,鮌,不能塞也,靖言庸違,不能惑也。秦二世隱藏其身,以信趙高,天下潰叛而不得聞;梁武帝信硃異,侯景向關而不得聞;隋煬帝信虞世基,賊遍天下而不得聞。故曰,君能兼聽,則奸人不得壅蔽,而下情通矣。」



 鄭仁基息女美而才,皇后建請為充華,典冊具。或言許聘矣。徵諫曰:「陛下處臺榭,則欲民有楝宇;食膏粱,則欲民有飽適;顧嬪御,則欲民有室家。今鄭已約昏,陛下取之,豈為人父母意!」帝痛自咎,即詔停冊。



 貞觀三年,以秘書監參豫朝政。高昌王曲文泰將入朝,西域諸國欲因文泰悉遣使者奉獻。帝詔文泰使人厭怛紇干迎之。徵曰:「異時文泰入朝,所過供擬不能具,今又加諸國焉,則瀕塞州縣以乏致罪者眾。彼以商賈來,則邊人為之利;若賓客之,中國蕭然耗矣。漢建武時,西域請置都護、送侍子,光武不許,不以蠻夷敝中國也。」帝曰:「善。」追止其詔。



 於是帝即位四年,歲斷死二十九,幾至刑措,米斗三錢。先是,帝嘗嘆曰:「今大亂之後,其難治乎?」徵曰:「大亂之易治,譬饑人之易食也。」帝曰:「古不云善人為邦百年,然後勝殘去殺邪?」答曰:「此不為聖哲論也。聖哲之治,其應如響,期月而可,蓋不其難。」封德彞曰:「不然。三代之後,澆詭日滋。秦任法律,漢雜霸道,皆欲治不能,非能治不欲。徵書生,好虛論,徒亂國家,不可聽。」徵曰:「五帝、三王不易民以教,行帝道而帝,行王道而王,顧所行何如爾。黃帝逐蚩尤,七十戰而勝其亂,因致無為。九黎害德,顓頊征之,已克而治。桀為亂,湯放之;紂無道,武王伐之。湯、武身及太平。若人漸澆詭,不復返樸,今當為鬼為魅,尚安得而化哉!」德彞不能對,然心以為不可。帝納之不疑。至是,天下大治。蠻夷君長襲衣冠,帶刀宿衛。東薄海,南逾嶺,戶闔不閉,行旅不齎糧,取給於道。帝謂群臣曰:「此徵勸我行仁義,既效矣。惜不令封德彞見之!」



 俄檢校侍中,進爵郡公。帝幸九成宮,宮御舍湋川宮下。僕射李靖、侍中王珪繼至,吏改館宮御以舍靖、珪。帝聞,怒曰:「威福由是等邪!何輕我宮人?」詔並按之。徵曰:「靖、珪皆陛下腹心大臣,宮人止後宮掃除隸耳。方大臣出,官吏諮朝廷法式;歸來,陛下問人間疾苦。夫官舍,固靖等見官吏之所,吏不可不謁也。至宮人則不然,供饋之餘無所參承。以此按吏,且駭天下耳目。」帝悟,寢不問。



 後宴丹霄樓,酒中謂長孫無忌曰:「魏徵、王珪事隱太子、巢刺王時,誠可惡,我能棄怨用才,無羞古人。然徵每諫我不從,我發言輒不即應,何哉?」徵曰:「臣以事有不可,故諫,若不從輒應,恐遂行之。」帝曰:「弟即應,須別陳論,顧不得?」徵曰:「昔舜戒群臣:『爾無面從,退有後言。』若面從可,方別陳論,此乃後言,非稷、蒐所以事堯、舜也。」帝大笑曰:「人言徵舉動疏慢,我但見其嫵媚耳!」徵再拜曰:「陛下導臣使言,所以敢然;若不受,臣敢數批逆鱗哉!」



 十年,為侍中。尚書省滯訟不決者,詔徵平治。徵不素習法,但存大體,處事以情,人人悅服。進左光祿大夫、鄭國公。多病,辭職,帝曰:「公獨不見金在鑛何足貴邪?善冶鍛而為器,人乃寶之。朕方自比於金,以卿為良匠而加礪焉。卿雖疾,未及衰,庸得便爾?」徵懇請,數卻愈牢。乃拜特進,知門下省事,詔朝章國典,參議得失,祿賜、國官、防閤並同職事。



 文德皇后既葬,帝即苑中作層觀,以望昭陵,引徵同升,徵孰視曰:「臣毛昏,不能見。」帝指示之,徵曰:「此昭陵邪?」帝曰:「然。」徵曰:「臣以為陛下望獻陵,若昭陵,臣固見之。」帝泣,為毀觀。尋以定五禮,當封一子縣男,徵請封孤兄子叔慈。帝愴然曰:「此可以勵俗。」即許之。



 後幸洛陽,次昭仁宮,多所譴責。徵曰:「隋惟責不獻食,或供奉不精,為此無限,而至於亡。故天命陛下代之,正當兢懼戒約,奈何令人悔為不奢。若以為足,今不啻足矣;以為不足,萬此寧有足邪?」帝驚曰:「非公不聞此言。」退又上疏曰:



 《書》稱「明德慎罰」,「惟刑之恤」。《禮》曰:「為上易事,為下易知,則刑不煩。」「上多疑,則百姓惑;下難知,則君長勞。」夫上易事,下易知,君長不勞,百姓不惑,故君有一德,臣無二心。夫刑賞之本,在乎勸善而懲惡。帝王所與,天下畫一,不以親疏貴賤而輕重者也。今之刑賞,或由喜怒,或出好惡。喜則矜刑於法中,怒則求罪於律外;好則鉆皮出羽,惡則洗垢索瘢。蓋刑濫則小人道長,賞謬則君子道消。小人之惡不懲,君子之善不勸,而望治安刑措,非所聞也。且暇豫而言,皆敦尚孔、老;至於威怒,則專法申、韓。故道德之旨未弘,而鍥薄之風先搖。昔州犁上下其手而楚法以敝,張湯輕重其心而漢刑以謬,況人主而自高下乎!頃者罰人,或以供張不贍,或不能從欲,皆非致治之急也。夫貴不與驕期而驕自至,富不與奢期而奢自至,非徒語也。



 且我之所代,實在有隋。以隋府藏況今之資儲,以隋甲兵況今之士馬,以隋戶口況今之百姓,挈長度大,曾何等級焉!然隋以富強而喪,動之也;我以貧寡而安,靜之也。靜之則安,動之則亂,人皆知之,非隱而難見、微而難察也。不蹈平易之塗,而遵覆車之轍,何哉?安不思危,治不念亂,存不慮亡也。方隋未亂,自謂必無亂;未亡,自謂必不亡。所以甲兵亟動,徭役不息,以至戮辱而不悟滅亡之所由也,豈不哀哉!夫監形之美惡,必就止水;監『政之安危,必取亡國。《詩》曰:「殷鑒不遠,在夏后之世。臣願當今之動靜,以隋為鑒,則存亡治亂可得而知。思所以危則安矣,思所以亂則治矣,思所以亡則存矣。存亡之所在,在節嗜欲,省游畋,息靡麗,罷不急,慎偏聽,近忠厚,遠便佞而已。夫守之則易,得之實難。今既得其所難,豈不能保其所易?保之不固,驕奢淫泆有以動之也。



 帝宴群臣積翠池,酣樂賦詩。徵賦《西漢》,其卒章曰:「終藉叔孫禮,方知皇帝尊。」帝曰:「徵言未嘗不約我以禮。」它日,從容問曰:「比政治若何?」徵見久承平,帝意有所忽,因對曰:「陛下貞觀之初,導人使諫。三年以後,見諫者悅而從之。比一二年,勉強受諫,而終不平也。」帝驚曰:「公何物驗之?」對曰:「陛下初即位,論元律師死,孫伏伽諫以為法不當死,陛下賜以蘭陵公主園,直百萬。或曰:『賞太厚。』答曰:『朕即位,未有諫者,所以賞之。』此導人使諫也。後柳雄妄訴隋資,有司得,劾其偽,將論死,戴胄奏罪當徒,執之四五然後赦。謂胄曰『弟守法如此,不畏濫罰。」此悅而從諫也。近皇甫德參上書言『修洛陽宮,勞人也;收地租,厚斂也;俗尚高髻,宮中所化也。』陛下恚曰:『是子使國家不役一人,不收一租,宮人無發,乃稱其意。』臣奏:『人臣上書,不激切不能起人主意,激切即近訕謗。』於時,陛下雖從臣言,賞帛罷之,意終不平。此難於受諫也。」帝悟曰:「非公,無能道此者。人苦不自覺耳!」



 先是,帝作飛仙宮,徵上疏曰:



 隋有天下三十餘年,風行萬里,威詹殊俗,一旦舉而棄之。彼煬帝者,豈惡治安、喜滅亡哉?恃其富強,不虞後患也。驅天下,役萬物,以自奉養,子女玉帛是求,宮宇臺榭是飾,徭役無時,干戈不休,外示威重,內行險忌,讒邪者進,忠正者退,上下相蒙,人不堪命,以致殞匹夫之手,為天下笑。聖哲乘機,拯其危溺。今宮觀臺榭,盡居之矣;奇珍異物,盡收之矣;姬姜淑媛,盡侍於側矣;四海九州,盡為臣妾矣。若能鑒彼所以亡,念我所以得,焚寶衣,毀廣殿,安處卑宮,德之上也。若成功不廢,即仍其舊,除其不急,德之次也。不惟王業之艱難,謂天命可恃,因基增舊,甘心侈靡,使人不見德而勞役是聞,斯為下矣。以暴易暴,與亂同道。夫作事不法,後無以觀。人怨神怒,則災害生;災害生,則禍亂作;禍亂作,而能以身名令終鮮矣。



 是歲,大雨,穀、洛溢,毀宮寺十九,漂居人六百家。徵陳事曰:



 臣聞為國基於德禮,保於誠信。誠信立,則下無二情;德禮形,則遠者來格。故德禮誠信,國之大綱,不可斯須廢也。傳曰:「君使臣以禮,臣事君以忠。」「自古皆有死,民無信不立。」又曰:「同言而信,信在言前;同令而行,誠在令外。」然則言而不行,言不信也;令而不從,令無誠也。不信之言,不誠之令,君子弗為也。



 自王道休明,綿十餘載,倉廩愈積,土地益廣,然而道德不日博,仁義不日厚,何哉?由待下之情,未盡誠信,雖有善始之勤,而無克終之美。故便佞之徒得肆其巧,謂同心為朋黨,告訐為至公,強直為擅權,忠讜為誹謗。謂之朋黨,雖忠信可疑;謂之至公,雖矯偽無咎。強直者畏擅權而不得盡,忠讜者慮誹謗而不敢與之爭。熒惑視聽,鬱於大道,妨化損德,無斯甚者。



 今將致治則委之君子,得失或訪諸小人,是譽毀常在小人,而督責常加君子也。夫中智之人,豈無小惠,然慮不及遠,雖使竭力盡誠,猶未免傾敗,況內懷奸利,承顏順旨乎?故孔子曰:「君子而不仁者有矣,未有小人而仁者。」然則君子不能無小惡,惡不積無害於正;小人時有小善,善不積不足以忠。今謂之善人矣,復慮其不信,何異立直木而疑其景之曲乎?故上不信則無以使下,下不信則無以事上。信之為義大矣!



 昔齊桓公問管仲曰:「吾欲使酒腐於爵,肉腐於俎,得無害霸乎?」管仲曰:「此固非其善者,然無害霸也。」公曰:「何如而害霸?」曰:「不能知人,害霸也;知而不能用,害霸也;用而不能任,害霸也;任而不能信,害霸也;既信而又使小人參之,害霸也。」晉中行穆伯攻鼓,經年而不能下,饋閑倫曰:「鼓之嗇夫,閑倫知之,請無疲士大夫,而鼓可得。」穆伯不應。左右曰:「不折一戟,不傷一卒,而鼓可得,君奚不為?」穆伯曰:「閑倫之為人也,佞而不仁。若使閑倫下之,吾不可以不賞,若賞之,是賞佞人也。佞人得志,是使晉國舍仁而為佞,雖得鼓,安用之!」夫穆伯,列國大夫,管仲,霸者之佐,猶能慎於信任,遠避佞人,況陛下之上聖乎?若欲令君子小人是非不雜,必懷之以德,待之以信,厲之以義,節之以禮,然後善善而惡惡,審罰而明賞,無為之化何遠之有!善善而不能進,惡惡而不能去,罰不及有罪,賞不加有功,則危亡之期或未可保。



 帝手詔嘉答。於是,廢明德宮玄圃院賜遭水者。



 它日,宴群臣,帝曰:「貞觀以前,從我定天下,間關草昧,玄齡功也。貞觀之後,納忠諫,正朕違,為國家長利,徵而已。雖古名臣,亦何以加!」親解佩刀,以賜二人。帝嘗問群臣:「徵與諸葛亮孰賢?」岑文本曰:「亮才兼將相,非徵可比。」帝曰:「徵蹈履仁義,以弼朕躬,欲致之堯、舜,雖亮無以抗。時上封者眾,或不切事,帝厭之,欲加譙黜,徵曰:「古者立謗木,欲聞己過。封事,其謗木之遺乎!陛下思聞得失,當恣其所陳。言而是乎,為朝廷之益;非乎,無損於政。」帝悅,皆勞遣之。



 十三年,阿史那結社率作亂,雲陽石然,自冬至五月不雨,徵上疏極言曰:



 臣奉侍帷幄十餘年,陛下許臣以仁義之道,守而不失;儉約樸素,終始弗渝。德音在耳,不敢忘也。頃年以來,浸不克終。謹用條陳,裨萬分一。



 陛下在貞觀初,清凈寡欲,化被荒外。今萬里遣使,市索駿馬,並訪怪珍。昔漢文帝卻千里馬,晉武帝焚雉頭裘。陛下居常論議,遠希堯、舜,今所為,更欲處漢文、晉武下乎?此不克終一漸也。子貢問治人。孔子曰:「懍乎若朽索之馭六馬。」子貢曰:「何畏哉?」對曰:「不以道導之,則吾仇也,若何不畏!」陛下在貞觀初,護民之勞,煦之如子,不輕營為。頃既奢肆,思用人力,乃曰:「百姓無事則易驕,勞役則易使。」自古未有百姓逸樂而致傾敗者,何有逆畏其驕而為勞役哉?此不克終二漸也。陛下在貞觀初,役己以利物,比來縱欲以勞人。雖憂人之言不絕於口,而樂身之事實切諸心。無慮營構,輒曰:「弗為此,不便我身。」推之人情,誰敢復爭?此不克終三漸也。在貞觀初,親君子,斥小人。比來輕褻小人,禮重君子。重君子也,恭而遠之;輕小人也,狎而近之。近之莫見其非,遠之莫見其是。莫見其是,則不待間而疏;莫見其非,則有時而暱。暱小人,疏君子,而欲致治,非所聞也。此不克終四漸也。在貞觀初,不貴異物,不作無益。而今難得之貨雜然並進,玩好之作無時而息。上奢靡而望下樸素,力役廣而冀農業興,不可得已。此不克終五漸也。貞觀之初,求士如渴,賢者所舉,即信而任之,取其所長,常恐不及。比來由心好惡,以眾賢舉而用,以一人毀而棄,雖積年任而信,或一朝疑而斥。夫行有素履,事有成跡,一人之毀未必可信,積年之行不應頓虧。陛下不察其原,以為臧否,使讒佞得行,守道疏間。此不克終六漸也。在貞觀初,高居深拱,無田獵畢弋之好。數年之後,志不克固,鷹犬之貢,遠及四夷,晨出夕返,馳騁為樂,變起不測,其及救乎?此不克終七漸也。在貞觀初,遇下有禮,群情上達。今外官奏事,顏色不接,間因所短,詰其細過,雖有忠款,而不得申。此不克終八漸也。在貞觀初,孜孜治道,常若不足。比恃功業之大,負聖智之明,長慠縱欲,無事興兵,問罪遠裔。親狎者阿旨不肯諫,疏遠者畏威不敢言。積而不已,所損非細。此不克終九漸也。貞觀初,頻年霜旱,畿內戶口並就關外,攜老扶幼,來往數年,卒無一戶亡去。此由陛下矜育撫寧,故死不攜貳也。比者疲於徭役,關中之人,勞弊尤甚。雜匠當下,顧而不遣。正兵番上,復別驅任。市物襁屬於廛,遞子背望於道。脫有一穀不收,百姓之心,恐不能如前日之帖泰。此不克終十漸也。



 夫禍福無門,惟人之召,人無釁焉,妖不妄作。今旱之災,遠被郡國,兇醜之孽,起於轂下,此上天示戒,乃陛下恐懼憂勤之日也。千載休期,時難再得,明主可為而不為,臣所以鬱結長嘆者也!



 疏奏,帝曰:「朕今聞過矣,願改之,以終善道。有違此言,當何施顏面與公相見哉!方以所上疏,列為屏障,庶朝夕見之,兼錄付史官,使萬世知君臣之義。」因賜黃金十斤,馬二匹。



 高昌平,帝宴兩儀殿,嘆曰:「高昌若不失德,豈至於亡!然朕亦當自戒,不以小人之言而議君子,庶幾獲安也。」徵曰:「昔齊桓公與管仲、鮑叔牙、寧戚四人者飲,桓公請叔牙曰:『盍起為寡人壽?』叔牙奉觴而起曰:『願公無忘在莒時,使管仲無忘束縛於魯時,使甯戚無忘飯牛車下時。』桓公避席而謝曰:『寡人與二大夫能無忘夫子之言,則社稷不危矣。』」帝曰:「朕不敢忘布衣時,公不得忘叔牙之為人也。」



 帝遣使者至西域立葉護可汗,未還,又遣使齎金帛諸國市馬。徵曰:「今立可汗未定,即詣諸國市馬,彼必以為意在馬,不在立可汗。可汗得立,必不懷恩。諸蕃聞之,以中國薄義重利,未必得馬而先失義矣。魏文帝欲求市西域大珠,蘇則以為惠及四海,則不求自至;求而得之,不足貴也。陛下可不畏蘇則言乎!」帝遂止。



 是後右僕射缺,欲用徵,徵讓,得不拜。皇太子承乾與魏王泰交惡,帝曰:「當今忠謇貴重無逾徵,我遣傅皇太子,一天下之望,羽翼固矣。」即拜太子太師。徵以疾辭,詔答曰:「漢太子以四皓為助,我賴公,其義也。公雖臥,可擁全之。」



 十七年,疾甚。徵家初無正寢,帝命輟小殿材為營構,五日畢,並賜素褥布被,以從其尚。令中郎將宿其第,動靜輒以聞,藥膳賜遺無算,中使者綴道。帝親問疾,屏左右,語終日乃還。後復與太子臻至徵第,徵加朝服,拖帶。帝悲懣,拊之流涕,問所欲。對曰:「嫠不恤緯,而憂宗周之亡!」帝將以衡山公主降其子叔玉。時主亦從,帝曰:「公強視新婦!」徵不能謝。是夕,帝夢徵若平生,及旦,薨。帝臨哭,為之慟,罷朝五日。太子舉哀西華堂。詔內外百官朝集使皆赴喪,贈司空、相州都督,謚曰文貞,給羽葆、鼓吹、班劍四十人,陪葬昭陵。將葬,其妻裴辭曰:「徵素儉約,今假一品禮,儀物褒大,非徵志。」見許,乃用素車,白布幨帷,無塗車、芻靈。帝登苑西樓,望哭盡哀。晉王奉詔致祭。帝作文於碑,遂書之。又賜家封戶九百。



 帝後臨朝嘆曰:「以銅為鑒,可正衣寇;以古為鑒,可知興替;以人為鑒,可明得失。朕嘗保此三鑒,內防己過。今魏徵逝,一鑒亡矣。朕比使人至其家,得書一紙,始半稿,其可識者曰:『天下之事,有善有惡,任善人則國安,用惡人則國弊。公卿之內,情有愛憎,憎者惟見其惡,愛者止見其善。愛憎之間,所宜詳慎。若愛而知其惡,憎而知其善,去邪勿疑,任賢勿猜,可以興矣。』其大略如此。朕顧思之,恐不免斯過。公卿侍臣可書之於笏,知而必諫也。」



 徵狀貌不逾中人,有志膽,每犯顏進諫,雖逢帝甚怒,神色不徙,而天子亦為霽威。議者謂賁、育不能過。嘗上塚還,奏曰:「向聞陛下有關南之行,既辦而止,何也?」帝曰:「畏卿,遂停耳。」始,喪亂後,典章湮散,徵奏引諸儒校集秘書,國家圖籍粲然完整。嘗以《小戴禮》綜匯不倫,更作《類禮》二十篇,數年而成。帝美其書,錄寘內府。帝本以兵定天下,雖已治,不忘經略四夷也。故徵侍宴,奏《破陣武德舞》,則俯首不顧,至《慶善樂》,則諦玩無斁,舉有所諷切如此。



 徵亡,帝思不已,登凌煙閣觀畫像,賦詩悼痛,聞者媢之,毀短百為。徵嘗薦杜正倫、侯君集才任宰相,及正倫以罪黜,君集坐逆誅,纖人遂指為阿黨;又言徵嘗錄前後諫爭語示史官褚遂良。帝滋不悅,乃停叔玉昏,而僕所為碑,顧其家衰矣。



 遼東之役,高麗、靺鞨犯陣,李勣等力戰破之。軍還,悵然曰:「魏徵若在,吾有此行邪!」即召其家到行在,賜勞妻子,以少牢祠其墓,復立碑,恩禮加焉。



 四子:叔玉、叔琬、叔璘、叔瑜。叔玉襲爵為光祿少卿。神龍初,以其子膺紹封。叔璘,禮部侍郎,武后時,為酷吏所殺。叔瑜,豫州刺史,善草隸,以筆意傅其子華及甥薛稷。世稱善書者「前有虞、褚,後有薛、魏」。華為檢校太子左庶子、武陽縣男。開元中,寢堂火,子孫哭三日,詔百官赴吊。徵五世孫謨。



 謨,字申之,擢進士第,同州刺史楊汝士闢為長春宮巡官。文宗讀《貞觀政要》,思徵賢,詔訪其後,汝士薦為右拾遺。謨姿宇魁秀,帝異之。



 邕管經略使董昌齡誣殺參軍衡方厚,貶漵州司戶,俄徙峽州刺史。謨諫曰:「王者赦有罪,唯故無赦。比昌齡專殺不辜,事跡暴章,家人銜冤,萬里投訴,獄窮罪得,特被矜貸,中外以為屈法。今又授刺史,復使治人,紊憲章,乖至治,不見其可。」有詔改洪州別駕。



 御史中丞李孝本,宗室子,坐李訓事誅死,其二女沒入宮。謨上言:「陛下即位,不悅聲色,於今十年,未始採擇。數月以來,稍意聲伎,教坊閱選,百十未已,莊宅收市,沄沄有聞。今又取孝本女內之後宮,宗姓不育,寵幸為累,傷治道之本,速塵穢之嫌。諺曰:『止寒莫若重裘,止謗莫若自修。』惟陛下崇千載之盛德,去一旦之玩好。」帝即出孝本女,詔曰:「乃祖在貞觀時,指事直言,無所避,每覽國史,朕與嘉之。謨為拾遺,屢有獻納。夫備灑埽於內,非曰聲妓,恤宗女之幼,不為漁取,然疑似之間,不可戶曉,謨辭深切,其惜我之失,不亦至乎?謨雖居位日淺,朕何愛一官,增直臣之氣,其以謨為右補闕。」



 先是,帝謂宰相曰:「太宗得徵,參裨闕失,朕今得謨,又能極諫,朕不敢仰希貞觀,庶幾處無過之地。」教坊有工善為新聲者,詔授揚州司馬,議者頗言司馬品高,郎官、刺史迭處,不可以授賤工,帝意右之。宰相諭諫官勿復言,謨獨固諫不可,工降潤州司馬。荊南監軍呂令琛縱傔卒辱江陵令,觀察使韋長避不發,移內樞密使言狀。謨劾長任察廉,知監軍侵屈官司,不以上聞,私白近臣,亂法度,請明其罰。不報。



 俄為起居舍人,帝問:「卿家書詔頗有存者乎?」謨對:「惟故笏在。」詔令上送。鄭覃曰:「在人不在笏。」帝曰:「覃不識朕意,此笏乃今甘棠。」帝因敕謨曰:「事有不當,毋嫌論奏。」謨對:「臣頃為諫臣,故得有所陳;今則記言動,不敢侵官。」帝曰:「兩省屬皆可議朝廷事,而毋辭也!」帝索起居注,謨奏:「古置左、右史,書得失,以存鑒戒。陛下所為善,無畏不書;不善,天下之人亦有以記之。」帝曰:「不然。我既嘗觀之。」謨曰:「向者取觀,史氏為失職。陛下一見,則後來所書必有諱屈,善惡不實,不可以為史,且後代何信哉?」乃止。



 中尉仇士良捕妖民賀蘭進興及黨與治軍中,反狀且,帝自臨問,詔命斬囚以徇。御史中丞高元裕建言:「獄當與眾共之。刑部、大理,法官也,決大獄不與知,律令謂何?請歸有司。」未報。謨上言:「事系軍,即推軍中。如齊民,宜付府縣。今獄不在有司,法有輕重,何從而知?」帝停決,詔神策軍以官兵留仗內,餘付御史臺。臺憚士良,不敢異,卒皆誅死。擢諫議大夫,兼起居舍人、弘文館直學士,謨固讓不見可,乃拜。



 始謨之進,李玨、楊嗣復實推引之。武宗立,謨坐二人黨,出為汾州刺史。俄貶信州長史。宣宗嗣位,移郢、商二州刺史。召授給事中,遷御史中丞,發駙馬都尉杜中立奸贓,權戚縮氣。俄兼戶部侍郎事,謨奏:「中丞,紀綱所寄,不宜雜領錢穀,乞專治戶部。」詔可。頃之,進同中書門下平章事。建言:「今天下粗治,惟東宮未立,不早以正人傅導之,非所以存副貳之重。」且泣下,帝為感動。自敬宗後,惡言儲嫡事,故公卿無敢開陳者。時帝春秋高,嫡嗣未辨,謨輔政,白發其端,朝議歸重。



 會詹毘國獻象,謨以為非土性,不可畜,請還其獻。詔可。河東節度使李業殺降虜,邊部震擾,業內恃憑藉,人無敢言者,謨奏徙滑州。遷中書侍郎。大理卿馬曙有犀鎧數十首,懼而瘞之。奴王慶以怨告曙藏甲有異謀,按之無它狀,投曙嶺外,慶免。議者謂奴訴主,法不聽。謨引律固爭,卒論慶死。累遷門下侍郎,兼戶部尚書。



 大中十年,以平章事領劍南西川節度使。上疾求代,召拜吏部尚書,因久疾,檢校尚書右僕射、太子少保。卒,年六十六,贈司徒。



 謨為宰相,議事天子前,它相或委抑規諷,惟謨讜切無所回畏。宣宗嘗曰:「謨名臣孫,有祖風,朕心憚之。」然卒以剛正為令狐綯所忌,讒罷之。



 贊曰:君臣之際,顧不難哉!以徵之忠,而太宗之睿,身歿未幾,猜譖遽行。始,徵之諫,累數十餘萬言,至君子小人,未嘗不反復為帝言之,以佞邪之亂忠也。久猶不免。故曰:「皓皓者易污,嶢嶢者難全」,自古所嘆云。唐柳芳稱「徵死,知不知莫不恨惜,以為三代遺直」。諒哉!謨之論議挺挺,有祖風烈,《詩》所謂「是以似之」者歟!



\end{pinyinscope}