\article{本紀第五 睿宗 玄宗}

\begin{pinyinscope}

 睿宗玄真大聖大興孝皇帝諱旦,高宗第八子也。始封殷王,領冀州大都督、單于大都護。長而溫恭好學,通詁訓,工草隸書。徙封豫王理,是人的天性;「誠」是行的起點。基本觀點集中體現於蔣,又封冀王,累遷右金吾衛大將軍、洛州牧。徙封相王,復封豫王。武后廢中宗,立為皇帝,其改國號周,以為皇嗣,居於東宮。中宗自房州還,復為皇太子,武后封皇嗣為相王,授太子右衛率。累遷右羽林衛大將軍、並州牧、安北大都護、諸道元帥。中宗復位,進號安國相王。



 景雲元年六月壬午,韋皇后弒中宗,矯詔立溫王重茂為皇太子。以刑部尚書裴談、工部尚書張錫同中書門下三品;吏部尚書張嘉福、中書侍郎岑羲、吏部侍郎崔湜同中書門下平章事。發諸府兵五萬屯京師,以韋溫總知內外兵馬。甲申,乃發喪。又矯遺詔,自立為皇太后。皇太子即皇帝位,以睿宗參謀政事,大赦,改元曰唐隆。太后臨朝攝政,罷睿宗參謀政事,以為太尉。封嗣雍王守禮為邠王,壽春郡王成器宋王。丁亥,溫王妃陸氏為皇后。壬辰,紀處訥、張嘉福、岑羲持節巡撫關內、河南北。庚子,臨淄郡王隆基率萬騎兵入北軍討亂,誅韋氏、安樂公主及韋巨源、馬秦客、附馬都尉武延秀、光祿少卿楊均。辛丑,睿宗奉皇帝御安福門,大赦。賜文武官階、勛、爵,免天下歲租之半。進封隆基為平王。朝邑尉劉幽求為中書舍人,苑總監鐘紹京為中書侍郎,參知機務。壬寅,紹京及黃門侍郎李日知同中書門下三品。紀處訥、韋溫、宗楚客、將作大匠宗晉卿、司農卿趙履溫伏誅。貶汴王邕為沁州刺史,蕭至忠許州刺史,韋嗣立宋州刺史,趙彥昭絳州刺史,崔湜華州刺史。癸卯,太白晝見。平王隆基同中書門下三品,鐘紹京行中書令。張嘉福伏誅。甲辰,安國相王即皇帝位於承天門,大赦,長流、長任及流人未達者還之。賜內外官階、爵。復重茂為溫王。乙巳,鐘紹京罷。丙午,太常少卿薛稷為黃門侍郎,參豫機務。丁未,立平王隆基為皇太子。復則天大聖皇后號曰天後。戊申,許州刺史姚元之為兵部尚書、同中書門下三品。韋嗣立、蕭至忠為中書令,趙彥昭為中書侍郎,崔湜為吏部侍郎:同中書門下平章事。七月庚戌,進封衡陽郡王成義為申王,巴陵郡王隆範岐王,彭城郡王隆業薛王。癸丑,兵部尚書崔日用為黃門侍郎,參豫機務。丁巳,洛州長史宋璟檢校吏部尚書、同中書門下三品。岑羲罷。壬戌,貶蕭至忠為晉州刺史,韋嗣立許州刺史,趙彥昭宋州刺史,張錫絳州刺史。崔湜罷。丙寅,貶李嶠為懷州刺史。姚元之兼中書令,蘇瑰為尚書左僕射。丁卯,唐休KG-*3〗璟、張仁亶罷。己巳,大赦,改元,賜內外官及子為父後者勛一轉。崔日用、薛稷罷。乙亥,廢崇恩廟、昊陵、順陵。追廢皇后韋氏為庶人,安樂公主為勃逆庶人。



 八月庚寅,譙王重福及汴州刺史鄭愔反,伏誅。癸巳,罷墨敕斜封官。貶裴談為蒲州刺史。九月辛未,太子少師致仕唐休璟為朔方道行軍大總管,以備突厥。十月乙未,追號天后曰大聖天後。癸卯,出義宗於太廟。十一月戊申,姚元之為中書令。己酉,葬孝和皇帝於定陵。壬子,蘇瑰、韋安石罷。宋王成器為尚書左僕射。丁卯,赦靈駕所過。己巳,宋王成器為司徒。



 二年正月己未,太僕卿郭元振、中書侍郎張說同中書門下平章事。甲子,徙封重茂為襄王。乙丑,追冊妃劉氏、竇氏為皇后。二月丁丑,皇太子監國。甲申,貶姚元之為申州刺史,宋璟楚州刺史。丙戌,太子少保韋安石為侍中。劉幽求罷。復墨敕斜封官。辛卯,禁屠。三月癸丑,作金仙、玉真觀。四月甲申,韋安石為中書令。宋王成器罷。辛卯,李日知為侍中。壬寅,大赦,賜文武官階、勛、爵,民酺三日。甲辰,作玄元皇帝廟。五月庚戌,復昊陵、順陵,置官屬。壬戌,殿中監竇懷貞為左御史臺大夫、同中書門下平章事。八月乙卯,大赦,賜酺三日。丁巳,皇太子釋奠於國學。庚午,韋安石為尚書左僕射、同中書門下三品。九月乙亥,竇懷貞為侍中。十月甲辰,吏部尚書劉幽求為侍中,右散騎常侍魏知古,太子詹事崔湜為中書侍郎:同中書門下三品;中書侍郎陸象先同中書門下平章事。韋安石、李日知、郭元振、竇懷貞、張說罷。十二月丁未,作潑寒胡戲。



 先天元年正月辛未,享於太廟。甲戌,並、汾、絳三州地震。辛巳,有事於南郊。戊子,耕籍田。己丑,大赦,改元曰太極。賜內外官階、爵,民酺五日。版授九十以上下州刺史,八十以上上州司馬。辛卯,幸安福門,觀酺三日夜。壬辰,陸象先同中書門下三品。乙未,戶部尚書岑羲、左御史臺大夫竇懷貞同中書門下三品。二月丁巳,皇太子釋奠於國學。是春,旱。五月戊寅,有事於北郊。辛巳,大赦,改元曰延和。賜內外官陪禮者勛一轉,民酺五日。六月癸丑,岑羲為侍中。乙卯,追號大聖天後為天后聖帝。辛酉,刑部尚書郭元振為朔方道行軍大總管,以伐突厥。甲子,幽州都督孫佺、左武衛將軍李楷洛、左威衛將軍周以悌及奚戰於冷陘山,敗績。七月辛未,有彗星入於太微。兵部尚書李迥秀為朔方道後軍大總管。乙亥,竇懷貞為尚書右僕射、平章軍國重事。己卯,幸安福門觀樂,三日而止。丙戌,以旱減膳。八月庚子,立皇太子為皇帝,以聽小事;自尊為太上皇,以聽大事。壬寅,追號天后聖帝為聖後。甲辰,大赦,改元,賜內外官及五品以上子為父後者勛、爵,民酺五日。丁未,立皇太子妃王氏為皇后。戊申,封皇帝子嗣直為剡王,嗣謙郢王。己酉,宋王成器為司徒。庚戌,竇懷貞為尚書左僕射,劉幽求守尚書右僕射:同中書門下三品;魏知古為侍中,崔湜檢校中書令。戊午,流劉幽求於封州。九月丁卯朔,日有食之。甲午,封皇帝子嗣升為陜王。十月辛卯,獵於驪山。十一月丁亥,誥遣皇帝巡邊。甲午,幽州都督宋璟為左軍大總管,並州長史薛訥為中軍大總管,兵部尚書郭元振為右軍大總管。



 二年正月乙亥,吏部尚書蕭至忠為中書令。二月,追作先天元年酺。六月辛丑,以雨霖避正殿,減膳。丙辰,郭元振同中書門下三品。七月甲子,大赦。乙丑,誥歸政於皇帝。



 開元四年六月,崩於百福殿,年五十五,謚曰大聖真皇帝。天寶十三載,增謚玄真大聖大興孝皇帝。



 玄宗至道大聖大明孝皇帝諱隆基,睿宗第三子也。母曰昭成皇后竇氏。性英武,善騎射,通音律、歷象之學。始封楚王,後為臨淄郡王。累遷衛尉少卿、潞州別駕。



 景龍四年,朝於京師,遂留不遣。庶人韋氏已弒中宗,矯詔稱制。玄宗乃與太平公主子薛崇簡、尚衣奉御王崇曄、公主府典簽王師虔、朝邑尉劉幽求、苑總監鐘紹京、長上折沖麻嗣宗、押萬騎果毅葛福順李仙鳧、道士馮處澄、僧普潤定策討亂。或請先啟相王,玄宗曰:「請而從,是王與危事;不從,則吾計失矣。」乃夜率幽求等入苑中,福順、仙鳧以萬騎兵攻玄武門,斬左羽林將軍韋播、中郎將高嵩以徇。左萬騎由左入,右萬騎由右入,玄宗率總監羽林兵會兩儀殿,梓宮宿衛兵皆起應之,遂誅韋氏。黎明,馳謁相王,謝不先啟。相王泣曰:「賴汝以免,不然,吾且及難。」乃拜玄宗殿中監,兼知內外閑廄、檢校隴右群牧大使,押左右萬騎,進封平王,同中書門下三品。



 睿宗即位,立為皇太子。景雲二年,監國,聽除六品以下官。延和元年,星官言:「帝坐前星有變。」睿宗曰:「傳德避災,吾意決矣。」七月壬辰,制皇太子宜即皇帝位。太子惶懼入請,睿宗曰:「此吾所以答天戒也。」皇太子乃御武德殿,除三品以下官。八月庚子,即皇帝位。先天元年十月庚子,享於太廟,大赦。



 開元元年正月辛巳,皇后親蠶。七月甲子,太平公主及岑羲、蕭至忠、竇懷貞謀反,伏誅。乙丑,始聽政。丁卯,大赦,賜文武官階、爵。庚午,流崔湜於竇州。甲戌,毀天樞。乙亥,尚書右丞張說檢校中書令。庚辰,陸象先罷。八月癸巳,劉幽求為尚書右僕射,知軍國大事。壬寅,宋王成器為太尉,申王成義為司徒,邠王守禮為司空。九月丙寅,宋王成器罷。庚午,劉幽求同中書門下三品,張說為中書令。十月,姚雋蠻設姚州,都督李蒙死之。己亥,幸溫湯。癸卯,講武於驪山。流郭元振於新州,給事中唐紹伏誅。免新豐來歲稅,賜從官帛。甲辰,獵於渭川。同州刺史姚元之為兵部尚書、同中書門下三品。乙巳,至自渭川。十一月乙丑,劉幽求兼侍中。戊子,群臣上尊號曰開元神武皇帝。十二月庚寅,大赦,改元,賜內外官勛。改中書省為紫微省,門下省為黃門省,侍中為監。甲午,吐蕃請和。巳亥,禁潑寒胡戲。壬寅,姚崇兼紫微令。癸丑,劉幽求罷。貶張說為相州刺史。甲寅,黃門侍郎盧懷慎同紫微黃門平章事。



 二年正月壬午,以關內旱,求直諫,停不急之務,寬系囚,祠名山大川,葬暴骸。甲申,並州節度大使薛訥同紫微黃門三品,以伐契丹。二月壬辰,避正殿,減膳,徹樂。突厥寇北庭,都護郭虔瓘敗之。己酉,慮囚。三月己亥,磧西節度使阿史那獻執西突厥都擔。四月辛未,停諸陵供奉鷹犬。五月辛亥,魏知古罷。六月,京師大風拔木。甲子,以太上皇避暑,徙御大明宮。七月乙未,焚錦繡珠玉於前殿。戊戌,禁採珠玉及為刻鏤器玩、珠繩帖綏服者,廢織錦坊。庚子,薛訥及奚、契丹戰於灤河,敗績。丁未,襄王重茂薨,追冊為皇帝。八月壬戌,禁女樂。乙亥,吐蕃寇邊,薛訥攝左羽林軍將軍,為隴右防禦大使,右驍衛將軍郭知運為副,以伐之。九月庚寅,作興慶宮。丁酉,宴京師侍老於含元殿庭,賜九十以上幾、杖,八十以上鳩杖,婦人亦如之,賜於其家。戊申,幸溫湯。十月戊午,至自溫湯。甲子,薛訥及吐蕃戰於武階,敗之。十二月乙丑,封子嗣真為鄫王,嗣初鄂王,嗣玄鄄王。



 三年正月丁亥,立郢王嗣謙為皇太子。降死罪,流以下原之。賜酺三日。癸卯,盧懷慎檢校黃門監。二月辛酉,赦囚非惡逆、造偽者。四月庚申,突厥部三姓葛邏祿來附。右羽林軍大將軍薛訥為源州鎮軍大總管,源州都督楊執一副之;右衛大將軍郭虔瓘為朔州鎮軍大總管,並州長史王晙副之。以備突厥。五月丁未,以旱錄京師囚。戊申,避正殿,減膳。七月庚辰朔,日有食之。十月辛酉,巂州蠻寇邊,右驍衛將軍李玄道伐之。壬戌,薛訥為朔方道行軍大總管,太僕卿呂延祚、靈州刺史杜賓客副之。癸亥,如眉阜,赦所過徒罪以下,賜侍老九十以上及篤疾者物。甲子,如鳳泉湯。戊辰,降大理系囚罪。十一月己卯,至自鳳泉湯。乙酉,幸溫湯。丁亥,相州人崔子巖反,伏誅。甲午,至自溫湯。乙未,禁白衣長發會。十二月乙丑,降鳳泉湯所過死罪以下。



 四年正月戊寅,朝太上皇於西宮。二月丙辰,幸溫湯。辛酉,吐蕃寇松州,廓州刺史蓋思貴伐之。丁卯,至自溫湯。癸酉,松州都督孫仁獻及吐蕃戰,敗之。六月甲子,太上皇崩。辛未,京師、華陜二州大風拔木。癸酉,大武軍子將郝靈佺殺突厥默啜。七月丁丑,吐蕃請和。丁酉,洛水溢。八月辛未,奚、契丹降。十月庚午,葬大聖真皇帝於橋陵。十一月己卯,盧懷慎罷。丁亥,選中宗於酉廟。丙申,尚書左丞源乾曜為黃門侍郎、同紫微黃門平章事。十二月乙卯,定陵寢殿火。丙辰,幸溫湯。乙丑,至自溫湯。閏月己亥,姚崇、源乾曜罷。刑部尚書宋璟為吏部尚書兼黃門監,紫微侍郎蘇頲同紫微黃門平章事。



 五年正月癸卯,太廟四室壞,遷神主於太極殿,素服避正殿,輟視朝五日。己酉,享於太極殿。辛亥,如東都。戊辰,大霧。二月甲戌,大赦,賜從官帛,給復河南一年,免河南北蝗、水州今歲租。三月丙寅,吐蕃請和。四月甲申,毀拜洛受圖壇。己丑,子嗣一卒。五月丙辰,詔公侯子孫襲封。七月壬寅,隴右節度使郭知運及吐蕃戰,敗之。九月壬寅,復紫微省為中書省,黃門省為門下省,監為侍中。十月戊寅,祔神主於太廟。甲申,命史官月奏所行事。



 六年正月辛丑,突厥請和。二月壬辰,朔方道行軍大總管王脧伐突厥。六月甲申,瀍水溢。八月庚辰,以旱慮囚。十月癸亥,賜河南府、懷汝鄭三州父老帛。十一月辛卯,至自東都。丙申,享於太廟。元皇帝以上三祖枝孫失官者授五品京官。皇祖妣家子孫在選者甄擇之。免知頓及旁州供承者一歲租稅。乙巳,改傳國璽曰「寶」。是月,突厥執單于副都護張知運。



 七年五月己丑朔,日有食之,素服,徹樂,減膳,中書門下慮囚。六月戊辰,吐蕃請和。閏七月辛巳,以旱避正殿,徹樂,減膳。甲申,慮囚。八月丙戌,慮囚。九月甲戌,徙封宋王憲為寧王。十月,作義宗廟於東都。辛卯,幸溫湯。癸卯,至自溫湯。十一月乙亥,皇太子入學齒胄,賜陪位官及學生帛。



 八年正月辛巳,宋璟、蘇頲罷。京兆尹源乾曜為黃門侍郎,並州大都督府長史張嘉貞為中書侍郎:同中書門下平章事。二月戊戌,子敏卒。三月甲子,免水旱州逋負,給復四鎮行人家一年。五月丁卯,源乾曜為侍中,張嘉貞為中書令。六月庚寅,洛、瀍、穀水溢。九月,突厥寇甘、涼,涼州都督楊敬述及突厥戰,敗績。丙寅,降京城囚罪,杖以下原之。壬申,契丹寇邊,王晙檢校幽州都督、節度河北諸軍大使,黃門侍郎韋抗為道朔方行軍大總管,以伐之。甲戌,中書門下慮囚。十月辛巳,如長春宮。壬午,獵於下邽。庚寅,幸溫湯。十一月乙卯,至自溫湯。



 九年正月,括田。丙寅,幸溫湯。乙亥,至自溫湯。二月丙戌,突厥請和。丁亥,免天下七年以前逋負。四月庚寅,蘭池胡康待賓寇邊。五月庚午,原見囚死、流罪隨軍郊力、徒以下未發者。七月己酉,王晙執康待賓。八月,蘭池胡康願子寇邊。九月乙巳朔,日有食之。癸亥,天兵軍節度大使張說為兵部尚書、同中書門下三品。十一月庚午,大赦,賜文武官階、爵,唐隆、先天實封功臣坐事免若死者加贈,賜民酺三日。十二月乙酉,幸溫湯。壬辰,至自溫湯。是冬,無雪。



 十年正月丁巳,如東都。二月丁丑,次望春頓,賜從官帛。四月己亥,張說持節朔方軍節度大使。五月戊午,突厥請和。辛酉,伊、汝水溢。閏月壬申,張說巡邊。六月丁巳,河決博、棣二州。七月庚辰,給復遭水州。丙戌,安南人梅叔鸞反,伏誅。九月,張說敗康願子於木盤山,執之。己卯,京兆人權梁山反,伏誅。癸未,吐蕃攻小勃律,北庭節度使張孝嵩敗之。十月甲寅,如興泰宮,獵於上宜川。庚申,如東都。十二月,突厥請和。



 十一年正月丁卯,降東都囚罪,徙以下原之。賜侍老物。庚辰,次潞州,赦囚,給復五年,以故第為飛龍宮。辛卯,次並州,改並州為北都。癸巳,赦太原府,給復一年,下戶三年元從家五年。版授侍老八十以上上縣令,婦人縣君;九十以上上州長史,婦人郡君;百歲以上上州刺史,婦人郡夫人。二月己酉,貶張嘉貞為幽州刺史。壬子,如汾陰,祠后土,賜文武官階、勛、爵、帛。癸亥,張說兼中書令。三月辛未,至自汾陰,免所過今歲稅,赦京城。四月甲子,張說為中書令。吏部尚書王晙為兵部尚書、同中書門下三品。五月乙丑,復中宗於太廟。己丑,王晙持節朔方軍節度大使。辛卯,遣使分巡天下。六月,王晙巡邊。八月戊申,追號宣皇帝曰獻祖,光皇帝曰懿祖。十月丁酉,幸溫湯,作溫泉宮。甲寅,至自溫湯。十一月戊寅,有事於南郊,大赦。賜奉祠官階、勛、爵,親王公主一子官,高年粟帛,孝子順孫終身勿事。天下酺三日,京城五日。十二月甲午,如鳳泉湯。戊申,至自鳳泉湯。庚申,貶王晙為蘄州刺史。



 十二年四月壬寅,詔傍繼國王禮當廢而屬近者封郡王。七月己卯,廢皇后王氏為庶人。十月,庶人王氏卒。十一月庚午,如東都。庚辰,溪州首領覃行章反,伏誅。辛巳,申王捴薨。閏十二月丙辰朔,日有食之。



 十三年正月戊子,降死罪,流以下原之。遣使宣慰天下。壬子,葬朔方隴右河西戰亡者。三月甲午,徙封郯王潭為慶王,陜王浚忠王,鄫王洽棣王,甄王滉榮王。封子涺為光王,濰儀王,沄潁王,澤永王,清壽王,洄延王,沐盛王,溢濟王。九月丙戌,罷奏祥瑞。十月辛酉,如兗州。庚午,次濮州,賜河南、北五百里內父老帛。十一月庚寅,封於泰山。辛卯,禪於社首。壬辰,大赦。賜文武官階、勛、爵,致仕官一季祿,公主、嗣王、郡縣主一子官,諸蕃酋長來會者一官。免所過一歲、兗州二歲租。賜天下酺七日。丙申,幸孔子宅,遣使以太牢祭其墓,給復近墓五戶。丁酉,賜徐、曹、毫、許、仙、豫六州父老帛。十二月己巳,如東都。



 十四年二月,邕州獠梁大海反,伏誅。四月丁巳,戶部侍郎李元紘為中書侍郎、同中書門下平章事。庚申,張說罷。丁卯,岐王範薨。六月戊午,東都大風拔木。壬戌,詔州縣長官言事。七月癸未,瀍水溢。八月丙午,河決魏州。九月己丑,磧西節度使杜暹檢校黃門侍郎、同中書門下平章事。十月甲寅,太白晝見。庚申,如廣成湯。己巳,如東都。十二月丁巳,獵於方秀川。



 十五年正月辛丑,河西、隴右節度使王君奐及吐蕃戰於青海,敗之。七月甲戌,震興教門觀,災。庚寅,洛水溢。己亥,降都城囚罪,徒以下原之。八月,澗、谷溢,毀澠池縣。己巳,降天下死罪、嶺南邊州流人,徒以下原之。九月丙子,吐蕃寇瓜州,執刺史田元獻。閏月庚子,寇安西,副大都護趙頤貞敗之。庚申,回紇襲甘州,王君奐死之。十月己卯,至自東都。十一月丁卯,獵於城南。十二月乙亥,幸溫泉宮。丙戌,至自溫泉宮。



 十六年正月壬寅,趙頤貞及吐蕃戰於曲子城,敗之。乙卯,瀧州首領陳行範反,伏誅庚申,許徒以下囚保任營農。三月辛丑,免營農囚罪。七月,吐蕃寇瓜州,刺史張守珪敗之。乙巳,隴右節度使張志亮、河西節度使蕭嵩克吐蕃大莫門城。八月辛卯,及吐蕃戰於祁連城,敗之。九月丙午,以久雨降囚罪,徒以下原之。十月己卯,幸溫泉宮。己丑,至自溫泉宮。十一月癸巳,蕭嵩為兵部尚書、同中書門下平章事。甲辰,弛陂澤禁。戊申,幸寧王憲第。庚戌,至自寧王憲第。十二月丁卯,幸溫泉宮。丁丑,至自溫泉宮。



 十七年二月丁卯,雋州都督張審素克雲南昆明城、鹽城。三月戊戌,張守珪及吐蕃戰於大同軍,敗之。四月癸亥,降死罪,流以下原之。乙亥,大風,震,藍田山崩。六月甲戌,源乾曜、杜暹、李元罷紘。蕭嵩兼中書令。戶部侍郎宇方融為黃門侍郎,兵部侍郎裴光庭為中書侍郎:同中書門下平章事。九月壬子,貶宇文融為汝州刺史。十月戊午朔,日有食之。十一月庚寅,享於太廟。丙申,拜橋陵,赦奉先縣。戊戌,拜定陵。己亥,拜獻陵。壬寅,拜昭陵。乙巳,拜乾陵。戊申,至自乾陵,大赦。免今歲稅之半。賜文武官階、爵,侍老帛。旌表孝子順孫、義夫節婦,終身勿事。唐隆兩營立功三品以上予一子官。免供頓縣今歲稅。賜諸軍行人勛兩轉。十二月辛酉,幸溫泉宮。壬申,至自溫泉宮。是冬,無雪。



 十八年正月辛卯,裴光庭為侍中。二月丙寅,大雨,雷震左飛龍廄,災。辛未,免囚罪杖以下。四月乙卯,築京師外郭。五月己酉,奚、契丹附於突厥。六月甲子,有彗星出於五車。癸酉,有星孛於華、昂。乙亥,瀍水溢。丙子,忠王浚為河北道行軍元帥。壬午,洛水溢。九月丁巳,忠王浚兼河東道諸軍元帥。十月戊子,吐蕃請和。庚寅,如鳳泉湯。癸卯,至自鳳泉湯。十一月丁卯,幸溫泉宮。丁丑,至自溫泉宮。



 十九年正月,殺瀼州別駕王毛仲。丙子,耕於興慶宮。己卯,禁捕鯉魚。四月壬午,降死罪以下。丙申,立太公廟。六月乙酉,大風拔木。七月癸丑,吐蕃請和。八月辛巳,以千秋節降死罪,流以下原之。十月丙申,如東都。十一月乙卯,次洛城南,賜從官帛。是歲,揚州穞稻生。



 二十年正月乙卯,信安郡王禕為河東、河北道行軍副元帥,以伐奚、契丹。二月甲戌朔,日有食之。壬午,降囚罪,徒以下原之。三月己巳,信安郡王禕及奚、契丹戰於薊州,敗之。五月戊申,忠王浚俘奚、契丹以獻。六月丁丑,浚為司徒。八月辛未朔,日有食之。九月乙巳,渤海靺鞨寇登州,刺史韋俊死之,左領軍衛將軍蓋福慎伐之。戊辰,以宋、滑、袞、鄆四州水,免今歲稅。十月壬午,如潞州。丙戌,中書門下慮巡幸所過囚。辛卯,赦潞州,給復三年,賜高年粟帛。十一月辛丑,如北都。癸丑,赦北都,給復三年。庚申,如汾陰,祠后土,大赦。免供頓州今歲稅。賜文武官階、勛、爵,諸州侍老帛,武德以來功臣後及唐隆功臣三品以上一子官。民酺三日。十二月辛未,至自汾陰。



 二十一年正月丁巳,幸溫泉宮。二月丁亥,至自溫泉宮。三月乙巳,裴光庭薨。甲寅,尚書右丞韓休為黃門侍郎、同中書門下平章事。閏月癸酉,幽州副總管郭英傑及契丹戰於都山,英傑死之。四月乙卯,遣宣慰使黜陟官吏,決擊囚。丁巳,寧王憲為太尉,薛王業為司徒。五月戊子,以皇太子納妃,降死罪,流以下原之。七月乙丑朔,日有食之。九月壬午,封子沔為信王,泚義王,漼陳王,澄豐王,潓恆王,漩涼王,滔深王。十月庚戌,幸溫泉宮。己未,至自溫泉宮。十二月丁巳,蕭嵩、韓休罷。京兆尹裴耀卿為黃門侍郎,中書侍郎張九齡:同中書門下平章事。



 二十二年正月己巳,如東都。二月壬寅,秦州地震,給復壓死者家一年,三人者三年。四月甲辰,降死罪以下。甲寅,北庭都護劉渙謀反,伏誅。五月戊子,裴耀卿為侍中,張九齡為中書令,黃門侍郎李林甫為禮部尚書、同中書門下三品。是日,大風拔木。六月壬辰,幽州節度使張守珪俘奚、契丹以獻。七月己巳,薛王業薨。十一月甲戌,免關內、河南八等以下戶田不百畝者今歲租。十二月戊子朔,日有食之。乙巳,張守珪及契丹戰,敗之,殺其王屈烈。



 二十三年正月乙亥,耕藉田。大赦。侍老百歲以上版授上州刺史,九十以上中州刺史,八十以上上州司馬。賜陪位官勛、爵。征防兵父母年七十者遣還。民酺三日。八月戊子,免鰥寡惸獨今歲稅米。十月戊申,突騎施寇邊。閏十一月壬午朔,日有食之。是冬,東都人劉普會反,伏誅。



 二十四年正月丙午,北庭都護蓋嘉運及突騎施戰,敗之。四月丁丑,降死罪以下。五月丙午,醴泉人劉志誠反,伏誅。八月甲寅,突騎施請和。乙亥,汴王璥薨。十月戊申,京師地震。甲子,次華州,免供頓州今歲稅,賜刺史、縣令中上考。降兩京死罪,流以下原之。丁卯,至自東都。十一月辛丑,東都地震。壬寅,裴耀卿、張九齡罷。李林甫兼中書令,朔方軍節度副大使牛仙客為工部尚書、同中書門下三品。十二月戊申,慶王琮為司徒。



 二十五年三月乙酉,張守珪及契丹戰於捺祿山,敗之。辛卯,河西節度副大使崔希逸及吐蕃戰於青海,敗之。四月辛酉,殺監察御史周子諒。乙丑,廢皇太子瑛及鄂王瑤、光王琚為庶人,皆殺之。十一月壬申,幸溫泉宮。乙酉,至自溫泉宮。十二月丙午,惠妃武氏薨。丁巳,追冊為皇后。



 二十六年正月甲戌,潮州刺史陳思挺謀反,伏誅。乙亥,牛仙客為侍中。丁丑,迎氣於東郊。降死罪,流以下原之。以京兆稻田給貧民,禁王公獻珍物,賜文武官帛。壬辰,李林甫兼隴右節度副大使。二月乙卯,牛仙客兼河東節度副大使。三月丙子,有星孛於紫微。癸巳,京師地震。吐蕃寇河西,崔希逸敗之,鄯州都督杜希望克其新城。四月己亥,有司讀時令。降死罪,流以下原之。五月乙酉,李林甫兼河酉節度副大使。六月庚子,立忠王璵為皇太子。七月己巳,大赦。賜文武九品以上及五品以上子為父後者勛一轉,侍老粟帛,加版授。免京畿下戶今歲租之半。賜民酺三日。九月丙申朔,日有食之。庚子,益州長史王昱及吐蕃戰於安戎城,敗績。十月戊寅,幸溫泉宮。壬辰,至自溫泉宮。



 二十七年正月壬寅,榮王琬巡按隴右。二月己巳,群臣上尊號曰開元聖文神武皇帝,大赦。免今歲稅。賜文武官階、爵。版授侍老百歲以上下州刺史,婦人郡君;九十以上上州司馬,婦人縣君;八十以上縣令,婦人鄉君。賜民酺五日。八月乙亥,磧西節度使蓋嘉運敗突騎施於賀邏嶺,執其可汗吐火仙。壬午,吐蕃寇邊河西、隴右節度使蕭炅敗之。十月丙戌,幸溫泉宮。十一月辛丑,至自溫泉宮。



 二十八年正月癸巳,幸溫泉宮。庚子,至自溫泉宮。三月丁亥朔,日有食之。壬子,益州司馬章仇兼瓊敗吐蕃,克安戎城。五月癸卯,吐蕃寇安戎城,兼瓊又敗之。十月甲子,幸溫泉宮。以壽王妃楊氏為道士,號太真。戊辰,以徐、泗二州無蠶,免今歲稅。辛巳,至自溫泉宮。十一月,牛仙客罷朔方、河東節度副大使。



 二十九年正月癸巳,幸溫泉宮。丁酉,立玄元皇帝廟,禁厚葬。庚子,至自溫泉宮。五月庚戌,求明《道德經》及《莊》、《列》、《文子》者。降死罪,流以下原之。七月乙亥,伊、洛溢。九月丁卯,大雨雪。十月丙申,幸溫泉宮。戊戌,遣使黜陟官吏。十一月庚戌,邠王守禮薨。辛酉,至自溫泉宮。己巳,雨木冰。辛未,寧王憲薨,追冊為皇帝,及其妃元氏為皇后。十二月癸未,吐蕃陷石堡城。



 天寶元年正月丁未,大赦,改元。詔京文武官材堪刺史者自舉。賜侍老八十以上粟帛,九品以上勛兩轉。甲寅,陳王府參軍田同秀言:「玄元皇帝降於丹鳳門通衢。」二月丁亥,群臣上尊號曰開元天寶聖文神武皇帝。辛卯,享玄元皇帝於新廟。甲午,享於太廟。丙申,合祭天地於南郊,大赦。侍老加版授,賜文武官階、爵。改侍中為左相,」中書令為右相,東都為東京,北都為北京,州為郡,刺史為太守。七月癸卯朔,日有食之。辛未,牛仙客薨。八月丁丑,刑部尚書李適之為左相。十月丁酉,幸溫泉宮。十一月己巳,至自溫泉宮。十二月戊戌,隴右節度使皇甫惟明及吐蕃戰於青海,敗之。庚子,河西節度使王倕克吐蕃漁海、游奕軍。朔方軍節度使王忠嗣及奚戰於紫乾河,敗之,遂伐突厥。是冬,無冰。



 二年正月乙卯,作升仙宮。丙辰,加號玄元皇帝曰大聖祖。三月壬子,享於玄元宮,追號大聖祖父周上御大夫敬曰先天太皇,咎繇曰德明皇帝涼武昭王曰興聖皇帝。改西京玄元宮曰太清宮,東京曰太微宮。四月己卯,皇甫惟明克吐蕃洪濟城。六月甲戌,震東京應天門觀,災。十月戊寅,幸溫泉宮。十一月乙卯,至自溫泉宮。十二月壬午,海賊吳令光寇永嘉郡。是冬,無雪。



 三載正月丙申,改年為載。降死罪,流以下原之。辛丑,幸溫泉宮。辛亥,有星隕於東南。二月庚午,至自溫泉宮。丁丑,河南尹裴敦復、晉陵郡太守劉同升、南海郡太守劉巨鱗討吳令光。閏月,令光伏誅。三月壬申,降死罷,流以下原之。八月丙午,拔悉蜜攻突厥,殺烏蘇米施可汗,來獻其首。十月甲午,幸溫泉宮。十一月丁卯,至自溫泉宮。十二月癸丑,祠九宮貴神於東郊,大赦。詔天下家藏《孝經》。賜文武官階、爵,侍老粟帛,民酺三日。



 四載正月丙戌,王忠嗣及突厥戰於薩河內山,敗之。三月壬申,以外孫獨孤氏女為靜樂公方,嫁於契丹松漠都督李懷節;楊氏女為宜芳公主,嫁於奚饒樂都督李廷寵。八月壬寅,立太真為貴妃。九月,契丹、奚皆殺其公主以叛。甲申,皇甫惟明及吐蕃戰於石堡城,副將褚誗死之。十月戊戌,幸溫泉宮。十二月戊戌,至自溫泉宮。



 五載正月乙亥,停六品以下員外官。三月丙子,遣使黜陟官吏。四月庚寅,李適之罷。丁酉,門下侍郎陳希烈同中書門下平章事。五月壬子朔,日有食之。七月,殺括蒼郡太守韋堅、播川郡太守皇甫惟明。十月戊戌,幸溫泉宮。十一月乙巳,至自溫泉宮。十二月甲戌,殺贊善大夫杜有鄰、著作郎王曾、左驍衛兵曹參軍柳勣、左司禦率府倉曹參軍王脩己、右武衛司戈盧寧、左威衛參軍徐徵。



 六載正月辛巳,殺北海郡太守李邕、淄川郡太守裴敦復。丁亥,享於太廟。戊子,有事於南郊,大赦,流人老者許致仕,停立仗鋜。賜文武官階、爵,侍老粟帛,民酺三日。三月甲辰,陳希烈為左相。七月乙酉,以旱降死罪,流以下原之。十月戊申,幸華清宮。十一月丁酉,殺戶部侍郎楊慎矜及其弟少府少監慎餘、洛陽令慎名。十二月癸丑,至自華清宮。是歲,安西副都護高仙芝及小勃律國戰,敗之。



 七載五月壬午,群臣上尊號曰開元天寶聖文神武應道皇帝,大赦,免來載租、庸。以魏、周、隋為三恪。賜京城父老物人十段。七十以上版授本縣令,婦人縣君,六十以上縣丞。天下侍老百歲以上上郡太守,婦人郡君;九十以上上郡司馬,婦人縣君;八十以上縣令,婦人鄉君。賜文武官勛兩轉,民酺三日。十月庚戌,幸華清宮。十二月辛酉,至自華清宮。



 八載四月,殺咸寧郡太守趙奉璋。六月乙卯,隴右節度使哥舒翰及吐蕃戰於石堡城,敗之。閏月丙寅,謁太清宮,加上玄元皇帝號曰聖祖大道玄元皇帝,增祖宗帝後謚。郡臣上尊號曰開元天地大寶聖文神武應道皇帝,大赦,男子七十、婦人七十五以上皆給一子侍,賜文武官階、爵,民為戶者古爵,酺三日。十月乙丑,幸華清宮。是月,特進何履光率十道兵以伐雲南。十一月丁巳,幸御史中丞楊釗莊。



 九載正月己亥,至自華清宮。丁巳,詔以十一月封華嶽。三月辛亥,華嶽廟災,關內旱,乃停封。五月庚寅,慮囚。九月辛卯,以商、周、漢為三恪。十月庚申,幸華清宮。太白山人王玄翼言:「玄元皇帝降於寶仙洞。」十二月乙亥,至自華清宮。是歲,雲南蠻陷雲南郡,都督張虔陀死之。



 十載正月壬辰,朝獻於太清宮。癸巳,朝享於太廟。甲午,有事於南郊,大赦,賜侍老粟帛,酺三日。丁酉,李林甫兼朔方軍節度副大使、安北副大都護。己亥,改傳國寶為「承天大寶」。戊申,安西四鎮節度使高仙芝執突騎施可汗及石國王。四月壬午,劍南節度使鮮於仲通及雲南蠻戰於西洱河,大敗績,大將王天運死之,陷雲南都護府。七月,高仙芝及大食戰於恆邏斯城,敗績。八月,範陽節度副大使安祿山及契丹戰於吐護真河,敗績。乙卯,廣陵海溢。丙辰,武庫災。十月壬子,幸華清宮。十一月乙未,幸楊國忠第。



 十一載正月丁亥,致自華清宮。二月庚午,突厥部落阿布思寇邊。三月乙巳,改尚書省八部名。四月乙酉,戶部郎中王銲、京兆人邢縡謀反,伏誅。丙戌,殺御史大夫王鉷。李林甫罷安北副大都護。五月戌申,慶王琮薨。甲子,東京大風拔木。六月壬午,御史大夫兼劍南節度使楊國忠敗吐蕃於雲南,克故洪城。十月戊寅,幸華清宮。十一月乙卯,李林甫薨。庚申,楊國忠為右相。十二月丁亥,至自華清宮。



 十二載五月己酉,復魏、周、隋為三恪。六月,阿布思部落降。八月,中書門下慮囚。九月甲寅,葛邏祿葉護執阿布思。十月戊寅,幸華清宮。



 十三載正月丙午,至自華清宮。二月壬申,朝獻於太清宮,加上玄元皇帝號曰大聖祖高上大道金厥玄元天皇大帝。癸酉,朝享於太廟,增祖宗謚。甲戌,群臣上尊號曰開元天地大寶聖文神武證道孝德皇帝,大赦,左降官遭父母喪者聽歸。賜孝義旌表者勛兩轉。侍老百歲以上版授本郡太守,婦人郡夫人;九十以上郡長史,婦人郡君;八十以上縣令,婦人縣君。太守加賜爵一級,縣令勛兩轉,民酺三日。丁丑,楊國忠為司空。是日,雨土。三月,隴右、河西節度使哥舒翰敗吐蕃,復河源九曲。辛酉,大風拔木。五月壬戌,觀酺於勤政樓,北庭都護程千里俘阿布思以獻。六月乙丑朔,日有食之。劍南節度留後李宓及雲南蠻戰於西洱河,死之。八月丙戌,陳希烈罷。文部侍郎韋見素為武部尚書、同中書門下平章事。是秋,瀍、洛水溢。十月乙酉,幸華清宮。十二月戊午,至自華清宮。



 十四載三月壬午,安祿山及契丹戰於潢水,敗之。五月,天有聲於浙西。八月辛卯,降死罪,流以下原之。免今載租、庸半。賜侍老米。十月庚寅,幸華清宮。十一月,安祿山反,陷河北諸郡。範陽將何千年殺河東節度使楊光翽。壬申,伊西節度使封常清為範陽、平盧節度使,以討安祿山。丙子,至自華清宮。九原郡太守郭子儀為朔方軍節度副大使,右羽林軍大將軍王承業為太原尹,衛尉卿張介然為河南節度採訪使,右金吾大將軍程千里為上黨郡長史,以討安祿山。丁丑,榮王琬為東討元帥,高仙芝副之。十二月丁亥,安祿山陷靈昌郡。辛卯,陷陳留郡,執太守郭納,張介然死之。癸巳,安祿山陷滎陽郡,太守崔無詖死之。丙申,封常清及安祿山戰於甕子穀,敗績。丁酉,陷東京,留守李憕、御史中丞盧弈、判官蔣清死之。河南尹達奚珣叛降於安祿山。己亥,恆山郡太守顏杲卿敗何千年,執之,克趙、鉅鹿、廣平、清河、河間、景城、樂安、博平、博陵、上谷、文安、信都、魏、鄴十四郡。癸卯,封常清、高仙芝伏誅。哥舒翰持節統領處置太子先鋒兵馬副元帥,守潼關。甲辰,郭子儀及安祿山將高秀巖戰於河曲,敗之。戊申,榮王琬薨。壬子,濟南郡太守李隨、單父尉賈賁、濮陽人尚衡以兵討安祿山。是月,平原郡太守顏真卿、饒陽郡太守盧全誠、司馬李正以兵討安祿山。



 十五載正月乙卯,東平郡太守嗣昊王祗以兵討安祿山。丙辰,李隋為河南節度使,以討安祿山。壬戌,祿山陷恆山郡,執顏杲卿、袁履謙,陷鄴、廣平、鉅鹿、趙、上谷、博陵、文安、魏、信都九郡。癸亥,朔方軍節度副使李光弼為河東節度副大使,以討祿山。甲子,南陽郡太守魯炅為南陽節度使,率嶺南、黔中、山南東道兵屯於葉縣。乙丑,安慶緒寇潼關,哥舒翰敗之。丁丑,真源令張巡以兵討安祿山。二月己亥,嗣吳王祗及祿山將謝元同戰於陳留,敗之。李光弼克常山郡,郭子儀出井陘會光弼,及安祿山將史思明戰,敗之。庚子,賈賁戰於雍丘,死之。三月,顏真卿克魏郡。史思明寇饒陽、平原。乙卯,張巡及安祿山將令狐潮戰於雍丘,敗之。丙辰,殺戶部尚書安思順、太僕卿安元貞。乙丑,李光弼克趙郡。四月乙酉,北海郡太守賀蘭進明以兵救平原。丙午,太子左贊善大夫來瑱為潁川郡太安、兼招討使。五月丁巳,魯炅及安祿山戰於滍水,敗績,奔於南陽。戊辰,嗣虢王巨為河南節度使。六月癸未,顏真卿及安祿山將袁知泰戰於堂邑,敗之。賀蘭進明克信都。丙戌,哥舒翰及安祿山戰於靈寶西原,敗績。是日,郭子儀、李光弼及史思明戰於嘉山,敗之。辛卯蕃將火拔歸仁執哥舒翰叛降於安祿山,遂陷潼關、上洛郡。甲午,詔親征。京兆尹崔光遠為西京留守、招討處置使。丙申,行在望賢宮。丁酉,次馬嵬,左龍武大將軍陳玄禮殺楊國忠及御史大夫魏方進、太常卿楊暄。賜貴妃楊氏死。是日,張巡及安祿山將翟伯玉戰於白沙堝,敗之。己亥,祿山陷京師。辛丑,次陳倉。閑廄使任沙門叛降於祿山。丙午,次河池郡。劍南節度使崔圓為中書侍郎、同中書門下平章昌事。七月甲子,次普安郡。憲部侍郎房琯為文部尚書、同中書門下平章事。丁卯,皇太子為天下兵馬元帥,都統朔方、河東、河北、平盧節度使,御史中丞裴冕、隴酉郡司馬劉秩副之。江陵大都督永王璘為山南東路黔中江南西路節度使,盛王琦為廣陵郡都督、江南東路淮南道節度使,豐王珙為武威郡都督、河西隴右安西北庭節度使。庚午,次巴西郡。以太守崔渙為門下侍郎、同中書門下平章事,韋見素為左相。庚辰,次蜀郡。八月壬午,大赦,賜文武官階、爵,為安祿山脅從能自歸者原之。癸巳,皇太子即皇帝位於靈武,以聞。庚子,上皇天帝誥遣韋見素、房琯、崔渙奉皇帝冊於靈武。十一月甲寅,憲部尚書李麟同中書門下平章事。十二月甲辰,永王璘反,廢為庶人。



 至德二載正月庚戌,誥求天下孝悌可旌者。甲子,劍南健兒賈秀反,伏誅。三月庚午,通化郡言玄元皇帝降。五月庚申,誥追冊貴嬪楊氏為皇后。七月庚戌,行營健兒李季反,伏誅。庚午,劍南健兒郭千仞反,伏誅。十月丁巳,皇帝復京師,以聞。誥降劍南囚罪,流以下原之。十二月丁未,至自蜀郡,居於興慶宮。三載,上號曰太上至道聖皇天帝。上元元年,徙居於西內甘露殿。元年建巳月,崩於神龍殿,年七十八。



 贊曰:睿宗因其子之功,而在位不久,固無可稱者。嗚呼,女子之禍於人者甚矣!自高祖至於中宗,數十年間,再罹女禍,唐祚既絕而復續,中宗不免其身,韋氏遂以滅族。玄宗親平其亂,可以鑒矣,而又敗以女子。方其勵精政事,開元之際,幾致太平,何其盛也!及侈心一動,窮天下之欲不足為其樂,而溺其所甚愛,忘其所可戒,至於竄身失國而不悔。考其始終之異,其性習之相遠也至於如此。可不慎哉!可不慎哉!



\end{pinyinscope}