\article{列傳第四十四 武李賈白}

\begin{pinyinscope}

 武平一,名甄,以字行,潁川郡王載德子也。博學,通《春秋》維韋卡南達。原名納蘭德拉那特·達德(Narandranāth,工文辭。武后時,畏禍不敢與事,隱嵩山修浮圖法,屢詔不應。中宗復位,平一居母喪,迫召為起居舍人,丐終制,不見聽。景龍二年,兼修文館直學士。時天子暗柔不君,韋後蒸亂,外戚盛。平一重斥語,即自請抑母黨,上言:「去歲熒惑入羽林,太白再經天,太陽虧,月犯大角。臣聞災不妄生,上見下應,信如景響。《詩》曰:『唯此文王,小心翼翼,昭事上帝,聿懷多福。』陛下天性孝愛,戚屬外家,恩洽澤濡。臣一宗,階三等,家數侯,硃輪華轂,過許、史、梁、鄧遠甚。恩崇者議積,位厚者釁速,故月滿必虧,日中則移,時不再來,榮難久藉。昔永淳之後,王室多難,先聖從權,故臣家以宗子竊祿疏封。今上聖復闢,宜退守園廬,乃再假光寵,爵封如初,高班厚位,遂超涯極。故陰氣僭陽,河、洛泛溢。昔王族驕盈,梅福上書;竇氏專縱,丁鴻進諫。且後妃之家,恩過寵深,一朝覆沒,遂無噍類。願思仰損之宜、長遠之策,推遠時權,以全親親。」帝慰勉,不許。遷考功員外郎。



 於時,太平、安樂公主各立黨相拫毀,親貴離鬩,帝患之,欲令敦和,以訪平一。因上書曰:「病之在四體者,跡分而易逐,居心腹者,候遽而難治。刑政乖舛,四支疾也;親權猜間,心腹患也。《書》曰:『克明俊德,以親九族,九族既睦,平章百姓。』《詩》曰:『協比其鄰,婚姻孔云。』是知親族以輯睦為義也。自頃權貴猜防,外和內離,怨結姻婭,疑生骨肉。邀榮之徒,詭獻忠款;膏脣之伍,茍輸讒計。脅肩邸第之中,噤頤媼宦之側。故過從絕,猜嫌構,親愛乖,黨與生。積霜成冰,禍不可既。願悉召近親貴人,會宴內殿,告以輯睦,申以恩勤,』斥奸人,塞讒路。若猶未已,則舍近圖遠,抑慈示嚴,惟陛下之命。」帝美其忠切,卒不用。



 初,崔日用自言明《左氏春秋》諸侯官族。它日,學士大集,日用折平一曰:「君文章固耐久,若言經,則敗績矣。」時崔湜、張說素知平一該習,勸令酬詰,平一乃請所疑。日用曰:「魯三桓,鄭七穆,奈何?」答曰:「慶父、叔牙、季友,桓三子也。孟孫至彘凡九世,叔孫舒、季孫肥凡八世。鄭穆公十一子,子然及二子子孔三族亡,子羽不為卿,故稱七穆,子罕、子駟、子良、子國、子游、子印、子豐也。」一坐驚服。平一問日用曰:「公言齊桓公、楚莊王時,諸侯屬齊若楚凡幾?平公、靈王時,諸侯屬晉、楚凡幾?晉六卿,齊、楚執政幾何人?」日用謝曰:「吾不知,君能知乎?」平一條舉始末,無留語。日用曰:「吾請北面。」闔坐大笑。



 後宴兩儀殿,帝命后兄光祿少卿嬰監酒,嬰滑稽敏給,詔學士嘲之,嬰能抗數人。酒酣,胡人襪子、何懿等唱「合生」,歌言淺穢,因倨肆,欲奪司農少卿宋廷瑜賜魚。平一上書諫曰:「樂,天之和,禮,地之序;禮配地,樂應天。故音動於心,聲形於物,因心哀樂,感物應變。樂正則風化正,樂邪則政教邪,先王所以達廢興也。伏見胡樂施於聲律,本備四夷之數,比來日益流宕,異曲新聲,哀思淫溺。始自王公,稍及閭巷,妖伎胡人、街童市子,或言妃主情貌,或列王公名質,詠歌蹈舞,號曰『合生』。昔齊衰,有《行伴侶》,陳滅,有《玉樹後庭花》,趨數驚驁僻,皆亡國之音。夫禮慊而不進即銷,樂流而不反則放。臣願屏流僻,崇肅雍,凡胡樂,備四夷外,一皆罷遣。況兩儀、承慶殿者,陛下受朝聽訟之所,比大饗群臣,不容以倡優媟狎虧污邦典。若聽政之暇,茍玩耳目,自當奏之後廷可也。」不納。



 玄宗立,貶蘇州參軍,徙金壇令。平一見寵中宗,時雖宴豫,嘗因詩頌規誡,然不能卓然自引去,故被謫。既謫而名不衰。開元末,卒。孫元衡、儒衡別傳。



 李乂,字尚真,趙州房子人。少孤。年十二,工屬文,中書令薛元超曰:「是子且有海內名。」第進士、茂才異等,累調萬年尉。長安三年,詔雍州長史薛季昶選部吏才中御史者,季昶以乂聞,擢監察御史。劾奏無避。景龍初,葉靜能怙勢,乂條其奸,中宗不納。遷中書舍人、修文館學士。帝遣使江南,發在所庫貲以贖生,乂上疏以為:「江南魚鱉之利,衣食所資。江湖之生無既,而府庫之財有限,與其拯物,不如憂民。且鬻生之徒惟利所視,錢刀日至,網罟歲廣,施之一朝,營之百倍。若回所贖之貲,減方困之徭,其澤鄉矣。



 韋氏之變,詔令嚴促,多乂草定。進吏部侍郎,仍知制誥。與宋璟等同典選事,請謁不行,時人語曰:「李下無蹊徑。」改黃門侍郎,封中山郡公。制敕不便,輒駁正。貴幸有求官者,睿宗曰:「朕非有靳,顧李乂不可耳!」諫罷金仙、玉真二觀,帝雖不從,優容之。太平公主干政,欲引乂自附,乂深自拒絕。



 開元初,姚崇為紫微令,薦為侍郎,外托引重,實去其糾駁權,畏乂明切也。未幾,除刑部尚書。卒,年六十八,贈黃門監,謚曰貞。遺令薄葬,毋還鄉里。



 乂沉正方雅,識治體,時稱有宰相器。葬日,蘇頲、畢構、馬懷素往祖之,哭曰:「非公為慟而誰慟歟!」乂事兄尚一、尚貞孝謹甚,又俱以文章自名,弟兄同為一集,號《李氏花萼集》,乂所著甚多。尚一終清源尉,尚貞博州刺史。



 賈曾,河南洛陽人。父言忠,貌魁梧,事母以孝聞,補萬年主薄。護役蓬萊宮,或短其苛,高宗廷詰,辯列詳諦,帝異之,擢監察御史。方事遼東,奉使稟軍餉,還,奏上山川道里,並陳高麗可破狀。帝問:「諸將材否?」對曰:「李勍舊臣,陛下所自悉。龐同善雖非鬥將,而持軍嚴。薛仁貴票勇冠軍,高偘忠果而府,契苾何力性沈毅,雖忌前,有統御才。然夙夜小心,忘身憂國,莫逮於勣者。」帝然所許,眾亦以為知言。累轉吏部員外郎。李敬玄兼尚書,言忠尚氣,及主選,不能下,貶邵州司馬。失武懿宗意,下獄幾死,左除建州司戶參軍,卒。



 曾少有名,景雲中,為吏部員外郎。玄宗為太子,遴選宮僚,以曾為舍人。太子數遣使採女樂,就率更寺肄習,曾諫曰:「作樂崇德,以和人神。《韶》、《夏》有容,《咸》、《英》有節,而女樂不與其間。昔魯用孔子幾霸,戎有由餘而強,齊、秦遺以女樂,故孔子行,由余出奔。良以冶容哇咬,蠱心喪志,聖賢疾之最甚。殿下渴賢之美未彰,好伎之聲先聞,非所以追啟誦、嗣堯舜之烈也。餘閑宴私,後廷伎樂,古亦有之,猶當秘隱,不以示人,況閱之所司,明示群臣哉!願下令屏倡優女子,諸使者採召,一切罷止。」太子手令嘉答。



 俄擢中書舍人,以父嫌名不拜,徙諫議大夫,知制誥。天子親郊,有司議不設皇地祗位,曾請合享天地如古制並從祀等坐。睿宗詔宰相禮官議,皆如曾請。開元初,復拜中書舍人,曾固辭。議者謂中書乃曹司,非官稱,嫌名在禮不諱,乃就職。與蘇晉同掌制誥,皆以文辭稱,時號「蘇賈」。後坐事貶洋州刺史。歷虔、鄭等州刺史,遷禮部侍郎,卒。子至。



 至字幼鄰,擢明經第,解褐單父尉。從玄宗幸蜀,拜起居舍人,知制誥。帝傳位,至當譔冊,既進稿,帝曰:「昔先天誥命,乃父為之辭,今茲命冊,又爾為之,兩朝盛典,出卿家父子手,可謂繼美矣。」至頓首,鳴咽流涕。歷中書舍人。



 至德中,將軍王去榮殺富平令杜徽,肅宗新得陜,且惜去榮材,詔貸死,以流人使自效。至諫曰:「聖人誅亂,必先示法令,崇禮義。漢始入關,約法三章,殺人者死,不易之法也。按將軍去榮以朔方偏裨提數千士,不能整行列,挾私怨殺縣令,有犯上之逆。或曰去榮善守,陜新下,非去榮不可守,臣謂不然。李光弼守太原,程千里守上黨,許叔冀守靈昌,魯炅守南陽,賈賁守雍丘,張巡守睢陽,初無去榮,未聞賊能下也。以一能而免死,彼弧矢絕倫、劍術無前者,恃能犯上,何以止之!若舍去榮,誅將來,是法不一而招罪人也。惜一去榮,殺十去榮之材,其傷蓋多。彼逆亂之人,有逆於此而順於彼乎?亂富平而治於陜乎?悖縣令,能不悖於君乎?律令者,太宗之律令,陛下不可以一士小材,廢祖宗大法。」帝詔群臣議,太子太師韋見素、文部郎中崔器等皆以為:「法者,天地大典,王者不敢專也。帝王不擅殺,而小人得擅殺者,是權過人主。開元以前,無敢專殺,尊朝廷也;今有之,是弱國家也。太宗定天下,陛下復鴻業,則去榮非至德罪人,乃貞觀罪人也。其罪祖宗所不赦,陛下可易之耶?」詔可。



 蒲州刺史以河東瀕賊,徹傅城廬舍五千室,不使賊得保聚,民大擾。詔遣至慰安,官助營完,蒲人乃安。坐小法,貶岳州司馬。



 寶應初,召復故官,遷尚書左丞。楊綰建請依古制,縣令舉孝廉於刺史,刺史升天子禮部。詔有司參議,多是綰言。至議以為:「自晉後,衣冠遷徙,人多僑處,因緣官族,所在占籍。今鄉舉取人未盡,請廣學校,增國子博士員,十道大州得置大學館,詔博士領之,召置生徒。使保桑梓者,鄉里舉焉;在流寓者,庠序推焉。」議者更附至議。轉禮部侍郎,待制集賢院。



 大歷初,徙兵部。累封信都縣伯,進京兆尹。七年,以右散騎常侍卒,年五十五,贈禮部尚書,謚曰文。



 白居易,字樂天,其先蓋太原人。北齊五兵尚書建,有功於時,賜田韓城,子孫家焉。又徙下邽。父季庚,為彭城令,李正己之叛,說刺史李洧自歸,累擢襄州別駕。



 居易敏悟絕人,工文章。未冠,謁顧況。況,吳人,恃才少所推可,見其文,自失曰:「吾謂斯文遂絕,今復得子矣!」貞元中,擢進士、拔萃皆中,補校書郎。元和元年,對制策乙等,調盩厔尉,為集賢校理,月中,召入翰林為學士。遷左拾遺。



 四年,天子以旱甚,下詔有所蠲貸,振除災沴。居易見詔節未詳,即建言乞盡免江淮兩賦,以救流瘠,且多出宮人。憲宗頗採納。是時,于頔入朝,悉以歌舞人內禁中,或言普寧公主取以獻,皆頔嬖愛。居易以為不如歸之,無令頔得歸曲天子。李師道上私錢六百萬,為魏徵孫贖故第,居易言:「徵任宰相,太宗用殿材成其正寢,後嗣不能守,陛下猶宜以賢者子孫贖而賜之。師道人臣,不宜掠美。」帝從之。河東王鍔將加平章事,居易以為:「宰相天下具瞻,非有重望顯功不可任。按鍔誅求百計,不恤雕瘵,所得財號為『羨餘』以獻。今若假以名器,四方聞之,皆謂陛下得所獻,與宰相。諸節度私計曰:『誰不如鍔?』爭裒割生人以求所欲。與之則綱紀大壞,不與則有厚薄,事一失不可復追。」是時,孫以禁衛勞,擢鳳翔節度使。張奉國定徐州,平李有功,遷金吾將軍。居易為帝言:「宜罷,進奉國,以竦天下忠臣心。」度支有囚系閺鄉獄,更三赦不得原。又奏言:「父死,縶其子,夫久系,妻嫁,債無償期,禁無休日,請一切免之。」奏凡十餘上,益知名。



 會王承宗叛,帝詔吐突承璀率師出討,居易諫:「唐家制度,每征伐,專委將帥,責成功,比年始以中人為都監。韓全義討淮西,賈良國監之;高崇文討蜀,劉貞亮監之。且興天下兵,未有以中人專統領者。神策既不置行營節度,即承璀為制將,又充諸軍招討處置使,是實都統。恐四方聞之,必輕朝廷。後世且傳中人為制將自陛下始,陛下忍受此名哉?且劉濟等洎諸將必恥受承璀節制,心有不樂,無以立功。此乃資承宗之奸,挫諸將之銳。」帝不聽。既而兵老不決,居易上言:「陛下討伐,本委承璀,外則盧攸史、範希朝、張茂昭。今承璀進不決戰,已喪大將,希朝、茂昭數月乃入賊境,觀其勢,似陰相為計,空得一縣,即壁不進,理無成功。不亟罷之,且有四害。以府帑金帛、齊民膏血助河北諸侯,使益富強,一也。河北諸將聞吳少陽受命,將請洗滌承宗,章一再上,無不許,則河北合從,其勢益固。與奪恩信,不出朝廷,二也。今暑濕暴露,兵氣熏蒸,雖不顧死,孰堪其苦?又神策雜募市人,不忸於役,脫奔逃相動,諸軍必搖,三也。回鶻、吐蕃常有游偵,聞討承宗歷三時無功,則兵之強弱,費之多少,彼一知之,乘虛入寇,渠能救首尾哉?兵連事生,何故蔑有?四也。事至而罷,則損威失柄,祗可逆防,不可追悔。」亦會承宗請罪,兵遂罷。



 後對殿中,論執強鯁,帝未諭,輒進曰:「陛下誤矣。」帝變色,罷,謂李絳曰:「是子我自拔擢,乃敢爾,我叵堪此,必斥之!」絳曰:「陛下啟言者路,故群臣敢論得失。若黜之,是箝其口,使自為謀,非所以發揚盛德也。」帝悟,待之如初。歲滿當遷,帝以資淺,且家素貧,聽自擇官。居易請如姜公輔以學士兼京兆戶曹參軍,以便養,詔可。明年,以母喪解,還,拜左贊善大夫。是時,盜殺武元衡,京都震擾。居易首上疏,請亟捕賊,刷朝廷恥,以必得為期。宰相嫌其出位,不悅。俄有言:「居易母墮井死,而居易賦《新井篇》,言浮華,無實行,不可用。」出為州刺史。中書舍人王涯上言不宜治郡,追貶江州司馬。既失志,能順適所遇,托浮屠生死說,若忘形骸者。久之,徙忠州刺史。入為司門員外郎,以主客郎中知制誥。



 穆宗好畋游,獻《續虞人箴》以諷,曰:



 唐受天命,十有二聖。兢兢業業,咸勤厥政。鳥生深林,獸在豐草。春曈冬狩,取之以道。鳥獸蟲魚,各遂其生。民野君朝,亦克用寧。在昔玄祖,厥訓孔彰:「馳騁畋獵,俾心發狂。」何以效之,曰羿與康。曾不是誡,終然覆亡。高祖方獵,蘇長進言:「不滿十旬,未足為歡。」上心既悟,為之輟畋。降及宋璟,亦諫玄宗。溫顏聽納,獻替從容。璟趨以出,鷂死握中。噫!逐獸於野,走馬於路。豈不快哉,銜橛可懼。審其安危,惟聖之慮。



 俄轉中書舍人。田布拜魏博節度使,命持節宣諭,布遺五百縑,詔使受之,辭曰:「布父讎國恥未雪,人當以物助之,乃取其財,誼不忍。方諭問旁午,若悉有所贈,則賊未殄,布貲竭矣。」詔聽辭餉。是時,河朔復亂,合諸道兵出討,遷延無功。賊取弓高,絕糧道,深州圍益急。居易上言:「兵多則難用,將眾則不一。宜詔魏博、澤潞、定、滄四節度,令各守境,以省度支貲餉。每道各出銳兵三千,使李光顏將。光顏故有鳳翔、徐、滑、河陽、陳許軍無慮四萬,可徑薄賊,開弓高糧路,合下博,解深州之圍,與牛元翼合。還裴度招討使,使悉太原兵西壓境,見利乘隙夾攻之,間令招諭以動其心,未及誅夷,必自生變。且光顏久將,有威名,度為人忠勇,可當一面,無若二人者。」於是,天子荒縱,宰相才下,賞罰失所宜,坐視賊,無能為。居易雖進忠,不見聽,乃丐外遷。為杭州刺史,始築堤捍錢塘湖,鐘洩其水,溉田千頃。復浚李泌六井,民賴其汲。久之,以太子左庶子分司東都。復拜蘇州刺史,病免。



 文宗立,以秘書監召,遷刑部侍郎,封晉陽縣男。太和初,二李黨事興,險利乘之,更相奪移,進退毀譽,若旦暮然。楊虞卿與居易姻家,而善李宗閔,居易惡緣黨人斥,乃移病還東都。除太子賓客分司。逾年,即拜河南尹,復以賓客分司。開成初,起為同州刺史,不拜,改太子少傅,進馮翊縣侯。會昌初,以刑部尚書致仕。六年,卒,年七十五,贈尚書右僕射,宣宗以詩吊之。遺命薄葬,毋請謚。



 居易被遇憲宗時,事無不言,湔剔抉摩,多見聽可,然為當路所忌,遂擯斥,所蘊不能施,乃放意文酒。既復用,又皆幼君,偃蹇益不合,居官輒病去,遂無立功名意。與弟行簡、從祖弟敏中友愛。東都所居履道里,疏詔種樹,構石樓香山,鑿八節灘,自號醉吟先生,為之傳。暮節惑浮屠道尤甚,至經月不食葷,稱香山居士。嘗與胡杲、吉日又、鄭據、劉真、盧真、張渾、狄兼謨、盧貞燕集,皆高年不事者,人慕之,繪為《九老圖》。



 居易於文章精切,然最工詩。初,頗以規諷得失,及其多,更下偶俗好,至數千篇,當時士人爭傳。雞林行賈售其國相,率篇易一金,甚偽者,相輒能辯之。初,與元稹酬詠,故號「元白」;稹卒,又與劉禹錫齊名,號「劉白」。其始生七月能展書,姆指「之」、「無」兩字,雖試百數不差;九歲暗識聲律。其篤於才章,蓋天稟然。敏中為相,請謚,有司曰文。後履道第卒為佛寺。東都、江州人為立祠焉。



 贊曰:居易在元和、長慶時,與元稹俱有名,最長於詩,它文未能稱是也,多至數千篇,唐以來所未有。其自敘言:「關美刺者,謂之諷諭;詠性情者,謂之閑適;觸事而發,謂之感傷;其它為雜律。」又譏「世人所愛惟雜律詩,彼所重,我所輕。至諷諭意激而言質,閑適思澹而辭迂,以質合迂,宜人之不愛也」。今視其文,信然。而杜牧謂:「纖艷不逞,非莊士雅人所為。流傳人間,子父女母交口教授,淫言媟語入人肌骨不可去。」蓋救所失不得不云。觀居易始以直道奮,在天子前爭安危,冀以立功,雖中被斥,晚益不衰。當宗閔時,權勢震赫,終不附離為進取計,完節自高。而稹中道徼險得宰相,名望漼然。鳴呼,居易其賢哉!



 行簡,字知退,擢進士,闢盧坦劍南東川府。罷,與居易自忠州入朝,授左拾遺。累遷主客員外郎,代韋詞判度支按,進郎中。長慶時,振武營田使賀拔志歲終結課最,詔行簡閱實,發其妄,志懼,自刺不殊。行簡敏而有辭,後學所慕尚。寶歷二年卒。



 敏中,字用晦,少孤,承學諸兄。長慶初,第進士,闢義成節度使李聽府,聽一見,許其遠到。遷右拾遺,改殿中侍御史,為符澈邠寧副使,澈卒以能政聞。御史中丞高元裕薦為侍御史,再轉左司員外郎。武宗雅聞居易名,欲召用之。是時,居易足病廢,宰相李德裕言其衰苶不任事,即薦敏中文詞類其兄而有器識。即日知制誥,召入翰林為學士。進承旨。宣宗立,以兵部侍郎同中書門下平章事,遷中書侍郎,兼刑部尚書。德裕貶,敏中抵之甚力,議者訾惡。德裕著書亦言「惟以怨報德為不可測」,蓋斥敏中云。歷尚書右僕射、門下侍郎,封太原郡公。自員外,凡五年,十三遷。



 崔鉉輔政,欲專任,患敏中居右。會黨項數寇邊,鉉言宜得大臣鎮撫,天子響其言,故敏中以司空、平章事兼邠寧節度、招撫、制置使。初,帝愛萬壽公主,欲下嫁士人。時鄭顥擢進士第,有閥閱,敏中以充選。顥與盧氏婚,將授室而罷,銜之。敏中自以居外,畏顥讒,自訴於帝。帝曰:「朕知久矣。若用顥言,庸相任耶?」顧左右取書一函,發視,悉顥所上,敏中乃安。及行,帝御安福樓以餞,頒璽書諭尉,賜通天帶,衛以神策兵,開府闢士,禮如裴度討淮西時。次寧州,諸將已破羌賊,敏中即說諭其眾,皆願棄兵為業。乃自南山並河按堡保,回繞千里。又規蕭關通靈威路,使為耕戰具。逾年,檢校司徒,徒劍南西川,增騾軍,完創關壁。治蜀五年,有勞,加兼太子太師,徙荊南。



 懿宗立,召拜司徒、門下侍郎,還平章事。數月足病不任謁,固求避位,不許,中使者勞問,俾對別殿,毋拜。右補闕王譜奏言:「敏中病四月,陛下坐朝,與他宰相語不三刻,安暇論天下事?願聽其請,無使有持寵曠貴之譏。」書聞,帝怒,斥譜陽翟令。給事中鄭公輿申救,不聽。譜者,侍中珪之遠裔。未幾,加敏中中書令。自裴度以勛德居,而敏中以恩澤進。



 咸通二年,南蠻擾邊,召敏中入議,許挾扶升殿。固求免,乃出為鳳翔節度使。三奏願歸守墳墓,除東都留守,不敢拜,許以太傅致仕。詔書未至,卒,冊贈太尉。博士曹鄴責其病不堅退,且逐諫臣,舉怙威肆行,謚曰醜。



\end{pinyinscope}