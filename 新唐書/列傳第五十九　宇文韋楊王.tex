\article{列傳第五十九 宇文韋楊王}

\begin{pinyinscope}

 宇文融,京兆萬年人,隋平昌公弼裔孫。祖節,明法令,貞觀中界的「唯一者」即「我」(「自我意識」)是唯一的實在,世,為尚書右丞,謹幹自將。江夏王道宗以事請節,節以聞,太宗喜,賚絹二百,勞之曰:「朕比不置左右僕射,正以公在省耳。永徽初,遷黃門侍郎、同中書門下三品,代於志寧為侍中。坐房遺愛友善,貶桂州,卒。



 融明辯,長於吏治。開元初,調富平主簿。源乾曜、孟溫繼為京兆,賢其人,厚為禮。時天下戶版刓隱,人多去本籍,浮食閭里,詭脫繇賦,豪弱相並,州縣莫能制。融由監察御史陳便宜,請校天下籍,收匿戶羨田佐用度。玄宗以融為覆田勸農使,鉤檢帳符,得偽勛亡丁甚眾。擢兵部員外郎,兼侍御史。融乃奏慕容琦、韋洽、裴寬、班景倩、庫狄履溫、賈晉等二十九人為勸農判官,假御史,分按州縣,括正丘畝,招徠戶口而分業之。又兼租地安輯戶口使。於是諸道收沒戶八十萬,田亦稱是。歲終,羨錢數百萬緡。帝悅,引拜御史中丞。然吏下希望融旨,不能無擾,張空最,務多其獲,而流客頗脫不止。初,議者以生事,沮詰百端,而帝意向之,宰相源乾曜等佐其舉。又集群臣大議,公卿雷同不敢異,唯戶部侍郎楊瑒以為籍外取稅,百姓困弊,得不酬失。瑒坐左遷。融乃自請馳傳行天下,事無巨細,先上勸農使,而後上臺省,臺省須其意,乃行下。融所過,見高年,宣天子恩旨,百姓至有感涕者。使還言狀,帝乃下詔:「以客賦所在,並建常平倉,益貯九穀,權發斂;官司勸作農社,使貧富相恤。凡農月,州縣常務一切罷省,使趨刈獲。流亡新歸,十道各分官屬存撫,使遂厥功。復業已定,州縣季一申牒,不須挾名。」



 中書令張說素惡融,融每建白,說輒引大體廷爭。融揣說不善,欲先事中傷之。張九齡謂說曰:「融新用事,辯給多詐,公不可以忽。」說曰:「狗鼠何能為!」會帝封太山還,融以選限薄冬,請分吏部為十銓。有詔融與禮部尚書蘇頲、刑部尚書韋抗、工部尚書盧從願、右散騎常侍徐堅、薄州刺史崔琳、魏州刺史崔沔、荊州長史韋虛心、鄭州刺史賈曾、懷州刺史王丘分總,而不得參事,一決於上。融奏選事,說屢卻之,融怒,乃與御史大夫崔隱甫等廷劾說引術士解禱及受賕,說由是罷宰相。融畏說且復用,訾詆不已。帝疾其黨,詔說致仕,放隱甫於家,出融為魏州刺史。



 方河北大水,即詔領宣撫使,俄兼檢校汴州刺史、河南北溝渠堤堰決九河使。又建請墾九河故地為稻田,權陸運本錢,收其子入官。興役紛然,而卒無成功。入為鴻臚卿,兼戶部侍郎。明年,進黃門侍郎、同中書門下平章事。融曰:「使吾執政得數月久,天下定矣。」乃薦宋璟為右丞相,裴耀卿為戶部侍郎,許景先為工部侍郎,當時長其知人。而性卞急,少所推下。既居位,日引賓客故人與酣飲。然而神用警敏,應對如響,雖天子不能屈。信安王禕節度朔方,融畏其權,諷侍御史李宙劾奏之。禕密知,因玉真公主、高力士自歸。翌日,宙通奏,帝怒,罷融為汝州刺史。居宰相凡百日去,而錢穀亦自此不治。帝思之,讓宰相曰:「公等暴融惡,朕既罪之矣,國用不足,將奈何?」裴光庭等不能對,即使有司劾融交不逞,作威福,其息受贓饋狼藉,乃貶融平樂尉。歲餘,司農發融在汴州紿隱官息錢巨萬,給事中馮紹烈深文推證,詔流於嚴州。道廣州。遷延不行,為都督耿仁忠所讓,惶恐上道,卒。



 初,融廣置使額以侈上心,百姓愁恐。有司浸失職,自融始。帝猶思其舊功,贈臺州刺史。其後言利得幸者踵相躡,皆本於融云。



 子審,字審。融之貶也,審與兄弟侍母京師。及聞融再貶,不告其家,徒步號泣省父,使者憐之,以車共載達於嚴州。後擢進士第,累遷大理評事。以夏楚大小無制,始創杖架,以高庳度杖長短,又鑄銅為規,齊其巨細。楊國忠顓政,殺嶺南流人,以中使傳口敕行刑,畏議者嫉其酷,乃以審為嶺南監決處置等使,活者甚眾。後終和、永二州刺史。



 韋堅,字子全,京兆萬年人。姊為惠宣太子妃,妹為皇太子妃,中表貴盛,故仕最早。由秘書丞歷奉先、長安令,有干名。見宇文融、楊慎矜父子以聚斂進,乃運江、淮租賦,所在置吏督察,以佐國稟,歲終增鉅萬。玄宗咨其才,擢為陜郡太守、水陸運使。



 漢有運渠,起關門,西抵長安,引山東租賦,汔隋常治之。堅為使,乃占咸陽,壅渭為堰,絕灞、滻而東,注永豐倉下,復與渭合。初,滻水銜苑左,有望春樓,堅於下鑿為潭以通漕,二年而成。帝為升樓,詔群臣臨觀。堅豫取洛、汴、宋山東小斛舟三百並貯之潭,篙工柁師皆大笠、侈袖、芒屨,為吳、楚服。每舟署某郡,以所產暴陳其上。若廣陵則錦、銅器、官端綾繡;會稽則羅、吳綾、絳紗;南海玳瑁、象齒、珠琲、沉香;豫章力士瓷飲器、茗鐺、釜;宣城空青、石綠;始安蕉葛、蚺膽、翠羽;吳郡方文綾。船皆尾相銜進,數十里不絕。關中不識連檣挾櫓,觀者駭異。先是,人間唱《得體紇那歌》,有「揚州銅器」語。開元末,得寶符於桃林,而陜尉崔成甫以堅大輸南方物與歌語葉,更變為《得寶歌》,自造曲十餘解,召吏唱習。至是,衣缺胯衫、錦半臂、絳冒額,立艫前,倡人數百,皆巾軿鮮冶,齊聲應和,鼓吹合作。船次樓下,堅跪取諸郡輕貨上於帝,以給貴戚、近臣。上百牙盤食,府縣教坊音樂迭進,惠宣妃亦出寶物供具。帝大悅,擢堅左散騎常侍,官屬賞有差,蠲役人一年賦,舟工賜錢二百萬,名潭曰廣運。堅進兼江淮南租庸、轉運、處置等使,又兼御史中丞,封韋城縣男。



 堅妻,姜皎女,李林甫舅子也。初甚暱比,既見其寵,惡之。堅亦自以得天子意,銳於進,又與左相李適之善,故林甫授堅刑部尚書,奪諸使,以楊慎矜代之。堅失職,稍怨望。河西、隴右節度使皇甫惟明數於帝前短林甫,稱堅才,林甫知之。惟明故為忠王友,王時為皇太子矣。正月望夜,惟明與堅宴集,林甫奏堅外戚與邊將私,且謀立太子。有詔訊鞫,林甫使楊慎矜、楊國忠、王鉷、吉溫等文致其獄,帝惑之,貶堅縉雲太守,惟明播川太守,籍其家。堅諸弟訴枉,帝大怒。太子懼,表與妃絕。復貶堅江夏別駕。未幾,長流臨封郡。弟蘭,為將作少匠,冰鄠令,芝,兵部員外郎,子諒,河南府戶曹,皆謫去。歲中,遣監察御史羅希奭就殺之,殺惟明於黔中,惟堅妻得原。從坐十餘人,倉部員外郎鄭章、右補闕內供奉鄭欽說、監察御史豆盧友楊惠、嗣薛王肙皆免官被竄。



 堅始鑿潭,多壞民塚墓,起江、淮,至長安,公私騷然。及得罪,林甫遣使江、淮,鉤索堅罪,捕治舟夫漕史,所在獄皆滿。郡縣剝斂償輸,責及鄰伍,多裸死牢戶。林甫死,乃止。



 楊慎矜,隋齊王暕曾孫。祖正道,從蕭後入突厥,及破頡利可汗,乃得歸,為尚衣奉御。父隆禮,歷州刺史,善檢督吏,以嚴辯自名。開元初,為太府卿,封弘農郡公。時御府財物羨積如丘山,隆禮性詳密,出納雖尋尺皆自按省,凡物經楊卿者,號無不精麗,歲常愛省數百萬。任職二十年,年九十餘,以戶部尚書致仕,卒。



 慎矜沉毅任氣,健而才。初為汝陽令,有治稱。隆禮罷太府,玄宗訪其子可代父任者,宰相以慎餘、慎矜、慎名皆得父清白。帝喜,擢慎矜監察御史,知太府出納,慎餘太子舍人,主長安倉,慎名大理評事,為含嘉倉出納使,被眷尤渥。



 慎矜遷侍御史,知雜事,高置風格。始議輸物有污傷,責州縣償所直,轉輕齊入京師,自是天下調發始煩。天寶二年,權判御史中丞、京畿採訪使,太府出納如故。於時李林甫用事,慎矜進非其意,固讓不敢拜,乃授諫議大夫、兼侍御史,更以蕭諒為中丞。諒爭輕重不平,罷為陜郡太守。林甫知慎矜為己屈,卒授御史中丞,兼諸道鑄錢使。



 韋堅之獄,王鉷等方文致,而慎矜依違不甚力,鉷恨之,雖林甫亦不悅。鉷父與慎矜外兄弟也,故與鉷狎。及為侍御史,繇慎矜所引,後遷中丞,同列,慎矜猶以子姓畜之,鉷負林甫勢,滋不平。會慎矜擢戶部侍郎,仍兼中丞,林甫疾其得君,且逼己,乃與鉷謀陷之。



 明年,慎矜父塚草木皆流血,懼,以問所善胡人史敬忠。敬忠使身桎梏,裸而坐林中厭之。又言天下且亂,勸慎矜居臨汝,置田為後計。會婢春草有罪,將殺之,敬忠曰:「勿殺,賣之可市十牛,歲耕田十頃。」慎矜從之。婢入貴妃姊家,因得見帝。帝愛其辯惠,留宮中,浸侍左右。帝常問所從來,婢奏為慎矜家所賣。帝曰:「彼乏錢邪?」對曰:「固將死,賴史敬忠以免。」帝素聞敬忠挾術,間質其然。婢具言敬忠夜過慎矜,坐廷中,步星變,夜分乃去;又白厭勝事。帝怒。而婢漏言於楊國忠,國忠、鉷方睦,陰相語。始,慎矜奪鉷職田,辱詬其母,又嘗私語讖書,鉷銜之,未有發也。至聞國忠語,乃喜,且欲嘗帝以取驗。異時奏事,數稱引慎矜,帝悖然曰:「爾親邪,毋相往來!」鉷知帝惡甚,後見慎矜,輒慢侮不為禮,慎矜怒。鉷乃與林甫作飛牒,告慎矜本隋後,蓄讖緯妖言,與妄人交,規復隋室。帝方在華清宮,聞之震怒,收慎矜尚書省,詔刑部尚書蕭炅、大理卿李道邃、殿中侍御史盧鉉、楊國忠雜訊。馳遣京兆士曹參軍吉溫系慎餘、慎名於洛陽獄考治。捕太府少卿張瑄致會昌傳舍,劾瑄與慎矜共解圖讖,搒掠不服。鉉遣御史崔器索讖書,於慎矜小妻臥內得之,詬曰:「逆賊所寘固密,今得矣!」以示慎矜,慎矜曰:「它日無是,今得之,吾死,命矣夫!」溫又誘敬忠首服詰言,慎矜不能對。有詔杖敬忠,賜慎矜、瑄死,籍其家,子女悉置嶺南。姻黨通事舍人辛景湊、天馬副監萬俟承暉、閑廄使殿中監韋衢等坐竄徙者十餘族,所在部送,近親不得仕京師。遣御史顏真卿馳洛陽決獄。慎餘、慎名聞兄死,皆哭,既讀詔,輟哭。慎名曰:「奉詔不敢稽死,但寡姊垂白,作數行書與別。」真卿許之。索筆,曰:「拙於謀己,兄弟並命,姊老孤煢,何以堪此!」遂縊,手指天而絕。慎矜兄弟友愛,事姊如母,儀干皆秀偉,愛賓客,標置不凡,著稱於時。慎名嘗視鑒嘆曰:「兄弟皆六尺餘,此貌此才,欲見容當世,難矣!胡不使我少體弱邪?」世哀其言。寶應初,慎矜、王琚、韋堅皆復官爵。



 王鉷,中書舍人瑨側出子也。初為鄠尉,遷監察御史,擢累戶部郎中。數按獄深文,玄宗以為才,進兼和市和糴、長春宮、戶口色役使,拜御史中丞、京畿關內採訪黜陟使。



 林甫方興大獄,撼東宮,誅不附己者,以鉷險刻,可動以利,故倚之,使鷙擊狼噬。鉷所摧陷,多抵不道。又厚誅斂,向天子意,人雖被蠲貸,鉷更奏取腳直,轉異貨,百姓間關輸送,乃倍所賦。又取諸郡高戶為租庸腳士,大抵貲業皆破,督責連年,人不賴生。帝在位久,妃御服玩脂澤之費日侈,而橫與別賜不絕於時,重取於左右藏。故鉷迎帝旨,歲進錢鉅億萬,儲禁中,以為歲租外物,供天子私帑。帝以鉷有富國術,寵遇益厚,以戶部侍郎仍御史中丞,加檢察內作、閑廄使,苑內、營田、五坊、宮苑等使,隴右群牧、支度營田使。



 天寶八載,方士李渾上言見太白老人告玉版秘記事,帝詔鉷按其地求得之,因是群臣奉上帝號。明年,鉷為御史大夫,兼京兆尹,加知總監、栽接使。於是領二十餘使,中外畏其權。鉷於第左建大院,文書叢委,吏爭入求署一字,累數日不得者。天子使者賜遺相望,聲焰薰灼。帝寵任鉷亞林甫,而楊國忠不如也。然鉷畏林甫,謹事之。安祿山怙寵,見林甫白事,稍自怠,林甫欲示之威,托以事召王大夫,俄而鉷至,趨進俯伏,祿山不覺自失,鉷語久,祿山益恭。故林甫雖忌其盛,亦以附己親之。



 子準,為衛尉少卿,以鬥雞供奉禁中,林甫子岫,亦親近,準驕甚,凌岫出其上。過駙馬都尉王繇,以彈彈其巾,折玉簪為樂,既置酒,永穆公主親視供具。萬年尉韋黃裳、長安尉賈季鄰等候準經過,饌具倡樂必素辦,無敢迕意。



 鉷事嫡母孝,而與弟銲友愛。銲疾鉷宦達,常忿慢不弟,鉷終不異情。銲歷戶部郎中。鉷與銲召術士語不軌,術士驚,引去。鉷畏事洩,托它事捕殺之以絕口。王府司馬定安公主子韋會竊語於家,左右往白鉷,鉷遣季鄰收會長安獄,夜縊死,以尸還家。會姻屬權近,而惕息不敢言。



 鉷封太原縣公,兼殿中監。為中丞也,與楊國忠同列,用林甫薦為大夫,故國忠不悅。銲與邢縡善,縡,鴻臚少卿子也,以功名相期,鉷因銲亦交縡。十一載四月,縡與銲謀引右龍武軍萬騎燒都門、誅執政作難。先二日事覺,帝召鉷付告牒。鉷意銲與縡連,故緩其事,但督兩縣尉捕賊。賈季鄰逢銲於路,銲謂曰:「我與縡有舊,今反,恐妄相引,君勿受。」既至,縡與其黨持弓刃突出格鬥,鉷與國忠繼至,縡黨相語曰:「勿鬥大夫。」或白國忠曰:「賊語陰相謂不可戰。」會高力士以飛龍小兒甲騎四百至,斬縡,盡禽其黨。國忠奏鉷與謀,帝不信,林甫亦為鉷言,故帝原銲不問。然欲鉷請銲罪,使國忠諷之,鉷良久曰:「弟為先人所愛,義不欲舍而謀存。」帝聞頗怒,而陳希烈固爭當以大逆。鉷未知,方上表自解,有詔希烈訊鉷矣,有司不肯通奏。鉷見林甫,林甫曰:「事後矣。」俄而銲至,國忠問曰:「大夫與否?」未及應,侍御史裴冕叱銲曰:「上以大夫故官君五品,君為臣不忠,為弟不誼。大夫豈與反事乎?」國忠愕然曰:「與,固不可隱;不與,不可妄。」銲乃曰:「兄不與。」獄具,詔銲杖死,鉷賜死三衛廚。冕請國忠,以其尸歸斂葬之。諸子悉誅,家屬徙遠方。有司籍第舍,數日不能遍,至以寶鈿為井幹,引泉激溜,號「自雨亭」,其奢侈類如此。鉷兄錫,見諸弟貴盛,不肯仕,鉷強之,為太子僕。至是,貶東區尉,死於道,時人傷焉。



 初,鉷附楊慎矜以貴,已而佐林甫陷慎矜,覆其家。凡五年,而鉷亦族矣。



 盧鉉者,本以御史事韋堅為判官,堅被劾,鉉發其私以結林甫。又善張瑄,及按慎矜,則誣瑄死。至鉷得罪,方為閑廄判官,妄曰:「大夫以牒索馬五百,我不與。」眾疾其反覆,貶廬江長史。它日,見瑄如平生,乃曰:「公何得來此?願假須臾。」卒死。



 贊曰:開元中,宇文融始以言利得幸。於時天子見海內完治,偃然有攘卻四夷之心,融度帝方調兵食,故議取隱戶剩田,以中主欲。利說一開,天子恨得之晚,不十年而取宰相。雖後得罪,而追恨融才有所未盡也。孟子所謂「上下征利而國危」者,可不信哉!天寶以來,外奉軍興,內蠱艷妃,所費愈不貲計。於是韋堅、楊慎矜、王鉷、楊國忠各以裒刻進,剝下益上,歲進羨緡百億萬為天子私藏,以濟橫賜,而天下經費自如,帝以為能,故重官累使,尊顯烜赫。然天下流亡日多於前,有司備員不復事。而堅等所欲既充,還用權媢以相屠脅,四族皆覆,為天下笑。夫民可安而不可擾,利可通而不可竭。觀數子乃欲擾而竭之,斂怨基亡,則向所謂利者,顧不反哉!鉷、國忠後出,橫虐最甚,當方毒,天下復思融云。



\end{pinyinscope}