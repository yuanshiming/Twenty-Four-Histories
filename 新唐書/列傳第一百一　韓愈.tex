\article{列傳第一百一 韓愈}

\begin{pinyinscope}

 韓愈,字退之,鄧州南陽人。七世祖茂,有功於後魏,封安定王。父仲卿隱喻象征雙重形式表達。尼德蘭哲學家西格爾(Sigerusde,為武昌令,有美政,既去,縣人刻石頌德。終秘書郎。愈生三歲而孤,隨伯兄會貶官嶺表。會卒,嫂鄭鞠之。愈自知讀書,日記數千百言,比長,盡能通《六經》、百家學。擢進士第。會董晉為宣武節度使,表署觀察推官。晉卒,愈從喪出,不四日,汴軍亂,乃去。依武寧節度使張建封,建封闢府推官。操行堅正,鯁言無所忌。調四門博士,遷監察御史。上疏極論宮市,德宗怒,貶陽山令。有愛在民,民生子多以其姓字之。改江陵法曹參軍。元和初,權知國子博士,分司東都,三歲為真。改都官員外郎,即拜河南令。遷職方員外郎。



 華陰令柳澗有罪,前刺史劾奏之,未報而刺史罷。澗諷百姓遮索軍頓役直,後刺史惡之,按其獄,貶澗房州司馬。愈過華,以為刺史陰相黨,上疏治之。既御史覆問,得澗贓,再貶封溪尉。愈坐是復為博士。既才高數黜,官又下遷,乃作《進學解》以自諭曰:



 國子先生晨入太學,召諸生立館下,誨之曰:「業精於勤,荒於嬉;行成於思,毀於隨。方今聖賢相逢,治具畢張,拔去兇邪,登崇畯良。占小善者率以錄,名一藝者無不庸。爬羅剔抉,刮垢磨光。蓋有幸而獲選,孰雲多而不揚?諸生業患不能精,無患有司之不明;行患不能成,無患有司之不公。」



 言未既,有笑於列者曰:「先生欺予哉!弟子事先生,於茲有年矣。先生口不絕吟於六藝之文,手不停披於百家之編。記事者必提其要,纂言者必鉤其玄。貪多務得,細大不捐。燒膏油以繼晷,常矻矻以窮年。先生之業,可謂勤矣。牴排異端,攘斥佛老。補苴罅漏,張皇幽眇。尋墜緒之芒芒,獨旁搜而遠紹。停百川而東之,回狂瀾於既倒。先生之於儒,可謂有勞矣。沈浸濃鬱,含英咀華。作為文章,其書滿家。上規姚姒,渾渾亡涯。周《誥》商《盤》,佶屈聱牙。《春秋》謹嚴,《左氏》浮誇。《易》奇而法,《詩》正而葩。下迨《莊》《騷》,太史所錄,子雲相如,同工異曲。先生之於文,可謂閎其中而肆其外矣。少始知學,勇於敢為。長通於方,左右具宜。先生之於為人,可謂成矣。然而公不見信於人,私不見助於友。跋前躓後,動輒得咎。暫為御史,遂竄南夷。三年博士,冗不見治。命與仇謀,取敗幾時?冬暖而兒號寒,年豐而妻啼饑。頭童齒豁,竟死何裨?不知慮此,而反教人為?」



 先生曰:「籲!子來前。夫大木為杗,細木為桷,欂櫨侏儒,椳闑磺楔,各得其所,施以成室者,匠氏之工也。玉札丹砂,赤箭青芝,牛溲馬勃,敗鼓之皮,俱收並蓄,待用無遺者,醫師之良也。登明選公,雜進巧拙,紆餘為妍,卓犖為傑,校短量長,唯器是適者,宰相之方也。昔者孟軻好辯,孔道以明;轍環天下,卒老於行。荀卿宗王,大倫以興;逃讒於楚,廢死蘭陵。是二儒者,吐詞為經,舉足為法,絕類離倫,優入聖域,其遇於世何如也?今先生學雖勤而不由其統,言雖多而不要其中;文雖奇而不濟於用,行雖修而不顯於眾。猶且月費俸錢,歲靡稟粟,子不知耕,婦不知織;乘馬從徒,安坐而食;踵常途之促促,窺陳編以盜竊。然而聖主不加誅,宰臣不見斥。茲非其幸歟?動而得謗,名亦隨之。投閑置散,乃分之宜。若夫商財賄之有無,計班資之崇庳,忘量己之所稱,指前人之瑕疵,是所謂詰匠氏之不以杙為楹,而訾醫師以昌陽引年,欲進其豨苓也。」



 執政覽之,奇其才,改比部郎中、史館修撰。轉考功,知制誥,進中書舍人。



 初,憲宗將平蔡,命御史中丞裴度使諸軍按視。及還,且言賊可滅,與宰相議不合。愈亦奏言:



 淮西連年脩器械防守,金帛糧畜耗於給賞,執兵之卒四向侵掠,農夫織婦餉於其後,得不償費。比聞畜馬皆上槽櫪,此譬有十夫之力,自朝抵夕,跳躍叫呼,勢不支久,必自委頓。當其已衰,三尺童子可制其命。況以三州殘弊困劇之餘而當天下全力,其敗可立而待也,然未可知者,在陛下斷與不斷耳。夫兵不多不足以取勝,必勝之師利在速戰,兵多而戰不速則所費必廣。疆場之上,日相攻劫,近賊州縣,賦役百端,小遇水旱,百姓愁苦。方此時,人人異議以惑陛下,陛下持之不堅,半塗而罷,傷威損費,為弊必深。所要先決於心,詳度本末,事至不惑,乃可圖功。



 又言:「諸道兵羈旅單弱不足用,而界賊州縣,百姓習戰鬥,知賊深淺,若募以內軍,教不三月,一切可用。」又欲「四道置兵,道率三萬,畜力伺利,一日俱縱,則蔡首尾不救,可以責功」。執政不喜。會有人詆愈在江陵時為裴均所厚,均子鍔素無狀,愈為文章,字命鍔謗語囂暴,由是改太子右庶子。及度以宰相節度彰義軍,宣慰淮西,奏愈行軍司馬。愈請乘遽先入汴,說韓弘使葉力。元濟平,遷刑部侍郎。



 憲宗遣使者往鳳翔迎佛骨入禁中,三日,乃送佛祠。王公士人奔走膜唄,至為夷法,灼體膚,委珍貝,騰沓系路。愈聞惡之,乃上表曰:



 佛者,夷狄之一法耳。自後漢時始入中國,上古未嘗有也。昔黃帝在位百年,年百一十歲;少昊在位八十年,年百歲;顓頊在位七十九年,年九十歲;帝嚳在位七十年,年百五歲;堯在位九十八年,年百一十八歲;帝舜在位及禹年皆百歲。此時天下太平,百姓安樂壽考,然而中國未有佛也。其後,湯亦年百歲,湯孫太戊在位七十五年,武丁在位五十年,書史不言其壽,推其年數,蓋不減百歲。周文王年九十七歲,武王年九十三歲,穆王在位百年。此時佛法亦未至中國,非因事佛而致然也。漢明帝時始有佛法,明帝在位才十八年。其後亂亡相繼,運祚不長。宋、齊、梁、陳、元魏以下,事佛漸謹,年代尤促。唯梁武帝在位四十八年,前後三舍身施佛,宗廟祭不用牲牢,晝日一食,止於菜果,後為侯景所逼,餓死臺城,國亦尋滅。事佛求福,乃更得禍。由此觀之,佛不足信,亦可知矣。



 高祖始受隋禪,則議除之。當時君臣識見不遠,不能深究先王之道、古今之宜,推闡聖明,以救斯弊,其事遂止。臣常恨焉!伏惟睿聖文武皇帝陛下,神聖英武,數千百年以來,未有倫比。即位之初,即不許度人為僧尼、道士,又不許別立寺觀。臣當時以為高祖之志,必行於陛下。今縱未能即行,豈可恣之令盛也!今陛下令群僧迎佛骨於鳳翔,御樓以觀,舁入大內,又令諸寺遞加供養。臣雖至愚,必知陛下不惑於佛,作此崇奉以祈福祥也。直以豐年之樂,徇人之心,為京都士庶設詭異之觀、戲玩之具耳。安有聖明若此,而肯信此等事哉?然百姓愚冥,易惑難曉,茍見陛下如此,將謂真心信佛,皆云:「天子大聖,猶一心信向;百姓微賤,於佛豈合更惜身命?」以至灼頂燔指,十百為群,解衣散錢,自朝至暮,轉相仿效,唯恐後時,老幼奔波,棄其生業。若不即加禁遏,更歷諸寺,必有斷臂臠身以為供養者。傷風敗俗,傳笑四方,非細事也。



 佛本夷狄之人,與中國言語不通,衣服殊制;口不道先王之法言,身不服先王之法服,不知君臣之義、父子之情。假如其身尚在,奉其國命來朝京師,陛下容而接之,不過宣政一見,禮賓一設,賜衣一襲,衛而出之於境,不令貳於眾也。況其身死已久,枯朽之骨,兇穢之餘,豈宜以入宮禁?孔子曰:「敬鬼神而遠之。」古之諸侯吊於其國,必令巫祝先以桃茢祓除不祥,然後進吊。今無故取朽穢之物,親臨觀之,巫祝不先,桃茢不用,君臣不言其非,御史不舉其失,臣實恥之。乞以此骨付之水火,永絕根本,斷天下之疑,絕前代之惑,使天下之人知大聖人之所作為,出於尋常萬萬也。佛如有靈,能作禍祟,凡有殃咎,宜加臣身。上天鑒臨,臣不怨悔。



 表入,帝大怒,持示宰相,將抵以死。裴度、崔群曰:「愈言訐牾,罪之誠宜。然非內懷至忠,安能及此?願少寬假,以來諫爭。」帝曰:「愈言我奉佛太過,猶可容;至謂東漢奉佛以後,天子感夭促,言何乖剌邪?愈,人臣,狂妄敢爾,固不可赦!」於是中外駭懼,雖戚里諸貴,亦為愈言,乃貶潮州刺史。



 既至潮,以表哀謝曰:



 臣以狂妄戇愚,不識禮度,陳佛骨事,言涉不恭,正名定罪,萬死莫塞。陛下哀臣愚忠,恕臣狂直,謂言雖可罪,心亦無他,特屈刑章,以臣為潮州刺史。既免刑誅,又獲祿食,聖恩寬大,天地莫量,破腦刳心,豈足為謝!



 臣所領州,在廣府極東,過海口,下惡水,濤瀧壯猛,難計期程,颶風鱷魚,患禍不測。州南近界,漲海連天,毒霧瘴氛,日夕發作。臣少多病,年才五十,發白齒落,理不久長。加以罪犯至重,所處遠惡,憂惶慚悸,死亡無日。單立一身,朝無親黨,居蠻夷之地,與魑魅同群,茍非陛下哀而念之,誰肯為臣言者?



 臣受性愚陋,人事多所不通,惟酷好學問文章,未嘗一日暫廢,實為時輩所見推許。臣於當時之文,亦未有過人者。至於論述陛下功德,與《詩》、《書》相表裏,作為歌詩,薦之郊廟,紀太山之封,鏤白玉之牒,鋪張對天之宏休,揚厲無前之偉績,編於《詩》、《書》之策而無愧,措於天地之間而無虧,雖使古人復生,臣未肯讓。



 伏以皇唐受命有天下,四海之內,莫不臣妾,南北東西,地各萬里。自天寶以後,政治少懈,文致未優,武克不剛,孽臣奸隸,蠹居棋處,搖毒自防,外順內悖,父死子代,以祖以孫,如古諸侯,自擅其地,不朝不貢,六七十年。四聖傳序,以至陛下。陛下即位以來,躬親聽斷,旋乾轉坤,關機闔開,雷厲風飛,日月清照,天戈所麾,無不從順。宜定樂章,以告神明,東巡泰山,奏功皇天,具著顯庸,明示得意,使永永年服我成烈。當此之際,所謂千載一時不可逢之嘉會,而臣負罪嬰釁,自拘海島,戚戚嗟嗟,日與死迫,曾不得奏薄伎於從官之內、隸御之間,窮思畢精,以贖前過。懷痛窮天,死不閉目,伏惟陛下天地父母,哀而憐之。



 帝得表,頗感悔,欲復用之,持示宰相曰:「愈前所論是大愛朕,然不當言天子事佛乃年促耳。」皇甫鎛素忌愈直,即奏言:「愈終狂疏,可且內移。」乃改袁州刺史。初,愈至潮州,問民疾苦,皆曰:「惡溪有鱷魚,食民畜產且盡,民以是窮。」數日,愈自往視之,令其屬秦濟以一羊一豚投溪水而祝之曰:



 昔先王既有天下,列山澤,罔繩擉刃以除蟲蛇惡物為民物害者,驅而出之四海之外。及德薄,不能遠有,則江、漢之間尚皆棄之以與蠻夷楚越,況湖、嶺之間去京師萬里哉?鱷魚之涵淹卵育於此,亦固其所。



 今天子嗣唐位,神聖慈武,四海之外,六合之內,皆撫而有之,況禹跡所掩,揚州之近地,刺史縣令之所治,出貢賦以供天地、宗廟、百神之祀之壤者哉?鱷魚其不可與刺史雜處此土也。刺史受天子命,守此土,治此民,而鱷魚旰然不安溪潭據處,食民畜熊豕鹿麞以肥其身,以種其子孫,與刺史拒爭為長雄。刺史雖駑弱,亦安肯為鱷魚低首下心,伈々睍斯,為吏民羞,以偷活於此也?承天子命以來為吏,固其勢不得不與鱷魚辨。鱷魚有知,其聽刺史。



 潮之州,大海在其南,鯨鵬之大,蝦蟹之細,無不容歸,以生以食,鱷魚朝發而夕至也。今與鱷魚約:「盡三日,其率醜類南徙於海,以避天子之命吏。三日不能,至五日;五日不能,至七日,七日不能,是終不肯徙也,是不有刺史、聽從其言也。不然,則是鱷魚冥頑不靈,刺史雖有言,不聞不知也。夫傲天子之命吏,不聽其言,不徙以避之,與頑不靈而為民物害者,皆可殺。刺史則選材技民,操強弓毒矢,以與鱷魚從事,必盡殺乃止,其無悔!」



 祝之夕,暴風震電起溪中,數日水盡涸,西徙六十里。自是潮無鱷魚患。袁人以男女為隸,過期不贖,則沒入之。愈至,悉計庸得贖所沒,歸之父母七百餘人。因與約,禁其為隸。召拜國子祭酒,轉兵部侍郎。



 鎮州亂,殺田弘正而立王廷湊,詔愈宣撫。既行,眾皆危之。元稹言:「韓愈可惜。」穆宗亦悔,詔愈度事從宜,無必入。愈至,廷湊嚴兵迓之,甲士陳廷。既坐,廷湊曰:「所以紛紛者,乃此士卒也。」愈大聲曰;「天子以公為有將帥材,故賜以節,豈意同賊反邪?」語未終,士前奮曰:「先太師為國擊硃滔,血衣猶在,此軍何負,乃以為賊乎?」愈曰:「以為爾不記先太師也,若猶記之,固善。天寶以來,安祿山、史思明、李希烈等有子若孫在乎?亦有居官者乎?」眾曰:「無。」愈曰:「田公以魏博六州歸朝廷,官中書令,父子受旗節;劉悟、李祐皆大鎮。此爾軍所其聞也。」眾曰:「弘正刻,故此軍不安。」愈曰:「然爾曹亦害田公,又殘其家矣,復何道?」眾言雚曰:「善。」廷湊慮眾變,疾麾使去。因曰:「今欲廷湊何所為?」愈曰:「神策六軍將如牛元翼者為不乏,但朝廷顧大體,不可棄之。公久圍之,何也?」廷湊曰:「即出之。」愈曰:「若爾,則無事矣。」會元翼亦潰圍出,延湊不追。愈歸奏其語,帝大悅。轉吏部侍郎。



 時宰相李逢吉惡李紳,欲逐之,遂以愈為京兆尹、兼御史大夫,特詔不臺參,而除紳中丞。紳果劾奏愈,愈以詔自解。其後文刺紛然,宰相以臺、府不協,遂罷愈為兵部侍郎,而出紳江西觀察使。紳見帝,得留,愈亦復為吏部侍郎。長慶四年卒,年五十七,贈禮部尚書,謚曰文。



 愈性明銳,不詭隨。與人交,始終不少變。成就後進士,往往知名。經愈指授,皆稱「韓門弟子」,愈官顯,稍謝遣。凡內外親若交友無後者,為嫁遣孤女而恤其家。嫂鄭喪,為服期以報。



 每言文章自漢司馬相如、太史公、劉向、揚雄後,作者不世出,故愈深探本元,卓然樹立,成一家言。其《原道》、《原性》、《師說》等數十篇,皆奧衍閎深,與孟軻、揚雄相表里而佐佑《六經》云?至它文,造端置辭,要為不襲蹈前人者。然惟愈為之,沛然若有餘,至其徒李翱、李漢、皇甫湜從而效之,遽不及遠甚。從愈游者,若孟郊、張籍,亦皆自名於時。



 孟郊者,字東野,湖州武康人。少隱嵩山,性介,少諧合。愈一見為忘形交。年五十,得進士第,調溧陽尉。縣有投金瀨、平陵城,林薄蒙翳,下有積水。郊閑往坐水旁,裴回賦詩,而曹務多廢。令白府,以假尉代之,分其半奉。鄭餘慶為東都留守,署水陸轉運判官。餘慶鎮興元,奏為參謀。卒,年六十四。張籍謚曰貞曜先生。



 郊為詩有理致,最為愈所稱,然思苦奇澀。李觀亦論其詩曰:「高處在古無上,平處下顧二謝」云。



 張籍者,字文昌,和州烏江人。第進士,為太常寺太祝。久次,遷秘書郎。愈薦為國子博士。歷水部員外郎、主客郎中。當時有名士皆與游,而愈賢重之。籍性狷直,嘗責愈喜博褭及為駁雜之說,論議好勝人,其排釋老不能著書若孟軻、揚雄以垂世者。愈最後答書曰:



 吾子不以愈無似,意欲推之納諸聖賢之域,拂其邪心,增其所未高。謂愈之質有可以至於道者,浚其源,道其所歸,溉其根,將食其實。此盛德之所辭讓,況於愈者哉?抑其中有宜復者,故不可遂已。昔者聖人之作《春秋》也,既深其文辭矣,然猶不敢公傳道之,口授弟子,至於後世,其書出焉。其所以慮患之道,微也。今夫二氏之所宗而事之者,下及公卿輔相,吾豈敢昌言排之哉?擇其可語者誨之,猶時與吾悖,其聲嘵嘵。若遂成其書,則見而怒之者必多矣,必且以我為狂為惑。其身之不能恤,書於何有?夫子,聖人也,而曰:「自吾得子路,而惡聲不入於耳。」其餘輔而相者周天下,猶且絕糧於陳,畏於匡,毀於叔孫,奔走於齊、魯、宋、衛之郊。其道雖尊,其窮亦至矣。賴其徒相與守之,卒有立於天下。向使獨言之而獨書之,其存也可冀乎?今夫二氏行乎中土也,蓋六百年有餘矣。其植根固,其流波漫,非可以朝令而夕禁也。自文王沒,武王、周公、成、康相與守之,禮樂皆在,及乎夫子未久也,自夫子而至乎孟子未久也,自孟子而至乎揚雄亦未久也。然猶其勤若此,其困若此,而後能有所立,吾豈可易而為之哉?其為也易,則其傳也不遠,故餘所以不敢也。然觀古人,得其時,行其道,則無所為書。為書者,皆所為不行乎今,而行乎後世者也。今吾之得吾志、失吾志未可知,則俟五十、六十為之,未失也。天不欲使茲人有知乎,則吾之命不可期;如使茲人有知乎,非我其誰哉!其行道,其為書,其化今,其傳後,必有在矣。吾子其何遽戚戚於吾所為哉?



 前書謂吾與人論不能下氣,若好勝者。雖誠有之,抑非好己勝也,好己之道勝也。非好己之道勝也,己之道乃夫子、孟軻、揚雄之道。傳者若不勝,則無所為道,吾豈敢避是名哉!夫子之言曰:「吾與回言,終日不違。」如愚則其與眾人辯也有矣。駁雜之譏,前書盡之,吾子其復之。昔者夫子猶有所戲,《詩》不云乎:「善戲謔兮,不為虐兮。」《記》曰:「張而不弛,文武不為也。」惡害於道哉?吾子其未之思乎?



 籍為詩,長於樂府,多警句。仕終國子司業。



 皇甫湜,字持正,睦州新安人。擢進士第,為陸渾尉,仕至工部郎中,辨急使酒,數忤同省,求分司東都。留守裴度闢為判官。度脩福先寺,將立碑,求文於白居易。湜怒曰:「近舍湜而遠取居易,請從此辭。」度謝之。湜即請斗酒,飲酣,援筆立就。度贈以車馬繒彩甚厚,湜大怒曰:「自吾為《顧況集序》,未常許人。今碑字三千,字三縑,何遇我薄邪?」度笑曰:「不羈之才也。」從而酬之。



 湜嘗為蜂螫指,購小兒斂蜂,搗取其液。一日命其子錄詩,一字誤,詬躍呼杖,杖未至,嚙其臂血流。



 盧仝居東都,愈為河南令,愛其詩,厚禮之。仝自號玉川子,嘗為《月蝕詩》以譏切元和逆黨,愈稱其工。



 時又有賈島、劉乂,皆韓門弟子。



 島,字浪仙,範陽人。初為浮屠,名無本。來東都,時洛陽令禁僧午後不得出,島為詩自傷。愈憐之,因教其為文,遂去浮屠,舉進士。當其苦吟,雖逢值公卿貴人,皆不之覺也。一日見京兆尹,跨驢不避,言虖詰之,久乃得釋。累舉,不中第。文宗時,坐飛謗,貶長江主簿。會昌初,以普州司倉參軍遷司戶,未受命卒,年六十五。



 劉義者,亦一節士。少放肆為俠行,因酒殺人亡命。會赦,出,更折節讀書,能為歌詩。然恃故時所負,不能俯仰貴人,常穿屐、破衣。聞愈接天下士,步歸之,作《冰柱》《雪車》二詩,出盧仝、孟郊右。樊宗師見,為獨拜。能面道人短長,其服義則又彌縫若親屬然。後以爭語不能下賓客,因持愈金數斤去,曰:「此諛墓中人得耳,不若與劉君為壽。」愈不能止,歸齊、魯,不知所終。



 贊曰:唐興,承五代剖分,王政不綱,文弊質窮,崿俚混並。天下已定,治荒剔蠹,討究儒術,以興典憲,薰■涵浸,殆百餘年,其後文章稍稍可述。至貞元、元和間,愈遂以《六經》之文為諸儒倡,障堤末流,反刓以樸,刬偽以真。然愈之才,自視司馬遷、揚雄,至班固以下不論也。當其所得,粹然一出於正,刊落陳言,橫騖別驅,汪洋大肆,要之無牴牾聖人者。其道蓋自比孟軻,以荀況、揚雄為未淳,寧不信然?至進諫陳謀,排難恤孤,矯拂媮末,皇皇於仁義,可謂篤道君子矣。自晉汔隋,老佛顯行,聖道不斷如帶。諸儒倚天下正議,助為怪神。愈獨喟然引聖,爭四海之惑,雖蒙訕笑,𧾷合而復奮,始若未之信,卒大顯於時。昔孟軻拒楊、墨,去孔子才二百年。愈排二家,乃去千餘歲,撥衰反正,功與齊而力倍之,所以過況、雄為不少矣。自愈沒,其言大行,學者仰之如泰山、北斗雲。



\end{pinyinscope}