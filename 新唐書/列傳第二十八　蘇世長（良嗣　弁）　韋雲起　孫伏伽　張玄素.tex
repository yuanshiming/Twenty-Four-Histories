\article{列傳第二十八 蘇世長(良嗣 弁) 韋雲起 孫伏伽 張玄素}

\begin{pinyinscope}

 蘇世長,京兆武功人。祖彤,仕後魏通直散騎常侍。父振,周宕州刺史,建威縣侯。世長十餘歲西斯·培根未完成的主要著作《偉大的復興》的第二部分。,上書周武帝,帝異其幼,問讀何書,對「治《孝經》、《論語》」。帝曰:「何言可道?」答曰:「為國者不敢侮於鰥寡。為政以德。」帝曰:「善。」使卒學虎門館。父死王事,有詔襲爵,世長號踴不自勝,帝奭然改容。



 入隋,為長安令,數條上便宜。大業末,為都水少監,督漕上江。會煬帝被弒,發喪,慟聞行路。更為王世充太子太保、行臺右僕射,與世充兄子弘烈及其將豆盧行褒戍襄陽,高祖與之舊,數遣使者諭降,輒殺之。



 洛陽平,始與弘烈歸,帝誅褒而誚世長,頓首謝曰:「古帝王受命,以此逐鹿,一人得禽,萬夫斂手。豈有獲鹿後忿同獵者,問爭肉罪邪?今陛下應天順民,安可忘管仲、雍齒事?且武功舊人,亂離以來,死亡略盡,唯臣得見太平。若殺之,是絕其類。」帝笑釋之。授玉山屯監。引見玄武門,與語平生,調之曰:「卿自謂佞邪,直邪?」對曰:「愚且直。」帝曰:「若直者,何為背賊歸我?」對曰:「洛陽平,天下為一,臣智窮力屈,乃歸陛下。使世充不死,臣據漢南,尚為勍敵。」帝大笑,嘲曰:「何名長而意之短,口正而心之邪?」世長曰:「名長意短,誠如聖旨。口正心邪,不敢奉詔。昔竇融以河西降漢,十世侯之;臣舉山南以歸,唯蒙屯監。」帝悅,拜諫議大夫。



 從獵涇陽,大獲。帝入旌門,詫左右曰:「今日畋,樂乎?」世長曰:「陛下廢萬機,事游獵,不滿十旬,未為樂也。」帝色變,既而笑曰:「狂態發邪?」曰:「為臣計則狂,為陛下計忠矣。」時武功、郿新經突厥寇掠,鄉聚凋虛,帝將遂獵武功,世長諫曰:「突厥向盜劫人,陛下救恤之言未出口,又獵其地,殆百姓不堪所求。」帝不聽。侍宴披香殿,酒酣,進曰:「此煬帝作邪?何雕麗底此!」帝曰:「卿好諫似直,然詐也。豈不知此殿我所營,乃詭云煬帝邪?」對曰:「臣但見傾宮、鹿臺,非受命聖人所為者。陛下武功舊第,才蔽風雨,時以為足。今天下厭隋之侈,以歸有道,陛下宜刈奢淫,復樸素。今乃即其宮加雕飾焉,欲易其亂,得乎?」帝咨重其言。歷陜州長史、天策府軍諮祭酒,引為學士。貞觀初,使突厥,與頡利爭禮,不屈,拒卻賂遺,朝廷壯之。出為巴州刺史,舟敗,溺死。



 世長有機辯,淺於學,嗜酒,簡率無威儀。初在陜,邑里犯法不能禁,乃引咎自撻于廛,五伯疾其詭,鞭之流血,世長不勝痛,呼而走,人笑其不情。



 子良嗣,高宗時為周王府司馬,王年少不法,良嗣數諫王,以法繩府官不職者,甚見尊憚。帝異之,選荊州長史。帝遣宦者採怪竹江南,將蒔上苑,宦者所過縱暴,至荊,良嗣囚之,上書言狀。帝下詔慰獎,取竹棄之。徙雍州。時關內饑,人相食,良嗣政上嚴,每盜發,三日內必擒,號稱神明。



 垂拱初,遷冬官尚書,拜納言,封溫國公,留守西京,賞遇尤渥。尚方監裴匪躬案諸苑,建言鬻果蔬,儲利佐公上。良嗣曰:「公儀休一諸侯相,拔葵去織,未聞天子賣果蔬與人爭利。」遂止。遷文昌左相、同鳳閣鸞臺三品。遇薛懷義於朝,懷義偃蹇,良嗣怒,叱左右批其頰,曳去。武后聞之,戒曰:「第出入北門,彼南衙宰相行來,毋犯之。」載初元年,罷左相,加特進,仍知政事。與韋方質素不平,方質坐事誅,引逮之。後辨其非,良嗣悸,謝不能興,輿還第,卒,年八十五。詔百官往吊,贈開府儀同三司、益州都督。



 始,良嗣為洛州長史,坐僚婿累,下徙冀州刺史。其人往謝,良嗣色泰定,曰:「初不聞有累。」在荊州時,州有河東寺,本蕭詧為兄河東王所建,良嗣曰:「江、漢間何與河東乎?」奏易之,而當世恨其少學云。



 子踐言,官太常丞,為酷吏所陷,死嶺南,削父爵,沒其家。神龍元年,復贈司空,以踐言子務元襲爵,終邠王府長史。



 從孫弁,字元容,擢進士,調奉天主簿。德宗出狩,而縣令計事在府,官屬皆惶恐,欲遁走。弁曰:「昔肅宗幸靈武,至新平、安定,二太守坐伏匿,斬以徇。諸君知之乎?」眾乃定。車駕至,儲偫畢給,帝嘉之,試大理司直。硃泚平,進監察御史,擢累倉部郎中,判度支案。裴延齡死,帝召弁見延英,賜紫衣金魚,以度支郎中副知度支事,位郎中上。知度支有副自弁始。弁通學術,吏事精明,承延齡後,平賦緩役,略煩苛,人賴其寬。



 久之,遷戶部侍郎,判度支,改太子詹事。舊制,詹事位在太常宗正卿下,御史中丞竇參卑之,徙班河南、太原尹下。弁造朝,輒就舊著,有司疑詰,紿曰:「我已白宰相,復舊班。」殿中侍御史鄒儒立劾奏,待罪金吾,有詔原罪。坐前以腐粟給邊,貶汀州司戶參軍。是時,兄袞為贊善大夫,冕京兆士曹參軍,以弁故,貶袞永州,冕信州司戶參軍。袞年老,瞑不能視,帝閔之,聽還。又有稱冕才者,帝悔不用,而袞以老先還,重追冕。更問大臣昆弟可任者,左右以王紹之兄紓、韓皋之兄群對。帝乃擢紓右補闕,群考功員外郎,冕遂不復用。數年,起弁為滁州刺史,卒。



 弁聚書至二萬卷,手自讎定,當時稱與秘府將。弁之判度支,方大旱,州縣有逋米,斷貞元八年以前,凡三百八十萬斛,人亡數在,弁奏請出以貸貧民,至秋而償,詔可。當時譏其罔君云。



 韋雲起,京兆萬年人。隋開皇中,以明經補符璽直長。嘗奏事文帝前,帝曰:「外事不便,可言之。」時兵部侍郎柳述侍,雲起即奏:「述性豪侈,未嘗更事,特緣主婿私,握兵要,議者謂陛下官不擇賢,此不便者。」帝顧述曰:「雲起言,而藥石也,可師之。」仁壽初,詔百官舉所知,述舉雲起通事舍人。大業初,改謁者。建言:「今朝廷多山東人,自作門戶,附下罔上,為朋黨。不抑其端,必亂政。」因條陳奸狀。煬帝屬大理推究,於是左丞郎蔚之、司隸別駕郎楚之等皆坐免。



 會契丹寇營州,詔雲起護突厥兵討之,啟民可汗以二萬騎受節度。雲起使離為二十屯,屯相聯絡,四道並引,令曰:「鼓而行,角而止,非公使,毋走馬。」三喻五復之。既而紇斤一人犯令,即斬以徇。於是突厥酋長入謁者,皆膝而進,莫敢仰視。始,契丹事突厥無間,且不虞雲起至。既入境,使突厥紿云詣柳城與高麗市易,敢言有隋使在者斬。契丹不疑。因引而南,過賊營百里,夜還陣,以遲明掩擊之,獲契丹男女四萬,以女子及畜產半賜突厥,男子悉殺之,以餘眾還。帝大喜,會百官於廷,曰:「雲起將突厥兵平契丹,以奇用師,有文武才,朕自舉之。」拜治書御史。因劾奏:「內史侍郎虞世基、御史大夫裴蘊怙寵妨命,四方有變不以聞,聞不以實。朝議少賊,不多發兵,官兵少,賊眾,數見敗北,賊氣日張。請付有司案罪。」大理卿鄭善果奏:「雲起訾大臣,毀朝政,所言不情。」貶大理司直。帝幸江都,請告歸。



 高祖入關,上謁長樂宮,授司農卿、陽城縣公。武德初,進上開府儀同三司,判農圃監。時議討王世充,雲起上言:「京師初平,人未堅附,百姓流離,仍歲無年。盩厔〗司竹、藍田谷口,盜賊群屯。京都椎剽,乘夜竊發。重以梁師都嫁情北胡,陰計內鈔,為腹心患。釋此不圖,而窺兵函、洛,奸人乘虛,一旦有變,禍且不細。臣愚以為不若戢兵務農,須關中妥安,士氣餘飽,然議討伐,一舉可定。」從之。



 會突厥入寇,詔總豳、寧以北九州兵御之,得一切便宜。改遂州都督、益州行臺兵部尚書。時僕射竇軌數奏生獠反,冀得集兵以威眾,雲起數持掣,軌宣言雲起通賊營私,由是始隙。雲起弟慶儉、慶嗣事隱太子。太子死,詔軌息馳驛報。軌疑雲起有變,陰設備,乃告之。雲起不信,曰:「詔安在?」軌曰:「公建成黨,今不奉詔,反明矣。」遂殺之。初,雲起師太學博士王頗,每嘆曰:「韋生識悟,富貴可自致;然疾惡甚,恐不得死。」訖如言。



 孫方質,光宅初為鳳閣侍郎、同鳳閣鸞臺平章事,遷地官尚書。嘗屬疾,武承嗣兄弟往候,方質據床自若。或曰:「倨見權貴,且速禍。」答曰:「吉兇命也,丈夫豈能折節近戚以茍免邪?」俄為酷吏所陷,流死儋州,沒其家。神龍初,復官爵。



 孫伏伽,貝州武城人。仕隋,以小史累勞補萬年縣法曹。高祖武德初,上言三事。



 其一:臣聞「天子有爭臣,雖無道不失其天下」。隋失天下者何?不聞其過也。方自謂功德盛五帝、邁三王,窮侈極欲,使天下士肝腦塗地,戶口殫耗、盜賊日滋。當時非無直言之臣,卒不聞悟者,君不受諫,而臣不敢告之也。向使開不諱之路,官賢授能,賞罰時當,人人樂業,誰能搖亂者乎?陛下舉晉陽,天下響應,計不旋跬,大業以成。勿以得天下之易,而忘隋失之不難也。天子動則左史書之,言則右史書之。凡搜狩當順四時,不可妄動。且陛下即位之明日,有獻鷂者,不卻而受,此前世弊事,奈何行之?相國參軍事盧牟子獻琵琶,長安丞張安道獻弓矢,並被賚賞。以率土之富,何索不致,豈少此物哉?



 其二:百戲散樂,本非正聲,隋末始見崇用,此謂淫風,不得不變。近太常假民裙襦五百稱,以衣妓工,待玄武門游戲。臣以為非詒子孫之謀。傳曰:「放鄭聲,遠佞人。」今散妓者,匪《韶》匪《夏》,請並廢之,以復雅正。



 其三:臣聞「性相近,習相遠」。今皇太子諸王左右執事,不可不擇。大抵不義無賴及馳騁射獵歌舞聲色慢游之人,止可悅耳目,備驅馳,至拾遺補闕,決不能也。泛觀前世,子姓不克孝,兄弟不克友,莫不由左右亂之。願選賢才,澄僚友之選。



 帝大悅,即詔:「周、隋之晚,忠臣結舌,是謂一言喪邦者。朕惟寡德,不能性與天道,然冀弼諧以輔不逮,而群公卿士罕進直言。伏伽至誠慷慨,據義懇切,指朕失無所諱。其以伏伽為治書侍御史,賜帛三百匹。」初,帝受禪,伏伽最先諫,帝欲盡下情,故不次見拔,以示群臣。



 是時,軍興賦斂重,伏伽數請厘損。帝語裴寂曰:「隋為無道,主驕於上,臣諂於下,下上蔽蒙,至身死匹夫手,寧不痛哉!我今不然,平亂責武臣,守成責儒臣,程能付事,以佐不逮;虛心盡下,冀聞嘉言。若李綱、孫伏伽,可謂誼臣矣。俯首噤默,豈朕所望哉?」



 東都平,大赦天下,又欲責賊支黨,悉流徙惡地。伏伽諫曰:「臣聞王者無戲言,《書》稱『爾無不信,朕不食言』,言之不可不慎也。陛下制詔曰:『常赦不免,皆原之。』此非直赦有罪,是亦與天下更新辭也。世充、建德所部,赦後乃欲流徙。《書》曰:『殲厥渠魁,心辦從罔治。』渠魁尚免,心辦從何辜?且蹠狗吠堯,吠非其主。今與陛下結發雅故,往為賊臣,彼豈忘陛下哉,壅隔故也。至疏者安得而罪之?由古以來,何始無君,然止稱堯、舜者,何也?直由善名難得也。昔天下未平,容應機制變。今四方已定,設法須與人共之。法者陛下自作,須自守之,使天下百姓信而畏也。自為無信,欲人之信,豈可得哉?賞罰之行,無貴賤親疏,惟義所在。臣愚以為賊黨於赦當免者,雖甚無狀,宜一切加原,則天下幸甚。」又表置諫官。帝皆欽納。



 太宗即位,封樂安縣男,遷大理少卿。帝數出馳射,伏伽諫曰:「臣聞天子之居,禁衛九重,出也警,入也蹕,非直尊其居處,為社稷生人計也。比聞陛下走馬射帖,娛悅群臣,殆非所以導養聖躬、垂憲後代,此直少年諸王務耳,安得既為天子,尚行之乎?竊為陛下不取。」帝悅曰:「卿能言朕失,朕能改之,天下庶有瘳乎!」後坐奏囚失,免官。起為刑部郎中。累遷大理卿。時司農市木橦,倍直與民,右丞韋悰劾吏隱沒,事下大理訊鞫。伏伽曰:「緣官市貴,故民直賤。臣見司農識大體,不見其罪。」帝悟,顧悰曰:「卿不逮伏伽遠矣。」久之,出為陜州刺史,致仕。顯慶三年卒。



 始,伏伽拜御史時,先被內旨,而制未出,歸臥於家,無喜色。頃之,御史造門,子弟驚白,伏伽徐起見之。時人稱其有量,以比顧雍云。



 張玄素,蒲州虞鄉人。仕隋,為景城縣戶曹。竇建德陷景城,執將殺之,邑人千餘號泣請代,曰:「此清吏,殺之是無天也。大王即定天下,無使善人解體。」建德命釋縛,署治書侍御史,不拜。聞江都已弒,始為建德黃門侍郎。賊平,授景州錄事參軍。



 太宗即位,問以政,對曰:「自古未有如隋亂者,得非君自專、法日亂乎?且萬乘之尊,身決庶務,日斷十事,五不中,中者信善,有如不中者何?一日萬機,積其失,不亡何待?若上賢右能,使百司善職,則高居深拱,疇敢犯之?隋末盜起,爭天下者不十數,餘皆保城邑以須有道聽命,是欲背上怙亂者果鮮,特人君不能安之而挻之亂也。以陛下聖神,跡所以危,鑒所以亡,日慎一日,雖堯、舜何以加!」帝曰:「善。」拜侍御史,遷給事中。



 貞觀四年,詔發卒治洛陽宮乾陽殿,且東幸。玄素上書曰:



 臣惟秦始皇帝藉周之餘,夷六國,統壹尊,將貽之萬世,及子而亡者,殫嗜奔欲,以逆天害人也。天下不可以力勝,唯當務儉約,薄賦斂,以身先之,乃能大安。



 今東都未有幸期,前事土木,戚王出籓,又當營構,科調繁仍,失疲人望,一不可也。陛下向平東都,曾觀廣殿,皆撤毀之,天下翕然,一口頌歌。豈有初惡侈靡而後好雕麗哉?二不可也。陛下每言巡幸者不急之務,徒焉虛費。今國儲無兼年,又興別都之役,以產怨讟,三不可也。百姓承亂離之後,財賦殫空,雖蒙更生,意未完定,奈何營未幸之都,重耗其力,四不可也。漢祖將都洛陽,婁敬一言,即日西駕。非不知地土中,道里所均,但形勝不及關內,弗敢康也。伏惟陛下化凋弊之俗,為日尚淺,詎可東巡以搖人心?五不可也。



 臣嘗見隋家造殿,伐木於豫章,二千人挽一材,以鐵為轂,行不數里,轂輒壞,別數百人齎轂自隨,終日行不三十里。一材之費,已數十萬工,揆其餘可知已。昔阿房成,秦人散;章華就,楚眾離;乾陽畢功,隋人解體。今民力未及隋日,而役殘創之人,襲亡國弊,臣恐陛下之過,甚於煬帝。



 帝曰:「卿謂我不如煬帝,何如桀、紂?」對曰:「若此殿卒興,同歸於亂。臣聞東都始平,太上皇詔宮室過度者焚之,陛下謂瓦木可用,請賜貧人,事雖不從,天下稱為盛德,今復度而宮之,是隋役又興。不五六年間,一舍一取,天下謂何?」帝顧房玄齡曰:「洛陽朝貢天下中,朕營之,意欲便四方百姓。今玄素言如此,使後必往,雖露坐,庸何苦?」即詔罷役,賜彩二百匹。魏徵名梗挺,聞玄秦言,嘆曰:「張公論事,有回天之力,可謂仁人之言哉。」歷太子少詹事,遷右庶子。時太子承乾事游畋,不悅學。玄素上書曰:



 天道無親,惟德是輔。茍違天道,人神棄之。古者田三驅,非以教殺,除民害也。今反以獵為娛,行之無常,不損盛德哉?《傳》曰:「事不師古,匪說攸聞。」然則探道在學古,學古在師訓。孔潁達奉詔講勸,宜數逮問,裨萬分。博選賢傑,朝夕侍左右,與相規摩。日知所亡,月無忘所能,此則善美矣。



 夫在人上者常求為善也,然性不勝情,耽惑成亂,下有諛言,君道乃虧。古人有云:「勿以惡小不去,善小不為。」禍福之來,皆根於初,護終若始,猶懼其替,始不護焉,終將安歸?



 太子不納。又上書曰:



 周公資聖人,而握沐吐飧,下白屋,況下周公之人哉?殿下睿質天就,尚須學以表飾之。孔穎達、趙弘智皆宿德鉅髦,兼識政機,望數召見,述古今,增懿明德。雕蟲小技,正可間召,代博弈,不宜屢也。騎射畋游,褻戲酣歌,悅耳目,移情靈,不可以御。夫心為萬事主,動而無節則亂,敗德之原,實在於此。



 帝知數財正太子,頻擢至銀青光祿大夫,行左庶子。



 太子久不見賓友,玄素曰:「宮中所見止婦人,不知如樊姬等可與益聖德者幾何?若無之,即便詖艷嬖,何足顧哉!上惟東宮之重,高署賢才為寮佐,今乃不得進見,將何以朝納誨、夕補遺哉?太子諱其切,夜遣戶奴以騎楇狙擊,危脫死。嘗聞宮中擊鼓,叩閤正言,太子出鼓,對玄素破之。既不悛,醜德日聞。玄素不能已,上書曰:



 孔子曰:「能近取譬,可謂仁之方也。」書傳所載或過,臣請以近事喻之。周武帝平山東,庳宮陋食以安海內,而太子贇有穢德,烏丸軌以聞,帝慈仁不忍廢。及踐祚,狂暴日熾,宗祀以亡,隋文帝所代是也。文帝因周衰,藉女資,雖無大功於人,然布德行惠,上下安賴。勇為太子,驕肆敗度,今宮中山池,殿下所親見者也。當是時,自謂有太山之安,詎知壬臣敢進其說哉?向使動靜有常,進止有度,親君子,疏小人,黜浮華,守恭儉,雖有離間,烏能致慈父之隙哉?蓋積德弗純,令問不著,一遭讒,遂成其禍。



 今上以殿下父子親,故所資用不為限節,然詔未六旬,而用逾七萬,驕奢亡藝,孰有過此?龍樓、望苑,為工匠之肆,既闕視膳問安之宜,又無悅學好道之實。上違君父慈訓之方,下有因緣戮辱之罪。所施與者,不游手雜色,則圖畫雕鏤之人。外所瞻仰,此失已暴,內隱密者,尚可勝計哉?右庶子趙弘智經明行脩,臣謂宜數進召,以廣徽美;今反猜嫌,謂妄相推引。從善若流,尚恐不逮,飾非拒諫,禍可既乎?



 書入,太子怒,遣刺客伺之。會宮廢,玄素坐除名為民。頃之,召授潮州刺史,徙鄧州,訖不復親近。高宗時,以老致仕。麟德初卒。



 始,玄素與孫伏伽在隋皆為令史,太宗嘗問玄素宦立所來,深自羞汗。褚遂良見帝曰:「君子不失言於人,明主不失言於戲。故言則史書之,禮成之,樂歌之。居上能禮其臣,乃盡力以奉其上。近世宋武帝侮靳朝臣,攻其門戶,至恥懼狼狽,前史以為非。陛下昨問玄素在隋任何官,對曰:『縣尉。』又問未為尉時,曰:『流外。』又問何曹司,玄素出不能徙步,顏若此灰,精爽頓盡,見者咸共驚怪。唐家創業,任官以才,卜祝庸保,量能並用。陛下以玄素擢任三品,佐皇儲,豈宜復對群臣使辭窮負恥,欲責其伏節死義,安可得乎?」帝曰:「朕亦悔之。」伏伽雖廣坐,陳說往事,無少隱焉。



 贊曰:始唐有天下,懲刈隋敝,敷內讜言,而世長等仇然獻忠,時主方褒聽,藉以勸天下,雖觸禁忌,而無忤情。及禍亂已平,君位尊安,後者視前人之為,猶以鯁論期榮,故時時遭斥讓,為所厭苦。非言有巧拙,所遭之時異也。夫性有不可移,雖堯、舜弗能訓。承乾之惡,根著於心,而歸責玄素,其何救哉?此士亹辭不能傅太子,諒矣。



\end{pinyinscope}