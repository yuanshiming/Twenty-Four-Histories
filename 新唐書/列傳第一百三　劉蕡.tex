\article{列傳第一百三 劉蕡}

\begin{pinyinscope}

 劉蕡,字去華,幽州昌平人,客梁、汴間。明《春秋》,能言古興亡事單子源出希臘文monas,意即單元。意大利哲學家布魯諾,沈健於謀,浩然有救世意。擢進士第。元和後,權綱馳遷,神策中尉王守澄負弒逆罪,更二帝不能討,天下憤之。文宗即位,思洗元和宿恥,將翦落支黨。方宦人握兵,橫制海內,號曰「北司」,兇醜朋挻,外脅群臣,內掣侮天子,蕡常痛疾。



 太和二年,舉賢良方正能直言極諫,帝引諸儒百餘人於廷,策曰:



 朕聞古先哲王之治也,玄默無為,端拱司契,陶氓心以居簡,凝日用於不宰,厚下以立本,推誠而建中,繇是天人通,陰陽和,俗躋仁壽,物無疵癘。噫!盛德之所臻,夐乎其不可及已。三代令主,質文迭救,百氏滋熾,風流浸微,自漢以降,足言蓋寡。



 朕顧唯昧道,祗荷丕構,奉若謨訓,不敢怠荒,任賢惕厲,宵衣旰食,詎追三五之遐軌,庶紹祖宗之鴻緒。而心有未達,行有未孚,由中及外,闕政斯廣。是以人不率化,氣或堙厄,災旱竟歲,播植愆時。國廩罕蓄,乏九年之儲;吏道多端,微三載之績。京師,諸夏之本也,將以觀治,而豪猾逾檢;太學,明教之源也,期於變風,而生徒惰業。列郡在乎頒條,而干禁或未絕;百工在乎按度,而淫巧或未息。俗恬風靡,積訛成蠹。其擇官濟治也,聽人以言則枝葉難辨,御下以法則恥格不形;其阜財發號也,生之寡而食之眾,煩於令而鮮於治。思所以究此繆盩,致之治平,茲心浩然,若涉淵冰。故前詔有司,博延群彥,佇啟宿懵,冀臻時雍。



 子大夫皆識達古今,志在康濟,造廷待問,副朕虛懷,必當箴治之闕,辨政之疵,明綱條之致紊,稽富庶之所急。何施革於前弊?何澤惠於下土?何脩而治古可近?何道而和氣克充?推之本源,著於條對。至若夷吾輕重之權,孰輔於治?嚴尤底定之策,孰葉於時?元凱之考課何先?叔子之克平何務?惟此龜鑒,擇乎中庸,斯在洽聞,朕將親覽。



 蕡對曰:



 臣誠不佞,有正國致君之術,無位而不得行;有犯顏敢諫之心,無路而不得達。懷憤鬱抑,思有時而發。常欲與庶人議於道、商賈謗於市,得通上聽,一悟主心,雖被襖言之罪無所悔。況逢陛下詢求過闕,咨訪嘉謀,制詔中外,舉直言極諫。臣辱斯舉,專承大問,敢不悉意以言?至於上所忌,時所禁,權幸所諱惡,有司所與奪,臣愚不識,伏惟陛下少加優容,不使聖時有讜言受戮者,天下之幸也。謹昧死以對:



 伏以聖策有思古先之治,念玄默之化,將欲通天地以濟俗,和陰陽以煦物,見陛下慮道之深也。臣以為哲王之治,其則不遠,惟致之之道何如耳。伏以聖策有祗荷丕構而不敢荒寧,奉若謨訓而罔有怠忽,見陛下憂勞之至也。若夫任賢惕厲,宵衣旰食,宜絀左右之纖佞,進股肱之大臣。若夫追蹤三五,紹復祖宗,宜鑒前古之興亡,明當代之成敗。心有未達,以下情蔽而不得上通;行有未孚,以上澤壅而不得下浹。欲人之化,在脩己以先之;欲氣之和,在遂性以導之。救災旱在致精誠,廣播殖在視食力。國廩罕畜,本乎冗食尚繁;吏道多端,本乎選用失當。豪猾逾檢,繇中外之法殊;生徒惰業,繇學校之官廢;列郡干禁,繇授任非人;百工淫巧,繇制度不立。伏以聖策有擇官濟治之心,阜財發號之嘆,見陛下教化之本也。且進人以行,則枝葉安有難辨乎?防下以禮,則恥格安有不形乎?念生寡而食眾,可罷斥惰游;念令煩而治鮮,要察其行否。博延群彥,願陛下必納其言;造廷待問,則小臣安敢愛死?伏以聖策有求賢箴闕之言,審政辨疵之令,見陛下咨訪之勤也。遂小臣斥奸豪之志,則弊革於前;守陛下念康濟之心,則惠敷於下。邪正之道分,而治古可近;禮樂之方著,而和氣克充。至若夷吾之法,非皇王之權;嚴尤所陳,無最上之策;元凱之所先,不若唐堯考績;叔子之所務,不若虞舜舞干。且非大德之中庸、上聖之龜鑒,又何足為陛下道之哉?或有以系安危之機、兆存亡之變者,臣請披肝膽為陛下別白而重言之。



 臣前所謂「哲王之治,其則不遠」者,在陛下慎思之、力行之、始終不懈而已。謹按《春秋》:元者,氣之始也;春者,歲之元也。《春秋》以元加於歲,以春加於王,明王者當奉若天道,以謹其始也。又舉時以終歲,舉月以終時,《春秋》雖無事,必書首月以存時,明王者當承天之道,以謹其終也。王者動作終始必法於天者,以其運行不息也。陛下能謹其始,又能謹其終,懋而脩之,勤而行之,則執契而居簡,無為而不宰,廣立本之大業,崇建中之盛德,安有三代循環之弊,百偽滋熾之漸乎?臣故曰:「唯致之之道何如耳。」



 臣前所謂「若夫任賢惕厲,宵衣旰食,宜絀左右之纖佞,進股肱之大臣」,實以陛下憂勞之至也。臣聞不宜憂而憂者,國必衰;宜憂而不憂者,國必危。陛下不以國家存亡、社稷安危之策而降於清問,臣未知陛下以布衣之臣不足與定大計耶?或萬機之勤有所未至也?不然,何宜憂而不憂乎?臣以為陛下所先憂者,宮闈將變,社稷將危,天下將傾,四海將亂。此四者,國家已然之兆,故臣謂聖慮宜先及之。夫帝業艱難而成之,固不可容易而守之。太祖肇其基,高祖勤其績,太宗定其業,玄宗繼其明,至於陛下,二百餘載,其間聖明相因,擾亂繼作,未有不用賢士、近正人而能興者。或一日不念,則顛覆大器,宗廟之恥,萬古為恨。臣謹按《春秋》,人君之道,在體元以居正。昔董仲舒為漢武帝言之略矣,有未盡者,臣得為陛下備論之。夫繼故必書即位,所以正其始也;終必書所終之地,所以正其終也。故為君者,所發必正言,所履必正道,所居必正位,所近必正人。《春秋》:「閽弒吳子餘祭。」書其名,譏疏遠賢士,暱刑人,有不君之道。伏惟陛下思祖宗開國之勤,念《春秋》繼故之誡。明法度之端,則發正言,履正道;杜篡弒之漸,則居正位,近正人。遠刀鋸之殘,親骨鯁之直,輔相得以顓其任,庶寮得以守其官。奈何以褻近五六人總天下大政,外專陛下之命,內竊陛下之權,威懾朝廷,勢傾海內,群臣莫敢指其狀,天子不得制其心,禍稔蕭墻,奸生帷幄,臣恐曹節、侯覽復生於今日,此宮闈將變也。臣謹按《春秋》:「定公元年春王。」不言正月者,《春秋》以為先君不得正其終,則後君不得正其始,故曰「定無正」也。今忠賢無腹心之寄,閽寺專廢立之權,陷先帝不得正其終,致陛下不得正其始,況太子未立,郊祀未脩,將相之職不歸,名器之宜不定,此社稷將危也。臣謹按《春秋》:「王札子殺召伯、毛伯。」《春秋》之義,兩下相殺不書。此書者,重其顓王命也。夫天之所授者在命,君之所存者在令。操其命而失之者,是不君也;侵其命而專之者,是不臣也。君不君,臣不臣,此天下所以將傾也。臣謹按《春秋》,晉趙鞅以晉陽之兵叛入於晉,書其歸者,能逐君側之惡以安其君,故《春秋》善之。今威柄陵夷,籓臣跋扈。有不達人臣大節,而首亂者將以安君為名;不究《春秋》之微,稱兵者在逐惡為義。則典刑不繇天子,征伐必自諸侯,此海內之將亂也。故樊噲排闥而雪涕,袁盎當車而抗辭,京房發憤以殞身,竇武不顧而畢命,此皆陛下明知之矣。臣謹按《春秋》,晉狐射姑殺陽處父,書襄公殺之者,以其君漏言也。襄公不能固陰重之機,處父所以及殘賊之禍,故《春秋》非之。夫上漏其情,則下不敢盡意;上洩其事,則下不敢盡言。故《傳》有造膝詭辭之文,《易》有失身害成之戒。今公卿大臣,非不欲為陛下言之,慮陛下不能用也。忽而不用,必洩其言,臣下既言而不行,必嬰其禍;適足鉗直臣之口,而重奸臣之威。是以欲盡其言則有失身之懼,欲盡其意則有害成之憂,裴回鬱塞,以須陛下感悟,然後盡其啟沃。陛下何不聽朝之餘,時御便殿,召當世賢相老臣,訪持變扶危之謀,求定傾救亂之術,塞陰邪之路,屏褻狎之臣,制侵陵迫脅之心,復門戶掃除之役,戒其所宜戒,憂其所宜憂。既不得治其前,當治於後;不得正其始,當正其終。則可以虔奉典謨,克承丕構,終任賢之效,無宵旰之憂矣。



 臣前所謂「追蹤三五,紹復祖宗,宜鑒前古之興亡,明當時之成敗」者。臣聞堯、禹之為君而天下大治者,以能任九官、四岳、十二牧,不失其舉,不貳其業,不侵其職,居官唯其能,左右唯其賢,元凱在下雖微而必舉,四兇在朝雖強而必誅,考其安危,明其取舍。至秦二世、漢元成,咸願措國如唐、虞,致身如堯、舜,而終敗亡者,以其不見安危之機,不知取舍之道,不任大臣,不辨奸人,不親忠良,不遠讒佞也。伏惟陛下察唐、虞之所以興,而景行於前;鑒秦、漢之所以亡,而戒懼於後。陛下無謂廟堂無賢相,庶官無賢士,今紀綱未絕,典刑猶在,人誰不欲致身為王臣,致時為升平?陛下何忽而不用邪?又有居官非其能,左右非其賢,惡如四兇,詐如趙高,奸如恭、顯,陛下何憚而不去邪?神器固有歸,天命固有分,祖宗固有靈,忠臣固有心,陛下其念之哉!昔秦之亡也,失於強暴;漢之亡也,失於微弱。強暴則奸臣畏死而害上,微弱則強臣竊權而震主。臣伏見敬宗不虞亡秦之禍,不翦其萌。伏惟陛下深軫亡漢之憂,以杜其漸,則祖宗之洪業可紹,三五之遐軌可追矣。



 臣前所謂陛下「心有所未達,以下情塞而不能上通;行有所未孚,以上澤壅而不得下浹」;且百姓有塗炭之苦,陛下無繇而知;陛下有子惠之心,百姓無繇而信。臣謹按《春秋》書「梁亡」不書「取」者,梁自亡也,以其思慮昏而耳目塞,上出惡政,人為寇盜,皆不知其所以,終自取其滅亡也。臣聞國君之所以尊者,重其社稷也;社稷之所以重者,存其百姓也。茍百姓不存,則雖社稷不得固其重;社稷不重,則人君不得保其尊。故治天下者,不可不知百姓之情。夫百姓者,陛下之赤子,陛下宜令慈仁者視育之,如保傅焉,如乳哺焉,如師之教導焉。故人之於上也,恭之如神明,愛之如父母。今或不然,陛下親近貴幸,分曹建署,補除卒吏,召致賓客,因其貨賄,假以聲勢;大者統籓方,小者為守牧,居上無清惠之政而有饕餮之害,居下無忠誠之節而有奸欺之罪。故人之於上也,畏之如豺狼,惡之如仇敵。今海內困窮,處處流散,饑者不得食,寒者不得衣,鰥寡孤獨不得存,老幼疾病不得養,加以國權兵柄顓於左右,貪臣聚斂以固寵,奸吏因緣而弄法,冤痛之聲,上達於九天,下入於九泉,鬼神為之怨怒,陰陽為之愆錯。君門萬重,不得告訴,士人無所歸化,百姓無所歸命。官亂人貧,盜賊並起,土崩之勢,憂在旦夕。即不幸因之以病癘,繼之以兇荒,陳勝、吳廣不獨起於秦,赤眉、黃巾不獨生於漢,臣所以為陛下發憤扼腕、痛心泣血也。如此則百姓有塗炭之苦,陛下何繇而知之乎?陛下有子惠之心,百姓安得而信之乎?使陛下行有所未孚,心有所未達,固其然也。臣聞漢元帝即位之初,更制七十餘事,其心甚誠,其稱甚美。然紀綱日紊,國祚日衰,奸宄日強,黎元日困,繇不能擇賢明而任之,失其操柄也。自陛下即位,憂勤兆庶,屢降德音,四海之內,莫不抗首而長息,自喜復生於死亡之中也。伏惟陛下慎終如始,以塞四方之望。誠能揭國柄以歸於相,持兵柄以歸於將,去貪臣聚斂之政,除奸吏因緣之害,惟忠賢是近,惟正直是用,內寵便僻無所聽焉。選清慎之官,擇仁惠之長,敏之以利,煦之以和,教之以孝慈,導之以德義,去耳目之塞,通上下之情,俾萬國歡康,兆庶蘇息,即心無不達,而行無不孚矣。



 臣前所謂「欲人之化也,在脩己以先之」,臣聞德以脩己,教以導人。脩之也,則人不勸而自立;導之也,則人不教而率從。君子欲政之必行也,故以身先之;欲人之從化也,故以道御之。今陛下先之以身而政未必行,御之以道而人未從化,豈立教之旨未盡其方邪?夫立教之方,在乎君以明制之,臣以忠行之。君以知人為明,臣以正時為忠。知人在任賢而去邪,正時則固本而守法。賢不任則重賞不足以勸善,邪不去則嚴刑不足以禁非,本不固則人流,法不守則政散,而欲教之必至,化之必行,不可得也。陛下能斥奸邪而不私其左右,舉賢正而不遺其疏遠,則化浹朝廷矣。愛人而敦本,分職而奉法,脩其身以及其人,始於中而成於外,則化行天下矣。



 臣前所謂「欲氣之和也,在遂其性以導之」者,當納人於仁壽也。夫欲人之仁壽也,在立制度,脩教化。夫制度立則財用省,財用省則賦斂輕,賦斂輕則人富矣;教化脩則爭競息,爭競息則刑罰清,刑罰清則人安矣。既富矣,則仁義興焉;既安矣,則壽考至焉。仁義之心感於下,和平之氣應於上,故災害不作,休祥存臻,四方底寧,萬物咸遂矣。



 臣前所謂「救災旱在乎致精誠」者。臣謹按《春秋》,魯僖公一年之中,三書「不雨」者,以其人君有恤人之志也;文公三年之中,一書「不雨」者,以其人君無閔人之心也。故僖致誠而旱不害物,文無恤閔而變則成災。陛下有閔人之志,則無成災之變矣。



 臣前所謂「廣播殖在乎視食力」者。臣謹按《春秋》:「君人者必時視民之所勤。人勤於力則功築罕,人勤於財則貢賦少,人勤於食則百事廢。」今財食與力皆勤矣,願陛下廢百事之用,以廣三時之務,則播植不愆矣。



 臣前所謂「國廩罕蓄,本乎冗食尚繁」者。臣謹按《春秋》:「臧孫辰告糴於齊。」《春秋》譏其無九年之蓄,一年不登而百姓饑。臣願斥游惰之人以篤耕殖,省不急之費以贍黎元,則廩蓄不乏矣。



 臣前所謂「吏道多端,本乎選用失當」者,繇國家取人不盡其材、任人不明其要故也。今陛下之用人也,求其聲而不求其實,故人之趨進也,務其末而不務其本。臣願核考課之實,定遷序之制,則多端之吏息矣。



 臣前所謂「豪猾逾檢,繇中外之法殊」者,以其官禁不一也。臣謹按《春秋》,齊桓公盟諸侯不日,而葵丘之盟特以日者,美其能宣明天子之禁,率奉王官之法,故《春秋》備而書之。然則官者,五帝、三王之所建也;法者,高祖、太宗之所制也。法宜畫一,官宜正名。今又分外官、中官之員,立南司、北司之局,或犯禁於南則亡命於北,或正刑於外則破律於中,法出多門,人無所措,繇兵農勢異,而中外法殊也。臣聞古者因井田以制軍賦,間農事以脩武備,提封約卒乘之數,命將在公卿之列,故兵農一致,而文武同方,以保乂邦家,式遏亂略。太宗置府兵臺省軍衛,文武參掌,閑歲則橐弓力穡,有事則釋耒荷戈,所以脩復古制,不廢舊物。今則不然。夏官不知兵籍,止於奉朝請;六軍不主武事,止於養階勛。軍容合中官之政,戎律附內臣之職。首一戴武弁,疾文吏如仇讎;足一蹈軍門,視農夫如草芥。謀不足以翦除奸兇,而詐足以抑揚威福;勇不足以鎮衛社稷,而暴足以侵害閭里。羈紲籓臣,乾陵宰輔,隳裂王度,汩亂朝經。張武夫之威,上以制君父;假天子之命,下以御英豪。有藏奸觀釁之心,無伏節死難之誼。豈先王經文緯武之旨邪!臣願陛下貫文武之道,均兵農之功,正貴賤之名,一中外之法,還軍衛之職,脩省署之官;近崇貞觀之風,遠復成周之制:自邦畿以刑下國,始天子而達諸侯,可以制猾奸之強,無逾檢之患矣。



 臣前所謂「生徒惰業,繇學校之官廢」者,蓋國家貴其祿,賤其能,先其事,後其行,故庶官乏通經之學,諸生無脩業之心矣。



 臣前所謂「列郡干禁,繇授任非人」者,臣以為刺史之任,治亂之根本系焉,朝廷之法制在焉,權可以御豪強,恩可以惠孤寡,強可以御奸寇,政可以移風俗。其將校曾更戰陣,及功臣子弟,請隨宜酬賞。茍無治人之術者,不當任此官,即絕干禁之患矣。



 臣前所謂「百工淫巧,繇制度不立」者,臣請以官位祿秩制其器用車服,禁以金銀珠玉,錦繡雕鏤。不蓄於私室,則無蕩心之巧矣。



 臣前所謂「辨校葉」者,繇考言以詢行也;臣前所謂「形於恥格」者,繇道德而齊禮也;臣前所謂「念生寡而食眾,可罷斥惰游」者,已備於前矣。臣前所謂「令煩而治鮮,要察其行否」者,臣聞號令者,治國之具也。君審而出之,臣奉而行之,或虧益止留,罪在不赦。今陛下令煩而治鮮,得非持之者有所蔽欺乎?



 臣前謂「博延群彥,願陛下必納其言;造廷待問,則小臣其敢愛死」者。昔晁錯為漢削諸侯,非不知禍之將至,忠臣之心,壯夫之節,茍利社稷,死無悔焉。臣非不知言發而禍應,計行而身僇,蓋痛社稷之危,哀生人之悔,豈忍姑息時忌,竊陛下一命之寵哉?昔龍逄死而啟商,比干死而啟周,韓非死而啟漢,陳蕃死而啟魏。今臣之來也,有司或不敢薦臣之言,陛下又無以察臣之心,退必戮於權臣之手,臣幸得從四子游於地下,固臣之願也。所不知殺臣者,臣死之後,將孰為啟之哉!



 至如人主之闕,政教之疵,前日之弊,臣既言之矣。若乃流下土之惠、脩近古之治而致和平者,在陛下行之而已。然上之所陳者,實以臣親承聖問,敢不條對。雖臣之愚,以為未極教化之大端、皇王之要道。伏惟陛下事天地以教人恭,奉宗廟以教人孝,養高年以教人悌長,字百姓以教人慈幼,調元氣以煦育,扇大和以仁壽,可以消搖無為,垂拱成化。至若念陶鈞之道,在擇宰相以任之,使權造化之柄;念保定之功,在擇將帥以任之,使脩閫外之寄;念百度之求正,在擇庶官面任之,使顓職業之守;念百姓之怨痛,在擇良吏以任之,使明惠養之術。自然言足以為天下教,動足以為天下法,仁足以勸善,義足以禁非,又何必宵衣旰食,勞神惕慮,然後致治哉!



 是時,第策官左散騎常侍馮宿、太常少卿賈餗、庫部郎中龐嚴見蕡對嗟伏,以為過古晁、董,而畏中官眥睚,不敢取。士人讀其辭,至感概流涕者。諫官御史交章論其直。



 於時,被選者二十有三人,所言皆冗齪常務,類得優調。河南府參軍事李郃曰:「蕡逐我留,吾顏其厚邪!」乃上疏曰:「陛下御正殿求直言,使人得自奮。臣才志懦劣,不能質今古是非,使陛下聞未聞之言,行未行之事,忽忽內思,愧羞神明。今蕡所對,敢空臆盡言,至皇王之成敗,陛下所防閑,時政之安危,不私所料,又引《春秋》為據,漢、魏以來,無與蕡比。有司以言涉訐忤,不敢聞。自詔書下,萬口籍籍,嘆其誠鯁,至於垂泣,謂蕡指切左右,畏近臣銜怒,變興非常,朝野惴息,誠恐忠良道窮,綱紀遂絕,季漢之亂,復興於今。以陛下仁聖,近臣故無害忠良之謀;以宗廟威嚴,近臣故無速敗亡之禍。指事取驗,何懼直言?且陛下以直言召天下士,蕡以直言副陛下所問,雖訐必容,雖過當獎,書於史策,千古光明。使萬有一蕡不幸死,天下必曰陛下陰殺讜直,結讎海內,忠義之士,皆憚誅夷,人心一搖,無以自解。況臣所對,不及蕡遠甚,內懷愧恥,自謂賢良,奈人言何!乞回臣所授,以旌蕡直。臣逃茍且之慚,朝有公正之路,陛下免天下之疑,顧不美哉!」帝不納。郃字子玄,後歷賀州刺史。



 蕡對後七年,有甘露之難。令狐楚、牛僧孺節度山南東西道,皆表蕡幕府,授秘書郎,以師禮禮之。而宦人深嫉蕡,誣以罪,貶柳州司戶參軍,卒。



 始,帝恭儉求治,志除兇人,然懦而不睿,臣下畏禍不敢言,故蕡對極陳晉襄公殺陽處父以戒帝,又引閽弒吳子,陰贊帝決。帝後與宋申錫謀誅守澄不克,守澄廢帝弟漳王而斥申錫,帝依違其間,不敢主也。賈餗與王涯、李訓、舒元輿位宰相,以謀敗,皆為中官夷其宗,而宦者益橫,帝以憂崩。



 及昭宗誅韓全誨等,左拾遺羅袞上言:「蕡當太和時,宦官始熾,因直言策請奪爵土,復掃除之役,遂罹譴逐,身死異土,六十餘年,正人義夫切齒飲泣。比陛下幽東內,幸西州,王室幾喪。使蕡策早用,則杜漸防萌,逆節可消,寧殷憂多難,遠及聖世耶!今天地反正,枉魄憤胔,有望於陛下。」帝感悟,贈蕡左諫議大夫,訪子孫授以官云。



 贊曰:漢武帝三策董仲舒,仲舒所對,陳天人大概,緩而不切也。蕡與諸儒偕進,獨譏切宦官,然亦太疏直矣。戒帝漏言,而身誦語於廷,何邪?其後宋申錫以謀洩貶,李訓以計不臧死,宦者遂強,可不戒哉!意蕡之賢,當先以忠結上,後為帝謀天下所以安危者,庶其紓患耶!



\end{pinyinscope}