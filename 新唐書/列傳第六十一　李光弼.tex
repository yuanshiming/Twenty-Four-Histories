\article{列傳第六十一 李光弼}

\begin{pinyinscope}

 李光弼,營州柳城人。父楷洛,本契丹酋長,武后時入朝,累官左羽林大將軍這些東西看作人的本質,並又推論為整個世界的基矗世界,封薊郡公。吐蕃寇河源,楷洛率精兵擊走之。初行,謂人曰:「賊平,吾不歸矣。」師還,卒於道,贈營州都督,謚曰忠烈。光弼嚴毅沉果,有大略,幼不嬉弄,善騎射。起家左衛親府左郎將,累遷左清道率,兼安北都護,補河西王忠嗣府兵馬使,充赤水軍使。忠嗣遇之厚,雖宿將莫能比。嘗曰:「它日得我兵者,光弼也。」俄襲父封。以破吐蕃、吐谷渾功,進雲麾將軍。朔方節度使安思順表為副,知留後事,愛其材,欲以子妻之,光弼引疾去。隴右節度使哥舒翰異其操,表還長安。



 安祿山反,郭子儀薦其能,詔攝御史大夫,持節河東節度副大使,知節度事,兼雲中太守。尋加魏郡太守、河北採訪使。光弼以朔方兵五千出土門,東救常山,次真定,常山團結子弟執賊將安思義降。自顏杲卿死,郡為戰區,露胔蔽野,光弼酹而哭之,出為賊幽閉者,厚恤其家。時賊將史思明、李立節、蔡希德攻饒陽,光弼得思義,不殺,問其計,答曰:「今軍行疲勞,逢敵不可支,不如按軍入守,料勝而出。虜兵焱銳,弗能持重,圖之萬全。」光弼曰:「善。」據城待。明日,思明兵二萬傅堞,光弼兵不得出,乃以勁弩五百射之,賊退,徙陣稍北。光弼出其南,夾滹沱而軍。思明雖數困,然恃近救,解鞍休士。是日,饒陽賊五千至九門,光弼諜知之,提輕兵,斂旗鼓,伺賊方飯,襲殺之且盡。思明懼,引去,以奇兵斷饟道。馬食薦藉,光弼命將取芻行唐,賊鈔擊之,兵負戶戰,賊不能奪。會郭子儀收雲中,詔悉眾出井陘,與光弼合擊賊九門西,思明大敗,挺身走趙郡,立節中流矢死,希德走鉅鹿。收稿城等十縣,遂攻趙。詔加光弼範陽大都督府長史、範陽節度使。思明繇鼓城入博陵,殺官吏。景城、河間、信都、清河、平原、博平六郡結營自守,以附光弼。光弼急攻趙,一日拔之。士多鹵掠,光弼坐譙門,收所獲,悉歸之民,城中大悅。進圍博陵,未下。與子儀合擊思明於嘉山,大破之。光弼以範陽本賊巢窟,當先取之,揠賊根本。會潼關失守,乃拔軍入井陘。



 肅宗即位,詔以兵赴靈武,更授戶部尚書、同中書門下平章事,節度如故。光弼以景城、河間兵五千入太原。前此,節度使王承業政弛謬,侍御史崔眾主兵太原,每侮狎承業,光弼素不平。及是,詔眾以兵付光弼。眾素狂易,見光弼長揖,不即付兵,光弼怒,收系之。會使者至,拜眾御史中丞。光弼曰:「眾有罪,已前系,今但斬侍御史。若使者宣詔,亦斬中丞。」使者內詔不敢出,乃斬眾以徇,威震三軍。



 至德二載,思明、希德率高秀巖、牛廷玠將兵十萬攻光弼。時銳兵悉赴朔方,而麾下卒不滿萬,眾議培城以守,光弼曰:「城環四十里,賊至治之,徒疲吾人。」乃徹民屋為摞石車,車二百人挽之,石所及輒數十人死,賊傷十二。思明為飛樓,障以木幔,築土山臨城,光弼遣穴地頹之。思明宴城下,倡優居臺上靳指天子,光弼遣人隧地禽取之。思明大駭,徙牙帳遠去,軍中皆視地後行。又潛溝營地,將沈其軍,乃陽約降。至期,以甲士守陴,遣裨校出,若送款者,思明大悅。俄而賊數千沒於塹,城上鼓噪,突騎出乘之,俘斬萬計。思明畏敗,乃去,留希德攻太原。光弼出敢死士搏賊,斬首七萬級,希德委資糧遁走。初,賊至,光弼設公幄城隅以止息,經府門不顧。圍解,閱三昔乃歸私寢。收清夷、橫野等軍。賊別將攻好畤,破大橫關,光弼追敗之。加檢校司徒,尋遷司空,封鄭國公,食實戶八百。



 乾元元年,入朝,詔朝官四品以上郊謁,進兼侍中。與九節度圍安慶緒於相州,大戰鄴西,敗之。光弼與諸將議:「思明勒兵魏州,欲以怠我,不如起軍逼之。彼懲嘉山之敗,不敢輕出,則慶緒可禽。」觀軍容使魚朝恩固謂不可。既而思明來援,光弼拒賊,戰尤力,殺略大當。會諸將驚潰,各引歸,所在剽掠,獨光弼整眾還太原。帝貸諸將罪,以光弼兼幽州大都督府長史,知諸道節度行營事。又代子儀為朔方節度使。未幾,為天下兵馬副元帥。



 光弼以河東騎五百馳東都,夜入其軍,且謂賊方窺洛,當扼虎牢,帥師東出河上。檄召兵馬使張用濟,用濟憚光弼嚴,教諸將逗留其兵。用濟單騎入謁,光弼斬之,以辛京杲代。復追都將僕固懷恩,懷恩懼,先期至。會滑汴節度使許叔冀戰不利,降賊,思明乘勝西向。光弼敦陣徐行,趨東京,謂留守韋陟曰:「賊新勝,難與爭鋒,欲詘之以計。然洛無見糧,危偪難守,公計安出?」陟曰:「益陜兵,公保潼關,可以持久。」光弼曰:「兩軍相敵,尺寸地必爭。今委五百里而守關,賊得地,勢益張。不如移軍河陽,北阻澤、潞,勝則出,敗則守,表裏相應,賊不得西,此猿臂勢也。夫辨朝廷之禮,我不如公;論軍旅勝負,公不如我。」陟不能答。判官韋損曰:「東都乃帝宅,公當守之。」光弼曰:「汜水、崿嶺盡為賊蹊,子能盡守乎?」遂檄河南縱官吏避賊,閈無留人,督軍取戰守備。



 思明至偃師,光弼悉軍趨河陽,身以五百騎殿。賊游騎至石橋,諸將曰:「並城而北乎?當石橋進乎?」光弼曰:「當石橋進。」夜甲,士持炬徐引,部曲重堅,賊不敢逼。已入三城,眾二萬,軍才十日糧,與卒伍均少棄甘。賊憚光弼,未敢犯宮闕,頓白馬祠,治塹溝,築月城以守。賊攻光弼,與戰中水單西,破逆黨,斬千級,溺死者甚眾,生執五千人。初,光弼謂李抱玉曰:「將軍能為我守南城二日乎?」抱玉曰:「過期何若?」曰:「棄之。」抱玉許諾。即紿賊曰:「吾糧盡,明日當降。」賊喜,斂兵待期。抱玉已繕完,即請戰。賊忿欺,急攻之。抱玉出奇兵夾擊,俘獲過當,賊帥周摯引卻。光弼自將治中水單,樹壁掘塹。摯舍南城攻中水單,光弼遣荔非元禮戰羊馬,賊大潰。摯收兵復振,與安太清合眾三萬攻北城。光弼斂軍入,登陴望曰:「彼軍雖銳,然方陣而囂,不足虞也,日中當破。」乃出戰,及期未決,召諸將曰:「彼強而可破者,亂也。今以亂擊亂,宜無功。」因問:「賊陣何所最堅?」曰:「西北隅。」召郝廷玉曰:「為我以麾下破之。」曰:「廷玉所將步卒,請騎五百。」與之三百。復問其次,曰:「東南隅。」召論惟貞,辭曰:「蕃將也,不知步戰,請鐵騎三百。」與之二百。乃出賜馬四十,分給廷玉等。光弼執大旗曰:「望吾旗,麾若緩,可觀便宜。若三麾至地,諸軍畢入,生死以之,退者斬!」既而馮堞望廷玉軍不能前,趣左右取其首來。廷玉曰:「馬中矢,非卻也。」乃命易佗馬。有裨將援矛刺賊,洞馬腹,中數人,又有迎賊不戰而卻者,光弼召援矛者賜絹五百匹,不戰者斬。光弼麾旗三,諸軍爭奮,賊眾奔敗,斬首萬餘級,俘八千餘人,馬二千,軍資器械以億計,禽周摯、徐璜玉、李秦授,惟太清挺身走。思明未知,猶攻南城,光弼驅所俘示之,思明大懼,築壘以拒官軍。始,光弼將戰,內刀於靴,曰:「戰,危事。吾位三公,不可辱於賊。萬有一不捷,當自刎以謝天子。」及是,西向拜舞,三軍感動。太清襲懷州,守之。



 上元元年,加太尉、中書令。進圍懷州,思明來救,光弼再逐北。思明見兵河清,聲度河絕餉路。光弼壁野水度,既夕還軍,留牙將雍希顥守,曰:「賊將高暉、李日越,萬人敵也,賊必使劫我。爾留此,賊至勿與戰,若降,與偕來。」左右竊怪語無倫。是日,思明果召日越曰:「光弼野次,爾以鐵騎五百夜取之,不然,無歸!」日越至壘,使人問曰:「太尉在乎?」曰:「去矣。」「兵幾何?」曰:「千人。」「將為誰?」曰:「雍希顥。」日越謂其下曰:「我受命云何,今顧獲希顥,歸不免死。」遂請降。希顥與俱至,光弼厚待之,表授特進,兼金吾大將軍。高暉聞,亦降。或問:「公降二將何易也?」光弼曰:「思明再敗,恨不得野戰,聞我野次,彼固易之,命將來襲,必許以死。希顥無名,不足以為功。日越懼死,不降何待?高暉材出日越之右,降者見遇,貳者得不思奮乎?」諸軍決丹水灌懷州,未下。光弼令廷玉由地道入,得其軍號,登陴大呼,王師乘城,禽太清、楊希仲,送之京師,獻俘太廟。進食實戶一千五百。



 思明使諜宣言賊將士皆北人,謳吟思歸。朝恩信然,屢上賊可滅狀。詔諭光弼,光弼固言賊方銳,未可輕動。僕固懷恩媢光弼功,陰佐朝恩陳掃除計。使者來督戰,光弼不得已,令李抱玉守河陽,出師次北邙。光弼使傅山陣,懷恩曰:「我用騎,今迫險,非便地,請陣諸原。」光弼曰:「有險,可以勝,可以敗;陣於原,敗斯殲矣。且賊致死於我,不如阻險。」懷恩不從。賊據高原,以長戟七百,壯士執刀隨之,委物偽遁。懷恩軍爭剽獲,伏兵發,官軍大潰。懷州復陷,光弼度河保聞喜,抱玉以兵寡,棄河陽。光弼請罪,帝以懷恩違令覆軍,優詔召光弼入朝。懇讓太尉,更拜開府儀同三司、中書令、河中尹、晉絳等州節度使。未幾,復拜太尉,兼侍中、河南副元帥,知河南、淮南東西、山南東、荊南五道節度行營事,鎮泗州。帝為賦詩以餞。



 朝義乘邙山之捷,進略申、光等十三州,光弼輿疾就道,監軍使以兵少,請保揚州。光弼曰:「朝廷以安危寄我,賊安知吾眾寡?若出不意,當自潰。」遂疾驅入徐州。時朝義圍李岑於宋州,使田神功擊走之。初,神功平劉展,逗留淮南,尚衡、殷仲卿相攻兗、鄆間,來瑱擅襄陽,及光弼至屯,朝義走,神功還河南,瑱、衡、仲卿踵入朝,其為諸將憚服類此。寶應元年,進封臨淮郡王。光弼收許州,斬賊贏千級,縛偽將二十二人。朝義分兵攻宋州,光弼破走之。浙東賊袁晁反臺州,建元寶勝,以建丑為正月,殘剽州縣。光弼遣麾下破其眾於衢州。廣德元年,遂禽晁,浙東平。詔增實封戶二千,與一子三品階,賜鐵券,名藏太廟,圖形凌煙閣。



 相州、北邙之敗,朝恩羞其策繆,故深忌光弼切骨,而程元振尤疾之。二人用事,日謀有以中傷者。及來瑱為元振讒死,光弼愈恐。吐蕃寇京師,代宗詔入援,光弼畏禍,遷延不敢行。及帝幸陜,猶倚以為重,數存問其母,以解嫌疑。帝還長安,因拜東都留守,察其去就。光弼以久須詔書不至,歸徐州收租賦為解。帝令郭子儀自河中輦其母還京。二年,光弼疾篤,奉表上前後所賜實封,詔不許。將吏問後事,答曰:「吾淹軍中,不得就養,為不孝子,尚何言哉!」取所餘絹布分遺部將。薨,年五十七。部將即以其布遂為光弼行喪,號哭相問。帝遣使吊恤其母,贈太保,謚曰武穆,詔百官送葬延平門外。



 光弼用兵,謀定而後戰,能以少覆眾。治師訓整,天下服其威名,軍中指顧,諸將不敢仰視。初,與郭子儀齊名,世稱「李郭」,而戰功推為中興第一。其代子儀朔方也,營壘、士卒、麾幟無所更,而光弼一號令之,氣色乃益精明云。



 子匯,有志操,廉介自將。從賈耽為裨將,奏兼御史大夫。元和初,分徐州苻離為宿州,光弼有遺愛,擢匯為刺史。後遷涇原節度使,罷軍中雜徭,出奉錢贖將士質賣子,還其家。卒,贈工部尚書。光弼弟光進,字太應。初為房琯裨將,將北軍戰陳濤斜,兵敗,奔行在,肅宗宥之。代宗即位,拜檢校太子太保,封涼國公。吐蕃入寇,至便橋,郭子儀為副元帥,光進及郭英乂佐之。自至德後與李輔國並掌禁兵,委以心膂。光弼被譖,出為渭北、邠寧節度使。永泰初,封武威郡王。累遷太子太保,卒。母李,有須數十,長五寸許,封韓國太夫人,二子節制皆一品。死葬長安南原,將相奠祭凡四十四幄,時以為榮。



 光弼所部將李懷光、僕固懷恩、田神功、李抱玉、董秦、哥舒曜、韓游環、渾釋之、辛京杲自有傳。若荔非元禮、郝廷玉、李國臣、白孝德、張伯儀、白元光、陳利貞、侯仲莊、柏良器,皆章章可稱列者,附次左方。



 荔非元禮起裨將,累兼御史中丞。光弼守河陽,周摯攻北城,光弼方壁中水單,摯聞,並兵從光弼。光弼使元禮守羊馬城,植小旗城東北隅,望摯軍。摯恃眾,直逼城,以車千乘載木鵝橦車,麾兵填塹,八道並進。光弼諭元禮曰:「中丞視賊過兵不顧,何也?」報曰:「公欲守邪?戰歟?」光弼曰:「戰。」曰:「方戰,賊為我實塹,復何怪?」光弼曰:「吾慮不及此,公勉之。」元禮遂出戰,摯軍小卻。元禮以敵堅,未可以馳,還軍示弱,怠其意。光弼怒,使召元禮,欲按軍法。答曰:「方戰,不及往,請破賊以見。」因休柵中,良久,顧麾下曰:「向公來召,殆欲斬我。鬥死有名,無庸受戮。」乃下馬持刀,瞋目直前,銳士堵而進,左右奮擊,一當數人,斬賊數百首,摯遁去。以功累遷驃騎大將軍、懷州刺史,知鎮西、北庭行營節度使。上元二年,光弼進收洛陽,軍敗,元禮徙軍翼成,為麾下所害。



 郝廷玉驍勇善格鬥,為光弼愛將。及保河陽,禽徐璜玉,功為多。累封安邊郡王,授神策將軍。吐蕃犯京畿,與馬璘屯中渭橋。它日,魚朝恩聞其善布陣,請觀之。廷玉申號令,鳴鼓角,部伍坐作進退若一。朝恩嘆曰:「吾處兵間久,今始識訓練法。」廷玉惻然曰:「此臨淮王遺法也。王善御軍,賞當功,罰適過,每校旗,不如令者輒斬。由是人皆自效,而赴蹈馳突,心破膽裂。自臨淮歿,無復校旗事,此安足賞哉?」累為秦州刺史。卒,贈工部尚書。



 李國臣,河西人,本姓安。力能抉關,以折沖從收魚海五城,遷中郎將。後為朔方將,積勞擢雲麾大將軍,賜姓李。從光弼守河陽,累封臨川郡王。大歷八年,為鹽州刺史。吐蕃敗渾瑊於黃菩原,將略汧、隴,國臣謂人曰:「虜乘勝,必擾京師,我趨秦原,彼當反顧。」乃引兵登安樂山,鳴鼓而西,日行三十里。吐蕃聞之,自百里城回軍,逾險,瑊因擊敗之。卒,贈揚州大都督。



 白孝德,安西人,事光弼為偏裨。史思明攻河陽,使驍將劉龍仙以騎五十挑戰,加右足馬鬣上,嫚罵光弼。光弼登城顧諸將曰:「孰能取是賊?」僕固懷恩請行,光弼曰:「是非大將所宜。」左右以孝德對。召問所須幾兵,對曰:「願出五十騎,見可而進,大軍鼓噪以張吾氣,足矣。」光弼撫其背遣之。孝德擁二矛,策馬絕河,半濟,懷恩賀曰:「事克矣。其攬轡便闢,可萬全者。」龍仙見,易之,不為動。將至,若引避然,孝德振手止之曰:「侍中使致辭,無它。」與語須之,瞋目曰:「賊識我乎?我,白孝德也。」龍仙罵之,乃躍馬前搏,城上因大噪,五十騎繼進,龍仙環堤走,追斬其首以還。後累功至北庭行營節度使,徙邠寧。僕固懷恩引吐蕃兵入寇,孝德擊敗之。永泰初,吐蕃、回紇圍涇陽,郭子儀說回紇約盟,吐蕃退走,子儀使渾瑊以兵五千出奉天,命孝德應之,大戰赤沙烽,斬獲甚眾。累封昌化郡王,歷太子少傅。建中元年卒,贈太保。



 張伯儀,魏州人,以戰功隸光弼軍。浙賊袁晁反,使伯儀討平之,功第一,擢睦州刺史。後為江陵節度使。樸厚不知書,然推誠遇人,軍中畏肅,民亦便之。李希烈反,詔與賈耽、張獻甫收安州。戰不利,伯儀中流矢,師卻,失所持節。賊追及,奮刀以御之,兩刃相向不得下,會救至,免。至漢水,拿野人船以達沔州。潰兵至江陵,哭於廷,伯儀妻勞勉,出其家帛給之,乃定。伯儀收散卒還。久之,除右龍武統軍。卒,贈揚州大都督。既請謚,博士李吉甫議以「中興三十年而兵未戢者,將帥養寇籓身也。若以亡敗為戒,則總干戈者必圖萬全,而不決戰。若伯儀雖敗,而其忠可錄。」遂謚曰恭。



 白元光,字元光,其先突厥人。父道生,歷寧、朔州刺史。元光初隸本軍,補節度先鋒。安祿山反,詔徙朔方兵東討,元光領所部結義營,長驅從光弼出土門。累遷太子詹事,封南陽郡王,為兩都游弈使。



 長安平,率兵清宮,進擊餘寇,身被數創,肅宗躬為傅藥。轉衛尉卿,兼朔方先鋒。史思明攻河陽,光弼召主騎軍。其後歷靈武留後、定遠城使。貞元二年卒,贈越州都督。



 陳利貞,幽州範陽人。初為平盧將,安祿山亂,從光弼軍河南。張巡被圍睢陽也,光弼遣郝廷玉及利貞救之,輕騎出入,廷玉稱為勝己,以子妻之。及歸,薦於光弼,自行間累遷檢校太子賓客,封靜戎郡王。李希烈叛,詔哥舒曜東討,利貞為前鋒,次郟城。賊眾大集,利貞出奇兵五百,橫搗其右,賊鋒詘,數月不敢前。及希烈攻曜襄城,利貞登陴捍守,七十日未嘗櫛沐,非議事不下城。硃泚反,利貞及張廷芝所統士皆幽、薊、河、隴人,故與廷芝合謀應泚,而利貞麾下亦從為亂。夜半,難作,利貞拔劍當軍門,大呼曰:「欲過門者,先殺我!」眾畏其勇,乃止。廷芝出奔。德宗嘉之,擢汝州防禦使。貞元五年,疽發首,卒。遺觀察使崔縱書,自陳受國恩,恨不得死所云。



 侯仲莊,字仲莊,蔚州人。為光弼先鋒,授忠武將軍。禽安太清有功,累加冠軍將軍。僕固懷恩以朔方反,仲莊為都將,訓兵自守,號為「平射」,人畏其鋒。懷恩敗,郭子儀代之,引為腹心。封上谷郡王,為神策京西將。德宗幸奉天,遷左衛將軍,為防城使。修壘堞,晝夜執戈徼巡。從幸興元,殿軍駱谷,授防禦招收使。帝還都,復鎮奉天,幾二十年。卒,贈洪州都督。



 柏良器,字公亮,魏州人。父造,以獲嘉令死安祿山難。乃學擊劍,欲報賊。父友王奐為光弼從事,見之曰:「爾額文似臨淮王,面黑子似顏平原,殆能立功。」乃薦之光弼。授兵平山越,遷左武衛中郎將。以部兵隸浙西,豫平袁晁、方清。其後潘獰虎、胡參分據小傷、蒸里,又擊破之。是時年二十四,更戰陣六十二。



 李希烈圍寧陵,遏水灌之,親令軍中明日拔城。良器以救兵至,擇弩手善游者,沿汴渠夜入,及旦,伏弩發,賊乘城者皆死。錄功封平原郡王,入為左神策軍大將軍、知軍事,圖形凌煙閣。募材勇以代士卒市販者,中尉竇文場惡之,坐友人闌入,換右領軍衛。自是軍政皆中官專之。終左領軍衛大將軍,贈陜州大都督。子耆,別傳。



 烏承玼,字德潤,張掖人。開元中,與族兄承恩皆為平盧先鋒,沉勇而決,號「轅門二龍」。契丹可突於殺其王邵固降突厥,而奚亦亂,其王魯蘇挈族屬及邵固妻子自歸。是歲,奚、契丹入寇,詔承玼擊之,破於捺祿山。二十二年,詔信安王禕率幽州長史趙含章進討,承玼請含章曰:「二虜固劇賊,前日戰而北,非畏我,乃誘我也。公宜畜銳以折其謀。」含章不信,戰白城,果大敗。承玼獨按隊出其右,斬首萬計,可突於奔北奚。



 渤海大武藝與弟門藝戰國中,門藝來,詔與太僕卿金思蘭發範陽、新羅兵十萬討之,無功。武藝遣客刺門藝於東都,引兵至馬都山,屠城邑。承玼窒要路,塹以大石,亙四百里,虜不得入。於是流民得還,士少休,脫鎧而耕,歲省度支運錢。安慶緒使史思明守範陽,思明恃兵強,為自固計。慶緒密遣阿史那承慶、安守忠就督事,且圖之。承玼勸思明曰:「唐家中興,與天下更始,慶緒偷肆晷刻,公殆與俱亡。有如束身本朝,湔洗前污,此反掌功耳。」思明善之,斬承慶等,奉表聽命。始,承恩為冀州刺史,失守,思明護送東都,故肅宗使自雲中趨幽州開說思明,與承玼謀投釁殺之,不克,死。承玼奔李光弼,表為冠軍將軍,封昌化郡王,為石嶺軍使。王思禮為節度使,軍政倚辦焉。久之,移疾還京師,卒,年九十六。子重胤,別傳。



 贊曰:李光弼生戎虜之緒,沉鷙有守。遭祿山變,拔任兵柄,其策敵制勝不世出,賞信罰明,士卒爭奮,毅然有古良將風。本夫終父喪不入妻室,位王公事繼母至孝,好讀班固《漢書》,異夫庸人武夫者。及困於口舌,不能以忠自明,奄侍內構,遂陷嫌隙,謀就全安,而身益危,所謂工於料人而拙於謀己邪。方攘袂徇國,天下風靡;一為遷延,而田神功等皆不受約束,卒以憂死。功臣去就,可不慎邪?嗚呼,光弼雖有不釋位之誅,然讒人為害,亦可畏矣,將時之不幸歟!



\end{pinyinscope}