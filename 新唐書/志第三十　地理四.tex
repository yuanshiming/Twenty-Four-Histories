\article{志第三十 地理四}

\begin{pinyinscope}

 山南道,蓋古荊、梁二州之域,漢南郡、武陵、巴郡、漢中、南陽及江夏、弘農、廣漢、武都郡地。江陵、峽、歸、夔、澧、朗、復、郢、襄、房為鶉尾分,鄧、隋、泌、均為鶉火分,興元、金、洋、鳳、興、成、文、扶、利、集、壁、巴、蓬、通、開、忠、萬、涪、閬、果、渠為鶉首分。為府二有《直講李先生文集》。參見「經濟」中的「李覯」。,州三十三,縣百六十一。其名山:嶓塚、熊耳、銅梁、巫、荊、峴。其大川:巴、漢、沮、淯。厥賦:絹、布、綿紬。厥貢:金、絲、紵、漆。



 江陵府江陵郡,本荊州南郡,天寶元年更郡名。肅宗上元元年號南都,為府。二年罷都,是年又號南都。尋罷都。土貢:方紋綾、貲布、柑、橙、橘、椑、白魚、糖蟹、梔子、貝母、覆盆、烏梅、石龍芮。戶三萬三百九十二,口十四萬八千一百四十九。縣八:有府一,曰羅含。有永安軍,乾元二年置。江陵,次赤。貞觀十七年省安興縣入焉。貞元八年,節度使嗣曹王皋塞古堤,廣良田五千頃,畝收一鐘。又規江南廢洲為廬舍,架江為二橋。荊俗飲陂澤,乃教人鑿井,人以為便。枝江,次畿。上元元年析江陵置長寧縣,二年省枝江入長寧。大歷六年復置枝江,省長寧。當陽,次畿。武德四年置平州,並析置臨沮縣。六年曰玉州。八年州廢,省臨沮,以當陽來屬。有南紫蓋山、北紫蓋山。長林,次畿。武德四年於東境置基州,並置章山縣。七年州廢,以章山隸郢州。郢州廢,來屬。八年省章山入長林。石首,次畿。武德四年置。松滋,次畿。公安,次畿。荊門。次畿。貞元二十一年析長林置。



 峽州夷陵郡,中。本治下牢戍,貞觀九年徙治步闡壘。土貢:紵葛、箭竹、柑、茶、蠟、芒硝、五加、杜若、鬼臼。戶八千九十八,口四萬五千六百六。縣四:夷陵,上。西北二十八里有下牢鎮。有黃牛山。宜都,中下。本宜昌,隸南郡。武德二年更名,以宜都及峽州之夷道置江州,六年曰東松州。貞觀八年州廢,省入宜都,來屬。長陽,中下。本隸南郡。武德四年以縣置睦州,並置巴山、鹽水二縣。八年州廢,省鹽水,以長陽、巴山隸東松州。州廢,來屬。天寶八載省巴山入長陽。遠安。中下。有神馬山,本白馬山,天寶元年更名。



 歸州巴東郡,下。武德二年析夔州之秭歸、巴東置。土貢:紵葛、茶、蜜、蠟。戶四千六百四十五,口二萬三千四百一十七。縣三:秭歸,中。有鹽;東南八十五里有太清鎮城。巴東,中下。有鹽,有鐵。興山。中下。武德三年析秭歸置。



 夔州雲安郡,下都督府。本信州巴東郡,武德二年更州名,天寶元年更郡名。土貢:紵錫布、熊、羆、山雞、茶、柑、橘、蜜、蠟。戶萬五千六百二十,口七萬五千。縣四:有府一,曰東陽。奉節,上。本人復,貞觀二十三年更名。有鐵。有永安井鹽官。雲安,上。有鹽官。巫山,中。有巫山。大昌。下。有鹽官。



 澧州澧陽郡,上。土貢:紋綾、紵綀縛巾、犀角、竹簟、光粉、柑、橘、恆山、蜀漆。戶萬九千六百二十,口九萬三千三百四十九。縣四:澧陽,望。有關山。安鄉,中。貞觀元年省孱陵縣入焉。石門,中。有鐵。慈利。中下。武德中置崇義縣,麟德元年省入焉。本故崇州。



 朗州武陵郡,下。土貢:葛、紵綀簟、柑、犀角。戶九千三百六,口四萬三千七百六十。縣二:武陵,上。北有永泰渠,光宅中,刺史胡處立開,通漕,且為火備;西北二十七里有北塔堰,開元二十七年,刺史李璡增修,接古專陂,由黃土堰注白馬湖,分入城隍及故永泰渠,溉田千餘頃;東北八十九里有考功堰,長慶元年,刺史李翱因故漢樊陂開,溉田千一百頃;又有右史堰,二年,刺史溫造增修,開後鄉渠,經九十七里,溉田二千頃;又北百一十九里有津石陂,本聖歷初,令崔嗣業開,翱、造亦從而增之,溉田九百頃。翱以尚書考功員外郎,造以起居舍人,出為刺史,故以官名。東北八十里有崔陂,東北三十五里有槎陂,亦嗣業所修以溉田,後廢。大歷五年,刺史韋夏卿復治槎陂,溉田千餘頃。十三年以堰壞遂廢;有枉山。龍陽。中上。



 忠州南賓郡,下。本臨州,義寧二年析巴東郡之臨江置,貞觀八年更名。土貢:生金、綿紬、蘇薰席、文刀。戶六千七百二十二,口四萬三千二十六。縣五:臨江,中下。有鹽。豐都,中下。義寧二年析臨江置。南賓,中下。武德二年析浦州之武寧置。有鐵。墊江,中下。桂溪。中下。本清水,武德二年析臨江置,天寶元年更名。



 涪州涪陵郡,下。武德元年以渝州之涪陵鎮置。土貢:麩金、文刀、獠布、蠟。戶九千四百,口四萬四千七百二十二。縣五:涪陵,中下。武德二年置,並置武龍縣。又析涪陵、巴縣地置永安縣。開元二十二年省永安入樂溫。賓化,下。本隆化,貞觀十一年置,先天元年更名。武龍,中下。樂溫,中下。武德二年析巴縣地置,隸南潾州,九年來屬。溫山。下。本隸南潾州,後來屬。



 萬州南浦郡,下。本南浦州,武德二年析信州置。八年州廢,以南浦、梁山隸夔州,武寧隸臨州。九年復置,曰浦州。貞觀八年更名。土貢:麩金、藥子。戶五千一百七十九,口二萬五千七百四十六。縣三:南浦,中。有塗祇監、漁陽監,鹽官二。武寧,中下。梁山。中下。



 襄州襄陽郡,望。土貢:綸巾,漆器,庫路真二品:十乘花文、五乘碎石文,柑,蔗,芋,姜。戶四萬七千七百八十,口二十五萬二千一。縣七:有府一,曰漢津。襄陽,望。貞觀八年省常平縣入焉。有峴山。鄧城,緊。本安養,天寶元年曰臨漢,貞元二十一年更名。穀城,上。武德四年以穀城、陰城置酂州,五年州廢,二縣來屬。貞觀八年省陰城入焉。有薤山。義清,中。貞觀八年省南漳入焉,南漳本臨沮。南漳,中下。本荊山,武德二年析南漳置,以縣置重州,並置重陽、平陽、渠陽、土門、歸義五縣。七年省渠陽入荊山,省平陽入重陽,省土門、歸義入房州之永清。貞觀元年州廢,以荊山來屬,徙重陽於故重州,隸遷州。八年省重陽入荊山。開元十八年徙於故南漳,因改名。有荊山。樂鄉,中下。本隸竟陵郡,武德四年以樂鄉及襄州之率道、上洪置鄀州。貞觀元年又領長壽,省上洪。八年州廢,以長壽隸溫州,樂鄉、率道來屬。宜城。上。本率道,貞觀八年省漢南縣入焉,天寶七載更名。有石梁山、陰山。



 泌州淮安郡,上。本昌州舂陵郡,治棗陽。武德五年以唐城山更名唐州,九年徙治比陽。天寶元年更郡名。天祐三年,硃全忠徙治泌陽,表更名。土貢:絹、布。戶四萬二千六百四十三,口十八萬二千三百六十四。縣七:泌陽,中。本上馬,貞觀元年省入湖陽,開元十三年復置,天寶元年更名。比陽,上。本淮安郡治,武德四年曰顯州,領比陽、慈丘、平氏、顯岡、桐柏五縣。貞觀二年省顯岡。九年州廢,縣皆來屬。慈丘,上。桐柏,中。武德初置純州,貞觀元年州廢,來屬。有桐柏山;有淮瀆祠。平氏,中。有祈中山。湖陽,中。武德四年以縣置湖州,貞觀元年州廢,來屬。有蓼山。方城。上。本淯陽郡治。武德二年曰北澧州,領方城、真昌二縣。貞觀元年省真昌。八年曰魯州,九年州廢,以方城來屬。



 隋州漢東郡,上。土貢:合羅、綾、葛、覆盆。戶二萬三千九百一十七,口十萬五千七百二十二。縣四:隋,上。武德四年省安貴縣入焉。五年省平林、順義縣入焉。光化,上。棗陽,上。本隸唐州。武德五年省唐州之清漳縣入焉。貞觀元年又省唐州之舂陵縣入焉。十年以棗陽來屬。有光武山。唐城。上。開元二十六年以客戶析棗陽地置。



 鄧州南陽郡,上。土貢:絲布、茅菊。戶四萬三千五十五,口十六萬五千二百五十七。縣六:穰,望。武德四年析置平晉縣,以新野置新州,尋廢新州,以新野來屬。六年省平晉縣。又領深陽縣,貞觀元年省,乾元元年省新野,皆入焉。南陽,緊。武德三年以南陽及舂陵郡之上馬置宛州,並置雲陽、上宛、安固三縣。八年州廢,以上馬隸唐州,省雲陽、上宛,以安固入南陽,來屬。聖歷元年曰武臺。神龍初復故名。有銅。向城,上。武德三年以縣置淯州。八年州廢,隸北澧州,州廢,來屬。聖歷元年曰武清。神龍初復故名。北八十里有魯陽關。臨湍,上。本新城。武德二年以縣置酈州,八年州廢,來屬。貞觀元年省冠軍縣入焉。天寶元年更名。又有順陽縣,武德二年析冠軍置,六年省。內鄉,上。本淅陽郡治。武德二年曰淅州,並置默水縣。貞觀八年州廢,省默水入內鄉,來屬。有岵山。菊潭。中。開元二十四年析新城置。



 均州武當郡,下。義寧二年析淅陽郡之武當、均陽置。貞觀元年州廢,二縣隸淅州。八年以武當、鄖鄉復置。土貢:山雞尾、麝香。戶九千六百九十八,口五萬八百九。縣三:有府一,曰至誠。武當,上。義寧二年析置平陵縣,武德七年省,八年省均陽入焉。東南百里有鹽池。有武當山。鄖鄉,上。本隸淅陽郡。武德元年以鄖鄉、安福置南豐州,並置堵陽、黃沙、白沙、固城四縣。八年省黃沙、白沙、固城,是年州廢,以鄖鄉、安福、堵陽隸淅州。貞觀元年省安福、堵陽入焉。有精舍山,本獨山,天寶中更名。豐利。上。有伏龍山;有錫義山,一名天心山。



 房州房陵郡,上。武德元年析遷州之竹山、上庸置。貞觀十年徙治房陵。土貢:蠟、蒼礬、麝香、鐘乳、雷丸、石膏、竹綍。戶萬四千四百二十二,口七萬一千七百八。縣四:房陵,上。本光遷,房陵郡治,武德元年曰遷州,並析置受陽、淅川、房陵三縣。五年省淅川。七年省房陵、受陽。貞觀十年州廢,來屬,更光遷曰房陵。永清,中下。本隸遷州,州廢,來屬。有房山。竹山,中下。武德元年析置武陵縣,貞觀十年省。上庸。上。



 復州竟陵郡,上。本沔陽郡,治竟陵。貞觀七年徙治沔陽。天寶元年更名。寶應二年復故治。土貢:白紵、白蜜。戶八千二百一十,口四萬四千八百八十五。縣三:沔陽,上。竟陵,上。有五花山;有石堰渠,咸通中,刺史董元素開。監利。中下。



 郢州富水郡,上。本竟陵郡,治長壽。貞觀元年州廢,以長壽隸鄀州,十七年復置,治京山,後還治長壽。土貢:紵布、葛、蕉、春酒麴、棗、節米。戶萬二千四十六,口五萬七千三百七十五。縣三:長壽,上。貞觀元年省藍水縣入焉。京山,上。本隸安州。武德四年以京山、富水二縣置溫州,貞觀十七年州廢,縣皆來屬。富水。上。有白沙山。



 金州漢陰郡,上。本西城郡,天寶元年曰安康郡,至德二載更名。土貢:麩金、茶牙、椒、乾漆、椒實、白膠香、麝香、杜仲、雷丸、枳殼、枳實、黃蘗。有橘官。戶萬四千九十一,口五萬七千九百二十九。縣六:有府一,曰洪義。西城,上。本金川,義寧二年更名。有牛山。漢水有金。洵陽,中下。武德元年以縣置洵州,並置洵城、驢川二縣,七年州廢,縣皆來屬。貞觀二年省驢川,八年省洵城。東有申口鎮城。淯陽,上。本黃土,天寶元年更名,大歷六年省入洵陽,長慶初復置。石泉,中下。聖歷元年曰武安,神龍元年復故名,大歷六年省入漢陰,永貞元年復置。漢陰,中下。本安康。武德元年以縣置西安州,並置寧鬱、廣德二縣。二年曰直州。貞觀元年州廢,省寧鬱,以廣德入安康,來屬。至德二載更名。西有方山關,貞觀十二年置。月川水有金。平利。中下。武德元年以故吉安置,大歷六年省入西城,長慶初復置。有女媧山。



 右東道採訪使,治襄州。



 興元府漢中郡,赤。本梁州漢川郡,開元十三年以「梁」「涼」聲相近,更名褒州,二十年復曰梁州,天寶元年更郡名,興元元年為府。土貢:縠、蠟、紅藍、燕脂、夏蒜、冬筍、糟瓜、柑、枇杷、茶。戶三萬七千四百七十,口十五萬三千七百一十七。縣五:有府一,曰麗水。南鄭,次赤。有旱山、玉女山、中梁山。褒城,次畿。義寧二年更名褒中。貞觀三年復故名。有牛頭山;北有甘寧關。城固,次畿。武德二年更名唐固,三年析置白雲縣,九年省。貞觀二年復故名。西,次畿。武德三年以縣置褒州,析利州之綿谷置金牛縣,八年州廢,二縣來屬。寶歷元年省金牛縣入焉。西南有百牢關。有錫,有鐵。三泉。次畿。武德四年析利州之綿谷置,以縣置南安州,並置嘉牟縣。八年州廢,省嘉牟,以三泉隸利州。天寶元年來屬。



 洋州洋川郡,雄。武德元年析梁州之西鄉、黃金、興勢置,天寶十五載徙治興道。土貢:白交梭、火麻布、野苧麻、蠟、白膠香、麝香。戶二萬三千八百四十九,口八萬八千三百二十七。縣四:興道。緊。本興勢,貞觀二十三年更名。有駱谷路,南口曰儻谷,北口曰駱谷。西鄉,上。武德四年析置洋源縣,寶歷元年省。有雲亭山。黃金,中。有子午谷路。真符。中。本華陽,開元十八年析興道置。天寶三載省。八載開清水谷路,復置,因鑿山得玉冊,更名,隸京兆府。十一載來屬。有太白山、金星洞。



 利州益昌郡,下都督府。本義城郡,天寶元年更名。土貢:金、絲布、粱米、蠟燭、鯄魚、天門冬、芎藭、麝香。戶萬三千九百一十,口四萬四千六百。縣六:綿谷,上。有鐵。葭萌,上。益昌,中下。嘉川,中下。胤山,中下。本義城,義寧二年曰義清。武德七年以義清、岐坪、隆州之奉國置西平州。貞觀二年州廢,以義清來屬,岐坪、奉國隸閬州。天寶元年更名。景谷。中下。武德四年以景谷及龍州之方維置沙州。貞觀元年州廢,省方維為鎮,以景谷來屬。寶歷元年省,尋復置。西有石門關。西北有白壩、魚老二鎮城。



 鳳州河池郡,下。土貢:布、蠟燭、麝香。戶五千九百一十八,口二萬七千八百七十七。縣三:有府一,曰歸昌。梁泉,中下。武德元年析置黃花縣,寶歷元年省。有銀,有鐵。兩當,中下。有銀。河池。中下。



 興州順政郡,下。土貢:蠟、漆、丹沙、蜜、筍。戶二千二百二十四,口萬一千四十六。縣二:順政,中。有鐵,南有興城關。長舉。中下。元和中,節度使嚴礪自縣而西疏嘉陵江三百里,焚巨石,沃醯以碎之,通漕以饋成州戍兵。州又領鳴水縣,長慶元年省入焉。有鐵。



 成州同谷郡,下。本漢陽郡,治上祿,天寶元年更名,寶應元年沒吐蕃,貞元五年,於同谷之西境泥公山權置行州,咸通七年復置,徙治寶井堡,後徙治同谷。土貢:蠟燭、麝香、鹿茸、防葵、狼毒。戶四千七百二十七,口二萬一千五百八。縣三:有府一,曰平陰。有靜戎軍,寶應元年徙馬邑州於鹽井城置。同谷,中下。武德元年以縣置西康州,貞觀元年州廢,來屬,咸通十三年復置。上祿,中。沒蕃後廢。有仇池山;有鹽。漢源。中下。沒蕃後廢。



 文州陰平郡,下。義寧二年析武都郡之曲水、正西、長松置。土貢:麩金、紬、綿、麝香、白蜜、蠟燭、柑。戶千九百八,口九千二百五。縣一:曲水。中下。貞觀元年省正西縣,貞元六年省長松縣,皆來屬。



 扶州同昌郡,下。乾元後沒吐蕃,大中二年,節度使鄭涯收復。土貢:麝香、當歸、芎藭。戶二千四百一十八,口萬四千二百八十五。縣四:有府二,曰安川、會川。同昌,中下。帖夷,中下。萬歲通天二年曰武進,神龍元年復故名。萬全,中下。本尚安,至德二年更名。鉗川。中下。



 集州符陽郡,下。武德元年,析梁州之難江,巴州之符陽、長池、白石置。土貢:蠟燭、藥子。戶四千三百五十三,口二萬五千七百二十六。縣三:難江,上。武德九年析置平桑縣,貞觀元年省,二年復置,六年省,又省長池縣入焉。大牟,下。武德二年徙靜州治狄平,更狄平曰地平。十七年廢靜州,以大牟、清化隸巴州,地平來屬。永泰元年以大牟隸集州,更地平曰通平,寶歷元年省。嘉川。下。本隸利州,貞觀二年隸靜州,州廢,還隸利州,永泰元年來屬。



 壁州始寧郡,下。武德八年析巴州之始寧縣地置。土貢:紬、綿、馬策。戶萬三千三百六十八,口五萬四千七百五十七。縣五:通江,上。本諾水,隸萬州。武德中省,八年又析巴州之始寧復置,天寶元年更名。廣納,中。武德三年析始寧、歸仁置,寶歷元年省,大中初復置。符陽,中。本隸清化郡,武德元年隸集州,八年來屬,貞觀八年復隸集州,長安三年來屬,景雲二年又隸集州,永泰元年來屬。白石,中。本隸清化郡,武德元年隸集州,八年來屬。東巴。中。本太平,開元二十三年置,天寶元年更名。



 巴州清化郡,中。土貢:麩金、綿、紬、貲布、花油、橙、石蜜。戶三萬二百一十,口九萬一千五十七。縣九:化城,上。盤道,中下。寶歷元年省入恩陽,長慶中復置。清化,上。武德元年置靜州,又析置大牟、狄平二縣。曾口,中。歸仁,中。始寧,中。其章,中。寶歷元年省,大中元年復置。恩陽,中。貞觀十七年省,萬歲通天元年復置。七盤。上。久視元年置。



 蓬州蓬山郡,下。本咸安郡,武德元年,以巴州之安固、伏虞,隆州之儀隴、大寅,渠州之宕渠、咸安置,開元二十九年徙治大寅,至德二載更郡名。土貢:綿、紬。戶萬五千五百七十六,口五萬三千三百五十三。縣七:蓬池,中。本大寅,廣德元年更名,後省,開成元年復置。良山,中。本安固,天寶元年更名,寶歷元年省入蓬池,大中中復置。儀隴,中。武德三年以縣置方州。八年州廢,還隸蓬州。伏虞,中。宕渠,中下。寶歷元年省入蓬山,大中中復置。蓬山,上。本咸安,至德二載更名。朗池。中。武德四年析果州之相如縣置,隸果州,寶應元年來屬。寶歷元年省,開成二年復置。



 通州通川郡,上。土貢:紬、綿、蜜、蠟、麝香、楓香、白藥實。戶四萬七百四十三,口十一萬八百四。縣九:通川,上。武德二年置思來縣,貞觀元年省入焉。永穆,上。本隸巴州。武德二年以永穆及歸仁置萬州,又置諾水、廣納、太平、恆豐四縣,七年省諾水。貞觀元年州廢,以歸仁還隸巴州,廣納隸壁州,省太平、恆豐入永穆,來屬。三岡,中。寶歷元年省,大中五年復置。石鼓,中。寶歷元年省,大中元年復置。東鄉,中。武德三年置南石州,又置下蒲、昌樂二縣。八年州廢,省昌樂入石鼓,下蒲入東鄉,來屬。宣漢,中下。武德元年置南井州,並析置東關縣。貞觀元年州廢,省東關,以宣漢來屬。有鹽,有金。新寧,中下。武德二年析通川置。大和三年隸開州,四年來屬。巴渠,中。永泰元年析石鼓置,大和三年隸開州,四年來屬。閬英。中。天寶九載置。



 開州盛山郡,下。本萬世郡,義寧二年,析巴東郡之盛山、新浦,通川郡之萬世、西流置,天寶元年更名。土貢:白紵布、柑、芣絪實。戶五千六百六十,口三萬四百二十一。縣三:開江,上。本盛山,貞觀元年省西流縣入焉,廣德元年更名。新浦,中下。萬歲。中下。本萬世,貞觀二十三年更名,寶歷元年省,尋復置。有鹽。東南五里有靈洞,貞元九年雷雨震開。



 閬州閬中郡,上。本隆州巴西郡,先天二年避玄宗名更州名,天寶元年更郡名。土貢:蓮綾、綿、絹、紬、縠。戶二萬九千五百八十八,口十三萬二千一百九十二。縣九:閬中,緊。本閬內,武德四年更名,是年析置思恭縣,七年省。有靈山;有鹽。晉安,中。本晉城,武德中避隱太子名更。南部,上。有鹽。蒼溪,中下。有雲臺山、紫陽山。西水,中下。奉國,上。武德七年隸西平州,貞觀元年州廢,還隸隆州。新井,中。武德元年析南部、晉安置。有鹽。新政,中。本新城,武德四年析南部、相如置,避隱太子名更。有鹽。岐坪。中。本隸利州,開元二十三年來屬,寶歷元年省入奉國、蒼溪,天復中,王建表置。



 果州南充郡,中。武德四年析隆州之南充、相如置,大歷六年更名充州,十年復故名。土貢:絹、絲布。戶三萬三千六百四,口八萬九千二百二十五。縣五:南充,上。有鹽。相如,中。有鹽。流奚,中。開耀元年析南充置。西充,上。武德四年析南充置。有鹽。岳池。中。萬歲通天二年析南充、相如置。有龍扶速山。



 渠州潾山郡,下。本宕渠郡,天寶元年更名。土貢:紬、綿、藥實、買子本實。戶九千九百五十七,口二萬六千五百二十四。縣三:流江,上。武德元年析置義興縣,別置賨城縣,八年皆省。渠江,中。本賨城,武德元年曰始安,又析置豐樂縣,八年省。天寶元年更名。潾山。中下。武德元年析潾水置,以縣置潾州,並置鹽泉縣及渠州之潾水、墊江以隸之。三年以潾水來屬。八年州廢,以墊江隸忠州,潾山來屬。久視元年分蓬州之宕渠置大竹縣,隸蓬州。至德二載來屬。寶歷元年省潾水、大竹入潾山。有鐵。



 右西道採訪使,治梁州。



 隴右道,蓋古雍、梁二州之境,漢天水、武都、隴西、金城、武威、張掖、酒泉、燉煌等郡,總為鶉首分。為州十九,都護府二,縣六十。其名山:秦嶺、隴坻、鳥鼠同穴、硃圉、西傾、積石、合黎、崆峒、三危。其大川:河、洮、弱、羌、休屠之澤。厥賦:布、麻。厥貢:金屑、礪石、鳥獸、革角。自祿山之亂,河右暨西平、武都、合川、懷道等郡皆沒於吐蕃,寶應元年又陷秦、渭、洮、臨,廣德元年復陷河、蘭、岷、廓,貞元三年陷安西、北廷,隴右州縣盡矣。大中後,吐蕃微弱,秦、武二州浙復故地,置官守。五年,張義潮以瓜、沙、伊、肅、鄯、甘、河、西、蘭、岷、廓十一州來歸,而宣、懿德微,不暇疆理,惟名存有司而已。



 秦州天水郡,中都督府。本治上邽,開元二十二年以地震徙治成紀之敬親川,天寶元年還治上邽,大中三年復徙治成紀。土貢:龍須席、芎藭。戶二萬四千八百二十七,口十萬九千七百四十。縣六:有府六,曰成紀、脩德、清德、清水、三度、長川。成紀,上。有銀,有銅,有鐵。上邽,上。有嶓塚山。伏羌,中。本冀城。武德二年更名,是年,以伏羌及渭州之隴西置伏州,八年州廢,縣還故屬。九年析置鹽泉縣,貞觀元年更名夷賓,三年省。有石臼山、硃圉山。隴城,下。武德二年以縣置文州,八年州廢,來屬。貞觀三年置長川縣,六年省入焉。有銀。清水,下。武德四年以縣置邽州,六年州廢,來屬。又有秦嶺縣,貞觀十七年省。大中二年先收復,權隸鳳翔府,三年來屬。東五十里有大震關;有銀。長道。中下。本隸成州,天寶末廢,咸通十三年復置,來屬。有鹽。



 河州安昌郡,下。本枹罕郡,天寶元年更名。土貢:麝香。戶五千七百八十二,口三萬六千八十六。縣三:西百八十里有鎮西軍,開元二十六年置;西八十里索恭川有天成軍,西百餘里雕窠城有振威軍,皆天寶十三載置;西南四十里有平夷守捉城。枹罕,中下。有可藍關。大夏,中下。貞觀元年省入枹罕,三年復置。鳳林。中下。本烏州,貞觀七年置,十一年州廢,更置安昌縣,來屬,天寶元年更名。北有鳳林關,有積石山。



 渭州隴西郡,中都督府。土貢:龍須席、麝香、秦艽。戶六千四百二十五,口二萬四千五百二十。縣四:有府四,曰渭源、平樂、臨源、萬年。襄武,上。隴西,上。鄣,下。天授二年曰武陽,神龍元年復故名。南二里有鹽井。渭源。上。高宗上元二年更名首陽,於渭源故縣別置渭源縣。儀鳳三年省首陽入渭源。有鳥鼠同穴山,一名青雀山。



 鄯州西平郡,下都督府。土貢:牸犀角。戶五千三百八十九,口二萬七千一十九。縣三:星宿川西有安人軍,西北三百五十里有威戎軍;西南二百五十里有綏和守捉城,南百八十里有合川守捉城。湟水,中。龍支,中。肅宗上元二年,州沒吐蕃,以龍支、鄯城隸河州。鄯城。中。儀鳳三年置。有土樓山,有河源軍,西六十里有臨蕃城,又西六十里有白水軍、綏戎城;又西南六十里有定戎城。又南隔澗七里有天威軍,軍故石堡城,開元十七年置,初曰振武軍,二十九年沒吐蕃,天寶八載克之,更名;又西二十里至赤嶺,其西吐蕃,有開元中分界碑。自振武經尉遲川、苦拔海、王孝傑米柵九十里至莫離驛。又經公主佛堂、大非川二百八十里至那錄驛,吐渾界也。又經暖泉、烈謨海,四百四十里渡黃河,又四百七十里至眾龍驛。又渡西月河,二百一十里至多彌國西界。又經犛牛河度藤橋,百里至列驛。又經食堂、吐蕃村、截支橋,兩石南北相當,又經截支川,四百四十里至婆驛。乃度大月河羅橋,經潭池、魚池,五百三十里至悉諾羅驛。又經乞量寧水橋,又經大速水橋,三百二十里至鶻莽驛,唐使入蕃,公主每使人迎勞於此。又經鶻莽峽十餘里,兩山相崟,上有小橋,三瀑水注如瀉缶,其下如煙霧,百里至野馬驛。經吐蕃墾田,又經樂橋湯,四百里至閤川驛。又經恕諶海,百三十里至蛤不爛驛,旁有三羅骨山,積雪不消。又六十里至突錄濟驛,唐使至,贊普每遣使慰勞於此。又經柳谷莽布支莊,有溫湯,湧高二丈,氣如煙雲,可以熟米。又經湯羅葉遺山及贊普祭神所,二百五十里至農歌驛。邏些在東南,距農歌二百里,唐使至,吐蕃宰相每遣使迎候於此。又經鹽池、暖泉、江布靈河,百一十里渡姜濟河,經吐蕃墾田,二百六十里至卒歌驛。乃渡臧河,經佛堂,百八十里至勃令驛鴻臚館,至贊普牙帳,其西南拔布海。



 蘭州金城郡,下。以皋蘭山名州。土貢:麩金、麝香、渼鼠。戶二千八百八十九,口萬四千二百二十六。縣二:有府二,曰金城、廣武。又有榆林軍。五泉,下。咸亨二年更名金城,天寶元年復故名。北有金城關。金城。下。本廣武縣,乾元二年更名。



 臨州狄道郡,下都督府。天寶三載析金城郡之狄道縣置。縣二:有臨洮軍,久視元年置,寶應元年沒吐蕃。狄道,下。長樂。下。本安樂,天寶後置,乾元後更名。



 階州武都郡,下。本武州,因沒吐蕃,廢,大歷二年復置為行州,咸通中始得故地,龍紀初遣使招葺之,景福元年更名,治皋蘭鎮。土貢:麝香、蜜、蠟燭、山雞尾、羚羊角。戶二千九百二十三,口萬五千三百一十三。縣三:將利,中下。州又領建威縣,貞觀元年省入焉。福津,中下。本覆津,景福元年更名。盤堤。中下。沒蕃後不復置。



 洮州臨洮郡,下。本治美相,貞觀八年徙治臨潭。開元十七年州廢,以縣隸岷州,二十年復置,更名臨州,二十七年復故名。土貢:甘草、麝香。戶二千七百,口萬五千六十。縣一;有府一,曰安西。有莫門軍,儀鳳二年置;西八十里磨禪川有神策軍,天寶十三載置。臨潭。中。本美相。貞觀四年徙治洪和城,以故地置旭州。五年又置臨潭縣。八年州廢,以臨潭來屬,徙州來治,遷于洮陽城。十二年省博陵縣,天寶中省美相縣,皆入臨潭。西百六十里有廣恩鎮,有西傾山。



 岷州和政郡,下。義寧二年析臨洮郡之臨洮、和政置。土貢:龍須席、甘草。戶四千三百二十五,口二萬三千四百四十一。縣三:有府三,曰祐川、臨洮、和政。溢樂,中下。本臨洮,義寧二年更名,貞觀二年析置當夷縣,神龍元年省。有岷山,西有崆峒山。祐川,中下。本基城,義寧二年置,先天元年更名。和政。中。有闊博山。



 廓州寧塞郡,下。本澆河郡,天寶元年更名。土貢:麩金、酥、大黃、戎鹽、麝香。戶四千二百六十一,口二萬四千四百。縣三:西有寧邊軍,本寧塞軍;西八十里宛秀城有威勝軍;西南百四十里洪濟橋有金天軍,其東南八十里百穀城有武寧軍;南二百里黑峽川有曜武軍;皆天寶十三載置。廣威,下。本化隆,先天元年曰化成,天寶元年又更名。達化,下。西有積石軍,本靜邊鎮,儀鳳二年為軍;東有黃沙戍。米川。下。貞觀五年置,又以縣置米州,十年州廢,隸河州。永徽六年來屬。



 疊州合川郡,下。武德二年析洮州之合川、樂川、疊川置。土貢:麝香。戶千二百七十五,口七千六百七十四。縣二:有府一,曰長利。合川,下。武德五年以黨項戶置安化、和同二縣,尋省。貞觀二年省樂川、疊川入焉。有渭礱山。常芬。下。武德元年以縣置芳州,並置丹嶺縣。四年以丹嶺隸洮州。貞觀二年置恆香縣,僑治恆香戍,復以丹嶺隸芳州。高宗上元二年陷吐蕃,神龍元年州廢,省丹嶺、恆香,以常芬來屬。



 宕州懷道郡,下。本宕昌郡,天寶元年更名。土貢:麩金、散金、麝香。戶千一百九十,口七千一百九十九。縣二:有府二,曰同歸、常吉。懷道,下。貞觀三年省和戎縣入焉。西百八十三里有蘇董戍。有同均山。良恭。下。貞觀元年以成州之潭水來屬,後省入焉。



 涼州武威郡,中都督府。土貢:白綾、龍須席、毯、野馬革、芎藭。戶二萬二千四百六十二,口十一萬二百八十一。縣五:有府六,曰明威、洪池、番禾、武安、麗水、姑臧。又有赤水軍,本赤烏鎮,有赤青泉,因名之,幅員五千一百八十里,軍之最大也;西二百里有大斗軍,本赤水守捉,開元十六年為軍,因大斗拔谷為名;東南二百里有烏城守捉,南二百里有張掖守捉,西二百里有交城守捉;西北五百里有白亭軍,本白亭守捉,天寶十四載為軍。姑臧,中下。北百八十里有明威戍,西北百六十里有武安戍;有武興鹽池、黛眉鹽池。神烏,下。武德三年置,貞觀元年省,總章元年復置,曰武威,神龍元年復故名。昌松,中。東北百五十里有白山戍。天寶,中下。本番禾,咸亨元年以縣置雄州,調露元年州廢來屬,天寶三載以山出醴泉更名。有通化鎮,有焉支山。嘉麟。神龍二年於故漢鸞鳥縣城置,景龍元年省,先天二年復置。



 沙州敦煌郡,下都督府。本瓜州,武德五年曰西沙州,貞觀七年曰沙州,土貢:子、黃礬、石膏。戶四千二百六十五,口萬六千二百五十。縣二:有府三,曰龍勒、效穀、懸泉。有豆盧軍,神龍元年置。敦煌,下。東四十七里有鹽池,有三危山。壽昌。下。武德二年析敦煌置,永徽元年省,乾封二年復置,開元二十六年又省,後復置,治漢龍勒城。西有陽關,西北有玉門關;有雲雨山。



 瓜州晉昌郡,下都督府。武德五年析沙州之常樂置。土貢:野馬革、緊鞓、草豉、黃礬、絳礬、胡桐律。戶四百七十七,口四千九百八十七。縣二:有府一,曰大黃。西北千里有墨離軍。晉昌,中下。本常樂,武德四年更名。東北有合河鎮,又百二十里有百帳守捉,又東百五十里有豹文山守捉,又七里至寧寇軍,與甘州路合。常樂。中下。武德五年別置。有拔河帝山。



 甘州張掖郡,下。土貢:麝香,野馬革,冬柰,茍杞實、葉。戶六千二百八十四,口二萬二千九十二。縣二:西北百九十里祁連山北有建康軍,證聖元年,王孝傑以甘、肅二州相距回遠,置軍;西百二十里有蓼泉守捉城。張掖,上。有祁連山、合黎山。北九百里有鹽池,西有鞏筆驛。刪丹。中下。北渡張掖河,西北行出合黎山峽口,傍河東壖屈曲東北行千里,有寧寇軍,故同城守捉也,天寶二載為軍;軍東北有居延海,又北三百里有花門山堡,又東北千里至回鶻衙帳。



 肅州酒泉郡,下。武德二年析甘州之福祿、瓜州之玉門置。土貢:麩金、野馬革、蓯蓉、柏脈根。戶二千二百三十,口八千四百七十六。縣三:有酒泉、威遠二守捉城。酒泉,中下。本福祿,唐初更名。西十五里有興聖皇帝陵,七十里有洞庭山,出金;有昆侖山。福祿,下。武德二年別置。東南百二十里有祁連戍,東北八十里有鹽池。玉門。中下。貞觀元年省,後復置。開元中沒吐蕃,因其地置玉門軍,天寶十四載廢軍為縣。北有獨登山,出鹽,以充貢;有神雨山。



 伊州伊吾郡,下。本西伊州,貞觀六年更名。土貢:香棗、陰牙角、胡桐律。戶二千四百六十七,口萬一百五十七。縣三:西北三百里甘露川有伊吾軍,景龍四年置。伊吾,下。貞觀四年置,並置柔遠縣,神功元年省入焉。在大磧外,南去玉門關八百里,東去陽關二千七百三十里。有折羅漫山,亦曰天山;南二里有咸池海。納職,下。貞觀四年以鄯善故城置,開元六年省,十五年復置。南六十里有陸鹽池。自縣西經獨泉、東華、西華駝泉,渡茨萁水,過神泉,三百九十里有羅護守捉;又西南經達匪草堆,百九十里至赤亭守捉,與伊西路合。別自羅護守捉西北上乏驢領,百二十里至赤谷;又出谷口,經長泉、龍泉,百八十里有獨山守捉;又經蒲類,百六十里至北庭都護府。柔遠。下。



 西州交河郡,中都督府。貞觀十四年平高昌,以其地置。開元中曰金山都督府。天寶元年為郡。土貢:絲、芃布、氈、刺蜜、蒲萄五物酒漿煎皺乾。戶萬九千一十六,口四萬九千四百七十六。縣五:有天山軍,開元二年置。自州西南有南平、安昌兩城,百二十里至天山西南入谷,經礌石磧,二百二十里至銀山磧;又四十里至焉耆界呂光館;又經盤石百里,有張三城守捉;又西南百四十五里經新城館,渡淡河,至焉耆鎮城。前庭,下。本高昌,寶應元年更名。柳中,下。交河,中下。自縣北八十里有龍泉館,又北入穀百三十里,經柳谷,渡金沙嶺,百六十里,經石會漢戍,至北庭都護府城。蒲昌,中。本隸庭州,後來屬。西有古屯城、弩支城,有石城鎮、播仙鎮。天山。下。有天山。



 北庭大都護府,本庭州,貞觀十四年平高昌,以西突厥泥伏沙缽羅葉護阿史那賀魯部落置,並置蒲昌縣,尋廢,顯慶三年復置,長安二年為北庭都護府。土貢:陰牙角、速霍角、阿魏截根。戶二千二百二十六,口九千九百六十四。縣四:有瀚海軍,本燭龍軍,長安二年置,三年更名,開元中蓋嘉運增築;西七百里有清海軍,本清海鎮,天寶中為軍;南有神山鎮。自庭州西延城西六十里有沙缽城守捉,又有馮洛守捉;又八十里有耶勒城守捉,又八十里有俱六城守捉,又百里至輪臺縣,又百五十里有張堡城守捉,又渡里移得建河,七十里有烏宰守捉,又渡白楊河,七十里有清鎮軍城,又渡葉葉河,七十里有葉河守捉,又渡黑水,七十里有黑水守捉,又七十里有東林守捉,又七十里有西林守捉;又經黃草泊、大漠、小磧,渡石漆河,逾車嶺,至弓月城;過思渾川、蟄失蜜城,渡伊麗河,一名帝帝河,至碎葉界;又西行千里至碎葉城,水皆北流入磧及入夷播海。金滿,下。輪臺,下。有靜塞軍,大歷六年置。後庭,下。本蒲類,隸西州,後來屬,寶應元年更名。有蒲類、郝遮、咸泉三鎮,特羅堡。西海。下。寶應元年置。



 安西大都護府,初治西州。顯慶二年平賀魯,析其地置濛池、昆陵二都護府,分種落列置州縣,西盡波斯國,皆隸安西,又徙治高昌故地。三年徙治龜茲都督府,而故府復為西州。咸亨元年,吐蕃陷都護府。長壽二年收復安西四鎮。至德元載更名鎮西。後復為安西。土貢:罔砂、緋氈、偏桃人。吐蕃既侵河、隴,惟李元忠守北庭,郭昕守安西,與沙陀、回紇相依,吐蕃攻之久不下。建中二年,元忠、昕遣使間道入奏,詔各以為大都護,並為節度。貞元三年,吐蕃攻沙陀、回紇,北庭、安西無援,遂陷。有保大軍,屯碎葉城;於闐東界有蘭城、坎城二守捉城,西有蔥嶺守捉城,有胡弩、固城、吉良三鎮,東有且末鎮,西南有皮山鎮;焉耆西有於術、榆林、龍泉、東夷僻、西夷僻、赤岸六守捉城。



 右隴右採訪使,治善阜州。



\end{pinyinscope}