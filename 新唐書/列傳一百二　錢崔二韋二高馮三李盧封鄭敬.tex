\article{列傳一百二 錢崔二韋二高馮三李盧封鄭敬}

\begin{pinyinscope}

 錢徽,字蔚章。父起,附見《盧綸傳》。徽中進士第,居谷城。穀城令王郢善接僑士游客,以財貸饋指出「百花齊放,百家爭鳴」是發展科學和藝術的基本的、長,坐是得罪。觀察使樊澤視其簿,獨徽無有,乃表署掌書記。蔡賊方熾,澤多募武士於軍。澤卒,士頗希賞,周澈主留事,重擅發軍廥,不敢給。時大雨雪,士寒凍,徽先冬頒衣絮,士乃大悅。又闢宣歙崔衍府。王師討蔡,檄遣採石兵會戰,戍還,頗驕蹇。會衍病亟,徽請召池州刺史李遜署副使,遜至而衍死,一軍賴以安。



 入拜左補闕,以祠部員外郎為翰林學士,三遷中書舍人,加承旨。憲宗嘗獨召徽,從容言它學士皆高選,宜預聞機密,廣參決,帝稱其長者。是時,內積財,圖復河湟,然禁無名貢獻,而至者不甚卻。徽懇諫罷之。帝密戒後有獻毋入右銀臺門,以避學士。梁守謙為院使,見徽批監軍表語簡約,嘆曰:「一字不可益邪!」銜之。以論淮西事忤旨,罷職,徙太子右庶子,出虢州刺史。



 入拜禮部侍郎。宰相段文昌以所善楊渾之、學士李紳以周漢賓並諉徽求致第籍。渾之者憑子也,多納古帖秘畫於文昌,皆世所寶。徽不能如二人請,自取楊殷士、蘇巢。巢者李宗閔婿,殷士者汝士之弟,皆與徽厚。文昌怒,方帥劍南西川,入辭,即奏徽取士以私。訪紳及元稹,時稹與宗閔有隙,因是共擠其非。有詔王起、白居易覆試,而黜者過半,遂貶江州刺史。汝士等勸徽出文昌、紳私書自直,徽曰:「茍無愧於心,安事辨證邪?」敕子弟焚書。



 初,州有盜劫貢船,捕吏取濱江惡少年二百人系訊,徽按其枉,悉縱去。數日,舒州得真盜。州有牛田錢百萬,刺史以給宴飲贈餉者,徽曰:「此農耕之備,可他用哉!」命代貧民租入。轉湖州。時宣、歙旱,左丞孔戣請徙徽領宣歙,宰相以其本文辭進,不用。戣曰:「相君宜知天下事,徽江、虢之治不及知,況其它邪?」還,遷工部侍郎,出為華州刺史。



 文宗立,召拜尚書左丞。會宣墨麻,群臣在廷,方大寒,稍稍引避,徽素恭謹,不去位,久而僕。因上疏告老,不許。太和初,復為華州。俄以吏部尚書致仕。卒,年七十五,贈尚書右僕射。



 徽與薛正倫、魏弘簡善,二人前死,徽撫其孤至婚嫁成立。任庶子時,韓公武以賂結公卿,遺徽錢二十萬,不納。或言非當路可無讓,徽曰:「取之在義不在官。」時稱有公望。



 子可復、方義。可復死鄭注時。方義終太子賓客。子珝,字瑞文,善文辭,宰相王摶薦知制誥,進中書舍人。摶得罪,珝貶撫州司馬。



 崔咸,字重易,博州博平人。元和初,擢進士第,又中宏辭。鄭餘慶、李夷簡皆表在幕府,與均禮。入朝為侍御史,處正特立,風採動一時。敬宗將幸東都,裴度在興元憂之,自表求覲,與章偕來。於是李逢吉當國,畏度復相,使京兆尹劉棲楚等十餘人悉力拫卻之,雖度門下賓客,皆有去就意。它日,度置酒延客,棲楚曲意自解,附耳語。咸嫉其矯,舉酒讓度曰:「丞相乃許所由官囁嚅耳語,願上罰爵。」度笑受而飲。棲楚不自安,趨出,坐上莫不壯之。累遷陜虢觀察使,日與賓客僚屬痛飲,未嘗醒;夜分輒決事,裁剖精明,無一毫差,吏稱為神。入拜右散騎常侍、秘書監。太和八年卒。



 咸素有高世志,造詣嶄遠。間游終南山,乘月吟嘯,至感慨泣下。諸文中歌詩最善。



 韋表微,字子明,隋郿城公元禮七世孫。羈昪能屬文。母訓諭稍厲,輒不敢食,以是未嘗讓責。



 韋皋鎮西川,王緯、司空曙、獨孤良弼、裴涚居幕府,皆厚相推挹。涚嘗謂表微似衛玠,自以不能及也。擢進士第,數闢諸使府。久之,入授監察御史裏行,不樂,曰:「爵祿譬滋味也,人皆欲之。吾年五十,拭鏡手翦白,冒游少年間,取一班一級,不見其味也。將為松菊主人,不愧陶淵明」云。俄為翰林學士。是時,李紳忤宰相,貶端州,龐嚴、蔣防皆謫去,學士缺,人人爭薦丞相所善者,表微獨薦韋處厚,人服其公。進知制誥。後與處厚議增選學士,復薦路隋。處厚以諸父事表微,因曰:「隋位崇,入且翁右,奈何?」答曰:「選德進賢,初不計私也。」久之,遷中書舍人。敬宗嘗語左右,欲相二韋,會崩。文宗立,獨相處厚,進表微戶部侍郎。丌志沼叛,詔李聽率師討之,次河上。天子憂無成功,表微曰:「以聽軍勢,不十五日必破賊。」及捷書上,止浹日。志沼殘兵六千奔昭義,宰相請推處首惡者誅之,歸脅從者於魏。表微上言:「逆子降,又殺之,非好生也。請以聽代史憲誠於魏,志沼之徒,可使招納。」不聽。以病痼罷學士。卒,年六十,贈禮部尚書。



 始,被病,醫藥不能具,所居堂寢隘陋,既沒,吊客咨嗟。篤故舊,雖庸下,與攜手語笑無間然。尤好《春秋》,病諸儒執一概,是非紛然,著《三傳總例》,完會經趣。又以學者薄師道,不如聲樂賤工能尊其師,著《九經師授譜》詆其違。



 高釴,字翹之,史失其何所人。與弟銖、鍇俱擢進士第。累遷右補闕、史館脩撰。元和末,以中人為和糴使,釴繼疏論執。轉起居郎,數陳政得失,穆宗嘉之,面賜緋、魚,召入翰林,為學士。張韶變興倉卒,釴從敬宗夜駐左軍。翌日,進知制誥,拜中書舍人。入見帝,因勸躬聽攬示憂勤,帝納其言,賜錦彩。俄罷學士。累進吏部侍郎,人善其振職。出為同州刺史。卒,贈兵部尚書,遺命薄葬。



 釴少孤窶,介然無黨援,以致宦達。諸弟皆檢願友愛,為搢紳景重。



 子湜,字澄之,第進士,累官右諫議大夫。咸通末,為禮部侍郎。時士多繇權要干請,湜不能裁,既而抵帽於地曰:「吾決以至公取之,得譴固吾分!」乃取公乘億、許棠、聶夷中等。以兵部侍郎判度支出為昭義節度使,為下所逐,貶連州司馬。以太子賓客分司東都,卒。億字壽仙,棠字文化,夷中字坦之,皆有名當時。



 銖,字權仲,既擢第,署太原張弘靖幕府,入遷監察御史。太和時,擢累給事中。文宗得李訓,驟拜侍講學士,銖率諫官伏閣言訓素行憸邪,不可任,必亂天下。帝遣使者諭曰:「朕留訓時時講繹,前命不可改。」當是時,已旱而水,彗變未息,鄭注權震赫,人情危駭,既銖等弗見省,群臣失色。明年,訓當國,出銖為浙東觀察使,歷義成節度使。大中初,遷禮部尚書判戶部,徙太常卿。嘗罰禮生,博士李愨慍見曰:「故事,禮院不關白太常,故卿蒞職,博士不參集。不宜罰小史,隳舊典。」銖嘆曰:「吾老不能退,乃為小兒所辱!」卒。



 鍇,字弱金,連中進士、宏辭科,闢河東府參謀,歷吏部員外郎,遷中書舍人。



 開成元年,權知貢舉。文宗自以題畀有司,鍇以籍上,帝語侍臣曰:「比年文章卑弱,今所上差勝於前。」鄭覃曰:「陛下矯革近制,以正頹俗,而鍇乃能為陛下得人。」帝曰:「諸鎮表奏太浮華,宜責掌書記,以誡流宕。」李石曰:「古人因事為文,今人以文害事,懲弊抑末,誠如聖訓。」即以鍇為禮部侍郎。閱三歲,頗得才實。始,歲取四十人,才益少,詔減十人,猶不能滿。遷吏部侍郎,出為鄂岳觀察使。卒,贈禮部尚書。



 子湘,字濬之,擢進士第,歷長安令、右諫議大夫。從兄湜與路巖親善,而湘厚劉瞻,巖既逐瞻,貶湘高州司馬。僖宗初,召為太子右庶子,終江西觀察使。



 馮宿,字拱之,婺州東陽人。父子華,廬親墓,有靈芝、白兔,號「孝馮家」。



 宿貞元中與弟定、從弟審、寬並擢進士第,徐州張建封表掌書記。建封卒,子愔為軍中脅主留事。李師古將乘喪復故地,愔大懼。於是,王武俊擁兵觀釁,宿以書說曰:「張公與公為兄弟,欲共力驅兩河歸天子,天下莫不知。今張公不幸,幼兒為亂兵所脅,內則誠款隔絕,外則強寇侵逼,公安得坐視哉?誠能奏天子不忘舊勛,赦愔罪,使束身自歸,則公有靖亂之功、繼絕之德矣。」武俊悅,即以表聞,遂授愔留後。宿不樂佐愔,更從浙東賈全觀察府。愔憾其去,奏貶泉州司戶參軍。



 召為太常博士。王士真死,子承宗阻命,不得謚,宿謂世勞不可遺,乃上佳謚,示不忘忠。再遷都官員外郎。裴度節度彰義軍,表為判官。淮西平,除比部郎中。長慶時,進知制誥。牛元翼徙節山南東道,為王廷湊所圍,以宿總留事。還,進中書舍人,出華州刺史,避諱不拜,徙左散騎常侍、兼集賢殿學士。拜河南尹。洛苑使姚文壽縱部曲奪民田,匿於軍,吏不敢捕。府大集,部曲輒與文壽偕來,宿掩取榜殺之。歷工部、刑部二侍郎。脩《格後敕》三十篇,行於時。累封長樂縣公。



 擢東川節度使,完城郛,增兵械十餘萬,詔分餘甲賜黔巫道。涪水數壞民廬舍,宿脩利防庸,一方便賴。疾革,將斷重刑,家人請宥之,宿曰:「命脩短,天也。撓法以求祐,吾不敢。」卒,年七十,贈吏部尚書,謚曰懿。治命薄葬,悉以平生書納墓中。



 子圖,字昌之,連中進士、宏辭科。大中時,終戶部侍郎、判度支。寬為起居郎。



 定,字介夫,偉儀觀,與宿齊名,人方漢二馮。于頔素善之。頔在襄陽,定徒步上謁,吏不肯白,乃亟去。頔聞,斥吏,歸錢五十萬,及諸境,定返其遺,以書讓頔不下士,頔大慚。



 第進士異等,闢浙西薛蘋府,以鄠尉為集賢校理。始,定居喪,號毀甚,故數移疾,大學士疑其簡怠,奪職。三遷祠部員外郎,出為郢州刺史。吏告定略民妻,乾沒庫錢,御史鞫治無狀。坐游宴不節免官。起為國子司業,再遷太常少卿。文宗嘗詔開元《霓裳羽衣舞》參以《雲韶》,肄於廷。定部諸工立縣間,端凝若植。帝異之,問學士李玨,玨以定對。帝喜曰:「豈非能古章句者邪?」親誦定《送客西江》詩,召升殿,賜禁中瑞錦,詔悉所著以上。遷諫議大夫。



 是歲,訓、注敗,多誅公卿,中外危惴。及改元,天子御前殿,仇士良請以神策仗衛殿門,定力爭罷之。又請許左右史從宰相至延英記所言,執政不悅,改太子詹事。鄭覃兼太子太師,上日欲會尚書省,定據禮當集詹事府,詔可。論者多其正。換衛尉卿,以散騎常侍致仕。卒,贈工部尚書,謚曰節。



 初,源寂使新羅,其國人傳定《黑水碑》、《畫鶴記》;韋休符使西蕃,所館寫定《商山記》於屏。其名播戎夷如此。



 審,字退思。開成中,為諫議大夫,拜桂管觀察使,歷國子祭酒。監有孔子碑,武后所立,睿宗署額。審請琢「周」著「唐」。終秘書監。



 子緘,字宗之。乾符初,歷京兆、河南尹。



 李虞仲,字見之。父端,附見《文藝傳》。虞仲第進士、宏辭,累遷太常博士。建言:「謚者,所以表德懲惡,《春秋》褒貶法也。茆土爵祿,僇辱流放,皆緣一時,非以明示百代,然而後之所以知其行者,惟謚是觀。古者將葬請謚,今近或二三年,遠乃數十年,然後請謚。人歿已久,風績湮歇,採諸傳聞,不可考信,誄狀雖在,言與事浮。臣請凡得謚者,前葬一月,請考功刺太常定議,其不請與請而過時者,聽御史劾舉。居京師不得過半期,居外一期。若善惡著而不請,許考功察行謚之。節行卓異,雖無官及官卑者,在所以聞。」詔可。



 寶歷初,以兵部郎中知制誥,進中書舍人,出為華州刺史,歷吏部侍郎。簡儉寡欲,時望歸重。卒,年六十五,贈吏部尚書。



 李翱,字習之,後魏尚書左僕射沖十世孫。中進士第,始調校書郎,累遷。元和初,為國子博士、史館脩撰。常謂史官紀事不得實,乃建言:「大氐人之行,非大善大惡暴於世者,皆訪於人。人不周知,故取行狀謚牒。然其為狀者,皆故吏門生,茍言虛美,溺於文而忘其理。臣請指事載功,則賢不肖易見。如言魏徵,但記其諫爭語,足以為直言;段秀實,但記倒用司農印追逆兵,笏擊硃泚,足以為忠烈。不者,願敕考功、太常、史館勿受。如此可以傳信後世矣。」詔可。又條興復太平大略曰:



 陛下即位以來,懷不廷臣,誅畔賊,刷五聖憤恥,自古中興之盛無以加。臣見聖德所不可及者,若淄青生口夏侯澄等四十七人,為賊逼脅,質其父母妻子而驅之戰,陛下俘之,赦不誅,詔田弘正隨材授職,欲歸者縱之。澄等得生歸,轉以相謂,賊眾莫不懷盛德,無肯拒戰。劉悟所以能一昔斬師道者,以三軍皆苦賊而暱就陛下,故不淹日成大功。一也。今歲關中麥不收,陛下哀民之窮,下明詔蠲賦十萬石,群臣動色,百姓歌樂遍畎畮。二也。昔齊遺魯以女樂,季桓子受之,君臣共觀,三日不朝,孔子行。今韓弘獻女樂,陛下不受,遂以歸之。三也。又出李宗奭妻女於掖廷,以田宅賜沈遵師,聖明寬恕,億兆欣感。臣愚不能盡識。若它詔令一皆類此,武德、貞觀不難及,太平可覆掌而致。



 臣聞定禍亂者,武功也;復制度、興太平者,文德也。今陛下既以武功定海內,若遂革弊事,復高祖、太宗舊制:用忠正而不疑;屏邪佞而不邇;改稅法,不督錢而納布帛;絕進獻,寬百姓租賦;厚邊兵,以制蕃戎侵盜;數引見待制官,問以時事,通壅蔽之路。此六者,政之根本,太平所以興。陛下既已能行其難,若何而不為其易者乎?



 以陛下資上聖,如不惑近習容悅之辭,任骨鯁正直,與之脩復故事,以興大化,可不勞而成也。若一日不事,臣恐大功之後,逸樂易生,進言者必曰:「天下既平矣,陛下可以高枕自安逸。」如是,則高祖、太宗之制度不可以復;制度不復,則太平未可以至。臣竊惜陛下當可興之時,而謙讓未為也。



 再遷考功員外郎。初,諫議大夫李景儉表翱自代。景儉斥,翱下除朗州刺史。久之,召為禮部郎中。翱性峭鯁,論議無所屈,仕不得顯官,怫鬱無所發,見宰相李逢吉,面斥其過失,逢吉詭不校,翱恚懼,即移病。滿百日,有司白免官,逢吉更表為廬州刺史。時州旱,遂疫,逋捐系路,亡籍口四萬,權豪賤市田屋牟厚利,而窶戶仍輸賦。翱下教使以田占租,無得隱,收豪室稅萬二千緡,貧弱以安。



 入為諫議大夫,知制誥,改中書舍人。柏耆使滄州,翱盛言其才。耆得罪,由是左遷少府少監。後歷遷桂管湖南觀察使、山南東道節度使,卒。翱始從昌黎韓愈為文章,辭致渾厚,見推當時,故有司亦謚曰文。



 盧簡辭,字子策。父綸,別傳。與兄簡能、弟弘止、簡求皆有文,並第進士。歷佐帥府,入遷侍御史,習知法令及臺閣舊事。寶歷中,黎干子煟詣臺請復葉縣故田,有司莫能知,簡辭獨詰曰:「按乾坐黨魚朝恩誅,貲田皆沒,大歷後數十年,比有赦令,無原洗之言,煟安得冒論?」不為治。福建鹽鐵院官盧昂坐贓,簡辭窮按,乃得金床、瑟瑟枕大如斗。敬宗曰:「禁中無此,昂為吏可知矣。」李程鎮太原,表為節度判官。入授考功員外郎,累擢湖南、浙西觀察使,以檢校工部尚書為忠武節度使。徙山南東道。坐事貶衢州刺史,卒。



 簡能,見《鄭注傳》。其子知猷,字子謨,中進士第,登宏辭,補秘書省正字。蕭鄴鎮荊南、劍南,再闢掌書記。入遷右補闕,出為饒州刺史,以政最聞。累進中書舍人。硃玫亂,避難不出。僖宗還京,召拜工部侍郎、史館脩撰。歷太常卿、戶部尚書,至太子太師。昭宗為劉季述所幽,感憤卒,贈太尉。知猷器量渾厚,世推為長者。善書,有楷法。文辭贍麗。子文度,亦貴顯。



 弘止,字子強,佐劉悟府,累擢監察御史。沈傳師表為江西團練副使。入拜侍御史。華州刺史宇文鼎、戶部員外盧允中坐贓,詔弘止按訊。文宗將殺鼎,弘止執據罪由允中,鼎乃連坐,不應死,帝釋之。累遷給事中。



 會昌中,詔河北三節度討劉稹。何弘敬、王元逵先取邢、洺、磁三州,宰相李德裕畏諸帥有請地者,乃以弘止為三州團練觀察留後。制未下,稹平,即詔為三州及河北兩鎮宣慰使。還,拜工部侍郎,以戶部領度支。初,兩池鹽法弊,得費不相償,弘止使判官司空輿檢鉤厘正,條上新法,即表輿兩池使,自是課入歲倍,用度賴之。逾年,出為武寧節度使。徐自王智興後,吏卒驕沓,銀刀軍尤不法,弘止戮其尤無狀者,終弘止治,不敢嘩。優詔褒勞。弘止羸病,丐身還東都,不許。徙宣武,卒於鎮,贈尚書右僕射。子虔灌,有美才,終秘書監。



 簡求,字子臧,始從江西王仲舒幕府,兩為裴度、元稹所闢,又佐牛僧孺鎮襄陽,入遷戶部員外郎。會昌中,討劉稹,以忠武節度使李彥佐為招討使,各選簡求副之,俾知後務。歷蘇、壽二州刺史。



 大中九年,黨項擾邊,拜涇原渭武節度使。徙義武、鳳翔、河東三鎮。簡求為政長權變,文不害,居邊善綏御,人皆安之。太原統退渾、契苾、沙陀三部,難馴制,它帥或與詛盟,質子弟,然寇掠不為止。簡求歸所質,開示至誠,虜憚其恩信,不敢亂。久之,辭疾,以太子少師致仕,還東都,治園沼林艿,與賓客置酒自娛。卒,年七十六,贈尚書左僕射。



 子嗣業、汝弼,皆中進士第。汝弼以祠部員外郎知制誥,從昭宗遷洛。方柳璨喪王室,汝弼懼,移疾去,客上黨。後依李克用,克用表為節度副使。太原府子亭,簡求所署多在,每宴亭中,未嘗居賓位,西向俯首,人美其有禮。



 嗣業子文紀,後貴顯。



 高元裕,字景圭,其先蓋渤海人。第進士,累闢節度府。以右補闕召,道商州,會方士趙歸真擅乘驛馬,元裕詆曰:「天子置驛,爾敢疾驅邪?」命左右奪之,還,具以聞。敬宗視朝不時,稍稍決事禁中,宦豎恣放,大臣不得進見。元裕諫曰:「今西頭勢乃重南衙,樞密之權過宰相。」帝頗寤而不能有所檢制,人皆危之。俄換侍御史內供奉,士始相賀。



 李宗閔高其節,擢諫議大夫,進中書舍人。鄭注入翰林,元裕當書命,乃言「以醫術侍」,注愧憾。及宗閔得罪,元裕坐出餞,貶閬州刺史。注死,復授諫議大夫、翰林侍講學士。



 莊恪太子立,擇可輔導者,乃兼賓客。進御史中丞。即建言:「紀綱地官屬須選,有不稱職者,請罷之。」於是監察御史杜宣猷、柳瑰、崔郢、侍御史魏中庸、高弘簡並奪職。故事,三司監院官帶御史者,號「外臺」,得察風俗,舉不法。元和中,李夷簡因請按察本道州縣。後益不職。元裕請監院御史隸本臺,得專督察。詔可。累擢尚書左丞,領吏部選。出為宣歙觀察使,入授吏部尚書。拜山南東道節度使,封渤海郡公,奏蠲逋賦甚眾。在鎮五年,復以吏部尚書召,卒於道,年七十六,贈尚書右僕射。



 元裕性勤約,通經術,敏於為吏,巖巖有風採,推重於時。自侍講為中丞,文宗難其代,元裕表言兄少逸才可任,因以命之,世榮其遷。



 少逸,長慶末為侍御史,坐失舉劾,貶贊善大夫,累遷諫議大夫,乃代元裕。稍進給事中,出為陜虢觀察使。中人責峽石驛吏供餅惡,鞭之,少逸封餅以聞。宣宗怒,召使者責曰:「山谷間是餅豈易具邪?」謫隸恭陵,中人皆斂手。以兵部尚書致仕,卒。



 元裕始名允中,太和中改今名。



 元裕子璩,字瑩之。第進士,累佐使府。以左拾遺為翰林學士,擢諫議大夫。近世學士超省郎進官者,惟鄭顥以尚主,而璩以寵升雲。懿宗時,拜劍南東川節度使。召拜中書侍郎、同中書門下平章事。閱月卒,贈司空。太常博士曹鄴建言:「璩,宰相,交游丑雜,取多蹊徑,謚法『不思妄愛曰刺』,請謚為刺。」從之。



 封敖,字碩夫,其先蓋冀州蓚人。元和中,署進士第,江西裴堪闢置其府,轉右拾遺,雅為宰相李德裕所器。會昌初,以左司員外郎召為翰林學士,三遷工部侍郎。敖屬辭贍敏,不為奇澀,語切而理勝。武宗使作詔書慰邊將傷夷者,曰:「傷居爾體,痛在朕躬。」帝善其如意出,賜以宮錦。劉稹平,德裕以定策功進太尉,時敖草其制曰:「謀皆予同,言不它惑。」德裕以能明專任己以成功,謂敖曰:「陸生恨文不迨意,如君此等語,豈易得邪?」解所賜玉帶贈之。未幾,拜御史中丞,與宰相盧商慮囚,誤縱死罪,復為工部侍郎。



 大中中,歷平盧、興元節度使。初,鄭涯開新路,水壞其棧,敖更治斜谷道,行者告便。蓬、果賊依雞山,寇三川,敖遣副使王贄捕平之。加檢校吏部尚書。還為太常卿。卿始視事,廷設九部樂,敖宴私第,為御史所劾,徙國子祭酒。復拜太常,進尚書右僕射。然少行檢,士但高其才,故不至宰相,卒。



 子彥卿、望卿,從子特卿,皆第進士。



 鄭薰,字子溥,亡鄉里世系。擢進士第。歷考功郎中、翰林學士。出為宣歙觀察使。前人不治,薰頗以清力自將。牙將素驕,共謀逐出之,薰奔揚州。貶棣王府長史,分司東都。



 懿宗立,召為太常少卿,擢累吏部侍郎。時數大赦,階正議光祿大夫者,得廕一子,門施戟。於是宦人用階請廕子,薰卻之不肯敘。宰相杜悰才其人,擬判度支,辭;又擬刑部兼御史中丞,固辭,乃免。久之,進左丞。性愛友,糾族百口,稟不充,求外遷。擬華州刺史,輒留中,為幸侍酬沮。後以太子少師致仕。



 薰端勁,再知禮部,舉引寒俊,士類多之。既老,號所居為「隱巖」,蒔松於廷,號「七松處士」云。



 敬晦,字日彰,河中河東人。祖括,字叔弓,進士及第,遷殿中侍御史。楊國忠惡不諧己,外除果州刺史,進累兵部侍郎。志簡淡,在職不求名。周智光已誅,議者健括才,選為同州刺史,拜御史大夫。隱然持重,弗以私害公。大歷中卒。



 晦進士及第,闢山南東道節度府,與馬曙聯舍。於是,帥不政,法制陵頹,曙引大吏廷責之。吏負兼軍職,不引咎,走訴諸府。牙將且十輩,方雜語以申吏枉,晦讓諸將曰:「吏冒軍名,公等不能詰,反引與為伍,奈何?」眾愧謝,闔府咨美。擢累諫議大夫。武宗時,趙歸真以詐營罔天子,御史平吳湘獄,得罪宰相。晦上疏極道非是,不少回縱。



 大中中,歷御史中丞、刑部侍郎、諸道鹽鐵轉運使、浙西觀察使。時南方連饉,有詔弛榷酒茗,官用告覂,晦處身儉勤,貲力遂充。徙兗州節度使,以太子賓客分司。卒,贈兵部尚書,謚曰肅。



 晦兄昕、暤,弟昈、煦,俱第進士籍。昕為河陽節度使,暤右散騎常侍,世寵其家。



 韋博,字大業,京兆萬年人。祖黃裳,浙西節度觀察使。博取進士第,浸遷殿中侍御史。開成中,蕭本詐窮得罪,詔與中人籍其財,中人利寶玉,欲竊取去,博奪還,簿無遺貲。



 回鶻入寇,以符澈為河東節度使,拜博為判官。久之,進主客郎中。時詔毀佛祠,悉浮屠隸主客。博言令太暴,宜近中,宰相李德裕惡之。會羌、渾叛,以何清朝為靈武節度使,詔博副之,擢右諫議大夫,召對,賜金紫。因行西北邊,商虜強弱,還奏有旨,進左大夫,為京兆尹。與御史中丞囂競不平,皆得罪,下除博衛尉卿。出為平盧節度使、檢校禮部尚書,徙昭義。卒,年六十二,贈兵部尚書。



 李景讓,字後己,贈太尉憕孫也。性方毅有守。寶歷初,遷右拾遺。淮南節度使王播以錢十萬市朝廷歡,求領鹽鐵,景讓詣延英亟論不可,遂知名。沈傳師觀察江西,表以自副。歷中書舍人、禮部侍郎、商華虢三州刺史。



 母鄭,治家嚴,身訓勒諸子。始,貧乏時,治墻得積錢,僮婢奔告,母曰:「士不勤而祿,猶菑其身,況無妄而得,我何取?」亟使閉坎。景讓自右散騎常侍出為浙西觀察使,母問行日,景讓率然對:「有日。」鄭曰:「如是,吾方有事,未及行。」蓋怒其不嘗告也。且曰:「已貴,何庸母行?」景讓重請罪,乃赦。故雖老猶加箠敕,已起,欣欣如初。嘗怒牙將,杖殺之,軍且謀變,母欲息眾言雚,召景讓廷責曰:「爾填撫方面而輕用刑,一夫不寧,豈特上負天子,亦使百歲母銜羞泉下!何面目見先大夫乎?」將鞭其背,吏大將再拜請,不許,皆泣謝,乃罷,一軍遂定。景讓家行脩治,閨門唯謹。



 入為尚書左丞,拜天平節度使,徙山南東道,封酒泉縣男。大中中,進御史大夫,甫視事,劾免侍御史孫玉汝、監察御史盧栯,威肅當朝。為大夫三月,蔣伸輔政,景讓名素出伸右,而宣宗擇宰相,盡書群臣當選者,以名內器中,禱憲宗神御前射取之,而景讓名不得。世謂除大夫百日,有他官相者,謂之「辱臺」。景讓愧艴不能平,見宰相,自陳考深當代,即拜西川節度使。以病丐致仕,或諫:「公廉潔亡素儲,不為諸子謀邪?」景讓笑曰:「兒曹詎餓死乎?」書聞,輒還東都。以太子少保分司。卒,年七十二,贈太子太保,謚曰孝。



 性獎士類,拔孤仄,如李蔚、楊知退皆所推引。始為左丞,蔣伸坐宴所,酌酒語客曰:「有孝於家、忠於國者飲此。」客肅然,景讓起卒爵。伸曰:「無宜於公。」所善蘇滌、裴夷直皆為李宗閔、楊嗣復所擢,故景讓在會昌時,抑厭不遷。宣宗銜穆宗舊怨,景讓建請遷敬、文、武三主,以猶子行為嫌,請還代宗以下主復入廟,正昭穆。事下百官議,不然,乃罷,德望稍衰矣。然清素寡欲,門無雜賓。李琢罷浙西,以同里訪之,避不見;及去,命貳其騙石焉。元和後,大臣有德望者,以居里顯,景讓宅東都樂和里,世稱清德者,號「樂和李公」云。



 弟景溫,字德己,歷諫議大夫、福建觀察使,徙華州刺史,以美政聞。累遷尚書右丞。盧攜當國,弟隱繇博士遷水部員外郎,材下資淺,人疾其冒,無敢繩,景溫不許赴省。時故事久廢,景溫既舉職,人皆韙其正。



 弟景莊,亦至顯官。



\end{pinyinscope}