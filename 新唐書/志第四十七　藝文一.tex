\article{志第四十七 藝文一}

\begin{pinyinscope}

 自《六經》焚於秦而復出於漢,其師傅之道中絕,而簡編脫亂訛缺,學者莫得其本真,於是諸儒章句之學興焉。其後傳注、箋解、義疏之流如太一。在目的論學說中是彼岸性的同義語。康德哲學中指,轉相講述,而聖道粗明,然其為說固已不勝其繁矣。至於上古三皇五帝以來世次,國家興滅終始,僭竊偽亂,史官備矣。而傳記、小說,外暨方言、地理、職官、氏族,皆出於史官之流也。自孔子在時,方脩明聖經以絀繆異,而老子著書論道德。接乎周衰,戰國游談放蕩之士,田駢、慎到、列、莊之徒,各極其辯;而孟軻、荀卿始專脩孔氏,以折異端。然諸子之論,各成一家,自前世皆存而不絕也。夫王跡熄而《詩》亡,《離騷》作而文辭之士興。歷代盛衰,文章與時高下。然其變態百出,不可窮極,何其多也。自漢以來,史官列其名氏篇第,以為六藝、九種、七略;至唐始分為四類,曰經、史、子、集。而藏書之盛,莫盛於開元,其著錄者,五萬三千九百一十五卷,而唐之學者自為之書者,又二萬八千四百六十九卷。嗚呼,可謂盛矣!



 《六經》之道,簡嚴易直而天人備,故其愈久而益明。其餘作者眾矣,質之聖人,或離或合。然其精深閎博,各盡其術,而怪奇偉麗,往往震發於其間,此所以使好奇博愛者不能忘也。然凋零磨滅,亦不可勝數,豈其華文少實,不足以行遠歟?而俚言俗說,猥有存者,亦其有幸不幸者歟?今著於篇,有其名而亡其書者,十蓋五六也,可不惜哉。



 初,隋嘉則殿書三十七萬卷,至武德初,有書八萬卷,重復相糅。王世充平,得隋舊書八千餘卷,太府卿宋遵貴監運東都,浮舟溯河,西致京師,經砥柱舟覆,盡亡其書。貞觀中,魏徵、虞世南、顏師古繼為秘書監,請購天下書,選五品以上子孫工書者為書手,繕寫藏於內庫,以宮人掌之。玄宗命左散騎常侍、昭文館學士馬懷素為脩圖書使,與右散騎常侍、崇文館學士褚無量整比。會幸東都,乃就乾元殿東序檢校。無量建議:御書以宰相宋璟、蘇頲同署,如貞觀故事。又借民間異本傳錄。及還京師,遷書東宮麗正殿,置修書院於著作院。其後大明宮光順門外、東都明福門外,皆創集賢書院,學士通籍出入。既而太府月給蜀郡麻紙五千番,季給上谷墨三百三十六丸,歲給河間、景城、清河、博平四郡兔千五百皮為筆材。兩都各聚書四部,以甲、乙、丙、丁為次,列經、史、子、集四庫。其本有正有副,軸帶帙簽皆異色以別之。



 安祿山之亂,尺簡不藏。元載為相,奏以千錢購書一卷,又命拾遺苗發等使江淮括訪。至文宗時,鄭覃侍講,進言經籍未備,因詔秘閣搜採,於是四庫之書復完,分藏於十二庫。黃巢之亂,存者蓋鮮。昭宗播遷,京城制置使孫惟晟斂書本軍,寓教坊於秘閣,有詔還其書,命監察御史韋昌範等諸道求購,及徙洛陽,蕩然無遺矣。



 甲部經錄,其類十一:一曰《易》類,二曰《書》類,三曰《詩》類,四曰《禮》類,五曰《樂》類,六曰《春秋》類,七曰《孝經》類,八曰《論語》類,九曰讖緯類,十曰經解類,十一曰小學類。凡著錄四百四十家,五百九十七部,六千一百四十五卷。不著錄一百一十七家,三千三百六十卷。



 《連山》十卷



 司馬膺注《歸藏》十三卷



 《周易》卜商《傳》二卷



 孟喜《章句》十卷



 京房《章句》十卷



 費直《章句》四卷



 馬融《章句》十卷



 荀爽《章句》十句



 鄭玄注《周易》十卷



 劉表《注》五卷



 董遇《注》十卷



 宋忠《注》十卷



 王肅《注》十卷



 王弼《注》七卷



 又《大衍論》三卷



 虞翻《注》九卷



 陸績《注》十三卷



 姚信《注》十卷



 荀輝《注》十卷



 蜀才《注》十卷



 王廙《注》十卷



 干寶《注》十卷



 又《爻義》一卷



 黃穎《注》十卷



 崔浩《注》十卷



 崔覲《注》十三卷



 何胤《注》十卷



 盧氏《注》十卷



 傅氏《注》十四卷



 王又玄《注》十卷



 王凱沖《注》十卷



 荀氏《九家集解》十卷



 馬、鄭、二王《集解》十卷



 王弼、韓康伯《注》十卷



 二王《集解》十卷



 張璠《集解》十卷



 又《略論》一卷



 謝萬注《系辭》二卷



 桓玄注《系辭》二卷



 荀諺注《系辭》二卷



 荀柔之注《系辭》二卷



 宋褰注《系辭》二卷



 宋明帝注《義疏》二十卷



 張該等《群臣講易疏》二十卷



 梁武帝《大義》二十卷



 又《大義疑問》二十卷



 蕭偉《發義》一卷



 又《幾義》一卷



 蕭子政《義疏》十四卷



 又《系辭義》二卷



 張譏《講疏》三十卷



 何妥《講疏》十三卷



 褚仲都《講疏》十六卷



 梁蕃《文句義疏》二十卷



 又《開題論序疏》十卷



 《釋序義》三卷



 劉瓛《系辭義疏》二卷



 又《乾坤義疏》一卷



 鐘會《周易論》四卷



 範氏《周易論》四卷



 應吉甫《明易論》一卷



 鄒湛《統略論》三卷



 阮長成、阮仲容《難答論》二卷



 宋處宗《通易論》一卷



 宣聘《通易象論》一卷



 欒肇《通易象論》一卷



 袁宏《略譜》一卷



 楊乂《卦序論》一卷



 沈熊《周易譜》一卷



 《雜音》三卷



 任希古注《周易》十卷



 《周易正義》十六卷國子祭酒孔穎達、顏師古、司馬才章、王恭,太學博士馬嘉運,太學助教趙乾葉、王談、於志寧等奉詔撰,四門博士蘇德融、趙弘智覆審。



 陸德明《周易文句義疏》二十四卷



 《文外大義》二卷



 陰弘道《周易新傳疏》十卷顥子,臨渙令。



 薛仁貴《周易新注本義》十四卷



 王勃《周易發揮》五卷



 玄宗《周易大衍論》三卷



 李鼎祚《集注周易》十七卷



 東鄉助《周易物象釋疑》一卷



 僧一行《周易論》卷亡。



 又《大衍玄圖》一卷



 《義決》一卷



 《大衍論》二十卷



 崔良佐《易忘象》卷亡。



 元載集注《周易》一百卷



 李吉甫注《一行易》卷亡。



 衛元嵩《元包》十卷蘇源明傳,李江注。



 高定《周易外傳》二十二卷郢子,京兆府參軍。



 裴通《易書》一百五十卷字又玄,士淹子,文宗訪以《易》義,令進所撰書。



 盧行超《易義》五卷字孟起,大中六合丞。



 陸希聲《周易傳》二卷



 右《易》類七十六家,八十八部,六百六十五卷。失姓名一家,李鼎祚以下不著錄十一家,三百二十九卷。



 《古文尚書》孔安國《傳》十三卷



 謝沈《注》十三卷



 王肅《注》十卷



 又《釋駁》五卷



 範甯《注》十卷



 李顒《集注》十卷



 又《新釋》二卷



 《要略》二卷



 姜道盛《集注》十卷



 徐邈注《逸篇》三卷



 伏勝注《大傳》三卷



 又《暢訓》一卷



 劉向《洪範五行傳論》十一卷



 馬融《傳》十卷



 王肅《孔安國問答》三卷



 鄭玄注《古文尚書》九卷



 又《注釋問》四卷



 王粲問,田瓊、韓益正。



 呂文優《義注》三卷



 伊說《釋義》四卷



 顧歡《百問》一卷



 巢猗《百釋》三卷



 又《義疏》十卷



 費甝《義疏》十卷



 任孝恭《古文大義》二十卷



 蔡大寶《義疏》三十卷



 劉焯《義疏》三十卷



 顧彪《古文音義》五卷



 又《文外義》一卷



 劉炫《述義》二十卷



 王儉《音義》四卷



 王玄度注《尚書》十三卷



 王元感《尚書糾繆》十卷



 《今文尚書》十三卷開元十四年,玄宗以《洪範》「無偏無頗」聲不協,詔改為「無偏無陂」。天寶三載,又詔集賢學士衛包改古文從今文。



 《尚書正義》二十卷國子祭酒孔穎達、太學博士王德韶、四門助教李子雲等奉詔撰。四門博士硃長才蘇德融、太學助教隋德素、四門助教王士雄趙弘智覆審。太尉揚州都督長孫無忌、司空李勣、左僕射於志寧、右僕射張行成、吏部尚書侍中高季輔吏部尚書褚遂良、中書令柳奭、弘文館學士谷那律劉伯莊、太學博士賈公彥範義郡齊威、太常博士柳士宣孔志約、四門博士趙君贊、右內率府長史弘文館直學士薛伯珍、國子助教史士弘、太學助教鄭祖玄周玄達、四門助教李玄植王真儒與王德韶、隋德素等刊定。



 王元感《尚書糾繆》十卷



 穆元休《洪範外傳》十卷



 陳正卿《續尚書》纂漢至唐十二代詔策、章疏、歌頌、符檄、論議成書,開元末上之。卷亡。



 崔良佐《尚書演範》卷亡。



 右《書》類二十五家,三十三部,三百六卷。王元感以下不著錄四家,二十卷。



 《韓詩》卜商《序》韓嬰《注》二十二卷



 又《外傳》十卷



 卜商《集序》二卷



 又《翼要》十卷



 毛萇《傳》十卷



 鄭玄箋《毛詩詁訓》二十卷



 又《譜》三卷



 王肅《注》二十卷



 又《雜義駁》八卷



 《問難》二卷



 葉遵《注》二十卷號《葉詩》。



 崔靈恩《集注》二十四卷



 《義注》五卷



 劉楨《義問》十卷



 王基《毛詩駁》五卷



 《毛詩雜答問》五卷



 《雜義難》十卷



 孫毓《異同評》十卷



 楊乂《毛詩辨》三卷



 陳統《難孫氏詩評》四卷



 又《表隱》二卷



 元延明《誼府》三卷



 張氏《義疏》五卷



 陸璣《草木鳥獸魚蟲疏》二卷



 謝沈《釋義》十卷



 劉氏《序義》一卷



 劉炫《述義》三十卷



 魯世達《音義》二卷



 鄭玄等《諸家音》十五卷



 王玄度注《毛詩》二十卷



 《毛詩正義》四十卷孔穎達、王德韶、齊威等奉詔撰,趙乾葉、四門助教賈普曜趙弘智等覆正。



 許叔牙《毛詩纂義》十卷



 成伯璵《毛詩指說》一卷



 又《斷章》二卷



 《毛詩草木蟲魚圖》二十卷開成中,文宗命集賢院脩撰並繪圖象,大學士楊嗣復、學士張次宗上之。



 右《詩》類二十五家,三十一部,三百二十二卷。失姓名三家,許叔牙以下不著錄三家,三十三卷。



 《大戴德禮記》十三卷



 又《喪服變除》一卷



 鄭玄注《小戴聖禮記》二十卷



 又《禮議》二十卷



 《禮記音》三卷曹耽解。



 《三禮目錄》一卷



 注《周官》十三卷



 《音》三卷



 注《儀禮》十七卷



 《喪服變除》一卷



 注《喪服紀》一卷



 盧植注《小戴禮記》二十卷



 馬融《周官傳》十二卷



 又注《喪服記》一卷



 王肅注《小戴禮記》三十卷



 又注《周官》十二卷



 注《儀禮》十七卷



 《音》二卷



 《喪服要記》一卷



 注《喪服紀》一卷



 鄭小同《禮記義記》四卷



 袁準注《儀禮》一卷



 孔倫《注》一卷



 陳銓《注》一卷



 蔡超宗《注》二卷



 田僧紹《注》二卷



 傅玄《周官論評》十二卷陳邵駁。



 杜預《喪服要集議》三卷



 賀循《喪服譜》一卷



 又《喪服要記》五卷謝微注。



 干寶注《周官》十二卷



 又《答周官駁難》五卷孫略問。



 李軌《小戴禮記音》二卷



 尹毅《音》二卷



 徐邈《音》三卷



 徐爰《音》二卷



 司馬伷《周官寧朔新書》八卷



 又《禮記寧朔新書》二十卷並王懋約注。



 戴顒《月令章句》十二卷



 又《中庸傳》二卷



 《緱氏要鈔》六卷



 王逡之注《喪服五代行要記》十卷



 徐廣《禮論問答》九卷



 範甯《禮問》九卷



 又《禮論答問》九卷



 射慈《小戴禮記音》二卷



 又《喪服天子諸侯圖》一卷



 崔游《喪服圖》一卷



 蔡謨《喪服譜》一卷



 《喪服要難》一卷趙成問,袁祈答。



 伊說注《周官》十卷



 孫炎注《禮記》三十卷



 葉遵《注》十二卷



 董勛《問禮俗》十卷



 劉俊《禮記評》十卷



 吳商《雜禮義》十一卷



 何承天《禮論》三百七卷



 顏延之《禮逆降議》三卷



 任預《禮論條牒》十卷



 又《禮論帖》三卷



 《禮論鈔》六十六卷



 庾蔚之《禮記略解》十卷



 又注《喪服要記》五卷



 《禮論鈔》二十卷



 王儉《禮儀答問》十卷



 又《禮雜答問》十卷



 《喪服古今集記》三卷



 荀萬秋《禮雜鈔略》二卷



 傅隆《禮議》一卷



 梁武帝《禮大義》十卷



 周舍《禮疑義》五十卷



 何佟之《禮記義》十卷



 又《禮答問》十卷



 戚壽《雜禮義問答》四卷



 賀瑒《禮論要鈔》一百卷



 賀述《禮統》十二卷



 崔靈恩《周官集注》二十卷



 又《三禮義宗》三十卷



 元延明《三禮宗略》二十卷



 皇侃《禮記講疏》一百卷



 又《義疏》五十卷



 《喪服文句義》十卷



 沈重《周禮義疏》四十卷



 又《禮記義疏》四十卷



 熊安生《義疏》四十卷



 劉芳《義證》十卷



 沈文阿《喪服經傅義疏》四卷



 又《喪服發題》二卷



 夏侯伏朗《三禮圖》十二卷



 《禮記隱》二十六卷



 《禮類聚》十卷



 《禮儀雜記故事》十一卷



 《禮統郊祀》六卷



 《禮論要鈔》十三卷



 《區分》十卷



 《禮論鈔略》十三卷



 《禮記正義》七十卷孔穎達、國子司業硃子奢、國子助教李善信賈公彥柳士宣範義郡、魏王參軍事張權等奉詔撰,與周玄達、趙君贊、王士雄、趙弘智覆審。



 賈公彥《禮記正義》八十卷



 又《周禮疏》五十卷



 《儀禮疏》五十卷



 魏征《次禮記》二十卷亦曰《類禮》。



 王玄度《周禮義決》三卷



 又注《禮記》二十卷



 元行沖《類禮義疏》五十卷



 《御刊定禮記月令》一卷集賢院學士李林甫、陳希烈、徐安貞、直學士劉光謙齊光乂陸善經、脩撰官史玄晏、待制官梁令瓚等注解。自第五易為第一。



 成伯璵《禮記外傳》四卷



 王元感《禮記繩愆》三十卷



 王方慶《禮經正義》十卷



 《禮雜問答》十卷



 李敬玄《禮論》六十卷



 張鎰《三禮圖》九卷



 陸質《類禮》二十卷



 韋彤《五禮精義》十卷



 丁公著《禮志》十卷



 《禮記字例異同》一卷元和十二年詔定。



 丘敬伯《五禮異同》十卷



 孫玉汝《五禮名義》十卷



 杜肅《禮略》十卷



 張頻《禮粹》二十卷



 右《禮》類六十九家,九十六部,一千八百二十七卷。失姓名七家,元行沖以下不著錄十六家,二百九十五卷。



 桓譚《樂元起》二卷



 又《琴操》一卷



 孔衍《琴操》二卷



 荀勖《太樂雜歌辭》三卷



 又《太樂歌辭》二卷



 《樂府歌詩》十卷



 謝靈運《新錄樂府集》十一卷



 信都芳刪注《樂書》九卷



 留進《管弦記》十二卷



 凌秀《管弦志》十卷



 公孫崇《鐘磬志》二卷



 梁武帝《樂社大義》十卷



 又《樂論》三卷



 沈重《鐘律》五卷



 釋智匠《古今樂錄》十三卷



 鄭譯《樂府歌辭》八卷



 又《樂府聲調》六卷



 蘇夔《樂府志》十卷



 李玄楚《樂經》三十卷



 元殷《樂略》四卷



 又《聲律指歸》一卷



 翟子《樂府歌詩》十卷



 又《三調相和歌辭》五卷



 劉氏、周氏《琴譜》四卷



 陳懷《琴譜》二十一卷



 《漢魏吳晉鼓吹曲》四卷



 《琴集歷頭拍簿》一卷



 《外國伎曲》三卷



 又一卷



 《論樂事》二卷



 《歷代曲名》一卷



 《推七音》一卷



 《十二律譜義》一卷



 《鼓吹樂章》一卷



 李守真《古今樂記》八卷



 蕭吉《樂譜集解》二十卷



 武後《樂書要錄》十卷



 趙邪利《琴敘譜》九卷



 張文收《新樂書》十二卷



 劉貺《太樂令壁記》三卷



 徐景安《歷代樂儀》三十卷



 崔令欽《教坊記》一卷



 吳兢《樂府古題要解》一卷



 郗昂《樂府古今題解》三卷一作王昌齡。



 段安節《樂府雜錄》一卷文昌孫



 竇璡《正聲樂調》一卷



 玄宗《金鳳樂》一卷



 蕭祜《無射商九調譜》一卷



 趙惟暕《琴書》三卷



 陳拙《大唐正聲新址琴譜》十卷



 呂渭《廣陵止息譜》一卷



 李良輔《廣陵止息譜》一卷



 李約《東杓引譜》一卷勉子,兵部員外郎。



 齊嵩《琴雅略》一卷



 王大力《琴聲律圖》一卷



 陳康士《琴譜》十三卷字安道,僖宗時人。



 又《琴調》四卷



 《琴譜》一卷



 《離騷譜》一卷



 趙邪利《琴手勢譜》一卷



 南卓《羯鼓錄》一卷



 右《樂》類三十一家,三十八部,二百五十七卷。失姓名九家,張文收以下不著錄二十家,九十三卷。



 左丘明《春秋外傳國語》二十卷



 董仲舒《春秋繁露》十七卷



 《春秋穀梁傳》十五卷尹更始注。



 《春秋公羊傳》五卷嚴彭祖述。



 賈逵《春秋左氏長經章句》二十卷



 又《解詁》三十卷



 《春秋三家訓詁》十二卷



 董遇《左氏經傳章句》三十卷



 王肅《注》三十卷



 又《國語章句》二十二卷



 王朗注《左氏》十卷



 土燮注《春秋經》十一卷



 杜預《左氏經傳集解》三十卷



 又《釋例》十五卷



 《音》三卷



 鄭眾《牒例章句》九卷



 潁容《釋例》七卷



 劉寔《條例》十卷



 方範《經例》六卷



 何休《左氏膏盲》十卷鄭玄箴。



 又《公羊解詁》十三卷



 《春秋漢議》十卷麋信注,鄭玄駁。



 《公羊條傳》一卷



 《墨守》一卷鄭玄發。



 《穀梁廢疾》三卷鄭玄釋,張靖箋。



 服虔《左氏解誼》三十卷



 又《膏盲釋痾》五卷



 《春秋成長說》七卷



 《塞難》三卷



 《音隱》一卷



 《駁何氏春秋漢議》十一卷



 王玢《達長義》一卷



 孫毓《左氏傳義注》三十卷



 又《賈服異同略》五卷



 梁簡文帝《左氏傳例苑》十八卷



 干寶《春秋函傳》十六卷



 《序論》一卷



 殷興《左氏釋滯》十卷



 何始真《春秋左氏區別》十二卷



 張沖《春秋左氏義略》三十卷



 嚴彭祖《春秋圖》七卷



 吳略《春秋經傳詭例疑隱》一卷



 京相璠《春秋土地名》三卷



 王延之《旨通》十卷



 顧啟期《大夫譜》十一卷



 李謐《叢林》十二卷



 崔靈恩《立義》十卷



 《申先儒傳例》十卷



 沈宏《經傳解》六卷



 又《文苑》六卷



 《嘉語》六卷



 沈文阿《義略》二十七卷



 劉炫《攻昧》十二卷



 又《規過》三卷



 《述議》三十七卷



 高貴鄉公《左氏音》三卷



 曹耽、荀訥《音》四卷



 李軌《音》三卷



 孫邈《音》三卷



 王元規《音》三卷



 孔氏《公羊集解》十四卷



 王愆期注《公羊》十二卷



 又《難答論》一卷庾翼難。



 高襲《傳記》十二卷



 荀爽、徐欽《答問》五卷



 劉寔《左氏牒例》二十卷



 又《公羊違義》三卷劉晏注。



 王儉《音》二卷



 《春秋穀梁傳》段肅《注》十三卷



 唐固注《穀梁》十二卷



 又注《國語》二十一卷



 麋信注《穀梁》十二卷



 又《左氏傳說要》十卷



 張靖《集解》十一卷



 程闡《經傳集注》十六卷



 孔衍《訓注》十三卷



 範甯《集注》十二卷



 徐乾《注》十三卷



 徐邈《注》十二卷



 又《傳義》十卷



 《音》一卷



 沈仲義《集解》十卷



 蕭邕《問傳義》三卷



 劉兆《三家集解》十一卷



 韓益《三傳論》十卷



 胡訥集撰《三傳經解》十一卷



 又《三傳評》十卷



 潘叔度《春秋成集》十卷



 又《合三傳通論》十卷



 江熙《公羊穀梁二傳評》三卷



 李鉉《春秋二傳異同》十二卷



 盧翻注《國語》二十一卷



 韋昭《注》二十一卷



 孔晁《解》二十一卷



 《春秋辨證明經論》六卷



 《左氏音》十二卷



 《左氏鈔》十卷



 《春秋辭苑》五卷



 《雜義難》五卷



 《左氏杜預評》二卷



 《春秋正義》三十六卷孔穎達、楊十勛、硃長才奉詔撰。馬嘉運、王德韶、蘇德融與隋德素覆審。



 楊士勛《穀梁疏》十二卷



 王玄度注《春秋左氏傳》卷亡。



 虞藏用《春秋後語》十卷



 高重《春秋纂要》四十卷字文明,士廉五代孫,文宗時翰林侍講學士。帝好《左氏春秋》,命重分諸國各為書,別名《經傳要略》。歷國子祭酒。



 許康佐等集《左氏傳》三十卷一作文宗御集。



 徐文遠《左傳義疏》六十卷



 又《左傳音》三卷



 陰弘道《春秋左氏傳序》一卷



 李氏《左傳異同例》十三卷開元中,右威衛錄事參軍,失名



 馮伉《三傳異同》三卷



 劉軻《三傳指要》十五卷



 韋表微《盧春秋三傳總例》二十卷



 王元感《春秋振滯》二十卷



 韓滉《春秋通》一卷



 陸質集注《春秋》二十卷



 又集傳《春秋纂例》十卷



 《春秋微旨》二卷



 《春秋辨疑》七卷



 樊宗師《春秋集傳》十五卷



 《春秋加減》一卷元和十三年,國子監脩定。



 李瑾《春秋指掌》十五卷



 張傑《春秋圖》五卷



 又《春秋指元》十卷



 裴安時《左氏釋疑》七卷字適之,大中江陵少尹。



 第五泰《左傳事類》二十卷字伯通,青州益都人,咸通鄂州文學。



 成玄《公穀總例》十卷字又玄,咸通山陽令。



 陸希聲《春秋通例》三卷



 陳岳《折衷春秋》三十卷唐末鐘傳江西從事。



 郭翔《春秋義鑒》三十卷



 柳宗元《非國語》二卷



 右《春秋》類六十六家,一百部,一千一百六十三卷。失姓名五家,王玄度以下不著錄二十二家。四百三卷。



 《古文孝經》孔安國《傳》一卷



 劉邵《注》一卷



 《孝經》王肅《注》一卷



 鄭玄《注》一卷



 韋昭《注》一卷



 孫熙《注》一卷



 蘇林《注》一卷



 謝萬《注》一卷



 虞盤佐《注》一卷



 孔光《注》一卷



 殷仲文《注》一卷



 殷叔道《注》一卷



 徐整《默注》二卷



 車胤《講孝經義》四卷



 荀勖《講孝經集解》一卷



 皇侃《義疏》三卷



 何約之《大明中皇太子講義疏》一卷



 梁武帝《疏》十八卷



 太史叔明《發題》四卷



 劉炫《述義》五卷



 張士儒《演孝經》十二卷



 《應瑞圖》一卷



 賈公彥《孝經疏》五卷



 魏克己注《孝經》一卷



 任希古《越王孝經新義》十卷



 《今上孝經制旨》一卷。玄宗。



 元行沖《御注孝經疏》二卷



 尹知章注《孝經》一卷



 孔穎達《孝經義疏》卷亡。



 王元感注《孝經》一卷



 李嗣真《孝經指要》一卷



 平貞諲《孝經議》卷亡。



 徐浩《廣孝經》十卷浩稱四明山人,乾元二年上,授校書郎。



 右《孝經》類二十七家,三十六部,八十二卷。失姓名一家,尹知章以下不著錄六家,一十三卷。



 《論語》鄭玄《注》十卷



 又注《論語釋義》一卷



 《論語篇目弟子》一卷



 王弼《釋疑》二卷



 王肅注《論語》十卷



 又注《孔子家語》十卷



 李充注《論語》十卷



 梁覬《注》十卷



 孟厘《注》九卷



 袁喬《注》十卷



 尹毅《注》十卷



 張氏《注》十卷



 何晏《集解》十卷



 孫綽《集解》十卷



 盈氏《集義》十卷



 江熙《集解》十卷



 徐氏《古論語義注譜》一卷



 虞喜《贊鄭玄論語注》十卷



 暢惠明《義注》十卷



 宋明帝補《衛瓘論語注》十卷



 欒肇《論語釋》十卷



 又《駁》二卷



 崔豹《大義解》十卷



 繆播《旨序》二卷



 郭象《體略》二卷



 戴詵《述議》二十卷



 劉炫《章句》二十卷



 皇侃《疏》十卷



 褚仲都《講疏》十卷



 《義注隱》三卷



 《雜義》十三卷



 《剔義》十卷



 徐邈《音》二卷



 《孔叢》七卷



 王勃《次論語》十卷



 賈公彥論《論語疏》十五卷



 韓愈注《論語》十卷



 張籍《論語注辨》二卷



 右《論語》類三十家,三十七部,三百二十七卷。失姓名三家,韓愈以下不著錄二家,十二卷。



 宋均注《易緯》九卷



 注《禮緯》三十卷



 注《詩緯》三卷



 注《樂緯》三卷



 注《春秋緯》三十八卷



 注《論語緯》十卷



 注《孝經緯》五卷



 鄭玄注《書緯》三卷



 注《詩緯》三卷



 右讖緯類二家,九部,八十四卷。



 劉向《五經雜義》七卷



 又《五經通義》九卷



 《五經要義》五卷



 許慎《五經異義》十卷鄭玄駁。



 楊方《五經金句沉》十卷



 楊思《五經咨疑》八卷



 元延明《五經宗略》四十卷



 劉炫《五經正名》十二卷



 沈文阿《經典玄儒大義序錄》十卷



 班固等《白虎通義》六卷



 鄭玄《六藝論》一卷



 譙周《五經然否論》五卷



 《鄭志》九卷



 《鄭記》六卷



 王肅《聖證論》十一卷



 梁武帝《孔子正言》二十卷



 簡文帝《長春義記》一百卷



 樊文深《七經義綱略論》三十卷



 又《質疑》五卷



 張譏《游玄桂林》二十卷



 《謚法》三卷荀顗演,劉熙注。



 沈約《謚例》十卷



 賀琛《謚法》三卷



 《集天名稱》三卷



 陸德明《經典釋文》三十卷



 顏師古《匡謬正俗》八卷



 趙英《五經對訣》四卷英,龍朔中汲令。



 劉迅《六說》五卷



 劉貺《六經外傳》三十七卷



 張鎰《五經微旨》十四卷



 韋表微《九經師授譜》一卷



 裴僑卿《微言注集》二卷開元中鄭縣尉。



 高重《經傳要略》十卷



 王彥威《續古今謚法》十四卷



 慕容宗本《五經類語》十卷字泰初,幽州人,大中時。



 劉氏《經典集音》三十卷熔,字正範,絳州正平,咸通晉州長史。



 右經解類十九家,二十六部,三百八十一卷。失姓名一家,趙英以下不著錄十家,一百二十七卷。



 《爾雅》李巡《注》三卷



 樊光《注》六卷



 孫炎《注》六卷



 沈■《集注》十卷



 郭璞《注》一卷



 又《圖》一卷



 《音義》一卷



 江灌《圖贊》一卷



 又《音》六卷



 李軌解《小爾雅》一卷



 楊雄《別國方言》十三卷



 劉熙《釋名》八卷



 韋昭《辨釋名》一卷



 李斯等《三蒼》三卷郭璞解。



 杜林《蒼頡訓詁》二卷



 張揖《廣雅》四卷



 又《埤蒼》三卷



 《三蒼訓詁》三卷



 《雜字》一卷



 《古文字訓》二卷



 樊恭《廣蒼》一卷



 史游《急就章》一卷曹壽解。



 顏之推《注》一卷



 司馬相如《凡將篇》一卷



 班固《在昔篇》一卷



 《太甲篇》一卷



 蔡邕《聖草章》一卷



 又《勸學篇》一卷



 《今字石經論語》二卷



 崔瑗《飛龍篇篆草勢合》三卷



 許慎《說文解字》十五卷



 呂忱《字林》七卷



 楊承慶《字統》二十卷



 馮幹《括字苑》十三卷



 賈魴《字屬篇》一卷



 葛洪《要用字苑》一卷



 戴規《辨字》一卷



 僧寶志《文字釋訓》三十卷



 周成《解文字》七卷



 王延《雜文字音》七卷



 王氏《文字要說》一卷



 阮孝緒《文字集略》一卷



 彭立《文字辨嫌》一卷



 王愔《文字志》三卷



 顧野王《玉篇》三十卷



 李登《聲類》十卷



 呂靜《韻集》五卷



 陽休之《韻略》一卷



 又《辨嫌音》二卷



 夏侯詠《四聲韻略》十三卷



 張諒《四聲部》三十卷



 趙氏《韻篇》十二卷



 陸慈《切韻》五卷



 郭訓《字旨篇》一卷



 《古文奇字》二卷



 衛宏《詔定古文字書》一卷



 虞龢《法書目錄》六卷



 衛恆《四體書勢》一卷



 蕭子云《五十二體書》一卷



 庾肩吾《書品》一卷



 顏之推《筆墨法》一卷



 僧正度《雜字書》八卷



 何承天《纂文》三卷



 顏延之《纂要》六卷



 又《詰幼文》三卷



 張推《證俗音》三卷



 顏愍楚《證俗音略》一卷



 李虔《續通俗文》二卷



 李少通《俗語難字》一卷



 諸葛潁《桂苑珠叢》一百卷



 硃嗣卿《幼學篇》一卷



 項峻《始學篇》十二卷



 王羲之《小學篇》一卷



 楊方《少學集》十卷



 顧凱之《啟疑》三卷



 蕭子範《千字文》一卷



 周興嗣《次韻千字文》一卷



 《演千字文》五卷



 《黃初篇》一卷



 《吳章篇》一卷



 《音隱》四卷



 《難要字》三卷



 《覽字知源》三卷



 《字書》十卷



 《敘同音》三卷



 《桂苑珠叢略要》二十卷



 《古今八體六文書法》一卷



 《古來篆隸詁訓名錄》一卷



 《筆墨法》一卷



 《鹿紙筆墨疏》一卷



 《篆書千字文》一卷



 《今字石經易篆》三卷



 《今字石經尚書本》五卷



 《今字石經鄭玄尚書》八卷



 《三字石經尚書古篆》三卷



 《今字石經毛詩》三卷



 《今字石經儀禮》四卷



 《三字石經左傳古篆書》十二卷



 《今字石經左傳經》十卷



 《今字石經公羊傳》九卷



 蔡邕《今字石經論語》二卷



 曹憲《爾雅音義》二卷



 又《博雅》十卷



 《文字指歸》四卷



 劉伯莊《續爾雅》一卷



 顏師古注《急就章》一卷



 武後《字海》一百卷凡武后所著書,皆元萬頃、範履冰、苗神客、周思茂、胡楚賓、衛業等撰。



 李嗣真《書後品》一卷



 徐浩《書譜》一卷



 《古跡記》一卷



 張懷瓘《書斷》三卷開元中翰林院供奉。



 又《評書藥石論》一卷



 張敬玄《書則》一卷貞元中處士。



 褚長文《書指論》一卷



 張彥遠《法書要錄》十卷



 弘靖孫,乾符初大理卿。



 裴行儉《草字雜體》卷亡。



 荊浩《筆法記》一卷浩稱洪谷子。



 二王、張芝、張昶等書一千五百一十卷太宗出御府金帛購天下古本,命魏徵、虞世南、褚遂良定真偽,凡得羲之真行二百九十紙,為八十卷,又得獻之、張芝等書,以「貞觀」字為印。草跡命遂良楷書小字以影之。其古本多梁、隋官書。梁則滿騫、徐僧權、沈熾文、硃異,隋則江總、姚察署記。帝令魏、褚卷尾各署名。開元五年,敕陸玄悌、魏哲、劉懷信檢校,分益卷秩。玄宗自書「開元」字為印。



 唐書王方慶《寶章集》十卷



 又《王氏八體書範》四卷



 《王氏工書狀》十五卷



 玄宗《開元文字音義》三十卷



 張參《五經文字》三卷



 唐玄度《九經字樣》一卷文宗時待詔。



 顏元孫《干祿字書》一卷



 歐陽融《經典分毫正字》一卷



 李騰《說文字源》一卷陽冰從子。



 僧慧力《像文玉篇》三十卷



 蕭鈞《韻音》二十卷



 孫愐《唐韻》五卷



 武元之《韻銓》十五卷



 玄宗《韻英》五卷天寶十四載撰,詔集賢院寫付諸道採訪使,傳布天下。



 顏真卿《韻海鏡源》三百六十卷



 李舟《切韻》十卷



 僧猷智《辨體補脩加字切韻》五卷



 右小學類六十九家,一百三部,七百二十一卷。失姓名二十三家,徐浩以下不著錄二十三家,二千四十五卷。



\end{pinyinscope}