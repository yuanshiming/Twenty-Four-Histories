\article{列傳第一百三十六 籓鎮鎮冀}

\begin{pinyinscope}

 李寶臣字為輔,本範陽內屬奚也。善騎射。範陽將張鎖高畜為假子,故冒其姓,名忠志。為盧龍府果毅,常覘虜陰山體方法,亦同樣需要認真學習和掌握。,追騎及,射六人盡殪,乃還。為安祿山射生,從入朝,留為射生子弟,出入禁中。祿山反,遁歸,更為祿山假子,使將驍騎十八人,劫太原尹楊光翽,挾以出,追兵萬餘不敢逼。又督精甲軍土門,以扼井陘。事安慶緒為恆州刺史。九節度師圍相州也,忠志懼,歸命於朝,肅宗即授故官,封密雲郡公。史思明度河,忠志復叛,勒兵三萬固守,賊將辛萬寶屯恆州相掎角。思明死,忠志不肯事朝義,使裨將王武俊殺萬寶,挈恆、趙、深、定、易五州以獻。雍王東討,開土門納王師,助攻莫州。朝義平,擢禮部尚書,封趙國公,名其軍曰成德,即拜節度使,賜鐵券許不死,它齎與不貲,賜姓及名。於是遂有恆、定、易、趙、深、冀六州地,馬五千,步卒五萬,財用豐衍,益招來亡命,雄冠山東。與薛嵩、田承嗣、李正己、梁崇義相姻嫁,急熱為表裹。先是天寶中,玄宗冶金自為象,州率置祠,更賊亂,悉毀以為貲,而恆獨存,故見寵異,加賜實封。



 始,寶臣與正己素為承嗣所易。其弟寶正,承嗣婿也,往依魏,與承嗣子維擊球,馬駭,觸維死,承嗣怒,囚之,以告寶臣,寶臣謝教不謹,進杖,欲使示責,而承嗣遂鞭殺之,由是交惡。乃與正己共劾承嗣可討狀。代宗欲其自相圖,則勢離易制,即詔寶臣與硃滔及太原兵攻其北,正己與滑亳、河陽、江淮兵攻其南。師會棗強,椎牛饗軍,寶臣厚賜士,而正己頗觳,軍怨望,正己懼有變,即引去。惟滔、寶臣攻滄州,歷年未下,擊宗城,殘之,斬二千級。承嗣弟廷琳方守貝州,遣高嵩巖將兵三千戍宗城,寶臣使張孝忠攻破之,斬嵩巖,逸所執將四十餘人。會王武俊執賊大將盧子期,遂降洺、瀛。當是時,河南諸將敗田悅於陳留,正己取德州,欲頗窮討。承嗣懼,乃甘言紿正己,正己止屯,諸軍亦莫敢進。



 於是天子遣中人馬希倩勞寶臣,寶臣歸使者百縑,使者恚,抵諸道,寶臣顧左右愧甚。諸將已休,獨武俊佩刀立所下,語之故。武俊計曰:「趙兵有功尚爾,使賊平,天子幅紙召置京師,一匹夫耳。」曰:「奈何?」對曰:「養魏以為資,上策也。」寶臣曰:「趙、魏有釁,何從而可?」對曰:「勢同患均,轉寇讎為父子,咳唾間耳。硃滔屯滄州,請禽送魏,可以取信。」寶臣然之。



 先是,承嗣知寶臣少長範陽,心常欲得之。乃勒石若讖者瘞之境,教望氣者云有王氣。寶臣掘得之,文曰:』二帝同功勢萬全,將田作伴入幽燕。」「帝」謂寶臣與正己為二。而陰使客說曰:「公與滔共攻滄,即有功,利歸天子,公於何賴?誠能赦承嗣罪,請奉滄州入諸趙,願取範陽以報。公以騎前驅,承嗣以步卒從,此萬全勢也。」寶臣喜得滄州,又見語與讖會,遂陰交承嗣而圖幽州,承嗣陳兵出次以自驗。寶臣謬謂滔使曰:「吾聞硃公貌若神,願繪而觀可乎?」滔即圖以示之。寶臣置圖射堂,大會諸將,熟視曰:「信神人也!」密選精卒二千,夜馳三百里欲劫滔,戒曰:「取彼貌如射堂者。」時二軍不相虞,忽聞變,滔大駭,戰瓦橋,敗,衣佗服得脫,禽類滔者以歸承嗣。承嗣知釁成,還軍入堡,使人謝寶臣曰:「河內方有警,未暇從公。石讖,吾戲為耳!」寶臣慚而還。俄進封隴西郡王,又拜同中書門下平章事。德宗立,拜司空。



 寶臣晚節尤猜忌,自顧子惟岳且暗弱,恐下不服,即殺骨鯁將辛忠義、盧俶、許崇俊、張南容、張彭老等二十餘人,籍入其貲,眾乃攜貳。寶臣既貯異志,引妖人作讖兆,為丹書、靈芝、硃草,齋別室,築壇置銀盤、金匜、玉,猥曰:「內產甘露液神酒。」刻玉印,告其下曰:「天瑞自至。」眾莫敢辨者。妖人復言:「當有玉印自天下,海內不戰而定。」寶臣大悅,厚齎金帛。既而畏事露且誅,詐曰:「公飲甘露液,可與天神接。」密置堇于液,寶臣已飲即瘖,三日死,年六十四。惟岳悉誅殺妖人,時建中二年也。遺表請以惟岳領軍,詒書執政諉家事,歸節於朝,詔贈太傅。



 惟岳少為行軍司馬、恆州刺史,寶臣死,軍中推為留後,求襲父位,帝不許。趣護喪還京師,以張孝忠代之。田悅為請,不聽。遂與悅、李正己謀拒命。府小史胡震、私人王他奴等專畫反計。府屬邵真泣曰:「先公位將相,恩甚厚,而大夫違命縗絰中,愚固惑焉。魏近且與國,不可遽絕,絕之速禍,請厚禮遣其使,徐更圖之;齊遠而交疏,不如械使者送京師,且請致討。上嘉大夫忠,所請宜許。」惟岳寤,使真作奏。震與將吏議不可,惟岳又從之。其舅穀從政,豪俊士也,切諫不納。



 於是張孝忠以易州歸天子,天子詔硃滔與孝忠合兵討惟岳,盡赦吏士,購惟嶽首有賞。惟岳與滔戰束鹿,大奔。遂圍深州。明年正月,率兵萬餘,使王武俊爭束鹿,田悅亦遣孟祐來助。武俊以精兵先陷陣,師卻。滔繢帛為狻猊,使壯士百人蒙以噪,趨惟岳軍,馬駭軍亂,因大敗,火其營去。於是深州日急,悅亦嬰城矣。惟岳懼,召真議遣使詣河東馬燧,令其弟惟簡見帝,斬大將謝罪,以兵屬鄭詵,身朝京師。孟祐知其謀,走告悅,悅使扈岌來讓曰:「敝邑暴兵,本為君索命節,豈為叛逆耶?雖見破於馬燧,而感激士大夫乘城拒守,以為後圖。今君信邵真讒間,欲歸悅之罪,以自湔蕩,何負而然!不則遣祐還軍,無遺王師禽。若能誅真以徇,請事公如初。」惟岳懦不能決,畢華見曰:「大夫與魏盟未久,魏雖被圍,彼多蓄積,未可下。齊兵勁地廣,裾帶山河,所謂東秦險固之國,與相持維,足以抗天下。夫背義不詳,輕慮生禍。且孟祐驍將,王武俊善戰,前日逐滔,滔僅免,今合兩將,破滔必矣。惟審圖之!」惟岳見深圍未解,畏祐還,乃斬真以謝悅。明日復戰,又大敗。而康日知舉趙州聽命,惟岳益困,乃付牙將衛常寧兵五千,而俾王武俊騎八百攻日知。



 武俊才雄,素為惟岳忌,及師行,謂常寧曰:「大夫信讒,吾朝不圖晏,是行勝與否,吾不復入恆矣!將以身托定州張公,安能持頸就刀乎?」常寧與副李獻誠曰:「君不聞詔書乎?斬大夫首以其官畀之。觀大夫勢終為滔滅,若倒戈還府,事實易圖,有如不捷,張公可歸也。」武俊然之。惟嶽使要藉官謝遵至武俊壁議事,武俊與謀,使內應。至期,啟城門,武俊入,殺人廷中,無亢者。乃傳令曰:「大夫叛命,今且取之,敢拒者族!」士不敢動。武俊使裨校任越牽惟岳出,縊之戟門下,並殺鄭詵、他奴等數十人,使子士真傳首京師。帝盡赦其府將士,給部中租役三年。



 真始事寶臣,掌文記,武俊表其忠,贈戶部尚書。其息呂擢冀州長史。



 常寧在武俊時用事,為內史監,其後謀亂,誅。



 惟嶽異母兄惟誠,尚儒術,謙裕,寶臣愛之,使決軍事,以惟岳正嫡,固讓不肯當。其妹妻李納,故寶臣請惟誠復故姓,而仕諸鄆,為納營田副使,四為州刺史。



 初,惟岳叛,弟惟簡以家僮票士百餘奉母鄭奔京師,帝拘於客省。及出奉天,惟簡將赴難,謀於鄭,鄭曰:「爾父立功河朔,位宰相,身未嘗至京師,兄死於人手。爾入朝,未識天子,不能效忠,吾不子汝矣!」督其行曰:「而能死王事,吾不朽矣!」乃斬關出,道更七戰,得及行在。帝見厚撫之,拜太子諭德,討賊有功。帝徙山南,惟簡以三十騎從,夜失道,馳至盩厔西,聞中人語,問天子所在,密語曰:「上在此。」帝見之流涕,執其手曰:「爾有母,乃能從朕耶?」對曰:「臣誓以死!」比明,北方有塵起,帝憂。惟簡登高曰:「渾瑊以騎來。」瑊至,遂決趨興元,惟簡前導。及帝還,封武安郡王,號元從功臣,圖形凌煙閣,賜鐵券。憲宗時,為左金吾衛大將軍,長史萬國俊奪興平民田,吏畏不敢治,至是訴於惟簡,即日廢國俊,以地與民。出為鳳翔節度使,市耕牛佃具給農,歲增墾數十萬畝。卒,年五十五,贈尚書右僕射。



 子元本,輕薄無行。長慶末,與薛渾私侍襄陽公主,事敗,主幽禁中,元本以功臣子,貸死,流嶺南。弟銖,好學多識,有儒者風。



 王武俊字元英,本出契丹怒皆部。父路俱,開元中,與饒樂府都督李詩等五千帳求襲冠帶,入居薊。武俊甫十五,善騎射,與張孝忠齊名,隸李寶臣帳下為裨將。寶應初,王師入井陘,武俊謂寶臣曰:「以寡敵眾,曲遇直,戰則離,守則潰,銳師遠鬥,庸可御乎!」寶臣遂以恆、定等五州自歸,共平餘賊,武俊謀也。奏兼御史中丞,封維川郡王。其子士真,亦沈悍有斷,寶臣倚愛,出入帳中,以女妻之。寶臣以疑殺許崇俊等,士真密結左右,故武俊免於難。



 惟岳拒命,或言武俊有他志,武俊知之,出入導從才一二,未嘗接賓客。惟岳雖內疑,然見其屈損,又惜善鬥,末忍殺。康日知以趙州降,惟岳謀伐之,皆曰:「武俊故心膂,先君命之使佐大夫,而士真又大夫女弟婿,今事急,宜去猜嫌以任之,不然,尚誰使?」乃遣與衛常寧將兵往。因謀執惟岳,而日知亦遣人邀說以禍福,武俊乃還兵,使人謂惟岳曰:「大夫與齊、魏同惡,今魏兵已敗,齊為趙州所限,幽州兵近在定,三軍且救死。聞有詔召大夫,宜亟歸。」惟岳惶遽出,遂縊。即遣其屬孟華奏天子。華辯對稱旨,德宗擢為兵部郎中,授武俊檢校秘書監兼御史大夫、恆冀觀察使。



 是時,惟岳將楊政義以定降,楊榮國以深降,硃滔受而戍之。帝以定賜張孝忠,而日知為深趙觀察使。武俊怨不得節度而失趙、定,滔亦怨失深州,二人相結。武俊即縛使者送滔,與之叛。帝聞,詔華諭解,不聽。



 時馬燧、李抱真、李芃、李晟討田悅,悅方困,武俊、滔救之,屯連篋山。帝詔李懷光督神策兵助討賊,軍就舍,氣銳甚,謂燧曰:「奉詔毋養寇,及壁壘未成擊之,可滅也。」乃縱兵入滔壁,殺千餘人。悅軍既屢北,不能陣。懷光緩轡觀之,武俊乘其怠,使趙萬敵等以二千騎橫突,而滔軍踵馳,王師亂,相蹈藉死,尸梗河為不流。懷光還走壁。武俊夜決河注王莽渠,斷燧餉路。燧計窮,而與滔素姻家,乃遣使謾謝滔曰:「老夫不自量,與諸君遇。王大夫善戰,天下無前,吾固宜敗,幸公圖之,使老夫得還河東,諸將亦罷兵,吾為言天子,以河北地付公。」滔亦陰忌武俊勝且不制,即謂武俊曰:「王師既敗,馬公卑約如此,不宜迫人以險。」答曰:「燧等皆國名臣,連兵十萬,一戰而北,貽羞國家,不知何面目見天子耶?彼行不五十里,必反拒我。」滔固許之。燧至魏縣,堅壁自固,師復振。滔慚謝,嫌隙始構矣。武俊使張鐘葵攻趙州,日知斬其首以聞。於是武俊與田悅等擅相王。武俊國號趙,以恆為真定府,命士真留守兼元帥;以畢華、鄭儒為左右內史,王士良司刑,王佑司文,士清司武,並為尚書;士則司文侍郎,宋端給事中,王洽內史舍人,張士清執憲大夫,衛常寧內史監,皇甫祝尚書右僕射,餘以次封拜。



 建中四年,抱真使客賈林詐降武俊,既見,曰:「吾來傳詔,非降也。」武俊色動,林曰:「天子知大夫登壇建國撫膺顧左右曰:『我本忠義,天子不省,故至是。』今諸軍數表大夫至誠,上見表動色曰:『朕前誤無及矣。朋友失意尚可謝,朕四海主,毫芒過失,返不得自新耶!』今大夫親斷逆首,而宰相闍於事宜,國家與大夫烏有細故哉?硃滔以利相動,公何取焉?誠能與昭義同心,曠然改圖,上不失君臣之義,下以為子孫計。」武俊曰:「僕虜人也,尚知撫百姓,天子固不務殺人以安天下。今山東連兵比戰,骨盡暴野,雖勝尚誰與居?今不憚歸國,業與諸軍盟,虜性樸強,不欲曲在我,天子若能以恩蕩刷之,我首倡歸命,有不從者,奉辭伐之,河北不五十日可定。」會帝出奉天,抱真將還澤潞,悅說武俊、滔踵襲之。林曰:「夫退軍,前輜重,後銳師,人心固壹,不可圖也。使戰勝得地,利歸於魏,不幸喪師,趙受其災。今滄、趙乃故地,故不取之?」武俊遂引而北,林復激之曰:「公異邦豪英,不應謀中夏。燕、魏幽險,彼王室強則須公之援,削則己欲並吞。且河北惟有趙、魏、燕耳,滔乃稱冀,心圖公冀州矣。使滔能制山東,大夫當臣事之,否則見攻。能臣滔乎?」武俊投袂曰:「二百年天子猶不能事,安能臣豎子耶!」乃定計通好抱真,而約馬燧盟。



 興元元年赦天下,武俊大集其軍,黜偽號。詔國子祭酒董晉與中人宣慰,拜檢校工部尚書、恆冀深趙節度使,又加檢校司空、同中書門下平章事,兼幽州盧龍節度使、瑯邪郡王。



 是時,滔悉幽、薊兵與回紇圍貝州,將絕白馬津,南趨洛,李懷光據河中,李希烈陷汴,南略江淮,李納方叛,唯李晟軍渭上。羽書調發天下十之三,人心惴恐。及田緒殺悅,林復說武俊曰:「滔素欲得魏博,會悅死,魏人氣闉,公不救,魏且下。滔益甲數萬,張孝忠將北面事滔,三道連衡,濟以回紇,長驅而南,昭義軍必保山西,則河朔舉入滔矣。今魏尚完,孝忠未附,公與昭義合兵破之,聲振關中,京邑可坐復,天子反正,不朽之業,誰與公參!」武俊大喜,與抱真相聞,自將屯南宮,抱真屯經城,兩軍相距十里而舍。武俊潛會抱真於軍,陳說忼慨,抱真亦傾意結納,約為兄弟,遂俱東壁貝州,距城三十里止。滔欲迎戰,武俊戒士飽食曰:「軍未合,毋妄動!」遣趙琳、趙萬敵兵五百蔽林以待。滔使票將馬寔、盧南史陣而西,李少成引回紇翼之。日中兵接,武俊與子士清引精騎望少成軍,抱真次之,滔馳騎二百出武俊東南,乘高鼓噪。武俊使步兵決戰,而自以騎當回紇,勒兵避其銳。回紇馬怒突而過,未及返,武俊急擊,琳等兵亦出,回紇驚,中斷,遂先奔。初,滔兵蹙武俊軍,不能傷,回紇既卻,即欲引還,因囂不能止,軍大奔,滔走還壁。武俊中流矢,謂抱真曰:「士少衰,盍以騎濟師,巢穴可覆也。」抱真使來希皓率勁騎薄滔營,盧玄真乘其後,滔懼,引眾去,希皓迫之,武俊邀於隘,滔大敗,免者八千人。會夜,各按屯,武俊營滔東北,抱真營西北。滔知不支,夜半焚車糧,遁歸幽州,火如晝,師大噪,其聲殷地。抱真以山東蝗,食少,歸於潞,武俊亦還。



 會有詔復滔官爵,武俊上還幽州盧龍節度。又詔以恆州為大都督府,即授武俊長史,賜德、棣二州,以士真為觀察使、清河郡王。天子至自梁,遇武俊益厚,子弟雖襁褓,悉官之。俄進檢校太尉兼中書令,得建廟京師,有司供擬。



 武俊善射,嘗與賓客獵,一日射雞兔九十五,觀者駭伏。貞元十七年死,年六十七。群臣奉慰天子,如渾瑊故事,贈太師。有司謚威烈,帝更為忠烈。士真襲位。



 士真,其長子也。少佐父立功,更患難。既得節度,息兵善守,雖擅置吏,私賦入,而歲貢數十萬緡,比燕、魏為恭。元和初,即拜同中書門下平章事。四年死,贈司徒,謚曰景襄。軍中推其子承宗為留後。



 始,河北三鎮自置副大使,常處嫡長,故承宗以御史大夫為之。及總留事,憲宗久不報,伺其變。承宗數上疏自言。帝聞劉濟、田季安俱大病,議更建節度。翰林學士李絳曰:「鎮州世相繼,人所狃習,惟拒命則討之。且諸道之賞饋百萬士,又燕、魏、淄青,勢同必合。方江、淮水潦,財力刓困,宜即詔承宗嗣領。季安等雖病,徐圖所宜。定四方有天時,不可速也。」帝然之,欲析鎮分建節度,使承宗歲輸賦如李師道。絳曰:「假令承宗奉詔,諸道以割地同怨,是官爵虛出而無當也。不如令使者諭之,無出上意。」帝乃詔京兆尹裴武慰撫,承宗奉詔恭甚,請上德、棣二州,遂以檢校工部尚書嗣領節度,而以德州刺史薛昌朝為保信軍節度使,統德、棣。



 昌朝,嵩子也,與承宗故姻家,帝因欲離其親將,故命之。詔未至,承宗馳騎劫而歸,囚之。詔更用棣州刺史田渙為二州團練守捉使,遣中人傳詔令歸昌朝,承宗拒命,帝怒,詔削官爵,遣中人吐突承璀將左右神策,率河中、河陽、浙西、宣歙兵討之。趙萬敵者,故武俊將,以健鬥聞,士真時入朝,上言討之必捷,令與承璀偕。有詔:「武俊忠節茂著,其以實封賜子士則,毋毀墳墓。」



 承璀至軍,無威略,師不振。神策大將酈定進號驍將,以禽劉闢功,王陽山郡,至是戰北,馳而僨,趙人曰:「酈王也」,害之,師氣益折。及吳少誠死,李絳奏:「蔡無四鄰援,攻討勢易,不如赦承宗,專事淮西。」帝不聽。昭義節度使盧從史市承宗,外自固,內實與之。太常卿權德輿諫曰:「神策兵市井屠販,不更戰陣,恐因勞憚遠,潰為盜賊。恆冀騎壯兵多,攻之必引時月,西戎乘間,則禁衛不可頓虛。山東,疥癬也;京師,心腹也。不可不深念。且師出半年,費緡錢五百萬。方夏甚暑水潦,疾疫且降,誠慮有潰橈之變。」又言:「山東諸侯,皆以息自副,人心不遠,誰肯為陛下盡力者。又盧從史倚寇為援,訹承璀邀寵利,宜召行營善將,令倍驛馳,度至半道,授以澤潞,而徙從史它鎮,破其奸圖,然後赦承宗,眾情必服。」帝未許。



 五年,河東軍拔其一屯,張茂昭破之木刀溝;帝患從史詐,卒以計縛送京師;劉濟又拔安平。承宗懼,遣其屬崔遂上書謝罪,且言:「往年納地,迫三軍不得專,而為盧從史賣以求利,願請吏入賦得自新。」是時宿師久無功,餉不屬,帝憂之。而淄青、盧龍數表請赦,乃詔浣雪,盡以故地畀之,罷諸道兵。昌朝歸京師,授右武衛將軍。承宗見兵薄境,已而罷,歸罪從史,得不詰,自謂計得,謷然無顧憚。



 七年,軍庫火,器鎧殆盡,殺守吏百餘人,不自安。及吳元濟反,承宗與李師道上書請宥,教其將尹少卿為蔡游說,見宰相語不遜,武元衡怒,叱遣之。承宗怨甚,與師道謀,遣惡少年數十曹伏河陰,乘昏射吏,吏奔潰,因火漕院,人趣火所,鬥死者十餘輩,縣大發民捕盜,亡去不獲,凡敗錢三十萬緡、粟數萬斛。未幾,張晏等賊宰相元衡,京師大索,天子為旰食。承宗嘗疏元衡過咎,留中。至是帝出表示群臣大議,咸請聲其罪伐之。詔乃絕承宗朝貢,竄其弟承系、承迪、承榮於遠方,以博野、樂壽故範陽地,命歸劉總。而所遣盜處處竊發,斷建陵門戟,燔獻陵寢宮,伏甲欲反洛陽,不克。承宗數出兵掠鄰鄙,田弘正上言承宗宜誅,帝使率師壓境。承宗揣詔旨兵不即進,即肆剽滄、景、易、定間,人苦之。



 十一年,詔削爵,以實封賜土平,使奉武俊後。令河東、義武、盧龍、橫海、魏博、昭義六節度兵進討,大抵數十萬,環地數千里,以分其勢。然營屯離置,主約不得一,故士觀望,獨昭義郗士美薄賊境,賊不敢犯。始,承宗不能葉諸父,皆奔京師。士則為神策大將軍,聞其叛,請占數京兆,裴度請用為邢州刺史,使隸昭義,以傾趙人。有王怡者,武俊從子,為承宗守南宮,士則招之,約歸命,謀洩遇害;子元伯奔還,擢監察御史,詔贈怡尚書左僕射。



 明年元濟平,承宗大恐,使牙將石泛奉二子至魏博,因田弘正求入侍,且請歸德、棣二州,入租賦,待天子署吏。弘正遣知感、知信詣闕下請命。前此,帝使尚書右丞崔從賜詔書許自新,承宗素服待罪。及是乃詔復官爵,以華州刺史鄭權為橫海節度使,統德、棣、滄、景等州,復承宗實封戶三百,以所部饑,賜帛萬匹。李師道平,奉法益謹,表所領州錄事、參軍、判司、縣主簿、令,皆丐王官。



 十五年死,贈侍中。軍中推其弟承元為留後。承元不敢世於鎮,詔用為義成軍節度使,事見本傳。



 王廷湊,本回紇阿布思之族,隸安東都護府。曾祖五哥之,為李寶臣帳下,驍果善鬥,王武俊養為子,故冒姓王,世為裨將。



 廷湊生駢脅,沈鷙少言,喜讀《鬼谷》、兵家諸書。王承宗時,為兵馬使。田弘正至鎮州,詔以度支緡錢百萬勞軍,不時致,廷湊暴其稽以觀眾心,眾果怨,由是害弘正,自稱留後,脅監軍表請節。又取冀州,殺刺史王進岌。穆宗怒,以弘正子布為魏博節度使,率軍進討,仍敕橫海、昭義、河東、義武軍並力。於是大將王位等謀執廷湊,不克,死者三千餘人。會硃克融囚張弘靖,以幽州亂,乃合從拒王師。



 有詔議攻討先後,劍南東川節度使王涯以為「範陽亂非宿謀,可先事鎮州,又有魏博之怨,濟以晉陽、滄德,掎角而進。夫用兵若鬥然,先扼喉領。今瀛莫、易定實賊咽喉,宜屯重兵,俾死生不得相聞,間諜不入,此莫勝之策。」帝乃詔義武節度使陳楚閉境,督諸軍三道攻。而滄德烏重胤最宿將,當一面。裴度以河東節度使兼幽、鎮招撫使,屯承天軍。重胤知時不可,案兵未肯前,帝浮於聽受,銳克伐,更以深冀行營節度使杜叔良代之。叔良素結中人,入見帝,大言曰:「賊不足破!」會度逐廷湊兵於會星,又入元氏,焚壁二十二。叔良率諸道兵救深州,戰博野,大奔,失所持節,以身免,貶歸州刺史。叔良者,將家子,本以附會至靈武節度使,坐不職罷,復階貴近,帥滄景。廷湊知其怯,故先犯之,師由是敗。



 當是時,帝賜賚無藝,府帑空,既集諸道兵,調發火馳,民不堪其勞。仰度支者大抵兵十五萬,有司懼不給,置南北供軍院。既薄賊鄙,餉道梗棘,樵蘇不繼,兵番休取芻蒸。廷湊乘間奪轉運車六百乘,食愈困,至所須衣帛,未半道,諸軍強取之,有司弗能制。其縣師深入者,不得衣食。又監軍宦人,悉取精票士自隨,疲瑣者備行陣,戰輒潰。二賊眾不過萬餘,王師統制不一,訖無功。宰相不知兵,為異議搖訹,裁報乖戾,深州圍益急。



 明年,魏牙將史憲誠叛,田布眾潰於南宮。帝不得已,乃赦廷湊,檢校右散騎常侍、成德軍節度使。會牛元翼出奔,廷湊遂取深州,詔兵部侍郎韓愈慰其軍。



 廷湊既原,則稍挺,與克融、憲誠深相結,為輔車援。滄州李全略死,子同捷求襲,文宗不許,更授兗海節度使。同捷逆命,乃以珍幣子女厚結廷湊,帝虞其變,故授檢校司徒。及幽、魏、徐、兗兵討同捷,廷湊橈魏北鄙以牽制之,而饋滄景鹺糧,囚鄰道使者不遣。帝怒,詔絕其輸貢。於是易定、柳公濟戰新樂,斬首三千級。昭義劉從諫戰臨城,敗之,引漳注深、冀。有詔:「同捷亂,廷湊同惡,宜削官爵,諸道以兵進討,有能斬廷湊者,賜錢二萬緡,優畀之官;以州鎮降者,等差為比。」公濟再戰行唐,皆克,焚柵十五。廷湊射蠟書求救於幽州,行營李載義獲之;又納魏叛將丌志沼。會同捷平,廷湊稍畏,表上景州,而弓高、樂陵、長河三縣固守,復上書謝。帝方厭兵,赦之,悉復官爵,還所上州。久之,進兼太子太傅、太原郡公。



 鎮冀自惟岳以來,拒天子命,然重鄰好,畏法,稍屈則祈自新。至廷湊湊資兇悖,肆毒甘亂,不臣不仁,雖夷狄不若也。大和八年死,贈太尉。軍中以元逵請命,帝聽襲節度。



 元逵,其次子也。識禮法,歲時貢獻如職。帝悅,詔尚絳王悟女壽安公主。元逵遣人納聘闕下,進千盤食、良馬、主妝澤奩具、奴婢,議者嘉其恭。其後劉稹叛,武宗詔元逵為北面招討使。詔下,即日師引道,拔宣務壁,破援軍堯山,攻邢州降之,累遷檢校司徒、同中書門下平章事。稹平,加兼太子太師,封太原郡公,食實封戶二百,進至兼太傅。大中八年死,年四十三,贈太師,謚曰忠。



 子紹鼎襲,字嗣先,累擢檢校尚書左僕射。其為人淫湎自放,性暴,厚裒斂,升樓彈射路人以為樂。眾忿其虐,欲逐之。會病死,贈司空。



 子幼未能事,宣宗以元逵次子紹懿為留後以嗣,俄為節度使,累封太原縣伯,加檢校司空。政簡易,咸通七年死,贈司徒。以紹鼎子景崇嗣。初,紹懿病篤,召景崇曰:「先君以政屬我,須爾長,將授之。今疾甚,爾雖少,勉總軍務,禮籓鄰,奉朝廷,則家業不墜矣。」監軍上狀,懿宗悅,擢景崇為留後,尋進節度使。



 景崇,字孟安,以公主嫡孫,尤被寵。龐勛反,景崇遣兵會王師平賊,進檢校尚書右僕射。主薨,謚曰章惠,景崇居喪如禮。母張卒,號慕羸心叕,當時稱之。以政委賓佐,檢戒親屬不得與。嘗欲引母昆弟為牙將,其佐張位曰:「軍中用人,有勞有能,若私其人,厚畀田宅祿食可也,何必以官。」景崇謝。進同中書門下平章事、檢校太尉兼中書令,封趙國公。乾符五年,進王常山。



 黃巢反,帝西狩,偽使齎詔至,景崇斬以徇,因發兵馳檄諸道,合定州處存連師西入關,問行在,貢輸相踵。每語及宗廟園陵,輒流涕。



 蔚州刺史蘇祐為沙陀所攻,乞師於幽州,屯美女谷,兵不利。祐將出奔,會詔徙濮州刺史,擁兵之官,道於鎮,景崇館於靈壽,肆其下剽奪,景崇殺之。



 嗣節度凡十四年,十三遷至檢校太傅。中和三年死,年三十七,贈太傅,謚曰忠穆。子鎔。



 鎔年十歲,軍中推為留後,授檢校工部尚書。李克用、楊復光攻黃巢,鎔凡再饋粟以濟師。僖宗還自蜀,獻馬牛戎械萬計。



 於是克用方擊孟方立於邢州,鎔歸芻糧。邢州平,克用遂謀山東,屯常山西,引輕騎涉滹沱諜軍,會大澍,平地水出,鎔兵奄至,克用匿林中以免。是時,幽州李匡威亦謀取易、定分其地。王處存方厚事克用,克用寵將李存孝已拔邢,則略鎔南鄙,別將李存信等出井陘會之。鎔侵堯山,存孝擊敗之,遂至深、越。鎔求救於匡威。存孝方攻臨城等數縣,聞匡威屯鄗,引師去。存信素忌存孝,妄曰「無擊賊意。」克用信之。存孝,飛狐人,所謂安敬思者,善騎射,攻葛從周,敗張浚、韓建,數有奇功。至是懼讒,挈邢州歸硃全忠,並結鎔為助。天子詔出鎮、幽、魏兵援之。景福元年,克用假道於鎔,以討存孝,鎔不答,乃與處存連兵侵鎔,拔堅固鎮,攻新市。鎔禽克用將薛萬金。匡威以兵三萬救鎔。克用自攻常山,度滹沱。鎔引騎十萬夜濟礠水,襲敗之,斬二萬級,奪鎧器三百乘,克用退壁欒城。天子有詔和解三鎮,克用還,然未得志,故復伐鎔。匡威以五千騎敗克用於元氏,鎔具牛酒會匡威槁城,餉金二十萬以謝。



 俄而匡威為弟匡籌所逐,鎔德其助己,迎而館之。匡威親忌日,鎔往吊,伏起,殺其府屬楊洽及親吏淡從,有甲者牽鎔袖。匡威曰:「與我四州,可不死!」鎔許之。將鎔入牙城,鎮軍噪而闔左門,坎垣出戰。會大雨風,木拔瓦飛。兵相接,有屠者墨君和袒而薄賊,眾披靡,乃挾鎔逾城入。既免,賞千金,與第一區,約宥十死。匡威走東園,兵圍之,與從事李抱貞俱死。明日,鎔以禮斂匡威,素服哭諸廷,遣使告匡籌。匡籌怒,移書詰兄所以死狀,表天子請討鎔,詔止之。又詔硃全忠平幽、鎮怨。



 克用聞匡威死,自率兵傅城下。鎔大驚,納縑二十萬,乃退。匡籌攻樂壽、武強,克用出縛馬關,敗鎮兵於平山,因進攻鎔外壘。鎔內失幽州助,因乞盟,進幣五十萬,歸糧二十萬,請出兵助討存孝,乃得解。



 克用屯欒城,存信屯琉璃陂,為邢人夜襲其營,存信軍亂,不克追。克用進薄邢,環城為溝堞,欲示久圍者;城中兵數出,溝壘不可成。裨將袁奉韜紿存孝曰:「君所畏唯王耳,王欲溝堞成則西歸,公何不聽之?」存孝兵不出,壘成,攻益急,城中食盡。存孝登城哭曰:「我誤計,使我生見王,死不恨!」克用遣家嫗招之,存孝出,泥首言為存信誣構,克用曰:「爾與鎔書,罵我多矣!」軒而尸於市。



 光化中,全忠討幽州劉仁恭,鎔遣兵屯蓚城,俄而仁恭敗,擊其歸,得十八。全忠既取邢、洺、礠,又得潞,因圖河東。使羅紹威諷鎔絕太原,共尊全忠。鎔猗違,全忠不悅。會克用將李嗣昭攻洺州,全忠自將擊走之,得鎔與嗣昭書,全忠怒,引軍攻鎔,次元氏。鎔謂其屬曰:「國危矣,奈何?」周式請見全忠,可以口舌罷也,許之。全忠迎折曰:「爾公朋附太原,今無赦矣!」即出書示式曰:「嗣昭在者,宜速遣。」式曰:「王公所與和者,息人鋒鏑間耳。況繼奉天子詔和解,能無一番紙墜北路乎?太原與趙本無恩,嗣昭庸肯入耶?公為唐桓、文,方以仁義成霸業,寧困人於險耶?」全忠喜,把式袂曰:「吾特戲耳!」延入帳中,議脩好。鎔以幣二十萬賂師,遣子昭祚質仕全忠府,全忠因妻之。鎔判官張澤謀曰:「失火之家,不可恃遠救。今定密邇,與太原親,宜使全忠圖之。」鎔遣式使全忠,全忠乃取定州,王郜遂奔太原。



 鎔母何,有婦德,訓鎔嚴。至母亡,鎔始黷貨財,姬侍千人,儀服僭上。又以房山有西王母祠,數游覽,妄求長年事,逾月不還。



 始廷湊賤微時,鄴有道士為卜,得《乾之坤》,曰:「君將有土。」及得鎮,迎事甚謹。復問壽幾何?子孫幾何?」答曰:「公三十年後,當有二王。」已而廷湊立十三年死,蓋廋文也,景崇、鎔皆王。廷湊嘗使至河陽,醉寢於路,有過其所者視之曰:「非常人也!」從者以告廷湊,馳及之,問其故,曰:「吾見君鼻之息,左若龍,右若虎,子孫當王百年。家有大樹,覆及堂,公興矣,」及害弘正,而樹適庇寢。自廷湊訖鎔,凡百年。



 贊曰:硃滔、王武俊南面稱王,地聯交暱。及泚僭天子,滔將應之,當時危矣。賈林以一語寤武俊,軋兵相仇,折幽、薊之銳,泚失其朋,不出孤城,終底覆夷,用林之功,賞不及身,德宗為不明哉!



\end{pinyinscope}