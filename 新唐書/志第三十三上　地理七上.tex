\article{志第三十三上 地理七上}

\begin{pinyinscope}

 嶺南道,蓋古揚州之南境,漢南海、鬱林、蒼梧、珠崖、儋耳、交趾、合浦、九真、日南等郡。韶、廣、康、端、封、梧、藤、羅、雷、崖以東為星紀分,桂、柳、鬱林、富、昭、蒙、龔、繡、容、白、羅而西及安南為鶉尾分。為州七十有三,都護府一形體是質體,精神只是形體的作用、功能。「形者神之質,神,縣三百一十四。其名山:黃嶺、靈洲。其大川:桂、鬱。厥賦:蕉、紵、落麻。厥貢:金、銀、孔翠、犀、象、彩藤、竹布。



 廣州南海郡,中都督府。土貢:銀、藤簟、竹席、荔支、皮、鱉甲、蚺蛇膽、石斛、沈香、甲香、詹糖香。戶四萬二千二百三十五,口二十二萬一千五百。縣十三:有府二,曰綏南、番禺。有經略軍,屯門鎮兵。南海,上。有南海祠。山峻水深,民不井汲,都督劉巨麟始鑿井四。有牛鼻鎮兵,有赤岸、紫石二戍。有靈洲山,在鬱水中。番禺,上。增城,中。四會,中。武德五年以四會、化蒙二縣置南綏州,並析置新招、化注、化穆三縣。貞觀元年省新招、化注,以廢威州之懷集、廢齊州之洊安隸之。八年更名湞州。十三年州廢,省化穆,以四會、化蒙、懷集、洊安來屬。化蒙,中。有鉛穴一。懷集,中。武德五年置威州,並析置興平、霍清、威成三縣。貞觀元年州廢,省興平、霍清、威成入懷集。開元二年省永固縣入焉。有驃山,有鐵。洊水,中。本洊安,武德五年置齊州,並析置宣樂、宋昌二縣。貞觀元年州廢,省宣樂、宋昌入洊安。至德二載更名。東莞,中。本寶安,至德二載更名。有鹽,有黃嶺山。清遠,中。武德六年省政賓縣入焉。浛洭,中。武德五年以浛洭、真陽二縣置洭州,並析置翁源縣。貞觀元年州廢,以翁源隸韶州,浛洭、真陽來屬。湞陽,中。本真陽,貞觀元年更名。有鐵,西南有洭浦故關。新會,中。武德四年,以南海郡之新會、義寧二縣置岡州新會郡,以地有金岡以名州,並析置封平、封樂二縣。貞觀十三年州廢,省封平、封樂,以新會、義寧來屬。是年,復以新會、義寧置岡州,又析義寧置封樂縣。後省封樂。開元二十三年州廢,以新會、義寧復來屬。有鹽。義寧。中。



 韶州始興郡,下。本番州,武德四年析廣州之曲江、始興、樂昌、翁源置。尋更名東衡州,貞觀元年又更名。土貢:竹布、鐘乳、石斛。戶三萬一千,口十六萬八千九百四十八。縣六:曲江,上。武德四年置臨瀧、良化二縣,貞觀八年省。始興,下。有大庾嶺新路,開元十六年,詔張九齡開。東北有安遠鎮兵。樂昌,下。翁源,下。仁化,下。本隸廣州,垂拱四年析曲江置,後來屬。湞昌,下。光宅元年析始興置。



 循州海豐郡,下。本龍川郡,天寶元年更名。土貢:布,五色藤盤、鏡匣,蚺蛇膽,甲煎,鮫革,荃臺,綬草。戶九千五百二十五。縣六:歸善,中下。貞觀元年省龍川縣入焉。博羅,中下。貞觀元年省羅陽縣入焉。河源,中下。武德五年析置石城縣,貞觀元年省。海豐,中下。武德五年析置陸安縣,貞觀元年省。興寧,貞觀元年省齊昌縣入焉。雷鄉。中下。天授二年置。



 潮州潮陽郡,下。本義安郡。土貢:蕉、鮫革、甲香、蚺蛇膽、龜、石井、銀石、水馬。戶四千四百二十,口二萬六千七百四十五。縣三:海陽,中下。有鹽。潮陽,中下。永徽初省,先天初復置。程鄉。中下。



 康州晉康郡,下。本南康州,武德六年析端州之端溪置,九年州廢。貞觀元年復置,十一年又廢,十二年復置,更名康州。土貢:金、銀。戶萬五百一十,口萬七千二百一十九。縣四:端溪,下。武德五年析端州之博林置撫納縣,後省。晉康,下。本遂安,至德二載更名。悅城,下。本樂城,隸端州,武德五年來屬,後更名。都城。下。



 瀧州開陽郡,下。本永熙郡,天寶元年更名。土貢:銀、石斛。戶三千六百二十七,口九千四百三十九。縣四:瀧水,下。武德四年析置正義縣,並領懷德縣。後省正義,以懷德隸竇州。開陽,下。武德四年析瀧水置。鎮南,下。本安南,武德四年置南建州,以永熙郡之安遂、永熙、永業三縣隸之。五年析瀧水置安南縣。貞觀八年更南建州曰藥州。十八年州廢,省安遂、永業,以永寧、安南來屬。至德二載更名。建水。下。本永熙,武德五年曰永寧,天寶元年復更名,以建水在西也。



 端州高要郡,下。本信安郡,天寶元年更名。土貢:銀、柑。戶九千五百,口二萬一千一百二十。縣二:高要,下。貞觀十三年省博林縣入焉。東有青岐鎮。平興。下。武德七年析置清泰縣,貞觀十三年省。



 新州新興郡,下。本新昌郡,武德四年以端州之新興置。土貢:金、銀、蕉。戶九千五百。縣二:新興,下。武德四年析置索盧、新昌、單牒、永順四縣。後省新昌、單牒,乾元後又省索盧。永順。下。



 封州臨封郡,下。本廣信郡,天寶元年更名。土貢:銀、鮫革、石斛。戶三千九百,口萬一千八百二十七。縣二:封川,下。武德四年析置封興縣,後省。開建。下。武德四年置。



 潘州南潘郡,下。本南宕州南巴郡,武德四年以合浦郡之南昌、定川置。本治南昌,貞觀元年徙治定川,八年更名,後徙治茂名。後廢,地入高州。永徽元年復以茂名、南巴、毛山三縣置。土貢:銀。戶四千三百,口八千九百六十七。縣三:茂名,下。本隸高州,以茂名水名,貞觀元年來屬。潘水,下。武德五年置,以潘水名,又析南昌、定川置陸川、思城、溫水、宕川四縣。貞觀八年省思成,後以定川、宕川隸牢州,陸川、溫水隸禺州,後省南昌。二十三年析潘水置毛山縣,以毛山名。其後省潘水縣。開元二年改毛山曰潘水。南有博畔鎮。南巴。下。本隸高州,武德五年置,永徽元年來屬。



 春州南陵郡,下。本陽春郡,武德四年以高涼郡之陽春置,天寶元年更郡名。土貢:銀、鐘乳、石斛。戶萬一千二百一十八。縣二:陽春,下。武德四年並置流南縣,五年又置西城縣,後皆省。有鉛。羅水。下。天寶後置。



 勤州雲浮郡,下。本銅陵郡,武德四年析春州置,五年州廢。萬歲通天二年復置,長安中復廢。開元十八年平春、瀧等州,首領陳行範餘黨保銅陵北山,廣州都督耿仁忠奏復置州,治富林洞,因以為縣。乾元元年徙治銅陵。土貢:金、銀、石斛。戶六百八十二,口千九百三十三。縣二:銅陵,下。本隸端州,武德五年隸春州,後來屬。有銅。富林。下。武德四年析銅陵置。州廢,隸春州,後縣亦廢,乾元元年復置。



 羅州招義郡,下。本石城郡,武德五年以高涼郡之石龍、吳川置,六年徙治石城。土貢:銀、孔雀、鸚鵡。戶五千四百六十,口八千四十一。縣四:廉江,下。本石城,以石城水名。武德五年,析石龍、吳川置南河、石城、招義、零綠、石龍、陵羅、龍化、羅辯、慈廉、羅肥十縣。後以石龍而下六縣隸南石州。天寶元年更名。大歷八年以南河隸順州。吳川,下。乾水,下。本石龍,武德五年曰招義,天寶元年更名,以干水名。零綠。下。以零綠水名。



 辯州陵水郡,下。本南石州石龍郡,武德六年,以羅州之石龍、陵羅、龍化、羅辯、慈廉、羅肥置。貞觀九年更名。天祐元年,硃全忠以「辯」「汴」聲近,表更名勛州。土貢:銀、竹奚。戶四千八百五十八,口萬六千二百九。縣二:石龍,下。貞觀元年省慈廉、羅肥二縣入焉。陵羅。下。



 高州高涼郡,下。武德六年分廣州之電白、連江置。本治高涼,貞觀二十三年徙治良德,大歷十一年徙治電白。土貢:銀、蚺蛇膽。戶萬二千四百。縣三:電白,下。良德,下。本隸瀧州,武德中來屬。保寧。下。本連江,開元五年曰保安,至德二載更名。



 恩州恩平郡,下。本齊安郡,貞觀二十三年以高州之西平、齊安、杜陵置。大順二年徙治恩平。土貢:金、銀。戶九千。縣三:有清海軍。恩平,下。本海安,武德五年曰齊安,至德二載更名。有西平縣,本高涼,亦武德五年更名,後省。杜陵,下。本杜原,武德五年更名。陽江。下。有銀。



 雷州海康郡,下。本南合州徐聞郡,武德四年以合浦郡之海康、隋康、鐵杷置。貞觀元年更名東合州,八年又更名。土貢:絲電、班竹、孔雀。戶四千三百二十,口二萬五百七十二。縣三:海康,中。遂溪,下。本鐵杷、椹川二縣,後並省,更名。徐聞。下。本隋康,貞觀二年更名。



 崖州珠崖郡,下。土貢:金、銀、珠、玳瑁、高良姜。戶八百一十九。縣三:舍城,下。以舍城水名。西南有勤連鎮兵。有顏城縣,本顏盧,貞觀元年更名,開元後省。澄邁,下。文昌。下。本平昌,武德五年置,貞觀元年更名。



 瓊州瓊山郡,下都督府。貞觀五年以崖州之瓊山置。自乾封後沒山洞蠻,貞元五年,嶺南節度使李復討復之。土貢:金。戶六百四十九。縣五:瓊山,下。貞觀十三年析置曾口、顏羅、容瓊三縣。貞元七年省容瓊。有鹽。臨高,下。本臨機,隸崖州,貞觀五年來屬,州沒隸崖州。開元元年更名。曾口,下。樂會,下。顯慶五年置。顏羅。下。



 振州延德郡,下。本臨振郡,又曰寧遠郡,天寶元年更名。土貢:金、五色藤盤、班布、食單。戶八百一十九,口二千八百二十一。縣五:寧遠,下。以寧遠水名。有鹽。延德,下。以延德水名。吉陽,下。貞觀二年析延德置。臨川,下。落屯。下。天寶後置。



 儋州昌化郡,下。本儋耳郡,隋珠崖郡治,天寶元年更名。土貢:金、糖香。戶三千三百九。縣五:義倫,下。有鹽。昌化,下。貞觀元年析置吉安縣,乾元後省。感恩,下。洛場,下。乾元後置。富羅。下。本毘善,武德五年更名。



 萬安州萬安郡,下。龍朔二年以崖州之萬安置。開元九年徙治陵水。至德二載更名萬全郡。貞元元年復治萬全,後復故名。土貢:金、銀。戶二千九百九十七。縣四:萬安,下。本隸瓊州,貞觀五年析文昌置,並置富雲、博遼二縣。十三年隸崖州,後來屬。至德二載曰萬全,後復故名。陵水,下。本隸振州,後來屬。富雲,下。博遼。下。



 邕州朗寧郡,下都督府。本南晉州,武德四年以隋鬱林郡之宣化置。貞觀八年更名。土貢:金、銀。有金坑。戶二千八百九十三,口七千三百二。縣七:有經略軍。宣化,中下。武德五年析置武緣、晉興、朗寧、橫山四縣。乾元後省橫山。鬱水自蠻境七源州流出,州民常苦之,景雲中,司馬呂仁引渠分流以殺水勢,自是無沒溺之害,民乃夾水而居。武緣,中下。西有都稜鎮。晉興,中下。朗寧,中下。思籠,中下。乾元後開山洞置。如和,中下。本隸欽州,武德五年析南賓、安京置,景龍二年來屬。封陵。中下。乾元後開山洞置。



 澄州賀水郡,下。本南方州,武德四年以鬱林郡之嶺方地置,貞觀八年更名。土貢:金、銀。戶千三百六十八,口八千五百八十。縣四:上林,下。武德四年,析嶺方縣地置無虞、瑯邪、思乾、上林、止戈五縣。無虞,下。止戈,下。賀水。下。本隸柳州,武德四年析馬平置,八年來屬。



 賓州嶺方郡,下。本安城郡,貞觀五年,析南方州之嶺方、思乾、瑯邪,南尹州之安城置。至德二載更名。土貢:藤器。戶千九百七十六,口八千五百八十。縣三:嶺方,中下。貞觀十二年省思乾縣。瑯邪,中下。保城。中下。本安城,至德二載更名。



 橫州寧浦郡,下。本簡州,武德四年以鬱林郡之寧浦、樂山置。六年曰南簡州,貞觀八年更名。土貢:金、銀。戶千九百七十八,口八千三百四十二。縣三:寧浦,中下。武德四年析置蒙澤縣。五年以貴州之嶺山來屬。貞觀十二年省蒙澤入焉,後又省嶺山。從化,中下。本淳風,武德四年析寧浦置,永貞元年更名。樂山。下。



 潯州潯江郡,下。貞觀七年以燕州之桂平、大賓置。十三年州廢,縣隸龔州,後復置。土貢:金、銀。戶二千五百,口六千八百三十六。縣三:桂平,下。本隸貴州,武德五年隸燕州。七年置陵江縣,十二年省入焉。皇化,下。本隸繡州,貞觀七年來屬。大賓,下。



 巒州永定郡,下。本淳州,武德四年以故秦桂林郡地置,永貞元年更名。土貢:金、銀。戶七百七十,口三千八百三。縣三:永定,下。武羅,下。靈竹。下。



 欽州寧越郡。土貢:金、銀、翠羽、高良姜。戶二千七百,口萬一百四十六。縣五:欽江,下。東南有西零戍。保京,下。本安京,至德二載更名,內亭,下。武德五年以內亭、遵化二縣置南亭州,貞觀二年州廢,二縣來屬。遵化,中下。靈山。下。本南賓,貞觀十年更名。



 貴州懷澤郡,下。本南定州鬱林郡,武德四年曰南尹州,貞觀八年曰貴州,天寶元年更郡名。土貢:金、銀、鉛器、紵布。戶三千二十六,口九千三百。縣四:有府一,曰龍山。鬱林,中下。懷澤,下。武德四年置。潮水,下。武德四年析鬱林置。義山。下。武德四年更馬嶺縣曰馬度。貞觀後省,天寶後更置,曰義山。



 龔州臨江郡,下。貞觀七年,以燕州故治,析潯州之武林、燕州之泰川置,後徙治平南。土貢:銀。戶九千,口二萬一千。縣五:平南,下。貞觀七年置,又置西平、歸政、大同三縣。十二年省泰川入平南,又省歸政、西平。武林,下。本隸藤州,貞觀七年來屬。隋建,下。本隸藤州,貞觀十三年來屬。大同,下。陽川。下。本陽建,後更名。



 象州象郡,下。本桂林郡,武德四年以始安郡之陽壽、桂林置,以象山為州名。貞觀十三年徙治武化,大歷十一年復治陽壽。土貢:銀、藤器。戶五千五百,口萬八百九十。縣三:陽壽,下。武德四年析桂林置武德、西寧、武仙三縣。貞觀十二年省西寧入武德,天寶元年省武德入陽壽。武仙,下。乾封元年省桂林縣入焉。武化。下。武德四年析桂州之建陵置,本隸封州,後隸晏州;又析陽壽置長風縣,隸晏州。州廢,縣皆來屬,大歷十一年省長風入焉。



 藤州感義郡,下。本永平郡,天寶元年更名。土貢:銀。戶三千九百八十。縣四:鐔津,中下。初州治永平,無鐔津,又有隋安、賀川、寧人等縣,皆貞觀後省並更置,而寧人隸容州,永平隸昭州。有鉛。感義,下。本淳民,武德中更名。義昌,下。本安昌,至德二載更名。寧風。下。武德五年以縣置燕州,以貴州之桂平隸之。貞觀三年又以藤州之大賓隸之,增領長恭、泰川、池陽、龍陽四縣,治長恭,五年置新樂、寧風、梁石、羅風四縣。七年更名泰州,徙治寧風,更池陽曰承恩,復以藤州之安基隸之;以梁石、羅風隸藤州;省長恭縣。八年徙治安基,復為燕州。十二年省龍陽、承恩二縣。十八年州廢,以寧風來屬。後省新樂、安基、梁石、羅風。



 巖州常樂郡,下。調露二年析橫、貴二州置,以巖岡之北因為名。天寶元年曰安樂郡,至德二載更名。土貢:金。戶千一百一十。縣四:常樂,下。本安樂,蕭銑分興德縣置。貞觀元年省,乾封元年復置,隸鬱林州,永隆元年來屬。至德二載更名。恩封,下。本伏龍洞,當牢、宜二州之境,調露二年與高城、石巖同置。高城,下。以高城水名。石巖。下。



 宜州龍水郡,下。唐開置,本粵州,乾封中更名。有銀、丹沙。戶千二百二十,口三千二百三十。縣四:龍水,下。崖山,下。東璽,下。天河,下。邕管所領,又有顯州、武州、沈州,後皆廢省。



 瀼州臨潭郡,下。貞觀十二年,清平公李弘節開夷獠置。戶千六百六十六。縣四:瀼江,下。波零,下。鵠山,下。弘遠。下。貞元後州、縣名存而已。



 籠州扶南郡,下。貞觀十二年,李弘節招慰生蠻置。戶三千六百六十七。縣七:武勤,下。武禮,下。羅籠,下。扶南,下。龍額,下。武觀,下。武江,下。



 田州橫山郡,下。開元中開蠻洞置,貞元二十一年廢,後復置。戶四千一百六十八。縣五:都救,下。惠佳,下。武龍,下。橫山,下。如賴。下。



 環州整平郡,下。貞觀十二年,李弘節開拓生蠻置。縣八:正平,下。福零,下。龍源,下。饒勉,下。思恩,下。武石,下。歌良,下。都蒙。下。



 桂州始安郡,中都督府。至德二載更郡曰建陵,後復故名。土貢:銀、銅器、麖皮鞾、簟。戶萬七千五百,口七萬一千一十八。縣十一:有經略軍。臨桂,上。本始安,武德四年置福祿縣,貞觀八年省入焉,更名。有相思埭,長壽元年築,分相思水使東西流。又東南有回濤堤,以捍桂水,貞元十四年築。有侯山。理定,中。本興安,武德四年置宣風縣,貞觀十二年省入焉。至德二載更名。西十里有靈渠,引漓水,故秦史祿所鑿,後廢。寶歷初,觀察使李渤立斗門十八以通漕,俄又廢。咸通九年,刺史魚孟威以石為鏵堤,亙四十里,植大木為斗門,至十八重,乃通巨舟。靈川,中。龍朔二年析始安置。陽朔,中下。武德四年置歸義縣,貞觀元年省入焉。荔浦,中下。武德四年,以始安郡之荔浦、建陵、隋化三縣置荔州,又析置崇仁、純義、東區三縣。五年以隋化、東區隸南恭州,貞觀元年以建陵隸晏州。十二年州廢,以荔浦、崇仁來屬。崇仁後省,純義隸蒙州。豐水,中下。本永豐,隸昭州,武德四年析陽朔置,後來屬。長慶三年更名。修仁,中下。本建陵,貞觀元年置晏州,並置武龍、武化、長風三縣。十二年州廢,省武龍,以武化、長風隸象州,建陵來屬。長慶三年更名。恭化,中下。本純化,武德四年析始安置,永貞元年更名。永福,中下。武德四年析始安置。全義,中下。本臨源,武德四年析始安置,大歷三年更名。古。乾寧二年析慕化置。



 梧州蒼梧郡,下。武德四年以靜州之蒼梧、豪靜、開江置。土貢:銀、白石英。戶千二百九。縣三:蒼梧,下。貞觀八年以賀州之綏越來屬。十二年省豪靜,其後又省綏越,而開江復隸富州。戎城,下。本隸藤州,永徽中來屬。光化四年,馬殷表以縣隸桂州。孟陵。下。本猛陵,隸藤州,蕭銑置。貞觀八年來屬,更名。光化中,馬殷表以縣隸桂州。



 賀州臨賀郡,下。本綏越郡,武德四年,以始安郡之富川、熙平郡之桂嶺、零陵郡之馮乘、蒼梧郡之封陽置。土貢:銀。戶四千五百五十二,口二萬五百七十。縣六:臨賀,下。武德四年置。東有銅冶,在橘山。桂嶺,下。朝岡、程岡皆有鐵。馮乘,下。有荔平關,有錫冶三。封陽,下。貞觀元年省,九年復置。富川,下。有富水。天寶中更名富水,後復故名。有錫,有鐘乳穴三。蕩山。下。天寶後置。



 連州連山郡,下。本熙平郡,天寶元年更名。土貢:赤錢、竹紵練、白紵細布、鐘乳、水銀、丹沙、白鑞。戶三萬二千二百一十,口十四萬三千五百三十三。縣三:桂陽,上。有桂陽山,本靈山,天寶八載更名。有銀,有鐵。陽山,中下。有鐵,有故秦湟溪關。連山。中。有金,有銅,有鐵。



 柳州龍城郡,下。本昆州,武德四年以始安郡之馬平置,是年,更名南昆州,貞觀八年又以地當柳星更名。土貢:銀、蚺蛇膽。戶二千二百三十二,口萬一千五百五十。縣五:馬平,下。武德四年析置新平、文安、賀水、歸德四縣,尋更名歸德曰脩德,文安曰樂沙。八年以賀水隸澄州。貞觀七年省樂沙,九年置崖山縣,十二年省新平。其後又省崖山,以脩德隸嚴州。龍城,下。武德四年置龍州,並置柳嶺縣。貞觀七年州廢,省柳嶺,以龍城來屬。象,下。本隸桂州,後來屬。洛曹,下。本洛封,元和十三年更名。洛容。下。貞觀中置。



 富州開江郡,下。本靜州龍平郡,武德四年,以始安郡之龍平、豪靜,蒼梧郡之蒼梧置,貞觀八年更名。土貢:銀、班布。戶千四百六十,口八千五百八十六。縣三:龍平,下。武德四年析置博勞、歸化、安樂、開江四縣,尋以蒼梧、豪靜、開江隸梧州,九年省安樂、歸化、博勞。思勤,下。天寶後置。馬江。下。本開江,後隸梧州,又復隸柳州。長慶三年更名。



 昭州平樂郡,下。本樂州,武德四年以始安郡之平樂置,貞觀八年更名。土貢:銀。戶四千九百一十八,口萬二千六百九十一。縣三:平樂,下。以平樂水名之。有鐘乳穴三。武德四年析置沙亭縣,貞觀七年省沙亭。恭城,下。蕭銑置。有鐘乳穴十二,在銀帳山。永平。下。本隸藤州,後來屬。



 蒙州蒙山郡,下。本南恭州,武德五年析荔州之隋化置,貞觀八年更名。土貢:麩金、銀。戶千五十九,口五千九百三十三。縣三:立山,下。本隋化,武德五年更名;又析置欽政縣,貞觀十二年省。東區,下。武德五年析立山置。貞觀六年隸燕州,十年來屬。正義。下。本純義,隸燕州,十年來屬。永貞元年更名。



 嚴州循德郡,下。乾封二年招致生獠,以秦故桂林郡地置。土貢:銀。戶千八百五十九,口七千五十一。縣三:來賓,下。乾封二年置。循德,下。本隸柳州,後來屬。歸化。下。乾封二年置。



 融州融水郡,下。武德四年析始安郡之義熙置。土貢:金、桂心。戶千二百三十二。縣二:融水,下。本義熙,武德四年析置臨牂、黃水、安脩三縣,六年更名。貞觀十三年省安脩入臨牂。武陽。下。天寶初並黃水、臨牂二縣更置。



 思唐州武郎郡,下。永隆二年析龔、蒙、象三州置。開元二十四年為羈縻州,建中元年為正州。土貢:銀。戶百四十一。縣二:武郎,下。思和。下。本平原,長慶三年更名。



 古州樂興郡,下。貞觀十二年,李弘節開夷獠置。土貢:蠟。戶二百八十五。縣三:樂山,本樂預,寶應元年更名。古書,下。樂興。下。



 容州普寧郡,下都督府。本銅州,武德四年以合浦郡之北流、普寧置。貞觀八年更名。元和中徙治普寧。土貢:銀、丹沙、水銀。戶四千九百七十,口萬七千八十五。縣六:有經略軍。普寧,下。北流,下。武德四年析置豪石、宕昌、南流、陵城、新安五縣。貞觀十一年省新安,後又省豪石、宕昌。北三十里有鬼門關,兩石相對,中闊三十步。陵城,下。渭龍,下。武德四年析普寧置。欣道,下。本寧人,隸藤州。貞觀二十三年更名,來屬。陸川。下。本隸東峨州,唐末來屬。



 牢州定川郡,下。本義州,武德二年以巴蜀徼外蠻夷地置。貞觀十一年以東北有牢石,因更名,徙治南流,後廢。乾封三年,將軍王杲平蠻獠復置。土貢:布、銀。戶千六百四十一,口萬一千七百五十六。縣三:南流,下。本隸容州,武德四年析北流置南流、定川、牢川三縣,以南百步有南流江名之,乾封三年皆來屬。定川,下。本隸潘州,定川水名之。宕川。下。本隸潘州,因瀘宕水名之。



 白州南昌郡,下。本南州,武德四年以合浦郡之合浦地置,六年更名。土貢:金、銀、珠。戶二千五百七十四,口九千四百九十八。縣四:博白,下。武德四年置,並置朗平、周羅、龍豪、淳良、建寧五縣。貞觀六年以廉州之大都隸之。十二年省郎平、淳良,後又省大都。大歷八年以龍豪隸順州。西南百里有北戍灘,咸通中。安南都護高駢募人平其險石,以通舟楫。建寧,下。周羅,下。南昌。下。本隸潘州,後來屬。



 順州順義郡,下。大歷八年,容管經略使王翃析禺、羅、辯、白四州置。土貢:銀。戶五百九。縣四:龍化,下。武德四年置,以西有龍化水名之,六年隸辯州。溫水,下。本隸禺州。南河,下。武德五年析石龍置,隸羅州。龍豪。武德四年析合浦置,隸白州。



 繡州常林郡,下。本林州,武德四年以鬱林郡之阿林縣及鬱平縣地置,六年更名。土貢:金。戶九千七百七十三。縣三:常林,中。武德四年置,又置羅繡、皇化、歸誠三縣。貞觀七年以皇化隸潯州,省歸誠。阿林,中下。羅繡。下。武德四年析置盧越縣,貞觀六年縣廢入焉。



 鬱林州鬱林郡,下。本鬱州,麟德二年析貴州之石南、興德、鬱平置,乾封元年更名。土貢:布。戶千九百一十八,口九千六百九十九。縣四:鬱平,下。本隸貴州,後來屬。興業,下。麟德二年析石南置,建中二年省石南入焉。興德,下。蕭銑析石南置,尋廢。武德四年析鬱平復置。潭慄。下。



 黨州寧仁郡,下。本鬱林州地,永淳元年開古黨洞置。土貢:金、銀。戶千一百四十九,口七千四百四。縣八:撫安,下。古西甌地。善勞,中下。善文,下。寧仁,下。容山,下。本安仁,永淳二年析黨州置平琴州平琴郡,領安仁、懷義、福陽、古符四縣。垂拱三年廢,神龍三年復置。至德中更安仁曰容山。建中二年州廢,縣皆來屬。懷義,下。福陽,下。古符。下。



 竇州懷德郡,下。本南扶州,武德四年以永熙郡之懷德置。以獠叛,僑治瀧州,後徙治信義。貞觀元年州廢,以縣隸瀧州。二年復置,五年又廢,以縣隸瀧州。六年復置,八年更名。土貢:銀。戶千一十九,口七千三百三十九。縣四:信義,中下。武德四年置,並析置潭峨縣,五年又析置特亮縣。懷德,中下。潭峨,下。特亮。下。



 禺州溫水郡,下。本東峨州,乾封三年,將軍王杲奏析白、辯、竇、容四州置,總章二年更名。土貢:銀。戶三千一百八十。縣四:峨石,下。總章二年析白州之溫水置,以南有峨石名之。羅辯,下。本陸川,隸辯州,後更名。本羅辯洞地。扶萊,下。武德五年析信義縣置,隸竇州,以扶萊水名之。貞觀中省,後復置。宕昌。下。本隸容州。



 廉州合浦郡,下。本合州,武德四年曰越州,貞觀八年更名,以本大廉洞地。土貢:銀。戶三千三十二,口萬三千二十九。縣四:合浦,中下。武德五年置安昌、高城、大廉、大都四縣。貞觀六年置珠池縣。後以大都隸白州。十二年省珠池、安昌入焉。封山,下。武德五年置姜州,並置東羅、蔡龍二縣。貞觀十年州廢,以封山、東羅、蔡龍來屬。後省東羅。蔡龍,下。以蔡龍洞名之。貞觀十二年省高城縣入焉。大廉。下。



 義州連城郡,下。本南義州,武德五年以永熙郡之永業縣地置。貞觀元年州廢,以縣隸南建州。二年復置,五年又廢,以縣隸南建州。六年復置,後第名義州。土貢:銀。戶千一百一十,口七千三百三。縣三:岑溪,下。本龍城,武德五年置,並置安義、義城二縣。至德中更龍城曰岑溪。其後又省義城。有郡山。永業,下。本安義,至德中更名。連城。下。武德五年析瀧州之正義置。



 安南中都護府,本交趾郡,武德五年曰交州,治交趾。調露元年曰安南都護府,至德二載曰鎮南都護府,大歷三年復為安南。寶歷元年徙治宋平。土貢:蕉、檳榔、鮫革、蚺蛇膽、翠羽。戶二萬四千二百三十,口九萬九千六百五十二。縣八:有經略軍。宋平,上。武德四年置宋州,並置弘教、南定二縣。五年析置交趾、懷德二縣,隸交州。六年曰南宋州。貞觀元年州廢,省弘教、懷德,徙交趾於故南慈州,來屬。南定,本隸宋州,武德四年析宋平置,五年隸交州。大歷五年省,貞元八年復置。太平,中下。本隆平,武德四年置,以縣置隆州,並置義廉、封溪二縣,治義廉。六年曰南隆州。貞觀元年州廢,省義廉,以封溪隸峰州,隆平來屬。先天元年更名。交趾,中下。武德四年置慈州,並置慈廉、烏延、武立三縣,以慈廉水因名之。六年曰南慈州。貞觀元年州廢,省三縣更置。硃鳶,上。武德四年置鳶州,並置高陵、定安二縣。貞觀元年州廢,省高陵、定安,以硃鳶來屬。龍編,中下。武德四年置龍州,並置武寧、平樂二縣。貞觀元年州廢,省武寧、平樂,以龍編隸仙州,州廢來屬。平道,中下。武德四年置道州,並置昌國縣。六年曰南道州,是年更名仙州。貞觀十年州廢,省昌國,以平道來屬。武平。中下。本隸道州,武德五年來屬。



 陸州玉山郡,下。本玉山州,武德五年以寧越郡之安海、玉山置。貞觀二年州廢,縣隸欽州。高宗上元二年復置,更名。土貢:銀、玳瑁、裛皮、翠羽、甲香。戶四百九十四,口二千六百七十四。縣三:烏雷,下。華清,下。本玉山,天寶中更名。寧海。下。本安海。武德四年又置海平縣,貞觀十二年省。至德二載更名。



 峰州承化郡,下都督府。武德四年以交趾郡之嘉寧置。土貢:銀、藤器、白鑞、蚺蛇膽、豆寇。戶千九百二十。縣五:嘉寧,下。武德四年置新昌、安仁、竹格、石堤四縣,又領封溪縣。貞觀元年省石堤、封溪入嘉寧,後又省安仁。承化,下。新昌,下。貞觀元年省竹格縣入焉。嵩山,元和後置。珠綠。元和後置。



 愛州九真郡,下。土貢:紗、施、孔雀尾。戶萬四千七百。縣六:九真,下。武德五年置松源、楊山、安預三縣。貞觀元年省楊山、安預,九年省松源。有金,有石磬。安順,下。武德五年置順州,並析置東河、建昌、邊河三縣。貞觀元年州廢,省三縣入安順,來屬。崇平,下。本隆安。武德五年置安州,並置教山、建道、都握三縣,又置山州,並置岡山、真潤、古安、西安、建初五縣。貞觀元年廢安州,省教山、建道、都握入隆安,來屬;又廢山州,省岡山、真潤、古安、西安入建初,來屬。八年省建初。先天元年更隆安曰崇安,至德二載又更名。軍寧,下。本軍安,武德五年以縣置永州,七年曰都州。貞觀元年州廢,隸南陵州。至德二載更名。日南,下。武德五年置積州,又置積善、津梧、方載三縣;又以移風縣地置前真州,並置九皋、建正、真寧三縣;又以胥浦縣置胥州,並置攀龍、如侯、博犢、鎮星四縣。九年更積州曰南陵州,貞觀元年曰後真州。是年廢前真州,省九皋、建正、真寧,以移風隸南陵州;又廢胥州,省攀龍、如侯、博犢、鎮星,以胥浦隸南陵州。十年州亦廢,以軍安、日南、移風、胥浦來屬,天寶中省移風、胥浦。長林。下。本無編。



 驩州日南郡,下都督府。本南德州,武德八年曰德州,貞觀元年又更名。土貢:金、金薄、黃屑、象齒、犀角、沈香、班竹。戶九千六百一十九,口五萬八百一十八。縣四:九德,中下。武德五年置安遠、曇羅、光安三縣。是年,以光安置源州,又置水源、安銀、河龍、長江四縣。貞觀八年更名阿州。十三年州廢,省水源、河龍、長江,以光安、安銀來屬。安遠、曇羅、光安、安銀後皆省。浦陽,下。越裳,下。武德五年置明州,並置萬安、明弘、明定三縣;又以日南郡之文谷、金寧二縣置智州,並置新鎮、闍員二縣。貞觀元年更曰南智州,省新鎮、闍員。十三年廢明州,省萬安、明弘、明定入越裳,隸智州。後廢智州,省文谷、金寧入越裳,來屬。初以隋林邑郡置林州,比景郡置七州。又更名七州曰南景州,貞觀二年綏懷林邑,乃僑治驩州之南境,領比景、硃吾二縣,並置由文縣。八年第名景州。九年置林州,亦寄治驩州之南境,領林邑、金龍、海界三縣。又置山州,領龍池、盆山二縣。有浦陽戍。戶千三百二十,口五千二百。後為龍池郡。皆貞元末廢。懷驩。下。本咸歡,武德五年置驩州,並置安人、扶演、相景、西源四縣,治安人。貞觀元年更名演州。十三年省相景。十六年州廢,省安人、扶演、西源,以咸驩來屬。後更咸驩曰懷驩。



 長州文楊郡,下。唐置。土貢:金。戶六百四十八。縣四:文陽,下。銅蔡,下。長山,下。其常。下。



 福祿州唐林郡,下。本福祿郡,總章二年,智州刺史謝法成招慰生獠昆明、北樓等七千餘落,以故唐林州地置。大足元年更名安武州,至德二載更郡曰唐林,乾元元年復州故名。土貢:白鑞、紫穀。戶三百一十七。縣三:柔遠,本安遠,至德二載更名。唐林,唐初以唐林、安遠二縣置唐林州,後州、縣皆廢,更置。福祿。下。



 湯州湯泉郡,下。唐以故秦象郡地置。土貢:金。縣三:湯泉,下。綠水,下。羅韶。下。



 芝州忻城郡,下。唐置。戶千二百,口五千三百。縣七:忻城,下。富川,下。平西,下。樂光,下。樂艷,下。多云,下。思龍。下。



 武峨州武峨郡,下。唐置。戶千八百五十,口五千三百二十。縣七:武峨,下。如馬,下。武義,下。武夷,下。武緣,下。武勞,下。梁山。下。



 演州龍池郡,下。本忠義郡,又曰演水郡。貞觀中廢,廣德二年析驩州復置。土貢:金。戶千四百五十。縣七:忠義,下。懷驩,下。龍池,下。思農,下。武郎,下。武容,下。武金。下。



 武安州武曲郡,下。土貢:金、朝霞布。戶四百五十。縣二:武安,下。臨江,下。



 開元中安南所領有龐州,土貢:孔雀尾、紫穀;又有南登州。後皆廢省。



 右嶺南採訪使,治廣州。



\end{pinyinscope}