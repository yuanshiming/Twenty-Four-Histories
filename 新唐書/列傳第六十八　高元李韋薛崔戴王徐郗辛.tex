\article{列傳第六十八 高元李韋薛崔戴王徐郗辛}

\begin{pinyinscope}

 高適,字達夫,滄州渤海人。少落魄,不治生事。客梁、宋間,宋州刺史張九皋奇之出《禮記·禮運》:「大道之行也,天下為公,選賢與能,講,舉有道科中第,調封丘尉,不得志,去。客河西,河西節度使哥舒翰表為左驍衛兵曹參軍,掌書記。祿山亂,召翰討賊,即拜適左拾遺,轉監察御史,佐翰守潼關。翰敗,帝問群臣策安出,適請竭禁藏募死士抗賊,未為晚,不省。天子西幸,適走間道及帝於河池,因言:「翰忠義有素,而病奪其明,乃至荒踣。監軍諸將不恤軍務,以倡優蒲飀相娛樂,渾、隴武士飯糲米日不厭,而責死戰,其敗固宜。又魚炅、何履光、趙國珍屯南陽,而一二中人監軍更用事,是能取勝哉?臣數為楊國忠言之,不肯聽。故陛下有今日行,未足深恥。」帝頷之。俄遷侍御史,擢諫議大夫,負氣敢言,權近側目。帝以諸王分鎮,適盛言不可,俄而永王叛。肅宗雅聞之,召與計事,因判言王且敗,不足憂。帝奇之,除揚州大都督府長史、淮南節度使。詔與江東韋陟、淮西來瑱率師會安陸,方濟師而王敗。李輔國惡其才,數短毀之,下除太子少詹事。



 未幾蜀亂,出為蜀、彭二州刺史。始,上皇東還,分劍南為兩節度,百姓弊於調度,而西山三城列戍。適上疏曰:「劍南雖名東、西川,其實一道。自邛關、黎、雅以抵南蠻,由茂而西,經羌中、平戎等城,界吐蕃。瀕邊諸城,皆仰給劍南。異時以全蜀之饒,而山南佐之,猶不能舉,今裂梓、遂等八州專為一節度,歲月之計,西川不得參也。嘉陵比困夷獠,日雖小定,而痍痏未平,耕紡亡業,衣食貿易皆資成都,是不可得役亦明矣。可稅賦者,獨成都、彭、蜀、漢四州而已,以四州耗殘當十州之役,其弊可見。而言利者,枘鑿萬端,窮朝抵夕,千案百牘,皆取之民,官吏懼譴,責及鄰保,威以罰抶,而逋逃益滋。又關中比饑,士人流入蜀者道路相系,地入有訖,而科斂無涯,為蜀計者,不亦難哉!又平戎以西數城,皆窮山之顛,蹊隧險絕,運糧束馬之路,坐甲無人之鄉。為戎狄言,不足利戎狄;為國家言,不足廣土宇。柰何以彈丸地而困全蜀太平之人哉?若謂已戍之城不可廢,已屯之兵不可收,願罷東川,以一劍南並力從事。不爾,非陛下洗蕩關東清逆亂之意也。蜀人又擾,則貽朝廷憂。」帝不納。



 梓屯將段子璋反,適從崔光遠討斬之。而光遠兵不戢,遂大掠,天子怒,罷光遠,以適代為西川節度使。廣德元年,吐蕃取隴右,適率兵出南鄙,欲牽制其力,既無功,遂亡松、維二州及雲山城。召還,為刑部侍郎、左散騎常侍,封渤海縣侯。永泰元年卒,贈禮部尚書,謚曰忠。



 適尚節義,語王霸袞袞不厭。遭時多難,以功名自許,而言浮其術,不為搢紳所推。然政寬簡,所涖,人便之。年五十始為詩,即工,以氣質自高。每一篇已,好事者輒傳布。其詒書賀蘭進明,使救梁、宋以親諸軍,與許叔冀書,令釋憾;未度淮,移檄將校,絕永王,俾各自白,君子以為義而知變。



 元結,後魏常山王遵十五代孫。曾祖仁基,字惟固,從太宗征遼東,以功賜宜君田二十頃,遼口並馬牝牡各五十,拜寧塞令,襲常山公。祖亨,字利貞,美姿儀。嘗曰:「我承王公餘烈,鷹犬聲樂是習,吾當以儒學易之。」霍王元軌聞其名,闢參軍事。父延祖,三歲而孤,仁基敕其母曰:「此兒且祀我。」因名而字之。逮長,不仕,年過四十,親婭強勸之,再調舂陵丞,輒棄官去,曰:「人生衣食,可適饑飽,不宜復有所須。」每灌畦掇薪,以為「有生之役,過此吾不思也」。安祿山反,召結戒曰:「而曹逢世多故,不得自安山林,勉樹名節,無近羞辱」云。卒年七十六,門人私謚曰太先生。



 結少不羈,十七乃折節向學,事元德秀。天寶十二載舉進士,禮部侍郎陽浚見其文,曰:「一第慁子耳,有司得子是賴!」果擢上第。復舉制科。會天下亂,沈浮人間。國子司業蘇源明見肅宗,問天下士,薦結可用。時史思明攻河陽,帝將幸河東,召結詣京師,問所欲言,結自以始見軒陛,拘忌諱,恐言不悉情,乃上《時議》三篇。其一曰:



 議者問:「往年逆賊,東窮海,南淮、漢,西抵函、秦,北徹幽都,醜徒狼扈,在四方者幾百萬,當時之禍可謂劇,而人心危矣。天子獨以匹馬至靈武,合弱旅,鉏強寇,師及渭西,曾不逾時,摧銳攘兇,復兩京,收河南州縣,何其易邪?乃今河北奸逆不盡,山林江湖亡命尚多,盜賊數犯州縣,百姓轉徙,踵系不絕,將士臨敵而奔,賢人君子遁逃不出。陛下往在靈武、鳳翔,無今日勝兵而能殺敵,無今日檢禁而無亡命,無今日威令而盜賊不作,無今日財用而百姓不流,無今日爵賞而士不散,無今日朝廷而賢者思仕,何哉?將天子能以危為安,而忍以未安忘危邪?」對曰:「此非難言之。前日天子恨愧陵廟為羯逆傷污,憤悵上皇南幸巴、蜀,隱悼宗戚見誅,側身勤勞,不憚親撫士卒,與人權位,信而不疑,渴聞忠直,過弗諱改。此以弱制強,以危取安之繇也。今天子重城深宮,燕和而居;凝冕大昕,纓佩而朝;太官具味,視時而獻,太常備樂,和聲以薦;國機軍務,參籌乃敢進;百姓疾苦,時有不聞;廄芻良馬、宮籍美女、輿服禮物、休符瑞諜,日月充備;朝廷歌頌盛德大業,聽而不厭;四方貢賦,爭上尤異;諧臣顐官,怡愉天顏;文武大臣至於庶官,皆權賞逾望。此所以不能以強制弱,以未安忘危。若陛下視今日之安,能如靈武時,何寇盜強弱可言哉!」



 其二曰:



 議者曰:「吾聞士人共自謀:『昔我奉天子拒兇逆,勝則家國兩全,不勝則兩亡,故生死決於戰,是非極於諫。今吾名位重,財貨足,爵賞厚,勤勞已極,外無仇讎害我,內無窮賤迫我,何苦當鋒刃以近死,忤人主以近禍乎?』又聞曰:『吾州里有病父老母、孤兄寡婦,皆力役乞丐,凍餒不足,況於死者,人誰哀之?』又聞曰:『天下殘破,蒼生危窘,受賦與役者,皆寡弱貧獨,流亡死徙,悲憂道路,蓋亦極矣。天下安,我等豈無畎畝自處?若不安,我不復以忠義仁信方直死矣!』人且如此,柰何?」對曰:「國家非欲其然,蓋失於太明太信耳。夫太明則見其內情,將藏內情則罔惑生下。能令必信,信可必矣,而太信之中,至奸尤惡之。如此遂使朝廷亡公直,天下失忠信,蒼生益冤結。將欲治之,能無端由?吾等議於野,又何所及?」



 其三曰:



 議者曰:「陛下思安蒼生,滅奸逆,圖太平,勞心悉精,於今四年,說者異之,何哉?」對曰:「如天子所思,說者所異,非不知之。凡有詔令丁寧事皆不行,空言一再,頗類諧戲。今有仁血阜之令,憂勤之誥,人皆族立黨語,指而議之。天子不知其然,以為言雖不行,猶足以勸。彼沮勸,在乎明審均當而必行也。天子能行已言之令,必將來之法,雜徭弊制,拘忌煩令,一切蠲蕩,任天下賢士,屏斥小人,然後推仁信威令,謹行不惑。此帝王常道,何為不及?」



 帝悅曰:「卿能破朕憂。」擢右金吾兵曹參軍,攝監察御史,為山南西道節度參謀。募義士於唐、鄧、汝、蔡,降劇賊五千,瘞戰死露胔於泌南,名曰哀丘。



 史思明亂,帝將親征,結建言:「賊銳不可與爭,宜折以謀。」帝善之,因命發宛、葉軍挫賊南鋒,結屯泌陽守險,全十五城。以討賊功遷監察御史裏行。荊南節度使呂諲請益兵拒賊,帝進結水部員外郎,佐諲府。又參山南東道來瑱府,時有父母隨子在軍者,結說瑱曰:「孝而仁者,可與言忠;信而勇者,可以全義。渠有責其忠信義勇而不勸之孝慈邪?將士父母,宜給以衣食,則義有所存矣。」瑱納之。瑱誅,結攝領府事。會代宗立,固辭,丐侍親歸樊上。授著作郎。益著書,作《自釋》,曰:



 河南,元氏望也。結,元子名也。次山,結字也。世業載國史,世系在家諜。少居商餘山,著《元子》十篇,故以元子為稱。天下兵興,逃亂入猗玗洞,始稱猗玗子。後家瀼濱,乃自稱浪士。及有官,人以為浪者亦漫為官乎,呼為漫郎。既客樊上,漫遂顯。樊左右皆漁者,少長相戲,更曰聱叟。彼誚以聱者,為其不相從聽,不相鉤加,帶笭箵而盡船,獨聱齖而揮車。酒徒得此,又曰:「公之漫其猶聱乎?公守著作,不帶笭箵乎?又漫浪於人間,得非聱齖乎?公漫久矣,可以漫為叟。」於戲!吾不從聽於時俗,不鉤加於當世,誰是聱者,吾欲從之!彼聱叟不慚帶乎笭箵,吾又安能薄乎著作?彼聱叟不羞聱齖於鄰里,吾又安能慚漫浪於人間?取而醉人議,當以漫叟為稱。直荒浪其情性,誕漫其所為,使人知無所存有,無所將待。乃為語曰:「能帶笭箵,全獨而保生;能學聱齖,保宗而全家。聱也如此,漫乎非邪!」



 久之,拜道州刺史。初,西原蠻掠居人數萬去,遺戶裁四千,諸使調發符牒二百函,結以人困甚,不忍加賦,即上言:「臣州為賊焚破,糧儲、屋宅、男女、牛馬幾盡。今百姓十不一在,耄孺騷離,未有所安。嶺南諸州,寇盜不盡,得守捉候望四十餘屯,一有不靖,湖南且亂。請免百姓所負租稅及租庸使和市雜物十三萬緡。」帝許之。明年,租庸使索上供十萬緡,結又奏:「歲正租庸外,所率宜以時增減。」詔可。結為民營舍給田,免徭役,流亡歸者萬餘。進授容管經略使,身諭蠻豪,綏定八州。會母喪,人皆詣節度府請留,加左金吾衛將軍。民樂其教,至立石頌德。罷還京師,卒,年五十,贈禮部侍郎。



 李承,趙州高邑人。幼孤,其兄曄養之。既長,以悌聞。擢明經,遷累大理評事,為河南採訪使判官。尹子奇陷汴州,拘承送洛陽,覘得賊謀,皆密啟諸朝。兩京平,例貶臨川尉。不三月,除德清令。尋擢監察御史,累遷吏部郎中,淮南西道黜陟使。奏置常豐堰于楚州,以御海潮,溉屯田脊鹵,收常十倍它歲。德宗將討梁崇義,李希烈揣知之,乃表崇義過惡,請先誅討,帝悅,數對左右稱其忠。會承使回,言希烈能立功,然恐後不可制,帝初謂不然,及崇義平,希烈果叛,始思其言,擢拜河中尹、晉絳觀察使。承廉正有雅望,以才顯於時。未幾,改山南東道節度使。時希烈猶據襄州,帝慮不受命,欲以禁兵衛送承,承辭,請以單騎入。既至,希烈舍承外館,迫脅日萬端,承晏然誓以死守。希烈不能屈,遂大掠去,襄、漢蕩然。承輯綏撫安之,居一年,闔境完復。初,希烈雖去,留部校守覘,往來踵舍,承因得使所厚臧叔雅結希烈腹心周會、王玢、姚詹。及曾等謀殺希烈,承首謀也。密詔褒美。尋檢校工部尚書、湖南觀察使。建中四年卒,年六十二,贈吏部尚書。



 韋倫,系本京兆。父光乘,在開元、天寶間為朔方節度使。倫以廕調藍田尉,幹力勤濟,楊國忠署為鑄錢內作使判官。國忠多發州縣齊人令鼓鑄,督非所習,雖箠失苛嚴,愈無功。倫請準直募匠,代無聊之人,繇是役用減,鼓鑄多矣。玄宗晚節盛營宮室,吏介以為欺,倫閱實工員,省費倍。從帝入蜀,以監察御史為劍南節度行軍司馬、置頓判官。時中人衛卒多侵暴,尤難治,倫以清儉自將,西人賴濟。中宦疾之,以讒貶衡州司戶參軍。度支使第五琦薦倫才,擢商州刺史、荊襄道租庸使。襄州裨將康楚元亂,自稱東楚義王,刺史王政棄城遁。賊南襲江陵,絕漢、沔餉道。倫調兵屯鄧州,厚撫降賊。寇益怠,乃擊禽楚元以獻,收租庸二百萬緡。召為衛尉卿,俄兼寧、隴二州刺史。



 乾元中,襄州亂,詔倫為山南東道節度使,而李輔國方恣橫,倫不肯謁,憾之,中罷為秦州刺史。吐蕃、黨項歲入邊,倫兵寡,數格虜,敗,貶巴州長史,徙務川尉。代宗立,連拜忠、臺、饒三州刺史。宦者呂太一反嶺南,詔拜倫韶州刺史、韶連郴都團練使。為太一反間,貶信州司馬,斥棄十年,客豫章。



 德宗嗣位,選使絕域者,擢倫太常少卿,充和吐蕃使。倫至,諭天子威德,贊普順悅,乃入獻。還,進太常卿,兼御史大夫。再使,如旨。倫處朝,數論政得失,宰相盧杞惡之,改太子少保。從狩奉天。及杞敗,關播罷為刑部尚書,倫在朝堂流涕曰;「宰相無狀,使天下至此,不失為尚書,後何勸?」聞者憚其公。帝後欲復用杞為刺史,倫苦諫,言懇至到,帝納之。進太子少師、郢國公,致仕。時李楚琳以僕射兼衛尉卿,李忠誠以尚書兼少府監,倫言:「楚琳逆節,忠誠戎醜,不當寵以官。」又請為義倉,以捍無年;擇賢者,任帝左右。謂吐蕃豺虎野心,不可事信約,宜謹備邊。帝善其言,厚禮之。居家以孝慈稱。卒,年八十三,贈揚州都督,謚曰肅。



 薛玨,字溫如,河中寶鼎人。以廕為懿德太子廟令,累遷乾陵臺令。歲中以清白聞,課第一,改昭應令,人請立石紀德,玨固讓。遷楚州刺史。初,州有營田,宰相遙領使,而刺史得專達,俸及它給百餘萬,田官數百,歲以優得遷,別戶三千,備刺史廝役。玨至,悉條去之,租入贏異時。觀察使惡其潔,誣以罪,左授峽州刺史。建中初,德宗命使者分諸道察官吏升黜焉,而李承狀玨之簡,趙贊言其廉,盧翰稱其肅,書參聞,於是拜中散大夫,賜金紫。劉玄佐表兼汴宋行軍司馬。李希烈棄汴州走,即拜玨刺史,遷河南尹。入為司農卿。是時,詔舉堪刺史、縣令者且百人,延問人間疾苦、吏得失,取尤通達者什二,宰相欲校以文詞,玨曰:「求良吏不可責文學,宜以上愛人之本為心也。」宰相多其計,所用皆稱職。為京兆尹,司農供三宮畜茹三十車,不足,請市京兆。是時,韋彤為萬年令,玨使彤禁鬻賣,民苦之。德宗怒,奪玨、彤俸。帝疑下情不達,因詔延英坐日許百司長官二員言闕失,謂之巡對。玨剛嚴,曉法治,勤身以勸下,然苛察,無經術大體。坐善竇參,改太子賓客,出為嶺南觀察使。卒,年七十四,贈工部尚書。



 子存慶,字嗣德,貌偉岸。及進士第,歷御史、尚書郎。五遷給事中,與韋弘景封駁詔書,時稱其直。劉總以幽州歸,穆宗謂宰相曰:「必用薛存慶,可以宣朕意。」對延英一刻,遣之,至鎮州,疽發於背卒,贈吏部侍郎。



 崔漢衡,博州博平人。沉懿博厚,善與人交。始為費令,滑州節度使令狐彰表掌書記。大歷六年,以檢校禮部員外郎為和蕃副使。還,遷右司郎中。建中二年,吐蕃請盟,擢殿中少監,為和蕃使,與其使區頰贊俱來約盟。改鴻臚卿,持節送區頰贊歸,遂定盟清水。德宗幸奉天,吐蕃以兵佐渾瑊,敗賊武功。轉秘書監。俄拜上都留守、兵部尚書、東都淄青魏博賑給宣慰使。又使幽州,還命稱指。貞元三年,豫吐蕃盟平涼,被執,虜將殺之,因夷言謂之曰:「我善結贊,無殺我!」而漢衡誠信素著,虜亦尊重,故至河州得還。明年,出為晉慈隰觀察使,卒,贈尚書左僕射。



 戴叔倫,字幼公,潤州金壇人。師事蕭穎士,為門人冠。劉晏管鹽鐵,表主運湖南,至雲安,楊惠琳反,馳客劫之曰:「歸我金幣,可緩死。」叔倫曰:「身可殺,財不可奪。」乃舍之。嗣曹王皋領湖南、江西,表在幕府。皋討李希烈,留叔倫領府事,試守撫州刺史。民歲爭溉灌,為作均水法,俗便利之。耕餉歲廣,獄無系囚。俄即真。期年,詔書褒美,封譙縣男,加金紫服。齊映、劉滋執政,叔倫勸以「屯難未靖,安之者莫先於兵,兵所藉者食,故金穀之司不輕易人。天下州縣有上、中、下,緊、望、雄、輔者,有司銓擬,皆便所私,此非為官擇人、為人求治之術。其尤切者,縣令、錄事參軍事,此二者,宜出中書、門下,無計資序限,遠近高卑,一以殿最升降,則人知勸。」映等重其言。遷容管經略使,綏徠夷落,威名流聞。其治清明仁恕,多方略,故所至稱最。德宗嘗賦《中和節詩》,遣使者寵賜。代還,卒於道,年五十八。



 王翃,字宏肱,並州晉陽人。少治兵家。天寶中,授翃衛尉、羽林軍宿衛。擢才兼文武科,出為辰州刺史。與討襄州康楚元有功,加兼秘書少監,遷朗州刺史。大歷中,擢容管經略使。初,安祿山亂,詔嶺南兵隸南陽魯炅。炅敗績,眾奔潰。溪洞夷獠相挻為亂,夷酋梁崇牽號「平南都統」,與別帥覃問合,又與西原賊張侯、夏永更誘嘯,因陷城邑,遂據容州。前經略使陳仁琇、元結、長孫全緒等皆僑治藤、梧。翃至,言於眾曰:「我,容州刺史,安可客治它所?必得容乃止。」即出私財募士,有功者放署吏,於是人自奮。不數月,斬賊帥歐陽珪。因至廣州,請節度使李勉出兵並力,勉不許,曰:「容陷賊久,獠方強,今速攻,祗自敗耳。」翃曰:「大夫即不出師,願下書州縣,陽言以兵為助,冀藉此聲,成萬一功。」勉許諾。翃乃移書義、藤二州刺史,約皆進討,引兵三千與賊鏖戰,日數遇。勉檄止之,輒匿不發,戰愈力,卒破賊,禽崇牽,悉復容州故地。捷書聞,詔更置順州,以定餘亂。翃凡百餘戰,禽首領七十,覃問遁去。復遣將李寔等分討西原,平鬱林等諸州。累兼御史中丞、招討處置使。會哥舒晃反,翃命寔悉師援廣州,問因合眾乘間來襲,翃設伏擊之,生禽問,嶺表平。代宗遣使慰勞,加金紫光祿大夫,賜第京師。



 時吐蕃入寇,郭子儀悉河中兵乘邊,召翃為河中少尹,領節度後務。悍將凌正數干法不逞,約其徒夜斬關逐翃。翃覺之,陰亂漏刻,以差其期,眾驚,不敢發。俄禽正誅之,一軍惕息。歷汾州刺史,為振武軍使綏,銀等州留後。入拜京兆尹。會起涇原兵討李希烈,次滻水,京兆主供擬,饔敗肉腐,眾怒曰:「食是而討賊乎?」遂叛。翃挺身走奉天,拜太子詹事。德宗還都,再遷大理卿,出為福建觀察使。徙東都留守,既至,開田二十餘屯,脩器械,皆良金壽革,練士卒,號令精明。俄而吳少誠叛,獨東畿為有備,關東賴之。貞元十八年卒,贈尚書右僕射,謚曰肅。



 翃雅善盧杞,杞之殺崔寧、沮李懷光不得朝,皆與其謀,議者以為訾。



 子正雅,字光謙,行謹飭,為崔邠所器。元和初,擢進士,遷累監察御史。穆宗時,京邑多盜賊,正雅以萬年令威震豪強。尹柳公綽言其能,就賜緋魚,擢累汝州刺史。屬監軍怙權,乃謝病去。入為大理卿,會爭宋申錫獄,堅甚,申錫得不死。大和中卒,贈左散騎常侍。



 翃兄翊,性謙柔,歷山南東道節度使。代宗目為純臣,世稱謹廉。卒,贈戶部尚書,謚曰忠惠。



 翊曾孫凝,字成庶,少孤,依其舅宰相鄭肅。舉明經、進士,皆中。歷臺省,浸知名,擢累禮部侍郎。不阿權近,出為商州刺史。驛道所出,吏破產不能給,而州有冶賦羨銀,常摧直以優吏奉。凝不取,則以市馬,故無橫擾,人皆慰悅。徙湖南觀察使。僖宗立,召為兵部侍郎,領鹽鐵轉運使。坐舉非其人,以秘書監分司東都,即拜河南尹。遷宣歙池觀察使,時乾符四年也。王仙芝之黨屠至德,勢益張,凝遣牙將孟琢助池守。賊益兵來攻,實欲襲南陵,凝遣樊儔以舟師扼青陽。儔違令,輕與賊戰,不勝,凝斬以徇,諸將聞,皆股慄,以死綴賊,賊不能進。時江南環境為盜區,凝以強弩據採石,張疑幟,遣別將馬穎解和州之圍。明年,賊大至,都將王涓自永陽赴敵,凝大宴,謂涓曰:「賊席勝而驕,可持重待之,慎毋戰。」涓意銳,日趨四舍,至南陵,未食即陣,死焉。監軍收餘卒數千,還走城,沮撓無去意,卒又恣橫不能禁,凝讓曰:「吏捕蝗者,不勝而仰食於民,則率暴以濟災也。今兵不能捍敵,又恣之犯民生業,何以稱朝廷待將軍意?」監軍詞屈,趣親吏入民舍奪馬,凝乘門望見,麾左右捕取殺之,由是不敢留,然益儲畜繕完以備賊,賊至不能加。會大星直寢庭墜,術家言宜上疾不視事以厭勝,凝曰:「東南,國有所出,而宣為大府,吾規脫禍可矣,顧一方何賴哉?誓與城相存亡,勿復言!」既而賊去。未幾,卒,年五十八,贈吏部尚書,謚曰貞。



 徐申,字維降,京兆人。擢進士第,累遷洪州長史。嗣曹王皋討李希烈,檄申以長史行刺史事,任職辦,皋表其能,遷韶州刺史。韶自兵興四十年,刺史以縣為治署,而令丞雜處民間。申按公田之廢者,募人假牛犁墾發,以所收半畀之,田久不治,故肥美,歲入凡三萬斛。諸工計所庸,受粟有差,乃徙治故州。未幾,邑閈如初。創驛候,作大市,器用皆具。州民詣觀察使,以其有功於人,請為生祠,申固讓,觀察使以狀聞,遷合州刺史。始來韶,戶止七千,比六年,倍而半之。會初置景州,授刺史,賜錢五十萬,加節度副使。遷邕管經略使。黃洞納質供賦,不敢桀。逾年,進嶺南節度使。前使死,吏盜印,署府職百餘員,畏事洩,謀作亂。申覺,殺之,詿誤一不問。遠俗以攻劫相矜,申禁切,無復犯。外蕃歲以珠、玳瑁、香、文犀浮海至,申於常貢外,未嘗賸索,商賈饒盈。劉闢反,表請發卒五千,循馬援故道,繇爨蠻抵蜀,手壽闢不備。詔可,加檢校禮部尚書,封東海郡公。詔未至,卒,年七十。贈太子少保,謚曰平。



 郗士美,字和夫,兗州金鄉人。父純,字高卿,舉進士、拔萃、制策皆高第,張九齡、李邕數稱之。自拾遺七遷至中書舍人。處事不回,為宰相元載所忌。時魚朝恩以牙將李琮署兩街功德使,琮恃勢桀橫,眾辱京兆尹崔昭於禁中,純曰:「此國恥也。」即詣載請速處其罪,載不納,遂辭疾還東都,號「伊川田父」,十年不出。德宗立,崔祐甫輔政,召為太子左庶子、集賢殿學士,不拜,以老乞身。改詹事,聽致仕。帝召見,褒嘆良久,賜金紫,公卿以下咸祖都門,世高其節。



 士美年十二,通《五經》、《史記》、《漢書》,皆能成誦。父友蕭穎士、顏真卿、柳芳與相論繹,嘗曰:「吾曹異日當交二郗之間矣。」未冠為陽翟丞,佐李抱真潞州幕府。以才,歷王虔休、李元,皆留不徙。久乃進房州刺史、黔中經略觀察使。溪州賊向子琪以眾八千岨山剽劫,士美討平之,加檢校右散騎常侍,封高平郡公。遷京兆尹,天子多所咨逮。



 出為鄂岳觀察使。時安黃節度使伊慎入朝,其子宥主後務,偃蹇,母死京師不發喪,欲固其權。士美知之,使府屬過其境,宥出迎,因以母訃告之,即為辦裝,宥惶遽上道。



 改河南尹,檢校工部尚書,充昭義節度使。昭義自李抱真以來皆武臣,私廚月費米六千石、羊千首、酒數十斛,潞人困甚。士美至,悉去之,出稟錢市物自給。又盧從史時,日具三百人膳以餉牙兵,士美曰:「卒衛於牙,固職也,安得廣費為私恩?」亦罷之。討王承宗也,遣大將王獻督萬人為前鋒,獻恣橫逗撓,士美即斬以徇,下令曰:「敢後者斬!」親鼓之,大破賊,下三營環柏鄉。時諸鎮兵合十餘萬繞賊,多玩寇犯法,獨士美兵銳整,最先有功。憲宗喜曰:「固知士美能辦吾事。」承宗大震懼。亡幾,會詔班師,然威震兩河。以疾召拜工部尚書。後檢校刑部尚書,為忠武節度使。卒,年六十四,贈尚書左僕射,謚曰景。生平與人交,已然諾,以是名重於世。



 辛秘,系出隴西。貞元中,擢明經第,授華原主簿。以判入等,調長安尉。其學於禮家尤洽,高郢為太常卿,奏為博士。再遷兵部員外郎,常兼博士。再闢禮儀使府。



 憲宗初,拜湖州刺史。李錡反,遣大將先取支州。蘇、常、杭、睦四刺史,或戰敗,或拘脅,獨秘以儒者,賊易之。未及至,秘召牙將丘知二夜開城收壯士,得數百,逆賊大戰,斬其將,進焚營保。錡平,賜金紫。僉謂秘材任將帥,會河東範希朝出討王承宗,召秘為希朝司馬,主留務。累遷汝、常州刺史,河南尹,進拜昭義軍節度使。是時,承討恆、趙之後,潞人雕耗。秘至,則約出入,嗇用度,比四年,儲錢十七萬緡、糧七十萬斛,器械堅良,隱然復為完鎮。召還,道病卒,年六十四,贈尚書左僕射,謚曰肅,後更謚懿。



 秘為大官,居不易第,服不改初,其奉祿悉與裏表親屬。病,自銘其墓,作書一通緘之。卒後發視,則送終制也,儉而不違於禮云。



\end{pinyinscope}