\article{列傳一百一十四 高趙田硃}

\begin{pinyinscope}

 高仁厚,亡其系出。初事劍南西川節度使陳敬瑄為營使。黃巢陷京師,天子出居成都,敬瑄遣黃頭軍部將李鋋鞏咸以兵萬五千戍興平,數敗巢軍。賊號蜀兵為「鵶兒」修民以孝悌忠信,利國強兵,以求天下統一。孟子生前,這,每戰,輒戒曰:「毋與鵶兒鬥。」敬瑄喜其兵可用,益選卒二千,使仁厚將而東。



 先是,京師有不肖子,皆著疊帶冒,持梃剽閭里,號「閑子」。京兆尹始視事,輒殺尤者以怖其餘。竇潏治京兆,至殺數十百人,稍稍憚戢。巢入京師,人多避難寶雞,閑子掠之,吏不能制。仁厚素知狀,下約入邑閭縱擊。軍入,閑子聚觀嗤侮,於是殺數千人,坊門反閉,欲亡不得,故皆死,自是閭里乃安。



 會邛州賊阡能眾數萬略諸縣,列壁數十,涪州刺史韓秀升等亂峽中,韓求反蜀州,諸將不能定。敬瑄召仁厚還,使督兵四討,屯永安。阡能遣諜者入軍中,吏執以獻,諜自言父母妻子囚於賊,約不得軍虛實且死。仁厚哀之,曰:「為我報賊,明日我且戰,有能釋甲迎我者,署背曰『歸順』,皆得復農矣。」縱諜去,命諸將毀柵,鼓而前。賊渠羅渾擎設伏詐降,仁厚遣將不持兵入諭其眾,皆真降。渾擎詐窮而逸,吏執之,仁厚曰:「愚人不足語。」降眾署背得免,則告諸壁:「大軍至。」賊帥句胡僧大驚,斬之,莫能禁,眾執胡僧以降。韓求知大賊已禽,徇諸壁曰:「敢出者斬!」眾罵之,求赴水死,眾鉤出,斬以徇,餘柵皆下。仁厚按轡徘徊視賊壘,吏請焚之,仁厚命取財糧,乃縱火,尸賊成都。仁厚還,天子御樓勞軍,授仁厚檢校尚書左僕射、眉州刺史。



 敬瑄與仁厚謀曰:「秀升未禽,貢輸梗奪,百官乏奉,民不鹽食。公能破賊,當以東川待公。」仁厚許之。詔拜行軍司馬。仁厚聞賊儲械、子女皆在屯,乃以銳兵瀕江,伐木頹水礙舟道,負岸而陣。使游軍逼賊,久不戰,則夜以千卒持短刀、強弩直薄營,火而噪之。秀升率舟兵救火,仁厚遣人鶩沒鑿舟,皆沈,眾懼,多潰。秀升斬潰兵,欲脅止之,眾怒,執秀升以降。仁厚問狀,對曰:「天子蒙塵,反者何獨我?」仁厚檻車送行在,斬於市。



 東川節度使楊師立初隸神策軍,累遷檢校司空、同中書門下平章事。聞敬瑄仁厚代己,有望言。敬瑄諷帝召師立以本官兼尚書右僕射,師立益怒,移檄言敬瑄十罪,殺監軍田繪,屯涪城,遣兵攻綿州,不克。又檄劍州刺史姚卓文共攻成都,假卓文為指揮應接使,卓文不應。帝乃下詔削官爵。敬瑄即表仁厚為東川節度留後,楊茂言為行軍副使,楊棠為諸軍都虞候,率兵三萬討之。師立遣大將張士安、鄭君雄守鹿頭關。仁厚次漢州,前軍戰德陽,師立嬰城,閱四旬,夜出兵擾北柵,仁厚設兩翼而伏,披柵門列炬,賊不敢進,伏發,擊走之。楊茂言謂仁厚且敗,引兵走,久乃還。明日,會諸將,仁厚曰:「副使當以死報天子。」斬而徇。於是士安不敢出,師立自督士,十戰皆北。仁厚約城中斬首惡者賞,君雄呼於軍曰:「天子所討,反者耳,吾等何與?」乃與士安嘩而進,以仁厚書示師立曰:「請以死謝眾。」自沈於池死。君雄悉誅其家,獻首天子。仁厚入府,縱系囚,賑貧絕。詔拜劍南東川節度使。



 光啟二年,遂據梓州,絕敬瑄。君雄時為遂州刺史,亦陷漢州,攻成都。敬瑄使部將李順之逆戰,君雄死。又發維、茂州羌軍擊仁厚,斬之。乾寧中,皆追贈司徒。



 趙犨,陳州宛丘人,世為忠武軍牙將。犨資警健,兒弄時好為營陣行列,自號令指顧,群兒無敢亂。父叔文見之,曰:「是當大吾門。」稍長,喜書,學擊劍,善射。會昌中,從伐潞州,收天井關,又從征蠻,忠武軍功多,遷大校。



 黃巢入長安,所在盜興,陳人詣節度府,請犨為刺史,表於朝,授之。既視事,會官屬計曰:「巢若不死長安,必東出關,陳其沖也。」乃培城疏塹,實倉庫,峙槁薪,為守計。民有貲者悉內之,繕甲兵,募悍勇,悉補子弟領兵。巢敗,果東奔。賊將孟楷以萬人寇項,犨擊禽之。僖宗嘉其功,遷累檢校司空。巢聞楷死,驚且怒,悉軍據溵水,與秦宗權合兵數十萬,繚長壕五周,百道攻之。州人大恐,犨令曰:「士貴建功立名節,今雖眾寡不敵,男子當死地求生,徒懼無益也。且死國,不愈生為賊乎?吾家食陳祿,誓破賊以保陳,異議者斬!」眾聽命。引銳士出戰,屢破賊。巢益怒,將必屠之,乃起八仙營於州左,僭象宮闕,列百官曹署,儲糧為持久計。宗權輸鎧仗軍須,賊益張。犨小大數百戰,勝負相當,故人心固,乃間道乞師於硃全忠。未幾,汴軍至,壁西北,陳人思奮,犨引兵急擊賊,破之。圍凡三百日而解。



 中和五年,擢彰義軍節度使。巢雖敗,宗權始熾,略地數千里,屠二十餘州,唯陳賴犨獨完,以功檢校司徒,加泰寧、浙西兩節度,皆在陳並領之。龍紀初,進同中書門下平章事、忠武軍節度,仍治陳州,流亡踵還。與弟昶至友愛,後將老,悉以軍事付之,乃卒,贈太尉。



 犨悉忠力以孤城抗賊,巢卒敗亡。然附全忠,亦賴其力復振,故委輸調發助全忠,常先它鎮云。



 昶,字大東,神採軒異,而內沈厚,有法度。破孟楷功多。巢之圍,昶夜手取師,疲而寢,如有神相之者。黎曙決戰,士爭奮死鬥,禽賊酋數人,斬級千餘。犨領泰寧,以昶為州刺史、檢校尚書右僕射。當時,方鎮言忠壯吏治,舉言犨、昶。犨之老,乃授留後,遷忠武節度使,亦留陳。進檢校司徒。劭勸農桑,於人有恩惠。加同中書門下平章事。乾寧二年卒,年五十三,贈太尉。



 犨子珝,字有節。雄毅喜書,善騎射。巢之難,激勵麾下,約皆死。以先塚邇賊,畏見殘齮,即夜縋死士取柩以入。庫有巨弩,機牙壞,不能張,珝以意調治,激矢至五百步,人馬皆洞,賊畏不敢逼。以勞檢校尚書右僕射,遙領處州刺史。



 昶帥忠武,珝遷行軍司馬。昶之喪,知忠武留後,政簡濟,上下安之。全忠表為忠武軍節度使。陳土惡善圮,珝疊甓表墉,遂無患。三加檢校太保。光化三年,同中書門下平章事,進兼侍中,封天水郡公。按鄧艾故跡,決翟王渠溉稻以利農。一家三節度,相繼二十餘年,陳人宜之。



 天復初,韓建帥忠武,以珝知同州節度留後。昭宗還長安,詔入朝,賜號「迎鑾功臣」。以檢校太傅為右金吾衛上將軍,從東遷。歲餘,以疾免。卒,年五十五,贈侍中,陳人為罷市。



 田頵,字德臣,廬州合肥人。略通書傳,沈果有大志。與楊行密同里,約為兄弟。應州募屯邊,遷主將。行密據廬州,頵謀為多。攻趙鍠於宣州,鍠出東溪,乘暴流以逸,阻水解甲,謂追騎不能及。頵乘輕舠追之,鍠驚,遂見禽。行密表頵為馬步軍都虞候。



 沙陀叛將安仁義奔淮南,行密大喜,屬以騎兵,使在頵右,兩人名冠軍中,共攻常州,殺刺史杜棱。錢鏐方屯潤州,一夕潰。會孫儒南略,頵等屯丹陽,儒火揚州,壁廣德,頵破其屯。與戰,頵走,行密怒,奪其兵。或諫行密曰:「強敵傅壘,不用頵,非計也。」行密復將頵。儒詒書仁義通好,以疑行密,行密待益厚,署行軍副使,卒用此二人功禽儒。乃表仁義為潤州刺史,頵寧國軍節度使。累遷檢校太保、同中書門下平章事。仁義至檢校太保。



 頵已平馮弘鐸,至揚州謝行密。左右求貲不已,獄吏亦有請,頵怒曰:「吏覬吾入獄邪!」又求池、歙為屬州,行密不許,頵始怨。將還,指府門曰:「吾不復入此。」



 是時,錢鏐部將徐綰叛,鏐入杭州逐綰,綰屯靈隱山迎頵。



 頵遣客何曉見鏐曰:「王宜東保會稽,無為虛屠士眾也。」鏐曰:「軍中小叛常然,公為人長,何助逆耶?」頵攻北門,鏐登城與語,射中麾下。頵築壘絕往來道,鏐患之,出金幣十輿,募能奪地者。陳璋以死士三百,免胄馳擊,奪其地,鏐授璋衢州刺史。頵攻城未能克,將濟江絕西陵,為鏐將所卻,圍益急。



 先是,行密欲女鏐子,鏐急,乃遣元鏚迎女,且告行密曰:「頵得志,為患必大,請以子為質,願召還頵。」行密使人謂頵曰:「不還,我遣人代守宣州。」頵不從。鏐輸錢二百萬緡犒軍,頵又請鏐子元瓘出質,乃與綰引兵還。然內怨行密與鏐,因移書曰:「侯王守方以奉天子,譬百川不朝於海,雖狂奔澶漫,終為涸土,不若順流無窮也。東南揚為大,刀布金玉積如阜,願公上天子常賦,頵請悉儲峙,單車以從。」行密答曰:「貢賦繇汴而達,適足資敵爾。」於是頵絕行密,大募兵。李神福白行密:「頵必叛,宜先圖之。」行密曰:「頵有大功,而反狀未明,殺之,諸將不為用。」頵遣其佐杜荀鶴至汴通好,全忠喜,屯宿州須變。行密以康儒在頵所,故授廬州刺史以間之。頵怒,族其家,儒曰:「公不用吾謀,死無地矣。」



 頵與安仁義連和攻升州,劫刺史李神福妻息厚養之。神福方與劉存攻鄂州,行密召之。神福謂諸將曰:「頵反,此心腹疾,宜速攻之。」頵遣李皋詒書神福曰:「公家在此,茍從我,當分地以王。」答曰:「吾以一卒從吳王,任上將,終不以妻子易意。」乃斬皋,破頵兵於曷山。始,頵將王壇等以舟師躡神福後,至吉陽磯,不戰。會日暮,壇掩神福軍半濟,神福反舟順流急擊,大破之,因縱火,士多死。明日,壇復戰,敗於皖口,頵乃自將來戰。神福曰:「賊棄城而來,此天亡也。」乃瀕水堅壁不出,請行密以兵塞頵走道。



 仁義焚東塘戰艦,夜攻常州,不克,轉戰至夾岡,立二幟,解甲而息,追兵莫敢向。頵陳舟蕪湖。行密遣將王茂章攻潤州。仁義以善射冠軍中,當時稱硃瑾槊,米志誠弩,皆為第一。仁義常曰:「志誠弩十,不當瑾槊之一;瑾槊十,不當吾弓之一。」人以為然。又其治軍嚴,善得士心。戰卒數百,濠梁不毀,開門斗,先告所當中,然後射之。茂章等不敢與確。行密遣使謂曰:「吾不忘公功,能自歸,當復為行軍副使,但不可處兵。」仁義欲降,其子固諫,乃止。



 行密召其將臺濛泣語曰:「人嘗告頵必反,我不忍負人,頵果負我。吾思為將者非公莫可。」濛頓首謝,率騎度江,為陣以行。士笑其怯,濛曰:「頵宿將多謀,備之何害?」與王壇等戰廣德,濛以行密書遺壇諸將,皆再拜氣奪。濛麾兵擊之,壇走。神福既以不戰困頵,頵紿言母病,還至蕪湖。聞壇敗,留精兵二萬屬郭行琮,身走城。濛之行,為狹營小舍,覘者以為才容二千人,頵輕之,不復召兵。與戰黃池,矢石始交而濛遁,兵爭逐北,遇伏,頵大敗,召蕪湖兵,不得入。行琮及壇皆歸行密,頵恚,自料死士數百,號「爪牙都」,身薄戰。濛退軍示弱,士超隍,濛殊死戰,軍潰。頵奔城,橋陷,為亂兵所殺,年四十六。其下猶鬥,示頵首,乃潰。



 頵始以元瓘歸,戰不勝,輒欲殺之,頵母護免。及鏐與行密合,頵曰:「今日不勝,必殺元瓘。」已而頵死,傳首至淮南,行密泣下,葬以庶人禮,亦葬康儒,還元瓘於杭。



 頵善為治,資寬厚,通利商賈,民愛之。善遇士,若楊夔、康軿、夏侯淑、殷文圭、王希羽等皆為上客。文圭有美名,全忠、鏐交闢不應。頵置田宅,迎其母,以甥事之,故文圭為盡力。夔知頵不足亢行密,著《溺賦》以戒,頵不用。



 行密使王茂章穴地取潤州,安仁義以家屬保城樓,兵不敢登。召李德誠曰:「汝可以委命。」乃抵弓矢就縛,父子斬揚州市。



 濛,字頂雲,亦合淝人。頵破,行密表為檢校太保、宣州觀察使。天祐初卒。



 硃延壽者,廬州舒城人。事行密,破秦彥、畢師鐸、趙鍠、孫儒功居多。行密欲以寬恕結人心,而延壽敢殺。時揚州多盜,捕得者,行密輒賜所盜遣之,戒曰:「勿使延壽知。」已而陰許延壽殺之。



 初,壽州刺史高彥溫舉州入硃全忠,行密襲之,諸將憚城堅不可拔,延壽鼓之,拔其城即表為淮南節度副使。全忠猶屯壽春,延壽以新軍出,每旗五伍為列,遣李厚以十旗擊西偏,不勝,將斬之,厚請益五旗,殊死戰,全忠引去。於是取黃、蘄、光三州,以功遷壽州團練使。



 昭宗在鳳翔,詔延壽圍蔡以披全忠勢,擢奉國軍節度使。全忠兵每至,延壽開門不設備,而不敢逼也。延壽用軍常以寡鬥眾,敗還者盡斬之。



 田頵之附全忠,延壽陰約曰:「公有所為,我願執鞭。」頵喜,二人謀絕行密。行密憂甚,紿病目,行觸柱殭。妻,延壽姊也,掖之。行密泣曰:「吾喪明,諸子幼,得舅代我,無憂矣。」遣辯士召之,延壽疑,不肯赴。姊遣婢報故,延壽疾走揚州,拜未訖,士禽殺之,而廢其妻。



 贊曰:全忠,唐之盜也,行密志梟其元而後已。田頵使出軍賦而助之,此其謀責難而絕之,非忠於唐也。棄所附而覬尊大,亦已妄矣。孔子稱孟公綽為趙、魏老則優,不可以為滕、薛大夫。如仁厚、田、硃,材不足為吳、蜀之老,可與事天子哉!



\end{pinyinscope}