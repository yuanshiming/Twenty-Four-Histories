\article{列傳第一百三十八 籓鎮淄青橫海}

\begin{pinyinscope}

 李正己,高麗人。為營州副將,從侯希逸入青州,希逸母即其姑,故薦為折沖都尉。寶應中天的純形式。知性的作用在於用先天的純範疇,把感性直觀,以軍候從討史朝義。時回紇恃功橫,諸軍莫敢抗。正己欲以氣折之,與大酋角逐,眾土皆墻立觀,約曰:「後者批之。」既逐而先,正己批其頰,回紇矢液流離,眾軍哄然笑。酋大慚,自是沮憚不敢暴。希逸以為兵馬使,沈毅得眾心,然陰忌之,因事解其職。軍中皆言不當廢,尋逐希逸出之,有詔代為節度使。本名懷玉,至是賜今名,遂有淄、青、齊、海、登、萊、沂、密、德、棣十州,與田承嗣、薛嵩、李寶臣、梁崇義輔牙相倚。嵩死,李靈耀反,諸道攻之,共披其地。正己復取曹、濮、徐、兗、鄆,凡十有五州。市渤海名馬,歲不絕,賦繇均約,號最強大。政令嚴酷,在所不敢偶語,威震鄰境。歷檢校司空,加同中書門下平章事,以司徒兼太子太保,封饒陽郡王。請附屬籍,許之。因徙治鄆,以子納及腹心將守諸州。



 建中初,聞城汴州,乃約田悅、梁崇義、李惟岳偕叛。自屯濟陰,陳兵按習,益師徐州以扼江、淮。天子於是改運道,檄天下兵為守備,河南騷然。會發疽死,年四十九。興元初,納順命,詔贈太尉。



 納,少時為奉禮郎,將兵防秋。代宗召見,擢殿中丞,賜金紫。入朝,擢兼侍御史。正己署為淄、青二州刺史,又為行軍司馬,濮、徐、兗、沂、海留後,進御史大夫。



 正己死,秘喪不發,以兵會田悅於濮陽。馬燧方擊悅,納使大將衛俊救之,為燧所破略盡,收洹水。德宗詔諸軍合討,其從父洧以徐州歸,大將李士真以德州、李長卿以棣州送款,納恚洧背己,且徐險集,悉兵攻洧。帝命宣武、劉玄佐督諸軍進援,大破其兵。納還濮陽,玄佐進圍之,殘其郛。納登陴見玄佐,泣且悔,遣判官房說與子弟質京師,因玄佐謝罪。時中人宋鳳朝以納窮,欲立功,建不可赦,帝乃械說等禁中。納於是還鄆,與悅、李希烈、硃滔、王武俊連和,自稱齊王,置百官。



 興元初,帝下詔罪己,納復歸命,授檢校工部尚書,復平盧帥節,賜鐵券,又同中書門下平章事,封隴西郡王。希烈圍陳州,納會諸軍破之城下,加檢校司空,實封五百戶,進檢校司徒。死年三十四,贈太傅。子師古、師道。



 師古,以廕累署青州刺史。納死,軍中請嗣帥,詔起為右金吾衛大將軍、本軍節度使。初,棣州有蛤朵鹽池,歲產鹽數十萬斛。李長卿以州入硃滔,獨蛤朵為納所據以專利。後德、棣入王武俊,納乃築壘德州南,跨河以守蛤朵,謂之三汊,通魏博以交田緒,盜掠德州,武俊患之。師古殆襲,武俊易其弱,且納時將無在,乃率兵取蛤朵、三汊。師古使趙鎬拒戰,武俊子士清兵先濟滴河,會營中火起,士大噪不敢前。德宗遣使者諭武俊罷兵。師古亦隳三汊聽命。



 嘗怒其僚獨孤造,使奏事京師,遣大將王濟縊殺之。貞元末,與杜佑、李欒皆得封妾媵以國為夫人,進同中書門下平章事。



 德宗崩,哀使未至,義成節度使李元素騰遺詔示之。師古幸國喪,欲攻掠州縣,即集將士告:「元素偽作遺詔,豈欲反邪?不可不討!」執使者,名討元素,勒兵出次,聞順宗立,乃罷。累加檢校司徒、兼侍中。元和初卒,贈太傅。



 師道,異母弟也。師古嘗曰:「是不更民間疾苦,要令知衣食所從。」乃署知密州。師古病,召親近高沐、李公度等曰:「即我不諱,欲以誰嗣?」二人未對。師古曰:「豈以人情屬師道邪?彼不服戎,以技自尚,慮覆吾宗,公等審計之。」及死,沐、公度與家奴卒立之,而請於朝。於是制書久不下,師道謀裒兵守境,沐爭止,更上書奉兩稅,守鹽法,請吏朝廷。宰相杜黃裳欲橈削其權,而憲宗方誅劉闢,未皇東討,故命建王審領節度大使,而以師道知留後。歲中,加檢校工部尚書,為副大使。自正己以來,雖外奉王命,而嘯引亡叛,有得罪於朝者厚納之。以嚴法持下,凡所付遣,必質其妻子;有謀順者,類夷其家。以故能脅污士眾,傳三世云。



 帝討蔡,詔興諸道兵而不及鄆,師道選卒二千抵壽春,陽言為王師助,實欲援蔡也。亡命少年為師道計曰:「河陰者,江、淮委輸,河南,帝都,請燒河陰敖庫,募洛壯士劫宮闕,即朝廷救腹心疾,此解蔡一奇也。」師道乃遣客燒河陰漕院錢三十萬緡,米數萬斛,倉百餘區。又有說師道曰:「上雖志討蔡,謀皆出宰相,而武元衡得君,願為袁盎事,後宰相必懼,請罷兵,是不用師,蔡圍解矣。」乃使人殺元衡,傷裴度。



 初,師道置邸東都,多買田伊闕、陸渾間,以舍山棚,遣將訾嘉珍、門察部分之,嵩山浮屠圓靜為之謀。元和十年,大饗士邸中,椎牛釃酒,既衷甲矣,其徒白官發之。留守呂元膺以兵掩邸,賊突出,轉略畿部,入山中數月,奪山棚所市,山棚怒,道官軍襲擊,盡殺之。圓靜者,年八十餘,嘗為史思明將,驍悍絕倫。既執,力士椎其脛,不能折,罵曰:「豎子,折人腳且不能,乃曰健兒!」因自置其足折之。且死,嘆曰:「敗吾事,不得見洛城流血!」於時,留守、防御將、都亭驛史數十人,皆陰受師道署職,使為言冋察,故無知者。及窮治,嘉珍、察乃害武元衡者。鹽鐵使王播又得嘉珍所藏弓材五千,並斷建陵戟四十七。



 始,師道欲知元濟虛實,使劉晏平間道走淮西。元濟日與宴,厚結歡。晏平歸,以為元濟暴師數萬,而晏然居內,與妻妾戲博,必敗之道也。師道本倚蔡為重,聞之怒,乃以它事殺晏平。及聞李光顏拔凌雲柵,始大懼,遣使歸順,帝重分兵支兩寇,故命給事中柳公綽慰撫之,加檢校司空。



 蔡平,又遣比部員外郎張宿諷令割地質子。宿謂曰:「公今歸國為宗姓,以尊卑論之,上叔父矣,不屈一也;以十二州事三百餘州天子,北面稱籓,不屈二也;以五十年傳爵,臣二百年天子,不屈三也。今反狀己暴,上猶許內省,宜遣子入宿衛,割地以贖罪。」師道乃納三州,遣子弘方入侍。宿既還,師道中悔,召諸將議,皆曰:「蔡數州,戰三四年乃克,公今十二州,何所虞?」大將崔承度獨進曰:「公初不示諸將腹心,而今委以兵,此皆嗜利者,朝廷以一漿十餅誘之去矣。」師道恚,遣承度詣京師,戒候吏時其還斬之。承度待命客省,不敢還。帝以其負約,用左散騎常侍李遜喻旨。既至,師道嚴兵以見,遜讓曰:「前已約,而今背之,何也?願得要言奏天子。」師道許之,然懦暗不自決。私奴婢媼爭言:「先司徒土地,奈何一旦割之?今不獻三州,不過戰耳,即不勝,割地未晚。」師道乃上書,以軍不協為解。帝怒,下詔削其官,詔諸軍進討。武寧節度使李願使將王智興破其眾,斬二千級,獲馬牛四千,略地至平陰。橫海節度使鄭權戰福城,斬五百級。武寧將李祐戰魚臺,敗之。宣武節度使韓弘拔考城。淮南節度使李夷簡命李聽趨海州,下沭湯、朐山,進戍東海。魏博節度使田弘正身將兵自陽劉濟河,拒鄆四十里而營,再接戰,破三萬眾,禽三千人。陳許節度使李光顏攻濮陽,收斗門、杜莊二屯。弘正又戰東阿,殘其眾五萬。師道每聞敗,輒悸成疾,及李祐取金鄉,左右莫敢白。



 初,遣大將劉悟屯陽谷,當魏博軍,疑其逗留,悟懼不免,引兵反攻城。師道晨起聞之,白其嫂裴曰:「悟兵反,將求為民,守墳墓。」即與弘方匿溷間,兵就禽之。師道請見悟,不許,復請送京師,悟使謂曰:「司空今為囚,何面目見天子!」猶俯仰祈哀,弘方曰:「不若速死!」乃並斬之,傳首京師。棄其尸,無敢收視者,有士英秀為殯城左。馬皛至,以士禮更葬。



 初,師古見劉悟,曰:「後必貴,然敗吾家者此人也。」田弘正之度河也,禽其將夏侯澄等四十七人,有詔悉赦之,給繒絮,還隸魏博、義成軍,父母在欲還者優遣,賊皆感慰相告,由是悟得行其謀。師道首傳弘正營,召澄驗之,澄舐目中塵,號絕良久。悟素與師道妻魏亂,妄言鄭公征之裔,不死,沒入掖廷,它宗屬悉遠徙。悟獨表師古子明安為朗州司戶參軍。親將王承慶,承宗弟也,師道以兄女妻之,潛約左右,欲因肄兵執師道,會悟入,出奔徐州,歸朝。



 程日華,定州安喜人,始名華,德宗以其有功,益曰日華。父元皓為安祿山帳下,偽署定州刺史,故日華籍本軍,為張孝忠牙將。滄,故成德部州也,孝忠絕李惟岳,德宗以滄畀義武。前刺史李固烈與惟岳姻屬,即牢守。孝忠令日華往喻之,固烈請還恆州。既治裝,悉帑以行,軍中怒曰:「馬瘠,士饑死,刺史不棄豪發血阜吾急,今刮地以去,吾等何望?」遂共殺固烈,屠其家。日華驚匿床下,將士迎出之曰:「暴吾軍者已死,何畏而亡?」共逼領州。孝忠亦以日華寬厚,遂假以刺史。



 硃滔叛,兵屯河間,以故滄、定道阻不相聞。滔及王武俊皆招日華,不納,即攻之。日華乘城自固。參軍事李宇謀曰:「城久圍,府兵不為援。今州十縣瀕海,有魚鹽利自給,此軍本號橫海,將軍能絕易定歸天子,自為一州,蜺甲訓兵,利則出,無利則守,可亢盜喉襟。君能用僕計,請至京師為天子言之。」日華謂然,乃遣宇西,帝果大喜,拜御史中丞、滄州刺史,復置橫海軍,即以為使,時建中三年也。拜檢校工部尚書。詔滄歲饋義武錢十二萬緡,糧數萬斛,以宇為判官。



 武俊欲得滄,遣人說日華歸己,日華紿曰:「敝邑為賊攻,力屈則下之。願假騎二百以抗賊,賊退,請以地授公。」武俊喜,歸之馬,日華留馬謝其使。武俊大怒,與滔方睦,懼有怨,乃止。久之,武俊歸命,日華乃還馬,以珍幣厚謝,復結好,武俊亦釋然。貞元二年卒,贈兵部尚書。



 子懷直擅知留事,帝以日華故,即拜權知滄州刺史。宇入朝,願析東光、景城二縣置景州,且請刺史。河朔刺史不廷授幾三十年,帝嘉其忠,以徐申為景州刺史。升橫海軍為節度,擢懷直為留後。明年,為節度使。九年來朝,寵遇加等,進檢校尚書右僕射,賜大第、宮女。



 懷直荒田獵,出輒數日不返,帳下程懷信乘眾怒,閉門不納。懷信,其從昆也。於是懷直入朝,帝不之罪,更以虔王為節度使,擢懷信留後,以懷直兼右龍武軍統軍。明年,懷信為節度矣。十六年,懷直卒,贈揚州大都督。後五年,懷信死,子權襲領軍務,詔授留後。元和元年,拜節度使,累進檢校兵部尚書,封邢國公。六年入朝,憲宗寵禮遣還鎮,加檢校尚書右僕射。權始名執恭,嘗夢滄諸門悉署「權」字,乃改名以應之。及淮西平,惕不安,丐入朝。至京師,固辭軍政,乃詔華州刺史鄭權代之。後以檢校司空為邠寧節度使。卒,贈司徒,宗族奉朝請宿衛者三十餘人。



 李全略,李王氏,名日簡,事王武俊為偏裨。承宗時,虐用其軍,故入朝,授代州刺史。田弘正遇害,穆宗以全略故鎮州將,召問所欲言,全略多陳利害,冀合帝意,且請盡死力以報,遂授德州刺史。是時,杜叔良兵敗博野,故以全略為橫海軍節度、滄德棣州觀察使,賜今姓名。未幾,貢錢千萬,使子同捷入朝。既還,即奏同捷為滄州長史,押中軍兵馬。帝不得已,可其請。全略陰規傳久計,選材武,以所私結士心。棣州刺史王稷善撫眾,而家富於財,全略內忌,以計殺之,族其家。未幾死,同捷領留後事,重賂鄰籓,求領父節,敬宗持久詔不下。俄而文宗立,同捷以帝新嗣位,必大開貸示四方,乃遣弟同志、同巽入朝,而使其屬崔長奉表請命,有詔拜兗海節度使,以烏重胤代之。同捷計窮,矯言軍中留己。於是,王智興請以全軍出討,魏博史憲誠令大將傳手詔入於軍,同捷不受,德、棣民多奔入鄆。乃下詔削官爵,命重胤率鄆、齊兵進討。憲誠、智興及汴滑李聽、平盧康志睦、易定張璠、幽州李載義以兵傅境。同捷自以與成德有舊,乃傾玉帛子女市河北三鎮驩。載義不許,絕其交,執使者並所遣奴婢四十七獻諸朝。王廷水奏本窺橫海,欲乘其隙取之,引軍來援。智興攻棣州,火譙門,引水灌城,凡七月,其將張叔連降。始,刺史欒濛以同捷叛,密上變,事洩,為所害,贈工部尚書。智興進圍滄州。



 是時,帝絕王廷水奏朝貢,且討之,兵須伙繁,調發不時,始置供軍糧料使,以濟兩河,諸將又多張俘首以冒賞。自重胤卒後,李寰、傳良弼不終事,更以左金吾衛大將軍李祐代,而智興將李君謀以輕兵絕河,夜殘無棣,降饒安壁五千兵。明年,祐拔無棣、平原。有詔行營堅壁務農,非被襲,勿決戰。而祐兵已薄德州,帝遣諫議大夫柏耆宣慰。祐攻拔德州,餘卒奔廷水奏。同捷益急,乞降,祐疑其詐。耆引兵直入城,取同捷及家屬馳西。祐入滄州,耆至將陵,斬同捷,使其下傳首京師。詔貸四州一年租賦,赦同捷母並妻息,徙湖南。流崔長商州。同巽等以異母貸死,得隨母流所云。



\end{pinyinscope}