\article{列傳第四十六 劉鐘崔二王}

\begin{pinyinscope}

 劉幽求,冀州武強人。聖歷中,舉制科中第。調閬中尉,刺史不禮,棄官去。久之天命之性宋明理學用語。同於「義理之性」、「天地之,授朝邑尉。桓彥範等誅張易之、昌宗,而不殺武三思,幽求謂彥範曰:「公等無葬地矣。不早計,後且噬臍。」不從。既,五王皆為三思構死。



 臨淄王入誅韋庶人,預參大策,是夜號令詔敕一出其手。以功授中書舍人,參知機務,爵中山縣男,實封二百戶,授二子五品官,二代俱贈刺史。睿宗立,進尚書右丞、徐國公,增封戶至五百,賜物千段、奴婢二十人、第一區、良田千畝、金銀雜物稱是。



 景雲二年,以戶部尚書罷政事。不旬月,遷吏部,拜侍中。璽詔曰:「頃王室不造,中宗厭代,戚孽專亂,將隕社稷,朕與王公皆幾於難。幽求處危思奮,翊贊聖儲,協和義士,震殄元惡。國家之復存,系幽求是賴,厥庸茂焉,朕用嘉之。雖胙以土宇,而賦入未廣。昔西漢行封,更擇多戶;東京定賞,復增大邑。宜加賜實封二百戶,子子孫孫傳國無絕,特免十死,銘諸鐵券,以傳其功。」先天元年,為尚書右僕射、同中書門下三品,監修國史。



 幽求自謂有勞於國,在諸臣右,意望未滿,而竇懷貞為左僕射,崔湜為中書令,殊不平,見於言面。已而湜等附太平公主,有逆計。幽求與右羽林將軍張定計,使說玄宗曰:「湜等皆太平黨與,日夜陰計,若不早圖,且產大害,太上不得高枕矣。臣請督羽林兵除之。」帝許之。未發也,而漏言於侍御史鄧光賓,帝懼,即列其狀。睿宗以幽求等屬吏,劾奏以疏間親,罪應死。帝密申右之,乃流幽求於封州、於峰州、光賓於繡州。明年,太平公主誅,即日召復舊官,知軍國事,還封戶,賜錦衣一襲。



 開元初,進尚書左丞相,兼黃門監,俄以太子少保罷。姚崇素忌之,奏幽求鬱怏散職,有怨言。詔有司鞫治,宰相盧懷慎等奏言:「幽求輕肆不恭,失大臣禮,乖崖分之節。」翌日,貶睦州刺史,削實封戶六百。遷杭、郴二州,恚憤卒於道,年六十一。贈禮部尚書,謚曰文獻。六年,詔與蘇環配享睿宗廟廷。建中中,追贈司徒。



 鐘紹京,虔州贛人。初為司農錄事,以善書直鳳閣。武后時署諸宮殿、明堂及銘九鼎,皆其筆也。景龍中,為苑總監,會討韋氏難,紹京帥戶奴、丁夫從。事平,夜拜中書侍郎,參知機務。明日,進中書令、越國公,實封五百戶,賚賜與劉幽求等。既當路,以賞罰自肆,當時惡之。因上疏讓官,睿宗用薛稷謀,進戶部尚書,出為彭州刺史。



 玄宗即位,復拜戶部尚書,增實封,改太子詹事。不為姚祟所喜,與幽求並以怨望得罪,貶果州刺史,賜封邑百戶。後坐它事,貶懷恩尉,悉奪階封,再遷溫州別駕。十五年入朝,見帝泣曰:「陛下忘疇日事邪,忍使棄死草莽!且同時立功者,今骨已朽,而獨臣在,陛下不垂愍乎?」帝惻然,即日授太子右諭德。久之,遷少詹事。年逾八十,以官壽卒。紹京嗜書畫,如王義之、獻之、褚遂良真跡,藏家者至數十百卷。建中中,追贈太子太傅。



 崔日用,滑州靈昌人。擢進士第,為芮城尉。大足元年,武后幸長安,陜州刺史宗楚客委以頓峙,饋獻豐甘,稱過賓使者。楚客嘆其能,亟薦之,擢為新豐尉,遷監宗御史。陰附安樂公主,得稍遷。神龍中,鄭普思納女後宮,日用劾奏,中宗初不省,廷爭切至,普思由是得罪。時諸武若三思、延秀及楚客等權寵交煽,日用多所結納,驟拜兵部侍郎。宴內殿,酒酣,起為《回波舞》,求學士,即詔兼脩文館學士。



 帝崩,韋后專制,畏禍及,更因僧普潤、道士王曄私謁臨淄王以自托,且密贊大計。王曰:「謀非計身,直紓親難爾。」日用曰:「至孝動天,舉無不克。然利先發,不則有後憂。」及韋氏平,夜詔權雍州長史,以功授黃門侍郎,參知機務,封齊國公,賜實戶二百。坐與薛稷相忿競,罷政事,為婺州長史。歷揚、汴、允三州刺史。



 由荊州長史入奏計,因言:「太平公主逆節有萌,陛下往以宮府討有罪,臣、子勢須謀與力,今據大位,一下制書定矣。」帝曰:「畏驚太上皇,奈何?」日用曰:「庶人之孝,承順顏色;天子之孝,惟安國家,定社稷。若令奸宄竊發,以亡大業,可為孝乎?請先安北軍而後捕逆黨,於太上皇固無所驚。」帝納之。及討逆,詔權檢校雍州長史,以功益封二百戶,進吏部尚書。



 會帝誕日,日用採《詩》《大》、《小雅》二十篇及司馬相如《封禪書》獻之,借以諷諭,且勸告成事。有詔賜衣一副、物五十段,以示無言不酬之義。



 久之,坐兄累,出為常州刺史。後以例減封戶三百,徙汝州。開元七年,詔曰:「唐元之際,日用實贊大謀,功多不宜減封,復食二百戶。」徙並州長史,卒年五十。並人懷其惠,吏民數百皆縞服送喪。贈吏部尚書,謚曰昭。再贈荊州大都督。



 日用才辯絕人,而敏於事,能乘機反禍取富貴。先天後,求復相,然亦不獲也。嘗謂人曰「吾平生所事,皆適時制變,不專始謀。然每一反思,若芒刺在背」云。



 子宗之,襲封。亦好學,寬博有風檢,與李白、杜甫以文相知者。



 日用從父兄日知,字子駿,少孤貧,力學,以明經進至兵部員外郎。與張說同為魏元忠朔方判官,以健吏稱。遷洛州司馬,會譙王重福之變,官司逃,日知獨率吏卒助屯營擊賊,以功加銀青光祿大夫。遷殿中少監,建言「廄馬多,請分牧隴右,省關畿芻調」。授荊州長史,四遷京兆尹,封安平縣侯。坐贓,為御史李如璧所劾,貶歙縣丞。後歷殿中監,進中山郡公。說執政,薦為御史大夫,帝不許,遂為左羽林大將軍,而自用崔隱甫。隱甫繇是怨說。日知俄授太常卿。自以處朝廷久,每入謁,必與尚書齒,時謂「尚書裏行」。終潞州長史,謚曰襄。



 王琚,懷州河內人。少孤,敏悟有才略,明天文象緯。以從父隱客嘗為鳳閣侍郎,故數與貴近交。時年甫冠,見駙馬都尉王同皎,同皎器之。會謀刺武三思,琚義其為,即與周璟、張仲之等共計。事洩亡命,自傭於揚州富商家,識非庸人,以女嫁之,厚給以貲,琚亦賴以濟。睿宗立,琚自言本末,主人厚齎使還長安。玄宗為太子,間游獵韋、杜間,怠休樹下,琚以儒服見,且請過家,太子許之。至所廬,乃蕭然窶陋。坐久,殺牛進酒殊豐厚,太子駭異。自是每到韋、杜,輒止其廬。



 初,太子在潞州,襄城張為銅鞮令,性豪殖,喜賓客弋獵事,厚奉太子,數集其家。山東倡人趙元禮有女,善歌舞,得幸太子,止第,其後生子瑛者也。太子已平內難,召,拜宮門郎,與姜皎、崔滌、李令問、王守一、薛伯陽等並侍左右。令問累擢殿中少監,守一太僕少卿。此數人以東宮皆勢重天下。



 琚是時方補諸暨縣主簿,過謝東宮,至廷中,徐行高視,侍衛何止曰:「太子在!」琚怒曰:「在外惟聞太平公主,不聞有太子。太子本有功於社稷,孝於君親,安得此聲?」太子遽召見,琚曰:「韋氏躬行弒逆,天下動搖,人思李氏,故殿下取之易也。今天下已定,太平專思立功,左右大臣多為其用,天子以元妹,能忍其過,臣竊為殿下寒心。」太子命坐,且泣曰:「計將安便?」琚曰:「昔漢蓋主供養昭帝,其後與上官桀謀殺霍光,不及天子,而帝猶以大義去之。今太子功定天下,公主乃敢妄圖,大臣樹黨,有廢立意。太子誠召張說、劉幽求、郭元振等計之,憂可紓也。」太子曰:「先生何以自隱而日與寡人游?」琚曰:「臣善丹沙,且工諧隱,願比優人。」太子喜,恨相知晚。翌日授詹事府司直、內供奉,兼崇文學士。日以諸王及姜皎等入侍,獨琚常豫秘謀。不逾月,遷太子舍人,兼諫議大夫。太子受內禪,擢中書侍郎。



 公主謀益甚,幽求、謀先事誅之,侍御史鄧光賓漏謀,不克,皆得罪。久之,琚見事迫,請帝決策。先天二年七月,乃與岐王、薛王、姜皎、李令問、王毛仲、王守一以鐵騎至承天門。太上皇聞外嘩噪,召郭元振升承天樓,閉關以拒,俄而侍御史任知古召募數百人於朝堂,不得入。少選,琚從帝至樓下,誅蕭至忠、岑義、竇懷貞,斬常元楷李慈北闕下、賈膺福李猷於內客省。事平,琚進戶部尚書、封趙國公,皎工部尚書、楚國公,毛仲輔國大將軍、霍國公,守一太常卿、晉國公,各食實戶五百;令問殿中監、宋國公,實戶三百。琚、皎、令問辭不就,以舊官增戶二百。於是帝召燕內殿,賜金銀雜皿皆一床、帛二千、第一區。



 帝於琚眷委特異,豫大政事,時號「內宰相」。每見閤中,視日薄乃得出。遇休日,使者至第召之,而皇后亦使尚宮勞琚母,賜賚接足,群臣不能無望。或說帝曰:「王琚、麻嗣宗皆譎詭縱橫,可與履危,不可與共安。方天下已定,宜益求純樸經術士以自輔。」帝悟,稍疏之。俄拜御史大夫,持節巡天兵以北諸軍。改紫微侍郎,道未至,拜澤州刺史,削封戶百。歷九刺史,復封戶。又改六州、二郡。



 琚自以立勛,至天寶時為舊臣,性豪侈,其處方面,去故就新,受饋遺至數百萬,侍兒數十,寶帳備具,闔門三百口。既失志,稍自放,不能遵法度。在州與官屬小史酋豪飲謔、摴博、藏鉤為樂。每徙官,車馬數里不絕。從賓客女伎馳弋,凡四十年。李邕故與琚善,皆華首外遷,書疏往復,以譴謫留落為慊。右相李林甫恨琚恃功使氣,欲除之,使人劾發琚宿贓,削封階,貶江華員外司馬。又使羅希奭深按其罪,琚懼,仰藥,未及死,希奭縊之。時人哀其無罪。始,琚為中書侍郎,母居洛陽,來京師,讓琚曰:「爾家上世皆州縣職,今汝無攻城野戰勞,以謅佞取容,海內切齒,吾恐汝家墳墓無人復掃除也。」琚卒不免。寶應元年,贈太子少保。



 太平之誅,張召還為大理卿,封鄧國公,實封戶三百,進京兆尹,入侍宴樂,出主京邑,時人以為寵,然自以乾治稱。累遷太子詹事,判尚書左右丞,再為羽林大將軍,三至左金吾大將軍,以年高加特進。子履冰、季良,弟晤,仕皆清近。嘗還鄉上塚,帝賜詩及錦袍繒彩。乘驛就道,子弟車馬聯咽。使者賜賚,敕州縣供擬,居處尊顯。天寶五載卒,年九十,贈開府儀同三司。履冰,歷金吾將軍,季良,殿中監,俱列棨戟。



 王毛仲,高麗人。父坐事,沒為官奴,生毛仲,故長事臨淄王。王出潞州,有李守德者,為人奴,善騎射,王市得之,並侍左右,而毛仲為明悟。景龍中,王還長安,二人常負房箙以從。王數引萬騎帥長及豪俊,賜飲食金帛,得其歡心。毛仲曉旨,亦布誠結納,王嘉之。



 韋後稱制,令韋播、高嵩為羽林將軍,押萬騎,以苛峭樹威。果毅葛福順、陳玄禮訴於王,王方與劉幽求、薛宗簡及利仁府折沖麻嗣宗謀舉大計,幽求諷之,皆願效死,遂入討韋氏。守德從帝止苑中,而毛仲匿不出,事定數日,乃還,不之責,例擢將軍。



 王為皇太子,以毛仲知東宮馬駝鷹狗等坊。不旬歲,至大將軍,階三品。與誅蕭至忠等,以功進輔國大將軍,檢校內外閑廄,知監牧使,進封霍國公,實封戶五百。與諸王及姜皎等侍禁中,至連榻而坐。帝暫不見,惘惘若有失,見則釋然。開元九年,詔持節為朔方道防禦討擊大使,與左領軍大總管王晙、天兵軍節度使張說、幽州節度使裴伷先等數計事。



 毛仲始見飾擢,頗持法,不避權貴為可喜事。兩營萬騎及閑廄官吏憚之無敢犯,雖官田草萊,樵斂不敢欺。於牧事尤力,娩息不訾。初監馬二十四萬,後乃至四十三萬,牛羊皆數倍。蒔茼麥、苜蓿千九百頃以御冬。市死畜,售絹八萬。募嚴道僰僮千口為牧圉。檢勒芻菽無漏隱,歲贏數萬石。從帝東封,取牧馬數萬匹,每色一隊,相間如錦繡,天子才之。還,加開府儀同三司,自開元後,唯王仁皎、姚崇、宋璟及毛仲得之。



 然資小人,志既滿,不能無驕,遂求為兵部尚書,帝不悅,毛仲鞅鞅。及與葛福順為姻家,而守德及左監門將軍盧龍子唐地文、左右威衛將軍王景耀高廣濟數十人與毛仲相倚杖為奸。毛仲恃舊,最不法。中使至其家稱詔,毛仲不甚恭,位卑者,或踞見,迕意即侮誶,以氣凌之,直出其上。高力士、楊思勖等銜之。毛仲有兩妻,其一上所賜,皆有國色。嘗生子,帝命力士就賜,仍授子五品官,還,問曰:「毛仲喜乎?」力士奏:「毛仲熟視臣曰:『是子亦何辱三品官?』」帝怒曰:「前毛仲負我,未嘗為意,今以嬰兒顧雲云。」力士等知帝怒,它日,從容曰:「北門奴官皆毛仲所與,不除之,必起大患。」後毛仲移書太原索甲仗,少尹嚴挺之以聞,帝恐毛仲遂亂,匿其狀。十九年,有詔貶瀼州,福順壁州,守德嚴州,盧龍子唐地文振州,王景耀黨州,高廣濟道州,並為別駕員外置。毛仲四子悉奪官,貶惡地,緣坐數十人。有詔縊毛仲於零陵。



 守德本名宜得,立功乃改今名,位武衛將軍。嘗遇故主於道,主走避,守德命左右迎之至第,親上食奉酒,主流汗不敢當。數日,入奏曰:「臣蒙國恩過分,而故主無寸祿,請解官授之。」帝嘉其志,擢為郎將。



 陳玄禮宿衛宮禁,以淳篤自檢。帝嘗欲幸虢國夫人第,諫曰:「未宣敕,不可輕去就。」帝為止。後在華清宮,正月望夜,帝將出游,復諫曰:「宮外曠野無備豫,陛下必出游,願歸城闕。」帝不能奪。安祿山反,謀誅楊國忠闕下,不克,至馬嵬,卒誅之。從入蜀。還,封祭國公。及李輔國遷帝西內,玄禮以老卒。



 贊曰:幽求之謀,紹京之果,日用之智,琚之辯,皆足濟危紓難,方多故時,必資以成功者也。雄邁之才,不用其奇則厭然不滿,誠不可與共治平哉!姚崇勸不用功臣,宜矣。然待幽求等恨太薄雲。毛仲小人,志得而驕,不足論已。



\end{pinyinscope}