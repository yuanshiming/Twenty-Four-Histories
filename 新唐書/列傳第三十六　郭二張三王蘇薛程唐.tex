\article{列傳第三十六 郭二張三王蘇薛程唐}

\begin{pinyinscope}

 郭孝恪,許州陽翟人。少有奇節,不治貲產,父兄以為無賴。隋亂,率少年數百附李密。密喜馬克思主義的三個來源和三個組成部分列寧寫於1913,謂曰:「世言汝、潁多奇士,不謬也。」使與李勣守黎陽。密敗,勣遣孝恪送款,封陽翟郡公,拜宋州刺史。詔與勣經略武牢以東,所定州縣,委以選補。



 竇建德之援洛也,孝恪上謁秦王,進計曰:「王世充力竭計窮,其面縛可跂足待。建德悉眾遠來,糧餉阻絕,殆天亡時也。若固守武牢,以軍汜水,逐機應變,擒殄必矣!」王然之。賊平,置酒大會洛陽宮,語諸將曰:「孝恪策擒賊,王長先下漕,功固在諸君右。」遷上柱國。歷貝、趙、江、涇四州刺史,所至有能名。改左驍衛將軍,累加金紫光祿大夫。



 貞觀十六年,拜涼州都督,改安西都護、西州刺史。其地高昌舊都,流徙罪人與鎮兵雜,限以沙磧,隔絕中國,孝恪推誠撫御,盡得其歡心。初,王師滅高昌,詔以所虜焉耆生口七百還焉耆王。王叛歸欲谷設可汗,孝恪請擊之,即拜西州道行軍總管,率步騎三千出銀山道,夜襲其王龍突騎支,虜之。帝悅,降璽書褒勞。



 俄拜昆丘道副大總管,進討龜茲,破其國城,乃自留守,遣餘軍分道進。龜茲國相那利遁去。孝恪以餘部未平,出營於外。國人有謂孝恪曰:「那利素得士心,今亡在外,勢必為變,城中頗有異志,願公備之。」孝恪忽其言,不設備。那利果率眾陰與城內胡為應,薄城鼓噪,始覺之,乃率千餘人合戰,城中舉應那利,孝恪殊死鬥,中流矢卒,子待詔亦歿。將軍曹繼叔進兵,復拔其城。太宗責孝恪斥候不明,至顛覆,奪其官。後愍死戰,更為舉哀。高宗即位,追還官爵,贈待詔游擊將軍,賻物三百段。



 次子待封,官左豹韜衛將軍。咸亨初,副薛仁貴討吐番,戰大非川,敗績,貸死為民。



 張儉,字師約,京兆新豐人。隋相州刺史、皖城郡公威孫。父植,車騎將軍、連城縣公。



 儉,高祖從外孫也。高祖起,儉以功除右衛郎將,遷朔州刺史。時頡利可汗方強,每有求取,所遣書輒稱詔敕,邊吏奉承不敢卻。及儉,獨拒不受。大教民營田,歲收穀數十萬斛。雖霜旱,勸百姓相振贍,免饑殍,州以完安。李靖既平突厥,有思結部者,窮歸於儉,儉受而安輯之。其在磧北者,親戚私相過省,儉不禁,示羈縻而已。儉徙勝州,後將不察其然,遽奏思結叛,朝廷議進討,時儉以母喪,奪服為使者撫納之。儉單騎入其部,召酋帥慰諭,推腹心,咸匍匐歸命,因舉徙代州,遂檢校代州都督。儉勸墾田力耕,歲數稔,私蓄富實。儉恐虜易驕,乃建平糴法,入之官,虜悅喜,由是儲斛流贏。



 遷營州都督,兼護東夷校尉。坐事免,詔白衣領職。營州部與契丹、奚、霫、靺鞨諸蕃切畛,高麗引眾入寇,儉率兵破之,俘斬略盡。復拜營州都督。太宗將征遼東,遣儉率蕃兵先進,略地至遼西,川漲,久未度。帝以為畏懦,召還。見洛陽宮,陳水草美惡、山川險易,並久不進狀。帝悅,拜行軍總管,使領諸蕃騎,為六軍前鋒。時高麗候者言莫離支且至,帝詔儉自新城路邀擊,虜不敢出。儉進度遼,趨建安城,破賊,斬數千級。累封皖城郡公。後改東夷校尉官為都護府,即以儉為都護。永徽初,加金紫光祿大夫。卒,年六十,謚曰密。



 儉兄大師,太僕卿、華州刺史、武功縣男。



 弟延師,左衛大將軍、範陽郡公。性謹畏,黃羽林兵三十年,未嘗有過。卒,贈荊州都督,謚曰敬,陪葬昭陵。



 儉兄弟三人門皆立戟,時號「三戟張家」。



 王方翼,字仲翔,並州祁人。祖裕,隨州刺史,尚同安大長公主,官開府儀同三司,卒,謚曰文。



 方翼早孤,哀毀如成人,時號孝童。母李,為主所斥,居鳳泉墅。方翼尚幼,雜庸保,執苦不棄日,墾田植樹,治林垠,既完墻屋,燎松丸墨,為富家。主薨,還京師。嘗夜行,見長人丈餘,引弓射僕之,乃配木也。太宗聞,擢右千牛。高宗立,而從祖女弟為皇后,調安定令,誅滅大姓,奸豪脅息。徙瀚海都護司馬,坐事下遷朔州尚德府果毅,歲餘代還。居母喪,哀瘠甚,帝遣侍醫療視。其友趙持滿誅死,尸諸道,親戚莫敢視,方翼曰:「欒布哭彭越,義也;周文王掩骼,仁也。絕友義,蔽主仁,何以事君?」遂往哭其尸,具禮收葬。金吾劾系,帝嘉之,不罪。



 再遷肅州刺史。州無隍塹,寇易以攻,方翼乃發卒建樓堞,廝多樂水自環,烽邏精明。儀鳳間,河西蝗,獨不至方翼境,而它郡民或餒死,皆重繭走方翼治下。乃出私錢作水磑,簿其贏,以濟饑瘵,構舍數十百楹居之,全活甚眾,芝產其地。



 裴行儉討遮匐,奏為副,兼檢校安西都護,徙故都護杜懷寶為庭州刺史。方翼築碎葉城,面三門,紆還多趣以詭出入,五旬畢。西域胡縱觀,莫測其方略,悉獻珍貨。未幾,徙方翼庭州刺史,而懷寶自金山都護更鎮安西,遂失蕃戎之和。



 永淳初,十姓阿史那車簿啜叛,圍弓月城,方翼引軍戰伊麗河。敗之,斬首千級。俄而三姓咽面兵十萬踵至,方翼次熱海,進戰,矢著臂,引佩刀斷去,左右莫知。所部雜虜謀執方翼為內應,方翼悉召會軍中,厚賜,以次出壁外,縛之。會大風,雜金鼓,而號呼無聞者,殺七千人。即遣騎分道襲咽面等,皆驚潰,烏鶻引兵遁去,擒首領突騎施等三百人,西戎震服。初,方翼次葛水,暴漲,師不可度,沉祭以禱,師涉而濟。又七月次葉河,無舟,而冰一昔合。時以為祥。



 西域平,以功遷夏州都督。屬牛疫,民廢田作,方翼為耦耕法,張機鍵,力省而見功多,百姓順賴。明年,召方翼議西域事,引見奉天宮,賜食帝前,帝見衣有污濯處,問其故,具對熱海苦戰狀。視其創,帝咨嗟久之,賜賚良厚。



 俄而妖賊白鐵余以綏州反,詔方翼與程務挺討之。飛■擊賊,火其柵,平之,封太原郡公。阿史那元珍入寇,被詔進擊。時庫無完鎧,方翼斷六板,畫虎文,鉤聯解合,賊馬忽見,奔駭,遂敗,獲大將二,因降桑乾、舍利二部。



 武后時,王後屬無在者,方翼自視功多,冀不坐,而後內欲因罪除之,未得也。及務挺被殺,即並坐方翼,追入朝,捕送獄,流崖州,卒於道,年六十三。神龍初,復官爵。方翼善書,與魏叔琬齊名。



 子珣,字伯玉,與兄璵、弟瑨以文學稱,時號「三王」。天授初,珣及進士第,應制科,遷藍田尉。以拔萃擢長安尉,因進見,武后召問刑政,嘉之。詢其族氏,對曰:「廢後,臣之姑也。」後不悅,左遷亳州司法參軍。神龍初,為河南丞,武三思矯制貶臨川令。宋璟輔政,召授侍御史。出許州長史。歲旱,珣時假刺史事,開廩振民,即自劾,玄宗赦之。累遷工部侍郎。而瑨至中書舍人。珣嘗為秘書少監,數年而瑨繼職。終右散騎常侍,卒。贈戶部尚書,謚曰孝。



 子金肙,天寶中歷右補闕、殿中侍御史。瑨子鉷,自有傳。



 蘇烈,字定方,以字行,冀州武邑人,後徙始平。父邕,當隋季,率里中數千人為本郡討賊。定方驍悍有氣決,年十五,從父戰,數先登陷陣。邕卒,代領其眾,破劇賊張金稱、楊公卿,追北數十里,自是賊不舍境,鄉黨賴之。



 貞觀初,為匡道府折沖,從李靖襲突厥頡利於磧口,率彀馬二百為前鋒,乘霧行,去賊一里許,霧霽,見牙帳,馳殺數十百人,頡利及隋公主惶窘各遁去,靖亦尋至,餘黨悉降。再遷左衛中郎將。與程名振討高麗,破之。拜右屯衛將軍、臨清縣公。



 從蔥山道大總管程知節徵賀魯,至鷹娑川,賀魯率二萬騎來拒,總管蘇海政連戰未決,鼠尼施等復引二萬騎為援。定方始休士,見塵起,率精騎五百,逾嶺馳搗賊營,賊眾大潰,殺千餘人,所棄鎧仗、牛馬藉藉山野不可計。副總管王文度疾其功,謬謂知節曰:「賊雖走,軍死傷者眾。今當結輜重陣間,被甲而趨,賊來即戰,是謂萬全。」又矯制收軍不深入。於是馬臒卒勞,無鬥志。定方說知節曰:「天子詔討賊,今反自守,何功之立哉?且公為大將,而閫外之事不得專,顧副將乃得專之,理不其然!胡不囚文度待天子命?」不從。至怛篤城,有胡人降,文度猥曰:「師還而降,且為賊,不如殺之,取其貲。」定方曰:「此乃自作賊耳,寧曰伐叛!」及分財,定方一不取。高宗知之,比知節等還,悉下吏,當死,貸為民。



 擢定方伊麗道行軍大總管,復徵賀魯,以任雅相、回紇婆潤為副。出金山北,先擊處木昆部,破之,俟斤嬾獨祿擁眾萬帳降,定方撫之,發其千騎並回紇萬人,進至曳咥河。賀魯率十姓兵十萬拒戰,輕定方兵少,舒左右翼包之。定方令步卒據高,攢槊外向,親引勁騎陣北原。賊三突步陣,不能入,定方因其亂擊之,鏖戰三十里,斬首數萬級,賊大奔。明日,振兵復進,五弩失畢舉眾降,賀魯獨與處木昆屈律啜數百騎西走。定方令副將蕭嗣業、回紇婆潤率雜虜兵趨邪羅斯川追北,定方與雅相領新附兵絕其後。會大雪,吏請少休,定方曰:「虜恃雪,方止舍,謂我不能進,若縱使遠遁,則莫能擒。」遂勒兵進至雙河,與彌射、步真合,距賀魯所百里,下令陣而行,薄金牙山。方賀魯將畋,定方縱擊,破其牙下數萬人,悉歸所部。賀魯走石國,彌射子元爽以兵與嗣業會,縛賀魯以還。由是脩亭障,列蹊隧,定強畛,問疾收胔,唐之州縣極西海矣。高宗臨軒,定方戎服奉賀魯以獻。策功拜左驍衛大將軍、邢國公,別封子慶節為武邑縣公。



 會思結闕俟斤都曼先鎮諸胡,劫所部及疏勒、硃俱波、喝般陀三國復叛,詔定方還為安撫大使。率兵至葉葉水,而賊堞馬頭川。定方選精卒萬、騎三千襲之,晝夜馳三百里,至其所。都曼驚,戰無素,遂大敗,走馬保城。師進攻之,都曼計窮,遂面縛降。俘獻於乾陽殿,有司請論如法。定方頓首請曰:「臣向諭陛下意,許以不死,願丐其命。」帝曰:「朕為卿全信。」乃宥之。蔥嶺以西遂定。加食邢州鉅鹿三百戶,遷左武衛大將軍。



 出為神丘道大總管,率師討百濟。自城山濟海至熊津口,賊瀕江屯兵,定方出左涯,乘山而陣,與之戰,賊敗,死者數千。王師乘潮而上,舳艫銜尾進,鼓而噪,定方將步騎夾引,直趨真都城。賊傾國來,酣戰,破之,殺虜萬人,乘勝入其郛,王義慈及太子隆北走。定方進圍其城,義慈子泰自立為王,率眾固守。義慈之孫文思曰:「王與太子出,而叔豈得擅為王?若王師還,我父子安得全?」遂率左右縋城下,人多從之,泰不能止。定方使士登城,建唐旗幟。於是泰開門請命,其將禰植與義慈降,隆及諸城送款,百濟平,俘義慈、隆、泰等獻東都。



 定方所滅三國,皆生執其王,賞賚珍寶不勝計,加慶節尚輦奉御。未幾,定方為遼東道行軍大總管,俄徙平壤道。破高麗之眾於浿江,奪馬邑山為營,遂圍平壤。會大雪,解圍還。拜涼州安集大使,以定吐蕃、吐谷渾。乾封二年卒,年七十六。帝悼之,責謂侍臣曰:「定方於國有功,當褒贈,若等不言,何邪?」乃贈左驍衛大將軍、幽州都督,謚曰莊。



 薛仁貴,絳州龍門人。少貧賤,以田為業。將改葬其先,妻柳曰:「夫有高世之材,要須遇時乃發。今天子自征遼東,求猛將,此難得之時,君盍圖功名以自顯?富貴還鄉,葬未晚。」仁貴乃往見將軍張士貴應募。



 至安地,會郎將劉君邛為賊所圍,仁貴馳救之,斬賊將,系首馬鞍,賊皆懾伏,由是知名。王師攻安市城,高麗莫離支遣將高延壽等率兵二十萬拒戰,倚山結屯,太宗命諸將分擊之。仁貴恃驍悍,欲立奇功,乃著白衣自標顯,持戟,腰鞬兩弓,呼而馳,所向披靡;軍乘之,賊遂奔潰。帝望見,遣使馳問:「先鋒白衣者誰?」曰:「薛仁貴。」帝召見,嗟異,賜金帛、口馬甚眾,授游擊將軍、雲泉府果毅,令北門長上。師還,帝謂曰:「朕舊將皆老,欲擢驍勇付閫外事,莫如卿者。朕不喜得遼東,喜得皦將。」遷右領軍中郎將。



 高宗幸萬年宮,山水暴至,夜突玄武門,宿衛皆散走,仁貴曰:「當天子緩急,安可懼死?」遂登門大呼,以警宮內,帝遽出乘高。俄而水入帝寢,帝曰:「賴卿以免,始知有忠臣也。」賜以御馬。



 蘇定方討賀魯,仁貴上疏曰:「臣聞兵出無名,事故不成,明其為賊,敵乃可服。今泥熟不事賀魯,為其所破,虜系妻子。王師有於賀魯部落轉得其家口者,宜悉取以還,厚加賚遣,使百姓知賀魯為暴而陛下至德也。」帝納之,遂還其家屬,泥熟請隨軍效死。



 顯慶三年,詔副程名振經略遼東,破高麗於貴端城,斬首三千級。明年,與梁建方、契苾何力遇高麗大將溫沙多門,戰橫山,仁貴獨馳入,所射皆應弦僕。又戰石城,有善射者,殺官軍十餘人,仁貴怒,單騎突擊,賊弓矢俱廢,遂生擒之。俄與辛文陵破契丹於黑山,執其王阿卜固獻東都。拜左武衛將軍,封河東縣男。



 詔副鄭仁泰為鐵勒道行軍總管。將行,宴內殿,帝曰:「古善射有穿七札者,卿試以五甲射焉。」仁貴一發洞貫,帝大驚,更取堅甲賜之。時九姓眾十餘萬,令驍騎數十來挑戰,仁貴發三矢,輒殺三人,於是虜氣懾,皆降。仁貴慮為後患,悉坑之。轉討磧北餘眾,擒偽葉護兄弟三人以歸。軍中歌曰:「將軍三箭定天山,壯士長歌入漢關。」九姓遂衰。



 鐵勒有思結、多覽葛等部,先保天山,及仁泰至,懼而降,仁泰不納,虜其家以賞軍,賊相率遁去。有候騎言:「虜輜重畜牧被野,可往取。」仁泰選騎萬四千卷甲馳,絕大漠,至仙萼河,不見虜,糧盡還。人饑相食,比入塞,餘兵才二十之一。仁貴亦取所部為妾,多納賕遣,為有司劾奏,以功見原。



 乾封初,高麗泉男生內附,遣將軍龐同善、高偘往慰納,弟男建率國人拒弗納,乃詔仁貴率師援送同善。至新城,夜為虜襲,仁貴擊之,斬數百級。同善進次金山,衄虜不敢前,高麗乘勝進,仁貴擊虜斷為二,眾即潰,斬馘五千,拔南蘇、木底、蒼巖三城,遂會男生軍。手詔勞勉。仁貴負銳,提卒二千進攻扶餘城,諸將以兵寡勸止。仁貴曰:「在善用,不在眾。」身帥士,遇賊輒破,殺萬餘人,拔其城,因旁海略地,與李勣軍合。扶餘既降,它四十城相率送款,威震遼海。有詔仁貴率兵二萬與劉仁軌鎮平壤,拜本衛大將軍,封平陽郡公,檢校安東都護,移治新城。撫孤存老,檢制盜賊,隨才任職,褒崇節義,高麗士眾皆欣然忘亡。



 咸亨元年,吐蕃入寇,命為邏娑道行軍大總管,率將軍阿史那道真、郭待封擊之,以援吐谷渾。侍封嘗為鄯城鎮守,與仁貴等夷,及是,恥居其下,頗違節度。初,軍次大非川,將趨烏海,仁貴曰:「烏海地險而瘴,吾入死地,可謂危道,然速則有功,遲則敗。今大非嶺寬平,可置二柵,悉內輜重,留萬人守之,吾倍道掩賊不整,滅之矣。」乃約齎,至河口,遇賊,破之,多所殺掠,獲牛羊萬計。進至烏海城,以待後援。待封初不從,領輜重踵進,吐蕃率眾二十萬邀擊取之,糧仗盡沒,待封保險。仁貴退軍大非川,吐蕃益兵四十萬來戰,王師大敗。仁貴與吐蕃將論欽陵約和,乃得還,吐谷渾遂沒。仁貴嘆曰:「今歲在庚午,星在降婁,不應有事西方,鄧艾所以死於蜀,吾固知必敗。」有詔原死,除名為庶人。



 未幾,高麗餘眾叛,起為雞林道總管。復坐事貶象州,會赦還。帝思其功,乃召見曰:「疇歲萬年宮,微卿,我且為魚。前日殄九姓,破高麗,爾功居多。人有言向在烏海城下縱虜不擊,以至失利,此朕所恨而疑也。今遼西不寧,瓜、沙路絕,卿安得高枕不為朕指麾邪?於是拜瓜州長史、右領軍衛將軍、檢校代州都督,率兵擊突厥元珍於雲州。突厥問曰:「唐將為誰?」曰:「薛仁貴。」突厥曰:「吾聞薛將軍流象州死矣,安得復生?」仁貴脫兜鍪見之,突厥相視失色,下馬羅拜,稍稍遁去。仁貴因進擊,大破之,斬首萬級,獲生口三萬,牛馬稱是。



 永淳二年卒,年七十。贈左驍衛大將軍、幽州都督,官給輿,護喪還鄉里。



 子訥,字慎言,起家城門郎,遷藍田令。富人倪氏訟息錢於肅政臺,中丞來俊臣受賕,發義倉粟數千斛償之。訥曰:「義倉本備水旱,安可絕眾人之仰私一家?」報上不與。會俊臣得罪,亦止。



 後突厥擾河北,武后以訥世將,詔攝左威衛將軍、安東道經略使。對同明殿,具言:「醜虜馮暴,以廬陵王藉言,今雖還東宮,議不堅信。若太子無動,賊不討而解。」後納其言。俄遷幽州都督、安東都護。改並州長史,檢校左衛大將軍。訥久處邊,有戰功。開元初,玄宗講武新豐,詔訥為左軍節度。時諸部頗失序,唯訥與解琬軍不動。帝令輕騎召之,至軍門,不得入。禮成,尤見慰勞。



 明年,契丹、奚、突厥連和,數入邊,訥建議請討,詔監門將軍杜賓客、定州刺史崔宣道與訥帥眾二萬出檀州。賓各議「方暑,士負戈贏糧深討,慮恐無功」,姚元崇亦持不可,訥獨曰:「夏草薦茂,羔犢方息,不費饋饟因盜資,振國威靈,不可失也。」天子方欲誇威四夷,喜奇功,乃聽訥言,而授紫微黃門三品以重之。師至灤河,與賊遇,諸將不如約,為虜覆,盡亡其軍。訥脫身走,而罪宣道及大將李思敬等八人,有詔斬以徇,獨賓客免,盡奪訥官爵。



 俄而吐蕃大酋坌達延、乞力徐等眾十萬寇臨洮,入蘭州,剽牧馬,詔訥白衣攝羽林將軍,為隴右防禦使,與王晙擊之。追及賊,戰武階驛,犄角劫之,破其眾;尾北至洮水,又戰長城堡,殺鹵數萬,擒其酋六指鄉彌洪,悉收所掠及仗械不貲。時帝欲自將北伐,及訥大克,乃止行。命紫微舍人倪若水即軍陟功狀,拜訥左羽林大將軍,復封平陽郡公,以子暢為朝散大夫。又授涼州鎮軍大總管,赤水、建康、河源邊州皆隸節度。俄為朔方行軍大總管。久之,以老致仕。卒,年七十二,贈太常卿,謚曰昭定。



 訥性沉勇寡言,其用兵,臨大敵益壯。



 弟楚玉,開元中為範陽節度使,以不職廢。生子嵩。



 嵩生燕、薊間,氣豪邁,不肯事產利,以膂力騎射自將。豫安祿山亂,晚為史朝義守相州。僕固懷恩破朝義,長驅河朔,嵩震懼,迎拜軍門,懷恩釋之,奏為檢校刑部尚書、相衛洺邢等州節度使。方大亂後,人亦厭禍,嵩謹奉職,頗有治名。大歷初,封高平郡王,實封二百戶,號其軍為昭義。遷檢校尚書右僕射,更封平陽。七年卒,贈太保。



 詔其弟曨知留後事,累加檢校太子少師。十年,為其將裴志清所逐,以兵歸田承嗣。曨奔洺州。請入朝,降服待罪銀臺門,赦之。乃分其地,以蒿族子擇為相州刺史,雄衛州刺史,堅洺州刺史。承嗣誘雄亂,不從,遣客刺殺之。



 初,嵩好蹴踘,隱士劉鋼勸止曰:「為樂甚眾,何必乘危邀晷刻歡?」嵩悅,圖其形坐右。嵩子平。



 平字坦途,年十二,為磁州刺史。父喪,軍吏以故事脅知留務,偽許之,已而讓曨,夕以喪歸。累授右衛將軍,宿衛三十年。宰相杜黃掌擢為汝州刺史,治有風績。王師討蔡,繇左龍武大將軍授鄭滑節度使,數戰有功。始,河溢瓠子,東泛滑,距城才二里所。平按求故道出黎陽西南,因命其佐裴弘泰往請魏博節度使田弘正,弘正許之。乃籍民田所當者易以它地,疏道二十里,以釃水悍,還壖田七百頃於河南,自是滑人無患。入為左金吾衛大將軍。未幾,復帥鄭滑。



 李師道平,詔分淄、青、齊、登、萊五州為平盧軍,徙平為節度使。王庭湊圍牛元翼,棣州危,詔平出援。平遣將李叔佐率兵二千往,刺史王稷饋餉陋狹,眾潰而歸,推突將馬士端為帥,劫屯士萬人,薄州堞。城中兵寡,平悉公帑家貲募銳卒二千迎戰,以奇兵掩賊輜重,賊狼顧,遂大敗,降,餘黨平。引謀亂者二千人斬堂皇下,脅從皆縱還田里,威震一方。詔遷檢校尚書右僕射,封魏國公。在鎮六年,兵鎧完礪,徭賦均一。寶歷初,入朝,民鄣路願留,數日得出。拜檢校司空、河中降隰節度使,復得隸晉、慈二州,益兵三千。進檢校司徒,更封韓召拜太子太保。以司徒致仕。卒,年八十,贈太傅。



 子從,字順之,以廕授左清道率府兵曹參軍,累遷汾州刺史,隄文谷、濾河二水,引溉公私田,汾人利之。徙濮州,儲粟二萬斛以備兇災。於是山東大水,詔右司郎中趙傑為賑恤使,傑表其才,擢將作監。終左領軍衛上將軍。贈工部尚書。



 程務挺,洺州平恩人。父名振,隋大業末,仕竇建德為普樂令,盜不跡境。俄棄賊自歸,高祖詔授永寧令,使率兵經略河北,即夜襲鄴縣,俘男女千餘人以歸,去數舍,閱婦人方乳者九十餘人,還之,鄴人感其仁。劉黑闥陷洺州,名振與刺史陳君賓自拔歸,母妻為賊所得。名振率眾千餘,掠冀、貝、滄、瀛等州,邀擊糧道,悉毀賊水陸餉具。黑闥怒,殺其母妻。賊平,請手斬黑闥,以其首祭母。拜營州長史,封東平郡公,賜物二千段、黃金三百兩。轉洺州刺史。太宗征遼東,召問方略,不合旨,帝勃然詰之,名振辯對益詳,帝意解,謂左右曰:「房玄齡常在朕前,見朕嗔餘人,色不能主。名振生平未識我,一旦誚讓,而辭吐不屈,奇士哉!」拜右驍衛將軍,平壤道行軍總管。攻沙卑城,破獨山陣,皆以少擊眾,號為名將。遷營州都督,兼東夷都護。擊高麗於貴端水,焚其新城。歷晉、蒲二州刺史,鏤方道總管。卒,贈右衛大將軍,謚曰烈。



 務挺少從父征討,以勇力聞,拜右領軍衛中郎將。破突厥六萬騎於雲州。會偽可汗阿史那伏念叛,總管李文暕等三將以次奔敗。詔裴行儉討之,以務挺副,檢校豐州都督。時伏念屯金牙山,務挺與副總管唐玄表引兵赴之,伏念懼,乃間道降於行儉,故裴炎以為非行儉功,遷務挺右武衛將軍,封平原郡公。



 綏州部落稽白鐵餘據城平叛,建偽號,署置百官,進攻綏德、大斌,殺官吏,火區舍。詔務挺與夏州都督王方翼討之,務挺生擒白鐵餘。進左驍衛大將軍,檢校左羽林軍。嗣聖初,與右領軍大將軍張虔勖等豫廢中宗、立豫王為皇帝,累被褒賚。以左武衛大將軍為單于道安撫大使,御突厥。務挺善綏御,士服其威愛,突厥憚之,不敢盜邊。



 裴炎下獄,務挺密表申治,又素與唐之奇、杜求仁善,或言務挺與炎及徐敬業潛相援結,後遣左鷹揚將軍裴紹業即軍中斬之,籍其家。突厥聞務挺死,率相慶,為立祠,每出師,輒禱焉。



 王孝傑,京兆新豐人。少以軍功進。儀鳳中,劉審禮討吐蕃,孝傑以副總管戰大非川,為虜執,贊普見之,曰「貌類吾父」,故不死,歸之。武后時,為右鷹揚衛將軍。孝傑居虜中久,悉其虛實。長壽元年,為武威道總管,與阿史那忠節討吐蕃,克龜茲、于闐、疏勒、碎葉等城。武后曰:「貞觀中,西境在四鎮,其後不善守,棄之吐蕃。今故土盡復,孝傑功也。」乃遷左衛大將軍。進夏官尚書、同鳳閣鸞臺三品,清源縣男。證聖初,復為朔方道總管,與吐蕃戰不利,免。



 會契丹李盡忠等叛,有詔起白衣為清邊道總管,將兵十八萬討之。軍至東硤石谷,與賊接。道隘虜眾,孝傑率銳兵先驅,出穀整陣,與賊戰,而後軍總管蘇宏暉以其軍退,援不至,為虜所乘,軍潰,孝傑墮穀死,士相蹂且盡。初,進軍平州,白鼠晝入營屯伏。皆謂「鼠坎精,胡象也,白質歸命,天亡之兆」。及戰,乃孝傑覆焉。時張說以管記還白狀,後問之,說具陳:「孝傑乃心國家,敢深入,以少當眾,雖敗,功可錄也。」乃贈夏官尚書、耿國公,以其子無擇為朝散大夫。遣使者斬宏暉,使未至而宏暉已立功,遂贖罪。



 唐璿,字休璟,以字行,京兆始平人。曾祖規,為後周驃騎大將軍。休璟少孤,授《易》於馬嘉運,傳《禮》於賈公彥,舉明經高第。為吳王府典簽,改營州戶曹參軍。會突厥誘奚、契丹叛,都督周道務以兵授休璟,破之於獨護山,數馘多,遷朔州長史。



 永淳中,突厥圍豐州,都督崔智辯戰死,朝廷議棄豐保靈、夏。休璟以為不可,上疏曰:「豐州控河遏寇,號為襟帶,自秦、漢以來,常郡縣之。土田良美,宜耕牧。隋季喪亂,不能堅守,乃遷就寧、慶,戎羯得以乘利而交侵,始以靈、夏為邊。唐初,募人以實之,西北一隅得以完固。今而廢之,則河傍地復為賊有,而靈、夏亦不足自安,非國家利也。」高宗從其言。垂拱中,遷安西副都護。會吐蕃破焉耆,安息道大總管韋待價等敗,休璟收其潰亡,以定西土,授靈州都督。乃陳方略,請復四鎮。武后遣王孝傑拔龜茲等城,自休璟倡之。



 聖歷中,授涼州都督、右肅政御史大夫、持節隴右諸軍副大使。吐蕃大將曲莽布支率騎數萬寇涼州,入洪源谷,休璟以兵數千臨高望之,見賊旗鎧鮮明,謂麾下曰:「吐蕃自欽陵死,贊婆降,莽布支新將兵,欲以示武,且其下皆貴臣酋豪子弟,騎雖精,不習戰,吾為諸君取之。」乃被甲先登,六戰皆克,斬二將,獲首二千五百,築京觀而還。吐蕃來請和,既宴,使者屢覘休璟,後問焉,對曰:「洪源之戰,是將軍多殺臣士卒,其勇無比,今願識之。」後嗟異,擢為右武威、金吾二衛大將軍。



 西突厥烏質勒失諸蕃和,舉兵相攻,安西道閉。武后詔休璟與宰相計議,不少選,畫所當施行者。既而邊州建請屯置,盡如休璟策。後曰:「恨用卿晚。」進拜夏官尚書、同鳳閣鸞臺三品。後誚楊再思、李嶠、姚元崇等曰:「休璟諫知邊事,卿輩十不當一。」改太子右庶子,仍知政事。



 會契丹入塞,復以夏官尚書檢校幽營等州都督、安東都護。時中宗為皇太子,休璟將行,進啟曰:「易之兄弟恩寵過幸,數入禁閫,非人臣所宜,願加防察。」帝復位,召授輔國大將軍、同中書門下三品、酒泉郡公。謂曰:「初欲召公計事,以有北狄憂,前日直言,今未忘也。」加特進、尚書右僕射,賜邑戶三百,封宋國公。



 是歲大水,上疏自劾免,不許。累遷檢校吏部尚書。景龍二年致仕。未幾,復起為太子少師、同中書門下三品,監脩國史。景雲初,以特進為朔方行軍大總管,備突厥;停舊封,別賜百戶。明年,復請老,給一品全祿。延和元年卒,年八十六,贈荊州大都督,謚曰忠。



 休璟以儒者號知兵,自碣石逾四鎮,其間綿地幾萬里,山川夷坦,障塞之要,皆能言之,故行師料敵未嘗敗。初得封,以賦絹數千散賙其族,又出財數十萬大為塋墓,盡葬其五服親,當時稱重。惟張仁願議築受降城,而休璟獨謂不可,卒就之,而漠南無虜患。始老,已逾八十,猶托倚權近求復用。於是賀婁尚宮方用事,附者輒榮赫,休璟乃為子娶其義女,故復起宰相,頗為時譏訾。其當國,亦無它毘益云。



 子先慎至陳州刺史,先擇為右金吾衛將軍。



 張仁願,華州下邽人。本名仁亶,以睿宗諱音近避之。有文武材。武后時,累遷殿中侍御史。御史郭弘霸者,稱後乃彌勒佛身,又鳳閣舍人張嘉福、王慶之請以武承嗣為皇太子,邀仁願聯章,仁願正色拒之。後王孝傑為吐刺軍總管,與吐蕃戰不利,仁願監其軍,因入言狀,孝傑坐免,擢仁願侍御史。



 萬歲通天中,監察御史孫承景監清邊軍,戰還,自圖先鋒當矢石狀。武后嘆曰:「御史乃能如是乎!」擢為右肅政臺中丞,詔仁願即敘其麾下功。仁願先問承景破敵曲折,承景實不行,所問皆窮。仁願劾奏承景罔上,虛列虜級。貶為崇仁令,以仁願代為中丞,檢校幽州都督。



 默啜寇趙、定,還出塞,仁願以兵邀之,賊引去,矢著其手,武后遣使勞問,賜藥注傅。遷並州都督長史。神龍中,進左屯衛大將軍,兼檢校洛州長史。會穀貴多盜,仁願一切捕殺,胔積府門,畿甸震懾,無敢犯。先是,賈敦頤嘗為長史,有政績,時人為之語曰:「洛有前賈後張,敵京兆三王。」



 三年,朔方軍總管沙吒忠義為突厥所敗,詔仁願攝御史大夫代之。既至,賊已去,引兵踵擊,夜掩其營,破之。始,朔方軍與突厥以河為界,北崖有拂雲祠,突厥每犯邊,必先謁祠禱解,然後料兵度而南。時默啜悉兵西擊突騎施,仁願請乘虛取漠南地,於河北築三受降城,絕虜南寇路。唐休璟以為「兩漢以來皆北守河,今築城虜腹中,終為所有」。仁願固請,中宗從之。表留歲滿兵以助功,咸陽兵二百人逃歸,仁願擒之,盡斬城下,軍中股慄,役者盡力,六旬而三城就。以拂雲為中城,南直朔方,西城南直靈武,東城南直榆林,三壘相距各四百餘里,其北皆大磧也,斥地三百里而遠。又於牛頭朝那山北置烽候千八百所。自是突厥不敢逾山牧馬,朔方益無寇,歲損費億計,減鎮兵數萬。初建三城也,不置壅門、曲敵、戰格。或曰:「邊城無守備,可乎?」仁願曰:「兵貴攻取,賤退守。寇至,當並力出拒,敢回望城者斬,何事守備,退忸其心哉!」後常元楷代為總管,始築壅門,議者益重仁願而輕元楷。



 景龍二年,拜左衛大將軍、同中書門下三品,封韓國公。春還朝,秋復督軍備邊,帝為賦詩祖道,賞賚不貲。遷鎮軍大將軍。睿宗立,乃致仕。加兵部尚書,稟祿全給。開元二年卒,贈太子少保。



 仁願為將,號令嚴,將吏信伏,按邊撫師,賞罰必直功罪。後人思之,為立祠受降城,出師輒享焉。宰相文武兼者,當時稱李靖、郭元振、唐休璟、仁願云。在朔方,奏用御史張敬忠、何鸞、長安尉寇泚、鄠尉王易從、始平主簿劉體微分總軍事,太子文學柳彥昭為管記,義烏尉晁良貞為隨機,皆著稱,後至大官,世名仁願知人。子之輔,至趙州刺史。



 張敬忠,自監察御史累遷吏部郎中,開元七年拜平盧節度使。



 王晙,滄州景城人,後徙洛陽。父行果,為長安尉,知名。晙少孤,好學。祖有方奇之,曰:「是子當興吾宗。」長豪曠,不樂為銜檢事。擢明經第,始調清苑尉,歷除殿中侍御史。會朔方元帥魏元忠討賊不利,劾奏副將韓思忠敗,律當誅。晙以「思忠偏裨,權不己制,且其人勇智可惜,不宜獨誅」,固爭,得釋,晙亦出為渭南令。


景龍末,授桂州都督。州有兵,舊常仰餉衡、永。晙始築羅郛,罷戍卒;埭江,開屯田數千頃,以息轉漕,百姓賴之。後求歸上塚,州人詣闕留。有詔:「桂往罹寇暴,戶口雕
 \gezhu{
  疒齊}
 ,宜即留,以須政成。」在桂逾期年,人丐刻石頌德。初,劉幽求放封州,廣州都督周利貞必欲殺之,道出晙所,晙知其故,留不遣。利貞移書督趣,幽求懼曰:「勢且難全,正恐累君,奈何?」晙曰:「公之坐,非朋友所絕。晙在,終不忍公無罪就死。俄崔湜等誅,幽求復執政,故詔幽求為刻石辭。遷鴻臚少卿,充朔方軍副大總管、安北大都護,豐安、安遠等城並授節度。進太僕少卿、隴右群牧使。



 開元二年,吐蕃以精甲十萬寇臨洮,次大來穀,其酋坌達延以兵踵而前。晙率所部二千與臨洮軍合,料奇兵七百,易胡服,夜襲,去賊五里,令曰:「前是寇,士大呼,鼓角應之。」賊驚,疑伏在旁,自相鬥死者萬計。俄而薛訥至武階,距大來二十里,賊陣兩軍間,互一舍而近。晙往迎訥,夜使壯士銜枚鏖突,虜駭引去,追至洮水,敗之,俘獲如積。以功加銀青光祿大夫、清源縣男,兼原州都督;以子珽為朝散大夫。又進並州都督長史。



 明年,突厥默啜為拔曳固所殺,其下多降,分置河曲。既而小殺繼降,降者稍稍叛去。晙上言:



 突厥向以國亂,故款塞,與部落無間也。延素北風,何嘗忘之?今徙處河曲,使內伺邊罅,久必為患。比者不受要約,兵已屢動,擅作烽區,閉障行李。虜脫南牧,降帳必與連衡,以相應接,表裏有敵,雖韓、彭、孫、吳,無所就功。請至農隙,令朔方軍大陳兵,召酋豪,告以禍福,啗以金繒,且言南方麋鹿魚米之饒,並遷置淮右、河南寬鄉,給之程糧。雖一時之勞,然不二十年,漸服諸華,料以充兵,則皆勁卒。議者若謂降狄不可以南處,則高麗舊俘置沙漠之西,城傍編夷居青、徐之右,何獨降胡不可徙歟?



 臣復料議者必曰:「故事,置於河曲,前日已寧,今無獨異。」且往者頡利破亡,邊鄙安定,故降戶得以久安。今虜未殄滅,此降人皆戚屬,固不與往年同已。臣請以三策料之:悉其部落置內地,獲精兵之實,閉黠虜之患,此上策也;亭障之下,蕃華參處,廣屯戍,為備擬,費甚人勞,下策也;置之胡塞,滋成禍萌,此無策也。不然,前至河冰,且必有變。



 書未報,而虜已叛,乃敕晙將並州兵濟河以討。晙間行,卷甲舍幕趨山谷,夜遇雪,恐失期,誓於神曰:「晙事君不以忠,不討有罪,天所殛者,當自蒙罰,士眾無罪。心誠忠,而天監之,則止雪反風,以獎成功。」俄而和霽。時叛胡分二道走,晙自東道追及之,獲級三千。以功遷左散騎常侍、朔方行軍大總管。改御史大夫。趯跌部及僕固都督勺磨等散保受降城之鄙,潛引突厥內擾,晙密言上,盡誘而誅之。拜兵部尚書,復為朔方軍大總管。



 九年,蘭池胡康待賓據長泉反,陷六州,詔郭知運與晙討平之。封清源公,官一子。玄宗以宮人賜知運等,晙獨不敢取,曰:「臣之事君,猶子事父,詎有常近圍掖而臣子敢當乎?誓死以免。」見聽。初,晙奏:「朔方兵力有餘,願罷知運,獨當戍。」未報,而知運至,故不協。晙所降附,知運輒縱擊,賊意晙賣己,乃復叛。晙坐貶梓州刺史。改太了詹事、中山郡公。進吏部尚書、太原尹。代張說為兵部尚書、同中書門下三品,充朔方軍節度大使,河北、河西、隴右、河東之軍盡屬。是冬,帝親郊,追會大禮,晙以冰壯,請留將兵待邊,手敕慰勉。會有人告許州刺史王喬謀反,辭逮晙,詔源乾曜、張說雜訊,無狀,以黨與貶蘄州刺史。遷定州。復以戶部尚書為朔方軍節度使。卒,贈尚書左丞相,謚曰忠烈。



 晙氣貌偉特,時謂為熊虎相。感慕節義,有古人風。其操下肅壹,吏人畏愛。始,二張之誣魏元忠,晙獨上疏申治。宋璟曰:「魏公全矣,子再觸逆鱗,其殆乎!」晙曰:「魏公以忠獲罪,茍得辨,雖死弗悔。」



 晙卒後,信安王禕討奚於幽州,各捷,且言「戰時,士咸見晙與部將高昭麾兵赴敵」,天子嗟異。戶部郎中陽伯成上疏,請封晙墓,表異之,優其子孫。帝乃遣使祭晙廟,進諸子官。



 贊曰:「唐所以能威振夷荒、斥大封域者,亦有虎臣為之牙距也。至師行數千萬里,窮討殊斗,獵取其國由鹿豕然,可謂選值其才歟!夫宰相代天秩物,燮化人神,惟有德者宜之。若休璟、仁願,用以丞弼,非強所不能邪?據功名之地,則綽綽矣。



\end{pinyinscope}