\article{列傳第一百四十七上 南蠻上}

\begin{pinyinscope}
南詔,或曰鶴拓,曰龍尾,曰苴咩,曰陽劍只是其中之一),人把這些形式和範疇加到感性的素材上去,,本哀牢夷後,烏蠻別種也。夷語王為「詔」。其先渠帥有六,自號「六詔」,曰蒙秀詔、越析詔、浪穹詔、
 \gezhu{
  辶登}
 睒詔、施浪詔、蒙舍詔。兵埒,不能相君,蜀諸葛亮討定之。蒙舍詔在諸部南,故稱南詔。居永昌、姚州之間,鐵橋之南,東距爨,東南屬交趾,西摩伽陀,西北與吐蕃接,南女王,西南驃,北抵益州,東北際黔、巫。王都羊苴咩城,別都曰善闡府。



 王坐東向,其臣有所陳,以狀言而不稱臣。王自稱曰元,猶朕也;謂其下曰昶,猶卿、爾也。官曰坦綽、曰布燮、曰久贊,謂之清平官,所以決國事輕重,猶唐宰相也;曰酋望、曰正酋望、曰員外酋望、曰大軍將、曰員外,猶試官也。幕爽主兵,琮爽主戶籍,慈爽主禮,罰爽主刑,勸爽主官人,厥爽主工作,萬爽主財用,引爽主客,禾爽主商賈,皆清平官、酋望、大軍將兼之。爽,猶言省也。督爽,總三省也。乞托主馬,祿托主牛,巨托主倉廩,亦清平官、酋望、大軍將兼之。曰爽酋、曰彌勤、曰勤齊,掌賦稅。曰兵獳司,掌機密。大府主將曰演習,副曰演覽;中府主將曰繕裔,副曰繕覽;下府主將曰澹酋,副曰澹覽;小府主將曰幕捴,副曰幕覽。府有陀酋,若管記;有陀西,若判官。大抵如此。凡調發,下文書聚邑,必占其期。百家有總佐一,千家有治人官一,萬家有都督一。凡田五畝曰雙。上官授田四十雙,上戶三十雙,以是而差。壯者皆為戰卒,有馬為騎軍。人歲給韋衫褲。以邑落遠近分四軍,以旗幟別四方,面一將統千人,四軍置一將。凡敵入境,以所入面將御之。王親兵曰硃弩佉苴。佉苴,韋帶也。擇鄉兵為四軍羅苴子,戴硃鞮鍪,負犀革銅盾而跣,走險如飛。百人置羅苴子統一人。


望苴蠻者,在蘭蒼江西。男女勇捷,不鞍而騎,善用矛劍,短甲蔽胸腹,鞮鍪皆插貓牛尾,馳突若神。凡兵出,以望苴子前驅。以清平子弟為羽儀。王左右有羽儀長八人,清平官見王不得佩劍,唯羽儀長佩之為親信。有六曹長,曹長有功補大軍將。大軍將十二,與清平官等列,日議事王所,出治軍壁稱節度,次補清平官。有內算官,代王裁處;外算官,記王所處分,以付六曹。外則有六節度,曰:弄棟、永昌、銀生、劍川、柘東、麗水。有二都督:會川、通海。有十瞼,夷語瞼若州,曰:雲南瞼、白厓瞼亦曰勃弄瞼、品澹瞼、
 \gezhu{
  辶登}
 川瞼、蒙舍瞼、大厘瞼亦曰史瞼、苴咩瞼亦曰陽瞼、蒙秦瞼、矣和瞼、趙川瞼。



 祁鮮山之西多瘴歊,地平,草冬不枯。自曲靖州至滇池,人水耕,食蠶以柘,蠶生閱二旬而繭,織錦縑精致。大和、祁鮮而西,人不蠶,剖波羅樹實,狀若絮,紐縷而幅之。覽瞼井產鹽最鮮白,惟王得食,取足輒滅灶。昆明城諸井皆產鹽,不征,群蠻食之。永昌之西,野桑生石上,其林上屈兩向而下植,取以為弓,不筋漆而利,名曰瞑弓。長川諸山,往往有金,或披沙得之。麗水多金麩。越睒之西,多薦草,產善馬,世稱越睒駿。始生若羔,歲中紐莎縻之,飲以米潘,七年可御,日馳數百里。



 王出,建八旗,紫若青,白斿;雉翣二;有旄鉞,紫囊之;翠蓋。王母曰信麼,亦曰九麼。妃曰進武。信麼出,亦建八旗,絳斿。自曹長以降,系金佉苴。尚絳紫。有功加錦,又有功加金波羅。金波羅,虎皮也。功小者,衿背不袖,次止於衿。婦人不粉黛,以蘇澤發。貴者綾錦裙襦,上施錦一幅。以兩股辮為鬟髻,耳綴珠貝、瑟瑟、虎魄。女、嫠婦與人亂,不禁,婚夕私相送。已嫁有奸者,皆抵死。俗以寅為正,四時大抵與中國小差。膾魚寸,以胡瓜、椒、菼和之,號鵝闕。吹瓢笙,笙四管。酒至客前,以笙推盞勸酹。以繒帛及貝市易。貝者大若指,十六枚為一覓。師行,人齎糧斗五升,以二千五百人為一營。其法,前傷者養治,後傷者斬。犁田以一牛三夫,前挽、中壓、後驅。然專於農,無貴賤皆耕。不繇役,人歲輸米二斗。一藝者給田,二收乃稅。



 王蒙氏,父子以名相屬。自舍尨以來,有譜次可考。舍尨生獨邏,亦曰細奴邏,高宗時遣使者入朝,賜錦袍。細奴邏生邏盛炎,邏盛炎生炎閣。武后時,盛炎身入朝,妻方娠,生盛邏皮,喜曰:「我又有子,雖死唐地足矣。」炎閣立,死開元時。弟盛邏皮立,生皮邏閣,授特進,封臺登郡王。炎閣未有子時,以閣羅鳳為嗣,及生子,還其宗,而名承閣,遂不改。



 開元末,皮邏閣逐河蠻,取大和城,又襲大厘城守之,因城龍口,夷語山陂陀為「和」,故謂「大和」,以處閣羅鳳。天子詔賜皮邏閣名歸義。當是時,五詔微,歸義獨強,乃厚以利啖劍南節度使王昱,求合六詔為一。制可。歸義已並群蠻,遂破吐蕃,浸驕大。入朝,天子亦為加禮。又以破渳蠻功,馳遣中人冊為雲南王,賜錦袍、金鈿帶七事。於是徙治大和城。天寶初,遣閣羅鳳子鳳迦異入宿衛,拜鴻臚卿,恩賜良異。



 七載,歸義死,閣羅鳳立,襲王,以其子鳳迦異為陽瓜州刺史。初,安寧城有五鹽井,人得煮鬻自給。玄宗詔特進何履光以兵定南詔境,取安寧城及井,復立馬援銅柱,乃還。



 鮮於仲通領劍南節度使,卞忿少方略。故事,南詔嘗與妻子謁都督,過雲南,太守張虔陀私之,多所求丐,閣羅鳳不應。虔陀數詬靳之,陰表其罪。由是忿怨,反,發兵攻虔陀,殺之,取姚州及小夷州凡三十二。明年,仲通自將出戎、巂州,分二道進次曲州、靖州。閣羅鳳遣使者謝罪,願還所虜,得自新,且城姚州;如不聽,則歸命吐蕃,恐雲南非唐有。仲通怒,囚使者,進薄白厓城,大敗引還。閣羅鳳斂戰胔,築京觀,遂北臣吐蕃,吐蕃以為弟。夷謂弟「鐘」,故稱「贊普鐘」,給金印,號「東帝」。揭碑國門,明不得已而叛,嘗曰:「我上世世奉中國,累封賞,後嗣容歸之。若唐使者至,可指碑澡祓吾罪也。」會楊國忠以劍南節度當國,乃調天下兵凡十萬,使侍御史李宓討之,輦餉者尚不在。涉海而疫死相踵於道,宓敗於大和城,死者十八。亦會安祿山反,閣羅鳳因之取巂州會同軍,據清溪關,以破越析,梟於贈,西而降尋傳、驃諸國。



 尋傳蠻者,俗無絲纊,跣履榛棘不苦也。射豪豬,生食其肉。戰,以竹籠頭如兜鍪。其西有裸蠻,亦曰野蠻,漫散山中,無君長,作檻舍以居。男少女多,無田農,以木皮蔽形,婦或十或五共養一男子。廣德初,鳳迦異築柘東城,諸葛亮石刻故在,文曰:「碑即僕,蠻為漢奴。」夷畏誓,常以石搘捂。



 大歷十四年,閣羅鳳卒,以鳳迦異前死,立其孫異牟尋以嗣。異牟尋有智數,善撫眾,略知書。母李,獨錦蠻女也。獨錦蠻亦烏蠻種,在秦藏川南。天寶中,命其長為蹄州刺史。世與南詔婚聘。



 異牟尋立,悉眾二十萬入寇,與吐蕃並力。一趨茂州,逾文川,擾灌口;一趨扶、文,掠方維、白壩;一侵黎、雅,叩邛郲關。令其下曰:「為我取蜀為東府,工伎悉送邏娑城,歲賦一縑。」於是進陷城聚,人率走山。德宗發禁衛及幽州軍以援東川,與山南兵合,大敗異牟尋眾,斬首六千級,禽生捕傷甚眾,顛踣厓峭且十萬。異牟尋懼,更徙苴咩城,築袤十五里,吐蕃封為日東王。



 然吐蕃責賦重數,悉奪其險立營候,歲索兵助防,異牟尋稍苦之。故西瀘令鄭回者,唐官也,往巂州破,為所虜。閣羅鳳重其惇儒,號「蠻利」,俾教子弟,得棰搒,故國中無不憚。後以為清平官。說異牟尋曰:「中國有禮義,少求責,非若吐蕃惏刻無極也。今棄之復歸唐,無遠戍勞,利莫大此。」異牟尋善之,稍謀內附,然未敢發。亦會節度使韋皋撫諸蠻有威惠,諸蠻頗得異牟尋語,白於皋,時貞元四年也。皋乃遣諜者遺書,吐蕃疑之,因責大臣子為質,異牟尋愈怨。後五年,乃決策遣使者三人異道同趣成都,遺皋帛書曰:



 異牟尋世為唐臣,曩緣張虔陀志在吞侮,中使者至,不為澄雪,舉部惶窘,得生異計。鮮於仲通比年舉兵,故自新無繇。代祖棄背,二蕃欺孤背約。神川都督論訥舌使浪人利羅式眩惑部姓,發兵無時,今十二年。此一忍也。天禍蕃廷,降釁蕭墻,太子弟兄流竄,近臣橫污,皆尚結贊陰計,以行屠害,平日功臣,無一二在。訥舌等皆冊封王;小國奏請,不令上達。此二忍也。又遣訥舌逼城於鄙,弊邑不堪。利羅式私取重賞,部落皆驚。此三忍也。又利羅式罵使者曰:「滅子之將,非我其誰?子所富當為我有。」此四忍也。



 今吐蕃委利羅式甲士六十侍衛,因知懷惡不謬。此一難忍也。吐蕃陰毒野心,輒懷搏噬。有如媮生,實污辱先人,辜負部落。此二難忍也。往退渾王為吐蕃所害,孤遺受欺;西山女王,見奪其位;拓拔首領,並蒙誅刈;僕固志忠,身亦喪亡。每虜一朝亦被此禍。此三難忍也。往朝廷降使招撫,情心無二,詔函信節,皆送蕃廷。雖知中夏至仁,業為蕃臣,吞聲無訴。此四難忍也。



 曾祖有寵先帝,後嗣率蒙襲王,人知禮樂,本唐風化。吐蕃詐紿百情,懷惡相戚。異牟尋願竭誠日新,歸款天子。請加戍劍南、西山、涇原等州,安西鎮守,揚兵四臨,委回鶻諸國,所在侵掠,使吐蕃勢分力散,不能為強,此西南隅不煩天兵,可以立功云。



 且贈皋黃金、丹砂。皋護送使者京師,使者奏異牟尋請歸天子,為唐籓輔。獻金,示順革;丹,赤心也。德宗嘉之,賜以詔書,命皋遣諜往覘。



 皋令其屬崔佐時至羊苴咩城。時吐蕃使者多在,陰戒佐時衣牂柯使者服以入。佐時曰:「我乃唐使者,安得從小夷服?」異牟尋夜迎之,設位陳燎,佐時即宣天子意。異牟尋內畏吐蕃,顧左右失色,流涕再拜受命。使其子閣勸及清平官與佐時盟點蒼山,載書四:一藏神祠石室,一沈西洱水,一置祖廟,一以進天子。乃發兵攻吐蕃使者殺之,刻金契以獻,遣曹長跼南羅、趙迦寬隨佐時入朝。



 初,吐蕃與回鶻戰,殺傷甚,乃調南詔萬人。異牟尋欲襲吐蕃,陽示寡弱,以五千人行,許之。即自將數萬踵後,晝夜行,大破吐蕃於神川,遂斷鐵橋,溺死以萬計,俘其五王。乃遣弟湊羅棟、清平官尹仇寬等二十七人入獻地圖、方物,請復號南詔。帝賜賚有加,拜仇寬左散騎常侍,封高溪郡王。



 明年夏六月,冊異牟尋為南詔王。以祠部郎中袁滋持節領使,成都少尹龐頎副之,崔佐時為判官;俱文珍為宣慰使,劉幽巖為判官。賜黃金印,文曰「貞元冊南詔印」。滋至大和城,異牟尋遣兄蒙細羅勿等以良馬六十迎之,金鍐玉珂,兵振鐸夾路陳。異牟尋金甲,蒙虎皮,執雙鐸韒。執矛千人衛,大象十二引於前,騎軍、徒軍以次列。詰旦,授冊,異牟尋率官屬北面立,宣慰使東向,冊使南向,乃讀詔冊。相者引異牟尋去位,跽受冊印,稽首再拜;又受賜服備物,退曰:「開元、天寶中,曾祖及祖皆蒙冊襲王,自此五十年。貞元皇帝洗痕錄功,復賜爵命,子子孫孫永為唐臣。」因大會其下,享使者,出銀平脫馬頭盤二,謂滋曰:「此天寶時先君以鴻臚少卿宿衛,皇帝所賜也。」有笛工、歌女,皆垂白,示滋曰:「此先君歸國時,皇帝賜胡部、龜茲音聲二列,今喪亡略盡,唯二人故在。」酒行,異牟尋坐,奉觴滋前,滋受觴曰:「南詔當深思祖考成業,抱忠竭誠,永為西南籓屏,使後嗣有以不絕也。」異牟尋拜曰:「敢不承使者所命。」滋還,復遣清平官尹輔酋等七人謝天子,獻鐸鞘、浪劍、鬱刃、生金、瑟瑟、牛黃、虎珀、氎、紡絲、象、犀、越睒統倫馬。鐸鞘者,狀如殘刃,有孔傍達,出麗水,飾以金,所擊無不洞,夷人尤寶,月以血祭之。鬱刃,鑄時以毒藥並治,取迎躍如星者,凡十年乃成,淬以馬血,以金犀飾鐔首,傷人即死。浪人所鑄,故亦名浪劍,王所佩者,傳七世矣。



 異牟尋攻吐蕃,復取昆明城以食鹽池。又破施蠻、順蠻,並虜其王,置白厓城;因定磨些蠻,隸昆山西爨故地;破茫蠻,掠弄棟蠻、漢裳蠻,以實云南東北。



 施蠻者,在鐵橋西北,居大施睒、斂尋睒。男子衣繒布;女分發直額,為一髻垂後,跣而衣皮。



 順蠻本與施蠻雜居劍、共諸川。咩羅皮、鐸羅望既失邆川、浪穹,奪劍、共地,由是徙鐵橋,在劍睒西北四百里,號劍羌。



 磨蠻、些蠻與施、順二蠻皆烏蠻種,居鐵橋、大婆、小婆、三探覽、昆池等川。土多牛羊,俗不澤,男女衣皮,俗好飲酒歌舞。



 茫蠻本關南種,茫,其君號也,或呼茫詔。永昌之南有茫天連、茫吐薅、大睒、茫昌、茫鮓、茫施,大抵皆其種。樓居,無城郭。或漆齒,或金齒。衣青布短褲,露骭,以繒布繚腰,出其餘垂後為飾。婦人披五色娑羅籠。象才如牛,養以耕。



 弄棟蠻,白蠻種也。其部本居弄棟縣鄙地,昔為褒州,有首領為刺史,誤殺其參軍,挈族北走。後散居磨些江側,故劍、共諸川亦有之。



 漢裳蠻,本漢人部種,在鐵橋。惟以朝霞纏頭,餘尚同漢服。



 十五年,異牟尋謀擊吐蕃,以邆川、寧北等城當寇路,乃峭山深塹修戰備,帝許出兵助力。又請以大臣子弟質於皋,皋辭。固請,乃盡舍成都,咸遣就學。且言:「昆明、巂州與吐蕃接,不先加兵,為虜所脅,反為我患。」請皋圖之。時唐兵比歲屯京西、朔方,大峙糧,欲南北並攻取故地。然南方轉餉稽期,兵不悉集。是夏,虜麥不熟,疫癘仍興,贊普死,新君立。皋揣虜未敢動,乃勸異牟尋:「緩舉萬全,愈於速而無功。今境上兵十倍往歲,且行營皆在巂州,扼西瀘吐蕃路,昆明、弄棟可以無虞。」異牟尋請期它年。



 吐蕃大臣以歲在辰,兵宜出,謀襲南詔,閱眾治道,將以十月圍巂州。軍屯昆明凡八萬,皆命一歲糧。贊普以舅攘鄀羅為都統,遣尚乞力、欺徐濫鑠屯西貢川。異牟尋與皋相聞,皋命部將武免率弩士三千赴之,亢榮朝以萬人屯黎州,韋良金以二萬五千人屯巂州,約南詔有急,皆進軍,過俄準添城者,南詔供饋。吐蕃引眾五萬自曩貢川分二軍攻雲南,一軍自諾濟城攻巂州。異牟尋畏東蠻、磨些難測,懼為吐蕃鄉導,欲先擊之。皋報:「巂州實往來道,捍蔽數州,虜百計窺之,故嚴兵以守,屯壁相望,糧械處處有之,東蠻庸敢懷貳乎?」異牟尋乃檄東、磨些諸蠻內糧城中,不者悉燒之。吐蕃顒城將楊萬波約降,事洩,吐蕃以兵五千守,皋將擊破之。萬波與籠官拔顒城以來,徙其人二千於宿川。皋將扶忠義又取末恭城,俘系牛羊千計。贊普大將既煎讓律以兵距十貢川一舍而屯,國師馬定德率種落出降。西貢節度監軍野多輸煎者,贊普乞立贊養子,當從先贊普殉,亦詣忠義降。於是虜氣衰,軍不振。欺徐濫鑠至鐵橋,南詔毒其水,人多死,乃徙納川,壁而待。是年,虜霜雪早,兵無功還,期以明年。吐蕃苦唐、詔掎角,亦不敢圖南詔。皋令免按兵巂州,節級鎮守,雖南詔境亦所在屯戍。吐蕃懲野戰數北,乃屯三瀘水,遣論妄熱誘瀕瀘諸蠻,復城悉攝。悉攝,吐蕃險要也。蠻酋潛導南詔與皋部將杜毘羅狙擊。十七年春,夜絕瀘破虜屯,斬五百級。虜保鹿危山,毘羅伏以待,又戰,虜大奔。於時,康、黑衣大食等兵及吐蕃大酋皆降,獲甲二萬首。又合鬼主破虜於瀘西。



 吐蕃君長共計,不得巂州,患未艾,常為兩頭蠻挾唐為輕重,謂南詔也。會虜薦饑,方葬贊普,調斂煩。至是,大料兵,率三戶出一卒,虜法為大調集。又聞唐兵三萬入南詔,乃大懼,兵戍納川、故洪、諾濟、臘、聿齎五城,欲悉師出西山、劍山,收巂州以絕南詔。皋即上言:「京右諸屯宜明斥候,蚤斂田,邠、隴焚萊,可困虜入。」皋遣將邢毘以兵萬人屯南、北路,趙昱萬人戍黎、雅州。異牟尋謂皋曰:「虜聲取巂州,實窺雲南,請武免督軍進羊苴咩。若虜不出者,請以來年二月深入。」時虜兵三萬攻鹽州,帝以虜多詐,疑繼以大軍,詔皋深鈔賊鄙,分虜勢。皋表:「賊精鎧多置南屯,今向鹽、夏非全軍,欲掠河曲黨項畜產耳」。俄聞虜破麟州,皋督諸將分道出,或自西山,或由平夷,或下隴陀和、石門,或徑神川、納川,與南詔會。是時,回鶻、太原、邠寧、涇原軍獵其北,劍南東川、山南兵震其東,鳳翔軍當其西;蜀、南詔深入,克城七,焚堡百五十所,斬首萬級,獲鎧械十五萬。圍昆明、維州不能克,乃班師。振武、靈武兵破虜二萬,涇原、鳳翔軍敗虜原州。惟南詔攻其腹心,俘獲最多。帝遣中人尹偕尉異牟尋。而吐蕃盛屯昆明、神川、納川自守。異牟尋比年獻方物,天子禮之。



\end{pinyinscope}