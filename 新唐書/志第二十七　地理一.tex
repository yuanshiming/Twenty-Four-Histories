\article{志第二十七 地理一}

\begin{pinyinscope}

 自秦變古,王制亡,始郡縣天下。下更漢、晉,分裂為南、北。至隋滅陳,天下始合為一社會發展的原因;「市民社會」是全部歷史的真正發源地和舞,乃改州為郡,依漢制置太守,以司隸、刺史相統治,為郡一百九十,縣一千二百五十五,戶八百九十萬七千五百三十六,口四千六百一萬九千九百五十六。其地東西九千三百里,南北一萬四千八百一十五里,東、南皆至海,西至且末,北至五原。



 唐興,高祖改郡為州、太守為刺史,又置都督府以治之。然天下初定,權置州郡頗多。太宗元年,始命並省,又因山川形便,分天下為十道:一曰關內,一曰河南,三曰河東,四曰河北,五曰山南,六曰隴右,七曰淮南,八曰江南,九曰劍南,十曰嶺南。至十三年定簿,凡州府三百五十八,縣一千五百五十一。明年,平高昌,又增州二,縣六。其後,北殄突厥頡利,西平高昌,北逾陰山,西抵大漠。其地東極海,西至焉耆,南盡林州南境,北接薛延陀界;東西九千五百一十一里,南北一萬六千九百一十八里。景雲二年,分天下郡縣,置二十四都督府以統之。既而以其權重不便,罷之。開元二十一年,又因十道分山南、江南為東、西道,增置黔中道及京畿、都畿,置十五採訪使,檢察如漢刺史之職。天寶盜起,中國用兵,而河西、隴右不守,陷於吐蕃,至大中、咸通,始復隴右。乾符以後,天下大亂,至於唐亡。然舉唐之盛時,開元、天寶之際,東至安東,西至安西,南至日南,北至單于府,蓋南北如漢之盛,東不及而西過之。開元二十八年戶部帳,凡郡府三百二十有八,縣千五百七十三,戶八百四十一萬二千八百七十一,口四千八百一十四萬三千六百九,應受田一千四百四十萬三千八百六十二頃。



 考隋、唐地理之廣狹、戶口盈耗與其州縣廢置,其盛衰治亂興亡可以見矣。蓋自古為天下者,務廣德而不務廣地,德不足矣,地雖廣莫能守也。嗚呼,盛極必衰,雖曰勢使之然,而殆忽驕滿,常因盛大,可不戒哉!



 關內道,蓋古雍州之域,漢三輔、北地、安定、上郡及弘農、隴西、五原、西河、雲中之境。京兆、華、同、鳳翔、邠、隴、涇、原、渭、武、寧、慶、鄜、坊、丹、延、靈、威、雄、會、鹽、綏、宥為鶉首分,麟、豐、勝、銀、夏、單于、安北為實沈分,商為鶉火分。為府二,都護府二,州二十七,縣百三十五。其名山:太白、九嵕、吳、岐、梁、華。其大川:涇、渭、灞、滻。厥賦:絹、綿、布、麻。京兆、同、華、岐調綿,餘州布、麻。開元二十五年,以關輔寡蠶,詔納米粟,其河南、河北非通漕州,皆調絹,以便關中。厥貢:毛、羽、革、角、布、席、弓、刀。



 上都,初曰京城,天寶元年曰西京,至德二載曰中京,上元二年復曰西京,肅宗元年曰上都。皇城長千九百一十五步,廣千二百步。宮城在北,長千四百四十步,廣九百六十步,周四千八百六十步,其崇三丈有半。龍朔後,皇帝常居大明宮,乃謂之西內,神龍元年曰太極宮。大明宮在禁苑東南,西接宮城之東北隅,長千八百步,廣千八十步,曰東內,本永安宮,貞觀八年置,九年曰大明宮,以備太上皇清暑,百官獻貲以助役。高宗以風痺,厭西內湫濕,龍朔三年始大興葺,曰蓬萊宮,咸亨元年曰含元宮,長安元年復曰大明宮。興慶宮在皇城東南,距京城之東,開元初置,至十四年又增廣之,謂之南內,二十年,築夾城入芙蓉園。京城前直子午谷。後枕龍首山,左臨灞岸,右抵澧水,其長六千六百六十五步,廣五千五百七十五步,周二萬四千一百二十步,其崇丈有八尺。



 京兆府京兆郡,本雍州,開元元年為府。厥貢:水土稻、麥、麰、紫稈粟、隔紗、粲席、鞾氈、蠟、酸棗人、地骨皮、櫻桃、藕粉。天寶元年領戶三十六萬二千九百二十一,口百九十六萬一百八十八。領縣二十:有府百三十一,曰真化、匡道、水衡、仲山、新城、竇泉、善信、鳳神、安業、平香、太清,餘皆逸。萬年,赤。本大興,武德元年更名。二年析置芷陽縣,七年省。總章元年析置明堂縣,長安二年省。天寶七載曰咸寧,至德三載復故名。有南望春宮,臨滻水,西岸有北望春宮,宮東有廣運潭。福陵在東二十五里,敬陵在東南四十里。長安,赤。總章元年析置乾封縣,長安二年省。有大安宮,本弘義,後更名。南五十裏太和谷有太和宮,武德八年置,貞觀十年廢,二十一年復置,曰翠微宮,籠山為苑,元和中以為翠微寺。有子午關。天寶二年,尹韓朝宗引渭水入金光門,置潭於西市,以貯材木。大歷元年,尹黎幹自南山開漕渠抵景風、延喜門,入苑以漕炭薪。咸陽,畿。武德元年析涇陽、始平置。有望賢宮;有便橋;有興寧陵,又有順陵,在咸陽原。興平,畿。本始平,景龍四年,中宗送金城公主降吐蕃至此,改曰金城,至德二載更名。西十八里有隋仙林宮。雲陽,赤。武德元年析置石門縣,三年以石門、溫秀置泉州。貞觀元年州廢,省溫秀,更石門曰雲陽,雲陽曰池陽。八年省雲陽,更池陽曰雲陽。天授二年以雲陽、涇陽、醴泉、三原置鼎州,大足元年州廢。有堯山、甘泉山,凡禁樵採者著於志。有古鄭、白渠。崇陵在北十五里嵯峨山,貞陵在西北四十里。涇陽,畿。三原,次赤。武德四年曰池陽,六年曰華池,析置三原,隸泉州,貞觀元年省,復華池曰三原。永康陵在北十八里,獻陵在東十八里,莊陵在西北五里,端陵在東十里。渭南,畿。武德元年隸華州,五年還隸雍州。天授二年析渭南、慶山置鴻門縣,以渭南、慶山、鴻門、高陵、櫟陽置鴻州,尋省鴻門,大足元年州廢。西十里有游龍宮,開元二十五年更置。東十五里有隋崇業宮。昭應,次赤。本新豐,垂拱二年曰慶山,神龍元年復故名。有宮在驪山下,貞觀十八年置,咸亨二年始名溫泉宮。天寶元年更驪山曰會昌山。三載,以縣去宮遠,析新豐、萬年置會昌縣。六載,更溫泉曰華清宮,治湯井為池,環山列宮室,又築羅城,置百司及十宅;七載省新豐,更會昌縣及山曰昭應。東三十五里有慶山,垂拱二年湧出。有清虛原,本鳳凰,有幽棲谷,本鸚鵡,中宗以韋嗣立所居更名。有旌儒鄉,有廟,故坑儒,玄宗更名。齊陵在東十六里。高陵,畿。武德元年析置鹿苑縣,貞觀元年省。西四十里有龍躍宮,武德六年,高祖以舊第置,德宗以為脩真觀。有古白渠,寶歷元年,令劉仁師請更水道,渠成,名曰劉公,堰曰彭城。同官,畿。有女迥山。富平,次赤。有荊山,有鹽池澤。定陵在西北十五里龍泉山,元陵在西北二十五里檀山,豐陵在東三十三里甕金山,章陵在西北二十里,簡陵在西北四十里。藍田,畿。武德二年析置白鹿縣,三年更曰寧民,又析藍田置玉山縣,貞觀三年皆省。有覆軍山;有藍田關,故嶢關;有庫谷,穀有關。武德六年,寧民令顏昶引南山水入京城。永淳元年作萬全宮,弘道元年廢。鄠,畿。有渼陂。東南三十里有隋太平宮,西南二十二里有隋甘泉宮。奉天,次赤。文明元年,析醴泉、始平、好畤、武功、豳州之永壽置,以奉乾陵,陵在北五里梁山。靖陵在東北十里。乾寧二年以縣置乾州,及覃王出鎮,又以畿內之好畤、武功、盩厔、醴泉隸之。好畤,畿。故上宜,武德二年析醴泉置好畤。貞觀八年廢上宜入岐陽,二十一年省好畤、岐陽,復置上宜,更上宜曰好畤。有大橫關。武功,畿。武德三年,以武功、好畤、盩厔及郇州之郿、鳳泉置稷州,又析始平置扶風縣,四年以岐州之圍川隸之,七年以郿隸岐州。貞觀元年州廢,省扶風,以圍川、鳳泉隸岐州,盩厔、武功隸雍州。天授二年,復以武功、始平、奉天、盩厔、好畤置稷州,大足元年州廢。有太一山,高十八里。有慶善宮,臨渭水。武德元年,高祖以舊第置宮,後廢為慈德寺。西原,殤帝所葬。醴泉,次赤。武德元年析置溫秀縣,後省醴泉。貞觀十年營昭陵,析雲陽、咸陽復置。有芳山,有九嵕山。昭陵在西北六十里九嵕山;建陵在東北十八里武將山,一名馮山。華原,畿。義寧二年以華原、宜君、同官置宜君郡,並置士門縣以隸之。武德元年曰宜州,貞觀十七年州廢,省宜君、土門,以華原、同官隸雍州。垂拱二年更華原曰永安。天授二年復以永安、同官、富平、美原置宜州,大足元年州廢。有永安宮,長安二年置。神龍元年復永安曰華原。有蒲萄園宮。天祐三年,李茂貞墨制以縣置耀州。美原。畿。咸亨二年,析富平、華原及同州之蒲城,以故土門縣置。天祐三年,李茂貞墨制以縣置鼎州。



 華州華陰郡,上輔。義寧元年析京兆郡之鄭、華陰置,垂拱二年避武氏諱曰太州,神龍元年復故名,上元二年又更名太州,寶應元年復故名。乾寧四年曰興德府,縣次畿、赤。光化三年復為州。土貢:鷂、烏鶻、伏苓、伏神、細辛。戶三萬三千一百八十七,口二十二萬三千六百一十三。縣四:有府二十,曰普樂、豐原、義全、清義、萬福、脩仁、神水、常興、義津、定城、延壽、羅文、鄭邑、宣義、相原、孝德、溫湯、宣化、懷德、懷仁。有鎮國軍,肅宗上元元年置。鄭,望。有少華山。東北三里有神臺宮,本隋普德宮,咸亨二年更名。西南二十三里有利俗渠,引喬谷水,東南十五里有羅文渠,引小敷谷水,支分溉田,皆開元四年詔陜州刺史姜師度疏故渠,又立堤以捍水害。華陰,望。垂拱元年更名仙掌。天授二年析置潼津縣,在關口,後隸虢州,聖歷二年來屬,長安中省。神龍元年復曰華陰,上元二年曰太陰,華山曰太山,寶應元年復故名。有嶽祠;有潼關,有渭津關;有漕渠,自苑西引渭水,因古渠會灞、滻,經廣運潭至縣入渭,天寶三載韋堅開;又有永豐倉,有臨渭倉;西十八里有瓊岳宮,故隋華陰宮,顯慶三年更名;東十三里有隋金城宮,武德三年廢,顯慶三年復置;西二十四里有敷水渠,開元二年,姜師度鑿,以洩水害,五年,刺史樊忱復鑿之,使通渭漕。下邽,望。本隸同州,垂拱元年來屬。東南二十里有金氏二陂,武德二年引白渠灌之,以置監屯。櫟陽。本畿。故萬年,隸雍州。武德元年更名,又析置平陵縣,二年更平陵曰粟邑,貞觀八年省;有煮鹽澤。天祐三年來屬。



 同州馮翊郡,上輔。武德元年更諸郡為州,其沒於賊者,事平乃更。天寶三載以州為郡,乾元元年復以郡為州。凡州、郡、縣無所更置者皆承隋舊。土貢:鞾郭二物、皺紋吉莫、麝、芑茨、龍莎、凝水石。戶六萬九百二十八,口四十萬八千七百五。縣八:有府二十六,曰濟北、唐安、秦城、太州、大亭、河東、興德、連邑、伏龍、溫陽、安遠、業善、南鄉、臨高、瀵陽、襄城、崇道、淅谷、吉安、長春、華池、永大、洪泉、善福、司御、效誠。馮翊,望。武德九年析置臨沮縣,貞觀九年省。有沙苑。南三十二里有興德宮,在志武里,高祖將趨長安所次。朝邑,望。有長春宮。武德三年析置河濱縣,貞觀元年省。北四里有通靈陂,開元七年,刺史姜師度引洛堰河以溉田百餘頃。乾元三年曰河西,隸河中府,大歷五年復曰朝邑,還隸同州。有河瀆祠、西海祠。小池有鹽。韓城,上。武德八年徙置西韓州,貞觀八年州廢,以韓城、河西、郃陽來屬,天祐二年更名韓原。有鐵;有梁山;有龍門山;有關。武德七年,治中云得臣自龍門引河溉田六千餘頃。郃陽,望。有陽班湫,貞元四年堰水誇谷水成。夏陽,本河西,武德三年析郃陽置,又以河西、郃陽、韓城置西韓州。乾元三年更河西曰夏陽,隸河中,後復來屬。白水,望。澄城,望。武德三年析置長寧縣,貞觀八年省。奉先。本次赤。故蒲城,開元四年更名,隸京兆府。橋陵在西北三十里豐山,泰陵在東北二十里金粟山,景陵在西北二十里金熾山,光陵在北十五里堯山,惠陵在西北十里。有鹵池二,大中二年,其一生鹽。天祐三年來屬。



 商州上洛郡,望。土貢:麝香、弓材。有洛源監錢官。貞元七年,刺史李西華自藍田至內鄉開新道七百餘里,回山取塗,人不病涉,謂之偏路,行旅便之。戶八千九百二十六,口五萬三千八十。縣六:有府二,曰洵水、玉京。有興平軍,初在郿縣東原,至德中徙。上洛,緊。有熊耳山。豐陽,上。洛南,上。有金,有銅,有鐵。商洛,望。東有武關。上津,上。義寧二年以上津、豐利、黃土置上津郡,並置長利縣。武德元年曰上州。貞觀元年省長利。八年州廢,以黃土隸金州,豐利隸均州,上津來屬。乾元。中下。本安業,萬歲通天元年析豐陽置,景龍三年隸雍州,景雲元年來屬,乾元元年更名,隸京兆,尋復還屬。



 鳳翔府扶鳳郡,赤上輔。本岐州,至德元載更郡曰鳳翔,二載復郡故名,號西京,為府。上元二年罷京,元年曰西都,未幾復罷都。士貢:榛實、龍須席、蠟燭。戶五萬八千四百八十六,口三十八萬四百六十三。縣九:有府十三,曰岐山、雍北、道清、洛邑、留谷、岐陽、文城、支阜邑、三交、鳳泉、望苑、邵吉、山泉。天興,次赤。本雍,至德二載曰鳳翔,仍析置天興縣,寶應元年省鳳翔入天興。岐山。次畿。貞觀七年,析扶風、岐山及京兆之上宜置岐陽縣,八年省上宜入岐山,永徽五年復置,元和三年省。有岐山。扶風,次畿。本湋川,武德三年析岐山置,以湋水名之,貞觀八年更名。麟游,次畿。義寧元年置,以麟游及京兆之上宜、扶風郡之普潤置鳳棲郡。二年以仁壽宮中獲白麟,更郡曰麟游,又以安定郡之鶉觚並析置靈臺縣隸之。武德元年曰麟州。貞觀元年州廢,省靈臺入麟游,以麟游、普潤來屬,上宜還隸雍州,鶉觚還隸涇州。西五里有九成宮,本隋仁壽宮,義寧元年廢,貞觀五年復置,更名,永徽二年曰萬年宮,乾封二年復曰九成宮,周垣千八百步,並置禁苑及府庫官寺等;又西三十里有永安宮,貞觀八年置。普潤,次畿。有隴右軍,貞元十年置,十一年以縣隸隴右經略使,元和元年更名保義軍。寶雞,次畿。本陳倉,至德二載更名。東有渠引渭水入升原渠,通長安故城,咸亨三年開。西南有大散關,有寶雞山。虢,次畿。貞觀八年省入岐山,天授二年復置。東北十里有高泉渠、如意元年開,引水入縣城;又西北有升原渠,引汧水至咸陽,垂拱初運岐、隴水入京城。郿,次畿。義寧二年置郿城郡,又析置鳳泉縣。武德元年曰郇州,以鳳泉來屬,三年州廢,以郿隸稷州,七年來屬,貞觀八年省鳳泉。大歷五年權隸京兆。有太白山,有鳳泉湯。盩厔。本畿,隸雍州。武德二年析置終南縣,貞觀八年省,天寶元年更名宜壽,至德二載復故名,乾寧中隸乾州,天復元年來屬。有駱谷關,武德七年置;有司竹園;東南三十二里有隋宜壽宮,有樓觀、老子祠。



 邠州新平郡,緊。義寧二年析北地郡之新平、三水置。邠,故作「豳」,開元十三年以字類「幽」改。土貢:剪刀、火筋、蓽豆、澡豆、白蜜、地膽。戶二萬二千九百七十七,口十二萬五千二百五十。縣四:有府十,曰嘉陽、宜祿、公劉、良社、胡陵、蜯川、萬敵、金池、舜城、宜山。新平,望。有永定壘二,太宗討薛舉置。三水,緊。有石門山。北二十里有萬壽湫,大歷八年因風雷而成。永壽,上。武德二年析新平置,神龍元年隸雍州,唐隆元年來屬。宜祿。中。貞觀二年析新平及涇州之保定、靈臺置。有淺水原,有長武城。



 右京畿採訪使,治京城內。



 隴州汧陽郡,上。本隴東郡,義寧二年,析扶風郡之汧源、汧陽、南由,安定郡之華亭置。天寶元年更郡曰汧陽。土貢:榛實、龍須席。戶二萬四千六百五十二,口十萬一百四十八。縣三:有府四,曰大堆、龍盤、開川、臨汧。汧源,上。垂拱二年更華亭曰亭川,神龍元年復故名,元和三年省入汧源。西有安戎關,在隴山,本大震關,大中六年,防禦使薛逵徙築,更名。有鐵;有五節堰,引隴川水通漕,武德八年,水部郎中姜行本開,後廢。華亭有義寧軍,大歷八年置。貞元十三年築永信城於平戎川。汧陽,上。有臨汧城,大和元年築。吳山。中。本長蛇,義寧二年置,貞觀元年更名,上元二年曰華山,尋復曰吳山。武德元年以南由縣置含州,四年州廢,元和三年省入焉。有西鎮吳山祠,有紫塠山。西有安夷關。



 涇州保定郡,上。本安定郡,至德元載更名。土貢:龍須席。戶三萬一千三百六十五,口十八萬六千八百四十九。縣五:有府六,曰涇陽、四門、興教、純德、肅清、仁賢。保定,上。本安定,至德元載更名,廣德元年沒吐蕃,大歷三年復置。有折墌故城。靈臺,上。本鶉觚,天寶元年更名。臨涇,中。良原,上。興元二年沒吐蕃,貞元四年復置。潘原。中。本陰盤,天寶元年更名,後省為彰信堡,貞元十一年復置。



 原州平涼郡,中都督府,望。廣德元年沒吐蕃,節度使馬璘表置行原州於靈臺之百里城。貞元十九年徙治平涼。元和三年又徙治臨涇。大中三年收復關、隴,歸治平高。廣明後復沒吐蕃,又僑治臨涇。土貢:氈、覆鞍氈、龍須席。戶七千三百四十九,口三萬三千一百四十六。縣二:有府二,曰彭陽、安善。平高,望。有崆峒山;西南有木峽關。州境又有石門、驛藏、制勝、石峽、木崝等關,並木峽、大盤為七關。又南有瓦亭故關。百泉。上。



 渭州,元和四年以原州之平涼縣置行渭州,廣明元年為吐蕃所破,中和四年,涇原節度使張鈞表置。凡乾元後所置州,皆無郡名;及其季世,所置州縣,又不列上、中、下之第。縣一。平涼。上。廣德元年沒吐蕃,貞元四年復置。及為行渭州,其民皆州自領之。西南隴山有六盤關;有銀,有銅,有鐵;西北五里有吐蕃會盟壇,貞元三年築。



 武州,中。大中五年以原州之蕭關置。中和四年僑治潘原。縣一:蕭關。中。貞觀六年以突厥降戶置緣州,治平高之他樓城。高宗置他樓縣,隸原州,神龍元年省,置蕭關縣。白草軍在蔚茹水之西,至德後沒吐蕃。



 寧州彭原郡,望。本北地郡,天寶元年更名。土貢:五色覆鞍氈、龍須席、芫青、亭長、庵褲、假蘇。戶三萬七千一百二十一,口二十二萬四千八百三十七。縣五:有府十一,曰彭池、高望、靜難、驎寶、天固、蒲川、東原、三會、大延、和泉、永寧。定安,望。義寧二年析置歸義縣,貞觀十七年省入定平。有定安故關。真寧,緊。本羅川。有要冊湫。天寶元年獲玉真人像二十七,因更名。襄樂,緊。彭原,緊。武德元年以縣置彭州,二年析置豐義縣。貞觀元年州廢,以彭原、豐義來屬。開元八年以豐義隸涇州,尋復還屬,唐末省。定平。上。武德二年析定安置,後隸邠州。元和三年復來屬,四年隸左神策軍。有高摭城。唐末以縣置衍州。



 慶州順化郡,中都督府。本弘化郡,天寶元年曰安化,至德元載更名。土貢:「胡女布、牛酥、麝、蠟。戶二萬三千九百四十九,口十二萬四千二百三十六。縣十:有府八,曰龍息、交水、同川、永清、蟠交、永業、樂蟠、永安。順化,中。本弘化,天寶元年曰安化,至德元載更名。合水,中。本合川,武德元年置,是年,又析置蟠交縣。貞觀元年省合川入弘化。天寶元年更蟠交曰合水。樂蟠,中。義寧元年析合水置。馬嶺,中。華池,下。武德四年置,以縣置林州,貞觀元年州廢。同川,中下。本三泉,義寧二年析彭原郡之彭原置,武德三年更名。洛源,中。貞觀三年置,四年隸北永州,五年徙州來治,八年州廢,來屬。延慶,中。本白馬,武德六年徙故豐州民析合水置,天寶元年更名。方渠,中下。神龍三年析馬嶺置。懷安。下。開元十一年括逃戶連黨項蕃落置。



 鄜州洛交郡,上。本上郡,天寶元年更名。土貢:龍須席。戶二萬三千四百八十四,口十五萬三千七百一十四。縣五:有府十一,曰洛昌、龍交、葦川、五交、大同、安光、洛安、銀方、杏林、脩武、安吉。有肅戎軍,大歷六年置,在鹿阜城。洛交,緊。洛川,上。三川,中。華池水、黑水、洛水所會。直羅,中。武德三年析三川、洛交因古直羅城置,羅水過城下,地平直,故名。甘泉。中。本伏陸,武德元年析洛交置,天寶元年更名。



 坊州中部郡,上。武德二年析鄜州之中部、鄜城置。土貢:龍須席、枲、弦麻。戶二萬二千四百五十八,口十二萬二百八。縣四:有府五,曰杏城、仁里、思臣、永平、安臺。中部,上。本內部,武德二年更名。周天和中,元皇帝為敷州刺史,置馬坊,高祖因以名州。有鐵。州郭無水,東北七里有上善泉,開成二年,刺史張怡架水入城,以紓遠汲。四年,刺史崔駢復增修之,民獲其利。後思之,為立祠。宜君,上。本隸宜州。有仁智宮,武德七年置。貞觀十七年州廢,縣亦省。二十年置玉華宮,復置縣,隸雍州。宮在北四里鳳凰穀。永徽二年廢宮為玉華寺,縣又省。龍朔三年析中部、同官復置,來屬。有鐵。升平,上。天寶十二載析宜君置,寶應元年省,後復置。鄜城。上。唐末以縣置翟州。



 丹州咸寧郡,上。本丹陽郡,義寧元年析延安郡之義川、汾川、咸寧縣置,天寶元年更名。土貢:龍須席、麝、蠟燭。戶萬五千一百五,口八萬七千六百二十五。縣四:有府五,曰宜城、通天、同化、丹陽、長松。義川,上。雲巖,中。武德元年析義川置。汾川,上。有烏仁關。咸寧。中。



 延州延安郡,中都督府。土貢:樺皮、麝、蠟。戶萬八千九百五十四,口十萬四十。縣十:有府七,敦化、延川、寧戎、因城、塞門、延安、金明。又儀鳳中,吐谷渾部落自涼州內附,置二府於金明西境,曰羌部落,曰閤門。膚施,上。有牢山鎮城。延長,中。本延安,武德二年置,以縣置北連州,並置義鄉、齊明二縣以隸之。貞觀二年州廢,省義鄉、齊明入延安,來屬。廣德二年更名。臨真,中。武德元年隸東夏州,貞觀二年州廢來屬。金明,中。武德二年析膚施置,以縣置北武州,並置開遠、全義、崇德、永安、定義五縣。貞觀二年州廢,省開遠、全義、崇德、永安、定義入金明,來屬。豐林,中。武德四年僑置雲州及雲中、榆林、龍泉三縣,八年州廢,省龍泉入臨真,省雲中、榆林入豐林。東北有合嶺關。延川,中。武德二年招慰稽胡置基州;又安撫使段德操表置義門縣,以義門置南平州。三年析綏州之城平置魏平縣。四年廢南平州,省義門、魏平。五年更基州曰北基州。貞觀八年州廢,來屬。敷政,中下。本因城,武德二年徙治金城鎮,更名金城;又東境置永州,並置洛盤、新昌、土塠三縣。貞觀四年徙州治洛源。及州廢,省洛盤、新昌、土塠入金城。天寶元年曰敷政。延昌,武德二年置北仁州,貞觀三年州廢,十年以其地置罷交縣,天寶元年更名。其北蘆子關。延水,中下。本安民,武德二年析延川置,以縣置西和州,並置修文、桑原二縣。貞觀二年州廢,省修文、桑原入安民,隸北基州。州廢,來屬。二十三年曰弘風,神龍元年更名。門山。上。武德三年析汾川置,隸丹州,廣德二年來屬。



 靈州靈武郡,大都督府。土貢:紅藍,甘草,花蓯蓉,代赭,白膠,青蟲,雕,鶻,白羽,麝,野馬、鹿革,野豬黃,吉莫鞾,郭,氈,庫利,赤檉,馬策,印鹽,黃牛臆。戶萬一千四百五十六,口五萬三千一百六十三。縣四:有府五,曰武略、河間、靜城、鳴沙、萬春。有朔方經略軍。黃河外有豐安、定遠、新昌等軍,豐寧、保寧等城。回樂,望。武德四年析置豐安縣。貞觀四年於回樂境置回州,以豐安隸回州。十三年州廢,省豐安。有溫泉鹽池;有特進渠,溉田六百頃,長慶四年詔開。靈武,上。懷遠,緊。武德六年廢豐州,省九原、永豐二縣入焉,隋九原郡也。有鹽池三:曰紅桃、武平、河池。保靜。上。本弘靜,神龍元年曰安靜,至德元載更名。



 威州郡闕。中。本安樂州。初,吐谷渾部落自涼州徙於鄯州,不安其居,又徙於靈州之境,咸亨三年以靈州之故鳴沙縣地置州以居之。至德後沒吐蕃。大中三年收復,更名。光啟三年徙治涼州鎮為行州。縣二:鳴沙,上。武德二年置會州,貞觀六年州廢,更置環州,以大河環曲為名。九年州廢,還隸靈州。神龍中為默啜所寇,移治故豐安城。大中三年復得故縣。溫池。上。本隸靈州,神龍元年置,大中四年來屬。有鹽池。



 雄州,在靈州西南百八十里。中和元年徙治承天堡為行州。



 警州,本定遠城,在靈州東北二百里,先天二年,朔方大總管郭元振置。其後為上縣,隸靈州。景福元年,靈威節度使韓遵表為州。羊馬城幅員十四里,信安王禕所築。



 會州會寧郡,上。本西會州,武德二年以平涼郡之會寧鎮置。貞觀八年以足食故更名粟州,是年又更名。土貢:駝毛褐、野馬革、覆鞍氈、鹿舌、鹿尾。戶四千五百九十四,口二萬六千六百六十。縣二:有新泉軍,開元五年廢為守捉。會寧,上。本涼川,武德二年更名。開元四年別置涼川縣,九年省。有黃河堰,開元七年,刺史安敬忠築,以捍河流。有河池,因雨生鹽。東南有會寧關。烏蘭。上。武德九年置。西南有烏蘭關。



 鹽州五原郡,下都督府。本鹽川郡。唐初沒梁師都。武德元年僑治靈州。貞觀元年州省,以縣隸靈州,二年,師都平,復置州。天寶元年更郡曰五原。貞元三年沒吐蕃,九年復城之。土貢:鹽山、木瓜、牛。戶二千九百二十九,口萬六千六百六十五。縣二:有府一,曰鹽川。有保塞軍,貞元十九年置。五原,上。有烏池、白池、細項池、瓦窯池鹽。白池。上。本興寧,貞觀元年與州同省,二年復置。景龍三年更名。



 夏州朔方郡,中都督府。土貢:氈、角弓、酥、拒霜薺。戶九千二百一十三,口五萬三千一十四。縣三:有府二,曰寧朔、順化。朔方,上。本鷿錄,貞觀三年更名。貞元七年開延化渠,引烏水入庫狄澤,溉田二百頃。有鹽池二。有天柱軍,天寶十四載置,寶應元年廢。長慶四年,節度使李祐築烏延、宥州、臨塞、陰河、陶子等城於蘆子關北,以護塞外。有木瓜嶺。靜德,中下。貞觀七年隸北開州,八年曰化州,十三年州廢。寧朔。中下。武德六年置南夏州。貞觀二年州廢,縣省入朔方,五年復置,來屬。長安二年省。開元四年又置,九年省,其後又置。



 綏州上郡,下。本雕陰郡地。唐初沒梁師都。武德三年以歸民於延州豐林縣僑置,六年徙治延川境,七年徙治魏平。貞觀二年,師都平,歸治上縣。天寶元年更郡名。士貢:胡女布、蠟燭。戶萬八百六十七,口八萬九千一百一十二。縣五:有府四,曰伏洛、義合、萬古、大斌。龍泉,中。本上縣,天寶元年更名。延福,中下。武德六年析置北吉州,並置歸義、洛陽二縣;又析置羅州,並置石羅、關善、萬福三縣;又析置匡州,並置安定、源泉二縣。貞觀二年州、縣皆廢。綏德,中下。武德二年置。六年析置雲州,並置信義、淳義二縣;又析置龍州,並置風鄉、義良二縣。貞觀二年州、縣皆廢。綏德,中下。武德三年置魏州,並置安故、安泉二縣。貞觀二年州廢,省安故、安泉。西南有魏平關。大斌。中下。武德七年徙治魏平城,取「稽胡懷化,文武雜半」以名。



 銀州銀川郡,下。貞觀二年析綏州之儒林、真鄉置。土貢:女稽布。戶七千六百二,口四萬五千五百二十七。縣四:儒林,中。東北有無定河。真鄉,中下。西北有茹盧水。開光,中。本隸綏州,貞觀二年置,八年隸柘州,十三年州廢,來屬。撫寧。中下。本隸綏州,貞觀八年來屬。



 宥州寧朔郡,上。調露元年,於靈、夏南境以降突厥置魯州、麗州、含州、塞州、依州、契州,以唐人為刺史,謂之六胡州。長安四年並為匡、長二州。神龍三年置蘭池都督府,分六州為縣。開元十年復置魯州、麗州、契州、塞州。十年平康待賓,遷其人於河南及江、淮。十八年復置匡、長二州。二十六年還所遷胡戶置宥州及延恩等縣,其後僑治經略軍。至德二載更郡曰懷德。乾元元年復故名。寶應後廢。元和九年於經略軍復置,距故州東北三百里。十五年徙治長澤,為吐蕃所破。長慶四年,節度使李祐復奏置。土貢:氈。戶七千八十三,口三萬二千六百五十二。縣二:延恩,中。開元二十六年以故匡州地置;又以故塞門縣地置懷德縣,以故蘭州之長泉縣地置歸仁縣。寶應後皆省。元和九年復置延恩。有經略軍,在榆多勒城,天寶中王忠嗣奏置。長澤。中下。本隸夏州,貞觀七年置長州,十三年州廢,隸夏州,元和十五年來屬。有胡洛鹽池。



 麟州新秦郡,下都督府。開元十二年析勝州之連谷、銀城置,十四年廢,天寶元年復置。土貢:青他鹿角。戶二千四百二十八,口萬九百三。縣三:新秦,中。開元二年置,七年又置鐵麟縣,十四年州廢,皆省。天寶元年復置新秦。連穀。中下。貞觀八年以隋連穀戍置。銀城。中下。貞觀二年置,四年隸銀州,八年隸勝州。



 勝州榆林郡,下都督府。武德中沒梁師都。師都平,復置。土貢:胡布、青他鹿角、芍藥、徐長卿。戶四千一百八十七,口二萬九百五十二。縣二:有義勇軍。榆林,中下。有隋故榆林宮。東有榆林關,貞觀十三年置。河濱。中下。貞觀三年置,以縣置雲州,四年曰威州,八年州廢,來屬。東北有河濱關,貞觀七年置。



 豐州九原郡,下都督府。貞觀四年以降突厥戶置,不領縣。十一年州廢,地入靈州。二十三年復置。土貢:白麥、印鹽、野馬胯革、駝毛褐、氈。戶二千八百一十三,口九千六百四十一。縣二:九原,中下。永徽四年置。有陵陽渠,建中三年浚之以溉田,置屯,尋棄之。有咸應、永清二渠,貞元中,刺史李景略開,溉田數百頃。永豐。中下。永徽元年置。麟德元年別置豐安縣,天寶末省。東受降城,景雲三年,朔方軍總管張仁願築三受降城。寶歷元年,振武節度使張惟清以東城濱河,徙置綏遠烽南。中受降城,有拂雲推祠。接靈州境有關,元和九年置;又有橫塞軍,本可敦城,天寶八載置,十二載廢。西二百里大同川有天德軍,大同川之西有天安軍,皆天寶十二載置。天德軍,乾元後徙屯永濟柵,故大同城也。元和九年,宰相李吉甫奏修復舊城。北有安樂戍。西受降城。開元初為河所圮,十年,總管張說於城東別置新城。北三百里有鷿鵜泉。



 單于大都護府,本云中都護府,龍朔三年置,麟德元年更名。土貢:胡女布、野馬胯革。戶二千一百五十五,口六千八百七十七。縣一:金河。中。天寶四年置。本後魏道武所都。有雲伽關,後廢,大和四年復置。



 安北大都護府,本燕然都護府,龍朔三年曰瀚海都督府,總章二年更名。開元二年治中受降城,十年徙治豐、勝二州之境,十二年徙治天德軍。土貢:野馬胯革。戶二千六,口七千四百九十八。縣二:陰山,上。天寶元年置。通濟。上。



 鎮北大都護府。土貢:筼牛尾。縣二。大同,上。長寧。上。



 右關內採訪使,以京官領。



\end{pinyinscope}