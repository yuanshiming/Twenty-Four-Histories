\article{列傳第四十七 魏韋郭}

\begin{pinyinscope}

 魏元忠,宋州宋城人。為太學生,跌蕩少檢,久不調,盩厔人江融曉兵術名。,元忠從之游,盡傳所學。儀鳳中,吐蕃數盜邊,元忠上封事洛陽宮,言命將用兵之要曰:



 天下之柄有二,文武而已,至制勝禦人,其道一也。今言武者先騎射,不稽之權略;言文者首篇章,不取之經綸。臣觀魏、晉、齊、梁才固不乏,然何益治亂哉!養由基射能穿札,不止鄢陵之奔,陸機識能辨亡,無救河橋之敗,斷可見已。



 夫才生於世,世實須才。何世而不生才?何才而不資世?故物有不求,未有無物之歲;士有不用,未有無士之時也。志士在富貴與賤貧,皆思立功名以傳於後,然知己難而所遇罕。士之懷琬琰就煨塵、抱棟幹困溝壑者,悠悠之人直睹此士之貧賤,安知其方略哉!故漢拜韓信,舉軍驚笑;蜀用魏延,群臣觖望。此富貴者易為善,貧賤者難為功也。昔漢文帝不知魏尚賢而囚之,知李廣才而不用,乃嘆其生不逢時。夫以廣之才,天下無雙,時方歲事匈奴,而卒不任。故近不知尚、廣之賢,而遠想廉頗、李牧,馮唐是以知其有而不能用也。此身為時主所知,不得盡其才也。晉羊祜謀舉吳,賈充、荀勖沮之,祜嘆曰:「天下事不如意十常七八。」以二人不同,終不大舉。此據立功之地,而不獲展其志也。布衣之人,懷奇抱策,而望朝奏夕召,豈易得哉?臣願歷訪文武五品以上,得無有智如羊祜、武如李廣而不得騁其才者乎?使各言其志,毋令久失職。



 又言:



 人無常俗,政有治亂;軍無常勝,將有能否。兵為王者大事,存亡系焉,將非其任,則殄人敗國。齊段孝玄有言:「持大兵如擎盤水,一致蹉跌,求止可得哉」」周亞夫堅壁以挫吳、楚,司馬懿閉營而困諸葛亮,此皆全軍制勝,不戰而卻敵。是知大將臨戎,以智為本。今之用人,類將家子,或死事孤兒,進非幹略,雖竭力盡誠,不免於傾敗,若之何用之?且建功者,言其所濟,不言所來;言其所能,不言所藉。若陳湯、呂蒙、馬隆、孟觀悉出貧賤,而勛伐甚高,不聞其家世將帥也。故陰陽不和,揠士為相;蠻貊不廷,擢校為將。今以四海之廣,億兆之眾,豈無卓越之士?臣恐未之思乎!



 又賞者禮之基,罰者刑之本。禮崇則謀夫竭其能,賞厚則義士輕其死,刑正故君子勖其心,罰重則小人懲其過。賞罰者軍國之綱紀,政教之藥石。吐蕃本非強敵,而薛仁貴、郭待封至棄甲喪師,脫身以免。國家寬政,罪止削除,網漏吞舟,何以過此。雖陛下顧收後效,然朝廷所少,豈此一二人乎?夫賞不勸,謂之止善,罰不懲,謂之縱惡。臣誠疏賤,乾非其事,豈欲間陛下君臣生薄厚哉?正以刑賞一虧,百年不復。故國無賞罰,雖堯、舜不能為。今罰既不行,賞復難信,故議者皆謂比日征行,虛立賞格,而無其實。蓋忘大體之臣恐賚勛庸,竭府庫,留意錐刀,以為益中國,所謂惜毫厘失千里者也。且黔首雖微,不可以欺,安有寓不信之令,設虛賞之格乎?自蘇定方平遼東,李勣破平壤,賞既不行,勛亦淹廢,歲月紛淆,真偽相錯。臣以吏不奉法,慢自京師,偽勛所由,主司過也,其則不遠,近在尚書省中。然未聞斬一臺郎、戮一令史,使天下知之。陛下何照遠而不照近哉?神州化首,文昌政本,治亂攸在,臣故冒死而言。夫明鑒所以照形,往事所以知今,臣請借近以為諭:貞觀中,萬年尉司馬玄景舞文飾智,以邀乾沒,太宗棄之都市;後征高麗,總管張君乂不進擊賊,斬之旗下。臣以為偽勛之罪,多於玄景;仁貴等敗,重於君乂。使早誅之,則諸將豈復有負哉?慈父多敗子,嚴家無格虜。且人主病不廣大,人臣病不節儉,臣恐陛下病之於不廣大,過在於慈父,斯日月一蝕也。



 又今將吏貪暴,所務口馬、財利,臣恐戎狄之平,未可旦夕望也。凡人識不經遠,皆言吐蕃戰,前隊盡,後隊方進,甲堅騎多,而山有氛瘴,官軍遠入,前無所獲,不積穀數百萬,無大舉之資。臣以為吐蕃之望中國,猶孤星之對太陽,有自然之大小、不疑之明暗,夷狄雖禽獸,亦知愛其性命,豈肯前盡死而後進哉!由殘迫其人,非下所願也。必其戰不顧死,則兵法許敵能鬥,當以智算取之,何憂不克哉!向使將能殺敵,橫尸蔽野,斂其頭顱以為京觀,則此虜聞官軍鐘鼓,望塵卻走,何暇前隊皆死哉!自仁貴等覆師喪氣,故虜得跳梁山谷。



 又師行必藉馬力,不數十萬,不足與虜爭。臣請天下自王公及齊人掛籍之口,人稅百錢;又弛天下馬禁,使民得乘大馬,不為數限,官籍其凡,勿使得隱。不三年,人間畜馬可五十萬,即詔州縣以所稅口錢市之,若王師大舉,一朝可用。且虜以騎為強,若一切使人乘之,則市取其良,以益中國,使得漸耗虜兵之盛,國家之利也。



 高宗善之,授秘書省正字,直中書省,仗內供奉。



 遷監察御史。帝嘗從容曰:「外以朕為何如主?」對曰:「周成、康,漢文、景也。」「然則有遺恨乎?」曰:「有之。王義方一世豪英,而死草萊。議者謂陛下不能用賢。」帝曰:「我適用之,聞其死,顧已無及。」元忠曰:「劉藏器行副於才,陛下所知,今七十為尚書郎。徒嘆彼而又棄此。」帝默然慚。



 遷殿中侍御史。徐敬業舉兵,詔元忠監李孝逸軍。至臨淮,而偏將雷仁智為賊敗,孝逸懼其鋒,按兵未敢前。元忠曰:「公以宗室將,天下安危系焉。海內承平久,聞狂狡竊發,皆傾耳翹心以待其誅。今軍不進,使遠近解情,萬有一朝廷以他將代公,且何辭?」孝逸然之,乃部分進討。時敬業保下阿谿,弟敬猷屯淮陰,咸請「先擊下阿,下阿敗,淮陰自破。今淮陰急,敬業必救,是敵在腹背也。」元忠曰:「不然。賊勁兵盡守下阿,利在一決,茍有負,則大事去矣。敬酋博徒不知戰,且其兵寡易搖,大軍臨之,勢宜克。敬業畏直搗江都,必將邀我中路,吾今乘勝進,又以逸擊勞,破之必矣。譬之逐獸,弱者先禽。今舍必禽之弱,而趨難敵之強,非計也。」孝逸乃引兵擊淮陰,敬猷脫身遁,遂進擊敬業,平之。還。授司刑正。



 遷洛陽令。陷周興獄當死,以平揚、楚功,得流。歲餘,為御史中丞,復為來俊臣所構。將就刑,神色不動,前死者宗室子三十餘,尸相枕藉於前,元忠顧曰:「大丈夫行居此矣。」俄敕鳳閣舍人王隱客馳騎免死,傳聲及於市,諸囚歡叫,元忠獨堅坐,左右命起,元忠曰:「未知實否。」既而隱客至,宣詔已,乃徐謝,亦不改容。流費州。復為中丞。歲餘,陷侯思止獄,仍放嶺南。酷吏誅,人多訟元忠者,乃召復舊官。因侍宴,武后曰:「卿累負謗鑠,何邪?」對曰:「臣猶鹿也,羅織之吏如獵者,茍須臣肉為之羹耳,彼將殺臣以求進,臣顧何辜?」



 聖歷二年,為鳳閣侍郎、同鳳閣鸞臺平章事,俄檢校並州長史、天兵軍大總管,以備突厥。遷左肅政臺御史大夫,兼檢校洛州長史,治號威明。張易之家奴暴百姓,橫甚,元忠笞殺之,權豪憚服。俄為隴右諸軍大使,以討吐蕃;又為靈武道行軍大總管御突厥。元忠馭軍持重,雖無赫然功,而亦未嘗敗。



 中宗在東宮,為檢校左庶子。時二張勢傾朝廷,元忠嘗奏曰:「臣承先帝之顧,且受陛下厚恩,不能徇忠,使小人在君側,臣之罪也。」易之等恨怒,因武后不豫,即共譖元忠與司禮丞高戩謀挾太子為耐久朋,遂下制獄。詔皇太子、相王及宰相引元忠等辨於廷,不能決。昌宗乃引張說為證,說初偽許之,至是迫使言狀,不應,後又促之,說曰:「臣不聞也。」易之等遽曰:「說與同逆。說曩嘗謂元忠為伊、周。夫伊尹放太甲,周公攝王位。此反狀明甚。」說曰:「易之、昌宗安知伊、周,臣乃能知之。伊尹、周公,歷古以為忠臣,陛下不遣學伊、周,將何效焉?」說又曰:「臣知附易之朝夕可宰相,從元忠則族滅。今不敢面欺,懼元忠之冤。」後寤其讒,然重違易之,故貶元忠高要尉。



 中宗復位,召為衛尉卿、同中書門下三品。不閱旬,遷兵部尚書,進侍中。武后崩,帝居喪,軍國事委元忠裁可,拜中書令,封齊國公。神龍二年,為尚書右僕射,知兵部尚書,當朝用事,群臣莫敢望。謁告上塚,詔宰相諸司長官祖道上東門,賜錦袍,給千騎四人侍,賜銀千兩。元忠到家,於親戚無所賑施。及還,帝為幸白馬寺迎勞之。



 安樂公主私請廢太子,求為皇太女,帝以問元忠,元忠曰:「公主而為皇太女,駙馬都尉當何名?」主恚曰:「山東木強安知禮?阿母子尚為天子,我何嫌?」宮中謂武後為阿母子,故主稱之。元忠固稱不可,自是語塞。



 武三思用事,京兆韋月將、渤海高軫上書言其惡,帝搒殺之,後莫敢言。王同皎謀誅三思,不克,反被族。元忠居其間,依違無所建明。初,元忠相武後,有清正名,至是輔政,天下傾望,冀干正王室,而稍憚權幸,不能賞善罰惡,譽望大減。陳郡男子袁楚客者以書規之曰:



 今皇帝新服厥德,任官惟賢才,左右惟其人,因以布大化,充古誼,以正天下。君侯安得事循默哉?茍利社稷,專之可也。夫安天下者先正其本,本正則天下固,國之興亡系焉。太子天下本,譬之大樹,無本則枝葉零悴,國無太子,朝野不安。儲君有次及之勢,故師保教以君人之道,用蘊崇其德,所以重天下也。今皇子既長,未定嫡嗣,是天下無本。天下無本,猶樹而亡根,枝葉何以存乎?願君侯以清宴之間言於上,擇賢而立之,此安天下之道。曠而不置,朝廷一失也。



 女有內則,男有外傅,豈相濫哉?幕府者,丈夫之職。今公主並開府置吏,以女處男職,所謂長陰抑陽也,而望陰陽不愆、風雨時若,得乎?此朝廷二失也。



 今度人既多,緇衣半道,不本行業,專以重寶附權門,皆有定直。昔之賣官,錢入公府,今之賣度,錢入私家。以茲入道,徒為游食。此朝廷三失也。



 唯名與器,不可以假人。故曰:「天工,人其代之。」夫代天,非材不可也。代非其人,必失天意。失天意而無患禍,未之有也。今倡優之輩,因耳目之好,遂授以官,非輕朝廷、亂正法邪?人君無私,私怒害物,私賞費財,況私人以官乎?此朝廷四失也。



 賢者邦家之光,任之致治,棄之生亂。近詔博求多士,雖有好賢之名,無得賢之實。蓋有司選士,非賄即勢,上失天心,下違人望,非為官擇吏,乃為人擇官。葛洪有言:「舉秀才,不知書;察孝廉,濁如泥;高第賢良吝如。此朝廷五失也。



 閹豎者,給宮掖掃除事,古以奴隸畜之。中古以來,大道乖喪,疏賢哲,親近習,乃委之以事,授之以權。故豎刁亂齊,伊戾敗宋。君側之人,眾所畏懼,所謂鷹頭之蠅、廟垣之鼠者也。後漢時用事尤甚,晚節卒亂天下。今大君中興,獨有閹豎坐升班秩,既無正闕,率授員外,乃盈千人,綰青紫,耗府藏。前事之驗,後事之師。此朝廷六失也。



 古者茅茨採椽,以儉約遺子孫,所以愛力也。今公主所賞傾庫府,所造皆官供,其疏築臺沼,崇峙觀廡,山無本石,木無近產,造之終歲,功用不絕。夫為君所以養人,非以害人,今外戚不助養而反害之,是使人主受謗天下。此朝廷七失也。



 官以安人,非以害於人也。先王欲人治必選材,欲人安必省事,此誠同天下憂也。人有樂,君共之,君有樂,人慶之,可謂同樂矣。如此,則上下無間,而均一體也。今天下困窮,州牧、縣宰,非以選進,割剝自私,人不聊生,是下有憂而上不恤也。而更員外置官,非助桀歟?夫人情自以員外吏,恐下不己畏也,必峻法懼之;恐財不己奉也,必枉道奪之。欲不亂,可得哉?古語有之,十羊九牧,羊既不得食,人亦不得息。《書》曰:「官不必備,惟其人。」此言正員猶難其備,況員之外乎!此朝廷八失也。



 政出多門,大亂之漸。近封數夫人,皆先帝宮嬪。以為備內職,則不當知外;不備內職,則自可處外。而令出入禁掖,使內言必出,外言必入,固將弄君之法,縱而不禁,非所以重宗廟、固國家。孔子曰:「彼婦之口,可以出走;彼婦之謁,可以死敗。」此朝廷九失也。



 不以道事其君者,所以危天下也,危天下之臣不可不逐,安天下之臣不可不任。今有引鬼神、執左道以惑主者,托鬼神為難知,故致其詐,而據非才之地,食非德之祿,此國盜也。《傳》曰:「國將興,聽於民,將亡,聽於神。」今幾聽於神乎?此朝廷十失也。



 君侯不正,誰與正之?



 元忠得書益慚。以三思專權,思有以誅之。會節愍太子起兵,與聞其謀。太子已誅三思,引兵走闕下,元忠子太僕少卿升遇於永安門,太子脅使從戰,已而被殺。議者未辨逆順,元忠誦言曰:「既誅賊謝天下,雖死鼎鑊所甘心。惟皇太子沒為恨耳。」帝以其嘗有功,且為高宗、武后素所禮,置不問。宗楚客、紀處訥大怒,固請夷其族,不聽。元忠不自安,上政事及國封,詔以特進、齊國公致仕,朝朔望。楚客等引右衛郎將姚廷筠為御史中丞,暴奏反狀,繇是貶渠州司馬。楊再思、李嶠皆希順楚客,傅致元忠罪,唯蕭至忠議當申宥之。楚客復遣再思與冉祖雍奏元忠緣逆不宜處內地,監察御史袁守一固請行誅,遂貶務川尉。守一又劾:「天后嘗不豫,狄仁傑請陛下監國,元忠止之,此其逆久萌。」帝謂楊再思曰:「守一非是。事君者一其心,豈有上少疾遽異論哉?朕未見元忠過也。」



 元忠至涪陵,卒,年七十餘。景龍四年,贈尚書左僕射、齊國公、本州刺史。睿宗詔陪葬定陵,以實封一百五十戶賜其子晃。開元六年,謚曰貞。



 元忠始名真宰,以諸生見高宗,高宗慰遣,不知謝即出,儀舉自安,帝目送謂薛元超曰:「是子未習朝廷儀,然名不虛謂,真宰相也。」避武后母諱,改今名。



 韋安石,京兆萬年人。曾祖孝寬,為周大司空、鄖國公。祖津,隋大業末為民部侍郎,與元文都等留守洛,拒李密,戰上東門,為密禽。後王世充殺文都而津獨免,密敗,復歸洛。世充平,高祖素與津善,授諫議大夫,檢校黃門侍郎,陵州刺史,卒。父琬,仕為成州刺史。



 安石舉明經,調乾封尉,雍州長史蘇良嗣器之。永昌元年,遷雍州司兵參軍。良嗣當國,謂安石曰:「大才當大用,徒勞州縣可乎?」薦於武後,擢膳部員外郎,遷並州司馬,有善政,後手制勞問,陟拜德、鄭二州刺史。安石性方重,不茍言笑,其政尚清嚴,吏民尊畏。



 久視中,遷文昌右丞,以鸞臺侍郎同鳳閣鸞臺平章事,兼太子左庶子,仍侍讀,尋知納言事。時二張及武三思寵橫,安石數折辱之。會侍宴殿中,易之引蜀商宋霸子等博塞後前,安石跪奏「商等賤類,不當戲殿上。」顧左右引出,坐皆失色,後以安石辭正,改容慰勉。鳳閣侍郎陸元方自以為不及,退告人曰:「韋公真宰相。」後嘗幸興泰宮,議趨疾道,安石曰:「此道板築所成,非自然之固。千金子且誡垂堂,況萬乘可輕乘危哉?」後為回輦。長安二年,同鳳閣鸞臺三品,俄又知納言,檢校揚州大都督府長史。神龍元年,罷政事,俄復同三品,遷中書令,兼相王府長史,封鄖國公,賜封三百戶,加特進,為侍中。中宗與韋後以正月望夜幸其第,賚賜不貲。帝嘗幸安樂公主池,主請御船,安石曰:「御輕舟,乘不測,非帝王事。」乃止。



 睿宗立,授太子少保,改封郇國,復為侍中、中書令,進開府儀同三司。太平公主有異謀,欲引安石,數因其婿唐晙邀之,拒不往。帝一日召安石曰:「朝廷傾心東宮,卿胡不察?」對曰:「太子仁孝,天下所稱,且有大功。陛下今安得亡國語?此必太平公主計也。」帝矍然曰:「卿勿言,朕知之。」主竊聞,乃構飛變,欲訊之,賴郭元振保護,免。遷尚書右僕射兼太子賓客、同三品,俄罷政事,留守東都。



 會妻薛怨婿婢,笞殺之,為御史中丞楊茂謙所劾,下遷蒲州刺史,徙青州。安石在蒲,太常卿姜皎有所請,拒之。皎弟晦為中丞,以安石昔相中宗,受遺制,而宗楚客、韋溫擅削相王輔政語,安石無所建正,諷侍御史洪子輿劾舉,子輿以更赦不從。監察御史郭震奏之,有詔與韋嗣立、趙彥昭等皆貶,安石為沔州別駕。皎又奏安石護作定陵,有所盜沒,詔籍其贓。安石嘆曰:「祗須我死乃已。」發憤卒,年六十四。開元十七年,贈蒲州刺史。天寶初,加贈左僕射、郇國公,謚文貞。二子:陟,、斌。



 陟字殷卿,與弟斌俱秀敏異常童。安石晚有子,愛之。神龍一年,安石為中書令,陟甫十歲,授溫王府東閣祭酒、朝散大夫。風格方整,善文辭,書有楷法,一時知名士皆與游。開元中居喪,以父不得志歿,乃與斌杜門不出八年。親友更往敦曉,乃強調為洛陽令。宋璟見陟嘆曰:「盛德遺範,盡在是矣。」累除吏部郎中,中書令張九齡引為舍人,與孫逖、梁涉並司書命,時號得才。



 遷禮部侍郎。陟於鑒裁尤長。故事,取人以一日試為高下。陟許自通所工,先就其能試之,已乃程考,由是無遺材。遷吏部侍郎,選人多偽集,與正調相冒,陟有風採,擿辨無不伏者,黜正數百員,銓綜號為公平。然任威嚴,或至詈詰,議者訾其峻。又自以門品可坐階三公,居常簡貴,視僚黨涘然;其以道誼合,雖後進布衣與均禮。



 李林甫惡其名高,恐逼己,出為襄陽太守,徙河南採訪使,以判官員錫善訊覆,支使韋元甫工書奏,時號「員推韋狀」,陟皆倚任之。俄襲郇國公,坐事貶守鐘離、義陽,後為河東太守。以失職,內怏怏,乃毀廉隅,頗餉謝權幸欲自結。天寶十二載,入考華清宮,楊國忠忌其才,謂拾遺吳豸之曰:「子能發陟罪乎?吾以御史相處。」豸之乃劾陟饋遺事,國忠又使甥婿韋元志左驗,陟惶悸,賄吉溫求救,由是俱得罪,陟貶桂嶺尉,坐不行,徙平樂。會安祿山陷洛陽,弟斌沒賊,國忠欲構陟與賊通,密諭守吏,令脅陟使憂死,州豪傑共說曰:「昔張說被竄,匿陳氏以免。今若詔書下,誰敢庇公?願公乘扁舟遁去,事寧乃出,不亦美乎?」陟慨然曰:「命當爾,其敢逃刑?」因謝遣,堅臥不出。



 歲餘,肅宗即位,起為吳郡太守,使者趣追,未至,會永王兵起,委陟招諭,乃授御史大夫、江東節度使。與高適、來瑱會安州,陟曰:「今中原未平,江淮騷離,若不齋盟質信,以示四方,知吾等協心戮力,則無以成功。」乃推瑱為地主,為載書,登壇曰:「淮西節度使瑱、江東節度使陟、淮南節度使適,銜國威命,糾合三垂,翦除兇慝,好惡同之,毋有異志。有渝此盟,墜命亡族,罔克生育。皇天后土,祖宗明神,實鑒斯言。」辭旨慷慨,士皆隕泣。



 永王敗,帝趣陟赴鳳翔。初,季廣琛從永王亂,非其本謀,陟表廣琛為歷陽太守,慰安之。至是,恐廣琛有後變,乃馳往諭詔恩釋其疑,而後趣召。帝雅聞陟名,欲倚以相,及是遷延,疑有顧望意,止除御史大夫。會杜甫論房琯,詞意迂慢,帝令陟與崔光遠、顏真卿按之,陟奏:「甫言雖狂,不失諫臣體。」帝繇是疏之。富平人將軍王去榮殺其縣令,帝將宥之,陟曰:「昔漢高帝約法,殺人者死。今陛下殺人者生,恐非所宜。」時朝廷尚新,群臣班殿中,有相吊哭者,帝以陟不任職,用顏真卿代之,更拜吏部尚書。久之,宗人伐墓柏,坐不相教,貶絳州刺史。還授太常卿,呂諲入輔,薦為禮部尚書、東京留守。史思明逼伊、洛,李光弼議守河陽,陟率東京安屬入關避之,詔授吏部尚書,令就保永樂,以圖收復。卒,年六十五,贈荊州大都督。



 陟早有名,而為林甫、國忠擯廢。及肅宗擇相,自謂必得,以後至不用。任事者皆新進,望風憚之,多言其驕倨。及入關,又不許至京師。鬱鬱不得志,成疾,且卒,嘆曰:「吾道窮於此乎!」性侈縱,喜飾服馬,侍兒閹童列左右常數十,侔於王宮主第。窮治饌羞,擇膏腴地藝穀麥,以鳥羽擇米,每食視庖中所棄,其直猶不減萬錢,宴公侯家,雖極水陸,曾不下箸。常以五採箋為書記,使侍妾主之,以裁答,受意而已,皆有楷法,陟唯署名,自謂所書「陟」字若五朵雲,時人慕之,號「郇公五雲體」。然家法脩整,敕子允就學,夜分視之,見其勤,旦日問安,色必怡;稍怠則立堂下不與語。雖家僮數十,然應門賓客,必允主之。



 永泰元年,贈尚書左僕射。太常博士程皓議謚「忠孝」,顏真卿以為許國養親不兩立,不當合二行為謚,主客員外郎歸崇敬亦駁正之。右僕射郭英乂無學術,卒用太常議云。



 斌,父為相時授太子通事舍人。少脩整,好文藝,容止嚴峭,有大臣體,與陟齊名。開元中,薛王業以女妻之,遷秘書丞。天寶中,為中書舍人,兼集賢院學士,改太常少卿。李林甫構韋堅獄,斌以宗累,貶巴陵太守,移臨汝。久之,拜銀青光祿大夫,列五品。時陟守河東,而從兄由為右金吾衛將軍,絳為太子少師,四第同時列戟,衣冠罕比者。祿山陷洛陽,斌為賊得,署以黃門侍郎,憂憤卒。乾元元年,贈秘書監。



 斌天性質厚,每朝會,不敢離立笑言。嘗大雪,在廷者皆振裾更立,斌不徙足,雪甚,幾至靴,亦不失恭。



 子況,少隱王屋山,孔述睿稱之,及述睿以諫議大夫召,薦況為右拾遺,不拜。未幾,以起居郎召,半歲,輒棄官去,徙家龍門。除司封員外郎,稱疾固辭。元和初,授諫議大夫,勉諭到職,數月,乞骸骨,以太子左庶子致仕,卒。況雖世貴,而志沖遠,不為聲利所遷,當時重其風操。



 叔夏,安石兄。通禮家學。叔父太子詹事琨嘗曰:「而能繼漢丞相業矣。」擢明經第,歷太常博士。高宗崩,恤禮亡缺,叔夏與中書舍人賈大隱、博士裴守真禋定其制,擢春官員外郎。武后拜治,享明堂,凡所沿改,皆叔夏、祝欽明、郭山惲等所裁討。每立一議,眾咨服之。累遷成均司業。後又詔:「五禮儀物,司禮博士有所脩革,須叔夏、欽明等評處,然後以聞。」進位春官侍郎。中宗復位,轉太常少卿,為建立廟社使,進銀青光祿大夫,累封沛郡公,國子祭酒。卒,贈兗州都督、脩文館學士,謚曰文。子縚。



 縚,開元時歷集賢修撰、光祿卿,遷太常。



 唐興,禮文雖具,然制度時時繆缺不倫。至顯慶中,許敬宗建言:「籩豆以多為貴,宗廟乃旂於天,請大祀十二、中祀十、小祀八。大祀、中祀、簠、簋、、俎皆一,小祀無。」詔可。二十三年,赦令以籩豆之薦,未能備物,宜詔禮官學士共議以聞。縚請「宗廟籩豆皆加十二。」又言「郊奠,爵容止一合,容小則陋,宜增大之。」



 兵部侍郎張均、職方郎中韋述議曰:「《禮》:『天之所生,地之所長,茍可薦者,莫不咸在。』聖人知孝子之情深,而物類無限,故為之節,使物有品,器有數,貴賤差降,不得相越。周制:王,食用六谷,膳用六牲,飲用六清,羞用百有二十品,珍用八物,醬用百有二十甕,而以四籩、四豆供祭祀。此祀與賓客豐省不得同,舊矣。且嗜好燕私之饌,與時而遷,故聖人一約以禮。雖平生所嗜,非禮則不薦;所惡,是禮則不去。屈建命去祥祭之芰曰:『祭典有之,不羞珍異,不陳庶侈。』此則禮外之食,前古不薦。今欲以甘旨肥濃皆充於祭,茍逾舊制,其何極焉。雖籩豆有加,不能備也。若曰以今之珍,生所嗜愛,求神無方,是簠、簋可去,而盤、盂、杯、案當御矣;韶、瑀可抵,而箜篌、笙、笛應奏矣。且自漢以來,陵有寢宮,歲時朔望,薦以常饌,固可盡孝子之心。至宗廟法享,不可變古從俗。有司所承,一升爵,五升散。《禮》:凡宗廟,貴者以爵,賤者以散,此貴小賤大,以示節儉。請如故。」



 太子賓客崔沔曰:「古者,有所飲食,必先嚴獻,未化火,則有毛血之薦,未麴糵,則有玄酒之奠。至後王,作酒醴、用犧牲,故有三牲、八簋、五齊、九獻。然神尚玄,可存而不可測也;祭主敬,可備而不可廢也。蓋薦貴新,味不尚褻,雖曰備物,猶有節制存焉。鉶、俎、籩、豆、簠、簋、尊、罍,周人時饌也,其用通於燕享賓客,周公乃與毛血玄酒共薦。晉中郎盧諶家祭,皆晉日食,則當時之食,不可闕於祀已。唐家清廟時享,禮饌備進,周法也;園寢上食,時膳具陳,漢法也。職貢助祭,致遠物也;有新必薦,順時令也。苑囿躬稼所入,搜田親發所中,皆因宜以薦,薦而後食。則濃腴鮮美盡在矣。又敕有司著於令,不必加籩豆之數也。大凡祭器,視物所宜。故大羹,古饌也,盛以,,古器也;和羹,時饌也,盛以鉶,鉶,時器也。有古饌而用時器者,則毛血於盤,玄酒於尊。未有進時饌用古器者,古質而今文,有所不稱也。雖加籩豆十二,未足盡天下之美,而措諸廟,徒以近侈而見訾抵。臣聞墨家者流,出於清廟,是廟貴儉不尚奢也。」禮部員外郎楊仲昌、戶部郎中陽伯成、左衛兵曹參軍劉秩等,請如舊便。宰相白奏,玄宗曰:「朕承祖宗休德,享祀粢盛,實貴豐潔。有如不應於法,亦不敢用。」乃詔太常,擇品味可增者稍加焉。縚又請室加籩、豆各六,每四時以新果珍饔實之。制「可」。又詔:「獻爵視藥升所容,以合古。」



 二十三年,詔書服紀所未通者,令禮官學士詳議。縚上言:「《禮》《喪服》:舅,緦麻三月。從母,小功五月,《傳》曰:『何以小功,以名加也。』而堂姨、舅母,恩所不及焉。外祖父母,小功五月,《傳》曰:『何以小功,以尊加也。』舅,緦麻三月,皆情親而屬疏也。外祖正尊,服同從母;姨、舅一等,而有輕重;堂姨、舅親未疏,不相為服;親舅母不如同爨。其亦古意有所未暢。且外祖小功,此為正尊,請進至大功;姨、舅儕親,服宜等,請進舅至小功;堂姨舅以疏降親舅從母一等;親舅母古未有服,請從袒免。」



 於是韋述議曰:「自高祖至玄孫並身謂之九族。由近及遠,差其輕重,遂為五服。《傳》曰:『外親服皆緦。』鄭玄曰:『外親之服異姓,正服不過緦。』外祖父母小功,以尊加;從母小功,以名加;舅、甥、外孫、中外昆弟,皆緦。以匹言之,外祖則祖也,舅則伯叔也,父母之恩不殊,而獨殺於外者有以也。禽獸知母而不知父,野人則父母等,都邑之士則知尊禰,大夫則知尊祖,諸侯及太祖,天子及始祖。聖人究天道,厚祖禰,系姓族,親子孫,則母黨之於本族,不同明甚。家無二尊,喪無二斬,人之所奉,不可貳也。為人後,降其父母喪。女子嫁,殺其家之喪。所存者遠,抑者私也。若外祖及舅加一等,而堂舅及姨著服,則中外其別幾何?且五服有上殺之義,伯叔父母服大功,從父昆弟亦大功,以其出於祖,服不得過於祖也。從祖祖父母、從祖父母、從祖昆弟皆小功,以其出於曾祖,服不得過曾祖也。族祖祖父母、族祖父母、族昆弟皆緦,以其出於高祖,服不得過高祖也。堂姨、舅出外曾祖,若為之服,則外曾祖父母、外伯叔祖父母亦可制服矣。外祖至大功,則外曾祖小功、外高祖緦。推而廣之,與本族無異。棄親錄疏,不可謂順。且服皆有報,則堂甥、外曾孫、侄女之子皆當服。聖人豈薄其骨肉恩愛哉?盡本於公者末於私,義有所斷,不得不然。茍可加也,則可減也,如是,禮可隳矣。請如古便。」楊仲昌又言:「舅服小功,魏徵嘗進之矣。今之所請,正同徵論。堂舅、堂姨、舅母,皆升袒免,則外祖父母進至大功,不加報於外孫乎?外孫而報以大功,則本宗之庶孫用何等邪?」



 帝手敕曰:「朕謂親姨、舅服小功,則舅母於舅有三年之喪,不得全降於舅,宜服緦。堂姨、舅古未有服,朕思睦厚九族,宜袒免。古有同爨緦,若比堂姨、舅於同爨,不已厚乎?《傳》曰:『外親服皆緦。』是亦不隔堂姨、舅也。若謂所服不得過本,而復為外曾祖父母、外伯叔父母制服,亦何傷?皆親親敦本意也。」



 侍中裴耀卿、中書令張九齡、禮部尚書李林甫奏言:「外服無降,甥為舅母服,舅母亦報之。夫之甥既報,則夫之姨、舅又當服,恐所引益疏。臣等愚,皆所不及。」詔曰:「從服六,此其一也。降殺於禮無文,皆自身率親為之數。姨、舅屬近,以親言之,亦姑伯之匹,可曰所引疏耶?婦人從夫者也,夫於姨舅既服矣,從夫而服,是謂睦親。卿等宜熟計。」耀卿等奏言:「舅母緦,堂姨舅袒免。請準制旨,自我為古,罷諸儒議。」制曰:「可。」



 初,帝詔歲率公卿迎氣東郊,至三時,常以孟月讀《時令》於正寢。二十六年,詔縚月奏《令》一篇,朔日於宣政側設榻,東向置案,縚坐讀之,諸司官長悉升殿坐聽。歲餘,罷。



 高宗上元三年,將袷享。議者以《禮緯》三年袷,五年禘;《公羊》家五年再殷祭。二家舛互,諸儒莫能決。太學博士史玄議曰:「《春秋》:僖公三十三年十二月薨。文公之二年八月丁卯,大享。《公羊》曰:『袷也。』則三年喪畢,新君之二年當袷,明年當禘群廟。又宣公八年,禘僖公。宣公八年皆有禘,則後禘距前禘五年。此則新君之二年袷、三年禘爾。後五年再殷祭,則六年當袷,八年禘。昭公十年,齊歸薨。十三年,喪畢當袷,為平丘之會。冬,公如晉,至十四年袷,十五年禘。《傳》曰『有事於武宮』是也。至十八年袷,二十年禘;二十三年袷,二十五年禘。昭公二十五年『有事於襄宮』是也。則禘後三年而袷,又二年而禘,合於禮。」議遂定。後睿宗喪畢,袷於廟。至開元二十七年,禘祭五,袷祭七。是歲,縚奏:「四月嘗已禘,孟冬又袷,祀禮叢數,請以夏禘為大祭之源。」自是相循,五年再祭矣。



 縚終太子少師。



 抗者,安石從父兄子。弱冠舉明經,累官吏部郎中。景雲初,為永昌令,輦轂繁要,抗不事威刑而治,前令無及者。遷右御史臺中丞,邑民詣闕留,不聽,乃立碑著其惠。開元三年,自太子左庶子為益州大都督府長兄,授黃門侍郎。河曲胡康待賓叛,詔持節慰撫。抗於武略非所長,稱疾逗留,不及賊而返。俄代王晙為御史大夫,兼按察京畿。弟拯方為萬年令,兄弟領本部,時以為榮。坐薦御史非其人,授安州都督,改薄州刺史。入為大理卿,進刑部尚書,分掌吏部選,卒。抗歷職以清儉,不治產,及終無以葬,玄宗聞之,特給槥車。贈太子少傅,謚曰貞。



 所表奉天尉梁升卿、新豐尉王倕、華原尉王燾為僚屬,後皆為顯人。升卿涉學工書,於八分尤工,歷廣州都督,書《東封朝覲碑》,為時絕筆。倕累遷河西節度使,天寶中,功聞於邊。它所闢舉,如王縉、崔殷等,皆一時選云。



 郭震,字元振,魏州貴鄉人,以字顯。長七尺,美須髯,少有大志。十六,與薛稷、趙彥昭同為太學生,家嘗送資錢四十萬,會有縗服者叩門,自言「五世未葬,願假以治喪」。元振舉與之,無少吝,一不質名氏。稷等嘆駭。十八舉進士,為通泉尉。任俠使氣,撥去小節,嘗盜鑄及掠賣部中口千餘,以餉遺賓客,百姓厭苦。武後知所為,召欲詰,既與語,奇之,索所為文章,上《寶劍篇》,後覽嘉嘆,詔示學士李嶠等,即授右武衛鎧曹參軍,進奉宸監丞。



 會吐蕃乞和,其大將論欽陵請罷四鎮兵,披十姓之地,乃以元振充使,因覘虜情。還,上疏曰:



 利或生害,害亦生利。國家所患,唯吐蕃與默啜耳,今皆和附,是將大利於中國也。若圖之不審,害且隨之。欽陵欲裂十姓地,解四鎮兵,此動靜之機,不可輕也。若直遏其意,恐邊患必甚於前,宜以策緩之,使其和望勿絕,而惡不得萌,固當取舍審也。夫患在外者,十姓、四鎮是也;患在內者,甘、涼、瓜、肅是也。關隴屯戍,向三十年,力用困竭,脫甘、涼有一日警,豈堪廣調發耶?



 善為國者,先料內以敵外,不貪外以害內,然後安平可保。欽陵以四鎮近己,畏我侵掠,此吐蕃之要;然青海、吐渾密邇蘭、鄯,易為我患,亦國家之要。今宜報欽陵曰:「四鎮本扼諸蕃走集,以分其力,使不得並兵東侵。今委之,則番力益強,易以擾動,保後無東意,當在吐渾諸部、青海故地歸於我,則俟斤部落還吐蕃矣。」此足杜欽陵口,而和議未絕。且四鎮久附,其倚國之心,豈與吐蕃等?今未知利害情實而分裂之,恐傷諸國意,非制御之算。



 後從之。



 又言:「吐蕃倦徭戍久矣,咸願解和;以欽陵欲裂四鎮,專制其國,故未歸款。陛下誠能歲發和親使,而欽陵常不從,則其下必怨,設欲大舉,固不能,斯離間之漸也。」後然其計。後數年,吐蕃君臣相猜攜,卒誅欽陵,而其弟贊婆等來降,因詔元振與河源軍大使夫蒙令卿率騎往迎。授主客郎中。



 久之,突厥、吐蕃聯兵寇涼州,後方禦洛城門宴,邊遽至,因輟樂,拜元振為涼州都督,即遣之。初,州境輪廣才四百里,虜來必傅城下。元振始於南硤口置和戎城,北磧置白亭軍,制束要路,遂拓境千五百里,自是州無虜憂。又遣甘州刺史李漢通闢屯田,盡水陸之利,稻收豐衍。舊涼州粟斛售數千,至是歲數登,至匹縑易數十斛,支廥十年,牛羊被野。治涼五歲,善撫御,夷夏畏慕,令行禁止,道不舉遺。河西諸郡置生祠,揭碑頌德。



 神龍中,遷左驍衛將軍、安西大都護。西突厥酋烏質勒部落盛強,款塞願和,元振即牙帳與計事。會大雨雪,元振立不動,至夕凍冽;烏質勒已老,數拜伏,不勝寒,會罷即死。其子娑葛以元振計殺其父,謀勒兵襲擊,副使解琬知之,勸元振夜遁,元振不聽,堅臥營為不疑者。明日,素服往吊,道逢娑葛兵,虜不意元振來,遂不敢逼,揚言迎衛。進至其帳,修吊贈禮,哭甚哀,為留數十日助喪事,娑葛感義,更遣使獻馬五千、駝二百、牛羊十餘萬。制詔元振為金山道行軍大總管。



 烏質勒之將闕啜忠節與娑葛交怨,屢相侵,而闕啜兵弱不支。元振奏請追闕啜入宿衛,徙部落置瓜、沙間。詔許之。闕啜遂行。至播仙城,遇經略使周以悌,以悌說之曰:「國家厚秩待君,以部落有兵故也。今獨行入朝,一矰旅胡人耳,何以自全?」乃教以重寶賂宰相,無入朝,請發安西兵導吐蕃以擊娑葛;求阿史那獻為可汗以招十姓;請郭虔使瓘拔汗那搜其鎧馬以助軍,既得復讎,部落更存。闕啜然之,即勒兵擊於闐坎城,下之。因所獲,遣人間道齎黃金分遺宗楚客、紀處訥,使就其謀。元振知之,上疏曰:



 國家往不與吐蕃十姓、四鎮而不擾邊者,蓋其諸豪泥婆羅等屬國自有攜貳,故贊普南征,身殞寇庭,國中大亂,嫡庶競立,將相爭權,自相翦屠,士畜疲癘,財力困窮,顧人事、天時兩不諧契,所以屈志於漢,非實忘十姓、四鎮也。如其有力,後且必爭。今忠節忽國家大計,欲為吐蕃鄉導主人,四鎮危機恐從此啟。吐蕃得志,忠節亦當在賊掌股,若為復得事我哉?往吐蕃於國無有恩力,猶欲爭十姓、四鎮;今若效力樹恩,則請分於闐、疏勒者,欲何理抑之?且其國諸蠻及婆羅門方自嫌阻,藉令求我助討者,亦何以拒之?是以古之賢人,不願夷狄妄惠,非不欲其力,懼後求無厭,益生中國事也。臣愚以為用吐蕃之力,不見其使。



 又請阿史那獻者,豈非以可汗子孫能招綏十姓乎?且斛瑟羅及懷道與獻父元慶、叔僕羅、兄俀子,俱可汗子孫也。往四鎮以他匐十姓之亂,請元慶為可汗,卒亦不能招來,而元慶沒賊,四鎮淪陷。忠節亦嘗請以斛瑟羅及懷道為可汗矣,十姓未附而碎葉幾危。又吐蕃亦嘗以俀子、僕羅並拔布為可汗矣,亦不能得十姓而皆自亡滅,此非它,其子孫無惠下之才,恩義素絕故也。豈止不能招懷,且復為四鎮患,則冊可汗子孫其效固試矣。獻又遠於其父兄,人心何繇即附,若兵力足取十姓,不必要須可汗子孫也。



 又請以郭虔瓘搜兵稅馬於拔汗那。往虔瓘已嘗與忠節擅入其國,臣時在疏勒,不聞得一甲一馬,而拔汗那挾忿侵擾,南導吐蕃。將俀子,以擾四鎮。且虔瓘往至拔汗那國,四面無助,若履虛邑,猶引俀子為敝。況今北有娑葛,知虔瓘之西,必引以相援,拔汗那倚堅城而抗於內,突厥邀伺於外,虔瓘等豈能復如往年得安易之幸哉?



 疏奏不省。



 楚客等因建遣攝御史中丞馮嘉賓持節安撫闕啜,以御史呂守素處置四鎮,以牛師獎為安西副都護,代元振領甘、涼兵,召吐蕃並力擊娑葛。娑葛之使娑臘知楚客謀,馳報之。娑葛怒,即發兵出安西、撥換、焉耆、疏勒各五千騎。於是闕啜在計舒河與嘉賓會,娑葛兵奄至,禽闕啜,殺嘉賓,又殺呂守素於僻城、牛師獎於火燒城,遂陷安西,四鎮路絕。元振屯疏勒水上,未敢動。楚客復表周以悌代元振,且以阿史那獻為十姓可汗,置軍焉耆以取娑葛。娑葛遺元振書,且言:「無仇於唐,而楚客等受闕啜金,欲加兵擊滅我,故懼死而鬥。且請斬楚客。」元振奏其狀。楚客大怒,誣元振有異圖,召將罪之。元振使子鴻間道奏乞留定西土,不敢歸京師。以悌乃得罪,流白州,而赦娑葛。



 睿宗立,召為太僕卿。將行,安西酋長有剺面哭送者,旌節下玉門關,去涼州猶八百里,城中爭具壺漿歡迎,都督嗟嘆以聞。景雲二年,進同中書門下三品,遷吏部尚書,封館陶縣男。先天元年,為朔方軍大總管,築豐安、定遠城,兵得保頓。明年,以兵部尚書復同中書門下三品。



 玄宗誅太平公主也,睿宗御承天門,諸宰相走伏外省,獨元振總兵扈帝,事定,宿中書者十四昔乃休。進封代國公,實封四百戶,賜一子官,物千段。俄又兼御史大夫,復為朔方大總管,以備突厥。未行,會玄宗講武驪山,既三令,帝親鼓之,元振遽奏禮止,帝怒軍容不整,引坐纛下,將斬之。劉幽求、張說扣馬諫曰:「元振有大功,雖得罪,當宥。」乃赦死,流新州。開元元年,帝思舊功,起為饒州司馬,怏怏不得志,道病卒,年五十八。十年,贈太子少保。



 元振雖少雄邁,及貴,居處乃儉約,手不置書,人莫見其喜慍。建宅宣陽里,未嘗一至諸院廄。自朝還,對親欣欣,退就室,儼如也。距國初仕至宰相而親具者,唯元振云。



 贊曰:魏、韋皆感概而奮,似矣。及在惸上側臣間,臨機會,不一引手揕奸邪之謀,誠可鄙哉。至牴後艷主以烝譖撼宗社,亦不肯從也。古所謂具臣者,諒乎!元振功顯節完,一跌未復,世恨其蚤歿云。



\end{pinyinscope}