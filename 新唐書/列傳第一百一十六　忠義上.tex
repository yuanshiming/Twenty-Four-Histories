\article{列傳第一百一十六 忠義上}

\begin{pinyinscope}

 夫有生所甚重者,身也;得輕用者,忠與義也。後身先義,仁也;身可殺,名不可死量變以及它們之間的內在聯系和規律性。否定之否定規律揭,志也。大凡捐生以趣義者,寧豫期垂名不朽而為之?雖一世成敗,亦未必濟也;要為重所與,終始一操,雖頹嵩、岱,不吾壓也。夷、齊排周存商,商不害亡,而周以興。兩人至餓死不肯屈,卒之武王蒙慚德,而夷、齊為得仁,仲尼變色言之,不敢少損焉。故忠義者,真天下之大閑歟!奸鈇逆鼎,搏人而肆其毒,然殺一義士,則四方解情,故亂臣賊子赩然疑沮而不得逞。何哉?欲所以為彼者,而為我也。義在與在,義亡與亡,故王者常推而褒之,所以砥礪生民而窒不軌也。雖然,非烈丈夫,曷克為之?彼委靡軟熟,偷生自私者,真畏人也哉!故次敘夏侯端以來凡三十三人於左方。



 夏侯端,壽州壽春人,梁尚書左僕射詳孫也。仕隋為大理司直。高祖微時與相友,大業中討賊河東,表端為副。端邃數術,密語高祖曰:「玉床搖,帝坐不安。晉得歲,真人將興,安天下之亂者,其在公乎!但上性沈忌,內惡諸李,今金才已誅,次且取公,宜蚤為計。」帝感其言。義師興,端在河東,吏捕送長安。帝入京師,釋囚,引入臥內,擢秘書監。



 李密之降,關東地未有所屬,端請假節招諭,乃拜大將軍,為河南道招尉使。即傳檄州縣,東薄海,南揵淮,二十餘州遣使順附。次譙州,會亳、汴二州刺史已降王世充,道塞,無所歸,計窮徬徨。麾下二千人糧盡不忍委端去,端乃殺馬宴大澤中,謂眾曰:「我奉王命,義無屈。公等有妻子,徒死無益。吾丐若首,持與賊以取富貴。」眾號泣不忍視,端亦泣,欲自刎,爭持之,乃止。行五日,餓死十四三。遇賊,眾潰,從者才三十餘人,遂東走,擷雰豆以食。端持節臥起,嘆曰:「平生不知死地乃在此!」縱其下令去,毋俱沒。會李公逸守杞州,勒兵迎端。時河南地悉入世充,公逸感端之節,亦固守。世充遣人以淮南郡公、尚書少吏部印綬召端,解所服衣以贈。端曰:「吾,天子使,寧污賊官邪!非持首去不可見。」即焚書及衣,因解節毛懷之,間道走宜陽,歷崖峭榛莽。比到,其下僅有在者,皆體發臒焦,人不堪視。端入謁,自謝無功,不及危困狀。帝閔之,復拜秘書監。出為梓州刺史。散祿稟周孤窮,不為子孫計。貞觀元年卒。



 劉感,岐州鳳泉人,後魏司徒豐生孫也。武德初,以驃騎將軍戍涇州,為薛仁杲所圍,糧盡,殺所乘馬啖士,而煮骨自飲,至和木屑以食。城垂陷,長平王叔良救之,賊乃解。與叔良出戰,為賊執,還圍涇州,令感約城中降。感紿諾,至城下大呼曰:「賊大饑,亡在朝暮,秦王數十萬眾且至,勉之無苦。」仁杲怒,執感埋其半土中,馳射之。至死,詈益甚。



 賊平,高祖購得其尸,祭以少牢,贈瀛州刺史,爵平原郡公,封戶二千,謚忠壯。詔其子嗣封爵,賜田宅焉。



 常達,陜州陜人。仕隋為鷹擊郎將。嘗從高祖征伐,與宋老生戰霍邑,軍敗自匿,帝意已死,久乃自歸。帝大悅,命為統軍,拜隴州刺史。



 時薛舉方強,達敗其子仁杲,斬首千級。舉遣將仵士政紿降,達不疑,厚加撫接。士政伺隙劫之,並其眾二千歸賊。舉指其妻謂達曰:「識皇后乎?」答曰:「彼癭老嫗,何所道?」舉奴張貴又曰:「亦識我否?」達瞋目曰:「若乃奴耳。」貴忿,舉笏擊其面,達不為懾,亦拔刀逐之,趙弘安為蔽捍,乃免。仁杲平,帝見達,勞曰:「君忠節,正可求之古人。」為執士政殺之,賜達布帛三百段,以達並劉感事授史臣令狐德棻云。終隴西刺史。



 敬君弘,絳州絳人,北齊尚書右僕射顯俊曾孫也。累功歷驃騎將軍,封黔昌侯。以屯營兵守玄武門。隱太子之死,左右解散。其車騎將軍馮立者,有材武,嘆曰:「生賴其寵,死不共難,我無以見士大夫!」乃與巢王親將謝叔方率兵攻玄武門,殊死鬥。君弘挺身出,或曰:「事未可判,當按兵待變,成列而鬥可也。」不從。與中郎將呂世衡呼而進,皆戰歿。立顧其下曰:「足以報太子矣。」遂解兵走。君弘等敗,秦府兵不振。尉遲敬德擲巢王首示叔方,叔方下馬慟,亦出奔。明日自歸,太宗曰:「義士也。」置之。俄而立又至,帝讓曰:「汝離我兄弟,罪一也;殺我將士,罪二也。何所逃死?」答曰:「出身事主,當戰之日,不知其它。」因伏地悲不自勝,帝亦勞遣之。詔贈君弘左屯衛大將軍,世衡右驍衛將軍。



 立已蒙貸,歸語人曰:「上赦吾罪,吾當以死報。」未幾,突厥犯便橋,立引數百騎與虜薄,敗之咸陽。帝喜,授廣州都督。前日牧守苛肆,為蠻夷患,故數叛。立至,不事家產,衣食弗求贏。嘗見貪泉曰:「此豈隱之所酌邪?吾雖日汲,庸易吾性哉?」遂極飲去。在職不三年,有惠愛,卒於官。



 叔方歷伊州刺史,善治軍,戎、華愛之。累加銀青光祿大夫,徙洪、廣二州都督。卒,謚曰勤。本萬年人,從巢王征討有功,王表為屈咥真府左軍騎雲。



 呂子臧,蒲州河東人。剛直,健於吏。隋大業末國南陽郡丞,捕擊盜賊有功。高祖入京師,遣馬元規慰輯山南,獨子臧堅守。元規遣士諷曉,子臧殺之。及煬帝已弒,帝更使其婿薛君傅齎詔,言隋所以亡,諭子臧。子臧為故君發喪訖,即送款,就拜鄧州刺史,封南陽郡公。



 武德初,硃粲新衄,子臧率兵與元規並力。元規軍不進,子臧曰:「乘賊新敗,上下惶沮,一戰可禽;若遷延,其眾稍集,吾食盡,致死於我,不可當也。」不納。子臧請以所部兵獨進,又不許。俄而粲得眾,復張,元規嬰城,子臧扼腕曰:「謀不見用,坐公死矣。」賊圍固。會霖雨,雉堞崩剝,或勸其降,子臧曰:「我,天子方伯,且降賊乎?」乃率麾下數百人赴敵死,城亦陷,元規死之。



 元規,安陸人。初以隊正從帝征伐,持節下南陽,得兵萬餘,然無謀,至於敗。



 王行敏,並州樂平人。隋末為盜長,高祖興,來降,拜潞州刺史,遷屯衛將軍。劉武周入並州,寇上黨,取長子、壺關。或言刺史郭子武懦不支,且失潞,帝遣行敏馳往。既至,與子武不葉,賊圍急,儲偫空乏,眾恫懼,行敏患之。會有告子武謀反,遂斬之。州民陳正謙者,以信義稱鄉里,出粟千石濟軍,由是人自奮,賊乃去。行敏又敗竇建德兵於武陟。武德四年,督兵徇燕、趙,與劉黑闥戰歷亭,破之。既而釋甲不設備,為黑闥所掩,縛致麾下。終不屈,賊遂斬之。且死,西向跪曰:「臣之忠,惟陛下知之。」帝聞而悼惜。



 黑闥之亂,死事者又有盧士叡、李玄通。



 士叡客韓城。隋亂,結納英豪。高祖與之舊,及兵興,率數百人上謁汾陰,又使兄子諭降劇賊孫華,與劉弘基敗隋將桑顯和於飲馬泉。擢累右光祿大夫,為瀛州刺史。黑闥遣輕騎破其郛,拒戰半日,士見親屬系虜,乃潰。士叡為賊擒,欲使說下城堡,不從,見殺。



 玄通,藍田人。為隋鷹揚郎將,高祖入關,率所部自歸,拜定州總管。為黑闥所破,愛其才,欲以為將。玄通曰:「吾當守節以報,烏能降志賊邪?」不聽,囚之。故吏有餉飲饋者,玄通曰:「諸君見哀,吾能一醉。」遂縱飲,謂守者曰:「吾能劍舞,可借刀。」守士與之。曲終,仰天太息曰:「大丈夫撫方面,不能保所守,尚可視息邪?」乃潰腹死。帝為流涕,擢其子伏護大將軍。



 羅士信,齊州歷城人。隋大業時,長白山賊王薄、左才相、孟讓攻齊郡,通守張須陀率兵擊賊。士信以執衣,年十四,短而悍,請自效。須陀疑其不勝甲,少之。士信怒,被重甲,左右鞬,上馬顧眄。須陀許之。擊賊濰水上,陣才列,執長矛馳入賊營,刺殺數人,取一級擲之,承以矛,戴而行,賊皆眙懼無敢亢。須陀乘之,大破賊。士信逐北,每殺一賊,輒劓鼻納諸懷,暨還,驗以代級。須陀嘆伏,遺以所乘馬。凡戰,須陀先登,士信副,以為常。煬帝遣使圖須陀、士信陣法上內史。



 後須陀為李密所殺,士信與裴仁基歸密,署總管,俾統所部討王世充。身被重創,見獲於世充。世充愛其才,厚遇之,與同寢食。後得密將邴元真等,故士信稍稍疏斥。士信恥與伍,率所部千餘人來降高祖,拜陜州道行軍總管,因謀世充。



 士信行則先鋒,反則殿,有所獲,悉散戲下有功者,或脫衣解馬賜之,士以故用命。然持法嚴,至親舊無少貸,其下亦不甚附。師次洛陽,攻千金堡,堡有惡言訽軍,士信怒,夜遣百人載嬰兒啼噪堡下,若自東都出奔者,既而陽悟曰:「非也,此千金堡耳。」因散去。堡兵開門追掠,士信伏入,屠之無類。賊平,授絳州總管,封郯國公。



 從秦王擊劉黑闥洛水上,得一城,王君廓戍之,賊急攻,潰而出。王語諸將:「孰能守此?」士信曰:「願以守。」乃命之。士信已入,賊悉眾攻,方雨雪,救軍不得進。城陷,黑闥欲用之,不屈而死,年二十八。王隱悼,購其尸以葬,謚曰勇。初,士信為仁基所禮,及東都平,出家財斂葬北邙以報德,且曰:「我死當墓其側。」至是,如所志。



 張道源,並州祁人,名河,以字顯。年十四,居父喪,士人賢其孝,縣令郭湛署所居曰復禮鄉至孝里。道源嘗與客夜宿,客暴死,道源恐主人忽怖,臥尸側,至署乃告,又徒步護送還其家。隋末政亂,辭監察御史,歸閭里。



 高祖興,署大將軍府戶曹參軍。至賈胡堡,復使守並州。京師平,遣撫慰山東,下燕、趙。有詔褒美,封累範陽郡公。淮安王神通略定山東,令守趙州,為竇建德所執。會建德寇河南,間遣人詣朝,請乘虛搗賊心脅。即詔諸將率兵影接。俄而賊平,還,拜大理卿。時何稠得罪,籍其家屬賜群臣。道源曰:「禍福何常,安可利人之亡,取其子女自奉?仁者不為也。」更資以衣食遣之。天子見其年耆,拜綿州刺史。卒,贈工部尚書,謚曰節。道源雖官九卿,無產貲,比亡,餘粟二斛。詔賜帛三百段。



 族孫楚金有至行,與兄越石皆舉進士。州欲獨薦楚金,固辭,請俱罷。都督李勣嘆曰:「士求才行者也。既能讓,何嫌皆取乎?」乃並薦之。累進刑部侍郎。儀鳳初,彗見東井,上疏陳得失。高宗欽納,賜物二百段。武后時,歷秋官尚書,爵南陽侯。有清概,然尚文刻,當時亦少之。為酷吏所構,流死嶺表。



 李育德,趙州人。祖諤,仕隋通州刺史,為名臣。世富於財,家僮百人。天下亂,乃私完械甲,嬰武陟城自保,人多從之,遂為長。劇賊來掠,不能克。隋亡,與柳燮等歸李密,私署總管。密為王世充所破,以郡來降,即拜陟州刺史。



 兄厚德,自賊所逃歸,度河復被執。賊使招育德,陽許之,故兄不死。賊帥段大師令裨校以兵守厚德,陰得其驤,,乃與州人賈慈行謀逐賊。慈行夜登城呼曰:「唐兵登矣!」厚德自獄擁群囚噪而出,斬長史,眾不敢動,大師縋城走。即拜殷州刺史。厚德省親,留育德以守,引兵拔賊河內堡三十一所。世充怒,悉銳士攻之,城陷,猶力戰,與三弟皆歿。



 時死節者又有李公逸、張善相,凡三人。



 公逸者,與族弟善行居雍丘,以才雄,為眾所歸。始附王世充,策其必敗,乃獻款高祖,因其地置杞州,即拜總管,封陽夏郡公。以善行為刺史。世充遣其弟將徐、亳兵攻之,公逸請援,未報,因使善行守,身入朝言狀。至襄城,為賊邏送洛陽。世充曰:「君越鄭臣唐,何哉?」答曰:「我於天下唯聞有唐。」賊怒斬之。善行亦死。帝悼惜,封其子襄邑縣公。



 善相,襄城人。大業末為里長,督兵跡盜,為眾附賴,乃據許州奉李密。密敗,挈州以來,詔即授伊州總管。王世充攻之,屢困賊,遣使三輩請救,朝廷未暇也。會糧盡,眾餓死,善相謂僚屬曰:「吾為唐臣,當效命。君等無庸死,斬吾首以下賊可也。」眾泣不肯,曰:「與公同死,愈於獨生。」城陷被執,罵賊見殺。高祖嘆曰:「吾負善相,善相不負我!」乃封其子襄城郡公。



 高叡,京兆萬年人,隋尚書左僕射穎孫也。舉明經,稍遷通義令,有治勞,人刻石載德。歷趙州刺史,平昌縣子。聖歷初,突厥默啜入寇,叡嬰城拒,虜攻益急。長史唐波若度且陷,即與虜通。叡覺之,力不能制,即自經。不得死,為虜執,使降諭諸縣,不肯應,見殺。初,虜至,有為叡計者:「突厥蜂銳,所向無完,公不能亢,且當下之。」答曰:「我,刺史,不戰而降,罪大矣。」武后嘆惜,贈冬官尚書,謚曰節。詔誅波若,籍其家。下制暴叡忠節、波若臣賊,使天下知之。



 子仲舒,通故訓學,擢明經,為相王府文學,王所欽器。開元初,宋璟、蘇頲當秉,多咨訪焉。時舍人崔琳練達政宜,璟等禮異之。當語人曰:「古事問高仲舒,時事問崔琳,何復疑?」終太子右庶子。



 安金藏,京兆長安人。在太常工籍。睿宗為皇嗣,少府監裴匪躬、中官範雲仙坐私謁皇嗣,皆殊死,自是公卿不復見,唯工優給使得進。俄有誣皇嗣異謀者,武后詔來俊臣問狀,左右畏慘楚,欲引服。金藏大呼曰:「公不信我言,請剖心以明皇嗣不反也。」引佩刀自剚腹中,腸出被地,眩而僕。後聞大驚,輿致禁中,命高醫內腸,褫桑紩之,閱夕而蘇。後臨視,嘆曰:「吾有子不能自明,不如爾之忠也。」即詔停獄,睿宗乃安。當是時,朝廷士大夫翕然稱其誼,自以為弗及也。



 神龍初,母喪,葬南闕口,營石墳,晝夜不息。地本卬燥,泉忽湧流廬之側,李冬有華,犬鹿相擾。本道使盧懷慎上其事,詔表闕於閭。景雲時,遷右武衛中郎將。玄宗屬其事於史官,擢右驍衛將軍,爵代國公。詔鑱其名於泰、華二山碑以為榮。卒,配饗睿宗廟廷。大歷中,贈兵部尚書,謚曰忠。以子承恩為廬州長史。中和中,又擢其遠孫敬則為太子右諭德。



 王同皎,相州安陽人,陳駙馬都尉寬曾孫也。陳亡,徙河北。長安中,尚太子女安定郡主,拜典膳郎。太子,中宗也。桓彥範等誅二張,遣同皎與李湛、李多祚即東宮迎太子,請至玄武門指授諸將。太子拒不許,同皎進曰:「逆豎反道,顯肆不軌,諸將與衙執事刻期誅之,須殿下到以系眾望。」太子曰:「上方不豫,得無不可乎?」同皎曰:「南將相毀家族以安社稷,奈何欲內之鼎鑊乎?太子能自出諭之,眾乃止。」太子猶豫,同皎即扶上馬,從至玄武門,斬關入。兵趨長生殿太后所,環侍嚴定,因奏誅易之等狀。帝復位,擢右千牛將軍,封瑯邪公,食實戶五百。主進封訟主,拜同皎駙馬都尉,遷光祿卿。



 神龍後,武三思烝濁王室,同皎惡之,與張仲之、祖延慶、周憬、李悛、冉祖雍謀,須武後靈駕發,伏弩射殺三思。會播州司兵參軍宋之愻以外妹妻延慶,延慶辭,之愻固請,乃成昏。延慶心厚之,不復疑。故之愻子曇得其實。之愻兄之問嘗舍仲之家,亦得其謀。令曇密語三思。三思遣悛上急變,且言同皎欲擁兵闕下廢皇后。帝殊不曉,大怒,斬同皎於都亭驛,籍其家。同皎且死,神色自如。仲之、延慶皆死。憬遁入比干廟自剄,將死,謂人曰:「比干,古忠臣,神而聰明,其知我乎!後、三思亂朝,虐害忠良,滅亡不久,可干吾頭國門,見其敗也。」憬,壽春人。後太子重俊誅三思,天下共傷同皎之不及見也。睿宗立,詔復官爵,謚曰忠壯。誅祖雍、悛等。



 先是,許州司戶參軍燕欽融再上書斥韋后擅政,且逆節已萌。後怒,勸中宗召至廷,撲殺之。宗楚客復私令衛士極力,故死。又博陵人郎岌亦表後及楚客亂,被誅。至是,俱贈諫議大夫,備禮改葬,賜欽融一子官。



 同皎子繇尚永穆公主,生子潛,字弘志。生三日,賜緋衣、銀魚。幼莊重,不喜兒弄。以帝外孫,補千牛,復選尚公主,固辭。元和中擢累將作監。吏或籍名北軍,輒驕墯不事,潛悉奏罷之,故不戒而辨。監無公食,而息錢舊皆私有,至潛,取以具食,遂為故事。



 遷左散騎常侍,拜涇原節度使。憲宗與對,大悅,曰:「吾知而善職,我自用之。」潛至鎮,繕壁壘,積粟,構高屋偫兵,利而嚴。遂引師自原州逾硤石,取虜將一人,斥烽候,築歸化、潘原三壘。請復城原州,度支沮議,故原州復陷。穆宗即位,封瑯邪郡公,更節度荊南。疏吏惡,榜之里閭,殺尤縱者。分射三等,課士習之,不能者罷,故無冗軍。大和初,檢校尚書左僕射。卒於官,贈司空。



 吳保安字永固,魏州從。氣挺特不俗。睿宗時,姚、巂蠻叛,拜李蒙為姚州都督,宰相郭元振以弟之子仲翔托蒙,蒙表為判官。時保安罷義安尉,未得調,以仲翔里人也,不介而見曰:「願因子得事李將軍可乎?」仲翔雖無雅故,哀其窮,力薦之。蒙表掌書記。保安後往,蒙已深入,與蠻戰沒,仲翔被執。蠻之俘華人,必厚責財,乃肯贖,聞仲翔貴胄也,求千縑。會元振物故,保安留巂州,營贖仲翔,苦無貲。乃力居貨十年,得縑七百。妻子客遂州,間關求保安所在,困姚州不能進。都督楊安居知狀,異其故,資以行,求保安得之。引與語曰:「子棄家急朋友之患至是乎!吾請貣官貲助子之乏。」保安大喜,即委縑於蠻,得仲翔以歸。始,仲翔為蠻所奴,三逃三獲,乃轉鬻遠酋,酋嚴遇之,晝役夜囚,沒凡十五年乃還。



 安居亦丞相故吏,嘉保安之誼,厚禮仲翔,遺衣服儲用,檄領近縣尉。久乃調蔚州錄事參軍,以優遷代州戶曹。母喪,服除,喟曰:「吾賴吳公生吾死,今親歿,可行其志。」乃求保安。於時,何安以彭山丞客死,其妻亦沒,喪不克歸。仲翔為服縗絰,囊其骨,徒跣負之,歸葬魏州,廬墓三年乃去。後為嵐州長史,迎保安子,為娶而讓以官。



 李憕,並州汶水人。或言其先出興聖皇帝,譜系疏晦,不復傳。父希倩,神龍初右臺監察御史。憕少秀敏,舉明經高第,授成安尉。張說罷宰相,為相州刺史,坐有善相者,說遍問官屬後孰當貴,工指憕及臨河尉鄭巖。說以女妻巖,而歸其甥陰於憕。會母喪免。自武功尉以政尤異遷主簿。說在並州,引心登置幕府。及執政,為長安尉。宇文融括天下田,高選官屬,多致賢以重其柄。表假憕監察御史,分道檢核。以課真拜御史。坐小累,下除晉陽令。三遷給事中。力於治,有任事稱,明簿最,下無敢紿。失李林甫意,出為河南少尹。尹蕭炅內倚權,骫法殖私,憕裁抑其謬,吏下賴之。道士孫甑生以左道幸,托祠事往來嵩、少間,干請亂吏治,憕不為應,故挾炅譖諸朝。天寶初,除清河太守。舉美政,遷廣陵長史,民為立祠賽祝,歲時不絕。以捕賊負,徙彭城太守。封酒泉縣侯。連徙襄陽、河東,並兼採訪處置使;入為京兆尹。楊國忠惡之,改光祿卿、東京留守。



 安祿山反,玄宗遣封常清募兵東京,憕與留臺御史中丞盧弈、河南尹達奚珣繕城壘,綏勵士卒,將遏賊西鋒。帝聞,擢禮部尚書。祿山度河,號令嚴密,候言冋不能知。已陷陳留、滎陽,殺張介然、崔詖,不數日,薄城下。常清兵皆白徒,戰不勝,輒北。憕收殘士數百,裒斷弦折矢堅守,人不堪鬥。憕約弈:「吾曹荷國重寄,雖力不敵,當死官。」部校皆夜縋去,憕坐留守府,弈守臺。城陷,祿山鼓而入,殺數千人,矢著闕門,執憕、奕及官屬蔣清,害之。有詔贈司徒,謚曰忠懿。河、洛平,再贈太尉,拜一子五品官。



 憕通《左氏春秋》,頗殖產伊川,占膏腴,自都至闕口,疇墅彌望,時謂「地癖」。巖仕終少府監,產利埒憕云。



 憕十餘子,江、涵、渢、瀛等同遇害,唯源、彭脫。



 源八歲家覆,俘為奴,轉側民間。及史朝義敗,故吏識源於洛陽者贖出之,歸其宗屬。代宗聞,授河南府參軍,遷司農主簿。以父死賊手,常悲憤,不仕不娶,絕酒葷。惠林佛祠者,憕舊墅也,源依祠居,闔戶日一食。祠殿,其先寢也,每過必趨,未始踐階。自營墓為終制,時時偃臥埏中。



 長慶初,年八十矣,御史中丞李德裕表薦源,曰:「賈誼稱:守圉捍敵之臣,死城郭封疆。天寶時,士罕伏節,逆羯始興,委符組、棄城郭者不為恥,而憕約義同列,守位自如,抵刃就終,臣節之光由憕始。而源天與至孝,絕心祿仕五十餘年,常守沈默,理契深要,一辭開析,百慮洗然。抱此真節,棄於清世,臣竊為陛下惜之。」穆宗下詔曰:「昔盜起幽陵,振蕩河、洛,贈太尉憕處難居首,正色就死,兩河聞風,再固危壁,殊節卓焉,到今稱之。源有曾參之行、巢父之操,泊然無營,汔此高年。夫褒忠,所以勸臣節也;旌孝,所以激人倫也;鎮澆浮,莫如尚義;厚風俗,莫如尊老。舉是四者,大儆于時。其以源守諫議大夫,賜緋魚袋。」河南尹遣官敦諭上道,帝自遣使者持詔書袍笏即賜,又賜絹二百匹。源頓首受詔,謂使者:「伏疾年耄,不堪趨拜。」即附表謝,辭吐哀愨,一無受。尋卒。敬宗時,擢憕孫為河南兵曹參軍。



 彭擢明經第。天寶中,選名臣子可用者,自咸寧丞遷右補闕。從天子入蜀。後憕數年卒。有孫景讓、景莊、景溫,別傳。



 武德功臣十六人,貞觀功臣五十三人,至德功臣二百六十五人。德宗即位,錄武德以來宰相及實封功臣子孫,賜一子正員官。史館考勛名特高者九十二人,以三等條奏。第一等,以其歲授官。第二等,以次年。第三等,子孫數訟於朝,有詔差為二等,增至百八十七人。每等,武德以來宰相為首,功臣次之,至德以來將相又次之。大中初,又詔求李峴、王珪、戴胄、馬周、褚遂良、韓瑗、郝處俊、婁師德、王及善、硃敬則、魏知古、陸象先、張九齡、裴寂、劉文靜、張柬之、袁恕已、崔玄、桓彥範、劉幽求、郭元振、房琯、寺履謙、李嗣業、張巡、許遠、盧弈、南霽雲、蕭華、張鎬、李勉、張鎰、蕭復、柳渾、賈耽、馬燧、李憕三十七人畫像,續圖凌煙閣雲。



 司空、太子太傅、知門下省事、梁國公房玄齡



 尚書右僕射、檢校侍中、萊國公杜如晦



 太子太保、同中書門下三品、宋國公蕭瑀



 開府儀同三司、同中書門下三品、知政事、上柱國、申國公高士廉



 太子太師、知政事、特進、鄭國公魏徵



 侍中、永寧郡公王珪



 吏部尚書、參豫朝政、道國公戴胄



 中書令、江陵縣子岑文本



 中書令、兼太子左庶子、檢校吏部尚書、高唐縣公馬周



 侍中、兼太子左庶子、檢校吏部禮部民部尚書事、清苑縣男劉洎



 尚書右僕射、同中書門下三品、河南郡公褚遂良



 太子太師、同中書門下三品、燕國公於志寧



 尚書右僕射、同中書門下三品、兼太子少傅、北平縣公張行成



 中書令、行侍中、兼太子少保、蓚縣公高季輔



 侍中、兼太子賓客、襲潁川縣公韓瑗



 中書令、兼太子詹事、南陽縣侯來濟



 侍中、兼太子賓客張文瓘



 侍中、甑山縣公郝處俊



 中書侍郎、同中書門下三品、兼太子右庶子、酒泉縣公李義琰



 內史、河東縣侯裴炎文昌左相、同鳳閣鸞臺三品、溫國公蘇良嗣



 內史、梁國公狄仁傑



 納言、檢校並州大都督府長史、天兵軍大總管、隴右諸軍大使、譙縣子婁師德



 鳳閣侍郎、同鳳閣鸞臺平章事、石泉縣公王方慶



 文昌左相、同鳳閣鸞臺三品、襲邢國公王及善



 尚書右僕射、兼中書令、知兵部尚書事、齊國公魏元忠



 紫微令、梁國公姚崇



 正諫大夫、同鳳閣鸞臺平章事硃敬則



 尚書左僕射、同中書門下平章事、許國公蘇瑰



 吏部尚書、兼侍中、廣平郡公宋璟



 黃門監、梁國公魏知古



 中書侍郎、同中書門下平章事、兗國公陸象先



 紫微侍郎、同紫微黃門平章事、許國公蘇頲



 中書令、河東縣侯張嘉貞



 中書侍郎、同中書門下平章事、清水縣公李元紘



 黃門侍郎、同中書門下平章事、宜陽縣子韓休



 中書令、始興縣伯張九齡



 司空、河東郡公裴寂



 納言、上柱國、魯國公劉文靜



 太尉、檢校中書令、同中書門下三品、揚州大都督、趙國公長孫無忌



 禮部尚書、河間郡王孝恭



 尚書右僕射、檢校中書令、行太子左衛率、上柱國、衛國公李靖



 司空、兼太子太師、英國公李勣



 開府儀同三司、鄜州都督、鄂國公尉遲敬德



 左光祿大夫、洛州都督、蔣國公屈突通



 陜東道大行臺、吏部尚書、鄖國公殷開山



 衛尉卿、夔國公劉弘基



 澤州刺史、邳國公長孫順德



 民部尚書、上柱國、莒國公唐儉



 右驍衛大將軍、駙馬都尉、譙國公柴紹



 右驍衛大將軍、褒國公段志玄



 洪州都督、渝國公劉政會



 左武候將軍、相州都督、郯國公張公謹



 右武衛大將軍、盧國公程知節



 左武衛大將軍、上柱國、胡國公秦叔寶



 弘文館學士、秘書監、永興縣公虞世南



 右衛大將軍、兼太子右衛率、工部尚書、武陽縣公李大亮



 左武衛大將軍、邢國公蘇定方



 夏官尚書、同中書門下三品、清邊道行軍總管、耿國公王孝傑



 中書令、漢陽郡公張柬之



 中書令、博陵郡公崔玄



 侍中、平陽郡公敬暉



 侍中、譙國公桓彥範



 中書令、南陽郡公袁恕已



 右武衛大將軍、同中書門下三品、韓國公張仁願



 尚書左丞相、兼黃門監、徐國公劉幽求



 黃門侍郎、參知機務、脩文館學士、齊國公崔日用



 兵部尚書、同中書門下三品、代國公郭元振



 尚書左承相、兼中書令、集賢院學士、燕國公張說



 紫微侍郎、上柱國、趙國公王琚



 兵部尚書、同中書門下三品、持節朔方軍節度大使、中山郡公王晙



 尚書左僕射、同中書門下平章事、兼河南江淮副元帥、東都留守、冀國公裴冕



 文部尚書、同中書門下平章事、清河縣公房琯



 門下侍郎、同中書門下平章事、衛國公桂鴻漸



 鎮西北庭行營節度使、開府儀同三司、衛尉卿、兼懷州刺史、虢國公李嗣業



 平盧軍節度使、柳城郡太守劉正臣



 恆州刺史、衛尉少卿、兼御史中丞顏杲卿



 常山郡太守袁履謙



 河南節度副使、左金吾衛將軍、檢校主客郎中、兼御史中丞張巡



 睢陽郡太守、兼御史中丞許遠



 御史中丞、留臺東都、知武選盧弈



 睢陽郡太守、特進左金吾衛將軍南霽雲



 右第一



 內史令、延安郡公竇威



 將作大匠、判納言、陳國公竇抗



 侍中、兼太子左庶子、江國公陳叔達



 納言、觀國公楊恭仁



 判吏部尚書、參議朝政、安吉郡公杜淹



 中書令、虞國公溫彥博



 中書侍郎、檢校刑部尚書、參知機務崔仁師



 中書令、兼檢校太子詹事、上柱國、安國公崔敦禮



 戶部尚書、平恩縣公許圉師



 兵部尚書、同中書門下三品、浿江道行軍總管任雅相



 度支尚書、同中書門下三品、範陽郡公盧承慶



 西臺侍郎、同東西臺三品、兼弘文館學士、楚國公上官儀



 右相、廣平郡公劉祥道



 左侍極、兼檢校左相、嘉興縣子陸敦信



 文昌左相、同鳳閣鸞臺三品、樂城縣公劉仁軌



 荊州大都督府長史、安平郡公李安期



 尚書右僕射、同中書門下三品、兼太子賓客、襲道國公戴至德



 司列少常伯、太子右中護、兼正諫大夫、同東西臺三品趙仁本



 中書令、趙國公李敬玄



 中書令、兼太子左庶子薛元超



 中書令、同中書門下三品崔知溫



 侍中、同中書門下三品、襲廣平郡公劉齊賢



 納言、樂平縣男王德真



 地官尚書、檢校納言、鉅鹿縣男魏玄同



 文昌左相、同鳳閣鸞臺三品、特進、輔國大將軍、鄧國公岑長倩



 鳳閣侍郎、同鳳閣鸞臺三品、臨淮縣男劉禕之



 納言、博昌縣男韋思謙



 地官尚書、同鳳閣鸞臺平章事格輔元



 司禮卿、判納言事、渤海縣子歐陽通



 內史李昭德



 鸞臺侍郎、同鳳閣鸞臺平章事陸元方



 鳳閣侍郎、同鳳閣鸞臺三品杜景佺



 尚書右僕射、兼太子賓客、同中書門下三品、鄖國公韋安石



 左散騎常侍、同中書門下三品、知東都留守、趙郡公李懷遠



 中書令、逍遙公韋嗣立



 守侍中、同中書門下三品、兼太子右庶子、常山縣男李日知



 檢校黃門監、漁陽縣伯盧懷慎



 中書令、左丞相、兼侍中、安陽郡公源乾曜



 黃門侍郎、同紫微黃門平章事、魏縣侯杜暹



 侍中、趙城侯裴耀卿



 左武衛大將軍、開府儀同三司、雀安王神通



 特進、太常卿、江夏王道宗



 荊州都督、周國公武士畐



 右屯衛大將軍、檢校晉州都督總管、譙國公竇琮



 少府監、葛國公劉義節



 右光祿大夫、羅國公張平高



 洛州都督、右衛大將軍、酂國公竇軌



 夔州都督、息國公張長愻



 金紫光祿大夫、夷國公李子和



 左監門衛大將軍、檢校右武候將軍、榮國公樊興



 左監門衛大將軍、巢國公錢九隴



 右驍衛大將軍、歸國公安興貴



 右武衛大將軍、申國公安脩仁



 殿中監、郢國公宇文士及



 右武衛大將軍、沔陽郡公公孫武達



 荊州都督、懷寧郡公杜君綽



 右驍衛將軍、濮國公龐卿惲



 代州都督、同安郡公鄭仁泰



 右翊衛將軍、遂安郡公李安遠



 幽州都督、歷陽郡公獨孤彥云



 始州刺史、左屯衛大將軍、襄武郡公劉師立



 右威衛大將軍、濟東郡公李孟嘗



 右監門衛大將軍、河南縣公元仲文



 右監門衛將軍、廬陵郡公秦師行



 左領軍大將軍、新興公馬三寶



 右衛大將軍、駙馬都尉、畢國公阿史那社尒



 鎮軍大將軍、虢國公張士貴、左衛大將軍、瑯邪郡公牛進達



 鎮軍大將軍、嘉州郡公周護



 陜州刺史、天水郡公丘行恭



 潭州都督、吳興郡公沈叔安



 散騎常侍、豐城縣男姚思廉



 太子少師、同中書門下三品、特進、朔方道行軍大總管,宋國公唐休璟



 左羽林軍大將軍、遼陽郡王李多祚



 左領軍大將軍、趙國公李湛



 刑部尚書、太子賓客、魏國公楊元琰



 殿中監、兼知總監、汝南郡公翟無言



 冠軍大將軍、左羽林軍大將軍、光祿卿、天水縣公趙承恩



 將作大匠裴思諒



 右羽林軍將軍、弘農郡公楊執一



 左衛將軍、河東郡公薛思行



 光祿卿、駙馬都尉、瑯邪郡公王同皎



 中書令、越國公鐘紹京



 太僕卿、立節郡王薛崇簡



 右金吾衛大將軍、涼國公李延昌



 太子中允同正、冀國公馮道力



 少府監、趙國公崔諤之



 左監門衛中候、光祿卿、申國公許輔乾



 左金吾大將軍、鄧國公張



 朔方道行軍大總管、左羽林軍大將軍、平陽郡公薛訥



 河南副元帥、太尉侍中、臨淮郡王李光弼



 河東節度副大使、守司空、兼兵部尚書、霍國公王思禮



 左相、豳國公韋見素



 太保、韓國公苗晉卿



 中書令、趙國公崔圓



 太原節度使、檢校尚書左僕射、同中書門下平章事、金城郡王辛云京



 河西隴右副元帥、兵部尚書、同中書門下平章事、涼國公李抱玉



 太子太師、檢校尚書右僕射、知省事、信都郡王田神功



 四鎮北庭涇原節度使、檢校尚書左僕射、知省事、扶風郡王馬璘



 左羽林軍大將軍、檢校戶部尚書、兼御史大夫薛景仙



 右散騎常侍、檢校禮部尚書、兼御史大夫尚衡



 太原尹、兼御史大夫、北都留守、河東節度副大使、南陽郡公鄧景山



 河東節度副使、兼雁門郡太守、光祿卿賈循



 禮部尚書、東京留守、酒泉縣侯李憕



 東平郡太守姚訚



 右第二



 盧弈,黃門監懷慎少子也。疏眉目,豐下,謹重寡欲,斤斤自脩。與兄奐名相上下,而剛毅過之。天寶初為鄠令,所治輒最,積功擢給事中,拜御史中丞。自懷慎、奐及弈,三居其官,清節似之,時傳其美。俄留臺東都,兼知武部選。



 安祿山陷東都,吏亡散。弈前遣妻子懷印間道走京師,自朝服坐臺。被執,將殺之,即數祿山罪,徐顧賊徒曰:「為人臣者當識逆順,我不蹈失節,死何恨?」觀者恐懼。弈臨刑,西向再拜而辭,罵賊不空口,逆黨為變色。肅宗詔贈禮部尚書,下有司謚。時以為洛陽亡,操兵者任其咎,執法吏去之可也,委身寇仇,以死誰懟?博士獨孤及曰:「荀息殺身於晉,不食其言也;玄冥勤其官水死,守位忘躬也;伯姬待姆而火死,先禮後身也。彼死之日,皆於事無補。然則祿山亂大於里、丕,弈廉察之任,切於玄冥之官。分命所系,不啻保姆;逆黨兵威,烈於水火。於斯時也,能與執干戈者同其戮力,挽之不來,推之不去,全操白刃之下,孰與夫懷安偷生者同其風?請謚曰貞烈。」詔可。



 子杞,別有傳。杞子元輔。



 元輔字子望,少以清行聞。擢進士,補崇文校書郎。杞死,德宗念之不忘,拜元輔左拾遺。歷杭、常、絳三州刺史,課當最,召授吏部郎中,進累兵部侍郎,為華州刺史,卒。



 元輔端靜介正,能紹其祖,故歷顯劇,而人不以杞之惡為累雲。



 張介然者,猗氏人,本名六朗。性慎願,長計畫。始為河、隴支郡太守。王忠嗣、皇甫惟明、哥舒翰踵領節度,並署營田、支度等使。入奏稱旨,賜與良渥。介然啟曰:「臣位三品,當給棨戟。若列於京師,雖富貴,不為饗人知,願得列戟故里。」玄宗許之,別賜戟京師第門,仍賜絹五百匹,宴閭里長老。本鄉得列戟,自介然始。翰薦為少府監,歷衛尉卿。



 祿山反,授河南節度採訪使,守陳留。陳留據水陸劇,居民孳伙,而太平久,不知戰。介然到屯不三日,賊已度河。車騎蹂騰,煙塵漫數十里,日為奪色。士聞鉦鼓聲,皆褫氣不能授甲。凡旬六日,城陷。初,有詔購賊首而暴誅慶宗狀。祿山入陳留,見詔書,拊膺大哭曰:「我何罪!吾子亦何罪,乃殺之!」即大恚憤,殺陳留降者萬人以逞,血流成川,斬介然於軍門。以偽將李廷望為節度使,守陳留。



 祿山已拔陳留,則鼓而前,無敢亢。中宿攻滎陽,太守崔無詖率眾乘城,聞師噪,自隊如雨,無詖與官屬皆死賊手。以偽將武令珣戍焉。



 無詖者,本韋後外家,博陵舊望也。始,無詖娶蕭至忠女,至忠敗,被貶。久乃為益州司馬。素善楊國忠,既用事,引為少府監,守滎陽。有詔贈禮部尚書,謚曰毅勇。



\end{pinyinscope}