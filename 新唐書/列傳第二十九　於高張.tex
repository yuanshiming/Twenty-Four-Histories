\article{列傳第二十九 於高張}

\begin{pinyinscope}

 於志寧,字仲謐,京兆高陵人。曾祖謹,有功於周,為太師、燕國公。父宣道無極指無形無象的宇宙原始狀態。《老子·二十八章》:,仕隋至內史舍人。大業末,志寧調冠氏縣長,山東盜起,棄官歸。



 高祖入關,率群從迎謁長春宮,詔授渭北道行軍元帥府記室,與殷開山參謀議。薛仁杲平,識褚亮於囚虜中,遷天策府中郎、文學館學士,引亮與同列。貞觀三年,為中書侍郎。太宗嘗宴近臣,問:「志寧安在?」有司奏:「敕召三品,志寧品第四。」帝悟,特詔預宴,因加散騎常侍、太子左庶子、黎陽縣公。是時議立七廟,君臣請以涼武昭王為始祖,志寧以涼非王業所因,獨建議違之。帝詔功臣世襲刺史,志寧奏:「古今異時,慕虛名,遺實患,非久安計。」帝皆從之。嘗謂志寧曰:「古者太子既生,士負之,即置輔弼。昔成王以周、召為師傅,日聞正道,習以成性。今太子幼,卿當輔以正道,無使邪僻啟其心。勉之,官賞可不次得也。」太子承乾數有過惡,志寧欲救止之,上《諫苑》以諷。帝見大悅,賜黃金十斤、絹三百匹。俄兼詹事,以母喪免,有詔起復本官,固請終喪,帝遣中書侍郎岑文本敦譬曰:「忠孝不兩立,今太子須人教約,卿強起,為我卒輔道之。」志寧乃就職。



 時太子以農時造曲室,累月不止,又好音樂過度。志寧諫,以為「今東宮乃隋所營,當時號為侈麗,豈容復事磨礱彩飾於其間?丁匠官奴皆犯法亡命,鉗鑿槌杵,往來出入,監門、宿衛、直長、千牛不得苛問。爪牙在外,廝役在內,其可無憂乎?又宮中數聞鼓聲,太樂伎兒輒留不出,往年口敕丁寧,殿下可不思之?」太子不納。而左右多任宦官,志寧復諫曰:「奄官者,體非全氣,專柔便佞,托親近為威權,假出納為禍福。故伊戾敗宋,易牙亂齊,趙高亡秦,張讓傾漢。近高齊任鄧長顒為侍中,陳德信為開府,內預宴私,外干朝政,齊卒顛覆。今殿下左右前後皆用寺人,輕忽高班,陵轢貴仕,品命失序,經紀不立,行路之人咸以為怪。」太子益不悅。東宮僕御舊得番休,而太子不聽,又私引突厥,與相狎比。志寧懷不能言,上疏極言曰:「竊見僕寺司馭,爰及獸醫,自春迄夏,不得番息。或家有慈親,以闕溫清,或室有幼弱,以虧撫養,殆非恕愛之意。又突厥達哥支等,人狀野心,不可以禮教期,不可以仁信待。狎而近之,無益令望,有損盛德。況引內閤中,使常親近,人皆震駭,而殿下獨安此乎?」太子大怒,遣張師政、紇干承基往刺之。二人者入其第,見志寧憔然在苫塊中,不忍殺,乃去。太子敗,帝知狀,謂曰:「聞公數諫,承乾不聽公,故至此。」是時宮臣皆罪廢,獨志寧蒙勞勉。



 晉王為皇太子,復拜左庶子,遷侍中,加光祿大夫,進封燕國公,監脩國史。永徽二年,洛陽人李弘泰誣告太尉長孫無忌反,有詔不待時斬之。志寧以為:「方春少陽用事,不宜行刑,且誣謀非本惡逆,請依律待秋分乃決。」從之。衡山公主既公除,將下嫁長孫氏。志寧以為:「《禮》,女十五而笄,二十而嫁,有故,二十三而嫁,固知遇喪須終三年。《春秋》,魯莊公如齊納幣,母喪未再期而圖婚,二家不譏,以其失禮明也。今議者云『公除從吉』,此漢文創制,為天下百姓耳。公主身服斬衰,服可以例除,情不可以例改。心喪成婚,非人情所忍。」於是詔公主待服除乃婚。拜尚書左僕射、同中書門下三品。頃之,兼太子少師。四年,隕石十八於馮翊,高宗問曰:「此何祥也?朕欲悔往脩來以自戒,若何?」志寧對:「《春秋》:『隕石於宋五。』內史過曰:『是陰陽之事,非吉兇所生。』物固有自然,非一系人事。雖然,陛下無災而戒,不害為福也。」俄遷太傅。嘗與右僕射張行成、中書令高季輔俱賜田,志寧奏:「臣家自周、魏來,世居關中,貲業不墜。今行成、季輔始營產土,願以臣有餘賜不足者。」帝嘉之,分其田以與二人。



 顯慶四年,以老乞骸骨,詔解僕射,更拜太子太師,仍同中書門下三品。王皇后之廢,長孫無忌、褚遂良固爭不見從,志寧不敢言。武后以其不右己,銜之,後因殺無忌,坐免官,出為滎州刺史,改華州,聽致仕。卒,年七十八,贈幽州都督,謚曰定。後追復左光祿大夫、太子太師。



 志寧愛賓客,樂引後進,然多嫌畏,不能有所薦達也,為士議所少。凡格式、律令、禮典,皆與論譔,賞賜以巨萬。



 初,志寧與司空李勣修定《本草》並圖,合五十四篇。帝曰:「《本草》尚矣,今復修之,何所異邪?」對曰:「昔陶弘景以《神農經》合雜家《別錄》注銘之,江南偏方,不周曉藥石,往往紕繆,四百餘物,今考正之,又增後世所用百餘物,此以為異。」帝曰:「《本草》、《別錄》何為而二?」對曰:「班固唯記《黃帝內外經》,不載《本草》,至齊《七錄》乃稱之。世謂神農氏嘗藥以拯含氣,而黃帝以前文字不傳,以識相付,至桐、雷乃載篇冊,然所載郡縣,多在漢時,疑張仲景、華佗竄記其語。《別錄》者,魏、晉以來吳普、李當之所記,其言華葉形色,佐使相須,附經為說,故弘景合而錄之。」帝曰:「善。」其書遂大行。



 曾孫休烈。休烈機鑒融敏,善文章,與會稽賀朝萬齊融、延陵包融齊名。開元初,第進士,又擢制科,歷秘書省正字。吐蕃金城公主請文籍四種,玄宗詔秘書寫賜。休烈上疏曰:「戎狄,國之寇;經籍,國之典也。戎之生心,不可以無備。昔東平王求《史記》、諸子,漢不與之,以《史記》多兵謀,諸子雜詭術也。東平,漢之懿戚,尚不示征戰之書,今西戎,國之寇仇,安可貽以經典?且吐蕃之性慓悍果決,善學不回。若達於《書》,則知戰;深於《詩》,則知武夫有師干之試;深於《禮》,則知《月令》有廢興之兵;深於《春秋》,則知用師詭之計;深於文,則知往來書檄之制:此何異假寇兵資盜糧也!臣聞魯秉周禮,齊不加兵;吳獲乘車,楚屢奔命。喪法危邦,可取鑒也。公主下嫁異國,當用夷禮,而反求良書,恐非本意,殆有奸人勸導其中。若陛下慮失其情,示不得已,請去《春秋》。夫《春秋》,當周德既衰,諸侯盛強,征伐競興,情偽於是乎生,變詐於是乎起,有以臣召君、取威定霸之事。誠與之,國之患也。狄固貪婪,貴貨易土,正可錫以錦彩,厚以金玉,無足所求以資其智。」疏入,詔中書門下議。侍中裴光庭曰:「吐蕃不識禮經,孤背國恩,今求哀啟顙,許其降附,漸以《詩》、《書》,陶一聲教,斯可致也。休烈但見情偽變詐於是乎生,不知忠信節義亦於是乎在。」帝曰:「善。」遂與之。累遷起居郎、直集賢殿學士、比部郎中。楊國忠為宰相,斥不附己者,出為中部郡太守。



 肅宗立,休烈奔行在,擢給事中,遷太常少卿,知禮儀事,兼修國史。帝嘗謂曰:「良史者,君舉必書。朕有過失,顧卿何如?」對曰:「禹、湯罪己,其興也勃焉。有德之君不忘規過。」於時經大盜後,史籍燔缺,休烈奏:「《國史》、《開元實錄》、《起居注》及餘書三千八百餘篇藏興慶宮,兵興焚煬皆盡,請下御史核史館所由,購府縣有得者,許上送官。一書進官一資,一篇絹十匹。」凡數月,止獲一二篇,唯韋述以其家藏《國史》百三十篇上獻。中興文物未完,休烈獻《五代論》,討著舊章,天子嘉之。轉工部侍郎,仍脩史。宰相李揆矜己護前,羞與同史任為等列,奏徙休烈為國子祭酒,權留史館脩撰,以卑下之,休烈安然無屑意。乾元初,始詔百官元日、冬至於光順門賀皇后。休烈奏:「周禮有命夫朝人君,命婦朝女君。自顯慶以來,則天皇后甫行此禮,而命婦與百官雜處,在禮不經。」帝罷之。



 代宗嗣位,甄別名品,元載稱其清諒。拜右散騎常侍,兼修國史,加禮儀使,遷太常卿。累進工部尚書,封東海郡公。雖歷清要,不治產。性恭儉仁愛,無喜慍之容。樂賢下善,推轂士甚眾。年老,篤意經籍,嗜學不厭。妻韋卒,天子嘉休烈父子著儒行,詔贈韋國夫人,葬給鹵簿、鼓吹。歲中,休烈亦卒,年八十一。帝為嘆息,贈尚書左僕射,謚曰元,遣謁者就第宣慰,為儒者榮。



 二子:益、肅,及休烈時,相繼為翰林學士。益,天寶初及進士第。肅,終給事中,贈吏部侍郎。



 肅子敖,字蹈中,擢進士,為秘書省校書郎。楊憑、李鄘、呂元膺相繼闢幕府。元和初,拜監察御史,五遷至右司郎中。進給事中、左拾遺。龐嚴為元稹、李紳所厚,與蔣防俱薦為翰林學士。李逢吉誣紳罪逐之,而出嚴為信州刺史,防汀州刺史。敖封還詔書,縉紳意申嚴枉,及駁奏下,乃論貶嚴太輕,眾皆嗤噪。逢吉乃厚敖,三遷至戶部侍郎,出為宣歙觀察使。敖修謹,家世用文學進,初為時所稱,及居官,無所建明,不迕物以自容,名益減。卒,贈禮部尚書。四子;球、珪、瑰、琮,皆清顯。琮知名。



 龐嚴者,字子肅,壽州壽春人。第進士,舉賢良方正,策第一,拜拾遺。辭章峭麗,累遷駕部郎中,知制誥。坐累出。復入,稍遷太常少卿。太和五年,權京兆尹,強幹不阿貴勢,然貪利,溺聲色。卒於官。



 琮字禮用,落魄不事事,以門資為吏,久不調,駙馬都尉鄭顥獨器之。宣宗詔選士人尚公主者,顥語琮曰:「子有美才,不飾細行,為眾毀所抑,能為之乎?」琮許諾。中書舍人李潘知貢舉,顥以琮托之,擢第,授左拾遺。初尚永福公主,主未降,食帝前,以事折匕箸,帝知其不可妻士大夫,更詔尚廣德公主。咸通中,以水部郎中為翰林學士,遷中書舍人。閱五月,轉兵部侍郎、判戶部。八年,同中書門下平章事,進中書侍郎,兼戶部尚書。為韋保衡所構,檢校司空、山南東道節度使,三貶韶州刺史。保衡敗,僖宗以太子少傅召,未幾,復為山南節度使,入拜尚書右僕射。黃巢陷京師,以病臥家,巢欲起為相,琮辭疾,賊迫脅不止,乃曰:「吾死在旦夕,位宰相,義不受污。」賊遂害之。



 高馮,字季輔,以字行,德州蓚人。居母喪,以孝聞。兄元道,仕隋為汲令,縣人反城應賊,殺元道。季輔率其黨與縣人戰,擒之,斬首以祭,賊眾畏伏,更歸附之,至數千人。俄與武陟李厚德將其眾降,授陟州總管府戶曹參軍。



 貞觀初,拜監察御史,彈治不避權要。累轉中書舍人,列上五事,以為:



 今天下大定,而刑未措,何哉?蓋謀猷之臣、臺閣之吏不崇簡易,而昧經遠,故執憲者以深刻為奉公,當官者以侵下為益國。如尚書八坐,人主所責成者也,宜擇溫厚脩絜者任之。敦樸素,革浮偽,使家識慈孝,人知廉恥,過行者被嗤於鄉,不暱者蒙擯於親,自然禮節興矣。



 陛下身帥節儉,而營繕未息,丁匠不能給驅使,又和雇以重勞費。人主所欲,何求而不得。願愛其財,毋使殫;惜其力,毋使弊。畿內數州,京師之本,土狹人庶,儲畜少而科役多,宜蒙優貸,令得休息,強本弱支之義也。至江南、河北,人頗舒閑,宜為差等,均量勞逸。



 公侯勛戚之家邑,入俸稍足以奉養,而貸息出舉,爭求什一,下民化之,競為錐刀,宜加懲革。



 今外官卑品,皆未得祿,故饑寒之切,夷、惠不能全其行。為政之道,期於易從,不恤其匱,而須其廉正,恐巡察歲出,輶軒繼軌,而侵漁不息也。宜及戶口之繁,倉庾且實,稍加稟賜,使得事父母、養妻子,然後督責其效,則官人畢力矣。



 密王元曉等俱陛下懿親,當正其禮。比見帝子拜諸叔,諸叔答拜。爵封既同,當明昭穆,願垂訓正,以為彞法。



 書奏,太宗稱善,進授太子右庶子。數上書言得失,辭誠切至。帝賜鐘乳一劑,曰:「而進藥石之言,朕以藥石相報。」後為吏部侍郎,善銓敘人物,帝賜金背鏡一,況其清鑒焉。



 久之,遷中書令、兼檢校吏部尚書,監脩國史,進爵蓚縣公。永徽初,加光祿大夫、侍中、兼太子少保。感疾歸第,有詔以其兄虢州刺史季通為宗正少卿,視疾,遣中使日候增損。卒,年五十八,贈開府儀同三司、荊州都督,謚曰憲。官給轜車,歸葬於鄉。



 子正業,仕至中書舍人。坐善上官儀,貶嶺表。



 張行成,字德立,定州義豐人。少師事劉炫,炫謂門人曰:「行成體局方正,廊廟才也。」隋大業末,察孝廉,為謁者臺散從員外郎。後為王世充度支尚書。世充平,以隋資補穀熟尉。家貧,代計吏集京師,擢制舉乙科,改陳倉尉。高祖謂吏部侍郎張銳曰:「今選吏豈無才用特達者?朕將用之。」銳言行成,調富平主簿,有能名。召補殿中侍御史,糾劾嚴正。太宗以為能,謂房玄齡曰:「古今用人未嘗不因介紹,若行成者,朕自舉之,無先容也。」



 嘗侍宴,帝語山東及關中人,意有同異。行成曰:「天子四海為家,不容以東西為限,是示人以隘矣。」帝稱善,賜名馬一、錢十萬、衣一稱。自是有大政事,令與議焉。累遷給事中。帝嘗謂群臣:「朕為人主,兼行將相事,豈不是奪公等名?舜、禹、湯、武得稷、蒐、伊、呂而四海安,漢高祖有蕭、曹、韓、彭而天下寧,茲事朕皆兼之。」行成退,上疏曰:「有隋失道,天下沸騰,陛下撥亂反正,拯人塗炭,何周、漢君臣所能比數。雖然,盛德含光,規模宏遠。左右文武誠無將相材,奚用大庭廣眾與之量校,損萬乘之尊,與臣下爭功哉?」帝嘉納之。轉刑部侍郎、太子少詹事。



 太子駐定州監國,謂曰:「吾乃送公衣錦過鄉邪!」令有司祠其先墓。行成薦里人魏唐卿、崔寶權、馬龍駒、張君劼皆以學行聞,太子召見,以其老不可任以事,厚賜遣之。太子使行成詣行在,帝見悅甚,賜勞尤渥。還為河南巡察大使,稱旨,檢校尚書左丞。是歲,帝幸靈州,詔皇太子從。行成諫曰:「皇太子宜留監國,對百寮日決庶務,既為京師重,且示四方盛德。」帝以為忠。遷侍中、兼刑部尚書。



 高宗即位,封北平縣公,監脩國史。時晉州地震不息,帝問之,對曰:「天,陽也,君象;地,陰也,臣象。君宜動,臣宜靜。今靜者顧動,恐女謁用事,人臣陰謀。又諸王、公主參承起居,或伺間隙,宜明設防閑。且晉,陛下本封,應不虛發,伏願深思以杜未萌。」帝然之,詔五品以上極言得失。俄拜尚書左僕射、太子少傅。永徽四年,自三月不雨至五月,行成懼,以老乞身,制答曰:「古者策免,乖罪己之義。此在朕寡德,非宰相咎。」乃賜宮女、黃金器,敕勿復辭。行成固請,帝曰:「公,朕之舊,奈何舍朕去邪?」泫然流涕。行成惶恐,不得已復視事。未幾,卒於尚書省舍,年六十七。詔九品以上就第哭。比斂,三遣使賜內衣服,尚宮宿其家護視。贈開府儀同三司、並州都督,祭以少牢,謚曰定。弘道元年,詔配享高宗廟廷。



 族子易之、昌宗。



 易之幼以門廕仕,累遷尚乘奉御。既冠,頎皙美姿制,音技多所曉通。武后時,太平公主薦其弟昌宗,得侍。昌宗白進易之材用過臣,善治煉藥石。即召見,悅之。兄弟皆幸,出入禁中,傅硃粉,衣紈錦,盛飾自喜。即日拜昌宗雲麾將軍、行左千牛中郎將,易之司衛少卿,賜甲第,帛五百段,給奴婢、橐它、馬牛充入之。不數日,進拜昌宗銀青光祿大夫,賜防閤,同京官朝朔望;追贈父希臧為襄州刺史,母韋、母臧並封太夫人,尚宮問省起居。詔尚書李迥秀私侍臧。昌宗興不旬日,貴震天下。諸武兄弟及宗楚客等爭造門,伺望顏色,親執轡棰,號易之為「五郎」,昌宗「六郎」。又加昌宗右散騎常侍。聖歷二年,始置控鶴府,拜易之為監。久之,更號奉宸府,以易之為令。乃引知名士閻朝隱、薛稷、員半千為供奉。



 後每燕集,則二張諸武雜侍,摴博爭道為笑樂,或嘲詆公卿,淫蠱顯行,無復羞畏。時無檢輕薄者又諂言昌宗乃王子晉後身,後使被羽裳、吹簫、乘寓鶴,裴回庭中,如仙去狀,詞臣爭為賦詩以媚後。後知醜聲甚,思有以掩覆之,乃詔昌宗即禁中論著,引李嶠、張說、宋之問、富嘉謨、徐彥伯等二十有六人譔《三教珠英》。加昌宗司僕卿、易之麟臺監,權勢震赫。皇太子、相王請封昌宗為王,後不聽,遷春官侍郎,封鄴國公,易之恆國公,實封各三百戶。



 後既春秋高,易之兄弟專政,邵王重潤與永泰郡主竊議,皆得罪縊死。御史大夫魏元忠嘗劾奏易之等罪,易之訴於後,反誣元忠與司禮丞高戩約曰:「天子老,當挾太子為耐久朋。」後問:「孰為證左?」易之曰:「鳳閣舍人張說。」翌日庭辯,皆不讎,然元忠、說猶皆被逐。其後易之等益自肆,奸贓狼藉,御史臺劾奏之,乃詔宗晉卿、李承嘉、桓彥範、袁恕己參鞫,而司刑正賈敬言窺望後旨,奏昌宗強市,罪當贖,詔曰可。承嘉、彥範進曰:「昌宗贓四百萬,尚當免官。」昌宗大言曰:「臣有功於國,不應免官。」後問宰相,內史令楊再思曰:「昌宗主煉丹劑,陛下餌之而驗,功最大者也。」即詔釋之,歸罪其兄昌儀、同休,皆貶官。已而後久疾,居長生院,宰相不得進見,惟昌宗等侍側。昌宗恐後不諱,禍且及,乃引支黨日夜與謀為不軌事。然小人疏險,道路皆知之,至有榜其事於衢左者。左臺御史中丞宋璟亟請按攝,後陽許璟,俄詔璟外按幽州都督屈突仲翔,更敕司刑卿崔神慶問狀。神慶妄奏云:「昌宗應原。」璟執奏「昌宗法當斬」。後不答,左拾遺李邕進曰:「璟之言,社稷計也,願可之。」後終不許。



 神龍元年,張柬之、崔玄等率羽林兵迎皇太子入,誅易之、昌宗於迎仙院,及其兄昌期、同休、從弟景雄皆梟首天津橋,士庶歡踴,臠取之,一夕盡。坐流貶者數十人。天寶九載,昌期女上表自言,楊國忠助之,詔復易之兄弟官爵,賜同休一子官。



 贊曰:於志寧諫太子承乾,幾遭賊殺,然未嘗懼,知太宗之明,雖匕首揕胸不愧也。及武后立,不敢出一言,知高宗之昧,雖死無益也。季輔,行成數進諫,然雍容有禮,皆長厚君子哉!



\end{pinyinscope}