\article{列傳第二十七 岑虞李褚姚令狐}

\begin{pinyinscope}

 岑文本,字景仁,鄧州棘陽人。祖善方,後梁吏部尚書,更家江陵。父之象而起作用。兩者的區別是相對的,由於事物聯系的復雜性和,仕隋為邯鄲令,坐為人訟,不得申。文本年十四,詣司隸理冤,辨對哀暢無所詘。眾屬目,命作《蓮華賦》,文成,合臺嗟賞,遂得直。



 性沈敏,有姿儀,善文辭,多所貫綜。郡舉秀才,不應。蕭銑僭號,召為中書侍郎,主文記。河間王孝恭平荊州,其下欲掠奪,文本說孝恭曰:「自隋無道,四海救死,延頸以望真主。蕭氏君臣決策歸命者,意欲去危就安。大王誠縱兵剽系,恐江、嶺以南,向化心沮,狼顧麕驚。不如厚撫荊州,勸未附,陳天子厚惠,誰非王人?」孝恭善之,遽下令止侵略,署文本別駕。從擊輔公祏,典檄符。進署行臺考功郎中。



 貞觀元年,除秘書郎,兼直中書省。太宗既藉田,又元日朝群臣,文本奏《藉田》、《三元頌》二篇,文致華贍。李靖復薦於帝,擢中書舍人。時顏師古為侍郎,自武德以來,詔誥或大事皆所草定。及得文本,號善職,而敏速過之。或策令叢遽,敕吏六七人泚筆待,分口占授,成無遺意。師古以譴罷,溫彥博為請帝曰:「師古練時事,長於文誥,人少逮者,幸得復用。」帝曰:「朕自舉一人,公毋憂。」乃授文本侍郎,專典機要。封江陵縣子。是時,魏王泰有寵,侈第舍,冠諸王。文本上疏,勸崇節儉,陳嫡庶分,宜有抑損。帝善之,賜帛三百段。



 逾年為令,從伐遼東,事一委倚,至糧漕最目、甲兵凡要、料配差序,籌不廢手,由是神用頓耗,容止不常。帝憂曰:「文本今與我同行,恐不與同返矣!」至幽州暴病,帝臨視流涕。卒,年五十一。是夕,帝聞夜嚴,曰:「文本死,所不忍聞。」命罷之。贈侍中、廣州都督,謚曰憲,陪葬昭陵。



 始,文本貴,常自以興孤生,居處卑,室無茵褥幃帟。事母以孝顯,撫弟侄篤恩義。生平故人,雖羈賤必鈞禮。帝每稱其忠謹:「吾親之信之」。晉王為皇太子,大臣多兼宮官,帝欲文本兼攝,辭曰:「臣守一職,猶懼其盈,不願希恩東宮,請一心以事陛下。」帝乃止,但詔五日一參東宮。每進見,太子答拜。始為中書令,有憂色,母問之,答曰:「非勛非舊,責重位高,所以憂也。」有來慶者,輒曰:「今日受吊不受賀。」或勸其營產業,文本嘆曰:「吾漢南一布衣,徒步入關,所望不過秘書郎、縣令耳。今無汗馬勞,以文墨位宰相,奉稍已重,尚何殖產業邪?」故口未嘗言家事。既任職久,賚錫豐饒,皆令弟文昭主之。文昭任校書郎,多交輕薄,帝不悅,謂文本曰:「卿弟多過,朕將出之。」文本曰:「臣少孤,母所鐘念者弟也,不欲離左右。今若外出,母必憂,無此弟,是無老母也!」泣下嗚咽。帝愍其意,召文昭讓敕,卒無過。孫羲。從子長倩。羲,字伯華,第進士,累遷太常博士。坐伯父長倩貶郴州司法參軍。遷金壇令。時弟仲翔為長洲令,仲休為溧水令,皆有治績。宰相宗楚客語本道巡察御史:「毋遺江東三岑。」乃薦羲為汜水令。武后令宰相舉為員外郎者,韋嗣立薦羲,且言惟長倩為累,久不進。後曰:「羲誠材,何諉之拘?」即拜天官員外郎。於是,坐親廢者皆得援而進矣。俄為中書舍人。中宗時,武三思用事,敬暉欲上表削諸武封王者,眾畏三思,不敢為草,獨羲為之,詞誼勁切,由是下遷秘書少監。進吏部侍郎。時崔湜、鄭愔及大理少卿李元恭分掌選,皆以賄聞,獨羲勁廉,為時議嘉仰。帝崩,詔擢右散騎常侍、同中書門下三呂。睿宗立,罷為陜州刺史,再遷戶部尚書。景雲初,復召同三品,進侍中,封南陽郡公。初,節愍太子之難,冉祖雍誣帝及太平公主連謀,賴羲與蕭至忠保護得免,羲監脩《中宗實錄》,自著其事。帝見之,賞嘆,賜物三百段、良馬一匹,下詔褒美。



 時羲兄獻為國子司業,仲翔陜州刺史,仲休商州刺史,兄弟子侄在清要者數十人。羲嘆曰:「物極則反,可以懼矣!」然不能抑退。坐豫太平公主謀誅,籍其家。



 長倩,少孤,為文本鞠愛。永淳中,累官至兵部侍郎、同中書門下平章事。垂拱初,自夏官尚書遷內史,知夏官事。俄拜文昌右相,封鄧國公。武后擅位,喜符瑞事,群臣爭言之。長倩懼,間亦開陳,請改皇嗣為武氏,且為周家儲貳。後順許,賜實封戶五百,加特進、輔國大將軍。鳳閣舍人張嘉福、洛州民王慶之建請以武承嗣為皇太子,長倩謂皇嗣在東宮,不宜更立,與格輔元不署,奏請切責嘉福等。和州浮屠上《大雲經》,著革命事,後喜,始詔天下立大雲寺。長倩爭不可,繇是與諸武忤,罷為武威道行軍大總管,征吐蕃。未至,召還,下獄。來俊臣脅誣長倩與輔元、歐陽通數十族謀反,斬於市,五子同賜死,發暴先墓。睿宗立,追復官爵,備禮改葬。



 輔元者,汴州俊儀人。父處仁,仕隋為剡丞,與同郡王孝逸、繁師玄、靖君亮、鄭祖咸、鄭師善、李行簡、盧協皆有名,號「陳留八俊」。輔元擢明經,累遷殿中侍御史,歷御史中丞、同鳳閣鸞臺平章事。既持承嗣不可,遂及誅。子遵,亦舉明經第,為太常寺太祝,亡命匿中牟十餘年。神龍初,訴父冤,擢累贊善大夫。



 輔無兄希元,洛州司法參軍,同章懷太子注範曄《後漢書》者。



 虞世南,越州餘姚人。出繼叔陳中書侍郎寄之後,故字伯施。性沉靜寡欲,與兄世基同受學於吳顧野王餘十年,精思不懈,至累旬不盥櫛。文章婉縟,慕僕射徐陵,陵自以類己,由是有名。陳天嘉中,父荔卒,世南毀不勝喪。文帝高荔行,知二子皆博學,遣使至其家護視,召為建安王法曹參軍。時寄陷於陳寶應,世南雖服除,仍衣布飯蔬;寄還,乃釋布啖肉。至德初,除西陽王友。陳滅,與世基入隋。世基辭章清勁過世南,而贍博不及也,俱名重當時,故議者方晉二陸。煬帝為晉王,與秦王俊交闢之。大業中,累至秘書郎。煬帝雖愛其才,然疾峭正,弗甚用,為七品十年不徙。世基佞敏得君,日貴盛,妻妾被服擬王者,而世南躬貧約,一不改。宇文化及已弒帝,間殺世基,而世南抱持號訴請代,不能得,自是哀毀骨立。從至聊城,為竇建德所獲,署黃門侍郎。秦王滅建德,引為府參軍,轉記室,遷太子中舍人。王踐祚,拜員外散騎侍郎、弘文館學士。時世南已衰老,屢乞骸骨,不聽,遷太子右庶子,固辭改秘書監,封永興縣子。世南貌儒謹,外若不勝衣,而中抗烈,論議持正。太宗嘗曰:「朕與世南商略古今,有一言失,未嘗不悵恨,其懇誠乃如此!」



 貞觀八年,進封縣公。會隴右山崩,大蛇屢見,山東及江、淮大水,帝憂之,以問世南,對曰:「春秋時,梁山崩,晉侯召伯宗問焉。伯宗曰:『國主山川,故山崩川竭,君為之不舉,降服,乘縵,徹樂,出次,祝幣以禮焉。』梁山,晉所主也,晉侯從之,故得無害。漢文帝元年,齊、楚地二十九山同日崩,水大出,詔郡國無來貢,施惠天下,遠近洽穆,亦不為災。後漢靈帝時,青蛇見御坐。晉惠帝時,大蛇長三百步,見齊地,經市入廟。蛇宜在草野,而入市,此所以為怪耳。今蛇見山澤,適其所居。又山東淫雨,江、淮大水,恐有冤獄枉系,宜省錄累囚,庶幾或當天意。」帝然之,於是遣使賑饑民,申挺獄訟,多所原赦。後星孛虛、危,歷氐,餘百日,帝訪群臣。世南曰:「昔齊景公時,彗見,公問晏嬰,嬰曰:『公穿池沼畏不深,起臺榭畏不高,行刑罰畏不重,是以天見彗為戒耳。』景公懼而修德,後十六日而滅。臣願陛下勿以功高而自矜,勿以太平久而自驕,慎終於初,彗雖見,猶未足憂。」帝曰:「誠然,吾良無景公之過,但年十八舉義兵,二十四平天下,未三十即大位,自謂三王以來,撥亂之主莫吾若,故負而矜之,輕天下士。上天見變,其為是乎?秦始皇劃除六國,隋煬帝有四海之富,卒以驕敗,吾何得不戒邪?」



 高祖崩,詔山陵一準漢長陵故事,厚送終禮,於是程役峻暴,人力告弊。世南諫曰:



 古帝王所以薄葬者,非不欲崇大光顯以榮其親,然高墳厚隴,寶具珍物,適所以累之也。聖人深思遠慮,安於菲薄,為長久計。昔漢成帝造延、昌二陵,劉向上書曰:「孝文居霸陵,淒愴悲懷,顧謂群臣曰:『嗟乎!以北山石為槨,用糸寧絮斮陳漆其間,豈可動哉?』張釋之曰:『使其中有可欲,雖錮南山猶有隙;使無可欲,雖無石槨,又何戚焉?』夫死者無終極,而國家有廢興。孝文寤焉,遂以薄葬。」



 又漢法,人君在位,三分天下貢賦之一以入山陵。武帝歷年長久,比葬,方中不復容物。霍光暗於大體,奢侈過度,其後赤眉入長安,破茂陵取物,猶不能盡。無故聚斂,為盜之用,甚無謂也。



 魏文帝為壽陵,作終制曰:「堯葬壽陵,因山為體,無封樹、寢殿、園邑,棺郭足以藏骨,衣衾足以朽肉。吾營此不食之地,欲使易代之後不知其處。無藏金銀銅鐵,一以瓦器。喪亂以來,漢氏諸陵無不發者,至乃燒取玉匣金縷,骸骨並盡,乃不重痛哉!若違詔妄有變改,吾為戮尸地下,死而重死,不忠不孝,使魂而有知,將不福汝。以為永制,藏之宗廟。」魏文此制,可謂達於事矣。



 陛下之德,堯、舜所不逮,而俯與秦、漢君同為奢泰,此臣所以尤戚也。今為丘隴如此,其中雖不藏珍寶,後世豈及信乎?臣愚以為霸陵因山不起墳,自然高顯。今所卜地勢即平,宜依周制為三仞之墳,明器一不得用金銀銅鐵,事訖刻石陵左,以明示大小高下之式,一藏宗廟,為子孫萬世法,豈不美乎!



 書奏,未報。又上疏曰:「漢家即位之初,便營陵墓,近者十餘歲,遠者五十年。今以數月之程,課數十年之事,其於人力不亦勞矣。漢家大郡,戶至五十萬,今人眾不逮往時,而功役一之,此臣所以致疑也。」時議者頗言宜奉遺詔,於是稍稍裁抑。



 帝嘗作宮體詩,使賡和。世南曰:「聖作誠工,然體非雅正。上之所好,下必有甚者,臣恐此詩一傳,天下風靡。不敢奉詔。」帝曰:「朕試卿耳!」賜帛五十匹。帝數出畋獵,世南以為言,皆蒙嘉納。嘗命寫《列女傳》於屏風,於時無本,世南暗疏之,無一字謬。帝每稱其五絕:一曰德行,二曰忠直,三曰博學,四曰文詞,五曰書翰。世南始學書於浮屠智永,究其法,為世秘愛。



 十二年,致仕,授銀青光祿大夫,弘文館學士如故,祿賜防閤視京官職事者。卒,年八十一,詔陪葬昭陵,贈禮部尚書,謚曰文懿。帝手詔魏王泰曰:「世南於我猶一體,拾遺補闕,無日忘之,蓋當代名臣,人倫準的。今其云亡,石渠、東觀中無復人矣!」後帝為詩一篇,述古興亡,既而嘆曰:「鐘子期死,伯牙不復鼓琴。朕此詩將何所示邪?」敕起居郎褚遂良即其靈坐焚之。後數歲,夢進讜言若平生,翌日,下制厚恤其家。



 子昶,終工部侍郎。



 李百藥,字重規,定州安平人。隋內史令德林子也。幼多病,祖母趙以「百藥」名之。七歲能屬文,父友陸乂等共讀徐陵文,有「刈瑯邪之稻」之語,嘆不得其事。百藥進曰:「《春秋》『鄅子藉稻』,杜預謂在瑯邪。」客大驚,號奇童。引廕補三衛長。乃性疏侻,喜劇飲。開皇初,授太子通事舍人,兼學士。被讒,輒謝病去。十九年,召見仁壽宮,襲父爵安平公。僕射楊素、吏部尚書牛弘愛其才,署禮部員外郎。奉詔定五禮、律令、陰陽書。



 初,以疾去舍人也,煬帝在揚州,召不赴,銜之。及即位,奪爵,為桂州司馬。官廢,還鄉里。大業九年,戍會稽,管崇亂,城守有功,帝顧其名謂虞世基曰:「是子故在,宜斥丑處。」乃授建安郡丞。至烏程,江都難作,沈法興、李子通、杜伏威更相滅,百藥轉側寇亂中,數被偽署,危得不死。會高祖遣使招伏威,百藥勸朝京師,既至歷陽,中悔,欲殺之,飲以石灰酒,因大利,瀕死,既而宿病皆愈。伏威詒書輔公祏使殺之,為王雄誕保護得免。公祏反,授吏部侍郎。或謂帝:「百藥與同反。」帝大怒。及平,得伏威所與公祏書,乃解,猶貶涇州司戶。



 太宗至涇州,召與語,悅之。貞觀元年,拜中書舍人,封安平縣男。明年,除禮部侍郎。時議裂土與子弟功臣,百藥上《封建論》,理據詳切,帝納其言而止。四年,授太子右庶子。太子數戲媟無度,乃作《贊道賦》以諷。它日,帝曰:「朕見卿賦,述古儲貳事,勸勵甚詳,向任卿,固所望耳!」賜彩三百段。遷散騎常侍,進左庶子、宗正卿,爵為子。久之,固乞致仕。帝嘗與偕賦《帝京篇》,嘆其工,手詔曰:「卿何身老而才之壯,齒宿而意之新乎?」卒,年八十四,謚曰康。



 百藥,名臣子,才行世顯,為天下推重。侍父母喪還鄉,徒跣數千里。服雖除,容貌臒瘠者累年。好獎薦後進,得俸祿與親黨共之。翰藻沈鬱,詩尤其所長,樵廝皆能諷之。所撰《齊史》行於時。



 子安期。安期亦七歲屬文。父貶桂州,遇盜,將加以刃,安期跪泣請代,盜哀釋之。貞觀初,為符璽郎。累除主客員外郎。高宗即位,遷中書舍人、司列少常伯,數豫決國事。帝屢責侍臣以不能進賢,眾不敢對。安期進曰:「邑十室且有忠信,天下至廣,不為無賢。比見公卿有所薦進,皆劾為朋黨,滯抑者未申,而主薦者已訾,所以人人爭噤默以避囂謗。若陛下忘其親仇,曠然受之,惟才是用,塞讒毀路,其誰敢不竭忠以聞上乎?」帝納之。尋檢校東臺侍郎、同東西臺三品,出為荊州大都督府長史。卒,謚曰烈。



 自德林至安期,三世掌制誥,孫羲仲,又為中書舍人。



 褚亮,字希明,杭州錢塘人。曾祖湮,父玠,皆有名梁、陳間。亮少警敏,博見圖史,一經目輒志於心。年十八,詣陳僕射徐陵,陵與語,異之。後主召見,使賦詩,江總諸詞人在席,皆服其工。累遷為尚書殿中侍郎。入隋,為東宮學士,遷太常博士。煬帝議改宗廟之制,亮請依古七廟,而太祖、高祖各一殿,法周文、武二祧,與始祖而三,餘則分室而祭,始祖二祧,不從迭毀。未及行,坐與楊玄感善,煬帝矜己嫉才,因是亦貶西海司戶。時博士潘徽貶威定主簿,亮與俱至隴山。徽死,為斂瘞,人皆義之。



 後為薛舉黃門侍郎。舉滅,秦王謂曰:「寡人受命而來,嘉於得賢。公久事無道君,得無勞乎?」亮頓首曰:「舉不知天命,抗王師,今十萬眾兵加其頸,大王釋不誅,豈獨亮蒙更生邪?」王悅,賜乘馬、帛二百段,即授王府文學。高祖獵,親格虎,亮懇愊致諫,帝禮納其言。王每征伐,亮在軍中,嘗預秘謀,有裨輔之益。貞觀中累遷散騎常侍,封陽翟縣侯,老於家。



 太宗征遼,子遂良從,詔亮曰:「疇日師旅,卿未嘗不在中,今朕薄伐,君已老。俯仰歲月,且三十載,眷言及此,我勞如何!今以遂良行,想君不惜一子於朕耳。善居加食。」帝頓首謝。及寢疾,帝遣醫、中使候問踵相逮。卒,年八十八,贈太常卿,陪葬昭陵,謚曰康。遂良自有傳。



 初,武德四年,太宗為天策上將軍,寇亂稍平,乃鄉儒,宮城西作文學館,收聘賢才,於是下教,以大行臺司勛郎中杜如晦、記室考功郎中房玄齡及於志寧、軍諮祭酒蘇世長、天策府記室薛收、文學褚亮姚思廉、太學博士陸德明孔穎達、主簿李玄道、天策倉曹參軍事李守素、王府記室參軍事虞世南、參軍事蔡允恭顏相時、著作郎攝記室許敬宗薛元敬、太學助教蓋文達、軍諮典簽蘇勖,並以本官為學士。七年,收卒,復召東虞州錄事參軍劉孝孫補之。凡分三番遞宿於閤下,悉給珍膳。每暇日,訪以政事,討論墳籍,榷略前載,無常禮之間。命閻立本圖象,使亮為之贊,題名字爵里,號「十八學士」,藏之書府,以章禮賢之重。方是時,在選中者,天下所慕問,謂之「登瀛洲」。



 劉孝孫者,荊州人。祖貞,周石臺太守。孝孫少知名。大業末,為王世充弟杞王辯行臺郎中。辯降,眾引去,獨孝孫攀援號慟,送於郊。貞觀六年,遷著作佐郎、吳王友。歷諮議參軍。遷太子洗馬,未拜,卒。



 李玄道者,本隴西人。世居鄭州。仕隋為齊王府屬。李密據洛口,署記室。密敗,為王世充所執,眾懼不能寐,獨玄道曰:「死生有命,憂能了乎?」寢甚安。及見世充,辭色不撓,釋縛,為著作佐郎。東都平,為秦王府主簿。貞觀初,累遷給事中,姑臧縣男。出為幽州長史,佐都督王君廓,專持府事。君廓不法,每以義裁糾之。嘗遺玄道婢,乃良家子為所掠,遣去不納,由是始隙。君廓入朝,玄道寓書房玄齡,玄齡本甥也。君廓發其書,不識草字,疑以謀己,遂反。坐是流巂州,未幾,擢常州刺史,風績清簡,下詔褒美,賜繒帛。久之,致仕,加銀青光祿大夫,以祿歸第,卒。



 李守素者,趙州人。王世充平,召署天策府倉曹參軍,通氏姓學,世號「肉譜」。虞世南與論人物,始言江左、山東,尚相酬對;至北地,則笑而不答,嘆曰:「肉譜定可畏。」許敬宗曰:「倉曹此名,豈雅目邪?宜有以更之。」世南曰:「昔任彥升通經,時稱『五經笥』,今以倉曹為『人物志』,可乎?」時渭州刺史李淹亦明譜學,守素所論,惟淹能抗之。



 姚思廉,本名簡,以字行,陳吏部尚書察之子。陳亡,察自吳興遷京兆,遂為萬年人。思廉少受《漢書》於察,盡傳其業。寡嗜欲,惟一於學,未嘗問家人生貲。



 仕陳會稽王主簿。入隋,為漢王府參軍事,以父喪免。服除,補河間郡司法書佐。初,察在陳,嘗脩梁、陳二史,未就,死,以屬思廉,故思廉表父遺言,有詔聽續。煬帝又詔與起居舍人崔祖浚脩《區宇圖志》。遷代王侍讀。高祖定京師,府僚皆奔亡,獨思廉侍王,兵將升殿,思廉厲聲曰:「唐公起義,本安王室,若等不宜無禮於王。」眾眙卻,布列階下。帝義之,聽扶王至順陽閤,泣辭去。觀者嘆曰:「仁者有勇,謂此人乎!」俄授秦王府文學。王討徐圓朗,嘗語隋事,慨然嘆曰:「姚思廉蒙素刃以明大節,古所難者。」時思廉在洛陽,遣使遺物三百段,致書曰:「景想節義,故有是贈。」



 王為皇太子,遷洗馬。即位,改著作郎、弘文館學士。詔與魏徵共撰《梁》、《陳書》,思廉採謝炅、顧野王等諸家言,推究綜括,為梁、陳二家史,以卒父業。賜雜彩五百段,加通直散騎常侍。以籓邸恩,凡政事得失,許密以聞,思廉亦展盡無所諱。帝幸九成宮,思廉以為「離宮游幸是秦皇、漢武事,非堯、舜、禹、湯所為」。帝諭曰:「朕嘗苦氣疾,熱即頓劇,豈為游賞者乎?」賜帛五十匹,拜散騎常侍、豐城縣男。卒,贈太常卿,謚曰康,陪葬昭陵。



 孫。



 贊曰:隋煬帝失德,高祖總豪英,興北方,鼓行入關,舉京師,轟若震霆。思廉以諸生侍孱王,奮然陳大義,挫虓虎而奪之氣,勇夫悍心,褫駭自卻,不敢加無禮於其君。誠使有國家者舉不失義,天下其何以抗之哉?宜太宗之尊表云。



 字令璋,少孤,撫昆媦友愛。力學,才辯掞邁。永徽中,舉明經第,補太子宮門郎。以論撰勞,進秘書郎。稍遷中書舍人,封吳興縣男。武后時,擢夏官侍郎。坐從弟敬節叛,貶桂州長史。後方以符瑞自神,取山川草樹名有「武」字者,以為上應國姓,裒類以聞。後大悅,拜檢校天官侍郎,擢文昌左丞、同鳳閣鸞臺平章事。永徽後,左右史唯對仗承旨,仗下謀議不得聞。以帝王謨訓不可闕紀,請仗下所言軍國政要,責宰相自撰,號《時政記》,以授史官。從之。時政有記自始。坐事,降司賓少卿。延載初,拜納言,有司以族犯法,不可為侍臣者,曰:「王敦犯順,導典樞機;嵇康被戮,紹以忠死。是能為累乎?」後曰:「此朕意,卿無恤浮言。」



 證聖初,加秋官尚書。明堂火,後欲避正殿,應天變。奏:「此人火,非天災也。昔宣榭火,周世延;建章焚,漢業昌。且彌勒成佛,七寶臺須臾散壞。聖人之道,隨物示化,況明堂布政之宮,非宗廟,不宜避正殿,貶常禮。」左拾遺劉承慶曰:「明堂所以宗祀,為天所焚,當側身思過,振除前犯。」挾前語以傾後意。後乃更御端門,大酺,燕群臣,與相娛樂,遂造天樞著己功德,命為使,董督之。功費浩廣,見金不足,乃斂天下農器並鑄。以功賜爵一級。後封嵩山,詔總知儀注,為封禪副使。更造明堂,又以使護作,加銀青光祿大夫。大食使者獻師子,曰:「是獸非肉不食,自碎葉至都,所費廣矣。陛下鷹犬且不蓄,而厚資養猛獸哉!」有詔大食停獻。時九鼎成,後欲用黃金塗之。奏:「鼎者,神器,貴質樸,不待外飾。臣觀其上先有五採雜昈,豈待塗金為符曜耶?」後乃止。



 契丹李盡忠盜塞,副梁王武三思為榆關道安撫使。坐累,下遷益州長史。始,蜀吏貪暴,擿發之,無所容貸。後聞,降璽詔慰勞,因謂左右曰:「為二千石清其身者易,使吏盡清者難,唯為兼之。」新都丞硃待闢坐贓應死,待闢所厚浮屠理中謀殺,據劍南。有密告後者,詔窮按。深探其獄,跡疑似皆捕逮,株黨牽聯數千人。獄具,後遣洛州長史宋玄爽、御史中丞霍獻可覆視,無所翻,坐沒入五十餘族,知反流徙者什八以上,道路冤噪。監察御史袁恕己劾奏獄不平,有詔勿治。召拜地官、冬官二尚書。久之,致仕。卒,年七十四,遺令薄葬。贈越州都督,謚曰成。



 弟班。班篤學有立志,擢明經。歷六州刺史,政皆有績,數被褒賜,累封宣城郡公。遷太子詹事,兼左庶子。時節愍太子稍失道,班凡四上書諫。



 其一曰:「臣聞賈誼稱『選天下端士,使與太子居處出入,故太子見正事,聞正言,行正道,左右前後皆正人也。夫習與正人居,不能無正;習與不正人居,不能無不正。教得而左右正,則太子正;太子正,天下定矣』。伏見內置作坊,諸工伎得入宮闈之內、禁衛之所,或言語內出,或事狀外通,小人無知,因為詐偽,有玷盛德。臣望悉出宮內造作付所司。」



 其二曰:「漢文帝身弋綈,足草舄。齊高帝闌檻用銅者,皆易以鐵。經侯帶玉具劍、環佩以過魏太子,太子不視。經侯曰:『魏國亦有寶乎?』太子曰:『主信臣忠,魏之寶也。』經侯委劍佩去,杜門不出。夫聖賢以簡素為貴,皇王以菲薄為德,惟殿下留心恭儉,損省玩好,以訓天下。」



 其三曰:「前世東宮門閤,往來皆有簿籍。殿下時有所須,唯門司宣令,奸偽乘之,因緣增損。近呂升之乃代署宣敕,賴殿下糾發其奸。以後墨令及覆事,並請內印畫署,冀免詐繆。」



 其四曰:「聖人不專其德,賢智必有所師。今司經無學士,供奉無侍讀。宜視膳時奏請其人,俾奉講勸。夫經所以立行修身,史所以諳識成敗,斯急務也。」太子雖稱善,不能用其言。及敗,索宮中,得班諫書,中宗嘉嘆。時宮臣皆得罪,獨班擢右散騎常侍,遷秘書監。睿宗立,拜戶部尚書。所歷定州刺史、尚書官,皆與相繼云。卒,年七十四。



 始,曾祖察嘗撰《漢書訓纂》,而後之注《漢書》者,多竊取其義為己說,班著《紹訓》以發明舊義云。



 令狐德棻,宜州華原人。父熙,隋鴻臚卿。其先乃燉煌右姓。德棻博貫文史。大業末,為藥城長,屬亂,不就官。淮安王神通據太平宮起兵,立總管府,署德棻府記室。高祖入關,引直大丞相府記室。武德初,為起居舍人,遷秘書丞。帝嘗問:「丈夫冠,婦人髻,比高大,何邪?」德棻對曰:「冠髻在首,君之象也。晉之將亡,君弱臣強,故江左士女,衣小而裳大。宋武帝受命,君德尊嚴,衣裳隨亦變改。此近事驗也。」帝然之。



 方是時,大亂後,經藉亡散,秘書湮缺,德棻始請帝重購求天下遺書,置吏稱錄。不數年,圖典略備。又建言:「近代無正史,梁、陳、齊文籍猶可據,至周、隋事多脫損。今耳目尚相及,史有所馮;一易世,事皆汩暗,無所掇拾。陛下受禪於隋,隋承周,二祖功業多在周,今不論次,各為一王史,則先烈世庸不光明,後無傳焉。」帝謂然。於是詔中書令蕭瑀、給事中王敬業、著作郎殷聞禮主魏,中書令封德彞、舍人顏師古主隋,大理卿崔善為、中書舍人孔紹安、太子洗馬蕭德言主梁,太子詹事裴矩、吏部郎中祖孝孫,秘書丞魏徵主齊,秘書監竇璡、給事中歐陽詢、文學姚思廉主陳,侍中陳叔達、大史令庾儉及德棻主周。整振論譔,多歷年不能就,罷之。



 貞觀三年,復詔撰定。議者以魏有魏收、魏澹二家,書為已詳,惟五家史當立。德棻更與秘書郎岑文本、殿中侍御史崔仁師次周史,中書舍人李百藥次齊史,著作郎姚思廉次梁、陳二史,秘書監魏征次隋史,左僕射房玄齡總監。脩撰之原,自德棻發之,書成,賜絹四百匹。遷禮部侍郎,兼修國史。累進爵彭城縣子。轉太子右庶子。太子承乾廢,坐除名為民。召拜雅州刺史,又坐事免。會修晉家史,房玄齡奏起之。預柬凡十有八人,德棻為先進,故類例多所諏定。除秘書少監。



 永徽初,復為禮部侍郎、弘文館學士,監修國史,遷太常卿。高宗嘗召宰相及弘文學士坐中華殿,問:「何脩而王?若而霸?又當孰先?」德棻曰:「王任德,霸任刑。夏、殷、周純用德而王,秦專刑而霸,至漢雜用之,魏、晉以降,王霸兩失。若用之,王為先,而莫難焉。」帝曰:「今茲何為而要?」對曰:「古者為政,清心簡事為本。今天下無虞,年穀豐衍,惟薄賦斂、省征役為要。」又問禹、湯、桀、紂所以興亡,對曰:「《傳》稱:『禹、湯罪己,其興也勃焉;桀、紂罪人,其亡也忽焉。』然二主惑嬖色,戮諫者,造砲烙之刑,此所以亡也。」帝悅,厚賜以答其言。遷國子祭酒、崇賢館學士,爵為公。以金紫光祿大夫致仕。卒,年八十四,謚曰憲。



 時又有鄧世隆、顧胤、李延壽、李仁實皆以史學稱當世。



 鄧世隆者,相州人。隋大業末,王世充兄子太戍河陽,引為賓客。秦王攻洛陽,遺書諭太,世隆報書誇慢。洛陽平,亡命,變姓名,號隱玄先生,棲白鹿山。貞觀初,召授國子主簿,與崔仁師、慕容善行、劉顗、庾安禮、敬播俱為修史學士。世隆內負罪,居不聊。太宗遣房玄齡諭曰:「爾為太作書,各忠其主耳。我為天子,尚甘心匹夫邪?毋有後疑!」改著作佐郎,歷衛尉丞。初,帝以武功定天下,晚始向學,多屬文賦詩,天格贍麗,意悟沖邁。十三年,世隆上疏,請加集錄,帝謙不許。終著作郎。



 顧胤,蘇州吳人。父覽,仕隋秘書學士。胤,永徽中累遷起居郎,兼脩國史,以撰《太宗實錄》勞,加朝散大夫、弘文館學士。論次國史,加朝請大夫,封餘杭縣男。終司文郎中。子琮,武后時為天官侍郎、同鳳閣鸞臺平章事。卒,後曰:「琮不幸,令雖不舉哀,然朕以股肱,特廢視事一日。」



 李延壽者,世居相州。貞觀中,累補太子典膳丞、崇賢館學士。以脩撰勞,轉御史臺主簿,兼直國史。初,延壽父太師,多識前世舊事,常以宋、齊、梁、陳、齊、周、隋天下參隔,南方謂北為「索虜」,北方指南為「島夷」。其史於本國詳,佗國略,往往訾美失傳,思所以改正,擬《春秋》編年,刊究南北事,未成而歿。



 延壽既數與論譔,所見益廣,乃追終先志。本魏登國元年,盡隋義寧二年,作本紀十二、列傳八十八,謂之《北史》;本宋永初元年,盡陳禎明三年,作本紀十、列傳七十,謂之《南史》。凡八代,合二書百八十篇,上之。其書頗有條理,刪落釀辭,過本書遠甚。時人見年少位下,不甚稱其書。遷符璽郎,兼脩國史,卒。



 嘗撰《太宗政典》,調露中,高宗觀之,咨美直筆,賜其家帛五十段,藏副秘閣,仍別錄以賜皇太子云。



 李仁實,魏州頓丘人。官至左史。著《格論》、《通歷》等書,行於時。



 峘,德棻五世孫。天寶末,及進士第。遇祿山亂,去隱南山豹林谷。楊綰微時,數從之游,而峘博學有口辯。綰為禮部侍郎,脩國史,薦峘,自華原尉拜右拾遺,兼史職。累遷起居舍人。撰《玄宗實錄》,屬《起居注》亡散,峘裒掇詔策,備一朝之遺。自開元、天寶間名臣事多漏略,拙於取棄,不稱良史。大歷中,以刑部員外郎判南曹。遷司封郎中,知制誥,兼史館脩撰。德宗立,詔元陵制度務極優厚,當竭帑藏奉用度。峘諫曰:「臣伏讀漢劉向論山陵之誡,良史咨欷。何者?聖賢勤儉,不作無益。昔舜葬蒼梧,弗變其肆;禹葬會稽,不改其列;周武葬畢陌,無丘壟處;漢文葬霸陵,不起山墳。禹非不忠,啟非不順,周公非不悌,景帝非不孝,其奉君親,皆以儉觳為無窮計。宋文公厚葬,《春秋》書華元為不臣;桓魋為石郭,夫子以為不如速朽。由是觀之,有德者葬薄,無德者葬厚,章章可見。陛下仁孝切於聖心,然尊親之義貴合於禮。先帝遺詔,送終之制,一用儉約,不得以金銀緣飾。陛下奉先志,無違物,若務優厚,是咈顧命,盩經誼,臣竊懼之。今赦令甫下,諸條未出,望速詔有司從遺制便。」詔答曰:「朕頃議山陵,荒哀迷謬,以違先旨。卿引據典禮,非唯中朕之失,亦使朕不遺君親於患。敢不聞義而從,奉以終始?雖古遺直,何以加焉!」



 峘在吏部,因尚書劉晏力。時楊炎為侍郎,故峘內德晏,至分闕,以善闕奉晏,惡闕與炎,炎心不平。建中初,峘為禮部侍郎,炎執政,不為憾。炎出故宰相杜鴻漸門下,其子封求弘文生,以托峘,峘謝使者曰:「得公手署,峘得以識。」炎不疑,署送之。峘即日奏言:「宰相迫臣以私,從之負陛下,不從則害臣。」帝以詰炎,炎具道所以然。帝怒曰:「此奸人,無可奈何!」欲殺之,炎苦救解,乃貶衡州別駕。遷刺史。李泌執政,召拜太子右庶子,復為脩撰。



 性愎且介,人人與為怨。孔述睿同脩史,峘忿細故,數侵之,述睿長者,無所校。貞元五年,坐守衡州冒前刺史戶口為己最,竇參素惡之,貶吉州別駕,稍遷刺史。齊映為江西觀察使,按部及州。峘輕映後出先至宰相,今雖屬刺史,自挾所以過映者,至迎謁,頗怏怏。以語其妻,妻曰:「君自視何如人,以白頭走小生前。君不以比見映,雖黜死,我無憾。」映至,峘入謁,從容步進,不襪首屬戎器,映以為恨。去至府,擿峘舉奏前刺史過失無狀,不宜按部,貶衢州別駕。刺史田敦,峘門生也,與峘昧生平,至是迎拜,分俸半以賙給之。在衢十年,順宗立,以秘書少監召,未至,卒。



 初,受詔撰《代宗實錄》,未就,會貶,詔聽在外成書。元和中,其子太僕丞丕獻之。以勞贈工部尚書。



 贊曰:「文本才猷,世南鯁諤,百藥之持論,亮、思廉之邃雅,德棻之辭章,皆治世華採,而淟汩於隋,光明於唐,何哉?蓋天下未嘗無賢,以不用亡;不必多賢,以見用興。夫典章圖史,有國者尤急,所以考存亡成敗,陳諸前而為之戒。方天下初定,德棻首發其議,而後唐之文物粲然,誠知治之本歟!



\end{pinyinscope}