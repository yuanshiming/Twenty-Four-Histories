\article{列傳第一百一十九 卓行}

\begin{pinyinscope}

 元德秀,字紫芝,河南河南人。質厚少緣飾。少孤,事母孝,舉進士,不忍去左右,自負母入京師。既擢第,母亡,廬墓側,食不鹽酪,藉無茵席。服除,以窶困調南和尉,有惠政。黜陟使以聞,擢補龍武軍錄事參軍。



 德秀不及親在而娶,不肯婚,人以為不可絕嗣,答曰:「兄有子,先人得祀,吾何娶為?」初,兄子襁褓喪親,無資得乳媼,德秀自乳之,數日湩流,能食乃止。既長,將為娶,家苦貧,乃求為魯山令。前此墮車足傷,不能趨拜,太守待以客禮。有盜系獄,會虎為暴,盜請格虎自贖,許之。吏白:「彼詭計,且亡去,無乃為累乎?」德秀曰:「許之矣,不可負約。即有累,吾當坐,不及餘人。」明日,盜尸虎還,舉縣嗟嘆。



 玄宗在東都,酺五鳳樓下,命三百里縣令、刺史各以聲樂集。是時頗言帝且第勝負,加賞黜。河內太守輦優伎數百,被錦繡,或作犀象,瑰譎光麗。德秀惟樂工數十人,聯袂歌《于蒍于》。《于蒍于》者,德秀所為歌也。帝聞,異之,嘆曰:「賢人之言哉!」謂宰相曰:「河內人其塗炭乎?」乃黜太守,德秀益知名。



 所得奉祿,悉衣食人之孤遺者。歲滿,笥餘一縑,駕柴車去。愛陸渾佳山水,乃定居。不為墻垣扃鑰,家無僕妾。歲饑,日或不爨。嗜酒,陶然彈琴以自娛。人以酒肴從之,不問賢鄙為酣飫。是時程休、邢宇、宇弟宙、張茂之、李崿、崿族子丹叔、惟岳、喬潭、楊拯、房垂、柳識皆號門弟子。德秀善文辭,作《蹇士賦》以自況。房琯每見德秀,嘆息曰:「見紫芝眉宇,使人名利之心都盡。」蘇源明常語人曰:「吾不幸生衰俗,所不恥者,識元紫芝也。」



 天寶十三載卒,家惟枕履簞瓢而已。潭時為陸渾尉,庀其葬。族弟結哭之慟,或曰:「子哭過哀,禮歟?」結曰:「若知禮之過,而不知情之至。大夫弱無固,性無專,老無在,死無餘,人情所耽溺、喜愛、可惡者,大夫無之。生六十年未嘗識女色、視錦繡,未嘗求足,無茍辭、佚色,未嘗有十畝之地、十尺之舍、十歲之僮,未嘗完布帛而衣,具五味之餐。吾哀之,以戒荒淫貪佞、綺紈粱肉之徒耳。」



 李華兄事德秀,而友蕭穎士、劉迅。及卒,華謚曰文行先生。天下高其行,不名,謂之元魯山。華於是作《三賢論》。或問所長,華曰:「德秀志當以道紀天下,迅當以《六經》諧人心,穎士當以中古易今世。德秀欲齊愚智,迅感一物不得其正,穎士呼吸折節而獲重祿,不易一刻之安易,於孔子之門,皆達者歟!使德秀據師保之位,瞻形容,乃見其仁。迅被卿佐服,居賓友,謀治亂根源,參乎元精,乃見其妙。穎士若百煉之剛,不可屈,使當廢興去就、一生一死間,而後見其節。德秀以為王者作樂崇德,天人之極致,而辭章不稱,是無樂也,於是作《破陣樂辭》以訂商、周。迅世史官,述《禮》、《易》、《書》、《春秋》、《詩》為《古五說》,條貫源流,備古今之變。穎士尤罪子長不編年而為列傳,後世因之,非典訓也。自《春秋》三家後,非訓齊生人不錄。然各有病,元病酒,劉病賞物,蕭病貶惡太亟、獎能太重。若取其節,皆可為人師也。」世謂篤論。



 休,字士美,廣平人。宇字紹宗,宙字次宗,河間人。茂之,字季豐,南陽人。崿字伯高,丹叔字南誠,惟嶽字謨道,趙人。潭字源,梁人。垂,字翼明,清河人。拯,字齊物,隋觀王雄後,舉進士,終右驍衛騎曹參軍。崿擢制科,遷南華令。大水,他縣饑,人至相屬,崿為具𩜾鬻,及去,糗糧送之,吏為立碑。安祿山亂,崿客清河,為乞師平原太守顏真卿,一郡獲全。歷廬州刺史。拯與崿名最著,潭、識以文傳後。



 權皋,字士繇,秦州略陽人,徙潤州丹徒,晉安丘公翼十二世孫。父倕與席豫、蘇源明以藝文相友,終羽林軍參軍。



 皋擢進士第,為臨清尉,安祿山籍其名,表為薊尉,署幕府。皋度祿山且叛,以其猜虐不可諫,欲行,慮禍及親。天寶十四載,使獻俘京師,還過福昌尉仲謨。謨妻,皋妹也,密約以疾召之,謨來,皋陽喑,直視謨而瞑。謨為盡哀,自含斂之。皋逸去,人無知者。吏以詔書還皋母,母謂實死,慟哭感行路,故祿山不之虞,歸其母。皋潛候於淇門,奉侍晝夜南奔,客臨淮,為驛亭保以言冋北方。既度江而祿山反,天下聞其名,爭取以為屬。高適表試大理評事、淮南採訪判官。



 永王舉兵,脅士大夫,皋詭姓名以免。玄宗在蜀聞之,拜監察御史,會母喪,得風痺疾,客洪州,南北梗否,逾年詔命不至。有中人過州,頗求取無厭,南昌令王遘欲按之,謀於皋。皋良久不答,泣曰:「今何由致天子使,而遽欲治之!」掩面去。遘悟,厚謝。浙西節度使顏真卿表為行軍司馬,召拜起居舍人,固辭。嘗曰:「吾潔身亂世,以全吾志,欲持是受名邪?」李季卿為江淮黜陟使,列其高行,以著作郎召,不就。



 自中原亂,士人率度江,李華、柳識、韓洄、王定皆仰皋節,與友善。洄、定常評皋可為宰輔、師保;華亦以為分天下善惡,一人而已。卒,年四十六,洄等制服行哭,詔贈秘書少監。元和中,謚為貞孝。子德輿,至宰相,別傳。



 甄濟,字孟成,定州無極人。叔父為幽、涼二州都督,家衛州,宗屬以伉俠相矜。濟少孤,獨好學,以文雅稱。居青巖山十餘年,遠近伏其仁,環山不敢畋漁。採訪使苗晉卿表之,諸府五闢,詔十至,堅臥不起。



 天寶十載,以左拾遺召,未至而安祿山入朝,求濟於玄宗,授範陽掌書記。祿山至衛,使太守鄭遵意致謁山中,濟不得已為起,祿山下拜鈞禮。居府中,論議正直。久之,察祿山有反謀,不可諫。濟素善衛令齊,因謁歸,具告以誠。密置羊血左右,至夜,若歐血狀,陽不支,舁歸舊廬。祿山反,使蔡希德封刀召之,曰:「即不起,斷其頭見我。」濟色不動,左手書曰:「不可以行。」使者持刀趨前,濟引頸待之,希德歔欷嗟嘆,止刀,以實病告。後慶緒復使強輿至東都安國觀。會廣平王平東都,濟詣軍門上謁泣涕,王為感動。肅宗詔館之三司署,使污賊官羅拜,以愧其心。授秘書郎,或言太薄,更拜太子舍人。



 來瑱闢為陜西襄陽參謀,拜禮部員外郎。宜城楚昭王廟坎地廣九十畝,濟立墅其左。瑱死,屏居七年。大歷初,江西節度使魏少游表為著作郎,兼侍御史,卒。



 濟生子,因其官字曰禮闈、曰憲臺。而禮闈死,憲臺更名逢,幼而孤。及長,耕宜城野,自力讀書,不謁州縣。歲饑,節用以給親里;大穰,則振其餘於鄉黨貧狹者。朋友有緩急,輒出家貲周贍,以義聞。



 逢常以父名不得在國史,欲詣京師自言。元和中,袁滋表濟節行與權皋同科,宜載國史。有詔贈濟秘書少監。而逢與元稹善,稹移書於史館修撰韓愈曰:「濟棄去祿山,及其反,有名號,又逼致之,執不起,卒不污其名。夫辨所從於居易之時,堅其操於利仁之世,而猶選懦者之所不為,蓋怫人之心難,而害己之避深也。至天下大亂,死忠者不必顯,從亂者不必誅,而眷眷本朝,甘心白刃,難矣哉!若甄生,弁冕不加其身,祿食不進其口,直布衣一男子耳。及亂,則延頸受刃,分死不回,不以不必顯而廢忠,不以不必誅而從亂。在古與今,蓋百一焉。」愈答曰:「逢能行身,幸於方州大臣,以標目其先人事,載之天下耳目,徹之天子,追爵其父第四品,赫然驚人,逢與其父俱當得書矣。」由是父子俱顯名。



 陽城,字亢宗,定州北平人,徙陜州夏縣,世為官族。資好學,貧不能得書,求為吏,隸集賢院,竊院書讀之,晝夜不出戶,六年,無所不通。及進士第,乃去隱中條山,與弟堦、域常易衣出。年長,不肯娶,謂弟曰:「吾與若孤煢相育,既娶則間外姓,雖共處而益疏,我不忍。」弟義之,亦不娶,遂終身。



 城謙恭簡素,遇人長幼如一。遠近慕其行,來學者跡接於道。閭里有爭訟,不詣官而詣城決之。有盜其樹者,城過之,慮其恥,退自匿。嘗絕糧,遣奴求米,奴以米易酒,醉臥於路。城怪其故,與弟迎之,奴未醒,乃負以歸。及覺,痛咎謝,城曰:「寒而飲,何責焉?」寡妹依城居,其子四十餘,癡不知人,城常負以出入。始,妹之夫客死遠方,城與弟行千里,負其柩歸葬。歲饑,屏跡不過鄰里,屑榆為粥,講論不輟。有奴都兒化其德,亦方介自約。或哀其餒,與之食,不納。後致糠核數杯,乃受。山東節度府聞城義者,發使遺五百縑,戒使者不令返。城固辭,使者委而去,城置之未嘗發。會里人鄭俶欲葬親,貸於人無得,城知其然,舉縑與之。俶既葬,還曰:「蒙君子之施,願為奴以償德。」城曰:「吾子非也,能同我為學乎?」俶泣謝,即教以書,俶不能業,城更徙遠阜,使顓其習。學如初,慚,縊而死。城驚且哭,厚自咎,為服緦麻瘞之。



 陜虢觀察使李泌數禮餉,城受之。泌欲闢致之府,不起,乃薦諸朝,詔以著作佐郎召,並賜緋魚。泌使參軍事韓傑奉詔至其家,城封還詔,自稱「多病老憊,不堪奔奉,惟哀憐」。泌不敢強。及為宰相,又言之德宗,於是召拜右諫議大夫,遣長安尉楊寧賚束帛詣其家。城褐衣到闕下辭讓,帝遣中人持緋衣衣之,召見,賜帛五十匹。



 初,城未起,縉紳想見風採。既興草茅,處諫諍官,士以為且死職,天下益憚之。及受命,它諫官論事苛細紛紛,帝厭苦,而城浸聞得失且熟,猶未肯言。韓愈作《爭臣論》譏切之,城不屑。方與二弟延賓客,日夜劇飲。客欲諫止者,城揣知其情,強飲客,客辭,即自引滿,客不得已。與酬酢,或醉,僕席上,城或先醉臥客懷中,不能聽客語,無得關言。常以木枕布衾質錢,人重其賢,爭售之。每約二弟:「吾所俸入,而可度月食米幾何,薪菜鹽幾錢,先具之,餘送酒家,無留也。」服用無贏副,客或稱其佳可愛,輒喜,舉授之。有陳萇者,候其得俸,常往稱錢之美,月有獲焉。居位八年,人不能窺其際。



 及裴延齡誣逐陸贄、張滂、李充等,帝怒甚,無敢言。城聞,曰:「吾諫官,不可令天子殺無罪大臣。」乃約拾遺王仲舒守延英閣上疏極論延齡罪,慷慨引誼,申直贄等,累日不止。聞者寒懼,城愈勵。帝大怒,召宰相抵城罪。順宗方為皇太子,為開救,良久得免,敕宰相諭遣。然帝意不已,欲遂相延齡。城顯語曰:「延齡為相,吾當取白麻壞之,哭於廷。」帝不相延齡,城力也。坐是下遷國子司業。引諸生告之曰:「凡學者,所以學為忠與孝也。諸生有久不省親者乎?」明日謁城還養者二十輩,有三年不歸侍者,斥之。簡孝秀德行升堂上,沈酗不率教者皆罷。躬講經籍,生徒斤斤皆有法度。



 薛約者,狂而直,言事得罪,謫連州。吏捕跡,得之城家。城坐吏於門,引約飲食訖,步至都外與別。帝惡城黨有罪,出為道州刺史,太學諸生何蕃、季償、王魯卿、李讜等二百人頓首闕下,請留城。柳宗元聞之,遺蕃等書曰:「詔出陽公道州,僕聞悒然。幸生不諱之代,不能論列大體,聞下執事,還陽公之南也。今諸生愛慕陽公德,懇悃乞留,輒用撫手喜甚。昔李膺、嵇康時,太學生徒仰闕執訴,僕謂訖千百年不可復見,乃在今日,誠諸生見賜甚厚,將亦陽公漸漬導訓所致乎!意公有博厚恢大之德,並容善偽,來者不拒。有狂惑小生,依托門下,飛文陳愚。論者以為陽公過於納污,無人師道。仲尼吾黨狂狷,南郭獻譏;曾參徒七十二人,致禍負芻;孟軻館齊,從者竊屨。彼聖賢猶不免,如之何其拒人也?俞、扁之門,不拒病夫;繩墨之側,不拒枉材;師儒之席,不拒曲士。且陽公在朝,四方聞風,貪冒茍進邪薄之夫沮其志,雖微師尹之位,而人實瞻望焉。與其化一州,其功遠近可量哉!諸生之言,非獨為己也,於國甚宜。」蕃等守闕下數日,為吏遮抑不得上。既行,皆泣涕,立石紀德。



 至道州,治民如治家,宜罰罰之,宜賞賞之,不以簿書介意。月俸取足則已,官收其餘。日炊米二斛,魚一大雚,置甌杓道上,人共食之。州產侏儒,歲貢諸朝,城哀其生離,無所進。帝使求之,城奏曰:「州民盡短,若以貢,不知何者可供。」自是罷。州人感之,以「陽」名子。前刺史坐罪下獄,吏有幸於刺史者,拾不法事告城,欲自脫,城輒搒殺之。賦稅不時,觀察使數誚責。州當上考功第,城自署曰:「撫字心勞,追科政拙,考下下。」觀察府遣判官督賦,至州,怪城不迎,以問吏,吏曰:「刺史以為有罪,自囚於獄。」判官驚,馳入,謁城曰:「使君何罪?我奉命來候安否耳。」留數日,城不敢歸,僕門闔,寢館外以待命。判官遽辭去。府復遣官來按舉,義不欲行,乃載妻子中道逃去。順宗立,召還城,而城已卒,年七十,贈左散騎常侍,賜其家錢二十萬,官護喪歸葬。



 蕃,和州人。事父母孝。學太學,歲一歸,父母不許。間二歲乃歸,復不許。凡五歲,慨然以親且老,不自安,揖諸生去,乃共閉蕃空舍中,眾共狀蕃義行,白城請留。會城罷,亦止。初,硃泚反,諸生將從亂,蕃正色叱不聽,故六館士無受污者。蕃居太學二十年,有死喪無歸者,皆身為治喪。償,魯人。魯卿,第進士,有名。



 司空圖,字表聖,河中虞鄉人。父輿,有風乾。當大中時,盧弘正管鹽鐵,表為安邑兩池榷鹽使。先是,法疏闊,吏輕觸禁,輿為立約數十條,莫不以為宜。以勞再遷戶部郎中。



 圖,咸通末擢進士,禮部侍郎王凝所獎待,俄而凝坐法貶商州,圖感知己,往從之。凝起拜宣歙觀察使,乃闢置幕府。召為殿中侍御史,不忍去凝府,臺劾,左遷光祿寺主簿,分司東都。盧攜以故宰相居洛,嘉圖節,常與游。攜還朝,過陜虢,屬於觀察使盧渥曰:「司空御史,高士也。」渥即表為僚佐。會攜復執政,召拜禮部員外郎,尋遷郎中。



 黃巢陷長安,將奔,不得前。圖弟有奴段章者,陷賊,執圖手曰:「我所主張將軍喜下士,可往見之,無虛死溝中。」圖不肯往,章泣下。遂奔咸陽,間關至河中。僖宗次鳳翔,即行在拜知制誥,遷中書舍人。後狩寶雞,不獲從,又還河中。龍紀初,復拜舊官,以疾解。景福中,拜諫議大夫,不赴。後再以戶部侍郎召,身謝闕下,數日即引去。昭宗在華,召拜兵部侍郎,以足疾固自乞。會遷洛陽,柳璨希賊臣意,誅天下才望,助喪王室,詔圖入朝,圖陽墮笏,趣意野耄。璨知無意於世,乃聽還。



 圖本居中條山王官穀,有先人田,遂隱不出。作亭觀素室,悉圖唐興節士文人,名亭曰休休,作文以見志曰:「休,美也,既休而美具。故量才,一宜休;揣分,二宜休;耄而聵,三宜休;又少也惰,長也率,老也迂,三者非濟時用,則又宜休。」因自目為耐辱居士。其言詭激不常,以免當時禍災云。豫為塚棺,遇勝日,引客坐壙中賦詩,酌酒裴回。客或難之,圖曰:「君何不廣邪?生死一致,吾寧暫游此中哉!」每歲時,祠禱鼓舞,圖與閭里耆老相樂。王重榮父子雅重之,數饋遺,弗受。嘗為作碑,贈絹數千,圖置虞鄉市,人得取之,一日盡。時寇盜所過殘暴,獨不入王官穀,士人依以避難。



 硃全忠已篡,召為禮部尚書,不起。哀帝弒,圖聞,不食而卒,年七十二。圖無子,以甥為嗣,嘗為御史所劾,昭宗不責也。



 贊曰:節誼為天下大閑,士不可不勉。觀皋、濟不污賊,據忠自完,而亂臣為沮計。天下士知大分所在,故傾朝復支。不有君子,果能國乎?德秀以德,城以鯁峭,圖知命,其志凜凜與秋霜爭嚴,真丈夫哉!



\end{pinyinscope}