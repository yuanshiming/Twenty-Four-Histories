\article{本紀第六 肅宗 代宗}

\begin{pinyinscope}

 肅宗文明武德大聖大宣孝皇帝諱亨,玄宗第三子也。母曰元獻皇后楊氏。初名嗣升,封陜王。



 開元四年,為安西大都護。性仁孝,好學,玄宗尤愛之,遣賀知章、潘肅、呂向、皇甫彬、邢等侍讀左右。



 十五年,更名浚,徙封忠王,為朔方節度大使、單于大都護。



 十八年,奚、契丹寇邊,乃以肅宗為河北道行軍元帥,遣御史大夫李朝隱等八總管兵十萬以伐之。居二歲,朝隱等敗奚、契丹於範陽北,肅宗以統帥功遷司徒。二十三年,又更名璵。



 二十五年,皇太子瑛廢死,明年,立為皇太子。有司行冊禮,其儀有中嚴、外辦,其服絳紗。太子曰:「此天子禮也。」乃下公卿議。太師蕭嵩、左丞相裴耀卿請改「外辦」為「外備」,絳紗衣為硃明服,乃從之。二十八年,又更名紹。天寶三載,又更名亨。



 安祿山來朝,太子識其有反相,請以罪誅之,玄宗不聽。祿山反。



 十五載,玄宗避賊,行至馬嵬,父老遮道請留太子討賊,玄宗許之,遣壽王瑁及內侍高力士諭太子,太子乃還。六月丁酉,至渭北便橋,橋絕,募水濱居民得三千餘人,涉而濟。遇潼關散卒,以為賊,與戰,多傷,既而覺之,收其餘以涉,後軍多沒者。夕次永壽縣,吏民稍持牛酒來獻。新平郡太守薛羽、保定郡太守徐聞賊且至,皆棄城走。己亥,太子次保定,捕得羽、,斬之。辛丑,次平涼郡,得牧馬牛羊,兵始振。朔方留後支度副使杜鴻漸、六城水陸運使魏少游、節度判官崔漪、支度判官崔簡金、關內鹽池判官李涵、河西行軍司馬裴冕迎大子治兵於朔方。庚戌,次豐寧,見大河之險,將保之,會天大風,迥趨靈武。七月辛酉,至於靈武。壬戌,裴冕等請皇太子即皇帝位。甲子,即皇帝位於靈武,尊皇帝曰上皇天帝,大赦,改元至德。賜文武官階、勛爵,版授老太守、縣令。裴冕為中書侍郎、同中書門下平章事。甲戌,安祿山寇扶風,太守薛景仙敗之。八月辛卯,張巡及安祿山將李廷望戰於雍丘,敗之。十月辛巳朔,日有食之。癸未,次彭原郡。詔御史諫官論事勿先白大夫及宰相。始鬻爵、度僧尼。房琯為招討西京、防禦蒲潼兩關兵馬元帥,兵部尚書王思禮副之。南軍入於宜壽,中軍入於武功,北軍入於奉天。辛卯,河南節度副使張巡及令狐潮戰於雍丘,敗之。辛丑,房琯以中軍、北軍及安祿山之眾戰於陳濤斜,敗績。癸卯,琯又以南軍戰,敗績。是月,遣永王璘朝上皇帝於蜀郡。璘反,丹徒郡太守閻敬之及璘戰於伊婁埭,死之。十一月辛卯,河西地震。戊午,崔渙為江南宣慰使。郭子儀率回紇及安祿山戰於河上,敗之。史思明寇太原。十二月,安祿山陷魯、東平、濟陰三郡。戊子,給復平原郡二載。安祿山陷潁川,執太守薛願及長史龐堅。是歲,吐蕃陷雋州,嶺南溪獠梁崇牽陷容州。



 二載正月,永王璘陷鄱陽郡。乙卯,安慶緒殺其父祿山。丙寅,河西兵馬使孟庭倫殺其節度使周佖,以武威郡反。乙亥,安慶緒將尹子奇寇睢陽郡,張巡敗之。二月戊子,次於鳳翔。李光弼及安慶緒之眾戰於太原,敗之。丁酉,關西節度兵馬使郭英乂及安慶緒戰於武功,敗績。慶緒陷馮翊郡,太守蕭賁死之。慶緒將蔡希德寇太原。戊戌,庶人璘伏誅。庚子,敦子儀及安慶緒戰於潼關,敗之。壬寅,河西判官崔偁克武威郡,孟庭倫伏誅。甲辰,郭子儀及安慶緒戰於永豐倉,敗之,大將李韶光、王祚死之。三月辛酉,韋見素、裴冕罷。憲部尚書致仕苗晉卿為左相。四月戊寅,郭子儀為關內、河東副元帥。壬午,瘞陣亡者。庚寅,郭子儀及安慶緒李歸仁戰於劉運橋,敗之。五月癸丑,子儀及慶緒將安守忠戰於清渠,敗績。丁巳,房琯罷,諫議大夫張鎬為中書侍郎、同中書門下平章事。六月癸未,尹子奇冠睢陽。丁酉,南充郡民何滔執其太守楊齊曾以反,劍南節度使盧元裕敗之。七月己酉,太白經天。丁巳,安慶緒將安武臣陷陜郡。八月丁丑,焚長春宮。甲申,崔渙罷。張鎬兼河南節度使,都統淮南諸軍事。靈昌郡太守許叔冀奔於彭城。癸巳,大閱。閏月甲寅,安慶緒寇好畤,渭北節度使李光進敗之。丁卯,廣平郡王俶為天下兵馬元帥,郭子儀副之,以朔方、安西、回紇、南蠻、大食兵討安慶緒。辛未,京畿採訪宣慰使崔光遠及慶緒戰於駱谷,敗之。行軍司馬王伯倫戰於苑北,死之。九月丁丑,慶緒陷上黨郡,執節度使程千里。壬寅,廣平郡王俶及慶緒戰於澧水,敗之。癸卯,復京師。慶緒奔於陜郡。尚書左僕射裴冕告太清宮、郊廟、社稷、五陵,宣慰百姓。十月戊申,廣平郡王俶及安慶緒戰於新店,敗之,克陜郡。壬子,復東京,慶緒奔於河北。興平軍兵馬使李奐及慶緒之眾戰於武關,敗之,克上洛郡。吐蕃陷西平郡。癸丑,安慶緒陷睢陽,太守許遠及張巡、鄆州刺史姚訚、左金吾衛將軍南霽雲皆死之。癸亥,給復鳳翔五載,版授父老官。遣太子太師韋見素迎上皇天帝於蜀郡。丁卯,至自靈武,饗於太廟,哭三日。己巳,關內節度使王思禮及安慶緒戰於絳郡,敗之。十一月丙子,張鎬率四鎮伊西北庭行營兵馬使李嗣業、陜西節度使來瑱、河南都知兵馬使嗣吳王祗克河南郡縣。庚子,作九廟神主,告享於長樂殿。十二月丙子,上皇天帝至自白蜀郡。甲寅,苗晉卿為中書侍郎、同中書門下平章事。戊午,大赦。靈武元從、蜀郡扈從官三品以上予一子官,四品以下一子出身。瘞陣亡者,致祭之,給復其家二載。免天下租、庸來歲三之一。禁珠玉、寶鈿、平脫、金泥、刺繡。復諸州及官名。以蜀郡為南京,鳳翔郡為西京,西京為中京。給復潞州五載,並鄧許滑宋五州、雍兵好畤奉先縣二載,益州三載。賜文武官階、勛、爵,父老八十以上版授,加緋衣、銀魚,民酺五日。廣平郡王俶為太尉,進封楚王。苗晉卿為侍中,崔圓為中書令,李麟同中書門下三品。進封子南陽郡王系為趙王,新城郡王僅鼓王,潁川郡王人閑袞王,東陽郡王侹涇王。封子僙為襄王,倕杞王,偲召王,佋興王,侗定王。乙丑,史思明降。壬申,達奚珣等伏誅。



 乾元元年正月戊寅,上皇天帝御宣政殿,授皇帝傳國、受命寶符,冊號曰光天文武大聖孝感皇帝。乙酉,出宮女三千人。庚寅,大閱。二月癸卯,安慶緒將能元皓以淄、青降,以元皓為河北招討使。乙巳,上上皇天帝冊號曰聖皇天帝。丁未,大赦,改元。贈死事及拒偽命者官。成都、靈州扈從三品以上予一子官,五品以上一子出身,六品以下敘進之。免陷賊州三歲稅。賜文武官階、爵。三月甲戌,徙封俶為成王。戊寅,立淑妃張氏為皇后。四月辛亥,祔神主於太廟。甲寅,朝享於太廟,有事於南郊。乙卯,大赦,賜文武官階、勛、爵,天下非租、庸毋輒役使,有能賑貧究寵以官爵,京官九品以上言事,二王、三恪予一子官。史思明殺範陽節度副使烏承恩以反。五月戊子,張鎬罷。乙未,崔圓、李麟罷。太常少卯王璵為中書侍郎、同中書門下平章事。七月,黨項羌寇邊。九月丙子,招討黨項使王仲升殺拓拔戎德。庚寅,郭子儀率李光弼、李嗣業、王思禮、淮西節度使魯炅、興平軍節度使李奐、滑濮節度使許叔冀、平盧兵馬使董秦、鄭蔡節度使季廣琛以討安慶緒。癸巳,大食、波斯寇廣州。十月甲辰,立成王俶為皇太子。大赦。賜文武官階、爵,五品以上子為父後者勛兩轉。舉忠正孝友甚東宮官者。十一月壬申,王思禮及安慶緒戰於相州,敗之。十二月庚戌,戶部尚書李亙都統淮南、江東、江酉節度使。丁卯,史思明陷魏州。



 二年正月己巳,群臣上尊號曰乾元大聖光天文武孝感皇帝。郭子儀及安慶緒戰於愁思岡,敗之。丁丑,祠九宮貴神。戊寅,耕籍田。二月壬戌,中書門下慮囚。三月己巳,皇后親蠶。壬申,九節度之師潰于滏水。史思明殺安慶緒。東京留守崔圓、河南尹蘇震、汝州刺史賈至奔於襄、鄧。郭子儀屯於東京。丁亥,以旱降死罪,流以下原之;流民還者給復三年。甲午,兵部侍郎呂諲同中書門下平章事。乙未,苗晉卿、王璵罷。京兆尹李峴為吏部尚書,中書舍人李揆為中書侍郎,戶部侍郎第五琦:同中書門下平章事。丙申,郭子儀為東畿、山南東、河南等道諸節度防禦兵馬元帥。四月庚子,王思禮及史思明戰於直千嶺,敗之。壬寅,詔減常膳服御,武德中尚作坊非賜蕃客、戎祀所須者皆罷之。五月辛巳,貶李峴為蜀州刺史。七月辛巳,趙王系為天下兵馬元帥,李光弼副之。辛卯,呂諲罷。八月乙巳,襄州防御將康楚元、張嘉廷反,逐其刺史王政。九月甲子,張嘉延陷刑州。丁亥,太子少保崔光遠為荊襄招討、山南東道處置兵馬使。庚寅,史思明陷東京及齊、汝、鄭、滑四州。十月乙巳,李光弼及史思明戰於河陽,敗之。壬戌,呂諲起復。十一月庚午,貶第五琦為忠州刺史。十二月乙巳,康楚元伏誅。史思明寇陜州,神策軍將衛伯玉敗之。



 上元元年三月丙子,降死罪,流以下原之。四月戊申,山南東道將張維瑾反,殺其節度使史翽。丁巳,有彗星出於婁、胃。己未,來瑱為山南東道節度使,以討張維瑾。閏月辛酉,有彗星出於西方。甲戌,徙封系為越王。己卯,大赦,改元,賜文武官爵。追封太公望為武成王。復死刑三覆奏。是月,大饑。張維瑾降。五月丙午,太子太傅苗晉卿為侍中。壬子,呂諲罷。六月乙丑,鳳翔節度使崔光遠及羌、渾、黨項戰於涇、隴,敗之。乙酉,又敗之於普潤。李光弼及史思明戰於懷州,敗之。七月丁未,聖皇天帝遷於西內。十一月甲午,揚州長史劉展反,陷潤州。丙申,陷升州。壬子,李峘、淮南節度使鄧景山及劉展戰於淮上,敗績。是歲,吐蕃陷廓州。西原蠻寇邊,桂州經略使邢濟敗之。



 二年正月甲寅,降死罪,流以下原之。乙卯,劉展伏誅。二月己未,奴剌、黨項羌寇寶雞,焚大散關,寇鳳州,刺史蕭心曳死之,鳳翔尹李鼎敗之。戊寅,李光弼及史思明戰於北邙,敗績。思明陷河陽。癸未,貶李揆為袁州長史。河中節度使蕭華為中書侍郎、同中書門下平章事。乙酉,來瑱及史思明戰於魯山,敗之。三月甲午,史朝義寇陜州,神策軍節度使衛伯玉敗之。戊戌,史朝義殺其父思明。李光弼罷副元帥。四月己未,吏部侍郎裴遵慶為黃門侍郎、同中書門下平章事。乙亥,青密節度使尚衡及史朝義戰,敗之。丁丑,袞鄆節度使能元皓又敗之。壬午,劍南東川節度兵馬使段子璋反,陷綿州,遂州刺史嗣虢王巨死之,節度使李奐奔於成都。五月甲午,史朝義將令狐彰以滑州降。戊戌,平盧軍節度使侯希逸及史朝義戰於幽州,敗之。庚子,李光弼為河南道副元帥。劍南節度使崔光遠克東川,段子璋伏誅。七月癸未朔,日有食之。八月辛巳,殿中監李國貞都統朔方、鎮西、北庭、興平、陳鄭、河中節度使。九月壬寅,大赦,去「乾元大聖光天文武孝感」號,去「上元」號,稱元年,以十一月為歲首,月以斗所建辰為名。賜文武官階、勛、爵,版授侍老官,先授者敘進之。停四京號。



 元年建子月癸巳,曹州刺史常休明及史朝義將薛崿戰,敗之。己亥,朝聖皇天帝於西內。丙午,衛伯玉及史朝義戰於永寧,敗之。己酉,朝獻於太清宮。庚戌,朝享於太廟及元獻皇后廟。建丑月辛亥,有事於南郊。己未,來瑱及史朝義戰於汝州,敗之。乙亥,侯希逸及朝義將李懷仙戰於範陽,敗之。寶應元年建寅月甲申,追冊靖德太子琮為皇帝,妃竇氏為皇后。乙酉,葬王公妃主遇害者。丙戌,盜發敬陵、惠陵。甲辰,李光弼克許州。吐蕃請和。戊申,史朝義陷營州。建卯月辛亥,大赦。賜文武官階、爵。五品以上清望及郎官、御史薦流人有行業情可矜者。停貢鷹、鷂、狗、豹。以京兆府為上都,河南府為東都,鳳翔府為西都,江陵府為南都,太原府為北都。壬子,羌、渾、奴剌寇梁州。癸丑,河東軍亂,殺其節度使鄧景山,都知兵馬使辛云京自稱節度使。乙丑,河中軍亂,殺李國貞及其節度使荔非元禮。戊辰,淮西節度使王仲升及史朝義將謝欽讓戰於申州,敗績。庚午,敦子儀知朔方、河中、北庭、潞儀澤沁節度行營,興平、定國軍兵馬副元帥。壬申,鄜州刺史成公意及黨項戰,敗之。建辰月壬午,大赦,官吏聽納贓免罪,左降官及流人罰鎮效力者還之。甲午,奴剌寇梁州。戊申,蕭華罷。戶部侍郎元載同中書門下平章事。建巳月庚戌,史朝義寇澤州,刺史李抱玉敗之。壬子,楚州獻定國寶玉十有三。甲寅,聖皇天帝崩。乙丑,皇太子監國。大赦,改元年為寶應元年,復以正月為歲首,建巳月為四月。丙寅,閑廄使李輔國、飛龍廄副使程元振遷皇后於別殿,殺越王系、兗王人閑。是夜,皇帝崩於長生殿,年五十二。



 代宗睿文孝武皇帝諱豫,肅宗長子也。母曰章敬皇后吳氏。玄宗諸孫百餘人,代宗最長,為嫡皇孫。聰明寬厚,喜慍不形於色,而好學強記,通《易》象。初名俶,封廣平郡王。



 安祿山反,玄宗幸蜀肅宗,留討賊,代宗常從於兵間。肅宗已即位,郭子儀等兵討安慶緒,未克。肅宗在岐,至德二載九月,以廣平郡王為天下兵馬元帥,率朔方、安西、回紇、南蠻、大食等兵二十萬以進討,百官送於朝堂,過闕而下,步出木馬門,然後復騎,以安西、北庭行營節度使李嗣業為前軍,朔方、河西、隴右節度使郭子儀為中軍,關內行營節度使王思禮為後軍,屯於香積寺。敗賊將安守忠,斬首六萬級。賊將張通儒守長安,聞守忠敗,棄城走,遂克京城,乃留思禮屯於苑中,代宗率大軍以東。安慶緒遣其將嚴莊拒於陜州,代宗及子儀、嗣業戰陜西,大敗之,安慶緒奔於河北,遂克東都。肅宗還京師。十二月,進封楚王。乾元元年三月,徙封成王。四月,立為皇太子。初,太子生之歲,豫州獻嘉禾,於是以為祥,乃更名豫。



 肅宗去上元三年號,止稱元年,月以斗所建辰為名。元年建巳月,肅宗寢疾,乃詔皇太子監國。而楚州獻定國寶十有三,因曰:「楚者,太子之所封,今天降寶於楚,宜以建元。」乃以元年為寶應元年。



 肅宗張皇后惡李輔國,欲圖之,召問太子,太子不許,乃與越王系謀之。肅宗疾革。四月丁卯,皇后與系將召太子入宮,飛龍副使程元振得其謀,以告輔國。輔國止太子無人,率兵入,殺系及袞王人閑,幽皇后於別殿。是夕,肅宗崩,乃迎太子見群臣於九仙門。明日,發喪。己巳,即皇帝位於柩前。癸酉,始聽政。甲戌,奉節郡王適為天下兵馬元帥,郭子儀罷副元帥。乙亥,進封適為魯王。五月壬午,李輔國為司空。庚寅,追尊母為皇太后。丙申,李光弼及史朝義戰於宋州,敗之。丁酉,大赦。刺史予一子官,賜文武官階、爵,子為父後者勛一轉。免民逋租宿負。進封子益昌郡王邈為鄭王,延慶郡王迥韓王。追復庶人王氏為皇后,瑛、瑤、琚皆復其封號。六月辛亥,追廢皇后張氏、越王系、袞王人閑皆為庶人。七月乙酉,殺山南東道節度使裴奰。癸巳,劍南西川兵馬使徐知道反。八月己未,知道伏誅。辛未,臺州人袁晁反。乙亥,徙封適為雍王。九月戊子,鳳州刺史呂日將及黨項羌戰於三嗟谷,敗之。丙申,回紇請助戰。壬寅,大閱。癸卯,袁晁陷信州。十月乙卯,陷溫、明二州。詔浙江水旱,百姓重困,州縣勿輒科率,民疫死不能葬者為瘞之。辛酉,雍王適討史朝義。壬戌,盜殺李輔國。癸酉,雍王適克懷州。甲戌,敗史朝義於橫水,克河陽、東都,史朝義將張獻誠以汴州降。十一月丁亥,朝義將薛嵩以相、衛、洺、邢四州降。丁酉,朝義將張忠志以趙、定、深、恆、易五州降。己亥,朔方行營節度使僕固懷恩為朔方、河北副元帥。十二月己酉,太府左藏庫火。戊辰,瘞京城內外暴骨。甲戌,李光弼及袁晁戰於衢州,敗之。是歲,舒州人楊昭反,殺其刺史劉秋子。西原蠻叛。吐蕃寇秦、成、渭三州。



 廣德元年正月癸未,京兆尹劉晏為吏部尚書、同中書門下平章事。甲申,史朝義自殺,其將李懷仙以幽州降,田承嗣以魏州降。壬寅,山陵使、山南東道節度使來瑱有罪,伏誅。三月甲辰,山南東道兵馬使梁崇義自南陽入於襄州。丁未,李光弼及袁晁戰,敗之。辛酉,葬至道大聖大明孝皇帝於泰陵。甲子,黨項羌寇同州,郭子儀敗之於黃堆山。庚午,葬文明武德大聖大宣孝皇帝於建陵。六月,同華節度使李懷讓自殺。七月壬寅,群臣上尊號曰寶應元聖文武孝皇帝。壬子,大赦,改元。免民逋負,戶三丁免其一庸、調;給復河北三年;回紇行營所經,免今歲租。賜內外官階、勛、爵。給功臣鐵券,藏名於太廟,圖形於凌煙閣。吐蕃陷隴右諸州。八月,僕固懷恩反。九月壬寅,裴遵慶宣慰僕固懷恩於汾州。乙丑,涇州刺史高暉叛附於吐蕃。十月庚午,吐蕃陷邠州。辛未,寇奉天、武功,京師戒嚴。壬申,雍王適為關內兵馬元帥,郭子儀副之。癸酉,渭北行營兵馬使呂日將及吐蕃戰於盩厔,敗之。乙亥,又戰於盩厔,敗績。丙子,如陜州。丁丑,次華陰。豐王珙有罪,伏誅。戊寅,吐蕃陷京師,立廣武郡王承宏為皇帝。辛巳,次陜州。癸巳,吐蕃潰,郭子儀復京師。南山五穀人高玉反。十一月壬寅,廣州市舶使呂太一反,逐其節度使張休。十二月辛未,劉晏宣慰上都。甲午,至自陜州。乙未,苗惡卿、裴遵慶罷。檢校禮部尚書李峴為黃門侍郎、同中書門下平章事。丙申,放承宏於華州。吐蕃陷松、維二州。西原蠻陷道州。



 二年正月丙午,詔舉堪御史、諫官、刺史、縣令者。乙卯,立雍王適為皇太子。癸亥,劉晏、李峴罷。右散騎常侍王縉為黃門侍郎,太常卿杜鴻漸為兵部侍郎:同中書門下平章事。郭子儀兼河東副元帥。二月辛未,僕固懷恩殺朔方軍節度留後渾釋之。癸酉,朝獻於太清宮。甲戌,朝享於太廟。乙亥,有事於南郊。己丑,大赦。賜內外官階、爵;武德功臣子孫予一人官;成都、靈武元從三品以上加賜爵一級,餘加一階;寶應功臣三品以上官一子,仍賜爵一極,餘加階、勛兩轉,五品以上為父後者勛兩轉。三月辛丑,給復河南府二年。甲子,盛王琦薨。四月甲午,禁鈿作珠翠。五月,洛水溢。六月丁卯,有星隕於汾州。七月庚子,初稅青苗。己酉,李光弼薨。八月丙寅,王縉為侍中,都統河南、淮南、山南東道節度行營事。壬申,王縉罷侍中。癸巳,吐蕃寇邠州,邠寧節度使白孝德敗之於宜祿。九月己未,劍南節度使嚴武及吐蕃戰於當狗城,敗之。是秋,有蜮。十月丙寅,吐蕃寇邠州。丁卯,寇奉天,京師戒嚴。庚午,嚴武克吐蕃鹽川城。辛未,朔方兵馬使郭晞及吐蕃戰於邠西,敗之。是月,突厥寇豐州,守將馬望死之。十一月乙未,吐蕃軍潰,京師解嚴。河西節度使楊志烈及僕固懷恩戰於靈州,敗績。癸丑,袁晁伏誅。免越州今歲田租之半,給復溫、臺、明三州一年。十二月乙丑,高玉伏誅。丙寅,眾星隕。是歲,西原蠻陷邵州。



 永泰元年正月癸巳,大赦,改元。是月,歙州人殺其刺史龐浚。二月戊寅,黨項羌寇富平。庚辰,儀王璲薨。三月庚子,雨木冰。庚戌,吐蕃靖和。辛亥,大風拔木。四月己巳,自春不雨,至於是而雨。是夏,盩厔穞麥生。七月辛卯,平盧、淄青兵馬使李懷玉逐其節度使侯希逸。八月庚辰,王縉為河南副元帥。僕固懷恩及吐蕃、回紇、黨項羌、渾、奴剌寇邊。九月庚寅,命百官觀浮屠象於光順門。辛卯,太白經天。甲辰、吐蕃寇醴泉、奉天,黨項羌寇同州,渾、奴剌寇盩厔,京師戒嚴。己酉,屯於苑,郭子儀屯於涇陽。丁巳,同華節度使周智光及吐蕃戰於澄城,敗之。智光入於鄜州,殺其刺史張麟,遂焚坊州。十月,沙陀殺楊志烈。己未,吐蕃至分阜州,與回紇寇邊。辛酉,寇奉天。癸亥,寇同州。乙丑,寇興平。丁卯,回紇、黨項羌請降。癸酉,郭子儀及吐蕃戰於靈臺,敗之。京師解嚴。閏月辛卯,朔方副將李懷光克靈州。辛亥,劍南西山兵馬使崔旴反,寇成都,節度使郭英乂奔於靈池,普州刺史韓澄殺之。癸丑,斂民貲作浮屠供。



 大歷元年二月,吐蕃遣使來朝。壬子,杜鴻漸為山南西道、劍南東西川、邙南、西山等道副元帥。三月癸未,劍南東川節度使張獻誠及崔旴戰於梓州,敗績。七月癸酉,洛水溢。九月辛巳,吐蕃陷原州。十一月甲子,大赦,改元,給復流民歸業者三年。十二月己亥,有彗星出於瓠瓜。癸卯,周智光反,殺虢州刺史龐充。是冬,無雪。鄭王邈為天下兵馬元帥。



 二年正月丁巳,郭子儀討周智光。己未,同華將李漢惠以同州降。甲子,周智光伏誅。淮西節度使李忠臣入於華州。戊寅,給復同、華二州二年。八月壬寅,殺附馬都尉姜慶初。九月甲寅,吐蕃寇靈州。乙卯,寇邠州。郭子儀屯於涇陽,京師戒嚴。乙丑,晝有星流於南方。是秋,桂州山獠反。十月戊寅,朔方軍節度使路嗣恭及吐蕃戰於靈州,敗之。京師解嚴。十一月辛未,雨木冰。壬申,京師地震。三年二月癸巳,商州兵馬使劉洽殺其刺史殷仲卿。三月乙巳朔,日有食之。五月乙卯,追號齊王倓為皇帝,興信公主女張氏為皇后。癸亥,地震。六月壬寅,幽州兵馬使硃希彩殺其節度使李懷仙,自稱留後。閏月庚午,王縉兼幽州盧龍軍節度使。七月壬申,濾州刺史楊子琳反,陷成都,劍南節度留後崔寬敗之,克成都。子琳殺夔州別駕張忠。戊寅,吐蕃遣使來朝。八月己酉,吐蕃寇靈州。丁卯,寇邠州,京師戒嚴。戊辰,邠寧節度使馬璘及吐蕃戰,敗之。庚午,王縉兼河東節度使。九月丁丑,濟王環薨。壬午,吐蕃寇靈州,朔方將白元光敗之。壬辰,又敗之於靈武。戊辰,京師解嚴。十二月辛酉,涇原兵馬使王童之謀反,伏誅。



 四年正月甲戌,殺潁州刺史李岵。二月乙卯,杜鴻漸罷副元帥。丙辰,京師地震。三月,遣御史稅商錢。甲戌,免京兆今歲稅。五月丙戌,京師地震。六月戊申,王縉罷副元帥、都統。七月癸未,降死罪,流以下原之。十月丁巳,大霧。十一月辛未,禁畿內弋獵。壬申,杜鴻漸罷。癸酉,元載權知門下省事。甲戌,吐蕃寇靈州,朔方軍節度留後常謙光敗之。丙子,左僕射裴冕同中書門下平章事。癸巳,裴冕兼河南、淮西、山南東道副元帥。十二月戊戌,裴冕薨。是歲,廣州人馮崇道、桂州人硃濟時反,容管經略使王翃敗之。



 五年正月辛卯,鳳翔節度使李抱玉為河西、隴右、山南西道副元帥。三月癸酉,內侍監魚朝恩有罪自殺。丙戌,以昭陵皇堂有光,赦京兆、關輔。四月庚子,湖南兵馬使臧玠殺其團練使崔灌。己未,有彗星出於五車。五月己卯,有彗星出於北方。六月己未,以彗星滅,降死罪,流以下原之。錄魏徵、王珪、李靖、李勣、房玄齡、杜如晦之後。是歲,湖南將王國良反,及西原蠻寇州縣。



 六年二月壬寅,李抱玉罷山南西道副元帥。三月,王翃敗梁崇牽,克容州。四月戊寅,藍田西原地陷。禁大繝、竭鑿六破錦及文紗吳綾為龍、鳳、麒麟、天馬、闢邪者。五月戊申,殺殿中侍御史陸珽、成都府司錄參軍事李少良、大理評事韋頌。



 七年二月庚午,江水泛溢。五月乙酉,大雨雹,大風拔木。乙未,以旱大赦,減膳,徹樂。是秋,幽州盧龍將李懷瑗殺其節度使硃希彩,經略軍副使硃泚自稱留後。十月乙亥,以淮南旱,免租、庸三之二。十一月庚辰,免巴、蓬、渠、集、壁、充、通、開八州二歲租、庸。十二月丙寅,雨土,有長星出於參。



 八年正月甲辰,詔京官三品以上及郎官、御史歲舉刺史、縣令一人。五月辛卯,鄭王邈薨。壬辰,赦京師。癸卯,降死罪,流以下原之。八月己未,吐蕃寇靈州,郭子儀敗之於七級渠。甲子,廢華州屯田給貧民。九月壬午,循州刺史哥舒晃反,殺嶺南節度使呂崇賁。戊子,詔京官五品以上、兩省供奉官、郎官、御史言事。十月庚申,吐蕃寇涇、邠。丙寅,朔方兵馬使渾瑊及吐蕃戰於宜祿,敗績。涇原節度使馬璘及吐蕃戰於潘原,敗之。



 九年二月辛未,徐州兵亂,逐其刺史梁乘。四月壬辰,大赦。十月壬申,信王瑝薨。乙亥,涼王璿薨。壬辰,降京師死罪,流以下原之。



 十年正月丁酉,昭義軍兵馬使裴志清逐其節度使薛崿,叛附於田承嗣。壬寅,壽王瑁薨。戊申,田承嗣反。癸丑,承嗣陷洺州。乙卯,劍南西川節度使崔寧及吐蕃戰於西山,敗之。二月乙丑,田承嗣陷衛州,刺史薛雄死之。辛未,封子述為睦王,逾郴王,連恩王,遘鄜王,造忻王,暹韶王,運嘉王,遇端王,遹循王,通恭王,逵原王,逸雅王。丙子,河陽軍亂,逐三城使常休明。三月甲午,陜州軍亂,逐其觀察使李國清。四月癸未,河東節度使薛兼訓等討田承嗣。給復昭義五州二年。甲申,大雨雹,大風拔木。五月乙未,魏博將霍榮國以礠州降。甲寅,大雨雹,大風拔木,震闕門。六月甲戌,成德軍節度使李寶臣及田承嗣戰於冀州,敗之。七月己未,杭州海溢。八月巳丑,田承嗣寇礠州。九月壬寅,降京師死罪,流以下原之。壬子,吐蕃寇臨涇。癸丑,寇隴州。丙辰,李抱玉敗之於義寧。丁巳,馬璘又敗之於百里城。十月辛酉朔,日有食之。甲子,昭義軍節度使李承昭及田承嗣戰於清水,敗之。丙寅,貴妃獨孤氏薨。丁卯,追冊為皇后。十一月丁酉,魏博將吳希光以瀛州降。丁未,嶺南節度使路嗣恭克廣州,哥舒晃伏誅。



 十一年正月庚寅,田承嗣降。辛亥,崔寧及吐蕃戰,敗之。五月,汴宋都虞候李靈耀反,殺濮州刺史孟鑒。七月庚寅,田承嗣寇滑州,永平軍節度使李勉敗績。八月甲申,淮西節度使李忠臣、河陽三城使馬燧及李勉討李靈耀。閏月丁酉,太白晝見。九月乙丑,李忠臣、馬燧及李靈耀戰於鄭州,敗績。十月乙酉,戰於中牟,敗之。壬辰,忠臣又敗之於西梁固。壬寅,淮南節度使陳少游及李靈耀戰於汴州,敗之。丙子,田承嗣以兵援靈耀,李忠臣敗之於匡城。甲寅,靈耀伏誅。



 十二年三月庚午,赦田承嗣。辛巳,元載有罪伏誅。貶王縉為括州刺史。四月壬午,太常卿楊綰為中書侍郎,禮部侍郎常袞為門下侍郎:同中書門下平章事。癸巳,詔諫官獻封事勿限時,側門論事者隨狀面奏,六品以上官言事投匭者無勒副章。丁酉,吐蕃寇黎、雅二州,崔寧敗之。是月,金州人卓英璘反。六月乙巳,英璘伏誅。給復金州二年。丁未,以旱降京師死罪,流以下原之。七月己巳,楊綰薨。丙子,詔尚書、御史大夫、左右丞、侍郎舉任刺史者。九月庚午,吐蕃寇坊州。是秋,河溢。十一月壬子,山南西道節度使張獻恭及吐蕃戰於岷州,敗之。十二月丁亥,崔寧及吐蕃戰於西山,敗之。



 是歲,恆、定、趙三州地震。冬,無雪。十三年正月戊辰,回紇寇並州。癸酉,河東節度留後鮑防及回紇戰於陽曲,敗績。二月庚辰,代州刺史張光晟成回紇戰於羊虎谷,敗之。四月甲辰,吐蕃寇靈州,常謙光敗之。十月己丑,禁京畿持兵器捕獵。是歲,郴州黃芩山崩。



 十四年二月癸未,魏博節度使田承嗣卒,其兄子悅自稱留後。三月丁未,汴宋將李希烈逐其節度使李忠臣,自稱留後。五月辛酉,不豫,詔皇太子監國。是夕,皇帝崩於紫宸內殿,年五十三。



 贊曰:天寶之亂,大盜遽起,天子出奔。方是時,肅宗以皇太子治兵討賊,真得其職矣!然以僖宗之時,唐之威德在人,紀綱未壞,孰與天寶之際?而僖宗在蜀,諸鎮之兵糾合戮力,遂破黃巢而復京師。由是言之,肅宗雖不即尊位,亦可以破賊矣。蓋自高祖以來,三遜於位以授其子,而獨睿宗上畏天戒,發於誠心,若高祖、玄宗,豈其志哉!代宗之時,餘孽猶在,平亂守成,蓋亦中材之主也!



\end{pinyinscope}