\article{列傳第九十五}

\begin{pinyinscope}
鄧繼曾劉最朱淛馬明衡陳逅林應聰楊言劉安薛侃喻希禮石金楊名黃直郭弘化劉世龍徐申羅虞臣張選{{黃正色包節弟孝謝廷王與齡周鈇楊思忠樊深凌儒王時舉方新

鄧繼曾,字士魯,資縣人。正德十二年進士。授行人。世宗即位之四月,以久雨,疏言:「明詔雖頒,而廢閣大半。大獄已定,而遲留尚多。擬旨間出於中人,奸諛漸倖於左右。禮有所不遵,孝有所偏重。納諫如流,施行則寡。是陛下修己親賢之誠,漸不如始,故天降霪雨以示警戒。伏願出令必信,斷獄不留,事惟咨於輔臣,寵勿啟於近習,割恩以定禮,稽古以崇孝,則一念轉移,可以銷天災,答天戒矣。」未幾,擢兵科給事中。疏陳杜漸保終四事:一、定君心之主宰,以杜蠱惑之漸;二、均兩宮之孝養,以杜嫌隙之漸;三、一政令,以杜欺蔽之漸;四、清傳奉,以杜假託之漸。尋言興府從駕官不宜濫授。帝納之。

嘉靖改元,帝欲尊所生為帝后。會掖庭火,廷臣多言咎在「大禮」。繼曾亦言:「去年五月日精門災,今月二日長安榜廊災,及今郊祀日,內廷小房又災。天有五行,火實主禮。人有五事,火實主言。名不正則言不順,言不順則禮不興。今歲未期而災者三,廢禮失言之郊也。」提督三千營廣寧伯劉佶久病,繼曾論罷之。宣大、關陜、廣西數有警,中原盜竊發。繼曾陳戰守方略及儲將練兵足食之計,多議行。

三年,帝漸疏大臣,政率內決。繼曾抗章曰:「比來中旨,大戾王言。事不考經,文不會理,悅邪說之諂媚則賜敕褒俞,惡師保之抗言則漸將放黜。臣目睹出涕,口誦吞聲。夫祖宗以來,凡有批答,必付內閣擬進者,非止慮獨見之或偏,亦防矯偽者之假託也。正德之世,蓋極弊矣,尚未有如今日之可駭可歎者。左右群小,目不知書,身未經事,乘隙招權,弄筆取寵,故言出無稽,一至於此。陛下不與大臣共政,而倚信群小,臣恐大器之不安也。」疏入,帝震怒,下詔獄掠治,謫金壇縣丞。給事中張逵、韓楷、鄭一鵬,御史林有孚、馬明衡、季本皆論救,不報。累遷至徽州知府,卒。

帝初踐阼,言路大開。進言者或過於切直,帝亦優容之。自劉最及繼曾得罪後,厭薄言官,廢黜相繼,納諫之風微矣。

最,字振廷,崇仁人。繼曾同年進士。由慈利知縣入為禮科給事中。世宗議定策功,大行封拜,最疏止之。尋請帝勤聖學,於宮中日誦《大學衍義》,勿令左右近習誘以匪僻。嘉靖二年,中官崔文以禱祠事誘帝。最極言其非,且奏文耗帑金狀。而帝從文言,命最自核侵耗數。最言:「帑銀屬內府,雖計臣不得稽贏縮。文乃欲假難行事,逃己罪,制言官」。疏入,忤旨,出為廣德州判官。言官論救,不納。已而東廠太監芮景賢奏最在途仍故銜,乘巨舫,取夫役,巡鹽御史黃國用復遣牌送之。帝怒,逮二人下詔獄。最充軍邵武,國用謫極邊雜職。法司及言官救之,責以黨比。最居戍所,久之赦還。家居二十餘年卒。

朱淛,字必東,莆田人。舉鄉試第一。嘉靖二年成進士。明年春與同縣馬明衡並授御史。甫閱月,會昭聖皇太后生辰,有旨免命婦朝賀。淛言:「皇太后親挈神器以授陛下,母子至情,天日昭鑒。若傳免朝賀,何以慰親心而隆孝治?」明衡亦言:「暫免朝賀,在恒時則可,在議禮紛更之時則不可。且前者興國太后令節,朝賀如儀,今相去不過數旬,而彼此情文互異。詔旨一出,臣民駭疑。萬一因禮儀末節,稍成嫌隙,俾陛下貽譏天下,匪細故也。」時帝亟欲尊所生,而群臣必欲帝母昭聖,相持未決。二人疏入,帝恚且怒。立捕至內廷,責以離間宮闈,歸過於上,下詔獄拷訊。侍郎何孟春、御史蕭一中論救,皆不聽。御史陳逅、季本、員外郎林應驄繼諫。帝愈怒,並下詔獄,遠謫之。帝必欲殺二人,變色謂閣臣蔣冕曰:「此曹誣朕不孝,罪當死。」冕膝行頓首請曰:「陛下方興堯、舜之治,奈何有殺諫臣名。」良久,色稍解,欲戍之。冕又固請,繼以泣。乃杖八十,除名為民,兩人遂廢。廷臣多論薦,不復召。

淛為人長者,不欺人,或為人欺亦不校。與明衡皆貧,淛尤甚。鄉里利病,必與有司言,雖忤弗顧。家居三十餘年卒。

明衡,字子萃。父思聰,死宸濠難,自有傳。明衡登正德十二年進士,授太常博士。甫為御史,即與淛同得罪。閩中學者率以蔡清為宗,至明衡獨受業於王守仁。閩中有王氏學,自明衡始。

陳逅,字良會,常熟人。正德六年進士。除福清知縣。入為御史。以救兩人謫合浦主簿。累官河南副使。帝幸承天,坐供具不辦,下獄為民。

林應驄,亦莆田人。明衡同年進士。授戶部主事。嘉靖初,尚書孫交核各官莊田。帝以其數稍參差,有旨詰狀。應驄言:「部疏,臣司檢視,即有誤,當罪臣。尚書總領部事,安能遍閱?今旬日間,戶、工二部尚書相繼令對狀,非尊賢優老之意。」疏入,奪俸。以救淛等,謫徐聞縣丞。代其長朝覲,疏陳時事,多議行。

楊言,字惟仁,鄞人。正德十六年進士。授行人。嘉靖四年擢禮科給事中。閱數日即上言:「邇者仁壽宮災,諭群臣修省。臣以為責在公卿而不在陛下,罪在諫官而不在聖躬。朝廷設六科,所以舉正欺蔽也。今吏科失職,致陛下賢否混淆,進退失當。大臣蔣冕、林俊輩去矣,小臣王相、張漢卿輩皆得禍矣,而張驄、桂萼始由捷徑以竊清秩,終怙威勢以賊良善。戶科失職,致陛下儉德不聞,而張崙輩請索無厭,崔和輩敢亂舊章。禮科失職,致陛下享祀未格於神,而廟社無帡幪之庇。兵科失職,致陛下綱紀廢弛,而錦衣多冒濫之官,山海攘抽分之利,匠役增收而不禁,奏帶踰額而不裁。刑科失職,致陛下用罰不中。元惡如藍華輩得寬籍沒之法,諍臣如郭楠輩反施鈕械之刑。工科失職,致陛下興作不常。局官陸宣輩支俸踰於常制,內監陳林輩抽解及於蕪湖。凡此,皆時弊之急且大,而足以拂天意者。願陛下勤修庶政,而罷臣等以警有位,庶可以格天心,弭災變。」帝以浮謗責之。

奸人何淵請建世室。言與廷臣爭,不聽。言復抗章曰:「祖宗身有天下,大宗也,君也。獻皇帝舊為籓王,小宗也,臣也。以臣並君,亂天下大分。以小宗並大宗,干天下正統。獻帝雖有盛德,非若周文、武創王業也,欲襲世室名,舛矣。如以獻帝為自出之帝,是前無祖宗;以獻帝為禰而宗之,是後無孝、武二帝。陛下前既罪醫士劉惠之言,今乃納淵之說。前既俞禮卿席書之議,今乃咈書之言。臣不知其何謂也。」

楊一清召入內閣,言請留之三邊。特旨拜張璁兵部侍郎。言以璁貪佞險躁,且新進,未更國家事,請罷璁,並劾吏部尚書廖紀引匪人。同官解一貫等亦諫。皆不納。有投匿名書御道者,言請即燒之,報可。

六年,錦衣百戶王邦奇借哈密事請誅楊廷和、彭澤等,下部議,未覆,而邦奇復誣大學士費宏、石珤陰庇廷和,詞連廷和子主事惇等,將興大獄。言抗疏曰:「先帝晏駕,江彬手握邊軍四萬,圖為不軌。廷和密謀行誅,俄頃事定,迎立聖主,此社稷之勳也。縱使有罪,猶當十世宥之。今既以奸人言罷其官、戍其長子矣,乃又聽邦奇之誣而盡逮其鄉里、親戚,誣為蜀黨,何意聖明之朝,忽有此事?至宏、珤乃天子師保之官,百僚之表也。邦奇心懷怨望,文飾奸言,詬辱大臣,熒惑聖聽。若窮治不已,株連益多,臣竊為國家大體惜也。」書奏,帝震怒,並收繫言,親鞫於午門。群臣悉集。言備極五毒,折其一指,卒無撓詞。既罷,下五府九卿議。鎮遠侯顧仕隆等覆奏邦奇言皆虛妄,帝責仕隆等徇情。然獄亦因是解,謫言宿州判官。御史程啟充請還言舊任,不聽。稍遷溧陽知縣,歷南京吏部郎中。坐事再謫知夷陵。累官湖廣參議。

言為吏,多著聲績。溧陽、夷陵皆祠祀之。

劉安,字汝勉,慈谿人。嘉靖五年進士。授南京工部主事,改河南道御史。入臺甫一月,上疏曰:「人君貴明不貴察。察,非明也。人君以察為明,天下始多事矣。陛下臨御八年而治理未臻,識者謂陛下之治功損於明察。夫治,可以緩圖,不可以急取;可以休養致,不可以督責成。以急切之心,行督責之政,於是躬親有司之事,指摘臣下之失,令出而復返,方信而忽疑。大小臣工救過不暇,多有不安其位者。孰能為陛下建長久之策,以圖平治哉?且朝廷者,四方之極也。內之君臣,習尚如此,則外而撫按守令之官,風從響應。上以苛察繩,下以苛察應,恐民窮為起盜之源,食寡無強兵之理。今明天子綜核於上,百執事振刷於下,叢蠹之弊十去其九,所少者元氣耳。伏望大包荒之量,重根本之圖,略繁文而先急務,簡細故而弘遠猷,不以一人之毀譽為喜怒,不以一言之順逆為行止,久任老成,優容言官,則君臣上下一德一心,人人各安其位,事事各盡其才,雍熙太和之治不難見矣。」帝閱疏大怒,逮赴錦衣衛拷訊。兵科給事中胡堯時救之,並逮治。獄具,謫堯時攸縣主薄,安餘干典史。築決堤數十丈,人稱劉公堤。再遷長沙同知,擢鳳陽知府。治行卓異,賜正三品服。以憂歸,卒。

薛侃,字尚謙,揭陽人。性至孝,正德十二年成進士,即以侍養歸。師王守仁於贛州,歸語兄助教俊。俊大喜,率群子侄宗鎧等往學焉。自是王氏學盛行於嶺南。

世宗立,侃授行人。母訃,隕絕,五日始食粥。嘉靖七年起故官。聞守仁卒,偕歐陽德輩為位,朝夕哭。時方議文廟祀典,侃請祀陸九淵、陳獻章。九淵得報允。已,進司正。十年秋疏言:「祖宗分封子弟,必留一人京師司香,有事居守,或代行祭饗。列聖相承,莫之或改。至正德初,逆瑾懷貳,始令就封。乞稽舊典,擇親籓賢者居京師,慎選正人輔導,以待他日皇嗣之生,此宗社大計。」帝方祈嗣,諱言之,震怒,立下獄廷鞫,究交通主使者。南海彭澤為吏部郎,無行。因議禮附張孚敬,遂與為腹心。後京察被黜,孚敬奏留之,復引為諭德,至太常卿。侃以疏草示澤。澤與侃及少詹事夏言同年生,而言是時數忤孚敬。澤默計儲副事觸帝諱,必興大獄,誣言同謀可禍也,紿侃槁示孚敬,因報侃曰:「張公甚稱善,此國家大事,當從中贊之。」與為期,趣之上。孚敬乃先錄侃槁以進,謂出於言,請勿先發以待疏至。帝許之。侃猶豫,澤頻趣之乃上。拷掠備至,侃獨自承,累日獄不具。澤挑使引言,侃嗔目曰:「疏,我自具。趣我上者,爾也。爾謂張少傅許助之,言何豫?」給事中孫應奎、曹汴揖孚敬避。孚敬怒。應奎等疏聞,詔並下言、應奎、汴詔獄,命郭勛、翟鑾及司禮中官會廷臣再鞫,具得其實。帝乃釋言等,出孚敬密疏二示廷臣,斥其忮罔,令致仕。侃為民,澤戍大同。澤在朝專為邪媚,及敗,天下快之。

侃至潞河,遇聖壽節,焚香叩祝甚謹。或報參政項喬曰:「小舟中有民服而祝聖者。」喬曰:「必薛中離也。」跡之,果然。中離者,侃自號也。歸家益力學,從游者百餘人。隆慶初,復官,贈御史。俊子宗鎧,自有傳。

侃歸數月,御史喻希禮、石金皆以言皇嗣得罪。希禮言:「陛下祈嗣禮成,瑞雪遂降,臣以為招和致祥,不盡於此。往者大赦,今歲免刑,臣民盡沾澤,獨議禮議獄得罪諸臣遠戍邊徼,乞量移近地,或特賜赦免,則和氣薰蒸,前星自耀。」帝大怒曰:「謂朕罪諸臣致遲嗣續耶?所司參議以聞。」議未上,金亦言:「陛下一日萬幾,經理勞瘁。何若中涵太虛,物來順應。凡人才之用舍,政事之敷施,始以九卿之詳度,繼以內閣之咨謀,其弗協於中者,付諸臺諫之公論。陛下恭默凝神,挈其綱領,使精神內蘊,根本充固,則百斯男之慶,自不期而至。王守仁首平逆籓,繼靖巨寇,乃因疑謗,泯其前勞。大禮大獄諸臣,久膺流竄,因鬱既久,物故已多。望錄守仁功,寬諸臣罪,則太和之氣塞宇宙間矣。」帝不悅曰:「金欲朕勿御萬幾,即古奸臣導其君不親政之意,其並察奏。」尚書夏言等言二人無他腸。帝益怒,下二人詔獄,而責言等陳狀。伏罪乃宥之。二人竟謫戍邊衛。久之,赦還,卒。隆慶初,俱贈光祿少卿。

喻希禮,麻城人。石金,黃梅人。巡按廣西,與姚鏌不協。後與守仁共撫盧蘇、王受。還臺,值張、桂用事。御史儲良才輩爭附之,金獨侃侃不阿,以是有名。

楊名,字實卿,遂寧人。童子時,督學王廷相奇其語,補弟子員。嘉靖七年,鄉試第一。明年以第三人及第,授編修。聞大母喪,請急歸。還朝,為展書官。

十一年十月,彗星見。名應詔上書,言帝喜怒失中,用舍不當。語切直,帝銜之,而答旨稱其納忠,令無隱。名乃復言:「吏部諸曹之首,尚書百官之表,而汪鋐小人之尤也。武定侯郭勛奸回險譎,太常卿陳道瀛、金贇仁粗鄙酣淫。數人者,群情皆曰不當用,而陛下用之,是聖心之偏於喜也。諸臣建言觸忤者,心實可原。大學士李時以愛惜人才為請,即荷嘉納,而吏部不為題覆。臣所謂虛文塞責者,豈盡無哉?夫此得罪諸臣,群情以為當宥,而陛下不終宥,是聖心之偏於怒也。真人邵元節猥以末術,過蒙採聽。嘗令設醮內府,且命左右大臣奔走供事,遂致不肖之徒有昏夜乞哀出其門者。書之史冊,後世其將謂何?凡此聖心之少有所偏者,故臣敢抒其狂愚。」疏入,帝震怒,即執下詔獄拷訊。鋐疏辨,謂:「名乃楊廷和鄉人。頃張孚敬去位,廷和黨輒思報復,故攻及臣。臣為上簡用,誠欲一振舉朝廷之法,而議者輒病臣操切。且內閣大臣率務和同,植黨固位,故名敢欺肆至此。」帝深入其言,益怒,命所司窮詰主使。名數瀕於死,無所承,言曾以疏草示同年生程文德,乃並文德下獄。侍郎黃宗明、候補判官黃直救之,先後皆下獄。法司再擬名罪,皆不當上指。特詔謫名戍,編伍瞿塘衛。明年釋還。屢薦終不復召。家居二十餘年,奉親孝。親歿,與弟台廬於墓。免喪,疾作,卒。

黃直,字以方,金谿人。受業於王守仁。嘉靖二年會試,主司發策極詆守仁之學。直與同門歐陽德不阿主司意,編修馬汝驥奇之,兩人遂中式。直既成進士,即疏陳隆聖治、保聖躬、敦聖孝、明聖鑒、勤聖學、務聖道六事。除漳州推官。以漳俗尚鬼,盡廢境內淫祠,易其材以葺橋梁、公廨。御史誣以罪,送吏部降用。行至中途,疏請早定儲貳。帝怒,遣緹騎逮問。無何得釋,貶沔陽判官。嘗署崇陽縣事,有惠政。

外艱歸,三年不御酒肉。服闋赴部,適名、宗明下獄。直抗疏言:「九經之首曰修身,其中曰敬大臣,體群臣。今楊名以直言置詔獄,非所以體群臣。黃宗明以論救與同罪,非所以敬大臣。二者未盡,天下後世疑陛下修身之道亦有所未盡矣。」帝大怒,並下詔獄拷掠,命發極邊,編戍雷州衛。赦還,貧甚,妻紡織以給朝夕,直讀書談道自如。久之,卒。隆慶初,贈光祿少卿。

郭弘化,字子弼,安福人。嘉靖二年進士。除江陵知縣,徵授御史。十一年冬,彗星見。弘化言:「按《天文志》:井居東方,其宿為木。今者彗出於井,則土木繁興所致也。臣聞四川、湖廣、貴州、江西、浙江、山西及真定諸府之採木者,勞苦萬狀。應天、蘇、松、常、鎮五府,方有造磚之役,民間耗費不貲,窯戶逃亡過半。而廣東以採珠之故,激民為盜,至攻劫會城。皆足戾天和,乾星變。請悉停罷,則彗滅而前星耀矣。」戶部尚書許贊等請聽弘化言。帝怒曰:「採珠,故事也,朕未有嗣,以是故耶?」責贊等附和,黜弘化為民。久之,言官會薦,報寢。卒於家。穆宗立,贈光祿少卿。

劉世龍,字元卿,慈谿人。正德十六年進士。授太倉知州,改國子助教,遷南京兵部主事。

嘉靖十三年,南京太廟災。世龍應詔陳三事:

一、杜諂諛以正風俗。天下風俗之不正,由於人心之壞。人心之壞,患得患失使然也。今天下刻薄相尚,變詐相高,諂媚相師,阿比相倚。仕者日壞於上,學者日壞於下,彼倡此和,磨然成風。惟陛下赫然矯正,勿以詭隨阿比者為賢,勿以正直骨鯁者為不肖,勿以私好有所賞,勿以私惡有所罰,虛心以防邪佞,謙受以來忠讜,更敕大小臣工,協恭圖治,無權勢相軋,朋黨相傾,則風俗正矣。

二、廣容納以開言路。陛下臨御之初,犯顏敢諫之臣比先朝為盛,所言或傷於激切,而放逐既久,悔悟日深。當宥其既往,以次錄用,死者則恤之。仍令大小臣工直言時政,以作忠義之氣。

三、慎舉動以存大體。立國者,在敬大臣,不遺故舊。蓋任之既重,則禮之宜優。今或忽然去之,忽然召之,甚至嬰三木,被箠楚,何以勵臣節哉!臣愚以為陛下歷試之餘,其人果無足取,則宜以禮使退。如素行無缺,偶以一時喜怒,輒從而顛倒之,陛下固付之無心,而天下有以窺陛下也。

至如張延齡憑寵為非,法難容假。側聞長老之言,孝宗時待之過厚,遂釀今日之禍。顧區區腐鼠,何足深惜!獨念孝廟在天之靈,太皇太后垂老之景,乃至不能自庇其骨肉,於情忍乎?恐陛下孝養兩宮,亦不能不為一動心也。頃創造神御閣、啟祥宮,特令大臣督理其事。臣以為南京太廟方被災,工役之急當無過此。今興作頻年,四方凋敝,正時絀舉贏之會,亦宜量酌緩急而為之以漸。此皆應天以實之道也。

疏入,帝震怒,謂世龍訕上庇逆。械繫至京,下詔獄拷掠。獄具,復廷杖八十,斥為民。張延齡者,昭聖太后弟也。帝必欲殺之,故世龍重得罪。後二年,又以大猾劉東山訐告,盡斥諸刑曹郎羅虞臣、徐申等,猶以延齡故也。

世龍家居五十年,自養親一肉外,蔬食終身。卒之日,族人為治衣冠葬之。

徐申,字周翰,崑山人。嘉靖初,由鄉舉除蘄水知縣。改知上鐃,徵授刑部主事。延齡之繫獄也,申奏記尚書聶賢、唐龍言:「太后春秋高,延齡旦暮戮,何以慰太后心?宜援議貴議親例請於帝。」賢等深然之,獄久不決。始延齡下獄,提牢主事沈椿不令入獄,置別所。繼者益寬假之,脫梏堣,通家人出入。會大猾劉東山亦繫獄,上告延齡有不軌謀。憾前主事羅虞臣笞己,因訐及椿等。帝震怒,命執先後提牢主事三十七人付詔獄搒掠,申與焉。獄具,當輸贖還職,帝命杖之廷,盡謫外任,而斥虞臣為民。虞臣,廣東順德人。歷吏部主事。好剛疾惡。既歸,結廬山中,讀書纂述。年僅三十五卒。

申既謫官,不赴,歸與同里魏校、方鳳輩優游歗詠為樂。久之,卒。

曾孫應聘,字伯衡,少有才名。萬歷十一年進士。改庶吉士,授檢討。二十一年京察,中蜚語當謫,拂衣歸。座主沈一貫當國,數招之,不出。家居十餘年,始起行人司副。遷尚寶司丞,再遷太僕少卿。卒官。

張選,字舜舉。黃正色,字士尚。皆無錫人。同登嘉靖八年進士。正色除仁和知縣,選知蕭山縣,又鄰境也。選治蕭山有聲。十二年冬,先入為戶科給事中。明年四月時享太廟,遣武定侯郭勛代。選上言:「宗廟之祭,惟誠與敬。孔子曰:『吾不與祭,如不祭』。傳曰:『神不歆非類』。孟春廟享,遣官暫攝,中外臣心知非得已。茲孟夏祫享,倘更不親行,則迹涉怠玩。如或聖體初復,未任趨蹌,宜明詔禮官先期告廟。陛下亦宜靜處齋宮,以通神貺。」帝閱疏大怒,下之禮部。尚書夏言等言:「代祭之文,載之《周官》。《語》曰:『子之所慎齋戰疾』。疾當慎,無異於祭,選言非是。但小臣無知,惟陛下曲赦。」帝愈怒,責言等黨比。命執選闕下,杖八十。帝出御文華殿聽之,每一人行杖畢,輒以數報。杖折者三。曳出,已死。帝怒猶未釋。是夕,不入大內,繞殿走,製《祭祀記》一篇。一夕鋟成,明旦分賜百官。而選出,家人投良劑得甦,帝竟削選籍。選居職甫三月,遽以言得罪,名震海內。

正色是時方憂居。已,補香山,旋改南海。座主霍韜宗人橫甚,正色繩以法。韜顧以為賢,豪強屏跡,縣中大理。十七年召為南京御史。劾兵部尚書張瓚奸貪,事甚有跡。而中有「歷官籓臬,無一善狀」語,瓚言己未任籓臬。帝以誣劾,奪俸兩月。明年,章聖太后梓宮南葬,命正色護視。事竣,劾中官鮑忠、駙馬都尉崔元、禮部尚書溫仁和所過納饋遺。帝召詰忠等。皆叩頭祈哀,因譖正色擅於梓宮前乘馬執扇,及江行涉險又不隨舟督護,大不敬。帝遂發怒,立捕下詔獄搒掠,遣戍遼東。

正色與選初同志相友善,至是先後以直節顯。正色居戍所三十年,其顛躓窮困視選尤甚。穆宗初,起選通政參議,以年老予致仕。召正色為大理丞,進少卿,尋遷南京太僕卿,亦引年致仕。選先卒,正色後數年卒。

包節,字元達,先世嘉興人,其父始遷華亭。節祖鼎,池州知府。為治清簡,早歲乞休,為鄉邑所重。節生五歲而孤,母躬教育之。登嘉靖十一年進士。授東昌推官。入為御史。劾兵部尚書張瓚貪穢。出按雲南。時仕者以荒徼憚不欲往,因設告就遠方之法。節言:「此曹志甘投荒,非年迫衰遲,則家貧急祿。志在為己,豈在恤民?滇中長吏所以多不得人也。請自今以附近選人充之,而州縣佐貳始用此曹,庶吏治可舉。」吏部請以節言概行於雲、貴、兩廣。制可。

以疾歸。起故官,再按湖廣。顯陵守備中官廖斌擅威福,節欲繩之,語先洩。斌俟節謁陵時,故獻膳羞,遽使撤去,詭稱節麾出之。鐘祥民王憲告斌黨庇奸豪周章等,節捕章,斃之杖下。斌益怒,遂奏節不以正旦謁陵,次日始謁,時當進膳,不旁立,褻慢大不敬。奏已入,節始奏斌前事。帝大怒,以節抵罪,逮詣詔獄搒掠,永戍莊浪衛。莊浪極邊,敗屋頹垣,節處之甚安。獨念其母,自傷不克終養,日飲泣。母訃至,晝夜哭。已,又聞弟孝卒,撫膺曰:「誰代吾奉祀者?」哭益悲。病死,遺言以衰絰殮。

孝,字元愛,後節三年成進士。由中書舍人為南京御史。疏論禮部尚書溫仁和主辛丑會試有奸弊,且劾庶子童承敘、贊善郜希顏、編修袁煒,帝皆不問。未幾,又劾巡撫孫襘、吳瀚,瀚罷去。

孝兄弟分居南北臺,並著風采,又皆有至情。節官北不得養母,孝遂以侍養歸。母亡,哀毀骨立,未終喪卒。節亦繼殞。時並稱其孝。

謝廷蒨,字子佩,富順人。嘉靖十一年進士。除新喻知縣,徵授吏科給事中。御史胡鰲言:「京師優倡雜處。請敕五城,諸非隸教坊兩院者,斥去之。」都御史王廷相等議可。帝惡熬言褻,謫鹽城丞,奪廷相等俸。廷蒨救之,被詔切責。雷震謹身殿,疏陳修省數事,語直。帝摘疏中訛字,停其俸。十八年偕同官曾廷,李逢、周珫諫帝南巡,忤旨。已,給事中戴嘉猷馳疏請回鑾,而車駕已發。帝大怒。甫還,即執嘉猷并廷蒨等下詔獄,謫廷蒨雲南典史。屢遷浙江僉事。以侍養歸,遂不出。隆慶元年,起故官山西,俄擢河南右參議,皆不拜。吏部高其行,請得以新秩歸老,許之。萬曆改元,四川巡撫曾省吾奏言:「廷隱居三十年,家徒四壁,樂道著書,宜特加京秩,風勵士林。」詔即加進太僕少卿。又數年卒。

王與齡,字受甫,寧鄉人。嘉靖八年進士。授蘇州推官。入為戶部主事,調吏部,進員外郎。二十一年遷文選郎中。澄清銓敘,所推薦皆廉靜老成。

大學士翟鑾為禮部主事張惟一求吏部,嚴嵩為監生錢可教求東陽知縣,俱書抵與齡。與齡偕員外郎吳伯亨、主事李大魁、周鈇,白之尚書許贊,具疏以聞。言:「平時請屬甚多。臣等違抗,積罪如山。非聖明覆庇,則二權奸主於中,群鷹犬和於外,臣等不為前選郎王嘉賓之斥,得為近日御史謝瑜之罷,幸矣。」疏入,鑾言惟一資望應遷。嵩抵無致書事,請逮可教訊治,因言:「聖明日覽奏章,革弊釐奸悉宸斷。而贊等妄意臣輩為之,借以修怨。然贊柔良,第受制所屬耳。」帝方信嵩,又見疏中引嘉賓、瑜事,遂發怒。切責贊,除與齡名,伯亨等俱調外。給事中周怡論之,廷杖繫獄。御史徐宗魯等亦以為言,皆奪俸。自是,諸司以與齡為戒,無復敢與嵩抗。

與齡既罷,錦衣遣使偵其裝,襆被外無長物,稱歎而去。里居,角巾躬稼圃,翛然自得。郡人為作《平陽四賢詩》美之。四賢者,尚書韓文、陶琰、張潤及與齡也。越二十餘年,卒。

周鈇,字汝威,榆次人。嘉靖五年進士。授行人。擢御史,巡按陜西。被俘民自塞外逃歸者,邊將殺以冒功。鈇請下詔嚴禁,有報降五人以上者賞之。詔可。再按山東,特改右春坊清紀郎兼翰林院侍書。俺答將入寇,總督侍郎翟鵬以聞。鈇以中樞無籌策,請早為計。帝以為浮詞亂政,責降廬州府知事。旋改國子監丞,擢吏部文選主事。坐與齡發嵩等私屬事,貶河間通判。已而吏部擬擢南京吏部主事。嵩言鈇調官甫四月,不得驟遷。帝怒,詰責尚書許贊等,令錄左降官遷擢者姓名。贊引罪,並列陳叔頤等十六人以聞。詔奪贊等俸,鐫文選郎鄭曉三級,鈇、叔頤等褫職為民。廷臣論薦鈇,以嵩在位,不復召。穆宗初,贈光祿少卿。

楊思忠,字孝夫,平定人。嘉靖二十年進士。歷禮科給事中。二十九年,孝烈皇后大祥。欲預祧仁宗,附后太廟,下廷議。尚書徐階以為非禮,思忠力贊階議,餘人莫敢言。帝使人覘知狀。及議上,嚴旨譙責,命階與思忠更定,二人復據禮對。帝益怒,竟祧仁宗。階故得帝眷,獨銜思忠。每當遷,輒報罷。逾三年,正旦日食,陰雲不見,六科合疏賀。帝摘疏中語,詰為不成文,曰:「思忠懷欺,不臣久矣。」杖百,斥為民,餘皆奪俸。隆慶元年起掌吏科。三遷右僉都御史,巡撫陜西。五年改南京戶部右侍郎。致仕卒。

世宗晚年,進言者多得重譴。二十九年,俺答薄都城。通政使樊深陳禦寇七事,中言仇鸞養寇要功。帝方眷鸞,立斥為民。四十二年正月,御史凌儒請重貪墨之罰,革虛冒之兵,搜遺佚之士。因薦羅洪先、陸樹聲、吳嶽、吳悌。帝惡其市恩,杖六十,除名。四十五年十月,御史王時舉劾刑部尚書黃光昇,言:「內官季永以訴事犯乘輿,本無死比,乃擬真犯;奸人王相私閹良民者三,本無生法,乃擬矜疑。宜勒令致仕。」帝怒,命編氓口外。踰月,御史方新上言:「黃河與北狄之患,自古有之。乃今豐、沛間陸地為渠,而興都有陵寢之憂,鳳陽有冰雹之厄,河南有饑饉之災,堯之洚水不烈於此矣。諸邊將惰卒驕,寇至輒巽觀望,而寧武有軍士之變,南贛有土兵之叛,徽州諸府有礦徒竊發之虞,舜之三苗不棘於此矣。夫洚水、三苗不足為累者,以堯、舜兢業於上,而禹、皋諸臣分憂於下也。今司論納者日獻禎祥,而疆場之臣,惟冒首功,隱喪敗。為國分憂者,誰也?斥罰之法,今不得不嚴。而陛下亦宜隨事自責,痛加修省,然後災變可息,而外患可弭也。」疏入,斥為民。

深,大同人。儒,泰州人。時舉,順天通州人。新,青陽人。穆宗嗣位,並復官。

深尋遷刑部右侍郎。齊康之劾徐階也,深劾康並詆高拱。時登極詔書赦死罪以下囚,而流徒已至配者,所司拘律令不遣。深言殊死猶赦,而此反不及,非所以廣皇仁。詔從其議。旋進左侍郎,罷歸。

儒既復御史,益發舒,亦以康事率同列劾拱。拱罷,又劾去大學士郭朴。頃之,劾罷撫治鄖陽都御史劉秉仁。又以永平失事劾總督劉燾、巡撫耿隨卿、總兵官李世忠罪。隨卿、世忠被逮,燾貶官。隆慶二年,儒再遷右僉都御史,理山西屯鹽。吏部追論其知永豐時貪墨,遂落職閒住。

時舉復官後,巡按貴州。聞給事中石星廷杖,且帝方廣市珠寶,馳疏救星,極陳奢靡之害。已,請陳后還中宮。章並報聞。萬曆初,都給事中雒遵、御史景嵩、韓必顯論譚綸被謫,時舉抗章救之。歷大理左少卿。

新終湖廣參議。

贊曰:賈山有言:「忠臣之事君也,言切直則不用而身危。」「然切直之言,明主之所亟欲聞,忠臣之所蒙死而竭知也。」鄧繼曾諸人箴主闕,指時弊,言切直矣,而杖斥隨之。伊尹曰:「有言逆於汝心,必求諸道。」有旨哉,有旨哉!


\end{pinyinscope}