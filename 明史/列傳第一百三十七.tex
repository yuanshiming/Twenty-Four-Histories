\article{列傳第一百三十七}

\begin{pinyinscope}
朱燮元(徐如珂劉可訓胡平表盧安世林兆鼎}}李枟史永安劉錫元王三善岳具仰等硃家民蔡復一沈儆炌{{袁善周鴻圖段伯炌胡從儀

朱燮元,字懋和,浙江山陰人。萬曆二十年進士。除大理評事。遷蘇州知府、四川副使,改廣東提督學校。以右參政謝病歸。起陜西按察使,移四川右布政使。

天啟元年,就遷左。將入覲,會永寧奢崇明反,蜀王要燮元治軍。永寧,古蘭州地。奢氏,惈羅種也,洪武時歸附,世為宣撫使。傳至崇周,無子,崇明以疏屬襲,外恭內陰鷙,子寅尤驍桀好亂。時詔給事中明時舉、御史李達征川兵援遼,崇明父子請行,先遣土目樊龍、樊虎以兵詣重慶。巡撫徐可求汰其老弱,餉復不繼,龍等遂反。殺可求及參政孫好古、總兵官黃守魁等,時舉、達負傷遁。時九月十有七日也。賊遂據重慶,播州遺孽及諸亡命奸人蜂起應之。賊黨符國禎襲陷遵義,列城多不守。

崇明僭偽號,設丞相五府等官,統所部及徼外雜蠻數萬,分道趨成都。陷新都、內江,盡據木椑、龍泉諸隘口。指揮周邦太降,冉世洪、雷安世、瞿英戰死。成都兵止二千,餉又絀。燮元檄徵石砫、羅綱、龍安、松、茂諸道兵入援,斂二百里內粟入城。偕巡按御史薛敷政、右布政使周著、按察使林宰等分陴守。賊障革裹竹牌鉤梯附城,壘土山,上架蓬蓽,伏弩射城中。燮元用火器擊卻之,又遣人決都江堰水注濠。賊治橋,得少息,因斬城中通賊者二百人,賊失內應。賊四面立望樓,高與城齊,燮元命死士突出,擊斬三賊帥,燔其樓。

既而援兵漸集。登萊副使楊述程以募兵至湖廣,遂合安綿副使劉芬謙、石主女土官秦良玉軍敗賊牛頭鎮,復新都。他路援兵亦連勝賊。然賊亦愈增,日發塚,擲枯骸。忽自林中大噪,數千人擁物如舟,高丈許,長五十丈,樓數重,牛革蔽左右,置板如平地。一人披髮仗劍,上載羽旗,中數百人挾機弩毒矢,旁翼兩雲樓,曳以牛,俯瞰城中,城中人皆哭。燮元曰:「此呂公車也。」乃用巨木為機關,轉索發炮,飛千鈞石擊之,又以大炮擊牛,牛返走,敗去。

有諸生陷賊中,遣人言賊將羅象乾欲反正。燮元令與象乾俱至,呼飲戍樓中,不脫其佩刀,與同臥酣寢。象乾誓死報,復縋而出。自是,賊中舉動無不知。乃遣部將詐降,誘崇明至城下。伏起,崇明跳免。會諸道援軍至,燮元策賊且走,投木牌數百錦江,流而下,令有司沉舟斷橋,嚴兵待。象乾因自內縱火,崇明父子遁走瀘州,象乾遂以眾來歸。城圍百二日而解。

初,朝廷聞重慶變,即擢燮元僉都御史,巡撫四川,以楊愈懋為總兵官,而擢河南巡撫張我續總督四川、貴州、雲南、湖廣軍。未至而成都圍解,官軍乘勢復州縣衛所凡四十餘,惟重慶為樊龍等所據。其地三面阻江,一面通陸,副使徐如珂率兵繞出佛圖關後,與良玉攻拔之。崇明發卒數萬來援,如珂迎戰,檄同知越其傑躡賊後,殺萬餘人。監軍僉事戴君恩令守備金富廉攻斬賊將張彤,樊龍亦戰死。帝告廟受賀,進君恩三官。燮元所遣他將復建武、長寧,獲偽丞相何若海,瀘州亦旋復。

先是,國禎陷遵義,貴州巡撫李枟已遣兵復之。永寧人李忠臣嘗為松潘副使,家居,陷賊,以書約愈懋為內應,事覺,合門遇害。賊即用其家僮紿愈懋,襲殺之,并殺順慶推官郭象儀等,再陷遵義,殺推官馮鳳雛。

當是時,崇明未平,而貴州安邦彥又起。安氏世有水西,宣慰使安位方幼,邦彥以故得倡亂。朝議錄燮元守城功,加兵部侍郎,總督四川及湖廣荊、岳、鄖、襄、陜西漢中五府軍務,兼巡撫四川,而以楊述中總督貴州軍務,兼制雲南及湖廣辰、常、衡、永十一府,代我續共辦奢、安二賊。然兩督府分閫治軍,川、貴不相應,賊益得自恣。三年,燮元謀直取永寧,集將佐曰:「我久不得志於賊,我以分,賊以合也。」乃盡掣諸軍會長寧,連破麻塘坎、觀音庵、青山崖、天蓬洞諸砦。與良玉兵會,進攻永寧。擊敗奢寅於土地坎,追至老軍營、涼傘鋪,盡焚其營。寅被二鎗遁,樊虎亦中槍死。復追敗之橫山,入青崗坪,抵城下,拔之,擒叛將周邦太,降賊二萬。副總兵秦衍祚等亦攻克遵義。崇明父子逃入紅崖大囤,官軍蹙而拔之。連拔天台、白崖、楠木諸囤,撫定紅潦四十八砦。賊奔入舊藺州城,五月為參將羅象乾所攻克。崇明父子率餘眾走水西龍場客仲壩,倚其女弟奢社輝以守。初,賊失永寧,即求救於安邦彥。邦彥遣二軍窺遵義、永寧,燮元敗走之。總兵官李維新等遂攻破客仲巢,崇明父子竄深箐。維新偕副使李仙品、僉事劉可訓、參將林兆鼎等搗龍場,生擒崇明妻安氏、弟崇輝,寅、國禎皆被創走。錄功,進燮元右都御史。

時蜀中兵十六萬,土、漢各半。漢兵不任戰,而土兵驕淫不肯盡力。成都圍解,不即取重慶;重慶復,不即搗永寧;及永寧、藺州並下,賊失巢穴,又縱使遠竄。大抵土官利養寇,官軍效之,賊得展轉為計。崇明父子方窘甚,燮元以蜀已無賊,遂不窮追。永寧既拔,拓地千里。燮元割膏腴地歸永寧衛,以其餘地為四十八屯,給諸降賊有功者,令歲輸賦於官,曰「屯將」,隸於敘州府,增設同知一人領之。且移敘州兵備道於衛城,與貴州參將同駐,蜀中遂靖。而邦彥張甚。

四年春陷貴州,巡撫王三善軍沒。明年,總理魯欽敗於織金,貴州總督蔡復一軍又敗。廷臣以三善等失事由川師不協助,議合兩督府。乃命燮元以兵部尚書兼督貴州、雲南、廣西諸軍,移鎮遵義;而以尹同皋代撫四川。燮元赴重慶,邦彥偵知之。六年二月,謀乘官軍未發,分犯雲南、遵義,而令寅專犯永寧。未行,寅被殺,乃已。寅凶淫甚,有阿引者,受燮元金錢,乘寅醉殺之。寅既死,崇明年老無能為,邦彥亦乞撫,燮元聞於朝,許之,乃遣參將楊明輝往撫。燮元旋以父喪歸,偏沅巡撫閔夢得來代。

先是,貴州巡撫王瑊謂督臣移鎮貴陽有十便,朝議從之。夢得乃陳用兵機宜,請自永寧始,次普市、摩泥、赤水,百五十里皆坦途,赤水有城可屯兵,進白巖、層臺、畢節、大方僅二百餘里。我既宿重兵,諸番交通之路絕,然後貴陽、遵義軍剋期進,賊必不能支。疏未報,夢得召還,代以尚書張鶴鳴,議遂寢。鶴鳴未至,明輝奉制書,僅招撫安位,不云赦邦彥。邦彥怒,殺明輝,撫議由此絕。鶴鳴視師年餘,未嘗一戰,賊得養其銳。

崇禎元年六月,復召燮元代之,兼巡撫貴州,仍賜尚方劍。錄前功,進少保,世廕錦衣指揮使。時寇亂久,里井蕭條,貴陽民不及五百家,山谷悉苗仲。而將士多殺降報功,苗不附。燮元招流移,廣開墾,募勇敢;用夢得前議,檄雲南兵下烏撒,四川兵出永寧,下畢節,而親率大軍駐陸廣,逼大方。總兵官許成名、參政鄭朝棟由永寧復赤水。邦彥聞之,分守陸廣、鴨池、三岔諸要害,別以一軍趨遵義,自稱四裔大長老,號崇明大梁王,合兵十餘萬,先犯赤水。燮元授計成名,誘賊至永寧,乃遣總兵官林兆鼎從三岔入。副將王國禎從陸廣入,劉養鯤從遵義入,合傾其巢。邦彥恃勇,擬先破永寧軍,還拒諸將,急索戰。四川總兵官侯良柱、副使劉可訓遇賊十萬於五峰山、桃紅壩,大破之。賊奔據山巔。諸將乘霧力攻,賊復大敗。又追敗之紅土川,邦彥、崇明皆授首,時二年八月十有七日也。捷聞,帝大喜。以成名與良柱爭功,賞久不行。

烏撒安效良死,其妻安氏招故沾益土酋安遠弟邊為夫,負固不服。燮元乘兵威脅走邊,遂復烏撒。燮元以境內賊略盡,不欲窮兵,乃檄招安位,位不決。燮元集將吏議曰:「水西地深險多箐篁,蠻煙僰雨,莫辨晝夜,深入難出。今當扼其要害,四面迭攻,賊乏食,將自斃。」於是攻之百餘日,斬級萬餘。養鯤復遣人入大方,燒其室廬。位大恐,三年春,遣使乞降。燮元與約四事:一、貶秩,二、削水外六目地歸之朝廷,三、獻殺王巡撫者首,四、開畢節等九驛。位請如約,率四十八目出降。燮元受之,貴州亦靖。遂上善後疏曰:「水西自河以外,悉入版圖。沿河要害,臣築城三十六所,近控蠻苗,遠聯滇、蜀,皆立邸舍,繕郵亭,建倉廩,賊必不敢猝入為寇。鴨池、安莊傍河可屯之土,不下二千頃,人賦土使自贍,鹽酪芻茭出其中。諸將士身經數百戰,咸願得尺寸地長子孫,請割新疆以授之,使知所激勸。」帝報可。

初,崇明、邦彥之死,實川中諸將功,而黔將爭之。燮元頗右黔將,屢奏於朝,為四川巡按御史馬如蛟所劾。燮元力求罷,帝慰留之。其冬討平定番、鎮寧叛苗,乃通威清等上六衛及平越、清平、偏橋、鎮遠四衛道路,凡一千六百餘里,繕亭障,置游徼。貴陽東北有洪邊十二馬頭,故宣慰宋嗣殷地也。嗣殷以助邦彥被剿滅,乃即其地置開州,又奏復故施秉縣,招流民實之。

四年,阿迷州土官普名聲作亂,陷彌勒州曲江所,又攻臨安及寧州,遠近震動。巡撫王伉、總兵官沐天波不能禦,伉逮戍。燮元遣兵臨之,遂就撫。

龍場壩者,鄰大方,邦彥以假崇明。崇明既滅,總兵侯良柱欲設官屯守以自廣。而安位謂己故地,數舉兵爭,燮元不之禁。會燮元劾良柱不職;良柱亦訐燮元曲庇安氏,納其重賄。章下四川巡按御史劉宗祥。宗祥亦劾燮元受賄,且以龍場、永寧不置邑衛為欺罔。帝以責燮元,燮元乃上言:「禦夷之法,來則安之,不專在攻取也。今水西已納款,惟明定疆界,俾自耕牧,以輸國賦。若設官屯兵,此地四面孤懸,中限河水,不利應援,築城守渡,轉運煩費。且內激藺州必死之斗,外挑水西扼吭之嫌,兵端一開,未易猝止,非國家久遠計。」帝猶未許。後勘其地,果如所議。論桃紅壩功,進少師,世廕錦衣指揮使。一品六年滿,加左柱國。再議平賊功,世廕錦衣指揮僉事。

十年,安位死,無嗣,族屬爭立。朝議又欲郡縣其地,燮元力爭。遂傳檄土目,布上威德。諸蠻爭納土,獻重器。燮元乃裂疆域,眾建諸蠻。復上疏曰:

水西有宣慰之土,有各目之土。宣慰公土,宜還朝廷。各目私土,宜畀分守,籍其戶口,徵其賦稅,殊俗內響,等之編氓。大方、西溪、谷里、北那要害之地,築城戍兵,足銷反側。夫西南之境,皆荒服也,楊氏反播,奢氏反藺,安氏反水西。滇之定番,小州耳,為長官司者十有七,數百年來未有反者。非他苗好叛逆,而定番性忠順也,地大者跋扈之資,勢弱者保世之策。今臣分水西地,授之酋長及有功漢人,咸俾世守。虐政苛斂,一切蠲除,參用漢法,可為長久計。

因言其便有九:

不設郡縣置軍衛,因其故俗,土漢相安,便一。地益墾闢,聚落日繁,經界既正,土酋不得侵軼民地,便二。黔地荒确,仰給外邦,今自食其地,省轉輸勞,便三。有功將士,酬以金則國幣方匱,酬以爵則名器將輕,錫以土田,於國無損,便四。既世其土,各圖久遠,為子孫計,反側不生,便五。大小相維,輕重相制,無事易以安,有事易以制,便六。訓農治兵,耀武河上,俾賊遺孽不敢窺伺,便七。軍民願耕者給田,且耕且守,衛所自實,無勾軍之累,便八。軍耕抵餉,民耕輸糧,以屯課耕,不拘其籍,以耕聚人,不世其伍,便九。

帝咸報可。無何,所撫土目有叛者,諸將方國安等軍敗,燮元坐貶一秩。已,竟破滅之。十一年春卒官,年七十三。

燮元長八尺,腹大十圍,飲啖兼二十人。鎮西南久,軍貲贖鍰,歲不下數十萬,皆籍之於官。治事明決,軍書絡繹,不假手幕佐。行軍務持重,謀定後戰,尤善用間。使人各當其材,犯法,即親愛必誅;有功,廝養不遺賞也。馭蠻以忠信,不妄殺,苗民懷之。初官陜西時,遇一老叟,載與歸,盡得其風角、占候、遁甲諸術。將別,語燮元曰:「幸自愛,他日西南有事,公當之矣。」內江牟康民者,奇士也,兵未起時,語人曰:「蜀且有變,平之者朱公乎?」已而果然。

徐如珂,字季鳴,吳縣人。萬曆二十三年進士。除刑部主事,歷郎中。主事謝廷贊疏請建儲。帝怒,盡貶刑曹官,如珂降雲南布政司照磨。累遷南京禮部郎中、廣東嶺南道右參議。暹羅貢使餽犀角、象牙,如珂不受。天啟元年,遷川東兵備副使。擊殺奢崇明黨樊龍,復重慶。奉檄搗藺州土城,賊借水西兵十萬來援,前軍少卻,捍子軍覃懋勛挽白竹弩連中之,賊大潰。轉戰數十里,斬首萬餘級,遂拔藺州,崇明父子竄水西去。乃召如珂為太僕少卿,轉左通政。

魏忠賢逐楊漣,如珂郊餞之,忠賢銜甚。遷光祿卿,修公廨竣,疏詞無所稱頌。六年九月,廷推南京工部右侍郎,遂削籍。歸里三月,治具飲客。頃之卒。崇禎初,以原推起用,死歲餘矣。尋錄破賊功,賜祭葬,進秩一等,官一子。

劉可訓,澧州人。萬曆中舉鄉試。歷官刑部員外郎。天啟元年,恤刑四川。會奢崇明反,圍成都,可訓佐城守有功,擢僉事,監軍討賊。崇明走龍場壩,可訓督諸將進剿,功最多。總督朱燮元匯奏文武將吏功,盛推可訓,乃遷威茂兵備參議。崇禎元年,改敘瀘副使,仍監諸將軍。二年,與總兵侯良柱破賊十萬眾於五峰山,斬崇明及安邦彥。御史毛羽健言:「可訓將孤軍,出入蠻煙瘴雨者多年。初無守土責,因奉命錄囚,而乃見危授命,解圍成都,奏捷永寧,掃除藺穴,逆寅授首。五路大戰,十道並攻,皆抱病督軍,誓死殉國。畀以節鉞,誰曰不宜?」帝頗納其言。未幾,畿輔被兵,可訓率師入衛。三年五月恢復遵化,擢右僉都御史,巡撫順天、永平,督薊鎮邊務。兵部尚書梁廷棟囑私人沈敏於可訓,敏遂交關為奸利。御史水佳允劾可訓,落職歸。後敘四川平寇功,復官,世廕錦衣千戶。未及起用,卒於家。

胡平表,雲南臨安人。萬曆中舉於鄉,歷忠州判官。天啟元年秋,樊龍陷重慶,平表縋城下,詣石砫土官秦良玉乞師,號泣不食飲者五晝夜,良玉為發兵。巡撫朱燮元檄平表監良玉軍。會擢新鄭知縣,燮元奏留之,改重慶推官,監軍兼副總兵,盡護諸軍將。戰數有功,擢四川監軍僉事,兼理屯田。遷貴州右參議。崇禎元年,總督張鶴鳴言:「平表偏州小吏,慷慨赴義。復新都,解成都圍,連戰白市驛、馬廟,進據兩嶺,俘斬無算。奪二郎關,擒賊帥黑蓬頭,追降樊龍,遂克重慶。用六千人敗奢、安二酋十萬兵。請以本官加督師御史銜,賜之專敕,必能梟逆賊首獻闕下。」部議格不行,乃進秩右參政,分守貴寧道,廕子錦衣世千戶。久之,擢貴州布政使。四年大計,坐不謹落職。十三年,督師楊嗣昌薦之,詔以武昌通判監標下軍事。嗣昌卒,乃罷歸。

盧安世,貴州赤水衛人。萬曆四十年舉於鄉,為富順教諭。天啟初,奢崇明反,遣賊逼取縣印,署令棄城走。安世收印,率壯士擊斬賊。無何,賊數萬猝至,安世單騎斗,手馘數人,詣上官請兵復其城。帝用大學士孫承宗言,超擢僉事,監軍討賊,屢戰有功。五年四月,總督朱燮元上言:「自遵義五路進兵,永寧破巢之後,大小數百戰,擒獲幾四萬人,降賊將百三十四人,招撫群賊及土、漢、苗仲二十九萬三千二百餘人,皆監司李仙品、劉可訓、鄭朝棟及安世等功,武將則林兆鼎、秦翼明、羅象乾,土官則陳治安、冉紹文、悅先民等。」帝納之。安世進貴州右參議,遷四川副使、遵義監軍,功復多。崇禎初,予世廕武職,進右參政。久之,解官,歸卒。

林兆鼎,福建人。天啟中,為四川參將,積功至總兵官,都督同知。崇禎三年,遣將討定番州苗,連破十餘寨,擒其魁。四年,遣將討湖廣苗黑酋,攻拔二百餘寨。加左都督,召僉南京右府。卒,贈太子少保。

李枟,字長孺,鄞人。曾祖循義,衡州知府。祖生威,鳳陽推官。枟登萬歷二十九年進士,授行人,擢御史。例轉廣東鹽法僉事,歷山東參議、陜西提學副使、山東參政、按察使。

四十七年秋,擢右僉都御史,巡撫貴州。貴州宣慰同知安邦彥者,宣慰使堯臣族子。堯臣死,子位幼,其母奢社輝代領其事。社輝,永寧宣撫奢崇明女弟也,邦彥遂專兵柄。會朝議征西南兵援遼,邦彥素桀黠,欲乘以起事,詣枟請行,枟諭止之。邦彥歸,益為反謀。枟累疏請增兵益餉,中朝方急遼事,置不問。

會枟被劾,乃六疏乞休。天啟元年始得請,以王三善代。而奢崇明已反重慶,陷遵義,貴陽大震,枟遂留視事。時城中兵不及三千,倉庫空虛。枟與巡按御史史永安貸雲南、湖廣銀四萬有奇,募兵四千,儲米二萬石,治戰守具,而急遣總兵官張彥方,都司許成名、黃運清,監軍副使朱芹,提學僉事劉錫元等援四川。屢捷,遂復遵義、綏陽、湄潭、真安、桐梓。

二年二月,或傳崇明陷成都,邦彥遂挾安位反,自稱羅甸王。四十八支及他部頭目安邦俊、陳其愚等蜂起相應,烏撒土目安效良亦與通。邦彥首襲畢節,都司楊明廷固守,擊斬數百人。效良助邦彥陷其城,明廷敗歿。賊遂分兵陷安順平壩,效良亦西陷霑益,而邦彥自統水西軍及羅鬼、苗仲數萬,東渡陸廣河,直趨貴陽,別遣王倫等下甕安,襲偏橋,以斷援兵。洪邊土司宋萬化糾苗仲九股陷龍里。

枟、永安聞變,亟議城守。會籓臬、守令咸入覲,而彥方鎮銅仁,運清駐遵義。城中文武無幾人,乃分兵為五,令錫元及參議邵應禎、都司劉嘉言、故副總兵劉岳分禦四門,枟自當北門之沖。永安居譙樓,團街市兵,防內變。學官及諸生亦督民兵分堞守。賊至,盡銳攻北城,枟迎戰,敗之。轉攻東門,為錫元所卻。乃日夕分番馳突,以疲官兵。為三丈樓臨城,用婦人、雞犬厭勝術。雲、永安烹彘雜斗米飯投飼雞犬,而張虎豹皮於城樓以祓之,乃得施炮石,夜縋死士燒其樓。賊又作竹籠萬餘,土壘之,高踰睥睨。永安急撤大寺鐘樓建城上,賊棄籠去,官軍出燒之。數出城邀賊糧,賊怒,盡發城外塚,遍燒村砦。又先後攻陷廣州、普定、威清、普安、安南諸衛。貴陽西數千里,盡為賊據。

初被圍,彥方、運清來救,敗賊於新添。賊誘入龍里,二將皆敗,乃縱之入城曰「使耗汝糧」,城中果大困。川貴總督張我續、巡撫王三善擁兵不進,枟、永安連疏告急,詔旨督責之。會彥方等出戰頻得利,賊退保澤溪,乃遣裨將商士傑等率九千人分控威清、新添二衛,且乞援兵。賊謂城必拔,沿山列營柵隔內外,間旬日一來攻,輒敗去。副總兵徐時逢、參將范仲仁赴援,遇賊甕城河。仲仁戰不利,時逢擁兵不救,遂大敗,諸將馬一龍、白自強等殲焉,援遂絕。賊聞三善將進兵,益日夜攻擊,長梯蟻附,城幾陷者數矣。枟奮臂一呼,士卒雖委頓,皆強起斫賊,賊皆顛踣死城下。王三善屢被嚴旨,乃率師破重圍而進。十二月七日,抵貴陽城下,圍始解。枟乃辭兵事,解官去。三善既破賊,我續無寸功,乾沒軍資六十萬,言官交劾,解職候勘。

我續,邯鄲人,刑部尚書國彥子。其後夤緣魏忠賢起戶部侍郎,進尚書,名麗逆案云。

方官廩之告竭也,米升直二十金。食糠核草木敗革皆盡,食死人肉,後乃生食人,至親屬相啖。彥方、運清部卒公屠人市肆,斤易銀一兩。枟盡焚書籍冠服,預戒家人,急則自盡,皆授以刀繯。城中戶十萬,圍困三百日,僅存者千餘人。孤城卒定,皆枟及永安、錫元功。熹宗用都御史鄒元標言,進枟兵部右侍郎,永安太僕少卿,錫元右參政。及圍解,當再敘功,御史蔣允儀言安位襲職時,枟索其金盆,致啟釁。章下貴州巡按侯恂核,未報,御史張應辰力頌枟功。恂核上,亦白其誣。帝責允儀。

初,永安遣運清往新添、平越趣援兵,懼其不濟,欲出城督之。錫元疑永安有去志,以咨枟,枟止永安。及錫元當絕食時,議發兵護枟、永安出城,身留死守,永安亦疑錫元。而運清因交構其間,三人遂相失。永安詆錫元議留身守城,欲輸城於賊,枟亦與謀,兩人上章辨。吏部尚書趙南星、左都御史孫瑋等力為三人解,而言永安功第一,當不次大用;枟已進官,當召還;錫元已進參政,當更優敘。詔可之。然枟竟不召,錫元亦無他擢,二人並還里。獨永安在朝,連擢太常卿、右僉都御史,巡撫寧夏,再以兵部右侍郎總督三邊,枟及諸將吏功,迄不敘。六年秋,御史田景新頌枟功,不納。

崇禎元年,給事中許譽卿再以金盆事劾枟。帝召咨廷臣,獨御史毛羽健為枟解,吏部尚書王永光等議如羽健言,給事中餘昌祚詆羽健曲庇。帝下川貴總督朱燮元等再核,羽健乃上疏曰:「安、奢世為婚姻,同謀已久。奢寅寇蜀,邦彥即寇黔,何用激變?當貴陽告急,正廣寧新破之日,舉朝皇皇,已置不問。後知枟不死,孤城尚存,始命王三善往救,至則圍已十月。安酋初發難,崇明欲取成都作家,邦彥欲圖貴陽為窟,西取雲南,東擾偏、沅、荊、襄,非枟扼其衝,東南盡塗炭。乃按臣永安不二三載躋卿貳,督師三邊,枟則投閒林壑,又以永安謗書為枟罪。案金盆之說發自允儀,當年已自承風聞,何至今猶執為實事?」貴州人亦爭為枟頌冤。燮元乃偕巡按御史越洪範交章雪其枉,枟事始白。

九年冬,敘守城功,進一秩,賚銀幣。久之,卒於家。

錫元,長洲人。崇禎中,任寧夏參政。

永安,武定人。共枟城守,功多。以在邊時建魏忠賢祠,後為御史甯光先論罷,不為人所重雲。

王三善,字彭伯,永城人。萬歷二十九年進士。由荊州推官入為吏部主事。齊、楚、浙三黨抨擊李三才,三善自請單騎行勘,遂為其黨所推。歷考功文選郎中,進太常少卿。

天啟元年十月,擢右僉都御史,代李枟巡撫貴州。時奢崇明已陷重慶。明年二月,安邦彥亦反,圍貴陽。枟及巡按御史史永安連章告急,趣三善馳援。三善始駐沅州,調集兵食。已次鎮遠,再次平越,去貴陽百八十里,方遣知府朱家民乞兵四川。兵未至,不敢進。疏請便宜從事,給空名部牒,得隨才委任。帝悉報可。

至十二月朔,知貴陽圍益困,集眾計曰:「失城死法,進援死敵,等死耳,盍死敵乎?」乃分兵為三:副使何天麟等從清水江進,為右部;僉事楊世賞等從都勻進,為左部;自將二萬人,與參議向日升,副總兵劉超,參將楊明楷、劉志敏、孫元謨、王建中等由中路,當賊鋒。舟次新安,抵龍頭營。超前鋒遇賊,眾欲退,斬二人乃定。賊酋阿成驍勇,超率步卒張良俊直前斬其頭,賊眾披靡。三善等大軍亦至,遂奪龍里城。諸將議駐師觀變,三善不可,策馬先。邦彥疑三善有眾數十萬,乃潛遁,餘賊退屯龍洞。官軍遂奪七里沖,進兵畢節鋪。元模、明楷連敗賊,其渠安邦俊中炮死,生獲邦彥弟阿倫,遂抵貴陽城下,賊解圍去。雲、永安請三善入城,三善曰:「賊兵不遠,我不可即安。」營於南門外。明日,破賊澤溪,賊走渡陸廣河。居數日,左右二部兵及湖廣、廣西、四川援兵先後至。

三善以二萬人破賊十萬,有輕敵心,欲因糧於敵。舉超為總兵官,令渡陸廣,趨大方,搗安位巢,以世賞監之;總兵官張彥方渡鴨池,搗邦彥巢,以天麟監之。漢、土兵各三萬。別將都司線補袞出黃沙渡。剋期並進。超等至陸廣,連戰皆捷,彥方部將秦民屏亦破賊五大寨,諸將益輕敵。邦彥先合崇明、效良兵誘官軍深入。三年正月,超渡陸廣,賊薄之,獨山土官蒙詔先遁,官軍大敗,爭渡河,超走免,明楷被執,諸將姚旺等二十六人殲焉。賊遂攻破鴨池軍,部將覃弘化先逃,諸營盡潰,彥方退保威清,惟補袞軍獨全。

諸苗見王師失利,復蜂起。土酋何中尉進據龍里,而邦彥使李阿二圍青巖,斷定番餉道,令宋萬化、吳楚漢為左右翼,自將趨貴陽,遠近大震。三善急遣遊擊祁繼祖等取龍里,王建中、劉志敏救青巖。繼祖燔上、中、下三牌及賊百五十砦,建中亦燔賊四十八莊,龍里、定番路皆通。三善又夜遣建中、繼祖搗楚漢八姑蕩,燔莊砦二百餘,薄而攻之。賊溺死無算。萬化不知楚漢敗,詐降,三善佯許,而令諸將捲甲趨之。萬化倉皇出戰,被擒,邦彥為奪氣。群苗復效順,三善給黃幟,令樹營中。邦彥望見不敢出,增兵守鴨池、陸廣諸要害。

時崇明父子屢敗,邦彥救之,為川師敗走。總理魯欽等剿擒中尉,彥方亦追賊鴨池,而賊復乘間陷普安。總督楊述中駐沅州,畏賊。朝命屢趣,始移鎮遠。議與三善左,三善屢求退,不許。會崇明為川師所窘,逃入貴州龍場,依邦彥。三善議會師進討,述中暨諸將多持不可。三善排群議,以閏十月,自將六萬人渡烏江,次黑石,連敗賊,斬前逃將覃弘化以徇。賊乃柵漆山,日遣遊騎掠樵採者。軍中乏食,諸將請退師。三善怒曰:「汝曹欲退,不如斬吾首詣賊降!」諸將乃不敢言。三善募壯士逼漆山。緋衣峨冠,肩輿張蓋,自督陣,語將士曰:「戰不捷,此即吾致身處也。」旁一山頗峻,麾左軍據其顛。賊倉皇拔柵爭山,將士殊死戰,賊大敗,邦彥狼狽走。

三善渡渭河,降者相繼。師抵大方,入居安位第。位偕母奢社輝走火灼堡,邦彥竄織金,先所陷將楊明楷乃得還。位窘,遣使詣述中請降。述中令縛崇明父子自贖,三善責并獻邦彥,往返之間,賊得用計為備。三善以賊方平,議郡縣其地,諸苗及土司咸惴恐,益合於邦彥。三善先約四川總兵官李維新滅賊,以餉乏辭。

三善屯大方久,食盡,述中弗為援,不得已議退師。四年正月,盡焚大方廬舍而東,賊躡之。中軍參將王建中、副總兵秦民屏戰歿。官軍行且戰,至內莊,後軍為賊所斷。三善還救,士卒多奔。陳其愚者,賊心腹,先詐降,三善信之,與籌兵事,故軍中虛實賊無不知。至是遇賊,其愚故縱轡衝三善墜馬,三善知有變,急解印綬付家人,拔刀自刎,不殊,群賊擁之去。罵不屈,遂遇害。同知梁思泰、主事田景猷等四十餘人皆死。賊拘監軍副使岳具仰以要撫,具仰遣人馳蠟書於外,被殺。

三善倜儻負氣,多權略。家中州,好交四方奇士俠客,後輒得其用。救貴陽時,得邸報不視,曰:「吾方辦賊,奚暇及此?且朝議戰守紛紛,閱之徒亂人意。」其堅決如此。然性卞急,不能持重,竟敗。先以解圍功,加兵部右侍郎,既歿,巡按御史陸獻明請優恤,所司格不行。崇禎改元,贈兵部尚書,世廕錦衣僉事,立祠祭祀。九年冬,再敘解圍功,贈太子少保。

大方之役,御史貴陽徐卿伯上言:「邦彥招四方奸宄,多狡計。撫臣得勝驟進,視蠢苗不足平。不知澤溪以西,渡陸廣河,皆鳥道,深林叢箐,彼誘我深入,以木石塞路,斷其郵書,阻餉道,遮援師,則彼不勞一卒,不費一矢,而我兵已坐困矣。」後悉如其言。

岳具仰,延安人。舉於鄉,歷瀘州知州、戶部郎中。貴州亂,朝議具仰知兵,用為監軍副使。內莊之敗,監軍四人,其三得脫還,惟具仰竟死。

田景猷,貴州思南人。天啟二年甫釋褐,憤邦彥反,疏請齎敕宣諭。廷議壯之,即擢職方主事。賊方圍貴陽,景猷單騎往,曉以禍福,令釋兵歸朝。邦彥不聽,欲屈景猷,日陳寶玩以誘之,不為動。賊乃留景猷,遣其徒恐以危禍,景猷怒,拔刀擊之,其人走免。羈賊中二年,至是遇害。具仰贈光祿卿,景猷太常少卿,並錄其一子。

楊明楷者,銅仁烏羅司人。內莊之敗,明楷為中軍,免死。後從魯欽討長田賊,功最,終副總兵。

朱家民,字同人,曲靖人。萬曆三十四年舉於鄉,為涪州知州。天啟二年官貴陽知府。奉三善命,乞援兵於四川,又借河南兵,共解其圍。乃撫傷殘,招流移,寬徭賦,遠邇悅服。丁父憂,奪情,擢安普監軍副使,加右參政。崇禎時,就遷按察使、左布政,以平寇功,加俸一級。久之,致仕歸,卒。自邦彥始亂,雲、貴諸土酋盡反,攻陷安南等上六衛,雲南路斷。其後路雖通,群苗猶出沒為患。家民率參將許成名等討平盤江外阿野、魯頗諸砦,於是相度盤江西坡、板橋、海子、馬場諸要害,築石城五,宿兵衛民。又於其間築六城,廨舍廬井畢備。群苗惕息,行旅晏然。盤江居雲、貴交,兩山夾峙,一水中絕,湍激迅悍,舟濟者多陷溺。家民仿瀾滄橋制,冶鐵為糸亙三十有六,長數百丈,貫兩崖之石而懸之,覆以板,類於蜀之棧,而道始通。

蔡復一,字敬夫,同安人。萬歷二十三年進士。除刑部主事,歷兵部郎中。居郎署十七年,始遷湖廣參政,分守湖北。進按察使、右布政使,以疾歸。光宗立,起故官,遷山西左布政使。

天啟二年,以右副都御史撫治鄖陽。歲大旱,布衣素冠,自繫於獄,遂大雨。奢崇明、安邦彥反,貴州巡撫王三善敗歿,進復一兵部右侍郎代之。兵燹之餘,斗米值一金,復一勞徠拊循,人心始定。尋代楊述中總督貴州、雲南、湖廣軍務,兼巡撫貴州,賜尚方劍,便宜從事。復一乃召集將吏,申嚴紀律,遣總理魯欽等救凱里,斬賊眾五百餘。賊圍普定,遣參將尹伸、副使楊世賞救,卻之,搗其巢,斬首千二百級。發兵通盤江路,斬逆酋沙國珍及從賊五百。欽與總兵黃鉞等復破賊於汪家沖、蔣義寨,斬首二千二百,長驅織金。織金者,邦彥巢也,緣道皆重關疊隘,木石塞山徑,將士用巨斧開之,或攀藤穿竇而入。賊戰敗,遁深箐,斬首復千級。窮搜不得邦彥,乃班師。是役也,焚賊巢數十里,獲牛馬、甲仗無算。復一以鄰境不協討,致賊未滅,請敕四川出兵遵義,抵水西,雲南出兵霑益,抵烏撒,犄角平賊。帝悉可之。因命廣西、雲南、四川諸郡鄰貴州者,聽復一節制。

五年正月,欽等旋師渡河。賊從後襲擊,諸營盡潰,死者數千人。時復一為總督,而硃燮元亦以尚書督四川、湖廣、陜西諸軍,以故復一節制不行於境外。欽等深入,四川、雲南兵皆不至。復一自劾,因論事權不一,故敗。巡按御史傅宗龍亦以為言,廷議移燮元督河道,令復一專督五路師。御史楊維垣獨言燮元不可易,帝從之,解復一任聽勘,而以王瑊為右僉都御史,代撫貴州。

復一候代,仍拮據兵事,與宗龍計,剿破烏粟、螺蝦、長田及兩江十五砦叛苗,斬七百餘級。賊黨安效良首助邦彥陷霑益,雲南巡撫沈儆炌遣兵討之,未定,遷侍郎去。代者閔洪學,招撫之,亦未定。及是見雲南出師,懼,約邦彥犯曲靖、尋甸。復一遣許成名往援,賊望風遁。又遣劉超等討平越苗阿秩等,破百七十砦,斬級二千三百有奇。至十月,復一卒於平越軍中。訃聞,帝嘉其忠勤,贈兵部尚書,謚清憲,任一子官。

復一好古博學,善屬文,耿介負大節。既歿,橐無遺貲。

瑊既至,見邦彥不易平,欲解去。夤緣魏黨李魯生,遷南京戶部右侍郎。崇禎初,被劾歸。流賊陷應城,遇害。

沈儆炌,字叔永,歸安人。父子木,官南京右都御史。儆炌登萬曆十七年進士。歷河南左布政使,入為光祿卿。四十七年,以右副都御史巡撫雲南。神宗末,詔增歲貢黃金二千,儆炌疏爭。會光宗立,如其請。

雲龍州土舍段進志掠永昌、大理,儆炌討擒之。安邦彥反,諸土目並起。安效良陷霑益,李賢陷平夷,祿千鐘犯尋甸、嵩明,張世臣攻武定,邦彥女弟設科掠曲靖,轉寇陸涼。儆炌起故參將雲南人袁善,令率守備金為貴、土官沙源等馳救嵩明,大破之。賊轉寇尋甸,復大敗去。乃請復善故官,與諸將分討賊,數有功。會儆蠙遷南京兵部右侍郎,而代者閔洪學至,乃以兵事委之去。後拜南京工部尚書,為魏忠賢黨石三畏所劾,落職閒住。崇禎初,復官,卒於家。子允培,禮科都給事中。

洪學既至,亦任用袁善。賊陷普安,圍安南,善攻破之,通上六衛道。王三善之歿,六衛復梗,善護御史傅宗龍赴黔,道復通。已而敗安效良於霑益,又敗賊於炎方、馬龍。七年,御史朱泰禎核上武定、嵩明、尋甸破賊功,大小百三十三戰,斬四千六百餘級,請宣捷告廟,從之。魏忠賢等並進秩,廕子。善加都督同知,世廕錦衣指揮僉事。崇禎初,卒官。

周鴻圖,字子固,即墨人。起家歲貢生,知宿遷縣。以侯恂薦,遷貴陽同知,監紀軍事,積軍功至知府。會勻哈叛苗與邦彥相倚為亂。天啟六年春,巡撫王瑊及御史傅宗龍使監胡從儀及都司張雲鵬軍,分道搜山,所向摧破。會聞魯欽敗,賊復趨龍場助邦彥。已而邦彥屢敗,賊返故巢。鴻圖、從儀等攻之,破焚一百餘寨,斬首千二百餘級。鴻圖擢副使,分巡新鎮道;從儀進副總兵。當是時,鴻圖駐平越,轄下六衛,參議段伯炌駐安莊,轄上六衛。千餘里間,奸宄屏息,兩人力也。鴻圖終陜西參政。

伯炌,雲南晉寧人。由鄉舉為鎮寧知州。力拒安邦彥,超擢僉事,分巡鎮寧。邦彥寇普定,偕從儀擊破之,由此擢參議。

胡從儀,山西人。天啟四年,以遊擊援普定,功多。既而破賊長田。以參將討平勻哈,後又與諸將平老蟲添。崇禎三年,討平苗賊汪狂、抱角,召為保定總兵官,卒於京邸。贈都督僉事。黔人愛之,為立真將軍碑。

贊曰:奢、安之亂,竊發於蜀,蔓延於黔,勞師者幾十載。燮元戡之以兵威,因俗制宜,開屯設衛,不亟亟焉郡縣其地,以蹈三善之覆轍,而西南由滋永寧,庶幾可方趙營平之制羌、韋南康之鎮蜀者歟。


\end{pinyinscope}