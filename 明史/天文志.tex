\article{天文志}

\begin{pinyinscope}
自司馬遷述《天官》,而歷代作史者皆志天文。惟《遼史》獨否,謂天象昭垂,千古如一,日食、天變既著本紀一說為前四篇。論及認識論、邏輯學,兼及自然科學。傳本,則天文志近於衍。其說頗當。夫《周髀》、《宣夜》之書,安天、窮天、昕天之論,以及星官占驗之說,晉史已詳,又見《隋志》,謂非衍可乎。論者謂天文志首推晉、隋,尚有此病,其他可知矣。然因此遂廢天文不志,亦非也。天象雖無古今之異,而談天之家,測天之器,往往後勝於前。無以志之,使一代制作之義泯焉無傳,是亦史法之缺漏也。至於彗孛飛流,暈適背抱,天之所以示儆戒者,本紀中不可盡載,安得不別志之。明神宗時,西洋人利瑪竇等入中國,精於天文、曆算之學,發微闡奧,運算制器,前此未嘗有也。茲掇其要,論著於篇。而《實錄》所載天象星變殆不勝書,擇其尤異者存之。日食備載本紀,故不復書。

▲兩儀

《楚詞》言「圜則九重,孰營度之」,渾天家言「天包地如卵裏黃」,則天有九重,地為渾圓,古人已言之矣。西洋之說,既不背於古,而有驗於天,故表出之。

其言九重天也,曰最上為宗動天,無星辰,每日帶各重天,自東而西左旋一周,次曰列宿天,次曰填星天,次曰歲星天,次曰熒惑天,次曰太陽天,次曰金星天,次曰水星天,最下曰太陰天。自恒星天以下八重天,皆隨宗動天左旋。然各天皆有右旋之度,自西而東,與蟻行磨上之喻相符。其右旋之度,雖與古有增減,然無大異。惟恒星之行,即古歲差之度。古謂恒星千古不移,而黃道之節氣每歲西退。彼則謂黃道終古不動,而恒星每歲東行。由今考之,恒星實有動移,其說不謬。至於分周天為三百六十度,命日為九十六刻,使每時得八刻無奇零,以之布算製器,甚便也。

其言地圓也,曰地居天中,其體渾圓,與天度相應。中國當赤道之北,故北極常現,南極常隱。南行二百五十里則北極低一度,北行二百五十里則北極高一度。東西亦然。亦二百五十里差一度也。以周天度計之,知地之全周為九萬里也。以周徑密率求之,得地之全徑為二萬八千六百四十七里又九分里之八也。又以南北緯度定天下之縱。凡北極出地之度同,則四時寒暑靡不同。若南極出地之度與北極出地之度同,則其晝夜永短靡不同。惟時令相反,此之春,彼為秋,此之夏,彼為冬耳。以東西經度定天下之衡,兩地經度相去三十度,則時刻差一辰。若相距一百八十度,則晝夜相反焉。其說與《元史》札馬魯丁地圓之旨略同。

▲七政

日月五星各有一重天,其天皆不與地同心,故其距地有高卑之不同。其最高最卑之數,皆以地半徑準之。太陽最高距地為地半徑者一千一百八十二,最卑一千一百零二。太陰最高五十八,最卑五十二。填星最高一萬二千九百三十二,最卑九千一百七十五。歲星最高六千一百九十,最卑五千九百一十九。熒惑最高二千九百九十八,最卑二百二十二。太白最高一千九百八十五,最卑三百。辰星最高一千六百五十九,最卑六百二十五。若欲得七政去地之里數,則以地半徑一萬二千三百二十四里通之。

又謂填星形如瓜,兩側有兩小星如耳。歲星四周有四小星,繞行甚疾。太白光有盈缺,如月之弦望。用窺遠鏡視之,皆可悉睹也。餘詳《曆志》。

▲恆星

崇禎初,禮部尚書徐光啟督修曆法,上《見界總星圖》。以為回回《立成》所載,有黃道經緯度者止二百七十八星,其繪圖者止十七座九十四星,並無赤道經緯。今皆崇禎元年所測,黃赤二道經緯度畢具。後又上《赤道兩總星圖》。其說謂常現常隱之界,隨北極高下而殊,圖不能限。且天度近極則漸狹,而《見界圖》從赤道以南,其度反寬,所繪星座不合仰觀。因從赤道中剖渾天為二,一以北極為心,一以南極為心。從心至周,皆九十度,合之得一百八十度者,赤道緯度也。周分三百六十度者,赤道經度也。乃依各星之經緯點之,遠近位置形勢皆合天象。

至於恒星循黃道右旋,惟黃道緯度無古今之異,而赤道經緯則歲歲不同。然亦有黃赤俱差,甚至前後易次者。如觜宿距星,唐測在參前三度,元測在參前五分,今測已侵入參宿。故舊法先觜後參,今不得不先參後觜,不可強也。

又有古多今少,古有今無者。如紫微垣中六甲六星今止有一,華蓋十六星今止有四,傳舍九星今五,天廚六星今五,天牢六星今二。又如天理、四勢、五帝內座、天柱、天床、大贊府、大理、女御、內廚,皆全無也。天市垣之市樓六星今二。太微垣之常陳七星今三,郎位十五星今十。長垣四星今二。五諸侯五星全無也。角宿中之庫樓十星今八。亢宿中之折威七星今無。氐宿中之亢池六星今四,帝席三星今無。尾宿中天龜五星今四。斗宿中之鱉十四星今十三,天籥、農丈人俱無。牛宿中之羅堰三星今二,天田九星俱無。女宿中之趙、周、秦、代各二星今各一,扶匡七星今四,離珠五星今無。虛宿中之司危、司祿各二星今各一,敗臼四星今二,離瑜三星今二,天壘城十三星今五。危宿中之人五星今三,杵三星今一,臼四星今三,車府七星今五,天鉤九星今六,天鈔十星今四,蓋屋二星今一。室宿中之羽林軍四十五星今二十六,螣蛇二十二星今十五,八魁九星今無。壁宿中之天廄十星今三。奎宿中之天溷七星今四。畢宿中之天節八星今七,咸池三星今無。觜宿中之座旗九星今五。井宿中之軍井十三星今五。鬼宿中之外廚六星今五。張宿中之天廟十四星今無。翼宿中之東甌五星今無。軫宿中之青丘七星今三,其軍門、土司空、器府俱無也。

又有古無今有者。策星旁有客星,萬歷元年新出,先大今小。南極諸星,古所未有,近年浮海之人至赤道以南,往往見之,因測其經緯度。其餘增入之星甚多,並詳《恒星表》。

其論雲漢,起尾宿,分兩派。一經天江、南海、市樓,過宗人、宗星,涉天津至螣蛇。一由箕、斗、天弁、河鼓、左右旗,涉天津至車府而會於螣蛇,過造父,直趨附路、閣道、大陵、天船,漸下而南行,歷五車、天關、司怪、水府,傍東井,入四瀆,過闕丘、弧矢、天狗之墟,抵天社、海石之南,踰南船,帶海山,置十字架、蜜蜂,傍馬腹,經南門,絡三角、龜、杵,而屬於尾宿,是為帶天一周。以理推之,隱界自應有雲漢,其所見當不誣。又謂雲漢為無數小星,大陵鬼宿中積尸亦然。考《天官書》言星漢皆金之散氣,則星漢本同類,得此可以相證。又言昴宿有三十六星,皆得之於窺遠鏡者。

凡測而入表之星共一千三百四十七,微細無名者不與。其大小分為六等:內一等十六星,二等六十七星,三等二百零七星,四等五百零三星,五等三百三十八星,六等二百一十六星。悉具黃赤二道經緯度。列表二卷,入光啟所修《崇禎歷書》中。

茲取二十八宿距星及一二等大星存之,其小而有名者,間取一二,備列左方。

表格略

▲黃赤宿度

崇禎元年所測二十八宿黃赤度分,皆不合於古。夫星既依黃道行,而赤道與黃道斜交,其度不能無增減者,勢也。而黃道度亦有增減者,或推測有得失,抑恒星之行亦或各有遲速歟。謹列其數,以備參考。

赤道宿度周天三百六十度,每度六十分。黃道同黃道宿度

角,一十一度四十四分。一十度三十五分。

亢,九度一十九分。一十度四十分。

氐,一十六度四十一分。一十七度五十四分。

房,五度二十八分。四度四十六分。

心,六度零九分。七度三十三分。

尾,二十一度零六分。一十五度三十六分。

箕,八度四十六分。九度二十分。

斗,二十四度二十四分。二十三度五十一分。

牛,六度五十分。七度四十一分。

女,一十一度零七分。一十一度三十九分。

虛,八度四十一分。九度五十九分。

危,一十四度五十三分。二十度零七分。

室,一十七度。一十五度四十一分。

壁,一十度二十八分。一十三度一十六分。

奎,一十四度三十分。一十一度二十九分。

婁,一十二度零四分。一十三度。

胃,一十五度四十五分。一十三度零一分。

昴,一十度二十四分。八度二十九分。

畢,一十六度三十四分。一十三度五十八分。

參,二十四分。一度二十一分。

觜,一十一度二十四分。一十一度三十三分。

井,三十二度四十九分。三十度二十五分。

鬼,二度二十一分。五度三十分。

柳,一十二度零四分。一十六度零六分。

星,五度四十八分。八度二十三分。

張,一十七度一十九分。一十八度零四分。

翼,二十度二十八分。一十七度。

軫,一十五度三十分。一十三度零三分。

▲黃赤宮界

十二宮之名見於《爾雅》,大抵皆依星宿而定。如婁、奎為降婁,心為大火,朱鳥七宿為鶉首、鶉尾之類。故宮有一定之宿,宿有常居之宮,由來尚矣。唐以後始用歲差,然亦天自為天,歲自為歲,宮與星仍舊不易。西洋之法,以中氣過宮,如日躔冬至,即為星紀宮之類。而恒星既有歲進之差,於是宮無定宿,而宿可以遞居各宮,此變古法之大端也。茲以崇禎元年各宿交宮之黃赤度,分列於左方,以志權輿云。

赤道交宮宿度黃道交宮宿度

箕,三度零七分,入星紀。箕,四度一十七分,入星紀。

斗,二十四度二十一分,入玄枵。牛,一度零六分,入玄枵

危,三度一十九分,入娵訾。危,一度四十七分,入娵訾。

壁,一度二十六分,入降婁。室,一十一度四十分,入降婁。

婁,六度二十八分,入大梁。婁,一度一十四分,入大梁。

昴,八度三十九分,入實沈。昴,五度一十三分,入實沈。

觜,一十一度一十七分,入鶉首。觜,一十一度二十五分,入鶉首。

井,二十九度五十三分,入鶉火。井,二十九度五十二分,入鶉火。

張,六度五十一分,入鶉尾。星,七度五十一分,入鶉尾。

翼,一十九度三十二分,入壽星。翼,一十一度二十四分,入壽星。

亢,一度五十分,入大火。亢,初度四十六分,入大火。

心,初度二十二分,入析木。房,二度一十二分,入析木。

▲儀象

璇璣玉衡為儀象之權輿,然不見用於三代。《周禮》有圭表、壺漏,而無璣衡,其制遂不可考。漢人創造渾天儀,謂即璣衡遺制,其或然歟。厥後代有制作。大抵以六合、三辰、四游、重環湊合者,謂之渾天儀;以實體圓球,繪黃赤經緯度,或綴以星宿者,謂之渾天象。其制雖有詳略,要亦青藍之別也。外此則圭表、壺漏而已。迨元作簡儀、仰儀、窺几、景符之屬,制器始精詳矣。

明太祖平元,司天監進水晶刻漏,中設二木偶人,能按時自擊鉦鼓。太祖以其無益而碎之。洪武十七年,造觀星盤。十八年,設觀象臺於雞鳴山。二十四年,鑄渾天儀。正統二年,行在欽天監正皇甫仲和奏言:「南京觀象臺設渾天儀、簡儀、圭表以窺測七政行度,而北京乃止於齊化門城上觀測,未有儀象。乞令本監官往南京,用木做造,挈赴北京,以較驗北極出地高下,然後用銅別鑄,庶幾占測有憑。」從之。明年冬,乃鑄銅渾天儀、簡儀於北京。御製《觀天器銘》。其詞曰:「粵古大聖,體天施治,敬天以心,觀天以器。厥器伊何?璇璣玉衡。璣象天體,衡審天行。歷世代更,垂四千祀,沿制有作,其制寢備。即器而觀,六合外儀,陽經陰緯,方位可稽。中儀三辰,黃赤二道,日月暨星,運行可考。內儀四遊,橫簫中貫,南北東西,低昂旋轉。簡儀之作,爰代璣衡,制約用密,疏朗而精。外有渾象,反而觀諸,上規下矩,度數方隅。別有直表,其崇八尺,分至氣序,考景咸得。縣象在天,制器在人,測驗推步,靡忒毫分。昔作今述,為制彌工,既明且悉,用將無窮。惟天勤民,事天首務,民不失寧,天其予顧。政純於仁,天道以正,勒銘斯器,以勵予敬。」十一年,監臣言:「簡儀未刻度數,且地基卑下,窺測日星,為四面臺宇所蔽。圭表置露臺,光皆四散,影無定則。壺漏屋低,夜天池促,難以注水調品時刻。請更如法修造。」報可。明年冬,監正彭德清又言:「北京北極出地度、太陽出入時刻與南京不同,冬夏晝長夜短亦異。今宮禁及官府漏箭皆南京舊式,不可用。」有旨,令內官監改造。景泰六年又造內觀象臺簡儀及銅壺。成化中,尚書周洪謨復請造璇璣玉衡,憲宗令自製以進。十四年,監臣請修晷影堂,從之。

弘治二年,監正吳昊言:「考驗四正日度,黃赤二道應交於壁軫。觀象臺舊制渾儀,黃赤二道交於奎軫,不合天象,其南北兩軸不合兩極出入之度,窺管又不與太陽出沒相當,故雖設而不用。所用簡儀則郭守敬遺制,而北極雲柱差短,以測經星去極,亦不能無爽。請修改或別造,以成一代之制。」事下禮部,覆議令監副張紳造木樣,以待試驗,黃道度許修改焉。正德十六年,漏刻博士朱裕復言:「晷表尺寸不一,難以準測,而推算曆數用南京日出分秒,似相矛盾。請敕大臣一員總理其事,鑄立銅表,考四時日中之影。仍於河南陽城察舊立土圭,以合今日之晷,及分立圭表於山東、湖廣、陜西、大名等處,以測四方之影。然後將內外晷影新舊曆書錯綜參驗,撰成定法,庶幾天行合而交食不謬。」疏入不報。嘉靖二年修相風桿及簡、渾二儀。七年始立四丈木表以測晷影,定氣朔。由是欽天監之立運儀、正方案、懸晷、偏晷、盤晷諸式具備於觀象臺,一以元法為斷。

萬曆中,西洋人利瑪竇制渾儀、天球、地球等器。仁和李之藻撰《渾天儀說》,發明製造施用之法,文多不載。其製不外於六合、三辰、四游之法。但古法北極出地,鑄為定度,此則子午提規,可以隨地度高下,於用為便耳。

崇禎二年,禮部侍郎徐光啟兼理曆法,請造象限大儀六,紀限大儀三,平懸渾儀三,交食儀一,列宿經緯天球一,萬國經緯地球一,平面日晷三,轉盤星晷三,候時鐘三,望遠鏡三。報允。已,又言:

定時之法,當議者五事:一曰壺漏,二曰指南鍼,三曰表臬,四曰儀,五曰晷。

漏壺,水有新舊滑濇則遲疾異,漏管有時塞時磷則緩急異。正漏之初,必於正午初刻。此刻一誤,靡所不誤。故壺漏特以濟晨昏陰晦儀晷表臬所不及,而非定時之本。

指南鍼,術人用以定南北,辨方正位咸取則焉。然鍼非指正子午,曩云多偏丙午之間。以法考之,各地不同。在京師則偏東五度四十分。若憑以造晷,冬至午正先天一刻四十四分有奇,夏至午正先天五十一分有奇。

若表臬者,即《考工》匠人置埶木之法,識日出入之影,參諸日中之影,以正方位。今法置小表於地平,午正前後累測日影,以求相等之兩長影為東西,因得中間最短之影為正子午,其術簡甚。

儀者,本臺故有立運儀,測驗七政高度。臣用以較定子午,於午前屢測太陽高度,因最高之度,即得最短之影,是為南北正線。

既定子午卯酉之正線,因以法分布時刻,加入節氣諸線,即成平面日晷。又今所用員石欹晷是為赤道晷,亦用所得正子午線較定。此二晷皆可得天之正時刻,所為晝測日也。若測星之晷,實《周禮》夜考極星之法。然古時北極星正當不動之處,今時久漸移,已去不動處三度有奇,舊法不可復用。故用重盤星晷,上書時刻,下書節氣,仰測近極二星即得時刻,所謂夜測星也。

七年,督修曆法右參政李天經言:

輔臣光啟言定時之法,古有壺漏,近有輪鐘,二者皆由人力遷就,不如求端於日星,以天合天,乃為本法,特請制日晷、星晷、望遠鏡三器。臣奉命接管,敢先言其略。

日晷者,礱石為平面,界節氣十三線,內冬夏二至各一線,其餘日行相等之節氣,皆兩節氣同一線也。平面之周列時刻線,以各節氣太陽出入為限。又依京師北極出地度,範為三角銅表置其中。表體之全影指時刻,表中之銳影指節氣。此日晷之大略也。

星晷者,治銅為柱,上安重盤。內盤鐫周天度數,列十二宮以分節氣,外盤鐫列時刻,中橫刻一縫,用以窺星。法將外盤子正初刻移對內盤節氣,乃轉移銅盤北望帝星與句陳大星,使兩星同見縫中,即視盤面銳表所指,為正時刻。此星晷之大略也。

若夫望遠鏡,亦名窺筒,其制虛管層疊相套,使可伸縮,兩端俱用玻璃,隨所視物之遠近以為長短。不但可以窺天象,且能攝數里外物如在目前,可以望敵施砲,有大用焉。

至於日晷、星晷皆用措置得宜,必須築臺,以便安放。

帝命太監盧維寧、魏國徵至局驗試用法。

明年,天經又請造沙漏。明初,詹希元以水漏至嚴寒水凍輒不能行,故以沙代水。然沙行太疾,未協天運,乃以斗輪之外復加四輪,輪皆三十六齒。厥後周述學病其竅太小,而沙易堙,乃更制為六輪,其五輪悉三十齒,而微裕其竅,運行始與晷協。天經所請,殆其遺意歟。

夫制器尚象,乃天文家之首務。然精其術者可以因心而作。故西洋人測天之器,其名未易悉數,內渾蓋、簡平二儀其最精者也。其說具見全書,茲不載。

▲極度晷影

宣城梅文鼎曰:

極度晷影常相因。知北極出地之高,即可知各節氣午正之影。測得各節氣午正之影,亦可知北極之高。然其術非易易也。圭表之法,表短則分秒難明,表長則影虛而淡。郭守敬所以立四丈之表,用影符以取之也。日體甚大,豎表所測者日體上邊之影,橫表所測者日體下邊之影,皆非中心之數,郭守敬所以於表端架橫梁以測之也,其術可謂善矣。但其影符之制,用銅片鑽鍼芥之孔,雖前低後仰以向太陽,但太陽之高低每日不同,銅片之欹側安能俱合。不合則光不透,臨時遷就,而日已西移矣。須易銅片以圓木,左右用兩板架之,如車軸然,則轉動甚易。更易圓孔以直縫,而用始便也。然影符止可去虛淡之弊,而非其本。必須正其表焉,平其圭焉,均其度焉,三者缺一,不可以得影。三者得矣,而人心有粗細,目力有利鈍,任事有誠偽,不可不擇也。知乎此,庶幾晷影可得矣。

西洋之法又有進焉。謂地半徑居日天半徑千餘分之一,則地面所測太陽之高,必少於地心之實高,於是有地半徑差之加。近地有清蒙氣,能升卑為高,則晷影所推太陽之高,或多於天上之實高,於是又有清蒙差之減。是二差者,皆近地多而漸高漸減,以至於無,地半徑差至天頂而無,清蒙差至四十五度而無也。

崇禎初,西洋人測得京省北極出地度分:北京四十度,周天三百六十度,度六十分立算,下同。南京三十二度半,山東三十七度,山西三十八度,陜西三十六度,河南三十五度,浙江三十度,江西二十九度,湖廣三十一度,四川二十九度,廣東二十三度,福建二十六度,廣西二十五度,雲南二十二度,貴州二十四度。以上極度,惟兩京、江西、廣東四處皆系實測,其餘則據地圖約計之。又以十二度度六十分之表測京師各節氣午正日影:夏至三度三十三分,芒種、小暑三度四十二分,小滿、大暑四度十五分,立夏、立秋五度六分,穀雨、處暑六度二十三分,清明、白露八度六分,春、秋分十度四分,驚蟄、寒露十二度二十六分,雨水、霜降十五度五分,立春、立冬十七度四十七分,大寒、小雪二十度四十七分,小寒、大雪二十三度三十分,冬至二十四度四分。

▲東西偏度

以京師子午線為中,而較各地所偏之度。凡節氣之早晚,月食之先後,胥視此。蓋人各以見日出入為東西為卯酉,以日中為南為午。而東方見日早,西方見日遲。東西相距三十度則差一時。東方之午乃西方之巳,西方之午乃東方之未也。相距九十度則差三時。東方之午乃西方之卯,西方之午乃東方之酉也。相距一百八十度則晝夜時刻俱反對矣。東方之午乃西方之子。西洋人湯若望曰:「天啟三年九月十五夜,戌初初刻望,月食,京師初虧在酉初一刻十二分,而西洋意大里雅諸國望在晝,不見。推其初虧在巳正三刻四分,相差三時二刻八分,以里差計之,殆距京師之西九十九度半也。故欲定東西偏度,必須兩地同測一月食,較其時刻。若早六十分時之二則為偏西一度,遲六十分時之二則為偏東一度。節氣之遲早亦同。今各省差數未得測驗,據廣輿圖計里之方約略條列,或不致甚舛也。南京應天府、福建福州府並偏東一度,山東濟南府偏東一度十五分,山西太原府偏西六度,湖廣武昌府、河南開封府偏西三度四十五分,陜西西安府、廣西桂林府偏西八度半,浙江杭州府偏東三度,江西南昌府偏西二度半,廣東廣州府偏西五度,四川成都府偏四十三度,貴州貴陽府偏西九度半,雲南雲南府偏西十七度。」

右偏度,載《崇禎曆書》交食曆指。其時開局修曆,未暇分測,度數實多未確,存之以備考訂云。

▲中星

古今中星不同,由於歲差。而歲差之說,中西復異。中法謂節氣差而西,西法謂恒星差而東,然其歸一也。今將李天經、湯若望等所推崇禎元年京師昏旦時刻中星列於後。

春分,戌初二刻五分昏,北河三中;寅正一刻一十分旦,尾中。清明,戌初三刻十三分昏,七星偏東四度;昏旦時或無正中之星,則取中前、中後之大星用之。距中三度以內者,為時不及一刻,可勿論。四度以上,去中稍遠,故紀其偏度焉。寅正初刻二分旦,帝座中。穀雨,戌正一刻七分昏,翼偏東七度;寅初二刻八分旦,箕偏東四度。立夏,戌正三刻二分昏,軫偏東五度;寅初初刻十三分旦,箕偏西四度。小滿,亥初初刻十二分昏,角中;丑正三刻三分旦,箕中。芒種,亥初一刻十二分昏,大角偏西六度;丑正二刻三分旦,河鼓二中。

夏至,亥初二刻五分昏,房中;丑正一刻一十分旦,須女中。小暑,亥初一刻十二分昏,尾中;丑正二刻三分旦,危中。大暑,亥初初刻十二分昏,箕偏東七度;丑正三刻三分旦,營室中。立秋,戌正三刻二分昏,箕中;寅初三刻十三分旦,婁偏東六度。處暑,戌正一刻七分昏,織女一中;寅初二刻八分旦,婁中。白露,戌初三刻十三分昏,河鼓二偏東四度;寅正初刻二分旦,昴偏東四度。

秋分,戌初二刻五分昏,河鼓二中;寅正一刻十一分旦,畢偏西五度。寒露,戌初初刻十四分昏,牽牛中;寅正三刻一分旦,參四中。霜降,酉正三刻十一分昏,須女偏西五度;卯初初刻四分旦,南河三偏東六度。立冬,酉正二刻一十分昏,危偏東四度;卯初一刻五分旦,輿鬼中。小雪,酉正一刻十二分昏,營室偏東七度;卯初二刻二分旦,張中。大雪,酉正一刻五分昏,營室偏西八度;卯初二刻一十分旦,翼中。

冬至,酉正一刻二分昏,土司空中;卯初二刻十三分旦,五帝座中。小寒,酉正一刻五分昏,婁中;卯初二刻一十分旦,角偏東五度。大寒,酉正一刻十三分昏,天囷一中;卯初二刻二分旦,亢中。立春,酉正二刻一十分昏,昴偏西六度;卯初一刻五分旦,氐中。雨水,酉正三刻十一分昏,參七中;卯初初刻四分旦,貫索一中。驚蟄,戌初初刻十四分昏,天狼中;寅正三刻一分旦,心中。

▲分野

《周禮·保章氏》以星土辨九州之地,所封之域皆有分星,以觀妖祥。唐貞觀中,李淳風撰《法象志》,因《漢書》十二次度數以唐州縣配,而一行則以為天下山河之象,存乎南北兩界,其說詳矣。洪武十七年,《大明清類天文分野書》成,頒賜秦、晉二王。其書大略謂「《晉天文志》分野始角、亢者,以東方蒼龍為首也。唐始女、虛、危者,以十二支子為首也。今始斗、牛者,以星紀為首也。古言天者皆由斗、牛以紀星,故曰星紀,是之取耳。」茲取其所配直隸十三布政司府州縣衛及遼東都司分星錄之。

斗三度至女一度,星紀之次也。直隸所屬之應天、太平、寧國、鎮江、池州、徽州、常州、蘇州、松江九府暨廣德州,屬斗分。鳳陽府壽、滁、六安三州,泗州之盱眙、天長二縣,揚州府高郵、通、泰三州,廬州府無為州,安慶府和州,皆斗分。淮安府,斗、牛分。浙江布政司所屬之杭州、湖州、嘉興、嚴州、紹興、金華、衢州、處州、寧波九府皆牛、女分。台州、溫州二府,斗、牛、須、女分。江西布政司所屬皆斗分。福建布政司所屬皆牛、女分。廣東布政司所屬之廣州府亦牛、女分。惠州,女分。肇慶、南雄二府,德慶州,皆牛、女分。潮州府,牛分。雷州、瓊州二府,崖、儋、萬三州,高州府化州,廣西布政司所屬梧州府之蒼梧、藤、岑溪、容四縣,皆牛、女分。

女二度至危十二度,玄枵之次也。山東布政司所屬之濟南府樂安、德、濱三州,皆危分。泰安州、青州府,皆虛、危分。萊州府膠州、登州府寧海州、東昌府高塘州,皆危分。東平州之陽穀、東阿、平陰三縣,北平布政司所屬之滄州,皆須、女、虛、危分。

危十三度至奎一度,娵訾之次也。河南布政司所屬之衛輝、彰德、懷慶三府,北平之大名府開州,山東東昌之濮州,館陶、冠、臨清三縣,東平州之汶上、壽張二縣,皆室、壁分。

奎二度至胃三度,降婁之次也。山東濟寧府之兗州滕、嶧二縣,青州府之莒州、安丘、諸城、蒙陰三縣,濟南府之沂州,直隸鳳陽府之泗、邳二州,五河、虹、懷遠三縣,淮安府之海州,桃源、清河、沭陽三縣,皆奎、婁分。

胃四度至畢六度,大梁之次也。北平之真定府,昴、畢分。定、冀二州,皆昴分。晉、深、趙三州,皆畢分。廣平、順德二府,皆昴分。祁州,昴、畢分。河南彰德府之磁州,山東高唐州之恩縣,山西布政司所屬之大同府應、朔、渾源、蔚四州,皆昴、畢分。

畢七度至井八度,實沈之次也。山西之太原府石、忻、代、平定、保德、岢嵐六州,平陽府,皆參分。絳、蒲、吉、隰、解、霍六州皆觜、參分。澤、汾二州,皆參分。潞、沁、遼三州,皆參、井分。

井九度至柳三度,鶉首之次也。陜西布政司所屬之西安府同、華、乾、耀、邠五州,鳳翔府隴州,延安府鄜、綏德、葭三州,漢中府金州,臨洮、平涼二府,靜寧州,皆井、鬼分。涇州,鬼分。慶陽府寧州,鞏昌府階、徽、秦三州,皆井、鬼分。四川布政司所屬惟綿州觜分,合州參、井分,餘皆井、鬼分。雲南布政司所屬皆井、鬼分。

柳四度至張十五度,鶉火之次也。河南之河南府陜州,皆柳分。南陽府鄧、汝、裕三州,汝寧府之信陽、羅山二縣,開封府之均、許二州,陜西西安府之商縣,華州之洛南縣,湖廣布政司所屬德安府之隨州,襄陽府之均州、光化縣,皆張分。

張十六度至軫九度,鶉尾之次也。湖廣之武昌府興國州,荊州府歸、夷陵、荊門三州,黃州府蘄州,襄陽、德安二府,安陸、沔陽二州,皆翼、軫分。長沙府軫旁小星曰長沙,應其地。衡州府桂陽州,永州府全、道二州,岳州、常德二府,澧州,辰州府沅州,漢陽府靖、郴二州,寶慶府武岡、鎮遠二州,皆翼、軫分。廣西所屬除梧州府之蒼梧、藤、容、岑溪四縣屬牛、女分,餘皆翼、軫分。廣東之連州、廉州府欽州、韶州府,皆翼、軫分。

軫十度至氐一度,壽星之次也。河南之開封府,角、亢分。鄭州,氐分。陳州,亢分。汝寧府光州,懷慶府之孟、濟源、溫三縣,直隸壽州之霍丘縣,皆角、亢、氐分。

氐二度至尾二度,大火之次也。河南開封府之杞、太康、儀封、蘭陽四縣,歸德、睢二州,山東之濟寧府,皆房、心分。直隸鳳陽府之潁州,房分。徐、宿二州,壽州之蒙城縣,潁州之亳縣,皆房、心分。

尾三度至斗二度,析木之次也。北平之北平府,尾、箕分。涿、通、薊三州,皆尾分。霸州、保定府,皆尾、箕分。易、安二州,皆尾分。河間府、景州,皆尾、箕分。永平府,尾分。灤州,尾、箕分。遼東都指揮司,尾、箕分。朝鮮,箕分。

▲月掩犯五緯

洪武元年五月甲申,犯填星。十二年三月戊辰朔,犯辰星。十四年十一月甲午,犯填星。十九年五月己未,犯歲星。二十三年四月丁酉,掩太白。十一月癸卯及永樂四年正月戊午,五年六月丙午,七年十二月壬子,俱犯熒惑。八年十二月壬子,九年四月庚子,十六年七月戊辰,俱犯歲星。十八年十一月辛卯,掩太白。二十年三月辛未,掩填星。二十二年八月乙丑,犯熒惑。

洪熙元年二月己未,掩填星。

宣德元年十二月丙子,掩熒惑。二年正月癸卯,犯熒惑。四月甲申,犯太白。六年十月丙申,掩太白。七年二月甲寅,犯填星。八年二月癸巳,掩歲星。四月戊子,犯歲星。

正統二年正月辛亥,掩歲星。四月癸酉、五月庚子,俱犯歲星。七月戊申,犯熒惑。四年正月乙酉,掩填星。八年三月庚申,犯填星。十一月丙寅,掩歲星。十年十一月辛卯,犯熒惑。十一年十二月甲寅,犯歲星。十二年正月辛巳,閏四月庚午,俱犯歲星。十四年四月壬子,犯太白。五月癸未,掩太白。

景泰二年四月戊子,犯歲星。九月甲辰,犯歲星於斗。五年二月丁亥,犯太白。六年正月甲寅,犯歲星。七年四月癸丑,犯填星。乙丑,犯太白。

天順五年十一月己亥,犯太白於斗。

成化五年二月丙申、癸亥,俱犯歲星。六年三月癸未,八年正月癸亥,俱犯太白。十二年十一月戊申,犯歲星於室。十三年十月乙卯,犯填星。十二月丁酉,犯太白。十四年三月戊辰,十八年二月戊午,俱犯填星。八月己酉,二十三年四月乙亥,俱掩熒惑。五月戊午,六月乙酉,俱犯歲星。十月甲戌,掩歲星。

弘治四年二月壬子,犯歲星。七年十一月戊申,犯熒惑。八年正月癸卯,犯歲星。十二月丙辰,掩填星。十一年四月甲申、九月庚子,俱犯歲星。十二年八月壬寅,犯熒惑。十四年七月丁卯,九月己丑,俱犯歲星。丙辰,掩歲星。十二月癸丑,犯熒惑。十七年十一月甲辰,犯歲星。十八年二月丙寅,掩歲星。九月乙巳,掩填星。

正德元年十一月己卯,犯太白。四年閏九月癸亥,犯歲星。八年正月己丑,犯填星。十六年二月丙戌,掩太白。

嘉靖二年五月戊子,掩歲星。十一月壬申,犯歲星。十七年十二月己未,犯填星。十八年十月丙戌,犯熒惑。二十年五月辛卯,犯歲星。二十一年四月甲寅,二十七年七月丁丑,俱犯太白。九月庚子,犯太白於角。三十一年五月辛丑,犯填星。九月庚寅,掩填星。十二月丁卯,犯歲星。四十二年五月庚辰,掩歲星。四十四年七月丁巳,犯熒惑。

萬曆二年九月己卯,犯熒惑於箕。十年八月戊申,犯熒惑於井。十四年八月己丑,犯太白於角。十五年六月乙丑,十九年九月辛未,俱犯熒惑。十二月甲辰,犯填星於井。二十四年正月甲申,犯填星於張。二十七年九月辛亥,犯太白。三十一年五月癸未,犯太白,三十五年六月乙未,犯填星於斗。三十七年八月辛酉,犯填星。四十一年九月癸未,犯歲星。

崇禎三年八月辛亥,掩太白。十一年四月己酉,掩熒惑於尾。

▲五緯掩犯

洪武六年三月戊申,熒惑犯填星。六月壬辰,太白犯歲星。八年三月癸亥,熒惑犯填星。二十二年六月丙辰,辰星犯太白。二十七年三月乙丑,熒惑犯歲星於奎。

永樂三年三月戊戌,太白犯歲星。十一月癸巳朔,太白犯辰星於箕。四年正月癸卯,太白犯歲星。五年七月甲子。熒惑犯填星。十二年十一月丁卯,太白犯歲星。十四年七月乙巳,太白犯填星。二十年九月乙亥,太白犯歲星。十月己酉,太白犯填星。

洪熙元年十一月丙午,太白犯填星。

宣德元年十一月戊戌,辰星犯填星。七年六月己酉,太白犯歲星。七月辛巳,太白犯熒惑。九年十一月己亥,太白犯填星。十年十月庚子,熒惑犯填星。

正統元年五月戊寅,太白犯熒惑於井。二年五月辛丑,熒惑犯填星。三年十二月戊寅,太白犯歲星。五年五月丙午,太白犯填星。七年九月戊午,太白犯熒惑於氐。十一年九月丁亥,太白犯歲星。十二年七月戊午,熒惑犯填星。十四年二月己卯,太白犯熒惑。七月丙午,熒惑犯填星。

景泰元年閏正月丁卯,熒惑犯歲星。

天順七年十一月乙卯朔,熒惑犯填星。

成化六年九月乙亥,太白犯歲星。十一年七月戊辰,太白犯填星。十三年九月丙寅,熒惑犯填星。十六年六月壬申,太白犯歲星。

弘治二年正月戊辰,太白犯歲星。十一月壬午,太白犯填星。三年正月庚申,太白犯填星。五年八月丁未,熒惑犯歲星。六年十一月己未,太白犯填星。七年九月甲寅及十年正月丙辰,熒惑犯歲星。十二月庚辰,辰星犯歲星。十七年閏四月癸酉,歲星犯填星。

正德二年十月癸未,熒惑犯填星。八年正月壬午及十六年十二月丙午,俱太白犯歲星。

嘉靖元年正月己未,太白犯歲星。十二月甲戌,太白犯填星。三年正月癸酉,太白犯歲星。二十九年六月庚辰,熒惑犯歲星守井。

萬曆五年十二月辛丑,太白犯填星於斗。九年十二月癸巳,太白犯填星入危。十一年六月丁丑,太白犯熒惑。十五年五月己亥,太白犯填星。二十四年四月己酉,太白犯歲星。二十五年七月甲辰,熒惑犯歲星。二十七年閏四月庚寅,辰星犯太白於井。三十四年十一月庚辰,熒惑掩歲星於危;甲辰,熒惑犯歲星。三十八年十一月辛亥,太白犯填星於虛。四十七年三月壬子,太白犯歲星於壁。

天啟元年八月丙申,熒惑與太白同度者兩日。

崇禎九年六月己亥,太白犯歲星於張。

▲五緯合聚

洪武十四年六月癸未,辰星、熒惑、太白聚於井。十七年六月丙戌,歲星、填星、太白聚於參。十八年二月乙巳,五星並見。三月戊子,填星、歲星、太白聚於井。二十年二月壬午朔,五星俱見。二十四年七月戊子,太白、熒惑、填星聚於翼。十一月乙未,辰星、歲星合於斗。十二月甲子,熒惑、辰星合於箕。二十五年正月辛丑,熒惑、歲星合於牛。二十六年十月壬辰,太白、填星同度。

永樂元年五月甲辰,五星俱見東方。二年四月戊子,太白、熒惑合於井。

正統十四年九月壬寅,太白、填星、熒惑聚於翼。十二月辛未,太白、歲星合於尾。

景泰元年十月壬申,太白、歲星合於箕。十二月己丑,辰星、歲星同度。二年九月庚申,太白、熒惑、填星聚於軫。四年三月乙丑,太白、歲星合於壁。五年正月戊辰,太白、歲星合於奎。六月己酉,熒惑、歲星合於胃。十一月己未,太白、填星合於氐。七年三月戊戌,太白、熒惑合於奎。十月戊申,歲星、熒惑合於鬼。

天順元年五月乙丑,太白、歲星合於井。十二月丙辰,太白、填星合於心。二年九月甲寅,太白、填星合於斗。三年九月乙巳,太白、歲星合於角。四年十月壬申,歲星、熒惑、辰星、太白聚於氐。五年十一月己亥,填星、熒惑合於牛。甲子,太白、熒惑合於虛。六年九月甲午,太白、熒惑合於張。七年十月庚寅,歲星、熒惑合於女。庚戌,太白、歲星合於女。八年二月丙午,填星、歲星、太白聚於危。

成化四年四月癸巳,歲星、熒惑合於井。壬子及七年七月庚子,太白、歲星合於井。十一年八月甲午,熒惑、填星同度。

弘治十三年四月癸丑,熒惑、太白、辰星聚於井。十六年八月庚申,熒惑、歲星、填星聚於井。十八年五月丙申,太白、歲星合於星。九月乙未,太白、歲星同度。

正德二年九月戊辰,辰星、歲星、太白聚於亢。

嘉靖三年正月壬午,五星聚於營室。十九年九月乙卯,太白、辰星、填星聚於角。二十三年正月癸卯,熒惑、歲星、填星聚於房。四十二年七月戊戌,太白、歲星、填星聚於井。四十三年四月庚子,歲星、填星、熒惑、太白聚於柳。

萬曆十七年十二月辛卯,太白、熒惑同度。二十年六月壬子,太白、辰星、填星聚於井。三十二年九月辛酉,歲星、填星、熒惑聚於危。

天啟四年七月丙寅,五星聚於張。

崇禎七年閏八月丙午至九月壬申,填星、熒惑、太白聚於尾。十年十一月己卯,歲星、熒惑合於亢。甲午,填星、辰星同度。

▲五緯掩犯恒星

△歲星

洪武六年九月庚申,犯鬼。十一月壬子,退行犯鬼。七年八月乙巳,犯軒轅大星。九年二月乙丑,退入太微,犯左執法。十年六月戊寅及戊戌,犯亢。十一月甲辰,犯房。十一年四月戊申,犯鍵閉。七月甲申,犯牛。八月丙午,犯房。十四年四月壬戌,犯壘壁。十七年閏十月癸卯,犯井。十九年四月丙申,入鬼。八月壬辰,犯軒轅。二十一年四月丁未,留太微垣。十一月甲戌,入亢。二十二年三月辛卯,退入亢。九月丁卯,犯氐。十一月甲午,入房。十二月壬戌,犯東咸。二十三年五月己未,守房。八月乙丑,犯東咸。二十六年二月丙子朔,犯壘壁。二十九年六月庚子,犯井鉞。七月丙辰朔,入井。十月癸卯,退入井。三十年八月庚辰朔,入鬼。

建文四年七月乙未,退犯東咸。十月丙辰,犯天江。

永樂元年正月丁未,犯建。十二月己丑,犯羅堰。六年三月己巳,犯諸王西第二星。四月甲午,犯東第一星。六月丙申,犯井。八年九月乙亥,犯靈臺。十八年七月己丑,犯天樽西北星。八月庚子,犯東北星。二十一年正月庚戌,犯上將。二十二年十一月戊寅,入氐。

宣德三年閏四月己酉,犯壘壁西第六星。十一月丙寅,又犯。七年七月丙寅,犯天樽。九年五月庚子,犯軒轅大星。

正統五年六月甲寅,犯壘壁。十一年十月戊戌,犯右執法。十四年正月丙申,犯房北第一星。二月丙子,退犯房。九月己卯,犯進賢。丙戌,犯房。

景泰元年閏正月庚午,與熒惑遞入斗杓。八月戊子,犯秦。二年二月庚午朔,犯牛。三年十月辛丑,犯亢。六年六月庚子,犯諸王。八月庚申,犯井鉞。七年九月癸未,入鬼。

天順元年九月癸亥,犯軒轅大星。二年八月癸未,犯右執法。十月己丑,三年正月辛卯,俱犯左執法。六月辛未,犯右執法。十二月癸亥,犯亢。四年閏十一月丙寅,犯房北第一星。庚午,犯鉤鈐。五年三月丁卯,退犯房上星。八月癸酉,犯鉤鈐。七年二月庚申朔,犯牛。八年二月丙午,犯壘壁。三月辛巳,又犯。

成化二年六月丁未,守昴。五年七月己酉,犯軒轅大星。六年三月癸卯,留守軒轅。七年三月丁丑,退入太微垣,犯執法。四月乙卯,入太微垣,留守端門。六月甲寅,犯右執法。十一月己亥,犯亢。八年十一月辛亥,犯房北第一星。癸丑,犯鉤鈐。九年三月丙辰,犯東咸。五月己酉,犯鉤鈐。六月乙丑,犯房第一星。十二年三月丁巳,犯壘壁。十三年閏二月己未,犯外屏。十五年三月甲子,犯天街。九月乙卯,犯井。辛巳,守井。十七年正月己卯,犯鬼。三月甲午,入鬼。庚子,犯積尸。十八年五月庚戌,犯靈臺。閏八月壬辰,犯左執法。二十年五月乙巳,守亢。八月癸酉,犯氐。

弘治四年七月癸巳,犯井。十一月壬辰,又犯。六年八月庚寅,犯靈臺。七年正月癸卯,犯壘壁。五月甲辰,犯靈臺。八年二月丁巳,犯進賢。七月辛丑,又犯。十月丁卯,犯亢。十一月己酉,犯氐。九年二月至三月庚寅,守氐。十二年五月己亥,犯壘壁。十三年八月戊申,又犯。十五年七月丙子,犯諸王。十六年七月己巳,犯井。八月壬子,犯天樽。十八年九月丁未,犯太微垣上相。

正德元年二月壬子,退犯右執法及上將。三月壬午,犯靈臺。十一月戊辰,犯牛。六年四月丁未,十二月壬午,俱犯壘壁。九年八月丙辰,犯諸王。十四年十月癸未,犯氐。

嘉靖元年四月戊寅,犯牛。十一月丙寅,犯羅堰。二年十一月壬辰,犯壘壁。二十年十一月庚寅,二十一年正月丁未,俱犯左執法。二十二年十二月丁亥,犯房北第一星。二十三年四月戊寅,又犯。三十五年五月壬戌,退行又犯。四十五年五月辛卯,退留守左執法。

隆慶元年二月戊午,退守亢。

萬曆三十九年十月己巳,天啟三年九月甲辰,俱犯軒轅。四年正月丙寅,犯軒轅大星。五年正月庚戌朔,退行犯左執法。七年三月乙酉,退行犯房北第一星。

崇禎七年閏八月丁未,犯積尸。九年冬,犯右執法。

△熒惑

洪武元年八月甲午,犯太微西垣上將。九月戊申,犯右執法。二年正月乙卯,犯房。六月壬辰,犯東咸。三年九月丙申,入太微垣。乙卯,留太微垣。四年九月乙卯,犯壘壁。五年十一月庚午,犯鉤鈐。九年三月辛酉,犯井。四月戊申,犯鬼。十年八月丙寅,犯天樽。十月乙卯,犯鬼。十一年二月壬戌,犯五諸侯。三月甲午,犯積尸。六月壬戌,犯右執法。十二年八月乙亥,犯鬼。戊寅,犯積尸。十二月庚寅,犯軒轅大星。十四年十月丙子,犯太微垣。十五年三月乙亥,犯右執法。九月乙丑,犯南斗。十六年八月辛卯,行軒轅中。九月辛酉,犯太微西垣上將。十七年正月乙卯,入氐。三月戊午,犯氐。十八年正月戊辰,犯外屏。十月丁酉,犯進賢。十九年正月壬戌,犯罰,二月丁未,犯箕。四月己亥,留斗。七月辛巳,犯斗。八月丁亥,犯斗。十月辛亥,十一月己巳,犯壘壁。二十一年正月丙申,入斗。四月丁未,七月庚辰,俱犯壘壁。十一月癸巳,犯外屏。二十二年正月丙戌,犯天陰。二月癸卯,行昴中。十月庚申,入氐。十一月甲午,犯東咸。十二月癸丑,犯天江。二十三年正月甲戌,入斗。三月辛卯,犯壘壁。五月戊戌,犯外屏。二十四年十二月甲子,與辰星同犯箕。二十五年二月己卯,犯壘壁。九月己卯朔,入井。二十六年三月庚戌,犯積薪。五月丙辰,犯軒轅。六月己丑,犯右執法。二十七年六月辛未,犯天街。八月癸巳,犯積薪。九月乙巳,犯鬼。二十八年二月壬午,又犯。四月戊子,入軒轅。五月戊午,犯靈臺。閏九月乙丑,犯東咸。二十九年五月丙寅,犯諸王。六月甲午,犯司怪。十月辛亥,犯上將。十二月癸卯,守太微垣。三十年三月壬午,入太微垣。五月戊午,犯右執法。八月丁亥,入氐。丁未,入房。十月癸未,犯斗杓。三十一年十月,守心。

建文四年八月戊辰,犯上將。甲戌,入太微垣右掖門。九月辛巳朔,犯右執法。壬辰,犯左執法。十月甲寅,犯進賢。甲子,入角。十一月壬午,入亢。己亥,入氐。

永樂元年五月癸未,犯壘壁西第四星。十月甲戌,犯東第五星。二年四月乙酉,犯天樽。九月乙卯,犯角。十一月壬子,犯鉤鈐。三年三月癸丑,犯壘壁。四年正月甲午,犯天陰。戊午,犯月星。五年七月癸酉,犯諸王。八月己酉,犯司怪南第二星。六年二月庚辰朔,犯北第二星。四月辛卯,犯鬼。七月辛亥,入太微垣右掖門。丙辰及八年六月丙午,十年五月壬辰,俱犯右執法。十一年十月戊午,犯上將。十二年二月癸酉,退入太微垣,犯上相。十三年九月丁酉,犯靈臺上星。癸卯,犯上將。十月庚午,犯左執法。十二月甲午朔,犯進賢。十五年九月庚申,犯左執法。十二月甲午,入房北第一星。十六年九月壬申,犯壘壁。十七年十二月庚辰,犯鉤鈐。二十年十月壬子,退犯天街上星。二十一年三月庚戌,犯積薪。二十二年十一月辛卯,退犯五諸侯。

洪熙元年正月庚辰,留井。四月癸卯,入鬼。

宣德元年十二月戊寅,犯軒轅。三年六月甲戌,犯積尸。十月戊子,犯太微西垣上將。四年三月癸亥,犯靈臺。戊辰,犯上將。四月丙申、戊戌,俱犯右執法。九月丙辰,犯天江。五年九月乙丑,犯靈臺。十月癸酉,犯上將。十一月己亥,犯左執法。丙午,犯進賢。六年三月乙卯,犯亢。六月甲寅、乙卯,俱犯氐。七月甲戌,犯房。九月癸亥,犯斗杓。七年九月辛酉,犯上將。十月己酉,犯進賢。八年正月丁卯,犯房。庚辰,犯東咸。八月丙午,犯斗魁。十月甲戌,犯壘壁。九年十一月己卯,犯氐。十二月己酉,犯鉤鈐。十年三月丁亥,犯壘壁。

正統元年二月乙丑,犯天街。十二月甲子,犯天江。二年四月乙亥,犯壘壁。三年三月甲辰,犯井。五月庚寅,犯積尸。四年閏二月己卯朔,犯壘壁。五年二月庚辰,三月辛未,俱犯井。七年五月己丑,犯右執法。八年八月辛丑,犯積尸。九年五月癸酉,犯左執法。十年十月辛丑,犯上將。十一年二月乙卯,三月丁酉,俱犯平道。七月丁亥,犯氐。九月辛未,犯天江。十三年正月丙午,犯房北第一星。二月戊午,犯罰。九月甲午,犯狗。十四年七月己卯朔,留守斗。九月壬寅,犯左執法。十月乙丑,犯進賢。十一月乙未,犯亢。十二月丁未朔,犯氐。丙子,犯房。

景泰元年九月丁未,犯壘壁西第三星。辛亥,犯第四星。庚申,犯第六星。十月辛未朔又犯。十二月己丑,犯第五星。二年十一月丙申,犯氐。癸亥,犯鉤鈐。三年四月甲申,與歲星同犯危。四年正月庚午,犯昴。五年六月戊戌,犯諸王。六年三月丙辰,犯井。五月乙巳朔,犯積尸。七年七月丁酉,入井。十月壬寅,犯鬼。

天順元年二月癸未,又犯。二年八月戊辰,入鬼。三年正月辛卯,犯軒轅。四月乙卯,犯靈臺。五月癸卯,犯右執法。四年七月戊子,犯天樽。八月丙辰,入鬼。十月庚午,犯上將。閏十一月庚申,犯上相。五年正月戊午,退入太微垣。三月癸亥,犯右執法。六年七月丙午,入鬼。九月乙卯,犯上將。十一月丙午,犯進賢。七年正月辛亥,入氐。四月辛酉,退犯氐西南星。七月壬辰,犯東南星。甲寅,犯房北第二星。八月己巳,犯斗杓。

成化元年正月丁巳,犯東咸。二月癸卯,犯天籥。五月戊午,留守斗。己巳,退犯魁第四星。七月癸酉,又犯。二年二月癸巳,犯天陰。三年八月乙未,犯壘壁。四年二月己亥,犯月星。己酉,犯天街。五月庚辰,犯鬼。癸未,犯積尸。十一年七月甲戌,犯積薪。八月癸未,入鬼。甲申,犯積尸。十月乙未,犯靈臺。十二年四月壬辰,犯上將及建。十三年九月癸未,犯上將。十一月庚辰,犯進賢。十四年正月乙丑,犯亢。二月甲辰,又犯。十五年九月乙丑,犯靈臺。閏十月庚申,犯進賢。十六年正月壬午朔,犯房。三月乙酉,犯天江。十月戊辰,犯壘壁。十七年三月庚辰,犯昴。十八年王月甲戌,八月丙辰,十月戊辰,俱犯壘壁。十九年十月庚辰,犯氐。十一月己酉,犯鉤鈐。壬子,犯東咸。二十一年正月戊子,犯天陰。十一月壬戌,犯天江。二十三年二月丁酉,犯井。

弘治元年六月庚戌,犯諸王。八月庚申,犯積薪。九月癸酉,犯鬼。甲戌,犯積尸。三年三月辛酉,犯鬼。四年六月戊子,犯諸王。五年六月己亥,犯積尸。七月癸酉,入井。十月乙巳,犯靈臺。十一月丙申,犯上相。六年二月庚子,犯平道。三月甲戌,犯上相。四月丙申,犯左執法。七年十二月癸亥,犯亢。八年二月戊寅,犯房。四月癸酉、六月癸亥,俱犯氐。十二月癸丑,犯壘壁。九年十二月己丑,犯鉤鈐。十一年十一月乙未,犯亢。十三年正月壬戌,犯天陰。十四年四月庚子,犯壘壁。十月乙卯,犯天街。十五年二月戊辰,犯井。十六年七月丁丑,犯諸王。十七年四月癸卯,十八年九月癸未,正德二年七月戊辰,俱犯積尸。十月癸未,犯上將。三年四月乙丑,犯右執法。四年十一月己未,犯進賢。五年三月癸亥,犯亢。六月丁卯,犯房北第二星。七月丙子,犯天關。八月乙未,犯天江。十六年二月庚子,犯鬼。六月壬午,犯右執法。

嘉靖元年八月乙未,犯積尸。二年正月庚戌,入太微垣,犯內屏。閏四月丙寅,犯右執法。三年十月癸巳,犯上將。十一月甲子,犯左執法。十二月癸丑,犯進賢。四年二月戊午,犯平道。五年九月癸未,犯上將。十八年十一月辛未,犯上相。十九年九月乙卯,二十一年八月戊戌,俱犯斗。二十三年正月壬寅,犯房北第一星。三月丁巳,入斗。六月乙亥,入箕退行二舍。二十四年十月丁巳,犯氐。二十七年十一月甲申,自畢退行至胃。二十九年十二月甲戌,退守井。三十一年九月辛卯,犯鬼。三十五年九月丁丑,犯上將。三十六年二月壬辰,自角退入軫。四月戊子,自軫退行二舍餘。三十九年十二月甲寅,犯鉤鈐。四十二年十月辛亥,自胃退行抵婁。四十四年十二月壬申,自井退二舍。

隆慶二年六月乙未,犯右執法。三年八月丁未,犯鬼。四年五月己卯,犯右執法。

萬歷二年二月癸亥,犯房。五月己卯,犯氐。五年十月辛丑,又犯。九年二月辛酉,犯井。十二年十二月辛亥,退行張次。十三年正月庚辰,退入軒轅。二月戊申,犯張,又自張歷柳。十五年正月丁酉,退入軫。二月丁卯,退行翼次。四月,犯翼。十七年二月己丑,犯氐。四月丁亥,自氐退入角。七月辛酉,犯房第二星。九月辛亥,犯斗杓。十九年四月乙巳,六月壬子,俱犯箕。七月丁亥,犯斗。二十年十一月戊辰,犯氐。二十一年七月辛巳,九月甲戌,俱犯室。二十二年五月,犯角。二十七年八月甲辰,犯奎。二十八年二月庚寅,犯鬼。三十年正月丁巳,退入太微垣。三十二年二月丁酉,退入角。三十四年四月己巳,犯心。五月戊寅,犯房。癸未,自心退入氐。三十七年十一月丙戌,犯氐。三十八年八月辛卯,退行婁次。四十二年十月,犯柳。四十四年十二月,犯翼。四十五年二月庚子,退行星度。四十七年正月,犯軫。二月丁巳,退入軫。辛未,退入翼。

泰昌元年八月辛亥,犯太微右將。

天啟元年閏二月癸巳,退入氐。三年正月甲午,犯房北第一星。四月,守斗百日。八月甲子,犯狗國。十月甲申,犯壘壁。四年二月,守斗。五年九月乙卯,自壁退入室。

崇禎三年三月己酉,入井,退舍復秬。居數月,又入鬼,犯積尸。四月己卯,復犯積尸。八月辛亥,犯斗魁。八年九月丁丑,犯太微垣。十一年,自春至夏守尾百餘日。四月己酉,退行尾八度,掩於月。五月丁卯,退尾入心。十五年五月,守心。

△填星

洪武十五年六丁亥,九月乙未,俱犯畢。十六年八月己卯,犯天關。十七年閏十月丙辰,犯井。十八年七月己巳,十九年三月甲戌,俱犯天樽。九月甲寅,入鬼。十月甲午,留鬼。二十二年二月癸卯,退行軒轅。二十三年正月戊子,五月壬子,俱犯靈臺。二十四年十月己未,犯太微東垣上相。二十五年二月辛酉,退犯上相。己卯,退入太微左掖。二十八年正月癸丑,守氐。四月乙丑,退入氐。二十九年十一月甲子,犯罰。三十年正月丙辰,犯東咸。五月壬子朔,又犯罰。

永樂元年九月丁丑,躔女留代。十二年七月戊子,犯井。十四年七月辛亥,犯鬼。十七年九月丙子,犯上將。

洪熙元年十一月辛酉,宣德元年三月庚戌,九月壬辰,俱犯鍵閉。

正統元年八月丁亥,退犯壘壁。三年十一月乙酉,犯外屏。八年十一月庚午,十二月壬子,俱犯井。十年三月丁丑,犯天樽。十三年九月丁亥,犯靈臺。

景泰元年閏正月己酉,入太微垣。九月庚戌,二年二月戊子,俱犯上相。庚寅,退入太微左掖。三年十月辛丑,犯亢。四年三月己未,退犯亢。七年七月己丑,犯罰。

天順三年正月辛卯,犯建。四月癸酉,守犯建。七年閏七月戊午朔,退犯壘壁。十月癸丑,又犯。

成化四年七月甲子,犯天囷。七年閏九月戊午,犯斗魁。辛酉,犯天高。十二年十月辛卯,守軒轅大星。十五年四月己丑,犯上將。十七年二月己未,犯進賢。二十一年正月庚戌,犯罰。

弘治六年三月壬申,八年十二月戊午,十年九月乙丑,俱犯壘壁。十四年十一月辛卯,犯諸王。十五年六月壬子,十二月辛丑,十六年正月己卯,俱犯井。七月辛卯,犯天樽。十七年七月辛亥,犯積尸。九月甲午,犯鬼。

正德二年八月癸巳,犯靈臺。十月甲戌,犯上將。三年五月甲子,犯靈臺。五年二月戊申,六月壬辰,俱犯上相。七年四月甲申,犯亢。十五年二月丁卯,犯羅堰。十六年七月乙卯,退犯代。

嘉靖元年八月庚辰,退犯壘壁。二十二年五月甲子,退守氐三十七日。

隆慶三年三月庚午,退犯上相。

萬曆三十五年正月至六月,退留斗。四十八年八月癸丑,犯井。

天啟元年正月丙戌,退入井。二年八月壬辰,犯守鬼。五年十月丙戌,犯上將。

△太白

洪武元年七月己巳朔,犯井。三年十一月甲寅,犯壘壁。九年六月丁亥,犯畢。庚戌,犯井。八月,犯上將。九月己未,犯右執法。十年十月壬子,犯進賢。十一年九月丁丑,犯氐。十二月辛丑,犯壘壁。十二年三月壬子,犯昴。六月丁亥,犯井。七月乙巳,犯鬼。十三年八月丙戌,犯心。十六年十一月乙卯,犯壘壁。十七年七月癸卯,犯天樽。十二月丙申,犯壘壁。十八年十月壬子,犯亢。十九年正月庚午,犯牛。二月己丑,犯壘壁。七月己卯,二十年八月己巳,俱入太微垣。二十一年六月壬戌,犯左執法。二十二年正月己卯,犯建。五月癸巳,犯諸王。十一月辛未,入斗。十二月丁巳,犯壘壁。二十三年四月壬戌,犯五諸侯。六月丁丑,留井。十月庚午,入亢。二十四年七月庚戌,入太微垣右掖。辛卯,犯右執法。十月丙辰,入斗。二十五年閏十一月乙酉,入壘壁。二十六年二月癸卯,犯天街。三月丙子朔,犯諸王。二十八年六月癸酉,犯畢。七月丙午,犯井。己酉,出井,犯東第三星。閏九月壬申,入角。十月戊申,犯東咸。二十九年七月戊辰,入角。八月癸丑,犯心中星。三十年正月壬戌,犯建。十二月戊戌,入壘壁。三十一年正月乙亥,犯外屏。五月丁未朔,犯五諸侯。

建文四年六月庚子,入太微右掖。八月甲子,入角。九月癸未,入氐。丙申,入房。十月癸亥,入斗杓。

永樂元年六月丙辰,犯畢。七月甲申,入井。八月己酉,犯鬼。九月丙子朔,犯軒轅左角。十月辛未,入氐。十一月丙戌,犯鍵閉。二年五月辛丑朔,犯鬼。七月己酉,入角。八月丁亥,入房南第二星。十一月丁巳,犯東咸。三年三月丙申朔,犯壘壁東第五星。十二月己巳,犯西第三星。四年二月癸未,犯天陰。五月庚寅朔,犯五諸侯。七月庚戌,犯井。八月丙申,犯御女。九月戊寅,犯進賢。十月乙卯,犯房北第一星。五年七月癸丑,犯右執法。八月己亥,犯氐。九月癸丑,犯東咸。十月癸未,犯斗魁。十一月辛未朔,犯秦。六年六月甲申,犯諸王。丙申,與歲星同犯井。七月戊申,犯天樽。七年二月丙戌,犯外屏。十一月丁亥,犯罰。八年九月壬辰,犯天江。十二年五月癸酉朔,犯五諸侯。閏九月己酉,犯左執法。十三年八月庚寅,犯房北第二星。十月乙丑朔,犯斗魁。十四年六月丁卯,犯諸王。十六年十一月甲子,犯壘壁。十七年七月戊午,犯天樽。八月癸巳,犯軒轅大星。十八年八月乙丑,犯心後星。十九年十月癸卯,犯天江。十二月丁酉,犯壘壁。

洪熙元年三月乙酉,犯昴。四月丙辰,犯井。十月辛未,犯平道。辛巳,犯亢。

宣德元年十月戊辰,犯斗杓。十一月己巳,犯壘壁。丙辰,又犯。二年正月丙申,犯外屏。七月癸巳,犯東井。八月丙辰朔,犯鬼。丁巳,又犯。乙亥,犯軒轅大星。九月丁巳,犯右執法。三年十一月甲子,犯罰。五年二月丁酉,犯昴。九月丁未,犯軒轅左角。十一月壬戌,犯鍵閉。六年九月丙戌,犯斗。七年七月乙酉,犯軒轅。八年十月癸亥,犯亢。十一月辛卯,犯罰。九年十一月壬辰,犯壘壁。十年正月甲戌,犯外屏。六月庚申,犯天關。八月丙辰,犯軒轅。九月壬申,犯上將。

正統三年九月己丑,十一年九月辛未,俱犯軒轅左角。己丑,犯右執法。十月乙未朔,犯左執法。丙午,犯進賢。十二年六月乙亥,犯上將。七月癸丑,犯譏。十四年正月丁亥,犯壘壁。四月庚申,犯井。五月丁亥,犯鬼。七月癸犯,犯亢。九月庚辰,犯天江。十一月丁亥,犯亢。

景泰元年正月丁亥,犯亢。閏正月庚申,入壘壁。八月甲申,犯亢。九月乙巳,犯鉤鈐。壬戌,犯天江。十一月辛酉,犯壘壁。二年六月戊辰朔,犯畢。八月壬寅,入太微右掖。三年四月丁卯,犯諸王。戊子,犯井。五月壬子,犯鬼。六月乙酉,犯靈臺。戊子,犯上將。庚寅,入太微右掖。七月壬寅,犯左執法。五年九月癸丑,掩犯軒轅左角。甲戌,犯左執法。六年六月辛巳,犯井。己丑,與熒惑同入太微右掖。八月戊午,犯房北第二星。九月甲午,犯斗魁。七年七月辛未,犯鬼。

天順元年十二月甲午,犯鍵閉。丁酉,犯罰。二年正月丁卯,犯建。七月丙申,行太微垣中。九月甲寅,犯斗杓。三年五月庚戌,犯畢。十月甲寅,犯亢。四年七月丁丑,犯右執法。甲申,犯左執法。六年九月乙未,犯軒轅左角。己未,犯左執法。十月己巳,犯進賢。七年九月丁丑,犯斗魁。乙酉,犯狗。八年二月丙午,與歲星同犯壘壁。

成化元年十二月丙午,犯鍵閉。二年正月乙卯,犯斗。三年二月丁未,犯婁。三月戊子,犯外屏。五月壬辰,犯畢。六月壬戌,犯井。七月甲申,入鬼,犯積尸。八月癸卯,入軒轅。四年六月戊申,犯靈臺。五年二月癸巳,犯牛。六年九月丙子朔,犯軒轅左角。甲午、庚子,俱犯左執法。七年九月壬午,犯房北第二星。閏九月戊午,犯斗魁。十二月乙未,犯牛及羅堰。八年二月甲申,犯壘壁。六月庚午,入井。十二月丙戌,犯壘壁。九年四月己卯,犯五諸侯。十月甲子,犯左執法。十一年三月甲戌,犯外屏。七月庚戌,犯天樽。八月丁酉,犯靈臺。庚子,犯上將。九月癸丑,犯左執法。十二年三月庚午,犯月星。四月甲午,犯井。十三年十二月甲午朔,犯壘壁。十五年九月庚辰,犯天江。十月庚子,犯斗魁。辛亥,犯狗。十七年二月丁卯,犯天陰。五月丁酉,犯軒轅。十九年八月丙寅,又犯。九月甲午,犯左執法。十月庚辰,犯房。二十年六月壬午,犯左執法。十二月庚辰,犯壘壁。二十二年六月庚子,犯井。八月甲午,犯軒轅。十一月乙亥,犯進賢。十二月庚戌,犯房。二十三年八月甲申,犯亢。

弘治元年二月癸丑,犯壘壁。六月庚戌,犯鬼。七月丙子,犯軒轅大星。癸未,犯左角。戊子,犯靈臺。二年正月庚辰,犯外屏。二月丁未,犯壘壁。十月己丑,犯左執法。三年正月壬申,犯羅堰。十一月戊戌,犯壘壁。四年六月癸丑,犯天關。六年二月庚子,犯羅堰。甲子,犯壘壁西第六星。三月甲申,犯東第四星。七年二月辛未,犯昴。七月壬子,犯鬼。八月辛巳,犯軒轅左角。九月丁亥,犯靈臺。壬寅,犯亢。十一月壬辰,犯房。乙未,犯罰。丙午,犯天江。九年二月戊午,犯羅堰。七月己未,犯軒轅大星。十年十月辛未,犯左執法。十二月戊辰朔,犯東咸。十一年十月辛未,犯天江。十二年七月辛未,犯鬼。九月戊午朔,犯左執法。十三年十一月乙未,犯罰。十四年正月辛酉,犯建。二月壬午,犯羅堰。十一月己亥,犯壘壁。十五年二月甲寅,犯昴。五月己丑,犯天高。十一月癸酉,犯牛。十六年三月辛卯,犯諸王。九月甲申,犯天江。十月丁未,犯斗魁。丁丑,犯狗。十一月辛巳,犯羅堰。十七年五月己亥,犯諸王。七月丙辰,犯上將。十八年九月丙午,犯右執法。

正德元年春,守軒轅。十二月癸丑,犯壘壁。二年三月壬申,犯外屏。五月己巳,犯天高。九月辛丑朔,犯進賢。三年十月丙戌,犯亢。四年正月己酉,犯建。五年八月己亥,犯軒轅大星。十月丙申,犯亢。六年七月辛酉,犯左執法。十月丁亥,犯斗。十一月癸亥,犯羅堰。七年閏五月丁酉,犯鉞。六月甲子,犯積尸。八年正月丙戌,犯外屏。七月丁亥,犯酒旗。八月戊申,犯軒轅右角。十年八月丁卯,犯上將。丁丑,犯左執法。十三年七月戊戌,犯井。己未,犯鬼。十四年十月戊辰,犯斗。癸未,犯狗。十六年四月癸卯,犯鬼。八月己丑,犯軒轅右角。九月乙亥,犯左執法。十月戊子,犯進賢。十一月丁卯,犯鍵閉。十二月庚子,犯建。

嘉靖元年正月丙辰,犯牛。十月戊子,犯斗杓。二年六月癸丑,犯井。七月丙子,犯鬼。八月辛酉,犯左執法。四年正月丁卯,犯建。五年六月庚辰,犯井。六年六月丁卯,犯靈臺。八年二月庚寅,犯天街。

隆慶元年十月甲申,入斗。

萬歷二十四年四月戊午,犯井。三十四年二月甲子,犯昴。四十六年四月乙卯,犯御女。

泰昌元年八月丙午朔,犯太微垣勾己。

天啟三年九月,犯心中星。五年九月壬申,犯左執法。甲申,犯御女。

△辰星

洪武十一年十二月庚戌,犯斗。十五年四月丁亥,犯東井。十八年八月丁酉,入太微垣。二十一年十月壬子,入氐。二十二年十月癸卯,犯氐。二十五年八月庚午,犯上將。二十七年七月辛丑,犯鬼。十一月庚子,犯鍵閉。二十八年正月丁酉,犯壘壁。五月甲辰,犯天樽。三十年十二月甲辰,犯建。

建文四年六月庚午,犯積薪。

永樂二年四月丁酉,犯畢。癸卯,犯諸王。五月丁卯,犯軒轅大星。十月己丑,犯斗杓。三年六月己卯,犯軒轅大星。六年正月庚戌朔,犯壘壁。二月癸巳,又犯。十六年六月戊子,犯軒轅大星。

宣德元年五月丁未,犯鬼。二年十一月丙戌,犯氐。五年閏十二月丁酉,犯建。戊戌,又犯。七年五月辛巳,犯積尸。

正統十三年十月丙辰,犯亢。

景泰四年五月己未,犯積薪。

成化十二年三月壬戌,犯昴。

弘治五年十一月庚辰,犯罰。十二年六月壬子,犯鬼。十月壬子,犯房北第一星。十七年七月丙辰,犯靈臺。十八年五月庚子,犯鬼。十一月戊子,犯鍵閉。

正德七年六月丙寅,犯鬼。

嘉靖元年正月戊午,犯羅堰。二年八月壬寅,犯上將。

天啟七年三月辛未,退犯房。

按兩星經緯同度曰掩,光相接曰犯,亦曰凌。緯星出入黃道之內外,凡恒星之近黃道者,皆其必由之道,凌犯皆由於此。而行遲則凌犯少,行速則多,數可預定,非如彗孛飛流之無常。然則天象之示炯戒者,應在彼而不在此。歷代史志凌犯多繫以事應,非附會即偶中爾。茲取緯星之掩犯恒星者次列之。比事以觀,其有驗者,十無一二,後之人可以觀矣。至於月道與緯星相似,而行甚速,其出入黃道也,二十七日而周,計其掩犯恆星殆無虛日,豈皆有休咎可占,今見於《實錄》者不及百分之一,然已不可勝書,故不書。

▲星晝見

恒星洪武十九年七月癸亥,二十年五月丁丑,七月壬寅,二十一年十二月丁卯,俱三辰晝見。弘治十八年九月甲午申刻,河鼓、北斗見。庚子,星晝見。正德元年二月癸酉,星斗晝見。天啟二年五月壬寅,有星隨日晝見。崇禎十六年十二月辛酉朔,星晝見。

歲星景泰二年九月甲辰,晝見。三年六月壬戌,四年五月丁丑,六月甲辰,五年七月庚戌、壬子、癸亥,六年七月丁酉,天順元年五月丙子,五年七月乙卯,六年八月庚午,七年三月乙巳,成化十四年六月庚子,八月丁酉,十六年七月丙申,十八年九月癸亥,二十年八月壬申,弘治元年六月甲寅,二年五月癸亥,六月甲午,五年十月己酉,六年九月癸卯,七年十一月癸卯,九年二月辛亥至甲寅,四月壬午,十年正月甲寅至丙辰,十一年八月甲申,十三年四月庚子至乙巳,十四年六月壬辰至乙未,並如之。十五年六月,連日晝見。十六年七月辛卯,十七年七月壬子,十八年五月乙未,八月辛巳至九月癸未,正德元年十一月乙酉,二年十一月辛酉至丁卯,六年三月壬寅至四月壬申,九年八月乙巳至甲寅,十二年十月甲子至乙巳,並如之。嘉靖二年三月辛未,二十九年八月戊寅,晝見守井。崇禎十一年四月壬子,晝見。

熒惑景泰三年八月甲子,晝見於未位。

太白洪武四年二月戊午,晝見。四月戊申,六月壬午朔,五年六月甲申至丁亥,十二月甲申,八年八月丁巳,九年二月乙巳至己酉,三月壬申,十二年閏五月戊戌,十三年七月甲午,十五年四月丁亥,七月戊申、辛酉,九月丁未朔,十六年十月壬辰至乙未,十八年四月己亥至辛丑,六月丙申至辛丑、辛亥,並如之。九月戊寅,經天與熒惑同度。乙酉,晝見。丁亥,又見,犯熒惑。十月癸巳至丙申,晝見。戊戌至辛丑,十九年十月甲申朔至庚寅,並如之。二十年六月戊戌,經天。七月壬寅至甲辰,晝見。二十一年四月己巳,七月丙申,二十三年三月丁亥,二十四年八月辛巳,二十五年二月辛酉,二十六年四月甲辰,並如之。八月庚子,與太陰同晝見。建文四年七月庚子,經天。永樂元年五月癸未、癸卯,俱與太陰同晝見。六月壬申,與太陰晝見。四年七月壬寅,晝見。五年八月丙申,六年二月甲辰,八年十月庚戌,十二年九月癸未,十五年七月己酉,八月庚戌,洪熙元年六月戊戌,七月乙巳,八月癸巳,宣德六年十月乙巳,八年九月戊戌至甲寅,九年十二月甲子,十年七月丁亥,正統四年七月壬子,十月丙申,六年五月庚戌,並如之。十一年七月甲申,經天。十三年二月辛酉,晝見。十四年正月辛亥,八月丙子,景泰元年十月乙酉,二年五月庚子、辛亥,並如之。壬子,經天。三年五月丁巳,晝見。十一月壬戌,五年正月甲戌,二月丙戌,六月癸卯,七年正月戊戌,天順元年四月甲午,八月壬子,二年十月己未,三年四月癸亥、癸酉,四年十一月庚寅,十二月丙戌,五年正月丁未,十二月癸巳,六年六月己丑,八月庚午,七年閏七月辛酉、癸未,八年正月庚申,成化元年二月癸未,三年四月癸丑,四年六月丙申,六年六月丙戌,七年八月癸卯,並如之。八年正月乙卯,經天,與日爭明。十一年五月己未,晝見。十二年十月丙戌,十三年十二月甲午,並如之。十四年六月庚子,與歲星俱晝見。八月甲午,晝見。十五年十二月丙子,十七年三月癸未,八月癸亥,十八年九月庚戌,十九年四月癸亥朔,並如之。二十年八月壬申,與歲星俱晝見。二十一年十一月丙辰,晝見。二十二年六月己丑,二十三年九月丙午,弘治元年五月庚午,二年正月壬戌,三月庚申,五月丙戌,八月癸巳、庚子,四年四月辛未,五年五月乙亥,十月辛酉,六年十二月乙丑,七年五月庚戌,八年七月戊子,九年二月己酉朔,十年正月甲子至丁卯,並如之。六月丙子未刻,經天。八月癸未及十一年十月辛巳,晝見。十二年三月戊辰至壬申,八月庚寅,並如之。十三年四月庚子至乙巳,與歲星同晝見。十月丁未、己酉,十四年十二月庚戌,十五年五月庚寅至癸巳,十六年七月壬辰,十七年二月戊戌及六月癸亥,十八年二月壬戌,並晝見。五月辛亥,經天。八月癸亥至戊辰,晝見。正德元年十月己未,如之。二年正月庚辰,經天。三月戊辰,晝見。三年五月乙巳至丁未,十月己卯、庚辰,四年十月戊戌至乙巳,五年五月丙子,六年七月壬申至八月癸未,八年正月丙戌至己丑,四月壬戌、癸亥,八月庚戌至乙卯,九年十一月甲申至十二月壬辰,十一年六月甲寅至己未,十四年八月丙寅至庚辰,十五年正月己未至二月辛酉,十六年八月丁亥,嘉靖元年九月辛未,並如之。二年三月辛未,與歲星俱晝見。三年四月庚戌,晝見。五年五月庚子,十一年四月癸巳,十月辛巳、戊子,十一月甲寅,十三年閏二月庚申,並如之。五月癸巳,與月同晝見。十七年九月辛卯,晝見。十八年四月癸亥,十一月壬寅,二十年十一月乙巳至丁未,二十二年七月丙午,二十三年二月辛巳,二十四年閏正月戊寅,二十五年十月辛卯,二十六年四月丙申,二十七年四月丁巳,十一月丙戌至乙未,二十八年十一月乙酉至己丑,二十九年六月戊申、甲寅,三十年六月丙子至辛巳,三十一年正月丙戌至丙申,三十二年二月辛未至甲戌,七月戊辰至辛未,三十五年五月壬午,十月癸卯至丙午,三十六年十二月庚辰朔,三十八年七月癸酉,三十九年正月庚寅至壬辰,並如之。四十年三月丙子,晝見,歷二十四日。八月辛未,晝見。四十一年九月乙未,四十二年四月己巳至壬申,四十三年五月甲寅,並如之。十月戊子,晝見,歷二十二日。四十五年正月己亥,晝見。隆慶元年七月辛酉,二年正月甲寅,並如之。三年三月甲子,晝見,歷二十二日,四年十一月乙丑至丁卯,晝見。萬曆十一年七月辛丑,十二年七月癸巳,十六年九月丁丑,二十一年八月甲午,二十四年十月丙寅,並如之。二十七年九月辛卯,經天。三十七年三月辛丑,晝見。三十八年十月辛巳,四十年五月壬寅,天啟二年二月丙戌,三年三月丁巳,十二月乙丑,五年四月癸未,並如之。七月癸酉,經天。崇禎元年七月壬戌,晝見。三年四月己卯,十二月丙辰,並如之。

▲客星

《史記·天官書》有客星之名,而不詳其形狀。敘國皇、昭明諸異星甚悉,而無瑞星、妖星之名。然則客星者,言其非常有之星,殆諸異星之總名,而非有專屬也。李淳風志晉、隋天文,始分景星、含譽之屬為瑞星,彗、孛、國皇之類為妖星,又以周伯老子等為客星,自謂本之漢末劉睿《荊州占》。夫含譽,所謂瑞星也,而光芒則似彗;國皇,所謂妖星也,而形色又類南極老人。瑞與妖果有定哉?且周伯一星也,既屬之瑞星,而云其國大昌。又屬之客星,而云其國兵起有喪。其說如此,果可為法乎?馬遷不復區別,良有以也。今按《實錄》,彗、孛變見特甚,皆別書。老人星則江以南常見,而燕京必無見理,故不書。餘悉屬客星而編次之。

洪武三年七月,太史奏文星見。九年六月戊子,有星大如彈丸,白色。止天倉,經外屏、卷舌,入紫微垣,掃文昌,指內廚,入於張。七月乙亥滅。十一年九月甲戌,有星見於五車東北,發芒丈餘。掃內階,入紫微宮,掃北極五星,犯東垣少宰,入天市垣,犯天市。至十月己未,陰雲不見。十八年九月戊寅,有星見太微垣,犯右執法,出端門。乙酉,入翼,彗長丈餘。至十月庚寅,犯軍門,彗掃天廟。二十一年二月丙寅,有星出東壁,占曰「文士效用」。帝大喜,以為將策進士兆也。

永樂二年十月庚辰,輦道東南有星如盞,黃色,光潤而不行。二十二年九月戊戌,有星見斗宿,大如碗,色黃白,光燭地,有聲,如撒沙石。

宣德五年八月庚寅,有星見南河旁,如彈丸大,色青黑,凡二十六日滅。十月丙申,蓬星見外屏南,東南行,經天倉、天庾,八日而滅。十二月丁亥,有星如彈丸,見九斿旁,黃白光潤,旬有五日而隱。六年三月壬午,又見。八年閏八月戊午,景星三,見西北方天門,青赤黃各一,大如碗,明朗清潤,良久聚半月形。丁丑,有黃赤色見東南方,如星非星,如雲非雲,蓋歸邪星也。

景泰三年十一月癸未,有星見鬼宿積尸氣旁,徐徐西行。

天順二年十一月癸卯,有星見於星宿,色白,西行,至丙午,其體微,狀如粉絮,在軒轅旁。庚戌,生芒五寸,犯爟位西北星,至十二月壬戌,沒於東井。五年六月壬辰,天市垣宗正旁,有星粉白,至乙未,化為白氣而消。六年六月丙寅,有星見策星旁,色蒼白,入紫微垣,犯天牢,至癸未,居中台下,形漸微。

弘治三年十二月丁巳,有星見天市垣,東南行。戊辰,見天倉下,漸向壁。七年十二月丙寅,有星見天江旁,徐行近斗,至八年正月庚戌,入危。十二年七月戊辰,有星見天市垣宗星旁,入紫微垣東籓,經少宰、尚書,抵太子后宮,出西籓少輔旁,至八月己丑滅。十五年十月戊辰,有星見天廟旁,自張抵翼,復退至張,戊寅滅。

正德十六年正月甲寅朔,東南有星如火,變白,長可六七尺,橫亙東西,復變勾屈狀,良久乃散。

嘉靖八年正月立春日,長星亙天。七月又如之。十一年二月壬午,有星見東南,色蒼白,有芒,積十九日滅。十三年五月丁卯朔,有星見螣蛇,歷天廄入閣道,二十四日滅。十五年三月戊午,有星見天棓旁,東行歷天廚,西入天漢,至四月壬辰沒。二十四年十一月壬午,有星出天棓,入箕,轉東北行,逾月沒。

萬歷六年正月戊辰,有大星如日,出自西方,眾星皆西環。十二年六月己酉,有星出房。三十二年九月乙丑,尾分有星如彈丸,色赤黃,見西南方,至十月而隱。十二月辛酉,轉出東南方,仍尾分。明年二月漸暗,八月丁卯始滅。三十七年,有大星見西南,芒刺四射。四十六年九月乙卯,東南有白氣一道,闊尺餘,長二丈餘,東至軫,西入翼,十九日而滅。十一月丙寅,旦有花白星見東方。天啟元年四月癸酉,赤星見於東方。

崇禎九年冬,天狗見豫分。

▲彗孛

彗之光芒傅日而生,故夕見者必東指,晨見者必西指。孛亦彗類,其芒氣四出,天文家言其災更甚於彗。

洪武元年正月庚寅,彗星見於昴、畢。三月辛卯,彗星出昴北大陵、天船間,長八尺餘,指文昌,近五車,四月己酉,沒於五車北。六年四月,彗星三入紫微垣。二十四年四月丙子,彗星二,一入紫微垣閶闔門,犯天床;一犯六甲,掃五帝內座。

永樂五年十一月丙寅,彗星見。

宣德六年四月戊戌,有星孛於東井,長五尺餘。七年正月壬戌,彗星出東方,長丈餘,尾掃天津,東南行,十月始滅。是月戊子,又出西方,十有七日而滅。八年閏八月壬子,彗星出天倉旁,長丈許。己巳,入貫索,掃七公。己卯,復入天市垣,掃晉星,二十有四日而滅。

正統四年閏二月己丑,彗星見張宿旁,大如彈。丁酉,長五丈餘,西行,掃酒旗,迤北,犯鬼宿。六月戊寅,彗星見畢宿旁,長丈餘,指西南,計五十有五日乃滅。九年七月庚午,彗星見太微東垣,長丈許,累日漸長,至閏七月己卯,入角沒。十四年十二月壬子,彗星見天市垣市樓旁,歷尾度,長二尺餘,至乙亥沒。

景泰元年正月壬午,彗星出天市垣外,掃天紀星。三年三月甲午朔,有星孛於畢。七年四月壬戌,彗星東北見於胃,長二尺,指西南。五月癸酉,漸長丈餘。戊子,西北見於柳,長九尺餘,掃犯軒轅星。甲午,見於張,長七尺餘,掃太微北,西南行。六月壬寅,入太微垣,長尺餘。十二月甲寅,彗星復見於畢,長五寸,東南行,漸長,至癸亥而沒。

天順元年五月丙戌,彗星見於危,若動搖者,東行一度,芒長五寸,指西南。六月癸巳朔,見室,長丈餘,由尾至東壁,犯天大將軍、卷舌第三星,井宿水位南第二星。十月己亥,彗星見於角,長五寸餘,指北,犯角北星及平道東星。五年六月戊戌,彗見東方,指西南,入井度。七月丙寅始滅。

成化元年二月,彗星見。三月,又見西北,長三丈餘,三閱月而沒。四年九月己未,有星見星五度,東北行,越五日,芒長三丈餘,尾指西南,變為彗星。其後晨見東方,昏見室,南犯三公、北斗、瑤光、七公,轉入天市垣。出垣漸小,犯天屏西第一星。十一月庚辰,始滅。七年十二月甲戌,彗星見天田,西指,尋北行,犯右攝提,掃太微垣上將及幸臣、太子、從官,尾指正西,橫掃太微垣郎位。己卯,光芒長大,東西竟天。北行二十八度餘,犯天槍,掃北斗、三公、太陽,入紫微垣內,正晝猶見。自帝星、北斗、魁、庶子、后宮、勾陳、天樞、三師、天牢、中台、天皇大帝、上衛、閣道、文昌、上台,無所不犯。乙酉,南行犯婁、天河、天陰、外屏、天囷。八年正月丙午,行奎宿外屏,漸微,久之始滅。

弘治三年十一月戊戌,彗星見天津南,尾指東北。犯人星,歷杵臼。十二月戊申朔,入營室。庚申,犯天倉。十三年四月甲午,彗星見壘壁陣上,入室壁間,漸長三尺餘。指離宮,掃造父,過太微垣,漸微。入紫微垣,近女史,犯尚書,六月丁酉沒。

正德元年七月己丑,有星見紫微西籓外,如彈丸,色蒼白。越數日,有微芒見參、井間,漸長二尺,如帚,西北至文昌。庚子,彗星見,有光,流東南,長三尺。越三日,長五尺許,掃下台上星,入太微垣。十五年正月,彗星見。

嘉靖二年六月,有星孛於天市。十年閏六月乙巳,彗星見於東井,長尺餘,掃軒轅第一星。芒漸長,至翼,長七尺餘。東北掃天樽,入太微垣,掃郎位,行角度,東南掃亢北第二星,漸斂,積三十四日而沒。十一年八月己卯,彗星見東井,長尺許。後東北行,歷天津,漸至丈餘。掃太微垣諸星及角宿、天門,至十二月甲戌,凡一百十五日而滅。十二年六月辛巳,彗星見於五車,長五尺餘,掃大陵及天大將軍。漸長丈餘,掃閣道,犯螣蛇,至八月戊戌而滅。十八年四月庚戌,彗星見,長三尺許,光指東南。掃軒轅北第八星,旬日始滅。三十三年五月癸亥,彗星見天權旁,犯文昌,行入近濁,積二十七日而沒。三十五年正月庚辰,彗星見進賢旁,長尺許,西南指,漸至三尺餘。掃太微垣,次相東北,入紫微垣,犯天床,四月二日滅。三十六年九月戊辰,彗星見天市垣列肆旁,東北指,至十月二十三日滅。

隆慶三年十月辛丑朔,彗星見天市垣,東北指,至庚申滅。

萬曆五年十月戊子,彗星見西南,蒼白色,長數丈,氣成白虹。由尾、箕越斗、牛逼女,經月而滅。八年八月庚申,彗星見東南方,每夜漸長,縱橫河漢凡七十日有奇。十年四月丙辰,彗星見西北,形如匹練,尾指五車,歷二十餘日滅。十三年九月戊子,彗星出羽林旁,長尺許。每夕東行,漸小,至十月癸酉滅。十九年三月丙辰,西北有星如彗,長尺餘。歷胃、室、壁,長二尺。閏三月丙寅朔,入婁。二十一年七月乙卯,彗星見東井。乙亥,逆行入紫微垣,犯華蓋。二十四年七月丁丑,彗星見西北,如彈丸。入翼,長尺餘,西北行。三十五年八月辛酉朔,彗星見東井,指西南,漸往西北。壬午,自房歷心滅。四十六年十月乙丑,彗星出於氐,長丈餘,指東南,漸指西北。掃犯太陽守星,入亢度,西北掃北斗、璇璣、文昌、五車,逼紫微垣右,至十一月甲辰滅。四十七年正月杪,彗見東南,長數百尺,光芒下射,末曲而銳,未幾見於東北,又未幾見於西。

崇禎十二年秋,彗星見參分。十三年十月丙戌,彗星見。

▲天變

洪武二十一年八月壬戌至甲子,天鼓鳴,晝夜不止。二十八年三月戊午,昏刻天鳴,如風水相搏,至一鼓止。九月戊戌,初鼓,天鳴如瀉水,自東北而南,至二鼓止。宣德元年八月戊辰,昏刻天鳴,如雨陣迭至,自東南而西南,良久乃息。辛未,東南天鳴,聲如萬鼓。正統十年三月庚寅,西北天鳴,如鳥群飛。正德元年二月壬子,夜東北天鳴,如風水相搏者五七次。隆慶二年八月甲辰,絳州西北天裂,自丑至寅乃合。萬曆十六年九月乙丑,甘肅石灰溝天鳴,雲中如犬狀亂吠,有聲。崇禎元年三月辛巳,昧爽,天赤如血,射窗牖皆紅。十年九月,每晨夕天色赤黃。

▲日變月變

洪武二年十二月甲子,日中有黑子。三年九月戊戌,十月丁巳,十一月甲辰,四年三月戊戌,五月壬子至辛巳,九月戊寅,五年正月庚戌,二月丁未,五月甲子,七月辛未,六年十一月戊戌朔,七年二月庚戌至甲寅,八年二月辛亥,九月癸未,十二月癸丑,十四年二月壬午至乙酉,十五年閏二月丙戌,十二月辛巳,並如之。

正統元年八月癸酉至己卯,月出入時皆有游氣,色赤無光。十四年八月辛未,月晝見,與日爭明。十月壬申,日上黑氣如煙,尋發紅光,散焰如火。

景泰二年四月己卯,月色如赭。七年九月丙子,日色變赤。

天順二年閏二月己巳,日無光,旋赤如赭。三年八月丁卯,日色如赭。六年十月丙子,日赤如血。七年四月癸未,如之。乙酉,日色變白。八年二月己亥,日無光。

成化五年閏二月己卯,日色變白。十一年二月己亥,日色如赭。四月辛卯,如之。十三年三月壬申,日白無光。十月辛卯,十四年三月庚午,十六年三月丙戌,並如之。十七年三月丁酉,日赤如赭。十八年四月壬寅,日赤無光。十二月癸酉,日赤如赭。二十年二月癸酉,如之。

弘治元年十一月己卯,月生芒如齒,長三尺餘,色蒼白。十八年八月癸酉至九月甲午,日無光。

嘉靖元年正月丁卯,日慘白,變青,無光。二十八年三月丙申至庚子,日色慘白。三十四年十二月庚申,晦,日忽暗,有青黑紫日影如盤數十相摩,久之千百,飛蕩滿天,向西北而散。

萬曆二十五年三月癸丑,黑日二三十餘,回繞日旁,移時雲隱不見。五月辛卯朔,日光轉蕩,旋為黑餅。三十年三月甲申,日光照地黃赤。三十五年十一月丙午,日赤無光,燭地如血。四十二年三月庚辰,日赤黃如赭如血者累日。四十四年八月戊辰,日中有黑光。四十六年閏六月丙戌至戊子,黑氣出入日中摩蕩。

天啟四年正月癸未,日赤無光,有黑子二三蕩於旁,漸至百許,凡四日。二月壬子,日淡黃無光。癸丑,黑日摩蕩日旁。四月癸酉,日中黑氣摩蕩。十二月辛巳,午刻,非煙非霧,覆壓日上,摩蕩如蓋如吞,通天皆赤。

崇禎四年正月戊戌,日色如血,照人物皆赤。二月乙巳朔,日赤如血,無光。十月丙午,月晝見。十一年十一月癸亥,日中有黑子及黑青白氣。日入時,日光摩蕩如兩日。十二年正月己未朔,日白無光。辛酉,日光摩蕩竟日,有氣從日中出,如鏡黛噴花。二月庚子,日旁有紅白丸,又白芒黑氣交掩,日光摩蕩。十三年九月己巳,兩日並出,辰刻乃合為一,入時又分為二。十四年正月壬寅,日青無光。後三年正月癸丑,有星入月。三月壬寅,日色無光者兩旬。

▲暈適

洪武六年三月戊辰,日交暈。十年正月己巳,白虹貫日。十二月甲子,白虹貫月。十二年四月庚申,日交暈。二四年正月壬子,日有珥,白虹貫之。九月甲辰,白虹貫日。十五年正月丁未,十九年三月己巳,二十二年十二月戊午,並如之。二十三年正月壬辰,日暈,白虹貫珥。二十八年十一月乙亥,日上赤氣長五丈餘,須臾又生直氣、背氣,皆青赤色。又生半暈,兩白虹貫珥,已而彌天貫日。三十年二月辛亥,白虹亙天貫日。

永樂十八年閏正月癸未,日生重半暈,上有青赤背氣,左右有珥,白虹貫之,隨生黃氣、璚氣。

洪熙元年正月乙未,日生兩珥,白虹貫之。四月丁未,如之,復生交暈。

宣德元年正月庚戌,日生青赤璚氣,隨生交暈,色黃赤。二月己卯,日兩珥,又生交暈,左右有珥,上重半暈及背氣。昏刻,月生兩珥,白虹貫之。二年十二月甲戌,月生交暈,左右珥,白虹貫之。三年三月庚寅,日生交暈,色黃赤,兩珥及背氣、戟氣各一,色皆青赤。丁酉,日暈,又交暈及戟氣二道。十二月己卯,日生交暈。五年正月癸亥,日暈,隨生交暈。二月甲午,日交暈,隨生戟氣。四月庚辰,日生兩珥,白虹貫之。六年二月甲寅,日暈,隨生交暈及重半暈璚氣。八年九月戊戌,辰刻,日暈,兩珥背氣,申刻諸氣復生。十年十二月辛亥,日暈,白虹貫兩珥,有璚氣,隨生重半暈及背氣。

正統元年二月己酉,白虹貫月。九月丁未,如之。十二月丙戌,月生背氣,左右珥,白虹貫之。三年四月庚辰,日生兩珥,白虹貫之,隨暈。十二月癸酉,月生兩珥,白虹貫之,隨生背氣。七年十二月辛丑,月暈,白虹貫之。十一年正月乙未,日生背氣,白虹彌天。十四年八月戊申,日暈,旁有戟氣,隨生左右珥及戴氣,東北虹霓如杵。

景泰元年二月壬午,酉刻,日上黑氣四道,約長三丈,離地丈許,兩頭銳而貫日,其狀如魚。十二月甲午,日交暈,上下背氣各一道,兩旁戟氣各一道。二年正月癸卯,日生左右珥,白虹貫之,隨生背氣。二月丙戌,日交暈。三年正月丙辰,日生左右珥及背氣、白虹。五年十一月壬戌,月暈,左右珥及背氣,又生白虹,貫右珥。七年六月丁丑,日暈,隨生重半暈及左右珥。

天順元年二月庚戌,辰刻,日交暈,左右珥,旋生抱氣及左右戟氣,白虹貫日。未刻,諸氣復生。辛亥,日交暈,左右珥及戟氣,白虹貫日,彌天者竟日。二年二月乙卯,日交暈,上有背氣,白虹貫日。七年正月戊戌,月生連環暈。

成化二年四月壬寅,日交暈,右有珥。十一年六月己酉,日重暈,左右珥及背氣。十二年正月甲子,日交暈。二十年二月己未,日生白虹,東北亙天。二十一年十月癸巳,巳刻,日暈,左右珥。未刻,復生,又生抱氣背氣。二十三年十二月癸巳,日暈,左右珥,又生背氣及半暈。

弘治二年正月甲戌,午刻,日暈,白虹彌天。丙戌,日交暈,左右珥,白虹彌天。二月壬寅,日生左右珥及背氣,又生交暈、半暈及抱、格二氣。十一月戊辰,月暈連環,貫左右珥。四年二月庚戌,午刻,日交暈,左右珥,下生戟氣,白虹彌天。六年十一月乙巳,月暈,左右珥,連環貫之。十八年二月己巳,月暈,左右珥,白虹彌天。

正德元年正月乙酉,日暈,上有背氣,左右有珥,白虹彌天。十二月辛酉,月暈,白虹彌天,甲子,如之。

嘉靖元年四月癸未,月生連環暈。二年正月己酉,月暈,連環左右珥。七年正月乙亥,日重暈,兩珥及戟氣,白虹彌天。十三年二月壬辰,白虹亙天,日暈,左右珥及戟氣。十八年十二月壬午,立春,日暈右珥,白虹亙天。二十一年十一月甲子,月暈連環。四十一年十一月辛丑,日暈,左右珥,上抱下戟,白虹彌天。

隆慶五年三月辛巳,日暈,有珥,白虹亙天。

萬曆三十五年正月庚午,日暈,黑氣蔽天。四十八年二月癸丑,日連環暈,下有背氣,左右戟氣,白虹彌天。

天啟元年二月甲午,日交暈,左右有珥,白虹彌天。三年十月辛巳,日生重半暈,左右珥。

崇禎八年二月丙午,白虹貫日。

▲星變

洪武二十八年閏九月辛巳,壘壁陣疏拆復聚。二十九年八月戊子,欽天監言,井宿東北第二星,近歲漸暗小,促聚不端列。三十一年五月癸亥,壘壁陣疏者就聚。正統元年九月丁巳,狼星動搖。十四年十月辛亥,如之。成化六年丁巳,熒惑無光。十三年九月乙丑朔,歲星光芒炫耀而有玉色。正德元年八月,大角及心中星動搖,北斗中璇、璣、權三星不明。萬曆四十四年,權星暗小,輔星沉沒。四十六年九月,太白光芒四映如月影。天啟五年七月壬申,熒惑色赤,體大,有芒。崇禎九年十二月,熒惑如炬,在太微垣東南。十二年十月甲午,填星昏暈。十三年六月,泰階拆。九月,五車中三柱隱。十月,參足突出玉井。後四年二月,熒惑怒角。三月壬辰,欽天監正戈承科奏,帝星下移。已,又軒轅星絕續不常,太小失次。文昌星拆,天津拆,瑤光拆,芒角黑青。

▲星流星隕

靈臺候簿飛流之記,無夜無有,其小而尋常者無關休咎,擇其異常者書之。

洪武三年十月庚辰,有赤星如桃,起天桴至壘壁陣,抵羽林軍,爆散有聲。五小星隨之,至士司空旁,發光燭天,忽大如碗,曳赤尾至天倉沒,須臾東南有聲。二十一年八月乙巳,赤星如杯,自北斗杓東南行三丈餘,分為二,又五丈餘,分為三,經昴宿復為二,經天廩合為一,沒於天苑。

永樂元年閏十一月丁卯,有星色蒼,大如斗,光燭地,出中天雲中。西南行,隆隆有聲,入雲中。二年五月丙午,有赤星大如斗,光燭地,出中天,西北行入雲中。十六年,有星大如斗,色青赤,光燭地,自柳東行至近濁。二十二年五月己亥,有星如盞,色青白,光燭地,起東南雲中。西北行,入雲中,有聲如砲。七月庚寅,有星如碗,色赤有光,自奎入參炸散,眾星搖動。

宣德元年十二月己巳,有星大如碗,光赤,出卷舌,東行過東井墜地,有聲如雷。

正統元年八月乙酉,昏刻至曉,大小流星百餘。四年八月癸卯,大小流星數百。十四年十月癸丑,有星大如杯,赤光燭地,自三師西北抵少弼,尾跡化蒼白氣,長五尺餘,曲曲西行。十二月戊申,有星大如杯,色青白,有聲,光燭地。自太乙旁東南行丈餘,發光大如斗,至天市西垣沒,四小星隨之。

景泰二年六月丙申,太小流星八十餘。八月壬午,有赤星二,一如桃,一如斗,光燭地。一出紫微西籓北行,至陰德,三小星隨之;一出天津,東南行至河南,十餘小星隨之。尾跡炸散,聲如雷。

天順三年四月癸丑,有星大如碗,赤光燭地,自左旗東南行抵女宿,尾跡炸散。八年二月壬子,有星如碗,光燭地,自天市至天津,尾化蒼白氣,如蛇形,長丈餘,良久散。

成化十二年十一月乙丑,延綏波羅堡有星二,形如轆軸,一墜樊家溝,一墜本堡,紅光燭天。二十年五月丙申,有大星墜番禺縣東南,聲如雷,散為小星十餘。既而天地皆晦,良久乃復。二十一年正月甲申朔,申刻,有火光自中天少西下墜,化白氣,復曲折上騰有聲。踰時,西方有赤星大如碗,自中天西行近濁,尾跡化白氣,曲曲如蛇行良久,正西轟轟如雷震。

弘治元年八月戊申,巳刻,南方流星如盞,自南行丈餘,大如碗,西南至近濁,尾化白雲,屈曲蛇行而散。四年十月丁巳,有星赤,光如電,自西南往東北,聲如鼓。隕光山縣,化為石如斗。光州商城亦見大星飛空,如光山所見。十一月甲戌,星隕真定西北,紅光燭天。西南天鳴如鼓,又若奔車。七年五月,宣府、山西、河南有星晝隕。八年四月辛未,有星如輪,流至西北,隕於鉛山縣,其聲如雷。九年閏三月戊午,平涼東南有流星如月,紅光燭地,至西北止,既而天鼓鳴。十年正月壬子,有星大如斗,色黃白,光長三十餘丈,一小星隨之,隕於寧夏西北隅。天鳴如雷者數聲。九月乙巳,有星如斗,光掩月,流自西北,隕於永平,有聲。十一年正月癸亥,有流星隕於肅州,大如房,響如雷,良久滅。十月壬申,曉,東方赤星如碗,行丈餘,光燭地,東南行,小星數十隨之。十四年閏七月辛巳,山東有星大如車輪,赤光燭天,自東南往西北,隕於壽光。天鼓鳴。十六年正月己酉,南京有星晝流。

正德元年十二月庚午,有星如碗,隕寧夏中衛,空中有紅光大二畝。二年八月己亥,寧夏有大星,自正南流西南而墜,後有赤光一道,闊三尺,長五丈。五年四月丁亥,雷州有大星如月,自東南流西北,分為二,尾如彗,隨沒,聲如雷。六年八月癸卯,有流星如箕,尾長四五丈,紅光燭天。自西北轉東南,三首一尾,墜四川崇慶衛。色化為白,復起綠焰,高二丈餘,聲如雷震。十五年正月丁未,酉刻,有星隕於山西龍舟谷巡檢司廳事,四月丙戌,陜西鞏昌府有星如日,色赤,自東方流西南而隕。天鼓鳴。

嘉靖十二年九月丙子,流星如盞,光照地,自中台東北行近濁,尾跡化為白氣。四更至五更,四方大小流星,縱橫交行,不計其數,至明乃息。十四年九月戊子,開封白晝天鼓鳴。有星如碗,東南流,眾小星從之如珠。十九年五月辛丑,星隕棗強,為石四。

萬曆三年五月癸亥,晝,景州天鼓鳴。隕星二,化為黑石。四年十一月甲午,有四星隕費縣,火光照地。質明,落赤點於城西北,色如硃砂,長二里,闊一二尺。是月,臨漳有星長尺許,白晝北飛。十三年七月辛巳,有星如碗,隕於沈丘蓮花集。天鼓鳴。十五年六月丙寅,平陽晝隕星。丁卯,辰刻,有星如斗,隕於平陰,震響如雷。十七年正月庚申,有星隕西寧衛,大如月。天鼓鳴。二十年二月丙辰,有三星隕閩縣東南。二十二年正月戊戌,保定青山口有大飛星,餘光若彗,長二十餘丈。二十七年三月庚子,蓋州衛天鼓鳴,連隕大星三。三十年九月己未朔,有大星見東南,赤如血,大如碗,忽化為五,中星更明,久之會為一,大如簏。辛巳,有大小星數百交錯行。十月壬辰,五更,流星起中天,光散七道,有聲如雷。三十三年九月戊子,有星如碗,墜於南京龍江後營,光如火,至地遊走如螢,移時滅。明日,復有星如月,從西北流至閱兵臺,分為三,墜地有聲。十一月,有星隕南京教場,入地無跡。三十五年十一月癸巳,有星隕於涇陽、淳化諸縣,大如車輪,赤色,尾長丈餘,聲如轟雷。三十八年二月癸酉,有星大如斗,墜陽曲西北,碎星不絕。天鼓齊鳴。四十一年正月庚子,真定天鼓鳴。流星晝隕有光。四十三年三月戊申,晝,星墜清豐東流邨,聲如雷。四十六年十月,辛酉,有星如斗,隕於南京安德門外,聲如霹靂,化為石,重二十一斤。

天啟三年九月甲寅,固原州星隕如雨。

崇禎十五年夏,星流如織。後二年三月己丑朔,有星隕於御河。

▲雲氣

洪武四年四月辛丑,五色雲見。戊申、乙酉,十一月壬戌,五年正月庚午、丙子,六月辛巳,七月己酉、壬子,八月己亥,六年六月丁丑,七月癸卯,七年四月丙午,五月丙戌、癸巳、甲午,六月乙未、乙卯,七月己卯,八月辛酉,八年正月壬申,四月丁未,五月庚午、癸未,六月壬辰、己亥,十月庚戌,九年八月癸巳,十四年九月甲申,十五年正月甲申,五月庚申,九月乙卯、丙寅,十一月辛酉,十八年四月癸巳、乙未,五月辛未、甲申,六月癸丑,十九年九月壬午,二十年十一月丁亥,五月乙酉,二十七年六月乙卯,並如之。

永樂元年六月甲寅,日下五色雲見。八月壬申,日珥隨五色雲見。八年二月庚戌,車駕次永安甸,日下五色雲見。十一年六月戊申朔,武當山頂五色雲見。十七年九月丙辰、十二月癸未,慶雲見。二十二年十一月丙戌,月下五色雲見。

洪熙元年二月癸酉、庚辰,三月乙未,俱五色雲見。

宣德元年八月庚辰,白雲起東南,狀如群羊驚走。十一月丙辰,北方有蒼白雲,東西竟天。二年十一月乙未,日下五色雲見。四年六月戊子,夜五色雲見。六年二月壬子,昏,西方有蒼白雲,南北竟天。十年三月丁亥,月生五色雲。

正統二年七月庚子,月生五色雲。十月己丑,日生五色雲。十二月癸亥,如之。三年七月己亥,夜,中天有蒼白雲,南北竟天,貫南北斗。八年十一月戊辰,夜,東南方有蒼白雲,東西亙天。九年十一月甲午,月生五色雲。十年九月丁酉,日生五色雲。十一月甲午,月生五色雲。十四年十月庚申,晝生蒼白雲,復化為三,東西南北竟天。

景泰元年六月乙酉,赤雲四道,兩頭銳如耕壟狀,徐徐東北行而散。八月甲戌,黑雲如山,化作龍虎麋鹿狀。九月丙寅,有蒼白雲氣,南北亙天。二年六月戊寅,日上五色雲。九月辛酉,夜蒼白雲三,東西竟天。三年正月癸亥,東南有黑雲,如人戴笠而揖。四年十一月丁卯,月生五色雲。天順二年十月壬申,四年十月戊午,亦如之。

成化二年三月辛未,白雲起南方,東西竟天。十一年正月丙寅,月生五色雲。十八年十月庚午,五色雲見於泰陵。二十一年閏四月壬辰,開、濮二州,清豐,金鄉,未、申時黑雲起西北,化為五色,須臾晦如夜。

弘治二年正月辛巳,日生五色雲。十四年三月己酉朔,嘉靖十七年九月戊子,並如之。十八年二月庚子朔,當午,日下有五色雲見,長徑二寸餘,形如龍鳳。

萬歷五年六月庚辰,祥雲繞月。

天啟四年六月癸巳,午刻,南方五色雲見。


\end{pinyinscope}