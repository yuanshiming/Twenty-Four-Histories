\article{列傳第五十九}

\begin{pinyinscope}
○王驥孫瑾徐有貞楊善李實趙榮霍瑄沈固王越

王驥,字尚德,束鹿人。長身偉幹,便騎射,剛毅有膽,曉暢戎略。中永樂四年進士。為兵科給事中,使山西,奏免鹽池逋課二十餘萬,尋遷山西按察司副使。

洪熙元年入為順天府尹。宣德初,擢兵部右侍郎,代顧佐署都察院。久之,署兵部尚書。九年命為真。

正統元年奉詔議邊事,越五日未奏。帝怒,執驥與侍郎鄺埜下之獄。尋得釋。阿台、朵兒只伯數寇甘、涼,邊將屢失利。侍郎柴車、徐晞,都御史曹翼相繼經理邊務,未能制。二年五月命驥往,許便宜行事。驥疾驅至軍,大會諸將,問「往時追敵魚兒海子,先退敗軍者誰」。僉曰「都指揮安敬」。驥先承密旨戮敬,遂縛敬斬轅門,並宣敕責都督蔣貴。諸將皆股心慄。驥乃大閱將士,分兵畫地,使各自防禦,邊境肅然。閱軍甘、涼,汰三之一。定更番法,兵得休息而轉輸亦省。

俄阿台復入寇。帝以任禮為平羌將軍,蔣貴、趙安為副,驥督軍。三年春,偕諸將出塞,以貴為前鋒,而自與任禮帥大軍後繼,與貴約曰:「不捷,無相見也。」貴擊敵石城,敵走兀魯乃。貴帥輕騎二千五百人出鎮夷,間道兼行,三日夜及之。擒左丞脫羅,斬首三百餘,獲金銀印各一,駝馬兵甲千計。驥與禮自梧桐林至亦集乃,擒樞密、同知、僉院十五人,萬戶二人。降其部落,窮進至黑泉。而趙安等出昌寧,至刁力溝,亦擒右丞、達魯花赤三十人。分道夾擊,轉戰千餘里,朵兒只伯遠遁。論功,貴、禮皆封伯,而驥兼大理卿,支二俸。尋召還,理部事。

久之,麓川之役起。麓川宣慰使思任發叛,數敗王師。黔國公沐晟討之,不利,道卒。以沐昂代。昂條上攻取策,徵兵十二萬人。中官王振方用事,喜功名,以驥可屬,思大舉。驥亦欲自效。

六年正月遂拜蔣貴平蠻將軍,李安、劉聚為副,而驥總督軍務,大發東南諸道兵十五萬討之。刑部侍郎何文淵、侍講劉球先後疏諫,不納。瀕行,賜驥、貴金兜鍪、細鎧、蟒繡緋衣、朱弓矢。驥請得以便宜從事。馳傳至雲南,部署諸將,遣參將冉保由東路趨孟定,大軍由中路至騰衝,分道夾擊。是年十一月,與貴以二萬人趨上江,圍其寨,五日不下。會大風,縱火焚柵,拔之,斬首五萬餘級。進自夾象石,渡下江,通高黎貢山道。閏月至騰衝,長驅抵杉木籠山。賊乘高據險,築七壘相救。驥遣參將宮聚、副將劉聚分左右翼緣嶺上,而自將中軍奮擊之,賊大潰,乘勝至馬鞍山。

踰月,抵賊巢。山陡絕,深塹環之。東南面江,壁立不可上。驥遣前軍覘賊,敗其伏兵。賊更自間道立柵馬鞍山,出大軍後。驥戒軍中無動,而令都指揮方瑛以六千人突賊寨,斬首數百,復誘敗其象陣。會東路軍冉保等已合木邦、車里、大侯諸土軍,破烏木弄、戛邦諸寨。遣別將守西峨渡,防賊軼,刻期與大軍會。驥乃督諸將環攻其七門,積薪縱火。風大作,賊焚死無算,溺江死者數萬人。思任發攜二子走孟養。獲其虎符、金牌、宣慰司印及所掠騰衝諸衛所印章三十有奇。犁其巢穴,留兵守之而還。

明年四月遣偏師討維摩土司韋郎羅。郎羅走安南,俘其妻子。傳檄安南,縛之以獻。五月,師還。帝遣戶部侍郎王質齎羊酒迎勞,賜宴奉天門。封推誠宣力武臣、特進榮祿大夫、上柱國、靖遠伯,歲祿千二百石,世襲指揮同知,賜貂蟬冠玉帶。貴進侯,劉聚等遷賞有差。從征少卿李蕡,郎中侯璡、楊寧皆擢侍郎,士卒賜予加等。府庫為竭。

思任發之竄緬甸也,其子思機發復帥餘眾居者藍,乞入朝謝罪。廷議因而撫之,王振不可。是年八月復命驥總督雲南軍務,帥參將冉保、毛福壽以往。未至而思機發遣弟招賽入貢,緬甸亦奏獲思任發,要麓川地。朝廷不納其貢,且敕驥圖緬甸,驥因請濟師。

八年五月復命蔣貴為平蠻將軍,調土兵五萬往,發卒轉餉五十萬人。驥初檄緬甸送思任發。緬人陽聽命,持兩端。是年冬,大軍逼緬甸,緬人以樓船載思任發覘官軍,而潛以他舟載之歸。驥知緬人資木邦水利為脣齒,且慮思機發將以獻其父故仇之,故終不肯獻思任發。驥乃趨者藍,破思機發巢,得其妻子部落,而思機發獨脫去。

明年召還,加祿三百石,命與都御史陳鎰巡延綏、寧夏、甘肅諸邊。初,寧夏備邊軍,半歲一更。後邊事亟,三年乃更。軍士日久疲罷,又益選軍餘防冬,家有五六人在邊者,軍用重困。驥請歲一更,當代者以十月至,而代者留至來年正月乃遣歸,邊備足而軍不勞。帝善其議,行之諸邊。當是時,緬人已以思任發來獻,而思機發竊駐孟養地,屢遣使入貢謝罪。中外咸願罷兵。振意終未慊,要思機發躬入朝謝。沐斌帥師至金沙江招之,不至。諭孟養執之以獻,亦不聽命。於是振怒,欲盡滅其種類。

十三年春復命驥總督軍務,宮聚為平蠻將軍,帥師十五萬人往。明年造舟浮金沙江,蠻人柵西岸拒守。官軍聯舟為浮橋以濟,拔其柵,進破鬼哭山,連下十餘寨,墜溺死者無算。而思機發終脫去,不可得。是時,官軍踰孟養。至孟冉阜海。地在金沙江西,去麓川千里,自古兵力所不至,諸蠻見大軍皆震怖。而大軍遠涉,驥慮餽餉不繼,亟謀引還。時思機發雖遁匿,而思任發少子思陸復擁眾據孟養。驥度賊終不可滅,乃與思陸約,立石表,誓金沙江上,曰:「石爛江枯,爾乃得渡。」遂班師。

驥凡三征麓川,卒不得思機發。議者咎驥等老師費財,以一隅騷動天下。而會川衛訓導詹英抗疏劾之,大略謂:「驥等多役民夫,舁彩繪,散諸土司以邀厚利。擅用腐刑,詭言進御,實充私役。師行無紀,十五萬人一日起行,互相蹂踐。每軍負米六斗,跋陟山谷,自縊者多。抵金沙江,彳旁徨不敢渡;既渡不敢攻;攻而失都指揮路宣、翟亨等。俟賊解,多捕魚戶為俘。以地分木邦、緬甸,掩敗為功。此何異李宓之敗,而楊國忠以捷聞也。」奏下法司。王振左右之,得不問。而命英從驥軍自效。英知往且獲罪,匿不去。

當是時,湖廣、貴州諸苗,所在蜂起,圍平越及諸城堡,貴州東路閉。驥至武昌,詔還軍討苗。會英宗北狩,群臣劾王振并及驥。以驥方在軍,且倚之平苗,置弗問。命佩平蠻將軍印,充總兵官,侍郎侯璡總督軍務。已而苗益熾,眾至十餘萬。平越被圍半歲,巡按御史黃鎬死守,糧盡掘草根食之,而驥頓軍辰、沅不進。景泰元年,鎬草疏置竹筒中,募人自間道出,聞於朝。更命保定伯梁珤為平蠻將軍,益兵二萬人。侯璡自雲南督之前,疾戰,大破賊,盡解諸城圍。而驥亦俘刬平王蟲富等以獻。

驥還,命總督南京機務。其冬,乞世券,與之。南畿軍素偷惰。驥至,以所馭軍法教之。于謙弗重也,朝廷以其舊臣寵禮之。三年四月,賜敕解任,奉朝請。驥年七十餘,躍馬食肉,盛聲伎如故。

久之,石亨、徐有貞等奉英宗復辟,驥與謀。賞稍後,上章自訟,言:「臣子祥入南城,為諸將所擠,墮地幾死。今論功不及,疑有蔽之者。」帝乃官祥指揮僉事。而命驥仍兵部尚書,理部事,加號奉天翊衛推誠宣力守正文臣、光祿大夫,餘如故。數月請老,又三年乃卒,年八十三。贈靖遠侯,謚忠毅。傳子瑺及孫添。添尚嘉善長公主。

再傳至孫瑾。嘉靖初,提督三千營,協守南京,還掌左府。久之,佩征蠻將軍印,鎮兩廣。廣東新寧、新興、思平間多高山叢箐,亡命者輒入諸瑤中,吏不得問,眾至萬餘人,流劫高要、陽江諸縣。官軍討之,輒失利。三十五年春,瑾與巡撫都御史談愷檄諸路土兵誅其魁陳以明,悉平諸巢。捷聞,加太子太保。而扶藜、葵梅諸山峒馮天恩等,據險為寇者亦數十年。瑾復督軍分道進剿,破巢二百餘,復以功蔭一子錦衣百戶。言官劾其暴橫,召還。爵傳至明亡乃絕。

徐有貞,字元玉,初名珵,吳人。宣德八年進士。選庶吉士,授編修。為人短小精悍,多智數,喜功名。凡天官、地理、兵法、水利、陰陽方術之書,無不諳究。

時承平既久,邊備媮惰,而西南用兵不息,珵以為憂。正統七年疏陳兵政五事,帝善之而不能用。十二年進侍講。十四年秋,熒惑入南斗。珵私語友人劉溥曰「禍不遠矣」,亟命妻子南還。及土木難作,郕王召廷臣問計。珵大言曰:「驗之星象,稽之歷數,天命已去,惟南遷可以紓難。」太監金英叱之,胡濙、陳循咸執不可。兵部侍郎於謙曰:「言南遷者,可斬也。」珵大沮,不敢復言。

景帝即位,遣科道官十五人募兵於外,珵行監察御史事,往彰德。寇退,召還,仍故官。珵急於進取,自創南遷議為內廷訕笑,久不得遷。因遺陳循玉帶,且用星術,言「公帶將玉矣。」無何,循果加少保,大喜,因屢薦之。而是時用人多決於少保于謙。珵屬謙門下士遊說,求國子祭酒。謙為言於帝,帝曰:「此議南遷徐珵邪?為人傾危,將壞諸生心術。」珵不知謙之薦之也,以為沮己,深怨謙。循勸珵改名,因名有貞。

景泰三年遷右諭德。河決沙灣七載,前後治者皆無功。廷臣共舉有貞,乃擢左僉都御史,治之。至張秋,相度水勢,條上三策:一置水門,一開支河,一濬運河。議既定,督漕都御史王竑以漕渠淤淺滯運艘,請急塞決口。帝敕有貞如軏議。有貞守便宜。言:「臨清河淺,舊矣,非因決口未塞也。漕臣但知塞決口為急,不知秋冬雖塞,來春必復決,徒勞無益。臣不敢邀近功。」詔從其言。有貞於是大集民夫,躬親督率,治渠建閘,起張秋以接河、沁。河流之旁出不順者,為九堰障之。更築大堰,楗以水門,閱五百五十五日而工成。名其渠曰「廣濟」,閘曰「通源」。方工之未成也,帝以轉漕為急,工部尚書江淵等請遣中書偕文武大臣督京軍五萬人往助役,期三月畢工。有貞言:「京軍一出,日費不貲,遇漲則束手坐視,無所施力。今泄口已合,決堤已堅,但用沿河民夫,自足集事。」議遂寢。事竣,召還,佐院事。帝厚勞之。復出巡視漕河。濟守十三州縣河夫多負官馬及他雜辦,所司趣之亟,有貞為言免之。七年秋,山東大水,河堤多壞,惟有貞所築如故。有貞乃修舊隄決口,自臨清抵濟寧,各置減水閘,水患悉平。還朝,帝召見,獎勞有加,進左副都御史。

八年正月,景帝不豫。石亨、張輒等謀迎上皇,以告太常卿許彬。彬曰:「此不世功也。彬老矣,無能為。徐元玉善奇策,盍與圖之。」亨即夜至有貞家。聞之,大喜,曰:「須令南城知此意。」軏曰:「陰達之矣。」令太監曹吉祥入白太后。辛巳夜,諸人復會有貞所。有貞升屋覽乾象,亟下曰:「時至矣,勿失。」時方有邊警,有貞令軏詭言備非常,勒兵入大內。亨掌門鑰,夜四鼓,開長安門納之。既入,復閉以遏外兵。時天色晦冥,亨、軏皆惶惑,謂有貞曰:「事當濟否?」有貞大言「必濟」,趣之行。既薄南城,門錮,毀牆以入。上皇燈下獨出問故,有貞等俯伏請登位,乃呼進輿。兵士惶懼不能舉,有貞率諸人助挽以行。星月忽開朗,上皇各問諸人姓名。至東華門,門者拒弗納,上皇曰「朕太上皇帝也」,遂反走。乃升奉天門,有貞等常服謁賀,呼「萬歲」。

景帝明當視朝,群臣咸待漏闕下。忽聞殿中呼噪聲,方驚愕。俄諸門畢啟,有貞出號於眾曰:「太上皇帝復位矣!」趣入賀。即日命有貞兼學士,入內閣,參預機務。明日加兵部尚書。有貞謂亨曰:「願得冠側注從兄後。」亨為言於帝,封武功伯兼華蓋殿大學士,掌文淵閣事,賜號「奉天翊衛推誠宣力守正文臣」,祿千一百石,世錦衣指揮使,給誥券。有貞遂誣少保于謙、大學士王文,殺之。內閣諸臣斥遂略盡。陳循素有德於有貞,亦弗救也。事權盡歸有貞,中外咸側目。而有貞愈益發舒,進見無時,帝亦傾心委任。

有貞既得志,則思自異於曹、石。窺帝於二人不能無厭色,乃稍稍裁之,且微言其貪橫狀,帝亦為之動。御史楊瑄奏劾亨、吉祥侵占民田。帝問有貞及李賢,皆對如瑄奏。有詔獎瑄。亨、吉祥大怨恨,日夜謀構有貞。帝方眷有貞,時屏人密語。吉祥令小豎竊聽得之,故洩之帝。帝驚問曰:「安所受此語?」對曰:「受之有貞,某日語某事,外間無弗聞。」帝自是疏有貞。會御史張鵬等欲糾亨他罪,未上,而給事中王鉉泄之亨、吉祥。二人乃泣訴於帝,謂內閣實主之。遂下諸御史獄,併逮繫有貞及李賢。忽雷雹交作,大風折木。帝憾悟,重違亨意,乃釋有貞出為廣東參政。

亨等憾未已,必欲殺之。令人投匿名書,指斥乘輿,云有貞怨望,使其客馬士權者為之。遂追執有貞於德州,並士權下詔獄,榜治無驗。會承天門災,肆赦。亨、吉祥慮有貞見釋,言於帝曰:「有貞自撰武功伯券辭云『纘禹成功』,又自擇封邑武功。禹受禪為帝,武功者曹操始封也。有貞志圖非望。」帝出以示法司,刑部侍郎劉廣衡等奏當棄市。詔徙金齒為民。

亨敗,帝從容謂李賢、王翱曰:「徐有貞何大罪?為石亨輩所陷耳。其釋歸田里。」成化初,復冠帶閒住。有貞既釋歸,猶冀帝復召,時時仰觀天象,謂將星在吳,益自負。常以鐵鞭自隨,數起舞。及聞韓雍征兩廣有功,乃擲鞭太息曰:「孺子亦應天象邪?」遂放浪山水間,十餘年乃卒。

有貞初出獄時,拊士權背曰:「子,義士也,他日一女相託。」金齒歸,士權時往候之,絕不及婚事。士權辭去,終身不言其事,人以是薄有貞而重士權。

楊善,字思敬,大興人。年十七為諸生。成祖起兵,預城守有勞,授典儀所引禮舍人。

永樂元年,改鴻臚寺序班。善偉風儀,音吐洪亮,工進止。每朝謁引進奏時,上目屬之。累進右寺丞。仁宗即位,擢本寺卿。宣德六年被劾下獄,褫冠帶,踰月。

正統六年,子容詐作中官書,假金於尚書吳中。事覺,謫戍威遠衛,置善不問。久之,擢禮部左侍郎,仍視鴻臚事。

十四年八月扈駕北征。及土木,師潰,善間行得脫。也先將入寇,改左副都御史,與都督王通提督京城守備。寇退,進右都御史,視鴻臚如故。景泰元年,廷臣朝正畢,循故事,相賀於朝房。善獨流涕曰:「上皇在何所,而我曹自相賀乎!」眾愧,為之止。是年夏,李實、羅綺使瓦剌,議罷兵,未還,而也先使至,言朝廷遣使報阿剌知院,而不遣大臣報可汗及太師,事必不濟。尚書王直等奏其言,廷議簡四人為正副使,與偕行,帝命俟李實還議之。已而實將至,乃命善及侍郎趙榮為使,齎金銀書幣往。

先是袁敏者,請齎服御物問上皇安,不納。及是,尚書胡濙等言,上皇蒙塵久,御用服食宜付善等隨行,亦不報。時也先欲還上皇,而敕書無奉迎語,自齎賜也先外,善等無他賜。善乃出家財,悉市彼中所需者,攜以往。

既至,其館伴與飲帳中,詫善曰:「土木之役,六師何怯也?」善曰:「彼時官軍壯者悉南征,王司禮邀大駕幸其里,不為戰備,故令汝得志耳。今南征將士歸,可二十萬。又募中外材官技擊,可三十萬。悉教以神鎗火器藥弩,百步外洞人馬腹立死。又用策士言,緣邊要害,隱鐵椎三尺,馬蹄踐輒穿。又刺客林立,夜度營幕若猿猱。」伴色動。善曰:「惜哉,今皆置無用矣。」問:「何故?」曰:「和議成,歡好且若兄弟,安用此?」因以所齎遺之。其人喜,悉以語也先。

明日謁也先,亦大有所遺,也先亦喜。善因詰之曰:「太上皇帝朝,太師遣貢使必三千人,歲必再賚,金幣載途,乃背盟見攻何也?」也先曰:「奈何削我馬價,予帛多剪裂,前後使人往多不歸,又減歲賜?」善曰:「非削也,太師馬歲增,價難繼而不忍拒,故微損之。太師自度,價比前孰多也?帛剪裂者,通事為之,事露,誅矣。即太師貢馬有劣弱,貂或敝,亦豈太師意耶?且使者多至三四千人,有為盜或犯他法,歸恐得罪,故自亡耳,留若奚為?貢使受宴賜,上名或浮其人數,朝廷核實而予之。所減乃虛數,有其人者,固不減也。」也先屢稱善。善復曰:「太師再攻我,屠戮數十萬,太師部曲死傷亦不少矣。上天好生,太師好殺,故數有雷警。今還上皇,和好如故,中國金幣日至,兩國俱樂,不亦美乎?」也先曰:「敕書何以無奉迎語?」善曰:「此欲成太師令名,使自為之。若載之敕書,是太師迫於朝命,非太師誠心也。」也先大喜,問:「上皇歸將復得為天子乎?」善曰:「天位已定,難再移。」也先曰:「堯、舜如何?」善曰:「堯讓舜,今兄讓弟,正相同也。」其平章昂克問善:「何不以重寶來購?」善曰:「若齎貨來,人謂太師圖利。今不爾,乃見太師仁義,為好男子,垂史策,頌楊萬世。」也先笑稱善。知院伯毅帖木耳勸也先留使臣,而遣使要上皇復位。也先懼失信,不可,竟許善奉上皇還。

時舉朝競奇善功,而景帝以非初遣旨,薄其賞。遷左都御史,仍蒞鴻臚事。二年,廷臣朝正旦畢,修賀朝房。善又曰:「上皇不受賀,我曹何相賀也?」三年正月加太子太保。六年以衰老乞致仕,優詔不許。

善狀貌魁梧,應對捷給。然無學術,滑稽,對客鮮莊語。家京師,治第郭外。園多善果,歲時饋公卿戚里中貴,無不得其歡心。王振用事,善媚事之。至是又與石亨、曹吉祥結。天順元年正月,亨、吉祥奉上皇復辟。善以預謀,封奉天翊衛推誠宣力武臣、特進光祿大夫、柱國、興濟伯,歲祿千二百石,賜世券,掌左軍都督府事。尚書胡濙頌善迎駕功,命兼禮部尚書,尋改守正文臣。善使瓦剌,攜子四人行,至是並得官。又為從子、養子乞恩,得官者復十數人。氣勢烜赫,招權納賄。亨輩嫉而間之,以是漸疏外。二年五月卒。贈興濟侯,謚忠敏。

善負才辨,以巧取功名,而憸忮為士論所棄。其為序班,坐事與庶吉士章樸同繫獄,久之,相狎。時方窮治方孝孺黨,樸言家有孝孺集,未及毀。善從借觀,密奏之。樸以是誅死,而善得復官。于謙、王文之戮,陳循之竄,善亦有力焉。子宗襲爵,後革「奪門」功,降金吾指揮使。孫增尚公主。

李實,字孟誠,合州人。正統七年進士。為人恣肆無拘檢,有口辨。景泰初,為禮科給事中。也先令完者脫歡議和,實請行。擢禮部右侍郎以往,少卿羅綺為副。至則見上皇,頗得也先要領,還言也先請和無他意。及楊善往,上皇果還。是年十月進右都御史,巡撫湖廣。五年召還,掌院事。初,實使謁上皇,請還京引咎自責,失上皇意。後以居鄉暴橫,斥為民。

趙榮,字孟仁,其先西域人。元時入中國,家閩縣。舅薩琦,官翰林,從入都,以能書授中書舍人。

正統十四年十月,也先擁上皇至大同,知府霍瑄謁見,慟哭而返。也先遂犯京師,奉上皇登土城,邀大臣出迓。榮慨然請行。大學士高穀拊其背曰:「子,忠義人也。」解所佩犀帶贈之,即擢大理右少卿,充鴻臚卿。偕右通政王復出城朝見,進羊酒諸物。也先以非大臣,遣之還,而邀于謙、石亨、王直、胡濙出。景帝不遣。改榮太常少卿,仍供事內閣。景泰元年七月擢工部右侍郎,偕楊善等往。敕書無奉迎語,善口辯,榮左右之,竟奉上皇歸。進左侍郎。

行人王晏請開沁河通漕運,再下廷議,言不便,遣榮往勘。還,亦言不便。尋奉敕會山東、河南三司相度河道。眾以榮不由科目,慢之。榮怒,多所撻辱,又自攝衣探水深淺。三司各上章言榮單馬馳走,驚駭軍民,杖傷縣官,鬻廩米多取其直。撫、按薛希璉、張琛亦以聞。章下治河僉都御史徐有貞核奏。法司言,榮雖失大體,終為急於國事,鬻米從人所為。諸臣侮大臣,抗敕旨,宜逮治,希璉、琛亦宜罪。帝令按臣責取諸臣供狀,宥之。

天順元年進尚書。曹欽反,榮策馬大呼於市曰:「曹賊作逆,壯士同我討罪。」果有至者,即率之往。賊平,英宗與李賢言,歎榮忠,命兼大理寺卿,食其俸。七年以疾罷。成化十一年卒。賜恤如制。

霍瑄,字廷璧,鳳翔人。由鄉舉入國學,授大同通判。正統十二年,以武進伯朱冕薦,就擢知府。也先擁英宗至城下,瑄與理餉侍郎沈固等出謁,叩馬號泣。眾露刃叱之,不為動。上皇命括城內金帛,瑄悉所有獻之,上皇嘉歎。寇數出沒大同、渾源,伺軍民樵採,輒驅掠。或幸脫歸,率殘傷肢體。遺民相率入城,無所棲,又乏食。瑄俱為奏之。老弱聽暫徙,發粟振,而所留城守丁壯除賦役。秩滿當遷,鎮巡諸臣乞留。詔加山西右參政,仍治府事。

英宗復位,徵拜瑄工部右侍郎,而固亦以石亨薦,起為戶部尚書。既而巡撫上瑄治行,賜誥旌異。初,瑄在大同,巡撫年富被逮,瑄資其家還里,為鎮守太監韋力轉所惡,撻之十餘。至是瑄以聞,且言力轉每宴輒用妓樂,服御僭侈如王者,強取部民女為妾。力轉亦訐瑄違法事。帝兩釋焉。其年轉左,賜二品服。成化初,屢為言官所劾。命致仕。卒於京師。

瑄初治郡有聲,晚節不檢。特以艱危時見知天子,遂久列顯位。

沈固,丹陽人。永樂中,起家鄉舉,積官至尚書。石亨敗,乞休去。

王越,字世昌,濬人。長身,多力善射。涉書史,有大略。登景泰二年進士。廷試日,旋風起,揚其卷去,更給卷,乃畢事。授御史,出按陜西。聞父訃,不俟代輒歸,為都御史所劾。帝特原之。天順初,起掌諸道章奏,超拜山東按察使。七年,大同巡撫都御史韓雍召還,帝難其代,喟然曰:「安得如雍者而任之。」李賢薦越,召見。越偉服短袂,進止便利。帝喜,擢右副都御史以行。甫至,遭母憂,奪情視事。越乃繕器甲,簡卒伍,修堡寨,減課勸商,為經久計。

成化三年,撫寧侯朱永征毛里孩,以越贊理軍務。其秋,兼巡撫宣府。

五年冬,寇入河套,延綏巡撫王銳請濟師,詔越帥師赴之。河套者,周朔方、秦河南地,土沃,豐水草。東距山西偏頭關,西距寧夏,可二千里。三面阻河,北拊榆林之背。唐三受降城在河外,故內地。明初,阻河為守,延綏亦無事。自天順間,毛里孩等三部始入為寇,然時出沒,不敢久駐。至是始屯牧其中,屢為邊患。越至榆林,遣遊擊將軍許寧出西路龍州、鎮靖諸堡,范瑾出東路神木、鎮羌諸堡,而自與中官秦剛按榆林城為聲援。寧戰黎家澗,瑾戰崖窯川,皆捷;右參將神英又破敵於鎮羌,寇乃退。

明年正月以捷聞,越引還。抵偏頭關,延綏告警。兵部劾越擅還。詔弗罪,而令越屯延綏近地為援。寇萬餘騎五路入掠,越令寧等擊退之。進右副都御史。是年三月,朝廷以阿羅出等擾邊不止,拜撫寧侯朱永為將軍,與越共圖之。破敵開荒川,諸將追奔至牛家寨,阿羅出中流矢走。論功,進右都御史。

又明年,越以方西征,辭大同巡撫。詔聽之,加總督軍務,專辦西事。然是時寇數萬,而官軍堪戰者僅萬人,又分散防守,勢不敵。永、越乃條上戰、守二策。尚書白圭亦難之,請敕諸將守。其年,寇復連入懷遠諸堡,永、越禦卻之。圭復請大舉搜套。

明年遣侍郎葉盛至軍議。時永已召還,越以士卒衣裝盡壞,馬死過半,請且休兵,與盛偕還。而廷議以套不滅,三邊終無寧歲;先所調諸軍已踰八萬,將權不一,迄無成功。宜專遣大將調度。乃拜武靖侯趙輔為平虜將軍,敕陜西、寧夏、延綏三鎮兵皆受節制,越總督軍務。比至,寇方深入環慶、固原飽掠,軍竟無功。

越、輔以滿都魯、孛羅忽、加思蘭方強盛,勢未可破,乃奏言:「欲窮搜河套,非調精兵十五萬不可。今饋餉煩勞,公私困竭,重加科斂,內釁可虞。宜姑事退守,散遣士馬,量留精銳,就糧鄜、延,沿邊軍民悉令內徙。其寇所出沒之所,多置烽燧,鑿塹築牆,以為保障。」奏上,廷議不決。越等又奏:「寇知我軍大集,移營近河,潛謀北渡,殆不戰自屈。但山、陜荒旱,芻糧缺供,邊地早寒,凍餒相繼。以時度之,攻取實難,請從防守之策,臣等亦暫還朝。」於是部科諸臣劾越、輔欺謾。會輔有疾,召還,以寧晉伯劉聚代。

明年,越與聚敗寇溫天嶺,進左都御史。是時三遣大將,皆以越總督軍務。寇每入,小擊輒去,軍罷即復來,率一歲數入。將士益玩寇,而寇勢轉熾。其年九月,滿都魯及孛羅忽、加思蘭留妻子老弱於紅鹽池,大舉深入,直抵秦州、安定諸州縣。越策寇盡銳西,不備東偏,乃率延綏總兵官許寧、遊擊將軍周玉各將五千騎為左右哨,出榆林,踰紅兒山,涉白鹽灘,兩晝夜行八百里。將至,暴風起,塵翳目。一老卒前曰:「天贊我也。去而風,使敵不覺。還軍,遇歸寇,處下風。乘風擊之,蔑不勝矣。」越遽下馬拜之,擢為千戶。分兵千為十覆,而身率寧、玉張兩翼,薄其營,大破之。擒斬三百五十,獲駝馬器械無算,焚其廬帳而還。及滿都魯等飽掠歸,則妻子畜產已蕩盡,相顧痛哭。自是遠徙北去,不敢復居河套,西陲息肩者數年。初,文臣視師者,率從大軍後,出號令行賞罰而已。越始多選跳盪士為腹心將,親與寇搏。又以間覘敵累重邀劫之,或剪其零騎,用是數有功。

十年春,廷議設總制府於固原,舉定西侯蔣琬為總兵官,越提督軍務,控制延綏、寧夏、甘肅三邊。總兵、巡撫而下,並聽節制。詔罷琬,即以越任之,三邊設總制自此始。論功,加太子少保,增俸一級。紀功郎中張謹、兵科給事中郭鏜等論劉聚等濫殺冒功,並劾越妄奏。越方自以功大賞薄,遂怏怏,稱疾還朝。

明年與左都御史李賓同掌院事,兼督十二團營。越素以才自喜,不修小節,為朝議所齮。至是乃破名檢,與群小關通。奸人韋英者,以官奴從徵延綏,冒功得百戶。汪直掌西廠用事,英為爪牙,趙因英自結於直。內閣論罷西廠,越遇大學士劉吉、劉珝於朝,顯謂之曰:「汪直行事亦甚公。如黃賜專權納賂,非直不能去。商、萬在事久,是非多有所忌憚。二公入閣幾日,何亦為此?」珝曰:「吾輩所言,非為身謀。使值行事皆公,朝廷置公卿大夫何為?」越不能對。

兵部尚書項忠罷,越當遷,而朝命予陜西巡撫餘子俊。越彌不平,請解營務,優詔不許。因自陳搗巢功,為故尚書白圭所抑,從征將士多未錄,乞移所加官酬之。子俊亦言越賞不酬功,乃進兵部尚書,仍掌院事。尋加太子太保。

越急功名。汪直初東征,越望督師,為陳鉞所沮。鉞驟寵,心益艷之。十六年春,延綏守臣奏寇潛渡河入靖虜,越乃說直出師。詔拜保國公朱永為平虜將軍,直監軍,而越提督軍務。越說直令永率大軍由南路,己與直將輕騎循塞垣而西,俱會榆林。越至大同,聞敵帳在威寧海子,則盡選宣、大兩鎮兵二萬,出孤店,潛行至貓兒莊,分數道。值大風雨雪晦冥,進至威寧,寇猶不覺,掩擊大破之。斬首四百三十餘級,獲馬駝牛羊六千,師不至榆林而還。永所出道迂,不見敵,無功。由是封越威寧伯,世襲,歲祿千二百石。越受封,不當復領都察院,而越不欲就西班。御史許進等頌其功,引王驥、楊善例,請仍領院事,提督團營。從之。明年復與直、永帥師出大同。適寇入掠,追擊至黑石崖,擒斬百二十餘人,獲馬七百匹。進太子太傅,增歲祿四百石。明制,文臣不得封公侯。越從勳臣例,改掌前軍都督府,總五軍營兵,督團營如故。自是真為武人,且望侯矣。其年五月,宣府告警,命佩平胡將軍印,充總兵官。復以直監督軍務,率京軍萬人赴之。比至,寇已去,因留屯其地。至冬,而直為其儕所間,寵衰。越等再請班師,不許。陳鉞居兵部,亦代直請。帝切責之,兩人始懼。已,大同總兵官孫鉞卒,即命越代之,而以直總鎮大同、宣府,悉召京營將士還。

明年,寇犯延綏。越等調兵援之,頗有斬獲,益祿五十石。帝是時益知越、直交結狀。大學士萬安等以越有智計,恐誘直復進,乃請調越延綏以離之。兩人勢益衰。明年,直得罪,言官並劾越。詔奪爵除名,謫居安陸,三子以功蔭得官者,皆削籍,且使使齎敕諭之。越聞使至,欲自裁,見敕有從輕語,乃稍自安。越既為禮法士所疾,自負豪傑,驁然自如。飲食供奉擬王者,射獵聲樂自恣,雖謫徙不少衰。故其得罪,時議頗謂太過,而竟無白之者。孝宗立,赦還。

弘治七年,越屢疏訟冤。詔復左都御史,致仕。越年七十,耄矣,復結中官李廣,以中旨召掌都察院事。給事中季源、御史王一言等交章論,乃寢。

十年冬寇犯甘肅。廷議復設總制官,先後會舉七人,不稱旨。吏部尚書屠滽以越名上,乃詔起原官,加太子太保,總制甘、涼邊務兼巡撫。越言甘鎮兵弱,非籍延、寧兩鎮兵難以克敵,請兼制兩鎮,解巡撫事。從之。明年,越以寇巢賀蘭山後,數擾邊,乃分兵三路進剿。斬四十三級,獲馬駝百餘。加少保,兼太子太傅。遂條上制置哈密事宜。會李廣得罪死,言官連章劾廣黨,皆及越。越聞憂恨,其冬卒於甘州。贈太傅,謚襄敏。

越姿表奇偉,議論飆舉。久歷邊陲,身經十餘戰,知敵情偽及將士勇怯,出奇制勝,動有成算。獎拔士類,籠罩豪俊,用財若流水,以故人樂為用。又嘗薦楊守隨、鐘、屠滽輩,皆有名於世。睦族敦舊,振窮恤貧,如恐不及。其膽智過絕於人。嘗與朱永帥千人巡邊,寇猝至,永欲走,越止之,列陣自固,寇疑未敢前。薄暮,令騎皆下馬,銜枚魚貫行,自率驍勇為殿,從山後行五十里抵城,謂永曰:「我一動,寇追擊,無噍類矣,示暇以惑之也。下馬行,無軍聲,令寇不覺耳。」性故豪縱。嘗西行謁秦王,王開筵奏妓。越語王:「下官為王吠犬久矣,寧無以相酬者?」因盡乞其妓女以歸。一夕大雪,方圍爐飲,諸妓擁琵琶侍。一小校詗敵還,陳敵情。未竟,越大喜,酌金卮飲之,命彈琵琶侑酒,即以金卮賜之。語畢益喜,指妓絕麗者目之曰:「若得此何如?」校惶恐謝。越大笑,立予之。校所至為盡死力。

越在時,人多咎其貪功。及死,而將餒卒惰,冒功糜餉滋甚,邊臣竟未有如越者。

贊曰:人非有才之難,而善用其才之難。王驥、王越之將兵,楊善之奉使,徐有貞之治河,其才皆有過人者。假使隨流平進,以幹略自奮,不失為名卿大夫。而顧以躁於進取,依附攀援,雖剖符受封,在文臣為希世之遇,而譽望因之隳損,甚亦不免削奪。名節所系,可不重哉!


\end{pinyinscope}