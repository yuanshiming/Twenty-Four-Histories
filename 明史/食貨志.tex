\article{食貨志}


《記》曰:「取財於地,而取法於天。富國之本,在於農桑。」明初,沿元之舊,錢法不通而用鈔,又禁民間以銀交易,宜若不便於民。而洪、永、熙、宣之際,百姓充實,府藏衍溢。蓋是時,劭農務墾闢,土無萊蕪,人敦本業,又開屯田、中鹽以給邊軍,餫餉不仰藉於縣官,故上下交足,軍民胥裕。其後,屯田壞於豪強之兼并,計臣變鹽法。於是邊兵悉仰食太倉,轉輸住往不給。世宗以後,耗財之道廣,府庫匱竭。神宗乃加賦重征,礦稅四出,移正供以實左藏。中涓群小,橫斂侵漁。民多逐末,田卒汙萊。吏不能拊循,而覆侵刻之。海內困敝,而儲積益以空乏。昧者多言復通鈔法可以富國,不知國初之充裕在勤農桑,而不在行鈔法也。夫韁本節用,為理財之要。明一代理財之道,始所以得,終所以失,條其本末,著於篇。

○戶口田制屯田莊田

太祖籍天下戶口,置戶帖、戶籍,具書名、歲、居地。籍上戶部,帖給之民。有司歲計其登耗以聞。及郊祀,中書省以戶籍陳壇下,薦之天,祭畢而藏之。洪武十四年詔天下編賦役黃冊,以一百十戶為一里,推丁糧多者十戶為長,餘百戶為十甲,甲凡十人。歲役里長一人,甲首一人,董一里一甲之事。先後以丁糧多寡為序,凡十年一周,曰排年。在城曰坊,近城曰廂,鄉都曰里。裡編為冊,冊首總為一圖。鰥寡孤獨不任役者,附十甲後為畸零。僧道給度牒,有田者編冊如民科,無田者亦為畸零。每十年有司更定其冊,以丁糧增減而升降之。冊凡四:一上戶部,其三則布政司、府、縣各存一焉。上戶部者,冊面黃紙,故謂之黃冊。年終進呈,送後湖東西二庫庋藏之。歲命戶科給事中一人、御史二人、戶部主事四人釐校訛舛。其後黃冊只具文,有司徵稅、編徭,則自為一冊,曰白冊云。

凡戶三等:曰民,曰軍,曰匠。民有儒,有醫,有陰陽。軍有校尉,有力士,弓、鋪兵。匠有廚役、裁縫、馬船之類。瀕海有鹽灶。寺有僧,觀有道士。畢以其業著籍。人戶以籍為斷,禁數姓合戶附籍。漏口、脫戶,許自實。里設老人,選年高為眾所服者,導民善,平鄉里爭訟。其人戶避徭役者曰逃戶。年饑或避兵他徙者曰流民。有故而出僑於外者曰附籍。朝廷所移民曰移徙。

凡逃戶,明初督令還本籍復業,賜復一年。老弱不能歸及不願歸者,令在所著籍,授田輸賦。正統時,造逃戶周知冊,核其丁糧。

凡流民,英宗令勘籍,編甲互保,屬在所里長管轄之。設撫民佐貳官。歸本者,勞徠安輯,給牛、種、口糧。又從河南、山西巡撫於謙言,免流民復業者稅。成化初,荊、襄寇亂,流民百萬。項忠、楊璇為湖廣巡撫,下令逐之,弗率者戍邊,死者無算。祭酒周洪謨著《流民說》,引東晉時僑置郡縣之法,使近者附籍,遠者設州縣以撫之。都御史李賓上其說。憲宗命原傑出撫,招流民十二萬戶,給閒田,置鄖陽府,立上津等縣統治之。河南巡撫張瑄亦請輯西北流民。帝從其請。

凡附籍者,正統時,老疾致仕事故官家屬,離本籍千里者許收附,不及千里者發還。景泰中,令民籍者收附,軍、匠、灶役冒民籍者發還。

其移徙者,明初,當徙蘇、松、嘉、湖、杭民之無田者四千餘戶,往耕臨濠,給牛、種、車、糧,以資遣之,三年不徵其稅。徐達平沙漠,徙北平山後民三萬五千八百餘戶,散處諸府衛,籍為軍者給衣糧,民給田。又以沙漠遺民三萬二千八百餘戶屯田北平,置屯二百五十四,開地千三百四十三頃。復徙江南民十四萬於鳳陽。戶部郎中劉九皋言:「古狹鄉之民,聽遷之寬鄉,欲地無遺利,人無失業也。」太祖採其議,遷山西澤、潞民於河北。眾屢徙浙西及山西民於滁、和、北平、山東、河南。又徙登、萊、青民於東昌、兗州。又徙直隸、浙江民二萬戶於京師,充倉腳夫。太祖時徙民最多,其間有以罪徙者。建文帝命武康伯徐理往北平度地處之。成祖核太原、平陽、澤、潞、遼、沁、汾丁多田少及無田之家,分其丁口以實北平。自是以後,移徙者鮮矣。

初,太祖設養濟院收無告者,月給糧。設漏澤園葬貧民。天下府州縣立義冢。又行養老之政,民年八十以上賜爵。復下詔優恤遭難兵民。然懲元末豪強侮貧弱,立法多右貧抑富。嘗命戶部籍浙江等九布政司、應天十八府州富民萬四千三百餘戶,以次召見,徙其家以實京師,謂之富戶。成祖時,復選應天、浙江富民三千戶,充北京宛、大二縣廂長,附籍京師,仍應本籍徭役。供給日久,貧乏逃竄,輒選其本籍殷實戶僉補。宣德間定制,逃者發邊充軍,官司鄰里隱匿者俱坐罪。弘治五年始免解在逃富戶,每戶徵銀三兩,與廂民助役。嘉靖中減為二兩,以充邊餉。太祖立法之意,本仿漢徙富民實關中之制,其後事久弊生,遂為厲階。

戶口之數,增減不一,其可考者,洪武二十六年,天下戶一千六十五萬二千八百七十,口六千五十四萬五千八百十二。弘治四年,戶九百十一萬三千四百四十六,口五千三百二十八萬一千一百五十八。萬曆六年,戶一千六十二萬一千四百三十六,口六千六十九萬二千八百五十六。太祖當兵燹之後,戶口顧極盛。其後承平日久,反不及焉。靖難兵起,淮以北鞠為茂草,其時民數反增於前。後乃遞減,至天順間為最衰。成、弘繼盛,正德以後又減。戶口所以減者,周忱謂:「投倚於豪門,或冒匠竄兩京,或冒引賈四方,舉家舟居,莫可蹤跡也。」而要之,戶口增減,由於政令張弛。故宣宗嘗與群臣論歷代戶口,以為「其盛也,本於休養生息,其衰也,由土木兵戎」,殆篤論云。

明土田之制,凡二等:曰官田,曰民田。初,官田皆宋、元時入官田地。厥後有還官田,沒官田,斷入官田,學田,皇莊,牧馬草場,城需苜蓿地,牲地,園陵墳地,公占隙地,諸王、公主、勳戚、大臣、內監、寺觀賜乞莊田,百官職田,邊臣養廉田,軍、民、商屯田,通謂之官田。其餘為民田。

元季喪亂,版籍多亡,田賦無準。明太祖即帝位,遣周鑄等百六十四人,核浙西田畝,定其賦稅。復命戶部核實天下土田。而兩浙富民畏避徭役,大率以田產寄他戶,謂之鐵腳詭寄。洪武二十年命國子生武淳等分行州縣,隨糧定區。區設糧長四人,量度田畝方圓,次以字號,悉書主名及田之丈尺,編類為冊,狀如魚鱗,號曰魚鱗圖冊。先是,詔天下編黃冊,以戶為主,詳具舊管、新收、開除、實在之數為四柱式。而魚鱗圖冊以土田為主,諸原阪、墳衍、下隰、沃瘠、沙鹵之別畢具。魚鱗冊為經,土田之訟質焉。黃冊為緯,賦役之法定焉。凡質賣田土,備書稅糧科則,官為籍記之,毋令產去稅存以為民害。又以中原田多蕪,命省臣議,計民授田。設司農司,開治河南,掌其事。臨濠之田,驗其丁力,計畝給之,毋許兼并。北方近城地多不治,召民耕,人給十五畝,蔬地二畝,免租三年。每歲中書省奏天下墾田數,少者畝以千計,多者至二十餘萬。官給牛及農具者,乃收其稅,額外墾荒者永不起科。二十六年核天下土田,總八百五十萬七千六百二十三頃,蓋駸駸無棄土矣。

凡田以近郭為上地,迤遠為中地、下地。五尺為步,步二百四十為畝,畝百為頃。太祖仍元里社制,河北諸州縣土著者以社分里甲,遷民分屯之地以屯分里甲。社民先占畝廣,屯民新占畝狹,故屯地謂之小畝,社地謂之廣畝。至宣德間,墾荒田永不起科及洿下斥鹵無糧者,皆核入賦額,數溢於舊。有司乃以大畝當小畝以符舊額,有數畝當一畝者。步尺參差不一,人得以意贏縮,土地不均,未有如北方者。貴州田無頃畝尺籍,悉徵之土官。而諸處土田,日久頗淆亂,與黃冊不符。弘治十五年,天下土田止四百二十二萬八千五十八頃,官田視民田得七之一。嘉靖八年,霍韞奉命修會典,言:「自洪武迄弘治百四十年,天下額田已減強半,而湖廣、河南、廣東失額尤多。非撥給於王府,則欺隱於猾民。廣東無籓府,非欺隱即委棄於寇賊矣。司國計者,可不究心?」是時,桂萼、郭弘化、唐能、簡霄先後疏請核實田畝,而顧鼎臣請履畝丈量,丈量之議由此起。江西安福、河南裕州首行之,而法未詳具,人多疑憚。其後福建諸州縣,為經、緯二冊,其法頗詳。然率以地為主,田多者猶得上下其手。神宗初,建昌知府許孚遠為歸戶冊,則以田從人,法簡而密矣。萬曆六年,帝用大學士張居正議,天下田畝通行丈量,限三載竣事。用開方法,以徑圍乘除,畸零截補。於是豪猾不得欺隱,里甲免賠累,而小民無虛糧。總計田數七百一萬三千九百七十六頃,視弘治時贏三百萬頃。然居正尚綜核,頗以溢額為功。有司爭改小弓以求田多,或掊克見田以充虛額。北直隸、湖廣、大同、宣府,遂先後按溢額田增賦云。

屯田之制:曰軍屯,曰民屯。太祖初,立民兵萬戶府,寓兵於農,其法最善。又令諸將屯兵龍江諸處,惟康茂才績最,乃下令褒之,因以申飭將士。洪武三年,中書省請稅太原、朔州屯卒,命勿徵。明年,中書省言:「河南、山東、北平、陜西、山西及直隸淮安諸府屯田,凡官給牛種者十稅五,自備者十稅三。」詔且勿徵,三年後畝收租一斗。六年,太僕丞梁埜仙帖木爾言:「寧夏境內及四川西南至船城,東北至塔灘,相去八百里,土膏沃,宜招集流亡屯田。」從之。是時,遣鄧愈、湯和諸將屯陜西、彰德、汝寧、北平、永平,徙山西真定民屯鳳陽。又因海運餉遼有溺死者,遂益講屯政,天下衛所州縣軍民皆事墾闢矣。

其制,移民就寬鄉,或召募或罪徙者為民屯,皆領之有司,而軍屯則領之衛所。邊地,三分守城,七分屯種。內地,二分守城,八分屯種。每軍受田五十畝為一分,給耕牛、農具,教樹植,復租賦,遣官勸輸,誅侵暴之吏。初畝稅一斗。三十五年定科則:軍田一分,正糧十二石,貯屯倉,聽本軍自支,餘糧為本衛所官軍俸糧。永樂初,定屯田官軍賞罰例:歲食米十二石外餘六石為率,多者賞鈔,缺者罰俸。又以田肥瘠不同,法宜有別,命官軍各種樣田,以其歲收之數相考較。太原左衛千戶陳淮所種樣田,每軍餘糧二十三石,帝命重賞之。寧夏總兵何福積穀尤多,賜敕褒美。戶部尚書郁新言:「湖廣諸衛收糧不一種,請以米為準。凡粟穀穈黍大麥蕎穄二石,稻穀薥秫二石五斗,穇稗三石,皆準米一石。小麥芝麻豆與米等。」從之,著為令。

又更定屯守之數。臨邊險要,守多於屯。地僻處及輸糧艱者,屯多於守,屯兵百名委百戶,三百名委千戶,五百名以上指揮提督之。屯設紅牌,列則例於上。年六十與殘疾及幼者,耕以自食,不限於例。屯軍以公事妨農務者,免徵子粒,且禁衛所差撥。於時,東自遼左,北抵宣、大,西至甘肅,南盡滇、蜀,極於交阯,中原則大河南北,在在興屯矣。宣宗之世,屢核各屯,以征戍罷耕及官豪勢要占匿者,減餘糧之半。迤北來歸就屯之人,給車牛農器。分遼東各衛屯軍為三等,丁牛兼者為上,丁牛有一為中,俱無者為下。英宗免軍田正糧歸倉,止徵餘糧六石。後又免沿邊開田官軍子粒,減各邊屯田子粒有差。景帝時,邊方多事,令兵分為兩番,六日操守,六日耕種。成化初,宣府巡撫葉盛買官牛千八百,並置農具,遣軍屯田,收糧易銀,以補官馬耗損,邊入稱便。

自正統後,屯政稍弛,而屯糧猶存三之二。其後屯田多為內監、軍官占奪,法盡壞。憲宗之世頗議釐復,而視舊所入,不能什一矣。弘治間,屯糧愈輕,有畝止三升者。沿及正德,遼東屯田較永樂間田贏萬八千餘頃,而糧乃縮四萬六千餘石。初,永樂時,屯田米常溢三之一,常操軍十九萬,以屯軍四萬供之。而受供者又得自耕。邊外軍無月糧,以是邊餉恒足。及是,屯軍多逃死,常操軍止八萬,皆仰給於倉。而邊外數擾,棄不耕。劉瑾擅政,遣官分出丈田責逋。希瑾意者,偽增田數,搜括慘毒,戶部侍郎韓福尤急刻。遼卒不堪,脅眾為亂,撫之乃定。

明初,募鹽商於各邊開中,謂之商屯。迨弘治中,葉淇變法,而開中始壞。諸淮商悉撤業歸,西北商亦多徙家於淮,邊地為墟,米石直銀五兩,而邊儲枵然矣。世宗時,楊一清復請召商開中,又請仿古募民實塞下之意,招徠隴右、關西民以屯邊。其後周澤、王崇古、林富、陳世輔、王畿、王朝用、唐順之、吳桂芳等爭言屯政。而龐尚鵬總理江北鹽屯,尋移九邊,與總督王崇古,先後區畫屯政甚詳。然是時因循日久,卒鮮實效。給事中管懷理言:「屯田不興,其弊有四:疆埸戒嚴,一也;牛種不給,二也;丁壯亡徙,三也;田在敵外,四也。如是而管屯者猶欲按籍增賦,非扣月糧,即按丁賠補耳。」

屯糧之輕,至弘、正而極,嘉靖中漸增,隆慶間復畝收一斗。然屯丁逃亡者益多。管糧郎中不問屯田有無,月糧止半給。沿邊屯地,或變為斥鹵、沙磧,糧額不得減。屯田御史又於額外增本折,屯軍益不堪命。萬曆時,計屯田之數六十四萬四千餘頃,視洪武時虧二十四萬九千餘頃,田日減而糧日增,其弊如此。時則山東巡撫鄭汝璧請開登州海北長山諸島田。福建巡撫許孚遠墾閩海壇山田成,復請開南日山、澎湖;又言浙江濱海諸山,若陳錢、金塘、補陀、玉環、南麂,皆可經理。天津巡撫汪應蛟則請於天津興屯。或留中不下,或不久輒廢。熹宗之世,巡按張慎言復議天津屯田。而御史左光斗命管河通判盧觀象大興水田之利,太常少卿董應舉踵而行之。光斗更於河間、天津設屯學,試騎射,為武生給田百畝。李繼貞巡撫天津,亦力於屯務,然仍歲旱蝗,弗克底成效也。明時,草場頗多,占奪民業。而為民厲者,莫如皇莊及諸王、勛戚、中官莊田為甚。太祖賜勛臣公侯丞相以下莊田,多者百頃,親王莊田千頃。又賜公侯暨武臣公田,又賜百官公田,以其租入充祿。指揮沒於陣者皆賜公田。勳臣莊佃,多倚威扞禁,帝召諸臣戒諭之。其後公侯復歲祿,歸賜田於官。

仁、宣之世,乞請漸廣,大臣亦得請沒官莊舍。然寧王權請灌城為庶子耕牧地,帝賜書,援祖制拒之。至英宗時,諸王、外戚、中官所在占官私田,或反誣民占,請案治。比案問得實,帝命還之民者非一。乃下詔禁奪民田及奏請畿內地。然權貴宗室莊田墳塋,或賜或請,不可勝計。御馬太監劉順家人進薊州草場,進獻由此始。宦官之田,則自尹奉、喜寧始。初,洪熙時,有仁壽宮莊,其後又有清寧、未央宮莊。天順三年,以諸王未出閣,供用浩繁,立東宮、德王、秀王莊田。二王之籓,地仍歸官。憲宗即位,以沒入曹吉祥地為宮中莊田,皇莊之名由此始。其後莊田遍郡縣。給事中齊莊言:「天子以四海為家,何必置立莊田,與貧民較利。」弗聽。弘治二年,戶部尚書李敏等以災異上言:「畿內皇莊有五,共地萬二千八百餘頃;勛戚、中官莊田三百三十有二,共地三萬三千餘頃。管莊官校招集群小,稱莊頭、伴當,占地土,斂財物,汙婦女。稍與分辯,輒被誣奏。官校執縛,舉家驚惶。民心傷痛入骨,災異所由生。乞革去管莊之人,付小民耕種,畝徵銀三分,充各宮用度。」帝命戒飭莊戶。又因御史言,罷仁壽宮莊,還之草場,且命凡侵牧地者,悉還其舊。

又定制,獻地王府者戍邊。奉御趙瑄獻雄縣地為皇莊,戶部尚書周經劾其違制,下瑄詔獄。敕諸王輔導官,導王奏請者罪之。然當日奏獻不絕,氣請亦愈繁。徽、興、岐、衡四王,田多至七千餘頃。會昌、建昌、慶雲三侯爭田,帝輒賜之。武宗即位,踰月即建皇莊七,其後增至三百餘處。諸王、外戚求請及奪民田者無算。

世宗初,命給事中夏言等清核皇莊田。言極言皇莊為厲於民。自是正德以來投獻侵牟之地,頗有給還民者,而宦戚輩復中撓之。戶部尚書孫交造皇莊新冊,額減於舊。帝命核先年頃畝數以聞,改稱官地,不復名皇莊,詔所司徵銀解部。然多為宦寺中飽,積逋至數十萬以為常。是時,禁勛戚奏討、奸民投獻者,又革王府所請山場湖陂。德王請齊、漢二庶人所遺東昌、兗州閒田,又請白雲等湖,山東巡撫邵錫按新令卻之,語甚切。德王爭之數四,帝仍從部議,但存籓封初請莊田。其後有奏請者不聽。

又定,凡公主、國公莊田,世遠者存什三。嘉靖三十九年遣御史沈陽清奪隱冒莊田萬六千餘頃。穆宗從御史王廷瞻言,復定世次遞減之限:勛臣五世限田二百頃,戚畹七百頃至七十頃有差。初,世宗時,承天六莊二湖地八千三百餘頃,領以中官,又聽校舍兼並,增八百八十頃,分為十二莊。至是始領之有司,兼并者還民。又著令宗室買田不輸役者沒官,皇親田俱令有司徵之,如勳臣例。雖請乞不乏,而賜額有定,徵收有制,民害少衰止。

神宗賚予過侈,求無不獲。潞王、壽陽公主恩最渥。而福王分封,括河南、山東、湖廣田為王莊,至四萬頃。群臣力爭,乃減其半。王府官及諸閹丈地徵稅,旁午於道,扈養廝役廩食以萬計,漁斂慘毒不忍聞。駕帖捕民,格殺莊佃,所在騷然。給事中官應震、姚宗文等屢疏諫,皆不報。時復更定勛戚莊田世次遞減法,視舊制稍寬。其後應議減者,輒奉詔姑留,不能革也。熹宗時,桂、惠、瑞三王及遂平、寧德二公主莊田,動以萬計,而魏忠賢一門,橫賜尤甚。蓋中葉以後,莊田侵奪民業,與國相終云。


○賦役

賦役之法,唐租庸調猶為近古。自楊炎作兩稅法,簡而易行,歷代相沿,至明不改。太祖為吳王,賦稅十取一,役法計田出夫。縣上、中、下三等,以賦十萬、六萬、三萬石下為差。府三等,以賦二十萬上下、十萬石下為差。即位之初,定賦役法,一以黃冊為準。冊有丁有田,丁有役,田有租。租曰夏稅,曰秋糧,凡二等。夏稅無過八月,秋糧無過明年二月。丁曰成丁,曰未成丁,凡二等。民始生,籍其名曰不成丁,年十六曰成丁。成丁而役,六十而免。又有職役優免者,役曰里甲,曰均徭,曰雜泛,凡三等。以戶計曰甲役,以丁計曰徭役,上命非時曰雜役,皆有力役,有雇役。府州縣驗冊丁口多寡,事產厚薄,以均適其力。

兩稅,洪武時,夏稅曰米麥,曰錢鈔,曰絹。秋糧曰米,曰錢鈔,曰絹。弘治時,會計之數,夏稅曰大小米麥,曰麥荍,曰絲綿并荒絲,曰稅絲,曰絲綿折絹,曰稅絲折絹,曰本色絲,曰農桑絲折絹,曰農桑零絲,曰人丁絲折絹,曰改科絹,曰棉花折布,曰苧布,曰土苧,曰紅花,曰麻布,曰鈔,曰租鈔,曰稅鈔,曰原額小絹,曰幣帛絹,曰本色絹,曰絹,曰折色絲。秋糧曰米,曰租鈔,曰賃鈔,曰山租鈔,曰租絲,曰租絹,曰粗租麻布,曰課程棉布,曰租苧布,曰牛租米穀,曰地畝棉花絨,曰棗子易米,曰棗株課米,曰課程苧麻折米,曰棉布,曰魚課米,曰改科絲折米。萬曆時,小有所增損,大略以米麥為主,而絲絹與鈔次之。夏稅之米惟江西、湖廣、廣東、廣西,麥荍惟貴州,農桑絲遍天下,惟不及川、廣、雲、貴,餘各視其地產。

太祖初立國即下令,凡民田五畝至十畝者,栽桑、麻、木棉各半畝,十畝以上倍之。麻畝徵八兩,木棉畝四兩。栽桑以四年起科。不種桑,出絹一疋。不種麻及木棉,出麻布、棉布各一疋。此農桑絲絹所由起也。

洪武九年,天下稅糧,令民以銀、鈔、錢、絹代輸。銀一兩、錢千文、鈔一貫,皆折輸米一石,小麥則減直十之二。棉苧一疋,折米六斗,麥七斗。麻布一疋,折米四斗,麥五斗。絲絹等各以輕重為損益,願人粟者聽。十七年,雲南以金、銀、貝、布、漆、丹砂、水銀代秋租。於是謂米麥為本色,而諸折納稅糧者,謂之折色。越二年,又令戶部侍郎楊靖會計天下倉儲存糧,二年外並收折色,惟北方諸布政司需糧餉邊,仍使輸粟。三十年諭戶部曰:「行人高稹言,陜西困逋賦。其議自二十八年以前,天下逋租,咸許任土所產,折收布、絹、棉花及金、銀等物,著為令。」於是戶部定:鈔一錠,折米一石;金一兩,十石;銀一兩,二石;絹一疋,石有二斗;棉布一疋,一石;苧布一疋,七斗;棉花一斤,二斗。帝曰:「折收逋賦,蓋欲蘇民困也。今賦重若此,將愈困民,豈恤之之意哉。金、銀每兩折米加一倍。鈔止二貫五百文折一石。餘從所議。」

永樂中,既得交阯,以絹,漆,蘇木,翠羽,紙扇,沉、速、安息諸香代租賦。廣東瓊州黎人、肇慶瑤人內附,輸賦比內地。天下本色稅糧三千餘萬石,絲鈔等二千餘萬。計是時,宇內富庶,賦入盈羨,米粟自輸京師數百萬石外,府縣倉廩蓄積甚豐,至紅腐不可食。歲歉,有司往往先發粟振貸,然後以聞。雖歲貢銀三十萬兩有奇,而民間交易用銀,仍有厲禁。

至正統元年,副都御史周銓言:「行在各衛官俸支米南京,道遠費多,輒以米易貨,貴買賤售,十不及一。朝廷虛糜廩祿,各官不得實惠。請於南畿、浙江、江西、湖廣不通舟楫地,折收布、絹、白金,解京充俸。」江西巡撫趙新亦以為言,戶部尚書黃福復條以請。帝以問行在戶部尚書胡濙。濙對以太祖嘗折納稅糧於陜西、浙江,民以為便。遂仿其制,米麥一石,折銀二錢五分。南畿、浙江、江西、湖廣、福建、廣東、廣西米麥共四百餘萬石,折銀百萬餘兩,入內承運庫,謂之金花銀。其後概行於天下。自起運兌軍外,糧四石收銀一兩解京,以為永例。諸方賦入折銀,而倉廩之積漸少矣。

初,太祖定天下官、民田賦,凡官田畝稅五升三合五勺,民田減二升,重租田八升五合五勺,沒官田一斗二升。惟蘇、松、嘉、湖,怒其為張士誠守,乃籍諸豪族及富民田以為官田,按私租簿為稅額。而司農卿楊憲又以浙西地膏腴,增其賦,畝加二倍。

故浙西官、民田視他方倍蓰,畝稅有二三石者。大抵蘇最重,松、嘉、湖次之,常、杭又次之。洪武十三年命戶部裁其額,畝科七斗五升至四斗四升者減十之二,四斗三升至三斗六升者俱止徵三斗五升,其以下者仍舊。時蘇州一府,秋糧二百七十四萬六千餘石,自民糧十五萬石外,皆官田糧。官糧歲額與浙江通省埒,其重猶如此。建文二年詔曰:「江、浙賦獨重,而蘇、松準私租起科,特以懲一時頑民,豈可為定則以重困一方。宜悉與減免,畝不得過一斗。」成祖盡革建文政,浙西之賦復重。宣宗即位,廣西布政使周乾巡視蘇、常、嘉、湖諸府還,言:「諸府民多逃亡,詢之耆老,皆云重賦所致。如吳江、崑山民田租,舊畝五升,小民佃種富民田,畝輸私租一石。後因事故入官,輒如私租例盡取之。十分取八,民猶不堪,況盡取乎。盡取,則民必凍餒,欲不逃亡,不可得也。仁和、海寧、崑山海水陷官、民田千九百餘頃,逮今十有餘年,猶徵其租。田沒於海,租從何出?請將沒官田及公、侯還官田租,俱視彼處官田起科,畝稅六斗。海水淪陷田,悉除其稅,則田無荒蕪之患,而細民獲安生矣。」帝命部議行之。宣德五年二月詔:「舊額官田租,畝一斗至四斗者各減十之二,四斗一升至一石以上者減十之三。著為令。」於是江南巡撫周忱與蘇州知府況鐘,曲計減蘇糧七十餘萬,他府以為差,而東南民力少紓矣。忱又令松江官田依民田起科,戶部劾以變亂成法。宣宗雖不罪,亦不能從。而朝廷數下詔書,蠲除租賦。持籌者輒私戒有司,勿以詔書為辭。帝與尚書胡濙言「計臣壅遏膏澤」,然不深罪也。正統元年令蘇、松、浙江等處官田,準民田起科,秋糧四斗一升至二石以上者減作三斗,二斗一升以上至四斗者減作二斗,一斗一升至二斗者減作一斗。蓋宣德末,蘇州逋糧至七百九十萬石,民困極矣。至是,乃獲少蘇。英宗復辟之初,令鎮守浙江尚書孫原貞等定杭、嘉、湖則例,以起科重者徵米宜少,起科輕者徵米宜多。乃定官田畝科一石以下,民田七斗以下者,每石歲徵平米一石三斗;官民田四斗以下者,每石歲徵平米一石五斗;官田二斗以下,民田二斗七升以下者,每石歲徵平米一石七斗;官田八升以下,民田七升以下者,每石歲徵平米二石二斗。凡重者輕之,輕者重之,欲使科則適均,而畝科一石之稅未嘗減云。

嘉靖二年,御史黎貫言:「國初夏秋二稅,麥四百七十餘萬石,今少九萬;米二千四百七十餘萬石,今少二百五十餘萬。而宗室之蕃,官吏之冗,內官之眾,軍士之增,悉取給其中。賦入則日損,支費則日加。請核祖宗賦額及經費多寡之數,一一區畫,則知賦入有限,而浮費不容不節矣。」於是戶部議:「令天下官吏考滿遷秩,必嚴核任內租稅,徵解足數,方許給由交代。仍乞朝廷躬行節儉,以先天下。」帝納之。既而諭德顧鼎臣條上錢糧積弊四事:

一曰察理田糧舊額。請責州縣官,於農隙時,令里甲等仿洪武、正統間魚鱗、風旗之式,編造圖冊,細列元額田糧、字圩、則號、條段、坍荒、成熟步口數目,官為覆勘,分別界址,履畝檢踏丈量,具開墾改正豁除之數。刊刻成書,收貯官庫,給散里中,永為稽考。仍斟酌先年巡撫周忱、王恕簡便可行事例,立為定規。取每歲實徵、起運、存留、加耗、本色、折色并處補、暫徵、帶徵、停徵等件數目,會計已定,張榜曉諭。庶吏胥不得售其奸欺,而小民免賠累科擾之患。一曰催徵歲辦錢糧。成、弘以前,里甲催徵,糧戶上納,糧長收解,州縣臨收。糧長不敢多收斛面,糧戶不敢攙雜水穀糠粃,兌糧官軍不敢阻難多索,公私兩便。近者,有司不復比較經催里甲負糧人戶,但立限敲撲糧長,令下鄉追徵。豪強者則大斛倍收,多方索取,所至雞犬為空。孱弱者為勢豪所凌,耽延欺賴,不免變產補納。至或舊役侵欠,責償新僉,一人逋負,株連親屬,無辜之民死於箠楚囹圄者幾數百人。且往時每區糧長不過正、副二名,近多至十人以上。其實收掌管糧之數少,而科斂打點使用年例之數多。州縣一年之間,輒破中人百家之產,害莫大焉。宜令戶部議定事例,轉行所司,審編糧長務遵舊規。如州縣官多僉糧長,縱容下鄉,及不委里甲催辦,輒酷刑限比糧長者,罪之。致人命多死者,以故勘論。

其二則議遣官綜理及復預備倉糧也。疏下,戶部言:「所陳俱切時弊,令所司舉行。」遷延數載如故。

糧長者,太祖時,令田多者為之,督其鄉賦稅。歲七月,州縣委官偕詣京,領勘合以行。糧萬石,長、副各一人,輸以時至,得召見,語合,輒蒙擢用。末年更定,每區正副二名輪充。宣德間,復永充。科斂橫溢,民受其害,或私賣官糧以牟利。其罷者,虧損公賦,事覺,至隕身喪家。景泰中,革糧長,未幾又復。自官軍兌運,糧長不復輸京師,在州里間頗滋害,故鼎臣及之。

未幾,御史郭弘化等亦請通行丈量,以杜包賠兼并之弊。帝恐紛擾,不從。給事中徐俊民言:「今之田賦,有受地於官,歲供租稅者,謂之官田。有江水泛溢溝塍淹沒者,謂之坍江。有流移亡絕,田棄糧存者,謂之事故。官田貧民佃種,畝入租三斗,或五六斗或石以上者有之。坍江、事故虛糧,里甲賠納,或數十石或百餘石者有之。夫民田之價十倍官田,貧民既不能置。而官田糧重,每病取盈,益以坍江、事故虛糧,又令攤納,追呼敲撲,歲無寧日。而奸富猾胥方且詭寄、那移,并輕分重。此小民疾苦,閭閻凋瘁,所以日益而日增也。請定均糧、限田之制。坍江、事故,悉與蠲免。而合官民田為一,定上、中、下三則起科以均糧。富人不得過千畝,聽以百畝自給,其羨者則加輸邊稅。如此,則多寡有節,輕重適宜,貧富相安,公私俱足矣。」部議:「疆土民俗各異,令所司熟計其便。」不行。

越數年,乃從應天巡撫侯位奏,免蘇州坍海田糧九萬餘石,然那移、飛灑之弊,相沿不改。至十八年,鼎臣為大學士,復言:「蘇、松、常、鎮、嘉、湖、杭七府,供輸甲天下,而里胥豪右蠹弊特甚。宜將欺隱及坍荒田土,一一檢核改正。」於是應天巡撫歐陽鐸檢荒田四千餘頃,計租十一萬石有奇,以所欺隱田糧六萬餘石補之,餘請豁免。戶部終持不下。時嘉興知府趙瀛建議:「田不分官、民,稅不分等則,一切以三斗起徵。」鐸乃與蘇州知府王儀盡括官、民田裒益之。履畝清丈,定為等則。所造經賦冊,以八事定稅糧:曰元額稽始,曰事故除虛,曰分項別異,曰歸總正實,曰坐派起運,曰運餘撥存,曰存餘考積,曰徵一定額。又以八事考里甲:曰丁田,曰慶賀,曰祭祀,曰鄉飲,曰科賀,曰恤政,曰公費,曰備用。以三事定均徭:曰銀差,曰力差,曰馬差。著為例。

徵一者,總徵銀米之凡,而計畝均輸之。其科則最重與最輕者,稍以耗損益推移。重者不能盡損,惟遞減耗米,派輕齎折除之,陰予以輕。輕者不能加益,為徵本色,遞增耗米加乘之,陰予以重。推收之法,以田為母,戶為子。時豪右多梗其議,鼎臣獨以為善,曰:「是法行,吾家益千石輸,然貧民減千石矣,不可易也。」顧其時,上不能損賦額,長民者私以己意變通。由是官田不至偏重,而民田之賦反加矣。

時又有綱銀、一串鈴諸法。綱銀者,舉民間應役歲費,丁四糧六總徵之,易知而不繁,猶網之有綱也。一串鈴,則夥收分解法也。自是民間輸納,止收本色及折色銀矣。

是時天下財賦,歲入太倉庫者二百萬兩有奇。舊制以七分經費而存積三分備兵、歉,以為常。世宗中年,邊供費繁,加以土木、禱祀,月無虛日,帑藏匱竭。司農百計生財,甚至變賣寺田,收贖軍罪,猶不能給。二十九年,俺荅犯京師,增兵設戍,餉額過倍。三十年,京邊歲用至五百九十五萬,戶部尚書孫應奎蒿目無策,乃議於南畿、浙江等州縣增賦百二十萬,加派於是始。

嗣後,京邊歲用,多者過五百萬,少者亦三百餘萬,歲入不能充歲出之半。由是度支為一切之法,其箕斂財賄、題增派、括贓贖、算稅契、折民壯、提編、均徭、推廣事例興焉。其初亦賴以濟匱,久之諸所灌輸益少。又四方多事,有司往往為其地奏留或請免:浙、直以備倭,川、貴以採木,山、陜、宣、大以兵荒。不惟停格軍興所徵發,即歲額二百萬,且虧其三之一。而內廷之賞給,齋殿之經營,宮中夜半出片紙,吏雖急,無敢延頃刻者。三十七年,大同右衛告警,賦入太倉者僅七萬,帑儲大較不及十萬。戶部尚書方鈍等憂懼不知所出,乃乘間具陳帑藏空虛狀,因條上便宜七事以請。既,又令群臣各條理財之策,議行者凡二十九事,益瑣屑,非國體。而累年以前積逋無不追徵,南方本色逋賦亦皆追徵折色矣。

是時,東南被倭,南畿、浙、閩多額外提編,江南至四十萬。提編者,加派之名也。其法,以銀力差排編十甲,如一甲不足,則提下甲補之,故謂之提編。及倭患平,應天巡撫周如斗乞減加派,給事中何煃亦具陳南畿困敝,言:「軍門養兵,工部料價,操江募兵,兵備道壯丁,府州縣鄉兵,率為民累,甚者指一科十,請禁革之。」命如煃議,而提編之額不能減。

隆、萬之世,增額既如故,又多無藝之徵,逋糧愈多,規避亦益巧。已解而愆限或至十餘年,未徵而報收,一縣有至十萬者。逋欠之多,縣各數十萬。賴行一條鞭法,無他科擾,民力不大絀。

一條鞭法者,總括一州縣之賦役,量地計丁,丁糧畢輸於官。一歲之役,官為僉募。力差,則計其工食之費,量為增減;銀差,則計其交納之費,加以增耗。凡額辦、派辦、京庫歲需與存留、供億諸費,以及土貢方物,悉併為一條,皆計畝徵銀,折辦於官,故謂之一條鞭。立法頗為簡便。嘉靖間,數行數止,至萬曆九年乃盡行之。

其後接踵三大征,頗有加派,事畢旋已。至四十六年,驟增遼餉三百萬。時內帑充積,帝靳不肯發。戶部尚書李汝華乃援征倭、播例,畝加三釐五毫,天下之賦增二百萬有奇。明年復加三釐五毫。明年,以兵工二部請,復加二釐。通前後九釐,增賦五百二十萬,遂為歲額。所不加者,畿內八府及貴州而已。

天啟元年,給事中甄淑言:「遼餉加派,易致不均。蓋天下戶口有戶口之銀,人丁有人丁之銀,田土有田土之銀,有司徵收,總曰銀額。按銀加派,則其數不漏。東西南北之民,甘苦不同,布帛粟米力役之法,徵納不同。惟守令自知其甘苦,而通融其徵納。今因人土之宜,則無偏枯之累。其法,以銀額為主,而通人情,酌土俗,頒示直省。每歲存留、起解各項銀兩之數,以所加餉額,按銀數分派,總提折扣,裒多益寡,期不失餉額而止。如此,則愚民易知,可杜奸胥意為增減之弊。且小民所最苦者,無田之糧,無米之丁,田鬻富室,產去糧存,而猶輸丁賦。宜取額丁、額米,兩衡而定其數,米若干即帶丁若干。買田者,收米便收丁,則縣冊不失丁額,貧民不致賠累,而有司亦免逋賦之患。」下部覆議,從之。

崇禎三年,軍興,兵部尚書梁廷棟請增田賦。戶部尚書畢自嚴不能止,乃於九釐外畝復徵三釐。惟順天、永平以新被兵無所加,餘六府畝徵六厘,得他省之半,共增賦百六十五萬四千有奇。後五年,總督盧象昇請加宦戶田賦十之一,民糧十兩以上同之。既而概徵每兩一錢,名曰助餉。越二年,復行均輸法,因糧輸餉,畝計米六合,石折銀八錢,又畝加徵一分四釐九絲。越二年,楊嗣昌督師,畝加練餉銀一分。兵部郎張若麒請收兵殘遺產為官莊,分上、中、下,畝納租八斗至二三斗有差。御史衛周胤言:「嗣昌流毒天下,剿練之餉多至七百萬,民怨何極。」御史郝晉亦言:「萬歷末年,合九邊餉止二百八十萬。今加派遼餉至九百萬。剿餉三百三十萬,業已停罷,旋加練餉七百三十餘萬。自古有一年而括二千萬以輸京師,又括京師二千萬以輸邊者乎?」疏語雖切直,而時事危急,不能從也。

役法定於洪武元年。田一頃出丁夫一人,不及頃者以他田足之,名曰均工夫。尋編應天十八府州,江西九江、饒州、南康三府均工夫圖冊。每歲農隙赴京,供役三十日遣歸。田多丁少者,以佃人充夫,而田主出米一石資其用。非佃人而計畝出夫者,畝資米二升五合。迨造黃冊成,以一百十戶為一里,里分十甲曰里甲。以上、中、下戶為三等,五歲均役,十歲一更造。一歲中諸歲雜目應役者,編第均之,銀、力從所便,曰均徭。他雜役。凡祗應、禁子、弓兵,悉僉市民,毋役糧戶。額外科一錢、役一夫者,罪流徙。

後法稍馳,編徭役里甲者,以戶為斷,放大戶而勾單小。於是議者言,均徭之法,按冊籍丁糧,以資產為宗,核人戶上下,以蓄藏得實也。稽冊籍,則富商大賈免役,而土著困;核人戶,則官吏里胥輕重其手,而小民益窮蹙。二者交病。然專論丁糧,庶幾古人租庸調之意。乃令以舊編力差、銀差之數當丁糧之數,難易輕重酌其中。役以應差,里甲除當復者,論丁糧多少編次先後,曰鼠尾冊,按而徵之。市民商賈家殷足而無田產者,聽自占,以佐銀差。正統初,僉事夏時創行於江西,他省仿行之,役以稍平。

其後諸上供者,官為支解,而官府公私所須,復給所輸銀於坊里長,責其營辦。給不能一二,供者或什伯,甚至無所給,惟計值年里甲祗應夫馬飲食,而里甲病矣。凡均徭,解戶上供為京徭,主納為中官留難,不易中納,往復改貿,率至傾產。其他役苛索之弊,不可毛舉。

明初,令天下貢土所有,有常額,珍奇玩好不與。即須用,編之里甲,出銀以市。顧其目冗碎,奸黠者緣為利孔。又大工營繕,祠官祝釐,資用繁溢。迨至中葉,倭寇交訌,仍歲河決,國用耗殫。於是里甲、均徭,浮於歲額矣。

凡役民,自里甲正辦外,如糧長、解戶、馬船頭、館夫、祗候、弓兵、皂隸、門禁、廚斗為常役。後又有斫薪、抬柴、修河、修倉、運料、接遞、站鋪、插淺夫之類,因事編僉,歲有增益。嘉、隆後,行一條鞭法,通計一省丁糧,均派一省徭役。於是均徭、里甲與兩稅為一,小民得無擾,而事亦易集。然糧長、里長,名罷實存,諸役卒至,復僉農氓。條鞭法行十餘年,規制頓紊,不能盡遵也。天啟時,御史李應昇疏陳十害,其三條切言馬夫、河役、糧甲、修辦、白役擾民之弊。崇禎三年,河南巡撫范景文言:「民所患苦,莫如差役。錢糧有收戶、解戶、驛遞有馬戶,供應有行戶,皆僉有力之家充之,名曰大戶。究之,所僉非富民,中人之產輒為之傾。自變為條鞭法,以境內之役均於境內之糧,宜少蘇矣,乃民間仍歲奔走,罄資津貼,是條鞭行而大戶未嘗革也。」時給事中劉懋復奏裁驛夫,徵調往來,仍責編戶。驛夫無所得食,至相率從流賊為亂云。

凡軍、匠、灶戶,役皆永充。軍戶死若逃者,於原籍勾補。匠戶二等:曰住坐,曰輪班。住坐之匠,月上工十日。不赴班者,輸罰班銀月六錢,故謂之輸班。監局中官,多占匠役,又括充幼匠,動以千計,死若逃者,勾補如軍。灶戶有上、中、下三等。每一正丁,貼以餘丁。上、中戶丁力多,或貼二三丁,下戶概予優免。他如陵戶、園戶、海戶、廟戶、幡夫、庫役,瑣末不可勝計。

明初,工役之繁,自營建兩京宗廟、宮殿、闕門、王邸,採木、陶甓,工匠造作,以萬萬計。所在築城、浚陂,百役具舉。迄於洪、宣,郊壇、倉庾猶未迄工。正統、天順之際,三殿、兩宮、南內、離宮,次第興建。弘治時,大學士劉吉言:「近年工役,俱摘發京營軍士,內外軍官禁不得估工用大小多寡。本用五千人,奏請至一二萬,無所稽核。」禮部尚書倪岳言:「諸役費動以數十萬計,水旱相仍,乞少停止。」南京禮部尚書童軒復陳工役之苦。吏部尚書林瀚亦言:「兩畿頻年凶災,困於百役,窮愁怨嘆。山、陜供億軍興,雲南、廣東西徵發剿叛。山東、河南、湖廣、四川、江西興造王邸,財力不贍。浙江、福建辦物料,視舊日增多。庫藏空匱,不可不慮。」帝皆納其言,然不能盡從也。武宗時,乾清宮役尤大。以太素殿初制樸儉,改作雕峻,用銀至二千萬餘兩,役工匠三千餘人,歲支工食米萬三千餘石。又修凝翠、昭和、崇智、光霽諸殿,御馬臨、鐘鼓司、南城豹房新房、火藥庫皆鼎新之。權倖閹宦莊園祠墓香火寺觀,工部復竊官銀以媚焉。給事中張原言:「工匠養父母妻子,尺籍之兵禦外侮,京營之軍衛王室,今奈何令民無所賴,兵不麗伍,利歸私門,怨叢公室乎?」疏入,謫貴州新添驛丞。世宗營建最繁,十五年以前,名為汰省,而經費已六七百萬。其後增十數倍,齋宮、秘殿並時而興。工場二三十處,役匠數萬人,軍稱之,歲費二三百萬。其時宗廟、萬壽宮災,帝不之省,營繕益急。經費不敷,乃令臣民獻助;獻助不已,復行開納。勞民耗財,視武宗過之。萬曆以後,營建織造,溢經制數倍,加以徵調、開採,民不得少休。迨閹人亂政,建第營墳,僭越亡等,功德私祠遍天下。蓋二百餘年,民力殫殘久矣。其以職役優免者,少者一二丁,多者至十六丁。萬歷時,免田有至二三千者。

至若賦稅蠲免,有恩蠲,有災蠲。太祖之訓,凡四方水旱輒免稅,豐歲無災傷,亦擇地瘠民貧者優免之。凡歲災,盡蠲二稅,且貸以米,甚者賜米布若鈔。又設預備倉,令老人運鈔易米以儲粟。荊、蘄水災,命戶部主事趙乾往振,遷延半載,怒而誅之。青州旱蝗,有司不以聞,逮治其官吏。旱傷州縣,有司不奏,許耆民申訴,處以極刑。孝感饑,其令請以預備倉振貸,帝命行人馳驛往,且諭戶部:自今凡歲饑,先發倉庾以貸,然後聞,著為令。」在位三十餘年,賜予布鈔數百萬,米百餘萬,所蠲租稅無數。成祖聞河南饑,有司匿不以聞,逮沼之。因命都御史陳瑛榜諭天下,有司水旱災傷不以聞者,罪不宥。又敕朝廷歲遣巡視官,目擊民艱不言者,悉逮下獄。仁宗監國時,有以發振請者,遣人馳諭之,言:「軍民困乏,待哺嗷嗷,尚從容啟請待報,不能效漢汲黯耶?」宣宗時,戶部請核饑民。帝曰:「民饑無食,濟之當如拯溺救焚,奚待勘。」蓋二祖、仁、宣時,仁政亟行。預備倉之外,又時時截起運,賜內帑。被災處無儲粟者,發旁縣米振之。蝗蝻始生,必遣人捕枌。鬻子女者,官為收贖。且令富人蠲佃戶租。大戶貸貧民粟,免其雜役為息,豐年償之。皇莊、湖泊皆馳禁,聽民採取。饑民還籍,給以口糧。京、通倉米,平價出糶。兼預給俸糧以殺米價,建官舍以處流民,給糧以收棄嬰,養濟院窮民各注籍,無籍者收養蠟燭、幡竿二寺。其恤民如此。世宗、神宗於民事略矣,而災荒疏至,必賜蠲振,不敢違祖制也。

振米之法,明初,大口六斗,小口三斗,五歲以下不與。永樂以後,減其數。

納米振濟贖罪者,景帝時,雜犯死罪六十石,流徒減三之一,餘遞減有差。捐納事例,自憲宗始。生員納米百石以上,人國子監;軍民納二百五十石,為正九品散官,加五十石,增二級,至正七品止。武宗時,富民納粟振濟,千石以上者表其門,九百石至二三百石者,授散官,得至從六品。世宗令義民出穀二十石者,給冠帶,多者授官正七品,至五百石者,有司為立坊。

振粥之法,自世宗始。

報災之法,洪武時不拘時限。弘治中,始限夏災不得過五月終,秋災不得過九月終。萬曆時,又分近地五月、七月,邊地七月、九月。

洪武時,勘災既實,盡與蠲免。弘治中,始定全災免七分,自九分災以下遞減。又止免存留,不及起運,後遂為永制云。

○漕運倉庫

歷代以來,漕粟所都,給官府廩食,各視道里遠近以為準。太祖都金陵,四方貢賦,由江以達京師,道近而易。自成祖遷燕,道里遼遠,法凡三變。初支運,次兌運、支運相參,至支運悉變為長運而制定。

洪武元年北伐,命浙江、江西及蘇州等九府,運糧三百萬石於汴梁。已而大將軍徐達令忻、崞、代、堅、臺五州運糧大同。中書省符下山東行省,募水工發萊州洋海倉餉永平衛。其後海運餉北平、遼東為定制。其西北邊則浚開封漕河餉陜西,自陜西轉餉寧夏、河州。其西南令川、貴納米中鹽,以省遠運。於時各路皆就近輸,得利便矣。

永樂元年納戶部尚書郁新言,始用淮船受三百石以上者,道淮及沙河抵陳州潁岐口跌坡,別以巨舟入黃河抵八柳樹,車運赴衛河輸北平,與海運相參。時駕數臨幸,百費仰給,不止餉邊也。淮、海運道凡二,而臨清倉儲河南、山東粟,亦以輸北平,合而計之為三運。惟海運用官軍,其餘則皆民運云。

自濬會通河,帝命都督賈義、尚書宋禮以舟師運。禮以海船大者千石,工窳輒敗,乃造淺船五百艘,運淮、揚、徐、兗糧百萬,以當海運之數。平江伯陳瑄繼之,頗增至三千餘艘。時淮、徐、臨清、德州各有倉。江西、湖廣、浙江民運糧至淮安倉,分遣官軍就近輓運。自淮至徐以浙、直軍,自徐至德以京衛軍,自德至通以山東、河南軍。以次遞運,歲凡四次,可三百萬餘石,名曰支運。支運之法,支者,不必出當年之民納;納者,不必供當年之軍支。通數年以為裒益,期不失常額而止。由是海陸二運皆罷,惟存遮洋船,每歲於河南、山東、小灘等水次,兌糧三十萬石,十二輸天津,十八由直沽入海輸薊州而已。不數年,官軍多所調遣,遂復民運,道遠數愆期。

宣德四年,瑄及尚書黃福建議復支運法,乃令江西、湖廣、浙江民運百五十萬石於淮安倉,蘇、松、寧、池、廬、安、廣德民運糧二百七十四萬石於徐州倉,應天、常、鎮、淮、揚、鳳、太、滁、和、徐民運糧二百二十萬石於臨清倉,令官軍接運入京、通二倉。民糧既就近入倉,力大減省,乃量地近遠,糧多寡,抽民船十一或十三、五之一以給官軍。惟山東、河南、北直隸則徑赴京倉,不用支運。尋令南陽、懷慶、汝寧糧運臨清倉,開封、彰德、衛輝糧運德州倉,其後山東、河南皆運德州倉。

六年,瑄言:「江南民運糧諸倉,往返幾一年,誤農業。令民運至淮安、瓜洲,兌與衛所。官軍運載至北,給與路費耗米,則軍民兩便。」是為兌運。命群臣會議。吏部蹇義等上官軍兌運民糧加耗則例,以地遠近為差。每石,湖廣八斗,江西、浙江七斗,南直隸六斗,北直隸五斗。民有運至淮安兌與軍運者,止加四斗,如有兌運不盡,仍令民自運赴諸倉,不願兌者,亦聽其自運。軍既加耗,又給輕齎銀為洪閘盤撥之費,且得附載他物,皆樂從事,而民亦多以遠運為艱。於是兌運者多,而支運者少矣。軍與民兌米,往往恃強勒索。帝知其弊,敕戶部委正官監臨,不許私兌。已而頗減加耗米,遠者不過六斗,近者至二斗五升。以三分為率,二分與米,一分以他物準。正糧斛面銳,耗糧俱平概。運糧四百萬石,京倉貯十四,通倉貯十六。臨、徐、淮三倉各遣御史監收。

正統初,運糧之數四百五十萬石,而兌運者二百八十萬餘石,淮、徐、臨、德四倉支運者十之三四耳。土木之變,復盡留山東、直隸軍操備。蘇、松諸府運糧仍屬民。景泰六年,瓦剌入貢,乃復軍運。天順末,兌運法行久,倉入覬耗餘,入庾率兌斛面,且求多索,軍困甚。憲宗即位,漕運參將袁佑上言便宜。帝曰:「律令明言,收糧令納戶平準,石加耗不過五升。今運軍願明加,則倉吏侵害過多可知。今後令軍自概,每石加耗五升,毋溢,勒索者治罪。」後從督倉中官言,加耗至八升。久之,復溢收如故,屢禁不能止也。

初,運糧京師,未有定額。成化八年始定四百萬石,自後以為常。北糧七十五萬五千六百石,南糧三百二十四萬四千四百石,其內兌運者三百三十萬石,由支運改兌者七十萬石。兌運之中,湖廣、山東、河南折色十七萬七千七百石。通計兌運、改兌加以耗米入京、通兩倉者,凡五百十八萬九千七百石。而南直隸正糧獨百八十萬,薊州一府七十萬,加耗在外。浙賦視蘇減數萬。江西、湖廣又殺焉。天津、蘇州、密雲、昌平,共給米六十四萬餘石,悉支兌運米。而臨、德二倉,貯預備米十九萬餘石,取山東、河南改兌米充之。遇災傷,則撥二倉米以補運,務足四百萬之額,不令缺也。

至成化七年,乃有改兌之議。時應天巡撫滕昭令運軍赴江南水次交兌,加耗外,復石增米一斗為渡江費。後數年,帝乃命淮、徐、臨、德四倉支運七十萬石之米,悉改水次交兌。由是悉變為改兌,而官軍長運遂為定制。然是時,司倉者多苛取,甚至有額外罰,運軍展轉稱貸不支。弘治元年,都御史馬文升疏論運軍之苦,言:「各直省運船,皆工部給價,令有司監造。近者,漕運總兵以價不時給,請領價自造。而部臣慮軍士不加愛護,議令本部出料四分,軍衛任三分,舊船抵三分。軍衛無從措辦,皆軍士賣資產、鬻男女以供之,以造船之苦也。正軍逃亡數多,而額數不減,俱以餘丁充之,一戶有三、四人應役者。春兌秋歸,艱辛萬狀。船至張家灣,又雇車盤撥,多稱貸以濟用,此往來之苦也。其所稱貸,運官因以侵漁,責償倍息。而軍士或自載土產以易薪米,又格於禁例,多被掠奪。今宜加造船費每艘銀二十兩,而禁約運官及有司科害搜檢之弊,庶軍困少蘇。」詔從其議。五年,戶部尚書葉淇言:「蘇、松諸府,連歲荒歉,民買漕米,每石銀二兩。而北直隸、山東、河南歲供宣、大二邊糧料,每石亦銀一兩。去歲,蘇州兌運已折五十萬石,每石銀一兩。今請推行於諸府,而稍差其直。災重者,石七錢,稍輕者,石仍一兩。俱解部轉發各邊,抵北直隸三處歲供之數,而收三處本色以輸京倉,則費省而事易集。」從之。自後歲災,輒權宜折銀,以水次倉支運之糧充其數,而折價以六七錢為率,無復至一兩者。

先是,成化間行長運之法。江南州縣運糧至南京,令官軍就水次兌支,計省加耗輸輓之費,得餘米十萬石有奇,貯預備倉以資緩急之用。至是,巡撫都御史以兌支有弊,請令如舊上倉而後放支。戶部言:「兌支法善,不可易。」詔從部議,以所餘就貯各衛倉,作正支銷。又從戶部言,山東改兌糧九萬石,仍聽民自運臨、德二倉,令官軍支運。正德二年,漕運官請疏通水次倉儲,言:「往時民運至淮、徐、臨、德四倉,以待衛軍支運,後改附近州縣水次交兌。已而并支運七十萬石亦令改兌。但七十萬石之外,猶有交兌不盡者,民仍運赴四倉,久無支銷,以致陳腐。請將浙江、江西、湖廣正兌糧米三十五萬石,折銀解京,而令三省衛軍赴臨、德等倉,支運如所折之數。則諸倉米不腐,三省漕卒便於支運。歲漕額外,又得三十五萬折銀,一舉而數善具矣。」帝命部臣議,如其請。六年,戶部侍郎邵寶以漕運遲滯,請復支運法。戶部議,支運法廢久,不可卒復,事遂寢。

臨、德二倉之貯米也,凡十九萬,計十年得百九十萬。自世宗初,災傷撥補日多,而山東、河南以歲歉,數請輕減,且二倉囤積多朽腐。於是改折之議屢興,而倉儲漸耗矣。嘉靖元年,漕運總兵楊宏,請以輕齎銀聽運官道支,為顧僦舟車之費,不必裝鞘印封,計算羨餘,以苦漕卒。給事、御史交駁之。戶部言:「科道官之論,主於防奸,是也。但輕齎本資轉般費,今慮官軍侵耗,盡取其贏餘以歸太倉,則以腳價為正糧,非立法初意也。」乃議運船至通州,巡倉御史核驗,酌量支用實數,著為定規。有羨餘,不輸太倉,即用以修船,官旗漁蠹者重罪。輕齎銀者,憲宗以諸倉改兌,給路費,始各有耗米;兌運米,俱一平一銳,故有銳米;自隨船給運四斗外,餘折銀,謂之輕齎。凡四十四萬五千餘兩。後頗入太倉矣。隆慶中,運道艱阻,議者欲開膠萊河,復海運。由淮安清江浦口,歷新壩、馬家壕至海倉口,徑抵直沽,止循海套,不泛大洋。疏上,遣官勘報,以水多沙磧而止。

神宗時,漕運總督舒應龍言:「國家兩都並建,淮、徐、臨、德,實南北咽喉。自兌運久行,臨、德尚有歲積,而淮、徐二倉無粒米。請自今山東、河南全熟時,盡徵本色上倉。計臨、德已足五十餘萬,則令納於二倉,亦積五十萬石而止。」從之。當是時,折銀漸多。萬曆三十年,漕運抵京,僅百三十八萬餘石。而撫臣議載留漕米以濟河工,倉場侍郎趙世卿爭之,言:「太倉入不當出,計二年後,六軍萬姓將待新漕舉炊,倘輸納愆期,不復有京師矣。」蓋災傷折銀,本折漕糧以抵京軍月俸。其時混支以給邊餉,遂致銀米兩空,故世卿爭之。自後倉儲漸匱,漕政亦益馳。迨於啟、禎,天下蕭然煩費,歲供愈不足支矣。

運船之數,永樂至景泰,大小無定,為數至多。天順以後,定船萬一千七百七十,官軍十二萬人。許令附載土宜,免徵稅鈔。孝宗時限十石,神宗時至六十石。

憲宗立運船至京期限,北直隸、河南、山東五月初一日,南直隸七月初一日,其過江支兌者,展一月,浙江、江西、湖廣九月初一日。通計三年考成,違限者,運官降罰。武宗列水程圖格,按日次填行止站地,違限之米,頓德州諸倉,曰寄囤。世宗定過淮程限,江北十二月者,江南正月,湖廣、浙江、江西三月,神宗時改為二月。又改至京限五月者,縮一月,七八九月者,遞縮兩月。後又通縮一月。神宗初,定十月開倉,十一月兌竣,大縣限船到十日,小縣五日。十二月開幫,二月過淮,三月過洪入閘。皆先期以樣米呈戶部,運糧到日,比驗相同乃收。

凡災傷奏請改折者,毋過七月。題議後期及臨時改題者,立案免覆。漂流者,抵換食米。大江漂流為大患,河道為小患;二百石外為大患,二百石內為小患。小患把總勘報,大患具奏,其後不計多寡,概行奏勘矣。

初,船用楠杉,下者乃用松。三年小修,六年大修,十年更造。每船受正耗米四百七十二石。其後船數缺少,一船受米七八百石。附載夾帶日多,所在稽留違限。一遇河決,即有漂流,官軍因之為奸。水次折乾,沿途侵盜,妄稱水火,至有鑿船自沉者。

明初,命武臣督海運,嘗建漕運使,尋罷。成祖以後用御史,又用侍郎、都御史催督,郎中、員外分理,主事督兌,其制不一。景泰二年始設漕運總督於淮安,與總兵、參將同理漕事。漕司領十二總,十二萬軍,與京操十二營軍相準。初,宣宗令運糧總兵官、巡撫、侍郎歲八月赴京,會議明年漕運事宜,及設漕運總督,則并令總督赴京。至萬曆十八年後始免。凡歲正月,總漕巡揚州,經理瓜、淮過閘。總兵駐徐、邳,督過洪入閘,同理漕參政管押赴京。攢運則有御史、郎中,押運則有參政,監兌、理刑、管洪、管廠、管閘、管泉、監倉則有主事,清江、衛河有提舉。兌畢過淮過洪,巡撫、漕司、河道各以職掌奏報。有司米不備,軍衛船不備,過淮誤期者,責在巡撫。米具船備,不即驗放,非河梗而壓幫停泊,過洪誤期因而漂凍者,責在漕司。船糧依限,河渠淤淺,疏濬無法,閘坐啟閉失時,不得過洪抵灣者,責在河道。

明初,於漕政每加優恤,仁、宣禁役漕舟,宥遲運者。英宗時始扣口糧均攤,而運軍不守法度為民害。自後漕政日馳,軍以耗米易私物,道售稽程。比至,反買倉米補納,多不足數。而糧長率攙沙水於米中,河南、山東尤甚,往往蒸濕浥爛不可食。權要貸運軍銀以罔取利,至請撥關稅給船料以取償。漕運把總率由賄得。倉場額外科取,歲至十四萬。世宗初政,諸弊多釐革,然漂流、違限二弊,日以滋甚。中葉以後,益不可究詰矣。

漕糧之外,蘇、松、常、嘉、湖五府,輸運內府白熟粳糯米十七萬四十餘石,內折色八千餘石,各府部糙粳米四萬四千餘石,內折色八千八百餘石,令民運。謂之白糧船。自長運法行,糧皆軍運,而白糧民運如故。穆宗時,陸樹德言:「軍運以充軍儲,民運以充官祿。人知軍運之苦,不知民運尤苦也。船戶之求索,運軍之欺陵,洪閘之守候,入京入倉,厥弊百出。嘉靖初,民運尚有保全之家,十年後無不破矣。以白糧令軍帶運甚便。」疏入,下部議。不從。

凡諸倉應輸者有定數,其或改撥他鎮者,水次應兌漕糧,即令坐派鎮軍領兌者給價,州縣官督車戶運至遠倉,或給軍價就令關支者,通謂之穵運。九邊之地,輸糧大率以車,宣德時,餉開平亦然,而蘭、甘、松潘,往往使民背負。永樂中,又嘗令廣東海運二十萬石給交址云。

明初,京衛有軍儲倉。洪武三年增置至二十所,且建臨濠、臨清二倉以供轉運。各行省有倉,官吏俸取給焉。邊境有倉,收屯田所入以給軍。州縣則設預備倉,東南西北四所,以振凶荒。自鈔法行,頗有省革。二十四年儲糧十六萬石於臨清,以給訓練騎兵。二十八年置皇城四門倉,儲糧給守禦軍。增京師諸衛倉凡四十一。又設北平、密雲諸縣倉,儲糧以資北征。永樂中,置天津及通州左衛倉,且設北京三十七衛倉。益令天下府縣多設倉儲,預備倉之在四鄉者移置城內。迨會通河成,始設倉於徐州、淮安、德州,而臨清因洪武之舊,并天津倉凡五,謂之水次倉,以資轉運。既,又移德州倉於臨清之永清壩,設武清衛倉於河西務,設通州衛倉於張家灣。宣德中,增造臨清倉,容三百萬石。增置北京及通州倉。京倉以御史、戶部官、錦衣千百戶季更巡察。外倉則布政、按察、都司關防之。各倉門,以致仕武官二,率老幼軍丁十人守之,半年一更。英宗初,命廷臣集議,天下司府州縣,有倉者以衛所倉屬之,無倉者以衛所改隸。惟遼東、甘肅、寧夏、萬全及沿海衛所,無府州縣者仍其舊。正統中,增置京衛倉凡七。自兌運法行,諸倉支運者少,而京、通倉不能容,乃毀臨清、德州、河西務倉三分之一,改為京、通倉。景泰初,移武清衛諸倉於通州。成化初,廢臨、德預備倉在城外者,而以城內空廒儲預備米。名臨清者曰常盈,德州者曰常豐。凡京倉五十有六,通倉十有六。直省府州縣、籓府、邊隘、堡站、衛所屯戍皆有倉,少者一二,多者二三十云。

預備倉之設也,太祖選耆民運鈔糴米,以備振濟,即令掌之。天下州縣多所儲蓄,後漸廢馳。於謙撫河南、山西,修其政。周忱撫南畿,別立濟農倉。他人不能也。正統時,重侵盜之罪,至僉妻充軍。且定納穀千五百石者,敕獎為義民,免本戶雜役。凡振饑米一石,俟有年,納稻穀二石五斗還官。弘治三年限州縣十里以下積萬五千石,二十里積二萬石;衛千戶所萬五千石,百戶所三百石。考滿之日,稽其多寡以為殿最。不及三分者奪俸,六分以上降調。十八年令贖罪贓罰,皆糴穀入倉。正德中,令囚納紙者,以其八折米入倉。軍官有犯者,納穀準立功。初,預備倉皆設倉官,至是革,令州縣官及管糧倉官領其事。嘉靖初,諭德顧鼎臣言:「成、弘時,每年以存留餘米入預備倉,緩急有備。今秋糧僅足兌運,預備無粒米。一遇災傷,輒奏留他糧及勸富民借穀,以應故事。乞急復預備倉糧以裕民。」帝乃令有司設法多積米穀,仍仿古常平法,春振貧民,秋成還官,不取其息。府積萬石,州四五千石,縣二三千石為率。既,又定十里以下萬五千石,累而上之,八百里以下至十九萬石。其後積粟盡平糶,以濟貧民,儲積漸減。隆慶時,劇郡無過六千石,小邑止千石。久之數益減,科罰亦益輕。萬歷中,上州郡至三千石止,而小邑或僅百石。有司沿為具文,屢下詔申飭,率以虛數欺罔而已。

弘治中,江西巡撫林俊嘗請建常平及社倉。嘉靖八年乃令各撫、按設社倉。令民二三十家為一社,擇家殷實而有行義者一人為社首,處事公平者一人為社正,能書算者一人為社副,每朔望會集,別戶上中下,出米四斗至一斗有差,斗加耗五合,上戶主其事。年饑,上戶不足者量貸,稔歲還倉。中下戶酌量振給,不還倉。有司造冊送撫、按,歲一察核。倉虛,罰社首出一歲之米。其法頗善,然其後無力行者。

兩京庫藏,先後建設,其制大略相同。內府凡十庫:內承運庫,貯緞匹、金銀、寶玉、齒角、羽毛,而金花銀最大,歲進百萬兩有奇。廣積庫,貯硫黃、硝石。甲字庫,貯布匹、顏料。乙字庫,貯胖襖、戰鞋、軍士裘帽。丙字庫,貯棉花、絲纊。丁字庫,貯銅鐵、獸皮、蘇木。戊字庫,貯甲仗。贓罰庫,貯沒官物。廣惠庫,貯錢鈔。廣盈庫,貯紵絲、紗羅、綾錦、紬絹。六庫皆屬戶部,惟乙字庫屬兵部,戊字、廣積、廣盈庫屬工部。又有天財庫,亦名司鑰庫,貯各衙門管鑰,亦貯錢鈔。供用庫,貯粳稻、熟米及上供物。以上通謂之內庫。其在宮內者,又有內東裕庫、寶藏庫,謂之裏庫。凡裏庫不關於有司。其會歸門、寶善門迤東及南城磁器諸庫,則謂之外庫。若內府諸監司局,神樂堂,犧牲所,太常、光祿寺,國子監,皆各以所掌,收貯應用諸物。太僕則馬價銀歸之。明初,嘗置行用庫於京城及諸府州縣,以收易昏爛之鈔。仁宗時罷。

英宗時,始設太倉庫。初,歲賦不徵金銀,惟坑冶稅有金銀,入內承運庫。其歲賦偶折金銀者,俱送南京供武臣祿。而各邊有緩急,亦取足其中。正統元年改折漕糧,歲以百萬為額,盡解內承運庫,不復送南京。自給武臣祿十餘萬兩外,皆為御用。所謂金花銀也。七年乃設戶部太倉庫。各直省派剩麥米,十庫中綿絲、絹布及馬草、鹽課、關稅,凡折銀者,皆入太倉庫。籍沒家財,變賣田產,追收店錢,援例上納者,亦皆入焉。專以貯銀,故又謂之銀庫。弘治時,內府供應繁多,每收太倉銀入內庫。又置南京銀庫。正德時,內承運庫中官數言內府財用不充,請支太倉銀。戶部執奏不能沮。嘉靖初,內府供應視弘治時,其後乃倍之。初,太倉中庫積銀八百餘萬兩,續收者貯之兩廡,以便支發。而中庫不動,遂以中庫為老庫,兩廡為外庫。及是時,老庫所存者僅百二十萬兩。二十二年特令金花、子粒銀應解內庫者,並送太倉備邊用,然其後復入內庫。三十七年令歲進內庫銀百萬兩外,加預備欽取銀,後又取沒官銀四十萬兩入內庫。隆慶中,數取太倉銀入內庫,承運庫中官至以空扎下戶部取之。廷臣疏諫,皆不聽。又數取光祿太僕銀,工部尚書朱衡極諫,不聽。初,世宗時,太倉所入二百萬兩有奇。至神宗萬曆六年,太倉歲入凡四百五十餘萬兩,而內庫歲供金花銀外,又增買辦銀二十萬兩以為常,後又加內操馬芻料銀七萬餘兩。久之,太倉、光祿、太僕銀,括取幾盡。邊賞首功,向發內庫者,亦取之太僕矣。

凡甲字諸庫,主事偕科道巡視。太倉庫,員外郎、主事領之,而以給事中巡視。嘉靖中,始兩月一報出納之數。時修工部舊庫,名曰節慎庫,以貯礦銀。尚書文明以給工價,帝詰責之,令以他銀補償,自是專以給內用焉。

其在外諸布政司、都司、直省府州縣衛所皆有庫,以貯金銀、錢鈔、絲帛、贓罰諸物。巡按御史三歲一盤查。各運司皆有庫貯銀,歲終,巡鹽御史委官察之。凡府州縣稅課司局、河泊所,歲課、商稅、魚課、引由、契本諸課程,太祖令所司解州縣府司,以至於部,部札之庫,其元封識,不擅發也。至永樂時,始委驗勘,中,方起解;至部復驗,同,乃進納。嘉靖時,建驗試廳,驗中,給進狀寄庫。月逢九,會巡視庫藏科道官,進庫驗收,不堪者駁易。正統十年設通濟庫於通州。世宗時罷。隆慶初,密雲、薊州、昌平諸鎮皆設庫,收貯主客年例、軍門公費及撫賞、修邊銀云。

凡為倉庫害者,莫如中官。內府諸庫監收者,橫索無厭。正德時,台州衛指揮陳良納軍器,稽留八載,至乞食於市。內府收糧,增耗嘗以數倍為率,其患如此。諸倉初不設中官,宣德末,京、通二倉始置總督中官一人,後淮、徐、臨、德諸倉亦置監督,漕輓軍民被其害。世宗用孫交、張孚敬議,撤革諸中官,惟督諸倉者如故。久之,從給事中管懷理言,乃罷之。

初,天下府庫各有存積,邊餉不借支於內,京師不收括於外。成化時,巡鹽御史楊澄始請發各鹽運提舉司贓罰銀入京庫。弘治時,給事中曾昂請以諸布政司公帑積貯征徭羨銀,盡輸太倉。尚書周經力爭之,以為有不足者,以識造、賞賚、齋醮、土木之故,必欲盡括天下財,非藏富於民意也。至劉瑾用事,遂令各省庫藏盡輸京師。世宗時,閩、廣進羨餘,戶部請責他省巡按,歲一奏獻如例。又以太倉庫匱,運南戶部庫銀八十萬兩實之。而戶部條上理財事宜,臨、德二倉積銀二十萬兩,錄以歸太倉。隆慶初,遣四御史分行天下,搜括庫銀。神宗時,御史蕭重望請核府縣歲額銀進部,未報上。千戶何其賢乞敕內官與己督之,帝竟從其請,由是外儲日就耗。至天啟中,用操江巡撫范濟世策,下敕督歲進,收括靡有遺矣。南京內庫頗藏金銀珍寶,魏忠賢矯旨取進,盜竊一空。內外匱竭,遂至於亡。

○鹽法茶法

煮海之利,歷代皆官領之。太祖初起,即立鹽法,置局設官,令商人販鬻,二十取一,以資軍餉。既而倍征之,用胡深言,復初制。丙午歲,始置兩淮鹽官。吳元年置兩浙。洪武初,諸產鹽地次第設官。都轉運鹽使司六:曰兩淮,曰兩浙,曰長蘆,曰山東,曰福建,曰河東。鹽課提舉司七:曰廣東,曰海北,曰四川,曰雲南;雲南提舉司凡四,曰黑鹽井,白鹽井,安寧鹽井,五井。又陜西靈州鹽課司一。

兩淮所轄分司三,曰泰州,曰淮安,曰通州;批驗所二,曰儀真,曰淮安;鹽場三十,各鹽課司一。洪武時,歲辦大引鹽三十五萬二千餘引。弘治時,改辦小引鹽,倍之。萬曆時同。鹽行直隸之應天、寧國、太平、揚州、鳳陽,廬州、安慶、池州、淮安九府,滁、和二州,江西、湖廣二布政司,河南之河南、汝寧、南陽三府及陳州。正統中,貴州亦食淮鹽。成化十八年,湖廣衡州、永州改行海北鹽。正德二年,江西贛州、南安、吉安改行廣東鹽。所輸邊,甘肅、延綏、寧夏、宣府、大同、遼東、固原、山西神池諸堡。上供光祿寺、神宮監、內官監。歲入太倉餘鹽銀六十萬兩。

兩浙所轄分司四,曰嘉興,曰松江,曰寧紹、曰溫台;批驗所四,曰杭州,曰紹興,曰嘉興,曰溫州;鹽場三十五,各鹽課司一。洪武時,歲辦大引鹽二十二萬四百餘引。弘治時,改辦小引鹽,倍之。萬曆時同。鹽行浙江,直隸之松江、蘇州、常州、鎮江、微州五府及廣德州,江西之廣信府。所輸邊,甘肅、延綏、寧夏、固原、山西神池諸堡。歲入太倉餘鹽銀十四萬兩。

明初,置北平河間鹽運司,後改稱河間長蘆。所轄分司二,曰滄州,曰青州;批驗所二,曰長蘆,曰小直沽;鹽場二十四,各鹽課司一。洪武時,歲辦大引鹽六萬三乾一百餘引。弘治時,改辦小引鹽十八萬八百餘引。萬曆時同。鹽行北直隸,河南之彰德、衛輝二府。所輸邊,宣府、大同、薊州。上供郊廟百神祭祀、內府羞膳及給百官有司。歲入太倉餘鹽銀十二萬兩。

山東所轄分司二,曰膠萊,曰濱樂;批驗所一,曰濼口;鹽場十九,各鹽課司一。洪武時,歲辦大引鹽十四萬三千三百餘引。弘治時,改辦小引鹽,倍之。萬歷時,九萬六千一百餘引。鹽行山東,直隸徐、邳、宿三州,河南開封府,後開封改食河東鹽。所輸邊,遼東及山西神池諸堡。歲入太倉餘鹽銀五萬兩。

福建所轄鹽場七,各鹽課司一。洪武時,歲辦大引鹽十萬四千五百餘引。弘治時,增七百餘引。萬曆時,減千引。其引曰依山,曰附海。依山納折色。附海行本色,神宗時亦改折色。鹽行境內。歲入太倉銀二萬二千餘兩。

河東所轄解鹽,初設東場分司於安邑,成祖時,增設西場於解州,尋復並於東。正統六年復置西場分司。弘治二年增置中場分司。洪武時,歲辦小引鹽三十萬四千引。弘治時,增入萬引。萬曆中,又增二十萬引。鹽行陜西之西安、漢中、延安、鳳翔四府,河南之歸德、懷慶、河南、汝寧、南陽五府及汝州,山西之平陽、潞安二府,澤、沁、遼三州。地有兩見者,鹽得兼行。隆慶中,延安改食靈州池鹽。崇禎中,鳳翔、漢中二府亦改食靈州鹽。歲入太倉銀四千餘兩,給宣府鎮及大同代府祿糧,抵補山西民糧銀,共十九萬兩有奇。

陜西靈州有大小鹽池,又有漳縣鹽井、西和鹽井。洪武時,歲辦鹽,西和十三萬一千五百斤有奇,漳縣五十一萬五千六百斤有奇,靈州二百八十六萬七千四百斤有奇。弘治時同。萬曆時,三處共辦千二百五十三萬七千六百餘斤。鹽行陜西之鞏昌、臨洮二府及河州。歲解寧夏、延綏、固原餉銀三萬六千餘兩。

廣東所轄鹽場十四,海北所轄鹽場十五,各鹽課司一。洪武時,歲辦大引鹽,廣東四萬六千八百餘引,海北二萬七千餘引。弘治時,廣東如舊,海北萬九千四百餘引。萬曆時,廣東小引生鹽三萬二百餘引,小引熟鹽三萬四千六百餘引;海北小引正耗鹽一萬二千四百餘引。鹽有生有熟,熟貴生賤。廣東鹽行廣州、肇慶、惠州、韶州、南雄、潮州六府。海北鹽行廣東之雷州、高州、廉州、瓊州四府,湖廣之桂陽、郴二州,廣西之桂林、柳州、梧州、潯州、慶遠、南寧、平樂、太平、思明、鎮安十府,田、龍、泗城、奉議、利五州。歲入太倉鹽課銀萬一千餘兩。

四川鹽井轄鹽課司十七。洪武時,歲辦鹽一千一十二萬七千餘斤。弘治時,辦二千一十七萬六千餘斤。萬曆中,九百八十六萬一千餘斤。鹽行四川之成都、敘州、順慶、保寧、夔州五府,潼川、嘉定、廣安、雅、廣元五州縣。歲解陜西鎮鹽課銀七萬一千餘兩。

雲南黑鹽井轄鹽課司三,白鹽井、安寧鹽井各轄鹽課司一,五井轄鹽課司七。洪武時,歲辦大引鹽萬七千八百餘引。弘治時,各井多寡不一。萬曆時與洪武同。鹽行境內。歲入太倉鹽課銀三萬五千餘兩。

成祖時,嘗設交阯提舉司,其後交阯失,乃罷。遼東鹽場不設官,軍餘煎辦,召商易粟以給軍。凡大引四百斤,小引二百斤。

鹽所產不同:解州之鹽風水所結,寧夏之鹽刮地得之,淮、浙之鹽熬波,川、滇之鹽汲井,閩、粵之鹽積鹵,淮南之鹽煎,淮北之鹽曬,山東之鹽有煎有曬,此其大較也。

有明鹽法,莫善於開中。洪武三年,山西行省言:「大同糧儲,自陵縣運至太和嶺,路遠費煩。請令商人於大同倉入米一石,太原倉入米一石三斗,給淮鹽一小引。商人鬻畢,即以原給引目赴所在官司繳之。如此則轉運費省而邊儲充。」帝從之。召商輸糧而與之鹽,謂之開中。其後各行省邊境,多召商中鹽以為軍儲。鹽法邊計,相輔而行。

四年定中鹽例,輸米臨濠、開封、陳橋、襄陽、安陸、荊州、歸州、大同、太原、孟津、北平、河南府、陳州、北通州諸倉,計道里近遠,自五石至一石有差。先後增減,則例不一,率視時緩急,米直高下,中納者利否。道遠地險,則減而輕之。編置勘合及底簿,發各布政司及都司、衛所。商納糧畢,書所納糧及應支鹽數,齎赴各轉運提舉司照數支鹽。轉運諸司亦有底簿比照,勘合相符,則如數給與。鬻鹽有定所,刊諸銅版,犯私鹽者罪至死,偽造引者如之,鹽與引離,即以私鹽論。

成祖即位,以北京諸衛糧乏,悉停天下中鹽,專於京衛開中。惟雲南金齒衛、楚雄府,四川鹽井衛,陜西甘州衛,開中如故。不數年,京衛糧米充羨,而大軍征安南多費,甘肅軍糧不敷,百姓疲轉運。迨安南新附,餉益難繼,於是諸所復召商中鹽,他邊地復以次及矣。

仁宗立,以鈔法不通,議所以斂之之道。戶部尚書夏原吉請令有鈔之家中鹽,遂定各鹽司中鹽則例,滄州引三百貫,河東、山東半之,福建、廣東百貫。宣德元年停中鈔例。三年,原吉以北京官吏、軍、匠糧餉不支,條上預備策,言:「中鹽舊則太重,商賈少至,請更定之。」乃定每引自二斗五升至一斗五升有差,召商納米北京。戶部尚書郭敦言:「中鹽則例已減,而商來者少,請以十分為率,六分支與納米京倉者,四分支與遼東、永平、山海、甘肅、大同、宣府、萬全已納米者。他處中納悉停之。」又言:「洪武中,中鹽客商年久物故,代支者多虛冒,請按引給鈔十錠。」帝皆從之,而命倍給其鈔。甘肅、寧夏、大同、宣府、獨石、永平道險遠,趨中者少,許寓居官員及軍餘有糧之家納米豆中鹽。

正統三年,寧夏總兵官史昭以邊軍缺馬,而延慶、平涼官吏軍民多養馬,乃奏請納馬中鹽。上馬一匹與鹽百引,次馬八十引。既而定邊諸衛遞增二十引。其後河州中納者,上馬二十五引,中減五引;松潘中納者,上馬三十五引,中減五引。久之,復如初制。中馬之始,驗馬乃掣鹽,既而納銀於官以市馬,銀入布政司,宗祿、屯糧、修邊、振濟展轉支銷,銀盡而馬不至,而邊儲亦自此告匱矣。於是召商中淮、浙、長廬鹽以納之,令甘肅中鹽者,淮鹽十七,浙鹽十三。淮鹽惟納米麥,浙鹽兼收豌豆、青稞。因淮鹽直貴,商多趨之,故令淮、浙兼中也。

明初仍宋、元舊制,所以優恤灶戶者甚厚,給草場以供樵採,堪耕者許開墾,仍免其雜役,又給工本米,引一石。置倉於場,歲撥附近州縣倉儲及兌軍餘米以待給,兼支錢鈔,以米價為準。尋定鈔數,淮、浙引二貫五百文,河間、廣東、海北、山東、福建、四川引二貫。灶戶雜犯死罪以上止予杖,計日煎鹽以贖。後設總催,多朘削灶戶。至正統時,灶戶貧困,逋逃者多,松江所負課六十餘萬。民訴於朝,命直隸巡撫周忱兼理鹽課。忱條上鑄鐵釜、恤鹵丁、選總催、嚴私販四事,且請於每年正課外,帶徵逋課。帝從其請。命分逋課為六,以六載畢徵。

當是時,商人有自永樂中候支鹽,祖孫相代不得者。乃議仿洪武中例,而加鈔錠以償之,願守支者聽。又以商人守支年久,雖減輕開中,少有上納者,議他鹽司如舊制,而淮、浙、長蘆以十分為率,八分給守支商,曰常股,二分收貯於官,曰存積,遇邊警,始召商中納。常股、存積之名由此始。凡中常股者價輕,中存積者價重,然人甚苦守支,爭趨存積,而常股壅矣。景帝時,邊圉多故,存積增至六分。中納邊糧,兼納穀草、秋青草,秋青草三當穀草二。

廣東之鹽,例不出境,商人率市守關吏,越市廣西。巡撫葉盛以為任之則廢法,禁之則病商,請令入米餉邊,乃許出境,公私交利焉。成化初,歲洊災,京儲不足,召商於淮、徐、德州水次倉中鹽。

舊例中鹽,戶部出榜召商,無徑奏者。富人呂銘等託勢要奏中兩淮存積鹽,中旨允之。戶部尚書馬昂不能執正,鹽法之壞自此始。勢豪多攙中,商人既失利,江南、北軍民因造遮洋大船,列械販鹽。乃為重法,私販、窩隱俱論死,家屬徙邊衛,夾帶越境者充軍。然不能遏止也。十九年頗減存積之數,常股七分,而存積三分。然商人樂有見鹽,報中存積者爭至,遂仍增至六分。淮、浙鹽猶不能給,乃配支長廬、山東以給之。一人兼支數處,道遠不及親赴,邊商輒貿引於近地富人。自是有邊商、內商之分。內商之鹽不能速獲,邊商之引又不賤售,報中寢怠,存積之滯遂與常股等。憲宗末年,閹宦竊勢,奏討淮、浙鹽無算,兩淮積欠至五百餘萬引,商引壅滯。

至孝宗時,而買補餘鹽之議興矣。餘鹽者,灶戶正課外所餘之鹽也。

洪武初制,商支鹽有定場,毋許越場買補;勤灶有餘鹽送場司,二百斤為一引,給米一石。其鹽召商開中,不拘資次給與。成化後,令商收買,而勸借米麥以振貧灶。至是清理兩淮鹽法,侍郎李嗣請令商人買餘鹽補官引,而免其勸借,且停各邊開中,俟逋課完日,官為賣鹽,三分價直,二充邊儲,而留其一以補商人未交鹽價。由是以餘鹽補充正課,而鹽法一小變。

明初,各邊開中商人,招民墾種,築臺堡自相保聚,邊方菽粟無甚貴之時。成化間,始有折納銀者,然未嘗著為令也。弘治五年,商人困守支,戶部尚書葉淇請召商納銀運司,類解太倉,分給各邊。每引輸銀三四錢有差,視國初中米直加倍,而商無守支之苦,一時太倉銀累至百餘萬。然赴邊開中之法廢,商屯撤業,菠粟翔貴,邊儲日虛矣。

武宗之初,以鹽法日壞,令大臣王瓊、張憲等分道清理,而慶雲侯周壽、壽寧侯張鶴各令家人奏買長蘆、兩淮鹽引。戶部尚書韓文執不可,中旨許之。織造太監崔杲又奏乞長蘆鹽一萬二千引,戶部以半予之。帝欲全予,大學士劉健等力爭,李東陽語尤切。帝不悅。健等復疏爭,乃從部議。權要開中既多,又許買餘鹽,一引有用至十餘年者。正德二年始申截舊引角之令,立限追繳,而每引增納紙價及振濟米麥。引價重而課壅如故矣。

先是成化初,都御史韓雍於肇慶、梧州、清遠、南雄立抽鹽廠,官鹽一引,抽銀五分,許帶餘鹽四引,引抽銀一錢。都御史秦紘許增帶餘鹽六引,抽銀六錢。及是增至九錢,而不復抽官引。引目積滯,私鹽通行,乃用戶部郎中丁致祥請,復紘舊法。而他處商人夾帶餘鹽,掣割納價,惟多至三百斤者始罪之。

淮、浙、長蘆引鹽,常股四分,以給各邊主兵及工役振濟之需;存積六分,非國家大事,邊境有警,未嘗妄開。開必邊臣奏討,經部覆允,未有商人擅請及專請淮鹽者。弘治間,存積鹽甚多。正德時,權倖遂奏開殘鹽,改存積、常股皆為正課,且皆折銀。邊臣緩急無備,而勢要占中賣窩,價增數倍。商人引納銀八錢,無所獲利,多不願中,課日耗絀。姦黠者夾帶影射,弊端百出。鹽臣承中璫風旨,復列零鹽、所鹽諸目以假之。世宗登極詔,首命裁革。未幾,商人逯俊等夤緣近倖,以增價為名,奏買殘餘等鹽。戶部尚書秦金執不允,帝特令中兩淮額鹽三十萬引於宣府。金言:「姦人占中淮鹽,賣窩罔利,使山東、長蘆等鹽別無搭配,積之無用。虧國用,誤邊儲,莫此為甚。」御史高世魁亦爭之。詔減淮引十萬,分兩浙、長蘆鹽給之。金復言:「宣、大俱重鎮,不宜令姦商自擇便利,但中宣府。」帝可之。已而俊等請以十六人中宣府,十一人中大同,竟從其請。

嘉靖五年從給事中管律奏,乃復常股存積四六分之制。然是時餘鹽盛行,正鹽守支日久,願中者少;餘鹽第領勘合,即時支賣,願中者多。自弘治時以餘鹽補正課,初以償逋課,後令商人納價輸部濟邊。至嘉靖時,延綏用兵,遼左缺餉,盡發兩淮餘鹽七萬九千餘引於二邊開中。自是餘鹽行。其始尚無定額,未幾,兩淮增引一百四十餘萬,每引增餘鹽二百六十五斤。引價,淮南納銀一兩九錢,淮北一兩五錢,又設處置、科罰名色,以苛斂商財。於是正鹽未派,先估餘鹽,商灶俱困。姦黠者藉口官買餘鹽,夾販私煎。法禁無所施,鹽法大壞。

十三年,給事中管懷理言:「鹽法之壞,其弊有六。開中不時,米價騰貴,召糴之難也。勢豪大家,專擅利權,報中之難也。官司科罰,吏胥侵索,輸納之難也。下場挨掣,動以數年,守支之難也。定價太昂,息不償本,取贏之難也。私鹽四出,官鹽不行,市易之難也。有此六難,正課壅矣,而司計者因設餘鹽以佐之。餘鹽利厚,商固樂從,然不以開邊而以解部,雖歲入距萬,無益軍需。嘗考祖宗時,商人中鹽納價甚輕,而灶戶煎鹽工本甚厚,今鹽價十倍於前,而工本不能十一,何以禁私鹽使不行也?故欲通鹽法,必先處餘鹽,欲處餘鹽,必多減正價。大抵正鹽賤,則私販自息。今宜定價,每引正鹽銀五錢,餘鹽二錢五分,不必解赴太倉,俱令開中關支,餘鹽以盡收為度。正鹽價輕,既利於商;餘鹽收盡,又利於灶。未有商灶俱利,而國課不充者也。」事下所司,戶部覆,以為餘鹽銀仍解部如故,而邊餉益虛矣。至二十年,帝以變亂鹽法由餘鹽,敕罷之。淮、浙、長蘆悉復舊法,夾帶者割沒入官,應變賣者以時估為準。御史吳瓊又請各邊中鹽者皆輸本色。然令甫下,吏部尚書許言贊即請復開餘鹽以足邊用。戶部覆從之,餘鹽復行矣。

先是,十六年令兩浙僻邑,官商不行之處,山商每百斤納銀八分,給票行鹽。其後多侵奪正引,官商課缺,引壅二百萬,候掣必五六載。於是有預徵、執抵、季掣之法。預徵者,先期輸課,不得私為去留。執抵者,執現在運鹽水程,復持一引以抵一引。季掣,則以納課先後為序,春不得遲於夏,夏不得超於春也。然票商納稅即掣賣,預徵諸法徒厲引商而已。靈州鹽池,自史昭中馬之議行,邊餉虧缺,甘肅米直石銀五兩,戶部因奏停中馬,召商納米中鹽。

二十七年令開中者止納本色糧草。三十二年令河東以六十二萬引為額,合正餘鹽為一,而革餘鹽名。時都御史王紳、御史黃國用議:兩淮灶戶餘鹽,每引官給銀二錢,以充工本,增收三十五萬引,名為工本鹽。令商人中額鹽二引,帶中工本鹽一引,抵主兵年例十七萬六千兩有奇。從其請。

初,淮鹽歲課七十萬五千引,開邊報中為正鹽,後益餘鹽納銀解部。至是通前額凡一百五萬引,額增三之一。行之數年,積滯無所售,鹽法壅不行。言事者屢陳工本為鹽贅疣。戶部以國用方絀,年例無所出,因之不變。江西故行淮鹽三十九萬引,後南安、贛州、吉安改行廣鹽,惟南昌諸府行淮鹽二十七萬引。既而私販盛行,袁州、臨江、瑞州則私食廣鹽,撫州、建昌私食福鹽。於是淮鹽僅行十六萬引。數年之間,國計大絀。巡撫馬森疏其害,請於峽江縣建橋設關,扼閩、廣要津,盡復淮鹽額,稍增至四十七萬引。未久橋毀,增額二十萬引復除矣。

三十九年,帝欲整鹽法,乃命副都御史鄢懋卿總理淮、浙、山東、長蘆鹽法。懋卿,嚴嵩黨也,苞苴無虛日。兩淮額鹽銀六十一萬有奇,自設工本鹽,增九十萬,懋卿復增之,遂滿百萬。半年一解。又搜括四司殘鹽,共得銀幾二百萬,一時詡為奇功。乃立剋限法,每卒一人,季限獲私鹽有定數;不及數,輒削其人雇役錢。邏卒經歲有不得支一錢者,乃共為私販,以矣大利,甚至劫估舶,誣以鹽盜而執之,流毒遍海濱矣。嵩失勢,巡鹽御史徐爌言:「兩淮鹽法,曰常股,曰存積,曰水鄉,共七十萬引有奇。引二百斤,納銀八分。永樂以後,引納粟二斗五升,下場關支,四散發賣,商人之利亦什五焉。近年,正鹽之外,加以餘鹽;餘鹽之外,又加工本;工本不足,乃有添單;添單不足,又加添引。懋卿趨利目前,不顧其後,是誤國亂政之尤者。方今災荒疊告,鹽場淹沒,若欲取盈百萬,必至逃亡。弦急欲絕,不棘於此。」於是悉罷懋卿所增者。

四十四年,巡鹽御史朱炳如奏罷兩淮工本鹽。自葉淇變法,邊儲多缺。嘉靖八年以後,稍復開中,邊商中引,內商守支。末年,工本鹽行,內商有數年不得掣者,於是不樂買引,而邊商困,因營求告掣河鹽。河鹽者,不上廩囷,在河徑自超掣,易支而獲利捷。河鹽行,則守支存積者愈久,而內商亦困,引價彌賤。於是姦人專以收買邊引為事,名曰囤戶,告掣河鹽,坐規厚利。時復議於正鹽外附帶餘鹽,以抵工本之數,囤戶因得賤賣餘鹽而貴售之,邊商與內商愈困矣。隆慶二年,屯鹽都御史龐尚鵬疏言:「邊商報中,內商守支,事本相須。但內商安坐,邊商遠輸,勞逸不均,故掣河鹽者以惠邊商也。然河鹽既行,淮鹽必滯,內商無所得利,則邊商之引不售。今宜停掣河鹽,但別邊商引價,自見引及起紙關引到司勘合,別為三等,定銀若干。邊商倉鈔已到,內商不得留難。蓋河鹽停則淮鹽速行,引價定則開中自多,邊商內商各得其願矣。」帝從之。四年,御史李學詩議罷官買餘鹽。報可。

是時廣西古田平,巡撫都御史殷正茂請官出資本買廣東鹽,至桂林發賣,七萬餘包可獲利二萬二千有奇。從之。

自嘉靖初,復常股四分,存積六分之制。後因各邊多故,常股、存積並開,淮額歲課七十萬五千餘引,又增各邊新引歲二十萬。萬曆時,以大工搜遠年違沒廢引六十餘萬,胥出課額之外,無正鹽,止令商買補餘鹽。餘鹽久盡,惟計引重科,加煎飛派而已。時兩淮引價餘銀百二十餘萬增至百四十五萬,新引日益,正引日壅。千戶尹英請配賣沒官鹽,可得銀六萬兩。大學士張位等爭之。二十六年,以鴻臚寺主簿田應璧奏,命中官魯保鬻兩淮沒官餘鹽。給事中包見捷極陳利害。不聽。保既視事,遂議開存積鹽。戶部尚書楊俊民言:「明旨核沒官鹽,而存積非沒官也。額外加增,必虧正課。保奏不可從。」御史馬從騁亦爭之。俱不聽。保乃開存積八萬引,引重五百七十斤,越次超掣,壓正鹽不行。商民大擾,而姦人蜂起。董璉、吳應麒等爭言鹽利。山西、福建諸稅監皆領鹽課矣。百戶高時夏奏浙、閩餘鹽歲可變價三十萬兩,巡撫金學會勘奏皆罔。疏入不省。於是福建解銀萬三千兩有奇,浙江解三萬七千兩有奇,借名苛斂,商困引壅。戶部尚書趙世卿指其害由保,因言:「額外多取一分,則正課少一分,而國計愈絀,請悉罷無名浮課。」不報。三十四年夏至明年春,正額逋百餘萬,保亦惶懼,請罷存積引鹽。保尋死。有旨罷之,而引斤不能減矣。

李太后薨,帝用遺誥蠲各運司浮課,商困稍蘇,而舊引壅滯。戶部上鹽法十議,正行見引,附銷積引,以疏通之。巡鹽御史龍遇奇立鹽政綱法,以舊引附見引行,淮南編為十綱,淮北編為十四綱,計十餘年,則舊引盡行。從之。天啟時,言利者恣搜括,務增引超掣。魏忠賢黨郭興治、崔呈秀等,巧立名目以取之,所入無算。論者比之絕流而漁。崇禎中,給事中黃承昊條上鹽政,頗欲有所釐革。是時兵餉方大絀,不能行也。

初,諸王府則就近地支鹽,官民戶口食鹽皆計口納鈔,自行關支。而官吏食鹽多冒增口數,有一官支二千餘斤,一吏支五百餘斤者。乃限吏典不得過十口,文武官不過三十口;大口鈔十二貫支鹽十二斤,小口半之。景泰三年始以鹽折給官吏俸糧,以百四十斤當米一石。京官歲遣吏下場,恣為姦利。錦衣吏益暴,率聯巨艦私販,有司不能詰。巡鹽御史乃定百司食鹽數,攟束以給吏,禁毋下場。納鈔、僦輓,費無所出,吏多亡。嘉靖中,吏部郎中陸光祖言於尚書嚴訥,疏請革之。自後百司停支食鹽,惟戶部及十三道御史歲支如故。軍民計口納鈔者,浙江月納米三升,賣鹽一斤,而商賈持鹽赴官,官為斂散,追徵之急過於租賦。正統時,從給事中鮑輝言,令民自買食鹽於商,罷納米令,且鬻十斤以下者勿以私鹽論,而鹽鈔不除。後條鞭法行,遂編入正賦。

巡鹽之官,洪、永時,嘗一再命御史視鹽課。正統元年始命侍郎何文淵、王佐,副都御史朱與言提督兩淮、長蘆、兩浙鹽課,命中官御史同往。未幾,以鹽法已清,下敕召還。後遂令御史視鹺,依巡按例,歲更代以為常。十一年以山東諸鹽場隸長蘆巡鹽御史。十四年命副都御史耿九疇清理兩淮鹽法。成化中,特遣中官王允中、僉都御史高明整治兩淮鹽法。明請增設副使一人,判官二人。孝宗初,鹽法壞,戶部尚書李敏請簡風憲大臣清理,乃命戶部侍郎李嗣於兩淮,刑部侍郎彭韶於兩浙,俱兼都御史,賜敕遣之。弘治十四年,僉都御史王璟督理兩淮鹽法。正德二年,兩淮則僉都御史王瓊,閩、浙則僉都御史張憲。後惟兩淮賦重,時遣大臣。十年,則刑部侍郎藍章。嘉靖七年,則副都御史黃臣。三十二年,則副都御史王紳。至三十九年,特命副都御史鄢懋卿總理四運司,事權尤重。自隆慶二年,副都御史龐尚鵬總理兩淮、長蘆、山東三運司後,遂無特遣大臣之事。

番人嗜乳酪,不得茶,則困以病。故唐、宋以來,行以茶易馬法,用制羌、戎,而明制尤密。有官茶,有商茶,皆貯邊易馬。官茶間徵課鈔,商茶輸課略如鹽制。

初,太祖令商人於產茶地買茶,納錢請引。引茶百斤,輸錢二百,不及引曰畸零,別置由帖給之。無由、引及茶引相離者,人得告捕。置茶局批驗所,稱較茶引不相當,即為私茶。凡犯私茶者,與私鹽同罪。私茶出境,與關隘不譏者,並論死。後又定茶引一道,輸錢千,照茶百斤;茶由一道,輸錢六百,照茶六十斤。既,又令納鈔,每引由一道,納鈔一貫。

洪武初,定令:凡賣茶之地,令宣課司三十取一。四年,戶部言:「陜西漢中、金州、石泉、漢陰、平利、西鄉諸縣,茶園四十五頃,茶八十六萬餘株。四川巴茶三百十五戶,茶二百三十八萬餘株。宜定令每十株官取其一。無主茶園,令軍士薅采,十取其八,以易番馬。」從之。於是諸產茶地設茶課司,定稅額,陜西二萬六千斤有奇,四川一百萬斤。設茶馬司於秦、洮、河、雅渚州,自碉門、黎、雅抵朵甘、烏思藏,行茶之地五千餘里。山後歸德諸州,西方諸部落,無不以馬售者。

碉門、永寧、筠、連所產茶,名曰剪刀粗葉,惟西番用之,而商販未嘗出境。四川茶鹽都轉運使言:「宜別立茶局,徵其稅,易紅纓、氈衫、米、布、椒、蠟以資國用。而居民所收之茶,依江南給引販賣法,公私兩便。」於是永寧、成都、筠、連皆設茶局矣。

川人故以茶易毛布、毛纓諸物以償茶課。自定課額,立倉收貯,專用以市馬,民不敢私採,課額每虧,民多賠納。四川布政司以為言,乃聽民採摘,與番易貨。又詔天全六番司民,免其徭役,專令蒸烏茶易馬。

初制,長河西等番商以馬入雅州易茶,由四川嚴州衛入黎州始達。茶馬司定價,馬一匹,茶千八百斤,於碉門茶課司給之。番商往復迂遠,而給茶太多。嚴州衛以為言,請置茶馬司於嚴州,而改貯碉門茶於其地,且驗馬高下以為茶數。詔茶馬司仍舊,而定上馬一匹,給茶百二十斤,中七十斤,駒五十斤。

三十年改設秦州茶馬司於西寧,敕右軍都督曰:「近者私茶出境,互市者少,馬日貴而茶日賤,啟番人玩侮之心。檄秦、蜀二府,發都司官軍於松潘、碉門、黎、雅、河州、臨洮及入西番關口外,巡禁私茶之出境者。」又遣駙馬都尉謝達諭蜀王椿曰:「國家榷茶,本資易馬。邊吏失譏,私販出境,惟易紅纓雜物。使番人坐收其利,而馬入中國者少,豈所以制戎狄哉!爾其諭布政司、都司,嚴為防禁,毋致失利。」

當是時,帝綢繆邊防,用茶易馬,固番人心,且以強中國。嘗謂戶部尚書郁新:「用陜西漢中茶三百萬斤,可得馬三萬匹,四川松、茂茶如之。,販鬻之禁,不可不嚴。」以故遣僉都御史鄧文鏗等察川、陜私茶;駙馬都尉歐陽倫以私茶坐死。又製金牌信符,命曹國公李景隆齎入番,與諸番要約,篆文上曰「皇帝聖旨」,左曰「合當差發」,右曰「不信者斬」。凡四十一面:洮州火把藏思囊日等族,牌四面,納馬三千五十匹;河州必里衛西番二十九族,牌二十一面,納馬七千七百五匹;西寧曲先、阿端、罕東、安定四衛,巴哇、申中、申藏等族,牌十六面,納馬三千五十匹。下號金牌降諸番,上號藏內府以為契,三歲一遣官合符。其通道有二,一出河州,一出碉門,運茶五十餘萬斤,獲馬萬三千八百匹。太祖之馭番如此。

永樂中,帝懷柔遠人,遞增茶斤。由是市馬者多,而茶不足。茶禁亦稍馳,多私出境。碉門茶馬司至用茶八萬餘斤,僅易馬七十匹,又多瘦損。乃申嚴茶禁,設洮州茶馬司,又設甘肅茶馬司於陜西行都司地。十三年特遣三御史巡督陜西茶馬。

太祖之禁私茶也,自三月至九月,月遣行人四員,巡視河州、臨洮、碉門、黎、雅。半年以內,遣二十四員,往來旁午。宣德十年,乃定三月一遣。自永樂時停止金牌信符,至是復給。未幾,番人為北狄所侵掠,徙居內地,金牌散失。而茶司亦以茶少,止以漢中茶易馬,且不給金牌,聽其以馬入貢而已。

先是,洪武末,置成都、重慶、保寧、播州茶倉四所,令商人納米中茶。宣德中,定官茶百斤,加耗什一。中茶者,自遣人赴甘州、西寧,而支鹽於淮、浙以償費。商人恃文憑恣私販,官課數年不完。正統初,都御史羅亨信言其弊,乃罷運茶支鹽例,令官運如故,以京官總理之。

景泰中,罷遣行人。成化三年命御史巡茶陜西。番人不樂御史,馬至日少。乃取回御史,仍遣行人,且令按察司巡察。已而巡察不專,兵部言其害,乃復遣御史,歲一更,著為令。又以歲饑待振,復令商納粟中茶,且令茶百斤折銀五錢。商課折色自此始。

弘治三年,御史李鸞言:「茶馬司所積漸少,各邊馬耗,而陜西諸郡歲稔,無事易粟。請於西寧、河西、洮州三茶馬司召商中茶,每引不過百斤,每商不過三十引,官收其十之四,餘者始令貨賣,可得茶四十萬斤,易馬四千匹,數足而止。」從之。十二年,御史王憲又言:「自中茶禁開,遂令私茶莫遏,而易馬不利。請停糧茶之例。異時或兵荒,乃更圖之。」部覆從其請。四川茶課司舊徵數十萬斤易馬。永樂以後,番馬悉由陜西道,川茶多浥爛。乃令以三分為率,一分收本色,二分折銀,糧茶停二年。延綏饑,復召商納糧草,中四百萬斤。尋以御史王紹言,復禁止,並罷正額外召商開中之例。

十六年取回御史,以督理馬政都御史楊一清兼理之。一清復議開中,言:「召商買茶,官貿其三之一,每歲茶五六十萬斤,可得馬萬匹。」帝從所請。正德元年,一清又建議,商人不願領價者,以半與商,令自賣。遂著為例永行焉。一清又言金牌信符之制當復,且請復設巡茶御史兼理馬政。乃復遣御史,而金牌以久廢。卒不能復。後武宗寵番僧,許西域人例外帶私茶。自是茶法遂壞。

番人之市馬也,不能辯權衡,止訂篦中馬。篦大,則官虧其直;小,則商病其繁。十年巡茶御史王汝舟酌為中制,每千斤為三百三十篦。

嘉靖三年,御史陳講以商茶低偽,悉徵黑茶,地產有限,乃第茶為上中二品,印烙篦上,書商名而考之。旋定四川茶引五萬道,二萬六千道為腹引,二萬四千道為邊引。芽茶引三錢,葉茶引二錢。中茶至八十萬斤而止,不得太濫。

十五年,御史劉良卿言:「律例:『私茶出境與關隘失察者,並凌遲處死。』蓋西陲籓籬,莫切於諸番。番人恃茶以生,故嚴法以禁之,易馬以酬之,以制番人之死命,壯中國之籓籬,斷匈奴之右臂,非可以常法論也。洪武初例,民間蓄茶不得過一月之用。弘治中,召商中茶,或以備振,或以儲邊,然未嘗禁內地之民使不得食茶也。今減通番之罪,止於充軍。禁內地之茶,使不得食,又使商私課茶,悉聚於三茶馬司。夫茶司與番為鄰,私販易通,而禁復嚴於內郡,是驅民為私販而授之資也。

以故大姦闌出而漏網,小民負升斗而罹法。今計三茶馬司所貯,洮河足三年,西寧足二年,而商、私、課茶又日益增,積久腐爛而無所用。茶法之弊如此。番地多馬而無所市,吾茶有禁而不得通,其勢必相求,而制之之機在我。今茶司居民,竊易番馬以待商販,歲無虛日,及官易時,而馬反耗矣。請敕三茶馬司,止留二年之用,每年易馬當發若干。正茶之外,分毫毋得夾帶。令茶價踴貴,番人受制,良馬將不可勝用。且多開商茶,通行內地,官榷其半以備軍餉,而河、蘭、階、岷諸近番地,禁賣如故,更重通番之刑如律例。洮、岷、河責邊備道,臨洮、蘭州責隴右分巡,西寧責兵備,各選官防守。失察者以罷軟論。」奏上,報可。於是茶法稍飭矣。

御史劉崙、總督尚書王以旂等,請復給諸番金牌信符。兵部議,番族變詐不常,北狄抄掠無已,金牌亟給亟失,殊損國體。番人納馬,意在得茶,嚴私販之禁,則番人自順,雖不給金牌,馬可集也。若私販盛行,吾無以繫其心、制其命,雖給金牌,馬亦不至。乃定議發勘合予之。

其後陜西歲饑,茶戶無所資,頗逋課額。三十六年,戶部以全陜災震,邊餉告急,國用大絀,上言:「先時,正額茶易馬之外,多開中以佐公家,有至五百萬斤者。近者御史劉良卿亦開百萬,後止開正額八十萬斤,並課茶、私茶通計僅九十餘萬。宜下巡茶御史議,召商多中。」御史楊美益言:「歲祲民貧,即正額尚多虧損,安有贏羨。今第宜守每年九十萬斤招番易馬之規。凡通內地以息私販,增開中以備振荒,悉從停罷,毋使與馬分利。」戶部以帑藏方匱,請如弘治六年例,易馬外仍開百萬斤,召納邊鎮以備軍餉。詔從之。末年,御史潘一桂言:「增中商茶頗壅滯,宜裁減十四五。」又言:「松潘與洮、河近,私茶往往闌出,宜停松潘引目,申嚴入番之禁。」皆報可。

四川茶引之分邊腹也,邊茶少而易行,腹茶多而常滯。隆慶三年裁引萬二千,以三萬引屬黎、雅,四千引屬松潘諸邊,四千引留內地,稅銀共萬四千餘兩,解部濟邊以為常。

五年令甘州仿洮、河、西寧事例,歲以六月開中,兩月內中馬八百匹。立賞罰例,商引一二年銷完者賞有差,踰三年者罪之,沒其附帶茶。

萬曆五年,俺答款塞,請開茶市。御史李時成言:「番以茶為命,北狄若得,藉以制番,番必從狄,貽患匪細。部議給百餘篦,而勿許其市易。自劉良卿馳內地之禁,楊美益以為非,其後復禁止。十三年,以西安、鳳翔、漢中不與番鄰,開其禁,招商給引,抽十三入官,餘聽自賣。御史鐘化民以私茶之闌出多也,請分任責成。陜之漢中,關南道督之,府佐一人專駐魚渡壩;川之保寧,川北道督之,府佐一人專駐雞猴壩。率州、縣官兵防守。」從之。

中茶易馬,惟漢中、保寧,而湖南產茶,其直賤,商人率越境私販,中漢中、保寧者,僅一二十引。茶戶欲辦本課,輒私販出邊,番族利私茶之賤,因不肯納馬。二十三年,御史李楠請禁湖茶,言:「湖茶行,茶法、馬政兩弊,宜令巡茶御史召商給引,願報漢、興、保、夔者,準中。越境下湖南者,禁止。且湖南多假茶,食之刺口破腹,番人亦受其害。」既而御史徐僑言:「漢、川茶少而直高,湖南茶多而直下。湖茶之行,無妨漢中。漢茶味甘而薄,湖茶味苦,於酥酪為宜,亦利番也。但宜立法嚴核,以遏假茶。」戶部折衷其議,以漢茶為主,湖茶佐之。各商中引,先給漢、川畢,乃給湖南。如漢引不足,則補以湖引。報可。

二十九年,陜西巡按御史畢三才言:「課茶徵輸,歲有定額。先因茶多餘積,園戶解納艱難,以此改折。今商人絕跡,五司茶空。請令漢中五州縣仍輸本色,每歲招商中五百引,可得馬萬一千九百餘匹。」部議,西寧、河、洮、岷、甘、莊浪六茶司共易馬九千六百匹,著為令。天啟時,增中馬二千四百匹。

明初嚴禁私販,久而姦弊日生。洎乎末造,商人正引之外,多給賞由票,使得私行。番人上駟盡入姦商,茶司所市者乃其中下也。番得茶,叛服自由;而將吏又以私馬竄番馬,冒支上茶。茶法、馬政、邊防於是俱壞矣。

其他產茶之地,南直隸常、盧、池、徽,浙江湖、嚴、衢、紹,江西南昌、饒州、南康、九江、吉安,湖廣武昌、荊州、長沙、寶慶,四川成都、重慶、嘉定、夔、瀘,商人中引則於應天、宜興、杭州三批驗所,徵茶課則於應天之江東瓜埠。自蘇、常、鎮、徽、廣德及浙江、河南、廣西、貴州皆徵鈔,雲南則徵銀。

其上供茶,天下貢額四千有奇,福建建寧所貢最為上品,有探春、先春、次春、紫筍及薦新等號。舊皆採而碾之,壓以銀板,為大小龍團。太祖以其勞民,罷造,惟令採茶芽以進,復上供戶五百家。凡貢茶,第按額以供,不具載。

○錢鈔坑冶附鐵冶銅場商稅市舶馬市

錢幣之興,自九府圜法,歷代遵用。鈔始於唐之飛錢,宋之交會,金之交鈔。元世始終用鈔,錢幾廢矣。

太祖初置寶源局於應天,鑄「大中通寶」錢,與歷代錢兼行。以四百文為一貫,四十文為一兩,四文為一錢。及平陳友諒,命江西行省置貨泉局,頒大中通寶錢,大小五等錢式。即位,頒「洪武通寶」錢,其制凡五等:曰「當十」、「當五」、「當三」、「當二」、「當一」。「當十」錢重一兩,餘遞降至重一錢止。各行省皆設寶泉局,與寶源局並鑄,而嚴私鑄之禁。洪武四年改鑄大中、洪武通寶大錢為小錢。初,寶源局錢鑄「京」字於背,後多不鑄,民間無「京」字者不行,故改鑄小錢以便之。尋令私鑄錢作廢銅送官,償以錢。是時有司責民出銅,民毀器皿輸官,頗以為苦。而商賈沿元之舊習用鈔,多不便用錢。

七年,帝乃設寶鈔提舉司。明年始詔中書省造大明寶鈔,命民間通行。以桑穰為料,其制方,高一尺,廣六寸,質青色,外為龍文花欄。橫題其額曰「大明通行寶鈔」。其內上兩旁,復為篆文八字,曰「大明寶鈔,天下通行」。中圖錢貫,十串為一貫。其下云「中書省奏準印造大明寶鈔與銅錢通行使用,偽造者斬,告捕者賞銀二十五兩,仍給犯人財產。」若五百文則畫錢文為五串,餘如其制而遞減之。其等凡六:曰一貫,曰五百文、四百文、三百文、二百文、一百文。每鈔一貫,準錢千文,銀一兩;四貫準黃金一兩。禁民間不得以金銀物貨交易,違者罪之;以金銀易鈔者聽。遂罷寶源、寶泉局。越二年,復設寶泉局,鑄小錢與鈔兼行,百文以下止用錢。商稅兼收錢鈔,錢三鈔七。十三年,以鈔用久昏爛,立倒鈔法,令所在置行用庫,許軍民商賈以昏鈔納庫易新鈔,量收工墨直。會中書省廢,乃以造鈔屬戶部,鑄錢屬工部,而改寶鈔文「中書省」為「戶部」,與舊鈔兼行。十六年,置戶部寶鈔廣源庫、廣惠庫;入則廣源掌之,出則廣惠掌之。在外衛所軍士,月鹽皆給鈔,各鹽場給工本鈔。十八年,天下有司官祿米皆給鈔,二貫五百文準米一石。

二十二年詔更定錢式:生銅一斤,鑄小錢百六十,折二錢半之,「當三」至「當十」,準是為差。更造小鈔,自十文至五十文。二十四年諭榷稅官吏,凡鈔有字貫可辯者,不問爛損,即收受解京,抑勒與偽充者罪之。二十五年設寶鈔行用庫於東市,凡三庫,各給鈔三萬錠為鈔本,倒收舊鈔送內府。令大明寶鈔與歷代錢兼行,鈔一貫準錢千文,提舉司於三月內印造,十月內止,所造鈔送內府充賞賚。明年罷行用庫,又罷寶泉局。時兩浙、江西、閩、廣民重錢輕鈔,有以錢百六十文折鈔一貫者,由是物價翔貴,而鈔法益壞不行。三十年乃更申交易用金銀之禁。

成祖初,犯者以姦惡論,惟置造首飾器皿,不在禁例。永樂二年詔犯者免死,徙家戍興州。陜西都司僉事張豫,坐抵易官鈔論戍。江夏民父死,以銀營葬具,當戍邊。帝以其迫於治葬,非玩法,特矜宥之。都御史陳瑛言:「比歲鈔法不通,皆緣朝廷出鈔太多,收斂無法,以致物重鈔輕。莫若暫行戶口食鹽法。天下人民不下千萬戶,官軍不下二百萬家,誠令計口納鈔食鹽,可收五千餘萬錠。」帝令戶部會群臣議。大口月食鹽一斤,納鈔一貫,小口半之。從其議。設北京寶鈔提舉司,稅糧課程贓罰俱折收鈔,其直視洪武初減十之九。後又令鹽官納舊鈔支鹽,發南京抽分場積薪、龍江提舉司竹木鬻之軍民,收其鈔。應天歲辦蘆柴,徵鈔十之八。帝初即位,戶部尚書夏原吉請更鈔板篆文為「永樂」。帝命仍其舊。自後終明世皆用洪武年號云。

仁宗監國,令犯笞杖者輸鈔。及即位,以鈔不行詢原吉。原吉言:「鈔多則輕,少則重。民間鈔不行,緣散多斂少,宜為法斂之。請市肆門攤諸稅,度量輕重,加其課程。鈔入官,官取昏軟者悉毀之。自今官鈔宜少出,民間得鈔難,則自然重矣。」乃下令曰:「所增門攤課程,鈔法通,即復舊,金銀布帛交易者,亦暫禁止。」然是時,民卒輕鈔。至宣德初,米一石用鈔五十貫,乃馳布帛米麥交易之禁。凡以金銀交易及匿貨增直者罰鈔,府縣衛所倉糧積至十年以上者,鹽糧悉收鈔,秋糧亦折鈔三分,門攤課鈔增五倍,塌房、店舍月納鈔五百貫,果園、CA車並令納鈔。戶部言民間交易,惟用金銀,鈔滯不行。乃益嚴其禁,交易用銀一錢者,罰鈔千貫,贓吏受銀一兩者,追鈔萬貫,更追免罪鈔如之。

英宗即位,收賦有米麥折銀之令,遂減諸納鈔者,而以米銀錢當鈔,弛用銀之禁。朝野率皆用銀,其小者乃用錢,惟折官俸用鈔,鈔壅不行。十三年復申禁令,阻鈔者追一萬貫,全家戍邊。天順中,始弛其禁。憲宗令內外課程錢鈔兼收,官俸軍餉亦兼支錢鈔。是時鈔一貫不能直錢一文,而計鈔徵之民,則每貫徵銀二分五釐,民以大困。

弘治元年,京城稅課司,順天、山東、河南戶口食鹽,俱收鈔,各鈔關俱錢鈔兼收。其後乃皆改折用銀。而洪武、永樂、宣德錢積不用,詔發之,令與歷代錢兼用。戶部請鼓鑄,乃復開局鑄錢。凡納贖收稅,歷代錢、制錢各收其半;無制錢即收舊錢,二以當一。制錢者,國朝錢也。舊制,工部所鑄錢入太倉、司鑰二庫;諸關稅錢亦入司鑰庫。共貯錢數千百萬,中官掌之,京衛軍秋糧取給焉,每七百當銀一兩。武宗之初,部臣請察核侵蝕;又以錢當俸糧者,僅及銀數三之一,請於承運庫給銀。時中官方用事,皆不聽。已而司鑰庫太監龐慄言:「自弘治間榷關折銀入承運庫,錢鈔缺乏,支放不給,請遵成化舊制,錢鈔兼收。」從之。正德三年,以太倉積錢給官俸,十分為率,錢一銀九。又從太監張永言,發天財庫及戶部布政司庫錢,關給徵收,每七十文徵銀一錢,且申私鑄之禁。嘉靖四年,令宣課分司收稅,鈔一貫折銀三釐,錢七文折銀一分。是時鈔久不行,錢亦大壅,益專用銀矣。

明初鑄洪武錢。成祖九年鑄永樂錢。宣德九年鑄宣德錢。弘治十六年以後,鑄弘治錢。至世宗嘉靖六年,大鑄嘉靖錢。每文重一錢三分,且補鑄累朝未鑄者。三十二年鑄洪武至正德九號錢,每號百萬錠,嘉靖錢千萬錠,一錠五千文。而稅課抽分諸廠,專收嘉靖錢。民患錢少,乃發內庫新舊錢八千一百萬文折給俸糧。又令通行歷代錢,有銷新舊錢及以銅造像製器者,罪比盜鑄。先是,民間行濫惡錢,率以三四十錢當銀一分。後益雜鉛錫,薄劣無形製,至以六七十文當銀一分。翦楮夾其中,不可辨。用給事中李用敬言,以制錢與前代雜錢相兼行,上品者俱七文當銀一分,餘視錢高下為三等,下者二十一文當銀一分;私造濫惡錢悉禁不行,犯者置之法。小錢行久,驟革之,民頗不便。又出內庫錢給文武官俸,不論新舊美惡,悉以七文折算。諸以俸錢市易者,亦悉以七文抑勒予民,民亦騷然。

屬連歲大侵,四方流民就食京師,死者相枕藉。論者謂錢法不通使然。於是御史何廷鈺條奏,請許民用小錢,以六十文當銀一分。戶部執不從。廷鈺訐奏尚書方鈍及郎中劉爾牧。帝怒,斥爾牧,採廷鈺議,命從民便。且定嘉靖錢七文,洪武諸錢十文,前代錢三十文,當銀一分。然諸濫惡小錢,以初禁之嚴,雖奉旨間行,竟不復用,而民間競私鑄嘉靖通寶錢,與官錢並行焉。

給事中殷正茂言:「兩京銅價大高,鑄錢得不償費。宜採雲南銅,運至岳州鼓鑄,費工本銀三十九萬,可得錢六萬五千萬文,直銀九十三萬餘兩,足以少佐國家之急。」戶部覆言:「雲南地僻事簡,即山鼓鑄為便。」乃敕巡撫以鹽課銀二萬兩為工本。未幾,巡撫王昺言費多入少,乞罷鑄。帝以小費不當惜,仍命行之。越數年,巡按王諍復言宜罷鑄。部議:「錢法壅滯者,由宣課司收稅以七文當一分。姦民乘機阻撓,錢多則惡濫相欺,錢少則增直罔利,故禁愈繁而錢愈滯。自今準折聽民便,不必定文數,而課稅及官俸且俱用銀。」乃罷雲南鑄錢,而從戶部議。

時所鑄錢有金背,有火漆,有鏇邊。議者以鑄錢艱難,工匠勞費,革鏇車用鑢鐋。於是鑄工競雜鉛錫便坐刂治,而輪郭粗糲,色澤黯黲。姦偽仿傚,盜鑄日滋,金背錢反阻不行。死罪日報,終不能止。帝患之,問大學士徐階。階陳五害,請停寶源局鑄錢,應支給錢者悉予銀。帝乃鞫治工匠侵料減工罪,而停鼓鑄。自後稅課徵銀而不徵錢。且民間止用制錢,不用古錢,而私鑄者多。

隆慶初,錢法不行,兵部侍郎譚綸言:「欲富民,必重布帛菽粟而賤銀,欲賤銀,必制錢法以濟銀之不足。今錢惟布於天下,而不以輸於上,故其權在市井。請令民得以錢輸官,則錢法自通。」於是課稅銀三兩以下復收錢,民間交易一錢以下止許用錢。時錢八文折銀一分,禁民毋得任意低昂。直隸巡按楊家相請鑄大明通寶錢,不識年號。部議格不行。高拱再相,言:「錢法朝議夕更,迄無成說。小民恐今日得錢,而明日不用,是以愈更愈亂,愈禁愈疑。請一從民便,勿多為制以亂人耳目。」帝深然之。錢法復稍稍通矣。寶鈔不用垂百餘年,課程亦鮮有收鈔者,惟俸錢獨支鈔如故。四年始以新鑄隆慶錢給京官俸云。

萬曆四年命戶工二部,準嘉靖錢式鑄「萬曆通寶」金背及火漆錢,一文重一錢二分五釐,又鑄鏇邊錢,一文重一錢三分,頒行天下,俸糧皆銀錢兼給。雲南巡按郭庭梧言:「國初京師有寶源局,各省有寶泉局,自嘉靖間省局停廢,民用告匱。滇中產銅,不行鼓鑄,而反以重價購海,非利也。」遂開局鑄錢。尋命十三布政司皆開局。採工部言,以五銖錢為準,用四火黃銅鑄金背,二火黃銅鑄火漆,粗惡者罪之。蓋以費多利少則私鑄自息也。久之,戶部言:「錢之輕重不常,輕則斂,重則散,故無壅閼匱乏之患。初鑄時,金背十文直銀一分,今萬曆金背五文,嘉靖金背四文,各直銀一分,火漆鏇邊亦如之。僅踰十年,而輕重不啻相半,錢重而物價騰踴,宜發庫貯以平其直。」從之。時王府皆鑄造私錢,吏不敢訐。古錢阻滯不行,國用不足,乃命南北寶源局拓地增爐鼓鑄。而北錢視南錢昂值三之一,南鑄大抵輕薄。然各循其舊,並行不廢。

天啟元年鑄泰昌錢。兵部尚書王象乾,請鑄當十、當百、當千三等大錢,用龍文,略仿白金三品之制,於是兩京皆鑄大錢。後有言大錢之弊者,詔兩京停鑄大錢,收大錢發局改鑄。當是時,開局遍天下,重課錢息。

崇禎元年,南京鑄本七萬九千餘兩,獲息銀三萬九千有奇;戶部鑄錢獲息銀二萬六千有奇。其所鑄錢,皆以五十五文當銀一錢,計息取盈,工匠之賠補,行使之折閱,不堪命矣。寶泉局銅本四十萬兩,舊例錢成還本太倉,次年再借,至是令永作鑄本。三年,御史鐃京言:「鑄錢開局,本通行天下,今乃苦於無息,旋開旋罷,自南北兩局外,僅存湖廣、陜西、四川、雲南及宣、密二鎮。而所鑄之息,不盡歸朝廷,復苦無鑄本,蓋以買銅而非采銅也。乞遵洪武初及永樂九年、嘉靖六年例,遣官各省鑄錢,采銅於產銅之地,置官吏駐兵,仿銀礦法,十取其三。銅山之利,朝廷擅之,小民所采,仍予直以市。」帝從之。是時鑄廠並開,用銅益多,銅至益少。南京戶部尚書鄭三俊請專官買銅。戶部議原籍產銅之人駐鎮遠、荊、常銅鉛會集處,所謂采銅於產銅之地也。帝俱從之。既,又采絳、孟、垣曲、聞喜諸州縣銅鉛。荊州抽分主事硃大受言:「荊州上接黔、蜀,下聯江、廣,商販銅鉛畢集,一年可以四鑄。四鑄之息,兩倍於南,三倍於北。」因陳便宜四事,即命大受專督之。遂定錢式,每文重一錢,每千直銀一兩。南都錢輕薄,屢旨嚴飭,乃定每文重八分。初,嘉靖錢最工,隆、萬錢加重半銖,自啟、禎新鑄出,舊錢悉棄置。然日以惡薄,大半雜鉛砂,百不盈寸,捽擲輒破碎。末年敕鑄當五錢,不及鑄而明亡。

初制,歷代錢與制錢通行。自神宗初,從僉都御史龐尚鵬議,古錢止許行民間,輸稅贖罪俱用制錢。啟、禎時廣鑄錢,始括古錢以充廢銅,民間市易亦擯不用矣。莊烈帝初即位,御平臺召對,給事中黃承昊疏有銷古錢之語。大學士劉鴻訓言:「北方皆用古錢,若驟廢之,於民不便。」帝以為然。既而以御史王燮言,收銷舊錢,但行新錢,於是古錢銷毀頓盡。蓋自隋世盡銷古錢,至是凡再見云。

鈔法自弘、正間廢,天啟時,給事中惠世揚復請造行。崇禎末,有蔣臣者申其說,擢為戶部司務。倪元璐方掌部事,力主之,然終不可行而止。

坑冶之課,金銀、銅鐵、鉛汞、硃砂、青綠,而金銀礦最為民害。徐達下山東,近臣請開銀場。太祖謂銀場之弊,利於官者少,損於民者多,不可開。其後有請開陜州銀礦者,帝曰:「土地所產,有時而窮。歲課成額,徵銀無已。言利之臣,皆戕民之賊也。」臨淄丞乞發山海之藏以通寶路,帝黜之。成祖斥河池民言採礦者。仁、宣仍世禁止,填番禺坑洞,罷嵩縣白泥溝發礦。然福建尤溪縣銀屏山銀場局爐冶四十二座,始於洪武十九年。浙江溫、處、麗水、平陽等七縣,亦有場局。歲課皆二千餘兩。

永樂間,開陜西商縣鳳皇山銀坑八所。遣官湖廣、貴州採辦金銀課,復遣中官、御史往核之。又開福建浦城縣馬鞍等坑三所,設貴州太平溪、交址宣光鎮金場局,葛容溪銀場局,雲南大理銀冶。其不產金銀者,亦屢有革罷。而福建歲額增至三萬餘兩,浙江增至八萬餘。宣宗初,頗減福建課,其後增至四萬餘,而浙江亦增至九萬餘。英宗下詔封坑穴,撤閘辦官,民大蘇息,而歲額未除。歲辦,皆洪武舊額也。閘辦者,永、宣所新增也。既而禁革永煎。姦民私開坑穴相殺傷,嚴禁不能止。下詔宥之,不悛。言者復請開銀場,則利歸於上,而盜無所容。乃命侍郎王質往經理,定歲課,福建銀二萬餘,浙江倍之。又分遣御史曹祥、馮傑提督,供億過公稅,民困而盜愈眾。鄧茂七、葉宗留之徒流毒浙、閩,久之始定。景帝嘗封閉,旋以盜礦者多,兵部尚書孫原貞請開浙江銀場,因並開福建,命中官戴細保提督之。天順四年命中官羅永之浙江,羅珪之雲南,馮讓之福建,何能之四川。課額浙、閩大略如舊,雲南十萬兩有奇,四川萬三千有奇,總十八萬三千有奇。成化中,開湖廣金場,武陵等十二縣凡二十一場,歲役民夫五十五萬,死者無算,得金僅三十五兩,於是復閉。而浙江銀礦以缺額量減,雲南屢開屢停。

弘治元年始減雲南二萬兩,溫、處萬兩餘,罷浦城廢坑銀冶。至十三年,雲南巡撫李士實言:「雲南九銀場,四場礦脈久絕,乞免其課。」報可。四川、山東礦穴亦先後封閉。武宗初,從中官秦文等奏,復開浙、閩銀礦。既而浙江守臣言礦脈已絕,乃令歲進銀二萬兩,劉瑾誅乃止。世宗初,閉大理礦場。其後薊、豫、齊、晉、川、滇所在進礦砂金銀,復議開採,以助大工。既獲玉旺峪礦銀,帝諭閣臣廣開採。戶部尚書方鈍等請令四川、山東、河南撫按嚴督所屬,一一搜訪,以稱天地降祥之意。於是公私交鶩礦利,而浙江、江西盜礦者且劫徽、寧,天下漸多事矣。

隆慶初,罷薊鎮開採。南中諸礦山,亦勒石禁止。萬曆十二年,姦民屢以礦利中上心。諸臣力陳其弊。帝雖從之,意怏怏。二十四年,張位秉政,前衛千戶仲春請開礦,位不能止。開採之端啟,廢弁白望獻礦峒者日至,於是無地不開。中使四出:昌平則王忠,真、保、薊、永、房山、蔚州則王虎,昌黎則田進,河南之開封、彰德、衛輝、懷慶、葉縣、信陽則魯坤,山東之濟南、青州、濟寧、沂州、滕、費、蓬萊、福山、樓霞、招遠、文登則陳增,山西之太原、平陽、潞安則張忠,南直之寧國、池州則郝隆、劉朝用,湖廣之德安則陳奉,浙江之杭、嚴、金、衢、孝豐、諸暨則曹金,後代以劉忠,陜西之西安則趙鑒、趙欽,四川則丘乘雲,遼東則高淮,廣東則李敬,廣西則沈永壽,江西則潘相,福建則高寀,雲南則楊榮。皆給以關防,並偕原奏官往。礦脈微細無所得,勒民償之。而姦人假開採之名,乘傳橫索民財,陵轢州縣。有司恤民者,罪以阻撓,逮問罷黜。時中官多暴橫,而陳奉尤甚。富家鉅族則誣以盜礦,良田美宅則指以為下有礦脈,率役圍捕,辱及婦女,甚至斷人手足投之江,其酷虐如此。帝縱不問。自二十五年至三十三年,諸璫所進礦稅銀幾及三百萬兩,群小藉勢誅索,不啻倍蓰,民不聊生。山西巡撫魏允貞上言:「方今水旱告災,天鳴地震,星流氣射,四方日報。中外軍興,百姓困敝。而嗜利小人,借開採以肆饕餮。倘釁由中作,則礦夫冗役為禍尤烈。至是而後,求投珠抵璧之說用之晚矣。」河南巡按姚思仁亦言:「開採之弊,大可慮者有八。礦盜哨聚,易於召亂,一也。礦頭累極,勢成土崩,二也。礦夫殘害,逼迫流亡,三也。雇民糧缺,饑餓噪呼,四也。礦洞遍開,無益浪費,五也。礦砂銀少,強科民買,六也。民皆開礦,農桑失業,七也。奏官強橫,淫刑激變,八也。今礦頭以賠累死,平民以逼買死,礦夫以傾壓死,以爭鬥死。及今不止,雖傾府庫之藏,竭天下之力,亦無濟於存亡矣。」疏入,皆不省。識者以為明亡蓋兆於此。

鐵冶所,洪武六年置。江西進賢、新喻、分宜,湖廣興國、黃梅,山東萊蕪,廣東陽山,陜西鞏昌,山西吉州二,太原、澤、潞各一,凡十三所,歲輸鐵七百四十六萬餘斤。河南、四川亦有鐵冶。十二年益以茶陵。十五年,廣平吏王允道言:「磁州產鐵,元時置官,歲收百餘萬斤,請如舊。」帝以民生甫定,復設必重擾,杖而流之海外。十八年罷各布政司鐵冶。既而工部言:「山西交城產雲子鐵,舊貢十萬斤,繕治兵器,他處無有。」乃復設。已而武昌、吉州以次復焉。末年,以工部言,復盡開,令民得自採鍊,每三十分取其二。永樂時,設四川龍州、遼東都司三萬衛鐵冶。景帝時,辦事吏請復陜西、寧遠鐵礦,工部劾其違法,下獄。給事中張文質以為不宜塞言路,乃釋之。弘治十七年,廣東歸善縣請開鐵冶,有司課外索賂,唐大鬢等因作亂,都御史劉大夏討平之。正德十四年,廣州置鐵廠,以鹽課提舉司領之,禁私販如鹽法。嘉靖三十四年開建寧、延平諸府鐵冶。隆、萬以後,率因舊制,未嘗特開云。

銅場,明初,惟江西德興、鉛山。其後四川梁山,山西五臺,陜西寧羌、略陽及雲南皆採水銀、青綠。太祖時,廉州巡檢言:「階州界西戎,有水銀坑冶及青綠、紫泥,願得兵取其地。」帝不許。惟貴州大萬山長官司有水銀、硃砂場局,而四川東川府會川衛山產青綠、銀、銅,以與外番接境,虞軍民潛取生事,特禁飭之。成化十七年封閉雲南路南州銅坑。弘治十八年裁革板場坑水銀場局。正德九年,軍士周達請開雲南諸銀礦,因及銅、錫、青綠。詔可,遂次第開採。嘉靖、隆、萬間,因鼓鑄,屢開雲南諸處銅場,久之所獲漸少。崇禎時,遂括古錢以供爐冶焉。關市之徵,宋、元頗繁瑣。明初務簡約,其後增置漸多,行齎居鬻,所過所止各有稅。其名物件析榜於官署,按而征之,惟農具、書籍及他不鬻於市者勿算,應徵而藏匿者沒其半。買賣田宅頭匹必投稅,契本別納紙價。凡納稅地,置店歷,書所止商氏名物數。官司有都稅,有宣課,有司,有局,有分司,有抽分場局,有河泊所。所收稅課,有本色,有折色。稅課司局,京城諸門及各府州縣市集多有之,凡四百餘所。其後以次裁併十之七。抽分在南京者,曰龍江、大勝港;在北京者,曰通州、白河、盧溝、通積、廣積;在外者,曰真定、杭州、荊州、太平、蘭州、廣寧。又令軍衛自設場分,收貯柴薪。河泊所惟大河以南有之,河北止鹽山縣。

凡稅課,徵商估物貨;抽分,科竹木柴薪;河泊,取魚課。又有門攤課鈔,領於有司。太祖初,徵酒醋之稅,收官店錢。即吳王位,減收官店錢,改在京官店為宣課司,府縣官店為通課司。

凡商稅,三十而取一,過者以違令論。洪武初,命在京兵馬指揮領市司,每三日一校勘街市度量權衡,稽牙儈物價;在外,城門兵馬,亦令兼領市司。彰德稅課司,稅及蔬果、飲食、畜牧諸物。帝聞而黜之。山西平遙主簿成樂秩滿來朝,上其考曰「能恢辦商稅」。帝曰:「稅有定額,若以恢辦為能,是剝削下民,失吏職也。州考非是。」命吏部移文以訊。十年,戶部奏:「天下稅課司局,征商不如額者百七十八處。遂遣中官、國子生及部委官各一人核實,立為定額。十三年,吏部言:「稅課司局歲收額米不及五百石者,凡三百六十四處,宜罷之。」報可。胡惟庸伏誅,帝諭戶部曰:「曩者姦臣聚斂,稅及纖悉,朕甚恥焉。自今軍民嫁娶喪祭之物,舟車絲布之類,皆勿稅。」罷天下抽分竹木場。明年令以野獸皮輸魚課,製裘以給邊卒。

初,京師軍民居室皆官所給,比舍無隙地。商貨至,或止於舟,或貯城外,駔儈上下其價,商人病之。帝乃命於三山諸門外,瀕水為屋,名塌房,以貯商貨。

永樂初定制,嫁娶喪祭時節禮物、自織布帛、農器、食品及買既稅之物、車船運己貨物、魚蔬雜果非市販者,俱免稅。準南京例,置京城官店塌房。七年遣御史、監生於收課處榷辦課程。二十一年,山東巡按陳濟言:「淮安、濟寧、東昌、臨清、德州、直沽,商販所聚。今都北平,百貨倍往時。其商稅宜遣人監榷一年,以為定額。」帝從之。

洪熙元年增市肆門攤課鈔。宣德四年,以鈔法不通,由商居貨不稅,由是於京省商賈湊集地、市鎮店肆門攤稅課,增舊凡五倍。兩京蔬果園不論官私種而鬻者,塌房、庫房、店舍居商貨者,騾驢車受人雇裝載者,悉令納鈔。委御史、戶部、錦衣衛、兵馬司官各一,於城門察收。舟船受人雇裝載者,計所載料多寡、路近遠納鈔。鈔關之設自此始。其倚勢隱匿不報者,物盡沒官,仍罪之。於是有漷縣、濟寧、徐州、淮安、揚州、上新河、滸墅、九江、金沙洲、臨清、北新諸鈔關,量舟大小修廣而差其額,謂之船料,不稅其貨。惟臨清、北新則兼收貨稅,各差御史及戶部主事監收。自南京至通州,經淮安、濟寧、徐州、臨清,每船百料,納鈔百貫。侍郎曹弘言:「塌房月鈔五百貫,良苦,有鬻子女輸課者。」帝令核除之。及鈔法通,減北京蔬地課鈔之半,船料百貫者減至六十貫。

正統初,詔凡課程門攤,俱遵洪武舊額,不得藉口鈔法妄增。未幾,以兵部侍郎于謙奏,革直省稅課司局,領其稅於有司;罷濟寧、徐州及南京上新河船料鈔,移漷縣鈔關於河西務;船料當輸六十貫者減為二十貫。商民稱便。九年,王佐掌戶部,置彰義門官房,收商稅課鈔,復設直省稅課司官,徵榷漸繁矣。景泰元年,于謙柄國,船料減至十五貫,減漲家灣及遼陽課稅之半。大理卿薛瑄忻言:「抽分薪炭等匿不報者,準舶商匿番貨罪,盡沒之,過重。請得比匿稅律。」帝從之。成化七年增置蕪湖、荊州、杭州三處工部官。初抽分竹木,止取鈔,其後易以銀,至是漸益至數萬兩。尋遣御史榷稅。孝宗初,御史陳瑤言:「崇文門監稅官以掊克為能,非國體。」乃命客貨外,車輛毋得搜阻。又從給事中王敞言,取回蕪湖、荊州、杭州抽分御史,以府州佐貳官監收其稅。十三年復遣御史。正德十一年始收泰山碧霞元君祠香錢,從鎮守太監言也。十二年,御史胡文靜請革新設諸抽分廠。未一年,太監鄭璽請復設於順德、廣平。工部尚書李鐩依阿持兩端,橫徵之端復起。尋命中官李文、馬俊之湖廣、浙江抽分廠,與主事中分榷稅。世宗初,抽分中官及江西、福建、廣東稅課司局多所裁革,又革真定諸府抽印木植中官。

京城九門之稅,弘治初歲入鈔六十六萬餘貫,錢二百八十八萬餘文,至末年,數大減。自正德七年以後,鈔增四倍,錢增三十萬。嘉靖三年,詔如弘治初年例,仍減錢三十萬。直省關稅,成化以來,折收銀,其後復收錢鈔。八年復收銀,遂為定制。始時鈔關估船料定稅,既而以估料難核,乃度梁頭廣狹為準,自五尺至三丈六尺有差。帝令以成尺為限,勿科畸零。太監李能請於山海關榷商稅,行之數年,主事鄔閱言:「廣寧八里鋪前屯衛既有榷場,不宜再榷。」罷之。其後復山海關稅,罷八里鋪店錢。四十二年令各關歲額定數之外,餘饒悉入公帑。隆慶二年始給鈔關主事關防敕書,尋令鈔關去府近者,知府收解;去府遠者,令佐貳官收貯府庫,季解部。主事掌核商所報物數以定稅數,收解無有所與。

神宗初,令商貨進京者,河西務給紅單,赴崇文門併納正、條、船三稅;其不進京者,河西務止收正稅,免條、船二稅。萬曆十一年革天下私設無名稅課。然自隆慶以來,凡橋梁、道路、關津私擅抽稅,罔利病民,雖累詔察革,不能去矣。迨兩宮三殿災,營建費不貲,始開礦增稅。而天津店租,廣州珠榷,兩淮餘鹽,京口供用,浙江市舶,成都鹽茶,重慶名木,湖口、長江船稅,荊州店稅,寶坻魚葦及門攤商稅、油布雜稅,中官遍天下,非領稅即領礦,驅脅官吏,務朘削焉。

榷稅之使,自二十六年千戶趙承勛奏請始。其後高寀於京口,暨祿於儀真,劉成於浙,李鳳於廣州,陳奉於荊州,馬堂於臨清,陳增於東昌,孫隆於蘇、杭,魯坤於河南,孫朝於山西,丘乘雲於四川,梁永於陜西,李道於湖口,王忠於密雲,張曄於盧溝橋,沈永壽於廣西,或徵市舶,或徵店稅,或專領稅務,或兼領開採。姦民納賄於中官,輒給指揮千戶札,用為爪牙。水陸行數十里,即樹旗建廠。視商賈懦者肆為攘奪,沒其全貲。負戴行李,亦被搜索。又立土商名目,窮鄉僻塢,米鹽雞豕,皆令輸稅。所至數激民變,帝率庇不問。諸所進稅,或稱遺稅,或稱節省銀,或稱罰贖,或稱額外贏餘。又假買辦、孝順之名,金珠寶玩、貂皮、名馬,雜然進奉,帝以為能。甚至稅監劉成因災荒請暫寬商稅,中旨仍徵課四萬,其嗜利如此。三十三年始詔罷採礦,以稅務歸有司,而稅使不撤。李道詭稱有司固卻,乞如舊便。帝遽從之。又聽福府承奉謝文銓言,設官店於崇文門外,以供福邸。戶部尚書趙世卿屢疏,不聽。世卿又言:「崇文門、河西務、臨清、九江、滸墅、揚州、北新、淮安各鈔關,歲徵本折約三十二萬五千餘兩,萬曆二十五年增銀八萬二千兩,此定額也。乃二十七年以後,歷歲減縮,至二十九年總解二十六萬六千餘兩。究厥所由,則以稅使苛斂,商至者少,連年稅使所供,即此各關不足之數也。」疏入不省。寶坻銀魚廠,永樂時設,穆宗時,止令估直備廟祀上供。及是始以中官坐採,又徵其稅,後并稅武清等縣非產魚之處。增葦網諸稅,且及青縣、天津。九門稅尤苛,舉子皆不免,甚至擊殺覲吏。事聞,詔法司治之,監豎為小戢。至四十二年,李太后遺命減天下稅額三之一,免近京畸零小稅。光宗立,始盡蠲天下額外稅,撤回稅監,其派入地畝、行戶、人丁、間架者,概免之。

天啟五年,戶部尚書李起元請復榷水陸衝要,依萬曆二十七八年例,量徵什一。允行之。崇禎初,關稅每兩增一錢,通八關增五萬兩。三年復增二錢,惟臨清僅半,而崇文門、河西務俱如舊。戶部尚書畢自嚴,議增南京宣課司稅額一萬為三萬。南京戶部尚書鄭三俊,以宣課所收落地稅無幾,請稅蕪湖以當增數。自嚴遂議稅蕪湖三萬兩,而宣課仍增一萬。三俊悔,疏爭不能已。九年復議增稅課款項。十三年增關稅二十萬兩,而商民益困矣。

凡諸課程,始收鈔,間折收米,已而收錢鈔半,後乃折收銀,而折色、本色遞年輪收,本色歸內庫,折色歸太倉。

明初,東有馬市,西有茶市,皆以馭邊省戍守費。海外諸國入貢,許附載方物與中國貿易。因設市舶司,置提舉官以領之,所以通夷情,抑姦商,俾法禁有所施,因以消其釁隙也。洪武初,設於太倉黃渡,尋罷。復設於寧波、泉州、廣州。寧波通日本,泉州通琉球,廣州通占城、暹羅、西洋諸國。琉球、占城諸國皆恭順,任其時至入貢。惟日本叛服不常,故獨限其期為十年,人數為二百,舟為二艘,以金葉勘合表文為驗,以防詐偽侵軼。後市舶司暫罷,輒復嚴禁瀕海居民及守備將卒私通海外諸國。

永樂初,西洋剌泥國回回哈只馬哈沒奇等來朝,附載胡椒與民互市。有司請徵其稅。帝曰:「商稅者,國家抑逐末之民,豈以為利。今夷人慕義遠來,乃侵其利,所得幾何,而虧辱大體多矣。」不聽。三年,以諸番貢使益多,乃置驛於福建、浙江、廣東三市舶司以館之。福建曰來遠,浙江曰安遠,廣東曰懷遠。尋設交址雲屯市舶提舉司,接西南諸國朝貢者。初,入貢海舟至,有司封識,俟奏報,然後起運。宣宗命至即馳奏,不待報隨送至京。

武宗時,提舉市舶太監畢真言:「舊制,泛海諸船,皆市舶司專理,近領於鎮巡及三司官,乞如舊便。」禮部議:市舶職司進貢方物,其汎海客商及風泊番船,非敕旨所載,例不當預。中旨令如熊宣舊例行。宣先任市舶太監也,嘗以不預滿剌加諸國番舶抽分,奏請兼理,為禮部所劾而罷。劉瑾私真,謬以為例云。

嘉靖二年,日本使宗設、宋素卿分道入貢,互爭真偽。市舶中官賴恩納素卿賄,右素卿,宗設遂大掠寧波。給事中夏言言倭患起於市舶。遂罷之。市舶既罷,日本海賈往來自如,海上姦豪與之交通,法禁無所施,轉為寇賊。二十六年,倭寇百艘久泊寧、台,數千人登岸焚劫。浙江巡撫朱紈訪知舶主皆貴官大姓,市番貨皆以虛直,轉鬻牟利,而直不時給,以是構亂。乃嚴海禁,毀餘皇,奏請鐫諭戒大姓,不報。二十八年,紈又言:「長澳諸大俠林恭等勾引夷舟作亂,而巨姦關通射利,因為嚮導,躪我海濱,宜正典刑。」部覆不允。而通番大猾,紈輒以便宜誅之。御史陳九德劾紈措置乖方,專殺啟釁。帝逮紈聽勘。紈既黜,姦徒益無所憚,外交內訌,釀成禍患。汪直、徐海、陳東、麻葉等起,而海上無寧日矣。三十五年,倭寇大掠福建、浙、直,都御史胡宗憲遣其客蔣洲、陳可願使倭宣諭。還報,倭志欲通貢市。兵部議不可,乃止。

三十九年,鳳陽巡撫唐順之議復三市舶司。部議從之。四十四年,浙江以巡撫劉畿言,仍罷。福建開而復禁。萬曆中,復通福建互市,惟禁市硝黃。已而兩市舶司悉復,以中官領職如故。

永樂間,設馬市三:一在開原南關,以待海西;一在開原城東五里,一在廣寧,皆以待朵顏三衛。定直四等:上直絹八疋,布十二,次半之,下二等各以一遞減。既而城東、廣寧市皆廢,惟開原南關馬市獨存。

大同馬市始正統三年,巡撫盧睿請令軍民平價市駝馬,達官指揮李原等通譯語,禁市兵器、銅鐵。帝從之。十四年,都御史沈固請支山西行都司庫銀市馬。時也先貢馬互市,中官王振裁其馬價,也先大舉入寇,遂致土木之變。

成化十四年,陳鉞撫遼東,復開三衛馬市。通事劉海、姚安肆侵牟,朵顏諸部懷怨,擾廣寧,不復來市。兵部尚書王越請令參將、布政司官各一員監之,毋有所侵剋。遂治海、安二人罪。尋令海西及朵顏三衛入市;開原月一市,廣寧月二市,以互市之稅充撫賞。正德時,令驗放入市者,依期出境,不得挾弓矢,非互市日,毋輒近塞垣。

嘉靖三十年,以總兵仇鸞言,詔於宣府、大同開馬市,命侍郎史道總理之。兵部員外郎楊繼盛諫。不從。俺答旋入寇抄,大同市則寇宣府,宣府市則寇大同。幣未出境,警報隨至。帝始悔之,召道還。然諸部嗜馬市利,未敢公言大舉,而邊臣亦多畏懾,以互市啖之。明年罷大同馬市,宣府猶未絕,抄掠不已,乃並絕之。隆慶四年,俺答孫把漢那吉來降,於是封貢互市之議起。而宣、大互市復開,邊境稍靜。然撫賞甚厚,朝廷為省客兵餉、減哨銀以充之。頻年加賞,而要求滋甚,司事者復從中乾沒,邊費反過當矣。

遼東義州木市,萬歷二十三年開,事具李化龍傳。二十六年從巡撫張思忠奏罷之,遂並罷馬市。其後總兵李成梁力請復,而薊遼總督萬世德亦疏於朝。二十九年復開馬、木二市,後以為常。

○上供採造採造柴炭採木珠池織造燒造俸餉會計

採造之事,累朝侈儉不同。大約靡於英宗,繼以憲、武,至世宗、神宗而極。其事目繁瑣,徵索紛紜。最鉅且難者,曰採木。歲造最大者,曰織造、曰燒造。酒醴膳羞則掌之光祿寺,採辦成就則工部四司、內監司局或專差職之,柴炭則掌之惜薪司。而最為民害者,率由中官。

明初,上供簡省。郡縣貢香米、人參、葡萄酒,太祖以為勞民,卻之。仁宗初,光祿卿井泉奏,歲例遣正官往南京採玉面貍,帝叱之曰:「小人不達政體。朕方下詔,盡罷不急之務以息民,豈以口腹細故,失大信耶!」宣宗時,罷永樂中河州官買乳牛造上供酥油者,以其牛給屯軍。命御史二人察視光祿寺,凡內外官多支及需索者,執奏。英宗初政,三楊當軸,減南畿孳牧黃牛四萬,糖蜜、果品、腒脯、酥油、茶芽、稉糯、粟米、藥材皆減省有差,撤諸處捕魚官。即位數月,多所撙節。凡上用膳食器皿三十萬七千有奇,南工部造,金龍鳳白瓷諸器,饒州造,硃紅膳盒諸器,營膳所造,以進宮中食物,尚膳監率乾沒之。帝令備帖具書,如數還給。景帝時,從於謙言,罷真定、河間採野味、直沽海口造乾魚內使。

天順八年,光祿果品物料凡百二十六萬八千餘斤,增舊額四之一。成化初,詔光祿寺牲口不得過十萬。明年,寺臣李春請增。禮部尚書姚夔言:「正統間,雞鵝羊豕歲費三四萬。天順以來增四倍,暴殄過多。請從前詔。」後二年,給事中陳鉞言:「光祿市物,概以勢取。負販遇之,如被劫掠。夫光祿所供,昔皆足用,今不然者,宣索過額,侵漁妄費也。」大學士彭時亦言:「光祿寺委用小人買辦,假公營私,民利盡為所奪。請照宣德、正統間例,斟酌供用,禁止買辦。」於是減魚果歲額十之一。弘治元年命光祿減增加供應。初,光祿俱預支官錢市物,行頭吏役因而侵蝕。乃令各行先報納而後償價,遂有游手號為報頭,假以供應為名,抑價倍取,以充私橐。御史李鸞以為言,帝命禁止。十五年,光祿卿王珩,列上內外官役酒飯及所畜禽獸料食之數,凡百二十事。乃降旨,有仍舊者,有減半者,有停止者。於是放去乾明門虎、南海子貓、西華門鷹犬、御馬監山猴、西安門大鴿等,減省有差,存者減其食料。自成化時,添坐家長隨八十餘員,傳添湯飯中官百五十餘員。天下常貢不足於用,乃責買於京師鋪戶。價直不時給,市井負累。兵部尚書劉大夏因天變言之,乃裁減中官,歲省銀八十餘萬。

武宗之世,各宮日進、月進,數倍天順時。廚役之額,當仁宗時僅六千三百餘名,及憲宗增四之一。世宗初,減至四千一百名,歲額銀撙節至十三萬兩。中年復增至四十萬。額派不足,借支太倉。太倉又不足,乃令原供司府依數增派。於是帝疑其乾沒,下禮部問狀,責光祿寺具數以奏。帝復降旨詰責,乃命御史稽核月進揭帖,兩月間省銀二萬餘兩,自是歲以為常。

先是上供之物,任土作貢,曰歲辦。不給,則官出錢以市,曰採辦。其後本折兼收,採辦愈繁。於是召商置買,物價多虧,商賈匿迹。二十七年,戶部言:「京師召商納貨取直,富商規避,應役者皆貧弱下戶,請核實編審。」給事中羅崇奎言:「諸商所以重困者,物價賤則減,而貴則不敢增。且收納不時,一遭風雨,遂不可用,多致賠累。既收之後,所司更代不常,不即給直,或竟沈閣。幸給直矣,官司折閱於上,番役齮齕於下,名雖平估,所得不能半。諸弊若除,商自樂赴,奚用編審。」帝雖納其言,而仍編審如戶部議。

穆宗朝,光祿少卿李鍵奏十事,帝乃可之,頗有所減省:停止承天香米、外域珍禽奇獸,罷寶坻魚鮮。凡薦新之物,領於光祿寺,勿遣中官。又從太監李芳請,停徵加增細稉米、白青鹽,命一依成、弘間例。御史王宗載請停加派。部議悉準原額,果品百七萬八千餘斤,牲口銀五萬八千餘兩,免加派銀二萬餘。未行,而神宗立,詔免之。世宗末年,歲用止十七萬兩,穆宗裁二萬,止十五萬餘,經費省約矣。萬曆初年,益減至十三四萬,中年漸增,幾三十萬,而鋪戶之累滋甚。時中官進納索賂,名鋪墊錢,費不訾,所支不足相抵,民不堪命,相率避匿。乃僉京師富戶為商。令下,被僉者如赴死,重賄營免。官司蜜鉤,若緝姦盜。宛平知縣劉曰淑言:「京民一遇僉商,取之不遺毫髮,貲本悉罄。請厚估先發,以蘇民困。」御史王孟震斥其越職,曰淑自劾解官去。至熹宗時,商累益重,有輸物於官終不得一錢者。

洪武時,宮禁中市物,視時估率加十錢,其損上益下如此。永樂初,斥言採五色石者,且以溫州輸礬困民,罷染色布。然內使之出,始於是時。工役繁興,徵取稍急,非土所有,民破產購之。軍器之需尤無算。仁宗時,山場、園林、湖池、坑冶、果樹、蜂蜜,官設守禁者,悉予民。宣宗罷閘辦金銀,其他紙靛、糸寧絲、紗羅、BF緞、香貨、銀硃、金箔、紅花、茜草、麂皮、香蠟、藥物、果品、海味、硃紅戧金龍鳳器物,多所罷減。副都御史弋謙言:「有司給買辦物料價,十不償一,無異空取。」帝嘉納之,諭工部察懲。又因泰安州稅課局大使郝智言,悉召還所遣官,敕自今更不許輒遣,自軍器、軍需外,凡買辦者盡停止。然寬免之詔屢下,內使屢敕撤還,而奉行不實,宦者輒名採辦,虐取於民。誅袁琦、阮巨隊等十餘人,患乃稍息。英宗立,罷諸處採買及造下西洋船木,諸冗費多敕省。正統八年,以買辦擾民,始令於存留錢糧內折納,就近解兩京。

先是仁宗時,令中官鎮守邊塞,英宗復設各省鎮守,又有守備、分守,中官布列天下。及憲宗時益甚,購書採藥之使,搜取珍玩,靡有孑遺。抑賣鹽引,私採禽鳥,糜官帑,納私賂,動以巨萬計。太嶽、太和山降真諸香,通三歲用七千斤,至是倍之。內府物料,有至五六倍者。孝宗立,頗有減省。甘肅巡撫羅明言:「鎮守、分守內外官競尚貢獻,各遣使屬邊衛搜方物,名曰採辦,實扣軍士月糧馬價,或巧取番人犬馬奇珍。且設膳乳諸房,僉廚役造酥油諸物。比及起運,沿途騷擾,乞悉罷之。」報可,然其後靡費漸多。至武宗任劉瑾,漁利無厭。鎮守中官率貢銀萬計,皇店諸名不一,歲辦多非土產。諸布政使來朝,各陳進貢之害,皆不省。

世宗初,內府供應減正德什九。中年以後,營建齋醮,採木採香,採珠玉寶石,吏民奔命不暇,用黃白蠟至三十餘萬斤。又有召買,有折色,視正數三倍。沈香、降香、海漆諸香至十餘萬斤。又分道購龍涎香,十餘年未獲,使者因請海舶入澳,久乃得之。方澤、朝日壇,爵用紅黃玉,求不得,購之陜西邊境,遣使覓於阿丹,去土魯番西南二千里。太倉之銀,頗取入承運庫,辦金寶珍珠。於是貓兒睛、祖母碌、石綠、撤孛尼石、紅剌石、北河洗石、金剛鑽、朱藍石、紫英石、甘黃玉,無所不購。穆宗承之,購珠寶益急。給事中李己、陳吾德疏諫。己下獄,吾德削籍。自是供億浸多矣。

神宗初,內承運庫太監崔敏請買金珠。張居正封還敏疏,事遂寢。久之,帝日黷貨,開採之議大興,費以鉅萬計,珠寶價增舊二十倍。戶部尚書陳蕖言庫藏已竭,宜加撙節。中旨切責。而順天府尹以大珠鴉青購買不如旨,鐫級。至於末年,內使雜出,採造益繁。內府告匱,至移濟邊銀以供之。熹宗一聽中官,採造尤夥。莊烈帝立,始務釐剔節省,而庫藏已耗竭矣。

永樂中,後軍都督府供柴炭,役宣府十七衛所軍士採之邊關。宣宗初,以邊木以扼敵騎,且邊軍不宜他役,詔免其採伐,令歲納銀二萬餘兩,後府召商買納。四年置易州山廠,命工部侍郎督之,僉北直、山東、山西民夫轉運,而後府輸銀召商如故。

初,歲用薪止二千萬餘斤。弘治中,增至四千萬餘斤。轉運既艱,北直、山東、山西乃悉輸銀以召商。正德中,用薪益多,增直三萬餘兩。凡收受柴炭,加耗十之三,中官輒私加數倍。逋負日積,至以三年正供補一年之耗。尚書李鐩議,令正耗相準,而主收者復私加,乃以四萬斤為萬斤,又有輸納浮費,民弗能堪。世宗登極,乃酌減之。隆慶六年,後府採納艱苦,改屬兵部武庫司。萬曆中,歲計柴價銀三十萬兩,中官得自徵比諸商,酷刑悉索,而人以惜薪司為陷阱云。

採木之役,自成祖繕治北京宮殿始。永樂四年遣尚書宋禮如四川,侍郎古樸如江西,師逵、金純如湖廣,副都御史劉觀如浙江,僉都御史史仲成如山西。禮言有數大木,一夕自浮大谷達於江。天子以為神,名其山曰神木山,遣官祠祭。十年復命禮採木四川。仁宗立,已其役。宣德元年修南京天地山川壇殿宇,復命侍郎黃宗載、吳廷用採木湖廣。未幾,因旱災已之。尋復採大木湖廣,而諭工部酌省,未幾復罷。其他處亦時採時罷。弘治時,發內帑修清寧宮,停四川採木。

正德時,採木湖廣、川、貴,命侍郎劉丙督運。太監劉養劾其不中梁棟,責丙陳狀,工部尚書李鐩奪俸。嘉靖元年革神木千戶所及衛卒。二十年,宗廟災,遣工部侍郎潘鑑、副都御史戴金於湖廣、四川採辦大木。二十六年復遣工部侍郎劉伯躍採於川、湖、貴州,湖廣一省費至三百三十九萬餘兩。又遣官核諸處遣留大木。郡縣有司,以遲誤大工逮治褫黜非一,並河州縣尤苦之。萬曆中,三殿工興,採楠杉諸木於湖廣、四川、貴州,費銀九百三十餘萬兩,徵諸民間,較嘉靖年費更倍。而採鷹平條橋諸木於南直、浙江者,商人逋直至二十五萬。科臣劾督運官遲延侵冒,不報。虛糜乾沒,公私交困焉。

廣東珠池,率數十年一採。宣宗時,有請令中官採東莞珠池者,繫之獄。英宗始使中官監守,天順間嘗一採之。至弘治十二年,歲久珠老,得最多,費銀萬餘,獲珠二萬八千兩,遂罷監守中官。正德九年又採,嘉靖五年又採,珠小而嫩,亦甚少。八年復詔採,兩廣巡撫林富言:「五年採珠之役,死者五十餘人,而得珠僅八十兩,天下謂以人易珠。恐今日雖以人易珠,亦不可得。」給事中王希文言:「雷、廉珠池,祖宗設官監守,不過防民爭奪。正德間,逆豎用事,傳奉採取,流毒海濱。陛下御極,革珠池少監,未久旋復。驅無辜之民,蹈不測之險,以求不可必得之物,而責以難足之數,非聖政所宜有。」皆不聽。隆慶六年詔雲南進寶石二萬塊,廣東採珠八千兩。神宗立,停罷。既而以太后進奉,諸王、皇子、公主冊立、分封、婚禮,令歲辦金珠寶石。復遣中官李敬、李鳳廣東採珠五千一百餘兩。給事中包見捷力諫,不納。至三十二年始停採。四十一年,以指揮倪英言,復開。

明制,兩京織染,內外皆置局。內局以應上供,外局以備公用。南京有神帛堂、供應機房,蘇、杭等府亦各有織染局,歲造有定數。

洪武時,置四川、山西諸行省,浙江紹興織染局。又置藍靛所於儀直、六合,種青藍以供染事。未幾悉罷。又罷天下有司歲織緞匹。有賞賚,給以絹帛,於後湖置局織造。永樂中,復設歙縣織染局。令陜西織造駝毼。正統時,置泉州織造局。天順四年遣中官往蘇、松、杭、嘉、湖五府,於常額外,增造彩緞七千匹。工部侍郎翁世資請減之,下錦衣獄,謫衡州知府。增造坐派於此始。孝宗初立,停免蘇、杭、嘉、湖、應天織造。其後復設,乃給中官鹽引,鬻於淮以供費。

正德元年,尚衣監言:「內庫所貯諸色紵絲、紗羅、織金、閃色,蟒龍、斗牛、飛魚、麒麟、獅子通袖、膝襴,並胸背斗牛、飛仙、天鹿,俱天順間所織,欽賞已盡。乞令應天、蘇、杭諸府依式織造。」帝可之。乃造萬七千餘匹。蓋成、弘時,頒賜甚謹。自劉瑾用事,幸璫陳乞漸廣,有未束髮而僭冒章服者,濫賞日增。中官乞鹽引、關鈔無已,監督織造,威劫官吏。至世宗時,其禍未訖。即位未幾,即令中官監織於南京、蘇、杭、陜西。穆宗登極,詔撤中官,已而復遣。

萬曆七年,蘇、松水災,給事中顧九思等請取回織造內臣,帝不聽。大學士張居正力陳年饑民疲,不堪催督,乃許之。未幾復遣中官。居正卒,添織漸多。蘇、杭、松、嘉、湖五府歲造之外,又令浙江、福建,常、鎮、徽、寧、揚、廣德諸府州分造,增萬餘匹。陜西織造羊絨七萬四千有奇,南直、浙江紵絲、紗羅、綾紬、絹帛,山西潞紬,皆視舊制加丈尺。二三年間,費至百萬,取給戶、工二部,搜括庫藏,扣留軍國之需。部臣科臣屢爭,皆不聽。末年,復令稅監兼司,姦弊日滋矣。

明初設南北織染局,南京供應機房,各省直歲造供用,蘇、杭織造,間行間止。自萬曆中,頻數派造,歲至十五萬匹,相沿日久,遂以為常。陜西織造絨袍,弘、正間偶行,嘉、隆時復遣,亦遂沿為常例。

燒造之事,在外臨清磚廠,京師琉璃、黑窯滄廠,皆造磚瓦,以供營繕。宣宗始遣中官張善之饒州,造奉先殿几筵龍鳳文白瓷祭器,磁州造趙府祭器。踰年,善以罪誅,罷其役。正統元年,浮梁民進瓷器五萬餘,償以鈔。禁私造黃、紫、紅、綠、青、藍、白地青花諸瓷器,違者罪死。宮殿告成,命造九龍九鳳膳案諸器,既又造青龍白地花缸。王振以為有璺,遣錦衣指揮杖提督官,敕中官往督更造。成化間,遣中官之浮梁景德鎮,燒造御用瓷器,最多且久,費不貲。孝宗初,撤回中官,尋復遣,弘治十五年復撤。正德末復遣。

自弘治以來,燒造未完者三十餘萬器。嘉靖初,遣中官督之。給事中陳皋謨言其大為民害,請罷之。帝不聽。十六年新作七陵祭器。三十七年遣官之江西,造內殿醮壇瓷器三萬,後添設饒州通判,專管御器廠燒造。,是時營建最繁,近京及蘇州皆有磚廠。隆慶時,詔江西燒造瓷器十餘萬。萬曆十九年命造十五萬九千,既而復增八萬,至三十八年未畢工。自後役亦漸寢。

國家經費,莫大於祿餉。洪武九年定諸王公主歲供之數:親王,米五萬石,鈔二萬五千貫,錦四十匹,紵絲三百匹,紗、羅各百匹,絹五百匹,冬夏布各千匹,綿二千兩,鹽二百引,花千斤,皆歲支。馬料草,月支五十匹。其緞匹,歲給匠料,付王府自造。靖江王,米二萬石,鈔萬貫,餘物半親王,馬料草二十匹。公主未受封者,紵絲、紗、羅各十匹,絹、冬夏布各三十匹,綿二百兩;已受封,賜莊田一所,歲收糧千五百石,鈔二千貫。親王子未受封,視公主;女未受封者半之。子已受封郡王,米六千石,鈔二千八百貫,錦十匹,紵絲五十匹,紗、羅減紵絲之半,絹、冬夏布各百匹,綿五百兩,鹽五十引,茶三百斤,馬料草十匹。女已受封及已嫁,米千石,鈔千四百貫,其緞匹於所在親王國造給。皇太子之次嫡子并庶子,既封郡王,必俟出閣然後歲賜,與親王子已封郡王者同。女俟及嫁,與親王女已嫁者同。凡親王世子,與已封郡王同,郡王嫡長子襲封郡王者,半始封郡王。女已封縣主及已嫁者,米五百石,鈔五百貫,餘物半親王女已受封者。郡王諸子年十五,各賜田六十頃,除租稅為永業,其所生子世守之,後乃令止給祿米。

二十八年詔以官吏軍士俸給彌廣,量減諸王歲給,以資軍國之用。乃更定親王萬石,郡王二千石,鎮國將軍千石,輔國將軍、奉國將軍、鎮國中尉以二百石遞減,輔國中尉、奉國中尉以百石遞減,公主及駙馬二千石,郡王及儀賓八百石,縣主、郡君及儀賓以二百石遞減,縣君、鄉君及儀賓以百石遞減。自後為永制。仁宗即位,增減諸王歲祿,非常典也。時鄭、越、襄、荊、淮、滕、梁七王未之籓,令暫給米歲三千石,遂為例。正統十二年定王府祿米,將軍自賜名受封日為始,縣主、儀賓自出閤成婚日為始,於附近州縣秋糧內撥給。景泰七年定郡王將軍以下祿米,出閤在前,受封在後,以受封日為始;受封在前,出閤在後,以出閤日為始。

宗室有罪革爵者曰庶人。英宗初,頗給以糧。嘉靖中,月支米六石。萬歷中減至二石或一石。

初,太祖大封宗籓,令世世皆食歲祿,不授職任事,親親之誼甚厚。然天潢日繁,而民賦有限。其始祿米盡支本色,既而本鈔兼支。有中半者,有本多於折者,其則不同。厥後勢不能給,而冒濫轉益多。姦弊百出,不可究詰。自弘治間,禮部尚書倪岳即條請節減,以寬民力。嘉靖四十一年,御史林潤言:「天下之事,極弊而大可慮者,莫甚於宗籓祿廩。天下歲供京師糧四百萬石,而諸府祿米凡八百五十三萬石。以山西言,存留百五十二萬石,而宗祿三百十二萬;以河南言,存留八十四萬三千石,而宗祿百九十二萬。是二省之糧,借令全輸,不足供祿米之半,況吏祿、軍餉皆出其中乎?故自郡王以上,猶得厚享,將軍以下,多不能自存,飢寒困辱,勢所必至,常號呼道路,聚詬有司。守土之臣,每懼生變。夫賦不可增,而宗室日益蕃衍,可不為寒心。宜令大臣科道集議於朝,且諭諸王以勢窮弊極,不得不通變之意。令戶部會計賦額,以十年為率,通計兵荒蠲免、存留及王府增封之數。共陳善後良策,斷自宸衷,以垂萬世不易之規。」下部覆議,從之。至四十四年乃定宗籓條例。郡王、將軍七分折鈔,中尉六分折鈔,郡縣主、郡縣鄉君及儀賓八分折鈔,他冒濫者多所裁減。於是諸王亦奏辭歲祿,少者五百石,多者至二千石,歲出為稍紓,而將軍以下益不能自存矣。

明初,勛戚皆賜官田以代常祿。其後令還田給祿米。公五千石至二千五百石,侯千五百石至千石,伯千石至七百石。百官之俸,自洪武初,定丞相、御史大夫以下歲祿數,刻石官署,取給於江南官田。,十三年重定內外文武官歲給祿米、俸鈔之制,而雜流吏典附焉。正從一二三四品官,自千石至三百石,每階遞減百石,皆給俸鈔三百貫。正五品二百二十石,從減五十石,鈔皆百五十貫。正六品百二十石,從減十石,鈔皆九十貫。正從七品視從六品遞減十石,鈔皆六十貫。正八品七十五石,從減五石,鈔皆四十五貫。正從九品視從八品遞減五石,鈔皆三十貫。勒之石。吏員月俸,一二品官司提控、都吏二石五斗,掾史、令史二石二斗,知印、承差、吏、典一石二斗;三四品官司令史、書吏、司吏二石,承差、吏、典半之;五品官司司吏一石二斗,吏、典八斗;六品以下司吏一石;光祿寺等吏、典六斗。教官之祿,州學正月米二石五斗,縣教諭、府州縣訓導月米二石。首領官之祿,凡內外官司提控、案牘、州吏目、縣典史皆月米三石。雜職之祿,凡倉、庫、關、場、司、局、鐵冶、遞運、批驗所大使月三石,副使月二石五斗,河泊所官月米二石,閘壩官月米一石五斗。天下學校師生廩膳米人日一升,魚肉鹽醯之屬官給之。宦官俸,月米一石。

二十五年更定百官祿。正一品月俸米八十七石,從一品至正三品,遞減十三石至三十五石,從三品二十六石,正四品二十四石,從四品二十一石,正五品十六石,從五品十四石,正六品十石,從六品八石,正七品至從九品遞減五斗,至五石而止。自後為永制。

洪武時,官俸全給米,間以錢鈔兼給,錢一千,鈔一貫,抵米一石。成祖即位,令公、侯、伯皆全支米;文武官俸則米鈔兼支,官高者支米十之四、五,官卑者支米十之六、八;惟九品、雜職、吏、典、知印、總小旗、軍,並全支米。其折鈔者,每米一石給鈔十貫。永樂二年乃命公、侯、伯視文武官吏,米鈔兼支。仁宗立,官俸折鈔,每石至二十五貫。宣德八年,禮部尚書胡濙掌戶部,議每石減十貫,而以十分為準,七分折絹,絹一匹抵鈔二百貫。少師蹇義等以為仁宗在春宮久,深憫官員折俸之薄,故即位特增數倍,此仁政也,詎可違?濙不聽,竟請於帝而行之,而卑官日用不贍矣。正統中,五品以上米二鈔八,六品以下米三鈔七。時鈔價日賤,每石十五貫者已漸增至二十五貫,而戶部尚書王佐復奏減為十五貫。成化二年從戶部尚書馬昂請,又省五貫。舊例,兩京文武官折色俸,上半年給鈔,下半年給蘇木、胡椒。七年從戶部尚書楊鼎請,以甲字庫所積之布估給,布一匹當鈔二百貫。是時鈔法不行,一貫僅直錢二三文,米一石折鈔十貫,僅直二三十錢,而布直僅二三百錢,布一匹折米二十石,則米一石僅直十四五錢。自古官俸之薄,未有若此者。

十六年又令以三梭布折米,每匹抵三十石。其後粗闊棉布亦抵三十石,梭布極細者猶直銀二兩,麤布僅直三四錢而已。久之,定布一匹折銀三錢。於是官員俸給凡二:曰本色,曰折色。其本色有三:曰月米,曰折絹米,曰折銀米。月米,不問官大小,皆一石。折絹,絹一匹當銀六錢。折銀,六錢五分當米一石。其折色有二:曰本色鈔,曰絹布折鈔。本色鈔十貫折米一石,後增至二十貫。絹布折鈔,絹每匹折米二十石,布一匹折米十石。公侯之祿,或本折中半,或折多於本有差。文武官俸,正一品者,本色僅十之三,遞增至從九品,本色乃十之七。武職府衛官,惟本色米折銀例,每石二錢五分,與文臣異,餘並同。其三大營副將、參、游、佐員,每月米五石,巡捕營提督、參將亦如之。巡捕中軍、把總官,月支口糧九斗,旗牌官半之。

天下衛所軍士月糧,洪武中,令京外衛馬軍月支米二石,步軍總旗一石五斗,小旗一石二斗,軍一石。城守者如數給,屯田者半之。民匠充軍者八斗,牧馬千戶所一石,民丁編軍操練者一石,江陰橫海水軍稍班、碇手一石五斗。陣亡病故軍給喪費一石,在營病故者半之。籍沒免死充軍者謂之恩軍。家四口以上一石,三口以下六斗,無家口者四斗。又給軍士月鹽,有家口者二斤,無者一斤,在外衛所軍士以鈔準。永樂中,始令糧多之地,旗軍月糧,八分支米,二分支鈔。後山西、陜西皆然,而福建、兩廣、四川則米七鈔三,江西則米鈔中半,惟京軍及中都留守司,河南、浙江、湖廣軍,仍全支米。已而定制,衛軍有家屬者,月米六斗,無者四斗五升,餘皆折鈔。

凡各衛調至京操備軍兼工作者,米五斗。其後增損不一,而本折則例,各鎮多寡不同,不能具舉。凡各鎮兵餉,有屯糧,有民運,有鹽引,有京運,有主兵年例,有客兵年例。屯糧者,明初,各鎮皆有屯田,一軍之田,足贍一軍之用,衛所官吏俸糧皆取給焉。民運者,屯糧不足,加以民糧。麥、米、豆、草、布、鈔、花絨運給戍卒,故謂之民運,後多議折銀。鹽引者,召商入粟開中,商屯出糧,與軍屯相表裏。其後納銀運司,名存而實亡。京運,始自正統中。後屯糧、鹽糧多廢,而京運日益矣。主兵有常數,客兵無常數。初,各鎮主兵足守其地,後漸不足,增以募兵,募兵不足,增以客兵。兵愈多,坐食愈眾,而年例亦日增云。

明田稅及經費出入之數,見於掌故進,皆略可考見。洪武二十六年,官民田總八百五十萬七千餘頃。夏稅,米麥四百七十一萬七千餘石,錢鈔三萬九千餘錠,絹二十八萬八千餘匹;秋糧,米二千四百七十二萬九千餘石,錢鈔五千餘錠。弘治時,官民田總六百二十二萬八千餘頃。夏稅,米麥四百六十二萬五千餘石,鈔五萬六千三百餘錠,絹二十萬二千餘匹;秋糧,米二千二百十六萬六千餘石,鈔二萬一千九百餘錠。萬曆時,官民田總七百一萬三千餘頃。夏稅,米麥總四百六十萬五千餘石,起運百九十萬三千餘石,餘悉存留,鈔五萬七千九百餘錠,絹二十萬六千餘匹;秋糧,米總二千二百三萬三千餘石,起運千三百三十六萬二千餘石,餘悉存留,鈔二萬三千六百餘錠。屯田六十三萬五千餘頃,花園倉基千九百餘所,徵糧四百五十八萬四千餘石。糧草折銀八萬五千餘兩,布五萬匹,鈔五萬餘貫,各運司提舉大小引鹽二百二十二萬八千餘引。

歲入之數,內承運庫,慈寧、慈慶、乾清三宮子粒銀四萬九千餘兩,金花銀一百一萬二千餘兩,金二千兩。廣惠庫、河西務等七鈔關,鈔二千九百二十八萬餘貫,錢五千九百七十七萬餘文。京衛屯鈔五萬六千餘貫。天財庫、京城九門鈔六十六萬五千餘貫,錢二百四十三萬餘文。京、通二倉,并薊、密諸鎮漕糧四百萬石。京衛屯豆二萬三千餘石。太倉銀庫,南北直隸、浙江、江西、山東、河南派剩麥米折銀二十五萬七千餘兩。絲綿、稅絲、農桑絹折銀九萬餘兩,綿布、苧布折銀三萬八千餘兩。百官祿米折銀二萬六千餘兩。馬草折銀三十五萬三千餘兩。京五草場折銀六萬三千餘兩。各馬房倉麥豆草折銀二十餘萬兩。戶口鹽鈔折銀四萬六千餘兩。薊、密、永、昌、易、遼東六鎮,民運改解銀八十五萬三千餘兩。各鹽運提舉餘鹽、鹽課、鹽稅銀一百萬三千餘兩。黃白蠟折銀六萬八千餘兩。霸、大等馬房子粒銀二萬三千餘兩。備邊并新增地畝銀四萬五千餘兩。京衛屯牧地增銀萬八千餘兩。崇文門商稅、牙稅一萬九千餘兩,錢一萬八千餘貫。張家灣商稅二千餘兩,錢二千八百餘貫。諸鈔關折銀二十二萬三千餘兩。泰山香稅二萬餘兩。贓罰銀十七萬餘兩。商稅、魚課、富戶、曆日、民壯、弓兵并屯折、改折月糧銀十四萬四千餘兩。北直隸、山東、河南解各邊鎮麥、米、豆、草、鹽鈔折銀八十四萬二千餘兩。諸雜物條目繁瑣者不具載。所載歲入,但計起運京邊者,而存留不與焉。

歲出之數,公、侯、駙馬、伯祿米折銀一萬六千餘兩。官吏、監生俸米四萬餘石。官吏折俸絹布銀四萬四千餘兩,錢三千三百餘貫。倉庫、草場、官攢、甲斗,光祿、太常諸司及內府監局匠役本色米八萬六千餘石,折色銀一萬三千餘兩。錦衣等七十八衛所官吏、旗校、軍士、匠役本色米二百一萬八千餘石,折色銀二十萬六千餘兩。官員折俸絹布銀二十六萬八千餘兩。軍士冬衣折布銀八萬二千餘兩。五軍、神樞、神機三大營將卒本色米十二萬餘石,冬衣折布銀二千餘兩,官軍防秋三月口糧四萬三千餘石,營操馬匹本色料二萬四千餘石,草八十萬餘束。巡捕營軍糧七千餘石。京營、巡捕營,錦衣、騰驤諸衛馬料草折銀五萬餘兩。中都留守司,山東、河南二都司班軍行糧及工役鹽糧折銀五萬餘兩。京五草場商價一萬六千餘兩。御馬三倉象馬等房,商價十四萬八千餘兩。

諸邊及近京鎮兵餉。

宣府:主兵,屯糧十三萬二千餘石,折色銀二萬二千餘兩,民運折色銀七十八萬七千餘兩,兩淮、長蘆、河東鹽引銀十三萬五千餘兩,京運年例銀十二萬五千兩;客兵,淮、蘆鹽引銀二萬六千餘兩,京運年例銀十七萬一千兩。

大同:主兵,屯糧本色七萬餘石,折色銀一萬六千餘兩,牛具銀八千餘兩,鹽鈔銀一千餘兩,民運本色米七千餘石,折色銀四十五萬六千餘兩,屯田及民運本色草二百六十八萬餘束,折草銀二萬八千餘兩,淮、蘆鹽四萬三千餘引,京運年例銀二十六萬九千餘兩;客兵,京運銀十八萬一千兩,淮、蘆鹽七萬引。

山西:主兵,屯糧二萬八千餘石,折色銀一千餘兩,草九萬五千餘束,民運本色米豆二萬一千餘石,折色銀三十二萬二千餘兩,淮、浙、山東鹽引銀五萬七千餘兩,河東鹽課銀六萬四千餘兩,京運銀十三萬三千餘兩;客兵,京運銀七萬三千兩。

延綏:主兵,屯糧五萬六千餘石,地畝銀一千餘兩,民運糧料九萬七千餘石,折色銀十九萬七千餘兩,屯田及民運草六萬九千餘束,淮、浙鹽引銀六萬七千餘兩,京運年例銀三十五萬七千餘兩;客兵,淮、浙鹽引銀二萬九千餘兩,京運年例銀二萬餘兩。

寧夏:主兵,屯糧料十四萬八千餘石,折色銀一千餘兩,地畝銀一千餘兩,民運本色糧千餘石,折色銀十萬八千餘兩,屯田及民運草一百八十三萬餘束,淮、浙鹽引銀八萬一千餘兩,京運年例銀二萬五千兩;客兵,京運年例銀萬兩。

甘肅:屯糧料二十三萬二千餘石,草四百三十餘萬束,折草銀二千餘兩,民運糧布折銀二十九萬四千餘兩,京運銀五萬一千餘兩,淮、浙鹽引銀十萬二千餘兩。

固原:屯糧料三十一萬九千餘石,折色糧料草銀四萬一千餘兩,地畝牛具銀七千一百餘兩,民運本色糧料四萬五千餘石,折色糧料草布花銀二十七萬九千餘兩,屯田及民運草二十萬八千餘束,淮、浙鹽引銀二萬五千餘兩,京運銀六萬三千餘兩,犒賞銀一百九十餘兩。

遼東:主兵,屯糧二十七萬九千餘石,荒田糧四百餘兩,民運銀十五萬九千餘兩,兩淮、山東鹽引銀三萬九千餘兩,京運年例銀三十萬七千餘兩;客兵,京運年例銀十萬二千餘兩。

薊州:主兵,民運銀九千餘兩,漕糧五萬石,京運年例銀二十萬六千餘兩;客兵,屯糧料五萬三千餘石,地畝馬草折色銀萬六千餘兩,民運銀萬八千餘兩,山東民兵工食銀五萬六千兩,遵化營民壯工食銀四千餘兩,鹽引銀萬三千餘兩,京運年例銀二十萬八千餘兩,撫賞銀一萬五千兩,犒軍銀一萬三千餘兩。

永平:主兵,屯糧料三萬三千餘石,民運糧料二萬七千餘石,折色銀二萬八千餘兩,民壯工食銀萬二千餘兩,京運年例銀十二萬二千餘兩;客兵,屯草折銀三千餘兩,民運草三十一萬一千餘束,京運銀十一萬九千餘兩。

密雲:主兵,屯糧六千餘石,地畝銀二百九十兩,民運銀萬兩有奇,漕糧十萬四千餘石,京運銀十六萬兩有奇;客兵,民運銀萬六千餘兩,民壯工食銀九百餘兩,漕糧五萬石,京運銀二十三萬三千餘兩。

昌平:主兵,屯糧折色銀二千四百餘兩,地畝銀五百餘兩,折草銀一百餘兩,民運銀二萬兩有奇,漕糧十八萬九千餘石,京運年例銀九萬六千餘兩;客兵,京運年例銀四萬七千餘兩。

易州:主兵,屯糧二萬三千餘石,地畝銀六百餘兩,民運銀三十萬六千餘兩;客兵,京運銀五萬九千兩。

井陘:主兵,屯糧萬四千餘石,地畝銀八千餘兩,民運本色米麥一萬七千餘石,折色銀四萬八千餘兩;客兵,京運年例銀三千餘兩。

他雜費不具載。
