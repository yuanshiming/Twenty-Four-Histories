\article{列傳第五十四}

\begin{pinyinscope}
○韓觀山雲蕭授吳亮方瑛陳友李震王信都勝郭鋐彭倫歐磐張祐

韓觀,字彥賓,虹人,高陽忠壯侯成子也。以舍人宿衛,忠謹為太祖所知,授桂林右衛指揮僉事。

洪武十九年討平柳州、融縣諸蠻,累遷廣西都指揮使。二十二年平富川蠻,設靈亭千戶所。二十五年平賓州上林蠻。二十七年會湖廣兵討全州、灌陽諸瑤,斬千四百餘人。明年捕擒宜山諸縣蠻,斬其偽王及萬戶以下二千八百餘人。以征南左副將軍從都督楊文討龍州土官趙宗壽,宗壽伏罪。移兵征南丹、奉議及都康、向武、富勞、上林、思恩、都亮諸蠻,先後斬獲萬餘級。

觀生長兵間,有勇略。性鷙悍,誅罰無所假。下令如山,人莫敢犯。初,群蠻所在蜂起,剽郡縣,殺守吏,勢甚熾。將士畏觀法,爭死鬥。觀得賊必處以極刑。間縱一二,使歸告諸蠻,諸蠻膽落。由是境內得安。

二十九年召還,進都督同知。明年復從楊文討平吉州及五開叛苗,與顧成討平水西諸蠻堡,還理左府事。建文元年練兵德州,禦燕師無功。成祖即位,委任如故。命往江西練軍城守,兼節制廣東、福建、湖廣三都司。

廬陵民嘯聚山澤。帝不欲用兵,遣行人許子謨齎敕招諭,命觀臨撫之。觀至,眾皆復業,賜璽書褒勞。命佩征南將軍印,鎮廣西,節制兩廣官軍。帝知觀嗜殺,賜璽書戒之曰:「蠻民易叛難服,殺愈多愈不治。卿往鎮,務綏懷之,毋專殺戮。」會群蠻復叛,帝遣員外郎李宗輔齎敕招之。觀大陳兵示將發狀,而遣使與宗輔俱。桂林蠻復業者六千家,惟思恩蠻未附。而慶遠、柳、潯諸蠻方殺掠吏民,乃上章請討。

永樂元年與指揮葛森等擊斬理定諸縣山賊千一百八十有奇,擒其酋五十餘人,斬以徇。還所掠男女於民,而撫輯其逃散者。明年遣都指揮朱輝諭降宜山、忻城諸山寨。荔波瑤震恐,乞為編戶。帝屬觀撫之,八十餘洞皆歸附。明年,潯、桂、柳三府蠻作亂,已撫復叛,遣硃輝以偏師破之。蠻大懼。會朝廷遣郎中徐子良至,遂來降,歸所掠人畜器械。

四年大發兵討安南,詔觀畫方略,轉粟二十萬石餉軍。已,復命偕大理卿陳洽選土兵三萬會太平,仍令觀偵安南賊中動靜。尋從大兵發憑祥,抵坡壘關,以所部營關下,伐木治橋梁,給軍食。安南平,命措置交阯緣途諸堡,而柳、潯諸蠻乘觀出,復叛。

五年,觀旋師抵柳州。賊望風遁匿,觀請俟秋涼深入,且請濟師。帝使使發湖廣、廣東、貴州三都司兵,又敕新城侯張輔遣都督朱廣、方政以征交阯兵協討。十月,諸軍皆集,分道進剿。觀自以貴州、兩廣兵由柳州攻馬平、來賓、遷江、賓州、上林、羅城、融縣,皆破之。會兵象州,復進武宣、東鄉、桂林、貴平、永福。斬首萬餘級,擒萬三千餘人,群蠻復定。捷聞,帝嘉勞之。

九年拜征夷副將軍,仍佩故印,總兵鎮交阯。明年復命轉粟給張輔軍。輔再出師定交阯,觀皆主饋運,不為將,故功不著。

觀在廣西久,威震南中,蠻人惴惴奉命。繼之者,自山雲外,皆不能及。十二年九月卒,無子。宣德二年,保定伯梁銘奏求觀南京故宅。帝許之。既聞觀妻居其中,曰:「觀,功臣地,雖歿,豈可奪之?」遂不許。令有司以他宅賜銘。

山雲,徐人。父青,以百戶從成祖起兵,積功至都督僉事。雲貌魁梧,多智略。初襲金吾左衛指揮使。數從出塞,有功。時幼軍二十五所,隸府軍前衛,掌衛者不任事,更命雲及李玉等五人撫戢之。仁守立,擢行在中軍都督僉事。

宣德元年改北京行都督府,命偕都御史王彰自山海抵居庸,巡視關隘,以便宜行事。帝征樂安,召輔鄭王、襄王居守。

明年,柳、慶蠻韋朝烈等掠臨桂諸縣。時鎮遠侯顧興祖以不救邱溫被逮,公侯大臣舉雲。帝亦自知之。三年正月命佩征蠻將軍印,充總兵官往鎮。雲至,討朝烈,破之。賊保山巔,山峻險,掛木於藤,壘石其上。官軍至,輒斷藤下木石,無敢近者。雲夜半束火牛羊角,以金鼓隨其後,驅向賊。賊謂官軍至,亟斷藤。比明,木石且盡,眾噪而登,遂盡破之。南安、廣源諸蠻悉下。是夏,忻城蠻譚團作亂,雲討擒之。四年春,討平柳、潯諸蠻。其秋,雒容蠻出掠,遣指揮王綸破之。雲上綸功,並劾其殺良民罪,帝宥綸而心重雲。廣西自韓觀卒後,諸蠻漸橫。雲以廣西兵少,留貴州兵為用,先後討平潯、柳、平樂、桂林、宜山、思恩諸蠻。九年又以慶遠、鬱林苗、瑤非大創不服,請濟師。詔發廣東兵千五百人益雲。雲分道剿捕,擒斬甚眾。復遣指揮田真攻大藤峽賊,破之。

雲在鎮,先後大戰十餘,斬首萬二千二百六十,降賊酋三百七十,奪還男女二千五百八十,築城堡十三,鋪舍五百,陶磚鑿石,增高益厚。自是瑤、僮屏跡,居民安堵。論功,進都督同知,璽書褒勞。

雲謀勇深沉,而端潔不茍取,公賞罰,嚴號令,與士卒同甘苦。臨機應變,戰無不捷。廣西鎮帥初至,土官率饋獻為故事。帥受之,即為所持。雲始至,聞府吏鄭牢剛直,召問曰:「饋可受乎?」牢曰:「潔衣被體,一污不可湔,將軍新潔衣也。」雲曰:「不受,彼且生疑,奈何?」牢曰:「黷貨,法當死。將軍不畏天子法,乃畏土夷乎?」雲曰:「善。」盡卻饋獻,嚴馭之。由是土官畏服,調發無敢後者。雲所至,詢問里老,撫善良,察誣枉,土人皆愛之。

英宗即位,雲墜馬傷股。帝遣醫馳視。以病請代,優詔不許。進右都督。正統二年上言:「潯州與大藤峽諸山相錯,瑤寇出沒,占耕旁近田。左右兩江土官,所屬人多田少。其狼兵素勇,為賊所畏。若量撥田州土兵於近山屯種,分界耕守,斷賊出入。不過數年,賊必坐困。」報可。嗣後東南有急,輒調用狼兵,自此始也。明年冬,卒於鎮。贈懷遠伯,謚忠毅。長子俊,襲府軍前衛指揮使。廣西人思雲不置,立祠肖像祀焉。

初,韓觀鎮廣西,專殺戮。慶遠諸生來迓。觀曰:「此皆賊覘我也。」悉斬之。雲平恕,參佐有罪,輒上請,不妄殺人,人亦不敢犯。鄭牢嘗逮事觀。觀醉,輒殺人。牢輒留之,醒乃以白。牢為士大夫所重,然竟以隸終。

蕭授,華容人。由千戶從成祖起兵,至都指揮同知。永樂十六年擢右軍都督僉事,充總兵官,鎮湖廣、貴州。

宣德元年,鎮遠邛水蠻銀總作亂。指揮祝貴往撫,被殺。授遣都指揮張名破斬之。貴州宣慰所轄乖西、巴香諸峒寨,山箐深險,諸蠻錯居。攻剽他部,傷官軍,發民塚。而昆阻比諸寨亦恃險不輸賦。二年,授遣都指揮蘇保會宣慰宋斌攻破昆阻比寨,窮追,斬偽王以下數百人。乖西諸蠻皆震懾歸命。

水西蠻阿閉妨宜作亂,授結旁寨酋,以計誅之。而西堡蠻阿骨等與寨底、豐寧、清平、平越、普安諸苗復相聚為寇,四川筠連諸蠻應之。授且捕且撫。諸蠻先後聽命,承制赦之。以豐寧酋稔惡,械送京師,伏誅。七年諭降安隆酋岑俊。已,討辰州蠻,擒其酋八十,斬馘無算。移兵擊江華苗,討富川山賊,先後破擒之。

先是,貴州治古、答意二長官司苗數出掠。授築二十四堡,環其地,分兵以戍,賊不得逞。久之,其酋吳不爾覘官軍少,復掠清浪,殺官吏。授遣張名擊破之。賊走湖廣境,結生苗,勢復張。授乃發黔、楚、蜀軍分道捕討。進軍筸子坪,誅不爾,斬首五百九十餘級,賊悉平。九年,都勻蠻為亂,引廣西賊入掠。授遣指揮陳原、顧勇分道邀擊,獲賊首韋萬良等,降下合江蔡郎等五十餘寨。

英宗即位,命佩征蠻副將軍印,鎮守如故。念授年老,以都督僉事吳亮副之。正統元年,普定蠻阿遲等叛,僭稱王,四出攻掠。授遣顧勇等搗其巢,破之。而廣西蒙顧十六洞與湖廣逃民相聚蜂起,授督兵圍之。再戰,悉擒斬其酋,餘黨就誅。捷聞,進右都督。上言:「靖州與廣西接壤,時苦苗患。永樂、宣德間,嘗儲糧數萬石,備軍興。比年儲糧少。有警,發人徒轉輸,賊輒先覺,以故不能得賊。乞於清浪、靖州二衛,各增儲五萬石,庶緩急可借。」報可。

四年,貴州計沙賊苗金蟲、苗總牌糾洪江生苗作亂,偽立「統千侯」、「統萬侯」號。授督兵抵計沙,分遣都指揮鄭通攻三羊洞,馬曄攻黃柏山,大破之。吳亮窮追至蒲頭、洪江,斬總牌,千戶尹勝誘斬金蟲,於是生苗盡降。授沉毅多計算,裨校皆盡其材,而馭軍嚴整。自鎮遠侯顧成歿,群蠻所在屯結。官軍討之,皆無功。授在鎮二十餘年,規畫多本於成。久益明練,威信大行,寇起輒滅,前後諸帥莫及也。論功,進左都督。是年六月召還,以老致仕。尋起視事右府。十年卒。贈臨武伯,謚靖襄。

吳亮,來安人。永樂初,為旗手衛指揮僉事。宣德中,署湖廣都指揮僉事。尋以右副總兵與王瑜督漕運。

英宗初,討新淦賊有功,累進都督僉事,副授鎮湖廣、貴州。破普定蠻,進都督同知。平計沙苗,進右都督。方政歿於麓川,召亮還京,命為副總兵,將兵五萬往討。至雲南,賊益熾,坐金齒參將張榮敗不救,逮下獄。左遷都督僉事,仍佩征南副將軍印,鎮湖廣、貴州,討平四川都掌蠻。尋召還,視右府事。正統十一年卒。

亮姿貌魁梧,性寬簡,不喜殺戮,所至蠻人懷附。好讀書,至老,手不釋卷。

方瑛,都督政之子。正統初,以舍人從父征麓川。父戰死,瑛發憤,矢報父仇。初襲指揮使,已,論政死事功,遷都指揮同知。

六年從王驥征麓川。帥兵六千突賊壘。賊渠衣黃衣帳中。瑛直前,左右擊斬數百人,躪死者無算,遂平其地。進都指揮使。尋復從驥破貢章、沙壩、阿嶺諸蠻。進都督僉事,蒞後府事,充右參將,協守雲南。十三年復從驥征麓川。破鬼山大寨,留鎮雲南。

景泰元年,廷議以瑛有將略,命都督毛福壽代,還,進都督同知。甫抵京,而貴州群苗叛,道梗。驥請瑛還討。其年四月拜右副總兵,與保定伯梁珤、侍郎侯璡次第破走之。進右都督。復破賞改諸寨,擒偽苗王王阿同等。璡卒,都御史王來代督軍務,分道擊賊香爐山。瑛入自龍場,大破平之。

三年秋,來劾瑛違法事,置不問。來召還,命瑛鎮守貴州。其冬,討白石崖賊,俘斬二千五百人,招降四百六十寨。進左都督。五年,四川草塘苗黃龍、韋保作亂,自稱「平天大王」,剽播州西坪、黃灘。瑛與巡撫蔣琳會川兵進剿,賊魁皆就縛。因分兵克中潮山及三百灘、乖西、谷種、乖立諸寨,執偽王谷蟻丁等,斬首七千餘。詔封南和伯。

瑛為將,嚴紀律,信賞罰,臨陣勇敢,善撫士。士皆樂為用,以故數有功。廷臣言宜委以禁旅,乃召還,同石亨督京營軍務。明年,琳奏瑛前守貴州,邊境寧,苗蠻畏服,乞遣還。帝不許。未幾,湖廣苗叛,拜瑛平蠻將軍,率京軍討之,而使御史張鵬偵其後。還奏,瑛所過秋毫不犯,帝大喜。

七年,賊渠蒙能攻平溪衛。都指揮鄭泰等擊卻之,能中火槍死,瑛遂進沅州。連破鬼板等一百六十餘寨。與尚書石璞移兵天柱,率陳友等分擊天堂諸寨,復大破之。克寨二百七十,擒偽侯伯以下一百二人。時英宗已復位。捷聞,璞召還,瑛留鎮貴州、湖廣。瑛討蒙能餘黨,克銅鼓藕洞一百九十五寨,覃洞、上隆諸苗各斬其渠納款。帝嘉瑛功,進侯。天順二年,東苗干把豬等僭偽號,攻都勻諸衛。命瑛與巡撫白圭合川、湖、雲、貴軍討之,克六百餘寨。邊方悉定。瑛前後克寨幾二千,俘斬四萬餘。平苗之功,前此無與比者。尋卒於鎮,年四十五。帝震悼,賜謚忠襄。

瑛天姿英邁,曉古兵法。嘗上練兵法及陣圖,老將多稱之。為人廉,謙和不伐。所至鎮以安靜,民思之,久而不忘。

子毅,嗣伯爵,誘祖母誣從父瑞不孝,坐奪爵閒住。卒,子壽祥嗣。正德中,歷鎮貴州、湖廣。傳爵至明亡乃絕。

陳友,其先西域入,家全椒。正統初,官千戶,累遷都指揮僉事。頻年使瓦剌有勞,尋復進都指揮使。九年充寧夏游擊將軍,與總兵官黃真擊兀良哈。多獲,進都督僉事。未幾,出塞招答哈卜等四百人來歸。

景帝即位,進都督同知,征湖廣、貴州苗。尋充左參將,守備靖州。景泰二年偕王來等擊賊香爐山,自萬潮山入,大破之。留鎮湖廣。論功,進右都督。四年春奏斬苗五百餘級,五年又奏斬苗三百餘。而都指揮戚安等八人戰死,兵部疑首功不實,指揮蔡昇亦奏友欺妄。命總督石璞廉之,斬獲僅三四十人,陷將士千四百人,宜罪。詔令殺賊自效。天順元年隨瑛徵天堂諸苗,大獲。命充左副總兵,仍鎮湖廣。已,又偕瑛破蒙能餘黨。召封武平伯,予世券。孛來犯邊,充游擊將軍,從安遠侯柳溥等往禦。率都指揮趙瑛等與戰,敵敗遁。再犯鎮番,復擊卻之,俘百六十人。尋佩將軍印,充總兵官,討寧夏寇。先是,寇大入甘、涼,溥及總兵衛穎等不能禦,惟友稍獲。至是巡撫芮釗列諸將失事狀,兵部請免友罪。詔並宥溥等。召還,進侯,卒。

傳子至孫綱,弘治中,請友贈謚。詔贈沔國公,謚武僖。綱傳子勳及熹。嘉靖中,吏部以友征苗功多冒濫,請停襲。帝不從。熹子大策復得嗣,至明亡乃絕。

李震,南陽人。父謙,都督僉事,震襲指揮使。正統九年從征兀良哈有功,進都指揮僉事。已,從王驥平麓川,進同知。

景帝即位,充貴州右參將。擊苗於偏橋,敗之。景泰二年從王來征韋同烈。破鎖兒、流源諸寨,俘斬千六百人,共克香爐山,獲同烈。進都指揮使,守靖州。尋坐罪徵還。方瑛討苗,乞震隨軍,詔許立功贖。已,從瑛大破天堂諸苗,仍充左參將。瑛平銅鼓諸賊,震亦進武岡,克牛欄等五十四寨。斬獲多,進都督僉事。

天順中,復從瑛平貴東苗干把豬。瑛卒,即以震充總兵官,代鎮貴州、湖廣。初,麻城人李添保以逋賦逃入苗中。偽稱唐太宗後,眾萬餘,僭王,建元「武烈」,剽掠遠近。震進擊,大破之。添保遁入貴州鬼池諸苗中,復誘群苗出掠。震擒之,送京師。尋破西堡苗。

五年春剿城步瑤、僮,攻橫水、城溪、莫宜、中平諸寨,皆破之。長驅至廣西西延,會總兵官過興軍,克十八團諸瑤,前後俘斬數千人。其冬命震專鎮湖廣,以李安充總兵,守貴州。明年夏率師由錦田、江華抵雲川、桂嶺、橫江諸寨,破瑤,俘斬二千八百餘人。七年冬,苗據赤谿湳洞長官司。震與安分道進,斬賊渠飛天侯等,破寨二百,遂復長官司。進都督同知。明年冬,廣西瑤侵湖南,夜入桂陽州大掠。震遣兵分道追擊,連敗之,俘斬千餘人。

成化改元,守備靖州。都指揮同知莊榮奏貴州黎平諸府密邇湖廣五開諸衛,非大將總領不可,乃復命震兼鎮貴州。未幾,獲賊首苗蟲蝦。

荊、襄賊劉千斤、石和尚為亂,震進討。賊屢敗,乘勝追及於梅溪賊巢。官軍不利,都指揮以下死者三十八人,有詔切責。白圭等大軍至,震自南漳進兵合擊,大破之,賊遂平。論功,進右都督。

時武岡、沅靖、銅鼓、五開苗復蜂起,而貴州亦告警。震言貴州終難遙制,請專鎮湖廣。許之,乃還兵。由銅鼓、天柱分四道進,連破賊,直抵清水江。因苗為導,深入賊境。兩月間破巢八百,焚廬舍萬三千,斬獲三千三百。而廣西瑤劫桂陽者,亦擊斬三千八百有奇。當是時,震威名著西南,苗、僚聞風畏懾,呼為「金牌李」。七年,與項忠討平流賊李原,招撫流民九十萬人,荊、襄遂定。語具忠傳。

十一年,苗復犯武岡、靖州,湖湘大擾。震與巡撫劉敷等分五道進,破六百二十餘寨,俘斬八千五百餘人,獲賊孥萬計。論功封興寧伯。時武靖侯趙輔、寧晉伯劉聚皆以功封,論者多訾議之,獨震功最高,人無異言。

參將吳經者,與震有隙。弟千戶綬為汪直腹心,經屬綬譖之。會直方傾項忠,詞連震,遂逮下獄。奪爵,降左都督,南京閒住。未幾,直遣校尉緝事,言震陰結守備太監覃包,私通貨賂。帝怒,遣直赴南京數包等罪,責降包孝陵司香,勒震回京。直敗,震訴復爵,尋卒。

震在湖湘久,熟知苗情,善用兵。一時征苗功,方瑛後震為最。然貪功好進,事交結,竟以是敗。

王信,字君實,南鄭人。生半歲,父忠征北戰歿,母岳氏苦節育之,後俱獲旌。正統中,信襲寬河衛千戶。

成化初,積功至都指揮僉事,守備荊、襄。劉千斤反,信以房縣險,進據之。民兵不滿千人,賊眾四千突至,圍其城。拒四十餘日,選死士,出城五六里舉砲。賊疑援至,驚走,追敗之。已,白圭統大軍至,以信為右參將,分道抵後巖山,賊遂滅。論功,進都指揮同知。賊黨石龍復陷巫山,信與諸將共平之。而流民仍嘯荊、襄、南陽間。信以為憂,言於朝,即命信兼督南陽軍務。賊首李原等果亂,信復與項忠討平之。擢署都督僉事,鎮守臨清。

十三年以本官佩平蠻將軍印,移鎮湖廣。永順、保靖二宣慰世相仇殺,信諭以禍福,兵即解。靖州及武岡蠻久不戢,守臣議剿之。信親詣,犒以牛酒,責其無狀,眾稽顙服罪。

十七年疏言:「湖廣諸蠻雖腹心蠹,實無能為。久不靖者,由我將士利其竊發以邀功也。選精銳,慎隄防,其患自息。荊、襄流逋,本避徭役,濫誅恐傷天和。南畝之氓咸無蓄積。收獲未竟,餱糧已空;機杼方停,布縷何在。乞選公正仁惠守令,加意撫綏。濫授冗員,無慮千百,無一矢勞,冒崇階之賞,乞察勘削奪。」部指揮劉斌、張全智勇,力薦於朝。且云:「英雄之士,處心剛正,安肯俯首求媚。若不加意延訪,則志士沉淪,朝廷安得而用之。」

二十一年,巡撫馬馴等言,副總兵周賢、參將彭倫官皆都督僉事,而信反止署職,宜量進一秩以重其權。兵部言信無軍功。帝特擢為都督同知。頃之,改總督漕運。帥府舊有湖,擅為利,信開以泊漕艘。勢要壅水,一裁以法,漕務修舉。明年卒。

信沉毅簡重,好觀書,被服儒雅。歷大鎮,不營私產。嘗曰:「儉足以久,死後不累子孫,所遺多矣。」故人婚喪,傾資助之。子繼善、從善皆舉進士。

繼信總漕運者,寧津都勝、合肥郭鋐。勝襲職南京羽林左衛指揮僉事,金宏襲彭城衛指揮使。成化初,勝擢署都指揮僉事,而鋐亦以從征荔浦功,進都指揮僉事,中武舉,遷同知。勝備倭揚州,擊敗鹽徒為亂者。尹旻等舉勝將才,鋐亦為張懋所舉。乃命勝充參將,協同漕運,而鋐代之備倭。陜西大饑,勝奉詔輸米百萬石往振。信卒,遂遷署都指揮使,充總兵官代之,鋐代勝為參將。弘治中,勝以都督僉事帶俸南京前府。時鋐已鎮守廣西副總兵,破府江僮賊,遂以時望擢總漕運。

鋐沉毅有將略。而勝無汗馬勛,徒以居官廉靜,故頻有任使。歷任五十七年,所處皆膏腴地,而自奉簡淡,日食止豆腐,時因以為號。鋐累進都督同知,凡軍民利病多陳於朝。嘗浚通州河二十里,置壩,令淺船搬運,歲省白金數萬。當孝宗時,朝政整肅,文武大臣率得人,鋐筦漕十三年不易。正德初,始召佐後府,尋卒。

彭倫,初職為湖廣永定衛指揮使,累功至都指揮同知。

成化初,從趙輔平大藤峽賊。進都指揮使,守備貴州清浪諸處,討破茅坪、銅鼓叛苗。賊掠乾溪,倫討之。賊還所掠,與盟而退。倫以賊入時,道邛水諸寨,不即邀遏,乃下令,賊入境能生致者予重賞,縱者置諸法。由是諸司各約所屬,凡生苗軼入,即擒之,送帳下者纍纍。倫大會所部目、把縛俘囚,置高竿,集健卒亂射殺之,復割裂肢體,烹啖諸壯士。罪輕者截耳鼻使去,曰:「以此識,再犯不赦矣。」因令諸寨樹牌為界,群苗股栗不敢犯。

明年充右參將,仍鎮清浪。益盡心邊計,戎事畢舉。妖賊石全州潛入絞洞,煽動古州苗,洪江、甘篆諸苗咸應之。倫遣兵截擒,並搜獲其妻子。諸苗將攻鎮遠,倫大敗之,斬首及墮崖死者無算。無何,邛水十四寨苗糾洪江生苗為逆。倫分五哨往,甫行,雨如注,倫曰:「賊不虞我,急趨之,可得志也。」競進夾攻,縶其魁,俘斬餘黨。賊盡平。

靖州苗亂,湖廣總兵官李震檄倫會討。軍至邛水江,諸熟苗驚,欲竄。倫與僉事李晃計曰「苗竄必助賊」,乃急撫定之。又緣道降天堂、小坪諸苗。既抵靖州,倫將右哨,出賊背布營。賊走據高山,倫軍仰攻之,賊敗走。遂渡江,搗其巢,大獲。乘勝攻白崖塘。崖高萬仞,下臨深淵,稱絕險。倫會左哨同進,得徑路。夜登,賊倉皇潰。追斬二千餘級,俘獲如之,盡夷其寨。

初,臻剖、六洞苗侵熟苗田,不輸賦,又不供驛馬,有司莫敢問。倫遣人諭之,頓首請如制。錄功,進都督僉事。久之,御史鄧庠、員外郎費瑄勘事貴州,總兵官吳經等皆被劾,獨薦倫智謀老成。弘治初,經論罷,即以倫代。

倫用師,先計後戰,故多功。四年以老致仕。卒,予恤如制。

歐磐,滁人。襲世職指揮使。成化中,擢廣東都指揮僉事。屢剿蠻寇有功。用總督朱英薦,充廣西右參將,分守柳州、慶遠。與左參將馬義討融縣八寨瑤,克之。師旋。餘賊復出掠,被劾。帝絀磐等功,但恤死事家。瑤賊方公強亂,兵部劾總鎮中官顧恒,並及磐,當謫戍。督撫奏:「磐所守乃瑤、僮出沒地。磐募死士,夜入賊巢,斬其渠胡公返,威震群蠻。論功,可贖罪。」帝乃宥之,還故任。二十三年,鬱林陸川賊黃公定、胡公明等亂。磐偕按察使陶魯等分五道攻破之。進都指揮同知。

弘治初,謝病解職。總督秦紘言磐多歷戰陣,有才有守,乞起用。詔還任。八年,府江永安諸僮亂。總督閔珪調兵六萬,分四哨往討。磐自象州、修仁直搗陸峒,所向摧破。已,偕諸軍連破山寨百八十,斬首六千有奇。進都指揮使,遷廣西副總兵。思恩土官岑浚築石城於丹良莊,截江括商利。帥府令毀之,不聽。磐自田州還,督兵將毀城。浚率眾拒,擊敗之,卒夷其城。都御史鄧廷瓚等以磐功多,言於朝,進都督僉事。十五年命佩平蠻將軍印,鎮守湖廣。

磐為將廉,能得士。久鎮南邦,蠻人畏服。十八年請老,又二年卒。祭葬如制。

張祐,字天祐,廣州人。幼好學能文。弘治中襲世職為廣州右衛指揮使。年十九,從總督潘蕃征南海寇CK元祖,先登有功。

正德二年擢署都指揮僉事,守備德慶、瀧水。瑤、僮負險者聞其威信,稍稍遁去。總督林廷選引為中軍,事無大小咨焉。守備惠、潮,搗盜魁劉文安、李通寶穴,平之。遷廣西右參將,分守柳、慶。總督陳金討府江賊,命祐進沈沙口,大破之。增俸一等,擢副總兵,鎮守廣西。尋進署都督僉事。

古田諸瑤、僮亂。祐言:「先年征討,率倚兩江土兵,賞不酬勞。今調多失期,乞定議優賚。」從之。督都指揮沈希儀等討臨桂、灌陽諸瑤,斬首五百餘級,璽書獎勞。又連破古田賊,俘斬四千七百,進署都督同知。已,復討平洛容、肇慶、平樂諸蠻。增俸一等,蔭子,世百戶。

嘉靖改元,母喪,哀毀骨立。尋以疾乞休,還衛。

初,上思州土目黃鏐作亂,祐購其黨黃廷寶縛獻之。總督張嵿惡祐不白己,至劾祐懷奸避難,逮繫德慶獄。數上書訟冤,釋令閒住。盧蘇、王受亂田州。總督姚鏌召至軍中,待以賓禮,多所裨贊。後王守仁代鏌,詢撫剿之宜,祐曰:「以夷治夷,可不煩兵而下。」守仁納之,蘇、受果效順。因命祐部分其眾。事寧,守仁言:「思、田初定,宜設一副總兵鎮之,請即以命祐。」報可。破封川賊盤古子,又剿廣東會寧劇賊丘區長等,斬首一千二百,勒銘大隆山。

十一年,楊春賊趙林花陷高州,總督陶諧檄祐討。深入,多所斬獲。忽中危疾卒,軍中為哀慟。

祐身長八尺,智識絕人。馭軍有節制,與下同甘苦,不營私產。性好書,每載以自隨,軍暇即延儒生講論。嘗過烏蠻灘,謁馬伏波祠,太息曰:「歿不俎豆其間,非夫也。」題詩而去。後田州人立祠橫山祀之。

贊曰:苗蠻阻險自固,易動難服,自其性然。而草薙禽獮,濫殺邀功,貪貨賄,興事端,控馭乖方,綏懷無策,則鎮將之過也。韓觀諸人,雖功最焯著,而皆以威信震懾蠻荒。若山雲、王信、張祐之廉儉有守,士君子何以過?故尤足尚云。


\end{pinyinscope}