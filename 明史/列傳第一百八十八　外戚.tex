\article{列傳第一百八十八 外戚}

\begin{pinyinscope}
明太祖立國,家法嚴。史臣稱后妃居宮中,不預一髮之政,外戚循理謹度,無敢恃寵以病民,漢、唐以來所不及。而高、文二后賢明,抑遠外氏。太祖訪得高后親族,將授以官。后謝曰:「國家爵祿,宜與賢士大夫共之,不當私妾家。」且援前世外戚驕佚致禍為辭。帝善后言,賜金帛而已。定國之封,文皇后謂非己志,臨終猶勸帝,毋驕畜外家。詒謀既遠,宗社奠安,而椒房貴戚亦藉以保福慶逮子孫,所全不已多乎。惟英宗時,會昌侯孫繼宗以奪門功,參議國是。自茲以下,其賢者類多謹身奉法,謙謙有儒者風。而一二怙恩負乘之徒,所好不過田宅、狗馬、音樂,所狎不過俳優、伎妾,非有軍國之權,賓客朋黨之勢。而在廷諸臣好為危言激論,汰如壽寧兄弟,庸駑如鄭國泰,已逐影尋聲,抨擊不遺餘力。故有明一代,外戚最為孱弱。然而惠安、新樂,舉宗殉國,嗚呼卓矣!成祖后家,詳《中山王傳》,餘採其行事可紀者,作《外戚傳》。

○陳公馬公呂本馬全張麒子昶升等胡榮孫忠子繼宗吳安錢貴汪泉杭昱周能子壽彧王鎮子源等萬貴邵喜張巒夏儒陳萬言方銳陳景行李偉王偉鄭承憲王昇劉文炳弟文耀等張國紀周奎

陳公,逸其名,淳皇后父也。洪武二年追封揚王,媼為王夫人,立祠太廟東。明年有言王墓在盱眙者,中都守臣按之信。帝乃命中書省即墓次立廟,設祠祭署,奉祀一人,守墓戶二百一十家,世世復。帝自製《揚王行實》,諭翰林學士宋濂文其碑,略曰:

王姓陳氏,世維揚人,不知其諱。當宋季,名隸尺籍伍符中,從大將張世傑扈從祥興。至元己卯春,世傑與元兵戰,師大潰,士卒多溺死。王幸脫死達岸,與一二同行者,累石支破釜,煮遺糧以療饑。已而絕糧,同行者聞山有死馬,將其烹食之。王疲極晝睡,夢一白衣人來曰:「汝慎勿食馬肉,今夜有舟來共載也。」王未之深信,俄又夢如初。至夜將半,夢中仿佛聞櫓聲,有衣紫衣者以杖觸王胯曰:「舟至矣。」王驚寤,身已在舟上,見舊所事統領官。

時統領已降於元將,元將令來附者輒擲棄水中。統領憐王,藏之艎板下,日取乾餱從板隙投之,王掬以食。復與王約,以足撼板,王即張口從板隙受漿。居數日,事洩,徬徨不自安。颶風吹舟,盤旋如轉輪,久不能進,元將大恐。統領知王善巫術,遂白而出之。王仰天叩齒,若指麾鬼神狀,風濤頓息。元將喜,因飲食之。至通州,送之登岸。

王歸維揚,不樂為軍伍,避去盱眙津里鎮,以巫術行。王無子,生二女,長適季氏,次即皇太后。晚以季氏長子為後,年九十九薨,遂葬焉,今墓是已。

臣濂聞君子之制行,能感於人固難,而能通於神明為尤難。今當患難危急之時,神假夢寐,挾以升舟,非精誠上通於天,何以致神人之佑至於斯也。舉此推之,則積德之深厚,斷可信矣。是宜慶鐘聖女,誕育皇上,以啟億萬年無疆之基,於乎盛哉!

臣濂既序其事,復再拜稽首而獻銘曰:皇帝建國,克展孝思。疏封母族,自親而推。錫爵維揚,地邇帝畿,立廟崇祀,玄冕袞衣。痛念宅兆,卜之何墟,閭師來告,今在盱眙。皇情悅豫,繼以涕洟,即詔禮官,汝往葺治,毋俾蕘豎,跳踉以嬉。惟我揚王,昔隸戎麾,獰風蕩海,糧絕阻饑。天有顯相,夢來紫衣,挾以登舟,神力所持,易死為生,壽躋期頤。積累深長,未究厥施,乃毓聖女,茂衍皇支。蘿圖肇開,鴻祚峨巍,日照月臨,風行霆馳。自流徂源,功亦有歸,無德弗酬,典禮可稽。聿昭化原,扶植政基,以廣孝治,以惇民彞。津里之鎮,王靈所依,於昭萬年,視此銘詩。

馬公,逸其名,高皇后父也,宿州人。元末殺人,亡命定遠。與郭子興善,以季女屬子興,後歸太祖,即高皇后也。

公及妻鄭媼皆前卒,洪武二年追封徐王,媼為王夫人,建祠太廟東。皇后親奉安神主,祝文稱「孝女皇后馬氏,謹奉皇帝命致祭。」四年命禮部尚書陶凱即宿州塋次立廟,帝自為文以祭。

文曰:「朕惟古者創業之君,必得賢后以為內助,共定大業。及天下已安,必追崇外家,以報其德。惟外舅、外姑實生賢女,正位中宮。朕既追封外舅為徐王,外姑為王夫人,以王無繼嗣,立廟京師,歲時致祭。然稽之古典,於禮未安。又念人生其土,魂魄必遊故鄉,故即塋所立廟,俾有司春秋奉祀。茲擇吉辰,遣禮官奉安神主於新廟,靈其昭格,尚鑒在茲。」

二十五年設祠祭署,奉祀、祀丞各一人。王無後,以外親武忠、武聚為之,置灑掃戶九十三家。永樂七年北巡,親謁祠下。守塚武戡為建陽衛鎮撫,犯法,責而宥之。十五年,帝復親祭,以戡為徐州衛指揮僉事。

呂本,壽州人,懿文太子次妃父也。仕元,為元帥府都事。後歸太祖,授中書省令史。洪武五年歷官吏部尚書。六年改太常司卿。明年四月,御史臺言:「本奉職不謹,郊壇牲角非繭慄,功臣廟壞不修。」詔免官,罰役功臣廟。已,釋為北平按察司僉事。帝召本及同時被命楊基、答祿與權,諭之曰:「風憲之設,在肅紀綱,清吏治,非專理刑名。爾等往修厥職,務明大體,毋傚俗吏拘繩墨。善雖小,為之不已,將成全德;過雖小,積之不已,將為大憝。不見干雲之臺,由寸土之積,燎原之火,由一爝之微,可不慎哉!」本等頓首受命,尋復累遷太常司卿。逾二年卒,無子,賜葬鐘山之陰。

馬全,洪武中為光祿少卿。其女,乃惠帝后也。燕兵陷都城,全不知所終。

張麒,永城人。洪武二十年以女為燕世子妃,授兵馬副指揮。世子為太子,進京衛指揮使,尋卒。仁宗即位,追封彭城伯,謚恭靖,後進侯。二子昶、昇,並昭皇后兄也。

昶從成祖起兵取大寧,戰鄭村壩,俱有功,授義勇中衛指揮同知。已,援蘇州,敗遼東軍,還佐世子守北平。永樂初,累官錦衣衛指揮使。昶嘗有過,成祖戒之曰:「戚畹最當守法,否則罪倍常人。汝今富貴,能不忘貧賤,驕逸何自生。若奢傲放縱,陵虐下人,必不爾恕,慎之。」昶頓首謝。仁宗立,擢中軍都督府左都督,俄封彭城伯,子孫世襲。洪熙改元,命掌五軍右哨軍馬。英宗嗣位,年幼,太皇太后召昶兄弟誡諭之,凡朝政弗令預。昶兄弟素恭謹,因訓飭益自斂。正統三年卒。

長子輔病廢,子瑾嗣。以伯爵封輔,命未下而輔卒。初,昶私蓄奄人,瑾匿不舉。事發,下獄,已,獲釋。瑾從弟,天順中,官錦衣衛副千戶。飲千戶呂宏家,醉抽刀刺宏死,法當斬,有司援議親末滅。詔不從,迄如律。成化十六年,瑾卒,子信嗣。其後裔嗣封,見《世表》。

升,字叔暉。成祖起兵,以舍人守北平有功,授千戶,歷官府軍衛指揮僉事。永樂十二年從北征。仁宗即位,拜後府都督同知。宣德初,進左都督掌左府事。四年二月敕論昇曰:「卿舅氏至親,日理劇務,或以吏欺謾連,不問則廢法,問則傷恩,其罷府事,朝朔望,官祿如舊,稱朕優禮保全之意。」九年北征,命掌都督府事,留守京師。英宗立,太皇太后令勿預政。大學士楊士奇稱昇賢,宜加委任,終不許。正統五年,兄昶已前卒,太后念外氏惟升一人,封惠安伯,予世襲。明年卒。

子沄早亡,孫琮嗣。琮卒,弟瑛嗣。瑛卒,無子,庶兄瓚嗣。瓚卒,子偉嗣。弘治十二年充陜西總兵官,鎮守固原。明年五月,孝宗御平臺,出兵部推舉京營大將疏,歷詢大學士劉健等,僉稱偉才。命提督神機營,御書敕以賜。正德元年令參英國公張懋、保國公朱暉提督團營。三年加太子太保。六年三月充總兵官,偕都御史馬中錫督京兵討流賊劉六等。朝議以偉擁兵自衛,責其玩寇殃民,召還。御史吳堂復劾其罪,兵部請逮偉及中錫,下獄論死。遇赦獲釋,停祿閒住。十年請給祿,詔給其半。十五年復督神機營。嘉靖初,兼提督團營。二年敘奉迎防守功,加太子太傅。十四年卒,贈太傅,謚康靖。

子鑭嗣。二十年,言官劾勛戚權豪家置店房、科私稅諸罪,鑭亦預,輸贖還爵。二十七年掌後府事。居三年卒。子元善嗣。隆慶四年僉書後府事。萬曆三十七年卒。子慶臻嗣。四十八年掌左府事。崇禎元年七月命提督京營。慶臻私請內閣,於敕內增入兼管捕營。捕營提督鄭其心訐慶臻侵職,帝怒,詰改敕故。大學士劉鴻訓至遣戍,慶臻以世臣停祿三年。後復起,掌都督府。十七年,賊陷都城,慶臻召親黨盡散貲財,闔家自燔死。南渡時,贈太師、惠安侯,謚忠武,合祀旌忠祠。初,世宗嘉靖八年革外戚世爵,惟彭城、惠安獲存,慶臻卒殉國難。

胡榮,濟寧人。洪武中,長女入宮為女官,授錦衣衛百戶。永樂十五年將冊其第三女為皇太孫妃,擢光祿寺卿,子安為府軍前衛指揮僉事,專侍太孫,不蒞事。後太孫踐阼,妃為皇后,安亦屢進官。宣德三年,后廢,胡氏遂不振。

孫忠,字主敬,鄒平人。初名愚,宣宗改曰忠。初,以永城主簿督夫營天壽山陵,有勞,遷鴻臚寺序班,選其女入皇太孫宮。宣宗即位,冊貴妃,授忠中軍都督僉事。三年,皇后胡氏廢,貴妃為皇后,封忠會昌伯。嘗謁告歸里,御製詩賜之,命中官輔行。比還,帝后臨幸慰勞。妻董夫人數召入宮,賜齎弗絕。正統中,皇后為皇太后。忠生日,太后使使賜其家。時王振專權,祭酒李時勉荷校國學門,忠附奏曰:「臣荷恩厚,願赦李祭酒使為臣客。坐無祭酒,臣不歡。」太后立言之帝,時勉獲釋。忠家奴貸子錢於濱州民,規利數倍,有司望風奉行,民不堪,訴諸朝,言官交章劾之。命執家奴戍邊,忠不問。景泰三年卒,年八十五,贈會昌侯,謚康靖。英宗復辟,加贈太傅、安國公,改謚恭憲。成化十五年再贈太師、左柱國。子五人:繼宗、顯宗、紹宗、續宗、純宗。

純宗官錦衣衛指揮僉事,早卒。

繼宗,字光輔,章皇后兄也。宣德初,授府軍前衛指揮使,改錦衣衛。景泰初,進都指揮僉事,尋襲父爵。天順改元,以奪門功,進侯,加號奉天翊衛推誠宣力武臣,特進光祿大夫、柱國,身免二死,子免一死,世襲侯爵;諸弟官都指揮僉事者,俱改錦衣衛。復自言:「臣與弟顯宗率子、婿、家奴四十三人預奪門功,乞加恩命。」由是顯宗進都指揮同知,子璉授錦衣衛指揮使,婿指揮使武忠進都指揮僉事,蒼頭輩授官者十七人。五月,命督五軍營戎務兼掌後軍都督府事。

左右又有為紹宗求官者,帝召李賢謂曰:「孫氏一門,長封侯,次皆顯秩,子孫二十餘人悉得官,足矣。今又請以為慰太后心,不知初官其子弟時,請於太后,數請始允,且不懌者累日,曰:『何功於國,濫授此秩,物盛必衰,一旦有罪,吾不能庇矣。』太后意固如此。」賢稽首頌太后盛德,因從容言祖宗以來,外戚不典軍政。帝曰:「初內侍言京營軍非皇舅無可屬,太后實悔至今。」賢曰:「侯幸淳謹,但後此不得為故事耳。」帝曰:「然。」已,錦衣逯杲奏英國公張懋、太平侯張瑾及繼宗、紹宗並侵官地,立私莊。命各首實,懋等具服,乃宥之,典莊者悉逮問,還其地於官。石亨之獲罪也,繼宗為顯宗、武忠及子孫、家人、軍伴辭職,帝止革家人、軍伴之授職者七人,餘不問。五年,曹欽平,進太保。尋以疾奏解兵柄,辭太保,不允。

憲宗嗣位,命繼宗提督十二團營兼督五軍營,知經筵事,監修《英宗實錄》。朝有大議,必繼宗為首。再核奪門功,惟繼宗侯如故。乞休,優詔不許。三年八月,《實錄》成,加太傅。十年,兵科給事中章鎰疏言:「繼宗久司兵柄,尸位固寵,亟宜罷退,以全終始。」於是繼宗上疏懇辭,帝優詔許解營務,仍蒞後府事,知經筵,預議大政。復辭,帝不許,免其奏事承旨。自景泰前,戚臣無典兵者,帝見石亨、張軏輩以營軍奪門,故使外戚親臣參之,非故事也。又五年卒,年八十五,贈郯國公,謚榮襄。再傳至曾孫杲,詳《世表》中。

吳安,丹徒人。父彥名,有女入侍宣宗於東宮,生景帝。宣德三年冊為賢妃,彥名已卒,授安錦衣衛百戶。景帝嗣位,尊妃為皇太后,安進本衛指揮使。屢遷前府左都督,弟信亦屢擢都督僉事。景泰七年封安安平伯。信早亡,官其弟敬為南京前軍左都督。英宗復辟,太后復稱賢妃,降安為府軍前衛指揮僉事。敬及其群從南京錦衣衛指揮僉事智、府軍前衛指揮同知喜山、指揮僉事廣林、錦衣衛千戶誠,俱革職原籍閒住。尋命安為錦衣衛指揮使,子孫世襲。

錢貴,海州人,英宗睿皇后父也。祖整,從成祖起兵,為燕山護衛副千戶。父通嗣職,官至金吾右衛指揮使。貴嗣祖職,數從成祖、宣宗北征,屢遷都指揮僉事。正統七年,後將正位中宮,擢貴中府都督同知。英宗數欲封之,后輒遜謝,故后家獨不獲封。

貴卒,長子欽為錦衣衛指揮使,與弟鐘俱歿於土木。欽無子,以鐘遺腹子雄為後,年幼,以父錦衣故秩予優給。天順改元,累擢都督同知。成化時,后崩。憲宗優生母外家周氏,而薄錢氏,故后家又不獲封。雄卒,子承宗亦屢官錦衣衛都指揮使。弘治二年,承宗祖母王氏援憲宗外家王氏例,請封。乃封承宗安昌伯,世襲。先是,勛臣莊田租稅皆有司代收,至是王氏乞自收,始命願自收者聽,而禁管莊者橫肆。嘉靖五年,承宗卒,謚榮僖。子維圻嗣。尋卒,承宗母請以庶長子維垣嗣,詔授錦衣衛指揮使。已又請嗣伯爵。世宗以外戚世封非祖制,下廷臣議。八年十月上議曰:「祖宗之制,非軍功不封。洪熙時,都督張昶封彭城伯,弟昇亦封惠安伯,外戚之封,自此始。循習至今,有一門數貴者,歲糜厚祿,踰分非法。臣等謹議:魏、定二公雖係戚里,實佐命元勛,彭城、惠安二伯即以恩澤封,而軍功參半。其餘外戚恩封,毋得請襲。有出特恩一時寵錫者,量授指揮,千、百戶之職,終其身。」制曰:「可。」命魏、定、彭城、惠安襲封如故,餘止終本身,著為令。維垣遂不得襲,以錦衣終。

汪泉,世為金吾左衛指揮使,家京師。正統十年,其子瑛有女將冊為郕王妃,授瑛為中城兵馬司指揮,食祿不視事。妃正位中宮,進泉都指揮同知府軍衛,帶俸,瑛錦衣衛指揮使。尋並擢左都督,瑛弟亦授錦衣千戶有差。英宗復位,泉仍居金吾舊職,瑛錦衣舊職,其四弟皆奪官還故里。尋命瑛錦衣指揮僉事,子孫世襲。

杭昱,女為景帝妃,生子見濟。景泰三年,帝欲廢英宗子而立己子,乃廢皇后汪氏,冊妃為后。昱累官錦衣衛指揮使。兄聚授錦衣千戶。聚尋卒,賜賻及祭葬。七年,后崩,官其弟敏錦衣百戶。英宗復闢,盡奪景帝所授外親官,尤惡杭氏,昱已前卒,敏削職還里。

周能,字廷舉,昌平人。女為英宗妃,生憲宗,是為孝肅皇太后。英宗復位,授能錦衣衛千戶,賜齎甚渥。能卒,長子壽嗣職。憲宗踐阼,擢左府都督同知。成化三年封慶雲伯,贈能慶雲侯。壽以太后弟,頗恣橫。時方禁勛戚請乞莊田,壽獨冒禁乞通州田六十二頃,不得已與之。嘗奉使,道呂梁洪,多挾商艘。主事謝敬不可,壽與哄,且劾之,敬坐落陽。十七年進侯,子弟同日授錦衣官者七人,能追贈太傅、寧國公,謚榮靖。孝宗立,壽加太保。時壽所賜莊田甚多,其在寶坻者已五百頃,又欲得其餘七百餘頃,詭言以私財相易。部劾其貪求無厭,執不許,孝宗竟許之。又與建昌侯張延齡爭田,兩家奴相毆,交章上聞。又數撓鹽法,侵公家利,有司厭苦之。十六年加太傅,弟長寧伯彧亦加太保,兄弟並為侯伯,位三公,前此未有也。武宗立,汰傳奉官,壽子姪八人在汰中,壽上章乞留,從之。正德四年卒,贈宣國公,謚恭和。

子瑛嗣,封殖過於父。嘉靖中,於河西務設肆邀商貨,虐市民,虧國課,為巡按御史所劾,停祿三月。而瑛怙惡如故,又為主事翁萬達所劾,詔革其廛肆,下家人於法司。時已革外戚世爵,瑛卒,遂不得嗣。

彧,太后仲弟也。成化時,累官左府都督同知。二十一年封長寧伯,世襲。弘治中,外戚經營私利,彧與壽寧侯張鶴齡至聚眾相鬥,都下震駭。九年九月,尚書屠滽偕九卿上言:

憲宗皇帝詔,勛戚之家,不得占據關津陂澤,設肆開廛,侵奪民利,違者許所在官司執治以聞。皇上踐極,亦惟先帝之法是訓是遵。而勛戚諸臣不能恪守先詔,縱家人列肆通衢,邀截商貨,都城內外,所在有之。觀永樂間榜例,王公僕從二十人,一品不過十二人。今勛戚多者以百數,大乖舊制。其間多市井無賴,冒名罔利,利歸群小,怨叢一身,非計之得。邇者長寧伯周彧、壽寧侯張鶴齡兩家,以瑣事忿爭,喧傳都邑,失戚里之觀瞻,損朝廷之威重。伏望綸音戒諭,俾各修舊好。凡在店肆,悉皆停止。更敕都察院揭榜禁戒,擾商賈、奪民利者,聽巡城巡按御史及所在有司執治。仍考永樂間榜例,裁定勛戚家人,不得濫收。

科道亦以為言,帝嘉納之。十八年進太保。彧求為侯,吏部言封爵出自朝廷,無請乞者,乃止。武宗立,悉擢彧子瑭等六人為錦衣官。彧尋卒。傳子瑭,孫大經,及曾孫世臣,降授錦衣衛指揮同知。

先是,孝肅有弟吉祥,兒時出游,去為僧,家人莫知所在,孝肅亦若忘之。一夕,夢伽藍神來,言后弟今在某所,英宗亦同時夢。旦遣小黃門,以夢中言物色,得之報國寺伽藍殿中,召入見。后且喜且泣,欲爵之不可,厚賜遣還。憲宗立,為建大慈仁寺,賜莊田數百頃。其後,周氏衰落,而慈仁寺莊田久猶存。

王鎮,字克安,上元人,憲宗純皇后父也。成化初,授金吾左衛指揮使。未幾,后將正位中宮,拜中軍都督同知。四年進右都督。鎮為人厚重清謹,雖榮寵,不改其素,有長者稱。十年六月卒。弘治六年追封阜國公,謚康穆。子三人:源,清,濬。

源,字宗本,后弟也。父卒,授錦衣衛都指揮使。外戚例有賜田,源家奴怙勢,多侵靜海縣民業。十六年,給事中王垣等言:「永樂、宣德間,許畿輔八郡民盡力墾荒,永免其稅,所以培國本重王畿也。外戚王源賜田,初止二十七頃,乃令其家奴別立四至,占奪民產至二千二百餘頃。及貧民赴告,御史劉喬徇情曲奏,致源無忌憚,家奴益橫。今戶部郎中張禎叔等再按得實,乞自原額外悉還氏,并治喬罪。」帝不悅,切責之。後詔禁外戚侵民產,源悉歸所占於民,人多其能改過。十八年擢中軍都督同知。二十年封瑞安伯。弘治六年進侯。十六年加太保。武宗登極,進太傅,增祿至七百石。嘉靖三年卒,贈太師,謚榮靖。清,成化十八年授錦衣衛千戶,累官中軍都督同知。弘治十年封崇善伯。武宗嗣位,加太保。嘉靖十三年卒。濬,成化十八年授錦衣衛百戶。兄清每遷職,輒以浚代,歷官中軍左都督。正德二年封安仁伯,踰月卒,贈侯。浚兄弟三人並貴顯,皆謙慎守禮,在戚里中以賢稱。源子橋、浚子桓,皆嗣伯。嘉靖中并清子極皆以例降革。

萬貴,憲宗萬貴妃父也,歷官錦衣衛指揮使。貴頗謹飭,每受賜,輒憂形於色曰:「吾起掾史,編尺伍,蒙天子恩,備戚屬,子姓皆得官。福過災生,未知所終矣。」時貴妃方擅寵,貴子喜為指揮使,與弟通、達等並驕橫。貴每見諸子屑越賜物,輒戒曰:「官所賜,皆著籍。他日復宣索,汝曹將重得罪。」諸子笑以為迂。成化十年卒,賻贈祭葬有加。十四年進喜都指揮同知,通指揮使,達指揮僉事。通少貧賤,業賈。既驟貴,益貪黷無厭,造奇巧邀利。中官韋興、梁芳等復為左右,每進一物,輒出內庫償,輦金錢絡繹不絕。通妻王出入宮掖,大學士萬安附通為同宗,婢僕朝夕至王所,謁起居。妖人李孜省輩皆緣喜進,朝野苦之。通死,帝眷萬氏不已,遷喜都督同知,達指揮同知。通庶子方二歲,養子方四歲,俱授官。憲宗崩。言官劾其罪狀。孝宗乃奪喜等官,而盡追封誥及內帑賜物,如貴言。

邵喜,昌化人,世宗大母邵太后弟也。世宗立,封喜昌化伯,明年卒。子蕙嗣,嘉靖六年卒,無子,族人爭嗣。初,太后入宮時,父林早歿。太后弟四人:宗、安、宣、喜。宗、宣無後,及蕙卒,帝令蕙弟萱嗣。蕙姪錦衣指揮輔、千戶茂言,萱非嫡派,不當襲,蕙母爭之,議久不決。大學士張璁等言:「邵氏子孫已絕,今其爭者皆旁枝,不宜嗣。」時帝必欲為喜立後,乃以喜兄安之孫傑為昌化伯。明年,《明倫大典》成,命武定侯郭勛頒賜戚畹,弗及傑。傑自請之,帝詰勛。勛怒,錄邵氏爭襲章奏,訐傑實他姓,請覆勘,帝不聽。會給事中陸粲論大學士桂萼受傑賂,使奴隸冒封爵。帝怒,下粲獄,而盡革外戚封,傑亦奪擊。

張巒,敬皇后父也。弘治四年封壽寧伯。立皇太子,進為侯。卒贈昌國公,子鶴齡嗣侯。十六年,其弟延齡亦由建昌伯進爵侯。巒起諸生,雖貴盛,能敬禮士大夫。

鶴齡兄弟並驕肆,縱家奴奪民田廬,篡獄囚,數犯法。帝遣侍郎屠勳、太監蕭敬按得實,坐奴如律。敬復命,皇后怒,帝亦佯怒。已而召敬曰:「汝言是也。」賜之金。給事中吳世忠、主事李夢陽皆以劾延齡幾得罪。他日,帝遊南宮,鶴齡兄弟入侍。酒半,皇后、皇太子及鶴齡母金夫人起更衣,因出遊覽。帝獨召鶴齡語,左右莫得聞,遙見鶴齡免冠首觸地,自是稍斂迹。正德中,鶴齡進太傅。世宗入繼,鶴齡以定策功,進封昌國公。時敬皇后已改稱皇伯母昭聖皇太后矣。帝以太后抑其母蔣太后故,銜張氏。嘉靖十二年,延齡有罪下獄,坐死,并革鶴齡爵,謫南京錦衣衛指揮同知,太后為請不得。

初,正德時,日者曹祖告其子鼎為延齡奴,與延齡謀不軌。武宗下之獄,將集群臣廷鞫之,祖仰藥死。時頗以祖暴死疑延齡,而獄無左證,遂解。指揮司聰者,為延齡行錢,負其五百金。索之急,遂與天文生董昶子至謀訐祖前所首事,脅延齡賄。延齡執聰幽殺之,令聰子昇焚其屍,而折所負券。升噤不敢言,常憤詈至。至慮事發,乃摭聰前奏上之。下刑部,逮延齡及諸奴雜治。延齡嘗買沒官第宅,造園池,僭侈踰制。又以私憾殺婢及僧,事並發覺。刑部治延齡謀不軌,無驗,而違制殺人皆實,遂論死。繫獄四年,獄囚劉東山發延齡手書訕上,東山得免戍,又陰構奸人劉琦誣延齡盜宮禁內帑,所告連數十百人。明年,奸人班期、于雲鶴又告延齡兄弟挾左道祝詛,辭及太后。鶴齡自南京赴逮,瘐死,期、雲鶴亦坐誣謫戍。又明年,東山以射父亡命,為御史陳讓所捕獲,復誣告延齡並構讓及遂安伯陳鏸等數十人,冀以悅上意而脫己罪。奏入,下錦衣衛窮治,讓獄中上疏言:「東山扇結奸黨,圖危宮禁。陛下有帝堯既睦之德,而東山敢為陛下言漢武巫蠱之禍。陛下有帝爵底豫之孝,而東山敢導陛下以暴秦遷母之謀。離間骨肉,背逆不道,義不可赦。」疏奏,帝頗悟。指揮王佐典其獄,鉤得東山情,奏之。乃械死東山,赦讓、鏸等,而延齡長繫如故。太后崩之五年,延齡斬西市。

夏儒,毅皇后父也。正德二年以后父封慶陽伯。為人長厚,父瑄疾,三年不去左右。既貴,服食如布衣時,見者不知為外戚也。十年以壽終,子臣嗣伯。嘉靖八年罷襲。

陳萬言,肅皇后父也,大名人,起家諸生。嘉靖元年授鴻臚卿,改都督同知,賜第黃華坊。明年詔復營第於西安門外,費帑金數十萬。工商尚書趙璜以西安門近大內,治第毋過高。帝怒,逮營繕郎翟璘下獄。言官餘瓚等諫,不省。尋封萬言泰和伯,子紹祖授尚寶司丞。又明年,萬言乞武清、東安地各千頃為莊田,詔戶部勘閒地給之。給事中張漢卿言:「萬言拔跡儒素,聯婚天室,當躬自檢飭,為戚里倡,而僭冒陳乞,違越法度。去歲深冬冱雪,急起大第,徒役疲勞,怨咨載道。方今災沴相繼,江、淮餓死之人,掘穴掩埋,動以萬計。萬言曾不動念,益請莊田。小民一廛一畝,終歲力作,猶不足於食,若又割而畀之貴戚,欲無流亡,不可得也。伏望割恩以義,杜漸以法,一切裁抑,令保延爵祿。」帝竟以八百頃給之。巡撫劉麟、御史任洛復言不宜奪民地,弗聽。七年,皇后崩,萬言亦絀。十四年卒,子不得嗣封。

方銳,世宗孝烈皇后父也,應天人。后初為九嬪,銳授錦衣正千戶。嘉靖十三年,張后廢,后由妃冊為皇后,遷銳都指揮使。扈蹕南巡,道拜左都督。既封安平伯,尋進封侯。卒,子承裕嗣。隆慶元年用主事郭諫臣言,罷襲。

陳景行,穆宗繼后陳皇后父也。先世建昌人,高祖政以軍功世襲百戶,調通州右衛,遂家焉。景行故將門,獨嗜學,弱冠試諸生高等。穆宗居裕邸,選其女為妃,授景行錦衣千戶。隆慶元年封固安伯。景行素恭敬,每遇遣祀、冊封諸典禮,必齋戒將事。家居,誡諸子以退讓。萬曆中卒,太后、帝及中宮、潞王、公主賻贈優厚,人皆榮之。子昌言、嘉言、善言、名言,皆官錦衣。昌言先景行卒,其子承恩引李文全例,請襲祖封。帝曰:「承恩,孫,文全,子也,不可比。」以都督同知授之。

李偉,字世奇,漷縣人,神宗生母李太后父也。兒時嬉里中,有羽士過之,驚語人曰:「此兒骨相,當位極人臣。」嘉靖中,偉夢空中五色彩輦,旌幢鼓吹導之下寢所,已而生太后。避警,攜家入京師。居久之,太后入裕邸,生神宗。隆慶改元,立皇太子,授偉都督同知。神宗立,封武清伯,再進武清侯。太后能約束其家,偉嘗有過,太后召入宮切責之,不以父故骫祖宗法。以是,偉益小心畏慎,有賢聲。萬曆十一年卒,贈安國公,謚莊簡。子文全嗣侯,卒,子銘誠嗣。天啟末,銘誠頌魏忠賢功德,建祠名鴻勛。莊烈帝定逆案,銘誠幸獲免。久之,大學士薛國觀請勒勛戚助軍餉。時銘誠已卒,子國瑞當嗣爵,其庶兄國臣與爭產,言父遺貲四十萬,願輸以佐軍興。帝初不允,至是詔借餉如國臣言,國瑞不能應。帝怒,奪國瑞爵,遂悸死,有司復繫其家人。國瑞女字嘉定伯周奎孫,奎請於莊烈后,后曰:「但迎女,秋毫無所取可也。」諸戚畹人人自危。會皇五子疾亟,李太后憑而言。帝懼,悉還李氏產,復武清爵,而皇五子竟殤。或云中人構乳媼,教皇五子言之也。未幾,國觀遂以事誅。

王偉,神宗顯皇后父也。萬曆五年授都督。尋封永年伯。帝欲加恩偉子棟及其弟俊,閣臣請俱授錦衣正千戶。帝曰:「正德時,皇親夏助等俱授錦衣指揮使,世襲,今何薄也?」大學士張居正等言:「正德時例,世宗悉已釐革,請授棟錦衣衛指揮僉事,俊千戶,如前議。」帝意未慊,居正固奏,乃止。偉卒,傳子棟及曾孫明輔,襲伯如制。

鄭承憲,神宗鄭貴妃父也。貴妃有寵,鄭氏父子、宗族並驕恣,帝悉不問。承憲累官至都督同知,卒。子國泰請襲,帝命授都指揮使。給事中張希皋言:「指揮使下都督一等,不宜授任子。妃家蒙恩如是,何以優後家。」不報。是時,廷臣疑貴妃謀奪嫡,群以為言。國泰不自安,上疏請立太子,其從子承恩亦言儲位不宜久虛。大學士沈一貫左右於帝,弗聽。詔奪國泰俸,而斥承恩為民,然言者終不息。萬歷二十六年,承恩復上疏劾給事中戴士衡、知縣樊玉衡,妄造《憂危竑議》,離間骨肉,污蔑皇貴妃。帝怒。《憂危竑議》者,不知誰所作,中言侍郎呂坤構通宮掖,將與國泰等擁戴福王。而士衡前嘗論坤與承恩相結,玉衡方抗言貴妃沮立太子,疏並留中,故承恩指兩人。帝怒,士衡、玉衡皆永戍。廷臣益貧鄭氏。久之,皇太子立。四十三年,男子張差持梃入東宮,被擒。言者皆言國泰謀刺皇太子。主事王之寀鞫差,差指貴妃宮監。主事陸大受、給事中何士晉遂直攻國泰。帝以貴妃故,不慾竟事,詳之寀等傳。國泰官左都督,病死,子養性襲職。天啟初,光祿少卿高攀龍、御史陳必謙追論其罪,且言養性結白蓮賊將為亂。詔勒養性出京師,隨便居住。魏忠賢用事,宥還。

王升,熹宗生母孝和太后弟也。父鉞。天啟元年封昇新城伯。尋以皇子生,進俟。卒,子國興嗣。崇禎十七年,京師陷,被殺。

劉文炳,字淇筠,宛平人。祖應元,娶徐氏,生女,入宮,即莊烈帝生母孝純皇太后也。應元早卒,帝即位,封太后弟效祖新樂伯,即文炳父也。崇禎八年卒,文炳嗣。是年,文炳大母徐年七十,賜寶鈔、白金、文綺。帝謂內侍曰:「太夫人年老,猶聰明善飯,使太后在,不知若何稱壽也。」因愴然泣下。九年進文炳為新樂侯,其祖、父世贈爵如之。

十三年,宮中奉太后像,或曰未肖。帝不懌,遣司禮監太監王裕民同武英殿中書至文炳第,敕徐口授,繪像以進,左右咸驚曰:「肖。」帝大喜,命卜日具鹵簿,帝俯伏歸極門,迎入,安奉奉慈殿,朝夕上食如生。因追贈應元瀛國公,封徐氏瀛國太夫人,文炳晉少傅,叔繼祖,弟文耀、文照俱晉爵有差。

文炳母杜氏賢,每謂文炳等曰:「吾家無功德,直以太后故,受此大恩,當盡忠報天子。」帝遣文炳視鳳陽祖陵,密諭有大事上聞。文炳歸,奏史可法、張國維忠正有方略,宜久任,必能滅賊,後兩人果殉國難。文炳謹厚不妄交,獨與宛平太學生申湛然、布衣黃尼麓及駙馬都尉鞏永固善。時天下多故,流賊勢益張,文炳與民麓等講明忠義,為守禦計。及李自成據三秦,破榆林,將犯京師。文炳知勢不支,慷慨泣下,謂永固曰:「國事至此,我與公受國恩,當以死報。」

十七年正月,帝召文炳、永固等回國事。二人請早建籓封,遣永、定二王之國。帝是之,以內帑乏,不果行。三月初一日,賊警益急,命文武勛戚分守京城。繼祖守皇城東安門,文耀守永定門,永固守崇文門。文炳以繼祖、文耀皆守城,故未有職事。十六日,賊攻西直門,勢益急。尼麓踉蹌至,謂文炳曰:「城將陷,君宜自為計。」文炳母杜氏聞之,即命侍婢簡笥絳於樓上,作七八繯,命家僮積薪樓下,隨遣老僕鄭平迎李氏、吳氏二女歸,曰:「吾母女同死此。」又念瀛國太夫人年篤老,不可俱燼,因與文炳計,匿之申湛然家。

十八日,帝遣內使密召文炳、永固。文炳歸白母曰:「有詔召兒,兒不能事母。」母拊文炳背曰:「太夫人既得所,我與若妻妹死耳,復何憾。」文炳偕永固謁帝,時外城已陷。帝曰:「二卿所糾家丁,能巷戰否?」文炳以眾寡不敵對,帝愕然。永固奏曰:「臣等已積薪第中,當闔門焚死,以報皇上。」帝曰:「朕志決矣。朕不能守社稷,朕能死社稷。」兩人皆涕泣誓效死,出馳至崇文門。須臾賊大至,永固射賊,文炳助之,殺數十人,各馳歸第。

十九日,文照方侍母飯,家人急入曰:「城陷矣!」文照碗脫地,直視母。母遽起登樓,文照及二女從之,文炳妻王氏亦登樓。懸孝純皇太后像,母率眾哭拜,各縊死。文照入繯墮,拊母背連呼曰:「兒不能死矣,從母命,留侍太夫人。」遂逃去。家人共焚樓。文炳歸,火烈不得入,入後園,適湛然、尼麓至,曰:「鞏都尉已焚府第,自刎矣。」文炳曰:「諾。」將投井,忽止曰:「戎服也,不可見皇帝。」湛然脫己幘冠之,遂投井死。繼祖歸,亦投井死。繼祖妻左氏見大宅火,亟登樓自焚,妾董氏、李氏亦焚死。初,文耀見外城破,突出至渾河,聞內城破,復入,見第焚,大哭曰:「文燿未死,以君與母在。今至此,何生為!」遂覓文炳死所,大書版井旁曰「左都督劉文耀同兄文炳畢命報國處」,亦投井死,闔門死者四十二人。是時,惠安伯張慶臻集妻子同焚死。新城侯王國興亦焚死。宣城伯衛時春懷鐵券,闔門赴井死。與永固射賊楊光陛者,駙馬都尉子也,被甲馳突左右射,與永固相失,矢盡,投觀象臺下井中死。而湛然以匿瀛國為賊所拷掠,終不言,體糜爛以死。福王時,謚文炳忠壯,文燿忠果。

張國紀,祥符人,熹宗張皇后父也。天啟初,封太康伯。魏忠賢與客氏忌皇后,因謀陷國紀,使其黨劉志選、梁夢環先後劾國紀謀占宮婢韋氏,矯中宮旨鬻獄。忠賢將從中究其事,以撼后。大學士李國普曰:「君后,猶父母也,安有勸父構母者?」國紀始放歸故郡,忠賢猶欲掎之,莊烈帝立,乃得免。崇禎末,以輸餉進爵為侯,旋死於賊。

周奎,蘇州人,莊烈帝周皇后父也。崇禎三年封嘉定伯,賜第於蘇州之葑門。帝嘗諭奎及田貴妃父弘遇、袁貴妃父祐,宜恪遵法度,為諸戚臣先。祐頗謹慎,惟弘遇驕縱,奎居外戚中,碌碌而已。李自成逼京師,帝遣內侍徐高密諭奎倡勛戚輸餉,奎堅謝無有。高憤泣曰:「後父如此,國事去矣。」奎不得已奏捐萬金,且乞皇后為助。及自成陷京師,掠其家得金數萬計,人以是笑奎之愚云。


\end{pinyinscope}