\article{列傳第三十四}

\begin{pinyinscope}
張武陳珪孟善鄭亨徐忠郭亮趙彞張信唐雲徐祥李浚孫巖房勝陳旭陳賢張興陳志王友

張武,瀏陽人。豁達有勇力,稍涉書史。為燕山右護衛百戶。從成祖起兵,克薊州,取雄縣,戰月漾橋,乘勝抵鄚州。與諸將敗耿炳文於真定。夾河之戰,帥壯士為前鋒,突陣,佯敗走。南軍追之,武還擊,南軍遂潰。攻西水寨,前軍夜失道,南軍來追。武引兵伏要路,擊卻之。戰小河,陳文歿於陣。武帥敢死士自林間突出,與騎兵合,大破南軍,斬首二萬級,溺死無算。累授都督同知。成祖即位,論功封成陽侯,祿千五百石,位次朱能下。是時侯者,陳珪、鄭亨、孟善、火真、顧成、王忠、王聰、徐忠、張信、李遠、郭亮、房寬十三人,武為第一。還守北平。永樂元年十月卒。出內廄馬以賻,贈潞國公,謚忠毅。無子,爵除。

陳珪,泰州人。洪武初,從大將軍徐達平中原,授龍虎衛百戶,改燕山中護衛。從成祖出塞為前鋒,進副千戶。已,從起兵,積功至指揮同知。還佐世子居守。累遷都督僉事,封泰寧侯,祿千二百石。佐世子居守如故。永樂四年董建北京宮殿,經畫有條理,甚見獎重。八年,帝北征,偕駙馬都尉袁容輔趙王留守北京。十五年命鑄繕工印給珪,並設官屬,兼掌行在後府。十七年四月卒,年八十五。贈靖國公,謚忠襄。

子瑜嗣。二十年從北征。失律,下獄死。兄子鐘嗣。再傳至瀛,歿土木,贈寧國公,謚恭愍。弟涇嗣。天順六年鎮廣西。明年九月,瑤賊作亂,涇將數千人駐梧州。是冬,大藤賊數百人夜入城,殺掠甚眾。涇擁兵不救。徵還,下獄論斬。尋宥之。卒。子桓嗣。弘治初,鎮寧夏。中貴人多以所親冒功賞,桓拒絕之,為所譖,召還。卒。數傳至延祚,明亡,爵除。

孟善,海豐人,仕元為山東樞密院同僉。明初歸附,從大軍北征,授定遠衛百戶。從平雲南,進燕山中護衛千戶。燕師起,攻松亭關,戰白溝河,皆有功。已,守保定。南軍數萬攻城,城中兵纔數千,善固守,城完。累遷右軍都督同知,封保定侯,祿千二百石。永樂元年鎮遼東。七年召還北京,鬚眉皓白。帝憫之,命致仕。十年六月卒。贈滕國公,謚忠勇。

子瑛嗣。將左軍,再從北征,督運餉。仁宗即位,為左參將,鎮交阯。坐庶兄常山護衛指揮賢永樂中謀立趙王事,並奪爵,毀其券,謫雲南。宣德六年放還,充為事官於宣府。英宗即位,授京衛指揮使。卒,子俊嗣官。天順初,以恩詔與伯爵。卒,子昂嗣。卒,爵除。

鄭亨,合肥人。父用,洪武時,積功為大興左衛副千戶。請老,亨嗣職。洪武二十五年,應募持檄諭韃靼,至斡難河。還,遷密雲衛指揮僉事。

燕師起,以所部降。戰真定,先登,進指揮使。襲大寧,至劉家口,諸將將攻關,成祖慮守關卒走報大寧得為備,乃令亨將勁騎數百卷旆登山,潛出關後,斷其歸路。急攻之,悉縛守關者,遂奄至大寧。進北平都指揮僉事。夜帥眾破鄭村壩兵,西破紫荊關,掠廣昌,取蔚州,直抵大同。還戰白溝河,逐北至濟南,進都指揮同知。攻滄州,軍北門,扼餉道東昌。戰敗,收散卒,還軍深州。明年戰夾河、槁城,略地至彰德,耀兵河上。還屯完縣。明年從破東平、汶上,軍小河。戰敗,王真死。諸將皆欲北還,惟亨與朱能不可。入京師,歷遷中府左都督,封武安侯,祿千五百石,予世券。留守北京。時父用猶在,受封爵視亨。

永樂元年,充總兵官,帥武成侯王聰、安平侯李遠備宣府。亨至邊,度宣府、萬全、懷來形便,每數堡相距,中擇一堡可容數堡士馬者,為高城深池,浚井蓄水,謹尞望。寇至,夜舉火,晝鳴炮,併力堅守。規畫周詳,後莫能易。三年二月召還,旋遣之鎮。七年秋,備邊開平。

明年,帝北征,命亨督運。出塞,將右哨,追敗本雅失里。大軍與阿魯台遇。亨帥眾先,大破之。論功為諸將冠。其冬仍出鎮宣府。十二年復從北征,領中軍。戰忽失溫,追敵中流矢卻,復與大軍合破之。二十年復從出塞,將左哨,帥卒萬人,治龍門道過軍,破兀良哈於屈裂河。將輜重還,擊破寇之追躡者,仍守開平。成祖凡五出塞,亨皆在行。

仁宗即位,鎮大同。洪熙元年二月,頒制諭及將軍印於各邊總兵官。亨佩征西前將軍印。在鎮墾田積穀,邊備完固,自是大同希寇患。宣德元年召掌行後府事。已,仍鎮大同,轉餉宣府。招降迤北部長四十九人,請於朝,厚撫之,歸附者相屬。九年二月卒於鎮。

亨嚴肅重厚,善撫士卒,恥掊克。在大同時,鎮守中官撓軍政,亨裁之以理,其人不悅,然其卒也,深悼惜之。贈漳國公,謚忠毅。妾張氏,自經以殉,贈淑人。子能嗣,傳爵至明亡。

徐忠,合肥人,襲父爵為河南衛副千戶。累從大軍北征,多所俘獲,進濟陽衛指揮僉事。洪武末,鎮開平。燕兵破居庸、懷來,忠以開平降。從徇灤河,與陳旭拔其城。李景隆攻北平,燕師自大寧還救。至會州,置五軍:張玉將中軍,硃能將左軍,李彬將右軍,房寬將後軍,忠號驍勇,使將前軍。遂敗陳暉於白河,破景隆於鄭村壩。白溝河之戰,忠單騎突陣。一指中流矢,未暇去鏃,急抽刀斷之。控滿疾驅,殊死戰。燕王乘高見之,謂左右曰:「真壯士也!」進攻濟南,克滄州,大戰東昌、夾河。攻彰德,破西水寨,克東阿、東平、汶上,大戰靈璧。遂從渡江入京師。自指揮同知累遷都督僉事。封永康侯,祿一千一百石,予世券。

忠每戰,摧鋒跳盪,為諸將先。而馭軍甚嚴,所過無擾。善撫降附,得其死力。事繼母以孝聞。夜歸,必揖家廟而後入。儉約恭謹,未嘗有過。成祖北巡,以忠老成,留輔太子監國。永樂十一年八月卒。贈蔡國公,謚忠烈。

傳爵至裔孫錫登,崇禎末,死於賊。從兄錫胤嘗襲侯,卒,無子。其妻朱氏,成國公純臣女也。夫歿,樓居十餘年,不履地。城陷,捧廟主自焚死。

郭亮,合肥人,為永平衛千戶。燕兵至永平,與指揮趙彞以城降,即命為守。時燕師初起,先略定旁郡邑。既克居庸、懷來,山後諸州皆下。而永平地接山海關,障隔遼東。既降,北平益無患,成祖遂南敗耿炳文於真定。既而遼東鎮將江陰侯吳高、都督楊文等圍永平,亮拒守甚固。援師至,內外合擊,高退走。未幾,高中讒罷,楊文代將,復率眾來攻。亮及劉江合擊,大敗之。累進都督僉事。成祖即位,以守城功封成安侯,祿千二百石,世伯爵。永樂七年守開平,以不檢聞。二十一年三月卒。贈興國公,謚忠壯。妾韓氏自經以殉,贈淑人。

子晟當嗣伯,仁宗特命嗣侯。宣德五年,坐扈駕先歸革爵,尋復之。無子,弟昂嗣伯,傳爵至明亡。

趙彞,虹人。洪武時,為燕山右衛百戶。從傅友德北征,城宣府、萬全、懷來,擢永平衛指揮僉事。降燕,歷諸戰皆有功,累遷都指揮使。成祖稱帝,封忻城伯,祿千石。永樂八年,鎮宣府,嘗從北征。坐盜餉下獄,得釋。尋以呂梁洪湍險,命彞鎮徐州經理。復以擅殺運丁、盜官糧,為都御史李慶所劾。命法司論治,復得釋。仁宗立,召還。宣德初卒。子榮嗣。數傳至之龍。崇禎末,協守南京,大清兵下江南,之龍迎降。

張信,臨淮人。父興,永寧衛指揮僉事。信嗣官,移守普定、平越,積功進都指揮僉事。

惠帝初即位,大臣薦信謀勇,調北平都司。受密詔,令與張昺、謝貴謀燕王。信憂懼不知所為。母怪問之,信以告。母大驚曰:「不可。汝父每言王氣在燕。汝無妄舉,滅家族。」成祖稱病,信三造燕邸,辭不見。信固請,入拜床下。密以情輸成祖,成祖戄然起立,召諸將定計,起兵,奪九門。成祖入京師,論功比諸戰將,進都督僉事。封隆平侯,祿千石,與世伯券。

成祖德信甚,呼為「恩張」。欲納信女為妃,信固辭,以此益見重。凡察籓王動靜諸密事,皆命信。信怙寵頗驕。永樂八年冬,都御史陳瑛言信「無汗馬勞,忝冒侯爵,恣肆貪墨,強占丹陽練湖八十餘里、江陰官田七十餘頃,請下有司驗治。」帝曰:「瑛言是也。昔中山王有沙洲一區,耕農水道所經,家僮阻之以擅利。王聞,即歸其地於官。今信何敢爾!」命法司雜治之。尋以舊勳不問。二十年從北征,督運餉。大閱於隰寧,信辭疾不至,謫充辦事官。已而復職。

仁宗即位,加少師,並支二俸,與世侯券。宣德元年,從征樂安。三年,帝巡邊,徵兀良哈,命居守。明年督軍萬五千人浚河西務河道。正統七年五月卒於南京。贈鄖國公,謚恭僖。

子鏞,自立功為指揮僉事,先卒。子淳嗣,傳爵至明亡。

有唐雲者,燕山中護衛指揮也,不知所自起。成祖既殺張昺、謝貴等,將士猶據九門,閉甕城,陳戈戟內向。張玉等夜襲之,已克其八,惟西直門不下。成祖令雲解甲,騎馬導從如平時,諭守者曰:「天子已聽王自制一方。汝等急退,後者戮。」雲於諸指揮中年最長,素信謹,將士以為不欺,遂散。時眾心未附,雲告以天意所嚮,眾乃定。雲從成祖久,出入左右,甚見倚任。先後出師,皆留輔世子。南兵數攻城,拒守甚力,戰未嘗失利。累遷都指揮使。成祖稱帝,封新昌伯,世指揮使。明年七月卒。賜賚甚厚。

徐祥,大冶人。初仕陳友諒,歸太祖於江州,積功至燕山右護衛副千戶。成祖以其謹直,命侍左右。從起兵,轉戰四年,皆有功,累進都指揮使。成祖即位,論功封興安伯,祿千石。時封伯者,祥及徐理、李濬、張輔、唐雲、譚忠、孫巖、房勝、趙彞、陳旭、劉才、茹瑺、王佐、陳瑄十四人,祥第一。祥在諸將中年稍長。及封,益勤慎。永樂二年五月卒。年七十三。

孫亨嗣。十二年從北征,為中軍副將。至土剌河,獲馬三千。還守開平,將輕騎往來興和、大同備邊。後屢從出塞。宣德元年,以右副將征交址,無功,奪爵。英宗即位,復之。正統九年,徵兀良哈,出界嶺口、河北川,進侯。出鎮陜西,召還。天順初卒,謚武襄。

子賢嗣伯,以跛免朝謁,給半祿,卒。子盛嗣,卒,無子。再從弟良嗣。良祖母,故小妻也。繼祖母,定襄伯郭登女。至是其孫爭襲。朝議以郭氏初嘗適人,法不當為正嫡,良竟得嗣。良時年五十,家貧,傭大中橋汲水。都督府求興安伯後,良乃謝其鄰而去,僉書南京中府。忤劉瑾,革祿二百石。傳爵至明亡。

李濬,和州人。父旺,洪武中燕山左護衛副千戶。濬嗣官,從起兵,奪九門。招募薊州、永平壯勇數千人,破南軍於真定。從收大寧。鄭村壩之戰,帥精騎突陣,眾鼓噪乘之,大捷。轉戰山東,為前鋒。至小河,猝與南軍遇,帥敢死士先斷河橋,南軍不能爭。成祖至,遂大敗之。累遷都指揮使,封襄城伯,祿千石。永樂元年出鎮江西。永新盜起,捕誅其魁。尋召還。三年十一月卒。

子隆,字彥平,年十五嗣封。雄偉有將略。數從北征,出奇料敵,成祖器之。即遷都,以南京根本地,命隆留守。仁宗即位,命鎮山海關。未幾,復守南京。隆讀書好文,論事侃侃,清慎守法,尤敬禮士大夫。在南京十八年,前後賜璽書二百餘。及召還,南都民流涕送之江上。正統五年入總禁軍。十一年巡大同邊,賜寶刀一,申飭戒備,內外凜凜。訖還,不僇一人。明年卒。子珍嗣。歿於土木,贈侯,謚悼僖。無子。

弟瑾嗣。成化三年,四川都掌蠻叛,命佩征夷將軍印,充總兵官往討。兵部尚書程信督之。師至永寧,分六路進。瑾與信居中節制,盡破諸蠻寨。前後斬首四千五百有奇,獲鎧仗、牲畜無算。分都掌地,設官建治控制之。師還,進侯,累加太保。弘治二年卒。贈芮國公,謚壯武。瑾性寬弘,能下士。兄璉以貌寢,不得嗣。瑾敬禮甚厚。璉卒,撫其子鄌如己子。瑾子黼嗣伯,數年卒。無子,鄌得嗣。

四傳至守錡,累典營務,加太子少保。崇禎初,總督京營,坐營卒為盜落職,憂憤卒。子國禎嗣。有口辯。嘗召對,指陳兵事甚悉,帝信以為才。十六年命總督京營,倚任之,而國禎實無他能。明年三月,李自成犯京師。三大營兵不戰而潰。再宿,城陷。賊勒國禎降,國禎解甲聽命。責賄不足,被拷折踝,自縊死。

孫巖,鳳陽人。從太祖渡江,累官燕山中護衛千戶,致仕。燕師起,通州守將房勝以城降。王以巖宿將,使與勝協守。南軍至,攻城甚急,樓堞皆毀。巖、勝多方捍禦。已,復突門力戰,追奔至張家灣,獲餉舟三百。累擢都指揮僉事。論功,以舊臣有守城功,封應城伯,祿千石。永樂十一年,備開平,旋移通州。以私憾椎殺千戶,奪爵,安置交阯。已而復之。十六年卒。贈侯,謚威武。子亨嗣,傳至明亡,爵除。

房勝,景陵人。初從陳友諒。來歸,累功至通州衛指揮僉事。燕兵起北平,勝首以通州降。成祖即位,以守城功,封富昌伯,祿千石,世指揮使。永樂四年卒。

陳旭,全椒人。父彬,從太祖為指揮僉事。旭嗣官,為會州衛指揮同知,舉城降燕。從徇灤河,功多。力戰真定。守德州,盛庸兵至,棄城走。置不問。從入京師,封雲陽伯,祿千石。永樂元年,命巡視中都及直隸衛所軍馬城池。四年從英國公張輔征交阯,為右參將。偕豐城侯李彬破西都。師還,與彬各加祿五百石。已而陳季擴叛,復從輔往剿。輔還,又命副沐晟。八年以疾卒於軍。無子,封絕。

陳賢,壽州人。初從太祖立功,授雄武衛百戶。從征西番、雲南;北征至捕魚兒海,皆有功。歷燕山右護衛指揮僉事。燕師起,從諸將轉戰,常突陣陷堅。軍中稱其驍勇。累遷都督僉事。永樂元年四月,成祖慮功臣封有遺闕,令邱福等議。福等言都督僉事李彬功不在房寬下,涇國公子懋、金鄉侯子通俱未襲爵,而陳賢、張興、陳志、王友功與劉才等。於是封彬豐城侯,懋、通與賢等四人並封伯,祿皆千石。賢封榮昌伯。八年充神機將軍,從北征。十三年十一月卒。

子智,前立功為常山右護衛指揮,嗣父爵。宣德中,以參將佩征夷將軍印,鎮交阯。怯不任戰,又與都督方政相失。黎利勢盛,不能禦,敗績。奪爵,充為事官。從王通立功。尋以棄地還,下獄。得釋。正統初,復為指揮使。

張興,壽州人。起卒伍,為燕山左護衛指揮僉事。從起兵,功多,累遷都指揮同知。從子勇,有力敢戰,從興行陣為肘腋。興嘗單騎追敵,被數十創,傷重不任戰。以勇嗣指揮使,代將其兵。再論功,興封安鄉伯。永樂五年正月卒。無子。

勇嗣。永樂八年從北征,失律,謫交阯。赦還復爵,卒。子安嗣。正統十三年鎮廣東。黃蕭養寇廣州,安帥舟師遇賊於戙船澳。安方醉臥,官軍不能支,退至沙角尾。賊薄之,軍潰。安溺死。傳爵至光燦,死流寇。

陳志,巴人。洪武中,為燕山中護衛指揮僉事。從起兵,累遷都指揮同知,封遂安伯。志素以恭謹受知,戮力戎行,始終不懈。永樂八年五月卒。

孫瑛嗣。屢從出塞,鎮永平、山海、薊州,城雲州、獨石。爽闓有將材。然貪殘,人多怨者。卒,子塤嗣。歿於土木,謚榮懷。弟韶嗣。卒。孫鏸嗣。總薊州兵。朵顏入寇,禦卻之。嘉靖初,敘奉迎功,加太子太保,進少保,委寄亞武定侯郭勛。嗣伯六十餘年卒。又五傳而明亡。

王友,荊州人。襲父職為燕山護衛百戶。從起兵,定京師。論功當侯,以驕縱,授都指揮僉事。及邱福等議上,乃封清遠伯。明年充總兵官,帥舟師沿海捕倭。倭數掠海上,友無功,帝切責之。已,大破倭。帝喜,降敕褒勞,尋召還。四年從征交阯,與指揮柳琮合兵破籌江柵,困枚、普賴諸山,斬首三萬七千餘級。六年七月進侯,加祿五百石,與世券。明年,再徵交阯,為副總兵。八年還,從北征,督中軍。別與劉才築城飲馬河上。會知院失乃干欲降,帝令友將士卒先行,諭以遇敵相機剿滅。友等至,與敵相距一程,迂道避之應昌。軍中乏食,多死者。帝震怒,屢旨切責,奪其軍屬張輔。還令群臣議罪。已而赦之。十二年,坐妾告友夫婦誹謗。有驗,奪爵。未幾卒。仁宗即位,官其子順為指揮僉事。

贊曰:張武、陳珪諸人,或從起籓封,或率先歸附,皆偏裨列校,非有勇略智計稱大將材也。一旦遘風雲之會,剖符策功,號稱佐命,與太祖開國諸臣埒,酬庸之義不亦厚歟!


\end{pinyinscope}