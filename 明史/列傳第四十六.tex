\article{列傳第四十六}

\begin{pinyinscope}
黃宗載顧佐邵已陳勉賈諒嚴升段民吾紳章敞徐琦劉戩吳訥朱與言魏驥魯穆耿九疇軒輗陳復黃孔昭

黃宗載,一名垕,字厚夫,豐城人。洪武三十年進士。授行人。奉使四方,未嘗受饋遺,累遷司正。

永樂初,以薦為湖廣按察司僉事。巨奸宿猾多謫戍銅鼓、五開間,陰持官吏短長。宗載榜數其罪,曰:「不改,必置之法。」眾莫敢犯。武陵多戎籍,民家慮與為婚姻徭賦將累己。男女至年四十尚不婚。宗載以理諭之,皆解悟,一時婚者三百餘家。鄰邑效之,其俗遂變。徵詣文淵閣修《永樂大典》。書成,受賜還任。董造海運巨艦數十艘,事辦而民不擾。車駕北征,徵兵湖廣,使者貪暴失期。宗載坐不舉劾,謫楊青驛驛夫。

尋起御史,出按交阯。時交阯新定,州縣官多用兩廣、雲南舉人及歲貢生員之願仕遠方者,皆不善撫字。宗載因言:「有司率不稱職。若俟九年黜陟,恐益廢弛。請任二年以上者,巡按御史及兩司核實舉按以聞。」帝是之。及歸,行李蕭然,不攜交阯一物。尚書黃福語人曰:「吾居此久,所接御史多矣,惟宗載知大體。」丁祖母憂,起復,改詹事府丞。

洪熙元年擢行在吏部侍郎。少師蹇義領部事,宗載一輔以正。宣德元年奉命清軍浙江。三年督採木湖湘。英宗初,以侍郎羅汝敬巡撫陜西,坐事戴罪辦事。汝敬妄引詔書復職,而吏部不言,為御史所劾,宗載及尚書郭璡俱下獄。未幾,得釋,遷南京吏部尚書。居九年,乞休,章四上,乃許。九年七月卒於家,年七十九。

宗載持廉守正,不矯不隨,學問文章俱負時望。公卿大夫齒德之盛,推宗載云。

顧佐,字禮卿,太康人。建文二年進士。除莊浪知縣。端陽日,守將集官僚校射。以佐文士,難之。持弓矢一發而中,守將大服。

永樂初,入為御史。七年,成祖在北京,命吏部選御史之才者赴行在,佐預焉。奉命招慶遠蠻。督採木四川。從北征,巡視關隘。遷江西按察副使,召為應天尹。剛直不撓,吏民畏服,人比之包孝肅。北京建,改尹順天。權貴人多不便之,出為貴州按察使。洪熙元年召為通政使。

宣德三年,都御史劉觀以貪被黜,大學士楊士奇、楊榮薦佐公廉有威,歷官並著風采,為京尹,政清弊革。帝喜,立擢右都御史,賜敕獎勉。命察諸御史不稱者黜之,御史有缺,舉送吏部補選。佐視事,即奏黜嚴暟、楊居正等二十人,謫遼東各衛為吏,降八人,罷三人;而舉進士鄧棨、國子生程富、謁選知縣孔文英、教官方瑞等四十餘人堪任御史。帝使歷政三月而後任之。居正等六人辨愬。帝怒,并諸為吏者悉戍之。既而暟自戍所潛還京,脅他賄,為佐所奏,且言豈謀害己。詔戮暟於市。帝北巡,命偕尚書張本等居守。還復賜敕。令約束諸御史。於是糾黜貪縱,朝綱肅然。

居歲餘,姦吏奏佐受隸金,私遣歸。帝密示士奇曰:「爾不嘗舉佐廉乎?」對曰:「中朝官俸薄,僕馬薪芻資之隸,遣隸半使出資免役。隸得歸耕,官得資費,中朝官皆然,臣亦然。先帝知之,故增中朝官俸。」帝歎曰:「朝臣貧如此。」因怒訴者曰:「朕方用佐,小人敢誣之,必下法司治!」士奇對曰:「細事不足乾上怒。」帝乃以吏狀付佐曰:「汝自治之。」佐頓首謝,召吏言:「上命我治汝,汝改行,吾當貸汝。」帝聞之益喜,謂佐得大體。或告佐不理冤訴。帝曰:「此必重囚教之。」命法司會鞫,果千戶臧清殺無罪三人當死,使人誣佐。帝曰:「不誅清,則佐法不行。」磔清於市。

八年秋,佐有疾,乞歸。不許。以南京右都御史熊概代理其事。踰年而概卒。佐疾良已,入見。帝慰勞之,令免朝賀,視事如故。

正統初考察御史不稱者十五人,降黜之。邵宗九載滿,吏部已考稱,亦與焉。宗奏辨,尚書郭璡亦言宗不應與在任者同考。帝遂責佐。而御史張鵬等復劾宗微過。帝以鵬朋欺,并切責佐。佐上章致仕去。賜敕獎慰,賚鈔五十貫,命戶部復其家。十一年九月卒。

佐孝友,操履清白,性嚴毅。每旦趨朝,小外廬,立雙藤戶外。百僚過者,皆折旋避之。入內直廬,獨處小夾室,非議政不與諸司群坐。人稱為「顧獨坐」云。然持法深,論者以為病。

時雩都陳勉、嶧縣賈諒先後為副都御史,與佐同舉臺職,而蘭谿邵官南京,與佐齊名,繁昌嚴升名亦亞於。

玘,字以先,永樂中進士。授御史。仁宗監國,知其廉直。每法司缺官,即命署,有重獄輒付之。歷仕中外,所過人不敢犯。宣德三年由福建按察使入為南京左副都御史。奏黜御史不職者十三人,簡黜諸司庸懦不肖者八十餘人,風紀大振。居二年,以疾卒官。負氣,好侮同列,治獄頗刻深。然持身廉潔,內行修,事母以孝聞。

陳勉,與同年進士。仁宗初,以楊士奇薦,由廣東副使擢左副都御史。信、豐諸縣盜起,命勉撫之。招徠三千六百餘人,亂遂定。景泰初,仕至南京右都御史,掌院事。致仕,卒。勉外和內剛,精通法律,吏不敢欺。

賈諒,字子信。永樂中由鄉舉入太學,選侍皇太孫說書,擢刑科給事中。宣德四年劾清軍侍郎金庠受賄,罷之。郎中胡玨、蕭翔等十一人,御史方鼎三人,以不職被劾。帝未信,命諒及張居傑密察之。得實,悉貶官。明年又劾陽武侯薛祿朋比不敬。廷中肅然。尋拜右副都御史。偕錦衣指揮王裕、參議黃翰、中官張義等巡視四川、江西、湖廣,按治豪強不少假。正統二年,江北、河南大水,命諒及工部侍郎鄭辰往振。芒、碭山盜為患,諒捕獲甚眾。四年還至德州,卒。諒內行修,當官有風采。

嚴升,建文時進士。歷官大理寺右少卿。清軍蘇、松,執法不撓。調南京僉都御史,與同心治事。剛果自信,嘗著《神羊賦》以見志焉。

段民,字時舉,武進人。永樂二年進士。選庶吉士。與章敞、吾紳輩俱讀書文淵閣,又俱授刑部主事。民旋進郎中。

山東妖婦唐賽兒作亂,三司官坐縱寇誅,擢民左參政。當是時索賽兒急,盡逮山東、北京尼及天下出家婦女,先後幾萬人。民力為矜宥,人情始安。

車駕北征,餉舟由濟寧達潞河,陸挽出居庸至塞外。民深計曲算,下不擾而事集。既還,敕與巡按御史考所過府縣吏廉墨以聞。

宣德三年召入京,命署南京戶部右侍郎,踰年實授。又明年改刑部。初,二部皆以不治聞。民至,紀綱修舉,宿弊以革。上元人有為姪毆者,憤甚,詣通政司告。時方令納米贖罪,而越訴禁甚嚴,犯者戍遼東。民上言:「依定例,卑幼之罪得贖,而尊長反遠竄,揆於理有未安,請更擬。」帝是之。帝以民廉介端謹,特賜敕,令考察南京百官。八年,詔書罪囚自十惡外並減一等。有重囚三十餘人,例不得赦,民亦減其罪。後有旨報決,乃復追還,而逃已數人。民自陳狀,給事中年富等劾民。帝知民賢,不問。

九年二月卒於官,年五十九。貧不能殮,都御史吳訥裞以衣衾。帝聞,命有司營葬。成化間,葉盛請褒恤不果。其後百有餘年,始追謚襄介。

吾紳,字叔縉,衢州人。官刑部主事,治獄有聲。歷郎中,拜禮部侍郎。成祖謂呂震曰:「紳出自翰林,可佐卿典禮矣。」既而為震所擠,出為廣東參政。尋召為南京刑部侍郎,奉敕考察兩廣、福建方面官。有故人官參政,素貪黷,權要多為之地。紳至,竟黜之,時稱其公。復改禮部。正統六年卒於官。

紳清彊有執,澹於榮利。初拜侍郎,賀者畢集。而一室蕭然,了無供具,眾笑而起。

章敞,字尚文,會稽人。由庶吉士授刑部主事。山西盜發,捕逮數百人。敞察其冤,留詞色異者一人,餘悉遣出。明日訊之,留者盜,餘非也。遷郎中,改吏部。

宣德六年擢禮部侍郎。偕徐琦使安南,命黎利權國事。利遣人白相見禮,敞曰:「汝敬使者,所以尊朝廷,奚白為?」利聽命,趨拜下坐。啖以聲色,不為動。還致厚贐,不受,利以付貢使。及關,悉閱貢物,封其贐,付關吏。利死,子麟嗣,敞復奉詔往,卻贐如初。

正統初,纂洪武以來條格,使諸司參酌,吏無能為奸。尚書胡濙寬大,敞佐以嚴肅。二年十二月卒。子瑾亦累官至禮部侍郎。

徐琦,字良玉。先世錢塘人,其祖謫戍寧夏,遂家焉。幼力學,通經史。永樂十三年舉進士,授行人。歷兵部員外郎。明敏有斷,居官務持大體。宣德六年擢右通政。副敞使安南,亦不受饋。還拜南京兵部右侍郎。八年,帝以安南貢賦不如額,南征士卒未盡返,命琦復往。時黎利已死,其子麟疑未決。琦曉以禍福,麟懼,鑄代身金人,貢方物以謝。帝悅,命落琦戍籍,宴賚甚厚。

正統初,與工部侍郎鄭辰考察南畿有司,黜不法者三十人。時災異屢見,琦陳弭災十事。悉嘉納。五年命參贊南京機務。十四年進尚書,參贊如故。有言往年分調南京軍,家屬悉宜北徙,朝議欲行之。琦奏:「安土重遷,人之情也。今驟徙數萬眾,人心一搖,事或叵測。」事得寢。軍衛無學校,琦請天下衛所視府州縣例皆立學。從之。

景泰元年,靖遠伯王驥贊機務,琦專理部事。驥解任,琦仍參贊。四年三月卒,年六十八。謚貞襄。

敞、琦皆以使安南不辱命著稱。安南多寶貨,後使者率從水道挾估客往以為利,交人頗輕之。

弘治時,侍講劉戩往頒詔,由南寧乘傳抵其國,交人大驚。戩依舊制,受陪臣拜謁,不交一語,越宿即行,餽遺一無所受。使人要於途,固致之,卒麾去,與敞、琦皆為交人所重。戩,字景元,安福人。

吳訥,字敏德,常熟人。父遵,任沅陵簿,坐事繫京師。訥上書乞身代。事未白而父歿,訥感奮力學。

永樂中,以醫薦至京。仁宗監國,聞其名,命教功臣子弟。成祖召對稱旨,俾日侍禁廷,備顧問。

洪熙元年,侍講學士沈度薦訥經明行修,授監察御史。敬慎廉直,不務矯飾。宣德初,出按浙江,以振風紀植綱常為務。時軍犯逃者,往往令家人妄愬,逮繫至千人。訥請嚴禁,即冤不得越告。從之。繼按貴州,恩威并行,蠻人畏服。將代還,部民詣闕乞留。不許。五年七月,進南京右僉都御史,尋進左副都御史。

正統初,光祿丞董正等盜官物,訥發之,謫戍四十四人。右通政李畛者,奉使蘇、松,行事多不謹。訥微誡之,畛不悅,誣訥稽延詔書等事。訥疏辯。互為臺省所劾,俱逮下獄,既而釋之。英宗初御經筵,錄所輯《小學集解》上之。四年三月,以老致仕,以朱與言代。

訥博覽,議論有根柢。於性理之奧,多有發明,所著書皆可垂於後。歸家,布衣蔬食,環堵蕭然。周忱撫江南,欲新其居,不可。家居十六年而卒,年八十六。謚文恪,鄉人祀之言偃祠。

朱與言,字一鶚,萬安人。永樂九年進士,授湖廣按察僉事。宣德中遷四川副使。合州盜起,督吏目熊鼎斬六十餘人,賊勢遂衰。事聞,擢鼎合州同知。雅州妖人為亂,與言執送京師,境內以寧。正統元年召為南京右副都御史,入代訥領院事。年老致仕,卒。與言剛方廉慎,為政務大體。數建白,多切時弊。家居門庭清肅,鄉人有不善,惟恐與言知之。

魏驥,字仲房,蕭山人。永樂中,以進士副榜授松江訓導。常夜分攜茗粥勞諸生。諸生感奮,多成就者。召修《永樂大典》。書成,還任。用師逵薦,還太常博士。帝謂曰:「劉履節為御史九年,高皇帝方授是官,不輕予人也。」

宣德初,遷吏部考功員外郎,歷南京太常寺少卿。正統三年,召試行在吏部左侍郎,踰年實授。屢命巡視畿甸遺蝗,問民疾苦。八年改禮部,尋以老請致仕。吏部尚書王直言驥未衰,如念其老,宜令去繁就簡。乃改南京吏部。復以老辭,不允。十四年進尚書。英宗北狩,驥率諸司條上時務,多施行。景泰元年,年七十七,致仕。

驥居官務大體。在太常,山川壇獲雙白兔,圻內生瑞麥,皆卻不進。在吏部,有進士未終制,求考功。同官將許之,驥持不可。法司因旱恤刑,有王綱者,惡逆當辟,或憫其少,欲緩之。驥曰:「此婦人之仁,天道不時,正此故也。」獄決而雨。

正統中,王振怙寵,凌公卿,獨嚴重驥,呼「先生」。景泰初,以請老至京師。大學士陳循,驥門生也,請間曰:「公雖位冢宰,然未嘗立朝。願少待,事在循輩。」驥正色曰:「君為輔臣,當為天下進賢才,不得私一座主。」退語人曰:「渠以朝廷事為一己事,安得善終。」竟致仕去。

驥端厚祗慎。顧勁直,好別白君子、小人。恒曰:「無是非之心,非人也。」家居,憂國憂民,老而彌篤。蕭山故多水患,有宋時縣令楊時湖隄遺跡。驥倡修螺山、石巖、畢公諸塘堰,捍江潮,興湖利。鄉人賴之。居恒布衣糲食,不殖生產。事兄教諭騏,雖耄益恭。時戴笠行田間。嘗遇錢塘主簿,隸訶之。答曰「蕭山魏驥也」。主簿倉皇謝慰而去。

成化七年,御史梁昉言:「臣先任蕭山,見致仕尚書臣魏驥里居,與里人稠處,教子孫孝弟力田,增隄浚湖,捍禦災患。所行動應禮法,倡理學,勖後進。雖在林野,有補治化。驥生平學行醇篤,心術正大。諳世事,尞國體。致仕二十餘年,年九十八歲,四方仰德,有如卿雲。百年化育,滋此人瑞。臣讀前史,有以歸老賜祿畢其身者,有尊養三老五更者,有安車蒲輪召者,有賜幾仗者,上齒德也。驥齒德有餘,爵在上卿,可稱達尊。乞下所司,酌前代故事施行。」帝覽奏嘉歎。遣行人存問,賜羊酒,命有司月給米三石。使命未至而驥卒。賜祭葬如禮,謚文靖。其子完以驥遺言詣闕辭葬,乞以其金振饑民。帝憮然曰:「驥臨終遺命,猶恐勞民,可謂純臣矣。」許之。蕭山民德驥不已,詣闕請祀於德惠祠,以配楊時。制曰「可」。

魯穆,字希文,天台人。永樂四年進士。家居,褐衣蔬食,足跡不入州府。比謁選,有司饋之贐,穆曰:「吾方從仕,未能利物,乃先厲州里乎?」弗受。除御史。仁宗監國,屢上封事。漢王官校多不法,人莫敢言。穆上章劾之,不報,然直聲振朝廷。

遷福建僉事。理冤濫,摧豪強。泉州人李某調官廣西,其姻富民林某遣僕酖李於道,而室其妻。李之宗人訴於官,所司納林賂,坐訴者,繫獄久。穆廉得其實,立正林罪。漳民周允文無子,以姪為後,晚而妾生子,因析產與姪,屬以妾子。允文死,侄言兒非叔子,逐去,盡奪其貲,妾訴之。穆召縣父老及周宗族,密置妾子群兒中。咸指兒類允文,遂歸其產。民呼「魯鐵面」。時楊榮當國,家人犯法,穆治之不少貸。榮顧謂穆賢,薦之朝。

英宗即位,擢右僉都御史。明年奉命捕蝗大名。還,以疾卒。命給舟歸其喪。

始穆入為僉都御史,所載不過囊衣,尚書吳中贈以器用,不受。至是中為治棺衾,乃克殯。子崇志,歷官應天尹,廉直有父風。

耿九疇,字禹範,盧氏人。永樂末進士。宣德六年授禮科給事中。議論持大體,有清望。

正統初,大臣言兩淮鹽政久壞,宜得重名檢者治之,於是推擇為鹽運司同知。痛革宿弊,條奏便宜五事,著為令。母喪去官,場民數千人詣闕乞留。十年正月起為都轉運使。節儉無他好,公退焚香讀書,廉名益振,婦孺皆知其名。

以事見誣,逮下吏,已,得白,即留為刑部右侍郎。屢辨疑獄,無所撓屈。禮部侍郎章瑾下獄,九疇及江淵等議貶其官。瑾婿給事中王汝霖銜之,與同官葉盛、張固、林聰等論刑部不公。九疇、淵遂劾盛等,且言汝霖父永和死土木,嬉笑自如,不宜居職。時景帝新立,急於用人,置汝霖等不問,瑾如奏。鳳陽歲凶,盜且起,敕往巡視招撫。奏留英武、飛熊諸衛軍耕守,招來流民七萬戶,境內以安。

兩淮自九疇去,鹽政復弛。景泰元年仍命兼理。尋敕錄諸府重囚,多所平反。十月命兼撫江北諸府。

三年三月代陳鎰鎮陜西。都指揮楊得青等私役操卒,九疇劾之。詔按治,且命諸邊如得青者,具劾以聞。邊將請增臨洮諸衛戍,九疇言:「邊城士卒非乏。將帥能嚴紀律,賞罰明信,則人人自奮。不然,徒冗食耳。」乃不增戍。邊民春夏出作田,秋冬輒徙入塞。九疇言:「邊將所以御寇,衛民也。今使民避寇失業,安用將帥?」因禁民入徙。有被寇者,治守帥罪。

四年,布政使許資言:「侍郎出鎮,與巡按御史不相統,事多拘滯,請改授憲職便。」乃轉右副都御史。大臣鎮守、巡撫皆授都御史,自九疇始。有旨市羊角為燈,九疇引宋蘇軾諫神宗買浙燈事,事乃寢。災異求言,請帝延儒碩,公賞罰,擇守令,簡將帥。優詔報焉。

天順初,議事京師。帝顧侍臣曰:「九疇,廉正人也。」留為右都御史。罪人繫都察院獄者不給米。九疇為言,乃日給一升,遂為令。已,上疏陳崇廉恥、清刑獄、勸農桑、節軍賞、重臺憲五事。帝皆嘉納。是年六月,御史張鵬等劾石亨、曹吉祥。亨等謂九疇實使之,遂并下獄。謫江西布政使,尋調四川。

明年,禮部缺尚書。帝問李賢。賢曰:「老成清介,無如九疇。」乃召還。既至,憐其老,改南京刑部尚書。四年卒。謚清惠。子裕,自有傳。

軒輗,字惟行,鹿邑人。永樂末年進士。授行人司副。宣德六年用薦改御史。按福建,剔蠹鋤奸,風采甚峻。

正統元年清軍浙江,劾不職官四十餘人。五年言:「祖宗設御史官,為職綦重。今內外諸司有事,多擅遣御史,非制,請禁之。」立報可。是年,超擢浙江按察使。前使奢汰,輗力矯之。寒暑一青布袍,補綴殆遍,居常蔬食,妻子親操井臼。與僚屬約:三日出俸錢市肉,不得過一斤。僚屬多不能堪。故舊至,食惟一豆。或具雞黍,則人驚以為異。時鎮守內臣阮隨、布政使孫原貞、杭州知府陳復、仁和知縣許璞居官皆廉,一方大治。

溫、處有銀場,洪武間歲課僅二千八百餘兩。永樂時增至八萬二千兩,民不堪命。帝即位,以大臣議罷之。至是參政俞士悅請復開,謂利歸於上,則礦盜自絕。下三司議,輗力持不可,乃止。既而給事中陳傅復請,朝廷遽從之,遂致葉宗留之變。

會稽趙伯泰,宋苗裔也。奏孝宗、理宗及福王陵墓,俱為豪民侵奪。御史王琳謂福王降於元,北去,山陰安得墓?伯泰不平,復訴。帝命輗及巡按御史歐陽澄覆按。輗言福王蓋衣冠之藏,伯泰言非誣。詔戍豪民於邊,停琳等俸。遭親喪,起復。十三年奏陳四事,俱切時弊,帝悉從之。

景帝立,以右副都御史鎮守浙江。景泰元年命兼理兩浙鹽課。閩賊吳金八等流劫青田諸縣,輗與原貞討平之。賊首羅丕、廖寧八復自閩抵浙。輗等防遏有功,進秩一等。明年改督南京糧儲。五年復改左副都御史,掌南院事。考黜御史不職者數人。

天順元年二月召拜刑部尚書。數月,引疾乞歸。帝召見,問曰:「昔浙江廉使考滿歸,行李僅一簏,乃卿耶?」輗頓首謝。賜白金慰遣之。明年,南京督理糧儲缺官,帝問李賢,大臣中誰曾居此職者。賢以輗對,且稱其廉。乃命以左都御史往。八年夏以老乞骸骨,不待報徑歸。抵家趣具浴,欠伸而卒。

輗孤峭,遇人無賢否,拒不與接。為按察使,嘗飲同僚家,歸撫其腹曰:「此中有贓物也。」在南都,都御史張純置酒延客。輗惡其汰,不往。徹饌遺之,亦不納。歲時詣禮部拜表慶賀,屏居一室,撤燭端坐,事竣竟歸,未嘗與僚友一語。僚友聞其來,亦輒避去,不樂與之處。量頗遍隘。御史有訐人陰私者,輒獎其能。嘗令御史劾南京祭酒吳節,節亦發輗私事,眾頗不直輗。然清操聞天下,與耿九疇齊名,語廉吏必曰軒、耿。

陳復,福建懷安人。輗同年進士,由戶部主事知杭州。廉靜無私,獄訟大省。日端坐堂皇,與曹掾講讀律令而已。遭喪,部民乞留,詔起復,未幾卒。輗倡僚屬助之,乃克斂。吏民相率致賻,其子盡卻之,稱貸歸。

黃孔昭,黃巖人。初名曜,後以字行,改字世顯。年十四,遭父母喪,哀毀骨立。舉天順四年進士,授屯田主事。奉使江南,卻饋弗受,進都水員外郎。

成化五年,文選郎中陳雲等為吏所訐,盡下獄貶官,尚書姚夔知孔昭廉,調之文選。九年進郎中。故事,選郎率閉門謝客。孔昭曰:「國家用才,猶富家積粟。粟不素積,豈足贍饑;才不預儲,安能濟用?茍以深居絕客為高,何由知天下才俊。」公退,遇客至,輒延見,訪以人才,書之於冊。除官,以其才高下配地繁簡。由是銓敘平允。其以私干者,悉拒之。嘗與尚書尹旻爭,至推案盛怒。孔昭拱立,俟其怒止,復言之。旻亦信其諒直。旻暱通政談倫,欲用為侍郎,孔昭執不可。旻卒用之,倫果敗。旻欲推故人為巡撫,孔昭不應。其人入都謁孔昭,至屈膝,孔昭益鄙之。旻令推舉,孔昭曰:「彼所少者,大臣體耳。」旻謂其人曰:「黃君不離銓曹,汝不能遷也。」

為郎中滿九載,始擢右通政。久之,遷南京工部右侍郎。有官地十餘區為勢家所侵,奏復之。奉詔薦舉方面,以知府樊瑩、僉事章懋應。後皆為名臣。郎官主藏者以羨銀數千進,斥退之。掘地得古鼎,急命工鐫文廟二字,送之廟中。俄中貴欲獻諸朝,見鐫字而止。

孔昭嗜學敦行,與陳選、林鶚、謝鐸友善,並為士類所宗。弘治四年卒。嘉靖中,贈禮部尚書,謚文毅。子俌,亦舉進士,為文選郎中。俌子綰,以議大禮至禮部尚書,自有傳。

贊曰:國家盛時,士大夫多以廉節自重,豈刻意勵行,好為矯飾名譽哉。亦其澹嗜欲,恥營競,介特之性然也。仁、宣之際,懲吏道貪墨,登進公廉剛正之士。宗載佐銓衡,顧佐掌邦憲,風紀為之一清。段民、吳訥、魏驥、魯穆皭然秉羔羊素絲之節。軒、耿、孔昭矯厲絕俗,物不能幹。章敞、徐琦、劉戩律己嚴正,異域傾心。廉之足尚也卓矣。


\end{pinyinscope}