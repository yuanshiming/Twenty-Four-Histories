\article{列傳第二十二}

\begin{pinyinscope}
何文輝徐司馬葉旺(馬云繆大亨(武德蔡遷陳文王銘寧正袁義金興旺費子賢花茂丁玉郭云王溥

何文輝,字德明,滁人。太祖下滁州,得文輝,年十四,撫為己子,賜姓硃氏。太祖初起,多蓄義子。及長,命偕諸將分守諸路。周舍守鎮江,道舍守寧國,馬兒守婺州,柴舍、真童守處州,金剛奴守衢州,皆義子也。金剛奴後無考。周舍即沐英,軍中又呼沐舍。柴舍者,朱文剛,與耿再成死處州難。又有朱文遜,史不傳其小字,亦以義子死太平。自沐英外,最著者唯道舍、馬兒,馬兒即徐司馬,而道舍即文輝也。文輝以天寧翼元帥守寧國,進江西行省參政。數攻江西,未下州縣。討新淦鄧仲廉,斬之。援安福,走饒鼎臣,平山尖寨。從徐達取淮東,復從下平江。賜文綺,進行省左丞,復其姓。

以征南副將軍與平章胡美由江西取福建,度杉關,入光澤,徇邵武、建陽,直趨建寧。元同僉達里麻、參政陳子琦閉門拒守。文輝與美環攻之。踰十日,達里麻不能支,夜潛至文輝營,乞降。詰旦,總管翟也先不花亦以眾降於文輝。美怒兩人不詣己,欲屠其城。文輝馳告美曰:「與公同受命至此,為安百姓耳。今既降,奈何以私忿殺人。」美乃止。師入城,秋毫無所犯。汀、泉諸州縣聞之,皆相次歸附。會車駕幸汴梁,召文輝扈從,因命為河南衛指揮使,定汝州餘寇。從大將軍取陜西,留守潼關。洪武三年,授大都督府都督僉事,予世襲指揮使。復以參將從傅友德等平蜀,賜金幣,留守成都。

文輝號令明肅,軍民皆德之。帝嘗稱其謀略威望。遷大都督府同知。五年命帥山東兵從李文忠出應昌。明年移鎮北平。文忠北征,文輝督兵巡居庸關,以疾召還。九年六月卒,年三十六。遣官營葬滁州東沙河上,恤賚甚厚。子環,成都護衛指揮使,徵迤北陣歿。

徐司馬,字從政,揚州人。元末兵亂年九歲,無所依。太祖得之,養為子,亦賜姓。即長,出入侍左右。及取婺州,除總制,命助元帥常遇春守婺。吳元年,授金華衛指揮同知。洪武元年從副將軍李文忠北征,擒元宗王慶生。擢杭州衛指揮使,尋進都指揮使。詔復姓。

九年遷鎮河南。時新建北京於汴梁,號重地。帝素賢司馬,特委任之。宋國公馮勝方練兵河南。會有星變,占在大梁。帝使使密敕勝,且曰:「并以此語馬兒知之。」既復敕二人曰:「天象屢見,大梁軍民錯處,尤宜慎防。今秦、晉二王還京,當嚴兵宿衛。王抵汴時,若宋國公出迓,則都指揮居守;都指揮出迓,則宋國公亦然。」敕書官而不名,倚重與宋公等。十九年入覲,遂擢中軍都督府僉事。二十五年,以左副總兵從藍玉征建昌,討越巂。明年正月還至成都,卒。追坐藍玉黨,二子皆獲罪。

司馬好文學,性謙厚,所至撫循士卒,甚得眾心。在河南久,尤有惠政。公暇退居,一室蕭然如寒素。雖戰功不及文輝,而雅量過之,並稱賢將云。

葉旺,六安人。與合肥人馬雲同隸長鎗軍謝再興,為千戶。再興叛,二人自拔歸。數從征,積功並授指揮僉事。洪武四年,偕鎮遼東。初,元主北走,其遼陽行省參政劉益屯蓋州,與平章高家奴相為聲援,保金、復等州。帝遣斷事黃儔齎詔諭益。益籍所部兵馬、錢糧、輿地之數來歸。乃立遼陽指揮使司,以益為指揮同知。未幾,元平章洪保保、馬彥翬合謀殺益。右丞張良佐、左丞商暠擒彥翬殺之,保保挾儔走納哈出營。良佐因權衛事,以狀聞。且言:「遼東僻,處海隅,肘腋皆敵境。平章高家奴守遼陽山寨,知院哈剌章屯沈陽古城,開元則右丞也先不花,金山則太尉納哈出。彼此相依,時謀入犯。今保保逃往,釁必起。乞留斷事吳立鎮撫軍民,而以所擒平章八丹、知院僧孺等械送京師。」帝命立、良佐、暠俱為蓋州衛指揮僉事。既念遼陽重地,復設都指揮使司統轄諸衛,以旺及雲並為都指揮使往鎮之。已,知儔被殺,納哈出將內犯,敕旺等預為備。

未幾,納哈出果以眾至,見備禦嚴,不敢攻,越蓋至金州。金州城未完,指揮韋富、王勝等督士卒分守諸門。乃剌吾者,敵驍將也,率精騎數百挑戰城下,中伏弩仆,為我兵所獲。敵大沮。富等縱兵擊,敵引退,不敢由故道,從蓋城南十里沿柞河遁。旺先以兵扼柞河。自連雲島至窟駝寨,十餘里緣河壘冰為墻,沃以水,經宿凝沍如城。布釘板沙中,旁設坑阱,伏兵以伺。雲及指揮周鶚、吳立等建大旗城中,嚴兵不動,寂若無人。已,寇至城南。伏四起,兩山旌旗蔽空,矢石雨下。納哈出倉皇趨連雲島,遇冰城,旁走,悉陷於阱,遂大潰。雲自城中出,合兵追擊至將軍山、畢栗河,斬獲及凍死者無算,乘勝追至豬兒峪。納哈出僅以身免。第功,進旺、雲俱都督僉事。時洪武八年也。

十二年命雲征大寧。捷聞,受賞,召還京。後數年卒。旺留鎮如故。會高麗遣使致書及禮物,而龍州鄭白等請內附。旺以聞。帝謂:人臣無外交。此間諜之漸,勿輕信。彼特示弱於我,以窺邊釁。還之,使無所藉口。明年,旺復送高麗使者周誼入京。帝以其國中弒逆,又詭殺朝使,反覆不可信,切責旺等絕之,而留誼不遣。十九年召旺為後軍都督府僉事。居三月,遼東有警,復命還鎮。二十一年三月卒。

旺與雲之鎮遼也,翦荊棘,立軍府,撫輯軍民,墾田萬餘頃,遂為永利。旺尤久,先後凡十七年。遼人德之。嘉靖初,以二人有功於遼,命有司立祠,春秋祀之。

繆大亨,定遠人。初糾義兵,為元攻濠,不克,元兵潰。大亨獨以眾二萬人與張知院屯橫澗山,固守月餘。太祖以計夜襲其營,破之,大亨與子走免。比明,復收散卒,列陣以待。太祖遣其叔貞諭降之,命將所部從征,數有功,擢元帥。總兵取揚州,克之。降青軍元帥張明鑑。

初,明鑑聚眾淮西,以青布為號,稱「青軍」;又以善長槍,稱「長槍軍」。由含山轉掠揚州,元鎮南王孛羅普化招降之,以為濠、泗義兵元帥。踰年,食盡,謀擁王作亂。王走,死淮安。明鑑遂據城,屠居民以食。大亨言於太祖,賊饑困,若掠食四出則難制矣,且驍鷙可用,無為他人得。太祖命大亨亟攻,明鑑降,得眾數萬、馬二千餘匹。悉送其將校妻子至應天。改淮海翼元帥府為江南分樞密院,以大亨為同僉樞密院事,總制揚州、鎮江。

大亨有治略,寬厚不擾,而治軍嚴肅,禁暴除殘,民甚悅之。未幾卒。太祖過鎮江,嘆曰:「繆將軍生平端直,未嘗有過,惜不見矣。」遣使祭其墓。

武德,安豐人。元至正中為義兵千戶。知元將亡,言於其帥張鑑曰:「吾輩才雄萬夫。今東衄西挫,事勢可知。不如早擇所依。」鑑然其言,相率歸太祖。隸李文忠,從赴池州,力戰,流矢中右股,拔去,戰自若。取於潛、昌化,克嚴州,皆預,進萬戶。苗帥楊完者軍烏龍嶺,德請曰:「此可襲而取也。」文忠問故。對曰:「乘高覘之,其部曲徙舉不安而聲囂。」文忠曰:「善。」即襲完者,覆其營。取蘭溪,克諸暨,攻紹興,皆先登陷陣,傷右臂不顧。文忠嘆曰:「將士人人如此,何戰不捷哉。」

蔣英、賀仁德之叛,浙東大震。從文忠定金華,又從攻處州。遇仁德於劉山,戈中右股,德引刀斷戈,追擊之。仁德再戰,再敗走,遂為其下所殺。德還師守嚴。後二年,定官制,改管軍百戶。從文忠破張士誠兵於諸暨,與諸將援浦城,所過山寨皆下。復從文忠下建、延、汀三州,悉定閩溪諸寨。進管軍千戶,移守衢,予世襲。最後從靖海侯吳禎巡海上。禎以德可任,令守平陽。在任八年,致仕。及征雲南,帝以德宿將,命與諸大帥偕行。

張鑑,又名明鑒,淮西人。既歸太祖,每攻伐必與德俱,先德卒。官至江淮行樞密院副使。

蔡遷,不詳其鄉里,元末從芝麻李據徐州。李敗,歸太祖,為先鋒。從渡江,下採石,克太平,取溧水,破蠻子海牙水寨及陳埜先,皆有功。定集慶,授千戶。從徐達取廣德、寧國,遷萬戶。進攻常州,獲黃元帥,遂為都先鋒。從征馬馱沙,克池州,攻樅陽,從征衢、婺二州,授帳前左翼元帥。敗陳友諒於龍江,進復太平,取安慶水寨,收九江,敗友諒八陣指揮於瑞昌,遂克南昌。從援安豐,攻合肥,戰鄱陽。從征武昌,進指揮同知。從常遇春討平鄧克明餘黨,進攻贛州。取南安、南雄諸郡,還兵追饒鼎臣於茶陵,遷龍驤衛同知。從徐達克高郵,破馬港,授武德衛指揮使,守淮安,移守黃州。從下湘潭、辰、全、道、永諸州,轉荊州衛指揮。進克廣西,遷廣西行省參政,兼靖江王相,討平諸叛蠻。洪武三年九月卒,詔歸葬京師,贈安遠侯,謚武襄。

遷為將十五年,未嘗獨任,多從諸將征討。身經數十戰,輒奮勇突出,橫刀左右擊,敵皆披靡,不敢近。既還,金瘡滿體,人視之不可堪。而遷略不為意,為太祖所愛重。及卒,尤痛惜之,親製文祭焉。

合肥陳文者,南北征伐,累立戰功,亦遷亞也。文少孤,奉母至孝,元季挈家歸太祖,積官都督僉事。卒,追封東海侯,謚孝勇。明臣得謚孝者,文一人而已。

王銘,字子敬,和州人。初隸元帥俞通海麾下,從攻蠻子海牙於采石。以銘驍勇,選充奇兵。戰方合,帥敢死士大噪突之,拔其水寨。自是數有功。與吳軍戰太湖,流矢中右臂,引佩刀出其鏃,復戰。通海勞之。復拔通州之黃橋、鵝項諸寨。賜白金文綺。龍灣之戰,逐北至采石,銘獨突敵陣。敵兵攢朔刺銘,傷頰。銘三出三入,所殺傷過當。賜文綺銀碗,選充宿衛。從取江州,戰康郎山及涇江口,復克英山諸寨,擢管軍百戶。從副將軍常遇春戰湖州之升山。再戰舊館,已,又戰烏鎮。前後數十戰,功多,命守松江。移太倉,捕斬倭寇千餘人,再賜金幣。

洪武四年,都試百戶諸善用槍者,率莫能與銘抗。累官至長淮衛指揮僉事,移守溫州。上疏曰:「臣所領鎮,外控島夷,城池樓櫓仍陋襲簡,非獨不足壯國威,猝有風潮之變,捍禦無所,勢須改為。」帝報可。於是繕城浚濠,悉倍於舊。加築外垣,起海神山屬郭公山,首尾二千餘丈,宏敞壯麗,屹然東浙巨鎮。帝甚嘉之,予世襲。銘嘗請告暫還和州。溫士女遮道送迎。長吏皆相顧歎曰:「吾屬為天子牧民,民視吾屬去來漠然,愧王指揮多矣。」歷右軍都督僉事,二十六年坐藍玉黨死。

寧正,字正卿,壽州人。幼為韋德成養子,冒韋姓。元末隨德成來歸,從渡江。德成戰歿宣州,以正領其眾。積功授鳳翔衛指揮副使。從定中原,入元都,招降元將士八千餘人。

傅友德自真定略平定州,以正守真定。已,從大軍取陜西。馮勝克臨洮,留正守之。大軍圍慶陽,正駐邠州,絕敵聲援。慶陽下,還守臨洮。從鄧愈破定西,克河州。

洪武三年,授河州衛指揮使。上言:「西民轉粟餉軍甚勞,而茶布可易粟。請以茶布給軍,令自相貿易,省挽運之苦。」詔從之。正初至衛,城邑空虛,勤於勞徠。不數年,河州遂為樂土。璽書嘉勞,始復甯姓。兼領寧夏衛事。修築漢、唐舊渠,引河水溉田,開屯數萬頃,兵食饒足。

十三年從沐英北征,擒元平章脫火赤、知院愛足,取全寧四部。十五年遷四川都指揮使,討平松、茂諸州。雲南初定,命正與馮誠共守之。思倫發作亂,正破之於摩沙勒寨,斬首千五百。已,敵眾大集,圍定邊。沐英分兵三隊,正將左軍,鏖戰,大敗之。語在《英傳》。土酋阿資叛,復從英討降之。英卒,詔授正左都督代鎮。已,復命為平羌將軍,總川、陜兵討平階、文叛寇張者。二十八年從秦王討平洮州番,還京。明年卒。

又袁義,廬江人,本張姓,德勝族弟也。初為雙刀趙總管,守安慶,敗趙同僉、丁普郎於沙子港。左君弼招之,弗從。德勝戰死,始來附。為帳前親軍元帥,賜姓名。數從征伐,積功為興武衛指揮僉事。從大將軍北征,敗元平章俺普達等於通州,走賀宗哲、詹同於澤、潞,功最。復從定陜西,敗元豫王兵。與諸將合攻慶陽。張良臣兵驟薄義營,義堅壁不為動,俟其懈,力擊破之。走擴廓軍於定西,南取興元。進本衛同知,調羽林衛,移鎮遼東。

已,從沐英征雲南,克普定諸城,留鎮楚雄。蠻人屢叛。義積糧高壘,且守且戰,以功遷楚雄衛指揮使。嘗入朝,帝厚加慰勞。以其老,命醫為染鬚鬢,俾還任以威遠人,且特賜銀印寵異之。歷二十年,墾田築堰,治城郭橋梁,規畫甚備。軍民德之。建文元年征還,為右軍都督府僉事,進同知,卒官。

金興旺,不詳所始。為威武衛指揮僉事,進同知。洪武元年,大將軍徐達自河南至陜西,請益兵守潼關。以興旺副郭興守之,進指揮使。

明年攻臨洮,移興旺守鳳翔,轉軍餉。未幾,賀宗哲攻鳳翔,興旺與知府周煥嬰城守。敵編荊為大箕,形如半舫。每箕五人,負之攻城,矢石不能入。投槁焚之,輒揚起。乃置鉤槁中,擲著其隙,火遂熾,敵棄箕走。復為地道薄城。城中以矛迎刺,敵死甚眾,而攻不已。興旺與煥謀曰:「彼謂我援師不至,必不敢出。乘其不意擊之,可敗也。」潛出西北門,奮戰,敵少卻。會百戶王輅自臨洮收李思齊降卒東還,即以其眾入城共守。敵拔營去。眾欲追之,輅曰:「未敗而退,誘我也。」遣騎偵之。至五里坡,伏果發。還師復圍城。眾議欲走。興旺叱曰:「天子以城畀我,寧可去耶!」以輅所將皆新附,慮生變,乃括城中貲畜積庭中,令曰:「敵少緩,當大犒新兵。」新兵喜,協力固守。相持十五日,敵聞慶陽下,乃引去。帝遣使以金綺勞興旺等。

明年,達入沔州,遣興旺與張龍由鳳翔入連雲棧,合攻興元。守將降,以興旺守之,擢大都督府僉事。蜀將吳友仁帥眾三萬寇興元,興旺悉城中兵三千禦敵。面中流矢,拔矢復戰,斬數百人。敵益眾,乃斂兵入城。友仁決濠填塹,為必克計。達聞之,令傅友德夜襲木槽關,攻斗山寨。人持十炬,連亙山上。友仁驚遁。興旺出兵躡之,墜崖石死者無算。友仁自是氣奪。時興旺威鎮隴蜀。

而國初諸都督中,城守功,興旺外尤推費子賢。子賢,亦不詳所始。從渡江,為廣德翼元帥。數有功。取武康,又取安吉。築城守之,張士誠兵數來犯,輒敗去。最後張左丞以兵八萬來攻,子賢所部僅三千人,而守甚固。設車弩城上,射殺其梟將二人,敵乃解去。以功進指揮同知。取福建,克元都、定西,俱有功,授大都督府僉事,世指揮使。

花茂,巢縣人。初從陳埜先,已而來歸。從定江左,滅陳友諒。平中原、山西、陜西。積功授武昌衛副千戶。征西蜀。克瞿唐關,入重慶。下左、右兩江及田州。進神策衛指揮僉事。調廣州左衛。平陽春、清遠、英德、翁源、博羅諸山寨叛蠻及東莞、龍川諸縣亂民,進指揮同知。平電白、歸善賊,再遷都指揮同知,世襲指揮使。數剿連州、廣西湖廣諸瑤賊。上言:「廣東南邊大海,姦宄出沒。東莞、筍岡諸縣逋逃蜒戶,附居海島。遇官軍則詭稱捕魚;遇番賊則同為寇盜。飄忽不常,難於訊詰。不若籍以為兵,庶便約束。」又請設沿海依山廣海、碣石、神電等二十四衛所。築城浚池,收集海島隱料無籍等軍。仍於山海要害地立堡屯軍,以備不虞。皆報可。進都指揮使。久之卒,賜葬安德門。

長子榮襲職。次子英,果毅有父風,亦以軍功為廣東都指揮使,有聲永樂中。

丁玉,初名國珍,河中人。仕韓林兒為御史,才辨有時譽。呂珍破安豐,玉來歸。隨征彭蠡,為九江知府。大兵還建康,彭澤山民叛,玉聚鄉兵討平之。太祖嘉其武略,命兼指揮,更名玉。從傅友德克衡州,以指揮同知鎮其地。復調守永州。玉有文武才,撫輯新附,威望甚著。

洪武元年,進都指揮使,尋兼行省參政,鎮廣西。十年召為右御史大夫。四川威茂土酋董貼里叛,以玉為平羌將軍討之。至威州,貼里降。承制設威州千戶所。十二年平松州,玉遣指揮高顯等城之,請立軍衛。帝謂松州山多田少,耕種不能贍軍,守之非策。玉言:松州為西羌要地,軍衛不可罷。遂設官築戍如玉議。會四川妖人彭普貴為亂,焚掠十四州縣。指揮普亮等不能克,命玉移軍討滅之。帝手敕褒美,轉左御史大夫。師還,拜大都督府左都督。十三年坐胡惟庸姻誅。

郭雲,南陽人。長八尺餘,狀貌魁偉。元季聚義兵保裕州泉白寨,累官湖廣行省平章政事。元主北奔,河南郡縣皆下,雲獨堅守。大將軍徐達遣指揮曹諒圍之,雲出戰,被執。大將軍呵之跪。雲植立,嫚罵求死。脅以刃,不動。大將軍壯之,繫送京師。太祖奇其狀貌,釋之。時帝方閱《漢書》,問識字否,對曰:「識。」因以書授之。雲誦其書甚習。帝大喜,厚加賞賜,用為溧水知縣,有政聲。帝益以為賢,特擢南陽衛指揮僉事,使還鄉收故部曲,就戍其地,凡數年卒。

長子洪,年甫十三。帝為下制曰:「雲出田間,倡義旗,保鄉曲,崎嶇累年,竭心所事。王師北伐,人神響應。而雲數戰不屈,勢窮援絕,終無異志。朕嘉其節概。試之有司,則閭閻頌德;俾鎮故鄉,則軍民樂業。雖無汗馬之勳,倒戈之效,治績克著,忠義凜然。子洪可入開國功臣列,授宣武將軍、飛熊衛親軍指揮使司僉事,世襲。」其同時以降將予世職者有王溥。

溥,安仁人。仕陳友諒為平章,守建昌。太祖命將攻之,不克。朱亮祖擊於饒之安仁港,亦失利。友諒將李明道之寇信州也,溥弟漢二在軍,俱為胡大海擒,歸於行省李文忠,文忠命二人招溥。是歲,太祖拔江州,友諒走武昌,溥乃遣使降,命仍守建昌。明年,太祖次龍興,帥其眾來見,數慰勞。從歸建康,賜第聚寶門外,號其街曰「宰相街」,以寵異之。尋遣取撫州及江西未附郡縣。從克武昌,進中書右丞。洪武元年,命兼詹事府副詹事。從大將軍北征,屢有功。賜文幣,擢河南行省平章,不署事。歲祿視李伯昇、潘元明。

初,溥未仕時,奉母葉氏避兵貴溪。遇亂,與母相失,凡十八年。嘗夢母若告以所在,至是從容言於帝,請歸省墳墓。許之,且命禮官具祭物。溥率士卒之貴溪,求不得,晝夜號泣。居人吳海言:「夫人為賊逼,投井中死矣。」溥求得井,有鼠自井出,投溥懷中,旋復入井。汲井索之,母屍在焉。哀呼不自勝,乃具棺斂,即其地以葬。溥卒,子孫世襲指揮同知。

贊曰:文輝、司馬任寄股肱,葉旺、馬雲效著邊域;大亨以端直見思,郭云以政績蒙寵。他如蔡遷、王銘、甯正、金興旺輩,或善戰,或善守,或善撫綏,要皆一時良將也。蓋明運初興,人材蔚起,鐵券、丹符之外,其可稱者猶如此。以視詩人《兔罝》之詠,何多讓哉。


\end{pinyinscope}