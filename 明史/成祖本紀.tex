\article{成祖本紀}

\begin{pinyinscope}
成祖啟天弘道高明肇運聖武神功純仁至孝文皇帝諱棣,太祖第四子也。母孝慈高皇后。洪武三年,封燕王。十三年,之籓北平。王貌奇偉,美髭髯。智勇有大略,能推誠任人。二十三年,同晉王討乃兒不花。晉王怯不敢進,王倍道趨迤都山,獲其全部而還,太祖大喜,是後屢帥諸將出征,並令王節制沿邊士馬,王威名大振。

三十一年閏五月,太祖崩,皇太孫即位,遺詔諸王臨國中,毋得至京師。王自北平入奔喪,聞詔乃止。時諸王以尊屬擁重兵,多不法。帝納齊泰、黃子澄謀,欲因事以次削除之。憚燕王強,未發,乃先廢周王橚,欲以牽引燕。於是告訐四起,湘、代、齊、岷皆以罪廢。王內自危,佯狂稱疾。泰、子澄密勸帝除王,帝未決。

建文元年夏六月,燕山百戶倪諒告變,逮官校於諒、周鐸等伏誅。下詔讓王,并遣中官逮王府傣,王遂稱疾篤。都指揮使謝貴、布政使張昺以兵守王宮。王密與僧道衍謀,令指揮張玉、朱能潛納勇士八百人入府守衛。

秋七月癸酉,匿壯士端禮門,紿貴,昺入,殺之學之一。,遂奪九門。上書天子指泰、子澄為奸臣,并援《祖訓》「朝無正臣,內有奸惡,則親王訓兵待命,天子密詔諸王統領鎮兵討平之」。書既發,遂舉兵。自署官屬,稱其師曰「靖難」。拔居庸關,破懷來,執宋忠,取密雲,克遵化,降永平。二旬眾至數萬。

八月,天子以耿炳文為大將軍,帥師致討。己酉,師至真定,前鋒抵雄縣。壬子,王夜渡白溝河,圍雄,拔其城,屠之。甲寅,都指揮潘忠、楊松自鄚州來援,伏兵擒之,遂據鄚州,還駐白溝。大將軍部校張保來降,言大將軍軍三十萬,先至者十三萬,半營滹沱河南,半營河北。王懼與北軍戰,南軍且乘之也,乃縱保歸,俾揚言王帥兵且至,誘其軍盡北渡河。壬戌,王至真定,與張玉、譚淵等夾擊炳文軍,大破之,獲其副將李堅、甯忠及都督顧成等,斬首三萬級。進圍真定,二日不下,乃引去。天子聞炳文敗,遣曹國公李景隆代領其軍。九月戊辰,江陰侯吳高以遼東兵圍永平。戊寅,景隆合兵五十萬,進營河間。王語諸將曰:「景隆色厲而中餒,聞我在必不敢遽來,不若往援永平以致其師。吳高怯不任戰,我至必走,然後還擊景隆。堅城在前,大軍在後,必成擒矣。」丙戌,燕師援永平。壬辰,吳高聞王至,果走,追擊敗這。遂北趨大寧。

冬十月壬寅,以計入其城。居七日,挾寧王權,拔大寧之眾及朵顏三衛卒俱南。乙卯,至會州。始立五軍,張玉將中軍,鄭亨、何壽副之,硃能將左軍,朱榮、李浚副之,李彬將右軍,徐理、孟善副之;徐忠將前軍,陳文、吳達副之;房寬將後軍,和允中、毛整副之。丁巳,入松亭關。景隆聞王征大寧,果引軍圍北平,築壘九門,世子堅守不戰。十一月庚午,王次孤山。邏騎還報曰白河流澌不可渡。王禱於神,至則冰合,乃濟師。景隆遣都督陳暉偵敵,道左,出王軍後。王分軍還擊之,暉眾爭渡河,冰忽解,溺死無算。辛未,與景隆戰於鄭村壩。王以精騎先破其七營,諸將繼至,景隆大敗,奔還。乙亥,復上書自訴。十二月,景隆調兵德州,期以明年春大舉。王乃謀侵大同,曰:「攻大同,彼必赴救,大同苦寒,南軍脆弱,且不戰疲矣。」庚申,降廣昌。

二年春正月丙寅,克蔚州。二月癸丑,至大同。景隆果由紫荊關來援。王已旋軍居庸,景隆兵多凍餒死者,不見敵而還。

夏四月,景隆進兵河間,與郭英、吳傑、平安期會白溝河。乙卯,王營蘇家橋。己未,遇平安兵河側。王以百騎前,佯卻,誘安陣動,乘之,安敗走。遂薄景隆軍,戰不利。暝收軍,王以三騎殿,夜迷失道,下馬伏地視河流,乃辨東西,渡河去。庚申,復戰。景隆橫陣數十里,破燕後軍。王自帥精騎橫擊之,斬瞿能父子。令丘福衝中堅,不得入。王盪其左,景隆兵乃繞出王後,大戰良久,飛矢雨注。王三易馬,矢盡揮劍,劍折走登堤,佯引鞭若招後繼者。景隆疑有伏,不敢前,高煦救至,乃解。時南軍益集,燕將士皆失色。王奮然曰:「吾不進,敵不退,有戰耳。」乃復以勁卒突出其背,夾攻之。會旋風起,折景隆旂,王乘風縱火奮擊,斬首數萬溺死者十餘萬人。郭英潰而西,景隆潰而南,盡喪其所賜璽書斧鉞,走德州。五月癸酉,王入德州,景隆走濟南。庚辰,攻濟南,敗景隆軍城下。鐵鉉、盛庸堅守,不克。

秋八月戊申,解圍還北平。九月,盛庸代李景隆將,復取德州,與吳傑、平安、徐凱相掎角,以困北平。時徐凱方城滄州,王佯出兵攻遼東,至通州,循河而南,渡直沽,晝夜兼行。

冬十月戊午,襲執徐凱,破其城,夜坑降卒三千人。遂渡河過德州。盛庸遣兵來襲,擊敗之。十一月壬申,至臨清。十二月丁酉,襲破盛庸將孫霖於滑口。乙卯,及庸戰於東昌,庸以火器勁弩殲王兵。會平安軍至,合圍數重,王大敗,潰圍以免,亡數萬人,張玉戰死。

三年春正月辛酉,敗吳傑、平安於威縣,又敗之於深州,遂還北平。二月乙巳,復帥師南下。三月辛巳,與盛庸遇於夾河,譚淵戰死。朱能、張武殊死鬥,庸軍少卻。會日暮,各斂兵入營。王以十餘騎逼庸營野宿,及明起視,已在圍中。乃從容引馬,鳴角穿營而去。諸將以天子有詔,毋使負殺叔父名,倉卒相顧愕貽,不敢發一矢。是日復戰,自辰至未,兩軍相勝負,東北風忽起,塵埃蔽天,燕兵大呼,乘風縱擊,庸大敗。走德州。吳傑、平安自真定引軍與庸會,未至八十里,聞敗引還。王以計誘之,傑、安出兵襲王。閏月戊戌,遇於槁城。己亥,與戰,大風拔木,傑、安敗走,追至真定城下。癸丑,至大名,聞齊泰、黃子澄已罷,上書請召還吳傑、平安、盛庸兵。天子使大理少卿薛巖來報,諭王釋甲,王不奉詔。

夏五月,傑、安、庸分兵斷燕餉道,王遣指揮武勝上書,詰其故。天子怒,下勝獄。王遂遣李遠略沛縣,焚糧舟萬計。

秋七月己丑,掠彰德。丙申,降林縣。平安乘虛搗北平,王遣劉江迎戰,安敗走。房昭屯易州西水寨,攻保定,王引兵圍之。

冬十月丁巳,都指揮花英援昭,敗之峨眉山下,斬首萬級,昭棄寨走。己卯,還北平。十一月乙巳,王自為文祭南北陣亡將士。當是時,王稱兵三年矣。親戰陣,冒矢石,以身先士卒,常乘勝逐北,然亦屢瀕於危。所克城邑,兵去旋復為朝廷守,僅據有北平、保定、永平三府而已。無何,中官被黜者來奔,具言京師空虛可取狀。王乃慨然曰:「頻年用兵,何時已平?要當臨江一決,不復返顧矣。」十二月丙寅,復出師。

四年春正月乙未,由館陶渡河。癸丑,徇徐州。三月壬辰,平安以四萬騎躡王軍,王設伏淝河,大敗之。丙午,遣譚清斷徐州餉道,還至大店,為鐵鉉軍所圍。王引兵馳援,清突圍出,合擊敗之。

夏四月丙寅,王營小河,為橋以濟,平安趨爭橋,陳文戰死。平安軍橋南,王軍橋北,相持數日。平安轉戰,遇王於北阪,王幾為安槊所及。番騎王騏躍入陣,掖王逸去。王曰:「南軍饑,更一二日餉至,猝未易破。」乃令千餘人守橋,夜半渡河而南,繞出安軍後。比旦,安始覺,適徐輝祖來會。甲戌,大戰齊眉山下。時燕連失大將,淮士盛暑蒸濕,諸將請休軍小河東,就麥觀釁。王曰:「今敵持久饑疲,遮其餉道,可以坐困,奈何北渡懈將士心。」乃下令欲渡河者左,諸將爭趨左。王怒曰:「任公等所之。」乃無敢復言。丁丑,何福等營靈璧,燕遮其餉道,平安分兵六萬人護之。己卯,王帥精銳橫擊。斷其軍為二。何福空壁來援,王軍少卻,高煦伏兵起,福敗走。辛巳,進薄其壘,破之,生擒平安、陳暉等三十七人,何福走免。五月己丑,下泗州,謁祖陵,賜父老牛酒。辛卯,盛庸扼淮南岸,朱能、丘福潛濟襲走之,遂克盱貽。

癸巳,王集諸將議所向,或言宜取鳳陽,或言先取淮安。王曰:「鳳陽樓櫓完,淮安多積粟,攻之未易下。不若乘勝直趨揚州,指儀真,則淮、鳳自震。我耀兵江上,京師孤危,必有內變。」諸將皆曰善。己亥,徇揚州,駐軍江北。天子遣慶成郡主至軍中,許割地以和,不聽。六月癸丑,江防都督僉事陳瑄以舟師叛,附於王。甲寅,祭大江。乙卯,自瓜州渡,盛庸以海艘迎戰,敗績。戊午,下鎮江。庚申,次龍潭。辛酉,天子復遣大臣議割地,諸王繼至,皆不聽。乙丑,至金川門,谷王橞、李景隆等開門納王,都城遂陷。是日,王分命諸將守城及皇城,還駐龍江,下令撫安軍民。大索齊泰、黃子澄、方孝孺等五十餘人,榜其姓名曰奸臣。丙寅,諸王群臣上表勸進。己巳,王謁孝陵。群臣備法駕,奉寶璽,迎呼萬歲。王升輦,詣奉天殿即皇帝位。復周王橚、齊王榑爵。壬申,葬建文皇帝。丁丑,殺齊泰、黃子澄、方孝孺,並夷其族。坐奸黨死者甚眾。戊寅,遷興宗孝康皇帝主於陵園,仍稱懿文太子。

秋七月壬午朔,大祀天地於南郊,奉太祖配。詔:「今年以洪武三十五年為紀,明年為永樂元年。建文中更改成法,一復舊制。山東、北平、河南被兵州縣,復徭役三年,未被兵者與鳳陽、淮安、徐、滁、揚三州蠲租一年,餘天下州縣悉蠲今年田租之半。」癸未,召前北平按察使陳瑛為左副都御史,盡復建文朝廢斥者官。甲申,復官制。癸巳,改封吳王允熥廣澤王,衡主允熞懷恩王,徐王允熙敷惠王,隨母妃呂氏居懿文太子陵園。癸卯,江陰侯吳高督河南、陜西兵備,撫安軍民。甲辰,尚書嚴震直、王鈍,府尹薛正言等巡視山西、山東、河南、陜西。

八月壬子,侍讀解縉、編修黃淮入直文淵閣。尋命侍讀胡廣,修撰楊榮,編修楊士奇,檢討金幼孜、故儼同入直,並預機務。執兵部尚書鐵鉉至,不屈,殺之。左軍都督劉真鎮遼東。丁巳,分遣御史察天下利弊。戊午,都督何福為征虜將軍,鎮寧夏,節制陜西行都司。都督同知韓觀練兵江西,節制廣東、福建。甲子,西平侯沐晟鎮雲南。九月甲申,論靖難功,封丘福淇國公,朱能成國公,張武等侯者十三人,徐祥等伯者十一人。論欸附功,封駙馬都尉王寧為侯,茹瑺、陳瑄及都督同知王佐皆為伯。甲午,定功臣死罪減祿例。乙未,徙山西民無田者實北平,賜之鈔,復五年,韓觀為征南將軍,鎮廣西。

冬十月丁巳,命北平州縣棄官避靖難兵者硃寧等二百一十九人入粟免死,戍興州。己未,脩《太祖實錄》。丙寅,鎮遠侯顧成鎮貴州。壬申,徙封谷王橞於長沙。甲戌,詔從征將士掠民間子女者還其家。十一月壬辰,立妃徐氏為皇后。廢廣澤王允熥、懷恩王允熞為庶人。十二月癸丑,蠲被兵州縣明年夏稅。

永樂元年春正月己卯朔,御奉天殿受朝賀,宴群臣及屬國使。乙酉,享太廟。辛卯,大祀天地於南郊。復周王橚、齊王榑、代王桂、岷王楩舊封。以北平為北京。癸巳,何定侯孟善鎮遼東。丁酉,宋晟為平羌將軍,鎮甘肅。二月庚戌,設北京留守行後軍都督府、行部、國子監,改北平曰順天府。乙卯,遣御史分巡天下,為定制。己未,徙封寧王權於南昌。貽書鬼力赤可汗,許其遣使通好。癸亥,耕耤田。乙丑,遣使徵尚師哈立麻於烏斯藏。己巳,振北京六府饑。辛未,命法司五日一引奏罪囚。壬申,瘞戰地暴骨。甲戌,高陽王高煦備邊開平。三月庚辰,江陰侯吳高鎮大同。壬午,改北平行都司為大寧都司,徙保定,始以大寧地畀兀良哈。戊子,平江伯陳瑄、都督僉事宣信充總兵官,督海運,餉遼東、北京,歲以為常。甲午,振直隸、北京、山東、河南饑。

夏四月丁未朔,安南胡𡗨乞襲陳氏封爵,遣使察實以聞。己酉,戶部尚書夏原吉治蘇、松、嘉、湖水患。辛未,岷王楩有罪,降其官屬。甲戌,襄城伯李濬鎮江西。五月丁丑,除天下荒田未墾者額稅。癸未,宥死罪以下,遞減一等。庚寅,捕山東蝗。丁酉,河南蝗,免今年夏稅。是月,再論靖難功,封駙馬都尉袁容等三人為侯。陳亨子懋等六人為伯。六月壬子,代王桂有罪,削其護衛。癸丑,遣給事中、御史分行天下,撫安軍民,有司奸貪者逮治。丁巳,改上高皇帝、高皇后尊謚。戊辰,武安侯鄭亨鎮宣府。

秋七月庚寅,復貽書鬼力赤。八月己巳,發流罪以下墾北京田。甲戌,徙直隸蘇州等十郡、浙江等九省富民實北京。九月癸未,命寶源局鑄農器給山東被兵窮民。庚寅,初遣中官馬彬使爪哇諸國。乙未,奪歷城侯盛庸爵,尋自殺。庚子,岷王楩有罪,削其護衛。

冬十一月乙亥朔,頒曆於朝鮮諸國,著為令。壬辰,罷遣浚河民夫。甲午,北京地震。乙未,命六科辦事官言事。丙申,韓觀討柳州山賊,平之。閏月丁卯,封胡𡗨為安南國王。

是年,始命內臣出鎮及監京營軍。朝鮮入貢者六,自是歲時貢賀為常。琉球中山、山北、山南,暹羅,占城,爪哇西王,日本,剌泥,安南入貢。

二年春正月乙卯,大祀天地於南郊。己巳,召世子高熾及高陽王高煦還京師。三月乙巳,賜曾棨等進士及第、出身有差。己酉,始選進士為翰林院庶吉士。庚戌,吏部請罪千戶違制薦士者,帝曰:「馬周不因常何進乎?果才,授之官,否則罷之可耳。」戊辰,改封敷惠王允熙甌寧王,奉懿文太子祀。

夏四月辛未朔,置東宮官屬。壬申,僧道衍為太子少師,復共姓姚,賜名廣孝。甲戌,立子高熾為皇太子,封高煦漢王,高燧趙王。壬午,封汪應祖為琉球國山南王。五月壬寅,豐城侯李彬鎮廣東,清遠伯王友充總兵官,率舟師巡海。六月丁亥,汰冗官。辛卯,振松江、嘉興、蘇州、湖州饑。甲午,封哈密安克帖木兒為忠順王。

秋七月壬戌,鄱陽民進書毀先賢,杖之,毀其書。丙寅,振江西、湖廣水災。八月丁酉,故安南國王陳日煃弟天平來奔。九月丙午,周王橚來朝,獻騶虞,百官請賀。帝曰:「瑞應依德而至,騶虞若果為祥,在朕更當修省。」丁卯,徙山西民萬戶實北京。命自今御史巡行察吏毋得摭拾人言,賢否皆具實蹟以聞。

冬十月丁丑,河決開封。乙酉,蒲城、河津黃河清。是月,籍長興侯耿炳文家,炳文自殺。十一月甲辰,御奉天門錄囚。癸丑,京師及濟南、開封地震,敕群臣修省。戊午,蠲蘇、松、嘉、湖、杭水災田租。十二月壬辰,同州、韓城黃河清。是月,下李景隆於獄。

是年,占城,別失八里,琉球山北、山南,爪哇,真臘入貢。暹羅,日本,琉球中山入貢者再。

三年春正月庚戌,大祀天地於南郊。甲寅,遣使責諭安南。庚申,復免順天、永平、保定田租二年。二月己巳,行部尚書雒僉以言事涉怨誹誅。癸未,趙王高燧居守北京。三月甲寅,免湖廣被水田租。

夏六月己卯,中官鄭和帥舟師使西洋諸國。庚辰,中官山壽等帥兵出雲州覘敵。甲申,夏原吉等振蘇、松、嘉、湖饑。免天下農民戶口食鹽鈔。庚寅,胡𡗨謝罪,請迎陳天平歸國。

秋九月丁酉,蠲蘇、松、嘉、湖水災田租,凡三百三十八萬石。丁巳,徙山西民萬戶實北京。

冬十月,盜殺駙馬都尉梅殷。丁卯,齊王榑有罪,三賜書戒之。戊子,頒《祖訓》於諸王。十二月戊辰,沐晟討八百,降之。庚辰,都督僉事黃中、呂毅以兵納陳天平於安南。

是年,蘇門答剌、滿剌加、古里、浡泥來貢,封其長為王。日本貢馬,並俘獲倭寇為邊患者。爪哇東、西,占城,碟里,日羅夏治,合貓里,火州回回入貢。暹羅,琉球山南、山北入貢者再,琉球中山入貢者三。

四年春正月丁未,大祀天地於南郊。丙辰,初御午朝,令群臣奏事得從容陳論。三月辛卯朔,釋奠於先師孔子。甲午,設遼東開原、廣寧馬市。乙巳,賜林環等進士及第、出身有差。丙午,胡𡗨襲殺陳天平於芹站,前大理卿薛巖死之,黃中等引兵還。

夏四月己卯,遣使購遺書。五月丁酉,振常州、廬州、安慶饑。庚戌,齊王榑有罪,削官屬護衛,留之京師。六月己未朔,日當食,陰雲不見,禮官請表賀,不許。丙寅,南陽獻瑞麥,諭禮部曰:「比郡縣屢奏祥瑞,獨比為豐年之兆。」命薦之宗廟。

秋七月辛卯,朱能為征夷將軍,沐晟、張輔副之,帥師分道討安南,兵部尚書劉俊參贊軍務,行部尚書黃福、大理卿陳洽督餉。詔曰:「安南皆朕赤子,惟黎季犛父子首惡必誅,他脅從者釋之。罪人既得,立陳氏子孫賢者。毋養亂,毋玩寇,毋毀廬墓,毋害禾稼,毋攘財貨掠子女,毋殺降。有一於此,雖功不宥。」乙巳,申誹謗之禁。閏月壬戌,詔以明年五月建北京宮殿,分遣大臣採木於四川、湖廣、江西、浙江、山西。八月丁酉,詔通政司,凡上書奏民事者,雖小必以聞。癸丑,齊王榑廢為庶人。九月戊辰,振蘇、松、常、杭、嘉、湖流民復業者十二萬餘戶。

冬十月戊子,成國公朱能卒於軍,張輔代領其眾。乙未,克隘留關。庚子,沐晟率師會於白鶴。十一月己巳,甘露降孝陵松柏,醴泉出神樂觀,薦之太廟,賜百官。十二月辛卯,赦天下殊死以下。張輔大破安南兵於嘉林江。丙申,拔多邦城。丁酉,克其東都。癸卯,克西都,賊遁入海。辛亥,甌寧王允熙邸第火,王薨。

是年,暹羅,占城,于闐,浡泥,日本,琉球中山、山南、婆羅入貢。爪哇東、西,真臘入貢者再。別失八里入貢者三。琉球進閹人,還之,回回結牙曲進玉椀,卻之。

五年春正月丁卯,大祀天地於南郊。己巳,張輔大敗安南兵於木丸江。二月庚寅,出翰林學士解縉為廣西參議。三月丁巳,封尚師哈立麻為大寶法王。辛巳,張輔大破安南兵於富良江。

夏四月己酉,振順天、河間、保定饑。五月甲子,張輔擒黎季犛、黎蒼獻京師,安南平,河南饑,逮治匿災有司。敕都察院,凡災傷不以實聞者罪之。六月癸未,以安南平,詔天下,置交阯布政司。己丑,山陽民丁珏訐其鄉人誹謗,擢為刑科給事中。甲午,詔自永樂二年六月後犯罪去官者,悉宥之。乙未,張輔移師會韓觀討潯、柳叛蠻。癸卯,命張輔訪交阯人才,禮遣赴京師。

秋七月乙卯,皇后崩。丁卯,河溢河南。八月乙酉,左都督何福鎮甘肅。庚子,錄囚,雜犯死罪減等論戍,流以下釋之。九月壬子,鄭和還。乙卯,御奉天門,受安南俘,大賚將士。

冬十月,潯、柳蠻平。

是年,琉球中山、山南,婆羅,日本,別失八里,阿魯,撒馬兒罕,蘇門答剌,滿剌加,小葛蘭入貢。

六年春正月丁巳,岷王楩復有罪,罷其官屬。辛酉,大祀天地於南郊。二月丁未,除北京永樂五年以前逋賦,免諸色課程三年。三月癸丑,寧陽伯陳懋鎮寧夏。乙卯,除河南、山東、山西永樂五年以前逋賦。

夏四月丙申,始命雲南鄉試。五月壬戌夜,京師地震。六月庚辰,詔罷北京諸司不急之務及買辦,以甦民困;流民來歸者復三年。丁亥,張輔、沐晟還。

秋七月癸丑,論平交阯功,進封張輔英國公,沐晟黔國公,王友清遠侯,封都督僉事柳升安遠伯,餘爵賞有差。八月乙酉,交阯簡定反,沐晟為征夷將軍,討之,劉俊仍參贊軍務。九月己酉,命刑部疏滯獄。癸亥,鄭和復使西洋。

冬十一月丁巳,錄囚。十二月丁酉,沐晟及簡定戰於生厥江,敗績,劉俊及都督僉事呂毅、參政劉昱死之。是月,柳升、陳瑄、李彬等率舟師分道沿海捕倭。

是年,鬼力赤為其下所弒,立本雅失里為可汗。浡泥國王來朝。瓦剌,占城,于闐,暹羅,撒馬兒罕,榜葛剌,馮嘉施蘭,日本,爪哇,琉球中山、山南入貢。

七年春正月癸丑,賜百官上元節假十日,著為令。乙卯,大祀天地於南郊。二月乙亥,遣使於巡狩所經郡縣存問高年,八十以上賜酒肉,九十加帛。丙子,徵致仕知府劉彥才等九十二人分署府州縣。辛巳,以北巡告天地宗廟社稷。壬午,發京師,皇太子監國。張輔、王友率師討簡定。戊子,謁鳳陽皇陵。三月甲辰,次東平州,望祭泰山。辛亥,次景州,望祭恒山。乙卯,平安自殺。壬戌,至北京。癸亥,大賚官吏軍民。丙寅,詔起兵時將士及北京效力人民雜犯死罪咸宥之,充軍者官復職,軍民還籍伍。壬申,柳升敗倭於青州海中,敕還師。

夏四月癸酉朔,皇太子攝享太廟。壬午,海寇犯欽州,副總兵李珪遣將擊敗之。閏月戊申,命皇太子所決庶務,六科月一類奏。丙辰,諭行在法司,重罪必五覆奏。五月己卯,營山陵於昌平,封其山曰天壽。乙未,封瓦剌馬哈木為順寧王,太平為賢義王,把禿孛羅為安樂王。六月壬寅,察北巡郡縣長吏,擢汶上知縣史誠祖治行第一,下易州同知張騰於獄。辛亥,給事中郭驥使本雅失里,為所殺。丁卯,斥御史洪秉等四人,詔自今御史勿用吏員。

秋七月癸酉,淇國公丘福為征虜大將軍,武成侯王聰、同安侯火真副之,靖安侯王忠、安平侯李遠為左、右參將,討本雅失里。八月甲寅,丘福敗績於臚朐河,福及聰、真、忠、遠皆戰死。庚申,張輔敗賊於鹹子關。九月庚午朔,日有食之。張輔敗賊於太平海口。甲戌,贈北征死事李遠莒國公、王聰漳國公,遂決意親征。丙子,武安侯鄭亨率師巡邊。壬午,成安候郭亮備禦開平。

冬十月丁未,削丘福封爵,徙其家於海南。十一月戊寅,張輔獲簡定於美良,送京師,誅之。十二月庚戌,賜濟寧至良鄉民頻年遞運者田租一年。乙丑,召張輔還。

是年,滿剌加,哈烈,撒馬兒罕,火州,古里,占城,蘇門答剌,琉球中山、山南入貢。暹羅、榜葛剌入貢者再。

八年春正月辛未,召寧陽侯陳懋隨征漠北。己卯,皇太子攝祀天地於南郊。癸巳,免去年揚州、淮安、鳳陽、陳州水災田租,贖軍民所鬻子女。二月辛丑,以北征詔天下,命戶部尚書夏原吉輔皇長孫瞻基留守北京。乙巳,皇太子錄囚,奏貰雜犯死罪以下,從之。丁未,發北京。癸亥,遣祭所過名山大川。乙丑,大閱。三月丁卯,清遠侯王友督中軍,安遠伯柳升副之,寧遠侯何福、武安侯鄭亨督左、右哨,寧陽侯陳懋、廣恩伯劉才督左、右掖,都督劉江督前哨。甲戌,次鳴鑾戍。乙亥,誓師。

夏四月庚申,次威虜鎮,以橐季駝所載水給衛士,視軍士皆食,始進膳。五月丁卯,更名臚朐河曰飲馬。甲戌,聞本雅失里西奔,遂渡飲馬河追之。己卯,及於斡難河,大敗之,本雅失里以七騎遁。丙戌,還次飲馬河,詔移師征阿魯台。丁亥,回回哈剌馬牙殺都指揮劉秉謙,據肅州衛以叛,千戶朱迪等討平之。六月甲辰,阿魯台偽降,命諸將嚴陣以待,果悉眾來犯。帝自將精騎迎擊。大敗之,追北百餘里。丁未,又敗之。己酉,班師。

秋七月丁卯,次開平。帝在軍,念士卒艱苦,每蔬食,是日宴賚,始復常膳。西寧侯宋琥鎮甘肅。辛巳,振安慶、徽州、鳳陽、鎮江饑。壬午,至北京,御奉天殿受朝賀。甲午,論功行賞有差。八月壬寅,進封柳升安遠侯。乙卯,何福自殺。庚申,河溢開封。九月己巳,幸天壽山。

冬十月丁酉,發北京。是月,倭寇福州。十一月甲戌,至京師。十二月癸巳,阿魯台遣使貢馬。戊午,陳季擴乞降,以為交阯右布政使,季擴不受命。

是年,失捏干寇黃河東岸,寧夏都指揮王俶敗沒。浡泥、呂宋、馮嘉施蘭、蘇門答剌、榜葛剌入貢。占城貢象。琉球山南、爪哇、暹羅貢馬。琉球中山入貢者三。

九年春正月甲戌,大祀天地於南郊。丙子,柳升鎮寧夏。巳卯,張輔為征虜副將軍,會沐晟討交阯。丙戌,豐城侯李彬、平江伯陳瑄率浙江、福建兵捕海寇。二月辛亥,陳瑛有罪,下獄死。丙辰,詔赦交阯。丁巳,倭陷昌化千戶所。己未,工部尚書宋禮開會通河。三月甲子,賜蕭時中等進士及第、出身有差。壬午,浚祥符縣黃河故道。戊子,劉江鎮遼東。

夏六月乙巳,鄭和還自西洋。是月,下交阯右參議解縉於獄。

秋七月丙子,張輔敗賊於月常江。九月戊寅,諭法司,凡死罪必五覆奏。壬午,命屯田軍以公事妨農務者,免徵子粒,著為令。

冬十月乙未,寬北京謫徙軍民賦役。癸卯,封哈密兔力帖木兒為忠義王。乙巳,復修《太祖實錄》。十一月戊午,蠲陜西逋賦。癸亥,張輔敗賊於生厥江。丁卯,立皇長孫瞻基為皇太孫。壬申,韓觀為征夷副將軍,改鎮交阯,都指揮葛森鎮廣西。丙子,敕法司決遣罪囚毋淹滯。是月,遣使督瘞戰場暴骨。十二月壬辰,敕宥福餘、朵顏、泰寧三衛罪,令入貢。閏月丁巳,命府部諸臣陳軍民利弊。

是年,浙江、湖廣、湖南、順天、揚州水,河南、陜西疫,遣使振之。滿剌加王來朝。爪哇、榜葛剌、古里、柯枝、蘇門答剌、阿魯、彭亨、急蘭丹、南巫里、暹羅入貢。阿魯台來貢馬,別失八里獻文豹。琉球中山入貢者三。

十年春正月己丑,命入覲官千五百餘人各陳民瘼,不言者罪之,言有不當勿問。丁酉,大祀天地於南郊。癸丑,振平陽饑,逮治布政使及郡縣官不奏聞者。二月辛酉,蠲山西、河南逋賦。庚辰,遼王植有罪,削其護衛。三月丁亥,豐城侯李彬討甘肅叛寇八耳思朵羅歹。戊子,賜馬鐸等進士及第、出身有差。甲辰,免北京水災租稅。

夏六月甲戌,諭戶部,凡郡縣有司及朝使目擊民艱不言者,悉逮治。

秋七月癸卯,禁中官干預有司政事。八月癸丑,張輔大破交阯賊於神投海。己未,敕邊將自長安嶺迤西迄洗馬林築石垣,深濠塹。

冬十月戊辰,獵城南武岡。十一月壬午,侍講楊榮經略甘肅。丙申,鄭和復使西洋。

是年,浡泥、占城、暹羅、滿剌加、榜葛剌、蘇門答剌、南浡利、球琉山南入貢。

十一年春正月辛己朔,日有食之,詔罷朝賀宴會。壬午,諭通政使、禮科給事中,凡朝覲官境內災傷不能聞為他人所奏者,罪之。辛卯,大祀天地於南郊。辛丑,豐城侯李彬鎮甘肅,召宋琥還。二月辛亥,始設貴州布政司。癸亥,令北京民戶分養孳生馬,著為令。甲子,幸北京,皇太孫從。尚書蹇義、學士黃淮、諭德楊士奇、洗馬楊溥輔皇太子監國。乙丑,發京師,命給事中、御史所過存間高年,賜酒肉及帛。丙寅,葬仁孝皇后於長陵。辛未,次鳳陽,謁皇陵。

夏四月己酉,至北京。五月丁未,曹縣獻騶虞,禮官請賀,不許。

秋七月戊寅,封阿魯台為和寧王。八月甲子,北京地震。乙丑,鎮遠侯顧成討思州、靖州叛苗。九月壬午,詔自今郡縣官每歲春行視境內,蝗蝻害稼即捕絕之,不如詔者二司并罪。

冬十月丙寅,以璽書命皇太子錄囚。十一月戊寅,以野蠶繭為衾,命皇太子薦太廟。壬午,瓦剌馬哈木兵渡飲馬河,阿魯台告警,命邊將嚴守備。甲申,寧陽侯陳懋,都督譚青、馬聚、朱崇巡寧夏、大同、山西邊,簡練士馬。尋命陜西、山西及潼關等五衛兵駐宣府,中都、遼東、河南三都指揮使司及武平等四衛兵會北京。乙巳,應城伯孫巖備開平。十二月壬子,張輔、沐晟大敗交阯賊於愛子江。

是年,馬哈木弒其主本雅失里,立答里巴為可汗。別失八里、滿剌加、占城、爪哇西王入貢。琉球中山入貢者四。琉球山南入貢者再。

十二年春正月庚寅,思州苗平。辛丑,發山東、山西、河南及鳳陽、淮安、徐、邳民十五萬,運糧赴宣府。二月己酉,大閱。庚戌,親征瓦剌,安遠侯柳升領大營,武安侯鄭亨領中軍,寧陽侯陳懋、豐城侯李彬領左、右哨,成山侯王通、都督譚青領左、右掖,都督劉江、朱榮為前鋒。庚申,振鳳翔、隴州饑,按長吏不言者罪。三月癸未,張輔俘陳季擴於老撾以獻,交阯平。庚寅,發北京,皇太孫從。

夏四月甲辰朔,次興和,大閱。己酉,頒軍中賞罰號令。庚戌,設傳令紀功官。丁卯,次屯雲谷,孛羅不花等來降。五月丁丑,命尚書、光祿卿、給事中為督陣官,察將士用命不用命者。六月甲辰,劉江遇瓦剌兵,戰於康哈里孩,敗之。戊申,次忽蘭忽失溫,馬哈木帥眾來犯,大敗之,追至土剌河,馬哈木宵遁。庚戌,班師,宣捷於阿魯台。戊午,次三峯山,阿魯台遣使來朝。己巳,以敗瓦剌詔天下。

秋七月戊子,次紅橋。詔六師入關有踐田禾取民畜產者,以軍法論。己亥,次沙河,皇太子遣使來迎。八月辛丑朔,至北京,御奉天殿受朝賀。丙午,蠲北京州縣租二年。戊午,賞從征將士。九月癸未,郭亮、徐亨備開平。丙戌,靖州苗平。甲午,費瓛鎮甘肅,劉江鎮遼東。閏月甲辰,以太子遣使迎駕緩,徵侍讀黃淮,侍講楊士奇,正字金問及洗馬楊溥、芮善下獄,未幾釋士奇復職。甲子,召吳高還。丁卯,都督朱榮鎮大同。

冬十一月甲辰,錄因。庚戌,廢晉王濟熺為庶人。庚申,蠲蘇、松、杭、嘉、湖水災田租四十七萬九千餘石。

是年,泥八剌國沙的新葛來朝,封為王。彭亨、烏斯藏入貢。真臘進金縷衣。琉球中山王貢馬。榜葛剌貢麒麟。

十三年春正月丙午,塞居庸以北隘口。丁未,馬哈木謝罪請朝貢,許之。壬子,北京午門災。戊午,敕內外諸司蠲諸宿逋,將士軍官犯罪者悉宥之。二月癸酉,遣指揮劉斌、給事中張磐等十二人巡視山西、山東、大同、陜西、甘肅、遼東軍操練、屯政,覈實以聞。甲戌,命行在禮部會試天下貢士。癸未,張輔等師還。戊子,論平交阯功,賞賚有差。三月己亥,策士於北京,賜陳循等進士及第、出身有差。丙午,廣西蠻叛,指揮同知葛森討平之。

夏四月戊辰,張輔鎮交阯。五月丁酉朔,日有食之。乙丑,鑿清江浦,通北京漕運。六月,振北京、河南、山東水災。

秋七月癸卯,鄭和還。乙巳,四川戎縣山都掌蠻平。八月庚辰,振山東、河南、北京順天州縣饑。九月壬戌,北京地震。

冬十月甲申,獵於近郊。壬辰,法司奏侵冒官糧者,帝怒,命戮之。及覆奏,帝曰:「朕過矣,仍論如律,自今死罪者皆五覆奏,著為令。」十二月,蠲順天、蘇州、鳳陽、浙江、湖廣、河南、山東州縣水旱田租。

是年,琉球山南、山北,爪哇西王,占城,古里,柯枝,南渤利,甘巴里,滿剌加,忽魯謨斯,哈密,哈烈,撒馬兒罕,火州,土魯番,蘇門答剌,俺都淮,失剌思入貢。麻林及諸番進麒麟、天馬、神鹿。琉球中山入貢者再。

十四年春正月己酉,北京、河南、山東饑,免永樂十二年逋租,發粟一百三十七萬石有奇振之。辛酉,都督金玉討山西廣靈山寇,平之。三月癸巳,都督梁福鎮胡廣、貴州。壬寅,阿魯台敗瓦剌,來獻捷。

夏四月壬申,禮部尚書呂震請封禪。帝曰:「今天下雖無事,四方多水旱疾疫,安敢自謂太平。且《六經》無封禪之文,事不師古,甚無謂也。」不聽。乙亥,胡廣為文淵閣大學士。六月丁卯,都督同知蔡福等備倭山東。

秋七月丁酉,遣使捕北京、河南、山東州縣蝗。壬寅,河決開封。乙巳,錦衣衛指揮使紀綱有罪伏誅。八月癸酉旦,壽星見,禮臣請上表賀,不許。丁亥,作北京西宮。九月癸卯,京師地震。戊申,發北京。

冬十月丁丑,次鳳陽,祀皇陵。癸未,至自北京,謁孝陵。十一月壬寅,詔文武群臣集議營建北京。丙午,召張輔還。戊申,漢王高煦有罪,削二護衛。徙山東、山西、湖廣流民於保安州,賜復三年。十二月丁卯,鄭和復使西洋。

是年,占城、古里、爪哇、滿剌加、蘇門答剌、南巫里、浡泥、彭亨、錫蘭山、溜山、南渤利、阿丹、麻林、忽魯謨斯、柯枝入貢。琉球中山入貢者再。

十五年春正月丁酉,大祀天地於南郊。壬子,平江伯陳瑄督漕,運木赴北京。二月癸亥,谷王橞有罪,廢為庶人。丁卯,豐城侯李彬鎮交阯。壬申,泰寧侯陳珪董建北京,柳升、王通副之。三月丁亥,交阯始貢士至京師。丙申,雜犯死罪以下囚,輸作北京贖罪。丙午,漢王高煦有罪,徙封樂安州。壬子,北巡,發京師,皇太子監國。

夏四月己巳,次邾城。申禁軍士毋踐民田稼,有傷者除今年租。或先被水旱逋租,亦除之。癸未,西宮成。五月丙戌,至北京。六月丁酉,李彬討交阯賊黎核,斬之。己亥,中官張謙使西洋還。敗倭寇於金鄉衛。

秋八月甲午,甌寧人進金丹。帝曰:「此妖人也。令自餌之,毀其方舊。」九月丁卯,曲阜孔了廟成,帝親製文勒石。

冬十月,李彬敗交阯賊楊進江,斬之。十一月癸酉,禮部尚書趙羾為兵部尚書,巡視塞北屯戍軍民利弊。

是年,西洋蘇祿東西峒王來朝。琉球中山、別失八里、琉球山南、真臘、浡泥、占城、暹羅、哈烈,撒馬兒罕入貢。

十六年春正月甲寅,交阯黎利反,都督朱廣擊敗之。甲戌,倭陷松門衛,按察司僉事石魯坐誅。興安伯徐亨、都督夏貴備開平。二月辛丑,交阯四忙縣賊殺知縣歐陽智以叛,李彬遣將擊走之。三月甲寅,賜李騏等進士及第、出身有差。都督僉事劉鑑備邊大同。

夏五月庚戌,重修《太祖實錄》成。丁巳,胡廣卒。

秋七月己巳,敕責陜西諸司:「比聞所屬歲屢不登,致民流莩,有司坐視不恤,又不以聞,其咎安在。其速發倉儲振之。」贊善梁潛、司諫周冕以輔導皇太子有闕,皆下獄死。

冬十二月戊子,諭法司:「朕屢敕中外官潔己愛民,而不肖官吏恣肆自若,百姓苦之。夫良農必去稂莠者,為害苗也。繼今,犯贓必論如法。」辛丑,成山侯王通馳傳振陜西饑。

是年,暹羅、占城、爪哇、蘇門答剌、泥八剌、滿剌加、南渤利、哈烈、沙哈魯、千里達、撒馬兒罕入貢。琉球中山入貢者再。

十七年春二月乙酉,興安伯徐亨備興和、開平、大同。

夏五月丙午,都督方政敗黎利於可藍柵。六月壬午,免順天府去年水災田租。戊子,劉江殲倭寇於望海堝,封江廣寧伯。

秋七月庚申,鄭和還。八月,中官馬騏激交阯乂安土知府潘僚反。九月丙辰,慶雲見,禮臣請表賀,不許。

冬十二月庚辰,諭法司曰:「刑者,聖人所慎。匹夫匹婦不得其死,足傷天地之和,召水旱之災,甚非朕寬恤之意。自今,在外諸司死罪,咸送京師審錄,三覆奏然後行刑。」乙未,工部侍郎劉仲廉覈實交阯戶口田賦,察軍民利病。

是年,哈密、土魯番、失剌思、亦思弗罕、真臘、占城、哈烈、阿魯、南渤利、蘇門答剌、八答黑商、滿剌加入貢。琉球中山入貢者四。

十八年春正月癸卯,李彬及都指揮孫霖、徐謜敗黎利於磊江。閏月丙子,翰林院學士楊榮、金幼孜為文淵閣大學士。庚辰,擢人材,布衣馬麟等十三人為布政使、參政、參議。二月己酉,蒲臺妖婦唐賽兒作亂,安遠侯柳升帥師討之。三月辛巳,敗賊於御石柵寨,都指揮劉忠戰沒,賽兒逸去。甲申,山東都指揮僉事衛青敗賊於安丘,指揮王真敗賊於諸城,獻俘京師。戊子,山東布政使儲埏、張海,按察使劉本等坐縱盜誅。戊戌,以逗留徵柳升下吏,尋釋之。

夏五月壬午,左都督朱榮鎮遼東。庚寅,交阯參政侯保、馮貴禦賊,戰死。六月丙午,北京地震。

秋七月丁亥,徐亨備開平。八月丁酉朔,日有食之。九月己巳,召皇太子。丁亥,詔自明年改京師為南京,北京為京師。

冬十月庚申,李彬遣指揮使方政敗黎利於老撾。十一月戊辰,以遷都北京詔天下。是月,振青、萊饑。十二月己未,皇太子及皇太孫至北京。癸亥,北京郊廟宮殿成。

是年,始設東廠,命中官剌事。古麻剌朗王來朝。暹羅、占城、爪哇、滿剌加、蘇門答剌、蘇祿西王入貢。

十九年春正月甲子朔,奉安五廟神主於太廟。御奉天殿受朝賀,大宴。甲戌,大祀天地於南郊。戊寅,大赦天下。癸巳,鄭和復使西洋。二月辛丑,都督僉事胡原帥師巡海捕倭。三月辛巳,賜曾鶴齡等進士及第、出身有差。

夏四月庚子,奉天、華蓋、謹身三殿災,詔群臣直陳闕失。乙巳,詔罷不便於民及不急諸務,蠲十七年以前逋賦,免去年被災田糧。己酉,萬壽節,以三殿災止賀。癸丑,蹇義等二十六人巡行天下,安撫軍民。五月乙丑,出建言給事中柯暹,御史何忠、鄭維桓、羅通等為知州。庚寅,令交阯屯田。

秋七月己巳,帝將北征,敕都督朱榮領前鋒,安遠侯柳升領中軍,寧陽侯陳懋領御前精騎,永順伯薛斌、恭順伯吳克忠領馬隊,武安侯鄭亨、陽武侯薛祿領左右哨,英國公張輔、成山侯王通領左右掖。八月辛卯朔,日有食之。

冬十一月辛酉,分遣中官楊實、御史戴誠等覈天下庫藏出納之數。丙子,議北征軍餉,下戶部尚書夏原吉、刑部尚書吳中於獄,兵部尚書方賓自殺。辛巳,下侍讀李時勉於獄。甲申,發直隸、山西、河南、山東及南畿應天等五府,滁、和、徐三州丁壯運糧,期明年二月至宣府。

是年,瓦剌賢義王太平、安樂王把禿孛羅來朝。忽魯謨斯、阿丹、祖法兒、剌撒、不剌哇、木骨都東、古里、柯枝、加異勒、錫蘭山、溜山、南渤利、蘇門答剌、阿魯、滿剌加、甘巴里、蘇祿、榜葛剌、浡泥、古麻剌朗王入貢。暹羅入貢者再。

二十年春正月己未朔,日有食之,免朝賀,詔群臣修省。辛未,大祀天地於南郊。壬申,豐城侯李彬卒於交阯。二月乙巳,隆平侯張信、兵部尚書李慶分督北征軍餉,役民夫二十三萬五千有奇,運糧三十七萬石。三月丙寅,詔有司遇災先振後聞。乙亥,阿魯台犯興和,都指揮王喚戰死。丁丑,親征阿魯台,皇太子監國。戊寅,發京師。辛巳,次雞鳴山,阿魯台遁。

夏四月乙卯,次雲州,大閱。五月乙丑,獵於偏嶺。丁卯,大閱。辛未,次西涼亭。壬申,大閱。乙酉,次開平。六月壬辰,令軍行出應昌,結方陣以進。癸巳,諜報阿魯台兵攻萬全,諸將請分兵還擊,帝曰:「詐也。彼慮大軍搗其巢穴,欲以牽制我師,敢攻城哉。」甲午,次陽和谷,寇攻萬全者果遁去。

秋七月己未,阿魯台棄輜重於濶欒海側北遁,發兵焚之,收其牲畜,遂旋師。謂諸將曰:「阿魯台敢悖逆,恃兀良哈為羽翼也。當還師翦之。」簡步騎二萬,分五道並進。庚午,遇於屈裂兒河,帝親擊敗之,追奔三十里,斬部長數十人。辛未,徇河西,捕斬甚眾。甲戌,兀良哈餘黨詣軍門降。是月,皇太子免南、北直隸、山東、河南郡縣水災糧芻共六十一萬有奇。八月戊戌,諸將分道者俱獻捷。辛丑,以班師詔天下。壬寅,鄭亨、薛祿守開平。鄭和還。九月壬戌,至京師。癸亥,下左春坊大學士楊士奇於獄。丙寅,下吏部尚書蹇義、禮部尚書呂震於獄,尋俱釋之。辛未,錄從征功,封左都督朱榮武進伯,都督僉事薛貴安順伯。

冬十月癸巳,分遣中官及朝臣八十人覈天下倉糧出納之數。十二月辛卯,朱榮鎮遼東。閏月戊寅,乾清宮災。

是年,暹羅、蘇門答剌、阿丹等國遣使隨貢方物。占城、琉球中山、卜花兒、哈密、瓦剌、土魯番、爪哇入貢。

二十一年春正月乙未,大祀天地於南郊。癸卯,交阯參將榮昌伯陳智追敗黎利於車來。二月己巳,都指揮使鹿榮討柳州叛蠻,平之。三月庚子,御史王愈等會決重囚,誤殺無罪四人,坐棄市。

夏五月癸未,免開封、南陽、衛輝、鳳陽等府去年水災田租。己丑,常山護衛指揮孟賢等謀逆,伏誅。六月庚戌朔,日有食之。

秋七月戊戌,復親征阿魯台,安遠侯柳升、遂安伯陳英領中軍,武安侯鄭亨、保定侯孟瑛領左哨,陽武侯薛祿、新寧伯譚忠領右哨,英國公張輔、安平伯李安領左掖,成山侯王通、興安伯徐亨領右掖,寧陽侯陳懋領前鋒。庚子,釋李時勉,復其官。辛丑,皇太子監國。壬寅,發京師。戊申,次宣府,敕居庸關守將止諸司進奉。八月己酉,大閱。庚申,塞黑峪、長安嶺諸邊險要。丁丑,皇太子免兩京、山東郡縣水災田租。九月戊子,次西陽河。癸巳,聞阿魯台為瓦剌所敗,部落潰散,遂駐師不進。

冬十月甲寅,次上莊堡,迤北王子也先土干帥所部來降,封忠勇王,賜姓名金忠。庚午,班師。十一月甲申,至京師。

是年,錫蘭山王來朝,又遣使入貢。占城、古里、忽魯謨斯、阿丹、祖法兒、剌撒、不剌哇、木骨都束、柯枝、加異勒、溜山、南渤利、蘇門答剌、阿魯、滿剌加、失剌思、榜葛剌、琉球中山入貢。

二十二年春正月甲申,阿魯台犯大同、開平,詔群臣議北征,敕邊將整兵俟命。丙戌,徵山西、山東、河南、陜西、遼東五都司及西寧、鞏昌、洮、岷各衛兵,期三月會北京及宣府。戊子,大祀天地於南郊。癸巳,鄭和復使西洋。三月戊寅,大閱,諭諸將親征。命柳升、陳英領中軍,張輔、朱勇領左掖,王通、徐亨領右掖,鄭亨、孟瑛領左哨,薛祿、譚忠領右哨、陳懋、金忠領前鋒。己卯,賜邢寬等進士及第、出身有差。

夏四月戊申,皇太子監國。己酉,發京師。庚午,次隰寧,諜報阿魯台走答蘭納木兒河,遂趨進師。五月己卯,次開平,使使招諭阿魯台諸部。乙酉,瘞道中遺骸。丁酉,宴群臣於應昌,命中官歌太祖御製詞五章,曰:「此先帝所以戒後嗣也,雖在軍旅何敢忘。」己亥,次威遠州。復宴群臣,自製詞五章,命中官歌之。皇太子令兔廣平、順德、揚州及湖廣、河南郡縣水災田租。六月庚申,前鋒至答蘭納木兒河,不見敵,命張輔等窮搜山谷三百里無所得,進駐河上。癸亥,陳懋等引兵抵白邙山,以糧盡還。甲子,班師,命鄭亨等以步卒西會於開平。壬申夜,南京地震。

秋七月庚辰,勒石於清水源之崖。戊子,遣呂震以旋師諭太子,詔告天下。己丑,次蒼崖戍,不豫。庚寅,至榆木川,大漸。遺詔傳位皇太子,喪禮一如高皇帝遺制。辛卯,崩,年六十有五。太監馬雲密與大學士楊榮、金幼孜謀,以六軍在外,祕不發喪,熔錫為椑以斂,載以龍轝,所至朝夕上膳如常儀。壬辰,楊榮偕御馬監少監海壽馳訃皇太子。壬寅,次武平鎮,鄭亨步軍來會。八月甲辰,楊榮等至京師,皇太子即日遣太孫奉迎於開平。己酉,次雕鶚谷,皇太孫至軍中發喪。壬子,及郊,皇太子迎入仁智殿,加殮納梓宮。九月壬午,上尊謚曰體天弘道高明廣運聖武神功純仁至孝文皇帝,廟號太宗,葬長陵。嘉靖十七年九月,改上尊謚曰啟天弘道高明肇運聖武神功純仁至孝文皇帝,廟號成祖。

贊曰:「文皇少長習兵,據幽燕形勝之地,乘建文孱弱,長驅內向,奄有四海。即位以後,躬行節儉,水旱朝告夕振,無有壅蔽。知人善任,表裏洞達,雄武之略,同符高祖。六師屢出,漠北塵清。至其季年,威德遐被,四方賓服,受朝命而入貢者殆三十國。幅隕之廣,遠邁漢、唐。成功駿烈,卓乎盛矣。然而革除之際,倒行逆施,慚德亦曷可掩哉。

\end{pinyinscope}