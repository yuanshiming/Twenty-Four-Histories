\article{列傳第一百七十五 文苑三}

\begin{pinyinscope}
○文徵明蔡羽等黃佐歐大任黎民表柯維騏王慎中屠應埈等高叔嗣蔡汝楠陳束任瀚熊過李開先田汝成子藝蘅皇甫涍弟沖汸濂茅坤子維謝榛盧柟李攀龍梁有譽等王世貞汪道昆胡應麟弟世懋歸有光子子慕胡友信

文徵明,長洲人,初名璧,以字行,更字徵仲,別號衡山。父林,溫州知府。叔父森,右僉都御史。林卒,吏民醵千金為賻。徵明年十六,悉卻之。吏民修故卻金亭,以配前守何文淵,而記其事。

徵明幼不慧,稍長,穎異挺發。學文於吳寬,學書於李應禎,學畫於沈周,皆父友也。又與祝允明、唐寅、徐禎卿輩相切劘,名日益著。其為人和而介。巡撫俞諫欲遺之金,指所衣藍衫,謂曰:「敝至此邪?」徵明佯不喻,曰:「遭雨敝耳。」諫竟不敢言遺金事。寧王宸濠慕其名,貽書幣聘之,辭病不赴。

正德末,巡撫李充嗣薦之,會徵明亦以歲貢生詣吏部試,奏授翰林院待詔。世宗立,預修《武宗實錄》,侍經筵,歲時頒賜,與諸詞臣齒。而是時專尚科目,徵明意不自得,連歲乞歸。

先是,林知溫州,識張璁諸生中。璁既得勢,諷征明附之,辭不就。楊一清召入輔政,徵明見獨後。一清亟謂曰:「子不知乃翁與我友邪?」徵明正色曰:「先君棄不肖三十餘年,茍以一字及者,弗敢忘,實不知相公與先君友也。」一清有慚色,尋與璁謀,欲徙徵明官。徵明乞歸益力,乃獲致仕。四方乞詩文書畫者,接踵於道,而富貴人不易得片楮,尤不肯與王府及中人,曰:「此法所禁也。」周、徽諸王以寶玩為贈,不啟封而還之。外國使者道吳門,望里肅拜,以不獲見為恨。文筆遍天下,門下士贗作者頗多,徵明亦不禁。嘉靖三十八年卒,年九十矣。長子彭,字壽承,國子博士。次子嘉,字休承,和州學正。並能詩,工書畫篆刻,世其家。彭孫震孟,自有傳。

吳中自吳寬、王鏊以文章領袖館閣,一時名士沈周、祝允明輩與並馳騁,文風極盛。徵明及蔡羽、黃省曾、袁CW、皇甫沖兄弟稍後出。而徵明主風雅數十年,與之遊者王寵、陸師道、陳道復、王穀祥、彭年、周天球、錢穀之屬,亦皆以詞翰名於世。

蔡羽,字九逵,由國子生授南京翰林院孔目。自號林屋山人,有《林屋》、《南館》二集。自負甚高。文法先秦、兩漢。或謂其詩似李賀,羽曰:「吾詩求出魏、晉上,今乃為李賀邪!」其不肯屈抑如此。

黃省曾,字勉之。舉鄉試。從王守仁、湛若水游,又學詩於李夢陽。所著有《五嶽山人集》。子姬水,字淳父,有文名,學書於祝允明。

袁CW,字永之,七歲能詩。舉嘉靖五年進士,改庶吉士。張璁惡之,出為刑部主事,累遷廣西提學僉事。兩廣自韓雍後,監司謁督府,率庭跪,CW獨長揖。無何,謝病歸。子尊尼;字魯望,亦官山東提學副使,有文名。

王寵,字履吉,別號雅宜。少學於蔡羽,居林屋者三年,既而讀書石湖。由諸生貢入國子,僅四十而卒。行楷得晉法,書無所不觀。

陸師道,字子傳。由進士授工部主事,改禮部,以養母請告歸。歸而游徵明門,稱弟子。家居十四年,乃復起,累官尚寶少卿。善詩文,工小楷古篆繪事。人謂徵明四絕,不減趙孟頫,而師道並傳之,其風尚亦略相似。平居不妄交游,長吏罕識其面。女字卿子,適趙宦光,夫婦皆有聞於時。

陳道復,名淳,以字行。祖璚,副都御史。淳受業徵明,以文行著,善書畫,自號白陽山人。

王穀祥,字祿之。由進士改庶吉士,歷官吏部員外郎。忤尚書汪鋐,左遷真定通判以歸。與師道俱有清望。

彭年,字孔嘉,其人亦長者。周天球,字公瑕;錢穀,字叔寶。天球以書,穀以畫,皆繼徵明表表吳中者也。其後,華亭何良俊亦以歲貢生入國學。當路知其名,用蔡羽例,特授南京翰林院孔目。良俊,字元朗。少篤學,二十年不下樓,與弟良傅並負俊才。良傅舉進士,官南京禮部郎中,而良俊猶滯場屋,與上海張之象,同里徐獻忠、董宜陽友善,並有聲。及官南京,趙貞吉、王維楨相繼掌院事,與相得甚歡。良俊居久之,慨然歎曰:「吾有清森閣在海上,藏書四萬卷,名畫百簽,古法帖彞鼎數十種,棄此不居,而僕僕牛馬走乎!」遂移疾歸。海上中倭,復居金陵者數年,更買宅居吳閶。年七十始返故里。

徐獻忠,字伯臣。嘉靖中,舉於鄉,官奉化知縣。著書數百卷。卒年七十七,王世貞私謚曰貞憲。

董宜陽,字子元。

張之象,字月鹿。祖萱,湖廣參議。父鳴謙,順天通判。之象由諸生入國學,授浙江按察司知事,以吏隱自命。歸益務撰著。晚居秀林山,罕入城市。卒年八十一。

黃佐,字才伯,香山人。祖瑜,長樂知縣,以學行聞。正德中,佐舉鄉試第一。世宗嗣位,始成進士,選庶吉士。嘉靖初,授編修,陳初政要務,又請修舉新政,疏皆留中。尋省親歸,便道謁王守仁,與論知行合一之旨,數相辨難,守仁亦稱其直諒。還朝,會出諸翰林為外僚,除江西僉事。旋改督廣西學校,聞母病,引疾乞休,不俟報竟去,下巡撫林富逮問。富言佐誠有罪,第為親受過,於情可原,乃令致仕。家居九年,簡宮僚,命以編修兼司諫,尋進侍讀,掌南京翰林院。召為右諭德,擢南京國子祭酒。母憂除服,起少詹事。謁大學士夏言,與論河套事不合。會吏部缺左侍郎,所司推禮部右侍郎崔桐及佐。給事中徐霈、御史艾朴言:「桐與左侍郎許成名競進,至相詬詈;而佐及同官王用賓亦爭覬望,惟恐或先之,宜皆止勿用。」言從中主之,遂皆賜罷。

佐學以程、朱為宗,惟理氣之說,獨持一論。平生譔述至二百六十餘卷。所著《樂典》,自謂洩造化之秘。年七十七卒。穆宗詔贈禮部右侍郎,謚文裕。

佐弟子多以行業自飭,而梁有譽、歐大任、黎民表詩名最著云。歐大任,字楨伯,順德人。由歲貢生歷官南京工部郎中,年八十而終。黎民表,字惟敬,從化人,御史貫子也。舉鄉試,久不第,授翰林孔目,遷吏部司務。執政知其能文,用為制敕房中書,供事內閣,加官至參議。

柯維騏,字奇純,莆田人。高祖潛,翰林學士。父英,徽州知府。維騏舉嘉靖二年進士,授南京戶部主事,未赴,輒引疾歸。張孚敬用事,創新制,京朝官病滿三年者,概罷免,維騏亦在罷中。自是謝賓客,專心讀書。久之,門人日進,先後四百餘人,維騏引掖靡倦。慨近世學者樂徑易而憚積累,竊二氏之說以文其固陋也,作左右二銘,訓學者務實。以辨心術、端趨向為實志,以存敬畏、密操履為實功,而其極則以宰理人物、成能天地為實用,作講義二卷。《宋史》與《遼》、《金》二《史》,舊分三書,維騏乃合之為一,以遼、金附之,而列二王於本紀。褒貶去取,義例嚴整,閱二十年而始成,名之曰《宋史新編》。又著《史記考要》、《續莆陽文獻志》,及所作詩文集並行於世。

維騏登第五十載,未嘗一日服官。中更倭亂,故廬焚燬,家困甚,終不妄取。世味無所嗜,惟嗜讀書。撫按監司時有論薦,不復起。隆慶初,廷臣復薦。所司以維騏年高,但授承德郎致仕。卒年七十有八。孫茂竹,海陽知縣。茂竹子昶,副都御史,巡撫山西。

王慎中,字道思,晉江人。四歲能誦詩,十八舉嘉靖五年進士,授戶部主事,尋改禮部祠祭司。時四方名士唐順之、陳束、李開先、趙時春、任瀚、熊過、屠應埈、華察、陸銓、江以達、曾忭輩,咸在部曹。慎中與之講習,學大進。十二年,詔簡部郎為翰林,眾首擬慎中。大學士張孚敬欲一見,辭不赴,乃稍移吏部,為考功員外郎,進驗封郎中。忌者讒之孚敬,因覆議真人張衍慶請封疏,謫常州通判。稍遷戶部主事、禮部員外郎,並在南京。久之,擢山東提學僉事,改江西參議,進河南參政。侍郎王杲奉命振荒,以其事委慎中,還朝,薦慎中可重用。會二十年大計,吏部註慎中不及。而大學士夏言先嘗為禮部尚書,慎中其屬吏也,與相忤,遂內批不謹,落其職。

慎中為文,初主秦、漢,謂東京下無可取。已悟歐、曾作文之法,乃盡焚舊作,一意師仿,尤得力於曾鞏。順之初不服,久亦變而從之。壯年廢棄,益肆力古文,演迤詳贍,卓然成家,與順之齊名,天下稱之曰王、唐,又曰晉江、毘陵。家居,問業者踵至。年五十一而終。李攀龍、王世貞後起,力排之,卒不能掩。攀龍,慎中提學山東時所賞拔者也。慎中初號遵巖居士,後號南江。

屠應埈,字文升,平湖人,刑部尚書勳子也。舉嘉靖五年進士。由郎中改翰林,官至右諭德。

華察,字子潛,無錫人。應埈同年進士。累官侍講學士,掌南京翰林院。

陸銓,字選之,鄞人。嘉靖二年進士。與弟編修釴爭大禮,並繫詔獄,被杖,後官廣西布政使。釴終山東提學副使,兄弟皆能文。

江以達,字子順,貴溪人。嘉靖五年進士。累官福建提學僉事。

高叔嗣,字子業,祥符人。年十六,作《申情賦》幾萬言,見者驚異。十八舉於鄉,第嘉靖二年進士。授工部主事,改吏部。歷稽勳郎中。出為山西左參政,斷疑獄十二事,人稱為神。遷湖廣按察使,卒官,年三十有七。

叔嗣少受知邑人李夢陽,及官吏部,與三原馬理、武城王道同署,以文藝相磨切。其為詩,清新婉約,雖為夢陽所知,不宗其說。陳束序其《蘇門集》,謂有應物之沖澹,兼曲江之沈雄,體王、孟之清適,具高、岑之悲壯。王世貞則曰:「子業詩,如高山鼓琴,沈思忽往,木葉盡脫,石氣自青;又如衛洗馬言愁,憔瘁婉篤,令人心折。」而蔡汝楠至推為本朝第一云。兄仲嗣,官知府,亦有才名。

汝楠,字子木。兒時隨父南京,聽祭酒湛若水講學,輒有解悟。年十八,成嘉靖十一年進士,授行人。從王慎中、唐順之及叔嗣輩學為詩。尋進刑部員外郎,徙南京刑部。善皇甫涍兄弟,尚書顧璘引為忘年友。廷議改歸德州為府,擢汝楠知其府事。以母憂歸,聚諸生石鼓書院,與說經。治民有惠政,既去,士民祠祀之。歷官江西左、右布政使,擢右副都御史,巡撫河南。召為兵部右侍郎,從諸大僚祝釐西宮,世宗望見其貌寢,改南京工部右侍郎,未幾卒。

汝楠始好為詩,有重名。中年好經學,及官江西,與鄒守一、羅洪先游,學益進,然詩由此不工去。

陳束,字約之,鄞人。生而聰慧絕倫,好讀古書。會稽侍郎董官翰林時,聞束才,召視之。東垂髫而前,試詞賦立就,遂字以女,攜至京,文譽益起。嘉靖八年廷對,世宗親擢羅洪先、程文德、楊名為一甲,而置唐順之及束、任瀚於二甲,皆手批其卷。無何,考庶吉士,得胡經等二十人,以束及順之、瀚曾奉御批,列經等首。座主張璁、霍韜以前此館選悉改他曹,引嫌,亦議改,乃寢前令,束授禮部主事。時有「嘉靖八才子」之稱,謂束及王慎中、唐順之、趙時春、熊過、任瀚、李開先、呂高也。四郊改建,都御史汪鋐請徙近郊居民墳墓,束疏諫,不報。遷員外郎,改編修。

束出璁、韜門,不肯親附。歲時上壽,望門投刺,輒馳馬過之。為所惡,出為湖廣僉事。分巡辰、沅,治有聲。稍遷福建參議,改河南提學副使。束故有嘔血疾,會科試期近,試八郡之士,三月而畢,疾增劇,竟不起,年才三十有三。妻董,亦能詩,束卒未幾亦卒,束竟無後。

當嘉靖初,稱詩者多宗何、李,束與順之輩厭而矯之。束早世,且槁多散逸,今所傳《后岡集》,僅十之一二云。

任瀚,字少海,南充人。嘉靖八年進士。改庶吉士,未上,授吏部主事。屢遷考功郎中。十八年,簡宮僚,改左春坊左司直兼翰林院檢討。明年,拜疏引疾,出郭戒行,疏再上,不報,復自引還。給事中周來劾瀚舉動任情,蔑視官守。帝令自陳,瀚語侵掌詹事霍韜。帝怒,勒為民。久之,遇赦,復官致仕。終世宗朝,中外屢薦,不復用。神宗嗣位,四川巡撫劉思潔、曾省吾先後疏薦,優旨報聞而已。瀚少懷用世志,百家二氏之書,罔不搜討。被廢,益反求《六經》,闡明聖學。晚又潛心於《易》,深有所得。文亦高簡。卒年九十三。

熊過,字叔仁,富順人。瀚同年進士。累官祠祭郎中,坐事貶秩,復除名為民。

李開先,字伯華,章丘人。束同年進士。官至太常少卿。性好蓄書,李氏藏書之名聞天下。

呂高,字山甫,丹徒人。亦束同年進士。歷官山東提學副使。鄉試錄文,舊多出學使者手,巡按御史葉經乞順之文。高心憾,寓書京師友人言經紕繆。嚴嵩惡經,遂置之死。及後大計,諸御史謂經禍由高,乃斥歸,於八子中,名最下。

田汝成,字叔禾,錢塘人。嘉靖五年進士。授南京刑部主事,尋召改禮部。十年十二月上言:「陛下以青宮久虛,祈天建醮,復普放生之仁,凡羈蹄金殺羽禁在上林者,咸獲縱釋。顧使囹圄之徒久纏徽纆,衣冠之侶流竄窮荒,父子長離,魂魄永喪,此獨非陛下之赤子乎!望大廣皇仁,悉加寬宥。」忤旨,切責,停俸二月。屢遷祠祭郎中,廣東僉事,謫知滁州。復擢貴州僉事,改廣西右參議,分守右江。龍州土酋趙楷、憑祥州土酋李寰皆弒主自立,與副使翁萬達密討誅之。努灘賊侯公丁為亂,斷藤峽群賊與相應。汝成復偕萬達設策誘擒公丁,而進兵討峽賊,大破之,又與萬達建善後七事,一方遂靖,有銀幣之賜。遷福建提學副使。歲當大比,預定諸生甲乙。比榜發,一如所定。

汝成博學工古文,尤善敘述。歷官西南,諳曉先朝遺事,撰《炎徼紀聞》。歸田後,般桓湖山,窮浙西諸名勝,撰《西湖游覽志》,並見稱於時。他所論著甚多,時推其博洽。子藝蘅,字子。十歲從父過采石,賦詩有警句。性放誕不羈,嗜酒任俠。以歲貢生為徽州訓導,罷歸。作詩有才調,為人所稱。

皇甫涍,字子安,長洲人。父錄,弘治九年進士。任重慶知府。生四子,沖、涍、汸、濂。沖、汸同登嘉靖七年鄉薦,明年,汸第進士。又三年,涍第進士。又十三年,濂亦第進士。而沖尚為舉子。兄弟並好學工詩,稱「皇甫四傑」。

沖,字子浚,善騎射,好談兵。遇南北內訌,譔《幾策》、《兵統》、《枕戈雜言》三書,凡數十萬言。涍,初授工部主事,改禮部。歷儀制員外郎,主客郎中。在儀制時,夏言為尚書,連疏請建儲,皆涍起草,故言深知涍才。比簡宮僚,遂用為春坊司直兼翰林檢討。言者論涍改官有私,謫廣平通判,量移南京刑部主事,進員外郎,遷浙江僉事。大計京官,以南曹事論罷,邑邑發病卒。涍沈靜寡與,自負高俊,稍不當意,終日相對無一言。居官砥廉隅,然頗操切,多忤物,故數被讒謗云。

汸,字子循,七歲能詩。官工部主事,名動公卿,沾沾自喜,用是貶秩為黃州推官。屢遷南京稽勳郎中,再貶開州同知,量移處州府同知。擢雲南僉事,以計典論黜。汸和易,近聲色,好狎游。於兄弟中最老壽,年八十乃卒。

濂,字子約,初授工部主事,母喪除,起故官,典惜薪廠。賈人偽增數罔利,濂按其罪。賈人女為尚書文明妾,明召濂切責之。濂抗言曰:「公掌邦政,縱奸人干紀,又欲奪郎官法守邪?」明為斂容謝。大計,謫河南布政司理問,終興化同知。

濂兄弟與黃魯曾、省曾為中表兄弟,文藻亦相似。其後,里人張鳳翼、燕翼、獻翼並負才名。吳人語曰:「前有四皇,後有三張。」鳳翼、燕翼終舉人。而獻翼為太學生,名日益高,年老矣,狂甚,為讎家所殺。

茅坤,字順甫,歸安人。嘉靖十七年進士。歷知青陽、丹徒二縣。母憂,服闋,遷禮部主事,移吏部稽勳司,坐累,謫廣平通判。屢遷廣西兵備僉事,轄府江道。坤雅好談兵。瑤賊據鬼子諸砦,殺陽朔令。朝議大征,總督應檟以問坤。坤曰:「大征非兵十萬不可,餉稱之,今猝不能集,而賊已據險為備。計莫若雕剿。條入殲其魁,他部必襲,謀自全,此便計也。」檟善之,悉以兵事委坤。連破十七砦,晉秩二等。民立祠祀之。遷大名兵備副使,總督楊博歎為奇才,特薦於朝。為忌者所中,追論其先任貪污狀,落職歸。時倭事方急,胡宗憲延之幕中,與籌兵事,奏請為福建副使。吏部持之,乃已。家人橫於里,為巡按龐尚鵬所劾,遂褫冠帶。坤既廢,用心計治生,家大起。年九十,卒於萬曆二十九年。

坤善古文,最心折唐順之。順之喜唐、宋諸大家文,所著文編,唐、宋人自韓、柳、歐、三蘇、曾、王八家外,無所取,故坤選《八大家文鈔》。其書盛行海內,鄉里小生無不知茅鹿門者。鹿門,坤別號也。少子維,字孝若,能詩,與同郡臧懋循、吳稼竳、吳夢陽,並稱四子。嘗詣闕上書,希得召見,陳當世大事,不報。

謝榛,字茂秦,臨清人。眇一目。年十六,作樂府商調,少年爭歌之。已,折節讀書,刻意為歌詩。西游彰德,為趙康王所賓禮。入京師,脫盧柟於獄。

李攀龍、王世貞輩結詩社,榛為長,攀龍次之。及攀龍名大熾,榛與論生平,頗相鐫責,攀龍遂貽書絕交。世貞輩右攀龍,力相排擠,削其名於七子之列。然榛游道日廣,秦、晉諸王爭延致,大河南北皆稱謝榛先生。趙康王卒,榛乃歸。萬曆元年冬,復游彰德,王曾孫穆王亦賓禮之。酒闌樂止,命所愛賈姬獨奏琵琶,則榛所製竹枝詞也。榛方傾聽,王命姬出拜,光華射人,藉地而坐,竟十章。榛曰:「此山人里言耳,請更制,以備房中之奏。」詰朝上新詞十四闋,姬悉按而譜之。明年元旦,便殿奏伎,酒止送客,即盛禮而歸姬於榛。榛游燕、趙間,至大名,客請賦壽詩百章,成八十餘首,投筆而逝。

當七子結社之始,尚論有唐諸家,各有所重。榛曰:「取李、杜十四家最勝者,熟讀之以會神氣,歌詠之以求聲調,玩味之以裒精華。得經三要,則浩乎渾淪,不必塑謫仙而畫少陵也。」諸人心師其言,厥後雖合力擯榛,其稱詩指要,實自榛發也。

盧柟,字少楩,濬縣人。家素封,輸貲為國學生。博聞強記,落筆數千言。為人跅馳,好使酒罵座。常為具召邑令,日晏不至,柟大怒,徹席滅炬而臥。令至,柟已大醉,不具賓主禮。會柟役夫被榜,他日牆壓死,令即捕柟,論死,繫獄,破其家。里中兒為獄卒,恨柟,笞之數百,謀以土囊壓殺之,為他卒救解。柟居獄中,益讀所攜書,作《幽鞫》、《放招》二賦,詞旨沈鬱。

謝榛入京師,見諸貴人,泣訴其冤狀曰:「生有一盧柟不能救,乃從千古哀沅而弔湘乎!」平湖陸光祖遷得浚令,因榛言平反其獄。柟出,走謁榛。榛方客趙康王所,王立召見柟,禮為上賓。諸宗人以王故爭客柟,柟酒酣罵座如故。及光祖為南京禮部郎,柟往訪之,遍游吳會無所遇,還益落魄嗜酒,病三日卒。柟騷賦最為王世貞所稱,詩亦豪放如其為人。

李攀龍,字于鱗,歷城人。九歲而孤,家貧,自奮於學。稍長為諸生,與友人許邦才、殷士儋學為詩歌。已,益厭訓詁學,日讀古書,里人共目為狂生。舉嘉靖二十三年進士,授刑部主事。歷員外郎、郎中,稍遷順德知府,有善政。上官交薦,擢陜西提學副使。鄉人殷學為巡撫,檄令屬文,攀龍怫然曰:「文可檄致邪?」拒不應。會其地數震,攀龍心悸,念母思歸,遂謝病。故事,外官謝病不再起,吏部重其才,用何景明便,特予告歸。予告者,例得再起。

攀龍既歸,構白雪樓,名日益高。賓客造門,率謝不見,大吏至,亦然,以是得簡傲聲。獨故交殷、許輩過從靡間。時徐中行亦家居,坐客恒滿,二人聞之,交相得也。歸田將十年,隆慶改元,薦起浙江副使,改參政,擢河南按察使。攀龍至是摧亢為和,賓客亦稍稍進。。無何,奔母喪歸,哀毀得疾,疾少間,一日心痛卒。

攀龍之始官刑曹也,與濮州李先芳、臨清謝榛、孝豐吳維岳輩倡詩社。王世貞初釋褐,先芳引入社,遂與攀龍定交。明年,先芳出為外吏。又二年,宗臣、梁有譽入,是為五子。未幾,徐中行、吳國倫亦至,乃改稱七子。諸人多少年,才高氣銳,互相標榜,視當世無人,七才子之名播天下。擯先芳、維岳不與,已而榛亦被擯,攀龍遂為之魁。其持論謂文自西京,詩自天寶而下,俱無足觀,於本朝獨推李夢陽。諸子翕然和之,非是,則詆為宋學。攀龍才思勁鷙,名最高,獨心重世貞,天下亦並稱王、李。又與李夢陽、何景明並稱何、李、王、李。其為詩,務以聲調勝,所擬樂府,或更古數字為己作,文則聱牙戟口,讀者至不能終篇。好之者推為一代宗匠,亦多受世抉摘云。自號滄溟。

梁有譽、宗臣、徐中行、吳國倫,皆嘉靖二十九年進士。有譽除刑部主事,居三年,以念母告歸,杜門讀書。大吏至,辭不見。卒年三十六。

宗臣,字子相,揚州興化人。由刑部主事調考功,謝病歸,築室百花洲上,讀書其中。起故官,移文選。進稽勳員外郎,嚴嵩惡之,出為福建參議。倭薄城,臣守西門,納鄉人避難者萬人。或言賊且迫,曰:「我在,不憂賊也。」與主者共擊退之。尋遷提學副使,卒官,士民皆哭。

徐中行,字子輿,長興人。美姿容,善飲酒。由刑部主事歷員外郎、郎中,稍遷汀州知府。廣東賊蕭五來犯,禦之,有功。策其且走,俾武平令徐甫宰邀擊之,讓功甫宰,甫宰得優擢。尋以父憂歸,補汝寧,坐大計,貶長蘆鹽運判官。行湖廣僉事,掩捕湖盜柯彩鳳,得其積貯,活饑民萬餘。累官江西左布政使,萬曆六年卒官。中行性好客,無賢愚貴賤,應之不倦,故其死也,人多哀之。

吳國倫,字明卿,興國人。由中書舍人擢兵科給事中。楊繼盛死,倡眾賻送,忤嚴嵩,假他事謫江西按察司知事。量移南康推官,調歸德,居二歲棄去。嵩敗,起建寧同知,累遷河南左參政,大計罷歸。國倫才氣橫放,好客輕財。歸田後聲名籍甚,求名之士,不東走太倉,則西走興國。萬曆時,世貞既沒,國倫猶無恙,在七子中最為老壽。

王世貞,字元美,太倉人,右都御史忬子也。生有異稟,書過目,終身不忘。年十九,舉嘉靖二十六年進士。授刑部主事。世貞好為詩古文,官京師,入王宗沐、李先芳、吳維岳等詩社,又與李攀龍、宗臣、梁有譽、徐中行、吳國倫輩相倡和,紹述何、李,名日益盛。屢遷員外郎、郎中。

奸人閻姓者犯法,匿錦衣都督陸炳家,世貞搜得之。炳介嚴嵩以請,不許。楊繼盛下吏,時進湯藥。其妻訟夫冤,為代草。既死,復棺殮之。嵩大恨。吏部兩擬提學皆不用,用為青州兵備副使。父忬以濼河失事,嵩構之,論死繫獄。世貞解官奔赴,與弟世懋日蒲伏嵩門,涕泣求貸。嵩陰持忬獄,而時為謾語以寬之。兩人又日囚服跽道旁,遮諸貴人輿,搏顙乞救。諸貴人畏嵩不敢言,忬竟死西市。兄弟哀號欲絕,持喪歸,蔬食三年,不入內寢。既除服,猶卻冠帶,苴履葛巾,不赴宴會。隆慶元年八月,兄弟伏闕訟父冤,言為嵩所害,大學士徐階左右之,復忬官。世貞意不欲出,會詔求直言,疏陳法祖宗、正殿名、慶恩義、寬禁例、修典章、推德意、昭爵賞、練兵實八事,以應詔。無何,吏部用言官薦,令以副使涖大名。遷浙江右參政,山西按察使。母憂歸,服除,補湖廣,旋改廣西右布政使,入為太僕卿。

萬曆二年九月以右副都御史撫治鄖陽,數條奏屯田、戍守、兵食事宜,咸切大計。有奸僧偽稱樂平王次子,奉高皇帝御容、金牒,行游天下。世貞曰:「宗籓不得出城,而言壽張如此,必偽也。」捕訊之,服辜。張居正枋國,以世貞同年生,有意引之,世貞不甚親附。所部荊州地震,引京房占,謂臣道太盛,坤維不寧,用以諷居正。居正婦弟辱江陵令,世貞論奏不少貸。居正積不能堪,會遷南京大理卿,為給事中楊節所劾,即取旨罷之。後起應天府尹,復被劾罷。居正歿,起南京刑部右侍郎,辭疾不赴。久之,所善王錫爵秉政,起南京兵部右侍郎。先是,世貞為副都御史及大理卿、應天尹與侍郎,品皆正三。世貞通理前俸,得考滿陰子。比擢南京刑部尚書,御史黃仁榮言世貞先被劾,不當計俸,據故事力爭。世貞乃三疏移疾歸。二十一年卒於家。

世貞始與李攀龍狎主文盟,攀龍歿,獨操柄二十年。才最高,地望最顯,聲華意氣籠蓋海內。一時士大夫及山人、詞客、衲子、羽流,莫不奔走門下。片言褒賞,聲價驟起。其持論,文必西漢,詩必盛唐,大曆以後書勿讀,而藻飾太甚。晚年,攻者漸起,世貞顧漸造平淡。病亟時,劉鳳往視,見其手蘇子瞻集,諷玩不置也。

世貞自號鳳洲,又號弇州山人。其所與遊者,大抵見其集中,各為標目。曰前五子者,攀龍、中行、有譽、國倫、臣也。後五子則南昌餘曰德、蒲圻魏裳、歙汪道昆、銅梁張佳胤、新蔡張九一也。廣五子則崑山俞允文、浚盧柟、濮州李先芳、孝豐吳維岳、順德歐大任也。續五子則陽曲王道行、東明石星、從化黎民表、南昌朱多火煃、常熟趙用賢也。末五子則京山李維楨、鄞屠隆、南樂魏允中、蘭谿胡應麟,而用賢復與焉。其所去取,頗以好惡為高下。

餘曰德,字德甫,張佳胤,字肖甫,張九一,字助甫,世貞詩所謂「吾黨有三甫」也。魏裳,字順甫,與曰德俱嘉靖二十九年進士。曰德終福建副使,裳終濟南知府。九一,嘉靖三十二年進士,終巡撫寧夏僉都御史。佳胤自有傳。

汪道昆,字伯玉,世貞同年進士。大學士張居正亦其同年生也,父七十壽,道昆文當其意,居正亟稱之。世貞筆之《藝苑卮》曰:「文繁而有法者於鱗,簡而有法者伯玉。」道昆由是名大起。晚年官兵部左侍郎,世貞亦嘗貳兵部,天下稱「兩司馬」。世貞頗不樂,嘗自悔獎道昆為違心之論云。

胡應麟,幼能詩。萬歷四年舉於鄉,久不第,築室山中,構書四萬餘卷,手自編次,多所撰著。攜詩謁世貞,世貞喜而激賞之,歸益自負。所著《詩藪》二十卷,大抵奉世貞《卮言》為律令,而敷衍其說,謂詩家之有世貞,集大成之尼父也。其貢諛如此。

世貞弟世懋,字敬美。嘉靖三十八年成進士,即遭父憂。父雪,始選南京禮部主事。歷陜西、福建提學副使,再遷太常少卿,先世貞三年卒。好學,善詩文,名亞其兄。世貞力推引之,以為勝己,攀龍、道昆輩因稱為「少美」。

世貞子士騏,字冏伯,舉鄉試第一,登萬曆十七年進士,終吏部員外郎,亦能文。

歸有光,字熙甫,崑山人。九歲能屬文,弱冠盡通《五經》、《三史》諸書,師事同邑魏校。嘉靖十九年舉鄉試,八上春官不第。徙居嘉定安亭江上,讀書談道。學徒常數百人,稱為震川先生。四十四年始成進士,授長興知縣。用古教化為治。每聽訟,引婦女兒童案前,刺刺作吳語,斷訖遣去,不具獄。大吏令不便,輒寢閣不行。有所擊斷,直行己意。大吏多惡之,調順德通判,專轄馬政。明世,進士為令無遷卒者,名為遷,實重抑之也。隆慶四年,大學士高拱、趙貞吉雅知有光,引為南京太僕丞,留掌內閣制敕房,修《世宗實錄》,卒官。

有光為古文,原本經術,好《太史公書》,得其神理。時王世貞主盟文壇,有光力相觸排,目為妄庸巨子。世貞大憾,其後亦心折有光,為之贊曰:「千載有公,繼韓、歐陽。餘豈異趨,久而自傷。」其推重如此。

有光少子子慕,字季思。舉萬曆十九年鄉試,再被放,即屏居江村,與無錫高攀龍最善。其歿也,巡按御史祁彪佳請於朝,贈翰林待詔。

有光制舉義,湛深經術,卓然成大家。後德清胡友信與齊名,世並稱歸、胡。

友信,字成之,隆慶二年進士。授順德知縣。歲賦率奸胥攬輸,稍以所入啖長吏,謂之月錢。友信與民約,歲為三限,多寡皆自輸,不取贏,閭里無妄費,而公賦以充。海寇竊發,官軍往討,民間驛騷。部內烏洲、大洲,賊所巢穴,諸惡少為賊耳目。友信悉勾得之,捕誅其魁,餘黨解散。鄉立四應社,一鄉有警,三鄉鼓而援之,不援者罪同賊,賊不敢發。歲大兇,民饑死無敢為惡者。

初,友信慮民輕法,涖以嚴,後令行禁止,更為寬大,或旬日不笞一人。其治縣如家,弊修墮舉,學校城池,咸為更新。督課邑子弟,教化興起。卒官,士民立祠奉祀。

友信博通經史,學有根柢。明代舉子業最擅名者,前則王鏊、唐順之,後則震川、思泉。思泉,友信別號也。


\end{pinyinscope}