\article{列傳第二百一 雲南土司}

\begin{pinyinscope}
明洪武十四年,大軍至滇,梁王走死,遂置雲南府。自是,諸郡以次來歸,垂及累世,規制咸定。統而稽之,大理、臨安以下,元江、永昌以上,皆府治也。孟艮、孟定等處則為司,新化、北勝等處則為州,或設流官,或仍土職。今以諸府州概列之土司者,從其始也。蓋滇省所屬,多蠻夷雜處,即正印為流官,亦必以土司佐之。而土司名目淆雜,難以縷析,故係之府州,以括其所轄。而於土司事迹,止摭其大綱有關乎治亂興亡者載之,俾控馭者識所鑒焉。

○雲南土司一

雲南大理臨安楚雄澄江景東廣南廣西鎮沅永寧順寧蒙化孟艮孟定耿馬安撫司附曲靖

雲南,滇國也。漢武帝時始置益州郡。蜀漢置雲南郡。隋置昆州,唐仍之。後為南詔蒙氏所據,改鄯闡府。歷鄭、趙、楊三氏,至大理段氏,以高智昇領鄯闡牧,遂世其地。元初,置鄯闡萬戶府。既改置中慶路,封子忽哥為雲南王鎮之,仍錄段氏子孫守其土。忽哥死,其子嗣封為梁王。

洪武六年,遣翰林待制王禕等齎詔諭梁王,久留不遣,卒遇害。八年復遣湖廣行省參政吳雲往,中途為梁使所害。十四年,征南將軍傅友德、藍玉、沐英率師至雲南城,梁王赴滇池死,定其地。改中慶路為雲南府,置都指揮使司,命都督僉事馮誠署司事。二月詔諭雲南諸郡蠻。十五年,友德等分兵攻諸蠻寨之未服者,土官楊苴乘隙作亂,集蠻眾二十餘萬攻雲南城。時城中食少,士卒多病,寇至,都督謝熊、馮誠等攖城固守,賊不能攻,遂遠營為久困計。時沐英方駐師烏撒,聞之,將驍騎還救。至曲靖,遣卒潛入報城中,為賊所得,紿之曰:「總兵官領三十萬眾至矣。」賊眾驚愕,拔營宵遁,走安寧、羅次、邵甸、富民、普寧、大理、江川等處,復據險樹柵,謀再寇。英分調將士剿降之,斬首六萬餘級,生擒四千餘人,諸部悉定。二十五年,英卒,命其子春襲封西平侯,仍鎮雲南。

自英平雲南,在鎮十年,恩威著於蠻徼;每下片楮,諸番部具威儀出郭叩迎,盥而後啟,曰:「此令旨也。」沐氏亦皆能以功名世其家。每大征伐,輒以征南將軍印授之,沐氏未嘗不在行間。數傳而西平裔孫當襲侯,守臣爭之,謂滇人知有黔國公,不知西平侯也。孝宗以為然,許之。自是,遂以公爵佩印,為故事。諸土司之進止予奪,皆咨稟。及承平久,文網周密,凡事必與太監撫、按、三司會議後行,動多掣肘,土官子孫承襲有積至二三十年不得職者。土官復慢令玩法,無所忌憚;待其罪大惡極,然後興兵征剿,致軍民日困,地方日壞。大學士楊一清等因武定安銓之亂,痛切陳之。黔國公沐紹勛亦以為言。雖得旨允行,亦不能更革。馴至神宗之世,朝廷惰媮,封疆敗壞日甚一日。緬、莽之叛,皆土官之失職者導之。雖稍奏膚功,而滇南喪敗,卒由土官沙定洲之禍。

沙定洲者,王弄山長官司沙源之子也。源驍勇有將材,萬曆中,數從徵調有功,巡撫委以王弄副長官事。繼以征建水功,以安南長官司廢地畀之。後征東川、水西、馬龍山等處,全雲南會城,稱首功,累加至宣撫使,時號沙兵。定洲,其仲子也。

崇禎中,元謀土知州吾必奎叛。總兵官沐天波剿之,調定洲從征。定洲不欲行,出怨言。會奸徒饒希之、余錫朋者逋天波金,無以償。錫朋常出入土司家,誇黔府富盛。定洲心動,陰結都司阮韻嘉諸人為內應。既定洲入城辭行,天波以家諱日不視事,定洲噪而入,焚劫其府。天波聞變,由小竇遁。時寧州土司祿永命在城,方巷戰拒賊,從官周鼎止天波,留討賊。天波疑鼎為定洲誘己,殺之,其母妻皆走城北自焚死。定洲據黔府,盤踞會城。劫巡撫吳兆元,使題請代天波鎮滇,傳檄州縣,全滇震動。祿永命與石屏州龍在田俱引所部去。

天波走楚雄,金滄副使楊畏知奉調駐城中,謂天波曰:「公何不走永昌,使楚得為備,而公在彼掎角,首尾牽制之,上策也。」天波從之。定洲至楚雄,城閉不得入,乃去。遣其黨王翔、李日芳等,攻陷大理、蒙化。畏知乘間檄城外居民盡入城,築陴浚隍,調土、漢兵守之。定洲聞祿永命等各固守,不敢至永昌,恐畏知截其歸路,急還兵攻楚雄。畏知坐城樓,賊發巨砲擊之,煙焰籠城櫓,眾謂畏知已死,而畏知端坐自如,賊相驚謂神。畏知伺賊間,輒出奇兵殺賊甚眾。賊引去,攻石屏不下,還攻寧州,祿永命戰死。賊計迤東稍稍定,乃復攻楚雄。分兵為七十二營,環城掘濠,為久困計。

會張獻忠死,其部將孫可望率餘眾由遵義入黔,稱黔國焦夫人弟來復仇。民久困沙兵,喜其來,迎之。定洲解楚雄圍,迎戰於草泥關,大敗,遁阿迷。可望破曲靖及交水,俱屠之。遂由陸涼、宜良入雲南城,分遣李定國徇迤東諸府。而可望自率兵西出,畏知禦於啟明橋,兵敗,被執。可望聞其名,不殺,語之曰:「吾與爾共討賊,何如?」畏知要以三事:「不用獻忠偽號,不殺百姓,不擄婦女,吾從爾。」可望皆許之。即折箭相誓,乃以書諭天波如畏知言,天波亦來歸。而李定國之徇臨安者,定洲部目李阿楚拒戰甚力。定國穴地置炮,砲發城陷,遂入。驅城中官民於城外白場殺之,凡七萬八千餘人,斬獲不與焉。當時皆意定國破臨安,必襲阿迷,取定洲,乃僅掠臨安子女而回,所過無不屠滅。迤西以畏知在軍,得保全。

始定洲歸,屯兵洱革龍,且借安南援自固。會可望與定國不協,聲其罪,杖之百,責以取定洲自贖。定國既至,定洲土目楊嘉方迎定洲就其營宴。定國偵知之,率兵圍營,相拒數日,乃出降。遂械定洲及妻萬氏數百人回雲南,剝其皮市中。可望遂據滇,而天波卒走死於緬甸。

大理,唐葉榆縣境也。麟德初,置姚州都督府。開元末,蒙詔皮羅閣建都於此,為南詔,治太和城。至閣羅鳳,號大蒙國,異牟尋改大禮國。其後,鄭買賜、趙善政、楊乾貞互篡奪,至五代晉時,段思平得之,更號大理國。元憲宗取雲南,至大理,段智興降附,乃設都元帥,封智興為摩訶羅嵯,管領八方。又以劉時中為宣撫使,同智興安輯其民。段氏有大理,傳十世至寶。聞太祖開基江南,遣其叔段真由會川奉表歸款。洪武十四年,征南將軍傅友德克雲南,授段明為宣慰使。明遣都使張元亨貽征南將軍書曰:「大理乃唐交綏之外國,鄯闡實宋斧畫之餘邦,難列營屯,徒勞兵甲。請依唐、宋故事,寬我蒙、段,奉正朔,佩華篆,比年一小貢,三年一大貢。」友德怒,辱其使。明再貽書曰:「漢武習戰,僅置益州。元祖親征,祗緣鄯闡。乞賜班師。」友德答書曰:「大明龍飛淮甸,混一區宇。陋漢、唐之小智,卑宋、元之淺圖。大兵所至,神龍助陣,天地應符。汝段氏接武蒙氏,運已絕於元代,寬延至今。我師已殲梁王,報汝世仇,不降何待?」

十五年,征南左將軍藍玉、右將軍沐英率師攻大理。大理城倚點蒼山,西臨洱河為固。聞王師至,聚眾扼下關。下關者,南詔皮羅閣所築龍尾關也,號極險。玉等至品甸,遣定遠侯王弼以兵由洱水東趨上關,為掎角勢,自率眾抵下關,造攻具。遣都督胡海洋由石門間道夜渡河,繞出點蒼山後,攀木援崖而上,立旗幟。昧爽,軍抵下關者望見,皆踴躍言雚噪,蠻眾驚亂。英身先士卒,策馬渡河,水沒馬腹,將士隨之,遂斬關入。蠻兵潰,拔其城,酋長段世就擒。世與明皆段寶子也。至京師,帝傳諭曰:「爾父寶曾有降表,朕不忍廢。」賜長子名歸仁,授永昌衛鎮撫;次子名歸義,授雁門鎮撫。大理悉定,因改大理路為大理府,置衛,設指揮使司。

十六年,品甸土酋杜惠來朝,命為千夫長。命六安侯王志、安慶侯仇成、鳳翔侯張龍督兵往雲南品甸,繕城池,立屯堡,置郵傳,安輯人民。十七年以土官阿這為鄧川知州,阿散為太和府正千夫長,李朱為副千夫長,楊奴為雲南縣丞。十九年置雲南洱海衛指揮使司,以賴鎮為指揮僉事。洱海,本品甸也。兵燹後,人民流亡,室廬無復存者。鎮至,復城池,建譙樓,治廬舍市里,修屯堡、隄防、斥堠,又開白鹽井,民始安輯。二十年詔景川侯曹震及四川都司選精兵二萬五千人,給軍器農具,即雲南品甸屯種,以俟征討。永樂以後,雲南諸土官州縣,率按期入貢,進馬及方物,朝廷賜予如制。嘉靖元年改十二關長官司於一泡江之西,從巡撫何孟春奏也。

臨安,古句町國。漢置縣。唐為羈縻牁州地。天寶末,南詔蒙氏於此置通海郡。元時內附,置阿僰部萬戶府。至元中改臨安路,屬臨安、廣西、元江等處宣慰司。洪武十四年,征南將軍下雲南,遣宣德侯金朝興分道取臨安。元右丞兀卜台、元帥完者都及土官楊政降,改路為府,廢宣慰司,置臨安衛指揮使司。十七年以土官和寧為阿迷知州,弄甥為寧州知州,陸羨為蒙自知縣,普少為納婁茶甸副長官;俱來朝貢,因給誥敕冠帶以命之。十八年,臨安府千戶納速丁等來朝,人賜米十石。

永樂九年,溪處甸長官司副長官自恩來朝,貢馬及金銀器,賜賚如例。自恩因言:「本司歲納海七萬九千八百索,非土所產,乞準鈔銀為便。」戶部以洪武中定額,難準折輸。帝曰:「取有於無,適以厲民,況彼遠夷,尤當寬恤,其除之。」

宣德五年,中官雲仙還自雲南,奏設東山口巡檢司,以故土官後普覺為巡檢。八年,虧容甸長官司奏:「河底自洪武中官置渡船,路通車里、八百。近年軍民有逃逸出境詐稱使者,迫令乘載,往往被害,又沿河時有劫盜出沒。乞置巡檢司,以故把事袁凱之子瑀為巡檢。」從之。嘉靖元年復設寧州流官知州,掌州事,土知州祿氏專職巡捕。寧州舊設流官,正德初,土官祿俸陰賄劉瑾罷之。遂交通彌勒州十八寨強賊為亂,為官軍捕誅,其子祿世爵復以罪諭死。撫按請仍設流官,從之。初,臨安阿迷州土官普柱,洪武中為土知州。後設流,錄其後覺為東山巡檢,既而以他事廢。正德二年以廣西維摩、王弄山與阿迷接壤,盜出沒,仍令普覺後納繼前職。

普維籓者,與寧州祿氏構兵,師殲焉。維籓子名聲,幼育於官,既長,有司俾繼父職。名聲收拾舊部,勇於攻戰,從討奢安有功,仍授土知州,漸驕恣。崇禎五年,御史趙洪範按部,名聲不出迎。已,出戈甲旗幟列數里。洪範大怒,謀之巡撫王伉,請討,得旨。官軍進圍州城,名聲恐,使人約降,而陰以重賄求援於元謀土官吾必奎。時官軍已調必奎隨征,必奎與名聲戰,兵始合,佯敗走。官軍望見,遂大潰,布政使周士昌戰死。朝廷以起釁罪伉,逮治,而名聲就撫。然驕恣益甚,當事者頗以為患。已而廣西知府張繼孟道出阿迷,以計毒殺之。必奎聞名聲死,遂反,連陷武定、祿豐、楚雄諸城。寧州土官祿永命、石屏州土目龍在田,俱與必奎、名聲從征著名,至是,黔國公沐天波檄之統兵,合剿擒必奎。名聲妻萬氏,本江西寄籍女,淫而狡。名聲死後,改嫁王弄山副長官沙源之子定洲。名聲有子曰服遠,與萬氏分寨居,定洲誘殺服遠,併其地。天波檄定洲取必奎,定洲不欲行,遂反,詳前傳。

臨安領州四,縣四。其長官司有九,曰納樓茶甸,曰教化三部,曰溪處甸,曰左能寨,曰王弄山,曰虧容甸,曰思陀甸,曰落恐甸,曰安南,其地皆在郡東南。西平侯征安南,取道於此。蓮花灘之外即交荒外,而臨安無南面之虞者,以諸甸為之備也。但地多瘴,流官不欲入,諸長官亦不請代襲,自相冠帶,日尋干戈。納樓部內有礦場三,曰中場、鵝黃、摩訶。封閉已久,亡命多竊取之。其安南長官司,本阿僰蠻所居,舊名褒古,後名捨資。元為捨資千戶所。以地近交址,改安南,屬臨安路。正德八年,蒙自土舍祿祥爭襲父職,鴆殺其嫡兄祿仁,安南長官司土舍那代助之以兵,遂稱亂,守臣討平之。事聞,命革蒙自土官,改長官司為新安守禦千戶所,調臨安衛中所官軍戍之。

楚雄,昔為威楚。元憲宗置威楚萬戶府。至元後,置威楚開南路宣撫司。洪武十五年,南雄侯趙庸取其地。十七年以土官高政為楚雄府同知,阿魯為定邊縣丞。永樂元年,楚雄府言:「所屬蠻民,不知禮義。惟僰種賦性溫良,有讀書識字者。府州已嘗設學教養,其縣學未設。縣所轄六里,僰人過半,請立學置官訓誨。」從之。

宣德五年命故土知府高政女襲同知。政初為同知,永樂中來朝,時仁宗監國,嘉其勤誠,升知府,子孫仍襲同知。政卒,無子,妻襲。又卒,其女奏乞襲知府。帝曰:「皇考有成命。」令襲同知。

八年陞南安州琅井土巡檢李保為州判官;以鄉老言:「本州俱羅舞、和泥、烏蠻雜類,稟性頑獷,以無土官管束,多致流移,差役賦稅,俱難理辦。眾嘗推保署州事,撫綏得宜,民皆向服,流移復歸,乞授本州土官。」吏部言:「南安舊無土官,難從其請。」帝以為治在順民情,從之。九年,黔國公沐晟等奏:「楚雄所屬黑石江及泥坎村銀場,軍民盜礦,千百為群,執兵攘奪。楚雄縣賊首者些糾合武定賊者惟等,劫掠軍民,殺巡檢張禎。又定邊縣阿苴里諸處強賊,聚眾抄掠景東等衛。大理、蒙化、楚雄、姚州皆有盜出沒。」帝敕責晟等,期以三年,討靖諸為亂者。

嘉靖四十三年,楚雄叛蠻阿方等兵起,先攻易門所,流劫嶍峨、昆陽、新化各州縣,僭稱王,約土官王一心、王行道為援。一心後悔,詣軍門請討賊自效。巡撫呂光洵許之,招降數百人。官軍分道進,擒獲賊黨。乘勝攻大、小木址二寨,克之,斬阿方首,餘賊悉平。

澄江,唐為南寧、昆二州地。天寶末,沒於蠻,號羅伽甸。宋時,大理段氏號羅伽部。元置羅伽萬戶府。至元中,改澄江路。洪武十五年,雲南平,澄江歸附,改澄江府。地居滇省之中,山川明秀,蠶衣耕食,民安於業。近郡之羅羅,性雖頑狠,然恭敬上官。官至,爭迎到家,刲羊擊豕,罄所有以供之,婦女皆出羅拜,故於諸府獨號安靜云。

景東,古柘南也,漢尚未有其地。唐南詔蒙氏始置銀生府,後為金齒白蠻所據。元中統三年討平之,以所部隸威楚萬戶。至元中,置開南州。洪武十五年平雲南,景東先歸附。土官俄陶獻馬百六十匹、銀三千一百兩、馴象二。詔置景東府,以俄陶知府事,賜以文綺襲衣。十八年,百夷思倫發叛,率眾十餘萬攻景東之北吉寨。俄陶率眾禦之,為所敗,率其民千餘家避於大理府之白崖川。事聞,帝嘉其忠,遣通政司經歷楊大用齎白金文綺賜之。二十三年,沐英討平思倫發,復景東地,因奏景東百夷要衝,宜置衛。以錦衣衛僉事胡常守之,俄陶仍舊職。二十四年,帝以景東為雲南要害,且多腴田,調白崖川軍士屯守。二十六年命洱海衛指揮同知賴鎮守景東,從沐春請也。

宣德五年置孟緬長官司。時景東奏所轄孟緬、孟梳,地方遐遠,屢被外寇侵擾。乞並孟梳於孟緬,設長官司,授把事姜嵩為長官,以隸景東,歲增貢銀五十兩。六年,大侯土知州刀奉漢侵據孟緬地,敕黔國公沐晟遣官撫諭。

正統中,思任發叛,官軍征麓川,知府陶瓚從征有功,進階大中大夫。弘治十五年正月,景東衛雲霧黑暗,晝夜不別者凡七日,巡撫陳金以聞。命廷臣議考察,以謝天變。南京刑部、都察院承旨,考黜文武官千二百員。嘉靖中,者東甸稱亂,劫景東府印去。土舍陶金追斬其頭目,奪印歸。

景東部皆僰種,性淳樸,習弩射,以象戰。歷討鐵索、米魯、那鑒、安銓、鳳繼祖諸役,皆調其兵及戰象。天啟六年,貴州水西安邦彥反,率眾二十萬入滇境,至馬龍後山,去會城十五里。總兵官調景東土舍陶明卿率兵伏路左。賊分道並至,官兵禦之,賊拒戰,勢甚銳。明卿乃以象陣從左翼衝出橫擊,賊潰,追奔十餘里。巡撫上功,推明卿第一。景東每調兵二千,必自效千餘,餉士之費,未嘗仰給公家,土司中最稱恭順。其府治東有邦泰山,頗險峻,土官陶姓所世居也。

廣南,宋時名特磨道。土酋儂姓,智高之裔也。元至元間,立廣南西路宣撫司。初領路城等五州,後惟領安寧、富二州。洪武十五年歸附,改廣南府,以土官儂郎金為同知。十八年,郎金來朝,賜錦綺鈔錠。二十八年,都指揮同知王俊奉命率雲南後衛官軍至廣南,築城建衛。郎金父貞佑不自安,結眾據山寨拒守。俊遣人招之,不服,時伏草莽中劫掠,覘官軍進退。俊乃遣指揮歐慶等分兵攻各寨,自將取貞佑;又以兵扼間道,絕其救援。諸寨悉破,眾潰,貞佑窮促就擒,械送京師。降郎金為府通判。

永樂六年,富州土知州沈絃經入貢,值仁孝皇后喪,絃經奉香幣致祭。宣德元年,土官儂郎舉來朝,貢馬。正統六年,廣南賊阿羅、阿思等劫掠,命總兵官沐昂等招撫之。時富州土官沈政與郎舉互訐糾眾侵地,帝命昂等勘處。七年,昂奏二人叛逆無實迹,因有隙相妄奏。兵部請治政等罪,帝以蠻人宥之。政、舉相仇殺已十餘年,時方征麓川,憚兵威不敢動。未幾,郎舉以從征功陞同知,死無嗣,四門舍目共推儂文舉署事,屢立戰功。萬曆七年,實授同知。子應祖從征三鄉,親獲賊首,詔賞銀百兩。播州之役,徵其兵三千討尋甸叛目,皆有功,賜四品服。

儂氏自文舉藉四門舍目推擁之力得授職,後儂氏襲替必因之。土官之政出於四門,租稅僅取十之一。道險多瘴,知府不至其地,印以臨安指揮一人署之。指揮出,印封一室,入取,必有瘟癘死亡。萬曆末,知府廖鉉者,避瘴臨安,以印付同知儂仕英子添壽。添壽死,家奴竊印並經歷司印以逃,既而歸印於其族叔儂仕祥。時仕英親弟仕獬例得襲,索仕祥印,仕祥不與,遂獻地與泗城土官岑接,與連婚構兵,滅仕獬家。及仕祥死,子琳以府印送接,而經歷司印又為琳弟瓊所有。巡撫王懋中調兵往問,瓊懼,還印於通判周憲,接亦出府印獻於官。時兵方調至境,遽遣歸。廷議治鉉擅離與守巡失撫之罪,瓊、接已輸服,勿問,詔可。未幾,儂紹湯兄弟爭襲,各糾交阯丘象,焚掠一空。

廣西,隋屬牂州,後為東僰、烏蠻等部所居。唐隸黔州都督府。後師宗、彌勒二部浸盛,蒙、段皆莫能制。元憲宗時始內屬。至元十二年籍二部為軍,置廣西路。洪武十四年歸附,以土官普德署府事。二十年,普德及彌勒知州赤善、師宗知州阿的各遣人貢馬,詔賜文綺鈔錠。二十四年,布政使張紞奏:「維摩、雲龍、永寧、浪渠、越順等州縣蠻民頑惡,不遵政教,宜置兵戍守,以控制之。」是後,朝貢賜予如制。

正統六年,總兵官沐昂奏師宗州及廣南府賊阿羅、阿思糾合為亂,命昂等招諭,未幾平。成化中,土知府昂貴有罪,革其職,安置彌勒州,乃置流官,始築土城。嘉靖元年設雲南彌勒州十八寨守禦千戶所。其部眾好擄掠,無紀律,至水西、烏撒用兵,始徵調之。崇禎間,巡按御史傅宗龍由滇入黔,招普兵以行。時滇中最勍稱沙普兵,亦曰昂兵。

鎮沅,古濮、洛雜蠻所居,《元史》謂是和泥、昔樸二蠻也。唐南詔蒙氏銀生府地。其後,金齒僰蠻據之。元時為威遠蠻棚府,屬元江路總管。洪武十五年,總管刀平與兄那直歸附,授千夫長。建文四年置鎮沅州,以刀平為知州。永樂三年,刀平率其子來朝,貢方物,賜鈔文綺。從征八百,又從攻石崖、者達寨外部。整線來降,入貢方物。升為府,以刀平為知府,置經歷、知事各一員。貢賜皆如例。成化十七年,以地方未平,免鎮沅諸土官朝覲。正統元年復免。

嘉靖中徵安銓,調鎮沅兵千人,命刀寧息領之。復調其子刀仁,亦率兵千人,徵那鑑,克魚復寨。初,鎮沅印為那氏所奪,至是得印以獻,命給之。領長官司一,曰祿谷寨,永樂十年置。

永寧,昔樓頭夾地,接吐蕃,又名答藍。唐屬南詔,後為麼些蠻所據。元憲宗時內附,至元間,置答藍管民官,尋改永寧州,隸北勝府。洪武平雲南時,屬鶴慶府。二十九年,改屬瀾滄衛。十二月,土賊卜百如加劫殺軍民,前軍都督僉事何福遣指揮李榮等討之。其子阿沙遁入革失瓦都寨,官軍齎三日糧,深入追之,會天大雨,眾饑疲,引還。

永樂四年設四長官司,隸永寧土官,以土酋張首等為長官,各給印章,賜冠帶彩幣。尋升永寧為府,隸布政司,升土知州各吉八合知府,遣之齎敕往大西番撫諭蠻眾。宣德四年,永寧蠻寨矢不剌非糾四川鹽井衛土官馬剌非殺各吉八合,官軍撫定之。命卜撒襲知府,復為矢不剌非所殺。已,命卜撒之弟南八襲,馬剌非又據永寧節卜、上、下三村,逐南八,大掠夜白、尖住、促卜瓦諸寨。事聞,帝命都督同知沐昂勒兵諭以禍福,並移檄四川行都司下鹽井衛諭馬剌非還所據村寨。正統二年,馬剌非為南八所攻,拔烏節等寨,南八亦言馬剌非殺害。詔鎮巡官驗問,令各歸侵地,乃寢。

永寧界,東至四川鹽井衛十五里,西至麗江寶山州,南至浪渠州,北至西番。領長官司四,曰剌次和,曰瓦魯之,曰革甸,曰香羅。

順寧府,本蒲蠻地,名慶甸。宋以前不通中國,雖蒙氏、段氏不能制。元泰定間始內附。天歷初,置順寧府並慶甸縣,後省入府。洪武十五年,順寧歸附,以土酋阿悅貢署府事。十七年命阿日貢為順寧知府。二十三年,土酋猛丘、土知府子丘等,不輸征賦,自相仇殺。大理衛指揮鄭祥征蒙化賊,移師至甸頭,破其寨。猛丘請降輸賦,乃還。猛丘死,把事阿羅等復起兵相攻擊。二十九年,西平侯沐春遣鄭祥與指揮李榮等,分道進討,擒阿羅等誅之。後貢賜如制。

順寧與大侯接境。萬歷中,大侯土舍奉赦、奉學兄弟不相能。奉學倚妻父土知府猛廷瑞,與兄赦日構兵。巡撫陳用賓檄參將李先著、副使邵以仁勘處。以仁襲執廷瑞,因請改順寧為流官。先著被檄,極言不可討,被謗語,逮下獄庾死。然廷瑞實無反謀,以參將吳顯忠覘其富,誣以助惡,索金不應,遂讒於巡按張應揚,轉告巡撫陳用賓。廷瑞大恐,不得已斬奉學以獻。顯忠益誣其陰事,傅以反狀,撫按會奏,得旨大剿。廷瑞出,獻印獻子以候命,不從。顯忠帥兵入其寨,盡取猛氏十八代蓄貲數百萬,誘廷瑞至會城執之,獻捷於朝。於是所部十三寨盡憤,始聚兵反,官兵悉剿除之,并殺其子。以仁超擢右都御史,蔭子。未幾坐大辟,繫獄,應揚亦病卒。人以為天道云。

順寧附境有猛猛、猛撒、猛緬,所謂三猛也。猛猛最強,部落萬人,時與二猛為難。其地田少箐多,射獵為業。猛緬地雖廣,而人柔弱。部長賜冠帶,最忠順。猛撒微弱,後折入於耿馬云。

蒙化,唐屬姚州都督府。蒙氏時,細奴邏築城居之,號蒙舍詔。段氏改開南縣。元為州,屬大理。洪武十七年以土酋左禾為蒙化州判官、施生為正千夫長。二十三年,西平侯沐英以蒙化所屬蠻火頭字青等梗化不服,請置衛。命指揮僉事李聚守蒙化。賊高天惠作亂,大理衛指揮使鄭祥捕斬之,傳首雲南。

永樂九年,土知州左禾、正千夫長阿束來朝,貢馬,賜予如例。既,左伽從征麓川,戰於大侯,功第一,進秩臨安知府,掌州事。正統中,陞州為府,以左伽為知府,世襲。所部江內諸蠻,性柔,頗馴擾,江外數枝,以勇悍稱。每應徵調,多野戰,無行伍。

成化十七年,巡撫奏地方未寧,免蒙化土官明年朝貢。正統元年詔復免。萬曆四十八年,雲龍土知州段龍死,子嘉龍立,養子進忠殺嘉龍爭襲,流劫殺掠。官軍進討,進忠從間道欲趨大理,官軍擒誅之,改設流官,授段氏世吏目一人。

孟艮,蠻名孟掯,自古不通中國。永樂三年來歸,設孟艮府,隸雲南都司,以土酋刀哀為知府,給印誥冠帶。時刀哀遣人來朝,請設治所,歲辦差發黃金六十兩。六年,土知府刀交遣弟刀哈哄貢象及金銀器。禮部言:「刀交嘗構兵攻劫鄰境,詐譎不誠,宜卻其貢。」帝曰:「蠻夷能悔過來朝,往事不足責。」命賜鈔及絨錦綺帛。是後,貢賜皆如例。宣德六年,命內官楊琳齎彩幣往賜孟艮知府刀光。正統間,孟艮地多為木邦所並。景泰中,入貢知府名慶馬辣,不知於刀氏何屬也。

孟艮在姚關東南二千里外,沃野千里,最殷富。地多虎,農者於樹杪結草樓以護稼。雲南知府趙混一嘗入其境,待之禮慢,後無復至者。

孟定,蠻名景麻。至元中,立孟定路軍民總管府,領二甸,隸大理、金齒等處宣慰司。洪武十五年,土酋刀名扛來朝,貢方物,賜綺帛鈔幣,設孟定府,以刀渾立為知府。永樂二年,孟定土官刀景發遣人貢馬,賜鈔羅綺。遣使往賜印誥、冠帶、襲衣,復頒信符、金字紅牌。四年,帝以孟定道里險遠,每歲朝貢不便,令自今三年一貢,如慶賀謝恩不拘例。

初,孟璉與孟定皆麓川地,其土目皆故等夷,惡相屬;後改孟璉隸雲南,多以互侵土地仇殺。宣德六年,土知府罕顏法以為言,敕黔國公沐晟遣官撫諭,俾各歸侵掠。正統中,麓川叛,孟定知府刀祿孟遁走。木邦土官罕葛從征有功,總督王驥奏令食孟定之土。嘉靖間,木邦罕烈據地奪印,令土舍罕慶守之,名為耿馬;地之所入,悉歸木邦。萬曆十二年,官兵取隴川,平孟定故地,以罕葛之後為知府。十五年頒孟定府印。崇禎末,孟定叛,降於緬甸。其地,自姚關南八日程,西接隴川,東連孟璉,南木邦,北鎮康。土瘠人稀,有馬援城在焉。領安撫司一,曰耿馬。萬曆十二年置,以們罕為安撫使。與孟定隔喳哩江。孟定居南,耿馬居北。罕死,弟們罕金護印,屢奉朝貢。時木邦思禮作亂,侵灣甸、鎮康,倚罕金為聲援。天啟二年,緬人攻猛乃、孟艮,罕金欲救之。緬移兵攻金,金厚賂之,乃解。後與木邦罕正構難不絕云。

曲靖,隋恭、協二州地。唐置南寧州,改恭州為曲州,分協州置靖州,至元初,置磨彌部萬戶,後改為曲靖路宣慰司。

洪武十四年,征南將軍下雲南,元曲靖宣慰司征行元帥張麟、行省平章劉輝等來降。十五年改曲靖千戶所為曲靖軍民指揮使司,置曲靖軍民府。十六年,沾益州土官安索叔、安磁等貢馬及羅羅刀甲、氈衫、虎皮。詔賜磁、冠帶、綺羅衣各一襲並文綺、鈔錠。羅雄州土酋納居來朝,賜鈔幣。十七年,亦佐縣土酋安伯作亂,西平侯沐英發兵討降之。

二十年,越州土酋阿資與羅雄州營長發束等叛。阿資者,土官龍海子也。越州,蠻呼為苦麻部。元末,龍海居之,所屬俱羅羅斯種。王師征南時,英駐兵其地之湯池山。龍海降,遂遣子入朝,詔以龍海為知州。尋為亂,英擒之,徙遼東,至蓋州病死。阿資繼其職,益桀驁,至是叛。帝命英會征南將軍傅友德進討。道過平夷,以其山險惡,宜駐兵屯守,遂遷其山民往居卑午村,留神策衛千戶劉成等將千人置堡其地,後以為平夷千戶所。阿資等率眾寇普安,燒府治,大肆剽掠。友德率兵擊之,斬其營長。二十二年,友德等進攻,土官普旦來降。阿資退屯普安,倚崖壁為寨。友德以精兵蹙之,蠻眾皆緣壁攀崖,墜死者不可勝數,生擒一千三百餘人,獲馬畜甚眾。阿資遁還越州,復追擊敗之,斬其黨五十餘人。阿資窮蹙請降。初,阿資之遁也,揚言曰:「國家有萬軍之勇,我地有萬山之險,豈能盡滅我輩。」英乃請置越州、馬龍二衛,扼其險要,復分兵追捕,至是遂降。

英等以陸涼西南要地,請設衛屯守。命洱海衛指揮僉事滕聚於古魯昌築城,置陸涼衛指揮使司。英又言:「曲靖指揮千戶哈刺不花,乃故元守禦陸涼千戶。今陸涼置衛,宜調於本衛鎮守,庶絕後患。」詔從之。帝以平夷尤當要衝,四面皆諸蠻部落,乃遣開國公常升往辰陽集民間丁壯五千人,統以右軍都督僉事王成,即平夷千戶所改置衛。二十三年置越州衛。二十四年徙越州衛於陸涼州;以英言雲南諸蠻皆降,惟阿資恃險屢叛,宜徙衛軍守禦。已,阿資復叛。命都督僉事何福為平羌將軍,率師進討,屢敗賊眾。會連月淫雨水溢,阿資援絕,與其眾降。福擇曠地列柵,以置其眾。西南有木蓉菁,賊常出沒處,復調普安衛官軍置寧越堡鎮之,然阿資終不悛。

二十七年,阿資復反。西平侯沐春及福率兵營於越州城北,遣壯士伏於岐路,而以兵挑戰。蠻兵悉眾出,伏起,大敗之,阿資脫身遁。初,曲靖土軍千戶阿保、張琳所守地,與越州接壤,部眾多相與貿易。春使人結阿保等,覘阿資所在及其經行地,星列守堡,絕其糧道,賊益困。二十八年,福潛引兵屯赤窩鋪,遣百戶張忠等搗賊巢,擒阿資,斬之,俘其黨,越州乃平。自是以後,諸土官按期朝貢,西南晏然。

正統二年,曲靖軍民知府晏毅言四事:一,土官承襲,或子孫,或兄弟,或妻繼夫,或妾繼嫡,皆無豫定次序,致臨襲爭奪,仇殺連年。乞敕該部移文所司,豫為定序造冊,土官有故,如序襲職。一,請恤陣亡子孫。一,請雲南官俸,悉如四川之例。一,均戶口田地。事下所司議行。毅復請設霑益州松韶巡檢,從之。

嘉靖中,羅雄知州者浚殺營長,奪其妻,生子繼榮,稍長即持刀逐浚。浚欲置之死,以其母故不忍。及浚請老,以繼榮代襲,繼榮遂逐濬。浚訴之鎮巡官,命迎濬歸。繼榮陽事之,實加禁錮。萬曆九年調羅雄兵征緬。繼榮將行,恐留浚為難,遂弒浚。時沾益土知州安世鼎死,妻安素儀署州事,亦提兵赴調。繼榮與之合營,通焉,且倚沾益兵力為助。師過越州,留土官資氏家,淫樂不進。知州越應奎白於兵備,將擒之,繼榮走,遂聚眾反。攻破陸涼鴨子塘、陡陂諸寨,築石城於赤龍山,據龍潭為險,廣六十里。名己所居曰「龍樓鳳閣」,環以群寨,實諸軍士妻女其中。十三年,巡撫劉世曾乃檄諸道進兵。適劉綎破緬解官回,世曾以兵屬綎。綎遂馳赴普鮓營,直搗赤龍寨,斬賊渠帥,繼榮遁去。綎復連破三寨,降其眾一萬七千人,追奔至阿拜江,斬繼榮,賊平。世曾請築城,改設流官,乃以何倓為知州,者繼仁為巡檢。未幾,蠻寇必大反,殺繼仁,執倓。參將蔡兆吉等討定之,乃改羅雄州曰羅平,設千戶所曰定雄。

時霑益安素儀無子,以烏撒土官子安紹慶為嗣。慶死,孫安遠襲。土婦設科作亂,逐安遠,糾眾焚掠霑益諸堡站,陷平夷衛。天啟三年,官兵擒設科,誅之。五年,安邊據沾益,從水西叛。事詳《烏撒傳》中。

初,越州阿資罪誅,永樂間以其子祿寧為土縣丞,與亦佐沙氏分土而居。其地南北一百二十里,士馬精強,徵調銀至三千八百兩。

曲靖境內有交水,去平夷衛二舍,與黔接壤,滇師出上六衛必由之道。天啟初,水西用兵,撫臣議:「曲靖鎖鑰全滇,交水當黔、滇之沖,乃阨塞要地。平夷右所宜移置交水,去險築城,俾與平夷衛相望,互為聲援,便。」報可。


\end{pinyinscope}