\article{列傳第一百九十四 閹黨}

\begin{pinyinscope}
明代閹宦之禍酷矣,然非諸黨人附麗之,羽翼之,張其勢而助之攻,虐焰不若是其烈也。中葉以前是由人的生理組織構造決定的。康德的「自在之物」只不過,士大夫知重名節,雖以王振、汪直之橫,黨與未盛。至劉瑾竊權,焦芳以閣臣首與之比,於是列卿爭先獻媚,而司禮之權居內閣上。迨神宗末年,訛言朋興,群相敵仇,門戶之爭固結而不可解。凶豎乘其沸潰,盜弄太阿,黠桀渠憸,竄身婦寺。淫刑痡毒,快其惡正醜直之私。衣冠填於狴犴,善類殞於刀鋸。迄乎惡貫滿盈,亟伸憲典,刑書所麗,迹穢簡編,而遺孽餘燼,終以覆國。莊烈帝之定逆案也,以其事付大學士韓爌等,因慨然太息曰:「忠賢不過一人耳,外廷諸臣附之,遂至於此,其罪何可勝誅!」痛乎哉,患得患失之鄙夫,其流毒誠無所窮極也!今錄自焦芳、張彩以下,迄天啟朝,為《閹黨列傳》,用垂鑒誡。其以功名表見,或晚節自蓋,如王驥、王越、楊維垣、張捷之徒,則仍別見焉。

○焦芳劉宇曹元張彩韓福等顧秉謙魏廣微等崔呈秀吳淳夫等劉志選梁夢環等曹欽程石三畏等王紹徽周應秋霍維華徐大化等閻鳴泰賈繼春田爾耕許顯純

焦芳,泌陽人。天順八年進士。大學士李賢以同鄉故,引為庶吉士,授編修,進侍講。滿九年考,當遷學士。或語大學士萬安:「不學如芳,亦學士乎?」芳聞大恚曰:「是必彭華間我也。我不學士,且刺華長安道中。」華懼,言于安,乃進芳侍講學士。先是,詔纂《文華大訓》,進講東宮,其書皆華等所為。芳恥不與,每進講,故摘其疵,揚言眾中。翰林尚文采,獨芳粗陋無學識,性陰很,動輒議訕,人咸畏避之。尹旻之罷也,芳與其子龍相比,謫桂陽州同知。芳知出華、安二人指,銜次骨。

弘治初,移霍州知州,擢四川提學副使,調湖廣。未幾,遷南京右通政,以憂歸。服闋,授太常少卿兼侍講學士,尋擢禮部右侍郎。怨劉健尼己,日於眾中嫚罵。健判牒不可意,即引筆抹去,不關白尚書。俄改吏部,轉左侍郎。馬文升為尚書,芳輒加姍侮,陰結言官,使抨擊素所不快及在己上者。又上言禦邊四事以希進用,為謝遷所抑,尤憾遷。每言及餘姚、江西人,以遷及華故,肆口詬詈。芳既積忤廷臣,復銳進,乃深結閹宦以自固,日夜謀逐健、遷,代其位。

正德初,戶部尚書韓文言會計不足。廷議謂理財無奇術,唯勸上節儉。芳知左右有竊聽者,大言曰:「庶民家尚須用度,況縣官耶?諺云『無錢揀故紙』。今天下逋租匿稅何限,不是檢索,而但云損上何也?」武宗聞之大喜。會文升去,遂擢為吏部尚書。韓文將率九卿劾劉瑾,疏當首吏部,以告芳。芳陰洩其謀於瑾。瑾遂逐文及健、遷輩,而芳以本官兼文淵閣大學士,入閣輔政,累加少師、華蓋殿大學士。居內閣數年,瑾濁亂海內,變置成法,荼毒縉紳,皆芳導之。每過瑾,言必稱千歲,自稱曰門下。裁閱章奏,一阿瑾意。四方賂瑾者先賂芳。子黃中,亦傲很不學,廷試必欲得第一。李東陽、王鏊為置二甲首,芳不悅。言於瑾,徑授翰林檢討,俄進編修。芳以黃中故,時時詈東陽。瑾聞之曰:「黃中昨在我家試石榴詩,甚拙,顧恨李耶?」

瑾怒翰林官傲己,欲盡出之外,為張彩勸沮。及修《孝家實錄》成,瑾又持前議,彩復力沮。而芳父子與檢討段炅輩,教瑾以擴充政事為名,乃盡出編修顧清等二十餘人於部曹。有司應詔舉懷材抱德之士,以餘姚人周禮、徐子元、許龍,上虞人徐文彪四人名上。瑾以禮等皆遷鄉人,而詔草出健,因下四人詔獄,欲併逮健、遷。東陽力解之。芳厲聲曰:「縱貰其罪,不當除名耶?」乃黜健、遷為民,而榜逐餘姚人之為京官者。

滿剌加使臣亞劉,本江西萬安人,名蕭明舉。以罪叛入其國,與其國人端亞智等來朝。既又謀入浡泥國索寶,且殺亞智等。事聞,方下所司勘奏。芳即署其尾曰:「江西土俗,故多玩法,如彭華、尹直、徐瓊、李孜省、黃景等,多被物議。宜裁減解額五十名,通籍者勿選京職,著為令。」且言:「王安石禍宋,吳澄仕元,宜榜其罪,使他日毋得濫用江西人。」楊廷和解之曰:「以一盜故,禍連一方,至裁解額矣。宋、元人物,亦欲併案耶?」乃止。

芳深惡南人,每退一南人,輒喜。雖論古人,亦必詆南而譽北,嘗作《南人不可為相圖》進瑾。其總裁《孝宗實錄》,若何喬新、彭韶、謝遷皆肆誣詆,自喜曰:「今朝廷之上,誰如我直者。」

始張彩為郎時,芳力薦以悅瑾,覬其為奸利。比彩為尚書,芳父子薦人無虛日,綵時有同異,遂有隙。而段炅見瑾暱彩,芳勢稍衰,轉附彩,盡發芳陰事於瑾。瑾大怒,數於眾中斥芳父子。芳不得已,乃乞歸。

黃中頠閣廕,以侍讀隨父還。瑾敗,給事、御史交劾,削其官,黜黃中為民。久之,芳使黃中齎金寶遺權貴,上章求湔雪復官,為吏科所駁。於是吏部覆奏,請械繫黃中法司,以彰天討。黃中狼狽遁走。

芳居第宏麗,治作勞數郡。大盜趙鐩入泌陽,火之,發窖多得其藏金,乃盡掘其先人塚墓,雜燒以牛馬骨。求芳父子不得,取芳衣冠被庭樹,拔劍斫其首,使群盜糜之,曰:「吾為天子誅此賊。」鐩後臨刑歎曰:「吾不能手刃焦芳父子以謝天下,死有餘恨!」瑾從孫二漢當死,亦曰:「吾死固當,第吾家所為,皆焦芳與張彩耳。今彩與我處極刑,而芳獨晏然,豈非冤哉。」芳父子竟良死。

劉宇,字至大,鈞州人。成化八年進士。由知縣入為御史,坐事謫,累遷山東按察使。弘治中,以大學士劉健薦,擢右僉都御史,巡撫大同,召為左副都御史。正德改元,吏部尚書馬文升薦之,進右都御史,總督宣府、大同、山西軍務。宇初撫大同,私市善馬賂權要。兵部尚書劉大夏因孝宗召見,語及之。帝密遣錦衣百戶邵琪往察,宇厚賂琪,為之抵諱。後大夏再召對,帝曰:「健薦宇才堪大用,以朕觀之,此小人,豈可用哉?由是知內閣亦未可盡信也。」宇聞,以大夏不為己地,深憾之。

劉瑾用事,宇介焦芳以結瑾。二年正月入為左都御史。瑾好摧折臺諫,宇緣其意,請敕箝制御史,有小過輒加笞辱,瑾以為賢。瑾初通賄,望不過數百金,宇首以萬金贄,瑾大喜曰;』劉先生何厚我。」尋轉兵部尚書,加太子太傅。子仁應殿試,求一甲不得。厚賄瑾,內批授庶吉士,踰年遷編修。時許進為吏部尚書,宇讒於瑾,遂代其位,而曹元代宇為兵部。宇在兵部時,賄賂狼籍。及為吏部,權歸選郎張綵,而文史贈遺又不若武弁,嘗悒悒歎曰:「兵部自佳,何必吏部也。」後瑾欲用彩代宇,乃令宇以原官兼文淵閣大學士。宇宴瑾閣中,極馭,大喜過望。明日將入閣辦事。瑾曰:「爾真欲相耶?此地豈可再入。」宇不得已,乃乞省墓去。踰年瑾誅,科道交章劾奏,削官致仕,子仁黜為民。

曹元,字以貞,大寧前衛人。柔佞滑稽,不修士行。舉成化十一年進士。授工部主事。正德二年累遷右副都御史,巡撫甘肅。分守中官張昭奉命捕虎豹,元以軍士出境搜捕,恐啟邊釁,上疏請止,不從。改撫陜西。踰年,召為兵部右侍郎,轉左,尋代宇為尚書兼督團營,加太子少保。將校遷除,皆惟瑾命。元所入亦不貲。五年拜吏部尚書兼文淵閣大學士。元與劉瑾有連,自瑾侍東宮,即與相結。及瑾得志,遂夤緣躐至卿相,然瑣刺無能,在閣中飲酒諧謔而已。瑾敗,元即日上疏請罪,詞極哀。詔許致仕,言官交劾,黜為民。元無子,病中自作墓志,歎曰:「我死,誰銘我者!」

當劉瑾時,廷臣黨附者甚眾。瑾誅,言官交劾。內閣則焦芳、劉宇、曹元。尚書則吏部張彩、戶部劉璣、兵部王敞、刑部劉璟、工部畢亨、南京戶部張澯、禮部朱恩、刑部劉纓、工部李善。侍郎則吏部柴昇、李瀚,前戶部韓福,禮部李遜學,兵部陸完、陳震,刑部張子麟,工部崔巖、夏昂、胡諒,南京禮部常麟、工部張志淳。都察院則副都御史楊綸、僉都御史蕭選。巡撫則順天劉聰、應天魏訥、宣府楊武、保定徐以貞、大同張禴、淮揚屈直、兩廣林廷選,操江王彥奇。前總督文貴、馬炳然。大理寺則卿張綸,少卿董恬,丞蔡中孚、張檜。通政司則通政吳釴、王雲鳳,參議張龍。太常則少卿楊廷儀、劉介。尚寶卿則吳世忠,丞屈銓。府尹則陳良器,府丞則石祿。翰林則侍讀焦黃中,修撰康海,編修劉仁,檢討段炅。吏部郎則王九思、王納誨。給事中則李憲、段豸。御史則薛鳳鳴、朱袞、秦昂、宇文鐘、崔哲、李紀、周琳。其他郎署監司又十餘人。於是彩論死,福謫戍,元、恩、震、聰、訥、武、恬、介、黃中、海、仁、憲、鳳鳴、鐘除名,亨、昂閒住,善、巖、諒、志淳、綸、直、彥奇、良器、哲致仕,選、以貞、示龠、中孚、龍、祿、銓、炅、豸、袞、紀、琳、九思,納誨謫外,朝署為清。

張彩,安定人。弘治三年進士。授吏部主事,歷文選司郎中。彩議論便利,善伺權貴指。初矯飾徹聲譽,尚書馬文升等皆愛之。給事中劉郤嘗劾其顛倒選法數事,文升悉為辯析,且譽其聰明剛正,為上下所推服。詔令辦事如故。彩即五疏移疾去,文升固留不得,時論稱之。越數日,給事李貫薦彩有將略。楊一清總制三邊,亦薦彩自代。而焦芳以綵與劉瑾同鄉,力薦於瑾。瑾欲致之,因著令,病過期不赴者,斥為民。彩乃就道。既見瑾,高冠鮮衣,貌白晳修偉,鬚眉蔚然,詞辯泉湧。瑾大敬愛,執手移時,曰:「子神人也,我何以得遇子!」時文選郎劉永已遷通政,次當驗封郎石確。疏既入,瑾令尚書許進追原疏,以彩易之。彩自是一意事瑾。瑾惡進不附己,彩因媒孽去進,以劉宇代之。宇雖為尚書,銓政率由彩,多不關白宇,即白宇,宇必溫言降接。彩抱案立語,宇俯僂不敢當。居文選半載,擢左僉都御史,與戶部右侍郎韓鼎同廷謝。鼎老,拜起不如儀,為谷大用、張永輩所竊笑。瑾方慚,而彩豐采英毅,大用等皆稱羨,瑾乃喜。越二日罷鼎,而彩踰年超拜吏部右侍郎。

鼎,合水人。弘治時,為給事中,負直聲。後遷右通政,治水安平有勞績,以通政使家居。至是為瑾所引,復挫歸,遂失其素望。

瑾欲大貴彩,乃命劉宇入內閣,以彩代之。一歲中,自郎署長六卿。僚友守官如故,咸惴惴白事尚書前,彩厲色無所假借。尋加太子少保。每瑾出休沐,公卿往候,自辰至哺未得見。彩故徐徐來,直入瑾小閣,歡飲而出,始揖眾人。眾以是益畏彩,見綵如瑾禮。彩與朝臣言,呼瑾為老者。凡所言,瑾無不從。因不時考察內外官,糾摘嚴急,間一用薄罰,而諸司臺諫謫辱日甚。變亂舊格,賄賂肆行,海內金帛奇貨相望塗巷間。性尤漁色。撫州知府劉介,其鄉人也,娶妾美。彩特擢介太常少卿,盛服往賀曰:「子何以報我?」介皇恐謝曰:「一身外,皆公物。」彩曰:「命之矣。」即使人直入內,牽其妾,輿戴而去。又聞平陽知府張恕妾美,索之不肯,令御史張禴按致其罪,擬戍。恕獻妾,始得論減。

彩既銜瑾恩,見瑾擅權久,貪冒無厭,天下怨之,因乘間說曰:「公亦知賄入所自乎?非盜官帑,即剝小民。彼借公名自厚,入公者未十一,而怨悉歸公,何以謝天下,」瑾大然之。會御史胡節巡按山東還,厚遺瑾。瑾發之,捕節下獄。少監李宣、侍郎張鸞、指揮同知趙良按事福建還,饋瑾白金二萬。瑾疏納金於官,而按三人罪。其他因賄得禍者甚眾。苛斂之害為少衰,中外或稱彩能導瑾為善矣。及瑾伏誅,彩以交結近侍論死,遇赦當免。改擬同瑾謀反,瘐死獄中,仍剉屍於市,籍其家,妻子流海南。

韓福者,西安前衛人也。成化十七年進士。為御史,按宣府、大同,數條奏軍民利病,邊人悅之。弘治中,遷大名知府,奸盜屏跡,道不拾遺,政績為畿輔冠。以卓異舉,遷浙江左參政,病免。

武宗立,言官交薦,召為大理右少卿。正德二年以右僉都御史督蘇、松糧儲。未幾,召入為右副都御史。坐累,下詔獄。獄上,劉瑾以同鄉故,立命出之。召與語,大悅,即用為戶部左侍郎。福強結幹吏,所在著能聲。至是受挫,為瑾所拔擢,遂精心事瑾,為效力。瑾亦時召與謀,委寄亞於彩。會湖廣以缺餉告,命兼僉都御史往理之。瑾喜操切,福希指,益務為嚴苛。湖廣民租自私弘治改元後,逋六百餘萬石,皆遇災蠲免。福欲追徵之,劾所司催科不力,自巡撫鄭時以下凡千二百人。奏至,舉朝駭愕,戶部尚書劉璣等議如福言。瑾忽怒福,取詔旨報曰:「湖廣軍民困敝,朕甚憫之。福任意苛斂,甚不稱朕意,令自劾,吏部舉堪代者以聞。」福引罪求罷,乃召還。四年復命核遼東屯田。福性故刻深,所攜同知劉玉等又奉行過當。軍士不能堪,焚掠將吏及諸大姓家。守臣發帑撫慰之。亂始定。給事中徐仁等極論之。瑾迫公議,勒福致仕。明年瑾敗,籍其貲,則福在湖廣時所餽白金數十萬兩,封識宛然,遂遣戍固原。

李憲,岐山人。為吏科給事中,諂事瑾,每率眾請事於瑾,盛氣獨前,自號六科都給事中。時袖白金示同列曰:「此劉公所遺也。」瑾敗,虞禍及,亦劾瑾六事。瑾在獄,笑曰:「李憲亦劾我乎?」卒坐除名。

張龍,順天人。官行人,邪媚無賴,與壽寧侯通譜系,因得交諸中人、貴戚,恃勢奪人田宅。正德三年夤緣為兵科給事中,出核遼東軍餉,得腐豆四石。請逮問監守諸臣,罰郎中徐璉以下米三百石有差。瑾以為能,擢通政參議。瑾敗,謫知灤州。後又結朱寧為父,起嘉興同知,遷登州知府。言官彈射無虛月。與山西左布政使倪天民、右布政使陳逵、右參議孫清並貪殘,天下目為「四害」。龍朝覲入都,中旨擢右通政,為寧通中外賄,所乾沒不貲。後以私取賄,為寧所覺,斥逐之。嘉靖初,下獄論死。

顧秉謙,崑山人。萬曆二十三年進士。改庶吉士,累官禮部右侍郎,教習庶吉士。天啟元年晉禮部尚書,掌詹事府事。二年,魏忠賢用事,言官周宗建等首劾之。忠賢於是謀結外廷諸臣,秉謙及魏廣微率先諂附,霍維華、孫傑之徒從而和之。明年春,秉謙、廣微遂與朱國禎、朱延禧俱入參機務。

廣微,南樂人,侍郎允貞子也。萬曆三十二年進士。由庶吉士歷南京禮部侍郎。忠賢用事,以同鄉同姓潛結之,遂召拜禮部尚書。至是,與秉謙俱以原官兼東閣大學士。七月,秉謙晉太子太保,改文淵閣。十一月晉少保、太子太傅。五年正月晉少傅、太子太師、吏部尚書,改建極殿。九月晉少師。

秉謙為人,庸劣無恥,而廣微陰狡。趙南星與其父允貞友善,嘗歎曰:「見泉無子。」見泉,允貞別號也。廣微聞之,恨刺骨。既柄政,三及南星門,閽人辭不見。廣微怫然曰:「他人可拒,相公尊,不可拒也。」益恨南星。楊漣之劾忠賢二十四罪也,忠賢懼,屬廣微為調旨,一如忠賢意。而秉謙以漣疏有「門生宰相」語,怒甚。會孟冬饗廟,且頒朔,廣微偃蹇後至,給事中魏大中、御史李應昇連劾之。廣微益憤,遂決意傾善類,與秉謙謀盡逐諸正人,點《縉紳便覽》一冊,若葉向高、韓爌、何如寵、成基命、繆昌期、姚希孟、陳子壯、侯恪、趙南星、高攀龍、喬允升、李邦華、鄭三俊、楊漣、左光斗、魏大中、黃尊素、周宗建、李應昇等百餘人,目為邪黨,而以黃克纘、王永光、徐大化、賈繼春、霍維華等六十餘人為正人,由閹人王朝用進之,俾據是為黜陟。忠賢得內閣為羽翼,勢益張。秉謙、廣微亦曲奉忠賢,若奴役然。

葉向高、韓爌相繼罷,何宗彥卒,秉謙遂為首輔。自四年十二月至六年九月,凡傾害忠直,皆秉謙票擬。《三朝要典》之作,秉謙為總裁,復擬御製序冠其首,欲用是鉗天下口。朝廷有一舉動,輒擬旨歸美忠賢,褒贊不已。廣微以札通忠賢,簽其函曰「內閣家報」,時稱曰「外魏公」。先是,內閣調旨,惟出首輔一人,餘但參議論而已。廣微欲擅柄,謀之忠賢,令眾輔分任,政權始分,後遂沿為故事。

楊漣等六人之逮也,廣微實與其謀,秉謙調嚴旨,五日一追比。尚書崔景榮懼其立死杖下,亟請廣微諫止。廣微不自安,疏言:「漣等在今日,誠為有罪之人,在前日實為卿寺之佐。縱使贓私果真,亦當輔付法司,據律論罪,豈可逐日嚴刑,令鎮撫追贓乎?身非木石,重刑之下,就死直須臾耳。以理刑之職,使之追贓,官守安在?勿論傷好生之仁,抑且違祖宗之制,將朝政日亂,與古之帝王大不相侔矣。」疏入,大忤忠賢意。廣微懼,急出景榮手書自明,而忠賢怒已不可解。乃具疏乞休,不許。居兩月,矯詔切責廷臣,中言「朕方率循舊章,而曰『朝政日亂』,朕方祖述堯、舜,而曰『大不相侔』」,蓋即指廣微疏語。廣微益懼,丐秉謙為解,忠賢意少釋。然廣微卒不自安,復三疏乞休,五年八月許之去。廣微先已加少保、太子太傅,改吏部尚書、建極殿大學士,至是復加少傅、太子太師,廕子中書舍人,賜白金百、坐蟒一、彩幣四表裏,乘傳,行人護歸。典禮優渥,猶用前好故也。居二年,卒於家,贈太傅,恤典如制。

秉謙票擬,事事徇忠賢指。初矯旨罪主考丁乾學,又調旨殺漣、光斗等。惟周順昌、李應升等下詔獄,秉謙請付法司,毋令死非其罪。內臣出鎮,秉謙撰上諭,已復與丁紹軾請罷。二事微有執爭。馮銓既入閣,同黨中日夜交輒,群小亦各有所左右。秉謙不自安,屢疏乞休,後廣微一年致仕去。崇禎元年,為言官祖重曄、徐尚勛、汪應元所糾,命削籍。已,坐交結近侍,入逆案中,論徒三年,贖為民。二年,崑山民積怨秉謙,聚眾焚掠其家。秉謙年八十,倉皇竄漁舟得免,乃獻窖藏銀四萬於朝,寄居他縣以死。廣微亦追論削奪,列逆案遣戍中。

自秉謙、廣微當國,政歸忠賢。其後入閣者黃立極、施鳳來、張瑞圖之屬,皆依媚取容,名麗逆案。

黃立極,字中五,元城人。萬曆三十二年進士。累官少詹事、禮部侍郎。天啟五年八月,忠賢以同鄉故,擢禮部尚書兼東閣大學士,與丁紹軾、周如磐、馮銓並參機務。時魏廣微、顧秉謙皆以附忠賢居政府。未幾廣微去,如磐卒。明年夏,紹軾亦卒,銓罷。其秋,施鳳來、張瑞圖、李國普入。己而秉謙乞歸,立極遂為首輔。

施鳳來,平湖人。張瑞圖,晉江人。皆萬曆三十五年進士。鳳來殿試第二,瑞圖第三,同授編修,同積官少詹事兼禮部侍郎,同以禮部尚書入閣。鳳來素無節概,以和柔媚於世。瑞圖會試策言:「古之用人者,初不設君子小人之名,分別起於仲尼。」其悖妄如此。忠賢生祠碑文,多其手書。莊烈帝即位,山陰監生胡煥猷劾立極、鳳來、瑞圖、國普等,「身居揆席,漫無主持。甚至顧命之重臣,斃於詔獄;五等之爵,尚公之尊,加於閹寺;而生祠碑頌,靡所不至。律以逢奸之罪,夫復何辭?」帝為除煥猷名,下吏。立極等內不自安,各上疏求罷,帝猶優詔報之。十一月,立極乞休去,來宗道、楊景辰並入閣,鳳來為首輔。御史羅元賓復疏糾,鳳來、瑞圖俱告歸。

宗道,蕭山人。立極同年進士,累官太子太保、禮部尚書,以本官兼內閣大學士,預機務。宗道官禮部時,為崔呈秀父請恤典,中有「在天之靈」語。編修倪元璐屢疏爭時事,宗道笑曰:「渠何事多言,詞林故事,止香茗耳。」時謂宗道清客宰相云。

景辰,瑞圖同縣人。萬曆四十一年進士。積官吏部右侍郎,與宗道同入閣。官翰林時,為《要典》副總裁,一徇奸黨指,又三疏頌忠賢。及朝局已變,乃請毀《要典》,給事、御史交劾之,與宗道同日罷。

其後定逆案,瑞圖、宗道初不與,莊烈帝詰之,韓爌等封無實狀。帝曰:「瑞圖為忠賢書碑,宗道稱呈秀父『在天之靈』,非實狀耶?」乃以瑞圖、宗道與顧秉謙、馮銓等坐贖徒為民,而立極、鳳來、景辰落職閒住。

崔呈秀,薊州人。萬曆四十一年進士。授行人。天啟初,擢御史,巡按淮、揚。卑汙狡獪,不修士行。見東林勢方盛,將出都,力薦李三才,求入其黨,東林拒不納。在淮、揚,贓私狼籍。霍丘知縣鄭延祚貪,將劾之,以千金賄免。延祚知其易與,再行千金,即薦之。其行事多類此。

四年九月還朝,高攀龍為都御史,盡發其貪污狀。吏部尚書趙南星議戍之,詔革職候勘。呈秀大窘,夜走魏忠賢所,叩頭乞哀,言攀龍、南星皆東林,挾私排陷,復叩頭涕泣,乞為養子。當是時,忠賢為廷臣交攻,憤甚,方思得外廷為助。涿州人馮銓,少年官侍從家居,與熊廷弼有隙,遺書魏良卿勸興大獄。忠賢冀假事端傾陷諸害己者,得呈秀,恨相見晚,遂用為腹心,日與計畫。明年正月,給事中李恒茂為呈秀訟冤。中旨即言呈秀被誣,復其官。呈秀乃首疏薦張鶴鳴、申用懋、王永光、商周祚、許弘綱等;而再疏請令京官自陳,由是清流多屏斥。尋督三殿工,忠賢以閱工故,日至外朝。呈秀必屏人密語,以間進《同志》諸錄,皆東林黨人。又進《天鑒錄》,皆不附東林者。令忠賢憑以黜陟,善類為一空。暮夜乞憐者,莫不緣呈秀以進,繩集蟻附,其門如市。累擢工部右侍郎並兼御史,督工如故。御史田景新言,侍郎兼御史非便,請改僉都御史,從之。

忠賢嘗修鄉縣肅寧城,呈秀首上疏稱美。六年二月,復疏頌忠賢督工功,請賜敕獎諭,末言:「臣非行媚中官者,目前千譏萬罵,臣固甘之。」疏出,朝野轟笑。閣臣顧秉謙輩撰敕八百餘言,褒忠賢,極口揚詡,前代九錫文不能過也。自是,中外章疏,無不頌忠賢德者矣。時方創《三朝要典》,呈秀疏陳耍典之源,追論並封、妖書、之籓三事,凡擁衛光宗者,悉加醜詆。忠賢悅,宣付史館。其年七月,進本部尚書。十月,皇極殿成,加太子太保兼左都御史,仍督大工。母死,不奔喪,奪情視事。呈秀負忠賢寵,嗜利彌甚。朝士多拜為門下士,以通於忠賢。其不附己及勢位相軋者,輒使其黨排去之,時有「五虎」之目,以呈秀為魁。請所傾陷,不可悉數,雖其黨亦深畏之。子鐸不能文,屬考官孫之獬,獲鄉薦。用其弟凝秀為浙江總兵官,女夫張元芳為吏部主事,妾弟優人蕭惟中為密雲參將,所司皆不敢違。明年八月冒寧、錦功,加太子太傅。俄敘三殿功,加少傅,世廕錦衣指揮僉事。其月遷兵部尚書,仍兼左都御史,並綰兩篆,握兵權憲紀,出入亙赫,勢傾朝野。無何,熹宗崩,廷臣入臨。內使十餘人傳呼崔尚書甚急,廷臣相顧愕眙。呈秀入見忠賢,密謀久之,語秘不得聞。或言忠賢欲篡位,呈秀以時未可,止之也。

莊烈帝即位,其黨知忠賢必敗,內相攜。副都御史楊所修首請允呈秀守制,御史楊維垣、賈繼春相繼力攻,呈秀乞罷。帝猶慰留。章三上,溫旨令乘傳歸。已而言者劾呈秀及工部尚書吳淳夫、兵部尚書田吉、太常卿倪文煥、副都御史李夔龍,號稱「五虎」,宜肆市朝。詔逮治,籍其貲。時忠賢已死,呈秀知不免,列姬妾,羅諸奇異珍寶,呼酒痛飲,盡一卮即擲壞之,飲已自縊。詔戮其屍,子鐸除名,弟凝秀遣戍。後定逆案,以呈秀為首。

淳夫,晉江人。萬曆三十八年進士。歷官陜西僉事,以京察罷。五年夤緣起兵部郎中,與文煥、吉、夔龍並由呈秀進,為忠賢義子。大學士馮銓釋褐十三年登宰輔,為忠賢所暱。呈秀妒之,淳夫即為攻銓。六年冬,擢太僕少卿,視職方事。旋擢太僕卿,歷工部添注右侍郎。冒寧、錦及三殿功,累進工部尚書,加太子太傅。歲中六遷,至極品。

倪文煥,江都人。由進士授行人,擢御史,巡視南城。山東多大猾,事發則走匿京師。參政王維章數牒文煥,文煥納其賄,反劾罷維章。嘗誤撻皇城守卒,為中官所糾,大懼,走謁呈秀求救,遂引入忠賢幕,為鷹犬。首劾兵部侍郎李邦華,御史李日宣,吏部員外郎周順昌、林枝橋。再劾戶部侍郎孫居相、御史夏之令及故吏部尚書崔景榮、吏部尚書李宗延等數十人。輕者削奪,重者拷死。呈秀首頌忠賢,文煥即繼之。出按畿輔,為忠賢建三祠。河南道缺掌印官,呈秀為懸缺待文煥,至越十餘人任之。冒寧、錦、殿功,加太僕卿,掌道如故。尋改太常卿。忠賢敗,文煥懼,乞終養歸。

田吉者,故城人。萬曆三十八年廷對懷挾,罰三科,以縣佐錄用。已,補試,由知縣歷兵部郎中。六年冬,遷淮揚參議,取中旨,擢太常少卿,視職方事。明年擢太常卿。未匝歲,連擢至兵部尚書,加太子太保。諸逆黨超擢,未有如吉者。

李夔龍,福建南安人。由進士歷吏部主事,被劾罷去。天啟五年夤緣復官,進郎中。專承呈秀指,引用邪人以媚忠賢。擢太常少卿,仍署選事。尋遷左僉都御史。三殿成,進左副都御史。

莊烈帝嗣位,淳夫、文煥、吉、夔龍,並以上林典薄樊維城、戶部員外郎王守履言,逮治論死。

方忠賢敗時,莊烈帝納廷臣言,將定從逆案。大學士韓爌、李標、錢錫不欲廣搜樹怨,僅以四五十人上。帝少之,令再議,又以數十人上。帝不懌,令以贊導、擁戴、頌美、諂附為目,且曰:「內侍同惡者亦當入。」爌等以不知內侍對,帝曰:「豈皆不知,特畏任怨耳。」閱日,召入便殿,案有布囊,盛章疏甚夥,指之曰:「此皆奸黨頌疏,可案名悉入。」爌等知帝意不可回,乃曰:「臣等職在調旨,三尺法非所習。」帝召吏部尚書王永光問之,永光以不習刑名對,乃詔刑部尚書喬允升、左都御史曹于汴同事,於是案名羅列無脫遺者。崇禎二年三月上之,帝為詔書頒示天下。

首逆凌遲者二人:魏忠賢,客氏。

首逆同謀決不待時者六人:呈秀及魏良卿,客氏子都督侯國興,太監李永貞、李朝欽、劉若愚。

交結近侍秋後處決者十九人:劉志選、梁夢環、倪文煥、田吉、劉詔、薛貞、吳淳夫、李夔龍、曹欽程,大理寺正許志吉,順天府通判孫如冽,國子監生陸萬齡,豐城侯李承祚,都督田爾耕、許顯純、崔應元、楊寰、孫雲鶴、張體乾。

結交近侍次等充軍者十一人:魏廣微、周應秋、閻嗚泰、霍維華、徐大化、潘汝禎、李魯生、楊維垣、張訥,都督郭欽,孝陵衛指揮李之才。

交結近侍又次等論徒三年輸贖為民者:大學士顧秉謙、馮銓、張瑞圖、來宗道,尚書王紹徽、郭允厚、張我續、曹爾禎、孟紹虞、馮嘉會、李春曄、邵輔忠、呂純如、徐兆魁、薛風翔、孫傑、楊夢袞、李養德、劉廷元、曹思誠,南京尚書范濟世、張樸,總督尚書黃運泰、郭尚友、李從心,巡撫尚書李精白等一百二十九人。

交結近侍減等革職閒住者,黃立極等四十四人。

忠賢親屬及內官黨附者又五十餘人。

案既定,其黨日謀更翻,王永光、溫體仁陰主之,帝持之堅,不能動。其後,張捷薦呂純如,被劾去。唐世濟薦霍維華,福建巡按應喜臣薦部內閒住通政使周維京,罪至謫戍。其黨乃不敢言。福王時,阮大鋮冒定策功,起用,其案始翻。於是太僕少卿楊維垣、徐景濂,給事中虞廷陛、郭如暗,御史周昌晉、陳以瑞、徐復陽,編修吳孔嘉,參政虞大復輩相繼而起,國亡乃止。

劉志選,慈谿人。萬曆中,與葉向高同舉進士。授刑部主事,偕同官劉復初、李懋檜爭鄭貴妃、王恭妃冊封事。後懋檜因給事中邵庶請禁諸曹言事,抗疏力爭,貶二秩。志選言:「陛下謫懋檜,使人箝口結舌,蒙蔽耳目,非國家福也。」帝怒,謫福寧州判官。稍遷合肥知縣,以大計罷歸,家居三十年。光宗、熹宗相繼立,諸建言得罪者盡起,志選獨以計典不獲與。會向高赴召,道杭州,志選與遊宴彌月。還朝,用為南京工部主事,進郎中,時已七十餘,嗜進彌銳,上疏追論「紅丸」,極詆孫慎行不道。魏忠賢喜,天啟五年九月召為尚寶少卿。在道,復力攻慎行,遂並及向高。忠賢益喜,出兩疏宣史館。

明年擢順天府丞。冬十月遂上疏劾張國紀。國紀者,后父也。忠賢忌后賢明,欲傾之。會有張匿名榜於厚載門者,列忠賢反狀,並其黨七十餘人。忠賢疑出國紀及被逐諸人手。邵輔忠、孫傑謀因此興大獄,盡殺東林諸人,而借國紀以搖中宮,事成則立魏良卿女為后,草一疏,募人上之。諸人慮禍不敢承。志選惑家人言,謂己老必先忠賢死,竟上之。極論國紀罪,而末言「毋令人訾及丹山之穴,藍田之種。」蓋前有死囚孫二言張后己所生,非國紀女也。疏上,事叵測。帝伉儷情篤,但令國紀自新而已。后為故司禮劉克敬所選,忠賢遷怒克敬,謫發鳳陽,縊殺之。未幾,志選疏頌《要典》,言:「命德討罪,無微不彰,即堯、舜之放四凶,舉元、愷,何以加焉,洵游、夏無能贊一詞者。」因力詆王之寀、孫慎行、楊漣、左光斗,而極譽劉廷元、岳駿聲、黃克纘、徐景濂、范濟世、賈繼春並及傅櫆、陳九疇。且言:「慷慨憂時,力障狂瀾於既倒者,魏廣微也,當還之揆席,以繼五臣之盛事。赤忠報國,弼成巨典於不日者,廠臣也,當增入簡端,以揚一德之休風。」又言:「之寀宜正典刑,慎行宜加謫戍。」忠賢大悅,於是駿聲等超擢,之寀被逮,慎行遣戍,悉如志選言。

七年擢右僉都御史,提督操江。其年,熹宗崩,忠賢敗,言官交劾,詔削籍。後定逆案,律無傾搖國母文,坐子罵母律,與梁夢環並論死。志選先自經。

夢環,廣東順德人。舉進士。歷官御史。父事忠賢,興汪文言獄,殺楊漣等。出巡山海關,會寧遠敘功,崔呈秀不獲與,夢環力敘其賢勞,遂進侍郎。劾熊廷弼乾沒軍資十七萬,廷弼已死,家益破。志選之劾國紀也,忠賢意未逞。夢環偵知之,七年二月馳疏極論國紀罪,且故詰「丹山、藍田」二語,冀傾后。顧事重,忠賢亦不能驟行,而國紀竟勒還籍。夢環建祠祀忠賢,三疏頌功德。寧、錦之役,復稱忠賢「德被四方,勛高百代」,於是有安平之封,夢環擢太僕卿。

又劉詔者,杞縣人。萬曆四十七年進士。授盧龍知縣。天啟二年超擢山東僉事。七年代閻鳴泰總督薊、遼、保定軍務。尋進兵部尚書,加太子太保。詔嗜利無恥,父事忠賢。釋褐九年,驟至極品。建四祠祀忠賢。忠賢敗,僅罷官聽勘。御史高弘圖言:「傾危社稷,搖動宮闈,如詔及劉志選、梁夢環三賊者,罪實浮於『五虎』『五彪』,而天討未加。且詔建祠薊州,迎忠賢像,五拜三稽首,呼九千歲。及聞先帝彌留,詔即整兵三千,易置將領,用崔呈秀所親蕭惟中主郵騎,直接都門,此其意何為。」由是三人皆被逮,論死。

邵輔忠,定海人。萬曆二十三年進士。為工部郎中,首劾李三才貪險假橫四大罪。尋謝病去,久之起故官。天啟五年附忠賢,驟遷至兵部尚書,視侍郎事。諸奸黨攻擊正人,多其所主使。七年三月護桂王之籓衡州,加太子太保。還朝,時事已變,移疾歸。尋麗逆案,贖徒為民。

孫傑,錢塘人。萬曆四十一年進士。官刑科右給事中,以附忠賢劾劉一燝、周嘉謨,為清議所棄。出為江西參議,引疾歸。忠賢召為大理丞,累擢工部右侍郎。大學士馮銓由李魯生、李蕃擁戴為首輔,素與崔呈秀璫。而傑與霍維華以呈秀最得忠賢懽,欲令入閣,謀之吳淳夫等,先擊去銓。又恐王紹徽為吏部,不肯推呈秀,令袁鯨疏攻紹徽,而龔萃肅上閣臣內外兼用疏以堅之。自是,魯生、蕃與傑等分途,其黨日相輒矣。傑官亦至尚書,加少保。忠賢誅,傑被劾罷,名麗逆案,贖徒三年。輔忠、傑本謀搖中宮,而事發於志選、夢環,故得輕論云。

曹欽程,江西德化人。舉進士。授吳江知縣,贓污狼籍,以淫刑博強項聲。巡撫周起元劾之,貶秩,改順天教授,調國子助教。諂附汪文言,得為工部主事。及文言敗,欽程力擠之,由座主馮銓父事魏忠賢,為「十狗」之一。銓欲害御史張慎言、周宗建,令李魯生草疏,屬欽程上之,因及李應昇、黃尊素,而薦魯生及傅櫆、陳九疇、張訥、李蕃、李恒茂、梁夢環輩十餘人。慎言等四人並削籍。欽程於群小中尤無恥,日夜走忠賢門,卑諂無所不至,同類頗羞稱之。欽程顧驕眾人以忠賢親己。給事中吳國華劾之,忠賢怒,除國華名,欽程益得志。給事中楊所修緣忠賢指,力薦其賢,遂由員外郎擢太僕少卿。後忠賢亦厭之,六年正月為給事中潘士聞所劾。忠賢責以敗群,削其籍。瀕行猶頓首忠賢前曰:「君臣之義已絕,父子之恩難忘。」絮泣而去。忠賢誅,入逆案首等,論死。繫獄久之,家人不復饋食,欽程掠他囚餘食,日醉飽。李自成陷京師,欽程首破獄出降。自成敗,隨之西走,不知所終。福王時,定從賊案,欽程復列首等。

當忠賢盛時,其黨爭搏擊清流,獻諂希寵。最著者,石三畏、張訥、盧承欽、門克新、劉徽、智鋌。

三畏,交河人。知文登、曹二縣,大著貪聲。以御史陳九疇薦,得行取。趙南星秉銓,出為王府長史。故事,外吏行取無為王官者,三畏以是大恨。及忠賢得志,三畏諂附之,遂授御史。首劾都給事中劉弘化護熊廷弼,太僕卿吳炯黨顧憲成,兩人獲嚴譴。追論京察三變,力詆李三才、王圖、孫丕揚、曹于汴、湯兆京、王宗賢、顧憲成、胡忻、王元翰、王淑抃、趙南星、張問達、王允成、塗一榛、王象春等十五人,而薦喬應甲、徐兆魁等十三人。於是三才等生者除名,死者追奪。已,極論三案,請以其疏付史館,而劾禮部侍郎周炳謨、南京尚書沈儆炌、大理丞張廷拱,三人亦獲譴。三畏為忠賢「十孩兒」之一。又倚呈秀為薦主,鍛成楊、左之獄,咆哮特甚。一日,赴戚畹宴,魏良卿在焉。三畏醉,誤令優人演《劉瑾酗酒》一劇。忠賢聞,大怒,削籍歸。忠賢殛,借忤廕名,起故官,為南京御史朱純所劾,罷去。

訥,閬中人。由行人擢御史,承忠賢指,首劾趙南星十大罪,並及御史王允成,吏部郎鄒維璉、程國祥、夏嘉遇。忠賢大喜,立除南星等名,且令再奏。乃羅織兵部侍郎李邦華,湖廣巡撫孫鼎相,舊給事中毛士龍、魏大中,光祿少卿史記事等十七人,誣以賄南星得官,諸人並獲罪。尋請毀東林、關中、江右、徽州諸書院。痛詆鄒元標、馮從吾、餘懋衡、孫慎行並及侍郎鄭三俊、畢懋良等,亦坐削奪。復劾罷江西巡撫韓光祐。訥為忠賢鷹犬,前後搏擊用力多。忠賢深德之,用其兄太僕少卿樸至南京戶部尚書,加太子太保。樸官宣大總督,為忠賢建四祠。兄弟並入逆案。

承欽,餘姚人。由中書舍人擢御史,首劾罷戶部侍郎孫居相等,因言:「東林自顧憲成、李三才、趙南星而外,如王圖、高攀龍等謂之副帥,曹于汴、湯兆京、史記事、魏大中、袁化中謂之『先鋒』,丁元薦、沈正宗、李朴、賀烺謂之『敢死軍人』,孫丕揚、鄒元標謂之『土木魔神。』請以黨人姓名、罪狀榜示海內。」忠賢大喜,敕所司刊籍,凡黨人已罪未罪者,悉編名其中。承欽官至太僕少卿卒。

克新,汝陽人。由青州推官擢御史,劾右庶子葉燦、光祿卿錢春、按察使張光縉倚傍門戶,且請速誅熊廷弼。忠賢大喜,立傳旨行刑。以閣臣固爭,乃令俟秋後,而除燦等名。御史吳裕中,廷弼姻也,憤曰:「廷弼已死人,何必疏促。」與克新絕,逆黨由此銜之。廷弼之禍,大學士丁紹軾有力焉。馮銓因使人嗾裕中劾紹軾,而先報忠賢曰:「裕中必為廷弼報仇。」裕中疏上,遂命於午門杖之百,舁至家死。魏廣微將謝政,克新言:「廣微砥柱狂瀾,厥功甚偉,宜錫之溫綸,優以禮數。」以是稍失忠賢意。太倉人孫文豸,與同里武進士顧同寅嘗客廷弼所。廷弼死,文豸為詩誄之,同寅題尺牘亦有追惜語,為邏卒所獲。克新遽以誹謗聞,兩人遂棄市,連及同郡編修陳仁錫、故修撰文震孟,並削籍。克新尋巡按山東,崇禎初,引疾去。

徽,清苑人。由臨淮知縣擢御史。陳朝輔劾馮銓,徽出疏繼之,且曰:「臣與銓同鄉,痛惡群小之誤銓,不忍銓坐失燕、趙本色。」聞者笑之。出督遼餉,乾沒不貲。初,梁夢環巡關,誣熊廷弼侵盜軍貲十七萬。徽言:「廷弼原領帑金三十萬,茫無所歸。其家貲不下百萬,而僅以十七萬還公家,何以申國法?」因誣給事中劉弘化、毛士龍,御史樊尚燝、房可壯贓賄事。忠賢喜,削弘化等籍,敕所司徵廷弼贓。尋加徽太僕少卿,先後頌忠賢至十一疏。忠賢敗,被劾回籍。

鋌,元氏人。舉鄉試,受業趙南星門,授知縣。由魏廣微通於忠賢,得擢御史,遂疏詆南星為元惡。先後劾罷禮部侍郎徐光啟等。鋌以乙榜起家,欲得忠賢歡,搏擊彌銳。忠賢大喜,加太僕少卿,以憂歸。崇禎初,禮部主事喬若雯劾鋌及陳九疇、張訥為魏廣微爪牙,詔奪職。後與三畏、訥、承欽、克新、徽並入逆案,訥遣戍,三畏等論徒。

當忠賢橫時,宵小希進乾寵,皆陷善類以自媒。始所擊皆東林也,其後凡所欲去者,悉誣以東林而逐之。自四年十月迄熹宗崩,斃詔獄者十餘人,下獄謫戍者數十人,削奪者三百餘人,他革職貶黜者不可勝計。

王紹徽,咸寧人,尚書用賓從孫也。舉萬曆二十六年進士。授鄒平知縣,擢戶科給事中。居官強執,頗以清操聞。湯賓尹號召黨與,圖柄用。吏部尚書孫丕揚以紹徽其門生,用年例出為山東參議,紹徽辭疾不就。泰昌時,起通政參議,遷太僕少卿,被劾引疾。尋以拾遺罷。

天啟四年冬,魏忠賢既逐去左光斗,即召紹徽代為左僉都御史。明年六月進左副都御史。尋進戶部侍郎,督倉場,甫視事,改左都御史。十二月拜吏部尚書。忠賢為從子良卿求世封,紹徽即為奏請良卿封伯。請推崇其三世,紹徽亦議如其言。至忠賢遣內臣出鎮,紹徽乃偕同官陳四不可。王恭廠、朝天宮並災,紹徽言誅罰過多。忤忠賢意,得譙讓。已復上言:「四方多事,九邊缺饟,難免催科,乞定分數,寬年限,以緩急之宜付撫按。正殿既成,兩殿宜緩,請敕工部裁省織造、瓷器諸冗費,用佐大工。奸黨削除已盡,恐藏禍蓄怨,反受中傷。逮繫重刑,加於封疆、顯過、三案巨奸,則人心悅服,餘宜少寬貸。」復忤忠賢意。

初,紹徽在萬曆朝,素以排擊東林為其黨所推,故忠賢首用居要地。紹徽仿民間《水滸傳》,編東林一百八人為《點將錄》,獻之,令按名黜汰,以是益為忠賢所喜。既而奸黨轉盛,後進者求速化,妒諸人妨己,擬次第逐之。孫傑乃謀使崔呈秀入閣,先擊去紹徽,令御史袁鯨、張文熙詆紹徽朋比。鯨再疏列其鬻官穢狀,遂落紹徽職,而以周應秋代。逆案既定,紹徽削籍論徒。

應秋,金壇人。萬曆中進士。歷官工部侍郎,生平無持操。天啟三年避東林謝病去。明年冬,魏忠賢起為南京刑部左侍郎。五年召拜刑部添註尚書。時忠賢廣樹私人,悉餌以顯爵,故兩京大僚多添註。尋改左都御史。家善烹飪,每魏良卿過,進豚蹄留飲,良卿大歡,時號「煨蹄總憲」。明年七月代紹徽為吏部尚書,與文選郎李夔龍鬻官分賄。清流未盡逐者,應秋毛舉細故,削奪無虛日。忠賢門下有「十狗」,應秋其首也。冒三殿功,屢加太子太師。初,楊漣等拷死,應秋夜半叩戶,語其館客曰:「天眼開,楊漣、左光斗死矣。」莊烈帝嗣位,被劾歸。已,入逆案,遣戍死。弟維持。天啟中為御史,請刊黨籍,盡毀天下書院。俄劾兵部尚書趙彥等,並削籍。以兄應秋在位,引嫌歸。崇禎初,起按浙江,被劾罷。兄弟並麗逆案。

霍維華,東光人。萬曆四十一年進士。除金壇知縣,徵授兵科給事中。天啟元年六月,中官王安當掌司禮監印,辭疾居外邸,冀得溫旨即視事。安與魏忠賢有隙,閹人陸藎臣者,維華內弟也,偵知之以告。維華故與忠賢同郡交好,遂乘機劾安,忠賢輒矯旨殺之。劉一燝、周嘉謨咸惡維華,用年例出為陜西僉事。其同官孫傑言,維華三月兵垣無過失,一燝、嘉謨仰王安鼻息,故擯於外。忠賢大喜,立逐兩人,而維華亦以外艱歸。

四年冬,朝事大變,南京御史呂鵬雲以外轉請告。忠賢傳旨令與被察徐大化、年例外轉孫傑俱擢京卿,維華及王志道、郭興治、徐景濂、賈繼春、楊維垣並復故官。維華得刑科。諸為趙南星斥者,競起用事。維華益銳意攻東林,劾罷御史劉璞、南京御史塗世業、黃公輔、萬言揚。追論三案,痛詆劉一燝、韓爌、孫慎行、張問達、周嘉謨、王之寀、楊漣、左光斗,而譽范濟世、王志道、汪慶百、劉廷元、徐景濂、郭如楚、張捷、唐嗣美、岳駿聲、曾道唯。請改《光宗實錄》,宣其疏史館。忠賢立傳旨削一燝等五人籍,逮之寀,免李可灼戍,擢濟世巡撫、志道等京卿,嗣美以下悉起用,實錄更撰,而以閣臣言免一燝等罪。尋言,總督張我續宜罪,尚書趙彥宜去,御史方震孺不宜逮,韓敬宜復官,湯賓尹宜雪。忤忠賢意,傳旨譙責之。五年冬擢太僕少卿。明年擢本寺卿。尋擢兵部右侍郎,署部事。每陳奏,必頌忠賢。七年,延綏奏捷,進右都御史,蔭子錦衣千戶。寧、錦敘功,進兵部尚書,視侍郎事,廕子如之。俄敘三殿功,加太子太保。

維華性憸邪,與崔呈秀為忠賢謀主。所親為近侍,宮禁事皆預知,因進仙方靈露飲。帝初甚甘之,已漸厭。及得疾,體腫,忠賢頗以咎維華。維華甚懼,而慮有後患,欲先自貳於忠賢,乃力辭寧、錦恩命,讓功袁崇煥,乞以己廕授之。忠賢覺其意,降旨頗厲。無何,熹宗崩,忠賢敗,維華與楊維垣等彌縫百方。其年十月,以兵部尚書協理戎政。

崇禎改元,附璫者多罷去,維華自如。遼東督師王之臣免,代者袁崇煥未至,維華謀行邊自固。帝已可之,給事中顏繼祖極論其罪,言「維華狡人也,璫熾則借璫,璫敗則攻璫。擊楊、左者,維華也。楊、左逮,而陽為救者,亦維華也。以一給事中,三年躐至尚書,無敘不及,有賚必加,即維華亦難以自解。」乃寢前命。頃之,言者踵至,維華乃引退。逆案既定,維華戍徐州,氣勢猶盛。七年,駱馬湖淤,維華言於治河尚書劉榮嗣,請自宿遷抵徐州,穿渠二百餘里,引黃河水通漕,冀敘功復職。榮嗣然其計,費金錢五十餘萬,工不成,下獄論死,維華意乃沮。九年,邊事急,都御史唐世濟薦維華邊才,至,下獄遣戍。維華遂憂憤死。

福王時,楊維坦翻逆案,為維華等訟冤,章下吏部。尚書張捷重述三朝舊事,力稱維華等忠,追賜恤典。贈廕祭葬謚全者,維華及劉廷元、呂純如、楊所修、徐紹吉、徐景濂六人。贈廕祭葬不予謚者,徐大化、范濟世二人。贈官祭葬者,徐揚先、劉廷宣、岳駿聲三人。復官不賜恤者,王紹徽、徐兆魁、喬應甲三人。他若王德完、黃克纘、王永光、章光岳、徐鼎臣、徐卿伯、陸澄源,名不麗逆案,而為清議所抑者,亦賜恤有差。

徐大化,會稽人,家京師。由庶吉士改御史,以京察貶官,再起再貶,至工部主事。孫丕揚典京察,坐不謹落職。故事,大計斥退官無復起者。萬曆末,群邪用事,文選郎陸卿榮破例起之。天啟初,屢遷刑部員外郎,結魏忠賢、劉朝,為之謀主。給事中周朝瑞劾其奸貪,御史張新詔抉其閨房之隱,大化頗愧沮。已,承要人指,力詆熊廷弼。及廷弼入關,又請速誅,與朝瑞相訐,尚書王紀劾罷之。尋復罹察典,削職。四年冬,中旨起大理丞,益與魏廣微比,助忠賢為虐。疏薦邵輔忠、姚宗文、陸卿榮、郭鞏等十三人,即召用。俄遷少卿。左僉都御史楊漣等之下獄也,大化獻策於忠賢曰:「彼但坐移宮罪,則無贓可指。若坐納楊鎬、熊廷弼賄,則封疆事重,殺之有名。」忠賢大悅,從之,由是諸人皆不免。尋進左副都御史,歷工部左、右侍郎。皇極殿成,加尚書,貪恣無忌,忠賢亦厭之。七年四月那移金錢事發,遂勒閒住。後入逆案,戍死。

李蕃,日照人。與李魯生皆萬曆四十一年進士。蕃由廬江知縣入為御史,魯生亦方居垣中,皆為魏忠賢心腹。孫承宗請入朝,蕃以王敦、李懷光為比,承宗遂還鎮。朱國禎當國,不為忠賢所喜,蕃希指劾去之。同官排擊忠良,多其代草。始與魯生諂事魏廣微,廣微敗,改事馮銓,銓寵衰,又改事崔呈秀,時號兩人為四姓奴。出督畿輔學政,建祠天津、河間、真定,呼忠賢九千歲。加太僕卿,視御史事。忠賢敗,被劾罷。

魯生,霑化人,知邢臺、邯鄲、儀封、祥符四縣。擢兵科給事中,由座主廣微通於忠賢,卑污奸險,常參密謀。周起元劾朱童蒙,魯生希忠賢指,攻罷起元。時中旨頻出,朝端以為憂。魯生獨上言:「執中者帝,用中者王,旨不從中出而誰出?」舉朝大駭。內閣缺人,詔舉老成幹濟者。馮銓資淺,年未及四十,魯生、蕃欲令入閣。魯生遂上言:「成即為老,而非必老乎年。幹乃稱濟,而即有濟於國。」銓果柄用。時有「十孩兒」之號,魯生其一也。嘗薦阮大鋮、陳爾翼、張素養、李嵩、張捷輩十一人,悉其私黨。疏詆家居大學士韓爌,削其籍。主事呂下問治徽州吳養春獄,株累者數百家,知府石萬程不能堪,棄官去。魯生反劾罷萬程。遷左給事中,典試湖廣,發策詬楊漣,因歷詆屈原、宋玉等。冒寧、錦功,進太僕少卿。莊烈帝即位,魯生知禍及,疏請免漣等追贓。給事中汪始亨、顏繼祖,御史張三謨交章發其奸,始罷去。御史汪應元再劾之,乃削籍。

又有李恒茂者,邢臺人。為禮科給事中,薦呈秀復官,與深相得。劾罷侍郎扶克儉、太僕少卿孫之益、太常少卿莊欽鄰,皆不附忠賢者也。恒茂、魯生、蕃日走吏、兵二部,交通請託,時人為之語曰:「官要起,問三李。」後忽與呈秀交惡,削籍歸。忠賢敗,起故官,為御史鄒毓祚劾罷。逆案既定。魯生遣戍,蕃、恒茂贖徒為民。

閻鳴泰,清苑人。萬曆中進士。除戶部主事,屢遷遼東參政,拾遺被劾罷歸。久之,起僉事,分巡遼海。開原既失,經略熊廷弼遣撫沈陽,半道慟哭而返。尋託疾謝歸。天啟二年,起故官,監軍山海關。旋進副使,受知孫承宗,屢疏推薦,而鳴泰實無才略,工諂佞,以虛詞罔上而已。其年八月,廷推鳴泰遼東經略,會承宗自請督師,乃擢右僉都御史,巡撫遼東。自王化貞棄地後,巡撫罷不設。至是承宗以重臣當關,事權獨操,鳴泰不能有所為。明年五月復移疾去,家居三年。魏忠賢竊柄,鳴泰潛結之,用御史智鋌薦,召為兵部右侍郎。

六年正月,寧遠告警,畿輔震驚。內閣顧秉謙等以順天巡撫吳中偉非禦侮才,薦鳴泰代之。未幾,代王之臣總督薊、遼、保定軍務。寧遠敘功,進本部尚書。以繕修山海關城,進太子太傅。尋召還,協理戎政。敘錦州功,加少保。三殿成,加少師兼太子太師。熹宗崩,代崔呈秀為兵部尚書。鳴泰由忠賢再起,專事諂諛。每陳邊事,必頌功德,於薊、遼建生祠,多至七所。其頌忠賢,有「民心依歸,即天心向順」語,聞者咋舌。崇禎初,為言者劾罷。後麗逆案,遣戍死。

生祠之建,始於潘汝禎。汝禎巡撫浙江,徇機戶請,建祠西湖。六年六月疏聞於朝,詔賜名「普德」。自是,諸方效尤,幾遍天下。其年十月,孝陵衛指揮李之才建之南京。七年正月,宣大總督張樸、宣府巡撫秦士文、宣大巡按張素養建之宣府、大同,應天巡撫毛一鷺、巡按王珙建之虎丘。二月,鳴泰與順天巡撫劉詔、巡按倪文煥建之景忠山,宣大總督樸、大同巡撫王點、巡按素養又建之大同。三月,鳴泰與詔、文煥,巡按御史梁夢環建之西協密雲丫髻山,又建之昌平、通州,太僕寺卿何宗聖建之房山。四月,鳴泰與巡撫袁崇煥又建之寧前,宣大總督樸、山西巡撫曹爾禎、巡按劉弘光又建之五臺山,庶吉士李若琳建之蕃育署,工部郎中曾國禎建之盧溝橋。五月,通政司經歷孫如冽、順天府尹李春茂建之宣武門外,巡撫朱童蒙建之延綏,巡視五城御史黃憲卿、王大年、汪若極、張樞、智鋌等建之順天,戶部主事張化愚建之崇文門,武清侯李誠銘建之藥王廟,保定侯梁世勳建之五軍營大教場,登萊巡撫李嵩、山東巡撫李精白建之蓬萊閣、寧海院,督餉尚書黃運泰,保定巡撫張鳳翼、提督學政李蕃、順天巡按文煥建之河間、天津,河南巡撫郭增光、巡按鮑奇謨建之開封,上林監丞張永祚建之良牧、嘉蔬、林衡三署,博平侯郭振明等建之都督府、錦衣衛。六月,總漕尚書郭尚友建之淮安。是月,順天巡按盧承欽、山東巡按黃憲卿、順天巡按卓邁,七月,長蘆巡鹽龔萃肅、淮揚巡鹽許其孝、應天巡按宋禎漢、陜西巡按莊謙,各建之所部。八月,總河李從心、總漕尚友、山東巡撫精白、巡按黃憲卿、巡漕何可及建之濟寧,湖廣巡撫姚宗文、鄖陽撫治梁應澤、湖廣巡按溫謨建之武昌、承天、均州。三邊總督史永安。陜西巡撫胡廷晏,巡按謙、袁鯨建之固原太白山。楚王華奎建之高觀山。山西巡撫牟志夔,巡按李燦然、劉弘光建之河東。

每一祠之費,多者數十萬,少者數萬,剝民財,侵公帑,伐樹木無算。開封之建祠也,至毀民舍二千餘間,創宮殿九楹,儀如帝者。參政周鏘、祥符知縣季寓庸恣為之,巡撫增光俯首而已。鏘與魏良卿善,祠成,熹宗已崩,猶抵書良卿,為忠賢設滲金像。而都城數十里間,祠宇相望。有建之內城東街者,工部郎中葉憲祖竊歎曰:「此天子幸辟雍道也,土偶能起立乎!」忠賢聞,即削其籍。上林一苑,至建四祠。童蒙建祠延綏,用琉璃瓦。詔建祠薊州,金像用冕旒。

幾疏詞揄揚,一如頌聖,稱以「堯天帝德,至聖至神。」而閣臣輒以駢語褒答,中外若響應。運泰迎忠賢像,五拜三稽首,率文武將吏列班階下,拜稽首如初。已,詣像前,祝稱某事賴九千歲扶植,稽首謝。某月荷九千歲拔擢,又稽首謝。還就班,復稽首如初禮。運泰請以遊擊一人守祠,後建祠者必守。其孝等方建祠揚州,將上梁,而熹宗哀詔至,既哭臨,釋縗易吉,相率往拜。監生陸萬齡至謂:「孔子作《春秋》,忠賢作《要典》。孔子誅少正卯,忠賢誅東林。宜建祠國學西,與先聖並尊。」司業朱之俊輒為舉行,會熹宗崩,乃止。而華奎、誠銘輩,以籓王之尊,戚畹之貴,亦獻諂希恩,祝釐恐後。最後,巡撫楊邦憲建祠南昌,毀周、程三賢祠,益其地,鬻澹臺滅明祠,曳其像碎之。比疏至,熹宗已崩,莊烈帝且閱且笑。忠賢覺其意,具疏偽辭,帝輒報允。無何,忠賢誅,諸祠悉廢,凡建祠者概入逆案云。

賈繼春,新鄉人。萬歷三十八年進士。歷知臨汾、任丘二縣,入為御史。李選侍移噦鸞宮,一時頗逼迫,然故無恙也。繼春聽流言,上書內閣方從哲等,略言:「新君御極,首導以違忤先皇,逼逐庶母,通國痛心。昔孝宗不問昭德,先皇優遇鄭妃,何不輔上取法?且先皇彌留,面以選侍諭諸臣,而玉體未寒,愛妾莫保。忝為臣子,夫獨何心。」給事中周朝瑞駁之,繼春再揭,謂「選侍雉經,皇八妹入井」,至稱選侍為未亡人。楊漣乃上移宮始末疏,謂:「宸宮未定,先帝之社稷為重,則平日之寵愛為輕。及宸居已安,既盡臣子防危之忠,即當體聖主如天之度。臣所以請移宮者如此。而蜚語謂選侍踉蹌徒跣,屢欲自裁,皇妹失所投井。恐釀今日之疑端,流為他年之實事。」帝於是宣敕數百言,極言選侍無狀,嚴責廷臣黨庇。

時繼春出按江西,便道旋里,馳疏自明上書之故,中有「威福大權,莫聽中涓旁落」語。王安激帝怒,嚴旨切責,令陳狀。於是御史張慎言、高弘圖連章為求寬。帝益怒,下廷臣雜議。尚書周嘉謨等言:「臣等意陛下篤念聖母,不能忘選侍。及誦敕諭,知聖心自體恤。而繼春誤聽風聞,慎言等又連疏瀆奏。然意本無他,罪當宥。」未報。御史王大年、張捷、周宗建、劉廷宣,給事中王志道、倪思輝等交章論救,給事、御史復合詞為請,諸閣臣又於講筵救之,乃停慎言、弘圖、大年俸,宥志道等。既而繼春回奏,詞甚哀,且隱「雉經、入井」二語。帝嚴旨窮詰,令再陳。嘉謨等復力救,帝不許。繼春益窘,惶恐引罪,言得之風聞。乃除名永錮,時天啟元年四月也。其後言者屢請召還,帝皆不納。

四年冬,魏忠賢既逐楊漣等,即以中旨召復官。至則重述移宮事,極言:「漣與左光斗目無先皇,罪不容死。且漣因傅櫆發汪文言事,知禍及,故上劾內疏,先發制人,天地祖宗所必殛。而止坐納賄結黨,則漣等當死之罪未大暴天下。宜速定爰書布中外,昭史冊,使後世知朝廷之罪漣等以不道無人臣禮也。」疏娓娓數百言,且請用楊所修言,亟修《三朝要典》,忠賢大喜。

莊烈帝即位,繼春方督學南畿,知忠賢必敗,馳疏劾崔呈秀及尚書田吉、順天巡撫單明詡、副都御史李夔龍,群小始自貳。旋由太常少卿進左僉都御史,與霍維華輩力扼正人。崇禎改元五月,給事中劉斯球極言其反覆善幻,乃自引歸。已,楊漣子之易疏訐之,詔削籍。初,繼春以移宮事詆漣結王安圖封拜,後見公議直漣,畏漣嚮用,俯首乞和,聲言疏非己意。還朝則極詆漣。及忠賢殛,又極譽高弘圖之救漣,且薦韓爌、倪元璐,以求容於清議。帝定逆案,繼春不列名,帝問故。閣臣言繼春雖反覆,持論亦可取。帝曰:「惟反覆,故為真小人。」遂引交結近侍律,坐徒三年,自恨死。

田爾耕,任丘人,兵部尚書樂孫也。用祖蔭,積官至左都督。天啟四年十月代駱思恭掌錦衣衛事。狡黠陰賊,與魏良卿為莫逆交。魏忠賢斥逐東林,數興大獄。爾耕廣布偵卒,羅織平人,鍛練嚴酷,入獄者率不得出。宵人希進者,多緣以達於忠賢,良卿復左右之,言無不納,朝士輻輳其門。魏廣微亦與締姻,時有「大兒田爾耕」之謠。又與許顯純、崔應元、楊寰、孫雲鶴有「五彪」之號。累加至少師兼太子太師,蔭錦衣世職者數人,歲時賞賚不可勝紀。顯純等加官亦如之。忠賢敗,言者交劾,下吏論死。崇禎元年六月與顯純並伏誅。

顯純,定興人,駙馬都尉從誠孫也。舉武會試,擢錦衣衛都指揮僉事。天啟四年,劉僑掌鎮撫司,治汪文言獄,失忠賢指,得罪,以顯純代之。顯純略曉文墨,性殘酷,大獄頻興,毒刑鍛練,楊漣、左光斗、周順昌、黃尊素、王之寀、夏之令等十餘人,皆死其手。諸人供狀,皆顯純自為之。每讞鞫,忠賢必遣人坐其後,謂之聽記,其人偶不至,即袖手不敢問。

應元,大興人。市井無賴,充校尉,冒緝捕功,積官至錦衣指揮。雲鶴,霸州人,為東廠理刑官。寰,吳縣人。隸籍錦衣,為東司理刑。凡顯純殺人事,皆應元等共為之。而寰為田爾耕心腹。及顯純論死,法司止當應元、雲鶴、寰戍。後定逆案,三人並論死,寰先死戍所。


\end{pinyinscope}