\article{地理志}

自黃帝畫野置監,唐、虞分州建牧,沿及三代,下逮宋、元,廢興因革此類推,萬物無所差別,「天與地卑,山與澤平」。其學說與,前史備矣。明太祖奮起淮右,首定金陵,西克湖、湘,東兼吳、會,然後遣將北伐,並山東,收河南,進取幽、燕,分軍四出,芟除秦、晉,訖於嶺表。最後削平巴、蜀,收復滇南。禹跡所奄,盡入版圖。近古以來,所未有也。

洪武初,建都江表,革元中書省,以京畿應天諸府直隸京師。後乃盡革行中書省,置十三布政使司,分領天下府州縣及羈縻諸司。又置十五都指揮使司以領衛所番漢諸軍,其邊境海疆則增置行都指揮使司,而於京師建五軍都督府,俾外都指揮使司各以其方附焉。成祖定都北京,北倚群山,東臨滄海,南面而臨天下,乃以北平為直隸,又增設貴州、交址二布政使司。仁、宣之際,南交屢叛,旋復棄之外徼。

終明之世,為直隸者二:曰京師,曰南京。為布政使司者十三:曰山東,曰山西,曰河南,曰陜西,曰四川,曰湖廣,曰浙江,曰江西,曰福建,曰廣東,曰廣西,曰雲南,曰貴州。其分統之府百有四十,州百九十有三,縣千一百三十有八。羈縻之府十有九,州四十有七,縣六。編裡六萬九千五百五十有六。而兩京都督府分統都指揮使司十有六,行都指揮使司五,曰北平、曰山西、曰陜西、曰四川、曰福建,留守司二。所屬衛四百九十有三,所二千五百九十有三,守御千戶所三百一十有五。又土官宣慰司十有一,宣撫司十,安撫司二十有二,招討司一,長官司一百六十有九,蠻夷長官司五。其邊陲要地稱重鎮者凡九:曰遼東,曰薊州,曰宣府,曰大同,曰榆林,曰寧夏,曰甘肅,曰太原,曰固原。皆分統衛所關堡,環列兵戎。綱維布置,可謂深且固矣。

計明初封略,東起朝鮮,西據吐番,南包安南,北距大磧,東西一萬一千七百五十里,南北一萬零九百四里。自成祖棄大寧,徙東勝,宣宗遷開平於獨石,世宗時復棄哈密、河套,則東起遼海,西至嘉峪,南至瓊、崖,北抵雲、朔,東西萬餘里,南北萬里。其聲教所訖,歲時納贄,而非命吏置籍,侯尉羈屬者,不在此數。嗚呼盛矣!

論者謂交趾之棄,未為失圖,而開平近遷,則守衛益薄,雖置萬全都指揮使司,不足以鎮伏山後諸部,故再傳而有土木之變。然睿皇自以失律蒙塵,非由經制未備。景帝任賢才,修守御,國步未移,乘輿旋復。由是觀之,三衛者,一隅之隘,而無關大計也審矣。至其季世,流寇首禍於西陲,浸尋蔓延,中原為之糜爛。金湯之固不足以制土崩,皈宇之廣不足以成掎角。疆圉不蹙於曩時,形勝無虧於初盛,而強弱懸殊,興亡異數者,天降喪亂,昏椓內訌,人事之乖,而非地利之失也。語曰:「在德不在險」,詎不信夫!今考其升降之差,沿革之故,具著於篇。作《地理志》。

京師南京

京師《禹貢》冀、兗、豫三州之域,元直隸中書省。洪武元年四月分屬河南、山東兩行中書省。二年三月置北平等處行中書省,治北平府。先屬山東、河南者皆復其舊。領府八,州三十七,縣一百三十六。八月置燕山都衛。與行中書省同治。八年十月改都衛為北平都指揮使司。九年六月改行中書省為承宣布政使司。永樂元年正月建北京於順天府,稱為「行在」。二月罷北平布政使司,以所領直隸北京行部;罷北平都指揮使司,以所領直隸北京留守行後軍都督府。十九年正月改北京為京師。罷北京留守行後軍都督府,直隸後軍都督府。衛所有實土者附見,無實土者不載。罷北京行部,直隸六部。洪熙初,仍稱行在。正統六年十一月罷稱行在,定為京師。府八,直隸州二,屬州十七,縣一百一十六。為里三千二百三十有奇。府州縣建置沿革,俱自元始。其沿革年月已見《元史志》者,不載。其未見《元史志》及明改元舊,並新增、新廢者,悉書。北至宣府,外為邊地。東至遼海,與山東界。南至東明,與山東、河南界。西至阜平,與山西界。洪武二十六,年編戶三十三萬四千七百九十二,口一百九十二萬六千五百九十五。

順天府元大都路,直隸中書省。洪武元年八月改為北平府。十月屬山東行省。二年三月改屬北平。三年四月建燕王府。永樂元年正月升為北京,改府為順天府。永樂四年閏七月詔建北京宮殿,修城垣。十九年正月告成。宮城周六里一十六步,亦曰紫禁城。門八:正南第一重曰承天,第二重曰端門,第三重曰午門,東曰東華,西曰西華,北曰玄武。宮城之外為皇城,周一十八里有奇。門六:正南曰大明,東曰東安,西曰西安,北曰北安,大明門東轉曰長安左,西轉曰長安右。皇城之外曰京城,周四十五里。門九:正南曰麗正,正統初改曰正陽;南之左曰文明,後曰崇文;南之右曰順城,後曰宣武;東之南曰齊化,後曰朝陽;東之北曰東直;西之南曰平則,後曰阜成;西之北曰彰儀,後曰西直;北之東曰安定;北之西曰德勝。嘉靖三十二年築重城,包京城之南,轉抱東西角樓,長二十八里。門七:正南曰永定,南之左為左安,南之右為右安,東曰廣渠,東之北曰東便,西曰廣寧,西之北曰西便。領州五,縣二十二。弘治四年編戶一十萬五百一十八,口六十六萬九千三十三。萬曆六年,戶一十萬一千一百三十四,口七十萬六千八百六十一。

大興倚。東南有大通河,亦日通惠河,水自玉河出,繞都城東南,下流至高麗莊,入白河,即元運河也。又有玉河,源自玉泉山,流經大內,出都城東南,注大通河。

宛平倚。西山在西。有桑乾河出山西馬邑縣,流千里入京師宛平縣境。出盧溝橋下,又東南分為二:一至通州,入白河;一至武清小直沽,合衛河,入於海。又有沙河、高梁河、清河,皆在西北。西又有沿河口守禦千戶所,有盧溝、王平口、石港口、齊家莊四巡檢司。

良鄉府西南。有琉璃河,即古聖水,下流入澱。北有天津關。

固安府西南。元固安州。洪武元年十二月降為縣。西南有拒馬河,即淶水。源自代郡,下流合易水為白溝,入三角澱。

永清府南。南有拒馬河。

東安府東南。元東安州,治在西。洪武元年十二月降為縣。三年徙今治。南有鳳河,即桑乾分流,南入三角澱。

香河府東南。元屬漷州。洪武十年二月省入州。十三年二月復置,改屬府。西有板罾口河,源出通州東之孤山,經縣界,入於白河。

通州洪武初,以州治潞縣省入。西有通惠河,西南有渾河,即桑乾,至州東張家灣,俱合於白河。有張家灣巡檢司。西南有弘仁橋巡檢司。西距府四十里。領縣四:

三河州東。北有泃河。又西有洳河,西南有鮑丘河,一名矣榆河,即東潞水,俱流入於泃河。西有泥窪鋪巡檢司,後移於夏店鋪。

武清州南。元屬漷州。洪武十二年來屬。有三角澱,在縣南,即古之雍奴,周二百餘里,諸水所聚。有直沽,在縣東南,衛河、白河、丁字沽合流於此入海。有巡檢司。又東北有河西務、東南有楊村二巡檢司。

漷縣州南。元漷州。洪武十四年二月降為縣來屬。有漷河,一名新河,東入於白河,即盧溝之下流。

寶坻州東南。元直隸大都路。洪武十年二月來屬。東有潮河。南有泃河。又縣東南有梁城守禦千戶所,建文二年,燕王置。有蘆臺巡檢司。

霸州洪武初,以州治益津縣省入。拒馬河舊在北,後徙治南。又南有沙河。東有苑家口巡檢司。北距府二百十里。領縣三:

文安州南少東。西有易水。東北有得勝、火燒等澱。

大城州東南。東北有黃汊河,源自交河,分流至縣境,入三角澱。

保定州南少西。洪武七年九月省入霸州。十三年十一月復置。玉帶河在北,東流入會通河。西南有磁河,東南與玉帶河合。

涿州洪武初以州治范陽縣省入。西有獨鹿山。北有涿水,西北有挾河,合焉。南有范水。東北距府百四十里。領縣一。

房山州北,少西。西有大房山。北有大安山。西南有青龍潭,其下流為挾河,一名韓村河,至涿州與胡良河合。北有磁家務巡檢司。

昌平州元昌平縣,直隸大都路。正德元年七月升為州,旋罷。八年復升為州。舊治白浮圖城,景泰元年築永安城於東,三年遷縣治焉。北有天壽山,成祖以下陵寢咸在。東南有白浮山。西南有駐蹕山。又南有榆河,一名溫餘河,下流為沙河,入於白河。又東南有鞏華城,嘉靖十九年築。東北有黃花鎮。弘治中,置渤海守禦千戶所於此,萬歷元年移於慕田峪,四年復故。西有鎮邊城,又有常峪城,俱正德十年五月築,各置守御千戶所。又有白陽守禦千戶所,亦正德中置。西北有居庸關。南距府九十里。領縣三:

順義州東少南。元順州。洪武元年十二月改為順義縣,屬府。正德元年七月來屬。東有白河,西南有榆河,又有潮河,俱流入焉。

懷柔州東北。洪武元年十一月省入檀州。十二月復分密雲、昌平二縣地置,屬府。正德元年七月來屬。東有黍谷山。西有白河。

密雲州東北。元檀州,後置縣,為州治。洪武元年十一月省縣入州。十二月復置縣,省州入焉,屬府。正德元年七月來屬。南有白檀山。西有白河。東有潮河。北有古北口,洪武十二年九月置守禦千戶所於此。三十年改為密雲後衛。又有石塘嶺、牆子嶺等關。

薊州洪武初,以州治漁陽縣省入。西北有盤山。東北有崆峒山。又泃水在北,沽河在南。州北有黃崖峪、寬佃峪等關。東又有石門鎮。西距府二百里。領縣四:

玉田州東南。東北有無終山,又有徐無山。又東有梨河。北有浭水。東南有興州左屯衛,永樂元年自故開平境移置於此。

豐潤州東南。南有沙河。西南有浭水。

遵化州東。東北有五峰山。南有靈靈山及龍門峽。又東有灤河。西南有梨河。北有喜峰口、馬蘭峪、松亭等關。

平谷州西北。洪武十年二月省入三河縣。十三年十一月復置。東南有泃河,又有洳河。西北有營州中屯衛,永樂元年自故龍山縣移置於此。又東有黃松峪關,與密云縣將軍石關相接。

保定府元保定路,直隸中書省。洪武元年九月為府。十月屬河南分省。二年三月來屬。領州三,縣十七。東北距京師三百五十里。弘治四年編戶五萬六百三十九,口五十八萬二千四百八十二。萬曆六年,戶四萬五千七百一十三,口五十二萬五千八十三。

清苑倚。北有徐河,一名大冊水,自滿城經縣北至安州,東入澱。又西有清苑河。又南有張登巡檢司,嘉靖十三年自滿城縣方順橋移置於此。

滿城府西少北。洪武十年五月省入慶都縣。十三年十一月復置。北有徐河。南有方順河。

安肅府北少東。元安肅州。洪武二年七月降為縣。易水在北。曹河在南。徐河在西。西南又有鮑河。又西有遂州,元屬保定路。洪武初降為縣。八年二月省。

定興府北少東。元屬易州。洪武六年五月改屬府。西有拒馬河,即淶水也。又易水自西來,合焉,謂之白溝河。南有河陽巡檢司,後移於清苑縣界之固城鎮。

新城府東北。元屬雄州。洪武初屬北平府。六年五月改屬府。南有白溝河。西南有巨河鎮巡檢司。

雄府東北。元雄州。洪武二年七月省州治歸信縣入焉。七年四月降為縣。北有白溝河。南有瓦濟河。

容城府東北。元屬雄州。洪武七年四月省入州。十三年十一月復置,來屬。舊治在拒馬河南,景泰二年遷於河北。西有易水,又有濡水。

唐府西,少南。西北有大茂山,即恆岳也,東麓有嶽嶺口巡檢司。又唐河在西,源出恒山,流經定州曰滱水,下流合於南易水。又西北有倒馬關,有巡檢司,後移於縣西之橫河口。又有周家鋪、軍城鎮二巡檢司。

慶都府西南,南有唐河。北有祁水。

博野府南。舊治在今蠡縣界,直隸保定路。洪武元年從今治,改屬祁州。六年五月還屬府。西北有博水。南有唐河,亦曰滱水。又有永安鎮巡檢司,有鐵燈盞巡檢司。

蠡府南少東。元蠡州,屬真定路。洪武二年七月來屬。八年正月降為縣。楊村河在南,滋、沙、唐三河之下流也,俗亦謂之唐河。

完府西。元完州。洪武二年七月降為縣。西有伊祁山,祁水出焉,其下流為方順河。

祁州洪武二年七月以州治蒲陰縣省入。北有唐河,西南有滋河,至州東南合沙河,流入易水。北距府百二十里。領縣二:

深澤州南少西。西有滋河。

束鹿州東南。北有故城。今治,天啟二年所徙。滹沱河在南。又南有百天口巡檢司。

安州洪武二年七月以州治葛城縣省入。七年降為縣。十三年十一月復升為州。北有易水,府境九河之水所匯也,下流至雄縣南,為瓦濟河。西距府七十里。領縣二:

高陽州南。元屬安州,洪武六年五月改屬府。尋屬蠡州。八年正月省入蠡縣。十三年十一月復置,還屬。故城在東,洪武三年圮於水,遷於今治。東有馬家河,其上流為蠡縣之楊村河。

新安州東少北。元直隸保定路。洪武七年七月省入安州。十三年十一月復置,來屬。西有長流河,一名長溝河,其上源為鮑河。南有曹河,又有徐河,經縣南,合流為溫義河,又南與長流河合,又東南入於瓦濟河。

易州洪武初,以州治易縣省入。西南有五迴山,雷溪出焉,徐河之上源也。西北有窮獨山,濡水所出。又南有易水,出州境之西山,與濡水並東流,而為白溝河,所謂北易水也。又有雹水,一名鮑河,出縣西南,東南流為長流河,所謂南易水也。西有紫荊關,洪武中置守禦千戶所於此。又有安座嶺、五迴嶺、金陂鎮、奇峰口、塔崖口五巡檢司。南距府百二十里。領縣一:

淶水州東北。東有淶水,亦曰拒馬河,源出山西代郡,下流合易水。北有乾河口、西北有黃兒莊二巡檢司。

河間府元河間路,直隸中書省。洪武元年十月為府,屬河南分省。二年三月來屬。領州二,縣十六。北距京師四百十里。弘治四年編戶四萬二千五百四十八,口三十七萬八千六百五十八。萬曆六年,戶四萬五千二十四,口四十一萬九千一百五十二。

河間倚。西南有滹沱河。西有滱水。西南有景和鎮巡檢司。

獻府南。元獻州。洪武初,省州治樂壽縣入焉。八年四月降為縣。有滹沱河自代郡流入境,經縣南,至青縣合衛河達於海。有單家橋巡檢司。

阜城府南。元屬景州。洪武七年改屬府。西北有胡盧河,即《禹貢》衡漳水。

肅寧府西。中堡河在縣東。

任丘府北少西。元屬莫州。洪武七年改屬府。西北有瓦濟河,下流為五官澱,注於滹沱河。北有莫州,元治莫亭縣,屬河間路。洪武七年七月,州縣俱省。

交河府東南。元屬獻州。洪武八年四月改屬府。十年五月省入獻縣。十三年十一月復置。東有衛河,源自衛輝,流入故城境,經縣東,過滄州,又東北至直沽入海,一名御河。又西北有高河,經縣南,合滹沱,謂之交河,下流入衛縣,以此名。又南有洚河。又東有泊頭鎮巡檢司。

青府東北。元清州。洪武初,以州治會川縣省入。八年四月降為縣,尋改清為青。滹沱河自縣南流入衛,謂之岔河口。其支流經縣之北者,曰獨流河。

興濟元屬清州。洪武初省。十三年復置,屬府。衛河在城西。

靜海府東北。元曰靖海,屬清州。洪武初,更名。八年四月改屬北平府。十年五月來屬。縣北有小直沽,衛河自西來,與白河合,入於海。又有丁字沽、堿水沽。又北有天津衛,永樂二年十一月置。

寧津府東南。南有土河,自山東德州流入,又東入山東樂陵縣界。

景州洪武初,以州治蓚縣省入。東有衛河。東北有胡盧河。又東有安陵、西北有宋門二巡檢司。又東北有李晏鎮。西北距府百八十五里。領縣三:

吳橋州東,少南。西有衛河。

東光州東北。洪武七年七月省入阜城縣。十三年十一月復置。南有衛河,又有胡盧河。

故城州南少西。有衛河,自山東武城縣流入境。又西南有索盧枯河。

滄州洪武初,以州治清池縣省入。舊治在東南。洪武二年五月徙於長蘆,即今治也。東濱海。西有衛河。南有浮河。北有長蘆巡檢司。西距府百五十里。領縣三:

南皮州西南。衛河在縣西。

鹽山州東南。東濱海,產鹽。東南有鹽山。

慶雲州東南。洪武六年六月析山東樂安州北地置,來屬。

真定府元真定路,直隸中書省。洪武元年十月為府。屬河南分省。二年正月屬山東。三月來屬。領州五,縣二十七。東北距京師六百三十里。弘治四年編戶五萬九千四百三十九,口五十九萬七千六百七十三。萬曆六年,戶七萬四千七百三十八,口一百九萬三千五百三十一。

真定倚。滹沱河在城南。又北有滋河,自山西靈丘縣流入,經行唐縣之張茂村伏流不見,至府北南孟社復出,下流合於南易水。

井徑府西南。元屬廣平路威州。洪武二年來屬。東南有城山,又有甘淘河,亦名冶河。南與綿蔓水合。又故關在其西。

獲鹿府西南。西有抱犢山,有西屏山。又有蓮花山,白鹿泉出焉,東流為西河,即洨水上源也。又有土門關在西,亦曰井徑關。

元氏府西南。西北有封龍山,汦水所出,下流入胡盧河。西南有槐水,下流曰野河。

靈壽府西北。東北有衛水,源出恒山,《禹貢》「恆、衛既從」即此。俗名雷溝河,東北入於滹沱。北有叉頭鎮巡檢司,後遷於慈峪鎮。

槁城府東南。北有滹沱河,又有滋河。

欒城府南,縣北有故城,今治洪武初所徙。西有洨河。

無極府東少北。元屬中山府。洪武初廢。四年七月復置,屬定州。七年四月改屬府。南有滋河。

平山府西少北。北有滹沱河,東北有冶河入焉。西北有房山。西有十八盤、下口村巡檢司。

阜平府西北。東北有大茂山。北有派河。西有龍泉關。

行唐府北。元屬保定路。洪武二年屬定州,正統十三年十月直隸真定府。西有滋河。西北有兩嶺口巡檢司。

定州元中山府。洪武二年正月改曰定州。三年以州治安喜縣省入。水寇水在北,沙河在南,下流合於滱水。西北有倒馬關守御千戶所。景泰二年置關,與紫荊、居庸為內三關。北有清風店巡檢司。西南距府百三十里。領縣二:

新樂州西南。西南有沙河。

曲陽州西北。元屬保定路。洪武二年來屬。恆山在西北,恆水出焉。又沙河在南,自山西繁峙縣流入。

冀州洪武二年以州治信都縣省入。西北有漳水。北有滹沱河。成化十八年,滹沱挾漳南注為州患。正德十二年,二水自寧晉縣南北流,患始息。又北有洚水,一名枯洚,下流合於漳。西北距府二百八十里。領縣四:

南宮州南少西。故城在縣西北,成化十六年遷於今治。漳水在北。洚水在南。東南有董家廟堡巡檢司。

新河州西少南。有清水河,成化後堙。

棗強州東少北。西北有索盧水,乃衛河之支流也,亦曰黃盧河。

武邑州東北。西有洚水。西北有漳水。

晉州洪武二年以州治鼓城縣省入。南有滹沱河。西距府九十里。領縣三:

安平州東北。滹沱河舊在縣南,萬曆二十三年自束鹿縣南行,始不經縣境。

饒陽州東北。北有滹沱河。西南有饒河,即滹沱河支流也。

武強州東。漳河在縣東。又南有滹沱河,舊合於漳,萬曆二十六年北出饒陽縣境,而縣之滹沱河始涸。

趙州洪武元年以州治平棘縣省入。南有洨河,下流入於胡盧河。北距府百二十里。領縣六:

柏鄉州南。東北有野河,即槐水也,下流入於胡盧河。

隆平州東南。洪武六年九月省入柏鄉縣。十三年十一月復置。東有灃水,東北與沙河合,下流入於胡盧河。沙河,亦槐水之別名也。又東北有大陸澤,亦曰廣阿,漳水所匯。

高邑州西南。北有黑水,即槐水也,流合縣南之水。

臨城州西南。南有敦輿山。西南有鐵山。西北有汦水,東經釣盤山下,與水合。

贊皇州西南。西南有贊皇山,水出焉,亦曰沙水。又城北有槐水。西北有黃沙嶺巡檢司。

寧晉州東少南。東南有胡盧河,其上流即漳水也,深、冀群川悉匯於此。東北有百尺口巡檢司。

深州洪武二年以州治靜安縣省入。南有故城,今治本吳家莊,永樂十年遷於此。滹沱河在東北。胡盧河在東南。有傅家池巡檢司,後廢。西距府二百五十里。領縣一:

衡水州南少東。故城在縣西南,永樂十三年遷於今治。西有漳水,南有洚水。又北有滹沱河,舊與漳合,成化八年北徙,不經縣界。西南有鹽池。

順德府元順德路,直隸中書省。洪武元年為府。十月屬河南分省。二年三月來屬。領縣九。距京師一千里。弘治四年編戶二萬一千六百一十四,口一十八萬一千八百二十五。萬曆六年,戶二萬七千六百三十三,口二十八萬一千九百五十七。

邢臺倚。西北有夷儀山,又有封山,一曰西山。又有黃榆嶺,上有黃榆關。又漳水在東南,自河南臨漳縣流入,下流為胡盧河,至交河縣合滹沱河,此為漳水經流也。又東南有百泉水,其下流為灃河,一名渦水,又名鴛鴦水。西有西王社巡檢司。

沙河府南。弘治四年以沙壅遷縣於西山小屯。十八年六月復還舊治。西南有磬口山,產鐵。南有沙河,亦名湡水。

南和府東少南。南有漳河,合縣西之灃河,又縣西北有汦水,蓋伏流而旁出者。

任府東北。東北有汦水。東有灃水。

內丘府北。東南有汦水。

唐山府東北。西北汦水。

平鄉府東少南。西南有漳河,西有沙河,又有洺河。東有滏陽河。萬歷三十年,漳挾滏陽河北出,會於沙、洺名諸河,而漳水之舊流益亂。

鉅鹿府東北。漳水舊在縣東,有大小二河,亦謂之新舊二河,其後北徙,不復至縣境,而二河遂成平陸。北有鉅鹿澤,即隆平縣之大陸澤也,澤畔舊有鹽泉。

廣宗府東少北。洪武十年六月省入平鄉、鉅鹿二縣。十三年十一月復置。漳水舊在西。又東有枯洚河。

廣平府元廣平路,直隸中書省。洪武元年為府。十月屬河南分省。二年三月來屬。領縣九。東北距京師千里。弘治四年編戶二萬七千七百六十四,口二十一萬二千八百四十六。萬曆六年,戶三萬一千四百二十,口二十六萬四千八百九十八。

永年倚。北有沙河。又有洺水,自河南武安縣流入。西南又有滏水,自河南臨漳縣流入,亦曰滏陽河。西有臨洺鎮巡檢司。西南又有黃龍鎮。

曲周府東北。西南有漳水。東有滏陽河。

肥鄉府東南。漳河在縣西北。

雞澤府東北。漳河在縣東。又西有洺河,又有沙河自南來合焉。

廣平府東南。北有漳河。

成安安府南。元屬磁州。洪武初廢。四年六月復置,來屬。西南有洹水,自河南臨漳縣流入,其下流合於衛河。又南有漳水,亦自河南臨漳縣流入。

威府東北。元威州。至正間,省州治洺水縣入州。洪武二年四月降為縣。漳水舊在南,洺水自西流入焉。

邯鄲府西南。元屬磁州。洪武元年來屬。西北有洺河。東有滏陽河。

清河府東北。元屬大名路。洪武六年九月來屬。東有衛河。

大名府元大名路,直隸中書省。洪武元年為府。十月屬河南分省。二年三月來屬。領州一,縣十。東北距京師千一百六十里。弘治四年編戶六萬六千二百七,口五十七萬四千九百七十二。萬曆六年,戶七萬一千一百八十,口六十九萬二千五十八。

元城倚。故城在東,洪武三十一年圮於衛河,徙此。東有沙麓山。西有漳河。北有衛河,即永濟渠也,自河南汲縣流入,下流合漳河。東北有小灘鎮巡檢司。

大名府南少東。元與元城縣同為大名府治。洪武十年五月省入魏縣。十五年二月復置。永樂九年移於今治。北有愜山,東南有衛河。

魏府西少北。舊治在縣西。洪武三年遷於此。南有魏河,又有新舊二漳河,下流俱合於衛河。

南樂府東。南有繁水,北入於衛河。

清豐府東南。元屬開州。洪武七年三月改屬府。西南有澶水,伏流至古繁水城西南,謂之繁水。

內黃府西南。元屬滑州。洪武七年三月改屬府。北有衛河。東有繁水。西有洹水。西北有回隆鎮,有回龍廟巡檢司。嘉靖三十六年,漳河決於此,入衛。

浚府西南。元浚州治在浮丘山之西。洪武二年四月降為縣,徙治於山東北之平坡。嘉靖二十九年復徙城於山巔,即今治也。東有大伾山,一名黎陽山,又名清澶山。西有衛河。北有淇水,自河南淇縣流入,經縣南,東入於衛,謂之黎水,亦謂之浚水。又西有長豐泊。西南有新鎮巡檢司。

滑府西南。元滑州。洪武二年四月省州治白馬縣入焉。七年三月降為縣。西北有衛河。東南有老岸鎮巡檢司。

開州洪武二年四月以州治水僕陽縣省入。大河故道在城南,正統十三年,河決入焉。景泰五年塞。北距府百六十里,領縣二:

長垣州南。舊治在縣東北,洪武二年以河患遷於古蒲城。南有黃河故道。東南有朱家口,正統十三年,河決於此。又南有大社口,萬曆十五年,河復決焉。又東南有大岡巡檢司,本治永豐裏,尋徙竹林,後徙大岡。

東明州北。洪武十年五月省入州及長垣縣。弘治三年九月復置,屬府。萬曆中,仍屬州,其舊治在今縣南。洪武初,徙今縣西。弘治三年始徙於今治。南有黃河,有杜勝集巡檢司。

永平府元永平路,直隸中書省。洪武二年改為平灤府。四年三月為永平府。領州一,縣五。西距京師五百五十里。弘治四年編戶二萬三千五百三十九,口二十二萬八千九百四十四。萬曆六年,戶二萬五千九十四,口二十五萬五千六百四十六。

盧龍倚。東南有陽山。西有灤河,自開平流經縣境,有漆河自北來入焉。東有肥如河,經城西入於漆。北有桃林口關。

遷安府西北。北有都山。東有灤河。又北有劉家口、冷口、青山口等關。

撫寧府東少南。舊治在陽河西,洪武六年十二月所徙。十三年又遷於兔耳山東。東南濱海。又東有榆河,又有陽河,一名洋河,俱自塞外流入,俱東南注於海。東有山海關。洪武十四年九月置山海衛於此。北有撫寧衛,永樂元年二月置。又有董家口、義院口等關。東有一片石口,一名九門水口。

昌黎府東南。西北有碣石山。東南有溟海,亦曰七里海,有黑陽河,自天津達縣之海道也。又有蒲泊,舊產鹽,置惠民鹽場於此。北有界嶺口、箭捍嶺等關。

灤州洪武二年九月以州治義豐縣省入。南濱海。東有灤河,又南有開平中屯衛,永樂元年二月自沙峪移置於此。東北距府四十里,領縣一:

樂亭州東南。南濱海。西有灤河,經縣北岳婆港分為二,東曰胡盧河,西曰定流河,各入於海。景泰中,胡盧河塞,定流河獨自入海,其水清碧,亦謂之綠洋溝。又西南有新橋海口巡檢司。萬歷四十三年移於灤州西之榛子鎮。

延慶州元龍慶州,屬大都路。洪武初,屬永平府。三年三月屬北平府,尋廢。永樂十二年三月置隆慶州,屬北京行部。十八年十一月直隸京師。隆慶元年改曰延慶州。西有阪泉山。南有八達嶺。東北有媯川,俗名清水河,下流注於桑乾河。又西南有沽河。東南有岔道口,與居庸關相接。關口有居庸關守御千戶所,洪武三年置。建文四年,燕王改為隆慶衛,隆慶元年曰延慶衛。東南又有柳溝營,隆慶初,置城於此,為防禦處。領縣一。東南距京師百八十里。弘治四年編戶一千七百八十七,口二千五百四十四。萬曆六年,戶二千七百五十五,口一萬九千二百六十七。

永寧本永寧衛,洪武十二年九月置。永樂十二年三月置縣於衛城。媯川在西。東有四海冶堡,天順八年置。西北有靖胡堡,東南有黑漢嶺堡,北有周四溝堡,俱嘉靖中置。又有劉斌堡,萬曆三十二年所置也。

保安州元屬上都路之順寧府。洪武初,廢。永樂二年閏九月置保安衛。十三年正月復置州於衛城,屬北京行部。十八年十一月直隸京師。舊州城在西南山下,景泰二年移於雷家站,即今治也。西南又有涿鹿山,涿水出焉。西北有磨笄山,亦曰雞鳴山,又有鷂兒嶺。又桑乾河在西南,自山西蔚州流入,東有媯川來入焉,謂之合和口。西有甯川,亦入於桑乾。東有東八里堡、良田屯堡、麻谷口堡,俱洪武二十五年置。南有美峪守禦千戶所,本在州西之美峪嶺,永樂十二年置。十六年二月徙於董家莊。景泰二年又移於此,與山西蔚州界。東南距京師三百里。弘治四年編戶四百四十五,口一千五百六十。萬曆六年,戶七百七十二,口六千四百四十五。

萬全都指揮使司元順寧府,屬上都路。洪武四年三月,府廢。宣德五年六月置司於此。領衛十五,蔚州、延慶左、永寧、保安四衛俱設於本州縣,守禦千戶所三,廣昌、美峪二所,亦設於本處,堡五。東南距京師三百五十里。

宣府左衛元宣德縣,為順寧府治。洪武四年,縣廢。二十六年二月置衛,屬山西行都司。二十八年四月改為宣府護衛,屬谷王府。三十五年十一月罷宣府護衛,復置,徙治保定。永樂元年二月直隸後軍都督府。宣德五年六月還故治,改屬。洪武二十四年四月建谷王府,永樂元年遷於湖廣長沙。西有灤河,源自炭山,下流入開平界。南有桑乾河,洋河東流入之。又有順聖川,延袤二百餘里,下流亦合於桑乾河。北有東西二城,其東城為順聖縣,元屬順寧府,西城為弘州,元屬大同路,洪武中俱廢。天順四年修築二城。又東北有大白陽、小白陽及龍門關等堡。東南有雞鳴驛堡。北有葛峪堡。西北有長峪口、青邊口、羊房等堡。

宣府右衛洪武二十六年二月置,與左衛同城,屬山西行都司。二十八年四月改為宣府護衛,屬谷王府。三十五年十一月罷宣府護衛,復置,徙治定州。永樂元年二月直隸後軍都督府。宣德五年六月還故治,改屬。

宣府前衛洪武二十六年置,治宣府城,屬山西行都司。永樂元年二月直,隸後軍都督府。宣德五年六月改屬。

萬全左衛元宣平縣,屬順寧府。洪武四年,縣廢。二十六年二月置衛,屬山西行都司。三十五年徙治山西蔚州。永樂元年二月徙治通州,直隸後軍都督府,尋還故治。宣德五年改屬。北有洋河,西海子自西來,流入之。又西北有沙城堡。西有會河堡。東有寧遠站堡。東距都司六十里。

萬全右衛洪武二十六年二月置,與左衛同城,屬山西行都司。三十五年徙治山西蔚州。永元元年二月徙治通州,直隸後軍都督府。二年徙治德勝堡。宣德五年改屬。北有翠屏山,又有野狐嶺。西北有西陽河,下流入灤河。東有張家口堡。西有新河口堡。北有膳房堡、上莊堡。西北有新開口、柴溝、洗馬林等堡。西南有渡口堡,又有西陽河堡。東距都司八十里。

懷安衛元懷安縣,屬興和路。洪武三年屬興和府,改屬山西大同府,尋廢。二十六年二月置衛,屬山西行都司。永樂元年二月直隸後軍都督府。宣德五年六月改屬。西北有花山。北有蕁麻嶺。南有水溝口河,東入於洋河。東北有威寧縣,元屬興和路,洪武中廢。又西有李信屯堡,嘉靖十六年置。東距都司百二十里。

保安右衛永樂十五年置於順聖川,直隸後軍都督府。十七年移治西沙城。二十年徙懷安城內。宣德五年六月改屬。

懷來衛元懷來縣,屬龍慶州。洪武二年屬永平府。三年三月屬北平府,尋廢。三十年正月置懷來守禦千戶所。永樂十五年改為懷來左衛,明年曰懷來衛,直隸後軍都督府。宣德五年六月改屬。北有螺山,或云即滏山也。東南有媯川。西有沽河。又西南有土木堡。東南有榆林堡,又有殷繁水。西北距都司百五十里。

延慶右衛本隆慶右衛,永樂二年置於居庸關北口,直隸後軍都督府。宣德五年六月來屬,徙治懷來城。隆慶元年更名。

開平衛本獨石堡,宣德五年築。六月自開平故城移衛,置於此。東有東山,韭菜川出焉,經城南,與氈帽山水合。又南有獨石水,下流合於龍門川。南有半壁店、貓兒峪等堡。東北有清泉堡。西南距都司三百里。

龍門衛宣德六年七月置於故龍門縣。東有紅石山,紅石水出焉,下流合於龍門川。西有大松山。北有洗馬嶺。西北有金家莊堡。東有三岔口堡。西距都司百二十里。

興和守禦千戶所永樂二十年自興和舊城徙宣府城內。宣德五年六月改屬。

龍門守禦千戶所宣德六年七月置於李家莊。西有西高山。東有白河。北有牧馬堡。東有寧遠堡。東北有長伸地、滴水涯等堡。東南有樣田堡。西南距都司二百四十里。

長安嶺堡永樂九年置。弘治二年置守禦千戶所於此。有長安嶺,名槍桿嶺。西北有鷹窩山泉。西南距都司一百四十里。

雕鶚堡宣德五年六月置。北有浩門嶺。南有南河,下流入於白河。西南距都司一百七十里。

赤城堡宣德五年六月置。東有赤城山,又有東河,即通州白河之上源也,又有西河,合焉。西北有鎮寧堡,弘治十一年置。西南距都司二百里。

雲州堡元雲州,屬上都路。洪武三年七月屬北平府。五年七月廢。宣德五年六月置堡。景泰五年置新軍千戶所於此。東北有龍門山,亦曰龍門峽,下為龍門川。又北有灤河。東北有金蓮川。西北有鴛鴦泊。又金蓮川東有鎮安堡,成化八年置。西南距都司二百十里。

馬營堡宣德七年置。西北有冠帽山。南有灤河。又西北有君子堡。西有松樹堡。東南有倉上堡。西南距都司二百里。

北平行都指揮使司本大寧都指揮使司,洪武二十年九月置。治大寧衛。二十一年七月更名。領衛十。永樂元年三月復故名,僑治保定府,而其地遂虛。景泰四年,泰寧等三衛乞居大寧廢城,不許,令去塞二百里外居住。天順後,遂入於三衛。西南距北平布政司八百里。

大寧衛元大寧路,治大定縣,屬遼陽行省。洪武十三年為府,屬北平布政司,尋廢。二十年八月置衛。九月分置左、右、中三衛,尋又置前、後二衛。二十八年四月改左、右、後三衛為營州左、右、中三護衛。永樂元年二月省,又徙中、前二衛於京師,直隸後軍都督府。洪武二十四年四月,寧王府建於此,永樂元年遷於江西南昌。南有土河。東南有大堿場。東北有惠和縣,又有武平縣。東有和眾縣。元俱屬大寧路,洪武中俱廢。

新城衛洪武二十年九月置。永樂元年廢。距行都司六十里。

富峪衛本富峪守禦千戶所。洪武二十二年二月置。二十四年五月改為衛。永樂元年二月徙置京師,直隸後軍都督府。距行都司一百二十里。

會州衛洪武二十年九月置。永樂元年廢。南有冷嶺。西北有馬孟山,廣袤千里,土河之源出焉,下流合於漌河,又南入於遼水。距行都司里。

榆木衛洪武二十年九月置。永樂元年廢。距行都司里。

全寧衛元全寧路,直隸中書省。洪武中廢。二十二年四月置衛。永樂元年廢。有潢河,又有黑龍江。西南距行都司二百里。

營州左屯衛洪武二十六年二月置。永樂元年三月徙治順義縣,屬大寧都司。南有塔山。距行都司里。

營州右屯衛元建州,屬大寧路。洪武中,州廢。二十六年二月置此衛。永樂元年三月徙治薊州,屬大寧都司。西北距行都司四百里。

營州中屯衛元龍山縣,屬大寧路。洪武中,縣廢。二十六年二月置此衛。永樂元年三月徙治平谷縣西,屬大寧都司。南有榆河。距行都司里。

營州前屯衛元興州,屬上都路。洪武三年七月屬北平府。五年七月廢。二十六年置此衛。永樂元年三月徙治香河縣,屬大寧都司。西有新開嶺。南有老河,源出馬孟山,流經此,又經行都司城南,東北入於潢河。西南有興安縣,元屬興州,順帝後至元五年四月廢。距行都司里。

營州後屯衛洪武二十五年八月置。永樂元年三月徙治三河縣,屬大寧都司。距行都司里。

興州左屯衛洪武中置。永樂元年二月徙治玉田縣,直隸後軍都督府。距行都司里。

興州右屯衛洪武中置。永樂元年二月徙治遷安縣,直隸後軍都督府。距行都司里。

興州中屯衛洪武中置。永樂元年二月徙治良鄉縣,直隸後軍都督府。距行都司里。

興州前屯衛洪武中置。永樂元年二月徙治豐潤縣,直隸後軍都督府。距行都司里。

興州後屯衛洪武中置。永樂元年二月徙治三河縣,直隸後軍都督府。距行都司里。

開平衛元上都路,直隸中書省。洪武二年為府,屬北平行省,尋廢府置衛,屬北平都司。永樂元年二月徙衛治京師,直隸後軍都督府。四年二月還舊治。宣德五年遷治獨石堡,改屬萬全都司,而令兵分班哨備於此,後廢。西北有臥龍山。南有南屏山,又有灤河。東北有香河,又有簸箕河、閭河,西南有兔兒河,下流俱合於灤河。又東有涼亭、沈阿、賽峰、黃崖四驛,路接大寧、古北口;西有桓州、威虜、明安、隰寧四驛,路接獨石。俱洪武中置,宣德後廢。又西北有寧昌路,東北有應昌路,北有泰寧路,又有德寧路,元俱直隸中書省。西有桓州,元屬上都路。洪武中皆廢。距北平都司里。

開平左屯衛洪武二十九年八月置於七合營。永樂元年廢。距都司里。

開平右屯衛洪武二十九年置於軍臺。永樂元年廢。距北平都司里。

開平中屯衛洪武二十九年置於沙峪。永樂元年二月徙治真定府,直隸後軍都督府。尋徙治灤州西石城廢縣。距都司里。

開平前屯衛洪武二十九年八月置於偏嶺。永樂元年廢。距北平都司里。

開平後屯衛洪武二十九年八月置於石塔。永樂元年廢。距北平都司里。

興和守禦千戶所元隆興路,直隸中書省。皇慶元年十月改為興和路。洪武三年為府,屬北平布政司。四年後,府廢。三十年正月置所。永樂元年二月直隸後軍都督府。二十年為阿魯台所攻,徙治宣府衛城,而所地遂虛。東北有凌霄峰。南有威遠川。西有魚兒濼。又西有集寧路,元直隸中書省。西北有寶昌州,元屬興和路。又有高原縣,元為興和路治。洪武中俱廢。距北平都司里。

寬河守禦千戶所洪武二十二年二月置。永樂元年二月徙治遵化縣,仍屬大寧都司。又僑置寬河衛於京師,直隸後軍都督府。東南有寬河,一名豹河,下流經遷安縣西北,又東合於灤河。距北平都司里。

宜興守禦千戶所元宜興縣,屬興州。致和元年八月升為宜興州。洪武二年兼置衛,屬永平府。三年三月屬北平府。六月改衛為守禦千戶所。五年七月,州廢,存所。永樂元年,所廢。距北平都司里。

南京《禹貢》揚、徐、豫三州之域。元以江北地屬河南江北等處行中書省,又分置淮東道宣慰使司治揚州路屬焉;江南地屬江浙等處行中書省。明太祖丙申年七月置江南行中書省。治應天府。洪武元年八月建南京,罷行中書省,以應天等府直隸中書省,衛所直隸大都督府。十一年正月改南京為京師。十三年正月己亥罷中書省,以所領直隸六部。癸卯改大都督府為五軍都督府,以所領直隸中軍都督府。永樂元年正月仍稱南京。統府十四,直隸州四,屬州十七,縣九十有七。為里萬三千七百四十有奇。北至豐、沛,與山東、河南界。西至英山,與河南、湖廣界。南至婺源,與浙江、江西界。東至海。距北京三千四百四十五里。

應天府元集慶路,屬江浙行省。太祖丙申年三月曰應天府。洪武元年八月建都,曰南京。十一年曰京師。永樂元年仍曰南京。洪武二年九月始建新城,六年八月成。內為宮城,亦曰紫禁城,門六:正南曰午門,左曰左掖,右曰右掖,東曰東安,西曰西安,北曰北安。宮城之外門六:正南曰洪武,東曰長安左,西曰長安右,東之北曰東華,西之北曰西華,北曰玄武。皇城之外曰京城,周九十六里,門十三:南曰正陽,南之西曰通濟,又西曰聚寶,西南曰三山,曰石城,北曰太平,北之西曰神策,曰金川,曰鐘阜,東曰朝陽,西曰清涼,西之北曰定淮,曰儀鳳。後塞鐘阜、儀鳳二門,存十一門。其外郭,洪武二十三年四月建,周一百八十里,門十有六:東曰姚坊、仙鶴、麒麟、滄波、高橋、雙橋,南曰上方、夾岡、鳳臺、大馴象、大安德、小安德,西曰江東,北曰佛寧、上元、觀音。領縣八。洪武二十六年編戶一十六萬三千九百一十五,口一百十九萬三千六百二十。弘治四年,戶一十四萬四千三百六十八,口七十一萬一千三。萬曆六年,戶一十四萬三千五百九十七,口七十九萬五百一十三。

上元倚。太祖丙申年遷縣治淳化鎮,明年復還舊治。東北有鐘山,山南有孝陵衛,洪武三十一年置。北有覆舟山。西北有雞鳴山、幕府山。又東北有攝山。東南有方山。北濱大江。東南有秦淮水,北流入城,又西出,入大江。又北有玄武湖。東有青溪,又有淳化鎮巡檢司。

江寧倚。南有聚寶山、牛首山。西南有三山、烈山、慈姥山。西濱大江。東北有靖安河。西南有大勝關、江寧鎮。東南有秣陵關。西有江東四巡檢司。北有龍江關,置戶分司於此。

句容府東。南有茅山。北有華山,秦淮水源於此。北濱大江。西北有龍潭巡檢司。

溧陽府東南。元溧陽州。洪武二年降為縣。東南有鐵山、銅山。西南有鐵冶山。北有長蕩湖,一名洮湖,與宜興、金壇二縣分界。西北有溧水,一名瀨水,上承丹陽湖,東流為宜興縣荊溪,入太湖,舊名永陽江,又曰中江也。西北有上興埠巡檢司,後廢。

溧水府東。元溧水州。洪武二年降為縣。東南有東廬山,秦淮水別源出焉。南有石臼湖,西連丹陽湖,注大江。

高淳府南。弘治四年以溧水縣高淳鎮置。西南有固城、丹陽、石臼諸湖。東南有廣通鎮,俗曰東壩,有廣通鎮巡檢司。

江浦府西。本六合縣浦子口巡檢司,洪武九年六月改為縣,析和、滁二州及江寧縣地益之。二十五年七月移於江北新開路口,仍置巡檢司於舊治。東南濱大江,有江淮衛,洪武二十八年正月置。又有西江口巡檢司。

六合府西北。元屬真州。洪武三年直隸揚州府。二十二年二月來屬。東有瓜步山,濱大江,滁河水自西來,入焉。有瓜埠巡檢司。

鳳陽府元濠州,屬安豐路。太祖吳元年升為臨濠府。洪武二年九月建中都,置留守司於此。六年九月曰中立府。七年八月曰鳳陽府。洪武二年九月建中都城於舊城西,三年十二月始成。周五十里四百四十三步。立門九:正南曰洪武,南之左曰南左甲第,右曰前右甲第,北之東曰北左甲第,西曰後右甲第,正東曰獨山,東之左曰長春,右曰朝陽,正西曰塗山。中為皇城,周九里三十步,正南門曰午門,北曰玄城,東曰東華,西曰西華。領州五,縣十三。距南京三百三十里。洪武二十六年編戶七萬九千一百七,口四十二萬七千三百三。弘治四年,戶九萬五千一十,口九十三萬一千一百八。萬曆六年,戶一十一萬一千七十,口一百二十萬二千三百四十九。

鳳陽倚。洪武七年八月析臨淮縣地置,為府治。十一年又割虹縣地益之。北濱淮,南有鏌邪山,西濠水出焉。又西南有皇陵城,洪武二年置衛。西北有長淮關,洪武六年置長淮衛於此。東北有洪塘湖屯田守禦千戶所,洪武十一年置。

臨淮府東北。元曰鐘離,為濠州治。洪武二年九月改曰中立。三年十一月改曰臨淮。七年為府屬。北濱淮。有二濠水,東源出濠塘山,西源出鏌邪山,至城西南合流,東入淮。

懷遠府西北。荊山在縣西南。塗山在縣東南。淮水經兩山峽間,有北肥水入焉。又北有渦水亦入淮,謂之渦口。又西南有洛水,與壽州分界,徑縣南新城村入淮。有洛河鎮巡檢司。

定遠府南。南有池河。。西有洛河。又有英武衛在北,飛熊衛在東北,俱洪武十一年置。

五河府東北。元屬泗州。洪武四年二月來屬。舊治在縣南,永樂元年圮於水,徙治西北界。嘉靖二十五年遷於澮河北,即今治也。東濱淮。東南有漴河,西北有澮河、沱河,東北有潼河,並流合淮,謂之五河口。又西有上店巡檢司,後廢。

虹府東北。元屬泗州。洪武七年七月來屬。南有汴河。東南有潼河。西有沱河。

壽州元安豐路,屬河南江北行省。太祖丙午年曰壽春府。吳元年曰壽州,屬臨濠府。洪武二年九月直隸中書省。四年二月還屬,後以州治壽春縣省入。北濱淮。淮水經山硤中,謂之硤石山,有西肥水來合焉。東北有八公山,東肥水經其下,西入淮,謂之肥口。又西北有潁水,亦入淮。又南有芍陂水,西有渒水,俱入淮。又北有下蔡縣,南有安豐縣,俱洪武中省,有下蔡鎮巡檢司。又東有北爐鎮、西有正陽鎮二巡檢司。東距府一百八十里,領縣二:

霍丘州西南。西南有大別山。北濱淮,史河、灃河俱流入焉。南有開順鎮、丁塔店,西有高唐店三巡檢司。

蒙城州北。北有渦水,又有北肥水。

泗州元屬淮安路。太祖吳元年屬臨濠府。洪武二年九月直隸中書省。四年二月還屬府,後以州治臨淮縣省入。南濱淮,有汴水自城北南流入焉。西距府二百十里,領縣二:

盱眙州南。東南有都梁山。東北有龜山。西有浮山。北濱淮,有池河自西來入焉。又東北有洪澤湖,淮水之所匯也。又西有舊縣巡檢司。

天長州東南。冶山在縣南。西北有石梁河,下流為五湖,接高郵州界。東北有城門鄉巡檢司。

宿州元屬歸德府。洪武四年二月來屬。龍山在西南,北肥水出焉。又北有睢河,自河南永城縣流入,下流至宿遷縣合淮,亦曰小河也。南有汴河,亦自永城縣流入,又有澮河與渙水合。又東南有沱水。東南距府二百三十三里。領縣一:

靈璧州東。西南有齊眉山。北有磬石山。黃河在東北。南有汴河。北有睢河。又南有固鎮巡檢司。

潁州元屬汝寧府。洪武四年二月來屬。淮河在南,自河南固始縣流入,下流合大河入海。又南有汝水,自河南息縣流入,經硃皋鎮入淮。又北有潁河,自河南沈丘縣流入。洪武二十四年,黃河決於河南,由陳州合潁,徑太和縣,又經州城北,又經潁上縣,至壽州同入於淮。永樂九年,河復故道。宣德、正統、成化、正德間,河、潁時通時塞,俗亦稱潁為小黃河。西北又有沈丘鎮巡檢司。東距府四百四十里。領縣二:

潁上州東南。東有潁河。南有淮河。東北有西肥水。

太和州西北。南有潁水,亦名沙河。北有西肥水。又有洪山、北原和二巡檢司。

亳州元屬歸德府。洪武初,以州治譙縣省入,尋降為縣,屬歸德州。六年屬潁州。弘治九年十月復升為州。西有渦河,自河南鹿邑縣流入,北有馬尚河,流合焉。南有西肥水,即夏肥水也。又東南有城父縣,洪武中廢。又有義門巡檢司。東南距府四百五十里。

淮安府元淮安路,屬淮東道宣慰司。太祖丙午年四月為府。領州二,縣九。西南距南京五百里。洪武二十六年,編戶八萬六百八十九,口六十三萬二千五百四十一。弘治四年,戶二萬七千九百七十八,口二十三萬七千五百二十七。萬曆六年,戶一十萬九千二百五,口九十萬六千三十三。

山陽倚。北濱淮。高家堰在其西南。南有運河,永樂中浚。西南有永濟河,萬歷九年開,長六十五里,亦謂之新運河。東南有射陽湖。東北有馬邏鄉、廟灣鎮、羊寨鄉三巡檢司。

清河府西。縣治濱黃河,崇禎末,遷治縣東南之甘羅城。南有淮河,東北與黃河合,謂之清口,舊謂之泗口。自徐州至此,皆泗水故道,為黃河所奪者也。南有洪澤湖,有洪澤巡檢司。又東有馬頭鎮巡檢司。

鹽城府東南。東濱海,有鹽場。北有射陽湖。西有清溝、西北有喻口鎮二巡檢司。

安東府東北。元安東州。洪武二年正月降為縣。東北朐山在南。東北有鬱洲山,在海中,洪武初,置東海巡檢司於此,後移於州南之新壩。西南有漣河,又有桑墟湖,濱海。南有淮水,東北過雲梯關,折旋入於海。自清口至此,皆淮水故道,為黃河所奪者也。又漣水自西北來,東南流入淮。又西北有碩項湖。東北有五港口、長樂鎮,東南有壩上三巡檢司。

桃源府西北。元曰桃園。洪武初,更名。北有大河,即泗水故道。西北有古城巡檢司。東有三義鎮巡檢司,崇禎末,移於縣西之白洋河鎮。

沭陽府北。元屬海寧州。洪武初改屬。東南有沭水,自山東郯城縣流入,其下流為漣水。又北有桑墟湖。海州元曰海寧州。洪武初,復曰海州,以州治朐山縣省入。北有於公、白溝等浦,皆產鹽。南有惠澤、西北有高橋二巡檢司。南距府二百七十里。領縣一:

贛榆州北。西北有羽山。東濱海。東北有荻水鎮、南有臨洪鎮二巡檢司。

邳州元屬歸德府。洪武初,以州治下邳縣省入。四年二月改屬中都。十五年來屬。北有艾山,接山東沂水縣界。西有沂水,自沂州西流,至下邳入泗。又西北有泇河。萬歷三十五年開泇以通運,自沛縣夏鎮迄直河口,長二百六十餘里,避黃河險者三百餘里。有直河口巡檢司。又西有新安巡檢司。東南距府四百五十里。領縣二:

宿遷州東南。北有峒峿山。南有大河,即泗水故道。又東南有睢水,入大河,曰睢口,亦曰小河口。又東南有白洋河,西北有駱馬湖,皆入大河。東北有劉家莊巡檢司。

睢寧州南。北濱大河。有睢水自西來,經縣界,至睢口入河。

揚州府元揚州路,屬淮東道宣慰司。太祖丁酉年十月曰淮海府。辛丑年十二月曰維揚府。丙午年正月曰揚州府。領州三,縣七。西距南京二百二十里。洪武二十六年編戶一十二萬三千九十七,口七十三萬六千一百六十五。弘治四年,戶一十萬四千一百四,口六十五萬六千五百四十七。萬曆六年,戶一十四萬七千二百一十六,口八十一萬七千八百五十六。

江都倚。元末廢。太祖辛丑年復置。西有蜀岡。東有官河,即古邗溝,今運河也。南濱大江。東北有艾陵湖。北有邵伯湖,有邵伯鎮巡檢司。又東有萬壽鎮、西北有上官橋、南有瓜洲鎮三巡檢司。又東有歸仁鎮巡檢司,後遷便益河口。

儀真府西。元真州,治揚子縣。洪武二年,州廢,改縣曰儀真。西北有大、小銅山。南濱江。南有運河。東南有舊江口巡檢司,尋移於縣南汊河口。

泰興府南。南濱江。西北有口岸鎮、東有黃橋鎮、南有印莊三巡檢司。

高郵州元高郵府,屬淮東道宣慰司。洪武元年閏七月降為州,以州治高郵縣省入。西有運河。西北有樊梁、甓社、新開等湖。西南有白馬塘。北有張家溝、東北有時堡二巡檢司。又西有北阿鎮。東有三垛鎮。西南距府百二十里。領縣二:

寶應州北。西有運河,又有汜光、白馬、射陽等湖。南有槐樓鎮、西南有衡陽二巡檢司。

興化州東。南有運河。東有得勝湖。東北有安豐巡檢司。又東北有鹽場。

泰州洪武初,以州治海陵縣省入。東濱海。南濱江。西有運河。東北有西溪鎮、北有寧鄉鎮、東南有海安鎮三巡檢司。西距府百二十里。領縣一:

如皋州東南。大江在縣南。運河在縣北。東有掘港、南有石莊、北有西場三巡檢司。又東南有白浦鎮。

通州洪武初,以州治靜海縣省入。南有狼山,臨大江,有狼山巡檢司。東南濱海,舊有海門島及布州夾。西有運鹽河。又東北有石港巡檢司。城南有利豐監,宋置。西距府四百里。領縣一:

海門州東。舊治禮安鄉圮於海,正德七年徙治余中場。嘉靖二十四年八月遷於金沙場以避水患。海在東,大江於此入海。又西有張港、東有吳陵、又有安東壩上、又有白塔河四巡檢司。東南有料角嘴。

蘇州府元平江路,屬江浙行省。太祖吳元年九月曰蘇州府。領州一,縣七。西距南京五百八十八里。洪武二十六年編戶四十九萬一千五百一十四,口二百三十五萬五千三十。弘治四年,戶五十三萬五千四百九,口二百四萬八千九十七。萬曆六年,戶六十萬七百五十五,口二百一萬一千九百八十五。

吳倚。西有姑蘇山。西南有橫山,又有穹窿、光福等山。又有太湖。湖縱廣三百八十三里,周三萬六千頃,跨蘇、常、嘉、湖四府之境,亦曰具區,亦曰五湖,中有包山、莫厘山。又南有吳淞江,亦曰松江,亦曰松陵江,亦曰笠澤,自太湖分流,東入海。又西有運河。西南有木瀆、東山、甪頭三巡檢司。又有橫金巡檢司,後廢。

長洲倚。西北有虎丘山,又有陽山,又有長蕩、陽城等湖。東有婁江,源出太湖。東南有運河。又北有吳塔、東南有陳墓二巡檢司。又東有唐湖巡檢司。後廢。

吳江府東南。元吳江州。洪武二年降為縣。西濱太湖。東有吳淞江,又有運河。又東南有白蜆江。又東有同里,南有平望,西南有震澤,東南有簡村、汾湖五巡檢司。又東有長橋、西南有瀾溪、東南有因瀆三巡檢司,後廢。

崑山府東。元崑山州。洪武二年降為縣。南有吳淞江。西有女婁江。東南有澱山湖。又南有千墩浦,東有夏駕浦,皆注於婁江。東南有石浦巡檢司,後移於千墩浦口。西北有巴城巡檢司,後移於縣西之真義鎮。

常熟府北。元常熟州。洪武二年降為縣。萬曆末避諱曰嘗熟。西北有虞山。北有福山,下臨大江。有福山浦,又東有白茆浦,東北有許浦,西北有奚浦、黃泗浦,為五大浦。皆分太湖西北之水,注於大江。南有運河。有許浦、白茅、黃泗、福山四巡檢司。

嘉定府東。元嘉定州。洪武二年降為縣。東濱海,南有運河。又南有吳淞江,東南有白鶴江,西南有青龍江,南有蟠龍江,皆匯吳淞江入海。又劉河在縣北,即婁江也。又東南有吳淞江守禦千戶所,洪武十九年置。又有寶山守禦千戶所,本協守吳淞中千戶所,嘉靖三十六年置,萬曆五年更名。又東有顧徑、東南有江灣二巡檢司。又西南有吳塘、南有南翔二巡檢司,後廢。

太倉州本太倉衛,太祖吳元年四月置。弘治十年正月置州於衛城,析崑山、常熟、嘉定三縣地益之。東濱海。海口有鎮海衛,洪武十二年十月置,後移於太倉衛城。南有劉河,其入海處曰劉河口,有劉家港巡檢司。北有七鴉浦,亦東入海。又東北有甘草巡檢司。又有唐茜涇口巡檢司,後移於東花浦口,尋廢。又有茜涇巡檢司,亦廢。西距府一百零五里。領縣一:

崇明州東。元崇明州,屬揚州路。洪武二年降為縣。八年改屬蘇州府。弘治十年正月來屬。舊治在縣東北曰東沙,為海所圮。永樂十九年、嘉靖八年、三十三年三遷,亦俱圮於水。萬曆十三年遷於平洋沙巡檢司,即今治也。四面環海。西有西沙、北有三沙二巡檢司。

松江府元直隸江浙行省。太祖吳元年正月因之。領縣三。西北距南京七百七十里。洪武二十六年編戶二十四萬九千九百五十,口一百二十一萬九千九百三十七。弘治四年,戶二十萬五百二十,口六十二萬七千三百一十三。萬曆六年,戶二十一萬八千三百五十九,口四十八萬四千四百一十四。

華亭倚。昆山在縣西北。東南濱海,有鹽場。又西北有澱山湖,西有泖湖。東南有黃浦,西北有趙屯、大盈、顧會、松子、磐龍等五浦,俱會吳淞江入海。東南有金山衛,又東有青村守御千戶所,俱洪武二十年二月置。西北有小貞村、西南有泖橋二巡檢司。南有金山巡檢司,本治張堰,後徙胡家巷。東南有南橋巡檢司,本戚睦,後徙治更名。又有陶宅巡檢司,後廢。又東南有柘林鎮,嘉靖間築城戍守。

上海府東北。東濱海,有鹽場。北有吳淞江,有巡檢司。東有黃浦,有巡檢司。東南有南匯觜守禦中、後千戶所,洪武二十年二月置。又有三林莊巡檢司。又有南蹌巡檢司,後廢。嘉靖三十六年築城曰川沙堡,置兵戍守焉。

青浦府西北。嘉靖二十一年四月以今縣東北之新徑巡檢司置,析華亭、上海二縣地益之。三十二年廢為青龍鎮,仍置新徑巡檢司。萬歷元年復於唐行鎮置縣,即今治也。北有吳淞江。東有顧會等浦。西南有澱山湖。又西有安莊鎮,殿山巡檢司置於此。

常州府元常州路,屬江浙行省。太祖丁酉年三月丁亥曰長春府,己丑曰常州府。萬曆末,避諱曰嘗州府。領縣五。西北距南京三百六十里。洪武二十六年編戶一十五萬二千一百六十四,口七十七萬五千五百一十三。弘治四年,戶五萬一百三十一,口二十二萬八千三百六十三。萬曆六年,戶二十五萬四千四百六十,口一百萬二千七百七十九。

武進倚。東為晉陵縣,元時同治郭內。太祖丁酉年三月改武進縣曰永定,晉陵縣曰京臨。尋以京臨省入永定。壬寅年八月仍改永定曰武進。東南有馬跡山,濱太湖。北有大江。西有孟瀆,又有得勝新河,俱北入江。南有運河。西南有水鬲湖。與宜興界。東有陽湖,與無錫界。西有魏村閘守御百戶所,洪武三年置。又有奔牛巡檢司。西北有小河巡檢司,舊在鄭港,後移小河寨,尋復遷孟河城。北有澡江巡檢司,舊在江北沙新河,後遷縣北於塘村。

無錫府東。元無錫州。洪武二年四月降為縣。西有慧山,梁溪出焉,西南入太湖,其別阜曰錫山。西南有太湖。東南有運河。又西北有高橋、東南有望亭二巡檢司。

宜興府南。元宜興州。太祖戊戌年十月曰建寧州,尋復曰宜興州。洪武二年降為縣。西南有荊南山,又有國山,又有龍池山。又東南有香蘭山,臨太湖。又有唐貢山,產茶。西北有掞山,有長蕩湖。北有運河。南有荊溪。西南有百瀆,疏荊溪之下流,注於太湖,後多堙廢。東北有下邾、北有鐘溪、東南有湖水父、西南有張渚四巡檢司。

江陰府西北。元江陰州,直隸江浙行省。太祖甲辰年曰連洋州,尋復曰江陰州。吳元年四月降為縣,來屬。北有君山,濱大江。西南又有秦望山。東有香山。南有運河。又申浦在西,又有黃田等港,俱注大江。東有石頭港巡檢司。西有利港巡檢司,後移於夏港。又東有范港巡檢司,後廢。又有楊舍鎮,嘉靖三十七年築城。

靖江府東北。成化七年閏九月以江陰縣馬馱沙置。大江舊分二派,繞縣南北。天啟後,潮沙壅積,縣北大江漸為平陸。西南有新港巡檢司。

鎮江府元鎮江路,屬江浙行省。太祖丙申年三月曰江淮府,十二月曰鎮江府。領縣三。西距南京城二百里。洪武二十六年編戶八萬七千三百六十四,口五十二萬二千三百八十三。弘治四年,戶六萬八千三百四十四,口一十七萬一千五百八。萬曆六年,戶六萬九千三十九,口一十六萬五千五百八十九。

丹徒倚。北有北固山,濱大江。江中西北有金山,東北有焦山。又城西江口有蒜山。又京峴山在東,圌山在北,濱江為險。又南有運河。西有高資鎮、東北有安巷、東有丹徒鎮、北有姜家觜四巡檢司。

丹陽府東南。北濱大江,又有練湖。南有運河。又東有呂城鎮巡檢司,尋移鎮東。又有包港巡檢司,尋移顧巷。

金壇府東南。西有茅山。東南有長蕩湖,一名洮湖,有湖溪巡檢司。北有白鶴溪。

廬州府元廬州路,屬河南江北行省。太祖甲辰年七月為府,置江淮中書行省於此,尋罷。領州二,縣六。距南京五百十里。洪武二十六年編戶四萬八千七百二十,口三十六萬七千二百。弘治四年,戶三萬六千五百四十八,口四十八萬六千五百四十九。萬曆六年,戶四萬七千三百七十三,口六十二萬二千六百九十八。

合肥倚。西有雞鳴山,肥水所出,東南流入巢湖。西南有紫篷山。東有浮槎山、橫山。又東南有四頂山,俯瞰巢湖,湖周四百餘里,中有姥山、孤山。又東北有滁水,源出龍潭,下流至六合縣入江。又東有店阜河,南有三汊河,皆入巢湖。東北有梁縣,洪武初省。西南有廬鎮關巡檢司,後徙於縣東之石梁鎮。

舒城府西南。西南有龍眠山,與桐城縣界。西有三角山。又巢湖在東。又南有北峽關,亦與桐城界。

廬江府南。元屬無為州。洪武初,改屬府。東北有冶父山。東有巢湖。東南有黃陂湖。西有冷水關,有巡檢司。

無為州洪武中,以州治無為縣省入。大江在東南。東有濡須水,一名天河,自巢湖分流,東北入江。又東有奧龍河鎮,東南有泥汊河鎮、土橋河鎮,北有黃落河鎮四巡檢司。西北距府二百八十里。領縣一:

巢州北。東南有七寶山,與含山縣濡須山相對峙,有西關在其上。巢湖在西,西北有柘皋河流入焉。南有石梁河,即濡須上流也,東南有清溪入焉。西南有焦湖巡檢司。

六安州洪武四年二月屬中都臨濠府,以州治六安縣省入。十五年改屬。西有淠水,亦曰沘水,下流至壽州入淮。西南有麻埠巡檢司,後廢。又西北有和尚灘巡檢司,弘治間屬霍山縣,後移於新店,仍來屬。東距府百八十里。領縣二:

英山州西南。縣治本直河鄉,崇禎十二年徙於西北之章山,十六年又遷於北境之添樓鄉。多雲山在西北,接湖廣羅田縣界。西有英山河,湖廣浠水之上源也。

霍山州西南。本六安州故埠鎮巡檢司,弘治二年改為縣。南有霍山,亦曰天柱山,亦曰衡山,又謂之南岳也。東南有鐵爐山,多鐵冶。又西南有四十八盤山,又淠河在東,源出霍山,下流至壽州入淮。西北有千羅畈、西南有上土市二巡檢司。

安慶府元安慶路,屬河南江北行省。太祖辛丑年八月曰寧江府,壬寅年四月曰安慶府。領縣六。北距南京六百五十里。洪武二十六年編戶五萬五千五百七十三,口四十二萬二千八百四。弘治四年,戶四萬六千五十,口六十一萬六千八十九。萬曆六年,戶四萬六千六百九,口五十四萬三千四百七十六。

懷寧倚。南濱大江,西有皖水流入焉,曰皖口。西北有觀音港巡檢司。東有長風沙鎮巡檢司。

桐城府東北。東有浮山,一名浮度山。西北有龍眠山。北有北峽山,與舒城界,有北峽關巡檢司。又北有西峽山,亦謂之南峽石,對壽州峽石則此為南也。東南濱江,有樅陽河,自西北流入焉。又東有六百丈、東南有馬踏石、源子港三巡檢司。

潛山府西北。元末廢。洪武初復置。西北有灊山,亦曰天柱山,亦曰皖公山,即霍山也,皖水出焉,別流曰灊水,合流注大江。又有天堂山,後部河所出,有天堂寨巡檢司。

太湖府西北。西北有司空山。城西有馬路河,即後部河之下流也,東合於灊水。又西北有南陽、白沙,東北有小池,北有後部四巡檢司。

宿松府西南。東有馬頭山。又小姑山在縣南大江中,與江西彭澤縣界,有小姑山巡檢司。又西南有歸林灘、南有涇江口二巡檢司。

望江府西南。南濱江。東有雷池,南入江,曰雷江口,亦曰雷港,有巡檢司。西有泊湖,北有慈湖,東北有漳湖,下流俱入江。又西南有楊灣鎮巡檢司。

太平府元太平路,屬江浙行省江東道。太祖乙未年六月為府。領縣三。東距南京百三十五里。洪武二十六年編戶三萬九千二百九十,口二十五萬九千九百三十七。弘治四年,戶二萬九千四百六十六,口一十七萬三千六百九十九。萬曆六年,戶三萬三千二百六十二,口一十七萬六千八十五。

當塗倚。城北有采石山,一名牛渚山,臨大江。西南有博望山,與和州梁山夾江相對,亦曰東梁山。又丹陽湖在東南,周三百餘里,分流蕪湖,西入江。南有姑熟溪,又有黃池河,西南有大信河,北有慈湖,皆入大江。有采石、大信二巡檢司。

蕪湖府西南。西南有戰鳥山,在大江中,西北有七磯。南有魯明江,一名魯港,又有石洈河,俱注大江。西有河口鎮巡檢司,後移於魯港鎮。

繁昌府西南。西北有磕山,在江中。又三山磯在東北,濱江。又西有荻港,入大江。有三山、荻港二巡檢司。

池州府元池州路,屬江浙行省江東道。太祖辛丑年八月曰九華府,尋曰池州府。領縣六。東北距南京五百五十里。洪武二十六年編戶三萬五千八百二十六,口一十九萬八千五百七十四。弘治四年,戶一萬四千九十一,口六萬九千四百七十入。萬曆六年,戶一萬八千三百七十七,口八萬四千八百五十一。

貴池倚。南有齊山。北濱江。東有梅根港。西有池口河,即貴池也,又西有李陽河,俱流入大江。有池口鎮、李陽河二巡檢司。

青陽府東。西南有九華山。北有青山。西有五溪水,出九華山,又南有臨城河,俱會流大通河入江。

銅陵府東北。南有銅官山。東有城山。西濱大江。又南有大通河,北有荻港河,俱入大江,有大通巡檢司。

石埭府東南。北有陵陽山。西有櫟山,官溪出焉,即池口河之源也。又舒溪在南,下流合蕪湖縣之魯港入江。

建德府西南。南有龍口河,東南入饒州府之獨山湖。又有堯城溪,下流為東流縣之江口河,入江。又西南有永豐鎮巡檢司。

東流府西。西南有馬當山,枕大江,與江西彭澤縣界。南有香口河,流入江,有香口鎮巡檢司,後移於吉陽鎮。

寧國府元寧國路,屬江浙行省。太祖丁酉年四月曰寧國府。辛丑年四月曰宣城府。丙午年正月曰宣州府。吳元年四月仍曰寧國府。領縣六。北距南京三百十里。洪武二十六年編戶九萬九千七百三十二,口五十三萬二千二百五十九。弘治四年,戶六萬三百六十四,口三十七萬一千五百四十三。萬曆六年,戶五萬二千一百四十八,口三十八萬七千一十九。

宣城倚。北有敬亭山。西有清弋江,西北至蕪湖縣入江。又東有宛溪,與東北之句溪合,北流入大江。又南湖亦在東北,流注於句溪。北有黃池鎮、東北有水陽鎮二巡檢司。

南陵府西。西有工山。南有呂山,淮水出焉。東有青弋江。又西南有漳水,與淮水合,入於青弋江。又南有峨嶺巡檢司。

徑府西。南有承流山。西有賞溪,亦曰涇溪,其上流即舒溪也。又東南有藤溪來合焉,下流入青弋江。東南有茹蔴嶺巡檢司。

寧國府東南。西有紫山。西北有文脊山。東南有千秋嶺,有關。東有東溪,出浙江於潛縣天目山。西有西溪,出績溪縣巃叢山,即句溪上源也。東南有嶽山巡檢司,舊置嶽山下,洪武中遷於紐口,復移於石口鎮。又西南有胡樂巡檢司。

旌德府南。北有石壁山。西有正山。西南有箬嶺,與太平、歙二縣界。東有徽水,自績溪縣流入,即藤溪上流也。東北有烏嶺巡檢司,廢。又北有三溪巡檢司。

太平府西南。南有黃山,與歙縣分界。西有龍門山,有巡檢司。南有麻川,與舒溪合流入涇縣,為賞溪。西南有宏潭巡檢司,後移於郭巖前。

徽州府元徽州路,屬江浙行省。太祖丁酉年七月曰興安府。吳元年曰徽州府。領縣六。北距南京六百八十里。洪武二十六年編戶一十二萬五千五百四十八,口五十九萬二千三百六十四。弘治四年,戶七千二百五十一,口六萬五千八百六十一。萬曆六年,戶一十一萬八千九百四十三,口五十六萬六千九百四十八。

歙倚。西北有黃山,亦曰黟山,新安江出焉,東南流為歙浦。又東曰新安江,至浙江建德縣,與東陽江合為浙江上源。又楊之水在西,亦曰徽溪,合於歙浦。東南有街口鎮、王乾寨二巡檢司。西北有黃山巡檢司。

休寧府西。東北有松蘿山。西有白岳山。東南有率山,率水出焉,新安江別源也。西南有浙溪,東流與率水合。又西有吉陽水,亦曰白鶴溪,下流合於浙溪。西南有黃竹嶺巡檢司,尋廢。東南有太廈巡檢司,後移於屯溪。

婺源府西南。元婺源州。洪武二年正月降為縣。北有浙嶺,浙溪水出焉,一名漸溪,新安江別源也。西北有大廣山,婺水所出,南流達於鄱陽湖。又西南有太白、東有大鏞嶺二巡檢司。又西有項村巡檢司。舊治澆嶺,後移縣西北之嚴田。萬歷九年復故。

祁門府西。東北有祁山。西有新安山,又有武陵嶺。北有大共山,大共水出焉,南流入江西浮梁縣界。有大共嶺巡檢司。又西南有良禾嶺巡檢司,後移於苦竹港。

黟府西。西南有林歷山。又有武亭山,橫江水出焉。又東北有吉陽山,吉陽水所出。南有魚亭山,魚亭水出焉。俱流合橫江。

績溪府東北。西北有徽嶺山。東有大鄣山,浙水出焉,亦新安江別源也。又巃叢山在東北,楊之水出焉,流合大鄣山水。有叢山關,與寧國縣界。東有西坑寨巡檢司,尋廢。西北有濠寨巡檢司。

徐州元屬歸德府。洪武四年二月屬中都臨濠府。十四年十一月直隸京師。東南有雲龍山。天啟四年遷州治於雲龍山。東北有盤馬山,產鐵。又有銅山。東南有呂梁山,泗水所經。大河自蕭縣流入,經州城北,遂奪泗水之道,東經百步洪、呂梁洪而入邳州界。有呂梁洪巡檢司。又睢水在南。領縣四。南距南京一千里。洪武二十六年編戶二萬二千六百八十三,口一十八萬八百二十一。弘治四年,戶三萬四千八百八十六,口三十五萬四千三百一十一。萬曆六年,戶三萬七千八百四十一,口三十四萬五千七百六十六。

蕭州西南。舊治在縣西北,今治,萬曆五年徙。南有永固山。北有大河,舊汴河所經道也。南有睢水。又西北有趙家圈巡檢司。嘉靖四十四年,大河決於此。

沛州西北。元屬濟寧路。太祖吳元年來屬。南有大河。東有泗河,自山東魚臺縣流入境。又泡河在西,薛河在東,又北有南沙河、北沙河,皆會於泗。又昭陽湖在縣東。又東北有夏鎮。

豐州西北。元屬濟寧路。太祖吳元年來屬。大河在南。北有豐水,即泡河也。

碭山州西。元屬濟寧路。太祖吳元年來屬。東南有碭山。其北有芒山。大河自河南虞城縣流入,舊經縣南,嘉靖三十七年徙在北。又南有睢水。

滁州元屬揚州路。洪武初,以州治清流縣省入。七年屬鳳陽府。二十二年二月直隸京師。南有瑯邪山。西南有清流山,清流關在其南,清流水出焉,合於滁水。又滁水自全椒縣流入,下流至六合縣入江。西有大鎗嶺巡檢司。領縣二。東距南京一百四十五里。洪武二十六年編戶三千九百四十四,口二萬四千七百九十七。弘治四年,戶四千八百四十,口四萬九千七百一十二。萬曆六年,戶六千七百一十七,口六萬七千二百七十七。

全椒州南。洪武初省,十三年十一月復置。東南有九斗山。西北有桑根山。又滁水在南,自合肥縣流入,有襄水自北流合焉。

來安州北。洪武初省,十三年十一月復置。東北有五湖山,下有五湖。北有石固山。又來安水在東,東南合清流河。又東南有湯河,南入滁河。東北有白塔鎮巡檢司。

和州元治歷陽縣,屬盧州路。洪武初,省州入縣。二年九月復改縣為州,仍屬廬州府。七年屬鳳陽府,尋直隸京師。梁山在南,與當塗縣博望山夾江相對,謂之天門山,亦曰西梁山。又東南有橫江,南對當塗縣之采石磯。西南有柵江,即濡須水,入江之口也。南有白石水,又有裕溪河,源出巢湖,皆南流注於江。西有麻湖,亦曰歷湖,永樂中堙。東北有烏江縣,洪武初省。東有浮沙口、南有裕溪鎮二巡檢司。又南有牛屯河巡檢司,後移於烏江鎮,即故烏江縣也。領縣一。東南距南京百三十里。洪武二十六年編戶九千五百三十一,口六萬六千七百一十一。弘治四年,戶七千四百五十,口六萬七千一十六。萬曆六年,戶八千八百,口一十萬四千九百六十。

含山州西。洪武初省,十三年十一月復置。南有白石山,白石水出焉。西南有濡須山,與無為州界。西對巢縣之七寶山,濡須水出其間,即東關口也。又南有三義河,東合裕溪入江。

廣德州元廣德路,屬江浙行省。太祖丙申年六月曰廣興府。洪武四年九月曰廣德州。十三年四月以州治廣德縣省入,直隸京師。西有橫山。南有靈山。西北有桐川,匯丹陽湖入江,亦名白石水。南有廣安、西南有陳陽、北有杭村三巡檢司。又東南有苦嶺關,路通浙江安吉州。又有四安鎮。領縣一。北距南京五百里。洪武二十六年編戶四萬四千二百六十七,口二十四萬七千九百七十九。弘治四年,戶四萬五千四十三,口一十二萬七千七百九十五。萬歷六年,戶四萬五千二百九十六,口二十二萬一千五十三。

建平州西北。西南有桐川,又有南碕湖,亦謂之南湖,與宣城縣界,流入丹陽湖。北有梅渚、南有陳村二巡檢司。

○山東山西

山東《禹貢》青、兗二州地。元直隸中書省,又分置山東東西道宣慰司治益都路屬焉。洪武元年四月置山東等處行中書省。治濟南府。三年十二月置青州都衛。治青州府。八年十月改都衛為山東都指揮使司。九年六月改行中書省為承宣布政使司。領府六,屬州十五,縣八十九。為里六千四百有奇。南至郯城,與南直界。北至無棣,與北直界。西至定陶,與北直、河南界。東至海。距南京一千八百五十里,京師九百里。洪武二十六年編戶七十五萬三千八百九十四,口五百二十五萬五千八百七十六。弘治四年,戶七十七萬五百五十五,口六百七十五萬九千六百七十五。萬曆六年,戶一百三十七萬二千二百六,口五百六十六萬四千九十九。

濟南府元濟南路,屬山東東西道宣慰司。太祖吳元年為府。領州四,縣二十六:

歷城倚。天順元年建德王府。南有歷山。東有華不注山。有大清河在西北,即濟水故道,自壽張縣流經縣界,東北至利津入海。又小清河,即濟之南源,一名濼水,出城西趵突泉,經城北,下流至樂安縣入海。又大明湖在城內。又東北有堰頭鎮巡檢司。

章丘府東。東有長白山,又有黌山。南有東陵山,又有長城嶺。又小清河在北。又東有淯河,一名繡江,合諸泉西北匯為白雲湖,下流入小清河。

鄒平府東北。西南有長白山,接章丘、長山二縣界。北有小清河。

淄川府東。元般陽路治此,屬山東東西道宣慰司。太祖吳元年改路為淄川州,縣仍為附郭。二年七月,州廢,來屬。西南有夾谷山。南有原山,與萊蕪縣界,其山陰淄水出焉。又西有孝婦河,自益都縣流入,合瀧、萌、般諸水,下流入小清河。

長山府東北。元屬般陽路。洪武二年七月來屬。西南有長白山。西北有小清河。南有孝婦河。

新城府東北。元屬般陽路。洪武二年七月來屬。七年十二月省入長山、高苑二縣,後復置。北有小清河。西北有孝婦河。東有烏河,其上流即時水,下流至高苑縣入小清河。

齊河府西。元屬德州。洪武二年七月改屬府。有大清河。

齊東府東。元屬河間路。洪武初來屬。北有大清河。東有減水河,成化元年開浚,洩小清河漲溢入大清河。

濟陽府北。南有大清河。

禹城府西北。元屬曹州。洪武二十年來屬。西有漯水枯河,俗名土河。

臨邑府北。元屬河間路。洪武初來屬。西北有盤河。

長清府西南。元屬泰安州。洪武二年七月改屬府。東南有青崖山、隔馬山、方山。西南有大清河。又有沙河,自縣南流入焉,亦曰沙溝河。又東南有石都寨巡檢司。

肥城府西南。元屬濟寧路。洪武二年七月來屬。西北有巫山,一名孝堂山,肥水出焉,西流入大清河。

青城府東北。元屬河間路。洪武二年省入鄒平、齊東二縣。十三年十一月復置,來屬。北有大清河。北有大石關,舊置巡檢司,後廢。

陵府西北。元德州,治安德縣,直隸中書省。洪武元年省安德縣入州。七年七月移州於故陵縣。十三年十一月置陵縣於此。東有德河,下流西入衛河。

泰安州元直隸中書省。洪武初來屬,以州治奉符縣省入。北有泰山,即岱宗也,亦曰東岳,汶水出焉,下流至汶上縣合大清河。又東南有徂徠山。南有梁父山。又城西有泰安巡檢司。北距府百八十里。領縣二:

新泰州東南。西北有宮山,本名新甫。西南有龜山。東北有小汶河,西流合汶水。又西有上四莊巡檢司。

萊蕪州東。洪武初,改屬濟南府。二年仍來屬。東北有原山,其山陽汶水別源出焉。又西南有冠山。西北有韶山。諸山多產銅鐵錫。

德州元陵州,屬河間路。洪武元年降為陵縣,屬濟寧府。二年七月改屬德州。七年七月省陵縣,移德州治焉。西有衛河。東南有故篤馬河,俗名土河。東南距府二百八十里。領縣二:

德平州東。東北有般河,亦曰盤河,或以為古鉤盤也。

平原州東南。

武定州元棣州,治厭次縣,屬濟南路。洪武初,州縣俱廢。六年六月復置州,改名樂安。宣德元年八月改為武定州。永樂十五年,漢王府遷於此。宣德元年除。南有大清河,又有土河,又有商河。東南有清河巡檢司。西南距府二百四十里。領縣四:

陽信州北。元屬棣州。東有商河。

海豐州東北。洪武六年六月析樂安州南地置,屬濱州,後來屬。東北濱海。又北有鬲津河,又有無棣縣,元屬棣州,洪武初省。東北有大沽河口巡檢司。

樂陵州西北。舊治在縣之咸平鎮,屬滄州,洪武元年改屬濟寧府,二年移治富平鎮,七月來屬。南有般河及鬲津河,又有土河。西南又有商河。西北有舊縣鎮巡檢司。

商河州西南。南有商河。

濱州洪武初,以州治渤海縣省入。東北濱海,產鹽。南有大清河。北有士傷河,即鬲津別名也。西南距府三百五十里。領縣三:

利津州東。東北濱海,有永阜等鹽場。東有大清河,流入海。又東北有豐國鎮巡檢司。

沾化州西北。東北濱海,有富國等鹽場。又有久山鎮巡檢司。

蒲臺州南。元屬般陽路。洪武二年七月來屬。東濱海。北有大清河。

兗州府元兗州,屬濟寧路。洪武十八年升為兗州府。領州四,縣二十三。東北距布政司三百五十里。

滋陽倚。洪武三年四月建魯王府。元曰嵫陽。洪武初,省入州。十八年復置。成化間,改為滋陽。泗水在東,又有沂水,自曲阜縣西流來合焉。

曲阜府東。東南有尼山,沂水所出。又東有防山。北有泗水。又有洙水,西南流入於沂水。又北有孔林。

寧陽府北。西北有汶水,支流為洸水。洸水者,洙水也,洸、洙相入受,通稱也,俱西南入運河。又東北有堽城堰,即汶、洸分流處也。

鄒府東南。元屬滕州。洪武二年七月改屬。東南有嶧山,亦曰邾嶧,又曰鄒嶧。東北有昌平山。西南有鳧山。又有泗河。

泗水府東。東有陪尾山,泗水出焉,經縣北,下流至南直清河縣入淮。

滕府東南。元滕州,治滕縣,屬益都路。洪武二年七月,州廢,縣屬濟寧府。十八年來屬。東南有桃山。東北有連青山。又西南有新運河,北自南陽,南至境山,長一百九十四里,嘉靖四十四年所開,又薛水,源自縣東高、薛二山間,西南流,合漷水,一名南沙河,至沛縣入運。又有北沙河在縣北,西流經魚臺入招湖。又南有沙溝集巡檢司。

嶧府東南。元嶧州,屬益都路。洪武二年降為縣,屬濟寧府,後來屬。東南有柱子山,舊名葛嶧山,水流其下。又北有君山,一名抱犢山,西泇水所出,東南流至三合村,有東泇河自沂水來會焉。又南合武河、彭、諸水注於泗,謂之泇口。萬曆中,改為運道,自夏鎮至直河口,凡二百六十餘里,避黃河之險者三百三十里。又西北有鄒塢鎮巡檢司,嘉靖中,移於縣西拖梨溝。又東南有臺莊巡檢司,萬歷三十四年置。

金鄉府西南。元屬濟寧路。洪武十八年來屬。金莎嶺在東。大河在西南。

魚臺府西南。元屬濟州。洪武元年屬徐州。二年七月屬濟寧府。十八年來屬。泗河在東,即運道也。北有菏水,一名五丈溝,東入泗。又東有穀亭鎮,嘉靖九年,黃河決於此。又南有塌場口,洪武、永樂間,為運道所經。

單府西南。元單州,屬濟寧路。洪武元年省州治單父縣入州。二年七月乃降州為縣,屬濟寧府。十八年來屬。舊城在南,正德十四年五月因河決改遷。南濱大河。

城武府西南。元屬曹州。洪武四年屬濟寧府。十八年來屬。縣城,正德十四年五月因河決改遷。南有故黃河,即洪武間之運道也,弘治後堙。

濟寧州元任城縣,為濟州治。至正八年罷濟州,徙濟寧路治此。太祖吳元年為濟寧府。十八年降為州,以州治任城縣省入。南臨會通河。西有馬腸湖。又東南有魯橋鎮巡檢司。東距府六十里。領縣三:

嘉祥州東。元屬單州。洪武二年來屬。南有塔山。東有會通河。北有故黃河,一名塔章河,即塌場口之上流也。

鉅野州西北。元為濟寧路治,至正八年徙路治任城縣,以縣屬焉。南有高平山。東有鉅野澤,元末為黃河所決,遂涸。東南有會通河。西南有故黃河,弘治後堙。西有安興集巡檢司。

鄆城州西北。西有灉水,又有故黃河,又有故濟水在西南。

東平州元東平路,直隸中書省。太祖吳元年為府。七年十一月降為州,屬濟寧府,以州治須城縣省入。十八年改屬。北有瓠山。東北有危山。西南有安山,亦曰安民山。下有積水湖,一名安山湖。山南有安山鎮,會通河所經也。汶水在南,西流入安山湖。又西北有金線閘巡檢司。東南距府百五十里。領縣五:

汶上州東南。西南有蜀山,其下為蜀山湖。又西為南旺湖,其西北則馬踏河,運道經其中而北出,即會通河也。又汶水在東北,舊時西流入大清河。永樂中,開會通河,堰汶水西南流,悉入南旺湖。

東阿州西北。故城在縣西南。今治本故穀城縣也,洪武八年徙於此。南有碻磝山。西有魚山。會通河自西南而北經此,始與大清河分流。又西有馬頰河,俗名小鹽河,東流入大清河。又張秋鎮在西南,弘治二年,河決於此。七年十二月塞,賜名安平鎮。

平陰州東北。南有汶河。西南有大清河,又有滑口鎮巡檢司,後廢。

陽穀州西北。東有會通河。又東有阿膠井。

壽張州西。洪武三年省入須城、陽穀二縣。十三年十一月復置,屬濟寧府,後來屬。東南有故城,元時縣治在焉。今治本王陵店,洪武十三年徙置。南有梁山水樂,即故大野澤下流。東北有會通河,又有沙灣,弘治前黃河經此,後堙。西南有梁山集巡檢司。

曹州正統十年十二月以曹縣之黃河北舊土城置。東有舊黃河,洪武初,引河入泗以通運處也。永樂中,亦嘗條浚。南有灉河。東南有菏澤,流為菏水。東北距府三百里。領縣二:

曹州東南。元曹州,治濟陰縣,直隸中書省。洪武元年省濟陰縣入州。二年,州治自北徙於盤石鎮。四年降為縣,屬濟寧府。正統十年十二月置州,以縣屬焉。西南有黃陵岡,與河南儀封縣界。弘治五年,黃河決於此,河遂在縣南,東入單縣界,至南直徐州,合泗入淮。又西有賈魯河,嘉靖前猶為運道,後廢。東南有楚丘縣,元屬曹州,洪武初省。又西北有安陵鎮巡檢司。

定陶州東南。元屬曹州。洪武元年屬濟寧府。十年五月省入城武縣。十三年十一月復置,仍屬濟寧府。正統十年十二月來屬。西有黃河故道。弘治前,河經此,至張秋之沙灣入會通河。

沂州元屬益都路,後省州治臨沂縣入州。洪武元年屬濟寧府。五年屬濟南府。七年十二月屬青州府。十八年來屬。弘治四年八月建涇王府,嘉靖十六年除。西有艾山。東有沂水,源自青州沂水縣,南流至州境,與枋水合,下流入泗。又有沭水,流經南直安東縣為漣水,入淮。又西南有泇水,亦曰東泇水,下流合嶧縣之西泇水入運。西南有羅藤鎮巡檢司。西距府五百六十里。領縣二:

郯城州東南。洪武初置。東有馬陵山,又有羽山,與南直贛榆縣界。又沭水在東。沂水在西。西有磨山鎮巡檢司,後廢。

費州西北。西北有蒙山。西南有大沫涸,又有祊水,東北有蒙陽水,下流俱入於沂河。西南有關陽鎮、西北有毛陽鎮二巡檢司。

東昌府元東昌路,直隸中書省。洪武初為府。領州三,縣十五。東距布政司二百九十里。

聊城倚。城東有會通河。西南有武水枯河,即漯河也,為會通河所截,中堙。

堂邑府西。東北有會通河。西有舊黃河。

博平府東北。洪武三年三月省,尋復置。西南有會通河。東北有故黃河。

茌平府東北。西有故黃河。又西北有故馬頰河。

莘府西南。北有弇山,舊有泉湧出,曰弇山泉。

清平府北。元屬德州。洪武元年屬恩州。二年七月屬高唐州。三年三月省,尋復置,改屬。西有會通河。西南有魏家灣巡檢司。

冠府西南。元冠州,直隸中書省。洪武三年降為縣,來屬。西北有衛河。又東有賈鎮堡,東北有清水鎮堡,俱嘉靖二十二年築。

臨清州元臨清縣,屬濮州。洪武二年七月改屬。弘治二年升為州。舊治在南,洪武二年徙治臨清閘。景泰元年又於閘東北三里築城,徙治焉。會通河在城南,有衛河自西來會,至天津直沽入海,為北運河。東南距府百二十里。領縣二:

丘州西。元直隸東昌路。弘治二年改屬州。東南有衛河,又有漳河。

館陶州西南。元屬濮州。洪武二年七月屬東昌府,三年三月省,尋復置,仍屬東昌府。弘治二年改屬州。西有衛河,自元城縣流入。又西南有漳河。又西南有南館陶鎮巡檢司。

高唐州元直隸中書省。洪武初,以州治高唐縣省入,來屬。西有漯河,溢涸無常。又有馬頰河,一名舊黃河。西南距府百二十里。領縣三:

恩州北。元恩州,直隸中書省。洪武二年降為縣,來屬。西有故城。今治本許官店,洪武七年七月徙於此。西北有衛河。東南有馬頰枯河。又高雞泊亦在縣西北。

夏津州西。洪武三年三月省,尋復置。西南有衛河。又東有馬頰故河。又西有裴家圈巡檢司。

武城州西北。西有衛河。東南有沙河。東北有甲馬營巡檢司。

濮州元直隸中書省。洪武二年以州治鄄城縣省入,來屬。故城在東,景泰三年以河患遷於王村,即今治也。東南有故黃河,永樂中,河流由此入會通河,後堙。又西南有濮水,一名洪河。東北距府二百里。領縣三:

范州東北。洪武三年三月省,尋復置。東南有故城,洪武二十五年圮於河,始遷今治。又東南有水保寨巡檢司。

觀城州西北。洪武三年三月省,尋復置。又東有馬頰河,有黑羊山水自西北流入焉。

朝城州北。洪武三年三月省,尋復置。西南有故漯河。

青州府元益都路,屬山東東西道宣慰司。太祖吳元年為青州府。領州一,縣十三。西距布政司三百二十里。

益都倚。洪武三年四月建齊王府,永樂四年廢。十三年建漢王府,十五年遷於樂安。成化二十三年建衡王府。南有雲門山,與劈山連。西北有堯山。又西有九回山,北陽水出焉,亦曰澠水,經治嶺山麓,曰五龍口,下流經樂安縣,入巨澱。又有南陽水,源出縣西南石膏山,流經城北,又東北合北陽水。又西有淄水,下流至壽光入海。又西南有顏神鎮,孝婦河出焉,入淄川縣界。有顏神鎮巡檢司,嘉靖三十七年築城。鎮西南有青石關。

臨淄府西北。南有牛山。又有鼎足山,女水出焉,下流合北陽水。又有蒨山。又有南郊山,其下為天齊淵。城東有淄水,又西有澠水,又有系水,下流俱入時水。其時水自西南而東北,亦曰耏水,又有澅水流入焉,下流俱至樂安縣入海。南有淄河店巡檢司,後廢。

博興府西北。元博興州。洪武二年降為縣。南有小清河,有時水。

高苑府西北。東南有商山。西南有小清河。西北有田鎮巡檢司。後廢。

樂安府北。東北濱海,有鹽場。北有小清河。東有時水。又東南有淄水,又有北陽水,又有巨洋水,俱匯流於縣東北之高家港入海。港即古之馬車瀆也。有高家港巡檢司。又西北有樂安鎮巡檢司。又東北有塘頭寨,有百戶所駐焉。

壽光府東北。北濱海,有鹽場。西有淄水,又有北陽水。又東有巨洋水。又西北有清水泊,即古之鉅定湖也,其北接樂安縣之高家港。又東北有廣陵鎮巡檢司。

昌樂府東。元屬濰州,尋省,後復置,仍屬濰州。洪武初,改屬。西北有故城。洪武中,徙於今治。東南有方山,東丹水所出,北徑昌樂故城,西丹水流合焉,下流至壽光縣入於海。又南有白狼水,至濰縣入海。

臨朐府東。南有朐山,又有大峴山,上有穆陵關巡檢司。又東有沂山,一名東泰山,沭水、水彌水俱發源於此。水彌水,一名巨洋水,西合石溝水,至壽光入海。又東北有丹山,一名丸山,西丹河及白狼水出焉。

安丘府東南。元屬密州。洪武二年七月,州廢,屬府。西南有牟山,又有峿山。又東北有岞山。東有濰水,下流經濰縣入海。又北有汶水,源亦出沂山,下流合濰水。

諸城府東南。元為密州治,屬益都路。洪武二年七月,州廢,屬府。東南有瑯邪山。西南有常山,又有馬耳山。北有濰水,東北有盧水,流合焉。南有信陽鎮巡檢司。又南有南龍灣海口巡檢司。

蒙陰府西南。元屬莒州。洪武二年七月改屬府。南有蒙陰山。東有長山,有蒙水,北流入沂水。東南有紫荊關巡檢司。萬歷間廢。

莒州元屬益都路。洪武初,以州治莒縣省入。西有浮來山。又西北有箕屋山,濰水出焉。又西南有沭水,流入沂州界。南有十字路、西南有葛溝店二巡檢司。北距府二百里。領縣二:

沂水州西北。西北有大弁山,與雕厓山連,沂水出焉,南流經沂州界入泗。東北有沭水。

日照州東北。東濱海,有鹽場。東南有夾倉鎮巡檢司。

萊州府元萊州,屬般陽路。洪武元年升為府。六年降為州。九年五月復升為府。領州二,縣五。西距布政司六百四十里。

掖倚。北濱海,有鹽場。又有三山島,在海南岸。東北有萬里沙。西南有掖水,北入海。東南有小沽河。又東北有王徐砦守禦千戶所,嘉靖中置。又西有海倉、北有柴葫寨二巡檢司。

平度州元膠水縣。洪武二十二年正月改置。北有萊山。西有膠水,下流至昌邑北入海。東有大沽河,源自黃縣蹲犬山,流經州,與小沽河合,通名為沽河,至即墨縣入海。小沽,即尤水也。又西南有亭口鎮巡檢司。北距府百里。領縣二:

濰州西。元濰州,屬益都路。洪武元年以州治北海縣省入。九年屬萊州府。十年五月降為縣。二十二年正月改屬州。南有濰水,東北入海。又東北有固堤店巡檢司。

昌邑州西北。元屬濰州。洪武十年五月省入濰縣。二十二年正月復置,來屬。東有濰水。東北有膠河。北有魚兒鎮巡檢司。

膠州元屬益都路。洪武初,以州治膠西縣省入。九年來屬。西南有鐵橛山,膠水所出,亦曰膠山。東北有沽河,南流入海。又東南海口有靈山衛,又有安東縣,俱洪武三十一年五月置。又有夏河寨千戶所,在靈山衛西南。石臼島寨千戶所,在安東衛南。俱弘治後置。又西南有古鎮巡檢司。北有逢猛鎮巡檢司。北距府二百二十里。領縣二:

高密州西北。元屬膠州。洪武元年屬青州府。九年五月屬萊州府,尋復屬州。東有膠水。西有濰水。又西南有密水,一名百尺溝,北會於濰水。

即墨州東。元屬膠州。洪武初,屬青州府。九年五月屬萊州府。十年五月仍屬州。東南有勞山,在海濱。又有田橫島,在東北海中。東有鰲山衛,洪武二十一年五月置。又東北有雄崖守御千戶所,南有浮山守禦千戶所,俱洪武中置。又東北有栲栳島巡檢司。又即墨營舊在縣南,宣德八年移置縣北,有城。

登州府元登州,屬般陽路。洪武元年屬萊州府。六年直隸山東行省。九年五月升為府。領州一,縣七。西距布政司一千零五十里。

蓬萊倚。洪武初廢。九年五月復置。北有丹崖山,臨大海。南有密神山,密水所出。西南有黑石山,黑水所出,經城南合流,北入於海。西有龍山,產鐵。東有高山巡檢司,本置於海中沙門島,後遷朱高山下。又東南有楊家店巡檢司。

黃府西南。東南有萊山。西南有蹲犬山,大沽水出焉。又東有黃水,東南有洚水,合流入海。又西有馬停鎮巡檢司。

福山府東南。東北有之罘山,三面臨海。西南有義井河,北入海。又奇山守禦千戶所在東北,洪武三十一年置。又北有孫夼鎮巡檢司。

棲霞府東南。東有岠禺山,嘗產金,亦名金山。又有百澗山,西北有北曲山,二山舊皆產鐵。又南有翠屏山,大河出焉,即義井河之上源也。

招遠府西南。元屬萊州。洪武九年五月來屬。東北有原畽河,北入海。西有東良海口巡檢司。

萊陽府南。元屬萊州。洪武九年五月來屬。東南有昌水,源發文登縣之昌山,一名昌陽水,南入海。東有豯養澤。又東南有大嵩衛,洪武三十一年五月置。衛西有大山千戶所,成化中置。又南有行村寨巡檢司。

寧海州元直隸山東東西道宣慰司。洪武初,以州治牟平縣省入,屬萊州府。九年改屬。東有金水河,一名沁水,西南有五丈河,俱北入海。又西南有乳山寨巡檢司。西距府二百二十里。領縣一:

文登州東南。元屬寧海州。洪武初,改屬萊州府。九年五月屬登州府,後仍屬州。東南有斥山。南有成山,又有鐵槎山。又西有鐵官山。東南濱海。南有靖海衛,東有成山衛,北有威海衛,皆洪武三十一年五月置。又寧津守禦千戶所在東南,亦洪武三十一年置。又東有海陽守禦千戶所,在靖海衛南。金山守禦千戶所,在威海衛西。百尺崖守禦千戶所,在威海衛北。尋山守禦千戶所,在成山衛東南。俱成化中置。又北有辛汪寨、東北有溫泉鎮、東有赤山鎮三巡檢司。

遼東都指揮使司元置遼陽等處行中書省,治遼陽路。洪武四年七月置定遼都衛。六年六月置遼陽府、縣。八年十月改都衛為遼東都指揮使司。治定遼中衛,領衛二十五,州二。十年,府縣俱罷。東至鴨綠江,西至山海關,南至旅順海口,北至開原。由海道至山東布政司,二千一百五十里。距南京一千四百里,京師一千七百里。

定遼中衛元遼陽路,治遼陽縣。洪武四年罷。六年復置。十年復罷。十七年置衛。西南有首山。南有千山。又東南有安平山,山有鐵場。又西有遼河,自塞外流入,至海州衛入海。又西北有渾河,一名小遼水,東北有太子河,一名大梁水,又名東梁水,下流俱入於遼水。又東有鴨綠江,東南入海。又東有鳳凰城,在鳳凰山東南,成化十七年築,為朝鮮入貢之道。又南有鎮江堡城。又連山關亦在東南。

定遼左衛、定遼右衛俱洪武六年十一月置。

定遼前衛洪武八年二月置。

定遼後衛本遼東衛,洪武四年二月置。八年二月改。九年十月徙治遼陽城北,尋復。

東寧衛本東寧、南京、海洋、草河、女直五千戶所,洪武十三年置。十九年七月改置。

自在州永樂七年置於三萬衛城,尋徙。

以上五衛一州,同治都司城內。

海州衛本海州,洪武初,置於舊澄州城。九年置衛。二十八年四月,州廢。西南濱海,有鹽場。西有遼河,匯渾河、太子河入海,謂之三岔河。又西有南、北通江,亦合於遼河。東有大片嶺關,有鹽場。東北距都司百二十里。

蓋州衛元蓋州,屬遼陽路。洪武四年廢。五年六月復置。九年十月置衛。二十八年四月,州復廢。東北有石城山。又北有平山,其下有鹽場。又東有駐蹕山,西濱海,有連雲島,上有關。又東有泥河,南有清河,東南有畢裏河,下流皆入於海。又南有永寧監城,永樂七年置。又西北有梁房口關,海運之舟由此入遼河,旁有鹽場。又東有石門關。西有鹽場。北有鐵場。北距都司二百四十里。

復州衛本復州,洪武五年六月置於舊復州城。十四年九月置衛。二十八年四月,州廢。西濱海。西南有長生島。又南有沙河,合麻河,西注於海。東有得利嬴城,元季士人築,洪武四年二月置遼東衛於此,尋徙。又南有樂古關。西有鹽場。北有鐵場。北距都司四百二十里。

金州衛本金州,洪武五年六月置於舊金州。八年四月置衛。二十八年四月,州廢。東有大黑山,小沙河出焉。又有小黑山,駱馬河、澄沙河俱出焉。衛東西南三面皆濱海。南有南關島。東有蓮花島。東南有金線島。又東有皮島,又有長行島。南有雙島及三山島。西南有鐵山島。東北有蕭家島,有關。又旅順口關在南,海運之舟由此登岸,有南、北二城,其北城有中左千戶所,洪武二十年置。又東南有望海堝石城,永樂七年置。又衛東有鐵場。東北有鹽場。北距都司六百里。

廣寧衛元廣寧府路。江武初廢。二十三年五月置衛。洪武二十五年三月建遼王府。建文中改封湖廣荊州府。西有醫無閭山。南濱海。東有路河,東北有珠子河,下流皆注於遼河。又板橋河在西,南流入海。北有白土廠關,又有分水嶺關。西北有魏家嶺關。又北有懿州,元屬遼陽路。洪武二十六年正月置廣寧後屯衛於此。永樂八年,州廢。徙衛於義州衛城。又西南有閭陽關,東北有望平縣,元俱屬廣寧路。又西北有川州,元屬大寧路。又東北有順州,西北有成州,元俱屬東寧路。又西南有鐘秀城,元置千戶所於此。俱洪武中廢。東距都司四百二十里。

廣寧中衛、廣寧左衛俱洪武二十六年正月置。二十八年四月廢。三十五年十一月復置。

廣寧右衛本治大凌河堡,洪武二十六年正月置。二十八年四月廢。三十五年十一月復置。

以上三衛,俱在廣寧衛城。

廣寧前衛、廣寧後衛俱洪武二十六年正月置。後俱廢。

義州衛元義州,屬大寧路。洪武初,州廢。二十年八月置衛。西北有大凌河,下流入海。東北有清河,下流合大凌河。東南距都司五百四十里。

廣寧後屯衛洪武二十六年正月置於舊懿州。永樂八年徙治義州衛城。

廣寧中屯衛元錦州,屬大寧路。洪武初,州廢。二十四年九月置衛。東有木葉山。西有東、西紅螺山。西南有杏山。東南有乳峰山。又東有大凌河、小凌河。又西有女兒河,與小凌河合。又南有松山堡,在松山西,宣德五年正月置中左千戶所於此,轄杏山驛至小凌河驛。東有大凌河堡,洪武二十六年正月置廣寧右衛,二十八年四月廢。宣德五年正月置中右千戶所於此,轄凌河驛至十三山驛。又城南有鹽場二,鐵場一。又西有鐵場。東南距都司六百里。

廣寧左屯衛洪武二十四年九月置於遼河西,後徙廣寧中屯衛城。

廣寧右屯衛元廣寧府地。洪武二十六年正月置於十三山堡。二十七年遷於舊閭陽縣之臨海鄉。北有十三山。山西有十三山堡。西有大凌河。又西南有望梅嶺。又南有鹽場,東有鐵場。東南距都司五百四十里。

廣寧前屯衛元瑞州,屬大寧路。洪武初,屬永平府。七年七月,州廢。二十六年正月置衛。西北有萬松山。北有十八盤山。西有麻子峪,有鐵場。東南為山口峪,有鹽場。東北有六州河,下流至蛇山務入海。西有山海關,與北直撫寧縣界。又有急水河堡,宣德五年正月置中前千戶所於此,轄山海東關至高嶺驛。又東有杏林堡,宣德五年正月置中後千戶所於此,轄沙河驛至東關驛。東距都司九百六十里。

寧遠衛宣德五年正月分廣寧前屯、中屯二衛地置,治湯池。西北有大團山。東北有長嶺山。南濱海。東有桃花島。東南有覺華島城。西有寧遠河,即女兒河也,又名三女河。又東有塔山,有中左千戶所,轄連山驛山至杏山驛,西有小沙河中右千戶所,轄東關驛至曹莊驛,俱宣德五年正月置。又南有鹽、鐵二場。東距都司七百七十里。

沈陽中衛元沈陽路。洪武初廢。三十一年閏五月置衛。洪武二十四年建沈王府。永樂六年遷於山西潞州。東有東牟山。南有渾河,又東有沈水入焉。又西有遼河。又東北有撫順千戶所,洪武二十一年置。所東有撫順關。北有蒲河千戶所,亦洪武二十一年置。南距都司百二十里。

沈陽左衛、沈陽右衛俱洪武中置。建文初廢。洪武三十五年七月復置,後仍廢。

沈陽中屯衛洪武三十一年閏五月置。建文中廢。洪武三十五年十一月復置,屬北平都司,後屬後軍都督府,寄治北直河間縣。

鐵嶺衛洪武二十一年三月以古鐵嶺城置。二十六年四月遷於古嚚州之地,即今治也。西有遼河,南有泛河,又南有小清河,俱流入於遼河。又南有懿路城,洪武二十九年置懿路千戶所於此。又範河城在衛南,亦曰泛河城,正統四年置汎河千戶所於此。東南有奉集縣,即古鐵嶺城也,接高麗界,洪武初置縣,尋廢。又有咸平府,元直隸遼東行省。至正二年正月降為縣。洪武初廢。南距都司二百四十里。

三萬衛元開元路。洪武初廢。二十年十二月置三萬衛於故城西,兼置兀者野人乞例迷女直軍民府。二十一年,府罷,徙衛於開元城。洪武二十四年建韓王府。永樂二十二年遷於陜西平涼。西北有金山。東有分水東嶺。北有分水西嶺。西有大清河,東有小清河,流合焉,下流入於遼河。又北有上河,東北有艾河,流合焉,謂之遼海,即遼河上源也。又北有金水河,北流入塞外之松花江。又鎮北關在東北。廣順關在江。又西有新安關。西南有清河關。南有山頭關。又北有北城,即牛家莊也,洪武二十三年三月置遼海衛於此。二十六年,衛徙。又南有中固城,永樂五年置。南距都司三百三十里。

遼海衛洪武二十三年三月置於牛家莊。二十六年徙三萬衛城。

安樂州永樂七年置,在三萬衛城。

山西《禹貢》冀州之域。元置河東山西道宣慰使司,治大同路。直隸中書省。洪武二年四月置山西等處行中書省。治太原路。三年十二月置太原都衛。與行中書省同治。八年十月改都衛為山西都指揮使司。九年六月改行中書省為承宣布政使司。領府五,直隸州三,屬州十六,縣七十九。為里四千四百有奇。東至真定,與北直界。北至大同,外為邊地。西南皆至河,與陜西、河南界。距南京二千四百里,京師千二百里。洪武二十六年編戶五十九萬五千四百四十四,口四百七萬二千一百二十七。弘治四年,戶五十七萬五千二百四十九,口四百三十六萬四百七十六。萬曆六年,戶五十九萬六千九十七,口五百三十一萬七千三百五十九。

太原府元冀寧路,屬河東山西道宣慰司。洪武元年十二月改為太原府,領州五,縣二十:

陽曲倚。洪武三年四月建晉王府於城外東北維。西有汾水,自靜樂縣流經此,下流至滎河縣合大河。西北有天門關巡檢司。東北有石嶺關巡檢司。

太原府西南。元曰平晉,治在今東北。洪武四年移於汾水西,故晉陽城之南關。八年更名太原。西有懸甕山,一名龍山,又名結絀山,晉水所出,下流入於汾。西北有蒙山。東有汾水。東南有洞渦水,源自樂平,下流入汾。

榆次府東南。東南有塗水,合小塗水西北流,入洞渦水。

太谷府東南。東南有馬嶺,路出北直邢臺縣,上有馬嶺關,有巡檢司。西有太谷,一名咸陽谷。東北有象穀水,流入汾。

祁府南少西。東南有胡甲山,隆舟水出焉,下流至平遙入汾。南有隆舟峪巡檢司。又東有團柏鎮。

徐溝府南。北有洞渦水,至此合汾。

清源府西南。北有清源水,東流,南入汾。

交城府西南。東北有羊腸山。東南有汾水。又西有文水。

文水府西南。西南有隱泉山。東有文水,南入汾。又東北有猷水,或以為即鄔澤也。

壽陽府東。西有殺熊嶺。南有洞渦水,黑水流合焉。

孟府東北。元孟州。洪武二年降為縣。東北有白馬山。北有滹沱河,東入北直平山縣界。東北有伏馬關,一名白馬關。又東有榆棗關。

靜樂府西北。元管州。洪武二年改為靜樂縣。東北有管涔山,汾水所出。又東北有燕京山,上有天池。又北有寧化守御千戶所,洪武二年置。又東南有兩嶺關,置故鎮巡檢司於此,後移於稍東順水村。又南有樓煩鎮巡檢司。又東北有沙婆嶺巡檢司,後移於陽曲縣天門關。

河曲府西北。元省。洪武十三年十一月復置。西有火山,臨大河。河濱有娘娘灘、太子灘,皆套中渡河險要處也。北有關河,以經偏頭關而名,西北流入大河。成化十一年十二月置偏頭關守禦千戶所,與寧武、雁門為三關。

平定州東有綿山,澤發水出焉,即冶河上源,合沾水,東流至平山縣入滹沱。西南有洞渦水,合浮化水,西流入汾。東南有新固關守御千戶所。又東有故關,即井陘關也,洪武三年置故關巡檢司於此。又有葦澤、盤石二關在縣東北,俱接井陘縣界。西北距府一百八十里。領縣一:

樂平州東南。東有皋落山,一名靈山。西南有少山,一名沾嶺,為沾水、清漳二水之發源。沾東流入澤發水,漳北流,折而西南,入和順縣之梁榆水。又西有陡泉嶺,洞渦水所出。又靜陽鎮在縣東南。

忻州洪武初,以州治秀容縣省入。北有滹沱河,又有忻水,一名肆盧川,自北流入焉。西南有牛尾莊巡檢司,後移於州北十里。又西有寨西巡檢司,西北有沙溝巡檢司,後俱廢。又忻口寨亦在州北。又東南有赤塘關。南距府百六十里。領縣一:

定襄州東少北。北有滹沱河。又南有叢象山,有三會水流合焉。東北有胡谷砦巡檢司,後廢。

代州洪武二年降為縣。八年二月復升為州。句注山在西,亦名西陘,亦曰鴈門山,其北為鴈門關,有鴈門守禦千戶所,洪武十二年十月置。又於關北置廣武營城。又東有夏屋山,一名下壺。又南有滹沱河,源自繁峙入州界,西南流經崞、忻、定襄,又東經五臺、盂,入真定界。又北有太和嶺、水勤口二巡檢司,後俱廢。西南距府三百五十里。領縣三:

五臺州東南。元臺州。洪武二年改為五臺縣。八年二月來屬。東北有五臺山,有清水河,東北流,合虒陽河,南入於滹沱。又東南有高洪口巡檢司。又東北有大谷口、飯仙山二巡檢司,後俱廢。

繁峙州東。元堅州。洪武二年改為繁峙縣。八年二月來屬。舊治在縣南,成化三年二月移治東義村。萬曆十四年十二月徙於河北之石龍崗。東北有秦戲山,滹沱河所出也,回環千三百七十里,至北直靜海縣入海。又北有茹越口、東北有北樓口、東有平刑嶺三巡檢司,後俱廢。又東有郎嶺關城,洪武十七年築。

崞州西南。元崞州。洪武二年降為縣。八年二月來屬。西南有崞山。東南有石鼓山,又有滹沱河。又西北有寧武關,有寧武守禦千戶所,景泰元年置。又有八角守御千戶所,嘉靖三年八月置。又西南有蘆板寨巡檢司。又西北有楊武峪、吊橋嶺、胡峪北口三巡檢司。

岢嵐州本岢嵐縣,洪武七年十月置。八年十一月升為州。北有岢嵐山,其東為雪山。西南有嵐漪河,北有蔚汾水,下流俱入大河。又西北有岢嵐鎮巡檢司,後廢。又北有天澗堡隘,路通朔州。西北有于坑堡隘,又有洪谷堡隘,俱通保德州。東南距府二百八十里。領縣二:

嵐州南少東。元嵐州。洪武初,降為縣。西南有黃尖山,蔚汾水所出。又北有二郎關、鹿徑嶺二巡檢司。

興州西南。元興州。洪武二年降為縣。八年十一月來屬。東北有石樓山。西濱大河,南有蔚汾水流入焉。又東有界河口、西南有孟家峪二巡檢司。

保德州洪武七年降為縣。八年十一月屬岢嵐州。九年正月復升為州西濱大河。東北有得馬水巡檢司,後廢。東南距府五百里。

平陽府元晉寧路,屬河東山西道宣慰司。洪武元年改為平陽府。領州六,縣二十八。東北距布政司五百九十里。

臨汾倚。西有姑射山。西南有平山,晉水、平水皆出於此,東流入於汾。

襄陵府西南。西南有三隥山。東有汾水,南有太平關,有巡檢司。

洪洞府北少東。東有九箕山。西有汾水。

浮山府東少南。西有浮山。北有澇水,東南有潏水,下流俱入汾。

趙城府北。元屬霍州。洪武三年改屬。西有羅雲山,又有汾水、霍水,自東南流入焉。

太平府西南。元屬絳州。洪武二年改屬。東有汾水。

岳陽府東北。東有沁水,流入澤州界。北有澗水。又南有赤壁水,西北流,會澗水入汾河。

曲沃府南。元屬絳州。洪武二年改屬。南有紫金山,產銅。北有喬山。西有汾水。西南有澮水,下流入汾。

翼城府東南。元屬絳州。洪武二年改屬。東南有澮高山,產銅,下有灤泉。又東有烏嶺山,澮水出焉。

汾西府北,少西。西有青山,產鐵。東有汾水。

蒲府西北。元屬隰州。洪武二年改屬。西有第一河,西流入大河。東有張村岔巡檢司。

靈石府北。元屬霍州。萬歷二十三年五月改屬汾州府。四十三年還屬府。東有綿山,即介山也。城北有汾水,又東有谷水流入焉。又北有靈石口巡檢司。西南有陰地關,又有汾水關。

蒲州元河中府。洪武二年改為蒲州,以州治河東縣省入。中條山在東南,即雷首山也,又名首陽山,跨臨晉、聞喜、垣曲、平陸、芮城、安邑、夏縣、解州之境。又南有歷山。又大河自榆林折而南,經州城西,又經中條山麓,又折而東,謂之河曲。臨河有風陵關巡檢司。又東南有涑水,即絳水下流,又南有媯汭水,俱注於大河。東北距府四百五十里。領縣五:

臨晉州東北。東南有王官谷。西有大河。南有涑水。又西有吳王寨巡檢司。

滎河州北少東。大河在城西,汾水至此入河。

猗氏州東北。南有涑水。東南有鹽池。

萬泉州東北。南有介山。

河津州東北。西北有龍門山,夾河對峙,下有禹門渡巡檢司。汾水舊由滎河縣北睢丘入河,隆慶四年東徙,經縣西南葫蘆灘入河。

解州洪武初,以州治解縣省入。南有檀道山,又有石錐山。東南有白徑嶺。南濱大河。東有鹽池。西北又有女鹽池。東北有長樂鎮巡檢司。東南有鹽池巡檢司。東北距府三百四十里。領縣五:

安邑州東北。西有司鹽城。北有鳴條岡。又有涑水。西南有鹽池。南有聖惠鎮巡檢司。西南有西姚巡檢司。

夏州東北。北有涑水。

聞喜州東北。東南有湯山,產銅。南有涑水。又東北有乾河,又有董澤。

平陸州東南。東北有虞山,一名吳山。又東有傅巖。南濱大河,中有底柱山。東有大陽津,上有關,亦曰茅津。有沙澗茅津渡巡檢司。又有白浪渡巡檢司。

芮城州西南。大河南經縣,西折而東。東南有陌底渡巡檢司。西北有萬壽堡。東有襄邑堡。

絳州洪武初,以州治正平縣省入。西北有九原山。南有汾水,澮水自東南流入焉。西有武平關。東北距府百五十里。領縣三:

稷山州西。南有稷神山,又有汾水。

絳州東南。東有太行山。東南有太陰山,又有陳村峪,涑水出焉,經聞喜、夏、安邑等縣,至蒲州入黃河。又西北有絳山,絳水出焉,西流入涑。又東南有教山,教水出焉,即乾河之源也。絳山產鐵。

垣曲州東南。西北有折腰山,山有銅冶。又東北有王屋山。南濱河,西有清水流入焉。又北有乾河。西北有橫嶺背巡檢司。西南有留莊隘。

霍州洪武初,以州治霍邑縣省入。東南有霍山,亦曰霍太山。西有汾水,又有霍水、彘水,俱出霍山,下流俱入汾。南距府百四十五里。

吉州西有孟門山,大河所經。西南有壺口山。又烏仁關在西,平渡關在西北,俱有巡檢司。東距府二百七十里。領縣一:

鄉寧州東南。西南有兩乳山。西有黃河。西北有龍尾磧巡檢司。

隰州洪武初,以州治隰川縣省入。西有蒲水,南入大河。東北有廣武莊巡檢司。東南距府二百八十里。領縣二:

大寧州西南。西濱大河。又東南有昕川,西注於河。西有馬鬥關,大河經其下,有巡檢司。

永和州西。西濱大河。西北有永和關,有巡檢司。又有興德關。西南有鐵羅關。三關俱與陜西濱河為界。

汾州府元汾州,屬冀寧路。洪武九年直隸布政司。萬曆二十三年五月升為府。領州一,縣七。東北距布政司二百里。

汾陽倚。元曰西河。洪武初,省入州。萬曆二十三年五月復置,更名。東有汾水。又東北有文水,一名萬谷河,自文水縣東南流入焉。西有金鎖關、黃蘆嶺二巡檢司。

教義府南少東。西北有狐岐山,勝水出焉,東流入汾。又縣南有雀鼠谷,與介休縣界,汾水自東北來經此。又西有溫泉鎮巡檢司。

平遙府東。南有麓臺山,一名蒙山,又名謁戾山。西有汾河。東有中都水,又有原祠水,合流注於汾河。又南有普同關巡檢司,後移於縣東北之洪善鎮。

介休府東南。有介山,亦曰綿山。西有汾水,東有石洞水,西流入焉。東北有鄔城泊,與平遙、文水二縣界,即昭餘祁藪之餘浸也,或亦謂之蒿澤。東南有關子嶺鎮巡檢司。

石樓府西少南。元屬晉寧路之隰州。萬曆四十年改屬。東南有石樓山。西有黃河,又有土軍川流入焉。又西北有上平關、西有永和關、東北有窟龍關三巡檢司。

臨府西北。元臨州,屬冀寧路。洪武二年降為縣。萬歷二十三年五月來屬。北濱黃河,東北有榆林河流入焉。西北有剋狐寨巡檢司。

永寧州元石州,屬冀寧路。洪武初,以州治離石縣省入。隆慶元年更名。萬曆二十三年五月來屬。大河在西。東有穀積山,下有石窟村,東川河出焉。北有赤堅嶺,一名離石山,離石水出焉,亦曰北川河,合流注於大河。又西有青龍流、北有赤堅嶺二巡檢司。又西有孟門關。東南距府百六十里。領縣一:

寧鄉州南。東南有樓子臺山。西有黃河。

潞安府元潞州,屬晉寧路。洪武二年直隸行中書省。九年直隸布政司。嘉靖八年二月升為潞安府。領縣八。西北距布政司四百五十里。

長治倚。永樂六年,沈王府自沈陽遷此。元上黨縣。洪武二年省入州。嘉靖八年二月復置,更名。東南有壺關山,舊置壺口關於山下。西南有潞水,即濁漳水,自長子縣流入,下流至河南臨漳縣,合清漳水。又西有藍水,東流與濁漳水合。

長子府西少南。東南有羊頭山。西南有發鳩山,一名鹿谷山,濁漳水發源於此。西北有藍水,南有梁水,皆流入漳水。

屯留府西北。西北有三峻山。又西南有盤秀山,藍水出乎其陽,絳水出乎其陰,下流俱合濁漳水。

襄垣府北,少西。南有濁漳水。西北有小漳水,又有涅水,自武鄉縣流入界,合小漳水,下流入濁漳水。西有五贊山巡檢司。

潞城府東北。西有三垂山。北有濁漳水,又有絳水,流合焉,謂之交漳。

壺關府東北。南有趙屋嶺,西南有大峪嶺,俱產鐵。東南有羊腸板。西北有壺水,西入濁漳。

黎城府東北。西北有濁漳水,東南入河南林縣界。東北又有清漳水,流入河南涉縣界。又東北有吾兒峪巡檢司。

平順嘉靖八年二月以潞城縣青羊里置,析黎城、壺關、潞城三縣地益之。東北有濁漳水。東南有虹梯關、玉峽關二巡檢司。

大同府元在同路,屬河東山西道宣慰司。洪武二年為府。領州四,縣七。南距布政司六百七十里。

大同倚。洪武二十五年三月建代王府。北有方山。西北有雷公山。東有紇真山。又東北有白登山。又西有大河。又南有桑乾河,自馬邑縣流經此,其下流至蔚州入北直境,為盧溝河。又西北有金河,又有紫河,皆流入大河。又西有武州山,武州川水出焉。又東有御河,一名如渾水,南有十里河流合焉,即武州川也,俗曰合河,南入於桑乾。北有威寧海子。又有孤店、開山、虎峪、白陽等口,俱在東北。又北有貓兒莊。

懷仁府西南。西有清涼山,西南有錦屏山,舊皆有鐵冶。南有桑乾河。西南有偏嶺等口。

渾源州南有恆山,即北嶽也,與北直曲陽縣界。東有五峰山。又南有翠屏山,滱水出焉,與嘔夷水合,下流為唐河。又北有桑乾河。西南有渾源川,下流入桑乾河。又東有亂嶺關、南有瓷窯口、東南有峪口巡檢司。西北距府百三十里。

應州洪武初,以州治金城縣省入。北有桑乾河。西有小石口巡檢司。東南有胡峪口巡檢司。南有茹越口巡檢司。又有北婁、大石等口,路通繁峙縣。北距府百二十里。領縣一:

山陰州西南。北有桑乾水。

朔州洪武初,以州治鄯陽縣省入。西南有翠峰山。西北有黃河。又南有灰河,下流入桑乾河。又西有武州,元屬大同路,洪武初省。北有沙凈口、西南有神池口二巡檢司。東北距府二百八十里。領縣一:

馬邑州東,少北。西北有洪濤山,水壘水出焉,俗名洪濤泉,即桑乾河上源也,至北直武清縣入海。東南有雁門關。又北有白陽。

蔚州元屬上都路之順寧府。至大元年十一月升為蔚昌府,直隸上都路。洪武二年仍為州。四年來屬,以州治靈仙縣省入。東有九宮山,又有雪山。又東南為小五臺山。北有桑乾水,東入北直保安州界。又北有壺流水,一名胡盧水,西南有滋水流入焉,下流入北直真定府界。東北有定安縣,元屬州,洪武初廢。西南有石門口,東南有神通溝鎮,東北有鴛鴦口、長寧鎮四巡檢司。又東有九宮口巡檢司,後移於州南黑石嶺。又東北有美峪口巡檢司,尋徙於董家莊。又有興寧口巡檢司。後移於北口關。西北距府三百五十里。領縣三:

廣靈州西,少北。北有九層山。東南有豐水,即葫蘆河上源也。又西南有滋水。北有平嶺關巡檢司,後徙於縣西南之林關口。

廣昌州東南。元曰飛狐,洪武初更名。東南有白石山。東有雕窠崖,舊有洞產銀。又桑乾河在北。唐河在南,即滱水也。又淶水在東,源出北崖古塔,與縣南之拒馬河合,東入北直淶水縣界。又紫荊關在東北,接北直易州界。倒馬關在南,接北直定州界。又飛狐關在北,今為黑石嶺堡,與蔚州界。

靈丘州西南。東南有隘門山,西北有槍峰嶺,即高是山也,嘔夷水出焉。又有枚回嶺,滋水出焉。

澤州元澤州,屬晉寧路。洪武初,以州治晉城縣省入。二年直隸行中書省。九年直隸布政司。東南有馬牢山。南有太行山,山頂有天井關,關南即羊腸阪。又東北有丹水,南有白水流入焉,下流注於沁河。東南有柳樹店、南有橫望嶺二巡檢司。領縣四。西北距布政司六百二十里。

高平州北少東。西北有仙公山,丹水出焉。又西南有空倉堡巡檢司。西北有長平關,又有磨磐寨。

陽城州西。西南有析城山,南有王屋山,與垣曲縣及河南濟源縣界。東有沁河,又西北有濩澤水入焉。

陵川州東北。西北有蒲水,西流入於丹水。南有永和隘巡檢司,後廢。

沁水州西北。東有沁河。又西有蘆河,下流入於沁水。西北有東烏嶺巡檢司。

沁州元屬晉寧路。洪武初,以州治銅鞮縣省入。二年直隸行中書省。九年直隸布政司。萬曆二十三年五月改屬汾州府,三十二年仍直隸布政司。西南有護甲山,涅水出焉。南有銅鞮山。正西有銅鞮水,有二流,一名小漳河,一名西漳河,下流入襄垣縣,合濁漳水。領縣二。西北距布政司三百十里。

沁源州西少南。北有綿山,沁水出焉,經縣東,下流至河南修武縣入大河,行九百七十餘里。又北有綿上巡檢司。

武鄉州東北。西有涅水,又西有武鄉水入焉。

遼州元屬晉寧路。洪武初,以州治遼山縣省入。二年直隸行中書省。九年直隸布政司。東南有太行山,洺水所出,上有黃澤嶺,嶺有十八盤巡檢司。又東有清漳水,分二流,至東南交漳村而合,南入黎城縣界。又西北有遼陽水,流合清漳水。領縣二。西北距布政司三百四十里。

榆社州西。西有榆水。西南有武鄉水。又西北有黃花嶺、馬陵關二巡檢司。

和順州北。東有黃榆嶺,北有松子嶺,西有八賦嶺,俱有巡檢司。又清漳水在西北,松嶺水及八賦水、梁榆水俱流入焉。

山西行都指揮使司本大同都衛,洪武四年正月置。治白羊城。八年十月更名。二十五年八月徙治大同府。二十六年二月領衛二十六,宣府左、右,萬全左、右,懷安五衛,改屬萬全都司。後領衛十四。朔州衛治州城,安東中屯衛寄治應州城。

大同前衛洪武七年二月置,與行都司同城。

大同後衛洪武二十五年八月置,與行都司同城,尋罷。二十六年二月復置,治行都司東,後仍徙行都司城。東有聚落城,天順三年築。嘉靖二年九月置聚落守禦千戶所於此,來屬。

大同中衛洪武二十五年八月置,與行都司同城,後罷。

大同左衛洪武二十五年八月置,與行都司同城。三十五年罷。永樂元年九月復置。七年徙治鎮朔衛城。

大同右衛洪武二十五年八月置,與行都司同城。三十五年罷。永樂元年九月復置。七年徙治定邊衛城。

鎮朔衛洪武二十六年二月置,屬行都司。永樂元年二月徙治北直薊州,直隸後軍都督府,而衛城遂虛。七年徙大同左衛來治。正統十四年又徙雲川衛來同治。東有雕嶺山。北有兔毛川,即武州川也。又西北有御河,自塞外流入,下流入於桑乾河。又北有鹽池。東北距行都司一百二十里。

定邊衛洪武二十六年二月置,屬行都司。永樂元年二月徙治北直通州,直隸後軍都督府,而衛城遂虛。七年徙大同右衛來治。正統十四年又徙玉林衛來同治。西有大青山。東北有海子窊,兔毛川出焉,分為二,其一東南流入左衛界,其一西北流自殺虎口出塞。又有南大河,經衛東南,合於兔毛川。東南距行都司一百九十里。

陽和衛元白登縣,屬大同路。洪武初,縣廢。二十六年二月置衛。宣德元年徙高山衛來同治。北有鴈門山,鴈門水出焉。南有桑乾河。西南距行都司一百二十里。

天成衛元天成縣,屬興和路。洪武四年五月改屬大同府,縣尋廢。二十六年二月置衛,後徙鎮虜衛來同治。桑乾河在南。南洋河在北,即鴈門水也,東入宣府西陽和堡界。西南距行都司一百二十里。

威遠衛正統三年三月以凈水坪置。南有大南山。西有小南山。又南有南大河,下流入於兔毛川。東距行都司一百八十里。

平虜衛成化十七年置,與行都司同城。嘉靖中徙今治。西有小青山,又有黃河自東勝衛流入。北有南大河。西北有雲內縣,本元雲內州,屬大同路,洪武五年廢。宣德中復置縣,屬豐州,正統十四年復廢。西北有平地縣,元屬大同路,亦洪武中廢。東北距行都司二百四十里。領千戶所一:

井坪守禦千戶所成化二十年七月置。

雲川衛洪武二十六年二月置,屬行都司。永樂元年二月徙治北直畿內,直隸後軍都督府。宣德元年還舊治,仍屬行都司。正統十四年徙治舊鎮朔衛城,與大同左衛同治,而衛城遂虛。東距行都司二百十里。

玉林衛洪武二十六年二月置,屬行都司。永樂元年二月徙治北直畿內,直隸後軍都督府。宣德元年還舊治,仍屬行都司。正統十四年徙治舊定邊衛城,與大同右衛同治,而衛城遂虛。東有玉林山,玉林川出焉。東距行都司二百四十里。

鎮虜衛洪武二十六年二月置,屬行都司。永樂元年二月徙治北直畿內,直隸後軍都督府。宣德元年還舊治,仍屬行都司。正統十四年徙治天成衛城,與天成衛同治,而衛城遂虛。東距行都司百十里。

高山衛洪武二十六年二月置,屬行都司。永樂元年二月徙治北直畿內,直隸後軍都督府。宣德元年徙陽和衛城,與陽和衛同治,仍屬行都司,而衛城遂虛。嘉靖二年九月置高山守禦千戶所於此,屬大同前衛。東有高山。西有兔毛川。東距行都司三十里。

宣德衛元宣寧縣,屬大同路。洪武中,縣廢。二十六年二月置宣德衛,後廢。東南距行都司八十里。

東勝衛元東勝州,屬大同路。洪武四年正月,州廢,置衛。二十五年八月分置東勝左、右、中、前、後五衛,屬行都司。二十六年二月罷中、前、衛三衛。永樂元年二月徙左衛於兆直盧龍縣,右衛於北直遵化縣,直隸後軍都督府。三月置東勝中、前、後三千戶所於懷仁等處守禦,而衛城遂虛。正統三年九月復置,後仍廢。北有赤兒山。西有黃河。西北有黑河,源出舊豐州之官山,西流入雲內州界,又東經此入於黃河。又有兔毛川,亦入於黃河。又有紫河,源出舊豐州西北之黑峪口,下流至云內州界,入於黑河。又西有金河泊,上承紫河,下流亦入於黃河。西北有豐州,元屬大同路,洪武中廢,宣德元年復置;正統中內徙,復廢。又有凈州路,元直隸中書省,亦洪武中廢。西距行都司五百里。領千戶所五:失寶赤千戶所、五花城千戶所、乾魯忽奴千戶所、燕只千戶所、甕吉刺千戶所,俱洪武四年正月置。


○河南陜西

河南《禹貢》豫、冀、揚、兗四州之域。元以河北地直隸中書省,河南地置河南江北行中書省。治汴梁路。洪武元年五月置中書分省。治開封府。二年四月改分省為河南等處行中書省。三年十二月置河南都衛。八年十月改都衛為都指揮使司。九年六月改行中書省為承宣布政使司。府八,直隸州一,屬州十一,縣九十六。為里三千八百八十有奇。北至武安,與北直、山西界。南至信陽,與江南、湖廣界。東至永城,與山東、江南界。西至陜州,與山西、陜西界。距南京一千一百七十五里,京師一千五百八十里。洪武二十六年編戶三十一萬五千六百一十七,口一百九十一萬二千五百四十二。弘治四年,戶五十七萬五千二百四十九,口四百三十六萬四百七十六。萬曆六年,戶六十三萬三千六十七,口五百一十九萬三千六百二。

開封府元汴梁路,屬河南江北行省。洪武元年五月曰開封府。八月建北京。十一年,京罷。領州四,縣三十:

祥符倚。洪武十一年正月建周王府。大河舊在城北。正統十三年,河決滎陽,東過城西南,而城遂在河北。東為開封縣,元時同治郭內,洪武中省。南有朱仙鎮。東北有陳橋鎮。

陳留府東少南。北有大河。東北有睢水,下流至南直宿遷縣合泗水。

巳府東南。北有睢水,又有舊黃河,洪武二十五年河決之故道也。嘉靖三十六年,全河合淮入海,而縣遂無河患。

通許府東南。西南有故黃河,弘治後北徙,不經縣界。

太康府東南。北有渦水,自通許縣流入,下流至南直懷遠縣入淮。東有馬廠集,正統十三年河決,自巳縣經此。

尉氏府南少西。西南有大溝,東北合康溝,入於黃河。

洧川府西南。南有故城,洪武二年以河患遷今治。又南有洧水,下流至西華縣合潁水。東南有南席店,弘治九年,河入栗家口,南行經此。

鄢陵府南少西。北有洧水。

扶溝府南少東。東有沙河,一名惠民河,又名小黃河,即宋蔡河故道也。成化中浚,下流達南直太和縣界。又北有洧水,自西流入焉。又東北有黃河故道,弘治二年淤。

中牟府西。東有故城,天順中,徙今治。大河在縣北。又有汴河,舊自滎陽而東,下流經祥符縣南,又東南至南直泗州入於淮。正統六年改從此入河,後淤。西北有圃田澤。

陽武府西北。北濱大河,自此至南直徐州,大河所行,皆唐、宋汴河故道。

原武府西北。北有黑陽山,下臨大河。洪武二十四年,河決於此。正統十二年復決焉。東南有安城縣,洪武初置,正統中廢。

封丘府北。南有大河。西南有荊隆口,一名金龍口。弘治二年、五年,萬曆十五年,崇禎四年、五年,河屢決於此。又西北有沁河,弘治六年淤。西南有中欒鎮巡檢司。

延津府西北。大河舊經縣北。成化十四年,河決,徙流縣南,而縣北之流遂絕。西北有沙門鎮,弘治十一年移項城縣西之香臺巡檢司於此。

蘭陽府東少北。北濱大河,有李景高口。萬歷十七年,河決於此。

儀封府東少北。元屬睢州。洪武十年五月改屬南陽府,後來屬。故城在縣北,洪武二十二年二月圮於河,徙日樓村,即今治也。東北有黃陵岡,大河舊經其下,入曹縣界。弘治五年,河決於此,尋塞之,改徙岡南入睢州界。又賈魯故河亦在縣北,正德四年,河決入焉。

新鄭府西南。元屬均州。隆慶五年七月改屬。西南有大隗山,一名具茨山,水異水出焉,一名魯固河,下流入潁。又南有陘山。北有大河。又有溱水,一曰澮水,流合縣南之洧水。

陳州洪武初,以州治宛丘縣省入。南有潁水。又西有沙水,亦曰小黃河,至潁岐口,與潁水合,下流分為二。崇禎間,屢決於西南之苑家埠口。又南有故黃河,喜靖時,黃河南出之道也。西北距府二百六十五里。領縣四:

商水州西南。洪武初廢。四年七月復置。北有潁水,又有水隱水,亦曰大水隱水。

西華州西少北。北有潁水,又有沙水,即小黃河也。西南有水隱水,又有常社鎮巡檢司。

項城州南。東北有故城。今治本南頓縣之殄寇鎮也,宣德三年遷。東有潁水,西有溵水流入焉。洪武二十四年,大河自陳州經縣界合潁,下入於淮。永樂九年,河始復故道。又東北有沙水。

沈丘州東南。元屬潁州。洪武初廢。弘治十年改乳香臺巡檢司置,來屬。東北有潁水,東入南直潁州界。又北有沙河,東入南直太和縣界。又東有界首巡檢司。又北有南頓縣,洪武初廢。景泰初,置南頓巡檢司於此。

許州洪武初,以州治長社縣省入。西有潁水。北有水異水。又東有東湖,一名秋湖。又西北有石固鎮,與長葛縣界。東北距府二百二十里。領縣四:

臨潁州東南。西有潁水,水異水自縣北流入焉。又西南有小水隱水。

襄城州西南。南有首山。東北有潁水。南有汝河。

郾城州東南。南有沙水,亦曰大溵水,上流即故汝水也,又東南有澧水來入焉。

長葛州西北。北有洧水。西有水異水。

禹州元曰鈞州。洪武初,以州治陽翟縣省入。萬曆三年四月避諱改曰禹州。成化二年七月建徽王府。嘉靖三十五年除。北有禹山,又西北有礦山,有鐵母山,舊俱產鐵。又北有潁水,下經襄城,一名渚水,至臨潁合沙河。東北距府三百二十里。領縣一:

密州西北。南有洧水,又有溱水。

鄭州洪武初,以州治管城縣省入。西南有梅山,鄭水出焉,下流舊入汴水,後堙。又西有須水,源出滎陽縣,舊亦入於汴水。正統八年嘗浚以分決河之流,後亦堙。東北距府百四十里。領縣四:

滎陽州西。南有大周山,汴水出焉。又東南有嵩渚山,京水出焉。又有索水,源出小徑山,北流與京水合,下流入於鄭水。又大河在北。東有須水鎮,崇禎十年築城。

滎澤州北少西。元直隸汴梁路。洪武中,改屬州。北有故城。洪武八年因河患徙於南。成化十五年正月又徙北,濱大河。東南有孫家渡,正統十三年,大河決於此。

河陰州西北。舊治在大峪口,洪武三年為水所圮,徙於此。東北有廣武山,與三皇山連。西有敖倉,北濱大河。

汜水州西。故城在縣東,洪武十一年七月徙於成皋。崇禎十六年又遷西北。北濱河,洛水自西,東至滿家溝合汜水入焉。又西有虎牢關,洪武四年九月改曰古崤關,有巡檢司。

河南府元河南府路,屬河南江北行中書省。洪武元年為府。領州一,縣十三。東距布政司三百八十里。

洛陽倚。洪武二十四年建伊王府。嘉靖四十三年廢。萬歷二十九年十月建福王府。北有北邙山,西南有闕塞山,亦曰闕口山,亦曰伊闕山,俗曰龍門山。又西北有穀城山,亦曰簪亭山,湹水所出。又東南有大谷,穀口有關。又大河在北。又有洛水,源自洛南塚嶺山,東經盧氏、永寧諸縣,至洛陽、偃師、鞏縣入於河。又東有伊水,自盧氏縣東北流至偃師縣而入洛。又北有朅水,西有澗水,俱流會於洛。又西南有孝水。

偃師府東少北。南有緱氏山。又有洛水,西有伊水流合焉。

鞏府東北。西南有軒轅山,上有關。北濱河。西北有洛水,舊經縣北入河,謂之洛汭,亦曰洛口。嘉靖後,東過汜水縣入河。又南有鄩水,會洛入河,亦曰鄩口也。又東南有石子河,西南有長羅川,皆流入洛水。又西南有黑石渡巡檢司。

孟津府東北。舊治在縣東,今治本聖賢莊,嘉靖十四年七月遷於此。西北有大河。又西有硤石津,又西有委粟津,又有高渚、馬渚、陶渚,皆大河津濟處。東北有孟津巡檢司。

宜陽府西南。西有女幾山。東南有鹿蹄山,一名非山,甘水出焉。又北有洛水。西有宜水,又有昌谷水,與甘水俱流注於洛。又西南有趙保鎮、木冊鎮二巡檢司。

永寧府西南。北有崤山,崤水出焉,北注於河。其東曰穀陽谷,穀水所出焉。又南有洛水。東北有刀軒川,下流為昌谷水。又有大宋川,下流為宜水。又西有崇陽鎮、又有高門關、東有崤底關三巡檢司。

新安府西。西有缺門山。北有大河。又南有澗水,穀水自北流入焉。東有慈澗水,亦流入穀水。又有函谷新關。

澠池府西。元屬陜州。洪武中改屬。東北有廣陽山,亦曰澠池山,北溪水出焉。又有白石山,澗水所出。西北濱河。南有穀水。又西北有南村巡檢司。

登封府東南。北有嵩山,即中嶽也,亦曰太室山。又西有少室山,潁水中源出焉;又有右源,出於山之南溪,又有左源,出於西南之陽乾山,合流至南直壽州入淮。又北有陽城山,洧水所出,下流至扶溝縣入沙河。又東南有崿嶺,即箕山也,上有崿阪關。又東南有五渡水,流入潁,亦曰三交水。又西南有少陽河,亦流入潁。

嵩府西南。元嵩州,屬南陽府。洪武二年四月降為縣,來屬。三塗山在西南。陸渾山在東北。又東有篩山,北有露寶山,西有大礦山,皆產錫。西南有伏牛山,即天息山也,山有分水嶺,汝水出焉,下流至南直潁州入淮,行千三百五十餘里。又南有伊水,西北有高都川流入焉。又西南有舊縣鎮巡檢司。西有沒大嶺巡檢司。

盧氏府西南。元屬嵩州。洪武元年四月屬南陽府。三年三月屬陜州。萬歷初,改屬府。西南有熊耳山,洛水自陜西商州流入境,經此。東南有巒山,一名悶頓嶺,伊水所出。北有鐵嶺,東澗水出焉,東南入洛。又東北有馬回川,亦入於洛。又東南有欒州鎮、西南有硃陽鎮、北有杜管鎮三巡檢司。又西有白華關。

陜州元屬河南府路。洪武元年四月改屬南陽府,以州治陜縣省入。東有底柱山,在大河中。山有三門,中曰神門,南曰鬼門,北曰人門,惟人門修廣可行舟,鬼門最險。又南有橐水,一名永定澗,亦曰漫澗,西北入河。又東南有硤石關,有巡檢司。又有鴈翎關。東距府三百里。領縣二:

靈寶州西少南。北濱河。又西有弘農澗。南有虢略鎮巡檢司。又有函谷故關。西南又有洪關。

閿鄉州西南。東南有誇父山,一名秦山,中有大谷關。北濱河,自山西芮城縣流入,東南至永城縣,入南直碭山縣界。西有湖水,又有盤澗水北流入焉。又西有潼關,與陜西華陰縣分界。

歸德府元直隸河南江北行省。洪武元年五月降為州,屬開封府。嘉靖二十四年六月升為府。領州一,縣八。西距布政司三百五十里。

商丘倚。元曰睢陽。洪武初省。嘉靖二十四年六月復置,更名。舊治在南,弘治十五年圮於河,十六年九月遷於今治。北濱河。正統後,河決而南。城嘗在河北,正德後,仍在河南。北有丁家道口巡檢司。東南有武津關巡檢司。

寧陵府西。南有睢水。北有桃源集巡檢司。

鹿邑府南。元屬亳州。洪武中改屬。南有潁水,又蔡河自西流入,謂之蔡河口,即沈丘縣之沙河也。又北有渦水,東流入南直亳州境。

夏邑府東。元曰下邑,洪武初更名。北濱大河。又東南有睢水。

永城府東南。洪武元年五月屬開封府。十一月來屬。北有碭山,又有芒山,皆與南直碭山縣界,又睢水、澮水皆在縣南。又南有泡水,弘治間淤塞。

虞城府東北。元屬濟寧路。洪武二年正月來屬。南有故城。嘉靖九年遷於今治。北有黃河。

睢州元屬汴梁路。洪武初,屬開封府,以州治襄邑縣省入。十年五月降為縣。十三年十一月復升為州。嘉靖二十四年六月來屬。北濱河。又有睢水亦在州東北。東距府百七十里。領縣二:

考城州北。元末省。洪武四年八月復置,屬開封府。十年五月復省。十三年十一月復置,屬州。舊治在縣東南。正統十三年徙。北濱大河。

柘城州東南。元末省。洪武四年八月復置,屬開封府。十年五月省入寧陵縣。十三年十一月復置,屬州。北有睢水。南有渦水。

汝寧府元直隸河南江北行省。洪武初,因之。領州二,縣十二。距布政司四百六十里。

汝陽倚。天順元年三月建秀王府,成化八年除。十年建崇王府。洪武初,縣廢,四年七月復置。北有汝水,源出天息山,東流入境,過新蔡東南入淮。又南有澺水,又有汶水,又有溱水,又西北有犋水,俗名泥河,下流俱入於汝。又城南有柴潭。東有陽埠巡檢司。

真陽府東。元屬息州。洪武四年省入汝陽縣。景泰四年置真陽鎮巡檢司於此。弘治十八年十二月仍置縣,而徙巡檢司於縣南銅鐘店,仍故名,尋廢。南有淮水。又汝水在縣東,北有滇水流入焉。

上蔡府北。洪武初廢,四年五月復置。西有汝水,西南有沙水流合焉。

新蔡府東少南。元屬息州,後廢。洪武四年五月復置,改屬。南有汝水,又澺水自城北流合焉。又東北有瓦店巡檢司。

西平府西北。北有汝水,源出縣西南雲莊、諸石二山。自元末堨斷故汝,而此水遂為汝源。嘉靖九年復塞,改為洪河之上流。

確山府西南。洪武十年五月省入汝陽縣,十三年十一月復置。成化十一年九月改屬信陽州。弘治二年八月仍屬府。西北有郎山,亦曰樂山。北有黃酉河,下流為練河,流入汝。又西有竹溝巡檢司。南有明港巡檢司。

遂平府西少北。西南有查牙山,其東南相接者曰馬鞍山。又西有洪山,龍陂之源出焉,自西平縣雲莊諸山之水既塞,遂以此為汝源。南有灈水,又有沙河,又北有石洋河,其下流皆入於汝。

信陽州元為信陽縣,屬信陽州,後廢。洪武元年十月置信陽州於此,屬河南分省。四年二月屬中都臨濠府。七年八月改屬。十年五月降為縣。成化十一年九月復升為州。西南有賢首山。南有士雅山,又有峴山。東南有石城山,亦曰冥山。北有淮水,又南有溮水流入焉。東北距府二百七十里。領縣一:

羅山州東。元信陽州治,後州縣俱廢。洪武元年十月置州於舊信陽縣,復置羅山縣屬焉。十年五月直隸汝寧府。成化十一年九月還屬州。北有淮水,又南有小黃河入焉。東南有大勝關巡檢司,與湖廣黃陂界。西南有九里關,好黃峴關,義陽三關之一,有巡檢司,與湖廣應山縣界。

光州洪武初,以州治定城縣省入。四年二月改屬中都臨濠府。十三年仍來屬。北有淮水。又南有潢水,北流入淮水。西南有陰山關。西北距府三百里。領縣四:

光山州西南。南有石盤山。北濱淮。南有潢水,亦曰官渡河。又南有木陵關。西南又有白沙、土門、斗木嶺、黃土嶺、修善衝等五關,與湖廣麻城縣界。東南有牛山鎮巡檢司,後移於長潭。又有沙窩鎮巡檢司,後廢。

固始州東北。南有白鹿崖。北濱淮。東有史河,西有淠河,俱入南直霍丘縣界,下流入淮。又東北有朱皋鎮,與南直潁州界,有巡檢司。

息州西北。元息州,洪武四年二月屬中都臨濠府。尋降為縣,屬潁州。七年仍來屬。南濱淮。東北有汝水。北有楊莊店巡檢司,後移於縣東北之固城倉。

商城州東南。成化十一年四月析固始縣地置。南有金剛臺山。又東南有竹根山。東有大蘇山,灌水出焉,流入南直霍丘縣。又東有牛山河,即史河上源也。西南有五水關河。又南有五河,下流俱入於史河。又南有金剛臺巡檢司,本置金剛臺山下,嘉靖二十七年移於縣東南之水東案。又南有長嶺關,東南有松子關,俱接湖廣羅田縣界。

南陽府元直隸河南江北行省。洪武初,因之。領州二,縣十一。距布政司六百八十里。

南陽倚。洪武二十四年建唐王府。城南有精山。北有百重山、雉衡山。又有分水嶺,其水北流入於汝水,南流入於淯水。西南有臥龍岡。東有淯水,一名白河,下流至湖廣襄陽縣界入漢水。西南有湍水,西北有洱水,皆流入淯水。

鎮平府西。洪武十年五月省入南陽縣。十三年十一月復置。西北有五朵山,產銅。東有潦河,流入淯河。

唐府東南。洪武三年以故比陽縣地置。南有唐子山。東北有大狐山,亦曰壺山,沘水所出。又西有黃淳水,又有泌水,下流皆入淯水。又東北有石夾口關。

泌陽府東。元為唐州治。洪武二年二月省入州。十三年十一月,州廢,復置縣。東有銅山,泌水出焉。又北有潕水,東北有瀙水,下流俱入汝水。又象河關在縣東北,有巡檢司。

桐柏府東南。本唐縣之桐柏鎮巡檢司。成化十二年十二月改置縣,而移巡檢司於毛家集。東有桐柏山,淮水所經,下流至南直安東縣入海,行二千三百餘里。又東有大復山。西北有胎簪山,淮水所出。又西有澧水,亦曰醴水,下流入泌水。

南召府北。成化十二年十二月以南陽縣南召堡置。北有丹霞山,一名留山。北有魯陽關,即三鴉路口也,與魯山縣界。有鴉路鎮巡檢司,成化十二年十二月移於窪石口。

鄧州元治穰縣。洪武二年二月,縣廢。十三年十一月復置縣。十四年五月復省入州。南有析隈山。西北有白崖山。北有湍水,又東有涅水,亦名趙河,自北來入焉。東北距府百二十里。領縣三:

內鄉州北少西。東有熊耳山,湍水所出。西南有淅水,又有丹水。又北有菊潭。東北有金斗山巡檢司,後廢。又西北有西硤口關巡檢司。又西南有黨子口關。又西有武關,路出陜西商州。

新野州東南。西有清水,又有湍水,又北有沘水,東有棘水,皆流入於淯水。

淅川州西。成化六年析內鄉縣地置。東南有太白山。又有丹崖山。東有均水,又西南有淅水,北有丹水俱流入焉,南入於漢水。西北有花園頭巡檢司,又有荊子口關。又西有峽口鎮,南接湖廣均州界。

裕州洪武初,以州治方城縣省入。東北有方城山,渚水出焉,下流入沘水。西南距府百二十里。領縣二:

舞陽州東北。汝水在縣北,舊入西平縣界,元末於渦河堨斷其流,使東歸潁,而西平之水始別為汝源。南有潕水,亦曰舞水,又有瀙水,下流俱入於汝寧府之汝水。西南有沙水,即水也。又北有澧水,下流歸故汝水。

葉州北少東。北有黃城山,一名長城山,有汝水。又北有湛水,流入汝。東北有沙水,一名水,又名泜水,又北有昆水入焉,下流入於汝。又北有昆陽關。

懷慶府元懷慶路,直隸中書省。洪武元年十月為府,屬河南分省。領縣六。東南距布政司三百里。

河內倚。永樂二十二年建衛王府。正統三年除。八年,鄭王府自陜西鳳翔府遷此。北有太行山,又有碗子城山,上有關。又有沁河,源出山西沁源縣,流入府境,下流至武陟入大河。又有丹河,自澤州流入,注於沁河。又西有柏鄉城,崇禎四年築。

濟源府西。元屬孟州。洪武十年五月改屬府。南濱大河。西有王屋山,接山西垣曲縣界,濟水出焉。西北有琮山,溴水出焉。又東北有沁水,經兩山之間,一名枋口水。又西北有軹關。西有邵原鎮巡檢司。

修武府東少北。西有沁水。

武陟府東。大河在縣南。東有沁河,至南賈口入焉。又東北有蓮花池,萬曆十五年,沁河決此。又西北有寧郭城,景泰中築。

孟府南少西。元孟州。洪武初,以州治河陽縣省入。十年五月降為縣。西南濱大河。

溫府東南。元屬孟州。洪武十年五月改屬府。南濱大河,溴水自西北流入焉。又西南有濟水,舊自濟源縣流經沇河鎮,南注於河,後其道盡入河中。

衛輝府元衛輝路,直隸中書省。洪武元年八月為府。十月屬河南分省。領縣六。東南距布政司一百六十里。

汲倚。弘治四年八月建汝王府。嘉靖二十年除。隆慶五年二月建潞王府。北有衛河,源出輝縣,下流至北直靜海縣入海,行二千餘里,又東北有淇門鎮。

胙城府東少南。洪武十年五月省入汲縣。十三年十一月復置。

新鄉府西南。北有衛河。西北有清水。又西南有大河故道,正統十三年河決縣之八柳樹由此,尋塞。西有古沁河,永樂十三年後,時決時涸。

獲嘉府西少南。洪武十年五月省入新鄉縣,十三年十一月復置。大河舊在縣南。天順六年中,河自武陟徙入原武,而縣界之流絕。北有清水,又有小丹河合焉。

淇府北。元淇州,後廢。洪武元年九月復置。十二月降為縣。西北有淇水,又清水自東北流入焉,下流入於衛河。

輝府西北。元輝州,後廢。洪武元年九月復置。十二月降為縣。西有太行山。西北有白鹿山。又有蘇門山,一名百門山,山有百門泉,泉通百道,其下流為衛水,故又名衛源。又西南有清水。又西北有侯趙川、西有鴨子口二巡檢司。

彰德府元彰德路,直隸中書省。洪武元年閏七月為府。十月屬河南分省。領州一,縣六。南距布政司三百六十里。

安陽倚。永樂二年四月建趙王府。元末,縣廢。洪武元年九月復置。東北有韓陵山。西北有銅山,舊產銅。北有安陽河,本名洹水,自林縣流入,至北直內黃縣入衛河。又北有濁漳水。

臨漳府東北。元末廢。洪武元年九月復置。西有清、濁二漳水,合流於此,曰交漳口,入北直界。又有滏水,下流入於漳河。西南又有洹水。

湯陰府南。元末廢。洪武元年九月復置。西有蕩水,經縣治北,下流入衛水。

林府西,少南。元林州,後廢。洪武元年九月復置。二年四月降為縣。西北有隆慮山,亦曰林慮,洹水出焉。又西南有天平山。西有太行山。又北有濁漳水,自山西平順縣流入。

磁州元治滏陽縣,屬廣平路,後州縣俱廢。洪武元年十一月復置州,屬廣平府。二年四月來屬。西北有神麇山,滏水出焉。又南有清漳水。北有車騎關巡檢司。南距府七十里。領縣二:

武安州西北。元末廢。洪武元年十一月復置。東南有滏山,滏水出焉。西南有磁山,產磁石。東北有洺河,流入北直邯鄲縣界。又西有固鎮巡檢司。

涉州西少北。元屬真定路,後廢。洪武元年十一月復置,屬真定府。二年四月來屬。南有涉水,即清漳水也,自山西黎城縣流入。又東北有偏店巡檢司,後移於縣西南之吾而峪口。

汝州元屬南陽府。洪武初,以州治梁縣省入。成化十二年九月直隸布政司。東南有霍山。又有魚齒山,涉水出於此,入葉縣界。又西南有鳴皋山。又有空峒山。南有汝水。西有廣成澤。領縣四。東北距布政司四百九十里。

魯山州西南。東有魯山。西有堯山,水所出,西南有波水流入焉。又西北有歇馬嶺關巡檢司。

郟州東少南。東南有汝水,西有扈澗水流入焉。

寶豐州東南。成化十一年四月析汝州地置。南有汝水,又有水。

伊陽州西少南。成化十二年十二月以汝州之伊闕故縣置,析嵩及魯山二縣地益之。西有伊陽山。又有堯山,即天息山也,上有分水嶺,水出焉,俗又名沙水。又南有汝水。西有伊水。西南有上店鎮巡檢司,成化十二年十二月移於常界嶺。又有普浗關巡檢司,廢。

陜西《禹貢》雍、梁二州之域。元置陜西等處行中書省,治奉元路。又置甘肅等處行中書省。治甘州路。洪武二年四月置陜西等處行中書省。治西安府。三年十二月置西安都衛。與行中書省同治。八年十月改都衛為陜西都指揮使司。九年六月改行中書省為承宣布政使司。領府八,屬州二十一,縣九十有五。為里三千五百九十七。東至華陰,與河南、山西界。南至紫陽,與湖廣、四川界。北至河套,西至肅州。外為邊地。距南京二千四百三十里,京師二千六百五十里。洪武二十六年編戶二十九萬四千五百二十六,口二百三十一萬六千五百六十九。弘治四年,戶三十萬六千六百四十四,口三百九十一萬二千三百七十。萬曆六年,戶三十九萬四千四百二十三,口四百五十萬二千六十七。

西安府元奉元路,屬陜西行省。洪武二年三月改為西安府。領州六,縣三十一:

長安倚。治西偏。洪武三年四月建秦王府。北有龍首山。南有終南山。西南有太一山,又有子午谷,谷中有關。北有渭水,源出鳥鼠山,流經縣界,至華陰入黃河。又西有灃水。又西北有鎬水,合滮水,又南有潏水,亦曰水穴水,合澇水,俱北流入渭。

咸寧倚。治東偏。渭水在南。東有滻水,合霸水流入渭。

咸陽府西北。舊治在渭河北,洪武二年徙於渭南。東北有涇水,東入渭。東南有灃水,北入渭。

涇陽府北。西北有甘泉山。南有涇水,源自開頭山,流經縣界,至高陵縣入謂。又北有冶谷水,合清谷水,下流入謂。

興平府西少北。南有渭水。

臨潼府東少北。東南有驪山,有溫泉。北有渭水。西有潼水,又東有戲水,俱北入渭。又東有泠水,一曰零水,至零口鎮亦入渭。又南有煮鹽驛,舊產鹽。

渭南府東。元屬華州。嘉靖三十八年十一月改屬府。北有渭水。

藍田府東南。南有七盤山,旁有糸爭坡,謂之七盤十二糸爭,藍關之險道。又有嶢山。東南有藍田山,有關。西有霸水,西北有長水,亦曰荊溪,又南有輞谷水,亦曰輞川,俱注於霸水。

鄠府西南。南有牛首山,澇水出焉。北有渭水。西南有甘泉,西有水美陂,俱流合澇水,注於渭。又灃水在南,合高觀谷、太平谷諸水,入長安縣界。

盩厔府西南。西南有駱谷,谷長四百二十里,穀口有關。谷中有十八盤、又有柴家關二巡檢司。北有渭水。南有龍水,西南有黑水流入焉。又東有駱谷水,東南有芒水,並北入謂。

高陵府東北。西南有渭水,涇水自西北流合焉。

富平府東北。元屬耀州。萬歷三十六年改屬府。西南有荊山。西北有漆沮水,舊經白水縣南入洛,自鄭渠堙廢,不復東入洛矣。東北有美原巡檢司,尋廢。

三原府北少東。元屬耀州。弘治三年十一月改屬府。西北有堯門山。東北有漆沮水。西有清水,下流注於渭。

醴泉府西北。元屬乾州。嘉靖三十八年十一月改屬府。西北有九峻山,又有武將山。東有涇水,又有甘谷水,流合焉。

華州南有少華山。北有渭水,與同州界。西有赤水,分大小二流,又有石橋水,俱北注渭。西距府二百里。領縣二:

華陰州東。南有華山,亦曰太華,即西嶽也。東有牛心谷。西南有車箱谷。東北有大河,自朝邑縣流入,至渭口,與渭水合,所謂渭汭也。南有敷水,北入渭。東北有潼水,入於大河。東有潼關。洪武七年置潼關守禦千戶所。九年十一月升為衛,屬河南都司。永樂六年直隸中軍都督府。

蒲城州西北。東有洛水。又西有西鹵池,南有東鹵池,舊產鹽。

商州洪武七年五月降為縣。成化十三年三月仍為州。東南有商洛山。西有熊耳山,伊水所出。南有丹崖山,舊產銅。又有冢嶺山,洛水所出,下流至河南汜水縣入大河。又南有丹水,流入河南內鄉縣界。東有武關、西有秦嶺二巡檢司。又東有龍駒寨。西北距府二百二十里。領縣四:

商南州東少南。成化十三年三月以商縣之層峰驛置,尋徙治於沭河西。西南有兩河,即丹水也,東有沭河,南有挾川,俱入焉。東有富水堡巡檢司。

雒南州北少東。元曰洛南,屬商州。洪武七年五月改屬華州。成化十三年三月復來屬。天啟初,改洛為雒。東北有魚難山,魚難水出焉,西北有玄扈山,玄扈水出焉,俱北入於洛。東南有三要、東北有石家坡二巡檢司。

山陽州南少東。本商縣之豐陽巡檢司,成化十二年十二月改為縣,而移巡檢司於縣東南之漫川里,仍故名。東南有天柱山。西南有甲河,流入湖廣上津縣界,注於漢水。又東有竹林關巡檢司。

鎮安州西南。景泰三年以咸寧縣野豬坪置,屬府。天順七年二月遷治謝家灣。成化十三年三月改屬州。西有泎水,合縣南洵水入洵陽縣界,注於漢江。北有舊縣、西有五郎壩二巡檢司。

同州北有商原。南有渭水。西南有沮水,一名洛水。西南距府二百六十里。領縣五:

朝邑州東。東有大河。南有渭水。又有洛水,舊自縣南經華陰縣西北葫蘆灘入謂;成化中,自縣南趙渡鎮徑入於河,不復入渭。東北有臨晉關,一名大慶關,即浦津關也,舊屬浦州,洪武九年八月來屬。有浦津關巡檢司。

郃陽州東北。東有黃河。

韓城州東北。西有梁山,一名呂梁山,濱大河。東北有龍門山,夾河對峙。

澄城州北,西有洛水。

白水州西北。南有故城。洪武初,徙於今治。西有洛水,白水流入焉。西北有馬蓮灘巡檢司。

耀州東有沮水,西有漆水流入焉。又有清水,流入三原縣界。南距府百八十里。領縣一:

同官州東北。北有神水峽,峽內有金鎖關巡檢司。又西北有北高山,漆水出焉,東南流與同官川水合。又東有沮水,南有安公谷水,其下流合於沮水。

乾州西北有梁山,接岐山縣界。其南有漠谷,漠谷水經其下,流為武水。又東北有甘谷水。又西有武亭水。東南距府百六十里。領縣二:

武功州西南。西南有太白山,又有武功山。東南有心享物山。南有渭水。又西有漠谷水,又有武亭水,自縣東北流合焉。俱匯於湋水。

永壽州北。東有涇水。西南有錦川河,下流為漠谷水。有土副巡檢司。又有穆陵關。

邠州元直隸陜西行省。洪武中來屬,以州治新平縣省入。北有涇水。西南有白土川,亦名漆水,東南注於渭水。與入洛之漆異。東南距府三百五十里。領縣三:

淳化州東。南有黃嶔山。西有涇水。東有清水,南流入耀州界。

三水州東北。成化十三年九月析淳化縣地置。東南有石門山。東有三水河,一名汃水,西南流入涇水。東南有石門巡檢司。

長武州西北。萬曆十一年三月以邠州宜祿鎮置。北有涇水,自涇州流入。南有汭水,一名宜祿水,亦自涇州流入,徑縣東停口鎮,與黑水河合,入於涇水。西有窯店巡檢司,本名宜祿,治宜祿鎮。弘治十七年遷於正東之冉杏,仍故名。萬歷十一年又遷,更名。

鳳翔府元屬陜西行省。洪武二年三月因之。領州一,縣七。東距布政司三百四十里。

鳳翔倚。永樂二十二年建鄭王府。正統八年遷於河南懷慶府。東北有杜陽山,杜水所出。西北有雍山,雍水出焉,下流合漆水入渭。又東南有橫水,亦曰橫渠,東入渭。

岐山府東。東北有岐山。又有梁山。又北有武將山。南有渭水,西北有岐水,又東有湋水,俱流入扶鳳縣界。又南有斜谷水,北入渭。

寶雞府西南。東南有陳倉山。西南有大散嶺,大散關在焉。又有和尚原,接鳳縣界。南有渭河,東有汧河流入焉。又東南有箕谷水,有洛谷水,俱北入渭。西南有益門鎮二里散關、東南有虢川二巡檢司。又東南有金牙關。

扶鳳府東。西南有渭河。東有漆河,又有雍水自東南流入焉,又南有湋河,俱流入武功縣界。

郿府東南。元屬奉元路。洪武二年來屬。西有衙嶺山,褒水出其南,流入沔,斜水出其北,流入渭。西有五丈原。又西南有斜谷,南入漢中,有斜谷關。北有渭水。

麟遊府東北。西有漆水,南有麟遊水,下流俱入於渭。西南又有杜水,亦曰杜陽川,東入漆。西北有石窯關巡檢司。

汧陽府西少北。元屬隴州。嘉靖三十八年十一月改屬府。舊治在縣西,嘉靖二十七年徙於今治。南有汧河。

隴州元屬鞏昌總帥府。延祐四年十一月省州治汧源縣入州。洪武二年來屬。西北有隴山,上有關曰隴關,亦曰大震關,一名故關,有故關大寨巡檢司。又有安夷關,亦曰新關。又西有小隴山,一名關山。又西南有岍山,汧水出焉。南有吳山,即吳嶽,古文以為岍山。西南有白環谷,白環水出焉。西有弦蒲藪,汭水出焉,下流合於涇水。南有渭水。西南有方山原。又南有隴安、西南有香泉二巡檢司。東南距府百八十里。

漢中府元興元路,屬陜西行省。洪武三年五月為府。六月改名漢中府。領州一,縣八。東北距布政司九百六十里。

南鄭倚。萬歷二十九年十月建瑞王府。西南有巴嶺山,南連孤雲、兩角、米倉諸山,達四川之巴州。南濱漢水,又曰沔水,源自嶓冢,經縣界,下流至湖廣漢陽府入大江。又有沮水,漢水別源也,又西北有褒水,俱流入漢水。南有青石關巡檢司。

褒城府西北。洪武十年六月省入南鄭縣,後復置。東北有褒谷,自此出連雲棧,北抵斜谷之道也。南有沔水,即漢水也。又有廉水,又城東有褒水,西南有讓水,一名遜水,下流俱入沔水。北有雞頭關巡檢司。又有虎頭關。西北有漢陽關。

城固府東少北。南有漢水。東北有壻水,又名智水,下流入漢水。又西北有黑水,或云即褒水之上源。

洋府東南。元洋州。洪武三年降為縣。十年六月省入西鄉,後復置。北有興勢山。東有黃金谷。南有漢水。西北有壻水,西有灙水,亦曰駱谷水,又東有酉水,俱南入漢。

西鄉府東南。東有饒風嶺,有關。北有漢水。東有洋水,即清涼川也,西北合木馬河入漢。東南有鹽場關、西南有大巴山、東北有子午鎮三巡檢司。

鳳府西北。元鳳州。洪武七年七月降為縣。南有武都山。北有嘉陵江,源出縣之嘉陵谷,下流至四川巴縣入於大江。又東有大散水,亦注於嘉陵江。東北有清風閣巡檢司。南有留壩巡檢司,後遷廢丘關,又遷柴關,仍故名。南有仙人關。西有馬嶺關。

沔府西。元沔州,屬四川廣元路。洪武三年改屬漢中府,省州治鐸水縣入州。七年七月降為縣。十年六月省入略陽,後復置。成化二十一年六月屬寧羌州。嘉靖三十八年十一月仍屬府。北有鐵山。東南有定軍山。南有漢水。西有沮水,又有大安水,南入於漢。西南有大安縣,洪武初廢。又西有石頂關。

寧羌州本寧羌衛。洪武三十年九月以沔縣之大安地置。成化二十一年六月置州,屬府。東北有五丁山,亦曰金牛峽。北有嶓塚山,漢水出焉,亦曰漾水,下流至湖廣漢陽縣合大江。又東有嘉陵江,西有西漢水合焉。西南有白水,自洮州衛流經此,亦曰葭萌水,有白水關,其下流至四川昭化縣合於嘉陵江。又東北有濜水,流入漾水,謂之濜口。又東有沮水。北有陽平關巡檢司。東北距府三百里。領縣一:

略陽州北。元屬沔州。洪武三年屬府。成化二十一年六月改來屬。西有盤龍山。東南有飛仙嶺,棧道所經也。東有沮水,為漢水之別源。南有嘉陵江,西北有犀牛江,即西漢水也。又西有白水江。東北有九股樹、西有罝口二巡檢司。又西北有白水鎮巡檢司,後廢。

延安府元延安路,屬陜西行省。洪武二年五月為府。領州三,縣十六。南距布政司七百四十里。

膚施倚。東有延水,又有清化水流入焉。

安塞府西北。西有洛水。北有延水,出縣西北之蘆關嶺,又東南有西川水,北有金明川,俱流入焉。又北有塞門守禦千戶所,洪武十二年置。西南有敷政巡檢司。

甘泉府西南。北有野豬峽。西有洛河,南有伏陸水流入焉。又東北有庫利川。

安定府東北。北有高柏山,懷寧河出焉,東流入於無定河。西北有白洛城,洪武三年築。

保安府西北。西南有洛河,有吃莫河流入焉。北有大鹽池。又西有靖邊守禦千戶所,隆慶元年二月置。北有順寧巡檢司。

宜川府東。南有孟門山,在大河中流。又西南有銀川水,北有汾川水,西南有丹陽諸川,俱流入大河。

延川府東少北。東濱大河。北有吐延川,合清澗水,流注於大河。又東北有永寧關,臨河。

延長府東。東濱河。南有延水,流入大河。

青澗府東北。元屬綏德州。嘉靖四十一年改屬府。東有黃河,東北有無定河流入焉。又西有青澗河。

鄜州東有洛水,南與單池水合,又名三川水。西有直羅巡檢司。北距府百八十里。領縣三:

洛川州東南。西南有洛水。東南有鄜城巡檢司。

中部州南。北有橋山,亦曰子午嶺,沮水出焉。西北有谷河及子午水,俱入於沮水。又東北有洛水。

宜君州南。西南有玉華山,又有鳳凰穀。東有洛水。東北有沮水。

綏德州洪武十年五月省入府,後復置。南有魏平關。東有黃河。城東有無定河,一名奢延水,亦曰水,西北有大理水流入焉。東北有官菜園渡口巡檢司。西南距府三百六十里。領縣一:

米脂州北。西有無定河。有大理水,又有小理水,西北有明堂川,俱流入無定河。北有碎金鎮、西南有克戎寨二巡檢司。又西有銀州關,成化七年修築。

葭州洪武七年十一月降為縣,屬綏德州。十三年十一月復升為州,屬府。東濱大河,西有葭蘆河,城東有真鄉川流合焉。西南距府五百八十里。領縣三:

吳堡州南。元屬州。洪武七年十一月改屬綏德,尋省。十三年十一月復置,還屬。東濱河。

神木州北。洪武初省。十三年十一月復置。西北有楊家城,正統五年移縣治焉。成化中,復還故治。南有大河。北有濁輪川。西南有屈野川。

府谷州東北。洪武初省。十三年十一月復置。東濱大河,北有清水川入焉。

慶陽府元屬鞏昌總帥府。洪武二年五月直隸行省。領州一,縣四。東南距布政司五百七十里。

安化倚。洪武二十四年四月建慶王府。二十六年遷於寧夏衛。元省。洪武中復置。東北有白於山,洛水所出。又城東有東河,西有西河,流合焉,下流為馬蓮河。又西有黑水河,源出縣北之太白山,下流至長武縣合於涇河。東北有槐安、北有定邊二巡檢司。又西南有驛馬關、又有靈州、又有大鹽池三巡檢司,廢。

合水府東南。東有建水,西有北岔河,流合焉,謂之合水,西南入馬蓮河。又東北有華池水,有平戎川流合焉,東入鄜州之洛河。有華池巡檢司。

環府西北。元環州,屬鞏昌總帥府。洪武初,降為縣,來屬。西有環河,出縣北青岡峽,下流為府城之西河。又南有黑水河,又有堿河,西南有甘河,俱注於環河。又西有葫蘆泉。西北有清平關。西北有安邊守禦千戶所,弘治中置。

真寧府東南。元屬寧州。萬歷二十九年改屬府。西有馬蓮河。南有大陵、小陵諸水,即九陵川之上源也。東有雕山嶺巡檢司。

寧州元屬鞏昌總帥府。洪武中來屬。東有橫嶺,又有九龍川,亦曰寧江,亦曰九陵川,西南流,會上流群川,而南注於涇河。東北有襄樂巡檢司。北距府百五十里。

平涼府元屬鞏昌總帥府。洪武三年五月直隸行省。領州三,縣七。東南距布政司六百五十里。

平涼倚。洪武二十四年建安王府。永樂十五年除。二十二年,韓王府自遼東開原遷此。西南有可藍山。西有崆峒山。又有笄頭山,涇水出焉,下流至高陵縣入渭。又西有橫河,東有湫峪河,俱流入涇河。又西有群牧監。洪武三十年置陜西行太僕寺。永樂四年置陜西苑馬寺,領長樂等六監,開成等二十四苑,俱在本府及慶陽、鞏昌境內。正統三年又並甘肅苑馬寺入焉。又東有通梢關。

崇信府東南。北有汭水。西南有赤城川,南有白石川流合焉。下流合於涇水。

華亭府南。西有小隴山。西北有瓦亭山,有瓦亭關巡檢司,所謂東瓦亭也。東北有涇河。東南有汭水。又東南有三鄉鎮,北有馬鋪嶺二巡檢司。

鎮原府東北。元鎮原州,屬鞏昌總帥府。洪武初,降為縣,來屬。西北有胡盧河,分二流,一北注於黃河,其支流東南注於涇河。南有高平川,流入胡盧河。西有安平寨巡檢司。西北有蕭關。西南有木峽關。又西有石峽關。南有驛藏、木靖二關。

隆德府西南。元屬靖寧州。嘉靖三十八年十一月改屬府。東有好水,西流與苦水合。西北有武延川,流入好水。東南有捺龍川,流入苦水。

涇州元直隸陜西行省。洪武三年以州治涇川縣省入,來屬。舊治在涇水北。今治本皇甫店,洪武三年徙於此。北有涇河,有汭水。東有金家凹巡檢司。西北距府百五十里。領縣一:

靈臺州東南。西北有白石原。東北有三香水,一名三交川,下流至邠州合涇水。又西南有細川水,東北流合於三交川。

靜寧州元屬鞏昌總帥府。洪武中來屬。南有隴山。北有橫山,即隴山支阜。南有水洛川,一名石門水,下流至秦州入略陽川。又西有苦水河,即高平川之上源。東距府二百三十里。領縣一:

莊浪州東南。元莊浪州,直隸陜西行省。洪武三年屬鳳翔府。八年三月降為縣,來屬。西有苦水川。

固原州本固原守禦千戶所,景泰三年以故原州城置。成化四年升為衛。弘治十五年置州,屬府。西南有六盤山,上有六盤關,東北有清水河出焉,下流合鎮原縣之胡盧河。又北有黑水,北流入於大河。又東西有二朝那湫,其下流注於高平川。南有開成州,元直隸陜西行省,治開成縣。洪武二年省州,以縣屬平涼府。成化三年廢縣。又東南有廣安州,元屬開成州,洪武二年省。又西有甘州群牧所,永樂中置。又西北有西安守禦千戶所,成化五年以舊西安州置。北有鎮戎守御千戶所,成化十二年以葫蘆峽城置。東北有平虜守御千戶所,弘治十四年以舊豫望城置。又北有下馬關,嘉靖五年置。東南距府百七十里。

鞏昌府元屬鞏昌總帥府。洪武二年四月直隸行省。領州三,縣十四。東距布政司千六十里。

隴西倚。西有首陽山,上有關。北濱渭水,東有赤亭水,西流入焉。

安定府北。元定西州,屬鞏昌總帥府。至正十二年三月改名安定州。洪武十年降為縣,屬府。北有車道峴。西有西河,東有東河,流合焉。北有巉口巡檢司。

會寧府東北。元會州,屬鞏昌總帥府。至正十二年三月改為會寧州。洪武十年降為縣,屬府。東有響水,北流入大河。東有青家巡檢司。

通渭府東北。北濱渭,西有華川,東流入焉。

漳府南。西南有故城。今治,正統中所徙。西北有西傾山。南有漳水,北流入渭。東南有鹽井。

寧遠府東。南有太陽山,舊產鐵。北有桃花峽,兩山夾峙,渭水經其中。西有廣吳水,又有山丹水,俱源出岷州,並流北注渭。

伏羌府東。西南有朱圉山,俗名白崖山。北有渭水,西南有永寧河,西有洛門川,俱東北注於渭。

西和府東南。元西和州,屬鞏昌總帥府。洪武十年降為縣,屬府。舊治在西南白石鎮,洪武中,移於今治。北有祁山。南有黑谷山,上有關。西北有西漢水,亦曰鹽官水。西南有濁水,即白硃江也。東北有鹽井。

成府東南。元成州,屬鞏昌總帥府。洪武十年降為縣,屬府。西北有仇池山。東南有西漢水。西南有濁水,又西有建安水,又有洛谷川,俱流入西漢水。又東有泥陽水,下流至徽州界入嘉陵江。又北有黃渚關巡檢司。

秦州元屬鞏昌總帥府。洪武二年屬府,省州治成紀縣入州。西南有嶓冢山,西漢水出焉,下流至寧羌州合嘉陵江。東北有渭水,有秦水東流入渭。又西有西谷水,下流入西漢水。又南有籍水,西南有段谷水流入焉。又東有長離水,即瓦亭川下流也,俱流入於渭。南有高橋巡檢司。又有石榴關。又有現子關。西距府三百里。領縣三:

秦安州北。東有大隴山。又東北有瓦亭山,所謂西瓦亭也。城南有渭水。又西有隴水,瓦亭川自東北流合焉。又東有松多川,下流入於秦水。又東有隴城關巡檢司。

清水州東。東有隴山,有盤嶺巡檢司。西南又有小隴山。西有渭水。東有秦水,南有清水流入焉。

禮州西南。元禮店文州軍民元帥府,屬吐蕃宣慰司。洪武四年十一月置禮店千戶所。十一年屬岷州衛。十五年改屬秦州衛。成化九年十二月置禮縣於所城,屬州。故城在東。洪武四年移於今治。東南有西漢水。西南有岷峨山,岷江出焉,東南流入階州界合於西漢水。又西有漩水鎮、南有板橋山二巡檢司。

階州元屬鞏昌總帥府。洪武四年降為縣,屬府。十年六月復為州。舊城在東南坻龍岡上。今城,洪武五年所置。北有白水江。東北有犀牛江,即西漢水也。又西北有羌水,下流合白水江。又東有七防關巡檢司。西北距府八百里。領縣一:

文州東南。元文州。至元九年十月置,屬吐蕃宣慰司。洪武四年降為縣,屬府。十年六月改屬州。二十三年三月省。成化九年十二月復置,仍屬州。東南有青唐嶺,路入四川龍安府。東有白水,西有黑水,流合焉。又北有羌水,一名太白水。東有文縣守禦千戶所,本文州番漢千戶所,洪武四年四月置。二十三年改文縣守御軍民千戶所。成化九年更今名。又東有玉壘關。西北有臨江關。

徽州元屬鞏昌總帥府。洪武十年六月降為縣,屬府,後復升為州。東南有鐵山。南有嘉陵江,又有河池水流入焉。又南有虞關巡檢司。西南有小河關。西北距府四百八十里。領縣一:

兩當州東。洪武十年六月省入徽縣,後復置,屬州。南有嘉陵江。

臨洮府元臨洮府,屬鞏昌總帥府。泰定元年九月改為臨兆路。洪武二年九月仍為府。領州二,縣三。南距布政司千二百六十里。

狄道倚。西南有常家山,與西傾山相接。北有馬寒山,浩尾河出於其北,阿乾河出於其南,俱東流入大河。又西南有洮河,自洮州衛流入。又東有東峪河,南有邦金川,皆流會洮河。北有摩雲嶺巡檢司。又北有打壁峪關,有結河關。南有南關,有下襯關,有八角關、十八盤關。西有三坌關,有分水嶺關。

渭源府東少南。西有南谷山,渭水所出。又有鳥鼠山,渭水所經,東至華陰縣入大河。又西有分水嶺,東流者入渭,西流者入洮,上有分水嶺關巡檢司。又西南有五竹山,清源河出焉,逕縣東南入渭。

蘭州元屬鞏昌總帥府。洪武二年九月降為縣,來屬。成化十三年九月復為州。建文元年,肅王府自甘州衛遷此。南有皋蘭山。北濱大河,所謂金城河也,湟水自西,洮水、阿乾河俱自南,先後流入焉。又西南有漓水,合於洮水。北有金城關,下有鎮遠浮橋,有河橋巡檢司。西北有京玉關,南有阿干鎮關。西南有鳳林關。南距府二百十里。領縣一:

金州東少南。元金州,屬鞏昌總帥府。洪武二年九月降為縣,屬府。成化十三年改屬州。舊城在南,洪武中,移於今治。北有大河,東北流亂山中,入靖虜衛界。又南有浩尾河,一名閃門河,入於大河。東北有一條城,萬曆二十五年置。

河州元河州路,屬吐蕃宣慰司。洪武四年正月置河州衛,屬西安都衛。六年正月置河州府,屬陜西行中書省。七年七月置西安行都衛於此,領河州、朵甘、烏斯藏三衛。八年十月改行都衛為陜西行都指揮使司。九年十二月,行都指揮使司廢,衛屬陜西都指揮使司。十年分衛為左右。十二年七月,府廢,改左衛於洮州,升右衛為軍民指揮使司。成化九年十二月置州,屬府,改軍民指揮使司為衛。西南有雪山,與洮州界。西北有小積石山,上有關。大河自塞外大積石山東北流,逕此,又逕榆林衛北,折而南,與山西中流分界,至潼關衛北,折而東,入河南界,回環陜西境四千餘里。南有大夏河,即漓水也,亦曰白石川。又西北有積石州,元屬吐蕃宣慰司,洪武四年正月改置積石州千戶所。西南有貴德州,元屬吐蕃宣慰司,洪武八年正月改置歸德守禦千戶所。又南有寧河縣,東北有安鄉縣,元俱屬河州路,洪武三年廢,六年復置。十二年復廢。又東南有定羌巡檢司。東北距府百八十里。

靈州元屬寧夏府路。洪武三年罷。弘治十三年九月復置,直隸布政司。大河在城北,洛浦河自南流入焉。南有小鹽池。距布政司九百九十三里。

興安州元金州,屬興元路。萬曆十一年八月更名。二十三年直隸布政司。舊治漢水北,後遷水南。萬歷十一年又遷故城南三里許。北有漢水。又西有衡河,亦曰恆河,下流入漢江。東北有乾祐關巡檢司,廢。領縣六。西北距布政司六百四十里。

平利州南少東。元末省。洪武三年置,屬四川大寧州。五年二月來屬。十年六月復省,後復置。東有女媧山,灌溪水出焉,西北與黃洋河合,入於漢。南有鎮坪巡檢司。

石泉州西。元末省。洪武三年置,屬四川大寧州。五年二月來屬。嘉靖三十八年十一月改屬漢中府。萬歷十一年還屬州。南有十八盤山,有漢江。西有饒風河,東有遲河,俱入漢。又西有饒風嶺巡檢司,本治縣東遲河口,後遷下饒風鋪,更名。

洵陽州東。元末省。洪武三年復置。五年二月來屬。東北有水銀山,產水銀、硃砂。南有漢江,東有旬水流入焉。又有乾祐河,自西北流入旬水。東有閭關、西北有三岔二巡檢司。

漢陰州西少北。元末省。洪武三年復置。十年六月省入石泉縣,後復置,屬州。嘉靖三十八年十一月改屬漢中府。萬歷十一年還屬州。南有漢水。東北有直水,又有恆河,俱流入漢水。又西有方山關。

白河州東南。成化十二年十二月以洵陽縣白河堡置,屬湖廣鄖陽府。十三年九月來屬。北有漢江,東入湖廣鄖西縣界。南有白石河,分二流,俱北注於漢。

紫陽州西南。正德七年十一月以金州紫陽堡置。初治紫陽灘之左,嘉靖三十五年遷於灘右。西有漢江。

洮州衛元洮州,屬吐蕃宣慰司。洪武四年正月置洮州軍民千戶所,屬河州衛。十二年二月升為洮州衛軍民指揮使司,屬陜西都司。西南有西傾山,桓水出焉,下流為白水江,又漒川亦出焉,一名洮水。又北有石嶺山,上有石嶺關。東有黑松嶺,上有松嶺關。又東有黑石關、三岔關、高樓關。北有羊撒關。西南有新橋關、洮州關。東南有舊橋關。南距布政司千六百七十里。

岷州衛元岷州,以舊祐川縣地置,屬吐番宣慰司。洪武四年正月置岷州千戶所,屬河州衛。十一年七月升為衛,屬陜西都司。十五年四月升軍民指揮使司。嘉靖二十四年又置州,改軍民指揮使司為衛。四十年閏五月,州廢,仍置軍民指揮使司。洪武二十四年建岷王府。二十六年遷雲南。北有岷山,洮河經其下。南有白水,一名臨江。又東有石關。東北有鐵州,元屬吐蕃宣慰司。洪武四年正月置鐵城千戶所,屬河州衛,後廢。領所一。南距布政司千五百五十里。

西固城守禦軍民千戶所衛南。本西固城千戶所,洪武七年三月置,屬鞏昌府。十五年四月改置,來屬。南有白水。北有化石關。西北有平定關。

榆林衛成化六年三月以榆林川置。其城,正統二年所築也。西有奢延水,西北有黑水,經衛南,為三岔川流入焉。又北有大河,自寧夏衛東北流經此,西經舊豐州西,折而東,經三受降城南,折而南,經舊東勝衛,又東入山西平虜衛界,地可二千里。大河三面環之,所謂河套也。洪武中,為內地。天順後,元裔阿羅出、毛里孩、孛羅出相繼居之。西南有鹽池,舊屬寧夏衛,嘉靖九年來屬。又衛東有長鹽池、紅鹽池。西有西紅鹽池、鍋底池。又東有長樂堡,分轄雙山等十二營堡,為中路。又有神木堡,分轄鎮羌等九營堡,為東路。西有安邊營,分轄永濟等十二營堡,為西路。俱成化後置。又北有邊墻,成化九年築,長一千七百七十餘里,東起清水營,接山西偏頭關界,西抵定邊營,接寧夏花馬池界。南距布政司千一百二十里。

寧夏衛元寧夏府路,屬甘肅行省。洪武三年為府。五年,府廢。二十六年七月置衛。二十八年四月罷。永樂元年正月復置。洪武二十六年,慶王府自慶陽府遷此。西有賀蘭山。又西南有峽口山,黃河流其中,一名青銅硤。黃河出硤東流,亦曰三岔河。又東有黑水河,南有清水河,即葫蘆河下流也,俱注於黃河。有寧夏群牧千戶所,洪武二十七年十二月置。領千戶所四。東南距布政司千四百里。

靈州守禦千戶所衛東南。洪武十六年十月置,治在河口。宣德三年二月徙於城東。弘治十三年九月復置靈州於所城。

興武守禦千戶所衛東南。正德元年以興武營置。

韋州守禦千戶所衛東南。弘治十年以故韋州置。西有大蠡山。南有小蠡山。東有東湖。

平虜千戶所衛北少東。嘉靖三十年以平虜城置。東北有老虎山,濱大河。北有鎮遠關。

寧夏前衛在寧夏城內,洪武十七年置。

○寧夏左屯衛

寧夏右屯衛亦俱在寧夏城內,洪武二十五年二月置,後廢。三十五年十二月復置。

寧夏後衛本花馬池守禦千戶所,成化十五年置。正德元年改衛。其城,正統九年所築也。東北有方山。西有花馬池。西北有大鹽池。又西有小鹽池。東有長城關,正德初置。東南距布政司千一百二十里。

寧夏中衛元應理州,屬寧夏府路。洪武三年州廢。永樂元年正月置衛。西有沙山,一名萬斛堆。大河在南。又西南有溫圍水,流入大河。又有裴家川。又東南有鳴沙州,元屬寧夏府路。洪武初廢。南距布政司千一百十里。

靖虜衛正統二年以故會州地置,屬陜西都司。南有烏蘭山,上有烏蘭關。北有大河。西南有祖厲河,東北有亥刺河,皆注於大河。西南有會寧關。南距布政司千二百二十里。

陜西行都指揮使司元甘肅等處行中書省,治甘州路。洪武五年十一月置甘肅衛。二十五年罷。二十六年,陜西行都指揮使司自莊浪徙置於此。領衛十二,守禦千戶所四。距布政司二千六百四十五里。

甘州左衛倚。元甘州路。洪武初廢。二十三年十二月置甘州左衛。二十七年十一月罷。二十八年六月復置。洪武二十五年三月建肅王府。建文元年遷於蘭縣。西南有祁連山。西北有合黎山。東北有人祖山,山口有關,曰山南,嘉靖二十七年置。又東北有居延海。西有弱水,出西南山谷中,下流入焉。又有張掖河,流合弱水,其支流曰黑水河,仍合於張掖河。又東南有盧水,亦曰沮渠川。

甘州右衛、甘州中衛俱洪武二十五年三月置。

甘州前衛、甘州後衛俱洪武二十九年置。四衛俱與甘州左衛同城。

肅州衛元肅州路,屬甘肅行省。洪武二十七年十一月置衛。西有嘉峪山,其西麓即嘉峪關也。弘治七年正月扁關曰鎮西。西南有小崑侖山,亦曰雪山,與甘州山相接。北有討來河,東會於張掖河。西南有白水,又西北有黑水,東南有紅水,俱流入白水,下流入西寧衛之西海。又東北有威虜衛,洪武中置,永樂三年三月省。東距行都司五百十里。

山丹衛元山丹州,直隸甘肅行省。洪武初廢。二十三年九月置衛,屬陜西都司,後來屬。東南有焉支山。西有刪丹河,即弱水也。北有紅鹽池。西距行都司百八十里。

永昌衛元永昌路,屬甘肅行省,至正三年七月改永昌等處宣慰司。洪武初廢。十五年三月置衛,屬陜西都司,後來屬。北有金山,麗水出焉。西南有白嶺山,亦曰雪山。西有水磨川,上有水磨關。又東南有蹇占河。西北距行都司三百十里。

涼州衛元西涼州,屬永昌路。洪武九年十月置衛,屬陜西都司,後來屬。南有天梯山,三岔河出焉。東南有洪池嶺。又東北有白亭海,有瀦野澤。又西有土彌乾川,即五澗水也,亦出天梯山,下流合於三岔河。又東有雜木口關。又有涼州土衛,洪武七年十月置。西北距行都司五百里。

鎮番衛本臨河衛,洪武中,以小河灘城置。三十年正月更名。建文中罷。永樂元年六月復置。西有黑河,即張掖河下流也。又東有三岔河。南有小河。西有鹽池。西南有黑山關。西距行都司五百五十里。

莊浪衛洪武五年十一月以永昌地置。十二年正月置陜西行都指揮使司於衛城。二十六年,行都司徙於甘州。建文中,改衛為守禦千戶所。洪武三十五年十月復改所為衛,屬陜西都司,後來屬。東有大松山。其北有小松山。西有分水嶺,南出者為莊浪河,北出者為古浪河。又南有大通河,與莊浪河合,北流經衛西,入於沙漠。北距行都司九百四十里。

西寧衛元西寧州,直隸甘肅行省。洪武初廢。六年正月置衛。宣德七年十一月升軍民指揮使司,屬陜西都司,後來屬。西南有小積石山,與河州接界。東南有峽口山,亦曰湟峽。南有大河,自西域流入,回環於陜西、山西、河南、山東四布政司,及南直隸之地,幾至萬里,至淮安府清河縣,南合長淮,又東至安東縣南入於海。又北有湟水,即蘇木連河也,東入大河。又西南有賜支河,又城北有西寧河,皆流入大河。又西北有浩亹水,西南有宗哥川,俱流合於湟水。又西有西海,亦名卑禾羌海,俗呼青海。西北有赤海。又有烏海鹽池。東南有綏遠關。西北距行都司千三百五十里。

碾伯守禦千戶所本碾北地。洪武十一年三月置莊浪分衛。七月改置碾北衛,後廢,而徙西寧衛右千戶所於此。成化中更名。南有碾伯河。西北距行都司千二百三十里。

沙州衛元沙州路,屬甘肅行省。洪武初廢。永樂元年置衛。正統間廢。南有鳴沙山。東南有三危山。又東有龍勒山,又有渥窪水。西有瓜州,元屬沙州路,洪武初廢。東距行都司千三百六十里。

鎮夷守禦千戶所洪武三十年以甘州衛地置。建文二年罷。永樂元年復置所,舊在西北,天順八年移於今治。南有黑河,即張掖河也。西南有鹽池。北有兔兒關。東南距行都司三百里。

古浪守禦千戶所正統三年六月以莊浪衛地置。古浪河在東。又南有古浪關。東有石峽關。東南距行都司六百四十里。

高臺守禦千戶所景泰七年以甘州衛之高臺站置。弱水在北。又西有合黎山。西南有白城山。東南距行都司一百六十里。

○四川江西

四川《禹貢》梁、荊二州之域。元置四川等處行中書省。治成都路。又置羅羅蒙慶等處宣慰司,治建昌路。屬雲南行中書省。洪武四年六月平明昇。七月置四川等處行中書省。九月置成都都衛。與行中書省同治。八年十月改都衛為四川都指揮使司。領招討司一,宣慰司二,安撫司五,長官司二十二及諸衛所。九年六月改行中書省為承宣布政使司。領府十三,直隸州六,宣撫司一,安撫司一,屬州十五,縣百十一,長官司十六。為里千一百五十有奇。北至廣元,與陜西界。東至巫山,與湖廣界。南至烏撒、東川,與貴州、雲南界。西至威茂,與西番界。距南京七千二百六十里,京師一萬七百一十里。洪武二十六年編戶二十一萬五千七百一十九,口一百四十六萬六千七百七十八。弘治四年,戶二十五萬三千八百三,口二百五十九萬八千四百六十。萬曆六年,戶二十六萬二千六百九十四,口三百一十萬二千七十三。

成都府元成都路。洪武四年為府。領州六,縣二十五:

成都倚。洪武十一年建蜀王府。

華陽倚。北有武擔山。又有外江,自灌縣分流經城北,繞城而南,一名清遠江。又有內江,亦自灌縣分流經城南,繞城而東,亦名石犀渠。合流南注於大江。此府城之內、外江也。東有寧州衛,洪武十一年四月置。東南有馬軍寨巡檢司。

雙流府西南。洪武十年五月省入華陽縣。十三年十一月復置。東南有牧馬川,即府在內、外江下流也。

郫府西。有內江,一名郫江,即府城內江之上流也。

溫江府西少南。西南有皁江,亦曰內江。

新繁府西北。洪武十年五月省入成都縣。十三年十一月復置。西北有沱江。又西有湔水臾口。

新都府北。東有雒水,自什邡縣流經此,下流至瀘州入大江,亦曰中水。北有湔水,即大江別流,自灌縣東北出,流經此,至漢州入雒水。東北有綿水,自漢州流至此入雒江。三水同流,亦曰郫江也。

彭府北。元彭州。洪武十年五月降為縣。北有九隴山,有葛王貴山,又有大隋山、中隋山。南有沱江,又北有濛水流合焉。又東有濛陽縣,元屬彭州。洪武十年五月省。又北有白石溝巡檢司。

崇寧府西北。元屬彭州。洪武四年屬府。十年五月省入灌縣。十三年十一月復置。南有沱江。

灌府西少北。元灌州。洪武中,降為縣。西北有灌口山。又有玉壘山,下有玉壘關,一名七盤關。又西南有青城山。又西有湔江,亦曰都江,亦曰湔堋江,古離堆也。岷江經此,正流引而南,支流分三道,繞成都境。有石渠水口。又有白沙水,下流入都江。又南有沱江,即郫江上源也。又西有蠶崖關巡檢司。西南有獠澤關。

金堂府東。洪武十年五月省入新都縣。十三年十一月復置。東北有三學山。南有雲頂山。有金堂峽,雒水經此,曰金堂河。東南又有懷口巡檢司。

仁壽府南少東。東有麗甘山,下有鹽井。東有三嵎山,又有蟠溪,下流入資江。又南有陵井,產鹽,亦曰仙井。

井研府南少東。洪武六年十二月置。十年五月省入仁壽縣。十三年十一月復置。東北有鐵山,舊產鐵。南有鹽井。

資府東。明玉珍置資州。洪武初,降為縣。南有珠江,即雒江也,東流為資江。東有銀山鎮巡檢司。

內江府東南。洪武中置。西有中江,即雒之異名。南有椑木鎮巡檢司。

安府北少東。元安州,治在西北。洪武中,降為縣,移於今治。南有浮山,黑水出焉,南流入羅江縣界。北有曲山關。東有小東壩關。又東南有睢水關,關西有綿堰堡,綿水發源處也。

簡州洪武六年降為縣。正德八年又升為州。舊治在絳河北。正德八年徙治河南。東北有石鼓山。西有分棟山。東有雁水,即雒水也,絳水自北來合焉,一名赤水,亦曰牛鞞水。又城內有牛皮井,產鹽。西有龍泉鎮巡檢司。西南有陽安關。西北距府百五十里。領縣一:

資陽州東。洪武六年十二月置,屬府。十年五月省入簡縣。成化元年七月復置,仍屬府。正德中,改屬州。西有資溪,流入雁水。東有資陽鎮巡檢司,後移治濛溪河。

崇慶州元治晉原縣。洪武中省縣入州。西有鶴鳴山。西北有鄩江,東流入新津界。又北有味江,東北有白馬江,皆岷江南出之別名也。西北有永康縣。東南有江源縣,明玉珍復置,洪武初省。西有清溪口巡檢司。東北距府百十里。領縣一:

新津州東。南有天社山。南枕大江,一名皁江。東有北江,亦曰新穿水,自府城南流經此合大江。

漢州明玉珍復置雒縣,為州治。洪武四年省縣入州。東有雒水,有綿水。又西南有湔水,流入雒。又北有雁水,亦流入雒,故雒水亦兼雁水之名。又東北有石亭水,流合綿水。東南有三水關巡檢司。西南距府百十里。領縣三:

什邡州西。洪武十年五月省入綿竹縣。十三年十一月復置。西北有章山,雒水出此,亦名雒通山。南有高鏡關,雒水經其南。又西有大逢山。

綿竹州西北。西北有紫巖山,綿水出焉。又有紫溪河,一名射水河。又北有睢水關。

德陽州東北。洪武十年五月省入漢州。十三年十一月復置。北有鹿頭山,上有鹿頭關。東有綿水。西南有石亭水。南有白馬關巡檢司。

綿州元屬潼川府。洪武三年來屬。十年五月降為縣。十三年十一月復為州。東有富樂山。西有涪水,源出松潘衛,泫經此,亦曰綿江,下流至合州,合於嘉陵江。又西北有安昌水,一名龍安水,東南流合涪水。又東有潺水,亦合於涪水。東有魏城巡檢司。西南距府三百六十里。領縣二:

羅江州南。洪武六年十二月省入綿州。十三年十一月復置。東北有羅江,涪水與安昌水會流處也。又西有黑水,自安縣流入界。又西南有白馬關巡檢司,關與德陽縣鹿頭關相對。

彰明州北。洪武十年五月省入綿縣。十三年十一月復置。東北有太華山。西有涪江,北有廉水,西有讓水,俱流入焉。

茂州元治汶山縣,屬陜西行省吐番宣慰司。洪武中省縣入州。十六年復置縣,後復省。南有岷山,即隴山之南首也。汶江自松潘衛流入,經山下,又東經州城西,東南流,回環於四川、湖廣、江西三布政司及南直隸之地,入於海,幾七千餘里。南有雞宗關、東有積水關、北有魏磨關三巡檢司。又南有七星關,又有雁門關。東有桃坪關。北有實大關。西北有黃崖關,有汶山長官司,又南有靜川長官司,東南有隴木頭長官司,西南有嶽希蓬長官司,俱洪武七年五月置,屬重慶衛。又北有長寧堡,本長寧安撫司,宣德中,平歷日諸蠻置,屬松潘衛。正統元年二月改屬壘溪所。八年六月改屬茂州衛。後廢為堡。東南距府五百五十里。領縣一:

汶川州西南。北有七盤山。西有玉輪江,即汶江也,有汶川長官司,洪武七年五月置。西有寒水關巡檢司。又南有徹底關。

威州元以州治保寧縣省入。明玉珍復置縣。洪武二十年五月復省縣入州。舊治在西北鳳坪里,宣德三年六月遷於保子岡河西。十年六月又遷於保子岡河東千戶所城內。東南有定廉山,鹽溪出焉。又西南有雪山,亦曰西山。北有汶江,西北有赤水,北有平谷水,俱流入焉。東有通化縣,洪武三年省。西北有保子關、徹底關。西南有鎮夷關。東南距府四百五十里。領縣一:

保州西北。洪武六年分保寧縣地置。東有汶江。西北有鎮安關。

保寧府元屬廣元路。洪武四年直隸行省。領州二,縣八。西南距布政司七百里。

閬中倚。成化二十三年建雍王府。弘治三年遷於湖廣衡州府。四年八月建壽王府。正德元年遷於湖廣德安府。舊治在縣東,明玉珍徙於此。東有蟠龍山,其北有鋸山關。又有靈山,其麓為梁山關。南有嘉陵江,即西漢水,自陜西寧羌州流入,至巴縣合大江,亦曰閬水,又曰巴水,其下流曰渝水。有南津關在城南,臨嘉陵江。又有滴水關,在城北玉臺山下。又東南有和溪關。

蒼溪府西北。洪武十年五月省入閬中縣。十三年十一月復置。大獲山在東,宋江環其下。東南有云臺山。西南有嘉陵江,宋江自西流入焉。北有八字堡巡檢司。

南部府南少東。洪武十年五月省入閬中縣。十三年十一月復置。南有南山,一名跨鰲山。東南有離堆山。東北有嘉陵江。

廣元府北少西。元廣元路,治綿谷縣。洪武四年改為府。九年四月降為州,來屬,以綿谷縣省入。十三年十一月復置綿谷縣。二十二年六月降州為縣,復省綿谷縣入焉。北有潭毒山,上有潭毒關,下臨大江。又有朝天嶺,上有朝天關。又有七盤嶺,上有七盤關,為陜西、四川分界處。又東北有大漫天嶺,其北有小漫天嶺。西有嘉陵江。北有渡口,在大、小二漫間。東有百丈關,北有望雲關,有龍門閣,北達陜西寧羌州。

昭化府西北。元屬廣元路。洪武十年五月省入廣元州。十三年十一月復置,屬府。西南有長寧山,有白衛嶺。又西有九曲山。東有嘉陵江,其津口曰桔柏津,渡口關在焉。北有白水,自陜西文縣流入,亦曰葭萌水,合於嘉陵江。又北有馬鳴閣,又有石櫃閣。

劍州元屬廣元路。洪武六年以州治普安縣省入,來屬。九年省。十三年十一月復置。北有大劍山,亦曰梁山,西北接小劍山,飛閣通衢,謂之劍閣,有大、小劍門關在其上。又有漢陽山。東有嘉陵江。西南有涪江。北有大劍溪、小劍溪,又有泥溪。東南距府三百二十里。領縣一:

梓潼州西南。西有梓潼水,亦日潼江水,下流入於涪江。又北有揚帆水,流合潼江水。又東有小潼水,下流入嘉陵江。

巴州元屬廣元路。洪武九年四月以州治化城縣省入,又改州為縣,來屬。正德九年復為州。東北有小巴山,與漢中大巴山接,巴江水出焉,經州東南,分為三,下流至合州入嘉陵江。南有清水江,流合巴江。東有曾口縣,元屬州,後廢。又北有米倉關巡檢司。本治小巴山之巔,尋徙大巴山下,後廢。東北距府三百五十里。領縣二:

通江州東少北。元至正四年置,屬府。正德九年改屬州。舊治在趙口坪,洪武中,徙於今治。東有得漢山。南有巴江。又有宕水,在縣西壁山下,亦曰諾水,流入巴江。東北有濛壩、北有羊圈山二巡檢司。又東北有桐柏關,相對樗林關。

南江州北。正德十一年置。北有兩角山。南有難江,源出南鄭縣米倉山,下流入巴江。西北有大壩巡檢司。

順慶府元順慶路。洪武中,為府。領州二,縣八。西南距布政司六百里。

南充倚。北有北津渡,縣舊治也。洪武中,徙今治。南有清居山。西有大、小方山。東有嘉陵江。西有曲水,又有流溪水,東有清水溪,又有大斗溪,俱流注於嘉陵江。又西有昆井,產鹽。府境州縣多鹽井。北有北津渡巡檢司。

西充府西北。洪武十年五月省入南充縣。十三年十一月復置。南有南岷山,上有九井、十三峰。西有西溪,即流溪也。

蓬州元屬順慶路。洪武中,以州治相如縣省入。東南有雲山。西有嘉陵江。東北有巴江。西南距府百四十里。領縣二:

營山州東少北。洪武十年五月省入蓬州。十三年十一月復置。東北有大、小蓬山。東有巴江。

儀隴州北少東。洪武十年五月省入蓬州。十三年十一月復置。西有伏虞山。北有金城山,一名金粟山。東有巴江。北有鰲水,流入嘉陵江。

廣安州元廣安府,屬順慶路。洪武四年降為州,來屬。十年五月以州治渠江縣省入。東北有篆江,即巴江,合渠江之下流也。江中有三十六灘,亦名洄水。又北有濃水,南流合於環水,至州南合洄水,并注合州之嘉陵江。西北距府二百十里。領縣四:

岳池州西北。東有岳池水。

渠州東北。元渠州,屬順慶路。至元二十六年五月省州治流江縣入焉。洪武九年四月降為縣。東北有八濛山。東有宕渠山,有渠江,下流合巴江。又北有衛渠關,正德中置。

鄰水州東少南。成化元年七月置。東南有鄰山,產鐵。有鄰水,下流入大江,縣以此名。

大竹州東少北。元屬渠州。洪武九年來屬。西有九盤山。東有東流溪,下流合於渠江。

夔州府元夔州路,屬四川南道宣慰司。洪武四年為府。九年四月降為州,屬重慶府。十年五月直隸布政司。十三年十一月復為府。領州一,縣十二。西距布政司千九百里。

奉節倚。洪武九年四月省。十三年十一月復置。東北有赤甲山。東有白帝山,又有白鹽山。南濱江。東出為瞿唐峽,峽口曰灩澦堆。又西有南鄉峽、虎鬚灘,東有龍脊灘,皆江流至險處。又東有大水襄水、東水襄水,俱流入江。南有尖山、又有金子山二巡檢司。又東有瞿唐關。東南有江關。南有八陣磧,磧旁有鹽泉。

巫山府東。東有巫山,亦曰巫峽,大江經其中,東入湖廣巴東縣界。東有大寧河,又有萬流溪,皆流入大江。

大昌府東。洪武十三年十一月置。西有千頃池。又有當陽鎮巡檢司。

大寧府東北。元大寧州。洪武九年降為縣。北有寶源山,有石穴,鹽泉出焉。又有馬連溪,亦曰昌溪。東北有袁溪巡檢司。北有青崖關。

雲陽府西。元雲陽州。洪武六年十二月降為縣。南濱江。東有湯溪,源自湖廣竹山,流經此,至奉節湯口入江。西有檀溪,上承巴渠水,入於湯水。北有鹽井。又西北有五溪、北有鐵檠二巡檢司。

萬府西少南。元萬州。洪武六年十二月降為縣。南濱江。西有苧溪。東有彭溪。又西有武寧縣,洪武四年省,有武寧巡檢司。又西南有銅羅關巡檢司。又西北有西柳關。

開府西少北。元開州。洪武六年八月置,九月降為縣。南有開江,彭溪之上流,有清江自縣東流合焉,亦曰疊江。又南有墊江,一名濁水,亦合流於開江。

梁山府西。元梁山州,治梁山縣。洪武六年十二月省州,存縣。十年五月改屬忠州,後來屬。北有高梁山,又有高都山。西南有桂溪,南有蟠龍溪,下流俱入於江。

新寧元屬達州。洪武三年改屬重慶府。十年五月省入梁山縣。十三年十一月復置,來屬。東有霧山,開江出焉。又東有豆山關。

建始府東南。元屬施州。洪武中來屬。西有石乳山,產麩金,上有石乳關,與湖廣施州衛界。南有清江,自施州衛流入,又東入湖廣巴東縣界。

達州元治通川縣。洪武九年四月降為縣,省通川縣入焉。正德九年復升為州。西有石城山。東有渠江,通川江之下流,西南入渠縣界,合於巴江,中有南昌灘,有土副巡檢司。又西有鐵山關。東北有深溪關。東南距府八百里。領縣二:

東鄉州東少北。成化元年七月置。通川江在城東。

太平州東北。正德十年析東鄉縣地置。東北有萬頃池,渠江、通川江出焉,下流為渠江。北有北江,又北入陜西紫陽縣界,名任河,入於漢江,東北有明通巡檢司。

重慶府元重慶路,屬四川南道宣慰司。洪武中,為府。領州三,縣十七。西北距布政司五百五十里。

巴倚。東有塗山。大江經城南,又東經明月峽,至城東,與涪江合。西北有魚鹿峽,涪江所經。東南有丹溪,東北有交龍溪,俱流入大江。東有大紅江巡檢司。西有佛圖關。西南有二郎關。東有銅鑼關。又南有南坪關。

江津府西南。北濱大江。東南有僰溪口,僰溪入江處,有清平巡檢司。

壁山成化十九年三月析巴縣地置。大江在南。涪江在北。又北有壁山巡檢司。

永川府西少南。洪武六年十二月置。

榮昌府西少南。洪武六年十二月置。西有雒江,即中水。西北有昌寧縣,明玉珍置,洪武七年省。

大足明玉珍置,屬合州。洪武四年改屬府。東有米糧關。北有化龍關。

安居成化十七年九月析銅梁、遂寧二縣地置。東有安居溪,一名瓊江,下流入涪江。

綦江府南少東。元綦江長官司,屬播州。明玉珍改為縣。洪武中來屬。南有綦江,即僰溪之上流,一名東溪,有東溪巡檢司,後徙縣南之趕水鎮。又南有三溪渡,有綦市關。

南川府東南。洪武十年五月省入綦江縣。十三年十一月復置。南有南江,北流為綦江,中有龍床灘,在縣北。又東有四十八渡水,流入南江。又南有馬勁關、雀子崗關。北有冷水關。

長壽府東少北。洪武六年九月置,屬涪州,尋改屬府。北濱大江。南有樂溫山,下有樂溫灘,大江所經。又東有桃花溪。

黔江府東。元屬紹慶府。洪武五年十二月省入彭水縣。十一年九月置黔江守禦千戶所。十四年九月復置縣,來屬。南有黔江,源出貴州思州府界,正流自涪江合大江,支流經此,下流為湖廣施州衛之清江。又東有石勝關,又有石牙關。西有白巖關。東南有老鷹關,與湖廣施州界。

合州府北。元治石照縣。明玉珍省縣入焉。東有釣魚山,嘉陵江經其北,涪江經其南。又東北有嘉渠口,嘉陵江與渠江合流處,經城東南,涪江自西流合焉,亦曰三江口,並流而南,入於大江。南距府百五十里。領縣二:

銅梁州南。北有涪江。

定遠州北。有舊城。今城本廟兒壩,嘉靖三十年徙此。東有武勝山。西南有涪江。東有嘉陵江。

忠州府東。元治臨江縣。洪武中,以縣省入。南濱大江,江中有倒須灘,西北有鳴玉溪流入江。西有臨江巡檢司。西距府八百里。領縣二:

酆都州西南。元曰豐都。洪武十年五月省入涪州。十三年十一月復置,曰酆都。南濱大江,有葫蘆溪自西南流入焉。東南有南賓縣,洪武中省。又有沙子關巡檢司。

墊江州西少北。明玉珍置,屬州。南有高灘溪,西南入長壽界,為桃花溪。

涪州大江自長壽縣流入,東逕黃草峽,又東逕鐵櫃山,又東逕州城北,繞城而東,又南有涪陵江流合焉,江口有銅柱灘。又東南有清溪關。西南有白雲關。又西有陽關。西距府四百三十里。領縣二:

武隆州南。元曰武龍。洪武十年五月省入彭水縣。十三年十一月復置,曰武隆。西南有涪陵江,亦曰黔江,亦曰巴江。

彭水州南。元紹慶府治此,屬四川南道宣慰司。洪武四年,府廢,改屬重慶府。洪武十年五月來屬。東有伏牛山,山左右有鹽井。城西有涪陵江。又東南有水德江,源自貴州思南流入涪陵江。東南有天池關。東北有亭子關。

遵義軍民府元播州宣慰司,屬湖廣行省。洪武五年正月改屬四川。十五年二月改屬貴州都司。二十七年四月改屬四川布政司。萬曆二十九年四月改置遵義軍民府。領州一,縣四。西北距布政司千七百里。

遵義倚。元播州總管。洪武五年正月改為播州長官司。萬歷二十九年四月改縣,與府同徙治白田壩,在故司城之西。北有龍巖山。其東為定軍山,又有大樓山,上有太平關,亦曰樓山關。又東有烏江,源自貴州水西,即涪陵江上源,中有九接灘,其南有烏江關。又東南有仁江,東有湘江、洪江,皆流合於烏江。又西南有落閩水,東有樂安水,亦俱流入焉。又東南有河度關。西南有老君關。又東有三度關。西有落濛關。西北有崖門關、黑水關。北有海龍囤,有白石口隘。

桐梓府東。萬曆二十九年四月以舊夜郎縣望草地置。北有僰溪,源出山箐,綦江之上流。

真安州元珍州思寧長官司。明玉珍改真州。洪武十七年置真州長官司。萬曆二十九年四月改置。南有芙蓉江,自烏江分流,東北入於黔江。又有三江,東南流合於虎溪,亦注於黔江。西南距府二百里。領縣二:

綏陽府東北。萬歷二十九年四月以舊綏陽縣地置。東有水德江,亦曰涪江,亦曰小烏江,流入彭水縣界。

仁懷州西。萬曆二十九年四月以舊懷陽縣地置。東南有芙蓉江,西南有仁水,其下流俱注於烏江。

敘州府元敘州路,屬敘南等處蠻夷宣撫司。至元二十三年正月降為縣。洪武六年六月置府。領州一,縣九。北距布政司千二百里。

宜賓倚。弘治四年八月建申王府,未之國,除。西有失提山,舊產銀。西南有石城山。又西北有朝陽崖,大江經其下,又東經城東南,馬湖江來合焉。又西南有石門江,俗呼橫江,北入馬湖江。又東南有黑水,一名南廣溪,北入江。又西北有宣化縣,洪武中省,有宣化巡檢司。又西南有橫江鎮巡檢司。又南有摸索關。

南溪府東。東濱大江,中有石筍灘,在縣西。又有銅鼓灘,在縣東。又南有青衣水,流入大江。

慶符府南。洪武十年五月省入宜賓縣。十三年十一月復置。南有石門山,石門江經其下。又西北有馬鳴溪,流入馬湖江。

富順府東北。元富順州。洪武中降為縣。西南有虎頭山。東有金川,亦曰中水,即雒江也。又西有榮溪,東有鰲溪,俱流合焉。又西有鹽井。東有趙化鎮巡檢司。

長寧府東南。元長寧軍,屬馬湖路。泰定二年十月改為州。洪武五年降為縣。治東西有二溪,並冷水溪,三溪合流入大江,曰三江口。又東出虞公峽,曰淯溪,亦曰武寧溪,其下流入於大江。又治北有淯井,產鹽。東有梅洞堡巡檢司。

興文府東南。元戎州,屬馬湖路。洪武四年降為縣,來屬。萬曆二年二月改曰興文。南有南壽山,又有思早江,又東有水車河,俱流入淯溪。西有武寧城,萬曆二年二月築,置建武守御千戶所於此。所南有九絲城,所東南有李子關。縣東北有板橋巡檢司,後遷兩河口,仍故名。

隆昌府東北。本富順縣隆橋馬驛。隆慶元年置縣,析榮昌、富順二縣及瀘州地屬之。西南有雒江。

高州元屬敘南宣撫司。洪武五年降為縣,屬府。正德十三年四月復為州。舊治懷遠寨。正德十三年遷治中壩。東有復寧溪,即黑水之上流。南有江口巡檢司。北距府百五十里。領縣二:

筠連州西。元筠連州,治騰川縣,屬永寧路,尋廢縣存州。洪武四年降州為縣,屬敘州府。六年十二月改屬綿州,尋仍屬敘州府。十年五月省入高縣。十三年十一月復置,仍屬敘州府。正德十三年四月來屬。西有定川溪,下流與淯溪合。東南有三岔巡檢司。

珙州東。元下羅計長官司,屬敘南宣撫司。明玉珍改為珙州。洪武四年降為縣。十年五月省入高縣。十三年十一月復置,屬府。正德十三年四月來屬。西南有珙溪,下流入淯溪。南有鹽水壩巡檢司,後遷歇馬堡,仍故名。

龍安府元龍州,屬廣元路。明玉珍置龍州宣慰司。洪武六年十二月復置龍州。十四年正月改松潘等處安撫司。二十年正月仍改為龍州。二十二年九月改龍州軍民千戶所。二十八年十月升龍州軍民指揮使司,後復曰龍州。宣德七年改龍州宣撫司,直隸布政司。嘉靖四十五年十二月改曰龍安府。領縣三。南距布政司四百八十里。

平武倚。本名寧武,萬曆十八年四月置,後更名。州舊治在江油縣界之雍村。洪武六年徙於青州所。二十二年又徙於盤龍壩箭樓山之麓,即今治也。東南有馬盤山,又有石門山。東有涪江,有青川溪,下流合白水,入嘉陵江。西北有胡空關,又有黃陽關。東有鐵蛇關,西有大魚關,羊昌關、和平關,俱永樂中置。又東有棧閣,道出陜西文縣。又西有永濟橋,鐵索為之,達松潘衛。又東有青川守禦千戶所,洪武四年十月以舊青川縣置,屬四川都司。嘉靖四十五年十二月來屬。所東有白水江。東北有明月關巡檢司。南有杲陽關。北有北雄關,接陜西文縣界。又有控夷關,萬曆中置。

江油府東南。元省。明玉珍復置。洪武十年五月省入梓潼縣。十三年十一月復置,屬劍州。嘉靖四十五年十二月來屬。西有大匡山,與彰明縣界。東北有竇圌山。北有涪水,水上有涪水關。

石泉府西南。元屬安州。洪武中,州廢,改屬成都府。嘉靖四十五年十二月來屬。北有三面山,龍安水出焉。又東有湔水,東至江陽入江,有馬坪口巡檢司。北有松嶺關。西有石板關。東有奠邊關。東北有大方關。西北有上雄關。

馬湖府元馬湖路,屬敘南宣撫司。洪武四年十二月為府。領縣一,長官司四。東北距布政司千一百里。

屏山倚。本泥溪長官司,洪武四年十二月置。萬曆十七年三月改縣。西有雷番山。南有馬湖江,其上源自黎州西徼外流入界,至此合金沙江,經府城東入宜賓縣界。中有結髮灘、鐵鎖灘、雞肝石灘,俱在府西。又有馬湖,湖在山頂,亦曰龍湖。東有悔泥溪巡檢司。又東有龍關。西有鳳關。又北有新鄉鎮,萬曆十七年三月建城,置戍焉。

平夷長官司府西。洪武四年十二月置。舊治在司東。萬歷中,移於今治。南有馬湖江,又南有大汶溪,東有小汶溪,俱流合焉。

蠻夷長官司府西少南。洪武四年十二月置。南濱馬湖江,西有什葛溪,東有大鹿溪,俱流合焉。南有戎寧巡檢司。

沐川長官司府西少北。元置。洪武四年十二月改為州,尋復。北有沐川,下流入大江。東有芭蕉溪,下流入馬湖江。

雷坡長官司府西南。洪武四年十二月置。二十六年省。

鎮雄府元芒部路,屬雲南行省。洪武十五年正月為府。十六年正月改屬四川布政司。十七年五月升為軍民府。嘉靖五年四月改府名。萬曆三十七年五月罷稱軍民府。北有樂安山,與敘州府界。又西有白水,亦曰八匡河,源出烏撒界,流經此,境內諸川俱流入焉,下流至敘州府入大江。又南有苴斗河,下流入烏撒之七星關河。又北有堿泉二,俱產鹽。有益良州、強州,元俱屬芒部路,洪武十七年後廢。又有阿頭、易溪、易娘三蠻部,元屬烏撒路,洪武十五年三月屬芒部府。十七年又改阿頭部為阿都府,屬四川布政司。後俱廢。南有阿赫關,與烏撒界。領長官司五。北距布政司千五百八十里。

白水江肸酬長官司正德十六年十一月置。

懷德長官司府西。本卻佐寨。

威信長官司府南。本母響寨。

歸化長官司府西南。本夷良寨。

安靜長官司府西北。本落角寨。四司,俱嘉靖五年四月改置。

烏蒙軍民府元烏蒙路,後至元元年九月屬四川行省。洪武十五年正月為府,屬雲南布政司。十六年正月改屬四川布政司。十七年五月升為軍民府。西有涼山。北有界堆山,與敘州府界。西南有金沙江,下流合於馬湖江。南有索橋,金沙江渡處。北有羅佐關。有歸化州,洪武十五年三月置,屬府,尋廢。東北距布政司千三百里。

烏撒軍民府元烏撒路,後至元元年九月屬四川行省。洪武十五年正月為府,屬雲南布政司。十六年正月改屬四川布政司。十七年五月升為軍民府。西有盤江,出府西亂山中,經府南為可渡河,入貴州畢節衛界。有可渡河巡檢司。又西有趙班巡檢司。又有阿赫關、鄔撒二巡檢司。東南有七星關。東有老鴉關,又有善欲關,皆與貴州畢節衛界。又南有倘唐驛,路出雲南沾益州。東北距布政司千八百五十里。

東川軍民府元東川路,屬雲南行省。洪武十五年正月為府。十七年五月升為軍民府,改屬四川布政司。二十一年六月廢。二十六年五月復置。西南有馬鞍山,府舊治在焉。尋移治萬額山之南。又西南有絳雲弄山,接雲南祿勸州界,下臨金沙江。又東南有牛欄江,自雲南尋甸府流入,至府北合金沙江。有藤索橋,在東北牛欄江上。東北距布政司千四百里。

潼川州元潼川府,直隸四川行省。洪武九年四月降為州,以州治郪縣省入,直隸布政司。北有涪江,南有中江流合焉。又西南有郪江,有鹽井。西南距布政司三百里。領縣七:

射洪州南。洪武十年五月省入鹽亭縣。十三年十一月復置。東有涪江。又東南有射江,亦曰瀰江,亦曰梓潼水,自鹽亭縣流入,經縣東南之獨坐山,合於涪江。又東南有沈水,亦入涪江。有鹽井。

中江州西。洪武十年五月省入州。十三年十一月復置。西南有可蒙山、銅官山,南有賴應山、私熔山,俱產銅。東南有中江,南有郪江,有鹽井。

鹽亭州東少北。北有紫金山。南有梓潼水。東有鹽亭水,自劍州南境流入,亦謂之瀰江。城東有鹽井。

遂寧州東南。元遂寧州。明玉珍省州治小溪縣入焉。洪武九年四月降州為縣。東有銅盤山,又有涪江,北有郪江流入焉,謂之郪口。西有倒流溪,有鹽井。

蓬溪州東南。元屬遂寧州。洪武十年五月省入遂寧縣。十三年十一月復置,徙治故城之西南。西有明月山,下為明月池。又有伏龍山,下有火井。北有蓬溪,下流合於涪江,有鹽井。

安岳州南。洪武四年於縣置普州。九年,州廢。西有岳陽溪,下流合於涪江,有鹽井。

樂至州南少西。成化元年七月置,屬州。正德九年改屬簡州。嘉靖元年四月還屬。有鹽井。

眉州元屬嘉定府路。洪武九年四月降為縣,仍屬嘉定州。十三年十一月復為州,直隸布政司。東有蟆頤山,西面臨江,下為蟆頤津。南有峨眉山。東有玻璃江,即大江也。南有思濛江,西南有金流江,一名難江,下流俱入大江。東南有魚耶鎮巡檢司。北距布政司百八十里。領縣三:

彭山州北。洪武十年五月省入眉縣。十三年十一月復置。東有彭亡山,亦曰平無山,俗呼為平模山。北有天社山。南有打鼻山。東北濱大江,內江自雙流縣流入焉,即牧馬川也,合流而南,亦曰武陽江,江中有鼓樓灘。又有赤水,亦自東北流入大江。

丹棱州西。洪武六年十二月置,屬嘉定府。十年五月省入眉縣。十三年十一月復置,來屬。東南有青衣水,源出盧山縣,流經此,下流至嘉定州入大江。

青神州南。洪武十年五月省入嘉定州。十三年十一月復還屬。西有熊耳山,青衣水經其下。又東有大江。東南有松柏灘。東有犁頭灣巡檢司。

邛州元屬嘉定府路。洪武九年四月降為縣,仍屬嘉定州。成化十九年二月復為州,直隸布政司。西有古城山,產鐵。又東南有銅官山,產銅。西有相臺山,下有火井,又有鹽井。南有邛水,自雅州流入,至新津縣入大江。南有夾門關巡檢司。西有火井壩巡檢司。後移於州南二十五里。東北距布政司三百里。領縣二:

大邑州北少東。洪武十年五月省入邛縣。十三年十一月復置,屬嘉定州。成化十九年二月還屬。西北有鶴鳴山,與崇慶州界。東有牙江,下流入邛水。

蒲江州東南。元省入州。洪武六年十二月復置,屬嘉定府。成化十九年二月還屬。南有蒲水,源出名山縣,流經此,東入邛州界。西有雙路巡檢司。

嘉定州元嘉定府路。洪武四年為府。九年四月降為州,以州治龍遊縣省入,直隸布政司。東有三龜山。又有九頂山。大江在城東,亦曰通江。又西有陽江,即大渡河,自峨眉縣流入,經城東烏尤山下,合於大江。又西南有青衣水,至城西雙湖,與陽江合。東南有金石井巡檢司,後廢。北距布政司二百六十里。領縣六:

峨眉州西。西南有峨眉山,有大峨、中峨、小峨,羅目江出焉。陽江在縣南,自黎州所夷界流入,與羅目江合。又西南有中鎮巡檢司,後徙治大圍山。又有土地關,接蠻界。

夾江州西北。西有青衣水,又有洪雅川,合焉。

洪雅州西北。元省入夾江。成化十八年五月復置。西北有青衣水。西有洪雅川。又有竹箐山巡檢司。

犍為州東南。舊治玉津鎮。今治懲非鎮,洪武中徙此。東有大江。東北有四望溪流入焉。有四望溪口巡檢司。又北有石馬關巡檢司。

榮州東。本榮州。洪武六年十二月置。九年四月降為縣。東有榮川水,有甕溪關、飛水關,俱洪武間置。又有大坪隘口,成化十二年八月置。

威遠州東。洪武六年十二月置,屬嘉定府。十年五月省入榮縣。十三年十一月復置。

瀘州元屬重慶路。洪武六年直隸四川行省。九年直隸布政司。舊治在州東茜草壩。洪武中徙此。城西有寶山。西南有方山。大江在東,一名瀘江,又名汶江,資水自州北來合焉,亦曰中江。又有瀘州衛,洪武二十一年十月置於州城,成化四年四月徙於州西南之渡船鋪。南有石棚鎮、北有李市鎮二巡檢司。又有江門、水流崖、洞掃等關堡,俱成化四年四月置。又南有龍透關,崇禎間修築。西北距布政司千五百五十里。領縣三:

納溪州西南。北濱大江,城西有納溪水,自蕃部西南流合焉。有納溪口巡檢司。南有倒馬關、石虎關,俱通雲南、交址路。

江安州西少南。北濱大江,有綿水西南流入之,謂之綿水口。又南有淯溪,又有涇灘,俱流合於綿水。有板橋巡檢司。

合江州東少北。舊治在神臂山南。洪武初徙安樂山之麓,即今治也。又南有榕山,俗名容子山。北濱大江,西有之溪、北溪入焉,因謂之合江。又南有安樂溪,西北流入江安縣。

雅州元屬陜西行省吐蕃宣慰司。洪武四年以州治嚴道縣省入,直隸布政司。東有蔡山,一名周公山,其下有經水,一名周公水。又東南有榮水,一名長濆河,又有小溪,一名百丈河,至州界,俱合流於青衣江。北有金雞關。東北有金沙關。東北距布政司四百五十里。領縣三:

名山州東北。洪武十年省入州。十三年十一月復置。東北有百丈山,旁有百丈縣,元屬州,洪武中省。西有蒙山。南有青衣江。

榮經州西南。明玉珍省入嚴道縣。洪武中復置。東北有銅山。東有邛崍山,與黎州所界,上有九折阪。西有大關山,邛崍關在焉。北有長濆河,南有周公水,並流入州界。西北有紫眼關,地接西番。又有碉門砦,亦曰和川鎮,元置碉門安撫司。洪武五年設碉門百戶所於此,其地興天全界。

蘆山州西北。元曰瀘山,後省。洪武六年十二月復置,改為蘆山。東有盧山,青衣水出焉。南有三江渡,其水經多功峽,下流入平羌江。西北有臨關,舊曰靈關,正統初更名。有臨關巡檢司。又南有飛仙關。

永寧宣撫司元永寧路。洪武七年為永寧長官司。八年正月升宣撫司。天啟三年廢,地屬敘州府。故城在西。洪武十五年遷於今治。東南有獅子山。西北有青山。南有永寧河,東北流經瀘州境,入於大江。又東南有赤水河。東有魚浮關,洪武四年置。領長官司二。距布政司千八百里。

九姓長官司司城西南。元九姓羅氏黨蠻夷長官千戶。洪武六年十二月改置。天啟六年改屬瀘州。南有通江溪,東北會於納溪之江門峽。西南有金鵝池。

太平長官司元大壩軍民府,洪武中廢。成化四年四月改置。

天全六番招討司元六番招討司。洪武六年十二月改置,直隸四川布政司。二十一年二月改隸都司。東有多功山。南有和水,一名始陽河,亦名多功河,流入雅州青衣江。又西番境內有可跋海,其下流合雲南樣備水,流入交址。又禁門關、紫石關亦俱在司西。又東有善所、張所、泥山、天全、思經、樂藹、始陽、樂屋、在城、靈關凡十百戶所。東距布政司五百五十里。

松潘衛元松州,屬雲南行省。洪武初因之。十二年四月兼置松州衛。十三年八月罷衛。未幾,復置衛。二十年正月罷州,改衛為松潘等處軍民指揮使司,屬四川都司。嘉靖四十二年罷軍民司,止為衛。東有雪欄山,上有關。南有紅花山。西北有甘松嶺。又北有大、小分水嶺。西有岷江,自陜西洮州衛流經此,亦曰潘州河。又東有涪江,出小分水嶺,東南流,入小河所界。北有潘州衛,洪武中,以故潘州置。二十年省入。又西有鎮夷關,永樂四年七月置。又西北有流沙關。又東有望山、雪欄、風洞、黑松林、三舍、小關子關。南有西寧、歸化、安化、新塘、北定、浦江六關。又有平夷關,萬曆十四年置。又南為鎮平關。又西北有漳臘堡,洪武十一年置。領千戶所一,長官司十六,官撫司五。東南距布政司七百六十里。

小河守禦千戶所宣德四年正月置。北有師家山,一名文山,山麓有文山關。南有小河,即涪水也,東流入龍安府界,有鐵索橋跨其上。

占藏先結簇長官司、蠟匝簇長官司、白馬路簇長官司、山洞簇長官司、阿昔洞簇長官司、北定簇長官司、麥匝簇長官司、者多簇長官司、牟力結簇長官司、班班簇長官司、祈命簇長官司、勒都簇長官司、包藏先結簇長官司以上十三司,俱洪武十四年正月置。,阿用簇長官司宣德十年五月置。,潘斡寨長官司正統五年七月置。,別思寨長官司宣德十年五月置。,八郎安撫司永樂十五年二月置。,麻兒匝安撫司宣德二年三月,以阿樂地置。,阿角寨安撫司、芒兒者安撫司二司俱正統五年七月置。,思曩日安撫司正統十一年七月置。

疊溪守禦軍民千戶所本疊溪右千戶所,洪武十一年以古翼州置,屬茂州衛。二十五年改置。直隸都司。南有排柵山。西有汶江,南有黑水流合焉,謂之翼水。又南有南橋、中橋、徹底三關,北有永鎮橋關、鎮平關,西有疊溪橋關,東有小關,俱洪武十一年置。領長官司二。東南距布政司五百八十里。

疊溪長官司所城北。、鬱即長官司所城西。俱永樂元年正月置。

黎州守禦軍民千戶所本黎州長官司,洪武九年七月置。十一年六月升安撫司,直隸布政司。萬曆二十四年降為千戶所,直隸都司。東北有聖鐘山,下有黎州,元屬陜西行省吐蕃宣慰司。洪武五年省州治漢源縣入州。永樂後廢。西北有飛越山,兩面皆接生羌界。西南有大田山,東麓為大田壩,萬歷二十四年立黎州土千戶所於此。又東有沖天山。南有避瘴山。西北又有筍筤山。南有大渡河,即古若水。洪武十五年六月置大渡河守御千戶所,後徙司城西北隅。又西南有漢水,源出飛越山之仙人洞,亦曰流沙河,下流至試劍山,入大渡河。河南即清溪關,與建昌行都司界。西有黑崖關,洪武十六年置。又有椒子關,路通長河西等處。東北距布政司六百九十里。

平茶洞長官司元溶江、芝子、平茶等處長官司。洪武八年正月置,屬酉陽宣撫司。十七年直隸布政司。西有百歲山。哨溪出於其東,滿溪出於其西,合流入買賽河。北距布政司千六百七十里。

溶溪芝麻子坪長官司元溶江、芝子、平茶等處長官司。洪武八年改置,屬湖廣思南宣慰司。十七年五月直隸四川布政司。

安寧宣撫司成化十三年二月置,領長官司二:

懷遠長官司、宣化長官司俱成化十三年二月,與宣撫司同置。

酉陽宣慰司元酉陽州,屬懷德府。明玉珍改沿邊溪洞軍民宣慰司。洪武五年四月仍置酉陽州,兼置酉陽宣慰司,州尋廢。八年正月改宣慰司為宣撫司,屬四川都司。永樂十六年改屬重慶衛。天啟元年升為宣慰司。東南有酉水,流合平茶水,至湖廣辰州府合流於江,有寧俊江巡檢司。西北距重慶府四百九十里。領長官司三:

石耶洞長官司司東南。元石耶軍民府。洪武八年正月改為長官司。邑梅洞長官司司南。元佛鄉洞長官司。明玉珍改邑梅沿邊溪洞軍民府。洪武八年正月改置。北有凱歌河,一名買賽河,自貴州平頭著可司流入,東入酉陽司界。麻兔洞長官司洪武八年正月置。

石砫宣慰司元石砫軍民宣撫司。明玉珍改安撫司。洪武八年正月為宣撫司,屬重慶衛。嘉靖四十二年改屬夔州衛。天啟元年升為宣慰司。東有石砫山。又有三江溪,即葫蘆溪之上流也。西南距夔州府七百五十里。

四川行都指揮使司元羅羅、蒙慶等處宣慰司,治建昌路,屬雲南行省。洪武十五年罷宣慰司。二十七年九月置四川行都指揮使司。治建昌衛。領衛五、所八、長官司四。東北距布政司千四百八十里。

建昌衛軍民指揮使司元建昌路,屬羅羅蒙慶宣慰司。洪武十五年正月為府,屬雲南布政司,兼置衛,屬雲南都司。十月,衛府俱改屬四川。二十五年六月,府廢,升衛為軍民指揮使司。二十七年九月來屬。領守禦千戶所四、長官司三。南有瀘水,流入金沙江。又北有長河,南有懷遠河,西南有寧遠河,下流俱合於瀘水。又東有建安州、永寧州,又東有里州,東南有闊州,西南有瀘州、隆州,元俱屬建昌路,洪武十五年三月俱屬建昌府。東有北社縣,元屬永寧州,洪武十五年三月因之,尋改為碧舍縣。又西有德州,元屬德昌路,洪武十五年三月屬德昌府。二十七年後,府州縣俱廢。又有建昌前衛指揮使司,洪武二十七年六月置,與建昌軍民衛同城,九月屬四川行都司,萬曆三年省。又東有建昌土衛,洪武十五年置,萬曆後廢。北有瀘沽巡檢司,即故瀘沽縣也。又南有麻刺巡檢司。又西南有打沖河、東南有白水、東有龍溪三巡檢司,後廢。又東北有老君關,有太平關。東南有甸沙關。又有金川堡。

○守禦禮州後千戶所

守禦禮州中中千戶所衛北。元禮州,屬建昌路。洪武十五年三月屬建昌府,兼置二守禦所,屬衛。二十七年後,州廢。北有瀘沽縣,元屬禮州,洪武十五年三月因之,亦二十七年後廢。

守禦打沖河中前千戶所衛西。洪武二十七年二月置。西有打沖河,蠻名黑惠江,一名納夷江,源出吐蕃,下流入金沙江。東北有水砦關。南有天星砦。

守禦德昌千戶所衛南。洪武十五年置。南有德昌路,元屬羅羅蒙慶宣慰司,洪武十五年三月為府,屬雲南布政司,十月改屬四川布政司,二十七年後廢。

昌州長官司衛南。元屬德昌路。洪武十五年三月屬德昌府。永樂二年七月改置。

威龍長官司衛東南。元威龍州,屬德昌路。洪武十五年三月以「龍」為「隆」,屬德昌府。永樂二年七月改置。

普濟長官司衛西南。元普濟州,屬德昌路。洪武十五年三月屬德昌府。永樂二年七月改置。

寧番衛軍民指揮使司元蘇州,屬建昌路。洪武十五年三月屬建昌府。二十一年十月兼置蘇州衛,屬四川都司。二十五年六月,州廢,升衛為軍民指揮使司。二十六年三月更名,屬四川都司。二十七年九月來屬。南有南山,產銅。東有長河,亦名白沙江,南流會於瀘水。又有中縣,元屬建昌路。洪武十五年三月改屬永寧州。十七年改屬蘇州,後廢。又有沙陀關、羅羅關、九盤關。南有烏角關。北有北山關。又西有定番堡,萬歷十五年置。南距行都司百九十里。領千戶所一:

守禦冕山橋後千戶所衛東。正統七年以冕山堡置。東有東河,與瀘沽河合,下流入金沙江。北有冕山關。

越巂衛軍民指揮使司洪武二十五年七月置,屬四川都司。二十七年九月來屬。西有阿露山,亦曰大雪山。北有大渡河,與黎州界。又有魚洞河,南有羅羅河,合流入大渡河。又北有青岡關,有海棠關,有曬經關。南有小相公嶺關。西北有刺伯關。南距行都司百九十里。領千戶所一、長官司一:

鎮西後千戶所衛北。弘治中置。

邛部長官司衛東。元邛部州,屬建昌路。洪武十五年三月屬建昌府,二十七年四月升軍民府,後仍為州,屬越巂衛。永樂元年五月改為長官司。東有平夷、歸化二堡,萬歷十五年開部夷地增置。

鹽井衛軍民指揮使司元柏興府,治閏鹽縣,屬羅羅蒙慶宣慰司。洪武十五年三月屬雲南布政司。二十四年二月降為州,省閏鹽縣入焉。二十六年六月,州廢,置衛,屬四川都司。二十七年九月來屬。南有柏林山。西有斛僰和山,產金。又西有鐵石山,出砮石。東北有打沖河,上有索橋。西有雙橋河,東有越溪河,俱流入打沖河。又治東有鹽井。北有金縣,元屬柏興府,洪武十五年三月因之,十七年後廢。又東有雙橋關。西有古德關。東南距行都司三百里。領千戶所一、長官司一:

打沖河守禦中左千戶所衛東北。洪武二十五年置。

馬刺長官司衛南。永樂初置。

會川衛軍民指揮使司本會川守禦千戶所,洪武十五年置,屬建昌衛。二十五年六月升軍民千戶所。十一月升會川衛軍民指揮使司,屬四川都司。二十七年九月來屬。東南有土田山,產石碌,有葛砧山,產石青,東有密勒山,產銀礦。西南有金沙江,自雲南武定府流入界。又西有瀘水,南入焉,南有瀘沽河,亦流入焉。又南有搭甲渡巡檢司。東南有瀘津關。南有迷郎關,又有松坪關。西有永昌關,有大龍關。北有甸沙關,接建昌衛界。有會川路,元屬羅羅蒙慶宣慰司。洪武十五年三月為府,屬雲南布政司。十月改屬四川布政司。二十六年四月,府廢。墮其城。二十七年四月復置府,後復廢。又西有永昌州,南有武安州,又有黎漢州,元俱屬會川路,洪武十五年三月俱屬會川府,十月俱改為縣,二十四年二月復俱為州。東南有姜州,元屬建昌路,又有會理州,元屬會川路,洪武十五年三月俱改屬東川府。北有麻龍州,元屬會川路,洪武十五年三月改屬東川府。又有麻龍縣,洪武十七年改屬麻龍州。二十七年後,府州縣俱廢。西北距行都司五百里。領千戶所一:

守禦迷易千戶所衛西北。洪武二十五年閏十一月置。

江西《禹貢》揚州之域。元置江西等處行中書省。治龍興路。太祖壬寅年正月因之。正月治吉安府。二月還治洪都。洪武三年十二月置江西都衛。與行中書省同治。八年十月改都衛為都指揮使司。九年六月改行中書省為承宣布政使司。領府十三,州一,縣七十七。為里九千九百五十六有奇。北至九江,與江南、湖廣界。東至玉山,與浙江界。南至安遠,與福建、廣東界。西至永寧,與湖廣界。距南京一千五百二十里,京師四千一百七十五里。洪武二十六年編戶一百五十五萬三千九百二十三,口八百九十八萬二千四百八十二。弘治四年,戶一百三十六萬三千六百二十九,口六百五十四萬九千八百。萬曆六年,戶一百三十四萬一千五,口五百八十五萬九千二十六。

南昌府元龍興路,屬江西行省。太祖壬寅年正月為洪都府。癸卯年八月改南昌府。領州一、縣七:

南昌倚。洪武十一年建豫王府。二十五年改為代王,遷山西大同。永樂初,寧王府自大寧衛遷此,正德十四年除。故城在東。今城,明太祖壬寅年改築。東湖在城東南隅。西有贛江,自豐城縣流入,東北入鄱陽湖,出湖口縣,入大江,亦曰章江。又東南有武陽水,上源自南豐縣汙江,北流經此,又東北入宮亭湖。南有市汊巡檢司。

新建倚。西有西山,跨南昌、新建、奉新、建昌四縣之境。北有吳城山,臨贛江。東有鄱陽湖,即彭蠡也,俗謂之東鄱湖;其西與宮亭湖相接,謂之西鄱湖西南有筠水,一名蜀江,自高安縣流入,合於章江。東北有趙家圍、西有烏山、北有吳城、西北有昌邑四巡檢司。

豐城府南少西。元富州。洪武九年十二月改為豐城縣。南有羅山,富水所出。又有柸山,豐水所出。西南有章江,豐水自南,富水自東南,俱流入焉。又東有雲韶水,自撫州流入,亦入於章江。南有沛源、西南有江滸口二巡檢司。又有河湖巡檢司,廢。又北有港口巡檢司,治大江口,後遷縣東北小江口,廢。

進賢府東南。西南有金山,產金。北有三揚水,又有軍山湖,又北有日月湖,下流俱入於鄱陽湖。東有潤陂、東北有鄔子寨、北有龍山、東南有花園四巡檢司。

奉新府西。西有百丈山,馮水所出,下流入於章江。又西有華林山,華林水出焉。又西北有藥王山,龍溪水出焉。二水合流,注於馮水。西有羅坊巡檢司。又有白沙巡檢司,廢。

靖安府西北。西有毛竹山,接寧州界,雙溪水出焉,下流入於馮水。北有桃源山,桃源水所出,流與雙溪水合。又西北有長溪,源出名山,下流入於修水。

武寧府西北。西有太平山。西北有九宮山。南有修水。

寧州府西。元分寧縣,為寧州治。洪武初,改縣為寧縣,省州入焉。弘治十六年,升縣為州。西有幕阜山,修水發源於此,下流入鄱陽湖。又東有鶴源水,源發九宮山,下流合修水。西有杉市巡檢司,後遷於崇鄉北村。南有定江、又有八疊嶺二巡檢司,廢。東南距府三百六十里。

瑞州府元瑞州路,屬江西行省。洪武二年為府。領縣三。東北距布政司二百里。

高安倚。北有米山。西北有華林山。又北有蜀江,自上高縣流入,東流匯於南昌之象牙潭而入章江,一名錦水。此別一蜀江,非出岷山之大江也。又南有曲水,亦東入章江。南有陰岡嶺、又有洪城二巡檢司,廢。

上高府西南。南有蒙山,舊產銀鉛。西有天嶺。又西有蜀江,自萬載縣流入,至縣西北凌江口合新昌縣之鹽溪水。又有斜口水,源出蒙山,至縣西亦流入焉。西有離婁橋、又有麻塘二巡檢司。

新昌府西。元新昌州。洪武初,降為縣。西有鹽溪水,一名若耶溪,南流至上高縣入於蜀江。又北有藤江,下流與鹽溪水合。西有黃岡洞、北有大姑嶺二巡檢司。

九江府元江州路,屬江西行省。太祖辛丑年為九江府。領縣五。南距布政司三百里。

德化倚。南有廬山,亦曰匡廬。東南有鄱陽湖,湖中有大孤山。縣北濱大江,亦曰潯陽江,北岸為湖廣黃梅縣,南岸經湖口、彭澤二縣,而入南直東流縣境。江中有桑落州,與南直宿松縣界。又西有湓浦,自瑞昌縣流入,經城西,注於大江,所謂湓口也。又東南有女兒浦,源出廬山,東北入鄱陽湖。西有城子鎮巡檢司。又東有南湖觜、西有龍開河二巡檢司,後廢。

德安府西南。南有博陽山,古文以為敷淺原,博陽川出焉,東南流入於鄱陽湖。東北有谷簾水,源出廬山,下流亦入鄱陽湖。

瑞昌府西。西有清湓山,湓水出焉。北有大江,北岸與湖廣廣濟縣分界。

湖口府東。北濱大江。南有上石鐘山。北有下石鐘山。又南有青山,在鄱陽湖中。西南即鄱陽湖,匯章、貢群川之水,由此入江。南有湖口鎮巡檢司,後遷上石鐘山。西北有茭石磯鎮巡檢司,後遷於黃茅潭。

彭澤府東少北。濱大江。北有小孤山在江中,江濱有彭浪磯,與小孤對。東北有馬當山,橫枕大江。有馬當鎮巡檢司。西南有峰山、磯鎮二巡檢司。

南唐府元南唐路,屬江西行省。太祖辛丑年八月為西寧府。壬寅年四月改曰南康府。領縣四。南距布政司三百里。

星子倚。西北有廬山。北有鞋山,在鄱陽湖中。湖東為宮亭湖,西北為落星湖。又西有谷簾水,下流入鄱陽湖。東有長嶺巡檢司,後遷縣南渚溪鎮,又遷縣東北青山鎮,仍故名。

都昌府東。西南有石壁山,臨章江,東南為鄱陽湖,北有後港河,合諸水入焉。西北有左蠡巡檢司,濱湖。東南有柴棚巡檢司,在湖中。

建昌府西南。元建昌州。洪武初,降為縣。西南有長山,南有修水,自寧州流入,亦謂之西河。東有蘆潭巡檢司。

安義府西南。正德十三年二月析建昌縣安義等五鄉置。東有東陽新逕水,南有龍江水,俱流合於修水。

饒州府元饒州路,屬江浙行省。太祖辛丑年八月為鄱陽府,隸江南行省。尋曰饒州府,來隸。領縣七。西南距布政司二百四十里。

鄱陽倚。正統元年,淮王府自廣東韶州府遷此。西北有鄱陽山,在鄱陽湖中。湖長三百里,闊四十里,亙南康、饒州、南昌、九江四府之境。南有鄱江,源出南直婺源縣及祁門縣,下流會於城東。又南則廣信上饒江來合焉,環城西北出,復分為二,俱入鄱陽湖,亦名雙港水。又東有東湖,一名督軍湖,流入鄱江。西北有棠陰巡檢司,遷於雙港口。北有石門鎮巡檢司。又東北有大陽埠。西有八字腦。

餘干府南。元饒幹州。洪武初,降為縣。西北有康郎山,濱鄱陽湖南涯,因名其水曰康朗湖。又西有族亭湖。又南有餘水,亦曰三餘水。又南有龍窟河,合於餘水,下入鄱江。有康山巡檢司,舊在康郎山上,後遷黃埠。西有瑞虹鎮,在鄱陽湖濱。

樂平府東。元樂平州。洪武初,降為縣。東北有鳳遊山。南有樂安江,即鄱江之上流也。北有八澗鎮巡檢司。南有仙鶴鎮巡檢司,後遷萬年縣之苛溪鎮。

浮梁府東。元浮梁州,洪武初降為縣。南有昌江,南直祁門縣之水俱流匯焉,鄱江之別源也。西北有桃樹鎮巡檢司,後遷縣東北勒上市。西南有景德鎮,宣德初,置御器廠於此。

德興府東。東有銀山,舊產銀。北有銅山,山麓有膽泉,浸鐵可以成銅。西南有建節水,自弋陽縣流入。北有大溪,自南直婺源縣流入。下流俱合於樂安江。東有白沙巡檢司。西南有永泰巡檢司,廢。

安仁府南少東。南有錦江,亦名安仁港,自貴溪縣流入,西北入餘干境,為龍窟河。又東有白塔河,流合於錦江。南有白塔、東有田南二巡檢司,後廢。

萬年府東南。正德七年以餘干縣之萬春鄉置,析鄱陽、樂平及貴溪三縣地益之。北有萬年山。東有桃源洞,桃源水出焉,經縣西南,下流為餘水。東北有荷溪鎮、北有石頭街二巡檢司,後俱廢。

廣信府元信州路,屬江浙行省。太祖庚子年五月為廣信府。領縣七。西北距布政司六百三十里。

上饒倚。西北有靈山,舊產水晶。南有丁溪山,產鐵。又南有銅山。北有上饒江,自玉山縣流入,經城北,下流至鄱陽縣合於鄱江。又西有櫧溪,源出靈山,亦曰靈溪,流入上饒江。南有八坊場、東北有鄭家坊二巡檢司。

玉山府東。有三清山。又有懷玉山,玉溪出焉,分二流,東入浙,西為上饒江。東南有柳都寨巡檢司。

弋陽府西。南有軍陽山,舊產銀。東有弋陽江,即上饒江下流也,又有弋溪流合焉。又有葛溪,源出上饒縣靈山,下流入鄱江。又有信義港,自福建邵武流入,合於葛溪。

貴溪府西。西南有象山,又有龍虎山,上清宮在焉。其南為仙嚴。又南有薌溪,亦名貴溪,上流即上饒江也。又有須溪,自福建光澤縣流入,來合焉。南有管界寨巡檢司。西有神前街巡檢司,本神峰寨,在縣北,後遷潭溪,更名。

鉛山府南。元鉛山州,直隸江浙行省,治在八樹嶺之南。洪武初,降為縣,遷於今治。西南有銅寶山,湧泉浸鐵可以為銅。又有鉛山,產鉛銅及青綠。北有鵝湖山。南有分水嶺,與福建崇安縣界,上有分水關巡檢司。又有紫溪嶺,紫溪水出焉。北有上饒江,至汭口,與紫溪、桐木、黃蘗諸水合流,入弋陽縣界,謂之鉛山河口。又東北有石溪,亦流合上饒江。西南有石佛寨巡檢司,後遷善政鄉湖坊街。又西有駐泊巡檢司,治汭口鎮,廢。

永豐府南。東南有平洋山,舊產銀礦。南有永豐溪,源出福建浦城縣界,下流至上饒縣界合玉溪。又東有永平溪,西會杉溪及諸溪谷之水,注於永豐溪。東有柘陽寨巡檢司。又有杉溪寨巡檢司,廢。

興安府西。嘉靖三十九年八月以弋陽縣之橫峰寨置,析上饒、貴溪二縣地益之。縣南有宋溪,源並出靈山,下流入上饒江。東有丫嚴寨巡檢司,後廢。

建昌府元建昌路,屬江西行省。太祖壬寅年正月為肇慶府,尋曰建昌府。領縣五。西北距布政司四百里。

南城倚。永樂二十二年建荊王府。正統十年遷於湖廣蘄州。成化二十三年建益王府。西南有麻姑山。東有旴江,一名建昌江,自南豐縣流入,下流入金溪縣。東有藍田、北有伏牛二巡檢司。又南有曾潭、北有嶽口二巡檢司,廢。又東南有杉關,接福建光澤縣界。

南豐府南少西。元南豐州,直隸江西行省。洪武初,降為縣,南有軍山。又東南有百丈嶺,與福建建寧縣分界。又有旴水。東南有龍池巡檢司,本黃沙源坪,在縣西南,後遷縣南雙港口,又遷縣東南百丈嶺,又遷刊都,尋又遷於此,更名。又南有太平、北有仙君二巡檢司,廢。

新城府東南。西有福山,黎水出焉,經縣西,下流會於旴江。又東有飛猿嶺,飛猿水出焉,下流至南城縣入於日於江。又有五福港,源出杉關,流與飛猿水合。東南有極高巡檢司,遷水口村,後遷縣南德勝關,又遷縣東洵口,仍故名。西南有同安巡檢司,後遷縣西樟村,尋復。

廣昌府西南。西北有金嶂山。西南有梅嶺。又南有血木嶺,旴水出焉,經城南,流入南豐縣。西南有秀嶺、南有泉鎮二巡檢司。

瀘溪府東南。本南城縣瀘溪巡檢司,萬歷六年十二月改為縣。東有瀘溪,源出福建崇安縣之五鳳山,流至縣,又北入於安仁港。

撫州府元撫州路,屬江西行省。太祖壬寅年正月為臨川府,尋曰撫州府。領縣六。北距布政司二百四十里。

臨川倚。南有靈谷山。西有銅山,舊產銅。城東有汝水,上源接旴江,自金谿縣流入,東合於章江。又西有臨水,源出崇仁縣,流合汝水。北有溫家圳、南有青泥、西有清遠三巡檢司。又有白竿巡檢司,後廢。

崇仁府西。南有巴山,一名臨川山,臨水出焉,亦曰巴水。又南有華蓋山,西寧水出焉,下流俱合於汝水。又西南有寶唐山,寶唐水出其下,北合縣境諸溪,入於臨水。東有周坊巡檢司。又西北有丁坊、南有河亭二巡檢司,廢。

金谿府東南。東有金窟山,舊產金。又有云林山,跨撫、信、建昌三府境。又有崖山,接貴溪縣界。南有福水,即旴水下流也,自南城縣流入,北合清江水,又北合石門港水。又北流為苦竹水,又西流為臨川縣之汝水。

宜黃府西南。東有宜黃水,下流入汝。南有止馬寺巡檢司。又有上勝巡檢司,廢。

樂安府西南。西北有大盤山,與新淦、永豐二縣界,寶唐水出焉,下流合於臨水。東有芙蓉山,鰲溪水出焉,下流合於贛水。北有龍義、又有望仙二巡檢司。又西北有南平巡檢司,後廢。

東鄉府東。正德七年八月以臨川縣之孝岡置,析金谿、進賢、餘干、安仁四縣地益之。西南有汝水。東北有橫山、西北有古熂二巡檢司,後廢。

吉安府元吉安路,屬江西行省。太祖壬寅年為府。領縣九。東北距布政司五百九十里。

廬陵倚。北有螺山,南有神岡山,兩山相望,贛江經其下。又北經城東,又北經虎口石,流入峽江縣,為清江。南有富田、西有井岡、西南有敖城三巡檢司。

泰和府南少西。元太和州。洪武二年正月改為泰和縣。東有王山,亦名匡山。贛江在城南,自萬安縣流入,經縣西之牛吼石,而東北入廬陵縣界。又南有雲亭江,一名繒水,源出興國縣,北流至珠林口注於贛江。西有旱禾市、東北有花石潭、東南有三顧山三巡檢司。

吉水府東北。元吉水州。洪武二年正月降為縣。東有東山。北有王嶺。又東北有吉文水,贛水之支流。北有白沙巡檢司,遷縣西北三曲灘上,仍故名。

永豐府東。東有郭山。南有石空嶺,又有恩江,下流入於贛江。東南有層山、南有沙溪、又有表湖三巡檢司。又東北有視田巡檢司,後廢。

安福府西少北。元安福州。洪武二年正月降為縣。西有盧蕭山,盧水出焉,經城北,東流與王江合,又東合禾水,至廬陵縣神岡山下入於贛江。南有黃茆巡檢司,治黃陂寨,後遷縣西時礱鎮,西有羅塘巡檢司,治洋澤,後遷江背,俱仍故名。

龍泉府西南。東南有錢塘山。西有石含山。南有遂水,東流入於贛江。西北有北鄉巡檢司。西南有禾源巡檢司,後遷縣西左安司,仍故名。西有秀洲巡檢司,本金田,在縣北,後遷治,更名。

萬安府南。東有蕉源山,產鐵。城西有贛江,江之灘三百里,在縣境者十八灘,皇恐為最險。又南有皁口江,自贛縣北注於贛江。有造口巡檢司,在縣西南。又東北有灘頭巡檢司,又東南有西平山巡檢司,廢。

永新府西南。元永新州。洪武二年正月降為縣。東南有義山。西有秋山,一名禾山,禾水出焉,一名永新江,下流至泰和縣入於贛江。東南有上坪寨、西北有慄傳寨、又有禾山寨、又有新安寨四巡檢司。

永寧府西南。北有七溪嶺。西有漿山水,源自湖廣茶陵州界,流經縣南,合於永新縣之禾江。西有升鄉寨巡檢司。西南有礱頭寨巡檢司,尋廢。

臨江府元臨江路,屬江西行省。太祖癸卯年為府。領縣四。東北距布政司二百七十里。

清江倚。東有閣皁山,亙二百餘里。南有贛江,一名清江,有清江鎮巡檢司。又有袁江,自新喻縣流入,至縣南合焉。西有蕭水,南有淦水,至縣東清江鎮,亦俱合於贛江。西南有太平市巡檢司,廢。

新淦府南。元新淦州。洪武初降為縣。西北有離嶺,淦水出焉。又西有清江。又南有象江,有泥江,俱流入於清江。東有枉山巡檢司,後遷藍橋,尋復。

新喻府西。元新喻州,洪武初,降為縣。西有銅山,舊產銅。北有蒙山。南有渝水,即袁江,潁江水北流入焉。北有水北墟巡檢司。

峽江府南。本新淦縣之峽江巡檢司,嘉靖五年四月改為縣,析新淦縣六鄉地益之。南有玉笥山,又有贛江,亦名峽江,有黃金水流合焉。

袁州府元袁州路,屬江西行省。太祖庚子年為府。領縣四。東北距布政司三百九十里。

宜春倚。南有蟠龍山,又有仰山。又秀江在城北,源出萍鄉縣,流經府西,亦曰稠江,即袁江之上源也。西有黃圃、南有澗富嶺二巡檢司。

分宜府東。東有鐘山峽。西有昌山峽。秀江經兩峽中,入新喻縣境,為渝水。

萍鄉府西。元萍鄉州,洪武初降為縣。東有羅霄山,羅霄水出焉,分二派。東流者為盧溪水,下流為秀江,入宜春縣界。西流者入湖廣醴陵縣界,合淥水。又西有萍川水,亦曰楊岐水,西流經縣南,下流合淥水。北有安樂鎮、東南有大安里二巡檢司。又西有草市巡檢司,後遷於插嶺關,仍故名。又西有湘東市。東有盧溪鎮。

萬載府北。北有龍江,下流即瑞州府之蜀江。東北有康樂水入焉。西有鐵山界巡檢司。又有高村鎮巡檢司,尋廢。

贛州府元贛州路,屬江西行省。太祖乙巳年為府。領縣十二。西北距布政司一千一百八十里。

贛倚。南有崆峒山,章、貢二水夾山左右,經城之東西。貢水一名東江,自福建長汀縣流入府界。章水一名西江,自湖廣宜章縣流入府界。至城北,合流為贛江。北有桂源巡檢司,後遷攸鎮。東北有磨刀寨巡檢司,後遷石院鋪。南有長洛巡檢司,後遷縣西黃金鎮。俱仍故名。

雩都府東。東北有高沙寶山。又北有雩山,雩水出焉,合寧都、會昌諸水,繞城而西,至贛縣,合於貢水。東北有平頭寨巡檢司。又有印山、又有青塘二巡檢司,後廢。

信豐府東南。東有桃江,自龍南縣流入,經縣北,為信豐江,下流入於貢水。東南有新田巡檢司。西有桃枝墟,又有黃田、覃塘,又東有新設四巡檢司,後廢。

興國府東北。北有覆笥山。東北有瀲江,西南流,合雩水入貢江。東有衣錦鄉、東北有迴龍寨二巡檢司。

會昌府東少南。元會昌州,洪武初降為縣。南有四望山,下有羊角水隘。北有湘洪水,即貢水,西北流,會雩水。南有湘鄉寨、北有承鄉鎮二巡檢司。又西有河口巡檢司,後廢。

安遠府南。元屬寧都州,洪武初改屬府。西有安遠水,亦曰廉水,流入會昌縣之貢水。又南有三百坑水,下流入廣東龍川縣。西北有板口巡檢司。

寧都府東北。元寧都州,洪武初降為縣。西北有金精山。北有梅嶺。南有寧都水,與散水、筼簹、曲陽、黃沙、長樂五水合,又東北有虔化水,下流俱入於雩水。又有梅川水,出梅嶺,下流亦經雩都縣入貢水。東南有下河寨巡檢司。

瑞金府東。元屬會昌州,洪武初改屬府。東北有陳石山,綿江出焉,流至縣南入貢水,又西入會昌縣,為湘洪水。西北有瑞林、東北有湖陂二巡檢司。東南有古城鎮,路出福建長汀縣。

龍南府南。元屬寧都州,洪武初改屬府。西南有冬桃山,桃水出焉,東北流會諸水,至縣北宮山下,與渥、濂二水合為三江口,又北流為信豐縣之桃江。有冬桃隘,崇禎初,移定南縣下歷巡檢司駐焉。

石城府東北。元元貞元年十一月屬寧都州,洪武初改屬府。北有牙梳山。東有霸水,西南合虔化水,入貢江。北有捉殺寨巡檢司,後遷縣西赤江市,仍故名。

定南府東南。隆慶三年三月以龍南縣之蓮莆鎮置。析安遠、信豐二縣地益之。西北有程嶺,又南有神仙嶺。東有指揮峰。東北有九洲河,下流會於信豐縣之桃江。東北有下歷巡檢司,後遷高砂蓮塘,又遷龍南縣冬桃隘。

長寧府東南。萬歷四年三月以安遠縣之馬蹄岡置,析會昌縣地益之。東南有頂山,又南有大帽山,俱接閩、廣境。又東有尋鄔水,流入廣東龍川縣界。西北有黃鄉巡檢司。南有新坪巡檢司,本大墩,後更名。北有雙橋、南有丹竹樓二巡檢司,後廢。

南安府元南安路,屬江西行省。太祖乙巳年為府。領縣四。東北距布政司一千五百二十里。

大庾倚。西南有大庾嶺,五嶺之一,亦名梅嶺,上有關曰梅關。又有章江,亦曰南江,亦曰橫江,下流與貢水合。西有鬱林鎮巡檢司,治晶都村,後遷浮江隘,又遷黃泥港,東北有赤石嶺巡檢司,治峰山里,後遷小溪城,又遷峰山新城,後遷峰山水西村,俱仍故名。又縣南有水南城,與府城隔江對峙,嘉靖四十年築。西北有新田城。又北有鳳凰城,又西有楊梅城,俱嘉靖四十四年築。又東有九所城。亦嘉靖四十四年築。

南康府東北。西北有禽山,禽水出焉,東流至南野口入於章江。北有羊嶺山。南有芙蓉江,即章江。東北有潭口鎮、北有相安鎮二巡檢司。

上猶府東北。元永清縣,洪武初更名。西有書山,一名太傅山。東有大猶山,猶水出焉,下流至南康縣,入於章江。西有浮龍巡檢司,後遷太傅村,仍故名。

崇義府北。正德十四年三月以上猶縣之崇義里置,析大庾、南康二縣地益之。西南有聶都山。西有桶岡。又有章江,自湖廣宜章縣流入,又有橫水,經縣南,又西南有左溪,下流俱合章江。西北有上保巡檢司,本過步,後遷治,更名。西南有鉛廠巡檢司,本在鉛山,後遷聶都,東南有長龍巡檢司,本治降平里,後遷縣東北尚德里江頭,俱仍故名。

○湖廣浙江

湖廣《禹貢》荊、揚、梁、豫四州之域。元置湖廣等處行中書省,治武昌路,又分置湖南道宣慰司治天臨路屬焉。又以襄陽等三路屬河南江北等處行中書省,又分置荊湖北道宣慰司治中興路并屬焉。太祖甲辰年二月平陳理,置湖廣等處行中書省。洪武三年十二月置武昌都衛。與行中書省同治。八年十月改都衛為湖廣都指揮使司。九年六月改行中書省為承宣布政使司。領府十五,直隸州二,屬州十七,縣一百有八,宣慰司二,宣撫司四,安撫司五,長官司二十一,蠻夷長官司五。為里三千四百八十有奇。北至均州,與河南、陜西界。南至九疑,與廣東、廣西界。東至蘄州,與江南、江西界。西至施州,與四川、貴州界。距南京一千七百一十五里,京師五千一百七十里。洪武二十六年編戶七十七萬五千八百五十一,口四百七十萬二千六百六十。弘治四年,戶五十萬四千八百七十,口三百七十八萬一千七百一十四。萬曆六年,戶五十四萬一千三百一十,口四百三十九萬八千七百八十五。

武昌府元武昌路,屬湖廣行省。太祖甲辰年二月為府。領州一,縣九:

江夏倚。洪武三年四月建楚王府於城內黃龍山。東有黃鵠山,下為黃鵠磯,臨大江。又南有金水,一名塗水,西流至金口入江,有金口鎮巡檢司。又北有滸黃洲、西南有占魚口鎮二巡檢司。

武昌府東。西有樊山,一名西山,產銀銅鐵及紫石英。南有神人山,其下為白鹿磯。西有西塞山,與大冶縣界。北濱江,中有蘆洲,亦曰羅洲。又西南有樊港,一名樊溪,又名袁溪,匯縣南湖澤凡九十九,北入大江,曰樊口。又東有南湖,一名五丈湖,通大江。東有金子磯鎮、又有赤土磯鎮、西南有白湖鎮三巡檢司。南有金牛鎮、西有三江口鎮二巡檢司,後廢。

嘉魚府西南。西有赤壁山,與江夏縣界。北岸對烏林。西北濱大江,有陸水流入焉,曰陸口,亦曰蒲圻口。東北有簰洲鎮、西南有石頭口鎮二巡檢司。

蒲圻府西南。西有蒲首山。南有蒲圻河,即陸水也。又西有蒲圻湖。西南有新店等湖,下流至嘉魚縣之石頭口,注於大江。西南有羊樓巡檢司。

咸寧府東南。陳友諒時徙治河北。洪武中復還故城,即今治也。西有淦水,即金水之別名。

崇陽府南。西有岩頭山。西南有龍泉山。東北有壺頭山,下有壺頭港,亦曰崇陽港,匯群川西合陸水,又名雋水。

通城府西南。南有錫山,舊產銀錫。北有陸水,自巴陵縣流入。

興國州元興國路,屬湖廣行省。太祖甲辰年二月為府。洪武九年四月降為州,以州治永興縣省入,來屬。北有銀山,西有黃姑山,舊俱產銀。南有太平山,與九宮山接。東有大坡山,產茶。東北有大江。東有富池湖,亦曰富水,北流注於江,有富池鎮巡檢司。又東北有黃顙口鎮巡檢司。西北距府三百八十里。領縣二:

大冶州西北。北有鐵山,又有白雉山,出銅礦。又東有圍爐山,出鐵。又西南有銅綠山,舊產銅。大江在北。有道士洑巡檢司。

通山州西少南。東南有九宮山,寶石河出焉,下流合於富水。東有黃泥壟巡檢司。

漢陽府元屬湖廣行省。洪武九年四月降為州,屬武昌府。十三年五月復為府。屬湖廣布政司,尋屬河南。二十四年六月還湖廣。領縣二。西北距布政司,隔江僅七里。

漢陽倚。洪武九年四月省。十三年五月復置。大別山在城東北,一名翼際山,又名魯山。漢水自漢川縣流入,舊逕山南襄河口入江。成化初,於縣西郭師口之上決,而東從山北注於大江,即今之漢口也,有漢口巡檢司。大江自巴陵縣西北接洞庭之水,流入府境,至此與漢水會。又西南有沔水,即漢水支流也,仍合漢入江。又有沌水,大江支流也,自沔陽州流入,仍入大江,謂之沌口,有沌口巡檢司。又有弇水,在大江南岸,至弇口入江。又北有灄水,亦漢水支流也,有淪水流合焉,下流注于大江。又西有太白湖,江北諸水多匯焉。西有蔡店鎮、西南有新灘鎮二巡檢司。又西南有百人磯鎮巡檢司,後遷於東江腦。

漢川府西少北。元屬漢陽路。洪武九年四月改屬武昌府。十三年五月還屬。南有小別山,一名甑山,又有陽臺山。西南有漢水。東有溳水,自雲夢縣來,南入漢,謂之溳口。北有劉家塥巡檢司。

黃州府元黃州路,屬河南江北行省。太祖甲辰年為府,屬湖廣行省。九年屬湖廣布政司,尋改屬河南。二十四年六月還屬湖廣。領州一,縣八。西南距布政司百八十里。

黃岡倚。南有故城。洪武初,徙於今治。南濱大江,西北岸有赤鼻磯,非嘉魚之赤壁。西有三江口,其上流水分三派,至此合流。中有新生洲,又有崢嶸洲。東有巴河,西有舉水,俱入於江。江濱西有陽邏鎮、北有團風鎮、又西北有中和鎮三巡檢司。又有鹿城關,有大活關。又東北有陰山關。

麻城府北。東有龜峰山,舉水出焉,流入黃岡縣。東南有長河,又南有縣前河流入焉,下流注於江西。有雙城鎮、鵝籠鎮,東北有虎頭關三巡檢司。又西北有木陵關,在木陵山上。東北有陰山關,在陰山上。又北有黃土關,與木陵、虎頭、白沙、大城為五關。又西有岐亭鎮,嘉靖五年築城。

黃陂府西。東南濱大江,有武湖自西來,入於江,曰武口,又曰沙武口,亦曰沙洑口。又西有灄水,自漢陽流入江,曰灄口。北有大城潭鎮巡檢司。又北有白沙關,即麻城五關之一也。

黃安府西北。嘉靖四十二年以麻城縣之姜家畈置,析黃岡、黃陂二縣地益之。東有三角山,接蘄水、羅田、蘄州界。又有東流河,下流出團風口入江。西有西河,又有雙河,合流出灄口,入漢。又北有雙山關巡檢司。西北有金扃關,亦曰金山關,與河南羅山縣界。

蘄水府東少南。元屬蘄州路。洪武九年四月屬蘄州。十一年十月改屬府。西南濱大江。南有浠水,源出英山縣,流經縣境西南入江。又東有蘭溪,東南流入浠水。又北有巴水,源出縣之板石山,流入黃岡縣界。有蘭溪鎮、巴河鎮二巡檢司。

羅田府東北。元屬蘄州路。洪武九年四月屬蘄州。十一年十月改屬府。東南有浠水。西北有平湖水。南有官渡河,亦名縣前河,平湖水流入焉,下流合黃岡縣之巴河,入大江。東北有多雲鎮巡檢司,又有栗子關,又有上岐嶺、中岐嶺、下岐嶺等關。西北又有平湖關。

蘄州元蘄州路,屬河南江北行省。太祖甲辰年為府。九年四月降為州,以州治蘄春縣省入,來屬。正統十年,荊王府自江西建昌遷此。東北有百家冶山,產蘄竹。南濱江。東北有蘄水,出大浮山,經州北,匯為赤東湖,西南流,接蘄水縣界,注於大江。西有茅山鎮、北有大同鎮二巡檢司。西距府二百十里。領縣二:

廣濟州東北。南濱江,江中有中洲。崇禎末遷治於此,尋復故。又有武山湖、馬口湖皆流通大江。南有武家穴鎮、西南有馬口鎮二巡檢司。

黃梅州東北。東南有礦山,舊產鐵。大江在南,江濱有太子洑。又南有縣前河,由小池口入江。西南有新開口鎮巡檢司,屢圮於江,內徙。又南有靖江觜鎮巡檢司。

承天府元安陸府,屬荊湖北道宣慰司。太祖乙巳年屬湖廣行省。洪武九年四月降為州,直隸湖廣布政司。二十四年六月改屬河南,未幾還屬。弘治四年,興王府自德安府遷此。嘉靖十年升州為承天府。十八年建興都留守司於此。領州二,縣五。東南距布政司五百七十里。

鐘祥倚。洪武二十四年建郢王府,永樂十二年除。二十二年建梁王府,正統六年除。元曰長壽縣,元末廢。洪武三年復置。九年四月省入州。嘉靖十年八月復置,更名。東有兩木山,一名青泥山。北有松林山,興獻王陵寢在焉,嘉靖十年賜名純德山,置顯陵縣於此。明末,縣廢。西濱漢水。北有直河,自隨州流入,有滶水流合焉。又有豐樂水,又東有臼水,俱注於漢水。

京山府東。南有縣河,下流至景陵縣,入漢江。又東北有撞河,自隨州流入,至漢川縣入漢江,或謂之富水。

潛江府東南。元屬中興路。洪武十年八月來屬。北有漢水。西北有潛水,即漢水分流,經縣東南入於漢。又東南有深江,又南有恩江,皆漢水支分也。西南有沱水,為江水之分流,經縣南,有重湖環繞,又東匯於漢水。

荊門州元治長林縣,屬荊湖北道宣慰司。洪武九年四月改為縣,省長林縣入焉,屬荊州府。十三年五月復為州,仍屬荊州府。嘉靖十年八月來屬。東南有章山,即內方山也。漢水逕其東,亦曰沔水。又西有權水,東南有直江,一名直河,又有陽水,一名建水,皆流入焉。南有荊門守禦千戶所。北有宜陽守禦千戶所。東南有建陽鎮、新城鎮,西北有仙居口,北有樂仙橋四巡檢司。東北距府九十里。領縣一:

當陽州西。元屬荊門州。洪武九年改屬荊州府。十年五月省入荊門縣。十三年五月復置,仍屬州。東南有方城,洪武初移治於此。十三年復故。南有玉泉山,玉泉水出焉。北有沮水,源出房縣,逕縣東南,合榕渡,與漳水會,下流至枝江縣,入於大江。北有漳河口巡檢司。

沔陽州元沔陽府,屬荊湖北道宣慰司。洪武九年四月降為州,以州治玉沙縣省入,直隸湖廣布政司,尋直隸河南。二十四年六月還直隸湖廣。嘉靖十年十二月來屬。東南有黃蓬山,其下為黃蓬湖。南有大江。北有漢水。東有太白湖,州西十四湖之水悉匯焉,由漢陽縣之沌口入於大江。又南有長夏河,江水支流也,亦曰夏水。西北有襄水,漢水支流也,至州東北瀦口合流,東入於沔水。東有沙鎮、西南有茅鎮二巡檢司。西北距府三百二十五里。領縣一:

景陵州西北。南有沔水。西南有楊水,北注沔,謂之楊口,亦曰中夏口,又曰楊林口。又有中水,流合楊水,曰中口。東有乾鎮巡檢司。

德安府元屬荊湖北道宣慰司。洪武元年十月屬湖廣行省。九年四月降為州,屬黃州府。十一月屬武昌府。十三年五月復為府,屬湖廣布政司。二十四年六月改屬河南,未幾還屬。領州一,縣五。東南距布政司四百里。

安陸倚。成化二十三年建興王府。弘治四年遷於安陸州。八年建岐王府,十四年除。正德元年,壽王府自四川保寧府遷此,嘉靖二十四年除。四十年建景王府,四十四年除。洪武初,縣省。十三年五月復置。東有章山,即豫章山。溳水在城西,俗稱府河,亦曰石潼河,又西有漳水入焉,謂之漳口。南有高核鎮巡檢司,後移於隨州之合河店。

雲夢府東南。西南有溳水。東有興安鎮巡檢司,後廢。

應城府西南。洪武九年四月屬黃州府。十年五月省入雲夢縣。十三年五月復置。西北有西河,下流入漢水。又峙山鎮巡檢司亦在西北。

孝感府東南。洪武九年四月屬黃州府。十年五月省入德安州。十三年五月復置。北有澴水,下流入於漢水。南有淪河,自溳河分流至漢陽,合灄水入江。北有小河溪、東南有馬溪河二巡檢司。

隨州洪武二年正月以州治隨縣省入。九年四月降為縣,屬黃州府。十年五月省入應山縣。十三年五月復升為州。西有大溪山,水員水出焉,下流至漢川縣入漢水。又西有大洪山,漳水所出。西北有溠水,源出栲栳山,又有水厥水流入焉。又南有浪水,源出大猿山,下流俱注於溳水。又西北有合河店、東北有出山鎮二巡檢司。東南距府百八十里。領縣一:

應山州東。洪武初省。十三年五月復置。西有雞頭山,澴水出焉。西南有溳水。東有白泉河,與澴水合,入孝感縣界。西北有杏遮關巡檢司,即平靖關,義陽三關之一。又西南有平里市巡檢司。又東北有武陽關,一名武勝關,又名禮山關,亦義陽三關之一。

岳州府元岳州路,屬湖廣行省。太祖甲辰年為府。洪武九年四月降為州,直隸布政司。十四年正月復為府。領州一,縣七。東北距布政司五百里。

巴陵倚。洪武九年四月省,十四年復置。西南有巴丘山。又有君山,在洞庭湖中。大江在西北。洞庭湖上納湘、澧二水,自西南來合,謂之三江口。湖之南有青草湖,又西曰赤沙湖,謂之三湖。沅、漸、元、辰、敘、酉、澧、澬、湘九水,皆匯於此,故亦名九江。東南有水邕湖,亦名翁湖。南有鹿角巡檢司。

臨湘府東北。東南有龍窖山,跨臨湘、通城、當陽、蒲圻四縣界。西南有城陵磯,又有道人磯,皆濱大江,有城陵磯巡檢司。又南有土門鎮、東北有鴨欄磯二巡檢司。

華容府西北。東有東山,又有石門山。大江在北。又有華容河,自大江分流,南達洞庭湖。南有澧水,東流入洞庭湖。西南有赤沙湖,與洞庭湖接。南有明山古樓巡檢司。又東北有黃家穴巡檢司,後移於塔市。北有北河渡巡檢司,後廢。

平江府東南。元平江州,洪武三年降為縣。北有永寧山。東北有幕阜山。東有汨水,西南流,昌水北流入焉。東北有長壽巡檢司。

澧州元澧州路,屬湖廣行省。太祖甲辰年為府。九年四月降為州,以州治澧陽縣省入,屬常德府。三十年三月來屬。元元貞末徙治新城。洪武五年復舊治。東有關山。西南有大浮山,跨石門、武陵、桃源三縣界。南有澧水,一名蘭江,亦曰繡水。其東有澹水,北有涔水,俱流入焉。東有嘉山鎮巡檢司。東距府二百七十里。領縣三:

安鄉州東南。西有澧水,一名長河。北有涔水。

石門州西。南有澧水。西北有渫水,亦名添平河,自添平所南流入焉。

慈利州西少南。元慈利州,洪武二年降為縣。西南有天門山,有檳榔洞,與瑤分界。又西有崇山。又有歷山,澧水出焉,下流至華容縣入於洞庭湖。又西有漊水,源出四川巫山縣,東流合諸溪澗之水,至縣西匯於澧水,亦曰後江。西南有永定衛,洪武中置,二十三年八月徙於永順宣慰司之芋岸坪。西北有龍伏關,東南有後平關、黑崇關,謂之永定三關。所屬曰大庸守禦千戶所,本大庸衛,在衛西,洪武九年四月置,三十一年改為所,曰茅岡長官司,在衛東北,正統中永定衛置。北有九溪衛,洪武二十三年六月置,有九淵、野牛、三江口、閘口四關。所屬曰守御添平千戶所,在衛北,洪武二年置。曰守禦安福千戶所,在衛西北,洪武二十三年九月置。曰守御麻寮千戶所,在衛北,洪武四年置。曰桑植安撫司,本桑植、荒溪等處宣撫司,在衛西北,太祖丙午年二月置,後廢,永樂四年十一月改置。

荊州府元中興路,屬荊湖北道。太祖甲辰年九月改為荊州府,屬湖廣行省。吳元年十月置湖廣分省於此,尋罷。九年屬湖廣布政司,尋改屬河南。二十四年還屬。領州二,縣十一。東距布政司千二百一十里。

江陵倚。洪武十一年正月建湘王府,建文元年四月除。永樂元年,遼王府自遼東廣寧遷於此,隆慶二年十月除。萬歷二十九年十月建惠王府。南濱江。東南有夏水,至沔陽州合於沔水,故沔水亦兼夏水之名。又有陽水,東北至景陵縣,入沔水。又東北有三海,沮、漳水匯流處。北有柞溪。又東有靈溪,亦曰零水,南入江,謂之零口。東北有龍彎市、東南有沙頭市、南有郝穴口、西南有虎渡口四巡檢司。

公安府東南。東北有舊城。今治崇禎元年所遷。北濱江,西北有油河流入焉,謂之油口,有油口巡檢司。東北有夏水。

石首府東南。元末治楚望山北,洪武中徙繡林山左,本宋時舊治也。北濱江,江中有石首山。又東有焦山,下有港,通洞庭湖。有調弦口巡檢司。

監利府東少南。南濱江。東南有魯洑江,亦曰夏水,自大江分流,下至沔陽州入沔。又西有涌水,南入江,謂之涌口。又東有瓦子灣、西有窯所、南有白螺磯、北有毛家口、又有分鹽所五巡檢司。

松滋府西南。西南有巴山。北濱大江。南有紅崖子巡檢司。又有西坪塞巡檢司,後廢。

枝江府西。洪武十年五月省入松滋縣。十三年五月復置。北濱大江,江中有百里洲,江水經此而分,故曰枝江。北有沮水,南入江,謂之沮口。

夷陵州元峽州路,屬荊湖北道宣慰司。太祖甲辰年為府。九月降為州,直隸湖廣行省。九年四月改州名夷陵,以州治夷陵縣省入,來屬。大江在南。西北有關曰下牢關,夾江為險。又有西陵、明月、黃牛三峽,峽中有使君、虎頭、狼尾、鹿角等灘,皆江流至險處也。西北有赤谿,東合大江。南有南津口巡檢司。又東有金竹坪巡檢司,後廢。又西有西津關。東北有白虎關。東距府三百四十里。領縣三:

長陽州西南。東南有清江。西有舊關堡、西南有蹇家園、南有漁洋關三巡檢司。南有古捍關。西有梅子八關。

宜都州東南。西北有荊門山,下臨大江,其對岸即虎牙山也。又西有清江,東流合大江,有清江口巡檢司。又西北有古江關、東北有普通鎮二巡檢司。

遠安州東北。舊治亭子山下。成化四年遷於東莊坪。崇禎十三年又遷鳳凰山麓,即今治也。東北有沮水。

歸州元治秭歸縣,直隸湖廣行省。洪武九年四月廢州入秭歸縣,屬夷陵州。十年二月改縣名長寧。十三年五月復改縣為歸州。舊治江北,後治白沙南浦。洪武初,徙治丹陽。四年徙長寧,在江南楚王臺下。嘉靖四十年復還江北舊治。東有馬肝、白狗、空舲等峽。大江在州北,經峽中,入夷陵界。其西有叱灘、蓮花灘、新灘,皆濱江。西北有牛口巡檢司,後遷於巴東縣利洲。東南有南邏口巡檢司,後遷於新灘。東距府五百二十里。領縣二:

興山州西北。洪武九年改屬夷陵州,後還屬。正統九年三月省入州。弘治三年五月復置。南有香溪,亦曰縣前河,南流入江。東北有高雞寨巡檢司。又東有桑林坪巡檢司,後廢。又北有貓兒關,達鄖、襄。

巴東州西。元屬歸州。洪武九年改屬夷陵州。隆慶四年還屬。北濱大江,自四川巫山縣流入,東經門扇、東奔、破石,謂之巴東三峽,下流至黃梅縣入南直宿松縣界。又南有清江,一名夷水,自四川建始縣流入,下流入於大江。又北有鹽井。西南有連天關巡檢司。南有野山關巡檢司,本治石柱,隆慶四年更名。

襄陽府元襄陽路,屬河南江北行省。太祖甲辰年為府,屬湖廣行省。九年屬湖廣布政司。二十四年六月改屬河南,未幾,還屬湖廣。領州一,縣六。東南距布政司六百八十里。

襄陽倚。正統元年,襄王府自長沙遷此。南有虎頭山,又有峴山。東南有鹿門山。又西有隆中山。漢水在城北,亦曰襄江。白河在城東北,與唐河合,南入漢,謂之白河口,亦曰三州口。又西北有青泥河,南有浮河,西南有檀溪,下流皆入於漢。北有樊城,有樊城關巡檢司,後移於縣東北之柳樹頭。又東北有雙溝口巡檢司。又西有油坊灘巡檢司,嘉靖十九年移於縣西北之北泰山廟鎮。

宜城府東南。東有漢水。西有蠻水,亦曰夷水,源出房縣,流至縣界,入漢水,其支流曰長渠。又有沶水,自漢中流入,合於蠻水,謂之沶口。又有疏水,在縣東北,自南漳縣流入,注漢,謂之疏口。

南漳府西南。西北有荊山。南有蠻水,又有沮水,又有漳河,流入當陽縣,合於沮水。東有方家堰、西南有金廂坪二巡檢司。又西有七里頭巡檢司,後移於保康縣之常平堡。

棗陽府東北。洪武十年五月省入宜城縣,後復置。東南有白水,南有濜水流合焉,西流於沔水,此縣內之白水也。又西南有滾河,流入襄陽之白河。東北有鹿頭店巡檢司。

穀城府西少北。東北有漢水,又有均水流入焉,謂之均口。又有築水,經縣治東南,注於漢水,曰築口。西有石花街巡檢司。

光化府西北。洪武十年省入穀城縣。十三年五月復置。舊治在西。隆慶末,改建於阜城衛,即今治也。東有馬窟山。北有漢水。東有白河,即水肓水,自河南新野縣流入,有泌河流合焉。西北有左旗營巡檢司,萬曆中,徙於縣舊城。

均州洪武二年七月以州治武當縣省入。南有武當山,永樂中,尊為太嶽太和山。山有二十七峰、三十六巖、二十四澗。北有漢江,一名滄浪水。東北有均水,自河南淅川縣流入。又東南有黑虎廟巡檢司。東南距府三百九十里。

鄖陽府成化十二年十二月置。領縣七。又置湖廣行都指揮使司於此。衛所俱無實土。東南距布政司千二百里。

鄖倚。元屬均州。成化十二年置鄖陽府,治此。漢水在南。東南有龍門山,龍門河出焉,下流入於漢水。西北有青桐關。東北有雷峰、椏鎮二巡檢司。

房府南少西。元房州,屬襄陽路。洪武十年五月以州治房陵縣省入,又降州為縣,仍屬襄陽府。成化十二年十二月來屬。西南有景山,一名雁山,沮水出焉,流入遠安縣界。又南有粉水,亦曰彭水,又有築水,俱流入穀城縣,注漢。西南有板橋山巡檢司,後移於縣東南之博磨坪。

竹山府西南。元屬房州。洪武十年五月省入房縣。十三年五月復置,屬襄陽府。成化十二年十二月來屬。東有方城山。西有築山,築水出焉,流入房縣界。又有上庸山,上庸水所出,南合孔陽水,下流入漢。又南有堵水,源出陜西平利縣界,東流入漢。西北有黃茅關、吉陽關二巡檢司。

竹溪府西南。本竹山縣之尹店巡檢司,成化十二年十二月改置縣,而移巡檢司於縣東之縣河鎮,尋又遷巡檢司於白土關。南有竹溪河。

上津府西北。洪武初置,屬襄陽府。十年五月省入鄖陽。十三年五月復置,仍屬襄陽府。成化十二年十二月來屬。西有十八盤山,又有吉水,西南流入漢,俗謂之夾河。南有江口鎮巡檢司。

鄖西府西北。成化十二年十二月以鄖縣之南門保置。南有漢江,自陜西白河縣流入,下流至漢陽縣入於江。

保康府東南。弘治十年十一月以房縣之潭頭坪置。北有粉水,東南有常平堡,嘉靖十九年移南漳縣之七里頭巡檢司於此。

長沙府元天臨路,屬湖南道宣慰司。太祖甲辰年為潭州府。洪武五年六月更名長沙。領州一,縣十一。東北距布政司八百八十里。

長沙倚。治西北。洪武三年四月建潭王府,二十三年除。永樂元年,谷王府自北直宣府遷於此,十五年除。二十二年建襄王府,正統元年遷于襄陽。天順元年三月建吉王府。縣舊治城外,洪武初,徙城中。十八年復徙北門外。萬曆二十四年徙朝宗門內。西有湘水,源出廣西興安縣,流入境,合瀟水、烝水北流,環府城,東北出至湘陰縣,達青草湖,注洞庭湖,行二千五百餘里。北有瀏陽水,西流入湘,謂之瀏口。又有麻溪,流入湘水,曰麻溪口。又西北有喬口巡檢司,喬江與澬江合流處。

善化倚。治東南。舊治在城外,洪武四年徙於城中。十年五月省入長沙縣。十三年五月復置,治在南門外。成化十八年仍徙城中。西南有嶽麓山,湘江繞其東麓。又有靳江,流入湘江。西有橘洲,在湘江中。南有暮雲市巡檢司。

湘陰府北。元湘陰州。洪武初降為縣。北有黃陵山。西有湘水,北達青草湖,謂之湘口。湖在縣北,與洞庭連,亦曰重湖。南有哀江。又北有汨羅江,汨水自平江縣流入,分流為羅水,會於屈潭,西流注湘,謂之汨羅口。西北有營田巡檢司。

湘潭府西南。元湘潭州。洪武三年三月降為縣。東有昭山,下有昭潭。西有湘水,西南有涓水流入焉。南有下灄市巡檢司。

瀏陽府東。元瀏陽州。洪武二年降為縣。北有道吾山。東北有大光山。又有大圍山,瀏水出焉,經縣南,入長沙縣界,曰瀏陽水。東南有渠城界、梅子園二巡檢司。又有翟家寨巡檢司,後廢。

醴陵府東南。元醴陵州。洪武二年降為縣。南有淥水,亦曰漉水,西北注於湘水,有淥口巡檢司。

寧鄉府西。西有大溈山。北有澬江,源出綏寧縣,經此入沅江縣界,注洞庭。

益陽府西北。元益陽州。洪武初降為縣。西南有澬江,亦曰益水。東有喬江,澬江之分流也,下流復合於澬江。

湘鄉府西南。元湘鄉州。太祖甲辰年降為縣。西有龍山,漣水出焉,經縣東南,下流入於湘水。又西有湄水,南有豐溪水,俱入於漣水。西南有武障市巡檢司。又有永豐市、虞磨市二巡檢司,後廢。

攸府南少東。元攸州。洪武三年三月降為縣。南有司空山。東有攸水,自江西安福縣流入,東南有洣水流合焉,下流至衡山縣,入於湘水。南有鳳嶺巡檢司,後廢。

安化府西。東有浮泥山,有大峰山。西北有辰山,西有澬江。又南有善溪,自武陵縣流注於澬江。

茶陵州元直隸湖南道。太祖甲辰年降為縣。成化十八年十月復為州。西有雲陽山。西北有洣水,自酃縣流入。又東南有茶水,源出江西永新縣之景陽山,西流來合焉,北入攸縣之攸水。東有視渡口巡檢司。北距府四百五十里。

常德府元常德路,屬湖廣行省。太祖甲辰年為府。領縣四。東北距布政司一千零五十里。

武陵倚。弘治四年八月建榮王府。東南有善德山。南有沅水,又有朗水流入焉,謂之郎口。又東北有漸水,即鼎水也,自九溪衛流入。

桃源府西。元桃源州。洪武二年降為縣。西有壺頭山,接武陵、沅陵界。南有沅水,東有朗溪,西南有泥溪,俱流入焉。又西南有高都巡檢司。又南有白馬巡檢司,本名蘇溪,治縣東後春村,尋徙,更名,後廢。

龍陽府東少南。元龍陽州。洪武三年三月降為縣。舊治在東,今治景泰元年十二月所徙。東有軍山。北有沅水,東北有鼎水流入焉,謂之鼎口,有鼎港口巡檢司。又東南有赤沙湖,一名蠡湖。又西北有小江口巡檢司。

沅江府東南。元屬龍陽州。洪武三年州廢,來屬。十年五月省入龍陽縣。十三年五月復置。西南有沅水。又有澬水、澧水,並流入縣境,至縣東北入洞庭湖。

衡州府元衡州路,屬湖南道宣慰司。太祖甲辰年為府。領州一,縣九。東北距布政司一千三百里。

衡陽倚。弘治十二年,雍王府自四川保寧府遷此,正德二年除。萬曆二十九年十月建桂王府。南有回雁峰,北有岣嶁峰。衡山之峰七十二,在縣者凡七,而二峰最著。東有湘水,又有蒸水自西南流入焉,謂之蒸口。又東北有耒水,注湘,謂之耒口。又東有酃湖。又東有新城縣,元末置。洪武十年五月省為新城市,江東巡檢司治此。西南有松柏市巡檢司。

衡山府東北。元屬天臨路。洪武間改屬。西有衡山,有七十二峰、十洞、十五岳、三十八泉、二十五溪、九池、九潭、九井,而峰之最大者曰祝融、紫蓋、雲密、石廩、天柱,惟祝融為最高。東有湘江。東南有茶陵江,即洣水也,自攸縣合攸水流入境,注於湘,曰茶陵口。東有草市、東南有雷家埠二巡檢司。

耒陽府東南。元耒陽州,直隸湖南道。洪武三年三月降為縣。耒水在北。東有侯計山,肥水出焉,西南入耒水。又西南有羅渡巡檢司。

常寧府南。元常寧州,直隸湖南道。洪武三年三月降為縣。西北有湘水,東有舂陵水合焉。

安仁府東少北。西有楊梅峰。南有小江水,自郴州流入,西北流至衡山縣,合於洣水。南有安平、北有潭湖二巡檢司。

靈阜府東。洣水在縣東,源出洣泉,西有雲秋水流合焉。

桂陽州元桂陽路,治平陽縣,屬湖南道宣慰司。洪武元年為府。九年四月降為縣,省平陽縣入焉。十三年五月升為州。西有大湊山。南有晉嶺山。北有潭流嶺。舊皆產銀鉛砂礦。西有藍山。西北有舂陵水,又西有巋水流合焉。北有泗州寨、南有牛橋鎮二巡檢司。西北距府三百里。領縣三:

臨武州南。西北有舜峰山。西有西山,武水出焉,經宜章縣合於章水。東北有兩路口巡檢司。又東有赤土巡檢,後廢。

藍山州西南。舊治在縣北,洪武元年徙於此,屬郴州。二年來屬。南有黃檗山。東南有華陰山。西南有九疑山,山有杞林峰,巋水出焉,亦名舜水,北流合舂陵水。又西有守禦寧溪千戶所,洪武二十九年三月置。東有毛俊鎮、北有乾溪鎮、西南有大橋鎮三巡檢司。又西有小山堡、張家陂二巡檢司,後廢。

嘉禾州西南。崇禎十二年以桂陽州之倉禾堡置,析臨武縣地益之。東南有巋水,自藍山縣流入,北經石門山,又東北入州界。

永州府元永州路,屬湖南道宣慰司。洪武元年為府。領州一,縣七。東北距布政司千八百二十里。

零陵倚。北有湘水,經城西,瀟水自南來合焉,謂之湘口,有湘口關。又南有永水,源出縣西南之永山,北流入於湘水。北有黃楊堡巡檢司,本高溪市,隆慶元年徙治,更名。

祁陽府東北。舊治在縣西,景泰元年十二月徙於今治。北有祁山,上有黃羆鎮。西北有四望山。西有湘水。又城北有祁水,源出邵陽縣,東北流入焉。南有浯溪,下流亦入湘水。又東有歸陽市、東南有白水市、西北有水隆太平市三巡檢司。又東北有湘江市巡檢司,後移於縣東北之排山。

東安府西北。八十四渡山在縣東。又東南有湘水,自廣西全州流入。又有盧洪江,源出縣北九龍巖,經城東,下流入湘水。有盧洪市巡檢司。又有結陂市巡檢司,後廢。

道州元道州路,屬湖南宣慰司。洪武元年為府。九年四月復降為州,以州治營道縣省入,來屬。西有營山,營水出焉,至泥江,與江華縣之沲水合。東有瀟江,至青口,合於沲水。又西有濂溪,源出州西安定山下,東北合宜水,謂之龍灘,下流俱入湘水。北距府百五十里。領縣四:

寧遠州東少北。南有九疑山,介衡、永、郴、道之間。山有硃明峰,瀟水出焉。又南有舜源水,北流與江華縣沲、瀟二水合為三江口。南有九疑、魯觀巡檢司,在九疑、魯觀二峒口。

江華州南。東南有故城。今治本寧遠衛右千戶所,洪武二十八年置。天順六年徙縣來同治。西有白芒嶺,即萌渚嶺,五嶺之第四嶺也。東有沲水,源出九疑山之石城、娥皇二峰,下流合於瀟水。又東南有砅水,源出九疑山之女英峰,流合沲水。又東有守禦錦田千戶所,洪武二十九年置。又有錦田巡檢司。又西南有錦岡巡檢司,又有濤墟市巡檢司,後移於寧遠縣之九疑、魯觀。

永明州西少南。北有永明嶺,即都龐嶺,五嶺之第三嶺也。南有遨水,自廣西富川縣流入,下流注於瀟水。東南有枇杷守御千戶所,西南有桃川守御千戶所,俱洪武二十九年置。又有桃川市巡檢司。又西南有白面墟巡檢司。

新田州東北。崇禎十二年以寧遠縣之新田堡置。西北有舂陵山,與寧遠縣界,舂陵水出焉,下流至常寧縣,合於湘水。東南有白面寨巡檢司。

寶慶府元寶慶路,屬湖南道宣慰司。洪武元年為府。領州一,縣四。東北距布政司千二百五十里。

邵陽倚。南有高霞山,東有烝水。又北有澬水,邵水自東流合焉,有五十三灘,又有四十八灘,皆澬水所經。西北有龍回巡檢司。又北有巨口關。東北有白馬關。

新化府北。南有上梅山,其下梅山在安化縣境。東南有澬水。西南有長鄄巡檢司,尋廢。又北有蘇溪巡檢司。

城步府西南。本武岡州之城步巡檢司。弘治十七年改置縣,析綏寧縣地益之,而遷巡檢司於縣東北之茅坪鋪,尋又遷山口,後廢。東南有羅漢山,又有巫水,下流入於澬水。

武岡州元武岡路,屬湖南道宣慰司。洪武元年為府。九年四月降為州,以州治武岡縣省入,來屬。永樂二十二年,岷王府自雲南遷於此。北有武岡山。南有雲山。又有澬水。西南有都梁水,東北流入焉。北有蓼溪隘、峽口鎮,南有石門隘,東有紫陽關四巡檢司。東有石羊關。東距府二百八十里。領縣一:

新寧州東南。舊治在縣東。景泰二年移於沙洲原。南有夫夷水,北流合都梁水。東南有靖位、西有新寨二巡檢司。

辰州府元辰州路,屬湖廣行省。太祖甲辰年為府。領州一,縣六。東北距布政司千七百里。

沅陵倚。西北有大酉山、小酉山。東有壺頭山。西南有沅水,辰水自東北流入焉。又東有百曳、高湧、九磯、清浪等灘。又酉水在西北,東南入沅水。東有大刺、西北有明溪、又有會溪、東北有池蓬四巡檢司。又有高巖巡檢司,後廢。

盧溪府西少南。南有沅水。西有武溪,即潕溪也,下流合於沅水。又西有鎮溪軍民千戶所,洪武三十年二月置。又南有溪洞巡檢司。又西有河溪、西南有院場坪二巡檢司,後廢。

辰溪府西南。東南有五城山。西北有沅水。西有辰水。又東有渡口鎮、南有晉市鎮二巡檢司,後廢。

漵浦府東南。東有紅旗洞。西有漵水,下流入沅水。南有龍潭、東北有鎮寧二巡檢司。

沅州元沅州路,直隸湖廣行省。太祖甲辰年為府。九年四月降為州,以州治盧陽縣省入,來屬。北有明山。南有沅江,其源出四川遵義縣,下流至沅江縣,入洞庭湖。西有舞水,即無水也,流入於沅水。西有晃州巡檢司。又西南有西關渡口巡檢司,後廢。東北距府二百七十里。領縣二:

黔陽州東南。東南有羅公山。南有雙石崖,一名屏風崖。景泰中,築寨置戍於此,名安江雙崖城。北有沅水。又東有洪江,西有郎江,南有黔江,俱流入焉。東有安江巡檢司。又西有托口寨。東有洪江寨。

麻陽州北少西。東有包茅山。西有蠟爾山,與保靖司及四川、貴州界,諸苗蠻在山下者凡七十四寨。南有辰水,自貴州銅仁府流入。西有錦水,下流入於辰州。東北有巖門巡檢司。

郴州元郴州路,屬湖南道宣慰司。洪武元年為府。九年四月降為州,以州治郴陽縣省入,直隸布政司。南有黃岑山,與宜章縣界,亦曰騎田嶺,五嶺之第二嶺也,其支嶺曰摺嶺。又東北有雲秋山,與靈阜縣界,雲秋水出焉。東有郴水,發源黃岑山,流合桂陽縣之耒水,下流入於湘水。又西南有桂水,下流合於耒水。西南有石陂巡檢司。領縣五。北距布政司千八百八十里。

永興州北少西。東南有土富山,舊有銀井。西有高亭山。東有郴水,又有白豹水,自西南流入焉,謂之森口。西有高亭、北有安福二巡檢司。

宜章州南。西南有莽山。東有漏天山。北有章水,支流曰小章水,源俱出黃岑山,有武水自西來合焉,下流入江西崇義縣界。東有赤石、南有白沙二巡檢司。

興寧州東北。南有耒水,東南有資興水流合焉。東有州門巡檢司。東南有滁口巡檢司,後移於西南之黃家摐。

桂陽州西南。南有耒山,耒水所出,西北會於郴水。又東有孤山水,流入江西崇義縣,達於贛水。東有守禦廣安千戶所,洪武二十九年三月置,後廢。宣德八年六月復置。東有益將、西有鎮安、南有長樂山口、北有濠村四巡檢司。

桂東州東。西北有小桂山,桂水所出,南有漚江來合焉。又南有高分嶺巡檢司。

靖州元靖州路,直隸湖廣行省。太祖乙巳年七月為靖州軍民安撫司。元年降為州。三年升為府。九年四月復降為州,以州治永平縣省入,直隸布政司。南有侍郎山,與廣西融縣分界。東有渠水,下流合會同縣之郎江而入沅水。西有零溪巡檢司。領縣四。東北距布政司千八百五十里。

會同州東北。西有沅水,又西南有郎水,自貴州黎平府流入,又東有雄溪,一名洪江,下流俱入於沅水。南有若水巡檢司。

通道州南。洪武十年五月省入州。十三年五月復置。北有福湖山。西有渠水,西北有播揚河,自貴州黎平府流合焉。有播揚巡檢司。又西南有收溪寨巡檢司。

綏寧州東。元屬武岡路。洪武元年屬武岡府。三年來屬。東有雙溪,即城步縣巫水之下流也。東北有青坡巡檢司,後移於武陽。西南有臨口巡檢司。

天柱州西北。本天柱守禦千戶所,洪武二十五年五月置。萬曆二十五年改為縣,析綏寧、會同二縣地益之。崇禎十年東遷龍塘,名龍塘縣。後東遷雷寨縣。後還舊治,復故名。東有沅水。西北有屯鎮汶溪後千戶所,洪武二十三年置。東有鎮遠巡檢司,後移上新市,又有江東巡檢司。

施州衛軍民指揮使司元施州,屬四川行省夔州路。洪武初省。十四年五月復置,屬夔州府。六月兼置施州衛軍民指揮使司,屬四川都司。十二月屬湖廣都司。後州廢,存衛。北有都亭山。東有連珠山,五峰關在山下。又東南有東門山。東北有清江,自四川黔江縣流入,一名夷水,亦曰黔江,衛境諸水皆入焉,下流至宜都縣入於大江。領所一,宣撫司四,安撫司九,長官司十三,蠻夷官司五。東北距布政司千七百里。

大田軍民千戶所洪武二十三年閏四月以散毛宣撫司之大水田置。東有小關山。西南有萬頃湖,與酉陽界。又南有深溪關。北有硝場,產硝。東北距衛二百二十里。

施南宣撫司元施南道宣慰司。洪武四年十二月因之,後廢。十六年十一月復置,屬施州衛。二十七年後復廢。永樂二年五月改置長官司,屬大田軍民千戶所。四年三月升宣撫司,仍屬衛。東有舊治。後遷夾壁龍孔,即今治也。西有前江,發源七藥山,西南流與後江合,入四川彭水縣界。北距衛一百里。領安撫司五:

東鄉五路安撫司元東鄉五路軍民府。洪武四年十二月改置長官司,後升安撫司。領長官司三,蠻夷官司二。

搖把峒長官司元又把峒安撫司,後廢。宣德三年五月改置。

○上愛茶峒長官司

下愛茶峒長官司二長官司俱元容美洞地。至大二年置懷德府,屬四川南道宣慰司。至順二年正月升宣撫司。至正中,升軍民宣慰司。太祖甲辰年六月改軍民宣撫司,後廢。宣德三年五月改置。

鎮遠蠻夷官司宣德三年五月置。

隆奉蠻夷官司元隆奉宣撫司。洪武四年十二月改長官司,後廢。宣德三年五月改置官司。

忠路安撫司明玉珍忠路宣撫司。洪武四年改安撫司,二十三年廢。永樂五年復置,領長官司一。

劍南長官司宣德三年五月置。

忠孝安撫司元置。洪武四年十二月改置長官司,尋復故。二十三年廢。永樂五年復置。

金峒安撫司元置。洪武四年十二月改長官司。永樂五年復故。宣德三年五月領蠻夷官司一。隆慶五年正月降為峒長。

西坪蠻夷官司宣德三年五月置。

中峒安撫司嘉靖初置。

散毛宣撫司元至元三十年四月置散毛洞蠻夷官。三十一年五月升為府,屬四川行省。至正六年七月改散毛誓厓等處軍民宣慰司。明玉珍改散毛宣慰使司都元帥。洪武七年五月改散毛沿邊宣慰司,屬四川重慶衛。二十三年廢。永樂二年五月置散毛長官司,屬大田軍民千戶所。四年三月升宣撫司,屬施州衛。南有白水河,一名酉溪,自忠建宣撫司流入,又東南入永順司界。東北距衛二百五十里。領安撫司二:

龍潭安撫司元龍潭宣撫司。明玉珍改長官司。洪武八年十二月改龍潭安撫司,屬四川重慶衛。二十三年廢。永樂四年三月復置,來屬。南有清江。

大旺安撫司明玉珍大旺宣撫司。洪武八年十二月因之,屬四川。永樂五年改置,領蠻夷官司二。

東流蠻夷官司洪武八年十二月置東流安撫司,屬四川,後廢。宣德三年五月改置,來屬。

臈壁峒蠻夷官司宣德三年五月置。

忠建宣撫司元忠建軍民都元帥府。明玉珍因之。洪武五年正月改長官司。六年升宣撫司。二十七年四月改安撫司,尋廢。永樂四年復置宣撫司,屬施州衛。南有白水河,源出將軍山,西南流,車東河自容美司來合焉。北距衛二百五十里。領安撫司二:

忠峒安撫司元湖南鎮邊宣慰司。明玉珍改沿邊溪洞宣撫司。洪武五年正月改沿邊溪洞長官司,後廢。永樂四年改置。西南有酉溪。

高羅安撫司元高羅宣撫司。明玉珍改安撫司。洪武六年廢。永樂四年三月復置。領長官司一。

思南長官司成化後置。

容美宣撫司元容美等處宣撫司,屬四川行省。太祖丙午年二月因之。吳元年正月改黃沙、靖安、麻寮等處軍民宣撫司。洪武五年二月改置長官司。七年十一月升宣慰司,後廢。永樂四年復置宣撫司,屬施州衛。西南有山河,即漊水之上源,東入九溪衛界。西北距衛二百十里。領長官司五:

盤順長官司元元統二年正月置盤順府。至正十五年四月升軍民安撫司。洪武五年三月改為長官司。

椒山瑪瑙長官司、五峰石寶長官司、石梁下峒長官司、水盡源通塔平長官司四長官司,俱洪武七年十一月置,十四年廢。永樂五年復置。

木冊長官司元木冊安撫司。明玉珍改長官司。洪武四年廢。永樂四年三月復置,屬高羅安撫司。宣德九年六月直隸施州衛。

鎮南長官司元宣化鎮南五路軍民府,尋改湖南鎮邊毛嶺峒宣慰司。明玉珍改鎮南宣撫司。太祖丙午年二月因之,尋廢。洪武八年二月復置,屬施州衛。二十三年復廢。永樂五年改置,直隸施州衛。有酉溪。

唐崖長官司元唐崖軍民千戶所。明玉珍改安撫司。洪武七年四月改長官司,後廢。永樂四年三月復置,直隸施州衛。南有黔水,即清江之上源。

永順軍民宣慰使司元至元中,置永順路,後改永順保靖南渭安撫司。至大三年四月改永順等處軍民安撫司。至正十一年四月升宣撫司,屬四川行省。洪武二年為州。十二月置永順軍民安撫司。六年十二月升軍民宣慰使司,屬湖廣行省,尋改屬都司。西南有水溪,即酉水也,下流入沅陵縣界。領州三,長官司六。東北距布政司二千里。

南渭州司西。元屬新添葛蠻安撫司,後廢。洪武二年復置,改屬。

施溶州司東南。元會溪施溶等處長官司,屬思州軍民安撫司,後廢。洪武二年改置,來屬。

上溪州司西。洪武二年置

臈惹洞長官司、麥著黃洞長官司、驢遲洞長官司、施溶溪長官司四長官司,元俱屬思州軍民安撫司。洪武三年改屬。

白崖洞長官司元屬新添葛蠻安撫司。洪武三年改屬。

田家洞長官司洪武三年置。

保靖州軍民宣慰使司元保靖州,屬新添葛蠻安撫司。太祖丙午年二月置保靖州軍民安撫司。洪武元年九月改宣慰司。六年十二月升軍民宣慰使司,直隸湖廣行省,尋改屬都司。北有北河,自酉陽司流入,東入永順司界。又有峒河,下流與盧溪縣之武溪合。領長官司二。東北距布政司千九百七十里。

五寨長官司司南。元置,洪武七年六月因之。

筸子坪長官司司南。太祖甲辰年六月置竿子坪洞元帥府,後廢。永樂三年七月改置。

浙江《禹貢》揚州之域。元置江浙等處行中書省,治杭州路。又分置浙東道宣慰使司,治慶元路。屬焉。太祖戊戌年十二月置中書分省。治寧越府。癸卯年二月移治嚴州府。丙午年十二月罷分省,置浙江等處行中書省。治杭州府。洪武三年十二月置杭州都衛。與行中書省同治。八年十月改都衛為浙江都指揮使司。九年六月改行中書省為承宣布政使司。領府十一,屬州一,縣七十五。為里一萬零八百九十九。西至開化,與江南界。南至平陽,與福建界。北至太湖,與江南界。東至海。距南京九百里,京師三千二百里。洪武二十六年編戶二百一十三萬八千二百二十五,口一千四十八萬七千五百六十七。弘治四年,戶一百五十萬三千一百二十四,口五百三十萬五千八百四十三。萬曆六年,戶一百五十四萬二千四百八,口五百一十五萬三千五。

杭州府元杭州路,屬江浙行省。太祖丙午年十一月為府。領縣九:

錢塘倚。洪武三年四月建吳王府。十一年正月改封周王,遷河南開封府。南有鳳凰山,有秦望山。西南有靈隱山。南有錢塘江,亦曰浙江,有三源:曰新安江,出南直歙縣;曰信安江,出開化縣;曰東陽江,出東陽縣。匯而東為錢塘江,至會稽縣三江海口入海。西有西湖,源出武林泉。又北有運河,至秀水縣北,而接南直運河。又有安溪,即苕溪也,自餘杭縣流入,下流至烏程縣東北,注於太湖。

仁和倚。東北有皋亭山,有臨平山,下有臨平湖,後塞。北有北新關,成化中設戶部分司於此。又有塘棲鎮。

海寧府東少北。元海寧州。洪武二年降為縣。南濱海,有捍海塘。西南有赭山,與蕭山縣龕山相對,浙江經其中,東接大海,謂之海門。東南有石墩鎮巡檢司,本置縣東北硤石鎮,後遷於此,更名。西南有赭山鎮巡檢司,本置縣西陳橋北,尋遷赭山,更名,又遷文堂山上,仍故名。又西北有長安鎮。

富陽府西。東有觀山。西南有湖洑山。東南臨富春江,即錢塘江也。西南有東梓巡檢司,後廢。

餘杭府西北。西南有大滌山。西北有徑山。南有苕溪,源出於潛縣天目山。東北有石瀨巡檢司,後廢。

臨安府西。舊治在縣西西墅鎮。洪武初徙於今所,本吳越衣錦軍也。西有天目山,亦曰東天目,其在於潛境者為西天目。西北有南溪,即東苕溪也,源出天目山,經縣南,亦曰新溪。

於潛府西。北有天目山,浮溪出焉。縣南為紫溪,下流至桐盧縣入浙江。

新城府西南。西有葛溪,又東北有松溪合焉,至峴口入於浙江。

昌化府西。東南有柳相山。南有銅坑山。西北有千頃山。西有昱嶺,上有關。又西北有黃花嶺,上亦有關。東南有柳溪,東流合於於潛之紫溪。又有雙溪,自縣治南流入柳溪。西有手甗嶺巡檢司,遷縣西南株柳村,又遷縣西湛村,又遷楊家塘,仍故名。

嚴州府元建德路,屬江浙行省。太祖戊戌年三月為建安府,尋曰建德府。壬寅年二月改曰嚴州府。領縣六。東北距布政司二百七十里。

建德倚。北有烏龍山。西有銅官山。又新安江自淳安縣流入,經城南,東陽江自西南來合焉。又東北有胥溪,來入江,謂之胥口,亦曰建德江。東有管界巡檢司。

桐廬府東北。西有富春山,一名嚴陵山。桐江在南,即浙江也,亦曰睦江。自建德縣流入,經富春山之釣臺下,曰七里瀨,又東經桐君山下,曰桐江。有桐溪自縣東北流入焉,謂之桐江口,其上源即分水縣之天目溪也。有桐江巡檢司,後遷桐君山,又遷窄溪埠。

淳安府西。南有雲濛山。西有都督山,又有威平洞,亦曰青溪洞,又名幫源洞。南有新安江,自南直歙縣流入,亦曰青溪。西有街口、又有永平、南有港口三巡檢司,後廢。東有錦溪關,嘉靖中置。

遂安府西少南。西有武強溪,有雙溪流合焉,曰三渡口,經城南,東北注於淳安之青溪。南有鳳林巡檢司,後廢。

壽昌府西南。東南有巖峒山。西有壽昌溪,東北流至建德縣,入新安江。南有常樂溪,東南流至蘭溪縣,入東陽江。西有社田、西南有上梅二巡檢司。

分水府東北。東有天目溪,上源即於潛縣之紫溪及昌化縣柳溪也,下流為桐廬縣之桐溪。又東南有前溪,自淳安縣流入,東流於天目溪。東有吳村巡檢司,後廢。

嘉興府元嘉興路,屬江浙行省。太祖丙午年十一月為府,直隸京師。十四年十一月改隸浙江。領縣七。西南距布政司百七十五里。

嘉興倚。南有南湖,亦曰鴛鴦湖,流合運河。又有長水塘,西南接海寧,東北接海鹽縣界。又東有雙溪,東出為華亭塘,南直松江府之漕舟,由此入運河。

秀水倚。宣德五年三月析嘉興縣地置。西有運河,北經聞家湖,達南直吳江縣之運河。東北有杉青閘、又有王江涇二巡檢司。

嘉善府東。本嘉興縣魏塘鎮巡檢司,宣德五年三月改為縣。南有華亭塘河,東有魏塘河,東北有清風涇,皆流合焉。西北有分湖,與南直吳江縣分界。又北有章練塘水,亦流合華亭塘河,達華亭縣之泖湖。東北有風涇、西北有陶莊二巡檢司,廢。

崇德府西南。元崇德州。洪武二年降為縣。西北有運河,自德清縣流入。東南有語溪,一名語兒中涇,又名沙渚塘。又東北有石門塘水,東南接運河,北達歸安之烏鎮。

桐鄉府西少南。宣德五年三月以崇德縣之鳳鳴鄉置。北有運河,與崇德縣接界。又有爛溪,北達吳江縣之鶯脰湖,西達湖州府潯溪。北有皁林鎮巡檢司。

平湖府東。宣德五年三月以海鹽縣之當塗鎮置。東南有故邑山。南有雅山,俗曰瓦山。又當湖在縣治東,下流出海鹽澉浦口入海。其西為市西河,自嘉興縣流入,入於當湖。其分流南出者,則由縣東南乍浦入海;北出者,則由縣東北蘆瀝浦入海。浦傍有蘆瀝鹽場。又北有東泖,即華亭三泖之上流。東有白沙灣巡檢司,治廣陳墅,後遷縣東南獨山。又東南有乍浦鎮巡檢司,後遷梁莊,仍故名。

海鹽府東南。元海鹽州。洪武二年降為縣。南有秦駐山,又有長墻山。西南有鳳凰山。東北有湯山,又有獨山,舊置鹽場於此。東臨海,有防海塘,洪武初,以石為之,南北計四千八百丈。又有東、西、南三海口,而西海口在縣東北,尤衝要。東北有呂港,港口有鹽場。西南有鮑郎市,有鹽課司。東北有守禦乍浦千戶所,東南有澉浦守御千戶所,俱洪武十九年十月置。城東有海口巡檢司,後徙砂腰村,南有澉浦巡檢司,後遷秦駐山,俱仍故名。

湖州府元湖州路,屬江浙行省。太祖丙午年十一月為府,直隸京師。十四年十一月改隸浙江。領州一,縣六。南距布政司百九十里。

烏程倚。北有卞山,亦曰弁山。西南有石城山。南有峴山,本名顯山。西南有銅山,一名銅峴山。北有太湖,接南直蘇、常二府界。東北有大錢湖、小梅湖二口,府境群水皆於此入太湖。又西有苕溪,源自孝豐天目之陰,流經毘山下,出大錢湖口。又南有餘不溪,即杭州境內之苕溪,自德清縣流經府南,匯為玉湖,復東北出而匯於苕水,亦曰霅溪。東有後潘村巡檢司,後遷南潯鎮,仍故名。東北有大錢湖口巡檢司。

歸安倚。南有金蓋山,亦名何山。又有衡山。東有昇山,亦曰烏山,一名歐餘山。又運河在城東,源自苕溪、餘不溪二水,分流為運河,東北經南潯鎮,入吳江縣界,合嘉興之運河。又南有荻塘,亦曰荻港,東北接運河。其枝流東南出烏鎮,合桐鄉之爛溪。又東有潯溪,即餘不溪支流也,流經南潯。東南有璉市巡檢司。又西南有上沃埠巡檢司,後廢。

長興府西北。元長興州。太祖丁酉年三月改名長官州,壬寅年復曰長興。洪武二年降為縣。西北有顧渚山,產茶,一名西顧山,一名吳望山。東北有太湖,與南直宜興縣分中流為界。西有箬溪,下流入太湖。西南有荊溪,東南入於苕溪。東北有皋塘、西南有四安二巡檢司。又西有合溪、南有和平二巡檢司,廢。

德清府南少東。東北有敢三山。東南有運河,有餘不溪,亦曰霅溪,即苕溪別名。東北有新市鎮巡檢司。又東有下塘巡檢司,後遷五柳港口。又東有荷葉浦巡檢司,廢。

武康府西南。東有封山,一名防風山。又有禺山。西南有覆舟山。南有前溪,東北有後溪流入焉,下流入德清餘不溪。

安吉州元安吉縣。正德元年十一月升為州。西南有故城。洪武徙於今治。東南有白陽山,舊產錫。西有苕溪。又有龍溪,即苕溪支流。東南有獨松關巡檢司,又有遞鋪巡檢司,廢。東北距府二十里。領縣一:

孝豐州西南。成化二十三年析安吉縣地置,屬府。正德二年改屬州。南有天目山,有天目山巡檢司。又西南為金石山,即天目最高處。又南有苕溪,出天目山,此為苕溪之別源。又西有松坑巡檢司。

紹興府元紹興路,屬浙東道宣慰司。太祖丙午年十二月為府。領縣八。西北距布政司百三十八里。

山陰倚。南有會稽山,其支山為雲門山,又有法華山。西南為蘭亭山。西北有塗山。北濱海,有三江口。三江者,一曰浙江;一曰錢清江,即浦陽江下流,其上源自浦江縣流入,至縣西錢清鎮,曰錢清江;一曰曹娥江,即剡溪下流,其上源自嵊縣流入,東折而北,經府東曹娥廟,為曹娥江,又西折而北,會錢清江、浙江而入海。又西有運河,自蕭山縣流入,又東南逕會稽縣,又東入上虞縣界。又南有鑒湖,長十四五里,俗曰白塔洋,有若耶溪合焉。又北有白水湖,旁通運河。北有三江守御千戶所,在浮山之陽,洪武二十年二月置。又有三江巡檢司,在浮山桃松莊。又西北有白洋巡檢司。

會稽倚。東南為會稽山,其東接宛委、秦望、天柱諸山。又東有銀山、錫山,舊產銀砂及錫。東南有若耶山。東有曹娥江。東南有平水溪,南合剡溪。東北有瀝海守御千戶所,洪武二十年二月置。又有黃家堰巡檢司,尋遷瀝海所西,後遷上虞縣界纂風鎮,仍故名。

蕭山府西北。西南有虎爪山,東南有龕山,俱下臨浙江。龕山傍有小山曰鱉子山,浙江自縣西東北流,出其中,東接大海,亦曰海門。東南有峽山,錢清江經其中,復北折而東,入山陰縣界。城西有運河,東接錢清江。又有湘湖。西南有漁浦巡檢司。又西有西興,亦曰西陵,往錢塘者由此渡江。

諸暨府西南。元諸暨州。大祖己亥年正月改諸全州。丙午年十二月降為諸暨縣。西南有新城,在五指山下,太祖癸卯年,李文忠所築。西有長山,又有五泄山。南有句乘山。又有浣江,即浦陽江,亦曰青弋江。又西南有長清關、西有陽塘關二巡檢司,廢。

餘姚府東北。元餘姚州。洪武初,降為縣。南有新城,與縣城隔江對峙,姚江經其中。南有四明山,北瀕海。姚江源自縣西南太平山,一名舜江,西北流至上虞縣,乃東北出,經縣南。又東為慈溪之前江。東北有燭溪湖,引流為東橫河。西有牟山湖,引流為西橫河,俱注於姚江。又西北有臨山衛,洪武二十年二月置。東北有三山守御千戶所,一名滸山,亦洪武二十年二月置。又東北有三山巡檢司,治金家山上,尋遷破山。北有眉山巡檢司,治眉山寨,尋遷縣西北湖海頭。又有廟山巡檢司,治廟山寨,尋遷上虞縣界中源堰,仍故名。

上虞府東。西北有夏蓋山,北枕海,南臨夏蓋湖。西南有東山。東有覆卮山,接嵊縣界。又東有通明江,即姚江上流。又有運河,在縣治前。又西北有白馬湖,北接夏蓋湖,其相連者有上妃湖,亦曰上陂湖,引流為五夫湖,東北達於餘姚之西橫河。又西有梁湖巡檢司,本治梁湖,尋遷百官市,仍故名。

嵊府東南。東有丹池山。東北有嵊山。北有雩山,又有清風嶺。西有太白山。南有剡溪,源出天臺諸山,下流為曹娥江。西有長樂鎮、西北有管解寨二巡檢司,廢。

新昌府東南。東有沃州山。東南有天姥山。又東有東溪,源出天台山,西北流入嵊縣界。南有彩霞鎮、又有豐樂、又有善政三巡檢司,後廢。

寧波府元慶元路,屬浙東道宣慰司。太祖吳元年十二月為明州府。洪武十四年二月改寧波。領縣五。西北距布政司三百六十里。

鄞倚。東有鄮山。西南有四明山,周八百餘里。東有灌頂山,舊產鐵。東南有阿育王山,有太白、天童諸山。東北濱海。有鄞江,一名甬江。東南有奉化江,西北有慈溪,皆流合焉。西南有小江湖,又西有廣德湖,東有東錢湖,皆引流入鄞江。北有龍山守禦千戶所,洪武十九年十一月置。東有甬東巡檢司,治甬東隅,後遷定海縣東南竹山海口,仍故名。又有岱山、又有螺峰二巡檢司,後廢。

慈谿府西北。元曰慈溪。永樂十六年改「溪」為「谿」。西南有車廄山。東北濱海。南有慈溪,一曰前江,即姚江下流也,藍溪、文溪諸水皆流合焉。西北有鳴鶴鹽課司。又觀海衛亦在西北,洪武十九年一月置。又有松浦巡檢司,治浦東,尋遷浦西。又有向頭巡檢司,治向頭寨,尋遷洋浦,廢,後復。

奉化府南。元奉化州。洪武二年降為縣。南有蓬島山,又有天門山。西北有雪竇山。北有奉化江,亦曰北渡江,又謂之剡溪。東有市河,東北有趙河,皆南流入焉。東有塔山、東南有鮚琦二巡檢司。又有公棠、連山、柵虛、東宿四巡檢司,廢。

定海府東北。東有候濤山,一名招寶山,上有威遠城,山麓有靖海城,俱嘉靖三十九年置。東北皆濱海。海中有舟山,有金塘山,有蛟門山,又有普陀落伽山,有大謝、小謝山。南有大浹江,其上流即鄞江,分流為小浹江,並入海。南有清泉等鹽場。又東北有定海衛,本定海守御千戶所,洪武十四年四月置,二十年二月升為衛。東南有穿山後千戶所,洪武二十七年九月置。又有霩衢守御千戶所,大嵩守御千戶所,俱洪武十九年十一月置。又有舟山中中千戶所,舟山中左千戶所,本元昌國州,洪武二年降為縣,二十年六月,縣廢。改置。南有上岸太平嶴、西有管界寨二巡檢司。又西北有施公山、南有長山二巡檢司,後廢。又南有霞嶼巡檢司,本名崎頭,正統間更名,後廢。又舟山東南有寶陀、西北有岑港,又舟山東有岱山、西南有螺峰四巡檢司,後廢。

象山府東南。南有石壇山,亦曰壇頭山。東南北三面皆濱海。其南有三萼山,一名三仙島,俱在海中。南有玉泉鹽場。又西南有昌國衛,本昌國守禦千戶所,洪武十二年十月置於舟山,十七年九月改為衛。二十年來徙縣南天門山,二十七年遷縣西南後門山。又山西南有石浦守禦前、後二千戶所,俱洪武二十年置。西北有錢倉守禦千戶所,洪武十九年十一月置。西有爵溪守御千戶所,洪武三十年十二月置。北有陳山巡檢司,治陳山,尋遷縣東南。西有爵溪巡檢司,遷治姜嶼渡。南有石浦巡檢司,遷治青山頭。又東有趙嶴巡檢司,自寧海縣遷此。俱仍故名。

台州府元台州路,屬浙東道宣慰司。洪武初,為府。領縣六。西北距布政司四百四十里。

臨海倚。西南有括蒼山,一名真隱山。又東南有海門山,有金鰲山,皆濱海。南有澄江,一名靈江,流合天臺、仙居諸山之水,至黃巖縣入海。又大海在東,中有芙蓉山、高麗頭山。又有杜瀆鹽場。又海門衛亦在縣江,洪武二十年二月置。其北為前千戶所,洪武二十八年置。東北有桃渚前千戶所,洪武二十年九月置。東有蛟湖巡檢司,遷治海口陶嶼。又有連盤巡檢司,遷治海口長沙。俱仍故名。

黃巖府東南。元黃巖州。洪武三年三月降為縣。南有委羽山。東有大海。西北有永寧江,即澄江下流。東南有鹽場,又有長浦巡檢司。

天台府西南。西有天臺山。北有赤城山,又有石橋山,皆天臺支阜也,其絕頂曰華頂峰。又西南有始豐溪,即澄江上源。又東有楢溪,產鐵。其東為甬溪。又西有胡竇巡檢司,廢。

仙居府西南。西北有蒼嶺,即括蒼山。又有永安溪,下流亦會於澄江。又西南有曹溪,東有彭溪,俱流合於永安溪。西有田寺巡檢司,後廢。

寧海府東北。北有天門山。西北有龍須山,舊產銅鐵。東濱海。東北有鄞江,與象山縣界。南有海游溪,有寧和溪,又有東溪,東有鐵砂,冶之成鐵,俱導流入海。又有梅嶴鎮,舊有鐵場。又南有健跳千戶所,洪武二十年九月置。東有越溪、又有長亭、北有鐵場、南有曼嶴、東南有竇嶴五巡檢司。

太平府東南。成化五年十二月以黃巖縣之太平鄉置,析樂清地益之。南有大雷山。西北有王城山。西南有靈山,與玉環山接。東南濱海,曰大閭洋,中有松門、石塘、大陳等山。又東有遷江,一名新建河,至縣北曰官塘河,北抵黃巖縣,東入海。東有松門衛,本松門千戶所,洪武十九年十二月置,二十年六月升為衛。東北有新河千戶所,洪武十九年十二月置。南有隘頑千戶所,西南有楚門千戶所,俱洪武二十年二月置。又東有盤馬、西有二山、又有蒲岐三巡檢司。南有沙角巡檢司,本治岐頭山下,後遷今治。西南有小鹿巡檢司,遷治楚門所之橫山後。西有溫嶺巡檢司,廢。

金華府元婺州路,屬浙東宣慰司。太祖戊戌年十二月為寧越府。庚子年正月曰金華府。領縣八。東北距布政司四百五十里。

金華倚。北有金華山。南有銅山,舊產銅。城南有東陽江,亦曰婺港,自東陽縣流經此。又有南溪,自縉雲縣來合焉,謂之雙溪,亦曰縠溪,合流至蘭谿而會於信安江。

蘭谿府西。元蘭谿州。洪武三年三月降為縣。東有銅山,舊產銅。西南有蘭溪,即彀溪也,亦曰大溪,一自衢州之衢港,一自金華之婺港,會於西南蘭陰山下,北入嚴州界。西北有平渡巡檢司。北有靈泉鄉、龍巖鄉二巡檢司,廢。

東陽府東。東南有大盆山,東陽江出焉,經縣北,謂之北溪,亦曰東溪,西南有畫溪,下流至義烏縣入焉。東有永寧巡檢司。又東南有瑞山、玉山。南有興賢、仁壽二巡檢司,廢。

義烏府東少北。南有烏傷溪,即東陽江。西有智者同義鄉、南有雙林明義鄉、北有龍祈鎮三巡檢司,廢。

永康府東南。東南有銅山,舊產銅。南有南溪,亦曰永康溪。又東有孝義寨、南有義豐鄉、東南有合德鄉三巡檢司,後廢。

武義府南少東。東北有永康溪,又有茭道市。西有苦竹市。又北有白溪口市。

蒲江府東北。西有深裊山,蒲陽江出焉,東流入諸暨縣界。東有楊家埠巡檢司,後廢。

湯溪府西南。成化七年正月析蘭溪、金華、龍游、遂昌四縣地置。南有銀嶺。西北有縠江,即信安江。

衢州府元衢州路,屬浙東道宣慰司。太祖己亥年九月為龍游府。丙午年為衢州府。領縣五。東北距布政司五百六十里。

西安倚。永樂二十二年建越王府,宣德二年除。西有巖山。南有爛柯山,又有爵豆山,舊出銀。又西北有銅山,舊出銅、錫、鉛。城西南有衢江,其上源曰大溪,自江山縣流入。又有西溪,亦曰信安溪,自開化縣發源,流至此與大溪合焉,曰雙港口。又東有定陽溪,一名東溪,自遂昌縣流入,合於衢江。西南有嚴剝、東南有板固二巡檢司。

龍游府東。東有龍丘山。北有梅嶺。又有縠溪,即衢江也,一名盈川溪,又南有靈溪,自遂昌縣流經縣南靈山下,又東北入焉。東有湖頭鎮巡檢司。又北有水北、南有靈山二巡檢司,廢。

常山府西。有三衢山。東有常山,即信安嶺也。北有金川,一名馬金溪,自開化縣流入。東有文溪,自江山縣流入,合於金川,為信安溪上源。北有下坑、東南有鎮平二巡檢司,廢。

江山府西。東南有江郎山,有仙霞嶺,仙霞關在其上。城東有大溪,仙霞嶺水所匯也。又西有文溪。南有東山巡檢司,本治仙霞嶺下,後遷嶺上。又有小竿嶺巡檢司,廢。

開化府西北。金溪在城東,其源一出馬金嶺,一出百際嶺,至城北合流而南,即金川上源也。北有金竹嶺巡檢司。又西有雲臺、北有低阪、又有馬金、南有華埠四巡檢司,廢。

處州府元處州路,屬浙東道宣慰司。太祖己亥年十一月為安南府,尋曰處州府。領縣十。北距布政司七百三十里。

麗水倚。大溪在城南,一名洄溪,自龍泉縣流經此,下流至永嘉縣,入於海。又東有好溪,本名惡溪,東南達於大溪。

青田府東南。西有大、小連雲山。南有南田山。又有南溪,即大溪也,亦曰青溪,自麗水縣流入。西南有小溪流合焉。南有淡洋巡檢司,又北有黃壇巡檢司,廢。

縉雲府北。東有仙都山,亦名縉雲山。又有管溪官山。西南有馮公嶺,一名木合嶺,一名桃花隘。又東有好溪,源出縣東北之大盆山,有管溪自東流合焉。又北有南源溪,亦曰南溪,下流為永康溪,入於東陽江。

松陽府西。北有竹客嶺。西有松溪,南有竹溪流入焉,下流至麗水縣,入於大溪。又西南有凈居巡檢司,廢。

遂昌府西。南有雙溪,有二源,至縣南合流。又東經西明山南,分為二,其一入龍泉縣之大溪,其一為東溪,入松陽縣,為松溪。北有馬步巡檢司。

龍泉府西南。南有匡山,建溪之水出焉。南有大溪,源出臺湖山,又有靈溪,自縣北流合焉,東入雲和縣界。南有慶元巡檢司,治查田市。

慶元府西南。洪武三年三月省。十三年十一月復置。西南有松源水,南流入福建,為松溪縣之松溪。

雲和府西南。景泰二年析麗水縣地置。南有大溪,西有黃溪流入焉,東入麗水縣界。又西有七赤渡。東有石塘隘。

宣平府北。本麗水縣之鮑村巡檢司。景泰三年改為縣,而徙巡檢司於縣之後陶,仍故名,尋廢。西北有礱坑山,舊產銀。南有玉岩山,又有會高山,產礦。又南有虎蹐溪,會流於麗水縣之大溪。

景寧府南。景泰五年析青田縣置。南有敕木山。東有礦坑嶺。西有彪溪,東北有大匯灘,下流皆注於青田縣之大溪。北有沐溪巡檢司,遷縣南大漈仍故名。又西有盧山巡檢司,後廢。東有龍首關,又有龍匯關、白鹿關,俱嘉靖中置。

溫州府元溫州路,屬浙東道宣慰司。洪武初,為府。領縣五。西北距布政司八百九十里。

永嘉倚。西有岷岡山,又有鐵場嶺。南有大羅山。東濱海。又永寧江在城北,一名甌江,一名永嘉江,自蒼括諸溪匯流入府界,又東注於海。江中有孤嶼山,與北岸羅浮相望。又西北有安溪,東北有楠溪,俱注於甌江。城西南又有會昌湖,東有寧村守禦千戶所,洪武二十年二月置。東南有中界山巡檢司,後遷縣東永昌堡。

瑞安府南。元瑞安州。洪武二年降為縣。正德六年五月徙縣城於故城西,去海三丈五尺,以避潮患。西有陶山。北有帆游山。城南有安陽江,源出福建政和縣及青田縣界,合流至此,曰瑞安江,亦曰飛雲江,渡處有飛雲關,東接海口。又縣東海岸中有鳳凰諸山。又縣東北有海安守禦千戶所,縣東南有沙園守禦千戶所,俱洪武二十年二月置。東有東山巡檢司,本名梅頭,治梅頭寨,後遷,更名。

樂清府東北。東有北雁蕩山。南濱海,有玉環山,在海中。又西北有荊溪。又縣治傍有東、西二溪。西南有館頭江。西有象浦河,東北有石馬港,下流皆達海。有長林鹽場。又西有盤石衛,洪武二十年二月置。東有盤石守御後千戶所,成化五年置。東北有蒲岐守御千戶所,亦洪武二十年二月置。西有館頭巡檢司,遷治縣西南岐頭寨。後復。東南有北監巡檢司,治玉環山下,尋遷縣東北蔡嶴,又遷縣東白沙嶺,又遷鶚頭,又遷窯嶴山下,仍故名。

平陽府西南。元平陽州。洪武三年降為縣。西南有南雁蕩山,有玉蒼山。又東南海中有大巖頭山,有南麂山。又西有前倉江,亦曰橫陽江,東南經江口關注於海。南有天富南鹽場。又南有金鄉衛,有蒲門守御千戶所,東北有壯士守御千戶所,皆洪武二十年二月置。東南有舥艚、又有斗門二巡檢司。南有江口巡檢司,治下埠,後遷渡頭。又東有仙口巡檢司,遷縣南麥城山,仍故名。又東南有龜峰巡檢司,廢。

泰順府西南。景泰三年以瑞安縣羅洋鎮置,析平陽縣地益之。南有分水山,上有關,為浙、閩分界處。又西有白溪,下流至福建寧德縣入海。又東有仙居溪,流入瑞安境入海。北有池村巡檢司。南有三冠巡檢司,本洋望,後更名。東南有鴉陽巡檢司,後廢。又羅陽第一關在縣東。

○福建廣東廣西

福建《禹貢》揚州之域。元置福建道宣慰使司,治福州路。屬江浙行中書省。至正十六年正月改宣慰司為行中書省。太祖吳元年十二月平陳友定。洪武二年五月仍置福建等處行中書省。七年二月置福州都衛。與行中書省同治。八年十月改福州都衛為福建都指揮使司。九年六月改行中書省為承宣布政使司。領府八,直隸州一,屬縣五十七。為里三千七百九十七。北至嶺,與浙江界。西至汀州,與江西界。南至詔安,與廣東界。東至海。距南京二千八百七十二里,京師六千一百三十三里。洪武二十六年編戶八十一萬五千五百二十七,口三百九十一萬六千八百六。弘治四年,戶五十萬六千三十九,口二百一十萬六千六十。萬曆六年,戶五十一萬五千三百七,口一百七十三萬八千七百九十三。

福州府元福州路,屬福建道。太祖吳元年為府。領縣九:

閩倚。南有釣臺山,亦曰南臺山。東南有鼓山。南有方山,一名甘果山,下有官母嶼,有巡檢司。東南濱海。南有閩江,亦曰建江,自南平縣流入府界。東南納群川之水,至府西曰洪塘江,分二流,南出曰陶江,東出曰南臺江,至鼓山下復合為一。又東南有馬頭江,自永福縣流入,曰西峽江,又東有東峽江流合焉,又東南至五虎門,入於海。東有閩安鎮巡檢司。

侯官倚。西有旂山,有雪峰山,有建江,又有西禪浦。西南有陽崎、吳山、鳳岡、澤苗、延澤、仙阪等六浦,皆建江支分,仍合正流入海。西北有懷安縣,洪武十二年移入郭內,與閩、侯官同治,萬歷八年九月省。西北有竹崎、又有五縣寨二巡檢司。

長樂府東少南。東濱海,有海堤。北有馬頭江。又東有守禦梅花千戶所,洪武二十一年二月置。東北有石梁蕉山、東南有松下鎮二巡檢司。又東有小祉山巡檢司,後移治大祉澳。

福清府南少東。元福清州。洪武二年二月降為縣。東南際海,有鹽場,海中有海壇山,又有小練山。南有龍江,又有逕江。東南有海口,江皆匯流入海。又東有鎮東衛,東南有守御萬安千戶所,俱洪武二十一年二月置。又有澤郎山、有牛頭門、又南有壁頭山三巡檢司。又東有海口鎮巡檢司,洪武二十年移於長樂縣之松下鎮。

連江府東北。東北濱海,海中有北茭鎮巡檢司。南有連江,東入海。東北有守禦定海千戶所,洪武二十一年二月置。

羅源府東北。東濱海。西有羅川,南流分三派入海。南有應德鎮。

古田府西北。建江在縣南,自南平縣流入,經城南,有大溪流合焉,謂之水口。又東南逕模天嶺下,江流至此始出險就平,東入閩清縣界。東有杉洋鎮,出銀坑,有巡檢司,後廢。又西南有谷口鎮、西北有西溪鎮二巡檢司,尋廢。

閩清府西北。西南有大帽山。北有建江,西南有梅溪流合焉。東有青窯鎮巡檢司,廢。

永福府西南。西南有高蓋山,又南有陳山。東有東溪,匯諸山溪之水,下流會於福清之龍江而入海。又有漈門巡檢司,後移於嵩口埕,尋復故。

興化府元興化路,屬福建道宣慰司。洪武元年為府。領縣二。北距布政司二百八十里。

莆田倚。東南濱海,海中有湄洲嶼,又有南日山,俱東與琉球國相望。又南有木蘭溪,北有延壽溪,東北有荻蘆溪,又有通應港,俱會流入海。又西北有興化縣,正統十三年四月省。東有平海衛,東南有守禦莆禧千戶所,俱洪武二十一年二月置。東有嵌頭、西北有大洋寨、東南有吉了三巡檢。東有沖沁巡檢司,本治尋陽,後徙興福。又有青山巡檢司,本治武盛里南哨,後徙奉國里。東南有南日山巡檢司,後徙新安。東北有迎仙寨巡檢司,後移鼓樓山。東有峙頭、東南有小峙二巡檢司,後廢。

仙游府西。北有二飛山。東北有何嶺。南臨九鯉湖,湖在萬山中,下流入莆田縣界,合於延壽溪。西有三會溪,即木蘭溪上源。西有白嶺巡檢司,後遷於文殊寨。南有楓亭市、西有潭邊市二巡檢司,後廢。

建寧府元建寧路,屬福建道宣慰司。洪武元年為府。領縣八。四年正月置建寧都衛於此。八年十月改為福建行都指揮使司。東南距布政司五百二十五里。

建安倚。東北有鳳凰山,產茶。東有東溪,即建江,自浙江慶元縣流經此,又西合於西溪。又東南有壽嶺巡檢司。

甌寧倚。西有西溪,源出崇安縣,東會諸溪之水,流入縣境,又東合於東溪,南入延平府界。西北有營頭街巡檢司。

建陽府西北。西北有西山。東南有錦江,亦曰交溪,有二源,合流於縣東東山下,南流達於建溪。

崇安府西北。南有武夷山,中有清溪,九曲流入崇溪。西北有分水嶺,上有分水關巡檢司。其水西流者入江西境,東流者入縣境,即崇溪源。俗謂之大溪,經城西而南出,亦謂之西溪。其別源出縣東北之岑陽山,亦曰東溪,西南流合於西溪,又南合武夷水而入建陽縣界,即錦江之上源也。又西北有溫林、岑陽、桐木、焦嶺、谷口、寮竹、觀音等關,與分水關為崇安入關。

浦城府東北。北有漁梁山,建溪之源出焉。又有蓋仙山,有黎嶺,又有楓嶺,一名大竿嶺,皆浙、閩通途。又東北有柘嶺,與浙江麗水縣分界,柘水出焉,流合大溪。又南有南浦溪,亦曰大溪,即建溪也,下合建陽之交溪。東有高泉、東北有溪源、西北有盆亭三巡檢司。

松溪府東。東有萬山。東北有鷲峰山,接浦城及浙江之龍泉界。南有松溪,源出浙江慶元縣,亦謂之松源水,又西有杉溪,下流俱入於建溪。北有二十四都巡檢司。南有東關巡檢司,後遷於烏鞍嶺,又遷於鐵嶺,又遷於峽橋。

政和府東。南有七星溪,源出縣東之銅盤山,下流合於松溪。又東有丹溪,流經福安縣入海。又東南有赤巖巡檢司。

壽寧府東。景泰六年八月以政和縣楊海村置,析福安縣地益之。東有蟾溪,即福寧州長溪上源也。東有漁溪巡檢司,後遷縣北之官臺山,又遷斜灘鎮。

延平府元延平路,屬福建道宣慰司。洪武元年為府。領縣七。東南距布政司四百五里。

南平倚。南有九峰山。東北有衍仙山。城東南有劍溪,即建江也,亦曰東溪,自建寧府流入,南經黯淡灘,又西逕劍津,與西溪合。西溪出汀、邵二府之境,至縣西,合於沙縣之沙溪,為沙溪口;又東至劍津,合於東溪;又南至尤溪口,合於大溪,亦名南溪;又東至福州府,入於海;俗亦謂之三溪。東南有蒼峽、西北有大曆二巡檢司。

將樂府西。南有天階山。西北有百丈山。南有將溪,亦曰大溪,即西溪之上源也。又西北有梅溪,自邵武界流入,合於大溪。又北有萬安寨巡檢司。

沙府西南。西北有幼山。縣治南有沙溪,亦名太史溪,自永安縣流入,經縣東,有霹靂等灘,下流合於西溪。北有北鄉寨巡檢司。

尤溪府南。北有丹溪嶺,一名桃木嶺,下有丹溪。東有尤溪,其上源一出龍巖縣,一出德化縣,合流於縣西南,又北流會湯泉等二十溪,北出尤口,入建溪,亦曰湖頭溪。西有英果砦、又有高才阪二巡檢司。

順昌府西少北。南有徘徊嶺。西北有順陽溪,源出建陽縣,又東經縣南,與將溪合,又東經沙口,合邵武縣之沙溪,又東經縣西,與西溪合,西溪即邵武縣之紫云溪也,又東入南平縣界,為南平之西溪。又西北有仁壽鎮巡檢司。

永安府西南。本沙縣之浮流巡檢司,正統十四年置永安千戶所於此。景泰三年改置縣,析尤溪縣地益之。東北有貢川山。東南有石羅山。西有燕溪,四源合流,經城東北,下流為沙縣之沙溪。又西有安砂鎮、西南有湖口寨二巡檢司。又西北有黃楊巡檢司,廢。

大田府西南。嘉靖十五年二月以尤溪縣之大田置,析永安、漳平、德化三縣地益之。北有五臺山。南有大仙山。東有銀瓶山,產銀鐵。又南有尤溪,自龍岩縣流入,又東入尤溪縣境。又東南有花橋巡檢司。又西南有桃源店巡檢司,本屬漳平縣,後來屬。北有英寨、西南有安仁隘二巡檢司,後廢。

汀州府元汀州路,屬福建道宣慰司。洪武元年為府。領縣八。東距布政司九百七十五里。

長汀倚。北有臥龍山。又北有新樂山,貢水出焉,流入江西界。西有新路嶺。東有鄞江,即東溪,亦曰左溪,自寧化縣流入,下流經廣東大埔縣入海,中有五百灘,亦謂之汀水。又東南有正溪,西有西溪,北有北溪,南有南溪,俱合於東溪。又西有古城寨巡檢司。

寧化府東北。南有潭飛漈。又有大溪,源出縣北萬斛泉,分流為清流縣之清溪,其正流入長汀縣,為鄞江上流。北有安遠寨巡檢司。

上杭府南。西有金山,上有膽泉,浸鐵能成銅。西南有羊廚山,產礦。南有大溪。

武平府西南。北有黃公嶺。南有化龍溪,下流入廣東程鄉縣。西南有武平城,洪武二十四年正月置武平千戶所於此。東南有象洞巡檢司,後移於縣西南之懸繩隘。北有永平寨巡檢司,後移縣西北之貝寨。

清流府東北。南有豐山,東南有鐵石山,南臨九龍溪,有鐵石磯頭巡檢司。西南有清溪,自寧化縣流入,東北合半溪,又東南經九龍灘而入永安縣界,亦曰龍溪,即燕溪之上源。

連城府東南。本曰蓮城,洪武十七年後改「蓮」曰「連」。東有蓮峰山。南有文溪,下流達於清流縣之清溪。西南有北園寨巡檢司,後遷於縣南之朗村隘,後又遷於縣西南之新泉隘。

歸化府東北。成化七年正月以清流縣之明溪鎮置,析將樂、沙縣、寧化三縣地益之。北有鐵嶺。南有歸化溪,下流合將樂縣之將溪。東有夏陽巡檢司。

永定府南。成化十四年以上杭縣溪南里之田心地置,析勝運等四里益之。西有大溪,即汀水,自上杭縣流經此,又東入廣東大埔縣界。東南有三層嶺巡檢司。東北有太平巡檢司,後徙高坡。西南有興化巡檢司,治溪南里古鎮,尋廢,復置,後遷於上杭縣之峰頭。

邵武府元邵武路,屬福建道宣慰司。太祖吳元年為府。領縣四。東南距布政司六百七十里。

邵武倚。東有三臺山。東南有七臺山,又有道人峰。又有樵溪,源自樵嵐山,經城內,出北門,合紫雲溪,流至順昌縣為順陽溪。又東南有水口巡檢司。又東有拿口、南有同巡、東北有楊坊三巡檢司,廢。

光澤府西北。北有雲際嶺。西北有杉嶺,杉關在其上,與江西南城縣接界。杭川出焉,亦名大溪,下流入紫雲溪。又有大寺寨巡檢司,在杉關東。又西北有黃土關。

泰寧府西南。西有金饒山。西北有大杉嶺。西有二十四溪,南有灘江流合焉,下流會於樵溪。

建寧府西南。北有百丈嶺,藍溪出焉。南有綏江,源出金饒山,一名濉江,亦名寧溪,至綏城口,合藍溪流入泰寧縣界。西有西安巡檢司,本治里心保,後遷丘坊隘,尋廢,後復置,後又遷新安保之黃泥鋪。

泉州府元泉州路,屬福建道宣慰司。洪武元年為府。領縣七。東距布政司四百十里。

晉江倚。東北有泉山,一名清源。東南有寶蓋山。南有靈源山。東南濱海,有鹽場。海中有彭湖嶼。南有晉江,自南安縣流入,經城西石塔山下,又東南至岱嶼入海。東北有洛陽江,南流入海。又東南有永寧衛,南有守禦福泉千戶所,俱洪武二十一年二月置。東南有祥芝、又有烏潯、南有深滬、又有圍頭四巡檢司。西南有安平城,嘉靖中築。東南有石湖城,萬曆中築。

南安府西少北。東南濱海。南有黃龍溪,即晉江之上流,西有桃林溪流入焉。南有石井巡檢司。又西北有澳頭、西南有達河二巡檢司,後廢。

同安府西南。西有文圃山。南濱海,有鹽場。西北有西溪,流合縣東之東溪、縣西之苧溪,又東南注於海。西南有守御金門千戶所,西有守御高浦千戶所,俱洪武二十一年二月置。又西南有永寧中左千戶所,在嘉禾嶼,即廈門也,洪武二十七年二月置。西有苧溪、南有塔頭山、東南有田浦、又有陳坑四巡檢司。又西南有白礁巡檢司,後移於縣西之珝口寨。東南有烈嶼巡檢司,後移於石潯港口。又有官澳巡檢司,後移於踏石寨。又有峰上巡檢司,後移於縣西之下店港口。

惠安府東北。東南濱海,有鹽場。西有洛陽江。又東南有守禦崇武千戶所,洪武二十一年二月置,嘉靖中移於縣東北。城東有黃崎、南有獺窟、東南有小水乍、東北有峰尾四巡檢司。又東北有塗嶺、又有沙格、東南有小兜三巡檢司,洪武二十年廢。

安溪府西。西北有佛耳山。南有藍溪。又西北有源口渡巡檢司,後遷白華堡,尋復。

永春府西北。西北有雪山,桃林溪出焉,東逕南安縣,北合藍溪,為雙溪口,又東逕南安縣,南合於黃龍溪。西有陳岩寨巡檢司,洪武中廢。

德化府西北。西北有戴雲山。西有太湖山。南有丁溪,又有滻溪,合而北流,入興化仙遊境。又西北有高鎮巡檢司,本東西團,後徙治,更名。東南有虎豹關。

漳州府元漳州路,屬福建道宣慰司。洪武元年為府。領縣十。東北距布政司七百里。

龍溪倚。東有岐山。西有天寶山。北有華封嶺,一名龍頭嶺。東南濱海,海中有丹霞等嶼。又東北有九龍江,亦名北溪,其上源出長汀及沙縣,流入縣界,歷龍頭嶺下,謂之峽中,至縣東出峽,為柳營江,又南有南溪流入焉。又東南為鎮門港,入於海。有柳營江巡檢司。又南有九龍嶺巡檢司。

漳浦府南。南有梁山,又東南有良山,與梁山相峙。東北有大武山。縣東南兩面皆濱海。南有漳江,亦曰雲霄溪,合李澳溪入於海。又有石塍溪。東北又有鎮海衛,東有守御六鰲千戶所。澳東南有古雷、又有後葛、東有井尾澳、西南有盤陀嶺四巡檢司。又東南有青山巡檢司,後徙治月峙,又西南有雲霄鎮,俱洪武二十一年二月內置。

龍巖府西。東有龍巖山,又有東寶山,舊產銀鉛。西有紫金山。北有九侯山。又南有龍川,下流入漳平界,為九龍江上源。東北有雁石巡檢司,後移於皞林口。

長泰府東。南有長泰溪,下流入九龍江。東南有朝天嶺巡檢司,後移於溪口。

南靖府西。舊治在西南,雙溪之北。嘉靖四十五年北徙大帽山麓。萬歷二十三年復還舊治。北有歐寮山。南有雙溪,入龍溪縣界,為南溪。北有永豐、西北有和溪二巡檢司。又有小溪、寒溪二巡檢司,後廢。

漳平府西北。成化六年以龍巖縣九龍鄉置,析居仁等五里地益之。東南有象湖山。南有百家畬洞,踞龍巖、安溪、龍溪、南靖、漳平五縣之交。又有九龍溪,自龍巖縣流經此,下流入龍溪縣。南有歸化巡檢司,後移於縣東之析溪口。又東北有溪南巡檢司,後廢。

平和府西南。正德十四年六月以南靖縣之河頭大洋陂置,析漳浦縣地益之。東南有三平山。東有大峰山,河頭溪所出,分數流達海,又西有盧溪流合焉。有盧溪巡檢司,後遷枋頭板,改名漳汀巡檢司。

詔安府南。本南詔守禦千戶所,弘治十八年置。嘉靖九年十二月改為縣。南臨海,海濱有川陵山,海中有南澳山。又東有東溪,為河頭溪分流,東南流入海。又南有守御玄鐘千戶所,東有守御銅山千戶所,俱洪武二十一年二月置。東有金石、洪淡二巡檢司。西南有分水關,漳、潮分界,巡檢司治焉。

海澄府東南。嘉靖四十五年十二月以龍溪縣之靖海館置,析漳浦縣地益之。東北濱海。西有南溪,自龍溪縣流入,與柳營江合流入海。東有海門巡檢司,後遷於青浦社。東北有濠門巡檢司,本治海滄洋,後遷縣東北之嵩嶼。東有島尾巡檢司。又西北有石馬鎮。

寧洋府西北。本龍巖縣之東西洋巡檢司,正統十一年置。嘉靖四十五年十二月改置縣,又析大田、永安二縣地益之。南有香寮山。東南有東洋,溪流所匯也。

福寧州元屬福州路。洪武二年八月降為縣,屬福州府。成化九年三月升為州,直隸布政司。北有龍首山。東有松山,山下有烽火門水寨,正統九年自海中三沙堡移此。東北有大姥山。東南濱海,海中有崳山、臺山、官澳山、屏風嶼。東有白水江。西有長溪,源出壽寧縣界,至縣西南古鎮門入海。東有福寧衛,南有守御大金千戶所,俱洪武二十一年二月置。西北有柘洋巡檢司,又有蘆門巡檢司,後移桐山堡。又東北有大筼簹巡檢司,後移秦嶼堡。又東有清灣巡檢司,後徙牙裏堡。南有高羅巡檢司,後移閭峽堡。又有延亭巡檢司,後移下滸堡。又東北有蔣洋,又有小瀾,西北有小澳、庫溪,西南有藍田,南有西臼六巡檢司,後廢。領縣二。西南距布政司五百四十五里。

寧德州西南。洪武二年屬福州府。成化九年來屬。北有霍童山,有龜嶼。東南濱海,中有官扈山,下有官井洋。又東有瑞峰,亦在海中。西有穹窿溪,西南有赤鑒湖,北有外渺溪,下流俱達於海。北有東洋麻嶺巡檢司,後徙涵村,又徙縣東北之雲淡門,又徙縣東之黃灣,後還故治。南有南靖關。東有長崎鎮。

福安州西北。洪武二年屬福州府。成化九年來屬。西南有城山。海在南。西北有長溪,東南入福寧州境。西北有白石巡檢司,後徙於縣東南之黃崎鎮。

廣東《禹貢》揚州之域及揚州徼外。元置廣東道宣慰使司,治廣州路。屬江西行中書省。又置海北海南道宣慰使司,治雷州路。屬湖廣行中書省。洪武二年三月以海北海南道屬廣西行中書省。四月改廣東道為廣東等處行中書省。六月以海南海北道所領并屬焉。四年十一月置廣東都衛。與行中書省同治。八年十月改都衛為廣東都指揮使司。九年六月改行中書省為承宣布政使司。領府十,直隸州一,屬州七,縣七十五。為里四千二十八。北至五嶺,與江西界。東至潮州,與福建界。西至欽州,與廣西界。南至瓊海。距南京四千三百里,京師七千八百三十五里。洪武二十六年編戶六十七萬五千五百九十九,口三百萬七千九百三十二。弘治四年,戶四十六萬七千三百九十,口一百八十一萬七千三百八十四。萬曆六年,戶五十三萬七百一十二,口五百四萬六百五十五。

廣州府元廣州路,屬廣東道宣慰司。洪武元年為府。領州一,縣十五:

南海倚。西北有石門山、雙女山。南濱海。又南有三江口。三江者,一曰西江,上流合黔、鬱、桂三水,自廣西梧州府流入;一曰北江,即湞水;一曰東江,即龍川水。俱與西江會,經番禺縣南,入於南海。西北有三江巡檢司,本治側水村,後遷村堡。又有金利、西南有神安、又有黃鼎、又有江浦四巡檢司。又南有五斗口巡檢司,後遷磨刀口,又遷佛山鎮。

番禺倚。在城有番、禺二山,縣是以名。東有鹿步、南有沙灣、北有慕德、東南有茭塘、又有獅嶺五巡檢司。

順德府西南。景泰三年,以南海縣大良堡置,析新會縣地益之。西北有西江,南有馬寧、北有紫泥二巡檢司。西有江村巡檢司,後遷縣西北查浦。北有寧都巡檢司,後遷都粘堡。又東南有馬岡巡檢司,後廢。

東莞府東南。南濱海,海中有三洲,有南頭、屯門、雞棲、佛堂門、十字門、冷水角、老萬山、零丁洋等澳。北有東江。西有中堂、西南有白沙、又有缺口鎮三巡檢司。東北有京山巡檢司,本治茶園,後遷京口村,更名。又西南有虎頭山關,洪武二十七年置。

新安府東南。本東莞守禦千戶所,洪武十四年八月置。萬歷元年改為縣。南濱海。有大鵬守御千戶所,亦洪武十四年八月置。東南有官富、西北有福永二巡檢司。

三水府北。嘉靖五年五月以南海縣之龍鳳岡置,析高安縣地益之。西江在南,北江在西。又西南有三江、北有胥江、東有西南鎮三巡檢司。又南有橫石巡檢司。

增城府東。東有增江,南有東江。西南有烏石、西北有茅田二巡檢司。

龍門府東。弘治六年以增城縣七星岡置,析博羅縣地益之。南有龍門水,亦曰九淋水,流入東江。東有上龍門巡檢司。

香山府南。南濱海。東有零丁洋。北有黃圃巡檢司。西北有大攬巡檢司,本名香山,後更名。

新會府西南。南濱海,中有崖山。東北有西江。西南有恩平江,一名峴岡水。東南有潮連、西有牛肚灣二巡檢司。又西北有樂逕巡檢司,後遷縣北之石螺岡。又東北有大瓦巡檢司,本治中樂都,後遷鸞臺村。又南有沙村巡檢司,本治大神岡,後遷仙洞村,又遷長沙村,後復故治。

新寧府西南。弘治十一年以新會縣德行都之上坑蓢置,析文章等五都地益之。南濱海。北有恩平江,一名長沙河。又南有廣海衛,洪武二十七年九月置。西有望高巡檢司,西南有城岡巡檢司,後廢。

從化府東北。弘治二年以番禺縣橫潭村置,析增城縣地益之。九年遷於流溪馬場曲。東北有流溪巡檢司,本治縣北石潭村,後遷神岡村。

清遠府北。東有中宿峽。西有大羅山。又湞水在縣東北,東南有潖水來合焉,謂之潖江口,有潖江巡檢司。又西南有回岐、西北有濱江二巡檢司。東北有橫石磯巡檢司,後廢。

連州元桂陽州,直隸廣東道。洪武二年三月省入連州。四月,連州廢,地屬連山。三年九月,連山廢,地屬陽山。十四年置連州於此,屬府。東北有桂水。西有湟水,亦曰洭水,自湖廣寧遠縣流入,東南合湞水。西北有朱岡巡檢司。又有西岸巡檢司,治仁內鄉,後徙陽山縣境。東南距府五百六十里。領縣二:

陽山州東北。元屬桂陽州。洪武二年三月,桂陽州廢,屬連州。四月,連州廢,屬韶州府。十四年四月改為連州,徙州於桂陽州舊治,復置縣,屬焉。南有陽溪,即洭水。西北有星子巡檢司。東有西岸巡檢司,自連州移此,治青蓮水口。又北有湟谿、陽山二關。

連山州西。元連州治此,直隸廣東道。洪武二年四月,州廢,屬韶州府。三年九月省入陽山。十三年十一月復置。十四年四月屬州。舊治在縣西北鐘山。永樂元年徙縣西程山下。天順六年又徙小坪。南有黃連山。北有高良水,又名大獲水,東至州界入湟水。西有宜善巡檢司,即程山下舊縣治。

肇慶府元肇慶路,屬廣江道。洪武元年為府。領州一,縣十一。南距布政司二百三十里。

高要倚。北有石室山。南有銅鼓山。東有高峽山、爛柯山。城南有西江,又南有新江,東南有蒼梧水,俱流入焉。東南有古耶巡檢司,治龍池都之馮村,後遷縣東之橫槎下都。東有祿步巡檢司,初在下村,後遷上村水口。東有橫槎巡檢司,初治上半都,後遷水口,尋廢。

高明府東南。本高要縣高明鎮巡檢司,成化十一年十二月改為縣,析清泰等都益之。南有倉步水,一名滄江,下流入於西江。東北有太平巡檢司,治太平都,後遷縣東都含海口。又遷縣西南山臺寺,又遷縣東清溪申石奇海濱。

四會府北。南有北江。東有南津巡檢司,治黃岡村,尋遷縣東南南津水口。

新興府南。元新州治,直隸廣東道。洪武二年四月州廢來屬。東有新江。西南有立將巡檢司。又南有祿緣巡檢司,後廢。

開平府南。本恩平縣之開平屯。明末改為縣,析新興、新會二縣地益之。南有恩平江,源出舊恩平縣西北平城山,東流合烏石水,下流入廣州新會縣界。東南有沙岡巡檢司,本治沙岡村,後遷平康都之長沙村。又南有松柏、北有四合二巡檢司。

陽春府南。元屬南恩州。洪武元年屬新州。二年四月,新州廢,屬府。西有漠陽江。北有古良巡檢司,尋廢,後復置於縣西,又遷南鄉都小水口。又北有思良巡檢司,後廢。

陽江府南。元南恩州治此,直隸廣東道。洪武元年,南恩州廢,改屬新州。二年四月,新州廢,屬府。南濱海。中有海陵山,山西北為鶴州山,海陵巡檢司在焉。西有漠陽江,源出古銅陵縣北雲浮山下,南流過陽春縣,會諸水,經南恩舊城,直通北津港門,入於海。東南有海朗守禦千戶所,西南有雙魚守御千戶所,俱洪武二十七年置。又東北有蓮塘堡,西有太平堡,俱嘉靖間築。

恩平府南。本陽江縣之恩平巡檢司,初治縣東北之恩平故縣,後遷恩平堡。成化十四年六月改堡為縣,析新興、新會二縣地益之,而遷巡檢司於縣東南之城村,仍故名,後又遷白蒙屯。縣南有恩平江。

廣寧府西北。嘉靖三十八年十月以四會縣地置。初治縣東南潭圃山下,後遷大圃村福星山下,即今治也。北有綏江,又有龍口屯田千戶所,亦嘉靖三十八年置。西北有金溪巡檢司。南有扶溪巡檢司,初治東鄉水口,後遷扶溪口,又遷官埠。

德慶州元德慶路,屬廣東道。洪武元年為府。九年四月降為州,以府治端溪縣省入,來屬。西有小湘峽,西江經其中,端溪自東北流入焉。東有悅城鄉巡檢司,治悅城故縣,後遷靈溪水口。東距府二百十里。領縣二:

封川州西。元封州治此,直隸廣東道。洪武二年三月,州廢,改屬。南有西江,西有賀江,西北有東安江,俱流入焉。北有文德巡檢司,初治縣西北大洲口,後遷縣西賀江口,後又遷於此。

開建州西北。元屬封州。洪武二年三月改屬。西有開江,一名封溪,即賀江之下流。北有古令巡檢司,治古令村,後遷縣東北之褥村。

韶州府元韶州路,屬廣東道宣尉司。洪武元年為府。領縣六。西距布政司八百里。

曲江倚。永樂二十二年建淮王府,正統元年遷於江西饒州府。南有蓮花山。東北有韶石山。西有桂山。湞水在東,東南有曹溪水,西有武水,俱流入焉,抱城回曲,故謂之曲江,下流即始興江。東北有平圃、南有蒙浬二巡檢司。

樂昌府西北。南有昌山。東北有靈君山。西有三瀧水,即武水。北有九峰、西北有黃圃、又有羅家灣三巡檢司。東有高勝巡檢司,後廢。

英德府西南。元英德州,直隸廣東道。洪武二年三月降為縣,來屬。南有皋石山,一名湞陽峽。又湞水在縣東,一名溱水,洭水在縣西,一名洸水,至縣西南合流,謂之洸口,有洸口巡檢司。又南有瀧頭水,與湞水合。又東有象岡、北有清溪、西有含洸三巡檢司。又南有南崖巡檢司,廢。

仁化府東北。治水西村,後遷城口村。西北有吳竹嶺,吳溪水出焉,下流為潼溪,入湞。東北有扶溪巡檢司。又北有恩村巡檢司。

乳源府西。本治虞塘,洪武元年遷於洲頭津。西有臈嶺,五嶺之一。西北有武水,自湖廣宜章縣流入,有武陽巡檢司。

翁源府東南。元屬英德州。洪武二年三月改屬。故城在西北,今治本長安鄉也,洪武初,遷於此。北有寶山。東有靈池山,滃溪出焉,即瀧頭水。東有桂丫山巡檢司,初治茶園鋪,後遷南浦。

南雄府元南雄路,屬廣東道。洪武元年為府。領縣二。西距布政司千九十里。

保昌倚。大庾嶺在北,亦曰梅嶺,上有梅關,湞水所出。西北有凌江水,流合焉,南至番禺入海,謂之北江。又縣東有小庾嶺。西北有百順、東南有平田二巡檢司。又東北有紅梅巡檢司,舊治梅關下,後遷於此。

始興府西。西有始興江,即湞水。南有清化徑巡檢司。又東北有黃塘巡檢司,本治瓔珞鋪,後遷黃塘江口,又遷黃田鋪。

惠州府元惠州路,屬廣東道宣慰司。洪武元年為府。領州一,縣十。西北距布政司三百六十里。

歸善倚。南濱海。西江在西南。東有東江,自江西安遠縣流入府境,亦曰龍川江,西南至番禺縣,會西江入海。東南有平海守禦千戶所,洪武二十七年九月置。又有內外管理、又有碧甲二巡檢司。

博羅府西北。西北有羅浮山。南有東江。西有石灣、又西北有善政里二巡檢司。

長寧府西。隆慶三年正月以歸善縣鴻雁洲置,析韶州府英德、翁源二縣地益之。萬歷元年徙治君子峰下。南有新豐江,下流入龍江。西有乍坪巡檢司。又北有黃峒巡檢司,後廢。

永安府東北。隆慶三年正月以歸善縣安民鎮置,析長樂縣地益之。西有東江。西南有寬仁里巡檢司,治苦竹派,後遷桃子園。又有馴雉里巡檢司,治鳳凰岡,後遷縣東烏石屯。尋俱還故治。

海豐府東。北有五坡嶺。南濱海,一名長沙海。又東南有碣石衛,東有甲子門守禦千戶所,俱洪武二十七年十月置。南有捷勝守禦千戶所,洪武二十八年二月置,初名捷徑,三月更名。有甲子門巡檢司。又西有鵝埠嶺巡檢司。又西南有長沙港巡檢司,後遷謝道。

龍川府東北。元循州治此,直隸廣東道。洪武二年四月,州廢來屬。東有霍山。南有龍江,即東江上流,自江西安遠縣流入。東有通衢巡檢司,後遷老龍埠,尋還故治。東北有十一都巡檢司。

長樂府東北。元屬循州。洪武四年四月來屬。舊治在紫金山北。洪武初,徙於今治。東南有興寧江。南有十二都巡檢司。又西有清溪巡檢司,後廢。

興寧府東北。元屬循州。洪武二年四月來屬。南有興寧江,東入潮州府程鄉縣界。東南有水口巡檢司,治水口隘,後廢,復置於下岸,尋遷於上岸水東。又北有十三都巡檢司,後遷白水砦,尋復故。

連平州本連平縣。崇禎六年以和平縣惠化都置,析長寧、河源二縣及韶州府翁源縣地益之。尋升為州。西有銀梅水,源出楊梅坪,即湞水上源。南有長吉里、東南有忠信里二巡檢司。東北距府百八十里。領縣二:

河源州北。舊屬府,崇禎六年改屬州。故城在西南。洪武二年徙於壽春市。萬曆十年遷於今治。南有槎江,即龍川江,下流為東江。又北有新豐江入焉。又東北有藍口巡檢司。

和平州少北。正德十三年八月以龍川縣之和平司置,析河源縣地益之,屬府。崇禎六年改屬州。北有九連山。西北有浰頭山,三浰水出焉,亦名和平水,有浰頭巡檢司。

潮州府元潮州路,屬廣東道宣慰司。洪武二年為府。領縣十一。西距布政司千一百九十里。

海陽倚。南濱海,有急水門。東有鱷溪,一名惡溪,亦名韓江,又名意溪,東入於海。西北有潘田巡檢司。又有楓洋巡檢司,尋遷縣南園頭村。

潮陽府南。東南濱海。西南有練江。南有海門守御千戶所,洪武二十七年置。東有招寧、西北有門闢、北有桑田三巡檢司。又有吉安巡檢司,治水戎水都南山下,後遷貴嶼村。

揭陽府西。西北有揭嶺。南有古溪。東南濱海。西有南寨巡檢司,本名湖口,治湖口村,後遷棉湖寨,更名。東有北寨巡檢司,本治縣西北岡頭山,後遷縣東北鳥石山南,尋還舊治,後又遷縣東桃山鋪前。

程鄉府西北。元梅州治此,直隸廣東道。洪武二年四月,州廢來屬。南有梅溪,即興亭江之下流,一名惡溪,西北有程江合焉。西有太平鄉巡檢司,治梅塘堡,後遷縣西北石鎮村旁。東南有豐順鄉巡檢司,本在縣西北平遠縣界,後遷松口市。

饒平府東北。成化十二年十月以海陽縣三饒地置,治下饒。東南濱海,海中有南澳山,有大成守御千戶所,洪武二十七年置。南有黃岡、西有鳳凰山二巡檢司。又東南有柘林寨。

惠來府西南。嘉靖三年十月以潮陽縣惠來都置,析惠州府海豐縣地益之。南濱海。西有三河,以大河、小河、清遠河三水交會而名,即韓江之上源。東南有靖海守禦千戶所,洪武二十七年置。南有神泉巡檢司,本名北山,治縣西北山村,後遷神泉村,更名。

鎮平府北。本平遠縣石窟巡檢司,崇禎六年改為縣,析程鄉縣地益之。西有石窟溪,下流入於程江。東有藍坊巡檢司,自石窟司遷治,更名。

大埔府東。嘉靖五年以饒平縣大埔村置,析水戀洲,清遠二都地益之。南有神泉河,即福建汀州府之鄞江。又西有惡溪。東北有虎頭沙、西有三河鎮二巡檢司。又南有大產巡檢司,後遷黃沙。西南有烏槎巡檢司,後遷高陂。

平遠府西北。嘉靖四十一年五月以程鄉縣豪居都之林子營置,析福建之武平、上杭,江西之安遠,惠州府之興寧四縣地益之,屬江西贛州府。四十二年正月還三縣割地,止以興寧程鄉地置縣,來屬。

普寧府西南。嘉靖四十二年正月以潮陽縣水戎水都置,析洋烏、黃坑二都地益之,寄治貴山都之貴嶼。萬曆十年移治黃坑,以洋烏、水戎水二都還潮陽。西有冬瓜山,冬瓜水出焉,下流為揭陽縣之古溪,與南、北二溪合,下流至澄海縣入於海。西南有雲落徑巡檢司。

澄海府東南。本海陽縣之闢望巡檢司。嘉靖四十二年正月改為縣,析揭陽、饒平二縣地益之,而徙闢望巡檢司於縣北之南洋府,仍故名。南濱海,亦曰鳴洋海。西南有蓬州守御千戶所,洪武二十六年四月置。又有駝浦巡檢司。

高州府元高州路,屬海北海南道,治電白。洪武元年為府。七年十一月降為州。九年四月復為府。後徙治茂名。領州一,縣五。東南距布政司一千里。

茂名倚。洪武七年十一月省,十四年五月復置。南濱海。城西有竇江,源出信宜縣,東北流,鑒江入焉,西南流入化州界。南有赤水巡檢司。東南有平山巡檢司,治紅花堡,後遷縣東北之電白故縣。又西南有博茂巡檢司,後廢。

電白府東。舊治在西北。今治本神電衛,洪武二十七年十月置。成化三年九月遷於此。東濱海。西北有立石巡檢司,後廢。

信宜府北。南有竇江。東北有中道巡檢司,治在懷德鄉黃僚寨之左,廢,後復置於羅馬村,尋又遷於三橋。

化州元化州路,屬海北海南道。洪武元年為府。七年十一月降為州,以州治石龍縣省入。九年四月又降為縣,來屬。十四年五月復為州。北有石城山,又有來安山。東北有茂名水,竇江之下流。又有陵水、羅水,俱自廣西北流縣流入,與茂名水合,至吳川縣為吳川水,南入於海。北有梁家沙巡檢司。東南距府九十里。領縣二:

吳川州南。元屬化州路。洪武九年四月屬高州府。十四年五月改屬州。南濱海,中有岡洲。有岡洲巡檢司,在洲南濱海,後遷洲上。東南有寧川守御千戶所,洪武二十七年四月置。又北有寧村巡檢司,治川水窖,後遷縣西北之地聚村,又遷於芷皞口。

石城州西。元屬化州路。洪武九年四月屬高州府。十四年五月改屬州。南濱海。西有零緣巡檢司。

雷州府元雷州路,屬海北海南道宣慰司。洪武元年為府。領縣三。東距布政司千四百五十里。

海康倚。東濱海。南有擎雷水,自擎雷山南流,東入於海。西有海康守禦千戶所,洪武二十七年十月置。西南有清道、東南有黑石二巡檢司。

遂溪府北。東西濱海。西南有樂民守禦千戶所,洪武二十七年十月置。西北有湛川巡檢司,治故湛川縣,後遷縣東南故鐵杷縣。又西南有潿洲巡檢司,治海島中博里村,後遷蠶村。

徐聞府南。東西南三面濱海。西有海安守禦千戶所,東有錦囊守御千戶所,俱洪武二十七年十月置。西南有東場、東有寧海二巡檢司。又西北有遇賢巡檢司,廢。

廉州府元廉州路,屬海北海南道宣慰司。洪武元年為府。七年十一月降為州。九年四月屬雷州府。十四年五月復為府。領州一,縣二。東距布政司千二百十里。

合浦倚。洪武七年十一月省,十四年五月復置。東有大廉山,州以此名。東南濱海,亦曰珠母海,以海中有珠池也。又城北有廉江,亦曰合浦江,自廣西容縣流入,逕州,江口分為五,西南注於海。又北有石康縣,成化八年省。東有永安守御千戶所,洪武二十七年置。東南有珠場、東北有永平二巡檢司。又北有高仰巡檢司,治馬欄墟,後遷於縣西南。

欽州元欽州路,屬海北海南道。洪武二年為府。七年十一月降為州,以州治安遠縣省入。九年四月降為縣,來屬。十四年五月復為州。西南濱海,中有烏雷山,入安南之要道也。又有分茅嶺,亦與安南分界。龍門江在城東,又東有欽江,俱入於海。南有淞海、西南有長墩、西北有管界三巡檢司。又西有如昔、又有佛淘二巡檢司,與交址接界,宣德二年入於安南,嘉靖二十一年復。又西南有千金鎮。東距府百四十里。領縣一:

靈山州北。元屬欽州。洪武九年四月屬廉州。十四年五月仍屬欽州。北有洪崖山,洪崖江出焉,經縣東,與羅陽山水合,為南岸江,南流為欽江。又南有林墟、西有西鄉二巡檢司。

瓊州府元乾寧軍民安撫司。元統二年十月改為乾寧安撫司,屬海北海南道宣慰司。洪武元年十月改為瓊州府。二年降為州。三年仍升為府。領州三,縣十。東北距布政司千七百五十里。

瓊山倚。南有瓊山。北濱海,有神應港,亦曰海口渡,有海口守禦千戶所,洪武二十年十月置。又西南有水蕉村,萬曆二十八年置水會守禦千戶所於此。南有石山。又有清瀾巡檢司,廢。

澄邁府西。北濱海。南有黎母江。東有澄江。西北有澄邁巡檢司,治石矍都。南有兔穎巡檢司,治曾家東都,後遷南黎都,廢。西南有銅鼓巡檢司,治新安都,後遷西黎都,廢。又有那拖巡檢司,治那拖市,後遷縣西森山市,廢。

臨高府西。北濱海。南有黎母江。南有田牌巡檢司,後遷墳橫岡。又東有定南、北有博鋪二巡檢司,廢。

安定府南。元至元二十九年六月置。天歷二年十月升為南建州。洪武元年十月復為縣。南有五指山,亦曰黎母山,黎人環居山下,外為熟黎,內為生黎。北有建江,繞郡境西北流,入南渡江。東有潭覽屯田千戶所,元置,洪武中因之,永樂四年廢。西有青寧巡檢司。又東有寧村巡檢司,治潭覽村,後遷縣東南南資都,仍故名。

文昌府東。西北有七星山。南有紫貝山。東北濱海。東南有文昌江,入於海。又東北有清瀾守御千戶所,洪武二十七年八月置,萬歷九年遷縣東南南毛都陳家村。西北有鋪前巡檢司。東北有青藍頭巡檢司,後遷縣東抱凌港。

會同府東南。元至元二十九年六月置。東濱海。西有黎盆溪,東有調囂巡檢司,治端趙都,尋遷縣東南南滄村。

樂會府東南。西有白石山。東濱海。西北有萬泉河,有黎盆水流入焉。

儋州元南寧軍,屬海北海南道宣慰司。洪武元年十月改為儋州,屬府。正統四年六月以州治宜倫縣省入。西北有龍門嶺。西濱海。北有倫江。西南有鎮南、又有安海二巡檢司。又東有歸姜巡檢司,廢。東北距府三百七十里。領縣一:

昌化州南。舊城在東南,今城本昌化守御千戶所,洪武二十五年置。正統六年五月徙縣治焉。西濱海。南有昌江。

萬州元萬安軍,屬海北海南道。洪武元年十月改為萬州,屬府。正統四年六月以州治萬安縣省入。北有六連山,龍滾河出焉。東南海中有獨洲山。東有蓮塘巡檢司,後廢。西北距府四百七十里。領縣一:

陵水州南。東北有舊縣城,今治本南山守禦千戶所,洪武二十七年置。正統間,遷縣於此。西有小五指山。東濱海,海中有雙女嶼。東北有牛嶺巡檢司。

崖州元吉陽軍,屬海北海南道宣慰司。洪武元年十月改為崖州,屬府。正統四年六月以州治寧遠縣省入。南有南山。北有大河,自五指山分流,南入海。東有滕橋、西有抱歲、又西北有通遠三巡檢司。北距府千四百一十里。領縣一:

感恩州西北。舊屬儋州。正統五年來屬。西濱海。南有南湘江,源自黎母山,西南入於海。東南有延德巡檢司。

羅定州元瀧水縣,屬德慶路。洪武元年屬德慶州。萬歷五年五月升為羅定州,直隸布政司。西南有瀧水,源出瑤境。又有瀧水、新寧、從化三千戶所,俱萬歷七年置。又有函江守御千戶所,萬曆五年五月置於西寧縣境,十六年遷於州界之鳷溝驛。南有開陽鄉、西北有晉康鄉二巡檢司。又東有建水巡檢司,治建水鄉,後遷縣東南古模村,又遷高要縣白坭村,尋復還白模。領縣二。東距布政司五百三十里。

東安州東。萬歷五年十一月以瀧水縣東山黃姜峒置,析德慶州及高要、新興二縣地益之。北有西江,西有瀧水流入焉。東北有南鄉守禦千戶所,西南有富霖守禦千戶所,俱萬曆五年五月置。東南有羅苛巡檢司。

西寧州西。萬曆五年十一月以瀧水縣西山大峒置,析德慶州及封川縣地益之。東北有西江,與德慶州分界。東南有瀧水。西南有封門守禦千戶所,萬曆五年五月置。北有都城鄉巡檢司。又西南有懷鄉巡檢司,後廢。

廣西《禹貢》荊州之域及荊、揚二州之徼外。元置廣西兩江道宣慰使司,治靜江路。屬湖廣行中書省。至正末,改宣慰使司為廣西等處行中書省。洪武二年三月因之。六年四月置廣西都衛。與行中書省同治。八年十月改都衛為都指揮使司。九年六月改行中書省為承宣布政使司。領府十一,州四十有八,縣五十,長官司四。為里一千一百八十三。北至懷遠,與湖廣、貴州界。東至梧州,與廣東界。西至太平,與貴州、雲南界。南至博白,與廣東界。距南京四千二百九十五里,京師七千四百六十二里。洪武二十六年編戶二十一萬一千二百六十三,口一百四十八萬二千六百七十一。弘治四年,戶四十五萬九千六百四十,口一百六十七萬六千二百七十四。萬曆六年,戶二十一萬八千七百一十二,口一百一十八萬六千一百七十九。

桂林府元靜江路。洪武元年為府。五年六月改為桂林府。領州二,縣七:

臨桂倚。洪武三年七月建靖江王府於獨秀峰前。東有桂山。東北有堯山。又有桂江,亦曰漓江,南有陽江來合焉,至蒼梧縣合於左、右江。東有蘆田市、西有兩江口二巡檢司。南有湘山渡巡檢司,後廢。

興安府北。南有海陽山,湘水出其北,流入湖廣永州府界,漓水出其南,南入梧州府界。北有越城嶺,亦曰始安嶠,五嶺之最西嶺,下有始安水流入漓水。西南有融江六峒、西有鹽砂寨、北有唐家鋪三巡檢司。又西南有巖關。

靈川府北。北有百丈山。東北有融江,源出融山二洞中,一名銀江,流經縣境,又南入靈川縣界,合於漓江。南有白石潭、東北有千秋峽二巡檢司。

陽朔府南。北有陽朔山。東有漓江。東南有伏荔市、南有都樂墟二巡檢司。西有白竹寨巡檢司,廢。

全州元全州路,屬湖廣道。洪武元年為府。九年四月降為州,省州治清湘縣入焉,屬湖廣永州府。二十七年八月來屬。西有湘山。南有湘水,又北有洮水流合焉。又西有西延、西南有建安、東北有柳浦三巡檢司。又東北有平塘巡檢司,廢。南距府二百五十里。領縣一:

灌陽州南少東。南有灌水,經州界,合於湘水。西南有吉寧鄉崇順里巡檢司。

永寧州元古縣。洪武十四年改為古田縣。隆慶五年三月升為永寧州。縣舊治在今州南三十里。洪武初,移於今州南八里。成化十八年又移今治。又有黃源水,下流入漓江。南有桐木鎮、又有常安鎮、西南有富椽鎮三土巡檢司。東距府百五十里。領縣二:

永福州東南。舊屬府,隆慶五年三月改屬州。西南有太和山,太和江環其下,東入柳州府,為雒清江。又西南有理定縣,元屬靜江舊路,正統五年九月省。又有蘭麻鎮、東北有銅鼓市二巡檢司,廢。

義寧州東北,舊屬府,隆慶五年三月改屬州。北有丁嶺,義江出焉,下流分為二,東流者為臨桂縣之相思水,入於漓江,南流者為永福縣之白石水,即太和江也。西北有桑江口巡檢司。

平樂府元大德五年十一月置。洪武元年因之。領州一,縣七。北距布政司百九十里。

平樂倚。東南有魯溪山。西北有漓江,又北有樂川水,東經昭潭流合焉。又東有榕津寨巡檢司,又有水滻營土巡檢司。又東有龍平寨巡檢司、昭平堡土巡檢司,廢。又東有團山堡,東南有廣運堡、足灘堡,又南有甑灘堡,俱弘治後置。

恭城府東北。南有樂川水,又東有勢江,南有南平江,北有平川江,西南有西水江,俱流合焉。東北有鎮峽寨、東有勢江源二巡檢司。又有白面寨、西嶺寨二土巡檢司。

富川府東少北。元屬賀州。洪武十年五月改屬潯州府,後來屬。西南有鐘山縣,舊治於此,洪武二十九年十一月移治靄石山下,而置邊蓬寨巡檢司於舊治。北有秦山,接湖廣道州界。東北有氓渚嶺,即臨賀嶺,與湖廣江華縣分界。又東有富江,南合賀水。西南有白霞寨、西北有寨下市二巡檢司。

賀府東南。元賀州,直隸廣西兩江道。洪武初,以州治臨賀縣省入,屬潯州府。十年五月降為縣,後來屬。東北有臨賀嶺,亦曰桂嶺,下有桂嶺縣,元末廢。東有賀江,至廣東封川縣合於西江。南有信都鄉巡檢司。北有沙田寨巡檢司,後遷縣西點燈寨,尋廢。又東北有大寧寨、樊字寨、白花洞三土巡檢司,後廢。

荔浦府西少南。舊屬桂林府,弘治四年來屬。舊治在今縣西。景泰七年移於後山,即今治。東有銅鼓嶺,一名火焰山。又荔江在南,下流入漓江。東南有峰門寨巡檢司,後遷中峒。西北有南原寨巡檢司,後遷縣東南下峒,又遷縣東延濱江。又西南有華蓋城,萬曆中築。

修仁府西少南。舊屬桂林府,弘治四年來屬。舊治在今縣西馬浪坪。景泰初,遷今縣南霸寨村。成化十五年遷於五福嶺,即今治。東北有荔江,有麗壁市土巡檢司。西南有石墻堡,萬歷間築。

昭平府南少東。萬曆四年四月析平樂、富川二縣地置。五年又析賀縣地益之。東有五指山。又有漓江。又有思勤江,下流入於漓江。東南有龍平縣,元屬府,洪武十八年廢。

永安州元立山縣,屬府。洪武十八年廢為立山鄉,屬荔浦縣。成化十三年二月置州,曰永安,屬桂林府。弘治三年九月改為長官司。五年復為州,來屬。東有蒙山,下有蒙水。南有古眉寨土巡檢司。北有群峰寨土巡檢司,後遷州西北杜莫寨,又遷州北貓兒堡。東南有仙迴營,萬歷中置。東北距府百二十里。

梧州府元梧州路。洪武元年為府。領州一,縣九。北距布政司五百八十里。

蒼梧倚。城西南有大江,江即黔、鬱二水,合流於潯州府城東,為潯江;入府界,東經立山下,又東經此,與桂江合,謂之三江口,下流為廣東之西江。東有長行、西有安平、北有東安、西南有羅粒四巡檢司。

藤府西。元藤州。直隸廣西兩江道。洪武二年九月省州治鐔津縣入焉。十月來屬。十年五月降為縣。北有藤江,亦曰鐔江,即潯江也。東南有繡江,西有幕僚江,俱流入焉。又西北有五屯守禦千戶所,嘉靖初置。西有白石寨、南有竇家寨、東北有赤水鎮三巡檢司。又東有溻洲、南有周村、西南有驛面、又南有思羅四巡檢司,廢。

容府西南。元容州,直隸廣西兩江道。洪武二年十月來屬。十年五月降為縣,省州治普寧縣入焉。西北有容山。南有容江,亦名繡江。又東有波羅里大洞、西南有粉壁寨二巡檢司。

岑溪府南少西。元屬藤州。洪武十年五月改屬府。東北有烏峽山。西有繡江。東南有上里平河村、西南有南渡二巡檢司。又東南有連城鄉義平巡檢司,廢。

懷集府東北。元屬賀州。洪武初,屬平樂府。十年五月來屬。西南有懷溪水。東有武城鄉、西有慈樂寨、西北有蘭峒寨三巡檢司。

鬱林州元直隸廣西兩江道。洪武二年九月以州治南流縣省入。十月來屬。南有南流江,至廣東合浦縣入海。有橫嶺、文俊二巡檢司,廢。東北距府三百三十里。領縣四:

博白州西南。西有雙角山,綠珠江出其下,流合縣南飲馬水,下流入南流江。南有周羅、西南有沙河二巡檢司。又有安定、春臺、平山、兆常四土巡檢司,尋廢。又東南有海門鎮,舊為入安南之道。

北流州北。元屬容州。洪武十年五月來屬。東北有勾漏山。東有銅石山,產水銀、硃砂。又南有扶來山,陵水出焉,西南有峨石山,羅水出焉,俱流入廣東化州界。又北有綠藍山,綠藍水出焉,分為二。東流者經城東登龍橋,與廣東高州府流入之繡江合,又東經容縣,為容江。西流者入鬱林州,為南流江。南有雙威寨巡檢司。西有都隴、又有中山、又有清灣三巡檢司,廢。又西有天門關,本名鬼門關,洪武初,改為桂門關;宣德中,更今名。

陸川州南少東。元屬容州。洪武十年五月來屬。舊為入安南之道。東有龍化江,下流合容江。南有溫水寨巡檢司。

興業州西少北。南有鐵城山。北有翻車嶺,龍母江出焉,下流入南流江。南有趙家寨、西有長寧寨、北有平安寨、又有棠木寨四巡檢司,後俱廢。

潯州府元潯州路。洪武元年為府。領縣三。東北距布政司九百八十里。

桂平倚。南有白石山。西北有大藤峽。北有黔江,一名北江,亦曰右江,南有鬱江,一名南江,亦曰左江,至城東匯為潯江。東北有武靖州,成化三年置,萬歷末廢。又東有大黃江口、北有靖寧鄉、東北有大宣鄉、又有思隆鄉、又有木盤浦、西南有常林鄉六巡檢司。又南有羅秀土巡檢司,又北有碧灘堡、鎮峽堡,俱成化中置。東有牛屎灣堡,西有淹沖堡、秀江堡,俱嘉靖中置。

平南府東。東南有龔江,即潯江也,東有白馬江流入焉。又有奉議衛,洪武二十八年八月置於奉議州,正統六年五月遷於此。東北有大同、西北有泰川、西南有武林三巡檢司。又南有峒心、東南有三堆、東北有大峽、西北有平嶺四土巡檢司。

貴府西。元貴州,直隸廣西兩江道,洪武二年十月降為縣,來屬。南山在南。又有東、西、北三山。南有鬱江,亦曰南江,群川悉流入焉。有向武軍民千戶所,本向武守御千戶所,洪武十八年十月置於向武州,三十年三月升軍民所,正統六年五月來遷縣北門外,萬歷二十三年又遷縣西北謝村鎮。東南有新安寨、北有北山寨二巡檢司。又南有橋頭墟、西有瓦塘渡、又有五州寨、又有東鋋渡、又有郭東里五巡檢司,廢。又東南有三江城,萬曆中築。

柳州府元柳州路。洪武元年為府。領州二,縣十。東北距布政司四百里。

馬平倚。元為府屬,洪武元年徙府治於此。南有柳江,亦曰潯水,亦曰黔江,上流自貴州黎平府流入府境,下流至桂平縣合於鬱江,亦曰右江。南有新興鎮、都博鎮二巡檢司。又有歸化鎮巡檢司,廢。

洛容府東北。舊治白龍岩,天順中,徙於硃峒。正德時,為瑤、僮所據,嘉靖三年十一月復,萬歷四年正月遷於靈塘,以硃峒舊治為平樂鎮,留兵百名守之。城南有洛清江,至馬平縣入於柳江。西南有江口鎮、又有運江二巡檢司。東有平樂鎮巡檢司,治石榴江,後遷縣東北中渡。又西南有章洛鎮巡檢司,廢。

柳城府西北。舊治龍江南,元為府治。洪武元年遷治龍江東,而府徙治馬平縣。龍江自天河縣流入,合於融江,即柳江上流。東有東泉鎮巡檢司。北有古枿鎮巡檢司,初治融江東岸,後遷馬頭驛。又東北有古清鎮、西有洛好鎮、又有廖洞鎮三巡檢司。

羅城府西北。洪武二年十月以羅城鄉置,屬融州。十年五月來屬。北有武陽江,下流合於融江。北有武陽鎮、又有莫離鎮、又有通道鎮三巡檢司。又舊有安湘鎮、樂善鎮、中峒鎮三巡檢司,廢。

懷遠府北。元屬融州。洪武十年廢,置三江鎮巡檢司。十三年十一月復置縣,來屬,治大融江、潯江之匯。萬曆十九年移治丹陽鎮。西北有九曲山,山南為石門山,兩山夾峙。福祿江自貴州永從縣流逕其中,至融縣為融江,至柳城縣為柳江。又東北有潯江,自湖廣靖州流合焉,有潯江鎮巡檢司。又西北有萬石鎮,又有宜良鎮、丹陽鎮三巡檢司。

融府西北。元融州,直隸廣西兩江道。洪武二年十月以州治融水縣省入,來屬。十年五月降為縣。東南有靈巖山。北有雲際山。其西曰上石門,以兩山夾峙,融江中流也。又東有寶積山,產鐵。東北有思管鎮、東南有清流鎮、西南有鵝頭隘三巡檢司。又北有長安鎮巡檢司,本在融江東岸,後遷西岸。又有大約鎮土巡檢司。又有保江鎮、理源鎮、西峒鎮三巡檢司,廢。

來賓府南。元屬象州。洪武十年五月來屬。西南有白牛洞。北有白雲洞。南有大江,亦曰都泥江。西有界牌鎮巡檢司,後遷縣南之南岡。

象州元直隸廣西兩江道。洪武二年十月來屬,以州治陽壽縣省入。西有象山。東有雷山。南有象江,即柳江。東北有龍門寨巡檢司。又有鵝頸鎮、尖山鎮二巡檢司,廢。西北距府百十三里,領縣一:

武宣州南。元曰武仙。宣德六年更名。舊治陰江。宣德六年三月徙於高立。東南有大藤峽,後名永通峽。西有柳江,又有都泥江,亦謂之橫水江,來入焉,下流為潯州府之右江,亦入於柳江。西北有安永鎮、西南有縣郭鎮二巡檢司。又東有東鄉、又有周沖、又有閑得三巡檢司,廢。

賓州元直隸廣西兩江道。洪武二年九月以州治領方縣省入。十月來屬。東南有鎮龍山。西南有燈臺山。西有古漏山,下有古漏關,古漏水出焉,入於賓水。賓水在南,即都泥江也。東有安城鎮巡檢司。又東有梁村巡檢司,後廢。北距府三百里。領縣二:

遷江州北。西有古黨山,有峒。東北有大江,即都泥江。東有遷江屯田千戶所,洪武二十五年九月置。東南有清水鎮巡檢司,又有羅目鎮、李廣鎮二巡檢司,廢。又東有石零堡,北有都歷堡,俱正德中築。

上林州西少北。西有大明山,澄江出焉,亦名南江,東合北江,又東入遷江縣之大江。西北有三里營,南丹衛在焉。衛舊在南丹州,洪武二十八年八月置,二十九年正月升軍民指揮使司,尋罷軍民,止為衛。永樂二年十二月徙上林縣東,正統六年五月徙賓州城,與賓州千戶所同治,萬曆八年徙於此。西南有周安堡,在八寨中,舊為瑤、僮所據,嘉靖三年討平之,萬歷七年改屬南丹衛。西北有三畔鎮巡檢司。又東北有琴水橋、東南有思龍鎮、又有三門灘鎮三巡檢司。

慶遠府元慶遠路。洪武元年為府。二年正月改慶遠南丹軍民安撫司。三年六月復曰慶遠府。領州四,縣五,長官司三。東北距布政司五百七十里。

宜山倚。北有龍江,東流入融縣,合於融江。西有河池守御千戶所,洪武二十八年十月置於河池縣,永樂六年徙於此。東有大曹鎮、西有懷遠鎮、又有德勝鎮、又有東江鎮四巡檢司。

天河府北少東。舊縣在高寨。洪武二年遷於蘭石。正統七年又遷甘場。嘉靖十三年又遷福祿鎮。萬歷十九年始移今治。西南有龍江,自貴州獨山州流入。北有東禪鎮巡檢司,又有思農鎮、歸仁鎮二土巡檢司。

忻城府南少東。西有烏泥江,即都泥江。北有三寨堡土巡檢司。

河池州元河池縣。弘治十七年五月升為州。縣舊治在州北懷德故城。天順六年遷屏風山。成化十三年還治懷德。嘉靖四年又遷鳳儀山南。西有智州山。東有金城江,下流合於都泥江。江北有金城鎮巡檢司。又東有都銘鎮、土堡鎮二巡檢司,後廢。東距府二百五十里。領縣二:

思恩州東北。舊屬府,正德元年二月改屬州。舊治在環江洲。永樂末,遷於清潭村。宣德三年十一月遷於白山寨。成化八年遷於歐家山。南有環江,北有帶溪,皆合流於龍江。有安化鎮、歸思鎮二巡檢司。又有普義鎮、吉安鎮、北蘭鎮三巡檢司,廢。

荔波州西北。洪武十七年九月析思恩縣地置,屬府。正統十二年改屬南丹州。成化十一年九月又屬府。正德元年來屬。州東南有勞村江,源出貴州陳蒙爛土長官司,流入州界,為金城江。又東有窮來、南有蒙石、又有方村三土巡檢司,後廢。

南丹州洪武七年七月置。二十八年廢,尋復置。西有孟英山,舊產銀。南有都泥江,自貴州定番州流入。東距府二百四十里。

東蘭州洪武十二年置。以西蘭州省入,又省安習、忠、文三州入焉。東南有隘洞江,一名都泥江,又名紅水河,又名烏泥江。東北距府四百二十里。

那地州元地州。洪武元年改置。北有都泥江,有布柳水流合焉。南有那州,洪武元年省。東北距府二百四十里。

永順長官司府西南。

永定長官司府南。二司皆弘治五年析宜山縣地置。

永安長官司弘治九年九月析天河縣十八里地置。

南寧府元南寧路。洪武元年為府。領州七,縣三。東北距布政司千二百里。

宣化倚。東有昆崙山,上有崑崙關。又有橫山,又有思玉山。北有馬退山。東南有望仙坡,與青、羅二山相對。城西南有大江,即鬱江,一曰夜郎豚水。其上流有二:一為南盤江,經府城南,曰右江;一為麗江,經府城西南,曰左江。合流處謂之合江鎮,下流為潯州府之左江。東有金城寨、西有那南寨、又有那龍寨、又有遷隆寨、南有八尺寨五巡檢司。

隆安府西北。嘉靖十二年四月析宣化縣那久地置。東有火焰山。城北有盤江,亦曰右江。西南有那樓寨、西北有馱演寨二巡檢司。

橫州元直隸廣西兩江道。洪武二年九月以州治寧浦縣省入,屬潯州府。十年五月降為縣,來屬。十三年十一月復為州,仍置寧浦縣為州治,縣尋廢。東有烏蠻山。南有鬱江,又東南有武流江,源自廣東靈山縣,流入境合焉。東有古江口、西有南鄉二巡檢司。又南有太平關,成化四年置。西北距府二百四十里。領縣一:

永淳州西。元屬橫州。洪武十年五月省入橫縣。十三年十一月復置,屬州。西臨鬱江。南有南里鄉、北有武羅鄉二巡檢司。又東北有修德鄉巡檢司,景泰間遷於縣西,後廢。

新寧州隆慶六年二月以宣化縣定祿洞地置。北有三峰山。城西有麗江,一名定祿江,又名文字水。東南有渠樂寨巡檢司。東距府二百里。

上思州元屬思明路。洪武初廢。二十一年正月復置,屬思明府。弘治十八年來屬。南有十萬山,上思江出焉,東流合西小江,西即交址所出之左江也。又有明江,亦出十萬山,西流入思明府界。又西有遷隆峒土巡檢司。東南距府三百里。

歸德州元屬田州路。洪武二年屬田州府。弘治十八年來屬。鬱江在西南。東南距府三百五十里。

果化州元屬田州路。洪武二年屬田州府。嘉靖九年十二月來屬。南盤江在西。東南距府三百二十里。

忠州元屬思明路。洪武初廢。二十一年正月復置,屬思明府。萬歷三年九月來屬。東北距府四百餘里。

下雷州元下雷峒。洪武初,屬鎮安府。嘉靖四十三年來屬。萬曆十八年升為州。南有邏水,自鎮安府流入,南寧府左江之別源也。東距府五百八十里。

思恩軍民府元思恩州,屬田州路。洪武二年屬田州府,後屬雲南廣西府。永樂二年八月直隸廣西布政司。正統四年十月升為府。六年十一月升軍民府。舊治在府西北。正統七年遷府東北之喬利。嘉靖七年七月又遷武緣縣止戈里之荒田驛,因割止戈二里屬之。西北有都陽山。東南有靖遠峰。北有紅水江,又有馱蒙江,一名清水江,流合焉。又有大攬江,出城東北大名山,下流俱入於鬱江。東有鳳化縣,正德六年七月置,嘉靖八年十月廢。東有古零,西有定羅、那馬、下旺,北有興隆,東北有白山、安定,西北有舊城、都陽九土巡檢司。領州二,縣二。東北距布政司千二百里。

奉議州元直隸廣西兩江道。洪武五年省入來安府。七年二月復置,直隸行省。二十八年復廢,尋復置,直隸布政司。嘉靖六年二月來屬。東有舊城。今治本砦林村也,洪武初,遷於此。北濱南盤江,有州門渡。距府百十里。

上映州元屬鎮安路。洪武五年廢為洞。萬歷三十二年復置,來屬。東北距府四百七十里。

上林府西南。元屬田州路。洪武二年屬田州府。嘉靖七年七月來屬。北有南盤江,南有大羅溪,東流合焉,即枯榕江之下流也。

武緣府南。元屬南寧路。萬曆五年十月來屬。西有西江,即大欖江也,東南有南流江合焉。東有鏌鎁寨、又有博澀寨、西有高井寨、西北有西舍寨四巡檢司。又南有橫山寨巡檢司,廢。

太平府元太平路,至元二十九年閏六月置。洪武二年七月為府。領州十七,縣三。東北距布政司二千五十里。

崇善倚。府治馱盧村,洪武二年徙治麗江。舊縣治在府西北,嘉靖十九年遷入郭內。北有青連山。東有將軍山,下有威震關,一名伏波關。南有府前江,即麗江,又西有邏水流入焉。北有壺關,正德三年置。又東北有保障關。

陀陵府東北。東有淥空山,淥空江出焉,亦名綠甕江。又南有麗江。

羅陽府東北。南有麗江。西有馱排江,源出永康縣,下流入於麗江。以上三縣,元俱屬太平路。

左州東有舊治。成化十三年遷於思崖村。正德十五年遷於今治,本古攬村也。西北有金山。南有麗江。西南距府百里。

養利州有舊州三,一在州北,一在西北,一在東北。又西北有養水。北有通利江,至崇善縣注於麗江。以上二州,元屬太平路。南距府百五十里。

永康州元永康縣,屬太平路。萬曆二十八年六月升為州。北有故城。萬歷中遷於今治。西有綠甕江,下流亦合麗江焉。西南有思同州,舊屬府。萬歷二十八年六月省。西南距府二百里。

上石西州元屬思明路。洪武末省。永樂二年復置。萬曆三十八年來屬。東有明江,西北流入麗江。東北距府三百三十里。

太平州自此以下十一州,元屬太平路。邏水在西,下流入麗江。東南距府八十里。

思城州南有教水,下流合於隴水。東南距府五百里。

安平州南有隴水,下流合於邏水。東南距府百十里。

萬承州西南有綠降水,亦名玉帶水。西南距府五十里。

全茗州西有通利江,一名大利江。南距府百六十里。

鎮遠州北有楊山。南有巖磨水。西南距府二百八十里。

茗盈州南有觀音巖,澗水出焉,下流入於麗江。西南距府六十里。

龍英州南有通利江,有三源,下流入於麗江。南距府二百十里。

結安州西有堰水,下流入麗江。西南距府二百二十里。

結倫州南有咘畢水。即堰水之上流。西南距府三百三十里。

都結州南有咘畢水。西南距府三百三十里。

上下凍州元屬龍州萬戶府。洪武初來屬。西有八峰山,太源水出焉。又北有青連山。南有拱天嶺。東距府二百二十里。

思明州元屬思明路。洪武二年屬思明府。萬歷十六年三月來屬。東有逐象山。東北有明江,自思明府流入。東北距府二百十里。

思明府元思明路。洪武二年七月為府,直隸行省。九年直隸布政司。南有明江,有永平寨巡檢司。領州三。北距布政司二千二百里。

下石西州元屬思明路。洪武二年屬府。舊治在東南。萬歷間,始遷今治。西距府百四十里。

西平州元屬思明路。洪武三年省。永樂二年復置。宣德元年與安南。

祿州元屬思明府。洪武三年省。二十一年正月復置,尋沒於交址。永樂三年收復。宣德元年與安南。

鎮安府元鎮安路。洪武二年為府。西有鎮安舊城。洪武二年徙於廢凍州,即今治也。南有馱命江,下流合鬱江。又有邏水,發源府北土山峽中,下流至胡潤寨,與歸順州之邏水合,有湖潤寨巡檢司。距布政司二千二百里。

田州元田州路。洪武二年七月為府。嘉靖七年六月降為州,徙治八甲,而置田寧府於府城。八年十月,府廢,州復還故治,直隸布政司。東南有南盤江。西有來安路,元屬廣西兩江道,洪武二年七月為府,領歸仁州、羅博州、田州,十七年復廢。北有上隆州,元屬田州路,洪武二年屬府,成化三年徙治潯州府東北,更名武靖州。又有恩城州,元屬路,洪武初屬府,弘治五年廢。東有床甲、拱甲、婪鳳,西有武隆、累彩,北有岊馬甲、篆甲,東北有下隆,東南有砦桑,西北有凌時,西南有萬岡陽院,又有大甲、子甲,又有縣甲、怕河、怕牙、思郎、思幼、候周十九土巡檢司。距布政司千六百里。

歸順州元屬鎮安路。洪武初,廢為洞。弘治九年八月復置,屬鎮安府。嘉靖初,直隸布政司。東北有龍潭水,南入交址高平府界。又南有邏水,發源西北鵝槽隘界。距布政司二千三百二十里。

泗城州元屬田州路。洪武七年直隸行中書省。九年直隸布政司。舊州在西南,洪武六年移於古勘洞。西有南盤江,自貴州慕役長官司流入,下流為南寧府之右江。又北有紅水江。東北有程縣,洪武二十一年以泗城州之程丑莊置,屬州,尋屬慶遠府,宣德初,還屬州,嘉靖元年廢。西南有利州,元屬田州路,洪武七年十一月直隸布政司,正統六年五月徙治泗城州古那甲,嘉靖二年廢。又西有上林長官司,永樂七年以州之上林洞置,直隸布政司,萬歷中,省入州,崇禎六年分司西地入雲南廣南府。有羅博關巡檢司。北距布政司一千八百一十五里。

向武州元屬田州路。洪武二年七月屬田州府。二十八年廢。建文二年復置,直隸布政司。舊州在東。萬歷四十五年遷於乃甲。南有枯榕江,下流入於右江。北有富勞縣,元屬田州路,洪武二年屬田州府,尋為夷僚所據,建文四年復置,後廢。東有武林縣,元亦屬田州路,洪武二年屬田州府,永樂初省入富勞縣。距布政司二千四百里。

都康州元屬田州路。洪武二年屬田州府,後為夷僚所據。建文元年復置,直隸布政司。西有岊爐江,下流合於通利江。距布政司二千五百四十里。

龍州元龍州萬戶府。洪武二年七月仍為州,屬太平府。九年六月直隸布政司。南有龍江,自交址廣源州流入,即麗江也,有明江流入焉,下流為南寧府之左江。距布政司二千三百里。

江州元屬思明路。洪武二十年直隸布政司。東有歸安水,西有綠眉水,下流俱合於麗江。領縣一。距布政司二千一百十里。

羅白州東北。洪武三年置,屬思明府,後來屬。南有隴冬水,下流入於麗江。

思陵州元屬思明路。洪武三年省入思明府。二十一年正月復置,直隸布政司。南有角硬山,角硬水出焉,又有淰削水合之,下流入思明府界。距布政司二千一百二十里。

憑祥州本憑祥縣。永樂二年五月以思明府之憑祥鎮置,屬思明府。成化十八年升為州,直隸布政司。西北有麗江,自交址廣源州流入。又南有鎮南關,一名大南關,即界首關也。距布政司二千四十里。

安隆長官司元致和元年三月置安隆州,屬雲南行省。後廢為寨,屬泗城州。洪武三十五年十二月置安隆長官司,仍屬泗城州,後直隸布政司。西有壩達山,渾水河經其下,即紅水江也,東入泗城州界。又西南有同舍河。距布政司里。

○雲南貴州

雲南《禹貢》梁州徼外。元置雲南等處行中書省。治中慶路。洪武十五年二月癸丑平雲南,置雲南都指揮使司。乙卯置雲南等處承宣布政使司。同治雲南府。領府五十八,州七十五,縣五十五,蠻部六。後領府十九,禦夷府二,州四十,禦夷州三,縣三十,宣慰司八,宣撫司四,安撫司五,長官司三十三,禦夷長官司二。北至永寧,與四川界。東至福州,與廣西界。西至乾崖,與西番界。南至木邦,與交址界。距南京七千二百里,京師一萬六百四十五里。洪武二十六年編戶五萬九千五百七十六,口二十五萬九千二百七十。弘治四年,戶一萬五千九百五十,口一十二萬五千九百五十五。萬曆六年,戶一十三萬五千五百六十,口一百四十七萬六千六百九十二。

雲南府元中慶路。洪武十五年正月改為雲南府。領州四,縣九:

昆明倚。洪武二十六年,岷王府自陜西岷州遷於此。永樂二十二年遷岷王府於湖廣武岡州,建滕王府於此,宣德元年除。東有金馬山,與西南碧雞山相對,俱有關,山下即滇池。池在城南,周五百里,其西南為海口,至武定府北,注於金沙江。又東有盤龍江,西注滇池。東有赤水鵬、清水江二巡檢司。

富民府西北。東有螳良川,源自滇池,下流入金沙江。東南有安寧河。

宜良府東少南。東有大池江,一名大河,亦曰巴盤江。西有湯池巡檢司。

羅次府西北。舊屬安寧州,弘治十三年八月改屬府。西有星宿河,自武定府流入。又有沙摩溪,即安寧河。南有鍊象關巡檢司。

晉寧州西有大堡河,下流入滇池。北距府百里。領縣二:

歸化東北有交七浦,滇池下流。

呈貢州北。西有滇池,北有落龍河,南流入焉。

安寧州西有呀嵕山,有煎鹽水,設鹽課提舉司,轄鹽井四。天啟三年改設於瑯井,此司遂廢。又南有螳良川。西有安寧河。又有祿脿、貼琉二巡檢司。東距府八十里。領縣一:

祿豐州西。西有南平山,上有關。東有大溪,即安寧河。西有星宿河,河東有老鴉關巡檢司。又西有蘭谷關。

昆陽州東南有渠濫川,東北入於滇池。北距府百五十里。領縣二:

三泊州西北。西有三泊溪,流入滇池。

易門州西。南有易門守御千戶所,洪武二十四年置,舊縣治在焉。萬曆三年復還縣治於此。又南有黎崖山,產異馬,一名馬頭山。西有九渡河,即祿豐縣大溪,下流入元江府界。

嵩明州洪武十五年三月改曰嵩盟。成化十八年復故。東北有羅錦山。東有秀嵩山。西北有東葛勒山。東南有烏納山,牧漾水出焉,西南入滇池。又東南有嘉利澤,亦曰楊林澤。又西有邵甸河,匯九十九泉,至昆明為備龍江。西有邵甸縣,洪武十五年三月屬州,尋廢。東南有楊林縣,成化十七年十月廢。又東有楊林守禦千戶所,洪武二十五年置。又西有兔兒關巡檢司。西南距府百二十里。

曲靖府元曲靖路。洪武十五年三月為府。二十七年四月升為軍民府。領州四,縣二。西距布政司二百九十里。

南寧倚。東南有石堡山,山西有元越州治,洪武二十八年正月廢。北有白石江,流合城南之瀟湘江,又東南合左小江,亦謂之南盤江,下流環雲南、澂江、廣西三府之境,至羅平州入貴州界。東北有白水關巡檢司。

亦佐府東。元屬羅雄州。永樂初,改屬府。西南有塊澤江。

霑益州東南有堆湧山。北有北盤江,其上流即貴州畢節衛之可渡河,流入州境,又東南入貴州安南衛。其西南又有南盤江,即南寧縣之東山河。南有交水縣,東南有羅山縣,東北有石梁縣,元皆屬州,洪武十五年皆廢。南有平夷衛,本平夷千戶所,洪武二十一年十一月置,二十三年四月改為衛,後廢,永樂元年復置衛。衛當貴州西入之衝,東有巒岡,西有定南嶺,北有豫順關、宣威關。州東南又有越州衛,洪武二十三年七月置,二十四年十二月徙於陸涼州,二十八年與州同廢,永樂元年九月復置。又州南有松韶鋪、阿幢橋二巡檢司。又南有炎方城,西南有松株城,俱天啟五年築。西南距府二百十三里。

陸涼州東有丘雄山,下有中涎澤,即南盤江所匯也。西北有木容山,有關。又西有部封山。又西有芳華縣,南有河納縣,元皆屬州,永樂初皆廢。西南有陸涼衛,洪武二十三年三月以古魯昌地置,西南有喬甸,萬曆二年立營置戍於此。四十八年復設法古甸、龍峒等營,協守其地。北距府百二十里。

馬龍州東南有木容箐山,洪武二十四年十二月置寧越堡於此。山下有木容溪,下流即瀟湘江。又西有楊磨山,一名關索嶺,上有關。西南有通泉縣,元屬州,永樂初廢。北有馬隆守禦千戶所,本馬隆衛,洪武二十三年七月置,二十八年十月改為所。南有魯婆伽嶺巡檢司。又有馬龍縣,元屬州,洪武十五年廢。西南有分水嶺關。東有三叉口關。東距府七十里。

羅平州元羅雄州。萬曆十五年四月更名。北有祿布山。東南有盤江,下流入貴州慕役長官司界。南有定雄守禦千戶所,萬曆十四年九月置。西北距府二百七十里。

尋甸府元仁德府。洪武十六年十月辛未升為仁德軍民府。丁丑改尋甸軍民府。成化十二年改為尋甸府。舊治在東。今治在鳳梧山下,嘉靖七年十月徙。西南有落隴雄山,又有哇山。西有果馬山,其泉流為龍巨江,下流入滇池。又西南有三棱山,上有九十九泉,即盤龍江之上源。又東有阿交合溪。又北有為美縣,西有歸厚縣,元屬府,洪武十五年三月因之,尋廢。東南有木密關,一名易龍堡,洪武二十三年四月置木密關守禦千戶所於此。西南距布政司二百六十里。

臨安府元臨安路。洪武十五年正月為府。領州六,縣五,長官司九。北距布政司四百二十里。

建水州倚。元時府在州北,洪武中移府治此。西南有寶山。西北右有火燄山。東有石巖山,瀘江水自石屏州流經此,伏流入巖洞中,東出為樂蒙河。又東北有曲江,東入於盤江,有曲江巡檢司。又西有禮社江,源出趙州,流經此。又有寧遠州,萬曆十四年析建水州置,四十八年廢。東南有納更山土巡檢司。

石屏州元曰石坪。洪武十五年三月改曰石平,後改今名。南有鐘秀山。東有菜玉山,產石似玉。有曲江。又有異龍湖,周百五十里,中有大、小、中三島,其大島、中島上皆有城,其水引流為瀘江。西有寶秀關巡檢司。東距府七十里。

阿迷州元阿甯萬戶。洪武十五年三月置州。東南有買吾山,萬曆初,改名雷公山。又南有盤江,東有樂蒙河流入焉。又東有火井,有東山口土巡檢司。又有部舊村巡檢司,後廢。又有阿迷守禦城,萬歷二年築。西距府百二十里。

寧州東南有登樓山。東有水角甸山,產蘆甘石。又東有婆兮江,源出澂江府撫仙湖,下流入盤江,又西南有浣江,流合焉。又東有西沙縣,元屬州,後省,洪武十五年三月復置,仍屬州,尋復省。西北有甸直巡檢司。西南距府百八十里。

通海府西北。元屬寧州,洪武十五年三月改屬府。南有秀山。北有通海湖。東有守禦通海前前、右右二千戶所,本元臨安路治。洪武初,徙府治建水州。十五年置守禦千戶所於此。

河西府西北。東有曲江。又西有祿卑江,自新興州流入,合於曲江。又東北有綠溪河,其下流即通海湖。又北有曲陀關巡檢司,後廢。

嶍峨府西北。元屬寧州。洪武十五年二月改屬府。東有曲江,自新興州流入,又南有合流江,西北有丁癸江,俱流合焉。又西南有伽羅關、西有興衣鄉二巡檢司。

蒙自府東南。西有目則山。東有雲龍山,又有羨裒山。又東南有黎花江,即禮社江也,東南注於交址清水江。有黎花舊市柵,宣德五年五月置臨安衛右千戶所於此。又西南有西溪二,出銀礦。又南有蓮花灘,即瀾滄江下流,交址洮江上流。西南有箐口關巡檢司,又有大窩關、楊柳河關。東南有廢果寨,又有賀謎寨,俱道通交址。

新平府西北。萬歷十九年置。東南有魯奎山。東有平甸河。南有南峒巡檢司。

新化州本馬龍他郎甸長官司。洪武十七年四月置,直隸布政司。弘治八年改為新化州。萬曆十九年來屬。北有徹崇山。西有馬籠山,蠻酋結寨處,元置馬籠部千戶於此,屬元江路,洪武十五年廢。又北有法龍山,亦蠻酋結寨處。又東南有馬籠江,即禮社江,亦曰摩沙勒江,有摩沙勒巡檢司。東北有阿怒甸。東南距府五百三十里。

寧遠州元至治三年二月置,直隸雲南行省。洪武十五年來屬。宣德元年與安南。

納樓茶甸長官司府西南。本納樓千戶所,洪武十五年置,屬和泥府。十七年四月改置。北有羚羊洞,產銀礦。又有祿豐江,即禮社江下流。又東有倘甸。

教化三部長官司府東南。元強現三部,洪武中改置。西南有魯部河,源出禮社江,下流合蒙自縣梨花江。

王弄山長官司府東南。元王弄山大小二部,洪武中改置。

虧容甸長官司府西南。元鐵容甸,屬元江路。洪武中改置,來屬。西有虧容江,源出沅江府,東經車人寨,出寧遠州境。

溪處甸長官司府西南。元溪處甸軍民副萬戶,屬元江路。洪武中改置,來屬。

思佗甸長官司府西南。元和泥路。洪武十五年三月為府,領納樓千戶所伴溪、七溪、阿撒三蠻部,十七年廢,後改置。

左能寨長官司府西南。本思佗甸寨,洪武中改置。

落恐甸長官司府西南。元伴溪落恐部軍民萬戶。洪武中改置。

安南長官司府東南。元捨資千戶,後改安南道防送軍千戶。洪武十五年三月仍曰捨資千戶所,尋改置長官司。正德六年省入蒙自縣。天啟二年復置。

澂江府元澂江路。洪武十五年三月府。領州二,縣三。西北距布政司八十里。

河陽倚。舊治在西。洪武中,遷繡球山上。弘治中,又遷縣東金蓮山。正德十三年又遷縣東暘溥山麓。嘉靖二十年又遷金蓮山南。隆慶四年又遷舞鳳山下,即今治。北有羅藏山。南有撫仙湖,一名羅伽湖,下流東會於盤江。又東有鐵池河,源出陸涼州,流至此,會撫仙湖,復引流為鐵赤河,入於盤江。

江川府西南。南有故城,崇禎七年圮於水,遷於舊江川驛,即今治。又南有星雲湖,東南入撫仙湖。北有關索嶺巡檢司。

陽宗府東北。北有明湖,一名陽宗湖,源出羅藏山,流入於盤江。

新興州東北有羅麼山,一名石崖山。西北有大棋山。又有蒙習山,山與晉寧州交界。又有大溪,下流至嶍峨縣,入於曲江。有羅麼溪,源出羅麼山,入於大溪。又北有普舍縣,南有研和縣,元俱屬州,洪武十五年三月因之,尋廢。又北有鐵爐關巡檢司。東距府二百里。

路南州西南有竹子山。東有答刂龍山,石可煉銅。西有巴盤江,源自陸涼州。又有鐵赤河合焉。東南有邑市縣,元屬州,弘治三年九月廢。東北有革泥巡檢司。西距府百三十里。

廣西府元廣西路。洪武十五年三月為府。西有阿盧山。西北有巴盤江。又西有南盤江。又南有矣邦池,一名龍甸海,跨彌勒州界,南入盤江。領州三。西北距布政司三百十里。

師宗州西有龜山,萬歷四十八年築督捕城於此。東有英武山。西有盤江,又西北有巴盤江合焉,東北入羅平州界。西南距府八十里。

彌勒州南有卜龍山。西有阿欲山。東南有盤江山,南盤江經其下。又東有八甸溪,南合南盤江。又西有十八寨山,嘉靖元年二月置十八寨守禦千戶所於此,直隸雲南都司。又南有捏招巡檢司。東北距府九十里。

維摩州元大德四年二月置。東北有小維摩山。東南有大維摩山,又有阿母山。又東北有寶寧溪,下流經廣南府界,合西洋江。西有三鄉城,萬曆二十二年築。西北距府二百二十里。

廣南府元廣南西路宣撫司。洪武十五年十一月改置廣南府。西北有牌頭山,土人築砦其上。南有西洋江,東南至廣西田州府,入於左江。領州一。西北距布政司七百九十里。

富州元至元十三年置,屬廣南西路。洪武十五年改屬府。東南有者鷂山。東北有西寧山。又東有楠木溪,至州南與南汪溪合,伏流十五里,東出於西洋江。西南有安寧州,東北有羅佐州,俱元至元十三年置,屬廣南西路。洪武十五年因之,後俱廢。西距府二百里。

元江軍民府元元江路。洪武十五年三月為府。永樂初,升軍民府。領州二。東北距布政司七百九十里。

奉化州倚。本因遠羅必甸長官司,洪武十八年四月置。嘉靖中,改州。東有羅盤山,亦名玉臺山。又有路通山。東南有元江,亦曰禮社江,東南入納樓茶甸長官司界。西南有瀾滄江,與車里宣慰司分界。又西有步日部,洪武中廢。又東有禾摩村巡檢司。

恭順州本他郎寨長官司,嘉靖中改州。

楚雄府元威楚開南路。洪武十五年三月改為楚雄府。領州二,縣五。東距布政司六百里。

楚雄倚。元曰威楚。洪武十五年二月更名。西有薇溪山,又有龍川江,經城北青峰下,曰峨𡸮江,下流入武定府,合金沙江。西有波羅澗,其麓有滷水,元設鹽課司於此,明廢。西北有呂合巡檢司。

廣通府東。元屬南安州。洪武十五年因之,後改屬府。東北有盤龍山,亦曰九盤山。西有羅苴甸山。東有鹽倉山,舊產鹽。又有臥象山,東南有臥獅山,俱產銀礦。又東北有阿陋雄山,有阿陋井、猴井,俱產鹽。又東有捨資河,自武定府流入,下流入於元江。又北有大河,西北入定遠縣之龍川江。東有捨資巡檢司,東北有沙矣舊,西有回蹬關二土巡檢司。

定遠府西北。西有赤石山。東有龍川江。又有黑鹽井,設提舉於此。又有瑯井提舉司,本置於安寧州,天啟三年移此,有黑井、瑯井二巡檢司。又西南有羅平關、南有會基關二巡檢司。

定邊府西。元至元十二年置,屬鎮南州。洪武中,改屬。北有螺盤山,上有自普關。又有無量山。南有定邊河,又有陽江,自蒙化府流合焉。

柷嘉府南。元置。西有黑初山。東北有卜門河,在卜門山下,又東北合馬龍江,流入新化州。又西有上江河,接南安州界。

南安州東有健林蒼山。又西南有表羅山,產銀。北有捨資河。西北距府五十里。

鎮南州東北有石吠山。東有五樓山。西南有馬龍江,其上流為定邊河,又東南入柷嘉縣界。又西有平夷川,龍川江之上流。又有沙橋巡檢司。又有鎮南關、英武關、阿雄關三土巡檢司。東南距府五十里。

姚安軍民府元姚安路。洪武十五年三月為府。二十七年四月升軍民府。領州一,縣一。東南距布政司七百里。

姚州倚。元屬大理路。洪武十五年三月來屬。東有東山,一名飽煙蘿山。東北有金沙江。南有青蛉河,源出三窠山,下流合大姚河。北有守禦姚安千戶所,洪武二十八年置。東有箭場、西有普昌、南有三窠、西南有普淜四巡檢司。

大姚府北。元屬姚州。洪武十五年三月因之,後改屬府。西北有赤石崖。北有大姚河,源出書案山。西北有龍蛟江,源出鐵索箐,一名苴泡江,產金。俱東北流入金沙江。南有白鹽井提舉司,轄鹽井九。又有白鹽井巡檢司。東有姚安中屯千戶所,洪武二十八年置。

武定府元武定路。洪武十五年三月為府,尋升軍民府。隆慶三年閏六月徙治獅子山。萬曆中,罷稱軍民。領州二,縣一。東南距布政司百五十里。

和曲州倚。舊城在南,元州治於此。隆慶三年十二月徙州為府附郭,令吏目領兵守焉。西北有三臺山。北有金沙江,源出吐蕃共龍川犁牛石,下流經麗江、鶴慶二府,至本府北界,東流入黎溪州,又東入四川會川衛界。有金沙江土巡檢司。又有烏龍河,流入金沙江。又西北有西溪河,即楚雄府龍川江下流。又有只舊、草起二鹽井。東有南甸縣,元路治,洪武十五年三月改屬州,成化二十年仍屬府,正德元年七月省。西北有乾海子、又有羅摩洱、又南有小甸關三巡檢司。西北有龍街關土巡檢司。

元謀府西北。西北有住雄山,又有竹沙雄山。北有金沙江,西有西溪河流入焉。

祿勸州北有法塊山,又有哇匿歪山。東北有幸丘山,又有烏蒙山,一名絳雲露山。北有金沙江,與四川東川府界。又東有普渡河,即螳良川,下流會掌鳩河水,入於金沙江。北有易籠縣,元屬州,洪武十七年省。東有石舊縣,元屬州,天啟元年七月省。又北有普渡河巡檢司。南有撒墨巡檢司,後廢。西距府二十里。

景東府元至順二年二月置。洪武十五年閏二月因之。三月降為州,屬楚雄府。十七年正月仍升為府。西有景董山,洪武中築景東衛城於其上,又築小城於山顛,謂之月城。北有蒙落山,一名無量山。西南有瀾滄江,源出金齒,流經府西南二百餘裏,南注車里,為九龍江,下流入交址,東南有大河,即定邊河之下流,又東入鎮南州,為馬龍江。又南有土井,產鹽。北有開南州,元屬威楚開南路。洪武十五年三月屬楚雄府,尋省。又東有三汊河、西北有保甸二土巡檢司。又北有安定關。南有母瓜關。東南有景蘭關。西南有蘭津橋,鐵索為之。東北距布政司千一百八十里。

鎮沅府本鎮沅州。洪武三十五年十二月置。永樂四年四月升為府。西有波弄山,山上下有鹽井六。南有杉木江,源出者樂甸,下流合威遠州之谷寶江。領長官司一。北距布政司千五十里。

祿谷寨長官司府東北。永樂十年四月以祿平寨置。北有馬容山。南有南浪江,西南流合杉木江。

大理府元大理路。洪武十五年三月為府。領州四,縣三,長官司一。東南距布政司八百九十里。

太和倚。西有點蒼山。東有西洱河,一名洱海,自浪穹縣流入,經天橋下,又東合點蒼山之十八川匯於此,中有三島、四洲、九曲。西有樣備江,一曰漾鼻水,自劍川州流入,經點蒼山後,合於西洱河,又西南流入瀾滄江。南有太和土巡檢司。又北有龍首關,亦曰上關。南有龍尾關,亦曰下關。

趙州洪武十五年三月改名趙喜州,尋復。南有九龍頂山。又有定西嶺,大江之源出焉,一名波羅江,西北入西洱河。又西南有樣備江,南入蒙化府界。東南有白崖江,源出定西嶺,下流為禮社江。有舊白崖城,嘉靖四十三年修築,更名彩雲城。又東有乾海子、南有迷度市二巡檢司。又有定西嶺上巡檢司。西北距府三十里。領縣一:

雲南州東。元雲南州。洪武十五年三月改為縣,屬府。十七年改屬州。西北有寶泉水,有一泡江。東北有周官些海子。西有品甸,洪武十九年四月置洱海衛於此。又東北有你場、又有楚甸、南有安南坡三巡檢司。

鄧川州北有鐘山,又有普陀江,一名蒲萄江,又名彌苴佉江,南入西洱河。又東有豪豬洞,一名銀坑。又有青索鼻土巡檢司。南距府七十里。領縣一:

浪穹州東。東北有佛光山,山半有洞,可容萬人,山後險仄,名一女關。又有蓮花山,有蒙次和山,皆險峻。西南有鳳羽山。北有罷谷山,洱水所出。西有樣備江。西北有寧湖,亦曰明河,即普陀江上源。又有五鹽井提舉司,洪武十六年置,萬歷四十二年廢。西南有鳳羽縣,洪武十五年三月置,屬鄧川州,尋省。有鳳羽鄉巡檢司。又東南有晉陀崆巡檢司,後廢。西有上江嘴、西南有下江嘴二土巡檢司。

賓川州弘治六年四月析趙州及太和、雲南二縣地置。西有雞足山,一名九曲巖。東北有金沙江,東入姚安府界。西有金龍湫,流入西洱河。又東有大羅衛,在鐘英山下,弘治六年四月與州同置。又東北有赤石崖、西南有賓居二巡檢司。西有神摩洞。又南有蔓神寨、北有白羊市二巡檢司,後廢。又北有金沙江土巡檢司。西距府百里。

雲龍州元云龍甸軍民府,至元末置。洪武十七年改為州,來屬。正統間屬蒙化府,後仍來屬。西有三峰山。東有瀾滄江。又西北有諾鄧等鹽井,東南有大井等鹽井,舊俱轄於五井提舉司,後改屬州。東有雲龍甸巡檢司,後廢。東北有順盪井、又有上五井、東有師井、北有箭捍場四巡檢司,又東有十二關土巡檢司,舊俱屬浪穹縣,後改屬。東南距府六十里。

十二關長官司府東。元十二關防送千戶所。洪武中改置。嘉靖元年五月徙於一泡江之西。

鶴慶軍民府元鶴慶路。洪武十五年三月為府。三十年十一月升軍民府。南有方丈山,又有半子山,產礦。東有金沙江。東南有漾共江,即鶴川,其下流入金沙江。有木按州,又有副州,元俱屬府,洪武十五年俱廢。東北有宣化關、西南有觀音山、又有清水江三巡檢司。領州二。東南距布政司千一百六十里。

劍川州元劍川縣。洪武十五年三月因之。十七年正月升為州。西南有石寶山。南有劍川湖,俗呼海子,樣備江之下流。又西南有彌沙井鹽課司。又有彌沙井巡檢司。東距府九十里。

順州元屬麗江路。洪武十五年三月屬北勝府,尋來屬。西有金沙江。東有浴海浦,與北勝州分界。西距府百二十里。

麗江軍民府元麗江路宣撫司。洪武十五年三月為府。三十年十一月升軍民府。領州四。東南距布政司千二百四十里。

通安州倚。西北有玉龍山,一名雪嶺。又有金沙江,古名麗水,源出吐蕃界犁牛石下,名犁水,「犁」訛「麗」,流經巨津、寶山二州,至武定府,北流入四川大江。西有石門關巡檢司。

寶山州西南有阿那山。南有金沙江。西距府二百四十里。

蘭州元屬麗江路。洪武十五年三月屬麗江府,尋屬鶴慶府,後仍來屬。北有福源山。西北有瀾滄江,源出吐蕃嵯和歌甸,流入境,南入雲龍州界。東北距府三百六十里。

巨津州南有華馬山。北有金沙江,流入州界,有鐵橋跨其上。西北有臨西縣,元屬州,洪武十五年三月因之,弘治後廢。又東北有雪山關。東南距府三百里。

永寧府元永寧州,屬麗江路。洪武十五年三月屬北勝府。十七年屬鶴慶府。二十九年改屬瀾滄衛。永樂四年四月升為府。金沙江在西。又東有瀘沽湖,周三百里,中有三島。又東南有魯窟海子,在乾木山下,下流入四川鹽井衛之打沖河。又北有勒汲河,自吐蕃流入,亦東流入打沖河。又南有羅易江,自蒗蕖州流入,注於瀘沽湖。領長官司四。東南距布政司千四百五十里。

剌次和長官司府東北、革甸長官司府西北、香羅甸長官司府西、瓦魯之長官司府北。四司,俱永樂四年四月置。

北勝州元北勝府,屬麗江路。洪武十五年三月屬布政司,尋降為州,屬鶴慶府。二十九年改屬瀾滄衛。正統七年九月直隸布政司。弘治九年徙治瀾滄衛城。瀾滄衛舊在州南,本瀾滄衛軍民指揮使司,洪武二十八年九月置,屬都司。弘治九年徙州來同治。尋罷軍民司,止為衛。西南有瀾滄山。南有九龍山。西有金沙江,環繞州治,亦曰麗江。又南有陳海,又有呈湖,東南有浪峨海,下流俱入金沙江。東有羅易江,下流入永寧府界。北有蒗蕖州,元屬麗江路,洪武十五年三月屬北勝府,尋屬鶴慶軍民府,二十九年改屬瀾滄衛,天啟中廢。東有寧番土巡檢司。南距布政司千二十五里。

永昌軍民府元永昌府,屬大理路。洪武十五年三月屬布政司。十八年二月兼置金齒衛,屬都司。二十三年十二月省府,升衛為金齒軍民指揮使司。嘉靖元年十月罷軍民司,止為衛,復置永昌軍民府。領州一,縣二,安撫司四,長官司三。東距布政司千二百里。

保山倚。本金齒千戶所,洪武中置。永樂元年九月又置永昌府守禦千戶所,俱屬金齒軍民司。嘉靖三年三月改二所為保山縣,東有哀牢山,本名安樂,夷語哀牢。西有九隆山。又東北有羅岷山,瀾滄江經其麓。又南有潞江,舊名怒江,一名喳里江,自潞江司流入。又北有清水河,經縣東南峽口山下,伏流東出,入瀾滄江。又有潞江州,宣德八年六月置,直隸布政司,正統二年五月廢。又東北有沙木和、西北有清水關二巡檢司。又北有甸頭、南有水眼二土巡檢司。

永平府東北。元屬永昌府。洪武二十三年屬金齒軍民司。嘉靖元年仍屬府。西南有博南山,一名金浪巔山,俗訛為丁當丁山,上有關。又有花橋山,產鐵礦。又東北有橫嶺山,驛道所經。東有銀龍江,下流入瀾滄江。又東北有勝備江,下流入蒙化府樣備江。又西南有花橋河,源出博南山,流入銀龍江,上有花橋關,亦曰玉龍關。又東北有上甸定夷關巡檢司。東有打牛坪土巡檢司。

騰越州元騰衝府,屬大理路。洪武十五年三月屬布政司,尋廢。永樂元年九月置騰衝守禦千戶所,屬金齒軍民司。宣德六年八月直隸都司。正統十年三月升所為騰衝軍民指揮使司。嘉靖三年十月置騰越州,屬府。十年十二月罷司為騰衝衛。東有球牟山。東南有羅生山。南有羅佐衝山,上有鎮夷關,有巡檢司。又東北有高黎共山,一名崑侖岡。西北有明光山,有銀礦銅礦。西有大盈江,亦曰大車江,自徼外流入,下流至比蘇蠻界,注於金沙江。又東北有龍川江,源出徼外蛾昌蠻地之七藏甸,下流合於大盈江,有藤橋在其上。有龍川江關巡檢司。又西南有疊水河,即大盈江之支流。又有騰衝土州,宣德五年六月置,屬金齒軍民司,後直隸布政司,正統三年五月仍屬金齒軍民司,尋廢。又西有古勇關。東北距府二百七十五里。

潞江安撫司元柔遠路。洪武十五年三月為府,後廢,屬麓川平緬司。永樂元年正月析置潞江長官司,直隸都司。十六年六月升安撫司。宣德元年六月改隸布政司。正統三年六月屬金齒軍民司。嘉靖元年十月屬府。北有潞江,一名怒江,源出吐蕃雍望甸,南流經此,折而東南入府界。東岸有潞江關,北岸有細甸。又西有鎮姚守禦千戶所,萬曆十三年置,治老姚關鳳山之阿。又西有全勝關。東北距府三百五十里。

鎮道安撫司、楊塘安撫司二司地舊屬西番,與麗江府接界。俱永樂四年正月置,屬金齒軍民司。嘉靖元年屬府。

瓦甸安撫司本瓦甸長官司。宣德二年置,屬金齒軍民司。九年二月直隸都司。正統三年五月仍屬金齒軍民司。五年十一月升為安撫司。嘉靖元年屬府。

鳳溪長官司府東。洪武二十三年十一月置,屬金齒軍民司。嘉靖元年改屬府。

施甸長官司府南。元石甸長官司。洪武十七年五月更名,屬府。二十三年屬金齒軍民司。嘉靖元年仍屬府。西有坪市河,下流入於怒江。東南有猛淋寨,萬曆十三年置鎮安守御千戶所於此。南有金齒巡檢司,治浦關。又南有石甸巡檢司。

茶山長官司永樂五年析孟養地置,屬金齒軍民司。嘉靖元年屬府。東有高黎共山。

蒙化府元蒙化州,屬大理路。洪武十五年三月因之。正統十三年六月升為府。北有龍宇圖山、又有甸頭山,一名天耳山。南有甸尾山。西有陽江,源出甸頭澗,下流至定邊縣,入定邊河。又西有樣備江,一名神莊江,與永平縣分界,南入順寧府境,為黑惠江。西南有瀾滄江。有甸頭、甸尾、樣備、瀾滄江四巡檢司。又西南有備溪江土巡檢司。又東有迷渡市,嘉靖初築。東距布政司八百六十里。

順寧府元泰定四年十一月置。洪武十五年三月庚戌因之。己未降為州,屬大理府。十七年正月仍升為府。西北有樂平山。南有把邊山,中有把邊關。東北有瀾滄江,又有黑惠江,即樣備江也,又名墨會江,南流至府東泮山下,合於瀾滄江。又城東有順寧河,源出甸頭村山箐,流入雲州之孟祐河。南有寶通州,又有慶甸縣,元俱與府同置,洪武十五年省。又西南有矣堵寨,萬歷三十年置右甸守御士千戶所於此。北有錫鉛寨、又有牛街、又有猛麻、又有錫蠟寨、董甕寨、蟒水寨、亦壁嶺七巡檢司。領州一。東距布政司千五百五十里。

雲州本大侯長官司。永樂元年正月析麓川平緬地置,直隸都司。宣德三年五月升為大侯禦夷州,直隸布政司。萬曆二十五年更名,來屬。舊治在南。萬曆三十年移於今治。南有瀾滄江,東有孟祐河流入焉。有臘丁鄉巡檢司,後廢。西距府百五十里。領長官司一:

孟緬長官司州西南。宣德五年六月以景東府之孟緬、孟梳地置,屬景東府,後直隸布政司。萬曆二十五年來屬。有大猛麻、又有猛撒二土巡檢司,與猛緬稱為「三猛」。

車里軍民宣慰使司元車里路,泰定二年七月置,即大徹里。洪武十五年閏二月為軍民府。十九年十一月改軍民宣慰使司。永樂中廢。宣德六年復置。東北有瀾滄江,與九龍江會,達於交址,為富良江,而入於海。又有沙木江。東有小徹里部,永樂十九年正月置車里靖安宣慰使司,宣德九年十月省入車里。又有元耿凍路,至正七年正月置,又有耿當、孟弄二州,亦元末置,洪武十五年俱省入車里。西北距布政司三十四程。

緬甸軍民宣慰使司本緬中宣慰司。洪武二十七年六月置,尋廢。永樂元年十月復置,更名。北有大金沙江,其上流即大盈江也,源出青石山,自孟養境內流經司北江頭城下,下流注於南海。東有阿瓦河,自孟養流入境,下流入大金沙江。又北有江頭城、太公城、馬來城、安正國城、蒲甘緬王城,謂之「緬中五城」。元後至元四年十二月置邦牙宣慰司於蒲甘緬王城,至正二年六月廢。至元二十六年置太公路於太公城,洪武十五年三月為府,後廢。領長官司一。東北距布政司三十八程。

東倘長官司宣德八年九月置。

木邦軍民宣慰使司元木邦路,至順元年三月置。洪武十五年三月為府,後廢。三十五年十二月復置。永樂二年六月改軍民宣慰司。北有慕義山。西有喳里江,即潞江,自芒市流入境,又西南入緬甸界。又北有蒙憐路、蒙來路,俱元置,洪武十五年三月俱為府,後俱廢。又西北有孟炎甸,有天馬關。東北距布政司三十五程。

八百大甸軍民宣慰使司元八百等處宣慰使司。洪武二十四年六月改置。東北有南格剌山,下有河,與車里分界。有八百者乃軍民宣慰使司,永樂二年四月分八百大甸地置,後廢。又有蒙慶宣慰司,元泰定四年閏月置,至正二年四月罷,洪武十五年三月復置府,後廢。又有孟絹路,元元統元年置,屬八百宣慰司,洪武十五年三月為府,後廢。又有木按、孟傑二路,俱元置,洪武十五年三月俱為府,後俱廢。北距布政司三十八程。

孟養軍民宣慰使司元雲遠路。洪武十五年三月為府。十七年改為孟養府,後廢。三十五年十二月復置。永樂二年六月改軍民宣慰使司。正統十三年廢。萬曆十三年改置長官司。東有鬼窟山,又有茫崖山。又有大金沙江,其上流即大盈江,南流入於緬甸。又南有密堵城,有速送城。又南有戛撒寨。西有猛倫,西南有孟拱、戛里、猛別、盞西諸部。東北距布政司三十七程。

老撾軍民宣慰使司永樂二年四月置。東南有三關,與安南界。西北距布政司六十八程。

南甸宣撫司元至元二十六年置南甸路。洪武十五年三月為府,後廢,屬騰衝守禦千戶所。永樂十二年正月置州,直隸布政司。正統三年五月改屬金齒軍民指揮使司。九年六月升宣撫司,仍直隸布政司。東有丙弄山,又有蠻乾山。南有沙木籠山,上有沙木籠關。西有大盈江。東北有小梁河,西南經南牙山下,曰南牙江,入乾崖境內。又東南有孟乃河,即騰越州之龍川江。又南有黃連坡關。東北有小隴川關。東北距布政司二十二程。

十崖宣撫司元鎮西路。洪武十五年三月為府,後廢,屬麓川平緬司。永樂元年正月析置乾崖長官司,直隸都司,後屬金齒軍民指揮使司。宣德五年六月復屬都司。正統三年五月復屬金齒軍民指揮使司。九年六月升宣撫司,直隸布政司。東有雲籠山。西有大盈江,又南有檳榔江,自吐蕃界流合焉。東有安樂河,即小梁河,下流經雲籠山下,曰雲籠江,經司治北,折而西,合於檳榔江。又西北有南賧,元置,洪武中廢。又西有雷弄、盞達等部。東北距布政司二十三程。

隴川宣撫司本麓川平緬軍民宣慰使司。正統六年廢,九年九月改置,治隴把。元平緬路,在隴把東北。洪武十五年閏三月置平緬宣慰使司。三月又改路為府,未幾府廢。十七年八月丙子升司為平緬軍民宣慰使司。甲午改麓川平緬軍民宣慰使司,省麓川路入焉。元麓川路在隴把南,洪武十五年三月為府,未幾府廢。十七年八月為麓川平緬軍民宣慰司治所,正統中,司廢,曰平麓城,亦曰孟卯城,萬曆十二年置宣撫同知於此。又西南有通西軍民總管府,元至元二十六年置,洪武十五年三月為府,後廢。又東南有遮放城,萬曆十二年置宣撫副使於此。北有馬鞍山。西北有大金沙江。又有麓川江,即龍川江,自南甸流入,與芒市分界,西南入於大金沙江。東北距布政司六十六程。

孟定禦夷府元孟定路,至元三十一年四月置。洪武十五年三月為府。東北有無量山,又有喳哩江,與麓川江合。東南有謀粘路,元泰定三年七月置。有木連路,元至正二十六年置。洪武十五年三月俱因之,後俱廢。領安撫司一。東北距布政司十八程。

耿馬安撫司萬曆十三年析孟定地置。西有三尖山。南有喳哩江,與孟定分界。北距府百里。

孟艮禦夷府永樂三年七月置,直隸都司,後直隸布政司。東有木朵路,又有孟隆路,俱元泰定三年九月置。東北有孟愛等甸軍民府,元至元二十六年置。洪武十五年三月俱為府,後俱廢。北距布政司三十八程。

威遠禦夷州元威遠州,屬威楚路,後改威遠蠻柵府。洪武十五年三月仍為威遠州,屬楚雄府。十七年升為府,後廢。三十五年十二月復置州,直隸布政司。北有蒙樂山,接景東府界。西北有威遠江,一名穀寶江,下流合瀾滄江。東北距布政司十九程。

灣甸禦夷州本灣甸長官司。永樂元年正月析麓川平緬地置,直隸都司。三年四月升為州,直隸布政司。西北有高黎共山。北有姚關,與順寧府界。東北距布政司二十程。

鎮康禦夷州元鎮康路。洪武十五年三月為府。十七年降為州,後廢,以其地屬灣甸州。永樂七年七月復置,直隸布政司。西有喳哩江,接潞江安撫司界。南有昔剌寨。西南有控尾寨。東北距布政司二十三程。

孟密宣撫司本孟密安撫司。成化二十年六月析木邦地置。萬曆十三年升為宣撫司。東北有南牙山,與南甸分界。西南有摩勒江,有大金沙江,俱與緬甸分界。又有寶井。北有猛乃、猛哈,東北有孟廣等部。東北距布政司三十三程。

蠻莫安撫司萬曆十三年析孟密地置。東北有等練山。西南有那莫江,下流入大金沙江。又西有孟木寨。東北距布政司三十一程。

者樂甸長官司永樂元年正月析麓川平緬地置,直隸都司,後改隸布政司。南有瀾滄江。又東有景來河,自景東府流入,下流入馬龍江。東北距布政司千一百七十里。

鈕兀禦夷長官司宣德八年十月以和泥之鈕兀、五隆二寨置,北距布政司十六程。

芒市禦夷長官司元芒施路。洪武十五年三月為府,後廢。正統八年四月改置,屬金齒軍民指揮司,後直隸布政司。西南有永昌乾山,又有孟契山。又有大盈江,西南經青石山下,又西有麓川江來合焉。東北距布政司二十三程

孟璉長官司舊為麓川平緬司地,後為孟定府。永樂四年四月置,直隸都司。東南有木來府,元置,洪武十五年三月因之,後廢。東北距布政司二十三程。

大古剌軍民宣慰使司在孟養西南。亦曰擺古,濱南海,與暹羅鄰、底馬撒軍民宣慰使司在大古剌東南、小古剌長官司、茶山長官司、底板長官司、孟倫長官司、八家塔長官司皆在西南極邊俱永樂四年六月置。

刺和莊長官司永樂四年十月置,直隸都司。

○促瓦長官司

散金長官司舊俱為麓川平緬司地。永樂六年四月置。

裡麻長官司永樂六年七月析孟養地置,直隸都司。

八寨長官司永樂十二年九月置,直隸都司。

底兀刺宣慰使司永樂二十二年三月置。地舊為大古剌所據,上諭還之,故置司。

廣邑州本金齒軍民司之廣邑寨。宣德五年五月升為州。八年十一月直隸布政司。正統元年三月徙於順寧府之右甸。

貴州《禹貢》荊、梁二州徼外。元為湖廣、四川、雲南三行中書省地。洪武十五年正月置貴州都指揮使司,治貴州宣慰司。其民職有司則仍屬湖廣、四川、雲南三布政司。永樂十一年置貴州等處承宣布政使司。與都指揮司同治。領府八,州一,縣一,宣慰司一,長官司三十九。後領府十,州九,縣十四,宣慰司一,長官司七十六。北至銅仁,與湖廣、四川界。南至鎮寧,與廣西、雲南界。東至黎平,與湖廣、廣西界。西至普安,與雲南、四川界。距南京四千二百五十里,京師七千六百七十里。弘治四年,編戶四萬三千三百六十七,口二十五萬八千六百九十三。萬曆六年,戶四萬三千四百五,口二十九萬九百七十二。

貴陽軍民府本程番府。成化十二年七月分貴州宣慰司地置,治程番長官司。隆慶二年六月移入布政司城,與宣慰司同治。三年三月改府名貴陽。萬曆二十九年四月升為軍民府。領州三,縣二,長官司十六:

新貴倚。本貴竹長官司,洪武五年正月置,屬宣慰司。萬曆十四年二月改置縣,來屬。西有獅子山。西北有木閣箐山,在水西境內。北有貴人峰。又西有白龍洞。北有烏江,源出水西,與四川遵義府分界,北流至四川彭水縣,入涪陵江。西北有陸廣河,下流入於烏江,有陸廣河巡檢司。又西有宅溪。又西北有蔡家關,一名響水關,又有闊水關。

貴定倚。萬曆三十六年析新貴縣及定番州地置。東有銅鼓山,有石門山。南有高連山,有南門河。又東有龍洞河,下流俱入陸廣河。

開州崇禎四年十一月以副宣慰洪邊舊地置。西南距府一百二十里。

廣順州本金築長官司。洪武五年三月置,屬四川行省。十年正月改安撫司。十九年十二月屬廣西。二十七年仍屬四川。二十九年屬貴州衛。正統三年八月直隸貴州布政司。成化十二年七月屬程番府。隆慶二年六月屬貴陽府。萬曆四十年置州。東南有天臺山。北有天生橋。南距府一百一十里。

定番州元程番武勝軍安撫司。洪武五年罷。成化十二年七月置程番府,領金築安撫司,上馬橋、大龍番、小龍番、程番、方番、韋番、臥龍番、洪番、小程番、盧番、羅番、金石番、盧山、木瓜、大華、麻響十六長官司。隆慶二年六月移府入布政司城。萬曆十四年三月置州。距府八十五里,領長官司十六:

程番長官司倚。洪武五年三月置,屬貴州衛。正統三年八月屬貴州宣慰司。成化十二年七月屬程番府。萬歷十四年三月屬州。北有青巖。南有都泥江,源出州西北亂山中,曰濛潭,經司南,州境之水皆流合焉,入廣西南丹州界。下十二司所屬人放此。

小程番長官司州西北。元小程番安撫司。洪武六年正月改置。

上馬橋長官司州西北。洪武十五年六月置。

盧番長官司州北。元盧番靜海軍安撫司。洪武六年正月改置,省元盧番蠻夷軍民長官司入焉。

韋番長官司州南。元韋番蠻夷長官司。洪武十五年六月改置。

方番長官司州南。元方番河中府安撫司。洪武五年改置。

洪番長官司州西。元洪番永盛軍安撫司。洪武六年正月改置。

臥龍番長官司州南。元臥龍番南寧州安撫司。洪武五年改置。

小龍番長官司州東南。元小龍番靜蠻軍安撫司。洪武六年正月改置。

大龍番長官司州東南。元大龍番應天府安撫司。洪武五年改置。

金石番長官司州東。元金石番太平軍安撫司。洪武五年改置。

羅番長官司州南。元羅番大龍遏蠻軍安撫司。洪武五年改置。

盧山長官司州南。元盧山等處蠻夷軍安撫司。洪武六年正月改置。

木瓜長官司元木瓜等處蠻夷軍民長官司。洪武五年改置,屬貴州衛。正統三年八月屬金築安撫司。成化十二年七月,屬程番府。萬歷十四年三月屬州。下二司仿此。

麻響長官司洪武七年六月置。

大華長官司洪武七年六月置。

貴州宣慰使司元改順元路軍民安撫司置,屬湖廣行省。洪武五年正月屬四川行省。九年六月屬四川布政司。永樂十一年二月來屬。有沙溪、的澄河二巡檢司。又有黃沙渡、龍谷二土巡檢司。領長官司七:

水東長官司宣慰司北。元水東寨長官司。洪武五年改置,後廢。永樂元年六月置,屬都司,後來屬。

中曹蠻夷長官司宣慰司東南。元中曹白納等處長官司,屬管番民總管。洪武五年改置,來屬。

龍里長官司宣慰司東南。元龍里等寨長官司,屬管番民總管。洪武五年改置,來屬。

白納長官司宣慰司東南。元茶山白納等處長官司。洪武五年並入中曹司。永樂四年五月置,來屬。

底寨長官司宣慰司北。元底寨等處長官司。洪武五年改置。

乖西蠻夷長官司宣慰司東北。元乖西軍民府,屬管番民總管。洪武五年改置,後廢。永樂元年六月復置,屬都司,後來屬。

養龍坑長官司宣慰司北。元養龍坑宿徵等處長官司。洪武五年改置。

安順軍民府元安順州,屬普定路。洪武十五年三月屬普定府。十八年直隸雲南布政司。二十五年八月屬四川普定衛。正統三年八月直隸貴州布政司。成化中,徙州治普定衛城。萬曆三十年九月升安順軍民府。普定衛舊在州西北,洪武十五年正月置,屬四川都司。三月升軍民指揮使司,正統三年改屬貴州都司。成化中,州自衛東南來同治。西北有舊坡山,兩峰相對,中有石關。東有巖孔山。北有歡喜嶺,又有思臘河,接水西界。西南有北盤江,自雲南霑益州流入。東南有九溪河。又東有元普定路,屬雲南行省,洪武十五年三月為府,屬雲南布政司,尋并軍民府,改屬四川布政司,十八年七月廢。領州三,長官司六。東距布政司百五十里。

寧谷寨長官司府西南。洪武十九年置,屬安順州。二十五年八月屬普定衛。正統三年八月仍來屬。下仿此。東南有乾海子。

西堡長官司府西北。建置所屬同上。北有浪伏山,元置習安州於山下,屬普定路,洪武十五年三月屬普定府,後廢。又北有白石巖。東南有楚油洞山。北有谷龍河,下流合烏江。

鎮寧州元至正十一年四月以火烘夷地置,屬普定路。洪武十五年三月屬普定府。二十五年八月屬普定衛,後僑治衛城。正統三年八月直隸貴州布政司。嘉靖十一年六月徙州治安莊衛城。萬曆三十年九月屬府。安莊衛舊在州西,洪武二十三年五月置,屬貴州都司。萬曆三十五年九月,州自衛東來同治。南有白水河,又有烏泥江,即都泥江,源出山箐中,東南流,入金築安撫司境。東距府五十五里。領長官司二:

十二營長官司州北。洪武十九年置,屬安順州。二十五年八月屬普定衛。正統三年八月來屬。下仿此。東北有天生橋,又有公具河。北有阿破河。

康佐長官司州東。建置所屬同上。

永寧州元以打罕夷地置,屬普定路。洪武十五年三月屬普定府。二十五年八月屬普定衛,後僑治衛城。正統三年八月直隸貴州布政司。嘉靖十一年三月徙州治關索嶺守禦千戶所城。萬曆三十年九月屬府。關索所舊在州西南,洪武二十五年置,屬安莊衛。萬曆三十年九月,州自所東北來同治。西北有紅崖山。西有北盤江,自普安州流入,有盤江河巡檢司。東北距府一百二十里。領長官司二:

慕役長官司州西。洪武十九年置,屬安順州。二十五年八月屬普定衛。正統三年八月來屬。下仿此。北有安籠箐山。西北有象鼻嶺。東有北盤江,與永寧州分界,東南流,南盤江自雲南羅平州來合焉,又南入廣西泗城州界。

頂營長官司州北。洪武四年置,所屬同上。東有關索嶺。西有盤江。

普安州本貢寧安撫司。建文中置,屬普安軍民府。永樂元年正月改普安安撫司,屬四川布政司。十三年十二月改為州,直隸貴州布政司。萬歷十四年二月徙治普安衛城。三十年九月屬府。普安衛舊在州南,洪武十五年正月置,屬雲南都司,後改屬貴州都司。二十二年三月升軍民指揮使司。萬曆十四年二月,州自衛北來同治。東有八部山,元普安路治山下,屬雲南行省,洪武十五年三月為府,屬雲南布政司,尋升軍民府,二十七年四月改屬四川,永樂後廢。東北有格孤山。又西北有番納牟山,一名雲南坡。又東南有得都山,一名白崖,產雄黃水銀。又東有盤江。東南有者卜河,下流入於盤江。東有芭蕉關。西有分水嶺關。東南有安籠箐關。又西南有樂民守禦千戶所,西有平夷守御千戶所,俱洪武二十二年置,又東南有安南守禦千戶所,又有安籠守禦千戶所,俱洪武二十三年置,皆屬普安衛。正統十年四月徙安南所於羅渭江。東北距府三百三十五里。

都勻府本都勻安撫司。洪武十九年十二月置。二十三年十月改都勻衛,屬貴州都司。二十九年四月升軍民指揮使司,屬四川布政司。永樂十七年仍屬貴州都司。弘治七年五月置都勻府於衛城。西有龍山。南有獨山鎮巡檢司。北有平定關,西有威鎮關,俱洪武二十四年置。領州二,縣一,長官司八。西北距布政司二百六十里。

都勻長官司府南。元上都勻等處軍民長官司。洪武十六年更名。南有都勻河,亦名馬尾河。

邦水長官司府西。元中都雲板水等處軍民長官司,屬管番民總管。洪武十六年更名。邦水河在東南,本名扳河,即都勻河上源。

平浪長官司府西。洪武十六年置。西南有凱陽山,上有滅苗鎮,即故凱口囤。東南有麥沖河。

平洲六洞長官司府西南。洪武十六年置。西南有六洞山。南有平洲河,中有沙洲。

麻哈州本麻哈長官司。洪武十六年置,屬平越衛。弘治七年五月升為州,來屬。南有麻哈江,即邦水河之上源。南距府六十里。領長官司二。

樂平長官司州西北。洪武二十四年五月置,屬雲南,後屬平越衛。弘治七年五月來屬。東北有馬場山。南有樂平溪。

平定長官司州西北。洪武二十二年置,屬平越衛。三十年屬清平衛。弘治七年五月來屬。東有山江河。

獨山州本九名九姓獨山州長官司。洪武十六年置,屬都勻衛。弘治七年五月升為獨山州,屬府。南有獨山,有獨山江,即都勻河下流,南入廣西天河縣界,為龍江。北距府百五十里。領縣一,長官司二:

清平府北。本清平長官司,洪武二十二年置,屬平越衛。三十年屬清平衛。弘治七年五月改為縣,屬麻哈州,後來屬。東有香爐山,嘉靖十二年四月徙清平衛中左所於此。北有雲溪洞。南有木級坡。又東有山江河,源出香爐山,有舟溪江流合焉,亦都勻河上源。又南有雞場關,北有羅沖關,俱洪武二十五年置。又東北有黎樹等寨。

合江洲陳蒙爛土長官司州東。洪武十六年置,屬都勻衛。弘治七年五月屬州。東南有梅花洞。

豐寧長官司州西南。洪武二十三年置,屬都勻衛。弘治七年五月屬州。西南有行郎山。

平越軍民府元平月長官司。洪武十四年置平越守禦千戶所。十五年閏二月改為平越衛。十七年二月升軍民指揮使司。領長官司五,屬四川布政司,尋屬貴州都司。萬曆二十九年四月置平越軍民府於衛城,以播州地益之,屬貴州布政司。東有峨黎山,又有七盤坡。東南有麻哈江,其上源即黃平州之兩岔江。南有馬場江,又有羊場河,俱東入於麻哈江。南有武勝關。西南有通津關。東南有羊場關。領衛二,州一,縣三,長官司二。西距布政司百八十里。

清平衛洪武二十三年六月置,屬貴州都司。萬曆二十九年來屬。衛治在清平縣北一里。西南距府六十里。

興隆衛洪武二十二年六月置,屬貴州都司。萬曆二十九年來屬。北有龍巖山,亦名龍洞山。又有截洞,其深險。東有飛雲巖。西南距府百二十里。

黃平州本黃平安撫司。洪武七年十一月置,屬播州宣慰司。萬曆二十九年四月改為州,來屬。東有七里谷。西南有兩岔江,以兩源合流而名。又東有冷水河。西北有黃平守禦千戶所,洪武十一年正月置,十五年正月改為衛,閏二月仍為千戶所。南距府三十里。

餘慶州西。本餘慶長官司,洪武十七年置,屬播州宣慰司。萬曆二十九年六月改為縣,來屬。東有白泥長官司,亦洪武十七年置,屬播州宣衛司,萬曆二十九年四月省入餘慶縣。南有小烏江,下流入於烏江。東南有白泥河,下流合於思南河。又有走馬坪寨,嘉靖三十四年置。

甕安州西北。本甕水安撫司,洪武初置。萬歷二十九年四月改為縣,來屬。東有草塘安撫司,洪武十七年六月置,又有重安長官司,永樂四年九月置,俱屬播州宣慰司,萬曆二十九年四月俱省入甕安縣。東南有萬丈山。西有烏江,縣境諸山溪之水皆流合焉。又有黃灘關。東北有飛練堡,有天邦囤,西有西坪等寨。

湄潭州北。萬曆二十九年四月以播州湄潭地置。西有容山長官司,洪武中置,屬播州宣慰司。萬曆二十九年省入湄潭縣。南有湄潭水,又西有三江水,下流俱入於烏江。

凱里長官司府東北。本凱里安撫司,嘉靖八年二月分播州宣慰司地置,屬清平衛。萬曆二十九年來屬。三十五年六月改為長官司。

楊義長官司府東南。洪武初置,屬平越衛。萬歷二十九年屬府。西有杉木箐山。又有清水江,上流自新添衛流入,經城西,又名皮隴江,北經乖西、巴香諸苗界,而入烏江。

黎平府本思州宣慰司地。洪武十八年正月置五開衛,屬湖廣都司,後廢。三十五年十一月復置。永樂十一年二月置黎平府於衛城,屬貴州布政司。弘治十年徙府治衛南。萬曆二十九年十一月改府屬湖廣。三十一年四月還屬貴州。南有寶帶山。東有摩天嶺。東北有銅鼓巖。西有新化江。又有福祿江,其上源為古州江,下流入廣西懷遠縣境。西南有黎平守御千戶所,洪武二十一年九月置,屬五開衛。領縣一,長官司十三。西距布政司六百三十里。

永從府南。本元福祿永從軍民長官司。洪武中改置福祿永從蠻夷長官司,後廢。永樂元年正月復置,屬貴州衛。十二年三月來屬。正統六年九月改為縣。南有福祿江,有彩江流合焉。又有永從溪。

潭溪蠻夷長官司府東南。元潭溪長官司。洪武三年正月改置,屬湖廣辰州衛。三月改屬湖廣靖州衛,後廢。永樂元年正月復置,屬貴州衛。十二年三月來屬。西南有銅關鐵寨山。南有潭溪。

八舟蠻夷長官司府北。元八舟軍民長官司。洪武五年改置,後廢。永樂元年正月復置,屬貴州衛。十二年三月來屬。西南有八舟江,源自府城,西為三十里江,北流經此,又東北為新化江。

洪舟泊里蠻夷長官司府東南。元洪舟泊里軍民長官司。洪武初改置,後廢。永樂元年正月復置,屬貴州衛。十三年三月來屬。北有洪舟江,下流合於湖廣靖州之渠河。西南有中潮守禦千戶所,洪武二十一年九月置,屬五開衛。

曹滴洞蠻夷長官司府西北。元曹滴等洞軍民長官司。洪武初改置,後廢。永樂元年正月復置,屬貴州衛。十二年三月來屬。西南有容江,源出苗地,北流入福祿江。

古州蠻夷長官司府西北。元古州八萬洞軍民長官司。洪武三年正月改置,屬湖廣辰州衛。三月改屬湖廣靖州衛,後廢。永樂元年正月復置,屬貴州衛。十二年三月來屬。有古州衛,洪武二十六年置,尋廢。東北有古州江。

西山陽洞蠻夷長官司府西南。洪武初置,後廢。永樂元年正月復置,屬貴州衛。十二年三月來屬。西北有大巖山,大巖江出焉,東南入於福祿江。

新化蠻夷長官司府東北。元新化長官司。洪武三年正月改置,屬湖廣辰州衛,三月改屬湖廣靖州衛,後廢。永樂元年正月復置,屬貴州衛。十一年二月置新化府於此,領湖耳、亮寨、歐陽、新化、中林驗洞、龍里六蠻夷長官司,赤溪湳洞長官司。宣德九年十一月府廢,以所領俱屬黎平府。西有六疊山。東南有新化江,又西北合於清水江。又東有新化亮寨守禦千戶所,洪武二十一年九月置,西南有新化屯千戶所,洪武二十五年置,俱屬五開衛。

湖耳蠻夷長官司府東北。元湖耳洞長官司。洪武三年正月改置,屬湖廣辰州衛。三月改屬湖廣靖州衛,後廢。永樂元年正月復置,屬貴州衛。十二年三月屬新化府,府廢來屬。西有銅鼓衛,本銅鼓守禦千戶所,洪武二十一年九月置,屬五開衛,三十年改所為衛,屬湖廣都司,後二年廢,三十五年十一月復置,屬湖廣都司。

亮寨蠻夷長官司府東北。本八萬亮寨蠻夷長官司。洪武三年正月置,屬湖廣辰州衛。三月屬湖廣靖州衛,後廢。永樂元年正月復置,改名,屬貴州衛。十二年三月屬新化府,府廢來屬。

歐陽蠻夷長官司府東北。元歐陽寨長官司。洪武三年正月改置,屬湖廣辰州衛。三月改屬湖廣靖州衛,後廢。永樂元年正月復置,屬貴州衛。十二年三月屬新化府,府廢來屬。

中林驗洞蠻夷長官司府北。洪武初置,後廢。永樂元年正月復置。十二年三月屬新化府,府廢來屬。下二司仿此。

赤溪湳洞蠻夷長官司府東北。

龍里蠻夷長官司府北。南有龍里守御千戶所,洪武二十五年置,屬五開衛。

思南府元思南宣慰司,屬湖廣行省。洪武四年改屬四川。六年十二月升為思南道宣慰使司,仍屬湖廣。永樂十一年二月改為府,屬貴州布政司。隆慶四年三月徙治平溪衛。尋復故。有都儒、五堡、二坑等處巡檢司。又有覃韓偏力水土巡檢司。又有板橋巡檢司,舊屬石阡府,後來屬。領縣三,長官司三。西南距布政司六百二十里。

安化倚。本水犄姜長官司,元屬思州安撫司。洪武初,改曰水德江,屬思南宣慰司。永樂十二年三月屬府。萬曆三十三年改置安化縣。西南有崖門山。南有萬勝山。又有烏江,自石阡府流入,經城西占魚峽北,入四川彭水縣界,合涪陵江。東南有水德江,即烏江之分流,又有思印江流合焉,下流亦入於涪陵江。舊有洪安、化濟二長官司,屬思南宣慰司,洪武二十六年五月省。東有水勝關。南有武勝關。北有太平關。

蠻夷長官司倚。洪武十年十月置,屬思南宣慰司。永樂十二年三月屬府。

婺川府北。元屬思州安撫司。洪武五年屬鎮遠州。十七年後仍屬思州。永樂十二年三月來屬。東有河只水,又有羅多水,下流俱注於水德江。

印江府東。本思印江長官司,元屬思南宣慰司。永樂十二年三月屬府。弘治七年六月改為印江縣。

沿河祐溪長官司府東北。洪武七年十月置,屬思南宣慰司。永樂十二年三月屬府。

朗溪蠻夷長官司府東。洪武七年十月置,屬思南宣慰司。永樂十二年三月屬烏羅府。正統三年五月,府廢,來屬。有厥溪蠻夷長官司,亦洪武七年十月置,尋廢。

思州府元思州宣慰司。永樂十一年二月改為府,屬貴州布政司。領長官司四。西距布政司七百五十里。

都坪峨異溪蠻夷長官司倚。洪武六年置,二十五年省。永樂十二年三月復置。南有峨山。西北有江頭山。東有異溪。東北有平溪,上有關,洪武二十二年三月置平溪衛於此,屬湖廣都司,萬歷二十九年十一月改屬貴州,三十一年四月還屬湖廣。又有占魚關。南有黃土關。又東北有晃州驛,路出湖廣沅州。

都素蠻夷長官司府西。永樂十二年三月置,屬府。

施溪長官司府北。元施溪樣頭長官司。洪武五年改名,屬湖廣沅州衛。永樂十二年三月來屬。東有施溪。

黃道溪長官司府東北。元屬思州宣慰司。永樂十二年三月屬府。西南有黃道溪。

鎮遠府元鎮遠府,屬思州安撫司。洪武四年降為鎮遠州,屬思南宣慰司。五年六月直隸湖廣。永樂十一年二月置鎮遠府於州治,屬貴州布政司。正統三年五月省州入焉。領縣二,長官司三。西距布政司五百三十里。

鎮遠倚。本鎮遠溪洞金容金達蠻夷長官司,洪武二年二月置,屬思南宣慰司。永樂十二年三月屬州。正統三年五月改屬府。弘治七年十月改為鎮遠縣。北有石崖山。東有中河山,以兩水夾流而名。東北有鐵山。又東有觀音山,有馬場坡。東南有巴邦山。西有平冒山。南有鎮陽江,一名鎮南江,亦曰潕水,上受興隆、黃平諸水,東流三百裏,入於沅江。又東北有鐵溪,出鐵山,下流入鎮陽江。又西有油榨關。有焦溪關、梅溪關。又有清浪關,清浪衛治於此,又西有偏橋,偏橋衛在焉,俱洪武二十三年四月置。西南有鎮遠衛,洪武二十二年七月置。俱屬湖廣都司,萬歷二十九年十一月俱改屬貴州,三十一年四月還屬湖廣。

施秉府西南。本施秉蠻夷長官司,洪武五年置,屬思南宣慰司。永樂十二年三月屬州。正統九年七月改為縣。天啟元年四月省。崇禎四年十一月復置。南有洪江,即鎮陽江。

偏橋長官司府西。元偏橋中寨蠻夷軍民長官司。洪武五年改置,屬思南宣慰司。永樂十二年三月來屬。

邛水十五洞蠻夷長官司府東。元邛水縣。洪武五年改置團羅、得民、曉隘、陂帶、邛水五長官司,屬思州宣慰司。二十九年以四司並入邛水司,屬思南宣慰司。永樂十二年三月屬府。

臻剖六洞橫坡等處長官司府西。本臻剖、六洞、橫坡三長官司,洪武二十二年置,屬鎮遠衛,後并為一司。

銅仁府本思州宣慰司地。永樂十一年二月置銅仁府。領縣一,長官司五。西南距布政司七百七十里。

銅仁倚。元銅人大小江等處蠻夷軍民長官司,屬思州安撫司。洪武初,改置銅仁長官司,屬思南宣慰司。永樂十二年三月置府治於此。萬曆二十六年四月改為縣。南有銅崖山。又有新坑山,產硃砂水銀。西南有銅仁大江,西北有小江流合焉,下流入沅州界,注於沅江。

省溪長官司府西。元省溪壩場等處蠻夷長官司,屬思州安撫司。洪武初改名,屬思南宣慰司。永樂十二年三月來屬。西有棨邏江,即省溪,產金。

提溪長官司府西。元提溪等處軍民長官司,屬思州安撫司。洪武初改名,屬思南宣慰司。永樂十二年三月來屬。東有印江。西有提溪,產砂金。

大萬山長官司府南。元大萬山蘇葛辦等處軍民長官司,屬思州安撫司。洪武初改名,屬思南宣慰司。永樂十二年三月來屬。

烏羅長官司府西。元烏羅龍幹等處長官司,屬思州安撫司。洪武初更名,屬思南宣慰司。永樂十一年二月置烏羅府,領朗溪蠻夷長官司,烏羅、答意、治古、平頭著可四長官司治於此。正統三年五月,府廢來屬。西有九龍山,銅仁大江源於此。又西南有觀音囤,亦曰烏羅洞。南有九江。又有木耳溪,亦曰九十九溪,下流亦入沅江。

平頭著可長官司府西北。元平頭著可通達等處長官司,屬思州安撫司。洪武七年十月改置,屬思南宣慰司。永樂十二年三月屬烏羅府,府廢來屬。又有答意長官司,治古寨長官司,俱永樂三年七月置,屬貴州宣慰司,十二年三月改屬烏羅府,正統三年五月俱與府同廢。

石阡府本思州宣慰司地。永樂十一年二月置石阡府。領縣一,長官司三。西南距布政司六百三十里。

石阡長官司倚。元石阡等處軍民長官司,屬思州安撫司。洪武初改置,屬思州宣慰司。永樂十二年三月為石阡府治。西有崖門山。南有秋滿洞。西有烏江,自四川遵義府流入,東北入思南府界。有石阡江,下流入於烏江。

龍泉府西。本龍泉坪長官司,元為思州安撫司治。洪武七年七月復置,屬思州宣慰司。永樂十二年三月來屬。萬曆二十九年四月改為縣。北有騰雲洞。南有鄧坎等寨。

苗民長官司府西南。洪武七年十月置,屬思州宣慰司。永樂十二年三月來屬。

葛彰葛商長官司府南。元屬思州安撫司。洪武中屬思州宣慰司。永樂十二年三月來屬。

龍里衛軍民指揮使司洪武二十三年四月置衛。二十九年四月升軍民指揮使司。西有蓮花涇,又有加牙河,下流入甕首河。東南有平伐長官司,本元平伐等處長官司,洪武十五年改置,屬貴州衛,二十八年屬龍里衛,萬歷十四年二月省入新貴縣。又西有長沖關。東有巃聳關。領長官司一。西距布政司五十里。

大平伐長官司衛南。洪武十九年置,屬貴州衛。二十八年來屬。東北有谷峽山。東南有甕首河,下流合清水江。

新添衛軍民指揮使司元新添葛蠻安撫司,後廢。洪武二十二年置新添千戶所,屬貴州衛。二十三年二月改為新添衛,屬貴州都司。二十九年四月升軍民指揮使司。領長官司五。西距布政司百十里。

新添長官司倚。洪武四年置。東有憑虛洞,一名豬母洞。西北有清水江。西南有甕城河,有甕城河土巡檢司。又東有谷忙關。

小平伐長官司衛西南。洪武十五年六月置,屬貴州衛,尋屬龍里衛。二十九年來屬。

把平寨長官司衛南。洪武十五年六月置,屬貴州衛,尋屬龍里衛。二十九年來屬。

丹平長官司衛西南。洪武三十年置,尋省。永樂二年復置。

丹行長官司衛西南。洪武三十年置,尋省。永樂二年復置。

安南衛洪武十五年正月置尾灑衛於此,尋廢。二十三年十二月復置,更名,屬貴州都司。南有尾灑山。東有盤江山,有清源洞。又有北盤江,自雲南沾益州流入,又南入安順府界。東南有者卜河,自普安州流入,注於盤江。西有江西陂,初置柵屯守於此,尋徙於尾灑,築城為衛。南有烏鳴關,亦洪武中置。東北距布政司三百四十里。

威清衛洪武二十三年六月置,屬貴州都司。北有羊耳山。西有的澄河,即陸廣河上流。西北有鴨池河,即烏江。西距布政司六十里。

平壩衛洪武二十三年閏四月置,屬貴州都司。東南有南仙洞,有馬頭山。東有東溪。西南距布政司八十里。

畢節衛洪武十七年二月置,屬貴州都司。東有木稀山,有關。又有響水河。南有善欲關,西有老鴉關,俱洪武中置。東北有層臺衛,洪武二十一年九月置,二十七年六月廢。領守禦所一。東南距布政司四百五十里。

守禦七星關後千戶所衛西。洪武二十一年置,屬烏撒衛。永樂中來屬。有七星關河,亦曰可渡河,源出四川烏撒府,即北盤江上流,七星關在其上,下流入雲南沾益州界。

赤水衛洪武二十一年十月置,北有雪山,上有關。東有赤水河,有赤水關。領所四。距布政司六百二十里。

摩尼千戶所衛北、白撒千戶所衛東南。二所俱洪武二十二年九月置、阿落密千戶所衛南、前千戶所衛南。二所俱洪武二十七年置。

普市守禦千戶所洪武二十三年三月析永寧宣撫司地置,直隸貴州都司。東有木案山。西南有水腦洞。又東南有龍泉澗。距布政司七百二十里。

敷勇衛本答刂佐長官司。洪武五年改元落邦札佐等處長官司置,屬貴州宣慰司。崇禎三年改置,屬貴州都司。東有陽明洞。西有三湘水。北有烏江,有陸廣河。領所四。南距布政司五十里。

於襄守禦千戶所衛西。本青山長官司,洪武五年改元青山遠地等處長官司置,屬貴州宣慰司。崇禎三年改置。

息烽守御千戶所衛東北。崇禎三年以貴州前衛故絕六屯並割底寨司地置。西有西望山。南有石天洞。北有烏江。

濯靈守御千戶所衛北。西有陸廣河,北流合烏江、修文守御千戶所衛東北。二所俱宣慰司水西地,崇禎三年同置。

鎮西衛崇禎三年以宣慰司水西地置。北有天柱洞,又有鴨池河,即烏江異名。領所四。西南距布政司六十里。

威武守御千戶所衛東、赫聲守御千戶所衛北。有鴨池河、柔遠守御千戶所衛□、定遠守御千戶所衛□。以上俱水西地,崇禎三年與衛同置。
