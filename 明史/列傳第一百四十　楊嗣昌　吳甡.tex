\article{列傳第一百四十 楊嗣昌 吳甡}

\begin{pinyinscope}
楊嗣昌,字文弱,武陵人。萬曆三十八年進士。改除杭州府教授。遷南京國子監博士,累進戶部郎中。天啟初,引疾歸。

崇禎元年,起河南副使,加右參政,移霸州。四年,移山海關飭兵備。父鶴,總督陜西被逮,嗣昌三疏請代,得減死。五年夏,擢右僉都御史,巡撫永平、山海諸處。嗣昌父子不附奄,無嫌於東林。侍郎遷安郭鞏以逆案謫戍廣西,其鄉人為訟冤。嗣昌以部民故,聞於朝,給事中姚思孝駁之,自是與東林卻。

七年秋,拜兵部右侍郎兼右僉都御史,總督宣、大、山西軍務。時中原饑,群盜蜂起,嗣昌請開金銀銅錫礦,以解散其黨。又六疏陳邊事,多所規畫。帝異其才。以父憂去,復遭繼母喪。

九年秋,兵部尚書張鳳翼卒,帝顧廷臣無可任者,即家起嗣昌。三疏辭,不許。明年三月抵京,召對。嗣昌通籍後,積歲林居,博涉文籍,多識先朝故事,工筆札,有口辨。帝與語,大信愛之。鳳翼故柔靡,兵事無所區畫。嗣昌銳意振刷,帝益以為能。每對必移時,所奏請無不聽,曰:「恨用卿晚。」嗣昌乃議大舉平賊。請以陜西、河南、湖廣、江北為四正,四巡撫分剿而專防;以延綏、山西、山東、江南、江西、四川為六隅,六巡撫分防而協剿;是謂十面之網。而總督、總理二臣,隨賊所向,專征討。福建巡撫熊文燦者,討海賊有功,大言自詭足辦賊。嗣昌聞而善之。會總督洪承疇、王家楨分駐陜西、河南。家楨故庸材,不足任,嗣昌乃薦文燦代之。因議增兵十二萬,增餉二百八十萬。其措餉之策有四,曰因糧,曰溢地,曰事例,曰驛遞。因糧者,因舊額之糧,量為加派,畝輸糧六合,石折銀八錢,傷地不與,歲得銀百九十二萬九千有奇;溢地者,民間土田溢原額者,核實輸賦,歲得銀四十萬六千有奇;事例者,富民輸資為監生,一歲而止;驛遞者,前此郵驛裁省之銀,以二十萬充餉。議上,帝乃傳諭:「流寇延蔓,生民塗炭,不集兵無以平寇,不增賦無以餉兵。勉從廷議,暫累吾民一年,除此腹心大患。其改因糧為均輸,布告天下,使知為民去害之意。」尋議諸州縣練壯丁捍本土,詔撫按飭行。

賊攻淅川,左良玉不救,城陷。山西總兵王忠援河南,稱疾不進,兵噪而歸。嗣昌請逮戮失事諸帥,以肅軍令,遂逮忠及故總兵張全昌。良玉以六安功,落職戴罪自贖。

嗣昌既建「四正六隅」之說,欲專委重文燦,文燦顧主撫議,與前策牴牾。帝譙讓文燦,嗣昌亦心望。既已任之,則曲為之解,乃上疏曰:「網張十面,必以河南、陜西為殺賊之地。然陜有李自成、惠登相等,大部未能剿絕,法當驅關東賊不使合,而使陜撫斷商、雒,鄖撫斷鄖、襄,安撫斷英、六,鳳撫斷亳、潁,而應撫之軍出靈、陜,保撫之軍渡延津。然後總理提邊兵,監臣提禁旅,豫撫提陳永福諸軍,并力合剿。若關中大賊逸出關東,則秦督提曹變蛟等出關協擊。期三月盡諸劇寇。巡撫不用命,立解其兵柄,簡一監司代之;總兵不用命,立奪其帥印,簡一副將代之;監司、副將以下,悉以尚方劍從事。則人人效力,何賊不平。」乃剋今年十二月至明年二月為滅賊之期。帝可其奏。

是時,賊大入四川,朝士尤洪承疇縱賊。嗣昌因言於帝曰:「熊文燦在事三月,承疇七年不效。論者繩文燦急,而承疇縱寇莫為言。」帝知嗣昌有意左右之,變色曰:「督、理二臣但責成及時平賊,奈何以久近藉之口!」嗣昌乃不敢言。文燦既主撫議,所加餉天子遣一侍郎督之,本藉以剿賊,文燦悉以資撫。帝既不復詰,廷臣亦莫言之。

至明年三月,嗣昌以滅賊踰期,疏引罪,薦人自代。帝不許,而命察行間功罪,乃上疏曰:「洪承疇專辦秦賊,賊往來秦、蜀自如,剿撫俱無功,不免於罪。熊文燦兼辦江北、河南、湖廣賊,撫劉國能、張獻忠,戰舞陽、光山,剿撫俱有功,應免罪。諸巡撫則河南常道立、湖廣餘應桂有功,陜西孫傳庭、山西宋賢、山東顏繼祖、保定張其平、江南張國維、江西解學龍、浙江喻思恂有勞,鄖陽戴東旻無功過,鳳陽朱大典、安慶史可法宜策勵圖功。總兵則河南左良玉有功,陜西曹變蛟、左光先無功,山西虎大威、山東倪寵、江北牟文綬、保定錢中選有勞無功,河南張任學、寧夏祖大弼無功過。承疇宜遣逮,因軍民愛戴,請削宮保、尚書,以侍郎行事。變蛟、光先貶五秩,與大弼期五月平賊,踰期並承疇逮治。大典貶三秩,可法戴罪自贖。」議上,帝悉從之。

嗣昌既終右文燦,而文燦實不知兵。既降國能、獻忠,謂撫必可恃。嗣昌亦陰主之,所請無不曲徇,自是不復言「十面張網」之策矣。是月,帝御經筵畢,嗣昌奏對有「善戰服上刑」等語,帝怫然,詰之曰:「今天下一統,非戰國兵爭比。小醜跳梁,不能伸大司馬九伐之法,奈何為是言?」嗣昌慚。

當是時,流賊既大熾,朝廷又有東顧憂,嗣昌復陰主互市策。適太陰掩熒惑,帝減膳修省,嗣昌則歷引漢永平、唐元和、宋太平興國事,蓋為互市地云。給事中何楷疏駁之,給事中錢增、御史林蘭友相繼論列,帝不問。

六月,改禮部尚書兼東閣大學士,入參機務,仍掌兵部事。嗣昌既以奪情入政府,又奪情起陳新甲總督,於是楷、蘭友及少詹事黃道周抗疏詆斥,修撰劉同升、編修越士春繼之。帝怒,並鐫三級,留翰林。刑部主事張若麒上疏醜詆道周,遂鐫道周六級,並同升、士春皆謫外。已而南京御史成勇、兵部尚書範景文等言之,亦獲譴。嗣昌自是益不理於人口。

我大清兵入墻子嶺、青口山,薊遼保定總督吳阿衡方醉,不能軍,敗死。京城戒嚴,召盧象升帥師入衛。象升主戰,嗣昌與監督中官高起潛主款,議不合,交惡。編修楊廷麟劾嗣昌誤國,嗣昌怒,改廷麟職方主事監象升軍,而戒諸將毋輕戰。諸將本恇怯,率藉口持重觀望,所在列城多破。嗣昌據軍中報,請旨授方略。比下軍前,則機宜已變,進止乖違,疆事益壞云。象升既陣亡,嗣昌亦貶三秩,戴罪視事。

十二年正月,濟南告陷,德王被執,游騎北抵兗州。二月,大清兵北旋,給事中李希沆言:「聖明御極以來,北兵三至。己巳之罪未正,致有丙子;丙子之罪未正,致有今日。」語侵嗣昌。御史王志舉亦劾嗣昌誤國四大罪,請用丁汝夔、袁崇煥故事。帝怒,希沆貶秩,志舉奪官。初,帝以嗣昌才而用之,非廷臣意,知其必有言,言者輒斥。嗣昌既有罪,帝又數逐言官,中外益不平。嗣昌亦不自安,屢疏引罪,乃落職冠帶視事。未幾,以敘功復之。

先是,京師被兵,樞臣皆坐罪。二年,王洽下獄死,復論大辟。九年,張鳳翼出督師,服毒死,猶削籍。及是,亡七十餘城,而帝眷嗣昌不衰。嗣昌乃薦四川巡撫傅宗龍自代。帝命嗣昌議文武諸臣失事罪,分五等:曰守邊失機,曰殘破城邑,曰失陷籓封,曰失亡主帥,曰縱敵出塞。於是中官則薊鎮總監鄧希詔、分監孫茂霖,巡撫則順天陳祖苞、保定張其平、山東顏繼祖,總兵則薊鎮吳國俊、陳國威,山東倪寵,援剿祖寬、李重鎮及他副將以下,至州縣有司,凡三十六人,同日棄市。而嗣昌貶削不及,物議益譁。

當戒嚴時,廷臣多請練邊兵。嗣昌因定議:宣府、大同、山西三鎮兵十七萬八千八百有奇,三總兵各練萬,總督練三萬,以二萬駐懷來,一萬駐陽和,東西策應。余授鎮監、巡撫以下分練。延綏、寧夏、甘肅、固原、臨兆五鎮兵十五萬五千七百有奇,五總兵各練萬,總督練三萬,以二萬駐固原,一萬駐延安,東西策應。余授巡撫、副將以下分練。遼東、薊鎮兵二十四萬有奇,五總兵各練萬,總督練五萬,外自錦州,內抵居庸,東西策應。余授鎮監、巡撫以下分練。汰通州、昌平督治二侍郎,設保定一總督,合畿輔、山東、河北兵,得十五萬七千有奇,四總兵各練二萬,總督練三萬,北自昌平,南抵河北,聞警策應。余授巡撫以下分練。又以畿輔重地,議增監司四人。於是大名、廣平、順德增一人,真定、保定、河間各一人。薊遼總督下增監軍三人。議上,帝悉從之。嗣昌所議兵凡七十三萬有奇,然民流餉絀,未嘗有實也。

帝又採副將楊德政議,府汰通判,設練備,秩次守備,州汰判官,縣汰主簿,設練總,秩次把總,並受轄於正官,專練民兵。府千,州七百,縣五百,捍鄉土,不他調。嗣昌以勢有緩急,請先行畿輔、山東、河南、山西,從之。於是有練餉之議。初,嗣昌增剿餉,期一年而止。後餉盡而賊未平,詔徵其半。至是,督餉侍郎張伯鯨請全徵。帝慮失信,嗣昌曰:「無傷也,加賦出於土田,土田盡歸有力家,百畝增銀三四錢,稍抑兼并耳。」大學士薛國觀、程國祥皆贊之。於是剿餉外復增練餉七百三十萬。論者謂:「九邊自有額餉,概予新餉,則舊者安歸?邊兵多虛額,今指為實數,餉盡虛糜,而練數仍不足。且兵以分防不能常聚,故有抽練之議,抽練而其餘遂不問。且抽練仍虛文,邊防愈益弱。至州縣民兵益無實,徒糜厚餉。」以嗣昌主之,事鉅莫敢難也。神宗末增賦五百二十萬,崇禎初再增百四十萬,總名遼餉。至是,復增剿餉、練餉,額溢之。先後增賦千六百七十萬,民不聊生,益起為盜矣。

五月,熊文燦所撫賊張獻忠反穀城,羅汝才等九營皆反。八月,傅宗龍抵京,嗣昌解部務,還內閣。未幾,羅犬英山敗書聞。帝大驚,詔逮文燦。特旨命嗣昌督師,賜尚方劍,以便宜誅賞。九月朔,召見平臺。嗣昌曰:「君言不宿於家,臣朝受命,夕啟行,軍資甲仗望敕所司遄發。」帝悅,曰:「卿能如此,朕復何憂。」翊日,賜白金百、大紅絺絲四表裏、斗牛衣一、賞功銀四萬、銀牌千五百、幣帛千。嗣昌條七事以獻,悉報可。四日召見賜宴,手觴三爵,御製贈行詩一章。嗣昌跪誦,拜且泣。越二日,陛辭,賜膳。二十九日抵襄陽,入文燦軍。文燦就逮,嗣昌猶為疏辯云。

十月朔,嗣昌大誓三軍,督理中官劉元斌,湖廣巡撫方孔炤,總兵官左良玉、陳洪範等畢會。賊賀一龍等掠葉,圍沈丘,焚項城之郛,寇光山。副將張琮、刁明忠率京軍踰山行九十里,及其巢。先驅射賊,殪絳袍而馳者二人,追奔四十里,斬首千七百五十。嗣昌稱詔頒賜。十一月,興世王王國寧以眾千人來歸,受之於襄陽,處其妻子樊城。表良玉平賊將軍。諸將積驕玩,無鬥志。獻忠、羅汝才、惠登相等八營遁鄖陽、興安山間,掠南漳、穀城、房、竹山、竹谿。嗣昌鞭刁明忠,斬監軍僉事殷大白以徇。檄巡撫方孔炤遣楊世恩、羅萬邦剿汝才、登相,全軍覆於香油坪。嗣昌劾逮孔炤,奏辟永州推官萬元吉為軍前監紀,從之。

當是時,李自成潛伏陜右,賀一龍、左金王等四營跳梁漢東,嗣昌專剿獻忠。獻忠屢敗於興安,求撫,不許。其黨托天王常國安、金翅鵬劉希原來降,獻忠走入川,良玉追之。嗣昌牒令還,良玉不從。十三年二月七日,與陜西副將賀人龍、李國奇夾擊獻忠於瑪瑙山,大破之,斬馘三千六百二十,墜巖谷死者無算。其黨掃地王曹威等授首,十反王楊友賢率眾降。是月也,帝念嗣昌,發銀萬兩犒師,賜斗牛衣、良馬、金鞍各二。使者甫出國門,而瑪瑙山之捷至,大悅,再發銀五萬,幣帛千犒師。論功,加太子少保。而湖廣將張應元、汪之鳳敗賊水石壩,獲其軍師。四川將張令、方國安敗之千江河。李國奇、賀人龍等敗之寒溪寺、鹽井。川、陜、湖廣諸將畢集,復連敗之黃墩、木瓜溪,軍聲大振。汝才、登相求撫,獻忠持之,斂兵南漳、遠安間,殺安撫官姚宗中,走大寧、大昌,犯巫山,為川中患。獻忠遁興安、平利山中,良玉圍而不攻,賊得收散亡,由興安、房縣走白羊山而西,與汝才等合。嗣昌以群賊合,其勢復張,乃由襄陽赴夷陵,扼其要害。帝念嗣昌行間勞苦,賜敕發賞功銀萬,賜鞍馬二。罷鄖陽撫治王鰲永,詔廢將猛如虎軍前立功。黃得功、宋紀大破賊商城,賀一龍五大部降而復叛。鄭嘉棟、賀人龍大破汝才、登相開縣。汝才偕小秦王東奔,登相越開縣而西,自是二賊始分。

當是時,諸部士馬居山谷,罹炎暑瘴毒,物故十二三。京兵之在荊門、雲南兵之在簡坪、湖廣兵之在馬蝗坡者,久屯思歸,夜亡多。關河大旱,人相食,土寇蜂起,陜西竇開遠、河南李際遇為之魁,饑民從之,所在告警。嗣昌以聞。帝發帑金五萬,營醫藥,責諸將進兵。而陜之長武,川之新寧、大竹,湖廣之羅田又相繼報陷。嗣昌乃下招撫令,為諭帖萬紙,散之賊中。七月,監軍孔貞會等大破汝才豐邑坪。其黨混世王、小秦王率其下降,賊魁整十萬及登相、王光恩亦相繼降,於是群賊盡萃於蜀中。嗣昌遂入川,以八月泛舟上,謂川地阨塞,諸軍合而蹙之,可盡殄。而人龍以秦師自開縣噪而西歸,應元等敗績於夔之土地嶺,獻忠勢復張,汝才與之合。聞督師西,遂急趨大昌,犯觀音巖,守將邵仲光不能禦,遂突凈壁,陷大昌。嗣昌斬仲光,劾逮四川巡撫邵捷春。賊遂渡河至通江,嗣昌至萬縣。賊攻巴州不下,嗣昌至梁山,檄諸將分擊。賊已陷劍州,趨保寧,將由間道入漢中。趙光遠、賀人龍拒之,賊乃轉掠,陷梓潼、昭化,抵綿州,將趨成都。十一月,嗣昌至重慶。賊攻羅江,不克,走綿竹。嗣昌至順慶,諸將不會師。賊轉掠至漢州,去中江百里,守將方國安避之去,賊遂縱掠什邡、綿竹、安縣、德陽、金堂間,所至空城而遁,全蜀大震。賊遂由水道下簡州、資陽。嗣昌徵諸將合擊,皆退縮。屢征良玉兵,又不至。賊遂陷榮昌、永川。十二月,陷瀘州。

自賊再入川,諸將無一邀擊者。嗣昌雖屢檄,令不行。其在重慶也,下令赦汝才罪,降則授官,惟獻忠不赦,擒斬者賚萬金,爵侯。翌日,自堂皇至庖湢,遍題「有斬督師獻者,賚白金三錢」,嗣昌駭愕,疑左右皆賊,勒三日進兵。會雨雪道斷,復戒期。三檄人龍,不奉令。初,嗣昌表良玉平賊將軍,良玉浸驕,欲貴人龍以抗之。既以瑪瑙山功不果,人龍慍,反以情告良玉,良玉亦慍,語載良玉、人龍傳。

嗣昌雖有才,然好自用,躬親簿書,過於繁碎。軍行必自裁進止,千里待報,坐失機會。王鰲永嘗諫之,不納。及鰲永罷官,上書於朝曰:「嗣昌用師一年,蕩平未奏,此非謀慮之不長,正由操心之太苦也。天下事,總挈大綱則易,獨周萬目則難。況賊情瞬息更變,今舉數千里征伐機宜,盡出嗣昌一人,文牒往返,動踰旬月,坐失事機,無怪乎經年之不戰也。其間能自出奇者,惟瑪瑙山一役。若必遵督輔號令,良玉當退守興安,無此捷矣。臣以為陛下之任嗣昌,不必令其與諸將同功罪,但責其提衡諸將之功罪。嗣昌之馭諸將,不必人人授以機宜,但核其機宜之當否,則嗣昌心有餘閒,自能決奇制勝。何至久延歲月,老師糜餉為哉?」先是,嗣昌以諸將進止不一,納幕下評事元吉言,用猛如虎為總統,張應元副之。比賊入瀘州,如虎及賀人龍、趙光遠軍至,賊復渡南溪,越成都,走漢州、德陽、綿州、劍州、昭化至廣元,又走巴州、達州。諸軍疲極,惟如虎軍躡其後。十四年正月,嗣昌知賊必出川,遂統舟師下雲陽,檄諸軍陸行追賊。人龍軍既噪而西,頓兵廣元不進,所恃惟如虎。比與賊戰開縣、黃陵城,大敗,將士死亡過半。如虎突圍免,馬騾關防盡為賊有。

初,賊竄南溪,元吉欲從間道出梓潼,扼歸路以待賊。嗣昌檄諸軍躡賊疾追,不得拒賊遠,令他逸。諸將乃盡從瀘州逐後塵。賊折而東返,歸路盡空,不可復遏,嗣昌始悔不用元吉言。賊遂下夔門,抵興山,攻當陽,犯荊門。嗣昌至夷陵,檄良玉兵,使十九返。良玉撤興、房兵趨漢中,若相避然。賊所至,燒驛舍,殺塘卒,東西消息中斷。鄖陽撫治袁繼咸聞賊至當陽,急謀發兵。獻忠令汝才與相持,而自以輕騎一日夜馳三百里,殺督師使者於道,取軍符。以二月十一日抵襄陽近郊,用二十八騎持軍符先馳呼城門督師調兵,守者合符而信,入之。夜半從中起,城遂陷。

獻忠縛襄王置堂下,屬之酒,曰:「吾欲斷楊嗣昌頭,嗣昌在遠。今借王頭,俾嗣昌以陷籓伏法。王努力盡此酒。」遂害之。未幾,渡漢水,走河南,與賀一龍、左金王諸賊合。嗣昌初以襄陽重鎮,仞深溝方洫而三環之,造飛梁,設橫互,陳利兵而譏訶,非符要合者不得渡。江、漢間列城數十,倚襄陽為天險,賊乃出不意而破之。嗣昌在夷陵,驚悸,上疏請死,下至荊州之沙市,聞洛陽已於正月被陷,福王遇害,益憂懼,遂不食。以三月朔日卒,年五十四。

廷臣聞襄陽之變,交章論列,而嗣昌已死矣。繼咸及河南巡按高名衡以自裁聞,其子則以病卒報,莫能明也。帝甚傷悼之,命丁啟睿代督師。傳諭廷臣:「輔臣二載辛勞,一朝畢命,然功不掩過,其議罪以聞。」定國公徐允禎等請以失陷城寨律議斬。上傳制曰;「故輔嗣昌奉命督剿,無城守專責,乃詐城夜襲之檄,嚴飭再三,地方若罔聞知。及違制陷城,專罪督輔,非通論。且臨戎二載,屢著捷功,盡瘁殞身,勤勞難泯。」乃昭雪嗣昌罪,賜祭,歸其喪於武陵。嗣昌先以剿賊功進太子少傅,既死,論臨、藍平盜功,進太子太傅。廷臣猶追論不已,帝終念之。後獻忠陷武陵,心恨嗣昌,發其七世祖墓,焚嗣昌夫婦柩,斷其屍見血,其子孫獲半體改葬焉。

吳甡,字鹿友,揚州興化人。萬曆四十一年進士。歷知邵武、晉江、濰縣。天啟二年征授御史。初入臺,趙南星擬以年例出之,甡乃薦方震孺等,而追論崔文昇、李可灼罪,遂得留。後又諫內操宜罷,請召還鄒元標、馮從吾、文震孟,乃積與魏忠賢忤。七年二月削其籍。

崇禎改元,起故官。溫體仁訐錢謙益,周延儒助之。甡恐帝即用二人,言枚卜大典當就廷推中簡用,事乃止。時大治忠賢黨,又值京察,甡言此輩罪惡非考功法所能盡,宜先定其罪,毋混察典。御史任贊化以劾體仁謫,甡論救,而力詆王永光媚璫,請罷黜。皆不納。出按河南。妖人聚徒劫村落,甡遍捕賊魁誅之。奉命振延綏饑,因諭散賊黨。帝聞,即命按陜西。劾大將杜文煥冒功,置之法。數為民請命,奏無不允。遷大理寺丞,進左通政。

七年九月,超擢右僉都御史,巡撫山西。甡歷陳防禦、邊寇、練兵、恤民四難,及議兵、議將、議餉、議用人四事。每歲暮扼河防秦、豫賊,連三歲,無一賊潛渡,以閒修築邊牆。八年四月上疏言:「晉民有三苦:一苦凶荒,無計糊口;一苦追呼,無力輸租;一苦殺掠,無策保全。由此悉為盜,請蠲最殘破地十州縣租。」帝即敕議行。戶部請稅間架,甡力爭,弗聽。其秋,我大清平察哈爾國,旋師略朔州,直抵忻、代,守將屢敗。總督楊嗣昌遣副將自代州往偵,亦敗走。甡鐫五級,嗣昌及大同巡撫葉廷桂鐫三級,俱戴罪視事。先是,定襄縣地震者再,甡曰:「此必有東師也。」飭有司繕守具,已而果入。定襄以有備,獨不被兵。山西大盜賀宗漢、劉浩然、高加計皆前巡撫戴君恩所撫,擁眾自恣。甡陽為撫慰,而密令參將虎大威、劉光祚等圖之,以次皆被殲。甡行軍樹二白旗,脅從及老弱婦女跪其下,即免死,全活甚眾。在晉四年,軍民戴若慈母。謝病歸。

十一年二月,起兵部左侍郎。其冬,尚書楊嗣昌言邊關戒嚴,甡及添注侍郎惠世楊久不至,請改推。帝怒,落職閒住。十三年冬起故官,明年命協理戎政。帝嘗問京營軍何以使練者盡精,汰者不嘩,甡對曰:「京營邊勇營萬二千專練騎射,壯丁二萬專練火器,廩給厚而技與散兵無異。宜行分練法,技精者,散兵拔為邊勇,否則邊勇降為散兵,壯丁亦然。老弱者汰補,革弊當以漸,不可使知有汰兵意。」帝然之。又問別立戰營,能得堪戰者五萬否,甡對:「京營兵合堪戰。承平日久,發兵剿賊,輒沿途雇充。將領利月餉,游民利剽敚,歸營則本軍復充伍。今練兵法要在選將,有戰將自有戰兵,五萬非難。但法忌紛更,不必別立戰營也。」帝顧兵部尚書陳新甲,令速選將,而諭甡具疏以聞。賜果餌,拜謝出。

十五年六月,擢禮部尚書兼東閣大學士。周延儒再相,馮銓力為多,延儒許復其冠帶。銓果以捐資振饑屬撫按題敘,延儒擬優旨下戶部。公議大沸,延儒患之。馮元飆為甡謀,說延儒引甡共為銓地,延儒默援之,甡遂得柄用。及延儒語銓事,甡唯唯,退召戶部尚書傅淑訓,告以逆案不可翻,寢其疏不覆。延儒始悟為甡紿。延儒欲起張捷為南京右都御史,甡力尼之。甡居江北,延儒居江南,各樹黨。延儒引用錦衣都督駱養性,甡持不可。後帝論諸司弊竇,甡言錦衣尤甚,延儒亦言緹騎之害,帝並納之。

十六年三月,帝以襄陽、荊州、承天連陷,召對廷臣,隕涕謂甡曰:「卿向歷巖疆,可往督湖廣師。」甡具疏請得精兵三萬,自金陵趨武昌,扼賊南下。帝方念湖北,覽疏不悅,留中。甡請面對,帝御昭文閣,諭以所需兵多,猝難集。南京隔遠,不必退守。甡奏:「左良玉跋扈甚,督師嗣昌九檄徵兵,一旅不發。臣不如嗣昌,而良玉踞江、漢甚於曩時,臣節制不行,徒損威重。南京從襄陽順流下,窺伺甚易,宜兼顧,非退守。」大學士陳演言:「督師出,則督、撫兵皆其兵。」甡言:「臣請兵,正為督、撫無兵耳。使臣束手待賊,事機一失,有不忍言者。」帝乃令兵部速議發兵。尚書張國維請以總兵唐通、馬科及京營兵共一萬畀甡,又言此兵方北征,俟敵退始可調。帝命姑俟之。甡屢請,帝曰:「徐之,敵退兵自集,卿獨往何益?」踰月,延儒出督師,朝受命,夕啟行。蔣德璟謂倪元璐曰:「上欲吳公速行,緩言相慰者,試之耳,觀首輔疾趨可見。」甡卒遲回不肯行。部所撥唐通兵,演又請留,云關門不可無備。甡不得已,以五月辭朝。先一日出勞從騎,帝猶命中官賜銀牌給賞,越宿忽下詔責其逗遛,命輟行入直。甡惶恐,兩疏引罪,遂許致仕。既行,演及駱養性交構之,帝益怒。至七月,親鞫吳昌時,作色曰:「兩輔臣負朕,朕待延儒厚,乃納賄行私,罔知國法。命甡督師,百方延緩,為委卸地。延儒被糾,甡何獨無?」既而曰:「朕雖言,終必無糾者,錦衣衛可宣甡候旨。」甡入都,敕法司議罪。十一月,遣戍金齒。南京兵部尚書史可法馳疏救,不從。

明年,行次南康,聞都城變。未幾,福王立於南京,赦還,復故秩。吏部尚書張慎言議召用甡,為勛臣劉孔昭等所阻。國變後,久之,卒於家。

贊曰:明季士大夫問錢穀不知,問甲兵不知,於是嗣昌得以才顯。然迄無成功者,得非功罪淆於愛憎,機宜失於遙制故耶?吳甡按山右有聲,及為相,遂不能有為。進不以正,其能正邦乎?抑時勢實難,非命世材,固罔知攸濟也。


\end{pinyinscope}