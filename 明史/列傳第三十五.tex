\article{列傳第三十五}

\begin{pinyinscope}
解縉黃淮胡廣金幼孜胡儼

解縉,字大紳,吉水人。祖子元,為元安福州判官。兵亂,守義死。父開,太祖嘗召見論元事。欲官之,辭去。

縉幼穎敏,洪武二十一年舉進士。授中書庶吉士,甚見愛重,常侍帝前。一日,帝在大庖西室,諭縉:「朕與爾義則君臣,恩猶父子,當知無不言。」縉即日上封事萬言,略曰:

臣聞令數改則民疑,刑太繁則民玩。國初至今,將二十載,無幾時不變之法,無一日無過之人。嘗聞陛下震怒,鋤根剪蔓,誅其姦逆矣。未聞褒一大善,賞延於世,復及其鄉,終始如一者也。

臣見陛下好觀《說苑》、《韻府》雜書與所謂《道德經》、《心經》者,臣竊謂甚非所宜也。《說苑》出於劉向,多戰國縱橫之論;《韻府》出元之陰氏,抄輯穢蕪,略無可採。陛下若喜其便於檢閱,則願集一二志士儒英,臣請得執筆隨其後,上溯唐、虞、夏、商、周、孔,下及關、閩、濂、洛。根實精明,隨事類別,勒成一經,上接經史,豈非太平制作之一端歟?又今《六經》殘缺。《禮記》出於漢儒,踳駁尤甚,宜及時刪改。訪求審樂之儒,大備百王之典,作樂書一經以惠萬世。尊祀伏羲、神農、黃帝、堯、舜、禹、湯、文、武、皋陶、伊尹、太公、周公、稷、契、夷、益、傅說、箕子於太學。孔子則自天子達於庶人,通祀以為先師,而以顏、曾、子思、孟子配。自閔子以下,各祭於其鄉。魯之闕里,仍建叔梁紇廟,贈以王爵,以顏路、曾曨、孔鯉配。一洗歷代之因仍,肇起天朝之文獻,豈不盛哉!若夫祀天宜復掃地之規,尊祖宜備七廟之制。奉天不宜為筵宴之所,文淵未備夫館閣之隆。太常非俗樂之可肄,官妓非人道之所為。禁絕倡優,易置寺閹。執戟陛墀,皆為吉士;虎賁趣馬,悉用俊良。除山澤之禁稅,蠲務鎮之徵商。木輅朴居,而土木之工勿起;布墾荒田,而四裔之地勿貪。釋、老之壯者驅之,俾復於人倫;經咒之妄者火之,俾絕其欺誑。絕鬼巫,破淫祀,省冗官,減細縣。痛懲法外之威刑,永革京城之工役。流十年而聽復,杖八十以無加。婦女非帷薄不修,毋令逮繫;大臣有過惡當誅,不宜加辱。治曆明時,授民作事,但申播植之宜,何用建除之謬。所宜著者,日月之行,星辰之次。仰觀俯察,事合逆順。七政之齊,正此類也。

近年以來,臺綱不肅。以刑名輕重為能事,以問囚多寡為勳勞,甚非所以勵清要、長風采也。御史糾彈,皆承密旨。每聞上有赦宥,則必故為執持。意謂如此,則上恩愈重。此皆小人趨媚效勞之細術,陛下何不肝膽而鏡照之哉?陛下進人不擇賢否,授職不量重輕。建不為君用之法,所謂取之盡錙銖;置朋姦倚法之條,所謂用之如泥沙。監生進士,經明行修,而多屈於下僚;孝廉人材,冥蹈瞽趨,而或布於朝省。椎埋嚚悍之夫,闒茸下愚之輩。朝捐刀鑷,暮擁冠裳。左棄筐篋,右綰組符。是故賢者羞為之等列,庸人悉習其風流。以貪婪茍免為得計,以廉潔受刑為飾辭。出於吏部者無賢否之分,入於刑部者無枉直之判。天下皆謂陛下任喜怒為生殺,而不知皆臣下之乏忠良也。

古者善惡,鄉鄰必記。今雖有申明旌善之舉,而無黨庠鄉學之規。互知之法雖嚴,訓告之方未備。臣欲求古人治家之禮,睦鄰之法,若古藍田呂氏之《鄉約》,今義門鄭氏之家範,布之天下。世臣大族,率先以勸,旌之復之,為民表帥。將見作新於變,至於比屋可封不難矣。

陛下天資至高,合於道微。神怪妄誕,臣知陛下洞矚之矣。然猶不免所謂神道設教者,臣謂不必然也。一統之輿圖已定矣,一時之人心已服矣,一切之姦雄已慴矣。天無變災,民無患害。聖躬康寧,聖子聖孫繼繼繩繩。所謂得真符者矣。何必興師以取寶為名,諭眾以神仙為徵應也哉。

臣觀地有盛衰,物有盈虛,而商稅之徵,率皆定額。是使其或盈也,姦黠得以侵欺;其歉也,良善困於補納。夏稅一也,而茶椒有糧,果絲有稅。既稅於所產之地,又稅於所過之津,何其奪民之利至於如此之密也!且多貧下之家,不免拋荒之咎。今日之土地,無前日之生植;而今日之徵聚,有前日之稅糧。或賣產以供稅,產去而稅存;或賠辦以當役,役重而民困。土田之高下不均,起科之輕重無別。膏腴而稅反輕,瘠鹵而稅反重。欲拯困而革其弊,莫若行授田均田之法,兼行常平義倉之舉。積之以漸,至有九年之食無難者。

臣聞仲尼曰:「王公設險以守其國。」近世狃於晏安,墮名城,銷鋒鏑,禁兵諱武,以為太平。一旦有不測之虞,連城望風而靡。及今宜敕有司整葺,寬之以歲月,守之以里胥,額設弓手,兼教民兵。開武舉以收天下之英雄,廣鄉校以延天下之俊乂。古時多有書院學田,貢士有莊,義田有族,皆宜興復而廣益之。

夫罪人不孥,罰弗及嗣。連坐起於秦法,孥戮本於偽書。今之為善者妻子未必蒙榮,有過者里胥必陷其罪。況律以人倫為重,而有給配婦女之條,聽之於不義,則又何取夫節義哉。此風化之所由也。

孔子曰:「名不正則言不順。」尚書、侍郎,內侍也,而以加於六卿;郎中、員外,內職也,而以名於六屬。御史詞臣,所以居寵臺閣;郡守縣令,不應迴避鄉邦。同寅協恭,相倡以禮。而今內外百司捶楚屬官,甚於奴隸。是使柔懦之徒,蕩無廉恥,進退奔趨,肌膚不保。甚非所以長孝行、勵節義也。臣以為自今非犯罪惡解官,笞杖之刑勿用。催科督厲,小有過差,蒲鞭示辱,亦足懲矣。

臣但知罄竭愚衷,急於陳獻,略無次序,惟陛下幸垂鑒焉。書奏,帝稱其才。已,復獻《太平十策》,文多不錄。

縉嘗入兵部索皁隸,語嫚。尚書沈溍以聞。帝曰:「縉以冗散自恣耶。」命改為御史。韓國公李善長得罪死,縉代郎中王國用草疏白其冤。又為同官夏長文草疏,劾都御史袁泰。泰深銜之。時近臣父皆得入覲。縉父開至,帝謂曰:「大器晚成,若以而子歸,益令進學,後十年來,大用未晚也。」歸八年,太祖崩,縉入臨京師。有司劾縉違詔旨,且母喪未葬,父年九十,不當舍以行。謫河州衛吏。時禮部侍郎董倫方為惠帝所信任,縉因寓書於倫曰:「縉率易狂愚,無所避忌,數上封事,所言分封勢重,萬一不幸,必有厲長、吳濞之虞。冉阜哈術來歸,欽承顧問,謂宜待之有禮,稍忤機權,其徒必貳。此類非一,頗皆億中。又嘗為王國用草諫書,言韓國事,為詹徽所疾,欲中以危法。伏蒙聖恩,申之慰諭,重以鏹賜,令以十年著述,冠帶來廷。《元史》舛誤,承命改修,及踵成《宋書》,刪定《禮經》,凡例皆已留中。奉親之暇,杜門纂述,漸有次第,洊將八載。賓天之訃忽聞,痛切欲絕。母喪在殯,未遑安厝。家有九十之親,倚門望思,皆不暇戀。冀一拜山陵,隕淚九土。何圖詿誤,蒙恩遠行。揚、粵之人,不耐寒暑,復多疾病。俯仰奔趨,伍於吏卒,誠不堪忍。晝夜涕泣,恆懼不測。負平生之心,抱萬古之痛。是以數鳴知感。冀還京師,得望天顏,或遂南還,父子相見,即更生之日也。」倫乃薦縉,召為翰林待詔。

成祖入京師,擢侍讀。命與黃淮、楊士奇、胡廣、金幼孜、楊榮、胡儼並直文淵閣,預機務。內閣預機務自此始。

尋進侍讀學士,奉命總裁《太祖實錄》及《列女傳》。書成,賜銀幣。永樂二年,皇太子立,進縉翰林學士兼右春坊大學士。帝嘗召縉等曰:「爾七人朝夕左右,朕嘉爾勤慎,時言之宮中。恒情,慎初易,保終難,願共勉焉。」因各賜五品服,命七人命婦朝皇后於柔儀殿,后勞賜備至。又以立春日賜縉等金綺衣,與尚書埒。縉等入謝,帝曰:「代言之司,機密所繫,且旦夕侍朕,裨益不在尚書下也。」一日,帝御奉天門,諭六科諸臣直言,因顧縉等曰:「王、魏之風,世不多有。若使進言者無所懼,聽言者無所忤,天下何患不治?朕與爾等共勉之。」其年秋,胡儼出為祭酒,縉等六人從容獻納。帝嘗虛己以聽。

縉少登朝,才高,任事直前,表裏洞達。引拔士類,有一善稱之不容口。然好臧否,無顧忌,廷臣多害其寵。又以定儲議,為漢王高煦所忌,遂致敗。先是,儲位未定,淇國公邱福言漢王有功,宜立。帝密問縉。縉稱:「皇長子仁孝,天下歸心。」帝不應。縉又頓首曰:「好聖孫。」謂宣宗也。帝頷之。太子遂定。高煦由是深恨縉。會大發兵討安南,縉諫。不聽。卒平之,置郡縣。而太子既立,又時時失帝意。高煦寵益隆,禮秩踰嫡。縉又諫曰:「是啟爭也,不可。」帝怒,謂其離間骨肉,恩禮浸衰。四年,賜黃淮等五人二品紗羅衣,而不及縉。久之,福等議稍稍傳達外廷,高煦遂譖縉洩禁中語。明年,縉坐廷試讀卷不公,謫廣西布政司參議。既行,禮部郎中李至剛言縉怨望,改交阯,命督餉化州。

永樂八年,縉奏事入京,值帝北征,縉謁皇太子而還。漢王言縉伺上出,私覲太子,徑歸,無人臣禮。帝震怒。縉時方偕檢討王偁道廣東,覽山川,上疏請鑿贛江通南北。奏至,逮縉下詔獄,拷掠備至。詞連大理丞湯宗,宗人府經歷高得抃,中允李貫,贊善王汝玉,編修朱紘,檢討蔣驥、潘畿、蕭引高并及至剛,皆下獄。汝玉、貫、紘、引高、得抃皆瘐死。十三年,錦衣衛帥紀綱上囚籍,帝見縉姓名曰:「縉猶在耶?」綱遂醉縉酒,埋積雪中,立死。年四十七。籍其家,妻子宗族徙遼東。

方縉居翰林時,內官張興恃寵笞人左順門外,縉叱之,興斂手退。帝嘗書廷臣名,命縉各疏其短長。縉言:「蹇義天資厚重,中無定見。夏原吉有德量,不遠小人。劉俊有才幹,不知顧義。鄭賜可謂君子,頗短於才。李至剛誕而附勢,雖才不端。黃福秉心易直,確有執守。陳瑛刻於用法,尚能持廉。宋禮戇直而苛,人怨不恤。陳洽疏通警敏,亦不失正。方賓簿書之才,駔儈之心。」帝以付太子,太子因問尹昌隆、王汝玉。縉對曰:「昌隆君子而量不弘。汝玉文翰不易得,惜有市心耳。」後仁宗即位,出縉所疏示楊士奇曰:「人言縉狂,觀所論列,皆有定見,不狂也。」詔歸縉妻子宗族。

縉初與胡廣同侍成祖宴。帝曰:「爾二人生同里,長同學,仕同官。縉有子,廣可以女妻之。」廣頓首曰:「臣妻方娠,未卜男女。」帝笑曰:「定女矣。」已而果生女,遂約婚。縉敗,子禎亮徙遼東,廣欲離婚。女截耳誓曰:「薄命之婚,皇上主之,大人面承之,有死無二。」及赦還,卒歸禎亮。

正統元年八月,詔還所籍家產。成化元年,復縉官,贈朝議大夫。始縉言漢王及安南事得禍。後高煦以叛誅。安南數反,置吏未久,復棄去。悉如縉言。

縉兄綸,洪武中亦官御史。性剛直。後改應天教授。子禎期,以書名。

黃淮,字宗豫,永嘉人。父性,方國珍據溫州,遁跡避偽命。淮舉洪武末進士,授中書舍人。成祖即位,召對稱旨,命與解縉常立御榻左,備顧問。或至夜分,帝就寢,猶賜坐榻前語,機密重務悉預聞。既而與縉等六人並直文淵閣,改翰林編修,進侍讀。議立太子,淮請立嫡以長。太子立,遷左庶子兼侍讀。永樂五年,解縉黜,淮進右春坊大學士。明年與胡廣、金幼孜、楊榮、楊士奇同輔導太孫。七年,帝北巡,命淮及蹇義、金忠、楊士奇輔皇太子監國。十一年,再北巡,仍留守。明年,帝征瓦剌還,太子遣使迎稍緩,帝重入高煦譖,悉征東宮官屬下詔獄,淮及楊溥、金問皆坐繫十年。

仁宗即位,復官。尋擢為通政使,兼武英殿大學士,與楊榮、金幼孜、楊士奇同掌內制。丁母憂,乞終制。不許。明年,進少保、戶部尚書,兼大學士如故。仁宗崩,太子在南京。漢王久蓄異志,中外疑懼,淮憂危嘔血。宣德元年,帝親征樂安,命淮居守。明年以疾乞休,許之。父性年九十,奉養甚歡。及性卒,賜葬祭,淮詣闕謝。值燈時,賜遊西苑,詔乘肩輿登萬歲山。命主會試,比辭歸,餞之太液池,帝為長歌送之,且曰:「朕生日,卿其復來。」明年入賀。英宗立,再入朝。正統十四年六月卒。年八十三,謚文簡。

淮性明果,達於治體。永樂中,長沙妖人李法良反。仁宗方監國,命豐城侯李彬討之。漢王忌太子有功,詭言彬不可用。淮曰:「彬,老將,必能滅賊,願急遣。」彬卒擒法良。又時有告黨逆者。淮言於帝曰:「洪武末年已有敕禁,不宜復理。」吏部追論「靖難」兵起時,南人官北地不即歸附者,當編戍。淮曰:「如是,恐示人不廣。」帝皆從之。阿魯台歸款,請得役屬吐蕃諸部。求朝廷刻金作誓詞,磨其金酒中,飲諸酋長以盟。眾議欲許之。淮曰:「彼勢分則易制,一則難圖矣。」帝顧左右曰:「黃淮論事,如立高岡,無遠不見。」西域僧大寶法王來朝,帝將刻玉印賜之,以璞示淮。淮曰:「朝廷賜諸番制敕,用『敕命』、『廣運』二寶。今此玉較大,非所以示遠人、尊朝廷。」帝嘉納。其獻替類如此。然量頗隘。同列有小過,輒以聞。或謂解縉之謫,淮有力焉。其見疏於宣宗也,亦謂楊榮言「淮病瘵,能染人」云。

胡廣,字光大,吉水人。父子祺,名壽昌,以字行。陳友諒陷吉安,太祖遣兵復之,將殺脅從者千餘人。子祺走謁帥,力言不可,得免。洪武三年,以文學選為御史,上書請都關中。帝稱善,遣太子巡視陜西。後以太子薨,不果。子祺出為廣西按察僉事,改知彭州。所至平冤獄,毀淫祀,修廢堰,民甚德之。遷延平知府,卒於任。廣,其次子也。建文二年,廷試。

時方討燕,廣對策有「親籓陸梁,人心搖動」語,帝親擢廣第一,賜名靖,授翰林修撰。

成祖即位,廣偕解縉迎附。擢侍講,改侍讀,復名廣。遷右春坊右庶子。永樂五年,進翰林學士,兼左春坊大學士。帝北征,與楊榮、金幼孜從。數召對帳殿,或至夜分。過山川阨塞,立馬議論,行或稍後,輒遣騎四出求索。嘗失道,脫衣乘驏馬渡河,水沒馬及腰以上,帝顧勞良苦。廣善書,每勒石,皆命書之。十二年再北征,皇長孫從,命廣與榮、幼孜軍中講經史。十四年,進文淵閣大學士,兼職如故。帝征烏思藏僧作法會,為高帝、高后薦福,言見諸祥異。廣乃獻《聖孝瑞應頌》,帝綴為佛曲,令宮中歌舞之。禮部郎中周訥請封禪,廣言其不可,遂不許。廣上《卻封禪頌》,帝益親愛之。

廣性縝密。帝前所言及所治職務,出未嘗告人。時人以方漢胡廣。然頗能持大體。奔母喪還朝,帝問百姓安否。對曰:「安,但郡縣窮治建文時姦黨,株及支親,為民厲。」帝納其言。十六年五月卒,年四十九。贈禮部尚書,謚文穆。文臣得謚,自廣始。喪還,過南京,太子為致祭。明年,官其子穜翰林檢討。仁宗立,加贈廣少師。

金幼孜,名善,以字行,新淦人。建文二年進士。授戶科給事中。成祖即位,改翰林檢討,與解縉等同直文淵閣,遷侍講。時翰林坊局臣講書東宮,皆先具經義,閣臣閱正,呈帝覽,乃進講。解縉《書》,楊士奇《易》,胡廣《詩》,幼孜《春秋》,因進《春秋要旨》三卷。

永樂五年,遷右諭德兼侍講,因諭吏部,直內閣諸臣胡廣、金幼孜等考滿,勿改他任。七年從幸北京。明年北征,幼孜與廣、榮扈行,駕駐清水源,有泉湧出。幼孜獻銘,榮獻詩,皆勞以上尊。帝重幼孜文學,所過山川要害,輒命記之。幼孜據鞍起草立就。使自瓦剌來,帝召幼孜等傍輿行,言敵中事,親倚甚。嘗與廣、榮及侍郎金純失道陷谷中。暮夜,幼孜墜馬,廣、純去不顧。榮為結鞍行,行又輒墜,榮乘以己騎,明日始達行在所。是夜,帝遣使十餘輩迹榮、幼孜,不獲。比至,帝喜動顏色。自後北徵皆從,所撰有北征前、後二《錄》。十二年命與廣、榮等纂《五經四書性理大全》,遷翰林學士。十八年與榮並進文淵閣大學士。

二十二年從北征,中道兵疲。帝以問群臣,莫敢對,惟幼孜言不宜深入,不聽。次開平,帝謂榮、幼孜曰:「朕夢神人語上帝好生者再,是何祥也?」榮、幼孜對曰:「陛下此舉,固在除暴安民。然火炎崑岡,玉石俱毀,惟陛下留意。」帝然之,即命草詔,招諭諸部。還軍至榆木川,帝崩。秘不發喪。榮訃京師,幼孜護梓宮歸。仁宗即位,拜戶部右侍郎兼文淵閣大學士。尋加太子少保兼武英殿大學士。是年十月命幼孜、榮、士奇會錄罪囚於承天門外。詔法司,錄重囚必會三學士,委寄益隆。帝御西角門閱廷臣制誥,顧三學士曰:「汝三人及蹇、夏二尚書,皆先帝舊臣,朕方倚以自輔。嘗見前代人主惡聞直言,雖素所親信,亦畏威順旨,緘默取容。賢良之臣,言不見聽,退而杜口。朕與卿等當深用為戒。」因取五人誥詞,親增二語云:「勿謂崇高而難入,勿以有所從違而或怠。」幼孜等頓首稱謝。洪熙元年進禮部尚書兼大學士、學士如故,並給三俸。尋乞歸省母。明年,母卒。

宣宗立,詔起復,修兩朝實錄,充總裁官。三年持節寧夏,冊慶府郡王妃。所過詢兵民疾苦,還奏之。帝嘉納焉。從巡邊,度雞鳴山。帝曰:「唐太宗恃其英武征遼,嘗過此山。」幼孜對曰:「太宗尋悔此役,故建憫忠閣。」帝曰:「此山崩於元順帝時,為元亡徵。」對曰:「順帝亡國之主,雖山不崩,國亦必亡。」宣德六年十二月卒。年六十四。贈少保,謚文靖。

幼孜簡易靜默,寬裕有容。眷遇雖隆,而自處益謙。名其宴居之室曰「退庵」。疾革時,家人囑請身後恩,不聽,曰:「此君子所恥也。」

胡儼,字若思,南昌人。少嗜學,於天文、地理、律曆、醫卜無不究覽。洪武中以舉人授華亭教諭,能以師道自任。母憂,服除,改長垣,乞便地就養,復改餘干。學官許乞便地自儼始。

建文元年,薦授桐城知縣。鑿桐陂水,溉田為民利。縣有虎傷人。儼齋沐告於神,虎遁去。桐人祀之朱邑祠。四年,副都御史練子寧薦於朝曰:「儼學足達天人,智足資帷幄。」比召至,燕師已渡江。

成祖即位,曰:「儼知天文,其令欽天監試。」既試,奏儼實通象緯、氣候之學。尋又以解縉薦,授翰林檢討,與縉等俱直文淵閣,遷侍講,進左庶子。父喪,起復。儼在閣,承顧問,嘗不欲先人,然少戇。永樂二年九月,拜國子監祭酒,遂不預機務。時用法嚴峻,國子生託事告歸者坐戍邊。儼至,即奏除之。七年,帝幸北京,召儼赴行在。明年北征,命以祭酒兼侍講,掌翰林院事,輔皇太孫留守北京。十九年,改北京國子監祭酒。

當是時,海內混一,垂五十年。帝方內興禮樂,外懷要荒,公卿大夫彬彬多文學之士。儼館閣宿儒,朝廷大著作多出其手,重修《太祖實錄》、《永樂大典》、《天下圖誌》皆充總裁官。居國學二十餘年,以身率教,動有師法。洪熙改元,以疾乞休,仁宗賜敕獎勞,進太子賓客,仍兼祭酒。致仕,復其子孫。

宣宗即位,以禮部侍郎召,辭歸。家居二十年,方岳重臣咸待以師禮。儼與言,未嘗及私。自處淡泊,歲時衣食才給。初為湖廣考官,得楊溥文,大異之,題其上曰:「必能為董子之正言,而不為公孫之阿曲。」世以為知人。正統八年八月卒,年八十三。

贊曰:明初罷丞相,分事權於六部。成祖始命儒臣直文淵閣,預機務。沿及仁、宣,而閣權日重,實行丞相事。解縉以下五人,則詞林之最初入閣者也。夫處禁密之地,必以公正自持,而尤貴於厚重不洩。縉少年高才,自負匡濟大略,太祖俾十年進學,愛之深矣。彼其動輒得謗,不克令終,夫豈盡嫉賢害能者力固使之然歟。黃淮功在輔導,胡廣、金幼孜勞著扈從,胡儼久於國學。觀諸臣從容密勿,隨事納忠,固非僅以文字翰墨為勛績已也。


\end{pinyinscope}