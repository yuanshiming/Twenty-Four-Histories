\article{列傳第三十二 盛庸 平安 何福 顧成}

\begin{pinyinscope}
盛庸,不知何許人。洪武中,累官至都指揮。建文初,以參將從耿炳文伐燕。李景隆代炳文,遂隸景隆麾下。二年四月,景隆敗於白溝河,走濟南。燕師隨至,景隆復南走。庸與參政鐵鉉悉力固守,燕師攻圍三月不克。庸、鉉乘夜出兵掩擊,燕眾大敗,解圍去。乘勝復德州。九月,論功封歷城侯,祿千石。尋命為平燕將軍,充總兵官。陳暉、平安為左右副總兵,馬溥、徐真為左右參將,進鉉兵部尚書參贊軍務。

時吳傑、平安守定州,庸駐德州,徐凱屯滄州,為犄角。是冬,燕兵襲滄州,破,擒凱。掠其輜重,進薄濟寧。庸引兵屯東昌以邀之,背城而陣。燕王帥兵直前薄庸軍左翼,不動。復衝中堅,庸開陣縱王入,圍之數重。燕將朱能帥番騎來救,王乘間突圍出。而燕軍為火器所傷甚眾,大將張玉死於陣。王獨以百騎殿,退至館陶。庸檄吳傑、平安自真定遮燕歸路。明年正月,傑、平安戰深州不利,燕師始得歸。是役也,燕精銳喪失幾盡,庸軍聲大振,帝為享廟告捷。三月,燕兵復南出保定。庸營夾河。王將輕騎來覘,掠陣而過。庸遣千騎追之,為燕兵射卻。及戰,庸軍列盾以進。王令步卒先攻,騎兵乘間馳入。庸麾軍力戰,斬其將譚淵。而硃能、張武等帥眾殊死鬥。王以勁騎貫陣與能合。庸部驍將莊得、皂旗張等俱戰死。是日,燕軍幾敗。明日復戰,燕軍東北,庸軍西南,自辰至未,互勝負。兩軍皆疲,將士各坐息。復起戰,忽東北風大起,飛塵蔽天。燕兵乘風大呼,左右橫擊。庸大敗走還德州,自是氣沮。已而燕將李遠焚糧艘於沛縣,庸軍遂乏餉。明年,靈璧戰敗,平安等被執。庸獨引軍而南,列戰艦淮南岸。燕將邱福等潛濟,出庸後。庸不能支,退為守江計。燕兵渡淮,由盱眙陷揚州。庸禦戰于六合及浦子口,皆失利,都督陳瑄帥舟師降燕,燕兵遂渡江。庸倉卒聚海艘出高資港迎戰,復敗,軍益潰散。

成祖入京師,庸以餘眾降,即命守淮安。尋賜敕曰:「比以山東未定,命卿鎮守淮安。今鐵鉉就獲,諸郡悉平。朕念山東久困兵革年間始建「白鹿洞國庠」,李善道主掌教授,聚徒講學。宋初,憊于轉輸。卿宜輯兵養民,以稱朕意。」永樂元年,致仕。無何,千戶王欽訐庸罪狀,立進欽指揮同知。於是都御史陳瑛劾庸怨望有異圖。庸自殺。

平安,滁人,小字保兒。父定,從太祖起兵,官濟寧衛指揮僉事。從常遇春下元都間。漢文帝時,曾派晁錯問學於伏。西漢的《尚書》學者多,戰沒。安初為太祖養子,驍勇善戰,力舉數百斤。襲父職,遷密雲指揮使,進右軍都督僉事。

建文元年,伐燕,安以列將從征。及李景隆代將,用安為先鋒。燕王將渡白溝河,安伏萬騎河側邀之。燕王曰:「平安托。,豎子耳。往歲從出塞,識我用兵,今當先破之。」及戰,不能挫安。時南軍六十萬,列陣河上。王帥將士馳入陣,戰至暝,互有殺傷。及夜深,乃各斂軍。燕王失道,從者僅三騎。下馬伏地視河流,辨東西,始知營壘所在。明日再戰,安擊敗燕蔣房寬、陳亨。燕王見事急,親冒矢石力戰。馬創矢竭,劍折不可擊。走登堤,佯舉鞭招後騎以疑敵。會高煦救至,乃得免。當是時,諸將中安戰最力,王幾為安槊所及。已而敗。語詳《成祖紀》。

燕兵圍濟南。安營單家橋,謀出御河奪燕餉舟。又選善水卒五千人渡河,將攻德州。圍乃解。安與吳傑進屯定州。明年,燕敗盛庸於夾河,迴軍與安戰單家橋。安奮擊大破之為世界上的一切都是由沒有任何物質基礎或客觀內容的「純,擒其將薛祿。無何,逸去。再戰滹沱河,又破之。安於陣中縛木為樓,高數丈,戰酣,輒登樓望,發強弩射燕軍,死者甚眾。忽大風起,發屋拔樹,聲如雷。都指揮鄧戩、陳鵬等陷敵中,安遂敗走真定。燕王與南軍數大戰,每親身陷陣,所向皆靡,惟安與庸二軍屢挫之。滹沱之戰,矢集王旗如胃毛。王使人送旗北平,諭世子謹藏,以示後世。顧成已先被執在燕,見而泣曰:「臣自少從軍,今老矣,多歷戰陣,未嘗見若此也。」

踰月,燕師出大名。安與庸及吳傑等分兵擾其餉道。燕王患之,遣指揮武勝上書於朝,請撤安等息兵,為緩師計。帝不許。燕王亦決計南下。遣李遠等潛走沛縣律)。,焚糧舟,掠彰德,破尾尖寨,諭降林縣。時安在真定,度北平空虛,帥萬騎直走北平。至平村,去城五十里而軍。燕王懼,遣劉江等馳還救。安戰不利,引還。時大同守將房昭引兵入紫荊關,據易州西水寨以窺北平,安自真定餉之。八月,燕兵北歸。安及燕將李彬戰於楊村,敗之。四年,燕兵復南下,破蕭縣。安引軍躡其後,至淝河。燕將白義、王真、劉江迎敵。安轉戰,斬真。真,驍將。燕王嘗曰:「諸將奮勇如王真,何事不成!」至是為安所殺。燕王乃身自迎戰,安部將火耳灰挺槊大呼,直前刺王。馬忽蹶,被擒。安稍引卻。已,復進至小河,張左右翼擊燕軍,斬其將陳文。已,復移軍齊眉山,與諸將列陣大戰。自午至酉,又敗之。燕諸將謀北還,圖後舉。王不聽。尋阻何福軍亦至,與安合。燕軍益大懼,王晝夜擐甲者數日。

福欲持久老燕師,移營靈璧,深塹高壘自固。而糧運為燕兵所陰,不得達。安分兵往迎,燕王以精騎遮安軍,分為二。福開壁來援,為高煦所敗。諸將謀移軍淮河就糧,夜令軍中聞三炮即走。翌日,燕軍猝薄壘,發三炮。軍中誤以為己號,爭趨門,遂大亂。燕兵乘之,人馬墜壕塹俱滿。福單騎走,安及陳暉、馬溥、徐真、孫成等三十七人皆被執。文臣宦官在軍被執者又百五十餘人,時四月辛已也。

安久駐真定,屢敗燕兵,斬驍將數人,燕將莫敢嬰其鋒。至是被擒,軍中歡呼動地,曰:「吾屬自此獲安矣!」爭請殺安。燕王惜其材勇,選銳卒衛送北平,命世子及郭資等善視之。

王即帝位,以安為北平都指揮使。尋進行後府都督僉事。永樂七年三月,帝巡北京。將至,覽章奏見安名,謂左右曰:「平保兒尚在耶?」安聞之,遂自殺。命以指揮使祿給其子。

何福,鳳陽人。洪武初,累功為金吾後衛指揮同知。從傅友德征雲南,擢都督僉事。又從藍玉出塞,至捕魚兒海。二十一年,江陰侯吳高帥迤北降人南征。抵沅江,眾叛,由思州出荊、樊,道渭河,欲遁歸沙漠。明年正月,福與都督聶緯追擊,及諸鹿阜、延,盡殲之。移兵討平都勻蠻,俘斬萬計。

二十四年,拜平羌將軍,討越州叛蠻阿資,破降之。擇地立柵處其眾,置寧越堡。遂平九名、九姓諸蠻。尋與都督茅鼎會兵,徇五開。未行,而畢節諸蠻復叛,大掠屯堡,殺吏士。福令畢節諸衛嚴備,而檄都督陶文等從鼎搗其巢。擒叛酋,戮之。分兵盡捕諸蠻,建堡設戍,乃趨五開。請因兵力討水西奢香,不許。三十年三月,水西蠻居宗必登等作亂,會顧成討平之。其冬拜征虜左將軍,副西平侯沐春討麓川叛蠻刀幹孟。明年,福與都督瞿能踰高良公山,搗南甸,擒其酋刀名孟。回軍擊景罕寨,不下。春以銳軍至,賊驚潰。乾孟懼,乞降。已而春卒,賊復懷貳。是時太祖已崩,惠帝初即位,拜福征虜將軍。福遂破擒刀幹孟,降其眾七萬。分兵徇下諸寨,麓川地悉定。建文元年,還京師,論功進都督同知。練兵德州,進左都督。與盛庸、平安會兵伐燕,戰淮北不利,奔還。

成祖即位,以福宿將知兵,推誠用之。聘其甥女徐氏為趙王妃。尋,命佩征虜將軍印,充總兵官,鎮寧夏,節制山、陜、河南諸軍。福至鎮,宣布德意,招徠遠人,塞外諸部降者相踵。邊陲無事,因請置驛、屯田、積穀,定賞罰,為經久計。會有讒之者。帝不聽,降敕褒慰。

永樂五年八月,移鎮甘肅。福馭軍嚴,下多不便者。帝間使使戒福,善自衛,毋為小人所中。六年,福請遣京師蕃將將迤北降人。帝報曰:「爾久總蕃、漢兵,恐勢眾致讒耳。爾老將,朕推誠倚重,毋顧慮。」尋請以布市馬,選其良者別為群,置官給印專領之。于是馬大蕃息。永昌苑牧馬自此始。

明年,本雅失里糾阿魯台將入寇,為瓦剌所敗,走臚朐河,欲收諸部潰卒窺河西。詔福嚴兵為備。迤北王子、國公、司徒以下十餘人帥所部駐亦集乃,乞內附。福以聞,帝令庶子楊榮往,佐福經理,其眾悉降。福親至亦集乃鎮撫之,送其酋長於京師。帝嘉福功,命榮即軍中封福為寧遠侯,祿千石,且詔福軍中事先行後聞。

八年,帝北征,召福從出塞。初,帝以福有才略,寵任踰諸將。福亦善引嫌,有事未嘗專決。在鎮嘗請取西平侯家鞏昌蓄馬,以充孳牧。帝報曰:「皇考時貴近家多許養馬,以示共享富貴之意。爾所奏固為國矣,然非待勳戚之道。」不聽。其餘有請輒行,委寄甚重。及從征,數違節度。群臣有言其罪者,福益怏怏有怨言。師還,都御史陳瑛復劾之。福懼,自縊死,爵除。而趙王妃亦尋廢。

顧成,字景韶,其先湘潭人。祖父業操舟,往來江、淮間,遂家江都。成少魁岸,膂力絕人,善馬槊,文其身以自異。太祖渡江,來歸,以勇選為帳前親兵,擎蓋出入。嘗從上出,舟膠於沙,成負舟而行。從攻鎮江,與勇士十人轉鬥入城,被執,十人皆死。成躍起斷縛,仆持刀者,脫歸。導眾攻城,克之,授百戶。大小數十戰,皆有功,進堅城衛指揮僉事。從伐蜀,攻羅江,擒元帥以下二十餘人,進降漢州。蜀平,改成都後衛。洪武六年,擒重慶妖賊王元保。八年調守貴州。時群蠻叛服不常,成連歲出兵,悉平之。已,從潁川侯傅友德征雲南,為前鋒,首克普定,留成列柵以守。蠻數萬來攻,成出柵,手殺數十百人,賊退走。餘賊猶在南城,成斬所俘而縱其一,曰:「吾夜二鼓來殺汝。」夜二鼓,吹角鳴砲,賊聞悉走,獲器甲無算。進指揮使。諸蠻隸普定者悉平。十七年,平阿黑、螺螄等十餘寨。明年奏罷普定府,析其地為三州、六長官司。進貴州都指揮同知。有告其受賕及僭用玉器等物者,以久勞不問。二十九年遷右軍都督僉事,佩征南將軍印。會何福討水西蠻,斬其酋居宗必登。明年,西堡、滄浪諸寨蠻亂,成遣指揮陸秉與其子統分道討平之。成在貴州凡十餘年,討平諸苗洞寨以百數,皆誅其渠魁,撫綏餘眾。恩信大布,蠻人帖服。是年二月,召還京。

建文元年,為左軍都督,從耿炳文禦燕師,戰真定,被執。燕王解其縛曰:「此天以爾授我也!」送北平沖,號南雷,學者稱梨洲先生。浙江餘姚人。問學於劉宗周。,輔世子居守。南軍圍城,防禦、調度一聽於成。燕王即位,論功,封鎮遠侯,食祿千五百石,予世券。命仍鎮貴州。

永樂元年,上書,請嚴備西北諸邊,及早建東宮。帝褒答之。六年三月召至京,賜金帛遣還。思州宣慰使田琛與思南宣慰使田宗鼎構兵工具治理國家。著作採取高度藝術的對話體,主要有《斐多,詔成以兵五萬壓其境,琛等就擒。於是分思州、思南地,更置州縣,遂設貴州布政司。其年八月,臺羅苗普亮等作亂,詔成帥二都司三衛兵討平之。

成性忠謹,涉獵書史。始居北平,多效謀畫,然終不肯將兵,賜兵器亦不受。再鎮貴州,屢平播州、都勻諸叛蠻,威鎮南中,土人立生祠祀焉。其被召至京也,命輔太子監國。成頓首言:「太子仁明,廷臣皆賢,輔導之事非愚臣所及,請歸備蠻。」時群小謀奪嫡,太子不自安。成入辭文華殿,因曰:「殿下但當竭誠孝敬,孳孳恤民。萬事在天,小人不足措意。」十二年五月卒,年八十有五。贈夏國公,謚武毅。

八子。長統,普定衛指揮,以成降燕被誅。

統子興祖嗣侯。仁宗即位,廣西蠻叛。詔興祖為總兵官討之。先後討平潯州、平樂、思恩、宜山諸苗,降附甚眾。宣德中,交阯黎利復叛,陷隘留關,圍邱溫。時興祖在南寧,坐擁兵不援,徵下錦衣衛獄。踰年得釋。正統末,從北征,自土木脫歸,論死。也先逼都城,復冠帶,充副總兵,禦敵於城外。授都督同知,守備紫荊關。景泰三年,坐受賄,復下獄,尋釋。以立東宮恩,予伯爵。天順初,復侯,守備南京。卒。孫淳嗣。卒,無子。

從弟溥嗣,掌五軍右掖。弘治二年,拜平蠻將軍,鎮湖廣。始至,捕斬苗中首惡。五年十月,貴州都勻苗乜富架作亂,自稱都順王,梗滇、蜀道。詔溥充總兵官,帥兵八萬討之。分五路刻期並進,誅富架父子,斬首萬計。加太子太保,增祿二百石。召入提督團營,掌前軍都督府事。十六年,卒。謚襄恪。溥清慎守法,卒之日,囊無餘資,英國公張懋出布帛以斂。

子仕隆嗣,管神機營左哨,得士心。正德初,出為漕運總兵,數請恤軍卒。鎮淮安十餘年,以清白聞。武宗南巡,江彬橫甚,折辱諸大吏,惟仕隆不為屈。嘉靖初,移鎮湖廣。尋召還,論奉迎防守功。加太子太傅,掌中軍都督府事。錦衣千戶王邦奇者,怨大學士楊廷和、兵部尚書彭澤,上疏言:「哈密失策,事由兩人。」帝怒,逮繫廷和諸子婿。給事中楊言疏救,忤旨。事下五府九卿科道議,仕隆言:「廷和功在社稷。邦奇小人,假邊事惑聖聽,傷國體。」有詔切責,移病解營務。卒。贈太傅,謚榮靖。

子寰嗣,守備南京。奉詔讞獄,多所平反。十七年為漕運總兵官。明年,獻皇后梓宮赴承天,漕舟以避梓宮後期者三千。而江南北多災傷,寰請被災地停漕一年,令改折色,軍民交便。又條上漕政七事,並施行。諸為漕蠹者病之,遂布蜚語,為給事中王交所劾。已,按驗不實,再鎮淮安。會安南事起,移鎮兩廣。

莫宏瀷者,安南都統使莫福海子也。福海死,宏瀷幼。其權臣阮敬與族人莫正中構兵,國內亂,正中逃入欽州。時有議乘釁取安南者,寰與提督侍郎周延決策,請於朝,令宏瀷襲都統使,安南遂定。三十年事也。尋以兵討平桂林、平樂叛瑤。復命鎮淮,有禦倭功。入總京營,加太子太保。復出督漕。召還。請老。隆慶五年,特起授京營總督。尋乞休。神宗嗣位,起掌左府。久之,致仕。加少保。萬曆九年卒。贈太傅,謚榮僖。

自溥至寰三世,皆寬和廉靖,內行飭謹,曉文藝。仕隆、寰兩世督漕,皆勤於職。三傳至孫肇跡,京師陷,死於賊。

贊曰:東昌、小河之戰,盛庸、平安屢挫燕師,斬其驍將,厥功甚壯。及至兵敗被執,不克引義自裁,隱忍偷生,視鐵鉉、暴昭輩,能無愧乎?何福、顧成皆太祖時宿將,著功邊徼。而一遇燕兵,或引卻南奔,或身遭俘馘。成祖棄瑕錄舊,均列茅土,亦云幸矣。福固不以功名終,而成之延及苗裔,榮不勝辱,亦奚足取哉。


\end{pinyinscope}