\article{列傳第五十七}

\begin{pinyinscope}
○高穀胡濙王直

高穀,字世用,揚州興化人。永樂十三年進士,選庶吉土,授中書舍人。仁宗即位,改春坊司直郎,尋遷翰林侍講。英宗即位,開經筵,楊士奇薦穀及苗衷、馬愉、曹鼐四人侍講讀。正統十年由侍講學士進工部右侍郎,入內閣典機務。

景泰初,進尚書,兼翰林學士,掌閣務如故。英宗將還,奉迎禮薄,千戶龔遂榮投書於穀,具言禮宜從厚,援唐肅宗迎上皇故事。穀袖之入朝,遍示廷臣曰:「武夫尚知禮,況儒臣乎!」眾善其言。胡濙、王直欲以聞。穀曰:「迎復議上,上意久不決。若進此書,使上知朝野同心,亦一助也。」都御史王文不可。已而言官奏之。詰所從得,穀對曰:「自臣所。」因抗章懇請如遂榮言。帝雖不從,亦不之罪。

二年進少保、東閣大學士。易儲,加太子太傅,給二俸。應天、鳳陽災,命祀三陵,振貧民。七年進謹身殿大學士,仍兼東閣。內閣七人,言論多齟齬。穀清直,持議正。王文由穀薦,數擠穀。穀屢請解機務,不許。都給事中林聰忤權要論死,穀力救,得薄譴。陳循及文構考官劉儼、黃諫,帝命禮部會穀復閱試卷。穀力言儼等無私,且曰:「貴胄與寒士競進,已不可。況不安義命,欲因此構考官乎?」帝乃賜循、文子中式,惟黜林挺一人,事得已。

英宗復位,循、文等皆誅竄,穀謝病。英宗謂穀長者,語廷臣曰:「穀在內閣議迎駕及南內事,嘗左右朕。其賜金帛襲衣,給驛舟以歸。」尋復賜敕獎諭。

穀既去位,杜門絕賓客。有問景泰、天順間事,輒不應。天順四年正月卒,年七十。

穀美豐儀,樂儉素,位至台司,敝廬瘠田而已。成化初,贈太保,謚文義。

胡濙,字源潔,武進人。生而髮白,彌月乃黑。建文二年舉進士,授兵科給事中。永樂元年遷戶科都給事中。

惠帝之崩於火,或言遁去,諸舊臣多從者,帝疑之。五年遣濙頒御製諸書,並訪仙人張邋遢,遍行天下州郡鄉邑,隱察建文帝安在。濙以故在外最久,至十四年乃還。所至,亦間以民隱聞。母喪乞歸,不許,擢禮部左侍郎。十七年復出巡江浙、湖、湘諸府。二十一年還朝,馳謁帝於宣府。帝已就寢,聞濙至,急起召入。濙悉以所聞對,漏下四鼓乃出。先濙未至,傳言建文帝蹈海去,帝分遣內臣鄭和數輩浮海下西洋,至是疑始釋。

皇太子監國南京,漢王為飛語謗太子。帝改濙官南京,因命廉之。濙至,密疏馳上監國七事,言誠敬孝謹無他,帝悅。

仁宗即位,召為行在禮部侍郎,濙陳十事,力言建都北京非便,請還南都,省南北轉運供億之煩。帝皆嘉納。既聞其嘗有密疏,疑之,不果召。轉太子賓客,兼南京國子祭酒。

宣宗即位,仍遷禮部左侍郎。明年來朝,乃留行在禮部,尋進尚書。漢王反,與楊榮等贊親征。事平,賚予甚厚。明年賜第長安右門外,給閽者二人,賜銀章四。生辰,賜宴其第。四年命兼理詹事府事。六年,張本卒,又兼領行在戶部。時國用漸廣,濙慮度支不足,蠲租詔下,輒沮格。帝嘗切戒之,然眷遇不少替。嘗曲宴濙及楊士奇、夏原吉、蹇義,曰:「海內無虞,卿等四人力也。」英宗即位,詔節冗費。濙因奏減上供物,及汰法王以下番僧四五百人,浮費大省。正統五年,山西災,詔行寬恤,既而有採買物料之命。濙上疏言詔旨宜信。又言軍旗營求差遣,因而擾民,宜罷之。皆報可。行在禮部印失,詔弗問,命改鑄。已,又失,被劾下獄。未幾,印獲,復職。九年,年七十,乞致仕,不許。英宗北狩,群臣聚哭於朝,有議南遷者。濙曰:「文皇定陵寢於此,示子孫以不拔之計也。」與侍郎於謙合,中外始有固志。

景帝即位,進太子太傅。楊善使也先,濙言上皇蒙塵久,宜附進服食,不報。上皇將還,命禮部具奉迎儀。濙等議遣禮部署迎於龍虎臺,錦衣具法駕迎居庸關,百司迎土城外,諸將迎教場門;上皇自安定門入,進東安門,於東上北門南面坐;皇帝謁見畢,百官朝見,上皇入南城大內。議上,傳旨以一轎二馬迎於居庸關,至安定門易法駕,餘如奏。給事中劉福等言禮太薄。帝報曰:朕尊大兄為太上皇帝,尊禮無加矣。福等顧云太薄,其意何居?禮部其會官詳察之。」濙等言:「諸臣意無他,欲陛下篤親親耳。」帝曰:「昨得太上皇書,具言迎駕之禮宜從簡損,朕豈得違之。」群臣乃不敢言。會千戶龔遂榮為書投大學士高穀,言奉迎宜厚,具言唐肅宗迎上皇故事。穀袖之以朝,與王直等共觀之。直與濙欲聞之帝,為都御史王文所阻,而給事中葉盛竟以聞。盛同官林聰復劾直、濙、穀等,皆股肱大臣,有聞必告,不宜偶語竊議。有詔索書。濙等因以書進,且言:「肅宗迎上皇典禮,今日正可仿行。陛下宜躬迎安定門外,分遣大臣迎龍虎臺。」帝不悅曰:「第從朕命,無事紛更。」上皇至,居南城宮。濙請帝明年正旦率群臣朝延安門,不許。上皇萬壽節,請令百官拜賀延安門,亦不許。三年正月與王直並進少傅。易太子,加兼太子太師。王文惡林聰,文致其罪,欲殺之。濙不肯署,遂稱疾,數日不朝。帝使興安問疾。對曰:「老臣本無疾,聞欲殺林聰,殊驚悸耳。」聰由是得釋。

英宗復位,力疾入朝,遂求去。賜璽書、白金、楮幣、襲衣,給驛,官其一子錦衣,世鎮撫。濙歷事六朝,垂六十年,中外稱耆德。及歸,有三弟,年皆七十餘,鬚眉皓白,燕聚一堂,因名之曰「壽愷」。又七年始卒,年八十九。贈太保,謚忠安。

濙節儉寬厚,喜怒不形於色,能以身下人。在禮部久,表賀祥瑞,以官當首署名,人因謂其性善承迎。南城人龔謙多妖術,濙薦為天文生,又薦道士仰彌高曉陰陽兵法,使守邊,時頗譏之。

王直,字行儉,泰和人。父伯貞,洪武十五年以明經聘至京。時應詔者五百餘人,伯貞對第一。授試僉事,分巡廣東雷州。復呂塘廢渠,清鹽法。會罷分巡官,召還為戶部主事。以父喪服闋,不時起,謫居安慶。建文初,復以薦知瓊州,崖州黎相仇殺,以反聞,且用兵。伯貞捕其首惡,兵遂罷。瓊田歲常三獲,以賦軍,軍不時受,俟民乏,乃急斂以要利。伯貞為立期,三輸之,弊始絕。居數年,大治,流民占籍者萬餘。憂歸,卒於家。

直幼而端重,家貧力學。舉永樂二年進士,改庶吉士,與曾棨、王英等二十八人同讀書文淵閣。帝善其文,召入內閣,俾屬草。尋授修撰。歷事仁宗、宣宗,累遷少詹事兼侍讀學士。

正統三年,《宣宗實錄》成。進禮部侍郎,學士如故。五年出蒞部事。尚書胡濙悉以部政付之,直處之若素習者。八年正月代郭璡為吏部尚書。十一年,戶部侍郎奈亨附王振,構郎中趙敏,詞連直及侍郎曹義、趙新,並下獄。三法司廷鞫,論亨斬,直等贖徒。帝宥直、義,奪亨、新俸。

帝將親征也先,直率廷臣力諫曰:「國家備邊最為謹嚴。謀臣猛將,堅甲利兵,隨處充滿,且耕且守,是以久安。今敵肆猖獗,違天悖理,陛下但宜固封疆,申號令,堅壁清野,蓄銳以待之,可圖必勝。不必親御六師,遠臨塞下。況秋署未退,旱氣未回,青草不豐,水泉猶塞,士馬之用未充。兵凶戰危,臣等以為不可。」帝不從,命直留守。王師覆於土木。大臣群請太后立皇子為皇太子,命成阜王攝政。已,勸王即位,以安反側。時變起倉卒,朝臣議屢上,皆直為首。而直自以不如於謙,每事推下之,雍容鎮率而已。加太子太保。

景泰元年,也先使使議和,且請還上皇,下禮部議未決。直率群臣上言曰:「太上皇惑細人言,輕身一出,至於蒙塵。陛下宵衣旰食,徵天下兵,與群臣兆姓同心僇力,期滅此朝食,以雪不共戴天之恥。乃者天誘其衷,也先有悔心之萌,而來求成於我,請還乘輿,此轉禍為福之機也。望陛下俯從其請,遣使往報,因察其誠偽而撫納之,奉太上皇以歸,少慰祖宗之心。陛下天位已定,太上皇還,不復蒞天下事。陛下第崇奉之,則天倫厚而天眷益隆,誠古今盛事也。」帝曰:「卿等言良然。但前後使者五輩往,終不得要領。今復遣使,設彼假送駕為名,來犯京師,豈不為蒼生患。賊詐難信,其更議之。」已而瓦剌別部阿剌使復至,胡濙等復以為言。於是帝御文華殿門,召諸大臣及言官諭以宜絕狀。直對曰:「必遣使,毋貽後悔。」帝不悅。于謙前為解,帝意釋。群臣既退,太監興安匍匐出呼曰:「若等固欲遣使,有文天祥、富弼其人乎?」直大言曰:「廷臣惟天子使,既食其祿,敢辭難乎!」言之再,聲色愈厲。安語塞,乃議遣使,命李實、羅綺往。

既行,而瓦剌可汗脫脫不花及也先使先後至,將遣歸。使者謂館伴曰:「中國關外十四城皆為我有。前阿剌知院使來,尚遣人偕往。今亦必得大臣同行,庶有濟。」胡濙以聞,下廷議。直等固請,乃遣楊善等報之。

比實還,又以也先使至,具言也先欲和狀。直與寧陽侯陳懋等上疏,請更遣使齎禮幣往迎上皇,不許。復上疏曰:「臣等與李實語,具得彼中情事。其所需衣物資斧者,上皇言也;而奉迎車駕,也先意也。昨者脫脫不花及阿剌知院使來,皆有報使。今也先使以迎請為辭,乃不遣使與偕,是疑敵而召兵也。」又不許。

已而實自言於帝。帝第報也先書,就令楊善迎歸而已。直等復上言:「今北使已發,願本上皇之心,順臣民之願,因彼悔心,遣使往報,以圖迎復,此不待計而決者也。不然,眾志難犯,違天不祥,彼將執為兵端,邊事益棘,京師亦不得高枕臥矣。」帝乃命群臣擇使,直與陳懋等請仍遣實。報曰:「候善歸議之。」御史畢鑾等復上疏,力言:「就令彼以詐來,我以誠往,萬一不測,則我之兵力固在。」帝終不聽。已而善竟奉上皇還。

二年,也先遣使入貢,且請答使。直屢疏言:「邊備未修,芻糧未積,瘡痍未復,宜如其請。遣使往以觀虛實,開導其善。」不許。無何,也先遣騎入塞,以報使為辭。直與群臣復請之,卒不許。直等乃上疏言:「陛下銳意治兵,為戰守計,真大有為之主。然使命不通,難保其不為寇。宜敕沿邊守臣,發兵遊徼,有警則入保,無事則力耕。陛下於機務之暇,時召京營總督、總兵,詢以方略,誠接而禮貌之,信賞罰以持其後,斯戰守可言也。」帝曰「善」。

明年正月進少傅。帝欲易太子。未發。會思明土知府黃矰以為請。帝喜,下禮部議。胡濙唯唯,文武諸臣議者九十一人當署名,直有難色。陳循濡筆強之,乃署,竟易皇太子。直進兼太子太師,賜金幣加等。頓足歎曰:「此何等大事,乃為一蠻酋所壞,吾輩愧死矣。」景帝疾亟,直、濙等會諸大臣臺諫,請復立沂王為皇太子,推大學士商輅草疏。未上,而石亨、徐有貞等奪門迎上皇復位,殺王文等。疏草留姚夔所,嘗出以示郎中陸昶,歎曰:「是疏不及進,天也。」直遂乞休。賜璽書、金綺、楮幣,給驛歸。

直為人方面修髯,儀觀甚偉。性嚴重,不茍言笑。及與人交,恂恂如也。在翰林二十餘年,稽古代言編纂紀注之事,多出其手。與金谿王英齊名,人稱「二王」,以居地目直曰「東王」,英曰「西王」。直以次當入閣,楊士奇不欲也。及長吏部,兼廉慎。時初罷廷臣薦舉方面大吏,專屬吏部。直委任曹郎,嚴抑奔競。凡御史巡方歸者,必令具所屬賢否以備選擢,稱得人。其子資為南國子博士。考績至部,文選郎欲留侍直,直不可,曰:「是亂法自我始也。」朝廷以直老,命何文淵為尚書佐之。文淵去,又命王翱,部遂有二尚書。直為尚書十四年,年益高,名德日益重。帝優禮之,免其常朝。

比家居,嘗從諸佃僕耕蒔,擊鼓歌唱。諸子孫更迭舉觴上壽,直歎曰:「曩者西楊抑我,令不得共事。然使我在閣,今上復辟,當不免遼陽之行,安得與汝曹為樂哉!」天順六年卒,年八十四。贈太保,謚文端。

資仕至翰林檢討,亦以學行稱。曾孫思,自有傳。

贊曰:高穀之清直,胡濙之寬厚,王直之端重,蓋皆有大臣之度焉。當英、景之間,國勢初更,人心觀望,執政任事之臣多阿意取容。而谷、濙心卷心卷於迎駕之儀,直侃侃於遣使之請,皆力持正議,不隨眾俯仰,故能身負碩望,始終一節,可謂老成人矣。


\end{pinyinscope}