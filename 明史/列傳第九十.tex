\article{列傳第九十}

\begin{pinyinscope}
廖紀王時中周期雍唐龍子汝楫王杲王周用宋景屠僑聞淵劉訒胡纘宗孫應奎餘姚孫應奎方鈍聶豹李默萬鏜周延潘恩賈應春張永明胡松績溪胡松趙炳然

廖紀,字時陳,東光人。弘治三年進士。授考功主事,屢遷文選郎中。正德中,歷工部右侍郎。提督易州山廠,羨金無所私。遷吏部左、右侍郎。世宗立,拜南京吏部尚書。調兵部,參贊機務。被論解職。

嘉靖三年,「大禮」議既定,吏部尚書楊旦赴召,道劾張璁、桂萼。璁、萼之黨陳洸遂劾旦而薦紀。帝罷旦,以紀代之。紀疏辭,言:「臣年已七十,精力不如喬宇,聰明不如楊旦。」時宇、旦方為帝所惡,不許。光祿署丞何淵請建世室,祀興獻帝,下廷議。紀等執不可,帝弗從。紀力爭曰:「淵所言,干君臣之分,亂昭穆之倫,蔑祖宗之制,臣謹昧死請罷勿議。」不納。會廷臣多諍者,議竟寢。已,條奏三事。其末言人材當惜,謂:「正德之季,宗社幾危。議者但知平定逆籓之功,而不知保護京師之力。自陛下繼統,老成接踵去,新進連茹登,以出位喜事為賢,以凌分犯禮為貴。伏望陛下於昔年致仕大臣,念其保護之勛,量行召用。其他降職、除名、遣戍者,使得以才自效。」帝但納其正士風、重守令二事而已。三邊總督楊一清召還內閣,璁等欲起王瓊,紀推彭澤、王守仁,帝不允。復以鄧廷璋、王憲名上,竟用憲。

五年正月,御史張袞、喻茂堅、朱實昌以世廟禮成,請宥議禮得罪諸臣,璁、萼亦以為請,章俱下吏部。紀等列上四十七人,卒報罷。御史魏有本以劾郭勛、救馬永謫官,給事中沈漢等論救,帝不聽。紀從容為言,且薦永及楊銳。帝納之,有本得無謫。紀在南都,持議與璁合,坐是劾罷。璁輩欲引助己,遂首六卿。而紀顧數與抵牾,璁輩亦不喜。年老稱病乞歸,許之去。初,《獻皇實錄》成,加太子太保。至是進少保,賜敕乘傳,夫廩視故事有加。卒,贈太保,謚僖靖。

王時中,字道夫,黃縣人。弘治三年進士。授鄢陵知縣。嘗出郊,旋風擁馬首。時中曰:「冤氣也。」跡得屍眢井,乃婦與所私者殺之,遂伏辜。召拜御史,督察畿輔馬政。

正德初,請革近畿皇莊,不報。吏部尚書馬文升致仕,時望屬劉大夏、閔珪。時中詆珪和媚,大夏昏耄。兩人各求退,焦芳遂得之,眾咸咎時中。出按宣、大,逮繫武職貪污者百餘,為東廠太監邱聚所奏。劉瑾捕時中下詔獄,荷重枷於都察院門。時中病甚,其妻往省,遇都御史劉宇,哭且詬。宇不得已言於瑾,釋之,謫戍鐵嶺衛。瑾誅,起四川副使,遷湖廣按察使。十二年以右僉都御史巡撫寧夏。

世宗立,召為右副都御史。父喪除,起故官。會上章聖太后尊號,時中言本生二字不當去。及上冊寶,百官陪列不至者九人,時中與焉。帝責對狀,已而貰之。歷兵部左侍郎,代李鉞為尚書。中官黃英等多所陳請,時中皆執不可。敘薊州平盜功,濫及通州守備鄢祐,為言官李鳴鶴等所劾。時中乞休,且詆言者。給事中劉世揚等言時中不當逞忿箝言官,帝乃切責時中,令歸聽勘。嘉靖十年四月起復為兵部尚書。御史郭希愈請重兵部侍郎之選,以邊臣有才者兩人分掌邊方、內地軍務。吏部議從之。時中言非祖宗臨時遣將意,帝遂從其議。帝欲用王憲於兵部,乃調時中刑部尚書。坐論御史馮恩獄,落職閒住。始,恩疏詆時中,及是以寬恩得罪,時稱為長者。久之,遇赦,復官致仕。

周期雍,字汝和,江西寧州人。正德三年進士。授南京御史。劉瑾既誅,為瑾斥者悉起,而給事中李光翰、任惠、徐蕃、牧相、徐暹、趙士賢,御史貢安甫、史良佐、曹閔、王弘、葛浩、姚學禮、張鳴鳳、王良臣、徐鈺、趙佑、楊璋、朱廷聲、劉玉,部郎李夢陽、王綸、孫磐等,以兼劾群閹未得錄。期雍偕同官王佩力請,皆召用。兵部尚書王敞附瑾進,期雍請斥之。焦芳、劉宇猶在列,而劉大夏、韓文、楊守隨、林瀚、張敷華未雪,期雍皆極論。陳金討江西賊,縱苗殺掠,期雍發其狀。尋清軍廣東,劾鎮守武定侯郭勛,金與勛皆被責。出為福建僉事。宸濠反,簡銳卒赴討。會賊平乃還。嘉靖初,為浙江參議。討平溫、處礦盜,予一子官。再遷湖廣按察使。九年擢右僉都御史,巡撫順天。薊州、密雲關堡數十,以避寇警移入內地,關外益無備,期雍悉修復之。數列上便宜。入為大理卿,歷刑部左、右侍郎,右都御史,拜刑部尚書。大計京官,言官劾期雍納賄。吏部白其誣,詔為飭言者。十九年,郭勛修前郤,因風霾勸帝罷免大臣,期雍遂去位。家居十年卒。

唐龍,字虞佐,蘭谿人。受業於同縣章懋,登正德三年進士。除郯城知縣。稟大盜劉六,數敗之,加俸二等。父喪,服除,徵授御史,出按雲南。錢寧義父參將盧和坐罪當死,寧為奏辯,下鎮撫覆勘。會遣官錄囚,受寧屬欲出和,為龍所持,卒正其罪。土官鳳朝明坐罪死,革世職。寧令滇人為保舉,而矯旨許之。龍抗疏爭,寢其事。再按江西,疏趣張忠、許泰班師。三司官從宸濠叛者猶居位,龍召數之曰:「脅從罔治,謂凡民耳。若輩讀書食祿,何壎顏乃爾。」立收其印綬。擢陜西提學副使,遷山西按察使,召為太僕卿。嘉靖七年改右僉都御史,總督漕運兼巡撫鳳陽諸府。奏罷淮西官馬種牛,罷壽州正陽關榷稅,通、泰二州虛田租及漕卒船料,民甚德之。召拜左副都御史,歷吏部左、右侍郎。

十一年,陜西大饑。吉囊擁眾臨邊,延綏告警。詔進龍兵部尚書,總制三邊軍務兼理振濟,齎帑金三十萬以行。龍奏行救荒十四事。時吉囊居套中,西抵賀蘭山,限以黃河不得渡,用十皮為渾脫,渡入山後。俺答亦自豐州入套為患。龍用總兵官王效、梁震,數敗敵,屢被獎賚。召為刑部尚書。大猾劉東山構陷建昌侯張延齡,興大獄。延齡,昭聖皇太后母弟,帝所惡也。吏坐獄不窮竟去者數十人,龍獨執正東山罪。「大禮」大獄及諸建言獲罪者,廷臣屢請寬,不能得。會九廟成,覃恩,龍錄上充軍應赦者百四十人,率得宥,所不原惟豐熙、楊慎、王元正、馬錄、呂經、馮恩、劉濟、邵經邦而已。考尚書六年滿,加太子少保。以母老乞歸侍養。久之,用薦起南京刑部尚書,就改吏部。兵部尚書戴金罷,召龍代之。太廟成,加太子太保。尋代熊浹為吏部尚書。龍有才,居官著勞績。及為吏部,每事咨僚佐。年老多疾,輒為所欺。御史陳九德劾前選郎高簡罔上行私,并論龍衰暮,乃下簡詔獄。龍引疾,未報。吏科楊上林、徐良輔復論簡。詔杖簡六十遣戍。上林、良輔以不早言罷職,龍黜為民。龍已有疾,輿出國門卒。後數年,子修撰汝楫疏辯。詔復官,贈少保,謚文襄。龍故與嚴嵩善。龍之罷,實夏言主之。而汝楫素附嵩,得第一人及第。官至左諭德。後坐嵩黨奪官。

王杲,字景初,汶上人。正德九年進士。授臨汾知縣。擢御史,巡視陜西茶馬。帝遣中官分守蘭、靖。杲言窮邊饑歲,不宜設官累民,不報。嘉靖三年,帝將遣中官督織造於蘇、杭,杲疏諫,不納。久之,擢太僕少卿,改大理,再遷左副都御史,進戶部右侍郎。河南大饑,命杲往振。杲請急發帑金,詔齎臨清倉銀五萬兩以行。既至,復請發十五萬兩。全活不可勝計。事竣,賜銀幣。尋以右都御史總督漕運。故事,繕運艘,軍三民七。總兵官顧寰以軍民困敝,請發兩淮餘鹽銀七十萬,戶部尚書李如圭不可。杲請改折兩年漕運十之三,以所省轉輸費治運艘,勿重困軍民,報可。踰年,入為戶部尚書。后父安平侯方銳乞張家莊馬房地。杲言此地二千餘頃,正供所出,不可許,宜以大慈恩寺入官地二十頃予之。帝從其議。時國儲告匱,諸邊請增餉無虛月,四方多水旱,給事中李文進請議廣儲蓄。杲列九事以獻,已又上制財用十事,帝咸納之。舊制,歲漕四百萬石。杲以粟有餘而用不足,遇災傷率改折以便民。一日,帝見改折者過半,大驚,以詰戶部,杲等引罪。敕自今務遵祖制,毋輕變。杲掌邦計,事無不辦,帝深倚之。後有詔買龍涎香,久不進,帝以此不悅。給事中馬錫劾杲及巡倉御史艾朴受賄,給事中厲汝進言倉場尚書王亦然,並下獄。杲、朴遣戍,斥為民。杲竟卒於雷州戍所。隆慶初,給事中辛自修等訟杲冤。詔復官,賜祭葬,贈太子太保。

王,句容人。由進士除吉安推官。從王守仁平宸濠,遷大理寺副。爭「大禮」,下獄廷杖。累遷右副都御史,巡撫江西。歷兩京戶部侍郎,出督漕運,進尚書。歷官著清操。

周用,字行之,吳江人。弘治十五年進士。授行人。正德初,擢南京兵科給事中。父憂服闋,留補禮科。已,乞南。改南京兵科。諫迎佛烏斯藏及以中旨遷黜尚書、都給事中等官,且請治鎮守江西中官黎安罪。出為廣東參議,預平番禺盜,有功。歷浙江、山東副使。擢福建按察使,改河南右布政使。代監司鞫南陽滯獄,獄為之空。嘉靖八年擢右副都御史,巡撫南、贛。召協理院事。歷吏部左、右侍郎。以起廢不當,尚書汪鋐委罪僚屬,乃調用南京刑部。就遷右都御史,工、刑二部尚書。九廟災,自陳致仕。用端亮有節概。既罷,中外皆惜之,頻有推薦。久之,以工部尚書起督河道,數月,改漕運。未上,召拜左都御史。二品九年滿,加太子少保。二十五年代唐龍為吏部尚書。明年卒官。贈太子太保,謚恭肅。曾孫宗建,自有傳。

用掌憲時,慎自持而已,無所獻替。其後宋景、屠僑繼之,大略皆廉潔,與用相似。景未久卒,而僑居職八年。屬嚴嵩柄政,風紀不振。議丁汝夔獄,受杖不能去。

宋景,字以賢,奉新人。弘治十八年進士。知睢州。正德五年入為河南道御史。故事,知州無改御史者,劉瑾創之也。瑾誅,景引疾去。嘉靖三年以薦補浙江僉事,進山西副使。民饑為盜,殺守稟指揮。景樹幟,令被脅者赴之。賊咸歸命,乃擒斬其魁。四遷山西左布政使,累官南京吏、工二部尚書。改兵部,參贊機務。入為左都御史。卒,贈太子少保、吏部尚書,謚莊靖。

屠僑,字安卿,吏部尚書滽再從子也。正德六年進士。授御史。巡視居庸諸關。武宗遣中官李嵩等捕虎豹,僑力言不可。世宗時,歷左都御史。卒,贈少保,謚簡肅。

聞淵,字靜中,鄞人。弘治十八年進士。初授禮部主事,已,改刑部。楊一清為吏部,調淵稽勳員外郎。歷考功郎中,改掌文選,遷南京右通政。嘉靖初,擢應天府尹,改尹順天。累遷南京兵部右侍郎,攝部事。薦馬永等十餘人。召為刑部右侍郎,遷左。進南京刑部尚書,就移吏部。召為刑部尚書。周用卒,代為吏部尚書。侍郎徐階得帝眷,前尚書率推讓之。淵自以前輩,事取獨斷。大學士夏言柄政,淵老臣,不能委曲徇。及後議言獄,淵謂言事只任意,跡涉要君,請帝自裁決。帝大怒,切責淵。嚴嵩既殺言,勢益橫,部權無不侵,數以小故奪淵俸。淵年七十矣,遂乞骸骨歸。家居十四年卒。先累加太子太保,卒贈少保,謚莊簡。

淵居官始終一節。晚扼權相,功名頗損。在南刑部時,張璁先為曹屬,嘗題詩於壁,屬淵勒石後堂。淵曰:「此尚書堂也,吾敢以相君故,為郎官勒石耶?」

劉訒,鄢陵人。父璟,刑部尚書。訒登正德十二年進士,為寧國推官,攝蕪湖縣事。武宗南巡,中貴索賄不得,繫訒詔獄。世宗立,復官。尋擢御史,遷南京通政參議。歷南京刑部尚書,召改北。

初,帝幸承天,河南巡撫胡纘宗嘗以事笞陽武知縣王聯。聯尋為巡按御史陶欽夔劾罷。聯素兇狡,嘗歐其父良,論死。久之,以良請出獄。復坐殺人,求解不得。知帝喜告訐,乃摭纘宗迎駕詩「穆王八駿」語為謗詛。言纘宗命己刊布,不從,屬欽夔論黜,羅織成大辟。候長至日,令其子詐為常朝官,闌入闕門訟冤。凡所不悅,若副都御史劉隅,給事中鮑道明,御史胡植、馮章、張洽,參議朱鴻漸,知府項喬、賈應春等百十人,悉構入之。帝大怒,立遣官捕纘宗等下獄,命訒會法司嚴訊。訒等盡得其誣罔,仍坐聯死,當其子詐冒朝官律斬,而為纘宗等乞宥。帝既從法司奏坐聯父子辟,然心嗛纘宗,頗多詰讓,下禮部都察院參議。嚴嵩為之解,乃革纘宗職,杖四十。訒亦除名,法司正貳停半歲俸,郎官承問者下詔獄。嵩以對制平獄有功,令兼支大學士俸,嵩辭乃允。時法官率骫法徇上意。稍執正,譴責隨至。訒於是獄能持法,身雖黜,而天下稱之。

胡纘宗,陜西秦安人。正德三年進士。由檢討出為嘉定判官。歷山東巡撫,改河南。

孫應奎,字文宿,洛陽人。正德十六年進士。授章邱知縣。嘉靖四年入為兵科給事中,上疏言:「輔臣之任,必忠厚鯁亮、純白堅定者乃足當之。今大學士楊一清雖練達國體,而雅性尚通,難以獨任。張璁學博性偏,傷於自恃,猶飭厲功名,當抑其過而用之。至於桂萼以梟雄桀驁之資,作威福,納財賄,阻抑氣節,私比黨與,勢侵六官,氣制言路,天下莫不怨憤。乞鑒別三臣賢否,以定用舍。」其意特右璁。而帝因其奏,慰留一清,戒諭璁、萼。既而同官王準、陸粲劾璁、萼罷相,準、粲亦下吏遠謫,以應奎首抗章不罪。未幾,劾吏部尚書方獻夫,帝頗納其言。獻夫援汪鋐為助,遂詘應奎議。再遷戶科左給事中。行人孽侃建言忤旨,下廷訊,詞連張璁。應奎與同官曹汴揖璁避,且上疏言狀。帝怒,下之詔獄,尋釋還職。十一年大計天下庶官,王準謫富民典史。應奎言汪鋐為璁、萼修郤,誣以不謹而黜之。乞復準官,責鋐,為黨比戒。吏部尚書王瓊亦言準當黜,乃謫應奎高平縣丞。屢遷湖廣副使,督采大木,坐累復逮繫。尋釋還。歷右副都御史,巡撫順天。召理院事,遷戶部侍郎,進尚書。

俺答犯京師後,羽書旁午徵兵餉。應奎乃建議加派。自北方諸府暨廣西、貴州外,其他量地貧富,驟增銀一百十五萬有奇,而蘇州一府乃八萬五千。御史郭仁,吳人也,詣應奎請減,不從。仁遂劾奏,應奎疏辨。帝以仁不當私屬,調之外。既而國用猶不足,應奎言:「今歲入二百萬,而諸邊費六百餘萬,一切取財法行之已盡。請令諸曹所隸官吏、儒士、廚役、校卒,悉去其冗者。而臣部出入贏縮之數,亦綜其大綱,列籍進御,使百司庶府咸知為國惜財。」報可。三十一年正月命應奎條上京邊備用芻糧之數。應奎言:「自臣入都至今,計正稅、加賦、餘鹽五百餘萬外,他所搜括又四百餘萬。而所出自諸邊年例二百八十萬外,新增二百四十五萬有奇,修邊振濟諸役又八百餘萬。」帝以耗費多,疑有侵冒,分遣科道官往諸邊核實。給事中徐公遴劾應奎粗疏自用,遂改南京工部尚書,以方鈍代。諸邊餉銀益增。鈍計無所出,請令諸臣條上理財策。議行二十九事,益纖屑傷大體。應奎就移戶部,致仕歸,卒。

應奎為諫官,屢犯權貴,以風節自厲。晚官計曹,一切為茍且計,功名大損於前。

有與應奎同姓名者,餘姚人,字文卿。由進士授行人,擢禮科給事中。疏劾汪鋐奸,忤旨下詔獄。已復杖闕下,謫華亭縣丞。鋐亦罷去。兩孫給諫之名,並震於朝廷。累官右副都御史,總理河道。踰年罷歸。為山東布政時,有創開膠萊河議者,應奎力言不可。入覲,與吏部尚書爭官屬賢否,時稱其直。

方鈍,巴陵人。掌戶部七年,廉慎無過。嚴嵩中之,詔改南京,遂乞骸骨歸。

聶豹,字文蔚,吉安永豐人。正德十二年進士。除華亭知縣。浚陂塘,民復業者三千餘戶。嘉靖四年召拜御史,巡按福建。出為蘇州知府。憂歸,補平陽知府。山西頻中寇,民無寧居。豹令富民出錢,罪疑者贖,得萬餘金,修郭家溝、冷泉、靈石諸關隘,練鄉勇六千守之。寇卻,廷議以豹為知兵。給事中劉繪、大學士嚴嵩皆薦之。擢陜西副使,備兵潼關。大計拾遺,言官論豹在平陽乾沒,大學士夏言亦惡豹,逮下詔獄,落職歸。

二十九年秋,都城被寇。禮部尚書徐階,豹知華亭時所取士也,為豹訟冤,言其才可大用。立召拜右僉都御史,巡撫順天。未赴,擢兵部右侍郎,尋轉左。仇鸞請調宣、大兵入衛,豹陳四慮,謂宜固守宣、大,宣、大安則京師安。鸞怒。伺豹過無所得,乃已。三十一年召翁萬達為兵部尚書,未至,卒,以豹代之。奏上防秋事宜,又請增築京師外城,皆報可。是年秋,寇大入山西,覆總兵官李淶軍,大掠二十日而去。總督蘇祐反以大捷聞,為巡按御史毛鵬所發,章下兵部。豹言:「寇雖有所掠,而我師斬獲過當,實上玄垂祐,陛下威靈所致。宜擇吉祭告,論功行賞。」帝喜。進秩任子者數十人,豹亦加太子少保,蔭錦衣世千戶。京師外城成,進太子少傅。南北屢奏捷,及類奏諸邊功,豹率歸功玄祐,祭告行賞如初。豹亦進太子太保。

當是時,西北邊數遭寇,東南倭又起,羽書日數至。豹本無應變才,而大學士嵩與豹鄉里,徐階亦入政府,故豹甚為帝所倚。久之,寇患日棘,帝深以為憂。豹卒無所謀畫,條奏皆具文,帝漸知其短。會侍郎趙文華陳七事致仕,侍郎朱隆禧請設巡視福建大臣,開海濱互市禁,豹皆格不行。帝大怒切責。豹震懾請罪,復辨增官、開市之非,再下詔譙讓。豹愈惶懼,條便宜五事以獻。帝意終不懌,降俸二級。頃之,竟以中旨罷,而用楊博代之。歸數年卒,年七十七。隆慶初,贈少保,謚貞襄。

豹初好王守仁良知之說,與辨難,心益服。後聞守仁歿,為位哭,以弟子自處。及繫獄,著《困辨錄》,於王守仁說頗有異同云。

李默,字時言,甌寧人。正德十六年進士。選庶吉士。嘉靖初,改戶部主事,進兵部員外郎。調吏部,歷驗封郎中。真人邵元節貴幸,請封誥,默執不予。十一年為武會試同考官。及宴兵部,默據賓席,欲坐尚書王憲上。憲劾其不遜,謫寧國同知。屢遷浙江左布政使,入為太常卿,掌南京國子監事。博士等官得與科道選,自默發之。歷吏部左、右侍郎,代夏邦謨為尚書。自正德初焦芳、張彩後,吏部無侍郎拜尚書者。默出帝特簡,蓋異數也。

嚴嵩柄政,擅黜陟權。默每持己意,嵩銜之。會推遼東巡撫,列布政使張臬、謝存儒以上。帝問嵩,嵩言其不任。奪默職為民,以萬鏜代。默掌銓僅七月。逾年,鏜罷,特旨復用默。已,命入直西內,賜直廬,許苑中乘馬。尋進太子少保。未幾,復命兼翰林學士。給事中梁夢龍劾默徇私,帝為責夢龍。會大計群吏,默戒門下謝賓客,同直大臣亦不得燕見,嵩甚恨。趙文華視師還,默氣折之。總督楊宜罷,嵩、文華欲用胡宗憲,默推王誥代,兩人恨滋甚。

初,文華為帝言餘倭無幾,而巡按御史周如斗以敗狀聞。帝疑,數詰嵩。文華謀所以自解,稔帝喜告訐。會默試選人策問,言「漢武、唐憲以英睿興盛業,晚節用匪人而敗」,遂奏默誹謗。且言:「殘寇不難滅,以督撫非人,敗衄。由默恨臣劾其同鄉張經,思為報復。臣論曹邦輔,即嗾給事中夏栻、孫浚媒孽臣。延今半載,疆事日非。昨推總督,又不用宗憲而用誥。東南塗炭何時解?陛下宵旰憂何時釋?」帝大怒,下禮部及法司議。奏默偏執自用,失大臣體;所引漢、唐事,非所宜言。帝責禮部尚書王用賓等黨護,各奪俸三月,而下默詔獄。刑部尚書何鰲遂引子罵父律絞。帝曰:「律不著臣罵君,謂必無也。今有之,其加等斬。」錮於獄,默竟瘐死。時三十五年二月也。

默博雅有才辨,以氣自豪。同考武試,得陸炳為門生。炳貴盛,力推轂。默由外吏驟顯,有所恃,不附嵩。凡有銓除,與爭可否,氣甚壯。然性褊淺,用愛憎為軒輊,頗私鄉舊,以恩威自歸,士論亦不甚附之。默既得罪,繼之者吳鵬、歐陽必進,視嵩父子意,承順惟謹,吏部權盡失。隆慶中,復默官,予祭葬。萬曆中,賜謚文愍。

萬鏜,字仕鳴,進賢人。父福,金華知府。鏜登弘治十八年進士。正德中,由刑部主事屢遷吏部文選郎中。司署火,下獄,贖還職。歷太常、大理少卿。世宗嗣位,以鏜嘗貽書知縣劉源清,令預防宸濠,賚金幣。尋遷順天府尹,累遷右副都御史。歷兵部侍郎、右都御史,皆南京。彗星見,應詔陳八事。中言:「人邪正相懸,而形迹易混。其大較有四:人主所取於下者,曰任怨,曰任事,曰恭順,曰無私;而邪臣之恣強戾、好紛更、巧逢迎、肆攻訐者,其跡似之。人主所惡於下者,曰避事,曰沽名,曰朋黨,曰矯激;而正臣之守成法、恤公議、體群情、規君失者,其迹似之。察之不精,則邪正倒置,而國是亂矣,此不可不慎也。治天下貴實不貴文。今陛下議禮制度考文,至明備矣,而於理財用人安民講武之道,或有缺焉。願輟聲容之繁飾,略太平之美觀,而專從事於實用,斯治天下之道得矣。至大禮大獄得罪諸臣,幽錮已久,乞量加寬錄。」帝大怒,斥為民,令吏部錮勿用。

家居十年,屢推薦,輒報罷。同年生嚴嵩柄政,援引之。湖廣蠟爾山蠻叛,起鏜副都御史,相機剿撫。鏜納土指揮田應朝策,誘致其酋,督兵破之。條上善後七事,帝咸報可。召鏜還。未幾,銅平酋龍子賢復叛,御史繆文龍言鏜剿撫皆失。詔下撫按官勘覆,歸罪於參將李經,事乃解。鏜得為兵部侍郎。遷南京刑、禮二部尚書。召掌刑部。俄代李默為吏部尚書。

鏜既為嵩所引,每事委隨,又頗通饋遺。撫治鄖陽都御史闕,鏜以通政使趙文華名上。會給事中朱伯辰劾文華,文華上言:「納言之職,例不外推。鏜意在出臣,又嗾所親伯辰論劾,欲去臣。且鏜以侍郎起用,乃朦朧奏二品九年滿,得加太子少保。又以不得一品,面謾腹誹,無大臣禮。」帝怒,遂與伯辰並黜為民。久之卒。隆慶初,復官,贈太子太保。

周延,字南喬,吉水人。嘉靖二年進士。除潛江知縣,改新會,擢兵科給事中。時議新建伯王守仁罪,將奪其爵。延抗疏為訟,坐謫太倉州判官。歷南京吏部郎中,出為廣東參政。撫安南,徵黎寇,皆預,有功。三遷廣東左布政使。以右副都御史巡撫應天。靖海寇林成亂。進兵部右侍郎,提督兩廣軍務。召為刑部左侍郎。歷南京右都御史,吏、兵二部尚書。

嘉靖三十四年召為左都御史。帝用給事中徐浦議,令廷臣及督撫各舉邊才。於是故侍郎郭宗皋,都御史曹邦輔、吳獄,祭酒鄒守益,修撰羅洪先,御史吳悌、方涯,主事唐樞,參政周大禮、曹亨,參議劉志,知府黃華在舉中。御史羅廷唯駁曰:「浦疏本言邊才,而今廷臣乃以清修、苦節、實學、懿行舉,去初議遠矣。況又有夤緣進者。是假明詔開倖門。」帝納其言,責吏部濫舉,命與都察院更議。延與尚書吳鵬等言所舉皆人望,公無私。帝終不悅,切責延等,而舉者悉報罷。世宗時,海內賢士大夫被斥者眾,及是舉上,稍冀復用,而為廷唯所阻,自是皆不復召矣。

延顏面寒峭,砥節奉公。權臣用事,政以賄成,延未嘗有染。然居臺端七年,無諫諍名。卒官,贈太子太保,謚簡肅。

延卒,歐陽必進代。踰月,遷吏部,乃以潘恩繼之。

恩,字子仁,上海人。嘉靖二年進士。授祁州知州,調繁鈞州。鈞,徽王封國也,宗戚豪悍,恩約束之。擢南京刑部員外郎。遷廣西提學僉事,署按察使事。有大猾匿靖江王所,捕之急,王不得已出之。憾恩,誣以事,按無實得免。累遷山東副使。御史葉經以試錄忤旨,並恩下詔獄,謫廣東河源典史。四遷,復為江西副使,進浙江左參政。按部海鹽,倭猝至,圍城數匝。恩與參將湯克寬、僉事姜頤力禦卻之。俄遷浙江左布政使,以右副都御史巡撫河南。偕按臣劾徽王載侖貪虐,遂奪國。伊王典示英驕橫,恩一切裁之。河南民素苦籓府,恩制兩悍王,名大著。久之,由刑部尚書改左都御史。

子允端,為刑部主事。吏部尚書郭朴,恩門生也,調之禮部。給事中張益劾允端奔兢,恩溺愛,朴徇私。帝置朴不問,改允端南京工部,令恩致仕。萬曆初,賜存問。卒年八十七。贈太子少保,謚恭定。

賈應春,字東陽,真定人。嘉靖二年進士。授南陽知縣,遷和州知州。入為刑部郎中。歷知潞安、開封二府。遷陜西副使。未赴,河南巡按陳蕙劾其貪濫,謫山東鹽運同知,蕙亦坐貶。久之,由漢陽知府復遷陜西副使,進右參政。寧羌賊起,會兵討平之。遷按察使,左、右布政使,皆在陜西。就拜右副都御史,巡撫其地。三十二年進兵部右侍郎,總督三邊軍務。俺答諸部歲擾邊,應春言:「諸邊間諜不通,每寇入莫測其向,我則無所不備。兵分勢孤,往往失事。夫寇將內犯,必聚眾治器,臘肉飼馬,傳箭祭旗,其形先露。而我民被掠者,間亦臨邊傳報,頗有左驗。使邊臣厚以官賞,令密偵候,視漫然散守者,功相十百。」乃定賞格以請。帝立從之。其秋,寇大入延綏,殺掠五千餘人。應春督諸將邀擊,獲首功二百四十,以捷聞。而巡按御史吉澄極言敗狀。帝竟錄應春功,官其一子。明年罷宣、大總督蘇祐,以應春代。時秋防將屆,代應春者江東未至,令仍舊任。套寇數萬人屯寧夏山後,先遣騎五百餘入掠。總兵官姜應熊守紅井以綴敵,而密遣精兵薄其營,斬首百四十餘級,進應春右都御史。踰月,寇別部入永昌、西寧,為守將所破。番人入鎮羌,總兵官王繼祖擊敗之,並賜應春銀幣。久之,寇五千騎犯環慶,為都督袁正所破,掠莊涼,守將邀斬百二十人,再予應春一子官。在鎮數載,築邊垣萬一千八百餘丈,以花馬池閒田二萬頃給軍屯墾,邊人賴之。徵拜南京戶部尚書。論邊垣功,進秩一等。旋召為刑部尚書,改戶部。國用不足,應春以為言。因命征不及七分者,所司毋遷官。漕政廢弛,運艘多逋負,亦以應春言重其罰。歲餘,致仕去。卒,贈太子太保。

張永明,字鐘誠,烏程人。嘉靖十四年進士。除蕪湖知縣。獻皇后梓宮南祔,所過繁費不貲。永明堊江岸佛舍為殿,供器飾箔金,財用大省。尋擢南京刑科給事中。寇入大同,山西總督樊繼祖,巡撫史道、陳講等不能禦,永明偕同官論其罪。已,又劾兵部尚書張瓚黷貨誤國,又劾大學士嚴嵩及子世蕃貪污狀。已,又劾兵部尚書戴金為御史巡鹽時,增餘鹽羨銀,阻壞邊計。疏雖不盡行,中外憚之。

出為江西參議。累遷雲南副使,山西左布政使。以右副都御史巡撫河南。伊王典示英恣橫,永明發其惡,後竟伏辜。四十年遷刑部右侍郎。未上,改吏部,進左。尋拜刑部尚書。居數月,改左都御史。條上飭歷撫按六事。御史黃廷聘按浙歸,道湘潭,慢知縣陳安。安發其裝,得所攜金銀貨幣。廷聘皇恐謝,乃還之。永明聞,劾罷廷聘。浙江參政劉應箕先為廷聘論罷,見廷聘敗,摭其陰事自辨。永明惡之,劾應箕,亦斥。

故事,京官考滿,自翰林外皆報名都察院,修庭謁禮。後吏部郎恃權,張濂廢報名,陸光祖廢庭謁。永明榜令遵故事,列儀節奏聞,詔諸司遵守。郎中羅良當考滿,先詣永明邸,約免報名庭謁乃過院。永明怒,疏言:「此禮行百年,非臣所能損益。良輕薄無狀,當罷。又卿貳大臣考滿,詣吏部與堂官相見訖,即詣四司門揖,司官輒南面答揖,亦非禮,當改正。」良疏辨,奪俸。詔禮部會禮科議之,奏言:「永明議是。自今吏部郎其承舊制。九卿翰林官揖四司,當罷。」詔可。

永明素清謹。掌憲在嚴嵩罷後,以整飭綱維為己任。會給事中魏時亮劾,永明力求去,詔許馳驛歸。明年卒。贈太子少保,謚莊僖。

胡松,字汝茂,滁人。幼嗜學,嘗輯古名臣章奏,慨然有用世志。登嘉靖八年進士,知東平州。設方略捕盜,民賴以安。再遷南京禮部郎中,歷山西提學副使。

三十年秋,上邊務十二事,謂:

去秋俺答掠興、嵐,即傳箭徵兵,克期深入。守臣皆諗聞之。而巡撫史道、總兵官王陛等備禦無素。待其壓境,始以求貢上聞。又陰致賄遺,令勿侵己分地,冀嫁禍他境。今山西之禍,實大同貽之。宜亟置重典,以厲諸鎮。

大同自兵變以來,壯士多逃漠北為寇用,今宜招使歸。有攜畜產器械來者,聽其自有。更給牛種費,優復數年。則我捐金十萬,可得壯士二萬。拊而用之,皆勁旅也。孰與棄之以資強敵哉!

大同最敵衝,為鎮巡者較諸邊獨難。今宜不拘資格,精擇其人。豐給祿廩,使得收召猛士,畜豢健丁。又久其期,非十年不得代。彼知不可驟遷,必不為茍且旦夕計,而邊圉自固。又必稍寬文綱,非大乾憲典,言官毋得輕劾,以壞其成功。

至用間之道,兵家所貴。今寇諜獲於山西者已數十人,他鎮類是。故我之虛實,彼無不知。今宜厚養死士,潛縱遣之。得間則斬其名王、部長及諸用事貴人。否亦可覘強弱虛實,而陰為備。

又寇貪而好利,我誠不愛金帛。東賂黃、毛三衛以牽其左,西收亦不刺遺種,予善地,以綴其右,使首尾掣曳,自相狼顧,則我可起承其敝,坐收全勝矣。

他所條析,咸切邊計。帝嘉其忠懇,進秩左參政。

松疏上,當事者已惡其侵官。及遷擢,益忌之。不畀以兵柄,令於三關聽用,欲因以陷之。寇大入,抵太原。給事中馮良知遂劾松建言冒賞,無寸功。紀功科道官張堯年、王珩劾總兵官張達等,並論松虛議無補,遂斥為民。家居十餘年。屢薦,輒報罷。至三十五年,以趙文華言,起陜西參政,分守平涼。復條嚴保甲、均賦稅、置常平、簡伉健數事。三遷江西左布政使,以右副都御史巡撫其地。所部多盜,松奏設南昌、南豐、萬安三營,遣將討捕,以次削平。進兵部右侍郎,巡撫如故。以會討廣東巨寇張璉及援閩破倭功,兩賜銀幣。居三年,召理部事。進左侍郎,改吏部。遷南京兵部尚書,參贊機務。代郭朴為吏部尚書。奏言:「撫按舉劾,每舉數十人,虛譽浮詞,往往失實。所劾犯贓,僅擬降調;罷軟貪殘,僅擬改教。賞罰不當,人何所激勸?且巡撫歲終例有冊,第屬吏賢否,今皆寢閣,乞申飭其欺玩者。」帝嘉納之。

松潔己好修,富經術,鬱然有聲望。晚主銓柄,以振拔淹滯為己任。甫七月,病卒。贈太子少保,謚恭肅。

時又有胡松者,字茂卿,績溪人。正德九年進士。嘉請時為御史。桂萼薦王瓊,松論之。忤旨,謫廉州推官。累官工部尚書。伊王欲拓其洛陽府第,計直十萬金,以十二賕嚴嵩,期必得。松據祖制爭,乃止。俺答入寇,仇鸞以邊眾入衛,欲悉召其眾實京師,移武庫仗於營,便給調。松言邊兵外也而內之,武庫仗內也而外之,非所以重肘腋,杜微慎防也,執弗許。尋引疾歸。卒年八十三。居家以孝友稱。

趙炳然,字子晦,劍州人。嘉靖十四年進士。除新喻知縣。徵拜御史。與給事中李文進核宣、大、山西兵餉。劾前後督撫樊繼祖、史道,監司楊銳,指揮馮世彪等一百七十七人侵冒罪,坐謫有差。條上備邊十二事。歷按雲南、浙江。擢大理寺丞,進少卿。尋改右僉都御史,巡撫湖廣。進左副都御史,協理院事。

浙江、福建總督胡宗憲下獄,詔罷總督毋設。大學士徐階以浙江寇甫平,請設巡撫綏輯,遂進炳然兵部右侍郎兼右僉都御史往任之。浙罹兵燹久,又當宗憲汰侈後,財匱力絀。炳然廉以率下,悉更諸政令不便者,仍奏減軍需之半。民皆尸祝之。

福建巡撫游震得請浙兵剿賊。詔發義烏精兵一萬,命副總兵戚繼光將以往,仍諭炳然協剿。炳然言:「福建所以致亂者,由將吏撫馭無術,民變為兵,兵變為盜耳。今又驅浙兵以赴閩急,竊懼浙之復為閩也。請令一意團練士著,使人各為用,家自為守,急則兵,緩則農,然後聚散兩有所歸。即不得已而召募,亦必先本土後鄰壤,庶無釀禍本。」又條上防海八事,中言:「蘇、松、浙江水師皆統於總兵,駐定海;陸師皆統於副總兵,駐金山衛,並受總督節制。今督府既革,則已判為二鎮,彼此牽制,不得調發。請畫地分轄,各兼水陸軍務。」俱報可。其年,繼光破賊,瀕海餘寇流入浙江。官軍迎戰於連嶼、陡橋、石坪,斬首百餘級。新倭復犯石坪,將士乘勝殲之。炳然以援剿功,再賜金幣,進右都御史兼兵部右侍郎。

給事中辛自修劾罷戎政都御史李鐩,請擇素知兵者代之。乃召炳然為兵部尚書,協理戎政。踰年,詔兼右都御史,總督宣、大、山西軍務。新平、平遠、保平三堡密邇宣府,舊屬大同。天城相去六十里,孤懸塞外,隔崇山,寇騎時出沒。炳然奏添設參將,別為一營,報可。尋以總兵官馬芳等卻敵功,被賚。已,召還部,代楊博為尚書。考滿,加太子少保。

炳然清勤練達,所至有聲績。隆慶初,以病乞休去。卒,贈太子太保,謚恭襄。

贊曰:世宗朝,璁、萼、言、嵩相繼用事,六卿之長不得其職。大都波流茅靡,淟,忍取容。廖紀以下諸人,其矯矯者與!應奎司邦計,不能節以制度,顧務加賦以病民。豹也碌碌,彌無足觀矣。


\end{pinyinscope}