\article{列傳第一百七十 儒林一}

\begin{pinyinscope}
粵自司馬遷、班固創述《儒林》,著漢興諸儒修明經藝之由,朝廷廣厲學官之路,與一代政治相表裏。後史沿其體製,士之抱遺經以相授受者,雖無他事業,率類次為篇。《宋史》判《道學》、《儒林》為二,以明伊、雒淵源,上承洙、泗,儒宗統緒,莫正於是。所關於世道人心者甚巨,是以載籍雖繁,莫可廢也。

明太祖起布衣,定天下,當干戈搶攘之時,所至徵召耆儒,講論道德大學問明王守仁著。為其在嵇山書院講授《大學》的記,修明治術,興起教化,煥乎成一代之宏規。雖天亶英姿,而諸儒之功不為無助也。制科取士,一以經義為先,網羅碩學。嗣世承平,文教特盛,大臣以文學登用者,林立朝右。而英宗之世,河東薛瑄以醇儒預機政,雖弗究於用,其清修篤學,海內宗焉。吳與弼以名儒被薦,天子修幣聘之殊禮,前席延見,想望風采,而譽隆於實,詬誶叢滋。自是積重甲科,儒風少替。白沙而後,曠典缺如。

原夫明初諸儒,皆朱子門人之支流餘裔,師承有自,矩矱秩然。曹端、胡居仁篤踐履,謹繩墨離合「相禪而無窮」。以「中庸」為道的準則,重視「觀」對,守儒先之正傳,無敢改錯。學術之分,則自陳獻章、王守仁始。宗獻章者曰江門之學,孤行獨詣,其傳不遠。宗守仁者曰姚江之學,別立宗旨,顯與朱子背馳,門徒遍天下,流傳逾百年,其教大行,其弊滋甚。嘉、隆而後,篤信程、朱,不遷異說者,無復幾人矣。要之,有明諸儒,衍伊、雒之緒言,探性命之奧旨,錙銖或爽,遂啟岐趨,襲謬承訛,指歸彌遠。至專門經訓授受源流,則二百七十餘年間,未聞以此名家者。經學非漢、唐之精專,性理襲宋、元之糟粕,論者謂科舉盛而儒術微,殆其然乎。

今差別其人,準前史例,作《儒林傳》。有事功可見,列於正傳者,茲不復及。其先聖、先賢後裔,明代亟為表章,衍聖列爵上公,與國終始。其他簪纓逢掖,奕葉承恩,亦儒林盛事也。考其原始,別自為篇,附諸末簡,以備一代之故云。

○范祖幹葉儀等謝應芳汪克寬梁寅趙汸陳謨薛瑄閻禹錫周蕙等胡居仁餘祐蔡清陳琛林希元等羅欽順曹端吳與弼胡九韶等陳真晟呂柟呂潛等邵寶王問楊廉劉觀孫鼎李中馬理魏校王應電王敬臣周瑛潘府崔銑何瑭唐伯元黃淳耀弟淵耀

范祖乾,字景先,金華人。從同邑許謙遊,得其指要。其學以誠意為主,而嚴以慎獨持守之功。太祖下婺州,與葉儀並召。祖乾持《大學》以進,太祖問治道何先,對曰:「不出是書。」太祖令剖陳其義,祖乾謂帝王之道,自修身齊家以至治國平天下,必上下四旁,均齊方正,使萬物各得其所,而後可以言治。太祖曰:「聖人之道,所以為萬世法。吾自起兵以來,號令賞罰,一有不平,何以服眾。夫武定禍亂,文致太平,悉是道也。」深加禮貌,命二人為諮議,祖乾以親老辭歸。李文忠守處州,特加敬禮,恒稱之為師。祖幹事親孝,父母皆八十餘而終。家貧不能葬,鄉里共為營辦,悲哀三年如一日。有司以聞,命表其所居曰純孝坊,學者稱為純孝先生。

葉儀,字景翰,金華人。受業於許謙,謙誨之曰:「學者必以五性人倫為本,以開明心術、變化氣質為先。」儀朝夕惕厲,研究奧旨。已而授徒講學,士爭趨之。其語學者曰:「聖賢言行,盡於《六經》、《四書》,其微詞奧義,則近代先儒之說備矣。由其言以求其心,涵泳從容,久自得之,不可先立己意,而妄有是非也。」太祖克婺州,召見,授為諮議,以老病辭。已而知府王宗顯聘儀及宋濂為《五經》師,非久亦辭歸,隱居養親。所著有《南陽雜槁》。吳沉稱其理明識精,一介不茍。安貧樂道,守死不變。

門人何壽朋,字德齡,亦金華人。窮經守志,不妄干人。洪武初,舉孝廉,以二親俱老辭。父歿,舍所居宅易地以葬。學者因其自號,稱曰歸全先生。

同邑汪與立,字師道,祖乾門人。其德行與壽朋齊名而文學為優。隱居教授,以高壽終。

謝應芳,字子蘭,武進人也。自幼篤志好學,潛心性理,以道義名節自勵。元至正初,隱白鶴溪上。構小室,顏曰「龜巢」,因以為號。郡辟教鄉校子弟,先質後文,諸生皆循循雅飭。疾異端惑世,嘗輯聖賢格言、古今明鑒為《辨惑編》。有舉為三衢書院山長者,不就。及天下兵起,避地吳中,吳人爭延致為弟子師。久之,江南底定,始來歸,年逾七十矣。徙居芳茂山,一室蕭然,晏如也。有司徵修郡志,強起赴之。年益高,學行益劭。達官縉紳過郡者,必訪於其廬,應芳布衣韋帶與之抗禮。議論必關世教,切民隱,而導善之志不衰。詩文雅麗蘊藉,而所自得者,理學為深。卒年九十七。

汪克寬,字德一,祁門人。祖華,受業雙峰饒魯,得勉齋黃氏之傳。克寬十歲時,父授以雙峰問答之書,輒有悟。乃取《四書》,自定句讀,晝夜誦習,專勤異凡兒。後從父之浮梁,問業於吳仲迂,志益篤。元泰定中,舉應鄉試,中選。會試以答策伉直見黜,慨然棄科舉業,盡力於經學。《春秋》則以胡安國為主,而博考眾說,會萃成書,名之曰《春秋經傳附錄纂疏》。《易》則有《程朱傳義音考》。《詩》有《集傳音義會通》。《禮》有《禮經補逸》。《綱目》有《凡例考異》。四方學士,執經門下者甚眾。至正間,蘄、黃兵至,室廬貲財盡遭焚掠。簞瓢屢空,怡然自得。洪武初,聘至京師,同修《元史》。書成將授官,固辭老疾。賜銀幣,給驛還。五年冬卒,年六十有九。

梁寅,字孟敬,新喻人。世業農,家貧,自力於學,淹貫《五經》、百氏。累舉不第,遂棄去。辟集慶路儒學訓導,居二歲,以親老辭歸。明年,天下兵起,遂隱居教授。太祖定四方,徵天下名儒修述禮樂。寅就徵,年六十餘矣。時以禮、律、制度,分為三局,寅在禮局中,討論精審,諸儒皆推服。書成,賜金幣,將授官,以老病辭,還。結廬石門山,四方士多從學,稱為梁五經,又稱石門先生。鄰邑子初入官,詣寅請教。寅曰:「清、慎、勤,居官三字符也。」其人問天德王道之要,寅微笑曰:「言忠信,行篤敬,天德也。不傷財,不害民,王道也。」其人退曰:「梁子所言,平平耳。」後以不檢敗,語人曰:「吾不敢再見石門先生。」寅卒,年八十二。

趙汸,字子常,休寧人。生而姿稟卓絕。初就外傅,讀朱子《四書》,多所疑難,乃盡取朱子書讀之。聞九江黃澤有學行,往從之游。澤之學,以精思自悟為主。其教人,引而不發。汸一再登門,乃得《六經》疑義千餘條以歸。已,復往,留二歲,得口授六十四卦大義與學《春秋》之要。後復從臨川虞集游,獲聞吳澄之學。乃築東山精舍,讀書著述其中。雞初鳴輒起,澄心默坐。由是造詣精深,諸經無不通貫,而尤邃於《春秋》。初以聞於黃澤者,為《春秋師說》三卷,復廣之為《春秋集傳》十五卷。因《禮記》經解有「屬辭比事《春秋》教」之語,乃復著《春秋屬辭》八篇。又以為學《春秋》者,必考《左傳》事實為先,杜預、陳傅良有得於此,而各有所蔽,乃復著《左氏補注》十卷。當是時,天下兵起,汸轉側干戈間,顛沛流離,而進修之功不懈。太祖既定天下,詔修《元史》,徵汸預其事。書成,辭歸。未幾卒,年五十有一。學者稱東山先生。

陳謨,字一德,泰和人。幼能詩文,邃於經學,旁及子史百家,涉流探源,辨析純駁,犁然要於至當。隱居不求仕,而究心經世之務。嘗謂:「學必敦本,莫加於性,莫重於倫,莫先於變化氣質。若禮樂、刑政、錢穀、甲兵、度數之詳,亦不可不講習。」一時經生學子多從之游。事親孝,友於其弟。鄉人有為不善者,不敢使聞。洪武初,徵詣京師,賜坐議學。學士宋濂、待制王禕請留為國學師,謨引疾辭歸。屢應聘為江、浙考試官,著書教授以終。

薛瑄,字德溫,河津人。父貞,洪武初領鄉薦,為元氏教諭。母齊,夢一紫衣人謁見,已而生瑄。性穎敏,甫就塾,授之《詩》、《書》,輒成誦,日記千百言。及貞改任滎陽,瑄侍行。時年十二,以所作詩賦呈監司,監司奇之。既而聞高密魏希文、海寧范汝舟深於理學,貞乃並禮為瑄師。由是盡焚所作詩賦,究心洛、閩淵源,至忘寢食。後貞復改官鄢陵。瑄補鄢陵學生,遂舉河南鄉試第一,時永樂十有八年也。明年成進士。以省親歸。居父喪,悉遵古禮。宣德中服除,擢授御史。三楊當國,欲見之,謝不往。出監湖廣銀場,日探性理諸書,學益進。以繼母憂歸。

正統初還朝,尚書郭璡舉為山東提學僉事。首揭白鹿洞學規,開示學者。延見諸生,親為講授。才者樂其寬,而不才者憚其嚴,皆呼為薛夫子。王振語三楊:「吾鄉誰可為京卿者?」以瑄對,召為大理左少卿。三楊以用瑄出振意,欲瑄一往見,李賢語之。瑄正色曰:「拜爵公朝,謝恩私室,吾不為也。」其後議事東閣,公卿見振多趨拜,瑄獨屹立。振趨揖之,瑄亦無加禮,自是銜瑄。

指揮某死,妾有色,振從子山欲納之,指揮妻不肯。妾遂訐妻毒殺夫,下都察院訊,已誣服。瑄及同官辨其冤,三卻之。都御史王文承振旨,誣瑄及左、右少卿賀祖嗣、顧惟敬等故出人罪,振復諷言官劾瑄等受賄,並下獄。論瑄死,祖嗣等末減有差。繫獄待決,瑄讀《易》自如。子三人,願一子代死,二子充軍,不允。及當行刑,振蒼頭忽泣於爨下。問故,泣益悲,曰:「聞今日薛夫子將刑也。」振大感動。會刑科三覆奏,兵部侍郎王偉亦申救,乃免。

景帝嗣位,用給事中程信薦,起大理寺丞。也先入犯,分守北門有功。尋出督貴州軍餉,事竣,即乞休,學士江淵奏留之。景泰二年,推南京大理寺卿。富豪殺人,獄久不決,瑄執置之法。召改北寺。蘇州大饑,貧民掠富豪粟,火其居,蹈海避罪。王文以閣臣出視,坐以叛,當死者二百餘人,瑄力辨其誣。文恚曰:「此老倔強猶昔。」然卒得減死。屢疏告老,不許。英宗復辟,拜禮部右侍郎兼翰林院學士,入閣預機務。王文、于謙下獄,下群臣議,石亨等將置之極刑。瑄力言於帝,後二日文、謙死,獲減一等。帝數見瑄,所陳皆關君德事。已,見石亨、曹吉祥亂政,疏乞骸骨。帝心重瑄,微嫌其老,乃許之歸。

瑄學一本程、朱,其修已教人,以復性為主,充養邃密,言動咸可法。嘗曰:「自考亭以還,斯道已大明,無煩著作,直須躬行耳。」有《讀書錄》二十卷,平易簡切,皆自言其所得,學者宗之。天順八年六月卒,年七十有二。贈禮部尚書,謚文清。弘治中,給事中張九功請從祀文廟,詔祀於鄉。已,給事中楊廉請頒《讀書錄》於國學,俾六館誦習。且請祠名,詔名「正學」。隆慶六年,允廷臣請,從祀先聖廟庭。

其弟子閻禹錫,字子與,洛陽人。父端,舉河南鄉試第一,為教諭,卒。禹錫方九歲,哭父幾滅性。長博涉群書,領正統九年鄉薦,除昌黎訓導。以母喪歸,廬墓三年,詔以孝行旌其閭。聞河津薛瑄講濂、洛之學,遂罷公車,往受業。久之,將歸,瑄送至里門,告之曰:「為學之要,居敬窮理而已。」禹錫歸,得其大指,益務力行。

天順初,大學士李賢薦為國子學正。請嚴監規以塞奔競,復武學以講備禦,帝皆從之。尋升監丞,忤貴幸,左遷徽州府經歷。諸生伏闕乞留,不允。再遷至南京國子監丞,掌京衛武學,四為同考官,超拜監察御史。督畿內學,取周子《太極圖》、《通書》為士子講解,一時多士皆知響學。成化十二年卒,年五十一。

周蕙,字廷芳,泰州人。為臨洮衛卒,戍蘭州。年二十,聽人講《大學》首章,惕然感動,遂讀書。州人段堅,薛瑄門人也,時方講學於里。蕙往聽之。與辨析,堅大服。誨以聖學,蕙乃研究《五經》。又從學安邑李昶。昶,亦瑄門人也,由舉人官清水教諭。學使者歎其賢,薦昶代己,命未下而卒。蕙從之久,學益邃。恭順侯吳瑾鎮陜西,欲聘為子師,固辭不赴。或問之,蕙曰:「吾軍士也,召役則可。若以為師,師豈可召哉?」瑾躬送二子於其家,蕙始納贄焉。後還居泰州之小泉,幅巾深衣,動必由禮。州人多化之,稱為小泉先生。以父久遊江南不返,渡揚子江求父,舟覆溺死。蕙門人著者,薛敬之、李錦、王爵、夏尚樸。

敬之,字顯思,渭南人。五歲好讀書,不逐群兒戲。長從蕙游,雞鳴候門啟,輒灑掃設座,跪而請教。嘗語人曰:「周先生躬行孝弟,學近伊、洛,吾以為師。陜州陳雲逵忠信狷介,事必持敬,吾以為友。」憲宗初,以歲貢生入國學,與同舍陳獻章並有盛名。會父母相繼歿,號哭徒行大雪中,遂成足疾。母嗜韭,終身不食韭。成化末,選應州知州,課績為天下第一。弘治九年遷金華同知。居二年,致仕,卒年七十四。所著有《道學基統》、《洙泗言學錄》、《爾雅便音》、《思庵埜錄》諸書。思庵者,敬之自號也。其門人呂柟最著,自有傳。

錦,字名中,咸寧人。舉天順六年鄉試。入國學,為祭酒邢讓所知。讓坐事下吏,錦率眾抗章白其非辜。幼喪父,事母色養,執喪盡禮,不作浮屠法。巡撫餘子俊欲延為子師,錦以齊衰不入公門,固辭。所居僅蔽風雨,布衣糲食,義不妄取。成化中選松江同知,卒官。

爵,字錫之,泰州人。弘治初,由國學生授保安州判官,有平允聲。其教門人也,務以誠敬為本。

胡居仁,字叔心,餘干人。聞吳與弼講學崇仁,往從之游,絕意仕進。其學以主忠信為先,以求放心為要,操而勿失,莫大乎敬,因以敬名其齋。端莊凝重,對妻子如嚴賓。手置一冊,詳書得失,用自程考。鶉衣簞食,晏如也。築室山中,四方來學者甚眾,皆告之曰:「學以為己,勿求人知。」語治世,則曰:「惟王道能使萬物各得其所。」所著有《居業錄》,蓋取修辭立誠之義。每言:「與吾道相似莫如禪學。後之學者,誤認存心多流於禪,或欲屏絕思慮以求靜。不知聖賢惟戒慎恐懼,自無邪思,不求靜未嘗不靜也。故卑者溺於功利,高者騖於空虛,其患有二:一在所見不真,一在功夫間斷。」嘗作《進學箴》曰:「誠敬既立,本心自存。力行既久,全體皆仁。舉而措之,家齊國治,聖人能事畢矣。」

居仁性行淳篤,居喪骨立,非杖不能起,三年不入寢門。與人語,終日不及利祿。與羅倫、張元禎友善,數會於弋陽龜峰。嘗言,陳獻章學近禪悟,莊昶詩止豪曠,此風既成,為害不細。又病儒者撰述繁蕪,謂朱子註《參同契》、《陰符經》,皆不作可也。督學李齡、鐘成相繼聘主白鹿書院。過饒城,淮王請講《易傳》,待以賓師之禮。是時吳與弼以學名於世,受知朝廷,然學者或有間言。居仁暗修自守,布衣終其身,人以為薛瑄之後,粹然一出於正,居仁一人而已。卒年五十一。萬歷十三年從祀孔廟,復追謚文敬。其弟子餘祐最著。

祐字子積,鄱陽人。年十九,師事居仁,居仁以女妻之。弘治十二年舉進士。為南京刑部員外郎,以事忤劉瑾,落職。瑾誅,起為福州知府。鎮守太監市物不予直,民群訴於祐。涕泣慰遣之,云將列狀上聞。鎮守懼,稍戢,然恚甚,遣人入京告其黨曰:「不去餘祐,鎮守不得自遂也。」然祐素廉,摭拾竟無所得。未幾,遷山東副使。父憂,服闋,補徐州兵備副使。中官王敬運進御物入都,多挾商船,與知州樊準、指揮王良詬。良發其違禁物,敬懼,詣祐求解,祐不聽。敬誣奏準等毆己,遂并逮祐,謫為南寧府同知。稍遷韶州知府,投劾去。嘉靖初,歷雲南布政使,以太僕寺卿召,未行,改吏部右侍郎,祐已先卒。祐之學,墨守師說,在獄中作《性書》三卷。其言程、朱教人,專以誠敬入。學者誠能去其不誠不敬者,不患不至古人。時王守仁作《朱子晚年定論》,謂其學終歸於存養。祐謂:「朱子論心學凡三變,存齋記所言,乃少時所見,及見延平,而悟其失。後聞五峰之學於南軒,而其言又一變。最後改定已發未發之論,然後體用不偏,動靜交致其力,此其終身定見也。安得執少年未定之見,而反謂之晚年哉?」其辨出,守仁之徒不能難也。

蔡清,字介夫,晉江人。少走侯官,從林玭學《易》,盡得其肯綮。舉成化十三年鄉試第一。二十年成進士,即乞假歸講學。已,謁選,得禮部祠祭主事。王恕長吏部,重清,調為稽勳主事,恒訪以時事。清乃上二札:一請振紀綱,一薦劉大夏等三十餘人。恕皆納用。尋以母憂歸,服闋,復除祠祭員外郎。乞便養,改南京文選郎中。一日心動,急乞假養父,歸甫兩月而父卒,自是家居授徒不出。正德改元,即家起江西提學副使。寧王宸濠驕恣,遇朔望,諸司先朝王,次日謁文廟。清不可,先廟而後王。王生辰,令諸司以朝服賀。清曰「非禮也」,去蔽膝而入,王積不悅。會王求復護衛,清有後言。王欲誣以詆毀詔旨,清遂乞休。王佯挽留,且許以女妻其子,競力辭去。劉瑾知天下議己,用蔡京召楊時故事,起清南京國子祭酒。命甫下而清已卒,時正德三年也,年五十六。

清之學,初主靜,後主虛,故以虛名齋。平生飭躬砥行,貧而樂施,為族黨依賴。以善《易》名。嘉靖八年,其子推官存遠以所著《易經》、《四書蒙引》進於朝,詔為刊布。萬曆中追謚文莊,贈禮部右侍郎。其門人陳琛、王宣、易時中、林同、趙逮、蔡烈並有名,而陳琛最著。

琛,字思獻,晉江人,杜門獨學。清見其文異之,曰:「吾得友此人足矣。」琛因介友人見清,清曰:「吾所發憤沉潛辛苦而僅得者,以語人常不解。子已盡得之,今且盡以付子矣。」清歿十年,琛舉進士,授刑部主事,改南京戶部,就擢考功主事,乞終養歸。嘉靖七年,有薦其恬退者,詔征之,琛辭。居一年,即家起貴州僉事,旋改江西,皆督學校,並辭不赴。家居,卻掃一室,偃臥其中,長吏莫得見其面。

同郡林希元,字懋貞,與琛同年進士。歷官雲南僉事,坐考察不謹罷歸。所著《存疑》等書,與琛所著《易經通典》、《四書淺說》,並為舉業所宗。

王宣,晉江人。弘治中舉於鄉,一赴會試不第,以親老須養,不再赴。嘗曰:「學者混朱、陸為一,便非真知。」為人廓落豪邁,俯視一世。

易時中,字嘉會,亦晉江人。舉於鄉,授東流教諭,遷夏津知縣,有惠政。稍遷順天府推官。以治胡守中獄失要人意,將中以他事,遂以終養歸。道出夏津,老稚爭獻果脯。將別,有哭失聲者。母年九十一而終,時中七十矣,毀不勝喪而卒。

趙逮,字子重,東平人。弘治中舉鄉試,受《易》於清。蔡氏《易》止行於閩南,及是北行齊、魯矣。居母喪毀瘠,後會試不第,遂抗志不出。生平好濂、洛諸子之學,於明獨好薛氏《讀書錄》。

蔡烈,字文繼,龍溪人。父昊,瓊州知府。烈弱冠為諸生,受知於清及莆田陳茂烈。隱居鶴鳴山之白雲洞,不復應試。嘉靖十二年詔舉遣佚,知府陸金以烈應,以母老辭。巡按李元陽檄郡邑建書院,亦固辭。忽山鳴三日,烈遂卒。主簿詹道嘗請論心,烈曰:「宜論事。孔門求仁,未嘗出事外也。堯、舜之道,孝弟而已。夫子之道,忠恕而已。」學士豐熙戍鎮海,見烈,歎曰:「先生不言躬行,熙已心醉矣。」

羅欽順,字允升,泰和人。弘治六年進士及第,授編修。遷南京國子監司業,與祭酒章懋以實行教士。未幾,奉親歸,因乞終養。劉瑾怒,奪職為民。瑾誅,復官,遷南京太常少卿,再遷南京吏部右侍郎,入為吏部左侍郎。世宗即位,命攝尚書事。上疏言久任、超遷,法當疏通,不報。大禮議起,欽順請慎大禮以全聖孝,不報。遷南京吏部尚書,省親乞歸。改禮部尚書,會居憂未及拜。再起禮部尚書,辭。又改吏部尚書,下詔敦促,再辭。許致仕,有司給祿米。時張總、桂萼以議禮驟貴,秉政樹黨,屏逐正人。欽順恥與同列,故屢詔不起。里居二十餘年,足不入城市,潛心格物致知之學。王守仁以心學立教,才知之士翕然師之。欽順致書守仁,略曰:「聖門設教,文行兼資,博學於文,厥有明訓。如謂學不資於外求,但當反觀內省,則『正心誠意』四字亦何所不盡,必於入門之際,加以格物工夫哉?」守仁得書,亦以書報,大略謂:「理無內外,性無內外,故學無內外。講習討論,未嘗非內也。反觀內省,未嘗遺外也。」反復二千餘言。欽順再以書辨曰:「執事云:『格物者,格其心之物也,格其意之物也,格其知之物也。正心者,正其物之心也。誠意者,誠其物之意也。致知者,致其物之知也。』自有《大學》以來,未有此論。夫謂格其心之物,格其意之物,格其知之物,凡為物也三。謂正其物之心,誠其物之意,致其物之知,其為物也,一而已矣。就三而論,以程子格物之訓推之,猶可通也。以執事格物之訓推之,不可通也。就一物而論,則所謂物,果何物耶?如必以為意之用,雖極安排之巧,終無可通之日也。又執事論學書有云:『吾心之良知,即所謂天理。致吾心良知之天理於事物,則事事物物皆得其理矣。致吾心之良知者,致知也。事事物物各得其理者,格物也。」審如所言,則《大學》當云『格物在致知』,不當云『致知在格物』,與『物格而后知至』矣。」書未及達,守仁已歿。

欽順為學,專力於窮理、存心、知性。初由釋氏入,既悟其非,乃力排之,謂:「釋氏之明心見性,與吾儒之盡心知性,相似而實不同。釋氏之學,大抵有見於心,無見於性。今人明心之說,混於禪學,而不知有千里毫釐之謬。道之不明,將由於此,欽順有憂焉。」為著《因知記》,自號整庵。年八十三卒,贈太子太保,謚文莊。

曹端,字正夫,澠池人。永樂六年舉人。五歲見《河圖》、《洛書》,即畫地以質之父。及長,專心性理。其學務躬行實踐,而以靜存為要。讀宋儒《太極圖》、《通書》、《西銘》,嘆曰:「道在是矣。」篤志研究,坐下著足處,兩磚皆穿。事父母至孝,父初好釋氏,端為《夜行燭》一書進之,謂:「佛氏以空為性,非天命之性。老氏以虛為道,非率性之道。」父欣然從之。繼遭二親喪,五味不入口。既葬,廬墓六年。

端初讀謝應芳《辨惑編》,篤好之,一切浮屠、巫覡、風水、時日之說屏不用。上書邑宰,毀淫祠百餘,為設里社、里穀壇,使民祈報。年荒勸振,存活甚眾。為霍州學正,修明聖學。諸生服從其教,郡人皆化之,恥爭訟。知府郭晟問為政,端曰:「其公廉乎。公則民不敢謾,廉則吏不敢欺。」晟拜受。遭艱歸,澠池、霍諸生多就墓次受學。服闋,改蒲州學正。霍、蒲兩邑各上章爭之,霍奏先得請。先後在霍十六載,宣德九年卒官,年五十九。諸生服心喪三年,霍人罷市巷哭,童子皆流涕。貧不能歸葬,遂留葬霍。二子瑜、琛,亦戶端墓,相繼死,葬暮側,後改葬澠池。

端嘗言:「學欲至乎聖人之道,須從太極上立根腳。」又曰:「為人須從志士勇士不忘上參取。」又曰:「孔、顏之樂仁也,孔子安仁而樂在其中,顏淵不違仁而不改其樂,程子令人自得之。」又曰:「天下無性外之物,而性無不在焉。性即理也,理之別名曰太極,曰至誠,曰至善,曰大德,曰大中,名不同而道則一。」初,伊、洛諸儒,自明道、伊川後,劉絢、李輩身及二程之門,至河南許衡、洛陽姚樞講道蘇門,北方之學者翕然宗之。洎明興三十餘載,而端起崤、澠間,倡明絕學,論者推為明初理學之冠。所著有《孝經述解》、《四書詳說》、《周易乾坤二卦解義》、《太極圖說通書》《西銘》釋文、《性理文集》、《儒學宗統譜》、《存疑錄》諸書。

霍州李德與端同時,亦講學於其鄉。及見端,退語諸生曰:「學不厭,教不倦,曹子之盛德也。至其知古今,達事變,末學鮮或及之。古云『得經師易,得人師難』,諸生得人師矣。」遂避席去。端亦高其行誼,命諸生延致之,講明正學。初,端作《川月交映圖》擬太極,學者稱月川先生。及歿,私謚靜修。正德中,尚書彭澤、河南巡撫李楨請從祀孔子廟庭,不果。

吳與弼,字子傳,崇仁人。父溥,建文時為國子司業,永樂中為翰林修撰。與弼年十九,見《伊洛淵源圖》,慨然響慕,遂罷舉子業,盡讀《四子》、《五經》、洛閩諸錄,不下樓者數年。中歲家益貧,躬親耕稼,非其義,一介不取。四方來學者,約己分少,飲食、教誨不倦。正統十一年,山西僉事何自學薦於朝,請授以文學高職。後御史塗謙、撫州知府王宇復薦之,俱不出。嘗歎曰:「宦官、釋氏不除,而欲天下治平,難矣。」景泰七年,御史陳述又請禮聘與弼,俾侍經筵,或用之成均,教育胄子。詔江西巡撫韓雍備禮敦遣,竟不至。天順元年,石亨欲引賢者為己重,謀於大學士李賢,屬草疏薦之。帝乃命賢草敕加束帛,遣行人曹隆,賜璽書,齎禮幣,徵與弼赴闕。比至,帝問賢曰:「與弼宜何官?」對曰:「宜以宮僚,侍太子講學。」遂授左春坊左諭德,與弼疏辭。賢請賜召問,且與館次供具。於是召見文華殿,顧語曰:「聞處士義高,特行徵聘,奚辭職為?」對曰:「臣草茅賤士,本無高行,陛下垂聽虛聲,又不幸有狗馬疾。束帛造門,臣慚被異數,匍匐京師,今年且六十八矣,實不能官也。」帝曰:「宮僚優閑,不必辭。」賜文綺酒牢,遣中使送館次。顧謂賢曰:「此老非迂闊者,務令就職。」時帝眷遇良厚,而與弼辭益力。又疏稱:「學術荒陋,茍冒昧徇祿,必且曠官。」詔不許。乃請以白衣就邸舍,假讀秘閣書。帝曰:「欲觀秘書,勉受職耳。」命賢為諭意。與弼留京師二月,以疾篤請。賢請曲從放還,始終恩禮,以光曠舉。帝然之,賜敕慰勞,齎銀幣,復遣行人送還,命有司月給米二石。與弼歸,上表謝,陳崇聖志、廣聖學等十事。成化五年卒,年七十九。

與弼始至京,賢推之上座,以賓師禮事之。編修尹直至,令坐於側。直大慍,出即謗與弼。及與弼歸,知府張璝謁見不得,大恚。募人代其弟投牒訟與弼,立遣吏攝之,大加侮慢,始遣還。與弼諒非弟意,友愛如初。編修張元楨不知其始末,遣書誚讓,有「上告素王,正名討罪,豈容先生久竊虛名」語。直後筆其事於《瑣綴錄》。又言與弼跋亨族譜,自稱門下士,士大夫用此訾與弼。後顧允成論之曰:「此好事者為之也。」與弼門人後皆從祀,而與弼竟不果。所著《日錄》,悉自言生平所得。

其門人最著者曰胡居仁、陳獻章、婁諒,次曰胡九韶、謝復、鄭伉。胡九韶,字鳳儀,少從與弼學。諸生來學者,與弼令先見九韶。及與弼歿,門人多轉師之。家貧,課子力耕,僅給衣食。成化中卒。謝復,字一陽,祁門人。聞與弼倡道,棄科舉業從之游。身體力行,務求自得。居家孝友,喪祭冠婚,悉遵古禮。或問學,曰:「知行並進,否則落記誦詁訓矣。」晚卜室西山之麓,學者稱西山先生。弘治末年卒,年六十五。鄭伉,字孔明,常山人。為諸生,試有司,不偶,即棄去,師與弼。辭歸,日究諸儒論議,一切折衷於朱子。事親孝。設義學,立社倉,以惠族黨。所著《易義發明》、《讀史管見》、《觀物餘論》、《蛙鳴集》,多燼於火。

陳真晟,字晦德,漳州鎮海衛人。初治舉赴鄉試,聞有司防察過嚴,無待士禮,恥之棄去,由是篤志聖賢之學。讀《大學或問》,見朱子重言主敬,知「敬」為《大學》始基。又得程子主一之說,專心克治,嘆曰:「《大學》,誠意為鐵門關,主一二字,乃其玉鑰匙也。」天順二年詣闕上《程朱正學纂要》。其書首取程氏學制,次采朱子論說,次作二圖,一著聖人心與天地同運,一著學者之心法天之運,終言立明師、輔皇儲、隆教本數事,以畢圖說之意。書奏,下禮部議,侍郎鄒乾寢其事。真晟歸,聞臨川吳與弼方講學,欲就問之。過南昌,張元禎止之宿,與語,大推服曰:「斯道自程、朱以來,惟先生得其真。如康齋者,不可見,亦不必見也。」遂歸閩,潛思靜坐,自號漳南布衣。卒於成化十年,年六十四。真晟學無師承,獨得於遺經之中。自以僻處海濱,出而訪求當世學者,雖未與與弼相證,要其學頗似近之。

呂柟,字仲木,高陵人,別號涇野,學者稱涇野先生。正德三年登進士第一,授修撰。劉瑾以柟同鄉欲致之,謝不往。又因西夏事,疏請帝入宮親政事,潛消禍本。瑾惡其直,欲殺之,引疾去。瑾誅,以薦復官。乾清宮災,應詔陳六事,其言除義子,遣番僧,取回鎮守太監,尤人所不敢言。是年秋,以父病歸。都御史盛應期,御史朱節、熊相、曹珪累疏薦。適世宗嗣位,首召柟。上疏勸勤學以為新政之助,略曰:「克己慎獨,上對天心;親賢遠讒,下通民志,庶太平之業可致。」大禮議興,與張、桂忤。以十三事自陳,中以大禮未定,諂言日進,引為己罪。上怒,下詔獄,謫解州判官,攝行州事。恤煢獨,減丁役,勸農桑,興水利,築隄護鹽池,行《呂氏鄉約》及《文公家禮》,求子夏後,建司馬溫公祠。四方學者日至,御史為闢解梁書院以居之。三年,御史盧煥等累薦,升南京宗人府經歷,歷官尚寶司卿。吳、楚、閩、越士從者百餘人。晉南京太僕寺少卿。太廟災,乞罷黜,不允。選國子監祭酒,晉南京禮部右侍郎,署吏部事。帝將躬祀顯陵,累疏勸止,不報。值天變,遂乞致仕歸。年六十四卒,高陵人為罷市者三日。解梁及四方學者聞之,皆設位,持心喪。訃聞,上輟朝一日,賜祭葬。

柟受業渭南薛敬之,接河東薛瑄之傳,學以窮理實踐為主。官南都,與湛若水、鄒守益共主講席。仕三十餘年,家無長物,終身未嘗有惰容。時天下言學者,不歸王守仁,則歸湛若水,獨守程、硃不變者,惟柟與羅欽順云。所著有《四書因問》、《易說翼》、《書說要》、《詩說序》、《春秋說志》、《禮問內外篇》、《史約》、《小學釋》、《寒暑經圖解》、《史館獻納》、《宋四子抄釋》、《南省奏槁》、《涇野詩文集》。萬曆、崇禎間,李禎、趙錦、周子義、王士性、蔣德璟先後請從祀孔廟,下部議,未及行。柟弟子涇陽呂潛,字時見,舉於鄉。官工部司務。張節,字介夫。咸寧李挺,字正五。皆有學行。

潛里人郭郛,字維籓,由舉人官馬湖知府。藍田王之士,字慾立。由舉人以趙用賢薦,授國子博士。兩人不及柟門,亦秦士之篤學者也。

邵寶,字國賢,無錫人。年十九,學於江浦莊昶。成化二十年舉進士,授許州知州。月朔,會諸生於學宮,講明義利公私之辨。正潁考叔祠墓。改魏文帝廟以祠漢愍帝,不稱獻而稱愍,從昭烈所謚也。巫言龍骨出地中為禍福,寶取骨,毀於庭,杖巫而遣之。躬課農桑,仿朱子社倉,立積散法,行計口澆田法,以備凶荒。

弘治七年入為戶部員外郎,歷郎中,遷江西提學副使。釋菜周元公祠。修白鹿書院學舍,處學者。其教,以致知力行為本。江西俗好陰陽家言,有數十年不葬父母者。寶下令,士不葬親者不得與試,於是相率舉葬,以千計。寧王宸濠索詩文,峻卻之。後宸濠敗,有司校勘,獨無寶跡。遷浙江按察使,再遷右布政使。與鎮守太監勘處州銀礦,寶曰:「費多獲少,勞民傷財,慮生他變。」卒奏寢其事。進湖廣布政使。

正德四年擢右副都御史,總督漕運。劉瑾擅政,寶至京,絕不與通。瑾怒漕帥平江伯陳熊,欲使寶劾之,遣校尉數輩要寶左順門,危言恐之曰:「行逮汝。」張綵、曹元自內出,語寶曰:「郡第劾平江,無後患矣。」寶曰:「平江功臣後,督漕未久,無大過,不知所劾。」二人默然出。越三日,給事中劾熊併及寶,勒致仕去。瑾誅,起巡撫貴州,尋遷戶部右侍郎,進左侍郎。命兼左僉都御史,處置糧運。及會勘通州城濠歸,奏稱旨。尋疏請終養歸,御史唐鳳儀、葉忠請用之留都便養,乃拜南京禮部尚書,再疏辭免。世宗即位,起前官,復以母老懇辭。許之,命有司以禮存問。久之卒,贈太子太保,謚文莊。

寶三歲而孤,事母過氏至孝。甫十歲,母疾,為文告天,願減己算延母年。及終養歸,得疾,左手不仁,猶朝夕侍親側不懈。學以洛、閩為的,嘗曰:「吾願為真士大夫,不願為假道學。」舉南畿,受知於李東陽。為詩文,典重和雅,以東陽為宗。至於原本經術,粹然一出於正,則其所自得也。博綜群籍,有得則書之簡,取程子「今日格一物,明日格一物」之義,名之曰日格子。所著《學史》、簡端二錄,巡撫吳廷舉上於朝,外《定性書說》、《漕政舉要》諸集若干卷。學者稱二泉先生。

其門人,同邑王問,字子裕,以學行稱。嘉靖十七年成進士。授戶部主事,監徐州倉,減羨耗十二三。以父老,乞便養,改南京職方,遷車駕郎中、廣東僉事。行未半道,乞養歸。父卒,遂不復仕。築室湖上,讀書三十年,不履城市,數被薦不起。工詩文書畫,清修雅尚,士大夫皆慕之。卒年八十,門人私謚曰文靜先生。

子鑑,字汝明。嘉靖末年進士。累官吏部稽勛郎中。念父老,謝病歸,奉養不離側。父歿久之,進尚寶卿,改南京鴻臚卿,引年乞休。進太僕卿,致仕。鑒亦善畫,有言勝其父者,遂終身不復作。

楊廉,字方震,豐城人。父崇,永州知府,受業吳與弼門人胡九韶。廉承家學,早以文行稱。舉成化末年進士,改庶吉士。弘治三年,授南京戶科給事中。明年,京師地震,劾用事大臣。五年以災異上六事。一,經筵停罷時,宜日令講官更直待問。二,召用言事遷謫官,不當限臺諫及登極以後。三,治兩浙、三吳水患,停額外織造。四,召林下恬退諸臣。五,刪法司條例。六,災異策免大臣。末言,遇大政,宜召大臣面議,給事、御史隨入駁正。帝頗納之。吏部尚書王恕被讒,廉請斥讒邪,無為所惑。母喪,服闋,起任刑科。請祀薛瑄,取《讀書錄》貯國學。明年三月有詔以下旬御經筵。廉言:「故事,經筵一月三舉,茍以月終起以月初罷,則進講有幾?且經筵啟而後日講繼之,今遲一日之經筵,即輟一旬之日講也。」報聞。以父老欲便養,復改南京兵科。中貴李廣死,得廷臣通賄籍。言官劾賄者,帝欲究而中止。廉率同官力爭,竟不納。已,請申明祀典,謂宋儒周、程、張、朱從祀之位,宜居漢、唐諸儒上。闕里廟,當更立木主。大成本樂名,不合謚法。皆不果行。遷南京光祿少卿。正德初,就改太僕,歷順天府尹。時京軍數出,車費動數千金,廉請大興遞運所餘銀供之。奏免夏稅萬五千石,慮州縣巧取民財,置歲辦簿,吏無能為奸。乾清宮災,極陳時政缺失,疏留中。明年擢南京禮部右侍郎。上疏諫南巡,不報。帝駐南京,命百官戎服朝見。廉不可,乞用常儀,更請謁見太廟,俱報許。世宗即位,就遷尚書。

廉與羅欽順善,為居敬窮理之學,文必根《六經》,自禮樂、錢穀至星歷、算數,具識其本末。學者稱月湖先生。嘗以帝王之道莫切於《大學》,自為給事即上言,進講宜先《大學衍義》,至是首進《大學衍義節略》。帝優詔答之。疏論大禮,引程頤、朱熹言為證,且言:「今異議者率祖歐陽修。然修於考之一字,雖欲加之於濮王,未忍絕之於仁宗。今乃欲絕之於孝廟,此又修所不忍言者。」報聞。八疏乞休,至嘉靖二年,賜敕、馳驛,給夫廩如制。家居二年卒,年七十四。贈太子少保,謚文恪。

劉觀,字崇觀,吉水人。正統四年成進士。方年少,忽引疾告歸。尋丁內艱,服除,終不出。杜門讀書,求聖賢之學。四方來問道者,坐席嘗不給。縣令劉成為築書院於虎丘山,名曰「養中」。平居,飯脫粟,服浣衣,翛然自得。每日端坐一室,無懈容。或勸之仕,不應。又作《勤》、《儉》、《恭》、《恕》四《箴》,以教其家,取《呂氏鄉約》表著之,以教其鄉。冠婚喪祭,悉如《朱子家禮》。族有孤嫠不能自存者周之。或請著述,曰:「朱子及吳文正之言,尊信之足矣。復何言。」吳與弼,其鄰郡人也,極推重之。

觀前有孫鼎,廬陵人。永樂中為松江府教授,以孝弟立教。後督學南畿,人稱為貞孝先生。又有李中,吉水人,官副都御史,號谷平先生,在觀後。是為吉水三先生。

馬理,字伯循,三原人。同里尚書王恕家居,講學著書。理從之遊,得其指授。楊一清督學政,見理與呂柟、康海文,大奇之,曰:「康生之文章,馬生、呂生之經學,皆天下士也。」登鄉薦,入國學,與柟及林慮馬鄉,榆次寇天敘,安陽崔銑、張士隆,同縣秦偉,日切劘於學,名震都下。高麗使者慕之,錄其文以去。連遭艱,不預試。安南使者至,問主事黃清曰:「關中馬理先生安在,何不仕也?」其為外裔所重如此。

正德九年舉進士。一清為吏部尚書,即擢理稽勳主事。調文選,請告歸。起考功主事,偕郎中張衍瑞等諫南巡。詔跪闕門,予杖奪俸。未幾,復告歸。教授生徒,從游者眾。嘉靖初,起稽勛員外郎,與郎中餘寬等伏闕爭大禮。下詔獄,再予杖奪俸。屢遷考功郎中。故戶部郎中莊繹者,正德時首導劉瑾核天下庫藏。瑾敗,落職。至是奏辨求復,當路者屬理,理力持不可,寢其事。五年大計外吏,大學士賈詠、吏部尚書廖幻以私憾欲去廣東副使魏校、河南副使蕭鳴鳳、陜西副使唐龍。理力爭曰:「三人督學政,名著天下,必欲去三人,請先去理。」乃止。明年大計京官,黜張總、桂萼黨吏部郎中彭澤,總、萼竟取旨留之。理擢南京通政參議,請急去。居三年,起光祿卿,未幾告歸。閱十年,復起南京光祿卿,尋引年致仕。三十四年,陜西地震,理與妻皆死。

理學行純篤,居喪取古禮及司馬光《書儀》、朱熹《家禮》折衷用之,與呂柟並為關中學者所宗。穆宗立,贈右副都御史。天啟初,追謚忠憲。

魏校,字子才,崑山人。其先本李姓,居蘇州葑門之莊渠,因自號「莊渠」。弘治十八年成進士。歷南京刑部郎中。守備太監劉郎藉劉瑾勢張甚,或自判狀送法司,莫敢抗者。校直行己意,無所徇。改兵部郎中,移疾歸。嘉靖初,起為廣東提學副使。丁憂,服闋,補江西兵備副使。累遷國子祭酒,太常卿,尋致仕。

校私淑胡居仁主敬之學,而貫通諸儒之說,擇執尤精。嘗與餘祐論性,略曰:「天地者,陰陽五行之本體也,故理無不具。人物之性,皆出於天地,然而人得其全,物得其偏。」又曰:「古聖賢論性有二:其一,性與情對言,此是性之本義,直指此理而言。其一,性與習對言,但取生字為義,非性之所以得名,蓋曰天所生為性,人所為曰習耳。先儒因『性相近』一語,遂謂性兼氣質而言,不知人性上下不可添一物,纔著氣質,便不得謂之性矣。荀子論性惡,楊子論性善惡混,韓子論性有三品,眾言淆亂,必折之聖。若謂夫子『性相近』一言,正論性之所以得名,則前後說皆不謬於聖人,而孟子道性善,反為一偏之論矣。孟子見之分明,故言之直捷,但未言性為何物,故荀、楊、韓諸儒得以其說亂之。伊川一言以斷之,曰『性,即理也』,則諸說皆不攻自破矣。」所著有《大學指歸》、《六書精蘊》。卒,謚恭簡。唐順之、王應電、王敬臣,皆其弟子也。順之,自有傳。

王應電,字昭明,崑山人。受業於校,篤好《周禮》,謂《周禮》自宋以後,胡宏、季本各著書,指摘其瑕釁至數十萬言。而餘壽翁、吳澄則以為《冬官》未嘗亡,雜見於五官中,而更次之。近世何喬新、陳鳳梧、舒芬亦各以己意更定。然此皆諸儒之《周禮》也。覃研十數載,先求聖人之心,溯斯禮之源;次考天象之文,原設官之意,推五官離合之故,見綱維統體之極。因顯以探微,因細而繹大,成《周禮傳詁》數十卷。以為百世繼周而治,必出於此。嘉靖中,家毀於兵燹,流寓江西泰和。以其書就正羅洪先,洪先大服。翰林陳昌積以師禮事之。胡松撫江西,刊行於世。

應電又研精字學,據《說文》所載為訛謬甚者,為之訂正,名曰《經傳正訛》。又著《同文備考》、《書法指要》、《六義音切貫珠圖》、《六義相關圖》。卒於泰和。昌積為經紀其喪,歸之崑山。

時有李如玉者,同安儒生,亦精於《周禮》,為《會要》十五卷。嘉靖八年詣闕上之,得旨嘉獎,賜冠帶。

王敬臣,字以道,長洲人,江西參議庭子也。十九為諸生,受業於校。性至孝,父疽發背,親自吮舐。老得瞀眩疾,則臥於榻下,夜不解衣,微聞響咳聲,即躍起問安。事繼母如事父,妻失母歡,不入室者十三載。初,受校默成之旨,嘗言議論不如著述,著述不如躬行,故居常杜口不談。自見耿定向,語以聖賢無獨成之學,由是多所誘掖,弟子從游者至四百餘人。其學,以慎獨為先,而指親長之際、衣任席之間為慎獨之本,尤以標立門戶為戒。鄉人尊為少湖先生。萬曆中,以廷臣薦,徵授國子博士,辭不行。詔以所授官致仕。二十一年,巡按御史甘士價復薦。吏部以敬臣年高,請有司時加優禮,詔可。年八十五而終。

周瑛,字梁石,莆田人。成化五年進士。知廣德州,以善政聞,賜敕旌異。遷南京禮部郎中,出為撫州知府,調知鎮遠。秩滿,省親歸。弘治初,吏部尚書王恕起瑛四川參政,久之,進右布政使,咸有善績,尤勵清節。給事、御史交章薦,大臣亦多知瑛,而瑛以母喪歸。服除,遂引年乞致仁。孝宗嘉之,詔進一階。正德中卒,年八十七。瑛始與陳獻章友,獻章之學主於靜。瑛不然之,謂學當以居敬為主,敬則心存,然後可以窮理。自《六經》之奧,以及天地萬物之廣,皆不可不窮。積累既多,則能通貫,而於道之一本,亦自得之矣,所謂求諸萬殊而後一本可得也。學者稱翠渠先生。子大謨,登進士,未仕卒。

潘府,字孔修,上虞人。成化末進士。值憲宗崩,孝宗踐阼甫二十日,禮官請衰服御西角門視事,明日釋衰易素,翼善冠、麻衣腰絰。帝不許,命俟二十七日後行之。至百日,帝以大行未葬,麻衣衰絰如故。府因上疏請行三年喪,略言:「子為父,臣為君,皆斬衰三年,仁之至,義之盡也。漢文帝遺詔短喪,止欲便天下臣民,景帝遂自行之,使千古綱常一墜不振。晉武帝欲行而不能,魏孝文行之而不盡,宋孝宗銳志復古,易月之外,猶執通喪,然不能推之於下,未足為聖王達孝也。先帝奄棄四海,臣庶銜哀,陛下惻恆由衷,麻衣視朝,百日未改。望排群議,斷自聖心,執喪三年一如三代舊制。詔禮官參考載籍,使喪不廢禮,朝不廢政,勒為彞典,傳之子孫,豈不偉哉。」疏入,衰絰待罪。詔輔臣會禮官詳議,並持成制,寢不行。

謁選,得長樂知縣,教民行《朱子家禮》。躬行郊野,勞問疾苦,田夫野老咸謂府親己,就求筆札,府輒欣然與之。遷南京兵部主事,陳軍民利病七事。父喪除,補刑部。值旱蝗、星變,北寇深入,孔廟災,疏請內修外攘,以謹天戒。又上救時十要。以便養乞南,改南京兵部,遷武選員外郎。尚書馬文升知其賢,超拜廣東提學副使。雲南晝晦七日,楚婦人鬚長三寸,上弭災三術。以母老乞休,不待命輒歸。已而吏部尚書楊一清及巡按御史吳華屢薦其學行,終不起。嘉靖改元,言官交薦,起太僕少卿,改太常,致仕。既歸,屏居南山,布衣蔬食,惟以發明經傳為事。時王守仁講學其鄉,相去不百里,頗有異同。嘗曰:「居官之本有三:薄奉養,廉之本也;遠聲色,勤之本也;去讒私,明之本也。」又曰:「薦賢當惟恐後,論功當惟恐先。」年七十三卒。故事,四品止予祭。世宗重府孝行,特詔予葬。

崔銑,字子鐘,安陽人。父升,官參政。銑舉弘治十八年進士,選庶吉士,授編修。預修《孝宗實錄》,與同官見太監劉瑾,獨長揖不拜,由是忤瑾。書成,出為南京吏部主事。瑾敗,召復故官,充經筵講官,進侍讀。引疾歸,作後渠書屋,讀書講學其中。世宗即位,擢南京國子監祭酒。嘉靖三年集議大禮,久不決。大學士蔣冕、尚書汪俊俱以執議去位,其他擯斥杖戍者相望,而張總、桂萼等驟貴顯用事。銑上疏求去,且劾總、萼等曰:「臣究觀議者,其文則歐陽修之唾餘,其情則承望意響,求勝無已。悍者危法以激怒,柔者甘言以動聽。非有元功碩德,而遽以官賞之,得毋使僥倖之徒踵接至與?臣聞天子得四海歡心以事其親,未聞僅得一二人之心者也。賞之,適自章其私暱而已。夫守道為忠,忠則逆旨;希旨為邪,邪則畔道。今忠者日疏,而邪者日富。一邪亂邦,況可使富哉!」帝覽之不悅,令銑致仕。閱十五年,用薦起少詹事兼侍讀學士,擢南京禮部右侍郎。未幾疾作,復致仕。卒,贈禮部尚書,謚文敏。

銑少輕俊,好飲酒,盡數斗不亂。中歲自厲於學,言動皆有則。嘗曰:「學在治心,功在慎動。」又曰:「孟子所謂良知良能者,心之用也。愛親敬長,性之本也。若去良能,而獨挈良知,是霸儒也。」又嘗作《政議》十篇,其《序》曰:「三代而上,並田封建,其民固,故道易行,三代而下,阡陌郡縣,其民散,故道難成。況沿而下趨至今日乎。然人心弗異,係乎主之者而已。」凡篇中所論說,悉仿此意。世多有其書,故不載。

何瑭,字粹夫,武陟人。年七歲,見家有佛像,抗言請去之。十九讀許衡、薛瑄遺書,輒欣然忘寢食。弘治十五年成進士,選庶吉士。閣試《克己復禮為仁論》,有曰:「仁者,人也。禮則人之元氣而已,則見侵於風寒暑濕者也。人能無為邪氣所勝,則元所復,元年復而其人成矣。」宿學咸推服焉。劉瑾竊政,一日贈翰林川扇,有入而拜見者。瑭時官修撰,獨長揖。瑾怒,不以贈。受贈者復拜謝,瑭正色曰:「何僕僕也!」瑾大怒,詰其姓名。瑭直應曰:「修撰何瑭。」知必不為瑾所容,乃累疏致仕。後瑾誅,復官。以經筵觸忌諱,謫開州同知。修黃陵岡隄成,擢東昌府同知,乞歸。嘉靖初,起山西提學副使,以父憂不赴。服闋,起提學浙江。敦本尚實,士氣丕變。未幾,晉南京太常少卿。與湛若水等修明古太學之法,學者翕然宗之。歷工、戶、禮三部侍郎,晉南京右都御史,未幾致仕。

是時,王守仁以道學名於時,瑭獨默如。嘗言陸九淵、楊簡之學,流入禪宗,充塞仁義。後學未得游、夏十一,而議論即過顏、曾,此吾道大害也。里居十餘年,教子姓以孝弟忠信,一介必嚴。兩執親喪,皆哀毀。後謚文定。所著《陰陽律呂》、《儒學管見》、《柏齋集》十二卷,皆行於世。

唐伯元,字仁卿,澄海人。萬歷二年進士。歷知萬年、泰和二縣,並有惠政,民生祠之。遷南京戶部主事,進郎中。伯元受業永豐呂懷,踐履篤實,而深疾王守仁新說。及守仁從祀文廟,上疏爭之。因請黜陸九淵,而躋有若及周、程、張、朱五子於十哲之列,祀羅欽順、章懋、呂柟、魏校、呂懷、蔡清、羅洪先、王艮於鄉。疏方下部,旋為南京給事中鐘宇淳所駁,伯元謫海州判官。屢遷尚寶司丞。吏部尚書楊巍雅不喜守仁學,心善伯元前疏,用為吏部員外郎。歷考功、文選郎中,佐尚書孫丕揚澄清吏治,苞苴不及其門。秩滿,推太常少卿,未得命。時吏部推補諸疏皆留中,伯元言:「賢愚同滯,朝野咨嗟,由臣擬議不當所致,乞賜罷斥。」帝不懌,特允其去,而諸疏仍留不下。居二年,甄別吏部諸郎,帝識伯元名,命改南京他部,而伯元已前卒。伯元清苦淡薄,人所不堪,甘之自如,為嶺海士大夫儀表。

黃淳耀,字蘊生,嘉定人。為諸生時,深疾科舉文浮靡淫麗,乃原本《六經》,一出以典雅。名士爭務聲利,獨澹漠自甘,不事征逐。崇禎十六年成進士。歸益研經籍,縕袍糲食,蕭然一室。京師陷,福王立南都,諸進士悉授官,淳耀獨不赴選。及南都亡,嘉定亦破。愾然太息,偕弟淵耀入僧舍,將自盡。僧曰:「公未服官,可無死。」淳耀曰:「城亡與亡,豈以出處貳心。」乃索筆書曰:「弘光元年七月二十四日,進士黃淳耀自裁於城西僧舍。鳴呼!進不能宣力王朝,退不能潔身自隱,讀書寡益,學道無成,耿耿不寐,此心而已。」遂與淵耀相對縊死,年四十有一。

淳耀弱冠即著《自監錄》、《知過錄》,有志聖賢之學。後為日歷,晝之所為,夜必書之。凡語言得失,念慮純雜,無不備識,用自省改。晚而充養和粹,造詣益深。所作詩古文,悉軌先正,卓然名家。有《陶庵集》十五卷。其門人私謚之曰貞文。淵耀,字偉恭,諸生,好學敦行如其兄。


\end{pinyinscope}