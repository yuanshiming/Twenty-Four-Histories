\article{列傳第一百五十五}

\begin{pinyinscope}
馬從聘耿蔭樓張伯鯨宋玫族叔應亨陳顯際趙士驥等范淑泰)高名衡王漢徐汧楊廷樞鹿善繼薛一鶚

馬從聘,字起莘,靈壽人。萬歷十七年進士。授青州推官,擢御史。勛衛李宗城冊封平秀吉逃歸,從聘言其父言恭不當復督戎政,不從。出理兩淮鹽課,言近日泰山崩離,坼者里餘,由開礦斷地脈所致,當速罷,不報。奸人田應璧請掣賣沒官餘鹽助大工,帝遣中官魯保督之。從聘極陳欺罔狀,不從。還朝,改按浙江,又按蘇、松,請免增蘇、松、常鎮稅課,亦不報。以久次擢太僕少卿,拜右僉都御史,巡撫延綏,失事奪俸。既而有搗巢功,未敘,引疾歸。加兵部右侍郎。家居凡二十餘年,終熹宗世不出。

崇禎十一年冬,大清破靈壽。從聘年八十有二矣,謂其三子曰:「吾得死所矣。」又曰:「吾大臣,義不可生,汝曹生無害也。」三子不從。從聘縊,三子皆縊。贈兵部尚書,謚介敏,官其一子。

耿蔭樓,從聘同邑人也,字旋極。天啟中,任臨淄知縣。久旱,囚服暴烈日中,哭於壇,雨立澍。攝壽光,禱雨如臨淄。崇禎中,入為兵部主事,調吏部,歷員外郎,乞假歸。城破,偕子參並死之。贈光祿少卿。

張伯鯨,字繩海,江都人。萬曆四十四年進士。歷知會稽、歸安、鄞三縣。天啟中,大計,調補盧氏。

崇禎二年,稍遷戶部主事,出督延、寧二鎮軍儲。自黃甫川西抵寧夏千二百里,不產五穀,芻粟資內地。賀蘭山沿黃河漢、唐二渠,東抵花馬池,素沃野,亦荒蕪甚。伯鯨疏陳其狀,為通商惠工,轉菽麥。又仿邊商中鹽意,立官市法以招之,軍民稱便。大盜起延綏,擢伯鯨兵備僉事,轄榆林中路。擊破賀思賢,斬一座城、金翅鵬,敗套寇於長樂堡。巡撫陳奇瑜上其功,詔進三階,為右參政,仍視兵備事。

七年春,奇瑜遷總督,遂擢伯鯨右僉都御史代之。督總兵王承恩等分道擊破插漢部長及套寇於雙山、魚河二堡,斬首三百。明年,以拾遺論罷。尋論延綏功,詔起用,蔭子錦衣千戶。

十年秋,楊嗣昌議大舉討賊,遣戶部一侍郎駐池州,專理兵食。帝命傅淑訓。明年,淑訓憂去,即家起伯鯨代之,如淑訓官。又明年,熊文燦撫事敗,嗣昌自出督師,移伯鯨襄陽。文燦之被逮也,言剿餉不至者六十餘萬,伯鯨坐貶秩。

十五年,召為兵部左侍郎。明年,尚書馮元飆在告,伯鯨攝部事。召對萬歲山,疾作,中官扶出,遂乞休。又明年,京城陷,微服遁還。福王立於南京,伯鯨家居不出。久之,揚州被圍,與當事分城守。城破,自經死。

宋玫,字文玉,萊陽人。父繼登,萬歷三十二年進士。歷官陜西右參議。天啟五年大計謫官。玫即以是年偕族叔應亨同舉進士,玫授虞城知縣,應亨得清豐。

崇禎元年,玫兄琮亦舉進士,知祥符,而玫以才調繁杞縣,三人壤地相接,並有治聲。應亨遷禮部主事,玫亦擢吏科給事中。嘗疏論用人,謂:「陛下求治之心愈急,則浮薄喜事之人皆飾詭而釣奇;陛下破格之意愈殷,則巧言孔壬之徒皆乘機而鬥捷。」眾韙其言。時應亨已改吏部,累遷稽勳郎中,落職歸。玫方除母喪,起故官,歷刑科都給事中。請熱審概行於天下。又言獄囚稽滯瘐死,與刑死幾相半,宜有矜釋。帝採納之。遷太常少卿,歷大理卿、工部右侍郎。玫父繼登已久廢,至是為浙江右參政。大學士周延儒客盛順者,為浙江巡撫熊奮渭營內召,果擢南京戶部侍郎,繼登父子信之。

十五年夏,廷推閣臣,順為玫營推甚力。會詔令再推,玫與焉。帝已中流言,疑諸臣有私。比入對,玫冀得帝意,侃侃敷奏。帝發怒,叱退之,與吏部尚書李日宣等並下獄。日宣等遣戍,玫除名,順乃驚竄。

閏十一月,臨清破,應亨與知縣陳顯際謀城守。應亨以城北庳薄,出千金建甕城,浹旬而畢。玫及邑人趙士驥亦出貲治守具。無何,大清兵薄城,城上火炮矢石並發,圍乃解。明年二月復至,城遂破,玫、應亨、顯際、士驥並死之。顯際,真定人,士驥官中書舍人,並起家進士。玫、應亨有文名。

沈迅,亦萊陽人也。崇禎四年舉進士,歷知新城、蠡二縣,與膠州張若騏同年友善。十一年行取入都。帝以吏部考選行私,親策諸臣,迅、若騏並得刑部主事。兩人大恚恨,結楊嗣昌,得改兵部。其年冬,畿輔被兵,迅請於廣平、河間、定州、蠡縣各設兵備一人。又請以天下僧人配尼姑,編入里甲,三丁抽一,可得兵數十萬。他條奏甚多。章下兵部,嗣昌盛稱迅言可用,乃命為兵科給事中。

迅欲自結於帝,數言事,皆中旨。當是時,軍興方棘,廷臣言兵者即以為知兵,大者推督撫,小者兵備,一當事任,罪累立至。於是上下諱言兵,章奏無敢及者。迅極言其弊,乞敕廷臣五日內陳方略。帝即從其言。迅考選時為掌河南道御史王萬象所抑,因事劾罷萬象,勢益張,與若騏盡把持山東事。會順天府丞戴澳誣劾平遠知縣王凝命、嘉興推官文德翼貪,迅上疏頌二人廉能,澳下吏削籍。迅累遷禮科都給事中。陳新甲主款,迅面斥其非,廷辨良久,又言:「楊嗣昌死有餘戮,借久案以邀功,陳新甲負罪不遑,移邊勞而錄廕,非論功議罪法。」帝是其言。迅本由嗣昌進,隨眾詆毀,時論訾薄之。

尋以保舉高斗光為鳳陽總督不當,謫國子博士,乞假歸。及新甲誅,命追論兵科不糾發罪,吏部上迅名。帝曰:「迅御前駁議,朕猶識之,可復故官。」未赴而京師陷。迅家居,與弟迓設砦自衛。迓短小精悍,馬上舞百斤鐵椎。兄弟率里中壯士,捕剿土寇略盡。大清兵至,破砦,迅闔門死之。

若騏劾黃道周以媚嗣昌。歷職方郎中,新甲遣赴寧、錦督戰,覆洪承疇等十餘萬軍,獨渡海逃還,論死繫獄。李自成陷都城,出降。

範淑泰,字通也,滋陽人。崇禎元年進士。授行人。五年冬,擢工科給事中。上疏陳刑獄繁多,乞敕刑官疏理,帝褒納之。流賊犯河南,追論先任巡撫樊尚璟罪,劾總兵鄧淫掠狀。時中官張彝憲言天下逋賦至一千七百餘萬,請遣科道官督征。帝大怒,責撫按回奏。淑泰言民貧盜起,逋賦難以督追,不從。給事中莊鰲獻、章正宸建言下吏,抗疏救之。

吏部張捷薦逆黨呂純如,淑泰極論其謬,並論大學士王應熊朋比行私,劾捷徇應熊意,用其私人王維章撫蜀。言:「維章官西寧,坐加征激變,落職閒住。捷朦朧啟事,明肆奸欺。」帝責捷自陳。捷詆淑泰黨同伐異,帝不問。時皇陵被毀,巡撫楊一鵬得罪。應熊以座主故,力庇之。淑泰發其停匿章奏狀,帝亦不究。淑泰乃摭應熊納賄數事上之,應熊損貲助陵工,淑泰又劾其召寇庇奸。帝責以挾私求勝,終不納。

十一年冬,上疏言:「今以措餉故,至搜括借助。即行之而得,再有兵事,能復行乎!治不規其可久,徒倉皇於補救之術,非所以為忠也。陛下方以清節風天下,而乃條敘百官金錢於多寡之間,是教之貪也。至借貸之說,尤不可行。京師根本重地,邇者物力困竭,富商大賈大半旋歸。內不安,何以攘外!乞立寢其說。」又言:「強兵莫如行法。今之兵,索餉則強,赴敵則弱;殺良冒功則強,除暴救民則弱。請明示法令,諸將能用命殺賊者,立擢為大將,否則死無赦。毋以降級戴罪,徒為不切身之痛癢。」帝是其言。

十五年遷吏科,典浙江鄉試,事竣還家。十二月,大清兵圍兗州,淑泰竭力固守。城破,死之。詔贈太僕少卿,官一子。

高名衡,字仲平,沂州人。崇禎四年進士。除如皋知縣,以才調興化,徵授御史。十二年出按河南。明年期滿,留再巡一年。

十四年正月,李自成陷洛陽,乘勝遂圍開封。巡撫李仙風時在河北,名衡集眾守。周王恭枵發庫金百萬兩,募死士殺賊,烝米屑麥,執爨以餉軍,凡七晝夜。仙風馳還開封,副將陳永福背城而戰,斬首二千。游擊高謙夾擊,斬首七百。賊解去。仙風既還,與名衡互訐奏。帝以陷福籓罪詔逮仙風,以襄陽兵備副使張克儉代。克儉已前死難,即擢名衡右僉都御史代之。以永福充總兵官都督僉事,鎮守河南。

當是時,賊連陷南陽、鄧、汝十餘州縣,唐、徽二王遇害,名稀不能救。開封周邸圖書文物之盛甲他籓,士大夫CV富,蓄積充牣。自成攻之不能克,然欲得而甘心焉。十二月杪,賊再圍開封。永福射自成,中其左目,炮殪上天龍等。自成大怒,急攻之。開封故宋汴都,金帝南遷所重築也,厚數丈,內堅致而疏外。賊用火藥放迸,火發即外擊,瓳飛鳴,賊騎皆糜爛,自成大驚。會楊文岳援兵亦至,乃解圍去。西華、郾、襄、睢、陳、太康、商丘、寧陵、考城俱陷。

十五年四月復至開封,圍而不攻,欲坐困之。六月,帝詔釋故尚書侯恂於獄,命督保定、山東、河北、湖北諸軍務,並轄平賊等鎮援剿官兵。拔知縣蘇京、王漢、王燮為御史。詔蘇京監延、寧、甘、固軍,趣孫傳庭出關;王漢監平賊鎮標楚、蜀軍,同侯恂等急擊;王燮監陽、懷東晉軍,刻期渡河。總兵許定國以晉軍次沁水,一夕潰去,寧武兵亦潰於懷慶,詔逮定國。七月,河上之兵潰。督師丁啟睿、保督楊文岳合左良玉、虎大威、楊德政、方國安諸軍,次於朱仙鎮。良玉走還襄陽,諸軍皆潰,啟睿、文岳奔汝寧。詔山東總兵官劉澤清援開封。城中食盡,名衡、永福偕監司梁炳、蘇壯、吳士講,同知蘇茂灼,通判彭士奇,推官黃澍等守益堅。澤清以兵來援,諸軍並集河北朱家寨不敢進。澤清曰:「朱家寨去開封八里。我以兵五千南渡,依河而營,引水環之。以次結八營,直達大堤。築甬道輸河北之粟,以餉城中。賊兵已老,可一戰走也。」諸軍皆曰:「善。」乃以兵三千人先渡立營。賊攻之,戰三晝夜,諸軍無繼者,甬道不就,澤清拔營歸。日夜望傳庭出關,不至。

賊圖開封者三,士馬損傷多,積憤,誓必拔之。圍半年,師老糧匱,欲決黃河灌之。以城中子女貨寶,猶豫不決。聞秦師已東,恐諸鎮兵夾擊,欲變計。會有獻計於巡按御史嚴雲京者,請決河以灌賊。雲京語名衡、澍,名衡、澍以為然。周王恭枵募民築羊馬墻,堅厚如高岸。賊營直傅大堤,河決賊可盡,城中無虞。我方鑿朱家寨口,賊知,移營高阜,艨艟巨筏以待,而驅掠民夫數萬反決馬家口以灌城。九月癸未望,夜半,二口並決。天大雨連旬,黃流驟漲,聲聞百里。丁夫荷鍤者,隨堤漂沒十數萬,賊亦沉萬人。河入自北門,貫東南門以出,流入於渦水。名衡、永福乘小舟至城頭,周王率其宮眷及寧鄉諸郡王避水棲城樓,坐雨絕食者七日。王燮以舟迎王,王從城上泛舟出,名衡等皆出。茂灼、士奇久餓不能起,並溺死。賊浮艦入城,遺民俱盡,拔營而西。城初圍時百萬戶,後饑疫死者十二三。汴梁佳麗甲中州,群盜心艷之,至是盡沒於水。帝聞,痛悼。猶念諸臣拒守勞,命敘功。加名衡兵部右侍郎,名衡辭以疾。即擢王漢右僉都御史,代名衡巡撫河南。名衡歸,未幾,大清兵破沂州,名衡夫婦殉難。

王漢,字子房,掖縣人。崇禎十年進士。除高平知縣。調河內,擒巨寇天壇山劉二。又乘雪夜破妖僧智善。夜半渡河,破賊楊六郎。李自成圍開封,漢然火金龍口柳林為疑兵,遣死士入賊中,聲言:「諸鎮兵來援,各數十萬至矣。」賊聞則驚走。

漢為人負氣愛士。人有一長,嗟歎之不容口。僚屬紳士陳民疾苦,或言己過,則瞿然下拜。用兵士卒同甘苦,人樂為之死。好用間,賊中虛實莫不知。攻天壇山賊,山陡絕,登者挽以布。漢持刀直上,人服其勇。時賊氛日熾,帝每臨朝而歎漢前後破賊功,降旨優敘。

十五年春,以減俸行取入都,與蘇京、王燮同召對,稱旨。命三臣皆以試御史監軍。漢監平賊鎮標楚、蜀軍,與督臣侯恂南援汴。

時兵部奏援剿兵十萬,十之四以屬京、燮,屬漢以其六。漢所監凡五萬九千,然大半已潰散,兵部空名使之。漢乃請自立標營兵千人,騎二百,報可。乃簡保營兵百餘人,募邯鄲、鉅鹿壯士三百人,又取故治河內所練義兵及修武、濟源素從征剿者五百人,及親故子弟,合千人。八月朔夜半,襲賊范家灘,斬一紅甲賊目。檄諸將合剿。自走襄陽,督左良玉兵救汴。至潼關,有詔漢巡按河南。時賊灌開封,漢聞,趣諸將自柳園夜半渡河,伏兵西岸。檄卜從善等夾攻之,斬首九十餘級,遂入汴,大張旗鼓為疑兵。追賊至朱仙鎮,連戰皆捷。巡撫高名衡謝病,即擢漢右僉都御史代之。漢乃廣間諜,收土豪,議屯田,謀所以圖賊。

無何,劉超反永城。超,永城人,跛而狡,為貴州總兵,坐罪免。上疏言兵計,陳新甲用為河南總兵。以私怨殺其鄉官御史魏景琦一家三十餘人,懼罪,遂據城反。漢上疏請討,語洩,超得為備。明年正月,漢入永城,聲言招撫,為賊所殺。參將陳治邦、遊擊連光耀父子皆戰死。遊擊馬魁負漢屍以出,面如生。詔贈兵部尚書,廕錦衣世百戶,建祠致祭。既而超伏誅,傳首九邊。

徐汧,字九一,長洲人。生未期而孤。稍長砥行,有時名,與同里楊廷樞相友善。廷樞,復社諸生所稱維斗先生者也。天啟五年,魏大中被逮過蘇州,汧貸金資其行。周順昌被逮,緹騎橫索錢,汧與廷樞斂財經理之。當是時,汧、廷樞名聞天下。

崇禎元年,汧成進士,改庶吉士,授檢討。三年,廷樞舉應天鄉試第一。中允黃道周以救錢龍錫貶官。倪元璐,道周同年生,請以己代謫,帝不允。汧上疏頌道周、元璐賢,且自請罷黜,帝詰責汧。汧曰:「推賢讓能,藎臣所務;難進易退,儒者之風。間者陛下委任之意希注外廷,防察之權輒逮閽寺,默窺聖意,疑貳漸萌。萬一士風日賤,宸嚮日移,明盛之時為憂方大。」帝不聽。汧尋乞假歸。還朝,遷右庶子,充日講官。

十四年奉使益王府,便道還家。當是時,復社諸生氣甚盛,汧與廷樞、顧杲、華允誠等往復尤契。居久之,京師陷。福王召汧為少詹事。汧以國破君亡,臣子不當叨位,且痛宗社之喪亡,由朋黨相傾,移書當事,勸以力破異同之見。既就職,陳時政七事,心卷心卷以化恩仇、去偏黨為言。而安遠侯柳祚昌疏攻汧,謂:「朝服謁潞王於京口,自恃東林巨魁,與復社楊廷樞、顧杲諸奸狼狽相倚。陛下定鼎金陵,彼為《討金陵檄》,所云『中原逐鹿,南國指馬』是何語?乞置汧於理,除廷樞、杲名,其餘徒黨,容臣次第糾彈。」時國事方棘,事亦竟寢。汧移疾歸。

明年,南京失守,蘇、常相繼下。汧慨然太息,作書戒二子,投虎丘新塘橋下死。郡人赴哭者數千人。時又有一人儒冠藍衫而來,躍虎丘劍池中,土人憐而葬之,卒不知何人也。

於是廷樞聞變,走避之鄧尉山中。久之,四方弄兵者群起,廷樞負重名,咸指目廷樞。當事者執廷樞,好言慰之,廷樞嫚罵不已,殺之蘆墟泗洲寺。首已墮,聲從項中出,益厲。門人迮紹原購其屍葬焉。

汧子枋,字昭法,舉十五年鄉試。枋依隱,有高行雲。

鹿善繼,字伯順,定興人。祖久征,萬曆中進士,授息縣知縣。時詔天下度田,各署上中下壤,息獨以下田報,曰:「度田以紓民,乃病民乎!」調襄垣,擢御史,以言事謫澤州判官,遷滎澤知縣,未任而卒。父正,苦節自礪。縣令某欲見之,方糞田,投鍤而往。急人之難,傾其家不惜,遠近稱鹿太公。

善繼端方謹愨。由萬曆四十一年進士,授戶部主事。內艱除,起故官。遼左餉中絕,廷臣數請發帑,不報。會廣東進金花銀,善繼稽舊制,金花貯庫,備各邊應用。乃奏記尚書李汝華曰:「與其請不發之帑,何如留未進之金?」汝華然之。帝怒,奪善繼俸一年,趣補進。善繼持不可,以死爭。乃奪汝華俸二月,降善繼一級,調外。汝華懼,卒補銀進。泰昌改元,復原官,典新餉。連疏請帑百萬,不報。

天啟元年,遼陽陷,以才改兵部職方主事。大學士孫承宗理兵部事,推心任之。及閱視關門,以善繼從。出督師,復表為贊畫。布衣羸馬,出入亭障間,延見將卒相勞苦,拓地四百里,收復城堡數十,承宗倚之若左右手。在關四年,累進員外郎、郎中。承宗謝事,善繼亦告歸。

先是,楊、左之獄起,魏大中子學洢、左光斗弟光明,先後投鹿太公家。太公客之,與所善義士容城舉人孫奇逢謀,持書走關門,告其難於承宗。承宗、善繼謀借巡視薊門,請入覲。奄黨大嘩,謂閣部將提兵清君側,嚴旨阻之。獄益急,五日一追贓,搒掠甚酷。太公急募得數百金輸之,而兩人者則皆已斃矣。至是,善繼歸,而周順昌之獄又起。順昌,善繼同年生,善繼又為募得數百金,金入而順昌又斃。奄黨居近善繼家,難家子弟僕從相望於道。太公曰:「吾不懼也。」崇禎元年,逆榼既誅,善繼起尚寶卿,遷太常少卿,管光祿丞事,再請歸。

九年七月,大清兵攻定興。善繼家在江村,白太公請入扞城,太公許之。與里居知州薛一鶚等共守。守六日而城破,善繼死。家人奔告太公,太公曰:「嗟乎,吾兒素以身許國,今果死,吾復何憾!」事聞,贈善繼大理卿,謚忠節,敕有司建祠。子化麟,舉天啟元年鄉試第一,伏闕訟父忠。逾年亦卒。

薛一鶚,字百當,由貢生為黃州通判。荊王姬誣他姬鴆世子,一鶚白其誣。奄人傳太妃命,欲竟其獄,卒直之。遷蘭州知州。州北有田沒於番,吏派其賦於他戶,後田復歸,為衛卒所據,而民出賦三十年,一鶚核除其害。至是佐善繼城守,遂同死。

贊曰:士大夫致政里居,無封疆民社之責,可遜跡自全,非以必死為勇也。然而慷慨捐軀,冒白刃而不悔,湛宗覆族,君子哀之。豈非名義所在,有重於生者乎!氣節凜然,要於自遂其志。其英風義烈,固不可泯沒於宇宙間矣。


\end{pinyinscope}