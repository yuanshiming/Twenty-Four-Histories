\article{職官志}


明官制,沿漢、唐之舊而損益之。自洪武十三年罷丞相不設,析中書省之政歸六部,以尚書任天下事,侍郎貳之。而殿閣大學士只備顧問學史料《名哲言行錄》的編纂者。此書多系間接材料編成,其,帝方自操威柄,學士鮮所參決。其糾劾則責之都察院,章奏則達之通政司,平反則參之大理寺,是亦漢九卿之遺意也。分大都督府為五,而徵調隸於兵部。外設都、布、按三司,分隸兵刑錢穀,其考核則聽於府部。是時吏、戶、兵三部之權為重。迨仁、宣朝,大學士以太子經師恩,累加至三孤,望益尊。而宣宗內柄無大小,悉下大學士楊士奇等參可否。雖吏部蹇義、戶部夏原吉時召見,得預各部事,然希闊不敵士奇等親。自是內閣權日重,即有一二吏、兵之長與執持是非,輒以敗。至世宗中葉,夏言、嚴嵩迭用事,遂赫然為真宰相,壓制六卿矣。然內閣之擬票,不得不決於內監之批紅,而相權轉歸之寺人。於是朝廷之紀綱,賢士大夫之進退,悉顛倒於其手。伴食者承意指之不暇,間有賢輔,卒蒿目而不能救。初,領五都督府者,皆元勛宿將,軍制肅然。永樂間,設內監監其事,猶不敢縱。沿習數代,勛戚紈褲司軍紀,日以惰毀。既而內監添置益多,邊塞皆有巡視,四方大征伐皆有監軍,而疆事遂致大壞,明祚不可支矣。跡其興亡治亂之由,豈不在用人之得失哉!至於設官分職,體統相維,品式具備,詳列後簡。覽者可考而知也。

宗人府三公三孤太子三師三少內閣吏部戶部附總督倉場禮部

兵部附協理京營戎政刑部工部附提督易州山廠

宗人府。宗人令一人,左、右宗正各一人,左、右宗人各一人,並正一品掌皇九族之屬籍,以時修其玉牒,書宗室子女適庶、名封、嗣襲、生卒、婚嫁、謚葬之事。凡宗室陳請,為聞於上,達材能,錄罪過。初,洪武三年置大宗正院。二十二年改為宗人府,並以親王領之。秦王樉為令,晉王㭎、燕王棣為左、右宗正,周王隸、楚王楨為左、右宗人。其後以勳戚大臣攝府事,不備官,而所領亦盡移之禮部。其屬,經歷司,經歷一人,正五品典出納文移。

太師、太傅、太保為三公,正一品少師、少傅、少保為三孤,從一品掌佐天子,理陰陽,經邦弘化,其職至重。無定員,無專授。洪武三年,授李善長太師,徐達太傅。先是,常遇春已贈太保。三孤無兼領者。建文、永樂間罷公、孤官,仁宗復設。永樂二十二年八月,復置三公、三少。宣德三年,敕太師、英國公張輔,少師、吏部尚書蹇義,少傅、兵部尚書、華蓋殿大學士楊士奇,少保兼太子少傅、戶部尚書夏原吉,各輟所領,侍左右,咨訪政事。公孤之官,幾於專授。逮義、原吉卒,士奇還領閣務。自此以後,公、孤但虛銜,為勛戚文武大臣加官、贈官。而文臣無生加三公者,惟贈乃得之。嘉靖二年加楊廷和太傅,辭不受。其後文臣得加三公惟張居正,萬曆九年加太傅,十年加太師。

太子太師、太子太傅、太子太保,並從一品掌以道德輔導太子,而謹護翼之。太子少師、太子少傅、太子少保,並正二品掌奉太子以觀三公之道德而教諭焉。太子賓客,正三品掌侍太子贊相禮儀,規誨過失。皆東宮大臣,無定員,無專授。洪武元年,太祖有事親征,慮太子監國,別設宮僚或生嫌隙,乃以朝臣兼宮職:李善長兼太子少師,徐達兼太子少傅,常遇春兼太子少保,治書侍御史文原吉、范顯祖兼太子賓客。三年,禮部尚書陶凱請選人專任東宮官,罷兼領,庶於輔導有所責成。帝諭以江充之事可為明鑒,立法兼領,非無謂也。由是東宮師傅止為兼官、加官及贈官。惟永樂間,成祖幸北京,以姚廣孝專為太子少師,留輔太子。自是以後,終明世皆為虛銜,於太子輔導之職無與也。

中極殿大學士,舊名華蓋殿建極殿大學士,舊名謹身殿文華殿大學士,武英殿大學士,文淵閣大學士,東閣大學士,並正五品掌獻替可否,奉陳規誨,點檢題奏,票擬批答,以平允庶政。凡上之達下,曰詔,曰誥,曰制,曰冊文,曰諭,曰書,曰符,曰令,曰檄,皆起草進畫,以下之諸司。下之達上,曰題,曰奏,曰表,曰講章,曰書狀,曰文冊,曰揭帖,曰制對,曰露布,曰譯,皆審署申覆而修畫焉,平允乃行之。凡車駕郊祀、巡幸則扈從。御經筵,則知經筵或同知經筵事。東宮出閣講讀,則領其事,敘其官,而授之職業。冠婚,則充賓贊及納徵等使。修實錄、史志諸書,則充總裁官。春秋上丁釋奠先師,則攝行祭事。會試充考試官,殿試充讀卷官。進士題名,則大學士一人撰文,立石於太學。大典禮、大政事,九卿、科道官會議已定,則按典制,相機宜,裁量其可否,斟酌入告。頒詔則捧授禮部。會敕則稽其由狀以請。宗室請名、請封,諸臣請謚,並擬上。以其授餐大內,常侍天子殿閣之下,避宰相之名,又名內閣。

先是,太祖承前制,設中書省,置左、右丞相,正一品。甲辰正月,初置左、右相國,以李善長為右相國,徐達為左相國。吳元年命百官禮儀俱尚左,改右相國為左相國,左相國為右相國。洪武元年改為左、右丞相。平章政事,從一品左、右丞,正二品參知政事,從二品以統領眾職。置屬官,左、右司,郎中,正五品員外郎正六品都事、檢校,正七品照磨、管勾,從七品參議府參議,正三品參軍、斷事官,從三品斷事、經歷,正七品知事,正八品都鎮撫司都鎮撫,正五品考功所,考功郎,正七品。甲辰十月以都鎮撫司隸大都督府。吳元年革參議府。洪武元年革考功所。二年革照磨、檢校所、斷事官。七年設直省舍人十人,尋改中書舍人。

洪武九年汰平章政事、參知政事。十三年正月,誅丞相胡惟庸,遂罷中書省。其官屬盡革,惟存中書舍人。九月,置四輔官,以儒士王本等為之。置四輔官,告太廟,以王本、杜佑、襲斅為春官,杜斅、趙民望、吳源為夏官,兼太子賓客。秋、冬官缺,以本等攝之。一月內分司上中下三旬。位列公、侯、都督之次。尋亦罷。十五年,仿宋制,置華蓋殿、武英殿、文淵閣、東閣諸大學士,禮部尚書邵質為華蓋,檢討吳伯宗為武英,翰林學士宋訥為文淵,典籍吳沉為東閣。又置文華殿大學士,徵耆儒鮑恂、餘詮、張長年等為之,以輔導太子。秩皆正五品。二十八年敕諭群臣:「國家罷丞相,設府、部、院、寺以分理庶務,立法至為詳善。以後嗣君,其毋得議置丞相。臣下有奏請設立者,論以極刑。」當是時,以翰林、春坊詳看諸司奏啟,兼司平駁。大學士特侍左右,備顧問而已。建文中,改大學士為學士。悉罷諸大學士,各設學士一人。又改謹身殿為正心殿,設正心殿學士。成祖即位,特簡解縉、胡廣、楊榮等直文淵閣,參預機務。閣臣之預務自此始。然其時,入內閣者皆編、檢、講讀之官,不置官屬,不得專制諸司。諸司奏事,亦不得相關白。

仁宗以楊士奇、楊榮東宮舊臣,陞士奇為禮部侍郎兼華蓋殿大學士,榮為太常卿兼謹身殿大學士,謹身殿大學士,仁宗始置,閣職漸崇。其後士奇、榮等皆遷尚書職,雖居內閣,官必以尚書為尊。景泰中,王文始以左都御史進吏部尚書,入內閣。自後,誥敕房、制敕房俱設中書舍人,六部承奉意旨,靡所不領,而閣權益重。世宗時,三殿成,改華蓋為中極,謹身為建極,閣銜因之。嘉靖以後,朝位班次,俱列六部之上。

吏部。尚書一人,正二品左、右侍郎各一人。正三品其屬,司務廳,司務二人,從九品文選、驗封、稽勳、考功四清吏司,各郎中一人,正五品員外郎一人,從五品主事一人,正六品。洪武三十一年增設文選司主事一人。正統十一年增設考功司主事一人。

尚書,掌天下官吏選授、封勛、考課之政令,以甄別人才,贊天子治。蓋古冢宰之職,視五部為特重。侍郎為之貳。

司務,掌催督、稽緩、勾銷、簿書。明初,設主事、司務各四人,為首領官,有主事印。洪武二十九年改主事為司官,裁司務二人。各部並同。

文選,掌官吏班秩遷升、改調之事,以贊尚書。凡文官之品九,品有正從,為級一十八。不及九品曰未入流。凡選,每歲有大選,有急選,有遠方選,有歲貢就教選,間有揀選,有舉人乞恩選。選人或登資簿,釐其流品,平其銓注,而序遷之。凡陞必考滿,若員缺當補,不待考滿,曰推升。類推上一人,單推上二人。三品以上,九卿及僉都御史、祭酒,廷推上二人或三人。內閣,吏、兵二部尚書,廷推上二人。凡王官不外調,王姻不內除,大臣之族不得任科道,僚屬同族則以下避上。外官才地不相宜,則酌其繁簡互換之。有傳陞、乞升者,並得執奏。以署職、試職、實授奠年資,以開設、裁併、兼攝適繁簡,以薦舉、起廢、徵召振幽滯,以帶俸、添注寄恩冗,以降調、除名馭罪過,以官程督吏治,以寧假悉人情。

驗封,掌封爵、襲蔭、褒贈、吏算之事,以贊尚書。凡爵非社稷軍功不得封,封號非特旨不得與。或世或不世,皆給誥券。衍聖公及戚里恩澤封,不給券。凡券,左右各一,左藏內府,右給功臣之家。襲封則徵其誥券,稽其功過,核其宗支,以第其世流降除之等。土官則勘其應襲與否,移文選司注擬。宣慰、宣撫、安撫、長官諸司領士兵者,則隸兵部。凡廕敘,明初,自一品至七品,皆得蔭一子以世其祿。洪武十六年,定職官子孫廕敘。正一品子,正五品用。從一品子,從五品用。正二品子,正六品用。從二品子,從六品用。正三品子,正七品用。從三品子,從七品用。正四品子,正八品用。從四品子,從八品用。正五品子,正九品用。從五品子,從九品用。正六品子,於未入流上等職內敘用。從六品子,于未入流中等職內敘用。正從七品子,於未入流下等職內敘用。後乃漸為限制,京官三品以上,考滿著績,始廕一子曰官生,其出自特恩者曰恩生。凡封贈,公、侯、伯之追封,皆遞進一等。三品以上政績顯異及死諫、死節、陣亡者,皆得贈官。其見任則初授散階,京官滿一考,及外官滿一考而以最聞者,皆給本身誥敕。七品以上皆得推恩其先。五品以上授誥命,六品以下授敕命。一品,三代四軸。二品、三品,二代三軸。四品、五品、六品、七品,一代二軸。八品以下流內官,本身一軸。一品軸以玉,二品軸以犀,三品、四品軸以鋈金,五品以下軸以角。曾祖、祖、父皆如其子孫官。公、侯、伯視一品。外內命婦視夫若子之品。生曰封,死曰贈。若先有罪譴則停給。文之散階四十有二,以歷考為差。正一品,初授特進榮祿大夫,升授特進光祿大夫。從一品,初授榮祿大夫,升授光祿大夫。正二品,初授資善大夫,升授資政大夫,加授資德大夫。從二品,初授中奉大夫,升授通奉大夫,加授正奉大夫。正三品,初授嘉議大夫,陞授通議大夫,加授正議大夫。從三品,初授亞中大夫,升授中大夫,加授大中大夫。正四品,初授中順大夫,升授中憲大夫,加授中議大夫。從四品,初授朝列大夫,陞授朝議大夫,加授朝請大夫。正五品,初授奉議大夫,升授奉政大夫。從五品,初授奉訓大夫,陞授奉直大夫。正六品,初授承直郎,升授承德郎。從六品,初授承務郎,升授儒林郎,吏材幹出身授宣德郎。正七品,初授承事郎,升授文林郎,吏材幹授宣議郎。從七品,初授從仕郎,升授徵仕郎。正八品,初授迪功郎,升授修職郎。從八品,初授迪功佐郎,升授修職佐郎。正九品,初授將仕郎,升授登仕郎。從九品,初授將仕佐郎,升授登仕佐郎。外命婦之號九。公曰某國夫人。侯曰某侯夫人。伯曰某伯夫人。一品曰夫人,後稱一品夫人。二品曰夫人。三品曰淑人。四品曰恭人。五品曰宜人。六品曰安人。七品曰孺人。因其子孫封者,加太字,夫在則否。凡封贈之次,七品至六品一次,五品一次,初制有四品一次,後省。三品、二品、一品各一次。三母不並封,兩封從優品。父職高於子,則進一階。父應停給及子為人後者,皆得移封。嫡在不封生母,生母未封不先封其妻。妻之封,止於一嫡一繼。其封贈後而以墨敗者,則追奪。

稽勛,掌勛級、名籍、喪養之事,以贊尚書。凡文勛十。正一品,左、右柱國。從一品,柱國。正二品,正治上卿。從二品,正治卿。正三品,資治尹。從三品,資治少尹。正四品,贊治尹。從四品,贊治少尹。正五品,修正庶尹。從五品,協正庶尹。自五品以上,歷再考,乃授勛。凡百官遷除、降調皆開寫年甲、鄉貫、出身。每歲十二月貼黃,春秋清黃,皆赴內府。有故,揭而去之。凡父母年七十,無兄弟,得歸養。凡三年喪,解職守制,糾擿其奪喪、匿喪、短喪者。惟欽天監官奔喪三月復任。

考功,掌官吏考課、黜陟之事,以贊尚書。凡內外官給由,三年初考,六年再考,並引請,九年通考,奏請綜其稱職、平常、不稱職而陟黜之。陟無過二等,降無過三等,其甚者黜之、罪之。京官六年一察,察以巳、亥年。五品下考察其不職者,降罰有差;四品上自陳,去留取旨。外官三年一朝,朝以辰、戌、丑、未年。前期移撫、按官,各綜其屬三年內功過狀註考,匯送覆核以定黜陟。倉場庫官一年考,巡檢三年考,教官九年考。府州縣官之考,以地之繁簡為差。吏之考,三、六年滿,移驗封司撥用。九年滿,又試授官。惟王官及欽天、御用等監官不考。凡內外官彈章,稽其功過,擬去留以請上裁。薦舉、保留,則核其政績旌異焉。

明初,設四部於中書省,分掌錢穀禮儀、刑名、營造之務。洪武元年始置吏、戶、禮、兵、刑、工六部,設尚書、侍郎、郎中、員外郎、主事,尚書正三品,侍郎正四品,郎中正五品,員外郎正六品,主事正七品。仍隸中書省。六年,部設尚書二人,侍郎二人。吏部設總部、司勳、考功三屬部,部設郎中、員外郎各一人,主事各二人。十三年,罷中書省,仿《周官》六卿之制,升六部秩,各設尚書、侍郎一人。惟戶部侍郎二人。每部分四屬部,吏部屬部加司封。每屬部設郎中、員外郎、主事各一人,尋增侍郎一人。二十二年,改總部為選部。二十九年,定為文選、驗封、稽勛、考功四司并五部屬,皆稱清吏司。建文中,改六部尚書為正一品,設左、右侍中,正二品位侍郎上,除去諸司清吏字。成祖初,悉復舊制。

永樂元年,以北平為北京,置北京行部尚書二人,侍郎四人,其屬置六曹清吏司。吏、戶、禮、兵、工五曹,郎中、員外郎、主事各一人。刑曹,郎中一人,員外郎一人,主事四人,照磨、檢校各一人,司獄一人。尋戶曹亦增設主事三人。後又分置六部,各稱行在某部。十八年定都北京,罷行部及六曹,以六部官屬移之北,不稱行在。其留南京者,加「南京」字。洪熙元年,復置各部官屬於南京,去「南京」字,而以在北京者加「行在」字,仍置行部。宣德三年復罷行部。正統六年,於北京去「行在」字,於南京仍加「南京」字,遂為定制。景泰中,吏部嘗設二尚書。天順初,復罷其一。

按吏部尚書,表率百僚,進退庶官,銓衡重地,其禮數殊異,無與並者。永樂初,選翰林官入直內閣。其後大學士楊士奇等加至三孤,兼尚書銜,然品敘列尚書蹇義、夏原吉下。景泰中,左都御史王文升吏部尚書,兼學士,入內閣,其班位猶以原銜為序次。自弘治六年二月,內宴,大學士丘濬遂以太子太保、禮部尚書,居太子太保、吏部尚書王恕之上。其後由侍郎、詹事入閣者,班皆列六部上矣。

戶部。尚書一人,正二品左、右侍郎各一人,正三品其屬,司務廳,司務二人,從九品浙江、江西、湖廣、陜西、廣東、山東、福建、河南、山西、四川、廣西、貴州、雲南十三清吏司,各郎中一人。正五品。宣德以後增設山西司郎中三人,陜西、貴州、雲南三司郎中各二人,山東司郎中一人。員外郎一人,從五品。宣德七年增設四川、雲南二司員外郎各一人,後仍革。主事二人,正六品宣德以後增設雲南司主事七人,浙江、江西、湖廣、陜西、福建、河南、山西七司主事各二人,山東、四川、貴州三司主事各一人。照磨所,照磨一人,正八品檢校一人,正九品。所轄,寶鈔提舉司,提舉一人,正八品,副提舉一人。正九品典史一人,後副提舉、典史俱革。鈔紙局,大使、副使各一人,後革副使。印鈔局,大使、副使各一人,後俱革。寶鈔廣惠庫,大使一人,正九品,副使二人,從九品,嘉靖中革。廣積庫,大使一人,正九品,副使一人,從九品,典史一人,嘉靖中,副使、典史俱革。贓罰庫,大使一人,正九品,副使二人,從九品,嘉靖中革。甲字、乙字、丙字、丁字、戊字庫,大使五人,正九品,副使六人,從九品,丁字庫二人,嘉靖中革一人,並革乙字、戊字二庫副使。廣盈庫,大使一人,從九品副使二人。嘉靖中革。外承運庫,大使二人,正九品副使二人,從九品。後大使、副使俱革。承運庫,大使一人,正九品副使一人。從九品。嘉靖中革。行用庫,大使、副使各一人,後俱革。太倉銀庫,大使、副使各一人。嘉靖中,革副使。御馬倉,大使一人,從九品副使一人。軍儲倉,大使一人,從九品副使一人,後大使、副使俱革。長安、東安、西安、北安門倉,各副使一人,東安門倉舊二人,萬歷八年革一人。張家灣鹽倉檢校批驗所,大使、副使各一人。隆慶六年並革。

尚書,掌天下戶口、田賦之政令。侍郎貳之。稽版籍、歲會、賦役實征之數,以下所司。十年攢黃冊,差其戶上下畸零之等,以周知其登耗。凡田土之侵占、投獻、詭寄、影射有禁,人戶之隱漏、逃亡、朋充、花分有禁,繼嗣、婚姻不如令有禁。皆綜核而糾正之。天子耕耤,則尚書進耒耜。以墾荒業貧民,以占籍附流民,以限田裁異端之民,以圖帳抑兼并之民,以樹藝課農官,以芻地給馬牧,以召佃盡地利,以銷豁清賠累,以撥給廣恩澤,以給除差優復,以鈔錠節賞賚,以讀法訓吏民,以權量和市糴,以時估平物價,以積貯之政恤民困,以山澤、陂池、關市、坑冶之政佐邦國,贍軍輸,以支兌、改兌之規利漕運,以蠲減、振貸、均糴、捕蝗之令憫災荒,以輸轉、屯種、糴買、召納之法實邊儲,以祿廩之制馭貴賤。洪武二十五年,重定內外文武官歲給祿俸之制。正一品,一千四十四石。從一品,八百八十八石。正二品,七百三十二石。從二品,五百七十六石。正三品,四百二十石。從三品,三百一十二石。正四品,二百八十八石。從四品,二百五十二石。正五品,一百九十二石。從五品,一百六十八石。正六品,一百二十石。從六品,九十六石。正七品,九十石。從七品,八十四石。正八品,七十八石。從八品,七十二石。正九品,六十六石。從九品,六十石。未入流,三十六石。俱米鈔本折兼支。

十三司,各掌其分省之事,兼領所分兩京、直隸貢賦,及諸司、衛所祿俸,邊鎮糧餉,並各倉場鹽課、鈔關。浙江司帶管在京羽林右、留守左、龍虎、應天、龍驤、義勇右、康陵七衛,神機營。江西司帶管在京旗手、金吾前、金吾後、金吾左、濟陽五衛。湖廣司帶管國子監、教坊司,在京羽林前、通州、和陽、豹韜、永陵、昭陵六衛,及興都留守司。福建司帶管順天府,在京燕山左、武驤左、武驤右、驍騎右、虎賁右、留守後、武成中、茂陵八衛,五軍、巡捕、勇士、四衛各營,及北直隸永平、保定、河間、真定、順德、廣平、大名七府,延慶、保安二州,大寧都司、萬全都司,並北直隸所轄各衛所,山口、永盈、通濟各倉。山東司帶管在京錦衣、大寧中、大寧前三衛及遼東都司,兩淮、兩浙、長蘆、河東、山東、福建各鹽運司,四川、廣東、海北、雲南黑鹽井、白鹽井、安寧、五井各鹽課提舉司,陜西靈州鹽課司,江西南贛鹽稅。山西司帶管在京燕山前、鎮南、興武、永清左、永清右五衛,及宣府、大同、山西各鎮。河南司帶管在京府軍前、燕山右、大興左、裕陵四衛,牧馬千戶所及直隸潼關衛、蒲州千戶所。陜西司帶管宗人府、五軍都督府、六部、都察院、通政司、大理寺、詹事府、翰林院、太僕寺、鴻臚寺、尚寶司、六科、中書舍人、行人司、欽天監、太醫院、五城兵馬司、京衛武學、文思院、皮作局,在京留守右、長陵、獻陵、景陵四衛,神樞、隨侍二營,及延綏、寧夏、甘肅、固原各鎮。四川司帶管在京府軍後、金吾右、騰驤左、騰驤右、武德、神策、忠義後、武功中、武功左、武功右、彭城十一衛及應天府、南京四十九衛,南直隸安慶、蘇州、松江、常州、鎮江、徽州、寧國、池州、太平、廬州、鳳陽、淮安、揚州十三府,徐、滁、和、廣德四州,中都留守司並南直隸所轄各衛所。廣東司帶管在京羽林左、留守中、鷹揚、神武左、義勇前、義勇後六衛,蕃牧、奠靖二千戶所。廣西司帶管太常寺、光祿寺、神樂觀、犧牲所、司牲司、太倉銀庫、內府十庫,在京沈陽左、沈陽右、留守前、寬河、蔚州左五衛,及二十三馬房倉,各象房、牛房倉,京府各草場。雲南司帶管在京府軍、府軍左、府軍右、虎賁左、忠義右、忠義前、泰陵七衛,及大軍倉、皇城四門倉、并在外臨清、德州、徐州、淮安、天津各倉。貴州司帶管上林苑監,寶鈔提舉司,都稅司,正陽門、張家灣各宣課司,德勝門、安定門各稅課司,崇文門分司,在京濟州、會州、富峪三衛,及薊州、永平、密雲、昌平、易州各鎮,臨清、許墅、九江、淮安、北新、揚州、河西務各鈔關。

條為四科:曰民科,主所屬省府州縣地理、人物、圖志、古今沿革、山川險易、土地肥瘠寬狹、戶口物產多寡登耗之數;曰度支,主會計夏稅、秋糧、存留、起運及賞賚、祿秩之經費;曰金科,主市舶、魚鹽、茶鈔稅課,及贓罰之收折;曰倉科,主漕運、軍儲出納料糧。凡差三等,由吏部選授曰註差,疏名上請曰題差,答刂委曰部差。或三年,或一年,或三月而代。

初,洪武元年置戶部。六年,設尚書二人,侍郎二人。分為五科:一科,二科,三科,四科,總科。每科設郎中、員外郎各一人,主事四人。惟總科郎中、員外郎各二人,主事五人。八年,中書省奏戶、刑、工三部事繁,戶部五科,每科設尚書、侍郎各一人,郎中、員外郎各二人,主事五人,內會總科主事六人,外牽照科主事二人,司計四人,照磨二人,管勾一人。又置在京行用庫,隸戶部。設大使一人,副使二人,典史一人,都監二人。十三年,升部秩,定設尚書一人,侍郎二人。分四屬部:總部,度支部,金部,倉部。每部郎中、員外郎各一人。總部主事四人,度支部、金部主事各三人,倉部主事二人。尋罷在京行用庫。二十二年,改總部為民部。二十三年,又分四部為河南、北平、山東、山西、陜西、浙江、江西、湖廣、廣東、廣西、四川、福建十二部。四川部兼領雲南。部設郎中、員外郎各一人,主事二人,各領一布政司戶口、錢糧等事,量其繁簡,帶管京畿。每一部內仍分四科管理。又置照磨、檢校各一人,稽文書出入之數而程督之。十九年,復置寶鈔提舉司。洪武七年,初置寶鈔提舉司,提舉一人,正七品;副提舉一人,從七品;吏目一人,省注。所屬鈔紙、印鈔二局,各大使一人,正八品;副使一人,正九品;典史一人,省注。寶鈔、行用二庫,各大使二人,正八品;副使二人,正九品;典史一人,省注。尋升提舉為正四品。十三年罷,至是年復置,秩正八品。二十六年,令浙江、江西、蘇松人毋得任戶部。二十九年,改十二部為十二清吏司。建文中,仍為四司。餘見吏部。成祖復舊制。永樂元年,改北平司為北京司。十八年,革北京司,設雲南、貴州、交阯三清吏司。宣德十年,革交阯司,定為十三司。其後歸併職掌。凡宗室、勛戚、文武官吏之廩祿,陜西司兼領之。北直隸府州衛所,福建司兼領之。南直隸府州衛所,四川司兼領之。天下鹽課,山東司兼領之。關稅,貴州司兼領之。漕運及臨、德諸倉,雲南司兼領之。御馬、象房諸倉,廣西司兼領之。明初,嘗置司農司,尋罷吳元年置司農司。卿,正三品;少卿,正四品;丞,正五品;庸田署令,正五品;典簿、司計,正七品。洪武元年罷。三年復置司農司,開治所於河南,設卿一人,少卿二人,丞四人,主簿、錄事各二人。四年又罷。後置判錄司,亦罷。洪武十三年置判錄司,掌在京官吏俸給、文移、勘合。設判錄一人,正七品;副判二人,從七品。尋改判錄為司正,副判為左,右司副。十八年罷。皆不隸戶部。

總督倉場一人,掌督在京及通州等處倉場糧儲。洪武初,置軍儲倉二十所,各設官司其事。永樂中,遷都北京,置京倉及通州諸倉,以戶部司員經理之。宣德五年,始命李昶為戶部尚書,專督其事,遂為定制。以後,或尚書,或侍郎,俱不治部事。嘉靖十五年,又命兼督西苑農事。隆慶初,罷兼理。萬歷二年,另撥戶部主事一人陪庫,每日偕管庫主事收放銀兩,季終更替。九年裁革,命本部侍郎分理之。十一年復設。二十五年,以右侍郎張養蒙督遼餉。四十七年,增設督餉侍郎。崇禎間,有督遼餉、寇餉、宣大餉,增設三四人。天啟五年,又增設督理錢法侍郎。

禮部。尚書一人,正二品左、右侍郎各一人正三品其屬,司務廳,司務二人,從九品儀制、祠祭、主客、精膳四清吏司,各郎中一人,正五品員外郎一人,從五品主事一人,正六品。正統六年增設儀制、祠祭二司主事各一人。又增設儀制司主事一人,教習駙馬。弘治五年增設主客司主事一人,提督會同館。所轄,鑄印局,大使一人,副使二人。萬歷九年革一人。

尚書,掌天下禮儀、祭祀、宴饗、貢舉之政令。侍郎佐之。

儀制,分掌諸禮文、宗封、貢舉、學校之事。天子即位,天子冠、大婚,冊立皇太子、妃嬪、太子妃,上慈宮徽號,朝賀、朝見,大饗、宴饗,大射、宴射,則舉諸儀注條上之。若經筵、日講、耕耤、視學、策士、傳臚、巡狩、親征、進曆、進春、獻俘、奏捷,若皇太子出閣、監國,親王讀書、之籓,皇子女誕生、命名,以及百官、命婦朝賀皇太子、后妃之禮,與諸王國之禮,皆頒儀式於諸司。凡傳制、誥,開讀詔、敕、表、箋及上下百官往來移文,皆授以程式焉。凡歲請封宗室王、郡王、將軍、中尉、妃、主、君,各以其親疏為等。百官於宗王,具官稱名而不臣。王臣稱臣於其王。凡宗室、駙馬都尉、內命婦、蕃王之誥命,則會吏部以請。凡諸司之印信,領其制度。內閣,銀印,直紐,方一寸七分,厚六分,玉箸篆文。征西、鎮朔、平羌、平蠻等將軍,銀印,虎紐,方三寸三分,厚九分,柳葉篆文。宗人府、五軍都督府,俱正一品,銀印,三臺,方三寸四分,厚一寸。六部都察院、各都司,俱正二品,銀印,二臺,方三寸二分,厚八分。衍聖公、張真人、中都留守司,俱正二品,各布政司,從二品,銀印,二臺,方三寸一分,厚七分。後賜衍聖公三臺銀印。順天、應天二府,俱正三品,銀印,方二寸九分,厚六分五釐。通政司、大理寺、太常寺、詹事府、京衛、各按察司、各衛,俱正三品,苑馬寺、宣慰司,俱從三品,銅印,方二寸七分,厚六分。太僕寺、光祿寺、各鹽運司,俱從三品,銅印,方二寸六分,厚五分五厘。鴻臚寺各府,俱正四品,國子監、宣撫司,俱從四品,銅印,方二寸五分,厚五分。翰林院、左右春坊、尚寶司、欽天監、太醫院、上林苑監、六部各司、宗人府經歷司、王府長史司、各衛千戶所,俱正五品,司經局、五府經歷司、招討司、安撫司,俱從五品,銅印,方二寸四分,厚四分五厘。各州,從五品,銅印,方二寸三分,厚四分。都察院經歷司、大理寺左右司、五城兵馬司,大興、宛平、上元、江寧四縣,僧錄司、道錄司、中都留守司經歷司、斷事司,各都司經歷司、斷事司,各衛百戶所、長官司,王府審理所,俱正六品,光祿司各署,各布政司經歷司、理問所,俱從六品,銅印,方二寸二分,厚三分五釐。六科行人司、通政司經歷司、工部營繕所、太常寺典簿廳、上林苑監各署、各按察司經歷司、各縣,俱正七品,中書舍人,順天應天二府經歷司、京衛經歷司、光祿寺典簿廳、太僕寺主簿廳、詹事府主簿廳、各衛經歷司、各鹽運司經歷司、苑馬寺主簿廳、宣慰司經歷司,俱從七品,銅印,方二寸一分,厚三分。戶部、刑部、都察院各照磨所,兵部典牧所,國子監繩愆廳、博士廳、典簿廳,鴻臚寺主簿廳,欽天監主簿廳,各布政司照磨所,各府經歷司,王府紀善、典寶、典膳、奉祀、良醫、工正各所,宣撫司經歷司,俱正從八品,銅印,方二寸,厚二分五厘。刑部、都察院各司獄司,順天、應天二府照磨所、司獄司,鴻臚寺各署,國子監典籍廳,上林苑監典簿廳,內府寶鈔等各庫,御馬倉、草倉,會同館,織染所,文思院,皮作局,顏料局,鞍轡局,寶源局,軍器局,都稅司,教坊司,留守司司獄司,各都司司獄司,各按察司照磨所、司獄司,各府照磨所、司獄司,王府長史司典簿廳、教授、典義所,各府衛儒學、稅課司,陰陽學、醫學、僧綱司、道紀司、各巡檢司,俱正從九品,銅印,方一寸九分,厚二分二厘。各州縣儒學、倉庫、驛遞、閘壩批驗所、抽分竹木局、河泊所、織染局、稅課局、陰陽學、醫學、僧正司、道正司、僧會司、道會司,俱未入流,銅條記,闊一寸三分,長二寸五分,厚二分一釐。已上俱直紐,九疊篆文。監察御史,銅印,直紐,有眼,方一寸五分,厚三分,八疊篆文。總制、總督、巡撫並鎮守、公差等官,銅關防,直紐,闊一寸九分五釐,長二寸九分,厚三分,九疊篆文。外國王印三等:曰金,曰鍍金,曰銀。刓敝則換給之。凡祥瑞,辨其名物,無請封禪以蕩上心。以學校之政育士類,以貢舉之法羅賢才,以鄉飲酒禮教齒讓,以養老尊高年,以制度定等威,以恤貧廣仁政,以旌表示勸勵,以建言會議悉利病,以禁自宮遏奸民。

祠祭,分掌諸祀典及天文、國恤、廟諱之事。凡祭有三,曰天神、地祇、人鬼。辨其大祀、中祀、小祀而敬供之。飭其壇遺、祠廟、陵寢而數省閱之。蠲其牢醴、玉帛、粢羹、水陸瘞燎之品,第其配侑、從食、功德之上下而秩舉之。天下神祇在祀典者,則稽諸令甲,播之有司,以時謹其祀事。督日官頒曆象於天下。日月交食,移內外諸司救護。有災異即奏聞,甚者乞祭告修省。凡喪葬、祭祀,貴賤有等,皆定其程則而頒行之。凡謚,帝十七字,后十三字,妃、太子、太子妃並二字,親王一字,郡王二字,以字為差。勛戚、文武大臣請葬祭贈謚,必移所司,核行能,傅公論,定議以聞。其侍從勤勞、忠諫死者,官品未應謚,皆得特賜。凡帝后愍忌,祀於陵,輟朝不廢務。凡天文、地理、醫藥、卜筮、師巫、音樂、僧道人,並籍領之,有興造妖妄者罪無赦。

主客,分掌諸蕃朝貢接待給賜之事。諸蕃朝貢,辨其貢道、貢使、貢物遠近多寡豐約之數,以定王若使迎送、宴勞、廬帳、食料之等,賞賚之差。凡貢必省閱之,然後登內府,有附載物貨,則給直。若蕃國請嗣封,則遣頒冊於其國。使還,上其風土、方物之宜,贈遺禮文之節。諸蕃有保塞功,則授敕印封之。各國使人往來,有誥敕則驗誥敕,有勘籍則驗勘籍,毋令闌入。土官朝貢,亦驗勘籍。其返,則以鏤金敕諭行之,必與銅符相比。凡審言事,譯文字,送迎館伴,考稽四夷館譯字生、通事之能否,而禁飭其交通漏泄。凡朝廷賜賚之典,各省土物之貢,咸掌之。

精膳,分掌宴饗、牲豆、酒膳之事。凡御賜百官禮食,曰宴,曰酒飯,為上中下三等,視其品秩。番使、土官有宴,有下程,宴有一次,有二次,下程有常例,有欽賜。皆辨其等。親王之籓,王、公、將軍來朝,及其使人,亦如之。凡膳羞、酒醴、品料,光祿是供,會其數,而程其出納焉。凡廚役,僉諸民,以給使於太常、光祿;年深者,得選充王府典膳。凡歲藏冰、出冰,移所司謹潔之。

初,洪武元年置禮部。六年,設尚書二人,侍郎二人。分四屬部:總部,祠部,膳部,主客部。每部設郎中、員外郎各一人,主事各三人。十三年,升部秩,設尚書、侍郎各一人,每屬部設郎中、員外郎、主事各一人。尋復增置侍郎一人。二十二年,改總部為儀部。二十九年,改儀部、祠部、膳部為儀制、祠祭、精膳,惟主客仍舊,俱稱為清吏司。

按周宗伯之職雖掌邦禮,而司徒既掌邦教,所謂禮者,僅鬼神祠祀而已。至合典樂典教,內而宗籓,外而諸蕃,上自天官,下逮醫師、膳夫、伶人之屬,靡不兼綜,則自明始也。成、弘以後,率以翰林儒臣為之。其由此登公孤任輔導者,蓋冠於諸部焉。

兵部。尚書一人,正二品左、右侍郎各一人。正三品其屬,司務廳,司務二人,從九品武選、職方、車駕、武庫四清吏司,各郎中一人,正五品。正統十年,增設武選、職方二司郎中各一人。成化三年,增設車駕司郎中一人。萬曆九年並革。員外郎一人,從五品。正統十年增設武選司員外郎一人。弘治九年增設武庫司員外郎一人。後俱革。嘉靖十二年,增設職方司員外郎一人。主事二人,正六品。洪武、宣德間,增設武選司主事三人,職方司主事四人。正統十四年,增設車駕、武庫二司主事各一人。後革。萬歷十一年,又增設車駕司主事一人。所轄,會同館大使一人,正九品副使二人,從九品大通關大使、副使各一人,俱未入流。

尚書,掌天下武衛官軍選授、簡練之政令。侍郎佐之。

武選,掌衛所土官選授、升調、襲替、功賞之事。凡武官六品,其勳十有二。正一品,左、右柱國。從一品,柱國。正二品,上護軍。從二品,護軍。正三品,上輕車都尉。從三品,輕車都尉。正四品,上騎都尉。從四品,騎都尉。正五品,驍騎尉。從五品,飛騎尉。正六品,雲騎尉。從六品,武騎尉。散階三十。正一品,初授特進榮祿大夫,升授特進光祿大夫。從一品,初授榮祿大夫,升授光祿大夫。正二品,初授驃騎將軍,升授金吾將軍,加授龍虎將軍。從二品,初授鎮國將軍,升授定國將軍,加授奉國將軍。正三品,初授昭勇將軍,升授昭毅將軍,加授昭武將軍。從三品,初授懷遠將軍,升授定遠將軍,加授安遠將軍。正四品,初授明威將軍,升授宣威將軍,加授廣威將軍。從四品,初授宣武將軍,升授顯武將軍,加授信武將軍。正五品,初授武德將軍,升授武節將軍。從五品,初授武略將軍,升授武毅將軍。正六品,初授昭信校尉,升授承信校尉。從六品,初授忠顯校尉,升授忠武校尉。歲凡六選。有世官,有流官。世官九等,指揮使,指揮同知,指揮僉事,衛鎮撫,正千戶,副千戶,百戶,試百戶,所鎮撫。皆有襲職,有替職。其幼也,有優給。其不得世也,有減革,有通革。流官八等,左右都督,都督同知,都督僉事,都指揮使,都指揮同知。都指揮僉事,正留守,副留守。以世官升授,或由武舉用之,皆不得世。即有世者,出特恩。非真授者曰署職,署職,遞加本職一級作半級,不支俸,非軍功,毋得實授。曰試職,試職作一級,支半俸,不給誥。曰納職,納職帶俸,不蒞事。戰功二等:奇功為上,頭功次之。首功四等:迤北為大,遼東次之,西番、苗蠻又次之,內地反寇又次之。凡比試,有舊官,洪武三十一年以前為舊。有新官,成祖以後為新。軍政,五年一考選,先期撫、按官上功過狀,覆核而去留之。五府、錦衣衛堂上各總兵官,皆自陳,取上裁。推舉上二人,都指揮以下上一人。凡土司之官九級,自從三品至從七品,皆無歲祿。其子弟、族屬、妻女、若婿及甥之襲替,胥從其俗。附塞之官,自都督至鎮撫,凡十四等,皆以誥敕辨其偽冒。贈官死於王事,加二等;死於戰陣,加三等。凡除授出自中旨者,必覆奏然後行之。以貼黃徵圖狀,以初績徵誥敕,以效功課將領,以比試練卒徒,以優養恩故絕,以褒恤勵死戰,以寄祿馭恩倖,以殺降、失陷、避敵、激叛之法肅軍機,以典刑、敗倫、行劫、退陣之科斷世祿。

職方,掌輿圖、軍制、城隍、鎮戍、簡練、征討之事。凡天下地里險易遠近,邊腹疆界,俱有圖本,三歲一報,與官軍車騎之數偕上。凡軍制內外相維,武官不得輒下符徵發。自都督府,都指揮司,留守司,內外衛守禦、屯田、群牧千戶所,儀衛司,土司,諸番都司衛所,各統其官軍及其部落,以聽徵調、守衛、朝貢、保塞之令。以時修浚其城池而閱視之。凡鎮戍將校五等:曰鎮守,曰協守,曰分守,曰守備,曰備倭。皆因事增置,視地險要,設兵屯戍之。凡京營操練,統以文武大臣,皆科道官巡視之。若將軍營練,將軍四衛營練,及勇士、幼官、舍人等營練,則討其軍實,稽其什伍,察其存逸閒否,以教其坐作、進退、疾徐、疏數之節,金鼓、麾旗之號。征討請命將出師,懸賞罰,調兵食,紀功過,以黜陟之。以堡塞障邊徼,以烽火傳聲息,以關津詰姦細,以緝捕弭盜賊,以快壯簡鄉民,以勾解、收充、抽選、併豁、疏放、存恤之法整軍伍。

車駕,掌鹵簿、儀仗、禁衛、驛傳、廄牧之事。凡鹵簿大駕,大典禮、大朝會設之;丹陛駕,常朝設之;武陳駕,世宗南巡時設之。皆辨其物數,以授所司。慈宮、中宮之鹵簿,東宮、宗籓之儀仗,亦如之。凡侍衛,御殿全直,常朝番直,守衛、親軍衛,畫前、後、左、右四門為四行,而日夜巡警之。守衛皇城,前午門為一行,後玄武門為一行,左東華門為一行,右西華門為一行。凡郵傳,在京師曰會同館,在外曰驛,曰遞運所,皆以符驗關券行之。凡馬政,其專理者,太僕、苑馬二寺,稽其簿籍,以時程其登耗,惟內廄不會。

武庫,掌戎器、符勘、尺籍、武學、薪隸之事。凡內外官軍有征行,移工部給器仗,籍紀其數,制敕下各邊徵發。及使人出關,必驗勘合。軍伍缺,下諸省府州縣勾之。以跟捕、紀錄、開戶、給除、停勾之法,核其召募、垛集、罪謫、改調營丁尺籍之數。凡武職幼官,及子弟未嗣官者,於武學習業,以主事一人監督之。考稽學官之賢否、肄習之勤怠以聞。諸司官署供應有柴薪,直衙有皁隸,視官品為差。

初,洪武元年置兵部。六年,增尚書一人,侍郎一人。置總部、駕部並職方三部,設郎中、員外郎、主事,如吏部之數。十三年,升部秩,設尚書、侍郎各一人,又增置庫部為四屬部,部設郎中、員外郎、主事各一人。十四年,增試侍郎一人。二十二年改總部為司馬部。二十九年,定改四部為武選、職方、車駕、武庫四清吏司。惟職方仍舊名。景泰中,增設尚書一人,協理部事,天順初罷。隆慶四年添註侍郎二人,尋罷。萬曆末年復置。

協理京營戎政一人,或尚書,或侍郎,或右都御史。掌京營操練之事。永樂初,設三大營,總於武將。景泰元年始設提督團營,命兵部尚書于謙兼領之,後罷。成化三年復設,率以本部尚書或都御史兼之。嘉靖二十年,始命尚書劉天和輟部務,另給關防,專理戎政。二十九年,以「總督京營戎政」之印畀仇鸞,而改設本部侍郎協理戎政,不給關防。萬曆九年裁革,十一年復設。天啟初,增設協理一人,尋革。崇禎二年復增一人,以庶吉士劉之綸為兵部侍郎充之。

刑部。尚書一人,正二品左、右侍郎各一人。正三品其屬,司務廳,司務二人。從九品浙江、江西、湖廣、陜西、廣東、山東、福建、河南、山西、四川、廣西、貴州、雲南十三清吏司,各郎中一人,正五品員外郎一人,從五品主事二人。正六品。正統六年,十三司俱增設主事一人。成化元年增設四川、廣西二司主事各一人,後革。萬曆中,又革湖廣、陜西、山東、福建四司主事各一人。照磨所,照磨,正八品檢校,正九品各一人。司獄司,司獄六人,從九品。

尚書,掌天下刑名及徒隸、勾覆、關禁之政令。侍郎佐之。

十三司,各掌其分省及兼領所分京府、直隸之刑名。浙江司帶管崇府、中軍都督府、刑科、內官、御用、司設等監,在京金吾前、騰驤左、沈陽右、留守中、神策、和陽、武功右、廣洋八衛,蕃牧千戶所,及兩浙鹽運司,直隸和州,涿鹿左、涿鹿中二衛。江西司帶管淮、益、弋陽、建安、樂安五府,前軍都督府,御馬監,火藥、酒醋、面觔等局,在京府軍前、燕山左、留守前、龍驤、寬河、忠義前、忠義後、永清右、龍江左、龍江右十衛,及直隸廬州府,廬州、六安、九江、武清、宣府前、龍門各衛。湖廣司帶管楚、岷、吉、榮、遼五府,右軍都督府,司禮、尚賓、尚膳、神宮等監,天財庫,在京留守右、虎賁右、忠義右、武功左、茂陵、永陵、江淮、濟川、水軍右九衛,及興都留守司,直隸寧國、池州二府,宣州、神武中、定州、茂山、保安左、保安右各衛,渤海千戶所。福建司帶管戶部、太僕寺、戶科、寶鈔提舉司、印綬、都知等監,甲字第十庫,在京金吾後、應天、會州、武成中、武功中、孝陵、獻陵、景陵、裕陵、泰陵十衛,牧馬千戶所,及福建鹽運司,直隸常州府、廣德州,中都留守左、留守中、定邊、開平中屯各衛,美峪千戶所。山東司帶管魯、德、衡、涇四府,左軍都督府,宗人府,兵部,尚寶司,兵科,典牧所,會同館,供用庫,戈戟司,司苑局,在京羽林右、沈陽左、長陵三衛,奠靖千戶所,及山東鹽運司,中都留守司,遼東都司,遼東行太僕寺,直隸鳳陽府,滁州、鳳陽、皇陵、長淮、泗州、壽州、滁州、沂州、德州、德州左、保定後各衛,安東中護衛,潮河、龍門、寧靖各千戶所。山西司帶管晉、代、沈、懷仁、慶成五府,翰林院,欽天監,上林苑監,南、北二城兵馬司,混堂司,甜食房,在京旗手、金吾右、驍騎右、龍虎、大寧中、義勇前、義勇後、英武八衛,及直隸鎮江府、徐州,鎮江、徐州、沈陽中屯各衛,沈陽中護衛,倒馬關、平定各千戶所。河南司帶管周、唐、趙、鄭、徽、伊、汝七府,禮部,太常寺,光祿寺,鴻臚寺,詹事府,國子監,禮科,中書舍人,神樂觀,犧牲所,兵仗局,靈臺、鐘鼓等司,東城兵馬司,教坊司,在京羽林左、府軍右、武德、留守後、神武左、彭城六衛,及兩淮鹽運司,直隸淮安、揚州二府,淮安、大河、邳州、揚州、高郵、儀真、宿州、武平、歸德、寧山、神武右各衛,海州、鹽城、通州、汝寧各千戶所。陜西司帶管秦、韓、慶、肅四府,後軍都督府,大理寺,行人司,尚衣監,針工局,西城兵馬司,在京府軍後、騰驤右、豹韜、鷹揚、興武、義勇右、康陵、昭陵、龍虎左、橫海、江陰十一衛,及河東鹽運司,陜西行太僕寺,甘肅行太僕寺,直隸太平府,建陽、保定左、保定右、保定中、保定前各衛,平涼中護衛。四川司帶管蜀府,工部,工科,巾帽、織染二局,僧道錄司,在京府軍、金吾左、濟川、武驤右、大寧前、蔚州左、永清左、廣武八衛,及直隸松江、大名二府,金山、懷安、懷來各衛,神木千戶所。廣東司帶管應天府,在京錦衣、府軍左、虎賁左、濟陽、留守左、水軍左、飛熊七衛,及直隸延慶州,懷來千戶所。廣西司帶管靖江府,通政司,五軍斷事司,中城兵馬司,寶鈔、銀作二局,在京羽林前、燕山右、燕山前、大興左、通州、武驤左、鎮南、富峪八衛,及直隸安慶、徽州二府,安慶、新安、通州左、通州右、延慶、延慶左、延慶右各衛。雲南司帶管順天府,太醫院,儀衛、惜薪等司,承運庫,及直隸永平、廣平二府,鎮海、真定、永平、山海、盧龍、東勝左、東勝右、撫寧、密雲中、密雲后、大同中屯、潼關、營州五屯、萬全左、萬全右各衛,寬河、武定、蒲州各千戶所。貴州司帶管吏部,吏科,司菜局,及長蘆鹽運司,大寧都司,萬全都司,直隸蘇州、保定、河間、真定、順德五府,蘇州、太倉、薊州、遵化、鎮朔、興州五屯,忠義中、涿鹿、河間、天津、天津左、天津右、德州、宣府左、宣府右、開平、保安、蔚州、永寧各衛,梁城、興和、廣昌各千戶所。

照磨、檢校,照刷文卷,計錄贓贖。司獄,率獄吏,典囚徒。凡軍民、官吏及宗室、勳戚麗於法者,詰其辭,察其情偽,傅律例而比議其罪之輕重以請。詔獄必據爰書,不得逢迎上意。凡有殊旨、別敕、詔例、榜例,非經請議著為令甲者,不得引比。凡死刑,即決及秋後決,並三覆奏。兩京、十三布政司,死罪囚歲讞平之。五歲請敕遣官,審錄冤滯。霜降錄重囚,會五府、九卿、科道官共錄之。矜疑者戍邊,有詞者調所司再問,比律者監候。夏月熱審,免笞刑,減徒、流,出輕繫。遇歲旱,特旨錄囚亦如之。凡大祭止刑。凡贖罪,視罪輕重,斬、絞、雜犯、徒末減者,聽收贖。詞訴必自下而上,有事重而迫者,許擊登聞鼓。四方有大獄,則受命往鞫之。四方決囚,遣司官二人往蒞。凡斷獄,歲疏其名數以聞,曰歲報;月上其拘釋存亡之數,曰月報。獄成,移大理寺覆審,必期平允。凡提牢,月更主事一人,修葺囹圄,嚴固扃鑰,省其酷濫,給其衣糧。囚病,許家人入視,脫械鎖醫藥之。簿錄俘囚,配沒官私奴婢,咸籍知之。官吏有過,並紀錄之。歲終請湔滌之。以名例攝科條,以八字括辭議,以、准、皆、各、其、及、即、若,以五服參情法,以墨涅識盜賊。籍產不入瑩墓,籍財不入度支,宗人不即市,宮人不即獄,悼耄疲癃不即訊。詳《刑法志》。

洪武元年置刑部。六年,增尚書、侍郎各一人。設總部、比部、都官部、司門部,部設郎中、員外郎各二人,惟都官各一人。總部、比部主事各六人,都官、司門主事各四人。八年,以部事浩繁,增設四科,科設尚書、侍郎、郎中各一人,員外郎二人,主事五人。十三年,升部秩,設尚書一人,侍郎一人,仍分四屬部,部設郎中、員外郎各一人,總部、比部主事各四人,都官、司門主事各二人,尋增侍郎一人。始分左、右侍郎。二十二年,改總部為憲部。二十三年,分四部為河南、北平、山東、山西、陜西、浙江、江西、湖廣、廣東、廣西、四川、福建十二部,浙江部兼領雲南。部各設官,如戶部之制。二十九年,改為十二清吏司。永樂元年以北平為北京。十八年,革北京司,增置雲南、貴州、交阯三司。宣德十年,革交阯司,遂定為十三清吏司。

工部。尚書一人,正二品左、右侍郎各一人。正三品其屬,司務廳,司務二人。從九品營繕、虞衡、都水、屯田四清吏司,各郎中一人,正五品,後增設都水司郎中四人。員外郎一人,從五品,後增設營膳司員外郎二人,虞衡司員外郎一人。主事二人。正六品,後增設都水司主事五人,營膳司主事三人,虞衡司主事二人,屯田司主事一人。所轄,營繕所,所正一人,正七品所副二人,正八品所丞二人。正九品文思院,大使一人,正九品副使二人。從九品皮作局,大使一人,正九品副使二人。從九品,後革。鞍轡局,大使一人,正九品副使一人。從九品。隆慶元年,大使、副使俱革。寶源局,大使一人,正九品副使一人。從九品,嘉靖間革。顏料局,大使一人,正九品,後革。軍器局,大使一人,正九品副使二人,後革一人。節慎庫,大使一人,從九品。嘉靖八年設。織染所、雜造局,大使一人,正九品副使一人。從九品廣積、通積、盧溝橋、通州、白河各抽分竹木局,大使各一人,副使各一人。大通關提舉司,提舉一人,正八品,萬歷二年革。副提舉二人,正九品典史一人。後副提舉、典史俱革。柴炭司,大使一人,從九品副使一人。

尚書,掌天下百官、山澤之政令。侍郎佐之。

營繕,典經營興作之事。凡宮殿、陵寢、城郭、壇場、祠廟、倉庫、廨宇、營房、王府邸第之役,鳩工會材,以時程督之。凡鹵簿、儀仗、樂器,移內府及所司,各以其職治之,而以時省其堅潔,而董其窳濫。凡置獄具,必如律。凡工匠二等:曰輪班,三歲一役,役不過三月,皆復其家;曰住坐,月役一旬,有稍食。工役二等,以處罪人輸作者,曰正工,曰雜工。雜工三日當正工一日,皆視役大小而撥節之。凡物料儲偫,曰神木廠,曰大木廠,以蓄材木,曰黑窯廠,曰琉璃廠,以陶瓦器,曰臺基廠,以貯薪葦,皆籍其數以供修作之用。

虞衡,典山澤採捕、陶冶之事。凡鳥獸之肉、皮革、骨角、羽毛,可以供祭祀、賓客、膳羞之需,禮器、軍實之用,歲下諸司採捕。水課禽十八、獸十二,陸課獸十八、禽十二,皆以其時。冬春之交,罝罛不施川澤;春夏之交,毒藥不施原野。苗盛禁蹂躪,穀登禁焚燎。若害獸,聽為陷阱獲之,賞有差。凡諸陵山麓,不得入斧斤、開窯冶、置墓墳。凡帝王、聖賢、忠義、名山、岳鎮、陵墓、祠廟有功德於民者,禁樵牧。凡山場、園林之利,聽民取而薄征之。凡軍裝、兵械,下所司造,同兵部省之,必程其堅致。凡陶甄之事,有歲供,有暫供,有停減,籍其數,會其入,毋輕毀以費民。凡諸冶,飭其材,審其模範,付有司。錢必準銖兩,進於內府而頒之。牌符、火器,鑄於內府,禁其以法式洩於外。凡顏料,非其土產不以征。

都水,典川澤、陂池、橋道、舟車、織造、券契、量衡之事。水利曰轉漕,曰灌田。歲儲其金石、竹木、卷埽,以時修其閘壩、洪淺、堰圩、隄防,謹蓄洩以備旱潦,無使壞田廬、墳隧、禾稼。舟楫、磑碾者不得與灌田爭利,灌田者不得與轉漕爭利。凡諸水要會,遣京朝官專理,以督有司。役民必以農隙,不能至農隙,則僝功成之。凡道路、津梁,時其葺治。有巡幸及大喪、大禮,則修除而較比之。凡舟車之制,曰黃船,以供御用,曰遮洋船,以轉漕於海,曰淺船,以轉漕於河,曰馬船、曰風快船,以供送官物,曰備倭船、曰戰船,以禦寇賊,曰大車,曰獨轅車,曰戰車,皆會其財用,酌其多寡、久近、勞逸而均劑之。凡織造冕服、誥敕、制帛、祭服、凈衣諸幣布,移內府、南京、浙江諸處,周知其數而慎節之。凡公、侯、伯鐵券,差其高廣。制式詳《禮志》。凡祭器、冊寶、乘輿、符牌、雜器皆會則於內府。凡度量、權衡,謹其校勘而頒之,懸式於市,而罪其不中度者。

屯田,典屯種、抽分、薪炭、夫役、墳塋之事。凡軍馬守鎮之處,其有轉運不給,則設屯以益軍儲。其規辦營造、木植、城磚、軍營、官屋及戰衣、器械、耕牛、農具之屬。凡抽分徵諸商,視其財物各有差。凡薪炭,南取洲汀,北取山麓,或徵諸民,有本、折色,酌其多寡而撙節之。夫役伐薪、轉薪,皆雇役。凡墳塋及堂碑、碣獸之制,第宗室、勳戚、文武官之等而定其差。墳塋制度,詳《禮志》。

洪武初,置工部及官屬,以將作司隸焉。吳元年置將作司,卿,正三品,少卿,正四品,丞,正五品。左、右提舉司提舉,正六品,同提舉,從六品,司程、典簿、副提舉,正七品。軍需庫大使,從八品,副使,正九品。洪武元年,以將作司隸工部。六年,增尚書、侍郎各一人,設總部、虞部、水部並屯田為四屬部。總部設郎中、員外郎各二人,餘各一人。總部主事八人,餘各四人。又置營造提舉司。洪武六年,改將作司為正六品,所屬提舉司,改正七品。尋更置營造提舉司及營造提舉分司,每司設正提舉一人,副提舉二人,隸將作司。八年,增立四科,科設尚書、侍郎、郎中各一人,員外郎二人,主事五人,照磨二人。十年,罷將作司。十三年定官制,設尚書一人,侍郎一人,四屬部,以屯田部為屯部,各郎中、員外郎一人,主事二人。十五年增侍郎一人。二十二年,改總部為營部。二十五年,置營繕所。改將作司為營繕所,秩正七品,設所正、所副、所丞各二人,以諸匠之精藝者為之。二十九年,又改四屬部為營繕、虞衡、都水、屯田四清吏司。嘉靖後添設尚書一人,專督大工。

提督易州山廠一人,掌督御用柴炭之事。明初,於沿江蘆洲並龍江、瓦屑二場,取用柴炭。永樂間,遷都於北,則於白羊口、黃花鎮、紅螺山等處採辦。宣德四年始設易州山廠,專官總理。景泰間,移於平山,又移於滿城,相繼以本部尚書或侍郎督廠事。天順元年仍移於易州。嘉靖八年罷革,改設主事管理。


都察院附總督巡撫通政司大理寺詹事府附左右春坊司經局翰林院國子監衍聖公附五經博士

都察院。左、右都御史,正二品左、右副都御史,正三品左、右僉都御史,正四品其屬,經歷司,經歷一人,正六品都事一人。正七品司務廳,司務二人,從九品。初設四人,後革二人。照磨所,照磨,正八品檢校,正九品司獄司,司獄,從九品。初設六人,後革五人。各一人。十三道監察御史一百十人,正七品浙江、江西、河南、山東各十人,福建、廣東、廣西、四川、貴州各七人,陜西、湖廣、山西各八人,雲南十一人。其在外加都御史或副、僉都御史銜者,有總督,有提督,有巡撫,有總督兼巡撫,提督兼巡撫,及經略、總理、贊理、巡視、撫治等員。巡撫之名,起於懿文太子巡撫陜西。永樂十九年,遣尚書蹇義等二十六人巡行天下,安撫軍民。以後不拘尚書、侍郎、都御史、少卿等官。事畢復命,即或停遣。初名巡撫,或名鎮守,後以鎮守侍郎與巡按御史不相統屬,文移窒礙,定為都御史。巡撫兼軍務者加提督,有總兵地方加贊理或參贊,所轄多、事重者加總督。他如整飭、撫治、巡治、總理等項,皆因事特設。其以尚書、侍郎任總督軍務者,皆兼都御史,以便行事。

都御史,職專糾劾百司,辯明冤枉,提督各道,為天子耳目風紀之司。凡大臣姦邪、小人構黨、作威福亂政者,劾。凡百官猥茸貪冒壞官紀者,劾。凡學術不正、上書陳言變亂成憲、希進用者,劾。遇朝覲、考察,同吏部司賢否陟黜。大獄重囚會鞫於外朝,偕刑部、大理讞平之。其奉敕內地,拊循外地,各專其敕行事。

十三道監察御史,主察糾內外百司之官邪,或露章面劾,或封章奏劾。在內兩京刷卷,巡視京營,監臨鄉、會試及武舉,巡視光祿,巡視倉場,巡視內庫、皇城、五城,輪值登聞鼓。後改科員。在外巡按,北直隸二人,南直隸三人,宣大一人,遼東一人,甘肅一人,十三省各一人。清軍,提督學校,兩京各一人,萬曆末,南京增設一人。巡鹽,兩淮一人,兩浙一人,長蘆一人,河東一人。茶馬,陜西。巡漕,巡關,宣德四年設立鈔關御史,至正統十年始遣主事。攢運,印馬,屯田。師行則監軍紀功,各以其事專監察。而巡按則代天子巡狩,所按籓服大臣、府州縣官諸考察,舉劾尤專,大事奏裁,小事立斷。按臨所至,必先審錄罪囚,弔刷案卷,有故出入者理辯之。諸祭祀壇場,省其牆宇祭器。存恤孤老,巡視倉庫,查算錢糧,勉勵學校,表揚善類,翦除豪蠹,以正風俗,振綱紀。凡朝會糾儀,祭祀監禮。凡政事得失,軍民利病,皆得直言無避。有大政,集闕廷預議焉。蓋六部至重,然有專司,而都察院總憲綱,惟所見聞得糾察。諸御史糾劾,務明著實跡,開寫年月,毋虛文泛詆,訐拾細瑣。出按復命,都御史覆劾其稱職不稱職以聞。凡御史犯罪,加三等,有贓從重論。

十三道各協管兩京、直隸衙門;而都察院衙門分屬河南道,獨專諸內外考察。浙江道協管中軍都督府,在京府軍左、金吾左、金吾右、金吾前、留守中、神策、應天、和陽、廣洋、武功中、武功後、茂陵十二衛,牧馬千戶所,及直隸廬州府,廬州、六安二衛。江西道協管前軍都督府,在京府軍前、燕山左、龍江左、龍江右、龍驤、豹韜、天策、寬河八衛,及直隸淮安府,淮安、大河、邳州、九江、武清、龍門各衛。福建道協管戶部,寶鈔提舉司,鈔紙、印鈔二局,承運、廣惠、廣積、廣盈、贓罰、甲乙丙丁戊字、天財、軍儲、供用、行用各庫,在京金吾後、武成中、飛熊、武功左、武功右、武功前、獻陵、景陵、裕陵、泰陵十衛,及直隸常州、池州二府,定邊、開平中屯二衛,美峪千戶所。四川道協管工部,營繕所,文思院,御用、司設、神宮、尚衣、都知等監,惜薪司,兵仗、銀作、巾帽、針工、器皿、盔甲、軍器、寶源、皮作、鞍轡、織染、柴炭、抽分竹木各局,僧、道錄司,在京府軍、濟州、大寧前、蔚州左、永清左五衛,蕃牧千戶所,及直隸松江府、廣德州,金山、懷安、懷來各衛,神木千戶所,播州宣慰司,石砫、西陽等宣撫司,天全六番招討司。陜西道協管後軍都督府,大理寺,行人司,在京府軍後、鷹揚、興武、義勇右、橫海、江陰、康陵、昭陵八衛,敢勇、報效二營,韓、秦、慶、安化四府,及直隸和州,保定左、右、中、前四衛。雲南道協管順天府,廣備庫,在京羽林前、通州二衛,及直隸永平、廣平二府,通州左、通州右、涿鹿、涿鹿左、涿鹿中、密雲中、密雲後、永平、山海、盧龍、撫寧、東勝左、東勝右、大同中屯、營州五屯、延慶、延慶左、延慶右、萬全左、萬全右各衛,居庸關、黃花鎮、寬河、武定各千戶所。河南道協管禮部,都察院,翰林院,國子監,太常寺,光祿寺,鴻臚寺,尚寶司,中書舍人,欽天監,太醫院,司禮、尚膳、尚寶、直殿等監,酒醋面局,鐘鼓司,教坊司,在京羽林左、留守前、留守後、神武左、神武前、彭城六衛,伊、唐、周、鄭四府,及兩淮鹽運司,直隸揚州、大名二府,揚州、高郵、儀真、歸德、寧山、潼關、神武右各衛,泰州、通州、汝寧各千戶所。廣西道協管通政司,六科,在京燕山右、燕山前、大興左、騰驤左、騰驤右、武驤左、鎮南、沈陽左、會州、富峪、忠義前、忠義後十二衛,及直隸安慶、徽州、保定、真定四府,安慶、新安、鎮武、真定各衛,紫荊關、倒馬關、廣昌各千戶所。廣東道協管刑部,應天府,在京虎賁左、濟陽、武驤右、沈陽右、武功左、武功右、孝陵、長陵八衛,及直隸延慶州,開平中屯衛。山西道協管左軍都督府,在京錦衣、府軍右、留守左、驍騎左、驍騎右、龍虎、龍虎左、大寧中、義勇前、義勇後、英武、水軍左十二衛,晉府長史司,及直隸鎮江、太平二府,鎮江、建陽、沈陽中屯各衛,平定、蒲州二千戶所。山東道協管宗人府,兵部,會同館,御馬監,典牧所,大通關,在京羽林右、永清右、濟川三衛,及中都留守司,遼東都司,直隸鳳陽府,徐、滁二州,中都留守左、留守中、鳳陽、鳳陽中、鳳陽右、皇陵、長淮、懷遠、徐州、滁州、泗州、壽州、宿州、武平、沂州、德州、德州左、保定後、沈陽中各衛,洪塘千戶所。湖廣道協管右軍都督府,五城兵馬司,在京留守右、武德、忠義右、虎賁右、廣武、水軍右、江淮、永陵八衛,遼、梁、岷、吉、華陽五府,荊、襄、楚三府長史司,及興都留守司,直隸寧國府,寧國、宣州、神武中、定州、茂山各衛。貴州道協管吏部,太僕寺,上林苑監,內官、印綬二監,在京旗手衛,及長蘆鹽運司,大寧都司,萬全都司,直隸蘇州、河間、順德三府,保安州、蘇州、太倉、鎮海、薊州、遵化、鎮朔、興州五屯,忠義中、河間、天津、天津左、天津右、宣府前、宣府左、宣府右、開平、保安右、蔚州、永寧各衛,嘉興、吳淞江、梁城、滄州、興和、長安、龍門各千戶所。

初,吳元年置御史臺,設左、右御史大夫,從一品御史中丞,正二品侍御史,從二品治書侍御史,正三品殿中侍御史,正五品察院監察御史,正七品經歷,從五品都事,正七品照磨、管勾。正八品以鄧愈、湯和為御史大夫,劉基、章溢為御史中丞,諭之曰:「國家立三大府,中書總政事,都督掌軍旅,御史掌糾察。朝廷紀綱盡繫於此,而臺察之任尤清要。卿等當正己以率下,忠勤以事上,毋委靡因循以縱姦,毋假公濟私以害物。」洪武九年汰侍御史及治書、殿中侍御史。十年七月,詔遣監察御史巡按州縣。十三年,專設左,右中丞,正二品左、右侍御史。正四品尋罷御史臺。十五年更置都察院,設監察都御史八人,秩正七品。分監察御史為浙江、河南、山東、北平、山西、陜西、湖廣、福建、江西、廣東、廣西、四川十二道,各道置御史或五人或三、四人,秩正九品。每道鑄印二,一畀御史久次者掌之,一藏內府,有事受印以出,既事納之,文曰「繩愆糾繆」。以秀才李原名、詹徽等為都御史,吳荃等為試監察御史。試御史,一年後實授。又有理刑進士、理刑知縣,理都察院刑獄,半年實授。正德中革。十六年,陞都察院為正三品,設左、右都御史各一人,正三品,左、右副都御史各一人,正四品,左、右僉都御史各二人,正五品,經歷一人,正七品,知事一人,正八品。十七年,升都御史正二品,副都御史正三品,僉都御史正四品,十二道監察御史正七品。二十三年,左副都御史袁泰言:「各道印篆相同,慮有詐偽。」乃更鑄監察御史印曰「某道監察御史印」,其巡按印曰「巡按某處監察御史印」。建文元年,改設都御史一人,革僉都御史。二年,改為御史府,設御史大夫,改十二道為左、右兩院,止設御史二十八人。成祖復舊制。永樂元年,改北平道為北京道。十八年,罷北京道,增設貴州、雲南、交阯三道。洪熙元年,稱行在都察院,同六部,又定巡按以八月出巡。宣德十年,罷交阯道,始定為十三道。正統中,去「行在」字。嘉靖中,以清屯,增副都御史三人,尋罷。隆慶中,以提督京營,增右都御史三人,尋亦罷。

總督漕運兼提督軍務巡撫鳳陽等處兼管河道一員。太祖時,嘗置京畿都漕運司,設漕運使。洪武元年置漕運使,正四品,知事,正八品,提控案牘,從九品,屬官監運,正九品,都綱,省注。十四年罷。永樂間,設漕運總兵官,以平江伯陳瑄治漕。宣德中,又遣侍郎、都御史、少卿等官督運。至景泰二年,因漕運不斷,始命副都御史王竑總督,因兼巡撫淮、揚、廬、鳳四府,徐、和、滁三州,治淮安。成化八年,分設巡撫、總漕各一員。九年復舊。正德十三年又分設。十六年又復舊。嘉靖三十六年,以倭警,添設提督軍務巡撫鳳陽都御史。四十年歸併,改總督漕運兼提督軍務。萬曆七年加兼管河道。

總督薊遼、保定等處軍務兼理糧餉一員。嘉靖二十九年置。先是,薊、遼有警,間遣重臣巡視,或稱提督。至是以邊患益甚,始置總督,開府密雲,轄順天、保定、遼東三巡撫,兼理糧餉。萬曆九年加兼巡撫順天等處。十一年復舊。天啟元年,置遼東經略。經略之名,起於萬曆二十年宋應昌暨後楊鎬。至天啟元年,又以內閣孫承宗督師經略山海關,稱樞輔。崇禎四年併入總督。十一年又增設總督於保定。

總督宣大、山西等處軍務兼理糧餉一員。正統元年,始遣僉都御史巡撫宣大。景泰二年,宣府、大同各設巡撫,遣尚書石璞總理軍務。成化、弘治間,有警則遣。正德八年設總制。嘉靖初,兼轄偏、保。二十九年,去偏、保,定設總督宣大、山西等處銜。三十八年令防秋日駐宣府。四十三年,移駐懷來。隆慶四年,移駐陽和。

總督陜西三邊軍務一員。弘治十年,火篩入寇,議遣重臣總督陜西、甘肅、延綏、寧夏軍務,乃起左都御史王越任之。十五年以後,或設或罷。至嘉靖四年,始定設,初稱提督軍務。七年改為總制。十九年避制字,改為總督,開府固原,防秋駐花馬池。

總督兩廣軍務兼理糧餉帶管鹽法兼巡撫廣東地方一員。永樂二年,遣給事中雷填巡撫廣西。十九年,遣郭瑄、艾廣巡撫廣東。景泰三年,苗寇起,以兩廣宜協濟應援,乃設總督。成化元年,兼巡撫事,駐梧州。正德十四年,改總督為總制,尋改提督。嘉靖四十五年,另設廣東巡撫,改提督為總督,止兼巡撫廣西,駐肇慶。隆慶三年,又設廣西巡撫,除兼職。四年,革廣東巡撫,改為提督兩廣軍務兼理糧餉,巡撫廣東。萬曆三年,仍改總督,加帶管鹽法。

總督四川、陜西、河南、湖廣等處軍務一員。正德五年設,尋罷。嘉靖二十七年,以苗患,又設總督四川、湖廣、貴州、雲南等處軍務。四十二年罷。天啟元年,以土官奢崇明反,又設四川、湖廣、雲南、貴州、廣西五省總督。四年,兼巡撫貴州。

總督浙江、福建、江南兼制江西軍務一員。嘉靖三十三年,以倭犯杭州置。四十一年革。

總督陜西、山西、河南、湖廣、四川五省軍務一員。崇禎七年置,或兼七省。十二年後,俱以內閣督師。

總督鳳陽地方兼制河南、湖廣軍務一員。崇禎十四年設。

總督保定地方軍務一員。崇禎十一年設。

總督河南、湖廣軍務兼巡撫河南一員。崇禎十六年設。

總督九江地方兼制江西、湖廣軍務一員。崇禎十六年設。

總理南直隸、河南、山東、湖廣、四川軍務一員。崇禎八年設,以盧象昇為之,與總督或分或併。

總理河漕兼提督軍務一員。永樂九年遣尚書治河,自後間遣侍郎、都御史。成化後,始稱總督河道。正德四年,定設都御史。嘉靖二十年,以都御史加工部職銜,提督河南、山東、直隸河道。隆慶四年,加提督軍務。萬曆五年,改總理河漕兼提督軍務。八年革。

總理糧儲提督軍務兼巡撫應天等府一員。宣德五年,初命侍郎總督糧儲兼巡撫。景泰四年,定遣都御史。嘉靖三十三年,以海警,加提督軍務,駐蘇州。萬曆中,移駐句容,已復駐蘇州。

巡撫浙江等處地方兼提督軍務一員。永樂初,遣尚書治兩浙農事。以後或巡視或督鹺,有事則遣。嘉靖二十六年,以海警,始命都御史巡撫浙江,兼管福建福、興、建寧、漳、泉海道地方,提督軍務。二十七年,改巡撫為巡視。二十八年罷。三十一年復設。

巡撫福建地方兼提督軍務一員。嘉靖二十六年,既設浙江巡撫兼轄福、興、漳、泉等處,三十五年,以閩、浙道遠,又設提督軍務兼巡福、興、漳、泉、福寧海道都御史。後改巡撫福建,統轄全省。

巡撫順天等府地方兼整飭薊州等處邊備一員。成化二年,始專設都御史贊理軍務,巡撫順天、永平二府,尋兼撫河間、真定、保定,凡五府。七年,兼理八府。八年,以畿輔地廣,從居庸關中分,設二巡撫,其東為巡撫順天、永平二府,駐遵化。崇禎二年,又於永平分設巡撫兼提督山海軍務,其舊者止轄順天。

巡撫保定等府提督紫荊等關兼管河道一員。成化八年,分居庸關以西,另設巡撫保定、真定、河間、順德、大名、廣平六府,提督紫荊、倒馬、龍泉等關,駐真定。萬曆七年,兼管河道。

巡撫河南等處地方兼管河道提督軍務一員。宣德五年,遣兵部侍郎於謙巡撫山西、河南。正統十四年,以左副都御史王來巡撫湖廣、河南。景泰元年,始專設河南巡撫。萬曆七年,兼管河道。八年,加提督軍務。

巡撫山西地方兼提督雁門等關軍務一員。宣德五年,以侍郎巡撫河南、山西。正統十三年,始命都御史專撫山西,鎮守雁門。天順、成化間暫革,尋復置。

巡撫山東等處地方督理營田兼管河道提督軍務一員。正統五年始設巡撫。十三年,定遣都御史。嘉靖四十二年,加督理營田。萬歷七年,兼管河道。八年,加提督軍務。

巡撫遼東地方贊理軍務一員。正統元年設,舊駐遼陽,後地日蹙,移駐廣寧,駐山海關,後又駐寧遠。

巡撫宣府地方贊理軍務一員。正統元年,命都御史出巡塞北,因奏設巡撫兼理大同。景泰二年,另設大同巡撫,後復併為一。成化十年,復分設。十四年,加贊理軍務。

巡撫大同地方贊理軍務一員。初與宣府共一巡撫,後或分或併。成化十年,復專設,加贊理軍務。

巡撫延綏等處贊理軍務一員。宣德十年,遣都御史出鎮。景泰元年,專設巡撫加參贊軍務。成化九年,徙鎮榆林。隆慶六年,改贊理軍務。

巡撫寧夏地方贊理軍務一員。正統元年,以右僉都御史郭智鎮撫寧夏,參贊軍務。天順元年罷。二年復設,去參贊。隆慶六年,加贊理軍務。

巡撫甘肅等處贊理軍務一員。宣德十年,命侍郎鎮守。正統元年,甘、涼用兵,命侍郎參贊軍務。景泰元年,定設巡撫都御史。隆慶六年,改贊理軍務。

巡撫陜西地方贊理軍務一員。宣德初,遣尚書、侍郎出鎮。正統間,命右都御史陳鎰、王文等出入更代。景泰初,耿九疇以刑部侍郎出鎮,文移不得徑下按察司,特改都御史巡撫。成化二年,加提督軍務,後改贊理,駐西安,防秋駐固原。

巡撫四川等處地方兼提督軍務一員。宣德五年,命都御史鎮撫,後停遣。正統十四年,始設巡撫。萬曆十一年,加提督軍務。

巡撫湖廣等處地方兼贊理軍務一員。正統三年,命都御史賈諒鎮守,以後或侍郎或大理卿出撫。景泰元年定設巡撫都御史兼贊理軍務。萬曆八年,改為提督軍務。十二年,仍為贊理。

巡撫江西地方兼理軍務一員。永樂後,間設巡撫鎮守。成化以後,定為巡撫,或有時罷遣。嘉靖六年始定設。四十年加兼理軍務。

巡撫南贛汀韶等處地方提督軍務一員。弘治十年,始設巡撫。正德十一年,改提督軍務。嘉靖四十五年,定巡撫銜,所轄南安、贛州、南雄、韶州、汀州並郴州地方,駐贛州。

巡撫廣東地方兼贊理軍務一員。永樂中,設巡撫,後以總督兼巡撫事,遂罷不設。嘉靖四十五年,復另設巡撫,加贊理軍務。隆慶四年又罷。

巡撫廣西地方一員。廣西舊有巡撫,沿革不常。隆慶三年復專設。

巡撫雲南兼建昌、畢節等處地方贊理軍務兼督川、貴糧餉一員。正統九年,命侍郎參贊軍務。十年,設鎮撫。天順元年罷。成化十二年復設。嘉靖三十年,加兼理軍務。四十三年,改贊理。隆慶二年,兼撫建昌、畢節等處。

巡撫貴州兼督理湖北、川東等處地方提督軍務一員。正統十四年,以苗亂置總督,鎮守貴州、湖北、川東等處。景泰元年,另設貴州巡撫。成化八年罷。十一年復設。正德二年又罷。五年又復設。嘉靖四十二年,裁革總督,令巡撫兼理湖北、川東等處提督軍務。

巡撫天津地方贊理軍務一員。萬曆二十五年,以倭陷朝鮮暫設,尋為定制。

巡撫登萊地方贊理軍務一員。天啟元年設。崇禎二年罷。三年復設。

巡撫安廬地方贊理軍務一員。崇禎十年設,以史可法為之。十六年,又增設安、太、池、廬四府巡撫。

巡撫偏沅地方贊理軍務一員。萬曆二十七年,以征播暫設,尋罷。天啟二年後,或置或罷。崇禎二年定設。

巡撫密雲地方贊理軍務一員。崇禎十一年設。

巡撫淮揚地方贊理軍務一員。崇禎十一年設。

巡撫承天贊理軍務一員。崇禎十六年設。

撫治鄖陽等處地方兼提督軍務一員,成化十二年,以鄖、襄流民屢叛,遣都御史安撫,因奏設官撫治之。萬歷二年以撫治事權不專,添提督軍務兼撫治職銜。九年裁革,十一年復設。

贊理松潘地方軍務一員。正統四年,以王翱為之。

通政使司。通政使一人,正三品左、右通政各一人,謄黃右通政一人,正四品左、右參議各一人,正五品其屬,經歷司,經歷一人,正七品知事一人。正八品

通政使,掌受內外章疏敷奏封駁之事。凡四方陳情建言,申訴冤滯,或告不法等事,於底簿內謄寫訴告緣由,齎狀奏聞。凡天下臣民實封入遞,即於公廳啟視,節寫副本,然後奏聞。即五軍、六部、都察院等衙門,有事關機密重大者,其入奏仍用本司印信。凡諸司公文、勘合辨驗允當,編號注寫,公文用「日照之記」、勘合用「驗正之記」關防之。凡在外之題本、奏本,在京之奏本,並受之,於早朝匯而進之。有徑自封進者則參駁。午朝則引奏臣民之言事者,有機密則不時入奏。有違誤則籍而匯請。凡抄發、照駁諸司公移及勘合、訟牒、勾提件數、給繇人員,月終類奏,歲終通奏。凡議大政、大獄及會推文武大臣,必參預。

初,洪武三年置察言司,設司令二人,掌受四方章奏,尋罷。十年置通政使司,以曾秉正為通政使,劉仁為左通政,諭之曰:「政猶水也,欲其常通,故以『通政』名官。卿其審命令以正百司,達幽隱以通庶務。當執奏者勿忌避,當駁正者勿阿隨,當敷陳者毋隱蔽,當引見者毋留難。」十二年,撥承敕監給事中、殿廷儀禮司、九關通事使隸焉。建文中,改司為寺,通政使為通政卿,通政參議為少卿,寺丞增置左、右補闕,左、右拾遺各一人。成祖復舊制。成化二年,置提督謄黃右通政,不理司事,錄武官黃衛所襲替之故,以征贊事。萬曆九年革。

大理寺。卿一人,正三品左、右少卿各一人,正四品左、右寺丞各一人。正五品其屬,司務廳,司務二人。從九品。左、右二寺,各寺正一人,正六品寺副二人,從六品,後革右寺副一人。評事四人。正七品。初設右評事八人,後革四人。

卿,掌審讞平反刑獄之政令。少卿、寺丞贊之。左、右寺分理京畿、十三布政司刑名之事。凡刑部、都察院、五軍斷事官所推問獄訟,皆移案牘,引囚徒,詣寺詳讞。左、右寺寺正,各隨其所轄而覆審之。既按律例,必復問其款狀,情允罪服,始呈堂准擬具奏。不則駁令改擬,曰照駁。三擬不當,則糾問官,曰參駁。有牾律失入者,調他司再訊,曰番異。猶不愜,則請下九卿會訊,曰圓審。已評允而招由未明,移再訊,曰追駁。屢駁不合,則請旨發落,曰制決。凡獄既具,未經本寺評允,諸司毋得發遣。誤則糾之。

初,吳元年置大理司卿,秩正三品。洪武元年革。三年,置磨勘司,凡諸司刑名、錢糧,有冤濫隱匿者,稽其功過以聞。尋亦革。洪武三年置磨勘司,設司令、司丞。七年增設司令一人,司丞五人,首領官五人,分為四科。十年革。十四年復置磨勘司,設司令一人,左、右司丞各一人,左、右司副各一人。二十年復罷。十四年,復置大理寺,改卿秩正五品,左、右少卿從五品,左、右寺丞正六品。其屬,左、右寺正各一人,寺副各二人,左評事四人,右評事八人。又置審刑司,共平庶獄。凡大理寺所理之刑,審刑司復詳議之。審刑司設左、右審刑各一人,正六品;左、右詳議各三人,正七品。十七年,改建刑部、都察院、大理寺、審刑司、五軍斷事官署於太平門外,名其所曰貫城。十九年罷審刑司。二十二年復,卿秩正三品。少卿二人,正四品,丞三人,正五品。其左、右寺官如故。二十九年又罷,盡移案牘於後湖。建文初復置,改左、右寺為司,寺正為都評事,寺副為副都評事,司務為都典簿。司務,洪武二十六年置。成祖初,仍置大理寺,其左、右寺設官,復如洪武時。又因左、右二寺評事多寡不等,所治事亦繁簡不均,以二寺評事均分,左、右各六人,如刑部、都察院十二司道,各帶管直隸地方審錄。初,太祖設左評事四員,分管在京諸司及直隸衛所、府州縣刑名。右評事八員,分管在外十三布政司、都司、衛所、府州縣刑名。永樂二年,仍復舊。後定都北京,又改分寺屬。兩京、五府、六部、京衛等衙門刑名,屬左寺。順天、應天二府,南、北直隸衛所、府州縣並在外浙江等布政司、都司、衛所刑名,屬右寺。弘治元年,裁減右評事四人。時天下罪囚,類不解審,右寺事顧簡於左寺。萬曆九年,更定左、右寺分理天下刑獄。浙江、福建、山東、廣東、四川、貴州六司道,左寺理之。江西、陜西、河南、山西、湖廣、廣西、雲南七司道,右寺理之。以能按律出人罪者為稱職。大理寺之設,為慎刑也。三法司會審,初審,刑部、都察院為主,覆審,本寺為主。明初,猶置刑具、牢獄。弘治以後,止閱案卷,囚徒俱不到寺。司務典出納文移。

詹事府。詹事一人,正三品少詹事二人。正四品府丞二人,正六品主簿廳,主簿一人,從七品錄事二人,正九品通事舍人二人。左春坊,大學士,正五品左庶子,正五品左諭德,從五品各一人,左中允,正六品左贊善,從六品左司直郎,從六品,後不常設。各二人,左清紀郎一人,從八品,不常設左司諫二人,從九品,不常設。右春坊,亦如之。司經局,洗馬一人,從五品校書,正九品正字,從九品各二人。

詹事,掌統府、坊、局之政事,以輔導太子。少詹事佐之。凡入侍太子,與坊、局翰林官番直進講《尚書》、《春秋》、《資治通鑑》、《大學衍義》、《貞觀政要》諸書。前期纂輯成章進御,然後赴文華殿講讀。講讀畢,率其僚屬,以朝廷所處分軍國重事及撫諭諸蕃恩義,陳說於太子。凡朝賀,必先奏朝廷,乃具啟本以進。凡府僚暨坊、局官與翰林院職互相兼,試士、修書皆與焉。

通事舍人,典東宮朝謁、辭見之禮,承令勞問之事,凡廷臣朝賀、進箋、進春、進曆於太子,則引入而舉案。春坊大學士,掌太子上奏請、下啟箋及講讀之事,皆審慎而監省之。庶子、諭德、中允、贊善各奉其職以從。凡東宮監國、撫軍、出狩,及朝會出入,覆啟,畫諾,必審署以移詹事。諸祥眚必啟告。內外庶政可為規鑒者,隨事而贊諭。伶人、僕御有改變新聲、導逢非禮者,則陳古義,申典制,糾正而請斥遠之。司直、清紀郎,掌彈劾宮僚,糾舉職事。文華殿講讀畢,諸臣班退,有獨留奏事及私謁者,則共糾之。司諫,掌箴誨鑒戒,以拾遺補過。凡有啟事於東宮,與司直、清紀執筆紀令旨,規正其偽繆者。洗馬,掌經史子集、制典、圖書刊輯之事。立正本、副本、貯本以備進覽。凡天下圖冊上東宮者,皆受而藏之。校書、正字,掌繕寫裝潢,詮其訛謬而調其音切,以佐洗馬。

先是,洪武初,置大本堂,充古今圖籍其中,召四方名儒訓導太子、親王。諸儒專經面授,分番夜直。已而太子居文華堂,諸儒迭班侍從,又選才俊之士入充伴讀,時時賜宴、賦詩,商榷今古,評論文學。是時東宮官屬,自太子少師、少傅、少保、賓客外,則有左、右詹事,同知詹事院事,副詹事,詹事丞,左、右率府使,同知左、右率府事,左、右率府副使,諭德,贊善大夫,皆以勳舊大臣兼領其職。又有文學、中舍、正字、侍正、洗馬、庶子及贊讀等官。十五年,更定左、右春坊官,各置庶子、諭德、中允、贊善、司直郎,又各設大學士。尋定司經局官,設洗馬、校書、正字。二十二年,以官聯無統,始置詹事院。二十五年,改院為府,定詹事秩正三品,春坊大學士正五品,司經局洗馬從五品。雖各有印,而事總於詹事府。二十九年,增設左、右春坊清紀郎、司諫、通事舍人。建文中,增少卿、寺丞各一人,賓客二人。又置資德院資德一人,資善二人。其屬,贊讀、贊書、著作郎各二人,掌典籍各一人。成祖復舊制。英宗初,命大學士提調講讀官。

按詹事府多由他官兼掌。天順以前,或尚書、侍郎、都御史,成化以後,率以禮部尚書、侍郎由翰林出身者兼掌之。其協理者無常員。春坊大學士,景泰間,倪謙、劉定之而後,僅楊廷和一任之,後不復設。其司直、司諫、清紀郎亦不常置。惟嘉靖十八年以陸深為詹事,崔銑為少詹事,王教、羅洪先、華察等為諭德、贊善、洗馬,皇甫涍、唐順之等為司直、司諫,皆天下名儒。自明初宋濂諸人後,宮僚莫盛於此。嗣是,出閣講讀,每點別員,本府坊局僅為翰林官遷轉之階。

翰林院。學士一人,正五品侍讀學士、侍講學士各二人,並從五品侍讀、侍講各二人,並正六品《五經》博士九人,正八品,並世襲,別見。典籍二人,從八品侍書二人,正九品,後不常設。待詔六人,從九品,不常設。孔目一人,未入流史官修撰,從六品編修,正七品檢討,從七品庶吉士,無定員。

學士,掌制誥、史冊、文翰之事,以考議制度,詳正文書,備天子顧問。凡經筵日講,纂修實錄、玉牒、史志諸書,編纂六曹章奏,皆奉敕而統承之。誥敕,以學士一人兼領。正統中,王直、王英以禮部侍郎兼學士,專領誥敕,後罷。弘治七年復設。正德中,白鉞、費宏等由禮部尚書入東閣,專典誥敕。嘉靖六年復罷,以講、讀、編、檢等官管之。大政事、大典禮,集諸臣會議,則與諸司參決其可否。車駕幸太學聽講,凡郊祀慶成諸宴,則學士侍坐於四品京卿上。

侍讀、侍講,掌講讀經史。《五經》博士,初置五人,各掌專經講義,繼以優給聖賢先儒後裔世襲,不治院事。史官,掌修國史。凡天文、地理、宗潢、禮樂、兵刑諸大政,及詔敕、書檄,批答王言,皆籍而記之,以備實錄。國家有纂修著作之書,則分掌考輯撰述之事。經筵充展卷官,鄉試充考試官,會試充同考官,殿試充收卷官。凡記注起居,編纂六曹章奏,謄黃冊封等咸充之。庶吉士,讀書翰林院,以學士一人教習之。侍書,掌以六書供侍。待詔,掌應對。孔目掌文移。

吳元年,初置翰林院,秩正三品,設學士,正三品侍講學士,正四品直學士,正五品修撰、典簿,正七品編修,正八品洪武二年,置學士承旨,正三品,改學士,從三品。侍講學士,正四品,侍讀學士,從四品,修撰,正六品。增設待制,從五品應奉,正七品典籍從八品等官。十三年,增設檢閱。從九品十四年,定學士為正五品,革承旨、直學士、待制、應奉、檢閱、典簿,設孔目、《五經》博士、侍書、待詔、檢討。令編修、檢討、典籍同左春坊左司直郎、正字、贊讀考駁諸司奏啟,平允則署其銜曰「翰林院兼平駁諸司文章事某官某」,列名書之。十八年,更定品員,如前所列,獨未有庶吉士。以侍讀先侍講。建文時,仍設承旨,改侍讀、侍講兩學士為文學博士,設文翰、文史二館,文翰以居侍讀、侍講、侍書、《五經》博士、典籍、待詔,文史以居修撰、編修、檢討。改孔目為典簿,改中書舍人為侍書,以隸翰林。又設文淵閣待詔及拾遺、補闕等官。成祖初復舊。其年九月,特簡講、讀、編、檢等官參預機務,簡用無定員。謂之內閣。然解縉、胡廣等既直文淵閣,猶相繼署院事。至洪熙以後,楊士奇等加至師保,禮絕百僚,始不復署。正統七年,翰林院落成,學士錢習禮不設楊士奇、楊溥公座,曰「此非三公府也」,二楊以聞。乃命工部具椅案,禮部定位次,以內閣固翰林職也。嘉、隆以前,文移關白,猶稱翰林院,以後則竟稱內閣矣。其在六部,自成化時,周洪謨以後,禮部尚書、侍郎必由翰林,吏部兩侍郎必有一由於翰林。其由翰林者,尚書則兼學士,六部皆然。侍郎則兼侍讀、侍講學士。其在詹事府暨坊、局官,視其品級,必帶本院銜。詹事、少詹事帶學士銜,春坊大學士不常設,庶子、諭德、中允、贊善、洗馬等則帶講、讀學士以下至編、檢銜。

史官,自洪武十四年置修撰三人,編修、檢討各四人。其後由一甲進士除授及庶吉士留館授職,往往溢額,無定員。嘉靖八年,復定講、讀、修撰各三人,編修、檢討各六人,皆從吏部推補,如諸司例。然未幾即以侍從人少,詔采方正有學術者以充其選,因改御史胡經、員外郎陳束、主事唐順之等七人俱為編修。以後仍循舊例,由庶吉士除授,卒無定額。崇禎七年,又考選推官、知縣為編修、檢討,蓋亦創舉,非常制也。

庶吉士,自洪武初有六科庶吉士。十八年以進士在翰林院、承敕監等近侍者,俱稱庶吉士。永樂二年,始定為翰林院庶吉士,選進士文學優等及善書者為之。三年試之。其留者,二甲授編修,三甲授檢討;不得留者,則為給事中、御史,或出為州縣官。宣德五年,始命學士教習。萬曆以後,掌教習者,專以吏、禮二部侍郎二人。

明初,嘗置弘文館學士,洪武三年置,以胡鉉為學士,又命劉基、危素、王本中、睢稼皆兼弘文館學士,未幾罷。宣德間,復建弘文閣於思善門右,以翰林學士楊溥掌閣印,尋併入文淵閣。祕書監,洪武三年置,秩正六品,除監丞一人,直長二人,尋定設令一人,丞、直長各二人,掌內府書籍。十三年並入翰林院典籍。起居注,甲辰年置。吳元年定秩正五品。洪武四年改正七品。六年升從六品。九年定起居注二人,後革。十四年復置,秩從七品,尋罷。至萬曆間,命翰林院官兼攝之。已復罷。尋皆罷。

國子監。祭酒一人。從四品司業一人。正六品其屬,繩愆廳,監丞一人,正八品博士廳,《五經》博士五人。從八品率性、修道、誠心、正義、崇志、廣業六堂,助教十五人,從八品學正十人,正九品學錄七人。從九品典簿廳,典簿一人。從八品典籍廳,典籍一人。從九品掌饌廳,掌饌二人。未入流

祭酒、司業,掌國學諸生訓導之政令。凡舉人、貢生、官生、恩生、功生、例生、土官、外國生、幼勳臣及勳戚大臣子弟之入監者,奉監規而訓課之,造以明體達用之學,以孝弟、禮義、忠信、廉恥為之本,以六經、諸史為之業,務各期以敦倫善行,敬業樂群,以修舉古樂正、成均之師道。有不率者,撲以夏楚,不悛,徙謫之。其率教者,有升堂積分超格敘用之法。課業仿書,季呈翰林院考校,文冊歲終奏上。每歲仲春秋上丁,遣大臣祀先師,則總其禮儀。車駕幸學,則執經坐講。新進士釋褐,則坐而受拜。監丞掌繩愆廳之事,以參領監務,堅明其約束,諸師生有過及廩膳不潔,並糾懲之,而書之於集愆冊。博士掌分經講授,而時其考課。凡經,以《易》、《詩》、《書》、《春秋》、《禮記》,人專一經,《大學》、《中庸》、《論語》、《孟子》兼習之。助教、學正、學錄掌六堂之訓誨,士子肄業本堂,則為講說經義文字,導約之以規矩。典簿,典文移金錢出納支受。典籍,典書籍。掌饌,掌飲饌。

明初,即置國子學。乙巳九月置國子學,以故集慶路學為之。洪武十四年,改建國子學於雞鳴山下。設博士、助教、學正、學錄、典樂、典書、典膳等官。吳元年,定國子學官制,增設祭酒、司業、典簿。祭酒,正四品,司業,正五品,博士,正七品,典簿,正八品,助教,從八品,學正,正九品,學錄,從九品,典膳,省注。洪武八年,又置中都國子學,秩正四品命國子學分官領之。十三年,改典膳為掌饌。十五年,改為國子監,秩從四品,設祭酒一人,司業一人,監丞、典簿各一人,博士三人,助教十六人,學正、學錄各三人,掌饌一人。各官品秩,如前所列。中都國子監制亦如之。十六年,以宋訥為祭酒,敕諭之曰:「太學天下賢關,禮義所由出,人材所由興。卿夙學耆德,故特命為祭酒。尚體朕立教之意,俾諸生有成,士習丕變,國家其有賴焉。」又命曹國公李文忠領監事,車駕時幸。以故監官不得中廳而坐,中門而行。二十四年,更定國子監品秩、員數。俱如前所列。中都國子監設祭酒、司業、監丞、典簿、博士、學正、學錄、掌饌各一人,助教二人,品秩與在京同。二十六年,罷中都國子監。建文中,升監丞為堂上官,革學正、學錄。成祖復舊制。永樂元年,置國子監於北京,設祭酒、司業、監丞、典簿、博士、學正、學錄、掌饌各一人,助教二人。後增設不常,助教至十五人,學正至十一人,學錄至七人。後革助教二人,學正四人,學錄二人。萬歷九年,又革助教四人,學錄一人。宣德九年,省司業。弘治十五年復設。明初,祭酒、司業,擇有學行者任之,後皆由翰林院官遷轉。

衍聖公,孔氏世襲,正二品。袍帶、誥命、朝班一品。洪武元年授孔子五十六代孫希學襲封。其屬,掌書、典籍、司樂、知印、奏差、書寫各一人。皆以流官充之。曲阜知縣,孔氏世職。洪武元年授孔子裔孫希大為曲阜世襲知縣。翰林院世襲《五經》博士,正八品孔氏二人,正德元年授孔子五十九世孫彥繩主衢州廟祀。宋孔端友從高宗南渡,家於衢州,此孔氏南宗也。正德二年,授孔聞禮奉子思廟祀。顏氏一人,景泰三年,授顏子五十九世孫希惠。曾氏一人,嘉靖十八年,授曾子六十代孫質粹。仲氏一人,萬歷十五年,授子路裔孫仲呂。孟氏一人,景泰三年,授孟子裔孫希文。周氏一人,景泰七年,授先儒周敦頤裔孫冕。程氏二人,景泰六年,授先儒程頤裔孫克仁。崇禎三年,授先儒程顥裔孫接道。邵氏一人,崇禎三年,授先儒邵雍裔孫繼祖。張氏一人,天啟二年,以先儒張載裔孫文運為博士。朱氏二人,景泰六年,授先儒朱熹裔孫梴。嘉靖二年又授墅為博士,主婺源廟祀。劉氏一人,景泰七年,授誠意伯劉基七世孫祿,後革。教授司,教授,從九品學錄、學司,並未入流孔、顏、曾、孟四氏,各一人。又尼山、洙泗二書院,各學錄一人。

先是,元代封孔子後裔為衍聖公,賜三品印。洪武元年,太祖既以孔希學襲封衍聖公,因謂禮臣曰:「孔子萬世帝王之師,待其後嗣,秩止三品,弗稱褒崇,其授希學秩二品,賜以銀印。」又命復孔、顏、孟三家子孫徭役。十八年,敕工部詢問,凡有聖賢子孫以罪輸作者,釋之。永樂二十二年,賜衍聖公宅於京師,加一品金織衣。正統元年,詔免凡聖賢子孫差役,選周、程、張、硃諸儒子孫聰明俊秀可教養者,不拘名數,送所在儒學讀書,仍給廩饌。成化元年,給孔、顏、孟三氏學印,令三年貢有學行者一人,入國子監。六年,命衍聖公始襲者在監讀書一年。

太常寺附提督四夷館光祿寺太僕寺鴻臚寺尚寶司六科中書舍人行人司欽天監太醫院上林苑監五城兵馬司順天府附宛平大興二縣武學僧道錄司教坊司宦官女官

太常寺。卿一人,正三品少卿二人,正四品寺丞二人。正六品其屬,典簿廳,典簿二人,正七品博士二人,協律郎二人,正八品,嘉靖中增至五人。贊禮郎九人,正九品,嘉靖中增至三十三人,後革二人。司樂二十人。從九品,嘉靖中增至三十九人,後革五人。天壇、地壇、朝日壇、夕月壇、先農壇、帝王廟、祈穀殿、長陵、獻陵、景陵、裕陵、茂陵、泰陵、顯陵、康陵、永陵、昭陵各祠祭署,俱奉祀一人,從七品祀丞二人。從八品犧牲所,吏目一人。從九品

太常,掌祭祀禮樂之事,總其官屬,籍其政令,以聽於禮部。凡天神、地祇、人鬼,歲祭有常。先冬十二月朔,奏進明年祭日,天子御奉天殿受之,乃頒於諸司。天子親祭,則贊相禮儀。大臣攝事,亦如之。凡國有冊立、冊封、冠婚、營繕、征討、大喪諸典禮,歲時旱澇大災變,則請告宗廟社稷。薦新則移光祿寺供其品物。祭祀先期請省牲,進祝版、銅人,上殿奏請齋戒,親署御名。省牲偕光祿卿。惟大祀車駕親省,大臣日一省之。凡祭,滌器、爨埋、香燭、玉帛,整拂神幄,必恭潔。掌燎、看燎、讀祝、奏禮、對引、司香、進俎、舉麾、陳設、收支、導引、設位、典儀、通贊、奉帛、執爵、司樽、司罍洗,卿貳屬各領其事,罔有不共。凡玉四等:曰蒼璧,以祀天曰黃琮,以祀地曰赤璋、白琥,以朝日、夕月曰兩圭有邸。以祭太社、太稷帛五等:曰郊祀制帛,祀天地曰奉先制帛,薦祖考曰禮神制帛,祭社稷、群神、帝王、先師曰展親制帛,祭享親王曰報功制帛,祭享功臣。牲四等:曰犢,曰牛,曰太牢,曰少牢。色尚騂或黝。大祀入滌三月,中祀一月,小祀一旬。樂四等:曰九奏,用祀天地曰八奏,神祇、太歲,曰七奏,大明、太社、太稷、帝王曰六奏。夜明、帝社、帝稷、宗廟、先師。舞二:曰文舞,曰武舞。樂器不徙。陵園之祭無樂。歲終合祭五禮之神,則少卿攝事。

初,吳元年置太常司,設卿,正三品少卿,正四品丞,正五品典簿、協律郎、博士,正七品贊禮郎。從八品洪武初,置各祠祭署,設署令、署丞。十三年,更定協律郎等官品秩。協律郎正八品,贊禮郎正九品,司樂從九品。三十四年改各署令為奉祀,署丞為祀丞。二十年改司為寺,官制仍舊。二十五年已定司丞正六品。建文中,增設贊禮郎二人,太祝一人,以及各祠祭署俱有更革。天壇祠祭署為南郊祠祭署,泗州祠祭署為泗濱祠祭署,宿州祠祭署為新豐祠祭署,孝陵置鐘山祠祭署,各司圃所增神樂觀知觀一人。成祖初,惟易天壇為天地壇,餘悉復洪武間制。建文時,南郊祠祭署為郊壇祠祭署,已又改為天地壇祠祭署。洪熙元年置犧牲所,吏目典掌文移。先是,洪武三年置神牲所,設廩牲令、大使、副使等官。四年革。世宗釐祀典,分天地壇為天壇、地壇,山川壇、耤田祠祭署為神祇壇,大祀殿為祈穀殿,增置朝日、夕月二壇,各設祠祭署。又增設協律郎、贊禮郎、司樂等員。隆慶三年,革協律郎等官四十八員,萬曆六年復設,如嘉靖間制。萬歷四年,改神祇壇為先農壇。

提督四夷館。少卿一人,正四品掌譯書之事。自永樂五年,外國朝貢,特設蒙古、女直、西番、西天、回回、百夷、高昌、緬甸八館,置譯字生、通事,通事初隸通政使司通譯語言文字。正德中,增設八百館。八百國蘭者哥進貢萬曆中,又增設暹羅館。初設四夷館隸翰林院,選國子監生習譯。宣德元年,兼選官民子弟,委官教肄,學士稽考程課。弘治七年,始增設太常寺卿、少卿各一員為提督,遂改隸太常。嘉靖中,裁卿,止少卿一人。按太常寺卿在南京者,多由科目。北寺自永樂間用樂舞生,累資升至寺卿,甚或加禮部侍郎、尚書掌寺,後多沿襲。至隆慶初,乃重推科甲出身者補任。譯字生,明初甚重。與考者,與鄉、會試額科甲一體出身。後止為雜流。其在館者,升轉皆在鴻臚寺。

光祿寺。卿一人,從三品少卿二人,正五品寺丞二人,從六品其屬,典簿廳,典簿二人,從七品錄事一人,從八品大官、珍羞、良醞、掌醢四署,各署正一人,從六品署丞四人,從七品監事四人,從八品司牲司,大使一人,從九品副使一人,後革司牧局,大使一人,從九品,嘉靖七年革。銀庫,大使一人。

卿,掌祭享、宴勞、酒醴、膳羞之事,率少卿、寺丞官屬,辨其名數,會其出入,量其豐約,以聽於禮部。凡祭祀,同太常省牲;天子親祭,進飲福受胙;薦新,循月令獻其品物;喪葬供奠饌。所用牲、果、菜物,取之上林苑;不給,市諸民,視時估十加一,其市直季支天財庫。四方貢獻果鮮廚料,省納惟謹。器皿移工部及募工兼作之,歲省其成敗。凡筵宴酒食及外使、降人,俱差其等而供給焉。傳奉宣索,籍記而覆奏之。監以科道官一員,察其出入,糾禁其姦弊。歲四月至九月,凡御用物及祭祀之品皆用冰。大官,供祭品宮膳、節令筵席、蕃使宴犒之事。珍羞,供宮膳肴核之事。良醞,供酒醴之事。掌醢,供餳、油、醯、醬、梅、鹽之事。司牲養牲,視其肥瘠而蠲滌之。司牧亦如之。

初,吳元年置宣徽院,設院使,正三品同知,正四品院判,正五品典簿。正七品以尚食、尚醴二局隸之。局設大使,從六品,副使,從七品洪武元年改為光祿寺,設光祿卿,正四品少卿,正五品寺丞,正六品主簿。正八品所屬尚食等局,又移太常司供需庫隸之。局庫官品仍舊。二年,設直長四人,遇百官賜食御前者,則令供事。四年,置法酒庫。設內酒坊大使,從八品,副使,從九品。八年,改寺為司,升卿秩,卿從三品,少卿從四品。以寺丞為司丞,從六品主簿為典簿,從七品增設錄事。從八品又置所屬大官、珍羞、良醞、掌醢四署,每署令一人,從六品丞一人,從七品監事一人。從八品孳牧所,大使一人,從九品副使一人。未入流十年,定光祿司散官品秩。時所用光祿司官,或內官,或流官,或庖人,出身不同,同授散官。至是定,內官除授者,照內官散官給授。流官除授者,照文官散官給授。庖人除授者,卿從三品,授尚膳大夫;少卿正五品,授奉膳大夫;司丞從六品,授司膳郎;客署丞從七品,授掌膳郎;監事從八品,授執膳郎。尋罷各局庫,置司牲司,又改孳牧所為司牧司。後為司牧局。三十年,復改為光祿寺,官制仍舊。少卿已定正五品。建文中,升少卿、寺丞品秩。少卿升四品,寺丞升五品。增設司圃所,改司牲司為孳牲所。升其品級成祖復舊制。正統六年,裁四署冗員。先是,光祿卿奈享以供應事繁,奏增各署官,至是復奏裁之。裁署正四人,署丞五人,監事七人。嘉靖七年,革司牧局。萬歷二年,添設銀庫大使一人。

太僕寺。卿一人,從三品少卿二人,正四品,正德十一年增設一人。寺丞四人。正六品其屬,主簿廳,主簿一人。從七品常盈庫,大使一人。所轄,各牧監,監正一人,正九品監副一人,從九品錄事一人。後監正、監副、錄事俱革。各群,群長一人。後革

卿,掌牧馬之政令,以聽於兵部。少卿一人佐寺事,一人督營馬,一人督畿馬。寺丞分理京衛、畿內及山東、河南六郡孳牧、寄牧馬匹。濟南、兗州、東昌、開封、彰德、衛輝。凡軍民孳牧,視其丁產,授之種馬。牡十之二,牝十之八,為一群。南方以四牝一牡為群。歲徵其駒,曰備用馬,齊其力以給將士。將士足,則寄牧於畿內府州縣,肥瘠登耗,籍其毛齒而時閱之。三歲偕御史一人印烙,選其健良而汰其羸劣。其草場已墾成田者,歲斂其租金,災祲則出之以佐市馬。其賠償折納,則征馬金輸兵部。主簿典勾省文移。大使典貯庫馬金。

初,洪武四年置群牧監於答答失里營所,隨水草利便立官署,專司牧養。六年,更置群牧監於滁州,旋改為太僕寺,秩從三品,設卿、少卿、寺丞,又設首領官知事、主簿各一人。七年,增設牧監、群官二十七處,隸太僕寺。尋定群牧監品秩。令,正五品,丞,正六品,鎮撫,從六品,群頭十人、吏目一人,省注。十年,增置滁陽等各牧監及所屬各群。改牧監令、丞為監正、監副。監正,從八品,監副,正九品,御良,從九品。後又定監正為正九品。二十二年,定滁陽等十二牧監,每監設監正一人,監副二人,錄事一人。來安等一百二十七群,每群設群長一人。初設群副二人,至是革。二十三年,增置江東、當塗二牧監及所屬各群。又罷烏衣等五十四群,改置永安等七群,定為牧監十四,滁陽、大興、香泉、儀真、定遠、天長、長淮、江都、句容、溧陽、江東、溧水、當塗、舒城。群九十有七。大勝關、柏子、騮興、保寧、草堂五群,隸滁陽監。永安、如皋、沿海、保全、朝陽、永昌、安定七群,隸大興監。大錢、銅城、永豐、龍勝、龍山、永寧、新安、慶安、襄安九群,隸香泉監。華陽、壽寧、廣陵、善應四群,隸儀真監。龍江、龍安、萬勝、龍泉四群,隸定遠監。天長、懷德、招信、得勝、武安五群,隸天長監。長安、白石、荊山、南山、團山、草平六群,隸長淮監。萬寧、廣生、萬驥、順德、大興、驥寧、崇德七群,隸江都監。句容、易風、仍信、福胙、通德、承佩、上容、政仁、練塘、壽安十群,隸句容監。舉福、從山、明義、永定、福賢、崇來、永城、永泰、奉安九群,隸溧陽監。開寧、泉水、惟政、清化、神泉、新亭、長泰、光澤八群,隸江東監。儀鳳、仙壇、立信、歸政、豐慶、安興、遊山、永寧八群,隸溧水監。石城、永保、化洽、姑熟、繁昌、多福、丹陽、德政八群,隸當塗監。棗林、海亭、伏龍、龍河、會龍、九龍、萬龍七群,隸舒城監。二十八年,悉罷群牧監,以其馬隸有司牧養。三十年,置行太僕寺於北平,秩如太僕寺。建文中,陞寺丞品秩,舊六品,升五品。又改其首領官職名,增設錄事,及典廄、典牧二署,肅騻等十八群,滁陽等八牧監,龍山等九十二群。成祖復舊制。永樂元年,改北平行太僕寺為北京行太僕寺。十八年定都北京,遂以行太僕寺為太僕寺。洪熙元年,復稱北京行太僕寺。正統六年,定為太僕寺。其舊在滁州者,改為南京太僕寺。寺丞,初置四人。正統中,又增八人,共十二人,以一人領京衛,一人領順德、廣平二府,一人領開封、衛輝、彰德三府,九人分領順天、保定、真定、河間、永平、大名、濟南、兗州、東昌九府孳牧、寄牧各馬匹。弘治六年革四人。正德九年復增一人,專領寄牧之事。嘉靖八年又革三人,共六人分領,三年更代,而以寄牧者令府州縣兼理。隆慶三年又革三人,止設三人,以一人提督庫藏兼協理京邊,二人分理東西二路各馬政。

鴻臚寺。卿一人,正四品左、右少卿各一人,從五品左、右寺丞各一人。從六品其屬,主簿廳,主簿一人。從八品司儀、司賓二署,各署丞一人,正九品鳴贊四人,從九品,後增設五人。序班五十人。從九品。嘉靖三十六年革八人。萬歷十一年復設六人。

鴻臚,掌朝會、賓客、吉凶儀禮之事。凡國家大典禮、郊廟、祭祀、朝會、宴饗、經筵、冊封、進曆、進春、傳制、奏捷,各供其事。外吏朝覲,諸蕃入貢,與夫百官使臣之復命、謝恩,若見若辭者,並鴻臚引奏。歲正旦、上元、重午、重九、長至賜假、賜宴,四月賜字扇、壽縷,十一月賜戴煖耳,陪祀畢,頒胙賜,皆贊百官行禮。司儀,典陳設、引奏,外吏來朝,必先演儀於寺。司賓,典外國朝貢之使,辨其等而教其拜跪儀節。鳴贊,典贊儀禮。凡內贊、通贊、對贊、接贊、傳贊咸職之。序班,典侍班、齊班、糾儀及傳贊。

初,吳元年置侍儀司,秩從五品。洪武四年定侍儀使,從七品引進使,正八品奉班都知,正九品通贊、通事舍人,從九品俱為七品以下官。九年,改為殿庭儀禮司,設使一人,正七品副三人,正八品丞奉一人,從八品鳴贊二人,正九品序班十六人,從九品九關通事使一人,正八品副六人。從八品十三年,改使為司正,分左、右司副各一人,增序班至四十四人,革承奉,增設司儀四人。二十二年,增設左、右司丞四人。正九品三十年,始改為鴻臚寺,升秩正四品,設官六十二員。卿以下員數、品級如前所列。又設外夷通事隸焉。建文中,陞少卿以下品秩。少卿升正五品,寺丞陞正六品。又改其首領官職名,與鳴贊、序班皆升品級。罷司儀、司賓二署,而以行人隸鴻臚寺。成祖初,悉復舊制。

尚寶司。卿一人,正五品少卿一人,從五品司丞三人。正六品。吳元年但設一人,後增二人。掌寶璽、符牌、印章,而辨其所用。

寶二十有四。舊寶十有七,嘉靖十八年增製者七。曰「皇帝奉天之寶」,為唐、宋傳璽,祀天地用之。若詔與赦,則用「皇帝之寶」;冊封、賜勞,則用「皇帝行寶」;詔親王、大臣及調兵,則用「皇帝信寶」;上尊號,則用「皇帝尊親之寶」;諭親王,則用「皇帝親親之寶」。其「天子之寶」,以祀山川、鬼神;「天子行寶」,以封外國及賜勞;「天子信寶」,以招外服及徵發。詔用「制誥之寶」;敕用「敕命之寶」;獎勵臣工,用「廣運之寶」;敕諭朝覲官,用「敬天勤民之寶」。若「御前之寶」,「表章經史之寶」,「欽文之寶」,則圖書文史等用之。世宗增製,為「奉天承運大明天子寶」,為「大明受命之寶」,為「巡狩天下之寶」,為「垂訓之寶」,為「命德之寶」,為「討罪安民之寶」,為「敕正萬民之寶。」。太子之寶一,曰「皇太子之寶」。凡寶之用,必奏請而待發。每大朝會,本司官二員,以寶導駕,俟升座,各置寶於案,立待殿中。禮畢,捧寶分行,至中極殿,置案而出。駕出幸,則奉以從焉。歲終,移欽天監,擇日和香物入水,洗寶於皇極門。籍奏一歲用寶之數。凡請寶、用寶、捧寶、隨寶、洗寶、繳寶,皆與內官尚寶監俱。

凡金牌之號五,以給勳戚侍衛之扈從及班直者、巡朝者、夜宿衛者:曰仁,其形龍,公、侯、伯、駙馬都尉佩之;曰義,其形虎,勳衛指揮佩之;曰禮,其形麟,千戶佩之;曰智,其形獅,百戶佩之;曰信,其形祥雲,將軍佩之。半字銅符之號四,以給巡城寺衛官:曰承,曰東,曰西,曰北。巡者左半,守者右半,合契而點察焉。令牌之號六:曰申,以給金吾諸衛之警夜者;曰木,曰金,曰土,曰火,曰水,以給五城之警夜者。銅牌之號一,以稽守卒,曰勇。牙牌之號五,以察朝參:公、侯、伯曰勳,駙馬都尉曰親,文官曰文,武官曰武,教坊司曰樂。嘉靖中,總編曰官字某號,朝參佩以出入,不則門者止之。私相借者,論如律。有故,納之內府。祭牌之號三:陪,祀官曰陪,供事官曰供,執事人曰執。雙魚銅牌之號二:曰嚴,以肅直衛錦衣校尉之止直者;曰善,以飾光祿胥役之供事者。符驗之號五:曰馬,曰水,曰達,曰通,曰信。符驗之制,上織船馬之狀,起馬用「馬」字,雙馬用「達」字,單馬用「通」字。起船者用「水」字,並船用「信」字。親王之籓及文武出鎮撫、行人通使命者,則給之。御史出巡察則給印,事竣,咸驗而納之。稽出入之令,而辨其數,其職至邇,其事至重也。

太祖初,設符璽郎,秩正七品。吳元年改尚寶司卿,秩正五品,以侍從儒臣、勛衛領之,如耿瑄以散騎舍人、黃觀以侍中、楊榮以庶子為卿。非有才能不得調。勛衛大臣子弟奉旨乃得補丞。其後多以恩廕寄祿,無常員。

吏、戶、禮、兵、刑、工六科。各都給事中一人,正七品左、右給事中各一人。從七品給事中,吏科四人,戶科八人,禮科六人,兵科十人,刑科八人,工科四人。並從七品,後增、減員數不常。萬歷九年裁兵科五人,戶、刑二科各四人,禮科二人。十一年復設戶、兵、刑三科各二人,禮科一人。六科,掌侍從、規諫、補闕、拾遺、稽察六部百司之事。凡制敕宣行,大事覆奏,小事署而頒之;有失,封還執奏。凡內外所上章疏下,分類抄出,參署付部,駁正其違誤。吏科,凡吏部引選,則掌科即都給事中,以掌本科印,故名,六科同。同至御前請旨。外官領文憑,皆先赴科畫字。內外官考察自陳後,則與各科具奏。拾遺糾其不職者。戶科,監光祿寺歲入金穀,甲字等十庫錢鈔雜物,與各科兼蒞之,皆三月而代。內外有陳乞田土、隱占侵奪者,糾之。禮科,監訂禮部儀制,凡大臣曾經糾劾削奪、有玷士論者紀錄之,以核贈謚之典。兵科,凡武臣貼黃誥敕,本科一人監視。其引選畫憑之制,如吏科。刑科,每歲二月下旬,上前一年南北罪囚之數,歲終類上一歲蔽獄之數,閱十日一上實在罪囚之數,皆憑法司移報而奏御焉。工科,閱試軍器局,同御史巡視節慎庫,與各科稽查寶源局。而主德闕違,朝政失得,百官賢佞,各科或單疏專達,或公疏聯署奏聞。雖分隸六科,其事屬重大者,各科皆得通奏。但事屬某科,則列其科為首。凡日朝,六科輪一人立殿左右,珥筆記旨。凡題奏,日附科籍,五日一送內閣,備編纂。其諸司奉旨處分事目,五日一註銷,核稽緩。內官傳旨必覆奏,復得旨而後行。鄉試充考試官,會試充同考官,殿試充受卷官。冊封宗室、諸蕃或告諭外國,充正、副使。朝參門籍,六科流掌之。登聞鼓樓,日一人,皆錦衣衛官監蒞。洪武元年,以監察御史一人監登聞鼓,後令六科與錦衣衛輪直。受牒,則具題本封上。遇決囚,有投牒訟冤者,則判停刑請旨。凡大事廷議,大臣廷推,大獄廷鞫,六掌科皆預焉。

明初,統設給事中,正五品,後數更其秩。與起居注同。洪武六年,設給事中十二人,秩正七品,始分為六科,每科二人,鑄給事中印一,推年長者一人掌之。九年,定給事中十人。十年,隸承敕監。十二年,改隸通政司。十三年,置諫院,左、右司諫各一人,正七品左、右正言各二人。從七品十五年,又置諫議大夫。以兵部尚書唐鐸為之。尋皆罷。二十二年,改給事中為源士,增至八十一人。初,魏敏、卓敬等凡八十一人為給事中。上以其適符古元士之數,改為元士。至是,又以六科為事之本源,改為源士。未幾,復為給事中。二十四年,更定科員,每科都給事中一人,正八品。左、右給事中二人,從八品。給事中共四十人,正九品。各科分設員數,如前所列。建文中,改都給事中,正七品,給事中,從七品,不置左、右給事中。增設拾遺、補闕。成祖初,革拾遺、補闕,仍置左、右給事中,亦從七品。尋改六科,置於午門外直房蒞事。六科衙門舊在磚門內尚寶司西。永樂中災,移午門外東西,每夜一科直宿。宣德八年,增戶科給事中,專理黃冊。

中書科。中書舍人二十人,從七品直文華殿東房中書舍人,直武英殿西房中書舍人,內閣誥敕房中書舍人,制敕房中書舍人。並從七品,無定員。

中書科舍人掌書寫誥敕、制詔、銀冊、鐵券等事。凡草請諸翰林,寶請諸內府,左券及勘籍,歸諸古今通集庫。誥敕,公侯伯及一品至五品誥命、六品至九品敕命。勘合籍,初用二十八宿,後用《急就章》為號。誥敕之號,曰仁、義、禮、智,公、侯、伯、蕃王、一品、二品用之;曰十二支,曰文、行、忠、信,文官三品以下用之;曰千字文,武官、續誥用之。皆以千號為滿,滿則復始。王府及駙馬都尉不編號,土官以文武類編。凡大朝會,則侍班。東宮令節朝賀,則導駕侍班於文華殿。冊封宗室,則充副使。其鄉試、會試、殿試,間有差遣,充授並如科員。大祀南郊,則隨駕而供事。員無正貳,印用年深者掌之。文華殿舍人,職掌奉旨書寫書籍。武英殿舍人,職掌奉旨篆寫冊寶、圖書、冊頁。內閣誥敕房舍人,掌書辦文官誥敕,番譯敕書,并外國文書、揭帖,兵部紀功、勘合底簿。制敕房舍人,掌書辦制敕、詔書、誥命、冊表、寶文、玉牒、講章、碑額、題奏、揭帖一應機密文書,各王府敕符底簿。

洪武七年,初設直省舍人十人,秩從八品,隸中書省。九年,為中書舍人,改正七品,尋又改從七品。十年,與給事中皆隸承敕監。建文中,革中書舍人,改為侍書,升正七品,入文翰館,隸翰林院。成祖復舊制。尋設中書科署於午門外,定設中書舍人二十人。其恩廕帶俸者,不在額內。宣德間,內閣置誥敕、制敕兩房,皆設中書舍人。嘉靖二十年,選各部主事,大理寺評事,帶原銜直誥敕、制敕兩房。四十四年,兩房員缺,令吏部考選舉人為中書舍人。隆慶元年,令兩房辦事官不得升列九卿。按洪武間,置承敕監、洪武九年置,設令一人,正六品,丞二人,從六品。尋改令正七品,丞正八品。十年改令、丞為承敕郎,設二人,從七品。給事中、中書舍人咸隸焉。後罷。司文監、洪武九年置,設令一人,正六品,丞二人,從六品。尋改令正七品,丞正八品。十年罷。考功監,洪武八年置,設令、丞。九年定設令一人,正六品,丞二人,從六品。尋改令正七品,丞正八品。十八年罷。參掌給授誥敕之事。永樂初,命內閣學士典機務,詔冊、制誥皆屬之。而謄副、繕正皆中書舍人入辦,事竣輒出。宣德初,始選能書者處於閣之西小房,謂之西制敕房。而諸學士掌誥敕者居閣東,具稿付中書繕進,謂之東誥敕房。此系辦事。若知制誥銜,惟大學士與諸學士可帶。正統後,學士不能視誥敕,內閣悉委於中書、序班、譯字等官,於是內閣又有東誥敕房。因劉鉉不與輔臣會食始。嘉靖末,復以翰林史官掌外制,而武官誥敕仍自其屬為之。若詔赦、敕革之類,必由閣臣,翰林諸臣不得預。其直文華、武英兩殿供御筆札者,初為內官職,繼以中書分直,後亦專舉能書者。大約舍人有兩途,由進士部選者,得遷科道部屬,其直兩殿、兩房舍人,不必由部選,自甲科、監生、生儒、布衣能書者,俱可為之。不由科甲者,初授序班,及試中書舍人,不得遷科道部屬,後雖加銜九列,仍帶銜辦事。楷書出身者,或加太常卿銜,沈度、沈粲、潘辰等有加至翰林學士、禮部尚書者。洪武初,又有承天門待詔一人,閣門使四人,觀察使十人,後俱革。

行人司。司正一人,正七品左、右司副各一人,從七品行人三十七人。正八品職專捧節、奉使之事。凡頒行詔赦,冊封宗室,撫諭諸蕃,徵聘賢才,與夫賞賜、慰問、賑濟、軍旅、祭祀,咸敘差焉。每歲朝審,則行人持節傳旨法司,遣戍囚徒,送五府填精微冊,批繳內府。

初,洪武十三年置行人司,設行人,秩正九品。左、右行人,從九品。尋改行人為司正,左、右行人為左、右司副,更設行人三百四十五人。二十七年升品秩,以所任行人多孝廉人材,奉使率不稱旨,定設行人司官四十員,咸以進士為之。非奉旨,不得擅遣,行人之職始重。建文中,罷行人司,而以行人隸鴻臚寺。成祖復舊制。

欽天監。監正一人,正五品監副二人。正六品其屬,主簿廳,主簿一人,正八品春、夏、中、秋、冬官正各一人,正六品五官靈臺郎八人,從七品,後革四人。五官保章正二人,正八品,後革一人。五官挈壺正二人,從八品,後革一人。五官監候三人,正九品,後革一人。五官司曆二人,正九品五官司晨八人,從九品,後革六人。漏刻博士六人。從九品,後革五人。

監正、副,掌察天文、定曆數、占候、推步之事。凡日月、星辰、風雲、氣色,率其屬而測候焉。有變異,密疏以聞。凡習業分四科:曰天文,曰漏刻,曰回回,曰曆。自五官正下至天文生、陰陽人,各分科肄業。每歲冬至日,呈奏明歲《大統曆》,成化十五年改頒明歲歷於十月朔日。移送禮部頒行。其《御覽月令曆》、《七政躔度曆》、《六壬遁甲曆》、《四季天象錄》,並先期進呈。凡曆註,御曆註三十事,如祭祀、頒詔、行幸等類。民曆三十二事,壬遁曆七十二事。凡祭日,前一年會選以進,移知太常。凡營建、征討、冠婚、山陵之事,則選地而擇日。立春,則預候氣於東郊。大朝賀,於文樓設定時鼓、漏刻報時,司晨、雞唱,各供其事。日月交食,先期算其分秒時刻、起復方位以聞,下禮部,移內外諸司救之,仍按占書條奏。若食不及一分,與《回回曆》雖食一分以上,則奏而不救。監官毋得改他官,子孫毋得徙他業。乏人,則移禮部訪取而試用焉。五官正推曆法,定四時。司歷、監候佐之。靈臺郎辨日月星辰之躔次、分野,以占候天文之變。觀象臺四面,面四天文生,輪司測候。保章正專志天文之變,定其吉凶之占。挈壺正知刻漏。孔壺為漏,浮箭為刻,以考中星昏旦之次。漏刻博士定時以漏,換時以牌,報更以鼓,警晨昏以鐘鼓。司晨佐之。

明初,即置太史監,設太史令,通判太史監事,僉判太史監事,校事郎,五官正,靈臺郎,保章正、副,挈壺正,掌曆,管勾等官。以劉基為太史令。吳元年,改監為院,秩正三品。院使,正三品,同知,正四品,院判,正五品,五官正,正六品,典簿、雨暘司、時敘郎、紀候郎,正七品,靈臺郎、保章正,正八品,副,從八品,掌曆、管勾,從九品。洪武元年,徵元太史張佑、張沂等十四人,改太史院為司天監,設監令一人,正三品少監二人,正四品監丞一人,正六品主簿一人,正七品主事一人,正八品五官正五人,正五品五官副五人,正六品靈臺郎二人,正七品保章正二人,從七品監候三人,正八品司辰八人,正九品漏刻博士六人。從九品又置回回司天監,設監令一人,正四品少監二人,正五品監丞二人。正六品徵元回回司天監鄭阿里等議歷。三年,改司天監為欽天監。四年,詔監官職專司天,非特旨不得陞調。又定監官散官。監令,正儀大夫;少監,分朔大夫;五官司,司玄大夫;監丞,靈臺郎;五官保章正,平秩郎;五官靈臺郎,司正郎;五官挈壺正,挈壺郎。十四年,改欽天監為正五品,設令一人,丞一人,屬官五官正以下,員數如前所列。俱從品級授以文職散官。二十二年,改令為監正,丞為監副。三十一年,罷回回欽天監,以其歷法隸本監。明初,又置稽疑司,以掌卜筮,未幾罷。洪武十七年,置稽疑司,設司令一人,正六品,左、右丞各一人,從六品,屬官司筮,正九品,無定員。尋罷。

太醫院。院使一人,正五品院判二人。正六品其屬,御醫四人,正八品,後增至十八人,隆慶五年定設十人。生藥庫、惠民藥局,各大使一人,副使一人。

太醫院掌醫療之法。凡醫術十三科,醫官、醫生、醫士,專科肄業:曰大方脈,曰小方脈,曰婦人,曰瘡瘍,曰鍼灸,曰眼,曰口齒,曰接骨,曰傷寒,曰咽喉,曰金鏃,曰按摩,曰祝由。凡醫家子弟,擇師而教之。三年、五年一試、再試、三試,乃黜陟之。凡藥,辨其土宜,擇其良楛,慎其條製而用之。四方解納藥品,院官收貯生藥庫,時其燥濕,禮部委官一員稽察之。診視御脈,使、判、御醫參看校同,會內臣就內局選藥,連名封記藥劑,具本開寫藥性、證治之法以奏。烹調御藥,院官與內臣監視。每二劑合為一,候熟,分二器,一御醫、內臣先嘗,一進御。仍置歷簿,用內印鈐記,細載年月緣由,以憑考察。王府請醫,本院奉旨遣官或醫士往。文武大臣及外國君長有疾,亦奉旨往視。其治療可否,皆具本覆奏。外府州縣置惠民藥局。邊關衛所及人聚處,各設醫生、醫士或醫官,俱由本院試遣。歲終,會察其功過而殿最之,以憑黜陟。

太祖初,置醫學提舉司,設提舉,從五品同提舉,從六品副提舉,從七品醫學教授,正九品學正、官醫、提領。從九品尋改為太醫監,設少監,正四品監丞。正六品吳元年,改監為院,設院使,秩正三品,同知,正四品,院判,正五品,典簿,正七品。洪武三年,置惠民藥局,府設提領,州縣設官醫。凡軍民之貧病者,給之醫藥。六年,置御藥局於內府,始設御醫。御醫局,秩正六品,設尚藥、奉御二人,直長二人,藥童十人,俱以內官、內使充之。設御醫四人,以太醫院醫士充之。凡收受四方貢獻名藥及儲蓄藥品,奉御一人掌之。凡供御藥餌,醫官就內局修制,太醫院官診視。十四年,改太醫院為正五品,設令一人,丞一人,吏目一人。屬官御醫四人,俱如文職授散官。二十二年,復改令為院使,丞為院判。嘉靖十五年,改御藥房為聖濟殿,又設御藥庫,詔御醫輪直供事。

上林苑監。左、右監正各一人,正五品左、右監副各一人,正六品,監正、監副後不常設,以監丞署職。左、右監丞各一人。正七品其屬,典簿廳,典簿一人。正九品良牧、蕃育、林衡、嘉蔬四署,各典署一人,正七品署丞一人,正八品錄事一人。正九品

監正掌苑囿、園池、牧畜、樹種之事。凡禽獸、草木、蔬果,率其屬督其養戶、栽戶,以時經理其養地、栽地而畜植之,以供祭祀、賓客、宮府之膳羞。凡苑地,東至白河,西至西山,南至武清,北至居庸關,西南至渾河,並禁圍獵。良牧,牧牛羊豕,蕃育,育鵝鴨雞,皆籍其牝牡之數,而課孳卵焉。林衡,典果實、花木,嘉蔬,典蒔藝瓜菜,皆計其町畦、樹植之數,而以時苞進焉。

洪武二十五年,議開上林院,度地城南。自牛首山接方山,西並河涯。比圖上,太祖謂有妨民業,遂止。永樂五年,始置上林苑監,設良牧、蕃育、嘉蔬、林衡、川衡、冰鑒及典察左右前後十屬署。洪熙中,併為蕃育、嘉蔬二署。以良牧、川衡併蕃育,冰鑒、林衡併嘉蔬,典察四署分併入。宣德十年,始定四署。正德間,增設監督內臣共九十九員。嘉靖元年,裁汰八十員,革蕃育、嘉蔬二署典署,林衡、嘉蔬二署錄事。

中、東、西、南、北五城兵馬指揮司。各指揮一人,正六品副指揮四人,正七品吏目一人。

指揮,巡捕盜賊,疏理街道溝渠及囚犯、火禁之事。凡京城內外,各畫境而分領之。境內有遊民、姦民則逮治。若車駕親郊,則率夫里供事。凡親、郡王妃父無官者,親王授兵馬指揮,郡王授副指揮,不管事。

明初,置兵馬指揮司,設都指揮、副都指揮、知事。後改設指揮使、副指揮使,各城門設兵馬。洪武元年,命在京兵馬指揮司并管市司,每三日一次校勘街市斛斗、秤尺,稽考牙儈姓名,時其物價。五年,又設兵馬指揮司分司於中都。十年,定京城及中都兵馬指揮司秩俱正六品。先是秩正四品。改為指揮、副指揮,職專京城巡捕等事,革知事。二十三年,定設五城兵馬指揮司,惟中城止稱中兵馬指揮司。俱增設吏目。建文中,改為兵馬司,改指揮、副指揮為兵馬、副兵馬。永樂元年復舊。二年,設北京兵馬指揮司。嘉靖四十一年,詔巡視五城御史,每年終,將各城兵馬指揮會本舉劾。隆慶間,御史趙可懷言:「五城兵馬司官,宜取科貢正途,職檢驗死傷,理刑名盜賊,如兩京知縣。不職者,巡城御史糾劾之。」

順天府。府尹一人,正三品府丞一人,正四品治中一人,正五品通判六人,正六品,嘉靖後革三人。推官一人,從六品儒學教授一人,從九品訓導一人。其屬,經歷司,經歷一人,從七品知事一人。從八品照磨所,照磨一人,從九品檢校一人。所轄,宛平、大興二縣,各知縣一人,正六品縣丞二人,正七品主簿無定員,正八品典史一人。司獄司,司獄一人。從九品都稅司,大使一人,從九品副使一人。宣課司,凡四,正陽門外、正陽門、張家灣、盧溝橋。稅課司,凡二,安定門外、安定門。各大使一人。從九品稅課分司,凡二,崇文門、德勝門。各副使一人。遞軍所、批驗所,各大使一人。

府尹,掌京府之政令。宣化和人,勸農問俗,均貢賦,節征徭,謹祭祀,閱實戶口,糾治豪強,隱恤窮困,疏理獄訟,務知百姓之疾苦。歲立春,迎春、進春,祭先農之神。月朔望,早朝,奏老人坊廂聽宣諭。孟春、孟冬,率其僚屬行鄉飲酒禮。凡勳戚家人文引,每三月一奏。市易平其物價。遇內官監征派物料,雖有印信、揭帖,必補牘面奏。若天子耕耤,行三推禮,則奉青箱播種於後。禮畢,率庶人終畝。府丞,貳京府,兼領學校。治中,參理府事,以佐尹丞。通判,分理糧儲、馬政、軍匠、薪炭、河渠、堤塗之事。推官,理刑名,察屬吏。二縣,職掌如外縣,以近蒞輦下,故品秩特優。

順天府即舊北平府。洪武二年置北平行省。九年改為北平布政司,皆以北平為會府。永樂初,改為順天府。十年,升為府尹,秩正三品,設官如應天府。順天府通判,舊六人,內一人管糧,一人管馬,一人清軍,一人管匠,一人管河,一人管柴炭。嘉靖八年革管河、管柴炭二人。萬曆九年革清軍、管匠二人。十一年復設一人,兼管軍匠。

武學。京衛武學,教授一人,從九品訓導一人。衛武學,教授一人,訓導二人或一人。掌教京衛各衛幼官及應襲舍人與武生,以待科舉、武舉、會舉,而聽於兵部。其無武學者,凡諸武生則隸儒學。

建文四年始置京衛武學,設教授一人。啟忠等十齋,各訓導二人。永樂中罷,正統六年復設。後漸置各衛武軍,設官如儒學之制。

僧錄司。左、右善世二人,正六品左、右闡教二人,從六品左右講經二人,正八品左、右覺義二人。從八品

導錄司。左、右正一二人,正六品左、右演法二人,從六品左、右至靈二人,正八品左、右玄義二人。從八品神樂觀提點一人,正六品知觀一人,從八品,嘉靖中革。龍虎山正一真人一人。正二品。洪武元年,張正常入朝,去其天師之號,封為真人,世襲。隆慶間革真人,止稱提點。萬曆初復之。法官、贊教、掌書各二人。閣皁山、三茅山各靈官一人,正八品太和山提點一人。

僧、道錄司掌天下僧道。在外府州縣有僧綱、道紀等司,分掌其事,俱選精通經典、戒行端潔者為之。神樂觀掌樂舞,以備大祀天地、神祇及宗廟、社稷之祭,隸太常寺,與道錄司無統屬。

洪武元年,立善世、玄教二院。四年革。五年,給僧道度牒。十一年,建神樂觀於郊祀壇西,設提點、知觀。初,提點從六品,知觀從九品。洪武十五年升提點正六品,知觀從八品。凡遇朝會,提點列於僧錄司左善世之下,道錄司左正一之上。十五年,始置僧錄司、道錄司。各設官如前所列。僧凡三等:曰禪,曰講,曰教。道凡二等:曰全真,曰正一。設官不給俸,隸禮部。二十四年,清理釋、道二教,限僧三年一度給牒。凡各府州縣寺觀,但存寬大者一所,併居之。凡僧道,府不得過四十人,州三十人,縣二十人。民年非四十以上、女年非五十以上者,不得出家。二十八年,令天下僧道赴京考試給牒,不通經典者黜之。其後,釋氏有法王、佛子、大國師等封號,道士有大真人、高士,高士等封號,賜銀印蟒玉,加太常卿、禮部尚書及宮保銜,至有封伯爵者,皆一時寵幸,非制也。

教坊司。奉鑾一人,正九品左、右韶舞各一人,左、右司樂各一人,並從九品掌樂舞承應。以樂戶充之,隸禮部。嘉靖中,又設顯陵供祀教坊司,設左、右司樂各一人。

宦官。十二監。每監各太監一員,正四品,左、右少監各一員,從四品,左、右監丞各一員,正五品,典簿一員,正六品,長隨、奉御無定員,從六品。此洪武舊制也。後漸更革,詳見各條下。司禮監,提督太監一員,掌印太監一員,秉筆太監、隨堂太監、書籍名畫等庫掌司、內書堂掌司、六科郎掌司、典簿無定員。提督掌督理皇城內一應儀禮刑名,及鈐束長隨、當差、聽事各役,關防門禁,催督光祿供應等事。掌印掌理內外章奏及御前勘合。秉筆、隨堂掌章奏文書,照閣票批硃。掌司各掌所司。典簿典記奏章及諸出納號簿。內官監,掌印太監一員,總理、管理、僉書、典簿、掌司、寫字、監工無定員,掌木、石、瓦、土、塔材、東行、西行、油漆、婚禮、火藥十作,及米鹽庫、營造庫、皇壇庫,凡國家營造宮室、陵墓,并銅錫妝奩、器用暨冰窨諸事。御用監,掌印太監一員,裏外監把總二員,典簿、掌司、寫字、監工無定員。凡御前所用圍屏、床榻諸木器,及紫檀、象牙、烏木、螺甸諸玩器,皆造辦之。又有仁智殿監工一員,掌武英殿中書承旨所寫書籍畫冊等,奏進御前。司設監,員同內官監,掌鹵簿、儀仗、帷幕諸事。御馬監,掌印、監督、提督太監各一員。騰驤四衛營各設監官、掌司、典簿、寫字、拿馬等員。象房有掌司等員。神宮監,掌印太監一員,僉書、掌司、管理無定員,掌太廟各廟灑掃、香燈等事。尚膳監,掌印太監一員,提督光祿太監一員,總理一員,管理、僉書、掌司、寫字、監工及各牛羊等房廠監工無定員,掌御膳及宮內食用並筵宴諸事。尚寶監,掌印一員,僉書、掌司無定員,掌寶璽、敕符、將軍印信。凡用寶,外尚寶司以揭帖赴監請旨,至女官尚寶司領取,監視外司用訖,存號簿,繳進。印綬監,員同尚寶,掌古今通集庫,并鐵券、誥敕、貼黃、印信、勘合、符驗、信符諸事。直殿監,員同上,掌各殿及廊廡掃除事。尚衣監,掌印太監一員,管理、僉書、掌司、監工無定員,掌御用冠冕、袍服及屨舄、靴襪之事。都知監。掌印太監一員,僉書、掌司、長隨、奉御無定員,舊掌各監行移、關知、勘合之事,後惟隨駕前導警蹕。

四司。舊制每司各司正一人,正五品;左、右司副各一人,從五品。後漸更易,詳下。惜薪司,掌印太監一員,總理、僉書、掌道、掌司、寫字、監工及外廠、北廠、南廠、新南廠、新西廠各設僉書、監工,俱無定員,掌所用薪炭之事。鐘鼓司,掌印太監一員,僉書、司房、學藝官無定員,掌管出朝鐘鼓,及內樂、傳奇、過錦、打稻諸雜戲。寶鈔司,掌印太監一員,僉書、管理、監工無定員,掌造粗細草紙。混堂司。掌印太監一員,僉書、監工無定員,掌沐浴之事。

八局。舊制每局大使一人,正五品;左、右副使各一人,從五品。兵仗局,掌印太監一員,提督軍器庫太監一員,管理、僉書、掌司、寫字、監工無定員,掌製造軍器。火藥司屬之。銀作局,掌印太監一員,管理、僉書、寫字、監工無定員,掌打造金銀器飾。浣衣局,掌印太監一員,僉書、監工無定員。凡宮人年老及罷退廢者,發此局居住。惟此局不在皇城內。巾帽局,掌印太監一員,管理、僉書、掌司、監工無定員,掌宮內使帽靴,駙馬冠靴及籓王之國諸旗尉帽靴。鍼工局,員同巾帽局,掌造宮中衣服。內織染局,員同上,掌染造御用及宮內應用緞匹。城西藍靛廠為此局外署。酒醋面局,員同上,掌宮內食用酒醋、糖醬、面豆諸物。與御酒房不相統轄。司苑局。員同上,掌蔬菜、瓜果。

十二監、四司、八局,所謂二十四衙門也。

其外有內府供用庫,掌印太監一員,總理、管理、掌司、寫字、監工無定員。掌宮內及山陵等處內官食米及御用黃蠟、白蠟、沉香等香。凡油蠟等庫俱屬之。舊制各庫設官同八局。司鑰庫,員同上,掌收貯制錢以給賞賜。內承運庫,掌印太監一員,近侍、僉書太監十員,掌司、寫字、監工無定員。掌大內庫藏,凡金銀及諸寶貨總隸之。十庫,甲字,掌貯銀硃、黃丹、烏梅、藤黃、水銀諸物。乙字,掌貯奏本等紙及各省所解胖襖。丙字,掌貯絲綿、布匹。丁字,掌貯生漆、桐油等物。戊字,掌貯所解弓箭、盔甲等物。承運,掌貯黃白生絹。廣盈,掌貯紗羅諸帛匹。廣惠,掌造貯巾帕、梳籠、刷抿、錢貫、鈔錠之類。贓罰,掌沒入官物。已上各掌庫一員,貼庫、僉書無定員。御酒房,提督太監一員,僉書無定員。掌造御用酒。御藥房,提督太監正、副二員,分兩班。近侍、醫官無定員。職掌御用藥餌,與太醫院官相表裏。御茶房,提督太監正、副二員,分兩班。近侍無定員。職司供奉茶酒、瓜果及進御膳。牲口房,提督太監一員,僉書無定員。收養異獸珍禽。刻漏房,掌房一員,僉書無定員。掌管每日時刻,每一時即令直殿監官入宮換牌,夜報刻水。更鼓房,有罪內官職司之。甜食房,掌房一員,協同無定員。掌造辦虎眼、窩絲等糖及諸甜食,隸御用監。彈子房,掌房一員,僉書數員。專備泥彈。靈臺,掌印太監一員,僉書近侍、看時近侍無定員。掌觀星氣雲物,測候災祥。條作,掌作一員,協同無定員。掌造各色兜羅絨及諸絳綬,隸御用監。盔甲廠,即舊鞍轡局,掌造軍器。安民廠,舊名王恭廠,各掌廠太監一員,貼廠、僉書無定員。掌造銃砲、火藥之類。午門,東華門,西華門,奉天門,玄武門,左、右順門,左、右紅門,皇宮門,坤寧門,宮左、右門。東宮春和門,後門,左、右門,皇城、京城內外諸門,各門正一員,管事無定員。司晨昏啟閉,關防出入。舊設門正、門副各一員。提督東廠,掌印太監一員,掌班、領班、司房無定員。貼刑二員,掌刺緝刑獄之事。舊選各監中一人提督,後專用司禮、秉筆第二人或第三人為之。其貼刑官,則用錦衣衛千百戶為之。凡內官司禮監掌印,權如外廷元輔;掌東廠,權如總憲。秉筆、隨堂視眾輔。各設私臣掌家、掌班、司房等員。提督西廠,不常設,惟汪直、穀大用置之。劉瑾又設西內廠。尋俱罷革。提督京營,提督太監,坐營太監,監槍、掌司、僉書俱無定員。始於景泰元年。文書房,掌房十員。掌收通政司每日封進本章,并會極門京官及各籓所上封本,其在外之閣票,在內之搭票,一應聖諭旨意御批,俱由文書房落底簿發。凡升司禮者,必由文書房出,如外廷之詹、翰也。禮儀房,提督太監一員,司禮、掌印或秉筆攝之,掌司、寫字、管事、長隨無定員。掌一應選婚、選駙馬、誕皇太子女、選擇乳婦諸吉禮。中書房,掌房一員,散官無定員。掌管文華殿中書所寫書籍、對聯、扇柄等件,承旨發寫,完日奏進。御前近侍,曰乾清宮管事,督理御用諸事,曰打卯牌子,掌隨朝捧劍,俱位居司禮、東廠提督守備之次。曰御前牌子,曰暖殿,曰管櫃子,曰贊禮,曰答應長隨,曰當差聽事,曰拿馬,尚冠、尚衣、尚履,皆近侍也。南京守備,正、副守備太監各一員。關防一顆,護衛留都,為司禮監外差。天壽山守備,太監一員。轄各陵守陵太監,職司護衛。湖廣承天府守備,太監一員。轄承德、荊、襄地方,護衛興寧。織造,提督太監南京一員,蘇州一員,杭州一員。掌織造御用龍衣。鎮守,鎮守太監始於洪熙,遍設於正統,凡各省各鎮無不有鎮守太監,至嘉靖八年後始革。市舶,廣東、福建、浙江三市舶司各設太監提督,後罷浙江、福建二司,惟存廣東司。監督倉場,各倉、各場俱設監督太監。諸陵神宮監,各陵俱設神宮監太監守陵。其外之監軍、採辦、糧稅、礦關等使,不常設者,不可勝紀也。

初,吳元年置內史監,設監令,正四品丞,正五品,奉御,從五品內史,正七品典簿。正八品皇門官設皇門使,正五品副。從五品後改置內使監、御用監,各設令一人,正三品丞二人,從三品奉御,正六品典簿。正七品皇門官門正,正四品副,從四品春宮門官正,正五品副,從五品御馬司司正,正五品副,從五品尚寶兼守殿、尚冠、尚衣、尚佩、尚履、尚藥、紀事等奉御。俱正六品洪武二年,定置內使監奉御六十人,尚寶一人,尚冠七人,尚衣十人,尚佩九人,尚藥七人,紀事二人,執膳四人,司脯二人,司香四人,太廟司香四人,涓潔二人。置尚酒、尚醋、尚面、尚染四局,局設正一人,副二人。置御馬、御用二司,司設正一人,副二人。內府庫設大使一人,副使二人。內倉監設令一人,丞二人。及置東宮典璽、典翰、典膳、典服、典藥、典乘兵六局,局設局郎一人,丞一人。又置門官,午門等十三門,各設門正一人,副一人。東宮門官,春和門等四門,各設門正一人,副一人。三年,置王府承奉司。設承奉一人,承奉副二人,典寶、典服、典膳三所,各設正一人,副一人,門官設門正一人,副一人。改內使監、御用監,秩皆從三品,令從三品,丞正四品。皇門官秩從四品。門正從四品,副正五品,春宮門官正、副同。四年,復悉差其品秩,授以散官。乃改內使監為正五品,皇門官為正六品。洪武四年,定內官散官。正四品,中正大夫。從四品,中侍大夫。正五品,中衛大夫。從五品,侍直大夫。正六品,內侍郎。從六品,內直郎。正七品,正奉郎。從七品,正衛郎。正八品,司奉郎。從八品,司直郎。尋定內使監令。正五品,授中衛大夫。丞,從五品,授侍直大夫。皇門正、局正、司正、東宮門正、局正,俱正六品,授內侍郎。尚寶、奉御、皇門副、局副、司副、東宮門副、局丞,王府承奉、門正、所正,俱從六品,授內直郎。尚冠等奉御、內府庫大使、內倉監令、王府承奉副、門副、所副,俱正七品,授正奉郎。庫副使、倉丞,俱從七品,授正衛郎。六年,改御用監為供奉司,秩從七品,設官五人。內倉監為內府倉,以監令為大使,監丞為副使。內府庫為承運庫。仍設大使、副使。尋置紀事司,以宦者張翊為司正。秩正七品。又考前代糾劾內官之法,置內正司,設司正一人,正七品司副一人,從七品專糾內官失儀及不法者。旋改為典禮司,又改為典禮紀察司,陞其品秩。司正升正六品,司副升從六品。十年,置神宮內使監,設監令,正五品丞,從五品,司香奉御,正七品典簿。從九品天地壇、神壇各祠祭署,設署令,正七品丞,從七品,司香奉御。正八品甲、乙、丙、丁、戊五庫,各設大使,正七品副使,從七品及皇城門官端門等十六門,各設門正,正七品副,從七品十二年,更置尚衣、尚冠、尚履三監,針工、皮作、巾帽三局。改尚佩局為尚佩監。十六年,置內府寶鈔廣源、廣惠二庫,職掌出納楮幣,入則廣源庫掌之,出則廣惠庫掌之。寶鈔廣源庫,設大使一人,正九品,用流官;副使一人,正九品,用內官。寶鈔廣惠庫,設大使二人,從九品;副使二人,從九品。俱流官、內官兼用。十七年,更定內官諸監、庫、局品職。內官監,設令一人,正六品丞二人,從六品典簿一人。正九品神宮監,設令一人,正七品丞一人,從七品奉御一人,正八品。尚寶監,設令一人正七品丞一人。從七品尚衣監,設令一人,正七品丞一人。從七品奉御四人。正八品尚膳監,設令一人,正七品丞一人。從七品司設監,設令一人,正七品丞一人,從七品奉御四人,正八品司禮監,設令一人,正七品丞一人,從七品御馬監,設令一人,正七品丞一人,從七品直殿監,設令一人,正七品丞四人,從七品小內使十五人。宮門承制,設奉御五人。正八品宮門守門官,設門正一人,正八品副四人。從八品內承運庫,設大使一人,正九品副使二人。從九品司鑰庫,設大使一人,正九品副使四人。從九品巾帽局,設大使一人,正九品副使一人,從九品針工局,設大使一人,正九品副使一人,從九品織染局,設大使一人,正九品副使一人。從九品顏料局,設大使一人。正九品司苑局,設大使一人。正九品司牧局,設大使一人。正九品皆於內官內選用。二十八年,重定內官監、司、庫、局與諸門官,并東宮六局、王府承奉等官職秩。凡內官監十一:曰神宮監,曰尚寶監,曰孝陵神宮監,曰尚膳監,曰尚衣監,曰司設監,曰內官監,曰司禮監,曰御馬監,曰印綬監,曰直殿監,皆設太監一人,正四品左、右少監各一人,從四品左、右監丞各一人,正五品典簿一人,正六品又設長隨、奉御。正六品各門官七:午門、東華門、西華門、玄武門、奉天門、左順門、右順門,皆設門正一人,正四品門副一人。從四品司二:曰鐘鼓司,曰惜薪司,皆設司正一人,正五品,左、右司副各一人。從五品局庫九:曰兵仗局,曰內織染局,曰針工局,曰巾帽局,曰司苑局,曰酒醋面局,曰內承運庫,曰司鑰庫,曰內府供用庫。每局庫皆設大使一人,正五品左、右副使各一人。從五品。東宮典璽、典藥、典膳、典服、典兵、典乘六局,各設局郎一人,正五品局丞二人,從五品惟典璽局增設紀事、奉御。正六品親王府承奉司設承奉正,正六品承奉副。從六品所三:曰典寶所,設典寶正一人,正六品副一人。從六品曰典膳所,設典膳正一人,正六品副一人。從六品曰典服所,設典服正一人,正六品副一人。從六品門官,設門正一人,正六品門副一人。從六品又設內使十人,司冠一人,司衣三人,司佩一人,司履一人,司藥二人,司矢二人。各公主位下設中使司,司正、司副各一人。三十年,置都知監,設太監一人,正四品左、右少監各一人,從四品左、右監丞各一人,正五品典簿一人。正六品又置銀作局,設大使一人,正五品副使一人。從五品

太祖嘗謂侍臣曰:「朕觀《周禮》,奄寺不及百人。後世至踰數千,因用階亂。此曹止可供灑掃,給使令,非別有委任,毋令過多。」又言:「此曹善者千百中不一二,惡者常千百。若用為耳目,即耳目蔽;用為心腹,即心腹病。馭之之道,在使之畏法,不可使有功。畏法則檢束,有功則驕恣。」有內侍事帝最久,微言及政事,立斥之,終其身不召。因定制,內侍毋許識字。洪武十七年鑄鐵牌,文曰:「內臣不得干預政事,犯者斬」,置宮門中。又敕諸司毋得與內官監文移往來。然二十五年命聶慶童往河州敕諭茶馬,中官奉使行事已自此始。成祖亦嘗云:「朕一遵太祖訓,無御寶文書,即一軍一民,中官不得擅調發。」有私役應天工匠者,立命錦衣逮治。顧中官四出,實始永樂時。元年,李興等齎敕勞暹羅國王,此奉使外國之始也。三年,命鄭和等率兵二萬,行賞西洋古里、滿剌諸國,此將兵之始也。八年,敕王安等監都督譚青等軍,馬靖巡視甘肅,此監軍、巡視之始也。及洪熙元年,以鄭和領下番官軍守備南京,遂相沿不改。敕王安鎮守甘肅,而各省鎮皆設鎮守矣。宣德四年,特設內書堂,命大學士陳山專授小內使書,而太祖不許識字讀書之制,由此而廢。賜王瑾、金英印記,則與諸密勿大臣同。賜金英、范弘等免死詔,則又無異勳臣之鐵券也。英之王振,憲之汪直,武之劉瑾,熹之魏忠賢,太阿倒握,威福下移。神宗礦稅之使,無一方不罹厥害。其他怙勢薰灼,不可勝紀。而蔭弟、蔭姪、封伯、封公,則撓官制之大者。莊烈帝初翦大憝,中外頌聖。既而鎮守、出征、督餉、坐營等事,無一不命中官為之,而明亦遂亡矣。

女官。六局。尚宮局,尚宮二人,正五品。六尚並同。尚宮掌導引中宮。凡六局出納文籍,皆印署之。若征辦於外,則為之請旨,牒付內官監。監受牒,行移於外。領司四:司記,司記二人,正六品;典記二人,正七品;掌記二人,正八品。掌宮內諸司簿書,出入錄目,番署加印,然後授行。女史六人,掌執文書,凡二十四司,二十四典,二十四掌,品秩並同。司言,司言二人,典言二人,掌言二人,女史四人,掌宣傳啟奏。凡令節外命婦朝賀中宮,司言傳旨。司簿,司簿二人,典簿二人,掌簿二人,女史六人,掌宮人名籍及廩賜之事。司闈。司闈六人,典闈六人,掌闈六人,女史四人,掌宮闈管鍵之事。尚儀局,尚儀二人,掌禮儀起居事。領司四:司籍,司籍二人,典籍二人,掌籍二人,女史十人,掌經籍、圖書、筆札、幾案之事。司樂,司樂四人,典樂四人,掌樂四人,女史二人,掌音樂之事。司賓,司賓二人,典賓二人,掌賓二人,女史二人,掌朝見、宴會、賜賚之事。司贊,司贊二人,典贊二人,掌贊二人,女史二人,掌朝見、宴會、贊相之事。彤史。彤史二人,正六品,掌宴見進御之事,凡后妃、群妾御於君所,彤史謹書其月日。尚服局,尚服二人,掌供服用採章之數。領司四:司寶,司寶二人,典寶二人,掌寶二人,女史四人,掌寶璽、符契。司衣,司衣二人,典衣二人,掌衣二人,女史四人,掌衣服、首飾之事。司飾,司飾二人,典飾二人,掌飾二人,女史二人,掌巾櫛、膏沐之事。司仗,司仗二人,典仗二人,掌仗二人,女史二人,凡朝賀,帥女官擎執儀仗。尚食局,尚食二人,掌膳羞品齊之數。凡以飲食進御,尚食先嘗之。領司四:司膳,司膳四人,典膳四人,掌膳四人,女史四人,掌割烹煎和之事。司醞,司醞二人,典醞二人,掌醞二人,女史二人,掌酒醴酏飲之事。司藥,司藥二人,典藥二人,掌藥二人,女史四人,掌醫方藥物。司饎。司饎二人,典饎二人,掌饎二人,掌廩餼薪炭之事。尚寢局,尚寢二人,掌天子之宴寢。領司四:司設,司設二人,典設二人,掌設二人,女史四人,掌床帷、茵席、汛掃、張設之事。司輿,司輿二人,典輿二人,掌輿二人,女史二人,掌輿輦、傘扇之事。司苑,司苑二人,典苑二人,掌苑二人,女史四人,掌園囿種值花果。司燈。司燈二人,典燈二人,掌燈二人,女史二人,掌燈燭事。尚功局,尚功二人,掌督女紅之程課。領司四:司製,司製二人,典製二人,掌製二人,女史四人,裳衣服裁製縫紉之事。司珍,司珍二人,典珍二人,掌珍二人,女史六人,掌金玉寶貨。司彩,司彩二人,典彩二人,掌彩二人,女史六人,掌繪綿絲絮事。司計,司計二人,典計二人,掌計二人,女史四人,掌度支衣服、飲食、柴炭之事。宮正司。宮正一人,正五品;司正二人,正六品;典正二人,正七品。掌糾察宮闈、戒令、謫罰之事。大事則奏聞。女史四人,記功過。

吳元年,置內職六尚局。洪武五年,定為六局一司。局曰尚宮,曰尚儀,曰尚服,曰尚食,曰尚寢,曰尚功。司曰宮正。尚宮二人,尚儀、尚服、尚食、尚寢、尚功各一人,宮正二人,俱正六品。六局分領二十四司,每司或二人或四人。司記、司言、司簿、司樂、司寶、司衣、司飾、司醞、司藥、司供、司輿、司苑、司珍、司彩、司計各二人。司闈、司籍、司賓、司贊、司仗、司饌、司設、司燈、司制各四人。女史十八人。尚功局六人,餘五局及宮正局各二人。十七年,更定品秩。尚宮、尚儀、尚服、尚食、尚寢、尚功、宮正各一人,俱改正五品;二十四司正六品。增設二十四掌,正七品。宮正司增設司正,正六品。二十二年,授宮官敕。服勞多者,或五載六載,得歸父母,聽婚嫁。年高者許歸,願留者聽。現授職者,家給與祿。二十七年,又重定品職。增設二十四典,正七品。改二十四掌為正八品。尚儀局增設彤史,正六品。宮正司增設典正,正七品。自六尚以下,員數俱如前所列。凡宮官一百八十七人,女史九十六人。六局各鑄印給之。永樂後,職盡移於宦官。其宮官所存者,惟尚寶四司而已。

南京宗人府吏部戶部附總督糧儲禮部兵部刑部工部都察院附提督操江通政司大理寺詹事府翰林院國子監太常寺光祿寺太僕寺鴻臚寺尚寶司六科行人司欽天監太醫院五城兵馬司應天府附上元江寧二縣已上南京官王府長史司布政司按察司各道行太僕寺苑馬寺都轉運鹽使司鹽課提舉司市舶提舉司茶馬司府州縣儒學巡檢司驛稅課司倉庫織染局河泊所附閘壩官批驗所遞運所鐵冶所醫學陰陽學僧綱司道紀司

南京宗人府。經歷司,經歷一人。南京官品秩,俱同北京。

吏部。尚書一人,右侍郎一人。六部侍郎,至弘治後始專設右。萬曆三年俱革。十一年復設。天啟中,每部增侍郎一人。崇禎間革。其屬,司務廳,司務一人。文選、考功、驗封、稽勳四清吏司,各郎中一人,主事一人。驗封、稽勛二司主事,後並革。凡南京官,六年考察,考功掌之,不由北吏部。

戶部。尚書一人,右侍郎一人,司務一人,照磨一人。十三司,郎中十三人,員外郎九人,浙江、江西、湖廣、廣東、廣西、福建、山西、陜西、雲南九司各一人,嘉靖三十七年,革山西、陜西三司員外郎各一人,隆慶中又革廣西、雲南二司員外郎各一人。主事十七人,山西、廣東、廣西、雲南四司各二人,隆慶三年革廣東司主事一人。所轄,寶鈔提舉司,提舉一人。廣積庫、承運庫、贓罰庫、甲乙丙丁戊五字庫、寶鈔廣惠庫、軍儲倉,各大使一人。長安門倉、東安門倉、西安門倉、北安門倉各副使一人。龍江鹽倉檢校批驗所,大使一人。隆慶三年,革寶鈔司提舉、軍儲倉大使。

總督糧儲一人。嘉靖以前,特設都御史。二十六年革,以戶部右侍郎加都御史銜領之。

禮部。尚書一人,右侍郎一人,司務一人。儀制、祠祭、主客、精膳四司,各郎中一人。儀制、祠祭二司,各主事一人。所轄,鑄印局,副使一人。教坊司,右韶舞一人,左右司樂各一人。

兵部。尚書參贊機務一人,右侍郎一人,司務一人。武選、職方、車駕、武庫四司,郎中四人,員外郎二人,武選、武庫無員外郎。主事五人。車駕主事二人。所轄,典牧所,提領一人。正八品會同館、大勝關,各大使一人。按參贊機務,自宣德八年黃福始。成化二十三年,始奉敕諭,專以本部尚書參贊機務,同內外守備官操練軍馬,撫恤人民,禁戢盜賊,振舉庶務,故其職視五部為特重云。

刑部。尚書一人,右侍郎一人,司務、照磨各一人。十三司郎中十三人,員外郎五人,惟浙江、江西、河南、陜西、廣東五司設。主事十四人,廣東司二人。分掌南京諸司,及公、侯、伯、五府、京衛所刑名之事。司獄二人。

工部。尚書一人,右侍郎一人,司務一人。營繕、虞衡、都水、屯田四司,郎中四人,員外郎二人,營繕司一人,都水司一人,嘉靖三十七年,革都水員外郎。主事八人。營繕司三人,屯田司一人,餘各二人。所轄,營繕所,所正、所副、所丞各一人。龍江、清江二提舉司,各提舉一人。副提舉後革。文思院、寶源局、軍器局、織染所、龍江抽分竹木局、瓦屑壩抽分竹木局,各大使一人。嘉靖三十七年,革文思院大使。

都察院。右都御史一人,右副都御史一人,右僉都御史一人,司務、經歷、都事、照磨各一人,司獄二人。嘉靖三十七年,革司獄一人。隆慶四年,革都事。浙江、江西、河南、山東、山西、陜西、四川、雲南、貴州九道,各御史二人。福建、湖廣、廣東、廣西四道,各御史三人。嘉靖後不全設,恆以一人兼數道。凡刷卷、巡倉、巡江、巡城、屯田、印馬、巡視糧儲、監收糧斛、點閘軍士、管理京營、比驗軍器,皆敘而差之。清軍,則偕兵部、兵科。核後湖黃冊,則偕戶部、戶科。

提督操江一人。以副僉都御史為之,領上、下江防之事。

通政使司。通政使一人,右通政一人,右參議一人,掌收呈狀,付刑部審理。經歷一人。

大理寺。卿一人,右寺丞一人,司務一人,左、右寺正各一人,左、右評事各三人。隆慶三年,革左、右評事各一人。

詹事府。主簿一人。

翰林院。學士一人,不常置,以翰林坊局官署職。孔目一人。

國子監。祭酒一人,司業一人,監丞一人,典簿一人,博士三人,助教六人,學正五人,學錄二人,典籍一人,學饌一人。嘉靖三十七年,革助教二人及掌饌。隆慶四年,革博士一人,學正一人。

太常寺。卿一人,少卿一人,典簿一人,博士一人,協律郎二人,贊禮郎七人,嘉靖中,革贊禮郎一人。司樂二人。各祠祭署合奉祀八人,祀丞七人。天、地壇奉祀一、祀丞一。山川壇、耤田奉祀一。祖陵奉祀、祀丞各一。皇陵奉祀、祀丞各二。孝陵、揚王墳、徐王墳各奉祀一,祀丞一。嘉靖後,革天地壇、祖陵、揚王墳三祠祭署祀丞。

光祿寺。卿一人,少卿一人,隆慶四年,革少卿。典簿一人。大官、珍羞、良醞、掌醢四署,各署正一人,署丞一人。嘉靖中,革良醞、掌醢二署署丞。萬歷中,革珍羞署丞。

太僕寺。卿一人,少卿二人,寺丞二人,隆慶中,革少卿一人,寺丞一人。主簿一人。

鴻臚寺。卿一人,主簿一人。司儀、司賓二署,各署丞一人,鳴贊四人,序班九人。

尚寶司。卿一人。

吏、戶、禮、兵、刑、工六科。給事中六人。又戶科給事中一人,管理後湖黃冊。

行人司。左司副一人。

欽天監。監正一人,監副一人,主簿一人。五官正一人,五官靈臺郎二人,五官監候一人,五官司曆一人。

太醫院。院判一人,吏目一人。惠民藥局、生藥庫,各大使一人。

五城兵馬司。指揮各一人,副指揮各三人,吏目各一人。萬曆中,革副指揮每城二人。

應天府。府尹一人,府丞一人,治中一人,通判二人,推官一人,經歷、知事、照磨、檢校各一人。儒學教授一人,訓導六人。所轄,上元、江寧二縣,各知縣一人,縣丞一人,主簿一人,典史一人。司獄司,司獄一人。織染局,大使一人,左、右副使各一人。都稅司、宣課司,凡四,龍江、江東、聚寶門、太平門。稅課局,凡二,龍江、龍潭。各大使一人,副使或一人或二人。龍江遞運所,大使、副使各一人。批驗所,大使一人。河泊所,官一人。龍江關、石灰山關,各大使一人,副使四人。洪武三年,改應天府知府為府尹,秩正三品,賜銀印。十三年,始立儒學。

南京官,自永樂四年成祖往北京,置行部尚書,備行在九卿印以從。是時,皇太子監國,大小庶務悉以委之。惟封爵、大辟、除拜三品以上文武職,則六科都給事中以聞,政本故在南也。十八年,官屬悉移而北,南京六部所存惟禮、刑、工三部,各一侍郎,在南之官加「南京」字於職銜上。仁宗時補設官屬,除「南京」字。正統六年,定制復如永樂時。

王府長史司。左、右長史各一人。正五品其屬,典簿一人,正九品所轄,審理所,審理正一人,正六品副一人,正七品典膳所,典膳正一人,正八品副一人,從八品奉祠所,奉祠正一人,正八品副一人,從八品典樂一人,正九品典寶所,典寶正一人,正八品副一人,從八品紀善所,紀善二人,正八品良醫所,良醫正一人,正八品副一人,從八品典儀所,典儀正一人,正九品副一人,從九品工正所,工正一人,正八品副一人,從八品。以上各所副官,嘉靖四十四年並革。伴讀四人,從九品,後止設一人。教授無定員,從九品引禮舍二人,後革二人。倉大使、副使各一人,庫大使、副使各一人。倉、庫副使後俱革。郡王府,教授一人,從九品典膳一人。正八品鎮國將軍教授一人。從九品

長史,掌王府之政訟,輔相規諷以匡王失,率府僚各供乃事,而總其庶務焉。凡請名、請封、請婚、請恩澤,及陳謝、進獻表啟、書疏,長史為王奏上。若王有過,則詰長史。曾經過犯之人,毋得選用是職。審理,掌推按刑獄,禁詰橫暴,無干國紀。典膳,掌祭祀、賓客,王若妃之膳羞。奉祠,掌祭祀樂舞。典寶,掌王寶符牌。紀善,掌諷導禮法,開諭古誼,及國家恩義大節,以詔王善。良醫,掌醫。典儀,掌陳儀式。工正,掌繕造修葺宮邸、廨舍。伴讀,掌侍從起居,陳設經史。教授,掌以德義迪王,校勘經籍。凡宗室年十歲以上,入宗學,教授與紀善為之師。引禮,掌接對賓客,贊相威儀。

洪武三年,置王相府,左、右相各一人,正二品左、右傅各一人。從二品參軍府,參軍一人,正五品錄事二人,正七品紀善一人。正七品各以其品秩列朝官之次。又置典簽司、諮議官。尋以王府武相皆勳臣,令居文相上,王相府官屬仍與朝官更互除授。是年置王府教授。四年,更定官制。左、右相,正二品,文武傅,從二品,參軍,從五品,錄事,正七品,審理正,正六品,副,正七品,紀善,正七品,各署典祠正、典寶正、典儀正、典膳正、典服正、工正、醫正,並正七品,副,並從七品,牧正,正八品,副,從八品,引禮舍人,省注。九年,改參軍為長史,罷王傅府及典簽司、諮議官,增設伴讀四人,選老成明經慎行之士任之,侍讀四人,收掌文籍,少則缺之。尋改王相府所屬奉祠、典寶、典膳、良醫、工正各所正並紀善俱正八品,副,從八品。十三年,并罷王相府,升長史司為正五品,置左、右長史各一人,典簿一人,定王府孳牲所、倉庫等官俱為雜職。二十八年,置靖江王府諮議所,諮議、記室、教授各一人。建文中,增置親王賓輔二人,伴讀、伴講、伴書各一人,長史三人。郡王賓友二人,教授一人,記室二人,直史一人,左、右直史各一人,吏目一人,典印、典祠、典禮、典饌、典藥五署官各一人,典儀二人,引禮舍人二人,儀仗司,吏目一人。其賓輔、三伴、賓友、教授進見時,侍坐,稱名而不稱臣,禮如賓師。成祖初,復舊制,改靖江王府諮議所為長史司。萬曆間,周府設宗正一人。後各府亦漸置。郡王府增設教授一人。又洪武七年,公主府設家令一人,正七品司丞一人,正八品錄事一人。正九品二十三年,改家令司為中使司,以內使為之。

承宣布政使司。左、右布政使各一人,從二品左、右參政,從三品左、右參議,無定員。從四品。參政、參議因事添設,各省不等,詳諸道。經歷司,經歷一人,從六品都事一人。從七品照磨所,照磨一人,從八品檢校一人。正九品理問所,理問一人,從六品副理問一人,從七品提控案牘一人。司獄司,司獄一人,從九品庫大使一人,從九品副使一人。倉大使一人。從九品副使一人。雜造局、軍器局、寶泉局、織染局,各大使一人,從九品副使一人。所轄衙門各省不同,詳見雜職。

布政使,掌一省之政,朝廷有德澤、禁令,承流宣播,以下於有司。凡僚屬滿秩,廉其稱職、不稱職,上下其考,報撫、按以達於吏部、都察院。三年,率其府州縣正官朝覲京師,以聽察典。十年,會戶版以登民數、田數。賓興貢,合省之士而提調之。宗室、官吏、師生、軍伍,以時班其祿俸、廩糧。祀典神祗,謹其時祀。民鰥寡孤獨者養之,孝弟貞烈者表揚之,水旱疾疫災祲,則請於上蠲振之。凡貢賦役,視府州縣土地人民豐瘠多寡而均其數。凡有大興革及諸政務,會都、按議,經畫定而請於撫、按若總督。其國慶國哀,遣僚貳朝賀弔祭於京師。天子即位,則左布政使親至。參政、參議分守各道,及派管糧儲、屯田、清軍、驛傳、水利、撫民等事,併分司協管京畿。兩京不設布、按,無參政,參議、副使、僉事,故於旁近布、按分司帶管,詳見各道。經歷、都事,典受發文移,其詳巡按、巡鹽御史文書,用經歷印。照磨、檢校典勘理卷宗。理問典刑名。

初,太祖下集慶,自領江南行中書省。戊戌,置中書分省於婺州。後每略定地方,即置行省,其官自平章政事以下,大略與中書省同。設行省平章政事,從一品左、右丞,正二品參知政事。從二品左、右司,郎中,從五品員外郎,從六品都事、檢校,從七品照磨、管勾。從八品理問所,正理問,正四品副理問,正五品,知事,從八品尋改知事為提控案牘。省注洪武九年,改浙江、江西、福建、北平、廣西、四川、山東、廣東、河南、陜西、湖廣、山西諸行省俱為承宣布政使司,罷行省平章政事,左、右丞等官,改參知政事為布政使,秩正二品,左、右參政,從二品,改左、右司為經歷司。十三年改布政使,正三品,參政,從三品。十四年,增置左、右參議,正四品。尋增設左、右布政使各一人。十五年,置雲南布政司。二十二年,定秩從二品。建文中,陞正二品,裁一人。成祖復舊制。永樂元年以北平布政司為北京。五年,置交阯布政司。十一年,置貴州布政司。止設使一人,餘官如各布政司。宣德三年,罷交址布政司,除兩京外,定為十三布政司。初置籓司,與六部均重。布政使入為尚書、侍郎,副都御史每出為布政使。宣德、正統間猶然,自後無之。

提刑按察使司。按察使一人,正三品副使,正四品僉事無定員。正五品。詳見諸道。經歷司,經歷一人,正七品知事一人。正八品照磨所,照磨一人,正九品檢校一人。從九品司獄司,司獄一人,從九品

按察使,掌一省刑名按劾之事。糾官邪,戢奸暴,平獄訟,雪冤抑,以振揚風紀,而澄清其吏治。大者暨都、布二司會議,告撫、按,以聽於部、院。凡朝覲慶弔之禮,具如布政司。副使、僉事,分道巡察,其兵備、提學、撫民、巡海、清軍、驛傳、水利、屯田、招練、監軍,各專事置,併分員巡備京畿。

明初,置提刑按察司。吳元年,置各道按察司,設按察使,正三品,副使,正四品,僉事,正五品。十三年,改使秩正四品,尋罷。十四年復置,并置各道按察分司。十五年,又置天下府州縣按察分司。以儒士王存中等五百三十一人為試僉事,人按二縣。凡官吏賢否、軍民利病,皆得廉問糾舉。十六年,盡罷試僉事,改按察使為從三品,副使二人,從四品,僉事從五品,多寡從其分道之數。二十二年,復定按察使為正三品。二十九年,改置按察分司為四十一道。直隸六:曰淮西道,曰淮東道,曰蘇松道,曰建安徽寧道,曰常鎮道,曰京畿道。浙江二:曰浙東道,曰浙西道,四川三:曰川東道,曰川西道,曰黔南道。山東三:曰濟南道、曰海右道,曰遼海東寧道。河南二:曰河南道,曰河北道。北平二:曰燕南道,曰燕北道。陜西五:曰關內道,曰關南道,曰河西道,曰隴右道,曰西寧道。山西三:曰冀寧道,曰冀北道,曰河東道。江西三:曰嶺北道,曰兩江道,曰湖東道,廣東三:曰嶺南道,曰海南道,曰海北道。廣西三:曰桂林蒼梧道,曰左江道,曰右江道。福建二:曰建寧道,曰福寧道。湖廣四:曰武昌道,曰荊南道,曰湖南道,曰湖北道。三十年,始置雲南按察司。先是,命布政司兼理。建文時,改為十三道肅政按察司。成祖初,復舊。永樂五年,置交阯按察司,又增設各按察司僉事。因督軍衛屯糧,增浙江、江西、廣東、廣西、湖廣、河南、雲南、四川各一人,陜西、福建、山東、山西各二人。此增設監司之始。十二年,置貴州按察司。宣德五年革交阯按察司。除兩京不設,共十三按察司。正統三年,增設理倉副使、僉事,又設僉事與布政司參議各一員於甘肅,監收倉糧。八年,增設僉事,專理屯田。景泰二年,增巡河僉事。自後,各省因事添設,或置或罷,不可勝紀。今總布、按二司所分諸道詳左。

布政司參政、參議分司諸道。督糧道,十三布政司各一員,俱駐省城。督冊道,江西、陜西等間設。分守道:浙江杭嘉湖道,寧紹台道,金衢嚴道,溫處道。俱駐省江西南瑞道,駐省湖東道,駐廣信湖西道,駐臨江饒南九江道,駐九江贛南道。駐南安山東濟南道,東兗道,海右道。俱駐省山西冀寧道,駐省河東道,駐蒲州冀北道,駐大同冀南道。駐汾州陜西關內道,駐省關西道,駐鳳翔西寧道,駐涼州關南道,駐興安,河西道,駐慶陽隴右道。駐鞏昌河南大梁道,駐省河南道,駐河南汝南道,駐南陽河北道。駐懷慶湖廣武昌道,下荊南道,駐鄖陽上荊南道,兼兵備,駐澧州。荊西道,兼兵備,駐安陸。上湖南道,下湖南道,上江防道,或駐荊州、岳州。下江防道。福建興泉道,駐泉州福寧道,駐興化漳南道,駐漳州建南道,駐延平汀漳道。駐上杭縣廣東嶺東道,駐潮州嶺西道,駐高州羅定道,兼兵備,駐羅定州。嶺北道,嶺南道。駐南雄四川川西道,川北道,駐保寧上下川東道,駐涪州上川南道,雅州、嘉定二署。下川南道,敘州、瀘州署廣西桂平道,駐省蒼梧道,駐梧州左江道,駐潯州右江道,駐柳州貴州安平道,貴寧道,駐省新鎮道,駐平越思仁道,駐思南雲南臨安道,騰衝道,瀾滄道。以上或參政,或參議

按察司副使、僉事分司諸道。提督學道,清軍道,驛傳道,十三布政司俱各一員,惟湖廣提學二員,浙江、山西、陜西、福建、廣西、貴州清軍兼驛傳,江西右布政使清軍。

分巡道:浙江杭嚴道,寧紹道,嘉湖道,金衢道。江西饒南九江道,駐饒州湖西道,駐吉安南昌道,湖東道,嶺北道。山東兗州道,駐沂州濟寧道,青州海防道,濟南道,移德州海右道,駐省海道,駐萊州登萊道,遼海道。山西冀寧道,冀南道,駐潞安雁門道。陜西關內道,駐邠州,關西道。駐平涼隴右道,駐秦州河西道,駐鄜州西寧道。河南大梁道,汝南道,駐信陽州河南道,駐汝州河北道。駐磁州湖廣武昌道,荊西道,駐沔陽上荊南道,下荊南道,湖北道,上湖南道,下湖南道,沅靖道。福建巡海道,兼理糧儲福寧道,興泉道,駐泉州建南道,駐建寧武平道,漳南道,駐上杭縣建寧道,海道,駐漳州汀漳道。廣東嶺東道,駐惠州嶺西道,駐肇慶嶺南道,駐省海北道,駐雷州海南道。駐瓊州四川上東道,駐重慶下東道,駐達州川西道,川北道,駐保寧下川南道,上川南道。廣西府江兵巡道。駐平樂桂林兵巡道,駐省蒼梧兵巡道,駐梧州,移鬱林州。左江兵巡道,駐南寧右江兵巡道,駐賓州。上五道俱兼兵備。貴州貴寧道,思石道,駐銅仁都清道。兼兵備,駐都勻雲南安普道,臨沅道,洱海道,金滄道。

整飭兵備道:浙江寧紹道,嘉興道,溫處道,台海道。江西南瑞道,廣建道,駐建昌山東臨清道,武德道,駐武定州曹濮道,駐曹州沂州道,遼東道。山西雁北道,駐代州大同道,二員,一駐大同,一駐朔州。陽和道,潞安道,岢嵐道。陜西肅州道,固原道,臨洮道,駐蘭州洮岷道,駐岷州靖遠道,榆林中路道,榆林東路道,駐神木縣寧夏河西道,駐寧夏寧夏河東兵糧道,駐花馬池莊浪道,漢羌道,潼關道。湖廣辰沅道。河南睢東道。福建兵備道,巡海道。廣東南韶道,南雄道。四川松潘道,威茂道,建昌道,重夔道,安綿道,敘瀘道。廣西,分巡兼兵備。五道俱見分巡貴州威清道,駐安順畢節道。雲南曲靖道。

其外又有協堂道,副使,河南、浙江間設。水利道,浙江屯田道,江西、河南、四川三省屯田兼驛傳。管河道,河南鹽法道,撫治道,陜西撫治商洛道,湖廣又有撫民、撫苗道。監軍道,因事,不常設招練道。山東間設其北直隸之道寄銜於山東者,則為密雲道,大名道,天津道,霸州道;寄銜於山西者,則為易州道,口北道,昌平道,井陘道,薊州、永平等道。南直隸之道寄銜於山東者,太倉道,潁州道,徐州道;寄銜浙江、江西、湖廣者,蘇松道,漕儲道,常鎮道,廬鳳道,徽寧池太道,淮揚道。

按明初制,恐守令貪鄙不法,故於直隸府州縣設巡按御史,各布政司所屬設試僉事。已罷試僉事,改按察分司四十一道,此分巡之始也。分守起於永樂間,每令方面官巡視民瘼。後遂定右參政、右參議分守各屬府州縣。兵道之設,仿自洪熙間,以武臣疏於文墨,遣參政副使沈固、劉紹等往各總兵處整理文書,商榷機密,未嘗身領軍務也。至弘治中,本兵馬文升慮武職不修,議增副僉一員敕之。自是兵備之員盈天下。兩京不設布、按二司,故督學以御史。後置守、巡諸員無所屬,則寄銜於鄰近省布、按司官。

行太僕寺。卿一人,從三品少卿一人,正四品寺丞無定員,正六品其屬,主簿一人,從七品掌各邊衛所營堡之馬政,以聽於兵部。凡騎操馬匹印烙、人表散、課掌、孳牧,以時督察之。歲春秋,閱視其增耗、齒色,三歲一稽比,布、按二司不得與。有瘠損,則聽兵部參罰。苑馬寺亦如之。

洪武三十年,置行太僕寺於山西、北平、陜西、甘肅、遼東。山西、北平、陜西,每寺設少卿一人,丞三人;甘肅、遼東,每寺設少卿、丞各一人,擇致仕指揮、千百戶為之。永樂四年,許令寺官按治所轄衛所鎮撫首領官吏。十八年,以北京行太僕寺為太僕寺。宣德七年,發雜犯死罪應充軍者,於陜西行太僕寺養馬。弘治十年,簡推素有才望者補本寺官,視太僕寺官升擢。嘉靖三年,從御史陳講請,增設陜西、甘肅二寺各少卿一員,分管延綏、寧夏。二十九年,令寺官遇聖節,輪年齎進表文。

苑馬寺。卿一人,從三品少卿一人,正四品寺丞無定員,正六品其屬,主簿一人,從七品各牧監,監正一人,正九品監副一人,從九品錄事一人。各苑,圉長一人。從九品掌六監二十四苑之馬政,而聽於兵部。凡苑,視廣狹為三等:上苑牧馬萬匹,中苑七千,下苑四千。凡牧地,曰草場,曰荒地,曰熟地,嚴禁令而封表之。凡牧人,曰恩軍,曰隊軍,曰改編之軍,曰充發之軍,曰召募之軍,曰抽選之軍,皆籍而食之。凡馬駒,歲籍其監苑之數,上於兵部,以聽考課。監正、副掌監苑之牧事,圉長帥群長而阜蕃馬匹。

永樂四年,置苑馬寺凡四:北直隸、遼東、平涼、甘肅。五年,增設北直隸苑馬寺六監二十四苑。順義、長春、咸和、馴良四苑,隸清河監。水州、隆萃、大牧、遂寧,隸金臺監。汧池、鹿鳴、龍河、長興,隸涿鹿監。遼陽、龍山、萬安、蕃昌,隸盧龍監。清流、廣蕃、龍泉、松林,隸香山監。河陽、崇義、興寧、永成,隸通州監。六年增甘肅、平涼二寺監。每寺各六監二十四苑。十八年,革北京苑馬寺,并入太僕。正統四年,革甘肅苑馬寺,改牧恩軍於黑水口,隸長樂監。弘治二年革平涼寺丞一員。十七年,都御史楊一清奏請行太僕、苑馬二寺員缺,簡選才望參政、副使補升卿,參議、僉事補陞少卿,以振馬政。十八年又請添設寺員。嘉靖三十二年,以遼東寺卿張思兼轄金、復、蓋州三衛軍民。四十二年,又命帶理兵備事。

都轉運鹽使司。都轉運使一人,從三品同知一人,從四品副使一人,從五品判官無定員。從六品其屬,經歷司,經歷一人,從七品知事一人,從八品庫大使、副使各一人。所轄,各場鹽課司大使、副使,各鹽倉大使、副使,各批驗所大使、副使,並一人。俱未入流

都轉運使。掌鹽監之事。同知、副判分司之。都轉運鹽使司凡六:曰兩淮,曰兩浙,曰長蘆,曰河東,曰山東,曰福建。分司十四:泰州、淮安、通州隸兩淮,嘉興、松江、寧紹、溫台隸兩浙,滄州、青州隸長蘆,膠萊、濱樂隸山東,解鹽東場、西場、中場隸河東。分副使若副判蒞之,督各場倉鹽課司,以總於都轉運使,共奉巡鹽御史或鹽法道臣之政令。福建、山東無巡鹽御史,餘詳《食貨志·鹽法》中。

鹽課提舉司。提舉一人,從五品同提舉一人,從六品副提舉無定員。從七品其屬,吏目一人,從九品庫大使、副使一人。所轄,各鹽倉大使、副使,各場、各井鹽課司大使、副使,並一人。提舉司凡七:曰四川,曰廣東海北,廉州曰黑鹽井,楚雄曰白鹽井,姚安曰安寧,曰五井,大理曰察罕腦兒。又有遼東煎鹽提舉司。提舉,正七品,同提舉,正八品,副提舉,正九品。其職掌皆如都轉運司。

明初,置都轉運司於兩淮。吳元年,置兩浙都轉運司於杭州,定都轉運使秩正三品,設同知,正四品副使,正五品運判,正六品經歷,正七品知事,正八品照磨、綱官,正九品鹽場設司令,從七品司丞,從八品百夫長。省注洪武二年置長蘆、河東二都轉運司,及廣東海北鹽課提舉司,尋又置山東、福建二都轉運司。三年,又於陜西察罕腦兒之地置鹽課提舉司,後漸增置各處。建文中,改廣東提舉為都轉運司。永樂初復故。十四年,初命御史巡鹽。景泰三年,罷長蘆、兩淮巡鹽御史,命撫、按官兼理。已復遣御史,其無御史者,分按察司理之。又洪武中,於四川置茶鹽都轉運司,洪武五年置,設官如都轉運鹽使司。十年罷。納溪、白渡二鹽馬司,洪武五年置,以常選官為司令,內使為司丞。十三年罷,尋復置。十五年,改設大使、副使各一人。後並革。又有順龍鹽馬司,亦革。

市舶提舉司。提舉一人,從五品副提舉二人。從六品其屬,吏目一人。從九品掌海外諸蕃朝貢市易之事,辨其使人表文勘合之真偽,禁通番,徵私貨,平交易,閑其出入而慎館穀之。

吳元年,置市舶提舉司。洪武三年,罷太倉、黃渡市舶司。七年,罷福建之泉州、浙江之明州、廣東之廣州三市舶司。永樂元年復置,設官如洪武初制,尋命內臣提督之。嘉靖元年,給事中夏言奏倭禍起於市舶,遂革福建、浙江二市舶司,惟存廣東市舶司。

茶馬司。大使一人,正九品副使一人,從九品掌市馬之事。洪武中,置洮州、秦州、河州三茶馬司,設司令、司丞。十五年改設大使、副使各一人,尋罷洮州茶馬司,以河州茶馬司兼領之。三十年,改秦州茶馬司為西寧茶馬司。又洪武中,置四川永寧茶馬司,後革,復置雅州碉門茶馬司。又於廣西置慶遠裕民司,洪武七年置,設大使一人,從八品,副使一人,正九品。市八番溪洞之馬,後亦革。

府。知府一人,正四品同知,正五品通判無定員,正六品推官一人。正七品其屬,經歷司經歷一人,正八品知事一人。正九品照磨所,照磨一人,從九品檢校一人。司獄司,司獄一人。所轄別見

知府,掌一府之政,宣風化,平獄訟,均賦役,以教養百姓。每三歲,察屬吏之賢否,上下其考,以達於省,上吏部。凡朝賀、弔祭,視布政使司,直隸府得專達。凡詔赦、例令、勘答刂至,謹受之,下所屬奉行。所屬之政,皆受約束於府,劑量輕重而令之,大者白於撫、按、布、按,議允乃行。凡賓興科貢,提調學校,修明祀典之事,咸掌之。若籍帳、軍匠、驛遞、馬牧、盜賊、倉庫、河渠、溝防、道路之事,雖有專官,皆總領而稽核之。同知、通判分掌清軍、巡捕、管糧、治農、水利、屯田、牧馬等事。無常職,各府所掌不同,如延安、延綏同知又兼牧民,餘不盡載。無定員。邊府同知有增至六、七員者。推官理刑名,贊計典。各府推官,洪武三年始設。經歷、照磨、檢校受發上下文移,磨勘六房宗卷。

明初,改諸路為府。洪武六年,分天下府三等:糧二十萬石以上為上府,知府秩從三品;二十萬石以下為中府,知府正四品;十萬石以下為下府,知府,從四品。已,並為正四品。七年,減北方府州縣官三百八人。十三年,選國子學生二十四人為府州縣官。六月罷各府照磨。二十七年復置。自宣德三年棄交阯布政司,計天下府凡一百五十有九。

州。知州一人,從五品同知,從六品判官無定員,從七品。里不及三十而無屬縣,裁同知、判官。有屬縣,裁同知。其屬,吏目一人,從九品。所轄別見。

知州,掌一州之政。凡州二:有屬州,有直隸州。屬州視縣,直隸州視府,而品秩則同。同知、判官,俱視其事州之繁簡,以供厥職。計天下州凡二百三十有四。

縣。知縣一人,正七品縣丞一人,正八品主簿一人,正九品其屬,典史一人。所轄別見

知縣,掌一縣之政。凡賦役,歲會實征,十年造黃冊,以丁產為差。賦有金穀、布帛及諸貨物之賦,役有力役、雇役、借債不時之役,皆視天時休咎,地利豐耗,人力貧富,調劑而均節之。歲歉則請於府若省蠲減之。凡養老、祀神、貢士、讀法、表善良、恤窮乏、稽保甲、嚴緝捕、聽獄訟,皆躬親厥職而勤慎焉。若山海澤藪之產,足以資國用者,則按籍而致貢。縣丞、主簿分掌糧馬、巡捕之事。典史典文移出納。如無縣丞,或無主簿,則分領丞簿職。縣丞、主簿,添革不一。若編戶不及二十里者並裁。

吳元年,定縣三等:糧十萬石以下為上縣,知縣從六品;六萬石以下為中縣,知縣正七品;三萬石以下為下縣,知縣從七品。已,並為正七品。凡新授郡縣官,給道里費。洪武元年,徵天下賢才為府州縣職,敕命厚賜,以勵其廉恥,又敕諭之至於再。三十七年,定府州縣條例八事,頒示天下,永為遵守。是時,天下府州縣官廉能正直者,必遣行人齎敕往勞,增秩賜金。仁、宣之際猶然,英、憲而下日罕。自後益重內輕外,此風絕矣。計天下縣凡一千一百七十有一。

儒學。府,教授一人,從九品訓導四人。州,學正一人,訓導三人。縣,教諭一人,訓導二人。教授、學正、教諭,掌教誨所屬生員,訓導佐之。凡生員廩膳、增廣,府學四十人,州學三十人,縣學二十人,附學生無定數。儒學官月課士子之藝業而獎勵之。凡學政遵臥碑,咸聽於提學憲臣提調,府聽於府,州聽於州,縣聽於縣。其殿最視鄉舉之有無多寡。

明初,置儒學提舉司。洪武二年,詔天下府州縣皆立學。十三年,改各州學正為未入流。先是從九品二十四年,定儒學訓導位雜職上。三十一年詔天下學官改授旁郡州縣。正統元年始設提督學校官,又有都司儒學,洪武十七年置,遼東始。行都司儒學,洪武二十三年置,北平始。衛儒學,洪武十七年置,岷州衛,二十三年置,大寧等衛始。以教武臣子弟。俱設教授一人,訓導二人。河東又設都轉運司儒學,制如府。其後宣慰、安撫等土官,俱設儒學。

巡檢司。巡檢、副巡檢,俱從九品主緝捕盜賊,盤詰奸偽。凡在外各府州縣關津要害處俱設,俾率徭役弓兵警備不虞。初,洪武二年,以廣西地接瑤、僮,始於關隘衝要之處設巡檢司,以警奸盜,後遂增置各處。十三年二月,特賜敕諭之,尋改為雜職。

驛。驛丞典郵傳迎送之事。凡舟車、夫馬、廩糗、庖饌、裯帳,視使客之品秩,僕夫之多寡,而謹供應之。支直於府若州縣,而籍其出入。巡檢、驛丞,各府州縣有無多寡不同。

稅課司。府曰司,縣曰局。大使一人,從九品典稅事。凡商賈、僧屠、雜市,皆有常征,以時榷而輸其直於府若縣。凡民間貿田宅,必操契券請印,乃得收戶,則徵其直百之三。明初,改在京官店為宣課司,府州縣官店為通課司,後改通課司為稅課司、局。

倉。大使一人,府從九品,州縣未入流副使一人,庫大使一人。州縣設。

織染雜造局。大使一人,從九品,州織染局未入流。副使一人。

河泊所官,掌收魚稅;閘官、壩官,掌啟閉蓄洩。洪武十五年,定天下河泊所凡二百五十二。歲課糧五千石以上至萬石者,設官三人;千石以上設二人;三百石以上設一人。

批驗所。大使一人,副使一人,掌驗茶鹽引。

遞運所。大使一人,副使一人,掌運遞糧物。洪武九年始置。先是,在外多以衛所戍守軍士傳送軍囚,太祖以其有妨練習守禦,乃命兵部增置各處遞運所,以便遞送。設大使、副使各一人,驗夫多寡,設百夫長以領之。後汰副使,革百夫長。

鐵冶所。大使一人,副使一人。洪武七年初置。凡十三所,每所置大使、副使各一人。初,大使,正八品,副使,正九品,後俱為未入流。

醫學。府,正科一人。從九品州,典科一人。縣,訓科一人。洪武十七年置,設官不給祿。

陰陽學。府,正術一人。從九品州,典術一人。縣,訓術一人。亦洪武十七年置,設官不給祿。

府僧綱司,都綱一人,從九品副都綱一人。州僧正司,僧正一人。縣僧會司,僧會一人。府道紀司,都紀一人,從九品副都紀一人。州道正司,道正一人。縣道會司,道會一人。俱洪武十五年置,設官不給祿。


公侯伯駙馬都尉附儀賓五軍都督府京營京衛錦衣衛附旗手等衛南京守備南京五軍都督府南京衛王府護衛附儀衛司總兵官留守司都司附行都司各衛各所宣慰司宣撫司安撫司招討司長官司附蠻夷長官司軍民府附土州土縣

公、侯、伯,凡三等,以封功臣及外戚,皆有流有世。功臣則給鐵券,封號四等:佐太祖定天下者,曰開國輔運推誠;從成祖起兵,曰奉天靖難推誠;餘曰奉天翊運推誠,曰奉天翊衛推誠。武臣曰宣力武臣,文臣曰守正文臣。歲祿以功為差。已封而又有功,仍爵或進爵,增祿。其才而賢者,充京營總督,五軍都督府掌僉書,南京守備,或出充鎮守總兵官,否則食祿奉朝請而已。年幼而嗣爵者,咸入國子監讀書。嘉靖八年,定外戚封爵毋許世襲,其有世襲一二代者,出特恩。

駙馬都尉,位在伯上。凡尚大長公主、長公主、公主,並曰駙馬都尉。其尚郡主、縣主、郡君、縣君、鄉君者,並曰儀賓。歲祿各有差,皆不得與政事。明初,駙馬都尉有典兵出鎮及掌府部事者。建文時,梅殷為鎮守淮安總兵官,李堅為左副將軍。成祖時,李讓掌北京行部事。仁宗時沐昕,宣宗時宋琥,並守備南京。英宗時,趙輝掌南京左府事。其餘惟奉祀孝陵,攝行廟祭,署宗人府事。往往受命,一充其任。若恩親侯李貞,永春侯王寧,京山侯崔元,以恩澤封侯,非制也。

中軍、左軍、右軍、前軍、後軍五都督府,每府左、右都督,正一品都督同知,從一品都督僉事,正二品,恩功寄祿,無定員。其屬,經歷司,經歷,從五品都事,從七品各一人。

都督府掌軍旅之事,各領其都司、衛所,詳見《兵志·衛所》中以達於兵部。凡武職,世官流官、土官襲替、優養、優給,所屬上之府,移兵部請選。既選,移府,以下之都司、衛所。首領官聽吏部選授,給由亦如之。凡武官誥敕、俸糧、水陸步騎操練、官舍旗役併試、軍情聲息、軍伍勾補、邊腹地圖、文冊、屯種、器械、舟車、薪葦之事,並移所司而綜理之。凡各省、各鎮鎮守總兵官,副總兵,並以三等真、署都督及公、侯、伯充之。有大征討,則掛諸號將軍或大將軍、前將軍、副將軍印總兵出,既事,納之。其各府之掌印及僉書,率皆公、侯、伯。間有屬老將之實為都督者,不能十一也。

初,太祖下集慶,即置行樞密院,自領之。又置諸翼統軍元帥府。尋罷樞密院,改置大都督府。以朱文正為大都督,節制中外諸軍事,設司馬、參軍、經歷、都事等官。又增設左、右都督,同知,副使,僉事,照磨各一人,并設斷事官。定制,大都督從一品,左、右都督正二品,同知都督從二品,副都督正三品,僉都督從三品,經歷從五品,都事從七品;統軍元帥府元帥正三品,同知元帥從三品,副使正四品,經歷正七品,知事從八品,照磨正九品;又以都鎮撫司隸大都督府,先是屬中書省秩從四品。尋罷統軍元帥府。吳元年,更定官制,罷大都督不設,以左、右都督為長官,正一品同知都督,從一品副都督,正二品僉都督,從二品俱升品秩。其屬,設參議,正四品經歷,斷事官,從五品都事,正七品照磨從七品洪武九年,罷副都督,改參議為掌判官。十二年,升都督僉事為正二品,掌判官為正三品。十三年,始改都督府為五軍都督府,分領在京各衛所,惟錦衣等親軍,上直衛不隸五府。及在外各都司、衛所,以中軍都督府斷事官為五軍斷事官。十五年,置五軍十衛參軍府,設左、右參軍。十七年,五軍各設左、右斷事二人,提控案牘一人,並從九品二十三年,陞五軍斷事官為正五品,總治五軍刑獄。分為五司,司設稽仁、稽義、稽禮、稽智、稽信五人,俱正七品,各理其軍之刑獄。二十九年,置五軍照磨所,專掌文牘。建文中,革斷事及五司官。永樂元年,設北京留守行後軍都督府,置左、右都督,都督同知,都督僉事,無定員,經歷、都事各一人。後又分五府,稱行在五軍都督府。十八年,除「行在」字,在應天者加「南京」字。洪熙元年,復稱行在,仍設行後府。宣德三年又革。正統六年,復除「行在」字。

京營,永樂二十二年,置三大營,曰五軍營,曰神機營,曰三千營。五軍、神機各設中軍、左右哨、左右掖;五軍、三千各設五司。每營俱選勳臣二人提督之。其諸營管哨、掖官,曰坐營,曰坐司。各哨、掖官,亦率以勳臣為之。又設把總、把司、把牌等官。又有圍子手、幼官、舍人、殫忠、效義諸營,俱附五軍營中。景泰元年選三營精銳立十團營,蒞以總兵,統以總督,監以內臣。其舊設者,號為老營。三老營凡六提督,內選其二領團營。成化三年,分團營為十二,每營又各分五軍、三千統騎兵,神機統火器。其各營統領,俱擇都督、都指揮或列爵充之,以總督統轄之。正德中,又選團營精銳,置東西兩官廳,另設總兵、參將統領。嘉靖二十九年,革團營官廳,仍併三大營,改三千曰神樞,設副、參、遊、佐、坐營、號頭、中軍、千把總等官。五軍營:戰兵一營,左副將一;戰兵二營,練勇參將一;車兵三營,參將一;車兵四營,遊擊將軍一;城守五營,佐擊將軍一;戰兵六營,右副將一;戰兵七營,練勇參將一;車兵八營,參將一;車兵九營,遊擊將軍一;城守十營,佐擊將軍一;備兵坐營官一,大號頭官一。已上部推。監鎗號頭官一,中軍官十一,隨征千總四,隨營千總二十,選鋒把總八,把總一百三十八。已上俱營推。神樞營:戰兵一營,左副將一;戰兵二營,練勇參將一;車兵三營,參將一;車兵四營,遊擊將軍一;城守五營,佐擊將軍一;戰兵六營,右副將一;車兵七營,練勇參將一;執事八營,參將一;城守九營,佐擊將軍一;城守十營,佐擊將軍一;備兵坐營官一,大號頭官一。已上部推。監鎗號頭官一,中軍官十一,千總二十,選鋒把總六,把總一百五十七。已上俱營推。神機營:戰兵一營,左副將一;戰兵二營,練勇參將一;車兵三營,遊擊將軍一;車兵四營,佐擊將軍一;城守五營,佐擊將軍一;戰兵六營,右副將一;車兵七營,練勇參將一;城守八營,佐擊將軍一;城守九營,佐擊將軍一;城守十營,佐擊將軍一;備兵坐營官一,大號頭官一。已上部推。監槍號頭官一,中軍官十一,千總二十,選鋒把總六,把總一百二十八。已上俱營推。通計三大營,共五百八十六員。統以提督總兵官一員。已,改提督曰總督,鑄「總督京營戎政」印,俾仇鸞佩之。更設侍郎一人,協理京營戎政。定巡視科道官歲一代更,悉革內侍官。增設巡視主事,尋亦革。隆慶初,仍以總督為提督,改協理為閱視,尋併改閱視為提督。四年二月,更京營制,三營各設提督,又各設右都御史一員提督之。九月,罷六提督,仍復總督戎政一人。天啟初,增設協理一人,已,仍革一人。崇禎初,復增一人。

京衛指揮使司,指揮使一人,正三品指揮同知二人,從三品指揮僉事四人。正四品鎮撫司,鎮撫二人,從五品其屬,經歷司,經歷,從七品知事,正八品吏目,從九品倉大使、副使各一人。所轄千戶所,多寡各不等。

京衛有上直衛,有南、北京衛,品秩並同。各有掌印,有僉書。其以恩廕寄祿,無定員。凡上直衛親軍指揮使司,二十有六。曰錦衣衛,曰旗手衛,曰金吾前衛。曰金吾後衛,曰羽林左衛,曰羽林右衛,曰府軍衛,曰府軍左衛,曰府軍右衛,曰府軍前衛,曰府軍後衛,曰虎賁左衛。是為上十二衛,洪武中置。曰金吾左衛,曰金吾右衛,曰羽林前衛,曰燕山左衛,曰燕山右衛,曰燕山前衛,曰大興左衛,曰濟陽衛,曰濟州衛,曰通州衛。是為上十衛,永樂中置。曰騰驤左衛,曰騰驤右衛,曰武驤左衛,曰武驤右衛。宣德八年置。番上宿衛名親軍,以護宮禁,不隸五都督府。其京衛隸都督府者,三十有三。曰留守左衛,曰鎮南衛,曰驍騎右衛,曰龍虎衛,曰沈陽左衛,曰沈陽右衛,隸左軍都督府。曰留守右衛,曰虎賁右衛,曰武德衛,隸右軍都督府。曰留守中衛,曰神策衛,曰應天衛,曰和陽衛,及牧馬千戶所、蕃牧千戶所,俱隸中軍都督府。曰留守前衛,曰龍驤衛,曰豹韜衛,隸前軍都督府。曰留守後衛,曰鷹揚衛,曰興武衛,曰大寧中衛,曰大寧前衛,曰會州衛,曰富峪衛,曰寬河衛,曰神武左衛,曰忠義右衛,曰忠義前衛,曰忠義後衛,曰義勇右衛,曰義勇前衛,曰義勇後衛,曰武成中衛,曰蔚州左衛,隸後軍都督府。又京衛非親軍而不隸都督府者,十有五。曰武功中衛,曰武功左衛,曰武功右衛,已上三衛以匠故,隸工部。曰永清左衛,曰永清右衛,曰彭城衛,曰長陵衛,曰獻陵衛,曰景陵衛,曰裕陵衛,曰茂陵衛,曰泰陵衛,曰康陵衛,曰永陵衛,曰昭陵衛。

明初,置帳前總制親軍都指揮使司,以馮國用為都指揮使。後改置金吾侍衛親軍都護府,設都護,從二品經歷,正六品知事,從七品照磨。從八品。又置各衛親軍指揮使司,設指揮使,正三品同知指揮使,從三品副使,正四品經歷,正七品知事,正八品照磨,正九品千戶所正千戶,正五品副千戶,從五品鎮撫、百戶。正六品因置武德、龍驤、豹韜、飛龍、威武、廣武、興武、英武、鷹揚、驍騎、神武、雄武、鳳翔、天策、振武、宣武、羽林十七衛親軍指揮使司,此設親軍衛之始。尋罷金吾侍衛親軍都護府。洪武、永樂間,增設親軍諸衛,名為上二十二衛,分掌宿衛。而錦衣衛主巡察、緝捕、理詔獄,以都督、都指揮領之,蓋特異於諸衛焉。留守五衛,舊為都鎮撫司,總領禁衛,先屬中書省,改隸大都督府,設都鎮撫,從四品副鎮撫,從五品知事,從八品尋改宿衛鎮撫司,設宿衛鎮撫、宿衛知事。洪武三年,改為留守衛指揮使司,專領軍馬守禦各城門,及巡警皇城與城垣造作之事。後升為留守都衛,統轄天策、豹韜、飛熊、鷹揚、江陰、廣洋、橫海、龍江、水軍左、右十衛。八年,復為留守衛,與天策等八衛俱為親軍指揮使司,惟水軍左、右二衛為指揮使司。並隸大都督府。十一年,改為留守中衛,增置留守左、右、前、後四衛,仍為親軍。十三年,始分隸五都督府。

錦衣衛,掌侍衛、緝捕、刑獄之事,恒以勳戚都督領之,恩蔭寄祿無常員。凡朝會、巡幸,則具鹵簿儀仗,率大漢將軍共一千五百七員等侍從扈行。宿衛則分番入直。朝日、夕月、耕耤、視牲,則服飛魚服,佩繡春刀,侍左右。盜賊奸宄,街途溝洫,密緝而時省之。凡承制鞫獄錄囚勘事,偕三法司。五軍官舍比試併鎗,同兵部蒞視。統所凡十有七。中、左、右、前、後五所,領軍士。五所分鑾輿、擎蓋、扇手、旌節、幡幢、班劍、斧鉞、戈戟、弓矢、馴馬十司,各領將軍校尉,以備法駕。上中、上左、上右、上前、上後、中後六親軍所,分領將軍、力士、軍匠。馴象所,領象奴養象,以供朝會陳列、駕輦、馱寶之事。

明初,置拱衛司,秩正七品,管領校尉,屬都督府。後改拱衛指揮使司,秩正三品。尋又改為都尉司。洪武三年,改為親軍都尉府,管左、右、中、前、後五衛軍士,而設儀鸞司隸焉。四年,定儀鸞司為正五品,設大使一人,副使二人。十五年,罷儀鸞司,改置錦衣衛,秩從三品,其屬有御椅等七員,皆正六品。設經歷司,掌文移出入;鎮撫司,掌本衛刑名,兼理軍匠。十七年,改錦衣衛指揮使為正三品。二十年,以治錦衣衛者多非法凌虐,乃焚刑具,出繫囚,送刑部審錄,詔內外獄咸歸三法司,罷錦衣獄。成祖時復置。尋增北鎮撫司,專治詔獄。成化間,刻印畀之,獄成得專達,不關白錦衣,錦衣官亦不得干預。而以舊所設為南鎮撫司,專理軍匠。

旗手衛,本旗手千戶所,洪武十八年改置。掌大駕金鼓、旗纛,帥力士隨駕宿衛。校尉、力士,僉民間壯丁為之。校尉專職擎執鹵簿儀杖,及駕前宣召官員,差遣乾辦,隸錦衣衛。力士專領金鼓、旗幟,隨駕出入,及守衛四門,隸旗手衛。凡歲祭旗頭六纛之神,八月於壇,十二月於承天門外,皆衛官蒞事,統所五。

府軍前衛,掌統領幼軍,輪番帶刀侍衛。明初,有帶刀舍人。洪武時,府軍等衛皆有習技幼軍。永樂十三年,為皇太孫特選幼軍,置府軍前衛,設官屬,指揮使五人,指揮同知十人,指揮僉事二十人,衛鎮撫十人,經歷五人。統所二十有五。

金吾、羽林等十九衛,掌守衛巡警,統所凡一百有二。

騰驤等四衛,掌帥力士直駕、隨駕,統所三十有二。

南京守備一人,協同守備一人。南京以守備及參贊機務為要職。守備,以公、侯、伯充之,兼領中軍都督府事。協同守備,以侯、伯、都督充之,領五府事。參贊機務,以南京兵部尚書領之。其治所在中府,掌南都一切留守、防護之事。永樂十九年遷都北京,命中府掌府事官守備南京,節制南京諸衛所。洪熙元年,始以內臣同守備。景泰三年,增設協同守備一人。

南京五軍都督府,左、右都督,都督同知,都督僉事,不全設。其掌印、僉書,皆以勳爵及三等都督為之。分掌南京衛所,以達於南京兵部。凡管領大教場及江上操備等事,各府奉敕分掌之。城門之管鑰,中府專掌之。初設城門郎,洪武十八年革,以門禁鎖鑰銅牌,命中軍都督府掌之。其屬,經歷、都事各一人。

南京衛指揮使司,設官詳京衛凡四十有九。分隸五都督府者三十有二。曰留守左衛,曰鎮南衛,曰水軍左衛,曰驍騎右衛,曰龍虎衛,曰龍虎左衛,曰英武衛,曰龍江右衛,曰沈陽左衛,曰沈陽右衛,隸左府。曰留守右衛,曰虎賁右衛,曰水軍右衛,曰武德衛,曰廣武衛,隸右府。曰留守中衛,曰神策衛,曰廣洋衛,曰廣天衛,曰和陽衛,及牧馬千戶所,隸中府。曰留守前衛,曰龍江左衛,曰龍驤衛,曰飛熊衛,曰天策衛,曰豹韜衛,曰豹韜左衛,隸前府。曰留守後衛,曰橫海衛,曰鷹揚衛,曰興武衛,曰江陰衛,隸後府。又親軍衛指揮使司十有七:曰金吾前衛,曰金吾後衛,曰金吾左衛,曰金吾右衛,曰羽林左衛,曰羽林右衛,曰羽林前衛,曰府軍衛,曰府軍左衛,曰府軍右衛,曰府軍後衛,曰虎賁左衛,曰錦衣衛,曰旗手衛,曰江淮衛,曰濟州衛,曰孝陵衛。與左府所屬十衛,右府所屬五衛,前府所屬七衛,後府所屬五衛,並聽中府節制。各衛領所一百一十有八。

王府護衛指揮使司,設官如京衛。

王府儀衛司。儀衛正一人,正五品儀衛副二人,從五品典仗六人。正六品儀衛,掌侍衛儀仗。護衛,掌防禦非常,護衛王邸。有徵調,則聽命於朝。明初,諸王府置護軍府。洪武三年,置儀衛司,司設正、副各一人,秩比正、副千戶;司仗六人,秩比百戶。四年,改司仗為典仗。五年,置親王護衛指揮使司,每王府設三護衛,衛設左、右、前、後、中五所,所千戶二人,百戶十人。又設圍子手二所,每所千戶一人。九年,罷護軍府。建文中,改儀衛司為儀仗司,增置吏目一人。成祖初復舊制。

總兵官、副總兵、參將、遊擊將軍、守備、把總,無品級,無定員。總鎮一方者為鎮守,獨鎮一路者為分守,各守一城一堡者為守備,與主將同守一城者為協守。又有提督、提調、巡視、備御、領班、備倭等名。

凡總兵、副總兵,率以公、侯、伯、都督充之。其總兵掛印稱將軍者,雲南曰征南將軍,大同曰征西前將軍,湖廣曰平蠻將軍,兩廣曰征蠻將軍,遼東曰征虜前將軍,宣府曰鎮朔將軍,甘肅曰平羌將軍,寧夏曰征西將軍,交阯曰副將軍,延綏曰鎮西將軍。諸印,洪熙元年制頒。其在薊鎮、貴州、湖廣、四川及人贊運淮安者,不得稱將軍掛印。宣德間,又設山西、陜西二總兵。嘉靖間,分設廣東、廣西、貴州、湖廣二總兵為四,改設福建、保定副總兵為總兵,又添設浙江總兵。萬曆間,又增設於臨洮、山海。天啟間,增設登萊。至崇禎時,益紛不可紀,而位權亦非復當日。蓋明初,雖參將、遊擊、把總,亦多有充以勛戚都督等官,至後則杳然矣。

鎮守薊州總兵官一人,舊設。隆慶二年,改為總理練兵事務兼鎮守,駐三屯營。協守副總兵三人。東路副總兵,隆慶二年添設,駐建昌營,管理燕河營、臺頭營、石門寨、山海關四路。中路副總兵,萬曆四年改設,駐三屯營,帶管馬蘭峪、松棚峪、喜峰口、太平寨四路。西路副總兵,隆慶三年添設,駐石匣營,管理墻子嶺、曹家寨、古北口、石塘嶺四路。分守參將十一人,曰通州參將,曰山海關參將,曰石門寨參將,曰燕河營參將,曰石塘嶺參將,曰臺頭營參將,曰太平寨參將,曰馬蘭峪參將,曰墻子嶺參將,曰古北口參將,曰喜峰口參將。遊擊將軍六人,統領南兵遊擊將軍三人,領班遊擊將軍七人,坐營官八人,守備八人,把總一人,提調官二十六人。

鎮守昌平總兵官一人,舊設副總兵,又有提督武臣。嘉靖三十八年,裁副總兵,以提督改為鎮守總兵,駐昌平城,聽總督節制。分守參將三人,曰居庸關參將,曰黃花鎮參將,曰橫嶺口參將。遊擊將軍二人,坐營官三人,守備十人,提調官一人。

鎮守遼東總兵官一人,舊設,駐廣寧。隆慶元年,令冬月移駐河東遼陽適中之地,調度防禦,應援海州、沈陽。協守副總兵一人,遼陽副總兵舊為分守,嘉靖四十五年改為協守,駐遼陽城,節制開原、海州、險山、沈陽等處。分守參將五人。曰開原參將,曰錦義右參將,曰海蓋右參將,曰寧遠參將,曰寬奠堡參將。遊擊將軍八人,守備五人,坐營中軍官一人,備禦十九人。

鎮守保定總兵官一人。弘治十八年,初設保定副總兵,後改為參將。正德九年,復為分守副總兵。嘉靖二十年,改為鎮守。三十年,改設鎮守總兵官。萬曆元年,令春秋兩防移駐浮圖峪,遇有警,移駐紫荊關,以備入援。分守參將四人,曰紫荊關參將,曰龍固二關參將,曰馬水口參將,曰倒馬關參將。遊擊將軍六人,坐營中軍官一人,守備七人,把總七人,忠順官二人。

鎮守宣府總兵官一人,舊設,駐宣府鎮城。協守副總兵一人,副總兵舊亦駐鎮城,嘉靖二十八年移駐永寧城。分守參將七人,曰北路獨石馬營參將,曰東路懷來永寧參將,曰上西路萬全右衛參將,曰南路順聖蔚廣參將,曰中路葛峪堡參將,曰下西路柴溝堡參將,曰南山參將。遊擊將軍三人,坐營中軍官二人,守備三十一人,領班備禦二人。萬歷八年革。

鎮守大同總兵官一人,舊設,駐大同鎮城。協守副總兵一人,舊為左副總兵,萬曆五年去左字,駐左衛城。分守參將九人,曰東路參將,曰北東路參將,曰中路參將,曰西路參將,曰北西路參將,曰井坪城參將,曰新坪堡參將,曰總督標下左掖參將,曰威遠城參將,萬曆八年革。遊擊將軍二人,入衛遊擊四人,坐營中軍官二人,守備三十九人。

鎮守山西總兵官一人,舊為副總兵,嘉靖二十年改設,駐寧武關。防秋移駐陽方口,防冬移駐偏關。協守副總兵一人,嘉靖四十四年添設,初駐偏關,後移駐老營堡。分守參將六人,曰東路代州左參將,曰西路偏頭關右參將,曰太原左參將,曰中路利民堡右參將,曰河曲縣參將,曰北樓口參將。遊擊將軍一人,坐營中軍官一人,守備十三人,操守二人。

鎮守延綏總兵官一人,舊設,駐鎮城。協守副總兵一人,定邊右副總兵,嘉靖四十一年添設,分守安定、鎮靜等處,提調大墻及墻口等處。分守參將六人,曰孤山參將,曰東路右參將,曰西路左參將,曰中路參將,曰清平參將,曰榆林保寧參將。遊擊將軍二人,入衛遊擊四人,守備十一人,坐營中軍官一人。

鎮守寧夏總兵官一人,舊設,駐鎮城。協守副總兵一人,亦舊設,同駐鎮城。分守參將四人,曰東路右參將,曰西路左參將,曰靈州左參將,曰北路平虜城參將。游擊將軍三人,入衛遊擊一人,萬曆八年革守備三人,備禦領班二人,萬歷九年革,坐營中軍官二人,管理鎮城都司一人,領班都司二人萬歷九年革管理水利屯田都司一人。

鎮守甘肅總兵官一人,舊設,駐鎮城。協守副總兵一人,甘肅左副總兵,舊設,嘉靖四十四年,移駐高臺防禦,隆慶四年,回駐鎮城。分守副總兵一人,涼州右副總兵,舊設。分守參將四人,曰莊浪左參將,曰肅州右參將,曰西寧參將,曰鎮番參將遊擊將軍四人,坐營中軍官一人,守備十一人,領班備禦都司四人。

鎮守陜西總兵官一人,舊駐會城,後移駐固原。分守副總兵一人,洮泯副總兵,萬曆六年改設,駐洮州。分守參將五人,曰河州參將,曰蘭州參將,曰靖虜參將,曰陜西參將,曰階文西固參將。遊擊將軍四人,坐營中軍官二人,守備八人。

鎮守四川總兵官一人,隆慶五年添設,駐建武所。分守副總兵一人,松潘副總兵,舊設協守參將二人,曰松潘東路左參將,曰松潘南路右參將。遊擊將軍二人,守備六人。

鎮守雲南總兵官一人,舊設,駐雲南府。分守參將三人,曰臨元參將,曰永昌參將,曰順蒙參將,守備二人。巡撫中軍坐營官一人。

鎮守貴州總兵官一人,舊設,嘉靖三十二年,加提督麻陽等處地方職銜,駐銅仁府。分守參將二人,曰提督清浪右參將,曰提督川貴迤西左參將。守備七人,巡撫中軍官一人。

鎮守廣西總兵官一人,舊為副總兵,嘉靖四十五年改設,駐桂林府。分守參將五人,曰潯梧左參將,曰柳慶右參將,曰永寧參將,曰思恩參將,曰昭平參將。守備三人,坐營官一人。

鎮守湖廣總兵官一人,舊設,嘉靖十年罷,十二年復設,萬曆八年又罷,十二年仍復設,駐省城。分守參將三人,曰黎平參將,曰鎮參將,曰鄖陽參將。守備十一人,把總一人。

鎮守廣東總兵官一人。舊為征蠻將軍、兩廣總兵官。嘉靖四十五年分設,駐潮州府。協守副總兵一人,潮漳副總兵,萬曆三年添設,駐南澳。分守參將七人,曰潮州參將,曰瓊崖參將,曰雷廉參將,曰東山參將,曰西山參將,曰督理廣州海防參將,曰惠州參將。練兵遊擊將軍一人,守備五人,坐營中軍官二人,把總四人。

提督狼山副總兵一人,嘉靖三十七年添設,駐通州。鎮守江南副總兵一人,舊係總兵官,駐福山港,後移駐鎮江、儀真二處。嘉靖八年裁革。十九年復設。二十九年仍革。三十二年,改設副總兵,駐金山衛。四十三年移駐吳淞。分守參將二人,曰徐州參將,曰金山參將。遊擊將軍一人,守備六人,鳳陽軍門中軍官一人,把總十三人。

鎮守浙江總兵官一人,嘉靖三十四年設,總理浙直海防。三十五年,改鎮守浙直。四十二年,改鎮守浙江,舊駐定海縣,後移駐省城。分守參將四人,曰杭嘉湖參將,曰寧紹參將,曰溫處參將,曰台金嚴參將。遊擊將軍二人,總捕都司一人,把總七人。

分守江西參將一人,曰南贛參將,嘉靖四十三年改設,駐會昌縣。守備四人,把總六人。

鎮守福建總兵官一人,舊為副總兵,嘉靖四十二年改設,駐福寧州。分守參將一人,曰南路參將守備三人,把總七人,坐營官一人。

鎮守山東總兵官一人,天啟中增設。總督備倭都司一人,領薊鎮班都司四人。又河南守備三人,領薊鎮班都司四人。

總督漕運總兵官一人。永樂二年,設總兵、副總兵,統領官軍海運。後海運罷,專督漕運。天順元年又令兼理河道。協同督運參將一人,天順元年設把總十二人,南京二,江南直隸二,江北直隸二,中都一,浙江二,山東一,湖廣一,江西一。

留守司。正留守一人,正二品副留守一人,正三品指揮同知二人。從三品其屬,經歷司,經歷,正六品都事。正七品斷事司,斷事,正六品副斷事,正七品吏目各一人。掌中都、興都守禦防護之事。洪武二年,詔以臨濠為中都,置留守衛指揮使司,隸鳳陽行都督府。十四年,始置中都留守司,統鳳陽等八衛,鳳陽衛,鳳陽中衛,鳳陽右衛,皇陵衛,留守左衛,留守中衛,長淮衛,懷遠衛。防護皇陵,設留守一人,左、右副留守各一人。屬官經歷以下,如前所列。嘉靖十八年,改荊州左衛為顯陵衛,置興都留守司,統顯陵、承天二衛,防護顯陵,設官如中都焉。

都指揮使司。都指揮使一人,正二品都指揮同知二人,從二品都指揮僉事四人。正三品其屬,經歷司,經歷,正六品都事。正七品斷事司,斷事,正六品副斷事,正七品吏目各一人。司獄司,司獄。從九品倉庫、草場,大使、副使各一人。行都指揮使司,設官與都指揮使司同。

都司,掌一方之軍政,各率其衛所以隸於五府,而聽於兵部。凡都司並流官,或得世官,歲撫、按察其賢否,五歲考選軍政而廢置之。都指揮使及同知僉事,常以一人統司事,曰掌印,一人練兵,一人屯田,曰僉書。巡捕、軍器、漕運、京操、備禦諸雜務,並選充之,否則曰帶俸。凡備倭守備行都指揮事者,不得建牙、升公座。凡朝廷吉凶表箋,序銜布、按二司上。經歷、都事,典文移。斷事,理刑獄。

明初,置各行省行都督府,設官如都督府。又置各都衛指揮使司。洪武四年,置各都衛斷事司,以理軍官、軍人詞訟。又以都衛節制方面,職係甚重,從朝廷選擇升調,不許世襲。七年,置西安行都衛指揮使司於河州。八年十月,詔各都衛並改為都指揮使司,凡改設都司十有三,燕山都衛為北平都司,西安都衛為陜西都司,太原都衛為山西都司,杭州都衛為浙江都司,江西都衛為江西都司,青州都衛為山東都司,成都都衛為四川都司,福州都衛為福建都司,武昌都衛為湖廣都司,廣東都衛為廣東都司。廣西都衛為廣西都司,定遼都衛為遼東都司,河南都衛為河南都司。行都司三,西安行都衛為陜西行都司,大同都衛為山西行都司,建寧都衛為福建行都司。十五年,增置貴州、雲南二都司。後以北平都司為北平行都司。永樂元年改為大寧都司。宣德中,增置萬全都司。計天下都司凡十有六。十三省都司外,有遼東、大寧、萬全三都司。又於建昌置四川行都司,於鄖陽置湖廣行都司。計天下行都司凡五。

明初,又於各行省置都鎮撫司,設都鎮撫,從四品副鎮撫,從五品知事。從八品吳元年改都鎮撫正五品,副鎮撫正六品,知事為提控案牘,省注。洪武六年罷。

衛指揮使司,設官如京衛。品秩並同外衛各統於都司、行都司或留守司。率世官,或有流官。凡襲替、升授、優給、優養及屬所軍政,掌印、僉事報都指揮使司,達所隸都督府,移兵部。每歲撫、按察其賢否,五歲一考選軍政,廢置之。凡管理衛事,惟屬掌印、僉書。不論指揮使、同知、僉事,考選其才者充之。分理屯田、驗軍、營操、巡捕、漕運、備禦、出哨、入衛、戍守、軍器諸雜務,曰見任管事;不任事入隊,曰帶俸差操。征行,則率其屬,聽所命主帥調度。

所,千戶所,正千戶一人,正五品副千戶二人,從五品,鎮撫二人,從六品其屬,吏目一人。所轄百戶所凡十,共百戶十人,正六品。升授、改調、增置無定員。總旗二十人,小旗百人。其守禦千戶所,軍民千戶所設官並同。凡千戶,一人掌印,一人僉書,曰管軍。千戶、百戶,有試,有實授。其掌印,恒以一人兼數印。凡軍政,衛下於所,千戶督百戶,百戶下總旗、小旗,率其卒伍以聽令。鎮撫無獄事,則管軍,百戶缺,則代之。其守禦千戶所,不隸衛,而自達於都司。凡衛所皆隸都司,而都司又分隸五軍都督府。浙江都司、山東都司、遼東都司,隸左軍都督府。陜西都司、陜西行都司、四川都司、四川行都司、廣西都司、雲南都司、貴州都司,隸右軍都督府。中都留守司、河南都司,隸中軍都督府。興都留守司、湖廣都司、湖廣行都司、福建都司、福建行都司、江西都司、廣東都司,隸前軍都督府。大寧都司、萬全都司、山西都司、山西行都司,隸後軍都督府。

明初,置千戶所,設正千戶,正五品副千戶,從五品鎮撫、百戶。正六品又立各萬戶府,設正萬戶,正四品副萬戶,從四品知事,從八品照磨。正九品尋以名不稱實,遂罷萬戶府,而設指揮使及千戶等官。核諸將所部有兵五千者為指揮使,千人者為千戶,百人者為百戶,五十人為總旗,十人為小旗。洪武二年,置刻期百戶所,選能疾行者二百人,以百戶領之。七年,申定衛所之制。先是,內外衛所,凡一衛統十千戶,一千戶統十百戶,百戶領總旗二,總旗領小旗五,小旗領軍十。至是更定其制,每衛設前、後、中、左、右五千戶所,大率以五千六百人為一衛,一千一百二十人為一千戶所,一百一十二人為一百戶所,每百戶所設總旗二人,小旗十人。二十年,始命各衛立掌印、僉書,專職理事,以指揮使掌印,同知、僉事各領一所。士卒有武藝不嫻、器械不利者,皆責所領之官。二十三年,又設軍民指揮使司、軍民千戶所,計天下內外衛凡五百四十有七,所凡二千五百九十有三。自衛指揮以下其官多世襲,其軍士亦父子相繼,為一代定制。

土官,宣慰使司,宣慰使一人,從三品同知一人,正四品副使一人,從四品僉事一人。正五品經歷司,經歷一人,從七品都事一人。正八品

宣撫司,宣撫使一人,從四品同知一人,正五品副使一人,從五品僉事一人。正六品經歷司,經歷一人,從八品知事一人,正九品照磨一人。從九品

安撫司,安撫使一人,從五品同知一人,正六品副使一人,從六品僉事一人。正七品其屬,吏目一人。從九品

招討司,招討使一人,從五品副招討一人。正六品其屬,吏目一人。從九品

長官司,長官一人,正六品副長官一人,從七品其屬,吏目一人。未入流蠻夷長官司,長官、副長官各一人。品同上又有蠻夷官、苗民官及千夫長、副千夫長等官。

軍民府、土州、土縣,設官如府州縣。

洪武七年,西南諸蠻夷朝貢,多因元官授之,稍與約束,定征徭差發之法。漸為宣慰司者十一,為招討司者一,為宣撫司者十,為安撫司者十九,為長官司者百七十有三。其府州縣正貳屬官,或土或流,大率宣慰等司經歷皆流官,府州縣佐貳多流官。皆因其俗,使之附輯諸蠻,謹守疆土,修職貢,供徵調,無相攜貳。有相仇者,疏上聽命於天子。又有番夷都指揮使司三,衛指揮使司三百八十五,宣慰司三,招討司六,萬戶府四,千戶所四十一,站七,地面七,寨一,詳見《兵志·衛所》中。並以附寨番夷官其地。

