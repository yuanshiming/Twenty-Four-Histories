\article{列傳第一百九十五 佞幸}

\begin{pinyinscope}
漢史所載佞倖,如藉孺、閎孺、鄧通、韓嫣、李延年、董賢、張放之屬,皆以宦寺弄臣貽譏千古,未聞以武夫、健兒、貪人、酷吏、方技、雜流任親暱承寵渥於不衰者也。明興,創設錦衣衛,典新軍,暱居肘腋。成祖即位,知人不附己,欲以威讋天下,特任紀綱為錦衣,寄耳目。綱刺廷臣陰事,以希上指,帝以為忠,被殘殺者不可勝數。英宗時,門達、逮杲之徒,並見親信。至其後,廠衛遂相表裏,清流之禍酷焉。憲宗之世,李孜省、僧繼曉以祈禱被寵任,萬安、尹直、彭華等至因之以得高位。武宗日事般遊,不恤國事,一時宵人並起,錢寧以錦衣幸,臧賢以伶人幸,江彬、許泰以邊將幸,馬昂以女弟幸。禍流中外,宗社幾墟。世宗入繼大統,宜矯前軌,乃任陸炳於從龍,寵郭勛於議禮,而一時方士如陶仲文、邵元節、藍道行之輩,紛然並進,玉杯牛帛,詐妄滋興。凡此諸人,口銜天憲,威福在手,天下士大夫靡然從風。雖以成祖、世宗之英武聰察,而嬖倖釀亂,幾與昏庸失道之主同其蒙蔽。彼第以親己為可信,而孰知其害之至於此也。至顧可學、盛端明、朱隆禧之屬,皆起家甲科,致位通顯,乃以秘術干榮,為世戮笑。此亦佞倖之尤者,附之篇末,用以示戒云。

○紀綱門達逮杲李孜省繼曉江彬許泰錢寧陸炳邵元節陶仲文顧可學盛端明等

紀綱,臨邑人,為諸生。燕王起兵過其縣,綱叩馬請自效。王與語,說之。綱善騎射,便辟詭黠,善鉤人意向。王大愛幸,授忠義衛千戶。既即帝位,擢錦衣衛指揮使,令典親軍,司詔獄。

都御史陳瑛滅建文朝忠臣數十族,親屬被戮者數萬人。綱覘帝旨,廣布校尉,日摘臣民陰事。帝悉下綱治,深文誣詆。帝以為忠,親之若肺腑。擢都指揮僉事,仍掌錦衣。綱用指揮莊敬、袁江,千戶王謙、李春等為羽翼,誣逮浙江按察使周新,致之死。帝所怒內侍及武臣下綱論死,輒將至家,洗沐好飲食之,陽為言,見上必請赦若罪,誘取金帛且盡,忽刑於市。

數使家人偽為詔,下諸方鹽場,勒鹽四百餘萬。還復稱詔,奪官船二十、牛車四百輛,載入私第,弗予直。構陷大賈數十百家,罄其資乃已。詐取交址使珍奇。奪吏民田宅。籍故晉王、吳王,乾沒金寶無算。得王冠服服之,高坐置酒,命優童奏樂奉觴,呼萬歲,器物僭乘輿。欲買一女道士為妾,都督薛祿先得之,遇祿大內,撾其首,腦裂幾死。恚都指揮啞失帖木不避道,誣以冒賞事,捶殺之。腐良家子數百人,充左右。詔選妃嬪,試可,令暫出待年,綱私納其尤者。吳中故大豪沈萬三,洪武時籍沒,所漏貲尚富。其子文度蒲伏見綱,進黃金及龍角、龍文被、奇寶異錦,願得為門下,歲時供奉。綱乃令文度求索吳中好女。文度因挾綱勢,什五而中分之。

綱又多蓄亡命,造刀甲弓弩萬計。端午,帝射柳,綱屬鎮撫龐瑛曰:「我故射不中,若折柳鼓噪,以覘眾意。」瑛如其言,無敢糾者。綱喜曰:「是無能難我矣。」遂謀不軌。十四年七月,內侍仇綱者發其罪,命給事、御史廷劾,下都察院按治,具有狀。即日磔綱於市,家屬無少長皆戍邊,列罪狀頒示天下。其黨敬、江、謙、春、瑛等,誅譴有差。

門達,豐潤人。襲父職為錦衣衛百戶。性機警沉鷙。正統末,進千戶,理鎮撫司刑。久之,遷指揮僉事,坐累解職。景泰七年復故官,佐理衛事兼鎮撫理刑。天順改元,與「奪門」功,進指揮同知。旋進指揮使,專任理刑。千戶謝通者,浙江人也,佐達理司事,用法仁恕,達倚信之。重獄多平反,有罪者以下禁獄為幸,朝士翕然稱達賢。然是時英宗慮廷臣黨比,欲知外事,倚錦衣官校為耳目,由是逯杲得大幸,達反為之用。

逯杲者,安平人也,以錦衣衛校尉為達及指揮劉敬腹心,從「奪門」。帝大治奸黨,杲縛錦衣百戶楊瑛,指為張永親屬,又執千戶劉勤於朝,奏其訕上,兩人並坐誅。用楊善薦,授本衛百戶。以捕妖賊功,進副千戶。又用曹吉祥薦,擢指揮僉事。帝以杲強鷙,委任之,杲乃摭群臣細故以稱帝旨。英國公張懋、太平侯張瑾、外戚會昌侯孫繼宗兄弟並侵官田,杲劾奏,還其田於官。懋等皆服罪,乃已。石亨恃寵不法,帝漸惡之,杲即伺其陰事。亨從子彪有罪下獄,命杲赴大同械其黨都指揮朱諒等七十六人。杲因發彪弟慶他罪,連及者皆坐,杲進指揮同知。明年復奏亨怨望,懷不軌,亨下獄死。有詔盡革「奪門」功,達、杲言臣等俱特恩,非以亨故。帝優詔留任,以杲發亨奸,益加倚重。

杲益發舒,勢出達上。白遣校尉偵事四方,文武大吏、富家高門多進伎樂貨賄以祈免,親籓郡王亦然。無賄者輒執送達,鍛煉成獄。天下朝覲官大半被譴,逮一人,數大家立破。四方奸民詐稱校尉,乘傳縱橫,無所忌。鼓城伯張瑾以葬妻稱疾不朝,而與諸公侯飲私第。杲劾奏,幾得重罪。杲所遣校尉誣寧府弋陽王奠壏母子亂,帝遣官往勘,事已白,靖王奠培等亦言無左驗。帝怒責杲,杲執如初,帝竟賜奠壏母子死。方舁尸出,大雷雨,平地水數尺,人咸以為冤。指揮使李斌嘗構殺弘農衛千戶陳安,為安家所訴,下巡按御史邢宥覆讞,石亨囑宥薄斌罪。至是,校尉言:「斌素藏妖書,謂其弟健當有大位,欲陰結外番為石亨報仇。」杲以聞,下錦衣獄,達坐斌謀反。帝兩命廷臣會訊,畏杲不敢平反。斌兄弟置極刑,坐死者二十八人。

杲本由石亨、曹吉祥進,訐亨致死,復奏吉祥及其從子欽陰事,吉祥、欽大恨。五年七月,欽反,入杲第斬之,取其首以去。事平,贈杲指揮使,給其子指揮僉事俸。

時達已掌衛事,仍兼理刑。杲被殺,達以守衛功,進都指揮僉事。初,杲給事達左右,及得志恣甚。達怒,力逐之。杲旋復官,欲傾達,達惴惴不敢縱。杲死,達勢遂張。欲踵杲所為,益布旗校於四方。告訐者日盛,中外重足立,帝益以為能。

外戚都指揮孫紹宗及軍士六十七人冒討曹欽功,達發其事。紹宗被責讓,餘悉下獄。盜竊戶部山西司庫金,巡城御史徐茂劾郎中趙昌、主事王珪、徐源疏縱。達治其事,皆下獄謫官。達以囚多,獄舍少,不能容,請城西武邑庫隙地增置之,報可。御史樊英、主事鄭瑛犯贓罪。給事中趙忠等報不以實。達劾其徇私,亦下獄謫官。給事中程萬里等五人直登聞鼓,有軍士妻醖冤,會齋戒不為奏。達劾諸人蒙蔽,詔下達治。已,劾南京戶部侍郎馬諒,左都御史石璞,掌前府忻城伯趙榮,都督同知范雄、張斌老聵,皆罷去。裕州民奏知州秦永昌衣黃衣閱兵。帝怒,命達遣官核,籍其貲,戮永昌,榜示天下。並逮布政使侯臣、按察使吳中以下及先後巡按御史吳琬等四人下獄,臣等停俸,琬等謫縣丞。御史李蕃按宣府,或告蕃擅撻軍職,用軍容迎送。御史楊璡按遼東,韓琪按山西,校尉言其妄作威福。皆下達治,蕃、琪並荷校死。陜西督儲參政婁良,糊廣參議李孟芳,陜西按察使錢博,福建僉事包瑛,陜西僉事李觀,四川巡按田斌,雲南巡按張祚,清軍御史程萬鐘及刑部郎中馮維、孫瓊,員外郎貝鈿,給事中黃甄,皆為校尉所發下獄。瑛守官無玷,不勝憤,自縊死,其他多遣戍。湖廣諸生馬雲罪黜,詐稱錦衣鎮撫,奉命葬親,布政使孫毓等八人咸賻祭。事覺,法司請逮問,卒不罪雲。達初欲行督責之術,其同列呂貴曰:「武臣不易犯,曹欽可鑒也。獨文吏易裁耳。」達以為然,故文吏禍尤酷。

都指揮袁彬恃帝舊恩,不為達下。達深銜之,廉知彬妾父千戶王欽誆人財,奏請下彬獄,論贖徒還職。有趙安者,初為錦衣力士役於彬,後謫戍鐵嶺衛,赦還,改府軍前衛,有罪,下詔獄。達坐安改補府軍由彬請託故,乃復捕彬,搒掠,誣彬受石亨、曹欽賄,用官木為私第,索內官督工者磚瓦,奪人子女為妾諸罪名。軍匠楊塤不平,擊登聞鼓為彬訟冤,語侵達,詔並下達治。當是時,達害大學士李賢寵,又數規己,嘗譖於帝,言賢受陸瑜金,酬以尚書。帝疑之,不下詔者半載。至是,拷掠塤,教以引賢,塤即謬曰:「此李學士導我也。」達大喜,立奏聞,請法司會鞫塤午門外。帝遣中官裴當監視。達欲執賢並訊,當曰:「大臣不可辱。」乃止。及訊,塤曰:「吾小人,何由見李學士,此門錦衣教我。」達色沮不能言,彬亦歷數達納賄狀,法司畏達不敢聞,坐彬絞輸贖,塤斬。帝命彬贖畢調南京錦衣,而禁錮塤。

明年,帝疾篤,達知東宮局丞王綸必柄用,預為結納。無何,憲宗嗣位,綸敗,達坐調貴州都勻衛帶俸差操。甫行,言官交章論其罪。命逮治,論斬繫獄,沒其貲巨萬,指揮張山同謀殺人,罪如之。子序班升、從子千戶清、婿指揮楊觀及其黨都指揮牛循等九人,謫戍、降調有差。後當審錄,命貸達,發廣西南丹衛充軍,死。

李孜省,南昌人。以布政司吏待選京職,贓事發,匿不歸。時憲宗好方術,孜省乃學五雷法,厚結中官梁芳、錢義,以符籙進。成化十五年,特旨授太常丞。御史楊守隨、給事中李俊等劾孜省贓吏,不宜典祭祀,乃改上林苑監丞。日寵幸,賜金冠、法劍及印章二,許密封奏請。益獻淫邪方術,與芳等表裏為奸,漸干預政事。十七年,擢右通政,寄俸本司,仍掌監事。同官王昶輕之,不加禮。孜省譖昶,左遷太僕少卿。故事,寄俸官不得預郊壇分獻,帝特以命孜省。廷臣懲昶事,無敢執奏者。

初,帝踐位甫踰月,即命中官傳旨,用工人為文思院副使。自後相繼不絕,一傳旨姓名至百十人,時謂之傳奉官,文武、僧道濫恩澤者數千。鄧常恩、趙玉芝、凌中、顧玒及奸僧繼曉輩,皆尊顯,與孜省相倚為奸,然權寵皆出孜省下。居二年,進左通政。給事中王瑞、御史張稷等交劾之。乃貶二秩,為本司左參議,他貶黜者又十二人。蓋特借以塞中外之望,孜省寵固未嘗替也。頃之,復遷左通政。

二十一年正月,星變求言。九卿大臣、給事御史皆極論傳奉官之弊,首及孜省、常恩等。帝頗感悟,貶孜省上林監丞,令吏部錄冗濫者名凡五百餘人。帝為留六十七人,餘皆斥罷,中外大說。孜省緣是恨廷臣甚,構逐主事張吉、員外郎彭綱,而益以左道持帝意。其年十月,再復左通政,益作威福。構罪吏部尚書尹旻及其子侍講龍。又假扶鸞術言江西人赤心報國,於是致仕副都御史劉敷、禮部郎中黃景、南京兵部侍郎尹直、工部尚書李裕、禮部侍郎謝一夔,皆因之以進。間採時望,若學士楊守陳、倪岳,少詹事劉健,都御史餘子俊,李敏諸名臣,悉密封推薦。搢紳進退,多出其口,執政大臣萬安、劉吉、彭華從而附麗之。通政邊鏞為僉都御史,李和為南京戶部侍郎,皆其力也。所排擠江西巡撫閔珪、洗馬羅璟、兵部尚書馬文升、順天府丞楊守隨,皆被譴,朝野側目。

吏部奏通政使缺,即以命孜省,而右通政陳政以下五人,遞進一官。時張文質方以尚書掌司事,通政故未嘗缺使也。已,復擢禮部右侍郎,掌通政如故。

常恩,臨江人,因中官陳喜進。玉芝,番禺人,因中官高諒進。並以曉方術,累擢太常卿。玉芝丁母憂,特賜祭葬,大治塋域,制度踰等。玒、中不知何許人。玒以扶鸞術,累官太常少卿,喪母賜祭,且給贈誥。故事,四品未三載無給誥賜祭者,憲宗特予之。吏部尚書尹旻因請並贈其父。未幾,進本寺卿。其二子經、綸,亦官太常少卿。中以善書供事文華殿,不數年為太常卿。踰月,以諫官言,降寺丞。孜省以星變貶,常恩亦貶本寺丞,而玉芝、玒、中並如故。孜省復通政,常恩亦復太常卿。

有李文昌者,試術不效,杖五十,斥還。岳州通判沈政以繪事夤緣至太常少卿,請斂天下貨財充內府。帝怒,下獄,杖謫廣西慶遠通判。人頗以為快。

然群奸中外蟠結,士大夫附者日益多。進士郭宗由刑部主事,以篆刻為中人所引,擢尚寶少卿,日與市井工技伍,趨走闕廷。兵科左給事中張善吉謫官,因秘術干中官高英,得召見,因自陳乞復給事中,士論以為羞。大學士萬安亦獻房中術以固寵。而諸雜流加侍郎、通政、太常、太僕、尚寶者,不可悉數。

憲宗崩,孝宗嗣位,始用科道言,盡汰傳奉官,謫孜省、常恩、玉芝、玒、中、經戍邊衛。又以中官蔣琮言,逮孜省、常恩、玉芝等下詔獄,坐交結近侍律斬,妻子流二千里。詔免死,仍戍邊。孜省不勝搒掠,瘐死。

繼曉,江夏僧也。憲宗時,以秘術因梁芳進,授僧錄司左覺義。進右善世,命為通元翊教廣善國師。日誘帝為佛事,建大永昌寺於西市,逼徙民居數百家,費國帑數十萬。員外郎林俊請斬芳、繼曉以謝天下,幾得重譴。繼曉虞禍及,乞歸養母,並乞空名度牒五百道,帝悉從之。帝初即位,即以道士孫道玉為真人。其後西番僧答刂巴堅參封萬行莊嚴功德最勝智慧圓明能仁感應顯國光教弘妙大悟法王西天至善金剛普濟大智慧佛,其徒答刂實巴、鎖南堅參、端竹也失皆為國師,錫誥命。服食器用,僭擬王者。出入乘梭輿,衛卒執金吾仗前導,錦衣玉食幾千人。取荒塚頂骨為數珠,髑髏為法碗。給事中魏元等切諫,不納。尋進答刂實巴為法王,班卓兒藏卜為國師,又封領占竹為萬行清脩真如自在廣善普慧弘度妙應掌教翊國正覺大濟法王西天圓智大慈悲佛,又封西天佛子答刂失藏卜、答刂失堅參、乳奴班丹、鎖南堅參、法領占五人為法王,其他授西天佛子、大國師、國師、禪師者不可勝計。羽流加號真人、高士者,亦盈都下。大國師以上金印,真人玉冠、玉帶、玉珪、銀章。繼曉尤奸黠竊權,所奏請立從。成化二十一年,星變,言官極論其罪,始勒為民,而諸番僧如故。

孝宗初,詔禮議汰。禮官言諸寺法王至禪師四百三十七人,刺麻諸僧七百八十九人。華人為禪師及善世、覺義諸僧官一百二十人,道士自真人、高士及正一演法諸道官一百二十三人,請俱貶黜。詔法王、佛子遞降國師、禪師、都綱,餘悉落職為僧,遣還本土,追奪誥敕、印章、儀仗諸法物。真人降左正一,高士降左演法,亦追奪印章及諸玉器。僧錄司止留善世等九員,道錄司留正一等八員,餘皆廢黜。而繼曉以科臣林廷玉言,逮治棄市。

江彬,宣府人。初為蔚州衛指揮僉事。正德六年,畿內賊起,京軍不能制,調邊兵。彬以大同遊擊隸總兵官張俊赴調。過薊州,殺一家二十餘人,誣為賊,得賞。後與賊戰淮上,被三矢,其一著面,鏃出於耳,拔之更戰。武宗聞而壯之。七年,賊漸平,遣邊兵還鎮大同、宣府。軍過京師,犒之,遂并宣府守將許泰皆留不遣。彬因錢寧得召見。帝見其矢痕,呼曰:「彬健能爾耶!」彬狡黠強很,貌魁碩有力,善騎射,談兵帝前,帝大說,擢都指揮僉事,出入豹房,同臥起。嘗與帝弈不遜,千戶周騏叱之。彬陷騏搒死,左右皆畏彬。彬導帝微行,數至教坊司;進鋪花氈幄百六十二間,制與離宮等,帝出行幸皆御之。

寧見彬驟進,意不平。一日,帝捕虎,召寧,寧縮不前。虎迫帝,彬趨撲乃解。帝戲曰:「吾自足辦,安用爾。」然心德彬而嗛寧。寧他日短彬,帝不應。彬知寧不相容,顧左右皆寧黨,欲籍邊兵自固,固盛稱邊軍驍悍勝京軍,請互調操練。言官交諫,大學士李東陽疏稱十不便,皆不聽。於是調遼東、宣府、大同、延綏四鎮軍入京師,號外四家,縱橫都市。每團練大內,間以角牴戲。帝戎服臨之,與彬聯騎出,鎧甲相錯,幾不可辨。

八年命許泰領敢勇營,彬領神威營。改太平倉為鎮國府,處邊兵。建西官廳於奮武營。賜彬、泰國姓。越二年,遷都督僉事。彬薦萬全都指揮李琮、陜西都指揮神周勇略,並召侍豹房,同賜姓為義兒。毀積慶、鳴玉二坊民居,造皇店酒肆,建義子府。四鎮軍,彬兼統之。帝自領群閹善射者為一營,號中軍。晨夕馳逐,甲光照宮苑,呼噪聲達九門。帝時臨閱,名過錦。諸營悉衣黃罩甲,泰、琮、周等冠遮陽帽,帽植天鵝翎,貴者三翎,次二翎。兵部尚書王瓊得賜一翎,自喜甚。

彬既心忌寧,欲導帝巡幸遠寧。因數言宣府樂工多美婦人,且可觀邊釁,瞬息馳千里,何鬱鬱居大內,為廷臣所制。帝然之。十二年八月,急裝微服出幸昌平,至居庸關,為御史張欽所遮,乃還。數日,復夜出。先令太監谷大用代欽,止廷臣追諫者。因度居庸,幸宣府。彬為建鎮國府第,悉輦豹房珍玩、女御實其中。彬從帝,數夜入人家,索婦女。帝大樂之,忘歸,稱曰家裏。未幾,幸陽和。迤北五萬騎入寇,諸將王勛等力戰。至應州,寇引去。斬首十六級,官軍死數百人,以捷聞京師。帝自稱威武大將軍朱壽,又自稱鎮國公,所駐蹕稱軍門。中外事無大小,白彬乃奏,或壅格至二三歲。廷臣前後切諫,悉置不省。

十三年正月還京,數念宣府。彬復導帝往,因幸大同。聞太皇太后崩,乃還京發喪。將葬,如昌平,祭告諸陵,遂幸黃花、密雲。彬等掠良家女數十車,日載以隨,有死者。永平知府毛思義忤彬,下獄謫官。典膳李恭疏請回鑾,指斥彬罪。未及止,彬逮恭死詔獄。帝駐大喜峰口,欲令朵顏三衛花當、把兒孫等納質宴勞,御史劉士元陳四不可,不報。帝既還,下詔稱總督軍務威武大將軍總兵官朱壽統率六軍,而命彬為威武副將軍。錄應州功,封彬平虜伯;子三人,錦衣衛指揮;泰,安邊伯;琮、周,俱都督。陞賞內外官九千五百五十餘人,賞賜億萬計。

彬又導帝由大同渡黃河,次榆林,至綏德,幸總兵官戴欽第,納其女。還,由西安歷偏頭關,抵太原,大徵女樂,納晉府樂工楊騰妻劉氏以歸。彬與諸近幸皆母事之,稱曰劉娘娘。初,延綏總兵官馬昂罷免,有女弟善歌,能騎射,解外國語,嫁指揮畢春,有娠矣。昂因彬奪歸,進於帝,召入豹房,大寵。傳升昂右都督,弟炅、昶並賜蟒衣,大璫皆呼為舅,賜第太平倉。給事、御史諫,不應。嘗幸昂第,召其妾。昂不聽,帝怒而起。昂復結太監張忠進其妾杜氏,遂傳陞炅都指揮,昶儀真守備。昂喜過望,又進美女四人謝恩。及是,納欽女,皆彬所導也。

十四年正月自太原還至宣府,命彬提督十二團營。帝東西遊幸,歷數千里,乘馬腰弓矢,涉險阻,冒風雪,從者多道病,帝無倦容。及還京,復欲南幸。刑部主事汪金疏陳九不可,且極言酣酒當戒,帝不省。廷臣百餘人伏闕諫,彬故激帝怒,悉下獄,多杖死者。彬亦意沮,議得寢。

會寧王宸濠反,彬復贊帝親征,下令諫者處極刑。命彬提督贊畫機密軍務,並督東廠錦衣官校辦事。是時,張銳治東廠,錢寧治錦衣,彬兼兩人之任,權勢莫與比,遂扈帝以行。尋止寧,令董皇店役,不得從。八月發京師。彬在途,矯旨輒縛長吏,通判胡琮懼,自縊死。十二月至揚州,即民居為都督府,遍刷處女、寡婦,導帝漁獵。以劉姬諫,稍止。至南京,又欲導帝幸蘇州,下浙江,抵湖、湘。諸臣極諫,會其黨亦勸沮,乃止。當是時,彬率邊兵數萬,跋扈甚。成國公朱輔為長跪,魏國公徐鵬舉及公卿大臣皆側足事之。惟參贊尚書喬宇、應天府丞寇天敘挺身與抗,彬氣稍折。

十五年六月幸牛首山。諸軍夜驚,言彬欲為逆,久之乃定。時宸濠已就擒,繫江上舟中,民間數訛傳將為變。帝心疑,欲歸。閏八月發南京。至清江浦,漁積水池,帝舟覆被溺,遂得疾。十月,帝至通州。彬尚欲勸帝幸宣府,矯旨召勛戚大臣議宸濠獄。又上言:「賴鎮國公硃壽指授方略,擒宸濠逆黨申宗遠等十五人,乞明正其罪。」乃下詔褒賜鎮國公,歲加彬祿米百石,蔭一子錦衣千戶。會帝體憊甚,左右力請乃還京。彬猶矯旨改團練營為威武團練營,自提督軍馬,令泰、周、琮等提督教場操練。

及帝崩,大學士楊廷和用遺命,分遣邊兵,罷威武團練營。彬內疑,稱疾不出,陰布腹心,衷甲觀變,令泰詣內閣探意。廷和以溫語慰之,彬稍安,乃出成服。廷和密與司禮中官魏彬計,因中官溫祥入白太后,請除彬。會坤寧宮安獸吻,即命彬與工部尚書李鐩入祭。彬禮服入,家人不得從。事竟將出,中官張永留彬、鐩飯,太后遽下詔收彬。彬覺,亟走西安門,門閉。尋走北安門,門者曰:「有旨留提督。」彬曰:「今日安所得旨?」排門者。門者執之,拔其鬚且盡。收者至,縛之。有頃,周、琮並縛至,罵彬曰:「奴早聽我,豈為人擒!」世宗即位,磔彬於市,周、琮與彬子勳、傑、鰲、熙俱斬,繪處決圖,榜示天下,幼子然及妻、女俱發功臣家為奴。時京師久旱,遂大雨。籍彬家,得黃金七十櫃,白金二千二百櫃,他珍珤不可數計。許泰,江都人。都督寧子,襲職為羽林前衛指揮使。中武會舉第一,擢署都指揮同知。尋充副總兵,協守宣府。正德六年,與郤永、江彬俱調剿流賊,敗賊霸州,追敗之東光半壁店。未幾,復敗賊棗強。劉六寇曹州,泰與馮楨、郤永擊卻之,乘勝擒斬千八百人。賊犯蠡縣、臨城,泰等不敢擊,被劾停俸。既而賊奔衛輝,泰為所敗。調赴萊陽,逗遛不進,詔革署都督僉事新銜,仍以都指揮同知辦賊。賊平,進署都督同知,留京師,與彬日侍左右,賜國姓,歷遷左都督。冒應州功,封安邊伯。

宸濠反,帝以泰為威武副將軍,偕中官張忠率禁軍先往。宸濠已為王守仁所擒。泰欲攘其功,疾馳至南昌,窮搜逆黨,士民被誣陷者不可勝計。誅求刑戮,甚於宸濠之亂。嫉守仁功,排擠之百方。執伍文定,窘辱備至。居久之,始旋師。世宗即位,廷臣交劾,文定亦備以虐民妒功狀上聞,下獄論死。夤緣貴近,減死徙邊」馬昂亦罷,炅等戍邊。

錢寧,不知所出,或云鎮安人。幼鬻太監錢能家為奴,能嬖之,冒錢姓。能死,推恩家人,得為錦衣百戶。正德初,曲事劉瑾,得幸於帝。性蝟狡,善射,拓左右弓。帝喜,賜國姓,為義子,傳陞錦衣千戶。瑾敗,以計免,歷指揮使,掌南鎮撫司。累遷左都督,掌錦衣衛事,典詔獄,言無不聽,其名刺自稱皇庶子。引樂工臧賢、回回人於永及諸番僧,以秘戲進。請於禁內建豹房、新寺,恣聲伎為樂,復誘帝微行。帝在豹房,常醉枕寧臥。百官候朝,至晡莫得帝起居,密伺寧,寧來,則知駕將出矣。

太監張銳領東廠緝事,橫甚,而寧典詔獄,勢最熾,中外稱曰「廠、衛」。司務林華、評事沈光大皆以杖繫校尉,為寧所奏,逮下錦衣獄,黜光大,貶華一級。錦衣千戶王注與寧暱,撻人至死,員外郎劉秉鑑持其獄急。寧匿注於家,而屬東廠發刑部他事。尚書張子麟亟造謝寧,立釋注,乃已。廠衛校卒至部院白事,稱尚書子麟輩曰老尊長。太僕少卿趙經初以工部郎督乾清宮工,乾沒帑金數十萬。經死,寧佯遣校尉治喪,迫經妻子扶櫬出,姬妾、帑藏悉據有之。中官廖常鎮河南,其弟錦衣指揮鵬肆惡,為巡撫鄧庠所劾,詔降級安置。鵬懼,使其嬖妾私事寧,得留任。

寧子永安,六歲為都督。養子錢傑、錢靖等,俱冒國姓,授錦衣衛官。念富貴已極,帝無子,思結強籓自全。為寧王宸濠營復護衛,又遣人往宸濠所,有異謀。又令宸濠數進金銀玩好於帝。謀召其世子司香太廟,為入嗣地。又以玉帶、彩糸寧附其典寶萬銳歸,詐稱上賜。凡宸濠所遣私人行賄京師,皆主伶人臧賢家,由寧以達帝左右。

宸濠反,帝心疑寧。寧懼,白帝收宸濠所遣盧孔章,而歸罪賢,謫戍邊,使校尉殺之途以滅口,又致孔章瘐死,冀得自全。然卒中江彬計,使董皇店役。彬在道,盡白其通逆狀。帝曰:「黠奴,我固疑之。」乃羈之臨清,馳收其妻子家屬。帝還京,裸縛寧,籍其家,得玉帶二千五百束、黃金十餘萬兩、白金三千箱、胡椒數千石。世宗即位,磔寧於市。養子傑等十一人皆斬,子永安幼,免死,妻妾發功臣家為奴。

陸炳,其先平湖人。祖墀,以軍籍隸錦衣衛為總旗。父松,襲職,從興獻王之國安陸,選為儀衛司典仗。世宗入承大統,松以從龍恩,遷錦衣副千戶。累官後府都督僉事,協理錦衣事。

世宗始生,松妻為乳媼,炳幼從母入宮中。稍長,日侍左右。炳武健沉鷙,長身火色,行步類鶴。舉嘉靖八年武會試,授錦衣副千戶。松卒,襲指揮僉事。尋進署指揮使,掌南鎮撫事。十八年從帝南幸,次衛輝。夜四更,行宮火,從官倉猝不知帝所在。炳排闥負帝出,帝自是愛幸炳。屢擢都指揮同知,掌錦衣事。

帝初嗣位,掌錦衣者硃宸,未久罷。代者駱安,繼而王佐、陳寅,皆以興邸舊人掌錦衣衛。佐嘗保持張鶴齡兄弟獄,有賢聲。寅亦謹厚不為惡。及炳代寅,權勢遠出諸人上。未幾,擢署都督僉事。又以緝捕功,擢都督同知。炳驟貴,同列多父行,炳陽敬事之,徐以計去其易己者。又能得閣臣夏言、嚴嵩歡,以故日益重。嘗捶殺兵馬指揮,為御史所糾,詔不問。言故暱炳,一日,御史劾炳諸不法事,言即擬旨逮治。炳窘,行三千金求解不得,長跪泣謝罪,乃已。炳自是嫉言次骨。及嵩與言構,炳助嵩,發言與邊將關節書,言罪死。嵩德炳,恣其所為,引與籌畫,通賕賂。後仇鸞得寵,陵嵩出其上,獨憚炳。炳曲奉之,不敢與鈞禮,而私出金錢結其所親愛,得鸞陰私。及鸞病亟,炳盡發其不軌狀。帝大驚,立收鸞敕印,鸞憂懼死,至剖棺戮屍。

炳先進左都督,錄擒哈舟兒功,加太子太保。以發鸞密謀,加少保兼太子太傅,歲給伯祿。三十三年命入直西苑,與嚴嵩、朱希忠等侍修玄。三十五年三月賜進士恩榮宴。故事,錦衣列於西。帝以炳故,特命上坐,班二品之末。明年疏劾司禮中官李彬侵盜工所物料,營墳墓,僭擬山陵,與其黨杜泰三人論斬,籍其貲,銀四十餘萬,金珠珍寶無算。尋加炳太保兼少傅,掌錦衣如故。三公無兼三孤者,僅於炳見之。

炳任豪惡吏為爪牙,悉知民間銖兩奸。富人有小過輒收捕,沒其家。積貲數百萬,營別宅十餘所,莊園遍四方,勢傾天下。時嚴嵩父子盡攬六曹事,炳無所不關說。文武大吏爭走其門,歲入不貲,結權要,周旋善類,亦無所吝。帝數起大獄,炳多所保全,折節士大夫,未嘗構陷一人,以故朝士多稱之者。二十九年卒官。贈忠誠伯,謚武惠,祭葬有加,官其子繹為本衛指揮僉事。隆慶初,用御史言,追論炳罪,削秩,籍其產,奪繹及弟太常少卿煒官,坐贓數十萬,繫繹等追償,久之貲盡。萬歷三年,繹上章乞免。張居正等言,炳救駕有功,且律非謀反叛逆奸黨,無籍沒者;況籍沒、追贓,二罪並坐,非律意。帝憫之,遂獲免。

邵元節,貴溪人,龍虎山上清宮道士也。師事范文泰、李伯芳、黃太初,咸盡其術。寧王宸濠召之,辭不往。世宗嗣位,惑內侍崔文等言,好鬼神事,日事齋醮。諫官屢以為言,不納。嘉靖三年,徵元節入京,見於便殿,大加寵信,俾居顯靈宮,專司禱祀。雨雪愆期,禱有驗,封為清微妙濟守靜修真凝玄衍範志默秉誠致一真人,統轄朝天、顯靈、靈濟三宮,總領道教,錫金、玉、銀、象牙印各一。

六年乞還山,詔許馳傳。未幾,趨朝。有事南郊,命分獻風雲雷雨壇。預宴奉天殿,班二品。贈其父太常丞、母安人,並贈文泰真人,賜元節紫衣玉帶。給事中高金論之,帝下金詔獄。敕建真人府於城西,以其孫啟南為太常丞,曾孫時雍為太常博士。歲給元節祿百石,以校尉四十人供灑掃,賜莊田三十頃,蠲其租。又遣中使建道院於貴溪,賜名仙源宮。既成,乞假還山。中途上奏,言為大學士李時弟員外旼所侮。時上章引罪,旼下獄獲譴。比還朝,舟至潞河,命中官迎入,賜蟒服及「闡都輔國」玉印。

先是,以皇嗣未建,數命元節建醮,以夏言為監禮使,文武大臣日再上香。越三年,皇子疊生,帝大喜,數加恩元節,拜禮部尚書,賜一品服。孫啟南、徒陳善道等咸進秩,贈伯芳、太初為真人。

帝幸承天,元節病不能從。無何死,帝為出涕,贈少師,賜祭十壇,遣中官錦衣護喪還,有司營葬,用伯爵禮。禮官擬謚榮靖,不稱旨,再擬文康。帝兼用之,曰文康榮靖。啟南官至太常少卿。善道亦封清微闡教崇真衛道高士。隆慶初,削元節稱謚。

陶仲文,初名典真,黃岡人。嘗受符水訣於羅田萬玉山,與邵元節善。

嘉靖中,由黃梅縣吏為遼東庫大使。秩滿,需次京師,寓元節邸舍。寓節年老,宮中黑眚見,治不效,因薦仲文於帝。以符水噀劍,絕宮中妖。莊敬太子患痘,禱之而瘥,帝深寵異。

十八年南巡,元節病,以仲文代。次衛輝,有旋風繞駕,帝問:「此何祥也?」對曰:「主火。」是夕行宮果火,宮人死者甚眾。帝益異之,授神霄保國宣教高士,尋封神霄保國弘烈宣教振法通真忠孝秉一真人。明年八月欲令太子監國,專事靜攝。太僕卿楊最疏諫,杖死,廷臣震懾。大臣爭諂媚取容,神仙禱祀日亟。以仲文子世同為太常丞,子婿吳濬、從孫良輔為太常博士。帝有疾,既而瘳,喜仲文祈禱功,特授少保、禮部尚書。久之,加少傅,仍兼少保。仲文起筦庫,不二歲登三孤,恩寵出元節上。乃請建雷壇於鄉縣,祝聖壽,以其徒臧宗仁為左至靈,馳驛往,督黃州同知郭顯文監之。工稍稽,謫顯文典史,遣工部郎何成代,督趨甚急,公私騷然。御史楊爵、郎中劉魁言及之。給事中周怡陳時事,有「日事禱祠」語。帝大怒,悉下詔獄,拷掠長繫。吏部尚書熊浹諫乩仙,即命削籍。自是,中外爭獻符瑞,焚修、齋醮之事,無敢指及之者矣。

帝自二十年遭宮婢變,移居西內,日求長生,郊廟不親,朝講盡廢,君臣不相接,獨仲文得時見;見輒賜坐,稱之為師而不名。心知臣下必議己,每下詔旨多憤疾之辭,廷臣莫知所指。小人顧可學、盛端明、朱隆禧輩,皆緣以進。其後,夏言以下冠香葉冠,積他釁至死。而嚴嵩以虔奉焚修蒙異眷者二十年。大同獲諜者王三,帝歸功上玄,加仲文少師,仍兼少傅少保。一人兼領三孤,終明世,惟仲文而已。久之,授特進光祿大夫柱國兼支大學士俸,廕子世恩為尚寶丞。復以聖誕加恩,給伯爵俸,授其徒郭弘經、王永寧為高士。時都御史胡纘宗下獄,株連數十人。二十九年春,京師災異頻見,帝以咨仲文。封言慮有冤獄,得雨方解。俄法司上纘宗等爰書,帝悉從輕典,果得雨。乃以平獄功,封仲文恭誠伯,歲祿千二百石,弘經、永寧封真人。仇鸞之追戮也,下詔稱仲文功,增祿百石,廕子世昌國子生。三十二年,仲文言:「齊河縣道士張演建大清橋,浚河得龍骨一,重千斤。又突出石沙一脈,長數丈,類有神相。」帝即發帑銀助之。時建元嶽湖廣太和山,既成,遣英國公張溶往行安神禮,仲文偕顧可學建醮祈福。明年,聖誕,加恩,蔭子錦衣百戶。

帝益求長生,日夜禱祠,簡文武大臣及詞臣入直西苑,供奉青詞。四方奸人段朝用、龔可佩、藍道行、王金、胡大順、藍田玉之屬,咸以燒煉符咒熒惑天子,然不久皆敗,獨仲文恩寵日隆重,久而不替,士大夫或緣以進。又創二龍不相見之說,青宮虛位者二十年。

三十五年,上皇考道號為三天金闕無上玉堂都仙法主玄元道德哲慧聖尊開真仁化大帝,皇妣號為三天金闕無上玉堂總仙法主玄元道德哲慧聖母天后掌仙妙化元君,帝自號靈霄上清統雷元陽妙一飛玄君,後加號九天弘教普濟生靈掌陰陽功過大道思仁紫極仙翁一陽真人元虛玄應開化伏魔真忠孝帝君,再號太上大羅天仙紫極長生聖智昭靈統元證應玉虛總掌五雷大真人玄都境萬壽帝君。明年,仲文有疾,乞還山,獻上歷年所賜蟒玉、金寶、法冠及白金萬兩。既歸,帝念之不置,遣錦衣官存問,命有司以時加禮,改其子尚寶少卿世恩為太常丞兼道錄司右演法,供事真人府。

仲文得寵二十年,位極人臣。然小心慎密,不敢恣肆。三十九年卒,年八十餘。帝聞痛悼,葬祭視邵元節,特謚榮康惠肅。世恩後至太常卿。隆慶元年坐與王金偽製藥物,下獄論死。仲文秩謚亦追削。

段朝用,合肥人。以燒煉乾郭勛,言所化銀皆仙物,用為飲食器,當不死。勛進之帝,帝大悅。仲文亦薦之,獻萬金助雷壇工費。帝嘉其忠,授紫府宣忠高士。朝用請歲進數萬金以資國用,帝益喜。已而術不驗,其徒王子巖攻發其詐。帝執子巖、朝用,付鎮撫拷訊,朝用所獻銀,故出勛資。事既敗,帝亦浸疏勛。明年,勛亦下獄,朝用乃脅勛賄,捶死其家人,復上疏瀆奏。帝怒,遂論死。

龔可佩,嘉定人。出家崑山為道士,通曉道家神名,由仲文進。諸大臣撰青詞者,時從可佩問道家故事,俱愛之,得為太常博士。帝命入西宮,教宮人習法事,累遷太常少卿。為中官所惡,誣其嗜酒,使使偵之,報可佩醉員外郎邵畯所。執下詔獄,並逮畯,俱杖六十。可佩杖死,屍暴潞河,為群犬所食,畯亦奪官。畯與可佩故無交,無敢白其枉者。

藍道行以扶鸞術得幸,有所問,輒密封遣中官詣壇焚之,所答多不如旨。帝咎中官穢褻,中官懼,交通道行,啟視而後焚,答始稱旨。帝大喜,問:「今天下何以不治?」道行故惡嚴嵩,假乩仙言嵩奸罪。帝問:「果爾,上仙何不殛之?」答曰:「留待皇帝自殛。」帝心動,會御史鄒應龍劾嵩疏上,帝即放嵩還。已,嵩詗知道行所為,厚賂帝左右,發其怙寵招權諸不法事。下詔獄,坐斬,死獄中。

胡大順者,仲文同縣人也。緣仲文進,供事靈濟宮。仲文死,大順以奸欺事發,斥回籍。後覬復用,偽撰萬壽金書一帙,詭稱呂祖所作,且言呂祖授三元大丹,可卻疾不老。遣其子元玉從妖人何廷玉齎入京,因左演法藍田玉、左正一羅萬象以通內官趙楹,獻之帝。

田玉者,鐵柱觀道士。嚴嵩罷歸,至南昌,值聖誕,田玉為帝建醮。會御史姜儆訪秘法至,嵩索田玉諸符籙進獻。田玉亦自以召鶴術託儆附奏,得召為演法,與萬象並以扶鸞術供奉西內,因交觀楹。時帝方幸此三人,故大順書由三人進。帝覽書問:「既云乩書,扶乩者何不來?」田玉遂詐為聖諭徵之,至則屢上書求見。帝語徐階曰:「自藍道行下獄,遂百孽擾宮。今大順來,可復用乎?」對曰:「扶乩之術,惟中外交通,間有驗者,否則茫然不知。今宮孽已久,似非道行所致。且用此輩,孽未必消。小人無賴,宜治以法。」帝悟,報曰:「田玉無狀,去冬代廷玉進水銀藥,遂詐傳密旨,徵取大順,不治無以儆將來。」階對:「水銀不可服食,詐傳詔旨罪尤重。倘置不問,群小互相朋結,恐釀大患。」乃命執大順、田玉、萬象等下錦衣獄,不知其奸由楹也。錦衣上獄詞,帝有意寬之,以問階。階力言不可不重治,乃下諸人法司,令重擬。楹伺間,具密奏,為諸人申理。帝大怒,付司禮拷訊,具得其交通狀,遂與大順、田玉、萬象、廷玉、元玉並論死。楹瘐死。帝以逆囚當顯戮,怒所司不如法,詔停刑部司官俸。嘉靖四十四年也。

世宗朝,奏章有前朝、後朝之說。前朝所奏者,諸司章奏也;他方士雜流有所陳靖,則從後朝入,前朝官不與聞,故無人摘發。賴帝晚年漸悟其妄,而政府力為執奏,諸奸獲正法云。

王金者,鄠縣人也。為國子生,殺人當死。知縣陰應麟雅好黃白術,聞金有秘方,為之解,得末減。金遂逃京師,匿通政使趙文華所。以仙酒獻文華,文華獻之帝。及文華視師江南,金落魄無所遇。一日,帝於秘殿扶乩,言服芝可延年,使使採芝天下。四方來獻者,皆積苑中;中使竊出市人,復進之以邀賞。金厚結中使,得芝萬本,聚為一山,號萬歲芝山,又偽為五色龜,欲因禮部以獻,尚書吳山不為進。山罷,金自進之。帝大喜,遣官告太廟禮官袁煒率廷臣表賀,而授金太醫院御醫。

先是,總督胡宗憲獻白鹿者再。帝喜,告謝玄極寶殿及太廟,進宗憲秩,百官表賀。已,宗憲獻靈芝五、白龜二。帝益喜,賜金幣、鶴衣,告廟表賀如初。不數日,龜死,帝曰:「天降靈物,朕固疑處塵寰不久也。」淮王獻白雁二,帝曰:「天降祥羽,其告廟。」嚴嵩孫鵠獻玉兔一、靈芝六十四,藍道行獻瑞龜。俱遣中官獻太廟,廷臣表賀。未幾,兔生二子,禮官請謝玄告廟。是月,兔又生二子,帝以為延生之祥,特建謝典告廟。已又生數子,皆稱賀。其他西苑嘉禾,顯陵甘露,無不告廟稱賀者。當是時,陶仲文已死,嚴嵩亦罷政,藍道行又以詐偽誅,宮中數見妖孽,帝春秋高,意邑邑不樂,中官因詐飾以娛之。四十三年五月,帝夜坐庭中,獲一桃御幄後,左右言自空中下。帝大喜曰:「天賜也。」修迎恩醮五日。明日復降一桃,其夜白兔生二子。帝益喜,謝玄告廟。未幾,壽鹿亦生二子,廷臣表賀。帝以奇祥三錫,天眷非常,手詔褒答。

時遣官求方士於四方,至者日眾。豐城人熊顯進仙書六十六冊,方士趙添壽進秘法三十二種,醫士申世文亦進三種。帝知其多妄,無殊錫。金思所以動帝,乃與世文及陶世恩、陶仿、劉文彬、高守中偽造《諸品仙方》、《養老新書》、《七元天禽護國兵策》,與所製金石藥並進。其方詭秘不可辨,性燥,非服食所宜。帝御之,稍稍火發能愈。世恩竟得遷太常卿,仿太醫院使,文彬太常博士。未幾,帝大漸,遺詔歸罪金等,命悉正典刑,五人並論死繫獄。隆慶四年十月,高拱柄國,盡反徐階之政,乃宥金等死,編口外為民。

顧可學,無錫人。舉進士,歷官浙江參議。言官劾其在部時盜官帑,斥歸,家居二十餘年。瞷世宗好長生,而同年生嚴嵩方柄國,乃厚賄嵩,自言能煉童男女溲為秋石,服之延年。嵩為言於帝,遣使齎金幣就其家賜之。可學詣闕謝,遂命為右通政。嘉靖二十四年超拜工部尚書,尋改禮部,再加至太子太保。時盛端明亦以方術承帝眷,可學獨揚揚自喜,請屬公事,人咸畏而惡之。帝惑乩仙言,手詔問禮部:「古用芝入藥,今產何所?」尚書吳山博引《本草》、《黃帝內經》、《漢舊儀》、王充《論衡》、《瑞命記》,言:「歷代皆以芝為瑞,然服食之法未有傳,所產地亦未敢預擬。」乃詔有司採之五嶽及太和、龍虎、三茅、齊雲、鶴鳴諸山。無何,宛平民獻芝五本。帝悅,賚銀幣。自是,來獻者接踵。時又採銀礦、龍涎香,中使四出,論者咸咎可學。可學尋以年老乞休。卒,賜祭葬,謚榮僖。

端明,饒平人。舉進士,歷官右副都御史,督南京糧儲,劾罷,家居十年。自言通曉藥石,服之可長生,由陶仲文以進,嚴嵩亦左右之,遂召為禮部右侍郎。尋拜工部尚書,改禮部,加太子少保,皆與可學並命。二人但食祿不治事,供奉藥物而已。端明頗負才名,晚由他途進,士論恥之。端明內不自安,引去,卒於家。賜祭葬,謚榮簡。隆慶初,二人皆褫官奪謚。

朱隆禧者,崑山人。由進士歷順天府丞,坐大計黜。二十七年,陶仲文赴太和山,隆禧邀至其家,以所傳長生秘術及所製香衲祈代進。仲文還朝,奏之。帝悅,即其家賜白金、飛魚服。隆禧入朝謝恩,帝以大計罷閒官例不復起,加太常卿致仕。居二年,加禮部右侍郎。會有邊警,仲文乘閒薦隆禧知兵。帝曰:「祖宗法不可廢。」卒不用。既卒,其妻請恤典,所司執不予,帝特諭予之。隆慶初,褫官。

帝晚年求方術益急,仲文、可學輩皆前死。四十一年冬,命御史姜儆、王大任分行天下,訪求方士及符籙秘書。儆,江南、山東、浙江、江西、福建、廣東、廣西;大任,畿輔、河南、湖廣、四川、山西、陜西、雲南、貴州。至四十三年十月還朝,上所得法秘數千冊,方士唐秩、劉文彬等數人。儆、大任擢侍講學士,秩等賜第京師。儆不自安,尋引退。大任入翰林,不為同官所齒。隆慶元年正月,言官劾兩人所進劉文彬等已正刑章,宜並罪,遂奪職。


\end{pinyinscope}