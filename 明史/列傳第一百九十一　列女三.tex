\article{列傳第一百九十一 列女三}

\begin{pinyinscope}
○徐貞女劉氏餘氏虞鳳娘林貞女王貞女倪美玉劉烈女上海某氏谷氏白氏高烈婦于氏臺氏胡氏王氏劉孝女崔氏高陵李氏烈婦柴氏周氏王氏荊媧宋氏李氏陳氏蘄水李氏婢阿來萬氏王氏五烈婦明倫堂女陳氏澤二李氏姜氏六安女石氏女謝氏莊氏馮氏唐烈妻陳氏劉氏唐氏顏氏盧氏于氏蕭氏楊氏仲氏女何氏趙氏倪氏王氏韓氏邵氏李氏江氏楊氏張氏石氏王氏等郭氏姚氏朱氏徐氏女定州李氏胡敬妻姚氏熊氏丘氏乾氏黃氏洗馬畈婦向氏雷氏商州邵氏呂氏曲周邵氏王氏吳之瑞妻張氏韓鼎允妻劉氏江都程氏六烈江都張氏蘭氏等張秉純妻劉氏陶氏田氏和州王氏方氏陸氏子道弘妻于氏項淑美王氏甬上四烈婦夏氏

徐貞女,宣城人。少字施之濟。年十五,里豪湯一泰艷之,倚從子祭酒賓尹,強委禽焉。女父子仁不受,夜趣施舁女歸。一泰恚甚,脅有司攝施婦,欲庭奪以歸,先使人捽之濟父子及媒妁數人,毆之府門,有司莫能制。徐氏被攝,候理,次城東旅舍,思不免,夜伺人靜,投池中死,衣上下縫紉不見寸體。觀者皆泣下,共舁古廟,盛夏鬱蒸,蠅不敢近。郡守張德明臨視,立祠城東祀之。

劉氏,京師人。有松江人戍邊者,詐稱無妻,娶劉。既而遇赦歸,紿劉曰:「吾暫歸省。」久之不復至,劉抵松訪之,夫拒不納。劉哭曰:「良人棄我,我將安歸。」乃翦髮為尼,行乞市上,人多憐而周之。劉置一棺,夜臥棺中數十年。鄰火起,劉入棺,呼曰:「乞與闔棺,以畢吾事。」遂焚死。

餘氏,黃岡宋蒙妾。蒙妻劉,舉子女各一人,餘無所出。及蒙卒,劉他適,妾辛勤育之。日事紡績,非丙夜不休。壺政嚴肅,親屬莫敢窺其門。踰二十年,忽謂子女曰:「吾命將盡,不能終視若輩,惟望若輩為上流人爾。」越數日,無疾而逝。

虞鳳娘,義烏人。其姊嫁徐明輝而卒,明輝聞鳳娘賢,懇其父欲聘為繼室。女知,泣謂父母曰:「兄弟未嘗同妻,即姊妹可知。」父執不聽,女絕口不言,自經死。

林貞女,侯官人。父舜道,官參政。女幼許長樂副都御史陳省子長源,既納幣,長源卒。女蓬首削脂澤,稱疾臥床,哭無聲而神傷。或謂未成婦,何自苦。答曰:「子名氏、歲月飾而櫝之以歸陳,忍自昧哉!」固請於父,欲赴陳喪,父為達其意。陳父答曰:「以凶歸,所不忍,以好歸,疇與主之?姑俟喪除。」女大悲吒曰:「是欲緩之,覬奪吾志也。」遂不食,積七日,嘔血死。

王貞女,崑山人,太僕卿宇之孫,諸生述之女,字侍郎顧章志孫同吉。未幾,同吉卒。女即去飾,白衣至父母前,不言亦不泣,若促駕行者。父母有難色,使嫗告其舅姑,舅姑掃庭內待之。女既至,拜柩而不哭,斂容見舅姑,有終焉之意。姑含淚曰:「兒不幸早亡,奈何累新婦。」女聞姑稱新婦,淚簌簌下,遂留執婦道不去。早晚跪奠柩前,視姑眠食外,輒自屏一室,雖至戚遣女奴候視,皆謝絕,曰:「吾義不見門以外人。」後姑病,女服勤,晝夜不懈。及病劇,女人候床前,出視藥灶,往來再三,若有所為。群婢窺之而莫得其跡,姑既進藥則睡,覺而病立間,呼女曰:「向飲我者何藥?乃速愈如是。」欲執其手勞之,女縮手有難進之狀。姑怪起視,已斷一指煮藥中矣。姑歎曰:「吾以天奪吾子,常憂老無所倚。今婦不惜支體以療吾疾,豈不勝有子耶!」流涕久之。人皆稱貞孝女云。

倪美玉,年十八歸董緒。緒居喪過毀得疾,謂妻曰:「吾無兄弟,又無子。吾死,父母祀絕矣。當以吾屋為小宗祠,置祀田數畝,小宗人遞主之,春秋享祀,吾父母獲與焉,吾無憾矣。汝必以此意告我叔父而行之。」緒卒,倪立從子為後。治喪畢,攜其女及田二十畝囑其姒曰:「以此累姆。」及夫叔父自外郡至,泣拜致夫命,叔父如其言。事竣,婦出拜謝,即入室臥不食。居數日,沐浴整衣曰:「亡夫召我矣。」舉手別父母親屬而逝,年二十二。

劉烈女,錢塘人。少字吳嘉諫。鄰富兒張阿官屢窺之,一夕緣梯入。女呼父母共執之,將訟官。張之從子倡言劉女誨淫,縛人取財。人多信之。女呼告父曰:「賊污我名,不可活矣,我當訴帝求直耳。」即自縊。盛暑待驗,暴日下無尸氣。嘉諫初惑人言,不哭。徐察之,知其誣也,伏屍大慟。女目忽開,流血淚數行,若對泣者。張延訟師丁二執前說,女傅魂於二曰:「若以筆污我,我先殺汝。」二立死。時江濤震吼,岸土裂崩數十丈,人以為女冤所致。有司遂杖殺阿官及從子。

上海某氏,既嫁,夫患瘋癩,舅姑謀奪以妻少子。婦覺,密告其夫,夫泣遣之歸寧。婦潛製殮具,夫既死,舅姑不以告,不闔棺,露置水濱,以俗忌惡疾也。婦聞,盂飯淪雞,偕幼妹至棺所,抱屍浴之,斂以衣衾,闔棺設祭。祭畢,與妹訣,以巾幕面,投水死。

谷氏,餘姚史茂妻。父以茂有文學,贅之於家。數日,鄰人宋思徵責於父,見氏美,遂指逋錢為聘物,訟之官。知縣馬從龍察其誣,杖遣之。及谷下階,茂將扶以行。谷故未嘗出閨閣,見隸人林立,而夫以身近己,慚發赬,推茂遠之。從龍望見,以谷意不屬茂也,立改判歸思。思即率眾擁輿中而去,谷母隨之至思舍。谷呼號求速死,斷髮屬母遺茂。思族婦十餘人,環相勸尉,不可解,乘間縊死。從龍聞之大驚,捕思,思亡去。茂感妻義,終身不娶。

白氏,清澗惠道昌妻。年十八,夫亡。懷娠六月,欲以死殉。眾諭之曰:「胡不少待,舉子以延夫嗣。」氏泣曰:「非不念良人無後,但心痛不能須臾緩耳。」七日不食而死。

高烈婦,博平諸生賈垓妻。垓卒,氏自計曰:「死節易,守節難。況當兵亂之際,吾寧為其易者。」執姑手泣曰:「婦不能奉事舅姑,反遺孤孫為累。然婦殉夫為得正,勿過痛也。」遂縊。

于氏,潁州鄧任妻。任病,家貧,藥餌不給,氏罄嫁笥救之。閱六月病革,氏聘簪二,綰一於夫髮,自綰其一,撫任頸哽咽曰:「妾必不負君。」納指任口中,令齧為信。任歿三日,縊死。

州又有臺氏,諸生張雲鵬妻。夫病,氏單衣蔬食,禱天願代,割臂為糜以進。夫病危,許以身殉,訂期三日。夫付紅帨為訣,氏號泣受之。越三日,結所授帨就縊,侍婢救不死,恨曰:「何物奴,敗我事!令我負三日約。」自是,水漿不入口,舉聲一號,熱血迸流。至七日,頓足曰:「遲矣,郎得毋疑我。」母偶出櫛沐,扃戶縊死。

胡氏,諸城人,遂平知縣麗明孫女也。年十七,歸諸生李敬中,生一女而夫卒。初哭踴甚哀,比三日不哭,盥櫛拜舅姑堂下,家人怪之,從容答曰:「婦不幸失所天,無子,將從死者地下,不得復事舅姑,幸強飯自愛。他日叔有子,為亡人立嗣,歲時奠麥飯足矣。」姑及其母泣止之,不可,乃焚香告柩前,顧家人曰:「洗含汝等親之,不可近男子。」遂入戶自經,母與姑槌門痛哭疾呼,終不顧而死。

王氏,淄川成象妻。夫死,痛哭三日,脣焦齒黑。父不忍,予之水,謝勿飲。又三日,氣息漸微,強起語父曰:「翁姑未葬,夫亦露殯,奈何?」父許任其事,氏就枕叩頭而瞑,年十七。

劉孝女,京師人。父蘭卒,矢志不嫁,以養其母。崇禎元年,年四十六矣,母病歿,女遂絕粒殉之。

崔氏,香河王錫田妻。崇禎二年,城破,氏與眾訣曰:「我義不受辱。」涕泣乳其女,將自縊,家人力持不得遂。兵及門,眾俱奔,氏倉皇縊於戶後,恐賊見其貌,或解之也。

高陵李氏,鎮撫劉光燦妻。夫歿,勵志苦守。崇禎四年,賊陷高陵。年七十九,其家掖之走,曰:「未亡人棄先夫室何往?」語未已,賊露刃入。即取刀自刺,流血淋漓。賊壯其烈,與飲食,怒不受,以碗擊賊,罵曰:「吾忍死四十九年,今啜賊食耶!」遂遇害。

烈婦柴氏,夏縣孫貞妻。崇禎四年,夫婦避賊山中。賊搜山,見氏悅之,執其手。氏以口齧肉棄之曰:「賊污吾手。」繼扳其肱,又以口齧肉棄之曰:「賊污吾肱。」賊舍之去,氏罵不絕聲,還殺之。

周氏,新城王永命妻,登州都督遇吉兄女也。幼通《孝經》、《列女傳》。崇禎五年,叛將耿仲明、李九成等據登州反,縱兵淫掠。一小校將辱之,氏紿之去,即投繯死。明日,賊至,怒其誑己,支解之。事平,永命偵賊所在,擊斬之,以其首祭墓。時蓬萊浦延禧妻王氏,年二十,守節撫孤。九成叛,城陷,叔允章至其家,問所向。答曰:「兒豈向患難中求活。」時有麻索在床頭,叔以手振之曰:「欲決計於此乎?」氏首肯,從容就縊。

荊媧,陜西淳化人,姓高氏。兄起鳳,邑諸生。崇禎五年,流賊掠繼母秦氏及荊媧去,起鳳馳赴賊營請贖。賊索二馬,起鳳傾貲得一馬,予之。賊止還其母。起鳳與妹訣曰:「我去,汝即死。」賊令勸妹從己,且欲留為書記。起鳳大罵不從,被殺。百計脅荊媧,大罵求死。賊悅其色,割髮裂衣以恐之。媧益罵不已,賊乃殺之,年甫十六。巡按吳甡上其事,兄妹皆旌。

陳丹余妻宋氏。丹餘為鄖陽諸生。崇禎六年,賊至被掠,並執其女,迫令入空室。前有古槐,母女抱樹立,罵曰:「吾母子死白日下,豈受污暗室中。」大罵不行。賊斷其手,益大罵,俱被害。

黃日芳妾李氏、陳氏。日芳知霍丘縣,崇禎八年,齎計簿入郡。流賊突至,圍城。二人相謂曰:「主君未還,城必不守,我兩人獨有一死耳。」密縫內外衣甚固,城陷,南望再拜,攜赴藏天澗死。越三日,日芳至,號哭澗側。兩屍應聲浮出,顏色如生,手尚相援。

蘄水李氏,諸生何之旦妻。流賊至蘄,執而逼之去,不從,則眾挾之。李罵益厲,齧賊求死。賊怒,刺之,創遍體,未嘗有懼色,賊斷其頸死。從婢阿來抱李幼女,守哭。賊奪女將殺之,不與,伏地以身庇之。刺數十創,婢、女俱死。

萬氏,和州儒士姚守中妻,泉州知府慶女孫也。生六子,皆有室。崇禎八年,流賊陷其城,慟哭孀姑前,命諸婦曰:「我等女子也,誓必死節。」諸子環泣,急麾之曰:「汝輩男子,當圖存宗祀,何泣焉?」長子承舜泣曰:「兒讀書,惟識忠孝字耳,願為厲鬼殺賊,何忍母獨死。」遂負母投於塘。諸婦女孫相隨死者十數人,僅存子希舜,求其屍,其聚塘坳,無一相離者。

流賊陷和州,王氏一時五烈婦;王用賓妻尹氏,用賢妻杜氏,用聘妻魯氏,用極妻戴氏,又王氏良器女,劉臺妻也。五人同匿城西別墅,誓偕死。及賊登陴,呼聲震地。五人相持泣曰:「亟死亟死,毋污賊刃。」結繯,繯斷,適用賢所佩劍掛壁上,杜趨拔之,爭磨以剄,次第死。州又有女,失其姓,與諸婦共匿明倫堂後。其四人已為賊執,用帛牽之。獨此女不肯就執,多方迫之不得。四婦勸之,泣曰:「我處女也,可同男子去耶?」以頭搶地。賊搴其足而曳之,女大罵。賊怒,一手搴足,以刀從下劈之,體裂為四。

陳氏,涇陽王生妻。有子方晬,生疾將死,以遺孩屬陳。陳曰:「吾當生死以之。」流賊至,陳抱子避樓上。賊燒樓,陳從樓簷跳下,不死。賊視其色麗,挾之馬上,陳躍身墜地者再。最後以索縛之,行數里,陳力斷所繫索,並鞍墜焉。賊知不可奪,乃殺之。賊退,家人收其屍,子呱呱懷中,兩手猶堅抱如故。

雞澤二李氏。一同邑田蘊璽妻。遇亂,蘊璽兄弟被殺。李抱女同姒王抱男而逃。王足創難行,令李速去。李曰:「良人兄弟俱死,當存此子以留田氏後。」遂棄己女,抱其子赴城,得無恙。一嫁曲周郭某。遭亂,舉家走匿。翁姑旋被殺,李攜幼男及夫弟方七歲者共逃,力罷,不能俱全。或教之舍叔而抱男,李曰:「翁姑死矣,叔豈再得乎!子雖難捨,然吾夫在外,或未死,尚可期也。」竟棄男,負叔而走。

宋德成妻姜氏,臨清人。德成知贊皇縣,寇入署,姜投井。賊出之,逼令食,罵曰:「待官兵剿汝,醢為脯,吾當食之。」以簪自剔一目示賊曰:「吾廢人也,速殺為幸。」賊怒殺之。

六安女,失其姓。崇禎中,流賊入境,見其美,將犯之。以帕蒙其頭,輒壞之,曰:「毋污我髮。」被以錦衣,又擲之曰:「毋污吾身。」強擁諸馬上,復投地大罵請死。賊怒刃之,既而歎曰:「真烈女。」

石氏女,失其邑里,隨父守仁寓五河。崇禎十年,流賊突至,執欲污之。女抱槐樹厲聲罵賊。賊使數人牽之不解,剒其兩手,罵如初。又斷其足,愈罵不絕,痛仆地佯死。賊就褫其衣,女以口齧賊指,斷其三,含血升許噴賊,乃瞑。賊擁薪焚之,厥後所焚地,血痕耿耿,遇雨則燥,暘則濕。村人駭異,掘去之,色亦入土三尺許。

又當塗舉人吳昌祚妻謝氏,為亂卒所掠。謝以手抱樹,大罵不止。卒怒,斷其附樹之指,復拾斷指擲卒面,卒磔殺之。

周彥敬妻莊氏。彥敬,棲霞知縣。氏讀書知大義,亂起,鄉人悉竄山穴中。莊以男女無別,有難色。彥敬強之曰:「不入,且見殺。」莊曰:「無禮不如死,君疑我難死乎!」既引刀自裁。彥敬感其義,終身不復娶。

梁凝禧妻馮氏。凝禧,隨州諸生。崇禎十年,聞賊警,夫婦買舟避難。行至西河,賊追急,登岸奔魏家砦。夫婦要同死,氏訣凝禧曰:「同死固甘,但君尚無子,老母在堂,幸速逃,明早可於此地尋我。」凝禧遂逃,次早果得屍於分手處。

唐烈妻陳氏。烈,孝感諸生。崇禎十年,從夫避難山砦。賊突至,夫與子俱奔散,陳獨行山谷間。砦人曰:「非唐氏嫗乎?事迫矣,可急入保。」陳問夫與子至未,曰:「未也。」陳泣曰:「我煢煢一婦人,靡因而至。諸君雖憐而生我,我何面目安茲土耶!夫存亡未知,依人以生不貞,棄夫之難不義。失貞與義,何以為人!吾其行也。」卒不入。已,賊至,逼去不從,大罵死。

又劉氏,懷寧人,應天府丞顏素之孫婦也。崇禎末,亂兵焚掠江市。其舅與夫先在南京。劉孑身出避,倉皇無所之,見男婦雜走登舟,慨然曰:「吾儕婦人,保姆不在,義不出帷,敢亂群乎!」遂投江死。

唐氏,廣濟潘龍躍妻。崇禎十三年避賊靈果山。賊至,加刃龍躍頸,索錢。唐跪泣,乞以身代夫,不許。女巽跪泣,乞以身代父,不許。唐知夫不免,投於塘,女從之。賊愴然釋其夫。

又顏氏,長樂諸生黃應運妻。城陷,兵至其家,欲殺應運生母詹氏,顏泣訴,願身代。及顏方受刃,妾曾又奔號曰:「此我主母,無所出,願殺我以全其命。」卒感其義,兩釋之。

潁州盧氏,王瀚妻。家貧,舂織終歲。崇禎十四年大饑,夫患疫。氏語夫曰:「君死,我當從。」及夫死,時溽暑,氏求親戚斂錢以葬曰:「我當死,但酷熱無衣棺,恐更為親戚累,遲之秋爽耳。」聞者咳之。及秋,盡糶其新穀,置粗布衣,餘買酒蔬祀夫墓。歸至家,市梨數十進姑,並貽妯娌,語人曰:「我可死矣。」夜半自縊。

于氏,汝州張鐸妻。崇禎十四年,賊破城,氏謂兩婢曰:「吾輩今日必死,曷若先出擊賊,殺賊而斃,不失為義烈鬼。」於是執梃而前,賊先入者三,出不意,悉為所踣。群賊怒,攢刺之,皆死。

蕭氏,萬安賴南叔妻。夫早喪,無子,遺一女。寇大起,築室與女共居。盜突至,率女持利刃遮門,詈曰:「昔寧化曾氏婦,立砦殺賊。汝謂我刃不利邪!犯我必殺汝。」賊怒,縱火焚之,二人咸燼。

又楊氏,安定舉人張國紘妾。崇禎十六年,賊賀錦攻城急。國紘與守者議,丁壯登陴,女子運石。楊先倡,城中女子從之,須臾四城皆遍。及城陷,楊死譙樓旁。事定,家人獲其屍,兩手猶抱石不脫。

仲氏女,湖州人,隨父賈漢陽。崇禎中,漢陽陷,從群婦將出城,賊守門者止之。有頃,賊大肆淫掠,見女美,執之。女■面披髮,大罵。賊具馬,命二賊挾之上,連墜傷額,終不肯往。賊露刃迫之曰:「身往何如頭往?」笑曰:「頭往善。」遂被害。

鄺抱義妻何氏。抱義,臨武諸生。崇禎末,氏為賊所執,乃垢面蓬髮紿以病疫,賊懼釋之。及賊退,家人咸喜,何泣曰:「平昔謁拜伯叔,猶赭顏汗發。今匿身不固,以面目對賊,牽臂引裾,雖免污辱,何以為人!」竟忿恚不食死。

湯祖契妻趙氏。祖契,睢州諸生。氏知書,有志節。崇禎十五年,賊陷太康,將抵睢。氏語家人曰:「州為兵衝,未易保也。脫變起,有死耳。」及城破,屬祖契負其母以逃,而己闔戶自經,家人解之,投井,復為家人所阻,怒曰:「賊至不死,非節也,死不以時,非義也。」賊至,環刃相向,牽之出,厲聲訶賊,遂遇害。

蕭來鳳妻倪氏。來鳳,商城貢生,慷慨有大節。賊逼受職,不屈死,倪自經從之。又有宋愈亨,深澤舉人,寇至投井死。妻王氏曰:「夫既如此,吾敢相負。」媳韓生男甫六日,願從死,相對縊。

邵氏,鄒縣張一桂妻,同妾李氏遇賊。欲迫李行,邵罵曰:「亡夫以妾託我,豈令受賊辱。」賊怒殺之。李知不免,紿曰:「我有簪珥埋後園井旁。」賊隨李發之,至則曰:「主母為我死,我豈獨生。」即投井。賊下井扶之,李披髮破面罵不已,扭其衣欲令併死井底,叫聲若雷。賊知不可強,乃刃之。

宗胤芳妻江氏,魯山人。子麟祥,進士。流賊之亂,江與麟祥妻袁氏率孫女、孫婦九人登樓,俱懸於梁。視其已死,乃引刀自剄。

曹復彬妻楊氏。復彬,江都諸生。城破,復彬創仆地,楊匿破屋中。長女蒨文,年十四,趣母決計。次女蒨紅,年十二,請更衣死。楊止之,復彬執不可,乃為三繯,次第而縊。

梁以樟妻張氏,大興人。以樟知商丘縣。崇禎十五年,流賊圍商丘,急積薪樓下,集婢女其上,俱令就縊。謂子燮曰:「汝父城守,命不可知,宗祀惟汝是賴。」屬乳媼匿民家。自縊死。家人舉火,諸屍俱燼。

鄭完我母石氏,甘州衛人。完我,南陽府同知,既之官,妻王氏奉石家居。崇禎十六年,賊圍甘州,石預戒家人積薪室中。及城陷,攜王及一孫女縱火自焚。寇退,出屍灰燼間,姑媳牽挽不釋手。女距三尺許,覆以甕,啟視色如生。

郭氏,長治宋體道妻。崇禎十五年,任國琦作亂,同居諸婦皆羅跪,呼郭不出,獨匿垝垣。賊怒,詰其不跪,瞪目厲聲曰:「我跪亦死,不跪亦死,已安排不活矣。」賊加數刃,迄死罵不絕口。

姚氏,桐城人,湘潭知縣之騏女,諸生吳道震妻。年十九,夫亡,以子德堅在襁褓,忍死撫之。越二十六年,至崇禎末,流賊掠桐城。兄孫林奉母避潛山,氏偕行。賊奄至,孫林格鬥死,德堅負氏逃。氏曰:「事急矣,汝書生焉能負我遠行,倘賊追及,即俱死,汝不能全母,顧反絕父祀乎!」叱之去,德堅泣弗忍,氏推之墜層崖下。須臾賊至,叱曰:「出金可免。」氏曰:「我流離遠道,安得有金。」賊令解衣驗之,罵曰:「何物賊奴,敢作此語!」賊怒,刃交下死。

硃氏,無為人,徐畢璋妻。年十七,歸璋。璋有妹名京,年十五,未字。崇禎十五年,流賊破城。朱方懷孕,奔井邊,謂京曰:「吾妊在懷,井口狹,可推而納之。」京曰:「唯。」納畢,即哭呼曰:「父母安在乎,吾伴嫂死矣!」躍而入。

李氏,定州人,廣平教授元薦女,歸同里郝生。崇禎十六年,州被兵。生將奉親避山中,留李與二子居其母家。生控馬將發,李哭拜馬前,指庭中井訣曰:「若有變,即潔身此中,以衣袂為識,旁有白線一行者,即我也。」比城破,藏二子他所,入井死。兵退,生出其屍,顏色如生。

胡敬妻姚氏。敬,孝感貢生。流賊陷孝感,姚乘舟避難南湖,欷歔不已。鄰舟婦解之曰:「賊入黃,從未殺人,何畏也?」姚曰:「我非畏殺,畏其不殺耳。」聞賊將入湖,歎曰:「賊至而死,辱矣。」遂攜二女僮投水死。

熊氏,武昌李藎臣妻,大名知縣正南女。藎臣父周華,官贛州知官,藎臣從父之任,留婦於家。崇禎十六年,武昌陷,婦匿林藪中,為賊所得,奪刀自刎。賊去,鄰嫗救活之。明年,李自成率殘卒南奔,婦隻身竄山谷。有胡姓者,欲為子娶之。婦曰:「吾頸可斷,汝不聞前事乎!」已,藎臣自江西歸,遇賊被殺。婦慟三日,自縊死。

丘氏,孝感劉應景妻。崇禎末,為賊所執,逼從,不可。賊曰:「刃汝。」丘曰:「得死為幸。」賊注油滿甕,漬其衣,語同類曰:「此婦倔彊,將巘之。」丘哂曰:「若謂死溺、死焚、死刃有間乎?官兵旦夕至,若求如我,得哉!」賊怒,束於木焚之,火熾,罵不絕口。同邑乾氏,年十七,歸高文煥。文煥卒,無子,拔刀自裁。母及姑救之,越三日復蘇。自是斷葷,日不再食。崇禎十六年,聞賊陷德安,將及孝感。從子高騫將扶避山砦,氏曰:「吾老矣,豈復出門求活。行吾四十年前之志,可也。」投後園池中死。

邑又有黃氏,張挺然妻。崇禎末,賊帥白旺陷德安,授挺然偽掌旅。黃泣止之,不聽。賊令挺然取婦為質,黃攜十歲兒匿青山砦。挺然誘以利,劫以兵,且使親戚招之,皆不應。已而破砦,焚己居以窮黃,黃匿愈深,竟不可得。挺然寄兒金簪,兒以綰髮,黃怒,拔棄之曰:「何為以賊物污首!」久之,賊敗,挺然走死襄陽,黃耕織以撫其子,鄉人義之。

蘄水洗馬畈某氏,為賊所執,不從。賊刃其腹,一手抱嬰兒,一手捧腹,使氣不即盡以待夫。夫至,付兒,放手而斃。

向氏,黃陂人。年十八,歸王旦士。未久,賊陷黃陂,被執。賊持刀迫之,氏罵不絕口。賊指眾曰:「若非汝父母,即舅姑兄弟,必盡殺,而後及汝。」氏曰:「我義不辱,與家人何與!」奪刃自刎。賊怒,立磔之。

劉長庚妾雷氏。長庚為同州諸生。賊陷潼關,將及州,長庚拜家廟,召妻及二子曰:「汝年長,且有子,當逃。」召雷及所生女曰:「汝年少,當從吾死。」雷曰:「妾志也。」長庚攜酒登樓,謂妾曰:「汝平日不飲,今當共醉。」妾欣然引滿。長庚且飲且歌,夜半遍題四壁,拔刀示妾曰:「可以行乎?」對曰:「請先之。」奪刀自刎。長庚乃解所繫條,縊於梁。女方七歲,橫刀於壁,以頸就之而死。

邵氏,商州人,布政使可立女,侍郎雒南薛國用子匡倫妻也。流賊將至,避之母家。商州陷,賊驅使執爨,罵曰:「吾大家女,嫁大臣子,肯為狗賊作飯耶!」賊怒,斫其足,罵益厲,斷舌寸磔之。

關陳諫妻呂氏。陳諫,雲夢諸生。族有安氏者,殉其夫關坤,呂每談及,輒感慨欷歔曰:「婦人義當如是。」崇禎末,寇陷鄰郡,呂謂夫曰:「賊焰方張,不如早為之所。」取魚網結其體甚固。俄寇至,俾縫衣,呂投剪破賊面,罵曰:「賊敢辱我鍼黹乎!手可斷,衣不可縫。」賊怒,磔之,投於水。

邵氏,曲周李純盛妻。寇至,姑姊妹俱避地洞中。邵為寇所得,問洞所在。紿之行,寇喜隨之,徑往井傍,投井死。洞中五十餘人俱獲免。

王氏,宛平劉應龍妻。年十六,嫁應龍。家貧,以女紅養舅姑。應龍父子相繼亡,王事姑撫子。閱二十年,賊陷都城,泣拜其姑曰:「留長孫奉事祖母,婦死已決。」遂攜幼子投井死。

吳之瑞妻張氏。之瑞,宿松諸生。福王時,城陷,軍士欲污之。張恐禍及夫與子,紿曰:「此吾家塾師,攜其子在此。吾醜之,若遣去,則惟命。」夫與二子去已遠,張乃厲聲唾罵,撞石死。

韓鼎允妻劉氏。鼎允為懷寧諸生。福王時,城潰。舅姑雙柩殯於堂,劉守不去。賊欲剖棺,劉抱棺號哭,賊釋之。一女年十三,賊欲縱火,而數盼其女。劉紿之曰:「茍不驚先柩,女非所惜也。」賊喜投炬,攜女去。劉送女,目門外池示之,女即投池死。賊怒,刃劉,劉罵不絕口死。

江都程氏六烈。程煜節者,江都諸生也。其祖姑有適林者,其姑有適李者,其叔母曰劉氏、鄒氏、胡氏。而煜節之妹曰程娥,未字。城被圍,與劉約俱死,各以大帶置袖中。城破,女理髮更衣,再拜別其母,遂縊死。劉有女甫一歲,啼甚慘。劉乳之,復以糕餌一器置女側,乃死。鄒與胡亦同死。適林者,投井死。適李者,遭掠,紿卒至井旁,大罵投井死。時稱一門六烈。

張氏,江都史著馨妻。年二十六,夫亡。及城陷,撫其子泣曰:「嚮也撫孤為難,今也全節為大。兒其善圖,吾不能顧矣。」遂赴水死。

又蘭氏,孫道升繼妻。其前妻女曰四,蘭所生女曰七,皆嫁古氏。次曰存,孫女曰巽,皆未嫁。其弟道乾、道新並先卒。道乾妻王氏,子天麟妻丁氏,道新妻古氏,其從弟子啟先妻董氏。江都之圍,諸婦女各手一刃一繩自隨。城破,巽先縊死。蘭時五十四,引繩自縊死。王氏、丁氏投舍後汪中死。古氏亦五十四,守節三十年,頭盡白,投井死。有女嫁於吳,生女曰睿,方八歲,適在外家,從死於井。董氏以帶繫門樞,縊死。存病足,力疾投井死。董氏之娣,有祖母曰陳氏,方寄居,與董氏同處,亦自縊死。四與七同縊於床死。

同時有張廷鉉者,妻薛氏,城破自縊死。廷鉉之妹曰五,遇卒鞭撻使從己,大呼曰:「殺即殺,何鞭為!」遂殺死。

張秉純妻劉氏。秉純,和州諸生。家故貧,氏操井臼,處之怡然。國亡,秉純絕粒死。氏一勺水不入口,閱十有六日,肌骨銷鑠,命子扶至柩前祭拜,痛哭而絕。

陶氏,當塗孫士毅妻,守節十年。南都覆,為卒所掠,縛其手介刃於兩指之間,曰:「從我則完,否則裂。」陶曰:「義不以身辱,速盡為惠。」兵不忍殺,稍創其指,血流竟手,曰:「從乎?」曰:「不從。」卒怒,裂其手而下,且剜其胸,寸磔死。陶母奔護,亦被殺。

田氏,儀真李鐵匠妻,姿甚美。高傑步卒掠江上,執犯之,田以死拒。挾馬上,至城南小橋,馬不能渡。田紿卒牽衣行,睹中流急湍,曳二卒赴水,並溺死。

王氏,和州諸生張侶顏妻。南都不守,劉良佐部卒肆掠。氏同母匿朝陽洞,卒攻洞急,氏以子付母曰:「賊勢洶洶,我少婦,即茍免,何面目回夫家。此張氏一線,善撫之。」言訖,挺身跳洞外,洞高數十仞,亂石巉巖若鋒刃,碎身死焉。

方氏,桐城錢秉鐙妻。避寇寓南都。歲禕,饘粥不給,以女紅易米食其夫,己與婢僕雜食糠籺。客過,潔茗治饌,取諸簪珥,與秉鐙遊者,未嘗知其貧也。秉鐙與阮大鋮同里,有隙,避吳中。方挈子女追尋,得之。已而吳中亦亂,方知不免,乃密紉上下服,抱女赴水死。

陸氏,嘉定黃應爵妻。少喪夫,家貧,紡績自給踰三十年。甫歿,嘉定城破。子道弘妻,亡其姓,持二女倉卒欲赴井。長女曰:「若使母先投,必戀念吾二女,不如先之。」乃挽妹亟入,道弘妻繼之,並溺死。

于氏,丹陽荊潹妻。潹父大澈為亂兵所殺。於聞變,知不免,謂潹曰:「請先殺妾。」潹不忍,怒曰:「君不自殺,欲留為亂兵污耶!」潹慟哭從之。

項淑美,淳安人,適方希文。希文好蓄書。杭州不守,大帥方國安潰兵掠江滸,數百里無寧宇。希文避山間,載書以往。會幼子病疹,希文出延醫,淑美與一嫗一婢處。是夕,亂兵突至,縱火肆掠。婢挽淑美衣,欲與俱出,正色叱曰:「出則死於兵,不出死於火,等死耳,死火不辱。」時嫗已先去,見火熾復入,呼曰:「火至,奈何弗出?」淑美不應,急取書霍左右,高與身等,坐其中。須臾火迫,書盡焚,遂死。賊退,希文歸,則餘燼旋而成堆,若護其骨者。一慟,灰即散,乃收骨瘞先兆。

先是,有慈谿王氏,歸同里方姓。甫逾月,火起,延及其屋。夫適他出,氏堅坐小樓不下,遂被焚,骸骨俱燼,惟心獨存。夫歸,捧之長號,未頃即化。

甬上四烈婦。錢塘張氏,鄞縣舉人楊文瓚妻。國變後,文瓚與兄文琦,友華夏、屠獻宸,俱坐死。張紉箴聯其首,棺殮畢,即盛服題絕命詩,遍拜族戚。吞腦子不死,以佩帶自縊而卒。文琦妻沈氏亦自縊。夏繼妻陸氏結帨於梁,引頸就縊,身肥重,帨絕墮地。時炎暑,流汗沾衣,乃坐而搖扇,謂其人曰:「余且一涼。」既復取帨結之而盡。有司聞楊、華三婦之縊,遣丐婦四人至獻宸家,防其妻朱氏甚嚴。朱不得間,陽為歡笑以接之,且時時誚三婦之徒自苦也。數日,防者稍懈,因謂之曰:「我將一浴,汝儕可暫屏。」丐婦聽之,闔戶自盡。時稱「甬上四烈婦。」

夏氏,黔國公沐天波侍女也。沙定州之亂,天波出走,母陳、妻焦亦避外舍。懼賊迫,焦謂姑曰:「吾輩皆命婦,可陷賊手乎!」舉火自焚死。夏歸其母家,獲免。後天波自永昌還,夏復歸府,則已薙為尼矣。天波感其義,俾佐內政。及天波從亡緬甸,夏遂自經。時城中大亂,死者載道,尸為烏犬所食,血肉狼籍,夏尸棄十餘日,獨無犯者。


\end{pinyinscope}