\article{列傳第一百八十四 孝義}

\begin{pinyinscope}
孝弟之行,雖曰天性,豈不賴有教化哉。自聖賢之道明,誼壁英君莫不汲汲以厚人倫、敦行義為正風俗之首務。旌勸之典,賁於閭閻玄注、三國魏王弼注及唐李鼎祚《周易集解》等多種。,下逮委巷。布衣之氓、匹夫匹婦、兒童稚弱之微賤,行修於閨闥之中,而名顯於朝廷之上。觀其至性所激,感天地,動神明,水不能濡,火不能爇,猛獸不能害,山川不能阻,名留天壤,行卓古今,足以扶樹道教,敦厲末俗,綱常由之不泯,氣化賴以維持。是以君子尚之,王政先焉。至或刑政失平,復仇洩忿,或遭時不造,荒盜流離,誓九死以不回,冒白刃而弗顧。時則有司之辜,民牧之咎,為民上者,當為之惻然動念。故史氏志忠孝義烈之行,如恐弗及,非徒以發側陋之幽光,亦以覘世變,昭法戒焉。

明太祖詔舉孝弟力田之士,又令府州縣正官以禮遣孝廉士至京師。百官聞父母喪,不待報,得去官。割股臥冰,傷生有禁。其後遇國家覃恩海內聯。它在中國殷周之際已產生,其後戰國的荀子、東漢的王,輒以詔書從事。有司上禮部請旌者,歲不乏人,多者十數。激勸之道,綦云備矣。實錄所載,莫可殫述,今採其尤者輯為傳。餘援《唐書》例,臚其姓氏如左。

其事親盡孝,或萬里尋親,或三年廬墓,或聞喪殞命,或負骨還鄉者,洪武時,則有麗水祝昆,上元徐真童、李某女,龍江衛丁歪頭,懷寧曹鏞、鏞妻王氏,徐州王僧兒,廣德姚觀壽,廣武衛陳禮關,桃源張注,江浦張二女勝奴,上海沈德,溧陽史以仁,丹徒唐川,邳州李英,北平東安王重,遵化張拾,保定顧仲禮,樂亭杜仁義妻韓氏,昌平劉驢兒,保定新城王興,祁陽郝安童,山東寧海姜瑜,汶上侯昱,孟縣李德,鞏縣給事中魏敏,登封王中,舞陽周炳,臨桂李文選。而鈞州張宗魯以瞽子有孝行,十七年被旌。

永樂間,則有大興王萬僧奴,東光回滿住,金吾右衛何黑廝,金吾後衛包三雅述明王廷相著。分上、中、下篇。集中闡述元氣自然,武功中衛蔣小保、周阿狗,錦州衛趙興祖,旗手衛周來保,大寧前衛滑中,保安衛徐宗賢,羽林前衛孫志,漢府左護衛千戶許信男斌,江寧浦阿住、沈得安、嚴分保,上元馮添孫、邵佛定,上海沈氏妙蘭,儀真韓福緣,江陰衛徐佛保,府軍衛浦良兒,府軍後衛王保兒、潘醜兒,水軍右衛黃阿回,廣武衛百戶劉玉,蘇州衛張阿童,廣洋衛鄭小奴,大河衛硃阿金,興武衛張彥昇,龍江提舉司匠張貴、胡佛保、聶廣,永新左興兒,濟陽張思名,泰安張翼,肥城趙讓,安邑張普圓,永寧王仕能,陽武劉大,靈寶賀貳,鈞州袁節,膚施陳七兒,鳳翔梁準。

洪熙間,則有江陰越鉉。

宣德間,則有慶都邊靖,南樂康祥、楊鐸,內黃崔克昇,江寧張繼宗,定遠王絅,舒城錢敏,徐州衛張文友,歸德衛任貴,浮梁洪信文,堂邑趙巖,汶上馬威,翼城劉原真,太康順孫陳智,鈞州楊鼐,延安衛指揮王永、安岳、李遇中。

正統間,則有大興劉懷義,元城穀真,邢臺劉鏞,獻縣崔鑒,通州左衛總旗孫雄,昌黎侯顯,新樂孫禮,定興魏整,交河田畯,柏鄉張本,歸德楊敬,井陘畢鸞,永年楊忠,永清右衛穆弘,武驤左衛成貴,江寧顧暘,舒城吏部主事胡紀、御史王紹,廬江張政,武進胡長寧,徐州金暠、王豫,桐城檀郁,歸德衛呂仲和,麻城趙說,聊城裴俊,陵縣虎賁左衛經歷張讓,費縣葛子成,樂安孫整,冠縣陳勉,臨清賈貴,郯城郭秉,東平張琛,德州張泰,平陰王福緣,猗氏王約,高平王起孝、太僕丞王璲,介休楊智,興縣郭安,朔州衛吳順,杞縣高朗,太康軒茂良,鄭州邢恭,祥符李斌,鳳翔石玫,膚施劉友得、張信,邠州郭元,延安衛薛廣,蘭州吳仕坤。

景泰間,則有成安張憲,威縣傅海,邳州岑義,鳳陽李忠,徐州朱環,宿州郭興、李寬,泗州衛蔡興,龍泉顧佛僧,龍游常州通判徐珙,武昌衛吳綬,靖州衛方觀,鄆城李逵,朝城王禮,聊城朱舉,洛陽昌黎訓導閻禹錫。

天順間,則有宛平龔然勝,遷安蔣盛,永清賈懋,任丘黃文,唐縣寇林、大寧指揮張英,平山衛房鎮,忠義衛總旗鐘通,潼關衛楊順通、順素,蒙城汪泉,六合胡琛,合肥高興、張俊,和州獲嘉知縣薛良,上元龍景華,杭州姚文、姚得,平湖夔州知府沈琮,金華宗祉,德州尹綸,東昌許通,臨汾續鳳,絳州陳璽,鄢陵解禮、順孫張縉,上蔡硃儉,同州侯智,醴泉張璉,西安前衛張軫,延安衛指揮柏英,太和楊寧,金齒衛徐訥。

成化間,則有神機營指揮方榮,大醫院生安陽郭本,順天舉人萬盛,順天東安昌樂訓導周尚文,武清柳芳,玉田李茂,無極李皚,開州任勉、陳璋、僉事侯英及弟侃、副使甘澤,贊皇劉哲,平山光祿署丞李傑,莘縣李志及子忱,邢臺井澍,豐潤馬敬,柏鄉高明,定州竇文真、王達,平鄉張翱、史諫、史誼,永平秦良、硃輝,武平衛成綱、楊升,隆慶左衛衛瑾,宣府左衛何文,潼關衛千戶藍瑄,遼東定遼左衛劉定、東寧衛序班劉鼎,江寧福建參議盧雍,徐州吳友直、路車、張棟,山陽楊旻、順孫王鋐,滁州黃正,長洲硃灝,無錫秦永孚、仲孚,合肥沈諲,六安黃用賢,沭陽支儉,休寧吳仲成,懷寧吳本清,沛縣蔡清,歸德衛沈忠,杭州右衛金洪,黃巖項茂,富陽何訥,浙江西安錦衣百戶鄭得,麗水葉伯廣,海寧董謙,浙江建德蔡廷茶,奉化陸洪,餘干桃源訓導張憲,永豐呂盛,晉江史惠,平谿汪浩,江夏傅實、周璽,監利劉祥,湘陰邵敏,東昌張銳,莘縣孔昭、趙全,恩縣王弘,汶上張鄜,堂邑王懽,陽谷錢道,單縣徐洲,聊城王安、孫良,歷城湖廣布政使王允,曹州黃表、張倫,臨清劉端,壽陽吳宗,潞州張倫,大同楊茂、楊瑞、焦鑒,渾源慶都縣丞王誠,高平李振民,平陽衛指揮僉事楊輔,安東中屯衛王經,許州何清,汜水張俊,信陽王綱、袁洪,汲縣張琛,封丘陳瑛,光州太平通判劉進,羅山王賓,衛輝徐寧,郟縣劉濟,西平尹冕,新鄉王興,確山劉政,長葛蒙陰訓導羅貴,陽武舉人蕭盛,弘農衛習潤,涇陽趙謐、駱森、趙遂,同州張鼎,洋縣武全,甘州左衛毛綱,華陰周祿,保安李端,合州陳伯剛,臨桂劉本,姚州土官高紫、潼賜。

弘治間,則有大興錢福,宛平序班夏琮,青縣張俸,南和張彪,曲周趙象賢,長垣王鼐,開州甘潤、馬宗範,薊州孟振,遷安韓廷玉,元氏王懋,深州王寧,天津衛鄭海,武平衛王矩,廣寧右衛李周,霍丘徐汝楫,海州定邊衛經歷徐謐,邳州丁友,懷遠徐本忠、劉澄,宣城吳宗周,潁上王翊,鳳陽衛張全,鳳陽張欽、王澄,嘉定縣沈輔、沈珵,昆山徐協祥,豐縣周潭,徐州權宇、楊輔,績谿許欽,英山段弘仁,六安張時厚,蕭縣唐鸞、南傑,錢塘朱昌,仁和陳璋、璋妻錢氏,餘姚黃濟之,桐廬王瑁,江西樂安謝紳,南昌左衛黃璉,安福劉珍,豐城餘壽,湖廣寧鄉同知劉端,湘陰甘準,祁陽張機,閩縣高惟一,龍谿王彞,濟南序班穀珍,莘縣白溥,鄒平辛恕,堂邑李尚質,益都冀琮,文登致仕縣丞劉鑑,臨清王祐,寧海州卜懷,陵川徐河、徐瑛,平遙趙澄,澤州宋甫、裴春、舉人李用,興縣白好古,解州李錦,陽曲薛敬,檢次趙復性,屯留衛李清,儀封謝欽,祥符陳鎧、周府儀賓史經,西平張文佐,河南唐縣李擴,登封王祺,嵩縣杜端,裕州劉宗周,閿鄉薛璋,洛陽護衛軍余章瀚,鈞州陣希全,新鄭張遂,郟縣黃錦,咸寧舉人楊時敷,涇陽熊玻、張憲,隴西李琦,甘州後衛徐行,博羅何宇新,雲南芮城李錦及子澤、澤子柄,太和楊謫仙,靖安陳伯瑄及子恩。

正德間,則有高邑湘潭驛丞董玹,槁城劉強,定州趙鵬,吳橋段興,直隸新城李瑟,沙河王得時,青陽李希仁,永康歸德訓導應剛,進賢趙氏郡珍,宜春易直,善化陳大用,湘陰蘇純,侯官黃文會,邵武謝思,長山許嗣聰,聊城梁瑾,曲阜孔承夏,日照張旻,臨汾李大經及子承芳,新鄭王科,蒲城雷瑜,嵩明陳大韶。

嘉靖以後,國史不詳載,姓名所可考者,嘉靖間,則有直隸趙進、黃流、張節,冀州王國臣,六安順孫李九疇,望江順孫龍湧,太湖呂腆,沛縣楊冕,潁上王敷政,華亭徐億,浙江龔曇、王晁、孫堪、樓階、丘敘、吳燧,江西餘冠雄、曾柏,福建吳毓嘉、孫炳、丘子能,莆田舉人方重傑,山東宮守禮、王選,河南馮金玉、劉一魁,信陽趙謨,孝婦韓氏、安氏,杞縣邊雲鵡,陜西黃驥、張琛、李實,環縣趙璋,新會容璊,四川李應麒,嘉定州舉人王表,祿豐唐文炳、文蔚,蒙化舉人範運吉、黃巖。又有旌表天下孝子鮑燦、陸爻、徐億等,俱軼其鄉里。

隆慶間,則有大興李彪,靜海周一念、周斐,遷安楊騰,松江舉人馮行可,新鄉張登元,興業何世錦,崇善何珵。

萬歷間,則有直隸韓錫,深州林基,井陘張民望,清豐侯燦,河間吳應奎,平山舉人邢雲衢,邳州張縝,直隸華亭楊應祈、高承順,太湖顧槐,盱眙蔣臚,六安何金,遂安毛存元,江西餘鑰、徐信,都昌曹珊,萬安劉靜,新建樊儆、舒泰,會昌歐於復,鄱陽李岐,奉新周勃,南昌曹必和,湖廣賈應進,光化蔡玉、蔡佩,黃岡唐治,浦城徐彪,泉州訓導王熺及熺子文升,晉江韋起宗,山東馬致遠,冠縣申一琴、一攀,岳陽王應科,河南侯鶴齡,歸德賈洙,密縣陳邦寵,舞陽楊愈光,汜水王謙,淅川劉待徵,陜西劉燧,涇陽韓汝復,寧州周大賢,成都後衛楊茂勳,井研曾海,大姚金鯉,蒙化範潤,四川孝女解氏。又有馬錦、張浩、杜惠、孝女楊氏等,不詳邑里。

天啟間,則有安州邵桂,棗強先自正,晉州張蘭,高邑孫喬,上海張秉介,高淳葛至學,旌德江景宗,山陽張致中,歙縣吳榮讓、孝童女胡之憲、玉娥,慈谿馮象臨,吉水郭元達,宜春鐘名揚,峽江黃國賓,臨川傅合,萬載彭夢瑞,南康楊可幸,萬安羅應齎,江西樂安曹希和,安福孝婦王三重妻謝氏,孝感施文星,福建李躍龍,甌寧陳榮,晉江丘應賓,浦城吳昂,禹城給事中楊士衡,泰安範希賢,曹縣王治寧,曲阜孔弘傳,德州紀紹堯,聞喜張學孔,陳州郭一肖,虞城呂桂芳,淅川何大縉,華州孫繩祖,梁山李資孝,又有王錫光不詳邑里。

崇禎間,則有應天王之卿,故城李華先,仁和沈尚志,江西王之範,福建吳宗烜,山東硃文龍,忻州趙裕心,稷山舉人史宗禹,淳化高起鳳,雲南趙文宿。又有王宅中、任萬庫、武世捷、孔維章、浦某、褚咸、孫良輔等,不詳邑里。皆以孝行旌其門。

其同居敦睦者,則有洪武時龍游夏文昭,四世同居。成化間,霸州秦貴,建德何永敬,蒲圻李,句容戴睿,饒陽耿寬,俱七世同居,石首王宗義五世同爨,宿遷張賓八世同爨,安東蘇勒,潞城韓錦、李昇,永州唐汝賢,豐城劉志清,俱六世同居。弘治間,密雲李琚,合肥鄭元,陵川徐梁,安東朱勇,五世同居,慶都黃鐘,定邊衛韓鵬,俱六世同居,孝感程昂七世同居,泰州王玉八世同爨。正德間,山陽丁震五世同居。嘉靖間,石偉十一世同居,遂安毛彥恭六世同居。萬歷間,蕭梅七世同居,滁州盧守一,長治仇大,六世同居,先後得節烈貞女二十三人,太平楊乙六累世同居。天啟間,南城吳煥八世同居。皆旌曰義門。

其輸財助官振濟者,則有正統間千戶胡文郁,訓術李昺,訓科劉文勝,吉安胡有初、謝子寬,浮梁範孔孫,榆次於敏,邳州鞏得海、岑仲暉、高興、葉旺、高宗泰,沭陽葛禎,清河王仲英,山陽鮑越,懷遠廖冠平、張簡,石州張雷,淮安梁闢、李成、俞勝、徐成,潞州李廷玉,羅山王必通,溧陽陸旺,餘干舒彥祥,溫州李倫、鄒有真,四安何仕能、王清。景泰間,江陰陳安常。天順間,潮陽郭吾,太原栗仲仁,代州李斌。弘治中,歸善吳宗益、宗義及宗義子璋。隆慶間,永寧王潔、胥瓚。萬歷間,少卿吳炯,浙江董欽等,臨清張氏,江西胡士琇、丁果、婁世潔、黎金球,山西孫光勛、高自修,亳州李文明,順義楊惟孝。天啟間南城吳煥。崇禎間席本楨等。皆旌為義門,或賜璽書褒勞。

○孝義一

鄭濂王澄徐允讓石永壽錢瑛曾鼎姚玭丘鐸李茂崔敏劉鎬顧琇周琬虞宗濟等伍洪劉文煥朱煦危貞昉劉謹李德成沈德四謝定住包實夫蘇奎章權謹趙紳向化陸尚質麴祥

鄭濂,字仲德,浦江人。其家累世同居,幾三百年。七世祖綺載《宋史·孝義傳》。六傳至文嗣,旌為義門,載《元史·孝友傳》。弟文融,字太和,部使者余闕表為東浙第一家。鄭氏家法,代以一人主家政。文融卒,嗣子欽繼之,嘗刺血療本生父疾。欽卒,弟鉅繼。鉅卒,弟銘當主家政,以兄子渭宗子也,相讓久之,始受事。銘受業於吳萊。銘卒,弟鉉繼。父喪,慟哭三日,發鬚盡白。元末兵起,大將數入其境,相戒無犯義門。樞密判官阿魯灰軍奪民財,鉉以利害折之,引去。明兵臨婺州,鉉挈家避,右丞李文忠為扃鑰其家,而遣兵護之歸。至正中卒,渭繼。渭卒,弟濂繼。

濂受知於太祖,昆弟由是顯。濂以賦長詣京師,太祖問治家長久之道。對曰:「謹守祖訓,不聽婦言。」帝稱善,賜之果,濂拜賜懷歸,剖分家人。帝聞嘉嘆,欲官之,以老辭。時富室多以罪傾宗,而鄭氏數千指獨完。會胡惟庸以罪誅,有訴鄭氏交通者,吏捕之,兄弟六人爭欲行,濂弟湜竟往。時濂在京師,迎謂曰:「吾居長,當任罪。」湜曰:「兄年老,吾自往辨。」二人爭入獄。太祖召見曰:「有人如此,肯從人為逆耶?」宥之,立擢湜為左參議,命舉所知。湜舉同郡王應等五人,皆授參議。湜,字仲持,居官有政聲。南靖民為亂,詿誤者數百家,湜言於諸將,盡釋免。居一歲,入覲,卒於京。

十九年,濂坐事當逮,從弟洧曰:「吾家稱義門,先世有兄代弟死者,吾可不代兄死乎?」詣吏自誣服,斬於市。洧,字仲宗,受業於宋濂,有學行,鄉人哀之,私謚貞義處士。

濂卒,弟渶繼。二十六年,東宮缺官,命廷臣舉孝弟敦行者,眾以鄭氏對。太祖曰:「其里王氏亦仿鄭氏家法。」乃徵兩家子弟年三十上者,悉赴京,擢濂弟濟與王懃為春坊左、右庶子。後又徵濂弟沂,自白衣擢禮部尚書,年餘,致仕。永樂元年入朝,留為故官。未幾,復謝去。濂從子幹官御史,棠官檢討。他得官者復數人,鄭氏愈顯。濟、棠皆學於宋濂,有文行。

初,渶嘗仕元為浙江行省宣使,主家政數年。建文帝表其門,渶朝謝,御書「孝義家」三字賜之。燕兵既入,有告建文帝匿其家者,遣人索之。渶家廳事中,列十大櫃,五貯經史,五貯兵器備不虞。使者至,所發皆經史,置其半不啟,乃免於禍,人以為至行所感云。成化十年,有司奏鄭永朝世敦行義,復旌以孝義之門。

自文融至渶,皆以篤行著。文融著《家範》三卷,凡五十八則,子欽增七十則,從子鉉又增九十二則,至濂弟濤與從弟泳、澳、湜,白于兄濂、源,共相損益,定為一百六十八則,刊行焉。

王澄,字德輝,亦浦江人。歲儉,出粟貸人,不取其息。有鬻產者,必增直以足之。慕義門鄭氏風,將終,集子孫誨之曰:「汝曹能合食同居如鄭氏,吾死目瞑矣。」子孫咸拜受教。澄生三子子覺、子麟、子偉,克承父志。子覺生應,即為鄭湜所舉擢參議者。子偉生懃,即與鄭濟並擢庶子者。義門王氏之名,遂埒鄭氏。

又有王燾者,蘄水人,七世同居,一家二百餘口,人無間言。洪武九年十一月,詔旌為孝義之門。

徐允讓,浙江山陰人。元末,賊起,奉父安走避山谷間。遇賊,欲斫安頸。允讓大呼曰:「寧殺我,勿殺我父!」賊遂舍安殺允讓。將辱其妻潘,潘紿曰:「吾夫已死,從汝必矣。若能焚吾夫,則無憾也。」賊許之,潘聚薪焚夫,投烈焰中死。賊驚歎去,安獲全。洪武十六年,夫婦並旌。

同時石永壽者,新昌人。負老父避賊,賊執其父將殺之,號泣請代,賊殺永壽而去。

錢瑛,字可大,吉水人。生八月而孤,年十三能應秋試。及長,值元季亂,奉祖本和及母避難,歷五六年。遇賊,縛本和,瑛奔救,并縛之。本和哀告貰其孫,瑛泣請代不已,賊憐而兩釋之。時瑛母亦被執,瑛妻張從伏莽中窺見,即趨出,謂賊曰:「姑老矣,請縛我。」賊從之,既就縛,擲袖中奚與姑,訣曰:「婦無用此矣。」且行且睨姑,稍遠即罵賊不肯行。賊持之急,罵益厲,賊怒,攢刃刺殺之。是定,有司知瑛賢,凡三薦,並以親老辭。子遂志成進士,官山東僉事。

同時曾鼎,字元友,泰和人。祖懷可、父思立,並有學行。元末,鼎奉母避賊。母被執,鼎跪而泣請代。賊怒,將殺母,鼎號呼以身翼蔽,傷頂肩及足,控母不舍。賊魁繼至,憫之,攜其母子入營療治,獲愈。行省聞其賢,闢為濂谿書院山長。洪武三年,知縣郝思讓闢教設學。鼎好學能詩,兼工八分及邵子數學。

姚玭,松江人。元至正中,苗帥楊完者兵入境。玭奉母避於野,阻河不可渡。母泣曰:「兵至,吾誓不受辱。」遂沉於水。玭急投水救之,負母而出。已,數遇盜,中矢,玭佯死伏屍間以免,以奉母過湖、淮。後母疾思食魚,暮夜無從得,家養一烏,忽飛去攫魚以歸。洪武初,行省聞其賢,闢之,以親老不就。

丘鐸,字文振,祥符人。元末,父為湖廣儒學提舉。值兵亂,鐸奉父母播遷,賣藥供甘旨。母卒,哀慟幾絕。葬鳴鳳山,結廬墓側,朝夕上食如生時。當寒夜月黑,悲風蕭瑟,鐸輒繞墓號曰:「兒在斯!兒在斯!」山深多虎,聞鐸哭聲避去。時稱真孝子。鐸初避寇慶元,從祖父母居故鄉者八人,貧不能自存,鐸悉迎養之。有姑年十八,夫亡守節,鐸養之終身。

後有李茂者,澄城諸生也。母患惡瘡。茂日吮膿血,夜則叩天祈代。及卒,結廬墓旁,朝夕悲泣。天大雨,懼沖其墓,伏墓而哭,雨止乃已。父卒,廬墓如之。成化二生旌。二子表、森,森為國子生。茂卒,兄弟同廬於墓。弘治五年旌。表子俊亦國子生,表卒,俊方弱冠,廬墓終喪。母卒,亦如初。正德四年旌。

崔敏,字好學,襄陵人。生四十日,其父仕元為綿竹尹,父子隔絕者三十年。敏依母兄以居。元季寇亂,母及兄俱相失。亂定,入陜尋母不得。由陜入川,抵綿竹,求父塚,無知者。復還陜,訪諸親故,始知父殯所在,乃啟攢負骸歸。時稱崔孝子。

同時劉鎬,江西龍泉人。父允中,洪武五年舉人,官憑祥巡檢,卒於任。鎬以道遠家貧,不能返柩,居常悲泣。父友憐之,言於廣西監司,聘為臨桂訓導。尋假公事赴憑祥,莫知葬處。鎬晝夜環哭,一蒼頭故從其父,已轉入交址。忽暮至,若有憑之者,因得塚所在。刺血驗之良是,乃負歸葬。

有顧琇者,字季粟,吳縣人。洪武初,父充軍鳳翔,母隨行,留琇守丘墓。越六年,母歿。琇奔赴,負母骨行數千里,寢則懸之屋梁,涉則戴之於頂。父釋歸卒。水漿不入口五日,不勝喪而死。

周琬,江寧人。洪武時,父為滁州牧,坐罪論死。琬年十六,叩閽請代。帝疑受人教,命斬之,琬顏色不變。帝異之,命宥父死,謫戍邊。琬復請曰:「戍與斬,均死爾。父死,子安用生為,顧就死以贖父戍。」帝復怒,命縛赴市曹,琬色甚喜。帝察其誠,即赦之,親題御屏曰「孝子周琬。」尋授兵科給事中。

同時子代父死者,更有虞宗濟、胡剛、陳圭。宗濟,字思訓,常熟人。父兄並有罪,吏將逮治。宗濟謂兄曰:「事涉徭役,國法嚴,往必死。父老矣,兄塚嗣,且未有後,我幸產兒,可代死。」乃挺身詣吏,白父兄無所預。吏疑而訊之,悉自引伏。洪武四年竟斬於市,年二十二。剛,浙江新昌人。洪武初,父謫役泗上,以逃亡當死,敕駙馬都尉梅殷監刑。剛時方走省,立河上俟渡。聞之,即解衣泅水而往,哀號泣代。殷憫之,奏聞,詔宥其父,并宥同罪者八十二人。圭,黃巖人。父為仇人所訐當死,圭詣闕上章曰:「臣為子不能諫父,致陷不義,罪當死,乞原父使自新。」帝大喜曰:「不謂今日有此孝子,宜赦其父,俟四方朝覲官至,播告之,以風勵天下。」刑部尚書開濟奏曰:「罪有常刑,不宜屈法開僥幸路。」乃聽圭代,而戍其父雲南。

十七年,左都御史詹徽奏言:「太平府民有毆孕婦至死者,罪當絞,其子請代。」章下大理卿鄒俊議,曰:「子代父死,情固可嘉。然死婦繫二人之命,冤曷由申;犯人當二死之條,律何可貸。與其存犯法之父,孰若全無罪之兒。」詔從其議。

伍洪,字伯宏,安福人。洪武四年進士。授績谿主簿,擢上元知縣。丁外艱,服除,以母老不復仕。推資產與諸弟,而己獨隱居養母。有異母弟得罪逃,使者捕弗獲,執其母,洪哭訴求代。母曰:「汝往必死,莫若吾自當之。」洪曰:「安有子在而累母者。」遂行,竟死於市。

時有劉文煥者,廣濟人。與兄文煇運糧愆期,當死。兄以長坐,文煥詣吏請代,叩頭流血。所司上其狀,命宥之,則兄已死矣。太祖特書「義民」二字獎之。

時京師有兄坐法,兩弟各自縛請代。太祖遣使問故,同詞對曰:「臣少失父,非兄無以至今日。兄當死,弟安敢愛其生。」帝陽許之,而戒行刑者曰:「有難色者殺之,否則奏聞。」兩人皆引頸就刃,帝大嗟異,欲並其兄貰之。左都御史詹徽持不可,卒殺其兄。

硃煦,仙居人。父季用,為福州知府。洪武十八年詔盡逮天下積歲官吏為民害者,赴京師築城。季用居官僅五月,亦被逮,病不能堪,謂煦曰:「吾辦一死耳,汝第收吾骨歸葬。」煦惶懼不敢頃刻離。時訴枉令嚴,訴而戍極邊者三人,抵極刑者四人矣。煦奮曰:「訴不訴,等死耳,萬一父緣訴獲免,即戮死無恨。」即具狀叩闕。太祖憫其意,赦季用,復其官。

有危貞昉者,字孟陽,臨海諸生。父孝先,洪武四年進士。官陵川縣丞,坐法輸作江浦。貞昉詣闕上疏曰:「臣父絓吏議輸作,筋力向衰,不任勞苦,而大母年逾九十,恐染霜露之疾,貽臣父終天之恨。臣犬馬齒方壯,願代父作勞,俾父獲歸養,死且不朽。」詔從之。貞昉力作不勝勞,閱七月病卒。

劉謹,浙江山陰人。洪武中,父坐法戍雲南。謹方六歲,問家人「雲南何在?」家人以西南指之,輒朝夕向之拜。年十四,矍然曰:「雲南雖萬里,天下豈有無父之子哉!」奮身而往,閱六月抵其地,遇父於逆旅,相持號慟。俄父患瘋痺,謹告官乞以身代。法令戍邊者必年十六以上,嫡長男始許代。時謹未成丁,伯兄先死,乃歸家攜兄子往。兄子亦弱未能自立,復歸悉鬻其產畀兄子,始獲奉其父還,孝養終身。

李德成,浹水人。幼喪父。元末,年十二,隨母避寇至河濱。寇騎迫,母投河死。德成長,娶婦王氏。摶土為父母像,與妻朝夕事之。方嚴冬,大雪,水堅至河底。德成夢母曰:「我處水下,寒不得出。」覺而大慟,旦與妻徒跣行三百里,抵河濱。臥水七日,水果融數十丈,恍惚若見其母,而他處堅凍如故。久之,乃歸。洪武十九年舉孝廉,屢擢尚寶丞。二十七年旌為孝子。建文中,燕兵逼濟南。德成往諭令還兵,燕兵不退。德成歸,以辱命下吏,已而釋之。永樂初復官,屢遷陜西布政使。

沈德四,直隸華亭人。祖母疾,刲股療之愈。己而祖父疾,又刲肝作湯進之,亦愈。洪武二十六年被旌。尋授太常贊禮郎。上元姚金玉、昌平王德兒亦以刲肝愈母疾,與德四同旌。

至二十七年九月,山東守臣言:「日照民江伯兒,母疾,割肋肉以療,不愈。禱岱嶽神,母疾瘳,願殺子以祀。已果瘳,竟殺其三歲兒。」帝大怒曰:「父子天倫至重。《禮》父服長子三年。今小民無知,滅倫害理,亟宜治罪。」遂逮伯兒,仗之百,遣戍海南。因命議旌表例。

禮臣議曰:「人子事親,居則致其敬,養則致其樂,有疾則醫藥籲禱,迫切之情,人子所得為也。至臥冰割股,上古未聞。倘父母止有一子,或割肝而喪生,或臥冰而致死,使父母無依,宗祀永絕,反為不孝之大。皆由愚昧之徒,尚詭異,駭愚俗,希旌表,規避里徭。割股不已,至於割肝,割肝不已,至於殺子。違道傷生,莫此為甚。自今父母有疾,療治罔功,不得已而臥冰割股,亦聽其所為,不在旌表例。」制曰:「可。」

永樂間,江陰衛卒徐佛保等復以割股被旌。而掖縣張信、金吾右衛總旗張法保援李德成故事,俱擢尚寶丞。迨英、景以還,即割股者亦格於例,不以聞,而所旌,大率皆廬墓者矣。

謝定住,大同廣昌人。年十二,家失牛。母抱幼子追逐,定住隨母後。虎躍出噬其母,定住奮前擊之,虎逸去。取弟抱之,扶母行。虎復追齧母頸,定住再擊之,虎復去。行數武,虎還嚙母足。定住復取石擊,虎乃舍去,母子三人並全。永樂十二年,帝召見嘉獎,賜米十石、鈔二百錠,旌其門。

先是,洪武中,有包實夫者,進賢人。授徒數十里外,途遇虎,銜衣入林中,釋而蹲。實夫拜請曰:「吾被食,命也,如父母失養何?」虎即舍去。後人名其地為拜虎岡。其後,嘉靖中,筠連諸生蘇奎章,從父入山,猝遇虎。奎章倉皇泣告,願舍父食己,虎曳尾徐去。後為岷府教授。

權謹,字仲常,徐州人。十歲喪父,即哀毀,奉母至孝。永樂四年薦授樂安知縣,遷光祿署丞,以省侍歸。母年九十終,廬墓三年,致泉湧免馴之異。有司以聞,仁宗命馳驛赴闕,出其事狀,令侍臣朗誦大廷,以示百僚,即拜文華殿大學士。謹辭,帝曰:「朕擢卿以風天下為子者,他非卿責也。」尋扈從皇太子監國南京。宣宗嗣位,以疾乞歸,改通政司右參議,賜白金文綺致仕。子倫,舉永樂中鄉試。養親二十年,親終不仕。倫子宇,父母卒,皆廬墓。成化十二年亦獲旌。

趙紳,字以行,諸暨人。父秩,永樂中為高郵州學正,考滿赴京,至武城縣墮水。紳奮身下救,河流湍悍,俱不能出。明日屍浮水上,紳兩手抱父臂不釋。宣德五年旌其門。

有向化者,靜海衛人。父上為衛指揮,墮海死。化號泣求尸不得,亦投於海。忽父屍浮出,衣服盡脫。天方晴霽,雷雨驟作。既息,化首頂父衣,浮至一處。眾異而收葬之。

陸尚質者,山陰人。父渡江遇風,飄舟將入海。尚質自崖見之,即躍入濤中,欲挽舟近岸。父舟獲濟,而尚質竟溺死。里人呼其處為陸郎渡。

麴祥,字景德,永平人。永樂中,父亮為金山衛百戶。祥年十四,被倭掠。國王知為中國人,召侍左右,改名元貴,遂仕其國,有妻子,然心未嘗一日忘中國也,屢諷王入貢。宣德中,與使臣偕來,上疏言:「臣夙遭俘掠,抱釁痛心,流離困頓,艱苦萬狀。今獲生還中國,夫豈由人。伏乞賜歸侍養,不勝至願。」天子方懷柔遠人,不從其請,但許給驛暫歸,仍還本國。祥抵家,獨其母在,不能識,曰:「果吾兒,則耳陰有赤痣。」驗之信,抱持痛哭。未幾別去,至日本,啟以帝意。國王允之,仍令入貢。祥乃復申前請,詔許襲職歸養。母子相失二十年,又有華夷之限,竟得遂其初志,聞者異之。


\end{pinyinscope}