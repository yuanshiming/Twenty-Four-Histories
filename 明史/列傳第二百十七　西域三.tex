\article{列傳第二百十七 西域三}

\begin{pinyinscope}
○烏斯藏大寶法王大乘法王大慈法王闡化王贊善王護教王闡教王輔教王西天阿難功德國西天尼八剌國朵甘烏斯藏行都指揮使司長河西魚通寧遠宣慰司董卜韓胡宣慰司

烏斯藏,在雲南西徼外,去雲南麗江府千餘里,四川馬湖府千五百餘里,陜西西寧衛五千餘里。其地多僧,無城郭。群居大土臺上,不食肉娶妻,無刑罰,亦無兵革,鮮疾病。佛書甚多,《楞伽經》至萬卷。其土臺外,僧有食肉娶妻者。元世祖尊八思巴為大寶法王,錫玉印,既沒,賜號皇天之下一人之上宣文輔治大聖至德普覺真智佐國如意大寶法王西天佛子大元帝師。自是,其徒嗣者咸稱帝師。

洪武初,太祖懲唐世吐蕃之亂,思制御之。惟因其俗尚,用僧徒化導為善,乃遣使廣行招諭。又遣陜西行省員外郎許允德使其地,令舉元故官赴京授職。於是烏斯藏攝帝師喃加巴藏卜先遣使朝貢。五年十二月至京。帝喜,賜紅綺禪衣及鞋帽錢物。明年二月躬自入朝,上所舉故官六十人。帝悉授以職,改攝帝師為熾盛佛寶國師,仍錫玉印及彩幣表裏各二十。玉人製印成,帝眎玉未美,令更製,其崇敬如此。暨辭還,命河州衛遣官齎敕偕行,招諭諸番之未附者。冬,元帝師之後鎖南堅巴藏卜、元國公哥列思監藏巴藏卜並遣使乞玉印。廷臣言已嘗給賜,不宜復予,乃以文綺賜之。

七年夏,佛寶國師遣其徒來貢。秋,元帝師八思巴之後公哥監藏巴藏卜及烏斯藏僧答力麻八剌遣使來朝,請封號。詔授帝師後人為圓智妙覺弘教大國師,烏斯藏僧為灌頂國師,並賜玉印。佛寶國師復遣其徒來貢,上所舉土官五十八人,亦皆授職。九年,答力麻八剌遣使來貢。十一年復貢,奏舉故官十六人為宣慰、招討等官,亦皆報允。十四年復貢。

其時喃加巴藏卜已卒,有僧哈立麻者,國人以其有道術,稱之為尚師。成祖為燕王時,知其名。永樂元年命司禮少監侯顯、僧智光齎書幣往征。其僧先遣人來貢,而躬隨使者入朝。四年冬將至,命駙馬都尉沐昕往迎之。既至,帝延見於奉天殿,明日宴華蓋殿,賜黃金百,白金千,鈔二萬,彩幣四十五表裏,法器、示因褥、鞍馬、香果、茶米諸物畢備。其從者亦有賜。明年春,賜儀仗、銀瓜、牙仗、骨朵、膋燈、紗燈、香合、拂子各二,手爐六,傘蓋一,銀交椅、銀足踏、銀杌、銀盆、銀罐、青圓扇、紅圓扇、拜褥、帳幄各一,幡幢四十有八,鞍馬二,散馬四。

帝將薦福於高帝后,命建普度大齋於靈谷寺七日。帝躬自行香。於是卿雲、甘露、青烏、白象之屬,連日畢見。帝大悅,侍臣多獻賦頌。事竣,復賜黃金百,白金千,寶鈔二千,彩幣表裏百二十,馬九。其徒灌頂圓通善慧大國師答師巴羅葛羅思等,亦加優賜。遂封哈立麻為萬行具足十方最勝圓覺妙智慧善普應佑國演教如來大寶法王西天大善自在佛,領天下釋教,賜印誥及金、銀、鈔、彩幣、織金珠袈裟、金銀器、鞍馬。命其徒孛隆逋瓦桑兒加領真為灌頂圓修凈慧大國師,高日瓦禪伯為灌頂通悟弘濟大國師,果欒羅葛羅監藏巴里藏卜為灌頂弘智凈戒大國師,並賜印誥、銀鈔、彩幣。已,命哈立麻赴五臺山建大齋,再為高帝后薦福,賜予優厚。六年四月辭歸,復賜金幣、佛像,命中官護行。自是,迄正統末,入貢者八。已,法王卒,久不奉貢。弘治八年,王葛哩麻巴始遣使來貢。十二年兩貢,禮官以一歲再貢非制,請裁其賜賚,從之。

正德元年來貢。十年復來貢。時帝惑近習言,謂烏斯藏僧有能知三生者,國人稱之為活佛,欣然欲見之。考永、宣間陳誠、侯顯入番故事,命中官劉允乘傳往迎。閣臣梁儲等言:「西番之教,邪妄不經。我祖宗朝雖嘗遣使,蓋因天下初定,藉以化導愚頑,鎮撫荒服,非信其教而崇奉之也。承平之後,累朝列聖止因其來朝而賞賚之,未嘗輕辱命使,遠涉其地。今忽遣近侍往送幢幡,朝野聞之,莫不駭愕。而允奏乞鹽引至數萬,動撥馬船至百艘,又許其便宜處置錢物,勢必攜帶私鹽,騷擾郵傳,為官民患。今蜀中大盜初平,瘡痍未起。在官已無餘積,必至苛斂軍民,金廷而走險,盜將復發。況自天全六番出境,涉數萬之程,歷數歲之久,道途絕無郵置,人馬安從供頓?脫中途遇寇,何以禦之?虧中國之體,納外番之侮,無一可者。所齎敕書,臣等不敢撰擬。」帝不聽。禮部尚書毛紀、六科給事中葉相、十三道御史周倫等並切諫,亦不聽。。

允行,以珠琲為幢幡,黃金為供具,賜其僧金印,犒賞以鉅萬計,內庫黃金為之罄盡。敕允往返以十年為期,所攜茶鹽以數十萬計。允至臨清,漕艘為之阻滯。入峽江,舟大難進,易以冓鹿,相連二百餘里。及抵成都,日支官廩百石,蔬菜銀百兩,錦官驛不足,取傍近數十驛供之。治入番器物,估直二十萬。守臣力爭,減至十三萬。工人雜造,夜以繼日。居歲餘,始率將校十人、士千人以行,越兩月入其地。所謂活佛者,恐中國誘害之,匿不出見。將士怒,欲脅以威。番人夜襲之,奪寶貨、器械以去。將校死者二人,卒數百人,傷者半之。允乘善馬疾走,僅免。返成都,戒部下弗言,而以空函馳奏,至則武宗已崩。世宗召允還,下吏治罪。

嘉靖中,法王猶數入貢,迄神宗朝不絕。時有僧鎖南堅錯者,能知已往未來事,稱活佛,順義王俺答亦崇信之。萬曆七年,以迎活佛為名,西侵瓦剌,為所敗。此僧戒以好殺,勸之東還。俺答亦勸此僧通中國,乃自甘州遺書張居正,自稱釋迦摩尼比丘,求通貢,饋以儀物。居正不敢受,聞之於帝。帝命受之,而許其貢。由是,中國亦知有活佛。此僧有異術能服人,諸番莫不從其教,即大寶法王及闡化諸王,亦皆俯首稱弟子。自是西方止知奉此僧,諸番王徒擁虛位,不復能施其號令矣。

大乘法王者,烏斯藏僧昆澤思巴也,其徒亦稱為尚師。永樂時,成祖既封哈立麻,又聞昆澤思巴有道術,命中官齎璽書銀幣徵之。其僧先遣人貢舍利、佛像,遂偕使者入朝。十一年二月至京,帝即延見,賜藏經、銀鈔、彩幣、鞍馬、茶果諸物,封為萬行圓融妙法最勝真如慧智弘慈廣濟護國演教正覺大乘法王西天上善金剛普應大光明佛,領天下釋教,賜印誥、袈裟、幡幢、鞍馬、傘器諸物,禮之亞於大寶法王。明年辭歸,賜加於前,命中官護行。後數入貢,帝亦先後命中官喬來喜、楊三保齎賜佛像、法器、袈裟、禪衣、絨錦、彩幣諸物。洪熙、宣德間並來貢。

成化四年,其王完卜遣使來貢。禮官言無法王印文,且從洮州入,非制,宜減其賜物。使者言,所居去烏斯藏二十餘程,涉五年方達京師,且所進馬多,乞給全賜,乃命量增。十七年來貢。

弘治元年,其王桑加瓦遣使來貢。故事,法王卒,其徒自相繼承,不由朝命。三年,輔教王遣使奉貢,奏舉大乘法王襲職。帝但納其貢,賜賚遣還,不命襲職。

正德五年遣其徒綽吉我些兒等,從河州衛入貢。禮官以其非貢道,請減其賞,並治指揮徐經罪,從之。已,綽吉我些兒有寵於帝,亦封大德法王。十年,僧完卜鎖南堅參巴爾藏卜遣使來貢,乞襲大乘法王。禮官失於稽考,竟許之。嘉靖十五年偕輔教、闡教諸王來貢,使者至四千餘人。帝以人數踰額,減其賞,並治四川三司官濫送之罪。

初,成祖封闡化等五王,各有分地,惟二法王以遊僧不常厥居,故其貢期不在三年之列。然終明世,奉貢不絕云。

大慈法王,名釋迦也失,亦烏斯藏僧稱為尚師者也。永樂中,既封二法王,其徒爭欲見天子邀恩寵,於是來者趾相接。釋迦也失亦以十二年入朝,禮亞大乘法王。明年命為妙覺圓通慈慧普應輔國顯教灌頂弘善西天佛子大國師,賜之印誥。十四年辭歸,賜佛經、佛像、法仗、僧衣、綺帛、金銀器,且御製贊詞賜之,其徒益以為榮。明年遣使來貢。十七年命中官楊三保齎佛像、衣幣往賜。二十一年復來貢。宣德九年入朝,帝留之京師,命成國公朱勇、禮部尚書胡濙持節,冊封為萬行妙明真如上勝清凈般若弘照普慧輔國顯教至善大慈法王西天正覺如來自在大圓通佛。

宣宗崩,英宗嗣位,禮官先奏汰番僧六百九十人,正統元年復以為請。命大慈法王及西天佛子如故,餘遣還,不願者減酒饌廩餼,自是輦下稍清。西天佛子者,能仁寺僧智光也,本山東慶雲人。洪武、永樂中,數奉使西國。成祖賜號國師,仁宗加號圓融妙慧凈覺弘濟輔國光範演教灌頂廣善大國師,賜金印、冠服、金銀器。至是復加西天佛子。

初,太祖招徠番僧,本藉以化愚俗,弭邊患,授國師、大國師者不過四五人。至成祖兼崇其教,自闡化等五王及二法王外,授西天佛子者二,灌頂大國師者九,灌頂國師者十有八,其他禪師、僧官不可悉數。其徒交錯於道,外擾郵傳,內耗大官,公私騷然,帝不恤也。然至者猶即遣還。及宣宗時則久留京師,耗費益甚。英宗初年,雖多遣斥,其後加封號者亦不少。景泰中,封番僧沙加為弘慈大善法王,班卓兒藏卜為灌頂大國師。英宗復辟,務反景帝之政,降法王為大國師,大國師為國師。

成化初,憲宗復好番僧,至者日眾。答刂巴堅參、答刂實巴、領占竹等,以祕密教得幸,並封法王。其次為西天佛子,他授大國師、國師、禪師者不可勝紀。四方奸民投為弟子,輒得食大官,每歲耗費鉅萬。廷臣屢以為言,悉拒不聽。孝宗踐阼,清汰番僧,法王、佛子以下,皆遞降,驅還本土,奪其印誥,由是輦下復清。

弘治六年,帝惑近習言,命取領占竹等詣京。言官交章力諫,事乃寢。十三年命為故西天佛子著領占建塔。工部尚書徐貫等言,此僧無益於國,營墓足矣,不當建塔,不從。尋命那卜堅參三人為灌頂大國師。帝崩,禮官請黜異教,三人並降禪師。

既而武宗蠱惑佞倖,復取領占竹至京,命為灌頂大國師,以先所降禪師三人為國師。帝好習番語,引入豹房,由是番僧復盛。封那卜堅參及答刂巴藏卜為法王,那卜領占及綽即羅竹為西天佛子。已,封領占班丹為大慶法王,給番僧度牒三千,聽其自度。或言,大慶法王,即帝自號也。

綽吉我些兒者,烏斯藏使臣,留豹房有寵,封大德法王。乞令其徒二人為正副使,還居本土,如大乘法王例入貢,且為二人請國師誥命,入番設茶。禮官劉春等執不可,帝不聽。春等復言:「烏斯藏遠在西方,性極頑獷。雖設四王撫化,而其來貢必為節制。若令齎茶以往,賜之誥命,彼或假上旨以誘諸番,妄有所干請。從之則非法,不從則生釁,害不可勝言。」帝乃罷設茶敕,而予之誥命。帝時益好異教,常服其服,誦習其經,演法內廠。綽吉我些兒輩出入豹房,與權倖雜處,氣焰灼然。及二人乘傳歸,所過驛騷,公私咸被其患。

世宗立,復汰番僧,法王以下悉被斥。後世宗崇道教,益黜浮屠,自是番僧鮮至中國者。

闡化王者,烏斯藏僧也。初,洪武五年,河州衛言:「烏斯藏怕木竹巴之地,有僧曰章陽沙加監藏,元時封灌頂國師,為番人推服。今朵甘酋賞竹監藏與管兀兒構兵,若遣此僧撫諭,朵甘必內附。」帝如其言,仍封灌頂國師,遣使賜玉印、彩幣。明年,其僧使酋長鎖南藏卜貢佛像、佛書、舍利。是時方命佛寶國師招諭番人,於是怕木竹巴僧等自稱輦卜闍,遣使進表及方物。帝厚賜之。輦卜闍者,其地首僧之稱也。八年正月設怕木竹巴萬戶府,以番酋為之。已而章陽沙加卒,授其徒鎖南扎思巴噫監藏卜為灌頂國師。二十一年上表稱病,舉弟吉剌思巴監藏巴藏卜自代,遂授灌頂國師。自是三年一貢。

成祖嗣位,遣僧智光往賜。永樂元年遣使入貢。四年封為灌頂國師闡化王,賜螭紐玉印,白金五百兩,綺衣三襲,錦帛五十匹,巴茶二百斤。明年命與護教、贊善二王,必力工瓦國師及必里、朵甘、隴答諸衛,川藏諸族,復置驛站,通道往來。十一年,中官楊三保使烏斯藏還,其王遣從子答刂結等隨之入貢。明年復命三保使其地,令與闡教、護教、贊善三王及川卜、川藏等共修驛站,諸未復者盡復之。自是道路畢通,使臣往還數萬里,無虞寇盜矣。其後貢益頻數。帝嘉其誠,復命三保齎佛像、法器、袈裟、禪衣及絨錦、彩幣往勞之。已,又命中官戴興往賜彩幣。

宣德二年命中官侯顯往賜絨錦、彩幣。其貢使嘗毆殺驛官子,帝以其無知,遣還,敕王戒飭而已。九年,貢使歸,以賜物易茶。至臨洮,有司沒入之,羈其使,請命。詔釋之,還其茶。

正統五年,王卒。遣禪師二人為正副使,封其從子吉剌思巴永耐監藏巴藏卜為闡化王。使臣私市茶彩數萬,令有司運致。禮官請禁之,帝念其遠人,但令自僦舟車。已,王卒,以桑兒結堅昝巴藏卜嗣。

成化元年,禮部言:「宣、正間,諸貢不過三四十人,景泰時十倍,天順間百倍。今貢使方至,乞敕諭闡化王,令如洪武舊制,三年一貢。」從之。五年,王卒,命其子公葛列思巴中柰領占堅參巴兒藏卜嗣。遣僧進貢,還至西寧,留寺中不去,又冒名入貢,隱匿所賜璽書、幣物。王使其下三人來趣,其僧閉之室中,剜二人目。一人逸,訴於都指揮孫鑑。鑑捕置之獄,受其徒賄,而復以聞。下四川巡按鞫治,坐僧四人死,鑒將逮治,會赦悉免。

十七年以長河西諸番多假番王名朝貢,命給闡化、贊善、闡教、輔教四王敕書勘合,以防奸偽。二十二年遣使四百六十人來貢,守臣遵新例,但納一百五十人。禮官以使者已入境,難固拒,請順其情概納之,為後日兩貢之數,從之。

弘治八年遣僧來貢,還至揚州廣陵驛,遇大乘法王貢使,相與殺牲縱酒,三日不去。見他使舟至,則以石投之,不容近陸。知府唐愷詣驛呼其舟子戒之,諸僧持兵仗呼噪擁而入。愷走避,隸卒力格鬥乃免,為所傷者甚眾。事聞,命治通事及伴送者罪,遣人諭王令自治其使者。其時王卒,子班阿吉江東答刂巴請襲,命番僧二人為正副使往封。比至,新王亦死,其子阿往答刂失答刂巴堅參即欲受封,二人不得已授之,遂具謝恩儀物,並獻其父所領勘合印章為左驗。至四川,守臣劾其擅封,逮治論斬,減死戍邊,副使以下悉宥。

正德三年,禮官以貢使踰額,令為後年應貢之數。嘉靖三年偕輔教王及大小三十六番請入貢。禮官以諸番不具地名、族氏,令守臣核實以聞。四十二年,闡化諸王遣使入貢請封。禮官循故事,遣番僧二十二人為正副使,序班朱廷對監之。至中途大騷擾,不受廷對約束,廷對還,白其狀。禮官請自後封番王,即以誥敕付使者齎還,或下守臣,擇近邊僧人齎賜。封諸藏之不遣京寺番僧,自此始也。番人素以入貢為利,雖屢申約束,而來者日增。隆慶三年再定令闡化、闡教、輔教三王,俱三歲一貢,貢使各千人,半全賞,半減賞。全賞者遣八人赴京,餘留邊上。遂為定例。

萬曆七年,貢使言闡化王長子札釋藏卜乞嗣職,如其請。久之卒,其子請襲。神宗許之,而制書但稱闡化王。用閣臣沈一貫言,加稱烏斯藏怕木竹巴灌頂國師闡化王。其後奉貢不替。所貢物有畫佛、銅佛、銅塔、珊瑚、犀角、氆氌、左髻毛纓、足力麻、鐵力麻、刀劍、明甲胃之屬,諸王所貢亦如之。

贊善王者,靈藏僧也。其地在四川徼外,視烏斯藏為近。成祖踐阼,命僧智光往使。永樂四年,其僧著思巴兒監藏遣使入貢,命為灌頂國師。明年封贊善王,國師如故,賜金印、誥命。十七年,中官楊三保往使。洪熙元年,王卒,從子喃葛監藏襲。宣德二年,中官侯顯往使。正統五年奏稱年老,請以長子班丹監坐刂代。帝不從其請,而授其子為都指揮使。

初,入貢無定期,自永樂迄正統,或間歲一來,或一歲再至。而歷朝遣使往賜者,金幣、寶鈔、佛像、法器、袈裟、禪服,不一而足。至成化元年始定三歲一貢之例。

三年命塔兒把堅粲襲封。故事,封番王誥敕及幣帛遣官齎賜,至是西陲多事,禮官乞付使者齎回,從之。

五年,四川都司言,贊善諸王不遵定制,遣使率各寺番僧百三十二種入貢,且無番王印文,今止留十餘人守貢物,餘已遣還。禮官言:「番地廣遠,番王亦多,若遵例並時入貢,則內郡疲供億。莫若令諸王於應貢之歲,各具印文,取次而來。今貢使已至,難拂其情。乞許作明年應貢之數。」報可。

十八年,禮官言:「番王三歲一貢,貢使百五十人,定制也。近贊善王連貢者再,已遣四百十三人。今請封請襲,又遣千五百五十人,違制宜卻。乞許其請封襲者,以三百人為後來兩貢之數,餘悉遣還。」亦報可。遂封喃葛堅粲巴藏卜為贊善王。弘治十六年卒,命其弟端竹堅昝嗣。嘉靖後猶入貢如制。

護教王者,名宗巴斡即南哥巴藏卜,館覺僧也。成祖初,僧智光使其地。永樂四年遣使入貢,詔授灌頂國師,賜之誥。明年遣使入謝,封為護教王,賜金印、誥命,國師如故。遂頻歲入貢。十二年卒,命其從子幹些兒吉剌思巴藏卜嗣。洪熙、宣德中並入貢。已而卒,無嗣,其爵遂絕。

闡教王者,必力工瓦僧也。成祖初,僧智光齎敕入番,其國師端竹監藏遣使入貢。永樂元年至京,帝喜,宴賚遣還。四年又貢,帝優賜,並賜其國師大板的達、律師鎖南藏卜衣幣。十一年乃加號灌頂慈慧凈戒大國師,又封其僧領真巴兒吉監藏為闡教王,賜印誥、綵幣。後比年一貢。楊三保、戴興、侯顯之使,皆齎金幣、佛像、法器賜焉。

宣德五年,王卒,命其子綽兒加監巴領占嗣。久之卒,命其子領占叭兒結堅參嗣。成化四年從禮官言,申三歲一貢之制。明年,王卒,命其子領占堅參叭兒藏卜襲。二十年,帝遣番僧班著兒齎璽書勘合往賜。其僧憚行,至半道,偽為王印信、番文復命,詔逮治。

正德十三年遣番僧領占答刂巴等封其新王。答刂巴等乞馬快船三十艘載食鹽,為入番買路之資。戶科、戶部並疏爭,不聽。答刂巴等在途科索無厭,至呂梁,毆管洪主事李瑜幾斃,恣橫如此。迄嘉靖世,闡教王修貢不輟。

輔教王者,思達藏僧也。其地視烏斯藏尤遠。成祖即位,命僧智光持詔招諭,賜以銀幣。永樂十一年封其僧南渴烈思巴為輔教王,賜誥印、彩幣,數通貢使。楊三保、侯顯皆往賜其國,與諸法王等。景泰七年,使來貢,自陳年老,乞令其子喃葛堅粲巴藏卜代。帝從之,封為輔教王,賜誥敕、金印、彩幣、袈裟、法器。以灌頂國師葛藏、右覺義桑加巴充正、副使往封。至四川,多雇牛馬,任載私物。禮官請治其罪,英宗方復辟,命收其敕書,減供應之半。

成化五年,王卒,命其子喃葛答刂失堅參叭藏卜嗣。六年申舊制,三年一貢,多不過百五十人,由四川雅州入。國師以下不許貢。弘治十二年,輔教等四王及長河西宣慰司並時入貢,使者至二千八百餘人。禮官以供費不貲,請敕四川守臣遵制遣送,違者卻還,從之。歷正德、嘉靖世,奉貢不絕。

西天阿難功德國,西方番國也。洪武七年,王卜哈魯遣其講主必尼西來朝,貢方物及解毒藥石。詔賜文綺、禪衣及布帛諸物。後不復至。

又有和林國師朵兒只怯烈失思巴藏卜,亦遣其講主汝奴汪叔來朝,獻銅佛、舍利、白哈丹布及元所授玉印一、玉圖書一、銀印四、銅印五、金字牌三,命宴賚遣還。明年,國師入朝,又獻佛像、舍利、馬二匹,賜文綺、禪衣。和林,即元太祖故都,在極北,非西番,其國師則番僧。與功德國同時來貢,後亦不復至。

尼八剌國,在諸藏之西,去中國絕遠。其王皆僧為之。洪武十七年,太祖命僧智光齎璽書、彩幣往,並使其鄰境地湧塔國。智光精釋典,負才辨,宣揚天子德意。其王馬達納羅摩遣使隨入朝,貢金塔、佛經及名馬方物。二十年達京師。帝喜,賜銀印、玉圖書、誥敕、符驗及幡幢、彩幣。二十三年再貢,加賜玉圖書、紅羅傘。終太祖時,數歲一貢。成祖復命智光使其國。永樂七年遣使來貢。十一年命楊三保齎璽書、銀幣賜其嗣王沙的新葛及地湧塔王可般。明年遣使來貢。封沙的新葛為尼八剌國王,賜誥及鍍金銀印。十六年遣使來貢,命中官鄧誠齎璽書、錦綺、紗羅往報之。所經罕東、靈藏、必力工瓦、烏斯藏及野藍卜納,皆有賜。宣德二年又遣中官侯顯賜其王絨錦、糸寧絲,地湧塔王如之。自後,貢使不復至。

又有速睹嵩者,亦西方之國。永樂三年遣行人連迪等齎敕往招,賜銀鈔、彩幣。其酋以道遠不至。

朵甘,在四川徼外,南與烏斯藏鄰,唐吐蕃地。元置宣慰司、招討司、元帥府、萬戶府,分統其眾。

洪武二年,太祖定陜西,即遣官齎詔招撫。又遣員外郎許允德諭其酋長,舉元故官赴京。攝帝師喃加巴藏卜及故國公南哥思丹八亦監藏等於六年春入朝,上所舉六十人名。帝喜,置指揮使司二,曰朵甘,曰烏斯藏,宣慰司二,元帥府一,招討司四,萬戶府十三,千戶所四,即以所舉官任之。廷臣言來朝者授職,不來者宜弗予。帝曰:「吾以誠心待人。彼不誠,曲在彼矣。萬里來朝,俟其再請,豈不負遠人歸嚮之心。」遂皆授之。降詔曰:「我國家受天明命,統御萬方,恩撫善良,武威不服。凡在幅員之內,咸推一視之仁。乃者攝帝師喃加巴藏卜率所舉故國公、司徒、宣慰、招討、元帥、萬戶諸人,自遠入朝。朕嘉其識天命,不勞師旅,共效職方之貢。已授國師及故國公等為指揮同知等官,皆給誥印。自今為官者務遵朝廷法,撫安一方。僧務敦化導之誠,率民為善,共享太平,永綏福祉,豈不休哉。」並宴賚遣還。初,元尊番僧為帝師,授其徒國公等秩,故降者襲舊號。

鎖南兀即爾者歸朝,授朵甘衛指揮僉事。以元司徒銀印來上,命進指揮同知。已而朵甘宣慰賞竹監藏舉首領可為指揮、宣慰、萬戶、千戶者二十二人。詔從其請,鑄分司印予之。乃改朵甘、烏斯藏二衛為行都指揮使司,以鎖南兀即爾為朵甘都指揮同知,管招兀即爾為烏斯藏都指揮同知,並賜銀印。又設西安行都指揮使司於河州,兼轄二都司。已,佛寶國師鎖南兀即爾等遣使來朝,奏舉故官賞竹監藏等五十六人。命增置朵甘思宣慰司及招討等司。招討司六:曰朵甘思,曰朵甘隴答,曰朵甘丹,曰朵甘倉溏,曰朵甘川,曰磨兒勘。萬戶府四:曰沙兒可,曰乃竹,曰羅思端,曰列思麻。千戶所十七。以賞竹監藏為朵甘都指揮同知,餘授職有差。自是,諸番修貢惟謹。

八年置俄力思軍民元帥府。尋置隴答衛指揮使司。十八年以班竹兒藏卜為烏斯藏都指揮使。乃更定品秩,自都指揮以下皆令世襲。未幾,又改烏斯藏俺不羅衛為行都指揮使司。二十六年,西番思曩日等族遣使貢馬,命賜金銅信符、文綺、襲衣,許之朝貢。

永樂元年改必里千戶所為衛,後置烏斯藏牛兒宗寨行都指揮使司,又置上邛部衛,皆以番人官之。十八年,帝以西番悉入職方,其最遠白勒等百餘寨猶未歸附,遣使往招,亦多入貢。帝以番俗惟僧言是聽,乃寵以國師諸美號,賜誥印,令歲朝。由是諸番僧來者日多,迄宣德朝,禮之益厚。九年命中官宋成等齎璽書、賜物使其地,敕都督趙安率兵送之畢力術江。

正統初,以供費不貲,稍為裁損。時有番長移書松潘守將趙得,言欲入朝,為生番阻遏,乞遣兵開道。詔令得遣使招生番,相率朝貢者八百二十九寨,翻賜賚遣歸。天順四年,四川三司言:「比奉敕書,番僧朝貢入京者不得過十人,餘留境上候賞。今蜀地災傷,若悉留之,動經數月,有司困於供億。宜如正統間制,宴待遣還。」報可。

成化三年,阿昔洞諸族土官言:「西番大小二姓為惡,殺之不懼。惟國師、剌麻勸化,則革心信服。」乃進禪師遠丹藏卜為國師,都綱子瑺為禪師,以化導之。六年,申諸番三歲一貢之例,國師以下不許貢,於是貢使漸希。

初,太祖以西番地廣,人獷悍,欲分其勢而殺其力,使不為邊患,故來者輒授官。又以其地皆食肉,倚中國茶為命,故設茶課司於天全六番,令以馬市,而入貢者又優以茶布。諸番戀貢市之利,且欲保世官,不敢為變。迨成祖,益封法王及大國師、西天佛子等,俾轉相化導,以共尊中國,以故西陲宴然,終明世無番寇之患。

長河西魚通寧遠宣慰司,在四川徼外,地通烏斯藏,唐為吐蕃。元時置碉門、魚通、黎、雅、長河西、寧遠六安撫司,隸吐蕃宣慰司。

洪武時,其地打煎爐、長河西土官元右丞剌瓦蒙遣其理問高惟善來朝,貢方物,宴賚遣還。十六年復遣惟善及從子萬戶若剌來貢。命置長河西等處軍民安撫司,以剌瓦蒙為安撫使,賜文綺四十八匹,鈔二百錠,授惟善禮部主事。二十年遣惟善招撫長河西、魚通、寧遠諸處,明年還朝,言:安邊之道,在治屯守,而兼恩威。屯守既堅,雖遠而有功;恩威未備,雖近而無益。今魚通、九枝疆土及巖州、雜道二長官司,東鄰碉門、黎、雅,西接長河西。自唐時吐蕃彊盛,寧遠、安靖、巖州漢民,往往為彼驅入九枝、魚通,防守漢邊。元初設二萬戶府,仍與盤陀、仁陽置立寨柵,邊民戍守。其後各枝率眾攻仁陽等柵。及川蜀兵起,乘勢侵陵雅、邛、嘉等州。洪武十年始隨碉門土酋歸附。巖州、雜道二長官司自國朝設,迨今十有餘年,官民仍舊不相統攝。蓋無統制之司,恣其猖獗,因襲舊弊故也。其近而已附者如此,遠而未附者何由而臣服之。且巖州、寧遠等處,乃古之州治。茍撥兵戍守,就築城堡,開墾山田,使近者向化而先附,遠者畏威而來歸,西域無事則供我徭役,有事則使之先驅。撫之既久,則皆為我用。如臣之說,其便有六。

通烏斯藏、朵甘,鎮撫長河西,可拓地四百餘里,得番民二千餘戶。非惟黎、雅保障,蜀亦永無西顧憂。一也。

番民所處老思岡之地,土瘠人繁,專務貿販碉門烏茶、蜀之細布,博易羌貨,以贍其生。若於巖州立市,則此輩衣食皆仰給於我,焉敢為非。二也。

以長河西、伯思東、巴獵等八千戶為外番掎角,其勢必固。然後招徠遠者,如其不來,使八千戶近為內應,遠為鄉導,此所謂以蠻攻蠻,誠制邊之善道。三也。

天全六番招討司八鄉之民,宜翻蠲其徭役,專令蒸造烏茶,運至巖州,置倉收貯,以易番馬。比之雅州易馬,其利倍之。且於打煎爐原易馬處相去甚近,而價增於彼,則番民如蟻之慕亶,歸市必眾。四也。

巖州既立倉易馬,則番民運茶出境,倍收其稅,其餘物貨至者必多。又魚通、九枝蠻民所種不陸之田,遞年無征。若令歲輸租米,並令軍士開墾大渡河兩岸荒田,亦可供給戍守官軍。五也。

碉門至巖州道路,宜令繕修開拓,以便往來人馬。仍量地里遠近,均立郵傳,與黎、雅烽火相應。庶可以防遏亂略,邊境無虞。六也。」帝從之。

後建昌酋月魯帖木兒叛,長河西諸酋陰附之,失朝貢,太祖怒。三十年春謂禮部臣曰:「今天下一統,四方萬國皆以時奉貢。如烏斯藏、尼八剌國其地極遠,猶三歲一朝。惟打煎爐長河西土酋外附月魯帖木兒、賈哈剌,不臣中國。興師討之,鋒刃之下,死者必眾。宜遣人諭其酋。若聽命來覲,一以恩待,不悛則發兵三十萬,聲罪徂征。」禮官以帝意為文馳諭之。其酋懼,即遣使入貢謝罪。天子赦之,為置長河西魚通寧遠宣慰司,以其酋為宣慰使,自是修貢不絕。初,魚通及寧遠、長河西,本各為部,至是始合為一。

永樂十三年,貢使言:「西番無他土產,惟以馬易茶。近年禁約,生理實艱,乞仍許開中。」從之。二十一年,宣慰使喃哩等二十四人來朝貢馬。正統二年,喃哩卒,子加八僧嗣。成化四年申諸番三歲一貢之令,惟長河西仍比歲一貢。六年頒定二年或三年一貢之例,貢使不得過百人。十七年,禮官言:「烏斯藏在長河西之西,長河西在松潘、越巂之南,壤地相接,易於混淆。烏斯藏諸番王例三歲一貢,彼以道險來少,而長河西番僧往往詐為諸王文牒,入貢冒賞。請給諸番王及長河西、董卜韓胡敕書勘合,邊臣審驗,方許進入,庶免詐偽之弊。或道阻,不許補貢。」從之。十九年,其部內灌頂國師遣僧徒來貢至千八百人,守臣劾其違制。詔止納五百人,餘悉遣還。二十二年,禮官言:「長河西以黎州大渡河寇發,連歲失貢,至是補進三貢。定制,道梗者不得再補。但今貢物已至,宜順其情納之,而量減賜賚。」報可。

弘治十二年,禮官言:「長河西及烏期藏諸番,一時並貢,使者至二千八百餘人。乞諭守臣無濫送。」亦報可。然其後來者愈多,卒不能卻。嘉靖三年定令不得過一千人。隆慶三年定五百人全賞、遣八人赴京之制,如闡教諸王。其貢物則珊瑚、氆氌之屬,悉準《闡化王傳》所載。諸番貢皆如之。

董卜韓胡宣慰司,在四川威州之西,其南與天全六番接。永樂九年,酋長南葛遣使奉表入朝,貢方物。因言答隆蒙、碉門二招討侵掠鄰境,阻遏道路,請討之。帝不欲用兵,降敕慰諭,使比年一貢,賜金印、冠帶。

正統三年奏年老,乞以子克羅俄堅粲代,從之。兇狡不循禮法。七年乞封王,賜金印,帝不許。命進秩鎮國將軍、都指揮同知,掌宣慰司事,給之誥命。益恃強,數與雜谷安撫及別思寨安撫饒蛒構怨。十年八月移牒四川守臣,謂:「別思寨本父南葛故地,分畀饒蛒父者。後饒蛒受事,私奏於朝,獲設安撫司。邇乃偽為宣慰司印,自稱宣慰使,糾合雜谷諸番,將侵噬己地。已拘執饒蛒,追出偽印,用番俗法剜去兩目。謹以狀聞。」守臣上其事。帝遣使齎敕責其專擅,令與使臣推擇饒蛒族人為安撫,仍轄其土地,且送還饒蛒,養之終身。

十三年十月,四川巡按張洪等奏:「近接董卜宣慰文牒言:『雜谷故安撫阿票小妻毒殺其夫及子,又賄威州千戶唐泰誣己謀叛。今備物進貢,欲從銅門山西開山通道,乞官軍於日駐迓之。』臣等竊以雜谷內聯威州、保縣,外鄰董卜韓胡。雜谷力弱,欲抗董卜,實倚重於威、保。董卜勢強,欲通威、保,卻受阻於雜谷。以此仇殺,素不相能。銅門及日駐諸寨,乃雜谷、威、保要害地。董卜欺雜谷妻寡子弱,瞰我軍遠征麓川,假進貢之名,欲別開道路,意在吞滅雜谷,手冓陷唐泰。所請不可許。」乃下都御史寇深等計度,其議迄不行。

時董卜比歲入貢,所遣僧徒強悍不法,多攜私物,強索舟車,騷擾道途,詈辱長吏。天子聞而惡之,景泰元年賜敕切責。尋侵奪雜谷及達思蠻長官司地,掠其人畜,守臣不能制。三年二月朝議獎其入貢勤誠,進秩都指揮使,令還二司侵地及所掠人民。其酋即奉命,惟舊維州之地尚為所據。俄饋四川巡撫李匡銀DN、金珀,求《御製大誥》、《周易》、《尚書》、《毛詩》、《小學》、《方輿勝覽》、《成都記》諸書。匡聞之於朝,因言:「唐時吐蕃求《毛詩》、《春秋》。于休烈謂,予之以書,使知權謀,愈生變詐,非中國之利。裴光廷謂,吐蕃久叛新服,因其有表,賜以《詩》、《書》,俾漸陶聲教,化流無外。休烈徒知書有權略變詐,不知忠信禮義皆從書出。明皇從之。今茲所求,臣以為予之便。不然彼因貢使市之書肆,甚不為難。惟《方輿勝覽》、《成都記》,形勝關塞所具,不可概予。」帝如其言。尋以其還侵地,賜敕獎勵。

六年,兵部尚書于謙等奏其僭稱蠻王,窺伺巴,蜀,所上奏章語多不遜,且招集群番,大治戎器,悖逆日彰,不可不慮,宜敕守臣預為戒備,從之。

克羅俄堅粲死,子答刂思堅粲藏卜遣使來貢,命為都指揮同知,掌宣慰司事。天順元年遣使入貢,乞封王。命如其父官,進秩都指揮使,仍掌宣慰司事。

成化五年,四川三司奏:「保縣僻處極邊,永樂五年特設雜谷安撫司,令撫輯舊維州諸處蠻塞。後與董卜構兵,維州諸地俱為侵奪,貢道阻絕。今雜谷恢復故疆,將遣使來貢,不知貢期,未敢擅遣。」帝從禮官言,許以三年為期。四年申諸番三年一貢之例,惟董卜許比年一貢。

六年,答刂巴堅粲藏卜卒,子綽吾結言千嗣為都指揮使。弘治三年卒,子日墨答刂思巴旺丹巴藏卜遣國師貢珊瑚樹、氆氌、甲胄諸物,請嗣父職,許之,賜誥命、敕書、彩幣。九年卒,子喃呆請襲,亦遣國師貢方物,詔授以父官。卒,子容中短竹襲。嘉靖二年再定令貢使不得過千人,其所隸別思寨及加渴瓦寺別貢。隆慶二年,董卜及別思寨貢使多至千七百餘人,命予半賞,遣八人赴京,為定制。迄萬曆後,朝貢不替。


\end{pinyinscope}