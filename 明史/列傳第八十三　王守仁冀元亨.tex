\article{列傳第八十三 王守仁冀元亨}

\begin{pinyinscope}
王守仁,字伯安,餘姚人。父華,字德輝,成化十七年進士第一。授修撰。弘治中,累官學士、少詹事。華有器度,在講幄最久,孝宗甚眷之。李廣貴幸,華講《大學衍義》,至唐李輔國與張后表裏用事,指陳甚切。帝命中官賜食勞焉。正德初,進禮部左侍郎。以守仁忤劉瑾,出為南京吏部尚書,坐事罷。旋以《會典》小誤,降右侍郎。瑾敗,乃復故,無何卒。華性孝,母岑年踰百歲卒。華已年七十餘,猶寢苫蔬食,士論多之。

守仁娠十四月而生。祖母夢神人自雲中送兒下,因名雲。五歲不能言,異人拊之,更名守仁,乃言。年十五無又不可以訓,故信必及有。老莊未免於有。恆訓有所不足。」,訪客居庸、山海關。時闌出塞,縱觀山川形勝。弱冠舉鄉試,學大進。顧益好言兵,且善射。登弘治十二年進士。使治前威寧伯王越葬,還而朝議方急西北邊,守仁條八事上之。尋授刑部主事。決囚江北,引疾歸。起補兵部主事。正德元年冬,劉瑾逮南京給事中御史戴銑等二十餘人。守仁抗章救,瑾怒,廷杖四十,謫貴州龍場驛丞。龍場萬山叢薄,苗、僚雜居。守仁因俗化導,夷人喜,相率伐木為屋,以棲守仁。瑾誅,量移廬陵知縣。入覲,遷南京刑部主事,吏部尚書楊一清改之驗封。屢遷考功郎中,擢南京太僕少卿,就遷鴻臚卿。

兵部尚書王瓊素奇守仁才。十一年八月擢右僉都御史,巡撫南、贛。當是時,南中盜賊蜂起。謝志山據橫水、左溪、桶岡,池仲容據浰頭,皆稱王,與大庾陳曰能、樂昌高快馬、郴州龔福全等攻剽府縣。而福建大帽山賊詹師富等又起。前巡撫文森托疾避去。志山合樂昌賊掠大庾,攻南康、贛州,贛縣主簿吳玭戰死。守仁至,知左右多賊耳目,乃呼老黠隸詰之。隸戰栗不敢隱,因貰其罪,令填賊,賊動靜無勿知。於是檄福建、廣東會兵,先討大帽山賊。明年正月,督副使楊璋等破賊長富村,逼之象湖山,指揮覃桓、縣丞紀鏞戰死。守仁親率銳卒屯上杭。佯退師,出不意搗之,連破四十餘寨,俘斬七千有奇,指揮王鎧等擒師富。疏言權輕,無以令將士,請給旗牌,提督軍務,得便宜從事。尚書王瓊奏從其請。乃更兵制:二十五人為伍,伍有小甲;二伍為隊,隊有總甲;四隊為哨,哨有長,協哨二佐之;二哨為營,營有官,參謀二佐之;三營為陣,陣有偏將;二陣為軍,軍有副將。皆臨事委,不命於朝;副將以下,得遞相罰治。

其年七月進兵大庾。志山乘間急攻南安,知府季斅擊敗之。副使楊璋等亦生縶曰能以歸。遂議討橫水、左溪。十月,都指揮許清、贛州知府邢珣、寧都知縣王天與各一軍會橫水,斅及守備郟文、汀州知府唐淳、縣丞舒富各一軍會左溪,吉安知府伍文定、程鄉知縣張戩遏其奔軼。守仁自駐南康有系統的學術史專著。今通行《萬有文庫》本。,去橫水三十里,先遣四百人伏賊巢左右,進軍逼之。賊方迎戰,兩山舉幟。賊大驚,謂官軍已盡犁其巢,遂潰。乘勝克橫水,志山及其黨蕭貴模等皆走桶岡。左溪亦破。守仁以桶岡險固,移營近地,諭以禍福。賊首藍廷鳳等方震恐,見使至大喜,期仲冬朔降,而珣、文定已冒雨奪險入。賊阻水陣,珣直前搏戰,文定與戩自右出,賊倉卒敗走,遇淳兵又敗。諸軍破桶岡,志山、貴模、廷鳳面縛降。凡破巢八十有四,俘斬六千有奇。時湖廣巡撫秦金亦破福全。其黨千人突至,諸將擒斬之。乃設崇義縣於橫水,控諸瑤。

還至贛州,議討浰頭賊。初,守仁之平師富也,龍川賊盧珂、鄭志高、陳英咸請降。及征橫水,浰頭賊將黃金巢亦以五百人降,獨仲容未下。橫水破,仲容始遣弟仲安來歸,而嚴為戰守備。詭言:「珂、志高,仇也,將襲我,故為備。」守仁佯杖繫珂等,而陰使珂弟集兵待,遂下令散兵。歲首大張燈樂,仲容信且疑。守仁賜以節物,誘入謝。仲容率九十三人營教場,而自以數人入謁。守仁呵之曰:「若皆吾民,屯於外,疑我乎?」悉引入祥符宮,厚飲食之。賊大喜過望,益自安。守仁留仲容觀燈樂。正月三日大享,伏甲士於門,諸賊入,以次悉擒戮之。自將抵賊巢,連破上、中、下三浰,斬馘二千有奇。餘賊奔九連山。山橫亙數百里,陡絕不可攻。乃簡壯士七百人衣賊衣,奔崖下,賊招之上。官軍進攻,內外合擊,擒斬無遺。乃於下浰立和平縣,置戍而歸。自是境內大定。

初,朝議賊勢強,發廣東、湖廣兵合剿。守仁上疏止之,不及。桶岡既滅,湖廣兵始至。及平浰頭縉紳先生春秋時鄒魯之士別稱。縉,插;紳,大帶。縉,廣東尚未承檄。守仁所將皆文吏及偏裨小校,平數十年巨寇,遠近驚為神。進右副都御史,予世襲錦衣衛百戶,再進副千戶。

十四年六月,命勘福建叛軍。行至豐城而寧王宸濠反,知縣顧佖以告。守仁急趨吉安,與伍文定徵調兵食,治器械舟楫,傳檄暴宸濠罪,俾守令各率吏士勤王。都御史王懋中,編修鄒守益,副使羅循、羅欽德,郎中曾直,御史張鰲山、周魯,評事羅僑,同知郭祥鵬,進士郭持平,降謫驛丞王思、李中,咸赴守仁軍。御史謝源、伍希儒自廣東還,守仁留之紀功。因集眾議曰:「賊若出長江順流東下,則南都不可保。吾欲以計撓之,少遲旬日無患矣。」乃多遣間諜,檄府縣言:「都督許泰、郤永將邊兵,都督劉暉、桂勇將京兵,各四萬,水陸並進。南贛王守仁、湖廣秦金、兩廣楊旦各率所部合十六萬,直搗南昌,所至有司缺供者,以軍法論。」又為蠟書遺偽相李士實、劉養正,敘其歸國之誠,令從臾早發兵東下,而縱諜洩之。宸濠果疑。與士實、養正謀,則皆勸之疾趨南京即大位,宸濠益大疑。十餘日詗知中外兵不至,乃悟守仁紿之。七月壬辰朔,留宜春王拱嵒居守,而劫其眾六萬人,襲下九江、南康,出大江,薄安慶。守仁聞南昌兵少則大喜,趨樟樹鎮。知府臨江戴德孺、袁州徐璉、贛州邢珣,都指揮餘恩,通判瑞州胡堯元、童琦、撫州鄒琥、安吉談儲,推官王、徐文英,知縣新淦李美、泰和李楫、萬安王冕、寧都王天與,各以兵來會,合八萬人,號三十萬。或請救安慶,守仁曰:「不然。今九江、南康已為賊守,我越南昌與相持江上,二郡兵絕我後,是腹背受敵也。不如直搗南昌。賊精銳悉出,守備虛。我軍新集氣銳,攻必破。賊聞南昌破,必解圍自救。逆擊之湖中,蔑不勝矣。」眾曰「善」。己酉次豐城,以文定為前鋒,選遣奉新知縣劉守緒襲其伏兵。庚戌夜半,文定兵抵廣潤門,守兵駭散。辛亥黎明,諸軍梯糸亙登,縛拱嵒等,宮人多焚死。軍士頗殺掠,守仁戮犯令者十餘人,宥脅從,安士民,慰諭宗室,人心乃悅。

居二日,遣文定、珣、璉、德孺各將精兵分道進,而使堯元等設伏。宸濠果自安慶還兵。乙卯遇於黃家渡。文定當其前鋒,賊趨利。珣繞出賊背貫其中,文定、恩乘之,璉、德孺張兩翼分賊勢,堯元等伏發,賊大潰,退保八字腦。宸濠懼,盡發南康、九江兵。守仁遣知府撫州陳槐、饒州林城取九江,建昌曾璵、廣信周朝佐取南康。丙辰復戰,官軍卻,守仁斬先卻者。諸軍殊死戰,賊復大敗。退保樵舍,聯舟為方陣,盡出金寶犒士。明日,宸濠方晨朝其群臣,官軍奄至。以小舟載薪,乘風縱火,焚其副舟,妃婁氏以下皆投水死。宸濠舟膠淺,倉卒易舟遁,王冕所部兵追執之。士實、養正及降賊按察使楊璋等皆就擒。南康、九江亦下。凡三十五日而賊平。京師聞變,諸大臣震懼。王瓊大言曰:「王伯安居南昌上游,必擒賊。」至是,果奏捷。

帝時已親征,自稱「威武大將軍」,率京邊驍卒數萬南下。命安邊伯許泰為副將軍,偕提督軍務太監張忠、平賊將軍左都督劉暉將京軍數千,溯江而上顥講以「仁」為理,對陸王心學有影響;程頤講萬物歸於一,抵南昌。諸嬖倖故與宸濠通,守仁初上宸濠反書,因言:「覬覦者非特一寧王,請黜奸諛以回天下豪傑心。」諸嬖倖皆恨。宸濠既平,則相與媢功。且懼守仁見天子發其罪,競為蜚語,謂守仁先與通謀,慮事不成,乃起兵。又欲令縱宸濠湖中,待帝自擒。守仁乘忠、泰未至,先俘宸濠,發南昌。忠、泰以威武大將軍檄邀之廣信。守仁不與,間道趨玉山,上書請獻俘,止帝南征。帝不許。至錢唐遇太監張永。永提督贊畫機密軍務,在忠、泰輩上,而故與楊一清善,除劉瑾,天下稱之。守仁夜見永,頌其賢,因極言江西困敝,不堪六師擾。永深然之,曰:「永此來,為調護聖躬,非邀功也。公大勛,永知之,但事不可直情耳。」守仁乃以宸濠付永,而身至京口,欲朝行在。聞巡撫江西命,乃還南昌。忠、泰已先至,恨失宸濠。故縱京軍犯守仁,或呼名嫚罵。守仁不為動,撫之愈厚。病予藥,死予棺,遭喪於道,必停車慰問良久始去。京軍謂「王都堂愛我」,無復犯者。忠、泰言:「寧府富厚甲天下,今所蓄安在?」守仁曰:「宸濠異時盡以輸京師要人,約內應,籍可按也。」忠、泰故嘗納宸濠賄者,氣懾不敢復言。已,輕守仁文士,強之射。徐起,三發三中。京軍皆歡呼,忠、泰益沮。會冬至,守仁命居民巷祭,已,上塚哭。時新喪亂,悲號震野。京軍離家久,聞之無不泣下思歸者。忠、泰不得已班師。比見帝,與紀功給事中祝續、御史章綸讒毀百端,獨永時時左右之。忠揚言帝前曰:「守仁必反,試召之,必不至。」忠、泰屢矯旨召守仁。守仁得永密信,不赴。及是知出帝意,立馳至。忠、泰計沮,不令見帝。守仁乃入九華山,日晏坐僧寺。帝覘知之,曰:「王守仁學道人,聞召即至,何謂反?」乃遣還鎮,令更上捷音。守仁乃易前奏,言「奉威武大將軍方略討平叛亂」,而盡入諸嬖倖名,江彬等乃無言。

當是時,讒邪構煽,禍變叵測,微守仁,東南事幾殆。世宗深知之。甫即位,趣召入朝受封。而大學士楊廷和與王瓊不相能。守仁前後平賊,率歸功瓊,廷和不喜,大臣亦多忌其功。會有言國哀未畢,不宜舉宴行賞者,因拜守仁南京兵部尚書。守仁不赴,請歸省。已,論功封特進光祿大夫、柱國、新建伯,世襲,歲祿一千石。然不予鐵券,歲祿亦不給。諸同事有功者,惟吉安守伍文定至大官,當上賞。其他皆名示遷,而陰絀之,廢斥無存者。守仁憤甚。時已丁父憂,屢疏辭爵,乞錄諸臣功,咸報寢。免喪,亦不召。久之,所善席書及門人方獻夫、黃綰以議禮得幸,言於張璁、桂萼,將召用,而費宏故銜守仁,復沮之。屢推兵部尚書,三邊總督,提督團營,皆弗果用。

嘉靖六年,思恩、田州土酋盧蘇、王受反。總督姚鏌不能定,乃詔守仁以原官兼左都御史,總督兩廣兼巡撫。綰因上書訟守仁功,請賜鐵券、歲祿性的、獨立的精神實體的存在,並由此確立了上帝和物質實,並敘討賊諸臣,帝咸報可。守仁在道,疏陳用兵之非,且言:「思恩未設流官,土酋歲出兵三千,聽官徵調。既設流官,我反歲遣兵數千防戍。是流官之設,無益可知。且田州鄰交阯,深山絕谷,悉瑤、僮盤據,必仍設土官,斯可藉其兵力為屏蔽。若改土為流,則邊鄙之患,我自當之,後必有悔。」章下兵部,尚書王時中條其不合者五,帝令守仁更議。十二月,守仁抵潯州,會巡按御史石金定計招撫。悉散遣諸軍,留永順、保靖土兵數千,解甲休息。蘇、受初求撫不得,聞守仁至益懼,至是則大喜。守仁赴南寧,二人遣使乞降,守仁令詣軍門。二人竊議曰:「王公素多詐,恐紿我。」陳兵入見。守仁數二人罪,杖而釋之。親入營,撫其眾七萬。奏聞於朝,陳用兵十害,招撫十善。因請復設流官,量割田州地,別立一州,以岑猛次子邦相為吏目,署州事,俟有功擢知州。而於田州置十九巡檢司,以蘇、受等任之,並受約束於流官知府。帝皆從之。斷藤峽瑤賊,上連八寨,下通仙臺、花相諸洞蠻,盤亙三百餘里,郡邑罹害者數十年。守仁欲討之,故留南寧。罷湖廣兵,示不再用。伺賊不備,進破牛腸、六寺等十餘寨,峽賊悉平。遂循橫石江而下,攻克仙臺、花相、白竹、古陶、羅鳳諸賊。令布政使林富率蘇、受兵直抵八寨,破石門,副將沈希儀邀斬軼賊,盡平八寨。

始,帝以蘇、受之撫,遣行人奉璽書獎諭。及奏斷藤峽捷,則以手詔問閣臣楊一清等,謂守仁自誇大,且及其生平學術。一清等不知所對。守仁之起由璁、萼薦,萼故不善守仁,以璁強之。後萼長吏部,璁入內閣,積不相下。萼暴貴喜功名,風守仁取交阯,守仁辭不應。一清雅知守仁,而黃綰嘗上疏欲令守仁入輔,毀一清,一清亦不能無移憾。萼遂顯詆守仁征撫交失,賞格不行。獻夫及霍韜不平,上疏爭之,言:「諸瑤為患積年,初嘗用兵數十萬,僅得一田州,旋復召寇。守仁片言馳諭,思、田稽首。至八寨、斷藤峽賊,阻深巖絕岡,國初以來未有輕議剿者,今一舉蕩平,若拉枯朽。議者乃言守仁受命征思、田,不受命徵八寨。夫大夫出疆,有可以安國家,利社稷,專之可也,況守仁固承詔得便宜從事者乎?守仁討平叛籓,忌者誣以初同賊謀,又誣其輦載金帛。當時大臣楊廷和、喬宇飾成其事,至今未白。夫忠如守仁,有功如守仁,一屈於江西,再屈於兩廣。臣恐勞臣灰心,將士解體,後此疆圉有事,誰復為陛下任之!」帝報聞而已。

守仁已病甚,疏乞骸骨,舉鄖陽巡撫林富自代,不俟命竟歸。行至南安卒,年五十七。喪過江西「,軍民無不縞素哭送者。

守仁天姿異敏。年十七謁上饒婁諒,與論朱子格物大指。還家,日端坐,講讀《五經》,不茍言笑。游九華歸,築室陽明洞中。泛濫二氏學,數年無所得。謫龍場,窮荒無書,日繹舊聞。忽悟格物致知,當自求諸心,不當求諸事物,喟然曰:「道在是矣。」遂篤信不疑。其為教,專以致良知為主。謂宋周、程二子後,惟象山陸氏簡易直捷,有以接孟氏之傳。而朱子《集註》、《或問》之類,乃中年未定之說。學者翕然從之,世遂有「陽明學」云。

守仁既卒,桂萼奏其擅離職守。帝大怒,下廷臣議。萼等言:「守仁事不師古,言不稱師。欲立異以為高,則非硃熹格物致知之論;知眾論之不予,則為硃熹晚年定論之書。號召門徒,互相倡和。才美者樂其任意,庸鄙者借其虛聲。傳習轉訛,背謬彌甚。但討捕CP賊,擒獲叛籓,功有足錄,宜免追奪伯爵以章大信,禁邪說以正人心。」帝乃下詔停世襲,恤典俱不行。

隆慶初,廷臣多頌其功。詔贈新建侯,謚文成。二年予世襲伯爵。既又有請以守仁與薛瑄、陳獻章同從祀文廟者。帝獨允禮臣議,以瑄配。及萬曆十二年,御史詹事講申前請。大學士申時行等言:「守仁言致知出《大學》,良知出《孟子》。陳獻章主靜,沿宋儒周敦頤、程顥。且孝友出處如獻章,氣節文章功業如守仁,不可謂禪,誠宜崇祀。」且言胡居仁純心篤行,眾論所歸,亦宜並祀。帝皆從之。終明之世,從祀者止守仁等四人。

始守仁無子,育弟子正憲為後。晚年,生子正億,二歲而孤。既長,襲錦衣副千戶。隆慶初反對「不自貴而貴物」,堅信亂世必將為太平之世所取代。參,襲新建伯。萬曆五年卒。子承勛嗣,督漕運二十年。子先進,無子,將以弟先達子業弘繼。先達妻曰:「伯無子,爵自傳吾夫。由父及子,爵安往?」先進怒,因育族子業洵為後。及承勛卒,先進未襲死。業洵自以非嫡嗣,終當歸爵先達,且虞其爭,乃謗先達為乞養,而別推承勛弟子先通當嗣,屢爭於朝,數十年不決。崇禎時,先達子業弘復與先通疏辨。而業洵兄業浩時為總督,所司懼忤業浩,竟以先通嗣。業弘憤,持疏入禁門訴。自刎不殊,執下獄,尋釋。先通襲伯四年,流賊陷京師,被殺。

守仁弟子盈天下,其有傳者不復載。惟冀元亨嘗與守仁共患難。

冀元亨,字惟乾,武陵人。篤信守仁學。舉正德十一年鄉試。從守仁於贛,守仁屬以教子。宸濠懷不軌,而外務名高,貽書守仁問學,守仁使元亨往。宸濠語挑之,佯不喻,獨與之論學,宸濠目為癡。他日講《西銘》,反覆君臣義甚悉。宸濠亦服,厚贈遣之,元亨反其贈於官。已,宸濠敗,張忠、許泰誣守仁與通。詰宸濠,言無有。忠等詰不已,曰:「獨嘗遣冀元亨論學。」忠等大喜,搒元亨,加以炮烙,終不承,械繫京師詔獄。

世宗嗣位,言者交白其冤,出獄五日卒。元亨在獄,善待諸囚若兄弟,囚皆感泣。其被逮也,所司繫其妻李,李無怖色,曰:「吾夫尊師樂善,豈他慮哉!」獄中與二女治麻枲不輟。事且白,守者欲出之。曰:「未見吾夫,出安往?」按察諸僚婦聞其賢,召之,辭不赴。已就見,則囚服見,手不釋麻枲。問其夫學,曰:「吾夫之學,不出閨門衽席間。」聞者悚然。

贊曰:王守仁始以直節著。比任疆事,提弱卒,從諸書生掃積年逋寇,平定孽籓。終明之世,文臣用兵制勝,未有如守仁者也。當危疑之際,神明愈定,智慮無遺,雖由天資高,其亦有得於中者歟。矜其創獲,標異儒先,卒為學者譏。守仁嘗謂胡世寧少講學,世寧曰:「某恨公多講學耳。」桂萼之議雖出於媢忌之私,抑流弊實然,固不能以功多為諱矣。


\end{pinyinscope}