\article{列傳第一百十五}

\begin{pinyinscope}
龐尚鵬宋儀望張岳李材陸樹德蕭廩賈三近李頤朱鴻謨蕭彥弟雍查鐸孫維城謝傑郭惟賢萬象春鐘化民吳達可

龐尚鵬,字少南,南海人。嘉靖三十二年進士。除江西樂平知縣。擢御史。偕給事中羅嘉賓出核南京、浙江軍餉,請罪參將戚繼光、張四維,而盡發胡宗憲失律、貪淫及軍興督撫侵軍需狀。還朝,出按河南。巡撫蔡汝楠欲會疏進白鹿,尚鵬不可。改按浙江。民苦徭役,為舉行一條鞭法。按治鄉官呂希周、嚴傑、茅坤、潘仲驂子弟僮奴,請奪希周等冠帶。詔盡黜為民。尚鵬介直無所倚。所至搏擊豪強,吏民震懾。已,督畿輔學政。隆慶元年,請帝時御便殿,延見大臣,恤建言得罪者馬從謙等。已,又申救給事中胡應嘉,論大學士郭樸無相臣體。擢大理右寺丞。

明年春,朝議興九邊屯、鹽。擢尚鵬右僉都御史,與副都御史鄒應龍、唐繼祿分理。尚鵬轄兩淮、長蘆、山東三運司,兼理畿輔、河南、山東、江北、遼東屯務。抵昌平,劾內侍張恩擅殺人,兩淮巡鹽孫以仁贓罪,皆獲譴。其秋,應龍等召還,命尚鵬兼領九邊屯務。疏列鹽政二十事,鹺利大興。乃自江北躬歷九邊,先後列上屯政便宜,江北者四,薊鎮者九,遼東、宣、大者各十一,寧夏者四,甘肅者七。奏輒報可。尚鵬權既重,自負經濟才,慷慨任事。諸御史督鹽政者以事權見奪,欲攻去之。河東巡鹽郜永春劾尚鵬行事乖違,吏部尚書楊博議留之。會中官惡博,激帝怒,譙讓,罷博而落尚鵬職,汰屯鹽都御史官。時三年十二月也。明年,復坐按浙時驗進宮幣不中程,斥為民。

神宗立,御史計坤亨等交薦,保定巡撫宋纁亦白其無罪。萬歷四年冬,始以故官撫福建。奏蠲逋餉銀,推行一條鞭法。劾罷總兵官胡守仁,屬吏咸奉職。張居正奪情,重譴言者。尚鵬移書救,居正深銜之。會拜左副都御史,居正令吏科陳三謨以給由歲月有誤劾之,遂罷去。家居四年卒。浙江、福建暨其鄉廣東皆以徭輕故德尚鵬,立祠祀。天啟中,賜謚惠敏。

宋儀望,字望之,吉安永豐人。嘉靖二十六年進士。授吳縣知縣。民輸白糧京師,輒破家。儀望令諸區各出公田,計役授田贍之。禁火葬,創子游祠,建書院,惠績甚著。征授御史。劾大將軍仇鸞挾寇自重,疏留中。已,陳時務十二策。巡鹽河東,請開桑乾河通宣、大餉道,言:「河發源金龍池下甕城驛古定橋,會眾水,東流千餘里,入盧溝橋。其間惟大同卜村有叢石,宣府黑龍灣石崖稍險,然不踰五十里,水淺者猶二三尺,疏鑿甚易。曩大同巡撫侯鉞嘗乘小艇赴懷來,歷卜村、黑龍灣,安行無虞。又自懷來溯流,載米三十石達之古定河,足利漕可徵。」時方行挖運,率三十石致一石。儀望疏至,下廷議。兵部尚書聶豹言:「河成便漕,兼制敵騎。」工部尚書歐陽必進言:「道遠役重。」遂報罷。

儀望尋省母歸。還朝,發胡宗憲、阮鶚奸貪狀,鶚被逮。二人皆嚴嵩私人,嵩由是不悅。及受命督三殿門工,嵩子世蕃私賈人金,屬必進俾與工事,儀望執不可。工竣,敘勞,擢大理右寺丞。世蕃以為德,儀望請急歸,無所謝,世蕃益怒。會災異考察京官,必進遷吏部,遂坐以浮躁,貶夷陵判官。嵩敗,擢霸州兵備僉事。請城涿州,除馬戶逋稅。進大名兵備副使,改福建。與總兵官戚繼光合兵破倭,因列海防善後事。詔從其請。隆慶二年,吏部尚書楊博欲黜儀望,考功郎劉一儒持之,乃鐫二秩,補四川僉事。四遷大理少卿。

萬曆二年,張居正當國,雅知儀望才,擢右僉都御史,巡撫應天諸府。奏減屬郡災賦。海警稍定,將吏諱言兵,儀望與副使王叔果修戰備。倭果至,御之黑水洋,斬獲多,進右副都御史。先有詔雪建文諸臣,儀望創表忠祠祀之南京。宋忠臣楊邦晙,儀望鄉人也,葬江寧,歲久漸湮,儀望為封其墓,載其祠祀典。故太常卿袁洪愈、祭酒姜寶皆不為居正所喜,儀望薦之朝,漸失居正意。四年,稍遷南京大理卿。踰年改北,被劾罷歸。

儀望少師聶豹,私淑王守仁,又從鄒守益、歐陽德、羅洪先遊。守仁從祀,儀望有力焉。家居數年卒。

張岳,字汝宗,餘姚人。嘉靖三十八年進士。授行人。擢禮科給事中。巡視內府庫藏,奏行釐弊八事。已,又陳時政,極言講學者以富貴功名鼓動士大夫,談虛論寂,靡然成風。又今吏治方清,獨兵部無振刷,推用總兵黃印、韓承慶等,非庸即狡。曹司條例淆亂無章,胥吏朋奸,搏噬將校,其咎必有所歸。時徐階當國,為講學會,而楊博在兵部,意蓋指二人也。博奏辨乞罷,帝慰留之。博自是惡嶽。及掌吏部,岳已遷工科左給事中,遂出為雲南參議。再遷河南參政。

萬曆初,張居正雅知岳,用為太僕少卿。再遷南京右僉都御史,督操江。甫到官,會居正父喪謀奪情,南京尚書潘晟及諸給事、御史,咸上疏請留居正。岳獨馳疏請令馳驛奔喪,居正大怒。會大計京官,給事中傅作舟等承風劾岳,貶一秩調外,嶽遂歸。久之,操江僉都御史呂藿、給事中吳綰知居正憾未釋,摭劾岳落職閒住。甫兩月,居正死,南京御史方萬山薦岳,劾作舟。作舟坐斥,起岳四川參議。旋擢右僉都御史,巡撫南、贛。入為左僉都御史,獻時政四議。其一言宗籓宜以世次遞殺,親盡則停,俾習四民之業。其一言治河之策,夏鎮固當開,沽頭亦不可廢。並報寢。進左副都御史,上疏評議廷臣賢否,為給事中袁國臣等所論。時已遷刑部右侍郎,坐罷歸。

李材,字孟誠,豐城人,尚書遂子也。舉嘉靖四十一年進士,授刑部主事。素從鄒守益講學。自以學未成,乞假歸。訪唐樞、王畿、錢德洪,與問難。隆慶中還朝。由兵部郎中稍遷廣東僉事。羅旁賊猖獗,材襲破之周高山,設屯以守。賊有三巢在新會境,調副總兵梁守愚由恩平,游擊王瑞由德慶入,身出肇慶中道,夜半斬賊五百級,毀廬舍千餘,空其地,募人田之。亡何,倭五千攻陷電白,大掠而去。材追破之石城,設伏海口,伺其遁而殲之,奪還婦女三千餘。會奸人引倭自黃山間道潰而東。材聲言大軍數道至以疑賊,而返故道迎擊,盡殺之。又追襲雷州倭至英利,皆遁去,降賊渠許恩於陽江。錄功,進副使。

萬曆初,張居正柄國,不悅材,遂引疾去。居正卒,起官山東。以才調遼東開原。尋遷雲南洱海參政,進按察使,備兵金騰。金騰地接緬甸,而孟養、蠻莫兩土司介其間,叛服不常。緬部目曰大曩長、曰散奪者,率數千人據其地。材謂不收兩土司無以制緬,遣人招兩土司來歸,而間討抗命夷阿坡。居頃之,緬遣兵爭蠻莫,材合兩土司兵敗緬眾,殺大曩長,逐散奪去。緬帥莽應裏益兵至孟養,復擊沈其舟,斬其將一人,乃退。有猛密者,地在緬境,數為緬侵奪,舉族內徙,有司居之戶碗。至是,緬勢稍屈,材資遣還故土。亡何,緬人驅象陣大舉復仇,兩土司告急。材遣游擊劉天俸率把總寇崇德等出威緬,渡金沙江,與孟養兵會遮浪,迎擊之。賊大敗,生擒繡衣賊將三人。巡撫劉世曾、總兵官沐昌祚以大捷聞,詔令覆勘。未上,而材擢右僉都御史,撫治鄖陽。

材好講學,遣部卒供生徒役,卒多怨。又徇諸生請,改參將公署為學宮。參將米萬春諷門卒梅林等大噪,馳入城,縱囚毀諸生廬,直趨軍門,挾賞銀四千,洶洶不解。居二日,萬春脅材更軍中不便十二事,令上疏歸罪副使丁惟寧、知府沈鈇等,材隱忍從之。惟寧責數萬春,萬春欲殺惟寧,跳而免,材遂復劾惟寧激變。詔下鈇等吏,貶惟寧三官,材還籍候勘。時十五年十一月也。御史楊紹程勘萬春首亂,宜罪。大學士申時行庇之,置不問,旋調天津善地去。而材又以雲南事被訐,遂獲重譴。初,有詔勘征緬功,巡按御史蘇贊阜言斬馘不及千,破城拓地皆無驗,猛密地尚為緬據,材、天俸等虛張功伐,副使陳嚴之與相附和,宜並罪。帝怒,削世曾籍,奪昌祚祿一年,材、嚴之、天俸俱逮下詔獄。刑部尚書李世達、左都御史吳時來、大理少卿李棟等,當材、天俸徒,嚴之鐫秩。帝不懌,奪郎中、御史、寺正諸臣俸,典詔獄李登雲等亦解官。於是改擬遣戍。特旨引紅牌說謊例,坐材、天俸斬,嚴之除名。大學士時行等數為解,給事中唐堯欽等亦言:「材以夷攻夷,功不可泯。奏報偶虛,坐以死,假令盡虛無實,掩罪為功,何以罪之?設不幸失城池,全軍不返,又何以罪之?」帝皆不聽。幽繫五年,論救者五十餘疏。已,天俸以善用火器,釋令立功,時行等復為材申理,皆不省。

亡何,孟養使入貢,具言緬人侵軼,天朝救援,破敵有狀,聞典兵者在獄,眾皆流涕。而楚雄士民閻世祥等亦相率詣闕訟冤。帝意乃稍解,命再勘。勘至,材罪不掩功。大學士王錫爵等再疏為言,帝故遲之,至二十一年四月,始命戍鎮海衛。

材所至,輒聚徒講學,學者稱見羅先生。繫獄時,就問者不絕。至戍所,學徒益眾。許孚遠方巡撫福建,日相過從,材以此忘羈旅。久之赦還。卒年七十九。

陸樹德,字與成,尚書樹聲弟也。嘉靖末進士。除嚴州推官。行取當授給事、御史,會樹聲拜侍郎,乃授刑部主事。隆慶四年,改禮科給事中。穆宗御朝講,不發一語。樹德言:「上下交為泰,今暌隔若此,何以劘君德,訓萬幾?」不報。屢遷都給事中。六年四月,詔輟東宮講讀,樹德言:「自四月迄八月,為時甚遙,請非盛暑,仍御講筵。」不聽。穆宗頗倦勤,樹德言:「日月交蝕,旱魃為災,當及時修省。」及帝不豫,又請謹藥餌,善保護,仲夏亢陽月,宜益慎起居。帝不悅,疏皆留中。內臣請祈福戒壇,已得旨,樹德言:「戒壇度僧,男女擾雜,導淫傷化。陛下欲保聖躬,宜法大禹之惡旨酒,成湯之不邇聲色,何必奉佛。」未幾,穆宗崩,神宗嗣位,中官馮保擠司禮孟沖而代之。樹德言:「先帝甫崩,忽傳馮保掌司禮監。果先帝意,何不傳示數日前,乃在彌留後?果陛下意,則哀痛方深,萬幾未御,何暇念中官?」疏入,保大恨。比議祧廟,樹德請毋祧宣宗,仍祀睿宗世室,格不行。已,極陳民運白糧之患,請領之漕臣,從之。

樹德居言職三年,疏數十上,率侃直。會樹聲掌禮部,乃量遷尚寶卿。歷太常少卿,南京太僕卿,以右僉都御史巡撫山東。樹德素清嚴,約束僚吏,屏絕聲伎。山東民壯改民兵,戍薊門,隆慶末令歲輸銀二萬四千,罷其戍役。尋命增輸三萬,樹德請如河南例罷之。帝不從,而為免增輸之數。德府白雲湖故民田,為王所奪,後已還民,王復結中官謀復之。樹德爭不得,乞休歸。久之卒。

蕭廩,字可發,萬安人。祖乾元,以御史劾劉瑾,廷杖下獄,終雲南副使。廩舉嘉靖末進士,授行人。隆慶三年擢御史。因地震,請加禮中宮。已,出核陜西四鎮兵食。斥將吏隱占卒數萬人歸伍。固原州海剌都之地,密邇松山,為楚府牧地。廩言楚府封武昌,牧地在塞下,與寇接,王所收四五百金,而奸宄窟穴,弊甚大,宜諭使獻之朝廷。詔可。已,改巡茶馬。七苑牧地,養馬八千七百餘匹,而占地五萬五千三百頃有奇。廩但給萬二千二百餘頃,歲益課二萬。萬曆元年,巡按浙江。請祀建文朝忠臣十二人,從祀王守仁於文廟。尋擢太僕少卿,再遷南京太僕卿。九年,由光祿卿改右僉都御史,巡撫陜西。時方核天下隱田,大吏爭希張居正指增賦,廩令如額而止。境內回回部常群行拾麥穗,間或草竊,耀州以變告。廩撫諭之,戮數人,變遂定;令拾麥毋帶兵器,儕偶不得至十人。進右副都御史,移撫浙江。先以賞貢使,歲增造彩幣二千。廩請均之福建及徽、寧諸府,從之。已,請減上供織造,不許。遷工部右侍郎,召改刑部。進兵部左侍郎,以官卒。贈尚書。

廩初從歐陽德、鄒守益遊。制行醇謹,故所至有立。

賈三近,字德修,嶧縣人。隆慶二年進士。選庶吉士,授吏科給事中。四年六月,疏言:「善治者守法以宜民,去其太甚而已。今廟堂之令不信於郡縣,郡縣之令不信於小民。蠲租矣而催科愈急,振濟矣而追逋自如,恤刑矣而冤死相望。正額之輸,上供之需,邊疆之費,雖欲損毫釐不可得。形格勢制,莫可如何。且監司考課,多取振作集事之人,而輕寬平和易之士,守令雖賢,安養之心漸移於苛察,撫字之念日奪於征輸,民安得不困!乞戒有司務守法。而監司殿最毋但取旦夕功,失惇大之體。」已,復疏言:「撫按諸臣遇州縣長吏,率重甲科而輕鄉舉。同一寬也,在進士則為撫字,在舉人則為姑息。同一嚴也,在進士則為精明,在舉人則為苛戾。是以為舉人者,非華顛豁齒不就選;人或裹足毀裳,息心仕進。夫鄉舉豈乏才良,宜令勉就是途,因行激勸。」詔皆俞允。再遷左給事中,勘事貴州。中道罷遣,遂請急歸。

神宗嗣位,起戶科給事中。萬曆元年,平江伯陳王謨以太后家姻,夤緣得鎮湖廣。三近劾其垢穢,乃不遣。給事中雒遵、御史景嵩、韓必顯劾譚綸被謫,三近率同列救之,詔增供用庫黃蠟歲二萬五千,三近等又諫,皆不從。時方行海運,多覆舟,以三近言罷其役。肅王縉貴,隆慶間用賄以輔國將軍襲封,至是又請復莊田,三近再疏爭,遂弗予。初,有令征賦以八分為率,不及者議罰。三近請地凋敝者減一分,詔從之。中官溫泰請盡輸關稅、鹽課於內庫,三近言課稅本餉邊,今屯田半蕪,開中法壞,塞下所資惟此,茍歸內帑,必誤邊計。議乃寢。頃之,擢太常少卿。再遷南京光祿卿,請假歸。十二年,召掌光祿,其秋,拜右僉都御史,巡撫保定。畿輔大饑,振貸有方。召拜大理卿。未上,以親老歸養。起兵部右侍郎,復以親老辭,不許。尋卒。

李頤,字惟貞,餘干人。隆慶二年進士。授中書舍人。博習典故,負才名。萬歷初,擢御史。同官胡涍、景嵩、韓必顯,給事中雒遵相繼獲譴,抗疏申救,不聽。清軍湖廣、廣西,請免士民遠戍,只充傍近衛所軍,制可。忤張居正,出為湖州知府。遷蘇松兵備副使、湖廣按察使。鄖陽兵變,知府沈鈇且得罪,頤為白其冤,而密殲首亂者。以母喪歸。

起故官,蒞陜西,進河南右布政使。擢右僉都御史,巡撫順天。進右副都御史。以定亂兵進兵部右侍郎。長昂桀驁,頤與總兵王保擒其心腹小郎兒等七人,賊遂軿。已,別部伯牙入寇,督將士敗之羅文峪,進左侍郎。久之,進右都御史。

時礦稅使四出。馬堂駐天津,王忠駐昌平,王虎駐保定,張曄駐通州。頤疏言:「燕京王氣所鐘,去陵寢近,開鑿必損靈氣。」又言:「畿輔地荒歲儉,而敕使誅求,不遺纖屑,恐臨清激變之慘,復見輦轂下。」已,遼東稅使高淮誣劾山海同知羅大器,頤復言:「內監外僚,初無統攝,且遼陽礦稅,何預薊門?若皆效淮所為,有司將無遺類。陛下奉天之權,制馭宇內,今盡落宦豎手,朝奏夕報,如響應聲。縱所劾當罪,尚非所以為名,何況無辜,暴加摧折。」皆不報。頤在鎮十年,威望大著。中使憚頤廉正,畿民少安。二十九年,以工部右侍郎代劉東星管理河道。議上築決口,下疏故道,為經久計。甫兩月,以勞卒。贈兵部尚書。

頤仕宦三十餘年,敝車羸馬,布衣蔬食。初為御史,首請祀胡居仁於文廟,寢未行。見居仁裔孫希祖幼且貧,字以女,養之於家。弟謙早卒,以己廕畀其子。

硃鴻謨,字文甫,益都人。隆慶五年進士。授吉安推官。識鄒元標於諸生,厚禮之。擢南京御史。元標及吳中行等得罪,鴻謨疏救,語侵居正,斥為民。鴻謨歸,杜門講學,不入城市。居正卒,起故官,出按江西。奏蠲水災賦,請減饒州磁器,不報。又疏薦建言削籍者,忤旨,奪俸。擢光祿少卿。由大理少卿擢右僉都御史,提督操江。改撫應天、蘇州十府。引二祖節儉之德,請裁上供織造,報聞。吳中徭役不均,令一以田為準,不及百畝者無役,縣為立籍,定等差。貴游子弟恣里中,無賴者與共為非,遠近訛言謂有不軌謀。鴻謨盡捕之,上疏告變。朝議將用兵,兵部主事伍袁萃亟言於尚書石星,令覆勘,乃解。鴻謨尋入為刑部右侍郎,卒官。不能斂,僚屬醵金以辦。贈刑部尚書,謚恭介。蕭彥,字思學,涇縣人。隆慶五年進士。除杭州推官。萬曆三年,擢兵科給事中。自塞上多警,邊吏輒假招降倖賞。彥言:「議招逆黨,為中國逋亡設耳,乃欲以此招漠北敵人。夫李俊、滿四等休養百年,稱亂一旦,降人不可處內地明矣。宜一切報罷。」從之。以工科左給事中閱視陜西四鎮邊務。還奏訓兵、儲餉十事,並允行。

尋進戶科都給事中。初,行丈量法,延、寧二鎮益田萬八千餘頃。總督高文薦請三年征賦,彥言:「西北墾荒永免科稅,祖制也。況二鎮多沙磧,奈何定永額,使初集流庸懷去志。」遂除前令。詔購金珠,已,停市,而命以其直輸內庫。彥言不當虛外府以實內藏,不聽。尋上言:「察吏之道,不宜視催科為殿最。昨隆慶五年詔征賦不及八分者,停有司俸。至萬曆四年則又以九分為及格,仍令帶征宿負二分,是民歲輸十分以上也。有司憚考成,必重以敲撲。民力不勝,則流亡隨之。臣以為九分與帶征二議,不宜並行。所謂寬一分,民受一分之賜也。」部議允行。未幾,浙江巡撫張佳胤復以舊例請,部又從之。彥疏爭,乃詔如新令。詔取黃金三千二百兩,彥請納戶部言減其半,不從。

擢太常少卿,以右僉都御史巡撫貴州。都勻答千巖苗叛,土官蒙詔不能制,彥檄副使楊寅秋破擒之。宣慰安國亨詭言獻大木,被賚。及徵木無有,為彥所劾。國亨懼,誣商奪其木,訐彥於朝。帝怒,欲罪彥。大學士申時行等言國亨反噬,輕朝廷,帝乃止。

改撫雲南。時用師隴川,副將鄧子龍不善御軍,兵大噪,守備姜忻撫定之。而其兵素驕,給餉少緩,遂作亂。鼓行至永昌,趨大理,抵瀾滄,過會城。彥調土、漢兵夾攻之,斬首八十,脅從皆撫散。事聞,賚銀幣。自緬甸叛,孟養、車里二宣慰久不貢。至是修貢,彥撫納之。

尋以副都御史撫治鄖陽。進兵部右侍郎,總制兩廣軍務。日本躪朝鮮。會暹羅入貢,其使請勤王,尚書石星因令發兵搗日本。彥言暹羅處極西,去日本萬里,安能飛越大海,請罷其議。星執不從。既而暹羅兵卒不出。召拜戶部右侍郎,尋卒。

彥從同縣查鐸學,有志行。服官明習天下事,所在見稱。後贈右都御史,謚定肅。

弟雍,廣東按察使。宦績亞於彥,而學過之。時稱「二蕭」。

查鐸,字子警,嘉靖四十五年進士。隆慶時,為刑科左給事中。忤大學士高拱,出為山西參議。萬歷初,官廣西副使,移疾歸。繕水西書院,講王畿、錢德洪之學,後進多歸之。

孫維城,字宗甫,丘縣人。隆慶五年進士。歷知浚、太康、任丘三縣。萬曆十年,擢南京御史。初,張居正不奔喪,寧國諸生吳仕期欲上書諫。未發,太平同知龍宗武告之操江胡檟,以聞於居正。會有偽為海瑞劾居正疏者,播之邸抄。宗武意仕期,遂置獄,榜掠七日而卒。居正死,仕期妻訟冤,維城疏言狀。檟已擢刑部侍郎,宗武湖廣參議,皆落職戍邊,天下快之。中官田玉提督太和山,請兼行分守事,帝許之,維城援祖制力陳不可。

俄以救言官範俊,奪俸一年。忤座主大學士許國,出為永平知府。遷赤城兵備副使。繕亭障二百六十所,招史、車二部千餘人。以功屢進按察使,兵備如故。部長安兔挾五千騎邀賞,維城請於督、撫,革其市賞而責之,戢不敢肆。尋以右布政使移守宣府,改廣東左布政使。二十九年,拜右僉都御史,巡撫延綏。河套常犯順,罷貢市十餘年。後復松山,築邊城,諸部長恐,益侵軼。至是,吉囊、卜莊等乞款。聞巡撫王見賓當去,請益切。在寧夏者曰著宰,亦請之巡撫楊時寧。兩鎮交奏,給事中桂有根請聽邊臣自主。維城方代見賓,時寧亦遷去,以黃嘉善代,二人並申約束。維城又條善後六事,款事復堅。

初,維城在宣府,與總兵官麻承恩不相能。會承恩亦移鎮延綏。一日,維城見城外積沙及城,命餘丁除之。承恩紿其眾曰:「食不宿飽,且塞沙可盡乎?」卒遂噪。維城曉之曰:「除城沙,以防寇耳,非謂塞上沙也。」卒悟而散。維城因自劾,帝慰留維城,治嘩者。然維城竟坐是得疾,不數月卒。將吏入視其橐,僅俸數金,賻而歸其喪。

謝傑,字漢甫,長樂人。萬曆初進士。除行人。冊封琉球,卻其餽。其使入謝,仍以金饋,卒言於朝而返之。歷兩京太常少卿。南京歲祀懿文太子,以祠祭司官代,傑言:「祝版署御名,而遣賤者將事,於禮為褻。請如哀沖、莊敬二太子例,遣列侯。」帝是之,乃用南京五府僉書。累遷順天府尹。以右副都御史巡撫南、贛。屬吏被薦者以賄謝,傑曰:「賄而後薦,干戈之盜;薦而後賄,衣冠之盜。」人以為名言。進南京刑部右侍郎。

二十五年春,傑以帝荒於政事,疏陳十規。言:「前此兩宮色養維一,今則定省久曠,慶賀亦疏。孝安太后發引,並不親送。前此太廟時饗皆躬親,今則皆遣代。前此經筵臨御,聖學日勤,今則講官徒設,講席久虛。前此披星視朝,今則高拱深居,累年不出。前此歲旱步禱郊壇,今則圜丘大報,久缺齋居;宸宮告災,亦忘修省。前此四方旱澇,多發帑金,今則採礦榷稅。前此用財有節,今則歲進月輸;而江右之磁,江南之糸寧,西蜀之扇,關中之絨,率取之逾額。前此樂聞讜言,今則封事甫陳,嚴綸隨降,但經廢棄,永不賜環。前此撫恤宗室,恩義有加,今則楚籓見誣,中榼旋出,以市井奸宄,間骨肉懿親。前此官盛任使,下無曠鰥,今則大僚屢虛,庶官不補。是陛下孝親、尊祖,好學、勤政、敬天、愛民、節用、聽言、親親、賢賢,皆不克如初矣。」不報。召為刑部左侍郎,擢戶部尚書督倉場。時四方遇災,率請改折,傑請歲運必三百萬以上方許議折,從之。三十二年,卒官。

初,傑父教諭廷袞家居老矣,族人假其名逋賦。縣令劉禹龍言於御史逮之。傑代訊,幾斃。後撫贛,禹龍家居,未嘗修隙,時服其量。

郭惟賢,字哲卿,晉江人。萬歷二年進士。自清江知縣拜南京御史。張居正既死,吳中行、趙用賢等猶未錄。會皇長子生,詔赦天下,惟賢因請召諸臣。馮保惡其言,謫江山丞。保敗,還故官。劾左都御史陳炌希權臣指,論罷御史趙耀、趙應元,不可總憲紀。炌罷去。又薦王錫爵、賈三近、孫鑨、何源、孫丕揚、耿定向、曾同亨、詹仰庇,皆獲召。主事董基諫內操被謫,惟賢救之,忤旨,調南京大理評事。給事中阮子孝、御史潘惟岳等交章救。帝怒,奪俸有差。惟賢尋遷戶部主事,歷順天府丞。

二十年,以右僉都御史巡撫湖廣。景王封德安,土田倍諸籓,國絕賦額猶存。及帝弟潞王之國衛輝,悉以景賦予王。王奏賦不及額,帝為奪監司以下俸,責撫按急奏報。惟賢言:「景府賦額皆奸民投獻,妄張其數。臣為王履畝,增賦二萬五千,非復如往者虛數,王反稱不足,何也?且潞去楚遠,莫若征之有司,轉輸潞府。《會典》皇莊及勛戚官莊,遇災蠲減視民田。今襄、漢水溢,王佃民流亡過半,請蠲如例。」又言:「長沙、寶慶、衡州三衛軍戍武岡,而永州、寧遠諸衛遠戍廣西,瘴癘死無數。請分番迭戍武岡,罷其戍廣西者。」帝悉報許。承天守備中官以徵興邸舊賦,請罪潛江知縣及諸佃民,旨下撫按勾捕。惟賢言:「臣撫楚,事無不當問。今中官問,而臣等為勾捕,臣實不能。」帝直其言而止。尋請以太和山香稅充王府逋祿,免加派小民,又請以周敦頤父輔成從祀啟聖。詔皆從焉。

入為左僉都御史。言行取不宜久停,言官不宜久繫,臺員不宜久缺。已,復言天下多故,乃自大僚至監司率有缺不補,政日廢弛,且建言獲譴者不下百餘人,效忠者皆永棄。帝不納。尋遷左副都御史。請早建皇儲,慎簡輔弼,亟行考選,盡下推舉諸疏。俱不報。久之,以憂歸。起戶部左侍郎,未上卒。贈右都御史。天啟初,謚恭定。

萬象春,字仁甫,無錫人。萬曆五年進士。選庶吉士,授工科給事中。皇女生,詔戶部、光祿寺各進銀十萬兩。象春力諫,不聽。屢遷禮科都給事中。鄭貴妃有盛寵,而帝耽於酒。象春因慈寧宮災疏諫,報聞。時宗室繁衍,歲祿不繼,象春議變通。會河南巡撫褚鈇亦奏其事,帝即命象春遍詣河南、山西、陜西諸王府,計畫以聞。象春抵河南,方集議,而周府諸宗人疑鈇疏出宗正睦挈意,群毆睦挈幾死。象春以狀聞,帝為奪諸人歲祿。象春復以次詣秦、晉諸籓,奏上便宜十五事,多著為令。真人張國祥乞三年一覲,象春言左道無民社寄,不當在述職之列。時詔許后父永年伯王偉乘肩輿,象春言:「勛戚不乘輿,祖制也。固安伯陳景行、武清伯李偉,太后父,衰白封,始賜肩輿。定國公徐文璧班首重臣,嗣爵久,故亦蒙殊典。今偉非三人比,乞寢前命。」皆不許。孟秋將享廟,帝齋宿宮中,象春言當在便殿,不當於內寢。帝怒,停俸三月。已,因災異,言:「外吏貪殘不當遣緹騎逮問,宮禁邃密不當宿重兵,廷臣建言貶黜當敘遷,內臣有犯當付外廷按治。」帝報聞。象春在諫垣久,前後七十十餘疏,多關軍國計。請復建文年號,加景帝廟謚,尤為時所稱。

出為山東參政。妖賊郭大通為亂,計擒之。歷山西左布政使。二十五年,以右副都御史巡撫山東。倭躪朝鮮,濱海郡邑悉戒嚴。象春拊軍民,供饋運,應機立辦。中使陳增以礦稅至,象春疏論其害。福山知縣韋國賢忤增被侵辱,象春力保持之,增遂劾國賢沮撓,象春黨庇。詔逮國賢,奪象春俸,遂引疾歸。起南京工部右侍郎,未上卒。贈右都御史。

鐘化民,字維新,仁和人。萬歷八年進士。授惠安知縣,多異政。御史安九域薦於朝,以俸未及期,移知樂平,治復最。征授御史。與同官何卓、王慎德交章請建儲,不報。出視陜西茶馬,言:「邊塞土寒,獨畜馬為業。今慮其闌出為厲禁,於是民間孳息與境內貿易俱廢,公私緩急亦無所資。請聽逾境販鬻,特不得入番中。又曩寧夏乏餉,歲發萬金易米二萬七千石,後所司乾沒,濫征之民。請以墾田粟補之,永停征派。」俱報可。巡按山東,歲旱,請蠲振先發後聞。坐寧夏時取官銀交際,為尚寶丞周弘禴所劾,調行人司正。累遷儀制郎中。沈王珵堯由支庶嗣,請封其庶子為郡王,化民持不可。帝傳諭曰:「第予虛名,令藉是婚娶耳。」化民奏曰:「沈王子與元子孰親?王子不即封,慮妨婚娶,元子不即立,不慮妨豫教乎?」帝怒,以化民辭直,無以難。帝命並封三王,化民與顧允成等面詰王錫爵於朝房。尋進光祿丞。二十二年,河南大饑,人相食,命化民兼河南道御史往振。荒政具舉,民大悅。既竣,繪圖以進。帝嘉之,褒諭者再。擢太常少卿。二十四年,以右僉都御史巡撫河南,討平南陽礦盜。夾河賊嘯聚數千人,復督兵破之。時方採礦,抗疏力諫。

化民短小精悍,多智計。居官勤厲,所至有聲。遍歷八府,延父老問疾苦。勞瘁卒官,士民相率頌於朝。詔贈右副都御史,賜祠曰忠惠。

吳達可,字安節,宜興人,尚書儼從孫也。萬歷五年進士。歷知會稽、上高、豐城,並有聲。選授御史。疏請御經筵勤學,時與大臣臺諫面議政務,報聞。大學士趙志皋久疾乞休,未得請。達可力言志皋衰庸,宜罷,不納。二十八年正月,請因始和布令,舉皇長子冊立冠婚禮,簡輔臣補臺諫,撤礦稅中使,不報。視鹽長蘆。歲侵,繪上饑民十四圖,力請振貸。稅使馬堂、張日華議加鹽稅,奸商妄稱嘉靖中大同用兵貸其貲三萬六千金,請於鹽課補給,戶部許之。達可皆抗爭,事得已。改按江西。稅使潘相毆折輔國將軍謀圮肢,並系宗人宗達,誣以劫課,劾上饒知縣李鴻主使。帝切責謀圮等,奪鴻官。達可言:「宗人無故受刑,又重之以詰責,將使天潢人人自危。鴻無辜,不當黜。願亟正相罪,復鴻官。」同官湯兆京亦極論相罪,且言遼東高淮、陜西梁永、山東陳增、廣東李鳳、雲南楊榮皆元惡,為民害,不可一日留。皆弗聽。鴻,吳人,大學士申時行之婿。萬歷十六年舉北闈鄉試,為吏部郎中高桂所攻。後七年成進士。至是,抗相,以彊直稱。相又請開廣信銅塘山,採取大木,鑿泰和斌姥山石膏,達可復極諫不可,閣臣亦爭之,乃寢。還掌河南道事。佐溫純大計京官。尋陳新政要機,痛規首輔沈一貫。疏留中。擢太僕少卿,再遷南京太僕卿。召改光祿,進通政使。鎮撫史晉以罪罷,妄投封章詆朝貴。達可封其疏而劾之,晉尋得罪。奏請正疏式、屏讒邪、重駁正、懲奸宄數事,帝嘉納焉。尋上疏乞休去。卒,贈右副都御史。

贊曰:龐尚鵬諸人歷官中外,才住幹局,咸有可稱。賈三近陳時政,多長者之言,其言資格,深中積弊。謝傑卻屬吏饋,亦無愧楊震雲。


\end{pinyinscope}