\article{列傳第一百三十四}

\begin{pinyinscope}
滿朝薦江秉謙侯震暘倪思輝朱欽相王心一王允成李希孔毛士龍

滿朝薦,字震東,麻陽人。萬曆三十二年進士。授咸寧知縣,有廉能聲。稅監梁永縱其下劫諸生橐,朝薦捕治之。永怒,劾其擅刑稅役,詔鐫一官。大學士沈鯉等論救,不聽。會巡撫顧其志極論永貪殘狀,乃復朝薦官,奪俸一歲。無何,永遣人蠱巡按御史餘懋衡。事覺,朝薦捕獲其人。永懼,率眾擐甲入縣庭。吏卒早為備,無所掠而去。城中數夜驚,言永反,或謂永宜自明,永遂下教,自白不反狀,然蓄甲者數百。而朝薦助懋衡操之急,諸惡黨多亡去。朝薦追之渭南,頗有所格傷。永懼,使使繫書髮中,入都訟朝薦劫上供物,殺數人,投屍河中。帝震怒,立遣使逮治,時三十五年七月也。既至,下詔獄搒掠,遂長繫。中外論救,自大學士朱賡以下,百十疏。最後,四十一年秋,萬壽節將屆,用大學士葉向高請,乃與王邦才、卞孔時並釋歸。

光宗立,起南京刑部郎中,再遷尚寶卿。天啟二年,遼東地盡失,海內多故,而廷臣方植黨逞浮議。朝薦深慮之,疏陳時事十可憂、七可怪,語極危切。尋進太僕少卿,復上疏曰:

比者,風霾曀晦,星月晝見,太白經天,四月雹,六月冰,山東地震,畿內霪潦,天地之變極矣。四川則奢崇明叛,貴州則安邦彥叛,山東則徐鴻儒亂,民人之變極矣。而朝廷政令乃顛倒日甚。

一乞骸耳,周嘉謨、劉一燝,顧命之元老,以中讒去;孫慎行,守禮之宗伯,以封典去;王紀,執法如山之司寇,以平反去;皆漠不顧惜。獨心卷心卷於三十疏劾之沈紘,即去而猶加異數焉。祖宗朝有是顛倒乎?一建言耳,倪思輝、朱欽相等之削籍,已重箝口之嗟;周朝瑞、惠世揚等之拂衣,又中一網之計。祖宗朝有是顛倒乎?一邊策耳,西部索百萬之貲,邊臣猶慮其未飽;健兒乞錙銖之餉,度支尚謂其過奢。祖宗朝有是顛倒乎?一棄城耳,多年議確之犯或以庇厚而緩求,旬日矜疑之輩反以妒深而苛督。祖宗朝有是顛倒乎?一緝奸耳,正罪自有常律,平反原無濫條。遼陽之禍,起於袁應泰之大納降人,降人盡占居民婦女,故遼民發憤,招敵攻城。事發倉卒,未聞有何人獻送之說也。廣寧之變,起於王化貞之誤信西部,取餉金以啖插而不給卒伍,以故人心離散。敵兵過河,又不聞西部策應,遂至手足無措,抱頭鼠竄。亦事發倉卒,未聞有何人獻送之說也。深求奸細,不過為化貞卸罪地耳。王紀不欲殺人媚人,反致削籍。祖宗朝有是顛倒乎?若夫閣臣之職,在主持清議。今章疏有妒才壞政者,非惟不斥也,輕則兩可,重則竟行其言矣。有殛奸報國者,非惟不納也,輕則見讓,重則遞加黜罰矣。尤有恨者,沈紘賄盧受得進,及受敗,又交通跋扈之奄以樹威。振、瑾僨裂之禍,皆紘作俑,而放流不加。他若戚畹,豈不當檢,何至以閹寺之讒,斃其三僕?三宮分有常尊,何至以傾國之暱,僭逼母儀。此皆顛倒之甚者也。顧成於陛下者什之一二,成於當事大臣者十之八九。臣誠不忍見神州陸沈,祈陛下終覽臣疏,與閣部大臣更絃易轍,悉軌祖宗舊章,臣即從逢、干於地下,猶生之年。

既奏,魏忠賢激帝怒,降旨切責,褫職為民。大學士向高申救甚力,帝不納。已,忠賢黨撰《東林同志錄》,朝薦與焉,竟不復用。崇禎二年薦起故官,未上卒。

江秉謙,字兆豫,歙人。萬曆三十八年進士。除鄞縣知縣。用廉能征,擬授御史。久不得命,以葬親歸。光宗立,命始下,入臺,侃侃言事。

天啟元年,首陳君臣虛己奉公之道,規切甚至。戶部尚書李汝華建議興屯,請專遣御史,三年課績,所墾足抵年例餉銀,即擢京卿。秉謙力駁其謬,因言汝華尸素,宜亟罷。汝華疏辨,秉謙再劾之。

沈陽既失,朝士多思熊廷弼,而給事中郭鞏獨論廷弼喪師誤國,請并罪閣臣劉一燝。秉謙憤,力頌廷弼保守危疆功,且曰:「今廷弼勘覆已明,議者猶以一人私情沒天下公論,寧壞朝廷封疆,不忘胸中畛域。」章下廷議。會遼陽復失,廷弼旋起經略。鞏坐妄議奪官,遂與秉謙為仇。廷弼既鎮山海,議遣使宣諭朝鮮發兵牽制。副使梁之垣請行,廷弼喜,請付二十萬金為軍貲。兵部尚書張鶴鳴不予,秉謙抗疏爭。鶴鳴怒,力詆秉謙朋黨。秉謙疏辨,帝不罪。

鶴鳴既抑廷弼,專庇巡撫王化貞,朝士多附會之。帝以經、撫不和,詔廷臣議。秉謙言:「陛下再起廷弼,委以重寄,曰『疆場事不從中制』。乃數月以來,廷弼不得措手足,呼號日聞,辨駁踵至。執為詞者曰『經、撫不和,化貞主戰,廷弼主守耳,夫廷弼非專言守,謂守定而後可戰也。化貞銳意戰,即戰勝,可無事守乎?萬一不勝,又將何以守?此中利害,夫人知之。乃一則無言不從,一則無策不棄。豈真不明於戰守之說,但從化貞、廷弼起見耳。陛下既命廷弼節制三方,則三方之進戰退守當一一聽其指揮。乃化貞欲進,則使廷弼從之進,欲退,則使廷弼隨之退。化貞倏進倏退,則使廷弼進不知所以戰,退不知所以守。是化貞有節制廷弼之權,而廷弼未嘗有節制三方之權也。故今日之事,非經、撫不和,乃好惡經、撫者不和;非戰守之議論不合,乃左右經、撫者之議論不合。請專責廷弼,實圖戰守。」末譏首輔葉向高兩可含糊,勢必兩可掣肘,安能責成功。語極切至。

後朝議方撤廷弼,而化貞已棄廣寧遁。秉謙益憤,以職方郎耿如杞附和鶴鳴,力助化貞排廷弼,致封疆喪失,連疏攻之。并援世宗戮丁汝夔故事,乞亟置鶴鳴於法。帝以鶴鳴方行邊,不當輕詆,奪秉謙俸半歲,如杞不問。秉謙復上疏言:「鶴鳴一入中樞,初不過鹵莽而無遠識,既乃至兇狠而動殺機。明知西部間諜俱虛,戰守參差難合,乃顧自欺以欺朝廷。何處有機會?而曰機會可乘。何日渡河?而曰渡河必勝。既欲驅經略以出關,而不肯付經略以節制,既欲置廷弼於廣寧,而未嘗移化貞於何地。破壞封疆之罪,可置弗問哉?且化貞先棄地先逃,猶曰功罪相半。即此一言,縱寸斬鶴鳴,不足贖其欺君誤國罪,乃猶敢哆口定他入罪案耶!」當是時,大學士沈紘潛結中官劉朝、乳媼客氏,募兵入禁中,興內操。給事中惠世揚、周朝瑞等十二人再疏力攻,秉謙與焉,并詆朝及客氏。內外胥怨,遂假劾鶴鳴疏,出秉謙於外。無何,郭鞏召還,交通魏忠賢,力沮秉謙。是冬,皇子生,言官被謫者悉召還,獨秉謙不與。家居四年,聞忠賢益亂政,憂憤卒。

居數月,忠賢黨御史卓邁追劾秉謙保護廷弼,遂削籍。崇禎初,復官。

侯震暘,字得一,嘉定人。祖堯封,監察御史。忤大學士張居正,外轉。累官至福建右參政,有廉直聲。震暘舉萬曆三十八年進士,授行人。

天啟初,擢吏科給事中。是時,保姆奉聖夫人客氏方擅寵,與魏忠賢及大學士沈紘相表裏,勢焰張甚。既遣出宮,熹宗思念流涕,至日旰不御食,遂宣諭復入。震暘疏言:「宮闈禁地,姦璫群小睥睨其側,內外鉤連,借叢煬灶,有不忍言者。王聖寵而煽江京、李閏之奸,趙嬈寵而構曹節、皇甫之變。麼麼里婦,何堪數暱至尊哉?」不省。

會遼事棘,經略熊廷弼、巡撫王化貞相牴牾,兵部尚書張鶴鳴右化貞,議者遂欲移廷弼,與化貞畫地任事。震暘逆知其必敗,疏言:「事勢至此,陛下宜遣問經臣。果能加意訓練,則進止遲速不從中制,雖撤撫臣,一以付之,無不可者。如不然,則督其條晰陳奏,以聽吏議,摭拾殘局,專任化貞。此一說也。不則移廷弼密雲,而出本兵為經略。鶴鳴素慷慨自命,與其事敗同罪,不若挺身報國。此又一說也。不則遂以經略授化貞,擇沈深有謀者代任巡撫,以資後勁。此又一說也。不則直移廷弼於登、萊,終其三方布置之策,與化貞相犄角。此又一說也。若復遷延猶豫,必僨國事。」疏上,方有旨集議,而大清兵已破廣寧矣。化貞、廷弼相率入關門,猶數奉溫旨,責以戴罪立功。

震暘大憤懣,再疏言:「臣言不幸驗矣,為今日計,論法不論情。河西未壞以前,舉朝所惜者,什七在化貞,今不能為化貞惜也。河西既壞以後,舉朝所寬者什九在廷弼,今亦不能為廷弼寬也。策撫臣者,謂宜責令還赴廣寧,聯屬西部。然而廥庫已竭,其能赤手效包胥乎?策經臣者,謂宜仍責守關。然所謂守者,將如廷弼前議三十萬兵數十萬餉,以圖後效乎?抑止令率殘卒出關外,姑示不殺乎?凡此無一可者。及今不定逃臣之律,殘疆其奚賴焉?」其後治失事罪,蓋略如震暘疏云。

已,遂劾大學士沈紘結納奉聖夫人及諸中官為朋黨,具發其構殺故監王安狀。忠賢即日傳旨謫震暘。震暘陛辭,復上田賦、河渠二議。以逐臣不當建議,再鐫二級以歸。

震暘在垣八月,章奏凡數十上。崇禎初,召復故官,震暘已前卒。因其子主事峒曾請,特贈太常少卿。

方震暘之論客氏也,給事中祁門倪思輝、臨川朱欽相疏繼之。帝大恚,並貶三官。大學士劉一燝、尚書周嘉謨等交章論救,皆不納。御史吳縣王心一言之尤切,帝怒,貶官如之。心一同官龍谿馬鳴起復抗疏諫,且言客氏六不可留。帝議加重譴,用一燝等言,奪俸一年。

先是,元年正月,客氏未出宮,詔給土田二十頃,為護墳香火貲。又詔魏進忠侍衛有功,待陵工告竣,並行敘錄。心一抗疏言:「陛下眷念二人,加給土田,明示優錄,恐東征將士聞而解體。況梓宮未殯,先念保姆之香火,陵工未成,強入奄侍之勤勞,於理為不順,於情為失宜。」不報。至是,與思輝、欽相並貶,廷臣請召還者十餘疏。皇子生,詔思輝、欽相、心一、鳴起並復故官。

欽相尋擢太僕少卿。楊漣既劾魏忠賢,欽相亦抗疏極論。五年以右僉都御史巡撫福建,討賊楊六、蔡三、鐘六等有功。旋以忤忠賢,除名。思輝,崇禎時終南京督儲尚書,心一終刑部侍郎,鳴起終南京右都御史。

王允成,字述文,澤州人。萬曆中舉於鄉,除獲鹿知縣。以治行異等,徵授南京御史。時甲科勢重,乙科多卑下之。允成體貌魁梧,才氣飆發,欲凌甲科出其上,首疏論遼左失事諸臣,請正刑辟。

熹宗即位,廷臣方爭論「梃擊」、「移宮」事,而帝降兩諭罪選侍,因言移宮後相安狀。大學士方從哲封還上諭。允成陳保治十事,中言:「張差闖宮,說者謂瘋癲。青宮豈發瘋之地?龐保、劉成豈並瘋之人?言念及此,可為寒心。今鄭氏四十年之恩威猶在,卵翼心腹寔繁有徒,陛下當思所以防之。比者,聖諭多從中出,當,則開煬灶之端;不當,而臣下爭執,必成反汗之勢,孰若事無大小,盡歸內閣。至元輔方從哲,屢劾不去。陛下於選侍移宮後,發一敕諭,不過如常人表明心跡耳,從哲輒封還。夫封后之命,都督之命,貶謫周朝瑞之命,何皆不封還?司馬昭之心,路人知之矣。」姚宗文閱視遼左,與熊廷弼相失,歸而鼓同列攻之。允成惡其奸,再疏論列。

天啟元年,疏請恤先朝直臣,列楊天民等三十六人以上,帝納之。俄陳任輔弼、擇經略、慎中樞、專大帥、更戎政、嚴賞罰數事,末言:「方今最可慮者,陛下孤立禁中。先朝怙權恃寵諸奄,與今日左右近習,互相忌嫉,恐乘機肆毒,彼此相戕。夫防護禁庭,責在內閣及司禮。務令潛消默化,俾聖躬與皇弟,並得高枕無憂,斯為根本至計。」時韙其言。

已,劾刑部尚書黃克纘倡言保護選侍,貽誤賈繼春,又曲庇盜寶內侍,至辨御史焦源溥綱常一疏,刺謬特甚。已,極論內降及留中之害,末復規切閣部大臣。忤旨,停俸。給事中毛士龍劾府丞邵輔忠,允成亦偕同官李希孔斥輔忠。已,極言綱紀廢弛,請戒姑息、破因循,指斥時事甚悉。

當是時,中貴劉朝、魏進忠與乳媼客氏相倚為奸。允成抗疏歷數其罪,略言:「內廷顧命之璫,犬食其餘,不蒙帷蓋之澤;外廷顧命之老,中旨趣出,立見田里之收。以小馬為馳騁之貲,誰啟盤於遊田之漸;以大臣為釋忿之地,誰啟咈其耇長之心。劉朝輩初亦不預外事,自沈紘、邵輔忠導之,遂恣肆無忌。浸假而王心一、倪思輝、朱欽相斥矣,浸假而司空用陪推矣,浸假而中旨用考官矣。是易置大臣之權在二豎也。近者弄權愈甚,逐大臣如振落,王紀、滿朝薦並削職為編氓。是驅除大臣之權在二豎也。科臣遷改,自有定敘,給假推升,往例皆然。乃惡周朝瑞之正直,忽有不許推用之旨。是轉遷百官之權在二豎也。秦籓以小宗繼大宗,諸子不得封郡王,祖制昭然。乃部科爭之不獲,相繼而去。是進退諸籓之權在二豎也。招權納賄,作福作威;二豎弄權於外,客氏主謀於中。王振、劉瑾之禍將復見今日。」疏入,進忠輩切齒。允成復特疏論秦府濫恩之謬,帝終不省。

三年六月,允成又劾進忠,進忠益恨。明年,趙南星為吏部,知允成賢,調之於北。未幾,南星被逐,御史張訥劾南星調允成非法,遂除名。後給事中陳維新復劾允成貪險,詔撫按提問,坐以贓私。莊烈帝嗣位,以允成嘗請保護皇弟,識其名,召復故官。未幾卒。

當天啟初,東林方盛,其主張聯絡者,率在言路。允成居南,與北相應和,時貴多畏其鋒。然諤諤敢言,屢犯近倖,其風采足重云。

李希孔,字子鑄,三水人。萬曆三十八年進士。授中書舍人,擢南京御史。給事中姚宗文閱遼東軍,排經略熊廷弼,希孔連疏劾之。已,又糾宗文阻抑考選,以「令旨」二字抗言繳還,遏先帝非常之德。泰昌元年冬,陳時政七事。天啟改元,與允成劾邵輔忠。已,請宥言官倪思輝、朱欽相、王心一。三年上《折邪議》,以定兩朝實錄,疏言:

昔鄭氏謀危國本,而左袒之者,莫彰著於三王並封之事。今秉筆者不謂非也,且推其功,至與陳平、狄仁傑並。此其說不可解也。當時並封未有旨,輔臣王錫爵蓋先有密疏請也。迨旨下禮部,而王如堅、硃維京、塗一臻、王學曾、岳元聲、顧允成、于孔兼等苦口力爭,又共責讓錫爵於朝房。於是錫爵始知大義之不可違,而天下之不我予,隨上疏檢舉,而封事停也。假令如堅等不死爭,不責讓,將並封之事遂以定,而子以母貴之說,且徐邀定策國老之勛。而乃飾之曰:「旋命旋引咎,事遂以止。」嗟乎,此可為錫爵諱乎哉!且聞錫爵語人曰:「王給事中遺悔否?」以故事關國本,諸臣稿項黃馘,終錫爵世不復起。不知前代之安劉、復唐者,誰阨王陵,使之不見天日乎?曾剪除張柬之、桓彥範等五人,而令齎志以沒乎?臣所以折邪議者,一也。

其次,莫彰於張差闖宮之事。而秉筆者猶謂無罪也,且輕其事,而列王大臣、貫高事為辭。此其說又不可解也。王大臣之徒手而闖至乾清宮門也,馮保怨舊輔高拱,置刃其袖,挾使供之,非實事也。張差之梃,誰授之而誰使之乎?貫高身無完膚,而詞不及張敖,故漢高得釋敖不問。可與張差之事,造謀主使口招歷歷者比乎?昔寬處之以全倫,今直筆之以存實,以戒後,自兩不相妨,而奈之何欲諱之?且諱之以為君父隱,可也;為亂賊輩隱,則何為?臣所以折邪議者,二也。

至封后遺詔,自古未有帝崩立后者。此不過貴妃私人謀假母后之尊,以弭罪狀。故稱遺詔,以要必行。奈何猶稱先志,重誣神祖,而陰為阿附傳封者開一面也?臣所以折邪議者,三也。

帝之令德考終,自不宜謂因藥致崩,被不美之名。而當時在內視病者,烏可於積勞積虛之後,投攻剋之劑。群議洶洶,方蓄疑慮變之深,而遽值先帝升遐,又適有下藥之事,安得不痛之恨之,疾首頓足而深望之?乃討奸者憤激而甚其詞,庇奸者借題以逸其罰。君父何人,臣子可以僥倖而嘗試乎?臣所以折邪議者,四也。

先帝之繼神廟棄群臣也,兩月之內,鼎湖再號。陛下孑然一身,怙恃無託,宮禁深閟,狐鼠實繁,其於杜漸防微,自不得不倍加嚴慎。即不然,而以新天子儼然避正殿,讓一先朝宮嬪,萬世而下謂如何國體。此楊漣等諸臣所以權衡輕重,亟以移宮請也。宮已移矣,漣等之心事畢矣,本未嘗居以為功,何至反以為罪而禁錮之、擯逐之,是誠何心?即選侍久侍先帝,生育公主,諸臣未必不力請於陛下,加之恩禮。今陛下既安,選侍又未嘗不安,有何冤抑,而汲汲皇皇為無病之沈吟?臣所以折邪議者,五也。

抑猶有未盡者。神祖與先帝所以處父子骨肉之際,仁義孝慈,本無可以置喙。即當年母愛子抱,外議喧嘩,然雖有城社媒孽之奸,卒不以易祖訓立長之序,則愈足見神祖之明聖,與先帝之大孝。何足諱、何必諱,又何可諱?若謂言及鄭氏之過,便傷神祖之明,則我朝仁廟監國危疑,何嘗為成祖之累。而當時史臣直勒之汗青,並未聞有嫌疑之避也。何獨至今而立此一說,巧為奸人脫卸,使昔日不能置之罪,今日不容著之書,何可訓也!今史局開,公道明,而坐視奸輩陰謀,辨言亂義,將令三綱紊,九法滅,天下止知有私交,而不知有君父。乞特敕纂修諸臣,據事直書,無疑無隱,則繼述大孝過於武、周,而世道人心攸賴之矣。

詔付史館參酌,然其後卒不能改也。已,又請出客氏於外,請誅崔文昇。忌者甚眾,指為東林黨。未幾,卒官,故不與璫禍。

毛士龍,字伯高,宜興人。萬歷四十一年進士。授杭州推官。熹宗即位,擢刑科給事中,首劾姚宗文閱視乖張。楊漣去國,抗疏請留。天啟改元正月疏論「三案」,力言孫慎行、陸夢龍、陸大受、何士晉、馬德灃、王之寀、楊漣等有功社稷,而魏浚輩醜正害直之罪。帝是之。

李選侍之移宮也,其內豎劉朝、田詔、劉進忠等五人,以盜貲下刑部獄。尚書黃克纘庇之,數稱其冤。帝不從,論死。是年五月,王安罷,魏進忠用事。詔等進重賂,令其下李文盛等上疏鳴冤,進忠即傳旨貸死。大學士劉一燝等執奏者再。旨下刑科,士龍抄參者三,旨幾中寢。克纘乃陳其冤狀,而請付之熱審。進忠不從,傳旨立釋。士龍憤,劾克纘阿旨骫法,不可為大臣,且數朝等罪甚悉。由是進忠及諸奄銜士龍次骨。進忠廣開告密,誣天津廢將陳天爵交通李承芳,逮其一家五十餘人,下詔獄。士龍即劾錦衣駱思恭及誣告者罪。進忠憾張后抑己,誣為死囚孫二所出,布散流言。士龍請究治妖言奸黨並主使逆徒,進忠益憾。

至九月,士龍劾順天府丞邵輔忠奸貪,希孔、允成亦劾之,輔忠大懼。朝等因誘以超擢,令攻士龍。輔忠遂訐士龍官杭州時盜庫納妓,進忠從中下其疏。尚書周嘉謨等言兩人所訐,風聞,請寬貸。進忠不從,削士龍籍,輔忠落職閒住。進忠後易名忠賢,顯盜國柄,恨士龍未已。四年冬,令其私人張訥劾之,再命削籍。明年三月入之汪文言獄詞,謂納李三才賄三千,謀起南京吏部,下撫按提訊追贓,遣戍平陽衛。已而輔忠起用,驟遷兵部侍郎。六年十二月,御史劉徽復摭輔忠前奏,劾士龍納訪犯萬金,下法司逮治。士龍知忠賢必殺己,夜中踰牆遁,其妾不知也,謂有司殺之,被髮號泣於道,有司無如之何。士龍乃潛至家,載妻子浮太湖以免。

莊烈帝嗣位,忠賢伏誅。朝士為士龍稱冤,詔盡赦其罪。士龍始詣闕謝恩,且陳被陷之故。帝憐之,命復官致仕,竟不召用。至崇禎十四年,里人周延儒再相,始起漕儲副使,督蘇、松諸郡糧。明年冬,入為太僕少卿。又明年春,擢左僉都御史。時左都御史李邦華、副都御史惠世揚皆未至,士龍獨掌院事。帝嘗語輔臣:「往例御史巡方,類微服訪民間。近高牙大纛,氣凌巡撫,且公署前後皆通竇納賄,每奉使,富可敵國,宜重懲。」士龍聞,劾逮福建巡按李嗣京。十月謝病歸。國變後卒。

贊曰:滿朝薦,健令也,出死力以抗兇鋒,幽深牢而弗悔。及躋言路,益發憤時事,庶幾強立不反者歟。江秉謙、侯震暘之論經撫,李希孔之論「三案」,皆切中事理。王允成直攻劉朝、魏進忠,而不與楊、左、周、黃諸人同難。毛士龍顧以譎免。蓋忠賢殺人皆成於附閹邪黨,彼其甘心善類,授之刃而假手焉且加功者,罪直浮於忠賢已。


\end{pinyinscope}