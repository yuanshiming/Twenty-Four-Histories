\article{列傳第九十八}

\begin{pinyinscope}
桑喬(胡汝霖}}謝瑜王曄伊敏生童漢臣等何維柏徐學詩葉經陳紹厲汝進查秉彞等王宗茂周冕趙錦吳時來張翀董傳策鄒應龍張檟林潤

桑喬,字子木,江都人。嘉靖十一年進士。十四年冬,由主事改御史,出按山西。所部頻寇躪,喬奏請盡蠲徭賦,厚恤死者家。參將葉宗等將萬人至荊家莊,陷賊伏中,大潰,賊遂深入。天城、陽和兩月間五遭寇。巡撫樊繼祖、總兵官魯綱以下,皆為喬劾,副將李懋及宗等六人並逮治。

十六年夏,雷震謹身殿,下詔求言。喬偕同官陳三事,略言營造兩宮山陵,多侵冒;吉囊恣橫,邊備積弛。而末言:「陛下遇災而懼,下詔修省。修省不外人事,人事無過擇官。尚書嚴嵩及林庭昂、張瓚、張雲皆上負國恩,下乖輿望,災變之來,由彼所致。」疏奏,四人皆乞罷。詔庭昂、雲致仕,留嵩、瓚如故。嵩再疏辨,且詆言者。給事中胡汝霖言:「大臣被論,引罪求退而已。嵩負穢行,召物議,逞辭奏辨,陰擠言官,無大臣體。」帝下詔戒飭如汝霖指。時嵩拜尚書甫半歲,方養交遊,揚聲譽,為進取地,舉朝猶未知其奸,喬獨首發之。

喬尋巡按畿輔,引疾。都御史王廷相以規避劾之,嵩因構其罪。逮下詔獄,廷杖,戍九江。居戍所二十六年而卒。隆慶初,贈恤如制。

胡汝霖,綿州人。由庶吉士除戶科給事中。二十年四月,九廟災。偕同官聶靜、御史李乘雲劾文武大臣救火緩慢者二十六人,嵩與焉。帝怒所劾不盡,下詔獄訊治,俱鐫級調外。汝霖得太平府經歷。既謫官,則請解於嵩,反附以進。累遷至右僉都御史,巡撫甘肅。及嵩敗,以嵩黨奪官。

謝瑜,字如卿,上虞人。嘉靖十一年進士。由南京御史改北。十九年正月,禮部尚書嚴嵩屢被彈劾求去,帝慰留。瑜言:「嵩矯飾浮詞,欺罔君上,箝制言官。且援明堂大禮、南巡盛事為解,而謂諸臣中無為陛下任事者,欲以激聖怒。奸狀顯然。」帝留疏不下。嵩奏辨,且言:「瑜擊臣不已,欲與朝廷爭勝」。帝於是切責瑜,而慰諭嵩甚至。居二歲,竟用嵩為相。

甫踰月,瑜疏言:「武廟盤遊佚樂,邊防宜壞而未甚壞。今聖明在上,邊防宜固而反大壞者,大臣謀國不忠,而陛下任用失也。自張瓚為中樞,掌兵而天下無兵,擇將而天下無將。說者謂瓚形貌魁梧,足稱福將。夫誠邊塵不聳,海宇晏然,謂之福可也。今瓚無功而恩廕屢加,有罪而褫奪不及,此其福乃一身之福,非軍國之福也。昔舜誅四凶,萬世稱聖。今瓚與郭勛、嚴嵩、胡守中,聖世之四凶。陛下旬月間已誅其二,天下翕然稱聖,何不並此二凶,放之流之,以全帝舜之功也?大學士翟鑾起廢棄中,授以巡邊之寄,乃優游曼衍,靡費供億。以盛苞苴者為才,獻淫樂者為敬,遂使邊軍益瘠,邊備更弛。行邊若此,將焉用之!故不清政本,天下必不治也。不易本兵,武功必不競也。」

疏入,留不下。嵩復疏辯,帝更慰諭,瑜復被譙讓。然是時帝雖響嵩,猶未深罪言者,嵩亦以初得政,未敢顯擠陷,故瑜得居職如故。未幾,假他事貶其官。又三載,大計,嵩密諷主者黜之。比疏上,令如貪酷例除名,瑜遂廢棄,終於家。

始瑜之為御史也,武定侯郭勛陳時政,極詆大小諸臣不足任,請復遣內侍出鎮守。詔從之。瑜抗章奏曰:「勛所論諸事,影響恍惚,而復設鎮守,則其本意所注也。勛交通內侍,代之營求,利他日重賄。其言:『官吏貪濁,由陛下無心腹耳目之人在四方』。又曰:『文武懷奸避事,許內臣劾奏,則奸貪自息』。果若勛言,則內臣用事莫如正德時,其為太平極治耶?陛下革鎮守內臣,誠聖明善政,而勛詆以偏私。在朝百官,孰非天子耳目?而勛詆以不足任。欲陛下盡疑天下士大夫,獨倚宦官為腹心耳目,臣不知勛視陛下為何如主?」會給事中朱隆禧亦以為言,勛奏始寢。瑜,隆慶初復贈太僕少卿。

王曄,字韜孟,金壇人。嘉靖十四年進士。授吉安推官,召拜南京吏科給事中。二十年九月偕同官上言:「外寇陸梁,本兵張瓚及總督尚書樊繼祖、新遷侍郎費寀不堪重寄」。帝下其章於所司。居兩月,復劾瓚,因及禮部尚書嚴嵩、總督侍郎胡守中,與巨奸郭勛相結納。嵩所居第宅,則勛私人代營之。踰月,御史伊敏生、鄭芸、陳策亦云嵩居宅乃勛私人孫澐所居,澐籍沒,嵩第應在籍中。帝怒,奪敏生等俸一級。嵩不問,而守中竟由曄疏獲罪。明年秋,嵩入內閣。吏科都給事中沈良才、御史喻時等交章劾嵩。踰月,山西巡按童漢臣章上。又踰月,曄與同官陳塏、御史陳紹等章亦上。大指皆論嵩奸貪,而曄疏並及嵩子世蕃,語尤剴切,帝皆不省。嵩憾甚,未有以中也。久之,為山東僉事,給由入都,道病後期,嵩遂奪其官。曄在臺,嘗劾罷方面官三十九人,直聲甚著。比歸,環堵蕭然,數年卒。

伊敏生,上元人。鄭芸、陳策,俱莆田人。敏生官至山東參政。策,台州知府。芸,終御史。

沈良才,泰州人。起家庶吉士,歷官至兵部侍郎。三十六年大計自陳,已調南京矣,嵩附批南京科道拾遺疏中,落其職。

喻時,光山人。官至南京兵部侍郎。

童漢臣,錢塘人。由魏縣知縣入為御史。寇大入宣府、大同,總督樊繼祖等掩敗,三以捷聞。漢臣等劾之,得罪。其按山西,督諸將擊卻俺答之薄太原者,會方劾嵩,觸其怒。明年,漢臣與巡撫李玨覈上繼祖等失事狀。章下吏部。漢臣前劾嵩並劾吏部尚書許讚,讚亦憾漢臣。因言漢臣劾遲延,宜並論。嵩遂擬旨鐫玨一階留任,謫漢臣湖廣布政司都事。舉朝皆知為嵩所中,莫能救也。久之,為泉州知府。倭賊薄城,有保障功。終江西副使。

陳塏,餘姚人。後為嵩斥罷。

何維柏,字喬仲,南海人。嘉靖十四年進士。選庶吉士,授御史。雷震謹身殿,維柏言四海困竭,所在流移,而所司議加賦,民不為盜不止。因請罷沙河行宮、金山功德寺工作,及安南問罪之師。帝頗嘉納。尋引疾歸。久之,起巡按福建。二十四年五月疏劾大學士嚴嵩奸貪罪,比之李林甫、盧杞。且言嵩進顧可學、盛端明修合方藥,邪媚要寵。帝震怒,遣官逮治。士民遮道號哭,維柏意氣自如。下詔獄,廷杖,除名。家居二十餘年。隆慶改元,召復官,擢大理少卿。遷左僉都御史。疏請日御便殿,召執政大臣謀政事,並擇大臣有才德者與講讀儒臣更番入直。宮中燕居,慎選謹厚內侍調護聖躬,俾游處有常,幸御有節。非隆冬盛寒,毋輟朝講。報聞。進左副都御史。母憂歸。萬曆初,還朝。歷吏部左、右侍郎,極論鬻官之害。御史劉臺劾大學士張居正,居正乞罷,維柏倡九卿留之。及居正遭父喪,詔吏部諭留。尚書張瀚叩維柏,維柏曰:「天經地義,何可廢也?」瀚從之而止。居正怒,取旨罷瀚,停維柏俸三月。旋出為南京禮部尚書。考察自陳,居正從中罷之。卒謚端恪。

徐學詩,字以言,上虞人。嘉靖二十三年進士。授刑部主事,歷郎中。二十九年,俺答薄京師。既退,詔廷臣陳制敵之策。諸臣多掇細事以應。學詩憤然曰:「大奸柄國,亂之本也。亂本不除,能攘外患哉?」即上疏言:

大學士嵩輔政十載,奸貪異甚。內結權貴,外比群小。文武遷除,率邀厚賄,致此輩掊克軍民,釀成寇患。國事至此,猶敢謬引佳兵不祥之說,以謾清問。近因都城有警,密輸財賄南還。大車數十乘,樓船十餘艘,水陸載道,駭人耳目。又納奪職總兵官李鳳鳴二千金,使鎮薊州,受老廢總兵官郭琮三千金,使督漕運。諸如此比,難可悉數。舉朝莫不歎憤,而無有一人敢牴牾者,誠以內外盤結,上下比周,積久勢成。而其子世蕃又兇狡成性,擅執父權。凡諸司奏請,必先白其父子,然後敢聞於陛下。陛下亦安得而盡悉之乎?

蓋嵩權力足以假手下石,機械足以先發制人,勢利足以廣交自固,文詞便給足以掩罪飾非。而精悍警敏,揣摩巧中,足以趨利避害;彌縫缺失,私交密惠,令色脂言,又足以結人歡心,箝人口舌。故前後論嵩者,嵩雖不能顯禍之於正言之時,莫不假事託人陰中之於遷除考察之際。如前給事中王曄、陳塏,御史謝瑜、童漢臣輩,于時亦蒙寬宥,而今皆安在哉?陛下誠罷嵩父子,別簡忠良代之,外患自無不寧矣。

帝覽奏,頗感動。方士陶仲文密言嵩孤立盡忠,學詩特為所私修隙耳。帝於是發怒,下之詔獄。嵩不自安,求去,帝優詔慰諭。嵩疏謝,佯為世蕃乞回籍,帝亦不許。學詩竟削籍。先劾嵩者葉經、謝瑜、陳紹與學詩皆同里,時稱「上虞四諫」。隆慶初,起學詩南京通政參議。未之官,卒。贈大理少卿。

初,學詩族兄應豐以善書擢中書舍人,供事無逸殿,悉嵩所為。嵩疑學詩疏出應豐指,會考察,屬吏部斥之。應豐詣迎和門辭,特旨留用,嵩恚益甚。居數年以誤寫科書譖於帝,竟杖殺之。

葉經,字叔明。嘉靖十一年進士。除常州推官,擢御史。嵩為禮部,交城王府輔國將軍表柙謀襲郡王爵,秦府永壽王庶子惟燱與嫡孫懷墡爭襲,皆重賄嵩,嵩許之。二十年八月,經指其事劾嵩。嵩懼甚,力彌縫,且疏辯。帝乃付襲爵事於廷議,而置嵩不問。嵩由是憾經。又二年,經按山東監鄉試。試錄上,嵩指發策語為誹謗,激帝怒。廷杖經八十,斥為民。創重,卒。提調布政使陳儒及參政張臬,副使談愷、潘恩,皆謫邊方典史,由嵩報復也。穆宗即位,贈經光祿少卿,任一子官。

陳紹終韶州知府。

厲汝進,字子修,灤州人。嘉靖十一年進士。授池州推官,徵拜吏科給事中。湖廣巡撫陸傑以顯陵工成,召為工部等郎。汝進言傑素犯清議,不宜佐司空,並劾尚書甘為霖、樊繼祖不職。不納。三遷至戶科都給事中。戶部尚書王杲下獄,汝進與同官海寧查秉彞、馬平徐養正、巴縣劉起宗、章丘劉祿合疏言:「兩淮副使張祿遣使入都,廣通結納。如太常少卿嚴世蕃、府丞胡奎等,皆承賂受囑有證。世蕃竊弄父權,嗜賄張焰。」詞連倉場尚書王。嵩上疏自理,且求援於中官以激帝怒。帝責其代杲解釋,命廷杖汝進八十,餘六十,並謫雲南、廣西典史。明年,嵩復假考察,奪汝進職。隆慶初,起故官。未至京,卒。

秉彞由黃州推官歷戶科左給事中。數建白時事。終順天府尹。

養正以庶吉士歷戶科右給事中。隆慶中,官至南京工部尚書。

起宗初除衢州推官。召為戶科給事中。延綏幾饑,請帑金振救。終遼東苑馬寺卿。

祿以行人司擢戶科給事。謫後,自免歸。

王宗茂,字時育,京山人。父橋,廣東布政使。從父格,太僕卿。宗茂登嘉靖二十六年進士,授行人。三十一年擢南京御史。時先後劾嚴嵩者皆得禍,沈金柬至謫佃保安。中外懾其威,益箝口。宗茂積不平,甫拜官三月,上疏曰:

嵩本邪諂之徒,寡廉鮮恥。久持國柄,作福作威。薄海內外,罔不怨恨。如吏、兵二部每選,請屬二十人,人索賄數百金,任自擇善地。致文武將吏盡出其門。此嵩負國之罪一也。

任私人萬寀為考功郎。凡外官遷擢,不察其行能,不計其資歷,唯賄是問。致端方之士不得為國家用。此嵩負國之罪二也。

往歲遭人論劾,潛輸家資南返,輦載珍寶,不可數計。金銀人物,多高二三尺者。下至溺器,亦金銀為之。不知陛下宮中亦有此器否耶?此嵩負國之罪三也。

廣布良田,遍於江西數郡。又於府第之後積石為大坎,實以金銀珍玩,為子孫百世計。而國計民瘼,一不措懷。此嵩負國之罪四也。

畜家奴五百餘人,往來京邸。所至騷擾驛傳,虐害居民,長吏皆怨怒而不敢言。此嵩負國之罪五也。

陛下所食大官之饌不數品,而嵩則窮極珍錯。殊方異產,莫不畢致。是九州萬國之待嵩有甚於陛下。此嵩負國之罪六也。

往歲寇迫京畿,正上下憂懼之日,而嵩貪肆益甚。致民俗歌謠,遍於京師,達於沙漠。海內百姓,莫不祝天以冀其早亡,嵩尚恬不知止。此嵩負國之罪七也。

募朝士為乾兒義子至三十餘輩。若尹耕、梁紹儒,早已敗露。此輩實衣冠之盜,而皆為之爪牙,助其虐焰,致朝廷恩威不出於陛下。此嵩負國之罪八也。

夫天下之所恃以為安者,財也,兵也。不才之文吏,以賂而出其門,則必剝民之財,去百而求千,去千而求萬,民奈何不困?不才之武將以賂而出其門,則必剋軍之餉,或缺伍而不補,或踰期而不發,兵奈何不疲?邇者,四方地震,其占為臣下專權。試問今日之專權者,寧有出於嵩右乎?陛下之帑藏不足支諸邊一年之費,而嵩所蓄積可贍儲數年。與其開賣官鬻爵之令以助邊,盍去此蠹國害民之賊,籍其家以紓患也?臣見數年以來,凡論嵩者不死於廷杖,則役於邊塞。臣亦有身家,寧不致惜,而敢犯九重之怒,攖權相之鋒哉?誠念世受國恩,不忍見祖宗天下壞於賊嵩之手也。

疏至,通政司趙文華密以示嵩,留數日始上,由是嵩得預為地。遂以誣詆大臣,謫平陽縣丞。

方宗茂上疏,自謂必死。及得貶,恬然出都。到官半歲,以母憂歸。嵩無以釋憾,奪其父橋官。橋竟憤悒卒。嵩罷相之日,宗茂亦卒。隆慶初,贈光祿少卿。

周冕,資縣人。嘉靖二十年進士。授太常博士,擢貴州道試御史。重建太廟成,奉安神主,帝將遣官代祭。御史鄢懋卿言其不可。帝怒,降手詔數百言諭廷臣,且言更有協君取譽者,必罪不宥。舉朝悚息,無敢復言,冕獨抗章爭之。帝震怒,立下冕詔獄搒掠。終以其言直,釋還職。是時太子生十一年矣,猶未出閣講學。冕極言教諭不可緩,請早降綸言,慎選侍從。帝又大怒,謫雲南通海縣典史。冕雖遠竄,意慷慨無所屈。

數遷至武選郎中。楊繼盛劾嚴嵩及嚴效忠冒功事,語侵歐陽必進。必進奏辯,章下兵部。冕上言:

臣奉詔檢得二十七年通政司狀,效忠年十六,因武會試未第,咨兩廣軍門聽用。已而必進及總兵官陳圭奏黎賊平,遣效忠報捷,授錦衣試所鎮撫。未踰月,嚴鵠言兄效忠曾斬首七級,並功加賞,應得署副千戶。今效忠身抱痼疾,鵠請代職。臣心疑其偽,方將核實以聞。嵩子世蕃乃自創一槁付臣,屬臣依違題覆。臣觀其槁,率誕謾舛戾,請得一一折之。

如效忠曾中武舉,何初無本籍起送文牒,今又稱民人,而不言武舉?如效忠果鵠之兄,世蕃之子,則世蕃數子俱幼,未有名效忠者。如效忠果斬首七級,則當時狀稱年止十六,豈能赴戰?何軍門諸將俱未聞斬獲功,獨宰相一孫乃驍勇冠三軍?如曰效忠對敵,脛臂受創,計臨陣及差委,相去未一月,何以萬里軍情即能馳報?如曰效忠到京以創甚疾故,何以鵠代職之日,止告不能受職?如曰效忠鎮撫當代,則奏捷功止及身,例無傳襲。如曰效忠功當並論,例先奏請,何止用通狀,而逼令司官奉行?

臣悉心廉訪,初未有名效忠者赴軍門聽用,鵠亦非效忠親弟。其姓名乃詭設,首級亦要買,而非有纖毫實跡也。必進既嵩鄉曲,圭又世蕃姻親,依阿朋比,共為欺罔。臣如不言,陛下何從知其奸?且自累朝以來,未聞有宰相之子孫送軍門報效者。今嵩不唯咨送軍門,而且詭託名姓,破壞祖宗之制,彼蔣應奎、唐國相輩何怪其效尤耶?臣職守攸關,義不敢陷,乞特賜究正,使天下曉然知朝廷有不可幸之功、不可犯之法。臣雖得罪,死無所恨。

疏奏,直聲震朝廷。嵩父子大懼,力事彌縫。帝責冕報復,下詔獄拷訊,斥為民。冕既得罪,而尚書覆奏如世蕃指矣。隆慶初,錄先朝直臣,起冕太僕少卿。遭母憂,未任,卒。

趙錦,字元樸,餘姚人。嘉靖二十三年進士。授江陰知縣,徵授南京御史。江洋有警,議設總兵官於鎮江。錦言:「小寇剽掠,不足煩重兵。」帝乃罷之。已,疏言:「淮兗數百里,民多流傭,乞寬租徭,簡廷臣督有司拊循。」報可。軍興,民輸粟馬,得官錦衣,錦極陳不可。尋清軍雲南。

三十二年元旦,日食。錦以為權奸亂政之應,馳疏劾嚴嵩罪。其略曰:

臣伏見日食元旦,變異非常。又山東、徐、淮仍歲大水,四方頻地震,災不虛生。昔太祖高皇帝罷丞相,散其權於諸司,為後世慮至深遠矣。今之內閣,無宰相之名,而有其實,非高皇帝本意。頃夏言以貪暴之資,恣睢其間。今大學士嵩又以佞奸之雄,繼之怙寵張威,竊權縱欲,事無鉅細,罔不自專。人有違忤,必中以禍,百司望風惕息。天下事未聞朝廷,先以聞政府。白事之官,班候於其門;請求之賂,幅輳於其室。銓司黜陟,本兵用舍,莫不承意指。邊臣失事,率朘削軍資納賕嵩所,無功可以受賞,有罪可以逭誅。至宗籓勳戚之襲封,文武大臣之贈謚,其遲速予奪,一視賂之厚薄。以至希寵干進之徒,妄自貶損。稱號不倫,廉恥掃地,有臣所不忍言者。

陛下天縱聖神,乾綱獨運,自以予奪由宸斷,題覆在諸司,閣臣擬旨取裁而已。諸司奏稿,並承命於嵩,陛下安得知之?今言誅,而嵩得播惡者,言剛暴而疏淺,惡易見,嵩柔佞而機深,惡難知也。嵩窺伺逢迎之巧,似乎忠勤,諂諛側媚之態,似乎恭順。引植私人,布列要地,伺諸臣之動靜,而先發以制之,故敗露者少。厚賂左右親信之人,凡陛下動靜意向,無不先得,故稱旨者多。或伺聖意所注,因而行之以成其私;或乘事機所會,從而鼓之以肆其毒。使陛下思之,則其端本發於朝廷;使天下指之,則其事不由於政府。幸而洞察於聖心,則諸司代嵩受其罰;不幸而遂傳於後世,則陛下代嵩受其諐。陛下豈誠以嵩為賢邪?自嵩輔政以來,惟恩怨是酬,惟貨賄是斂。群臣憚陰中之禍,而忠言不敢直陳;四方習貪墨之風,而閭閻日以愁困。

頃自庚戌之後,外寇陸梁。陛下嘗募天下之武勇以足兵,竭天下之財力以給餉,搜天下之遺逸以任將,行不次之賞,施莫測之威,以風示內外矣。而封疆之臣卒未有為陛下寬宵旰憂者。蓋緣權臣行私,將吏風靡,以掊克為務,以營競為能。致朝廷之上,用者不賢,賢者不用;賞不當功,罰不當罪。陛下欲致太平,則群臣不足承德於左右;欲遏戎寇,則將士不足禦侮於邊疆。財用已竭,而外患未見底寧;民困已極,而內變又虞將作。陛下躬秉至聖,憂勤萬幾,三十二年於茲矣,而天下之勢其危如此,非嵩之奸邪,何以致之?

臣願陛下觀上天垂象,察祖宗立法之微,念權柄之不可使移,思紀綱之不可使亂,立斥罷嵩,以應天變,則朝廷清明,法紀振飭。寇戎雖橫,臣知其不足平矣。

當是時,楊繼盛以劾嵩得重譴,帝方蓄怒以待言者。周冕爭冒功事亦下獄,而錦疏適至。帝震怒,手批其上,謂錦欺天謗君,遣使逮治,復慰諭嵩備至。於是錦萬里就徵,屢墮檻車,瀕死者數矣。既至,下詔獄拷訊,搒四十,斥為民。父塤,時為廣西參議,亦投劾罷。

錦家居十五年,穆宗即位,起故官。擢太常少卿,未上,進光祿卿。江陰歲進子鱭萬斤,奏減其半。隆慶元年以右副都御史巡撫貴州,破擒叛苗龍得鮓等。宣慰安氏素桀驁,畏錦,為效命。入為大理卿,歷工部左、右侍郎。嘗署部事,有所爭執。

萬曆二年,遷南京右都御史,改刑部尚書。張居正遭喪,南京大臣議疏留。錦及工部尚書費三暘不可而止。移禮部,又移吏部,俱在南京。錦以居正操切,頗訾議之。語聞,居正令給事中費尚伊劾錦講學談禪,妄議朝政,錦遂乞休去。居正死,給事、御史交薦,起故官。十一年召拜左都御史。是時,方籍居正貲產。錦言:「世宗籍嚴嵩家,禍延江西諸府。居正私藏未必逮嚴氏,若加搜索,恐貽害三楚,十倍江西民。且居正誠擅權,非有異志。其翊戴沖聖,夙夜勤勞,中外寧謐,功亦有不容泯者。今其官廕贈謚及諸子官職並從褫革,已足示懲,乞特哀矜,稍寬其罰。」不納。

二品六年滿,加太子少保,尋加兵部尚書,掌院事如故。錦摘陳御史封事可採者數條,請旨行之。四川巡按雒遵憾錦,假條奏指錦為奸臣。御史周希旦、給事中陳與郊不直遵,交章論列,遂調遵外任。帝幸山陵,再奉敕居守。其冬,以繼母喪歸。十九年召拜刑部尚書。年七十六矣,再辭,不許。次蘇州卒。贈太子太保,謚端肅。

錦始終厲清操,篤信王守仁學,而教人則以躬行為本。守仁從祀孔廟,錦有力焉。始忤嚴嵩,得重禍。及之官貴州,道嵩里,見嵩葬路旁,惻然憫之,屬有司護視。後忤居正罷官,居正被籍,復為營救。人以是稱錦長者。

吳時來,字惟修,仙居人。嘉靖三十二年進士。授松江推官,攝府事。倭犯境,鄉民攜妻子趨城,時來悉納之。客兵獷悍好剽掠,時來以恩結其長,犯即行法,無嘩者。賊攻城,驟雨,城壞數丈。時來以勁騎扼其衝,急興版築,三日城復完,賊乃棄去。

擢刑科給事中。劾罷兵部尚書許論、宣大總督楊順及巡按御史路楷。皆嚴嵩私人,嵩疾之甚。會將遣使琉球,遂以命時來。三十七年三月,時來抗章劾嵩曰:「頃陛下赫然震怒,逮治僨事邊臣,人心莫不欣快。邊臣朘軍實、饒執政,罪也。執政受其饋,朋奸罔上,獨得無罪哉?嵩輔政二十年,文武遷除,悉出其手。潛令子世蕃出入禁所,批答章奏。世蕃因招權示威,頤指公卿,奴視將帥,筐篚苞苴,輻輳山積,猶無饜足。用所親萬寀為文選郎,方祥為職方郎,每行一事,推一官,必先稟命世蕃而後奏請。陛下但知議出部臣,豈知皆嵩父子私意哉!他不具論。如趙文華、王汝孝、張經、蔡克廉以及楊順、吳嘉會輩,或祈免死,或祈遷官,皆剝民膏以營私利,虛官帑以實權門,陛下已洞見其一二。言官如給事中袁洪愈、張墱,御史萬民英亦嘗屢及之。顧多旁指微諷,無直攻嵩父子者。臣竊謂除惡務本。今邊事不振由於軍困,軍困由官邪,官邪由執政之好貨。若不去嵩父子,陛下雖宵旰憂勞,邊事終不可為也。」

時張翀、董傳策與時來同日劾嵩。而翀及時來皆徐階門生,傳策則階邑子,時來先又官松江,於是嵩疑階主使。密奏三人同日構陷,必有人主之,且時來乃憚琉球之行,借端自脫。帝入其言,遂下三人詔獄,嚴鞫主謀者。三人瀕死不承,第言「此高廟神靈教臣為此言耳。」主獄者乃以三人相為主使讞上。詔皆戍煙瘴,時來得橫州。

隆慶初,召復故官。進工科給事中。條上治河事宜,又薦譚綸、俞大猷、戚繼光宜用之蘇鎮,專練邊兵,省諸鎮徵調。帝皆從之。撫治鄖陽。僉都御史劉秉仁被劾且調用,時來言秉仁薦太監李芳,無大臣節,秉仁遂坐罷。帝免喪既久,臨朝未嘗發言,時來上保泰九答刂,報聞。尋擢順天府丞。

隆慶二年,拜南京右僉都御史提督操江。移巡撫廣東。將行,薦所屬有司至五十九人。給事中光懋等劾其濫舉。會高拱掌吏部,雅不喜時來,貶雲南副使。復為拱門生給事中韓楫所劾,落職閒住。

萬曆十二年,始起湖廣副使。俄擢左通政,歷吏部左侍郎。十五年拜左都御史。誠意伯劉世延怙惡,數抗朝令,時來劾之,下所司訊治。時來初以直竄,聲振朝端。再遭折挫,沈淪十餘年。晚節不能自堅,委蛇執政間。連為饒伸、薛敷教、王麟趾、史孟麟、趙南星、王繼光所劾,時來亦連乞休歸。未出都,卒。贈太子太保,謚忠恪。尋為禮部郎中于孔兼所論,奪謚。

張翀,字子儀,柳州人。嘉靖三十二年進士。授刑部主事。疾嚴嵩父子亂政,抗章劾之。其略曰:

竊見大學士嵩貴則極人臣,富則甲天下。子為侍郎,孫為錦衣、中書,賓客滿朝班,親姻盡朱紫。犬馬尚知報主,乃嵩則不然。臣試以邊防、財賦、人才三大政言之。

國家所恃為屏翰者,邊鎮也。自嵩輔政,文武將吏率由賄進。其始不核名實,但通關節,即與除授。其後不論功次,但勤問遺,即被超遷。託名修邊建堡,覆軍者得蔭子,濫殺者得轉官。公肆詆欺,交相販鬻。而祖宗二百年防邊之計盡廢壞矣。

戶部歲發邊餉,本以贍軍。自嵩輔政,朝出度支之門,暮入奸臣之府。輸邊者四,饋嵩者六。臣每過長安街,見嵩門下無非邊鎮使人。未見其父,先饋其子。未見其子,先饋家人。家人嚴年富已踰數十萬,嵩家可知。私藏充溢,半屬軍儲;邊卒凍餒,不保朝夕。而祖宗二百年豢養之軍盡耗弱矣。

邊防既隳。邊儲既虛,使人才足供陛下用,猶不足憂也。自嵩輔政,藐蔑名器,私營囊橐。世蕃以狙獪資,倚父虎狼之勢,招權罔利,獸攫鳥鈔。無恥之徒,絡繹奔走,靡然成風,有如狂易。而祖宗二百年培養之人才盡敗壞矣。

夫嵩險足以傾人,詐足以惑世,辨足以亂政,才足以濟奸。附己者加諸膝,異己者墜之淵。箝天下口使不敢言,而其惡日以恣。此忠義之士,所以搤腕憤激,懷深長之憂者也。陛下誠賜斥譴,以快眾憤,則緣邊將士不戰而氣自倍,百司庶府不令而政自新。

書奏,逮下詔獄拷訊,謫戍都勻。

穆宗嗣位,召為吏部主事,再遷大理少卿。隆慶二年春,以右僉都御史巡撫南、贛。所部萬羊山跨湖廣、福建、廣東境,故盜藪,四方商民種藍其間。至是,盜出劫,翀遣守備董龍剿之。龍聲言搜山,諸藍戶大恐。盜因煽之,嘯聚千餘人。兵部令二鎮撫臣協議撫剿之宜,久乃定。南雄劇盜黃朝祖流劫諸縣,轉掠湖廣,勢甚熾。翀討擒之。移撫湖廣。召拜大理卿,進兵部右侍郎。以侍養歸。

萬曆初,起故官,督漕運。召為刑部右侍郎,不拜,連章乞休。卒於家。天啟初,贈兵部尚書,謚忠簡。

董傳策,字原漢,松江華亭人。嘉靖二十九年進士。除刑部主事。

三十七年抗疏劾大學士嚴嵩,略言:

嵩諗惡誤國,陛下豈不洞燭其奸?特以輔政故,尚為優容,令自省改。而嵩恬不知戒,負恩愈深。居位一日,天下受一日之害。臣竊痛之。

夫邊疆督撫將帥欲得士卒死力,必資財用。今諸邊軍饟歲費百萬,強半賂嵩。遂令軍士饑疲,寇賊深入。此其壞邊防之罪一也。

吏、兵二部持選簿就嵩填註。文選郎萬寀、職方郎方祥甘聽指使,不異卒隸。都門諺語至以「文武管家」目之。此其鬻官爵之罪二也。

侍郎劉伯躍以採木行部,擅斂民財及郡縣贓罪,輦輸嵩家,前後不絕。其他有司破冒攘敓,入獻於嵩者更不可數計。嵩家私藏,富於公帑。此其蠹國用之罪三也。

趙文華以罪放逐,嵩沒其囊橐巨萬,而令人護送南還。恐喝州縣,私役民夫,致道路驛騷,公私煩費。此其黨罪人之罪四也。

天下籓臬諸司,歲時問遺,動以千計,勢不得不掊克小民。民財日殫,嵩貲日積。於是水陸舟車載還其鄉,月無虛日。所至要索供億,勢如虎狼。此其騷驛傳之罪五也。

嵩久握重權,灸手而熱。干進無恥之徒,附亶逐穢,麕集其門。致士風日偷,官箴日喪。此其壞人才之罪六也。

嵩以蔽欺行其專權,生死予奪惟意所為。而世蕃又以無賴之子,竊威助惡。父子肆凶,中外飲憤。有臣如此,非國法可容。臣待罪刑曹,宜詰奸慝。陛下誠不惜嚴氏以謝天下,則臣亦何惜一死以謝權奸!

疏入,下詔獄。謫戍南寧。

穆宗立,召復故官。歷郎中。隆慶五年累遷南京大理卿,進工部右侍郎。萬曆元年就改禮部。言官劾傳策受人賄,免歸。繩下過急,竟為家奴所害。

鄒應龍,字雲卿,長安人。嘉靖三十五年進士。授行人,擢御史。嚴嵩擅政久,廷臣攻之者輒得禍,相戒莫敢言。而應龍知帝眷已潛移,其子世蕃益貪縱,可攻而去也,乃上疏曰:

工部侍郎嚴世蕃憑藉父權,專利無厭。私擅爵賞,廣致賂遺。使選法敗壞,市道公行。群小競趨,要價轉鉅。刑部主事項治元以萬三千金轉吏部,舉人潘鴻業以二千二百金得知州。夫司屬郡吏賂以千萬,則大而公卿方岳,又安知紀極?

平時交通贓賄,為之居間者不下百十餘人,而其子錦衣嚴鵠、中書嚴鴻、家人嚴年、幕客中書羅龍文為甚。年尤桀黠,士大夫無恥者至呼為鶴山先生。遇嵩生日,年輒獻萬金為壽。臧獲富侈若是,主人當何如?

嵩父子故籍袁州,乃廣置良田美宅於南京、揚州,無慮數十所,以豪僕嚴冬主之。抑勒侵奪,民怨入骨。外地牟利若是,鄉里又何如?

尤可異者,世蕃喪母,陛下以嵩年高,特留侍養,令鵠扶櫬南還。世蕃乃聚狎客,擁艷姬,恒舞酣歌,人紀滅絕。至鵠之無知,則以祖母喪為奇貨。所至驛騷,要索百故。諸司承奉,郡邑為空。

今天下水旱頻仍,南北多警。而世蕃父子方日事掊克,內外百司莫不竭民脂膏,塞彼谿壑。民安得不貧?國安得不病?天人災變安得不迭至也?臣請斬世蕃首懸之於市,以為人臣凶橫不忠之戒。茍臣一言失實,甘伏顯戮。嵩溺愛惡子,召賂市權,亦宜亟放歸田,用清政本。

帝頗知世蕃居喪淫縱,心惡之。會方士藍道行以扶乩得幸,帝密問輔臣賢否。道行詐為乩語,具言嵩父子弄權狀,帝由是疏嵩而任徐階。及應龍奏入,遂勒嵩致仕,下世蕃等詔獄,擢應龍通政司參議。然帝雖罷嵩,念其贊修玄功,意忽忽不樂,手札諭階:「嵩已退,其子已伏辜,敢再言者,當并應龍斬之。」應龍深自危,不敢履任,賴階調護始視事。御史張檟巡鹽河東,不知帝指,上疏言:「陛下已顯擢應龍,而王宗茂、趙錦輩首發大奸未召,是曲突者不賞也。」帝大怒,立逮至,杖六十,斥為民。久之,世蕃誅,應龍乃自安。

隆慶初,以副都御史總理江西、江南鹽屯。遷工部右侍郎。鎮守雲南黔國公沐朝弼驕恣,廷議遣大臣有威望者鎮之,乃改應龍兵部侍郎兼右僉都御史巡撫雲南。至則發朝弼罪,朝弼竟被逮。萬曆改元,鐵索箐賊作亂,討平之。已,番人栂犬發反,合土漢兵進討,斬獲各千餘人。

應龍有才氣,初以劾嚴嵩得名,驟致通顯。及為太常,省牲北郊,東廠太監馮保傳呼至,導者引入,正面爇香,儼若天子。應龍大駭,劾保僭肆,保深銜之。至是,京察自陳,保修郤,令致仕。臨安土官普崇明、崇新兄弟構爭。崇明引廣南儂兵為助,崇新則召交兵。已,交兵退,儂兵尚留,應龍命部將楊守廉往剿。守廉掠村聚,殺人。儂賊乘之,再敗官軍,人以咎應龍。應龍聞罷官,不俟代徑歸。代者王凝欲自以為功,力排應龍。給事中裴應章遂劾應龍僨事。巡按御史郭廷梧雅不善應龍,勘如凝言。應龍遂削籍,卒於家。

十六年,陜西巡撫王璇言應龍歿後,遺田不及數畝,遺址不過數楹,恤典未被,朝野所恨。帝命復應龍官,予祭葬。

張檟,江西新城人。嘉靖三十八年進士。居臺中,敢言。穆宗初,復官。屢疏抗中官,嘗劾大學士高拱。拱復入閣掌吏部,檟已遷太僕少卿,坐不謹罷歸。萬曆中,累官工部右侍郎。

林潤,字若雨,莆田人。嘉靖三十五年進士。授臨川知縣。以事之南豐,寇猝至,為畫計卻之。征授南京御史。嚴世蕃置酒召潤,潤談辨風生,世蕃心憚之。既罷,屬客謂之曰:「嚴侍郎謝君,無刺當世事。」潤到官,首論祭酒沈坤擅殺人,置之理。已,劾副都御史鄢懋卿五罪,嚴嵩庇之,不問。伊王典楧不道,數遭論列不悛,潤復糾之。典楧累奏辨,詆潤挾私。部科交章論王抗朝命,脅言官。世蕃納其賄,下詔責讓而已。潤因言宗室繁衍,歲祿不繼,請亟議變通。帝為下所司集議。

會帝用鄒應龍言,戍世蕃雷州,其黨羅龍文尋州。世蕃留家不赴。龍文一詣戍所,即逃還徽州,數往來江西,與世蕃計事。四十三年冬,潤按視江防,廉得其狀,馳疏言:「臣巡視上江,備訪江洋群盜,悉竄入逃軍羅龍文、嚴世蕃家。龍文卜築深山,乘軒衣蟒,有負險不臣之心。而世蕃日夜與龍文誹謗時政,搖惑人心。近假名治第,招集勇士至四千餘人。道路恟懼,咸謂變且不測。乞早正刑章,以絕禍本。」帝大怒,即詔潤逮捕送京師。世蕃子紹庭官錦衣,聞命亟報世蕃,使詣戍所。方二日,潤已馳至。世蕃猝不及赴,乃械以行,龍文亦從梧州捕至。遂盡按二人諸不法事,二人竟伏誅。

潤尋擢南京通政司參議,歷太常寺少卿。隆慶元年以右僉都御史巡撫應天諸府。屬吏懾其威名,咸震心慄。潤至,則持寬平,多惠政,吏民皆悅服。居三年,卒官。年甫四十。

潤鄉郡興化陷倭,特疏請蠲復三年,發帑金振恤。鄉人德之。喪歸,遮道四十里,為位祭哭凡三日。

贊曰:世宗非庸懦主也。嵩相二十餘年,貪裛盈貫。言者踵至,斥逐罪死,甘之若飴,而不能得君心之一悟。唐德宗言:「人謂盧杞奸邪,朕殊不覺。」各賢其臣,若蹈一轍,可勝歎哉!世蕃之誅,發於鄒應龍,成於林潤。二人之忠,非過於楊繼盛,其言之切直,非過於沈鍊、徐學詩等,而大憝由之授首。蓋惡積滅身,而鄒、林之彈擊適會其時歟!


\end{pinyinscope}