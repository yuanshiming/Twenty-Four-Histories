\article{列傳第一百九十六 奸臣}

\begin{pinyinscope}
《宋史》論君子小人,取象於陰陽,其說當矣。然小人世所恒有,不容概被以奸名。必其竊弄威柄、構結禍亂、動搖宗祏、屠害忠良、心跡俱惡、終身陰賊者,始加以惡名而不敢辭。有明一代,巨奸大惡,多出於寺人內豎,求之外廷諸臣,蓋亦鮮矣。當太祖開國之初,胡惟庸兇狡自肆,竟坐叛逆誅死。陳瑛在成祖時,以刻酷濟其奸私,逢君長君,荼毒善類。此其所值,皆英武明斷之君,而包藏禍心,久之方敗。令遇庸主,其為惡可勝言哉!厥後權歸內豎,懷奸固寵之徒,依附結納,禍流搢紳。惟世宗朝,閹宦斂迹,而嚴嵩父子濟惡,貪DF無厭。莊烈帝手除逆黨,而周延儒、溫體仁懷私植黨,誤國覆邦。南都末造,本無足言,馬士英庸瑣鄙夫,饕殘恣惡。之數人者,內無閹尹可依,而外與群邪相比,罔恤國事,職為亂階。究其心迹,殆將與巳、檜同科。吁可畏哉!作《奸臣傳》。

○胡惟庸陳寧陳瑛馬麟等嚴嵩趙文華等周延儒溫體仁馬士英阮大鋮

胡惟庸,定遠人。歸太祖於和州,授元帥府奏差。尋轉宣使,除寧國主簿,進知縣,遷吉安通判,擢湖廣僉事。吳元年,召為太常少卿,進本寺卿。洪武三年拜中書省參知政事。已,代汪廣洋為左丞。六年正月,右丞相廣洋左遷廣東行省參政,帝難其人,久不置相,惟庸獨專省事。七月拜右丞相。久之,進左丞相,復以廣洋為右丞相。

自楊憲誅,帝以惟庸為才,寵任之。惟庸亦自勵,嘗以曲謹當上意,寵遇日盛,獨相數歲,生殺黜陟,或不奏徑行。內外諸司上封事,必先取閱,害己者,輒匿不以聞。四方躁進之徒及功臣武夫失職者,爭走其門,饋遺金帛、名馬、玩好,不可勝數。大將軍徐達深疾其奸,從容言於帝。惟庸遂誘達閽者福壽以圖達,為福壽所發。御史中丞劉基亦嘗言其短。久之基病,帝遣惟庸挾醫視,遂以毒中之。基死,益無所忌。與太師李善長相結,以兄女妻其從子佑。學士吳伯宗劾惟庸,幾得危禍。自是,勢益熾。其定遠舊宅井中,忽生石筍,出水數尺,諛者爭引符瑞,又言其祖父三世塚上,皆夜有火光燭天。惟庸益喜自負,有異謀矣。

吉安侯陸仲亨自陜西歸,擅乘傳。帝怒責之,曰:「中原兵燹之餘,民始復業,籍戶買馬,艱苦殊甚。使皆效爾所為,民雖盡鬻子女,不能給也。」責捕盜於代縣。平諒侯費聚奉命撫蘇州軍民,日嗜酒色。帝怒,責往西北招降蒙古,無功,又切責之。二人大懼。惟庸陰以權利脅誘二人,二人素戇勇,見惟庸用事,密相往來。嘗過惟庸家飲,酒酣,惟庸屏左右言:「吾等所為多不法,一旦事覺,如何?」二人益惶懼,惟庸乃告以己意,令在外收集軍馬。又嘗與陳寧坐省中,閱天下軍馬籍,令都督毛驤取衛士劉遇賢及亡命魏文進等為心膂,曰:「吾有所用爾也。」太僕寺丞李存義者,善長之弟,惟庸婿李佑父也,惟庸令陰說善長。善長已老,不能強拒,初不許,已而依違其間。惟庸益以為事可就,乃遣明州衛指揮林賢下海招倭,與期會。又遣元故臣封績致書稱臣于元嗣君,請兵為外應。事皆未發。會惟庸子馳馬於市,墜死車下,惟庸殺挽車者。帝怒,命償其死。惟庸請以金帛給其家,不許。惟庸懼,乃與御史大夫陳寧、中丞塗節等謀起事,陰告四方及武臣從己者。

十二年九月,占城來貢,惟庸等不以聞。中官出見之,入奏。帝怒,敕責省臣。惟庸及廣洋頓首謝罪,而微委其咎於禮部,部臣又委之中書。帝益怒,盡囚諸臣,窮詰主者。未幾,賜廣洋死,廣洋妾陳氏從死。帝詢之,乃入官陳知縣女也。大怒曰:「沒官婦女,止給功臣家。文臣何以得給?」乃敕法司取勘。于是惟庸及六部堂屬咸當坐罪。明年正月,塗節遂上變,告惟庸。御史中丞商暠時謫為中書省吏,亦以惟庸陰事告。帝大怒,下廷臣更訊,詞連寧、節。廷臣言:「節本預謀,見事不成,始上變告,不可不誅。」乃誅惟庸、寧并及節。

惟庸既死,其反狀猶未盡露。至十八年,李存義為人首告,免死,安置崇明。十九年十月,林賢獄成,惟庸通倭事始著。二十一年,藍玉征沙漠,獲封績,善長不以奏。至二十三年五月,事發,捕績下吏,訊得其狀,逆謀益大著。會善長家奴盧仲謙首善長與惟庸往來狀,而陸仲亨家奴封帖木亦首仲亨及唐勝宗、費聚、趙庸三侯與惟庸共謀不軌。帝發怒,肅清逆黨,詞所連及坐誅者三萬餘人。乃為《昭示奸黨錄》,布告天下。株連蔓引,迄數年未靖云。

陳寧,茶陵人。元末為鎮江小吏,從軍至集慶,館於軍帥家,代軍帥上書言事。太祖覽之稱善,召試檄文,詞意雄偉,乃用為行省掾吏。時方四征,羽書帝午,寧酬答整暇,事無留滯,太祖益才之。淮安納款,奉命徵其兵,抵高郵,為吳人所獲。寧抗論不屈,釋還,擢廣德知府。會大旱,乞免民租,不許。寧自詣太祖奏曰:「民饑如此,猶征租不已,是為張士誠驅民也。」太祖壯而聽之。

辛丑除樞密院都事。癸卯遷提刑按察司僉事。明年改浙東按察使。有小隸訟其隱過,寧已擢中書參議,太祖親鞫之,寧首服,繫應天獄一歲。吳元年,冬盡將決,太祖惜其才,命諸將數其罪而宥之,用為太倉市舶提舉。洪武元年召拜司農卿,遷兵部尚書。明年出為松江知府。用嚴為治,積蠹弊,多所釐革。尋改山西行省參政。召拜參知政事,知吏、戶、禮三部事。寧,初名亮,至是賜名寧。

三年,坐事出知蘇州。尋改浙江行省參政,未行,用胡惟庸薦,召為御史中丞。太祖嘗御東閣,免冠而櫛。寧與侍御史商暠入奏事,太祖見之,遂移入便殿,遣人止寧毋入。櫛已,整冠出閣,始命入見。六年命兼領國子監事。俄拜右御史大夫。八月遣釋奠先師。丞相胡惟庸、參政馮冕、誠意伯劉基不陪祀而受胙,太祖以寧不舉奏,亦停俸半月。自是,不預祭者不頒胙。久之,進左御史大夫。

寧有才氣,而性特嚴刻。其在蘇州徵賦苛急,嘗燒鐵烙人肌膚。吏民苦之,號為陳烙鐵。及居憲臺,益務威嚴。太祖嘗責之,寧不能改。其子孟麟亦數諫,寧怒,捶之數百,竟死。太祖深惡其不情,曰:「寧於其子如此,奚有於君父耶!」寧聞之懼,遂與惟庸通謀。十三年正月,惟庸事發,寧亦伏誅。

陳瑛,滁人。洪武中,以人才貢入太學。擢御史,出為山東按察使。建文元年調北平僉事。湯宗告瑛受燕王金錢,通密謀,逮謫廣西。燕王稱帝,召為都察院左副都御史,署院事。

瑛天性殘忍,受帝寵任,益務深刻,專以搏擊為能。甫蒞事,即言:「陛下應天順人,萬姓率服,而廷臣有不順命、效死建文者,如侍郎黃觀、少卿廖升、修撰王叔英、紀善周是修、按察使王良、知縣顏伯瑋等,其心與叛逆無異,請追戮之。」帝曰:「朕誅奸臣,不過齊、黃數輩,後二十九人中如張紞王鈍、鄭賜、黃福、尹昌隆,皆宥而用之。況汝所言,有不與此數者,勿問。」後瑛閱方孝孺等獄詞,遂簿觀、叔英等家,給配其妻女,疏族、外親莫不連染。胡閏之獄,所籍數百家,號冤聲徹天。兩列御史皆掩泣,瑛亦色慘,謂人曰:「不以叛逆處此輩,則吾等為無名。」於是諸忠臣無遺種矣。

永樂元年擢左都御史,益以訐發為能。八月劾歷城侯盛庸怨誹,當誅,庸自殺。二年劾曹國公李景隆謀不軌,又劾景隆弟增枝知景隆不臣不諫,多置莊產,蓄佃僕,意叵測,俱收繫。又劾長興侯耿炳文僭,炳文自殺。劾駙馬都尉梅殷邪謀,殷遇害。三年,行部尚書雒僉言事忤帝意,瑛劾僉貪暴,僉坐誅死。又劾駙馬都尉胡觀強取民間女子,娶娼為妾,預景降逆謀,以親見宥不改。帝命勿治,罷觀朝請。已,又劾其怨望,逮下獄。八年劾降平侯張信占練湖及江陰官田,命三法司雜治之。

瑛為都御史數年,所論劾勳戚、大臣十餘人,皆陰希帝指。其他所劾順昌伯王佐,都督陳俊,指揮王恕,都督曹遠,指揮房昭,僉都御史俞士吉,大理少卿袁復,御史車舒,都督王瑞,指揮林泉、牛諒,通政司參議賀銀等,先後又數十人,俱得罪。帝以為能發奸,寵任之,然亦知其殘刻,所奏讞不盡從。中書舍人芮善弟夫婦為盜所殺,心疑其所親,訟於官。刑部驗非盜,縱之。善白帝刑部故出盜,帝命御史鞫治,果非盜。瑛因劾善妄奏,當下獄。帝曰:「兄弟同氣,得賊惟恐逸之,善何罪,其勿問。車里宣慰使刀暹答侵威遠州地,執其知州刀算黨以歸。帝遣使諭之,刀暹答懼,歸地及所執知州,遣弟刀臘等貢方物謝罪。瑛請先下刀臘法司,且逮治刀暹答。帝曰:「蠻僚之性稍不相得則相仇,改則已。今服罪而復治之,何以處不服者。」遂赦弗問。知嘉興縣李鑒廷見謝罪,帝問故。瑛言:「鑒籍奸黨姚瑄,瑄弟亨當連坐,而鑒釋亨不籍,宜罪。」鑑言:「都察院文止籍瑄,未有亨名。」帝曰:「院文無名而不籍,不失為慎重。」鑒得免。戶部人材高文雅言時政,因及建文事,辭意率直,帝命議行之。瑛劾文雅狂妄,請置之法。帝曰:「草野之人何知忌諱,其言有可采,奈何以直而廢之。瑛刻薄,非助朕為善者。」以文雅付吏部,量材授官。海運糧漂沒,瑛請治官軍罪,責之償。帝曰:「海濤險惡,官軍免溺死,幸矣。」悉釋不問。瑛之奸險附會,一意苛刻,皆此類也。

帝北巡,皇太子監國。瑛言兵部主事李貞受皂隸葉轉等四人金,請下貞獄。無何,貞妻擊登聞鼓訴冤。皇太子命六部大臣廷鞫之,自辰至午,貞等不至,惟葉轉至。訊之,云貞不承,不勝拷掠死,三皂錄皆笞死三日矣,貞實未嘗受金。先是,袁綱、覃珩兩御史俱至兵部索皂隸,貞猝無以應,兩御史銜之,興此獄。於是刑科給事中耿通等言瑛及綱、珩朋奸蒙蔽,擅殺無辜,請罪瑛。皇太子曰:「瑛大臣,蓋為下所欺,不能覺察耳。」置勿問,械繫綱、珩,以其罪狀奏行在。又有學官坐事謫充太學膳夫者,皇太子令法司與改役,瑛格不行,中允劉子春等復劾瑛方命自恣。皇太子謂瑛曰:「卿用心刻薄,不明政體,殊非大臣之道。」時太子深惡瑛,以帝方寵任,無如何。久之,帝亦浸疏瑛。九年春,瑛得罪下獄死,天下快之。

帝以篡得天下,御下多用重典。瑛首承風旨,傾誣排陷者無算。一時臣工多效其所為,如紀綱、馬麟、丁玨、秦政學、趙緯、李芳,皆以傾險聞。綱在《佞倖傳》。

麟,鞏人。洪武末為工科給事中,建文時坐罪謫雲南為吏。成祖即位,悉復建文朝所罷官,麟得召還。尋進兵科都給事中。麟無他建白,專以訐發為能。帝久亦厭之,諭麟等曰:「奏牘一字之誤皆喋喋,煩碎甚矣。偽謬即改正,不必以聞。」麟等言:「奏內有不稱臣者,不可宥。」帝曰:「彼亦偶脫漏耳。言官當陳軍國大務,細故可略也。」久之,擢右通政。帝一日顧侍臣曰:「四方頻奏水旱,朕甚不寧。」麟遽進曰:「水旱天數,堯、湯不免。一二郡有之,未害。」帝曰:「《洪範》恒雨恒暘,皆本人事,可委天數哉?爾此言,不學故也。」麟慚而退。麟居言路,糾彈諸司無虛日。嘗署兵部事,甫一日,輒有過,為人所奏,自是稍戢。居通政八年,卒於官。

玨,山陽人。永樂四年,里社賽神,誣以聚眾謀不軌,坐死者數十人。法司因稱玨忠,特擢刑科給事中。伺察百僚小過,輒上聞。居官十年,貪黷不顧廉恥。母喪未期,起復視事,輒隨眾大祀齋宮,復與慶成宴,為御史俞信等所劾,論大不敬當死。帝曰:「朕素疑其奸邪,若悉行所言,廷臣豈有一人免耶?」遂謫戍邊。

政學,慈谿人。永樂二年進士。歷行在禮部郎中,務掇人過失,肆為奸貪。十六年春,有罪伏誅。

緯初為大興教諭,燕兵起,與城守有勞。擢禮科給事中,坐罪謫思南宣慰司教授。永樂七年,復原官,務捃摭朝士過。久之,遷浙江副使。後入朝,仁宗見其名曰:「此人尚在耶!是無異蛇蠍。」遂謫嘉興典史。

芳,潁上人。永樂十三年進士。歷刑科給事中。宣宗數御便殿,與大臣議事。芳言:「洪武中,大臣面議時政,必給事中二人與俱,請復其舊。」帝是之。芳輒自矜,百司所為,少不如意,即詣帝前奏之,人比之紀綱。久之,帝亦惡其奸,黜為海鹽丞,棄官歸。

嚴嵩,字惟中,分宜人。長身戍削,疏眉目,大音聲。舉弘治十八年進士,改庶吉士,授編修。移疾歸,讀書鈐山十年,為詩古文辭,頗著清譽。還朝,久之進侍講,署南京翰林院事。召為國子祭酒。嘉靖七年歷禮部右侍郎,奉世宗命祭告顯陵,還言:「臣恭上寶冊及奉安神床,皆應時雨霽。又石產棗陽,群鸛集繞,碑入漢江,河流驟漲。請命輔臣撰文刻石,以紀天眷。」帝大悅,從之。遷吏部左侍郎,進南京禮部尚書,改吏部。

居南京五年,以賀萬壽節至京師。會廷議更修《宋史》,輔臣請留嵩以禮部尚書兼翰林學士董其事。及夏言入內閣,命嵩還掌部事。帝將祀獻皇帝明堂,以配上帝。已,又欲稱宗入太廟。嵩與群臣議沮之,帝不悅,著《明堂或問》示廷臣。嵩惶恐,盡改前說,條畫禮儀甚備。禮成,賜金幣。自是,益務為佞悅。帝上皇天上帝尊號、寶冊,尋加上高皇帝尊謚聖號以配,嵩乃奏慶雲見,請受群臣朝賀。又為《慶雲賦》、《大禮告成頌》奏之,帝悅,命付史館。尋加太子太保,從幸承天,賞賜與輔臣埒。

嵩歸日驕。諸宗籓請恤乞封,挾取賄賂。子世蕃又數關說諸曹。南北給事、御史交章論貪污大臣,皆首嵩。嵩每被論,亟歸誠於帝,事輒已。帝或以事諮嵩,所條對平無奇,帝必故稱賞,欲以諷止言者。嵩科第先夏言,而位下之。始倚言,事之謹,嘗置酒邀言,躬詣其第,言辭不見。嵩布席,展所具啟,跽讀。言謂嵩實下己,不疑也。帝以奉道嘗御香葉冠,因刻沈水香冠五,賜言等。言不奉詔,帝怒甚。嵩因召對冠之,籠以輕紗。帝見,益內親嵩。嵩遂傾言,斥之。言去,醮祀青詞,非嵩無當帝意者。

二十一年八月拜武英殿大學士,入直文淵閣,仍掌禮部事。時嵩年六十餘矣。精爽溢發,不異少壯。朝夕直西苑板房,未嘗一歸洗沐,帝益謂嵩勤。久之,請解部事,遂專直西苑。帝嘗賜嵩銀記,文曰「忠勤敏達。」尋加太子太傅。翟鑾資序在嵩上,帝待之不如嵩。嵩諷言官論之,鑾得罪去。吏部尚書許贊、禮部尚書張璧同入閣,皆不預聞票擬事,政事一歸嵩。贊嘗歎曰:「何奪我吏部,使我旁睨人。」嵩欲示厚同列,且塞言者意,因以顯夏言短,乃請凡有宣召,乞與成國公朱希忠、京山侯崔元及贊、璧偕入,如祖宗朝謇、夏、三楊故事,帝不聽,然心益喜嵩,累進吏部尚書、謹身殿大學士、少傅兼太子太師。

久之,帝微覺嵩橫。時贊老病罷,璧死,乃復用夏言,帝為加嵩少師以慰之。言至,復盛氣陵嵩,頗斥逐其黨,嵩不能救。子世蕃方官尚寶少卿,橫行公卿間。言欲發其罪,嵩父子大懼,長跪榻下泣謝,乃已。知陸炳與言惡,遂與比而傾言。世蕃遷太常少卿,嵩猶畏言,疏遣歸省墓。嵩尋加特進,再加華蓋殿大學士。窺言失帝眷,用河套事構言及曾銑,俱棄市。已而南京吏部尚書張治、國子祭酒李本以疏遠擢入閣,益不敢預可否。嵩既傾殺言,益偽恭謹。言嘗加上柱國,帝亦欲加嵩,嵩乃辭曰:「尊無二上,上非人臣所宜稱。國初雖設此官,左相國達,功臣第一,亦止為左柱國。乞陛下免臣此官,著為令典,以昭臣節。」帝大喜,允其辭,而以世蕃為太常卿。

嵩無他才略,惟一意媚上,竊權罔利。帝英察自信,果刑戮,頗護己短,嵩以故得因事激帝怒,戕害人以成其私。張經、李天寵、王忬之死,嵩皆有力焉。前後劾嵩、世蕃者,謝瑜、葉經、童漢臣、趙錦、王宗茂、何維柏、王曄、陳塏、厲汝進、沈練、徐學詩、楊繼盛、周鈇、吳時來、張翀、董傳策皆被譴。經、煉用他過置之死,繼盛附張經疏尾殺之。他所不悅,假遷除考察以斥者甚眾,皆未嘗有跡也。

俺答薄都城,慢書求貢。帝召嵩與李本及禮部尚書徐階入對西苑。嵩無所規畫,委之禮部。帝悉用階言,稍輕嵩。嵩復以間激帝怒,杖司業趙貞吉而謫之。兵部尚書丁汝夔受嵩指,不敢趣諸將戰。寇退,帝欲殺汝夔。嵩懼其引己,謂汝夔曰:「我在,毋慮也。」汝夔臨死始知為嵩紿。

大將軍仇鸞,始為曾銑所劾,倚嵩傾銑,遂約為父子。已而鸞挾寇得帝重,嵩猶兒子蓄之,浸相惡。嵩密疏毀鸞,帝不聽,而頗納鸞所陳嵩父子過,少疏之。嵩當入直,不召者數矣。嵩見徐階、李本入西內,即與俱入。至西華門,門者以非詔旨格之。嵩還第,父子對泣。時陸炳掌錦衣,與鸞爭寵,嵩乃結炳共圖鸞。會鸞病死,炳訐鸞陰事,帝追戮之。於是益信任嵩,遣所乘龍舟過海子召嵩,載直西內如故。世蕃尋遷工部左侍郎。倭寇江南,用趙文華督察軍情,大納賄賂以遣嵩,致寇亂益甚。及胡宗憲誘降汪直、徐海,文華乃言:「臣與宗憲策,臣師嵩所授也。」遂命嵩兼支尚書俸無謝,自是褒賜皆不謝。

帝嘗以嵩直廬隘,撤小殿材為營室,植花木其中,朝夕賜御膳、法酒。嵩年八十,聽以肩輿入禁苑。帝自十八年葬章聖太后後,即不視朝,自二十年宮婢之變,即移居西苑萬壽宮,不入大內,大臣希得謁見,惟嵩獨承顧問,御札一日或數下,雖同列不獲聞,以故嵩得逞志。然帝雖甚親禮嵩,亦不盡信其言,間一取獨斷,或故示異同,欲以殺離其勢。嵩父子獨得帝窾要,欲有所救解,嵩必順帝意痛詆之,而婉曲解釋以中帝所不忍。即欲排陷者,必先稱其DG,而以微言中之,或觸帝所恥與諱。以是移帝喜怒,往往不失。士大夫輻輳附嵩,時稱文選郎中萬寀、職方郎中方祥等為嵩文武管家。尚書吳鵬、歐陽必進、高耀、許論輩,皆惴惴事嵩。

嵩握權久,遍引私人居要地。帝亦浸厭之,而漸親徐階。會階所厚吳時來、張翀、董傳策各疏論嵩,嵩因密請究主使者,下詔獄,窮治無所引。帝乃不問,而慰留嵩,然心不能無動,階因得間傾嵩。吏部尚書缺,嵩力援歐陽必進為之,甫三月即斥去。趙文華忤旨獲譴,嵩亦不能救。有詔二王就婚邸第,嵩力請留內。帝不悅,嵩亦不能力持。嵩雖警敏,能先意揣帝指,然帝所下手詔,語多不可曉,惟世蕃一覽了然,答語無不中。及嵩妻歐陽氏死,世蕃當護喪歸,嵩請留侍京邸。帝許之,然自是不得入直所代嵩票擬,而日縱淫樂於家。嵩受詔多不能答,遣使持問世蕃。值其方耽女樂,不以時答。中使相繼促嵩,嵩不得已自為之,往往失旨。所進青詞,又多假手他人不能工,經此積失帝歡。會萬壽宮火,嵩請暫徙南城離宮,南城,英宗為太上皇時所居也,帝不悅。而徐階營萬壽營甚稱旨,帝益親階,顧問多不及嵩,即及嵩,祠祀而已。嵩懼,置酒要階,使家人羅拜,舉觴屬曰:「嵩旦夕且死,此曹惟公乳哺之。」階謝不敢。

未幾,帝入方士藍道行言,有意去嵩。御史鄒應龍避雨內侍家,知其事,抗疏極論嵩父子不法,曰:「臣言不實,乞斬臣首以謝嵩、世蕃。」帝降旨慰嵩,而以嵩溺愛世蕃,負眷倚,令致仕,馳驛歸,有司歲給米百石,下世蕃於理。嵩為世蕃請罪,且求解,帝不聽。法司奏論世蕃及其子錦衣鵠、鴻,客羅龍文,戍邊遠。詔從之,特宥鴻為民,使侍嵩,而錮其奴嚴年於獄,擢應龍通政司參議。時四十一年五月也。龍文官中書,交關為奸利,而年最黠惡,士大夫競稱萼山先生者也。

嵩既去,帝追念其贊玄功,意忽忽不樂,諭階欲遂傳位,退居西內,專祈長生。階極陳河,帝曰:「卿等不欲,必皆奉君命,同輔玄修乃可。嚴嵩既退,其子世蕃已伏法,敢更言者,並應龍俱斬。」嵩知帝念己,乃賂帝左右,發道行陰事,繫刑部,俾引階。道行不承,坐論死,得釋。嵩初歸至南昌,值萬壽節,使道士藍田玉建醮鐵柱宮。田玉善召鶴,嵩因取其符籙,并己祈鶴文上之,帝優詔褒答。嵩因言:「臣年八十有四,惟一子世蕃及孫鵠皆遠戍,乞移便地就養,終臣餘年。」不許。其明年,南京御史林潤奏:「江洋巨盜多入逃軍羅龍文、嚴世蕃家。龍文居深山,乘軒衣蟒,有負險不臣之志。世蕃得罪後,與龍文日誹謗時政。其治第役眾四千,道路皆言兩人通倭,變且不測。」詔下潤逮捕,下法司論斬,皆伏誅,黜嵩及諸孫皆為民。嵩竊政二十年,溺信惡子,流毒天下,人咸指目為奸臣。其坐世蕃大逆,則徐階意也。又二年,嵩老病,寄食墓舍以死。

世蕃,短項肥體,眇一目,由父任入仕。以築京師外城勞,由太常卿進工部左侍郎,仍掌尚寶司事。剽悍陰賊,席父寵,招權利無厭。然頗通國典,曉暢時務。嘗謂天下才,惟己與陸炳、楊博為三。炳死,益自負。嵩耄昏,且旦夕直西內,諸司白事,輒曰:「以質東樓。」東樓,世蕃別號也。朝事一委世蕃,九卿以下浹日不得見,或停至暮而遣之。士大夫側目屏息,不肖者奔走其門,筐篚相望於道。世蕃熟諳中外官饒瘠險易,責賄多寡,毫髮不能匿。其治第京師,連三四坊,堰水為塘數十畝,羅珍禽奇樹其中,日擁賓客縱倡樂,雖大僚或父執,虐之酒,不困不已。居母喪亦然。好古尊彝、奇器、書畫,趙文華、鄢懋卿、胡宗憲之屬,所到輒輦致之,或索之富人,必得然後已。被應龍劾戍雷州,未至而返,益大治園亭。其監工奴見袁州推官郭諫臣,不為起。

御史林潤嘗劾懋卿,懼相報,因與諫臣謀發其罪,且及冤殺楊繼盛、沈練狀。世蕃喜,謂其黨曰:「無恐,獄且解。」法司黃光昇等以讞詞白徐階,階曰:「諸公欲生之乎?」僉曰:必欲死之。」曰:「若是,適所以生之也。夫楊、沈之獄,嵩皆巧取上旨。今顯及之,是彰上過也。必如是,諸君且不測,嚴公子騎款段出都門矣。」為手削其草,獨按龍文與汪直姻舊,為交通賄世蕃乞官。世蕃用彭孔言,以南昌倉地有王氣,取以治第,制擬王者。又結宗人典楧陰伺非常,多聚亡命。龍文又招直餘黨五百人,謀為世蕃外投日本,先所發遣世蕃班頭牛信,亦自山海衛棄伍北走,誘致外兵,共相響應。即日令光昇等疾書奏之。世蕃聞,詫曰:「死矣。」遂斬於市。籍其家,黃金可三萬餘兩,白金二百萬餘兩,他珍寶服玩所直又數百萬。

趙文華,慈谿人。嘉靖八年進士。授刑部主事。以考察謫東平州同知。久之,累官至通政使。性傾狡,未第時在國學,嚴嵩為祭酒,才之。後仕於朝,而嵩日貴幸,遂相與結為父子。嵩念己過惡多,得私人在通政,劾疏至,可預為計,故以文華任之。文華欲自結於帝,進百華仙酒,詭曰:「臣師嵩服之而壽。」帝飲甘之,手敕問嵩。嵩驚曰;「文華安得為此!」乃宛轉奏曰:「臣生平不近藥餌,犬馬之壽誠不知何以然。」嵩恨文華不先白己,召至直所詈責之。文華跪泣,久不敢起。徐階、李本見之為解,乃令去。嵩休沐歸,九卿進謁,嵩猶怒文華,令從吏扶出之。文華大窘,厚賂嵩妻。嵩妻教文華伺嵩歸,匿於別室,酒酣,嵩妻為之解,文華即出拜,嵩乃待之如初。以建議築京師外城,加工部右侍郎。

東南倭患棘,文華獻七事。首以祭海神為言,請遣官望祭於江陰、常熟。次訟有司掩骼輕徭。次增募水軍。次蘇、松、常、鎮民田,一夫過百畝者,重科其賦,且預徵官田稅三年。次募富人輸財力自效,事寧論功。次遣重臣督師。次招通番舊黨並海鹽徒,易以忠義之名,令偵伺賊情,因以為間。兵部尚書聶豹議行其五事,惟增田賦、遣重臣二事不行。帝怒,奪豹官,而用嵩言即遣文華祭告海神,因察賊情。當是時,總督尚書張經方征四方及狼士兵,議大舉,自以位文華上,心輕之。文華不悅。狼兵稍有斬獲功,文華厚犒之,使進剿,至漕涇戰敗,亡頭目十四人。文華恚,數趣經進兵。經慮文華輕淺洩師期,不以告。文華益怒,劾經養寇失機,疏方上,經大捷王江涇。文華攘其功,謂己與巡按胡宗憲督師所致,經竟論死。又劾浙江巡撫李天寵罪,薦宗憲代,天寵亦論死。帝益以文華為賢,命鑄督察軍務關防,即軍中賜之。文華自此出總督上,益恣行無忌。欲分蘇松巡撫曹邦輔滸墅關破賊功,不得,則以陶宅之敗,重劾邦輔。陶宅之戰,實文華、宗憲兵先潰也。兵科給事中夏栻得其情,劾文華欺誕。吏科給事中孫浚亦白邦輔冤狀。帝終信文華言,邦輔坐遣戍。文華既殺經、天寵,復先後論罷總督周琉、楊宜,至是又傾邦輔,勢益張。文武將吏爭輸貨其門,顛倒功罪,牽制兵機,紀律大乖,將吏人人解體,徵兵半天下,賊寇愈熾。文華又陳防守事宜,請籍閒田百萬畝給兵,為屯守計,而令里居縉紳,分督郡邑兵事。為兵部所駁而寢。

官軍既屢敗,文華知賊未易平,欲委責去。會川兵破賊周浦,俞大猷破賊海洋,文華遂言水陸成功,江南清晏,請還朝。帝悅,許之。比還,敗報踵至,帝疑其妄,數詰嵩,嵩曲為解,帝意終不釋。會吏部尚書李默發策試選人,中言「漢武征四夷,而海內虛耗。唐憲復淮、蔡,而晚業不終。」文華劾其謗訕,默坐死。帝以是謂文華忠,進工部尚書,且加太子太保。是時,嵩年老,慮一旦死,有後患,因薦文華文學,宜供奉青詞,直內閣。帝不許。而東南警遝至,部議再遣大臣督師,已命兵部侍郎沈良材矣,嵩令文華自請行,為帝言江南人矯首望文華。帝以為然,命兼右副都御史,總督江南、浙江諸軍事。時宗憲先以文華薦代楊宜為總督,及文華再出,宗憲欲藉文華以通於嵩,諂奉無不至。文華素不知兵,亦倚宗憲,兩人交甚歡。已而宗憲平徐海,俘陳東,文華以大捷聞,歸功上玄。帝大喜,祭告郊廟社稷,加文華少保,蔭子錦衣千戶。召還朝,文華乃推功元輔嵩,辭升蔭,帝優詔不允。

文華既寵貴,志日驕,事中貴及世蕃,漸不如初,諸人憾之。帝嘗遣使賜文華,值其醉,拜跪不如禮,帝聞惡其不敬。又嘗進方士藥,帝服之盡,使小璫再索之,不應。西苑造新閣,不以時告成。帝一日登高,見西長安街有高甍,問誰宅。左右曰:「趙尚書新宅也。」旁一人曰:「工部大木,半為文華作宅,何暇營新閣。」帝益慍。會三殿災,帝欲建正陽門樓,責成甚亟,文華猝不能辦。帝積怒,且聞其連歲視師黷貨要功狀,思逐之,乃諭嵩曰:「門樓庀材遲,文華似不如昔。」嵩猶未知帝意,力為掩覆,且言:「文華觸熱南征,因致疾,宜增侍郎一人專督大工。」帝從之。文華因上章稱疾,請賜假靜攝旬月。帝手批曰:「大工方興,司空是職。文華既有疾,可回籍休養。」制下,舉朝相賀。

帝雖逐文華猶以為未盡其罪,而言官無攻者,帝怒無所洩。會其子錦衣千戶懌思以齋祀停封章日請假送父,帝大怒,黜文華為民,戍其子邊衛。以禮科失糾劾,令對狀。於是都給事中謝江以下六人,並廷杖削籍。文華故病蠱,及遭譴臥舟中,意邑邑不自聊,一夕手捫其腹,腹裂,臟腑出,遂死。後給事中羅嘉賓等核軍餉,文華所侵盜以十萬四千計。有詔徵諸其家,至萬曆十一年征猶未及半,有司援恩詔祈免。神宗不許,戍其子慎思於煙瘴地。

鄢懋卿,豐城人。由行人擢御史,屢遷大理少卿。三十五年,轉左僉都御史。尋進左副都御史。懋卿以才自負,見嚴嵩柄政,深附之,為嵩父子所暱。會戶部以兩浙、兩淮、長蘆、河東鹽政不舉,請遣大臣一人總理,嵩遂用懋卿。舊制,大臣理鹽政,無總四運司者。至是懋卿盡握天下利柄,倚嚴氏父子,所至市權納賄,監司郡邑吏膝行蒲伏。

懋卿性奢侈,至以文錦被廁床,白金飾溺器。嵊時遺嚴氏及諸權貴,不可勝紀。其按部,常與妻偕行,製五彩輿,令十二女子舁之,道路傾駭。淳安知縣海瑞、慈谿知縣霍與瑕,以抗忤罷去。御史林潤嘗劾懋卿要索屬吏,饋遺巨萬,濫受民訟,勒富人賄,置酒高會,日費千金,虐殺不辜,怨咨載路,苛斂淮商,幾至激變,五大罪。帝置不問。四十年召為刑部右侍郎。兩淮餘鹽,歲徵銀六十萬兩,及懋卿增至一百萬。懋卿去,巡鹽御史徐爌極言其害,乃復六十萬之舊。嵩敗,御史鄭洛劾懋卿及大理卿萬寀朋奸黷貨,兩人皆落職。既而寀匿嚴氏銀八萬兩,懋卿紿得其二萬,事皆露,兩人先後戍邊。

時坐嚴氏黨被論者,前兵部右侍郎柏鄉魏謙吉、工部左侍郎南昌劉伯躍、南京刑部右侍郎德安何遷、右副都御史信陽董威、僉都御史萬安張雨、應天府尹祥符孟淮、南京光祿卿南昌胡植、南京光祿少卿武進白啟常、右諭德蘭谿唐汝楫、南京太常卿掌國子監事新城王材、太僕丞新喻張春及嵩婿廣西副使袁應樞等數十人,黜謫有差。植與嵩鄉里,嘗勸嵩殺楊繼盛。啟常官禮部郎,匿喪遷光祿,與材、汝楫俱為世蕃狎客。啟常至以粉墨塗面供歡笑。而材、汝楫俱出入嵩臥內,關通請屬,尤為人所惡云。

周延儒,字玉繩,宜興人。萬曆四十一年會試、殿試皆第一。授修撰,年甫二十餘。美麗自喜,與同年生馮銓友善。天啟中,遷右中允,掌司經局事。尋以少詹事掌南京翰林院事。

莊烈帝即位,召為禮部右侍郎。延儒性警敏,善伺意指。崇禎元年冬,錦州兵嘩,督師袁崇煥請給餉。帝御文華殿,召問諸大臣,皆請發內帑。延儒揣帝意,獨進曰:「關門昔防敵,今且防兵。寧遠嘩,餉之,錦州嘩,復餉之,各邊且效尤。」帝曰:「卿謂何如?」延儒曰:「事迫,不得不發。但當求經久之策。」帝頷之,降旨責群臣。居數日,復召問,延儒曰:「餉莫如粟,山海粟不缺,缺銀耳。何故嘩?嘩必有隱情,安知非驕弁構煽以脅崇煥邪?」帝方疑邊將要挾,聞延儒言,大說,由此屬意延儒。十一月,大學士劉鴻訓罷,命會推,廷臣以延儒望輕置之,列成基命、錢謙益、鄭以偉、李騰芳、孫慎行、何如寵、薛三省、盛以弘、羅喻義、王永光、曹于汴十一人名上。帝以延儒不預,大疑。及溫體仁訐謙益,延儒助之。帝遂發怒,黜謙益,盡罷會推者不用。二年三月召對延儒於文華殿,漏下數十刻乃出,語秘不得聞。御史黃宗昌劾其生平穢行,御史李長春論獨對之非。延儒乞罷,不允。南京給事中錢允鯨言:「延儒與馮銓密契,延儒柄政,必為逆黨翻局。」延儒疏辨,帝優詔褒答。其年十二月,京師有警,特旨拜延儒禮部尚書兼東閣大學士,參機務。明年二月加太子太保,改文淵閣。六月,體仁亦入。九月,成基命致仕,延儒遂為首輔。尋加少保,改武英殿。

體仁既並相,務為柔佞,帝意漸響之。而體仁陽曲謹媚延儒,陰欲奪其位,延儒不知也。體仁與吏部尚書王永光謀起逆案王之臣、呂純如等。或謂延儒曰:「彼將翻逆案,而外歸咎於公。」延儒愕然。會帝以之臣問,延儒曰:「用之臣,亦可雪崔呈秀矣。」帝悟而止。體仁益欲傾延儒。四年春。延儒姻婭陳于泰廷對第一,及所用大同巡撫張廷拱、登萊巡撫孫元化皆有私,時論籍籍。其子弟家人暴邑中,邑中民熱其廬,發其先壟,為言官所糾。兄素儒冒錦衣籍,授千戶,又用家人周文郁為副總兵,益為言者所詆。

五年正月,叛將李九成等陷登州,囚元化。侍郎劉宇烈視師無功,言路咸指延儒庇宇烈。於是給事中孫三傑、馮元飆,御史餘應桂、衛景瑗、尹明翼、路振飛、吳執御、王道純、王象雲等,屢劾延儒。應桂並謂延儒納巨盜神一魁賄。而監視中官鄧希詔與總督曹文衡相訐奏,語侵延儒。給事中李春旺亦論延儒當去。延儒數上疏辯,帝雖慰留,心不能無動。已而延儒令于泰陳時政四事,宣府太監王坤承體仁指,直劾延儒庇于泰。給事中傅朝佑言中官不當劾首揆,輕朝廷,疑有邪人交構,副都御史王志道亦言之。帝怒,削志道籍,延儒不能救。體仁各處嗾給事中陳贊化劾延儒「暱武弁李元功等,招搖罔利。陛下特恩停刑,元功以為延儒功,索獄囚賕謝。而延儒至目陛下為羲皇上人,語誖逆。」帝怒,下元功詔獄,且窮詰贊化語所自得。贊化言得之上林典簿姚孫渠、給事中李世祺,而副使張鳳翼亦具述延儒語。帝益怒。錦衣衛帥王世盛拷掠元功無所承。獄上,鐫世盛五級,令窮治其事。延儒覬體仁為援,體仁卒不應,且陰黜與延儒善者,延儒大困。六年六月引疾乞歸,賜白金、彩緞,遣行人護行。體仁遂為首輔矣。

始延儒里居,頗從東林游,善姚希孟、羅喻義。既陷錢謙益,遂仇東林。及主會試,所取士張溥、馬世奇等,又皆東林也。至是歸,失勢,心內慚。而體仁益橫,越五年始去。去而張至發、薛國觀相繼當國,與楊嗣昌等並以娼嫉稱。一時正人鄭三俊、劉宗周、黃道周等,皆得罪。溥等憂之,說延儒曰:「公若再相,易前轍,可重得賢聲。」延儒以為然。溥友吳昌時為交關近侍,馮銓復助為謀。會帝亦頗思延儒,而國觀適敗。十四年二月詔起延儒。九月至京,復為首輔。尋加少師兼太子太師,進吏部尚書、中極殿大學士。

延儒被召,溥等以數事要之。延儒慨然曰:「吾當銳意行之,以謝諸公。」既入朝,悉反體仁輩弊政。首請釋漕糧白糧欠戶,蠲民間積逋,凡兵殘歲荒地,減見年兩稅。蘇、松、常、嘉、湖諸府大水,許以明年夏麥代漕糧。宥戍罪以下,皆得還家。復注誤舉人,廣取士額及召還言事遷謫諸臣李清等。帝皆忻然從之。延儒又言:「老成名德,不可輕棄。」於是鄭三俊長吏部,劉宗周掌都察院,范景文長工部,倪元璐佐兵部,皆起自廢籍。其他李邦華、張國維、徐石麒、張瑋、金光辰等,布滿九列。釋在獄傅宗龍等,贈已故文震孟、姚希孟等官。中外翕然稱賢。嘗燕侍,帝語及黃道周,時道周方謫戍辰州。延儒曰:「道周氣質少偏,然學與守皆可用。」蔣德璟請移道周戍近地。延儒曰:「上欲用即用之耳,何必移戍。」帝即日復道周官。其因事開釋如此。

帝尊禮延儒特重,嘗於歲首日東向揖之,曰:「朕以天下聽先生。」因遍及諸閣臣。然延儒實庸駑無材略,且性貪。當邊境喪師,李自成殘掠河南,張獻忠破楚、蜀,天下大亂,延儒一無所謀畫。用侯恂、范志完督師,皆僨事,延儒無憂色。而門下客盛順、董廷獻因緣為奸利。又信用文選郎吳昌時及給事中曹良直、廖國遴、楊枝起、曾應遴輩。

昌時,嘉興人。有幹材,頗為東林效奔走。然為人墨而傲,通廠衛,把持朝官,同朝咸嫉之。行人司副熊開元廷劾延儒納賄狀,觸帝怒,與給事中姜埰俱廷杖,下詔獄。左都御史宗周、僉都御史光辰以救開元、埰罷,尚書石麒又以救宗周等罷,延儒皆弗救,朝議皆以咎延儒。會昌時以年例出言路十人於外,言路大嘩。掌科給事中吳麟徵、掌道御史祁彪佳劾昌時挾勢弄權,延儒頗不自安。

初,延儒奏罷廠衛緝事,都人大悅。朝士不肖者因通賂遺,而廠衛以失權,胥怨延儒。又傲同官陳演,演銜刺骨。掌錦衣者駱養性,延儒所薦也,養性狡狠背延儒,與中官結,刺延儒陰事。十六年四月,大清兵略山東,還至近畿,帝憂甚。大學士吳甡方奉命辦流寇,延儒不得已自請視師。帝大喜,降手敕,獎以召虎、裴度,賜章服、白金、文綺、上駟,給金帛賞軍。延儒駐通州不敢戰,惟與幕下客飲酒娛樂,而日騰章奏捷,帝輒賜璽書褒勵。偵大清兵去,乃言敵退,請下兵部議將吏功罪。既歸朝,繳敕諭,帝即令藏貯,以識勳勞。論功,加太師,廕子中書舍人,賜銀幣、蟒服。延儒辭太師,許之。居數日,養性及中官盡發所刺軍中事。帝乃大怒,諭府部諸臣責延儒蒙蔽推諉,事多不忍言,令從公察議。陳演等公揭救之,延儒席槁待罪,自請戍邊。帝猶降溫旨,言「卿報國盡忱,終始勿替,」許馳驛歸,賜路費百金,以彰保全優禮之意。及廷臣議上,帝復諭延儒功多罪寡,令免議。延儒遂歸。

既去,給事中郝絅疏請除奸,以指延儒。帝不聽。山東僉事雷縯祚糾范志完,亦及延儒。已而御史蔣拱宸劾吳昌時贓私巨萬,大抵牽連延儒,而中言昌時通中官李端、王裕民,洩漏機密,重賄入手,輒預揣溫旨告人。給事中曹良直亦劾延儒十大罪。帝怒甚,御中左門,親鞫昌時,折其脛,無所承,怒不解,拱宸面訐其通內,帝察之有迹,乃下獄論死,始有意誅延儒。初,薛國觀賜死,謂昌時致之。其門人魏藻德新入閣有寵,恨昌時甚,因與陳演共排延儒,養性復騰蜚語。帝遂命盡削延儒職,遣緹騎逮入京師。時舊輔王應熊被召,延儒知帝怒甚,宿留道中,俟應熊先入,冀為請。帝知之,應熊既抵京,命之歸。延儒至,安置正陽門外古廟,上疏乞哀,不許。法司以戍請,同官申救,皆不許。冬十二月,昌時棄市,命勒延儒自盡,籍其家。

溫體仁,字長卿,烏程人。萬曆二十六年進士。改庶吉士,授編修,累官禮部侍郎。崇禎初遷尚書,協理詹事府事。為人外謹而中猛鷙,機深刺骨。

崇禎元年冬,詔會推閣臣,體仁望輕,不與也。侍郎周延儒方以召對稱旨,亦弗及。體仁揣帝意必疑,遂上疏訐謙益關節受賄,神奸結黨,不當與閣臣選。先是,天啟二年,謙益主試浙江,所取士錢千秋者,首場文用俚俗詩一句,分置七義結尾,蓋奸人紿為之。為給事中顧其仁所摘,謙益亦自發其事。法司戍千秋及奸人,奪謙益俸,案久定矣。至是體仁復理其事,帝心動。次日,召對閣部科道諸臣於文華殿,命體仁、謙益皆至。謙益不虞體仁之劾己也,辭頗屈,而體仁盛氣詆謙益,言如湧泉,因進曰:「臣職非言官不可言,會推不與,宜避嫌不言,但枚卜大典,宗社安危所係。謙益結黨受賄,舉朝無一人敢言者,臣不忍見皇上孤立於上,是以不得不言。」帝久疑廷臣植黨,聞體仁言,輒稱善。而執政皆言謙益無罪,吏科都給事中章允儒爭尤力,且言:「體仁熱中觖望,如謙益當糾,何俟今日。」體仁曰:「前此,謙益皆閒曹,今者糾之,正為朝廷慎用人耳。如允儒言,乃真黨也。」帝怒,命禮部進千秋卷,閱意,責謙益,謙益引罪。歎曰:「微體仁,朕幾誤!」遂叱允儒下詔獄,并切責諸大臣。時大臣無助體仁者,獨延儒奏曰:「會推名雖公,主持者止一二人,餘皆不敢言,即言,徒取禍耳。且千秋事有成案,不必復問諸臣。」帝乃即日罷謙益官,命議罪。允儒及給事中瞿式耜、御史房可壯等,皆坐謙益黨,降謫有差。

亡何,御史毛九華劾體仁居家時,以抑買商人木,為商人所訴,賂崔呈秀以免。又困杭州建逆祠,作詩頌魏忠賢。帝下浙江巡撫核實。明年春,御史任贊化亦劾體仁娶娼、受金,奪人產諸不法事。帝怒其語褻,貶一秩調外。體仁乞罷,因言:「比為謙益故,排擊臣者百出。而無一人左袒臣,臣孤立可見。」帝再召內閣九卿質之,體仁與九華、贊化詰辯良久,言二人皆謙益死黨。帝心以為然,獨召大學士韓爌等於內殿,諭諸臣不憂國,惟挾私相攻,當重繩以法。體仁復力求去以要帝,帝優詔慰答焉。已,給事中祖重曄、南京給事中錢允鯨、南京御史沈希詔相繼論體仁熱中會推,劫言者以黨,帝皆不聽。法司上千秋獄,言謙益自發在前,不宜坐。詔令再勘。體仁復疏言獄詞皆出謙益手。於是刑部尚書喬允升,左都御史曹于汴,大理寺卿康新民,太僕寺卿蔣允儀,府丞魏光緒,給事中陶崇道,御史吳甡、樊尚璟、劉廷佐,各疏言:「臣等雜治千秋,觀聽者數千人,非一手一口所能掩。體仁顧欺岡求勝。」體仁見于汴等詞直,乃不復深論千秋事,惟詆于汴等黨護而已。謙益坐杖論贖,而九華所論體仁媚璫詩,亦卒無左驗。當是時,體仁以私憾撐拒諸大臣,展轉不肯詘。帝謂體仁孤立,益響之。未幾,延儒入閣。其明年六月,遂命體仁以禮部尚書兼東閣大學士。

體仁既藉延儒力得輔政,勢益張。踰年,吏部尚書王永光去,用其鄉人閔洪學代之,凡異己者,率以部議論罷,而體仁陰護其事。又用御史史褷、高捷及侍郎唐世濟、副都御史張捷等為腹心,忌延儒居己上,并思傾之。初,帝殺袁崇煥,事牽錢龍錫,論死。體仁與延儒、永光主之,將興大獄,梁廷棟不敢任而止,事詳龍錫傳。比龍錫減死出獄,延儒言帝盛怒解救殊難,體仁則佯曰:「帝固不甚怒也。」善龍錫者,因薄延儒。其後太監王坤、給事中陳贊化先後劾延儒,體仁默為助,延儒遂免歸。始與延儒同入閣者何如寵,錢象坤踰歲致政去,無何,如寵亦去。延儒既罷,廷臣惡體仁當國,勸帝復召如寵。如寵屢辭,給事中黃紹傑言:「君子小人不並立,如寵瞻顧不前,則體仁宜思自處。」帝為謫紹傑於外,如寵卒辭不入,體仁遂為首輔。

體仁荷帝殊寵,益忮橫,而中阻深。所欲推薦,陰令人發端,己承其後。欲排陷,故為寬假,中上所忌,激使自怒。帝往往為之移,初未嘗有迹。姚希孟為講官,以才望遷詹事。體仁惡其偪,乃以冒籍武生事,奪希孟一官,使掌南院去。禮部侍郎羅喻義,故嘗與基命、謙益同推閣臣,有物望。會進講章中有「左右未得人」語,體仁欲去之,喻義執不可。體仁因自劾:「日講進規例從簡,喻義駁改不從,由臣不能表率。」帝命吏部議,洪學等因謂:「聖聰天亶,何俟喻義多言。」喻義遂罷歸。時魏忠賢遺黨日望體仁翻逆案,攻東林。會吏部尚書、左都御史缺,體仁陰使侍郎張捷舉逆案呂純如以嘗帝。言者大嘩,帝亦甚惡之。捷氣沮,體仁不敢言,乃薦謝升、唐世濟為之。世濟尋以薦逆案霍維華得罪去。維華之薦,亦體仁主之也,體仁自是不敢訟言用逆黨,而愈側目諸不附己者。

文震孟以講《春秋》稱旨,命入閣。體仁不能沮,薦其黨張至發以間之,而日伺震孟短,遂用給事中許譽卿事,逐之去。先是,秦、楚盜起,議設五省總督,兵部侍郎彭汝楠、汪慶百當行,憚不敢往,體仁庇二人,罷其議。賊犯鳳陽,南京兵部尚書呂維祺等議,令淮撫、操江移鎮,體仁又卻不用。既而賊大至,焚皇陵。譽卿言:「體仁納賄庇私,貽憂要地,以皇陵為孤注,使原廟震驚,誤國孰大焉。」體仁素忌譽卿,見疏益憾。會謝陞以營求北缺劾譽卿,體仁擬旨降調,而故重其詞。帝果命削籍,震孟力爭之,大學士何吾騶助為言。體仁訐奏震孟語,謂言官罷斥為至榮,蓋以朝廷賞罰為不足懲勸,悖理蔑法。帝遂逐震孟並罷吾騶。震孟既去,體仁憾未釋。庶吉士鄭鄤與震孟同建言,相友善也,其從母舅大學士吳宗達謝政歸。體仁劾鄤假乩仙判詞,逼父振先杖母,言出宗達。帝震怒,下鄤獄。其後體仁已去,而帝怒鄤甚,不俟左證,磔死。滋陽知縣成德,震孟門人,以彊直忤巡按御史禹好善,被誣劾,震孟為不平。體仁劾德,杖戍之。

體仁輔政數年,念朝士多與為怨,不敢恣肆,用廉謹自結於上,苞苴不入門。然當是時,流寇躪畿輔,擾中原,邊警雜沓,民生日困,未嘗建一策,惟日與善類為仇。誠意伯劉孔昭劾倪元璐,給事中陳啟新劾黃景昉,皆奉體仁指。禮部侍郎陳子壯嘗面責體仁,尋以議宗籓事忤帝指,竟下獄削籍。其所引與同列者,皆庸材,茍以充位,且藉形己長,固上寵。帝每訪兵餉事,輒遜謝曰:「臣夙以文章待罪禁林,上不知其駑下,擢至此位。盜賊日益眾,誠萬死不足塞責。顧臣愚無知,但票擬勿欺耳。兵食之事,惟聖明裁決。」有詆其窺帝意旨者,體仁言:「臣票擬多未中窾要,每經御筆批改,頌服將順不暇,詎能窺上旨。」帝以為樸忠,愈親信之。

自體仁輔政後,同官非病免物故,即以他事去。獨體仁居位八年,官至少師兼太子太師,進吏部尚書、中極殿大學士,階左柱國,兼支尚書俸,恩禮優渥無與比。而體仁專務刻核,迎合帝意。帝以皇陵之變,從子壯言,下詔寬恤在獄諸臣,吏部以百餘人名上。體仁靳之,言於帝,僅釋十餘人。秋決論囚,帝再三諮問,體仁略無平反。陜西華亭知縣徐兆麟涖任甫七日,以城陷論死,帝頗疑之。體仁不為救,竟棄市。帝憂兵餉急,體仁惟倡眾捐俸助馬修城而已。所上密揭,帝率報可。

體仁自念排擠者眾,恐怨歸己,倡言密勿之地,不宜宣洩,凡閣揭皆不發,并不存錄閣中,冀以滅迹,以故所中傷人,廷臣不能盡知。當國既久,劾者章不勝計,而劉宗周劾其十二罪、六奸,皆有指實。宗籓如唐王聿鍵,勳臣如撫寧侯朱國弼,布衣如何儒顯、楊光先等,亦皆論之,光先至輿櫬待命。帝皆不省,愈以為孤立,每斥責言者以慰之,至有杖死者。庶吉士張溥、知縣張采等倡為復社,與東林相應和。體仁因推官周之夔及奸人陸文聲訐奏,將興大獄。嚴旨察治,以提學御史倪元珙、海道副使馮元颺不承風指,皆降謫之。最後復有張漢儒訐錢謙益、瞿式耜居鄉不法事。體仁故仇謙益,擬旨逮二人下詔獄嚴訊。謙益等危甚,求解於司禮太監曹化淳。漢儒偵知之,告體仁。體仁密奏帝,請并坐化淳罪。帝以示化淳,化淳懼,自請案治,乃盡得漢儒等奸狀及體仁密謀。獄上,帝始悟體仁有黨。會國弼再劾體仁,帝命漢儒等立枷死。體仁乃佯引疾,意帝必慰留。及得旨竟放歸,體仁方食,失匕箸,時十年六月也。踰年卒,帝猶惜之,贈太傅,謚文忠。

崇禎末,福王立於南京,以尚書顧錫疇議,削其贈謚,天下快焉。尋用給事中戴英言,復如初。體仁雖前死,其所推薦張至發、薛國觀之徒,皆效法體仁,蔽賢植黨,國事日壞,以至於亡。

馬士英,貴陽人。萬曆四十四年,與懷寧阮大鋮同中會試。又三年,士英成進士,授南京戶部主事。天啟時,遷郎中,歷知嚴州、河南、大同三府。崇禎三年,遷山西陽和道副使。五年,擢右僉都御史,巡撫宣府。到官甫一月,檄取公帑數千金,饋遺朝貴,為鎮守太監王坤所發,坐遣戍。尋流寓南京。時大鋮名掛逆案,失職久廢,以避流賊至,與士英相結甚歡。

大鋮機敏猾賊,有才藻。天啟初,由行人擢給事中,以憂歸。同邑左光斗為御史有聲,大鋮倚為重。四年春,吏科都給事中缺,大鋮次當遷,光斗招之。而趙南星、高攀龍、楊漣等以察典近,大鋮輕躁不可任,欲用魏大中。大鋮至,使補工科。大鋮心恨,陰結中璫寢推大中疏。吏部不得已,更上大鋮名,即得請。大鋮自是附魏忠賢,與霍維華、楊維垣、倪文煥為死友,造《百官圖》,因文煥達諸忠賢。然畏東林攻己,未一月遽請急歸。而大中掌吏科,大鋮憤甚,私謂所親曰:「我猶善歸,未知左氏何如耳。」已而楊、左諸人獄死,大鋮對客詡詡自矜。尋召為太常少卿,至都,事忠賢極謹,而陰慮其不足恃,每進謁,輒厚賄忠賢閽人,還其刺。居數月,復乞歸。忠賢既誅,大鋮函兩疏馳示維垣。其一專劾崔、魏。其一以七年合算為言,謂天啟四年以後,亂政者忠賢,而翼以呈秀,四年以前,亂政者王安,而翼以東林。傳語維垣,若時局大變,上劾崔、魏疏,脫未定,則上算疏。會維垣方並指東林、崔、魏為邪黨,與編修倪元璐相詆,得大鋮疏,大喜,為投合算疏以自助。崇禎元年,起光祿卿。御史毛羽健劾其黨邪,罷去。明年定逆案,論贖徒為民,終莊烈帝世,廢斥十七年,DHDH不得志。

流寇偪皖,大鋮避居南京,頗招納遊俠為談兵說劍,覬以邊才召。無錫顧杲、吳縣楊廷樞、蕪湖沈士柱、餘姚黃宗羲、鄞縣萬泰等,皆復社中名士,方聚講南京,惡大鋮甚,作《留都防亂揭》逐之。大鋮懼,乃閉門謝客,獨與士英深相結。周延儒內召,大鋮輦金錢要之維揚,求湔濯。延儒曰:「吾此行,謬為東林所推。子名在逆案,可乎?」大鋮沉吟久之,曰:「瑤草何如?」瑤草,士英別字也,延儒許之。十五年六月,鳳陽總督高斗光以失五城逮治。禮部侍郎王錫兗薦士英才,延儒從中主之,遂起兵部右侍郎兼右僉都御史,總督廬、鳳等處軍務。

永城人劉超者,天啟中以征安邦彥功,積官至四川遵義總兵官,坐罪免,數營復官不得。李自成圍開封,超請募士冠協擊,乃用為保定總兵官,令率兵赴救。超憚不敢行,宿留家中,以私怨殺御史魏景琦等三家,遂據城反。巡撫王漢討之,被殺。帝乃命士英偕太監盧九德、河南總兵官陳永福進討。明年四月,圍其城,連戰,賊屢挫,築長圍困之。超官貴州時,與士英相識,緣舊好乞降。士英佯許之,超出見,不肯去佩刀。士英笑曰:「若既歸朝,安用此?」手解其刀。已,潛去其親信,遂就縛。獻俘於朝,磔死。時流寇充斥,士英捍禦數有功。

十七年三月,京師陷,帝崩,南京諸大臣聞變,倉卒議立君。而福王由崧、潞王常淓俱避賊至淮安,倫序當屬福王。諸大臣慮福王立,或追怨「妖書」及「挺擊」、「移宮」等案;潞王立,則無後患,且可邀功。陰主之者,廢籍禮部侍郎錢謙益,力持其議者兵部侍郎呂大器,而右都御史張慎言、詹事姜曰廣皆然之。前山東按察使僉事雷演祚、禮部員外郎周鑣往來遊說。時士英督師廬、鳳,獨以為不可,密與操江誠意伯劉孔昭,總兵高傑、劉澤清、黃得功、劉良佐等結,而公致書於參贊機務兵部尚書史可法,言倫序親賢,無如福王。可法意未決。及廷臣集議,吏科給事中李沾探士英指,面折大器。士英亦自廬、鳳擁兵迎福王至江上,諸大臣乃不敢言。王之立,士英力也。

當王監國時,廷推閣臣,劉孔昭攘臂欲得之,可法折以勳臣無入閣例。孔昭乃訟言:「我不可,士英何不可?」於是進士英東閣大學士兼兵部尚書、都察院右副都御史,與可法及戶部尚書高弘圖並命,士英仍督師鳳陽。士英大慍,令高傑、劉澤清等疏趣可法督師淮、揚,而士英留輔政,仍掌兵部,權震中外。尋論定策功,加太子太師,蔭錦衣衛指揮僉事。九月,敘江北歷年戰功,加少傅兼太子太師、建極殿大學士,蔭子如前。十二月,進少師。明年,進太保。當是時,中原郡縣盡失,高傑死睢州,諸鎮權侔無統。左良玉擁兵上流,跋扈有異志。而士英為人貪鄙無遠略,復引用大鋮,日事報復,招權罔利,以迄於亡。

初,可法、弘圖及姜曰廣、張慎言等皆宿德在位,將以次引海內人望,而士英必欲起大鋮。有詔廣搜人材,獨立逆案不可輕議。士英令孔昭及侯湯國祚、伯趙之龍等攻慎言去之,而薦大鋮知兵。初,大鋮在南京,與守備太監韓贊周暱。京師陷,中貴人悉南奔,大鋮因贊周遍結之,為群奄言東林當日所以危貴妃、福王者,俾備言於王,以潛傾可法等。群奄更極口稱大鋮才,士英亦言大鋮從山中致書與定策謀,為白其附璫贊導無實跡。遂命大鋮冠帶陛見。大鋮乃上守江策,陳三要、兩合、十四隙疏,并自白孤忠被陷,痛詆孫慎行、魏大中、左光斗,且指大中為大逆。於是大學士姜曰廣、侍郎呂大器、懷遠侯常延齡等並言大鋮逆案巨魁,不可召。士英為大鋮奏辨,力攻曰廣、大器,益募宗室統昚、建安王統鏤輩,連疏交攻。而以大學士高弘圖為御史時嘗詆東林,必當右己,乃言「弘圖素知臣者。」弘圖則言先帝欽定逆案一書,不可擅改。士英與爭,弘圖因乞罷。士英意稍折,遲回月餘,用安遠侯柳祚昌薦,中旨起大鋮兵部添註右侍郎。左都御史劉宗周言:「殺大中者魏璫,大鋮其主使也。即才果足用,臣慮黨邪害正之才,終病世道。大鋮進退,實係江左興亡,乞寢成命。」有旨切責。未幾,大鋮兼右僉都御史,巡閱江防。尋轉左侍郎。明年二月進本部尚書兼右副都御史,仍閱江防。

呂大器、姜曰廣、劉宗周、高弘圖、徐石麒皆與士英齟齬,先後罷歸。士英獨握大柄,內倚中官田成輩,外結勛臣劉孔昭、朱國弼、柳祚昌,鎮將劉澤清、劉良佐等,而一聽大鋮計。盡起逆案中楊維垣、虞廷陛、郭如闇、周昌晉、虞大復、徐復陽、陳以瑞、吳孔嘉;其死者悉予贈恤,而與張捷、唐世濟等比;若張孫振、袁弘勛、劉光斗皆得罪先朝,復置言路為爪牙。朝政濁亂,賄賂公行。四方警報狎至,士英身掌中樞,一無籌畫,日以鋤正人引兇黨為務。

初,舉朝以逆案攻大鋮,大鋮憾甚。及見北都從逆諸臣有附會清流者,因倡言曰:「彼攻逆案,吾作順案與之對。」以李自成偽國號曰順也。士英因疏糾從逆光時亨等;時亨名附東林,故重劾之。大鋮又誣逮顧杲及左光斗弟光先下獄,劾周鑣、雷縯祚殺之。時有狂僧大悲出語不類,為總督京營戎政趙之龍所捕。大鋮欲假以誅東林及素所不合者,因造十八羅漢、五十三參之目,書史可法、高弘圖、姜曰廣等姓名,內大悲袖中,海內人望,無不備列。錢謙益先已上疏頌士英,且為大鋮訟冤修好矣,大鋮憾不釋,亦列焉,將窮治其事。獄詞詭秘,朝士皆自危,而士英不欲興大獄,乃當大悲妖言律斬而止。

張縉彥以本兵首從賊,賊敗,縉彥竄歸河南,自言集義勇收復列城,即授原官,總督河北、山西、河南軍務,便宜行事。其他大僚降賊者,賄入,輒復其官。諸白丁、錄役輸重賂,立躋大帥。都人為語曰:「職方賤如狗,都督滿街走。」其刑賞倒亂如此。大清兵抵宿遷、邳州,未幾引還。史可法以聞,士英大笑不止,坐客楊士聰問故。士英曰:「君以為誠有是事耶?」乃史公妙用也。歲將暮,防河將吏應敘功,耗費軍資應稽算,此特為序功、稽算地耳。」侍講衛胤文兼給事中,監高傑軍。傑死,胤文窺士英指,論可法督師為贅。士英即擢胤文兵部右侍郎,總督傑營將士以分其權,可法益不得展布。

先是,左良玉接監國詔書,不肯拜,袁繼咸強之,乃開讀如禮。而屬承天守備何志孔、巡按御史黃澍入賀,陰伺朝廷動靜。澍挾良玉勢,當陛見,面數士英奸貪不法,且言嘗受張獻忠偽兵部尚書周文江重賄,為題授參將,罪當斬。志孔亦論士英岡上行私諸罪。司禮太監韓贊周叱志孔退,士英跪乞處分,澍舉笏直擊其背曰:「願與奸臣同死。」士英大號呼,王搖首不言者久之,贊周即執志孔候命。王因澍言意頗動,夜諭贊周,欲令士英避位。士英佯引疾,而賂福邸舊奄田成等向王泣曰:「上非馬公不得立,逐馬公,天下將議上背恩矣。且馬公去,誰念上者?」王默然,即慰留士英。士英亦畏良玉,請釋志孔,而命澍速還湖廣。故都督掌錦衣衛劉僑者,嘗遣戍,由周文江賄張獻忠,受偽命,為錦衣指揮使。及良玉復蘄、黃,僑削髮逃去,澍持之急。而士英納僑賄,令訐澍,遂復僑官,削澍職。尋以楚府中尉言,逮澍。良玉令部將群嘩,欲下南京索餉,因保救澍。袁繼咸為上疏代澍申理,士英不得已,乃免逮。澍遂匿良玉軍中,良玉與士英由此有隙。及偽太子獄起,良玉遂假為兵端。

太子之來也,識者指其偽,而都下士民嘩然是之。時又有童氏者,自稱王妃,亦下獄。督撫、鎮將交章爭太子及童妃事。王亟出獄詞,遍示中外,眾論益籍籍,謂士英等朋奸,導王滅絕倫理。澍在良玉軍中,日夜言太子冤狀,請引兵除君側惡。良玉亦上疏請全太子,斥士英等為奸臣。又以士英裁其餉,大憾,移檄遠近,聲士英罪。復上疏言:「自先帝之變,士英利災擅權,事事為難。逆案先帝手定,士英首翻之。《要典》先帝手焚,士英復修之。越其傑貪焚遣戍濫授節鉞。張孫振贓污絞犯,驟畀京卿。他如袁弘勳、楊文、劉泌、王燧、黃鼎等,或行同狗彘,或罪等叛逆,皆用之當路。己為首輔,用腹心阮大鋮為添註尚書。又募死士伏皇城,詭名禁軍,動曰廢立由我。陛下即位之初,恭儉明仁,士英百計誑惑,進優童艷女,傷損盛德。復引用大鋮,睚眥殺人,如雷演祚、周鑣等,鍛煉周內,株連蔓引。尤其甚者,借三案為題,凡生平不快意之人,一網打盡。令天下士民,重足解體。目今皇太子至,授受分明。大鋮一手握定抹殺識認之方拱乾,而信朋謀之劉正宗,忍以十七年嗣君,付諸幽囚。凡有血氣,皆欲寸磔士英、大鋮等,以謝先帝。乞立肆市朝,傳首抒憤。」疏上,遂引兵而東。士英懼,乃遣阮大鋮、朱大黃、黃得功、劉孔昭等禦良玉,而撤江北劉良佐等兵,從之西。時大清兵日南下,大理少卿姚思孝,御史喬可聘、成友謙請無撤江北兵,亟守淮、揚。士英厲聲叱曰:「若輩東林,猶藉口防江,欲縱左逆入犯耶?北兵至,猶可議款。左逆至,則若輩高官,我君臣獨死耳!」力排思孝等議,淮、揚備禦益弱。會良玉死,其子夢庚連陷郡縣,率兵至采石。得功等與相持,大鋮、孔昭方虛張捷音,以邀爵賞,而大清兵已破揚州,逼京城。

五月三日,王出走太平,奔得功軍。孔昭斬關遁。明日,士英奉王母妃,以黔兵四百人為衛,走浙江。經廣德州,知州趙景和疑其詐,閉門拒守。士英攻破,執景和殺之,大掠而去。走杭州,守臣以總兵府為母妃行宮。不數日,大鋮、大典、方國安俱倉皇至,則得功已兵敗死,王被擒,次日,請潞王監國,不受。未幾,大兵至,王率眾降,尋同母妃北去。此即大器等之所議欲立者也。

杭州既降,士英欲謁監國魯王,魯王諸臣力拒之。大鋮投朱大典於金華,亦為士民所逐,大典乃送之嚴州總兵方國安軍。士英,國安同鄉也,先在其軍中。大鋮掀髯指掌,日談兵,國安甚喜。而士英以南渡之壞,半由大鋮,而己居惡名,頗以為恨。已,我兵擊敗士英、國安。無何,士英、國安率眾渡錢塘,窺杭州,大兵擊敗之,溺江死者無算。士英擁殘兵欲入閩,唐王以罪大不許。明年,大兵巢湖賊,士英與長興伯吳日生俱擒獲,詔俱斬之。事具國史。大鋮偕謝三賓、宋之晉、蘇壯等赴江干乞降,從大兵攻仙霞關,殭僕石上死。而野乘載士英遁至臺州山寺為僧,為我兵搜獲,大鋮、國安先後降。尋唐王走順昌。我大兵至,搜龍扛,得士英、大鋮、國安父子請王出關為內應疏,遂駢斬士英、國安於延平城下。大鋮方游山,自觸石死,仍戮尸云。


\end{pinyinscope}