\article{列傳第八十五}

\begin{pinyinscope}
席書弟春篆霍韜子與瑕熊浹黃宗明黃綰陸澄

席書,字文同,遂寧人。弘治三年進士。授郯城知縣。入為工部主事,移戶部,進員外郎。十六年,雲南晝晦地震,命侍郎樊瑩巡視,奏黜監司以下三百餘人。書上疏言:「災異係朝廷,不係雲南。如人元氣內損,然後瘡瘍發四肢。朝廷,元氣也。雲南,四肢也。豈可舍致毒之源,專治四肢之末?今內府供應數倍往年,冗食官數千,投充校尉數萬,齋醮寺觀無停日,織造頻煩,賞賚踰度;皇親奪民田,宦官增遣不已;大獄據招詞不敢辯,刑官亦不敢伸;大臣賢者未起用,小臣言事謫者未復;文武官傳陞,名器大濫。災異之警,偶泄雲南,欲以遠方外吏當之,此何理也?漢遣八使巡行天下,張綱獨曰:『豺狼當道,安問狐狸。』今樊瑩職巡察,不能劾戚畹、大臣,獨考黜雲南官吏,舍本而治末。乞陛下以臣所言弊政,一切釐革。他大害當祛,大政當舉者,悉令所司條奏而興革之。」時不能用。

武宗時,歷河南僉事、貴州提學副使。時王守仁謫龍場驛丞,書擇州縣子弟,延守仁教之,士始知學。屢遷福建左布政使。寧王宸濠反,急募兵二萬討之。至則賊已平,乃返。尋以右副都御史巡撫湖廣。中官李鎮、張暘假進貢及御鹽名斂財十餘萬,書疏發之。嘉靖元年改南京兵部右侍郎。江南北大饑,奉命振江北。令州縣十里一廠,煮糜哺之,全活無算。

初,書在湖廣,見中朝議「大禮」未定,揣帝向張璁、霍韜,獻議言:「昔宋英宗以濮王第十三子出為人後,今上以興獻王長子入承大統。英宗入嗣在袞衣臨御之時,今上入繼在宮車晏駕之後。議者以陛下繼統武宗,仍為興獻帝之子,別立廟祀,張璁、霍韜之議未為非也。然尊無二帝。陛下於武宗親則兄弟,分則君臣。既奉孝宗為宗廟主,可復有他稱乎?宜稱曰『皇考興獻王』,此萬世不刊之典。禮臣三四執奏,未為失也。然禮本人情,陛下尊為天子,慈聖設無尊稱,可乎?故尊所生曰帝后,上慰慈闈,此情之不能已也。為今日議,宜定號曰『皇考興獻帝』。別立廟大內,歲時祀太廟畢,仍祭以天子之禮,似或一道也。蓋別以廟祀則大統正而昭穆不紊,隆以殊稱則至愛篤而本支不淪,尊尊親親,並行不悖。至慈聖宜稱皇母某后,不可以興獻加之。獻,謚也,豈宜加於今日?」議既具,會中朝競詆張璁為邪說,書懼不敢上,而密以示桂萼,萼然其議。三年正月,萼具疏並上之。帝大喜,趣召入對。無何,詔改稱獻帝為本生皇考,遂寢召命。會禮部尚書汪俊以爭建廟去位,特旨用書代之。故事,禮部長貳率用翰林官。是時廷臣排異議益力,書進又不由廷推,因交章詆書,至訾其振荒無狀,多侵漁。書亦屢辭新命,並錄上《大禮考議》,且乞遣官勘振荒狀。帝為遣司禮中官,戶、刑二部侍郎,錦衣指揮往勘,而趣書入朝益急。比至德州,則廷臣已伏闕哭爭,盡繫詔獄。書馳疏言:「議禮之家,名為聚訟。兩議相持,必有一是。陛下擇其是者,而非者不必深較。乞宥其愆失,俾獲自新。」不允。

其年八月入朝,帝慰勞有加。踰月乃會廷臣大議,上奏曰:

三代之法,父死子繼,兄終弟及,自夏歷漢二千年,未有立從子為皇子者也。漢成帝以私意立定陶王,始壞三代傳統之禮。宋仁宗立濮王子,英宗即位,始終不稱濮王為伯。今陛下生於孝宗崩後二年,乃不繼武宗大統,超越十有六年上考孝宗,天倫大義固已乖悖。又未嘗立為皇子,與漢、宋不同。自古天子無大宗、小宗,亦無所生、所後。《禮經》所載,乃大夫士之禮,不可語於帝王。伯父子姪皆天經地義,不可改易。今以伯為父,以父為叔,倫理易常,是為大變。

夫得三代傳統之義,遠出漢、唐繼嗣之私者,莫若《祖訓》。《祖訓》曰「朝廷無皇子,必兄終弟及。」則嗣位者實繼統,非繼嗣也。伯自宜稱皇伯考,父自宜稱皇考,兄自宜稱皇兄。今陛下於獻帝、章聖已去本生之稱,復下臣等大議。臣書、臣璁、臣萼、臣獻夫及文武諸臣皆議曰:世無二首,人無二本。孝宗皇帝,伯也,宜稱皇伯考。昭聖皇太后,伯母也,宜稱皇伯母。獻皇帝,父也,宜稱皇考。章聖皇太后,母也,宜稱聖母。武宗仍稱皇兄,莊肅皇后宜稱皇嫂。尤願陛下仰遵孝宗仁聖之德,念昭聖擁翊之功,孝敬益隆,始終無間,大倫大統兩有歸矣。奉神主而別立禰室,於至親不廢,隆尊號而不入太廟,於正統無干,尊親兩不悖矣。一遵《祖訓》,允合聖經。復三代數千年未明之典禮,洗漢、宋悖經違禮之陋習,非聖人其孰能之。

議上,詔布告天下,尊稱遂定。

帝既加隆所生,中外獻諛希恩者紛然遝至。錦衣百戶隨全、光祿錄事錢子勛既以罪褫,希旨請遷獻帝顯陵梓宮北葬天壽山。工部尚書趙璜等斥其謬,帝復下廷議。書乃會廷臣上言:「顯陵,先帝體魄所藏,不可輕動。昔高皇帝不遷祖陵,文皇帝不遷孝陵。全等諂諛小人,妄論山陵,宜下法司按問。」帝報曰:「先帝陵寢在遠,朕朝夕思望,不勝哀痛,其再詳議以聞。」書復集眾議,極言不可,乃已。

書以「大禮」告成,宜有以答天下望,乃條新政十二事以獻,帝優旨報焉。大同軍變,殺巡撫張文錦,毀總兵官江桓印,而出故帥朱振於獄,令代桓。帝因而命之,諭禮部鑄新印。書持不可,請討之,與政府忤。時執政者費宏、石珤、賈詠,書心弗善也,乃力薦楊一清、王守仁入閣,且曰:「今諸大臣皆中材,無足與計天下事。定亂濟時,非守仁不可。」帝曰:「書為大臣,當抒猷略,共濟時艱,何以中材自諉。」守仁迄不獲柄用。

四年,光祿寺丞何淵請建世室,祀獻皇帝於太廟。帝命禮官集議,書等上議;「《王制》:『天子七廟,三昭三穆』。周以文、武有大功德,乃立世室,與后稷廟皆百世不遷。我太祖立四親廟,德祖居北,後改同堂異室。議祧則以太祖擬文世室,太宗擬武世室。今獻皇帝以籓王追崇帝號,何淵乃欲比之太祖、太宗,立世室於太廟,甚無據。」不報。頃之,張璁特奏上,力言不可,書亦三疏如璁議。帝遣中官即其家諭之,書復密疏切諫。帝不悅,責以畏眾飾奸。乃議別立禰廟,而世室之議竟寢。五年秋,章聖太后將謁世廟,禮官議不合。書以目眚在告,上言:「母后謁廟,事出創聞,禮官實無所據,惟聖明裁酌。且世廟既成,宜有肆赦之典,請盡還議禮遣戍諸臣。所謂合萬國之歡心以祀先王,此天子大孝也。」報聞。

書以議禮受帝知,倚為親臣。初進《大禮集議》,加太子太保,尋以《獻帝實錄》成,進少保。眷顧隆異,雖諸輔臣莫敢望。而書得疾不能視事,屢疏乞休,舉羅欽順自代,帝輒慰留不允。其後疾篤,請益力,詔加武英殿大學士,賜第京師,支俸如故。甫聞命而卒。贈太傅,謚文襄,任一子尚寶丞,異數也。

書遇事敢為,性頗偏愎。初,長沙人李鑑為盜,知府宋卿論之死。書方巡撫湖廣,發卿贓私,因劾卿故入鑑罪。帝遣大臣按,不如書言。而書時已得幸,乃命逮鑑入京再訊。書遂言:「臣以議禮犯眾怒,故刑官率右卿而重鑒罪,請敕法司辨雪。」及法司讞上無異詞,帝重違書意,特減鑒死遣戍。其他庇陳洸,排費宏,率恣行私意,為時論所斥。

弟春、篆。春由庶吉士授御史,巡雲南。以兄為都御史,改翰林檢討。預修《武宗實錄》成,當進秩。內閣費宏以春由他官入,與檢討劉夔並擬按察僉事。夔亦故御史,以避兄侍郎龍改授者也。書大怒,疏言:「故事,無纂修書成出為外任者。」帝以書故留春,擢修撰,而夔亦留,擢編修。書由是怨宏,數詆諆。及書卒,帝念其議禮功,累進春翰林學士。嘉靖十二年由禮部右侍郎改吏部。詔舉堪翰林者,春欲召還故翰林楊惟聰、陳沂,尚書汪鋐不可,遂有隙。後鋐有所推舉,不與春議,春怒詬鋐。鋐訐春前附楊廷和排議禮諸臣,遂落職。卒於家。

篆為戶科給事中。黔國公沐崑劾按察使沈恩等,篆與同官李長私語崑奏多誣,長即劾崑。武宗責長誣重臣,下詔獄。詞連篆,並繫治謫外,篆得夷陵判官。世宗嗣位,復故官,未上卒。予祭,贈光祿少卿。

霍韜,字渭先,南海人。舉正德九年會試第一。謁歸成婚,讀書西樵山,經史淹洽。世宗踐阼,除職方主事。楊廷和方柄政,韜上言:「閣臣職參機務,今止票擬,而裁決歸近習。輔臣失參贊之權,近習起干政之漸。自今章奏,請召大臣面決施行,講官、臺諫,班列左右,眾議而公駁之。宰相得取善之名,內臣免招權之謗。」因言錦衣不當典刑獄,東廠不當預朝議,撫按兵備官不當以軍功授秩廕,興府護衛軍不當盡取入京概授官職,御史謝源、伍希儒赴難有功不當罷黜,平逆籓功自安慶、南昌外,不當濫敘。帝嘉納之。

及「大禮」議起,禮部尚書毛澄力持考孝宗,韜私為《大禮議》駁之。澄貽書相質難,韜三上書極辨其非。已,知澄意不可回,其年十月上疏曰:

按廷議謂陛下宜以孝宗為父,興獻王為叔,別擇崇仁王子為獻王後,考之古禮則不合,質之聖賢之道則不通,揆之今日之事體則不順。

考《儀禮·喪服》章云「斬衰為所後者」。又云「為人後者,為其父母報」。是於所後者,蓋無稱為父母之說,而於本生父母又無改稱伯叔父母之云也。漢儒不明其義,謬為邪說曰「為人後者為之子」。果如其言,則漢宣帝當為昭帝後矣。然昭帝從祖也,宣帝從孫也,孫將謂祖為父,可乎?唐宣宗當為武宗後矣,然武宗姪也,宣宗叔也,叔反謂姪為父,可乎?吳諸樊兄弟四人以國相授受,蓋迭相為後矣,是兄弟自具高曾祖考也,而可乎?故曰考之古禮則不合也。

天下者,天下之天下,非一人所得私也。宋人之告其君曰:「仁宗於宗室中特簡聖明,授以大業,陛下所以負扆端冕,富有四海,子孫萬世相承,皆先帝之德。」蓋謂仁宗以天下授英宗,宜舍本生父母而以仁宗為父母也。臣以聖賢之道觀之,孟子言舜為天子,瞽瞍殺人,皋陶執之,舜則竊負而逃,是父母重而天下輕也。若宋儒之說,則天下重而父母輕矣。故曰求之聖賢之道則不通也。

武宗嗣孝宗歷十有六年,考宗非無嗣也。今強欲陛下重為孝宗之嗣,何為也哉?夫陛下為孝宗子矣,誰為武宗子乎?孝宗有兩嗣子矣,武宗獨無嗣子,可乎?臣子於君父一也,既不忍孝宗之無嗣,獨忍武宗之無嗣乎?若曰武宗以兄,固得享弟之祀,則孝宗以伯,獨不得享姪之祀乎?既可越武宗直繼孝宗矣,獨不可並越孝宗直繼憲宗乎?武宗無嗣,無可如何矣。孝宗有嗣,復強繼其嗣,而絕興獻之嗣,是於孝宗無所益,而於興獻不大有損乎?故曰揆之今日之事體則不順也。

然臣下之為此議也,其故有三:曰前代故事之拘也,曰不忘孝宗之德也,曰避迎合之嫌也。今陛下既考孝宗矣,尊興獻王以帝號矣,則將如斯而已乎?臣竊謂帝王之相繼也,繼其統而已矣,固不屑屑於父子之稱也。惟繼其統,則不惟孝宗之統不絕,即武宗之統亦不絕矣。然則如之何而可乎?惟陛下於興獻王得正父子之稱,以不絕天性之恩。於國母之迎,得正天子之母之禮。復於昭聖太后、武宗皇后處之有其道,事之盡其誠,則於尊尊親親兩不悖矣。

帝得疏喜甚,迫群議不遽行。而朝士咸指目韜為邪說。韜意不自得,尋謝病歸。

嘉靖三年,帝議尊崇所生益急,兩詔召韜。韜辭疾不赴,馳疏言:

今日大禮之議,兩端而已。曰崇正統之大義也,曰正天倫之大經也。徒尊正統,其弊至於利天下而棄父母;徒重天倫,其弊至於小加大而卑踰尊。故臣謂陛下宜稱孝宗曰皇伯考,獻帝曰皇考。此天倫之當辨者也。尊崇之議,則姑在所緩,此大統之當崇者也。乃廷議欲陛下上考孝宗,又兼考獻帝,此漢人兩統之失也。本原既差,則愈議愈失。臣之愚慮,則願陛下預防未然之失,毋重將來之悔而已。始陛下尊昭聖皇太后為母,雖於禮未合,然宮闈之內亦既相安。今一旦改稱,大非人情所堪。願陛下以臣等建議之情,上啟皇太后,必中心悅預無疑貳之隙。萬一未喻,亦得歸罪臣等,加賜誅斥,然後委曲申請,務得其歡心。陛下朝夕所以承迎其意,慰釋其憂者,亦無所不用其極,庶名分正而嫌隙消,天下萬世無所非議,此臣愚慮者一也。

昭聖之嫡嗣,武宗一人而已。武宗無嗣,莊肅皇后之屬望已矣。臣謂陛下之事昭聖,禮秩雖極尊崇,然其勢日輕;陛下之事聖母,尊稱雖或未至,然其勢日重。故今日廷臣心卷心卷以尊大統,母昭聖為請者,蓋預防陛下將來之失,而追報孝宗之職分也。臣嘗伏讀明詔,正統大義,不敢有違。知陛下尊昭聖,敬莊肅,此心可上質天地,下信士庶矣。但恐左右之人不達聖意,妄生疑間。或以彌文小節,遂構兩宮之隙,此不可不早慮而預防之也。願陛下以臣等建議之情,上啟聖母曰,昭聖皇太后實大統嫡宗,至尊無對,伏願聖母時自謙抑,示尊敬至意。莊肅皇后母儀天下十六年,聖母接見之儀,不可輕忽,凡正旦、賀壽,聖母每致謙讓不敢納之意。俾宮闈大權一歸昭聖,而聖母若無與焉,則天下萬世稱頌懿德與天無極。萬一聖母意猶未喻,亦得歸罪臣等,加賜誅斥,然後委曲申請,務得允從,庶宗統正而嫌隙消,天下萬世無所非議,此臣愚慮者二也。

帝深嘉其忠義,趣令趨朝。明年擢少詹事兼侍講學士。韜固辭。且請令六部長貳、翰林、給事、御中俱調外任,練政體;監司、守令政績卓異,即擢卿丞,有文學者擢翰林;舉貢入仕皆得擢翰林,陞部院,不宜困資格。帝不允辭,趣令赴職。下其奏於有司,悉格不用。

六年,還朝,命直經筵日講。韜自以南音力辭日講,請撰《古今政要》及《詩書直解》以進。帝褒許之。其年九月遷詹事兼翰林學士,韜復固辭,言:「自楊榮、楊士奇、楊溥以及李東陽、楊廷和顓權植黨,籠翰林為屬官,中書為門吏,故翰林遷擢不由吏部,而中書至有進秩尚書者。臣嘗建議,謂翰林去留,盡屬吏部,庶不陰倚內閣為腹心,內閣亦不陰結翰林為羽翼。且欲京官補外以均勞逸,議未即行,躬自蹈之,而又躐居學士徐縉上,何愧如之。」帝優詔不允。明年四月進禮部右侍郎。韜力辭,且舉康海、王九思、李夢陽、魏校、顏木、王廷陳、何瑭自代,帝不允。再辭,乃允之。

六月,「大禮」成,超拜禮部尚書,掌詹事府事。韜因言翰林院修書遷官、日講廕子、及巡撫子弟廕武職之非,而以為己不能力挽,不可隨眾趨。且稱給事中陳洸冤,薦監生陳雲章才可用。帝優詔褒答,不允辭。韜復奏曰:「今異議者謂陛下特欲尊崇皇考,遂以官爵餌其臣,臣等二三臣茍圖官爵,遂阿順陛下之意。臣嘗自慨,若得禮定,決不受官,俾天下萬世知議禮者非利官也。茍疑議禮者為利官,則所議雖是,彼猶以為非,何以塞天下口?」因固辭不拜,帝猶不允。三辭,乃允之。

韜先後薦王守仁、王瓊諸人,帝皆納用。嘗因災異陳時弊十餘事,多議行。張璁、桂萼之罷政也,韜謂言官陸粲等受楊一清指使,兩疏力攻一清,奪其職,而璁、萼召還。帝從夏言議,將分祀天地,建二郊,韜極言其非。帝不悅,責韜罔上自恣。言亦疏辨,力詆韜。韜素護前自遂,見帝怒,不敢辨,乃遺言書,痛詆之,復錄其書送法司。言怒,疏陳其狀,且劾韜無君七罪,並以其書進呈。帝大怒,責韜謗訕君上,醜正懷邪,遂下都察院獄。韜從獄中上書祈哀,璁亦再申救,帝皆不納。南京御史鄧文憲言,宜察韜心,容其戇,且天地分祀是置父母異處,郊外親蠶是廢內外防閑。帝怒,謫之邊方。韜繫獄踰月,帝終念其議禮功,令輸贖還職。尋以母喪歸。廣東僉事龔大稔訐韜及方獻夫居鄉不法事,大稔反被逮削籍。

十二年,韜起歷吏部左、右侍郎。時部事多主於尚書,兩侍郎率不預。韜爭於尚書汪鋐,侍郎始獲參部事。韜素剛愎,屢與鋐爭,鋐等亦嚴憚之。既而鋐罷,帝久不置尚書,以韜掌部事。閣臣李時傳旨,用鴻臚卿王道中為順天府丞。韜言:「輔臣承天語無可疑,然臣等猶當奏請,用杜矯偽。」因守故事,列道中及應天府丞郭登庸二人名上。帝嘉其守法,乃用登庸,而改道中大理少卿。久之,出韜為南京禮部尚書。

順天府尹劉淑相坐所親贓私被鞫,疑禮部尚書夏言姻通判費完陷之,訐言請屬事。帝怒,下淑相詔獄。淑相與韜善,言亦疑韜主之,遂訐韜扈蹕謁陵,遠遊銀山寺大不敬。韜自訴,因論言:「請謚故少師費宏為文憲,不敘宏累被劾狀,按律,增減緊關情節者斬。且『憲』乃純皇帝廟號,人臣安得用?」會南京給事中曾鈞騎馬,不避尚書劉龍、潘珍轎,龍與鈞互訐奏。韜劾鈞,且請禁小臣乘轎。給事中李充濁、曹邁等交章,言近侍之臣不當避道,雜舉公會宴次得與尚書同列以證,語頗侵韜。韜疑充濁倚言為內主,訐充濁為奸黨,復摭言他事。言益怒,奏韜大罪十餘事。且言彭時、宋濂皆於正德間謚文憲,不避廟號,韜陋不知故事。帝方不直韜,淑相復從獄中摭言他事,帝益怒,考訊之。辭服韜主使,乃斥淑相為民,降韜俸一級。當議乘轎時,言被劾不預,都御史王廷相會禮部侍郎黃宗明、張璧請禁飭小臣如韜奏,而南京諸給事、御史自如。韜以為言,帝復申飭,眾情滋不悅。曹邁及同官尹相等遂與韜忿爭。相劾韜遷南部怨望;擅取海子魚,與鄉人群飲郊壇松下;侍郎袁宗儒期喪不當進表,逼使行。韜上疏自理。下廷議。帝為停韜俸四月,相等亦停二月。韜既與言交惡,及言柄用,韜每欲因事陷之。上言:「頃吏部選劉文光等為給事中,尋忽報罷,人皆曰閣臣抑之。給事中李鶴鳴考察謫官,尋復故,人皆曰賄得。宜諭吏部毋受當事頤指,使天下知威福出朝廷,而大臣有李林甫、秦檜者,不得播弄於左右。」其意為言發也。於是鶴鳴上疏自白,並摭韜居鄉不法諸事。帝兩置之。無何,韜劾南京御史龔湜、郭本。湜等自辨,亦劾韜。帝並置不問。

十八年簡補宮僚,命韜以太子少保、禮部尚書協掌詹事府事。疏辭加秩,且詆大臣受祿不讓,晉秩不辭,或有狐鼠鑽結,陰固寵權,怨氣召災。實有所自。其意亦為言發。既屢擊言不勝,最後見郭勛與言有隙,乃陰比勛,與共齮齕言。時中外訛言帝復南幸,韜因顯頌勛,言:「六飛南狩時,臣下多納賄不法。文官惟袁宗儒,武官惟郭勛不受饋。今訛言復播,宜有以禁戢之。」帝既下詔安群情,乃詰韜曰:「朕昨南巡,卿不在行,受賄事得自何人?據實以奏。」韜對,請問諸郭勛。帝責其支詞,務令指實。韜窘,乃言:「扈從諸臣無不受饋遺、折取夫隸直者,第問之夏言,令自述。至各官取賄實跡,勛具悉始末,當不欺。如必欲臣言,請假臣風憲職,循途按之,當備列以奏。」章下所司。韜懼不當帝旨,尋赴京,列所遇進鮮船內臣貪橫狀,帝亦不問。明年十月卒於官,年五十有四。贈太子太保,謚文敏。

韜學博才高,量褊隘,所至與人競。帝頗心厭之,故不大用。先後多所建白,亦頗涉國家大計。且嘗薦「大禮」大獄得罪諸臣,及廢籍李夢陽、康海等。在南都,禁喪家宴飲,絕婦女入寺觀,罪娼戶市良人女,毀淫祠,建社學,散僧尼,表忠節。既去,士民思之。始與璁、萼結,既而比郭勛。舉進士出毛澄門下,素執弟子禮,議禮不合,遂不復稱為座主。及總裁己丑會試,亦遂不以唐順之等為門生。其議禮時,詆司馬光。後議薛瑄從祀,至追論光不可祀孔廟。其不顧公論如此。

子與瑕,舉進士。授慈溪知縣。鄢懋卿巡鹽行部,與瑕不禮,為所劾罷。起知鄞縣,終廣西僉事。

熊浹,字悅之,南昌人。正德九年進士。授禮科給事中。寧王宸濠將為變,浹與同邑御史熊蘭草奏,授御史蕭淮上之。濠倉卒舉事,卒敗,本兩人早發之力。出核松潘邊餉。副總兵張傑倚江彬勢,贓累巨萬,誘殺熟番上功啟邊釁,箠死千戶以下至五百人。又嘗率家眾遮擊副使胡澧。撫、按莫敢言。浹至,盡發其狀,傑遂褫職。

世宗踐阼,廷議追崇禮未定。浹馳疏言:「陛下起自籓服,入登大寶,倘必執為後之說,考孝宗而母慈壽,則興獻母妃當降稱伯叔父母矣。不知陛下承懽內庭時,將仍舊稱乎,抑改而從今稱乎?若仍舊稱,而不得尊之為后,則於慈壽徒有為後之虛文,於母妃則又缺尊崇之大典,無一而可也。臣愚謂興獻王尊以帝號,別建一廟,以示不敢上躋於列聖。母妃則尊為皇太后,而少殺其徽稱,以示不敢上同於慈壽。此於大統固無所妨,而天性之恩亦得以兼盡。」疏至,會興王及妃已稱為帝后,下之禮官。

嘉靖初,由右給事中出為河南參議。外艱歸。六年,服闋,召修《明倫大典》。超擢右僉都御史,協理院事。明年四月遷大理寺卿,俄遷右副都御史。《大典》成,轉左。八年二月遂擢右都御史,掌院事。京師民張福訴里人張柱殺其母,東廠以聞,刑部坐柱死。不服,福姊亦泣訴官,謂母福自殺之,其鄰人之詞亦然。詔郎中魏應召覆按,改坐福。東廠奏法司妄出人罪,帝怒,下應召詔獄。浹是應召議,執如初。帝愈怒,褫浹職。給事中陸粲、劉希簡爭之,帝大怒,並下兩人詔獄。侍郎許贊等遂抵柱死,應召及鄰人俱充軍,杖福姊百,人以為冤。當是時,帝方深疾孝、武兩后家,柱實武宗后家夏氏僕,故帝必欲殺之。

浹家居十年。至帝幸承天與近臣論舊人,乃召為南京禮部尚書,改兵部,參贊機務。二十一年召為兵部尚書,掌都察院事。居二年,代許贊為吏部尚書。帝於禁中築乩仙臺,間用其言決威福,浹論其妄。帝大怒,欲罪之,以前議禮故不遽斥。二品六年滿,加太子太保,坐事奪俸者再。浹知帝意終不釋,遂稱病乞休。帝大怒,褫職為民。又十年卒。

浹少有志節,自守嚴。雖由議禮顯,然不甚黨比,尤愛護人才。故其去吏部也,善類多思之。隆慶初,復官,予祭葬,謚恭肅。

黃宗明,字誠甫,鄞人。正德九年進士。除南京兵部主事,進員外郎。嘗從王守仁論學。寧王宸濠反,上江防三策。武宗南征,抗疏諫,尋請告歸。嘉靖二年,起南京刑部郎中。張璁、桂萼爭「大禮」,自南京召入都,未上。三年四月,璁、萼、黃綰及宗明聯疏奏曰:「今日尊崇之議,以陛下與為人後者,禮官附和之私也。以陛下為入繼大統者,臣等考經之論也。人之言曰,兩議相持,有大小眾寡不敵之勢。臣等則曰,惟理而已。大哉舜之為君,視天下悅而歸己,猶草芥也,惟不順於父母,如窮人無所歸。今言者徇私植黨,奪天子之父母而不顧,在陛下可一日安其位而不之圖乎?此聖諭令廷臣集議,終日相視莫敢先發者,勢有所壓,理有所屈故也。臣等大懼欺蔽因循,終不能贊成大孝。陛下何不親御朝堂,進百官而詢之曰:『朕以憲宗皇帝之孫,孝宗皇帝之姪,興獻帝之子,遵太祖兄終弟及之文,奉武宗倫序當立之詔,入承大統,非與為人後者也。前者未及詳稽,遽詔天下,尊孝宗皇帝為皇考,昭聖太后為聖母,而興獻帝后別加本生之稱,朕深用悔艾。今當明父子大倫,繼統大義,改稱孝宗為皇伯考,昭聖為皇伯母,而去本生之稱,為皇考恭穆獻皇帝,聖母章聖皇太后,此萬世通禮。爾文武廷臣尚念父子之親,君臣之義,與朕共明大倫於天下。』如此,在朝百工有不感泣而奉詔者乎,更以此告於天下萬姓,其有不感泣而奉詔者乎,此即《周禮》詢群臣詢萬民之意也。」奏入,帝大悅,卒如其言。宗明亦遂蒙帝眷。

明年出為吉安知府,遷福建鹽運使。六年召修《明倫大典》,以母憂歸。服闋,徵拜光祿卿。十一年擢兵部右侍郎。其冬,編修楊名以劾汪鋐下詔獄,詞連同官程文德,亦坐繫。詔書責主謀者益急。宗明抗疏救,且曰:「連坐非善政。今以一人妄言,必究主使,廷臣孰不懼?況名搒掠已極,當嚴冬或困斃,將為仁明累。」帝大怒,謂宗明即其主使,並下詔獄,謫福建右參政。帝終念宗明議禮功,明年召拜禮部右侍郎。遼東兵變,捶辱巡撫呂經。而帝務姑息,納鎮守中官王純等言,將逮經。宗明言:「前者遼陽之變,生於有激。今重賦苛徭悉已釐正,廣寧復變,又誰激之?法不宜復赦。請令新撫臣韓邦奇勒兵壓境,揚聲討罪,取其首惡,用振國威,不得專事姑息。」帝不從,經卒被逮。宗明尋轉左侍郎,卒於官。

初,議禮諸臣恃帝恩眷,驅駕氣勢,恣行胸臆。宗明雖由是驟顯,持論頗平,於諸人中獨無畏惡之者。

黃綰,字宗賢,黃巖人,侍郎孔昭孫也。承祖蔭官後府都事。嘗師謝鐸、王守仁。嘉靖初,為南京都察院經歷。

張璁、桂萼爭「大禮」,帝心嚮之。三年二月,綰亦上言曰:「武宗承孝宗之統十有六年,今復以陛下為孝宗之子,繼孝宗之統,則武宗不應有廟矣。是使孝宗不得子武宗,乃所以絕孝宗也。由是,使興獻帝不得子陛下,乃所以絕興獻帝也。不幾於三綱淪,九法棨哉!」奏入,帝大喜,下之所司。其月,再上疏申前說。俄聞帝下詔稱本生皇考,復抗疏極辨。又與璁、萼及黃宗明合疏爭,「大禮」乃定。綰自是大受帝知。及明年,何淵請建世室,綰與宗明斥其謬。尋遷南京刑部員外郎,再謝病歸。帝念其議禮功,六年六月召擢光祿少卿,預修《明倫大典》。

王守仁中忌者,雖封伯,不給誥券歲祿;諸有功若知府邢珣、徐璉、陳槐,御史伍希儒、謝源,多以考察黜。綰訟之於朝,且請召守仁輔政。守仁得給賜如制,珣等亦敘錄。綰尋遷大理左少卿。其年十月,璁、萼逐諸翰林於外,引己所善者補之,遂用綰為少詹事兼侍講學士,直經筵。以任子官翰林,前此未有也。

明年,《大典》成,進詹事。錦衣僉事聶能遷者,初附錢寧得官,用登極詔例還為百戶。後附璁、萼議「大禮」,且交關中貴崔文,得復故職。《大典》成,諸人皆進秩,能遷獨不與,大恨。囑罷閒主事翁洪草奏,誣王守仁賄席書得召用,詞連綰及璁。綰疏辨,且乞引避。帝優旨留之,而下能遷法司,遣之戍,洪亦編原籍為民。

綰與璁輩深相得。璁欲用為吏部侍郎,且令典試南京,並為楊一清所抑,又以其南音不令與經筵。綰大恚,上疏醜詆一清而不斥其名。帝心知其為一清也,以浮詞責之。其年十月,出為南京禮部右侍郎,遍攝諸部印。十二年召拜禮部左侍郎。初,綰與璁深相結。至是,夏言長禮部,帝方嚮用,綰乃潛附之,與璁左。其佐南禮部也,郎中鄒守益引疾,詔綰核實。久不報,而守益竟去。吏部尚書汪鋐希璁指,疏發其事,詔奪守益官,令鋐覆核,鋐遂劾綰欺蔽。璁調旨削三秩,出之外。會禮部請祈穀導引官,帝留綰供事。鋐於是再疏攻綰,且掇及他事,帝復命調外。綰上疏自理,因詆鋐為璁鷹犬,乞賜罷黜以避禍。帝終念綰議禮功,仍留任如故。綰自是顯與璁貳矣。

初,大同軍變,殺總兵官李瑾,據城拒守。總制侍郎劉源清、提督郤永議屠之。城中恟懼,外勾蒙古為助,塞上大震。巡撫潘人放急請止兵,源清怒,馳疏力詆人放。璁及廷議並右源清,綰獨言非策。及源清罷,侍郎張瓚往代。未至,而郎中詹榮等已定亂。叛卒未盡獲,軍民瘡痍甚,代王請遣大臣綏緝之。疏下禮部,夏言以為宜許,而極詆前用兵之謬,語侵璁。璁怒,力持不欲遣。帝委曲諭解之,乃特以命綰,且令察軍情,勘功罪,得便宜行事。綰馳至大同,宗室軍民牒訴官軍暴掠者以百數,無告叛軍者。綰一無所問,以安其心。有為叛軍使蒙古歸者,綰執戮之,反側者復相煽。綰大集軍民,曉以禍福。罹害者陳牒,綰佯不問,而密以牒授給振官,按里核實,一日捕首惡數十人。卒尚欽殺一家三人,懼不免,夜鳴金倡亂,無應者,遂就擒。綰復圖形購首惡數人,軍民乃不復虞詿誤。遂令有司樹木柵,設保甲四隅,創社學,教軍民子弟,城中大安。還朝,列上文武將吏功罪,極詆源清、永。綰以勞增俸一等,璁及兵部庇源清,陰抑綰。綰累疏論,帝亦意嚮之,源清、永卒被逮。綰尋以母憂歸。

十八年,禮官以恭上皇天上帝大號及皇祖謚號,請遣官詔諭朝鮮。時帝方議討安南,欲因以覘之,乃曰:「安南亦朝貢之國,不可以邇年叛服故,不使與聞。其擇大臣有學識者往。」廷臣屢以名上,皆不用。特起綰禮部尚書兼翰林學士為正使,諭德張治副之。帝方幸承天,趣綰詣行在受命。綰憚往,至徐州先馳使奏疾不能前,致失期。帝責綰不馳赴行在,而舟詣京師為大不敬,令陳狀,已而釋之。綰數陳便宜,請得節制兩廣、雲、貴重臣,遣給事御史同事,吏、禮、兵三部擇郎官二人備任使。帝悉從之。最後為其父母請贈,且援建儲恩例請給誥命如其官。帝怒,褫尚書新命,令以侍郎閒住,使事亦竟寢。久之,卒於家。

綰起家任子,致位卿貳。初附張璁,晚背璁附夏言,時皆以傾狡目之。方「大禮」之興也,首繼璁上疏者為襄府棗陽王祐楒。其言曰:「孝廟止宜稱『皇伯考』,聖父宜稱『皇考興獻大王』。即興國之陵廟祀用天子禮樂,祝稱孝子皇帝某。聖母宜上徽號稱太妃,迎養宮中。庶繼體之道不失,天性之親不泯。」時世宗登極歲之八月也。自時厥後,諸希寵干進之徒,紛然而起。失職武夫、罷閒小吏亦皆攘臂努目,抗論廟謨。即璁、萼輩亦羞稱之,不與為伍。故自璁等八人外,率無殊擢。至致仕教諭王價,遂請加諸臣貶竄誅戮之刑,懲朋黨欺蔽之罪。而最陋者南京刑部主事歸安陸澄。初極言追尊之非,逮服闋入都,《明倫大典》已定,璁、萼大用事,澄乃言初為人誤,質之臣師王守仁乃大悔恨。萼悅其言,請除禮部主事。而帝見澄前疏惡之,謫高州通判以去。

嘉靖四年七月,席書將輯《大禮集議》,因言:「近題請刊布,多繫建言於三年以前,若臣書及璁、萼、獻夫、韜,所正取者不過五人。禮科右給事中熊浹、南京刑部郎中黃宗明、都察院經歷黃綰、通政司經歷金述、監生陳雲章、儒士張少璉及楚王、棗陽王二宗室外,所附取者不過六人。有同時建議,若監生何淵、主事王國光、同知馬時中、巡檢房濬,言或未純,義多未正,亦在不取。其他罷職投閒之夫,建言於璁、萼等召用後者,皆望風希旨,有所覬覦,亦一切不錄。其錦衣百戶聶能遷、昌平致仕教諭王價建言三年二三月,未經採入。今二臣奏乞附名,應如其請。」帝從之。因詔「大禮」已定,自今有假言陳奏者,必罪不宥。

至十二年正月,蒲州諸生秦鏜伏闕上書,言:「孝宗之統訖於武宗,則獻皇帝於孝宗實為兄終弟及。陛下承獻皇帝之統,當奉之於太廟,而張孚敬議禮,乃別創世廟以祀之,使不得預昭穆之次,是幽之也。」又謂:「分祀、天、地、日、月於四鄰,失尊卑大小之序。去先師王號,撤其塑像,損其禮樂,增啟聖祠,皆非聖祖之意。請復其初。」帝得奏,大怒。責以毀上不道,下詔獄嚴訊,令供主謀。鏜服妄議希恩,實無主使者。乃坐妖言律論死,繫獄。其後又從豐坊之請,入廟稱宗,以配上帝,則璁輩已死,不及見矣。

贊曰:席書等亦由議禮受知,而持論差平。然事以激成,末流多變。蓋至入廟稱宗,則亦非諸人倡議之初心矣。書、韜在官頗有所建樹,浹、宗明能自斂戢,時論為優。至綰之傾狡,乃不足道矣。


\end{pinyinscope}