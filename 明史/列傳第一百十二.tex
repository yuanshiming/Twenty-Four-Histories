\article{列傳第一百十二}

\begin{pinyinscope}
嚴清宋纁}}陸光祖孫鑨子如法陳有年孫丕揚蔡國珍楊時喬

嚴清,字公直,雲南後衛人。嘉靖二十三年進士。除富順知縣。公廉恤民,治聲大起。憂歸,補邯鄲。入為工部主事,歷郎中。董作京師外城,修九陵,吏無所侵牟,工成加俸。連丁內外艱。服除,補兵部,擢保定知府。故事,歲籍民充京師庫役,清罷之。振荒弭盜,人以比前守吳嶽。歷遷易州副使,陜西參政,四川按察使、右布政使。並以清望,薦章十餘上。隆慶二年,以右僉都御史巡撫貴州。未上,改四川。清久宦川中,僚吏憚其風采,相率厲名行,少墨敗者。郡縣卒歲團操成都,清罷之。番人入貢,裁為定額。痛絕強宗悍吏,毀者亦眾。陜西賊流入境,巡按御史王廷瞻劾清縱寇。大學士趙貞吉言:「賊起鄖、陜,貽害川徼,即有罪,當罪守土臣,不宜專責巡撫。臣蜀人,深知清約己愛人,省事任怨。今蜀地歲荒民流,方倚清如父母,奈何棄之!任事臣欲為國家利小民,必得罪豪右。論者不察,動以深文求之。頃海瑞既去,若清復罷,是任事之臣皆不免彈擊,惟全軀保位為得計矣。」疏奏,不允,命解官聽調。清遂不出。

萬歷二年,起撫山西。未赴,改貴州。歷兩京大理卿,三遷刑部尚書。張居正當國,尚書不附麗者獨清。居正既卒,籍馮保家,得廷臣饋遺籍,獨無清名,神宗深重焉。會吏部尚書梁夢龍罷,即以清代。日討故實,辨官材,自丞佐下皆親署,無一倖進者。中外師其廉儉,書問幾絕。甫半歲,得疾歸。帝數問閣臣:「嚴尚書病愈否?」十五年,兵部缺尚書,用楊博故事,特詔起補。遣使趣行,而清疾益甚,不能赴。又三年卒。贈太子太保,謚恭肅。

清初拜尚書,不能具服色,束素犀帶以朝。或嘲之曰:「公釋褐時,七品玳瑁帶猶在耶?」清笑而已。

宋纁,字伯敬,商丘人。嘉靖三十八年進士。授永平推官。擢御史,出視西關,按應天諸府。隆慶改元,再按山西。俺答陷石州,將士捕七十七人,當斬。纁訊得其誣,釋者殆半。靜樂民李良雨化為女,纁言此陽衰陰盛之象,宜進君子退小人,以挽氣運。帝嘉納之。擢順天府丞,尋以右僉都御史巡撫保定諸府。核缺伍,汰冗兵,罷諸道援兵防禦,省餉無算。

萬曆初,與張居正不合,引疾歸。居正卒,廷臣交薦,以故官撫保定。獲鹿諸縣饑,先振後以聞。帝以近畿宜俟命,令災重及地遠者便宜振貸,餘俱奏聞。尋遷南京戶部右侍郎。召還部,進左,改督倉場。請減額解贖銀,民壯弓兵諸役已裁者,勿徵民間工食。十四年,遷戶部尚書。民壯工食已減半,復有請盡蠲者,纁因并曆日諸費奏裁之。有司徵賦懼缺額,鞭撻取盈,纁請有司考成,視災傷為上下。山西連歲荒,賴社倉獲濟,纁請推行天下,以紙贖為糴本,不足則勸富人,或令民輸粟給冠帶。又言:「邊儲大計,最重屯田、鹽策。近諸邊年例銀增至三百六十一萬,視弘治初八倍,宜修屯政,商人墾荒中鹽。」帝皆稱善。聖節賞賚,詔取部帑銀二十萬兩,纁執奏,不從。潞王將之國,復取銀三十萬兩市珠寶,纁亦力爭,乃減三之一。故事,金花銀歲進百萬兩,帝即位之六年,增二十萬,遂以為常。纁三請停加額,終不許。

纁為戶部五年,值四方多災。為酌盈虛,籌緩急,奏報無需時,上下賴之。而都御史吳時來以吏部尚書楊巍年老求去,忌纁名出己上,兩疏劾,纁因杜門乞休,帝不許。及巍去,卒以纁代之。巍在部,不能止吏奸,且遇事輒請命政府。纁絕請寄,獎廉抑貪,罪黠吏百餘人,於執政一無所關白。會文選員外郎缺官,纁擬起鄒元標。奏不下,再疏趣之。大學士申時行遂擬旨切責,斥元標南京。頃之,以序班盛名昭註官有誤,時行劾奏之。序班劉文潤遷詹事府錄事,時行又劾文潤由輸粟進,不當任清秩。時殿閣中書無不以貲進者,時行獨爭一錄事。纁知其意,五疏乞休。福建僉事李琯言:「時行庇巡撫秦燿,而纁議罷之。仇主事高桂,而纁議用之。以故假小事齮齕,使不得安其位。」帝不納琯言,亦不允纁請。無何,纁卒官。詔贈太子太保,謚莊敬。

纁凝重有識,議事不茍。石星代為戶部,嘗語纁曰:「某郡有奇羨,可濟國需。」纁曰:「朝廷錢穀,寧蓄久不用,勿使搜括無餘。主上知物力充羨,則侈心生矣。」星憮然。有郎言漕糧宜改折,纁曰:「太倉之儲,寧紅腐不可匱絀,一旦不繼,何所措手?」中外陳奏,帝多不省,或直言指斥,輒曰「此沽名耳」,不罪。于慎行稱帝寬大,纁愀然曰:「言官極論得失,要使人主動心;縱罪及言官,上意猶有所儆省。概置勿問,則如痿痺不可療矣。」後果如其言。

陸光祖,字與繩,平湖人。祖淞,父杲,皆進士。淞,光祿卿。杲,刑部主事。光祖年十七,與父同舉於鄉。尋登嘉靖二十六年進士,除浚縣知縣。兵部尚書趙錦檄畿輔民築塞垣,光祖言不便。錦怒,劾之。光祖言於巡撫,請輸雇值,民乃安。郡王奪民產,光祖裁以法。

遷南京禮部主事,請急歸。補祠祭主事,歷儀制郎中。嚴訥為尚書,雅重光祖,議無不行。及訥改吏部,調光祖驗封郎中,改考功。王崇古、張瀚、方逢時、王一鶚掛物議,力雪之。既而改文選,益務汲引人才,登進耆碩幾盡。又破格擢廉能吏王化、江東、邵元善、張澤、李珙、郭文通、蔡琮、陳永、謝侃。或由鄉舉貢士,或起自書吏。由是下僚競勸,訥亦推心任之,故光祖得行其志。左侍郎朱衡銜光祖,有後言,御史孫丕揚遂以專擅劾光祖。時已遷太常少卿,坐落職閒住。

大學士高拱掌吏部,謀傾徐階。階賓客皆避匿,光祖獨為排解。及拱罷,楊博代為吏部,義之,特起南京太僕少卿。未上,擢本寺卿。又就進大理卿。半道丁父艱。萬曆五年,起故官。張居正以奪情杖言者,光祖遺書規之。及王用汲劾居正,居正將中以危禍,光祖時入為大理卿,力解得免。居正與光祖同年相善,欲援為助,光祖無詭隨。及遷工部右侍郎,以議漕糧改折懺居正,御史張一鯤論之,光祖遽引歸。

十一年冬,薦起南京兵部右侍郎。甫旬日,召為吏部。悉引居正所擯老成人,布九列。李植、江東之力求居正罪,光祖言居正輔翼功不可泯,與言路左。植輩以丁此呂故攻尚書楊巍,光祖右巍詆言者。言者遂群攻光祖,乃由左侍郎出為南京工部尚書。御史周之翰劾光祖附宗人炳得清華,帝不問。御史楊有仁遂劾光祖受賕請屬,巍力保持之,事得寢,光祖竟引疾去。

十五年,起南京刑部尚書,就改吏部。率同官劾東廠太監張鯨,且乞宥李沂。已,言國本未定,由鯨構謀,請除之以安宗社。及帝召還鯨,復率同官極諫。入為刑部尚書。帝嘗書其名御屏。吏部尚書宋纁卒,遂用光祖代,而以趙錦代光祖。御史王之棟言二人不當用。帝怒,貶之棟雜職。時部權為內閣所奪,纁力矯之,遂遭挫,光祖不為懾。嘗以事與大學士申時行迕。時行不悅,光祖卒無所徇。時行謝政,特旨用趙志皋、張位,時行所密薦也。光祖言,輔臣當廷推,不當內降。帝命不為後例。

二十年,大計外吏,給事中李春開、王遵訓、何偉、丁應泰,御史劉汝康皆先為外吏,有物議,悉論黜之。又舉許孚遠、顧憲成等二十二人,時論翕然稱焉。頃之,以推用饒伸、萬國欽懺旨,文選郎王教以下盡逐。光祖謂事由己,引罪乞休,為郎官祈宥,不許。及會推閣臣,廷臣循故事,首光祖名。詔報曰:「卿前請廷推,推固宜首卿。」光祖知不能容,日懷去志。無何,以王時槐、蔡悉、王樵、沈節甫老成魁艾,特推薦之,給事中喬胤遂劾光祖及文選郎鄒觀光。光祖遂力求去,許馳驛。在籍五年卒。贈太子太保,謚莊簡。

光祖清強有識,練達朝章。每議大政,一言輒定。初官禮部,將擢尚寶少卿,力讓時槐。丕揚劾罷光祖,後再居吏部,推轂之甚力。趙用賢、沈思孝以論此呂事與光祖左,後亦數推挽之。御史蔡時鼎、陳登雲嘗劾光祖,光祖引登雲為知己。時鼎視鹺兩淮,以建言罷,商人訐於南刑部,光祖時為尚書,雪其誣,罪妄訴者。人服其量。

孫鑨,字文中。父升,字志高,都御史燧季子也。嘉靖十四年進士及第。授編修,累官禮部侍郎。嚴嵩枋國,陞其門生也,獨無所附麗。會南京禮部尚書缺,眾不欲行,陞獨請往。卒,贈太子少保,謚文恪。陞嘗念父死宸濠之難,終身不書寧字,亦不為人作壽文。居官不言人過,時稱篤行君子。四子,鑨、鋌、錝、鑛。鋌,南京禮部右侍郎。錝,太僕卿。鑛自有傳。

鑨舉嘉靖三十五年進士,授武庫主事。歷武選郎中,尚書楊博深器之。世宗齋居二十年,諫者輒獲罪。鑨請朝群臣,且力詆近倖方士,引趙高、林靈素為喻。中貴匿不以聞,鑨遂引疾歸。隆慶元年,起南京文選郎中。萬曆初,累遷光祿卿。引疾歸。里居十年,坐臥一小樓,賓客罕見其面。起故官,進大理卿。都御史吳時來議律例,多紕盭,鑨力爭之。帝悉從駁議。歷南京吏部尚書,尋改兵部,參贊機務。命甫下,會陸光祖去,廷推代者再,乃召為吏部尚書。

吏部自宋纁及光祖為政,權始歸部。至鑨,守益堅。故事,冢宰與閣臣遇不避道,後率引避。光祖爭之,乃復故。然陰戒騶人異道行,至鑨益徑直。張位等不能平,因欲奪其權。建議大僚缺,九卿各舉一人,類奏以聽上裁,用杜專擅。鑨言:「廷推,大臣得共衡可否,此『爵人於朝,與眾共之』之義,類奏啟倖途,非制。」給事中史孟麟亦言之。詔卒如位議。自是吏部權又漸散之九卿矣。

二十一年,大計京朝官,力杜請謁。文選員外郎呂胤昌,鑨甥也,首斥之。考功郎中趙南星亦自斥其姻。一時公論所不予者貶黜殆盡,大學士趙志皋弟預焉。由是執政皆不悅。王錫爵方以首輔還朝,欲有所庇。比至而察疏已上,庇者在黜中,亦不能無憾。會言官以拾遺論劾稽勳員外郎虞淳熙、職方郎中楊於廷、主事袁黃。鑨議謫黃,留淳熙、于廷。詔黃方贊畫軍務,亦留之。給事中劉道隆遂言淳熙、于廷不當議留,乃下嚴旨責部臣專權結黨。鑨言:「淳熙,臣鄉人,安貧好學。于廷力任西事,尚書石星極言其才。今寧夏方平,臣不敢以功為罪。且既名議覆,不嫌異同。若知其無罪,以諫官一言而去之,自欺欺君,臣誼不忍為也。」帝以鑨不引罪,奪其俸,貶南星三官,淳熙等俱勒罷。

鑨遂乞休,且白南星無罪。左都御史李世達以己同掌察,而南星獨被譴,亦為南星、淳熙等訟。帝皆不聽。於是僉都御史王汝訓,右通政魏允貞,大理少卿曾乾亨,郎中于孔兼,員外郎陳泰來,主事顧允成、張納升、賈嚴,助教薛敷教交章訟南星冤,而泰來詞尤切,其略曰:

臣嘗四更京察。其在丁丑,張居正以奪情故,用御史朱璉謀,借星變計吏,箝制眾口。署部事方逢時、考功郎中劉世亨依違其間。如蔡文範、習孔教輩並掛察籍,不為眾所服。辛巳,居正威福已成,王國光唯諾惟謹,考功郎中孫惟清與吏科秦耀謀盡錮建言諸臣吳中行等。今輔臣趙志皋、張位、撫臣趙世卿亦掛名南北京察,公論冤之。丁亥,御史王國力折給事中楊廷相、同官馬允登之邪議。而尚書楊巍素性模棱,考功郎徐一檟立調停之畫。涇、渭失辯,亦為時議所議。獨今春之役,旁咨博採,核實稱情,邪諂盡屏,貪墨必汰;乃至鑨割渭陽之情,南星忍秦、晉之好,公正無踰此者。元輔錫爵兼程赴召,人或疑其欲乾計典。今其親故皆不能庇,欲甘心南星久矣。故道隆章上,而專權結黨之旨旋下。夫以吏部議留一二庶僚為結黨,則兩都大僚被拾遺者二十有二人,而閣臣議留者六,詹事劉虞夔以錫爵門生而留,獨可謂之非黨耶?且部權歸閣,自高拱兼攝以來,已非一日。尚書自張瀚、嚴清而外,選郎自孫鑛、陳有年而外,莫不奔走承命。其流及於楊巍,至劉希孟、謝廷寀而掃地盡矣。尚書宋纁稍欲振之,卒為故輔申時行齮齕以死。尚書陸光祖、文選郎王教、考功郎鄒觀光矢志澄清,輔臣王家屏虛懷以聽,銓敘漸清。乃時行身雖還里,機伏垣牆,授意內榼張誠、田義及言路私人,教、觀光遂不久斥逐。今祖其故智,借拾遺以激聖怒,是內榼與閣臣表裏,箝勒部臣,而陛下未之察也。

疏入,帝怒,謫孔兼、泰來等。世達又抗疏論救,帝怒,盡斥南星、淳熙、于廷黃為民。鑨乃上疏言:「吏部雖以用人為職,然進退去留,必待上旨。是權固有在,非臣部得專也。今以留二庶僚為專權,則無往非專矣;以留二司屬為結黨,則無往非黨矣。如避專權結黨之嫌,畏縮選心耎,使銓職之輕自臣始,臣之大罪也。臣任使不效,徒潔身而去,俾專權結黨之說終不明於當時,後來者且以臣為戒,又大罪也。」固請賜骸骨,仍不允。鑨遂杜門稱疾。疏累上,帝猶溫旨慰留,賜羊豕、酒醬、米物,且敕侍郎蔡國珍暫署選事,以需鑨起。鑨堅臥三月,疏至十上,乃許乘傳歸。居三年卒。贈太子太保,謚清簡。

鑨嘗曰:「大臣不合,惟當引去。否則有職業在,謹自守足矣。」其志節如此。

子如法,官刑部主事。以諫阻鄭貴妃進封,貶潮陽典史。久之,移疾歸。廷臣累薦,悉報寢。卒,贈光祿少卿。

陳有年,字登之,餘姚人。父克宅,字即卿,正德九年進士。嘉靖中官御史。哭爭「大禮」,有大僚欲去,克宅扼其項曰:「奈何先去為人望?」其人愧而止。俄繫獄廷杖。獲釋,先後按貴州、河南,多所彈劾。吏部尚書廖紀姻為所劾罷,惡之,出為松潘副使。累遷右副都御史,巡撫貴州。都勻苗王阿向作亂,據凱口囤。克宅與總兵官楊仁攻斬阿向。論功,進秩。旋移撫蘇、松。既行,而阿向黨復叛,坐罷官候勘。巡撫汪珊討平賊,推功克宅。克宅已卒,乃賜恤典。

有年舉嘉靖四十一年進士,授刑部主事。改吏部,歷驗封郎中。萬曆元年,成國公朱希忠卒,其弟錦衣都督希孝賄中官馮保援張懋例乞贈王,大學士張居正主之。有年持不可,草奏言:「令典:功臣歿,公贈王,侯贈公,子孫襲者,生死止本爵。懋贈王,廷議不可,即希忠父輔亦言之。後竟贈,非制。且希忠無勛伐,豈當濫寵。」左侍郎劉光濟署部事,受指居正,為刪易其稿。有年力爭,竟以原奏上。居正不懌,有年即日謝病去。

十二年起稽勛郎中,歷考功、文選,謝絕請寄。除目下,中外皆服。遷太常少卿,以右僉都御史巡撫江西。尚方所需陶器,多奇巧難成,後有詔許量減,既而如故。有年引詔旨請,不從。內閣申時行等固爭,乃免十之三。南畿、浙江大祲,詔禁鄰境閉糴,商舟皆集江西,徽人尤眾。而江西亦歲儉,群乞有年禁遏。有年疏陳濟急六事,中請稍弛前禁,令江西民得自救。南京御史方萬山劾有年違詔。帝怒,奪職歸。薦起督操江,累遷吏部右侍郎。改兵部,又改吏部。尚書孫鑨、左侍郎羅萬化皆鄉里,有年力引避,朝議不許。

尋由左侍郎擢南京右都御史。二十一年與吏部尚書溫純共典京察,所黜咸當。未幾,遂代純位。其秋,鑨謝事,召拜吏部尚書。止宿公署中,見賓則於待漏所。引用僚屬,極一時選。明年,王錫爵將謝政,廷推閣臣,詔無拘資品。有年適在告,侍郎趙參魯、盛訥、文選郎顧憲成往咨之,列故大學士王家屏、故禮部尚書沈鯉、故吏部尚書孫鑨、禮部尚書沈一貫、左都御史孫丕揚、吏部侍郎鄧以贊、少詹事馮琦七人名上。蓋鑨丕揚非翰林,為不拘資,琦四品,為不拘品也。家屏以爭國本去位,帝意雅不欲用。又推及吏部尚書、左都御史非故事,嚴旨責讓。謂:「不拘資品乃昔年陸光祖自為內閣地。今推鑨、丕揚,顯屬徇私。前吏部嘗兩推閣臣,可具錄姓名以上。」於是備列沈鯉、李世達、羅萬化、陳於陛、趙用賢、朱賡、於慎行、石星、曾同亨、鄧以豎等。而世達故左都御史也,帝復不悅。謂:「詔旨不許推都御史,何復及世達。家屏舊輔臣,不當擅議起用。」乃用命于陛、一貫入閣,而謫憲成及員外郎黃縉、王同休,主事章嘉禎、黃中色為雜職。錫爵首疏救,有年及參魯等疏繼之,帝並不納。趙志皋張位亦佯為言。而二人者故不由廷推,因謂:「輔臣當出特簡,廷推由陸光祖交通言路為之,不可為法。」帝喜。隆旨再譙責,遂免縉等貶謫,但停俸一年。給事中盧明諏疏救憲成。帝怒,貶明諏秩,斥憲成為民。

有年抗疏言:「閣臣廷推,其來舊矣。曩楊巍秉銓,臣署文選,廷推閣臣六人,今元輔錫爵即是年所推也。臣邑前有兩閣臣,弘治時謝遷,嘉靖時呂本,並由廷推,官止四品,而耿裕、聞淵則以吏部尚書居首。是廷推與推及吏部,皆非自今創也。至不拘資品,自出聖諭,臣敢不仰承。」因固乞骸骨。帝得疏,以其詞直,溫旨慰答。有年自是累疏稱疾乞罷。帝猶慰留,賚食物、羊酒。有年請益力。最後,以身雖退,遺賢不可不錄,力請帝起廢。帝報聞。有年遂杜門不出。數月中,疏十四上。乃予告,乘傳歸。歸裝,書一篋,衣一笥而已。二十六年正月卒,年六十有八。四月詔起南京右都御史,而有年已前卒。贈太子太保,謚恭介。

故事,吏部尚書未有以他官起者。屠滽掌都察院,楊博、嚴清掌兵部,皆用原銜領之。南京兵部尚書楊成起掌南院,亦領以故銜。有年以右都御史起,蓋帝欲用之,而政府陰抑之也。有年風節高天下。兩世朊仕,無宅居其妻孥,至以油幙障漏。其歸自江西,故廬火,乃僦一樓居妻孥,而身棲僧舍。其刻苦如此。

孫丕揚,字叔孝,富平人。嘉靖三十五年進士。授行人。擢御史。歷按畿輔、淮、揚,矯然有風裁。隆慶中,擢大理丞。以嘗劾高拱,拱門生給事中程文誣劾丕揚,落職候勘。拱罷,事白,起故官。

萬歷元年擢右僉都御史,巡撫保守諸府。以嚴為治,屬吏皆惴惴。按行關隘,增置敵樓三百餘所,築邊牆萬餘丈。錄功,進右副都御史。中官馮保家在畿內,張居正屬為建坊,丕揚拒不應。知二人必怒,五年春引疾歸。

其冬大計京官,言路希居正指劾之。詔起官時,調南京用。御史按陜西者,知保等憾不已,密諷西安知府羅織其贓。知府遣吏報御史,吏為虎噬。及再報,則居正已死,事乃解。起應天府尹。召拜大理卿,進戶部右侍郎。

十五年,河北大饑。丕揚鄉邑及鄰縣蒲城、同官至採石為食。丕揚傷之,進石數升於帝,因言:「今海內困加派,其窮非止啖石之民也。宜寬賦節用,罷額外徵派有諸不急務,損上益下,以培蒼生大命。」帝感其言,頗有所減罷。

尋由左侍郎擢南京右都御史,以病歸。召拜刑部尚書。丕揚以獄多滯囚,由公移牽制。議刑部、大理各置籍,凡獄上刑部,次日即詳讞大理,大理審允,次日即還刑部,自是囚無淹繫。尋奏:「五歲方恤刑,恐冤獄無所訴。請敕天下撫按,方春時和,令監司按行州縣,大錄繫囚,按察使則錄會城囚。死罪矜疑及流徒以下可原者,撫按以達於朝,期毋過夏月。輕者立遣,重者仍聽部裁,歲以為常。」帝報從之。已,條上省刑省罰各三十二事。帝稱善,優詔褒納。自是刑獄大減。有內堅殺人,逃匿禁中。丕揚奏捕,卒論戍。改左都御史。陳臺規三事,請專掌印、重巡方、久巡城,著為令。已,又言:「閭閻民瘼非郡邑莫濟,郡邑吏治非撫按監司莫清。撫按監司風化,非部院莫飭。請立約束頒天下,獎廉抑貪,共勵官箴。」帝咸優詔報許。

二十二年,拜吏部尚書。丕揚挺勁不撓,百僚無敢以私干者,獨患中貴請謁。乃創為掣簽法,大選急選,悉聽其人自掣,請寄無所容。一時選人盛稱無私,然銓政自是一大變矣。二十三年,大計外吏。九江知府沈鐵嘗為衡州同知,發巡撫秦耀罪,江西提學僉事馬猶龍嘗為刑部主事,定御史祝大舟贓賄,遂為庇者所惡。考功郎蔣時馨黜之,丕揚不能察。及時馨為趙文炳所劾,丕揚力與辨雪。謂釁由丁此呂,此呂坐逮。丕揚又力詆沈思孝,於是思孝及員外郎岳元聲連章訐丕揚。丕揚請去甚力。其冬,帝以軍政故,貶兩京言官三十餘人。丕揚猶在告,偕九卿力諫,弗納。已而帝惡大學士陳于陛論救,謫諸言官邊方。丕揚等復抗疏諫,帝益怒,盡除其名。

初,帝雖以夙望用丕揚,然不甚委信。有所推舉,率用其次。數請起廢,輒報罷。丕揚以志不行,已懷去志,及是杜門踰半歲。疏十三上,多不報。至四月,溫諭勉留,乃復起視事。主事趙學仕者,大學士志皋族弟也,坐事議調,文選郎唐伯元輒注饒州通判。俄學仕復以前事被訐,給事中劉道亨因劾吏部附勢,語侵丕揚。博士周獻臣有所陳論,亦頗侵之。丕揚疑道亨受同官周孔教指,獻臣又孔教宗人,益疑之,復三疏乞休。最後貽書大學士張位,懇其擬旨允放。位如其言。丕揚聞,則大恚,謂位逐己,上疏詆位及道亨、孔教、獻臣、思孝甚力。帝得疏,不直丕揚。位亦疏辯求退,帝復詔慰留,而位同官陳于陛、沈一貫亦為位解。丕揚再被責讓,許馳傳去。

久之,起南京吏部尚書,辭不就。及吏部尚書李戴免,帝艱其代,以侍郎楊時喬攝之。時喬數請簡用尚書。帝終念丕揚廉直,三十六年九月,召起故官。屢辭,不允。明年四月始入都,年七十有八矣。三十八年大計外吏,黜陟咸當。又奏舉廉吏布政使汪可受、王佐、張人思等二十餘人,詔不次擢用。

先是,南北言官群擊李三才、王元翰,連及里居顧憲成,謂之東林黨。而祭酒湯賓尹、諭德顧天颭各收召朋徒,干預時政,謂之宣黨、崑黨;以賓尹宣城人,天颭崑山人也。御史徐兆魁、喬應甲、劉國縉、鄭繼芳、劉光復、房壯麗,給事中王紹徽,硃一桂、姚宗文、徐紹吉、周永春輩,則力排東林,與賓尹、天颭聲勢相倚,大臣多畏避之。至是,繼芳巡按浙江,有偽為其書抵紹徽、國縉者,中云「欲去福清,先去富平;欲去富平,先去耀州兄弟」。又言「秦脈斬斷,吾輩可以得志」。福清謂葉向高,耀州謂王國、王圖,富平即丕揚也。國時巡撫保定,圖以吏部侍郎掌翰林院,與丕揚皆秦人,故曰「秦脈」。蓋小人設為挑激語,以害繼芳輩,而其書乃達之丕揚所。丕揚不為意。會御史金明時居官不職,慮京察見斥,先上疏力攻圖,并詆御史史記事、徐縉芳,謂為圖心腹。及圖、縉芳疏辯,明時再劾之,因及繼芳偽書事。國縉疑書出縉芳及李邦華、李炳恭、徐良彥、周起元手,因目為「五鬼」;五人皆選授御史候命未下者也。當是時,諸人日事攻擊,議論紛呶,帝一無所問,則益植黨求勝,朝端哄然。

及明年三月,大計京官。丕揚與侍郎蕭雲舉、副都御史許弘綱領其事,考功郎中王宗賢、吏科都給事中曹于汴、河南道御史湯光京、協理御史喬允升佐之。故御史康丕揚、徐大化,故給事中鐘兆斗、陳治則、宋一韓、姚文蔚,主事鄭振先、張嘉言及賓尹、天颭、國縉咸被察,又以年例出紹徽、應甲於外。群情翕服,而諸不得志者深銜之。當計典之初舉也,兆京謂明時將出疏要挾,以激丕揚。丕揚果怒,先期止明時過部考察,特疏劾之。旨下議罪,而明時辯疏復犯御諱。帝怒,褫其職。其黨大嘩。謂明時未嘗要挾兆京,只以劾圖一疏實之,為圖報復。於是刑部主事秦聚奎力攻丕揚,為賓尹、大化、國縉、紹徽、應甲、嘉言辨。時部院察疏猶未下,丕揚奏趣之,因發聚奎前知績溪、吳江時貪虐狀。帝方向丕揚,亦褫聚奎職。由是黨人益憤,謂丕揚果以偽書故斥紹徽、國縉,且二人與應甲嘗攻三才、元翰,故代為修隙,議論洶洶。弘綱聞而畏之。累請發察疏,亦若以丕揚為過當者。黨人藉其言,益思撼丕揚。禮部主事丁元薦甫入朝,慮察疏終寢,抗章責弘綱,因盡發崑、宣黨構謀狀。於是一桂、繼芳、永春、光魁、宗文爭擊元薦,為明時等訟冤。賴向高調獲,至五月察疏乃下。給事中彭惟成、南京給事中高節,御史王萬祚、曾成易猶攻訐不已。丕揚以人言紛至,亦屢疏求去,優詔勉留。先是,楊時喬掌察,斥科道錢夢皋等十人,特旨留任。至是丕揚亦奏黜之,群情益快。

丕場以白首趨朝,非薦賢無以報國。先後推轂林居耆碩,若沈鯉、呂坤、郭正域、丘度、蔡悉、顧憲成、趙南星、鄒元標、馮從吾、于玉立、高攀龍、劉元珍、龐時雍、姜士昌、范淶、歐陽東鳳輩。帝雅意不用舊人,悉寢不報。丕揚又請起故御史錢一本等十三人,故給事中鐘羽正等十五人,亦報罷。丕揚齒雖邁,帝重其老成清德,眷遇益隆。而丕揚乞去不已,疏復二十餘上。既不得請,則於明年二月拜疏徑歸。向高聞之,急言於上。詔令乘傳,且敕所司存問。既而丕揚疏謝,因陳時政四事,帝復優詔報之。家居二年卒,年八十三。贈太保。天啟初,追謚恭介。

蔡國珍,字汝聘,奉新人。嘉靖三十五年進士。鄉人嚴嵩當國,欲羅致門下。國珍不應,乞就南,為刑部主事。盜七十餘人久繫,讞得其情,減釋過半。就改吏部,進郎中。出為福建提學副使,以侍養歸。遭母喪。服除,遂不出。家居垂二十年。張居正既卒,朝議大起廢籍。萬曆十一年,仍以故官蒞福建。遷湖廣右參政,分守辰沅。洞蠻亂,將吏議剿,國珍檄諭之,遂定。歷浙江左布政使,以右僉都御史提督操江。召為左副都御史,歷吏部左、右侍郎,與尚書孫鑨、陳有年綜核銓政。擢南京吏部尚書。

二十四年閏八月,孫丕揚去國,帝久不除代。部事盡弛,其年十二月竟廢大選。閣臣及言官數為言,明年二月,始命國珍為吏部尚書。三殿災,率諸臣請修省。旋有詔起廢。國珍列三等,人品正大、心術光明者,文選郎王教等二十四人;才有足錄、過無可棄者,給事中喬允等三十三人;因人詿誤、釁非己作者,給事中耿隨龍等三十六人,並請錄用。竟報寢。明年三月,倡廷臣詣文華門請舉皇長子冊立、冠婚,言必得請方退。帝遣中官諭曰:「此大典,稍需時耳,何相挾若是!」乃頓首出。給事中戴士衡劾文選郎白所知贓私,國珍為辨,且求罷。帝不聽,除所知名。御史況上進因諭國珍八罪。帝察其誣,不問。國珍遂稱疾,累疏乞休。先是,丕揚坐忤張位去官,位欲援同己者為助,以國珍鄉人,汲引甚力。及秉銓,一守成憲,不為位用。位惡之,國珍乃懷去志。至是,帝忽怒吏部,貶黜諸郎二十二人。國珍求去益力,許乘傳歸。

初,楊巍為吏部,與內閣相比,得居位八年。自宋纁、陸光祖力與閣抗,權雖歸部,身不容,故自纁至國珍卒未浹歲去,惟丕揚閱二年。時咸議閣臣怯,而惜糸熏等用未竟也。國珍素以學行稱,風力不及孫鑨、陳有年,而清操似之,均為時望所屬。家居十三年卒,年八十四。贈太子太保,謚恭靖。楊時喬,字宜遷,上饒人。嘉靖四十四年進士。除工部主事。榷稅杭州,令商人自署所入,輸之有司,無所預。隆慶元年冬,上時政要務,言:「幾之當慎者三,以日勤朝講為修德之幾,親裁章奏為出令之幾,聽言能斷為圖事之幾。弊之最重者九:曰治體怠弛,曰法令數易,曰賞罰無章,曰用度太繁,曰鬻官太濫,曰莊田擾民,曰習俗侈靡,曰士氣卑弱,曰議論虛浮。勢之偏重者三:宦寺難制也,宗祿難繼也,邊備難振也。」疏入,帝褒納,中外傳誦焉。

擢禮部員外郎,遷南京尚寶丞。萬曆初,以養親去。服除,起南京太僕丞,復遷尚寶。移疾歸。時喬雅無意榮進,再起再告。閱十七年始薦起尚寶卿,四遷南京太常卿。疏請議建文帝謚,祠禮死節諸臣。就遷通政使。秩滿,連章乞休,不允。三十一年冬,召拜吏部左侍郎。時李戴已致仕,時喬至即署部事。絕請謁,謝交遊,止宿公署,苞苴不及門。及大計京朝官,首輔沈一貫欲庇其所私,憚時喬方正,將令兵部尚書蕭大亨主之,次輔沈鯉不可而止。時喬乃與都御史溫純力鋤政府私人。若給事中錢夢皋、御史張似渠、于永清輩,咸在察中,又以年例出給事中鐘兆斗於外。一貫大慍,密言於帝,留察疏不下。夢皋亦假楚王事再攻郭正域,謂主察者為正域驅除。帝意果動,特留夢皋;已,盡留科道之被察者,而嚴旨責時喬等報復。時喬等惶恐奏辨,請罷斥,帝不問。夢皋既留,遂合兆斗累疏攻純,并侵時喬。時喬求去。已而員外郎賀燦然請斥被察科道,亦詆純挾權斗捷,顧獨稱時喬。又言:「陛下睿斷躬操,非閣臣所能竊弄」,意蓋為一貫解。時喬以與純共事,復疏請貶黜,不報。及純去,夢皋、兆斗亦引歸。帝復降旨譙讓,謂「祖宗朝亦常留被察科道,何今日揣疑君父,誣詆輔臣」。因責諸臣朋比,令時喬策勵供職,而盡斥燦然及劉元珍、龐時雍輩。時喬歎曰:「主察者逐,爭察者亦竄矣,尚可靦顏居此乎?」九疏引疾,竟不得請。時中外缺官多不補,而群臣省親養病給假,及建言詿誤被譴者,充滿林下,率不獲召。時喬乃備列三百餘人,三疏請錄用。三十四年,皇長孫生,有詔起廢,時喬復列上遷謫鄒元標等九十六人,削籍范俊等一百十人。帝卒不省。

明年,大計外吏。時喬已偕副都御史詹沂受事,居數日,帝忽命戶部尚書趙世卿代時喬,遂中輟;蓋去冬所批察疏,至是誤發之也。輔臣朱賡謂非體,立言於帝。帝亦覺其誤,即日收還。時喬堅辭不肯任,吏科陳治則劾其怨懟無人臣禮。有旨詰責,時喬乃再受事。永年伯王棟卒,其子明輔請襲。時喬以外戚不當傳世,固爭之,弗聽。時一貫已罷,言路爭擊其黨。而李廷機者,一貫教習門生也,閣臣闕,眾多推之;惟給事中曹于汴、宋一韓、御史陳宗契持不可。時喬卒從眾議。未幾,又推黃汝良、全天敘為侍郎,諸攻一貫者益不悅。給事中王元翰、胡忻遂交劾時喬。時喬疏辨,力求罷。

當是時,帝委時喬銓柄,又不置右侍郎,一人獨理部事,銓敘平允。然堂陛扞格,曠官廢事,日甚一日,而中朝議論方囂,動見掣肘。時喬官位未崇,又自溫純去,久不置都御史,益無以鎮厭百僚。由是上下相凌,紀綱日紊,言路得收其柄。時喬亦多委蛇,議者諒其苦心,不甚咎也。秉銓凡五年。最後起故尚書孫丕揚。未至,而時喬已卒。篋餘一敝裘,同列賻襚以殮。詔贈吏部尚書,謚端潔。

時喬受業永豐呂懷,最不喜王守仁之學,闢之甚力,尤惡羅汝芳。官通政時具疏斥之曰:「佛氏之學,初不溷於儒。乃汝芳假聖賢仁義心性之言,倡為見性成佛之教,謂吾學直捷,不假修為。於是以傳註為支離,以經書為糟粕,以躬行實踐為迂腐,以綱紀法度為桎梏。踰閑蕩檢,反道亂德,莫此為甚。望敕所司明禁,用彰風教。」詔從其言。

贊曰:古者冢宰統百官,均四海,即宰相之任也。後代政柄始分,至明中葉,旁撓者眾矣。嚴清諸人,清公素履,秉正無虧,彼豈以進退得失動其心哉。孫丕揚創掣簽法,雖不能辨材任官,要之無任心營私之弊,茍非其人,毋寧任法之為愈乎!蓋與時宜之,未可援古義以相難也。


\end{pinyinscope}