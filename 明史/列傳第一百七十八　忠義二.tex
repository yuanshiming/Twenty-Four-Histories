\article{列傳第一百七十八 忠義二}

\begin{pinyinscope}
○王冕龔諒陳聞詩董倫王鈇錢泮錢錞唐一岑朱裒齊恩孫鏜杜槐黃釧陳見等王德叔沛汪一中王應鵬唐鼎蘇夢暘韋宗孝龍旌張振德章文炳等董盡倫李忠臣高光等龔萬祿李世勛翟英等管良相李應期等徐朝綱楊以成孫克恕鄭鼎姬文胤孟承光朱萬年秦三輔等張瑤王與夔等何天衢楊于陛

王冕,字服周,洛陽人。正德十二年進士。除萬安知縣。宸濠反,長吏多奔竄。冕募勇壯士,得死士數千人,從王守仁攻復南昌。宸濠解安慶圍,還救,至鄱陽湖,兩軍相拒。濠盡出金制犒士,殊死戰,官軍不利。冕密白守仁,以小艇實葦於中,擬建昌人語,就賊艦,乘風舉火。濠兵大驚,遂潰敗,焚溺死者無算。濠易舟,挾宮人遁。冕部卒棹漁舟,追執之。宸濠平,守仁封新建伯,而冕未及敘,坐他事落職。既而錄前功,擢兵部主事,巡視山海關。

嘉靖三年十二月,遼東妖賊陸雄、李真等作亂,突入關。侍吏欲扶冕趨避,冕不可,曰:「吾有親在。」急趨母所,執兵以衛。賊至,母被傷,冕奮前救之,被執。脅以刃,大罵,遂見害。詔贈光祿少卿,有司祠祀。

世宗嗣位之歲,寧津盜起,轉掠至德平。知縣龔諒率吏民禦之,力屈,被殺。贈濟南通判,恤其家。

陳聞詩,字廷訓,柘城人。嘉靖中舉於鄉,以親老,絕意仕進。親歿,居喪哀毀。三十二年秋,賊師尚詔陷歸德,聞聞詩名,欲劫為帥。已,陷柘城,擁之至,誘說百端,不屈。引其家數人斬之,曰:「不從,滅而族。」聞詩紿曰:「必欲吾行,毋殺人,毋縱火。」賊許諾,擁以行。聞詩遂不食,至鹿邑自經死。

董倫,歸德檢校也。尚詔入歸德,知府及守衛官皆遁。倫率民兵巷戰,被執,垂死猶手刃數賊。妻賈及童僕皆從死。詔贈聞詩鳳陽同知,倫歸德同知,並立祠死所。

王鈇,字德威,順天人。嘉靖二十九年進士。授常熟知縣。濱海多大猾,匿亡命作奸,鈇悉貰其罪。倭患起,鈇語諸猾曰:「何以報我?」咸請效死,於是立耆長,部署子弟得數百人,合防卒訓練。縣故無城,鈇率士卒城之。倭來薄,數禦卻之。已,自三丈浦分掠常熟、江陰。參政任環令鈇與指揮孔燾分統官民兵三千,破其寨,斬首百五十有奇,焚二十七艘,餘倭皆遁。復掠勞縣,將由尚湖還海。鈇憤曰:「賊尚敢涉吾地邪!必擊殺之。」

會邑人錢泮,字鳴聲者,以江西參政里居,仇倭爇其父柩,力從臾贊鈇。乃用小艇數十躡倭,倭夾擊之隘中,獨耆長數人從,皆力鬥死。鈇陷淖,瞋目大呼,腹中刃死。泮被數鎗,殺三賊而死。時三十四年五月也。詔贈泮光祿卿,鈇太僕少卿,並廕錦衣世百戶,遣官諭祭,立祠死所,歲時奉祀。

錢錞,字鳴叔,鐘祥人。嘉靖二十九年進士。授江陰知縣。初至官,倭已熾。三十三年入犯,鄉民奔入城者萬計,兵備道王從古不納。錞曰:「民死不救,守空城奚為!」遂開門縱之入,而身自搏戰於斜橋,三戰卻之。明年六月,倭據蔡涇閘,分眾犯塘頭。錞提狼兵戰九里山,薄暮,雷雨大作,伏四起,狼兵悉奔,錞戰死。

時唐一岑知崇明縣,建新城成,議徙居,為千戶高才、翟欽所沮。倭突入,一岑戰且詈,遂為亂軍所殺。詔贈錞、一岑光祿少卿,錞世廕錦衣百戶,岑廕國子生,並建祠祀。

朱裒,字崇晉,鄖西人。嘉靖中舉於鄉,署鞏縣教諭事。遷武功知縣,抑豪強,祛積弊,關中呼為鐵漢。遷揚州同知,吏無敢索民一錢。三十四年,倭入犯,擊敗之沙河,殲其酋,還所掠牲畜甚眾。未幾,復大至,薄城東門。督兵奮擊,兵潰,死焉。贈左參政,錄一子。

明年,倭犯無為州,同知齊恩率舟師敗倭於圌山北等港,斬首百餘級。子嵩,年十八,最驍勇,擊倭至安港,伏發被圍,恩家二十餘人俱力戰死,惟嵩等三人獲全。贈恩光祿丞,錄一子,厚恤其家,建祠祀之。

孫鏜,莒州人。商販吳、越。倭擾松江,謁郡守自請輸貲佐軍。守薦之參政翁大立,試以只刀,若飛,錄為士兵。擊走倭,出參政任環圍中。遣人還莒,括家貲,悉召里兒為爪牙,吳中倚鏜若長城。倭舟渡泖滸,鏜突出,酣戰竟日,援兵不至,還至石湖橋,半渡,伏大起,鏜墮死,中刃死。贈光祿丞,錄一子,亦建祠祀。

杜槐,字茂卿,茲谿人。倜儻任俠。倭寇至,縣僉其父文明為部長,令團結鄉勇。槐傷父老,以身任之,數敗倭。副使劉起安委槐守餘姚、慈谿、定海。遇倭定海之白沙,一日戰十三合,斬三十餘人,馘一酋,身被數槍,墮馬死。文明擊倭鳴鶴場,斬一人,倭驚遁,稱為杜將軍。無何,追至奉化楓樹嶺,戰歿。文明贈府經歷,槐贈光祿丞,建祠並祀,廕槐子國子生。

黃釧,字珍夫,安溪人。由舉人歷官溫州同知。嘉靖三十四年,倭入犯,釧擊走之。知倭必復來,日夜為備。又三年,倭果大至。釧出城逆擊,分軍為三,釧將中軍,其二軍帥皆紈褲子,約左右應援。及與倭遇,倭遣眾分掩二軍,而以銳卒當中軍。釧發勁弩巨礮,戰良久,倭方不支,二軍帥望敵而潰。倭合兵擊釧,釧腹背受敵,遂被執。脅之降,不屈,責以金贖,釧笑且罵曰:「爾不知黃大夫不愛錢邪!」倭怒,裸而寸斬之。子購屍不獲,具衣冠葬。事聞,贈浙江參議,官一子,有司建祠。

是年,倭陷福清,舉人陳兒率眾禦之,與訓導鄔中涵被執,大罵而死。倭乘勝犯惠安,知縣番禺林咸拒守五晝夜,倭引去。已,復至,咸擊之鴨山,窮追逐北,陷伏死。贈泉州同知,賜祠,任一子。

其陷興化,延平同知奚世亮署府事,守踰月,城陷,力戰死。贈右參議,廕子,賜葬。世亮,字明仲,黃岡人。

先是,三十一年,台州知事溧水武追倭釣魚嶺,力戰死,上官不以聞。其子尚寶訴於朝,乃贈太僕丞,而廕尚實為國子生。

王德,字汝修,永嘉人。嘉靖十七年進士。歷戶科給事中。定國公徐延德丐無極諸縣閒田為業,且言私置莊田,不宜以災傷免賦。德抗疏劾之,俺答圍都城,屢陳軍國便宜,悉報可。時城門盡閉,避難者不得入,號呼徹西內。德以為言,民始獲入。寇退,命募兵山東,所得悉驍勇,為諸道最。還朝,會李默長吏部,怒德投刺倨,出為嶺南兵備僉事。與巡撫爭事,投劾徑歸。默復起吏部,用前憾,落職閒住。德鄉居,以倭亂,奉母居城中,傾貲募健兒為保障計。三十七年夏,倭自梅頭至,大掠。德偕族父沛督義兵擊之,宵遁。俄一舟突來犯,沛及族弟崇堯、崇修殲焉。亡何,倭復至,大掠。德憤怒,勒所部追襲至龍灣,軍敗,手射殺數人,罵賊死。然倭自是不敢越德鄉侵郡城矣。事聞,贈太僕少卿,世廕錦衣百戶,立祠曰愍忠。沛贈太僕丞,立祠,予廕。

汪一中,字正叔,歙人。嘉靖二十三年進士。由開封推官歷江西副使。四十年,鄰境賊入寇,薄泰和。一中方宴,投著起曰:「賊鼓行而西,掩我不備,不早計,且無唯類,豈飲酒時乎!」當路遂以討賊屬之。先是,泰和巡檢劉芳力戰死,賊怒磔其尸。一中至,率諸將吏祭曰:「爾職抱關,猶死疆事。吾待罪方面,不滅賊,何以生為!」遂誓師,列陣鼓之,俘五人,斬首以徇。旦日,陣如前,會賊至,左右軍皆潰,賊悉赴中軍,中軍亦潰。一中躍馬當賊鋒,射殺二人,手刃一人,而左脅中槍二,臂中刃三,與指揮王應鵬、千戶唐鼎皆死。妻程投於井,家人出之,喪至,不食五日死。一中贈光祿卿,給祭葬,謚忠愍,妻程並贈恤如制。

蘇夢暘,萬曆間,為雲南祿豐知縣。三十五年十二月,武定賊鳳騰霄反,圍雲南府城,轉寇祿豐。夢暘率民兵出城力戰,賊退去。明年元旦,方朝服祝釐,賊出不意襲陷其城,執之去,不屈死。贈光祿少卿,有司建祠,錄一子。

當祿豐之未陷也,賊先犯嵩明州,吏目韋宗孝出禦而敗,合門死之。贈本州同知,廕子入國學。

有龍旌者,趙州人,由歲貢生為嵩明州學正。賊薄城,被執,罵賊死。贈國子博士。

張振德,字季修,崑山人。祖情,從祖意,皆進士。情福建副使,意山東副使。振德由選貢生授四川興文知縣。縣故九絲蠻地,萬曆初,始建士牆數尺,戶不滿千。永寧宣撫奢崇明有異志,潛結奸人,掠賣子女。振德捕奸人,論配之,招還被掠者三百餘人。崇明賄以二千金,振德怒卻之,裂其牘。

天啟元年方赴成都與鄉闈事,而崇明部將樊龍殺巡撫徐可求,副使駱日升、李繼周等。重慶知府章文炳、巴縣知縣段高選皆抗節死,賊遂據重慶。時振德兼署長寧,去賊稍遠,從者欲走長寧。振德曰:「守興文,正也。」疾趨入城。長寧主簿徐大禮與振德善,以騎來迎,振德卻之。督鄉兵與戰,不敵,退集居民城守。會大風雨,賊毀士城入。振德命妻錢及二女持一劍坐後堂,曰:「若輩死此,吾死前堂。」乃取二印繫肘後,北向拜曰:「臣奉職無狀,不能殺賊,惟一死明志。」妻女先伏劍死。乃命家人舉火,火熾自剄。一門死者十二人。賊至火所,見振德面如生,左手繫印,右手握刀,忿怒如赴敵狀,皆駭愕,羅拜而去。事聞,賜祭葬,贈光祿卿,謚烈愍。敕有司建祠,世廕錦衣千戶。

振德既死,興文教諭劉希文代署縣事。甫半載,賊復薄城,誓死不去。妻白亦慷慨願同死。城破,夫婦罵賊,並死。

大禮守長寧,城亦陷。大禮曰:「吾不可負張公。」一家四人仰藥死。贈重慶同知,世廕百戶。

文炳,長泰人。萬曆四十一年進士。歷戶部郎中,遷知府,治行廉潔,吏民愛之。賊既殺巡撫可求等,文炳罵賊亦被殺。後知其賢,為覓屍殯而歸之,喪出江上,夾岸皆大哭。贈太僕少卿,再贈太常卿,世廕外衛副千戶。

高選,雲南劍川縣人。萬曆四十七年進士。適在演武場,聞變,立遣吏歸印於署,厲聲叱賊。賊魁戒其下勿殺,而高選罵不絕聲,遂遇害。父汝元,母劉,側室徐及一子一女,聞變,皆自盡。僕冒死覓主屍,亦被害。初贈尚寶卿,世廕百戶。崇禎元年,子暄援振德例,叩閽請優恤,贈光祿卿,世廕錦衣千戶,建祠奉禮。汝元等亦獲旌。十五年復以謚請,賜謚恭節。

時先後殉難者,灌縣知縣左重,率壯士追賊成都,力戰馬蹶,罵賊死。南溪知縣王碩輔,城陷自盡,賊支解之。桐梓知縣洪維翰,城陷,奪印,不屈死。典史黃啟鳴亦死。郫縣訓導趙愷,率眾擊賊,被刺死。遵義推官馮鳳雛,挺身禦賊,被創死。遵義司獄蘇樸、威遠經歷袁一修,義不汙賊,墜城死。大足主簿張志譽、典史寧應皋,集兵奮戰,力屈死。所司上其狀,贈重、碩輔、維翰尚寶卿,世廕千戶。啟鳴重慶通判,愷重慶同知,俱世廕試百戶。崇禎十二年,重子廷皋援高選例乞恩,命如其請。

崇明父子據永寧,貴陽同知嘉興王昌胤分理永寧衛事,死難。贈僉事,賜祭。崇禎初,其子監生世駿言:「賊踞永寧,臣父刺血草三揭,繳印上官,以次年五月再拜自縊。賊恨之,焚其屍。二孫、一孫女及僕婢十三人,同日被害。乞如張振德例,優加恤典。」報可。

董盡倫,字明吾,合州人。萬歷中舉於鄉,除清水知縣,調安定,咸有惠政。秩滿,安定人詣闕奏留,詔加鞏昌同知,仍視縣事。久之,以同知理甘州軍餉,解職歸。天啟初,奢崇明反,率眾薄城。盡倫偕知州翁登彥固守。賊遣使說降,盡倫大怒,手刃賊使,抉其晴啖之,屢挫賊鋒,城獲全。復率眾援銅梁有功,尋被檄搗重慶,孤軍深入,伏四起,遂虞死。贈光祿少卿,世廕百戶,建祠奉祀,尋改廕指揮僉事。崇禎初,論全城功,改廕錦衣千戶。

其時里居士大夫死節者,有李忠臣,永寧人,官松潘參政。家居,陷賊。募死士,密約總兵官楊愈懋,令以大兵薄城,己為內應。事洩,合門遇害。高光,瀘州人,嘗為應天通判。城陷,薙髮為僧,與子在崑募壯士,殺賊百餘。賊怒,追至大葉壩,光罵賊不屈,與家眾十二人同死。胡縝,永寧舉人。預策崇明必反,上書當事,不納。賊起,被執,嚴刑錮獄中。弟緯傾家救免,乃糾義徒,潛結賊將張令等,執其偽相。部勒行陣,自當一面,數斬馘,賊甚畏之。既而為火藥焚死。聶繩昌,富順舉人。毀家募義勇禦賊,戰死。吳長齡,瀘州監生。率眾恢復瀘州,尋中伏,父子俱戰死。胡一夔,興文人。仕龍陽縣丞,被執,不屈死。皆未予恤。

龔萬祿,貴州人。目不知書,有膽志,膂力過人。從劉綎征楊應龍,先登海龍囤,署守備,戍建武所。奢崇明反,眾推萬祿遊擊將軍,主兵事。指揮李世勛,名位先萬祿,亦受節制,戮力固守。崇明謀犯成都,憚萬祿牽其後,遣部將張令說降。令與萬祿結,紿崇明以降。崇明果遣他將來戍,萬祿脅降之,誘殺無算。復微服走敘州,說副使徐如珂曰:「賊精騎萃成都,留故巢者悉老弱,誠假萬祿萬人搗其巢,彼必還救,成都圍立解矣。」如珂奇其計,而不能用。未幾,賊悉眾攻建武,萬祿邀擊十里外,兵少敗還,城遂陷。世勛具衣冠再拜,率家屬自焚死。萬祿手刃兩妾、兩孫,自刎不殊,乃握槊馳出,大呼:「我龔萬祿也,孰能追我者!」賊相視不敢逼。走至敘州,乞師巡撫朱燮元,遂以兵復建武。會官軍敗於江門,賊四面來攻,萬祿力戰三日,手刃數十人,與子崇學並死。詔贈都督僉事,立祠賜祭,世廕百戶。

時成都衛指揮翟英扼賊龍泉驛,成都後衛指揮韓應泰赴援成都,遇賊草堂寺,小河所鎮撫郁聯若鏖賊城西,茂州百戶張羽救援郫縣,皆力戰死。

管良相者,烏撒衛指揮也,為人慷慨負奇節。天啟初,樊龍等反於四川,巡撫李枟召至麾下,與籌軍事。良相策安邦彥必反,佐枟為固守計。尋以祖母疾,乞假婦,泣語枟曰:「烏撒孤城,密邇水西,且與安效良相仇。水西有變,禍必首及,良相無子,願以死報國。乞建長策,保此一方。」逾月,邦彥果反,圍其城,良相固守不下。久之,外援不至,城陷,自縊死。

同官李應期、朱運泰、蔣邦俊亦遇害。時普定衛王明重、威清衛丘述堯、平壩衛金紹勛、壩陽把總簡登、龍里故守備劉皋、皋子景並死難,而訓導劉三畏,賊至不避,兀坐齋中,見殺,人稱「龍里三劉」。

徐朝綱,雲南晉寧人。萬歷二十八年舉於鄉。天啟元年,授安順推官,至即署府事。明年,安邦彥反,來攻城,朝綱督兵民共守。士官溫如璋等開門迎賊,朝綱奮怒督戰,賊執之,逼降,不屈。索其印,罵曰:「死賊奴,吾頭可斷,印不可得!」賊怒,刀斧交下而死。其妻聞之,登樓自縊。長子婦急舉火焚舍,挈十歲女躍烈焰中死。孫應魁,年十六,持矛潰圍出城覓其祖,遇賊被殺。婢僕從死者十一人。

五年正月恤殉難諸臣,贈朝綱光祿少卿,廕子入國學。子天鳳甫第進士,即奔喪歸,服闋,授戶部主事。疏言:「臣家一門,臣死忠,妻死節,婦死姑,孫死祖,婢僕死主。此從來未有之節烈,乞如張振德例,再加優恤。臣母、臣嫂,一體旌表。」帝深嘉之,再贈光祿卿,改廕錦衣世千戶,賜祭葬,立祠建坊,諸從死者皆附祀。

同時殉難者:

楊以成,雲南路南人。萬歷中,由貢生授貴陽通判,理畢節衛事。秩滿,進同知,仍治畢節。邦彥圍貴陽,以成具蠟書乞援於雲南巡撫沈儆炌。書發而賊已至,戰卻之。賊來益眾,以成遣吏懷印間道趨省,身督吏民拒守。會援兵至,賊方夜逃,而衛吏阮世爵為內應,城遂陷。以成倉皇投繯,賊系之去。乃為書述賊中情形,置竹筒中,遣弟以恭赴雲南告變,至散納溪,賊搜得其書,并以成殺之,家屬死者十三人。贈按察僉事,賜葬。

鄭鼎,字爾調,龍溪人。由鄉舉為廣順知州。策安邦彥必反,上書當事言狀。州故無城,督民樹柵實以士。無何,邦彥果反,來攻城,鼎誓死固守。或言賊勢盛,宜走定番。鼎曰:「吾守土吏也,義當與城存亡。」及賊入,與士官金粲端坐堂上,並為賊所殺,婢僕從死者六人。吏目胡士統被執,亦不屈死。巡撫李枟上於朝,贈僉事,賜祭。崇禎元年,以成子舉人興南,鼎子舉人崑禎皆援朝綱例,請加恤,並贈光祿卿,世廕錦衣千戶,予祭葬,有司建祠立坊,以恭亦附祀。崑禎後舉進士,歷御史,尚寶卿。

時有孫克恕者,字推之,馬平人。舉於鄉,歷官貴州副使,分巡思石道。禦賊戰死,有虎守其骸不去,蠻人嗟異。事聞,贈太僕卿,賜祭葬。

姬文胤,字士昌,華州人。舉於鄉。天啟二年授滕縣知縣。視事甫三日,白蓮賊徐鴻儒薄城,民什九從亂。文胤徒步叫號,驅吏卒登陴,不滿三百,望賊輒走,存者纔數十。問何故從賊,曰:「禍由董二。」董二者,故延綏巡撫董國光子也,居鄉貪暴,民不聊生,故從賊。文胤憑城諭曰:「良民以董二故,挺而從賊。吾將執二置諸法,為若雪憤,可乎?」文胤身長赤面,鬚髯戟張,賊望見,駭為神人,皆讙呼羅拜。俄而發箭西隅,斃二賊。視之,延綏沙柳竿也。賊謂文胤紿之,大憤,肉薄登城,眾悉潰。文胤緋衣坐堂皇,嚼齒罵賊。賊前,搏裂冠裳,械繫之,罵不屈。三日潛解印,畀小吏魏顯照及家僮李守務,北向拜闕,遂自經。賊搒掠顯照索印,顯照潛授其父,而與守務罵賊,並死之。事聞,贈太僕少卿,立祠致祀,錄一子,優恤顯照、守務家。董二踰城遁去。

時賊陷鄒縣,博士孟承光被執,詬詈不屈死。贈尚寶少卿,世蔭錦衣千戶。承光,字永觀,亞聖裔,世廕《五經》博士也。

朱萬年,黎平人。萬曆中,舉於鄉。歷萊州知府,有惠政。崇禎五年,叛將李九成等陷登州,率眾來犯。萬年率吏民固守。時山東巡撫徐從治、登萊巡撫謝璉並在城中,被圍,堅守數月,從治中礮死。賊詭乞降,璉率萬年往受,為所執。萬年曰:「爾執我無益,盍以精騎從我,呼守者出降。」賊以精騎五百擁萬年至城下,萬年大呼曰:「我被擒,誓必死。賊精銳盡在此,急發礮擊之,毋以我為念!」守將楊御蕃不忍,萬年復頓足大呼,賊怒殺之。城上人見萬年已死,遂發畐駮,賊死過半。事聞,贈太常卿,賜祭葬,有司建祠,官一子。

初,賊掠新城,知縣秦三輔、訓導王協中禦之,並死。其陷黃縣,知縣吳世揚罵賊死,縣丞張國輔、參將張奇功、守備熊奮渭皆力戰死。陷平度,知州陳所聞自縊死。三輔、世揚贈光祿少卿,所聞贈太僕少卿,並賜祭葬,建祠,廕子。協中、國輔、奇功亦贈恤有差。三輔,三原人。世揚,洛陽人。所聞,畿輔人。並起家乙榜。

張瑤,蓬萊人。天啟五年進士。授開封府推官,絕請寄,抑豪強,吏民畏如神。崇禎四年行取入都,吏科宋鳴梧力援宋玫為給事,而抑瑤,授府同知。瑤怒,疏摭玫行賄狀。吏部尚書閔洪學劾瑤饋遺奔競,鳴梧復極論之,謫河州判官,未赴。明年正月,李九成等逼登州,瑤率家眾登陴拒守。城陷,瑤猶揮石奮擊。賊擁執之,大罵不屈,被殺。妻女四人並投井死。贈光祿少卿。

先是,賊陷新城,舉人王與羲、張儼然死之。其陷他縣者,貢生張聯臺、蔣時行亦死之。皆格於例,不獲旌。禮部侍郎陳子壯上言:「舉貢死難,無恤典,舊制也。然名既登於天府,恩獨後於流官,九泉之下,能無怨恫。比者,武舉李調禦賊捐軀,已蒙贈恤。武途如此,文儒安得獨遺。乞量贈一官,永為定制。」可之。乃贈與夔、儼然宛平知縣,聯台、時行順天府教授。其後地方死難,若舉人李讓、吳之秀、賈煜、張慶雲,貢生張茂貞、張茂恂,皆贈官如前制。

何天衢,字升宇,阿迷州人。有勇略,土酋普名聲招為頭目,使駐三鄉。崇禎三年,名聲反,謀出三路兵,至昆明會戰。令天衢自維摩羅平入,以礮手三百人助之。天衢慨然曰:「此大丈夫報國秋也,吾豈為逆賊用哉!」坑殺礮手數十人,率眾歸附,署維摩州同知李嗣泌開城納之。名聲已陷彌勒,聞大懼,急撤兩路兵歸。巡撫王伉上其事,授為守備。後數與嗣泌進剿有功。及名聲死,妻萬氏代領其眾,屢攻天衢。天衢屢挫之,錄功,進參將。十三年擢副總兵。萬氏贅沙定洲為婿,益以南安兵,且厚賂黔國公用事者,令毀天衢。天衢請兵餉皆不應,賊悉力攻之,食盡,舉家自焚死。

初,名聲之亂,有楊於陛者,劍州人。舉於鄉。歷官武定府同知。巡撫伉令監紀軍事,兵敗被執,死之。贈太僕少卿,建祠曰精忠。


\end{pinyinscope}