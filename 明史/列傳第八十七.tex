\article{列傳第八十七}

\begin{pinyinscope}
李鉞子惠王憲胡世寧子純繼李承勛王以旂范掞王邦瑞子正國鄭曉

李鉞,字虔甫,祥符人。弘治九年進士。除御史。巡視中城,理河東鹽政,歷有聲績。正德改元,天鳴星變。偕同官陳數事,論中官李興、寧謹、苗逵、高鳳等罪,而請斥尚書李孟抃、都督神英。武宗不能用。以喪歸。劉瑾惡鉞劾其黨,假他事罰米五百石輸邊。瑾敗,起故官,出為鞏昌知府,尋遷四川副使。巡撫林俊委鉞與副使何珊討敗流賊方四等,賜金加俸。遷陜西按察使,擢右僉都御史巡撫山西。寇入白羊口。鉞度宣、大有備,必窺岢嵐、五臺間,乃亟畫戰守。寇果犯岢嵐,鉞與延綏援將安國、杭雄敗之。加俸一級。尋討平內寇武廷章等。召入理院事。

世宗即位,歷兵部左、右侍郎,出總制陜西三邊軍務。鉞長軍旅,料敵多中。初至固原,寇入犯,援兵未集。鉞下令大開諸營門,晝夜不閉。寇疑有備,未敢逼。乃炮擊之,寇引去。以其間增築墩堡,謹烽堠,廣儲蓄,選壯勇為備。未幾,寇復深入平涼、邠州。鉞令遊擊時陳、周尚文等,分伏要害遏其歸,斬獲多。鉞策寇失利必東犯延綏,檄諸將設伏待。寇果至,又敗去。已而言官論邠州失事罪,請罷總兵官劉淮、巡撫王珝等,並及鉞。詔奪淮職,責鉞圖後效。鉞自劾乞休,不許。盜楊錦等剽延綏,殺指揮翟相,鉞討擒之。嘉靖二年,以塞上無警召還。給事中劉世揚請留鉞陜西,而久任諸邊巡撫。帝卒召鉞,進右都御史,總督漕運,巡撫鳳陽諸府,入掌都察院事。

四年,代金獻民為兵部尚書兼督團營。中官刁永等多所陳乞,帝皆許之。又錄司禮扶安家八人官錦衣。南京守備已三人,復命卜春添注以往。御馬監閻洪因軍政,請自考騰驤四衛及牧馬所官。鉞累疏力爭,帝皆不納,至責以抗旨,令對狀。鉞引罪乃罷。武定侯郭勛以會武宴列尚書下,疏爭之。鉞言:「中府官之有會武宴,猶禮部之有恩榮宴也。恩榮,禮部為主,會武,中府為主,故皆列諸尚書之次。宴圖可徵,不得引團營故事。」帝竟從勛言。錦衣革職百戶李全奏乞復任,鉞請治其違旨罪,帝不問。於是官旗鄭彪等皆援全例以請,鉞執奏如初,而疏有「猿攀狐媚」語。帝惡之,復責對狀,奪俸一月。

鉞既屢諫不用,失上意,且知為近倖所嫉。會病,遂再疏乞休,許馳驛,未行,卒。贈太子少保,遣官護喪歸葬。久之,賜謚恭簡。

子惠,正德十二年進士,官行人。諫武宗南巡,死於廷杖。贈監察御史。

王憲,字維綱,東平人。弘治三年進士。歷知阜平、滑二縣。召拜御史。正德初,擢大理寺丞。遷右僉都御史。清理甘肅屯田。進右副都御史,巡撫遼東。歷鄖陽、大同。以應州禦寇功,廕錦衣,世百戶。遷戶部右侍郎,改撫陜西,入為兵部右侍郎。近畿盜起,偕太監張忠、都督朱泰捕之,復以功蔭錦衣。武宗南征,命率戶、兵、工三部郎各一人督理軍儲。駕旋,以中旨代王瓊為兵部尚書。世宗即位,為給事中史道劾罷。

嘉靖四年,廷推鄧璋及憲為三邊總制,言官持不可,帝竟用憲。部將王宰、史經連敗寇,璽書褒諭。吉囊數萬騎渡河從石臼墩深入,憲督總兵官鄭卿、杭雄、趙瑛等分據要害擊之,都指揮卜雲斷其歸路。寇至青羊嶺,大敗去。五日四捷,斬首三百餘級,獲馬駝器仗無算。帝大喜,加憲太子太保,復予一子廕。至是,凡三廕錦衣世百戶矣。中官織花絨於陜,憲請罷之。又因九廟成,請釋還議禮得罪者,頗為士大夫所稱。張璁、桂萼欲用王瓊為總制,乃改憲南京兵部尚書。已,入為左都御史。朔州告急,廷推憲總督宣、大。憲不肯行,曰:「我甫入中臺,何見驅亟也。」給事中夏言、趙廷瑞劾憲託疾避難,復罷歸。

未幾,帝追念憲,召為兵部尚書。小王子入寇,條上平戎及諸邊防禦事宜。又請立京營分伍操練法,諸將不得藉內府供事,規避營操。帝皆嘉納。舊制,軍功論敘,有生擒、斬首、當先、殿後、奇功、頭功諸等,其後濫冒日多。憲定軍功襲替格,自永樂至正德,酌其輕重大小之差,臚析以上。詔著之《會典》為成式。尋兼督團營。西番諸國來貢,稱王號者百餘人。憲與禮臣夏言等請如成化、弘治間例,答敕止國王一人,仍限貢期、人數。議乃定。

大同兵變,憲初言首亂當誅,餘宜散遣。而大學士張孚敬與總督劉源清力主用兵,憲乃不敢堅前議。源清攻城不能下,北寇又內侵,請別遣大臣禦北寇,己得專攻城。憲亦議從其奏,論者多尤憲。會帝悟大同重鎮,不宜破壞,乃寢其事,亂亦旋定。源清竟得罪去。居數年,憲引年歸,卒。贈少保,謚康毅。子汝孝,副都御史。見《丁汝夔傳》。

胡世寧,字永清,仁和人。弘治六年進士。性剛直,不畏彊禦,且知兵。除德安推官。岐王初就籓,從官驕,世寧裁之。他日復請湖田,持不可。遷南京刑部主事。應詔陳邊備十策,復上書極言時政闕失。時孝宗已不豫,猶頷之。再遷郎中。與李承勛、魏校、餘祐善,時稱「南都四君子」。

遷廣西太平知府。太平知州李濬數殺掠吏民,世寧密檄龍英知州趙元瑤擒之。思明叛族黃文昌四世殺知府,占三州二十七村。副總兵康泰偕世寧入思明,執其兄弟三人。而泰畏文昌夜遁,委世寧空城中,危甚。諸土酋德世寧,發兵援,乃得還。文昌懼,歸所侵地,降。土官承襲,長吏率要賄不時奏,以故諸酋怨叛。世寧令:「生子即聞府。應世及者,年十歲以上,朔望謁府。父兄有故,按籍請官於朝。」土官大悅。

母喪歸。服闋赴京。道滄州,流寇攻城急。世寧即馳入城,畫防守計。賊攻七日夜,不能拔,引去。再知寶慶府。岷王及鎮守中官王潤皆嚴憚之。遷江西副使。與都御史俞諫畫策擒盜,討平王浩八。以暇城廣昌、南豐、新城。當是時,寧王宸濠驕橫有異志,莫敢言,世寧憤甚。正德九年三月上疏曰:「江西之盜,剿撫二說相持,臣愚以為無難決也。已撫者不誅,再叛者毋赦,初起者亟剿,如是而已。顧江西患非盜賊。寧府威日張,不逞之徒群聚而導以非法,上下諸司承奉太過。數假火災奪民廛地,採辦擾旁郡,蹂籍遍窮鄉。臣恐良民不安,皆起為盜。臣下畏禍,多懷二心,禮樂刑政漸不自朝廷出矣。請於都御史俞諫、任漢中專委一人,或別選公忠大臣鎮撫。敕王止治其國,毋撓有司,以靖亂源,銷意外變。」章下兵部。尚書陸完議,令諫往計賊情撫剿之宜,至所言違制擾民,疑出偽託,宜令王約束之。得旨報可。宸濠聞,大怒。列世寧罪,遍賂權幸,必殺世寧。章下都察院。右都御史李士實,宸濠黨也,與左都御史石玠等上言,世寧狂率當治。命未下,宸濠奏復至,指世寧為妖言。乃命錦衣官校逮捕世寧。世寧已遷福建按察使,取道還里。宸濠遂誣世寧逃,馳使令浙江巡按潘鵬執送江西。鵬盡繫世寧家人,索之急。李承勛為按察使,保護之。世寧乃亡命抵京師,自投錦衣獄。獄中三上書言宸濠逆狀,卒不省。繫歲餘,言官程啟充、徐文華、蕭鳴鳳、邢寰等交章救,楊一清復以危言動錢寧,乃謫戍沈陽。

居四年,宸濠果反。世寧起戍中為湖廣按察使。尋擢右僉都御史,巡撫四川。道聞世宗即位,疏以司馬光仁、明、武三言進,因薦魏校、何瑭、邵銳可講官;林俊、楊一清、劉忠、林廷玉可輔弼;知府劉蒞、徐鈺先為諫官有直聲,宜擢用。時韙其言。松潘所部熟番,將吏久不能制,率輸貨以假道。番殺官軍,憚不敢詰。官軍殺番,輒抵罪。世寧陳方略,請選將益兵,立賞罰格,嚴隱匿禁,修烽堠,時巡徼,以振軍威,通道路。詔悉行之。又劾罷副總兵張傑、中官趙欽。甫兩月,召為吏部右侍郎。未上,以父憂歸。

既免喪家居,朝廷方議「大禮」,異議者多得罪。世寧意是張璁等,疏乞早定追崇「大禮」。未上,語聞京師。既有議遷顯陵祔天壽山者,世寧極言不可,乃並前疏上之。帝深嘉歎。無何,聞廷臣伏闕爭,有杖死者,馳疏言:「臣向以仁、明、武三言進,然尤以仁為本。仁,生成之德;明,日月之臨,皆不可一日無。武則雷霆之威,但可一震而已。今廷臣忤旨,陛下赫然示威,辱以箠楚,體羸弱者輒斃。傳之天下,書之史冊,謂鞭撲行殿陛,刑辱及士夫,非所以光聖德。新進一言偶合,後難保必當;舊德老成,一事偶忤,後未必皆非。望陛下以三無私之心,照臨於上,無先存適莫於中。」帝雖不能從,亦不忤。尋召為兵部左侍郎。條戍邊時所見險塞利害二十五事以上。又請善保聖躬,毋輕餌藥物。獻《大學·秦誓》章、《洪範》「惟辟威福」、《繫辭·節》「初爻」講義,並乞留中。給事中餘經遂劾世寧啟告密之漸。世寧乞罷,不許。「大禮」成,進秩一等。復陳用人二十事。工匠趙奎等五十四人以中官請,悉授職。世寧言賞過濫,不納。屢疏引疾。改南京吏部,就遷工部尚書。已,復召為左都御史,加太子少保。辭宮銜,許之。

世寧故方嚴。及掌憲,務持大體。條上憲綱十餘條,末言:「近士習忌刻,一遭讒毀,則終身廢棄。僉事彭祺發豪強罪,受謗奪官。諸如此者,宜許大臣申理。」帝採其言,惟祺報寢。執政請禁私謁,世寧曰:「臣官以察為名。人非接其貌,聽其言,無由悉其才行。」帝以為然,遂弗禁。俄改刑部尚書。每重獄,別白為帝言之,帝輒感悟。中官剛聰誣漕卒掠御服,坐二千人,世寧劾其妄。已,聰情得抵罪,帝乃益信世寧。王瓊修郤陳九疇,將致之死。以世寧救,得戍。

兵部尚書王時中罷,以世寧代,加太子太保。再辭不得命,乃陳兵政十事,曰定武略、崇憲職、重將權、增武備、更賞罰、馭土夷、足邊備、絕弊源、正謬誤、惜人才。所言多破常格,帝優旨答之。土魯番貢使乞歸哈密城,易降人牙木蘭。王瓊上其事。世寧言:「先朝不惜棄大寧、交阯,何有於哈密。況初封忠順為我外籓,而自罕慎以來三為土魯番所執,遂狎與戎比,以疲我中國,耗財老師,戎得挾以邀索。臣以為此與國初所封元孽和順、寧順、安定三王等耳。安定在哈密內,近甘肅,今存亡不可知。我一切不問,獨重哈密何也?宜專守河西,謝絕哈密。牙木蘭本曲先衛人,反正歸順,非納降比,彼安得索之?唐悉怛謀事可鑒也。」張璁等皆主瓊議,格不用,獨留牙木蘭不遣。居兵部三月求去,帝不許,免朝參。世寧又上備邊三事。固稱疾篤,乃聽乘傳歸,給廩隸如制。歸數月,復起南京兵部尚書,固辭不拜。九年秋卒。贈少保,謚端敏。

世寧風格峻整,居官廉。疾惡若仇,而薦達賢士如不及。都御史馬昊、陳九疇坐累廢;副使施儒、楊必進考察被黜;御史李潤、副使范輅為時所抑,連章薦之。與人語,吶不出口。及具疏,援據古今,洞中窾會。與李承勛善,而持議不茍合。承勛欲授隴勝官,復芒部故地,世寧言勝非隴氏子,芒氏不當復立。始以議禮與張璁、桂萼合,璁、萼德之,欲援以自助。世寧不肯附會,論事多牴牾。萼議欲銷兵,世寧力折之。昌化伯以他姓子冒封,下廷議。世寧言:「吾輩不得以厚賂故,誣朝廷」,萼為色變。萼方為吏部,而世寧引疾,言:「天變人窮,盜賊滋起,咎在吏、戶、兵三部不得人。兵部尤重,請避賢路。」又以哈密議,語侵璁,諸大臣皆忌之。帝始終優禮不替。

子純、繼。純以父任知肇慶府,有才行。繼幼不慧,不為世寧知。世寧在江西出討賊,部將入見繼。繼為指陣法,進退離合甚詳,凡三日。世寧歸閱,大異之。知其故,嘆曰:「吾有子不自識,何也?」自是擊賊,輒令繼從,與策方略。世寧十不失三,繼十不失一。世寧方草疏論宸濠,繼請曰:「是且重得禍。」世寧曰:「吾已許國,遑恤其他。」及世寧下獄,繼念其父,病死。

李承勛,字立卿,嘉魚人。父田,進士,官右副都御史,巡撫順天。有操執,為政不苛。承勛舉弘治六年進士。由太湖知縣遷南京刑部主事。歷工部郎中,遷南昌知府。

正德六年,贛州賊犯新淦,執參政趙士賢。靖安賊據越王嶺瑪瑙岸,華林賊又陷瑞州。諸道兵不敢前。承勛督民兵剿,數有功。華林賊殺副使周憲,憲軍大潰。承勛單騎入憲營,眾乃復集。都御史陳金即檄承勛討之。賊黨王奇聽撫,搜得其衷刃,縱使還。奇感泣,誓以死報。承勛令奇密入寨,說降其黨為內應,而親率所部登山。奇夜拔柵,官軍奮而前,降者自內出,賊遂潰。已,從金斬賊渠羅光權、胡雪二,華林賊平。鎮守中貴黎安誣承勛擅易賊首王浩八獄詞,坐下吏。大理卿燕忠即訊,得白。舉治行卓異,超遷浙工按察使。歷陜西、河南左、右布政使。以右副都御史巡撫遼東。邊備久弛,開原尤甚。士馬纔十二,牆堡墩臺圮殆盡。將士依城塹自守,城外數百里悉為諸部射獵地,承勛疏請修築。會世宗立,發帑銀四十餘萬兩。承勛命步將四人各一軍守要害,身負畚鍤先士卒。凡為城塹各九萬一千四百餘丈,墩堡百八十有一。招逋逃三千二百人,開屯田千五百頃。又城中固、鐵嶺;斷陰山、遼河之交;城蒲河、撫順,扼要衝。邊防甚固。錄功,進秩一等。又數陳軍民利病,咸報可。以疾歸。起故官,蒞南院。三遷刑部尚書,加太子少保。

帝以京營多弊,欲振飭之。遂加承勛太子太保,改兵部尚書兼左都御史,專督團營。尋兼掌都察院。以疾,三疏乞休,且言:「山西潞城賊以四道兵討之,不統於一人,故無功。川、貴芒部之役措置乖方,再勝再叛,宜命伍文定深計,毋專用兵。豐、沛河工,二年三易大臣,工不就,宜令知水利者各陳所見,而俾侍郎潘希曾度可否。其尤要者,在決壅蔽患。仿唐、宋轉對、次對故事,不時召見大臣。」帝不允辭,下其議於所司。時秦、晉、楚、蜀歲祲,詔免田賦。承勛言:「有司例十月始征賦。今九月矣,恐官吏督趣,陰圖乾沒。宜及其未征,遣官馳告以所蠲數。山陬僻壤,俾悉戶曉。有司不能奉宣德意者,罪之。撫按失舉奏,并坐。」帝褒納之。奏奪京營把總湯清職。郭勛為求復,語侵承勛。承勛因求退,給事中王準等劾勛恣。乃敕責勛,而下清法司。兵部尚書胡世寧致仕,詔承勛還部代之。疏言:「朝廷有大政及推舉文武大臣,必下廷議。議者率相顧不發,拱手聽。宜及未議前,備條所議,布告與議者,俾先諗其故,然後平心商質,各盡所懷。議茍不合,聽其別奏。庶足盡諸臣之見,而所議者公。」帝然其言,下詔申飭。尋命兼督團營。言官攻張璁、桂萼黨,并及承勛。承勛連章求退,帝復溫旨答之。中官出鎮者,率暴橫。承勛因諫官李鳳毛等言,先後裁二十七人,又革錦衣官五百人,監局冒役數千人。獨御馬監未汰,復因給事中田秋奏,多所裁減。而請以騰驤四衛屬部,核詭冒,制可。中官言曩彰義門破也先,東市剿曹賊,皆四衛功,以直內故易集,隸兵部不便。承勛言:「彰義門之戰,禍由王振。東市作賊,即曹吉祥也。」帝卒從承勛議,歸兵部。寇犯大同,議遣大臣督兵。眾推都御史王憲,憲不肯行。給事中夏言謂承勛曰:「事急,公當請行。」承勛竟不請。給事中趙廷瑞並劾之。會寇退,罷。

十年春,大風晝晦,帝憂邊事。承勛言:「去歲冰合,敵騎盡入河套。延、寧、固原皆當警備。甘肅軍餉專仰河東,宜於蘭州糴貯,以備緩急。曩河西患土魯番,今亦卜喇又深入。兩寇雲擾,孤危益甚。套寇出入,並經莊浪。急宜繕塞設險,斷臂截踵,使不得相合。兀良哈最近京師,不善撫,即門庭寇。雲南安鳳之叛,軍民困敝,臨安、蒙自盜賊復興,曠日淹時,恐釀大患。交阯世子流寓老撾,異日歸命請援,或據地求封,皆未可測。惟急用人理財,俾邊鄙無虞。」帝嘉納焉。

承勛沉毅有大略。帝所信任,自輔臣外,獨承勛與胡世寧,大事輒咨訪。二人亦孜孜奉國,知無不言。世寧卒半歲,承勛亦卒,帝深嗟悼。贈少保,謚康惠。所賚予,常典外,特賜白金、綵幣、米蔬諸物。承勛官四十年,家無餘資。其議「大禮」,亦與世寧相合云。

王以旂,字士招,江寧人。正德六年進士。除上高知縣。華林賊方熾,以旂訓鄉兵禦之,賊不敢犯。征授御史,出按河南。宸濠反,鎮守太監劉璟倡議停鄉試。以旂言河南去江西遠,罷試無名。乃止。璟又言,帝親征,道且出汴,牒取供頓銀四萬兩。巡撫議予之,以旂執不予。世宗即位,欲加興獻帝皇號,以旂抗言不可。已,上弭災要務,言:「司禮取中旨免張漢贓科,臣不預聞,此啟矯偽之漸也。」帝不聽。累遷兵部右侍郎。徐、呂二洪竭,漕舟膠。命兼右僉都御史總理河漕。踰年,渠水通,進秩一等。尋拜南京右都御史。召為工部尚書,改左都御史,代陳經為兵部尚書兼督團營。

三邊總督曾銑議復河套,大學士夏言主之。數下優旨獎銑,令以旂集廷臣議。以旂等力主銑議。議上,帝意忽變,嚴旨咎銑,令再議。以旂等惶恐,盡反前說。帝逮銑,令以旂代之。套寇自西海還,肆掠永昌,鎮羌總兵官王繼祖禦卻之。已,復來犯,并及鎮番、山丹。部將蔡勛、馬宗援三戰皆捷。前後斬首一百四十餘級。論功,廕以旂一子。已而寇數萬復屯寧夏塞外,將大入。官軍擊之,斬首六十餘級,寇宵遁。延綏、寧夏開馬市,二鎮市五千匹。其長狠台吉等約束所部,終市無嘩。以旂以聞。詔大賚二鎮文武將吏,以旂復賜金幣。錄延綏將士破敵功,再廕一子。在鎮六年,修延綏城堡四千五百餘所,又築蘭州邊垣,加官至太子太保。比卒,軍民為罷市。贈少保,謚襄敏,再予一子官。

范金,字平甫,其先江西樂平人,遷沈陽。金登正德十二年進士,授工部主事,遷員外郎。嘉靖三年,伏闕爭「大禮」,下獄廷杖。由戶部郎中改長蘆鹽運司同知,遷河南知府。歲大饑,巡撫都御史潘塤駁諸請振文牒,候勘實乃發。金不待報,輒開倉振之,全活十餘萬。民爭謳頌金,語聞禁中。帝為責戶部及塤與巡按御史匿災狀。塤歸罪金以自解,被劾罷去,金名由此顯。遷兩淮鹽運使,條上鹺政十要。歷四川參政,湖廣按察使,浙江、河南左、右布政使。

二十年,擢右副都御史,巡撫寧夏。金為人持重,有方略。既蒞重鎮,不上首功。一意練步騎,廣儲蓄,繕治關隘亭障,寇為遠徙,俘歸者五百人。上疏言:「邊將各有常祿,無給田之制。自武定侯郭勛奏以軍餘開墾田園給將領,委奸軍為莊頭,害殊大。宜給還軍民,任耕種便。」帝從其請。居數年,引疾歸。

起故官,撫河南。尋召為兵部右侍郎,轉左。尚書王以旂出督三邊,金署部事。頃之,奉詔總理邊關阨隘。奏上經略潮河川、居庸關諸處事宜,請於古道門外蜂窩嶺增墩臺一為外屏,浚濠設橋,以防衝突。川西南兩山對處,各設敵臺,以控中流,分戍兵番直守要害。又薊鎮五里垛、劃車、開連口、慕田谷等地,宜設墩臺。惡谷、紅土谷、香爐石等地,宜斬崖塹。居庸關外諸口,在宣府為內地,在居庸則為邊籓,宜敕東中路文武臣修築。加潮河川提督為守備,增副將居庸關,領天壽山、黃花鎮。設橫嶺守備,塞懷來路,增置新軍二千餘人,資團練。又議紫荊、倒馬、龍泉等關及山海關、古北口經略事宜,請於紫荊之桑谷,倒馬之中AG關峪,龍泉之陡石嶺諸要害,創築城垣,增設敵樓營舍。薊州所轄燕河、太平、馬蘭、密雲四路,修築未竟者,括諸司贖鍰竣之。而浮圖峪、插箭嶺尤為紫荊、倒馬二關衝,移參將分駐石門杜家莊,俾保定總兵駐紫荊。薊、遼懸絕千里,移建昌營遊擊於山海關。三屯等營缺軍,應速募,馬不足者補入。其常戍之兵介胄不備,量給鎧仗,番上者悉予行糧,毋俾荷戈枵腹。又言:「諸路緩急,以密雲之分守為最。各關要害,以密雲之迤西為最。若燕河之冷口,馬蘭之黃崖,太平之榆木嶺、擦崖子,皆所急也。宜敕撫鎮督諸將領分各營士馬,兼側近按伏之兵,迭為戰守。」兵部言:「軍戍久,戀土。猝移置,恐他變。莫若山海關增置能將一員,募軍三千屯駐,聽薊、遼撫臣調度,援燕河。」餘如金言,下守臣議。

帝才金甚。會兵部尚書趙廷瑞罷,即命金入代。金以老辭,且言隨事通變,乏將順之宜。帝怒,責金不恭,削其籍。時嚴嵩當國,而金本由徐階薦,天下推為長者,惜其去,不以罪。然金罷,帝召翁萬達,甫至以憂去,丁汝夔代之。明年,俺答逼都城,汝夔遂誅死。而金歸久之乃卒。隆慶元年復官。

王邦瑞,字惟賢,宜陽人。早有器識。為諸生,山東盜起,上剿寇十四策於知府。正德十二年成進士。改庶吉士。與王府有連,出為廣德知州。嘉靖初,祖憂,去。補滁州。屢遷南京吏部郎中,出為陜西提學僉事。坐歲貢不中式五名以上,貶濱州知州。再遷固原兵備副使。涇、邠巨盜李孟春,流劫河東、西,剿平之。以祖母憂去。服除,復提學陜西,轉參政。母憂解職。起擢右僉都御史,巡撫寧夏。寇乘冰入犯,設伏敗之。改南京大理卿。未上,召為兵部右侍郎。

改吏部,進左。俺答犯都城,命邦瑞總督九門。邦瑞屯禁軍郭外,以巡捕軍營東、西長安街,大啟郭門,納四郊避寇者。兵部尚書丁汝夔下獄,命邦瑞攝其事,兼督團營。寇退,請治諸將功罪,且濬九門濠塹,皆報可。邦瑞見營制久弛,極陳其弊。遂罷十二團營,悉歸三大營,以咸寧侯仇鸞統之。邦瑞亦改兵部左侍郎,專督營務。復條上興革六事。中言宦官典兵,古今大患,請盡撤提督監槍者。帝報從之。又舉前編修趙時春、工部主事申知兵,並改兵部,分理京營事。未幾,帝召兵部尚書翁萬達未至,遲之,遂命邦瑞代。條上安攘十二事。

仇鸞構邦瑞於帝,帝眷漸移。會鸞奏革薊州總兵官李鳳鳴、大同總兵官徐玨任,而薦京營副將成勳代鳳鳴,密雲副將徐仁代玨。旨從中下。邦瑞言:「朝廷易置將帥,必采之公卿,斷自宸衷,所以慎防杜漸,示臣下不敢專也。且京營大將與列鎮將不相統攝,何緣京營,乃黜陟各鎮。今曲徇鸞請,臣恐九邊將帥悉奔走托附,非國之福也。」帝不悅,下旨譙讓。鸞又欲節制邊將,罷築薊鎮邊垣。邦瑞皆以為不可。鸞大憾,益肆讒構。會邦瑞復陳安攘大計,遂嚴旨落職,以冠帶辦事。居數日,大計自陳。竟除名,以趙錦代。邦瑞去,鸞益橫。明年誅死,錦亦坐黨比遣戍,於是帝漸思之。踰十年,京營缺人,帝曰:「非邦瑞不可。」乃起故官。

既至,疏便宜數事,悉允行。踰年卒。贈太子少保,謚襄毅,遣行人護喪歸葬。

邦瑞嚴毅有識量。歷官四十年,以廉節著。子正國,南京刑部侍郎。

鄭曉,字窒甫,海鹽人。嘉靖元年舉鄉試第一。明年成進士,授職方主事。日披故牘,盡知天下阨塞、士馬虛實、強弱之數。尚書金獻民屬撰《九邊圖志》,人爭傳寫之。以爭「大禮」廷杖。大同兵變,上疏極言不可赦。張孚敬柄政,器之,欲改置翰林及言路,曉皆不應。父憂,歸,久之不起。

許言贊為吏部尚書,調之吏部。歷考功郎中。夏言罷相,帝惡言官不糾劾,詔考察去留。大學士嚴嵩因欲去所不悅者,而曉去喬佑等十三人,多嵩所厚。嵩大憾曉,調文選。嵩欲用趙文華為考功,曉言於言贊曰:「昔黃禎為文選,調李開先考功,皆山東人,詔不許。今調文華,曉避位而已。」言贊以謝嵩。嵩欲以子世蕃為尚寶丞,曉曰:「治中遷知府,例也。遷尚寶丞,無故事。」嵩益怒。以推謫降官周鈇等,貶曉和州同知。稍遷太僕丞,歷南京太常卿。召拜刑部右侍郎。

俄改兵部,兼副都御史總督漕運。大江南北皆中倭,漕艘幾阻。曉請發帑金數十萬,造戰舸,築城堡,練兵將,積芻糗。詔從之。中國奸民利倭賄,多與通。通州人顧表者尤桀黠,為倭導。以故營寨皆據要害,盡知官兵虛實。曉懸重賞捕戮之。募鹽徒驍悍者為兵,增設泰州海防副使,築瓜洲城,廟灣、麻洋、雲梯諸海口皆增兵設堠。遂破倭於通州,連敗之如皋、海門,襲其軍呂泗,圍之狼山,前後斬首九百餘。賊潰去。錄功,再增秩,三賚銀幣。時賊多中國人。曉言:「武健才住之徒。困無所逞,甘心作賊。非國家廣行網羅,使有出身之階,恐有如孫恩、盧循輩出乎其間,禍滋大矣。洪武時倭寇近海州縣。以高皇帝威靈,兼謀臣宿將,築城練兵,經略數年,猶未乂安。乃招漁丁、島人、鹽徒、蜑戶籍為水軍至數萬人,又遣使出海宣布威德。久之,倭始不為患。今江北雖平,而風帆出沒,倏忽千里。倭恃華人為耳目,華人借倭為爪牙,非詳為區畫,後患未易弭也。」帝頗採納之。

尋召為吏部左侍郎,遷南京吏部尚書。帝以曉知兵,改右都御史協理戎政。尋拜刑部尚書。俺答圍大同右衛急,帝命兵部尚書楊博往督大師,乃以曉攝兵部。曉言:「今兵事方棘,而所簡聽徵京軍三萬五千人,乃令執役赴工,何以備戰守?乞歸之營伍。」帝立從之。

尋還視刑部事。嚴嵩勢益熾。曉素不善嵩。而其時大獄如總督王忬以失律,中允郭希顏以言事,曉並予輕比,嵩則置重典。南都叛卒周山等殺侍郎黃懋官,海寇汪直通倭為亂,曉置重典,嵩故寬假之。惟巡撫阮鶚、總督楊順、御史路楷,以嵩曲庇,曉不能盡法,議者譏其失出云。故事,在京軍民訟,俱投牒通政司送法司問斷。諸司有應鞫者,亦參送法司,無自決遣者。後諸司不復遵守,獄訟紛拿。曉奏循故事,帝報許,於是刑部間捕囚畿府。而巡按御史鄭存仁謂訟當自下而上,檄州縣,法司有追取,毋輒發。曉聞,率侍郎趙大祐、傅頤守故事爭,存仁亦據律執奏。章俱下都察院會刑科平議。議未上,曉疏辨。嵩激帝怒切讓,遂落曉職,兩侍郎亦貶二秩。

曉通經術,習國家典故,時望蔚然。為權貴所扼,志不盡行。既歸,角巾布衣與鄉里父老遊處,見者不知其貴人也。既卒,子履淳等訟曉禦倭功於朝,詔復職。隆慶初,贈太子少保,謚端簡。履淳自有傳。

贊曰:李鉞諸人皆以威略幹濟顯於時。鉞與王憲、王以旂之治軍旅,李承勛、范金之畫邊計,才力均有過人者。胡世寧奮不顧身,首發奸逆,危言正色,始終一節。《易》稱「王臣蹇蹇」,世寧近之矣。王邦瑞抵抗權倖,躓而復起,鄭曉諳悉掌故,博洽多聞,兼資文武,所在著效,亦不愧名臣云。


\end{pinyinscope}