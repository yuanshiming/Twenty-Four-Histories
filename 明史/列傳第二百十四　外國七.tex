\article{列傳第二百十四 外國七}

\begin{pinyinscope}
古里柯枝小葛蘭大葛蘭錫蘭山榜葛剌沼納樸兒祖法兒木骨都束不剌哇竹步阿丹剌撒麻林忽魯謨斯溜山比剌孫剌南巫里加異勒甘巴里急蘭丹沙里灣泥底里千里達失剌比古里班卒剌泥夏剌比奇剌泥窟察泥捨剌齊彭加那八可意烏沙剌踢坎巴阿哇打回白葛達黑葛達拂菻意大里亞

古里,西洋大國。西濱大海,南距柯枝國,北距狼奴兒國,東七百里距坎巴國。自柯枝舟行三日可至,自錫蘭山十日可至,諸蕃要會也。

永樂元年命中官尹慶奉詔撫諭其國,賚以彩幣。其酋沙米的喜遣使從慶入朝,貢方物。三年達南京,封為國王,賜印誥及文綺諸物,遂比年入貢。鄭和亦數使其國。十三年偕柯枝、南渤利、甘巴里、滿剌加諸國入貢。十四年又偕爪哇、滿剌加、占城、錫蘭山、木骨都束、溜山、南渤利、不剌哇、阿丹、蘇門答剌、麻木、剌撒、忽魯謨斯、柯枝、南巫里、沙里灣泥、彭亨諸國入貢。是時,諸蕃使臣充斥於廷,以古里大國,序其使者於首。十七年偕滿剌加十七國來貢。十九年又偕忽魯謨斯等國入貢。二十一年復偕忽魯謨斯等國,遣使千二百人入貢。時帝方出塞,敕皇太子曰:「天時向寒,貢使即令禮官宴勞,給賜遣還。其以土物來市者,官酬其直。」宣德八年,其王比里麻遣使偕蘇門答剌等國使臣入貢。其使入留都下,正統元年乃命附爪哇貢舟西還。自是不復至。

其國,山多地瘠,有穀無麥。俗甚淳,行者讓道,道不拾遺。人分五等,如柯枝,其敬浮屠、鑿井灌佛亦如之。每旦,王及臣民取牛糞調水塗壁及地,又煆為灰抹額及股,謂為敬佛。國中半崇回教,建禮拜寺數十處。七日一禮,男女齋沐謝事。午時拜天於寺,未時乃散。王老不傳子而傳甥,無甥則傳弟,無弟則傳於國之有德者。國事皆決於二將領,以回回人為之。刑無鞭笞,輕者斷手足,重者罰金珠,尤重者夷族沒產。鞫獄不承,則置其手指沸湯中,三日不爛即免罪。免罪者,將領導以鼓樂,送還家,親戚致賀。

富家多植椰子樹至數千。其嫩者漿可飲,亦可釀酒,老者可作油、糖,亦可作飯。乾可構屋,葉可代瓦,殼可製杯,穰可索綯,煆為灰可鑲金。其他蔬果、畜產,多類中國。所貢物有寶石、珊瑚珠、琉璃瓶、琉璃枕、寶鐵刀、拂郎雙刃刀、金繫腰、阿思模達塗兒氣、龍涎香、蘇合油、花氈單、伯蘭布、苾布之屬。

柯枝,或言即古盤盤國。宋、梁、隋、唐皆入貢。自小葛蘭西北行,順風一日夜可至。

永樂元年,遣中官尹慶齎詔撫諭其國,賜以銷金帳幔、織金文綺、綵帛及華蓋。六年復命鄭和使其國。九年,王可亦里遣使入貢。十年,鄭和再使其國,連二歲入貢。其使者請賜印誥,封其國中之山。帝遣鄭和齎印賜其王,因撰碑文,命勒石山上。其詞曰:王化與天地流通,凡覆載之內、舉納於甄陶者,體造化之仁也。蓋天下無二理,生民無二心,憂戚喜樂之同情,安逸飽暖之同欲,奚有間於遐邇哉。任君民之寄者,當盡子民之道。《詩》云「邦畿千里,惟民所止,肇域彼四海」。《書》云「東漸于海,西被于流沙,朔南暨聲教,訖于四海。」朕君臨天下,撫治華夷,一視同仁,無間彼此。推古聖帝明王之道,以合乎天地之心。遠邦異域,咸使各得其所,聞風向化者,爭恐後也。

柯枝國遠在西南,距海之濱,出諸蕃國之外,慕中華而歆德化久矣。命令之至,拳跽鼓舞,順附如歸,咸仰天而拜曰:「何幸中國聖人之教,沾及於我!」乃數歲以來,國內豐穰,居有室廬,食飽魚鱉,衣足布帛,老者慈幼,少者敬長,熙熙然而樂,凌厲爭競之習無有也。山無猛獸,溪絕惡魚,海出奇珍,林產嘉木,諸物繁盛,倍越尋常。暴風不興,疾雨不作,札沴殄息,靡有害菑。蓋甚盛矣。朕揆德薄,何能如是,非其長民者之所致歟?乃封可亦里為國王,賜以印章,俾撫治其民。並封其國中之山為鎮國之山,勒碑其上,垂示無窮。而系以銘曰:「截彼高山,作鎮海邦,吐煙出雲,為下國洪龐。肅其煩高,時其雨暘,祛彼氛妖,作彼豐穰。靡菑靡沴,永庇斯疆,優游卒歲,室家胥慶。於戲!山之嶄兮,海之深矣,勒此銘詩,相為終始。」自後,間歲入貢。

宣德五年,復遣鄭和撫諭其國。八年,王可亦里遣使偕錫蘭山諸國來貢。正統元年,遣其使者附爪哇貢舶還國,並賜敕勞王。

王,瑣里人,崇釋教。佛座四旁皆水溝,復穿一井。每旦鳴鐘鼓,汲水灌佛,三浴之,始羅拜而退。

其國與錫蘭山對峙,中通古里,東界大山,三面距海。俗頗淳。築室,以椰子樹為材,取葉為苫以覆屋,風雨皆可蔽。

人分五等:一曰南昆,王族類;二曰回回,三曰哲地,皆富民;四曰革全,皆牙儈;五曰木瓜。木瓜最貧,為人執賤役者。屋高不得過三尺。衣上不得過臍,下不得過膝。途遇南昆、哲地人,輒伏地,俟其過乃起。

氣候常熱。一歲中,二三月時有少雨,國人皆治舍儲食物以俟。五六月間大雨不止,街市成河,七月始晴,八月後不復雨,歲歲皆然。田瘠少收,諸穀皆產,獨無麥。諸畜亦皆有,獨無鵝與驢云。

小葛蘭,其國與柯枝接境。自錫蘭山西北行六晝夜可達。東大山,西大海,南北地窄,西洋小國也。永樂五年遣使附古里、蘇門答剌入貢,賜其王錦綺、紗羅、鞍馬諸物,其使者亦有賜。

王及群下皆瑣里人,奉釋教。重牛及他婚喪諸禮,多與錫蘭同。俗淳。土薄,收獲少,仰給榜葛剌。鄭和嘗使其國。厥貢惟珍珠傘、白棉布、胡椒。

又有大葛蘭者,波濤湍悍,舟不可泊,故商人罕至。土黑墳,本宜穀麥,民懶事耕作,歲賴烏爹之米以足食。風俗、物產,多類小葛蘭。

錫蘭山,或云即古狼牙修。梁時曾通中國。自蘇門答剌順風十二晝夜可達。

永樂中,鄭和使西洋至其地,其王亞烈苦奈兒欲害和,和覺,去之他國。王又不睦鄰境,屢邀劫往來使臣,諸蕃皆苦之。及和歸,復經其地,乃誘和至國中,發兵五萬劫和,塞歸路。和乃率步卒二千,由間道乘虛攻拔其城,生擒亞烈苦奈兒及妻子、頭目,獻俘於朝。廷臣請行戮,帝憫其無知,并妻子皆釋,且給以衣食。命擇其族之賢者立之。有邪把乃那者,諸俘囚咸稱其賢,乃遣使齎印誥,封為王,其舊王亦遣歸。自是海外諸蕃益服天子威德,貢使載道,王遂屢入貢。

宣德五年,鄭和撫諭其國。八年,王不剌葛麻巴忽剌批遣使來貢。正統元年命附爪哇貢舶歸,賜敕諭之。十年偕滿剌加使者來貢。天順三年,王葛力生夏剌昔利把交剌惹遣使來貢。嗣後不復至。

其國,地廣人稠,貨物多聚,亞於爪哇。東南海中有山三四座,總名曰翠藍嶼。大小七門,門皆可通舟。中一山尤高大,番名梭篤蠻山。其人皆巢居穴處,赤身髡髮。相傳釋迦佛昔經此山,浴於水,或竊其袈裟,佛誓云:「後有穿衣者,必爛其皮肉。」自是,寸布掛身輒發瘡毒,故男女皆裸體。但紉木葉蔽其前後,或圍以布,故又名裸形國。地不生穀,惟啖魚蝦及山芋、波羅密、芭蕉實之屬。自此山西行七日,見鸚哥嘴山。又二三日抵佛堂山,即入錫蘭國境。海邊山石上有一足跡,長三尺許。故老云,佛從翠藍嶼來,踐此,故足跡尚存。中有淺水,四時不乾,人皆手蘸拭目洗面,曰「佛水清凈」。山下僧寺有釋迦真身,側臥床上。旁有佛牙及舍利,相傳佛涅槃處也。其寢座以沉香為之,飾以諸色寶石,莊嚴甚。王所居側有大山,高出雲漢。其顛有巨人足跡,入石深二尺,長八尺餘,云是盤古遺跡。此山產紅雅姑、青雅姑、黃雅姑、昔剌泥、窟沒藍等諸色寶石。每大雨,衝流山下,土人競拾之。海旁有浮沙,珠蚌聚其內,光彩瀲灩。王使人撈取,置之地,蚌爛而取其珠,故其國珠寶特富。

王,瑣里國人。崇釋教,重牛,日取牛糞燒灰塗其體,又調以水,遍塗地上,乃禮佛。手足直舒,腹貼於地以為敬,王及庶民皆如之。不食牛肉,止食其乳,死則瘞之,有殺牛者,罪至死。氣候常熱,米粟豐足,民富饒,然不喜啖飯。欲啖,則於暗處,不令人見。遍體皆毫毛,悉薙去,惟髮不薙。所貢物有珠、珊瑚、寶石、水晶、撒哈剌、西洋布、乳香、木香、樹香、檀香、沒藥、硫黃、藤竭、蘆薈、烏木、胡椒、碗石、馴象之屬。

榜葛剌,即漢身毒國,東漢曰天竺。其後中天竺貢於梁,南天竺貢於魏。唐亦分五天竺,又名五印度。宋仍名天竺。榜葛剌則東印度也。自蘇門答剌順風二十晝夜可至。

永樂六年,其王靄牙思丁遣使來朝,貢方物,宴賚有差。七年,其使凡再至,攜從者二百三十餘人。帝方招徠絕域,頒賜甚厚。自是比年入貢。十年,貢使將至,遣官宴之於鎮江。既將事,使者告其王之喪。遣官往祭,封嗣子賽勿丁為王。十二年,嗣王遣使奉表來謝,貢麒麟及名馬方物。禮官請表賀,帝勿許。明年遣侯顯齎詔使其國,王與妃、大臣皆有賜。正統三年貢麒麟,百官表賀。明年又入貢。自是不復至。

其國,地大物阜。城池街市,聚貨通商,繁華類中國。四時氣候常如夏。土沃,一歲二稔,不待耔耘。俗淳龐,有文字,男女勤於耕織。容體皆黑,間有白者。王及官民皆回回人,喪祭冠婚,悉用其禮。男子皆薙髮,裹以白布。衣從頸貫下,用布圍之。曆不置閏。刑有笞杖徒流數等。官司上下,亦有行移。醫卜、陰陽、百工、技藝悉如中國,蓋皆前世所流入也。

其王敬天朝。聞使者至,遣官具儀物,以千騎來迎。王宮高廣,柱皆黃銅包飾,雕琢花獸。左右設長廊,內列明甲馬隊千餘,外列巨人,明盔甲,執刀劍弓矢,威儀甚壯。丹墀左右,設孔雀翎傘蓋百餘,又置象隊百餘於殿前。王飾八寶冠,箕踞殿上高座,橫劍於膝。朝使入,令拄銀杖者二人來導,五步一呼,至中則止;又拄金杖者二人,導如初。其王拜迎詔,叩頭,手加額。開讀受賜訖,設絨毯於殿,宴朝使;不飲酒,以薔薇露和香蜜水飲之。贈使者金盔、金繫腰、金瓶、金盆,其副則悉用銀,從者皆有贈。厥貢:良馬、金銀琉璃器、青花白瓷、鶴頂、犀角、翠羽、鸚鵡、洗白苾布、兜鑼綿、撒哈剌、糖霜、乳香、熟香、烏香、麻藤香、烏爹泥、紫膠、藤竭、烏木、蘇木、胡椒、粗黃。

沼納樸兒,其國在榜葛剌之西。或言即中印度,古所稱佛國也。永樂十年遣使者齎敕撫諭其國,賜王亦不剌金絨錦、金織文綺、彩帛等物。十八年,榜葛剌使者醞其國王數舉兵侵擾,詔中官侯顯齎敕諭以睦鄰保境之義,因賜之彩幣;所過金剛寶座之地,亦有賜。然其王以去中國絕遠,朝貢竟不至。

祖法兒,自古里西北放舟,順風十晝夜可至。永樂十九年遣使偕阿丹、剌撒諸國入貢,命鄭和齎璽書賜物報之。二十一年,貢使復至。宣德五年,和再使其國,其王阿里即遣使朝貢,八年達京師。正統元年還國,賜璽書獎王。

其國東南大海,西北重山,天時常若八九月。五穀、蔬果、諸畜咸備。人體頎碩。王及臣民悉奉回回教,婚喪亦遵其制。多建禮拜寺。遇禮拜日,市絕貿易,男女長幼皆沐浴更新衣,以薔薇露或沉香油拭面,焚沉、檀、俺八兒諸香土CL,人立其上以薰衣,然後往拜。所過街市,香經時不散。天使至,詔書開讀訖,其王遍諭國人,盡出乳香、血竭、蘆薈、沒藥、蘇合油、安息香諸物,與華人交易。乳香乃樹脂。其樹似榆而葉尖長,土人砍樹取其脂為香。有駝雞,頸長類鶴,足高三四尺,毛色若駝,行亦如之,常以充貢。

木骨都束,自小葛蘭舟行二十晝夜可至。永樂十四年遣使與不剌哇、麻林諸國奉表朝貢,命鄭和齎敕及幣偕其使者往報之。後再入貢,復命和偕行,賜王及妃彩幣。二十一年,貢使又至。比還,其王及妃更有賜。宣德五年,和復頒詔其國。

國濱海,山連地曠,磽瘠少收。歲常旱,或數年不雨。俗頑嚚,時操兵習射。地不產木。亦如忽魯謨斯,壘石為屋,及用魚臘以飼牛羊馬駝云。

不剌哇,與木骨都束接壤。自錫蘭山別羅里南行,二十一晝夜可至。永樂十四年至二十一年,凡四入貢,並與木骨都束偕。鄭和亦兩使其國。宣德五年,和復往使。

其國,傍海而居,地廣斥鹵,少草木,亦壘石為屋。其鹽池。但投樹枝於中,已而取起,鹽即凝其上。俗淳。田不可耕,蒜葱之外無他種,專捕魚為食。所產有馬哈獸,狀如麞;花福祿,狀如驢;及犀、象、駱駝、沒藥、乳香、龍涎香之類,常以充貢。

竹步,亦與木骨都束接壤。永樂中嘗入貢。其地戶口不繁,風俗頗淳。鄭和至其地。地亦無草木,壘石以居,歲多旱,皆與木骨都束同。所產有獅子、金錢豹、駝蹄雞、龍涎香、乳香、金珀、胡椒之屬。

阿丹,在古里之西,順風二十二晝夜可至。永樂十四年遣使奉表貢方物。辭還,命鄭和齎敕及彩幣偕往賜之。自是,凡四入貢,天子亦厚加賜賚。宣德五年,海外諸番久缺貢,復命和齎敕宣諭。其王抹立克那思兒即遣使來貢。八年至京師。正統元年始還。自後,天朝不復通使,遠番貢使亦不至。前世梁、隋、唐時,並有丹丹國,或言即其地。

地膏腴,饒粟麥。人性強悍,有馬步銳卒七八千人,鄰邦畏之。王及國人悉奉回回教。氣候常和,歲不置閏。其定時之法,以月為準,如今夜見新月,明日即為月朔。四季不定,自有陰陽家推算。其日為春首,即有花開;其日為秋初,即有葉落;及日月交食、風雨潮汐,皆能預測。

其王甚尊中國。聞和船至,躬率部領來迎。入國宣詔訖,遍諭其下,盡出珍寶互易。永樂十九年,中官周姓者往,市得貓睛,重二錢許,珊瑚樹高二尺者數枝,又大珠、金珀、諸色雅姑異寶、麒麟、獅子、花貓、鹿、金錢豹、駝雞、白鳩以歸,他國所不及也。

蔬果、畜產咸備,獨無鵝、豕二者。市肆有書籍。工人所製金首飾,絕勝諸蕃。所少惟無草木,其居亦皆壘石為之。麒麟前足高九尺,後六尺,頸長丈六尺有二,短角,牛尾,鹿身,食粟豆餅餌。獅子形似虎,黑黃色無斑,首大、口廣、尾尖,聲吼若雷,百獸見之皆伏地。

嘉靖時製方丘朝日壇玉爵,購紅黃玉於天方、哈密諸蕃,不可得。有通事言此玉產於阿丹,去土魯番西南二千里,其地兩山對峙,自為雌雄,或自鳴,請如永樂、宣德故事,齎重賄往購。帝從部議,已之。

剌撒,自古里順風二十晝夜可至。永樂十四年遣使來貢,命鄭和報之。後凡三貢,皆與阿丹、不剌哇諸國偕。宣德五年,和復齎敕往使,竟不復貢。國傍海而居,氣候常熱,田瘠少收。俗淳,喪葬有禮。有事則禱鬼神。草木不生,久旱不雨。居室,悉與竹步諸國同。所產有乳香、龍涎香、千里駝之類。

麻林,去中國絕遠。永樂十三年遣使貢麒麟。將至,禮部尚書呂震請表賀,帝曰:「往儒臣進《五經四書大全》,請上表,朕許之,以此書有益於治也。麟之有無,何所損益,其已之。」已而麻林與諸蕃使者以麟及天馬、神鹿諸物進,帝御奉天門受之。百僚稽首稱賀,帝曰:「此皇考厚德所致,亦賴卿等翊贊,故遠人畢來。繼自今,益宜秉德迪朕不逮。」十四年又貢方物。

忽魯謨斯,西洋大國也。自古里西北行,二十五日可至。永樂十年,天子以西洋近國已航海貢琛,稽顙闕下,而遠者猶未賓服,乃命鄭和齎璽書往諸國,賜其王錦綺、彩帛、紗羅,妃及大臣皆有賜。王即遣陪臣已即丁奉金葉表,貢馬及方物。十二年至京師。命禮官宴賜,酬以馬直。比還,賜王及妃以下有差。自是凡四貢。和亦再使。後朝使不往,其使亦不來。

宣德五年復遣和宣詔其國。其王賽弗丁乃遣使來貢。八年至京師,宴賜有加。正統元年附爪哇舟還國。嗣後遂絕。

其國居西海之極。自東南諸蠻邦及大西洋商舶、西域賈人,皆來貿易,故寶物填溢。氣候有寒暑,春發葩,秋隕葉,有霜無雪,多露少雨。土瘠穀麥寡,然他方轉輸者多,故價殊賤。民富俗厚,或遭禍致貧,眾皆遺以錢帛,共振助之。人多白晰豐偉,婦女出則以紗蔽面,市列廛肆,百物具備。惟禁酒,犯者罪至死。醫卜、技藝,皆類中華。交易用銀錢。書用回回字。王及臣下皆遵回教,婚喪悉用其禮。日齋戒沐浴,虔拜者五。地多鹹,不產草木,牛羊馬駝皆啖魚臘。壘石為屋,有三四層者,寢處庖廁及待客之所,咸在其上。饒蔬果,有核桃、把聃、松子、石榴、葡萄、花紅、萬年棗之屬。境內有大山,四面異色。一紅鹽石,鑿以為器,盛食物不加鹽,而味自和;一白土,可塗垣壁;一赤土、一黃土,皆適於用。所貢有獅子、麒麟、駝雞、福祿、靈羊;常貢則大珠、寶石之類。

溜山,自錫蘭山別羅里南去,順風七晝夜可至;自蘇門答剌過小帽山西南行,十晝夜可至。永樂十年,鄭和往使其國。十四年,其王亦速福遣使來貢。自後三貢,並與忽魯謨斯諸國偕。宣德五年,鄭和復使其國,後竟不至。

其山居海中,有三石門,並可通舟。無城郭,倚山聚居。氣候常熱,土薄穀少,無麥,土人皆捕魚,暴乾以充食。王及群下盡回回人,婚喪諸禮,多類忽魯謨斯。山下有八溜,或言外更有三千溜,舟或失風入其處,即沉溺。

又有國曰比剌,曰孫剌。鄭和亦嘗齎敕往賜。以去中華絕遠,二國貢使竟不至。

南巫里,在西南海中。永樂三年遣使齎璽書、彩幣撫諭其國。六年,鄭和復往使。九年,其王遣使貢方物,與急蘭丹、加異勒諸國偕來。賜其王金織文綺、金繡龍衣、銷金幃幔及傘蓋諸物,命禮官宴賜遣之。十四年再貢。命鄭和與其使偕行,後不復至。

加異勒,西洋小國也。永樂六年遣鄭和齎詔招諭,賜以錦綺、紗羅。九年,其酋長葛卜者麻遣使奉表,貢方物。命賜宴及冠帶、彩幣、寶鈔。十年,和再使其國,後凡三入貢。宣德五年,和復使其國。八年又偕阿丹等十一國來貢。

甘巴里,亦西洋小國。永樂六年,鄭和使其地,賜其王錦綺、紗羅。十三年遣使朝貢方物。十九年再貢,遣鄭和報之。

宣德五年,和復招諭其國。王兜哇剌札遣使來貢,八年抵京師。正統元年附爪哇舟還國,賜敕勞王。

其鄰境有阿撥把丹、小阿蘭二國,亦以六年命鄭和齎敕招諭,賜亦同。

急蘭丹,永樂九年,王麻哈剌查苦馬兒遣使朝貢。十年命鄭和齎敕獎其王,賚以錦綺、紗羅、彩帛。

沙里灣泥,永樂十四年遣使來獻方物,命鄭和齎幣帛還賜之。

底里,永樂十年遣使奉璽書招諭其王馬哈木,賜絨錦、金織文綺、彩帛諸物。其地與沼納樸兒近,並賜其王亦不剌金。

千里達,永樂十六年遣使貢方物。賜其使冠帶、糸寧絲、紗羅、彩帛及寶鈔。比還,賜其王有加。

失剌比,永樂十六年遣使朝貢。賜其使冠帶、金織文綺、襲衣、彩幣、白金有差,其王亦優賜。

古里班卒,永樂中,嘗入貢。其土瘠穀少,物產亦薄。氣候不齊,夏多雨,雨即寒。

剌泥,永樂元年,其國中回回哈只馬哈沒奇剌泥等來貢方物,因攜胡椒與民市。有司請徵其稅,帝曰:「徵稅以抑逐末之民,豈以為利。今遠人慕義來,乃取其貨,所得幾何,而虧損國體多矣。其已之。」剌泥而外,有數國:曰夏剌比,曰奇剌泥,曰窟察泥,曰捨剌齊,曰彭加那,曰八可意,曰烏沙剌踢,曰坎巴,曰阿哇,曰打回。永樂中,嘗遣使朝貢。其國之風土、物產,無可稽。

白葛達,宣德元年遣其臣和者里一思入貢。其使臣言:「遭風破舟,貢物盡失,國主心捲心卷忠敬之忱,無由上達。此使臣之罪,惟聖天子恩貸,賜之冠帶,俾得歸見國主,知陪臣實詣闕廷,庶幾免責。」帝許之,使附鄰國貢舟還國,諭之曰:「倉卒失風,豈人力能制。歸語爾主,朕嘉王之誠,不在物也。」宴賜悉如禮。及辭歸,帝謂禮官曰:「天時漸寒,海道遼遠,可賜路費及衣服。」其國,土地瘠薄,崇釋教,市易用鐵錢。

又有黑葛達,亦以宣德時來貢。國小民貧,尚佛畏刑。多牛羊,亦以鐵鑄錢。

拂菻,即漢大秦,桓帝時始通中國。晉及魏皆曰大秦,嘗入貢。唐曰拂菻,宋仍之,亦數入貢。而《宋史》謂歷代未嘗朝貢,疑其非大秦也。

元末,其國人捏古倫入市中國,元亡不能歸。太祖聞之,以洪武四年八月召見,命齎詔書還諭其王曰:「自有宋失馭,天絕其祀。元興沙漠,入主中國百有餘年,天厭其昏淫,亦用隕絕其命。中原擾亂十有八年,當群雄初起時,朕為淮右布衣,起義救民。荷天之靈,授以文武諸臣,東渡江左,練兵養士,十有四年。西平漢王陳友諒,東縛吳王張士誠,南平閩、粵,戡定巴、蜀,北定幽、燕,奠安方夏,復我中國之舊疆。朕為臣民推戴即皇帝位,定有天下之號曰大明,建元洪武,於今四年矣。凡四夷諸邦皆遣官告諭,惟爾拂菻隔越西海,未及報知。今遣爾國之民捏古倫齎詔往諭。朕雖未及古先哲王,俾萬方懷德,然不可不使天下知朕平定四海之意,故茲詔告。」已而復命使臣普剌等齎敕書、彩幣招諭,其國乃遣使入貢。後不復至。

萬曆時,大西洋人至京師,言天主耶穌生於如德亞,即古大秦國也。其國自開闢以來六千年,史書所載,世代相嬗,及萬事萬物原始,無不詳悉。謂為天主肇生人類之邦,言頗誕謾不可信。其物產、珍寶之盛,具見前史。

意大里亞,居大西洋中,自古不通中國。萬歷時,其國人利瑪竇至京師,為《萬國全圖》,言天下有五大洲。第一曰亞細亞洲,中凡百餘國,而中國居其一。第二曰歐羅巴洲,中凡七十餘國,而意大里亞居其一。第三曰利未亞洲,亦百餘國。第四曰亞墨利加洲,地更大,以境土相連,分為南北二洲。最後得墨瓦臘泥加洲為第五。而域中大地盡矣。其說荒渺莫考,然其國人充斥中土,則其地固有之,不可誣也。

大都歐羅巴諸國,悉奉天主耶穌教,而耶穌生於如德亞,其國在亞細亞洲之中,西行教於歐羅巴。其始生在漢哀帝元壽二年庚申,閱一千五百八十一年至萬曆九年,利瑪竇始汎海九萬里,抵廣州之香山澳,其教遂沾染中土。至二十九年入京師,中官馬堂以其方物進獻,自稱大西洋人。

禮部言:「《會典》止有西洋瑣里國無大西洋,其真偽不可知。又寄居二十年方行進貢,則與遠方慕義特來獻琛者不同。且其所貢《天主》及《天主母圖》,既屬不經,而所攜又有神仙骨諸物。夫既稱神仙,自能飛升,安得有骨?則唐韓愈所謂凶穢之餘,不宜入宮禁者也。況此等方物,未經臣部譯驗,徑行進獻,則內臣混進之非,與臣等溺職之罪,俱有不容辭者。及奉旨送部,乃不赴部審譯,而私寓僧舍,臣等不知其何意。但諸番朝貢,例有回賜,其使臣必有宴賞,乞給賜冠帶還國,勿令潛居兩京,與中人交往,別生事端。」不報。八月又言:「臣等議令利瑪竇還國,候命五月,未賜綸音,毋怪乎遠人之鬱病而思歸也。察其情詞懇切,真有不願尚方錫予,惟欲山棲野宿之意。譬之禽鹿久羈,愈思長林豐草,人情固然。乞速為頒賜,遣赴江西諸處,聽其深山邃谷,寄跡怡老。」亦不報。

已而帝嘉其遠來,假館授粲,給賜優厚。公卿以下重其人,咸與晉接。瑪竇安之,遂留居不去,以三十八年四月卒於京。賜葬西郭外。

其年十一月朔,日食。曆官推算多謬,朝議將修改。明年,五官正周子愚言:「大西洋歸化人龐迪我、熊三拔等深明曆法。其所攜曆書,有中國載籍所未及者。當令譯上,以資採擇。」禮部侍郎翁正春等因請仿洪武初設回回曆科之例,令迪我等同測驗。從之。

自瑪竇入中國後,其徒來益眾。有王豐肅者,居南京,專以天主教惑眾,士大夫暨里巷小民,間為所誘。禮部郎中徐如珂惡之。其徒又自誇風土人物遠勝中華,如珂乃召兩人,授以筆札,令各書所記憶。悉舛謬不相合,乃倡議驅斥。四十四年,與侍郎沈水隺、給事中晏文輝等合疏斥其邪說惑眾,且疑其為佛郎機假托,乞急行驅逐。禮科給事中餘懋孳亦言:「自利瑪竇東來,而中國復有天主之教。乃留都王豐肅、陽瑪諾等,煽惑群眾不下萬人,朔望朝拜動以千計。夫通番、左道並有禁。今公然夜聚曉散,一如白蓮、無為諸教。且往來壕鏡,與澳中諸番通謀,而所司不為遣斥,國家禁令安在?」帝納其言,至十二月令豐肅及迪我等俱遣赴廣東,聽還本國。命下久之,遷延不行,所司亦不為督發。

四十六年四月,迪我等奏:「臣與先臣利瑪竇等十餘人,涉海九萬里,觀光上國,叨食大官十有七年。近南北參劾,議行屏斥。竊念臣等焚修學道,尊奉天主,豈有邪謀敢墮惡業。惟聖明垂憐,候風便還國。若寄居海嶼,愈滋猜疑,乞並南都諸處陪臣,一體寬假。」不報,乃怏怏而去。豐肅尋變姓名,復入南京,行教如故,朝士莫能察也。

其國善製礮,視西洋更巨。既傳入內地,華人多效之,而不能用。天啟、崇禎間,東北用兵,數召澳中人入都,令將士學習,其人亦為盡力。

崇禎時,曆法益疏舛,禮部尚書徐光啟請令其徒羅雅谷、湯若望等,以其國新法相參較,開局纂修。報可。久之書成,即以崇禎元年戊辰為歷元,名之曰《崇禎歷》。書雖未頒行,其法視《大統曆》為密,識者有取焉。

其國人東來者,大都聰明特達之士,意專行教,不求祿利。其所著書多華人所未道,故一時好異者咸尚之。而士大夫如徐光啟、李之藻輩,首好其說,且為潤色其文詞,故其教驟興。

時著聲中土者,更有龍華民、畢方濟、艾如略、鄧玉函諸人。華民、方濟、如略及熊三拔,皆意大里亞國人,玉函,熱而瑪尼國人,龐迪我,依西把尼亞國人,陽瑪諾,波而都瓦爾國人,皆歐羅巴洲之國也。其所言風俗、物產多誇,且有《職方外紀》諸書在,不具述。


\end{pinyinscope}