\article{列傳第二十六}

\begin{pinyinscope}
陳修滕毅趙好德翟善李仁吳琳楊思義滕德懋范敏費震張琬周禎劉惟謙周湞端復初李質黎光劉敏楊靖凌漢嚴德氏單安仁朱守仁薛祥秦逵趙翥趙俊唐鐸沈溍開濟

陳修,字伯昂,上饒人。從太祖平浙東,授理官。援引律令,悉本寬厚,盡改元季弊政。擢兵部郎中,遷濟南知府。時亂後,比戶彫殘,且多衛將練兵屯田其間。修撫治有方,兵民相安,流亡復業。帝嘉之。

洪武四年拜吏部尚書。六部之設,始自洪武元年。鎮江滕毅首長吏部,佐省臺裁定銓除考課諸法略具。至是修與侍郎李仁詳考舊典,參以時宜,按地衝僻,為設官煩簡。凡庶司黜陟及課功核實之法,皆精心籌畫,銓法秩然。未幾,卒官。其後部制屢創。令入覲官各舉所知,定內外封贈廕敘之典,自浮山李信始。天下朝正官各造事蹟文冊,圖畫土地人民以進,及撥用吏員法,自崑山餘熂始。仿《唐六典》,自五府、六部、都察院以下諸司設官分職,編集為書曰《諸司職掌》;定吏役考滿,給由法以為司、衛、府、縣首領;選監生能文章者兼除州縣官及學正、教諭,自泰興翟善始。三年一朝,考核等第,自沂水杜澤始。此洪武時銓政大略也。

六部初屬中書省,權輕,多仰承丞相意指。毅、修及詹同、吳琳、趙好德輩,居吏部稱賢,然亦無大建豎。至十三年,中書省革,部權乃專,而銓衡為尤要。顧帝用法嚴,熂以排宋訥誅,善貶,澤拜尚書。未數月罷。惟信歷侍郎,拜尚書,幾二載,卒於官云。

滕毅,字仲弘。太祖征吳,以儒士見,留徐達幕下。尋除起居注。命與楊訓文集古無道之君若桀、紂、秦始皇、隋煬帝行事以進。曰:「吾欲觀喪亂之由,以為炯戒耳。」吳元年出為湖廣按察使。尋召還,擢居吏部一月,改江西行省參政,卒。

趙好德,字秉彝,汝陽人。由安慶知府入為戶部侍郎。進尚書,改吏部。帝嘉其典銓平,嘗召與四輔官入內殿。坐論治道,命畫史圖像禁中。終陜西參政。子毅,永樂中,官至工部侍郎。

翟善,字敬夫,以貢舉歷官吏部文選司主事。二十六年,尚書詹徽、侍郎傅友文誅,命善署部事,再遷至尚書。明於經術,奏對合帝意。帝曰:「善雖年少,氣宇恢廓,他人莫及也。」欲為營第於鄉,善辭。又欲除其家戍籍,善曰:「戍卒宜增,豈可以臣破例。」帝益以為賢。二十八年坐事,降宣化知縣以終。

李仁,唐縣人。初仕陳友諒。王師克武昌,來歸。以常遇春薦,代陶安知黃州府。歷官侍郎,進尚書。坐事謫青州,政最。擢戶部侍郎,致仕。

吳琳,黃崗人。太祖下武昌,以詹同薦,召為國子助教。經術逾于同。吳元年除浙江按察司僉事,復入為起居注。命齎幣帛求書於四方。洪武六年,自兵部尚書改吏部,嘗與同迭主部事。踰年,乞歸。帝嘗遣使察之。使者潛至旁舍,一農人坐小杌,起拔稻苗布田,貌甚端謹。使者前曰:「此有吳尚書者,在否?」農人斂手,對曰:「琳是也。」使者以狀聞。帝為嘉歎。

楊思義,不詳其籍里。太祖稱吳王,授起居注。初,錢穀隸中書省。吳元年始設司農卿,以思義為之。明年設六部,改為戶部尚書。大亂之後,人多廢業。思義請令民間皆植桑麻,四年始徵其稅。不種桑者輸絹,不種麻者輸布,如《周官》里布法。詔可。帝念水旱不時,緩急無所恃,命思義令天下立預備倉,以防水旱。思義首邦計,以農桑積貯為急。凡所興設,雖本帝意,而經畫詳密,時稱其能。調陜西行省參政,卒於官。

終洪武朝,為戶部尚書者四十餘人,皆不久於職,績用罕著。惟茹太素、楊靖、滕德懋、范敏、費震之屬,差有聲。太素、靖自有傳。

德懋,字思勉,吳人。由中書省掾歷外任。洪武三年,召拜兵部尚書,尋改戶部。為人有才辨,器量弘偉。長於奏疏,一時招徠詔諭之文,多出其手。以事免官,卒。

范敏,閿鄉人。洪武八年舉秀才,擢戶部郎中。十三年授試尚書。薦耆儒王本等,皆拜四輔官。帝以徭役不均,命編造黃冊。敏議:百一十戶為里,丁多者十人為里長,鳩一里之事以供歲役。十年一周。餘百戶為十甲。後遂仍其制不廢。明年以不職罷。

費震,鄱陽人。洪武初,以賢良徵,為吉水知州,寬惠得民,擢知漢中。歲凶盜起,發倉粟十餘萬斛貸民,俾秋成還倉。盜聞,皆來歸。令占宅自為保伍,得數千家。帝聞而嘉之。後坐事被逮,以有善政,特釋為寶鈔提舉。十一年,帝謂吏部曰:「資格為常流設耳,有才能者當不次用之。」超擢者九十五人。而拜震戶部侍郎,尋進尚書。奉命定丞相、御史大夫以下負祿之制。出為湖廣布政使,以老致仕。

洪武初,有張琬者,鄱陽人。以貢士試高等,授給事中,改戶部主事。一日,帝問天下財賦、戶口之數。口對無遺。帝悅,立擢左侍郎。謹身殿災,上言時政。歲饑,請蠲民租百萬餘石。俱見嘉納。琬才敏,有心計,年二十七,卒於官。時人惜之。

周禎,字文典,江寧人。元末流寓湖南。太祖平武昌,用為江西行省僉事,歷大理卿。太祖以唐、宋皆有成律斷獄,惟元以一時行事為條格,胥吏易為奸,詔禎與李善長、劉基、陶安、滕毅等定律令。少卿劉惟謙、丞周湞與焉。書成,太祖稱善。

洪武元年設刑部,以禎為尚書。尋改治書侍御史。明年出為廣東行省參政。時省治初開,正官多缺,吏治鮮勸懲。香山丞沖敬有治行,以勞卒官。禎為文祭之,聞者感動。一時郡邑良吏雷州同知餘騏孫、惠州知府萬迪、乳源知縣張安仁、清流知縣李鐸、揭陽縣丞許德、廉州知府脫因、歸善知縣木寅,禎皆列其政績以聞。寅,土司。脫因,蒙古人也。於是屬吏益勸。三年九月召為御史中丞。尋引疾致仕。帝初即位,懲元寬縱,用法太嚴,奉行者重足立。律令既具,吏士始知循守。其後數有釐正,皆以禎書為權輿云。

劉惟謙,不詳何許人。吳元年以才學舉。洪武初,歷官刑部尚書。六年命詳定新律,刪繁損舊,輕重得宜。帝親加裁定頒行焉。後坐事免。

周湞,字伯寧,鄱陽人。江西十才子之一也,官亦至刑部尚書。

終洪武世,為刑部者亦幾四十人,楊靖最著,而端復初、李質、黎光、劉敏亦有名。

復初,字以善,溧水人。子貢裔也,從省文,稱端氏。元末為小吏。常遇春鎮金華,召致幕下。未幾,辭去。太祖知其名,召為徽州府經歷。令民自實田,BZ為圖籍,積弊盡刷。稍遷至磨勘司令。時官署新立,案牘填委,復初鉤稽無遺。帝嘗廷譽之。性嚴峭,人不敢干以私。僚屬多貪敗,復初獨以清白免。洪武四年,超拜刑部尚書,用法平。杭州飛糧事覺,逮繫百餘人。詔復初往治,誠偽立辨,知府以下皆服罪。明年出為湖廣參政。令民來歸者,復其賦一年。流亡畢集。以治辦聞。坐事召還,卒。子孝文,翰林待詔;孝思,翰林侍書。先後使朝鮮,並著清節,朝鮮人為立「雙清館」云。

李質,字文彬,德慶人。有材略。元末居何真麾下,嘗募兵平德慶亂民,旁郡多賴其保障。名士客嶺南者,茶陵劉三吾、江右伯顏子中、羊城孫蕡、建安張智等,皆禮之。洪武元年,從真降,授中書斷事。明年改都督府斷事,強力執法。五年擢刑部侍郎,進尚書,治獄平恕。遣振饑山東,御製詩餞之。尋出為浙江行省參政。居三年,惠績著聞。帝念質老,召還。嘗入見便殿,訪時政。質直言無隱。拜靖江王右相。王罪廢,質竟坐死。

黎光,東莞人。以鄉薦拜御史,巡蘇州,請振水災,全活甚眾。巡鳳陽,上封事悉切時弊,帝嘉之。洪武九年,擢刑部侍郎,執法不阿,為御史大夫陳寧所忌,坐事死貶所。

劉敏,肅寧人。舉孝廉,為中書省吏。嘗暮市蘆龍江,旦載於家,俾妻織蓆,鬻以奉母。而後,入治事。性廉介,或遺之瓷瓦器,亦不受。為楚相府錄事,中書以沒官女婦給文臣家,眾勸其請給以事母。敏固辭曰:「事母,子婦事,何預他人。」及省臣敗,吏多坐誅,敏獨無所預。帝賢之,擢工部侍郎,改刑部。出為徽州府同知,有惠政,卒於官。

楊靖,字仲寧,山陽人。洪武十八年進士,選吏科庶吉士。明年擢戶部侍郎。時任諸司者,率進士及太學生,然時有不法者。帝製《大誥》,舉通政使蔡瑄、左通政茹瑺、工部侍郎秦逵及靖以諷厲之曰:「此亦進士太學生也,能率職以稱朕心。」其見稱如此。

二十二年進尚書。明年五月詔在京官三年皆遷調,著為令。乃以刑部尚書趙勉與靖換官。諭曰:「愚民犯法,如啖飲食。設法防之,犯者益眾。推恕行仁,或能感化。自今惟犯十惡並殺人者死,餘罪皆令輸粟北邊。」又曰:「在京獄囚,卿等覆奏,朕親審決,猶恐有失。在外各官所擬,豈能盡當?卿等當詳讞,然後遣官審決。」靖承旨研辨,多所平反。帝嘉納之。嘗鞫一武弁,門卒撿其身,得大珠,屬僚驚異。靖徐曰:「偽也,安有珠大如此者乎。」碎之。帝聞,嘆曰:「靖此舉,有四善焉。不獻朕求悅,一善也;不窮追投獻,二善也;不獎門卒,杜小人僥倖,三善也;千金之珠卒然而至,略不動心,有過人之智,應變之才,四善也。」

二十六年,兼太子賓客,並給二祿。已,坐事免。會征龍州趙宗壽,詔靖諭安南輸粟餉師。以白衣往。安南相黎一元以陸運險艱,欲不奉詔。靖宣示反覆開諭,且許以水運。一元乃輸粟二萬,至沲海江別造浮橋以達龍州。帝大悅,拜靖左都御史。靖公忠有智略,善理繁劇,治獄明察而不事深文。寵遇最厚,同列無與比。三十年七月,坐為鄉人代改訴冤狀草,為御史所劾。帝怒,遂賜死。時年三十八。

時有凌漢,字斗南,原武人。以秀才舉,獻《烏鵲論》。授官,歷任御史。巡按陜西,疏所部疾困數事。帝善之,召其子賜衣鈔。漢鞫獄平允。及還京,有德漢者,邀置酒,欲厚贈以金。漢曰:「酒可飲,金不可受也。」帝聞之嘉歎,擢右都御史。時詹徽為左,論議不合,每面折徽,徽銜之。左遷刑部侍郎,改禮部。後為徽所劾,降左僉都御史。帝憫其衰,令歸田里。漢以徽在,有後憂,不敢去。歲餘徽誅,復擢右僉都御史。尋致仕歸。漢出言不檢,居官屢躓。然以廉直見知於帝,故終得保全。

又吳人嚴德氏,由御史擢左僉都御史,以疾求歸。帝怒,黥其面,謫戍南丹。遇赦放還。布衣徒步,自齒齊民。宣德中猶存。嘗以事為御史所逮,德氏跪堂下,自言曾在臺勾當公事,曉三尺法。御史問何官。答言:「洪武中臺長,所謂嚴德氏是也。」御史大驚,揖起之。次日往謁,則擔囊徙矣。有教授與飲,見其面黥,戴敝冠,問:「老人犯何法?」氏述前事,因言「先時國法甚嚴,仕者不保首領,此敝冠不易戴也。」乃北面拱手,稱「聖恩,聖恩」云。

單安仁,字德夫,濠人。少為府吏。元末江淮兵亂,安仁集義兵保鄉里,授樞密判官。從鎮南王孛羅普化守揚州。時群雄四起,安仁歎曰:「此輩皆為人驅除耳。王者之興,當自有別。」鎮南王為長槍軍所逐,安仁無所屬,聞太祖定集慶,乃曰:「此誠是已。」率眾歸附。太祖悅,即命將其軍守鎮江。嚴飭軍伍,敵不敢犯。移守常州。其子叛,降張士誠,太祖知安仁忠謹,弗疑也。久之,遷浙江副使。悍帥橫斂民,名曰「寨糧」,安仁置於法。進按察使,徵為中書左司郎中,佐李善長裁斷。調瑞州守禦千戶,入為將作卿。

洪武元年擢工部尚書,仍領將作事。安仁精敏多智計,諸所營造,大小中程,甚稱帝意。逾年改兵部尚書。請老歸,賜田三千畝,牛七十角,歲給尚書半俸。六年起山東參政。懇辭,許之。家居,嘗奏請濬儀真南壩至朴樹灣,以便官民輸挽;疏轉運河江都深港以防淤淺;移瓜州倉CC置揚子橋西,免大江風潮之患。帝善其言,再授兵部尚書,致仕。初,尚書階正三品。十三年,中書省罷,始進為正二。而安仁致仕在前。帝念安仁勛舊,二十年特授資善大夫。其年十二月卒,年八十五。

徐州硃守仁者,字元夫。元末亦以保障功官樞密同知,守舒城。明兵下廬州,以城來歸。歷官工部侍郎。洪武四年進尚書,奉命察山東官吏,稱旨。尋改北平行省參政,以饋餉不繼,謫蒼梧知縣。初,守仁知袁州,撫安創殘,民甚德之。至是連知容州、高唐州,皆有善政。十年進四川布政使,治尚簡嚴。以年老致仕。坐事罰輸作,特宥之。十五年,雲南平,改威楚、開南等路宣撫司為楚雄府,遂命守仁知府事。招集流移,均徭役,建學校,境內大治。二十八年上計入朝,郡人垂涕送之。拜太僕卿。首請立牧馬草場於江北滁州諸處。所轄十四監九十八群。馬大蕃息。馬政之修,自守仁始。久之,致仕。永樂初,入朝,遇疾卒。

薛祥,字彥祥,無為人。從俞通海來歸。渡江,為水寨管軍鎮撫。數從征有功。洪武元年轉漕河南,夜半抵蔡河。賊驟至,祥不為動,好語諭散之。帝聞大喜。以方用兵,供億艱,授京畿都漕運使,分司淮安。浚河築堤,自揚達濟數百里,徭役均平,民無怨言。有勞者立奏,授以官。元都下,官民南遷,道經淮安,祥多方存恤。山陽、海州民亂,駙馬都尉黃琛捕治,詿誤甚眾。祥會鞫,無驗者悉原之。治淮八年,民相勸為善。及考滿還京,皆焚香,祝其再來,或肖像祀之。

八年授工部尚書。時造鳳陽宮殿。帝坐殿中,若有人持兵鬥殿脊者。太師李善長奏諸工匠用厭鎮法,帝將盡殺之。祥為分別交替不在工者,並鐵石匠皆不預,活者千數。營謹身殿,有司列中匠為上匠,帝怒其罔,命棄市。祥在側,爭曰:「奏對不實,竟殺人,恐非法。」得旨用腐刑。祥復徐奏曰:「腐,廢人矣,莫若杖而使工。」帝可之。明年改天下行省為承宣布政司。以北平重地,特授祥,三年治行稱第一。為胡惟庸所惡,坐營建擾民,謫知嘉興府。惟庸誅,復召為工部尚書。帝曰:「讒臣害汝,何不言?」對曰:「臣不知也。」明年,坐累杖死,天下哀之。子四人,謫瓊州,遂為瓊山人。

孫遠,正統七年進士。景泰時,官戶部郎中。天順元年,擢本部右侍郎,改工部。奉詔塞開封決河。還,仍改戶部。成化初,督兩廣軍餉,位至南京兵部尚書,以忤汪直免官。

其繼祥為工部尚書有名者,有秦逵等。

逵,字文用,宣城人。洪武十八年進士。歷事都察院。奉檄清理囚徒,寬嚴得宜。帝嘉其能,擢工部侍郎。時營繕事繁,部中缺尚書,凡興作事皆逵領之。初,議籍四方工匠,驗其丁力,定三年為班,更番赴京,三月交代,名曰「輸班匠」。未及行,至是逵議量地遠近為班次,置籍,為勘合付之,至期齎至部,免其家徭役,著為令。帝念逵勤勩,詔有司復其家。二十二年進尚書。明年改兵部。未幾,復改工部。帝以學校為國儲材,而士子巾服無異胥吏,宜更易之。命逵製式以進。凡三易,其製始定。賜監生藍衫、絳各一,以為天下先。明代士子衣冠,蓋創自逵云。

有趙翥者,永寧人。有志節,以學行聞。由訓導舉賢良,擢贊善大夫,拜工部尚書。奏定天下歲造軍器之數,及議定籓王宮城制度。

趙俊者,不知何許人。自工部侍郎進尚書。帝以國子監所藏書板,歲久殘剝,命諸儒考補,工部督匠修治。俊奉詔監理,古籍始備。洪武十二年,翥改署刑部。尋致仕去。俊,十七年免。而逵於二十五年九月坐事自殺。

唐鐸,字振之,虹人。太祖初起兵,即侍左右。守濠州,從定江州,授西安縣丞。召為中書省管勾。洪武元年,湯和克延平,以鐸知府事,拊輯新附,士民安之。居三年,入為殿中侍御史,復出知紹興府。六年十二月,召拜刑部尚書。明年改太常卿。丁母憂,特給半俸。十四年,服闋,起兵部尚書。

明年,初置諫院,以為諫議大夫。帝嘗與侍臣論歷代興廢,曰:「使朕子孫如成、康,輔弼如周、召,則可祈天永命。」鐸因進曰:「豫教元良,選左右為輔導,宗社萬年福也。」帝又謂鐸曰:「人有公私,故言有邪正。正言務規諫,邪言務謗諛。」鐸曰:「謗近忠,諛近愛。不為所眩,則讒佞自遠。」未幾,左遷監察御史。請選賢能京官遍歷郡縣,訪求賢才,體察官吏。選歷練老成、望隆名重者,居布政、按察之職。帝從之。既復擢為右副都御史,歷刑、兵二部尚書。二十二年,置詹事院,命吏部曰:「輔導太子,必擇端重之士。三代保傅,禮甚尊嚴。兵部尚書鐸,謹厚有德量,以為詹事。食尚書俸如故。」以鐸嘗請豫教故也。其年,致仕。

二十六年,起太子賓客,進太子少保。二十八年,龍州土官趙宗壽以奏鄭國公常茂死事不實,被召,又不至。帝怒,命楊文統大軍往討。而命鐸招諭。鐸至,廉得茂實病死,宗壽亦伏罪來朝。乃詔文移兵征奉議諸州叛蠻,即以鐸參議軍事。逾月,諸蠻平。鐸相度形勢,請設奉議衛及向武、河池、懷集、武仙、賀縣諸處守禦千戶所,鎮以官軍。皆報可。

鐸為人長者,性慎密,不妄取予。帝以故舊遇之,嘗曰:「鐸自友及臣至今三十餘年,其與人交不至變色,絕亦不出惡聲。」又曰:「都御史詹徽剛斷嫉惡,胥吏不得肆其貪,謗訕滿朝。唐鐸重厚,又謂懦而無為。人心不古,有若是耶!」後徽卒坐罪誅死,而鐸恩遇不替。三十年七月,卒於京師,年六十九。賻贈甚厚,命有司護其喪歸葬。

沈溍,字尚賢,錢塘人。與鐸同官兵部,以明敏稱。帝嘗以勛臣子弟多骫法,撰《大誥》二十二篇,諭天下武臣,皆令誦習,使知儆惕。已,又以諭戒八條,頒示將士。時溍以試兵部侍郎掌部事,一切訓飭事宜,皆承旨行之。尋進尚書。廣西都司建譙樓、青州衛造軍器,皆擅科民財。溍請凡都司衛所營作,必都督府奏準。官給物料,毋擅役民。違者治罪。仍禁武臣預民事。時干戈甫息,武臣暴橫,數捍文法,至是始戢,溍力也。帝嘗諭致治之要在進賢、退不肖。溍因言:「君子常少,小人常多,在上風厲之耳,賢者舉而不仁者遠矣。」帝善其言。二十三年,以溍與工部尚書秦逵換官,賜誥獎諭。尋復舊任,後以事免。

明初,衛所世籍及軍卒勾補之法,皆溍所定。然名目瑣細,簿籍煩多,吏易為奸。終明之世頗為民患,而軍衛亦日益耗減。語詳《兵志》。潮州生陳質,父在戍籍。父沒,質被勾補,請歸卒業。帝命除其籍。溍以缺軍伍,持不可。帝曰:「國家得一卒易,得一士難。」遂除之。然此皆特恩云。

開濟,字來學,洛陽人。元末為察罕帖木兒掌書記。洪武初,以明經舉。授河南府訓導,入為國子助教。以疾罷歸。十五年七月,御史大夫安然薦濟有吏治才,召試刑部尚書,踰年實授。

濟以綜核為己任,請天下諸司設文簿,日書所行事,課得失。又各部勘合文移,立程限,定功罪。又言,軍民以細故犯罪者,宜即決遣。數月間,滯牘一清。帝大以為能。會都御史趙仁言,曩者以「賢良方正」、「孝弟力田」諸科所取士列置郡縣,多不舉職。宜核其去留。濟條議,以「經明行修」為一科、「工習文詞」為一科、「通曉書義」為一科、「人品俊秀」為一科、「練達治理」為一科、「言有條理」為一科。六科備者為「上」;三科以上為「中」;不及三科者為「下」。從之。

濟敏慧有才辯。凡國家經制、田賦、獄訟、工役、河渠事,眾莫能裁定,濟一算畫,即有條理品式,可為世守。以故帝甚信任,數備顧問,兼預他部事。人以是忌之,謗議滋起。然濟亦深刻,好以法中傷人。嘗奉命定詐偽律。濟議法巧密。帝曰::「張密網以羅民,可乎?」又設籍曰「寅戌之書」,以程僚屬出入。帝切責曰:「古人以卯酉為常。今使趨事者朝寅暮戌。奉父母,會妻子,幾何時耶!」又為榜戒其僚屬,請揭文華殿。帝曰:「告誡僚屬之言,欲張殿廷,豈人臣禮?」濟慚謝。

尋令郎中仇衍脫囚死,為獄官所發。濟與侍郎王希哲、主事王叔徵執獄官,斃之。其年十二月,御史陶垕仲等發其事,且言:「濟奏事時,置奏答刂懷中,或隱而不言,覘伺上意,務為兩端,奸狡莫測。役甥女為婢。妹早寡,逐其姑而略其家財。」帝怒,下濟獄,併希哲、衍等皆棄市。

贊曰:六部之制仿於《周官》,所以佐王理邦國,熙庶績,任至重也。明興,建官分職,立法秩然。又三途用人,求賢彌廣。若陳修、滕毅之典銓法,楊思義、範敏之治賦役,周禎之定律令,單安仁之領將作,以至沈溍、開濟輩之所經畫,皆委曲詳備,細大不遺。考其規模,固一代政治之權輿者歟。


\end{pinyinscope}