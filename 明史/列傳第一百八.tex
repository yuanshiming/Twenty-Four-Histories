\article{列傳第一百八}

\begin{pinyinscope}
萬士和王之誥劉一儒吳百朋劉應節徐栻王遴畢鏘舒化李世達曾同享弟乾亨辛自修溫純趙世卿李汝華

萬士和,字思節,宜興人。父吉,桐廬訓導,有學術。士和成嘉靖二十年進士,改庶吉士,授禮部主事。父喪除,乞便養母,改南京兵部。累遷江西僉事,歲裁上供瓷器千計。遷貴州提學副使,進湖廣參政。撫納叛苗二十八砦,以功賚銀幣。三殿工興,採木使者旁午。士和經畫備至,民賴以安。遷江西按察使,之官踰期,劾免。起山東按察使,再行廣東左布政使。政事故專決於左,士和曰:「朝廷設二使,如左右手,非有軒輊。」乃約右使分日治事。召拜應天府尹,道遷右副都御史。督南京糧儲,奏請便民六事。隆慶初,進戶部右侍郎,總督倉場。尋改禮部,進左。引疾歸。神宗立,起南京禮部侍郎,署國子監事。萬曆元年,禮部尚書陸樹聲去位。張居正用樹聲言,召士和代之。條上崇儉數事。又以災祲屢見,奏乞杜倖門,容戇直,汰冗員,抑干請,多犯時忌。俺答及所部貢馬,邊臣請加官賞。士和言賞賚有成額,毋徇邊臣額外請,從之。方士倚馮保求官,士和持不可。成國公朱希忠歿,居正許贈王,士和力爭。給事中餘懋學言事得罪,士和言直臣不當斥。於是積忤居正。給事中朱南雍承風劾之,遂謝病去。居正歿,起南京禮部尚書,再疏引年不赴。卒,年七十一。贈太子少保,謚文恭。

王之誥,字告若,石首人。嘉靖二十三年進士。授吉水知縣。遷戶部主事,改兵部員外郎,出為河南僉事。討師尚詔有功,轉參議。調大同兵備副使。以搗板升功,增俸一級,進山西右參政,擢右僉都御史,巡撫遼東。大興屯田,每營墾田百五十頃,役軍四百人。列上便宜八事,行之。召為兵部右侍郎。尋以左侍郎總督宣、大、山西軍務。

隆慶元年,就進右都御史。俺答犯石州,之誥令山西總兵官申維岳、參將劉寶、尤月、黑雲龍四營兵尾之南下,而檄大同總兵官孫吳、山西副總兵田世威等出天門關,遏其東歸。巡撫王繼洛駐代州不出,維岳不敢前,石州遂陷。殺人數萬,所過無孑遺,大掠十有四日而去。事聞,維岳、世威、寶論死,繼洛戍邊,吳落職。之誥以還守南山,止貶二秩。

明年,詔之誥以左侍郎巡視薊、遼、保定、宣、大、山西,侍郎劉燾巡陜西、延綏、寧夏、甘肅。之誥以疾辭,代以冀練。已,復因給事中張鹵言,皆罷不遣。三年,起督京營。進右都御史,總督陜西三邊軍務。以延寧將士搗巢功,予一子官,遷南京兵部尚書。神宗嗣位,召拜刑部尚書。張居正專政,之誥與有連,每規切之。萬曆三年,乞假送母歸,踰時不至,被劾。會之誥亦奏請終養,遂報許。後居正喪父奪情,杖言者闕下。歸葬還闕,之誥以召還直臣收人心為勸。卒,贈太子太保,謚端襄。

時有夷陵劉一儒者,字孟真,亦居正姻也。嘉靖三十八年進士。屢官刑部侍郎。居正當國,嘗貽書規之。居正歿,親黨皆坐斥,一儒獨以高潔名。尋拜南京工部尚書。甫半歲,移疾歸。初,居正女歸一儒子,珠琲紈綺盈箱篋,一儒悉扃之別室。居正死,貲產盡入官,一儒乃發向所緘物還之。南京御史李一陽請還一儒於朝,以厲恬讓。帝可其奏。一儒竟不赴召,卒於家。天啟中,追謚莊介。

吳百朋,字維錫,義烏人。嘉靖二十六主年進士。授永豐知縣。徵拜御史,歷按淮、揚、湖廣。擢大理寺丞,進右少卿。

四十二年夏,進右僉都御史,撫治鄖陽。改提督軍務,巡撫南、贛、汀、漳。與兩廣提督吳芳討平河源賊李亞元、程鄉賊葉丹樓,又會師破倭海豐。

初,廣東大埔民藍松山、余大眷倡亂,流劫潼、延、興、泉間。官軍擊敗之,奔永春。與香寮盜蘇阿普、范繼祖連兵犯德化,為都指揮耿宗元所敗,偽請撫。百朋亦陽罷兵,而誘賊黨為內應,先後悉擒之,惟三巢未下。三巢者,和平李文彪據岑岡,龍南謝允樟據高沙,賴清規據下歷。朝廷以倭患棘,不討且十年。文彪死,子珍及江月照繼之,益猖獗。四十四年秋,百朋進右副都御史,巡撫如故。上疏曰:「三巢僭號稱王,旋撫旋叛。廣東和平、龍川、興寧,江西龍南、信豐、安遠,蠶食過半。不亟討,禍不可言。三巢中惟清規跨江、賡六縣,最逆命,用兵必自下歷始。」帝采部義,從之。百朋乃命守備蔡汝蘭討擒清規於苦竹嶂,群賊震懾。

隆慶初,吏部以百朋積苦兵間,稍遷大理卿。給事中歐陽一敬等請留百朋剿賊,詔進兵部右侍郎兼右僉都御史,巡撫如故。百朋奏,春夏用兵妨耕作,宜且聽撫,帝從之。尋擢南京兵部右侍郎。乞終養,不許。改刑部右侍郎。父喪歸,起改兵部。萬曆初,奉命閱視宣、大、山西三鎮。百朋以糧餉、險隘、兵馬、器械、屯田、鹽法、番馬、逆黨八事核邊臣,督撫王崇古、吳兌、總兵郭琥以下,陞賞黜革有差。又進邊圖,凡關塞險隘,番族部落,士馬強弱,亭障遠近,歷歷如指掌。以省母歸。起南京右都御史,召拜刑部尚書。踰年卒。

劉應節,字子和,濰人。嘉靖二十六年進士。授戶部主事。歷井陘兵備副使,兼轄三關。三關屬井陘道自此始。四十三年,以山西右參政擢右僉都御史,巡撫遼東。母喪歸。隆慶元年,起撫河南。俺答寇石州,山西騷動,詔應節赴援。已,寇退。會順天巡撫耿隨卿坐殺平民充首功逮治,改應節代之。建議永平西門抵海口距天津止五百里,可通漕,請募民習海道者赴天津領運,同運官出海達永平。部議以漕卒冒險不便,發山東、河南粟十萬石儲天津,令永平官民自運焉。

四年秋,進右副都御史,巡撫如故。旋進兵部右侍郎兼右僉都御史,代譚綸總督薊、遼、保定軍務。奏罷永平、密雲、霸州采礦。又因御史傅孟春言,議諸鎮積貯,當計歲豐歉。常時以折色便軍,可以積粟;凶歲以本色濟荒,可以積銀。又明年建議通漕密雲,上疏曰:「密雲環控潮、白二水,天設之以便漕者也。向二水分流,到牛欄山始合。通州運艘至牛欄山,以上陸運至龍慶倉,輸挽甚苦。今白水徙流城西,去潮水不二百武,近且疏渠植壩,合為一流,水深漕便。舊昌平運額共十八萬石有奇,今止十四萬,密雲僅得十萬,惟賴召商一法,而地瘠民貧,勢難長恃。聞通倉粟多紅朽。若漕五萬石於密雲,而以本鎮折色三萬五千兩留給京軍,則通倉無腐粟,京軍沾實惠,密雲免僉商,一舉而三善備矣。」報可。

給事中陳渠以薊鎮多虛伍,請核兵省餉。應節上疏曰:「國初設立大寧,薊門猶稱內地。既大寧內徙,三衛反覆,一切防禦之計,與宣、大相埒,而額兵不滿三萬。倉卒召外兵,疲於奔命,又半孱弱。於是議減客兵,募土著,而游食之徒,饑聚飽颺。請清勾逃軍,而所勾皆老稚,又未必安於其伍。本鎮西起鎮邊,東抵山海,因地制兵,非三十萬不可。今主、客兵不過十三萬而已。且宣府地方六百里,額兵十五萬;大同地方千餘里,額兵十三萬五千;今薊、昌地兼二鎮,而兵力獨不足。援彼例此,何以能守?以今上計,發精兵二十餘萬,恢復大寧,控制外邊,俾畿輔肩背益厚,宣、遼聲援相通,國有重關,庭無近寇,此萬年之利也。如其不然,集兵三十萬,分屯列戍,使首尾相應,此百年之利也。又不然,則選主、客兵十七萬,訓練有成,不必仰藉鄰鎮,亦目前茍安之計。今皆不然,徵兵如弈棋,請餉如乞糴,操練如摶沙,教戰如談虎。邊長兵寡,掣襟肘見。今為不得已之計,姑勾新軍補主兵舊額十一萬,與入衛客兵分番休息,庶軍不告勞,稍定邊計。」部議行所司清軍,而補兵之說卒不行。

萬曆元年,進右都御史兼兵部右侍郎,總督如故。進南京工部尚書,召為戎政尚書,改刑部。錦衣馮邦寧者,太監保從子,道遇不引避,應節叱下之,保不悅。會雲南參政羅汝芳奉表至京,應節出郭與談禪,給事中周良寅疏論之,遂偕汝芳劾罷。卒,贈太子少保。

初,王宗沐建議海運,應節與工部侍郎徐栻請開膠萊河,張居正力主之。用栻樣兼僉都御史以往,議鑿山引泉,計費百萬。議者爭駁之。召式還,罷其役。栻,常熟人,累官南京工部尚書。

王遴,字繼津,霸州人。嘉靖二十六年進士。除紹興推官。入為兵部主事,歷員外郎。峭直矜節概,不妄交。同官楊繼盛劾嚴嵩及其孫效忠冒功事,下部覆。世蕃自為稿,以屬武選郎中周冕。冕發之,反得罪。尚書聶豹懼,趣所司以世蕃稿上。遴直前爭,豹怒,竟覆如世蕃言。繼盛論死,遴為資粥饘,且以女字其子應箕。嵩父子大恚,摭他事下之詔獄。事白復官。及繼盛死,收葬之。遷山東僉事,再遷岢嵐兵備副使。有威名,為巡撫所忌,劾去。官民相率訟冤,詔許起用。

四十五年,擢右僉都御史,巡撫延綏。寇大入定邊、固原,總兵官郭江戰歿。總督陳其學、陜西巡撫戴才坐免,遴貶俸一秩。隆慶改元,寇六入塞,皆失利去。而御史溫如玉論遴不已,解官候勘。後御史楊BQ勘上其功,遂以故官巡撫宣府。總兵官馬芳驍勇,寇不敢深入。遴乃大興屯田,邊儲賴之。秩滿,進右副都御史。尋召拜兵部右侍郎。省親歸,起協理戎政。

神宗立,張居正秉政。遴其同年生,然雅不相能。會議閱邊,遴請行。命往陜西四鎮。峻絕饋遺。事竣,遽移疾歸。居正歿,始起南京工部尚書。尋改兵部,參贊機務。守備中官丘得用濫役營軍,遴奏禁之,因奏行計安留都十二事。召拜戶部尚書。先奉詔蠲除及織造議留共銀百七十六萬餘兩,命於太倉庫補進,遴言:「陛下歷十餘年之儲積,僅三百餘萬,今因一載蠲除,即收補於庫。計十餘年之積,不足償二年取補之資。矧金花額進歲當百萬,自六年以後增進二十萬,今合六年計之,不啻百萬矣。庫積非源泉,歲進不已,後將何繼?」因言京、通二倉糧積八百萬石,足供九年之需,請量改折百五十萬石,三年而止。詔許一年。

時尚寶丞徐貞明、御史徐待開京東水田,遴力贊之,議遂決。故事,戶部銀專供軍國,不給他用。帝大婚,暫取濟邊銀九萬兩為織造費,至是復欲行之,遴執爭。未幾,詔取金四千兩為慈寧宮用,遴又力持。皆不納。已,陳理財七事,請崇節儉、重農務、督逋負、懲貪墨、廣儲蓄、飭貢市。帝報曰:「事關朕躬者已知之。餘飭所司議行。」時釋教大盛,遴請汰其莊者歸農,聚眾修齋者坐左道罪。禮部尚書沈鯉請如遴言。詔已許,后妃宦官多言不便,事中止。

改兵部尚書。遼東總兵官李成梁賂遺遍輦轂,不敢至遴門。遴在戶部頻執爭,已為中官所嫉。會帝閱壽宮,中官持御批索馬。遴以為題本當鈐印,司禮傳奉由科發部,無徑下部者,援故事執奏。帝不悅。大學士申時行嘗以管事指揮羅秀屬遴補錦衣僉書,遴格不許。時行乃調旨責遴擅留御批,失敬上體。御史因交章劾遴,遴乞休去,張佳胤代之。給事中張養蒙言:「羅秀本太監滕祥奴,賄入禁衛。往歲營僉書,尚書遴持正,為所中傷去。未幾秀即躐用,物議沸騰。」於是黜秀,佳胤亦罷。遴雖退,聲望愈重,以年高存問者再三。三十六年卒。贈太子太保。天啟中,追謚恭肅。

畢鏘,字廷鳴,石埭人。嘉靖三十二年進士。授刑部主事。歷郎中,擢浙江提學副使,遷廣西右參政,進按察使,再遷湖廣左布政使。召為太僕卿,未至,改應天尹。海瑞撫江南,移檄京府,等於屬吏,鏘卻不受。瑞察鏘政,更與善。進南京戶部右侍郎,督理糧儲。

萬曆二年,入為刑部右侍郎。改戶部,總督倉場。擢南京戶部尚書,謝病去。起南京工部尚書,就改吏部,徵為戶部尚書。帝以風霾諭所司陳時政,鏘以九事上。中言:「錦衣旗校至萬七千四百餘人,內府諸監局匠役數亦稱是。此冗食之尤,宜屏除冒濫。州縣丈田滋弊,雲南鼓鑄不酬工直,官已裁而復置,田欲墾而再停。請酌土俗人情,毋率意更改。至袍服錦綺,歲有積餘,何煩頻織。天燈費巨萬,尤不經。濫予不可不裁,淫巧不可不革。」他所奏,並多切要。近倖從中撓之,不盡行。鏘乃引年乞罷。予馳驛歸。鏘遇事守正,有物望。年及八十,賜存問,加太子少保。後凡存問者再。其孫汝楩奉表入謝,詔以為太學生。年九十三而卒。贈太子太保,謚恭介。舒化,字汝德,臨川人。嘉靖三十八年進士。授衡州推官。改補鳳陽,擢戶科給事中。

隆慶初,三遷刑科給事中。帝任宦官,旨多從中下。化言:「法者天下之公,大小罪犯宜悉付法司。不當,則臣等論劾。若竟自敕行,則喜怒未必當,而法司與臣等俱虛設。」詔是其言。冬至郊天,聞帝咳聲,推論陰陽姤復之漸,請法天養微陽,詞甚切直。有詔言災眚洊至,由部院政事不修,令廠衛密察。化偕同列言:「廠衛徼巡輦下,惟詰奸宄、禁盜賊耳。駕馭百官,乃天子權,而糾察非法,則責在臺諫,豈廠衛所得乾。今命之刺訪,將必開羅織之門,逞機阱之術,禍貽善類,使人人重足累息,何以為治。且廠衛非能自廉察,必屬之番校。陛下不信大臣,反信若屬耶?」御史劉思賢等亦極陳其害。帝並不從。已而事竟寢。校尉負屍出北安門,兵馬指揮孫承芳見之,疑有奸,繫獄鞫訊,詞連內官李陽春。陽春懼,訴於帝。言尉所負非死者,出外乃死,承芳妄生事,刑校尉。帝信之,杖承芳六十,斥為民。化請以陽春所奏下法司勘問,不納。

四年熱審,請釋纍臣鄭履淳、李芳,及朝審,又請釋李已,皆得宥。時高拱當國,路楷、楊順以構殺沈煉論死。拱欲為楷地,謂順首禍,順死,楷可勿坐。化取獄牘示拱曰:「獄故無煉名。有之,自楷始。楷誠罪首。」拱又議宥方士王金等罪,化言:「此遺詔意,即欲勿罪,宜何辭?」忤拱,出為陜西參政。再疏致仕歸。

萬曆初,累擢太僕少卿。復以疾歸。由南京大理卿召拜刑部左侍郎。雲南緬賊平,帝御午門樓受俘。化讀奏詞,音吐洪亮,進止有儀,帝目屬之。會刑部缺尚書,手詔用化。化言:「陛下仁心出天性。知府錢若賡、知州方復乾以殘酷死戍。請飭大小臣僚各遵律例,毋淫刑。《大明律》一書,高皇帝揭之兩廡,手加更定。今未經詳斷者或命從重擬議,已經定議者又詔加等處斬,是謂律不足用也。去冬雨雪不時,災異頻見,咎當在此。」帝優詔答之。會續修《會典》,因輯嘉靖三十四年以後事例與刑名相關者三百八十二條,奏之。詔頒示中外。

十四年,應詔陳言。請信詔令,清獄訟,速訊讞,嚴檢驗,禁冤濫,而以格天安民歸本聖心。帝嘉納焉。帝慮群下欺罔,間有訐發,輒遣官逮捕,牽引證佐,文案累積。化言:「主術貴執要,不當侵有司;徒使人歸過於上,而下得緣以飾非。」潞王府小校以事為兵馬司吏目所笞,帝怒,逮吏目下詔獄,掠死,又罪其捕卒七人。化爭之。詔罪為首一人,餘並獲宥。明年,京察拾遺,南京科道論及化。遂三疏乞歸。帝不許。會當慮囚,復起視事。中貴傳帝意宥重辟三十餘人,化爭不可。詔卒從其議。尋稱病篤,乃聽歸。卒,贈太子少保,謚莊僖。李世達,字子成,涇陽人。嘉靖三十五年進士。授戶部主事。改吏部,歷考功、文選郎中,與陸光祖並為尚書所倚。隆慶初,丁曾祖憂。起右通政,歷南京太僕卿。

萬曆二年以右僉都御史巡撫山東。尋進右副都御史,總理河道。未上,改撫浙江。旋移疾歸,起督漕運兼巡撫鳳陽。黃河南侵,淮安告警,世達請修石堤捍城;寶應氾光湖風濤險惡,歲漂溺,請開越河殺水勢。俱報可。遷南京兵部右侍郎。召改戶部,復改吏部,進左侍郎。擢南京吏部尚書,就改兵部,參贊機務。

俄召為刑部尚書。中官張德毆人死,世達請置於理,刑科唐堯欽亦言之,德遂屬吏。大興知縣王階坐撻樂舞生下吏,帝密遣兩校尉偵之,讞日為巡風主事孫承榮所拒。校尉還奏,帝怒詰世達。世達言偵伺非大體。承榮竟奪俸。東廠太監張鯨有罪,言官交劾,帝曲貸之。世達執奏,帝乃屏鯨於外。駙馬都尉侯拱宸僕斃平民抵法,世達請並坐拱宸。乃革其任,命國學肄禮。罪人焦文粲法不當死,帝怒入之。會朝審,命戶部尚書宋纁主筆。世達言於纁,薄文粲罪。忤旨,詰問,復據法以對。帝卒不從。時帝燕居多暴怒,近侍屢以非罪死,世達因災異上書以諷。浙江饑,或請令罪人出粟除罪。世達言:「法不可廢,寧赦毋贖。赦則恩出於上,法猶存。贖則力出於下,人滋玩。」識者韙之。改左都御史。兵馬指揮何價虐死三人,御史劉思瑜庇之。世達劾奏,帝鐫思秩。復劾罷御史韓介等數人。帝深惡言官,下詔申飭,責以挾私報復。世達言:「效忠持正者,語雖過激,心實無他。即或心未可知,而言不可廢,並宜容納。惟緘默依阿,然後加黜罰。則讜言日進,邪說漸消。」報聞。二十一年,與吏部尚書孫鑨同主京察,斥政府私人殆盡。考功郎中趙南星被劾貶官,世達力爭之,反除南星等名,遂求去,不許。其秋,吏部侍郎趙用賢以絕婚事被訐,世達白其無罪。郎中楊應宿、鄭材疏詆世達,遂連章乞休去。歸七年卒。贈太子太保,謚敏肅。

曾同亨,字於野,吉水人。父存仁,雲南布政使。同亨舉嘉靖三十八年進士。授刑部主事。改禮部,遷吏部文選主事。故事,丞簿以下官,聽胥吏銓注,同亨悉躬親之。與陸光祖、李世達齊名。隆慶初,為文選郎中,薦用遺佚幾盡。進太常少卿,請急去。萬曆初,起大理少卿。歷順天府尹,以右副都御史巡撫貴州。御史劉臺得罪張居正,同亨,臺姊夫也,給事中陳三謨欲並逐之,奏同亨羸不任職。詔調南京,遂移疾歸。九年,京察拾遺,給事中奏燿、御史錢岱等復希居正指,列同亨名。勒休致。

居正卒,起南京太常卿。召為大理卿,遷工部右侍郎。督治壽宮,節浮費三十餘萬。由左侍郎進尚書。軍器自外輸,率不中程,奏請半收其直,又請減織造之半。皆報可。汝安王妃乞橋稅,同亨拒之。帝竟如妃請。內府工匠,隆慶初數至萬五千八百人,尋汰二千五百人,而中官濫增不已。同亨疏請清釐。已得旨,中官復奏寢之。給事中楊其休疏爭,弗納。同亨弟乾亨請裁冗員以裕經費,京衛諸武臣謂減己月俸也,大嘩,伺同亨出朝,圍而噪之。同亨再乞休,不得請。九門工成,加太子少保。力乞去,詔乘傳歸。起南京吏部尚書,辭不拜。久之,再起故官,累辭乃就職。稅使所在虐民,同亨極諫。三十三年,大計京官,與考功郎徐必達持正不撓。是年,北察失執政意,中旨留給事中錢夢皋等;南察及同亨自陳疏,亦久不下。同亨適給由入都,遂引疾。詔加太子太保致仕。

同亨初入吏部,嚴嵩其鄉人,尚書吳鵬則父同年也,同亨無私謁。嘗止宿署舍,彌月不歸。雅與羅汝芳、耿定向善。尚書楊博痛詆偽儒,同亨曰:「此中多暗修,非可概斥。即使陽假名義,視呈身進取、恬不知恥者,孰愈哉?」卒年七十有五。贈少保,謚恭端。

弟乾亨,字于健。從羅洪先學。登萬曆五年進士,除合肥知縣,調休寧。擢御史。給事中馮景劾李成梁被謫,乾亨以尚書張學顏右成梁也,並劾之。帝怒,黜為海州判官。稍遷大名推官,歷光祿少卿。十八年冬,敕兼監察御史,閱視大同邊務。劾罷總兵官以下十餘人。大同士兵歲餉萬二千石,兵自征之,民不勝擾。乾亨議留兵二百,餘盡汰之。屢奏邊備事宜,輒中機要。諸武弁之詬同亨也,大學士王家屏遣諭之曰:「天下有叛軍,寧有叛臣?若曹於禁地辱大臣,罪且死。」諸人乃散去。尚書石星言貴臣被辱,大傷國體,給事中鐘羽正亦言之。不報。家屏密揭力爭,乃奪掌後府定國公徐文璧祿半歲,而治首事者以法。乾亨尋進大理丞,遷少卿。考功郎趙南星以考察事被斥,乾亨論救,侵執政,復移書辨之。廷推巡撫者三,俱不用。遂引疾歸,未幾卒。乾亨言行不茍,與其兄並以名德稱。

辛自修,字子吉,襄城人。嘉靖三十五年進士。除海寧知縣。擢吏科給事中,奏言:「吏部銓注,遴才要矣,量地尤急。邇京府屬吏以大計去者十之五,豈畿輦下獨多不肖哉?地艱而事猥也。請量地劇易以除官,量事繁簡以注考。」吏部善其言,請令撫按舉劾如自修議。巡視京營,劾典營務鎮遠侯顧寰、協理僉都御史李燧,請戒寰罷燧。從之。歷遷禮科都給事中。誠意伯劉世延不法,自修極論其奸。詔革任禁錮。隆慶元年,給事中胡應嘉言事斥,自修疏救。未幾,論奪尚書顧可學、徐可成,侍郎朱隆禧、郭文英贈謚;以可成由黃冠,文英由工匠,可學、隆禧俱以方藥進也。擢太僕少卿,引疾歸。

萬曆六年,起應天府丞,再遷光祿卿。以右僉都御史巡撫保定六府。奏減均徭里甲銀六萬兩,增築雄、任丘二縣堤,以禦滹沱水患。每歲防秋,巡撫移駐易州,征所部供費,防秋已罷,徵如故,自修奏已之。入歷大理卿,兵部左、右侍郎,擢南京右都御史。御史沈汝梁者,巡視下江,用餽遺為名,盡括所部贖鍰,自修劾奏之。帝方欲懲貪吏,乃命逮治汝梁,而召自修為左都御史。

十五年,大計京官,政府欲庇私人,去異己。吏部尚書楊巍承意指惟謹,自修患之,先期上奏,請勿以愛憎為喜怒,排抑孤立之人。帝善其言,而政府不悅。有貪競者十餘輩,皆政府所厚,自修欲去之。給事中陳與郊自度不免,遂言憲臣將以一眚棄人,一舉空國。於是自修所欲斥者悉獲免。已而御史張鳴岡等拾遺,首工部尚書何起鳴。起鳴故以督工與中官張誠厚,而雅不善自修,遂訐自修仇主使。與郊及給事中吳之佳助之。御史高維崧、趙卿、張鳴岡、左之宜不平,劾起鳴飾非詭辨。帝先入張誠言,頗疑自修。得疏益不悅,曰:「朝廷每用一人,言官輒紛紛排擊。今起鳴去,爾等舉堪此任者。」維崧等具疏引罪,無他舉。帝怒,悉出之外。給事中張養蒙申救,亦奪俸。刑部主事王德新復疏爭,語侵嬖倖。帝下之詔獄,酷刑究主者。無所承,乃削其籍。自修不自安,亟引疾歸。

自修之進也,非執政意,故不為所容。久之,起南京刑部尚書。復以工部尚書召。未上,卒。贈太子太保,謚肅敏。

德新,安福人,後起官至光祿丞。

溫純,字景文,三原人。嘉靖四十四年進士。由壽光知縣徵為戶科給事中。隆慶三年,穆宗既禫除,猶不與大臣接。純請遵祖制延訪群工,親決章奏,報聞。屢遷兵科都給事中。倭陷廣東廣海衛,大殺掠而去。總兵劉燾以戰卻聞,純劾燾欺罔。時方召燾督京營,遂置不問。黔國公沐朝弼有罪,詔許其子襲爵。純言事未竟,不當遽襲。中官陳洪請封其父母,純執不可。言官李已、石星獲譴,疏救之。初,趙貞吉更營制,三營各統一大將。以恭順侯吳繼爵典五軍,而都督袁正、焦澤典神樞、神機。繼爵恥與同列,固辭。帝為罷二人,盡易以勛臣。純請廣求將才,毋拘世爵,不納。已,復命文臣三人分督之,時號「六提督」。純以政令多門,極陳不便,遂復舊制。俺答請貢市,高拱定議許之。純以為弛邊備,非中國利。出為湖廣參政,引疾歸。

萬歷初,用薦起河南參議。十二年,以大理卿改兵部右侍郎兼右副都御史,巡撫浙江。入為戶部左侍郎,進右副都御史,督倉場。母憂去。進南京吏部尚書。召拜工部尚書。父老,乞養歸。終喪,召為左都御史。

礦稅使四出,有司逮繫纍纍,純極論其害,請盡釋之,不報。已,諸閹益橫,所至剽奪,汙人婦女。四方無賴奸人蜂起言利:有請開雲南塞外寶井者;或又言海外呂宋國有機易山,素產金銀,歲可得金十萬、銀三十萬;或言淮、揚饒鹽利,用其策,歲可得銀五十萬。帝並欣然納之,遠近駭震。純言:「緬人方伺隙,寶井一開,兵端必起。余元俊一鹽犯,數千贓不能輸,而欲得五十萬金,將安取之?機易山在海外,必無遍地金銀,任人往取;不過假借詔旨,闌出禁物與番人市易,利歸群小,害貽國家。乞盡捕諸奸人,付臣等行法,而亟撤稅監之害民者。」亦不報。當是時,中外爭請罷礦稅,帝悉置不省。純等憂懼不知所出,乃倡諸大臣伏闕泣請。帝震怒,問誰倡者,對曰:「都御史臣純。」帝為霽威,遣人慰諭曰:「疏且下。」乃退。已而卒不行。廣東李鳳、陜西梁永、雲南楊榮並以礦稅激民變,純又抗言:「稅使竊弄陛下威福以十計,參隨憑藉稅使聲勢以百計,地方奸民竄身為參隨爪牙以萬計。宇內生靈困於水旱,困於採辦、營運、轉輸,既囂然喪其樂生之心,安能復勝此千萬虎狼耶!願即日罷礦稅,逮鳳等置於理。」亦不報。

先是,御史顧龍楨巡按廣東,與布政使王泮語不合,起毆之,泮即棄官去。純劾罷龍楨。御史於永清按陜西貪,懼純舉奏,倡同列救龍楨,顯與純異,以脅制純,又與都給事中姚文蔚比而傾純。純不勝憤,上疏盡發永清交構狀,並及文蔚,語頗侵首輔沈一貫。一貫等疏辨。帝為下永清、文蔚二疏,而純劾疏留不下。純益憤,三疏論之,因力丐罷,乃謫永清。純遂與一貫忤。給事中陳治則、鐘兆斗皆一貫私人,先後劾純。御史湯兆京不平,疏斥其妄。純求去,章二十上,杜門者九閱月。帝雅重純,諭留之。純不得已,強起視事。及妖書事起,力為沈鯉、郭正域辨誣。楚宗人戕殺撫臣,純復言無反狀。一貫怨益深。三十二年,大計京朝官。純與吏部侍郎楊時喬主之,一貫所欲庇者兆斗及錢夢皋等皆在謫中。疏入久之,忽降旨切責,盡留被察科道官,而察疏仍不下。純求去益力。夢皋、兆斗既得留,則連章訐純楚事。言純曲庇叛人,且誣以納賄。廷臣大駭,爭劾夢皋等。夢皋等亦再疏劾純求勝。俱留中。已,南京給事中陳嘉訓等極論二人陰有所恃,朋比作奸,當亟斥之,而聽純歸,以全大臣之體。帝竟批夢皋等前疏,予純致仕,夢皋、兆斗亦罷歸。

純清白奉公。五主南北考察,澄汰悉當。肅百僚,振風紀,時稱名臣。卒,贈少保。天啟初,追謚恭毅。

趙世卿,字象賢,歷城人。隆慶五年進士。授南京兵部主事。張居正當國,政尚嚴。州縣學取士不得過十五人;布按二司以下官,雖公事毋許乘驛馬;大辟之刑,歲有定額;征賦以九分為率,有司不及格者罰;又數重譴言事者。世卿奏匡時五要。請廣取士之額,寬驛傳之禁,省大辟,緩催科,而末極論言路當開,言:「近者臺諫習為脂韋,以希世取寵。事關軍國,卷舌無聲。徒摭不急之務,姑塞言責。延及數年,居然高踞卿貳,誇耀士林矣。然此諸人豈盡矩詬無節,忍負陛下哉,亦有所懲而不敢耳。如往歲傅應禎、艾穆、沈思孝、鄒元標皆以建言遠竄,至今與戍卒伍。此中才之士,所以內自顧恤,寧自同於寒蟬也。宜特發德音,放還諸人,使天下曉然知聖天子無惡直言之意,則士皆慕義輸誠,效忠於陛下矣。」居正欲重罪之。吏部尚書王國光曰:「罪之適成其名,請為公任怨。」遂出為楚府右長史。明年京察,復坐以不謹,落職歸。

居正死,起戶部郎中,出為陜西副使。累遷戶部右侍郎,督理倉場。世卿饒心計。凡所條奏,酌劑贏縮,軍國賴焉。戶部尚書陳垞有疾,侍郎張養蒙避不署事,帝怒,並罷之,而進世卿為尚書。時礦稅使四出為害,江西稅監潘相至擅捕繫宗室。曩時關稅所入歲四十餘萬,自為稅使所奪,商賈不行,數年間減三之一,四方雜課亦如之。歲入益寡,國用不支,邊儲告匱,而內供日繁。歲增金花銀二十萬,宮帑日充羨。世卿請復金花銀百萬故額,罷續增數,不許。乞發內庫銀百萬及太僕馬價五十萬以濟邊儲,復忤旨切責。世卿又請正潘相罪,且偕九卿數陳其害,皆不納。世卿復言脂膏已竭,閭井蕭然,喪亂可虞,揭竿非遠,不及今罷之,恐後將無及。帝亦不省。

三十二年,蘇、松稅監劉成以水災請暫停米稅。帝以歲額六萬,米稅居半,不當盡停,今以四萬為額。世卿上言:「鄉者既免米稅,旋復再徵,已失大信於天下。今成欲免稅額之半,而陛下不盡從,豈惻隱一念,貂榼尚存,而陛下反漠然不動心乎?」不報。

其夏,雷火毀祖陵明樓,妖蟲蝕樹,又大雨壞神道橋梁。帝下詔咨實政。世卿上疏曰:

今日實政,孰有切於罷礦稅者!古明主不貴異物,今也聚悖入之財,斂蒼生之怨,節儉之謂何?是為君德計,不可不罷者一。多取所以招尤,慢藏必將誨盜。鹿臺、鉅橋,足致倒戈之禍。是為宗社計,不可不罷者二。古者國家無事則預桑土之謀,有事則議金湯之策。安有鑿四海之山,榷三家之市,操弓挾矢,戕及良民,毀室踰垣,禍延雞犬,經十數年而不休者!是為國體計,不可不罷者三。貂榼漁獵,翼虎咆哮。毀掘冢墓,則枯骨蒙殃,奸虐子女,而良家飲恨。人與為怨,言雚噪屢聞,此而不已,後將何極!是為民困計,不可不罷者四。國家財賦不在民則在官,今盡括入奸人之室。故督逋租而逋租絀,稽關稅而關稅虧,搜庫藏而庫藏絕,課鹽策而鹽策薄,徵贖鍰而贖鍰消。外府一空,司農若埽。是為國課計,不可不罷者五。天子之令,信如四時。三載前嘗曰「朕心仁愛,自有停止之時」,今年復一年,更待何日?天子有戲言,王命委草莽。是為詔令計,不可不罷者六。

陛下試思:服食宮室,以至營造征討,上何事不取之民,民何事不供之上?嗟此赤子,曾無負於國,乃民方懽呼以供九重之欲,而陛下不少遂其欲;民方奔走以供九重之勞,而陛下不少慰其勞;民方竭蹶以赴九重之難,而陛下不少恤其難。返之於心,必有不自安者矣。陛下勿謂蠢蠢小民可駕馭自我,生殺自我,而不足介意也。民之心既天之心,今天譴頻仍,雷火妖蟲,淫雨疊至,變不虛生,其應非遠。故今日欲回天意在恤民心,欲恤民心在罷礦稅,無煩再計而決者。

帝優答之,而不行。至三十四年三月,始詔罷礦使,稅亦稍減。然遼東、雲南、四川稅使自若,吏民尤苦之。雲南遂變作,楊榮被戕。而西北水旱時時見告,世卿屢請減租發振,國用益不支。踰月復奏請捐內帑百萬佐軍用,不從。世卿遂連章求去,至十五上,竟不許。先是,福王將婚,進部帑二十七萬,帝猶以為少,數遣中使趣之。中使出誶語,且劾世卿抗命。世卿以為辱國,疏聞於朝,帝置不問。至三十六年,七公主下嫁,宣索至數十萬。世卿引故事力爭,詔減三之一。世卿復言:「陛下大婚止七萬,長公主下嫁止十二萬,乞陛下再裁損,一仿長公主例。」帝不得已從之。福王新出府第,設崇文稅店,爭民利,世卿亦諫阻。

世卿素勵清操,當官盡職。帝雅重之。吏部缺尚書,嘗使兼署,推舉無所私。惟楚宗人與王相訐,世卿力言王非偽,與沈一貫議合。李廷機輔政,世卿力推之。廷臣遂疑世卿黨比。於是給事中杜士全、鄧去霄、何士晉、胡忻,御史蘇為霖、馬孟禎等先後劾之,世卿遂杜門乞去。章復十餘上,不報。三十八年秋,世卿乃拜疏出城候命。明年十月,乘柴車徑去。廷臣以聞,帝亦不罪也。家居七年卒,贈太子少保。

李汝華,字茂夫,睢州人。萬曆八年進士。授兗州推官。征授工科給事中,嘗劾戎政尚書鄭洛不職。及出閱甘肅邊務,洛方經略西事,主和戎。汝華疏洛畏敵貽患,且劾諸將吏侵軍資,復請盡墾甘肅閒田。還朝,歷吏科都給事中,多所糾擿。

尋遷太常少卿,擢右僉都御史,巡撫南、贛。稅使四出,議括關津諸稅輸內府。汝華以稅本餉軍,力爭止之。既而詔四方稅務盡領於有司,以其半輸稅監,進內府,半輸戶部。獨江西潘相勒有司悉由己輸。汝華極論相違詔,帝竟如相議,且推行之四方。

汝華在贛十四年,威惠甚著,進秩兵部右侍郎,召拜戶部左侍郎。尚書趙世卿去位,遂掌部事。福王莊田四萬頃,詔旨屢趣,不能及額。汝華數偕廷臣執爭,僅減四之一。及王既之國,詔許自遣使督租,所在驛騷。內使閻時詣汝州,杖二人死。汝華請遵祖制隸有司,盡撤還使者,不納。畿輔、山東大饑,因汝華言,出倉米平糶,且發銀以振。汝華復奏行救荒數事,兩地賴之。先是,山東饑,蠲歲賦七十萬。是年盡蠲又百七十餘萬。汝華以邊餉不繼,請天下稅課未入內藏者,暫留一年補其缺,輔臣亦助為言。疏三上,不報。已,進尚書。

四十六年,鄭繼之去,兼攝吏部事。畿輔、陜西大饑,汝華請振,皆不報。遼東兵事興,驟增餉三百萬。汝華累請發內帑不得,則借支南京部帑,括天下庫藏餘積,徵宿逋,裁工食,開事例。而遼東巡撫周永春請益兵加賦,汝華議:天下田賦,自貴州外,畝增銀三釐五毫,得餉二百萬。明年,復議益兵增賦如前。又明年四月,兵部以募兵市馬,工部以制器,再議增賦。於是畝增二釐,為銀百二十萬。先後三增賦,凡五百二十萬有奇,遂為歲額。當是時,內帑山積,廷臣請發,率不應。計臣無如何,遂為一切茍且之計,苛斂百姓。而樞臣徵兵,乃遠及蠻方,致奢崇明、安邦彥相繼反,用師連年。又割四川、雲南、廣西、湖廣、廣東所加之賦以餉之,而遼餉仍不充,天下已不可支矣。

汝華練達勤敏,立朝無黨阿。官戶部久,於國計贏縮,邊儲虛實,與鹽漕屯牧諸大政,皆殫心裁劑。歲比不登,意常主寬恤,獨加賦之議不能力持,馴致萬方虛耗,內外交訌。天啟元年得疾乞休,加太子太保致仕。卒,謚恭敏。從子夢辰,自有傳。

贊曰:古稱文昌政本,七卿之任,蓋其重矣。萬士和諸人奉職勤慮,異夫依阿保位之流;劉應節、王遴、舒化、李世達尤其卓然者哉。李汝華司邦計,值兵興餉絀,請帑不應,乃不能以去就爭,而權宜取濟,遂與裒刻聚斂者同譏。時事至此,其可嘆也夫!


\end{pinyinscope}