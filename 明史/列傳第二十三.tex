\article{列傳第二十三}

\begin{pinyinscope}
陳遇秦從龍葉兌范常潘庭堅宋思顏夏煜郭景祥李夢庚王濂毛騏楊元杲阮弘道汪河孔克仁

陳遇,字中行,先世曹人。高祖義甫,宋翰林學士,徙居建康,子孫因家焉。遇天資沉粹,篤學博覽,精象數之學。元末為溫州教授,已而棄官歸隱。學者稱為靜誠先生。太祖渡江,以秦從龍薦,發書聘之,引伊、呂、諸葛為喻。遇至,與語,大悅,遂留參密議,日見親信。太祖為吳王,授供奉司丞,辭。即皇帝位,三授翰林學士,皆辭。乃賜肩輿一乘,衛士十人護出入,以示榮寵。

洪武三年,奉命至浙江廉察民隱,還賜金帛。除中書左丞,又辭。明年召對華蓋殿,賜坐,命草《平西詔》。授禮部侍郎,兼弘文館大學士,復辭。西域進良馬,遇引漢故事以諫。除太常少卿,固辭。強之,不可。最後除禮部尚書,又固辭。帝沉吟良久,從之。自是不復強以官。帝嘗從容言欲官其子,遇曰:「臣三子皆幼,學未成,請俟異日。」帝亦弗強也。

遇自開基之始,即侍帷幄。帝嘗問保國安民至計,遇對:「以不嗜殺人,薄斂,任賢,復先王禮樂為首務。」廷臣或有過被譴責,遇力為解,多得全釋。其計畫多秘不傳,而寵禮之隆,勳戚大臣無與比者。數監幸其第,語必稱「先生」,或呼為「君子」。命爵輒辭,終成其高。十七年卒,賜葬鐘山。

子恭,舉人,累官工部尚書,有能聲。遇弟遠,字中復,嘗隨遇侍帝。永樂初,為翰林待詔,精繪事。遠子孟顒,善書。

秦從龍,字元之,洛陽人。仕元,官江南行臺侍御史。兵亂,避居鎮江。徐達之攻鎮江也,太祖謂之曰:「聞有秦元之者,才器老成,汝當詢訪,致吾欲見意。」達下鎮江,訪得之。太祖命從子文正、甥李文忠奉金綺造其廬聘焉。從龍與妻陳偕來,太祖自迎之於龍江。

時太祖居富民家,因邀從龍與同處,朝夕訪以時事。已,即元御史臺為府,居從龍西華門外,事無大小悉與之謀。嘗以筆書漆簡,問答甚密,左右皆不能知。從龍生日,太祖與世子厚有贈遺,或親至其家燕飲。至正二十五年冬,從龍子澤死,請告歸。太祖出郊握手送之。尋病卒,年七十,太祖驚悼。時方督軍至鎮江,親臨哭之,厚恤其家,命有司營葬。

葉兌,字良仲,寧海人。以經濟自負,尤精天文、地理、卜筮之書。元末,知天運有歸,以布衣獻書太祖。列一綱三目,言天下大計。時太祖已定寧越,規取張士誠、方國珍。而察罕兵勢甚盛,遣使至金陵招太祖,故兌書於三者籌之為詳。其略曰:

愚聞:取天下者,必有一定之規模。韓信初見高祖,畫楚、漢成敗;孔明臥草廬,與先主論三分形勢者是也。今之規模,宜北絕李察罕,南併張九四。撫溫、台,取閩、越,定都建康,拓地江、廣。進則越兩淮以北征,退則畫長江而自守。夫金陵,古稱龍蟠虎踞帝王之都。藉其兵力資財,以攻則克,以守則固,百察罕能如吾何哉?江之所備,莫急上流。今義師已克江州,足蔽全吳。況自滁、和至廣陵,皆吾所有。非直守江,兼可守淮矣。張氏傾覆可坐而待,淮東諸郡亦將來歸。北略中原,李氏可併也。今聞察罕妄自尊大,致書明公,如曹操之招孫權。竊以元運將終,人心不屬,而察罕欲效操所為,事勢不侔。宜如魯肅計,鼎足江東,以觀天下之釁,此其大綱也。

至其目有三。張九四之地,南包杭、紹,北跨通、泰,而以平江為巢穴。今欲攻之,莫若聲言掩取杭、紹、湖、秀,而大兵直搗平江。城固難以驟拔,則以鎖城法困之。於城外矢石不到之地別築長圍,分命將卒四面立營,屯田固守,斷其出入之路。分兵略定屬邑,收其稅糧以贍軍中。彼坐守空城,安得不困?平江既下,巢穴已傾,杭、越必歸,餘郡解體。此上計也。

張氏重鎮在紹興。紹興懸隔江海,所以數攻而不克者,以彼糧道在三江斗門也。若一軍攻平江,斷其糧道;一軍攻杭州,絕其援兵,紹興必拔。所攻在蘇、杭,所取在紹興,所謂多方以誤之者也。紹興既拔,杭城勢孤,湖、秀風靡,然後進攻平江,犁其心腹,江北餘孽隨而瓦解。此次計也。

方國珍狼子野心,不可馴狎。往年大兵取婺州,彼即奉書納款。後遣夏煜、陳顯道招諭,彼復狐疑不從。顧遣使從海道報元,謂江東委之納款,誘令張昶齎詔而來。且遣韓叔義為說客,欲說明公奉詔。彼既降我,而反欲招我降元。其反覆狡獪如是,宜興師問罪。然彼以水為命,一聞兵至,挈家航海,中原步騎無如之何。夫上兵攻心,彼言杭、趙一平,即當納土,不過欲款我師耳。攻之之術,宜限以日期,責其歸順。彼自方國璋之沒,自知兵不可用。又叔義還稱義師之盛,氣已先挫。今因陳顯道以自通,正可脅之而從也。事宜速不宜緩。宣諭之後,更置官吏,拘集舟艦,潛收其兵權,以消未然之變。三郡可不勞而定。

福建本浙江一道,兵脃城陋。兩浙既平,必圖歸附。下之一辯士力耳。如復稽遲,則大兵自溫、處入,奇兵自海道入,福州必不支。福州下,旁郡迎刃解矣。威聲已震,然後進取兩廣,猶反掌也。

太祖奇其言,欲留用之,力辭去。賜銀幣襲衣。後數歲,削平天下,規模次第,略如兌言。

范常,字子權,滁人。太祖軍滁,杖策謁軍門。太祖夙知其名,與語意合,留置幕下。有疑輒問,常悉以實對。諸將克和州,兵不戢。常言於太祖曰:「得一城而使人肝腦塗地,何以成大事?」太祖乃切責諸將。搜軍中所掠婦女,還其家,民大悅。太祖以四方割據,戰爭無虛日,命常為文,禱於上帝。其辭曰:「今天下紛紜,生民塗炭,不有所屬,物類盡矣。倘元祚未終,則群雄當早伏其辜。某亦在群雄中,請自某始。若已厭元德,有天命者宜歸之,無使斯民久阽危苦。存亡之機,驗於三月。」太祖嘉其能達己意,命典文牘,授元帥府都事。取太平,命為知府,諭之曰:「太平,吾股肱郡,其民數困於兵,當令得所。」常以簡易為治,興學恤民。官廩有穀數千石,請給民乏種者,秋稔輸官,公私皆足。居三年,民親愛之,召入為侍儀。

洪武元年,擢翰林直學士兼太常卿。帝銳意稽古禮文。群臣集議,間有異同。常能參合眾言,委曲當上意。尋以病免歸。歲餘,手詔徵詣闕,仍故官。帝宴閒,輒命儒臣列坐,賦詩為樂。常每先成,語多率。帝笑曰:「老范詩質樸,殊似其為人也。」遷起居注。常有足疾,數在告,賜以安車。尋乞歸,帝賦詩四章送之。賜宅於太平。子祖,歷官雲南左參政,有修潔稱。

潘庭堅,字叔聞,當塗人。元末為富陽教諭,謝去。太祖駐太平,以陶安薦,徵庭堅為帥府教授。慎密謙約,為太祖所稱。下集慶,擢中書省博士。婺州下,改為金華府,以庭堅同知府事。時上游諸郡次第平定,擇儒臣撫綏之。先後用陶安、汪廣洋於江西,而庭堅與王愷守浙東。太祖為吳王,設翰林院,與安同召為學士。而庭堅已老,遂告歸。洪武四年復召至,主會試。

子黼,字章甫。有文名,官至江西按察使。會修律令,留為議律官。書成,卒。黼謹飭類父,而文采清雅過之。父子皆以鄉校顯,時以為榮。

宋思顏,不知何許人。太祖克太平,以思顏居幕府。及定集慶,置江南行中書省,太祖總省事,以李善長及思顏為參議。同時所設省中官李夢庚、郭景祥、侯元善、楊元杲、陶安、阮弘道、孔克仁、王愷、欒鳳、夏煜等數十人。而思顏獨與善長並授參議,其任較諸人為重。已,建大都督府,以思顏兼參軍事。太祖嘗視事東閣,天暑,汗沾衣。左右更以衣進,皆數經浣濯者。思顏曰:「主公躬行節儉,真可示法子孫,惟願終始如一。」太祖嘉其直,賜之幣。他日又進曰:「句容虎為害,既捕獲,宜除之,今豢養民間何益?」太祖欣然,即命殺虎。其隨事納忠類如此。後出為河南道按察僉事,坐事死。

夏煜,字允中,江寧人。有俊才,工詩,辟為中書省博士。婺州平,調浙東分省,兩使方國珍,咸稱旨。太祖征陳友諒,儒臣惟劉基與煜侍。鄱陽戰勝,太祖所與草檄賦詩者,煜其一也。洪武元年使總制浙東諸府,與高見賢、楊憲、凌說四人以伺察搏擊為事,後俱以不良死。

郭景祥,濠人。與鳳陽李夢庚皆從渡江,典文書,佐謀議,分任行中書省左右司郎中。既同調浙東分省,尋復同入為大都督府參軍。景祥性諒直,博涉書史,遇事敢言,太祖親信之。嘗曰:「景祥文吏,而有折衝禦侮才,能盡忠於我,可大任也。」先是,克滁州、太平、溧陽。以城郭不完,輒命景祥董治之。既而和州守臣言,州城久廢,命景祥相度,即故址城之,九旬而工畢。太祖以為能,授和州總制。景祥益治城隍樓櫓,廣屯田,練士卒,威望肅然。和遂為重鎮。璽書褒勞。仕終浙江行省參政。

謝再興之守諸全也,部將私販易吳境。太祖怒殺部將,召諭再興,命夢庚往諸全總制軍事。再興還鎮,忿夢庚出己上,遂叛。執夢庚降於吳,夢庚死之。其時,參佐行省者,又有毛騏、王濂。

濂,字習古,定遠人,李善長婦兄也。才嗜學,事親孝。初從汝、潁賊,太祖克集慶,乃渡江來歸。善長為言,得召見,除執法官,讞獄平允。遷中書省員外郎,出為浙江按察僉事,治行著聞。大風晝晦,濂應詔言民瘼,請緩征。太祖納之。洪武三年卒。帝謂善長曰:「濂有王佐才,今死,朕失一臂。」後善長坐事,帝歎曰:「使王濂在,必不至是。」

騏,字國祥,與濂同里。太祖自濠引兵趨定遠,騏扶縣令出降。太祖喜,留與飲食,籌兵事,悉當意。取滁州,擢總管府經歷。典倉廩,兼掌晨昏曆,稽將帥之失伍者。從渡江,擢兵省郎中。是時太祖左右惟善長及騏,文書機密,皆兩人協贊。尋授參議官。征婺州,命權理中書省事,委以心膂。俄病卒,太祖親為文哭之,臨視其葬。

子驤,管軍千戶,積功擢親軍指揮僉事。從定中原,進指揮使。滕州段士雄反,驤討平之。捕倭浙東,斬獲多,擢都督僉事,見親任,嘗掌錦衣衛事,典詔獄。後坐胡惟庸黨死。

楊元杲、阮弘道,皆滁人,家世皆儒者。從渡江,同為行省左右司員外郎,與陶安等更番掌行機宜文字。元杲以郎中擢理軍儲於金華,而弘道亦於是歲以郎中從大都督文正守南昌,皆有功。二人皆於太祖最故,又皆儒雅,嗜文學,練達政體,而元杲知慮尤周密。帝嘗曰:「文臣從渡江,掌簿書文字,勤勞十餘年,無如楊元杲、阮弘道、李夢庚、侯元善、樊景昭者。」其後,元杲歷應天府尹,弘道歷福建、江西行省參政,皆卒官。

元杲子賁,博學強記,以詞翰知名,薦授大名知縣,仕至周府紀善。

元善,全椒人,歷官參知政事,與樊景昭俱無所表見。

又汪河者,舒城人。嘗師餘闕,以文章名。從渡江,為行中書省掾,數陳時務。太祖高其才,進大都督府都事。使察罕,議論稱旨。後奉命偕錢楨至河南,報擴廓聘,為所留。太祖前後七致擴廓書,終不報。洪武元年,大軍下河、洛,擴廓走定西,河始得歸,被拘凡六年。帝甚嘉之,進吏部侍郎,備陳西征方略。二年改御史臺侍御史。九年,拜晉王左相,親御便殿諭遣之。居數歲,卒於官。

孔克仁,句容人。由行省都事進郎中。嘗偕宋濂侍太祖,太祖數與論天下形勢及前代興亡事。陳友諒既滅,太祖志圖中原,謂克仁曰:「元運既隳,豪傑互爭,其釁可乘。吾欲督兩淮、江南諸郡之民,及時耕種,加以訓練。兵農兼資,進取退守。仍於兩淮間饋運可通之處,儲糧以俟。兵食既足,中原可圖。卿以為何如?」克仁對曰:「積糧訓兵,觀釁待時,此長策也。」當是時,江左兵勢日盛,太祖以漢高自期,嘗謂克仁曰:「秦政暴虐,漢高帝起布衣,以寬大馭群雄,遂為天下主。今群雄蜂起,皆不知修法度以明軍政,此其所以無成也。」因感歎久之。又曰:「天下用兵,河北有孛羅帖木兒,河南有擴廓帖木兒,關中有李思齊、張良弼。然有兵而無紀律者河北也;稍有紀律而兵不振者河南也;道途不通、饋餉不繼者關中也。江南則惟我與張士誠耳。士誠多奸謀,尚間諜,御眾無紀律。我以數十萬眾,修軍政,任將帥,相時而動,其勢有不足平者。」克仁頓首曰:「主上神武,當定天下於一矣。」

嘗閱《漢書》,濂與克仁侍。太祖曰:「漢治道不純者何?」克仁對曰:「王霸雜故也。」太祖曰:「誰執其咎?」克仁曰:「責在高祖。」太祖曰:「高祖創業,遭秦滅學,民憔悴甫蘇,禮樂之事固所未講。孝文為令主,正當制禮作樂,以復三代之舊。乃逡巡未遑,使漢業終於如是。帝王之道,貴不違時。三代之王有其時而能為之,漢文有其時而不為,周世宗則無其時而為之者也。」又嘗問克仁:「漢高起徒步為萬乘主,所操何道?」克仁對曰:「知人善任使。」太祖曰:「項羽南面稱孤,仁義不施,而自矜功伐。高祖知其然,承以柔遜,濟以寬仁,卒以勝之。今豪傑非一,我守江左,任賢撫民,以觀天下之變。若徒與角力,則猝難定也。」及徐達等下淮東、西,又謂克仁曰:「壬辰之亂,生民塗炭。中原諸將,孛羅擁兵犯闕,亂倫干紀,行已夷滅。擴廓挾太子以稱戈,急私仇,無敵愾之志。思齊輩碌碌,竊據一方,民受其害。士誠外假元名,反覆兩端。明玉珍父子據蜀僭號,喜於自用而無遠謀。觀其所為,皆不能有成。予揆天時,審人事,有可定之機。今師西出襄、樊,東踰淮、泗,首尾相應,擊之必勝。大事可成,天下不難定。既定之後,生息猶難,方勞思慮耳。」

克仁侍帷幄最久,故獲聞太祖謀略居多。洪武二年四月,命克仁等授諸子經,功臣子弟亦令入學。已,出知江州,入為參議,坐事死。

贊曰:太祖起布衣,經營天下。渡江以來,規模宏遠,聲教風馳。雖曰天授,抑亦左右丞弼多國士之助歟。陳遇見禮不下劉基,而超然利祿之外。葉兌於天下大計,籌之審矣,亦能抗節肥遯,其高致均非人所易及。孔克仁無可稱述,以太祖之雄謀大略具其事中,故敘列於篇。


\end{pinyinscope}