\article{列傳第一百四十二}

\begin{pinyinscope}
喬允升易應昌等曹于汴孫居相弟鼎相曹珖陳於廷鄭三俊李日宣張瑋金光辰

喬允升,字吉甫,洛陽人。萬曆二十年進士。除太谷知縣。以治行高等征授御史。歷按宣、大、山西、畿輔,並著風采。

三十九年,大計京官。允升協理河南道,力鋤匪類。而主事秦聚奎、給事中朱一桂咸為被察者訟冤。察疏猶未下,允升慮帝意動搖,三疏別白其故,且劾吏部侍郎蕭雲舉佐察行私,事乃獲竣,雲舉亦引去。尋遷順天府丞,進府尹。齊、楚、浙三黨用事,移疾歸。

天啟初,起歷刑部左、右侍郎。三年,進尚書。魏忠賢逐吏部尚書趙南星,廷推允升代。忠賢以允升為南星黨,并逐主議者,允升復移疾歸。既而給事中薛國觀劾允升主謀邪黨,詔落職閒住。

崇禎初,召拜故官。時訟獄益繁,帝一切用重典。允升執法不撓,多所平反。先是,錢謙益典試浙江。有奸人金保元、徐時敏偽作關節,授舉子錢千秋,千秋故有文,獲薦,覺保元、時敏詐,與之哄。事傳京師,為部、科磨勘者所發。謙益大駭,詰知二奸所為,疏劾之,并千秋俱下吏。罪當戍,二奸瘐死,千秋更赦釋還。事已七年矣,溫體仁以枚卜不與,疑謙益主之,復發其事。詔逮千秋再訊。帝深疑廷臣結黨,蓄怒以待,而體仁又密伺於旁,廷臣相顧惕息。允升乃會都御史曹于汴、大理卿康新民等讞鞫者再,千秋受拷無異詞,允升等具以聞。帝不悅,命覆勘。體仁慮謙益事白,己且獲譴,再疏劾法官六欺,且言獄詞盡出謙益手。允升憤,求去。帝雖慰留,卒如體仁言,奪謙益官閑住。千秋荷校死。

二年冬,我大清兵薄都城,獄囚劉仲金等百七十人破械出,欲逾城,被獲。帝震怒,下允升及左侍郎胡世賞、提牢主事敖繼榮獄,欲置之死。中書沈自植乘間摭劾允升他罪,章並下按問。副都御史掌院事易應昌以允升等無死罪,執奏再三,帝益怒,并下應昌獄,鐫僉都御史高弘圖、大理寺卿金世俊級,奪少卿周邦基以下俸,令再讞。弘圖等乃坐允升絞,而微言其年老可念。帝謂允升法當死,特高年篤疾減死,與繼榮俱戍邊,世賞贖杖為民。尚書胡應台等上應昌罪,帝以為輕。杖郎中徐元嘏於廷,鐫應台秩視事,應昌論死。四年四月,久旱求言,多請緩刑,乃免應昌及工部尚書張鳳翔、御史李長春、給事中杜齊芳、都督李如楨死,遣戍邊衛。允升赴戍所,未幾死。允升端方廉直,揚歷中外,具有聲績,以詿誤獲重譴,天下惜之。

易應昌,字瑞芝,臨川人。萬曆四十一年進士。熹宗時,由御史累遷大理少卿。逆黨劾為東林,削籍。崇禎二年,起左僉都御史,進左副都御史,與曹于汴持史褷、高捷起官事,為時所重,至是獲罪。福王時,召復故官,遷工部右侍郎。國變後卒。

帝在位十七年,刑部易尚書十七人。薛貞以奄黨抵死,蘇茂相半歲而罷,王在晉,未任改兵部去,允升遣戍,韓繼思坐議獄除名,胡應台獨得善去,馮英被劾遣戍,鄭三俊坐議獄逮繫,劉之鳳坐議獄論絞,瘐死獄中,甄淑坐納賄下詔獄,改繫刑部,瘐死,李覺斯坐議獄削籍去,劉澤深卒於位,鄭三俊再為尚書,改吏部,范景文未任,改工部,徐石麒坐議獄落職閒住,胡應台再召不赴,繼其後者張忻,賊陷京師,與子庶吉士端並降。

曹于汴,字自梁,安邑人。萬歷十九年舉鄉試第一。明年成進士,授淮安推官。以治行高第,授吏科給事中。疏劾兩京兵部尚書田樂、邢玠及雲南巡撫陳用賓,樂、玠遂引去。吏部郎趙邦清被誣,於汴疏雪之。謁告歸,僦屋以居,不蔽風日。

起歷刑科左、右給事中。朝房災,請急補曠官,修廢政。遼左有警,朝議增兵,于汴言:「國家三歲遣使者閱邊,盛獎邊臣功伐,蟒衣金幣之賜,官秩之增,未嘗或靳。今廢防至此,宜重加按問。邊道超擢,當於秩滿時閱實其績,毋徒循資俸,坐取建牙開府。」進吏科都給事中。給事中胡嘉棟發中官陳永壽兄弟奸,永壽反訐嘉棟。于汴極論永壽罪。故事,章疏入會極門,中官直達之御前,至是必啟視然後進御。于汴謂乖祖制,洩事機,力請禁之。三十八年典外察,去留悉當。明年,典京察,屏湯賓尹、劉國縉等,而以年例出王紹徽、喬慶甲於外。其黨群起力攻,于汴持之堅,卒不能奪。以久次擢太常少卿,疏寢不下,請告又不報,候命歲餘,移疾歸。

光宗立,始以太常少卿召。至則改大理少卿,遷左僉都御史,佐趙南星京察。事竣,進左副都御史。天啟三年秋,吏部缺右侍郎,廷推馮從吾,以於汴副,中旨特用于汴。於汴以從吾名位先己,義不可越,四辭不得,遂引疾歸。明年,起南京右都御史,辭不拜。時紹徽、應甲附魏忠賢得志,必欲害于汴,屬其黨石三畏以東林領袖劾之,遂削奪。

崇禎元年,召拜左都御史。振舉憲規,約敕僚吏,臺中肅然。明年京察,力汰匪類,忠賢餘黨幾盡,仕路為清。溫體仁訐錢謙益,下錢千秋法司,訊不得實,體仁以于汴謙益座主也,並訐之。于汴亦發體仁欺罔狀。帝終信體仁,謙益竟獲罪。

先是,詔定逆案。於汴與大學士韓爌、李標、錢龍錫,刑部尚書喬允升平心參決,不為已甚,小人猶惡之。故御史高捷、史褷素憸邪,為清議所擯,吏部尚書王永光力薦之。故事,御史起官,必都察院咨取,于汴惡其人,久弗咨。永光憤,再疏力爭。已得請,于汴猶以故事持之,兩人遂投牒自乞,于汴益惡之,卒持不予。兩人竟以部疏起官,遂日夜謀傾于汴。

中書原抱奇者,賈人子也,嘗誣劾大學士爌。至是再劾爌及于汴並及尚書孫居相、侍郎程啟南、府丞魏光緒,目為「西黨」,請皆放黜,以五人籍山西也。帝絀抱奇言不聽。而工部主事陸澄源復劾於汴朋奸六罪。帝雖謫澄源,于汴卒謝事去。及辭朝,以敦大進規。七年卒,年七十七。贈太子太保。

于汴篤志正學,操履粹白。立朝,正色不阿,崇獎名教,有古大臣風。

孫居相,字伯輔,沁水人。萬歷二十年進士。除恩縣知縣。征授南京御史。負氣敢言。嘗疏陳時政,謂:「今內自宰執,外至郡守縣令,無一人得盡其職。政事日廢,治道日乖,天變人怨,究且瓦解土崩。縱珠玉金寶亙地彌天,何救危亂!」帝不省。誠意伯劉世延屢犯重辟,廢為庶人,錮原籍。不奉詔,久居南京,益不法,妄言星變,將勒兵赴闕。居相疏發其奸,并及南京勳臣子弟暴橫狀。得旨下世延吏,安遠、東寧、忻城諸侯伯子弟悉按問,強暴為戢。稅使楊榮激變雲南,守太和山中官黃勛嗾道士毆辱知府,居相皆極論其罪。

時中外多缺官,居相兼攝七差,署諸道印,事皆辦治。大學士沈一貫數被人言,居相力詆其奸貪植黨,一貫乃去,居相亦奪祿一年。連遭內外艱。服闋,起官,出巡漕運,還發湯賓尹、韓敬科場事。廷議當褫官,其黨為營護,旨下法司覆勘。居相復發敬通賄狀,敬遂不振。故事,御史年例外轉,吏部、都察院協議。王時熙、魏雲中之去,都御史孫瑋不與聞。居相再疏劾尚書趙煥,煥引退。及鄭繼之代煥,復以私意出宋槃、潘之祥於外,居相亦據法力爭。吏部侍郎方從哲由中旨起官,中書張光房等五人以持議不合時貴,擯不與科道選,居相並抗章論列。

當是時,朋黨勢成,言路不肖者率附吏部,以驅除異己,勢張甚。居相挺身與抗,氣不少沮。於是過庭訓、唐世濟、李徵儀、劉光復、趙興邦、周永春、姚宗文、吳亮嗣、汪有功、王萬祚輩群起為難,居相連疏搘拄,諸人迄不能害。至四十五年,亦以年例出居相江西參政,引疾不就。

天啟改元,起光祿少卿。改太僕,擢右僉都御史,巡撫陜西。四年春,召拜兵部右侍郎。其冬,魏忠賢盜柄,復引疾歸。無何,給事中陳序謂居相出趙南星門,與楊漣交好,序同官虞廷陛又劾居相力薦李三才,遙結史記事,遂削奪。

崇禎元年,起戶部右侍郎,專督鼓鑄。尋改吏部,進左侍郎,以戶部尚書總督倉場。轉漕多雇民舟,民憊甚,以居相言獲蘇。高平知縣喬淳貪虐,為給事中楊時化所劾,坐贓二萬有奇。淳家京師,有奧援,乞移法司覆訊,且訐時化請囑致隙。時化方憂居,通書居相,報書有「國事日非,邪氛益惡」語,為偵事者所得,聞於朝。帝大怒,下居相獄,謫戍邊。七年,卒於戍所。

弟鼎相,歷吏部郎中、副都御史,巡撫湖廣,亦有名東林中。

曹珖,字用韋,益都人。萬歷二十九年進士。授戶部主事,督皇城四門。倉衛軍貸群璫子錢,償以月餉,軍不支餉者三年。及餉期,群璫抱券至,珖命減息,璫大嘩。珖曰:「并私券奏聞,聽上處分耳。」群璫請如命,軍困稍蘇。以憂去。

起補兵部武選主事,歷職方郎中。大璫私人求大帥,珖不可。東廠太監盧受疏申職掌,珖亦請敕受約束部卒,毋陷良民。稍遷河東參政,引疾歸。久之,起南京太常少卿。光宗驟崩,馳疏言:「先帝春秋鼎盛,奄棄群臣,道路咸知奸黨陰謀,醫藥雜進,以至於此。天下之弒逆,有毒而非鴆、戕而非刃者,此與先年梃擊,同一奸宄。乞明詔輔臣,直窮奸狀,以雪先帝之仇。」報聞。

天啟初,敘職方時邊功,加光祿卿,進太常大理卿。魏忠賢亂政,大獄紛起,珖請告歸。尋為給事中潘士聞所劾,落職閒住。御史盧承欽歷攻東林,詆珖狎主邪盟,遂削奪。

崇禎元年,起戶部右侍郎,督錢法,尋遷左侍郎。三年,拜工部尚書。珖初名珍,避仁宗諱,始改名。五年,陵工成,加太子少保。桂王重建府第,議加江西、河南、山東、山西田賦十二萬有奇;浙江逋織造銀十餘萬,巡撫陸完學請編入正額。珖皆持不可。

中官張彞憲總理戶、工兩部事,議設座於部堂,珖不可。右侍郎高弘圖履任,彞憲欲共設公座,珖與弘圖約,比彞憲至,皆曰「事竣矣」,撤座去,彞憲怏怏。及主事金鉉、馮元颺交疏劾彞憲,彞憲疑出珖,日捃摭其隙。會山永巡撫劉宇烈請料價萬五千兩、鉛五萬斤,工部無給銀例,與鉛之半,宇烈怒,奏鉛皆濫惡。彞憲取粗鉛進曰「庫鉛盡然」,欲以罪珖。嚴旨盡熔庫鉛,司官中毒死者三人,內外官多獲罪。彞憲乃糾巡視科道許國榮等十一人,珖疏救,忤旨詰責。彞憲又指閘工冒破齮齕之,珖累疏乞骸骨歸,五月得請。屢薦不起。家居十四年卒。

陳于廷,字孟諤,宜興人。萬曆二十三年進士。歷知光山、唐山、秀水三縣,徵授御史。甫拜命,即論救給事中汪若霖,詆大學士朱賡甚力,坐奪俸一年。頃之,劾職方郎中申用懋、趙拱極、黃克謙為宰相私人,不宜處要地,又劾賡及王錫爵當斥。已,言諭德顧天颭素干清議,不宜久玷詞林。語皆峻切。視鹺河東,劾稅使張忠撓鹽政。正陽門災,極陳時政闕失。父喪歸。服除,起按江西。時稅務已屬有司,而中官潘相欲親督湖口稅,于廷劾其背旨虐民。淮府庶子常洪作奸,論置之法。改按山東。

光宗立,擢太僕少卿,徙太常。議「紅丸」事,極言崔文昇、李可灼當斬。尚書王紀被斥,特疏申救。再進大理卿、戶部右侍郎,改吏部,進左侍郎。尚書趙南星既逐,于廷署事。大學士魏廣微傳魏忠賢意,欲用其私人代南星,且許擢于廷總憲,于廷不可,以喬允升、馮從吾、汪應蛟名上。忠賢大怒,謂所推仍南星遺黨,矯旨切責,并楊漣、左光斗盡斥為民。文選郎張可前、御史袁化中、房可壯亦坐貶黜。自是清流盡逐,小人日用事矣。

崇禎初,起南京右都御史。與鄭三俊典京察,盡去諸不肖者。南御史差竣,便聽北考,於廷請先考於南,報可。召拜左都御史。以巡方責重,列上糾大吏、薦人才、修荒政、核屯鹽、禁耗羨、清獄囚、訪奸豪、弭寇盜八事,請於回道日核實課功。優詔褒納。給事中馬思理,御史高倬、餘文縉坐事下吏,並抗疏救之。秩滿,加太子少保。三疏乞休,不允。

兩浙巡鹽御史祝徽、廣西巡按御史畢佐周並擅撻指揮,非故事。事聞,帝方念疆場多故,欲倚武臣,旨下參核。于廷等言:「軍官起世胄,率不循法度,概列彈章,將不勝擾,故小過薄責以懲。凡御史在外者盡然,不自二臣始。」帝以指揮秩崇,非御史得杖,令會兵部稽典制以聞,典制實無杖指揮事,乃引巡撫敕書提問四品武職語以對。帝以比擬不倫,責令再核,于廷等終右御史,所援引悉不當帝意。疏三上三卻,竟削籍歸。家居二年卒。福王時,贈少保。

于廷端亮有守。周延儒當國,於廷其里人,無所附麗。與溫體仁不合,故卒獲重譴去。

鄭三俊,字用章,池州建德人。萬曆二十六年進士。授元氏知縣。累遷南京禮部郎中、歸德知府、福建提學副使。家居七年,起故官,督浙江糧儲。

天啟初,召為光祿少卿,改太常。未上,陳中官侵冒六事。時魏忠賢、客氏離間后妃,希得見帝,而三俊疏有「篤厚三宮,妖冶不列於御」語。忠賢遣二豎至閣中,摘「妖冶」語,令重其罪,閣臣力爭,而擬旨則以先朝故事為辭。三俊復疏言:「近日麋爛荼毒,無踰中璫,閣臣悉指為故事。古人言奄豎聞名,非國之福。今聞名者已有人,內連外結,恃閣臣彈壓抑損之,而閣臣輒阿諛自溺其職,可為寒心。」忠賢益怒,以語侵內閣,留中不下。擢左僉都御史,疏陳兵食大計,規切內外諸司。吏部郎中徐大相言事被謫,抗疏救之。

四年正月,遷左副都御史。戶部右侍郎楊漣劾忠賢,三俊亦上疏極論。尋署倉場事。太倉無一歲蓄,三俊奏行足儲數事。忠賢盡逐漣等,三俊遂引疾去。明年,忠賢黨張訥請毀天下書院,劾三俊與鄒元標、馮從吾、孫慎行、餘懋衡合污同流,褫職閒住。

崇禎元年,起南京戶部尚書兼掌吏部事。南京諸僚多忠賢遺黨,是年京察,三俊澄汰一空。京師被兵,大臣大獲譴。明年春,三俊以建儲入賀,力言:「皇上憂勞少過,人情鬱結未宣。百職庶司,救過不贍,上下睽孤,足為隱慮。願保聖躬以保天下,收人心以收封疆。」帝褒納之。南糧歲額八十二萬七千有奇,積逋至數百萬,而兵部又增兵不已。三俊初至,倉庫不足一月餉。三俊力祛宿弊,糾有司尤怠玩者數人,屢與兵部爭虛冒,久之,士得宿飽。萬曆時,稅使四出,蕪湖始設關,歲征稅六七萬,泰昌時已停。至是,度支益絀,科臣解學龍請增天下關稅,南京宣課司亦增二萬。三俊以為病民,請減其半,以其半征之蕪湖坐賈,戶部遂派蕪湖三萬,復設關征商。三俊請罷征,併於工部分司計舟輸課,不稅貨物,皆不從,遂為永制。蕪湖、淮安、杭州三關皆隸南戶部,所遣司官李友蘭、霍化鵬、任人叔皆貪,三俊悉劾罷之。

居七年,就移吏部。八年正月,復當京察,斥罷七十八人,時服其公。旋上議官評、杜請屬、慎差委三事,帝皆採納。流寇大擾江北,南都震動,三俊數陳防禦策。禮部侍郎陳子壯下獄,抗疏救之。

考績入都,留為刑部尚書,加太子少保。帝以陰陽愆和,命司禮中官錄囚,流徒以下皆減等。三俊以文武諸臣詿誤久繫者眾,請令出外候讞。因論告訐株蔓之弊,乞敕「內外諸臣行惻隱實政。內而五城訊鞫,非重辟不必參送法司;外而撫按提追,非真犯不必盡解京師;刑曹決斷,以十日為期。」帝皆從之。代州知州郭正中因天變,請舉寒審之典,帝命考故事。三俊稽歷朝寶訓,得祖宗冬月錄囚數事,備列上奏,寢不行。前尚書馮英坐事遣戍,其母年九十有一,三俊乞釋還侍養,不許。

初,戶部尚書侯恂坐屯豆事下獄,帝欲重譴之。三俊屢讞上,不稱旨。讒者謂恂與三俊皆東林,曲法縱舍。工部錢局有盜穴其垣,命按主者罪,三俊亦擬輕典。帝大怒,褫其官下吏。應天府丞徐石麒適在京,上疏力救,忤旨切責。帝御經筵,講官黃景昉稱三俊至清,又偕黃道周各疏救。帝不納,切責三俊欺罔。以無贓私,令出獄候訊。宣大總督盧象昇復救之,大學士孔貞運等復以為言,乃許配贖。

十五年正月,召復故官。會吏部尚書李日宣得罪,即命三俊代之。時值考選,外吏多假繕城、墾荒名,減俸行取,都御史劉宗周疏論之。諸人乃夤緣周延儒,囑兵部尚書張國維以知兵薦,帝即欲召對親擢。三俊言:「考選者部、院事,天子且不得專,況樞部乎?乞先考定,乃請聖裁。」帝不悅,召三俊責之,對不屈。宗周復言:「三俊欲俟部、院考後,第其優劣純疵,恭請欽定。若但以奏對取人,安能得真品?」帝不從,由是倖進者眾。帝下詔求賢,三俊舉李邦華、劉宗周自代,且薦黃道周、史可法、馮元颺、陳士奇四人。姜埰、熊開元言事下獄,及宗周獲嚴譴,三俊皆懇救。先後奏罷不職司官數人,銓曹悉廩廩。大僚缺官,三俊數引薦,賢士之廢斥者多復用。刑部尚書徐石麒獲罪,率同官合疏乞留。

三俊為人端嚴清亮,正色立朝。惟引吳昌時為屬,頗為世詬病。時文選缺郎中,儀制郎中吳昌時欲得之。首輔周延儒力薦於帝,且以囑三俊,他輔臣及言官亦多稱其賢,三俊遂請調補。帝特召問,三俊復徇眾意以對。帝頷之,明日即命下。以他部調選郎,前此未有也。帝惡言官不職,欲多汰之,嘗以語三俊,三俊與昌時謀出給事四人、御史六人於外。給事、御史大嘩,謂昌時紊制弄權,連章力攻,并詆三俊。三俊懇乞休致,詔許乘傳歸。國變後,家居十餘年乃卒。

李日宣,字晦伯,吉水人。萬曆四十一年進士。授中書舍人,擢御史。

天啟元年,遼陽破。請帝時召大僚,面決庶政。尋請宥侯震暘以開言路,厚中宮以肅名分。忤旨,切責。已,又薦丁元薦、鄒維璉、麻僖等十餘人,乞召還朱欽相、劉廷宣等,帝以濫薦逐臣,停俸三月。旋出理河東鹽政。還朝,以族父邦華佐兵部,引嫌歸。五年七月,逆黨倪文煥劾邦華、日宣為東林邪黨,遂削籍。

莊烈帝即位,復故官,以邦華在朝,久不出。崇禎三年,起故官,巡按河南。還朝,掌河南道事。中官王坤訐大學士周延儒,日宣率同官言:「內臣監兵,不宜侵輔臣,且插款中疑,邊情多故,坤責亦不可逭。」報聞。遷大理丞,屢進太常卿。九年冬,擢兵部右侍郎,鎮守昌平。久之,進左侍郎,協理戎政。尋敘護陵功,加兵部尚書。十三年九月,擢吏部尚書。

十五年五月,會推閣臣,日宣等以蔣德璟、黃景昉、姜曰廣、王錫袞、倪元璐、楊汝成、楊觀光、李紹賢、鄭三俊、劉宗周、吳甡、惠世揚、王道直名上。帝令再推數人,而副都御史房可壯、工部右侍郎宋玫、大理寺卿張三謨與焉。大僚不獲推者,為流言入內,且創二十四氣之說,帝深惑之。踰月,召日宣及與推諸臣入中左門,偕輔臣賜食。已,出御中極殿,令諸臣奏對。玫陳九邊形勢甚辯,帝惡其干進,叱之,乃命德璟、景昉、甡入閣,而以徇情濫舉責日宣等回奏。奏上,帝怒不解,復御中左門,太子及定、永二王侍。帝召日宣,聲甚厲。次召吏科都給事中章正宸、河南道御史張煊,及玫、可壯、三謨,詰其妄舉。日宣奏辯。帝曰:「汝嘗言秉公執法,今何事不私?」正宸奏:「日宣多游移,臣等常劾之。然推舉事,實無所徇。」日宣復為玫等三人解。帝命錦衣官提下日宣等六人,並褫冠帶就執。時帝怒甚,侍臣皆股栗失色。德璟、景昉、甡叩頭辭新命,因言:「臣等並在會推中。若諸臣有罪,臣等豈能安。」大學士周延儒等亦乞優容。帝皆不許,遂下刑部。廷臣交章申救,不納,帝疑其未就獄,責刑部臣剋期三日定讞。侍郎惠世揚、徐石麒擬予輕比,帝大怒,革世揚職,鐫石麒二秩,郎中以下罪有差。御史王漢言:「枚卜一案,日宣等無私。陛下懷疑,重其罪,刑官莫知所執。」不聽。獄上,日宣、正宸、煊戍邊,玫、可壯、三謨削籍。久之,赦還,卒。

張瑋,字席之,武進人。少孤貧,取糠秕自給,不輕受人一飯,為同里薛敷教所知。講學東林書院,師孫慎行。其學以慎獨研幾為宗。

萬曆四十年,舉應天鄉試第一。越七年,成進士,授戶部主事。調兵部職方,歷郎中,出為廣東提學僉事。粵俗奢麗,督學至,宮室供張輿馬餼牽之奉甲他省,象犀文石,名花珠具,磊砢璀璨,瑋悉屏去弗視也。大吏建魏忠賢祠,乞上梁文於瑋,瑋即日引去。瑋廉,歸而布袍草履,授徒於家。

莊烈帝即位,起江西參議,歷福建、山東副使。大學士吳宗達謂瑋難進而易退,言之吏部,召為尚寶卿,進太僕少卿。坐事調南京大理丞,引疾去。久之,起應天府丞。是歲,四方大旱,瑋以軍食可虞,奏請:「禁江西、湖廣遏糴,而令應天、常、鎮、淮、揚五郡折輸漕糧銀,赴彼易米,則小民免催科之苦,太倉無顆粒之虧。他十庫所收銅、錫、顏料、皮布,非州縣土產者,悉解折色,且盡改民解為官解,以救民湯火。」所司多議行。

遷南京光祿卿,召入為右僉都御史,遷左副都御史。時劉宗周、金光辰並總憲紀,瑋乃上《風勵臺班疏》曰:「懲往正以監來。今極貪則原任巡按蘇松御史王志舉,極廉則原任南京試御史成勇。勇與臣曾不相知,家居聞勇被逮,士民泣送者萬輩,百里不休。後入南都,始知勇在臺不濫聽一辭,不輕贖一鍰,不受屬吏一蔬一果;傑紳悍吏為民害者,不少假借;委曲開導民以孝弟。臣離南中,輒扳轅願借成御史,惠我南人。雖前奉嚴譴,宜召為諸御史勸。」疏上,一時稱快。詔下志舉法司逮治,成勇敘用。

瑋旋以病謝歸,未幾卒。福王時,贈左都御史,謚清惠。

金光辰,字居垣,全椒人。崇禎元年進士。授行人。擢御史,巡視西城。內使周二殺人,牒司禮監捕之,其人方直御前,叩頭乞哀。帝曰:「此國家法,朕不得私。」卒抵罪。出按河南,條奏至三百餘章,彈劾不避權勢。九年,還朝。京師戒嚴,光辰分守東直門,劾兵部尚書張鳳翼三不可解,一大可憂。帝以鳳翼方在行間,寢其奏。

時帝久罷內遣,然以邊警,諸臣類萎腇不任,仍分遣中官盧維寧等總監通、津、臨、德等處兵馬糧餉,而意頗諱言之。光辰疏請罷遣,帝怒,召對平臺。風雨驟至,侍臣立雨中,至以袖障溜。久之,帝召光辰責之。光辰對曰:「皇上以文武諸臣無實心任事,委任內臣。臣愚以任內臣,諸臣益弛卸不任。」帝大怒,聲色俱厲,將重譴光辰,而迅雷直震御座,風雨聲大作。光辰因言:「臣往在河南,見皇上撤內臣而喜。」語未終,帝沉吟,即云「汝言毋復爾」,然意亦稍解。人謂光辰有天幸云。時張元佐以兵部右侍郎出守昌平,同時內臣提督天壽山者即日往。帝顧閣臣曰:「內臣即日往,侍臣三日未出,朕之用內臣過耶?」翼日有詔,光辰鐫三級調外。

久之,由浙江按察司照磨召為大理寺正,進太僕丞。十三年五月,復偕諸大臣召對平臺,咨以御邊、救荒、安民之策。光辰班最後,時已夜,光辰獨對燭影中,娓娓數百言,帝為聳然聽。明日諭諸臣各繕疏以進。尋移尚寶丞。陳罷練總、換授、私派、僉報數事,報聞。歷光祿少卿、左通政。十五年五月,復偕諸臣召對德政殿,備陳賊形勢。帝悅,擢左僉都御史。無何,以救劉宗周,仍鐫三級調外,事具《宗周傳》。明年丁父憂。福王時,起故官。未赴,國變,家居二十餘年卒。

贊曰:明自神宗而後,士大夫峻門戶而重意氣。其賢者敦厲名檢,居官有所執爭,即清議翕然歸之。雖其材識不遠,耳目所熟習,不能不囿於風會,抑亦一時之良也。遭時孔棘,至救過不暇,顧安得責以挽回乾濟之業哉?


\end{pinyinscope}