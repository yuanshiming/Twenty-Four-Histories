\article{列傳第四十九}

\begin{pinyinscope}
周新李昌祺蕭省身陳士啟應履平林碩況鐘朱勝陳本深羅以禮莫愚趙泰彭勖孫鼎夏時黃潤玉楊瓚王懋葉錫趙亮劉實陳選夏寅陳壯張昺宋端儀

周新,南海人。初名志新,字日新。成祖常獨呼「新」,遂為名,因以志新字。洪武中以諸生貢入太學。授大理寺評事,以善決獄稱。

成祖即位,改監察御史。敢言,多所彈劾。貴戚震懼,目為「冷面寒鐵」。京師中至以其名怖小兒,輒皆奔匿。巡按福建,奏請都司衛所不得凌府州縣,府衛官相見均禮,武人為之戢。改按北京。時令吏民罪徒流者耕北京閒田,監禁詳擬,往復待報,多瘐死。新請從北京行部或巡按詳允就遣,以免淹滯。從之。且命畿內罪人應決者許收贖。帝知新,所奏無不允。

還朝,即擢雲南按察使。未赴,改浙江。冤民繫久,聞新至,喜曰:「我得生矣。」至果雪之。初,新入境,群蚋迎馬頭,跡得死人榛中,身繫小木印。新驗印,知死者故布商。密令廣市布,視印文合者捕鞫之,盡獲諸盜。一日,視事,旋風吹葉墜案前,葉異他樹。詢左右,獨一僧寺有之。寺去城遠,新意僧殺人。發樹,果見婦人屍。鞫實,磔僧。一商暮歸,恐遇劫,藏金叢祠石下,歸以語其妻。旦往求金不得,訴於新。新召商妻訊之,果商妻有所私。商驟歸,所私尚匿妻所,聞商語,夜取之。妻與所私皆論死。其他發奸摘伏,皆此類也。

新微服行部,忤縣令。令欲拷治之,聞廉使且至,繫之獄。新從獄中詢諸囚,得令貪污狀。告獄吏曰:「我按察使也。」令驚謝罪,劾罷之。永樂十年,浙西大水,通政趙居任匿不以聞,新奏之。夏原吉為居任解。帝命覆視,得蠲振如新言。嘉興賊倪弘三劫旁郡,黨數千人,累敗官軍。新督兵捕之,列木柵諸港汊。賊陸走,追躡之桃源,縶以獻。當是時,周廉使名聞天下。

錦衣衛指揮紀綱使千戶緝事浙江,攫賄作威福。新欲按治之,遁去。頃之,新齎文冊入京,遇千戶涿州,捕繫州獄。脫走訴於綱,綱誣奏新罪。帝怒,命逮新。旗校皆錦衣私人,在道榜掠無完膚。既至,伏陛前抗聲曰:「陛下詔按察司行事,與都察院同。臣奉詔擒奸惡,奈何罪臣?」帝愈怒,命戮之。臨刑大呼曰:「生為直臣,死當作直鬼!」竟殺之。

他日,帝悔,問侍臣曰:「周新何許人?」對曰:「南海。」帝嘆曰:「嶺外乃有此人,枉殺之矣!」後帝若見人緋衣立日中,曰「臣周新已為神,為陛下治奸貪吏」云。後紀綱以罪誅,事益白。

妻有節操。新未遇時,縫紉自給。及貴,偶赴同官妻內宴,荊布如田家婦。諸婦慚,盡易其衣飾。新死無子。妻歸,貧甚。廣東巡撫楊信民曰:「周志新當代第一人,可使其夫人終日餒耶?」時時賙給之。妻死,浙人仕廣東者皆會葬。

李昌祺,名禎,以字行,廬陵人。永樂二年進士。選庶吉士。預修《永樂大典》,僻書疑事,人多就質。擢禮部郎中,遷廣西左布政使。坐事謫役,尋宥還。洪熙元年,起故官河南。與右布政使蕭省身繩豪猾,去貪殘,疏滯舉廢,救災恤貧,數月政化大行。憂歸,宣宗已命侍郎魏源代。而是時河南大旱,廷臣以昌祺廉潔寬厚,河南民懷之,請起昌祺。命奪喪赴官,撫恤甚至。正統改元,上書言三事,皆報可。四年致仕。家居二十餘年,屏跡不入公府,故廬裁蔽風雨,伏臘不充。景泰二年卒。

蕭省身,泰和人。與昌祺同舉進士。洪熙元年,布政考滿,當給誥命。奏父年八十餘,願以給父。帝嘉而許之,後遂為例。居河南十二年,治行與昌祺等。

陳士啟,名雷,以字行,泰和人。永樂二年進士。選庶吉士,擢禮部郎中。尚書呂震險忮,屬吏皆憚之,承奉唯謹,士啟獨不少徇。

十二年三月,吏部言布、按二司多缺官。帝曰:「布政、按察,吾方岳臣。方數千里地懸數人手,其簡廷臣賢能者,分別用之。」於是諸曹郎、給事中出為監司者二十餘人,而士啟得山東右參政。盡心吏事,不為察察名。督徭賦,不峻期約。青州饑,疏請振之粟。使至,而饑民倍。士啟復上疏,先出粟予民,謂使者曰:「有罪吾獨任。」廷議竟從之。

坐唐賽兒亂下獄,數月,釋還職。高煦謀不軌,士啟自青州暮馳歸語三司,密聞於朝。高煦既執,從薛祿、張本錄餘黨,撫安人民。事竣,命清理山東軍籍。宣德六年卒於官。

應履平,奉化人。建文二年進士。授德化知縣。歷官吏部郎中,出為常德知府。

宣宗初,擢貴州按察使。所至祛除奸蠹,數論時政。舊制,都督府遣使於外,必領內勘合,下都司,不敢輒下衛。至是軍府浸橫,使者挾關文四馳,歷諸衛,朘軍伍。宣德七年,履平抗疏言:「勘合之設,所以防詐偽。今右軍府遣發至黔者,不遵故事,小人恁勢橫求,詐冒何從省。」宣宗善其言,都督陳政引罪。帝令諸司永守之,軍府為之戢。

山雲鎮廣西以備蠻,歲調貴州軍萬人,春秋更代,還多逃亡,則取原衛軍以補,不逐逃者。履平奏:「貴州四境皆苗蠻,軍伍虛,有急孰與戰守?今衛軍逃於廣西,而以在衛者補。不數年,貴州軍伍盡空,邊釁且起。」帝乃命雲嚴責廣西諸衛,追還逃軍,俟足用,即遣歸。罷貴州戍卒。雲,名將,鎮粵有功,輕履平書生。正統元年,履平劾雲弄權,擅作威福,帝令雲自陳。雲大驚,引罪。帝宥之。

明年,上書言四事。一,鎮遠六府,自湖廣改屬貴州,當食川鹽。去蜀道遠,仍食淮鹽為便。一,軍衛糧支於重慶,舟楫不通,易就輕齎多耗費,請以鎮遠秋糧輸湖廣者就近支給。一,停黎平諸府歲辦黃白蠟。一,貴州初開,三司月俸止一石,今糧漸充裕,請增給。並從之。

時方面以公事行部者,例不給驛。履平言僦車舟必擾民,請給驛便。又以軍伍不足,請令衛所官旂犯雜死及徒流者,俱送鎮將立功,期滿還伍;邊軍犯盜及土官民與官旂罪輕者,入粟缺儲所贖罪。並從之。三年遷雲南左布政使。時麓川用兵,屢奏勞績。八年致仕歸。

林碩,字懋弘,閩縣人。永樂十年進士。授御史,出按山東。

宣德初,按浙江。為治嚴肅,就擢按察使。千戶湯某結中官裴可烈為奸利,碩將繩以法。中官誣碩毀詔書,被逮。碩叩頭言:「臣前為御史,官七品。今擢按察使,官三品。日夜淬勵,思報上恩。小人不便,欲去臣,唯陛下裁察。」帝動容曰:「朕固未之信,召汝面訊耳。」立釋碩,復其官,敕責可烈。碩在浙久,人懷其惠。

正統三年誤引赦例出人死,僉事耿定劾之。逮訊,輸贖還職。其冬遷廣東布政使,未及任而卒。其後寧波知府鄭珞劾可烈不法,可烈竟罷去。

況鐘,字伯律,靖安人。初以吏事尚書呂震,奇其才,薦授儀制司主事。遷郎中。

宣德五年,帝以郡守多不稱職,會蘇州等九府缺,皆雄劇地,命部、院臣舉其屬之廉能者補之。鐘用尚書蹇義、胡濙等薦,擢知蘇州,賜敕以遣之。

蘇州賦役繁重,豪猾舞文為奸利,最號難治。鐘乘傳至府。初視事,群吏環立請判牒。鐘佯不省,左右顧問,惟吏所欲行止。吏大喜,謂太守暗,易欺。越三日,召詰之曰:「前某事宜行,若止我;某事宜止,若強我行;若輩舞文久,罪當死。」立捶殺數人,盡斥屬僚之貪虐庸懦者。一府大震,皆奉法。鐘乃蠲煩苛,立條教,事不便民者,立上書言之。

清軍御史李立勾軍暴,同知張徽承風指,動以酷刑抑配平人。鐘疏免百六十人,役止終本身者千二百四十人。屬縣逋賦四年,凡七百六十餘萬石。鐘請量折以鈔,為部議所格,然自是頗蠲減。又言:「近奉詔募人佃官民荒田,官田準民田起科,無人種者除賦額。崑山諸縣民以死徙從軍除籍者,凡三萬三千四百餘戶,所遺官田二千九百八十餘頃,應減稅十四萬九千餘石。其他官田沒海者,賦額猶存,宜皆如詔書從事。臣所領七縣,秋糧二百七十七萬九千石有奇。其中民糧止十五萬三千餘石,而官糧乃至二百六十二萬五千餘石,有畝徵至三石者,輕重不均如此。洪、永間,令出馬役於北方諸驛,前後四百餘匹,期三歲遣還,今已三十餘歲矣。馬死則補,未有休時。工部征三梭闊布八百匹,浙江十一府止百匹,而蘇州乃至七百,乞敕所司處置。」帝悉報許。

當是時,屢詔減蘇、松重賦。鐘與巡撫周忱悉心計畫,奏免七十餘萬石。凡忱所行善政,鐘皆協力成之。所積濟農倉粟歲數十萬石,振荒之外,以代民間雜辦及逋租。其為政,韱悉周密。嘗置二簿識民善惡,以行勸懲。又置通關勘合簿,防出納奸偽。置綱運簿,防運夫侵盜。置館夫簿,防非理需求。興利除害,不遺餘力。鋤豪強,植良善,民奉之若神。

先是,中使織造採辦及購花木禽鳥者踵至。郡佐以下,動遭笞縛。而衛所將卒,時凌虐小民。鐘在,斂跡不敢肆。雖上官及他省吏過其地者,咸心憚之。

鐘雖起刀筆,然重學校,禮文儒,單門寒士多見振贍。有鄒亮者,獻詩於鐘。鐘欲薦之,或為匿名書毀亮。鐘曰:「是欲我速成亮名耳。」立奏之朝。召授吏、刑二部司務。遷御史。

初,鐘為吏時,吳江平思忠亦以吏起家,為吏部司務,遇鐘有恩。至是鐘數延見,執禮甚恭,且令二子給侍,曰:「非無僕隸,欲籍是報公耳。」思忠家素貧,未嘗緣故誼有所幹。人兩賢之。

鐘嘗丁母憂,郡民詣闕乞留。詔起復。正統六年,秩滿當遷,部民二萬餘人,走訴巡按御史張文昌,乞再任。詔進正三品俸,仍視府事。明年十二月卒於官。吏民聚哭,為立祠。

鐘剛正廉潔,孜孜愛民,前後守蘇者莫能及。鐘之後李從智、硃勝相繼知蘇州,咸奉敕從事,然敕書委寄不如鐘矣。

李從智,宜賓人。

朱勝,金華人。勝廉靜精敏,下不能欺。嘗曰:「吏貪,吾不多受牒。隸貪,吾不行杖。獄卒貪,吾不繫囚。」由是公庭清肅,民安而化之。居七年,超遷江南左布政使。

初與鐘同薦者,戶部郎中羅以禮知西安,兵部郎中趙豫知松江,工部郎中莫愚知常州,戶部員外郎邵旻知武昌,刑部員外郎馬儀知杭州,陳本深知吉安,御史陳鼎知建昌,何文淵知溫州,皆賜敕乘傳行。

陳本深,字有源,鄞人。永樂初,由鄉舉入國子監。授刑部主事。善發奸。畿內盜殺人,亡匿。有司繫無辜十八人於獄。本深以計獲盜,十八人皆免。遷員外郎。

與況鐘等同受敕為知府,本深知吉安。吉安多豪強,好訐訟。巨猾彭摶等十九人橫閭里,本深遣人與相結。為具召與飲,伏壯士後堂,拉殺之,皆曳其屍以出,一府大驚。樂安大盜曾子良據大盤山,眾萬餘。本深設伏大破之,斬子良。

本深為政舉大綱,不屑苛細。大猾既殲,府中無事。晨起,鼓而升堂,吏無所白,輒鼓而休。間有所訟,呼至榻前,析曲直遣之,亦不受狀。有抑不伸者,雖三尺童子,皆得往白。久之,民恥爭訟。尤折節士人,飾治學宮,奏新先儒歐陽修、周必大、楊邦乂、胡銓、楊萬里、文天祥祠廟。正統六年,滿九載當遷,郡人乞留,詔予正三品俸。廨前民嫁女,本深聞鼓樂聲,笑曰:「吾來時,乳下兒也。今且嫁,我尚留此耶?」遂請老。前後守吉安十八年,既去,郡人肖像祀之。

羅以禮,桂陽人。永樂十三年進士。由郎中知西安府。遭喪,補紹興。再以喪去。代者不稱職,部民追思,乞以禮於朝。詔起復視事。歲滿,進秩復任。已,移知建昌。所至皆有惠愛。歷三郡,凡二十七年,乃致仕。

莫愚,臨桂人。由鄉舉,以郎中出知常州。奏請減宜興歲進茶數,禁公差官凌虐有司,嚴核上官薦劾之實。皆報可。郡民陳思保年十二,世業漁。其父兄行劫,思保在舟中,有司以為從論,當斬。愚疏言:「小兒依其父兄,非為從比。令全家舟居,將舉家坐耶?」宣宗命釋之,謂廷臣曰:「為守能言此,可謂有仁心矣。」正統六年秩滿,郡民乞留,巡撫周忱以聞。詔進二階復往。

與愚同時為同知者,潞城趙泰,字熙和。由鄉舉入國子監。歷事都察院,授常州同知。濬孟瀆、得勝二河,作魏村閘。周忱、況鐘議減蘇州重糧,泰亦檢常州官田租,請並減之。遷工部郎中,命塞東昌決河。忱薦為協同都運,益勤其職。亡何,疾卒。

彭勖,字祖期,永豐人。七歲,入佛寺不拜。僧強之,叱曰:「彼不衣冠而袒跣,何拜為!」

永樂十三年舉進士。親老,乞近地以養,除南雄府教授。學舍後有祠,數現光怪。學官弟子率禱祀,勖撤而焚之。滿考,補建寧教授。副使王增有疾,醫者許宗道誣諸生游亨魘魅,以舍旁童五郎祠為征。增怒,置亨家七人重罪,下近祠居民獄四百家。勖抗論游氏非巫者,五郎非邪神,初捐地築城人也,事載郡志中。增愕,索圖經證之,大慚悔,事得解。建寧朱子故宅,有祠無祭。勖疏請春秋祭,蠲子孫徭。又創尊賢堂,祀胡安國、蔡沈、真德秀。諸生翕然向學。

正統元年,以楊士奇薦,召授御史。時初設提學官,命督南畿學校。詳立教條,士風大振。疏言:「國朝祠祭,載在禮官。修齋起梁武帝,設醮起宋徽宗,宜一切除之。禁立庵院,罷給僧尼度牒。」又言:「真定、保定、山東民逃鳳陽、潁州以萬計,皆守令匿災暴斂所致,乞厚軫恤。守令課績,宜以戶口增耗為殿最。」又請設南京諸衛武學。皆報可。所至葺治先賢墳祠。母憂歸,以孫鼎代。勖起復,改吏部考功郎中,出為山東副使。土木之變,數言兵事。以直不容於時,致仕歸。

孫鼎,字宜鉉,廬陵人。永樂間舉人。歷松江教授。正統八年,楊溥薦為御史,董南畿學政。置「本源錄」,錄諸生善行。行部不令人知,單輿猝至。諸生謁,輒閉門試之,即日定甲乙。諸生試歸,榜已揭通衢,請託者無所措手。通州旱饑,奏蠲糧三千四百餘石。英宗北狩,鼎試罷,謂諸生曰:「故事當簪花宴,今臣子枕戈之秋,不敢陷諸君不義。」設茗飲,步送諸門。既而詣闕上書,請隨所用效死。不報。未幾,以親老致仕。知府張瑄疏言:「鼎孝追曾、閔,學繼朱、程,宜起居論思之職。」帝不允。天順元年卒於家。

夏時,字以正,錢塘人。永樂十六年進士。授戶科給事中。

洪熙元年議改鈔法。時力言其擾市肆,無裨國用,疏留中。鈔果大沮,民多犯禁。議竟寢。帝思時言,命侍皇太子祀孝陵,所過有災傷,輒白太子,發粟以振。留署南京戶科。

宣德初,一日三上封事。稱旨,命署尚寶司,兼理吏、禮、兵、刑四科,視七篆,無留事。命核後湖黃冊,陳便宜十四事。邳、徐、濟寧、臨清、武清旱,以時請,遣官振之。尋擢江西僉事。

正統三年奏:「今守令多刻刑無辜,傷和乾紀。乞令御史、按察司官遍閱罪囚,釋冤滯。逮按枉法官吏。」從之。遷參議。七年奏恤民六事,多議行。十二年以大臣薦,超擢廣西左布政使。前後所上又十餘疏,雖不盡用,天下壯其敢言。年未七十,致仕歸,卒。其為僉事時,進知州柯暹所撰《教民條約》及《均徭冊式》,刊為令,人皆便之。

時為人廉潔好義。親歿,廬墓有異徵。歿而鄉人祀之,名其祠曰「孝廉」。

黃潤玉,字孟清,鄞人。五歲,侍母疾,夜不就寢。十歲,道見遺金不拾。永樂初,徙南方富民實北京,潤玉請代父行,官少之。對曰:「父去,日益老,兒去,日益長。」官異其言,許之。

十八年舉順天鄉試。授建昌府學訓導。父喪除,改官南昌。宣德中,用薦擢交阯道御史。出按湖廣,斥兩司以下不職者至百有二十人。

正統初,詔推舉提學官。以楊士奇薦,擢廣西僉事,提督學政。時寇起軍興,有都指揮妄掠子女萬餘口,潤玉劾而歸之。副使李立入民死罪至數百人,亦為辨釋。南丹衛處萬山中,戍卒冒瘴多死,為奏徙夷曠地。

母憂歸,起官湖廣。論罷巡撫李實親故二人。實憤,奏潤玉不諳刑律,坐謫含山知縣。以年老歸。歸二十年,年八十有九卒。學者稱「南山先生」。

楊瓚,蠡縣人。永樂末進士。知趙城縣,課績為山西最,超擢鳳陽知府。正統十年大計天下群吏,始命舉治行卓異者,瓚及王懋、葉錫、趙亮等與焉。鳳陽帝鄉,勛臣及諸將子孫多犯令。瓚請立戶稽出入,由是始遵約束。瓚言民間子弟可造者多,請增廣生員毋限額。禮部採瓚言,考取附學。天下學校之有附學生,由瓚議始。

擢浙江右布政使。與鎮守侍郎孫原貞共平陶得二之亂。景泰二年,瓚以湖州諸府官田賦重,請均之民田賦輕者,而嚴禁詭寄之弊。詔與原貞督之,田賦稱平。久之,卒官。

王懋,修武人。永樂末進士,為海豐知縣。後超擢西安知府,亦有聲。

葉錫,永嘉人。宣德五年進士。為吳縣知縣,舉卓異遷。奸民訐於朝,將逮繫。吳人群詣闕頌錫,乃令視事如故,抵誣者罪。尋擢寧國知府。而趙亮為慶雲典史,亦在舉中,同被宴賚。時人以為榮。秩滿,擢知本縣。

劉實,字嘉秀,安福人。宣德五年舉進士。居三年,選庶吉士。正統初,授金華府通判。仍歲荒旱,請蠲租,且贖還饑民子女。義門鄭氏族大,不能自給,又買馬出丁,供山西郵傳,困甚,亦以實言獲免。母喪歸,廬墓三載,起順天府治中。

景泰時,侍臣薦其文學。召修《宋元通鑒綱目》。實為人耿介,意所不可,雖達官貴人不稍遜。然頗自是。見同曹所纂不當,輒大笑,聲徹廷陛,人亦以此忌之。

天順初,還原任。四年擢知南雄府。商稅巨萬,舊皆入守橐。實無所私。中官至南雄,入譖言,府僚參謁,留實折辱之。民競前擁之出,中官慚,將召謝之,實不往。中官去,至韶州,聞韶人言:「南雄守且訟於朝矣。」懼,馳奏,誣實毀敕,大不敬。逮下詔獄。實從獄中上書言:「臣官三十年,未嘗以妻子自隨,食麤衣敝,為國家愛養小民,不忍困之,以是忤朝使。」帝覽書,意稍解,且釋之,而實竟瘐死。

實苦節自持。政務紛遝,未嘗廢書,士大夫重其學行。其歿也,南雄人哀而祠之。孫丙,自有傳。

陳選,字士賢,臨海人。父員韜,宣德五年進士。為御史,出按四川,黜貪獎廉,雪死囚四十餘人。正統末,大軍征鄧茂七,往撫其民,釋被誣為賊者千餘家。都指揮蔣貴要所部賄,都督范雄病不能治軍,皆劾罷之。歷廣東右參政,福建右布政使。廣東值黃蕭養亂後,而福建亦寇盜甫息,員韜所至,拊循教養,得士民心。

選自幼端愨寡言笑,以聖賢自期。天順四年會試第一,成進士。授御史,巡按江西,盡黜貪殘吏。時人語曰:「前有韓雍,後有陳選。」廣寇流入贛州,奏聞,不待報,遣兵平之。

憲宗即位,嘗劾尚書馬昂、侍郎吳復、鴻臚卿齊政,救修撰羅倫,學士倪謙、錢溥。言雖不盡行,一時憚其風采。已,督學南畿。頒冠、婚、祭、射儀於學宮,令諸生以時肄之。作《小學集註》以教諸生。按部常止宿學宮,夜巡兩廡,察諸生誦讀。除試牘糊名之陋,曰:「己不自信,何以信於人?」

成化六年遷河南副使。尋改督學政,立教如南畿。汪直出巡,都御史以下皆拜謁,選獨長揖。直問:「何官?」選曰:「提學副使。」直曰:「大於都御史耶?」選曰:「提學何可比都御史,但忝人師,不敢自詘辱。」選詞氣嚴正,而諸生亦群集署外。直氣懾,好語遣之。

久之,進按察使。決遣輕繫數百人,重囚多所平反,囹圄為空。治尚簡易,獨於贓吏無所假。然受賂百金以上者,坐六七環而止。或問之,曰:「奸人惜財亦惜命,若盡挈所賂以貨要人,即法撓矣。」歷廣東左、右布政使。肇慶大水,不待報,輒發粟振之。

二十一年詔減省貢獻,而市舶中官韋眷奏乞均徭戶六十人添辦方物。選持詔書爭,帝命與其半,眷由是怒選。番人馬力麻詭稱蘇門答剌使臣欲入貢,私市易。眷利其厚賄,將許之,選立逐之去。撒馬兒罕使者自甘肅貢獅子,將取道廣東浮海歸,云欲往滿喇加更市以進。選疏言不可許,恐遺笑外番,輕中國。帝納其言,而眷憾選甚。

先是,番禺知縣高瑤沒眷通番資鉅萬,選移檄獎之,且聞於朝。至是眷誣奏選、瑤朋比為貪墨。詔遣刑部員外郎李行會巡按御史徐同愛訊之。選有所黜吏張褧,眷意其怨選,引令誣證選。褧堅不從,執褧拷掠無異辭。行、同愛畏眷,竟坐選如眷奏,與瑤俱被徵。士民數萬號泣遮留,使者辟除乃得出。至南昌,病作。行阻其醫藥,竟卒。年五十八。

編修張元禎為選治喪,殮之。褧聞選死,哀悼,乃上書曰:

臣聞口能鑠金,毀足銷骨。竊見故罪人選,抱孤忠,孑處群邪之中,獨立眾憎之地。太監眷通番敗露,知縣瑤按法持之。選移文獎厲,以激貪懦,固賢監司事也。都御史宋旻及同愛怯勢養奸,致眷橫行胸臆,穢蔑清流。勘官行頤指鍛煉,竟無左證。臣本小吏,詿誤觸法,被選黜罷,實臣自取。眷意臣憾選,厚賂啖臣,臣雖胥役,敢昧素心。眷知臣不可誘,嗾行等逮臣致理,拷掠彌月。臣忍死籲天,終無異口。行等乃依傍眷語,文致其詞。劾選勘災不實,擅便發倉,曲庇屬官,意圖報謝。必如所云,是毀共姜為夏姬,詬伯夷為莊蹻也。

頃年嶺外地震水溢,漂民廬舍。屬郡交牒報災,老弱引領待哺。而撫、按、籓臬若罔聞知。選獨抱隱憂,食不下咽。謂展轉行勘,則民命垂絕,所以便宜議振,志在救民,非有他也。選故剛正,不堪屈辱,憤懣旬日,嬰疾而殂。行幸其殞身,陰其醫療。訖命之日,密走報眷,小人佞毒,一至於此!臣擯黜罪人,秉耒田野,百無所圖,誠痛忠良銜屈,而為聖朝累也。不報。

員韜父子皆持操甚潔。而員韜量能容物,選務克己,因自號克菴,遇物亦稍峻。人謂員韜德性,四時皆備。選得其秋焉。嘗割田百四十畝贍其族人,暨卒,族人以選子戴貧,還之,戴不可而止。弘治初,主事林沂疏雪選冤,詔復官禮葬。正德中,追贈光祿卿,謚忠愍。

夏寅,字正夫,松江華亭人。正統十三年舉進士。授南京吏部主事。力學,為文以宏CH稱。進郎中。

成化元年考滿入都,上言:「徐州旱澇,民不聊生。饑餒切身,必為盜賊。乞特遣大臣鎮撫,蠲租發廩。沿途貢船,丁夫不足,役及老稚。而所載官物僅一箱,餘皆私齎,乞嚴禁絕。淮、徐、濟寧軍士,赴京操練,然其地實南北要衝,宜各設文武官鎮守,訓兵屯田,常使兩京聲勢聯絡,倉猝可以制變。」章下所司行之,唯不設文武官。

遷江西副使,提督學校。其教務先德行。進浙江右參政。處州民苦虐政,走山谷。寅檄招之,眾皆解散。久之,進山東右布政使。弘治初,致仕歸。

寅清直無黨援。嘗語人曰:「君子有三惜:此生不學,一可惜。此日閒過,二可惜。此身一敗,三可惜。」世傳為名言。

陳壯,字直夫,其先浙江山陰人。祖坐事謫戍交阯,後調京衛,遂家焉。壯舉天順八年進士,授南京御史。編修章懋等建言得罪,抗疏救之。帝遣中官采花木,復疏諫。尚書陳翌請以馬豆代百官俸,壯言飼馬之物,不可養士大夫。事乃寢。

壯家素寠,常祿外一無所取。父母歿,廬墓側,居喪一循古禮。歷江西僉事,致仕歸。家居十餘年。弘治中,以尚書張悅薦,起官福建。居二年,又乞致仕。時倪岳為吏部,素賢之,擢河南副使。歲荒振饑,民懷其惠。僉都御史林俊謝病,舉以自代。未及遷,而壯又乞致仕。巡撫孫需奏留之。又二年,竟致仕去。

張昺,字仲明,慈谿人,都御史楷孫也。舉成化八年進士,授鉛山知縣。性剛明,善治獄。有嫁女者,及婿門而失女,互以訟於官,不能決。昺行邑界,見大樹妨稼,欲伐之。民言樹有神巢其巔。昺不聽,率眾往伐。有衣冠三人拜道左。昺叱之,忽不見。比伐樹,血流出樹間。昺怒,手斧之,卒仆其樹。巢中墮二婦人,言狂風吹至樓上。其一即前所嫁女也。有巫能隱形,淫人婦女。昺執巫痛杖之,無所苦。已,並巫失去。昺馳縛以歸,印巫背鞭之,立死。乃盡毀諸淫祠。寡婦惟一子,為虎所噬,訴於昺。昺與婦期五日,乃齋戒祀城隍神。及期,二虎伏庭下,昺叱曰:「孰傷吾民,法當死。無罪者去。」一虎起,斂尾去。一虎伏不動,昺射殺之,以畀節婦。一縣稱神。鉛山俗,婦人夫死輒嫁;有病未死,先受聘供湯藥者。昺欲變其俗,令寡婦皆具牒受判。署二木。曰「羞」,嫁者跪之。曰「節」,不嫁者跪之。民傅四妻祝誓死守,舅姑紿令跪「羞」木下,昺判從之,祝投後園池中死。邑大旱,昺夢婦人泣拜,覺而識其里居姓氏,往詰其狀。及啟土,貌如生。昺哭之慟曰:「殺婦者,吾也。」為文以祭,改葬焉,天遂大雨。諸異政多類此。

擢南京御史。弘治元年七月偕同官上言:「邇臺諫交章論事矣,而扈蹕糾儀者不免錦衣捶楚之辱,是言路將塞之漸也。經筵既舉矣,而封章累進,卒不能回寒暑停免之說,是聖學將怠之漸也。內倖雖斥梁芳,而賜祭仍及便辟,是復啟寵倖之漸也。外戚雖罪萬喜,而莊田又賜皇親,是驕縱姻婭之漸也。左道雖斥,而符書尚揭於官禁,番僧旋復於京師,是異端復興之漸也。傳奉雖革,而千戶復除張質,通政不去張苗,是傳奉復啟之漸也。織造停矣,仍聞有蟒衣牛斗之織,淫巧其漸作乎?寶石廢矣,又聞有戚里不時之賜,珍玩其漸崇乎?《詩》云『靡不有初,鮮克有終』,願陛下以為戒。」帝嘉納之。

先是,昺以雷震孝陵柏樹,與同官劾大學士劉吉等十餘人,給事中周紘亦與同官劾吉,吉銜之。其冬,昺、紘奉命閱軍,軍多缺伍。兩人欲劾奏守備中官蔣琮,琮先事劾兩人。章下內閣,吉修隙,擬黜之外。尚書王恕抗章曰:「不治失伍之罪,而罪執法之臣,何以服天下!」再疏爭,言官亦論救。乃調昺南京通政司經歷,紘南京光祿寺署丞。

久之,昺用薦遷四川僉事。富豪殺人,屢以賄免。御史檄昺治,果得其情。尋進副使。守備中官某將進術士周慧於朝,昺擒慧,論徙之極邊。歲餘,引疾歸。環堵蕭然,擁經史自娛。都御史王璟以振荒至,饋昺百金,堅拒不得,授下戶饑民粟以答其意。知縣丁洪,昺令鉛山所取士也,旦夕候起居,為具蔬食。昺曰:「吾誠不自給,奈何以此煩令君。」卒弗受。炊煙屢絕,處之澹如。及卒,含斂不具,洪為經紀其喪。

宋端儀,字孔時,莆田人。成化十七年進士。官禮部主事。雲南缺提學官,部議屬端儀,吏先期洩之。端儀曰:「啟事未登,已喧眾口,人其謂我干乞乎!」力辭之。已,進主客員外郎,貢使以贄見,悉卻不納。

初在國學,為祭酒丘濬所知。及濬柄政,未嘗一造其門。廣東提學缺,部以端儀名上,濬竟沮之。濬卒,始以按察僉事督廣東學校。卒官。

端儀慨建文朝忠臣湮沒,乃搜輯遺事,為《革除錄》。建文忠臣之有錄,自端儀始也。

贊曰:明初重監司守牧之任。尚書有出為布政使,而侍郎為參政者,監司之入為卿貳者,比比也。守牧稱職,增秩或至二品。天順而後,巡撫之寄專,而監司守牧不得自展布,重內輕外之勢成矣。夫賦政於外,於民最親。李昌祺、陳本深之屬,靜以愛民,況鐘、張昺能於其職。所謂承宣德化,為天子分憂者,非耶?周新、陳選,冤死為可哀。讀張褧書,又以見公正之服人者至,而直道之終不泯也。


\end{pinyinscope}