\article{列傳第七十六}

\begin{pinyinscope}
○劉抃呂翀艾洪葛嵩趙佑朱廷聲等戴銑李光翰等陸崑薄彥徽等蔣欽周璽塗禎湯禮敬王渙何紹正許天錫周鑰等徐文溥翟唐王鑾張士隆張文明陳鼎等范輅張欽周廣曹琥石天柱

劉蒨,字惟馨,涪州人。弘治十二年進士。授戶科給事中。劾戶部尚書掞鐘縱子受賕,論外戚慶雲侯、壽寧侯家人侵牟商利,阻壞鹺法,又論文選郎張彩顛倒銓政。有直聲。

武宗踐阼,未數月,漸改孝宗之政。蒨疏諫曰:「先帝大漸,召閣臣劉健、李東陽、謝遷於榻前,託以陛下。今梓宮未葬,德音猶存,而政事多乖,號令不信。張瑜、劉文泰方藥弗慎,致先帝升遐,不即加誅,容其奏辨。中官劉郎貽害河南,宜按治,僅調之薊州。戶部奏汰冗員,兵部奏革傳奉,疏皆報罷。夫先帝留健等輔陛下,乃近日批答章奏,以恩侵法,以私掩公,是閣臣不得與聞,而左右近習陰有干預矣。願遵遺命,信老成,政無大小,悉咨內閣,庶事無壅蔽,權不假竊。」報聞。

正德元年,吏部尚書馬文升致仕,廷議推補。御史王時中以閔珪、劉大夏不宜在推舉之列。蒨恐耆德益疏,上疏極論其謬。章下所司,是蒨言,詔為飭言官毋挾私妄奏。孝宗在位時,深悉內臣出鎮之害,所遣皆慎選。劉瑾竊柄,盡召還之,而代以其黨。蒨言:「用新人不若用舊人,猶養饑虎不若養飽虎。」不聽。尋與給事中張文等極言時政缺失五事,忤旨,奪俸三月。

劉健、謝遷去位,蒨與刑科給事中呂翀各抗章乞留,語侵瑾。先是,兵科都給事中艾洪劾中官高鳳姪得林營掌錦衣衛。諸疏傳至南京守備武靖伯趙承慶所,應天尹陸珩錄以示諸僚,兵部尚書林瀚聞而太息。於是給事中戴銑、御史薄彥徽等,各馳疏極諫,請留健、遷。瑾等大怒,矯旨逮銑、彥徽等,下詔獄鞫治,並蒨、翀、洪俱廷杖削籍,承慶停半祿閒住,瀚、珩貶秩致仕。既而列健、遷等五十三人為奸黨,及翀、洪並預焉。

瑾敗,起蒨金華知府,舉治行卓異,未及遷輒告歸。嘉靖初,起知長沙,遷江西副使卒。御史范永奎訟於朝,特予祭葬。

呂翀,廣信永豐人。弘治十二年進士。其請留健、遷言:「二臣不可聽去者有五。孔子稱孟莊子之孝,以不改父之臣為難。二臣皆先帝所簡以遺陛下,今陵土未幹,無故罷遣,何以慰在天之靈?不可一也。二臣雖以老疾辭,實由言違計沮,不得其職而去。陛下聽之,亦以其不善將順,非實有意優老也。在二臣得去就之義,在陛下有棄老成之嫌。不可二也。今民窮財殫,府藏虛罄,水旱盜賊、星象草木之變迭見雜出,萬一禍生不測,國無老成,誰與共事?不可三也。自古剛正者難容,柔順者易合。二臣既去,則柔順之人必進,將一聽陛下所為,非國家之福。不可四也。書曰『無遺壽耇』。健等諳練有素,非新進可侔,今同日去國,天下後世將謂陛下喜新進而厭舊人。不可五也。」既削籍歸,後起雲南僉事。遷四川副使,修成都江堰以資灌溉,水利大興。嘉靖初卒。

艾洪,濱州人。弘治九年進士。授兵科給事中。武宗立,詔清核騰驤諸衛及在京七十二衛軍。給事中葛嵩剔抉無所徇,得各監局占役者七千五百餘人,有旨送各營備操。既而中官魏興、蕭壽等撓之,格不行。洪率同官抗論,竟不能得。又劾英國公張懋、懷寧侯孫應爵、新寧伯譚佑、彭城伯張信,並請斥陜西鎮監劉雲、薊州鎮監劉瑯阜。不聽。雲尋調南京守備,乞以其養子偉為錦衣千戶。洪復率同官劾之,事乃寢。洪在兵科久,諫疏多可稱。削籍後,復罰米二百石輸宣府。後起官,終福建左參政。

葛嵩,字鐘甫,無錫人。弘治十二年進士。由行人擢禮科給事中。閱薊州軍儲,核貴戚所侵地,歸之民。正德初,以釐營弊力抗權倖。請出先朝宮人,諫射獵,因劾魏國公徐俌。又偕九卿請誅劉瑾。瑾怒,斥為奸黨,罷歸。

趙佑,字汝翼,雙流人。弘治十二年進士。由繁昌知縣召為御史。

正德元年六月,災異求言,佑上言:「太監劉瑾、丘聚、馬永成輩日獻鷹犬,導騎射,萬一有銜橛之變,豈不為兩宮憂?鎮守內臣鄧原、麥秀頗簡靜,而劉璟、梁裕擠代之。戶部議馬房草場召民佃種,寧瑾竟自奏止。李興擅伐陵木,已坐大辟,乃欲賂左右祈免。他如南京守備劉雲,倉場監督趙忠、韋雋、段循,俱夤緣增設。乞置瑾等於法,罷璟、裕毋遣,而汰革額外冗員。自今政事必諮大臣、臺諫,不為近習所搖,則災變自弭。」奏入,群奄大恨。

帝將大婚,詔取太倉銀四十萬兩。佑言:「左右以婚禮為名,將肆無厭之欲。計臣懼禍而不敢阻,閣臣避怨而不敢爭。用如泥沙,坐致耗國。不幸興師旅,遘饑饉,將何以為計哉?」九月,宛平郊外李花盛開,佑言:「此陰擅陽權,非偶然也。」帝皆不納。

是時,中官益橫,佑與同官朱廷聲、徐鈺交章極論。章下閣議,將重罪中官。事忽中變,劉健、謝遷去位。瑾遂大逐廷臣忤己者,指佑與廷聲、鈺及陳琳、潘鏜等為奸黨,勒罷之。瑾誅,佑用薦起山西僉事。卒。

硃廷聲,字克諧,進賢人。弘治十二年進士。嘉靖中,終刑部右侍郎。

徐鈺,字用礪,江夏人。弘治九年進士,終四川左布政使。

陳琳,字玉疇,甫田人。弘治九年進士。由庶吉士改御史,上端本修政十五事。出督南畿學政。劉瑾逐健、遷,逮戴銑、陸崑等,琳抗章言:「南京窮冬雷震,正旦日食。正宜修德弭災,委心元寮,博采忠言。豈宜自棄股肱、隔塞耳目?」瑾大怒,謫揭陽丞。瑾敗,遷嘉興同知。世宗時,終南京兵部右侍郎。

潘鏜,字宗節,六安人。弘治九年進士。有孝行。為滿城知縣,憂歸。繼知滑縣,擢御史,陳時務大計四事。孝宗嘉納之。正德初,以論高鳳為中人所惡,傳旨鏜黨太監王岳,除其名。八年起廣東僉事,謝病歸。

戴銑,字寶之,婺源人。弘治九年進士,改庶吉士,授兵科給事中,數有建白。久之,以便養調南京戶科。武宗嗣位,偕同官請敕六科檢詳弘治間所行進賢、退奸、節財、訓兵、重祀、慎刑、救災、恤困諸大政,備錄進覽,凡裁決機務悉以為準。報聞。踰月,言四方歲辦多非土產,勞費滋甚,宜蠲其所無。又請勤御經筵,俾密勿大臣從容獻納。既乃與給事中李光翰、徐蕃、牧相、任惠、徐暹及御史薄彥徽等連章奏留劉健、謝遷,且劾中官高鳳。帝怒,逮繫詔獄,廷杖除名。銑創甚,遂卒。世宗立,追贈光祿少卿。

李光翰,新鄉人。弘治十二年進士。授南京戶科給事中。正德改元,災異求言。光翰偕同官疏劾太監苗逵、高鳳、李榮及保國公朱暉,且言大學士劉健等疏陳鹽法事,留中不報,將使老臣不安其位。帝不省。既削籍歸,後起台州知府,與蕃同舉治行卓異,尋卒。

徐蕃,泰州人。弘治六年進士。授南京禮科給事中。武宗嗣位,復先朝所汰諸冗費,蕃等力爭,不納。後起江西參議,從都御史陳金討平東鄉寇。嘉靖時,累官工部右侍郎。

牧相,餘姚人。弘治十二年進士。授南京兵科給事中。論救宣府都御史雍泰,又公疏請罷禮部尚書崔志端等,皆不聽。正德元年奉命與御史呂鏜清查御馬監,因陳濫役濫費之弊,及太監李棠珝詔旨營私罪。至是,受杖歸,授徒養母。後復官,擢廣西參議。命下,相已前卒。

任惠,灤州人。弘治九年進士。由行人擢南京吏科給事中。正德元年九月,偕同官諫佚遊,語切直。後起山東僉事,未任卒。

徐暹,歷城人。弘治十五年進士。武宗即位,擢南京工科給事中。正德改元,因災異上言七事,且請斥英國公張懋、尚書張昇等,撤諸添註內官,明正張瑜、劉文泰用藥失宜致誤先帝,及太監李興擅伐陵木,新寧伯譚佑、侍郎李鐩同事不舉之罪。帝下之所司。後起山西僉事,進副使。平巨盜混天王,民德之。卒於官。

陸崑,字如玉,歸安人。弘治九年進士。授清豐知縣。以廉幹征,擢南京御史。

武宗即位,疏陳重風紀八事:一,獎直言。古者,臣下不匡,其刑墨。宋制,御史入臺,踰十旬無言,有辱臺之罰。今郎署建言,如李夢陽、楊子器輩,當加旌擢,而言官考績,宜以章疏多寡及當否為殿最。二,復面劾。舊制,御史上殿,被劾者趨出待罪,即唐人對仗讀彈文遺意。近率封章奏聞,批答未行,彌縫先入。乞遵舊典面奏,立取睿裁。三,明淑慝。尚書劉大夏、王軾以病乞休;侍郎張元禎、陳清屢劾不去。賢不肖倒置,實治亂消長之關。宜勉留二人,放元禎等還田里。四,核命令。近者言妨左右,頻見留中。事涉所私,輒收成命。乞令諸曹章奏俱具數送閣,已行者得考稽,未行者易奏請。五,養銳氣。御史與都御史,例得互相糾繩,行事不宜牽制。六,均差遣。御史以南北為限,顯分重輕。自今除巡按面命外,其他差遣及遷轉資格,宜均擬上請,以示一體。七,專委任。河南道有考核之責,請擇人專任。八,勵庶官。郎中田岩、姚汀、張憲,員外郎李承勛、胡世寧、張嵿、顧璘等二十人,皆宜顯擢。章下所司。又劾中官高鳳、苗逵、保國公朱暉,因請汰南京增設守備內臣,廣開言路,屏絕宴遊騎射。帝不能從。

時「八黨」竊柄,朝政日非。崑偕十三道御史薄彥徽、葛浩、貢安甫、王蕃、史良佐、李熙、任諾、姚學禮、張鳴鳳、蔣欽、曹閔、黃昭道、王弘、蕭乾元等,上疏極諫曰:「自古奸臣欲擅主權,必先蠱其心志。如趙高勸二世嚴刑肆志,以極耳目之娛;和士開說武成毋自勤約,宜及少壯為樂;仇士良教其黨以奢靡導君,勿使親近儒生,知前代興亡之故。其君惑之,卒皆受禍。陛下嗣位以來,天下顒然望治。乃未幾,寵倖奄寺,顛覆典刑。太監馬永成、魏彬、劉瑾、傅興、羅祥、谷大用輩共為蒙蔽,日事宴遊。上干天和,災寢疊告,廷臣屢諫,未蒙省納。若輩必謂『宮中行樂,何關治亂』,此正奸人欺君之故術也。陛下廣殿細旃,豈知小民窮簷蔀屋風雨之不庇;錦衣玉食,豈知小民祁寒暑雨凍餒之弗堪;馳騁宴樂,豈知小民疾首蹙頞赴訴之無路。昨日雷震郊壇,彗出紫微,夏秋亢旱,江南米價騰貴,京城盜賊橫行。可恣情縱欲,不一顧念乎?閣部大臣受顧命之寄,宜隨事匡救,弘濟艱難,言之不聽,必伏闕死諫,以悟聖意。顧乃怠緩悅從,巽順退託。自為謀則善矣,如先帝付委、天下屬望何?伏望側身修行,亟屏永成輩以絕禍端,委任大臣,務學親政,以還至治。」疏至,朝事已變,劉健、謝遷皆被逐。於是彥徽為首,復上公疏,請留健、遷,而罪永成、瑾等。瑾怒,悉逮下詔獄,各杖三十,除名。昭道、弘、乾元逮捕未至,命即南京闕下杖之。江西清軍御史王良臣聞崑等被逮,馳疏救,並逮下詔獄,杖三十,斥為民。後列奸黨五十三人,崑、彥徽等並與焉。瑾誅,復崑官致仕。世宗初,起用,未行而卒。薄彥徽,陽曲人。弘治九年進士。授四川道御史。嘗劾崔志端以羽士玷春卿,有直聲。至是,被杖歸,未及起官卒。

葛浩,字天宏,上虞人。弘治九年進士。由五河知縣擢御史,數陳時政闕失,孝宗多采納。

正德元年,帝允司禮中官高鳳請,令其從子得林掌錦衣衛事。浩等爭之,言:「先帝詔錦衣官悉由兵部推舉,陛下亦悉罷傳奉乞官。今得林由傳奉,不關兵部,廢先帝命,壞銓舉法,虛陛下詔,一舉三失,由鳳致之。乞治鳳罪,而罷得林。」御史潘鏜亦言:「鳳、得林操中外大柄,中人效尤,弊將安底。」帝皆不聽。浩既削籍,瑾憾未釋,復坐先所劾武昌知府陳晦不實,與安甫、蕃、熙、學禮、昆六人,逮杖闕下。瑾誅,起浩知邵武府。入覲,陳利弊五事,悉施行。嘉靖中,歷官兩京大理卿。帝郊祀,有犯蹕者,法司欲置重典,浩執奏,得不死。十年夏,雷震午門,自劾致仕歸,年九十二卒。

貢安甫,字克仁,江陰人。弘治九年進士。授長垣知縣。孝宗時,擢御史,嘗疏劾壽寧侯張鶴齡。正德初,考功郎楊子器以山陵事下詔獄,安甫疏力救。兵部尚書劉大夏為中官所扼,謝病去,戶部侍郎陳清遷南京工部尚書,安甫率御史請還大夏而罷清。報聞。彥徽等公疏,安甫筆也,瑾知之,故列奸黨以安甫首南御史。家居十年,終歲不入城市。後起山東僉事,甫三月,引疾歸。

史良佐,字禹臣,亦江陰人。弘治十二年進士。由行人擢御史。後起雲南副使。平十八寨苗,賜白金文綺。濬海田,溉田千頃,滇人頌之。

李熙,上元人。弘治九年進士。由將樂知縣擢御史。十八年,奸人徐俊等造謠言:帝遣官齎駕帖至南京,有所捕治。已而知其妄。熙公疏言:「陛下於此事威與明少損矣。倘奸人效尤,妄以蜚語中善類,害何可勝言!」事下法司,亦力言駕帖之害,帝納之。正德元年九月,以災異,復偕御史陳十事。瑾誅,得禍者皆起,熙獨廢。世宗嗣位,始起饒州知府,遷浙江副使,以清操聞。

姚學禮,巴人,家京師。弘治六年進士。正德元年,公疏諫佚遊,不納。後起雲南僉事,終參議。

張鳴鳳,清平人。弘治九年進士,為永康知縣。有政績,擢御史。後起湖廣僉事,進副使,母憂歸,卒。蔣欽杖死,別有傳。

曹閔,上海人。弘治九年進士,為沙縣知縣。被徵,民號泣攀留,累日不得去。既與崑等同得罪。後當起官,以養母不出。母終,枕塊,得寒疾卒。

黃昭道,平江人,弘治十二年進士。後起廣西僉事,再遷雲南參政。撫木邦、孟密有功。終左布政使。

王弘,六合人,弘治六年進士。

蕭乾元,萬安人,弘治十二年進士。王蕃、任諾鞫獄時,抵不與知,不足載。

王良臣,陳州人。弘治六年進士。官南京御史。瑾誅,起山東副使,終按察使。

蔣欽,字子修,常熟人。弘治九年進士。授衛輝推官。征擢南京御史,數有論奏。

正德元年,劉瑾逐大學士劉健、謝遷,欽偕同官薄彥徽等切諫。瑾大怒,逮下詔獄,廷杖為民。居三日,欽獨具疏曰:「劉瑾,小豎耳。陛下親以腹心,倚以耳目,待以股肱,殊不知瑾悖逆之徒,蠹國之賊也。忿臣等奏留二輔,抑諸權奸,矯旨逮問,予杖削職。然臣思畎畝猶不忘君,況待命衽席,目擊時弊,烏忍不言。昨瑾要索天下三司官賄,人千金,甚有至五千金者。不與則貶斥,與之則遷擢。通國皆寒心,而陛下獨用之於左右,是不知左右有賊,而以賊為腹心也。給事中劉蒨指陛下闇於用人,昏於行事,而瑾削其秩,撻辱之。矯旨禁諸言官,無得妄生議論。不言則失於坐視,言之則虐以非法。通國皆寒心,而陛下獨用之於前後,是不知前後有賊,而以賊為耳目股肱也。一賊弄權,萬民失望,愁歎之聲動徹天地。陛下顧懵然不聞,縱之使壞天下事,亂祖宗法。陛下尚何以自立乎?幸聽臣言,急誅瑾以謝天下,然後殺臣以謝瑾。使朝廷一正,萬邪不能入;君心一正,萬欲不能侵,臣之願也。今日之國家,乃祖宗之國家也。陛下茍重祖宗之國家,則聽臣所奏。如其輕之,則任瑾所欺。」疏入,再杖三十,繫獄。

越三日,復具疏曰:「臣與賊瑾勢不兩立。賊瑾蓄惡已非一朝,乘間起釁,乃其本志。陛下日與嬉遊,茫不知悟。內外臣庶,凜如冰淵。臣昨再疏受杖,血肉淋漓,伏枕獄中,終難自默,願借上方劍斬之。朱雲何人,臣肯少讓?陛下試將臣較瑾,瑾忠乎,臣忠乎?忠與不忠,天下皆知之,陛下亦洞然知之,何仇於臣,而信任此逆賊耶?臣骨肉都銷,涕泗交作,七十二歲老父,不顧養矣。臣死何足惜,但陛下覆國喪家之禍起於旦夕,是大可惜也!陛下誠殺瑾梟之午門,使天下知臣欽有敢諫之直,陛下有誅賊之明。陛下不殺此賊,當先殺臣,使臣得與龍逢、比干同遊地下,臣誠不願與此賊並生。」疏入,復杖三十。

方欽屬草時,燈下微聞鬼聲。欽:「念疏上且掇奇禍,此殆先人之靈欲吾寢此奏耳。」因整衣冠立曰:「果先人,盍厲聲以告。」言未已,聲出壁間,益悽愴。欽歎曰:「業已委身,義不得顧私,使緘默負國為先人羞,不孝孰甚!」復坐,奮筆曰:「死即死,此稿不可易也!」聲遂止。杖後三日,卒於獄,年四十九。瑾誅,贈光祿少卿。嘉靖中,賜祭葬,錄一子入監。

周璽,字天章,廬州衛人。弘治九年進士。授吏科給事中。三遷禮科都給事中。慷慨好言事。

武宗初即位,請毀新立寺觀,屏逐法王、真人,停止醮事,並論前中官齊玄煉丹糜金罪。頃之,以久雨,偕同官劾侍郎李溫、太監苗逵。九月,以星變,復劾溫及尚書崔志端、熊翀、賈斌,都御史金澤、徐源等,翀、溫、澤因是罷。帝遣中官韋興守鄖陽,璽力言不可。尋復偕同官言:「邇者聰明日蔽,膏澤未施。講學一暴而十寒,詔令朝更而夕改。冗員方革復留,鎮監撤還更遣。解戶困於交收,鹽政壞於陳乞。厚戚畹而駕帖頻頒,私近習而帑藏不核。不可不亟為釐正。」不聽。

正德元年復應詔陳八事,中劾大寮賈斌等十一人,中官李興等三人,勳戚張懋等七人,邊將朱廷、解端、李稽等三人。未幾,言:「陛下即位以來,鷹犬之好,糜費日甚。如是不已,則酒色游觀,便佞邪僻,凡可以悅耳目蕩心志者,將無所不至。光祿上供,視舊十增七八,新政已爾,何以克終?」御史何天衢等亦以為言。章下禮部,尚書張昇請從之。帝雖不加譴,不能用也。

明年擢順天府丞。璽論諫深切,率與中官牴牾,劉瑾等積不能堪。至是,命璽與監丞張淮、侍郎張縉、都御史張鸞、錦衣都指揮楊玉勘近縣皇莊。玉,瑾黨,三人皆下之。璽辭色無假,且公移與玉止牒文。玉奏璽侮慢敕使,瑾即矯旨逮下詔獄,搒掠死。瑾誅,詔復官賜祭,恤其家。嘉靖初,錄一子。

又御史塗禎,新淦人也。弘治十二年進士。初為江陰知縣。正德初,巡鹽長蘆。瑾縱私人中鹽,又命其黨畢真託取海物,侵奪商利,禎皆據法裁之。比還朝,遇瑾止長揖。瑾怒,矯旨下詔獄。江陰人在都下者,謀斂錢賂瑾解之,禎不可,喟然曰:「死耳,豈以污父老哉!」遂杖三十,論戍肅州,創重竟死獄中。瑾怒未已,取其子樸補伍。瑾誅,樸乃還,禎復官賜祭。

湯禮敬,字仁甫,丹徒人。弘治九年進士。授行人,擢刑科給事中。

正德初,上言:「陛下踐阼以來,上天屢示災譴。不謹天戒,惟走馬射獵,遊樂無度。頃四月中旬,雷電雨雹,當六陽用事時,陰氣乃與之抗,此倖臣竊權,忠鯁疏遠之應也。」已,又論兩廣鎮監韋經,又偕九卿伏闕請誅「八黨」。劉瑾銜之,尋以其請當審奏囚決之日,有愬冤者屏勿奏,指為變祖制,謫薊州判官。後列奸黨給事中十六人,禮敬居首,罷歸。未幾卒。

瑾惡言官譏切時政多刺己,輒假他事坐之。禮敬得罪後,有王渙、何紹正。

王渙,字時霖,象山人。弘治九年進士。由長樂知縣徵授御史。正德元年,應詔條上應天要道五事,語多斥宦官。明年出視山海諸關,以病謝事未行。盜發其部內,都御史劉宇承瑾指劾渙失報。逮下詔獄,杖之,斥為民。瑾敗,復官致仕。

何紹正,淳安人。弘治十五年進士。授行人。正德三年擢吏科給事中。中官廖堂鎮河南,奏保方面數人,且擅擬遷調。吏部尚書許進等不能難,紹正劾之。瑾不得已責堂自陳,而心甚銜紹正。及冬,坐頒曆導駕失儀,杖之闕下,謫海州判官。屢遷池州知府,築銅陵五十餘圩以備旱潦。宸濠反,攻安慶,池人震恐,紹正登陴固守。事平,增俸一級,遷江西參政致仕。池人為立祠,與宋包拯並祀。

許天錫,字啟衷,閩縣人。弘治六年進士。改庶吉士。思親成疾,陳情乞假。孝宗賜傳以行。還朝,授吏科給事中。時言官何天衢、倪天明與天錫並負時望,都人有「臺省三天」之目。

十二年,建安書林火。天錫言:「去歲闕里孔廟災,今茲建安又火,古今書版蕩為灰燼。闕里,道所從出;書林,文章所萃聚也。《春秋》書宣榭火,說者曰:『榭所以藏樂器也。天意若曰不能行政令,何以禮樂為?禮樂不行,天故火其藏以戒也。』頃師儒失職,正教不修。上之所尚者浮華,下之所習者枝葉。此番災變,似欲為儒林一掃積垢。宜因此遣官臨視,刊定經史有益之書。其餘晚宋陳言,如論範、論草、策略、策海、文衡、文髓、主意、講章之類,悉行禁刻。其於培養人才,實非淺鮮。」所司議從其言,就令提學官校勘。

大同失事,天錫往核,具得其狀,巡撫洪漢、中官劉雲、總兵官王璽以下咸獲罪。內使劉雄怒儀真知縣徐淮廚傳不飭,訴之南京守備中官以聞,逮淮繫詔獄。天錫及御史馮允中論救,卒調淮邊縣。御史文森、張津、曾大有言事下吏,崔志端由道士擢尚書,天錫皆力爭。

十七年五月,天變求言。上疏曰:「外官三年考察,又有撫按監臨,科道糾劾,其法已無可加。惟兩京堂上官例不考核。而五品以下雖有十年考察之條,居官率限九載,或年勞轉遷,或服除改補,不能及期。今請以六年為期,通行考察。其大寮曾經彈劾者,悉令自陳而簡去之,用儆有位。古者,災異策免三公,陰霖輒避位。今大臣不引咎,陛下又不行策免,宜且革公孤銜,俟天心既回,徐還厥職。祖宗御內官,恩不泛施,法不輕貸。內府二十四監局及在外管事者,並有常員。近年諸監局掌印、僉事多至三四十人,他管事無數,留都亦然。憑陵奢暴,蠹蝕民膏,第宅連雲,田園遍野,膏粱厭於輿臺,文繡被乎狗馬。凡若此類,皆足召變。乞敕司禮監會內閣嚴行考察,以定去留。此後,或三年、五年一行,永為定制。」帝善之。於是令兩京四品以上並自陳聽命,五品下六年考察,遂著為令。惟大臣削公孤及內官考察,事格不行。尋與御史何深核牛馬房,條上便利十四事,歲省芻豆費五十餘萬。

武宗即位之七月,因災異上疏,請痛加修省,廣求直言,遷工科左給事中。正德改元,奉使封安南,在道進都給事中。三年春,竣事還朝。見朝事大變,敢言者皆貶斥,而劉瑾肆虐加甚,天錫大憤。六月朔,清核內庫,得瑾侵匿數十事。知奏上必罹禍,乃夜具登聞鼓狀。將以尸諫,令家人於身後上之,遂自經。時妻子無從者,一童侍側,匿其狀而遁。或曰瑾懼天錫發其罪,夜令人縊殺之。莫能明也。時有旨,令錦衣衛點閱六科給事中,不至者劾之。錦衣帥劾天錫三日不至。訊之,死矣。聞者哀之。

方瑾用事,橫甚,尤惡諫官,懼禍者往往自盡。

海陽周鑰,弘治十五年進士。為兵科給事中,勘事淮安,與知府趙俊善。俊許貸千金,既而不與。時奉使還者,瑾皆索重賄。鑰計無所出,舟行至桃源,自刎。從者救之,已不能言,取紙書「趙知府誤我」,遂卒。事聞,繫俊至京,責鑰死狀,竟坐俊罪。

平定郗夔,弘治十五年進士,為禮科給事中。正德五年,出核延綏戰功,瑾屬其私人。夔念從之則違國典,不從則得禍,遂自經死。

瓊山馮顒,弘治九年進士。為御史,嘗以事忤瑾,為所誣,自經死。顒初為主事,官軍討叛黎符南蛇久不克,顒歷陳致變之由,請購已革土官子孫,俾召集舊卒,以夷攻夷,有功則復舊職。尚書劉大夏亟稱之,奏行其策。正德初,偕中官高金勘涇王所乞莊地,清還二千七百餘頃。而不得其死,人皆惜之。

瑾誅,天錫、鑰、夔、顒俱復官賜祭,且恤其家。嘉靖中,天錫子春訟冤,復賜祭葬。

方瑾敗時,刑部員外郎夾江宿進疏陳六事,言:「忤逆瑾死者,內臣如王岳、范亨,言官如許天錫、周鑰,並宜恤贈。又附瑾大臣,如兵部尚書王敞等及內侍餘黨,俱宜斥。」疏入,帝怒將親鞫之,命張永召閣臣李東陽。東陽語永曰:「後生狂妄,且日暮非見君時,幸少寬之。」永入,少頃執進至午門,杖五十,削籍歸,未幾卒。世宗初,贈光祿少卿。

徐文溥,字可大,開化人。正德六年進士。授南京禮科給事中。劾尚書劉櫻、都御史李士實、侍郎呂獻、大理卿茆欽,而請召還致仕尚書孫交、傅珪。時論以為當。

寧王宸濠求復護衛,文溥諫曰:「曩因寧籓不靖,英廟革其護衛、屯田。及逆瑾亂政,重賄謀復。瑾既伏誅,陛下又革之,正欲制以義而安全之耳。乃曰『驅使乏人』。夫晏居深邃,靡征討之勞,安享尊榮,無居守之責,何所用而乏人?且王暴行大彰:剝削商民,挾制官吏,招誘無賴,廣行劫掠。致舟航斷絕,邑里蕭條,萬民莫不切齒。乃今止之,猶恐不逮,顧可縱之加恣,假翼於虎乎?貢獻本有定制,乃無故馳騁飛騎,出入都城,伺察動靜。況今海內多故,天變未息,意外之虞實未易料。宜裁以大義,勿徇私情,罪其獻謀之人,逐彼偵事之使,宗社幸甚。」時宸濠奧援甚眾,疏入,人咸危之,帝但責其妄言而已。又請擇建儲貳,不報。

十年四月復偕同官上疏曰:「頃因災異,禮部奏請修省。伏讀聖諭,謂『事關朕躬者,皆已知之』。臣惟茲一念之誠,足以孚上帝迓休命矣。雖然,知之非艱,行之維艱。陛下誠能經筵講學,早朝勤政;布寬恤以安人心,躬獻享以重宗廟;孝養慈闈,敬事蒼昊;舍豹房而居大內,遠嬖倖而近儒臣;禁中不為貿易,皇店不以罔財;還邊兵於故伍,斥番僧於外寺;毋暱俳優,盡屏義子;馬氏已醮之女弗留乎後宮,馬昂梟獍之族立奪其兵柄;停諸路之織造,罷不急之土木;汰倉局門戶之內官,禁水陸舟車之進奉;出留中奏牘以達下情,省傳奉冗員以慎名器。則陛下所謂『事關朕躬』,非徒知之,且一一行之,而不轉禍為福者,未之有也。」報聞。

初,帝聽中官崔瑤、史宣、劉瑯阜、於喜誣奏,先後逮知府翟唐,部曹王鑾、王瑞之,御史施儒、張經等,又入中官王堂譖,下僉事韓邦奇獄。文溥言:「朝廷刑威所及,乃在奄侍一言。旗校繹絡於道途,縉紳駢首於狴犴,遠近震駭,上下屏氣。向一瑾亂政於內,今數瑾縱橫於外。乞并下堂法司,且追治瑤等誣罔罪。」帝不聽,遂引疾去。

世宗即位,廷臣交薦,起河南參議。未幾,以念母乞歸。撫按請移近地便養,乃改福建。尋遷廣東副使。上言十事,多涉權要,恐貽母憂,復引疾歸。行至玉山卒。

翟唐,字堯佐,長垣人。弘治十二年進士。由壽光知縣召為御史。正德四年出按湖廣,奏言:「四川賊首劉烈僭號設官,必將為大患。湖廣、陜西壤地相接,入竹山可抵荊、襄,入漢中可抵秦、隴。今內外壅蔽,獎諭切責率皆虛文,宜切圖預備之策。」時劉瑾竊柄,以唐言「壅蔽」,尤惡之。兵部尚書王敞希指,言今蕩滌宿弊,唐乃云然,宜令指實。會瑾怒稍解,乃切責而宥之。久之,遷知寧波府。市舶中官崔瑤藉貢物擾民,為唐所裁抑,且杖其黨王臣,臣尋病死。瑤奏唐阻截貢獻,笞殺貢使。帝怒,逮下詔獄。巡按御史趙春等交章救之。給事中范洵亦言唐被逮日,軍民遮道涕泣,請宥令還任。帝不聽,謫雲南嵩明知州。再遷陜西副使卒。

王鑾,字廷和,大庾人。正德三年進士。授邵武知縣。入為都水主事,出轄徐沛閘河。十一年,織造中官史宣過其地,索挽夫千人,沛縣知縣胡守約給其半。宣怒,自至縣捕吏,鑾助守約與抗。宣誣奏於朝,逮繫詔獄。以言官論救,守約罷官,鑾輸贖還職。已,分司南旺,又捕誅中官廖堂姪廖鵬之黨。嘉靖初,遷武昌知府。鎮守中官李景儒歲進魚鮓多科率,鑾疏請罷之。楚府征稅,茶商重困。鑾謂稅當歸官,力與爭,王詆為毀辱親王。鑾遂請終養,不待報竟歸。後吏部坐以擅離職守,奪官。

張士隆,字仲修,安陽人。弘治八年舉鄉試,入太學。與同縣崔銑及寇天敘、馬卿、呂柟輩相砥礪,以學行聞。十八年成進士,授廣信推官。

正德六年入為御史。巡鹽河東,劾去貪污運使劉愉。建正學書院,興起文教。九年,乾清宮災,上疏曰:「陛下前有逆瑾之變,後遭薊盜之亂,猶不知警。方且興居無度,狎暱匪人。積戎醜於禁中,戲干戈於臥內。徹旦燕遊,萬幾不理。寵信內侍,濁亂朝綱。致民困盜起,財盡兵疲。禍機潛蓄,恐大命難保。夫裒衣博帶之雅,孰與市井狡儈之群?廣廈細旃之娛,孰與鞍馬驅馳之險?」不報。

出按鳳陽。織造中官史宣列黃梃二於騶前,號為「賜棍」,每以抶人,有至死者,自都御史以下莫敢問,士隆劾奏之。又劾錦衣千戶廖鎧奸利事,且曰:「鎧虐陜西,即其父鵬虐河南故習也。河南以鵬故召亂,鎧又欲亂陜西。乞置鎧父子於法,并召還廖鑾,以釋陜人之憤。」鑾,鎧所從鎮陜西者也。錢寧素暱鎧,見疏大恨,遂因士隆按薛鳳鳴獄以陷之。鳳鳴者,寶坻人,先為御史,坐罪削籍,諂事諸佞倖,尤善寧。與從弟鳳翔有隙,嗾緝事者發其私,下吏論死。刑部疑有冤,并捕鞫鳳鳴。鳳鳴懼,使其妾訴枉,自剄長安門外,詞連寶坻知縣周在及素所仇者數十人,悉逮付法司,而鳳鳴得釋。士隆與御史許完先後按治,復捕鳳鳴對簿,釋在還職。寧怒,令鳳鳴女告士隆、完治獄偏枉。遂下詔獄,謫士隆晉州判官。久之,擢知州。

世宗立,詔復故官,出為陜西副使。漢中賊王大等匿豪家,結回回為亂。士隆下令:匿賊者罪及妻孥,無赦。賊無所容,遂就擒滅。築堰溉田千頃,民利之。卒於官。

張文明,字應奎,陽曲人。正德六年進士。授行人,擢御史,巡按遼東。尋按陜西。鎮守中官廖堂貪恣,文明捕治其爪牙二十四人,堂大恨。

十三年,車駕幸延綏。文明馳疏諫,極陳災異,且言江彬逢惡導非,亟宜行誅。朝臣匡救無聞,亦當罰治。帝不省。既而文明朝行在。諸權倖扈從者,文明裁抑之,所需多不應。司禮太監張忠等譖於帝,言諸生毆旗校,文明縱勿治。帝怒,命械赴京師,下詔獄。明年春,言官交章請宥,不報。比駕旋,命執至豹房,帝將親鞫。文明自謂必死。及見帝,命釋之,謫電白典史。時劉瑾雖誅,佞幸猶熾,中外諫官被禍者不可勝數。文明止於貶謫,人以為幸。

世宗立,召復故官,尋出為松江知府。甫抵任,卒。巡按御史馬錄頌其忠,詔贈太常少卿。

陳鼎,字大器,其先宣城人。高祖尚書迪,死惠帝之難,子孫戍登州衛,遂占籍焉。鼎舉弘治十八年進士。正德四年授禮科試給事中。鎮守河南中官廖堂,福建人也,弟鵬之子鎧冒籍中河南鄉試。物議沸騰,畏堂莫敢與難。鼎上章發其事,鎧遂除名,堂、鵬大恨。會流寇起,鼎陳弭盜機宜。堂囑權倖摘其語激帝怒,下詔獄掠治。謂鼎前籍平江伯資產,附劉瑾增估物價,疑有侵盜。尚書楊一清救之,乃釋為民。世宗立,復故官,遷河南參議。妖人馬隆等為亂,鼎督兵誅之。改陜西副使,擢浙江按察使,廉介正直,不通私謁。召為應天府尹,未任卒。

賀泰,字志同,吳縣人。弘治十二年進士。由衢州府推官入為御史。武宗收京師無賴及宦官廝養為義子,一日而賜國姓者百二十七人,泰抗言其非。諸人激帝怒,謫衢州推官,終廣東參議。

張璞,字中善,江夏人。弘治十八年進士。由歸安知縣召授御史。正德八年出按雲南。鎮守中官梁裕貪橫,璞裁抑之。為所誣,逮赴詔獄,死獄中。世宗嗣位,贈太僕少卿,賜祭葬。

成文,大同山陰人。弘治十五年進士。由知縣擢御史。正德中,阿爾禿廝、亦不剌與小王子戰敗,引所部駐甘肅塞外,時入寇,掠陷堡寨五十有三。巡撫張翼、鎮守太監朱彬等反冒奏首功千九百有餘,以捷奏者十有一。文出巡按,盡發其奸。翼等賄中人傾文。會文劾僉事趙應龍,應龍亦訐文細事,遂逮文,斥為民。嘉靖中起用,累官右副都御史巡撫遼東,告歸,卒。

李翰臣,大同人。正德三年進士。官御史,巡按山東。吏部主事梁穀誣歸善王當沍謀叛,翰臣劾穀挾私。近倖方欲邀功,責翰臣為叛人掩飾。逮繫詔獄,謫德州判官。終山東副使。

張經,興州左衛人。正德六年進士。官御史。出按宣府,劾鎮守中官於喜貪肆罪。為喜所訐,逮繫詔獄,謫雲南河西典史。尋卒。世宗初,贈祭如張璞。

毛思義,陽信人。弘治十五年進士。官永平知府。正德十三年駕幸昌平,民間婦女驚避。思義下令言:「大喪未舉,車駕必不遠出。非有文書,妄稱駕至擾民者,治以法。」鎮守中官郭原與思義有隙,以聞。立逮下詔獄,繫半歲,謫雲南安寧知州。嘉靖中,累遷副都御史、應天巡撫。

胡文璧,耒陽人。弘治十二年進士。正德初,由戶部郎中改御史。出知鳳陽,遷天津副使。中官張忠督直沽皇莊,縱群小牟利,文璧捕治之。為所構,械繫詔獄,謫延安府照磨。嘉靖初,累官四川按察使。

王相,光山人。正德三年進士。官御史。十二年巡按山東。鎮守中官黎鑑假進貢苛斂,相檄郡縣毋輒行。鑒怒,誣奏於朝。逮繫詔獄,謫高郵判官。未幾卒。嘉靖初,贈光祿少卿。

董相,嵩縣人。正德六年進士。官御史,巡視居庸諸關。江彬遣小校米英執人於平谷,恃勢橫甚。相收而仗之,將以聞。彬遽譖於帝,械繫詔獄,謫判徐州。嘉靖初,召復故官。終山東副使。

劉士元,彭縣人。正德六年進士。官御史,巡按畿輔。十三年,帝獵古北口,將招朵顏衛花當、把兒孫等燕勞。士元陳四不可。先是,帝幸河西務,指揮黃勳假供奉擾民,士元按之。勳懼,逃赴行在,因嬖倖譖於帝,云:士元聞駕至,令民間盡嫁其女,藏匿婦人。帝怒,命裸縛面訊之。野次無杖,取生柳乾痛笞之四十,幾死,囚檻車馳入京。並執知縣曹俊等十餘人,同繫詔獄。都御史王璟及科道陳霑、牛天麟等交章論救,不報。謫麟山驛丞。世宗立,復故官,出為湖州知府,遷湖廣副使。修荒政,積粟百萬餘石。事聞,被旌勞。嘉靖九年,屢遷右副都御史,巡撫貴州。居三年罷。

范輅,字以載,桂陽人。正德六年進士。授行人,除南京御史。武宗久無子,輅偕同官請擇宗室賢者育宮中,以宋仁宗為法,不報。先後劾中官黎安、劉瑯及衛官簡文、王忠罪。又論馬姬有娠,不當入宮。語皆切直。

尋命清軍江西。寧王宸濠令諸司以朝服見。輅不可。奏言:「高帝定制,王府屬僚稱官。後乃稱臣,其餘文武及京官出使者皆稱官。朝使相見以便服。今天下王府儀注,制未畫一。臣以為尊無二上,凡不稱臣者,皆不宜具朝服,以嚴大防。」章下禮官議。宸濠馳疏爭之,廷議請如輅言。宸濠伶人秦榮僭侈,輅劾治之。又劾鎮守太監畢真貪虐十五事,疏留不下。真乃摭他事誣之,遂逮下詔獄。值帝巡幸,淹繫經年。至十四年四月始謫龍州宣撫司經歷。未幾,宸濠及真謀逆誅,御史謝源、伍希儒等交章薦輅。未及召,世宗立,復故官。遷福建僉事,轉江西副使,致仕歸。又用胡世寧薦,起密雲兵備副使。討礦賊有功,歷江西、福建左、右布政使。卒官。

張欽,字敬之,順天通州人。正德六年進士。由行人授御史,巡視居庸諸關。

十二年七月,帝聽江彬言,將出關幸宣府。欽上疏諫曰:「臣聞明主不惡切直之言以納忠,烈士不憚死亡之誅以極諫。比者,人言紛紛,謂車駕欲度居庸,遠遊邊塞。臣度陛下非漫遊,蓋欲親征北寇也。不知北寇猖獗,但可遣將徂征,豈宜親勞萬乘?英宗不聽大臣言,六師遠駕,遂成己巳之變。且匹夫猶不自輕,陛下奈何以宗廟社稷之身蹈不測之險。今內無親王監國,又無太子臨朝。外之甘肅有土番之患,江右有皞賊之擾,淮南有漕運之艱,巴蜀有採辦之困;京畿諸郡夏麥少收,秋潦為沴。而陛下不虞禍變,欲縱轡長驅,觀兵絕塞,臣竊危已。」已,聞朝臣切諫皆不納,復疏言:「臣愚以為乘輿不可出者有三:人心搖動,供億浩繁,一也;遠涉險阻,兩宮懸念,二也;北寇方張,難與之角,三也。臣職居言路,奉詔巡關,分當效死,不敢愛身以負陛下。」疏入,不報。

八月朔,帝微行至昌平,傳報出關甚急。欽命指揮孫璽閉關,納門鑰藏之。分守中官劉嵩欲詣昌平朝謁,欽止之曰:「車駕將出關,是我與君今日死生之會也。關不開,車駕不得出,違天子命,當死。關開,車駕得出,天下事不可知。萬一有如『土木』,我與君亦死。寧坐不開關死,死且不朽。」頃之,帝召璽。璽曰:「御史在,臣不敢擅離。」乃更召嵩。嵩謂欽曰:「吾主上家奴也,敢不赴。」欽因負敕印手劍坐關門下曰:「敢言開關者,斬。」夜草疏曰:「臣聞天子將有親征之事,必先期下詔廷臣集議。其行也,六軍翼衛,百官扈從,而後有車馬之音,羽旄之美。今寂然一不聞,輒云『車駕即日過關』,此必有假陛下名出邊勾賊者。臣請捕其人,明正典刑。若陛下果欲出關,必兩宮用寶,臣乃敢開。不然萬死不奉詔。」奏未達,使者復來。欽拔劍叱之曰:「此詐也。」使者懼而返,為帝言「張御史幾殺臣」。帝大怒,顧朱寧:「為我趣捕殺御史。」會梁儲、蔣冕等追至沙河,請帝歸京師。帝徘徊未決,而欽疏亦至,廷臣又多諫者,帝不得已乃自昌平還,意怏怏未已。又二十餘日,欽巡白羊口。帝微服自德勝門出,夜宿羊房民舍,遂疾馳出關,數問「御史安在」?欽聞,追之,已不及。欲再疏諫,而帝使中官谷大用守關,禁毋得出一人。欽感憤,西望痛哭。於是京師盛傳「張御史閉關三疏」云。明年,帝從宣府還。至關,笑曰:「前御史阻我,我今已歸矣」,然亦不之罪也。

世宗嗣位,出為漢中知府。累官太僕卿。嘉靖十七年以右副都御史巡撫四川。召為工部左侍郎,被論罷。

欽初姓李。既通顯,始復其姓。事父母孝。有不悅,長跪請,至解乃已。

周廣,字克之,崑山人。弘治十八年進士。歷知莆田、吉水二縣。

正德中,以治最征授御史,疏陳四事,略言:

三代以前,未有佛法。況剌麻尤釋教所不齒。耳貫銅環,身衣赭服,殘破禮法,肆為淫邪。宜投四裔,以禦魑魅。奈何令近君側,為群盜興兵口實哉!昔禹戒舜曰:「毋若丹朱傲,惟慢遊是好。」周公戒成王曰:「毋若商王紂之迷亂,酗於酒德。」今之伶人,助慢遊迷亂者也。唐莊宗與伶官戲狎,一夫夜呼,倉皇出走。臣謂宜遣逐樂工,不得籍之禁內,乃所以放鄭聲也。

陛下承祖宗統緒,而群小獻媚熒惑,致三宮鎖怨,蘭殿無征。雖陛下春秋鼎盛,獨不思萬世計乎?中人稍有資產,猶畜妾媵以圖嗣續。未有專養螟蛉,不顧祖宗繼嗣者也。義子錢寧本宦豎蒼頭,濫寵已極,乃復攘敚貨賄,輕蔑王章。甚至投刺於人,自稱皇庶子。僭踰之罪所不忍言。陛下何不慎選宗室之賢者,置諸左右,以待皇嗣之生。諸義兒、養子俱奪其名爵,乃所以遠佞人也。

近兩京言官論大臣禦寇不職者,陛下率優容,即武將失律亦赦不誅。故兵氣不揚,功成無日,川原白骨,積如丘山。夫出師十萬,日費千金。今海內困憊已骨見而肉消矣,諸統兵大臣如陳金、陸完輩可任其優游玩寇,不加切責哉!請定期責令成功,以贖前罪。

寧見疏大怒,留之不下,傳旨謫廣東懷遠驛丞。主事曹琥救之,亦被謫。寧怒不已,使人遮道刺廣。廣知之,易姓名,變服,潛行四百餘里乃免。武定侯郭勛鎮廣東,承寧風旨以白金試廣,廣拒不受。伺廣謁御史,攝致軍門,箠繫幾死,御史救之始解。越二年,遷建昌知縣,有惠政。寧矯旨再謫竹寨驛丞。

世宗即位,復故官,歷江西副使,提督學校。嘉靖二年舉治行卓異,擢福建按察使。鎮守中官以百金饋,廣貯之庫,將劾之。中官懼,謝罪,自是不敢撓。六年,以右僉都御史巡撫江西,墨吏望風去。將限豪右田,不果。明年拜南京刑部右侍郎。居二年,暴疾卒。嘉靖末,贈右都御史。

廣初以鄉舉入太學,師章懋。在里閈,與魏校友善。平生嚴冷無笑容。居官公彊,弗受請託,士類莫不憚之。

曹琥,字瑞卿,巢人。弘治十八年進士。授南京工部主事,改戶部。既抗疏救廣,吏部擬調河南通判。寧欲遠竄,乃改尋甸,再遷廣信同知。寧王暨鎮守中貴託貢獻,頻有徵斂。琥攝府事,堅持不予,士民德之。擢鞏昌知府,未任卒。嘉靖初,贈光祿卿。

石天柱,字季瞻,岳池人。正德三年進士。當除給事中,吏科李憲請如御史例,試職一年,授戶科試給事中。乾清宮災,上言:「今日外列皇店,內張酒館。寵信番僧,從其鬼教。招集邊卒,襲其衣裝。甚者結為昆弟,無復尊卑。數離深宮,馳驅郊外。章疏置之高閣,視朝月止再三。視老成為贅疣,待義子以心腹。時享不親,慈闈罕至。不思前星未耀,儲位久虛。既不常御宮中,又弗預選宗室。何以消禍本,計久長哉!」屢遷工科都給事中。

十一年,都督馬昂進其女弟,已有娠,帝嬖之。天柱率同官合詞抗論,未報。又上疏曰:「臣等請出孕婦,未蒙進止。竊疑陛下之意將遂立為己子歟?秦以呂易羸而羸亡,晉以牛易馬而馬滅。彼二君者,特出不知,致墮奸計。謂陛下亦為之耶?天位至尊,神明之胄,尚不易負荷,而況么麼之子。借使以陛下威力成於一時,異日諸王宗室肯坐視祖宗基業與他人乎?內外大臣肯俯首立於其朝乎?望急遣出,以清宮禁,消天下疑。」卒不報。

泰山有碧霞元君祠,中官黎鑑請收香錢為修繕費。天柱言祀典惟有東嶽神,無所謂碧霞元君者。淫祀非禮,不可許。十二年四月詔毀西安門外鳴玉、積慶二坊民居,有所營建,天柱等疏請停止。帝皆不省。

是年,帝始巡遊塞外,營鎮國府於宣府,天柱率同官力諫。孝貞純皇后將葬,帝假啟土為名,欲復巡幸。天柱念帝盤遊無度,廷臣雖諫,帝意不回,思所以感動之者,乃刺血草疏。略曰:

臣竊自念,生臣之身者,臣之親也。成臣之身者,累朝之恩也。感成身之恩欲報之於陛下者,臣之心也。因刺臣血,以寫臣心,明臣愚忠,冀陛下憐察。數年以來,星變地震,大水奇荒,災異不可勝數,而陛下不悟,禍延太皇太后。天之意,欲陛下居衰絰中,悔過自新,以保大業也。尚或不悟,天意或幾乎息矣。喪禮大事,人子所當自盡。陛下於太皇太后未能盡孝,則群臣於陛下必不能盡忠。不忠,將無所不至,猝有變故,人心瓦解矣。夫大位者,奸之窺也。昔太康田於洛、汭,煬帝行幸江都,皆以致敗,可不鑒哉!方今朝廷空,城市空,倉廩空,邊鄙空,天下皆知危亡之禍,獨陛下不知耳。治亂安危,在此行止。此臣所痛心為陛下惜,復昧死為陛下言也。凡數千言。當天柱刺血時,恐為家人所阻,避居密室,雖妻子不知。既上,即易服待罪。聞者皆感愴,而帝不悟也。

踰月,兵部尚書王瓊欲因哈密事殺都御史彭澤。廷臣集議,瓊盛氣以待,眾不敢發言。天柱與同官王爌力明澤無罪,乃得罷為民。瓊怒,取中旨出兩人於外,天柱得臨安推官。世宗即位,召復舊職。遷大理丞,未幾卒。久之,子請恤,特予祭。

贊曰:諫臣之職,在糾慝弼違。諸臣戒盤遊,斥權倖,引義力爭,無忝厥職矣。武宗主德雖荒,然文明止於遠竄,入關不罪張欽,其天姿固非殘暴酷烈者比。而義兒、閹豎,煬灶為奸。桁楊交錯於闕庭,忠直負痛於狴戶。批鱗者尚獲生全,投鼠者必陷死地。元氣日削,朝野震驚,祚以不延,統幾中絕。風愆之訓,垂戒不亦切乎。


\end{pinyinscope}