\article{宣宗本紀}

\begin{pinyinscope}
宣宗憲天崇道英明神聖欽文昭武寬仁純孝章皇帝,諱瞻基,仁宗長子也。母誠孝昭皇后。生之前夕,成祖夢太祖授以大圭曰:「傳之子孫,永世其昌。」既彌月,成祖見之曰:「兒英氣溢面,符吾夢矣。」比長,嗜書,智識傑出。

永樂七年,從幸北京,令觀農具及田家衣食,作《務本訓》授之。八年,成祖征沙漠聞於齊,人稱賢師。,命留守北京。九年十一月,立為皇太孫,始冠。自是,巡幸征討皆從。嘗命學士胡廣等即軍中為太孫講論經史。每語仁宗曰:「此他日太平天子也。」仁宗即位,立為皇太子。

夏四月,以南京地屢震,命往居守。五月庚辰,仁宗不豫,璽書召還。六月辛丑,還至良鄉,受遺詔,入宮發喪。庚戌,即皇帝位。大赦天下,以明年為宣德元年。辛亥,諭邊將嚴守備。甲寅,趣中官在外採辦者還,罷所市物。

秋七月乙亥,尊皇后為皇太后,立妃胡氏為皇后。辛卯,鎮遠侯顧興祖討大藤峽蠻,平之。乙未,諭法司慎刑獄。閏月戊申,安順伯薛貴、清平伯吳成、都督馬英、都指揮梁成帥師巡邊。乙丑,楊溥入直文淵閣。八月戊辰,都指揮李英討安定曲先叛番,大敗之,定定王桑兒加失夾詣闕謝罪。壬申,詔內外群臣舉廉潔公正堪牧民者。癸未,大理卿胡概、參政葉春巡撫南畿、浙江。設巡撫自此始。九月壬寅,葬昭皇帝於獻陵。

冬十月戊寅,南京地震,戊子,敕公、侯、伯、五府、六部、大學士、給事中審覆重囚。十一月戊戌,顧興祖討平思恩蠻。辛酉,恭祿為鎮朔大將軍巡邊。十二月甲申,顧興祖討平宜山蠻。

是年,哈密回回、滿剌撒丁、占城、琉球中山、爪哇、烏斯藏、瓦剌、浡泥入貢。

宣德元年春正月癸卯,享太廟。丁未,太祀天地於南郊。癸丑,赦死罪以下運糧宣府自贖。己未,遣恃郎黃宗載十五人清理天下軍伍。後遣使,著為令。二月戊辰,祭社稷。丁丑,耕耤田。丙戌,謁長陵、獻陵。丁亥,還宮。三月己亥,榮昌伯陳智、都督方政討黎利,敗績於茶籠州,乂安知府琴彭死之。癸丑,行在禮部侍郎張瑛兼華蓋殿大學士,直文淵閣。

夏四月乙丑,成山侯王通為征夷將軍充總兵官,討黎利,尚書陳洽參贊軍務,陳智、方政奪官從立功。五月甲午朔,錄囚。丙申,詔赦交阯,許黎利自新。丙午,敕郡縣瘞遺骸。庚申,召薛祿還。

秋七月癸巳,京師地震,乙未,免山東夏稅。己亥,諭六科,凡中官傳旨,必覆奏始行。壬子,罷湖廣採木。八月壬戌,漢王高照反。丙寅,宥武臣殊死以下罪,復其官。己巳,親征高煦,命鄭王瞻、襄王瞻墡居守,陽武侯薛祿、清平伯吳成將前鋒,大賚五軍將士。辛未,發京師。辛巳,至樂安,帝兩遣書諭降,又以敕繫矢射城中諭禍福。壬午,高煦出降。癸未,改樂安曰武定州。九月乙酉,班師。丙申,至自武定州,錮高煦於西內。戊戌,法司鞫高煦同謀者,詞連晉王、趙王,詔勿問。

冬十月戊寅,釋李時勉,復為侍讀。十一月乙未,成山侯王通擊黎利於應平,敗績,尚書陳洽死之。十二月辛酉,免六師所過秋糧。辛未,錄囚,宥免三千餘人。乙酉,征南將軍總兵官黔國公沐晟帥興安伯徐亨、新寧伯譚忠,征虜副將軍安遠侯柳升帥保定伯梁銘都督崔聚,由雲南、廣西分道討黎利,兵部尚書李慶參贊軍務。

是年,爪哇、暹羅、琉球、蘇門答剌、滿剌加、白葛達、撒馬兒罕、土魯番、哈密、烏斯藏入貢。

二年春正月庚子,大祀天地於南郊。丁未,有司奏歲問囚數。帝謂百姓輕犯法,由於教化未行,命申教化。二月癸亥,行在戶部待郎陳山為本部尚書兼謹身殿大學士,直文淵閣。乙丑,黎利攻交阯城,王通擊敗之。三月辛卯,賜馬愉等進士及第、出身秀差。

夏四月庚申,黎利陷昌江,都指揮李任,指揮顧福、劉順,知府劉子輔,中官馮智死之。甲子,晉王濟熿有罪,廢為庶人。己巳,王通許黎利和。五月癸巳,薛祿督餉開平。己亥,仁宗神主祔太廟。丙午,錄囚。六月戊寅,錄囚。

秋七月己亥,黎利陷隘留關,鎮遠侯顧興祖擁兵不救,逮治之。庚子,錄囚。辛丑,命都督同知陳懷充總兵官,帥師討松潘蠻。丁未,薛祿敗敵於開平。八月甲子,黃淮致仕。免兩京、山西、河南州縣被災稅糧。九月壬辰,錄囚。乙未,柳升師次倒馬坡,遇伏戰死。是日,保定伯梁銘病卒。丙申,尚書李慶病卒。師大潰,參將崔聚,郎中史安,主事陳鏞、李宗昉死之。

冬十月戊寅,王通棄交阯,與黎利盟。十一月乙酉,赦黎利,遣侍郎李琦、羅汝敬立陳暠為安南國王,悉如文武吏士還。己亥,以皇長子生大赦天下,免明年稅糧三之一。十二月丁丑,振陜西饑。并給絹布十五萬疋。

是年,爪哇、占城、暹羅、琉球、瓦剌、哈密、亦力把里、撒馬兒罕入貢。

三年春正月甲午,大祀天地於南郊。丙申,陳懷平松潘蠻。二月戊午,立皇長子祁鎮為皇太子。是月,作《帝訓》成。三月癸未,廢皇后胡氏,立貴妃孫氏為皇后。壬辰,錄囚。

夏四月癸亥,敕凡官民建言章疏,尚書、都御史、給事中會議以聞,勿諱。閏月壬寅,錄囚。免山西旱災稅糧。甲辰,命有司振恤。庚戌,論棄交阯罪,王通等及布政使弋謙、中官山壽、馬騏下獄論死,籍其家,鎮遠侯顧興祖并下獄。五月壬子,李琦、羅汝敬還。黎利表陳暠卒,子孫并絕,乞守國俟命。辛酉,錄囚。己巳,復遣羅汝敬等諭黎利立陳氏後。辛未,贈交阯死事諸臣。壬申,免北京被災夏稅。六月丙戌,免陜西被災夏稅。丁未,都御史劉觀巡視河道。

秋七月戊辰,錄囚。八月辛卯,罷北京行部及行後軍都督府。丁未,帝自將巡邊。九月辛亥,次右門驛。兀良哈寇會州,帝帥精卒三千人往擊之。己卯,出喜峰口,擊寇於寬河。帝親射其前鋒,殪三人,兩翼軍並發,大破之。寇望見黃龍旂,下馬羅拜請降,皆生縛之,斬渠酋。甲子,班師。癸酉,至自喜峰口。

冬十一月癸酉,錦衣指揮鐘法保請採珠東莞,帝曰:「是欲擾民以求利也」,下之獄。十二月庚子,廣西總兵官山雲討擒忻城蠻。

是年,占城、暹羅、爪哇、琉球、瓦剌、哈密、安南、曲先、土魯番、亦力把里、撒馬兒罕入貢。

四年春正月,兩京地震。己未,大祀天地於南郊。二月己丑,南京獻騶虞二,禮部請表賀,不許。三月甲戌,遣李琦再諭黎利訪立陳氏後。

夏四月辛巳,山雲討平柳、潯蠻。戊子,工部尚書黃福、平江伯陳瑄經略漕運。五月壬子,錄囚。六月甲午,罷文吏犯贓贖罪例。己亥,寇犯開平,鎮撫張信等戰死。庚子,薛祿督餉開平。

秋七月己未,幸文淵閣。八月己卯,起復楊溥。九月癸亥,釋顧興祖於獄。

冬十月庚辰,幸文淵閣。癸未,以天寒諭法司錄囚。丙戌,製《猗蘭操》賜廷臣,諭以薦賢為國之道。庚寅,張瑛、陳山罷。甲午,閱武於近郊。乙未,獵於峪口。戊戌,還宮。十一月癸卯,薛祿及恭順侯吳克忠帥師巡宣府。十二月乙亥,京師地震。壬辰,罷中官松花江造船。

是年,爪哇、占城、琉球、榜葛刺、哈密、土魯番、亦力把里、撒馬兒罕入貢。

五年春正月癸丑,大祀天地於南郊。戊辰,尚書夏原吉卒。二月壬辰,罷工部採木。癸巳,頒寬恤之令,省災傷,寬馬政,免逋欠薪芻,招流民賜復一年,罷採買,減官田舊科十之三,恤工匠,禁司倉官包納,戒法司慎刑獄。乙未,奉皇太后謁陵。三月戊申,道見耕者,下馬問農事,取耒三推,顧侍臣曰:「朕三推巳不勝勞,況吾民終歲勤動乎。」命賜所過農民鈔,己酉,還宮。辛亥,李琦還,黎利稱陳氏無後,上表請封。丙辰,免山西去歲被災田租。丁巳,賜林震等進士及第、出身有差。

夏四月戊寅,薛祿帥師築赤城、雕鶚、雲州、獨石、團山城堡。五月癸卯,追奪贓吏誥敕,著為令。丙辰,修預備倉,出官錢收糴備荒。癸亥,擢郎中況鐘、御史何文淵九人為知府,賜敕遣之。六月己卯,遣官捕近畿蝗,諭戶部曰:「往年捕蝗之使害民不減於蝗,宜知此弊。」因作《捕蝗詩》示之。

秋七月癸亥,甄別守令。八月己巳朔,日食,陰雨不見,禮官請表賀,不許。九月丙午,擢御史於謙、長史周忱六人為侍郎,巡撫兩京、山東、山西、河南、江西、浙江、湖廣。乙卯,巡近郊。己未,還宮。

冬十月乙亥,阿魯台犯遼東,遼海衛指揮同知皇甫斌力戰死。丙子,巡近郊。己卯,獵於坌道。丙戌,至洗馬林,遍閱城堡兵備。壬辰,還宮。十二月癸巳,曲先叛番平。閏月己未,敕內外諸司,久淹獄囚者罪之。

是年,占城、琉球、爪哇、瓦剌、哈密、罕東、土魯番、撒馬兒罕、亦力把里入貢。

六年春正月丁丑,大祀天地於南郊。庚辰,大雨雷電。二月丁酉,侍郎羅汝敬督陜西屯田。己亥,濬金龍口,引河達徐州以便漕。三月乙亥,命吏部考察外官自布政、按察二司始,著為令。

夏四月己酉,侍郎柴車經理山西屯田。六月己亥,遣使詔黎利權署安南國事。

秋七月己巳,錄囚。壬午,許朵顏三衛市易。冬十月甲辰,陳懷平松潘蠻。十一月丙子,始命官軍兌運民糧。乙酉,分遣御史往逮貪暴中官袁琦等。十二月乙未,袁琦等十一人棄市,榜其罪示天下。丁未,金幼孜卒。庚戌,遣御史巡視寧夏甘州屯田水利。

是年,占城、琉球、瓦剌、哈密、蘇門答剌、亦力把里入貢。

七年春正月辛酉朔,日有食之,免朝賀。癸酉,大祀天地於南郊。二月甲午,以春和諭法司錄囚。三月庚申,下詔行寬恤之政。辛酉,諭禮部曰:「朕以官田賦重,十減其三。乃聞異時蠲租詔下,戶部皆不行,甚者戒約有司,不得以詔書為辭。是廢格詔令,使澤不下究也。自今令在必行,毋有所遏。」

夏四月辛丑,免山西逋賦。壬寅,募商中鹽輸粟入邊。六月癸卯,錄囚。癸丑,罷中官入番市馬。是月,作《官箴》成,凡三十五篇,示百官。

秋八月乙未,敕京官三品以上舉才行文學之士,吏部、都察院黜方面有司不職者。九月庚午,諸將巡邊。是秋,免兩畿及嘉興。湖州水災稅糧。

冬十一月辛酉,召督漕平江伯陳瑄、侍郎趙新等歲終至京議糧賦利弊。

是年,占城、琉球、哈密、哈烈、瓦剌、亦力把里入貢。

八年春正月丁卯,大祀天地於南郊。二月壬子,錄囚,宥免五千餘人。三月丙辰,賜曹鼐等進士及第、出身有差。庚辰,諭內外優恤軍士,違者風憲官察奏罪之。是春,以兩京、河南、山東、山西久旱,遣使振恤。

夏四月戊戌,詔蠲京省被災逋和、雜課,免今年夏稅,賜復一年。理冤獄。減殊死以下,赦軍匠在逃者罪。有司各舉賢良方正一人。巡按御史、按察使糾貪酷吏及使臣生事者。五月丁巳,總兵官都督蕭授討平貴州烏羅蠻。丁卯,山雲討平宜山蠻。六月乙酉,禱雨不應,作《閔旱詩》示群臣。辛丑,詔中外疏決罪囚。是夏,復振兩京、河南、山東、山西、湖廣饑,免稅糧。

秋七月壬申,免江西水災稅糧。八月癸巳,汰京師冗官。閏月辛亥,西域貢麒麟。戊午,景星見。禮官請表賀,皆不許。九月乙酉,遣官錄天下重囚。己亥,阿魯台部昝卜寇涼州,總兵官劉廣擊斬之。

冬十二月乙亥,諭法司宥京官過犯。

是年,暹羅、占城、琉球、安南、滿剌加、天方、蘇門答剌、古里、柯枝、阿丹、錫蘭山、佐法兒、甘巴里、加異勒、忽魯謨斯、哈密、瓦剌、撒馬兒罕、亦力把里入貢。

九年春正月辛卯,大祀天地於南郊。二月庚戌,振鳳陽、淮安、揚州、徐州饑。乙卯,申兩京、山東、山西、河南寬恤之令。三月戊寅,山雲討癥思恩叛蠻。

夏四月己未,黎利死,子麟來告喪,命麟權署安南國事。戊辰,錄囚。五月壬午,瘞暴骸。

秋七月甲申,遣給事中、御史、錦衣衛官督捕兩畿、山東、山西、河南蝗。八月庚戌,振湖廣饑。甲子,敕兩京、湖廣、江西、河南巡撫、巡按御史、三司官行視災傷,蠲秋糧十之四。乙丑,罷工部採辦。己巳,瓦剌脫歡攻殺阿魯台,來告捷。九月癸未,自將巡邊。乙酉,度居庸關。丙戌,獵於坌道。乙未,阿魯台子阿卜只俺來歸。丁酉,至洗馬林,閱城堡兵備。己亥,大獵。

冬十月丙午,還宮。丙辰,都督方政討平松潘叛蠻。甲子,罷陜西市馬。丁卯,兩畿、浙江、湖廣、江西饑,以應運南京及臨清倉粟振濟。十一月戊戌,停刑。庚子,免四川被災稅糧。十二月甲子,帝不豫,衛王瞻埏攝享太廟。

是年,暹羅、占城、琉球、蘇門答剌、哈密、瓦剌入貢。

十年春正月癸酉朔,不視朝,命群臣謁皇太子於文華殿。甲戌,大漸。罷買、營造諸使。乙亥,崩於乾清宮,年三十有八。遺詔國家重務白皇太后。丁酉,上尊謚,廟號宣宗,葬景陵。

贊曰:仁宗為太子,失愛於成誼。其危而復安,太孫蓋有力焉。即位以後,吏稱其職,政得其平,綱紀修明,倉庾充羨,閭閻樂業。歲不能災。蓋明興至是歷年六十,民氣漸舒,蒸然有治平之象矣。若乃強籓猝起,旋即削平,掃蕩邊塵,狡寇震懾,帝之英姿睿略,庶幾克繩祖武者歟。


\end{pinyinscope}