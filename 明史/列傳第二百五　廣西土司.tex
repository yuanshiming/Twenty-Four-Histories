\article{列傳第二百五 廣西土司}

\begin{pinyinscope}
廣西瑤、僮居多,盤萬嶺之中,當三江之險,六十三山倚為巢穴,三十六源踞其腹心,其散布於桂林、柳州、慶遠、平樂諸郡縣者,所在蔓衍。而田州、泗城之屬,尤稱強悍。種類滋繁,莫可枚舉。蠻勢之眾,與滇為埒。今就其尤著者列於篇。觀其叛服不常,沿革殊致,可以覘中國之德威,知夷情之順逆,為籌邊者之一助云。

○廣西土司一

△桂林柳州慶遠平樂梧州潯州南寧

桂林,自秦置郡,漢始安,唐桂州,天寶改建陵,宋靜江府,元靜江路。明初,改桂林府為廣西布政司治所,屬內地,不當列於土司。然廣西惟桂林與平樂、潯州、梧州未設土官,而無地無瑤、僮。桂林之古田,平樂之府江,潯州之藤峽,梧州之岑溪,皆煩大征而後克,卒不能草薙而獸獮之,設防置戍,世世為患,是亦不得而略焉。

洪武七年,永、道、桂陽諸州蠻竊發,命金吾右衛指揮同知陸齡率兵討平之。二十二年,富川縣逃吏首賜糾合苗賊盤大孝等為亂,殺知縣徐元善等,往來劫掠。廣西都指揮韓觀遣千戶廖春等討之,擒殺大孝等二百餘人。觀因言:「靈亭鄉乃瑤蠻出入地,雖征剿有年,未盡殄滅,宜以桂林等衛贏餘軍士,置千戶所鎮之。」詔從其請。二十七年,全州灌陽等縣平川諸源瑤民,聚眾為亂。命湖廣、廣西二都司發兵討之,擒殺千四百餘人,諸瑤奔竄遁去,置灌陽守禦千戶所。初,灌陽縣隸湖廣,因廣西平川等三十六源瑤賊作亂,攻擊縣治,詔寶慶衛指揮孫宗總兵討平之。縣丞李原慶因奏灌陽去湖廣遠,隸廣西近,遂以灌陽隸桂林府千戶所,命廣西都指揮同知陶瑾領兵築城守之。

永樂二年,總兵韓觀奏:「潯、桂、柳三郡蠻寇黃田等累行劫掠,殺擄人畜。已調都指揮朱輝追剿,斬獲頗多。尋蒙遣官齎敕撫安,其黃田等瑤皆已向化,悉歸所擄人畜。」帝命觀,復業者善撫恤之。宣德六年,都督山雲奏:「廣西左、右兩江設土官衙門大小四十九處,蠻性無常,仇殺不絕。朝廷每命臣同巡按御史三司官理斷,緣諸處皆瘴鄉,兼有蠱毒,三年之間,遣官往彼,死者凡十七人,事竟不完。今同眾議,凡土官衙門軍務重事,徑詣其處。其餘爭論詞訟,就所近衛理之。」報可。

景泰五年,廣西古丁等洞賊首藍伽、韋萬山等,糾合蠻類,劫掠南寧、上林、武緣諸處。鎮守副總兵陳旺以聞,詔令總督馬昂等剿捕之。初,桂林、古田僮種甚繁,最強者曰韋,曰閉,曰白,而皆並於韋。賊首韋朝威據古田,縣官竄會城,遣典史入縣撫諭,烹食之。弘治間,大征,殺副總兵馬俊、參議馬鉉。正德初再徵,殺通判、知縣、指揮等官。嘉靖初,又徵之,殺指揮舒松等。時韋銀豹與其從父朝猛攻陷洛容縣,據古田,分其地為上、下六里。銀豹出掠,挾下六里人行,而上六里不與焉。四十五年,提督吳桂芳因其閑,遣典史廖元入上六里撫諭之,諸僮復業者二千人,銀豹勢孤請降。久之,復猖獗,嘗挾其五子據鳳皇、連水二寨,襲殺昭平知縣魏文端。更自永福入桂林劫布政司庫,殺署事參政黎民衷,縋城而去,官軍追不及。久之,臨桂、永福各縣兵群起捕賊,始得賊黨扶嫩、土婆顯等三十餘人於各山寨中。

時首惡未獲,隆慶三年,朝議以廣西專設巡撫,推江西按察使殷正茂為僉都御史以往。正茂至,奏請剿賊,合土漢兵十萬,集眾議。時八寨助逆,眾議先剿,敕書亦有先平八寨,徐圖古田之語。正茂獨不謂然,先給榜諭八寨,八寨聽命。然後分兵七哨,以總兵俞大猷統之,使副總兵門崇文,參將王世科、黃應甲,都司董龍、魯國賢,遊擊丁山等各領一哨,復分土兵為二隊,更番清道,必先清數里而後行。及至其巢,合營攻之,斬七千四百六十餘級,生擒朝猛,梟於軍,俘獲男女千餘口。銀豹窮蹙,擇肖己者斬首獻,捷聞。既而生縛銀豹并其子扶枝膠送京師,斬之。古田平。乃並八寨與龍哈、咘咳為十寨,立長官司,以黃昌等為長官及土舍,聽守禦調度。更升古田縣為永寧州。已而永寧僮韋狼要與其黨黃銀成有隙,相仇殺,常安巡檢欲窮治之。狼要遂與右江荔浦山灣諸僮稱亂。命指揮徐民瞻將兵捕之,民瞻伏兵執狼要,諸瑤大訌。總制殷正茂、巡撫郭應聘乃檄徵田州、向武、都康諸土兵,屬參將王瑞進剿,斬廖金鑑、廖金盞、韋銀花、韋狼化等。萬曆六年,總制潛雲翼、巡撫吳文華大征河池、咘咳諸瑤,斬首四萬八百餘級,嶺表悉平。

柳州置自唐貞觀中,明初移治于馬平。所屬州二,縣十。內屬千餘年,惟上林縣尚為土官,而賓、象、融、羅諸瑤蠻蟠結為寇,城外五里即賊巢,軍民至無地可田。後屢加征剿,置土巡檢於各峒隘,稍稱寧焉。

洪武二年,中書省臣言:「廣西諸峒雖平,宜遷其人入內地,可無邊患。」帝曰:「溪洞蠻僚雜處,其人不知禮義,順之則服,逆之則變,未可輕動。惟以兵分守要害以鎮服之,俾日漸教化,數年後,可為良民,何必遷也。」

永樂七年,柳州道村寨蠻韋布黨等作亂,都指揮周誼率兵討擒之。命斬布黨,梟其首於寨。廣西洞蠻韋父、融州羅城洞蠻潘父旂各聚眾為亂,柳州等衛官軍捕斬之。九年,賓州遷江縣、象州武仙縣古逢等洞蠻僚作亂。詔發柳州、南寧、桂林等衛兵討之。十四年,融州瑤民作亂,官軍討平之。十七年,象州土吏覃仁用言,其父景安,故元時常任本州巡檢,有兵僮二百人,今皆為民,請收集為軍。帝不許。十九年,融縣蠻賊五百餘人,群聚剽掠,廣西參政耿文彬率民兵會桂林衛指揮平之。柳州等府上林等縣僮民梁公竦等六千戶,男女三萬三千餘口,及羅城縣土酋韋公、成乾等三百餘戶復業。初,韋公等倡亂,僮民多亡入山谷,與之相結。事聞,遣御史王煜等招撫復業,至是俱至,仍隸籍為民。

宣德初,蠻寇覃公旺作亂,據思恩縣大、小富龍三十餘峒,固守險阻,以拒官軍。總兵官顧興祖等督兵分道攻之,斬公旺並其黨千五十餘人。捷至,帝曰:「蠻民亦朕赤子,殺至千數,豈無脅從非辜者。以後宜開示恩信,撫慰而降之,如賈琮戍交州可也。」元年,柳州僮首韋敬曉等歸附。二年,廣西三司奏:「柳慶等府賊首韋萬黃、韋朝傳等聚眾劫殺為民害。」敕興祖進兵剿平之。

懷遠為柳州屬邑,在右江上游,旁近靖綏、黎平,諸瑤竊據久。隆慶時,大征古田,懷遠知縣馬希武欲乘間築城,召諸瑤役之,許犒不與。諸瑤遂合繩坡頭、板江諸峒,殺官吏反。總制殷正茂請於朝,遣總兵官李錫、參將王世科統兵進討。官兵至板江,瑤賊皆據險死守。正茂知諸瑤獨畏永順鉤刀手及狼兵,乃檄三道兵數萬人擊太平、河裏諸村,大破之,連拔數寨,斬賊首榮才富、吳金田等,前後捕斬凡三千餘,俘獲男婦及牛馬無算。事聞,議設兵防,改萬石、宜良、丹陽為土巡司,屯土兵五百人,且耕且守。

萬曆元年,洛容知縣邵廷臣以養歸,主簿謝漳行縣事。會上元夜,單騎巡檄山中。僮蠻韋朝義率上油、古底諸僮夜半出掠,逐漳,追至城,殺漳,奪縣印去。是夜,指揮朱昌胤、土巡檢韋顯忠共提兵決戰,斬首三十一級,兵校文斌獲朝義,奪還縣印,守巡官以聞。乃命總兵李錫,參將王瑞、康仁等剿之,破上油、古底諸寨,斬覃金狼等二千八百三十餘級,俘二百二十餘人,牛馬器械稱是。後殘僮黃朝貴復合融縣瑤號萬人,聲言欲入富福鎮。王世科復引兵擊之,斬五十餘人。始洛容在萬山中,城小無雉堞,縣官皆寓府城,知縣餘涵請遷城於白龍巖,不果,至是謝漳遂及於難。

又韋王朋者,馬平僮也。初平馬平時,因建營堡,使土舍韋志隆提兵屯其地。王朋視堡兵如仇,常率東歐、大產諸蠻要挾營堡。兵備周浩使千總往撫,遂殺千總,劫村落,總兵王尚父剿平之。

慶遠,秦象郡,漢交阯、日南二郡界,後淪於蠻。唐始置粵州,天寶初,改龍水郡,屬嶺南道,乾符中,更宜州。宋陞慶遠軍節度,咸淳初,改慶遠府。元為慶元路。洪武元年仍改慶遠府。時征南將軍楊文既平廣西,二年,行省臣言:「慶遠府地接八番溪洞,所轄南丹、宜山等處,宋、元皆用其土酋安撫使統之。天兵下廣西,安撫使莫天護首來款附,宜如宋、元制,錄用以統其民,則蠻情易服,守兵可滅。」帝從之,詔改慶遠府為慶遠南丹軍民安撫司,置安撫使、同知、副使、經歷、知事各一員,以天護為同知,王毅為副使。三年,行省臣言:「慶遠故府也,今為安撫司,其地皆深山曠野,其民皆安撫莫天護之族。天護素庸弱,宗族強者,動肆跋扈,至殺河池縣丞蓋讓,與諸蠻相煽為亂,此豈可姑息以胎禍將來。乞罷安撫司,仍設府置衛,以守其地。」報可。乃命莫天護赴京。七年,賜廣西土官莫金文綺六匹,置南丹州,隸慶遠府,以莫金為知州。八年,那地縣土官羅貌來朝,以貌知縣事。

二十八年,都指揮韓觀率兵捕獲宜山等縣蠻寇二千八百餘人,斬偽大王韋召,偽萬戶趙成秀、韋公旺等,傳首京師。時嶺南盛暑,官軍多病瘴,帝命觀班師。南丹土官莫金叛,帝命征南將軍楊文,龍州平後,移師討南丹、奉議等處。龍州趙宗壽來朝謝罪,貢方物。大軍進徵奉議,調參將劉真分道攻南丹,破之,執莫金併俘其眾。後遣寶慶衛指揮孫宗等分兵擊巴蘭等寨,蠻僚懼,焚寨遁去,官兵追捕斬之,蠻地悉定。詔置南丹、奉議、慶遠三衛,以官軍守之。

二十九年,廣西布政司言:「新設南丹等三衛及富川千戶所,歲用軍餉二十餘萬石,有司所征,不足以給。」帝命俱置屯田,給耕種。尋遣中使至桂林等府市牛給南丹、奉議諸衛軍士。都指揮姜旺、童勝率兵抵思恩縣鎮寧等村洞,殺獲叛蠻三千餘人,降一千一百餘戶,得故宋銅印一來上。

永樂二年,慶遠府言:「忻城、宜山二縣洞蠻陳公宣等出沒為寇,請剿捕。」帝命都指揮朱輝親往撫諭,公宣等相率歸附,凡千三十五戶。荔波縣民覃真保上言:「縣自洪武至今,人民安業,惟八十二洞瑤民未隸編籍。今聞朝廷加恩撫綏,咸願為民,無由自達,乞遣使招撫。」乃命右軍都督府移文都督韓觀遣人撫諭,其願為民者,量給賜賚,復其徭役三年。

宣德五年,總兵官山雲討慶遠蠻寇,斬首七千四百,平之。九年,雲奏:「思恩縣蠻賊覃公砦等累年作亂,今委都指揮彭義等率兵剿捕,斬賊首梁公成、潘通天等梟之,仍督官軍搜捕餘黨。」帝賜敕慰勞。又奏:「慶遠、鬱林等州縣蠻寇出沒,必宜剿除,而兵力不足。」帝命廣東都司調附近衛所精銳士卒千五百人,委都指揮一員,赴廣西,聽雲調用。十年,南丹土官莫禎來朝,貢馬,賜彩幣。正統四年,莫禎奏:「本府所轄東蘭等三州,土官所治,歷年以來,地方寧靖。宜山等六縣,流官所治,溪峒諸蠻,不時出沒。原其所自,皆因流官能撫字附近良民,而溪峒諸蠻恃險為惡者,不能鈐制其出沒。每調軍剿捕,各縣居民與諸蠻結納者,又先漏洩軍情,致賊潛遁。及聞招撫,詐為向順,仍肆劫掠,是以兵連禍結無寧歲。臣竊不忍良民受害,願授臣本州土官知府,流官總理府事,而臣專備蠻賊,務擒捕殄絕積年為害者。其餘則編伍造冊,使聽調用。據巖險者,拘集平地,使無所恃。擇有名望者立為頭目,加意撫恤,督勵生理。各村寨皆置社學,使漸風化。三五十里設一堡,使土兵守備,凡有寇亂,即率眾剿殺。如賊不除,地方不靖,乞究臣誑罔之罪。」帝覽其奏,即敕總兵官柳溥曰:「以蠻攻蠻,古有成說。今莫禎所奏,意甚可嘉,彼果能效力,省我邊費,朝廷豈惜一官,爾其酌之。」

弘治九年,總督鄧廷瓚言:「廣西瑤、僮數多,土民數少,兼各衛軍士十亡八九,凡有徵調,全倚土兵。乞令東蘭土知州韋祖鋐子一人,領土兵數千於古田、蘭麻等處撥田耕守,候平古田,改設長官司以授之。」廷議以古田密邇省治,其間土地多良民世業,若以祖鋐子為土官,恐數年之後,良民田稅皆非我有。欲設長官司,祗宜於土民中選補。廷瓚又言:「慶遠府天河縣舊十八里,後漸為僮賊所據,止餘殘民八里,請分設一長官司治之。」部議增設永安長官司,授土人韋萬妙等為正、副長官,并流官吏目一員。是年,裁忻城縣流官,留土官知縣掌縣事,亦從廷瓚奏也。十二年,韋祖鋐率兵五千助思恩岑浚攻田州,殺掠男女八百餘人,驅之溺水死者無算。副總兵歐磐詣田州,兵乃解。

嘉靖二十七年,那地州土官羅廷鳳聽調有勞,命襲替,免赴京。四十二年錄平瑤功,授東蘭州、那地州土官職。

慶遠領州四。河池,弘治中以縣陞州,改流官。其東蘭、那地、南丹皆土官。縣五,忻城土官。又長官司二,曰永安,永順。

東蘭州,在府城西南四百二十里。宋時有韋君朝者,居文蘭峒為蠻長,傳子宴鬧。崇寧五年內附,因置蘭州,以宴鬧知州事,俾世其官。元改為東蘭州,韋氏世襲如故。洪武十二年,土官韋富撓遣家人韋錢保詣闕,上元所授印,貢方物。錢保匿富撓名,以己名上,因以錢保知東蘭州。既而錢保徵斂暴急,民不堪命,擁富撓作亂。廣西都司討平之,執錢保正其罪,仍以其地歸韋氏。

那地州,在府城西南二百四十里。宋熙寧初,土人羅世念來降,授世職。崇寧五年,諸蠻納土,遂置地、那二州,以羅氏世知地州。大觀中,析地州置孚州。元仍為地、那二州。洪武元年,土官羅黃貌歸附,詔并那入地,為那地州,予印,授黃貌世襲土知州,以流官吏目佐之。

南丹州,宋開寶初,土官莫洪胭內附。元豐三年置南丹州,管轄諸蠻,歷世承襲。元至正末,莫國麒納土,命為慶遠南丹谿洞安撫使。明洪武初,安撫使莫天讓歸附。七年置州,授莫金知州,世襲,佐以流官吏目。金以叛誅,廢州置衛。後因其地多瘴,遷之賓州。既而蠻民作亂,復置土官知州,以金子莫祿為之。

忻城,宋慶曆間置縣,隸宜州。元以土官莫保為八仙屯千戶。洪武初,設流官知縣,罷管兵官,籍其屯兵為民,莫氏遂徙居忻城界。宣、正後,瑤、僮狂悻,知縣蘇寬不任職。瑤老韋公泰等舉莫保之孫誠敬為土官,寬為請於上官,具奏,得世襲知縣。由是邑有二令,權不相統,流官握空印,僦居府城而已。弘治間,總督鄧廷瓚奏革流官,土人韋保為內官,陰主之,始獨任土官。

永順司、永安司,舊為宜山縣。正統六年,因蠻民弗靖,有司莫能控禦,耆民黃祖記與思恩土官岑瑛交結,欲割地歸之思恩,因謀於知縣朱斌備。時瑛方雄兩江,大將多右之,斌備亦欲藉以自固,遂為具奏,以地改屬思恩。土民不服,韋萬秀以復地為名,因而倡亂。成化二十二年,覃召管等復亂,屢征不靖。弘治元年委官撫之,眾願取前地,別立長官司。都御史鄧廷瓚為奏,置永順、永安二司,各設長官一,副長官一,以鄧文茂等四人為之,皆宜山洛口、洛東諸里人也。自是宜山東南棄一百八十四村地,宜山西南棄一百二十四村地。議者以忻城自唐、宋內屬已二百餘年,一旦舉而棄之於蠻,為失策云。

平樂,初為縣,元大德中改平樂府,明因之。洪武二十一年,廣西都指揮使言:「平樂府富川縣靈亭山、破紙山等洞瑤二千餘人,占耕內地,嘯聚劫奪,居民被擾,恭城、賀縣及湖廣道州、永明等縣之民亦被害。比調衛兵收捕,即逃匿巖谷,兵退復肆跳梁。臣等欲於秋成時,統所部會永、道諸軍,列屯賊境,扼其要路,收其所種穀粟。彼無糧食,勢必自窮,乘機擒戮,可絕後患。」。從之。二十九年遷富川縣於富川千戶所。時富川千戶所新立於矮石城,典史言:「縣治無城,恐蠻寇竊發,無以守禦,宜遷城內為便。」從之。弘治元九年,總督鄧廷瓚言:「平樂府之昭平堡介在梧州、平樂間,瑤、僮率出為患,乞令上林土知縣黃瓊、歸德土知州黃通各選子弟一人,領土兵各千人,往駐其地。仍築城垣,設長官司署領,撥平樂縣仙回峒閒田與之耕種。其冠帶千夫長龍彪改授昭平巡檢,造哨船三十,使往來府江巡哨,流官停選。」廷議以昭平堡係內地,若增土官,恐貽後患。況府江一帶,近已設按察司副使一員,整飭兵備,土官不必差遣,止令每歲各出土兵一千聽調。詔從其議。

府江有兩崖三洞諸僮,皆屬荔浦,延袤千餘里,中間巢峒盤絡,為瑤、僮窟穴。江上諸賊倚為黨援,日與府江酋長楊公滿等掠荔浦、平樂及峰門、南源,執永安知州楊惟執,殺指揮胡翰、千戶周濂、土舍岑文及兵民無算。而遷江之北三,來賓之北五,皆右江僮,亦時與東歐、西里及三都、五都諸賊相倚附,馬多人勁,俗號為劃馬賊。常陳兵走嶺東,掠三水、清遠諸縣,還入南寧、平南、武宣、來賓、藤、貴,劫府庫。已而劫來賓所千戶黃元舉,殺土吏黃勝及其子四人,兵七十餘人,又殺明經諸生王朝經、周松、李茂、姜集等,白晝劫殺,道絕行人。隆慶六年,巡撫郭應聘、總督殷正茂請討。詔總兵官李錫督軍進剿,並調東蘭、龍英、泗城、南丹、歸順諸土兵,而以土吏韋文明等統之,攻古西、巖口、筍山、古造及兩峰、黃洞等寨,斬獲賊渠,餘黨竄入仙回、古帶諸山,搜捕殆盡。乃移檄北三、北五,趣其歸降。峒老韋法真同被擄來賓、遷江民蒙演等詣軍前乞降,許之,乃定善後六策以聞。初,荔浦之峰門、南源,修仁之麗壁,永安之古眉諸巡司,為諸僮所奪。至是議改土巡檢,推擇有才武者,給冠帶管事,三載稱職,始世襲。

萬曆六年,北山蠻譚公柄挾毒弩,肆傷行旅,每一出十百為群。自殺黃勝後,復聚黨以三千人出DJ鳳山、龜鱉塘,與河塘韋宋武傍江結寨。時義寧、永寧、永福諸僮群起,相殺掠,道路不通。會咘咳寨藍公潺執土吏黃如金,奪其司。巡撫吳文華檄守巡道吳善、陳俊徵永順白山兵及狼兵剿之,平橫山、咘咳諸巢。諸瑤請還侵地及所擄生口,願輸賦為良民,遂班師。

右江十寨,隆慶中,總督殷正茂擊破古田,即以檄趣八寨歸降,得貸死。於是寨老樊公懸、韋公良等踵軍門上謁,自言十寨共一百二十八村,環村而居者二千一百二十餘家,皆請受賦。右江兵備鄭一龍、參將王世科,謂十寨既請為氓,當以十家為率,賦米一石。村立一甲長,寨立一峒老,為征賦計。而以思古、周安、落紅、古卯、龍哈立一州,屬向武土官黃九疇;羅墨、古缽、古憑、都北、咘咳立一州,屬那地土官黃暘;皆為土知州。已,移思恩守備於周安堡,而布政使以為不便,總制乃議立八寨為長官司,以兵八千人屬黃暘為長官,黃昌、韋富皆給冠帶為土舍,亦各引兵二百守焉。久之,十寨復聚黨作亂,據民田產,白晝入都市剽掠,甚至攻城劫庫,戕官民。總制劉堯誨、巡撫張任急統兵進剿,斬首一萬六千九百有奇,獲器仗三千二百,牛馬二百三十九。帝乃升賞諸土吏功,復分八寨為三鎮,各建一城,而以東蘭州韋應鯤、韋顯能及田州黃馮克為土巡檢,留兵一千人戍之。於三里增建二堡,自楊渡水為界,墾田屯種,給南丹衛,通道慶遠、賓州,使思恩、三里聯絡不絕,於是右江十寨復安輯輸賦。

三十二年,桂林、平樂瑤、僮據險肆亂,殺知縣張士毅,焚劫無虛月。總督應檟檄總兵官顧寰督兵進剿,擒斬四百八十四,俘獲男女三百四十,牛馬器械甚眾。守臣以捷聞,並上僉事茅坤、參將王寵、都指揮鐘坤秀、參政張謙、百戶吳通等功狀,各升廕有差。

平樂界桂、梧,西北近楚,清湘、九嶷,鬱相樛結。東南入梧,溪洞林箐,多為瑤人盤據。自數經大征後,刊山通道,展為周行,而又增置樓船,繕修校壘,居民行旅皆帖席,瑤、僮亦駸驍馴習於文治云。

梧州,漢之蒼梧郡也。元至元中,改置梧州路。洪武元年,征南將軍廖永忠、參政朱亮祖等既平廣東,引兵至梧州境。元達魯花赤拜住率官吏父老迎降,亮祖駐兵滕州。於是潯、貴等州縣以次降附。二年併南流縣於鬱林州,普寧縣於容州,并藤綿皆隸梧州府。四年置梧州守禦千戶所。二十三年置容縣守禦千戶所。

廣西全省惟蒼梧一道無土司,瑤患亦稀。萬曆初,岑溪有潘積善者,僭號平天王,與六十三山、六山、七山諸瑤、僮據山為寇,居民請剿。會大兵征羅旁不暇及,總制凌雲翼檄以禍福,積善願歸降輸賦,乃貸其死,且以其子入學。議者謂七山為蒼、藤信地,六山為容縣、北流中衝,北科為六十三山咽喉,懷集為賀縣諸村出入之所。因立五大營,營六百人,合得三千人,設參將及屯堡三十治焉。而懷集瑤賊,在正德中已雄據十五寨,環二百餘里,為州縣患。官軍屢討之,歸降,然盤互如故,往往相結諸峒蠻劫掠,殺百戶朱裳及把總羅定朝,村民畏之,東西走匿。都御史吳善檄總兵戚繼光徵兵於羅定、泗城、都康諸土司,分五道,命參將戴應麟等擊金雞、松柏諸寨,斬渠魁,撫四百餘人。時鬱林瑤亦桀驁,數糾諸生瑤破諸村寨,入寇興業縣。兵巡道副使王原相告於總制,調兵擊破之,諸瑤悉平。

潯州,江曰潯江,東城門曰潯陽,郡名取焉。洪武八年,潯州大藤峽瑤賊竊發,柳州衛官軍擒捕之。二十年,知府沈信言:「府境接連柳、象、梧、藤等州,山谿險峻,瑤賊出沒不常。近者廣西布政司參議楊敬恭為大亨、老鼠、羅碌山生瑤所殺,官軍討之,賊登巖攀樹,捷如猿狖,追襲不及。若久駐兵,則瘴癘時發,兵多疾疫,又難進取,兵退復出為患。臣以為桂平、平南二縣,舊附瑤民,皆便習弓弩,慣歷險阻。若選其少壯千餘人,免其差徭,給以軍器衣裝,俾各團村寨置烽火,與官兵相為聲援,協同捕逐,可以殲之。」帝以蠻夷梗化,夙昔固然,但當謹其防禦,使不為患。如為寇不已,則發兵討之,何必團寨。

永樂三年,總兵韓觀奏桂平縣蠻民為亂,請發兵剿捕。帝命姑撫之,勿用兵。宣德四年,總兵山雲討潯、柳二州寇,並誅從寇二千四百八十人,梟首境上。七年,雲奏斬獲桂平等縣蠻寇覃公專等首級數。帝顧左右曰:「蠻寇害我良民,譬之蟊賊害稼,不可不去。然殺之過多,亦所不忍。雖彼自取滅亡,朕自以天地之心為心也。」九年,雲奏潯州等處蠻寇劫掠良民,指揮田真率兵於大藤峽等處,前後斬首九十六級,歸所掠男婦二百三人。

正統元年,兵部尚書王驥奏:「桂平大藤峽等處蠻寇,攻劫鄉村。因調廣東官軍二千人,今已逾年,軍器衣裝損壞,宜如貴州諸軍例,予踐更。」從之。二年,山雲奏:「潯州府平南等縣耆民言:『大藤峽等山,瑤寇不時出沒,劫掠居民,阻絕行旅。近山荒田,為賊占耕,而左、右兩江,人多食少,其狼兵素勇,為賊所憚。若選委頭目,屯種近山荒田,斷賊出沒之路,不過數年,賊徒坐困,地方寧靖矣。』臣已會同巡按諸司計議,量撥田州等府族目土兵,分界耕守,即委土官都指揮黃竑領之。遇賊出沒,協同剿殺。」從之。七年,瑤賊藍受貳等恃所居大藤峽山險,糾集大信等山山老、山丁數百人,遞年殺掠。千戶滿智等誘殺十人,帝命梟之,家口給賜有功之家。十一年,大藤峽蠻賊流劫鄉村,侵犯諸縣,巡按萬節以聞。景泰七年,大藤峽賊糾合荔浦等處賊,劫掠縣治,殺擄居民,命總兵柳溥等剿之。

天順五年,鎮守廣東中官阮隨奏:「大藤峽瑤賊出沒兩廣,為惡累年,邇來愈甚。雖常會兵剿捕,緣地里遼遠,且兩廣軍馬不相統屬,未易成功,宜大舉搗其巢穴,庶絕民患。」乃命都督僉事顏彪佩征夷將軍印,調南京、江西及直隸九江等衛官軍一萬隸之。六年,彪奏:「臣率軍進剿大藤,攻破七百二十一寨,斬首三千二百七十一級,復所掠男婦五百餘口。」帝敕獎之。

七年,大藤峽賊夜入梧州城。時總兵官泰寧侯陳涇駐兵城中,會太監朱祥、巡按吳璘、副使周璹、僉事董應軫、參議陸禎、都指揮杜衡、土官都指揮岑瑛等議調兵。夜半,賊駕梯上城,涇等不覺,遂入府治,劫庫放囚,殺死軍民無算,大掠城中,執副使周璹為質,殺訓導任璩。涇等倉卒無計,惟擁兵自衛,隨軍器械并備賞銀物,皆為賊有。布政使宋欽時致仕家居,挺身出,以大義諭賊,為所害。黎明,賊聲言官軍若動,則殺周副使。涇等乃遣人與賊講解,晡時,縱之出城。賊既出,乃縱璹還。時官軍數千,賊僅七百而已。都指揮邢斌奏至,帝曰:「梧州蕞爾小城,總兵、鎮、巡、三司俱擁重兵駐城中,乃為小賊所蔑視,況遇大敵乎!爾兵部其即議處行。」

八年,國子監生封登奏:「潯州夾江諸山,含岈DK,峽中有大藤如斗,延亙兩崖,勢如徒杠,蠻眾蟻渡,號大藤峽,最險惡,地亦最高。登藤峽巔,數百里皆歷歷目前,軍旅之聚散往來,可顧盼盡,諸蠻倚為奧區。桂平大宣鄉崇姜里為前庭,象州東鄉、武宣北鄉為後戶,藤縣五屯障其左,貴縣龍山據其右,若兩臂然。峽北巖峒以百計,仙人關、九層崖極險峻,峽以南有牛腸、大岵諸村,皆緣江立寨。藤峽、府江之間為力山,力山之險倍於藤峽。又南則為府江,其中多冥巖奧谷,絕壁層崖,十步九折,失足隕身。中產瑤人,藍、胡、侯、槃四姓為渠魁。力山又有僮人,善傅毒藥弩矢,中人無不立斃,四姓瑤亦憚之。自景泰以來,嘯聚至萬人,隳城殺吏。而修仁、荔浦、平樂、力山諸瑤應之,其勢益張。渠長侯大狗嘗懸千金購,莫能得。鬱林、博白、新會、信宜、興安、馬平、來賓亦煽動,所至丘墟,為民害。乞選良將,多調官軍、狼兵急滅賊。」報聞。

成化元年,編修丘浚條上兩廣用兵機宜。兵部尚書王竑奏言:「峽賊稱亂日久,皆由守臣以招撫為功,致釀大患,非大創不止。」因薦浙江參政韓雍有文武才。命以雍為僉都御史,都督同知趙輔為征夷將軍,和勇為遊擊將軍,率師討之。時大藤峽賊三千餘陷平南縣,殺典史周誠,擄其妻子,并劫縣印。又入藤縣城,掠官庫,劫縣印,鎮守總兵歐信以聞。於是總兵官趙輔率軍至,奏言:「大藤蠻賊以修仁、荔浦為羽翼,今大軍壓境,宜先剿之。」乃合諸軍十六萬人,分五道進,先破修仁,窮追至力山,生擒千二百餘人,斬首七千三百餘級。

二年,趙輔、韓雍等奏:「元年十一月,師次潯州,謀深入以覆其巢。遂調總兵官歐信等分兵五哨,取道山北以進。臣及指揮白全分兵八哨,直抵潯州,以搗山南。復令參將孫震分兵二哨,從水路入。別遣指揮潘鐸等以兵分守諸山隘口,剋期十二月朔日,水陸並進,腹背交攻。賊知師至,先移妻子錢米入桂州橫石塘等處藏匿。乃於山南各寨,立柵自固,用木石鏢鎗藥弩,憑險拒守。官軍用團牌、扒山虎等器,魚貫而進。士殊死戰,一日之間,攻破山南、石門、林峒、沙田、古營諸巢,縱火焚其積聚,賊皆奔潰。復督兵追躡,剷山開路,直抵橫石塘及九層樓等山。賊已據險立柵數重,復用木石、鎗弩拒守。臣等多設疑兵,誘賊拋擲木石幾盡,別遣壯士於賊所不備處,高山絕頂,舉炮為號。諸軍緣木攀蘿,蟻附而上,四面夾攻,連日鏖戰,賊不能支。破賊寨三百二十四所,斬首三千二百七級,生擒七百八十二人,獲賊婦女二千七百一十八人,戰溺死者不可勝計。已將大藤峽改為斷藤峽,刻石紀之,以昭天討。」捷聞,帝降敕褒諭,仍敕輔計議長策,永絕後患。未幾,雍奏斷藤峽殘賊侯鄭昂等七百餘人,夜入潯州府城,焚軍營城樓,奪百戶所印三顆,殺掠男婦數十人。旋為參將孫震、指揮張英率軍擊斬賊魁,餘黨仍奔入巢。既雍又奏:「諸瑤之性,憚見官吏,攝以流官,終難靖亂。請改設武宣縣東鄉等巡檢司,以土人李升等為副巡檢;設武靖州於峽內,以上隆州知州岑鐸知州事,土人覃仲英世襲土官吏目。」然府江東西兩岸,大、小桐江、洛口與斷藤峽、朦朧、三黃等處,村巢接壤,路道崎嶇,聚眾劫掠,終不能除。

正德十一年,總督陳金復督調兩廣官軍土兵,分為六大哨,按察使宗璽,布政使吳廷舉,副總兵房閏,鎮守太監傅倫,參將牛桓,都指揮魯宗貫、王瑛將之,水陸並進,斬七千五百六十餘級。金謂諸蠻利魚鹽耳,乃與約,商船入峽者,計船大小,給之魚鹽。蠻就水濱受去,如榷稅然,不得為梗。蠻初獲利聽約,道頗通。金以此法可久,易峽名永通。諸蠻緣此無忌,大肆掠奪,稍不愜,即殺之。因循猖獗,江路為斷。時總督王守仁定田州還,兩江父老遮道言峽賊阻害狀。守仁上疏請討,報可。守仁率湖南兵至南寧,約日會兵。寇聞湖兵且至,皆逃匿。守仁故為散遣諸兵狀,寇弛不為備,乃令官軍突進,連破油窄、石壁、大皮等寨,賊奔斷藤峽,復追擊破之。賊奔渡橫石江,溺死六百餘人,俘斬甚眾,賊潰散。遂移兵仙臺、花相、白竹、古陶、羅鳳諸處,賊不支,奔入永安力山,官軍次第破之,擒斬三千餘,俘獲無算。八寨平,兩江悉定。守仁遂以土官岑猛子邦佐為武靖知州,使靖遺孽。

邦佐不能輯眾,且貪得賊賄,峽北賊復獗。有侯勝海者為首,指揮潘翰臣誘殺之,勝海弟公丁聚眾噪城下。僉事鄔閱、參議孫繼祖言於都御史潘旦,請討之。參將沈希儀以為宜需春江漲,順流下,乃可破賊,不聽。閱與繼祖以千人往擊,賊遁,斬一尪寇還。漫言賊退,請置堡。堡成,閱令土目黃貴、韋香以三百人往戍。初,貴、香利勝海田廬,故說翰臣殺海,至是往戍,遂奪勝海田廬。於是諸瑤俱恚,邦佐又陰黨之,公丁遂嘯聚二千餘人,乘夜陷堡城,殺戍兵二百人,貴、香走免。巡按以聞,乃罷閱與繼祖,旦亦代去,命侍郎蔡經督兵討之。會朝議欲征安南,事遂已。公丁等益橫,時出殺掠。久之,經乃會安遠侯柳珣決計發兵,以兵事屬副使翁萬達。萬達廉得百戶許雄通賊狀,詰之。雄懼,請自效。萬達佯庇公丁,捕繫訐訟公丁者數人。公丁果遣人自列,萬達佯許之,又令雄假稱貸為賄,公丁喜,益信雄。會萬達巡他郡,以事屬參議田汝成。汝成召雄申飭之,雄紿公丁詣汝成自列,言寇堡事由他瑤,汝成亦慰遣之。乃密授意城中居民被賊害者家,出毆公丁,一市皆嘩,遊檄并逮公丁入繫獄。遣雄諭其黨曰:「寇堡事公丁委罪諸瑤,果否?」諸瑤遂言事自公丁,聽論坐,不敢黨。乃檻致公丁於軍門,礫之。汝成因言於經,謂首惡既誅,宜乘勢進兵討賊。乃以副總兵張經、都指揮高乾分將左右二軍,萬達及副使梁廷振監之,副使蕭畹紀功,參政林士元及汝成督餉。

嘉靖十八年二月,兩軍齊發:左軍三萬五千人,分六道,攻紫荊、石門、梅嶺、木昴、藤沖、大坑等巢;右軍萬六千人,分四道,攻碧灘,羅淥上、中、下洞等巢。南北夾擊,賊大窘,遂擁眾奔林峒而東。王良輔邀擊之,中斷,復西奔。諸軍合擊,大破之,斬首千二百級,追至羅運山,又斬百餘級。平南縣有小田、羅應、古陶、古思諸瑤亦據險勿靖。萬達等移兵剿之,招降賊黨二百餘人,江南胡姓諸瑤歸順者亦千餘人,藤峽復平。

隆慶三年,右江諸瑤、僮復亂,巡撫郭應聘請給餉剿除。給事中梁問孟以賊黨眾,不可盡滅,宜令守臣熟計。兵部言:「府江自正德十二年都御史陳金征討之後,且六十年。而右江北三、北五等巢,素未懲創,生齒日繁,遂肆猖獗。頃者大征古田,各巢咸畏威斂戢,獨府江、右江恃險為亂,若復縱之,非惟無以固八寨懷遠之招,亦恐以啟古田攜貳之漸,剿之便。但兵在萬全,宜即以科臣所慮,備行提督殷正茂及巡撫郭應騁等便宜行之。」應聘遂檄總兵官李錫等將兵往討,以捷聞。南寧,唐之邕州也。元,邕州路。泰定中,改南寧路。洪武二年命潭州衛指揮同知丘廣為總兵官,寶慶衛指揮僉事胡海、廣西衛指揮僉事左君弼副之,率兵討左江上思州蠻賊黃龍冠等。龍冠一名英傑,時聚眾萬餘,寇鬱林州。知州趙鑑、同知王彬集民丁拒守,賊圍半月不下。海北等衛官軍來援,賊夜遁,追至上思州境,破之,賊走還,仍結聚不解。事聞,故命廣等討之。廣等兵至上思州,賊拒戰,擊敗之,擒從賊黃權等。英傑走十萬山,官軍追及,斬之,上思州平。

三年,置南寧、柳州二衛。時廣西省臣言:「廣西地接雲南、交阯,所治皆溪洞苗蠻,性狼戾多畔。府衛兵遠在靖江數百里外,卒有警,難相援,乞立衛置兵以鎮。」又言:「廣海俗素獷戾,動相仇殺,蓋緣郡縣無兵以馭之。近盜寇鬱林,同知集民兵拒守,潯州經歷徐成祖亦以民兵千餘敗賊,是土兵未始不可用。乞令邊境郡縣輯民丁之壯者,置衣甲器械,籍之有司,有事則捕賊,無事則務農。」詔從之。遂置衛,益兵守禦,賞王彬、徐成祖等有功者。

五年,宣化盜起,劫掠南寧府,詔發廣西官軍討平之。初,南寧衛指揮僉事左君弼核民之無藉者為軍,又縱所部入山伐木,民多擾,遂相構為盜。至是討平,命大都督府按君弼罪。

南寧故稱邕管,牂牁峙其西北,交阯踞其西南,三十六洞錯壤而居,延袤幾千里,橫山、永平尤要害。歷唐及宋,建牙置帥,與桂州等。又郡地夷曠,可宿數萬師。成化時,徵田州及經略安南,舉弭節茲土。後因瑤蠻不靖,往往仗狼兵,急則藉為前驅,緩則檄為守禦。諸瑤乃稍稍驕恣,不可盡繩以法。議邕事者謂宜開重鎮,以復邕州督府之舊云。南寧領州四。曰新寧,曰橫州,為流官。曰上思州,曰下雷州,為土官。縣三,曰宣化,曰隆化,曰永淳。

歸德州,宋熙寧中置。元屬田州路。洪武二年,土官黃隍城歸附,授知州,以流官吏目佐之。

果化州,宋始置。元屬田州路。洪武二年,土官趙榮歸附,授世襲知州,以流官吏目佐之。洪熙元年,果化州土官趙英遣族人趙誠等貢馬及方物。弘治間,州與歸德皆為田州所侵削,因改隸於南寧。

上思州,唐始置。元屬思明路,洪武初,土官黃中榮內附,授知州,子孫畔服不常。弘治十八年改流官,屬南寧府。正德六年,土目黃錙聚眾攻城,都御史林廷選捕之,下獄。已,越獄復叛,官軍禦之,詐降,攻破州城,復捕獲之,伏誅。嘉靖元年,都御史張嵿言:「上思州本土官,後改流,遂致土人稱亂。宜仍其舊,擇土吏之良者任之。」議以為然,仍以土官襲。

下雷州,宋置。明初,印失,廢為峒,在湖潤寨,屬鎮安府。峒長許永通奉調有功,給冠帶。傳世烈、國仁繼襲峒事。嘉靖十四年獲舊印。國仁及子宗廕屢立戰功。四十三年改屬南寧府。萬歷十八年以地逼交南,奏升為州,頒印,授宗廕子應珪為土判官,流官吏目佐之。


\end{pinyinscope}