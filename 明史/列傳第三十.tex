\article{列傳第三十}

\begin{pinyinscope}
鐵鉉暴昭侯泰陳性善陳植王彬崇剛張昺謝貴彭二葛誠余逢辰宋忠餘瑱馬宣曾浚卜萬硃鑒石撰瞿能莊得楚智皁旂張王指揮楊本張倫陳質顏伯瑋唐子清黃謙向朴鄭恕鄭華王省姚善錢芹陳彥回張彥方

鐵鉉,鄧人。洪武中,由國子生授禮科給事中,調都督府斷事。嘗讞疑獄,立白。太祖喜,字之曰「鼎石」。建文初,為山東參政。李景隆之北伐也,鉉督餉無乏。景隆兵敗白溝河,單騎走德州,城戍皆望風潰。鉉與參軍高巍感奮涕泣,自臨邑趨濟南,偕盛庸、宋參軍等誓以死守。燕兵攻德州,景隆走依鉉。德州陷,燕兵收其儲蓄百餘萬,勢益張。遂攻濟南,景隆復大敗,南奔。鉉與庸等乘城守禦。燕兵隄水灌城,築長圍,晝夜攻擊。鉉以計焚其攻具,間出兵奮擊。又遣千人出城詐降。燕王大喜,軍中皆懽呼。鉉伏壯士城上,候王入,下鐵板擊之。別設伏、斷橋。既而失約,王未入城板驟下。王驚走,伏發,橋倉卒不可斷,王鞭馬馳去。憤甚,百計進攻。凡三閱月,卒固守不能下。當是時,平安統兵二十萬,將復德州,以絕燕餉道。燕王懼,解圍北歸。

燕王自起兵以來,攻真定二日不下,即舍去。獨以得濟南,斷南北道,即畫疆守,金陵不難圖。故乘大破景隆之銳,盡力以攻,期於必拔,而竟為鉉等所挫。帝聞大悅,遣官慰勞,賜金幣,封其三世。鉉入謝,賜宴。凡所建白皆採納。擢山東布政使。尋進兵部尚書。以盛庸代景隆為平燕將軍,命鉉參其軍務。是年冬,庸大敗燕王於東昌,斬其大將張玉。燕王奔還北平。自燕兵犯順,南北日尋干戈,而王師克捷,未有如東昌者。自是燕兵南下由徐、沛,不敢復道山東。

比燕兵漸逼,帝命遼東總兵官楊文將所部十萬與鉉合,絕燕後。文師至直沽,為燕將宋貴等所敗,無一至濟南者。四年四月,燕軍南綴王師於小河,鉉與諸將時有斬獲。連戰至靈璧,平安等師潰被擒。既而庸亦敗績。燕兵渡江,鉉屯淮上,兵亦潰。

燕王即皇帝位,執之至。反背坐廷中嫚罵,令其一回顧,終不可,遂磔於市。年三十七。子福安,戍河池。父仲名,年八十三,母薛,並安置海南。

宋參軍者,逸其名。燕兵攻濟南不克,舍之南去。參軍說鉉直搗北平。鉉以卒困甚,不果。後不知所終。

暴昭,潞州人。洪武中,由國子生授大理寺司務。三十年,擢刑部右侍郎。明年進尚書。耿介有峻節,布衣麻履,以清儉知名。建文初,充北平採訪使,得燕不法狀,密以聞,請預為備。燕兵起,設平燕布政司於真定,昭以尚書掌司事,與鐵鉉輩悉心經畫。平安諸軍敗,召歸。金川門陷,出亡,被執。不屈,磔死。

繼昭為刑部尚書者侯泰,字順懷,南和人。以薦舉起家。建文初,仕至尚書。燕王舉兵,力主抗禦之策。嘗督餉於濟寧、淮安。京師不守,行至高郵,被執下獄,與弟敬祖,子,俱被殺。

陳性善,名復初,以字行,山陰人。洪武三十年進士。臚唱過御前,帝見其容止凝重,屬目久之,曰:「君子也。」授行人司副,遷翰林檢討。性善工書,嘗召入便殿,繙錄誠意伯劉基子璉所獻其父遺書。帝威嚴,見者多惴恐,至惶汗,不成一字。性善舉動安祥,字畫端好。帝大悅,賜酒饌,留竟日出。

惠帝在東宮,習知性善名。及即位,擢為禮部侍郎,薦起流人薛正言等數人。雲南布政使韓宜可隸謫籍,亦以性善言,起副都御史。一日,帝退朝,獨留性善賜坐,問治天下要道,手書以進。性善盡所言,悉從之。已,為有司所格,性善進曰:「陛下不以臣不肖,猥承顧問。既僭塵聖聽,許臣必行。未幾輟改,事同反汗。何以信天下?」帝為動容。

燕師起,改副都御史,監諸軍。靈璧戰敗,與大理丞彭與明、欽天監副劉伯完等皆被執。已,悉縱還。性善曰:「辱命,罪也,奚以見吾君?」朝服躍馬入於河以死。餘姚黃墀、陳子方與性善友,亦同死。燕王入京師,詔追戮性善,徙其家於邊。

與明,萬安人。貢入太學,歷給事中。建文初,為大理右丞,廉勤敏達。以督軍被執。縱歸,慚憤裂冠裳。變姓名,與伯完俱亡去,不知所終。

時以侍郎監軍者,有廬江陳植。植,元末舉鄉試,不仕。洪武間,官吏部主事。建文二年官兵部右侍郎。燕兵臨江,植監戰江上。慷慨誓師。部將有議迎降者,植責以大義,甚厲。部將殺之以降,且邀賞。燕王怒,立誅部將,具棺殮葬植白石山上。

燕師之至江北也,御史王彬巡按江淮。駐揚州,與鎮撫崇剛嬰城堅守。時盛庸兵既敗,人無固志。守將王禮謀舉城降,彬執之及其黨,繫獄。剛出練兵,彬修守具,晝夜不懈。有力士能舉千斤,彬嘗以自隨。燕兵飛書城中:「縛王御史降者,官三品。」左右憚力士,莫敢動。禮弟崇賂力士母,誘其子出。乘彬解甲浴,猝縛之。出禮於獄,開門納燕師。彬與剛皆不屈死。彬,字文質,東平人。洪武中進士。剛,逸其里籍。

又兵部主事樊士信,應城人。守淮,力拒燕兵,不勝,死之。

張昺,澤州人。洪武中,以人材累官工部右侍郎。謝貴者,不知所自起,歷官河南衛指揮僉事。建文初,廷臣議削燕,更置守臣。乃以昺為北平布政使,貴為都指揮使,並受密命。時燕王稱疾久不出,二人知其必有變,乃部署在城七衛及屯田軍士,列九門防守,將執王。昺庫吏李友直預知其謀,密以告王,王遂得為備。建文元年七月六日,朝廷遣人逮燕府官校。王偽縛官校置廷中,將付使者。紿昺、貴入,至端禮門,為伏兵所執,俱不屈死。

燕將張玉、朱能等帥勇士攻九門,克其八,獨西直門不下。都指揮彭二躍馬呼市中曰:「燕王反,從我殺賊者賞!」集兵千餘人,將攻燕府。會燕健士從府中出,格殺二,兵遂散,盡奪九門。

初,昺被殺,喪得還。「靖難」後,出昺屍焚之,家人及近戚皆死。

葛誠,不知所由進。洪武末,為燕府長史。嘗奉王命奏事京師。帝召見,問府中事,誠具以實對。遣還。王佯病,盛暑擁爐坐,呼寒甚。昺、貴等入問疾。誠言:「王實無病,將為變。」又密疏聞於帝。及昺、貴將圖王,誠與護衛指揮盧振約為內應。事敗,誠、振俱被殺,夷其族。

又伴讀余逢辰,字彥章,宣城人。有學行。王信任之,以故得聞異謀,乘間力諫。知變將作,貽書其子,誓必死。兵起,復泣諫,言「君、父兩不可負」,死之。

北平人杜奇者,才雋士也。燕王起兵,徵入府,奇因極諫「當守臣節」,王怒,立斬之。

宋忠,不知何許人。洪武末,為錦衣衛指揮使。有百戶以非罪論死,忠疏救。御史劾之,太祖曰:「忠率直無隱,為人請命,何罪?」遂宥百戶。尋為僉都御史劉觀所劾,調鳳陽中衛指揮使。三十年,平羌將軍齊讓征西南夷無功,以忠為參將,從將軍楊文討之。師旋,復官錦衣。

建文元年,以都督奉敕總邊兵三萬屯開平,悉簡燕府護衛壯士以從。又以都督徐凱屯臨清,耿瓛屯山海關,相犄角。北平故有永清左、右衛,忠調其左屯彰德,右屯順德以備燕。及張昺、謝貴謀執燕王,忠亦帥兵趨北平。未至而燕兵起,居庸失守,不得進,退保懷來。燕王度忠必爭居庸,帥精兵八千,卷甲倍道趨懷來。時北平將士在忠部下者,忠告以「家屬並為燕屠滅,盍努力復仇報國恩」。燕王偵知之,急令其家人張故旂幟為前鋒,呼父兄子弟相問勞。將士咸喜曰:「我家固亡恙,宋總兵欺我。」遂無鬥志。忠倉卒布陣,未成列。燕王一麾渡河,鼓噪進。忠敗,死之。

忠之守懷來也,都指揮餘瑱、彭聚、孫泰與俱。及戰,瑱被執,不屈死。泰中流矢,血被甲,裹創力鬥,與聚俱沒於陣。當是時,諸將校為燕所俘者百餘人,皆不肯降,以死。惜姓名多不傳。

馬宣,亦不知何許人。官都指揮使。宋忠之趨居庸,宣亦自薊州帥師赴北平。聞變,走還。燕王既克懷來,旋師欲南下。張玉進曰:「薊州外接大寧,多騎士,不取恐為後患。」會宣發兵將攻北平,與燕兵戰公樂驛,敗歸,與鎮撫曾濬城守。玉等往攻之,宣出戰被擒,罵不絕口,與濬俱死。

燕兵之襲大寧也,守將都指揮卜萬與都督劉真、陳亨帥兵扼松亭關。亨欲降燕,畏萬,不敢發。燕行反間,貽萬書,盛稱萬;極詆亨。厚賞所獲大寧卒,緘書衣中,俾密與萬。故使同獲卒見之,亦縱去而不與賞。不得賞者發其事。真、亨搜卒衣,得書。遂執萬下獄死,籍其家。萬忠勇而死於間,論者惜之。及大寧陷,指揮使朱鑑力戰,不屈死。

寧府左長史石撰者,平定人。以學行稱。燕王舉兵,撰輒為守禦計,每以臣節諷寧王,王亦心敬之。及城陷,憤詈不屈,支解死。

瞿能,合肥人。父通,洪武中,累官都督僉事。能嗣官,以四川都指揮使從藍玉出大渡河擊西番,有功。又以副總兵討建昌叛酋月魯帖木兒,破之雙狼寨。燕師起,從李景隆北征。攻北平,與其子帥精騎千餘攻彰義門,垂克。景隆忌之,令候大軍同進。於是燕人夜汲水沃城。方大寒,冰凝不可登,景隆卒致大敗。已,又從景隆進駐白溝河,與燕師戰。能父子奮擊,所向披靡。日暝,各收軍。明日復戰,燕王幾為所及。王急佯招後軍以疑之,得脫去。薄暮,能復引眾搏戰,大呼「滅燕」,斬馘數百。諸將俞通淵、滕聚復帥眾來會。會旋風起,王突入馳擊。能父子死於陣。通淵、聚俱死。精兵萬餘並沒。南軍由是不振。

時與北兵戰死者,有都指揮莊得、楚智、皁旗張等。

得,故隸宋忠。懷來之敗,一軍獨全。後從盛庸戰夾河,斬燕將譚淵。已而燕王以驍騎乘暮掩擊,得力戰,死。

智,嘗從馮勝、藍玉出塞有功。建文初,守北平。尋召還。及討燕,帥兵從景隆。戰輒奮勇,北人望旂幟股慄。至是,馬陷被執,死。

皂旗張,逸其名。或曰張能力挽千斤,每戰輒麾皂旗先驅,軍中呼「皂旗張」。死時猶執旗不仆。

又王指揮者,臨淮人。常騎小馬,軍中呼「小馬王」。戰白溝河被重創,脫胄付其僕曰:「吾為國捐軀,以此報家人。」立馬植戈而死。二人死尤異云。

又中牟楊本,初為太學生,通禽遁術,應募授錦衣鎮撫。從景隆討燕有功,景隆忌之,不以聞。尋劾景隆喪師辱國,遂以孤軍獨出,被擒,繫北平獄,後被殺。

張倫,不知何許人。河北諸衛指揮使也,勇悍負氣,喜觀古忠義事。馬宣自薊州起兵攻北平,不克,死。倫發憤,合兩衛官帥所部南奔,結盟報國。從李景隆、盛庸戰,皆有功。燕王即帝位,招倫降。倫笑曰:「張倫將自賣為丁公乎!」死之。京師陷,武臣皆降附。從容就義者,倫一人而已。

又陳質者,以參將守大同。進中軍都督同知。助宋忠保懷來。忠敗,退守大同。代王欲舉兵應燕,質持之不得發。及燕兵攻大同不下,蔚州、廣昌附於燕,質復取之。成祖即位,以質劫制代王,剽掠已附,誅死。

顏伯瑋,名瑰,以字行,廬陵人。唐魯國公真卿後。建文元年,以賢良徵,授沛縣知縣。李景隆屯德州,沛人終歲輓運。伯瑋善規畫,得不困。會設豐、沛軍民指揮司,乃集民兵五千人,築七堡為備禦計。尋調其兵益山東,所存疲弱不任戰。燕兵攻沛,伯瑋遣縣丞胡先間行,至徐州告急。援不至,遂命其弟玨、子有為還家侍父。題詩公署壁上,誓必死。燕兵夜入東門,指揮王顯迎降。伯瑋冠帶升堂,南向拜,自經死。有為不忍去,復還,見父屍,自刎其側。

主簿唐子清、典史黃謙俱被執。燕將欲釋子清。子清曰:「願隨顏公地下。」遂死之。遣謙往徐州招降。謙不從,亦死。

又向朴,慈谿人。力學養親。洪武末,以人才召見,知獻縣。縣無城郭,燕將譚淵至,朴集民兵與戰,被執,懷印死。

鄭恕,仙居人。蕭縣知縣。燕將王聰破蕭,不屈死。二女當配,亦死之。

鄭華,臨海人。由行人貶東平吏目。燕兵至,州長貳盡棄城走。華謂妻蕭曰:「吾義,必死。奈若年少何?」蕭泣曰:「君不負國,妾敢負君?」華曰:「足矣。」帥吏民憑城固守,城破,力戰,不屈死。

王省,字子職,吉水人。洪武五年領鄉舉。至京,詔免會試,命吏部授官。省言親老,乞歸養。尋以文學徵。太祖親試,稱旨,當殊擢。自陳「才薄親老」,乞便養。授浮梁教諭。凡三為教官,最後得濟陽。燕兵至,為游兵所執。從容引譬,詞義慷慨。眾舍之。歸坐明倫堂,伐鼓聚諸生,謂曰:「若等知此堂何名,今日君臣之義何如?」因大哭,諸生亦哭。省以頭觸柱死。女靜,適即墨主簿周岐鳳。聞燕兵至濟陽,知父必死,三遣人往訪,得遺骸歸葬。

姚善,字克一,安陸人。初姓李。洪武中由鄉舉歷祁門縣丞,同知廬州、重慶二府。三十年遷蘇州知府。初,太祖以吳俗奢僭,欲重繩以法,黠者更持短長相攻訐。善為政持大體,不為苛細,訟遂衰息,吳中大治。好折節下士,敬禮隱士王賓、韓奕、俞貞木、錢芹輩。以月朔會學宮,迎芹上座,請質經義。芹曰:「此非今所急也。」善悚然起問。芹乃授以一冊。視之,皆守禦策。

時燕兵已南下,密結鎮、常、嘉、松四郡守,練民兵為備。薦芹於朝,署行軍斷事。善尋至京師。會朝廷以燕王上書貶齊泰、黃子澄於外,善言不當貶,遂復召二人。建文四年詔兼督蘇、松、常、鎮、嘉興五府兵勤王。兵未集,燕王已入京師。時子澄匿善所,約共航海起兵。善謝曰:「公,朝臣,當行收兵圖興復。善守土,與城存亡耳。」子澄去,善為麾下許千戶者縛以獻,不屈,死。年四十三。子節等四人俱戍配。

芹,字繼忠。少好奇節。元末,干諸將,不遇。洪武初,辟大都督府掾,從中山王出北平至大漠。還解職。家居二十年,甘貧樂道。以善薦起。從李景隆北行,遣入奏事。道病將卒,猶條上兵事。年七十三。

陳彥回,字士淵,莆田人。父立誠,為歸安縣丞,被誣論死。彥回謫戍雲南,家人從者多道死。比至蜀,唯彥回與祖母郭在。會赦,又弗原,監送者憐而縱之。貧不能歸,依鄉人知縣黃積良,冒黃姓。久之,以閬中教諭嚴德政薦,授保寧訓導。考滿至京,召見以為平江知縣。逾年,太祖崩,彥回入臨。又以給事中楊維康薦,擢徵州知府。建文元年,以循良受上賞。祖母郭卒,當去,百姓走京師乞留。彥回衰糸至赴闕自陳,乞復姓。當彥回之戍雲南也,其弟彥蒦亦戍遼東。至是,詔除彥蒦籍。連乞終喪,不許。葬郭徽城北十里北山之陽。時走墓下,哭甚哀。人目之曰「太守山」。嘗對百姓泣曰:「吾罪人也,向亡命冒他姓。以祖母存,恐陳首獲罪,隱忍二十年。今祖母沒,宜自請死。上特宥我,終當死報國耳。」燕兵逼京師,彥回糾義勇赴援。已而被擒,械至京,死之。

張彥方,龍泉人。初為給事中,以便養乞改樂平知縣。應詔勤王,帥所部抵湖口。被執,械至樂平斬之。梟其首譙樓。當署月,一蠅不集,經旬面如生。邑人竊葬之清白堂後。

同時以勤王死者,有松江同知,死尤烈云。同知姓名不可考,或曰周繼瑜也。勤王詔下,榜募義勇入援。極言大義,感動人心。並斥「靖難」兵乖恩悖道。械至京,磔於市。

贊曰:燕師之南向也,連敗二大將,其鋒蓋不可當。鐵鉉以書生竭力抗禦於齊、魯之間,屢挫燕眾。設與耿、李易地而處,天下事固未可知矣。張昺、謝貴、葛誠圖燕於肘腋,而事不就。宋忠、馬宣東西繼敗,瞿能諸將垂勝戰亡,燕兵卒得長驅南下。而姚善、陳彥回之屬,欲以郡邑之甲奮拒於大勢已去之後,此黃鉞所謂「兵至江南,禦之無及」者也。


\end{pinyinscope}