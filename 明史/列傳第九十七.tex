\article{列傳第九十七}

\begin{pinyinscope}
楊最顧存仁高金王納言馮恩子行可時可宋邦輔薛宗鎧會翀楊爵浦鋐周天佐周怡劉魁沈束沈鍊楊繼盛{{何光裕龔愷楊允繩馬從謙孫允中狄斯彬

楊最,字殿之,射洪人。正德十二年進士。授工部主事。督逋山西,憫其民貧,不俟奏報輒返。尚書李鐩劾之,有詔復往。最乃與巡按御史牛天麟極陳歲災民困狀,請緩其徵。從之。

歷郎中,治水淮、揚。值世宗即位,上言:「寶應氾光湖西南高,東北下。運舟行湖中三十餘里。而東北隄岸不踰三尺,雨霪風厲,輒衝決,陰阻運舟,監城、興化、通、泰良田悉遭其害。宜如往年白圭修築高郵康濟湖,專敕大臣加修內河,培舊隄為外障,可百年無患,是為上策。其次於緣河樹杙數重,稍障風波,而增舊隄,毋使庳薄,亦足支數年。若但窒隙補闕,茍冀無事,一遇霪潦,蕩為巨浸,是為無策。」部議用其中策焉。出為寧波知府。請罷浙東貢幣,詔悉以銀充,民以為便。累遷貴州按察使,入為太僕卿。

世宗好神仙。給事中顧存仁、高金、王納言皆以直諫得罪。會方士段朝用者,以所煉白金器百餘因郭勛以進,云以盛飲食物,供齋醮,即神仙可致也。帝立召與語,大悅。朝用言:帝深居無與外人接,則黃金可成,不死藥可得。帝益悅,諭廷臣令太子監國,「朕少假一二年,親政如初。」舉朝愕不敢言。最抗疏諫曰:「陛下春秋方壯,乃聖諭及此,不過得一方士,欲服食求神仙耳。神仙乃山棲澡練者所為,豈有高居黃屋紫闥,兗衣玉食,而能白日翀舉者?臣雖至愚,不敢奉詔。」帝大怒,立下詔獄,重杖之,杖未畢而死。

最既死,監國議亦罷。明年,勛以罪瘐死。朝用詐偽覺,亦伏誅。隆慶元年,贈最右副都御史,謚忠節。

顧存仁,字伯剛,太倉人。嘉靖十一年進士。除餘姚知縣,徵為禮科給事中。十七年冬疏陳五事。首言宜廣曠蕩恩,赦楊慎、馬錄、馮恩、呂經等。末云:「敗俗妨農,莫甚釋氏。葉凝秀何人,而敢乞度?」帝方崇道家言。凝秀,道士也。帝以為刺已,且惡其欲釋楊慎等,遂責存仁妄指凝秀為釋氏,廷杖之六十,編氓口外。往來塞上,幾三十年。穆宗即位,召為南京通政參議。歷太僕卿。未幾,致仕。存仁困阨久,方見用,遽勇退,世尤高之。萬曆初,卒。

高金,石州人。為兵科給事中。嘉靖九年上疏言:「陛下臨御之初,盡斥法王、國師、佛子,近又黜姚廣孝配享。臣每歎大聖人作為,千古莫及。乃有真人邵元節者,誤蒙殊恩,為聖德累。夫元節,一道流耳。有勞,優以金帛足矣,乃加崇秩,復賜其師李得晟贈祭。廣孝不可配享於太廟,則二人益不可拜寵於聖朝。望削元節真人號,並奪得晟恩恤,庶異端CR、正道昌。」帝方欲受長生術,大怒,立下詔獄拷掠。終以其言直,釋之。尋偕御史唐愈賢稽核御用監財物,劾奉御李興等侵蝕狀,置諸獄。後累官蘇州兵備副使。

王納言,信陽人。為戶科給事中。請斥太常卿陳道瀛等,坐下詔獄,謫湖廣布政司照磨。累官陜西僉事。

馮恩,字子仁,松江華亭人。幼孤,家貧,母吳氏親督教之。比長,知力學。除夜無米且雨,室盡濕,恩讀書床上自若。登嘉靖五年進士,除行人。出勞兩廣總督王守仁,遂執贄為弟子。

擢南京御史。故事,御史有所執訊,不具獄以移刑部,刑部獄具,不復牒報。恩請尚書仍報御史。諸曹郎言雚,謂御史屬吏我。恩曰:「非敢然也。欲知事本末,得相檢核耳。」尚書無以難。已,巡視上江。指揮張紳殺人,立置之辟。大計朝覲吏,南臺例先糾。都御史汪鋐擅權,請如北臺,既畢事,始許論列。恩與給事中林土元等疏爭之,得如故。

帝用閣臣議分建南北郊,且欲令皇后蠶北郊,詔廷臣各陳所見,而詔中屢斥異議者為邪徒。恩上言:「人臣進言甚難,明詔令直諫,又詆之為邪徒,安所適從哉?此非陛下意,必左右奸佞欲信其說者陰詆之耳。今士風日下,以緘默為老成,以謇諤為矯激,已難乎其忠直矣。若預恐有異議,而逆詆之為邪,則必雷同附和,而後可也。況天地合祀已百餘年,豈宜輕改?《禮》:『男不言內,女不言外』。皇后深居九重,豈宜遠出郊野?願速罷二議,毋為好事希寵者所誤。」恩草疏時,自意得重譴。乃疏奏,帝不之罪,恩於是益感奮。

十一年冬,彗星見,詔求直言。恩以天道遠,人道邇,乃備指大臣邪正,謂:

大學士李時小心謙抑,解棼撥亂非其所長。翟鑾附勢持祿,惟事模棱。戶部尚書許贊謹厚和易,雖乏剸斷,不經之費必無。禮部尚書夏言,多蓄之學,不羈之才,駕馭任之,庶幾救時宰相。兵部尚書王憲剛直不屈,通達有為。刑部尚書王時中進退昧幾,委靡不振。工部尚書趙璜廉介自持,制節謹度。吏部尚書左侍郎周用才學有餘,直諒不足。右侍郎許誥講論便捷,學術迂邪。禮部左侍郎湛若水聚徒講學,素行未合人心。右侍郎顧鼎臣警悟疏通,不局偏長,器足任重。兵部左侍郎錢如京安靜有操守。右侍郎黃宗時雖擅文學,因人成事。刑部左侍郎聞淵存心正大,處事精詳,可寄以股肱。右侍郎硃廷聲篤實不浮,謙約有守。工部左侍郎黎奭滑稽淺近,才亦有為。右侍郎林昂才器可取,通達不執。

而極論大學士張孚敬、方獻夫,右都御史汪鋐三人之奸,謂:

孚敬剛惡兇險,媢嫉反側。近都給事中魏良弼已痛言之,不容復贅。獻夫外飾謹厚,內實詐奸。前在吏部,私鄉曲,報恩讎,靡所不至。昨歲偽以病去,陛下遣使徵之,禮意懇至。彼方倨傲偃蹇,入山讀書,直俟傳旨別用,然後忻然就道。夫以吏部尚書別用,非入閣而何?此獻夫之病所以痊也。今又遣兼掌吏部,必將呼引朋類,播弄威福,不大壞國事不止。若鋐,則如鬼如蜮,不可方物。所仇惟忠良,所圖惟報復。今日奏降某官,明日奏調某官,非其所憎惡則宰相之所憎惡也。臣不意陛下寄鋐以腹心,而鋐逞奸務私乃至此極。且都察院為綱紀之首。陛下不早易之以忠厚正直之人,萬一御史銜命而出,效其鍥薄以希稱職,為天下生民害,可勝言哉!故臣謂孚敬,根本之彗也;鋐,腹心之彗也;獻夫,門庭之彗也。三彗不去,百官不和,庶政不平,雖欲弭災,不可得已。

帝得疏大怒,逮下錦衣獄,究主使名。恩日受搒掠,瀕死者數,語卒不變。惟言御史宋邦輔嘗過南京,談及朝政暨諸大臣得失。遂并逮邦輔下獄,奪職。

明年春移恩刑部獄。帝欲坐以上言大臣德政律,致之死。尚書王時中等言:「恩疏毀譽相半,非專頌大臣,宜減戍。」帝愈怒,曰:「恩非專指孚敬三臣也,徒以大禮故,仇君無上,死有餘罪。時中乃欲欺公鬻獄耶?」遂褫時中職,奪侍郎聞淵俸,貶郎中張國維、員外郎孫雲極邊雜職,而恩竟論死。長子行可年十三,伏闕訟冤。日夜匍匐長安街,見冠蓋者過,輒攀輿號呼乞救,終無敢言者。時金宏已遷吏部尚書,而王廷相代為都御史。以恩所坐未當,疏請寬之,不聽。

比朝審,鋐當主筆,東向坐,恩獨向闕跪。鋐令卒拽之西面,恩起立不屈。卒呵之,恩怒叱卒,卒皆靡。鋐曰:「汝屢上疏欲殺我,我今先殺汝。」恩叱曰:「聖天子在上,汝為大臣,欲以私怨殺言官耶?且此何地,而對百僚公言之,何無忌憚也!吾死為厲鬼擊汝。」鋐怒曰:「汝以廉直自負,而獄中多受人餽遺,何也?」恩曰:「患難相恤,古之義也。豈若汝受金錢,鬻官爵耶?」因歷數其事,詆鋐不已。鋐益怒,推案起,欲毆之。恩聲亦愈厲。都御史王廷相、尚書夏言引大體為緩解。鋐稍止,然猶署情真。恩出長安門,士民觀者如堵。皆歎曰:「是御史,非但口如鐵,其膝、其膽、其骨皆鐵也。」因稱「四鐵御史」。恩母吳氏擊登聞鼓訟冤。不省。

又明年,行可上書請代父死,不許。其冬,事益迫,行可乃刺臂血書疏,自縛闕下,謂:「臣父幼而失怙。祖母吳氏守節教育,底於成立,得為御史。舉家受祿,圖報無地,私憂過計,陷於大辟。祖母吳年已八十餘,憂傷之深,僅餘氣息。若臣父今日死,祖母吳亦必以今日死。臣父死,臣祖母復死,臣煢然一孤,必不獨生。冀陛下哀憐,置臣辟,而赦臣父,茍延母子二人之命。陛下僇臣,不傷臣心。臣被僇,不傷陛下法。謹延頸以俟白刃。」通政使陳經為入奏。帝覽之惻然,令法司再議。尚書聶賢與都御史廷相言,前所引律,情與法不相麗,宜用奏事不實律,輸贖還職,帝不許。乃言恩情重律輕,請戍之邊徼。制可。遂遣戍雷州。而鋐亦後兩月罷矣。

越六年,遇赦還。家居,專為德於鄉。穆宗即位,錄先朝直言。恩年已七十餘,即家拜大理寺丞,致仕。復從有司言,旌行可為孝子。恩年八十一,卒。

行可既脫父於死,越數年登鄉薦。久之,不第。謁選,得光祿署正。遷應天府通判,有善政。弟時可,隆慶五年進士。累官按察使。以文名。

宋邦輔,字子相,東流人。既罷歸,躬耕養親,妻操井臼,子樵牧。歲時與田夫會飲,醉即作歌相和,高鳳動遠邇。士大夫造其門者,屏輿從而後入焉。

薛宗鎧,字子修,行人司正侃從子也。嘉靖二年與從父僑同成進士。授貴溪知縣,補將樂,調建陽。求朱子後,復之,以主祀事。歲饑振倉粟,先發後聞。給由赴京,留拜禮科給事中,以逋賦還任。至則民爭輸,課更最,仍詔入垣。再遷戶科左給事中。吏部尚書汪鋐以私憾斥王臣等,宗鎧白其枉。語具《戚賢傳》。其後,鋐愈驕。會御史曾翀、戴銑劾南京尚書劉龍、聶賢等九人。鋐覆疏,具留之。帝召大學士李時,言:鋐有私,留三人而斥其六。宗鎧與同官孫應奎復言:鋐肆奸植黨,擅主威福,巧庇龍等,上格明詔,下負公論,且縱二子為奸利。鋐疏辨乞休,帝不許。而給事御史翁溥、曹逵等更相繼劾鋐。鋐又抗辨,且極詆宗鎧等挾私。翀復言:「鋐一經論劾,輒肆中傷,諍臣杜口已三年。蔽塞言路,罪莫大,乞立正厥辟。」帝果罷鋐官,而責宗鎧言不早。又惡翀「諍臣杜口」語,執下鎮撫司鞫訊。詞連應奎,逵及御史方一桂,皆杖闕下。斥宗鎧、翀、一桂為民,鐫應奎、溥、逵等級,調外。宗鎧、翀死杖下。時十四年九月朔也。隆慶初,復宗鎧官,贈太常少卿。

曾翀,字習之,霍丘人。以進士授南京刑部主事,改御史。廷杖垂斃,曰:「臣言已行,臣死何憾!」神色無變。隆慶初,贈太常少卿。

楊爵,字伯珍,富平人。年二十始讀書。家貧,燃薪代燭。耕隴上,輒挾冊以誦。兄為吏,忤知縣繫獄。爵投牒直之,並系。會代者至,爵上書訟冤。代者稱奇士,立釋之,資以膏火。益奮於學,立意為奇節。從同郡韓邦奇遊,遂以學行名。

登嘉靖八年進士,授行人。帝方崇飾禮文,爵因使王府還,上言:「臣奉使湖廣,睹民多菜色,挈筐操刃,割道殍食之。假令周公制作,盡復於今,何補老贏饑寒之眾!」奏入,被俞旨。久之,擢御史,以母老乞歸養。母喪,廬墓,冬月筍生。推車糞田,妻饁於旁,見者不知其御史也。服闋,起故官。

帝經年不視朝。歲頻旱,日夕建齋醮,修雷壇,屢興工作。方士陶仲文加宮保,而太僕卿楊最諫死,翊國公郭勛尚承寵用事。二十年元日,微雪。大學士夏言、尚書嚴嵩等作頌稱賀。爵撫膺太息,中宵不能寐。踰月乃上書極諫曰:

今天下大勢,如人衰病已極。腹心百骸,莫不受患。即欲拯之,無措手地。方且奔競成俗,賕賂公行,遇災變而不憂,非祥瑞而稱賀,讒諂面諛,流為欺罔,士風人心,頹壤極矣。諍臣拂士日益遠,而快情恣意之事無敢齟齬於其間,此天下大憂也。去年自夏入秋,恒暘不雨。畿輔千里,已無秋禾。既而一冬無雪,元日微雪即止。民失所望,憂旱之心遠近相同。此正撤樂減膳,憂懼不寧之時,而輔臣言等方以為符瑞,而稱頌之。欺天欺人,不已甚乎!翊國公勛,中外皆知為大奸大蠹,陛下寵之,使諗惡肆毒,群狡趨赴,善類退處。此任用匪人,足以失人心而致危亂者,一也。

臣巡視南城,一月中凍餒死八十人。五城共計,未知有幾。孰非陛下赤子,欲延須臾之生而不能。而土木之功,十年未止。工部屬官增設至數十員,又遣官遠修雷壇。以一方士之故,朘民膏血而不知恤,是豈不可以已乎?況今北寇跳梁,內盜竊發,加以頻年災沴,上下交空,尚可勞民糜費,結怨天下哉?此興作未已,足以失人心而致危亂者,二也。

陛下即位之初,勵精有為,嘗以《敬一箴》頒示天下矣。乃數年以來,朝御希簡,經筵曠廢。大小臣庶,朝參辭謝,未得一睹聖容。敷陳復逆,未得一聆天語。恐人心日益怠媮,中外日益渙散,非隆古君臣都俞吁咈、協恭圖治之氣象也。此朝講不親,足以失人心而致危亂者,三也。

左道惑眾,聖王必誅。今異言異服列於朝苑,金紫赤紱賞及方外。夫保傅之職坐而論道,今舉而畀之奇邪之徒。流品之亂莫以加矣。陛下誠與公卿賢士日論治道,則心正身修,天地鬼神莫不祐享,安用此妖誕邪妄之術,列諸清禁,為聖躬累耶!臣聞上之所好,下必有甚。近者妖盜繁興,誅之不息。風聲所及,人起異議。貽四方之笑,取百世之譏,非細故也。此信用方術,足以失人心而致危亂者,四也。

陛下臨御之初,延訪忠謀,虛懷納諫。一時臣工言過激切,獲罪多有。自此以來,臣下震於天威,懷危慮禍,未聞復有犯顏直諫以為沃心助者。往歲,太僕卿楊最言出而身殞,近日贊善羅洪先等皆以言罷斥。國體治道,所損甚多。臣非為最等惜也。古今有國家者,未有不以任諫而興,拒諫而亡。忠藎杜口,則讒諛交進,安危休戚無由得聞。此阻抑言路,足以失人心而致危亂者,五也。

望陛下念祖宗創業之艱難,思今日守成為不易,覽臣所奏,賜之施行,宗社幸甚。

先是,七年三月,靈寶縣黃河清,帝遣使祭河神。大學士楊一清、張璁等屢疏請賀,御史鄞人周相抗疏言:「河未清,不足虧陛下德。今好諛喜事之臣張大文飾之,佞風一開,獻媚者將接踵。願罷祭告,止稱賀,詔天下臣民毋奏祥瑞,水旱蝗蝻即時以聞。」帝大怒,下相詔獄拷掠之,復杖於廷,謫韶州經歷。而諸慶典亦止不行。

及帝中年,益惡言者,中外相戒無敢觸忌諱。爵疏詆符瑞,且詞過切直。帝震怒,立下詔獄搒掠,血肉狼籍,關以五木,死一夕復甦。所司請送法司擬罪,帝不許,命嚴錮之。獄卒以帝意不測,屏其家人,不許納飲食。屢濱於死,處之泰然。既而主事周天佑、御史浦鋐以救爵,先後箠死獄中,自是無敢救者。

踰年,工部員外郎劉魁,再踰年,給事中周怡,皆以言事同繫,歷五年不釋。至二十四年八月,有神降於乩。帝感其言,立出三人獄。未踰月,尚書熊浹疏言乩仙之妄。帝怒曰:「我固知釋爵,諸妄言歸過者紛至矣。」復令東廠追執之。爵抵家甫十日,校尉至。與共麥飯畢,即就道。尉曰:「盍處置家事?」爵立屏前呼婦曰:「朝廷逮我,我去矣。」竟去不顧,左右觀者為泣下。比三人至,復同繫鎮撫獄,桎梏加嚴,飲食屢絕,適有天幸得不死。二十六年十一月,大高玄殿災,帝禱於露臺。火光中若有呼三人忠臣者,遂傳詔急釋之。

居家二年,一日晨起,大鳥集於舍。爵曰:「伯起之祥至矣。」果三日而卒。隆慶初,復官,贈光祿卿,任一子。萬曆中,賜謚忠介。

爵之初入獄也,帝令東廠伺爵言動,五日一奏。校尉周宣稍左右之,受譴。其再至,治廠事太監徐府奏報。帝以密諭不宜宣,亦重得罪。先後繫七年,日與怡、魁切劘講論,忘其困。所著《周易辨說》、《中庸解》,則獄中作也。

浦鋐,字汝器,文登人。正德十二年進士。授洪洞知縣,有異政。嘉靖初,召為御史。刑部尚書林俊去國,中官秦文已斥復用,鋐疏力爭之。且言武定侯郭勛奸貪,宜罷其兵權。忤旨,奪俸三月。以養母歸。母喪除,起掌河南道事。給事中饒秀考察黜,訐鋐與同官張祿、段汝礪,給事中李鳳來,考功郎餘胤緒,談省署得失,鋐等坐罷。

家居七年,廷臣交薦。起故官,出按陜西,連上四十餘疏。總督楊守禮請破格超擢,未報。而楊爵以直諫繫詔獄,鋐馳疏申救曰:「臣惟天下治亂,在言路通塞。言路通,則忠諫進而化理成;言路塞,則奸諛恣而治道隳。御史爵以言事下獄,幽囚已久,懲創必深。臣行部富平,皆言爵愨誠孚鄉里,孝友式風俗,有古賢士風。且爵本以論郭勛獲罪。今勛奸大露,陛下業致之理,則爵前言未為悖妄。望弘覆載之量,垂日月之照,賜之矜釋,使列朝端,爵必能盡忠補過,不負所學。」疏奏,帝大怒,趣緹騎逮之。秦民遠近奔送,舍車下者常萬人,皆號哭曰:「願還我使君。」鋐赴征,業已病。既至,下詔獄,搒掠備至。除日復杖之百,錮以鐵柙。爵迎哭之,鋐息已絕,徐張目曰:「此吾職也,子無然。」繫七日而卒。穆宗嗣位,恤典視爵等。

周天佐,字子弼,晉江人。嘉靖十四年進士。授戶部主事。屢分司倉場,以清操聞。

二十年夏四月,九廟災,詔百官言時政得失。天佐上書曰:「陛下以宗廟災變,痛自修省,許諸臣直言闕失,此轉災為祥之會也。乃今闕政不乏,而忠言未盡聞,蓋示人以言,不若示人以政。求言之詔,示人以言耳。御史楊爵獄未解,是未示人以政也。國家置言官,以言為職。爵繫獄數月,聖怒彌甚。一則曰小人,二則曰罪人。夫以盡言直諫為小人,則為緘默逢迎之君子不難也。以秉直納忠為罪人,又孰不能為容悅將順之功臣哉?人君一喜一怒,上帝臨之。陛下所以怒爵,果合於天心否耶?爵身非木石,命且不測,萬一溘先朝露,使諍臣飲恨,直士寒心,損聖德不細。願旌爵忠,以風天下。」帝覽奏,大怒。杖之六十,下詔獄。

天佐體素弱,不任楚。獄吏絕其飲食,不三日即死,年甫三十一。比屍出獄,曒日中,雷忽震,人皆失色。天佐與爵無生平交。入獄時,爵第隔扉相問訊而已。大興民有祭於柩而哭之慟者,或問之,民曰:「吾傷其忠之至,而死之酷也。」穆宗即位,贈光祿少卿。天啟初,謚忠愍。

周怡,字順之,太平縣人。為諸生時,嘗曰:「鼎鑊不避,溝壑不忘,可以稱士矣。不然,皆偽也。」從學於王畿、鄒守益。登嘉靖十七年進士,除順德推官。舉卓異,擢吏科給事中。疏劾尚書李如圭、張瓚、劉天和。天和致仕去,如圭還籍待勘,瓚留如故。頃之,劾湖廣巡撫陸傑、工部尚書甘為霖、採木尚書樊繼祖。立朝僅一歲,所摧擊,率當事有勢力大臣。在廷多側目,怡益奮不顧。

二十二年六月,吏部尚書許贊率其屬王與齡、周鈇訐大學士翟鑾、嚴嵩私屬事。帝方響嵩,反責讚,逐與齡等。怡上疏曰:

人臣以盡心報國家為忠,協力濟事為和。未有公卿大臣爭於朝、文武大臣爭於邊,而能修內治、廩外侮者也。大學士鑾、嵩與尚書贊互相詆訐,而總兵官張鳳、周尚文又與總制侍郎翟鵬、督餉侍郎趙廷瑞交惡,此最不祥事,誤國孰甚?

今陛下日事禱祠而四方災祲未銷,歲開輸銀之例而府庫未充,累頒蠲租之令而百姓未蘇,時下選將練士之命而邊境未寧。內則財貨匱而百役興,外則寇敵橫而九邊耗。乃鑾、嵩恁藉寵靈,背公營私,弄播威福,市恩酬怨。夫輔臣真知人賢不肖,宜明告吏部進之退之,不宜挾勢徇私,屬之進退。嵩威靈氣焰,凌轢百司。凡有陳奏,奔走其門,先得意旨而後敢聞於陛下。中外不畏陛下,惟畏嵩久矣。鑾淟涊委靡,言贊雖小心謹畏,然不能以直氣正色銷權貴要求之心,柔亦甚矣。

且直言敢諫之臣,於權臣不利,於朝廷則大利也。御史謝瑜、童漢臣以劾嵩故,嵩皆假他事罪之。諫諍之臣自此箝口,雖有檮杌、驩兜,誰復言之?

帝覽疏大怒,降詔責其謗訕,令對狀。杖之闕下,錮詔獄者再。

隆慶元年起故官。未上,擢太常少卿。陳新政五事,語多刺中貴。時近習方導上宴遊,由是忤旨,出為登萊兵備僉事。給事中岑用賓為怡訟,不納。改南京國子監司業。復召為太常少卿,未任卒。天啟初,追謚恭節。

劉魁,字煥吾,泰和人。正德中登鄉薦。受業王守仁之門。嘉靖初,謁選,得寶慶府通判。歷鈞州知州,潮州府同知。所至潔己愛人,扶植風教。入為工部員外郎,疏陳安攘十事,帝嘉納。二十一年秋,帝用方士陶仲文言,建祐國康民雷殿於太液池西。所司希帝意,務宏侈,程工峻急。魁欲諫,度必得重禍,先命家人鬻棺以待。遂上帝曰:「頃泰享殿、大高玄殿諸工尚未告竣。內帑所積幾何?歲入幾何?一役之費動至億萬。土木衣文繡,匠作班朱紫,道流所居擬於宮禁。國用已耗,民力已竭,而復為此不經無益之事,非所以示天下後世。」帝震怒,杖於廷,錮之詔獄。時御史楊爵先已逮繫,既而給事中周怡繼至,三人屢瀕死,講誦不輟。繫四年得釋,未幾復追逮之。魁未抵家,緹騎已先至,繫其弟以行。魁在道聞之,趣就獄,復與爵、怡同繫。時帝怒不測,獄吏懼罪,窘迫之愈甚,至不許家人通飲食。而三人處之如前,無幾微尤怨。又三年,與爵、怡同釋,尋卒。隆慶初,贈恤如制。

沈束,字宗安,會稽人。父侭,邠州知州。束登嘉靖二十三年進士,除徽州推官,擢禮科給事中。時大學士嚴嵩擅政。大同總兵官周尚文卒,請恤典,嚴嵩格不予。束言:「尚文為將,忠義自許。曹家莊之役,奇功也。雖晉秩,未壽勛,宜贈封爵延子孫。他如董暘、江瀚,力抗強敵,繼之以死。雖已廟祀,宜賜祭,以彰死事忠。今當事之臣,任意予奪,冒濫或悻蒙,忠勤反捐棄,何以鼓士氣,激軍心?」疏奏,嵩大恚,激帝怒,下吏部都察院議。聞淵、屠僑等言束無他腸,第疏狂當治。帝愈怒,奪淵、僑俸,下束詔獄。已,刑部坐束奏事不實,輸贖還職。特命杖於廷,仍錮詔獄。時束入諫垣未半歲也。踰年,俺答薄都城。司業趙貞吉以請寬束得罪,自是無敢言者。

束繫久,衣食屢絕,惟日讀《周易》為疏解。後同邑沈練劾嵩,嵩疑與束同族為報復,令獄吏械其手足。徐階勸,得免。迨嵩去位,束在獄十六年矣,妻張氏上書言:「臣夫家有老親,年八十有九,衰病侵尋,朝不計夕。往臣因束無子,為置妾潘氏。比至京師,束已繫獄,潘矢志不他適。乃相與寄居旅舍,紡織以供夫衣食。歲月積深,悽楚萬狀。欲歸奉舅,則夫之饘粥無資。欲留養夫,則舅又旦暮待盡。輾轉思維,進退無策。臣願代夫繫獄,令夫得送父終年,仍還赴繫,實陛下莫大之德也。」法司亦為請,帝終不許。

帝深疾言官,以廷杖遣戍未足遏其言,乃長繫以困之。而日令獄卒奏其語言食息,謂之監帖。或無所得,雖諧語亦以聞。一日,鵲噪於束前,束謾曰:「豈有喜及罪人耶?」卒以奏,帝心動。會戶部司務何以尚疏救主事海瑞,帝大怒,杖之,錮詔獄,而釋束還其家。

束還,父已前卒。束枕塊飲水,佯狂自廢。甫兩月,世宗崩,穆宗嗣位。起故官,不赴。喪除,召為都給事中。旋擢南京右通政。復辭疾。布衣蔬食,終老於家。束繫獄十八年。比出,潘氏猶處子也,然束竟無子。

沈鍊,字純甫,會稽人。嘉靖十七年進士。除溧陽知縣。用伉倨,忤御史,調茬平。父憂去,補清豐,入為錦衣衛經歷。

鍊為人剛直,嫉惡如仇,然頗疏狂。每飲酒輒箕踞笑傲,旁若無人。錦衣帥陸炳善遇之。炳與嚴嵩父子交至深,以故鍊亦數從世蕃飲。世蕃以酒虐客,鍊心不平,輒為反之,世蕃憚不敢較。

會俺答犯京師,致書乞貢,多嫚語。下廷臣博議,司業趙貞吉請勿許。廷臣無敢是貞吉者,獨鍊是之。吏部尚書夏邦謨曰:「若何官?」鍊曰:「錦衣衛經歷沈鍊也。大臣不言,故小吏言之。」遂罷議。鍊憤國無人,致寇猖狂,疏請以萬騎護陵寢,萬騎護通州軍儲,而合勤王師十餘萬人,擊其惰歸,可大得志。帝弗省。

嵩貴幸用事,邊臣爭致賄遺。及失事懼罪,益輦金賄嵩,賄日以重。鍊時時搤腕。一日從尚寶丞張遜業飲,酒半及嵩,因慷慨罵詈,流涕交頤。遂上疏言:「昨歲俺答犯順,陛下奮揚神武,欲乘時北伐,此文武群臣所願戮力者也。然制勝必先廟算,廟算必先為天下除奸邪,然後外寇可平。今大學士嵩,貪婪之性疾入膏肓,愚鄙之心頑於鐵石。當主憂臣辱之時,不聞延訪賢豪,咨詢方略,惟與子世蕃規圖自便。忠謀則多方沮之,諛諂則曲意引之。要賄鬻官,沽恩結客。朝廷賞一人,曰:『由我賞之』;罰一人,曰:『由我罰之』。人皆伺嚴氏之愛惡,而不知朝廷之恩威,尚忍言哉!姑舉其罪之大者言之。納將帥之賄,以啟邊陲之釁,一也。受諸王餽遺,每事陰為之地,二也。攬吏部之權,雖州縣小吏亦皆貨取,致官方大壞,三也。索撫按之歲例,致有司遞相承奉,而閭閻之財日削,四也。陰制諫官,俾不敢直言,五也。妒賢嫉能,一忤其意,必致之死,六也。縱子受財,斂怨天下,七也。運財還家,月無虛日,致道途驛騷,八也。久居政府,擅寵害政,九也。不能協謀天討,上貽君父憂,十也。」因併論邦謨諂諛黷貨狀。請均罷斥,以謝天下。帝大怒,搒之數十,謫佃保安。

既至,未有館舍。賈人某詢知其得罪故,徙家舍之。里長老亦日致薪米,遣子弟就學。鍊語以忠義大節,皆大喜。塞外人素戇直,又諗知嵩惡,爭詈嵩以快鍊。鍊亦大喜,日相與詈嵩父子為常。且縛草為人,象李林甫、秦檜及嵩,醉則聚子弟攢射之。或踔騎居庸關口,南向戟手詈嵩,復痛哭乃歸。語稍稍聞京師,嵩大恨,思有以報鍊。

先是,許論總督宣、大,常殺良民冒功,鍊貽書誚讓。後嵩黨楊順為總督。會俺答入寇,破應州四十餘堡,懼罪,欲上首功自解,縱吏士遮殺避兵人,逾於論。鍊遺書責之加切。又作文祭死事者,詞多刺順。順大怒,走私人白世蕃,言鍊結死士擊劍習射,意叵測。世蕃以屬巡按御史李鳳毛。鳳毛謬謝曰:「有之,已陰散其黨矣。」既而代鳳毛者路楷,亦嵩黨也。世蕃屬與順合圖之,許厚報。兩人日夜謀所以中鍊者。會蔚州妖人閻浩等素以白蓮教惑眾,出入漠北,泄邊情為患。官軍捕獲之,詞所連及甚眾。順喜,謂楷曰:「是足以報嚴公子矣。」竄鍊名其中,誣浩等師事鍊,聽其指揮,具獄上。嵩父子大喜。前總督論適長兵部,竟覆如其奏。斬鍊宣府市,戍子襄極邊。予順一子錦衣千戶,楷待銓五品卿寺。時三十六年九月也。順曰:「嚴公薄我賞,意豈未愜乎?」取鍊子袞、褒杖殺之,更移檄逮襄。襄至,掠訊方急,會順、楷以他事逮,乃免。

後嵩敗,世蕃坐誅。臨刑時,鍊所教保安子弟在太學者,以一帛署鍊姓名官爵於其上,持入市。觀世蕃斷頭訖,大呼曰:「沈公可瞑目矣。」因慟哭而去。

隆慶初,詔褒言事者。贈鍊光祿少卿,任一子官。襄乃上書,言順、楷殺人媚奸狀。給事中魏時亮、陳瓚亦相繼論之。遂下順、楷吏,論死。天啟初,謚忠愍。

楊繼盛,字仲芳,容城人。七歲失母。庶母妒,使牧牛。繼盛經里塾,睹里中兒讀書,心好之。因語兄,請得從塾師學。兄曰:「若幼,何學?」繼盛曰:「幼者任牧牛,乃不任學耶?」兄言於父,聽之學,然牧不廢也。年十三歲,始得從師學。家貧,益自刻厲。舉鄉試,卒業國子監,徐階丞賞之。嘉靖二十六年登進士。授南京吏部主事。從尚書韓邦奇遊,覃思律呂之學,手製十二律,吹之聲畢和。邦奇大喜,盡以所學授之,繼盛名益著。召改兵部員外郎。

俺答躪京師,咸寧侯仇鸞以勤王故有寵。帝命鸞為大將軍,倚以辦寇。鸞中情怯,畏寇甚。方請開互市市馬,冀與俺答媾,幸無戰鬥,固恩寵。繼盛以為讎恥未雪,遽議和示弱,大辱國,乃奏言十不可、五謬。大略謂:

互市者,和親別名也。俺答蹂躪我陵寢,虔劉我赤子。天下大讎也,而先之和。不可一。往下詔北伐,天下曉然知聖意,日夜征繕助兵食。忽更之曰和,失信於天下。不可二。以堂堂中國,與之互市,冠履倒置。不可三。海內豪傑爭磨礪待試,一旦委置無用。異時欲號召,誰復興起?不可四。使邊鎮將帥以和議故,美衣媮食,馳懈兵事。不可五。往時邊卒私通境外,吏率裁禁,今乃導之使與通。不可六。盜賊伏莽,徒懾國威不敢肆耳,今知朝廷畏怯,睥睨之漸必開。不可七。俺答往歲深入,乘我無備故也。備之一歲,以互市終。彼謂國有人乎?不可八。或俺答負約不至;至矣,或陰謀伏兵突入;或今日市,明日復寇;或以下馬索上直。不可九。歲帛數十萬,得馬數萬匹。十年以後,帛將不繼。不可十。

議者曰:「吾外為市以羈縻之,而內修我甲。」此一謬也。夫寇欲無厭,其以釁終明甚。茍內修武備,安事羈縻?曰:「吾陰市,以益我馬」。此二謬也。夫和則不戰,馬將焉用?且彼寧肯予我良馬哉?曰:「市不已,彼且入貢」。此三謬也。夫貢之賞不貲,是名美而實大損也。曰:「俺答利我市,必無失信」。此四謬也。吾之市,能盡給其眾乎?能信不給者之無入掠乎?曰:「佳兵不祥」。此五謬也。敵加己而應之,何佳也?人身四肢皆癰疽,毒日內攻,而憚用藥石可乎?

夫此十不可、五謬,明顯易見。蓋有為陛下主其事者,故公卿大夫知而莫為一言。陛下宜奮獨斷,悉按諸言互市者,發明詔選將練兵。不出十年,臣請為陛下竿俺答之首於槁街,以示天下萬世。

疏入,帝頗心動,下鸞及成國公朱希忠,大學士嚴嵩、徐階、呂本,兵部尚書趙錦,侍郎聶豹、張時徹議。鸞攘臂詈曰:「豎子目不睹寇,宜其易之。」諸大臣遂言遣官已行,勢難中止。帝尚猶豫,鸞復進密疏。乃下繼盛詔獄,貶狄道典史。其地雜番,俗罕知詩書。繼盛簡子弟秀者百餘人,聘三經師教之。鬻所乘馬,出婦服裝,市田資諸生。縣有煤山,為番人所據,民仰薪二百里外。繼盛召番人諭之,咸服曰:「楊公即須我曹穹帳亦舍之,況煤山耶?」番民信愛之,呼曰「楊父」。

已而俺答數敗約入寇,鸞奸大露,疽發背死,戮其屍。帝乃思繼盛言,稍遷諸城知縣。月餘調南京戶部主事,三日遷刑部員外郎。當是時,嚴嵩最用事。恨鸞凌己,心善繼盛首攻鸞,欲驟貴之,復改兵部武選司。而繼盛惡嵩甚於鸞。且念起謫籍,一歲四遷官,思所以報國。抵任甫一月,草奏劾嵩,齋三日乃上奏曰:

臣孤直罪臣,蒙天地恩,超擢不次。夙夜祗懼,思圖報稱,蓋未有急於請誅賊臣者也。方今外賊惟俺答,內賊惟嚴嵩,未有內賊不去,而可除外賊者。去年春雷久不聲,占曰:「大臣專政」。冬日下有赤色,占曰:「下有叛臣」。又四方地震,日月交食。臣以為災皆嵩致,請以嵩十大罪為陛下陳之。

高皇帝罷丞相,設立殿閣之臣,備顧問視制草而已,嵩乃儼然以丞相自居。凡府部題覆,先面白而後草奏。百官請命,奔走直房如市。無丞相名,而有丞相權。天下知有嵩,不知有陛下。是壞祖宗之成法。大罪一也。

陛下用一人,嵩曰「我薦也」;斥一人,曰「此非我所親,故罷之」。陛下宥一人,嵩曰「我救也」;罰一人,曰「此得罪於我,故報之」。伺陛下喜怒以恣威福。群臣感嵩甚於感陛下,畏嵩甚於畏陛下。是竊君上之大權。大罪二也。

陛下有善政,嵩必令世蕃告人曰:「主上不及此,我議而成之」。又以所進揭帖刊刻行世,名曰《嘉靖疏議》,欲天下以陛下之善盡歸於嵩。是掩君上之治功。大罪三也。

陛下令嵩司票擬,蓋其職也。嵩何取而令子世蕃代擬?又何取而約諸義子趙文華輩群聚而代擬?題疏方上,天語已傳。如沈鍊劾嵩疏,陛下以命呂本,本即潛送世蕃所,令其擬上。是嵩以臣而竊君之權,世蕃復以子而盜父之柄,故京師有「大丞相、小丞相」之謠。是縱姦子之僭竊。大罪四也。

嚴效忠、嚴鵠,乳臭子耳,未嘗一涉行伍。嵩先令效忠冒兩廣功,授錦衣所鎮撫矣。效忠以病告,鵠襲兄職。又冒瓊州功,擢千戶。以故總督歐陽必進躐掌工部,總兵陳圭幾統後府,巡按黃如桂亦驟亞太僕。既藉私黨以官其子孫,又因子孫以拔其私黨。是冒朝廷之軍功。大罪五也。

逆鸞先已下獄論罪,賄世蕃三千金,薦為大將。鸞冒擒哈舟丹兒功,世蕃亦得增秩。嵩父子自誇能薦鸞矣,及知陛下有疑鸞心,復互相排詆,以泯前迹。鸞勾賊,而嵩、世蕃復勾鸞。是引背逆之姦臣。大罪六也。

前俺答深入,擊其惰歸,此一大機也。兵部尚書丁汝夔問計於嵩,嵩戒無戰。及汝夔逮治,嵩復以論救紿之。汝夔臨死大呼曰:嵩誤我。是誤國家之軍機。大罪七也。

郎中徐學詩劾嵩革任矣,復欲斥其兄中書舍人應豐。給事厲汝進劾嵩謫典史矣,復以考察令吏部削其籍。內外之臣,被中傷者何可勝計?是專黜陟之大柄。大罪八也。

凡文武遷擢,不論可否,但衡金之多寡而畀之。將弁惟賄嵩,不得不朘削士卒;有司惟賄嵩,不得不掊剋百姓。士卒失所,百姓流離,毒遍海內。臣恐今日之患不在境外而在域中。是失天下之人心。大罪九也。

自嵩用事,風俗大變。賄賂者薦及盜跖,疏拙者黜逮夷、齊。守法度者為迂疏,巧彌縫者為才能。勵節介者為矯激,善奔者為練事。自古風俗之壞,未有甚於今日者。蓋嵩好利,天下皆尚貪。嵩好諛,天下皆尚諂。源之弗潔,流何以澄?是敝天下之風俗。大罪十也。

嵩有是十罪,而又濟之以五奸。知左右侍從之能察意旨也,厚賄結納。凡陛下言動舉措,莫不報嵩。是陛下之左右皆賊嵩之間諜也。以通政司之主出納也,用趙文華為使。凡有疏至,先送嵩閱竟,然後入御。王宗茂劾嵩之章停五日乃上,故嵩得展轉遮飾。是陛下之喉舌乃賊嵩之鷹犬也。畏廠衛之緝訪也,令子世蕃結為婚姻。陛下試詰嵩諸孫之婦,皆誰氏乎?是陛下之爪牙皆賊嵩之瓜葛也。畏科道之多言也,進士非其私屬,不得預中書、行人選。推官、知縣非通賄,不得預給事、御史選。既選之後,入則杯酒結歡,出則餽食盡相屬。所有愛憎,授之論刺。歷俸五六年,無所建白,即擢京卿。諸臣忍負國家,不敢忤權臣。是陛下之耳目皆賊嵩之奴隸也。科道雖入籠絡,而部寺中或有如徐學詩之輩亦可懼也,令子世蕃擇其有才望者,羅置門下。凡有事欲行者,先令報嵩,預為布置,連絡蟠結,深根固蒂,各部堂司大半皆其羽翼。是陛下之臣工皆賊嵩之心膂也。陛下奈何愛一賊臣,而忍百萬蒼生陷於塗炭哉?

至如大學士徐階蒙陛下特擢,乃亦每事依違,不敢持正,不可不謂之負國也。願陛下聽臣之言,察嵩之奸。或召問裕、景二王,或詢諸閣臣。重則置憲,輕則勒致仕。內賊既去,外賊自除。雖俺答亦必畏陛下聖斷,不戰而喪膽矣。

疏入,帝已怒。嵩見召問二王語,喜謂可指此為罪,密構於帝。帝益大怒,下繼盛詔獄,詰何故引二王。繼盛曰:「非二王誰不懾嵩者!」獄上,乃杖之百,令刑部定罪。侍郎王學益,嵩黨也。受嵩屬,欲坐詐傳親王令旨律絞,郎中史朝賓持之。嵩怒,謫之外。於是尚書何鰲不敢違,竟如嵩指成獄,然帝猶未欲殺之也。繫三載,有為營救於嵩者。其黨胡植、鄢懋卿怵之曰:「公不睹養虎者耶,將自貽患。」嵩頷之。會都御史張經、李天寵坐大辟。嵩揣帝意必殺二人,比秋審,因附繼盛名並奏,得報。其妻張氏伏闕上書,言:「臣夫繼盛誤聞市井之言,尚狃書生之見,遂發狂論。聖明不即加戮,俾從吏議。兩經奏讞,俱荷寬恩。今忽闌入張經疏尾,奉旨處決。臣仰惟聖德,昆蟲草木皆欲得所,豈惜一回宸顧,下垂覆盆?倘以罪重,必不可赦,願即斬臣妾首,以代夫誅。夫雖遠禦魑魅,必能為疆場效死,以報君父。」嵩屏不奏,遂以三十四年十月朔棄西市,年四十。臨刑賦詩曰:「浩氣還太虛,丹心照千古。生平未報恩,留作忠魂補。」天下相與涕泣傳頌之。

初,繼盛之將杖也,或遺之蚺蛇膽。卻之曰:「椒山自有膽,何蚺蛇為!」椒山,繼盛別號也。及入獄,創甚。夜半而蘇,碎磁碗,手割腐肉。肉盡,筋掛膜,復手截去。獄卒執燈顫欲墜,繼盛意氣自如。朝審時,觀者塞衢,皆歎息,有泣下者。後七年,嵩敗。穆宗立,恤直諫諸臣,以繼盛為首。贈太常少卿,謚忠愍,予祭葬,任一子官。已,又從御史郝傑言,建祠保定,名旌忠。

後繼盛論馬市得罪者,何光裕、龔愷。光裕,字思問,梓潼人。嘉靖二十年進士。改庶吉士,除刑科給事中。偕同官楊上林、齊譽請召遺佚。帝可之,已而報罷。巡視京營,劾罷尚書路迎。與給事中謝登之、御史曾佩建議節財,冗費大省。邊事迫,命清理諸陵守衛軍,條上祛弊七事,多報可。

屢遷兵科都給事中。都指揮呂元夤緣得錦衣,總旗王松冒功襲千戶,光裕皆舉奏之。兵部尚書趙錦疏辯,帝斥元,下松都察院獄,而奪錦等俸。

仇鸞之開馬市也,命尚書史道主之。徇俺答請,以粟豆易牛羊。光裕與御史龔愷等劾道:「委靡遷就。馬市既開,復請封號。今其表意在請乞,而道以為謝恩。況表文非出賊手。道不去,則彼有無厭之求,我無必戰之志,誤國事不小。」時帝方響鸞,責光裕等借道論鸞,以探朝廷。杖光裕、愷八十,餘奪俸。光裕不勝杖,卒。隆慶初,贈太常不卿。

愷既杖,官如故。尋列靖江王驕恣狀,疏止大征粵寇。終湖廣副使。愷,字次元,松江華亭人。嘉靖二十六年進士。

楊允繩,字翼少,松江華亭人。嘉靖二十三年進士。授行人。久之,擢兵科給事中。嚴嵩獨相,有詔廷推閣員。允繩偕同官王德、沈束、陳慎簡輔臣、收錄遺佚二事。未幾,奉命會英國公張溶、撫寧侯朱岳、定西侯蔣傳等簡應襲子弟於閱武場。指揮鄭璽忽傳寇至,溶等皆懼走,允繩獨不動,因奏之。褫璽職,奪溶、岳營務,罰傳等俸,由是知名。又劾罷兵部尚書趙廷瑞。

居諫垣未幾,疏屢上。言提學憲臣宜簡行誼,府州縣職宜量地煩簡為三等,皆報可。俺答入犯,朝議急兵事。允繩請令五軍都督府、府軍前衛及錦衣衛堂上官,每遇考選軍政之歲,各具疏自陳,聽科道官拾遺;騰驤四衛及錦衣衛指揮以下,聽兵部考察。詔皆從之,著為令。已,又陳禦邊四事,報可。再遷戶科左給事中。謝病歸。久之,起故官。

三十四年九月上疏言倭患,因推弊原,謂:「近者督撫命令不行於有司,非官不尊、權不重也。督撫蒞任,例賂權要,名『謝禮』。有所奏請,佐以苞苴,名『候禮』。及俸滿營遷,避難求去,犯罪欲彌縫,失事希芘覆,輸賄載道,為數不貲。督撫取諸有司,有司取諸小民。有司德色以事上,督撫壎顏以接下。上下相蒙,風俗莫振。不肖吏又乾沒其間,指一科十。孑遺待盡之民必將挺而為盜,陷憂不止海島間也。」

其冬巡視光祿。光祿丞胡膏偽增物直,允繩與同事御史張巽言劾之。下法司按驗。膏窘,言:「玄典隆重,所用品物,不敢徒取充數。允繩憎臣簡別太精,斥言醮齋之用,取具可耳,何必精擇?其欺謗玄修如此。」帝遂大怒,下允繩及膏詔獄。刑部尚書何鰲當允繩儀仗內訴事不實律絞,帝命仍與巽言杖於廷。巽言奪三官。膏調外任。居五年,允繩竟死西市。先是,有馬從謙者,以謗醮齋杖死。穆宗即位,贈允繩光祿少卿,予一子官。天啟初,謚忠恪。膏尋以貪墨被劾,誅。

馬從謙,字益之,溧陽人。嘉靖十年舉順天鄉試第一。越三年成進士,授工部主事。出治二洪,有政績。改官主客,擢尚寶丞,掌內閣制誥。章聖太后崩,勸帝行三年喪,不報。稍進光祿少卿。提督中官杜泰乾沒歲鉅萬,為從謙奏發,泰因誣從謙誹謗。巡視給事中孫允中、御史狄斯彬劾泰,如從謙言。帝方惡人言醮齋,而從謙奏頗及之,怒下從謙及泰詔獄。所司言誹謗無左證,帝益怒。下從謙法司,以允中、斯彬黨庇,謫邊方雜職。法司擬從謙戍遠邊。帝命廷杖八十,戍煙瘴,竟死杖下。而泰以能發謗臣罪,宥之。時三十一年十二月也。久之,光祿寺災,帝曰:「此馬從謙餘孽所致耳。」隆慶初,恤先朝建言杖死諸臣。中官追恨從謙,沮之。給事中王治、御史龐尚鵬力爭。帝以從謙所犯,比子罵父,終不許。

允中,太原人。後屢遷應天府丞。斯彬,從謙同邑人。

贊曰:語有之:「君仁則臣直」。當世宗之代,何直臣多歟!重者顯戮,次乃長繫,最幸者得貶斥,未有茍全者。然主威愈震,而士氣不衰,批鱗碎首者接踵而不可遏。觀其蒙難時,處之泰然,足使頑懦知所興起,斯百餘年培養之效也。


\end{pinyinscope}