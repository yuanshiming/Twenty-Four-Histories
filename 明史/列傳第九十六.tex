\article{列傳第九十六}

\begin{pinyinscope}
張芹汪應軫蕭鳴鳳(高公韶}}齊之鸞袁宗儒許相卿顧濟子章志章僑餘珊汪珊韋商臣黎貫王汝梅彭汝實鄭自璧戚賢劉繪子黃裳錢薇洪垣方瓘呂懷周思兼顏鯨

張芹,字文林,峽江人。弘治十五年進士。授福州推官。正德中,召為南京御史。寧夏既平,大學士李東陽亦進官廕子。芹抗疏曰:「東陽謹厚有餘,正直不足;儒雅足重,節義無聞。逆瑾亂政,東陽為顧命大臣,既不能遏之於始,及惡跡既彰,又不能力與之抗。脂韋順從,惟其指使。今叛賊底平,東陽何力?冒功受賞,何以服人心?乞立賜罷斥,奪其加恩,為大臣事君不忠者戒。」疏出,東陽涕泣不能辯。帝責芹沽名,令對狀。芹請罪,停俸三月。

給事中竇明言事下獄,芹疏救之。帝嘗馳馬傷,編修王思切諫,坐遠戍。芹曰:「彼非諫官尚爾,吾儕可坐視乎!」遂上疏曰:「孟子言:『從獸無厭謂之荒』。老聃曰:『馳騁田獵,使人心發狂』。心狂志荒,何事不忘?皆甚言無益有害也。今輕萬乘之尊,乘危冒險,萬一有不可諱,皇嗣未誕,如宗廟社稷何!」帝不省。

尋出為徽州知府。寧王宸濠反,言者以芹家江西,慮賊劫其親屬,取道出徽。乃改知杭州。已,復還徽州。嘉靖初,遷浙江海道副使。歷右參政、右布政使。坐為海道時倭人爭貢誤傷居民,罷歸。

芹事繼母孝,持身儉素,枲袍糲食終其身。

汪應軫,字子宿,浙江山陰人。少有志操。正德十二年成進士,選庶吉士。十四年,詔將南巡。應軫抗言:「自下詔以來,臣民旁皇,莫有固志。臨清以南,率棄業罷市,逃竄山谷。茍不即收成命,恐變生不測。昔谷永諫漢成帝,謂:『陛下厭高美之尊號,好匹夫之卑字。數離深宮,挺身晨夜,與群小相逐。典門戶奉宿衛者,執干戈而守空宮』。其言切中於今。夫谷永,諧諛之臣;成帝,庸闇之主。永言而成帝容之。豈以陛下聖明,不能俯納直諫哉?」疏入,留中。繼復偕修撰舒芬等連章以請。跪闕門,受杖幾斃。

教習竣,擬授給事中。有旨補外,遂出為泗州知州。土瘠民惰,不知農桑。應軫勸之耕,買桑植之。募江南女工,教以蠶繅織作。由是民足衣食。帝方南征,中使驛騷道路。應軫率壯夫百餘人列水次,舟至,即挽之出境。車駕駐南京,命州進美婦善歌吹者數十人。應軫言:「州子女荒陋,無以應敕旨。臣向募有桑婦,請納之宮中,傳受蠶事。」事遂寢。

世宗踐阼,召為戶科給事中。山東礦盜起,掠東昌、袞州,流入畿輔、河南境。應軫奏言:「弭盜與禦寇不同。禦寇之法,驅之境外而已。若弭盜而縱使出境,是嫁禍於鄰國也。凡一方有警,不行撲滅,致延蔓他境者,俱宜重論。」報可。在科歲餘,所上凡三十餘疏,咸切時弊。以便養,乞改南,遂調南京戶科。張璁、桂萼在南京,方議追尊獻皇帝。雅知應軫名,欲倚以自助。應軫與議不合,即奏請遵禮經、崇正統,以安人心。不報。

嘉靖三年春,出為江西僉事。居二年,具疏引疾,不俟命而歸,為巡按所劾。詔所司逮問。應軫自陳親老,鮮兄弟,乞休侍養。吏部為之請,乃免逮。久之,廷臣交薦,起故官,視江西學政。父艱歸,病卒。

蕭鳴鳳,字子雝,浙江山陰人。少從王守仁遊,舉鄉試第一。正德九年成進士,授御史。副使胡世寧下獄,抗章救之。同官內江高公韶劾王瓊誤邊計,言:「松潘副將吳坤請增設總兵於成都,瓊即以坤任之。花當本我屬衛,日憑陵。由本兵非人,致小醜輕中國。」瓊怒,奏訐公韶。中旨責公韶陰結外蕃,交通間諜,令首實。鳴鳳上疏曰:「公韶劾瓊,所論者天下之事。瓊不當逞忿恣辯,以箝諫官口。」中旨責鳴鳳黨庇,而謫公韶富民典史。鳴鳳又劾江彬恃寵恣肆,蔓將難圖。士論壯之。尋巡視山海諸關。武宗將出塞捕虎,鳴鳳疏諫,因具陳官司掊剋,軍民疾苦狀。不報。引疾歸。

起督南畿學政。諸生以比前御史陳選,曰「陳,泰山;蕭,北斗」。嘉靖初,遷河南副使,仍督學政。考察拾遺被劾。吏部惜其學行,調為湖廣兵備副使。明年復改督廣東學政。鳴鳳三督學政,廉無私。然性剛狠,以憤撻肇慶知府鄭璋。璋慚恚,投劾去,由是物論大嘩。八年考察,兩京言官交章論,坐降調。已,與璋相詆訐。皆下巡按御史逮治。鳴鳳遂不出。

公韶,正德中為御史,嘗劾總兵官郭勛罪。朵顏花當入寇,又劾總兵官遂安伯陳鏸、中官王欣、巡撫王倬,鏸坐解職。世宗立,起謫籍。歷右副都御史,巡撫江西。終戶部右侍郎。

齊之鸞,字瑞卿,桐城人。正德六年進士。改庶吉士,授刑科給事中。十一年冬,帝將置肆於京城西偏。之鸞上言:「近聞有花酒鋪之設,或云車駕將臨幸,或云朝廷收其息。陛下貴為天子,富有四海,乃至競錐刀之利,如倡優館舍乎?」應州奏捷,帝降敕:「總督軍務威武大將軍總兵官硃壽剿寇有功,宜特加公爵」。制下,舉朝大駭。之鸞偕諸給事中上言:「自古天子亦有親臨戰陣勘定禍亂者,成功之後,不過南面受賀,勒之金石,播之歌頌已耳,未有加爵酬勞,如今日之顛倒者。不知陛下何所取義,為此不祥之舉,以駴天下耳目,貽百世之譏笑也。」

未幾,請召還編修王思,給事中張原、陳鼎,御史周廣、高公韶、李熙、徐文華、李穩、施儒、劉寓生,僉事韓邦奇,評事羅僑,皆不聽。帝將巡邊,復自稱威武大將軍。御史袁宗儒疏諫,大學士楊廷和、蔣冕、毛紀以去就爭。之鸞偕同官言:「三臣居師保之重,身係安危,邇者先後稱疾。今六飛臨邊踰月矣,宗廟社稷百官萬姓寄空城中。人心危疑,幾務叢積,復杜門求決去。萬一事起倉卒,至於僨敗,三臣將何辭謝天下?乞陛下以社稷為重,亟返宸居,與大臣共圖治理。」已而御史李潤等復爭之,卒不省。

之鸞再遷兵科左給事中。中官馬永成死,詔授其家九十餘人官。之鸞言:「永成貴顯,用事十有餘年,兄弟子姪皆高爵美官。而其儕復為陳乞,將及百人。永成何功,恩濫如此,恐天下聞而解體也。」帝將南巡,之鸞偕同官及御史楊秉中等交章力諫。章入二日,未報。之鸞等不知所出,伏闕俟命,自辰至申。帝令中官傳諭,乃退。明日託疾免朝,欲以為之鸞等罪。會諸曹郎黃鞏等聯章力諫,乃止不行。然鞏等下獄杖譴,之鸞輩亦不敢救也。宸濠反,張忠、許泰等南征,命之鸞偕左給事中祝續從軍紀功。未至,賊已滅。群小忌王守仁,譖毀百端,之鸞力白其誣。忠、泰廣搜逆黨,株引無辜,之鸞多所開釋。且請蠲田租、停力役、寬逋負,帝頗採納。初冒徐姓,至是始復焉。

世宗踐阼,首上疏言:「祖宗法制,悉紛更於群小。補救之道,在先定聖志,次廣言路。先朝元凶雖去,根據盤互,連蔓滋多,猶恐巧相營結,或邀定策之賞,或假迎扈之勞,以取憐固寵。天下事豈堪若輩更壞!言者久遏於權奸,欲吐忠鯁懣憤之氣,必有不顧忌諱,至於逆耳者,在嘉納而優容之。若稍或抑裁,則小人又乘之以讎忠直。言路一塞,不可復開,大為新政累矣。陛下誠舉邇年亂政,盡返其初,中興之烈可以立睹。」帝嘉納之。又劾許泰及兵部尚書王憲,二人竟獲譴。

其秋大計京官,被中傷,謫崇德丞。屢遷寧夏僉事。饑民採蓬子為食,之鸞為取二封,一進於帝,一以貽閣臣。且言時事可憂者三,可惜者四,語極切。帝付之所司。時方大修邊牆,之鸞董役。巡撫胡東皋稱其能,舉以自代。歷河南、山東副使。召為順天府丞。未行,盜發,留鎮撫。尋擢河南按察使。卒官。

袁宗儒,字醇夫,雄縣人。正德三年進士。授御史。十二年冬,帝在大同,以郊祀將回鑾,既而復止。宗儒率同官力諫。明年夏,孝貞純皇后將葬,帝還京。宗儒等復引災異,力請罷皇店,遣邊兵,既又諫帝巡邊。語極危切。皆不報。擢大理寺丞。嘉靖三年爭「大禮」,廷杖。歷官右僉都御史,巡撫貴州。吏部尚書桂萼議宗儒改調,遂解職歸。未幾,起鄖陽,改山東。坐屬吏振饑無術,不能覺察,罷免。以薦起左副都御史。扈蹕承天,還京卒。

許相卿,字伯台,海寧人。正德十二年進士。世宗立,授兵科給事中。宦官張銳、張忠有罪論死,帝復寬之。給事中顧濟疏爭,帝下所司議,卒欲寬其死。相卿言:「天下望陛下為孝皇,陛下奈何自處以正德?」帝議加興獻帝皇號,相卿復爭之。

嘉靖二年詔廕中官張欽義子李賢為錦衣世襲指揮。相卿言:「于謙子冕止錦衣千戶,王守仁子正憲止錦衣百戶。賢中官廝養,反過之。忠勛大臣裔曾不若近倖奴,殉國勤事之臣誰不解體?部臣彭澤、科臣許復禮、安磐相繼言之,悉拒不納。毋乃重內侍而輕士大夫哉!」

尋復言:「天下政權出於一則治,二三則亂;公卿大夫參議則治,匪人僭乾則亂。陛下繼統之初,登用老成,嘉納忠讜,裁抑僥倖,竄殛憸邪,可謂明且剛矣。曾未再期,偏聽私暱,秕政亟行,明少蔽,剛少遜,操權未得其術,而陰伺旁竊者得居中制之。如崔文以左道罔上,師保臺諫言之而不聽。羅洪載守職逮繫,廷臣疏七十上而不行。近又庇崔文奴奪法司之守,斥林俊以違旨,怒言官之奏擾。事涉中人,曲降溫旨,犯法不罪,請乞必從。此與正德朝何異哉!俊,國之望也,其去志決矣。俊去,類俊者必不留。陛下將與二三近習私人共理天下乎?今日天下,與先朝異。武宗時,勢已阽危,然元氣猶壯,調劑適宜,可以立起。何也?承孝宗之澤也。今日病雖稍蘇,而元氣已竭,調劑無方,將至不起。何也?承武宗之亂也。伏願深察亂機,收還政柄,取文輩置之重典。然後務學親賢,去讒遠色,延訪忠言,深恤民隱。務使宮府一體,上下一心,而後天下可為也。」同官趙漢等亦皆以文為言,帝卒不聽。未幾,以給事中李學曾、章僑、主事林應驄皆言事奪俸,復上疏諫。指帝氣驕志怠,甘蹈過諐。詞甚切。

為給事三年,所言皆不聽,遂謝病歸。八年,詔養病三年以上不赴都者,悉落職閒住,相卿遂廢。夏言故與同僚相善,既秉政,招之,謝弗應。

顧濟,字舟卿,崑山人。正德十二年進士。授行人,擢刑科給事中。武宗自南都還,臥病豹房,惟江彬等侍。濟言:「陛下孤寄於外,兩宮隔絕,骨肉日疏。所恃以為安者,果何人哉?漢高帝臥病數日,樊噲排闥,警以趙高之事。今群臣中豈無噲憂者!願陛下慎擇廷臣更番入直,起居動息咸使與聞。一切淫巧戲劇,傷生敗德之事,悉行屏絕,則保養有道,聖躬自安。」不報。再踰月而帝崩。

世宗即位之月,濟上疏曰:「陛下踐阼,除弊納諫,臣民踴躍,思見德化之成。然立法非難,守法為難;聽諫非難,樂諫為難。今新政所釐,多不便於奸豪權悻。臣恐盤據既深,玩縱未已,非依怙宮闈,必請託左右。持法不固,則此輩將叢聚而壞之。此守法之難也。唐太宗貞觀初,每導群臣使言。及至晚年,諫者乃多忤旨。陛下首闢言路,臣工靡不因事納忠。高遠者似涉於迂闊,切直者或過於犯顏。若怒其犯顏,其言必不入;視為迂闊,則計必不行。此樂諫之難也。」尋復言:「內臣張雄、張銳等,詿誤先帝,業已逮治,又獲寬假。願斷以大義。俾無所售奸。」帝頗嘉納。既又劾司禮蕭敬黨庇銳等,而三法司會訊依違,無大臣節。不聽。帝欲加興獻帝皇號,濟言不可。尋請侍養歸,越數年卒。

子章志,嘉靖三十二年進士。累官南京兵部侍郎。奏減進奉馬快船額,南都人祀之。

章僑,字處仁,蘭谿人。正德十二年進士。授行人。嘉靖元年擢禮科給事中。疏劾中官蕭敬、芮景賢等。又言:「三代以下正學莫如朱熹。近有聰明才智,倡異學以號召,天下好高務名者靡然宗之。取陸九淵之簡便,詆朱熹為支離。乞行天下,痛為禁革。」御史梁世驃亦言之。帝為下詔申禁。

尋又請依祖宗故事,早朝班退,許百官以次啟事。經筵日講,賜清問,密勿大臣勤召對。又簡儒臣十數人,更番直便殿,以備咨訪。上納其言,而不能用。奸人何淵請立世室於太廟東北,僑力言其不可。未幾,又言:「添設織造內臣,貪橫殊甚。行戶至廢產鬻子以償。惟急停革,與天下更始。」疏入,不省。又因條列營務,劾定國公徐光祚、陽武侯薛倫不職,倫遂解任。尋請斥張璁、霍韜等,不聽。

孝陵司香谷大用乞還京治疾。僑言:「大用初連逆瑾,後引寧、彬,樹『八黨』之凶,釀十六年之禍,至先帝不得正其終。若不早遏絕,恐乘間伺隙,群兇競起,不至復亂天下不止。」章下所司。吳廷舉請召家居大臣議禮,僑劾其陰附邪說。孟秋時享太廟,帝遣京山侯崔元。僑言:「奉命臨時,倉皇就位,誠敬何存?」帝怒,奪其俸二月。歷禮科左給事中。出知衡州府,終福建布政使。

餘珊,字德輝,桐城人。正德三年進士。授行人,擢御史。庶吉士許成名等罷教習,留翰林者十七人。珊以為濫,疏論之。語侵內閣,不納。乾清宮災,疏陳弊政,極指義子、西僧之謬。巡鹽長蘆,發中官奸利事。為所誣,械繫詔獄,謫安陸判官。移知澧州。

世宗立,擢江西僉事,討平梅花峒賊。遷四川副使,備兵威、茂。嘉靖四年二月應詔陳十漸,其略曰:

陛下有堯、舜、湯、武之資,而無稷、契、伊、周之佐,致時事漸不克終者有十。

正德間,逆瑾專權,假子亂政,不知紀綱為何物,幸陛下起而振之。未幾而事樂因循,政多茍簡,名實乖謬,宮府異同,紛拏泄沓。以為在朝廷而不在朝廷,以為在宮省而不在宮省,遂至天子以其心為心,百官萬民亦各以其心為心。此紀綱之頹,其漸一也。

正德間,士大夫寡廉鮮恥,趨附權門,幸陛下起而作之。乃今則前日之去者復來,來者不去。自夫浮沉一世之人擢掌銓衡,首取軟美脂韋。重富貴薄名檢者,列之有位,致諛佞成風,廉恥道薄。甚者侯伯專糾彈,罷吏議禮樂。市門復開,賈販仍舊。此風俗之壞,其漸二也。

正德間,國柄下移,王靈不振,是以有安化、南昌之變,賴陛下起而整肅之。乃塞上戍卒近益驕恣。曩殺許巡撫而姑息,頃遂殺張巡撫而效尤。曩縛賈參將以立威,近又縛桂總兵而報怨。致榆關妖賊效之而戕主事,北邊庫吏仿之而賊縣官。陛下惑鄙儒姑息之談,牽俗吏權宜之計,遂使廟堂號令出於二三戍卒之口。此國勢之衰,其漸三也。

自逆瑾以來,以苞苴易將帥,故邊防盡壞,賴陛下起而申嚴之。然積弊已久,未能驟復。今朵顏蹢躅於遼海,羌戎跳梁於西川,北狄蹂躪於沙漠。寇勢方張,而食肉之徒不能早見預料,亟求制馭之方,乃假鎮靜之虛名,掩無能之實跡。甚且詐飾捷功,濫邀賞賚,虛張勞伐,峻取官階,而塞上多事日甚。此外裔之強,其漸四也。

自逆瑾以來,盡天下之脂膏,輸入權貴之室,是以有劉、趙、藍、鄢之亂,賴陛下起而保護之。乃近年以來,黃紙蠲放,白紙催征;額外之斂,下及雞豚;織造之需,自為商賈。江、淮母子相食,袞、豫盜賊橫行,川、陜、湖、貴疲於供餉。田野嗷嗷,無樂生之心。此邦本之搖,其漸五也。

正德朝,衣冠蒙禍,家國幾空,幸陛下起而收錄之。乃未幾而狂瞽之言,一鳴輒斥。昔猶謫遷外任,今或編配遐荒。昔猶禁錮終身,今至箠死殿陛。蓋自呂柟、鄒守益等去而殿閣空,顧清、汪俊等去而部寺空,張原、胡瓊等死而言路空。間有一二忠直士,又為權奸排擠而違之,俾不通,致陛下耳囂目眩,忽不自知其在鮑魚之肆矣。此人才之凋,其漸六也。

正德朝,奸邪迭進,忠諫不聞,幸陛下起而開通之。顧閱時未久,而此風復見。降心未懲其憤,逆耳或動諸顏。不剿說而折人以言,即臆度而虞人以詐。朝進一封,暮投千里。甚至三木囊頭,九泉含泣。此言路之塞,其漸七也。

正德朝,忠賢排斥,天下幾危,賴陛下起而主持之。豈期一轉瞬間,憸邪投隙而起。飾六藝以文奸言,假《周官》而奪漢政。堅白異同,模棱兩可。是蓋大奸似忠,大詐似信。王莽匿情於下士之日,安石垢面於入相之初。雖有聖哲,誰其辨之?臣恐正不敵邪,群陰日盛。此邪正之淆,其漸八也。

正德之世,大臣日疏,小人日親,致政事乖亂,賴陛下紹統,堂廉復親。乃自大禮議起,凡偶失聖意者,譴謫之,鞭笞之,流竄之,必一網盡焉而後已。由是小人窺伺,巧發奇中,以投主好,以弋功名。陛下既用先入為主,順之無不合,逆之無不怒。由是大臣顧望,小臣畏懼,上下乖戾,浸成睽孤,而泰交之風息矣。此君臣之睽,其漸九也。

正德之世,天鳴地震,物怪人妖,曾無虛歲,賴陛下紹統,災異始除。乃頃歲以來,雨雹殺禽獸,雷風拔樹屋,婦人產子兩頭,無極晝晦如夜,四方早潦,奏報不絕,曾何異正德之季乎?且京師陰霾之氣,上薄太陽,白晝冥冥,罕有暉采,尤為可畏。此災異之臻,其漸十也。

此十者,天子有一,無以保四海。陛下聖明,何以致此?無乃輔弼召之歟?竊見今日之為輔弼第一人者,徒以奸佞,伴食怙恩。致上激天變,下召民災,中失物望。臣逆知其非天下之第一流,而陛下乃任信之,不至於魚爛不已。願亟去其人,更求才兼文武如前大學士楊一清,老成厚重如今大學士石珤者,並置左右,庶弊政可除,天下可治。

臣又聞獻皇帝好賢下士,容物恕人,天下所共知也。今議禮諸臣,一言未合,輒以悖逆加之。謫配死徙,朝寧為空。此豈獻皇帝意?茍非其意,雖尊以天下,無當也。陛下何不起而用之,使駿奔清廟,以慰獻皇帝在天之靈哉!

疏反覆萬四千言,最為剴切,帝付之所司。其所斥輔弼第一人,謂費宏也。

珊律己清嚴,居官有威惠。外艱歸,士民祠之名宦。後副使胡東皋謁祠,獨顧珊嘆曰:「此吾師也。」服闋,以故官蒞廣東。終四川按察使。

先是,有御史汪珊者,於嘉靖元年七月疏陳十漸。略言:「陛下初即位,天下忻然望治,邇來漸不如初。初每事獨斷,今戚里左右,或潛移陰奪。初每事咨訪大臣,今禮貌雖隆,而實意日疏。初罷諸不經淫祠,今稍稍議復。初屏絕玩好,今教坊諸司或以新聲巧伎進。初日覽奏章,今或置不省,輒令左右可否。初釐革冗食冗費,今騰驤勇士不行核實,御馬實數不得稽察。初裁革錦衣冒濫,今大臣近侍以迎立授世陰,舊邸旗校盡補親軍。初中官有罪,懲以成法,今犯者多貸死,舉朝爭不得。初中官有過不復任用,今鎮守守備營求易置,悻門復啟。初納諫如流,今政事不便者,言官論奏,直曰『有旨』,訑訑拒人。」帝頗納其說。未幾,出為河南副使,歷官至南京戶部右侍郎。珊,字德聲,貴池人。正德六年進士。巡撫貴州時,討都勻叛苗有功。

韋商臣,字希尹,長興人。嘉靖二年進士。授大理評事。明年冬,商臣以「大禮」初定,廷臣下吏貶謫者無虛日,乃上疏曰:「臣所居官,以平獄為職。乃自授任以來,竊見群臣以議禮忤旨者,左遷則吏部侍郎何孟春一人,謫戍則學士豐熙等八人,杖斃則編修王思等十七人,以咈中使逮問,則副使劉秉鑑,布政馬卿,知府羅玉、查仲道等十人,以失儀就繫,則御史葉奇、主事蔡乾等五人,以京朝官為所屬訐奏下獄,則少卿樂頀、御史任洛等四人。此皆不平之甚,上干天象,下駭眾心。臣竊以為皆所當宥。況比者水旱疫癘,星隕地震,山崩泉湧,風雹蝗蝻之害,殆遍天下,有識莫不寒心。及今平反庶獄,復戍者之官,錄死者之後,釋逮繫者之囚,正告訐者之罪,亦弭災禳患之一道也。」帝責以沽名賣直,謫清江丞,量移德安推官。

遷河南僉事。討平永寧巨寇,以功受賞。伊王虐殺其妃,商臣論如律。嘗治里居給事中杜桐殺人罪。桐構之吏部尚書汪鋐。甫遷四川參議,遂以考察落職歸。言官薛宗鎧、戚賢、戴銑輩交章救,不納。家居數十年,卒。

黎貫,字一卿,從化人。正德十二年進士。改庶吉士,授御史。刷卷福建,劾鎮守內官尚春侵官帑狀,悉追還之。世宗入繼,貫請復起居注之制,命詞臣編類章奏備纂述,從之。登極詔書禁四方貢獻,後鎮守中貴貢如故。貫上言:「陛下明詔甫頒,而諸內臣曲說營私,希恩固寵。其假朝命以征取者謂之額,而自挾以獻者謂之額外,罔虐百姓,致朝廷之澤壅而不流,非所以昭大信,彰君德也。」

嘉靖二年,帝從玉田伯蔣輪請,於承天立興獻帝家廟,以輪子榮奉祀。貫言:「陛下信一諛臣之說,委祀事於外戚。神不歆非類,獻帝必將吐之。」不聽。尋疏言:「國初,夏秋二稅,麥四百七十一萬,而今損九萬。米二千四百七十三萬,而今損二百五十萬。以歲入則日減,以歲出則日增。乞敕所司通稽祖宗以來賦額及今日經費之數,列籍上聞。知賦入有限,則費用不容不節。」帝嘉納焉。

出按江西,父喪歸。久之,起故官。會帝從張孚敬議,去孔子王號,改稱先師,並損籩豆佾舞之數。編修徐階以諫謫。御製《改正祀典說》,頒示廷臣;而孚敬復為《祀典或問》,以希合帝意。議已定,貫率同官合疏爭之。帝震怒,曰:「貫等謂朕已尊皇考為皇帝,孔子豈反不可稱王?奸逆甚矣。其悉下法司按治。」於是都御史汪鋐言:「比者言官論事,每挾眾以凌人曰:『此天下公議也』,不知倡之者止一人。請究倡議之人,明正其罪。」帝然之。已而刑部尚書許贊等上其獄,當贖杖還職,帝特命褫貫為民。久之,卒於家。

方貫等上疏時,禮科都給事中華陽王汝梅亦率同官抗論,且曰:「陛下萬幾之餘,留神典禮,甚盛舉也。但恐生事之臣望風紛起,今日獻一議,謂某制當革,明日進一說,謂某制當復,國家自此多事矣。況祖宗成法,守之百六十年,縱使少不如古,循而行之,亦未為過,何必紛紛事更易乎?」帝覽奏,斥其違旨,以《祀典說》示之。

汝梅,字濟元,由行人歷禮科都給事中。八年二月以災異求言。汝梅言:「比來章奏多逢迎,請分別忠佞,毋信諛言。大臣奏事,近多留中,請悉付之公論。人主之學,詞命非所重。今一事之行,動煩宸翰,亦少褻矣。宜仿祖宗故事,時御平臺,召見宰執,面決大議,既省筆札之勞,且絕壅蔽之害。」疏入,忤旨。及夏言請分祀天地,汝梅復偕同官力爭。尋出為浙江參政,卒官。

彭汝實,字子充,嘉定州人。正德十六年進士。授南京吏科給事中。嘉靖三年疏言:「九江盜起,殺傷官軍。操江伍文定不即議剿,應城伯孫鉞擁兵不出,俱宜切責。」帝並從之。呂柟、鄒守益下獄,汝實抗章救。又因災異上言:「邇者黃風黑霧,春早冬雷,地震泉竭,揚沙雨土。加以群小盛長,盜賊公行,萬民失業。木異草妖,時時見告。天變於上,地變於下,人物變於中,而修省之詔無過具文。廷陛之間,忠邪未辨,以逢迎為合禮,以守正為沽直。長鯨巨鮞決綱自如,腴田甲第橫賜無已。陛下春秋已逾志學,而經筵進講略無問難,黃閣票擬依常批答。棄燕閒於女寵,委腹心於貂璫。二廖諸張尚然緩死,李隆、蘇晉竟得無他。如此而望天意回,人心感,不可得矣。」

大學士費宏以子坐事被論不出,禮部侍郎溫仁和以慶王台浤事聽勘。汝實言宜聽二臣避位,以明進退之義。因薦石珤、羅欽順、顧清、蔣冕可代宏,李廷相、崔銑、湛若水、何瑭、許誥可代仁和。章下所司。

奸人王邦奇之訐楊廷和、彭澤也,汝實言:「邦奇先後兩疏,始為惶駭之語,終雜鄙褻之辭。中所引事,多顛倒淆惑,至謂費宏、石珤夜入楊一清門。今不聞召問一清,一清又久不為白,何也?陛下即位之初,廷和裁省冗員數萬,坐此叢怒罷去。今其長子業以狂愚發遣,亦可已矣。而群小蓄忿,蔓連不已,并其次子及婿又復下獄。夫誣告之律,視其所誣輕重反坐,此國法也。願追究主使之人,與告人同罪,毋令茍免,貽譏外蕃。」不聽。

汝實數言時政缺失,又嘗力爭「大禮」,為璁、萼等所惡。以親老再疏請改近地教職,而舉貢士高任說、王表自代。章下,吏部承璁、萼指,言:「汝實倡言鼓眾,撓亂大禮,且與御史方鳳、程啟充朋黨通賄。自知考察不容,乃欲辭尊居卑,不當聽其倖免。」遂奪職閒住。與啟充及徐文華、安磐皆同里,時稱「嘉定四諫」。

鄭自璧,字采東,祥符人,隸籍京師。正德十二年進士。改庶吉士,除工科給事中。

世宗踐阼,中外競言時政。自璧請采有關化理者,類輯成書,以備觀覽,從之。初,正德中,奄人多奪民業為莊田,至是因民訴,遣使往勘。自璧復備言其弊,帝命勘者嚴治,民患稍除。嘉靖二年,后父陳萬言辭黃華坊賜第,請西安門外新宅,詔予之。自璧以所請宅已鬻之民,不當奪,與安磐力爭。不聽。明年爭「大禮」受杖。

三遷至兵科都給事中。中官李能以修墩堡為詞,請定山海關稅額。中官張忠、尚書金獻民等論甘肅功,廕子錦衣,其下參隨皆進秩。鎮守江西中官黎鑑,參隨踰常額。中官武忠從子英冒功,擢副千戶。錦衣官裁革者多夤緣復職,而司禮監奏收已汰諸匠近五百人。孝陵凈軍于喜擅赴京奏辨。安邊伯許泰戍死,其子請襲祖職。中官扶安黃英先後死,官其親屬。自璧皆抗疏爭,帝多不聽。嘗偕同官劾郭勛奸貪。及李福達事起,復劾勛交結妖人。帝以勛故,降旨責自璧。六年三月,宣府失事。復劾總兵傅鐸,并及鎮守中官王玳、巡撫周金、副將時陳等罪。鐸逮問,陳褫冠帶,而玳、金責立功贖罪。禮部侍郎桂萼請起王瓊於邊。自璧率同官與御史譚纘等言瓊罪宜追治,萼引奸邪,請并論。不納。

自璧最敢言,所言皆權倖,直聲震朝野。側目者共為蜚語,聞於上。吏部以資推太僕少卿,不用。至是科道共劾,中旨降二級,調外任,遂謫江陰縣丞。命下,大臣幸其去,無救者。後廷臣屢論薦,竟不召。

戚賢,字秀夫,全椒人。嘉靖五年進士。授歸安知縣。縣有蕭總管廟,報賽無虛日。會久早,賢禱不驗,沉本偶於河。居數日,舟過其地,木偶躍入舟,舟中人皆驚。賢徐笑曰:「是特未焚耳。」趣焚之。潛令健隸入岸傍社,誡之曰:「水中人出,械以來。」已,果獲數人。蓋奸民募善泅者為之也。

知府萬雲鵬操下急,賢數忤之。當上計,有毀雲鵬者,將被黜。賢走吏部白其枉,雲鵬竟得免。而尚書桂萼獨心異賢,喪去,起知唐縣。召為吏科給事中。

十四年春,當大計外吏。大計罷者,例永不用,而是時言事諸臣忤柄臣意,率假計典錮之。賢乃先事言所黜有未當者,宜聽言官論救。帝稱善,從其請。會參議王存、韋商臣言事忤要人,前給事中葉洪劾汪鋐被謫,果在黜中。賢方勘事陜西,給事中薛宗鎧因據賢疏伸救。吏部持不可,帝遂命已之。及賢還朝,以鋐恣橫,實張孚敬庇之,乃條其罪狀曰:「輔臣孚敬布腹心以操吏部之權,懸利害以箝言官之口。即如考察一事,陛下曲聽臣言,許其申雪,正防大臣行私也。今言官為洪等辯救,孚敬乃曲庇冢臣,巧言阻遏。陛下有堯、舜知人之明,輔臣負伯鯀方命之罪。放流之典具在,惟陛下以威斷之。」帝內嘉賢言,而重違孚敬、鋐意,洪等竟不復。

再以喪去。補刑科都給事中。夏言柄國,會當選庶吉士,不能無所徇。賢疏陳請屬之弊,帝納其言。久之,劾郭勛吞噬遍天下。太廟災,復劾勛及尚書張瓚、樊繼祖等,而薦聞淵、熊浹、劉天和、王畿、程文德、徐樾、萬鏜、呂柟、魏校、程啟充、馬明衡、魏良弼、葉洪、王臣可任用。言滋不悅,激帝怒,謫山東布政司都事。諸被薦者皆奪俸。

賢尋以父老自免。歸十餘年,卒。賢少聞王守仁說,心契之。及官於浙,遂執弟子禮。

劉繪,字子素,一字少質,光州人。祖進,太僕少卿。繪長身修髯,磊落負奇氣。好擊劍,力挽六石弓。舉鄉試第一,登嘉靖十四年進士,授行人,改戶科給事中。

二十年,詔兩京言官會薦邊才。給事中邢如默等薦毛伯溫、劉天和等二十人,而故御史段汝礪、副都御史翟瓚、參議王洙與焉。繪言:「汝礪乃大學士翟鑾姻戚,瓚、洙則夏言諭指如默排群議而薦之者。相臣挾權以遏言官,言官懼勢而弗公議,上下雷同,非社稷福。乞罷鑾、言,罪如默,為徇私植黨者戒。」帝是其言,出如默於外。言適罷政,鑾置不問。

明年,寇大入山西。繪上疏曰:「俺答方彊,必為腹心患。議者謂宜守不宜戰,以故邊將多自全,或拾殘騎報首功。督巡諸臣亦第列士馬守要害,名曰清野,實則避鋒;名曰守險,實則自衛。請專任翟鵬,得便宜從事。馳發宣、大、山西士馬,合十七八萬人。三路並舉,有進無退,寇雖多,可計日平也。」帝壯其言。令假鵬便宜,得戮都指揮以下。然鵬竟不能出塞。頃之,劾山西巡撫劉臬結納夏言,且請罷吏部尚書許瓚、宣府巡撫楚書。臬、書由是去職。

繪兩劾言,言憾之,出為重慶知府。土官爭地相讎,檄諭之,即定。上官交薦,而言再入政府,屬言者論罷之。家居二十年,卒。

子黃裳,兵部員外郎。倭陷朝鮮,命贊畫侍郎宋應昌軍務。渡鴨綠江,抵平壤,大敗賊兵。賊遁,黃裳追逐,又連破之。錄功,進郎中。

錢薇,字懋垣,海鹽人。嘉靖十一年進士。受業湛若水。官行人,泊然自守。與同年生蔣信輩朝夕問學。擢禮科給事中。請令將帥家丁得自耕塞下田,毋徵其賦,總督大臣假便宜,專制閫外。格不行。又疏劾大學士李時、禮部尚書夏言、工部尚書溫仁和、外戚蔣輪。

進右給事中。郭勛請復鎮守內官,擅易置宿衛將校。薇憤,疏其不法七事。帝眷勛,然素知其橫,兩不問。已,因星變,極言主德之失,帝深銜之未發。疏諫南巡,坐奪俸。內閣夏言輩所選宮僚,多以徇私劾罷。薇偕同官呂應祥、任萬里乞如會推故事,集內閣九卿公舉。帝特命並斥為民。累薦,皆報寢。

集鄉里晚進與講學,足跡不及公府。倭患起,請於巡撫王忬,集兵為備。鄉人德之。卒年五十三。隆慶初贈太常少卿。

洪垣,字峻之,婺源人。嘉靖十一年進士。禮部侍郎湛若水講學京師,垣受業其門。授永康知縣,徵授御史。十八年,世宗南巡,冊立皇太子,命閣臣夏言、顧鼎臣選宮僚。垣再疏言溫仁和、張衍慶、薛僑、胡守中、屠應颭、華察、胡經、史際、白悅、皇甫涍等皆庸流,不可使輔導青宮。帝亦已從他諫官言,廢黜者數人。未幾,劾文選郎中黃禎先「賄選郎楊育秀,得為考功。及居文選,貪婪欺罔。知州王顯祖等考察調簡,而補大州。知縣何瑚年過六十,而選御史。皆非制。今當大計京官,乃以猥瑣之曹世盛為考功郎,誤國甚」。帝下其章都察院,令會吏科參核。乃下禎詔獄,及育秀、顯祖等,咸斥為民。因詰責吏部尚書許贊、都御史王廷相,而令十三道御史公舉隱年冒進若瑚者。御史王之臣等坐調者四人,世盛亦改他部。垣一疏,而御史、曹郎以下得罪者至二十餘人。

出按廣東。以安南款附,增俸一級。未竣,出為溫州知府。歲饑,有閉糴者,饑民殺之,垣坐落職歸。復與同里方瓘往從若水,若水為建二妙樓居之。家食四十餘年,年九十。

瓘絕意仕進。嘗自廣東還,同行友瘴死。舟中例不載屍,瓘秘不以告,與同寢纍日,至韶州始發之。

垣同年呂懷,廣信永豐人,亦若水高弟子。由庶吉士授兵科給事中,改春坊左司直郎,歷右中允,掌南京翰林院事。每言王氏之良知與湛氏體認天理同旨,其要在變化氣質。作《心統圖說》以明之。終南京太僕少卿。

周思兼,字督夜,華亭人。少有文名。嘉靖二十六年進士。除平度知州。躬巡郊野,坐藍輿中,攜飯一盂,令鄉民以次舁行。因盡得閭閻疾苦狀,悉蠲除之。王府奄人縱莊奴奪民產,監司杖奴斃,奄迫王奏聞,巡撫彭黯令思兼讞之。王置酒欲有所囑,竟席不敢言。思兼閱獄詞曰:「此決杖不如法。罪當杖,以王故,加一等。奄誣告,罪當戍,以王故,末減。」監司竟得復故秩。旁郡饑民掠食,所司持之急,且為亂,上官檄思兼治之。作小木牌數千散四郊,令執牌就撫,悉振以錢穀,事遂定。入覲,舉治行第一,當遷。州人走闕下以請,乃復留一年。

擢工部員外郎,督臨清磚廠,士民遮道泣送。同年生貌類思兼,使經平度,民競走謁。見非是,各歎息去。河將決,思兼募民築隄,身立赤日中。隄成三日而秋漲大發,民免於災。進郎中,出為湖廣僉事。岷府宗室五人封爵皆將軍,殺人掠貲財,監司避不入武岡者二十年。思兼廉得奸狀,縛其黨,悉繫之獄。五人藏利刃入,思兼與揖,而捫其臂曰:「吾為將軍百口計,將軍乃為此曹死耶?」皆沮退。乃列其罪奏聞,悉錮之高牆,還田宅子女於民。遭內艱去官,不復出。居久之,起廣西提學副使,未聞命而卒。

顏鯨,字應雷,慈谿人。嘉靖三十五年進士。授行人。擢御史,出視倉場。奸人馬漢怙定國公勢,貸子錢漕卒。償不時,則沒入其糧,為怨家所訴。漢持定國書至,鯨立論殺。四十一年,畿輔、山東西、河南北大諗。鯨請州縣贓罰銀毋輸京師,盡易粟備振,且發之。內府新錢為糴本。帝悉報可。已,上漕政便宜六事。

明年出按河南。伊王典楧怙惡,久結掖廷中官、嚴嵩父子,內外應援,所請奏立下,爪牙率礦盜。鯨欲除之,與參政耿隨卿計,持王承奉王金盬罪,金盬日告王所謀。時嵩已敗,鯨乃奏記徐階,說諸大璫絕其援,又盡捕王偵事飛騎。託言防寇,檄知府兵分屯要害地。乃會巡撫胡堯臣劾典楧抗旨、矯敕、僭擬、淫虐十大罪。王護衛及諸亡命幾萬人,不敢發。帝震怒,廢王為庶人,錮之高牆,沒其貲,削世封。兩河人鼓舞相慶。景王之國,越界奪民產為莊田,鯨執治其爪牙。魏國公侵民產,假欽賜名樹碑為界。鯨仆其碑,戍其人。錦衣帥受諸俠少金,署名校尉籍中,為民害。列侯使王府,道路驛騷。王府內官進奉,駕龍舟,所過恣橫。鯨請校尉缺從兵部補,冊封改文臣,王府進奉遣屬吏。詔冊親王及妃遣列侯,餘皆如鯨議。

改督畿輔學政。大興知縣高世儒奉詔核逃役,都督朱希孝以勾軍劾之,下部議。鯨劾希孝亂法,言:「世儒等按籍召行戶,非勾禁軍。此乃禁軍子弟家人倚城社,冒禁衛名,致吏不敢問。富人得抗詔,而貧者為溝中瘠。世儒無罪,罪在錦衣。」帝怒,責鯨詆誣勛臣,貶安仁典史。隆慶元年歷,湖廣提學副使。以試恩貢生失張居正指,降山東參議。改行太僕少卿。都御史海瑞薦鯨異才,不報。

鯨按河南時,黜新鄭知縣,其人高拱所庇也。在湖廣,王篆欲祀其父鄉賢,執不許。至是,拱掌吏部,篆為考功,遂以不謹落鯨職。萬曆中,給事中鄒元標、御史饒位交章薦之,報寢。御史顧雲程言:「陛下大起遺佚,獨鯨及管志道以考察格之。夫相與冢宰賢,則黜幽為公典,否則驅除異己而已。近又登用被察吳中行、艾穆、魏時亮、趙世卿,獨靳鯨、志道何也?」給事中姜應麟、李弘道亦言之,僅以湖廣副使致仕。中外論薦十餘疏,不果用。

贊曰:傳稱:「未信而諫,則以為謗己」。然志節之士,心卷心卷忠愛,何忍以不信自外其君哉?張芹等懷抱悃忱,激昂論事。其言雖不盡用,要與緘默者異矣。


\end{pinyinscope}