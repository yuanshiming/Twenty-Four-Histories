\article{列傳第一百九十 列女二}

\begin{pinyinscope}
○歐陽氏徐氏馮氏方氏葉氏潘氏楊氏張烈婦蔡氏鄭氏王烈婦許烈婦吳氏沈氏六節婦黃氏張氏張氏葉氏範氏劉氏二女孫烈女蔡烈女陳諫妻李氏胡氏戴氏胡氏許元忱妻胡氏郃陽李氏吳節婦楊氏徐亞長蔣烈婦楊玉英張蟬云倪氏彭氏劉氏劉氏二孝女黃氏邵氏婢楊貞婦倪氏楊氏丁氏尤氏李氏孫氏方孝女解孝女李氏項貞女壽昌李氏玉亭縣君馬氏王氏劉氏楊氏譚氏張氏李烈婦黃烈婦須烈婦陳節婦馬氏謝烈婦張氏王氏戚家婦金氏楊氏王氏李孝婦洪氏倪氏劉氏

歐陽氏,九江人,彭澤王佳傅妻也。事姑至孝。夫亡,氏年方十八,撫遺腹子,紡績為生。父母迫之嫁,乃鍼刺其額,為誓死守節字,墨涅之,深入膚裏,里人稱為黑頭節婦」又徐氏,烏程人。年十六,嫁潘順。未期而夫病篤,顧徐曰:「母老,汝年少,奈何?」徐泣下,即引刀斷左小指,以死誓。夫死,布衣長齋。年七十八卒。遺命取斷指入棺中。家人出其指,所染爪紅色尚存。

馮氏,宣城劉慶妻。年十九,夫亡,誓守節。其娣姒諷之曰:「守未易言,非咬斷鐵釘者不能。」馮即投袂起,拔壁上釘齧之,DC然有齒痕。復抉臂肉,釘著壁上曰:「脫有異志,此即狗彘肉不若。」已而遺腹生子,曰大賢。長娶李氏,大賢又夭,姑婦相守至老。卒,取視壁釘肉,尚韌不腐,齒痕如新。

方氏,金華軍士袁堅妻。堅嗜酒敗家,卒殯城北濠上。方貧無所依,乃即殯處置棺,寢處其中,饑則出飲於濠。久之不復出,則死矣。郡守劉郤為封土祭之。

又葉氏,蘭谿人。適神武中衛舍人許伸。伸家素饒於財,以不檢,蕩且盡,攜妻投所親,卒於通州。氏守屍,晝夜跪哭。或遺之食,或饋金,或勸以改嫁,俱卻不應。水漿不入口者十四日,竟死尸傍,年二十餘,州人為買棺合葬。

潘氏,海寧人。年十六,歸許釗,生子淮。甫期年,釗卒,既殮,潘自經。死已兩日矣,有老嫗過之曰:「是可活也。」投之藥,更蘇。釗族兄欲不利於孤,嗾潘改適,潘毀容自矢。族兄者,夜率勢家僕數十人誣以債,椎門入。潘負子,冒風雨,踰垣逸。前距大河,追者迫,潘號慟投於河。適有木浮至,憑以渡,達母家,遂止不歸。淮年十九,始歸。淮補諸生,娶婦生五子。潘年五十,宗人聚而祝,族兄者亦至。潘曰:「氏所以得有今日,賴伯氏玉成。」目淮酌酒飲伯,卒爵,北向拜曰:「未亡人,三十年來瀕死者數矣,而顧強生,獨以淮故耳。今幸成立,且多子,復何憾。」語畢入室。頃之宴徹,諸宗人同淮入謝,則縊死室中矣。

楊氏,桐城吳仲淇妻。仲淇卒,家貧,舅欲更嫁之。楊曰:「即饑死,必與舅姑俱。」舅不能奪。數年,家益貧,舅謀於其父母,將以償債。楊仰天呼曰:「以吾口累舅姑,不孝。無所助於貧,不仁。失節則不義。吾有死而已。」因咽髮而死。張烈婦,蕪湖諸生繆釜妻。年十八,歸釜。越四年,釜病,屬張善自託。張泣曰:「夫以吾有二心乎?有子則守志奉主,妻道也。無子則潔身殉夫,婦節也。」乃沐浴更衣,闔戶自縊。閱日,而釜乃卒。又蔡烈婦,松陽葉三妻。三負薪為業,蔡小心敬事。三久病,織糸任供藥餌。病篤,執婦手訣曰:「及我生而嫁,無受三年苦。」婦梳洗更衣,袖刀前曰:「我先嫁矣。」刎頸死。三驚歎,尋死。又鄭氏,安陸趙鈓妻。性剛烈,閨房中言動不涉非禮。某寡婦更適人,饋以茶餅。鄭怒,命傾之。夫戲曰:「若勿罵,幸夫不死耳。」鄭正色曰:「君勿憂,我豈為此者。」後鈓疾將死,回視鄭,瞪目不瞑。鄭曰:「君得毋疑我乎?」即自縊於床楣。鈓少蘇,回盼,出淚而絕。

王烈婦,上元人。夫嗜酒廢業,僦居破屋一間,以竹篷隔內外。婦日塞戶,坐門扉績麻自給。夫與博徒李游。李悅婦姿,謀亂之。夫被酒,以狂言餂婦,婦奔母家避之。夫逼之歸,夜持酒脯與李俱至,引婦坐,婦駭走且罵。夫以威挾之,婦堅拒,大被搒笞。婦度不免,夜攜幼女坐河干,慟哭投河死。是夜,大風雨,屍不漂沒。及曙,女尚熟睡草間。

又許烈婦,松江人許初女。夫飲博不治生。諸博徒聚謀曰:「若婦少艾,曷不共我輩懽,日可得錢治酒。」夫即以意喻婦,婦叱之,屢加箠撻不從。一日,諸惡少以酒肴進。婦走避鄰嫗家,泣顧懷中女曰:「而父不才,吾安能靦顏自存,俟汝之成民也。」少間,聞闔戶聲。嫗覘之,則拔刀刎頸仆地矣。父挈醫來視,取熱雞皮封之,復抓去。明旦氣絕,年二十五。

吳氏,永豐人,名姞姑。年十八,適寧集略。未一年,夫卒,六日不食。所親百方解譬,始食粥,朝暮一溢米。服除,母憐其少,欲令改適。往視之,同寢食三年,竟不敢出一語。歸謂諸婦曰:「此女鐵石心,不可動也。」

慈谿沈氏六節婦。章氏,祚妻。周氏,希魯妻。馮氏,信魁妻。柴氏,惟瑞妻。孟氏,弘量妻。孫氏,琳妻。所居名沈思橋,近海。族眾二千人,多驍黠善鬥。嘉靖中,倭賊入犯,屢殲其魁,奪還虜掠。賊深仇之。一日,賊大至,沈氏豪誓於眾曰:「無出婦女,無輦貨財,共以死守,違者誅。」章亦集族中婦女誓曰:「男子死鬥,婦人死義,無為賊辱。」眾竦息聽命。賊圍合,群婦聚一樓以待。既而賊入,章先出投於河,周與馮從之。紫方為夫礪刃,即以刃斫賊,旋自刃。孟與孫為賊所得,奪賊刃自刺死。時宗婦死者三十餘人,而此六人尤烈。

黃氏,沙縣王珣妻。嘉靖中,倭亂,流劫其鄉。鄉之比鄰,皆操舟為業。賊至,眾婦登舟,匿艙中,黃兀坐其外。眾婦呼之曰:「不虞賊見乎?」黃曰:「篷窗安坐,恐賊至不得脫,我居外,便投水耳。」賊至,黃躍入水中死。時同縣羅舉妻張氏,從夫避亂巖穴間。賊至,張與妾及妾子俱為所獲。賊見張美,欲犯之,不從。至中途,張解髮自縊,賊斷之。張又解行纏,賊又覺之,徒跣驅至營。賊魁欲留之,張厲聲曰:「速賜一死。」賊曰:「不畏死,吾殺汝妾。」張引頸曰:「請代妾,留撫孩嬰。」賊曰:「吾殺孩嬰。」張引頸曰:「請代孩嬰,存夫嗣。」賊令牽出殺之。張先行,了無懼色。賊方猶豫,張罵不絕口,遂遇害。投屍於河,數日屍浮如生。

張氏,政和游銓妻。倭寇將至,婦數語其女曰:「婦道惟節是尚,值變之窮,有溺與刃耳,汝謹識之。」銓聞,以為不祥。婦曰:「使婦與女能如此,祥孰大焉。」未幾,賊陷政和,張度不脫,連呼女曰:「省前誨乎?」女頷之,即赴井。張含笑隨之,並死。

又葉氏,松溪江華妻,陳氏,葉弟惠勝妻,偕里人避倭長潭。值歲除,里嫗覓刀為幼男薙髮弗得,葉出諸懷中。眾問故,曰:「以備急耳。」及倭圍長潭,執二婦,共繫一繩。葉謂陳曰:「我二人被縶,縱生還,亦被惡名,死為愈。」陳唯唯。葉探刀於懷,則已失,各抱幼女跳潭中死。同時林壽妻范氏,亦與眾婦匿山塢。倭搜得眾婦,偕至水南,范獨與抗。或謂姑順之,家且來贖。答曰:「身可贖,辱可贖哉!我則寧死。」賊聞言,殺其幼女恐之,不為勛。曰:「併及汝矣。」厲聲曰:「固我願也!」賊殺之。

劉氏二女,興化人。嘉靖四十一年與里中婦同為倭所掠,繫路傍神祠中。倭飲酣,遍視繫中,先取其姊。姊厲聲曰:「我名家女也,肯污賊乎?」倭笑慰之曰:「若從我,當詢父母歸汝。」女曰:「父母未可知,此時尚論歸耶?」倭尚撫背作款曲狀。女怒,大罵。時黃昏,倭方縱火,女即赴火死。已復侵其妹,妹又大罵。倭露刃脅之,不為動,曰:「欲殺,即殺。」倭欲強犯之,女紿曰:「吾固願從,俟姊骨燼乃可,否則不忍也。」倭喜負薪益火,火熾,女又赴火死。時同死者四十七人,二女為最。

孫烈女,五河人。性貞靜,不茍嬉笑。母朱卒,繼母李攜前夫子鄭州兒來。州兒恃母欲私女,嘗以手挑之,忿批其頰。一日,女方治面,州兒從後摟之。女揪髮覓刃,州兒齧其臂得脫。女奔訴於姊,觸地慟哭曰:「母不幸,父又他出,賊子敢辱我,必刃之而後死。」姊曲撫慰。乃以臂痕示李,使戒戢之。州兒不悛,紿李曰:「兒採薪,臂力不勝,置遺束於路。」李往取之,歸則戶扃甚嚴。從母舒氏亦趨至,曰:「初聞如小犢悲鳴,繼又響震如雷,必有異。」並力啟之,州兒死閾下,項幾斷,女亦倚壁死。蓋州兒誑母出,調女。女陽諾而使之閉門,既躡其後殺之也。又蔡烈女,上元人。少孤,與祖母居。一日,祖母出,有逐僕為僧者來乞食,挑之,不從。挾以刃,女徒手搏之,受傷十餘處,罵不絕,宛轉死灶下。賊遁去,官行驗,忽來首伏。官怪問故。賊曰:「女拘我至此。」遂抵罪。

陳諫妻李氏,番禺人。諫,嘉靖十一年進士。為太平推官,兩月卒,其弟扶櫬歸。李曰:「吾少嫠也,豈可與叔萬里同歸哉!」遂不食死。

胡氏,會稽人。字同里沈DD。將嫁,而DD遘父煉難,二兄袞、褒杖死塞上,DD與兄襄並逮繫宣府獄。總督楊順逢嚴嵩意,必欲置二子死,搒掠數百,令夜分具二子病狀。會順為給事中吳時來所劾,就檻車去,襄等乃得釋。自是病嘔血,扶父喪歸,比服闋始婚,胡年已二十七。踰六月,DD卒,胡哀哭不絕聲,盡出奩具治喪事。有他諷者,斷髮剺面絕之。終日一室中,即同產非時不見。晚染疾,家人將迎醫,告其父曰:「寡婦之手豈可令他人視。」不藥而卒,年五十一。以襄子嗣。

戴氏,莆田人,名清。歸蔡本澄,年甫十四。居二年,本澄以世籍戍遼東,買妾代婦行。戴父與約曰:「遼左天末,五年不歸,吾女當改嫁矣。」至期,父語清如約。泣不從,獨居十有五年。本澄歸,生一子,未晬,父子相繼亡。清哀毀幾絕。父潛受吳氏聘,清聞之曰:「人呼女蔡本澄婦耳,何又云吳耶?」即往父家,使絕婚。吳訟之官,令守節,表曰寡婦清之門。時莆又有歐茂仁妻胡氏,守節嚴苦,內外重之。郡有獄久不斷,人曰:「太守可問胡寡婦。」守乃過婦問之,一言而決。

胡氏,鄞許元忱妻。元忱為徐祝師養子,習巫祝事。胡鄙之,勸夫改業,且勸歸許宗。未果,而元忱疫死。氏殯之許氏廬,苫臥柩傍,夜擁一刀臥。里某求氏為偶,氏毀面截鬢髮,斷左手三指,流血淋漓,某驚遁。族婦尊行抱持之,大慟,因立應後者,令子之。氏服喪三年,不浣不櫛。畢葬,乃為子娶婦。夫有弟少流移於外,復為返之,許氏賴以復起。

李氏,郃陽安尚起妻。尚起商河南,病亡。氏聞訃,盡變產完夫債,且置棺以待夫櫬歸,跪告族黨曰:「煩舉二棺入地。」閉戶將自縊,鄰婦欲生之,排闥曰:「爾尚有所逋,何遽死?」氏啟門應曰:「然吾資已盡,奈何?請復待一日。」乃紉履一雙往畀之,曰:「得此足償矣。」歸家,遂縊死。

吳節婦,無為周凝貞妻。凝貞卒,婦年二十四,毀容誓死,不更適,傭女工以奉孀姑。姑老臥病,齒毀弗能食。婦絕其兒乳以乳姑,冬月臥擁姑背以煖之,宛轉床席者三年。姑卒,哀毀骨立,年七十五終。又楊氏,清苑劉壽昌妻。年十九,夫卒,誓死殉。念姑病無依,乃不死。母家來迎,以姑老不忍去側,竟不歸寧。閱三十年,姑卒,葬畢,哀號夫墓曰:「妾今得相從地下矣。」遂絕粒。家人問遺言。曰:「姑服在身,殮以布素。」遂瞑。

徐亞長,東莞徐添男女。添男為徐姓僕,生亞長四歲而死。母以亞長還其主,去而別適。比長,貞靜寡言笑,居群婢中,凜然有難犯之色。家童進旺欲私之,不可。亞長奉主命薙草豆田中,進旺跡而迫之,力拒獲免,因哭曰:「聞郎君讀書,有寡婦手為人所引,斧斷其手,況我尚女也,何以生為!」遂投江死。

蔣烈婦,丹陽姜士進妻。幼穎悟,喜讀書。弟文止方就外傅。夜歸,輒以餅餌啖之,令誦日所授書,悉能記憶,久之遂能文。歸士進數年,士進病瘵死。婦屑金和酒飲之,並飲鹽鹵。其父數偵知,奔救免。不食者十二日,父啟其齒飲之藥,復不死。禮部尚書寶,士進從父也,知婦嗜讀書,多置古圖史於其寢所,令續劉向《列女傳》。婦許諾,家人備之益謹。一日,婦命於糸惠帳前掘坎埋大缸貯水,笑謂家人:「吾將種白蓮於此,此花出泥淖無所染,令亡者知予心耳。」於是日纂輯不懈。書將成,防者稍不戒,則濡首缸中死矣。為文脫稿即毀,所存《列女傳》及《哭夫文》四篇、《夢夫賦》一篇,皆文止竊而得之者。御史聞於朝,榜其門曰文章貞節。初,其兄見女能文,以李易安、朱淑真比之,輒頻蹙曰:「易安更嫁,而淑真不慊其夫,雖能文,大節虧矣。」其幼時志操已如此。

楊玉英,建寧人。涉獵書史,善吟詠。年十八,許字官時中。時中有非意之獄,父母改受他聘。玉英聞之,囑其婢曰:「吾篋有佩囊、布奚諸物,異日以遺官官人。」婢弗悟,諸之。於是竊入寢室,自經死,目不瞑。時中聞訃,具禮往祭,以手掩之,遂瞑。婢出所遺物,付父母啟之,得詩云:「崑山一片玉,既售與卞和。和足苦被刖,玉堅不可磨。若再付他人,其如平生何!」又張蟬雲,蒲城人,許字俞檜。萬歷中,檜被誣繫獄。女聞可賄脫,謀諸母,欲貨妝奩助之。母不可,曰:「汝未嫁,何為若此。」女方食,即以碗擲地,恚不語。入暮自縊死。

陳襄妻倪氏。襄為鄞諸生,早卒。婦年三十,無子,家貧,力女紅養姑。有慕其姿者,遣媒白姑。婦煎沸湯自漬其面,左目爆出,又以煙煤塗傷處,遂成獰惡狀。媒過之,驚走,不敢復以聘告。歷二十年,姑壽七十餘卒,婦哀慟不食死。

彭氏,安丘人。幼字王枚皋。未嫁,枚皋卒,誓不再適。濰縣丁道平密囑其父欲娶之。彭察知,六日不食。道平悔而止,心敬女節烈,後聞其疾革不起,贈以棺。彭語父曰:「可束葦埋我,亟還丁氏棺,地下欲見王枚皋也。」遂死。又劉氏,潁州劉梅女,許聘李之本。之本歿,女泣血不食,語父曰;「兒為李郎服三年,需弟稍長,然後殉。寄語翁,且勿為郎置槨。」遂盡去鉛華,教弟讀書,親正句讀。越一年,梅潛許田家。女聞,中夜開篋,取李幣,挑燈製衣,衣之,縊死。知府謝詔臨其喪,鄰里弔者如市。田家亦具奠賻,舉酒方酹,柩前承灌瓦盆劃然而碎,起高丈餘,繞簷如蝶墜。觀者震色。

劉氏二孝女,汝陽人。父玉生七女,家貧力田。嘗至隴上,歎曰:「生女不生男,使我扶犁不輟。」其第四、第六女聞之惻然,誓不嫁,著短衣代父耕作。及父母相繼卒,無力營葬,二女即屋為丘,不離親側。隆慶四年,督學副使楊俊民、知府史桂芳詣其舍請見,二女年皆逾六十矣。

黃氏,江寧陳伯妻。年十八,歸伯。父死,母欲改節,氏苦諫不從。一日,母來省,女閉門不與相見,母慚去。後伯疾篤,黃誓不獨生。一日,姑扶伯起坐,黃熟視曰:「嗟乎!病至此,吾無望矣。」走灶下,碎食器刺喉不殊,以廚刀自刎死,年二十一。

邵氏,丹陽大俠邵方家婢也。方子儀,令婢視之。故相徐階、高拱並家居,方以策干階,階不用,即走謁拱,為營復相,名傾中外。萬歷初,拱罷,張居正屬巡撫張佳胤捕殺方,並逮儀。儀甫三歲,捕者以日暮未發,閉方所居宅,守之。方女夫武進沈應奎,義烈士,負氣有力,時為諸生,念儀死,邵氏絕,將往救之。而府推官與應奎善,固邀飲,夜分乃罷。武進距方居五十里,應奎踰城出,夜半抵方家,逾牆入,婢方坐燈下,抱儀泣曰:「安得沈郎來,屬以此子。」應奎倉卒前,婢立以儀授之,頓首曰:「邵氏之祀在君矣。此子生,婢死無憾。」應奎匿儀去,晨謁推官。旦日,捕者失儀,繫婢毒掠,終無言。或言於守曰:「必應奎匿之。」奎所善推官在坐,大笑曰:「冤哉!應奎夜飲於餘,晨又謁餘也。」會有為方解者,事乃寢,婢撫其子以老。

楊貞婦,潼關衛人,字郭恒。萬歷初,客遊湖南,久不歸。父議納他聘,女不可,斷髮自守。家有巖壁,穴牆居之,垂橐以通飲食,如是者二十六年。恆歸,乃成禮。又有倪氏,歸安人,許聘陳敏。敏從征,傳為已死。踰五十載始歸。倪守志不嫁,至是成婚,年六十一矣。

楊氏,寧國饒鼎妻。鼎以單衣溺死湖中,楊招魂葬之,課二子成立,冬不衣袷。萬曆初,年八十,竟單衣入宅旁池中,端坐死。

丁氏,五河王序禮妻。序禮弟序爵客外,為賊所殺,其妻郭氏懷孕未即殉。及生子越月,投繯死。時丁氏適生女,泣謂序禮曰:「叔不幸客死,嬸復殉,棄孤不養,責在君與妾也。妾初舉女,後尚有期,孤亡則斬叔之嗣,且負嬸矣。」遂棄女乳姪。未幾,序禮亦死,竟無子女。氏年方少,撫侄長,絕無怨悔。

尤氏,崑山貢生鏞女。嫁諸生趙一鳳,早死,將殉之,顧二子方襁褓,為彊食。二子復殤,慟曰:「可以從夫矣。」痛夫未葬,即營窀穸。惡少年艷其色,訾其目曰:「彼盼美而流,烏能久也。」婦聞之,夜取石灰手挼目,血出立枯。置棺自隨。夫葬畢,即自縊,或解之,乃觸石裂額,趨臥棺中死。

李氏,王寵麟繼妻。寵麟仕知府卒,氏年二十餘,哭泣不食,經四十日疾革。知族人利其資,必以惡語傾前妻子,預戒家人置己棺中,勿封殮。眾果蝟集,噪孤殺母。氏從棺中言:「已知汝輩計必出此也。」眾大慚而去,然後瞑。

孫氏,甌寧人。幼解經史,字吳廷桂。廷桂死,孫欲左喪,家人止不得,父為命輿。曰:「奔喪而輿,可乎?」入夜,徒步往,挾納採雙金雀以見舅姑。拜畢,奠柩側,遂不離次,期必死。吳家故貧,所治棺,取具而已。好事者助以美檟,孫視之曰:「木以美逾吾夫,非禮矣。」卻之。以槥櫝來,乃許。屆期縊死,書衣帶中云:「男毋附尸,女毋啟衣。」

方孝女,莆田人。父瀾,官儀制郎中,卒京師。女年十四,無他兄弟,與叔父扶櫬歸。渡揚子江,中流舟覆,櫬浮。女時居別舟,皇遽呼救,風濤洶怒,人莫敢前。女仰天大哭,遂赴水死。經三日,屍浮,傍父櫬,同泊南岸。又有解孝女,寧陵人。年十四,同母浣衣。母誤溺水,女四顧無人,號泣投水。俄兄紹武至,泅而得之,母女皆死。女手挽母甚堅,兄救母,久之復蘇。女手仍不解,兄哭撫之曰:「母已生,妹可慰矣。」乃解。

李氏,東鄉何璇妻。璇客死。李有殊色,父迫之嫁。遂以簪入耳中,手自拳之至沒,復拔出,血濺如注。姑覺,呼家人救,則已死矣。

項貞女,秀水人。國子生道亨女,字吳江周應祁。精女工,解琴瑟,通《列女傳》,事祖母及母極孝。年十九,聞周病瘵,即持齋、燃香燈禮佛,默有所祝,侍女輩竊聽,微聞以身代語。一日,謂乳媼曰:「未嫁而夫亡,當奈何?」曰:「未成婦,改字無害。」女正容曰:「昔賢以一劍許人,猶不忍負,況身乎?」及訃聞,父母秘其事,然傳吳江人來,女已喻。祖母屬其母入視,女留母坐,色甚溫,母釋然去。夜伺諸婢熟睡,獨起以素絲約髮,衣內外悉易以縞,而紉其下裳。檢衣物當勞諸婢者,名標之,列諸床上。大書於几日:「上告父母,兒不得奉一日驩,今為周郎死矣。」遂自縊。兩家父母從其志,竟合葬焉。

李氏,壽昌人。年十三,受翁應兆聘。應兆暴卒,女盡取備嫁衣飾焚之,以身赴火,為父母救止。乃赴翁家,哀告舅姑乞立嗣,復乞一小樓,設夫位,坐臥於旁,奠食相對,非姑不接面。舅亡,家落,忍饑紡績以養姑。未幾,姑亦亡,鄰火大起,夜半達旦,延百餘家。鄰婦趨上樓,勸之避,婦曰:「此正我授命時也。」抱夫木主待焚。須臾四面皆燼,小樓獨存。

玉亭縣君,伊府宗室典柄女。年二十四,適楊仞。不兩月仞卒,號慟不食。或勸以舅姑年老,且有遺孕,乃忍死襄事。及生男,家日落。萬曆二十一年,河南大饑,宗祿久缺,紡績三日,不得一飧,母子相持慟哭。夜分夢神語曰:「汝節行上聞於天,當有以相助。」晨興,母子述所夢皆符,頗怪之。其子曰:「取屋後土作坯,易粟。」其日掘土,得錢數百。自是,每掘輒得錢。一日,舍傍地陷,得石炭一窖,取以供爨。延兩月餘,官俸亦至,人以為苦節所感。

馬節婦,年十六,歸平湖諸生劉濂。十七而寡。翁家甚貧,利其再適,必欲奪其志。不與飲食,百計挫之,志益厲。嘗閉門自經,或救之,則繫絕而墜於地死矣。急解之,漸蘇。翁又陰納沈氏聘,其姑誘與俱出,令女奴抱持納沈舟。婦投河不得,疾呼天救我。須臾風雨晝晦,疾雷擊舟,欲覆者數四。沈懼,乃旋舟還之。事聞於縣,縣令婦別居。時父兄盡歿,無可歸,假寓一學舍,官贍之以老。

王氏,東莞葉其瑞妻。其瑞貧,操舟往來鄰境,一月一歸。婦紡績易食。萬歷二十四年,嶺南大饑,民多鬻妻子。其瑞將鬻婦博羅民家,券成,載其人俱來。入門見氏羸甚,問之,不饘粥數日矣。其瑞泣語之故,且示之金,婦笑而許之。及舟發寶潭,躍入潭中死。兩岸觀者如堵,皆謂水迅,屍流無所底。其瑞至,從上流哭數聲,屍忽湧出,去所投處,已逆流數十步矣。

劉氏,博平吳進學妻。楊氏,進性妻。進學疫死,既葬,劉夜匍匐縊於墓所。未幾,進性亦疫死,楊一慟幾絕。姑議嫁之,楊曰:「我何以不如姒。」遂縊死。

譚氏,南海方存業妻。生子三月,夫亡,悲號欲殉。母乃姑交止之,且諷改適。氏垂涕曰:「吾久不樂生,特念姑與兒耳。」哽咽流涕不止,二人不敢復言。及子七歲,遣就塾師,先令拜姑,微示付託意,竊自喜曰:「吾今可以遂志矣。」一日,媒氏至,復勸改適,氏愈憤,中夜縊死。又張氏,臨清林與岐妻。夫亡,欲自縊,舅姑慰之曰:「爾死,如遺孤何?」氏以衣物倩乳嫗育其子,三月,知子安乳嫗,遂不食死。

李烈婦,餘姚吳江妻。年二十,夫與舅俱卒,家酷貧,婦紡績養姑,己恒凍餒。有黃某者,謀娶之,賄夫族某使鉺其姑,未即從。某乃陰與黃及父家約,詭稱其母暴病,肩輿來迎。婦倉卒升輿,既及門,非父家也。姑亦尋至,布几席,速使成禮。婦佯曰:「所以不欲嫁者,為姑老無依耳。姑既許,復何言。然妾自夫歿未嘗解帶,今願一洗沐。」又問:「聘財幾何?」姑以數對。曰:「亟懷之去。姑在,我即從人,殊赧顏也。」眾喜,促姑行,為具湯。湯至,久不出,闢戶視之,則縊死矣。其後,崇禎十五年,餘姚又有黃烈婦者,金一龍妻。夫早歿,黃截指自誓,立從子為嗣,與姑相依。熊氏子欲娶之,母黨利其財,紿令還家,間道送於熊。黃知勢不可挽,願搜括所有以償聘金,不聽,相持至夜深,引刀自刎未殞。其姑聞之,急趨視,黃曰:「婦所以未即死者,欲姑一面耳,今復何求。」遂剜喉以絕。郡邑聞之,斃熊氏子獄中。

須烈婦,吳縣人。夫李死,市兒悅其色,爭欲娶之。婦泣曰:「吾方送一夫,旋迎一夫。且利吾夫之死而妻我,不猶殺我夫耶!」市兒乃糾黨聚謀,將掠之。婦驚奔母,母懼不敢留。返於姑,姑懼知母。投姊,姊益不敢留,婦泣而歸。鄰人勸之曰:「若即死,誰旌若節者,何自苦若此?」婦度終不免,自經死。

陳節婦,安陸人。適李姓,早寡,孑然一身,歸父家守志,坐臥小樓,足不下樓者三十年。臨終,謂其婢曰:「吾死,慎勿以男子舁我。」家人忽其言,令男子登樓舉之,氣絕踰時矣,起坐曰:「始我何言,而令若輩至此。」家人驚怖而下,目乃瞑。

馬氏,山陰劉晉嘯妻。萬曆中,晉嘯客死,馬年二十許,家無立錐。伯氏有樓,遂與母寄居其上,以十指給養,不下梯者數十年。常用瓦盆貯新土,以足附之。鄰婦問故,曰:「吾以服土氣耳。」年六十五卒。

謝烈婦,名玉華,番禺曹世興妻。世興為馮氏塾師,甫成婚,即負笈往。亡何病歸,不能起,婦誓不改適。曹族之老嘉之,議分祭田以贍。或謂婦年方盛,當俟襄事畢,令歸寧,婦佯諾。及期,駕輿欲行,別諸姒,多作訣語,徐入室閉戶,以刀自斷其頸。家人亟穴板入,血流滿衣,尚未絕,見諸人入,亟以左手從斷處探喉出之,右手引刀一割,乃瞑。

張氏,桐城李棟妻。棟死無子,張自經於床。母救之,奮身起,引斧斫左臂者三。家人奪斧,抑而坐之蓐間,張瞆悶不語。家人稍退,張遽揜身出戶投於水。水方冰,以首觸穴入,遂死。」邑又有烈婦王氏,高文學妻。文學死,父道美來弔,謂王曰:「無過哀。事有三等,在汝自為之。」王輟泣問之,父曰:「其一從夫地下為烈,次則冰霜以事翁姑為節,三則恒人事也。」王即鍵戶,絕粒不食,越七日而死。又有戚家婦者,寶應人。甫合巹,而夫暴歿。婦哭之哀,投門外汪中死。後人名其死所為戚家汪云。

金氏,通渭劉大俊妻。年十九,夫病風痺,金扶浴溫泉。暴風雨,山水陡發,夫不能動,令金急走。金號泣堅持不肯舍,並溺死。屍流數十里而出,手猶挽夫不釋云。又應山諸生王芳妻楊氏。芳醉墜塘中,氏赴水救之。夫入水益深,氏追深處偕死。

王氏,山陰沈伯燮妻。議婚數年,伯燮病厲,手攣髮禿,父母有他意。女問:「沈郎病始何日?」父曰:「初許時固佳兒,今乃病。」女曰:「既許而病,命也,違命不祥。」竟歸之。伯燮病且憊,王奉事無少怠。居八年卒,嗣其從子。更出簪珥佐舅買妾,更得子。踰年,舅姑相繼亡,王獨撫二幼孤,鬻手食之,並成立。

李孝婦,臨武人,名中姑,適江西桂廷鳳。姑鄧患痰疾,將不起,婦涕泣憂悼。聞有言乳肉可療者,心識之。一日,煮藥,巘香禱灶神,自割一乳,昏仆於地,氣已絕。廷鳳呼藥不至,出視,見血流滿地,大驚呼救,傾駭城市,邑長佐皆詣其廬,命亟治。俄有僧踵門曰:「以室中蘄艾傅之,即愈。」如其言,果蘇,比求僧不復見矣。乃取乳和藥奉姑,姑竟獲全。又洪氏,懷寧章崇雅妻。崇雅早卒,洪守志十年。姑許,疾不能起,洪剜乳肉為羹而飲之,獲愈,餘肉投池中,不令人知。數日後,群鴨自水中銜出,鳴噪迴翔,小童獲以告姑。姑起視之,乳血猶淋漓也。其夫兄崇古亦早亡,姒朱氏誓死靡他,妯娌相守五十年云。

倪氏,興化陸鰲妻。性純孝,舅早世,憫姑老,朝夕侍寢處,與夫睽異者十五年。姑鼻患疽垂斃,躬為吮治,不愈,乃夜焚香告天,割左臂肉以進,姑啖之愈。遠近稱孝婦。

劉氏,張能信妻,太僕卿憲寵女,工部尚書九德婦也。性至孝,姑病十年,侍湯藥不離側。及病劇,舉刀刲臂,侍婢驚持之。舅聞,囑醫言病不宜近腥膩,力止之。踰日,竟刲肉煮糜以進,則乃姑已不能食,乃大悔恨曰:「醫紿我,使姑未鑒我心。」復刲肉寸許,慟哭奠簀前,將闔棺,取所奠置棺中曰:「婦不獲復事我姑,以此肉伴姑側,猶身事姑也。」鄉人莫不稱其孝。


\end{pinyinscope}