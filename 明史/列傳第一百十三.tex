\article{列傳第一百十三}

\begin{pinyinscope}
張瀚王國光梁夢龍楊巍李戴趙煥鄭繼之

張瀚,字子文,仁和人。嘉靖十四年進士。授南京工部主事。歷廬州知府,改大名。俺答圍京師,詔遣兵部郎中徵畿輔民兵入衛。瀚立閱戶籍,三十丁簡一人,而以二十九人供其餉,得八百人。馳至真定,請使者閱兵,使者稱其才。累遷陜西左布政使,擢右副都御史,巡撫其地。甫半歲,入為大理卿。進刑部右侍郎,俄改兵部,總督漕運。

隆慶元年,改督兩廣軍務,時兩廣各設巡撫官,事不關督府。瀚請如三邊例,乃悉聽節制。大盜曾一本寇掠廣州,詔切責瀚,停總兵官俞大猷、郭成俸。已,一本浮海犯福建,官軍迎擊,大破之,賚銀幣。已,復犯廣東,陷碣石衛,叛將周雲翔等殺雷瓊參將耿宗元,與賊合。廷議鐫瀚一秩調用。已而成大破賊,獲雲翔。詔還瀚秩,即家俟召。再撫陜西。遷南京右都御史,就改工部尚書。

萬曆元年,吏部尚書楊博罷,召瀚代之。秩滿,加太子少保。時廷推吏部尚書,首左都御史葛守禮,次工部尚書朱衡,次瀚。居正惡守禮戇,厭衡驕,故特拔瀚。瀚資望淺,忽見擢,舉朝益趨事居正,而瀚進退大臣率奉居正指。即出己意,輿論多不協。以是為御史鄭準、王希元所劾。居正顧之厚,不納也。御史劉臺劾居正,因論瀚撫陜狼籍,又唯諾居正狀。

比居正遭喪,謀奪情,瀚心非之。中旨令瀚諭留居正,居正又自為牘,風瀚屬吏,以覆旨請。瀚佯不喻,謂「政府奔喪,宜予殊典,禮部事也,何關吏部。」居正復令客說之,不為動,乃傳旨責瀚久不奉詔,無人臣禮。廷臣惴恐,交章留居正,瀚獨不與,撫膺太息曰:「三綱淪矣!」居正怒,嗾給事中王道成、御史謝思啟摭他事劾之,勒致仕歸。居正歿,帝頗念瀚。詔有司給月廩,年及八十,特賜存問。卒,贈太子少保,謚恭懿。

王國光,字汝觀,陽城人。嘉靖二十三年進士。授吳江知縣。鄰邑有疑獄來質,訊輒得情。調儀封,擢兵部主事。改吏部,歷文選郎中。屢遷戶部右侍郎,總督倉場。謝病去。隆慶四年,起刑部左侍郎,拜南京刑部尚書。未上,改戶部,再督倉場。神宗即位,還理部事。時簿牒繁冗,自州縣達部,有繕書、輸解、交納諸費,公私苦之。國光疏請裁併,去繁文十三四,時稱簡便。戶部十三司,自弘治來,以公署隘,惟郎中一人治事,員外郎、主事止除官日一赴而已。郎中力不給,則委之吏胥,弊益滋。國光盡令入署,職務得修舉。邊餉告匱,而諸邊歲出及屯田、監課無可稽。國光請敕邊臣核實,且畫經久策以聞。甘肅巡撫廖逢節等各條上其數,耗蠹為損。

萬歷元年,奏言:「國初,天下州縣存留夏稅秋糧可一千二百萬石。其時議主寬大,歲用外,計贏銀百萬有餘。使有司歲征無缺,則州縣積貯自豐,水旱盜賊不能為災患。今一遘兵荒,輒留京儲,發內帑。由有司視存留甚緩,茍事催科,則謂擾民,弊遂至此。請行天下撫按官,督所司具報出入、存留、逋負之數,臣部得通融會計,以其餘濟邊。有司催征不力者,悉以新令從事。」制可。京軍支糧通州者,候伺甚艱。國光請遣部郎一人司之,名坐糧廳,投牒驗發,無過三日,諸軍便之。天下錢穀散隸諸司,國光請歸併責成:畿輔府州縣歸福建司,南畿歸四川司,鹽課歸山東司,關稅歸貴州司,淮、徐、臨、德諸倉歸雲南司,御馬、象房及二十四馬房芻料歸廣西司。遂為定制。

三年,京察拾遺。國光為南京給事、御史所劾。再疏乞罷,帝特留之。明年復固以請,乃詔乘傳歸。瀕行,以所輯條例名《萬曆會計錄》上之。帝嘉其留心國計,令戶部訂正。及書成,詔褒諭焉。五年冬,吏部尚書張瀚罷,起國光代。陳采實政、別繁簡、責守令、恤卑官、罷加納數事,皆允行。尋以考績,加太子太保。八年,當考察外吏,請毋限日期。詔許之,且命詿誤者聽從公辯雪。明年大計京朝官,徇張居正意,置吳中行等五人於察籍。

國光有才智。初掌邦計,多所建白。及是受制執政,聲名損於初。給事中商尚忠論國光銓選私所親,而給事中張世則出為河南僉事,憾國光,劾其鬻官黷貨。國光再奏辯,帝再慰留,責世則挾私,貶儀真丞。及居正卒,御史楊寅秋劾國光六罪。帝遂怒,落職閒住。已,念其勞,命復官致仕。

梁夢龍,字乾吉,真定人。嘉靖三十二年進士,改庶吉士。授兵科給事中,首劾吏部尚書李默。帝方顧默厚,不問。出核陜西軍儲。劾故延綏巡撫王輪、督糧郎中陳燦等,廢斥有差。歷吏科都給事中。帝怒禮部尚書吳山,夢龍惡獨劾山得罪清議,乃并吏部尚書吳鵬劾罷之。嘗上疏,言:「相臣賢否,關治道污隆。請毋拘資格,敕在廷公舉名德宿望之臣,以光聖治。」帝疑諸臣私有所推引,責令陳狀。夢龍惶恐謝罪,乃奪俸。擢順天府丞。坐京察拾遺,出為河南副使。河決沛縣,尚書朱衡議開徐、邳新河,夢龍董其役。三遷河南右布政使。

隆慶四年,擢右僉都御史,巡撫山東。是秋,河決宿遷,覆漕糧八百艘。朝議通海運,以屬夢龍。夢龍言:「海道南自淮安至膠州,北自天津至海倉,各有商艇往來其間。自膠州至海倉,島人及商賈亦時出入。臣等因遣人自淮安轉粟二千石,自膠州轉麥千五百石,入海達天津,以試海道,無不利者。由淮安至天津,大要兩旬可達。歲五月以前,風勢柔順,揚帆尤便。況舟由近洋,洋中島嶼聯絡,遇風可依。茍船非朽敝,按占候以行,自可無虞。較元人殷明略故道,安便尤甚。丘浚所稱『傍海通運』,即此是也。請以河為正運,海為備運。萬一河未易通,則海運可濟,而河亦得悉心疏濬,以圖經久。又海防綦重,沿海衛所玩心妻歲久,不加繕飭,識者有未然之憂。今行海運兼治河防,非徒足裨國計,兼於軍事有補。」章下戶部,部議海運久廢,猝難盡復,請令漕司量撥糧十二萬石,自淮入海以達天津。工部給銀,為海艘經費。報可。已而海運卒不行,事具《王宗沐傳》。明年冬,遷右副都御史,移撫河南。

神宗初,張居正當國。夢龍其門下士,特愛之,召為戶部右侍郎。尋改兵部,出賚遼東有功將士。五年,以兵部左侍郎進右都御史,總督薊、遼、保定軍務。李成梁大破土蠻於長定堡,帝為告廟宣捷,大行賞賚,官夢龍一子。已,給事中光懋言:「此乃保塞內屬之部,游擊陶承嚳假犒賚掩襲之,請坐以殺降罪。」兵部尚書方逢時曲為解,夢龍等亦辭免恩廕。及土蠻三萬騎入東昌堡,成梁擊敗之。寧前復警,夢能親率勁卒三千出山海關為成梁聲援,分遣兩參將遮擊,復移繼光駐一片石邀之,敵引去。前後奏永奠堡、丁字泊、馬蘭峪、養善木、紅土城、寬奠、廣寧右屯、錦、義、大寧堡諸捷,累賜敕獎勵,就加兵部尚書。以修築黃花鎮、古北口邊牆,加太子少保,再廕子至錦衣世千戶。召入掌部務,疏陳軍政四事。尋錄防邊功,加太子太保。

十年六月,居正歿,吏部尚書王國光劾罷,夢龍代其位。踰月,御史江東之劾夢龍浼徐爵賄保得吏部,以孫女聘保弟為子婦,御史鄧練、趙楷復劾之,遂令致仕。家居十九年卒。天啟中,趙南星訟其邊功,贈少保。崇禎末,追謚貞敏。

楊巍,字伯謙,海豐人。嘉靖二十六年進士。除武進知縣。擢兵科給事中。操江僉都御史史褒善已遷大理卿,巍言:「東南倭患方劇,參贊、巡撫俱論罪,褒善獨倖免,又夤緣美遷,請并吏部罰治。」帝怒,停選司俸,還褒善故官。巍既忤吏部,遂出為山西僉事。已,遷參議,分守宣府。寇入犯,偕副將馬芳擊斬其部長,賚銀幣。尋為陽和兵備副使。擢右僉都御史,巡撫宣府。錄搗巢功,進秩二級。踰年,以養母歸。歸二年,召起巡撫陜西。增補屯戍軍伍,清還屯地之奪於籓府者。隆慶初,進右副都御史,移撫山西。所部驛遞銀歲征五十四萬,巍請減四之一。修築沿邊城堡,檄散大盜李九經黨。復乞養母去。

神宗立,起兵部右侍郎。萬曆二年,改吏部,進左,又以終養歸。母年逾百歲卒。十年,起南京戶部尚書,旋召為工部尚書。有詔營建行宮,近功德寺。巍爭之,乃止。明年,改戶部,遷吏部尚書。明制,六部分蒞天下事,內閣不得侵。至嚴嵩,始陰撓部權。迨張居正時,部權盡歸內閣,逡巡請事如屬吏,祖制由此變。至是,申時行當國。巍素厲清操,有時望,然年耄骫骳,多聽其指揮。御史丁此呂論科場事,時行及餘有丁、許國輩皆惡之。巍論謫此呂,為御史江東之、李植等所攻,與時行俱乞罷。帝從諸大臣請,慰留巍等而戒諭言者,巍乃起復視事。

當居正初敗,言路張甚,帝亦心疑諸大臣朋比,慾言官摘發之以杜壅蔽。諸大臣懼見攻,政府與銓部陰相倚以制言路。先是,九年京察,張居正令吏部盡除異己者。十五年,復當大計,都御史辛自修欲大有所澄汰,巍徇政府指持之。出身進士者,貶黜僅三十三人,而翰林、吏部、給事、御史無一焉。賢否混淆,群情失望。十七年夏,帝久不視朝,中外疑帝以張鯨不用故託疾。巍率同列請以秋日御殿。至十月,巍等復請。帝不悅,責以沽名。

巍初歷中外,甚有聲。及秉銓,素望大損。然有清操,性長厚,不為刻核行。明年,以年幾八十,屢疏乞歸。詔乘傳、給廩隸如故事。歸十五年,年九十二而卒。贈少保。

李戴,字仁夫,延津人。隆慶二年進士。除興化知縣,有惠政。擢戶科給事中。廣東以軍興故,增民間稅。至萬曆初亂定,戴奏正之。累遷禮科都給事中。出為陜西右參政,進按察使。張居正尚名法,四方大吏承風刻核,戴獨行之以寬。由山西左布政使擢右副都御史,巡撫山東。歲凶,累請蠲振。入為刑部侍郎。累進南京戶部尚書,召拜工部尚書,以繼母憂去。

二十六年,吏部尚書蔡國珍罷。廷推代者七人,戴居末,帝特擢用之。當是時,趙志皋、沈一貫輔政,雖不敢撓部權,然大僚缺人,九卿及科道掌印者咸得自舉聽上裁,而吏部諸曹郎亦由九卿推舉,尚書不得自擇其屬,在外府佐及州縣正、佐官則盡用掣簽法,部權日輕。戴視事,謹守新令,幸無罪而已。明年,京察。編修劉綱、中書舍人丁元薦、南京評事龍起雷嘗以言事忤當路,咸置察中,時議頗不直戴。而是時國本未定,皇長子冠婚久稽,戴每倡廷臣直諫。及礦稅害劇,戴率九卿言:「陳增開礦山東,知縣吳宗堯逮。李道抽分湖口,知府吳寶秀等又逮。天下為增、道者何限,有司安所措手足?且今水旱頻仍,田里蕭耗,重以東征增兵益餉,而西事又見告矣。民不聊生,奸宄方竊發,奈何反為發其機,速其變哉!」不報。

山西稅使張忠奏調夏縣知縣韓薰簡僻。戴以內官不當擅舉刺,疏爭之。湖廣陳奉屢奏逮有司,戴等又極論,且言:「奉及遼東高淮擅募勁卒橫民間,尤不可不問。」帝亦弗聽。已,復偕同列言:「自去夏六月不雨至今,路殣相望,巡撫汪應蛟所奏饑民十八萬人。加以頻值寇警,屢興征討之師,按丁增調,履畝加租,賦額視二十年前不啻倍之矣。瘡痍未起,而採榷之害又生。不論礦稅有無,概勒取之民間,此何理也。天下富室無幾,奸人肆虐何極。指其屋而恐之曰『彼有礦』,則家立破矣;『彼漏稅』,則橐立罄矣。持無可究詰之說,用無所顧畏之人,蚩蚩小民,安得不窮且亂也。湖廣激變已數告,而近日武昌尤甚。此輩寧不愛性命哉?變亦死,不變亦死,與其吞聲獨死,毋寧與仇家俱糜。故一發不可遏耳。陛下可視為細故耶?」亦不報。

三十年二月,帝有疾,詔罷礦稅、釋繫囚、錄建言譴謫諸臣。越日,帝稍愈,命礦稅採榷如故。戴率同官力諫。時釋罪、起廢二事,猶令閣臣議行,戴即欲疏名上請,而刑部尚書蕭大亨謂釋罪必當奏聞。方具疏上,太僕卿南企仲以二事久稽,劾戴等不能將順。帝怒,并停前詔。戴引罪求罷,帝不許。自是請起廢者再,率九卿乞停礦稅者四,皆不省。稽勳郎中趙邦清素剛介,為給事中張鳳翔所劾,疑出文選郎中鄧光祚、驗封郎中侯執躬意,辨疏侵之。御史沈正隆、給事中田大益交章劾邦清。邦清憤,盡發光祚、執躬私事。光祚亦騰疏力攻,部中大哄,戴無所裁抑。御史左宗郢、李培遂劾戴表率無狀,戴引疾乞去。帝諭留,為貶邦清三秩,允光祚執躬歸,群囂乃息。

明年冬,妖書事起。錦衣官王之楨等與同官周嘉慶有隙,言妖書嘉慶所為,下詔獄窮治。嘉慶,戴甥也,比會鞫,戴引避。帝聞而惡之。會王士騏通書事發,下部議。士騏奏辨。帝謂士騏不宜辨,責戴不能鉗屬官。戴引罪,而疏紙誤用印,復被譙讓,罪其司屬。戴疏謝,用印如故。帝怒,令致仕,奪郎中以下俸。

戴秉銓六年,溫然長者,然聲望出陸光祖諸人下。趙志皋、沈一貫柄政,戴不敢為異,以是久於其位,而銓政益頹廢矣。卒贈少保。

趙煥,字文光,掖縣人。嘉靖四十四年進士。授烏程知縣。入為工部主事,改御史。萬歷三年,中官張宏請遣其黨榷真定材木,煥及給事中侯于趙執奏,不從。張居正遭父喪,言官交章請留,煥獨不署名。擢順天府丞,累遷左僉都御史。

十四年三月,風霾求言。煥請恢聖度,納忠言,謹頻笑,信政令,時召大臣商榷治理,次第舉行實政,弊在內府者一切報罷,而飭戒督撫有司務求民瘼。帝嘉納焉。尋遷工部右侍郎。改吏部,進左。乞假去。起南京右都御史,以親老辭。時煥兄遼東巡撫僉都御史燿亦乞歸養。吏部言二人情同,燿為長子,且任封疆久,可聽其歸。乃趣煥就職。尋召為刑部尚書。議日本貢事,力言非策。男子諸龍光訐奏李如松通倭下吏,并及其黨陳仲登枷赤日中,期滿戍瘴鄉。煥以盛暑必斃,而二人罪不當死,兩疏力爭。忤旨,詰責。復以議浙江巡按彭應參獄失帝意,遂引疾歸。再起南京右都御史,就改吏部尚書,皆不赴。家居十六年。召拜刑部尚書,尋兼署兵部。

四十年二月,孫丕揚去,改署吏部。時神宗怠於於事,曹署多空。內閣惟葉向高,杜門者已三月。六卿止一煥在,又兼署吏部,吏部無復堂上官。兵部尚書李化龍卒,召王象乾未至,亦不除侍郎。戶、禮、工三部各止一侍郎而已。都察院自溫純罷去,八年無正官。故事,給事中五十人,御史一百十人,至是皆不過十人。煥累疏乞除補。帝皆不報。其年八月,遂用煥為吏部尚書,諸部亦除侍郎四人。既而考選命下,補給事中十七人,御史五十人,言路稱盛。

然是時朋黨已成,中朝議論角立。煥素有清望,驟起田間,於朝臣本無所左右,顧雅不善東林。諸攻東林者乘間入之。所舉措往往不協清議,先後為御史李若星、給事中孫振基所劾。帝皆優詔慰留之。已,兵部主事卜履吉為署部事都御史孫瑋所論。煥以履吉罪輕,擬奪俸三月。給事中趙興邦劾煥徇私。煥疏辨,再乞罷。向高言:「今國事艱難,人才日寡。在野者既賜環無期,在朝者復晨星無幾,乃大小臣工,日尋水火,甚非國家福也。臣願自今已後共捐成心,憂國事,議論聽之言官,主張聽之當事。使大臣得展布而毋苦言官之掣肘,言官得發舒而毋患當事之摧殘,天下事尚可為也。」因請諭煥起視事,煥乃出。

明年春,以年例出振基及御史王時熙、魏雲中於外。三人嘗力攻湯賓尹、熊廷弼者,又不移咨都察院,於是御史湯兆京守故事爭,且詆煥。煥屢疏訐辯,杜門不出,詔慰起之。兆京以爭不得,投劾徑歸。其同官李邦華、周起元、孫居相,及戶部郎中賀烺交章劾煥擅權,請還振基等於言路。帝為奪諸臣俸,貶烺官以慰煥。煥請去益力。九月,遂叩首闕前,出城待命。帝猶遣諭留。給事中李成名復劾煥伐異黨同,煥遂稱疾篤,堅不起。踰月,乃許乘傳歸。

四十六年,吏部尚書鄭繼之去國。時黨人勢成,清流斥逐已盡。齊黨亓詩教摯尤張。以煥為鄉人老而易制,力引煥代繼之,年七十有七矣。比至,一聽詩教指揮,不敢異同,由是素望益損。帝終以煥清操,委信之。及明年七月,遼東告警,煥率廷臣詣文華門固請帝臨朝議政。抵暮,始遣中官諭之退,而諸軍機要務廢閣如故。煥等復具疏趣之,且作危語曰:「他日薊門蹂躪,敵人叩閽,陛下能高枕深宮,稱疾謝卻之乎?」帝由是嗛焉。考滿當增秩,寢不報。煥尋卒,恤典不及。光宗立,始賜如制。熹宗初,贈太子太保。

鄭繼之,字伯孝,襄陽人。嘉靖四十四年進士。除餘幹知縣。遷戶部主事,歷郎中。遷寧國知府,進四川副使,以養親歸。服除,久之不出。萬歷十九年,用給事中陳尚象薦,起官江西,進右參政。召為太僕少卿,累遷大理卿。東征師罷,吏部尚書李戴議留戍兵萬五千,令朝鮮供億。繼之曰:「既留兵,自當轉餉,柰何疲敝屬國。」議者韙之。為大理九年,擢南京戶部尚書,就改吏部。

四十一年,吏部尚書趙煥罷。時帝雖倦勤,特謹銓部選,久不除代。以繼之有清望,明年二月,乃召之代煥。繼之久處散地,無黨援。然是時言路持權,齊、楚、浙三黨尤橫,大僚進退,惟其喜怒。繼之故楚產,習楚人議論,且年八十餘,耄而憒,遂一聽黨人意指。文選郎中王大智者,繼之所倚信。其秋以年例出御史宋匋、潘之祥,給事中張鍵,南京給事中張篤敬於外,皆嘗攻湯賓尹、熊廷弼者也。時定制,科道外遷必會都察院吏科,繼之不令與聞。比考選科道,中書舍人張光房,知縣趙運昌、張廷拱、曠鳴鸞、濮中玉當預,而持議頗右於玉立、李三才,遂見抑,改授部曹。大智同官趙國琦以為言。大智怒,構於繼之逐之去。由是御史孫居相、張五典、周起元等援年例故事以爭,且為光房等五人稱枉,吏科都給事中李瑾亦以失職抗疏劾大智。御史唐世濟則右吏部,詆居相等。居相、瑾怒,交章劾世濟。給事中、御史復助世濟排擊居相。居相再疏力攻大智,大智乃引疾去。繼之亦覺其非,不為辯。

至明年二月,胡來朝為文選,出兵科都給事中張國儒、御史馬孟禎、徐良彥於外,復不咨都察院、吏科。國儒已陪推京卿,法不當出外;孟禎、良彥則素忤黨人,故來朝抑之。繼之不能禁。時居相等已去國,獨瑾再爭,詆繼之、來朝甚力。來朝等不能難,其黨思以眾力勝之,於是諸御史群起攻瑾。瑾爭之強,疏三上。來朝等亦三疏詆訐,詞頗窮。來朝乃言:「年例協贊之旨,實秉國者調停兩袒,非可為制,乞改前令從事。」帝一無所處分。瑾方奉使,自引去。其秋,給事中梅之煥、御史李若星、張五典年例外轉,所司復不預聞。吏科韓光裕、御史徐養量稍言之,然勢孤,竟不能爭也。時縉雲李鋕以刑部尚書兼署都察院,亦浙黨所推轂。四十五年,大計京官,繼之與鋕司其事,考功郎中趙士諤、給事中徐紹吉、御史韓浚佐之。所去留悉出紹吉等意,繼之受成而已。一時與黨人異趣者,貶黜殆盡,大僚則中以拾遺,善類為空。

繼之以篤老累疏乞休,帝輒慰留不允。明年春,稽首闕下,出郊待命。帝聞,命乘傳歸。又數年卒,年九十二。贈少保。

贊曰:張瀚、王國光、梁夢龍皆以才辦稱,楊巍、趙煥、鄭繼之亦負清望,及秉銓政,蒙詬議焉。於時政府參懷,言路脅制,固積重難返,然以公滅私之節,諸人蓋不能無愧云。


\end{pinyinscope}