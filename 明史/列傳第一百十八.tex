\article{列傳第一百十八}

\begin{pinyinscope}
蔡時鼎萬國欽王教饒伸兄位劉元震元霖湯顯祖李琯立盧明諏楊恂冀體硃爵姜士昌宋燾馬孟禎汪若霖

蔡時鼎,字臺甫,漳浦人。萬曆二年進士。歷知桐鄉、元城,為治清嚴。征授御史。太和山提督中官田玉兼分守事,時鼎言不可,並及玉不法狀。御史丁此呂以劾高啟愚被謫,時鼎論救,語侵楊巍、申時行。報聞。已,巡鹽兩淮。悉捐其羨為開河費,置屬邑學田。

還朝,會戚畹子弟有求舉不獲者,誣順天考官張一桂私其客馮詩、童維寧及編修史鈳子記純,又濫取冒籍者五人。帝怒,命詩、維寧荷枷,解一桂、鈳官。時行等為之解。帝益怒,奪鈳職,下詩、維寧吏。法司廷鞫無驗,忤旨被讓。卒枷二人一月,而調一桂南京。時鼎以事初糾發不由外廷,徑從中出,極言「宵人蜚語直達御前,其漸不可長;且盡疑大臣言官有私,則是股肱耳目舉不可信,所信者誰也?」帝怒,手札諭閣臣治罪。會時行及王錫爵在告,許國、王家屏僅擬停俸,且請稍減詩、維寧荷校之期,以全其命。帝不從,責時鼎疑君訕上,降極邊雜職。又使人詗知發遣冒籍者多寬縱,責府尹沈思孝對狀。國、家屏復上言:「人君貴明不貴察。茍任一己見聞,猜防苛密,縱聽斷精審,何補於治;且使姦人乘機得中傷善類,害胡可言!願停察訪以崇大體,宥言官以彰聖度。」帝不懌,手詔詰讓。是日,帝思時行,遣中使就第勞問。而國等既被責,具疏謝,執爭如初。會帝意稍解,乃報聞。時鼎竟謫馬邑典史,告歸。居二年,吏部擬序遷,不許。御史王世揚請如石星、海瑞、鄒元標例,起之廢籍,不報。已,起太平推官,進南京刑部主事,就改吏部。

十八年冬,復疏劾時行,略言:「比年天災民困,紀綱紊斁,吏治混淆。陛下深居宮闕,臣民呼籲莫聞。然群工進言,猶蒙寬貸。乃輔臣時行則樹黨自堅,忌言益甚。不必明指其失,即意向稍左,亦輒中傷。或顯斥於當時,或徐退於後日。致天下諛佞成風,正氣消沮。方且內托之乎雅量,外托之乎清明,此聖賢所以重似是之防,嚴亂德之戒也。夫營私之念重,則奉公之意必衰;巧詐之機熟,則忠誠之節必退。自張居正物故,張四維憂去,時行即為首輔。懲前專擅,矯以謙退;鑒昔嚴苛,矯以寬平。非不欲示休休之量,養和平之福,無如患得患失之心勝,而不可則止之義微。貌退讓而心貪競,外包容而中忮刻。私偽萌生,欲蓋彌著。夫居正之禍在徇私滅公,然其持法任事,猶足有補於國。今也改革其美,而紹述其私;盡去其維天下之心,而益巧其欺天下之術。徒思邀福一身,不顧國禍,若而人者,尚可俾相天下哉!」因歷數其十失,勸之省改。疏留中。尋進南京禮部郎中。卒官。貧不具含殮,士大夫賻而治其喪。

萬國欽,字二愚,新建人。萬曆十一年進士。授婺源知縣。徵拜御史。言事慷慨,不避權貴。十八年,劾吏部尚書楊巍,被詰讓。里居尚書董份,大學士申時行、王錫爵座主也,屬浙江巡按御史奏請存問。國欽言份諂事嚴嵩,又娶尚書吳鵬已字子女,居鄉無狀,不宜加隆禮,事遂寢。

初,吏部員外郎趙南星、戶部主事姜士昌疏斥政府私人。給事中李春開以出位糾南星、士昌,而其黨陳與郊為助。刑部主事吳正志上疏,言春開、與郊媚政府,乾清議,且論御史林祖述保留大臣之非。於是御史赫瀛集諸御史於朝堂,議合疏糾正志,以臺體為辭。國欽與周孔教獨不署名。瀛大恚,盛氣讓國欽。國欽曰:「冠豸冠,服豸服,乃日以保留大臣傾善類為事,我不能茍同。」瀛氣奪,疏不果上,而正志竟謫宜君典史。奄人袁進等毆殺平民,國欽再疏劾之。

十八年夏,火落赤諸部頻犯臨洮、鞏昌。七月,帝召見時行等於皇極門,咨以方略,言邊備廢弛,督撫乏調度,欲大有所振飭。時行以款貢足恃為言。帝曰:「款貢亦不足恃。若專務媚敵,使心驕意大,豈有饜足時?」時行等奉諭而退。未幾,警報狎至,乃推鄭洛為經略尚書行邊,實用以主款議也。國欽抗疏劾時行,曰:「陛下以西事孔棘,特召輔臣議戰守,而輔臣於召對時乃飾詞欺罔。陛下怒賊侵軼,則以為攻抄熟番。臨、鞏果番地乎?陛下責督撫失機,則以為咎在武臣。封疆僨事,督撫果無與乎?陛下言款貢難恃,則云通貢二十年,活生靈百萬。西寧之敗,肅州之掠,獨非生靈乎?是陛下意在戰,時行必不欲戰;陛下意在絕和,時行必欲與和。蓋由九邊將帥,歲餽金錢,漫無成畫。寇已殘城堡,殺吏民,猶謂計得。三邊總督梅友松意專媚敵。前奏順義謝恩西去矣,何又圍我臨、鞏?後疏盛誇戰績矣,何景古城全軍皆覆?甘肅巡撫李廷儀延賊入關,不聞奏報,反代請贖罪。計馬牛布帛不及三十金,而殺掠何止萬計!欲仍通市,臣不知於國法何如也。此三人皆時行私黨,故敢朋奸誤國乃爾。」因列上時行納賄數事。帝謂其淆亂國事,誣污大臣,謫劍州判官。初,國欽疏上,座主許國責之曰:「若此舉,為名節乎,為國家乎?」國欽曰:「何敢為名節,惟為國事耳。即言未當,死生利害聽之。」國無以難。

二十年,吏部尚書陸光祖擬量移國欽為建寧推官,饒伸為刑部主事。帝以二人皆特貶,不宜遷,切責光祖,而盡罷文選郎中王教、員外郎葉隆光、主事唐世堯、陳遴瑋等。大學士趙志皋疏救,亦被譙責。國欽後歷南京刑部郎中,卒。

王教,淄川人。佐光祖澄清吏治。給事中胡汝寧承權要旨劾之,事旋白。竟坐推國欽、伸,斥為民。

饒伸,字抑之,進賢人。萬歷十一年進士。授工部主事。十六年,庶子黃洪憲典順天試,大學士王錫爵子衡為舉首,申時行婿李鴻亦預選。禮部主事于孔兼疑舉人屠大壯及鴻有私。尚書硃賡、禮科都給事中苗朝陽欲寢其事。禮部郎中高桂遂發憤謫可疑者八人,並及衡,請得覆試。錫爵疏辨,與時行並乞罷。帝皆慰留之,而從桂請,命覆試。禮部侍郎於慎行以大壯文獨劣,擬乙置之。都御史吳時來及朝陽不可。桂直前力爭,乃如慎行議,列甲乙以上。時行、錫爵調旨盡留之,且奪桂俸二月。衡實有才名,錫爵大憤,復上疏極詆桂。伸乃抗疏言:「張居正三子連占高科,而輔臣子弟遂成故事。洪憲更謂一舉不足重,居然置之選首。子不與試,則錄其婿,其他私弊不乏聞。覆試之日,多有不能文者。時來罔分優劣,蒙面與桂力爭,遂朦朧擬請。至錫爵訐桂一疏,劍戟森然,乖對君之體。錫爵柄用三年,放逐賢士,援引憸人。今又巧護己私,欺罔主上,勢將為居正之續。時來附權蔑紀,不稱憲長。請俱賜罷。」

疏既入,錫爵、時行並杜門求去。而許國以典會試入場,閣中遂無一人。中官送章奏於時行私第,時行仍封還。帝驚曰:「閣中竟無人耶?」乃慰留時行等,而下伸詔獄。給事中胡汝寧、御史林祖述等復劾伸及桂,以媚執政。御史毛在又侵孔兼,謂桂疏其所使。孔兼奏辨求罷。於是詔諸司嚴約所屬,毋出位沽名,而削伸籍,貶桂三秩,調邊方,孔兼得免。伸既斥,朝士多咎錫爵。錫爵不自安,屢請敘用。起伸南京工部主事,改南京吏部。引疾歸,遂不復出。熹宗即位,起南京光祿寺少卿。天啟四年累官刑部左侍郎。魏忠賢亂政,請告歸。所輯《學海》六百餘卷,時稱其浩博。

兄位。累官工部右侍郎。母年百歲,與伸先後以侍養歸。

先是,任丘劉元震、元霖兄弟俱官九列,以母年近百歲,先後乞養親歸,與伸兄弟相類。一時皆以為榮。元震,字元東,隆慶五年進士。由庶吉士萬曆中歷官吏部侍郎。天啟中,贈禮部尚書,謚文莊。元霖,萬歷八年進士。歷官工部尚書。福王開邸洛陽,有所營建。元霖執奏,罷之。卒,贈太子太保。

湯顯祖,字若士,臨川人。少善屬文,有時名。張居正欲其子及第,羅海內名士以張之。聞顯祖及沈懋學名,命諸子延致。顯祖謝弗往,懋學遂與居正子嗣修偕及第。顯祖至萬曆十一年始成進士。授南京太常博士,就遷禮部主事。十八年,帝以星變嚴責言官欺蔽,並停俸一年。顯祖上言曰:「言官豈盡不肖,蓋陛下威福之柄潛為輔臣所竊,故言官向背之情,亦為默移。御史丁此呂首發科場欺蔽,申時行屬楊巍劾去之。御史萬國欽極論封疆欺蔽,時行諷同官許國遠謫之。一言相侵,無不出之於外。於是無恥之徒,但知自結於執政。所得爵祿,直以為執政與之。縱他日不保身名,而今日固已富貴矣。給事中楊文舉奉詔理荒政,徵賄巨萬。抵杭,日宴西湖,鬻獄市薦以漁厚利。輔臣乃及其報命,擢首諫垣。給事中胡汝寧攻擊饒伸,不過權門鷹犬,以其私人,猥見任用。夫陛下方責言官欺蔽,而輔臣欺蔽自如。失今不治,臣謂陛下可惜者四:朝廷以爵祿植善類,今直為私門蔓桃李,是爵祿可惜也。群臣風靡,罔識廉恥,是人才可惜也。輔臣不越例予人富貴,不見為恩,是成憲可惜也。陛下御天下二十年,前十年之政,張居正剛而多欲,以群私人,囂然壞之;後十年之政,時行柔而多欲,以群私人,靡然壞之。此聖政可惜也。乞立斥文舉、汝寧,誡諭輔臣,省愆悔過。」帝怒,謫徐聞典史。稍遷遂昌知縣。二十六年,上計京師,投劾歸。又明年大計,主者議黜之。李維禎為監司,力爭不得,竟奪官。家居二十年卒。

顯祖意氣慷慨,善李化龍、李三才、梅國楨。後皆通顯有建豎,而顯祖蹭蹬窮老。三才督漕淮上,遣書迎之,謝不往。

顯祖建言之明年,福建僉事李琯奉表入都,列時行十罪,語侵王錫爵。言惟錫爵敢恣睢,故時行益貪戾,請並斥以謝天下。帝怒,削其籍。甫兩月,時行亦罷。琯,豐城人。萬曆五年進士。嘗官御史。既斥歸,家居三十年而卒。

顯祖子開遠,自有傳。

逯中立,字與權,聊城人。萬曆十七年進士。由行人擢吏科給事中。遇事敢言。行人高攀龍,御史吳弘濟,南部郎譚一召、孫繼有、安希範咸以爭趙用賢之罷被斥,中立抗疏曰:「諸臣率好修士,使跧伏田野,誠可惜也。陛下怒言者,則曰『出朕獨斷』,輔臣王錫爵亦曰『至尊親裁』。臣謂所斥者非正人也,則斷自宸衷,固陛下去邪之明;即擬自輔臣,亦大臣為國之正。若所斥者果正人也,出於輔臣之調旨,而有心斥逐者為妒賢;即出於至尊之親裁,而不能匡救者為竊位。大臣以人事君之道,當如是乎?陛下欲安輔臣,則罷言者;不知言者罷,輔臣益不自安。」疏入,忤旨,停俸一歲。

尋進兵科右給事中。有詔修國史,錫爵舉故詹事劉虞夔為總裁。虞夔,錫爵門生也,以拾遺劾罷。諸御史言不當召。而中立詆虞夔尤力,並侵錫爵,遂寢召命。未幾,文選郎顧憲成等以會推閣臣事被斥,給事中盧明諏救之,亦貶秩。中立上言:「兩年以來,銓臣相繼屏斥。尚書孫鑨去矣,陳有年杜門求罷矣,文選一署空曹逐者至再三,而憲成又繼之。臣恐今而後,非如王國光、楊巍,則不能一日為塚宰;非如徐一檟、謝廷寀、劉希孟,則不能一日為選郎。臧否混淆,舉錯倒置,使黜陟重典寄之權門,用舍斥罰視一時喜怒,公議壅閼,煩言滋起。此人才消長之機,理道廢興之漸,不可不深慮也。且會推閣臣,非自十九年始。皇祖二十八年廷推六員,而張治、李本二臣用;即今元輔錫爵之入閣,亦會推也。蓋特簡與廷推,祖宗並行已久。廷推必諧於僉議,特簡或由於私援。今輔臣趙志皋等不稽故典,妄激聖怒,即揭救數語,譬之強笑,而神不偕來,欲以動聽難矣。方今疆埸交聳,公私耗敝,群情思亂,識者懷憂。乃朝議紛紜若爾,豈得不長嘆息哉!」帝怒,嚴旨責讓,斥明諏為民,而貶中立陜西按察司知事。引疾歸,家居二十年卒。熹宗時,贈光祿少卿。

盧明諏,黃巖人。萬曆十四年進士。

楊恂,字伯純,代人。萬曆十一年進士。授行人,擢刑科給事中。錦衣冗官多至二千人,請大加裁汰,不用。累遷戶科都給事中。朝鮮用兵,冒破帑金不貲。恂請嚴敕邊臣,而劾武庫郎劉黃裳侵耗罪。黃裳卒罷去。尋上節財四議,格不行。

王錫爵謝政,趙志皋代為首輔。御史柳佐、章守誠劾之。志皋乞罷,不許。御史冀體極論志皋不可不去。帝怒,責對狀。體抗辭不屈,貶三秩,出之外,以論救者眾,竟斥為民。恂復論志皋,並及張位。其略曰:「今之議執政者,僉曰擬旨失當也,貪鄙無為也。是固可憂,而所憂有大於是者。許茂橓罷閒錦衣,厚齎金玉為奸,被人緝獲。使大臣清節素孚,彼安敢冒昧如此!乃緝獲者被責,而行賄者不問。欲天下澄清,其可得耶?可憂者一。楊應龍負固不服,執政貪其重餌,與之交通。如近日綦江捕獲奸人,得所投本兵及提督巡捕私書。其餘四緘及黃金五百、白金千、虎豹皮數十,不言所投。臣細詢播人,始囁嚅言曰『求票擬耳』。夫票擬,輔臣事也,而使小醜得以利動哉?可憂者二。推陞者,吏部職也。邇來創專擅之說以蠱惑聖聰,陛下入其言而疑之。於是內托上意,外諉廷推,或正或陪,惟意所欲。茍兩者俱無當,則駁令更推;少不如意,譴謫加焉。倘謂簡在帝心,非政府所預,何所用者非梓里姻親,則門牆密契也?如是而猶曰吏部專擅乎?可憂者三。言官天子耳目,糾繩獻納,其職也。邇來進朋黨之說以激聖怒,陛下納其譖而惡之。於是假托天威,肆行胸臆。非顯斥於建白之時,則陰中於遷除之日。倘謂斷自宸衷,無可挽救,何所斥者非宿昔積怨,則近日深仇也?如是而猶謂言官結黨乎?可憂者四。首輔志皋日薄西山,固無足責。位素負物望,乃所為若斯;且其機械獨深,朋邪日眾,將來之禍,更有難言者。請罷志皋而防位,嚴飭陳于陛、沈一貫,毋效二人所為。」疏入,忤旨。命鐫一級,出之外。志皋、位疏辨,且乞宥恂,於陛、一貫亦論救。乃以原品調陜西按察經歷。引疾歸。久之,吏部尚書蔡國珍奉詔起廢。及恂,未召卒。

冀體,武安人。被廢,累薦不起,卒於家。

其時以論志皋獲譴者又有硃爵,開州人。由茌平知縣召為吏科給事中。嘗論時政闕失,因疏志皋、位寢閣壅蔽罪,不報。尋切諫三王並封,且論救朱維京、王如堅等,復劾志皋、位私同年羅萬化為吏部。坐謫山西按察知事,卒於家。天啟中,贈太僕少卿。

姜士昌,字仲文,丹陽人。父寶,字廷善。嘉靖三十二年進士。官編修。不附嚴嵩,出為四川提學僉事。再轉福建提學副使,累遷南京國子監祭酒。請罷納粟例,復積分法,又請令公侯伯子弟及舉人盡入監肄業,詔皆從之。累官南京禮部尚書。嘗割田千畝以贍宗族。

士昌五歲受書,至「惟善為寶」,以父名輟讀拱立。師大奇之。舉萬曆八年進士,除戶部主事,進員外郎。請帝杜留中,錄遺直,舉召對,崇節儉。尋進郎中。以省親去。還朝,言吏部侍郎徐顯卿構陷張位,少詹事黃洪憲力擠趙用賢,宜黜之以警官邪;主事鄒元標、參政呂坤、副使李三才素著直讜,宜拔擢以厲士節。又請復連坐之法,慎巡撫之選,旌苦節之士,重贓吏之罰。疏入,給事中李春開劾其出位。遂下詔禁諸司毋越職刺舉。已,因風霾,請早建國本。貴妃父鄭承憲乞改造父塋,詔與五千金。士昌言:「太后兄陳昌言止五百金,而妃家乃十之,何以示天下?」弗納。稍遷陜西提學副使,江西參政。

三十四年,大學士沈一貫、沈鯉相繼去國。明年秋,士昌齎表入都,上疏曰:

皇上聽一貫、鯉並去,輿論無不快一貫而惜鯉。夫一貫招權罔利,大壞士風吏道,恐天下林居貞士與己齟齬,一切阻遏,以杜將來。即得罪張居正諸臣,皇上素知其忠義、注意拔擢者,皆擯不復用,甚則借他事處之。其直道左遷諸人、久經遷轉在告者,一貫亦擯不復用。在廷守正不阿、魁磊老成之彥,小有同異,亦巧計罷之。且空部院以便於擇所欲用,空言路以便於恣所欲為,空天下諸曹與部院、言路等,使人不疑。至於己所欲用所欲為者,又無不可置力而得志;所不欲者,輒流涕語人曰「吾力不能得之皇上」。善則歸己,過則歸君,人人知其不忠。

夫鯉不肥身家,不擇利便,惟以眾賢效之君,較一貫忠邪遠甚。一貫既歸,貨財如山,金玉堆積;鯉家徒壁立,貧無餘貲,較一貫貪廉遠甚。一貫患鯉邪正相形,借妖書事傾害,非皇上聖明,幾至大誤。臣以為輔臣若一貫憸邪異常,直合古今奸臣盧巳、章惇而三矣。然竟無一人以鯉、一貫之賢奸為皇上正言別白者,臣竊痛之。

且一貫之用,由王錫爵所推轂。今一貫去,以錫爵代首揆,是一貫未嘗去也。錫爵素有重名,非一貫比。然器量褊狹,嫉善如仇。高桂、趙南星、薛敷教、張納陛、于孔兼、高攀龍、孫繼有、安希範、譚一召、顧憲成、章嘉禎等一黜不復。頃聞錫爵有疏請錄遺佚。謂宜如其所請,召還諸臣,然後敦趣就道,不然,恐錫爵無復出理也。至論劾一貫諸臣,如劉元珍、龐時雍、陳嘉訓、朱吾弼,亦亟宜召復,以為盡忠發奸者之勸。至於他臣,以觸忤被中傷異同致罷去者,請皆以次拂拭用之。

說者謂皇上於諸臣,雖三下明詔,意若向用,實未欲用者,臣獨以為不然。皇上初嘗罷傅應禎、餘懋學、鄒元標、艾穆、沈思孝、吳中行、趙用賢、硃鴻謨、孟一脈、趙世卿、郭惟賢、王用汲等,後又嘗謫魏允貞、李三才、黃道瞻、譚希恩、周弘禴、江東之、李植、曾乾亨、馮景隆、馬應圖、王德新、顧憲成、李懋檜、董基、張鳴岡、饒伸、郭實、諸壽賢、顧允成、彭遵古、薛敷教、吳正志、王之棟等,旋皆擢用。頃年改調銓曹鄒觀光、劉學曾、李復陽、羅朝國、趙邦柱、洪文衡等於南京,亦俱漸還清秩。而鄒元標起自戍所,累蒙遷擢,其後未有一言忤主,而謂皇上忽復怒之,而調之南,而錮不復用,豈不厚誣皇上也哉。蓋皇上本無不用諸臣之心,而輔臣實決不用諸臣之策也。說者謂俗流世壞,宜用潔清之臣表率之。然古今廉相,獨推楊綰、杜黃裳,以其能推賢薦士耳。王安石亦有清名,乃用其學術驅斥諸賢,竟以禍宋。為輔臣者可不鑒於此哉。

其意以陰諷李廷機。廷機大恚,疏辨曰:「人才起用,臣等不惟不敢干至尊之權,亦何敢侵吏部職。」士昌見疏,復貽書規之,廷機益不悅,然帝尚未有意罪士昌也。會朱賡亦疏辨如廷機指,帝乃下士昌疏,命罪之。吏部侍郎楊時喬、副都御史詹沂請薄罰,不許。詔鐫三秩為廣西僉事。御史宋燾論救,復詆一貫,刺廷機。帝益怒,謫燾平定判官,再謫士昌興安典史。

士昌好學,勵名檢。居恆憤時疾俗,欲以身挽之。故雖居散僚,數有論建,竟齟齬以終。士昌謫之明年,禮部主事鄭振先劾賡等大罪十二,亦鐫三秩,調邊方用。

宋燾,泰安人。萬歷二十九年進士。自庶吉士授御史,任氣好搏擊。出按應天諸府,疏斥首輔硃賡。廷臣繼有請,皆責備輔臣,其端自燾發也。及坐謫,旋請假歸。卒於家。天啟初,贈士昌太常少卿,燾光祿少卿。

馬孟禎,字泰符,桐城人。萬歷二十六年進士。授分宜知縣。將內召,以征賦不及四分,為戶部尚書趙世卿所劾,詔鐫二秩。甫三日,而民逋悉完。鄒元標、萬國欽輩亟稱之。續授御史。文選郎王永光、儀制郎張嗣誠、都給事中姚文蔚、陳治則,以附政府擢京卿,南京右都御史沈子木年幾八十未謝政,孟禎並疏論之。大學士李廷機被劾奏辨,言入仕以來,初無大謬。孟禎駁之曰:「廷機在禮部暱邪妄司官彭遵古,而聶雲翰建言忤時,則抑之至死。秉政未幾,姜士昌、宋燾、鄭振先皆得罪。姚文蔚等濫授京堂,陳用賓等屢擬寬旨。猶不謂之謬哉?」王錫爵辭召,密疏痛詆言者。孟禎及南京給事中段然並上疏極論。尋陳僉商之害,發工部郎陳民志、範鈁黷貨罪。又陳通壅蔽、錄直臣、決用舍、恤民窮、急邊餉五事。請召用鄒元標、趙南星、王德完,放廷機還田里。皆不報。

三十九年夏,怡神殿災。孟禎言:「二十年來,郊廟、朝講、召對、面議俱廢,通下情者惟章奏。而疏入旨出悉由內侍,其徹御覽與果出聖意否,不得而知,此朝政可慮也。臣子分流別戶,入主出奴,愛憎由心,雌黃信口,流言蜚語,騰入禁庭,此士習可慮也。畿輔、山東、山西、河南,比歲旱饑。民間賣女鬻兒,食妻啖子,鋌而走險,急何能擇。一呼四應,則小盜合群,將為豪傑之藉,此民情可慮也。」帝亦不省。

吏部侍郎蕭雲舉佐京察,有所庇,孟禎首疏攻之。論者日眾,雲舉引去。山海參將李獲陽忤稅監,下獄死,孟禎為訟冤,因請貸卞孔時、王邦才、滿朝薦、李嗣善等之在獄者,且言:「楚宗一獄,死者已多,今被錮高墻者,誰非高皇帝子孫,乃令至是。」皆弗聽。四十二年冬,考選科道,中書舍人張光房、知縣趙運昌、張廷拱、曠鳴鸞、濮中玉,以言論忤時,抑不得與。孟禎不平,具疏論之。是時三黨勢張,忌孟禎讜直,出為廣東副使。移疾不赴。天啟初,起南京光祿少卿,召改太僕。以憂歸。魏忠賢得志,為御史王業浩所論,遂削籍。崇禎初,復官。

孟禎少貧。既通顯,家無贏資。惟銜趙世卿抑己,既入臺即疏劾世卿,人以為隘。

汪若霖,字時甫,光州人。父治,保定知府。若霖舉萬曆二十年進士,授行人。三十三年,擢戶科給事中。言「有司貪殘,率從輕論,非律;邊吏竭脂膏,外媚敵,內媚要津,而京軍十萬半虛冒,非計。」兵部尚書蕭大亨被劾求去,吏部議留,若霖力詆部議。雲南民變,殺稅使楊榮,詔從巡撫陳用賓言,命四川丘乘雲兼領。若霖言:「用賓養成榮惡,今不直請罷稅,而倡議領於四川,負國甚。乞亟斥用賓,追寢前命。」皆不報。

進禮科右給事中。自正月至四月不雨,若霖上疏曰:「臣稽《洪範傳》,言之不從,是謂不晙,厥罰恆暘。今郊廟宜親,朝會宜舉,東宮講習宜開,此下累言之,而上不從者也。又有上言之而中變者:稅務歸有司,權璫猶侵奪;起廢有明詔,啟事猶沉閣是也。有上屢言之而久不決、下數言之而上不斷者:中外大僚之推補,被劾諸臣之進退是也。凡此皆言不從之類。積鬱成災,天人恆理。陛下安得漠然而已哉!」時南京戶、工二部缺尚書,禮部缺侍郎,廷推故尚書徐元泰、貴州巡撫郭子章、故詹事範醇敬。若霖言:「三人不足任,且舉者不能無私。請自今廷推勿以一人主持,眾皆畫諾。宜籍舉主姓名,復祖宗連坐之法。」詔申飭如若霖言,所推悉報寢。兵部主事張汝霖,大學士朱賡婿也。典試山東,所取士有篇章不具者。若霖疏劾之,停其俸。中官楊致中枉法拷殺指揮鄭光擢,若霖率同官列其十罪,不報。朱賡獨相,朝事益弛。若霖言:「陛下獨相一賡,而又畫接無聞,補牘莫應,此最大患也。方今紀綱壞,政事壅,人才耗,庶職空,民力窮,邊方廢,宦豎橫,盜賊繁,士大夫幾忘廉恥禮義,而小民愁苦冤痛之聲徹於宇內。輔臣宜慨然任天下重,收拾人心,以效之當寧。如徒謙讓未遑,或以人言,輕懷去就,則陛下何賴焉?」賡乃緣若霖指,力請帝急行新政。帝亦不省。五月朔,大雨雹。若霖謂用人不廣,大臣專權之象,具疏切言之。已而京師久雨,壞田廬。若霖復言大臣比周相倚,小臣趨風,其流益甚;意復詆賡及新輔李廷機輩也。三十六年,巡視庫藏,見老庫止銀八萬,而外庫蕭然,諸邊軍餉逋至百餘萬。疏請集議長策,亦留中。

先是,吏部列上考選應授科道者,知縣新建汪元功、進賢黃汝亨、南昌黃一騰與焉。賡黨給事中陳治則推轂元功、汝亨。若霖劾二人囂競,吏部因改擬部曹。治則怒劾一騰交構。帝以言官紛爭,留部疏。廷臣屢請乃下,而責若霖首昌煩言,並元功、汝亨、一騰各貶一級,出之外。廷臣論救,皆不省。若霖遂出為潁州判官,卒。

贊曰:明至中葉以後,建言者分曹為朋,率視閣臣為進退。依阿取寵則與之比,反是則爭。比者不容於清議,而爭則名高。故其時端揆之地,遂為抨擊之叢,而國是淆矣。雖然,所言之是非,閣臣之賢否,黑白判然,固非私怨惡之所得而加,亦非可盡委之沽直好事,謂人言之不足恤也。


\end{pinyinscope}