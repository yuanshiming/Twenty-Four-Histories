\article{列傳第一百九十八 土司}

\begin{pinyinscope}
西南諸蠻,有虞氏之苗,商之鬼方,西漢之夜郎、靡莫、邛、莋、僰、爨之屬皆是也。自巴、夔以東及湖、湘、嶺嶠,盤踞數千里,種類殊別。歷代以來,自相君長。原其為王朝役使,自周武王時孟津大會,而庸、蜀、羌、髳、微、盧、彭、濮諸蠻皆與焉。及楚莊蹻王滇,而秦開五尺道,置吏,沿及漢武,置都尉縣屬,仍令自保,此即土官、土吏之所始歟。

迨有明踵元故事,大為恢拓,分別司郡州縣,額以賦役,聽我驅調,而法始備矣。然其道在於羈縻。彼大姓相擅,世積威約,而必假我爵祿,寵之名號,乃易為統攝,故奔走惟命。然調遣日繁,急而生變,恃功怙過,侵擾益深,故歷朝徵發,利害各半。其要在於撫綏得人,恩威兼濟,則得其死力而不足為患。《實錄》載成化十八年馬平主簿孔性善言:「谿峒蠻僚,雖常梗化,亂豈無因。昔陳景文為令,瑤、僮皆應差徭,厥后撫字乖方,始仍反側。誠使守令得人,示以恩信,諭以禍福,亦當革心。」帝嘉納之,惜未能實究其用,此可為治蠻之寶鑒矣。

嘗考洪武初,西南夷來歸者,即用原官授之。其土官銜號曰宣慰司,曰宣撫司,曰招討司,曰安撫司,曰長官司。以勞績之多寡,分尊卑之等差,而府州縣之名亦往往有之。襲替必奉朝命,雖在萬里外,皆赴闕受職。天順末,許土官繳呈勘奏,則威柄漸弛。成化中,令納粟備振,則規取日陋。孝宗雖發憤釐革,而因循未改。嘉靖九年始復舊制,以府州縣等官隸驗封,宣慰、招討等官隸武選。隸驗封者,布政司領之;隸武選者,都指揮領之。於是文武相維,比於中土矣。其間叛服不常,誅賞互見。茲據其事績尤著者,列於篇。

○湖廣土司

湖南,古巫郡、黔中地也。其施州衛與永、保諸土司境,介於岳、辰、常德之西,與川東巴、夔相接壤,南通黔陽。谿峒深阻,易於寇盜,元末滋甚。陳友諒據湖、湘間,啖以利,資其兵為用。諸苗亦為盡力,有乞兵旁寨為之驅使者,友諒以此益肆。及太祖殲友諒於鄱陽,進克武昌,湖南諸郡望風歸附,元時所置宣慰、安撫、長官司之屬,皆先後迎降。太祖以原官授之,已而梗化。

洪武三年,慈利安撫使覃垕連構諸蠻入寇,征南將軍周德興平之。五年,復命鄧愈為征南將軍,率師平散毛等三十六洞,而副將軍吳良復平五開、古州諸蠻凡二百二十三洞,籍其民一萬五千,收集潰散士卒四千五百餘人,平其地。未幾,五開、五谿諸蠻亂,討平之。十八年,五開蠻吳面兒反,勢獗甚。命楚王楨將征虜將軍湯和,擊斬九谿諸處蠻僚,俘獲四萬餘人,諸苗始懼。而靖、沅、道、澧之間,十年內亦尋起尋滅。雖開國之初,師武臣力,實太祖控制之道恩威備焉。

永樂初,苗告繼絕,襲冠帶,益就銜勒。垂百年,而五開、銅鼓間又紛紛多警。時英宗北狩,中原所在侵擾,苗勢殊熾。景泰初,總兵官宮聚奏:「蠻賊西至貴州龍里,東至湖廣沅州,北至武岡,南至播州之境,不下二十萬,圍困焚掠諸郡邑。臣所領官軍不及二萬,前後奔赴不能解平越之圍。乞急調京邊軍及徵麓川卒十萬前來,以資調遣。」久而師征不至,更易他帥,浸淫六七載。至天順元年,總督石璞調總兵官方瑛,始克期徵剿。破天堂、小坪、墨溪二百二十七寨,擒偽王侯伯等百餘人,斬賊首千四百餘級,奪回軍人男婦千三百餘口,於是苗患漸平。蓋萌發於貴州,而蔓銜於湖南,皆生苗為梗。諸土司初無動搖,而永、保諸宣慰,世席富強,每遇征伐,輒願荷戈前驅,國家亦賴以撻伐,故永、保兵號為虓雄。嘉、隆以還,徵符四出,而湖南土司均備臂指矣。

△施州施南宣撫司散毛宣撫司忠建宣撫司容美宣撫司永順軍民宣慰使司保靖州軍民宣慰使司

施州,隋為清江郡,改施州。明初仍之。洪武十四年改置施州衛軍民指揮使司,屬湖廣都司。領軍民千戶所一:曰大田。領宣撫司三:日施南,曰散毛,曰忠建。領安撫司八:曰東鄉五路,曰忠路,曰忠孝,曰金峒,曰龍潭,曰大旺,曰忠峒,曰高羅。領長官司七:曰搖把峒,曰上愛茶峒,曰下愛茶峒,曰劍南,曰木冊,曰鎮南,曰唐崖。領蠻夷長官司五:曰鎮遠,曰隆奉,曰西泙,曰東流,曰臘壁峒。又有容美宣撫司者,亦在境內,領長官司四:曰椒山瑪瑙,曰五峰石寶,曰石梁下峒,曰水盡源通塔平。

初,太祖即吳王位,甲辰六月,湖廣安定宣撫使向思明遣長官硬徹律等,以元所授宣撫敕印來上,請改授。乃命仍置安定等處宣撫司二,以思明及其弟思勝為之。又置懷德軍民宣撫司一,以向大旺為之,統軍元帥二,以南木、潘仲玉為之。抽攔、不用、黃石三洞,各置長官一,以沒葉、大蟲、硬徹律為之。簳坪洞設元帥府一,以向顯祖為之。梅梓、麻寮二洞,各置長官一,以向思明、唐漢明為之。皆新降者。丙午二月,,容美洞宣撫使田光寶遣弟光受等,以元所授宣撫敕印來上。命光寶為四川行省參政,行容美洞等處軍民宣撫司事,仍置安撫元帥治之。並立太平、臺宜、麻寮等十寨長官司。

洪武四年,宣寧侯曹良臣帥兵取桑植,容美洞元施南道宣慰使覃大勝弟大旺、副宣慰覃大興、光寶子答谷等皆來朝,納元所授金虎符。命以施州宣慰司為從三品,東鄉諸長官司為正六品,以流官參用。五年,忠建元帥墨池遣其子驢吾,率所部溪洞元帥阿巨等來歸附,納元所授金虎符并銀印、銅章、誥敕。置忠建長官司及沿邊溪洞長官司,以墨池等為長官。二月,容美宣撫田光寶復遣子答谷來朝。征南將軍鄧愈平散毛、柿谿、赤谿、安福等三十九峒,散毛宣慰司都元帥覃野旺上偽夏所授印。

十四年,江夏侯周德興移師討水盡源、通塔平、散毛諸峒,置施州衛軍民指揮使司。十五年,置施南宣撫司,隸施州衛。十七年,散毛、沿邊安撫司安撫覃野旺之子起刺來朝,命為本司僉事。景川侯曹震言:「散毛等洞蠻時寇掠為民患,已令施州衛及施南宣撫覃大勝招之,如負固,請發兵討。」

二十二年命忠建宣撫田思進之子忠孝代父職。時思進年八十餘,乞致仕,故有是命。明年,涼國公藍玉克散毛洞,擒刺惹長官覃大旺等萬餘人。置大田軍民千戶所,隸施州衛。以藍玉奏散毛、鎮南、大旺、施南等洞蠻叛服不常,黔江、施州衛兵相去遠,難應援。今散毛地與大水田連,宜置千戶所守御,乃改散毛為大田,命千戶石山等領土兵一千五百人,置所鎮之。時忠建、施南叛蠻結寨於龍孔,玉遣指揮徐玉將兵攻之,擒宣撫覃大勝,餘蠻退走。玉復分兵搜之,殺獲男女一千八百餘人,械大勝及其黨八百二十人送京師。磔大勝於市,餘戍開元,給衣糧遣之。

永樂二年復設散毛、施南二長官司。先是,洪武初,諸土司長官來降者,皆予原官。蠻苗吳面兒之難,諸土司地多荒廢,長官亦罷承襲。至是,故土官之子覃友諒等以招復蠻民,請仍設治所。以其戶少,降為長官司,隸大田軍民千戶所。以友諒為散毛,長官,覃添富為施南長官。四年,改施南、散毛仍為宣撫司,以友諒、添富來朝故也。以田應虎為龍潭安撫。時應虎來朝,言其祖父自宋、元來,俱為安撫,自蠻亂併其地入散毛隔遠難治,乞仍舊,從之。時高羅安撫田大民言,招復蠻民四百餘戶,乞還原職治所。木冊長官田谷佐、唐崖長官覃忠孝,並言父祖世為安撫,洪武時大軍平蜀,民驚潰,治所廢,今穀佐等招集三百餘戶,請襲,許之。五年,鎮南長官覃興等來朝,稱係世職,洪武中廢,今招來蠻民三百戶,乞仍舊,既五峰石寶長官張再武亦以襲職請,從之。同時,設東鄉五路安撫,以覃忠為之,隸施南。設石梁下峒、椒山瑪瑙、水盡源通塔平三長官司,以向潮文、劉再貴、唐思文為之,隸容美。既復設忠路、忠孝、金峒三安撫司,隸施州衛,以覃英、田大英、覃添貴為之。皆因洪武間蠻亂民散,廢其治,今忠等以故官子侄來朝,奏請復設,並從之,各賜印章冠帶。

宣德二年設劍南長官司,隸忠路安撫;搖把峒、上愛下愛二茶峒三長官司及鎮邊、隆奉二蠻夷官司,皆隸東鄉五路安撫;東流、臘壁峒二蠻夷官司,隸散毛宣撫;石關峒長官司、西泙蠻夷官司,隸金峒安撫。皆以其酋長為之。先是,忠路安撫司等各奏,前元故土官子孫牟酋蠻等,各擁蠻民,久據谿洞,今就招撫,請設官司,授以職事。兵部以聞,帝以馭蠻當順其情,所授諸司,宜有等殺。兵部議以四百戶以上者設長官司,四百戶以下者設蠻夷官司。元土官子孫量授以職,從所招官司管屬。皆從之。令三年一朝貢如故事。九年,木冊長官田谷佐奏:「高羅安撫常倚勢凌轢,侵奪其土地人民,已蒙朝廷分理,然彼宿怨未平,恐復加害。乞徑隸施州衛。」從之。正統三年命散毛宣撫覃友諒子瑄試職。初,友諒以罪械赴京,中路逃匿,後為官軍所獲,斃獄。至是,本司以其子為蠻民信服,乞襲職。帝以友諒罪重宜革,第以蠻故詘法信恩,命瑄試職圖後效。景泰二年,禮部奏:「散毛宣撫司副使黃縉瑄謀殺親兄,律應斬。其妻譚氏遣子忠等貢馬贖罪,然縉瑄罪重,法不可宥。宜給鈔以酬馬直。」從之。天順元年,容美宣撫田潮美老疾,請子保富代職,從之。五年,禮部奏:「施州木冊長官司土舍譚文壽兇暴,并造不法誹謗之言,罪當刑。今其母向氏進馬以贖,恐不可從。」帝命給鈔百錠以慰其母,其子仍禁錮之。

成化二年,搖把洞長官向麥答踵奏:「鄰近洗羅峒長,窺知本洞土兵調征兩廣,村寨空虛,煽誘土蠻攻劫,乞調官軍剿治。」五年,禮部奏:「容美宣撫司田保富等,遣人進貢方物不及數,恐使者侵盜,宜停其賞,仍移知所司。」施州等衛八安撫司各奏,成化五年朝覲進馬,已付邊衛騎操,而諸衛收馬文移不至,恐有虛詐,宜勘實給賞,皆從之。弘治二年,木冊長官田賢及容美致在田保富各進馬,為土人譚敬保等贖罪。刑部言:「蠻民納馬贖罪,輕者可原,重者難宥,宜下按臣察核。」八年,容美宣撫貢馬及香,禮部以香不及數,馬多道斃,又無文驗,命予半賞。九年,金峒安撫覃彥龍奏:「境內產杉木,嘗鬻金三千貯庫。今彥龍年老,子惟一人,恐身后土人爭奪,乞解部。」工部議非貢典,卻之。

正德四年,容美宣撫并椒山瑪瑙長官司所遣通事劉思朝等赴京進貢,沿途驛傳多需索,為偵事所發,自魯橋以北計千餘金。部臣以聞,帝以遠蠻宥之。散毛宣撫并五峰石寶、水盡源通塔平長官司入貢後期,部議半賞,從之。九年命大田千戶所冉霖彡子舜卿為指揮僉事,以自陳討川寇功也。十一年,容美宣撫田秀愛其幼子,將逐其兄白俚俾,而以幼子襲。白俚俾恨之,賊殺其父及其弟。事聞,下鎮巡官驗治,磔死。土官唐勝富、張世英等為白俚俾奏辨,罪亦當坐。詔以蠻僚異類,難盡繩以法,免其並坐,戒飭之。十五年,容美宣撫司同知田世瑛,奏獲鎮南軍民府古印,為始祖田始進開熙二年頒給,乞改陞宣撫司為軍民府。禮部議,以開設宣撫,頒印已久,不當更,古印宜繳,從之。

嘉靖七年,容美宣撫司、龍潭安撫司每朝貢率領千人,所過擾害,鳳陽巡撫唐龍以聞。禮部按舊制,進貢不過百人,赴京不過二十人,命所司申飭。忠孝安撫司把事田春者數十人稱入貢,偽造關文,騷擾驛傳,應天巡撫以聞。兵部議,土司違例入貢,且所過橫索,恐有他虞,宜嚴禁諭。二十六年,臘壁峒等長官司入貢,禮部驗印文詐偽,詔革其賞,並下按臣勘問。

三十三年詔湖廣川貴總督並節制容美十四司。初,容美土官田世爵與土官向元楫累世相仇。元楫幼,世爵佯為講好,以女嫁之,謀奪其產,因誣元楫以奸。有司恐激變,令自捕元楫,下獄論死。世爵遂發兵,盡俘向氏,並籍其土,皆沒入之。久之,撫按知其謀,責與元楫對狀,世爵不出,陰與羅峒土舍黃中等謀叛。於是湖廣巡按御史周如斗請移荊南道分巡施州衛,以便控制,調廣西清浪等戍軍,以實行伍。疏下督臣馮岳等議,岳等言:「施州地勢孤懸,不可久居,戍軍亦非一時可集。當移荊瞿守備於施州,九永守備於九谿,上荊南道備巡歷。至世爵驕橫,有司不能攝治,獨久系元楫何為。宜假督臣以節制容美之權,問世爵抗違之罪,如不悛,即繩以法。」從之。

時龍潭安撫黃俊素貪暴,據支羅洞寨,以睚爔殺人,繫獄。會白草番反,俊子中請立功為父貰罪,已又自求為副指揮,賄當事者許之。俊出益驕,乃與中及群盜李仲實等,恣行於四川之雲陽、奉節間,副使熊逵等計擒俊與仲實。俊死於獄,中自縛出降,執餘黨譚景雷等自贖。帝命追戮俊,梟示,仲實等論斬,中謫戍,而賞有功者。三十五年,命容美宣撫田九霄襲職,賜紅糸寧衣一襲,以浙江黃宗山擊倭之功也。

隆慶元年,吏科給事朱繪等言,湖廣施州衛忠路安撫覃大寧一日奏五上,語多不實,請究治。都察院議,金峒安撫上舍覃璧爭印相殺,及磁峒不當轄四川。俱下撫按官勘報。四年,覃璧作亂,傷官軍,撫按請治失事諸臣罪。兵部言:「本衛孤懸境外,事起倉猝,宜從寬貰,以責後功。」帝然之,命所司相機剿撫。五年,巡撫劉愨以覃璧平,條議五事:「一,請以川東所轄巫山、建始、黔江、萬縣改屬上荊道。一,以荊州去施州衛遠,不便巡歷。夷陵西有傅友德所闢取蜀故道,名百里荒者,抵衛僅五百餘里。請以巴東之石砫司巡檢、施州衛之州門驛、三會驛並移近地,俾閭井聯絡。而於百里荒及東卜壟仍創建哨堡,令千戶一員,督班軍百人戍守。一,施州衛延袤頗廣,物產最饒,衛官朘削,致民逃夷地為亂。宜裁通判設同知,撫治民蠻,均平徭賦,勿額外橫索。一,金峒世官不宜遽絕,貸覃勝罪,降安撫為峒長,聽支羅所百戶提調。一,施州所轄十四司應襲官舍,必先白道院,始許理事。其擅立名號者,請嚴治,并令兵巡道每歲經歷施州,豫行調集各官舍獎諭,令赴學觀化。」俱從之。

萬曆十一年,湖廣撫按奏:「施州衛施南等宣撫司各官,仍聽鎮筸參將節制,載入敕書,以一事權。」從之。

崇禎十二年,容美宣撫田元疏言:「六月間,穀賊復叛,撫治兩臣調用土兵。臣即捐行糧戰馬,立遣土兵七千,令副長官陳一聖等將之前行。悍軍鄧維昌等憚於徵調,遂與譚正賓結七十二村,鳩銀萬七千兩,賂巴東知縣蔡文升以逼民從軍之文上報,阻忠義而啟邊釁。」帝命撫按核其事。時中原寇盜充斥,時事日非,即土司徵調不至,亦不能問矣。

永順,漢武陵、隋辰州、唐溪州地也。宋初為永順州。嘉祐中,溪州刺史彭仕羲叛,臨以大兵,仕羲降。熙寧中,築下溪州城,賜名會溪。元時,彭萬潛自改為永順等處軍民安撫司。洪武五年,永順宣慰使順德汪倫、堂厓安撫使月直遣人上其所受偽夏印,詔賜文綺襲衣。遂置永順等處軍民宣慰使司,隸湖廣都指揮使司。領州三,曰南渭,曰施溶,曰上谿;長官司六,曰臘惹洞,曰麥著黃洞,曰驢遲洞,曰施溶溪,曰白崖洞,曰田家洞。九年,永順宣慰彭添保遣其弟義保等貢馬及方物,賜衣幣有差。自是,每三年一入貢。永樂十六年,宣慰彭源之仲率土官部長六百六十七人貢馬。

宣德元年,禮部以永順宣慰彭仲子英朝正後期,請罪之。帝以遠人不無風濤疾病之阻,仍賜予如例。總兵官蕭綬奏:「西陽宋農里、石提洞軍民被臘惹洞長謀古賞等連年攻劫,又及後溪,招之不從,乞調兵剿之。」謀古賞等懼,願罰人馬贖罪,乃罷兵。正統元年命彭仲子世雄襲職。天順二年諭世雄調士兵會剿貴州東苗。

成化三年,兵部尚書程信請調永順兵征都掌蠻。十三年以征苗功,命宣慰彭顯英進散官一階,仍賜敕獎勞。十五年免永順賦。弘治七年,貴州奏平苗功,以宣慰彭世麒等與有勞,世麒乞升職。兵部言非例,請進世麒階昭勇將軍,仍賜敕褒獎,從之。八年,世麒進馬謝恩。十四年,世麒以北邊有警,請帥土兵一萬赴延綏助討賊。兵部議不可,賜敕獎諭,並賜奏事人路費鈔千貫,免其明年朝覲,以方聽調徵賊婦米魯故也。

正德元年以世麒從征有功,賜紅織金麒麟服,世麒進馬謝恩。二年進馬賀立中宮,命給賞如例。五年,永順與保靖爭地相攻,累年不決,訴於朝,命各罰米三百石。六年,四川賊藍廷瑞、鄢本恕等及其黨二十八人倡亂兩川,鳥合十餘萬人,僭王號,置四十八營,攻城殺吏,流毒黔、楚。總制尚書洪鐘等討之,不克。已而為官軍所遏,乏食,乃佯聽撫,劫掠自如。廷瑞以女結婚於永順土舍彭世麟,冀緩兵。世麟偽許之,因與約期。廷瑞、本恕及王金珠等二十八人皆來會,世麟伏兵擒之,餘賊潰渡河,官兵追圍之,擒斬及溺死者七百餘人。總制、巡撫以捷聞,獎賚有差,論者以是役世麟為首功云。七年,賊劉三等自遂平趨東皋,宣慰彭明輔及都指揮曹鵬等以土軍追擊之,賊倉卒渡河,溺死者二千人,斬首八十餘級。巡撫李士實以聞。命永順宣慰格外加賞,仍給明輔誥命。

十年,致仕宣慰彭世麒獻大木三十,次者二百,親督運至京,子明輔所進如之。賜敕褒諭,賞進奏人鈔千貫。十三年,世麒獻大楠本四百七十,子明輔亦進大木備營建。詔世麒陞都指揮使,賞蟒衣三襲,仍致仕;明輔授正三品散官,賞飛魚服三襲,賜敕獎勵,仍令鎮巡官宴勞之。時政出權倖,恩澤皆由於干請。於是郴州民頌世麒征賊時號令嚴明,其土官彭芳等亦頌世麒功,乞蟒衣玉帶。兵部格不可,乃已。世麒辭賞,請立坊,賜名曰表勞。會有保靖兩宣慰爭兩江口之議,詞連明輔,主者議逮治。明輔乃令蠻民奏其從征功,悉辭香爐山應得升賞,以贖逮治之辱。部議悉已之。

嘉靖六年,論擒岑猛功,免應襲宣慰彭宗漢赴京,而加宗漢父明輔、祖世麒銀幣。二十一年,巡撫陸傑言:「酉陽與永順以採木仇殺,保靖又煽惑其間,大為地方患。」乃命川、湖撫臣撫戢,勿釀兵端。是年,免永順秋糧。

三十三年冬,調永順土兵協剿倭賊於蘇、松。明年,永順宣慰彭翼南統兵三千,致仕宣慰彭明輔統兵二千,俱會於松江。時保靖兵敗賊於石塘灣。永順兵邀擊,賊奔王江涇,大潰。保靖兵最,永順次之,帝降敕獎勵,各賜銀幣,翼南賜三品服。

先是,永順兵剿新場倭,倭故不出,保靖兵為所誘遽先入,永順土官田菑、田豐等亦爭入,為賊所圍,皆死之。議者皆言督撫經略失宜,致永順兵再戰再北。及王江涇之戰,保靖掎之,永順角之,斬獲一千九百餘級,倭為奪氣,蓋東南戰功第一云。時邀功者方行賞,翼南遂授昭毅將軍。已,升右參政管宣慰事,與明輔俱受銀幣之賜。時保、永二宣慰破倭後,兵驕,所過皆劫掠,緣江上下苦之。御史請究治,部議以土兵新有功,遽加罰,失遠人心,宜諭責之。並令浙、直練鄉勇,嗣後不得輕調土兵。

四十二年以獻大木功再論賞,加明輔都指揮使,賜蟒衣,其子掌宣慰司事,右參政彭翼南為右布政使,賜飛魚服,仍賜敕獎勵。四十四年,永順復獻大木,詔加明輔、翼南二品服。

萬曆二十五年,東事棘,調永順兵萬人赴援。宣慰彭元錦請自備衣糧聽調,既而支吾,有要挾之跡,命罷之。三十八年賜元錦都指揮銜,給蟒衣一襲,妻汪氏封夫人。四十七年,永順貢馬後期,減賞。兵部言:「前調宣慰元錦兵三千援遼,已半載,至關者僅七百餘人。」命究主兵者。四十八年進元錦都督僉事。先是,元錦以調兵三千為不足立功,願以萬兵往。朝廷嘉其忠,加恩優渥。既而檄調八千,僅以三千,塞責,又上疏稱病,為巡撫所劾,得旨切責。元錦不得已行,兵抵通州北,聞三路敗恤,遂大潰。於是巡撫徐兆魁言:「調永順兵八千,費逾十萬,今奔潰,虛糜無益。」罷之。

保靖,唐溪州地,宋置保靜州,元為保靖州安撫司。明太祖之初起也,安撫使彭世雄率其屬歸附,命仍為保靖安撫使。洪武元年,保靖安撫使彭萬里遣子德勝奉表貢馬及方物,詔升安撫司為保靖宣慰司,以萬里為之,隸湖廣都指揮使司。自是,朝貢如制。

永樂元年以保靖族屬大蟲可宜等互仇殺,遣御史劉從政齎敕撫諭之。三年,辰州衛指揮龔能等招諭筸子坪等三十五寨生苗廖彪等,各遣子入貢,因設筸子坪長官司,以彪為之,隸保靖。九年,宣慰彭勇烈遣人來貢。十二年,筸子坪賊吳者泥自稱苗玉,與蠻民苗金龍等為亂,總兵梁福平之。未幾,者泥子吳擔竹復誘苗吳亞麻糾貴州答意諸蠻叛,都督蕭授斬平之。二十一年,宣慰彭藥哈俾遣人貢馬。

宣德元年,宣慰彭大蟲可宜遣子順來貢。四年,兵部奏:「保靖舊有二宣慰,一為人所殺,一以殺人當死,其同知以下官皆缺,請改流官治之。」帝以蠻性難馴,流官不諳土俗,令都督蕭授擇眾所推服者以聞。正統十四年,保靖宣慰與族人彭南木答等相訐奏,既而講和,願輸米贖誣奏罪,從之。

景泰七年命調保靖土兵協剿銅鼓、五開、黎平諸蠻,先頒賞犒之。天順二年敕宣慰彭捨怕俾即選兵進討。三年,保靖奏夏災。成化二年,以保靖宣慰彭顯宗征蠻有功,命給誥命。三年復調保靖兵征都掌蠻。五年免保靖宣慰諸土司成化二年稅糧八百五十三石,以屢調徵廣西及荊、襄、貴州有功也。七年,顯宗老不任事,命其子仕瓏代。十三年,以平苗功,顯宗、仕瓏皆進一階。十五年以災免保靖租賦。仁瓏奏,兩江口長官彭勝祖違例進貢,下部臣議,宜逮問,命鎮巡官諭之。

弘治十二年,永順宣慰司奏,仕瓏擅率兵攻長官彭世英,仇殺多年,構禍不已,乞發兵征剿。部覆以屢行按問不報,宜諭鎮巡官速勘奏聞,從之。十四年,以保靖宣慰等方聽調,免明年朝覲,時有徵貴州賊婦米魯之役故也。初,保靖安撫彭萬里以洪武元年歸附,即其地設保靖宣慰司,授萬里宣慰使,領白崖、大別、大江、小江等二十八村寨。萬里卒,子勇烈嗣。勇烈卒,子藥哈俾嗣,年幼。萬里弟麥穀踵之子大蟲可宜,諷土人奏己為副宣慰,同理司事,因殺藥哈俾而據其十四寨。事覺,逮問,死獄中,革副宣慰,而所據寨如故。其後,勇烈之弟勇傑嗣,傳子南木杵,孫顯宗,曾孫仕瓏;與大蟲可宜之子忠,忠子武,武子勝祖及其子世英,代為仇敵。而武以正統中隨征有功,授兩江口長官,勝祖成化中亦以功授前職,並隨司理事,無印署。弘治初,勝祖以年老,世英無官,恐仕瓏奪其地,援例求世襲,奏行核實,仕瓏輒沮之,以是仇恨益甚,兩家所轄土人亦各分黨仇殺。永順宣慰使彭世麒取勝祖女,復左右之,以是互相攻擊,奏訴無寧歲。弘治十年,巡撫沈暉奏言,令世英入粟嗣父職,將以平之,而仕瓏奏訐不止。是時,敕調世英從徵貴州,而兵部移文有「兩江口長官司」字,仕瓏疑世英得設官署,將不聽約束,復奏言之。於是巡撫閻仲宇、巡按王約等請以前後章奏下兵部、都察院,議:「令世英歸所據小江七寨於仕瓏,止領大江七寨,聽仕瓏約束。其原居兩江口系襟喉要地,請調清水溪堡官兵守之。而徙世英於沱埠,以絕爭端。以後土官應襲子弟,悉令入學,漸染風化,以格頑冥。如不入學者,不準承襲。世麒黨於世英,法當治,但從征湖廣頗效忠勤,已有旨許以功贖。仕瓏、世英並逮問,勝祖照常例發遣。」奏上,從之。弘治十六年六月事也。

正德十四年,保靖兩江口土舍彭惠既以祖大蟲可宜與彭藥哈俾世仇,至是與宣慰彭九霄復構怨。永順宣慰彭明輔與之連姻,助以兵力,遂與九霄往復仇殺,數年不息,死者五百餘人,前後訐奏累八十餘章。守巡官系惠於獄,明輔率眾劫之去,尋復捕系。事聞,詔都御史吳廷舉勘處。廷舉乃令鎮巡議,以為惠罪當誅,但土蠻難盡以法繩,宜徙惠置辰、常城中,令九霄出價以易兩江口故地。仍用文官左遷者二人為首領官,以勸相之。俟數年後革心向化,請敕獎諭,仍擢用為首領。下兵部議,以惠徙內地,恐貽後患,令廷舉再議。於是廷舉等復請以大江之右五寨歸保靖,大江之左二寨屬辰州,設大刺巡檢司,流官一人主之。惠免遷徙,仍居沱埠,以土舍名目協理巡檢事。部覆如廷舉言。

嘉靖六年以擒岑猛功進九霄湖廣參政,賜銀幣。長子虎臣戰歿,贈指揮僉事,次子良臣襲職時,免赴京。二十六年免保靖秋糧。三十三年詔調宣慰彭藎臣帥所部三千人赴蘇、松征倭。明年遇倭於石塘彎,大戰,敗之。賊北走平望,諸軍尾之於王江涇,大破之。錄功,以保靖為首,敕賜藎臣銀幣並三品服,令統兵益擊賊。先是,都司李經率保靖兵追倭至新場,倭二千人伏不出,保靖土舍彭翅引軍探之,中伏,與所部皆死,贈翅一官並賜棺殮具。及是,以王江涇捷,進藎臣為昭毅將軍。既又調保靖土兵六千赴總督軍前,從胡宗憲請也。時已敘趙文華、宗憲功,復加藎臣右參政,管宣慰司事,仍賞銀幣。

萬歷四十七年調保靖兵五千,命宣慰彭象乾親統援遼。四十八年加象乾指揮使。象乾至涿州病,中夜兵逃散者三千餘人,部臣以聞。帝嚴旨責統兵者,並敕監軍道沿途招撫。明年,象乾病不能行,遣其子侄率親兵出關,戰於渾河,全軍皆歿。天啟二年進象乾都督僉事,贈彭象周、彭緄、彭天祐各都司僉書,以渾河之役一門殉戰,義烈為諸土司冠云。


\end{pinyinscope}