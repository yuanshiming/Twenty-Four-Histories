\article{列傳第三十三}

\begin{pinyinscope}
姚廣孝張玉(子輗軏從子信)朱能邱福李遠王忠王聰火真譚淵王真陳亨子懋徐理房寬劉才

姚廣孝,長洲人,本醫家子。年十四,度為僧,名道衍,字斯道。事道士席應真,得其陰陽術數之學。嘗游嵩山寺,相者袁珙見之曰:「是何異僧!目三角,形如病虎,性必嗜殺,劉秉忠流也。」道衍大喜。

洪武中,詔通儒書僧試禮部。不受官,賜僧服還。經北固山,賦詩懷古。其儕宗泐曰:「此豈釋子語耶?」道衍笑不答。高皇后崩,太祖選高僧侍諸王,為誦經薦福。宗泐時為左善世,舉道衍。燕王與語甚合,請以從。至北平,住持慶壽寺。出入府中,跡甚密,時時屏人語。及太祖崩,惠帝立,以次削奪諸王。周、湘、代、齊、岷相繼得罪。道衍遂密勸成祖舉兵。成祖曰:「民心向彼,奈何?」道衍曰:「臣知天道,何論民心。」乃進袁珙及卜者金忠。於是成祖意益決。陰選將校,勾軍卒,收材勇異能之士。燕邸,故元宮也,深邃。道衍練兵後苑中。穴地作重屋,繚以厚垣,密甃翎甋瓶缶,日夜鑄軍器,畜鵝鴨亂其聲。建文元年六月,燕府護衛百戶倪諒上變。詔逮府中官屬。都指揮張信輸誠於成祖,成祖遂決策起兵。適大風雨至,簷瓦墮地,成祖色變。道衍曰:「祥也。飛龍在天,從以風雨。瓦墮,將易黃也。」兵起,以誅齊泰、黃子澄為名,號其眾曰「靖難之師。」道衍輔世子居守。其年十月,成祖襲大寧,李景隆乘間圍北平。道衍守禦甚固,擊卻攻者。夜縋壯士擊傷南兵。援師至,內外合擊,斬首無算。景隆、平安等先後敗遁。成祖圍濟南三月,不克。道衍馳書曰:「師老矣,請班師。」乃還。復攻東昌,戰敗,亡大將張玉,復還。成祖意欲稍休,道衍力趣之。益募勇士,敗盛庸,破房昭西水寨。道衍語成祖:「毋下城邑,疾趨京師。京師單弱,勢必舉。」從之。遂連敗諸將於淝河、靈璧,渡江入京師。

成祖即帝位,授道衍僧錄司左善世。帝在籓邸,所接皆武人,獨道衍定策起兵。及帝轉戰山東、河北,在軍三年,或旋或否,戰守機事皆決於道衍。道衍未嘗臨戰陣,然帝用兵有天下,道衍力為多,論功以為第一。永樂二年四月,拜資善大夫、太子少師。復其姓,賜名廣孝,贈祖父如其官。帝與語,呼少師而不名。命蓄髮,不肯。賜第及兩宮人,皆不受。常居僧寺,冠帶而朝,退仍緇衣。。出振蘇、湖。至長洲,以所賜金帛散宗族鄉人。重修《太祖實錄》,廣孝為監修。又與解縉等纂修《永樂大典》。書成,帝褒美之。帝往來兩都、出塞北征,廣孝皆留輔太子於南京。五年四月,皇長孫出閣就學,廣孝侍說書。

十六年三月,入觀,年八十有四矣,病甚,不能朝,仍居慶壽寺。車駕臨視者再,語甚懽,賜以金睡壺。問所慾言,廣孝曰:「僧溥洽繫久,願赦之。」溥洽者,建文帝主錄僧也。初,帝入南京,有言建文帝為僧遁去,溥洽知狀,或言匿溥洽所。帝乃以他事禁溥洽。而命給事中胡濙等遍物色建文帝,久之不可得。溥洽坐繫十餘年。至是,帝以廣孝言,即命出之。廣孝頓首謝。尋卒。帝震悼,輟視朝二日,命有司治喪,以僧禮葬。追贈推誠輔國協謀宣力文臣、特進榮祿大夫、上柱國、榮國公,謚恭靖。賜葬房山縣東北。帝親製神道碑誌其功。官其養子繼尚寶少卿。

廣孝少好學,工詩。與王賓、高啟、楊孟載友善。宋濂、蘇伯衡亦推獎之。晚著《道餘錄》,頗毀先儒,識者鄙焉。其至長洲,候同產姊,姊不納。訪其友王賓,賓亦不見,但遙語曰:「和尚誤矣,和尚誤矣。」復往見姊,姊詈之。廣孝惘然。

洪熙元年,加贈少師,配享成祖廟庭。嘉靖九年,世宗諭閣臣曰:「姚廣孝佐命嗣興,勞烈具有。顧係釋氏之徒,班諸功臣,侑食太廟,恐不足尊敬祖宗。」於是尚書李時偕大學士張璁、桂萼等議請移祀大興隆寺,太常春秋致祭。詔曰:「可」。

張玉,字世美,祥符人。仕元為樞密知院。元亡,從走漠北。洪武十八年來歸。從大軍出塞,至捕魚兒海,以功授濟南衛副千戶,遷安慶衛指揮僉事。又從征遠順、散毛諸洞。北逐元人之擾邊者,至鴉寒山還,調燕山左護衛。從燕王出塞,至黑松林。又從征野人諸部。以驍果善謀畫為王所親任。

建文元年,成祖起兵。玉帥眾奪北平九門,撫諭城內外,三日而定。師將南,玉獻計,遣硃能東攻薊州,殺馬宣,降遵化。分兵下永平、密雲,皆致其精甲以益師。擢都指揮僉事。是時朝廷遣大兵討燕:都督徐凱軍河間;潘忠、楊松軍鄚州;長興侯耿炳文以三十萬眾軍真定。玉進說曰:「潘、楊勇而無謀,可襲而俘也。」成祖命玉將親兵為前鋒,抵樓桑。值中秋,南軍方宴會。夜半,疾馳破雄縣。忠、松來援,邀擊於月漾橋,生擒之。遂克鄚州。自以輕騎覘炳文軍。還言:「軍無紀律,其上有敗氣,宜急擊。」成祖遂引兵西,至無極,顧諸將謀所向。諸將以南軍盛,請屯新樂。玉曰:「彼雖眾,皆新集。我軍乘勝徑趨真定,破之必矣。」成祖喜曰:「吾倚玉足濟大事!」明日抵真定,大破炳文軍,獲副將李堅、甯忠,都督顧成等,斬首三萬。復敗安陸侯吳傑軍。燕兵由是大振。

江陰侯吳高以遼東兵圍永平。曹國公李景隆引數十萬眾將攻北平。成祖與玉謀,先援永平。至則高遁走,玉追斬甚眾。遂從間道襲大寧,拔其眾而還,次會州。初立五軍,以玉將中軍。時李景隆已圍北平,成祖旋師,大戰於鄭村壩,景隆敗。成祖乘勝抵城下。城中兵鼓噪出,內外夾攻,南軍大潰。

明年從攻廣昌、蔚州、大同。諜報景隆收潰卒,號百萬,且復至。玉曰:「兵貴神速,請先據白溝河,以逸待勞。」駐河上三日,景隆至。以精騎馳擊,復大敗之。進拔德州,追奔至濟南,圍其城三月,解圍還。尋再出,破滄州,擒徐凱。進攻東昌,與盛庸軍遇。成祖以數十騎繞出其後。庸圍之數重,成祖奮擊得出。玉不知成祖所在,突入陣中力戰,格殺數十人,被創死。年五十八。

燕兵起,轉鬥三年,鋒銳甚。至是失大將,一軍奪氣。師還北平,諸將叩頭請罪。成祖曰:「勝負常事,不足計,恨失玉耳。艱難之際,失吾良輔。」因泣下不能止,諸將皆泣。其後譚淵沒於夾河,王真沒於淝河,雖悼惜,不如玉也。建文四年六月,成祖稱帝,贈玉都指揮同知。九月甲申,追贈榮國公,謚忠顯。洪熙元年三月,加封河間王,改謚忠武,與東平王朱能、金鄉侯王真、榮國公姚廣孝並侑享成祖廟廷。

子三人,長輔,次輗。次軏,從子信。輔自有傳。

輗,以功臣子為神策衛指揮使。正統五年,英國公輔訴輗毆守墳者,斥及先臣,詞多悖慢。帝命錦衣衛鞫實,錮之,尋釋。三遷至中府右都督,領宿衛。景泰三年加太子太保。英宗復位,以軏迎立功,并封輗文安伯,食祿千二百石。天順六年卒。贈侯,謚忠僖。子斌嗣,坐詛咒,奪爵。

軏,永樂中入宿衛,為錦衣衛指揮僉事。從宣宗征高煦,又從成國公朱勇出塞至帽山。正統十三年,以副總兵征麓川。還,討貴州叛苗。積功為前府右都督,總京營兵。景泰二年,坐驕淫不道下獄,尋釋。景帝不豫,與石亨、曹吉祥迎上皇於南城。封太平侯,食祿二千石。于謙、王文、范廣之死,軏有力焉。納賄亂政,亞于亨。天順二年卒,贈裕國公,謚勇襄。子瑾嗣。成化元年,革「奪門」功,奪侯,授指揮使。

信,舉建文二年鄉試第一。永樂中,歷刑科都給事中,數言事。擢工部右侍郎。奉命視開封決河,請疏魚王口至中灤故道二十餘里。詔如其議,詳《宋禮傳》。出治浙江海塘,坐事謫交阯。洪熙初,召為兵部左侍郎。帝嘗謂英國公輔:「有兄弟可加恩者乎?」輔頓首言:「輗、軏蒙上恩,備近侍,然皆奢侈。獨從兄侍郎信賢,可使也。」帝召見信曰:「是英國公兄耶?」趣武冠冠之,改錦衣衛指揮同知,世襲。時去開國未遠,武階重故也。居職以平恕稱。宣德六年遷四川都指揮僉事。在蜀十五年致仕。

朱能,字士弘,懷遠人。父亮,從太祖渡江,積功至燕山護衛副千戶。能嗣職,事成祖籓邸。嘗從北征,降元太尉乃兒不花。

燕兵起,與張玉首謀殺張昺、謝貴,奪九門。授指揮同知。帥眾拔薊州,殺馬宣,下遵化。從破雄縣,戰月漾橋,執楊松、潘忠,降其眾於鄚州。長驅至真定,大敗耿炳文軍。獨與敢死士三十騎追奔至滹沱河,躍馬大呼突南軍,軍數萬人皆披靡,蹂藉死者甚眾,降三千餘人。成祖以手札勞之,進都指揮僉事。從援永平,走吳高,襲克大寧。還,將左軍,破李景隆於鄭村壩。從攻廣昌、蔚州、大同,戰白溝河,為前鋒,再敗平安軍。進攻濟南,次鏵山。南軍乘高而陣,能以奇兵繞其後,襲破之,降萬餘人。從攻滄州,破東門入,斬首萬餘級。東昌之戰,盛庸、鐵鉉圍成祖數重。張玉戰死。事急,能帥周長等殊死鬥,翼成祖潰圍出。復從戰夾河,譚淵死,燕師挫。能至,再戰再捷,軍復振。與平安戰槁城,敗之。追奔至真定,略地彰德、定州,破西水寨。將輕騎千人掠衡水,獲指揮賈榮。克東阿、東平,盡破汶上諸寨。既而王真戰死淝河,燕軍屢敗。諸將議旋師,能獨按劍曰:「漢高十戰九敗,終有天下。今舉事連得勝。小挫輒歸,更能北面事人耶!」成祖亦叱諸將曰:「任公等所之!」諸將乃不敢言。遂引兵南,敗平安銀牌軍。都督陳暉來援,又敗之。遂拔靈璧軍,擒平安等,降十萬眾。累遷右軍都督僉事。進克泗州,渡淮,敗盛庸兵。拔盱眙,下揚州,渡江,入金川門。

九月甲申論功,次邱福。授奉天靖難推誠宣力武臣、特進榮祿大夫、右柱國、左軍都督府左都督,封成國公,祿二千二百石,與世券。永樂二年兼太子太傅,加祿千石。四年七月詔能佩征夷將軍印,西平侯沐晟為左副將軍,由廣西、雲南分道討安南,帝親送之龍江。十月行次龍州,卒於軍。年三十七。

能於諸將中年最少,善戰,張玉善謀,帝倚為左右手。玉歿後,軍中進止悉諮能。能身長八尺。雄毅開豁,居家孝友。位列上公,未嘗以富貴驕人。善撫士卒。卒之日,將校皆為流涕。敕葬昌平,追封東平王,謚武烈。洪熙時,配享成祖廟廷。

子勇嗣,以元勳子特見任用。歷掌都督府事,留守南京。永樂二十二年從北征。宣宗即位,從平漢庶人,徵兀良哈。張輔解兵柄,詔以勇代。勇以南北諸衛所軍備邊轉運,錯互非便,請專令南軍轉運,北軍備邊。又言:「京軍多遠戍,非居重馭輕之道。請選精兵十萬益之。」又請令公、侯、伯、都督子弟操練。皆報可。正統九年出喜峰口,擊朵顏諸部,至富峪川而還,為兵部尚書徐晞所劾。詔不問。尋論功,加太保。

勇赬面虯鬚,狀貌甚偉,勇略不足,而敬禮士大夫。十四年從駕至土木,迎戰鷂兒嶺,中伏死,所帥五萬騎皆沒。于謙等追論勇罪,奪封。景泰元年,勇子儀乞葬祭。帝以勇大將,喪師辱國,致陷乘輿,不許。已,請襲。禮部尚書胡濙主之,又以立東宮恩得嗣,減歲祿至千石。天順初,追封勇平陰王,謚武愍。儀及子輔皆守備南京。

又三傳至希忠,從世宗幸承天,掌行在左府事。至衛輝,行宮夜火,希忠與都督陸炳翼帝出,由是被恩遇,入直西苑。歷掌後、右兩府,總神機營,提督十二團營及五軍營,累加太師,益歲祿七百石。代郊天者三十九,賞賚不可勝紀。卒,追封定襄王,謚恭靖。萬曆十一年,以給事中餘懋學言,追奪王爵。弟希孝亦至都督,加太保。卒,贈太傅,謚忠僖。

希忠五傳至曾孫純臣,崇禎時見倚任。李自成薄京師,帝手刺純臣總督中外諸軍,輔太子。敕未下,城已陷,為賊所殺。

邱福,鳳陽人。起卒伍,事成祖籓邸。積年勞,授燕山中護衛千戶。燕師起,與硃能、張玉首奪九門。大戰真定,突入子城。戰白溝河,以勁卒搗中堅。夾河、滄州、靈璧諸大戰,皆為軍鋒。盛庸兵扼淮,戰艦數千艘蔽淮岸。福與朱能將數百人,西行二十里,自上流潛濟,猝薄南軍。庸驚走,盡奪其戰艦,軍乃得渡。累遷至中軍都督同知。

福為人樸戇鷙勇,謀畫智計不如玉,敢戰深入與能埒。每戰勝,諸將爭前效虜獲,福獨後。成祖每歎曰:「丘將軍功,我自知之。」即位,大封功臣,第福為首。授奉天靖難推誠宣力武臣、特進榮祿大夫、右柱國、中軍都督府左都督,封淇國公,祿二千五百石,與世券。命議諸功臣封賞,每奉命議政,皆首福。

漢王高煦數將兵有功,成祖愛之。福武人,與之善,數勸立為太子。帝猶豫久之,竟立仁宗。以福為太子太師。六年加歲祿千石。尋命與蹇義、金忠等輔導皇長孫。明年七月將大軍出塞,至臚朐河,敗沒。

先是,本雅失里殺使臣郭驥,帝大怒,發兵討之。命福佩征虜大將軍印,充總兵官。武城侯王聰、同安侯火真為左、右副將,靖安侯王忠、安平侯李遠為左、右參將,以十萬騎行。帝慮福輕敵,諭以:「兵事須慎重。自開平以北,即不見寇。宜時時如對敵,相機進止,不可執一。一舉未捷,俟再舉。」已行,又連賜敕,謂軍中有言敵易取者,慎勿信之。福出塞,帥千餘人先至臚朐河南。遇遊騎,擊敗之,遂渡河。獲其尚書一人,飲之酒,問本雅失里所在。尚書言:「聞大兵來,惶恐北走,去此可三十里。」福大喜曰:「當疾馳擒之。」諸將請俟諸軍集,偵虛實而後進。福不從。以尚書為鄉導,直薄敵營。戰二日,每戰,敵輒佯敗引去,福銳意乘之。李遠諫曰:「將軍輕信敵閒,懸軍轉鬥。敵示弱誘我深入,進必不利,退則懼為所乘,獨可結營自固。晝揚旂伐鼓,出奇兵與挑戰;夜多燃炬鳴砲,張軍勢,使彼莫測。俟我軍畢至,併力攻之,必捷。否,亦可全師而還。始上與將軍言何如,而遂忘之乎?」王聰亦力言不可。福皆不聽,厲聲曰:「違命者斬!」即先馳,麾士卒隨行。控馬者皆泣下。諸將不得已與俱。俄而敵大至,圍之數重。聰戰死,福及諸將皆被執遇害,年六十七,一軍皆沒。敗聞,帝震怒。以諸將無足任者,決計親征。奪福世爵,徙其家海南。

李遠,懷遠人。襲父職為蔚州衛指揮僉事。燕兵攻蔚州,舉城降。南軍駐德州,運道出徐、沛間。遠以輕兵六千,詐為南軍袍鎧,人插柳一枝於背,徑濟寧、沙河至沛,無覺者。焚糧舟數萬,河水盡熱,魚鱉皆浮死。南將袁宇三萬騎來追,伏兵擊敗之。建文四年正月,燕軍駐蠡縣。遠分哨至槁城,遇德州將葛進步騎萬餘,乘冰渡滹沱河。遠迎擊之。進繫馬林間,以步兵接戰。遠佯卻,潛分兵出其後,解所繫馬,再戰。進引退失馬,遂大敗。斬首四千,獲馬千匹。成祖以歲首大捷,賜書嘉勞曰:「將軍以輕騎八百,破敵數萬,出奇應變,雖古名將不過也。」復遣哨淮上,敗守淮將士,斬千餘級。累功為都督僉事,封安平侯,祿千石,予世伯券。永樂元年,偕武安侯鄭亨備宣府。

遠沈毅有膽略,言論慷慨。既從邱福出塞,至臚朐河。諫福,不聽,師敗。遠帥五百騎突陣,殺數百人,馬蹶被執,罵不絕口死。年四十六。追封莒國公,謚忠壯。

子安,嗣伯爵。洪熙元年為交阯參將,失律,謫為事官。已,從王通棄交阯還,下獄奪券,謫赤城,立功。英宗即位,起都督僉事。征阿台朵兒只伯。遷都督同知,充總兵官,鎮松潘。正統六年,副定西伯蔣貴徵麓川。貴令安駐軍潞江護餉,而自帥大軍進。賊破。安恥無功,聞有餘賊屯高黎貢山,徑往擊之。為所敗,失士卒千餘人,都指揮趙斌等皆死。逮下獄,謫戍獨石。卒。詔授子清都指揮同知。

王忠,孝感人。與李遠同降於蔚州。每戰,帥精騎為奇兵,多斬獲。累遷都督僉事,封靖安侯,祿千石。出塞戰歿,年五十一,爵除。

王聰,蘄水人。以燕山中護衛百戶從起兵。取薊州,攻遵化,徇涿州。轉戰茌平、滑口,破南軍,獲馬千五百。還守保定。從次江上,略南軍舟濟師。累遷都指揮使。封武城侯,祿千五百石。偕同安侯火真備禦宣府。屢奉詔巡邊。從邱福出塞,戰死,年五十三。追封漳國公,謚武毅。子琰嗣。聰及遠嘗諫福,故得褒恤。

火真,蒙古人,初名火里火真。洪武時歸附,為燕山中護衛千戶。從攻真定,先馳突耿炳文陣,大軍乘之,遂捷。從襲大寧,戰鄭村壩。日暝,天甚寒,真斂敝鞍爇火成祖前。甲士數人趨附火,衛士止之。成祖曰:「吾衣重裘猶寒。此皆壯士,勿止也。」聞者感泣。真嘗將騎兵,每戰輒有斬獲,呼噪歸營,眾服其勇。累遷都督僉事,封同安侯,祿千五石。出塞戰歿,年六十一。爵除。子孫世襲觀海衛千戶。

裔孫斌,嘉靖中武舉。倭寇浙東,帥海舟與賊戰。賊然火球擲斌舟,斌輒手接之,還燒賊舟。賊屯補陀山,斌直搗其營,多殺傷。後軍不繼,被擒。不屈,賊支解之。官為建祠曰「忠勇」。

譚淵,清流人。嗣父職為燕山右護衛副千戶。燕兵起,從奪九門。破雄縣。潘忠、楊松自鄚州來援,淵帥壯士千餘人,伏月漾橋水中,人持茭草一束,蒙頭通鼻息。南軍已過,即出據橋。忠等戰敗,趨橋不得渡,遂被擒。累進都指揮同知。

淵驍勇善戰,引兩石弓,射無不中。然性嗜殺。滄州破,成祖命給牒散降卒。未遣者三千餘人,待明給牒。淵一夜盡殺之。王怒。淵曰:「此曹皆壯士,釋之為後患。」王曰:「如爾言,當盡殺敵。敵可盡乎?」淵慚而退。

夾河之戰,南軍陣動塵起。淵遽前搏戰,馬蹶被殺。成祖悼惜之。即位,贈都指揮使,追封崇安侯,謚壯節,立祠祀之。

子忠,從入京師有功。又以淵故封新寧伯,祿千石。永樂二十一年,將右哨從征沙漠。。宣德元年從征樂安。三年坐徵交阯失律,下獄論死,已得釋,卒。子璟乞嗣。吏部言忠罪死,不當襲。帝曰:「券有免死文,其予嗣。」再傳至孫祐。成化中,協守南京。還,掌前府提督團營,累加太傅,嗣伯,六十九年始卒。謚莊僖。子綸嗣。嘉靖十四年鎮湖廣,剿九溪蠻有功,益祿。坐占役軍士奪爵。數傳至弘業,國亡,死於賊。

王真,咸寧人。洪武中,起卒伍。積功至燕山右護衛百戶。燕兵起,攻九門。戰永平、真定,下廣昌,徇雁門。從破滄州,追南兵至滑口,俘獲七千餘人。累遷都指揮使。淝河之戰,真與白義、劉江各帥百騎誘平安軍。縛草置囊中為束帛狀,安追擊,真等佯棄囊走,安軍士競取之。伏發,兩軍鏖戰。真帥壯士直前,斬馘無算。後軍不繼,安軍圍之數匝。真被重創,連格殺數十人,顧左右曰:「我義不死敵手。」遂自刎。成祖即位,追封金鄉侯,謚忠壯。

真勇健有智略。成祖每追悼之曰:「奮武如王真,何功不成!不死,功當冠諸將。」仁宗時,追封寧國公,加號效忠。子通自有傳。

陳亨,壽州人。元末揚州萬戶。從太祖於濠,為鐵甲長,擢千戶。從大將軍北征,守東昌。敵數萬奄至,亨固守,出奇兵誘敗之。復從徇未下諸城。洪武二年守大同。積功至燕山左衛指揮僉事。數從出塞,遷北平都指揮使。及惠帝即位,擢都督僉事。

燕師起,亨與劉真、卜萬守大寧。移兵出松亭關,駐沙河,謀攻遵化。燕兵至,退保關。當是時,李景隆帥五十萬眾將攻北平。北平勢弱,而大寧行都司所領興州、營州二十餘衛,皆西北精銳;朵顏、泰寧、福餘三衛,元降將所統番騎彍卒,尤驍勇。卜萬將與景隆軍合。成祖懼,以計紿亨囚萬,遂從劉家口間道疾攻大寧。亨及劉真自松亭回救,中道聞大寧破,乃與指揮徐理、陳文等謀降燕。夜二鼓,襲劉真營。真單騎走廣寧,亨等帥眾降。成祖盡拔諸軍及三衛騎卒,挾寧王以歸。自是衝鋒陷陣多三衛兵。成祖取天下,自克大寧始。

亨、理既降,累從破南軍。白溝河之戰,亨中創幾死。已,攻濟南,與平安戰鏵山,大敗。創甚,輿還北平。進都督同知。成祖還軍,親詣亨第勞問。其年十月卒。成祖自為文以祭。比即位,追封涇國公,謚襄敏。長子恭,嗣都督同知。

少子懋,初以舍人從軍,立功為指揮僉事。已而將亨兵,功多,累進右都督。永樂元年,封寧陽伯,祿千石。六年三月佩征西將軍印,鎮寧夏,善撫降卒。明年秋,故元丞相昝卜及平章、司徒、國公、知院十餘人,皆帥眾相繼來降。已而平章都連等叛去,懋追擒之黑山,盡收所部人口畜牧。進侯,益祿二百石。八年從北征,督左掖。十一年巡寧夏邊。尋命將山西、陜西二都司及鞏昌、平涼諸衛兵,駐宣府。明年從北征,領左哨。戰忽失溫,與成山侯王通先登,都督硃崇等乘之,遂大捷。明年復鎮寧夏。二十年從北征。領御前精騎破敵於屈裂河。別將五千騎循河東北,捕餘寇,殲之山澤中。師還,武安侯鄭亨將輜重先行,懋伏隘以待。敵來躡,伏起縱擊,敵死過半。還京,賜龍衣玉帶,冊其女為麗妃。明年將陜西、寧夏、甘肅三鎮兵,從征阿魯台,為前鋒。又明年復領前鋒,從北征。

成祖之崩於榆木川也,六軍在外,京師守備虛弱。仁宗召懋與陽武侯薛祿帥精騎三千馳歸,衛京師。命掌前府,加太保,與世侯。宣德元年,從討樂安。還,仍鎮寧夏。三年奏徙靈州城。得黑白二兔以獻,宣宗喜,親畫馬賜之。懋在鎮久,威名震漠北。顧恃寵自恣,乾沒鉅萬。屢被劾,帝曲宥之,命所司徵其贓。懋自陳用已盡,詔貸免。

英宗即位,命偕張輔參議朝政,出為平羌將軍,鎮甘肅。其冬,寇掠鎮番,懋遣兵援之,解去,以斬獲聞。參贊侍郎柴車劾懋失律致寇,又取所遺老弱,冒為都指揮馬亮等功受CD賞,論斬。詔免死,奪祿。久之還祿,奉朝請。十三年,福建賊鄧茂七反。都御史張楷討之無功,乃詔懋佩征南將軍印,充總兵官,帥京營、江浙兵往討。至浙江,有欲分兵扼海口者,懋曰:「是使賊致死於我也。」明年抵建寧,茂七已死,餘賊聚尤溪、沙縣。諸將欲屠之,懋曰:「是堅賊心也。」乃下令招撫,賊黨多降。分道逐捕,悉平之。已而沙縣賊復熾,久不定。會英宗北狩,景帝立,遂詔班師。言官劾之,以賊平不問。仍加太保,掌中府,兼領宗人府事。英宗復位,益祿二百石。天順七年卒,年八十四。贈濬國公,謚武靖。

懋修髯偉貌,聲如洪鐘。胸次磊落,敬禮士大夫。「靖難」功臣至天順時無在者,惟懋久享祿位,數廢數起,卒以功名終。

長子晟有罪,弟潤嗣。潤卒,弟瑛嗣,減祿之半,嗣侯。十六年而晟子輔已長,乃令輔嗣,瑛免為勳衛。輔後坐事失侯。卒,無子。復封瑛孫繼祖為侯,傳爵至明亡。

徐理,西平人。洪武時,為永清中護衛指揮僉事,改營州衛。既降,為右軍副將。每戰先登,有功。成祖將襲滄州,命理及陳旭潛於直沽造浮橋,以濟師。累進都指揮僉事,封武康伯。還守北平。理馭下寬,得士卒心。永樂六年卒。再傳至孫勇,無子絕封。

陳文,降後為前軍左副將。戰小河,死於陣。

房寬,陳州人。洪武中,以濟寧左衛指揮從徐達練兵北平,遂為北平都指揮同知,移守大寧。寬在邊久,凡山川夋塞,殊域情偽,莫不畢知,然不能撫士卒。燕兵奄至,城中縛寬以降。成祖釋之,俾領其眾。戰白溝河,將右軍,失利。從克廣昌、彰德,進都督僉事。以舊臣,略其過。封思恩侯,祿八百石,世指揮使。永樂七年卒。

劉才,字子才,霍丘人。元末為元帥,明興歸附,歷營州中護衛指揮僉事。燕師襲大寧,才降。從戰有功,封廣恩伯,祿九百石,世指揮同知。永樂八年,從北征,督右掖。失律議罪,既而宥之。二十一年偕隆平侯張信理永平、山海邊務。明年復從北征,至懷來,以疾還。才悃愊無華,不為茍合,亦不輕訾毀人,甚為仁宗所重。宣德五年卒。

贊曰:惠帝承太祖遺威餘烈,國勢初張,仁聞昭宣,眾心悅附。成祖奮起方隅,冒不韙以爭天下,未嘗有萬全之計也。乃道衍首贊密謀,發機決策。張玉、朱能之輩戮力行間,轉戰無前,隕身不顧。於是收勁旅,摧雄師,四年而成帝業。意者天之所興,群策群力,應時並濟。諸人之得為功臣首也,可不謂厚幸哉!


\end{pinyinscope}