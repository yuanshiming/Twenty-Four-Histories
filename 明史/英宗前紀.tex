\article{英宗前紀}

\begin{pinyinscope}
英宗法天立道仁明誠敬昭文憲武至德廣孝睿皇帝,諱祁鎮,宣宗長子也。母貴妃孫氏。生四月,立為皇太子,遂冊貴妃為皇后。

宣德十年春正月,宣宗崩,壬午,即皇帝位。遵遺詔大事白皇太后行。大赦天下,以明年為正統元年。始罷午朝。丁亥,尚書蹇義卒。辛丑,戶部尚書黃福參贊南京守備機務。二月戊申,尊皇太后為太皇太后。庚戌,尊皇后為皇太后。辛亥,封弟祁鈺為郕王。甲寅,罷諸司冗費。三月戊寅,放都坊司樂工三千八百餘人。辛巳,罷山陵夫役萬七千人。丙申,諭三法司,死罪臨決。三覆奏然後加刑。

夏四月壬戌,以元學上吳澄從祀孔子廟庭。丁卯,以久旱考察布、按二司及府州縣官。戊辰,遣給事中、御史捕畿南、山東、河南、淮安蝗。五月壬午,戶部言浙江、蘇、松荒田稅糧減除二百七十七萬餘石,請加覆核。帝以核實必增額為民患。不許。六月丁未,令天下瘞暴骸。辛酉,葬章皇帝於景陵。

秋七月丙子,免山西夏稅之半。八月丙午,減光祿寺膳夫四千七百餘人。九月壬辰,詔督漕總兵及諸巡撫官,歲以八月至京會廷臣議事。是月,王振掌司禮監。

冬十月壬寅,遣使諭阿台朵兒只伯。辛亥,詔天下衛所皆立學。十一月戊辰朔,日有食之。十二月壬子,阿台朵兒只怕犯涼州鎮番,總兵官陳懋敗之於黑山。

是年,琉球中山、暹羅、日本、占城、安南、滿剌加、哈密、瓦剌入貢。

正統元年春正月丙戌,罷銅仁金場。庚寅,發禁軍三萬人屯田畿輔。三月己巳,賜周旋等進士及第、出身有差。乙亥,御經筵。

夏四月丁酉朔,享太廟。五月丁卯,阿台朵兒只伯寇肅州。壬辰,設提督學校官。

秋八月甲戌,右都督蔣貴充總兵官,都督同知趙安副之,帥師討阿台朵兒只伯。九月癸卯,遣侍郎何文淵、王佐,副都御史朱與言督兩淮、長蘆、浙江鹽課。欽差巡鹽自此始。庚申,封黎利子麟為安南國王。

冬十一月乙卯,詔京官三品以上舉堪任御史者,四品及侍從言官舉堪任知縣者,各一人。免湖廣被災稅糧。十二月丁丑,以邊議稽緩,下兵部尚書王驥、侍郎鄺野於獄,尋釋之。乙酉,湖廣、貴州總兵官蕭授討廣西蒙顧十六洞賊,平之。

是年,琉球中山、爪哇、安南、烏斯藏、占城、瓦剌入貢。遣宣德時來貢古里、蘇門答剌十一國使臣還國。

二年春正月甲午,宣宗神主祔太廟。己亥,大同總兵官方政、都指揮楊洪會寧夏、甘肅兵出塞討阿台朵兒只伯。三月甲午,錄囚。戊午,御史金敬撫輯大名及河南、陜西逃民。

夏四月,免河南被災田糧。五月庚寅,兵部尚書王驥經理甘肅邊務。壬寅,刑部尚書魏源經理大同邊務。丁未,免陜西平涼六府旱災夏稅。六月乙亥,以宋胡安國、蔡沈、真德秀從祀孔子廟庭。庚辰,副都御史賈諒、侍郎鄭辰振河南、江北饑。

冬十月甲子,鎮守甘肅左副總兵任禮充總兵官,都督蔣貴、都督同知趙安為左、右副總兵,兵部侍郎柴車,僉都御史曹翼、羅亨信參贊軍務,討阿台朵兒只伯。兵部尚書王驥、太監王貴監督之。十一月乙巳,振河南饑,免稅糧。

是年,琉球中山、撒馬兒罕、暹羅、土魯番、瓦剌、哈密入貢。

三年春三月己亥,京師地震。辛丑,振陜西饑。

夏四月乙卯,王驥、任禮、蔣貴、趙安襲擊阿台朵兒只伯,大破之,追至黑泉還。癸未,立大同馬市。六月癸酉,以旱讞中外疑獄。乙亥,都督方政、僉事張榮同征南將軍黔國公沐晟、右都督沐昂,討麓川叛蠻思任發。

秋七月癸未,下禮部尚書胡濙於獄。辛卯,下戶部尚書劉中敷於獄。尋俱釋之。八月乙亥,以陜西饑,令雜犯死囚以下輸銀贖罪,送邊吏易米。九月癸巳,蠲兩畿、湖廣逋賦。

冬十月癸丑,再振陜西饑。十二月丙辰,下刑部尚書魏源、右都御史陳智等於獄。

是年,榜葛剌貢麒麟,中外表賀。琉求中山、進羅、占城、瓦剌入貢四年春正月壬午,方政破麓川蠻於大寨,追至空泥,敗沒。二月丁巳,總兵官蕭授平貴州計砂叛苗。閏月辛丑,釋魏源、陳智等,復其官,并宥棄交阯王通、馬騏罪。三月己酉詔赦天下。壬子,賜施槃等進士及第、出身有差。庚申,廢遼王貴烚為庶人。丁卯,黔國公沐晟卒於軍。癸酉,增南京及在外文武官軍俸廩。

夏五月庚戌,右都督沐昂為征南將軍,充總兵官,討思任發。丁卯,錄中外囚。六月乙未,京師地震。丁酉,以京畿水災祭告天地,諭群臣修省。戊戌,下詔寬恤,求直言。

秋七月庚戌,免兩畿、山東、江西、河南被災稅糧。壬申,汰冗官。八月戍戌,增設沿海備倭官。己亥,京師地震。

冬十二月丁丑,都督同知李安充總兵官,僉都御史王翱參贊軍務,討松潘祈命族叛番。

是年,琉球、占城、安南、瓦剌、榜葛剌、滿剌加、哈密入貢。

五年春正月己未,大祀天地於南郊。二月乙亥,侍講學士馬愉、侍講曹鼐入閣預機務。甲申,僉都御史張純、大理少卿李畛振撫畿內流民。三月戊申,建北京宮殿。

夏四月壬申,免山西逋賦。丙戌,祈命簇番降。五月,徵麓川,參將張榮敗績於芒市。六月丁丑,免兩畿被來田糧。戊寅,錄囚。

秋七月辛丑遣刑部侍郎何文淵等分行天下,修備荒之政。壬寅,楊榮卒。八月乙未,令各邊修舉荒政。九月壬寅,蠲雲南逋賦。

冬十一月壬寅,振浙江饑。壬子,免蘇、松、常、鎮、嘉、湖水災稅糧。丁巳,廣西僧楊行祥偽稱建文帝,械送京師,錮錦衣衛獄死。乙丑,沐昂討平師宗叛蠻。十二月壬午,免南畿浙江、山東、河南被災稅糧。

是年,占城、琉球中山、哈密、烏斯藏入貢。

六年春正月己亥朔,日當食,不見,禮官請表賀,不許。庚戌,大祀天地於南郊。乙卯,以莊浪地屢震,躬禮郊廟,遣使祭西方嶽鎮。大舉征麓川,定西伯蔣貴為平蠻將軍,都督同知李安、僉事劉聚副之,兵部尚書王驥總督軍務。三月庚子,下兵部侍郎于謙於獄。

夏四月甲午,以災異遣使省天下疑獄。五月甲寅,刑部侍郎何文淵、大理卿王文錄在京刑獄,巡撫侍郎周忱、刑科給事中郭瑾錄南京刑獄。釋于謙為大理少卿。

秋七月丁未,振浙江、湖廣饑。

冬十月丁丑,戶部尚書劉中敷,侍郎吳璽、陳瑺荷校於長安門,旬餘釋還職。庚寅,免畿內被災稅糧。十一月甲午朔,乾清、坤寧二宮,奉天、華蓋、謹身三殿成,大赦。定都北京,文武諸司不稱行在。癸卯,王驥拔麓川上江寨。癸丑,免河南、山東及鳳陽等府被災稅糧。閏月甲戌,復下劉中敷、吳璽、陳瑺於獄。踰年,釋中敷為民,璽、瑺戍邊。十二月,王驥克麓川,思任發走孟養。丁未,班師。右副總兵李安攻餘賊於高黎貢山,敗績。

是年,占城、瓦剌、哈密入貢。

七年春正月甲戌,大祀天地於南郊。二月庚申,如天壽山。三月甲子,還宮。乙亥,免陜西屯糧十之五。戊寅,賜劉儼等進士及第、出身有差。

夏四月甲午,振陜西饑。是月,免山西、河南、山東被災稅糧。五月壬申,論平麓川功,進封蔣貴為侯,王驥靖遠伯。戊寅,立皇后錢氏。丁亥,倭陷大嵩所。六月壬子,戶部侍郎焦宏備倭浙江。

秋七月丙寅,振陜西饑民,贖民所鬻子女。八月壬寅,復命王驥總督雲南軍務。九月甲戌,陜西進嘉禾,祀臣請表賀,不許。

冬十月壬辰,兀良哈犯廣寧。乙巳,太皇太后崩。十二月,葬誠孝昭皇后於獻陵。

是年,占城、瓦剌、哈密、琉球中山、安南、爪哇、土魯番、烏斯藏入貢。

八年春正月丁卯,大祀天地於南郊。二月己丑,汰南京冗官。戊戌,淮王瞻墺來朝。丙午,荊王瞻堈來朝。

夏五月己巳,復命平蠻將軍蔣貴、王驥帥師征麓川思任發子思機發。戊寅,雷震奉天殿鴟吻,敕修省。壬午,大赦。六月丁亥,侍講劉球陳十事,下錦衣衛獄,太監王振使指揮馬順殺之。甲辰,下大理少卿薛瑄於獄。

秋七月戊午,祭酒李時勉荷校於國子監門三日。九月甲子,思機發請降。

冬十一月,宣宗廢后胡氏卒。十二月癸未,免山東復業民稅糧二年。丙戌,駙馬都尉焦敬荷校於長安右門。

是年,占城、安南、瓦剌、哈密、爪哇入貢。

九年春正月甲寅,右都御史王文巡延安、寧夏邊。辛酉,大祀天地於南郊。辛未,成國公朱勇,興安伯徐亨,都督馬亮、陳懷,同太監僧保、曹吉祥、劉永誠、但住分道討兀良哈。二月丙午,王驥擊走思機發,俘其孥以獻。召驥還。三月辛亥朔,新建太學成,釋奠於先師孔子。甲子,朱勇等師還。楊士奇卒。乙丑,敘征兀良哈功,封陳懷平鄉伯,馬亮招遠伯,成國公朱勇等進秩有差。

夏四月丙戌,翰林學士陳循直文淵閣,預機務。丁亥,振沙州及赤斤蒙古饑。五月己未,命法司錄在京刑獄,刑部侍郎馬昂錄南京刑獄。六月壬午,振湖廣、貴州蠻饑。

秋七月己酉,下駙馬都尉石璟於獄。處州賊葉宗留資福安銀礦,殺福建參議竺淵。癸丑,免河南被災稅糧。閏月戊寅,復開福建、浙江銀場。甲申,瘞暴骸。壬寅,雷震奉先殿鴟吻。八月庚戌,免陜西被災稅糧,贖民所鬻子女。甲戌,敕邊將備瓦剌也先。九月丁亥,靖遠伯王驥、右都御史陳鑒經理西北邊備。

冬十月丙午朔,日有食之。庚午,兀良哈貢馬謝罪。

是年,兩畿、山東、河南、浙江、湖廣大水,江河皆溢。暹羅、琉球中山、瓦剌、安南、烏斯藏、滿剌加入貢。

十年春正月丙戊,大祀天地於南郊。戊子,詔舉智勇之士。二月丁巳,京師地震。己未,免陜西逋賦。丙寅,兀良哈貢馬,請貸犯邊者罪,不許。壬申,如天壽山。三月丙子,還宮。庚辰,思機發入貢謝罪。庚寅,賜商輅等進士及第、出身有差。

夏四月甲辰朔,日有食之。庚申,詔所在有司飼逃民復業及流移就食者。六月乙丑,振陜西饑。免田租三之二。

秋七月乙未,減糶河南、懷慶倉粟、濟山、陜饑。八月癸丑,免湖廣旱災秋糧。丙辰,免蘇、松、嘉、湖十四府州水災秋糧。

冬十月戊辰,侍讀學士苗衷為兵部侍郎,侍講學士高穀為工部侍郎,並入閣預機務。十二月丙辰,緬甸獲思任發,斬其首送京師。壬戌,輸河南粟振陜西饑。廣西總兵官安遠侯柳溥討平慶遠叛蠻。

是年,琉球中山、哈密、亦力把里、安南、占城、滿剌加、錫蘭山、撒馬兒罕、烏斯藏入貢。

十一年春正月己卯,大祀天地於南郊。庚辰,予太監王振等弟姪世襲錦衣衛官。二月辛酉,異氣見華蓋、奉天殿,遣官祭告天地。癸亥,詔恤刑獄。三月戊辰,下戶部尚書王佐、刑部尚書金濂、右都御史陳鎰等於錦衣衛獄,尋釋之。壬申,御史柳華督福建、浙江、江西兵討礦賊。癸酉,如天壽山。庚辰,還宮。

夏六月丙辰,京師地震。

秋七月癸酉,增市廛稅鈔。庚辰,楊溥卒。八月戊戌,免湖廣被災秋糧。庚申,下吏部尚書王真等於獄,尋釋之。九月辛巳,廣西瑤叛,執化州知州茅自得,殺千戶汪義。

冬十月甲寅,遣給事中、御史分賚諸邊軍士。十一月壬申,減殊死以下罪。

是年,琉球中山、暹羅、安南、爪哇、回回哈密、占城、亦力把里、撒馬兒罕、烏斯藏入貢。

十二年春正月癸酉,大祀天地於南郊。三月癸亥,如天壽山。庚午,還宮。丙子,免杭嘉、湖被災秋糧。

夏四月丁巳,免蘇、松、常、鎮被災秋糧。五月己亥,大理少卿張驥振濟寧及淮、揚饑。

秋七月甲辰,敕各邊練軍備瓦剌。八月庚申朔,日有食之。九月乙未,馬榆卒。

是年,琉球中山、安南、占城、瓦剌、爪哇、哈密、暹羅入貢。

十三年春正月丁酉,大祀天地於南郊。三月戊子,詔責孟養宣慰司獻思機發。壬寅,賜彭時等進士及第、出身有差。王驥仍總督軍務,都督同知宮聚為平蠻將軍,充總兵官,帥師討思機發。

夏四月,免浙江、江西、湖廣被災秋糧。五月丙戌,遣使捕山東蝗。甲辰,刑部侍郎丁鉉撫輯河南、山東災民。

秋七月乙酉,河決大名,沒三百餘里,遣使蠲振。己酉河決河南、沒曹、濮、東昌,潰壽張沙灣,壞運這,工部侍郎王永和治之。八月乙卯,福建賊鄧茂七作亂。甲戌,命御史丁瑄捕之。

冬十一月丙戌,寧陽侯陳懋充總兵官,保定伯梁珤、平江伯陳豫副之,太監曹吉祥、王瑾提督火器,刑部尚書金濂參贊軍務,討鄧茂七。甲辰,處州賊流劫金華諸縣。庚戌,永康侯徐安備倭山東。十二月庚午,廣東瑤賊作亂。

是年,琉球中山、安南、占城入貢。瓦剌貢使三千人,賞不如例,遂構釁。

十四年春正月申午,大祀天地於南郊。乙巳,免浙江、福建銀課。二月丁巳,御史丁瑄、指揮劉福擊斬鄧茂七於延平。己巳,王驥破思機發於金沙江,又破之鬼哭山,班師。辛未,指揮僉事徐恭元總兵官,討處州賊葉宗留,工部尚書石璞參贊軍務。三月戊子,如天壽山,癸巳,還宮。

夏四月庚戌,處州賊犯崇安,殺都指揮吳剛。壬戌,湖廣、貴州苗賊大起,命王驥討之,乙丑,遣御史十三人同中官督福建、浙江銀課。五月丙戌,陳懋擊破沙縣賊。壬辰,旱,太監金英同法司錄囚。己亥,侍讀學士張益直文淵閣,預機務。庚子,巡按福建御史汪澄棄市,并殺前巡按御史柴文顯。六月庚戌,靖州苗犯辰溪,都指揮高亮戰死。丙辰,南京謹身諸殿災。甲子,修省,詔河南、山西班軍番休者盡赴大同、宣府。乙丑,西寧侯宋瑛總督大同兵馬。己巳,赦天下。戊寅,平鄉伯陳懷,駙馬都尉井源,都督王貴、吳克勤,太監林壽,分練京軍於大同、宣府,備瓦剌。

秋七月己丑,瓦剌也先寇大同,參將吳浩戰死,下詔親征。吏部尚書王直帥群臣諫,不聽。癸巳,命郕王居守。是日,西寧侯宋瑛、武進伯朱冕與瓦剌戰於陽和,敗沒。甲午,發京師。乙未,次龍虎臺。軍中夜驚。丁酉,次居庸關。辛丑,次宣府。群臣屢請駐蹕,不許。丙午,次陽和。八月戊申,次大同。鎮守太監郭敬諫,議旋師。己酉,廣寧伯劉安為總兵官,鎮大同。庚戌,師還。丁巳,次宣府。庚申,瓦剌兵大至,恭順侯吳克忠、都督吳克勤戰沒,成國公朱勇、永順伯薛綬救之,至鷂兒嶺遇伏,全軍盡覆。辛酉,次土木,被圍。壬戌,師潰,死者數十萬。英國公張輔,奉寧侯陳瀛,駙馬都尉並源,平鄉伯陳懷,襄城伯李珍,遂安伯陳塤,修武伯沈榮,都督梁成、王貴,尚書王佐、鄺野,學士曹鼐、張益,侍郎丁鉉、王永和,副都御史鄧棨等,皆死,帝北狩。甲子,京師聞敗,群臣聚器於朝,侍講徐珵請南遷,兵部侍郎於謙不可。乙丑,皇太后命郕王監國。戊辰,帝至大同。己巳,皇太后命立皇子見深為皇太子。辛未,帝至威寧海子。甲戌,至黑河。九月癸未,郕王即位,遙尊帝為太上皇帝。


\end{pinyinscope}