\article{列傳第一百七十四 文苑二}

\begin{pinyinscope}
○林鴻鄭定等王紱夏昶沈度弟粲滕用亨等聶大年劉溥蘇平等張弼張泰陸釴陸容程敏政羅儲巏李夢陽康海王九思王維楨何景明徐禎卿楊循吉祝允明唐寅桑悅邊貢顧璘弟瑮陳沂等鄭善夫殷雲霄方豪等陸深王圻王廷陳李濂

林鴻,字子羽,福清人。洪武初,以人才薦,授將樂縣訓導,歷禮部精膳司員外郎。性脫落,不善仕,年未四十自免歸。閩中善詩者,稱十才子,鴻為之冠。十才子者,閩鄭定,侯官王褒、唐泰,長樂高棅、王恭、陳亮,永福王偁及鴻弟子周玄、黃玄,時人目為二玄者也。

鴻論詩,大指謂漢、魏骨氣雖雄,而菁華不足。晉祖玄虛,宋尚條暢,齊、梁以下但務春華,少秋實。惟唐作者可謂大成。然貞觀尚習故陋,神龍漸變常調,開元、天寶間聲律大備,學者當以是為楷式。閩人言詩者率本於鴻。

晉府引禮舍人浦源,字長源,無錫人也。慕鴻名,踰嶺訪之。造其門,二玄請誦所作,曰:「吾家詩也。」鴻延之入社。

鄭定,字孟宣,嘗為陳友定記室。友定敗,浮海亡交、廣間。久之,還居長樂。洪武中,徵授延平府訓導,歷國子助教。

王褒,字中美,鴻之兄子婿也。為長沙學官,遷永豐知縣。永樂中,召入,預修《大典》,擢漢府紀善。

唐泰,字亨仲。洪武二十七年進士。歷陜西副使。

高棅,字彥恢,更名廷禮,別號漫士。永樂初,以布衣召入翰林,為待詔,遷典籍。性善飲,工書畫,尤專於詩。其所選《唐詩品匯》、《唐詩正聲》,終明之世,館閣宗之。

王恭,字安中,隱居七巖山,自稱皆山樵者。永樂初,以儒士薦起待詔翰林,年六十餘,與修《大典》。書成,授翰林院典籍。

陳亮,字景明。自以故元儒生,明興累詔不出,作《陳摶傳》以見志。結草屋滄洲中,與三山耆彥為九老會,終其身不仕。

王偁,字孟易又。父翰仕元,抗節死,偁方九歲,父友吳海撫教之。洪武中,領鄉薦,入國學,陳情養母。母歿,廬墓六年。永樂初,用薦授翰林檢討,與修《大典》。學博才雄,最為解縉所重。自負無輩行,獨推讓同官王洪。

王洪者,字希範,錢塘人。八歲能文,十八成進士,授吏科給事中。改翰林檢討,偕偁等與修《大典》。歷修撰、侍講。帝頒佛曲於塞外,命洪為文,逡巡不應詔。為同列所排,不復進用,卒官。而偁後坐累謫交址,復以縉事連及,繫死獄中。

黃玄,字玄之,將樂人。聞鴻棄官歸,遂攜妻子居閩縣,以歲貢官泉州訓導。

周玄,字微之,閩縣人。永樂中,以文學徵,授禮部員外郎。嘗挾書千卷止高棅家,讀十年,辭去,盡棄其書,曰:「在吾腹笥矣。」同時趙迪、林敏、陳仲宏、鄭關、林伯璟、張友謙亦以能詩名,皆鴻之弟子。

王紱,字孟端,無錫人。博學,工歌詩,能書,寫山木竹石,妙絕一時。洪武中,坐累戍朔州。永樂初,用薦,以善書供事文淵閣。久之,除中書舍人。

紱未仕時,與吳人韓奕為友,隱居九龍山,遂自號九龍山人。於書法,動以古人自期。畫不茍作,游覽之頃,酒酣握筆,長廊素壁淋漓霑灑。有投金幣購片楮者,輒拂袖起,或閉門不納,雖豪貴人勿顧也。有諫之者,紱曰:「丈夫宜審所處,輕者如此,重者將何以哉!」在京師,月下聞吹簫聲,乘興寫《石竹圖》,明旦訪其人贈之,則估客也。客以紅氍毹饋,請再寫一枝為配。紱索前畫裂之,還其饋。一日退朝,黔國公沐晟從後呼其字,紱不應。同列語之曰:「此黔國公也。」紱曰:「我非不聞之,是必與我索畫耳。」晟走及之,果以畫請,紱頷之而已。踰數年,晟復以書來,紱始為作畫。既而曰:「我畫直遺黔公不可。黔公客平仲微者,我友也,以友故與之,俟黔公與求則可耳。」其高介絕俗如此。

昆山夏昶者,亦善畫竹石,亞於紱。畫竹一枝,直白金一錠,然人多以饋遺得之。昶,字仲昭,永樂十三年進士,改庶吉士,歷官太常寺卿。昶與上元張益,同中進士,同以文名,同善畫竹。其後,昶見益《石渠閣賦》,自謂不如,遂不復作賦。益見昶所畫竹石,亦遂不復畫竹。益死土木之難。

仲微,名顯,錢塘人。嘗知滕縣事,謫戍雲南。其為詩頗豪放自喜,雲南詩人稱平、居、陳、郭,顯其一也。

沈度,字民則。弟粲,字民望。松江華亭人。兄弟皆善書,度以婉麗勝,粲以遒逸勝。度博涉經史,為文章絕去浮靡。洪武中,舉文學,弗就。坐累謫雲南,岷王具禮幣聘之,數進諫,未幾辭去。都督瞿能與偕入京師。成祖初即位,詔簡能書者入翰林,給廩祿,度與吳縣滕用亨、長樂陳登同與選。是時解縉、胡廣、梁潛、王璉皆工書,度最為帝所賞,名出朝士右。日侍便殿,凡金版玉冊,用之朝廷,藏秘府,頒屬國,必命之書。遂由翰林典籍擢檢討,歷修撰,遷侍講學士。粲自翰林待詔遷中書舍人,擢侍讀,進階大理少卿。兄弟並賜織金衣,鏤姓名於象簡,泥之以金。贈父母如其官,馳傳歸,告於墓。

崑山夏昺者,字孟晹,與其弟昶以善書畫聞,同官中書舍人,時號大小中書,而度、粲號大小學士。

度性敦實,謙以下人,嚴取與。有訓導介其友求書,請識姓字於上。度沈思曰:「得非曩訐奏有司者耶?」遽卻之。其友固請,終不肯書姓名。其在內廷備顧問,必以正對。粲篤於事兄,己有賜,輒歸其兄。

滕用亨,初名權,字用衡。精篆隸書。被薦時年七十矣,召見,大書麟鳳龜龍四字以進,又獻《貞符詩》三篇。授翰林待詔,與修《永樂大典》。用亨善鑒古,嘗侍帝觀畫卷,未竟,眾目為趙伯駒,用亨曰:「此王詵筆也。」至卷尾,果然。

陳登,字思孝。初仕羅田縣丞,改蘭谿,再改浮梁。選入翰林,仍給縣丞祿,歷十年始授中書舍人。登於六書本原,博考詳究,用力甚勤。自周、秦以來,殘碑斷碣,必窮搜摩拓審度而辨定之。得其傳者,太常卿南城程南雲也。

聶大年,字壽卿,臨川人。父同文,洪武中,官翰林侍書、中書舍人。燕王入京師,迎謁,道曷死,死後五月而大年生,母胡撫之。比長,博學,善詩古文。葉盛稱其詩,謂三十年來絕唱也。書得歐陽率更法。宣德末,薦授仁和訓導。母卒,歸葬,哀感行路。里人列其母子賢行上之有司,詔旌其門。服闋,分教常州,遷仁和教諭。景泰六年薦入翰林,未幾得疾卒。

始,尚書王直以詩寄錢塘戴文進索畫,自序昔與文進交,嘗戲作詩一聯,至是十年始成之。大年題其後曰:「公愛文進之畫,十年不忘。使以是心待天下賢者,天下寧復有遣賢哉。」直聞其言,不怒亦不薦。及大年疾篤,作詩貽直,有「鏡中白髮孰憐我,湖上青山欲待誰」句,直曰,「此欲吾志其墓耳」,遂為之志。

劉溥,字原博,長洲人。祖彥,父士賓,皆以醫得官。溥八歲賦《溝水詩》,時目為聖童。長侍祖父遊兩京,研究經史兼通天文、曆數。宣德時,以文學徵。有言溥善醫者,授惠民局副使,調太醫院吏目。恥以醫自名,日吟詠為事。其詩初學西昆,後更奇縱,與湯胤勣、蘇平、蘇正、沈愚、王淮、晏鐸、鄒亮、蔣忠、王貞慶號「景泰十才子」,溥為主盟。

胤勣,東甌王和曾孫,自有傳。蘇平,字秉衡,弟正,字秉貞,海寧人。兄弟並以布衣終。沈愚,字通理,崑山人,業醫終其身。王淮,字柏源,慈谿人。晏鐸,字振之,富順人。由庶吉士授御史,歷按兩畿、山東,所至有聲。坐言事謫上高典史,鄰境寇發,官兵不能討,鐸捕滅之,歸所掠於民。鄒亮,字克明,長洲人。用況鐘薦,擢吏部司務,遷御史。蔣忠,字主忠,儀真人,徙居句容。王貞慶,字善甫,駙馬都尉寧子也。折節好士,有詩名,時稱金粟公子。

張弼,字汝弼,松江華亭人。成化二年進士。授兵部主事,進員外郎。遷南安知府,地當兩廣衝,奸人聚山谷為惡,悉捕滅之。毀淫祠百數十區,建為社學。謝病歸,士民為立祠。弼自幼穎拔,善詩文,工草書,怪偉跌宕,震撼一世。自號東海。張東海之名,流播外裔。為詩,信手縱筆,多不屬稿,即有所屬,以書故,輒為人持去。與李東陽、謝鐸善。嘗自言:「吾平生,書不如詩,詩不如文。」東陽戲之曰:「英雄欺人每如此,不足信也。」鐸稱其好學不倦,詩文成一家言。子弘至,自有傳。

張泰,字亨父,太倉人。陸釴,字鼎儀,崑山人。陸容,字文量,亦太倉人。三人少齊名,號「婁東三鳳」。泰舉天順八年進士,選庶吉士,授檢討,遷修撰。為人恬淡自守,詩名亞李東陽。弘治間,藝苑皆稱李懷麓、張滄洲,東陽有《懷麓堂集》,泰有《滄洲集》也。釴與泰同年進士,殿試第二。授編修,歷修撰、諭德。孝宗立,以東宮講讀勞,進太常少卿兼侍讀,得疾歸。泰、釴皆早卒。容,成化中進士。授南京主事,進兵部職方郎中。西番進獅子,奏請大臣往迎,容諫止之。遷浙江參政,罷歸。

程敏政,字克勤,休寧人,南京兵部尚書信子也。十歲侍父官四川,巡撫羅綺以神童薦。英宗召試,悅之,詔讀書翰林院,給廩饌。學士李賢、彭時咸愛重之,賢以女妻焉。成化二年進士及第,授編修,歷左諭德,直講東宮。翰林中,學問該博稱敏政,文章古雅稱李東陽,性行真純稱陳音,各為一時冠。孝宗嗣位,以宮僚恩擢少詹事兼侍講學士,直經筵。

敏政名臣子,才高負文學,常俯視儕偶,頗為人所疾。弘治元年冬,御史王嵩等以雨災劾敏政,因勒致仕。五年起官,尋改太常卿兼侍讀學士,掌院事。進禮部右侍郎,專典內閣誥敕。十二年與李東陽主會試,舉人徐經、唐寅預作文,與試題合。給事中華昶劾敏政鬻題,時榜未發,詔敏政毋閱卷,其所錄者令東陽會同考官覆校。二人卷皆不在所取中,東陽以聞,言者猶不已。敏政、昶、經、寅俱下獄,坐經嘗贄見敏政,寅嘗從敏政乞文,黜為吏,敏政勒致仕,而昶以言事不實調南太僕主簿。敏政出獄憤恚,發癰卒。後贈禮部尚書。或言敏政之獄,傅瀚欲奪其位,令昶奏之。事秘,莫能明也。

羅,字景鳴,南城人。博學,好古文,務為奇奧。年四十困諸生,輸粟入國學。丘浚為祭酒,議南人不得留北監。固請不已,浚罵之曰:「若識幾字,倔彊乃爾!」仰對曰:「惟中秘書未讀耳。」濬姑留之,他日試以文,乃大驚異。成化末,領京闈鄉試第一。明年舉進士,選庶吉士,授編修。益肆力古文,每有作,或據高樹,或閉坐一室,瞑目隱度,形容灰槁。自此文益奇,亦厚自負。

尤尚節義。臺諫救劉遜盡下獄,言當優容以全國體。中官李廣死,遺一籍,具識大臣賄交者。帝怒,命言官指名劾奏。上言曰:「大臣表正百僚,今若此,固宜置重典。然天下及四裔皆仰望之,一旦指名暴其惡,啟遠人慢朝廷心。言官未見籍記,憑臆而論,安辨玉石?一經攻摘,且玷終身。臣請降敕密諭,使引疾退,或斥以他事,庶不為朝廷羞,而仕路亦清。」李夢陽下獄,言:「壽寧侯託肺腑,當有以保全之。夢陽不保,為侯累。」帝深納焉。秩滿,進侍讀。

正德初,遷南京太常少卿。劉瑾亂政,李東陽依違其間。,東陽所舉士也,貽書責以大義,且請削門生之籍。尋進本寺卿,擢南京吏部右侍郎。遇事嚴謹,僚屬畏憚。畿輔盜縱橫,而皇儲未建,疏論激切,且侵執政者。七年冬,考績赴都,遂引疾致仕歸。寧王宸濠慕其名,遣使饋,避之深山。及叛,已病,馳書守臣約討賊,事未舉而卒。嘉靖初,賜謚文肅,學者稱圭峰先生。

儲巏,字靜夫,泰州人。九歲能屬文。母疾,刲股療之,卒不起。家貧,力營墓域。旦哭塚,夜讀書不輟。成化十九年鄉試,明年會試,皆第一。授南京考功主事。孝宗嗣位,疏薦前直諫貶謫者,主事張吉、王純,中書舍人丁璣,進士李文祥,吉等皆錄用。久之,進郎中。吏部尚書耿裕知其賢,調北部,考注臧否,一出至公。嘗核實一官,裕欲改其評,巏正色曰:「公所執,何異王介甫!」群僚咸在側,裕大慚,徐曰:「郎中言是,然非我莫能容也。」擢太僕少卿,請命史官記注言動,如古左右史,時不能用。進本寺卿。武宗立,塞上有警,條禦邊五事,又陳馬政病民者四事,多議行。正德二年改左僉都御史,總督南京糧儲。召為戶部右侍郎,尋轉左,督倉場,所至宿弊盡釐。劉瑾用事,數陵侮大臣,獨敬巏,稱為先生。巏憤其所為,五年春,引疾求去。詔許乘傳,有司俟疾痊以聞。其秋,瑾敗,以故官召,辭不赴。後起南京戶部左侍郎,就改吏部,卒官。

巏體貌清羸,若不勝衣;淳行清修,介然自守。工詩文。好推引知名士,闢遠非類,不惡而嚴。進士顧璘嘗謁尚書邵寶,寶語曰:「子立身,當以柴墟為法。」柴墟者,巏別號也。嘉靖初,賜謚文懿。

李夢陽,字獻吉,慶陽人。父正,官周王府教授,徙居開封。母夢日墮懷而生,故名夢陽。弘治六年舉陜西鄉試第一,明年成進士,授戶部主事。遷郎中,榷關,格勢要,構下獄,得釋。

十八年,應詔上書,陳二病、三害、六漸,凡五千餘言,極論得失。末言:「壽寧侯張鶴齡招納無賴,罔利賊民,勢如翼虎。」鶴齡奏辨,摘疏中「陛下厚張氏」語,誣夢陽訕母后為張氏,罪當斬。時皇后有寵,后母金夫人泣愬帝,帝不得已繫夢陽錦衣獄。尋宥出,奪俸。金夫人愬不已,帝弗聽,召鶴齡閒處,切責之,鶴齡免冠叩頭乃已。左右知帝護夢陽,請毋重罪,而予杖以洩金夫人憤。帝又弗許,謂尚書劉大夏曰:「若輩欲以杖斃夢陽耳,吾寧殺直臣快左右心乎!」他日,夢陽途遇壽寧侯,詈之,擊以馬箠,墮二齒,壽寧侯不敢校也。

孝宗崩,武宗立,劉瑾等八虎用事,尚書韓文與其僚語及而泣。夢陽進曰:「公大臣,何泣也?」文曰:「奈何?」曰:「比言官劾群奄,閣臣持其章甚力,公誠率諸大臣伏闕爭,閣臣必應之,去若輩易耳。」文曰:「善」,屬夢陽屬草。會語洩,文等皆逐去。瑾深憾之,矯旨謫山西布政司經歷,勒致仕。既而瑾復摭他事下夢陽獄,將殺之,康海為說瑾,乃免。瑾誅,起故官,遷江西提學副使。令甲,副使屬總督,夢陽與相抗,總督陳金惡之。監司五日會揖巡按御史,夢陽又不往揖,且敕諸生毋謁上官,即謁,長揖毋跪。御史江萬實亦惡夢陽。淮王府校與諸生爭,夢陽笞校。王怒,奏之,下御史按治。夢陽恐萬實右王,訐萬實。詔下總督金行勘,金檄布政使鄭岳勘之。夢陽偽撰萬實劾金疏以激怒金,並構岳子涷通賄事。寧王宸濠者浮慕夢陽,嘗請撰《陽春書院記》,又惡岳,乃助夢陽劾岳。萬實復奏夢陽短,及偽為奏章事。參政吳廷舉亦與夢陽有隙,上疏論其侵官,不俟命徑去。詔遣大理卿燕忠往鞫,召夢陽,羈廣信獄。諸生萬餘為訟冤,不聽。劾夢陽陵轢同列,挾制上官,遂以冠帶閒住去。亦褫岳職,謫戍澐,奪廷舉俸。

夢陽既家居,益跅弛負氣,治園池,招賓客,日縱俠少射獵繁臺、晉丘間,自號空同子,名震海內。宸濠反誅,御史周宣劾夢陽黨逆,被逮。大學士楊廷和、尚書林俊力救之,坐前作《書院記》,削籍。頃之卒。子枝,進士。

夢陽才思雄鷙,卓然以復古自命。弘治時,宰相李東陽主文柄,天下翕然宗之,夢陽獨譏其萎弱。倡言文必秦、漢,詩必盛唐,非是者弗道。與何景明、徐禎卿、邊貢、朱應登、顧璘、陳沂、鄭善夫、康海、王九思等號十才子,又與景明、禎卿、貢、海、九思、王廷相號七才子,皆卑視一世,而夢陽尤甚。吳人黃省曾、越人周祚,千里致書,願為弟子。迨嘉靖朝,李攀龍、王世貞出,復奉以為宗。天下推李、何、王、李為四大家,無不爭效其體。華州王維楨以為七言律自杜甫以後,善用頓挫倒插之法,惟夢陽一人。而後有譏夢陽詩文者,則謂其模擬剽竊,得史遷、少陵之似,而失其真云。

康海,字德涵,武功人。弘治十五年殿試第一,授修撰。與夢陽輩相倡和,訾議諸先達,忌者頗眾。正德初,劉瑾亂政。以海同鄉,慕其才,欲招致之,海不肯往。會夢陽下獄,書片紙招海曰:「對山救我。」對山者,海別號也。海乃謁瑾,瑾大喜,為倒屣迎。海因設詭辭說之,瑾意解,明日釋夢陽。踰年,瑾敗,海坐黨,落職。

王九思,字敬夫,鄠人。弘治九年進士。由庶吉士授檢討。尋調吏部,至郎中,亦以瑾黨謫壽州同知。復被論,勒致仕。

海、九思同里、同官,同以瑾黨廢。每相聚沜東鄠、杜間,挾聲伎酣飲,製樂造歌曲,自比俳優,以寄其怫鬱。九思嘗費重貲購樂工學琵琶。海搊彈尤善。後人傳相仿效,大雅之道微矣。

王維楨,字允寧。嘉靖十四年進士。擢庶吉士,累官南京國子祭酒。家居,地大震,壓死。維楨頎而晰,自負經世才,職文墨,不得少效於世,使酒謾罵,人多畏而遠之。於文好司馬遷,於詩好杜甫,而其意以夢陽兼此二人。終身所服膺效法者,夢陽也。

何景明,字仲默,信陽人。八歲能詩古文。弘治十一年舉於鄉,年方十五,宗籓貴人爭遺人負視,所至聚觀若堵。十五年第進士,授中書舍人。與李夢陽輩倡詩古文,夢陽最雄駿,景明稍後出,相與頡頏。正德改元,劉瑾竊柄。上書吏部尚書許進勸其秉政毋撓,語極激烈。已,遂謝病歸。踰年,瑾盡免諸在告者官,景明坐罷。瑾誅,用李東陽薦,起故秩,直內閣制敕房。李夢陽下獄,眾莫敢為直,景明上書吏部尚書楊一清救之。九年,乾清宮災,疏言義子不當畜,邊軍不當留,番僧不當寵,宦官不當任。留中。久之,進吏部員外郎,直制敕如故。錢寧欲交歡,以古畫索題,景明曰:「此名筆,毋污人手。」留經年,終擲還之。尋擢陜西提學副使。廖鵬弟太監鑾鎮關中,橫甚,諸參隨遇三司不下馬,景明執撻之。其教諸生,專以經術世務。遴秀者於正學書院,親為說經,不用諸家訓詁,士始知有經學。嘉靖初,引疾歸,未幾卒,年三十有九。

景明志操耿介,尚節義,鄙榮利,與夢陽並有國士風。兩人為詩文,初相得甚歡,名成之後,互相詆諆。夢陽主摹仿,景明則主創造,各樹堅壘不相下,兩人交游亦遂分左右袒。說者謂景明之才本遜夢陽,而其詩秀逸穩稱,視夢陽反為過之。然天下語詩文必並稱何、李,又與邊貢、徐禎卿並稱四傑。其持論,謂:「詩溺於陶,謝力振之,古詩之法亡於謝。文靡於隋,韓力振之,古文之法亡於韓。」錢謙益撰《列朝詩》,力詆之。

徐禎卿,字昌穀,吳縣人。資穎特,家不蓄一書,而無所不通。自為諸生,已工詩歌,與里人唐寅善,寅言之沈周、楊循吉,由是知名。舉弘治十八年進士。孝宗遣中使問禎卿與華亭陸深名,深遂得館選,而禎卿以貌寢不與。授大理左寺副,坐失囚,貶國子博士。禎卿少與祝允明、唐寅、文徵明齊名,號「吳中四才子」。其為讀,喜白居易、劉禹錫。既登第,與李夢陽、何景明游,悔其少作,改而趨漢、魏、盛唐,然故習猶在,夢陽譏其守而未化。卒,年二十有三。禎卿體臒神清,詩熔煉精警,為吳中詩人之冠,年雖不永,名滿士林。子伯虯,舉人,亦能詩。

楊循吉,字君謙,吳縣人。成化二十年進士。授禮部主事。善病,好讀書,每得意,手足踔掉不能自禁,用是得顛主事名。一歲中,數移病不出。弘治初,奏乞改教,不許。遂請致仕歸,年纔三十有一。結廬支硎山下,課讀經史,旁通內典、稗官。父母歿,傾貲治葬,寢苫墓側。性狷隘,好持人短長,又好以學問窮人,至頰赤不顧。清寧宮災,詔求直言,馳疏請復建文帝尊號,格不行。武宗駐蹕南都,召賦《打虎曲》,稱旨,易武人裝,日侍御前為樂府、小令。帝以優俳畜之,不授官。循吉以為恥,閱九月辭歸。既復召至京,會帝崩,乃還。嘉靖中,獻《九廟頌》及《華陽求嗣齋儀》,報聞而已。晚歲落寞,益堅癖自好。尚書顧璘道吳,以幣贄,促膝論文,歡甚。俄郡守邀璘,璘將赴之,循吉忽色變,驅之出,擲還其幣。明日,璘往謝,閉門不納。卒,年八十九。其詩文,自定為《松籌堂集》,他所作又十餘種,幾及千卷。

祝允明,字希哲,長洲人。祖顯,正統四年進士。內侍傳旨試能文者四人,顯與焉,入掖門,知欲令教小內豎也,不試而出。由給事中歷山西參政。並有聲。允明以弘治五年舉於鄉,久之不第,授廣東興寧知縣。捕戮盜魁三十餘,邑以無警。稍遷應天通判,謝病歸。嘉靖五年卒。

允明生而枝指,故自號枝山,又號枝指生。五歲作徑尺字,九歲能詩,稍長,博覽群集,文章有奇氣,當筵疾書,思若湧泉。尤工書法,名動海內。好酒色六博,善新聲,求文及書者踵至,多賄妓掩得之。惡禮法士,亦不問生產,有所入,輒召客豪飲,費盡乃已,或分與持去,不留一錢。晚益困,每出,追呼索逋者相隨於後,允明益自喜。所著有詩文集六十卷,他雜著百餘卷。子續,正德中進士,仕至廣西左布政使。

唐寅,字伯虎,一字子畏。性穎利,與里狂生張靈縱酒,不事諸生業。祝允明規之,乃閉戶浹歲。舉弘治十一年鄉試第一,座主梁儲奇其文,還朝示學士程敏政,敏政亦奇之。未幾,敏政總裁會試,江陰富人徐經賄其家僮,得試題。事露,言者劾敏政,語連寅,下詔獄,謫為吏。寅恥不就,歸家益放浪。寧王宸濠厚幣聘之,寅察其有異志,佯狂使酒,露其醜穢。宸濠不能堪,放還。築室桃花塢,與客日般飲其中,年五十四而卒。

寅詩文,初尚才情,晚年頹然自放,謂後人知我不在此,論者傷之。吳中自枝山輩以放誕不羈為世所指目,而文才輕艷,傾動流輩,傳說者增益而附麗之,往往出名教外。

時常熟有桑悅者,字民懌,尤怪妄,亦以才名吳中。書過目,輒焚棄,曰:「已在吾腹中矣。」敢為大言,以孟子自況。或問翰林文章,曰:「虛無人,舉天下惟悅,其次祝允明,又次羅。」為諸生,上謁監司,曰「江南才子」。監司大駭,延之較書,預刊落以試悅,文義不屬者,索筆補之。年十九舉成化元年鄉試,試春官,答策語不雅訓,被斥。三試得副榜,年二十餘耳,年籍誤二為六,遂除泰和訓導。學士丘浚重其文,屬學使者善遇之。使者至,問:「悅不迎,豈有恙乎?」長吏皆銜之,曰:「無恙,自負才名不肯謁耳。」使者遣吏召不至,益兩使促之。悅怒曰:「始吾謂天下未有無耳者,乃今有之。與若期,三日後來,瀆則不來矣。」使者恚,欲收悅,緣浚故,不果。三日來見,長揖使者。使者怒,悅脫帽竟去。使者下階謝,乃已。遷長沙通判,調柳州。會外艱歸,遂不出。居家益狂誕,鄉人莫不重其文,而駭其行。初,悅在京師,見高麗使臣市本朝《兩都賦》,無有,以為恥,遂賦之。居長沙,著《庸言》,自以為窮究天人之際。所著書,頗行於世。

邊貢,字廷實,歷城人。祖寧,應天治中。父節,代州知州。貢年二十舉於鄉,第弘治九年進士。除太常博士,擢兵科給事中。孝宗崩,疏劾中官張瑜,太醫劉文泰、高廷和用藥之謬,又劾中官苗逵、保國公朱暉、都御史史琳用兵之失。改太常丞,遷衛輝知府,改荊州,並能其官。歷陜西、河南提學副使,以母憂家居。嘉靖改元,用薦,起南京太常少卿,三遷太常卿,督四夷館,擢刑部右侍郎,拜戶部尚書,並在南京。貢早負才名,美風姿,所交悉海內名士。久官留都,優閒無事,游覽江山,揮毫浮白,夜以繼日。都御史劾其縱酒廢職,遂罷歸。

顧璘,字華玉,上元人。弘治九年進士。授廣平知縣,擢南京吏部主事,晉郎中。正德四年出為開封知府,數與鎮守太監廖堂、王宏忤,逮下錦衣獄,謫全州知州。秩滿,遷台州知府。歷浙江左布政使,山西、湖廣巡撫,右副都御史,所至有聲。遷吏部右侍郎,改工部。董顯陵工畢,遷南京刑部尚書。罷歸,年七十餘卒。

璘少負才名,與何、李相上下。虛己好士,如恐不及。在浙,慕孫太初一元不可得見。道衣幅巾,放舟湖上,月下見小舟泊斷橋,一僧、一鶴、一童子煮茗,笑曰:「此必太初也。」移舟就之,遂往還無間。撫湖廣時,愛王廷陳才,欲見之,廷陳不可。偵廷陳狎游,疾掩之,廷陳避不得,遂定交。既歸,構息園,大治幸舍居客,客常滿。

從弟瑮,字英玉,以河南副使歸,居園側一小樓,教授自給。璘時時與客豪飲,伎樂雜作。呼瑮,瑮終不赴,其孤介如此。

初,璘與同里陳沂、王韋,號「金陵三俊」。其後寶應朱應登繼起,稱四大家。璘詩,矩矱唐人,以風調勝。韋婉麗多致,頗失纖弱。沂與韋同調。應登才思泉涌,落筆千言。然璘、應登羽翼李夢陽,而韋、沂則頗持異論。三人者,仕宦皆不及璘。

陳沂,字魯南。正德中進士。由庶吉士歷編修、侍講,出為江西參議,量移山東參政。以不附張孚敬、桂萼,改行太僕卿致仕。

王韋,字欽佩。父徽,成化時給事中,直諫有聲。韋舉弘治中進士,由庶吉士歷官太僕少卿。子逢元,亦能詩。

朱應登,字升之。弘治中進士,歷雲南提學副使,遷參政。恃才傲物,中飛語,罷歸。子日籓,嘉靖間進士,終九江知府。能文章,世其家。

南都自洪、永初,風雅未暢。徐霖、陳鐸、金琮、謝璿輩談藝正德時,稍稍振起。自璘王詞壇,士大夫希風附塵,厥道大彰。許穀,陳鳳,璿子少南,金大車、大輿、金鑾,盛時泰,陳芹之屬,並從之游。穀等皆里人,鑾僑居客也。儀真蔣山卿、江都趙鶴亦與璘遙相應和。沿及末造,風流未歇云。

鄭善夫,字繼之,閩縣人。弘治十八年進士。連遭內外艱,正德六年始為戶部主事,榷稅滸墅,以清操聞。時劉瑾雖誅,嬖倖用事。善夫憤之,乃告歸,築草堂金鰲峰下,為遲清亭,讀書其中,曰:「俟天下之清也。」寡交游,日晏未炊,欣然自得。起禮部主事,進員外郎。武宗將南巡,偕同列切諫,杖於廷,罰跪五日。善夫更為疏草,置懷中,屬其僕曰:「死即上之。」幸不死,歎曰:「時事若此,尚可靦顏就列哉!」乞歸未得,明年力請,乃得歸。嘉靖改元,用薦起南京刑部郎中,未上,改吏部。行抵建寧,便道游武夷、九曲,風雪絕糧,得病卒,年三十有九。善夫敦行誼,婚嫁七弟妹,貲悉推予之,葬母黨二十二人。所交盡名士,與孫一元、殷雲霄、方豪尤友善。作詩,力摹少陵。

雲霄,字近夫,壽張人,善夫同年進士。作蓄艾堂,聚書數千卷,以作者自命。正德中,官南京給事中。武宗納有娠女子馬姬宮中,雲霄偕同官疏諫,引李園、呂不韋事為諷,不報。卒官,年三十有七。鄉人穆孔暉畏雲霄峭直,曰:「殷子恥不善,不啻負穢然。」

方豪,字思道,開化人。正德三年進士。除崑山知縣,遷刑部主事。諫武宗南巡,跪闕下五日,復受杖。歷官湖廣副使,罷歸。一元,見《隱逸傳》。

閩中詩文,自林鴻、高棅後,閱百餘年,善夫繼之。迨萬曆中年,曹學佺、徐勃輩繼起,謝肇淛、鄧原岳和之,風雅復振焉。

學佺詳見後傳。勃,字興公,閩縣人。兄熥,萬曆間舉人。勃以布衣終。博聞多識,善草隸書。積書鰲峰書舍至數萬卷。

肇淛,字在杭。萬曆三十年進士。官工部郎中,視河張秋,作《北河紀略》,具載河流原委及歷代治河利病。終廣西右布政使。原岳,字汝高,亦閩縣人,肇淛同年進士,終湖廣副使。

陸深,字子淵,上海人。弘治十八年進士,二甲第一。選庶吉士,授編修。劉瑾嫉翰林官亢己,悉改外,深得南京主事。瑾誅,復職,歷國子司業、祭酒,充經筵講官。奏講官撰進講章,閣臣不宜改竄。忤輔臣,謫延平同知。晉山西提學副使,改浙江。累官四川左布政使。松、茂諸番亂,深主調兵食,有功,賜金幣。嘉靖十六年召為太常卿兼侍讀學士。世宗南巡,深掌行在翰林院印,御筆刪侍讀二字,進詹事府詹事,致仕。卒,謚文裕。深少與徐禎卿相切磨,為文章有名。工書,仿李邕、趙孟頫。嘗鑒博雅,為詞臣冠。然頗倨傲,人以此少之。

同邑有王圻者,字元翰。嘉靖四十四年進士。除清江知縣,調萬安。擢御史,忤時相,出為福建按察僉事,謫邛州判官。兩知進賢、曹縣,遷開州知州。歷官陜西布政參議,乞養歸,築室淞江之濱,種梅萬樹,目曰梅花源。以著書為事,年踰耄耋,猶篝燈帳中,丙夜不輟。所撰《續文獻通考》諸書行世。

初,圻以奏議為趙貞吉所推。張居正與貞吉交惡,諷圻攻之,不應。高拱為圻座主,時方修隙徐階,又以圻為私其鄉人不助己,不能無恚,遂摭拾之。

王廷陳,字穉欽,黃岡人。父濟,吏部郎中。廷陳穎慧絕人,幼好弄,父抶之,輒大呼曰:「大人奈何虐天下名士!」正德十二年成進士,選庶吉士,益恃才放恣。故事,兩學士為館師,體嚴重,廷陳伺其退食,獨上樹杪,大聲叫呼。兩學士無如之何,佯弗聞也。武宗下詔南巡,與同館舒芬等七人將疏諫,館師石珤力止之。廷陳賦《烏母謠》,大書於壁以刺,珤及執政皆不悅。已而疏上,帝怒,罰跪五日,杖於廷。時已改吏科給事中,乃出為裕州知州。廷陳不習為吏,又失職怨望,簿牒堆案,漫不省視。夏日裸跣坐堂皇,見飛鳥集庭樹,輒止訟者,取彈彈之。上官行部,不出迎。已而布政使陳鳳梧及巡按御史喻茂堅先後至,廷陳以鳳梧座主,特出迓。鳳梧好謂曰:「子候我固善,御史即來,候之當倍謹。」廷陳許諾。及茂堅至,銜其素驕蹇,有意裁抑之,以小過榜州吏。廷陳為跪請,茂堅故益甚。廷陳大罵曰:「陳公誤我。」直上堂搏茂堅,悉呼吏卒出,鎖其門,禁絕供億,且將具奏。茂堅大窘,鳳梧為解,乃夜馳去。尋上疏劾之,適裕人被案者逸出,奏廷陳不法事,收捕繫獄,削籍歸。世宗踐阼,前直諫被謫者悉復官,獨廷陳以畦吏議不與。

屏居二十餘年,嗜酒縱倡樂,益自放廢。士大夫造謁,多蓬髮赤足,不具賓主禮。時衣紅紫窄袖衫,騎牛跨馬,嘯歌田野間。嘉靖十八年詔修《承天大志》,巡撫顧璘以廷陳及顏木、王格薦。書成,不稱旨,賜銀幣而已。廷陳才高,詩文重當世,一時才士鮮能過之。木,應山人,官亳州知州。格,京山人,官河南僉事。

李濂,字川父,祥符人。舉正德八年鄉試第一,明年成進士。授沔陽知州,稍遷寧波同知,擢山西僉事。嘉靖五年以大計免歸,年纔三十有八。濂少負俊才,時從俠少年聯騎出城,搏獸射雉,酒酣悲歌,慨然慕信陵君、侯生之為人。一日作《理情賦》,友人左國璣持以示李夢陽,夢陽大嗟賞,訪之吹臺,濂自此聲馳河、雒間。既罷歸,益肆力於學,遂以古文名於時。初受知夢陽,後不屑附和。里居四十餘年,著述甚富。


\end{pinyinscope}