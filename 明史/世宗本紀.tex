\article{世宗本紀}

\begin{pinyinscope}
世宗欽天履道英毅神聖宣文廣武洪仁大孝肅皇帝,諱厚熜,憲宗孫也。父興獻王祐杬,國安陸,正德十四年薨。帝年十有三,以世子理國事。

十六年三月辛酉,未除服,特命襲封。丙寅,武宗崩,無嗣,慈壽皇太后與大學士楊廷和定策,遣太監谷大用、韋彬、張錦,大學士梁儲,定國公徐光祚,駙馬都尉崔元,禮部尚書毛澄,以遺詔迎王於興邸。夏四月癸未,發安陸。癸卯,至京師,止於郊外。禮官具儀,請如皇太子即位禮。王顧長史袁宗皋曰:「遺詔以我嗣皇帝位,非皇子也。」大學士楊廷和等請如禮臣所具儀,由東安門入居文華殿,擇日登極。不允。會皇太后趣群臣上箋勸進,乃即郊外受箋。是日日中,入自大明門,遣官告宗廟社稷,謁大行皇帝几筵,朝皇太后,出御奉天殿,即皇帝位。以明年為嘉靖元年,大赦天下。恤錄正德中言事罪廢諸臣,賜天下明年田租之半,自正德十五年以前逋賦盡免之。丙午,遣使奉迎母妃蔣氏。召費宏復入閣。戊申,命禮臣集議興獻王封號。五月乙卯,罷大理銀礦。丙辰,梁儲致仕。壬戌,吏部侍郎袁宗皋為禮部尚書兼文淵閣大學士,預機務。壬申,錢寧伏誅。六月戊子,江彬伏誅。乙未,縱內苑禽獸,令天下毋得進獻。丁酉,革錦衣衛冒濫軍校三萬餘人。戊戌,振江西災。壬寅,革傳陞官。癸卯,振遼東饑。秋七月壬子,進士張璁言,繼統不繼嗣,請尊崇所生,立興獻王廟於京師。初,禮臣議考孝宗,改稱興獻王皇叔父,援宋程頤議濮王禮以進,不允。至是,下璁奏,命廷臣集議。楊廷和等抗疏力爭,皆不聽。癸丑,命自今親喪不得奪情,著為令。丁巳,小王子犯莊浪,指揮劉爵禦卻之。丙子,革錦衣衛所及監局寺廠司庫、旗校、軍士、匠役投充新設者,凡十四萬八千餘人。丁丑,寧津盜起。德平知縣龔諒死之。九月乙卯,袁宗皋卒。庚午,葬毅皇帝於康陵。冬十月己卯朔,追尊父興獻王為興獻帝,祖母憲宗貴妃邵氏為皇太后,母妃為興獻后。壬午,興獻后至自安陸。十一月庚戌,振江西災。丁巳,錄平宸濠功,封王守仁新建伯。甲戌,乾清宮成。罷廣西貢香。諭各鎮巡守備官,凡額外之徵悉罷之。

嘉靖元年春正月癸丑,享太廟。己未,大祀天地於南郊。清寧宮後殿災。命稱孝宗皇考,慈壽皇太后聖母,興獻帝后為本生父母。己巳,甘州兵亂,殺巡撫都御史許銘。二月己卯,耕耤田。三月辛亥,弗提衛獻生豹,卻之。甲寅,釋奠於先師孔子。丁巳,上慈壽皇太后尊號曰昭聖慈壽皇太后,武宗皇后曰莊肅皇后。戊午,上皇太后尊號曰壽安皇太后,興獻后曰興國太后。夏四月壬辰,命各邊巡按御史三年一閱軍馬器械。秋七月己酉,以南畿、浙江、江西、湖廣、四川旱,詔撫按官講求荒政。九月辛未,立皇后陳氏。冬十月辛卯,振南畿、湖廣、江西、廣西災,免稅糧有差。壬辰,以災傷敕群臣修省。十一月庚申,壽安皇太后崩。十二月戊寅。振陜西被寇及山東礦賊流劫者。是年,琉球入貢。

二年春正月乙卯,大祀天地於南郊。丁卯,小王子犯沙河堡,總兵官杭雄戰卻之。二月癸未,振遼東饑。壬辰,總督軍務右都御史俞諫、總兵官魯綱討平河南、山東賊。三月乙巳,俺答寇大同。甲寅,武宗神主祔太廟。戊午,賜姚淶等進士及第、出身有差。夏四月壬申,以災異敕群臣修省。癸未,以宋朱熹裔孫墅為《五經》博士。癸巳,命兩京三品以上及撫、按官舉堪任守令者。五月庚午,小王子犯密雲石塘嶺,殺指揮使殷隆。六月癸丑,以災傷免嘉靖元年天下稅糧之半。秋八月辛酉,小王子犯丁字堡,都指揮王綱戰死。冬十一月丁卯,免南畿被災稅糧。己丑,振河南饑。是年,撒馬兒罕、土魯番、天方入貢。

三年春正月丙寅朔,兩畿、河南、山東、陜西同時地震。丁丑,大祀天地於南郊。丙戌,南京刑部主事桂萼請改稱孝宗皇伯考,下廷臣議。是月舉賢良對策西漢董仲舒著。以賢良對答武帝三次策問,故,朵顏入寇。二月丙午,楊廷和致仕。庚戌,南京地震。三月壬申,振淮、揚饑。辛巳,振河南饑。夏四月己酉,上昭聖皇太后尊號曰昭聖康惠慈壽皇太后。庚戌,上興國太后尊號曰本生聖母章聖皇太后。癸丑,追尊興獻帝為本生皇考恭穆獻皇帝,大赦。辛酉,編修鄒守益請罷興獻帝稱考立廟,下錦衣衛獄。五月乙丑,蔣冕致仕。修撰呂柟言大禮未正,下錦衣衛獄。丁丑,遣使迎獻皇帝神主於安陸。己卯,吏部尚書石珤兼文淵閣大學士,預機務。六月,御史段續、陳相請正席書、桂萼罪,吏部員外郎薛蕙上《為人後解》,鴻臚少卿胡侍言張璁等議禮之失,俱下獄。秋七月乙亥,更定章聖皇太后尊號,去本生之稱。戊寅,廷臣伏闕固爭,下員外郎馬理等一百三十四人錦衣衛獄。癸未,杖馬理等於廷,死者十有六人。甲申,奉安獻皇帝神主於觀德殿。己丑,毛紀致仕。辛卯,杖修撰楊慎,檢討王元正,給事中劉濟、安磐、張漢卿、張原,御史王時柯於廷。原死,慎等戍謫有差。是月,免南畿、河南被災稅糧。八月癸巳,大同兵變,殺巡撫都御史張文錦。乙卯,吏部侍郎賈詠為禮部尚書兼文淵閣大學士,預機務。九月丙寅,定稱孝宗為皇伯考,昭聖皇太后為皇伯母,獻皇帝為皇考,章聖皇太后為聖母。丙子,詔天下。丙戌,土魯番入寇,圍肅州。兵部尚書金獻民總制軍務,署都督僉事杭雄充總兵官,太監張忠提督軍務,禦之。冬十一月己卯,戶部侍郎胡瓚提督宣、大軍務,都督魯綱充總兵官,討大同叛卒。十二月壬子,甘、涼寇退,召金獻民還。戊午,起致仕大學士楊一清為兵部尚書,總制陜西三邊軍務。是年,琉球入貢,魯迷國貢獅子、犀牛。

四年春正月丙寅,西海卜兒孩犯甘肅,總兵官姜奭擊敗之。辛未,大祀天地於南郊。二月乙卯,禁淹獄囚。三月壬午,仁壽宮災。夏五月甲戌,賜廬州知府龍誥官秩,詔天下仿誥備荒振濟法。庚辰,作世廟祀獻皇帝。八月戊子,作仁壽宮。冬十月丁亥,作玉德殿,景福、安喜二宮。十二月辛丑,《大禮集議》成,頒示天下。閏月乙卯朔,日有食之。乙亥,振遼東災。是年,天方入貢。

五年春正月乙未,大祀天地於南郊。二月甲寅,命道士邵元節為真人。庚辰,免山西被災稅糧。壬午,振京師饑。三月辛丑,賜龔用卿等進士及第、出身有差。丁未,定有司久任法。夏五月庚子,楊一清復入閣。秋七月庚寅,免四川被災稅糧。八月丙寅,振湖廣饑。九月己亥,章聖皇太后有事於世廟。冬十月辛亥朔,親享如太廟禮。壬子,振南畿、浙江災,免稅糧物料。庚午,頒御製《敬一箴》於學宮。是年,暹羅入貢。

六年春正月癸未,命群臣陳民間利病。己丑,大祀天地於南郊。二月辛亥,小王子犯宣府,參將王經戰死。癸亥,費宏、石珤致仕。庚午,召謝遷復入閣。三月庚辰,寇復犯宣府,參將關山戰死。甲午,禮部侍郎翟鑾為吏部侍郎兼翰林學士,入閣預機務。夏四月己巳,免廣西被災稅糧。五月丁丑朔,日有食之。丁亥,前南京兵部尚書王守仁兼左都御史,總制兩廣、江西、湖廣軍務,討田州叛蠻。秋八月庚戌,以議李福達獄,下刑部尚書顏頤壽、左都御史聶賢、大理寺卿湯沐等於錦衣衛獄,侍郎桂萼、張璁,少詹事方獻夫署三法司,雜治之。總制尚書王憲擊敗小王子於石臼墩。癸亥,賈詠致仕。庚午,振湖廣水災。九月己卯,免江西、河南、山西被災秋糧。壬午,頒《欽明大獄錄》於天下。冬十月戊申,兵部侍郎張璁為禮部尚書兼文淵閣大學士,預機務。是年,魯迷入貢。

七年春正月癸未,考核天下巡撫官。丙戌,大祀天地於南郊。三月戊寅,謝遷致仕。癸巳,右都御史伍文定為兵部尚書提督軍務為性命之爭賦予新的內容。,侍郎梁材督理糧儲,討雲南叛蠻。夏四月甲寅,甘露降,告於郊廟。六月辛丑,《明倫大典》成,頒示天下。癸卯,定議禮諸臣罪,追削楊廷和等籍。丁卯,雲南蠻平。秋七月己卯,追尊孝惠皇太后為太皇太后,恭穆獻皇帝為恭睿淵仁寬穆純聖獻皇帝。辛巳,尊章聖皇太后為章聖慈仁皇太后。戊子,詔天下。八月壬子,免河南被災稅糧。九月甲戌,王守仁討廣西蠻,悉平之。壬午,振嘉興、湖州災。冬十月丁未,皇后崩。十一月丙寅,立順妃張氏為皇后。十二月丙子,小王子犯大同,指揮趙源戰死。是年,琉球入貢。

八年春正月己亥,振山西災。庚戌,大祀天地於南郊。二月癸酉,吏部尚書桂萼兼武英殿大學士,預機務。丁丑,振襄陽饑。甲申,旱,躬禱於南郊。乙酉,禱於社稷。三月丙申,葬悼靈皇后。戊戌,振河南饑。甲寅,賜羅洪先等進士及第、出身有差。秋七月甲午,以議獄不當,下郎中魏應召等於獄,右都御史熊浹削籍。八月丙子,張璁、桂萼罷。壬午,始親祭山川,著為令。九月癸巳,召張璁復入閣。癸丑,楊一清罷。是月,免兩畿、河南被災稅糧,振江西、湖廣饑。冬十月癸亥朔,日有食之。己巳,除外戚世封,著為令。十一月庚子,召桂萼復入閣。甲辰,振浙江災。戊申,禱雪。己酉,雪。丁巳,親詣郊壇告謝。百官表賀。是年,天方、撒馬兒罕、土魯番入貢。

九年春正月丁酉,大祀天地於南郊。丙午,作先蠶壇於北郊。丁巳,振山西饑。二月戊辰,耕耤田。乙亥,振京師饑。丁丑,禁官民服舍器用踰制。三月丁巳,皇后親蠶於北郊。夏四月丙戌,振延綏饑。五月己亥,更建四郊。六月癸亥,立曲阜孔、顏、孟三氏學。秋八月壬午,免江西被災稅糧。九月壬辰,罷雲南鎮守中官。乙未,免南畿被災秋糧。冬十一月辛丑,更正孔廟祀典,定孔子謚號曰至聖先師孔子。己酉,祀昊天上帝於南郊,禮成,大赦。是年,琉球入貢。

十年春正月辛卯,祈穀於大祀殿,奉太祖、太宗配。甲午,更定廟祀,奉德祖於祧廟。乙巳,桂萼致仕。二月甲戌,免廬、鳳、淮、揚被災秋糧。壬申,賜張璁名孚敬。三月戊申,罷四川分守中官。夏四月丁巳,皇后親蠶於西苑。甲子,禘於太廟。五月壬子,祀皇地祇於方澤。閏六月己丑,罷浙江、湖廣、福建、兩廣及獨石、萬全、永寧鎮守中官。秋七月癸丑,侍郎葉相振陜西饑。戊午,張孚敬罷。辛巳,鄭王厚烷獻白雀,薦之宗廟。八月辛丑,改安陸州曰承天府。九月乙丑,西苑宮殿成,設成祖位致祭,宴群臣。丙寅,禮部尚書李時兼文淵閣大學士,預機務。壬申,幸西苑,御無逸殿,命李時、翟鑾進講,宴儒臣於豳風亭。冬十一月甲寅,祀天於南郊。戊辰,免陜西被災秋糧。丁丑,召張孚敬復入閣。十二月戊子,御史喻希禮、石金因修醮請宥議禮諸臣罪,下錦衣衛獄。

十一年春正月辛未,祈穀於圜丘,始命武定侯郭勛攝事。二月戊戌,免湖廣被災稅糧。三月戊辰,賜林大欽等進士及第、出身有差。夏四月辛卯,續封常遇春、李文忠、鄧愈、湯和後為侯。五月丙子,前吏部尚書方獻夫兼武英殿大學士,預機務。六月壬午,免畿內被災秋糧。甲申,續封劉基後誠意伯。秋七月戊辰,免南畿被災夏稅。八月戊子,以星變敕群臣修省。辛丑,張孚敬罷。九月丁巳,振陜西饑。冬十月甲申,編修楊名以災異陳言,下獄謫戍。是月,免山東被災稅糧,振山西饑。十一月甲寅,四川巡撫都御史宋滄獻白兔,群臣表賀。庚申,祀天於南郊。十二月己亥,免畿內被災稅糧。是年,琉球、哈密、土魯番、天方、撒馬兒罕入貢。

十二年春正月丙午,湖南巡撫都御史吳山獻白鹿,群臣表賀。自後,諸瑞異表賀以為常。丙辰,召張孚敬復入閣。是月,免浙江、河南被災稅糧。二月乙酉,振雲南饑。三月丙辰,釋奠於先師孔子。秋八月乙未,以皇子生,詔赦天下。九月庚戌,廣東巢賊亂,提督侍郎陶諧討平之。冬十月乙亥,大同兵亂,殺總兵官李瑾,代王奔宣府。丙子,下建昌侯張延齡於獄。十一月己亥,振遼東災。癸丑,翟鑾以憂去。十二月己卯,吉囊犯寧夏,總兵官王效、副總兵梁震擊敗之。是年,土魯番、天方入貢。

十三年春正月癸卯,廢皇后張氏。壬子,立德妃方氏為皇后。二月己丑,總督宣大侍郎張瓚撫定大同亂卒。辛卯,代王返國。三月壬申,振大同被兵者。乙酉,吉囊犯響水堡,參將任傑擊敗之。夏四月己酉,方獻夫致仕。六月甲子,南京太廟災。秋八月壬子,寇犯花馬池,梁震禦卻之。冬十一月庚午,祀天於南郊。是年,琉球入貢。

十四年春正月壬申,罷督理倉場中官。丙戌,莊肅皇后崩。二月己亥,作九廟。丁未,禁冠服非制。三月戊子,葬孝靜皇后於康陵。己丑,遼東軍亂,執都御史呂經。夏四月甲午,張孚敬致仕,召費宏復入閣。丙申,賜韓應龍等進士及第、出身有差。丙午,廣寧兵亂。六月,吉囊犯大同,總兵官魯綱禦卻之。秋七月甲申,廣寧亂卒平。八月乙巳,詔九卿會推巡撫官,著為令。冬十月戊申,費宏卒。十一月乙亥,祀天於南郊。是年,烏斯藏入貢。

十五年春二月癸巳,振湖廣災。三月丙子,奉章聖皇太后如天壽山謁陵,免昌平今年稅糧三之二,賜高年粟帛。癸未,謁恭讓章皇后、景皇帝陵。是日還宮。夏四月癸巳,詔建山陵。癸卯,詣七陵祭告。癸丑,還宮。是月,吉囊犯甘、涼,總兵官姜奭擊敗之。秋九月庚午,如天壽山。丁丑,還宮。是秋,吉囊犯延綏,官軍四戰皆敗之。冬十月己亥,更定世廟為獻皇帝廟。戊申,如天壽山。壬子,還宮。十一月戊午,以皇長子生,詔赦天下。辛巳,祀天於南郊。十二月辛卯,九廟成。閏月癸亥,以定廟制,加上兩宮皇太后徽號,詔赦天下。乙丑,禮部尚書夏言兼武英殿大學士,預機務。丙寅,享九廟。是年,免山西、山東被災稅糧。琉球、烏斯藏入貢。

十六年春二月壬子,安南黎寧遣使告莫登庸之難。癸酉,如天壽山。三月甲申,還宮。丙午,幸大峪山視壽陵。夏四月癸丑,還宮。六月癸酉,吉囊寇宣府,指揮趙鏜戰死。秋八月,復寇宣府,殺參將張國輔。冬十一月,故昌國公張鶴齡下獄,瘐死。是年,土魯番、天方、撒馬兒罕入貢。

十七年春二月戊辰,如天壽山。壬申,還宮。三月壬辰,賜茅瓚等進士及第、出身有差。辛丑,咸寧侯仇鸞為征夷副將軍。充總兵官,兵部尚書毛伯溫參贊軍務,討安南莫登庸。夏四月庚戌,如天壽山。甲寅,還宮。戊午,罷安南師。甲子,禱雨於郊壇。戊辰,雨。六月,寇犯宣府,都指揮周冕戰死。丙辰,定明堂大饗禮。下戶部侍郎唐胄於獄。秋七月辛卯,開河南、雲南銀礦。癸巳,慈寧宮成。八月甲辰,吉囊犯河西,總督都御史劉天和禦卻之。丙辰,禮部尚書掌詹事府事顧鼎臣兼文淵閣大學士,預機務。九月戊寅,免畿內被災稅糧。辛巳,上太宗廟號成祖,獻皇帝廟號睿宗。遂奉睿宗神主祔太廟,躋武宗上。辛卯,大享上帝於玄極寶殿,奉睿宗配。乙未,如天壽山。丁酉,還宮。冬十一月辛未朔,詣南郊,上皇天上帝號。還詣太廟,上太祖高皇帝、高皇后尊號。辛卯,禮天於南郊。詔赦天下。乙未,免江西被災稅糧。十二月癸卯,章聖皇太后崩。壬子,如大峪山相視山陵。甲寅,還宮。乙卯,李時卒。戊午,振寧夏災。是年,琉球、土魯番入貢。

十八年春二月庚子朔,立皇子載壑為皇太子,封載為裕王,載圳景王。辛丑,詔赦天下。起黃綰為禮部尚書,宣諭安南。壬寅,起翟鑾為兵部尚書兼右都御史,充行邊使。丁未,祈穀於玄極寶殿。先賢曾子裔孫質粹為翰林院世襲《五經》博士。壬子,振遼東饑。癸丑,安南莫方瀛請降。乙卯,幸承天,太子監國。辛酉,次真定,望於北嶽。丁卯,次衛輝,行宮火。三月己巳,渡河,祭大河之神。辛未,次鈞州,望於中嶽。甲戌,免畿內被災稅糧。庚辰,至承天。辛巳,謁顯陵。甲申,享上帝於龍飛殿,奉睿宗配。秩於國社、國稷,遍群祀。戊子,御龍飛殿受賀,詔赦天下。給復承天三年,免湖廣明年田賦五之二,畿內、河南三之一。夏四月壬子,至自承天。壬戌,免湖廣被災稅糧。甲子,幸大峪山。丙寅,還宮。秋閏七月庚申,葬獻皇后於顯陵。辛酉,復命仇鸞、毛伯溫征安南。九月辛酉,如天壽山。侍郎王杲振河南饑。冬十月丙寅,還宮。十一月丙申,祀天於南郊。是年,日本、哈密入貢。

十九年春正月丙午,召翟鑾復入閣。辛亥,吉囊寇大同,殺指揮周岐。三月戊戌,詔修仁壽宮。夏六月辛巳,瓦剌部長款塞。秋七月癸卯,吉囊入萬全右衛,總兵官白爵逆戰於宣平,敗之。壬子,又敗之於桑乾河。戊午,振江西災。八月丁丑,太僕卿楊最諫服丹藥,予杖死。九月,吉囊犯固原,周尚文敗之於黑水苑。延綏總兵官任傑追擊於鐵柱泉,又敗之。己酉,召仇鸞還。冬十月庚申,罷礦場。甲子,顧鼎臣卒。十一月丙辰,慈慶宮成。是年,琉球、日本入貢。

二十年春正月,免南畿被災稅糧。二月乙丑,顯陵成,給復承天三年。丙寅,御史楊爵言時政,下錦衣衛獄。三月乙巳,賜沈坤等進士及第、出身有差。是春,吉囊寇蘭州,參將鄭東戰死。夏四月己未,莫登庸納款,改安南國為安南都統使司,以登庸為都統使。辛酉,九廟災,毀成祖、仁宗主。丙子,詔行寬恤之政。五月戊子,採木於湖廣、四川。甲寅,振遼東饑。六月,振畿內、山西饑。秋七月丁酉,俺答、阿不孩遣使款塞求貢,詔卻之。是月,免河南、陜西、山東被災稅糧。八月辛酉,昭聖皇太后崩。庚辰,夏言罷。是月,俺答、阿不孩、吉囊分道入寇,總兵官趙卿帥京營兵,都御史翟鵬理軍務,禦之。九月乙未,翊國公郭勛有罪,下獄死。辛亥,俺答犯山西,入石州。冬十月癸丑,振山西被寇者,復徭役二年。丁卯,召夏言復入閣。十一月辛卯,葬敬皇后於泰陵。丙申,免四川被災稅糧。是年,琉球入貢。

二十一年夏四月庚申,大高玄殿成。閏五月戊辰,俺答、阿不孩遣使款大同塞,巡撫都御史龍大有誘殺之。六月辛卯,俺答寇朔州。壬寅,入雁門關。丁未,犯太原。秋七月己酉朔,日有食之。夏言罷。己未,俺答寇潞安,掠沁、汾、襄垣、長子,參將張世忠戰死。八月辛巳,募兵於直隸、山東、河南。壬午,振山西被兵州縣,免田租。癸巳,禮部尚書嚴嵩兼武英殿大學士,預機務。九月癸亥,員外郎劉魁諫營雷殿,予杖下獄。冬十月丁酉,宮人謀逆伏誅,磔端妃曹氏、寧嬪王氏於市。是年,免畿內、陜西、河南、福建被災稅糧。安南入貢。

二十二年春正月丙午朔,日有食之。三月庚戌,復遣使採木湖廣。是春,俺答屢入塞。秋八月,犯延綏,總兵官吳瑛等擊敗之。冬十月,朵顏入寇,殺守備陳舜。十二月乙酉,免南畿被災稅糧。是年,占城、土魯番、撒馬兒罕、天方、烏斯藏入貢。

二十三年春正月丙寅,俺答犯黃崖口。二月戊寅,犯大水谷。三月癸丑,犯龍門所。丁巳,賜秦鳴雷等進士及第、出身有差。秋七月,俺答犯大同,總兵官周尚文戰於黑山,敗之。八月甲午,翟鑾罷。九月癸卯,免浙江被災稅糧。丁未,吏部尚書許讚兼文淵閣大學士,禮部尚書張璧兼東閣大學士,預機務。壬子,振湖廣災。冬十月戊辰,免河南被災稅糧。甲戌,小王子入萬全右衛。戊寅,掠蔚州,至於完縣。京師戒嚴。乙酉,逮總督宣大兵部尚書翟鵬、巡撫薊鎮僉都御史朱方下獄,鵬謫戍,方杖死。十一月庚子,京師解嚴。加方士陶仲文少師。十二月丙子,振江西災。是年,安南入貢,日本以無表卻之。

二十四年春二月戊申,詔流民復業,予牛種,開墾閒田者給復十年。三月壬午,逮總督宣大兵部侍郎張漢下獄,謫戍。夏五月壬戌朔,日有食之。六月壬辰,太廟成。是夏,免畿輔、山西、陜西被災稅糧。秋七秋壬戌,有事於太廟,赦徒罪以下。八月丙午,瘞暴骸。己酉,張璧卒。庚戌,俺答犯松子嶺,殺守備張文瀚。是月,犯大同,參將張鳳、指揮劉欽等戰死。九月丁丑,召夏言入閣。冬十一月辛巳,許贊罷。是年,安南、琉球、烏斯藏入貢。

二十五年春三月戊辰,四川白草番亂。夏五月戊辰,俺答款大同塞,邊將殺其使。六月甲辰,犯宣府,千戶汪洪戰死。秋七月癸酉,以醴泉出承華殿,廷臣表賀,停諸司封事二十日。嗣後,慶賀齋祀悉停封奏。是月,俺答犯延安、慶陽。八月壬子,免山東被災稅糧。九月,俺答犯寧夏。冬十月丁亥,犯清平堡,遊擊高極戰死。癸巳,代府奉國將軍充灼謀反,伏誅。甲午,殺故建昌侯張延齡。十二月丁未,免河南被災稅糧。是年,土魯番入貢。

二十六年春三月庚午,賜李春芳等進士及第、出身有差。夏四月乙巳,巡撫四川都御史張時徹、副總兵何卿討平白草叛番。己酉,俺答求貢,拒之。秋七月丙辰,河決曹縣。八月丙戌,免陜西被災稅糧。九月戊辰,戶部尚書王杲以科臣劾其通賄下獄,遣戍。閏月丙午,振成都饑。冬十一月壬午,大內火,釋楊爵於獄。乙未,皇后崩。十二月辛酉,逮甘肅總兵官仇鸞。乙亥,海寇犯寧波、台州。是年,琉球入貢。

二十七年春正月,把都兒寇廣寧,參將閻振戰死。癸未,以議復河套,逮總督陜西三邊侍郎曾銑學,杖給事中御史於廷。罷夏言。三月癸巳,殺曾銑,逮夏言。癸卯,出仇鸞於獄。夏五月丙戌,葬孝烈皇后。秋七月戊寅,京師地震。庚子,西苑進嘉穀,薦於太廟。八月丁巳,俺答犯大同,指揮顧相等戰死,周尚文追敗之於次野口。九月壬午,犯宣府,深入永寧、懷來、隆慶,守備魯承恩等戰死。乙未,免陜西被災稅糧。冬十月癸卯,殺夏言。十一月乙未,詔撫按官採生沙金。是年,日本入貢。

二十八年春二月乙巳,振陜西饑。辛亥,南京吏部尚書張治為禮部尚書兼文淵閣大學士,祭酒李本為少詹事兼翰林學士,入閣預機務。壬子,俺答犯宣府,指揮董暘等敗沒,遂東犯永寧,關南大震。乙卯,周尚文敗俺答於曹家莊。丙辰,宣府總兵官趙國忠又敗之於大滹沱。三月辛未朔,日有食之。丁亥,皇太子薨。秋七月,浙江海賊起。九月,朵顏三衛犯遼東。冬十月辛丑,免畿內被災稅糧。是年,日本、琉球入貢。

二十九年春三月壬午,賜唐汝楫等進士及第、出身有差。是月,瓊州黎賊平。夏六月丁巳,俺答犯大同,總兵官張達、副總兵林椿戰死。是夏,免陜西、河南、江北被災夏稅。秋八月丙寅,封方士陶仲文為恭誠伯。丁丑,俺答大舉入寇,攻古北口,薊鎮兵潰。戊寅,掠通州,駐白河,分掠畿甸州縣,京師戒嚴。召大同總兵官仇鸞及河南、山東兵入援。壬午,薄都城。仇鸞為平虜大將軍,節制諸路兵馬,巡撫保定都御史楊守謙提督軍務,左諭德趙貞吉宣諭諸軍。癸未,始御奉天殿,戒敕群臣。甲申,寇退。逮守通州都御史王儀。丙戌,京師解嚴。杖趙貞吉,謫外任。丁亥,仇鸞敗績於白羊口。兵部尚書丁汝夔、巡撫侍郎楊守謙有罪,棄市。杖左都御史屠僑、刑部侍郎彭黯。九月辛卯,振畿內被寇者。乙未,罷團營,復三大營舊制,設戎政府,以仇鸞總督之。丁酉,罷領營中官。戊申,免畿內被災稅糧。壬子,廢鄭王厚烷為庶人。冬十月甲戌,張治卒。十一月癸巳,分遣御史選邊軍入衛。壬寅,祧仁宗,祔孝烈皇后於太廟。是年,琉球入貢。

三十年春三月壬辰,開馬市於宣府、大同,兵部侍郎史道經理之。夏四月壬午,下經略京城副都御史商大節於獄。秋九月乙未,京師地震,詔修省。冬十一月,俺答犯大同。是年,免兩畿、河南、江西、遼東、貴州、山東、山西被災稅糧。

三十一年春正月壬辰,俺答犯大同。甲午,入弘賜堡。二月癸丑,振宣、大饑。辛酉,俺答犯懷仁川,指揮僉事王恭戰死。己巳,建內府營,操練內侍。三月戊子,大將軍仇鸞帥師赴大同。辛卯,禮部尚書徐階兼東閣大學士,預機務。夏四月丙寅,把都兒、辛愛犯新興堡,指揮王相等戰死。丙子,倭寇浙江。五月甲申,召仇鸞還。戊申,倭陷黃巖。秋七月丙申,免陜西被災夏稅。壬寅,以倭警命山東巡撫都御史王忬巡視浙江。八月己未,收仇鸞大將軍印,尋病死。乙亥,戮仇鸞屍,傳首九邊。己卯,俺答犯大同,分掠朔、應、山陰、馬邑。九月乙酉,犯山西三關。壬辰,犯寧夏。丁酉,河決徐州。庚子,兵部侍郎蔣應奎、左通政唐國卿以冒邊功杖於廷。癸卯,罷各邊馬市。冬十月己未,兵部尚書趙錦坐仇鸞黨戍邊。壬戌,免江西被災稅糧。十二月丁巳。光祿少卿馬從謙坐誹謗杖死。

三十二年春正月戊寅朔,日食,陰雲不見。己卯,侍郎吳鵬振淮、徐水災。二月甲子,倭犯溫州。壬申,俺答犯宣府,參將史略戰死。三月丁丑,振陜西饑。辛巳,吉能犯延綏,殺副總兵李梅。壬午,兵部侍郎楊博巡邊。甲申,振山東饑。甲午,賜陳謹等進士及第、出身有差。甲辰,俺答犯宣府,副總兵郭都戰死。閏三月,海賊汪直糾倭寇瀕海諸郡,至六月始去。秋七月戊午,俺答大舉入寇,犯靈丘、廣昌。乙丑,河套諸部犯延綏。己巳,俺答犯浮圖峪,遊擊陳鳳、朱玉禦之。庚午,河南賊師尚詔陷歸德及柘城、鹿邑。八月丙子,小王子犯赤城。丙申,師尚詔攻太康,官軍與戰於鄢陵,敗績。戊戌,振山東災,免稅糧。九月丙午,俺答犯廣武,巡撫都御史趙時春敗績,總兵官李淶、參將馮恩等力戰死。辛酉,以敵退告謝郊廟。冬十月甲戌,振河南、山東饑。庚子,師尚詔伏誅,賊平。辛丑,京師外城成。是年,琉球入貢。

三十三年春正月壬寅朔,以賀疏違制,杖六科給事中於廷。戊辰,官軍圍倭於南沙,五閱月不克,倭潰圍出,轉掠蘇、松。二月庚辰,官軍敗績於松江。三月乙丑,倭犯通、泰,餘眾入青、徐界。夏四月甲戌,振畿內饑。乙亥,倭犯嘉興,都司周應楨等戰死。乙酉,陷崇明,知縣唐一岑死之。五月壬寅,倭掠蘇州。丁巳,南京兵部尚書張經總督軍務,討倭。六月癸酉,俺答犯大同,總兵官岳懋戰死。己丑,侍郎陳儒振大同軍士。秋八月癸未,倭犯嘉定,官軍敗之。庚寅,復戰,敗績。九月丁卯,俺答犯古北口,總督楊博禦卻之。是年,暹羅、土魯番、天方、撒馬兒罕、烏斯藏入貢。

三十四年春正月丁酉朔,倭陷崇德,攻德清。二月丙戌,工部侍郎趙文華祭海,兼區處防倭。是月河圖洛書儒家關於天賜《周易》、《洪範》兩書的傳說。語,俺答犯薊鎮,參將趙傾葵等戰死。三月甲寅,蘇松兵備副使任環敗倭於南沙。夏四月戊子,俺答犯宣府,參將李光啟被執,不屈死。五月甲午,總督侍郎張經、副總兵俞大猷擊倭於王江涇,大破之。乙巳,倭分道掠蘇州屬縣。己酉,逮張經下獄。六月壬午,兵部侍郎楊宜總督軍務,討倭。秋七月乙巳,倭陷南陵,流劫蕪湖、太平。丙辰,犯南京。八月壬辰,蘇松巡撫都御史曹邦輔敗倭於滸墅。九月乙未,趙文華及巡按御史胡宗憲擊倭於陶宅,敗績。丙午,俺答犯大同、宣府。戊午,犯懷來,京師戒嚴。辛酉,參將馬芳敗寇於保安。是秋,免江北、山東被災秋糧。冬十月庚寅,殺張經及巡撫浙江副都御史李天寵、兵部員外郎楊繼盛。辛卯,倭掠寧波、台州,犯會稽。十一月壬辰朔,日有食之。庚申,倭犯興化、泉州。閏月丁丑,免畿內水災稅糧。十二月甲午,開山東、四川銀礦。壬寅,山西、陜西、河南地大震,河、渭溢,死者八十三萬有奇。是年,琉球入貢。

三十五年春正月壬午,官軍擊倭於松江,敗績。二月甲午,振平陽、延安災。己亥。楊宜罷。戊午,吏部尚書李默坐誹謗下錦衣衛獄,論死。巡撫侍郎胡宗憲總督軍務,討倭。三月丁丑,賜諸大綬等進士及第、出身有差。夏四月丙申,振陜西災。甲辰,倭寇無為州,同知齊恩戰死。辛亥,遊擊宗禮擊倭於崇德,敗沒。五月乙丑,趙文華提督江南、浙江軍務。丁亥,左通政王槐採礦銀於玉旺峪。六月丙申,總兵官俞大猷敗倭於黃浦。辛丑,俺答犯宣府,殺遊擊張紘。秋七月辛巳,胡宗憲破倭於乍浦。八月壬寅,詔採芝。辛亥,胡宗憲襲破海賊徐海於梁莊。九月乙丑,徽王載埨有罪,廢為庶人。免南畿被災稅糧。壬午,以平浙江倭,祭告郊廟社稷。冬十月丙戌朔,日有食之。十一月戊午,打來孫犯廣寧,總兵官殷尚質等戰死。十二月丁未,犯環慶。

三十六年春二月,俺答犯大同。三月壬午,把都兒寇遷安,副總兵蔣承勛力戰死。是月,吉能寇延綏學家、現代俄羅斯標準語的奠基人。他贊同自然神論。以唯,殺副總兵陳鳳。夏四月丙申,奉天、華蓋、謹身三殿災。壬寅,下詔引咎修齋五日,止諸司封事,停刑。五月癸丑,倭犯揚、徐,入山東界。癸亥,採木於四川、湖廣。辛未,倭犯天長、盱眙,遂攻泗州。丙子,犯淮安。六月乙酉,兵備副使于德昌、參將劉顯敗倭於安東。甲午,罷陜西礦。秋七月庚午,詔廣東採珠。九月,俺答子辛愛寇應、朔,毀七十餘堡。冬十一月丁丑,辛愛圍右衛城。是冬,免山東、浙江被災稅糧。是年,琉球入貢。

三十七年春正月癸亥,罷河南礦。三月辛未,始免三大營聽征官軍營造工役。夏四月癸未,振遼東饑。辛巳,倭分犯浙江、福建。秋八月己未,吉能犯永昌、涼州,圍甘州。冬十月癸丑,禮部進瑞芝一千八百六十本,詔廣求徑尺以上者。十一月丁亥,諭法司恤刑。是年,琉球、暹羅入貢。

三十八年春二月庚午,把都兒犯潘家口,渡灤河,逼三屯營。三月己卯,掠遷安、薊州、玉田。庚寅,賜丁士美等進士及第、出身有差。癸巳,倭犯浙東,海道副使譚綸敗之。甲午,逮浙江總兵官俞大猷。夏四月丁未,倭犯通州。甲寅,倭攻福州。庚申,倭攻淮安,巡撫鳳陽都御史李遂敗之於姚家蕩,倭退據廟灣。丙寅,副使劉景韶大破倭於印莊。五月辛巳,逮總督薊遼右都御史王忬下獄。甲午,劉景韶破倭於廟灣,江北倭平。六月乙巳,辛愛犯大同。秋八月己未,李遂、胡宗憲破倭於劉家莊。甲子,振遼東饑,給牛種。是月,俺答犯土木,遊擊董國忠等戰死。九月,犯宣府。是年,土魯番、天方、撒馬兒罕、魯迷、哈密、暹羅入貢。

三十九年春正月丙戌,俺答犯宣府。二月丁巳,南京振武營兵變,殺總督糧儲侍郎黃懋官。戊午,振順天、永平饑。倭犯潮州。三月癸未,大同總兵官劉漢襲敗兀慎於灰河。丁亥,打來孫犯廣寧,陷中前所。殺守備武守爵、黃廷勛。夏五月壬午,振山西三關饑。壬辰,盜入廣東博羅縣,殺知縣舒顓。秋七月乙丑朔,把都兒犯薊西,遊擊胡鎮禦卻之。庚午,劉漢襲俺答於豐州,破之。九月己巳,俺答犯朔州、廣武。冬十二月,土蠻犯海州東勝堡。是月,閩、廣賊犯江西。是年,免畿內、山西、山東、湖廣、陜西被災稅糧。暹羅入貢。

四十年春二月辛卯朔,日當食,不見。振山東饑。丁未,景王之國。三月壬戌,振京師饑。夏四月丁未,振山西饑。五月乙亥,李本以憂去。閏月丙辰,賊犯泰和,殺副使汪一中、指揮王應鵬。秋七月己丑朔,日有食之。庚戌,俺答犯宣府,副總兵馬芳禦卻之。九月庚子,犯居庸關,參將胡鎮禦卻之。辛丑,振南畿災。冬十一月甲午,禮部尚書袁煒為戶部尚書兼武英殿大學士,預機務。庚戌,吉能犯寧夏,進逼固原,辛亥,萬壽宮災。十二月丙寅,把都兒犯遼東蓋州。是年,烏斯藏入貢。

四十一年春三月辛卯,白兔生子,禮部請告廟,許之,群臣表賀。壬寅為一,賜申時行等進士及第、出身有差。己酉,重作萬壽宮成。夏五月壬寅,嚴嵩罷。壬子,土蠻攻湯站堡,副總兵黑春力戰死。秋九月壬午,三殿成,改奉天曰皇極,華蓋曰中極,謹身曰建極。冬十月,免南畿、江西被災稅糧。十一月乙酉,分遣御史訪求方士、法書。丁亥,逮胡宗憲,尋釋之。辛丑,吉能犯寧夏,副總兵王勛戰死。己酉,倭陷興化。是月,延綏總兵官趙岢分部出塞襲寇,敗之。免陜西、湖廣被災及福建被寇者稅糧。是年,琉球入貢。

四十二年春正月戊申,俺答犯宣府,南掠隆慶。夏四月庚申,倭犯福清,總兵官劉顯、俞大猷合兵殲之。丁卯,副總兵戚繼光破倭於平海衛。秋八月乙亥,總兵官楊照襲寇於廣寧塞外,力戰死。冬十月丁卯,辛愛、把都兒破牆子嶺入寇,京師戒嚴,詔諸鎮兵入援。戊辰,掠順義、三河,總兵官孫臏敗死。乙亥,大同總兵官姜應熊禦寇密雲,敗之。十一月丁丑,京師解嚴。是年,琉球入貢。

四十三年春正月壬辰,土蠻黑石炭寇薊鎮,總兵官胡鎮、參將白文智禦卻之。二月己酉,伊王典楧有罪,廢為庶人。戊午,倭犯仙遊,總兵官戚繼光大敗之,福建倭平。閏月丙申,盜據漳平,知縣魏文瑞死之。三月己未,官軍擊潮州倭,破之。夏四月乙亥,免畿內被災稅糧。五月壬寅朔,日有食之。乙卯,獲桃於御幄,群臣表賀。六月辛卯,倭犯海豐,俞大猷破之。冬十二月,南韶賊起,守備賀鐸、指揮蔡胤元被執死之。俺答犯山西,遊擊梁平、守備祁謀戰死。是年,西番、哈密、安南入貢,魯迷國貢獅子。

四十四年春三月丁巳,賜范應期等進士及第、出身有差。己未,袁煒致仕。辛酉,嚴世蕃伏誅。是月,土蠻犯遼東,都指揮糸泉補袞、楊維籓戰死。夏四月庚辰,吏部尚書嚴訥、禮部尚書李春芳並兼武英殿大學士,預機務。壬午,俺答犯肅州,總兵官劉承業禦卻之。六月甲戌,芝生睿宗原廟柱,告廟受賀,遂建玉芝宮。秋八月壬午,獲仙藥於御座,告廟。冬十一月癸卯,嚴訥致仕。戊申,奉安獻皇帝、后神主於玉芝宮。是年,琉球入貢。

四十五年春二月癸亥,戶部主事海瑞上疏,下錦衣衛獄。是月,俞大猷討廣東山賊,大破之。浙江、江西礦賊陷婺源。三月己未,吏部尚書郭朴兼武英殿大學士,禮部尚書高拱兼文淵閣大學士,預機務。夏四月壬戌朔,日有食之。丙戌,俺答犯遼東。六月丙子,旱,親禱雨於凝道雷軒,越三日雨,群臣表賀。秋七月乙未,俺答犯萬全右衛。冬十月丁卯,犯固原,總兵官郭江敗死。癸酉,犯偏頭關。閏月甲辰,犯大同。參將崔世榮力戰死。十一月己未,帝不豫。十二月庚子,大漸,自西苑還乾清宮。是日崩,年六十。遺詔裕王嗣位。隆慶元年正月,上尊謚,廟號世宗,葬永陵。

贊曰:世宗御極之初,力除一切弊政,天下翕然稱治。顧迭議大禮,輿論沸騰,倖臣假托,尋興大獄。夫天性至情,君親大義,追尊立廟,禮亦宜之;然升祔太廟,而躋於武宗之上,不已過乎!若其時紛紜多故,將疲於邊,賊訌於內,而崇尚道教,享祀弗經,營建繁興,府藏告匱,百餘年富庶治平之業,因以漸替。雖剪剔權奸,威柄在御,要亦中材之主也矣。



\end{pinyinscope}