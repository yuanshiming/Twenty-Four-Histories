\article{列傳第一百五}

\begin{pinyinscope}
王家屏陳于陛沈鯉于慎行李廷機吳道南

王家屏,字忠伯,大同山陰人。隆慶二年進士。選庶吉士,授編修,預修《世宗實錄》。高拱兄捷前為操江都御史,以官帑遺趙文華,家屏直書之,時拱方柄國,囑稍諱,家屏執不可。萬曆初,進修撰,充日講官。敷奏剴摯,帝嘗斂容受,稱為端士。張居正寢疾,詞臣率奔走禱祈,獨家屏不往。再遷侍講學士。十二年,擢禮部右侍郎,改吏部。甫踰月,命以左侍郎兼東閣大學士,入預機務。去史官二年即輔政,前此未有也。

申時行當國,許國、王錫爵次之,家屏居末。每議事,秉正持法,不亢不隨。越二年,遭繼母憂。詔賜銀幣,馳傳,行人護行。服甫闋,詔進禮部尚書,遣行人召還。抵京師,三月未得見。家屏以為言,請因聖節御殿受賀,畢發留中章奏,舉行冊立皇太子禮。不報。復偕同官疏請。帝乃於萬壽節強一臨御焉。俄遣中官諭家屏,獎以忠愛。家屏疏謝,復請帝勤視朝。居數日,帝為一御門延見,自是益深居不出矣。

評事雒於仁進四箴,帝將重罪之。家屏言:「人主出入起居之節,耳目心志之娛,庶官不及知、不敢諫者,輔弼之臣得先知而預諫之,故能防欲於微渺。今於仁以庶僚上言,而臣備位密勿,反緘默茍容,上虧聖明之譽,下陷庶僚蒙不測之威,臣罪大矣,尚可一日立於聖世哉!」帝不懌,留中,而於仁得善去。

十八年,以久旱乞罷,言:「邇年以來,天鳴地震,星隕風霾,川竭河涸,加以旱潦蝗螟,疫癘札瘥,調燮之難,莫甚今日。況套賊跳梁於陜右,土蠻猖獗於遼西,貢市屬國復鴟張虎視於宣、大。虛內事外,內已竭而外患未休;剝民供軍,民已窮而軍食未裕。且議論紛紜,罕持大體;簿書凌雜,只飾靡文。綱維縱弛,心妻玩之習成;名實混淆,僥倖之風啟。陛下又深居靜攝,朝講希臨。統計臣一歲間,僅兩覲天顏而已。間嘗一進瞽言,竟與諸司章奏並寢不行。今驕陽爍石,小民愁苦之聲殷天震地,而獨未徹九閽。此臣所以中夜旁皇,飲食俱廢,不能自已者也。乞賜罷歸,用避賢路。」不報。

時儲位未定,廷臣交章請冊立。其年十月,閣臣合疏以去就爭。帝不悅,傳諭數百言,切責廷臣沽名激擾,指為悖逆。時行等相顧錯愕,各具疏再爭,杜門乞去。獨家屏在閣,復請速決大計。帝乃遣內侍傳語,期以明年春夏,廷臣無所奏擾,即於冬間議行,否則待踰十五歲。家屏以口敕難據,欲帝特頒詔諭,立具草進。帝不用,復諭二十年春舉行。家屏喜,即宣示外廷,外廷歡然。而帝意實猶豫,聞家屏宣示,弗善也,傳諭詰責。時行等合詞謝,乃已。明年秋,工部主事張有德以冊立儀注請。帝復以為激擾,命止其事。國執爭去,時行被人言,不得已亦去,錫爵先以省親歸,家屏遂為首輔。以國諫疏己列名,不當獨留,再疏乞罷。不允,乃視事。家屏制行端嚴,推誠秉公,百司事一無所撓。性忠讜,好直諫。冊立期數更,中外議論紛然。家屏深憂之,力請踐大信,以塞口語,消宮闈釁。不報。

二十年春,給事中李獻可等請豫教,帝黜之。家屏封還御批力諫。帝益怒,譴謫者相屬。家屏遂引疾求罷,上言:

漢汲黯有言:「天子置公卿輔弼之臣,寧令從臾承意陷主於不義乎!」每感斯言,惕然內愧。頃年以來,九閽重閉,宴安懷毒,郊廟不饗,堂陛不交。天災物怪,罔徹宸聰;國計民生,莫關聖慮。臣備員輔弼,曠職鰥官,久當退避。今數月間,請朝講,請廟饗,請元旦受賀,請大計臨朝,悉寢不報。臣犬馬微誠,不克感回天意,已可見矣。至豫教皇儲,自宣早計,奈何厭聞直言,概加貶謫。臣誠不忍明主蒙咈諫之名,熙朝有橫施之罰,故冒死屢陳。若依違保祿,淟涊茍容,汲黯所謂「陷主不義」者,臣死不敢出此,願賜骸骨還田里。

帝得奏不下。次輔趙志皋亦為家屏具揭。帝遂責家屏希名託疾。家屏復奏,言:

名非臣所敢棄,顧臣所希者,陛下為堯、舜之主,臣為堯、舜之臣,則名垂千載,沒有餘榮。若徒犯顏觸忌,抗爭僨事,被譴罷歸,何名之有!必不希名,將使臣身處高官,家享厚祿,主愆莫正,政亂莫匡,可謂不希名之臣矣,國家奚賴焉?更使臣棄名不顧,逢迎為悅,阿諛取容,許敬宗、李林甫之姦佞,無不可為,九廟神靈必陰殛臣,豈特得罪於李獻可諸臣已哉!

疏入,帝益不悅。遣內侍至邸,責以徑駁御批,故激主怒,且託疾要君。家屏言:「言涉至親,不宜有怒。事關典禮,不宜有怒。臣與諸臣但知為宗社大計,盡言效忠而已,豈意激皇上之怒哉?」於是求去益力。或勸少需就大事。家屏曰:「人君惟所欲為者,由大臣持祿,小臣畏罪,有輕群下心。吾意大臣不愛爵祿,小臣不畏刑誅,事庶有濟耳。」遂復兩疏懇請。詔馳傳歸。家屏柄國止半載,又強半杜門,以戇直去國,朝野惜焉。閱八年,儲位始定。遣官齎敕存問,賚金幣羊酒。又二年卒,年六十八。贈少保,謚文端。熹宗立,再贈太保,任一子尚寶丞。

家屏家居時,朝鮮用兵。貽書經略顧養謙曰:「昔衛為狄滅,齊桓率諸侯城楚丘,《春秋》高其義;未聞遂與狄仇,連諸侯兵以伐之也。今第以保會稽之恥,激厲朝鮮,以城楚丘之功,獎率將吏,無為主而為客,則善矣。」養謙不能用,朝鮮兵數年無功。其深識有謀,皆此類也。

陳于陛,字元忠,大學士以勤子也。隆慶二年進士。選庶吉士,授編修。萬歷初,預修世、穆兩朝實錄,充日講官。累遷侍講學士,擢詹事,掌翰林院。疏請早建東宮。十九年,拜禮部右侍郎,領詹事府事。明年,改吏部,進左侍郎,教習庶吉士。奏言元子不當封王,請及時冊立豫教,又請早朝勤政,皆不報。又明年,進禮部尚書,仍領詹事府事。

于陛少從父以勤習國家故實。為史官,益究經世學。以前代皆修國史,疏言:「臣考史家之法,紀、表、志、傳謂之正史。宋去我朝近,制尤可考。真宗祥符間,王旦等撰進太祖、太宗兩朝正史。仁宗天聖間,呂夷簡等增入真宗朝,名《三朝國史》。此則本朝君臣自修本朝正史之明證也。我朝史籍,止有列聖實錄,正史闕焉未講。伏睹朝野所撰次,可備採擇者無慮數百種。倘不及時網羅,歲月浸邈,卷帙散脫,耆舊漸凋,事跡罕據。欲成信史,將不可得。惟陛下立下明詔,設局編輯,使一代經制典章,犁然可攷,鴻謨偉烈,光炳天壤,豈非萬世不朽盛事哉!」詔從之。二十二年三月,遂命詞臣分曹類纂,以于陛及尚書沈一貫、少詹事馮琦為副總裁,而閣臣總裁之。

其年夏,首輔王錫爵謝政,遂命于陛兼東閣大學士,入參機務。疏陳親大臣、錄遺賢、獎外吏、核邊餉、儲將才、擇邊吏六事。末言:「以肅皇帝之精明,而末年貪黜成風,封疆多事,則倦勤故也。今至尊端拱,百職不修,不亟圖更始,後將安極。」帝優詔答之,而不能用。帝以軍政失察,斥兩都言官三十餘人。于陛與同官申救至再,又獨疏請宥,俱不納。以甘肅破賊功,加太子少保。乾清、坤寧兩宮災,請面對,不報。乞罷,亦不許。其秋,二品三年滿,改文淵閣,進太子太保。時內閣四人。趙志皋、張位、沈一貫皆于陛同年生,遇事無齟齬。而帝拒諫益甚,上下否隔。于陛憂形於色,以不能補救,在直廬數太息視日影。二十四年冬,病卒於位,史亦竟罷。贈少保,謚文憲。終明世,父子為宰輔者,惟南充陳氏。世以比漢韋、平焉。沈鯉,字仲化,歸德人。祖瀚,建寧知府。鯉,嘉靖中舉鄉試。師尚詔作亂,陷歸德,已而西去。鯉策賊必再至,急白守臣,捕殺城中通賊者,嚴為守具。賊還逼,見有備,去。奸人倡言屠城,將驅掠居民,鯉請諭止之,眾始定。四十四年,成進士,改庶吉士,授檢討。大學士高拱,其座主又鄉人也,旅見外,未嘗以私謁。

神宗在東宮,鯉為講官。嘗令諸講官書扇,鯉書魏卞蘭《太子頌》以進,因命陳清大義甚悉。神宗咨美,遂蒙眷。比即位,用宮寮恩,進編修。旋進左贊善。每直講,舉止端雅,所陳說獨契帝心。帝亟稱之。連遭父母喪,帝數問沈講官何在,又問服闋期,命先補講官俟之。萬曆九年還朝。屬當輟講,特命展一日,示優異焉。

明年秋,擢侍講學士,再遷禮部右侍郎。尋改吏部,進左侍郎。屏絕私交,好推轂賢士,不使知。十二年冬,拜禮部尚書。去六品甫二年,至正卿。素負物望,時論不以為驟。久之,《會典》成,加太子少保。鯉初官翰林,中官黃錦緣同鄉以幣交,拒不納。教習內書堂,侍講筵,皆數與巨璫接,未嘗與交。及官愈高,益無所假借,雖上命及政府指,不徇也。

十四年春,貴妃鄭氏生子,進封皇貴妃。鯉率僚屬請冊建皇長子,進封其母,不許。未幾,復以為言,且請宥建儲貶官姜應麟等。忤旨譙讓。帝既卻群臣請,因詔諭少俟二三年。至十六年,期已屆,鯉執前旨固請,帝復不從。

鯉素鯁亮。其在部持典禮,多所建白。念時俗侈靡,稽先朝典制,自喪祭、冠婚、宮室、器服率定為中制,頒天下。又以士習不端,奏行學政八事。又請復建文年號,重定《景帝實錄》,勿稱戾王。大同巡撫胡來貢議移祀北岳於渾源,力駁其無據。太廟侑享,請移親王及諸功臣於兩廡,毋與帝后雜祀。進世廟諸妃葬金山者,配食永陵。諸帝陵祀,請各遣,官毋兼攝。諸王及妃墳祝版稱謂未協者,率請裁定。帝憂旱,步禱郊壇,議分遣大臣禱天下名山大川。鯉言使臣往來驛騷,恐重困民,請劉齋三日,以告文授太常屬致之,罷寺觀勿禱,帝多可其奏。鄭貴妃父成憲為父請恤,援后父永年伯例,鯉力駁之。詔畀葬資五千金,鯉復言過濫。順義王妻三娘子請封,鯉不予妃號,但稱夫人。真人張國祥言肅皇享國久長,由虔奉玄修所致,勸帝效之,鯉劾國祥詆誣導諛,請正刑辟。事亦寢。秦王誼璟故由中尉入繼,而乞封其弟郡王,中貴為請,申時行助之,鯉不可。唐府違帛請封妾子,執不從,帝並以特旨許之。京師久旱,鯉備陳恤民實政以崇儉戒奢為本,且請減織造。已,京師地震,又請謹天若戒,恤民窮。畿輔大侵,請上下交修,詞甚切。帝以四方災,敕廷臣修省,鯉因請大損供億營建,振救小民。帝每嘉納。

初,籓府有所奏請,賄中貴居間,禮臣不敢違,輒如志。至鯉,一切格之,中貴皆大怨,數以事間於帝。帝漸不能無疑,累加詰責,且奪其俸。鯉自是有去志。而時行銜鯉不附己,亦忌之。一日,鯉請告,遽擬旨放歸。帝曰:「沈尚書好官,奈何使去?」傳旨諭留。時行益忌。其私人給事中陳與郊為人求考官不得,怨鯉,屬其同官陳尚象劾之。與郊復危言撼鯉,鯉求去益力。帝有意大用鯉,微言:「沈尚書不曉人意。」有老宮人從子為內豎者,走告鯉;司禮張誠亦屬鯉鄉人內豎廖某密告之。鯉並拒之,曰:「禁中語,非所敢聞。」皆恚而去。鯉卒屢疏引疾歸。累推內閣及吏部尚書,皆不用。二十二年,起南京禮部尚書,辭弗就。

二十九年,趙志皋卒,沈一貫獨當國。廷推閣臣,詔鯉以故官兼東閣大學士,入參機務,與朱賡並命。屢辭不允。明年七月始入朝,時年七十有一矣。一貫以士心夙附鯉,深忌之,貽書李三才曰:「歸德公來,必奪吾位,將何以備之?」歸德,鯉邑名,欲風鯉辭召命也。三才答書,言鯉忠實無他腸,勸一貫同心。一貫由此並憾三才。鯉既至,即具陳道中所見礦稅之害。他日復與賡疏論。皆弗納。楚假王被訐事起,禮部侍郎郭正域請行勘,鯉是之。及奸人所撰《續憂危竑議》發,一貫輩張皇其事,令其黨錢夢皋誣奏正域、鯉門生,協造妖言,並羅織鯉奸贓數事。帝察其誣,不問。而一貫輩使邏卒日夜操兵圍守其邸。已而事解,復譖鯉詛咒。鯉嘗置小屏閣中,列書謹天戒、恤民窮、開言路、發章奏、用大僚、補庶官、起廢棄、舉考選、釋冤獄、撤稅使十事,而上書「天啟聖聰,撥亂反治」八字。每入閣,輒焚香拜祝之,讒者遂指為詛咒。帝取入視之,曰:「此豈詛咒耶?」讒者曰:「彼詛咒語,固不宣諸口。」賴帝知鯉深,不之信。

先是,閣臣奏揭不輕進,進則無不答者。是時中外扞格,奏揭繁,多寢不下。鯉以失職,累引疾求退。獎諭有加,卒不能行其所請。三十二年,敘皮林功,加太子太保。尋以秩滿,加少保,改文淵閣。

鯉初相,即請除礦稅。居位數年,數以為言。會長陵明樓災,鯉語一貫、賡各為奏,俟時上之。一日大雨,鯉曰:「可矣。」兩人問故,鯉曰:「帝惡言礦稅事,疏入多不視,今吾輩冒雨素服詣文華奏之,上訝而取閱,亦一機也。」兩人從其言。帝得疏,曰:「必有急事。」啟視,果心動,然不為罷。明年長至,一貫在告,鯉、賡謁賀仁德門。帝賜食,司禮太監陳矩侍,小璫數往來竊聽,且執筆以俟。鯉因極陳礦稅害民狀,矩亦戚然。鯉復進曰:「礦使出,破壞天下名山大川靈氣盡矣,恐於聖躬不利。」矩嘆息還,具為帝道之。帝悚然遣矩咨鯉所以補救者。鯉曰:「此無他,急停開鑿,則靈氣自復。」帝聞,為首肯。一貫慮鯉獨收其功,急草疏上。帝不懌,復止。然越月果下停礦之命,鯉力也。

鯉遇事秉正不撓。壓於一貫,志不盡行。而是時一貫數被論,引疾杜門,鯉乃得行閣事。皇孫生,詔赦天下。中官請徵茶蠟夙逋,鯉以戾詔旨,再執奏,竟報寢,帝乳母翊聖夫人金氏,其夫官都督同知,歿,請以從子繼。鯉言都督非世官,乃已。真人張國祥謂皇孫誕生,己有祝釐功,乞三代誥命且世襲詹事主簿。鯉力斥其謬,乃賚以金幣。帝惑中貴言,將察核畿輔牧地,諭鯉撰敕。鯉言:「近年以來,百利之源,盡籠於朝廷,常恐勢極生變。況此牧地,豈真有豪右隱占新墾未科者?奸民所傳,未足深信。」遂止。雲南武弁殺稅使楊榮。帝怒甚,將遣官逮治。鯉具陳榮罪狀,請誅為首殺榮者,而貸其餘,乃不果逮。陜西稅使梁永求領鎮守事,亦以鯉言罷。遼東稅使高淮假進貢名,率所統練甲至國門。鯉中夜密奏其不可,詔責淮而止。時一貫雖稱疾杜門,而章奏多即家擬旨,鯉力言非故事。

鯉既積忤一貫,一貫將去,慮鯉在,貽己後憂欲與俱去,密傾之。帝亦嫌鯉方鯁,因鯉乞休,遽命與一貫同致仕。賡疏乞留鯉,不報。既抵家,疏謝,猶極陳怠政之弊,以明作進規。年八十,遣官存問,賚銀幣。鯉奏謝,復陳時政要務。又五年卒,年八十五。贈太師,謚文端。

于慎行,字無垢,東阿人。年十七,舉於鄉。御史欲即鹿鳴宴冠之,以未奉父命辭。隆慶二年成進士。改庶吉士,授編修。萬曆初,《穆宗實錄》成,進修撰,充日講官。故事,率以翰林大僚直日講,無及史官者。慎行與張位及王家屏、沈一貫、陳于陛咸以史官得之,異也。嘗講罷,帝出御府圖畫,令講官分題。慎行不善書,詩成,屬人書之,具以實對。帝悅,嘗大書「責難陳善」四字賜之,詞林傳為盛事。

御史劉臺以劾張居正被逮,僚友悉避匿,慎行獨往視之。及居正奪情,偕同官具疏諫。呂調陽格之,不得上。居正聞而怒,他日謂慎行曰:「子吾所厚,亦為此耶?」慎行從容對曰:「正以公見厚故耳。」居正怫然。慎行尋以疾歸。居正卒,起故官。進左諭德,日講如故。時居正已敗,侍郎丘橓往籍其家。慎行遺書,言居正母老,諸子覆巢之下,顛沛可傷,宜推明主帷蓋恩,全大臣簪履之誼。詞極懇摯,時論韙之。由侍講學士擢禮部右侍郎。轉左,改吏部,掌詹事府。尋遷禮部尚書。慎行明習典制,諸大禮多所裁定。先是,嘉靖中孝烈后升祔,祧仁宗。萬曆改元,穆宗升祔,復祧宣宗。慎行謂非禮,作《太廟祧遷考》,言:「古七廟之制,三昭三穆,與太祖之廟而七。劉歆、王肅並以高、曾、祖、禰及五世、六世為三昭三穆。其兄弟相傳,則同堂異室,不可為一世。國朝成祖既為世室,與太祖俱百世不遷,則仁宗以下,必實歷六世,而後三昭三穆始備。孝宗與睿宗兄弟,武宗與世宗兄弟,韶穆同,不當各為一世。世宗升,距仁宗止六世,不當祧仁宗。穆宗升祔,當祧仁宗,不當祧宣宗。」引晉、唐、宋故事為據,其言辨而核。事雖不行,識者服其知禮。又言:「南昌、壽春等十六王,世次既遠,宜別祭陵園,不宜祔享太廟。」亦寢不行。

十八年正月,疏請早建東宮,出閣講讀。及冬,又請。帝怒,再嚴旨詰責。慎行不為懾,明日復言:「冊立臣部職掌,臣等不言,罪有所歸。幸速決大計,放歸田里。」帝益不悅,責以要君疑上,淆亂國本,及僚屬皆奪俸。山東鄉試,預傳典試者名,已而果然。言者遂劾禮官,皆停俸。慎行引罪乞休。章累上,乃許。家居十餘年,中外屢薦,率報寢。三十三年,始起掌詹事府。疏辭,復留不下。居二年,廷推閣臣七人,首慎行。詔加太子少保兼東閣大學士,入參機務。再辭不允,乃就道。時慎行已得疾。及廷謝,拜起不如儀,上疏請罪。歸臥於家,遂草遺疏,請帝親大臣、錄遣逸、補言官。數日卒,年六十三。贈太子太保,謚文定。

慎行學有原委,貫穿百家。神宗時,詞館中以慎行及臨朐馮琦文學為一時冠。李廷機,字爾張,晉江人。貢入太學,順天鄉試第一。萬歷十一年,會試復第一,以進土第二授編修。累遷祭酒。故事,祭酒每視事,則二生共舉一牌詣前,大書「整齊嚴肅」四字。蓋高皇帝所製,以警師儒者。廷機見之惕然,故其立教,一以嚴為主。

久之,遷南京吏部右侍郎,署部事。二十七年,典京察,無偏私。嘗兼署戶、工二部事,綜理精密。奏行軫恤行戶四事,商困大蘇。外城陵垣,多所繕治,費皆取公帑奇羨,不以煩民。召為禮部右侍郎,四辭不允,越二年始受任。時已進左侍郎,遂代郭正域視部事。會楚王華奎因正域發其餽遺書,誣訐正域不法數事。廷機意右楚王,而微為正域解。大學士沈一貫欲藉妖書傾正域,廷機與御史沈裕、同官涂宗浚俱署名上趣定皦生光獄,株連遂絕。三十三年夏,雷震郊壇。既率同列條上修省事宜,復言今日闕失,莫如礦稅,宜罷撤。不報。其冬,類上四方災異。秦王誼漶由中尉進封,其庶長子應授本爵,夤緣欲封郡王,廷機三疏力持。王遣人居間,廷機固拒,特旨許之。益府服內請封,亦持不可。

廷機遇事有執,尤廉潔,帝知之。然性刻深,亦頗偏愎,不諳大體。楚宗人華勣以奏訐楚王,撫按官既擬奪爵,錮高牆,廷機授《祖訓》謀害親王例,議置之死。言路勢張,政府暨銓曹畏之,不敢出諸外,年例遂廢。禮部主事聶雲翰論之,廷機希言路意,中雲翰察典。給事中袁懋謙劾之。廷機求退,不允。

時內閣止朱賡一人。給事中王元翰等慮廷機且入輔,數陰詆之。三十五年夏,廷推閣臣,廷機果與焉。給事中曹于忭、宋一韓、御史陳宗契不可。相持久之,卒列以上。帝雅重廷機,命以禮部尚書兼東閣大學士,入參機務。廷機三辭始視事。元翰及給事中胡忻攻之不已,帝為奪俸,以慰廷機。已而姜士昌、宋燾復以論廷機被黜,群情益憤。廷機力辨求罷,又疏陳十宜去,帝慰諭有加。明年四月,主事鄭振先論賡十二罪,並及廷機。廷機累疏乞休,杜門數月不出。言者疑其偽,數十人交章力攻。廷機求去不已,帝屢詔勉留,且遣鴻臚趣出,堅臥不起。待命踰年,乃屏居荒廟,廷臣猶有繁言。至四十年九月,疏已百二十餘上,乃陛辭出都待命。同官葉向高言廷機已行,不可再挽,乃加太子太保。賜道里費,乘傳,以行人護歸。居四年卒。贈少保,謚文節。

廷機繫閣籍六年,秉政止九月,無大過。言路以其與申時行、沈一貫輩密相授受,故交章逐之。輔臣以齮晷受辱,屏棄積年而後去,前此未有也。廷機輔政時,四川巡撫喬璧星銳欲討鎮雄安堯臣,與貴州守臣持議不決。廷機力主撤兵,其後卒無事,議者稱之。閩人入閣,自楊榮、陳山後,以語言難曉,垂二百年無人,廷機始與葉向高並命。後周如磐、張瑞圖、林釬、蔣德璘、黃景昉復相繼云。

吳道南,字會甫,崇仁人。萬曆十七年進士及第。授編修,進左中允。直講東宮,太子偶旁矚,道南即輟講拱俟,太子為改容。歷左諭德少詹事。擢禮部右侍郎,署部事。歷城、高苑牛產犢,皆兩首兩鼻,道南請盡蠲山東諸稅,召還內臣,又因災異言貂璫斂怨,乞下詔罪己,與天下更新。皆不報。尋請追謚建文朝忠臣。京師久旱,疏言:「天下人情鬱而不散,致成旱災。如東宮天下本,不使講明經術,練習政務,久置深闈,聰明隔塞,鬱一也。法司懸缺半載,讞鞫無人,囹圄充滿,有入無出,愁憤之氣,上薄日星,鬱二也。內藏山積,而閭閻半菽不充,曾不發帑振救,坐視其死亡轉徙,鬱三也。纍臣滿朝薦、卞孔時,時稱循吏,因權璫構陷,一繫數年,鬱四也。廢棄諸臣,實堪世用,一斥不復,山林終老,鬱五也。陛下誠渙發德音,除此數鬱,不崇朝而雨露遍天下矣。」帝不省。

道南遇事有操執,明達政體。朝鮮貢使歸,請市火藥,執不予。土魯番貢玉,請勿納。遼東議開科試士,以巖疆當重武,格不行。父喪歸。服闋,即家拜禮部尚書兼東閣大學士,預機務,與方從哲並命。三辭不允,久之始入朝。故事,廷臣受官,先面謝乃蒞任。帝不視朝久,皆先蒞任。道南至,不獲見,不敢入直。同官從哲為言,帝令先視事,道南疏謝。居數日,言:「臣就列經旬,僅下瑞王婚禮一疏。他若儲宮出講、諸王豫教、簡大僚、舉遺失、撤稅使、補言官諸事,廷臣舌敝以請者,舉皆杳然,豈陛下簡置臣等意。」帝優詔答之,卒不行。迨帝因「梃擊」之變,召見群臣慈寧宮。道南始得面謝,自是不獲再見。

織造中官劉成死,遣其黨呂貴往護,貴嗾奸民留己督造。中旨許之,命草敕。道南偕從哲爭,且詢疏所從進,請永杜內降,弗聽。鄱陽故無商稅,中官為稅使,置關湖口徵課。道南極言傍湖舟無所泊,多覆沒,請罷關勿征,亦不納。

道南輔大政不為詭隨,頗有時望。歲丙辰,偕禮部尚書劉楚先典會試。吳江舉人沈同和者,副都御史季文子,目不知書,賄禮部吏,與同里趙鳴陽聯號舍。其首場七篇,自坊刻外,皆鳴陽筆也。榜發,同和第一,鳴陽亦中式,都下大嘩。道南等亟檢舉,詔令覆試。同和竟日構一文。下吏,戍煙瘴,鳴陽亦除名。

先是,湯賓尹科場事,實道南發之,其黨側目。御史李嵩、周師旦遂連章論道南,而給事中劉文炳攻尤力。道南疏辨乞休,頗侵文炳。文炳遂極詆御史張至發助之。道南不能堪,言:「臺諫劾閣臣,職也,未有肆口嫚罵者。臣辱國已甚,請立罷黜。」帝雅重道南,謫文炳外任,奪嵩等俸。御史韓浚、朱堦救文炳,復詆道南。道南益求去。杜門踰年,疏二十七上,帝猶勉留。會繼母訃至,乃賜道里費,遣行人護歸。天啟初,以覃恩即家進太子太保。居二年卒。贈少保,謚文恪。

贊曰:《傳》稱「道合則服從,不合則去」,其王家屏、沈鯉之謂乎!廷機雖頗叢物議,然清節不污。若於陛之世德,慎行之博聞,亦足稱羽儀廊廟之選矣。


\end{pinyinscope}