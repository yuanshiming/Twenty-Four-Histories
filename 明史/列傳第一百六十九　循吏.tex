\article{列傳第一百六十九 循吏}

\begin{pinyinscope}
明太祖懲元季吏治縱弛,民生凋敝,重繩貪吏,置之嚴典。府州縣吏來朝,陛辭,諭曰:「天下新定,百姓財力俱困,如鳥初飛,木初植,勿拔其羽,勿撼其根。然惟廉者能約己而愛人,貪者必朘人以肥己,爾等戒之。」洪武五年,下詔有司考課,首學校、農桑諸實政。日照知縣馬亮善督運,無課農興士效,立命黜之。一時守令畏法,潔己愛民,以當上指,吏治煥然丕變矣。下逮仁、宣,撫循休息,民人安樂,吏治澄清者百餘年。英、武之際,內外多故,而民心無土崩瓦解之虞者,亦由吏鮮貪殘,故禍亂易弭也。嘉、隆以後,資格既重甲科,縣令多以廉卓被徵,梯取臺省,而龔、黃之治,或未之覯焉。神宗末年,徵發頻仍,礦稅四出,海內騷然煩費,郡縣不克修舉厥職。而廟堂考課,一切以虛文從事,不復加意循良之選。吏治既以日媮,民生由之益蹙。仁、宣之盛,邈乎不可復追,而太祖之法蔑如矣。重內輕外,實政不修,謂非在上者不加之意使然乎!

漢史丞相黃霸,唐史節度使韋丹,皆入《循吏傳》中。今自守令超擢至公卿有勳德者,事皆別見,故採其終於庶僚,政績可紀者,作《循吏傳》。

○陳灌方克勤吳履廖欽等高斗南餘彥誠等史誠祖吳祥等謝子襄黃信中夏升貝秉彞劉孟雍等萬觀葉宗人王源翟溥福李信圭孫浩等張宗璉李驥王瑩等李湘趙豫趙登等曾泉范衷周濟范希正劉綱段堅陳鋼丁積田鐸唐侃湯紹恩徐九思龐嵩張淳陳幼學

陳灌,字子將,廬陵人也。元末,世將亂,環所居築場種樹,人莫能測。後十年,盜蜂起。灌率武勇結屯林中,盜不敢入,一鄉賴以全。太祖平武昌,灌詣軍門謁見。與語奇之,擢湖廣行省員外郎,累遷大都督府經歷。從大將軍徐達北征。尋命築城泰州,工竣,除寧國知府。時天下初定,民棄《詩》《書》久。灌建學舍,延師,選俊秀子弟受業。訪問疾苦,禁豪右兼并。創戶帖以便稽民。帝取為式,頒行天下。伐石築堤,作水門蓄洩,護瀕江田,百姓咸賴。有坐盜麥舟者,論死數十人。灌覆按曰:「舟自漂至,而愚民哄取之,非謀劫也。」坐其首一人,餘悉減死。灌豐裁嚴正,而為治寬恤類此。洪武四年召入京,病卒。

方克勤,字去矜,寧海人。元末,台州盜起,吳江同知金剛奴奉行省命,募水兵禦之。克勤獻策弗納,逃之山中。洪武二年辟縣訓導,母老辭歸。四年征至京師,吏部試第二,特授濟寧知府。時始詔民墾荒,閱三歲乃稅。吏征率不俟期,民謂詔旨不信,輒棄去,田復荒。克勤與民約,稅如期。區田為九等,以差等徵發,吏不得為奸,野以日闢。又立社學數百區,葺孔子廟堂,教化興起。盛夏,守將督民夫築城,克勤曰:「民方耕耘不暇,奈何重困之畚鍤。」請之中書省,得罷役。先是久旱,遂大澍。濟寧人歌之曰:「孰罷我役?使君之力。孰活我黍?使君之雨。使君勿去,我民父母。」視事三年,戶口增數倍,一郡饒足。

克勤為治,以德化為本,不喜近名,嘗曰:「近名必立威,立威必殃民,吾不忍也。」自奉簡素,一布袍十年不易,日不再肉食。太祖用法嚴,士大夫多被謫,過濟寧者,克勤輒周恤之。永嘉侯朱亮祖嘗率舟師赴北平,水涸,役夫五千浚河。克勤不能止,泣禱於天。忽大雨,水深數尺,舟遂達,民以為神。八年入朝,太祖嘉其績,賜宴,遣還郡。尋為屬吏程貢所誣,謫役江浦,復以空印事連,逮死。

子孝聞、孝孺。孝聞,十三喪母,蔬食終制。孝孺,自有傳。

吳履,字德基,蘭谿人。少受業於聞人夢吉,通《春秋》諸史。李文忠鎮浙東,聘為郡學正。久之,舉於朝,授南康丞。南康俗悍,謂丞儒也,易之。居數月,摘發奸伏如老獄吏,則皆大驚,相率斂跡。履乃改崇寬大,與民休息。知縣周以中巡視田野,為部民所詈。捕之不獲,怒,盡縶其鄉鄰。履閱獄問故,立釋之,乃白以中。以中益怒,曰:「丞慢我。」履曰:「犯公者一人耳,其鄰何罪?今縶者眾,而捕未已,急且有變,奈何?」以中意乃解。邑有淫祠,每祀輒有蛇出戶,民指為神。履縛巫責之,沉神像於江,淫祠遂絕。為丞六年,百姓愛之。

遷安化知縣。大姓易氏保險自守,江陰侯吳良將擊之,召履計事。履曰:「易氏逃死耳,非反也,招之當來。不來,誅未晚。」良從之,易氏果至。良欲籍農故為兵者,民大恐。履曰:「世清矣,民安於農。請籍其願為兵者,不願,可勿強。」遷濰州知州。山東兵常以牛羊代秋稅,履與民計曰:「牛羊有死瘠患,不若輸粟便。」他日,上官令民送牛羊之陜西,他縣民多破家,濰民獨完。會改州為縣,召履還,濰民皆涕泣奔送。履遂乞骸骨歸。

是時河內丞廖欽並以廉能稱。居八年,調吳江,後坐事謫戍。久之,以老病放歸。道河內,河內民競持羊酒為壽,且遺之縑,須臾裒數百匹。欽固辭不得,一夕遁去。

他若興化丞周舟以績最,特擢吏部主事。民爭乞留,乃遣還之。歸安丞高彬、曹縣主簿劉郁、衡山主簿紀惟正、沾化典史杜濩皆坐事,以部民乞宥,復其官,而惟正立擢陜西參議。其後州縣之佐貳知名者,在仁、宣時則易州判官張友聞、壽州判官許敏、許州判官王通、靈璧丞田誠、安平丞耿福緣、嘉定丞戴肅、大名丞賀禎、昌邑主簿劉整、襄垣主簿喬育、貴池典史黃金蘭、深澤典史高聞;英、景時則養利判官汪浩、泰州判官王思旻、上海丞張禎、吳江丞王懋本、歷城丞熊觀、黔陽主簿古初、雲南南安州瑯井巡檢李保。或超遷,或遷任,皆因部民請云。

高斗南,字拱極,陜西徽州人。貌魁梧,語音若鐘。洪武中,由薦舉授四川定遠知縣。才識精敏,多善政。二十九年,與知府永州餘彥誠,知縣齊東鄭敏、儀真康彥民、岳池王佐、安肅范志遠、當塗孟廉及丞懷寧蘇億、休寧甘鏞、當塗趙森並坐事,先後被徵。其耆民奔走闕下,具列善政以聞。太祖嘉之,賜襲衣寶鈔遣還,并賜耆民道路費。諸人既還任,政績益著。尋舉天下廉吏數人,斗南與焉,列其名於《彰善榜》、《聖政記》以示勸。九載績最,擢雲南新興知州,新興人愛之不異定遠。居數年,以衰老乞歸,薦子吏科給事中恂自代,成祖許之。年七十而卒。

恂,字士信,博學能詩文。官新興,從大軍征交址,有協贊功。師旋,卒於官。

彥誠,德興人。初知安陸州,以征稅愆期,當就逮,其父老伏闕乞留。太祖賜宴嘉賞,遣還,父老亦預宴。久之,擢知永州府,終河東鹽運使。

敏,常坐事被逮,部民數千人守闕下求宥。帝宴勞,復其官,賜鈔百錠,衣三襲。居數年,考滿入朝。部民復走京師,乞再任,帝從其請。及是,再獲宥。

彥民,泰和人。洪武二十七年進士。先知青田,調儀真,後歷巴陵、天台,並著名績。永樂初罷歸。洪熙元年,御史巡按至天台。縣民二百餘人言彥民廉公有為,乞還之天台,慰民望。御史以聞,宣宗歎曰:「彥民去天台二十餘年,民猶思之,其有善政可知。」乃用為江寧縣丞。

億、廉、森三人既釋還,明年復以事當逮。縣民又走闕下頌其廉勤,帝亦釋之。

時太祖操重典繩群下,守令坐小過輒逮繫。聞其賢,旋遣還,且加賞賚,有因以超擢者。二十九年,知縣靈璧周榮、宜春沈昌、昌樂于子仁,丞新化葉宗並坐事逮訊,部民為叩閽。太祖喜,立擢四人為知府,榮河南,昌南安,子仁登州,宗黃州。由是長吏競勸,一時多循良之績焉。

榮,字國華,蓬萊人。初為靈璧丞,坐累逮下刑部,耆老群赴輦下稱其賢。帝賜鈔八十錠,綺羅衣各一襲。禮部宴榮及耆老而還之。無何,擢榮靈璧知縣。及知河南,亦有聲。後建言稱旨,擢河南左布政使。

史誠祖,解州人。洪武末,詣闕陳鹽法利弊。太祖納之,授汶上知縣,為治廉平寬簡。永樂七年,成祖北巡,遣御史考核郡縣長吏賢否,還言誠祖治第一。賜璽書勞之曰:「守令承流宣化,所以安利元元。朕統御天下,夙夜求賢,共圖治理。往往下詢民間,皆言苦吏苛急,能副朕心者實鮮。爾敦厚老成,恪共乃職;持身勵志,一於廉公。平賦均徭,政清訟簡,民心悅戴,境內稱安。方古良吏,亦復何讓。特擢爾濟寧知州,仍視汶上縣事。其益共乃職,慎終如始,以永嘉譽,欽哉。」並賜內醞一尊,織金紗衣一襲,鈔千貫。御史又言貪吏虐民無若易州同知張騰,遂徵下獄。誠祖既得旌,益勤於治。土田增闢,戶口繁滋,益編戶十四里。成祖過汶上,欲徙其民數百家於膠州,誠祖奏免之。屢當遷職,輒為民奏留。閱二十九年,竟卒於任。士民哀號,留葬城南,歲時奉祀。

是時,縣令多久任。蠡縣吳祥,永樂時知嵩縣,至宣德中,閱三十二年卒於任。臨汾李信,永樂時由國子生授遵化知縣,至宣德中,閱二十七年始擢無為知州。以年老不欲赴,遂乞歸。涓縣房巖,宣德間為鄒縣知縣,至正統中,閱二十餘年卒於任,吏民皆愛戴之。而吉水知縣武進錢本忠有廉名,詿誤罷官。父老奔走,號泣乞留,郡人胡廣力保之,得還任。民聞本忠復來,空閭井迎拜。永樂中卒官,民哀慕,留葬吉水,爭負土營墳,其得民如誠祖云。

謝子襄,名袞,以字行,新淦人。建文中,由薦舉授青田知縣。永樂七年,與錢塘知縣黃信中、開化知縣夏升並九載課最,當遷。其部民相率訴於上官,乞再任,上官以聞。帝嘉之,即擢子襄處州知府,信中杭州,升衢州,俾得治其故縣。子襄治處州,聲績益著。郡有虎患。歲旱蝗。禱於神,大雨二日,蝗盡死,虎亦循去。有盜竊官鈔,子襄檄城隍神。盜方閱鈔密室,忽疾風捲墮市中,盜即伏罪。民鬻牛於市,將屠之。牛逸至子襄前,俯首若有訴,乃捐俸贖還其主。叛卒吳米據山谷為亂,朝廷發兵討之,一郡洶洶。子襄力止軍城中毋出,而自以計掩捕之,獲其魁,餘悉解散。為人廉謹,歷官三十年,不以家累自隨。二十二年卒。

信中,餘干人。先知樂清縣。奸人紿寡婦至京,誣告鄉人謀叛,而己逸去。有司繫其婦以聞,詔行所司會鞫。信中廉得其情,力詆為誣,獲全者甚眾。盜殺一家三人,獄久不決。信中禱於神,得真盜,遠近稱之。升,鹽城人。

貝秉彞,名恒,以字行,上虞人。永樂二年進士。授邵陽知縣,以憂去,補東阿。善決獄,能以禮義導民。歲大侵,上平糴備荒議。帝從之,班下郡縣如東阿式。邑西南有巨浸,積潦為田害。秉彞相視高下,鑿渠,引入大清河,涸之,得沃壤數百頃,民食其利。尤善綜畫,凡廢鐵、敗皮、朽索、故紙悉藏之。暇令工匠煮膠、鑄杵、搗紙、絞索貯於庫。會成祖北巡,敕有司建席殿。秉彞出所貯濟用,工遂速竣。帝將召之,東阿耆老百餘人詣闕自言,願留貝令,帝許之。九載考滿入都,詔進一階,仍還東阿。嘗坐累,罰役京師。民競代其役,三罰三代,乃復官。秉彞為吏明察而仁恕。素善飲,已仕,遂已之。宣德元年卒官。

時龍溪知縣南昌劉孟雍、鄒縣知縣龍溪朱瑤、建安知縣崑山張準、婺源知縣建安吳春、歙縣知縣江西樂平石啟宗,皆有惠利,民率懷思不忘云。

萬觀,字經訓,南昌人。弱冠成永樂十九年進士。帝少之,令歸肄學。尋召為御史,改嚴州知府。府東境七里瀧,有漁舟數百艇,時剽行旅。觀編十舟為一甲,令畫地巡警。不匝月,盜屏跡。乃勵學校,勸農桑,奏減織造,以銀代絲稅,民皆便之。九年考績,治行為海內第一。既以憂去,將除服,嚴州民豫上章願復得觀為守,金、衢民亦上章乞之。朝廷異焉,補平陽府,政績益茂。有芝生堯祠棟上,士民皆言使君德化所致。觀曰:「太守知奉職而已,芝,非吾事也。」考滿,擢山東布政使,卒於官。

葉宗人,字宗行,松江華亭人。永樂中,尚書夏原吉治水東南。宗人以諸生上疏,請浚范家港引浦水入海,禁瀕海民毋作壩以遏其流。帝令赴原吉所自效。工竣,原吉薦之,授錢塘知縣。縣為浙江省會,徭重,豪有力往往構黠吏得財役貧民。宗人令民自占甲乙,書於冊,以次簽役,役乃均。嘗視事,有蛇升階,若有所訴。宗人曰:「爾有冤乎?吾為爾理。」蛇即出,遣隸尾之,入餅肆爐下。發之,得僵屍,蓋肆主殺而瘞之也。又常行江中,有死人挂舟舵,推問,則里無賴子所沉者。遂俱伏法,邑民以為神。按察使周新,廉介吏也,尤重宗人。一日,伺宗人出,潛入其室,見廚中惟銀魚臘一裹。新歎息,攜少許去。明日召宗人共食,飲至醉,用儀仗導之歸。時呼為「錢塘一葉清」。十五年督工匠往營北京,卒於塗,新哭之累日。

王源,字啟澤,龍巖人。永樂二年擢進士,授庶吉士。改深澤知縣。修學舍,築長隄,勸民及時嫁娶,革其爭財之俗。數上書論事,被詔徵入都,又論時政得失,忤旨下吏。會赦復官,奏免逋負。歲饑,輒發粟振救,坐是被逮。民爭先輸納,得贖還。召為春坊司直郎,侍諸王講讀。遷衛府紀善,移松江同知,奏捐積逋數十萬石。以母老乞歸養,服闋,除刑部郎中。

英宗踐阼,擇廷臣十一人為知府,賜宴及敕,乘傳行。源得潮州府。城東有廣濟橋,歲久半圮壞,源斂民萬金重築之。以其餘建亭,設先聖、四配、十哲像。刻《藍田呂氏鄉約》,擇民為約正、約副、約士,講肄其中,而時偕僚寀董率焉。西湖山上有大石為怪,源命鑿之,果獲石骷髏,怪遂息。乃琢為碑,大書「潮州知府王源除怪石」。會杖一民死,民子訴諸朝,并以築橋建亭為源罪。逮至京,罪當贖徒。潮人相率叩閽,乃復其官。久之,乞休。潮人奏留不獲,祠祀之。

翟溥福,字本德,東莞人。永樂二年進士。除青陽知縣。九華虎為患,溥福檄山神,虎即殄。久之,移新淦,遷刑部主事,進員外郎,為尚書魏源所器。正統元年七月詔舉廷臣堪為郡守者,源以溥福應,乃擢南康知府。

先是歲歉,民擅發富家粟,及收取漂流官木者,前守悉坐以盜,當死者百餘人。溥福閱實,杖而遣之。地濱鄱陽湖,舟遇風濤無所泊,為築石堤百餘丈,往來者便之。廬山白鹿書院廢,溥福倡眾興復,延師訓其子弟,朔望躬詣講授。考績赴部,以年老乞歸。侍郎趙新嘗撫江西,大聲曰:「翟君此邦第一賢守也,胡可聽其去。」懇請累日,乃許之。辭郡之日,父老爭贐金帛,悉不受。眾挽舟涕泣,因建詞湖堤祀之,又配享白鹿書院之三賢祠。三賢者,唐李渤,宋周敦頤、朱熹也。

李信圭,字君信,泰和人。洪熙時舉賢良,授清河知縣。縣瘠而衝,官艘日相銜,役夫動以千計。前令請得沐陽五百人為助,然去家遠,艱於衣食。信圭請免其助役,代輸清河浮征三之二,兩邑便之。俗好發塚縱火,信圭設教戒十三條,令里民書於牌,月朔望儆戒之。且令書其民勤惰善惡以聞,俗為之變。宣德三年上疏言:「本邑地廣人稀,地當衝要,使節絡繹,日發民挽舟。丁壯既盡,役及老稚,妨廢農桑。前年兵部有令,公事亟者舟予五人,緩者則否。今此令不行,役夫無限,有一舟至四五十人者。凶威所加,誰敢詰問。或遇快風,步追不及,則官舫人役沒其所齎衣糧,俾受寒餒。乞申明前令,哀此憚人。」從之。八年春,又言:「自江、淮達京師,沿河郡縣悉令軍民挽舟,若無衛軍則民夫盡出有司,州縣歲發二三千人,晝夜以俟。而上官又不分別雜泛差役,一體派及。致土田荒蕪,民無蓄積。稍遇歉歲,輒老稚相攜,緣道乞食,實可憫傷。請自儀真抵通州,盡免其雜徭,俾得盡力農田,兼供夫役。」帝亦從之。自是,他郡亦蒙其澤。

正統元年,用侍郎章敞薦,擢知蘄州。清河民詣闕乞留,命以知州理縣事。民有湖田數百頃,為淮安衛卒所奪,民代輸租者六十年。信圭奏之,詔還民。饑民攘食人一牛,御史論死八人。信圭奏之,免六人。天久雨,淮水大溢,沒廬舍畜產甚眾。信圭奏請振貸,并停歲辨物件及軍匠廚役、浚河人夫,報可。南北往來道死不葬者,信圭為三大塚瘞之。十一年冬,尚書金濂薦擢處州知府,其在清河已二十二年矣。處州方苦旱,信圭至輒雨。未幾,卒於官。清河民為立祠祀之。

自明興至洪、宣、正統間,民淳俗富,吏易為治。而其時長吏亦多勵長者行,以循良見稱。其秩滿奏留者,不可勝紀,略舉數人列於篇。

孫浩,永樂中知邵陽,遭喪去官。洪熙元年,陜西按察使頌浩前政,請令補威寧。宣宗嘉歎,即命起復。久之,超擢辰州知府。

薛慎知長清,以親喪去。洪熙元年,長清民知慎服闋,相率詣京師乞再任。吏部尚書蹇義以聞,言長清別除知縣已久,即如民言,又當更易。帝曰:「國家置守令,但欲其得民心,茍民心不得,雖屢易何害。」遂還之。

吳原知吳橋,洪熙中,九載考績赴部。縣民詣闕乞留,帝從之。

陳哲知博野,以舊官還職,解去。宣德元年,部民懇訴於巡按御史,乞還哲。御史以聞,報可。

暢宣知泰安,以母憂去。民頌於副使鄺埜,以聞,仁宗命服闋還任。宣德改元,宣服闋,吏部以請。帝曰:「民欲之,監司言之,固當從,況有先帝之命乎。」遂如其請。

劉伯吉知碭山,以親喪去。服除,碭山民守闕下,求再任。吏部言新令已在碭山二年矣。帝曰:「新者勝舊,則民不復思。今久而又思,其賢於新者可知矣。」遂易之。

孔公朝,永樂時知寧陽,坐與同僚飲酒忿爭,並遣戍。部民屢叩閽乞還,皆不許。宣德二年詔求賢,有以公朝薦者,寧陽人聞之,又相率叩閽乞公朝。帝顧尚書蹇義曰:「公朝去寧陽已二十餘載,民奏乞不已,此非良吏耶?可即與之。」

郭完知會寧,為奸人所訐被逮。里老伏闕訟冤乞還,帝亦許之。

徐士宗知貴溪,宣德六年三考俱最。民詣闕乞留,詔增二秩還任。

郭南知常熟,正統十二年以老致仕。父老乞還任,英宗許之。

張璟知平山,秩滿,士民乞留,英宗命進秩復任。景泰初,母憂去。復從士民請,奪情視事。

徐榮知槁城,親喪去官。服闋,部民乞罷新令而還榮,英宗如其請。景泰初,秩滿。復徇民請,留之。

何澄知安福,被劾。民詣闕乞留,英宗命還任。乃築寅陂,浚渠道,復密湖之舊,大興水利。秩滿當遷,侍講劉球為民代請,帝復留之。

田玉知桐鄉,丁艱去。英宗以部民及巡撫周忱請,還其任。

其他若內丘馬旭、桐廬楊信、北流李禧、洋縣王黼、保安張庸、獲鹿吳韞、扶風宋端,皆當宣宗之世,以九載奏最。為民乞留,即加秩留任者也。時帝方重循良,而吏部尚書蹇義尤慎擇守令,考察明恕。沿及英宗,吏治淳厚,部民奏留率報可。然其間亦有作奸者。永寧稅課大使劉迪刲羊置酒,邀耆老請留。宣宗怒,下之吏。漢中同知王聚亦張宴求屬吏保奏為知府。事聞,宣宗并屬吏罪之。自後,部民奏留,率下所司核實云。

張宗璉,字重器,吉水人。永樂二年進士。改庶吉士,授刑部主事,錄囚廣東。仁宗即位,擢左中允。會詔朝臣舉所知,禮部郎中況鐘以宗璉名上。帝問少傅楊士奇曰:「人皆舉外吏,鐘舉京官,何也?」對曰:「宗璉賢,臣與侍讀學士王直將舉之,不意為鐘所先耳。」帝喜,曰:「鐘能知宗璉,亦賢矣。」由是知鐘,而擢宗璉南京大理丞。宣德元年,詔遣吏部侍郎黃宗載等十五人出釐各省軍籍,宗璉往福建。明年坐奏事忤旨,謫常州同知。朝遣御史李立理江南軍籍,檄宗璉自隨。立受黠軍詞,多逮平民實伍,宗璉數爭之。立怒,宗璉輒臥地乞杖,曰「請代百姓死」,免株累甚眾。初,宗璉使廣東,務廉恕。至是見立暴橫,心積不平,疽廢背卒。常州民白衣送喪者千餘人,為建祠君山。宗璉蒞郡,不攜妻子,病亟召醫,室無燈燭。童子從外索取油一盂入,宗璉立卻之,其清峻如此。

李驥,字尚德,郯城人。舉洪武二十六年鄉試。入國學,居三年,授戶科給事中。時關市譏商旅,發及囊篋,驥奏止之。尋坐事免。建文時,薦起新鄉知縣,招流亡,給以農具,復業者數千人。內艱去官,民相率奏留者數四,不許。永樂初,服闋,改知東安。事有病民,輒奏於朝,罷免之。有嫠婦子齧死,訴於驥。驥禱城隍神,深自咎責。明旦,狼死於其所。侍郎李昶等交薦,擢刑部郎中。奏陳十餘事,多見採納。坐累,謫役保安。

洪熙時,有詔求賢,薦為御史。陳經國利民十事,仁宗嘉納。宣德五年巡視倉場,軍高祥盜倉粟,驥執而鞫之。祥父妾言,祥與張貴等同盜,驥受貴等賄故獨罪祥。刑部侍郎施禮遂論驥死。驥上章自辨,帝曰:「御史即擒盜,安肯納賄!」命偕都察院再訊,驥果枉。帝乃切責禮,而復驥官。其年十一月,擇廷臣二十五人為郡守,奉敕以行。驥授河南知府,肇慶則給事中王瑩〗,瓊州則戶部郎中徐鑑,汀州則禮部員外郎許敬軒,寧波則刑部主事鄭珞,撫州則大理寺正王昇,後皆以政績著。

河南境多盜,驥為設火甲,一戶被盜,一甲償之。犯者,大署其門曰盜賊之家。又為《勸教文》,振木鐸以徇之。自是人咸改行,道不拾遺。郡有伊王府,王數請囑,不從。中官及校卒虐民,又為驥所抑,恨甚。及冬至,令驥以四更往陪位行禮。及驥如期往,誣驥後期,執而桎梏之,次日乃釋。驥奏聞,帝怒,貽書讓王,府中承奉、長史、典儀悉逮置於理。

驥持身端恪,晏居雖几席必正。蒞郡六年卒,年七十。士民赴弔,咸哭失聲。

王瑩,鄞人,起家舉人。居肇慶九年,進秩二等,後徙知西安。

徐鑑,宜興人。在瓊四年卒,郡人祀之九賢祠。

許敬軒,天台人。起家國子生。守汀特糾參政陳羽貪暴,宣宗為逮治羽。卒官,士民爭賻之。

鄭珞,閩縣人。起家進士。守寧波,以艱去。會海寇入犯,民數千詣闕乞留,詔奪情復任。嘗劾中使呂可烈無狀,帝為誅可烈。久之,擢浙江參政。

王昇,龍溪人。起家進士。在郡九載,以部民乞留,增秩還任。以疾歸。

李湘,字永懷,泰和人。永樂中,由國子生理刑都察院。以才擢東平知州,常祿外一無所取,訓誡吏民若家人然。城東有大村壩,源出岱嶽,雨潦輒為民患,奏發丁夫堤之。州及所轄五邑,地多荒蕪,力督民墾闢,公私皆實。會舊官還任,將解去。民群乞於朝,帝從其請。成祖晚年數北征,令山東長吏督民轉餉,道遠多死亡,惟東平人無失所。奸人誣湘苛斂民財,訐於布政司。縣民千三百人走訴巡按御史暨布、按二司,力白其冤。耆老七十人復奔伏闕下,發奸人誣陷狀。及布政司繫湘入都,又有耆老九十人隨湘訟冤。通政司以聞,下刑曹閱實,乃復湘官,而抵奸人於法。蒞州十餘年,至正統初,詔大臣舉郡守,尚書胡蒞以湘應,遂擢懷慶知府。東平民扶攜老幼,泣送數十里。懷慶有軍衛,素挾勢厲民。湘隨時裁制,皆不敢犯。居三年卒。

趙豫,字定素,安肅人。燕王起兵下保定,豫以諸生督賦守城。永樂五年授泌陽主簿,未上,擢兵部主事,進員外郎。內艱起復。洪熙時進郎中。宣德五年五月簡廷臣九人為知府,豫得松江,奉敕往。時衛軍恣橫,豫執其尤者,杖而配之邊,眾遂貼然。一意拊循,與民休息。擇良家子謹厚者為吏,訓以禮法。均徭節費,減吏員十之五。巡撫周忱有所建置,必與豫議。及清軍御史李立至,專務益軍,勾及姻戚同姓。稍辨,則酷刑榜掠。人情大擾,訴枉者至一千一百餘人。鹽司勾灶丁,亦累及他戶,大為民害。豫皆上章極論之,咸獲蘇息。有詔滅蘇、松官田重租,豫所轄華亭、上海二縣,減去十之二三。

正統中,九載考績。民五千餘人列狀乞留,巡按御史以聞,命增二秩還任。及十年春,大計群吏,始舉卓異之典。豫與寧國知府袁旭皆預焉,賜宴及襲衣遣還。在職十五年,清靜如一日。去郡,老稚攀轅,留一履以識遺愛,後配享周忱祠。

方豫始至,患民俗多訟。訟者至,輒好言諭之曰:「明日來。」眾皆笑之,有「松江太守明日來」之謠。及訟者踰宿忿漸平,或被勸阻,多止不訟。

始與豫同守郡者,蘇州況鐘、常州莫愚、吉水陳本深、溫州何文淵、杭州馬儀、西安羅以禮、建昌陳鼎,並皦皦著名績,豫尤以愷悌稱。

是時,列郡長吏以惠政著聞者:

湖州知府祥符趙登,秩滿當遷。民詣闕乞留,增秩再任,自宣德至正統,先後在官十七年。登同里岳璿繼之,亦有善政,民稱為趙、岳。淮安知府南昌彭遠被誣當罷,民擁中官舟,乞為奏請,宣帝命復留之。正統六年超擢廣東布政司。荊州知府大庾劉永遭父喪,軍民萬八千餘人乞留,英宗命奪情視事。鞏昌知府鄞縣戴浩擅發邊儲三百七十石振饑,被劾請罪,景帝原之。徽州知府孫遇秩滿當遷,民詣闕乞留,英宗令進秩視事。先後在官十八年,遷至河南布政使。惟袁旭在寧國為督學御史程富所誣劾,逮死獄中。而寧國人惜之,立祠祀焉。

曾泉,泰和人。永樂十八年進士。選庶吉士,改御史。宣德初,都御史邵甄別屬僚,泉謫汜水典史,卒。

正統四年,河南參政孫原貞上言:「泉操行廉潔,服官勤敏,不以降黜故有偷惰心。躬督民闢荒土,收穀麥,伐材木,備營繕,通商賈,完逋責,官有儲積,民無科擾。造舟楫,置棺槨,膽民器用。百姓婚喪不給者,咸資於泉。死之日,老幼巷哭。臣行部汜水,泉沒已三年矣,民懷其惠,言輒流涕,雖古循吏,何以加茲。若使海內得泉等數十人分治郡邑,可使朝廷恩澤滂流,物咸得所。雖在異代,猶宜下詔褒美。而獎錄未及,官階未復,使泉終蒙貶謫之名,不獲顯於當世,良可矜恤。請追復泉爵,褒既往以風方來。」帝從之。

范衷,字恭肅,豐城人。永樂十九年進士。除壽昌知縣。闢荒田二千六百畝,興水利三百四十有六區。正統五年三考報最,當遷。邑人頌德乞留,御史以聞,朝廷許之。尋以外艱去,服闋,起知汝州。吏部尚書王直察舉天下廉吏數人,衷為第一。性至孝,廬父墓,瓜生連枝,有白兔三,馴擾暮側。鄉人莫不高其行。

周濟,字大亨,洛陽人。永樂中,以舉人入太學,歷事都察院。都御史劉觀薦為御史,固辭。宣德時,授江西都司斷事。艱歸,補湖廣。正統初,擢御史。大同鎮守中官以驕橫聞,敕濟往廉之。濟變服負薪入其宅,盡得不法狀,還報,帝大嘉之。已,巡按四川。威州土官董敏、王允相仇殺,詔濟督官兵進討。濟曰:「朝廷綏安遠人,宜先撫而後徵。」馳檄諭之,遂解。十一年出為安慶知府,歲比不登,民間鬻子女充衣食,方舟而去者相接。濟借漕糧以振,而禁鬻子女者。且上疏請免租,詔許之,全活甚眾。又為定婚喪制,禁侈費,愆嫁葬期者有罰,風俗一變。

饑民聚掠富家粟,富家以盜劫告。濟下令曰:「民饑故如此,然得穀當報太守數,太守當代爾償。」掠者遂解散。濟卒官,民皆罷市巷哭云。

范希正,字以貞,吳縣人。宣德三年舉賢良方正,授曹縣知縣。有奸吏受賕,希正按其罪,械送京師。吏反誣希正他事,坐逮。曹民八百餘人詣京白通政司,言希正廉能,橫為奸吏誣枉。侍郎許廓以公事過曹,曹父老二百餘人遮道稽顙,泣言朝廷奪我賢令。事並聞,帝乃釋希正使還縣。正統十年,山東饑。惟曹以希正先積粟,得無患。大理寺丞張驥振山東,聞之。因請升曹縣為州,而以希正為知州,從之。時州民負官馬不能償,多逃竄。希正節公費代償九十餘匹,逃者皆復業。吉水人誣曹富民殺其兄,連坐甚眾。希正密移吉水,按其人姓名皆妄,事得白。治曹二十三年,歷知州,再考乃致仕。

當是時,潞州知州咸寧燕雲、徐州知州楊秘、全州知州錢塘周健、霸州知州張需、定州知州王約,皆大著聲績。祕、健進秩視事,約賜詔旌異。需忤太監王振戍邊,人尤惜之。而得民最久者,無若希正與寧州知州劉綱。綱,字之紀,禹州人。建文二年進士。由府谷知縣遷是職。蒞州三十四年,仁宗嘗賜酒饌,人以為榮。正統中,請老去,民送之,涕泣載道。及卒,寧民祀之狄仁傑祠中。其孫,即大學士宇也。

段堅,字可大,蘭州人。早歲受書,即有志聖賢。舉於鄉,入國子監。景泰元年,上書請悉徵還四方監軍,罷天下佛老宮。疏奏,不行。五年成進士,授福山知縣。刊布小學,俾士民講誦。俗素陋,至是一變,村落皆有糸玄誦聲。成化初,賜敕旌異,超擢萊州知府。期年,化大行。以憂去,服除,改知南陽。召州縣學官,具告以古人為學之指,使轉相勸誘。創志學書院,聚秀民講說《五經》要義,及濂、洛諸儒遺書。建節義祠,祀古今烈女。訟獄徭賦,務底於平。居數年,大治,引疾去。士民號泣送者,踰境不絕。及聞其卒,立祠,春秋祀之。

堅之學,私淑河東薛瑄,務致知而踐其實,不以諛聞取譽,故能以儒術飾吏治。

子炅,進士,翰林檢討。諂附焦芳,劉瑾敗,落職,隤其家聲焉。

陳鋼,字堅遠,應天人。舉成化元年鄉試,授黔陽知縣。楚俗,居喪好擊彭歌舞。鋼教以歌古哀詞,民俗漸變。縣城當沅、湘合流,數決,壞廬舍。鋼募人採石甃隄千餘丈,水不為害。南山崖官道數里,徑窄甚,行者多墮崖死。鋼積薪燒山,沃以醯,拓徑丈許,行者便之。鋼病,民爭籲神,願減已算益鋼壽。遷長沙通判,監修吉王府第。工成,王賜之金帛,不受。請王故殿材修岳麓書院,王許之。弘治元年丁母憂歸。卒,黔陽、長沙並祠祀之。子沂,官侍講,見《文苑傳》。

丁積,字彥誠,寧都人。成化十四年進士。授新會知縣,至即師事邑人陳獻章。為政以風化為本,而主於愛民。中貴梁芳,邑人也,其弟長橫於鄉,責民逋過倍,復訴於積。積追券焚之,且收捕繫獄,由是權豪屏跡。申洪武禮制,參以《朱子家禮》,擇耆老誨導百姓。良家子墮業,聚廡下,使日誦小學書,親為解說,風俗大變。民出錢輸官供役,名均平錢。其後吏貪,復令甲首出錢供用,曰當月錢,貧者至鬻子女。積一切杜絕。俗信巫鬼,為痛毀淫祠。既而歲大旱,築壇圭峰頂。昕夕伏壇下者八日,雨大澍。而積遂得疾以卒,士民聚哭於途。有一嫗夜哭極哀,或問之,曰:「來歲當甲首,丁公死,吾無以聊生矣。」

田鐸,字振之,陽城人。成化十四年進士。授戶部主事,遷員外郎、郎中。弘治二年奉詔振四川,坐誤遺敕中語,謫蓬州知州。州東南有江洲八十二頃,為豪右所據,鐸悉以還民。建大小二十四橋,又鑿三溪山以便行者。御史行部至蓬,寂無訟者,訝之。已,乃知州無冤民也,太息而去。薦於朝,擢廣東僉事。遷四川參議,不赴,以老疾告歸。正德時,劉瑾矯詔,言鐸理廣東鹽法,簿牒未明,逮赴廣。未就道而瑾誅,或勸鐸毋行,鐸不聽,行次九江卒,年八十二矣。

唐侃,字廷直,丹徒人。正德八年舉於鄉,授永豐知縣。之官不攜妻子,獨與一二童僕飯蔬豆羹以居。久之,吏民信服。永豐俗刁訟,尚鬼,尤好俳優,侃禁止之。進武定知州。會清軍籍,應發遣者至萬二千人。侃曰:「武定戶口三萬,是空半州也」。力爭之。又有議徙州境徒駭河者,侃復言不宜朘民財填溝壑。事並得寢。章聖皇太后葬承天,諸內奄迫脅所過州縣吏,索金錢,宣言供張不辦者死,州縣吏多逃。侃置空棺旁舍中,奄迫之急,則經至棺所,指而造之曰:「吾辦一死,金錢不可得也。」諸奄皆愕眙去。稍遷刑部主事,卒。

初,侃少時從丁璣學。鄰女夜奔之,拒勿納。其父坐繫,侃請代不得,藉草寢地。逾歲,父獲宥,乃止。其操行貞潔,蓋性成也。

湯紹恩,安岳人。父佐,弘治初進士,仕至參政。紹恩以嘉靖五年擢第。十四年由戶部郎中遷德安知府,尋移紹興。為人寬厚長者,性儉素,內服疏布,外以父所遺故袍襲之。始至,新學宮,廣設社學。歲大旱,徒步禱烈日中,雨即降。緩刑罰,恤貧弱,旌節孝,民情大和。山陰、會稽、蕭山三邑之水,匯三江口入海,潮汐日至,擁沙積如丘陵。遇霪潦則水阻,沙不能驟洩,良田盡成巨浸,當事者不得已決塘以瀉之。塘決則憂旱,歲苦修築。紹恩遍行水道,至三江口,見兩山對峙,喜曰:「此下必有石根,餘其於此建閘乎?」募善水者探之,果有石脈橫互兩山間,遂興工。先投以鐵石,繼以籠盛甃屑沉之。工未半,潮衝蕩不能就,怨讟煩興。紹恩不為動,禱於海神,潮不至者累日,工遂竣。修五十餘尋,為閘二十有八,以應列宿。於內為備閘三,曰經漊,曰撞塘,曰平水,以防大閘之潰。閘外築石隄四百餘丈扼潮,始不為閘患。刻水則石間,俾後人相水勢以時啟閉。自是,三邑方數百里間無水患矣。士民德之,,立廟閘左,歲時奉祀不絕。屢遷山東右布政使,致仕歸,年九十七而卒。

初,紹恩之生也,有峨嵋僧過其門,曰:「他日地有稱紹者,將承是兒恩乎?」因名紹恩,字汝承,其後果驗。

徐九思,貴溪人。嘉靖中,授句容知縣。始視事,恂恂若不能。俄有吏袖空牒竊印者,九思摘其奸,論如法。郡吏為叩頭請,不許,於是人人惴恐。為治於單赤務加恩,而御豪猾特嚴。訟者,抶不過十。諸所催科,預為之期,逾期,令里老逮之而已,隸莫敢至鄉落。縣東西通衢七十里,塵土積三尺,雨雪,泥沒股。九思節公費,甃以石,行旅便之。朝廷數遣中貴醮神三茅山,縣民苦供應。九思搜故牒,有鹽引金久貯於府者,請以給嘗,民無所擾。歲侵,穀湧貴。巡撫發倉穀數百石,使平價糶而償直於官。九思曰:「彼糴者,皆豪也。貧民雖平價不能糴。」乃以時價糶其半,還直於官,而以餘穀煮粥食餓者。穀多,則使稱力分負以去,其山谷遠者,則就旁富人穀,而官為償之,全活甚眾。嘗曰:「即天子布大惠,安能人人蠲租賜復,第在吾曹酌緩急而已。」久之,與應天府尹不合,為巡撫所劾,吏部尚書熊浹知其賢,特留之。

積九載,遷工部主事,歷郎中,治張秋河道。漕河與鹽河近而不相接,漕水溢則泛濫為田患。九思議築減水橋於沙灣,俾二水相通,漕水溢,則有所洩以入海,而不侵田,少則有所限而不至於涸。工成,遂為永利。時工部尚書趙文華視師東南,道河上。九思不出迎,遣一吏齎牒往謁,文華嫚罵而去。會遷高州知府。文華歸,修舊怨,與吏部尚書吳鵬合謀構之,遂坐九思老,致仕。句容民為建祠茅山。九思家居二十二年,年八十五,抱疾,抗手曰「茅山迎我」,遂卒。子貞明,自有傳。

龐嵩,字振卿,南海人。嘉靖十三年舉於鄉。講業羅浮山,從遊者雲集。二十三年歷應天通判,進治中,先後凡八年。府缺尹,屢攝其事。始至,值歲饑,上官命督振。公粟竭,貸之巨室富家,全活者六萬七千餘人。乃蠲積逋,緩征徭,勤勞徠,復業者又十萬餘人。留都民苦役重,力為調劑,凡優免戶及寄居客戶、詭稱官戶、寄莊戶、女戶、神帛堂匠戶,俾悉出以供役,民困大蘇。江寧縣葛仙、永豐二鄉,頻遭水患,居民止存七戶。嵩為治隄築防,得田三千六百畝,立惠民莊四,召貧民佃之,流移盡復。屢剖冤獄,戚畹王湧、舉人趙君寵占良人妻,殺人,嵩置之法。

早遊王守仁門,淹通《五經》。集諸生新泉書院,相與講習。歲時單騎行縣,以壺漿自隨。京府佐貳鮮有舉其職者,至嵩以善政特聞。府官在六年京察例,而復與外察。嵩謂非體,疏請止之,遂為永制。遷南京刑部員外郎,進郎中。撰《原刑》、《司刑》、《祥刑》、《明刑》四篇,曰《刑曹志》,時議稱之。遷雲南曲靖知府,亦有政聲。中察典,以老罷,而年僅五十。復從湛若水游,久之卒。應天、曲靖皆祠之名宦,葛仙鄉專祠祀之。

張淳,字希古,桐城人。隆慶二年進士,授永康知縣。吏民素多奸黠,連告罷七令。淳至,日夜閱案牘。訟者數千人,剖決如流,吏民大駭,服,訟浸減。凡赴控者,淳即示審期,兩造如期至,片晷分析無留滯。鄉民裹飯一包即可畢訟,因呼為「張一包」,謂其敏斷如包拯也。巨盜盧十八剽庫金,十餘年不獲,御史以屬淳。淳刻期三月必得盜,而請御史月下數十檄。及檄累下,淳陽笑曰:「盜遁久矣,安從捕。」寢不行。吏某婦與十八通,吏頗為耳目,聞淳言以告十八,十八意自安。淳乃令他役詐告吏負金,繫吏獄。密召吏責以通盜死罪,復教之請以婦代繫,而己出營貲以償。十八聞,亟往視婦,因醉而擒之。及報御史,僅兩月耳。

民有睚眥嫌,輒以人命訟。淳驗無實即坐之,自是無誣訟者。永人貧,生女多不舉。淳勸誡備至,貧無力者捐俸量給,全活無數。歲旱,劫掠公行,下令劫奪者死。有奪五斗米者,淳佯取死囚杖殺之,而榜其罪曰「是劫米者」,眾旨懾服。久之,以治行第一赴召去永,甫就車,顧其下曰:「某盜已來,去此數里,可為我縛來。」如言跡之,盜正濯足於河,繫至,盜服辜。永人駭其事,謂有神告。淳曰:「此盜捕之急則遁,今聞吾去乃歸耳。以理卜,何神之有。」

擢禮部主事,歷郎中,謝病去。起建寧知府,進浙江副使。時浙江有召募兵,撫按議散之,兵皆洶洶。淳曰:「是憍悍者,留則有用,汰則叵測。不若汰其老弱,而留其壯勇,則留者不思亂,汰者不能亂矣。」從之,事遂定。官終陜西布政。

陳幼學,字志行,無錫人。萬曆十七年進士。授確山知縣。政務惠民,積粟萬二千石以備荒,墾萊田八百餘頃,給貧民牛五百餘頭,核黃河退地百三十餘頃以賦民。里婦不能紡者,授紡車八百餘輛。置屋千二百餘間,分處貧民。建公廨八十間,以居六曹吏,俾食宿其中。節公費六百餘兩,代正賦之無徵者。栽桑榆諸樹三萬八千餘株,開河渠百九十八道。

布政使劉渾成弟燦成助妾殺妻,治如律。行太僕卿陳耀文家人犯法,立捕治之。汝寧知府邱度慮幼學得禍,言於撫按,調繁中牟。秋成時,飛蝗蔽天。幼學捕蝗,得千三百餘石,乃不為災。縣故土城,卑且圮。給饑民粟,俾修築,工成,民不知役。縣南荒地多茂草,根深難墾。令民投牒者,必入草十斤。未幾,草盡,得沃田數百頃,悉以畀民。有大澤,積水,占膏腴地二十餘里。幼學疏為河者五十七,為渠者百三十九,俱引入小清河,民大獲利。大莊諸里多水,為築隄十三道障之。給貧民牛種,貧婦紡具,倍於確山。越五年,政績茂著。以不通權貴,當考察拾遺,掌道御史擬斥之,其子爭曰:「兒自中州來,咸言中牟治行無雙。今予殿,何也?」乃已。

稍遷刑部主事。中官採御園果者,怒殺園夫母,棄其屍河中。幼學具奏,逮置之法。嘉興人袁黃妄批削《四書》、《書經集註》,名曰《刪正》,刊行於時。幼學駁正其書,抗疏論列。疏雖留中,鏤板盡毀。以員外郎恤刑畿輔,出矜疑三百餘人。進郎中。

遷湖州知府,甫至,即捕殺豪惡奴。有施敏者,士族子,楊陞者,人奴也,橫郡中。幼學執敏置諸獄。敏賂貴人囑巡撫檄取親鞫,幼學執不予,立杖殺之。敏獄辭連故尚書潘季馴子廷圭,幼學言之御史,疏劾之,下獄。他奸豪復論殺數十輩,獨楊陞畏禍斂跡,置之。已,念己去,升必復逞,遂捕置之死,一郡大治。霪雨連月,禾盡死。幼學大舉荒政,活饑民三十四萬有奇。御史將薦之,徵其治行,推官閻世科列上三十六事,御史以聞。詔加按察副使,仍視郡事。久之,以副使督九江兵備。幼學年已七十,其母尚在,遂以終養歸。母卒,不復出。天啟三年起南京光祿少卿,改太常少卿,俱不赴。明年卒,年八十四矣。中矣、湖州並祠祀之。


\end{pinyinscope}