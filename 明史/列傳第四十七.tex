\article{列傳第四十七}

\begin{pinyinscope}
熊概葉春陳鎰李儀丁璿陳泰李棠曾翬賈銓王宇崔恭劉孜宋傑邢宥李侃雷復李綱原傑彭誼牟俸夏壎子鍭高明楊繼宗

熊概,字元節,豐城人。幼孤,隨母適胡氏,冒其姓。永樂九年進士。授御史。十六年擢廣西按察使。峒谿蠻大出掠,布政使議請靖江王兵遏之。概不可,曰:「吾等居方面,寇至無捍禦,顧煩王耶?且寇必不至,戒嚴而已。」已而果然。久之,調廣東。

洪熙元年正月,命以原官與布政使周乾、參政葉春巡視南畿、浙江。初,夏原吉治水江南還,代以左通政趙居任,兼督農務。居任不恤民,歲以豐稔聞。成祖亦知其誣罔。既卒,左通政岳福繼之,庸懦不事事。仁宗監國時,嘗命概以御史署刑部,知其賢,故有是命。是年八月,乾還,言有司多不得人,土豪肆惡,而福不任職。宣宗召福還,擢概大理寺卿,與春同往巡撫。南畿、浙江設巡撫自此始。

浙西豪持郡邑短長為不法。海鹽民平康暴橫甚,御史捕之,遁去。會赦還,益聚黨八百餘人。概捕誅之。已,悉捕豪惡數十輩,械至京,論如法。於是奸宄帖息。諸衛所糧運不繼,軍乏食。概以便宜發諸府贖罪米四萬二千餘石贍軍,乃聞於朝。帝悅,諭戶部勿以專擅罪概。概用法嚴,姦民憚之,騰謗書於朝。宣德二年,行在都御史劾概與春所至作威福,縱兵擾民。帝弗問,陰使御史廉之,無所得。由是益任概。明年七月賜璽書獎勵。概亦自信,諸當興革者皆列以聞。時屢遣部官至江南造紙、市銅鐵。概言水澇民饑,乞罷之。

五年還朝,始復姓。亡何,遷右都御史,治南院事。行在都御史顧佐疾,驛召概代領其職,兼署刑部。九年十月錄囚,自朝至日宴,未暇食,忽風眩卒。賜祭,給舟歸其喪。

概性剛決,巡視江南,威名甚盛。及掌臺憲,聲稱漸損於初。

葉春者,海鹽人。起家掾吏,歷禮部郎中兩淮鹽運使,改四川右參政。與概巡撫江、浙諸府。既復奉命與錦衣指揮任啟、御史賴英、太監劉寧巡視。先後凡三涖浙西,治事於鄉,人無議其私者。概遷都御史。春同日進刑部右侍郎。卒於官。

陳鎰,字有戒,吳縣人。永樂十年進士。授御史。遷湖廣副使,歷山東、浙江,皆有聲。

英宗即位之三月,擢右副都御史,與都督同知鄭銘鎮守陜西。北方饑民多流移就食。鎰道出大名見之,疏陳其狀,詔免賦役。正統改元,鎰言陜西用兵,民困供億,派征物料,乞悉停免。詔可。明年五月,以勞績下敕獎勵,因命巡延綏、寧夏邊。所至條奏軍民便宜,多所廢置。所部六府饑,請發倉振。帝從輔臣請,修荒政。鎰請遍行於各邊,由是塞上咸有儲蓄。六年春,以鎰久勞於外,命與王翱歲一更代。七年,翱調遼東,鎰復出鎮。歲滿當代,以陜人乞留,詔仍舊任。時倉儲充溢,有軍衛者足支十年,無者直可支百年。鎰以陳腐委棄可惜,請每歲春夏時,給官軍為月餉,不復折鈔。從之。

九年春進右都御史,鎮守如故。秦中饑,乞蠲租十之四,其餘米布兼收。時瓦剌也先漸強,遣人授罕東諸衛都督喃哥等為平章,又置甘肅行省名號。鎰以聞,請嚴為之備。已,命與靖遠伯王驥巡視甘肅、寧夏、延綏邊務,聽便宜處置。以災沴頻仍,條上撫安軍民二十四事,多議行。

鎰嘗恐襄、漢間流民嘯聚為亂,請命河南、湖廣、陜西三司官親至其地撫恤之。得旨允行,而當事者不以為意。王文亦相繼力言有司怠忽,恐遺禍。至成化時,乃有項忠之役,人益思鎰言。

英宗北狩,景帝監國,鎰合大臣廷論王振。於是振姪王山伏誅。也先將入犯,以于謙薦,出撫畿內。事寧,召還,進左都御史。

景泰二年,陜西饑,軍民萬餘人,「願得陳公活我。」監司以聞,帝復命之。鎰至是凡三鎮陜,先後十餘年,陜人戴之若父母。每還朝,必遮道擁車泣。再至,則歡迎數百里不絕。其得軍民心,前後撫陜者莫及也。

三年春召還,加太子太保,與王文並掌都察院。文威嚴,諸御史畏之若神。鎰性寬恕,少風裁,譽望損於在陜時。明年秋以疾致仕。卒,贈太保,謚僖敏。天順七年,詔官其子伸為刑部照磨。

李儀,涿人。永樂間以薦舉授戶部主事。宣宗既平高煦,義請去趙王護衛。尚書張本亦言:「往歲孟賢謀逆,趙王未必不知。高煦亦謂與趙合謀。儀言是。」帝不聽。既而言者益眾。帝封其詞,遣使諭王如儀指。王即獻護衛,趙卒無事。儀尋出知九江府,有惠政。

英宗即位之歲,始設諸邊巡撫。僉都御史丁璿方督大同、宣府軍儲,而儀以右僉都御史巡撫其地,盛有所建置。明年請以大同東西二路分責於總兵官羅文、方政。從之。時朝議遣方政、楊洪出塞,與甘肅將蔣貴、史昭合擊朵兒只伯。儀言:「四裔為患,自古有之,在備禦有方耳。和寧殘部,窮無所歸,乍臣乍叛,小為邊寇。邊將謹待之,將自遁,何必窮兵。萬一乘虛襲我,少有失,適足為笑,乞敕政等無窮追。」不納。

督糧參政劉璉不職,儀劾之。璉乃誣儀淫亂事。適參將石亨欲奏鎮守中官郭敬罪,先咨儀。儀誤緘咨牒於核餉主事文卷中,戶部以聞,致亨、敬相奏訐。詔儀、璉自陳,而切責敬等。璉止停俸二歲。儀雖引罪,自負其直,詞頗激,遂被劾下吏瘐死。正統二年二月也。儀居官廉謹,邊人素德之。聞其死,建昭德祠以祀。

丁璿,上元人。永樂中進士。由御史擢居是職。正統五年將征麓川,命乘傳往備儲餉。尋言用兵便宜,遂命撫雲南。麓川平,召為左副都御史,所至有聲。

陳泰,字吉亨,光澤人。幼從外家曹姓,既貴,乃復故。舉鄉試第一,除安慶府學訓導。

正統初,廷臣交薦,擢御史,巡按貴州。官軍征麓川,歲取土兵二千為鄉導,戰失利,輒殺以冒功,泰奏罷之。再按山西。時百官俸薄,折鈔又不能即得。泰上章乞量增祿廩,俾足養廉,然後治贓污,則貪風自息。事格不行。六年夏言:「連歲災異,咎在廷臣,請敕御史給事中糾彈大臣,去其尤不職者,而後所司各考核其屬。」帝從之。於是御史馬謹等交章劾吏部尚書郭璡等數十人。已,復出按山東。泰素勵操行,好搏擊。三為巡按,懲奸去貪,威棱甚峻。

九年超擢四川按察使,與鎮守都御史寇深相失。十二年八月,參議陳敏希深指,劾泰擅杖武職,毆輿夫至死。逮刑部獄,坐斬。泰奏辯,大理卿俞士悅亦具狀以聞。皆不聽。

景帝監國,赦復官。於謙薦守紫荊關。也先入犯,關門不守,復論死。景帝宥之,命充為事官,從總兵官顧興祖築關隘自效。景泰元年擢大理右少卿,守備白羊口。四月,都督同知劉安代寧遠伯任禮巡備涿、易、真、保諸城,命泰以右僉都御史參其軍務。三年兼巡撫保定六府。尋命督治河道。自儀真至淮安,浚渠百八十里,塞決口九,築壩三,役六萬人,數月而畢。七年移撫蘇、松。

天順改元,罷巡撫官,改廣東副使,以憂去。四川盜起,有言泰嘗蒞其地,有威名,乃復故官,往巡撫。八年進右副都御史,總督漕運兼巡撫淮、揚諸府。蒞淮三年,謝政歸。成化六年卒。

李棠,字宗楷,縉雲人。宣德五年進士。授刑部主事,為尚書魏源所器。金濂代源,以剛嚴懾下。棠與辯論是非,譴訶不為動,濂亦器之。進員外郎。錄囚南畿,多所平反,進郎中。景帝嗣位,超擢本部侍郎。未幾,巡撫廣西,提督軍務。所部多寇,棠以次討平之。正己帥下,令行政舉。

景泰三年,思明土知府黃夌老,子鈞嗣。夌庶兄矰使其子殺夌父子,滅其家,而以他盜為亂告。棠檄右參政曾翬副使劉仁宅按其事。翬等誘執矰父子下之獄。矰窘則遣使走京師,上書請帝廢太子立己子。帝大喜,立擢矰都督同知,出其子於獄。事具《懷獻太子》及《土司傳》。棠既不得竟黃矰獄,鬱鬱累疏謝病歸。不攜嶺表一物,以清節顯。

曾翬,字時升,泰和人。宣德八年進士。治秦府永興王葬,卻有司饋遺。歷刑部員外郎。尚書金濂器之,俾典奏牘。有重獄,諸郎不能決,輒以屬翬。秦王訐巡撫陳鎰狎妓。翬按得其情,劾籓府誣大臣,鎰得白。

正統十三年進郎中。以何文淵薦,擢廣西右參政。李棠檄翬及副使劉仁宅按黃矰父子。矰使人持千金賄於道,且擁精兵挾之。二人佯許諾,已,誘執矰下之獄。棠以聞。未幾,矰以上書擢都督同知,父子俱出獄,翬等太息而已。尋以憂去。服闋,起官河南御史。清軍者利得軍,多枉及民,翬辨釋甚眾。南陽諸府多流戶,眾議驅逐,人情惶急,翬與巡撫撫安之。

天順五年遷山東右布政使。民墾田無賦者,姦民指為閒田,獻諸戚畹。部使者來勘,翬曰:「祖制,民墾荒田,永不科稅,奈何奪之?」使者奏如言,乃免。成化初,轉左。河南歲饑,計開封積粟多,奏請平糶,貧民賴以濟。召拜刑部左侍郎,仍食從二品俸。尋巡視浙江,考察官吏,奏罷不職者百餘人,他弊政多所釐革。還朝,久之,謝病去。

翬操行謹,所至有聲。及歸,生計蕭然,絕跡公府,鄉人以為賢。

賈銓,字秉鈞,邯鄣人。永樂末進士。宣德四年授禮科給事中,數有參駁。

英宗踐阼,既肆赦,復命讞在京重囚,多所原宥。從銓請,推之南京。秩滿,出為大理知府。王驥征麓川,饋運有勞。驥薦之。麓川平,擢雲南左參政,仍知府事。尋以驥言,還治司事。正統十二年,左布政使闕,軍民數萬人頌銓,參贊軍務侍郎侯璡等亦疏請,銓遂得擢。土官十餘部,歲當貢馬輸差發銀及海,八府民歲當輸食鹽米鈔。至景泰初,皆積逋不能償。銓等為言除之。治行聞,賜誥旌異。景泰七年,九載滿,當入都,軍民乞留。命還任。

天順四年與梁楘等舉政績卓異。戶部初闕尚書,王翱欲擢銓。帝問李賢,賢曰:「聞其名,未見其人也。」及是來覲,帝命賢視之,還奏貌寢。乃以為右副都御史巡撫山東,尋兼撫河南。山東歲侵,請召還清軍御史。河南饑,請停徵課馬。皆許之。成化初,左都御史李秉督師遼東,召銓署院事。中官唐慎等從征荊、襄還,杖死淮安知事谷淵,自奏丐免。銓請罪之。乃付慎等司禮監,命法司罪其從人。未幾,卒官。謚恭靖。

銓在雲南,治行為一時冠。比為巡撫,清靜不自表暴,吏民亦安之。

王宇,字仲宏,祥符人。童丱時,日記萬言,巡撫侍郎于謙奇之。登正統四年進士,授南京戶部主事。秩滿當轉郎中,吏部以宇才,特用為撫州知府。為政簡靜,而鋤強遏姦,凜不可犯,一府大治。

天順元年,所司上其治行,詔賜誥命。頃之,抉山東右布政使,命撫恤所屬饑民。明年遷右副都御史,巡撫宣府。中官嚴順、都督張林等令家人承納芻糧。宇劾奏。都御史寇深為解,帝切責深。尋命兼撫大同。石亨及從子彪驕恣,大同其舊鎮地,徵索尤橫。宇抗疏論其姦,乞置之法。疏雖不行,聞者敬憚。督餉郎中楊益不能備芻槁,為宇所劾。戶部庇之,宇并劾尚書沈固等。皆輸罪。遭喪,起復為大理卿。固辭,不許。

宇剛介,所至有盛名。居大理,平反為多。七年卒。

崔恭,字克讓,廣宗人。正統元年進士。除戶部主事。出理延綏倉儲,有能聲。以楊溥薦,擢萊州知府。內地輸遼東布,悉貯郡庫,歲久朽敝,守者多破家。恭別構屋三十楹貯之,請約計歲輸外,餘以充本府軍餉,遂放遣守者八百人。也先犯京師,遣民兵數千入援。廷議城臨清,檄發役夫。恭以方春民乏食,請俟秋成。居府六年,萊人以比漢楊震。

景泰中,超遷湖廣右布政使。諸司供給,率取之民。恭與僚佐約,悉罷之。公安、監利流民擅相殺。恭下令願附籍者聽,否則迨秋遣歸,眾遂定。尋遷江西左布政使。司有廣濟庫,官吏乾沒五十萬。恭白於巡撫韓雍,典守者咸獲罪。定均徭法,酌輕重,十年一役,遂為定例。

天順二年,寧王奠培不法,恭劾之。削其護衛,王稍戢。遷右副都御史,代李秉巡撫蘇、松諸府。按部,進耆老言利病,為興革。與都督徐恭浚儀真漕河,又浚常、鎮河,避江險。已,大治吳淞江。起崑山夏界口,至上海白鶴江,又自白鶴江至嘉定卞家渡,迄莊家涇,凡浚萬四千二百餘丈。又浚曹家港、蒲匯塘、新涇諸水。民賴其利,目曹家港為「都堂浦」。初,周忱奏定耗羨則例,李秉改定以賦之輕重遞盈縮。其例甚平,而難於稽算,吏不勝煩擾。恭乃罷去,悉如忱舊。

吏部缺右侍郎,李賢、王翱舉恭。遂召用。置「勸懲簿」,有聞皆識之。翱甚倚恭,轉左。父憂起復。憲宗即位,乞致仕。不允。成化五年,尚書李秉罷。商輅欲用姚夔,彭時欲用王概,而北人居言路者,謂時實逐秉,喧謗於朝。時稱疾不出,侍讀尹直以時、概皆已鄉人,恐因此得罪,急言於輅,以恭代秉。越五月,母喪歸。服除,起南京吏部,劾罷諸司不識者數人。十一年春命參贊機務。居三年,致仕。又二年卒。贈太子少保,謚莊敏。

劉孜,字顯孜,萬安人。正統十年進士。授御史,出按遼東。景帝即位,有建南遷議者。孜馳奏,乞斬言者以定人心。期滿當代,朝議邊務方殷,復留一歲。再按畿輔。時方築淪州城,以孜言罷。擢山東按察使。

天順四年,吏部舉天下治行卓異,按察使惟孜一人,遷左布政使。明年春,以右副都御史巡撫江南十府。蘇、松財賦,自周忱立法後,代者多紛更。孜首訪忱遺積,斟酌行之,民稱便。成化元年,應天饑,方振貸,而江北饑民就食者眾。孜請盡發諸縣廩,全活無算。時民間多積困:瀕江官田久廢沒,仍責輸賦;蘇、松、杭、嘉諸府僉補富戶;南京廊房既傾圮,猶徵鈔;上元、江寧農民代河泊所綱戶採鰣魚;應天都稅宣課諸司額外增稅;江陰諸縣民戶償納荒租;六合、江浦官牛歲徵犢。孜皆疏罷之。

召拜南京刑部尚書,以宋傑代。四年致仕,道卒。

孜廉慎,治事精審。然持法過嚴,時議其刻。傑為人長者。居二年,罷去,而邢宥代。

宥,文昌人。正統十三年進士。授御史,出巡福建。民十人被誣為盜,當刑呼冤。宥為緩之,果得真盜。天順中,出為台州知府,有治績。坐累謫晉江丞。憲宗復其職,改知蘇州。姦民攬納秋賦,置之法,得其贓萬緡,以隄沙河,甓官道。大水,民饑,不待奏輒發米二十萬斛以振。宥素廉介,及治蘇,嚴而不苛。傑薦於朝,詔加浙江左參政仍理府事,賜璽書。居半歲,遂以右僉都御史代傑巡撫。開丹陽河,築奔牛閘,省兌運冗費,民以為便。尋兼理兩浙鹽政,考察屬吏,奏黜不識者百七十餘人。居數載,引疾歸。

李侃,字希正,東安人。正統七年進士。授戶科給事中。景帝監國,陳簡將才、募民壯、用戰車三事。也先逼京師議者欲焚城外馬草。侃言敵輕剽,無持久心,乞勿焚,免復斂為民累。皆報許。時父母在容城,侃曉夜悲泣,乞假,冒險迎之。景泰初,議錄扈從死事諸臣後。侃因言避難偷生者,宜嚴譴以厲臣節。上皇將還,與同官劉福等言禮宜從厚。忤旨,被詰,尚書胡濙為解,乃已。

再遷都給事中。軍興,減天下學校師儒俸廩。侃奏復之。戶部尚書金濂違詔征租,侃論濂,下之吏。石亨從子彪侵民業,侃請置重典,並嚴禁勳戚、中官不得豪奪細民,有司隱者同罪。帝宥亨、彪,餘如其請。時給事中敢言者,林聰稱首,侃亦矯抗有直聲。廷議易儲,諸大臣唯唯。侃泣言東宮無失德,聰與御史朱英亦言不可,時議壯之。擢詹事府丞。

天順元年改太常丞,進太僕卿。明年復設山西巡撫,遷侃右僉都御史任之。奏言:「塞北之地,與窮荒無異。非生長其間者,未有能寧居而狎敵者也。今南人戍西北邊,怯風寒,聞寇股栗。而北人戍南,亦不耐暑,多潛逃。宜令南北清勾之軍,各就本土補伍,人情交便,戎備得修。」時不能用。奏發巡按李傑罪,傑亦訐侃。按傑事有驗,除名。侃無贓罪,獲宥。六年考察屬吏,奏罷布政使王允、李正芳以下百六十人。因言:「諸臣年與臣若、不堪任事者,臣悉退之,臣亦當罷。」詔不許。侃性剛方,力振風紀,貪墨者屏跡。其年冬以母喪歸,軍民擁泣,至不得行。服除,遂不出,家居十餘年卒。

侃事親孝,好學安貧,歿幾不能殮。弘治初,國子生江紀等言,前祭酒胡儼,都御史高明、李侃學行事功,彰著耳目,並乞賜謚。寢不行。侃二子:德恢,嚴州知府;德仁,河東鹽運使。

雷復,字景暘,湖廣寧遠人。正統初進士。授行人,歷官廣西副使。藤縣民胡趙成構瑤陷縣治,復與參將范信討斬之。成化初以大臣會薦,擢山東右布政使。七年徵拜禮部右侍郎。尋改右副都御史,巡撫山西。繼李侃後,端恪守法,得軍民心。敗寇紅沙煙,再敗之煙寺溝、石人村,賜敕獎勞。時山西大浸,而廷議以陜西用兵,令預征芻餉,轉輸榆林。復上言:「自山西至榆林,道路險絕,民齎銀往易,價騰湧,不免稱貸,償責多破產。今雨雪愆違,饑民疾病流離,困悴萬狀,而應輸綾帛、藥果諸物,又不下萬計。乞依山東例蠲除,仍發帑振贍。」帝從之。及發金三萬不足,請鬻鹽四十萬引,并令民入粟授散官。皆報可。十年夏卒於官。

李綱,字廷張,長清人。幼從父入都,墜車下,車轢體過,竟不傷,人咸異之。登天順元年進士,授御史。歷按南畿、浙江。劾去浙江贓吏至四百餘人,時目為「鐵御史」。奉敕編集陜西延綏土兵。還,遷太僕寺少卿,巡畿輔馬政,盡卻有司饋。按冀州,遇盜。問隸人曰:「太僕李公耶?是何從得金。」不啟篋而去。成化十三年遷右僉都御史。轉左,出督漕運,與平江伯陳銳共事。踰年卒。銳見笥中惟敝衣,揮涕曰:「君子也。」為具棺斂,聞其清節於朝。帝特命賜祭葬,不為令。綱清剛似李侃,為時所重。

原傑,字子英,陽城人。正統十年進士。又二年,授南京御史,尋改北。巡按江西,捕誅劇盜,姦宄斂跡。復按順天諸府。大水,牧官馬者乏芻,馬多斃,有司責償,傑請免之。開中鹽引入米振饑。疏入,為部所格,景帝卒從傑議。超擢江西按察使。發寧王奠培淫亂事,革其護衛。治行聞,賜誥旌異,遷山東左布政使。

成化二年就拜右副都御史,巡撫其地。歲凶振救,民無流移。召為戶部左侍郎。時黃河遷決不常,彼陷則此淤。軍民就淤墾種。姦徒指為園場屯地,獻王府邀賞,王府輒據有之。傑請獻者謫戍,並罪受獻者。從之。江西盜起,以傑嘗再蒞其地得民,詔往治。捕戮六百餘人,餘悉解散。改左副都御史,還佐院事。

荊、襄流民數十萬,朝廷以為憂。祭酒周洪謨嘗著《流民圖說》,謂當增置府縣,聽附籍為編氓,可實襄、鄧戶口,俾數百年無患。都御史李賓以聞,帝善之。十二年,遂命傑出撫。遍歷山谿,宣朝廷德意,諸流民欣然願附籍。於是大會湖廣、河南、陜西撫、按官籍之,得戶十一萬三千有奇,口四十三萬八千有奇。其初至,無產及平時頑梗者,驅還其鄉,而附籍者用輕則定田賦。民大悅。因相地勢,以襄陽所轄鄖縣,居竹、房、上津、商、洛諸縣中,道路四達,去襄陽五百餘里。山林阻深,將吏鮮至,猝有盜賊,府難遙制。乃拓其城,置鄖陽府,以縣附之。且置湖廣行都司,增兵設戍,而析竹山置竹谿,析鄖置鄖西,析漢中之洵陽置白河,與竹山、上津、房咸隸新府。又於西安增山陽,南陽增南召、桐柏,汝州增伊陽,各隸其舊府。制既定,薦知鄧州吳遠為鄖陽知府,諸縣皆擇鄰境良吏為之。流人得所,四境乂安。將還,以地界湖廣、河南、陜西,事無統紀,因薦御史吳道宏自代。詔即擢道宏大理少卿,撫治鄖陽、襄陽、荊州、南陽、西安、漢中六府。鄖陽之有撫治,自此始也。傑以功進右都御史。

傑數揚歷於外,既居內臺,不欲出。荊、襄之命,非其意也。事竣,急請還朝。會南京兵部缺尚書,以傑任之。傑疏辭。不許。遂卒於南陽,年六十一。鄖、襄民為立祠,詔贈太子太保,錄其子宗敏為國子生。

彭誼,字景宜,東莞人。正統中,由鄉舉除工部司務。嘗與尚書辯事,無所阿。景帝立,用薦改御史。從尚書石璞塞沙灣決河,進秩二等。復決,再往塞之。

景泰五年,以從大學士王文巡視江、淮,擒獲蘇州賊,擢大理寺丞。時年二月擢右僉都御史,提督紫荊、倒馬諸關。劾都指揮胡璽納賄縱軍罪。天順初,罷巡撫官。中朝有不悅誼者,下遷紹興知府。歲饑,輒發廩振貸。吏白當俟朝命,誼曰:「民方急,安得循故事耶?」築白馬閘障海潮。歷九載,多惠政。超擢山東左布政使,入為工部左侍郎。

成化四年,遼東巡撫張岐得罪,吏部舉代者。帝曰:「遼東自王翱後,屢更巡撫,多不稱,可於大臣中求之。」乃改誼右副都御史以往,鎮守中官橫徵諸屬衛。誼下令,凡文牒不經巡撫審定者,所司毋輒行,虐焰為息。十年冬,戶部檄所司開黑山金場。誼奏永樂中太監王彥等開是山,督夫六千人,三閱月止得金八兩,請罷之。遂止。

誼好古博學,通律曆、占象、水利、兵法之屬。平居謙厚簡默,臨事毅然有斷。鎮遼八年,軍令振肅。年未老,四疏告歸,家居四十餘年卒。

牟俸,巴人。景泰初進士。授御史,巡按雲南。南寧伯毛勝鎮金齒,俸列其違縱罪,將吏皆聳。天順元年出為福建僉事。成化初,進秩副使。久之,遷江西按察使,政尚嚴厲,入為太僕卿。

八年以左僉都御史巡撫山東。歲祲,請發濟南倉儲減價以糶,令臨清關稅收米麥濟振。皆從之。時大饑,雖獲振,饑民眾,轉徙益多。俸請敕鄰境撫、按隨所在安輯,秋成資遣復業。又乞開中淮、浙鹽百萬引,盡蠲州縣逋課。詔如所請,更命移臨清倉粟十萬石振之。至七月,俸又言公私困竭,救荒靡策,乞開納粟例,令胥吏得就選,富民授散官,且截留漕糧備振。十月復言:「今救荒者止救其饑,不謀其寒。縱得食,終不免僵死,乞貸貧民布棉。」帝皆嘉納。俸又檄發東昌、濟寧倉粟十萬餘石為軍士月糧,而以德州、臨清寄庫銀易米振濟,奏請伏專擅罪。帝特宥之。已,復以俸奏免柴夫折價銀,移河南輸邊粟濟山東,而別給銀為邊餉,山東輸京租二十萬石,給本地用。十年又饑,請發倉儲出貸。撫山東五年,盡心荒政,活饑民不可勝數。

以右副都御史改撫蘇、松。俸性嚴。以所部多巨室,欲故摧抑之,乃禁索私租,勸富家出穀備振動千計,怨謗紛然。中官汪直有事南京,或譖俸。直歸,未發也。俸初在山東,與布政陳鉞負氣不相下。後鉞從容言俸短,直信之。十四年,俸議事至京,直請執俸下詔獄。先是,所親學士江朝宗除服還朝,俸迓之九江,聯舟並下。所至,有司供張頗盛。直因謂朝宗有所關說,並下獄。詞連僉事吳扁等十餘人,俱被逮。繫獄半歲,謫戍湖廣。

俸在江西時,共成許聰獄,人多議其深文。至是被禍,皆知為直誣,然無白其冤者。踰年,卒戍所。

夏壎,字宗成,天台人。景泰二年進士。授御史。天順初,巡按福建,繼清軍江西,發鎮守中官葉達恣橫狀,達為斂威。以薦超擢廣東按察使。時用師歲久,役民守城,壎至悉遣之。

成化初,奏:「瑤、僮弗靖,用兵無功,由有司撫字乖方,賊因得誘良民為徒黨。劇寇數百,脅從萬千,進則驅之當前,退則殺以抒憤,害常在民,而利常在彼。況用兵不已,供斂日增。以易搖之人心,責無窮之軍費。恐外患未除,內變先作。請慎選監司守令,撫綏遺民,彼被脅之眾自聞風來歸。」帝深納其言。尋遷布政使,調江西。

八年以右副都御史巡撫四川。苗、僚時為寇。壎立互知會捕法,賊為之戢。古州苗萬餘,居爛土久,時議逐之,壎謂非計。松潘參將堯彧請益戍兵三千,又力陳不可。皆得寢。已,奏所部將校多犯法,奏請踰時,輒至遁逸。請先逮繫,然後奏聞。帝可之。

壎剛介,善聽斷,所至民不冤。在蜀二年,民夷畏服。然厭繁劇,與時多齟齬。子鍭獻詩勸歸,壎欣然納焉。年未五十,即求退。章四上,得請。既歸,杜門養親,不按賓客。又五年卒。

鍭,舉進士。弘治四年謁選入都,上書請復李文祥、鄒智等官,罷大學士劉吉。忤旨,下獄,得釋。久之,除南京大理評事。疏論賦斂、徭役、馬政、鹽課利弊及宗籓、戚里侵漁狀。不報。鍭素無宦情。居官僅歲餘,念母老,乞侍養,遂歸。家居三十餘年,竟不復出。

高明,字上達,貴溪人。幼事母以孝聞。登景泰二年進士,授御史。聞內苑造龍舟,切諫。有指揮為大臣所陷,論死,辯出之。徐州民訴有司於朝,時例,越訴者戍邊。明言:「戍邊,防誣訴也。今訴不誣,法止當杖。」民有為妖言者,吏貪功,誣以謀反。明按無反狀,止坐妖言律。皆報許。

巡撫河南,黜屬吏六十人。再按畿輔,入總諸道章奏。天順初,尚書陳汝言有罪,偕諸御史劾,下之獄。四年,御史趙明等劾天下朝覲官,觸帝怒,詰草疏主名。眾大懼,明獨自承。都御史寇深言:「頻年章疏,盡出明手,幸勿以細故加罪。」帝意解,反稱明能。石亨既誅,僮僕皆收。明言不宜,坐免者百人。擢大理寺丞。

憲宗立,拜南京右僉都御史。以留都春夏淫雨,請修人事以回天意。時納馬入監者至萬餘人,明請區別。薦郎中孫瓊、陳鴻漸、梅倫、何宜,主事宋瑛,皆端方廉潔,恬於進取,宜顯擢以風有位。疏下所司。

成化三年,揚州鹽寇起,守兵失利,詔明討之。造巨艦,名曰「籌亭」,往來江上督戰,並江置邏堡候望。賊縱跡無所匿,遂平之。內官鬻私鹽,據法沒入,鹽政大治。因條上利病十餘事,多議行。仍還原任,以親老乞終養歸。

十四年,上杭盜發。詔起巡撫福建,督兵往討。擒誅首惡,餘皆減死遣戍。以上杭地接江西、廣東,盜易嘯聚,請析置永定縣。移疾徑歸。久之,卒。楊繼宗,字承芳,陽城人。天順初進士。授刑部主事。囚多疫死,為時其食飲,令三日一櫛沐,全活甚眾。又善辨疑獄。河間獲盜,遣里民張文、郭禮送京師,盜逸。文謂禮曰:「吾二人並當死。汝母老,鮮兄弟,以我代盜,庶全汝母子命。」禮泣謝,從之。文桎梏詣部,繼宗察非盜,竟辨出之。

成化初,用王翱薦,擢嘉興知府。以一僕自隨,署齋蕭然。性剛廉孤峭,人莫敢犯。而時時集父老問疾苦,為祛除之。大興社學,民間子弟八歲不就學者,罰其父兄;遇學官以賓禮。師儒競勸,文教大興。御史孔儒清軍,里老多撻死。繼宗榜曰:「御史杖人至死者,詣府報名。」儒怒。繼宗入見曰:「為治有體。公但剔姦弊,勸懲官吏。若比戶稽核,則有司事,非憲體也。」儒不能難,而心甚銜之。瀕行,突入府署,發篋視之,敝衣數襲而已。儒慚而去。中官過者,繼宗遺以菱芡、歷書。中官索錢,繼宗即發牒取庫金,曰:「金具在,與我印券。」中官咋舌不敢受。入覲,汪直欲見之,不可。憲宗問直:「朝覲官孰廉?」直對曰:「天下不愛錢者,惟楊繼宗一人耳。」九載秩滿,超遷浙江按察使。數與中官張慶忤。慶兄敏在司禮,每於帝前毀繼宗。帝曰:「得非不私一錢之楊繼宗乎?」敏惶恐,遺書慶曰:「善遇之,上已知其人矣。」聞母喪,立出。止驛亭下,盡籍廨中器物付有司。惟攜一僕、書數卷而還。

服除,以右僉都御史巡撫順天。畿內多權貴莊田,有侵民業者,輒奪還之。按行關塞,武備大飭。星變,應詔陳言,歷指中官及文武諸臣貪殘狀,且請召還中官出鎮者。益為權貴所嫉。治中陳翼訐其過,權貴因中之,左遷雲南副使。

孝宗立,遷湖廣按察使。既至,命汲水百斛,洗滌廳事而後視事,曰:「吾以除穢也。」居無何,復以僉都御史巡撫雲南。三司多舊僚,相見歡然。既而出位揖之曰:「明日有公事,諸君幸相諒。」遂劾罷不職者八人。未幾卒。

繼宗力持風節,而居心慈厚,自處必以禮。為知府,謁上官必衣繡服,朝覲謁吏部亦然。或言不可,笑曰:「此朝廷法服也,此而不服,將安用之?」為浙江按察時,倉官十餘人坐缺糧繫獄,至鬻子女以償。繼宗欲寬之而無由。一日,送月俸至,命量之,則溢原數。較他司亦然。因悟倉吏缺糧之由,將具實以聞。眾懼,請於繼宗,願捐俸代償。由是十人者獲釋。嘗監鄉試得二卷,具朝服再拜曰:「二子當大魁天下,吾為朝廷得人賀耳。」及拆卷,王華、李旻也,後果相繼為狀元。人服其鑑。天啟初,謚貞肅。

贊曰:明初以十五布政司分治天下,諸邊要害則遣侯伯勳臣鎮扼之。永樂之季,敕蹇義等二十六人巡行天下,安撫軍民,事竣還朝,不為經制。宣德初,始命熊概巡撫蘇、松、兩浙。越數年,而江西、河南諸省以次專設巡撫官。天順初,暫罷復設,諸邊亦稍用廷臣出鎮或參贊軍務。蓋以地大物眾,法令滋章,三司謹奉教條,修其常職;而興利除弊,均賦稅,擊貪濁,安善良,惟巡撫得以便宜從事。熊概以下諸人,強幹者立聲威,愷悌者流惠愛,政績均有可紀。于謙、周忱巡撫最為有名,而勳業尤盛,故別著焉。


\end{pinyinscope}