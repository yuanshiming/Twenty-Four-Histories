\article{列傳第五十六}

\begin{pinyinscope}
陳循蕭鎡王文江淵許彬陳文萬安彭華劉珝子鈗劉吉尹直

陳循,字德遵,泰和人。永樂十三年進士第一。授翰林修撰。習朝廷典故。帝幸北京,命取秘閣書詣行在,遂留侍焉。

洪熙元年,進侍講。宣德初,受命直南宮,日承顧問。賜第玉河橋西,巡幸未嘗不從。進侍講學士。正統元年兼經筵官。久之,進翰林院學士。九年入文淵閣,典機務。

初,廷議天下吏民建言章奏,皆三楊主之。至是榮、士奇已卒,循及曹鼐、馬愉在內閣,禮部援故事請。帝以楊溥老,宜優閒,令循等預議。明年進戶部右侍郎,兼學士。土木之變,人心洶懼。循居中,所言多採納。進戶部尚書,兼職如故。也先犯京師,請敕各邊精騎入衛,馳檄回番以疑敵。帝皆從其計。

景泰二年,以葬妻與鄉人爭墓地,為前後巡按御史所不直,循輒訐奏。給事中林聰等極論循罪。帝是聰言,而置循不問。循本以才望顯,及是素譽隳焉。

二年十二月進少保兼文淵閣大學士。帝欲易太子,內畏諸閣臣,先期賜循及高穀白金百兩,江淵、王一寧、蕭鎡半之。比下詔議,循等遂不敢諍,加兼太子太傅。尋以太子令旨賜百官銀帛。踰月,帝復賜循等六人黃金五十兩,進華蓋殿大學士,兼文淵閣如故。循子英及王文子倫應順天鄉試被黜,相與構考官劉儼、黃諫,為給事中張寧等所劾。帝亦不罪。

英宗復位,于謙、王文死,杖循百,戍鐵嶺衛。

循在宣德時,御史張楷獻詩忤旨。循曰「彼亦忠愛也」,遂得釋。御史陳祚上疏,觸帝怒,循婉為解,得不死。景帝朝,嘗集古帝王行事,名《勤政要典》,上之。河南江北大雪,麥苗死,請發帑市麥種給貧民。因事進言,多足採者。然久居政地,刻躁為士論所薄。其嚴譴則石亨輩為之,非帝意也。

亨等既敗,循自貶所上書自訟,言:「天位,陛下所固有。當天與人歸之時,群臣備法駕大樂,恭詣南內,奏請臨朝。非特宮禁不驚,抑亦可示天下萬世。而亨等儌倖一時,計不出此,卒皆自取禍敗。臣服事累葉,曾著微勞,實為所擠,惟陛下憐察。」詔釋為民,一年卒。成化中,于謙事雪,循子引例請恤,乃復官賜祭。

同邑蕭鎡。字孟勤。宣德二年進士,需次於家。八年,帝命楊溥合選三科進士,拔二十八人為庶吉士,鎡為首。英宗即位,授編修。正統三年進侍讀。久之,代李時勉為國子監祭酒。景泰元年以老疾辭。既得允,監丞鮑相率六館生連章乞留。帝可其奏。明年以本官兼翰林學士,與侍郎王一寧並入直文淵閣。又明年進戶部右侍郎,兼官如故。易儲議起,鎡曰:「無易樹子,霸者所禁,矧天朝乎。」不聽。加太子少師。《寰宇通志》成,進戶部尚書。帝不豫,諸臣議復憲宗東宮。李賢私問鎡,鎡曰:「既退,不可再也。」英宗復位,遂削籍。天順八年卒。成化中,復官賜祭。鎡學問該博,文章爾雅。然性猜忌,遇事多退避云。

王文,字千之,初名強,束鹿人。永樂十九年進士。授監察御史。持廉奉法,為都御史顧佐所稱。宣德末,奉命治彰德妖賊張普祥獄。還奏稱旨,賜今名。

英宗即位,遷陜西按察使。遭父憂,命奔喪,起視事。正統三年正月擢右副都御史,巡撫寧夏,五年召為大理寺卿。明年與刑部侍郎何文淵錄在京刑獄,尋遷右都御史。九年出視延綏、寧夏邊務。劾治定邊營失律都督僉事王禎、都督同知黃真等罪,邊徼為肅。明年代陳鎰鎮守陜西,平涼、臨洮、鞏昌饑,奏免其租。尋進左都御史。在陜五年,鎮靜不擾。

景泰改元,召掌院事。文為人深刻有城府,面目嚴冷,與陳鎰同官,一揖外未嘗接談。諸御史畏之若神,廷臣無敢干以私者,然中實柔媚。初,按大理少卿薛瑄獄,希王振指,欲坐瑄死。至是治中官金英縱家奴不法事,但抵奴罪。給事中林聰等劾文、鎰畏勢長奸,下詔獄。二人俱伏,乃宥之。二年六月,學士江淵上言法司斷獄多枉。文及刑部尚書俞士悅求罷。且言淵嘗私以事,不聽,故見誣。帝兩置之。

三年春,加太子太保。時陳鎰鎮陜西,將還,文當代。諸御史交章留之,乃改命侍郎耿九疇。南京地震,江、淮北大水,命巡視。偕南九卿議上軍民便宜九事。又言徐、淮間饑甚,而南京儲蓄有餘,請盡發徐、淮倉粟振貸,而以應輸南京者輸徐、淮,補其缺。皆報可。

是時,陳循最任,好剛自用。高穀與循不相能,以文彊悍,思引與共政以敵之,乃疏請增閣員。循舉其鄉人蕭維禎,穀遂舉文。而文得中官王誠助,於是詔用文。尋自江、淮還朝,改吏部尚書,兼翰林院學士,直文淵閣。二品大臣入閣自文始。尋遭母喪,奪哀如前。文雖為穀所引,而穀遲重,循性明決,文反與循合而不附穀。其後以子倫故,欲傾考官,又用穀言而罷。由是兩人卒不相得。

五年三月,江、淮大水,復命巡視。先是蘇、松、常、鎮四府糧四石折白銀一兩,民以為便。後戶部復徵米,令輸徐、淮,凡一百十餘萬石。率三石而致一石,有破家者。文用便宜停之。又發廩振饑民三百六十餘萬。時年饑多盜,文捕長洲盜許道師等二百人。欲張其功,坐以謀逆。大理卿薛瑄辨其誣。給事中王鎮乞會廷臣勘實,得為盜者十六人置之法,而餘得釋。還進少保,兼東閣大學士。再進謹身殿大學士,仍兼東閣。

初,英宗之還也,廷臣議奉迎禮。文時為都御史,厲聲曰:「公等謂上皇果還耶?也先不索土地、金帛而遽送駕來耶?」眾素畏文,皆愕然不決而罷。及易儲議起,文率先承命。景帝不豫,群臣欲乞還沂王東宮。文曰:「安知上意誰屬?」乃疏請早選元良。以是中外喧傳文與中官王誠等謀召取襄世子。

英宗復位,即日與于謙執於班內。言官劾文與謙等謀立外籓,命鞫於廷。文力辯曰:「召親王須用金牌信符,遣人必有馬牌,內府兵部可驗也。」辭氣激壯。逮車駕主事沈敬按問,無迹。廷臣遂坐謙、文召敬謀未定,與謙同斬於市,諸子悉戍邊。敬亦坐知謀反故縱,減死,戍鐵嶺。文之死,人皆知其誣。以素刻忮,且迎駕、復儲之議不愜輿論,故冤死而民不思。成化初,赦其子還,尋復官,贈太保,謚毅愍。

倫,改名宗彞。成化初進士。歷戶部郎中,出理遼東餉。中官汪直東征,言宗彞督餉勞,擢太僕少卿。弘治中,累官南京禮部尚書。卒,謚安簡。

江淵,字世用,江津人。宣德五年庶吉士,授編修。正統十二年詔與杜寧、裴綸、劉儼、商輅、陳文、楊鼎、呂原、劉俊、王玉共十人,肄業東閣,曹鼐等為之師。

郕王監國,徐有貞倡議南遷,太監金英叱出之,踉蹌過左掖門。淵適入,迎問之。有貞曰:「以吾議南遷不合也。」於是淵入,極陳固守之策。遂見知於王,由侍講超擢刑部右侍郎。也先薄京師,命淵參都督孫鏜軍事。

景泰元年出視紫荊、倒馬、白羊諸關隘,與都指揮同知翁信督修雁門關。其秋遂以本官兼翰林學士,入閣預機務。尋改戶部侍郎,兼職如故。明年六月以天變條上三事:一,厚結朵顏、赤斤諸衛,為東西籓籬;一,免京軍餘丁,以資生業;一,禁訐告王振餘黨,以免枉濫。詔悉從之。又明年二月改吏部,仍兼學士。是春,京師久雨雪。淵上言:「漢劉向曰,凡雨陰也,雪又雨之陰也。仲春少陽用事,而寒氣脅之,占法謂人君刑法暴濫之象。陛下恩威溥洽,未嘗不赦過宥罪,竊恐有司奉行無狀,冤抑或有未伸。且向者下明詔,免景泰二年田租之三。今復移檄追徵,則是朝廷自失大信於民。怨氣鬱結,良由此也。」帝乃令法司申冤濫,詰戶部違詔,下尚書金濂於獄,卒免稅加詔。東宮既易,加太子少師。四川巡撫僉都御史李匡不職,以淵言罷之。母憂起復。初侍講學士倪謙遭喪,淵薦謙為講官,謙遂奪哀。至是御史周文言淵引謙,正自為今日地。帝以事既處分,不問,而令自今群臣遭喪無濫保。

五年春,山東、河南、江北饑,命同平江侯陳預往撫。淵前後條上軍民便宜十數事。并請築淮安月城以護常盈倉,廣徐州東城以護廣運倉。悉議行。時江北洊饑,淮安糧運在塗者,淵悉追還備振,漕卒乘機侵耗。事聞,遣御史按實。淵被劾。當削籍。廷臣以淵守便宜,不當罪。帝宥之。

閣臣既不相協,而陳循、王文尤刻私。淵好議論,每為同官所抑,意忽忽不樂。會兵部尚書于謙以病在告,詔推一人協理部事。淵心欲得之。循等佯推淵,而密令商輅草奏,示以「石兵江工」四字,淵在旁不知也。比詔下,調工部尚書石璞於兵部,而以淵代璞。淵大失望。英宗復位,與陳循等俱謫戍遼東,未幾卒。

初,黃矰之奏易儲也,或疑淵主之。丘濬曰:「此易辨也,廣西紙與京師紙異。」索奏視之,果廣西紙,其誣乃白。成化初,復官。

許彬,字道中,寧陽人。永樂十三年進士。改庶吉士,授檢討。正統末,累遷太常少卿,兼翰林待詔,提督四夷館。上皇將還,遣彬至宣府奉迎。上皇命書罪己詔及諭群臣敕,遣祭土木陣亡官軍。以此受知上皇。還擢本寺卿。石亨等謀復上皇,以其謀告彬,彬進徐有貞,語具有貞傳。英宗復位,進禮部左侍郎,兼翰林院學士。入直文淵閣。未幾,為石亨所忌,出為南京禮部右侍郎,甫行,貶陜西參政。至則乞休去。憲宗立,命以侍郎致仕,尋卒。

彬性坦率,好交游,不能擇人,一時浮蕩士多出其門。晚參大政,方欲杜門謝客,而客惡其變態,競相騰謗,竟不安其位。

陳文,字安簡,廬陵人。鄉試第一,正統元年進士及第,授編修。十二年命進學東閣。秩滿,遷侍講。

景泰二年,閣臣高穀薦文才,遂擢雲南右布政使,貴州比歲用兵,資餉雲南,民困轉輸。文令商賈代輸,而民倍償其費,皆稱便。稅課額鈔七十餘萬,吏俸所取給,典者侵蝕,吏或累歲不得俸。文悉按治,課日羨溢。雲南產銀,民間用銀貿易,視內地三倍。隸在官者免役,納銀亦三之,納者不為病。文曰:「雖如是,得無傷廉乎?」損之,復令減隸額三之一。名譽日起,遷廣東左布政使,母憂未赴。

英宗即復位,一日謂左右曰:「向侍朕編修,皙而長者安在?」左右以文對,即召為詹事。乞終制。不允。入侍東宮講讀。學士呂原卒,帝問李賢誰可代者,曰:「柯潛可。」出告王翱,翱曰:「陳文以次當及,奈何抑之?」明日,賢入見,如翱言。

七年二月進禮部右侍郎兼學士,入內閣。文既入,數撓賢以自異,曰:「吾非若所薦也。」侍讀學士錢溥與文比舍居,交甚歡。溥嘗授內侍書。其徒多貴幸,來謁,必邀文共飲。英宗大漸,東宮內侍王綸私詣溥計事,不召文。文密覘之。綸言:「帝不豫,東宮納妃,如何?」溥謂:「當奉遺詔行事。」已而英宗崩,賢當草詔。文起奪其筆曰:「無庸,已有草者。」因言綸、溥定計,欲逐賢以溥代之,而以兵部侍郎韓雍代尚書馬昂。賢怒,發其事。是時憲宗初立,綸自謂當得司禮,氣張甚。英宗大殮,綸衰服襲貂,帝見而惡之。太監牛玉恐其軋己,因數綸罪,逐之去。溥謫知順德縣,雍浙江參政。詞所連,順天府尹王福,通政參議趙昂,南寧伯毛榮,都督馬良、馮宗、劉聚,錦衣都指揮僉事門達等皆坐謫。雍亦文素所不悅者也。改吏部左侍郎,同知經筵事。

成化元年進禮部尚書。羅倫論賢奪情。文內媿,陰助賢逐倫,益為時論所鄙。三年春,帝命戶部尚書馬昂、副都御史林聰及給事中潘禮、陳越清理京營。文奏必得內臣共事,始可刬除宿弊,因薦太監懷恩。帝從之。《英宗實錄》成,加太子少保,兼文淵閣大學士。四年卒。贈少傅,謚莊靖。

文素以才自許,在外頗著績效,士大夫多冀其進用。及居宮端,行事鄙猥。既參大政,無所建明。朝退則引賓客故人置酒為曲宴,專務請屬。性卞急,遇睚眥怨必報。及賢卒,文益恣意行,名節大喪。歿後,禮部主事陸淵之、御史謝文祥皆疏論文不當得美謚。帝以事已施行,不許。

萬安,安循吉,眉州人。長身魁顏,眉目如刻畫,外寬而深中。正統十三年進士。改庶吉士,授編修。

成化初,屢遷禮部左侍郎。五年命兼翰林學士,入內閣參機務。同年生詹事李泰,中官永昌養子也,齒少於安。安兄事之,得其歡。自為同官,每當遷,必推安出己上。至是議簡閣臣,泰復推安曰:「子先之,我不患不至。」故安得入閣,而泰忽暴病死。

安無學術,既柄用,惟日事請託,結諸閹為內援。時萬貴妃寵冠後宮,安因內侍致殷勤,自稱子姪行。妃嘗自媿無門閥,聞則大喜,妃弟錦衣指揮通,遂以族屬數過安家。其妻王氏有母至自博興。王謂母曰:「嚮家貧時,以妹為人娣,今安在?」母曰:「第憶為四川萬編修者。」通心疑是安,訪之則安小婦,由是兩家婦日往來。通妻著籍禁內,恣出入,安得備知宮中動靜,益自固。侍郎刑讓、祭酒陳鑑與安同年不相能。安構獄,除兩人名。

七年冬,彗見天田,犯太微。廷臣多言君臣否隔,宜時召大臣議政。大學士彭時、商輅力請。司禮中官乃約以御殿日召對,且曰:「初見,情未洽,勿多言,姑俟他日。」將入,復約如初。比見,時言天變可畏,帝曰:「已知,卿等宜盡心。」時又言:「昨御史有疏,請減京官俸薪,武臣不免觖望,乞如舊便。」帝可之。安遂頓首呼萬歲。欲出,時、輅不得已,皆叩頭退。中官戲朝士曰:「若輩嘗言不召見。及見,止知呼萬歲耳。」一時傳笑,謂之「萬歲閣老」。帝自是不復召見大臣矣。

其後尹直入閣,欲請見帝計事。安止之曰:「往彭公請召對,一語不合,輒叩頭呼萬歲,以此貽笑。今吾輩每事盡言,太監擇而聞之,上無不允者,勝面對多矣。」其容悅不識大體,且善歸過於人如此。

九年進禮部尚書。久之,改戶部。十三年加太子少保,俄改文淵閣大學士。孝宗出閤,進吏部尚書、謹身殿大學士,尋加太子太保。時彭時已歿,商輅以忤汪直去,在內閣者劉珝、劉吉。而安為首輔,與南人相黨附;珝與尚書尹旻、王越又以北人為黨,互相傾軋。然珝疏淺而安深鷙,故珝卒不能勝安。

十八年,汪直寵衰,言官請罷西廠。帝不許。安具疏再言之,報可,中外頗以是稱安。《文華大訓》成,進太子太傅、華蓋殿大學士。復進少傅、太子太師,再進少師。

當是時,朝多秕政,四方災傷日告。帝崇信道教,封金闕、玉闕真君為上帝,遣安祭於靈濟宮。而李孜省、鄧常恩方進用,安因彭華潛與結,藉以排異己。於是珝及王恕、馬文升、秦紘、耿裕諸大臣相繼被逐,而華遂由詹事遷吏部侍郎,入內閣。朝臣無敢與安牴牾者。

華,安福人,大學士時之族弟,舉景泰五年會試第一。深刻多計數,善陰伺人短,與安、孜省比。嘗嗾蕭彥莊攻李秉,又逐尹旻、羅璟,人皆惡而畏之。踰年,得風疾去。

孝宗嗣位,安草登極詔書,禁言官假風聞挾私,中外嘩然。御史湯鼐詣閣。安從容言曰:「此裏面意也。」鼐即以其語奏聞,謂安抑塞言路,歸過於君,無人臣禮。於是庶吉士鄒智,御史文貴、姜洪等交章列其罪狀。先是,歙人倪進賢者,粗知書,無行,諂事安,日與講房中術。安暱之,因令就試,得進士。授為庶吉士,除御史。帝一日於宮中得疏一小篋,則皆論房中術者,末署曰「臣安進」。帝命太監懷恩持至閣曰:「此大臣所為耶?」安媿汗伏地,不能出聲。及諸臣彈章入,復令恩就安讀之。安數跪起求哀,無去意。恩直前摘其牙牌曰:「可出矣。」始惶遽索馬歸第,乞休去。時年已七十餘。尚於道上望三台星,冀復用。居一年卒,贈太師,謚文康。

初,孝穆皇太后之薨,內庭籍籍指萬貴妃。孝宗立,魚臺縣丞徐項上書發其事。廷臣議逮鞫萬氏戚屬曾出入宮掖者。安驚懼不知所為,曰:「我久不與萬氏往來矣。」而劉吉先與萬氏姻,亦自危。其黨尹直尚在閣,共擬旨寢之。孝宗仁厚,亦置不問,安、吉得無事。

安在政府二十年,每遇試,必令其門生為考官,子孫甥婿多登第者。子翼,南京禮部侍郎。孫弘璧,翰林編修。安死無幾,翼、弘璧相繼死,安竟無後。

劉珝,字叔溫,壽光人。正統十三年進士。改庶吉士,授編修。天順中,歷右中允,侍講東宮。

憲宗即位,以舊宮僚屢遷太常卿,兼侍讀學士,直經筵日講。成化十年進吏部左侍郎,充講官如故。珝每進講,反覆開導,詞氣侃侃,聞者為悚。學士劉定之稱為講官第一,憲宗亦愛重之。明年詔以本官兼翰林學士,入閣預機務。帝每呼「東劉先生」,賜印章一,文曰「嘉猷贊翊」。尋進吏部尚書,再加太子少保、文淵閣大學士。《文華大訓》成。加太子太保,進謹身殿大學士。

珝性疏直。自以宮僚舊臣,遇事無所回護。員外郎林俊以劾梁芳、繼曉下獄,珝於帝前解之。李孜省輩左道亂政,欲動搖東宮。珝密疏諫,謀少阻。素薄萬安,嘗斥安負國無恥。安積忿,日夜思中珝。初,商輅之劾汪直也,珝與萬安、劉吉助之爭,得罷西廠。他日,珝又折王越於朝,越慚而退。已而西廠復設,珝不能有所諍。至十八年,安見直寵衰,揣知西廠當罷,邀珝同奏。珝辭不與,安遂獨奏。疏上,帝頗訝無珝名。安陰使人訐珝與直有連。會珝子鎡邀妓狎飲,里人趙賓戲為《劉公子曲》,或增飾穢語,雜教坊院本奏之。帝大怒,決意去珝。遣中官覃昌召安、吉赴西角門,出帝手封書一函示之。安等佯驚救。次日,珝具疏乞休。令馳驛,賜月廩、歲隸、白金、楮幣甚厚。其實排珝使去者,安、吉兩人謀也。

時內閣三人,安貪狡,吉陰刻。珝稍優,顧喜譚論,人目為狂躁。珝既倉卒引退,而彭華、尹直相繼入內閣,安、吉之黨乃益固。珝初遭母憂,廬墓三年。比歸,侍父盡孝。父歿,復廬於墓。弘治三年卒,謚文和。嘉靖初,以言官請,賜祠額曰「昭賢」,仍遣官祭之。

子鈗,字汝中。八歲時,憲宗召見,愛其聰敏,且拜起如禮,即命為中書舍人。宮殿門閾高,同官楊一清常提之出入。帝慮牙牌易損,命易以銀。歷官五十餘年,嘉靖中至太常卿,兼五經博士,仍供事內閣誥敕房。博學有行誼,與長洲劉棨並淹貫故實,時稱「二劉」。

劉吉,字祐之,博野人。正統十三年進士。改庶吉士,授編修,充經筵官。《寰宇通志》成,進修撰。天順四年侍講讀於東宮,以憂歸。

憲宗即位。召纂《英宗實錄》。至京,上疏乞終制。不允,進侍讀。《實錄》成,遷侍讀學士,直經筵。累遷禮部左侍郎。

成化十一年與劉珝同受命,兼翰林學士,入閣預機務。尋進禮部尚書。孝宗出閤,加太子少保兼文淵閣大學士。十八年遭父喪,詔起復。吉三疏懇辭,而陰屬貴戚萬喜為之地,得不允。《文華大訓》成,加太子太保,進武英殿大學士。久之,進戶部尚書、謹身殿大學士,尋加少保兼太子太傅。

孝宗即位,庶吉士鄒智、御史姜洪力詆萬安、尹直及吉皆小人,當斥。吉深銜之。安、直皆去,吉獨留,委寄愈專。慮言者攻不已,乃建議超遷科道官,處以不次之位。詔起廢滯,給事中賀欽、御史強珍輩十人已次第擬擢,吉復上疏薦之。部曹預薦者惟林俊一人,冀以此籠絡言路,而言者猶未息。庶子張升,御史曹璘、歐陽旦,南京給事中方向,御史陳嵩等相繼劾吉。吉憤甚,中昇逐之。數興大獄,智、向囚繫遠貶,洪亦謫官。復與中官蔣琮比,逐南御史姜綰等,臺署為空。中外側目,言者亦少衰。

初,吉與萬安、劉珝在成化時,帝失德,無所規正,時有「紙糊三閣老,泥塑六尚書」之謠。至是見孝宗仁明,同列徐溥、劉健皆正人,而吉於閣臣居首,兩人有論建,吉亦暑名,復時時為正論,竊美名以自蓋。

弘治二年二月旱,帝令儒臣撰文禱雨。吉等言:「邇者奸徒襲李孜省、鄧常恩故術,見月宿在畢,天將陰雨,遂奏請祈禱,覬一驗以希進用。倖門一開,爭言祈禱,要寵召禍,實基於此。祝文不敢奉詔。」帝意悟,遂已之。五月以災異請帝修德防微,慎終如始。八月又以災異陳七事。代王獻海青,吉等言登極詔書已卻四方貢獻,乞勿受。明年三月偕同列上言:「陛下聖質清羸,與先帝不同。凡宴樂游觀,一切嗜好之事,宜悉減省。左右近臣有請如先帝故事者,當以太祖、太宗典故斥退之。祖宗令節宴游皆有時,陛下法祖宗可也。」土魯番使者貢獅子還,帝令內閣草敕,遣中官送之。吉等言不宜優寵太過,使番戎輕中國。事遂寢。既又言:「獅子諸獸,日飼二羊,歲當用七百二十,又守視校尉日五十人,皆繁費。宜絕諸獸食,聽自斃。」帝不能用。十二月,星變,又言:「邇者妖星出天津,歷杵臼,迫營室,其占為兵,為饑,為水旱。今兩畿、河南、山西、陜西旱蝗;四川、湖廣歲不登。倘明年復然,恐盜賊竊發,禍亂將作。願陛下節用度,罷宴游,屏讒言,斥異教,留懷經史,講求治道。沙河修橋,江西造瓷器,南海子繕垣牆,俱非急務,宜悉停止。」帝嘉納之。帝惑近習言,頗崇祈禱事,發經牌令閣臣作贊,又令擬神將封號。吉等極言邪說當斥。

吉自帝初即位進少傅,兼太子太師,吏部尚書。及《憲宗實錄》成。又進少師、華蓋殿大學士。吉柄政久,權勢烜赫。帝初傾心聽信,後眷頗衰。而吉終無去志。五年,帝欲封后弟伯爵,命吉撰誥券。吉言必盡封二太后家子弟方可。帝不悅,遣中官至其家,諷令致仕,始上章引退。良賜敕,馳驛如故事。

吉多智數,善附會,自緣飾,銳於營私,時為言路所攻。居內閣十八年,人目之為「劉綿花」,以其耐彈也。吉疑其言出下第舉子,因請舉人三試不第者,不得復會試。時適當會試期,舉子已群集都下,禮部為請。詔姑許入試,後如令。已而吉罷,令亦不行。吉歸,踰年卒。贈太師,謚文穆。

尹直,字正言,泰和人。景泰五年進士。改庶吉士,授編修。

成化初,充經筵講官,與修《英宗實錄》。總裁欲革去景泰帝號,引漢昌邑、更始為比。直辨曰:「《實錄》中有初為大臣,後為軍民者。方居官時,則稱某官某,既罷去而後改稱。如漢府以謀逆降庶人,其未反時,書王書叔如故也。豈有逆計其反,而即降從庶人之號者哉!且昌邑旋立旋廢,景泰帝則為宗廟社稷主七年。更始無所受命,景泰帝則策命於母后。當時定傾危難之中,微帝則京師非國家有。雖易儲失德,然能不惑於盧忠、徐振之言,卒全兩宮,以至今日。其功過足相準,不宜去帝號。」時不能難。既成,進侍讀,歷侍讀學士。

六年上疏乞纂修《大明通典》,並續成《宋元綱目》,章下所司。十一年遷禮部右侍郎,辭,不許。丁父憂,服除,起南京吏部右侍郎,就改禮部左侍郎。

二十二年春,召佐兵部。占城王古來為安南所逼,棄國來求援。議者欲送之還,直曰:「彼窮來歸,我若驅使還國,是殺之也。宜遣大臣即詢,量宜處置。」詔從之,命都御史屠滽往。貴州鎮巡官奏苗反,請發兵,廷議將從之。直言起釁邀功,不可信。命官往勘,果無警。是年九月改戶部兼翰林學士,入內閣。踰月,進兵部尚書,加太子太保。

直明敏博學,練習朝章,而躁於進取。性矜忌,不自檢飭,與吏部尚書尹旻相惡。直初覬禮部侍郎,而旻薦他人。直以中旨得之。次日遇旻於朝,舉笏謝。旻曰:「公所謂簡在帝心者。」自是怨益深。後在南部八年,鬱鬱不得志,屬其黨萬安、彭華謀內召,旻輒持不可。諸朝臣亦皆畏直,幸其在南。及推兵部左、右侍郎,吏部列何琮等八人。詔用琮,而直以安、華及李孜省力,中旨召還。至是修怨,與孜省等比。陷旻父子得罪,又構罷江西巡撫閔珪,物論喧然不平。刑部郎袁清者,安私人,又幸於內侍郭閏。勘事浙江,輘轢諸大吏,吏部尚書李裕惡之。比還,即除紹興知府。清懼,累章求改,裕極論其罪,下詔獄。安、閏以屬直,為言於孜省,取中旨赦之,改知鄖陽。

孝宗立,進士李文祥,御史湯鼐、姜洪、繆樗,庶吉士鄒智等連章劾直。給事中宋琮及御史許斌言直自初為侍郎以至入閣,夤緣攀附,皆取中旨。帝於是薄其為人,令致仕。弘治九年表賀萬壽,並以太子年當出閤,上《承華箴》,引先朝少保黃淮事,冀召對。帝卻之。正德中卒,謚文和。

贊曰:《易》稱內君子外小人,為泰;外君子內小人,為否。況端揆之寄,百僚具瞻者乎!陳循以下諸人,雖不為大奸慝,而居心刻忮,務逞己私。同己者比,異己者忌;比則相援,忌則相軋。至萬安、劉吉要結近幸,蒙恥固位。猶幸同列多賢,相與彌縫匡救,而穢跡昭彰,小人之歸,何可掩哉!


\end{pinyinscope}