\article{列傳第二十五}

\begin{pinyinscope}
劉三吾(汪睿硃善))安然王本等吳伯宗鮑恂任亨泰吳沉桂彥良李希顏徐宗實陳南賓劉淳董子莊趙季通楊黼金實等宋訥許存仁張美和聶鉉貝瓊趙俶錢宰蕭執李叔正劉崧羅復仁孫汝敬

劉三吾,茶陵人。初名如孫,以字行。兄耕孫、燾孫皆仕元。耕孫,寧國路推官,死長槍賊難。燾孫,常寧州學正,死僚寇。三吾避兵廣西,行省承制授靜江路儒學副提舉。明兵下廣西,乃歸茶陵。洪武十八年,以茹瑺薦召至,年七十三矣,奏對稱旨,授左贊善,累遷翰林學士。時天下初平,典章闕略。帝銳意制作,宿儒凋謝,得三吾晚,悅之。一切禮制及三場取士法多所刊定。

三吾博學,善屬文。帝製《大誥》及《洪範注》成,皆命為序。敕修《省躬錄》、《書傳會選》、《寰宇通志》、《禮制集要》諸書,皆總其事,賜賚甚厚。帝嘗曰:「朕觀奎壁間嘗有黑氣,今消矣,文運其興乎。卿等宜有所述作,以稱朕意。」帝製詩,時令屬和。嘗賜以朝鮮玳瑁筆。朝參,命列侍衛前;燕享,賜坐殿中。與汪睿、朱善稱「三老。」既而三吾年日益老,才力日益減,往往忤意,禮遇亦漸輕。二十三年,授晉世子經,吏部侍郎侯庸劾其怠職。降國子博士,尋還職。

三吾為人慷慨,不設城府,自號「坦坦翁」。至臨大節,屹乎不可奪。懿文太子薨,帝御東閣門,召對群臣,慟哭。三吾進曰:「皇孫世嫡承統,禮也。」太孫之立由此。戶部尚書趙勉者,三吾婿也,坐贓死。三吾引退。許之。未幾,復為學士。三十年偕紀善白信蹈等主考會試。榜發,泰和宋琮第一,北士無預者。於是諸生言三吾等南人,私其鄉。帝怒,命侍講張信等覆閱,不稱旨。或言信等故以陋卷呈,三吾等實屬之。帝益怒,信蹈等論死,三吾以老戍邊,琮亦遣戍。帝親賜策問,更擢六十一人,皆北士。時謂之「南北榜」,又曰「春夏榜」云。建文初,三吾召還,久之,卒。

琮起刑部檢校。鄉人楊士奇輩貴顯,琮無所攀援。宣德中猶以檢討掌助教事,卒官。

汪睿,字仲魯,婺源人。元末與弟同集眾保鄉邑,助復饒州。授浮梁州同知,不就。胡大海克休寧,睿兄弟來附,設星源翼分院於婺源,以同為院判。睿歸田里。庚子秋,同將兵爭鄱陽,不克,棄妻孥,亡之浙西。幕府疑之,檄睿入應天為質。已,聞同為張士誠所殺,乃授睿安慶稅令。未幾,徵參贊川蜀軍事。以疾辭去。洪武十七年,復召見,命講《西伯戡黎》篇,授左春坊左司直。常命續《薰風自南來》詩及他應制,皆稱旨。請春夏停決死罪,體天地生物之仁,從之。踰年,疾作,請假歸。睿敦實閒靜,不妄言笑,及進講,遇事輒言。帝嘗以「善人」呼之。

朱善,字備萬,豐城人。九歲通經史大義,能屬文。元末兵亂,隱山中,事繼母以孝聞。洪武初,為南昌教授。八年,廷對第一,授修撰。踰年,奏對失旨,改典籍,放還鄉。復召為翰林待詔。上疏論婚姻律曰:「民間姑舅及兩姨子女,法不得為婚。仇家詆訟,或已聘見絕,或既婚復離,甚至兒女成行,有司逼奪。按舊律:尊長卑幼相與為婚者有禁。蓋謂母之姊妹,與己之身,是為姑舅兩姨,不可以卑幼上匹尊屬。若姑舅兩姨子女,無尊卑之嫌。成周時,王朝相與為婚者,不過齊、宋、陳、巳。故稱異姓大國曰「伯舅」,小國曰「叔舅」。列國齊、宋、魯、秦、晉,亦各自為甥舅之國。後世,晉王、謝,唐崔、盧,潘、楊之睦,朱、陳之好,皆世為婚媾。溫嶠以舅子娶姑女,呂滎公夫人張氏即其母申國夫人姊女。古人如此甚多,願下群臣議,馳其禁。」帝許之。十八年擢文淵閣大學士。嘗講《家人卦》、《心箴》,帝大悅。未幾,請告歸。卒年七十二。著有《詩經解頤》、《史輯》傳於世。正德中,謚文恪。

安然,祥符人,徙居潁州。元季以左丞守萊州。明兵下山東,率眾歸附。累官山東參政。撫綏流移,俸餘悉給公用,帝聞而嘉之。洪武二年,召為工部尚書,出為河南參政,歷浙江布政使,入為御史臺右大夫。十三年改左中丞,坐事免。未幾,召為四輔官。

先是,胡惟庸謀反伏誅,帝以歷代丞相多擅權,遂罷中書省,分其職於六部。既又念密勿論思不可無人,乃建四輔官,以四時為號,詔天下舉賢才。戶部尚書范敏薦耆儒王本、杜佑、龔斅,杜斅、趙民望、吳源等。召至,告於太廟,以本、佑、龔斅為春官;杜斅、民望、源為夏官。秋、冬闕,命本等攝之。位都督次,屢賜敕諭,隆以坐論之禮,命協贊政事,均調四時。會立冬,朔風釀寒。帝以為順冬令,乃本等功,賜敕嘉勉。又月分三旬,人各司之,以雨暘時若,驗其稱職與否。刑官議獄,四輔及諫院覆核奏行,有疑讞,四輔官封駁。

居無何,斅等四人相繼致仕,召然代之。本後坐事誅。諸人皆老儒,起田家,惇朴無他長。獨然久歷中外,練達庶務,眷注特隆。十四年八月卒。帝念然來歸之誠,親製文祭之。繼然為四輔者,李乾、何顯周。乾出為知府,佑、顯周俱罷去,是官遂廢不復設。

本,不詳其籍里。佑,安邑人。嘗三主本布政司鄉試,稱得人。龔斅,鉛出人。以行誼重於鄉。致仕後,復起為國子司業,歷祭酒。坐放諸生假不奏聞,免。杜斅,字致道,壺關人。舉元鄉試第一,歷官臺州學正。歸家教授。通《易》、《詩》、《書》三經。源,莆田人。亦再徵為國子司業,卒於官。民望,槁城人。幹,絳州人。顯周,內黃人。

吳伯宗,名祐,以字行,金谿人。洪武四年,廷試第一。時開科之始,帝親製策問。得伯宗甚喜,賜冠帶袍笏,授禮部員外郎,與修《大明日曆》。胡惟庸用事,欲人附己,伯宗不為屈。惟庸銜之,坐事謫居鳳陽。上書論時政,因言惟庸專恣不法,不宜獨任,久之必為國患。辭甚愷切。帝得奏,召還,賜衣鈔。奉使安南,稱旨。除國子助教,命進講東宮。首陳正心誠意之說。改翰林典籍。帝製十題命賦,援筆立就,詞旨雅潔。賜織金錦衣。除太常司丞,辭。改國子司業,又辭。忤旨,貶金縣教諭。未至,詔還為翰林檢討。十五年進武英殿大學士。明年冬,坐弟仲實為三河知縣薦舉不實,詞連伯宗,降檢討。伯宗為人溫厚,然內剛,不茍媕阿,故屢躓。踰年,卒於官。伯宗成進士,考試官則宋濂、鮑恂也。

恂,字仲孚,崇德人。受《易》於臨川吳澄。好古力行,著《大易傳義》,學者稱之。元至正中,以薦授溫州路學正。尋召入翰林,不就。洪武四年,初科舉取士,召為同考官。試已,辭去。十五年與吉安餘詮、高郵張長年、登州張紳,皆以明經老成為禮部主事劉庸所薦,召至京。恂年八十餘,長年、詮亦皆踰七十矣。賜坐顧問。翌日並命為文華殿大學士,皆以老疾固辭,遂放還。紳後至,以為鄠縣教諭,尋召為右僉都御史,終浙江左布政使。其明年以耆儒征者,曰全思誠,字希賢,上海人,亦授文華殿大學士。又明年請老,賜敕致仕。

伯宗之使安南也,以名德為交人所重。其後,襄陽任亨泰亦舉洪武二十一年進士第一,以禮部尚書使安南,交人以為榮。前後使安南者,並稱吳、任云。

亨泰為禮部尚書時,日照民江伯兒以母病殺其三歲子祀岱嶽。有司以聞。帝怒其滅絕倫理,杖百,戍海南。因命亨泰定旌表孝行事例。亨泰議曰:「人子事親,居則致其敬,養則致其樂,有疾則謹其醫藥。臥冰割股,事非恒經。割股不已,致於割肝,割肝不已,至於殺子。違道傷生,莫此為甚。墮宗絕祀,尤不孝之大者,宜嚴行戒諭。倘愚昧無知,亦聽其所為,不在旌表之例。」詔曰「可」。明年,議秦王喪禮,因定凡世子襲爵之禮。會討龍州趙宗壽,命偕御史嚴震直使安南,諭以謹邊方,無納逋逃。時帝以安南篡弒,絕其貢使。至是聞詔使至,震恐。亨泰為書,述朝廷用兵之故以安慰之,交人大悅。使還,以私市蠻人為僕,降御史。未幾,思明土官與安南爭界,詞復連亨泰,坐免官。

吳沉,字浚仲,蘭溪人。元國子博士師道子也,以學行聞。太祖下婺州,召沉及同郡許元、葉瓚玉、胡翰、汪仲山、李公常、金信、徐孳、童冀、戴良、吳履、孫履、張起敬會食省中,日令二人進講經史。已,命沉為郡學訓導。

洪武初,郡以儒士舉,誤上其名曰信仲,授翰林院待制。沉謂修撰王釐曰:「名誤不更,是欺罔也。」將白於朝。厘言:「恐觸上怒」。沉不從,牒請改正。帝喜曰:「誠愨人也。」遂眷遇之,召侍左右。以事降編修。給事中鄭相同言:「故事啟事東宮,惟東宮官屬稱臣,朝臣則否。今一體稱臣,於禮未安。」沉駁之曰:「東宮,國之大本。尊東宮,所以尊主上也。相同言非是。」帝從之。尋以奏對失旨,降翰林院典籍。已,擢東閣大學士。

初,帝謂沉曰:「聖賢立教有三:曰敬天,曰忠君,曰孝親。散在經卷,未易會其要領。爾等以三事編輯。」至是書成,賜名《精誠錄》,命沉撰序。居一年,降翰林侍書,改國子博士,以老歸。沉嘗著辯,言「孔子封王為非禮」。後布政使夏寅、祭酒丘浚皆沿其說。至嘉靖九年,更定祀典,改稱「至聖先師」,實自沉發之也。

桂彥良,名德偁,以字行,慈谿人。元鄉貢進士,為平江路學教授,罷歸。張士誠、方國珍交辟,不就。洪武六年,徵詣公車,授太子正字。帝嘗出御製詩文,彥良就御座前朗誦,聲徹殿外,左右驚愕,帝嘉其朴直。時選國子生蔣學等為給事中,舉人張唯等為編修,肄業文華堂。命彥良及宋濂、孔克表為之師。嘗從容有所咨問,彥良對必以正。帝每稱善,書其語揭便殿。七年冬至,詞臣撰南郊祝文用「予」、「我」字。帝以為不敬。彥良曰:「成湯祭上帝曰『予小子履』;武王祀文王之詩曰「『我將我享』。古有此言。」帝色霽曰:「正字言是也。」時御史臺具獄,令詞臣覆讞。彥良所論釋者數十人。

遷晉王府右傅。帝親為文賜之。彥良入謝。帝曰:「江南大儒,惟卿一人。」對曰:「臣不如宋濂、劉基。」帝曰「濂,文人耳;基,峻隘,不如卿也。」彥良至晉,製《格心圖》獻王。後更王府官制,改左長史。朝京師,上太平十二策。帝曰:「彥良所陳,通達事體,有裨治道。世謂儒者泥古不通今,若彥良可謂通儒矣。」十八年請告歸,越二年卒。

明初,特重師傅。既命宋濂教太子,而諸王傅亦慎其選。彥良與陳南賓等皆宿儒老生,而李希顏與駙馬都尉胡觀傅徐宗實,尤以嚴見憚。

李希顏,字愚庵,郟人。隱居不仕。太祖手書徵之,至京,為諸王師。規範嚴峻,諸王有不率教者,或擊其額。帝撫而怒。高皇后曰:「烏有以聖人之道訓吾子,顧怒之耶?」太祖意解,授左春坊右贊善。諸王就籓,希顏歸舊隱。閭里宴集,常著緋袍戴笠往。客問故,笑曰:「笠本質,緋,君賜也。」

徐宗實,名垕,以字行,黃巖人。少穎悟。篤於學。洪武中,被薦,除銅陵簿。請告迎養,忤帝意,謫戍淮陰驛。會東川侯胡海子觀選尚主,帝為觀擇師,難其人,以命宗實。中使援他府例,置駙馬位中堂南向,而布師席於西階上東向。宗實手引駙馬位使下,然後為說書。左右大驚,相顧以目。帝聞而嘉之,召宗實慰勞數四。

洪武末,授蘇州通判。奏發官粟二十萬石以活饑民。春水暴,齧隄,倡議修築,吳人皆以為便。請旌元節婦王氏,禮部以前朝事,不當允。宗實言:「武王封比干墓,獨非前朝事乎!」遂得旌。建文二年,超擢兵部右侍郎。坐事貶官,尋復職。燕事急,使兩浙招義勇。成祖即位,疏乞歸。逾二年,以事被逮,道卒。

陳南賓,名光裕,以字行,茶陵人。元末為全州學正。洪武三年,聘至都,除無棣丞,歷膠州同知,所至以經術為治。召為國子助教。嘗入見,講《洪範》九疇。帝大喜,書姓名殿柱。後御注《洪範》,多採其說。擢蜀府長史。蜀獻王好學,敬禮尤至,造安車以賜,為構第,名「安老堂。」二十九年,與方孝孺同為四川考試官。詩文清勁有法。卒年八十。其後諸王府長史劉淳、董子莊、趙季通、楊黼、金實、蕭用道、宋子環之屬,皆有名。

劉淳,南陽人。洪武末為原武訓導。周王聘為世子師。尋言於朝,補右長史,以正輔王。端禮門槐盛夏而枯。淳陳咎徵進戒。王用其言修省,枯枝復榮。王旌其槐曰「攄忠」。致仕十餘年而卒,年九十有七。

董子莊,名琰,以字行,江西樂安人。有學行。洪武中,以學官遷知茂名縣。永樂時,由國子司業出為趙王府右長史,隨事匡正。王多過,帝輒以責長史。子莊以能諫,得無過。十八年春當陪祀國社,夙起,衣冠端坐而卒。

趙季通,字師道,天台人。亦由教官歷知永豐、龍溪,與修《太祖實錄》,累進司業。出為趙王府左長史,與子莊同心輔導,籓府賢僚首稱趙、董云。

楊黼,吉水人。官御史。仁宗即位,上疏言十事。擢衛王府右長史。盡心獻替,未嘗茍取一錢。宣德初,卒。

金實,開化人。永樂初,上書言治道。帝嘉之。復對策,稱旨,除翰林典籍。與修《太祖實錄》、《永樂大典》,選為東宮講官。歷左春坊左司直。仁宗立,除衛府左長史。正統初,卒。為人孝友,敦行誼。閱經史,日有程限,至老不輟。

蕭用道,泰和人。建文中,舉懷才抱德,詣闕試文章。擢靖江王府長史,召入翰林,修《類要》。燕師渡淮,與周是修同上書,指斥用事者。永樂時,預修《太祖實錄》,改右長史,從王之籓桂林。嘗為王陳八事,曰:慎起居、寡嗜慾、勤學問、養德性、簡鞭撲之刑、無侵下人利、常接府僚以通群情、簡擇謹厚人以備差遣。又作《端禮》、《體仁》、《遵義》、《廣智》四門箴獻王。久之,以疾乞歸。成祖怒,貶宣府鷂兒嶺巡檢,卒。子晅,由進士官湖廣左布政使。天順四年,舉治行卓異,拜禮部尚書。初,兩京尚書缺,多用布政使為之。自晅後,遂無拜尚書者。晅重厚廉靜,而不善奏對。調南京,卒。

宋子環,廬陵人。由庶吉士歷考功郎中。從師逵採木湖廣,以寬厚得眾心。仁宗即位,授梁府右長史,改越府。和易澹泊,所至有賢聲。宣德中,卒官。自是以後,王府官不為清流,遂無足紀者矣。

宋訥,字仲敏,滑人。父壽卿,元侍御史。訥性持重,學問該博。至正中,舉進士,任鹽山尹,棄官歸。洪武二年,徵儒士十八人編《禮》、《樂》諸書,訥與焉。事竣,不仕歸。久之,用四輔官杜斅薦,授國子助教。以說經為學者所宗。十五年超遷翰林學士,命撰《宣聖廟碑》,稱旨,賞賚甚厚。改文淵閣大學士。嘗寒附火,燎脅下衣,至膚始覺。帝制文警之。未幾,遷祭酒。時功臣子弟皆就學,及歲貢士嘗數千人。訥為嚴立學規,終日端坐講解無虛晷,夜恒止學舍。十八年復開進士科,取士四百七十有奇,由太學者三之二。再策士,亦如之。帝大悅。製詞褒美。助教金文徵等疾訥,構之吏部尚書餘熂,牒令致仕。訥陛辭,帝驚問,大怒,誅熂、文徵等,留訥如故。訥嘗病,帝曰:「訥有壽骨,無憂也。」尋愈。帝使畫工瞷訥,圖其像,危坐,有怒色。明日入對,帝問:「昨何怒?」訥驚對曰:「諸生有趨踣者,碎茶器。臣愧失教,故自訟耳。且陛下何自知之?」帝出圖。訥頓首謝。

長子麟,舉進士,擢御史,出為望江主簿。帝念訥老,召還侍。二十三年春,訥病甚,乃止學舍。麟請歸私第,叱曰:「時當丁祭,敢不敬耶!」祭畢,舁歸舍而卒,年八十。帝悼惜,自為文祭之。又遣官祭於家,為治葬地。文臣四品給祭葬者,自訥始。正德中。謚文恪。

訥嘗應詔陳邊事,言:「海內乂安,惟沙漠尚煩聖慮。若窮追遠擊,未免勞費。陛下為聖子神孫計,不過謹邊備而已。備邊在乎實兵,實兵在乎屯田。漢趙充國將四萬騎,分屯緣邊九郡,而單于引卻。陛下宜於諸將中選謀勇數人,以東西五百里為制,立法分屯,布列要害,遠近相應。遇敵則戰,寇去則耕。此長策也。」帝頗採用其言。訥既卒,帝思之。官其次子復祖為司業,誡諸生守訥學規,違者罪至死。

明開國時即重師儒官。許存仁、魏觀為祭酒,老成端謹;訥稍晚進,最蒙遇。與訥定學規者,司業王嘉會、龔斅。三人年俱高,鬚髮皓白,終日危坐,堂上肅然。而張美和、聶鉉、貝瓊等皆名儒,當洪武時,先後為博士、助教、學錄,以故諸生多所成就。魏觀事別載。

嘉會,字原禮,嘉興人。以薦征,累官國子監司業。十六年,亦以老請歸,優詔留之。年八十卒,賻恤甚厚。

許存仁。名元,以字行,金華許謙子也。太祖素聞謙名,克金華,訪得存仁。與語大悅,命傅諸子。擢國子博士。嘗命講《尚書·洪範》休咎徵之說。又嘗問《孟子》何說為要。存仁以行王道、省刑、薄賦對。吳元年擢祭酒。存仁出入左右垂十年,自稽古禮文事,至進退人才,無不與論議。既將議即大位,而存仁告歸。司業劉丞直曰:「主上方應天順人,公宜稍待。」存仁不聽,果忤旨。僉事程孔昭劾其隱事,遂逮死獄中。

張美和,名九韶,以字行,清江人。能詞賦。元末,累舉不仕。洪武三年,以薦為縣學教諭。後遷國子助教,改翰林院編修。致仕歸,帝親為文賜之。復與錢宰等並徵,修《書》傳,既成,遣還。

聶鉉,字器之,美和同邑人。洪武四年進士。為廣宗丞,疏免旱災稅。秩滿入覲,獻《南都賦》及《洪武聖德詩》。授翰林院待制,改國子助教,遷典籍。與美和同賜歸。十八年復召典會試,欲留用之。乞便地自養。令食廬陵教諭俸,終其身。

貝瓊,字廷琚,崇德人。性坦率,篤志好學,年四十八,始領鄉薦。張士誠屢辟不就。洪武初,聘修《元史》。既成,受賜歸。六年以儒士舉,除國子助教。瓊嘗慨古樂不作,為《大韶賦》以見志。宋濂之為司業也,建議立四學,並祀舜、禹、湯、文為先聖。太祖既絀其說,瓊復為《釋奠解》駁之,識者多是瓊議。與美和、鉉齊名,時稱「成均三助」。九年改官中都國子監,教勛臣子弟。瓊學行素優,將校武臣皆知禮重。十一年致仕,卒。

趙俶,字本初,山陰人。元進士。洪武六年,徵授國子博士。帝嘗御奉天殿,召俶及錢宰、貝瓊等曰:「汝等一以孔子所定經書為教,慎勿雜蘇秦、張儀縱橫之言。」諸臣頓首受命。俶因請頒正定《十三經》於天下,屏《戰國策》及陰陽讖卜諸書,勿列學宮。明年擇諸生穎異者三十五人,命俶專領之,教以古文。尋擢李擴、黃義等入文華、武英二堂說書,皆見用。九年,御史臺言:「博士俶以《詩經》教成均四年,其弟子多為方岳重臣及持節各部者。今年逾懸車,請賜骸骨。」於是以翰林院待制致仕,賜內帑錢治裝。宋濂率同官暨諸生千餘人送之。卒年八十一。子圭玉,兵部侍郎,出知萊州,有聲。

錢宰,字子予,會稽人。吳越武肅王十四世孫。至正間中甲科,親老不仕。洪武二年,徵為國子助教。作《金陵形勝論》、《歷代帝王樂章》,皆稱旨。十年乞休。進博士,賜敕遣歸。至二十七年,帝觀蔡氏《書傳》,象緯運行,與朱子《詩傳》相悖,其他注與鄱陽鄒季友所論有未安者。徵天下宿儒訂正之。兵部尚書唐鐸舉宰及致仕編修張美和、助教靳權等。行人馳傳征至,命劉三吾總其事。江東諸門酒樓成,賜百官鈔,宴其上。宰等賦詩謝。帝大悅。諭諸儒年老願歸者,先遣之。宰年最高,請留。帝喜。書成,賜名《書傳會選》,頒行天下。厚賜,令馳驛歸。卒年九十六。

又蕭執者,字子所,泰和人。洪武四年鄉舉,為國子學錄。明年夏至,帝有事北郊,召尚書吳琳、主事宋濂率文學士以從。執偕陶凱等十二人入見齋所。令賦詩,復令賦山梔花。獨喜執作,遍示諸臣,寵眷傾一時。時帝留意文學,往往親試廷臣,執與陳觀知遇尤異。

觀以訓導入覲,試《王猛捫CB論》,立擢陜西參政。尋召還侍左右。應制作《鐘山賦》,賜金幣。在陜以廉謹稱。或問陜產金何狀。觀大驚曰:「吾備位籓寮,何金之問!」其卒也,妻子幾無以自存。而執以親老乞歸,親沒廬墓側。申國公鄧鎮剿龍泉寇,不戢下。執往責之,鎮為禁止,邑人以安。兩人皆篤行君子也。

李叔正,字克正,初名宗頤,靖安人。年十二能詩,長益淹博。時江西有十才子,叔正其一也。以薦授國子學正。洪武初,告歸。未幾,復以薦為學正,遷渭南丞。同州蒲城人爭地界,累年不決。行省以委叔正,單騎至,剖數語立決。渭南歲輸糧二萬,豪右與猾吏為奸,田無定額,叔正履畝丈量。立法精密,諸弊盡剔。遷興化知縣。尋召為禮部員外郎。以年老乞歸,不許,改國子助教。於是叔正三至太學矣。帝方銳意文治,於國學人材尤加意。然諸生多貴胄,不率教。叔正嚴立規條,旦夕端坐,督課無倦色。朝論賢之。擢監察御史,奉命巡嶺表。瓊州府吏訐其守踞公座簽表文,叔正鞫之,守得白,抵吏罪。太祖嘉之曰:「人言老御史懦,乃明斷如是耶。」累官禮部侍郎。十四年進尚書,卒於官。叔正妻夏氏,陳友諒陷南昌時,投井死。叔正感其義,終身不復娶。

劉崧,字子高,泰和人,舊名楚。家貧力學,寒無壚火,手皸裂而鈔錄不輟。元末舉於鄉。洪武三年舉經明行修,改今名。召見奉天殿,授兵部職方司郎中。奉命徵糧鎮江。鎮江多勛臣田,租賦為民累,崧力請得少減。遷北平按察司副使,輕刑省事。招集流亡,民咸復業。立文天祥祠於學宮之側。勒石學門,示府縣勿以徭役累諸生。嘗請減僻地驛馬,以益宛平。帝可其奏,顧謂侍臣曰:「驛傳勞逸不均久矣,崧能言之。牧民不當如是耶?」為胡惟庸所惡,坐事謫輸作。尋放歸。十三年,惟庸誅,徵拜禮部侍郎。未幾,擢吏部尚書。雷震謹身殿,帝廷諭群臣陳得失。崧頓首,以修德行仁對。尋致仕。明年三月,與前刑部尚書李敬並徵。拜敬國子祭酒,而崧為司業。賜鞍馬,令朝夕見,見輒燕語移時。未旬日卒。疾作,猶強坐訓諸生。及革,敬問所欲言。曰:「天子遣崧教國子,將責以成功,而遽死乎!」無一語及家事。帝命有司治殯殮,親為文祭之。

崧幼博學,天性廉慎。兄弟三人共居一茆屋,有田五十畝。及貴,無所增益。十年一布被,鼠傷,始易之,仍葺以衣其子。居官未嘗以家累自隨。之任北平,攜一童往,至則遣還。晡時史退,孤燈讀書,往往達旦。善為詩,豫章人宗之為《西江派》云。

羅復仁,吉水人。少嗜學,陳友諒辟為編修。已,知其無成,遁去。謁太祖於九江,留置左右。從戰鄱陽,齎蠟書諭降江西未下諸郡,授中書諮議。從圍武昌,太祖欲招陳理降,以復仁故友諒臣也,遣入城諭,且曰:「理若來,不失富貴。」復仁頓首曰:「如陳氏遣孤得保首領,俾臣不食言於異日,臣死不憾。」太祖曰:「汝行,吾不汝誤也。」復仁至城下,號慟者竟日,理縋之入。見理大哭,陳太祖意,且曰:「大兵所向皆摧,不降且屠,城中民何罪?」理聽其言,遂率官屬出降。

遷國子助教,以老特賜乘小車出入。每宴見,賜坐飲食。已,復使擴廓。前使多拘留,復仁議論慷慨,獨得還。洪武元年,擢編修,復偕主事張福往諭安南還占城侵地。安南奉詔,遺復仁金、貝、土產甚厚,悉卻不受。帝聞而賢之。三年置弘文館,以復仁為學士,與劉基同位。在帝前率意陳得失。嘗操南音。帝顧喜其質直,呼為「老實羅」而不名。間幸其舍,負郭窮巷,復仁方堊壁,急呼其妻抱杌以坐帝。帝曰:「賢士豈宜居此。」遂賜第城中。天壽節製《水龍吟》一闋以獻。帝悅,厚賜之。尋乞致仕。陛辭,賜大布衣,題詩衣襟上褒美之。已,又召至京師。奏減江西秋糧。報可。留三月,賜玉帶、鐵拄杖、坐墩、裘馬、食具遣還,以壽終。

孫汝敬,名簡,以字行。永樂二年庶吉士,就學文淵閣,誦書不稱旨,即日遣戍江南。數日復之。自此刻厲為學,累遷侍講。仁宗時,上言時政十五事,忤旨下獄。既與李時勉同改御史,直聲震一時。宣宗初,上書大學士楊士奇曰:「太祖高皇帝奄有四海,太宗文皇帝再造寰區。然猶翼翼兢兢,無敢豫怠。先皇帝嗣統未及期月,奄棄群臣。揆厥所由,皆憸壬小夫,獻金石之方以致疾也。去冬,簡以愚戇應詔上書,言涉不敬,罪當萬死。先皇帝憐其孤直,寬雷霆之誅,俾居方路,撫躬循省,無可稱塞。伏見今年六月,車駕幸天壽山,躬謁二陵,京師之人瞻望咨嗟,以為聖天子大孝。既而道路喧傳,禮畢即較獵講武,扈從惟也先士干與其徒數百人。風馳電掣,馳逐先後。某聞此言,心悸膽落。夫蒐苗獼狩,固有國之常經。然以謁陵出,而與降將較獵於山谷間,垂堂之戒,銜橛之虞,不可不深慮也。執事四朝舊臣,二聖元輔,於此不言,則孰得而言之者?惟特加採納,以弘靖獻之思,光弼直之義。」

尋擢工部右侍郎,兩使安南。時黎利言其主陳皓已死,而張筵設女樂。汝敬叱之,利懼謝。還督兩浙漕運,理陜西屯田,多所建置。坐受饋,充為事官。英宗立,遇赦,汝敬誤引詔復職,復逮繫。以在陜措置勞,宥死戍邊。尋復職,蒞故任。塞上有警,汝敬往督餉。遇敵紅城子,中流矢,墜馬得免。以疾告歸,卒。

贊曰:明始建國,首以人材為務。征辟四方,宿儒群集闕下,隨其所長而用之。自議禮定制外,或參列法從,或預直承明,而成均胄子之任尤多稱職,彬彬乎稱得人焉。夫諸臣當元之季世,窮經績學,株守草野,幾於沒齒無聞。及乎泰運初平,連茹利見,乃各展所蘊,以潤色鴻猷,黼黼文治。昔人謂天下不患無才,惟視上之網羅何如耳,顧不信哉!


\end{pinyinscope}