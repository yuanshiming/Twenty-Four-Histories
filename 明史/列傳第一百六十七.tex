\article{列傳第一百六十七}

\begin{pinyinscope}
呂大器文安之樊一蘅範文光詹天顏吳炳侯偉時王錫兗堵胤錫嚴起恒朱天麟張孝起楊畏知吳貞毓高勣等

呂大器,字儼若,遂寧人。崇禎元年進士。援行人,擢吏部稽勳主事,更歷四司,乞假歸。以邑城庳惡,倡議修築。工甫竣,賊至,佐有司拒守,城獲全。詔增秩一等。出為關南道參議,遷固原副使。巡撫丁啟睿檄大器討長武賊,用穴地火攻法滅之。

十四年,擢右僉都御史,巡撫甘肅。劾總兵官柴時華不法,解其職,立遣副將王世寵代之。時華乞兵西部及土魯番為變,大器令世寵討敗時華及西部,時華自焚死。塞外爾迭尼、黃台吉等擁眾乞賞,謀犯肅州,守臣拒走之。大器假賞犒名,毒飲馬泉,殺其眾無算。又遣總兵官馬爌督副將世寵等討群番為亂者,斬首七百餘級,撫三十八族而還。又擊敗其餘黨。西陲略定。

十五年六月,擢兵部添註右侍郎。大器負才,性剛躁,善避事。見天下多故,懼當軍旅任,力辭,且投揭吏科,言已好酒色財,必不可用。帝趣令入京,詭稱疾不至。嚴旨切責,亦不至,命所司察奏。明年三月始至,命以本官兼右僉都御史,總督保定、山東、河北軍務。時畿輔未解嚴,大器及諸將和應薦、張汝行馳扼順義牛欄山。總督趙光抃集諸鎮師大戰螺山,應薦陣亡,他將亦多敗。大器所部無失事,增俸一等。

五月,以保定息警,罷總督官,特設江西、湖廣、應天、安慶總督,駐九江,大器任之。湖北地已失,武昌亦陷,左良玉駐九江,稱疾不進。以侯恂故疑大器圖己,語具良玉傳中。大器詣榻前與慰勞,疑稍釋。而張獻忠大躪湖南,分兵陷袁州、吉安。大器急遣部將及良玉軍連破之樟樹鎮,峽江、永新二郡皆復。已而建昌、撫州陷,良玉、大器不和,兵私鬥,焚南昌關廂。廷議因改大器南京兵部右侍郎,以袁繼咸代。

十七年四月,京師報陷,南京大臣議立君。大器主錢謙益、雷縯祚言,立潞王。議未定而馬士英及劉澤清諸將擁福王至。福王立,遷大器吏部左侍郎。大器以異議絀,自危,乃上疏劾士英。言其擁兵入朝,靦留政地,翻先皇手定逆案,欲躋阮大鋮中樞。其子以銅臭為都督,女弟夫未履行陣,授總戎,姻婭越其傑、田仰、楊文驄先朝罪人,盡登膴仕,亂名器。「夫吳甡、鄭三俊,臣不謂無一事失,而端方諒直,終為海內正人之歸;士英、大鋮,臣不謂無一技長,而奸回邪慝,終為宗社無窮之禍」。疏入,以和衷體國答之。

未幾,澤清入朝,劾大器、縯祚懷異圖。大器遂乞休去,以手書監國告廟文送內閣,明無他。士英憾未已,令太常少卿李沾劾之。遂削大器籍,復命法司逮治之。以蜀地盡失,無可蹤跡而止。大器既去,沾得超擢左都御史。謙益亦以附士英、大鋮,得為禮部尚書。獨縯祚論死。

明年,唐王召為兵部尚書兼東閣大學士。道梗,久之至。汀州失,奔廣東,與丁魁楚等擁永明王監國,令以原官兼掌兵部事。久之,進少傅,盡督西南諸軍,代王應熊,賜劍,便宜從事。至涪州,與將軍李占春深相結。他將楊展、于大海、胡雲風、袁韜、武大定、譚弘、譚詣、譚文以下,皆受大器約束。宗室朱容籓自稱天下兵馬副元帥,據夔州。大器檄占春、大海、雲風討殺容籓。大器至思南得疾,次都勻而卒,王謚為文肅。

文安之,夷陵人。天啟二年進士。改庶吉士,授檢討,除南京司業。崇禎中,就遷祭酒,為薛國觀所構,削藉歸。久之,言官交薦,未及召而京師陷。

福王時,起為詹事。唐王復召拜禮部尚書。安之方轉側兵戈間,皆不赴。永明王以瞿式耜薦,與王錫兗並拜東閣大學士,亦不赴。順治七年六月,安之謁王梧州。安之敦雅操,素淡宦情,遭國變,絕意用世。至是見國勢愈危,慨然思起扶之,乃就職。時嚴起恒為首輔,王化澄、朱天麟次之,起恒讓安之而自處其下。

孫可望再遣使乞封秦王,安之持不予。其後桂林破,王奔南寧。大兵日迫,雲南又為可望據,不可往。安之念川中諸鎮兵尚強,欲結之,共獎王室,乃自請督師,加諸鎮封爵。王從之,加安之太子太保兼吏、兵二部尚書,總督川、湖諸處軍務,賜劍,便宜從事。進諸將王光興,郝永忠、劉體仁、袁宗第、李來亨、王友進、塔天寶、馬雲翔、郝珍、李復榮、譚弘、譚詣、譚文、黨守素等公侯爵,即令安之齎敕印行。可望聞而惡之,又素銜前阻封議,遣兵伺於都勻,邀止安之,追奪光興等敕印。留數月,乃令人湖廣。安之遠客他鄉,無所歸,復赴貴州,將謁王於安龍。可望坐以罪,戍之畢節衛。

先是,可望欲設六部、翰林等官,慮人議其僭,乃以范礦、馬兆義、任僎、萬年策為吏、戶、禮、兵尚書,並加行營之號。後又以程源代年策。而僎最寵,與方于宣屢勸進,可望令待王入黔議之。王久駐安龍,可望遂自設內閣六部等官,以安之為東閣大學士。安之不為用,久之走川東,依劉體仁以居。

李赤心,高必正等久竄廣西賓、橫、南寧間。赤心死,養子來亨代領其眾,推必正為主。必正又死,其眾食盡,且畏大兵逼,率眾走川東,分據川、湖間,耕田自給。川中舊將王光興、譚弘等附之,眾猶數十萬。

順治十六年正月,王奔永昌。安之率體仁、宗第、來亨等十六營由水道襲重慶。會譚弘、譚詣殺譚文,諸將不服。安之欲討弘、詣,弘、詣懼,率所部降於大兵,諸鎮遂散。時王已入緬甸,地盡失,安之不久鬱鬱而卒。

樊一蘅,字君帶,宜賓人。父垣,常德知府。一蘅舉萬曆四十七年進士,知安義、襄陽,累官吏部郎中,請告歸。崇禎三年秋,遷榆林兵備參議。流賊多榆林人,又久荒,饑民益相挻為盜。一蘅撫創殘,修戎備,討斬申在庭、馬丙貴,平不沾泥。累被薦,遷監軍副使,再遷右參政,分巡關南。總兵曹文詔敗歿,群賊迫西安。總督洪承疇令一蘅監左光先、張應昌軍,連破賊,擊走混天星。賊逼漢中,瑞王告急,一蘅偕副將羅尚文往救。會承疇大軍至,賊乃走。進按察使,偕副將馬科、賀人龍屢挫祁總管於漢中,降之。十二年,擢右僉都御史,代鄭崇儉巡撫寧夏,被劾罷歸。十六年冬,用薦起兵部右侍郎,總督川、陜軍務,道阻,命不達。

順治元年,福王立於南京,復申前命。時張獻忠已據全蜀,惟遵義未陷,一蘅與王應熊避其地。既拜命,檄諸郡舊將會師大舉。會巡撫馬乾復重慶,松潘副將朱化龍、同知詹天顏擊斬賊將王運行,復龍安、茂州。一蘅乃起舊將甘良臣為總統,副以侯天錫、屠龍,合參將楊展,游擊馬應試、餘朝宗所攜潰卒,得三萬人。明年三月攻敘州,應試、朝宗先登,展等繼至,斬馘數千級。偽都督張化龍走,遂復其城。一蘅乃犒師江上。

初,乾復重慶,賊將劉廷舉走,求救於獻忠。獻忠命養子劉文秀攻重慶,水陸並進。副將曾英與參政劉麟長自遵義至,與部將于大海、李占春、張天相等夾擊,破賊兵數萬。英威名大振,諸別將皆屬,兵二十餘萬,奉一蘅節制。

楊展既復敘州,賊將馮雙禮來寇,每戰輒敗,孫可望以大眾援之。隔江持一月,糧盡,一蘅退屯古藺州,展退屯江津。賊退截朱化龍及僉事蔡肱明於羊子嶺,化龍率番騎數百衝賊兵,賊驚潰,死者滿山谷。化龍以軍孤,還守舊地。他將復連敗賊於摩泥、滴水。

一蘅乃命展、應試取嘉定、邛、眉,故總兵官賈連登及其中軍楊維棟取資、簡,天錫、高明佐取瀘州,占春、大海守忠、涪。其他據城邑奉徵調者,洪、雅則曹勛及監軍副使范文光,松、茂則監軍僉事詹天顏,夔、萬則譚弘、譚詣。一蘅乃移駐納溪,居中調度,與督師應熊會瀘州,檄諸路刻欺並進。獻忠頗懼,盡屠境內民,沈金銀江中,大焚宮室,火連月不滅,將棄成都走川北。

明年春,展盡取上川南地,屯嘉定,與勛等相聲援。而應熊及王祥在遵義,乾、英在重慶,皆宿重兵。賊勢日蹙,惟保寧、順慶為賊將劉進忠所守,進忠又數敗。獻忠怒,遣孫可望、劉文秀、王尚禮、狄三品、王復臣等攻川南郡縣。應熊、一蘅急令展、天錫、龍、應試及顧存志、莫宗文、張登貴連營犍為、敘州以禦之。賊連戰不利,英、祥乘間趨成都,獻忠立召可望等還。又聞大清兵入蜀境,劉進忠降,大懼。七月,棄成都走順慶,尋入西充之風凰山。至十二月,大清兵奄至,射殺獻忠,賊降及敗死者二三十萬。可望等率殘卒南奔,驟至重慶。英出不意,戰敗,死於江。賊遂陷綦江,應熊避之畢節衛。踰月,賊陷遵義,入貴州。大清兵追至重慶,巡撫乾敗死,遂入遵義。以餉乏,旋師。王祥等復取保、寧二郡。一蘅再駐江上,為收復全蜀計,乃列上善後事宜及諸將功狀於永明王。拜一蘅戶、兵二部尚書,加太子太傅,祥、展、天錫等進爵有差。時應熊已卒,而宗室朱容籓、故偏沅巡撫李乾德並以總制至,楊喬然、江爾文以巡撫至,各自署置,官多於民。諸將袁韜據重慶,于大海據雲陽,李占春據涪州,譚詣據巫山,譚文據萬縣,譚弘據天字城,侯天錫據永寧,馬應試據蘆衛,王祥據遵義,楊展據嘉定,朱化龍、曹勛仍據故地。搖、黃諸家據夔州夾江兩岸,而李自成餘孽李赤心等十三家亦在建始縣。一蘅令不行,保敘州一郡而已。

順治五年,容籓自稱楚世子,建行臺夔州,稱制封拜。時喬然已進總督,而范文光、詹天顏巡撫川南、北,呂大器以大學士來督師,皆惡容籓,謀誅之。六年春,容籓遂為占春所敗,走死雲陽。初,展與祥有隙,遣子璟新攻之。璟新先襲殺應試,與祥戰敗歸。乾德利展富,說韜、大定殺展,分其貲。一蘅誚乾德,諸鎮亦皆憤,有離心。

秋九月,孫可望遣白文選攻殺祥,降其眾二十餘萬,盡得遵義、重慶。一蘅益孤。七年秋,可望又使劉文秀大敗武大定兵,長驅至嘉定。大定、韜皆降,乾德投水死。文秀兵復東,譚弘、譚詣、譚文盡降。占春、大海降於大清。明年正月,文秀還雲南,留文選守嘉定,劉鎮國守雅州。三月,大清兵南征,文選、鎮國挾曹勛走,文光、天顏、化龍相繼死。一蘅時已謝事,避山中。至九月,亦遘疾死。文武將吏盡亡。

范文光,內江人。天啟初,舉於鄉。崇禎中,歷官工部主事,南京戶部員外郎,告歸。十七年,張獻忠亂蜀,文光偕邛州舉人劉道貞,蘆山舉人程翔風,雅州諸生傅元修、洪其仁等舉義兵,奉鎮國將軍朱平檙為蜀王,推黎州參將曹勛為副總兵,統諸將,而文光以副使為監軍,道貞等授官有差。勛敗賊雅州龍鸛山,追至城下,反為所敗,退守小關山。十一月,文光督參將黎神武攻雅州,不克。明年九月,神武合雅州土、漢兵再擊賊將艾能奇於雅州,敗績。偽監司郝孟旋守錦州,文光、翔鳳遣間使招之,孟旋襲殺守雅州賊,以城來歸,文光等入居之。獻忠死,文光保境如故。永明王命為右僉都御史,巡撫川南,而以安綿道詹天顏巡撫川北。總督李乾德殺楊展,文光惡之,遂入山不視事。大清兵克嘉定,文光賦詩一章,仰藥死。天顏兵敗被執,亦死之。天顏,龍巖人,起家選貢生。

吳炳,宜興人。萬曆末進士。授蒲圻知縣。崇禎中,歷官江西提學副使。江西地盡失,流寓廣東。永明王擢為兵部右侍郎,從至桂林,令以本官兼東閣大學士,仍掌部事。又從至武岡。大兵至,王倉猝奔靖州,令炳扈王太子走城步,吏部主事侯偉時從之。既至,城已為大兵所據,遂被執,送衡州。炳不食,自盡於湘山寺,偉時亦死之。

偉時,公安人。崇禎中進士,歷官吏部考功主事,罷官。至是補官數月,即遘難。

王錫袞,祿豐人。天啟二年進士。改庶吉士,授檢討。崇禎中,累官少詹事。十三年擢禮部右侍郎。明年秋,尚書林欲楫出視孝陵,錫袞以左侍郎掌部事。帝禁內臣干預外政,敕禮官稽先朝典制以聞。錫袞等備列諸監局職掌,而不及東廠。提督內臣王德化言:「東廠之設,始永樂十八年,《國朝典匯》可據。禮官覆議不及,請解臣職,停廠不設。」錫袞等言:「《典匯》雖載此條,但係下文箋註。臣等以正史無文,故不敢妾引。」帝不聽。錫袞復抗疏,請罷廠,亦不允。二月,帝再耕耤田。錫袞因言頻歲旱蝗,三餉疊派,請量除加征,嚴核蠹餉,俾農夫樂生。又以時方急才,請召還故侍郎陳子壯、顧錫疇,故祭酒倪元璐、文安之,且乞免黃道周永戍。給事中沈胤培請增天下解額,錫袞因言南畿、浙江人文更盛,宜倍增。又言舉人不第,有三十年不謁選者,宜定制。數科不售,即令服官。從之。

欲楫還朝,錫袞調吏部尚書。李日宣下獄,遂掌部事。帝性純孝,嘗以秋夜感念聖母孝純太后,遂欲終身蔬食。錫袞疏諫,帝嘉其寓愛於規,進秩一等。尋解部務,直講筵。十六年憂歸。

唐王立,拜禮部尚書兼東閣大學士。永明王立,申前命。皆不至。土酋沙定洲作亂,執至會城,詭草錫袞疏上永明王,言定洲忠勇,請代黔國公鎮雲南。疏既行,以稿示之。錫袞大恨,愬上帝祈死。居數日,竟卒。

堵胤錫,字仲緘,無錫人。崇禎十年進士。歷官長沙知府。山賊掠安化、寧鄉,官軍數敗,胤錫督鄉兵破滅之,又殺醴陵賊魁,遂以知兵名。十六年八月,賊陷長沙。胤錫朝覲還,賊已退。明年六月,福王命為湖廣參政,分守武昌、黃州、漢陽。左良玉稱兵,總督何騰蛟奔長沙,令攝湖北巡撫事,駐常德。唐王立,拜右副都御史,實授巡撫。

李自成死,眾擁其兄子錦為主,奉自成妻高氏及高氏弟一功,驟至澧州。擁眾三十萬,言乞降,遠近大震。胤錫議撫之,騰蛟亦馳檄至。乃躬入其營,開誠慰諭,稱詔賜高氏命服,錦、一功蟒玉金銀器,犒其軍,皆踴躍拜謝。乃即軍中宴之,導以忠孝大義數千言。明日,高氏出拜,謂錦曰:「堵公,天人也,汝不可負!」別部田見秀、劉汝魁等亦來歸。唐王大喜,加胤錫兵部右侍郎兼右僉都御史,總制其軍,手書獎勞。授錦御營前部左軍,一功右軍,並掛龍虎將軍印,封列侯。賜錦名赤心,一功名必正,他部賞賚有差,號其營曰忠貞。封高氏貞義夫人,賜珠冠彩幣,命有司建坊,題曰:「淑贊中興」。胤錫遂與赤心等深相結,倚以自強。然赤心書疏猶稱自成先帝,稱高氏太后云。

已而袁宗第、劉體仁諸營先歸騰蛟者,亦引與赤心合,眾益盛。胤錫以芻糧難繼,令散處江北就食。明年正月,騰蛟大舉,期諸軍盡會岳州。獨赤心先至,餘逗遛,卒不進。永明王立,進胤錫兵部尚書,總制如故。

順治四年,永明王令赤心等攻荊州。月餘,大清兵援荊州。赤心等大敗,步走入蜀,數日不得食。乃散入施州衛,聲言就食湖南。時王在武岡,劉承胤懼為赤心所並,計非胤錫不能禦,乃加胤錫東閣大學士,封光化伯,賜劍,便宜從事。胤錫疏請得給空敕鑄印,頒賜秦中舉兵者,時頗議其專。承胤欲殺騰蛟,胤錫劾其罪。

八月,大兵破武岡及寶慶、常德、辰、沅、胤錫走永順土司。尋赴貴陽,抵遵義,乞師於皮熊王祥。又入施州,請忠貞營軍。會楚宗人朱容籓偽稱監國天下兵馬副元帥,擅居夔州,御史錢邦芑傳檄討之。五年正月,胤錫見容籓,責以大義,曉譬利害,散其黨。

未幾,金聲桓、李成棟叛我大清,以江西、廣東附永明王。於是馬進忠、王進才、曹志建、李赤心、高必正等乘間取常德、桃源、澧州、臨武、藍山、道州、靖州、荊門、宜城諸州縣,進忠、赤心,必正皆封公。胤錫與進忠有隙,令赤心、必正爭進忠所取常德,進忠盡焚廬舍而去。赤心等棄空城引而東,所至守將皆燒營棄城走,湖南已復州縣為一空。胤錫乃率赤心等入湘潭,與騰蛟會。騰蛟令胤錫向江西,而自率進忠等向長沙。六年正月,兵方逼長沙,騰蛟在湘潭被執,諸軍遂散。赤心等走廣西,緣道掠衡、永、郴、桂。胤錫與胡一青守衡州,戰敗走桂陽。

初,赤心等入廣西,龍虎關守將曹志建惡其淫掠,並惡胤錫。胤錫不知也。或說志建,胤錫將召忠貞營圖志建。志建夜發兵圍胤錫,殺從卒千餘。胤錫及子逃入富川瑤峒。志建索之急,瑤潛送胤錫於監軍僉事何圖復,間關達梧州。會王遣大臣嚴起恒、劉湘客安輯忠貞營。至梧而赤心等已走賓、橫二州,乃載胤錫謁王於肇慶。志建遷怒圖復,誘殺之,闔門俱盡。

胤錫至肇慶,時馬吉翔及李元胤、袁彭年等皆專柄,各樹黨。胤錫乃結歡於吉翔,激赤心等東來,與元胤為難。移書瞿式耜,欲間元胤,託言王有密敕,令己與式耜圖元胤,王頗不悅。元胤黨丁時魁、金堡又論其喪師失地,乃令總統兵馬,移駐梧州。胤錫以赤心等不足恃,欲遙結孫可望為強援,矯王命封為平遼王。胤錫尋至潯州,自恨發病,十一月卒。王贈胤錫潯國公,謚文忠。

嚴起恒,浙江山陰人。崇禎四年進士。歷廣州知府,遷衡永兵備副使。十六年,張獻忠躪湖南,吏民悉逋竄。起恒獨堅守永州。賊亦不至。唐王時,擢戶部右侍郎,總督湖南錢法。永明王立,令兼督湖南軍餉。順治四年,王駐武岡,拜起恒禮部尚書兼東閣大學士,仍領錢法。王走靖州,起恒從不及,避難萬村。已知王在柳州,閑道往從之。從返桂林,復從至柳州、南寧。李成棟叛大清,以廣東附於王。起恒從王至肇慶,與王化澄、朱天麟同入直。無何,化澄、天麟相繼罷。黃士俊繼何吾騶為首輔,起恒次之。

時朝政決於成棟子元胤,都御史袁彭年,少詹事劉湘客,給事中丁時魁、金堡、蒙正發五人附之,攬權植黨,人目為五虎。起恒居其間,不能有所匡正。然起恒潔廉,遇事持平,與文安侯馬吉翔、司禮中官龐天壽共患難久,無所忤。而五虎憾起恒,競詆為邪黨。王在梧州,尚書吳貞毓等十四人合疏攻五虎,下湘客等獄,欲置之死。起恒顧跪王舟力救,貞毓等並惡之,乃請召還化澄,而合攻起恒。給事中雷德復劾其二十餘罪,比之嚴嵩。王不悅,奪德復官。起恒力求罷,王挽留之不得,放舟竟去。

會鄖國公高必正入覲王,貞毓欲藉其力以傾起恒,言:「朝事壞於五虎,主之者,起恒也。公入見,請除君側奸,數言決矣。」必正許之。有為起恒解者,謂必正曰:「五虎攻嚴公,嚴公反力救五虎。此長者,奈何以為奸?」必正見王,乃力言起恒虛公可任,請手敕邀與俱還。文安之入朝,起恒讓為首輔。桂林破,從王奔南寧。

先是,孫可望據雲南,遣使乞封王。天麟議許之,起恒持不可。後胡執恭矯詔封為秦王,可望知其偽,遣使求真封。起恆又持不可,可望大怒。至是,可望知王播遷,遣其將賀九儀、張勝等率勁卒五千,迎王至南寧,直上起恒舟,怒目攘臂,問王封是「秦」非「秦」。起恒曰:「君遠迎主上,功甚偉,朝廷自有隆恩。若專問此事,是挾封,非迎主上也。」九儀怒,格殺之,投屍於江。遂殺給事中劉堯珍、吳霖、張載述,追殺兵部尚書楊鼎和於崑崙關,皆以阻封議故。時順治八年二月也。起恒既死,屍流十餘里,泊沙渚間。虎負之登崖,葬於山麓。

朱天麟,字游初,崑山人。崇禎元年進士。授饒州推官,有惠政。考選入都,貧不能行賂,擬授部曹。帝御經筵,講官並為稱屈。及臨軒親試,乃改翰林編修。十七年正月,奉命祭淮王,抵山東而京師陷。及南都破,走福州,唐王擢少詹事,署國子監事。天麟見鄭芝龍跋扈,乞假至廣東。聞汀州變,又走廣西,入安平土州。

順治四年,永明王居武岡,以禮部侍郎召。天麟疏請王自將,倡率諸鎮,毋坐失事機。辭不至。明年,王在南寧,擢禮部尚書,尋拜東閣大學士。天麟請親率士兵略江右,不聽,乃趨謁王。會李成棟反大清,從王至潯州。而潯帥陳邦傳請世居廣西如黔國公故事,天麟執不允。邦傳怒,以慶國公印、尚方劍擲天麟舟中,要必得,仍執不允。已而成棟奉王駐肇慶,天麟謂機可乘,復勸王亟頒親征詔,規取中原。王優詔答之。

當是時,朝臣各樹黨。從成棟至者,曹曄、耿獻忠、洪天擢、潘曾緯、毛毓祥、李綺,自誇反正功,氣凌朝士。從廣西扈行至者,天麟及嚴起恒、王化澄、晏清、吳貞毓、吳其雷、洪士彭、雷德復、尹三聘、許兆進、張孝起,自恃舊臣,詆曹、耿等嘗事異姓。久之復分吳、楚兩黨。主吳者,天麟、孝起、貞毓、李用楫、堵胤錫、王化澄、萬翱、程源、郭之奇,皆內結馬吉翔,外結陳邦傳。主楚者,袁彭年、丁時魁、蒙正發、劉湘客、金堡,皆外結瞿式耜,內結李元胤。元胤者,惠國公成棟子,為錦衣指揮使,進封南陽伯,握大權。彭年等倚為心腹,勢張甚。

彭年嘗論事王前,語不遜。王責以君臣之義,彭年勃然曰:「儻向者惠國以五千鐵騎,鼓行而西,君臣義安在?」王變色,大惡之。彭年等謀攻去吉翔、邦傳,權可獨擅也。而堡居言路,有鋒氣,乃疏陳八事,劾慶國公邦傳十可斬,文安侯吉翔,司禮中官龐天壽,大學士起恒、化澄與焉。起恒、化澄乞去,天麟奏留之。堡與給事中時魁等復相繼劾起恒、吉翔、天壽無已。太后召天麟面諭,武岡危難,賴吉翔左右,令擬諭嚴責堡等。天麟為兩解,卒未嘗罪言者,而彭年輩怒不止。王知群臣水火甚,令盟於太廟,然黨益固不能解。

明年春,邦傳訐堡官臨清嘗降流賊,受其職,且請堡為己監軍。天麟因擬諭譏堡,堡大憤。時魁乃鼓言官十六人詣閣詆天麟,至登殿陛大嘩,棄官擲印而出。王方坐後殿,與侍臣論事,大驚,兩手交戰,茶傾於衣,急取還天麟所擬而罷。天麟遂辭位,王慰留再三,不可。陛辭,叩頭泣。王亦泣曰:「卿去,餘益孤矣。」

初,時魁等謂所擬出起恒意,欲入署毆之。是日,起恒不入,而天麟獨自承。遂移怒天麟,逐之去,天麟移居慶遠。化澄貪鄙無物望,亦為時魁等所攻,碎冠服辭去。王乃召何吾騶、黃士俊入輔。未幾,吾騶亦為堡等排去,獨士俊、起恒在,乃復召天麟,天麟不至。堡等既連逐三相,益橫,每闌入閣中,授閣臣以意指。王不得已,建文華殿於正殿旁,令閣臣侍坐擬旨以避之。堡又連劾堵胤錫及侍郎萬翱、程源、郭之奇,尚書吳貞毓。貞毓等欲排去之,畏元胤為援,不敢發。

七年春,王赴梧州,元胤留肇慶,陳邦傳適遣兵入衛。貞毓、之奇、翱、源乃合諸給事御史劾彭年、湘客、時魁、堡、正發把持朝政,罔上行私罪。王謂彭年反正有功,免議,下堡等獄。堡又以語觸忌,與時魁並謫戍。湘客、正發贖配追臟。王乃再召天麟,天麟疏言:「年來百爾構爭,盡壞實事。昔宋高宗航海,猶有退步。今則何地可退?當奮然自將,文武諸臣盡擐甲胄。臣亦抽峒丁,擇土豪,募水手,經略嶺北、湖南,為六軍倡。若徒責票擬,以為主持政本,今政本安在乎?」

時大兵益逼,孫可望請王赴雲南。初,起恒持可望封,天麟及化澄獨謂宜許。及可望使至,天麟力請從之。請臣以起恒被殺故,皆不可。天麟乃奉命經略左、右兩江土司,以為勤王之助。兵未集,大兵逼南寧,王倉皇出走,天麟扶病從之。明年四月抵廣南,王已先駐安龍。天麟病劇,不能入覲,卒於西阪村。

張孝起,吳江人。舉於鄉,授廉州推官。大兵至,逼海濱,舉兵謀恢復。戰敗被獲,妻妾俱投海死。孝起羈軍中,會李成棟叛大清,孝起乃脫去。永明王以為吏科給事中。清真介直,不與流俗伍。王至梧州。劉湘客、丁時魁、金堡、蒙正發以失李元胤援,並辭職。王報許,以孝起代時魁,掌吏科印。俄與廷臣共排去湘客等,遂為其黨所疾。高必正,湘客鄉人也,尤疾之,怒罵於朝,王為解乃已。久之,擢孝起右僉都御史,巡撫高、雷、廉、瓊四府。城破,走避龍門島。島破,被執,不食七日死。

楊畏知,寶雞人。崇禎中,歷官雲南副使,分巡金滄。乙酉秋,武定土官吾必奎反,連陷祿豐、廣通諸縣及楚雄府。畏知督兵復楚雄,駐其地。必奎伏誅,而阿迷土官沙定洲繼亂,據雲南,黔國公沐天波走楚雄。巡撫吳兆元不能制,許為奏請鎮雲南。定洲遂西追天波,畏知說天波走永昌,而己以楚雄當定洲。定洲至,畏知復紿之曰:「若所急者,黔國爾,今已西。待爾定永昌還,朝命當已下,予出城以禮見。今順逆未分,不能為不義屈也。」定洲恐失天波,與盟而去。分兵陷大理、蒙化。畏知乘間清野繕堞,徵鄰境援兵,姚安、景東俱響應。定洲聞,不敢至永昌,還攻楚雄,不能下。畏知伺賊懈,輒出擊,殺傷多。乃引去,還攻石屏、寧州、嶍峨,皆陷之。復西攻楚雄,迄不能下。明年,孫可望等入雲南,定洲還救,大敗,遁歸阿迷,可望等遂據會城。

初,唐王聞畏知抗賊,進授右僉都御史,巡撫雲南,以巡撫吳兆元為總督。及可望等至,以畏知同鄉,甚重之。尋與劉文秀西略,畏知拒戰敗,投水不死,踞而罵。可望下馬慰之曰:「聞公名久。吾為討賊來,公能共事,相與匡扶明室,非有他也。」畏知瞪目視之曰:「紿我爾。」可望曰:「不信,當折矢誓。」畏知曰:「果爾,當從我三事:一不得仍用偽西年號,二不得殺人,三不得焚廬舍,淫婦女。」可望皆許諾。乃與至楚雄,略定大理諸郡,使文秀至永昌迎天波歸。迤西八府免屠戮,畏知力也。

時永明王已稱號於肇慶,而詔令不至。前御史臨安任僎議尊可望為國主,以干支紀年,鑄興朝通寶錢。畏知憤甚,有所忤,輒抵掌謾罵。可望數欲殺之,李定國、劉文秀為保護得免。可望與劉、李同輩,一旦自尊,兩人不為下。聞肇慶有君,李錦、李成棟等並加封爵,念得朝命,加王封,庶可相制,乃議遣使奉表。畏知亦素以尊主為言。歲已丑,遣畏知及永昌故兵部郎中龔彞赴肇慶進可望表,請王封,為金堡等所持。畏知乃曰:「可望欲權出劉、李上爾。今晉之上公,而卑劉、李侯爵可也。」乃議封可望景國公,賜名朝宗;定國、文秀皆列侯。遣大理卿趙昱為使,加畏知兵部尚書,彞兵部侍郎,同行。

時堵胤錫曾賜空敕,得便宜行事。昱乃就與謀,矯命改封可望平遼王,易敕書以往。武康伯胡執恭者,慶國公陳邦傳中軍也,守泗城。州與雲南接,欲自結可望,言於邦傳,先矯命封可望秦王,曰:「藉其力可制李赤心也。」邦傳乃鑄金章曰:「秦王之寶」,填所給空敕,令執恭齎行。可望大喜,郊迎。亡何,畏知等至。可望駭不受,曰:「我已封秦王矣。」畏知曰:「此偽也。」執恭亦曰:「彼亦偽也,所封實景國公,敕印故在。」可望怒,辭敕使,下畏知及執恭獄,而遣使至梧州問故,廷臣始知矯詔事。文安侯馬吉翔請封可望澄江王,使者言,非「秦」不敢復命。大學士嚴起恒持不可,兵部侍郎楊鼎和助之,且請卻所獻白金玉帶。會鄖國公高必正等入朝,召使者言:「本朝無異姓封王例。我破京師,逼死先帝,滔天大罪,蒙恩宥赦,亦止公爵爾。張氏竊據一隅,罪固減等,封上公足矣,安敢冀王爵。自今當與我同心報國,洗去賊名,毋欺朝廷孱弱,我兩家士馬足相當也。」又致書可望,詞義嚴正。使者唯唯退,議遂寢。必正者,李自成妻弟,同陷京師者也。

可望不得封,益怒。其年九月親率兵至貴州。十一月,大兵破廣州、桂林,王走南寧。事急,遣編修劉襜封可望冀王,可望仍不受。畏知曰:「『秦』『冀』等爾,假何如真?」可望不聽。定國等勸可望遣畏知終其事,可望許之。明年二月先遣部將賀九儀、張勝、張明志赴南寧索沮「秦」封者起恒、鼎和及給事中劉堯珍、吳霖、張載述殺之,乃真封可望秦王。而畏知旋至,痛哭自劾,語多侵可望。遂留為東閣大學士,與吳貞毓同輔政。可望聞之怒,使人召至貴陽,面責數之。畏知大憤,除頭上冠擊可望,遂被殺。楚雄人以畏知守城功,為立祠以祀。

吳貞毓,字元聲,宜興人。崇禎十六年進士。事唐王為吏部文選主事。事敗,擁立永明王,進郎中。王駐全州,加太常少卿,仍掌選事。已,擢吏部右侍郎,從至肇慶,拜戶部尚書。廣東、西會城先後失,王徙潯州,再徒南寧,貞毓並從。貞毓與嚴起恒共阻孫可望秦王封,可望殺起恒,貞毓以奉使獲免。及還,進東閣大學士,代起恒。可望自雲南遷貴陽,議移王自近,挾以作威。其將掌塘報者曹延生惎貞毓,言不可移黔。

時順治八年,大兵南征,勢日迫。王召諸臣議,有請走海濱就李元胤者,有議入安南避難者,有議泛海抵閩依鄭成功者。惟馬吉翔、龐天壽結可望,堅主赴黔。貞毓因前阻封議,且入延生言,不敢決。元胤疏請出海。王不欲就可望,而以海濱遠,再下廷議,終不決。亡何,開國公趙印選、衛國公胡一青殿後軍,戰敗奔還。請王速行,急由水道走土司,抵瀨湍。二將報大兵益近,相距止百里。上下失色,皆散去。已,次羅江土司,追騎相距止一舍。會日晡引去,乃稍安。次龍英,抵廣南,歲己暮。

可望遣兵以明年二月迎王入安隆所,改為安龍府,奉王居之。宮室庳陋,服御粗惡,守護將悖逆無人臣禮,王不堪其憂。吉翔掌戎政,天壽督勇衛營,諂事可望,謀禪代。惡貞毓不附己,令其黨冷孟銋、吳象元、方祚亨交章彈擊。且語孟銋等曰:「秦王宰天下,我具啟,以內外事盡付戎政、勇衛二司。大權歸我,公等為羽翼,貞毓何能為!」吉翔遂遣門生郭璘說主事胡士瑞擁戴秦王。士瑞怒,歷聲叱退之。他日,吉翔遣璘求郎中古其品畫《堯舜禪受圖》以獻可望,其品拒不從。吉翔譖於可望,杖殺其品,而可望果以朝事盡委吉翔、天壽。於是士瑞與給事中徐極,員外郎林青陽、蔡縯,主事張鐫連章發其奸謀。王大怒。兩人求救於太后,乃免。

前御史任僎、中書方于宣勸可望設內閣九卿科道官,改印交為八疊,盡易其舊,立太廟,定朝儀,擬改國號曰:「後明」,日夕謀篡位。王聞憂懼,密謂中官張福祿、全為國曰:「聞晉王李定國已定廣西,軍聲大振。欲密下一敕,令統兵入衛。若等能密圖乎?」二人言徐極、林青陽、張鐫、蔡縯、胡士瑞曾疏劾吉翔、天壽,宜可與謀,王即令告之。五人許諾,引以告貞毓。貞毓曰:「主上憂危,正我輩報國之秋。諸君中誰能充此使者?」青陽請行。乃令佯乞假歸葬,而使員外郎蔣乾昌撰予定國敕,主事朱東旦書之,福祿等持入用寶。青陽於歲盡間道馳至定國所。定國接敕感泣,許以迎王。

明年夏,青陽久未還,王將擇使往促,貞毓以翰林孔目周官對。都督鄭胤元曰:「吉翔晨夕在側,假他事出之外,庶有濟。」王乃令吉翔奉使祭先王及王太后陵於梧州、南寧,而遣周官詣定國。吉翔在道,微知青陽密敕事,遣人至定國營偵之。主事劉議新者,道遇吉翔,意其必預謀也,告以兩使齎敕狀。吉翔驚駭,啟報可望。可望大怒,並疑吉翔預謀,遣其將鄭國赴南寧逮之。會鐫、士瑞及李元開以王親試,極、縯、東旦及御史林鍾以久次,皆予美官。天壽及吉翔弟都督雄飛忌甚,與其黨郭璘方謀陷之。而鍾、縯、極、鐫、士瑞亦知事洩,倉皇劾吉翔、天壽表裏為奸。王見事急,即下廷臣議罪。天壽懼,與雄飛馳貴陽,告可望。

初,青陽還至南寧,為守將常榮所留,密遣親信劉吉告之王。王喜,改青陽給事中,諭貞毓再撰敕,鑄「屏翰親臣」金印,令吉還付青陽。至廉州,周官與青陽遇,偕至高州賜定國,定國拜受命。

而是時鄭國已械吉翔至安龍,與諸臣面質。貞毓謝不知,國怒,因挾貞毓直入王所居文華殿,迫脅王,索主謀者。王懼,不敢正言,謂必外人假敕寶為之。國遂努目出,與天壽至朝房,械貞毓並胤元、鍾、縯、乾昌、元開、極、鐫、士瑞、東旦及太僕少卿趙賡禹,御史周允吉、朱議篸,員外郎任斗墟,主事易士佳繫私室。又入宮擒福祿、為國而出。其黨冷孟銋、蒲纓、宋德亮、朱企鋘等迫王速具主名,王悲憤而退。翊日,國等嚴刑拷掠,獨貞毓以大臣免。眾不勝楚,大呼二祖列宗,且大罵。時日已暮,風雷忽震烈。縯厲聲曰:「今日縯等直承此獄,稍見臣子報國苦衷。」由是眾皆自承。國又問曰:「主上知否?」縯大聲曰:「未經奏明。」乃復收繫,以欺君誤國盜寶矯詔為罪,報可望。可望請王親裁,王不勝憤,下廷議。吏部侍郎張佐辰及纓、德亮、孟銋、企鋘、蔣御曦等謂國曰:「此輩盡當處死。儻留一人,將為後患。」於是御曦執筆,佐辰擬旨,以鐫、福祿、為國為首罪,凌遲,餘為從罪,斬。王以貞毓大臣,言於可望罪絞。吉翔以福祿等內侍,謂王后知情,將廢之,令主事蕭尹歷陳古廢后事。后泣訴於王,乃已。諸人就刑,神色不變,各賦詩大罵而死。其家人合瘞於安龍北關之馬場。已而青陽逮至,亦被殺,獨官走免。時順治十一年三月也。

居二載,定國竟奉前敕護王入雲南。乃贈貞毓少師、太子太師、吏部尚書、中極殿大學士,賜祭,謚文忠,廕子錦衣,世千戶,餘贈恤有差。已,建廟於馬場,勒碑大書「十八先生成仁處」以旌其忠。

定國既奉王居滇,即捕吉翔及其家人,令部將靳統武收繫,將殺之。吉翔日媚統武,定國客詣統武,吉翔復媚之。因相與譽吉翔於定國,而微為辨冤。定國召吉翔,吉翔入謁,即叩頭言:「王再造功,千古無兩。吉翔幸望見顏色,死且不朽,他是非,何足辨也。」定國乃大喜。吉翔因日諂定國客,令說定國薦己入內閣,遂與定國客蟠結,盡握中外權,天壽亦復用事。後從王入緬甸,天壽先死,吉翔為緬人所殺。

高勣,字無功,紹興人。事永明王,歷官光祿少卿。馬吉翔、龐天壽構殺吳貞毓等,李定國奉王至雲南,捕吉翔將殺之。已,為其所諛,遂免死,且薦入閣,遂得盡握中外權,而天壽亦用事。定國與劉文秀時詣二人家,定國時封晉王,文秀蜀王也。勣與御史鄔昌期患之,合疏言二人功高望重,不當往來權佞之門,恐滋奸弊,復蹈秦王故轍。疏上,二王遂不入朝。吉翔激王怒,命各杖一百五十,除名。定國客金維新走告定國曰:「勣等誠有罪,但不可有殺諫官名。」定國即偕文秀入救,乃復官。

及定國敗孫可望兵,自以為無他患,武備盡弛。勣與郎官金簡進諫曰:「今內難雖除,外憂方大。伺我者頓刃待兩虎之斃,而我酣歌漏舟之中,熟寢爇薪之上,能旦夕安耶?二王老於兵事,胡泄泄如死。」定國訴之王前,頗激。王擬杖二臣以解之,朝士多爭不可,移時未能決。而三路敗書至,定國始逡巡引謝,二臣獲免。簡,字萬藏,勣鄉人。後王入緬甸,二人扈行,並死之。

有李如月者,東莞人,官御史。王駐安龍時,孫可望獲叛將陳邦傳父子,去其皮,傳屍至安龍。如月劾可望不請旨,擅殺勛鎮,罪同莽、操,而請加邦傳惡謚,以懲不忠。王知可望必怒,留其疏。召如月入,諭以謚本褒忠,無惡謚理。小臣妄言亂制,杖四十,除名,意將解可望。而可望大怒,遣人至王所,執如月至朝門外,抑使跪。如月向闕叩頭,大呼太祖高皇帝,極口大罵。其人遂剔其皮,斷手足及首,實草皮內紉之,懸於通衢。

又有任國璽者,官行人。順治十五年,永明王將出奔,國璽獨請死守。章下廷議,李定國等言:「行人議是。但前途尚寬,暫移蹕,捲土重來,再圖恢復,未晚也。」乃扈王入緬甸。緬俗以中秋日大會群蠻,令黔國公沐天波偕諸酋椎髻跣足,以臣禮見。天波不得已從之,歸泣告眾曰:「我所以屈辱者,懼驚憂主上耳。否則彼將無狀,我罪益大。」國璽與禮部侍郎楊在抗疏劾之。

時龐天壽已死,李國泰代掌司禮監印,吉翔復與表裏為奸。國璽集宋末大臣賢奸事為一書,進之王,吉翔深恨之。王覽止一日,國泰即竊去。國璽尋進御史,疏論時事三不可解,中言禍急然眉,當思出險。吉翔不悅,即令國璽獻出險策。國璽忿然曰:「時事至此,猶抑言官使不言耶!」

時緬甸弟弒兄自立,欲盡殺文武諸臣,遣人來言曰:「蠻俗貴詛盟,請與天朝諸公飲咒水。」吉翔、國泰邀諸臣盡往。至則以兵圍之,令諸臣以次出外,出輒殺之,凡殺四十二人。國璽及在、天波、吉翔、國泰、華亭侯王維恭、綏寧伯蒲纓、都督馬雄飛、吏部侍郎鄧士廉等皆預焉。惟都督同知鄧凱以傷足不行,獲免。時順冶十八年七月也。自是由榔左右無人。至十二月,緬人遂送之出境,事具國史。

初,由榔之走緬甸也,昆明諸生薛大觀歎息曰:「不能背城戰,君臣同死社稷,顧欲走蠻邦以茍活,不重可羞耶!」顧子之翰曰:「吾不惜七尺軀,為天下明大義,汝其勉之!」之翰曰:「大人死忠,兒當死孝。」大觀曰:「汝有母在。」時其母適在旁,顧之翰妻曰:「彼父子能死忠孝,吾兩人獨不能死節義耶?」其侍女方抱幼子,問曰:「主人皆死,何以處我?」大觀曰:「爾能死,甚善。」於是五人偕赴城北黑龍潭死。次日,諸屍相牽浮水上,幼子在侍女懷中,兩手堅抱如故。大觀次女已適人,避兵山中,相去數十里,亦同日赴火死。

有那嵩者,沅江土官也。世為知府。嵩嗣職,循法無過。王走緬甸,過沅江,嵩與子燾迎謁,供奉甚謹,設宴皆金銀器。宴畢,悉以獻,曰:「此行上供者少,聊以佐缺乏耳。」後李定國號召諸土司兵,嵩即起兵應之。已而城破,登樓自焚,闔家皆死,其士民亦多巷戰死。

贊曰:明自神宗而後,浸微浸滅,不可復振。揆厥所由,國是紛呶,朝端水火,寧坐視社稷之淪胥,而不能破除門戶之角立。故至桂林播越,旦夕不支,而吳、楚之樹黨相傾,猶仍南都翻案之故態也。顛覆之端,有自來矣,於當時任事諸臣何責哉。


\end{pinyinscope}