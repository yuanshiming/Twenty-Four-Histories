\article{列傳第一百三十九}

\begin{pinyinscope}
李標(李國普(周道登}}劉鴻訓錢龍錫錢士升士晉成基命何如寵兄如申錢象坤徐光啟鄭以偉林釬文震孟周炳謨蔣德璟黃景昉方岳貢邱瑜瑜子之陶

李標,字汝立,高邑人。萬歷三十五年進士。改庶吉士,授檢討。泰昌時,累遷少詹事。天啟中,擢拜禮部右侍郎,協理詹事府。標師同邑越南星,黨人忌之,列名《東林同志錄》中。標懼禍,引疾歸。

莊烈帝嗣位,即家拜禮部尚書兼東閣大學士。崇禎元年三月入朝。未幾,李國普、來宗道、楊景辰相繼去,標遂為首輔。帝銳意圖治,恒召大臣面決庶政。宣府巡撫李養沖疏言旂尉往來如織,蹤跡難憑,且慮費無所出。帝以示標等曰:「邊情危急,遣旂尉偵探,奈何以為偽?且祖宗朝設立廠衛,奚為者?」標對曰:「事固宜慎。養沖以為不賂恐毀言日至,賂之則物力難勝耳。」帝默然。同官劉鴻訓以增敕事為御史吳玉所糾,帝欲置鴻訓於法,標力辯其納賄之誣。溫體仁訐錢謙益引己結浙闈事為詞,給事中章允儒廷駁之。帝怒,並謙益將重譴,又欲罪給事中瞿式耜、御史房可壯等。標言:「陛下處分謙益、允儒,本因體仁言,體仁乃不安求罷。乞陛下念謙益事經恩詔,姑令回籍;於允儒仍許自新,而式耜等概從薄罰。諸臣安,體仁亦安。」帝不從,自是深疑朝臣有黨,標等遂不得行其志。是冬,韓爌還朝,標讓為首輔,尋與爌等定逆案。

三年正月,爌罷,標復為首輔,累加至少保兼太子太保、戶部尚書、武英殿大學士。先是,與標並相者六人,宗道、景辰以附璫斥,鴻訓以增敕戍,周道登、錢龍錫被攻去,獨標在,遂五疏乞休。至三月得請。家居六年卒。贈少傅,謚文節。

李國普,字元治,高陽人。萬曆四十一年進士。由庶吉士歷官詹事。天啟六年七月,超擢禮部尚書入閣。釋褐十四年即登宰輔,魏忠賢以同鄉故援之也。然國普每持正論。劉志選劾張國紀以撼中宮,國普言:「子不宜佐父難母,而況無間之父母乎!」國紀乃得免罪。御史方震孺及高陽令唐紹堯繫獄,皆力為保全。崇禎初,以登極恩進左柱國、少師兼太子太師、吏部尚書、中極殿大學士。國子監生胡煥猷劾國普等褫衣冠,國普薦復之,時人稱為長厚。元年五月得請歸里,薦韓爌、孫承宗自代。卒,贈太保,謚文敏。宗道、景辰事見《黃立極傳》中。

周道登,吳江人。萬曆二十六年進士。由庶吉士歷遷少詹事。天啟時,為禮部左侍郎,頗有所爭執。以病歸。五年秋,廷推禮部尚書,魏忠賢削其籍。崇禎初,與李標等同入閣。道登無學術,奏對鄙淺,傳以為笑。御史田時震、劉士禎、王道直、吳之仁、任贊化,給事中閻可陛交劾之,悉下廷議。吏部尚書王永光等言道登黨護樞臣王在晉及宗生朱統飾、鄉人陳于鼎館選事,俱有實跡,乃罷歸。閱五年而卒。

劉鴻訓,字默承,長山人。父一相,由進士歷南京吏科給事中。追論故相張居正事,執政忌之,出為隴右僉事。終陜西副使。

萬曆四十一年,鴻訓登第,由庶吉士授編修。神、光二宗相繼崩,頒詔朝鮮。甫入境,遼陽陷。朝鮮為造二洋舶,從海道還。沿途收難民,舶重而壞。跳淺沙,入小舟,飄泊三日夜,僅得達登州報命。遭母喪,服闋,進右中允,轉左諭德,父喪歸。天啟六年冬,起少詹事,忤魏忠賢,斥為民。

莊烈帝即位,拜禮部尚書兼東閣大學士,參預機務,遣行人召之。三辭,不允。崇禎元年四月還朝。當是時,忠賢雖敗,其黨猶盛,言路新進者群起抨擊之。諸執政嘗與忠賢共事,不敢顯為別白。鴻訓至,毅然主持,斥楊維垣、李恒茂、楊所修、田景新、孫之獬、阮大鋮、徐紹吉、張訥、李蕃、賈繼春、霍維華等,人情大快。而御史袁弘勛、史褵、高捷本由維垣輩進,思合謀攻去鴻訓,則黨人可安也。弘勛乃言所修、繼春、維垣夾攻表裏之奸,有功無罪,而誅鋤自三臣始;又詆鴻訓使朝鮮,滿載貂參而歸。錦衣僉事張道浚亦訐攻鴻訓,鴻訓奏辯。給事中顏繼祖言:「鴻訓先朝削奪。朝鮮一役,舟敗,僅以身免。乞諭鴻訓入直,共籌安攘之策。至弘勛之借題傾人,道浚之出位亂政,非重創未有已也。」帝是之。給事中鄧英乃盡發弘勛贓私,且言弘勛以千金贄維垣得御史。帝怒,落弘勛職候勘。已而高捷上疏言鴻訓斥擊奸之維垣、所修、繼春、大鋮,而不納孫之獬流涕忠言;謬主焚毀《要典》,以便私黨孫慎行進用。帝責以妄言,停其俸。史褷復佐捷攻之。言路多不直兩人,兩人遂罷去。

七月,以四川賊平,加鴻訓太子太保,進文淵閣。帝數召見廷臣。鴻訓應對獨敏,謂民困由吏失職,請帝久任責成。以尚書畢自嚴善治賦,王在晉善治兵,請帝加倚信。帝初甚向之。關門兵以缺餉鼓噪,帝意責戶部,而鴻訓請發帑三十萬,示不測恩,由是失帝指。

至九月而有改敕書之事。舊例,督京營者,不轄巡捕軍。惠安伯張慶臻總督京營,敕有「兼轄捕營」語,提督鄭其心以侵職論之。命核中書賄改之故,下舍人田佳璧獄。給事中李覺斯言:「稿具兵部,送輔臣裁定,乃令中書繕寫。寫訖,復審視進呈。兵部及輔臣皆當問。」十月,帝御便殿,問閣臣,皆謝不知。帝怒,令廷臣劾奏;尚書自嚴等亦謝不知,帝益怒。給事中張鼎延、御史王道直咸言慶臻行賄有跡,不知誰主使。御史劉玉言:「主使者,鴻訓也。」慶臻曰:「改敕乃中書事,臣實不預知。且增轄捕卒,取利幾何,乃行重賄?」帝叱之。閱兵部揭有鴻訓批西司房語,佳璧亦供受鴻訓指,事遂不可解,而侍郎張鳳翔詆之尤力。閣臣李標、錢龍錫言鴻訓不宜有此,請更察訪。帝曰:「事已大著,何更訪為?」促令擬旨。標等逡巡未上,禮部尚書何如寵為鴻訓力辯,帝意卒不可回。乃擬旨,鴻訓、慶臻並革職候勘。無何,御史田時震劾鴻訓用田仰巡撫四川,納賄二千金;給事中閻可陛劾副都御史賈毓祥由賂鴻訓擢用。鴻訓數被劾,連章力辯,因言「都中神奸狄姓者,詭誆慶臻千金,致臣無辜受禍。」帝不聽,下廷臣議罪。

明年正月,吏部尚書王永光等言:「鴻訓、慶臻罪無可辭,而律有議貴條,請寬貸。兵部尚書王在晉、職方郎中苗思順贓證未確,難懸坐。」帝不許。鴻訓謫戍代州,在晉、思順並削籍,慶臻以世臣停祿三年。覺斯、鼎延、道直、玉、時震以直言增秩一級。

鴻訓居政府,銳意任事。帝有所不可,退而曰:「主上畢竟是沖主。」帝聞,深銜之,欲置之死。賴諸大臣力救,乃得稍寬。七年五月卒戍所。福王時,復官。

錢龍錫,字稚文,松江華亭人。萬曆三十五年進士。由庶吉士授編修,屢遷少詹事。天啟四年擢禮部右侍郎,協理詹事府。明年改南京吏部右侍郎。忤魏忠賢,削籍。

莊烈帝即位,以閣臣黃立極、施鳳來、張瑞圖、李國普皆忠賢所用,不足倚,詔廷臣推舉,列上十人。帝仿古枚卜典,貯名金甌,焚香肅拜,以次探之,首得龍錫,次李標、來宗道、楊景辰。輔臣以天下多故,請益一二人,復得周道登、劉鴻訓,並拜禮部尚書兼東閣大學士。明年六月,龍錫入朝,立極等四人俱先罷,宗道、景辰亦以是月去。標為首輔,龍錫、鴻訓協心輔理,朝政稍清。尋以蜀寇平,加太子太保,改文淵閣。

帝好察邊事,頻遣旂尉偵探。龍錫言:「舊制止行於都城內外,若遠遣恐難委信。」海寇犯中左所,總兵官俞咨皋棄城遁,罪當誅。帝欲並罪巡撫朱一馮。龍錫言:「一馮所駐遠,非棄城者比,罷職已足蔽辜。」瑞王出封漢中,請食川鹽。龍錫言:「漢中食晉鹽,而瑞籓獨用川鹽,恐奸徒借名私販,莫敢譏察。」故事,纂修實錄,分遣國學生採事跡於四方,龍錫言「實錄所需在邸報及諸司奏牘,遣使無益,徒滋擾,宜停罷。」烏撒土官安效良死,其妻改適霑益土官安邊,欲兼有烏撒,部議將聽之。龍錫言:「效良有子其爵,立其爵以收烏撒,存亡繼絕,於理為順。安邊淫亂,不可長也。」帝悉從之。明年,帝以漕船違禁越關,欲復設漕運總兵官。龍錫言:「久裁而復,宜集廷臣議得失。」事竟止。廷議汰冗官,帝謂學官尤冗。龍錫言:「學官舊用歲貢生,近因舉人乞恩選貢,纂修占缺者多,歲貢積至二千六百有奇,皓首以歿,良可憫。且祖宗設官,於此稍寬者,以師儒造士需老成故也。」帝亦納之。言官鄒毓祚、韓一良、章允儒、劉斯琜獲譴,並為申救。

御史高捷、史褷既罷,王永光力引之,頗為龍錫所扼,兩人大恨。逆案之定,半為龍錫主持,奸黨銜之次骨。及袁崇煥殺毛文龍,報疏云:「輔臣龍錫為此一事低徊過臣寓。」復上善後疏言:「閣臣樞臣,往復商確,臣以是得奉行無失。」時文龍擁兵自擅,有跋扈聲,崇煥一旦除之,即當寧不以為罪也。其冬十二月,大清兵薄都城。帝怒崇煥戰不力,執下獄,而捷、褷已為永光引用。捷遂上章,指通款殺將為龍錫罪,且言祖大壽師潰而東,由龍錫所挑激。帝以龍錫忠慎,戒無過求。龍錫奏辯,言:「崇煥陛見時,臣見其貌寢,退謂同官『此人恐不勝任』。及崇煥以五年復遼自詭,往詢方略,崇煥云:『恢復當自東江始。文龍可用則用之,不可用則去之易易耳。』迨崇煥突誅文龍,疏有『臣低徊』一語。臣念文龍功罪,朝端共知,因置不理。奈何以崇煥誇詡之詞,坐臣朋謀罪?」又辯挑激大壽之誣,請賜罷黜。帝慰諭之,龍錫即起視事。捷再疏攻,帝意頗動。龍錫再辯,引疾,遂放歸。時兵事旁午,未暇竟崇煥獄。

至三年八月,褷復上疏言:「龍錫主張崇煥斬帥致兵,倡為款議,以信五年成功之說。賣國欺君,其罪莫逭。龍錫出都,以崇煥所畀重賄數萬,轉寄姻家,巧為營幹,致國法不伸。」帝怒,敕刑官五日內具獄。於是錦衣劉僑上崇煥獄詞。帝召諸臣於平臺,置崇煥重辟。責龍錫私結邊臣,蒙隱不舉,令廷臣議罪。是日,群議於中府,謂:「斬帥雖龍錫啟端,而兩書有『處置慎重』語,意不在擅殺,殺文龍乃崇煥過舉。至講款,倡自崇煥,龍錫始答以『酌量』,繼答以『天子神武,不宜講款』。然軍國大事,私自商度,不抗疏發奸,何所逃罪?」帝遂遣使逮之。十二月逮至,下獄。復疏辯,悉封上崇煥原書及所答書,帝不省。時群小麗名逆案者,聚謀指崇煥為逆首,龍錫等為逆黨,。更立一逆案相抵。謀既定,欲自兵部發之,尚書梁廷棟憚帝英明,不敢任而止。乃議龍錫大辟,且用夏言故事,設廠西市以待。帝以龍錫無逆謀,令長繫。

四年正月,右中允黃道周疏言龍錫不宜坐死罪。忤旨,貶秩調外,而帝意浸解矣。夏五月大旱,刑部尚書胡應台等乞宥龍錫,給事中劉斯琜繼言之,詔所司再讞。乃釋獄,戍定海衛。在戍十二年,兩遇赦不原。其子請輸粟贖罪,會周延儒再當國,尼不行。福王時,復官歸里。未幾卒,年六十有八。

錢士升,字抑之,嘉善人。萬曆四十四年殿試第一,授修撰。天啟初,以養母乞歸。久之,進左中允,不赴。高邑趙南星、同里魏大中受璫禍,及江西同年生萬燝杖死追贓,皆力為營護,破產助之,以是為東林所推。

崇禎元年起少詹事,掌南京翰林院。明年以詹事召。會座主錢龍錫被逮,送之河干,即謝病歸。四年,起南京禮部右侍郎,署尚書事。祭告鳳陽陵寢,疏陳戶口流亡之狀甚悉。六年九月,召拜禮部尚書兼東閣大學士,參預機務。明年春入朝。請停事例,罷鼓鑄,嚴贓吏之誅,止遣官督催新舊餉,第責成於撫按。帝悉從之。

帝操切,溫體仁以刻薄佐之,上下囂然。士升因撰《四箴》以獻,大指謂寬以御眾,簡以臨下,虛以宅心,平以出政,其言深中時病。帝雖優旨報聞,意殊不懌也。

無何,武生李璡請括江南富戶,報名輸官,行首實籍沒之法。士升惡之,擬旨下刑部提問,帝不許,同官溫體仁遂改輕擬。士升曰:「此亂本也,當以去就爭之。」乃疏言:「自陳啟新言事,擢置省闥。比來借端倖進者,實繁有徒,然未有誕肆如璡者也。其曰縉紳豪右之家,大者千百萬,中者百十萬,以萬計者不能枚舉。臣不知其所指何地。就江南論之,富家數畝以對,百計者什六七,千計者什三四,萬計者千百中一二耳。江南如此,何況他省。且郡邑有富家,固貧民衣食之源也。地方水旱,有司令出錢粟,均糶濟饑,一遇寇警,令助城堡守禦,富家未嘗無益於國。《周禮》荒政十二,保富居一。今以兵荒歸罪於富家朘削,議括其財而籍沒之,此秦皇不行於巴清、漢武不行於卜式者,而欲行聖明之世乎?今秦、晉、楚、豫已無寧宇,獨江南數郡稍安。此議一倡,無賴亡命相率而與富家為難,不驅天下之民胥為流寇不止。或疑此輩乃流寇心腹,倡橫議以搖人心,豈直借端倖進已哉!」疏入,而璡已下法司提問。帝報曰:「即欲沽名,前疏已足致之,毋庸汲汲。」前疏謂《四箴》也。士升惶懼,引罪乞休,帝即許之。

士升初入閣,體仁頗援之。體仁推轂謝升、唐世濟,士升皆為助。文震孟被擠,士升弗能救,論者咎之。至是乃以讜言去位。

弟士晉,萬曆中由進士除刑部主事。恤刑畿輔,平反者千百人。崇禎時,以山東右布政擢雲南巡撫。築師宗、新化六城,浚金針、白沙等河,平土官岑、儂兩姓之亂,頗著勞績。已而經歷吳鯤化訐其營賄,體仁即擬嚴旨,且屬同官林釬弗洩,欲因弟以逐其兄。命下,而士晉已卒,事乃已。士升,國變後七年乃卒。

成基命,字靖之,大名人,後避宣宗諱,以字行。萬曆三十五年進士。改庶吉士,歷司經局洗馬,署國子監司業事。天啟元年,疏請幸學不先白政府,執政者不悅,令以原官還局,遂請告歸。尋起少詹事。累官禮部右侍郎兼太子賓客,改掌南京翰林院事。六年,魏忠賢以基命為楊漣同門生,落職閒住。

崇禎元年,起吏部左侍郎。明年十月,京師戒嚴,基命請召還舊輔孫承宗,省一切浮議,仿嘉靖朝故事,增設樞臣,帝並可之。踰月,拜禮部尚書兼東閣大學士,入閣輔政。庶吉士金聲薦僧申甫為將。帝令基命閱其所部兵,極言不可用,後果一戰而敗。袁崇煥、祖大壽入衛,帝召見平臺,執崇煥屬吏,大壽在旁股慄。基命獨叩頭請慎重者再,帝曰:「慎重即因循,何益?」基命復叩頭曰:「敵在城下,非他時比。」帝終不省。大壽至軍,即擁眾東潰,帝憂之甚。基命曰;「令崇煥作手札招之,當歸命也。」時兵事孔棘,基命數建白,皆允行。及戒嚴,召對文華殿,帝言法紀廢弛,宜力振刷。基命曰:「治道去太甚,譬理亂絲,當覓其緒,驟紛更益擾亂。」帝曰:「慢則糾之以猛,何謂紛更?」其後溫體仁益導帝以操切,天下遂大亂。

三年二月,工部主事李逢申劾基命欲脫袁崇煥罪,故乞慎重。基命求罷,帝為貶逢申一秩。韓爌、李標相繼去,基命遂為首輔,與周延儒、何如寵、錢象坤共事。以恢復永平敘功,並加太子太保,進文淵閣。至六月,溫體仁、吳宗達入,延儒、體仁最為帝所眷,比而傾基命,基命遂不安其位矣。方崇煥之議罪也,基命病足不入直。錦衣張道浚以委卸劾之,工部主事陸澄源疏繼上。基命奏辯曰:「澄源謂臣當兩首廷推,皆韓爌等欲藉以救崇煥。當廷推時,崇煥方倚任,安知後日之敗,預謀救之。其說祖逢申、道浚,不逐臣不止,乞放歸。」帝慰留之。卒三疏自引去。

基命性寬厚,每事持大體。先是,四城未復,兵部尚書梁廷棟銜總理馬世龍,將更置之,以撼樞輔承宗,基命力調劑,世龍卒收遵、永功。尚書張鳳翔、喬允升、韓繼思相繼下吏,並為申理。副都御史易應昌下詔獄,以基命言,改下法司。御史李長春、給事中杜齊芳坐私書事,將置重典。基命力救,不聽,長跪會極門,言:「祖宗立法,真死罪猶三覆奏,豈有詔獄一訊遽置極刑?」自辰至酉未起。帝意解,得遣戍。逢申初劾基命,後以炮炸下獄擬戍,帝猶以為輕,亦以基命言得如擬。為首輔者數月,帝欲委政延儒,遂為其黨所逐。八年卒於家。贈少保,謚文穆。

何如寵,字康侯,桐城人。父思鰲,知棲霞縣,有德於民。如寵登萬歷二十六年進士,由庶吉士累遷國子監祭酒。天啟時,官禮部右侍郎,協理詹事府。五年正月,廷推左侍郎,魏廣微言如寵與左光斗同里友善,遂奪職閒住。

崇禎元年,起為吏部右侍郎。未至,拜禮部尚書。宗籓婚嫁命名,例請於朝。貧者為部所稽,自萬曆末至是,積疏累千,有白首不能完家室,骨朽而尚未名者。用如寵請,貧宗得嫁娶者六百餘人。大學士劉鴻訓以增敕事,帝怒不測,如寵力為剖析,得免死戍邊。明年冬,京師戒嚴,都人桀黠者,請以私財聚眾助官軍,朝議壯之。如寵力言其叵測,不善用,必啟內釁。帝召問,對如初。帝出片紙示之,則得之偵事,與如寵言合,由是受知。十二月,命與周延儒、錢象坤俱以本官兼東閣大學士,入閣輔政。帝欲族袁崇煥,以如寵申救,免死者三百餘口。累加少保、戶部尚書、武英殿大學士。

四年春,副延儒總裁會試。事竣,即乞休,疏九上乃允。陛辭,陳惇大明作之道。抵家,復請時觀《通鑑》,察古今理亂忠佞,語甚切。六年,延儒罷政,體仁當為首輔。而延儒憾體仁排己,謀起如寵以抑之,如寵畏體仁,六疏辭,體仁遂為首輔。

如寵性孝友。母年九十,色養不衰。操行恬雅,與物無競,難進易退,世尤高之。十四年卒。福王時,贈太保,謚文端。

兄如申,與如寵同舉進士。官戶部郎中,督餉遼東。有清操,軍士請復留二載。終浙江右布政使。

錢象坤,字弘載,會稽人。萬曆二十九年進士。改庶吉士,授檢討,進諭德,轉庶子。泰昌改元,官少詹事,直講筵。講畢,見中官王安與執政議事,即趨出。安使人延之,堅不入。天啟中,給事中論織造,語侵中貴,詔予杖,閣臣救不得。象坤語葉向高講筵面奏之,乃免。時行立枷法,慘甚,象坤白之帝,多所寬釋。再遷禮部右侍郎兼太子賓客。

四年七月,向高辭位。御史黃公輔慮象坤柄政,請留向高,詆象坤甚力。象坤遂辭去。六年,廷推南京禮部尚書。魏忠賢私人指為繆昌期黨,落職閒住。

崇禎元年,召拜禮部尚書,協理詹事府。明年冬,都城被兵,條禦敵三策。奉命登陴分守,祁寒不懈。帝覘知,遂與何如寵並相。明年,溫體仁入,象坤其門生,讓而居其下。累加少保,進武英殿。象坤在翰林,與龍錫、謙益、士升並負物望,有「四錢」之目。及體仁相,無附和跡。

四年,御史水佳允連劾兵部尚書梁廷棟,廷棟不待旨即奏辯。廷棟故出象坤門,佳允疑象坤洩之,語侵象坤。延儒以廷棟嘗發其私人贓罪,惡之,並惡象坤。象坤遂五疏引疾去,廷棟落職。給事中吳執御、傅朝佑稱象坤難進易退,不當以門生累,不聽。家居十年,無病而卒。贈太保,謚文貞,廕一子中書舍人。

徐光啟,字子先,上海人。萬歷二十五年舉鄉試第一,又七年成進士。由庶吉士歷贊善。從西洋人利瑪竇學天文、曆算、火器,盡其術。遂遍習兵機、屯田、鹽策、水利諸書。

楊鎬四路喪師,京師大震。累疏請練兵自效,神宗壯之,超擢少詹事兼河南道御史。練兵通州,列上十議。時遼事方急,不能如所請。光啟疏爭,乃稍給以民兵戎械。

未幾,熹宗即位。光啟志不得展,請裁去,不聽。既而以疾歸。遼陽破,召起之。還朝,力請多鑄西洋大炮,以資城守。帝善其言。方議用,而光啟與兵部尚書崔景榮議不合,御史邱兆麟劾之,復移疾歸。天啟三年起故官,旋擢禮部右侍郎。五年,魏忠賢黨智鋌劾之,落職閒住。

崇禎元年召還,復申練兵之說。未幾,以左侍郎理部事。帝憂國用不足,敕廷臣獻屯鹽善策。光啟言屯政在乎墾荒,鹽政在嚴禁私販。帝褒納之,擢本部尚書。時帝以日食失驗,欲罪臺官。光啟言:「臺官測候本郭守敬法。元時嘗當食不食,守敬且爾,無怪臺官之失占。臣聞曆久必差,宜及時修正。」帝從其言,詔西洋人龍華民、鄧玉函、羅雅谷等推算曆法,光啟為監督。

四年春正月,光啟進《日躔曆指》一卷、《測天約說》二卷、《大測》二卷、《日躔表》二卷、《割圜八線表》六卷、《黃道升度》七卷、《黃赤距度表》一卷、《通率表》一卷。是冬十月辛丑朔日食,復上測候四說。其辯時差里差之法,最為詳密。

五年五月,以本官兼東閣大學士,入參機務,與鄭以偉並命。尋加太子太保,進文淵閣。光啟雅負經濟才,有志用世。及柄用,年已老,值周延儒、溫體仁專政,不能有所建白。明年十月卒。贈少保。

鄭以偉,字子器,上饒人。萬曆二十九年進士。改庶吉士,授檢討,累遷少詹事。泰昌元年,官禮部右侍郎。天啟元年,光宗祔廟,當祧憲宗,太常少卿洪文衡以睿宗不當入廟,請祧奉玉芝宮,以偉不可而止,論者卒是文衡。尋以左侍郎協理詹事府。四年,以偉直講筵,與璫忤,上疏告歸。崇禎二年,召拜禮部尚書。久之,與光啟並相,再辭,不允。以偉修潔自好,書過目不忘,文章奧博,而票擬非其所長。嘗曰:「吾富於萬卷,窘於數行,乃為後進所藐。」章疏中有「何況」二字,誤以為人名也,擬旨提問,帝駁改始悟。自是詞臣為帝輕,遂有館員須歷推知之諭,而閣臣不專用翰林矣。以偉累乞休,不允。明年六月,卒官,贈太子太保。御史言光啟、以偉相繼沒,蓋棺之日,囊無餘貲,請優恤以愧貪墨者。帝納之,乃謚光啟文定,以偉文恪。

其後二年,同安林釬為大學士,未半歲而卒。亦有言其清者,得謚文穆。釬,字實甫,萬曆四十四年殿試第三人,授編修。天啟時,任國子司業。監生陸萬齡請建魏忠賢祠於太學旁,具簿醵金,強釬為倡。釬援筆塗抹,即夕挂冠欞星門徑歸,忠賢矯旨削其籍。崇禎改元,起少詹事。九年由禮部侍郎入閣,有謹愿誠恪之稱。

久之,帝念光啟博學強識,索其家遣書。子驥入謝,進《農政全書》六十卷。詔令有司刊布,加贈太保,其孫為中書舍人。

文震孟,字文起,吳縣人,待詔徵明曾孫也。祖國子博士彭,父衛輝同知元發,並有名行。震孟弱冠以《春秋》舉於鄉,十赴會試。至天啟二年,殿試第一,授修撰。

時魏忠賢漸用事,外廷應之,數斥逐大臣。震孟憤,於是冬十月上《勤政講學疏》,言:「今四方多故,無歲不蹙地陷城,覆軍殺將,乃大小臣工臥薪嘗膽之日。而因循粉飾,將使祖宗天下日銷月削。非陛下大破常格,鼓舞豪傑心,天下事未知所終也。陛下昧爽臨朝,寒暑靡輟,政非不勤,然鴻臚引奏,跪拜起立,如傀儡登場已耳。請按祖宗制,唱六部六科,則六部六科以次白事,糾彈敷奏,陛下與輔弼大臣面裁決焉。則聖智日益明習,而百執事各有奮心。若僅揭帖一紙,長跪一諾,北面一揖,安取此鴛行豸繡、橫玉腰金者為?經筵日講,臨御有期,學非不講,然侍臣進讀,鋪敘文辭,如蒙師誦說已耳。祖宗之朝,君臣相對,如家人父子。咨訪軍國重事,閭閻隱微,情形畢照,奸詐無所藏,左右近習亦無緣蒙蔽。若僅尊嚴如神,上下拱手,經傳典謨徒循故事,安取此正笏垂紳、展書簪筆者為?且陛下既與群臣不洽,朝夕侍御不越中涓之輩,豈知帝王宏遠規模?於是危如山海,而閣臣一出,莫挽偷安之習;慘如黔圍,而撫臣坐視,不聞嚴譴之施。近日舉動,尤可異者。鄒元標去位,馮從吾杜門,首揆冢宰亦相率求退。空人國以營私窟,幾似濁流之投;詈道學以逐名賢,有甚偽學之禁。唐、宋末季,可為前鑒。」疏入,忠賢屏不即奏。乘帝觀劇,摘疏中「傀儡登場」語,謂比帝於偶人,不殺無以示天下,帝頷之。一日,講筵畢,忠賢傳旨,廷杖震孟八十。首輔葉向高在告,次輔韓爌力爭。會庶吉士鄭鄤疏復入,內批俱貶秩調外。言官交章論救,不納。震孟亦不赴調而歸。六年冬,太倉進士顧同寅、生員孫文豸坐以詩悼惜熊廷弼,為兵馬司緝獲。御史門克新指為妖言,波及震孟,與編修陳仁錫、庶吉士鄭鄤並斥為民。

崇禎元年以侍讀召。改左中允,充日講官。三年春,輔臣定逆案者相繼去國,忠賢遺黨王永光輩日乘機報復,震孟抗疏糾之。帝方眷永光,不報。震孟尋進左諭德,掌司經局,直講如故。五月,復上疏曰:「群小合謀,欲借邊才翻逆案。天下有無才誤事之君子,必無懷忠報國之小人。今有平生無恥,慘殺名賢之呂純如,且藉奧援思辯雪。永光為六卿長,假竊威福,倒置用舍,無事不專而濟以狠,發念必欺而飾以朴,以年例大典而變亂祖制,以考選盛舉而擯斥清才。舉朝震恐,莫敢訟言。臣下雷同,豈國之福!」帝令指實再奏。震孟言:「殺名賢者,故吏部郎周順昌。年例則抑吏科都給事中陳良訓,考選則擯中書舍人陳士奇、潘有功是也。」永光窘甚,密結大奄王永祚謂士奇出姚希孟門,震孟,希孟舅也。帝心疑之。永光辯疏得溫旨,而責震孟任情牽詆。然群小翻案之謀亦由是中沮。

震孟在講筵,最嚴正。時大臣數逮繫,震孟講《魯論》「君使臣以禮」一章,反覆規諷,帝即降旨出尚書喬允升、侍郎胡世賞於獄。帝嘗足加於膝,適講《五子之歌》,至「為人上者,奈何不敬」,以目視帝足,帝即袖掩之,徐為引下。時稱「真講官」。既忤權臣,欲避去。出封益府,便道歸,遂不復出。

五年,即家擢右庶子。久之,進少詹事。初,天啟時,詔修《光宗實錄》,禮部侍郎周炳謨載神宗時儲位臲卼及「妖書」、「梃擊」諸事,直筆無所阿。其後忠賢盜柄,御史石三畏劾削炳謨職。忠賢使其黨重修,是非倒置。震孟摘尤謬者數條,疏請改正。帝特御平臺,召廷臣面議,卒為溫體仁、王應熊所沮。

八年正月,賊犯鳳陽皇陵。震孟歷陳致亂之源,因言:「當事諸臣,不能憂國奉公,一統之朝,強分畛域,加膝墜淵,總由恩怨。數年來,振綱肅紀者何事,推賢用能者何人,安內攘外者何道,富國強兵者何策?陛下宜奮然一怒,發哀痛之詔,按失律之誅,正誤國之罪,行撫綏之實政,寬閭閻之積逋。先收人心以遏寇盜,徐議濬財之源,毋徒竭澤而漁。盡斥患得患失之鄙夫,廣集群策群力以定亂,國事庶有瘳乎!」帝優旨報之,然亦不能盡行也。

故事,講筵不列《春秋》。帝以有裨治亂,令擇人進講。震孟,《春秋》名家,為體仁所忌,隱不舉。次輔錢士升指及之,體仁佯驚曰:「幾失此人!」遂以其名上。及進講,果稱帝旨。

六月,帝將增置閣臣,召廷臣數十人,試以票擬。震孟引疾不入,體仁方在告。七月,帝特擢震孟禮部左侍郎兼東閣大學士,入閣預政。兩疏固辭,不許。閣臣被命,即投刺司禮大奄,兼致儀狀,震孟獨否。掌司禮者曹化淳,故屬王安從奄,雅慕震孟,令人輾轉道意,卒不往。震孟既入直,體仁每擬旨必商之,有所改必從,喜謂人曰:「溫公虛懷,何云奸也?」同官何吾騶曰:「此人機深,詎可輕信?」越十餘日,體仁窺其疏,所擬不當,輒令改,不從,則徑抹去。震孟大慍,以諸疏擲體仁前,體仁亦不顧。

都給事中許譽卿者,故劾忠賢有聲,震孟及吾騶欲用為南京太常卿。體仁忌譽卿伉直,諷吏部尚書謝升劾其與福建布政使申紹芳營求美官。體仁擬以貶謫,度帝欲重擬必發改,已而果然。遂擬斥譽卿為民,紹芳提問。震孟爭之不得,弗然曰:「科道為民,是天下極榮事,賴公玉成之。」體仁遽以聞。帝果怒,責吾騶、震孟徇私撓亂。吾騶罷,震孟落職閒住。

方震孟之拜命也,即有旨撤鎮守中官。及次輔王應熊之去,忌者謂震孟為之。由是有譖其居功者,帝意遂移。震孟剛方貞介,有古大臣風,惜三月而斥,未竟其用。

歸半歲,會甥姚希孟卒,哭之慟,亦卒。廷臣請恤,不允。十二年,詔復故官。十五年,贈禮部尚書,賜祭葬,官一子。福王時,追謚文肅。二子秉、乘。乘遭國變,死於難。

周炳謨,子仲覲,無錫人。父子義,嘉靖中庶吉士,萬曆中仕至吏部侍郎,卒謚文恪。炳謨,萬曆三十二年進士。當重修《光宗實錄》時,炳謨已先卒。崇禎初,贈禮部尚書,謚文簡。父子皆以學行稱於世。

蔣德璟,字申葆,晉江人。父光彥,江西副使。德璟,天啟二年進士。改庶吉士,授編修。

崇禎時,由侍讀歷遷少詹事,條奏救荒事宜。尋擢禮部右侍郎。時議限民田,德璟言:「民田不可奪,而足食莫如貴粟。北平、山、陜、江北諸處,宜聽民開墾,及課種桑棗,修農田水利。府縣官考滿,以是為殿最。至常平義倉,歲輸本色,依令甲行之足矣。」十四年春,楊嗣昌卒於軍,命九卿議罪。德璟議曰:「嗣昌倡聚斂之議,加剿餉、練餉,致天下民窮財盡,胥為盜,又匿失事,飾首功。宜按仇鸞事,追正其罪。」不從。

十五年二月,耕耤禮成,請召還原任侍郎陳子壯、祭酒倪元璐等,帝皆錄用。六月,廷推閣臣,首德璟。入對,言邊臣須久任,薊督半載更五人,事將益廢弛。帝曰:「不稱當更。」對曰:「與其更於後,曷若慎於初。」帝問:「天變何由弭?」對曰:「莫如拯百姓。近加遼餉千萬,練餉七百萬,民何以堪!祖制,三協止一督、一撫、一總兵,今增二督、三撫、六總兵,又設副將數十人,權不統一,何由制勝!」帝頷之。首輔周延儒嘗薦德璟淵博,可備顧問,文體華贍,宜用之代言。遂擢德璟及黃景昉、吳甡為禮部尚書兼東閣大學士,同入直。延儒、甡各樹門戶,德璟無所比。性鯁直,黃道周召用,劉宗周免罪,德璟之力居多。開封久被圍,自請馳督諸將戰,優詔不允。

明年,進《御覽備邊冊》,凡九邊十六鎮新舊兵食之數,及屯、鹽、民運、漕糧、馬價悉志焉。已,進《諸邊撫賞冊》及《御覽簡明冊》。帝深嘉之。諸邊士馬報戶部者,浮兵部過半,耗糧居多,而屯田、鹽引、民運,每鎮至數十百萬,一聽之邊臣。天津海道輸薊、遼歲米豆三百萬,惟倉場督臣及天津撫臣出入,部中皆不稽核。德璟語部臣,合部運津運、各邊民運、屯、鹽,通為計畫,餉額可足,而加派之餉可裁。因復條十事以責部臣,然卒不能盡釐也。

一日召對,帝語及練兵。德璟曰:「《會典》,高皇帝教練軍士,一以弓弩刀鎗行賞罰,此練軍法。衛所總、小旂補役,以槍勝負為升降。凡武弁比試,必騎射精嫻,方準襲替,此練將法。豈至今方設兵?」帝為悚然。又言:「祖制,各邊養軍止屯、鹽、民運三者,原無京運銀。自正統時始有數萬,迄萬曆末,亦止三百餘萬。今則遼餉、練餉並舊餉計二千餘萬,而兵反少於往時,耗蠹乃如此。」又言:「文皇帝設京衛七十二,計軍四十萬。畿內八府,軍二十八萬。又有中部、大寧、山東、河南班軍十六萬。春秋入京操演,深得居重馭輕勢。今皆虛冒。且自來征討皆用衛所官軍,嘉靖末,始募兵,遂置軍不用。至加派日增,軍民兩困。願憲章二祖,修復舊制。」帝是之,而不果行。

十七年,戶部主事蔣臣請行鈔法,言歲造三千萬貫,一貫價一兩,歲可得銀三千萬兩。侍郎王鰲永贊行之。帝特設內寶鈔局,晝夜督造,募商發賣,無一人應者。德璟言:「百姓雖愚,誰肯以一金買一紙。」帝不聽。又因局官言,責取桑穰二百萬斤於畿輔、山東、河南、浙江。德璟力爭,帝留其揭不下,後竟獲免。先以軍儲不足,歲僉畿輔、山東、河南富戶,給值令買米豆輸天津,多至百萬,民大擾。德璟因召對面陳其害,帝即令擬諭罷之。

二月,帝以賊勢漸逼,令群臣會議,以二十二日奏聞。都御史李邦華密疏云輔臣知而不敢言。翼日,帝手其疏問何事。陳演以少詹事項煜東宮南遷議對,帝取視默然。德璟從旁力贊,帝不答。

給事中光時亨追論練餉之害。德璟擬旨:「向來聚斂小人倡為練餉,致民窮禍結,誤國良深。」帝不悅,詰曰:「聚斂小人誰也?」德璟不敢斥嗣昌,以故尚書李待問對。帝曰:「朕非聚斂,但慾練兵耳。」德璟曰:「陛下豈肯聚斂。然既有舊餉五百萬,新餉九百餘萬,復增練餉七百三十萬,臣部實難辭責。且所練兵馬安在?薊督練四萬五千,今止二萬五千。保督練三萬,今止二千五百;保鎮練一萬,今止二百;若山、永兵七萬八千,薊、密兵十萬,昌平兵四萬,宣大、山西及陜西三邊各二十餘萬,一經抽練,原額兵馬俱不問,并所抽亦未練,徒增餉七百餘萬,為民累耳。」帝曰:「今已并三餉為一,何必多言!」德璟曰:「戶部雖並為一,州縣追比,仍是三餉。」帝震怒,責以朋比。德璟力辯,諸輔臣為申救。尚書倪元璐以鈔餉乃戶部職,自引咎,帝意稍解。明日,德璟具疏引罪。帝雖旋罷練餉,而德璟竟以三月二日去位。給事中汪惟效、檢討傅鼎銓等交章乞留,不聽。德璟聞山西陷,未敢行。及知廷臣留己,即辭朝,移寓外城。賊至,得亡去。

福王立於南京,召入閣。自陳三罪,固辭。明年,唐王立於福州,與何吾騶、黃景昉並召。又明年以足疾辭歸。九月,王事敗,而德璟適病篤,遂以是月卒。

黃景昉,字太稚,亦晉江人。天啟五年進士。由庶吉士歷官庶子,直日講。崇禎十一年,帝御經筵,問用人之道。景昉言「近日考選不公,推官成勇、朱天麟廉能素著,乃不得預清華選。」又言「刑部尚書鄭三俊四朝元老,至清無儔,不當久繫獄。」退復上章論之,三俊旋獲釋,勇等亦俱改官。

景昉尋進少詹事。嘗召對,言:「近撤還監視中官高起潛,關外輒聞警報,疑此中有隱情。臣家海濱,見沿海將吏每遇調發,即報海警,冀得復留。觸類而推,其情自見。」帝頷之。十四年以詹事兼掌翰林院。時庶常停選已久,景昉具疏請復,又請召還修撰劉同升、編修趙士春,皆不報。

十五年六月召對稱旨,與蔣德璟、吳甡並相。明年並加太子少保,改戶部尚書、文淵閣。南京操江故設文武二員,帝欲裁去文臣,專任誠意伯劉孔昭。副都御史惠世揚遲久不至,帝命削其籍。景昉俱揭爭,帝不悅,遂連疏引歸。唐王時,召入直,未幾,復告歸。國變後,家居十數年始卒。

方岳貢,字四長,穀城人。天啟二年進士。授戶部主事,進郎中。歷典倉庫,督永平糧儲,並以廉謹聞。

崇禎元年,出為松江知府。海濱多盜,捕得輒杖殺之。郡東南臨大海,颶潮衝擊,時為民患,築石堤二十里許,遂為永利。郡漕京師數十萬石,而諸倉乃相距五里,為築城垣護之,名曰「倉城」。他救荒助役、修學課士,咸有成績,舉卓異者數矣。薛國觀敗,其私人上海王陛彥下吏,素有隙,因言岳貢嘗饋國觀三千金,遂被逮。士民詣闕訟冤,巡撫黃希亦白其誣,下法司讞奏。一日,帝晏見輔臣,問:「有一知府積俸十餘年,屢舉卓異者誰也?」蔣德璟以岳貢對。帝曰:「今安在?」德璟復以陛彥株連對,帝頷之。法司讞上,言行賄無實跡,宜復官。帝獎其清執,報可。

無何,給事中方士亮薦岳貢及蘇州知府陳洪謐,乃擢山東副使兼右參議,總理江南糧儲。所督漕艘,如期抵通州。帝大喜。吏部尚書鄭三俊舉天下廉能監司五人,岳貢與焉。帝趣使入對,見於平臺,問為政何先,對曰:「欲天下治平,在擇守令;察守令賢否,在監司;察監司賢否,在巡方;察巡方賢否,在總憲。總憲得人,御史安敢以身試法。」帝善之,賜食,日晡乃出。越六日,即超擢左副都御史。嘗召對,帝適以事詰吏部尚書李遇知。遇知曰:「臣正糾駁。」岳貢曰:「何不即題參?」深合帝意。翼日,命以本官兼東閣大學士,時十六年十一月也。故事,閣臣無帶都御史銜者,自岳貢始。

岳貢本吏材。及為相,務勾檢簿書,請核赦前舊賦,意主搜括,聲名甚損。十七年二月命以戶、兵二部尚書兼文淵閣大學士總督漕運、屯田、練兵諸務,駐濟寧。已而不行。

李自成陷京師,岳貢及邱瑜被執,幽劉宗敏所。賊索銀,岳貢素廉,貧無以應,拷掠備至。搜其邸,無所有,松江賈人為代輸千金。四月朔日與瑜並釋。十二日,賊既殺陳演等,令監守者并殺二人,監守者奉以繯,二人並縊死。

邱瑜,宜城人。天啟五年進士。由庶吉士授檢討。崇禎中,屢遷少詹事。襄陽陷,瑜上恤難宗、擇才吏、旌死節、停催征、蘇郵困、禁勞役六事。帝採納焉。歷禮部左右侍郎。因召對,言:「督師孫傳庭出關,安危所係,慎勿促之輕出。俾鎮定關中,猶可號召諸將,相機進剿。」帝不能從。十七年正月以本官兼東閣大學士,同范景文入閣。都城陷,受拷掠者再,搜獲止二千金,既而被害。

瑜子之陶,年少有幹略。李自成陷宜城,瑜父民忠罵賊而死。之陶被獲,用為兵政府從事,尋以本府侍郎守襄陽。襄陽尹牛人全,賊相金星子,其倚任不如也。之陶以蠟丸書貽傳庭曰:「督師與之戰,吾詭言左鎮兵大至,搖其心,彼必返顧。督師擊其後,吾從中起,賊可滅也。」傳庭大喜,報書如其言,為賊邏者所得。傳庭恃內應,連營前進,之陶果舉火,報左兵大至。自成驗得其詐,召而示以傳庭書,責其負己。之陶大罵曰:「吾恨不斬汝萬段,豈從汝反耶!」賊怒,支解之。

贊曰:莊烈帝在位僅十七年,輔相至五十餘人。其克保令名者,數人而已,若標等是也。基命能推轂舊輔以定危難,震孟以風節顯,德璟諳悉舊章。以陸喜之論薛瑩者觀之,所謂侃然體國,執正不懼,斟酌時宜,時獻微益者乎。至於扶危定傾,殆非易言也。嗚呼,國步方艱,人材亦與俱盡,其所由來者漸矣。


\end{pinyinscope}