\article{列傳第八十八}

\begin{pinyinscope}
姚鏌子淶張嵿伍文定邢珣等蔡天佑胡瓚張文錦詹榮劉源清劉天和楊守禮張岳李允簡郭宗皋趙時春

姚鏌,字英之,慈谿人。弘治六年進士。除禮部主事,進員外郎。擢廣西提學僉事。立宣成書院,延《五經》師以教士子。桂人祀山魈卓旺。鏌毀像,俗遂變。遷福建副使,未幾改督學政。正德九年擢貴州按察使。十五年拜右副都御史,巡撫延綏。上邊務六事,皆議行。嘉靖元年,吉囊入涇陽。鏌遣遊擊彭楧出西路,釋指揮卜雲於獄,使副之。夜半邀擊,斬其二將,乃遁。璽書褒諭。尋召為工部右侍郎,出督漕運,改兵部左侍郎。

四年,遷右都御史,提督兩廣軍務兼巡撫。田州土官岑猛謀不軌。鏌調永順、保靖兵,使沈希儀與張經、李璋、張佑、程鑒各統兵八萬,分道討。而鏌與總兵官朱麒等攻破定羅、丹梁。用希儀計,結猛婦翁岑璋使為內應,大破之,斬猛子邦彥。璋誘殺猛,獻其首。詔進鏌左都御史,加太子少保,任一子官,諸將進秩有差。鏌請改設流官,陳善後七事,制可。乃命參議汪必東、僉事申惠與參將張經以兵萬人鎮其地。必東、惠移疾他駐。猛黨盧蘇、王受等詐言猛不死,借交阯兵二十萬且至,夷民信之。蘇等薄城,經突圍走,城遂陷。王受亦攻入思恩府。巡按御史石金劾鏌失策罔上,並論前總督盛應期。帝以鏌有功,許便宜撫剿。蘇、受數求赦,鏌不許,將大討之。會廷議起王守仁督兩廣軍,令鏌與同事。鏌引疾乞罷,許馳驛歸。

初,廣東提學道魏校毀諸寺觀田數千畝,盡入霍韜、方獻夫諸家。鏌至廣,追還之官。韜、獻夫恨甚,與張璁、桂萼合排鏌。謂大同當征而反撫,田州當撫而反徵,皆費宏謀國不臧,釀成南北患。時宏雖去,猶借鏌以排之也。鏌既得請,方候代,千夫長韋貴、徐伍攻復思恩。鏌上其狀。詔先賞貴等,而以撫剿事宜俟守仁處置。既而鏌奏辯石金前疏,詆金阻撓養寇。金亦再疏詆鏌。帝先入璁等言,落鏌職閒住。

其後,蘇、受復叛,帝漸思鏌。十三年,三邊闕總制。大學士費宏、李時同召對。宏薦鏌,時亦助之。遂命以兵部尚書總制三邊軍務。未赴,宏卒。鏌辭。帝不悅,仍落職閒住。鏌既罷,薦者至二十疏,不用。家居數年卒。

子淶,字維東。嘉靖二年殿試第一。授翰林修撰。爭「大禮」,廷杖。又議郊祀合祀,不當輕易。召修《明倫大典》,懇辭不與。累官侍讀學士。

張嵿,字時俊,蕭山人。成化二十三年進士。弘治初,修《憲宗實錄》,命往蘇、松諸府採軼事。事竣,授上饒知縣。遷南京兵部主事,就進刑部郎中。正德初,遷興化知府。隆平侯張祐無子,弟祿與族人爭襲,訴於南京法司,久不決,復訴京師。劉瑾方擅政,遂削尚書樊瑩、都御史高銓籍。嵿以郎承勘,為民。瑾敗,起知南雄。擢江西參政,進右布政使。舉治行卓異,遷左。寧王宸濠欲拓地,廣其居,嵿執不可。大恚,遣人饋之。嵿發視,則棗梨姜芥,蓋隱語也。未幾,召為光祿卿。以右副都御史巡撫保定諸府,忤中貴,移疾歸。

世宗即位,命以右都御史總督兩廣軍務。廣西上思州賊黃鏐糾峒兵劫州縣,嵿討擒之。廣東新寧、恩平賊蔡猛三等剽掠,眾至數萬。嵿合兵三萬餘人擊新寧諸賊,破巢二百,擒斬一萬四千餘人,俘賊屬五千九百餘人,猛三等皆授首。自嶺南用兵,以寡勝眾未有若是役者。捷聞,獎賚。程鄉賊梁八尺等與福建上杭流賊相應。遣都指揮李皋等會福建官兵夾擊,俘斬五百餘人。歸善李文積聚奸宄拒捕,討之,久弗克。山頂遣參政徐度等剿之,俘斬千餘人。佛郎機國人別都盧剽劫滿剌加諸國,復率其屬灊世利等擁五舟破巴西國,遂入寇新會。嵿遣將出海擒之,獲其二舟,賊乃遁。尋召掌南京都察院事,就改工部尚書。六年大計京官,拾遺被劾,致仕。後數年卒。

伍文定,字時泰,松滋人。父琇,貴州參議。文定登弘治十二年進士。有膂力,便弓馬,議論慷慨。授常州推官,精敏善決獄,稱彊吏。魏國公徐俌與民爭田,文定勘,歸之民。劉瑾入俌重賄,興大獄,巡撫艾樸以下十四人悉被逮。文定已遷成都同知,亦下詔獄,斥為民。瑾敗,起補嘉興。

江西姚源賊王浩八等流劫浙江開化,都御史俞諫檄文定與參將李隆、都指揮江洪、僉事儲珊討之,軍華埠。而都指揮白弘與湖州知府黃衷別營馬金。賊黨劉昌三破,執弘,官軍大挫。浩八突華埠,洪、文定擊敗之,追及於孔埠。隆、珊亦追至池淮,破其巢,進攻淫田。洪以奇兵深入,中賊誘,與指揮張琳等皆被執。文定等殿後得還,賊亦遁歸江西。諫等上文定忠勇狀,詔所司獎勞。擢河南知府,計擒劇賊張勇、李文簡。以才任治劇,調吉安。計平永豐及大茅山賊。已,佐巡撫王守仁平桶岡、橫水。宸濠反,吉安士民爭亡匿。文定斬亡者一人,眾乃定。乃迎守仁入城。知府邢珣、徐璉、戴德孺等先後至,共討賊。文定當大帥。丙辰之戰,身犯矢石、火燎須不動。賊平,功最,擢江西按察使。張忠、許泰至南昌,欲冒其功,而守仁已俘宸濠赴浙江。忠等失望,大恨。文定出謁,遂縛之。文定罵曰:「吾不恤九族,為國家平大賊,何罪?汝天子腹心,屈辱忠義,為逆賊報仇,法當斬。」忠益怒,椎文定仆地。文定求解任,不報。

尋遷廣東右布政使。未赴,而世宗嗣位。上忠等罪狀,且曰:「曩忠、泰與劉暉至江西,忠自稱天子弟,暉稱天子兒,泰稱威武副將軍,與天子同僚。折辱命吏,誣害良民。需求萬端,漁獵盈百萬。致餓殍遍野,盜賊縱橫。雖寸斬三人,不足謝江西百姓。今大憝江彬、錢寧皆已伏法,三人實其黨與。乞速正天誅,用章國典。」又請發宸濠資財,還之江西,以資經費;矜釋忠、泰所陷無辜及寧府宗人不預謀者,以清冤獄。帝並嘉納之。論功,進右副都御史,提督操江。嘉靖三年,討獲海賊董效等二百餘人,賜敕獎勞。尋謝病歸。

六年召拜兵部右侍郎。其冬擢右都御史,代胡世寧掌院事。雲南士酋安銓反,敗參政黃昭道,攻陷尋甸、嵩明。明年,武定土酋鳳朝文亦反,殺同知以下官,與銓合兵圍雲南。詔進文定兵部尚書兼前職,提督雲南、四川、貴州、湖廣軍討之,以侍郎梁材督餉。會芒部叛酋沙保子普奴為亂,并以屬文定。文定未至雲南,銓等已為巡撫歐陽重所破,遂移師征普奴。左都御史李承勛極言川、貴殘破,不當用兵,遂召還,命提督京營。文定至湖廣,疏乞省祭歸。已,四川巡按御史戴金復上言:「叛酋稱亂之初,勢尚可撫。而文定決意進兵,一無顧惜。飛芻挽糧,糜數十萬。及有詔罷師,尚不肯已。又極論土酋阿濟等罪。軍民訛言,幾復生變。臣愚以為文定可罪也。」尚書方獻夫、李承勛因詆文定好大喜功,傷財動眾,乃令致仕。

文定忠義自許,遇事敢為,不與時俯仰。芒部之役,憤小醜數亂,欲為國伸威,為議者旁撓。廟堂專務姑息,以故功不克就。九年七月卒於家。天啟初,追謚忠襄。

邢珣,當塗人,弘治六年進士。正德初,歷官南京戶部郎中。忤劉瑾,除名。瑾誅,起南京工部,遷贛州知府。招降劇盜滿總等,授廬給田,撫之甚厚。後討他盜,多藉其力。守仁征橫水、桶岡,珣常為軍鋒。功最,增二秩。宸濠反,以重賞誘總。總執其使送珣,遂從珣共平宸濠。

徐璉,朝邑人。文定同年進士。由戶部郎中出為袁州知府。從討宸濠,獲首功千餘。事定,珣、璉遷江西右參政。世宗錄功,各增秩二等。嘉靖二年大計,給事御史劾監司不職者二十二人,珣、璉與焉。吏部以軍功未酬,請進秩布政使致仕,從之。二人竟廢。

珣子埴嘗學於張璁。嘉靖初登鄉薦。璁貴顯,屢欲援之,辭不應。授浦城知縣。有徐浦者,役公府。埴一見異之,令與子同學,為娶妻。後登第為給事中。其家世世祀埴。弟址,進士,歷御史,終山東鹽運使。以清操聞。

戴德孺,臨海人。弘治十八年進士。歷工部員外郎。監蕪湖稅,有清名。再遷臨江知府。宸濠反,遣使收府印,德孺斬之。與家人誓曰:「吾死守孤城。脫有急,若輩沉池中,吾不負國也。」即日戒嚴。旋與守仁共滅宸濠。以憂去。世宗以德孺馭軍最整,獨增三秩,為雲南右布政使。舟次徐州,覆水死。後贈光祿寺卿,予一子官。

珣、璉等倡義討賊,月餘成大功。當事者以嫉守仁故,痛裁抑之。或賞或否,又往往借考功法逐之去。守仁之再疏辭爵也,為諸人訟曰:

宸濠變初起,勢焰猖熾,人心疑懼退阻。當時首從義師,自伍文定、邢珣、徐璉、戴德孺諸人外,又有知府陳槐、曾璵、胡堯元等,知縣劉源清、馬津、傅南喬、李美、李楫及楊材、王冕、顧佖、劉守緒、王軾等,鄉官都御史王懋中,編修鄒守益,御史張鰲山、伍希儒、謝源等。或摧鋒陷陣,或遮邀伏擊,或贊畫謀議,監錄經紀,所謂同功一體者也。帳下之士,若聽選官雷濟,已故義官蕭禹,致仕縣丞龍光,指揮高睿,千戶王佐等,或詐為兵檄以撓其進止,壞其事機,或偽書反間以離其心腹,散其黨與。今聞紀功文冊,改造者多所刪削。舉人冀元亨為臣勸說寧王,反為奸人構陷,竟死獄中,尤傷心慘目,負之冥冥之中者。

夫宸濠積威凌劫,雖在數千里外,無不震駭失措。而況江西諸郡縣切近剝床,觸目皆賊兵,隨處有賊黨,非真有捐軀赴難之義,戮力報主之忠,孰肯甘齏粉之禍,從赤族之誅,蹈必死之地,以希萬一難冀之功乎!

今臣獨崇封爵,而此同事諸人者,或賞不行而並削其績,或賞未及而罰已先行,或虛受升職之名而因使退閑,或冒蒙不忠之號而隨以廢斥。非獨為已斥諸權奸所誣構挫辱而已也。群憎眾嫉,惟事指摘搜羅以為快,曾未見有鳴其不平、伸其屈抑者,臣竊痛之。

奏入,卒寢不行。

蔡天祐,字成之,睢州人。父晟,濟南知府,以廉惠聞。天祐登弘治十八年進士,改庶吉士,授吏科給事中,出為福建僉事。歷山東副使,分巡遼陽。歲歉,活饑民萬餘。闢濱海圩田數萬頃,民名之曰「蔡公田」。累遷山西按察使。

嘉靖三年,大同兵亂,巡撫張文錦遇害。詔曲赦亂卒,改巡撫宣府都御史李鐸撫之。鐸以母憂不至,乃擢天祐右僉都御史,巡撫大同。天祐從數騎馳入城,諭軍士獻首惡,眾心稍定。會尚書金獻民、總兵官杭雄出師甘肅,道大同,亂卒疑見討,復鼓噪。天祐懼,急請再赦。兵部言「元惡不除無以警後」。請特遣大臣總督宣、大軍務,以制其變。乃命戶部侍郎胡瓚偕都督魯綱統京軍三千人以往。瓚等未發而進士李枝齎餉銀至。亂卒曰:「此承密詔盡殺大同人,為軍犒也。」夜中火起,圍枝館,出牒示之乃解。尋復殺知縣王文昌,圍代王府,脅王奏乞赦。王急攜二郡王走宣府。巡按御史王官言:「亂卒方囂,大兵壓境,是趣之叛也。請亟止禁軍,容臣密圖。」乃命瓚駐兵宣府。頃之,天祐奏總兵官桂勇已捕五十四人,請止京軍勿遣。帝責以阻撓,令必獲首惡郭鑑等。既而瓚次陽和,勇、天祐令千戶苗登擒斬鑒等十一人,函首送瓚,請班師。

甫二日,鑑父郭疤子復糾徐氈兒等夜殺勇家人,又毀苗登家。瓚言非盡殲不可。帝乃切讓天祐,召勇還京,以故總兵朱振代之,敕瓚仍駐宣府。居無何,天祐捕戮徐氈兒等,瓚等遂班師。明年正月,侍郎李昆、孟春,總兵官馬永交章言疤子潛逃塞外,必為後患。帝將遣使勘,會瓚還京言逃卒無足患,帝乃罷勘官勿遣。疤子復潛入城,焚振第。明旦,天祐閉城大索。獲疤子及其黨三十四人,悉斬以徇。盡宥脅從,人心乃大定。事聞,賚銀幣。已,進副都御史,巡撫如故。

尋就進兵部右侍郎。久之,召還部。天祐以籓祿久缺,又歲當繕邊垣,用便宜增淮鹽引價,每引萬加銀五千,被訐。帝宥之。至是,御史李宗樞復追論前事,天祐因引疾去。居二年,奉詔起用。未至京,得疾告歸,卒。年九十五。

天祐有才智。兵變時。左右皆賊耳目,幕府動靜悉知之。天祐廣招星卜藝士往來軍中,因具得其情,卒賴以成功。在鎮七年,威德大著,父老為立安輯祠。

胡瓚,字伯珩,永平人。進士。官終南京工部尚書。

張文錦,安丘人。弘治十二年進士,授戶部主事。正德初,為劉瑾所陷,逮繫詔獄,斥為民。瑾誅,起故官。再遷郎中。督稅陜西,條上籌邊裕民十事。遷安慶知府。度寧王宸濠必反,與都指揮楊銳為禦備計。宸濠果反,浮江下。文錦等慮其攻南都,令軍士登城詬之。宸濠乃留攻,卒不能克。事具《楊銳傳》。璽書褒美,擢太僕少卿。嘉靖元年,拜右副都御史,巡撫大同。文錦性剛。以拒賊得重名,遂銳意振刷,操切頗無序。大同北四望平衍,寇至無可禦。文錦曰:「寇犯宣府不能近鎮城者,以葛谷、白陽諸堡為外蔽也。今城外即戰場,何以示重?」議於城北九十里外,增設五堡,曰水口、宣寧、只河、柳溝、樺溝。參將賈鑑督役嚴,卒已怨。及堡成,欲徙鎮卒二千五百家戍之。眾憚行,請募新丁,僚吏咸以為言。文錦怒曰:「如此,則令不行矣。鎮親兵先往,孰敢後!」親兵素游惰有室。聞當發,大恐。請孑身往,得分番。又不聽,嚴趣之。鑑承風,杖其隊長。諸邊卒自甘州五衛殺巡撫許銘,朝廷處之輕,頗無忌。至是,卒郭鑑、柳忠等乘眾憤,遂倡亂。殺賈鑒,裂其屍,走出塞,屯焦山墩。文錦恐與外寇連,令副將時陳等招之入城,即索治首亂者。郭鑑等大懼,復聚為亂,焚大同府門,入行都司縱獄囚,又焚都御史府門。文錦踰垣走,匿博野王府第。亂卒欲燔王宮。王懼,出文錦。郭鑒等殺之,亦裂其屍,遂焚鎮守總兵公署。出故總兵朱振於獄,脅為帥。時嘉靖三年八月也。事聞,帝命侍郎李昆赦亂卒。昆為文錦請恤典,不報,久之,文錦父政訟其子守安慶功,禮部為之請,終不許。文錦妻李氏復上疏哀請。帝怒,命執齎疏者治之。副都御史陳洪謨言:「文錦僨事,朝廷戮之可也。假手士卒,傳之四方,損國威不小。」復降旨詰責。自是,廷臣不敢言。萬曆中,始贈右都御史。天啟初,追謚忠愍。

詹榮,字仁甫,山海衛人。嘉靖五年進士。授戶部主事,歷郎中。

督餉大同,值兵變,殺總兵官李瑾。總督劉源清率師圍城,久不下。榮素有智略,善應變。叛卒掠城中,無犯榮者。外圍益急,榮密約都指揮紀振、游擊戴濂、鎮撫王寧同盟討賊。察叛卒馬昇、楊麟無逆志,乃陽令寧持官民狀詣源清所,為叛卒乞原,而陰以榮謀告,請宥昇、麟死,畀三千金,俾募死士自效。會源清已罷,巡撫樊繼祖許之。升、麟遂結心腹,擒首惡黃鎮等九人戮之。榮乃開城門,延繼祖入,復捕斬二十六人。錄功,擢光祿寺少卿,再遷太常寺少卿。

二十二年,以右僉都御史巡撫甘肅。魯迷貢使留甘州者九十餘人,總兵官楊信驅以御寇,死者十之一。榮言:「彼以好來,而用之鋒鏑,失遠人心,且示中國弱。」詔奪信官,槥死者送之歸。番人感悅。踰年,以大同巡撫趙錦與總兵官周尚文不相能,詔榮與錦易任。俺答數萬騎入掠,榮與尚文破之黑山陽,進右副都御史。寇復大舉犯中路,參將張鳳等陣歿。榮與尚文及總督翁萬達嚴兵備陽和,而遣騎邀擊,多所殺傷,寇乃引去。代府奉國將軍充灼行剽,榮奏奪其祿。充灼等結小王子入寇,謀據大同。榮告尚文捕得,皆伏辜。榮以大同無險,乃築東路邊牆百三十八里,堡七,墩臺百五十四。又以守邊當積粟。而近邊弘賜諸堡三十一所,延亙五百餘里,闢治之皆膏腴田,可數十萬頃。乃奏請召軍佃作,復其租徭,移大同一歲市馬費市牛賦之;秋冬則聚而遏寇。帝立從焉。寇入犯,與尚文破之彌陀山,斬一部長。

榮先以靖亂功,進兵部右侍郎,又以繕邊破敵,累被獎賚。召還理部事,進左。尚書趙廷瑞罷,榮署部務,奏行秋防十事。已而翁萬達入為尚書,遭母喪,榮復當署部務,辭疾乞休。帝怒,奪職閒住。越二年卒。

當榮之撫大同也,萬達為總督,尚文為總兵。三人皆有才略,寇數入不能得志。自後代者不能任,寇無歲不入躪邊,人益思榮等。明年,俺答薄京師,萬達、榮皆已去。論者謂二人在,寇未必至此。萬曆中,榮孫延為順天通判,上書訟榮功。贈工部尚書,予恤如制。

劉源清,字汝澄,東平人。正德九年進士。授進賢知縣。

宸濠反,源清積薪環室,命家人曰:「事急,火吾家。」一僕逸,手刃以徇。縣中諸惡少與賊通者,悉杖殺之。宸濠妃弟婁伯歸上饒募兵,源清邀戮之。賊檄至,立斬其使。會餘干知縣馬津、龍津驛丞孫天祐亦起兵拒賊。賊七殿下者,奪運舟於龍津,天祐與戰,殺數人。賊黨募兵過龍津,天祐追殺之,焚其舟。婁氏家眾西下,亦為天祐所遏,擒七十餘人。賊兵不敢經湖東以窺兩浙者,三人力也。賊平,源清徵為御史。嘉靖改元,津亦入為御史。津,滁州人。終福建副使。源清尋遷大理丞,謝病歸。

六年夏,以右僉都御史巡撫宣府。滴水崖賊郭春據城叛,稱王。源清遣卒捕之,為所覺。副總兵劉淵令曰「止擒元惡」,以旗繞城而呼。其黨皆散,春等自剄死。總兵官郤永虐下,源清劾罷之。進副都御史。十二年,以邊警遷兵部左侍郎,總制宣、大、山西、保定諸鎮軍務。大同總兵官李瑾浚天城左孤店濠四十里,趣工急。卒王福勝等焚殺瑾,因焚巡撫潘人放署。人放奏瑾激變,帝命源清同總兵郤永討之。源清榜令解散。而榜言五堡變,處之過寬,五堡遺孽大懼。師次陽和,人放等密捕亂卒杖死十餘人,繫賊首王保等七十餘人以獻,請旋師。源清懲昔胡瓚事,不欲已,以囚屬御史蘇祐。囚妄言前總兵朱振失職首亂,且多引無辜。源清遣參將趙綱入城大索。城中訛言城且屠,亂卒遂鼓噪,殺千戶張欽。會僉事孫允中自源清所至,諭源清意,撫慰之始定。振前為亂卒所擁,實不反,詣源清自明。不能白,發憤自殺。

永兵至城下大掠,五堡遺孽遂盡反。迎戰,殺遊擊曹安。官軍攻據四關,晝夜圍擊。亂卒出前參將黃鎮等於獄,奉為帥,死守。人放與鎮國將軍俊隱等登城,止毋攻。俊隱出見永請緩兵,皆不聽。允中縋城出,言將士妄殺狀。源清叱曰:「汝為賊游說耶!」欲囚之。允中不敢歸。源清因多設邏卒,遏王府及有司軍民章疏,而請益師至五萬。帝命侍郎錢如京、都督江桓統京軍八千以往。已忽悟,罷弗遣,專責源清、永討賊。人放馳疏言,將士妄殺激變,速旋師,亂中已。源清亦詆人放媚賊。張孚敬主源清,侍郎顧鼎臣、黃綰言用兵謬,帝不能決。城圍久大困,毀王府及諸廨舍供爨。兵部復下安撫令,源清亦樹幟招降,叛卒稍稍自投。首惡黃鎮等亦分日出見,乞通樵採路,永許諾。翌日採薪者出,永悉執之。城中人益懼,亂卒復叛,勾外寇為助。永遇之,大敗而遁。叛卒遂引寇十餘騎入城,指代府曰:「以此為那顏居。」「那顏」者,華言大人也。城中人聞之,皆巷哭。明日,外寇攻東南二關,叛卒與犄角,官軍殊死戰,互有殺傷。寇知叛卒不足賴,倒戈擊之,大詬而去。是時,寇游騎南掠至朔、應。源清請募九邊兵,增總制官禦之,己得一意攻城,帝不許。源清乃百道攻,穴城,為毒煙熏死者相籍。復請壅水灌之。帝大不懌,奪其職閒住,以兵部侍郎張瓚代之。瓚未至,郎中詹榮等已悉捕首惡。

黃綰勘功罪,言源清、永實罪魁,具劾其婪賄不貲狀。兵科曾忭等言,宸濠亂,源清有保障功,當蒙八議之貸。帝怒,下忭等詔獄,逮源清治之。獄久不決,綰憂去,乃減死,斥為民。俺答薄京師,即家起之,未赴而卒。隆慶初,贈兵部尚書。

劉天和,字養和,麻城人。正德三年進士。授南京禮部主事。劉瑾黜御史十八人,改他曹二十四人補之,天和與焉。出按陜西。鎮守中官廖堂奉詔辦食御物於蘭州,天和謂非所部,辭不往。堂奏天和拒命,詔逮之。部民哭送者萬人。錮詔獄久不釋,吏部尚書楊一清疏救,法司奏當贖杖還職,中旨謫金壇丞。刑部主事孫繼芳抗章救,不報。屢遷湖州知府,多惠政。

嘉靖初,擢山西提學副使。累遷南京太常少卿。以右僉都御史督甘肅屯政。請以肅州丁壯及山、陜流民於近邊耕牧,且推行於諸邊。尋奏當興革者十事,田利大興。改撫陜西。請撤鎮守中官及罷為民患者三十餘事,帝皆從之。洮、岷番四十二族蠢動,天和誅不順命者。又討平湖店大盜及漢中妖賊,就進右副都御史。

母憂,服闋以故官總理河道。黃河南徙,歷濟、徐皆旁溢。天和疏汴河,自朱仙鎮至沛飛雲橋,殺其下流。疏山東七十二泉,自鳧、尼諸山達南旺河,浚其下流。役夫二萬,不三月訖工。加工部右侍郎。故事,河南八府歲役民治河,不赴役者人出銀三兩。天和因歲饑,請盡蠲旁河受役者課,遠河未役者半之。詔可。

十五年改兵部左侍郎,總制三邊軍務。兵車皆雙輪,用二十人,遇險即困,又行遲不適於用。天和請仿前總督秦紘隻輪車,上置砲槍斧戟,廂前樹狻猊牌,左右虎盾,連二車可蔽三四十人。一人挽之,推且翼者各二人。戰則護騎士其中,敵遠則施火器,稍近發弓弩,又近乃出短兵,敵走則騎兵追。復製隨車小帳,令士不露宿。又毒弩矢,修邊牆濠塹。皆從之。

吉囊十萬眾屯賀蘭山後,遣別部寇涼州,副將王輔逐奪其纛。寇莊浪,總兵官姜奭屢敗之。進天和右都御史。寇復大集兵將入犯。天和策寇瞰西有備必東,密檄延綏副將白爵宵行,與參將吳瑛合。寇果東入黑河墩,遇爵伏兵,大創而去。既又入蒺藜川,爵尾擊之,寇多死。尋入寇家澗、張家塔,為爵、瑛所敗。犯寧夏者,總兵官王效復破之。帝大喜,進天和左都御史。吉囊犯河西,天和禦卻之,進兵部尚書。寇將入平虜城,天和伏兵花馬池。寇戰不勝,走河上。遇伏兵,多死於水。吉囊乘虛寇固原,剽掠且饜。會淫潦,弓矢盡膠,無鬥志。而諸將多畏縮,天和斬指揮二人,召故總兵周尚文令立功。會陜西總兵官魏時角寇至黑水苑,尚文盡銳夾擊,殺吉囊子小十王。寇退寧夏,巡撫楊守禮、總兵官任傑等復邀擊,敗之鐵柱泉,斬獲共四百四十餘級。論功,加天和太子太保,廕一子錦衣千戶,前後齎銀幣十數。遷南京戶部尚書,召為兵部尚書督團營。言官論天和衰老,遂乞休歸。家居三年卒。贈少保,謚莊襄。

天和初舉進士,劉瑾欲與敘宗姓,謝不往。晚年內召,陶仲文以刺迎,稱戚屬。天和返其刺曰:「誤矣,吾中外姻連無是人。」仲文恚,其罷官有力焉。

楊守禮,字秉節,蒲州人。正德六年進士。除戶部主事。嘉靖初,屢遷湖廣僉事。以計擒公安賊魁。坐事謫敘州通判。累遷右副都御史,巡撫四川。與副將何卿平諸番亂,齎銀幣。初,守禮貶敘州,為僉事張文奎所辱。至是,文奎遷四川參議,恐守禮修隙,先以所摭事奏。詔二人俱解職歸。

守禮才器敏達,中外以為能。居家未久,工部尚書秦金等會薦,起河南參政。再遷右副都御史,巡撫寧夏。寇犯固原,為總督劉天和所敗。欲自寧夏去,守禮與總兵任傑等邀敗之。會天和召還,進守禮右都御史總督軍務代之。錄前功,進兵部尚書。總兵官李義、楊信連卻吉囊,三賜璽書銀幣。尋上疏乞休,帝惡其避難,降俸二級。

其秋,寇三萬騎抵綏德。游擊張鵬卻之,總兵官吳英等追至塞外,東路參將周文兵亦至,夾擊敗之。巡按御史殷學言,寇入內地五百里,請治諸將罪。部議延綏游兵俱調宣、大,寇方避實擊虛。而我能以寡勝眾,宜錄其功。乃加守禮太子少保,學謫外。守禮尋以憂去。俺答薄都城,廷臣首以守禮薦,詔趣上道。寇退,止不行。久之卒。

張岳,字維喬,惠安人。自幼好學,以大儒自期。登正德十一年進士,授行人。武宗寢疾豹房。請令大臣侍從,臺諫輪直起居,視藥餌,防意外變。不報。與同官諫南巡,杖闕下,謫南京國子學正。世宗嗣位,復故官,遷右司副。母老乞便養,改南京武選員外郎,歷主客郎中。方議大禘禮。張璁求始祖所自出者實之,禮官皆唯唯。岳言於尚書李時曰:「不如為皇初祖位,毋實以人。」時大喜,告璁。璁不謂然,以初議上。帝竟令題皇初祖主,如岳言。璁銜之,出為廣西提學僉事。行部柳州,軍缺餉大譁,城閉五日。岳令守城啟門,召詰嘩者予餉去。尋以計擒首惡,置之理。入賀,改提學江西。不謝璁,璁黜廣西選貢七人,謫岳廣東鹽課提舉。遷廉州知府。督民墾棄地,教以桔槔運水。廉民多盜珠池。岳居四年,未嘗入一珠。

帝使使往安南詰莫登庸殺主,岳言於總督張經曰:「莫氏篡黎,可無勘而知也,使往受謾詞辱國,請留使者毋前。」經不可。知欽州林希元上書請決討莫氏,岳貽書止之,復條上不可討六事。為書貽執政曰:「據邊民報,黎賙襲封無嗣,以兄子譓為子。陳暠作亂,賙遇害,暠篡。未幾國人擁立譓,暠奔諒山。言惠立七年,為莫登庸所逼,出居升華。登庸立譓幼弟騑而相之,卒弒騑自立,國分為三。黎在南,莫居中,陳在西北。後諒山亦為登庸有,陳遂絕。而黎所居即古日南地,與占城鄰,限大海,登庸不能踰之南,故兩存。近登庸又以交州付其孫福海,而自營海東府地都齋居之。蓋安南諸府,惟海東地最大,即所謂王山郡也。此賊負篡逆名,常練兵備我,又時揚言求入貢。邊人以非故王也,弗敢聞。愚以為彼內亂未嘗有所侵犯,可且置之,待其亂定乃貢。若必用兵,勝負利純非岳所敢知。」執政得書不能決。已,毛伯溫來視師,張經一以軍事委岳。又以翁萬達才,進二人於伯溫。岳與伯溫語數日,伯溫曰:「交事屬君矣。」許登庸如岳議。會岳遷浙江提學副使,又遷參政,伯溫馳奏留之,乃改廣東參政,分守海北。登庸降,加岳俸一級,賜銀幣。尋以征瓊州叛黎功,加俸及賜如之。

塞上多事,言官薦岳邊才。伯溫言:「岳可南,翁萬達可北也。」遂擢岳右僉都御史,撫治鄖陽。旋移撫江西,進右副都御史,總督兩廣軍務兼巡撫。討破廣東封川僮蘇公樂等,進兵部右侍郎。平廣西馬平諸縣瑤賊,先後俘斬四千,招撫二萬餘人,誅賊魁韋金田等,增俸一級。召為刑部右侍郎,以御史徐南金言命留任。連山賊李金與賀縣賊倪仲亮等,出沒衡、永、郴、桂,積三十年不能平,岳大合兵討擒之。蒞鎮四年,巨寇悉平,召拜兵部左侍郎。

湖貴間有山曰蠟爾,諸苗居之。東屬鎮溪千戶所筸子坪長官司,隸湖廣;西屬銅仁、平頭二長官司,隸貴州;北接四川酉陽,廣袤數百里。諸苗數反,官兵不能制。侍郎萬鏜征之,四年不克。乃授其魁龍許保冠帶。湖苗暫息,而貴苗反如故。鏜班師,龍許保及其黨吳黑苗復亂。貴州巡撫李義壯告警,乃命岳總督湖廣、貴州、四川軍務,討之。進右都御史。義壯持鏜議欲撫,岳劾其阻兵,罷之。先義壯撫貴州者,僉都御史王學益與鏜附嚴嵩,主撫議,數從中撓岳。岳持益堅。許保襲執印江知縣徐文伯及石阡推官鄧本忠以去,岳坐停俸。乃使總兵官沈希儀、參將石邦憲等分道進,躬入銅仁督之。先後斬賊魁五十三人,獨許保、黑苗跳不獲。岳以捷聞,言貴苗漸平,湖苗聽撫,請遣土兵歸農,朝議許之。未幾,酉陽宣慰冉元嗾許保、黑苗突思州,劫執知府李允簡。邦憲兵邀奪允簡還,允簡竟死。嵩父子故憾岳,欲逮治之,徐階持不可。乃奪右都御史,以兵部侍郎督師。邦憲等旋破賊。岳搜山箐,餘賊獻思州印及許保。湖廣兵亦破擒首惡李通海等。岳以黑苗未獲,不敢報功。已而冉元謀露,岳發其奸。元賄嚴世蕃責岳絕苗黨。邦憲竟得黑苗以獻,苗患乃息。

岳卒於沅州。喪歸,沅人迎哭者不絕。已,敘功,復右都御史,贈太子少保,謚襄惠。

岳博覽,工文章,經術湛深,不喜王守仁學,以程、朱為宗。

李允簡,融縣人。由舉人起家。以郡境多寇,道孥歸,獨與孫炳文居。祖孫皆被執,許保挾以求厚贖。允簡則傳語邦憲令亟進兵。在賊中自投高崖下,賊拽出,棄之途。思人舁還,至清浪衛而卒。詔贈貴州副使,賜祭葬,官一子。

郭宗皋,字君弼,福山人。嘉靖八年進士。選庶吉士。尋詔與選者皆改除,得刑部主事。擢御史。十二年十月,星隕如雨。未幾,哀沖太子薨,大同兵亂。宗皋勸帝惇崇寬厚,察納忠言,勿專以嚴明為治。帝大怒,下詔獄,杖四十釋之。歷按蘇、松、順天。行部乘馬,不御肩輿。會廷推保定巡撫劉夔還理院事,宗皋論夔嘗薦大學士李時子,諂媚無行,不任風紀,坐奪俸兩月。尋出為雁門兵備副使,轉陜西參政,遷大理少卿。

二十三年十月,寇入萬全右衛,抵廣昌,列營四十里。順天巡撫朱方下獄,擢宗皋右僉都御史代之,寇已去。宗皋言:「密雲最要害,宜宿重兵。乞敕馬蘭、太平、燕河三屯歲發千人,以五月赴密雲,有警則總兵官自將赴援。居庸、白楊,地要兵弱,遇警必待部奏,不能及事。請預擬借調之法,令建昌三屯軍,平時則協助密雲,遇警則移駐居庸。」俱報可。久之,宗皋聞敵騎四十萬欲分道入,奏調京營、山東、河南兵為援。已竟無實,坐奪俸一年。故事,京營歲發五軍詣薊鎮防秋。宗皋請罷三軍,以其犒軍銀充本鎮募兵費。又請發修邊餘銀,增築燕河營、古北口。帝疑有侵冒,令罷歸聽勘。既而事得白。起故官,巡撫大同,與宣府巡撫李仁易鎮。

尋進兵部右侍郎,總督宣、大、山西軍務。俺答三萬騎犯萬全左衛,總兵官陳鳳、副總兵林椿與戰鷂兒嶺,殺傷相當,宗皋坐奪俸。明年再犯大同,總兵官張達及椿皆戰死,宗皋與巡撫陳耀坐奪俸。給事中唐禹追論死事狀,因言全軍悉陷,乃數十年未有之大衄。帝乃逮宗皋及耀,各杖一百,燿遂死,宗皋戍陜西靖虜衛。

隆慶改元,從戍所起刑部右侍郎,改兵部,協理戎政。旋進南京右都御史,就改兵部尚書參贊機務。給事中莊國禎劾宗皋衰庸,宗皋亦自以年老求去,詔許之。萬曆中,再存問,歲給廩隸。十六年,宗皋年九十,又遣行人存問。是年卒。贈太子太保,謚康介。

趙時春,字景仁,平涼人。幼與群兒嬉,輒列旂幟,部勒如兵法。年十四舉於鄉。踰四年,為嘉靖五年,會試第一。選庶吉士。以張璁言改官,得戶部主事。尋轉兵部。九年七月,上疏曰:「陛下以災變求言已旬月,大小臣工率浮詞面謾。蓋自靈寶知縣言河清受賞,都御史汪鋐繼進甘露,今副都御史徐贊、訓導范仲斌進瑞麥,指揮張楫進嘉禾,鋐及御史楊東又進鹽華,禮部尚書李時再請表賀。仲斌等不足道,鋐、贊司風紀,時典三禮,乃罔上欺君,壞風傷政。」帝責其妄言,且令獻讜言善策。時春惶恐引咎未對。帝趣之,於是時春上言:

當今之務最大者有四,最急者有三。最大者,曰崇治本。君之喜怒,賞罰所自出,勿以逆心事為可怒,則賞罰大公而天下治。曰信號令。無信一人之言,必參諸公論;毋狃一時之近,必稽之永遠。茍利十而害一,則利不必興;功百而費半,則功不必舉。如是而天下享安靜之福矣。曰廣延訪。宜仿古人輪對及我朝宣召之制,使大臣、臺諫、侍從各得敷納殿陛間,群吏則以其職事召問之。曰勵廉恥。大臣宜待以禮,取大節略小過。臺諫言是者用之,非者寬容之。庶臣工自愛,不敢不勵。

其最急者,曰惜人才。凡得罪諸臣,其才不當棄,其過或可原,宜霈然發命,召還故秩。且因南郊禮成,除謫戍之罪,與之更始。曰固邊圉。敗軍之律宜嚴,臨陣而退者,裨將得以戮士卒,大將得以戮裨將,總制官得以戮大將,則人心震悚,而所向用命。曰正治教。請復古冠婚、喪祭之禮,絕醮祭、禱祀之術。凡佛老之徒有假引符籙、依託經懺、幻化黃白、飛昇遐景以冒寵祿者,即賜遣斥,則正道修明而民志定。

帝覽之,益怒,下詔獄掠治,黜為民。久之,選東宮官屬,起翰林編修兼司經局校書。

帝有疾,時春與羅洪先、唐順之疏請東宮御殿,受百官正旦朝賀。帝大怒,復黜為民。京師被寇,朝議以時春知兵,起兵部主事,贊理京營務,統民兵訓練。大將軍仇鸞倡馬市,時春憤曰:「此秦檜續耳。身為大將,而效市儈,可乎?」忤鸞,為所構,幾重得罪。稍遷山東僉事,進副使。

三十二年,擢僉都御史,巡撫山西。時春慷慨負奇氣,善騎射。慨寇縱橫,將帥不任職,數謂人:「使吾領選卒五千,俺答、邱福不足平也。」作《禦寇論》,論戰守甚悉。既秉節鉞,益思以武功自奮。其年九月,寇入神池、利民諸堡,時春率馬步兵往禦之。至廣武,諸將畢會。謀報寇騎二千餘,去兩舍。時春擐甲欲馳,大將李淶固止之。時春大言曰:「賊知吾來必遁,緩追即不及。」遂策馬前。及於大蟲嶺,伏兵四起,敗績。倉皇投一墩,守卒縋之上乃得免,淶軍竟覆。被論,解官聽調。時春喜談兵,至是一戰而敗。然當是時將帥率避寇不擊。為督撫者安居堅城,遙領軍事,無躬搏寇者。時春功雖不就,天下皆壯其氣。

時春讀書善強記,文章豪肆,與唐順之、王慎中齊名。詩,伉浪自喜類其為人。

贊曰:姚鏌等封疆宣其擘畫,軍務暢其機謀,勛績咸有可紀。伍文定從王守仁平宸濠之難,厥功最懋。趙時春將略自命,一出輒躓。夫危事而易言之,固知兵者所弗取乎。


\end{pinyinscope}