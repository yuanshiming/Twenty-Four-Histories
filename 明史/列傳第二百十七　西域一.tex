\article{列傳第二百十七 西域一}

\begin{pinyinscope}
○哈密衛柳城火州土魯番

哈密,東去嘉峪關一千六百里,漢伊吾盧地。明帝置宜禾都尉,領屯田。唐為伊州。宋入於回紇。元末以威武王納忽里鎮之,尋改為肅王,卒,弟安克帖木兒嗣。

洪武中,太祖既定畏兀兒地,置安定等衛,漸逼哈密。安克帖木兒懼,將納款。

成祖初,遣官招諭之,許其以馬市易,即遣使來朝,貢馬百九十匹。永樂元年十一月至京,帝喜,賜賚有加,命有司給直收其馬四千七百四十匹,擇良者十匹入內廄,餘以給守邊騎士。

明年六月復貢,請封,乃封為忠順王,賜金印,復貢馬謝恩。已而迤北可汗鬼力赤毒死之,其國人以病卒聞。三年二月遣官賜祭,以其兄子脫脫為王,賜玉帶。脫脫自幼俘入中國,帝拔之奴隸中,俾列宿衛,欲令嗣爵。恐其國不從,遣官問之,不敢違,請還主其眾。因賜其祖母及母彩幣,旋遣使貢馬謝恩。

四年春,甘肅總兵官宋晟奏,脫脫為祖母所逐。帝怒,敕責其頭目曰:「脫脫朝廷所立,即有過,不奏而擅逐之,是慢朝廷也。老人昏耄,頭目亦不知朝廷耶?即迎歸,善匡輔,俾孝事祖母。」由是脫脫得還,祖母及頭目各遣使謝罪。三月立哈密衛,以其頭目馬哈麻火者等為指揮、千百戶等官,又以周安為忠順王長史,劉行為紀善,輔導。冬,授頭目十九人為都指揮等官。

明年,宋晟奏,頭目陸十等作亂,已誅,慮他變,請兵防禦。帝命晟發兵應之,而以安克帖木兒妻子往依鬼力赤,恐誘賊侵哈密,敕晟謹備。晟卒,以何福代,又敕福開誠撫忠順。會頭目請設把總一人理國政,帝敕福曰:「置把總,是增一王也;政令不一,下安適從。」寢其議。自是,比歲朝貢,悉加優賜,其使臣皆增秩授官。

帝眷脫脫特厚,而脫脫顧凌侮朝使,沈湎昏聵,不恤國事,其下買柱等交諫不從。帝聞之怒,八年十一月遣官賜敕戒諭之。未至,而脫脫以暴疾卒。訃聞,遣官賜祭。擢都指揮同知哈剌哈納為都督僉事,鎮守其地,賜敕及白金、綵幣。且封脫脫從弟兔力帖木兒為忠義王,賜印誥、玉帶,世守哈密。十年,貢馬謝恩,自是修貢惟謹,故王祖母亦數奉貢。

十七年,帝以朝使往來西域者,忠義王致禮延接,命中官齎綺帛勞之,賜其母妻金珠冠服、彩幣,及其部下頭目。其使臣及境內回回尋貢馬三千五百餘匹及貂皮諸物,詔賜鈔三萬二千錠、綺百、帛一千。二十一年貢駝三百三十、馬千匹。

仁宗踐阼,詔諭其國。洪熙元年再入貢,賀即位。仁宗崩,宣宗繼統,其王兔力帖木兒亦卒,使來告哀。

宣德元年遣官賜祭,命故王脫脫子卜答失里嗣忠順王,且以登極肆赦,命其國中亦赦,復貢馬謝恩。明年遣弟北斗奴等來朝,貢駝馬方物。授北斗奴都督僉事,因命中官諭王,遣故忠義王弟脫歡帖木兒赴京。三年以卜答失里年幼,命脫歡帖木兒嗣忠義王,同理國事。自是,二王並貢,歲或三四至,奏求婚娶禮幣,命悉予之。

正統二年,脫歡帖木兒卒,封其子脫脫塔木兒為忠義王,未幾卒。已而忠順王亦卒,封其子倒瓦答失里為忠順王。五年遣使三貢,廷議以為煩,定令每年一貢。

初,成祖之封忠順王也,以哈密為西域要道,欲其迎護朝使,統領諸番,為西陲屏蔽。而其王率庸懦,又其地種落雜居。一曰回回,一曰畏兀兒,一曰哈剌灰,其頭目不相統屬,王莫能節制。眾心離渙,國勢漸衰。及倒瓦答失里立,都督皮剌納潛通瓦剌猛可卜花等謀殺王,不克。王父在時,納沙州叛亡百餘家,屢敕王令還,止遣其半,其貢使又數辱驛吏卒,呵叱通事,當四方貢使大宴日,惡言詬詈,天子不加罪,但令慎擇使臣,以是益無忌。其地,北瓦剌,西土魯番,東沙州、罕東、赤斤諸衛,悉與構怨。由是鄰國交侵。罕東兵抵城外,掠人畜去。沙州、赤斤先後兵侵,皆大獲。瓦剌酋也先,王母弩溫答失里弟也,亦遣兵圍哈密城,殺頭目,俘男婦,掠牛馬駝不可勝計,取王母及妻北還,脅王往見,王懼不敢往,數遣使告難。敕令諸部修好,迄不從,惟王母妻獲還。

十年,也先復取王母妻及弟,並撒馬兒罕貢使百餘人掠之,又數趣王往見。王外順朝命,實懼也先。十三年夏,親詣瓦剌,居數月方還;而遣使誑天子,謂守朝命不敢往。天子為賜敕褒嘉。已,知其詐,嚴旨詰責,然其王迄不能自振。會也先方東犯,不復還故土,以是哈密獲少安。

景泰三年遣其臣捏列沙朝貢,請授官。先是,使臣至京必加恩命。是時于謙掌中樞,言哈密世受國恩,乃敢交通瓦剌。今雖歸款,心猶譎詐。若加官秩,賞出無名。乃止。終景泰世,使臣無授官者。

天順元年,倒瓦答失里卒,弟卜列革遣使告哀,即封為忠順王。時都指揮馬雲使西域,聞迤北酋加思蘭梗道,不敢進。會哈蜜王報道已通,雲乃行,至哈密。而賊兵實未退,且謀劫朝使。帝疑王與賊通,遣使切責。

四年,王卒,無子,母弩溫答失里主國事。初,也先被誅,其弟伯都王及從子兀忽納走居哈密。王母為上書乞恩,授伯都王都督僉事,兀忽納指揮僉事。自卜列革之亡,親屬無可繼,命國人議當襲者。頭目阿只等言脫歡帖木兒外孫把塔木兒官都督同知,可繼。王母謂臣不可繼君,而安定王阿兒察與忠順王同祖,為請襲封。七年冬,奏上,禮官言:「加思蘭見哈密無主,謀據其地,勢危急,乞從其請。」帝命都指揮賀玉往。至西寧逗遛不進,哈密使臣苦兒魯海牙請先行,又不許。帝逮玉下吏,改命都指揮李珍,而敕安定、罕東護使臣偕往。阿兒察以哈密多難,力辭不行,珍乃返。

哈密素衰微,又婦人主國,眾益離散。加思蘭乘隙襲破其城,大肆殺掠,王母率親屬部落走苦峪,猶數遣使朝貢,且告難。朝廷不能援,但敕其國人速議當繼者而已。其國殘以破故,來者日眾。

成化元年,禮官姚夔等言:「哈密貢馬二百匹,而使人乃二百六十人。以中國有限之財,供外蕃無益之費,非策。」帝下廷臣議,定歲一入貢,不得過二百人,制可。

明年,兵部言王母避苦峪久,今賊兵已退,宜令還故土,從之。已而貢使言其地饑寒,男婦二百餘人隨來丐食,不能歸國。命人給米六斗、布二疋,遣之。

初,國人請立把塔木兒,以王母不肯,無王者八年。至是頭目交章請,詞極哀。乃擢把塔木兒為右都督,攝行國王事,賜之誥印。五年,王母陳老病乞藥物,帝即賜之。尋與瓦刺、土魯番遣使三百餘人來貢,邊臣以聞。廷議貢有定期,今前使未回後使又至,且瓦剌強寇,今乃與哈密偕;非哈密挾其勢以邀利,即瓦剌假其事以窺邊。帝乃卻其獻,令邊臣宴賚,遣還。貢使堅不受賜,必欲親詣闕下,乃命遣十之一赴京。

八年,把塔木兒子罕慎以父卒請嗣職。帝許之,而不命其主國事,國中政令無所出。土魯番速檀阿力乘機襲破其城,執王母,奪金印,以忠順王孫女為妾,據守其地。九年四月,事聞,命邊臣謹戒備,敕罕東、赤斤諸衛協力戰守。尋遣都督同知李文、右通政劉文赴甘肅經略。抵肅州,遣錦衣千戶馬俊奉敕往諭。時阿力留其妹婿牙蘭守哈密,而己攜王母、金印已返土魯番。俊至,諭以朝命,抗詞不遜,羈俊月餘。一日,牙蘭忽至,言大兵三萬即日西來,阿力乃宴勞俊等,舁王母出見。王母懼不敢言,夜潛遣人來云:「為我奏天子,速發兵救哈密。」文等以聞,遂檄都督罕慎及赤斤、罕東、乜克力諸部集兵進討。十年冬,兵至卜隆吉兒川,諜報阿力集眾抗拒,且結別部謀掠罕東、赤斤二衛。文等不敢進,令二衛還守本土,罕慎及乜克力、畏兀兒之眾退居苦峪,文等亦引還肅州。帝乃命罕慎權主國事,因其請給米布,且賜以穀種。文等無功而還。

土魯番久據哈密,朝命邊臣築苦峪城,移哈密衛於其地。十八年春,罕慎糾罕東、赤斤二衛,得兵一千三百人,與己所部共萬人,夜襲哈密城破之,牙蘭遁走;乘勢連復八城,遂還居故土。巡撫王朝遠以聞,帝喜,賜敕獎勵,並獎二衛。朝遠請封罕慎為王,且言土魯番亦革心向化,與罕慎議和,宜乘時安撫,取還王孫女及金印,俾隨王母共掌國事,哈密國人亦乞封罕慎。廷議不從,乃進左都督,賚白金百兩、彩幣十表裏,特敕獎勞,將士升賞有差。

弘治元年從其國人請,封罕慎為忠順王。土魯番阿力已死,而其子阿黑麻嗣為速檀,偽與罕慎結婚,誘而殺之,仍令牙蘭據其地。哈密都指揮阿木郎來奔求救,廷臣請諭土魯番貢使,令復還侵地,並敕赤斤、罕東,共圖興復。明年,哈密舊部綽卜都等率眾攻牙蘭,殺其弟,奪其叛臣者盼卜等人畜以歸。事聞,進秩加賞。先是,罕慎遣使來貢,未還而遘難,其弟奄克孛剌率部眾逃之邊方,朝命以賜罕慎者還賜其弟。阿黑麻之去哈密也,止留六十人佐牙蘭。阿木郎覘其單弱,請邊臣調赤斤、罕東兵,夜襲破其城,牙蘭遁去,斬獲甚多,有詔獎賚。

當是時,阿黑麻桀傲甚,自以地遠中國,屢抗天子命。及破哈密,貢使頻至,朝廷仍善待之,由是益輕中國。帝乃薄其賜賚,或拘留使臣,卻其貢物,敕責令悔罪。己,訪獲惠順王族孫陜巴,將輔立之。阿黑麻漸警懼,三年遣使叩關,願獻還哈密及金印,釋其拘留使臣。天子納其貢,仍留前使者。明年,果以城印來歸,乃從馬文升言,還其所拘使臣。文升又言:「番人重種類,且素服蒙古,哈密故有回回、畏兀兒、哈剌灰三種,北山又有小列禿、乜克力相侵逼,非得蒙古後裔鎮之不可。今安定王族人陜巴,乃故忠義王脫脫近屬從孫,可主哈密。」天子以為然,而諸番亦共奏陜巴當立。五年春立陜巴為忠順王,賜印誥、冠服及守城戎器,擢阿木郎都督僉事,與都督同知奄克孛剌共輔之。

已而諸番索陜巴犒賜不得,皆怨。阿木郎又引乜克力人掠土魯番牛馬,阿黑麻怒,六年春潛兵夜襲哈密,殺其人百餘,逃及降者各半。陜巴與阿木郎據大土剌以守。大土剌,華言大土臺也。圍三日不下。阿木郎急調乜克力、瓦剌二部兵來援,俱敗去。乃執陜巴,擒阿木郎支解之。牙蘭復據守,並移書邊臣訴阿木郎罪。時土魯番先後貢使皆未還。邊臣以其書不遜,且僭稱可汗,乞命將遣兵先剿除牙蘭,然後直抵土魯番,馘阿黑麻之首,取還陜巴。否則降敕嚴責,令還陜巴,乃宥其罪。廷議從後策,令守臣拘貢使,縱數人還,齎敕曉示禍福。帝如其請,命廷推大臣赴甘肅經略。

初,哈密變聞,丘濬謂馬文升曰:「西陲事重,須公一行。」文升曰:「國家有事,臣子義不辭難。然番人嗜利,不善騎射,自古未有西域能為中國患者,徐當靖之。」濬復以為言,文升請行。廷臣僉言北寇強,本兵未可遠出,乃推兵部右侍郎張海、都督同知緱謙二人。帝賜敕指授二人,而二人皆庸才,但遣土魯番人歸諭其主,令獻還侵地,駐甘州待之。明年,阿黑麻遣使叩關求貢,詭言願還陜巴及哈密,乞朝廷亦還其使者。海等以聞,請再降敕宣諭。廷議言,先已降敕,今若再降,有傷國體,宜令海等自遣人往諭。不從命,則仍留前使,且盡驅新使出關,永不許貢,仍與守臣檄罕東、赤斤諸部兵,直搗哈密,襲斬牙蘭。如無機可乘,則封嘉峪關,毋納其使。陜巴雖封王,其還與否,於中國無損益,宜別擇賢者代之。帝以陜巴既與中國無損益,則哈密城池已破,如獻還,當若何處之。廷臣復言陜巴乃安定王千奔之姪,忠順王之孫,向之封王,欲令鎮撫一方爾。今被虜,孱弱可知,即使復還,勢難復立。宜革其王爵,居之甘州,犒賚安定王,諭以不復立之故。令都督奄克孛剌總理哈密事,與回回都督寫亦虎仙,哈剌灰都督拜迭力迷失等分領三種番人以輔之。且修浚苦峪城塹,凡番人散處甘、涼者,令悉還其地,給以牛具口糧。若陜巴未還,不必索取,我不急陜巴,彼將自還也。帝悉如其言,敕諭海等。海等見敕書將棄陜巴,甚喜,即逐其貢使,閉嘉峪關,繕修苦峪城,令流寓番人歸其地,拜疏還朝。八年正月至京,言官交章劾其經略無功,並下吏貶秩,而哈密終不還。

文升銳意謀興復,用許進巡撫甘肅以圖之。進偕大將劉寧等潛師夜襲,牙蘭逸去,斬其遺卒,撫降餘眾而還。自明初以來,官軍無涉其地者,諸番始知畏,阿黑麻亦欲還陜巴。然哈密屢破,遺民入居者旦暮虞寇。阿黑麻果復來攻,固守不下,訖散去。諸人自以窮窘難守,盡焚室廬,走肅州求濟。邊臣以聞,詔賜牛具、穀種,並發流寓三種番人及哈密之寄居赤斤者,盡赴苦峪及瓜、沙州,俾自耕牧,以圖興復。

時哈密無王,奄克孛剌為之長。十年遣其黨寫亦虎仙等來貢,給幣帛五千酬其直,使臣猶久留,大肆咆烋。禮官徐瓊等極論其罪,乃驅之去。時諸番以朝廷閉關絕貢不得入,咸怨阿黑麻,阿黑麻悔,送還陜巴及哈密之眾,乞通貢如故。廷議謂無番文不可驟許,必令具文乃從其請。陜巴前議廢,今使暫居甘州,俟眾頭目俱歸心,然後修復哈密城塹,令復舊業。帝悉從之。冬,起王越總制三邊軍務兼經理哈密。十一年秋,越言哈密不可棄,陜巴亦不可廢,宜仍其舊封,令先還哈密,量給修城、築室之費,犒賜三種番人及赤斤、罕東、小列禿、乜克力諸部,以獎前勞,且責後效。帝亦報可。自是哈密復安,土魯番亦修貢惟謹。

奄克孛剌者,罕慎弟也,與陜巴不相能。當事患之,令陜巴娶罕慎女,與之結好。陜巴嗜酒掊剋,失眾心,部下阿孛剌等咸怨。十七年春,陰構阿黑麻迎其幼子真帖木兒主哈密。陜巴懼,挈家走苦峪。奄克孛剌與寫亦虎仙在肅州,邊臣以二人為番眾所服,令還輔陜巴,與百戶董傑偕行。傑有膽略。既抵哈密,阿孛剌與其黨五人約夜以兵來劫。傑知之,與奄克孛剌等謀,召阿孛剌等計事,立斬之,其下遂不敢叛。乃令陜巴還哈密,真帖木兒還土魯番。真帖木兒年十三,其母即罕慎女也,聞父已死,兄滿速兒嗣為速檀與諸弟相仇殺,懼不敢歸,願倚奄克孛剌,曰:「吾外祖也。」邊臣慮與陜巴隙,居之甘州。十八年冬,陜巴卒,其子拜牙即自稱速檀,命封為忠順王。

正德三年,寫亦虎仙入貢,不與通事偕行,自攜邊臣文牒投進。大通事王永怒,疏請究治,寫酋亦奏永需求。永供奉豹房,恃寵恣橫。詔勿究治,兩戒諭之。寫酋自是益輕朝廷,潛懷異志。

初,拜牙即嗣職,滿速兒與通和,且遣使求真帖木兒,邊臣言與之便。樞臣謂土魯番稔惡久,今見我扶植哈密,聲勢漸張,乃卑詞求貢,以還弟為名。我留其弟,正合古人質其親愛之意,不可遽遣。帝從之。六年始命寫亦虎仙偕都督滿哈剌三送之西還,至哈密,奄克孛剌欲止之,二人不可。護至土魯番,遂以國情輸滿速兒,且誘拜牙即叛。拜牙即素昏愚,性又淫暴,心怵屬部害已,而滿速兒又甘言誘之,即欲偕奄克孛剌同往,不從,奔肅州。八年秋,拜牙即棄城叛入土魯番。滿速兒遣火者他只丁據哈密,又遣火者馬黑木赴甘肅言拜牙即不能守國,滿速兒遣將代守,乞犒賜。

九年四月,事聞,命都御史彭澤往經略。澤未至,賊遣兵分掠苦峪、沙州,聲言予我金幣萬,即歸城印。澤抵甘州,謂番人嗜利,可因而款也。遣通事馬驥諭令還侵地及王,當予重賞。滿速兒偽許之,澤即畀幣帛二千及白金酒器一具。十一年五月,拜疏言:「臣遣通事往宣國威,要以重賞,其酋悔過效順,即以金印及哈密城付之。滿哈剌三、寫亦虎仙二人召還他只丁,並還所奪赤斤衛印。惟忠順王在他所,未還。請錄效勞人役功,賜臣骸骨歸田里。」帝即令還朝。忠順王迄不返,他只丁亦不肯退,復要重賞,始以城來歸。

明年五月,甘肅巡撫李昆上言:「得滿速兒牒,謂拜牙即不可復位,即還故土,已失人心,乞別立安定王千奔後裔。此言良然。如必欲其復國,乞敕滿速兒兄弟送還,仍厚賜繒帛,冀其效順。」廷議:「經略西陲已踰三載,而忠順迄無還期,宜興師絕貢,不可遂其要求,損我威重。但城印歸,國體具在,宜敕責滿速兒背負國恩,求取無厭。仍量賜其兄弟,令其速歸忠順。不從,則閉關絕貢,嚴兵為備。」從之。

初,寫亦虎仙與滿速兒深相結,故首倡逆謀。已而有隙,滿速兒欲殺之,大懼,求他只丁為解,許賂幣千五百匹,期至肅州畀之,且啖之入寇,曰肅州可得也。滿速兒喜,令與其婿馬黑木俱入貢,以覘虛實,且徵其賂。邊臣以同來火者撒者兒,乃火者他只丁弟,懼為變,并其黨虎都寫亦羈之甘州,而督寫亦虎仙出關,懼不肯去。他只丁聞其弟被拘,怒,復又奪哈密城,請滿速兒移居之,分兵脅據沙州,擁眾入寇,至兔兒壩,遊擊芮寧與參將蔣存禮,都指揮黃榮、王琮各率兵往禦。寧先抵娑子壩,遇賊。賊悉眾圍寧,而分兵綴諸將,寧所部七百人皆戰沒。賊薄肅州城,索所許幣。副使陳九疇固守,且先絕其內應,賊知事洩,慮援兵至,大掠而去。

十二年正月,羽書聞,廷議復命彭澤總制軍務,偕中官張永、都督郤永率師西征。賊還至瓜州,副總兵鄭廉合奄克孛剌兵擊敗之,斬七十九級。賊乃遁去,又與瓦剌相攻,力不敵,移書求款,澤等乃罷行。

先是,寫亦虎仙與子米兒馬黑木、婿火者馬黑木及其黨失拜煙答俱以內應繫獄,失拜煙答被捶死。及事平,械寫亦虎仙赴京,下刑部獄,其子仍繫甘州。失拜煙答子米兒馬黑麻者,寫亦虎仙姪婿他,以入貢在京,探知王瓊欲傾彭澤,突入長安門訟父冤,下錦衣獄。會兵部、法司請行甘肅訊報,瓊欲因此興大獄,奏遣科道二人往勘。明年,勘至,於澤無所坐。瓊怒,劾澤欺罔辱國,斥為民。坐昆、九疇激變,逮下吏,並獲重譴。明年,寫亦虎仙亦減死,遂夤緣錢寧,與其婿得侍帝左右。帝悅之,賜國姓,授錦衣指揮,扈駕南征。

滿速兒犯邊後,屢求通貢,不得。十五年歸先所掠將卒及忠順王家屬,復求貢。廷議許之,而王迄不還。巡按御史潘仿力言貢不當許,不聽。明年,世宗嗣位,楊廷和以寫亦虎仙稔中國情實,歸必為邊患,於遺詔中數其罪,并其子婿伏誅,而用陳九疇為甘肅巡撫。

時滿速兒比歲來貢,朝廷待之若故,亦不復問忠順王事。嘉靖三年秋,擁二萬騎圍肅州,分兵犯甘州。九疇及總兵官姜奭等力戰敗之,斬他只丁,賊乃卻去。事聞,命兵部尚書金獻民西討,抵蘭州,賊已久退,乃引還。九疇因力言賊不可撫,乞閉關絕貢,專固邊防,可之。明年秋,賊復犯肅州,分兵圍參將雲冒,而以大眾抵南山。九疇時已解職,他將援兵至,賊始遁。

當是時,番屢犯邊城,當局者無能振國威,為邊疆復仇雪恥,而一二新進用事者反借以修怨。由是,封疆之獄起。百戶王邦奇者,素憾楊廷和、彭澤,六年春,上言:「今哈密失國,番賊內侵,由澤賂番求和,廷和論殺寫亦虎仙所致。誅此兩人,庶哈密可復,邊境無虞。」桂萼、張璁輩欲藉此興大獄,斥廷和、澤為民,盡置其子弟親黨於理,有自殺者。復遣給事、錦衣官往按。番酋牙蘭言非敢獲罪天朝,所以犯邊,由冤殺寫亦虎仙、失拜煙答二人故。今願獻還城印贖前罪。事下兵部,尚書王時中等言:「番酋乞貢數四,先已下總制尚書王憲,因其貢使鐫責。所請當不妄,第其詞出牙蘭,非真求貢之文,或詐以款我。若果悔罪,必先歸城印及所掠人畜,械送首惡,稽首關門,方可聽許。」帝納之。萼以前獄未竟,必欲重興大獄,請留質牙蘭,遣譯者諭其主還侵地。而與禮、兵二部尚書方獻夫、王時中等協議,為挑激之詞,言番人上書者四輩,皆委咎前吏,雖詞多詆飾,亦事發有因。宜遣官嚴核激變虛實,用服其心,其他具如前議。九疇報捷時,言滿速兒、牙蘭已斃炮石下,二人實未死。帝固疑之。覽萼等議,益疑邊臣欺罔,手詔數百言,切責九疇,欲置之死,而戒首輔楊一清勿黨庇,遂遣官逮九疇。尚書金獻民、侍郎李昆以下,坐累者四十餘人。

七年正月,九疇逮至下獄。萼等必欲殺之,並株連廷和、澤。刑部尚書胡世寧力救,帝稍悟,免死戍邊,澤、獻民等皆落職。番酋氣益驕,而萼又薦王瓊督三邊,盡釋還九疇所繫番使,許之通貢。番酋迄不悔罪,侮玩如故。時以牙蘭獲罪其主,率部帳來歸,邊臣受之。滿速兒怒,其部下虎力納咱兒引瓦剌二千餘騎犯肅州,至老鸛堡,值撒馬兒罕貢使在堡中,賊呼與語,遊擊彭浚急引兵擊之。賊言欲問信通和,浚不聽,進戰,破之。賊遁走赤斤,使人持番文求貢,委罪瓦剌,詞多悖謾。瓊希時貴指,必欲議撫,因言番人且悔,宜原情赦罪,以罷兵息民,並上浚及副使趙載功狀。章下兵部。

初,胡世寧之救陳九疇也,欲棄哈密不守,言:「拜牙即久歸土魯番,即還故土,亦其臣屬,其他族裔無可繼者。回回一種,早已歸之。哈剌灰、畏兀兒二族逃附肅州已久,不可驅之出關。然則哈密將安興復哉?縱得忠順嫡派,畀之金印,助之兵食,誰與為守?不過一二年,復為所奪,益彼富強,辱我皇命,徒使再得城印,為後日要挾之地。乞聖明熟籌,如先朝和寧交址故事,置哈密勿間。如其不侵擾,則許之通貢。否則,閉關絕之,庶不以外番疲中國。」詹事霍韜力駁其非。至是,世寧改掌兵部,上言:「番酋變詐多端,欲取我肅州,則漸置奸回於內地。事覺,則多縱反間,傾我輔臣。乃者許之朝貢,使方入關,而賊兵已至,河西幾危。此閉關與通貢,利害較然。今瓊等既言賊薄我城堡,縛我士卒,聲言大舉,以恐嚇天朝,而又言賊方懼悔,宜仍許通貢,何自相牴牾。霍韜又以賊無印信番文為疑,臣謂即有印信,亦安足據。第毋墮其術中,以間我忠臣,弛我邊備,斯可矣。牙蘭本我屬番,為彼掠去,今束身來歸,事屬反正,宜即撫而用之。招彼攜貳,益我籓籬。至於興復哈密,臣等竊以為非中國所急也。夫哈密三立三絕,今其王已為賊用,民盡流亡。借使更立他種,彼強則入寇,弱則從賊,難保為不侵不叛之臣。故臣以為立之無益,適令番酋挾為奸利耳。乞賜瓊璽書,令會同甘肅守臣,遣番使歸諭滿速兒,詰以入寇狀。倘委為不知,則令械送虎力納咱兒。或事出瓦剌,則縛其人以自贖。否則羈其使臣,發兵往討,庶威信並行,賊知斂戢。更敕瓊為國忠謀,力求善後之策,以通番納貢為權宜,足食固圉為久計,封疆幸甚。」疏入,帝深然之,命瓊熟計詳處,毋輕信番言。

至明年,甘肅巡撫唐澤亦以哈密未易興復,請專圖自治之策。瓊善之,據以上聞,帝報可。自是置哈密不問,土魯番許之通貢,西陲藉以息肩。而哈密後為失拜煙答子米兒馬黑木所有,服屬土魯番。朝廷猶令其比歲一貢,異於諸番,迄隆慶、萬曆朝猶入貢不絕,然非忠順王苗裔矣。

柳城,一名魯陳,又名柳陳城,即後漢柳中地,西域長史所治。唐置柳中縣。西去火州七十里,東去哈密千里。經一大川,道旁多骸骨,相傳有鬼魅,行旅早暮失侶多迷死。出大川,渡流沙,在火山下,有城屹然廣二三里,即柳城也。四面皆田園,流不環繞,樹木陰翳。土宜穄麥豆麻,有桃李棗瓜胡蘆之屬。而葡萄最多,小而甘,無核,名鎖子葡萄。畜有牛羊馬駝。節候常和。土人純朴,男子椎結,婦人蒙皁布,其語音類畏兀兒。

永樂四年,劉帖木兒使別失八里,因命齎彩幣賜柳城酋長。明年,其萬戶瓦赤剌即遣使來貢。七年,傅安自西域還,其酋復遣使隨入貢。帝即命安齎綺帛報之。十一年夏,遣使隨白阿兒忻台入貢。冬,萬戶觀音奴再遣使隨安入貢。二十年與哈密共貢羊二千。

宣德五年,頭目阿黑把失來貢。正統五年、十三年並入貢。自後不復至。

柳城密爾火州、土魯番,凡天朝遣使及其酋長入貢,多與之偕。後土魯番強,二國並為所滅。

火州,又名哈剌,在柳城西七十里,土魯番東三十里,即漢車師前王地。隋時為高昌國。唐太宗滅高昌,以其地為西州。宋時回鶻居之,嘗入貢。元名火州,與安定、曲先諸衛統號畏兀兒,置達魯花赤監治之。

永樂四年五月命鴻臚丞劉帖木兒護別失八里使者歸,因齎彩幣賜其王子哈散。明年遣使貢玉璞方物。使臣言,回回行賈京師者,甘、涼軍士多私送出境,洩漏邊務。帝命御史往按,且敕總兵官宋晟嚴束之。七年遣使偕哈烈、撒馬兒罕來貢。十一年夏,都指揮白阿兒忻台遣使偕俺的乾、失剌思等九國來貢。秋,命陳誠、李暹等以璽書、文綺、紗羅、布帛往勞。十三年冬,遣使隨誠來貢。自是久不至。正統十三年復貢,後遂絕。

其地多山,青紅若火,故名火州。氣候熱。五穀、畜產與柳城同。城方十餘里,僧寺多於民居。東有荒城,即高昌國都,漢戊己校尉所治。西北連別失八里。國小,不能自立,後為土魯番所並。

土魯番,在火州西百里,去哈密千餘里,嘉峪關二千六百里。漢車師前王地。隋高昌國。唐滅高昌,置西州及交河縣,此則交河縣安樂城也。宋復名高昌,為回鶻所據,嘗入貢。元設萬戶府。

永樂四年遣官使別失八里,道其地,以彩幣賜之。其萬戶賽因帖木兒遣使貢玉璞,明年達京師。六年,其國番僧清來率徒法泉等朝貢。天子欲令化導番俗,即授為灌頂慈慧圓智普通國師,徒七人並為土魯番僧綱司官,賜賚甚厚。由是其徒來者不絕,貢名馬、海青及他物。天子亦數遣官獎勞之。

二十年,其酋尹吉兒察與哈密共貢馬千三百匹,賜賚有加。已而尹吉兒察為別失八里酋歪思所逐,走歸京師。天子憫之,命為都督僉事,遣還故土。尹吉兒察德中國,洪熙元年躬率部落來朝。宣德元年亦如之。天子待之甚厚,還國病卒。三年,其子滿哥帖木兒來朝。正統六年,朝議土魯番久失貢,因米昔兒使臣還,令齎鈔幣賜其酋巴剌麻兒。明年遣使入貢。

初,其地介于闐、別失八里諸大國間,勢甚微弱。後侵掠火州、柳城,皆為所并,國日強,其酋也密力火者遂僭稱王。以景泰三年,偕其妻及部下頭目各遣使入貢。天順三年復貢,其使臣進秩者二十有四人。先後命指揮白全、都指揮桑斌等使其國。

成化元年,禮官姚夔等定議,土魯番三年或五年一貢,貢不得過十人。五年遣使來貢,其酋阿力自稱速檀,奏求海青、鞍馬、蟒服、彩幣、器用。禮官言物多違禁,不可盡從,命賜彩幣、布帛。明年復貢,奏求忽撥思箏、鼓羅、占鐙、高麗布諸物。廷議不許。

時土魯番愈強,而哈密以無主削弱,阿力欲并之。九年春,襲破其城,執王母,奪金印,分兵守之而去。朝廷命李文等經略,無功而還。阿力修貢如故,一歲中,使來者三,朝廷仍善待之,未嘗一語嚴詰。貢使益傲,求馴象。兵部言象以備儀衛,禮有進獻,無求索,乃卻其請。使臣復言已得哈密城池及瓦剌奄檀王人馬一萬,又收捕曲先并亦思渴頭目倒刺火只,乞朝廷遣使通道,往來和好。帝曰:「迤西道無阻,不須遣官。阿力果誠心修貢,朝廷不計前愆,仍以禮待。」使臣復言赤斤諸衛素與有仇,乞遣將士護行,且謂阿力雖得哈密,止以物產充貢,願質使臣家屬於邊,賜敕歸諭其王,獻還城印。帝從其護行之請,而賜敕諭阿力獻王母及城印,即和好如初。使臣還,復遣他使再入貢,而不還哈密。

十二年八月,甘州守臣言,番使謂王母已死,城印俱存,俟朝廷往諭即獻還。帝已卻其貢使,復俾入京。時大臣專務姑息,致遐方小醜無顧忌。

十四年,阿力死,其子阿黑麻嗣為速檀,遣使來貢。十八年,哈密都督罕慎潛師搗哈密,克之。賊將牙蘭遁走。阿黑麻頗懼。朝議罕慎有功,將立為王。阿黑麻聞之,怒曰:「罕慎非忠順族,安得立!」乃偽與結婚。

弘治元年躬至哈密城下,誘罕慎盟,執殺之,復據其城,而遣使入貢;稱與罕慎締姻,乞賜蟒服及九龍渾金膝襴諸物。使至甘州,而罕慎之變已聞,朝廷亦不罪,但令還諭其主,歸我侵地。番賊知中國易與,不奉命,復遣使來貢。禮官議薄其賞,拘使臣,番賊稍懼。

三年春,偕撒馬兒罕貢獅子,願獻還城印,朝廷亦還其使臣。禮官請卻勿納,帝不從。及使還,命內官張芾護行,諭內閣草敕。閣臣劉吉等言:「阿黑麻背負天恩,殺我所立罕慎,宜遣大將直搗巢穴,滅其種類,始足雪中國之憤。或不即討,亦當如古帝王封玉門關,絕其貢使,猶不失大體。今寵其使臣,厚加優待,又遣中使伴送,此何理哉!陛下事遵成憲,乃無故召番人入大內看戲獅子,大賚御品,誇耀而出。都下聞之,咸為駭嘆,謂祖宗以來,從無此事。奈何屈萬乘之尊,為奇獸之玩,俾異言異服之人,雜遝清嚴之地。況使臣滿剌土兒即罕慎外舅,忘主事仇,逆天無道。而阿黑麻聚集人馬,謀犯肅州,名雖奉貢,意實叵測。兵部議羈其使臣,正合事宜。若不停張芾之行,彼使臣還國,阿黑麻必謂中土帝王可通情希寵,大臣謀國,天子不聽,其奈我何。長番賊之志,損天朝之威,莫甚於此。」疏入,帝止芾行,而問閣臣興師、絕貢二事。吉等以時勢未能,但請薄其賜賚。因言飼獅日用二羊,十歲則七千二百羊矣,守獅日役校尉五十人,一歲則一萬八千人矣。若絕其餧養,聽其自斃,傳之千載,實為美談。帝不能用。

秋,又遣使從海道貢獅子,朝命卻之,其使乃潛詣京師。禮官請治沿途有司罪,仍卻其使,從之。當是時,中外乂安,大臣馬文升、耿裕輩,咸知國體,於貢使多所裁損,阿黑麻稍知中國有人。四年秋,遣使再貢獅子,願還金印,及所據十一城。邊臣以聞,許之,果以城印來歸。明年封陜巴為忠順王,納之哈密,厚賜阿黑麻使臣,先所拘者盡釋還。

六年春,其前使二十七人還,未出境,後使三十九人猶在京師,阿黑麻復襲陷哈密,執陜巴以去。帝命侍郎張海等經略,優待其使,俾得進見。禮官耿裕等諫曰:「朝廷馭外番,宜惜大體。番使自去年入都,久不宣召,今春三月以來,宣召至再,且賜幣帛羊酒,正當謾書投入之時,小人何知,將謂朝廷恩禮視昔有加,乃畏我而然。事干國體,不可不慎。況此賊倔強無禮,久蓄不庭之心。所遣使臣,必其親信腹心,乃令出入禁掖,略無防閑。萬一奸宄窺伺,潛逞逆謀,雖悔何及。今其使寫亦滿速兒等宴賚已竣,猶不肯行,曰恐朝廷復宣召。夫不寶遠物,則遠人格。獅本野獸,不足為奇,何至上煩鑾輿,屢加臨視,致荒徼小醜,得覲聖顏,藉為口實。」疏入,帝即遣還。張海等抵甘肅,遵朝議,卻其貢物,羈前後使臣一百七十二人於邊,閉嘉峪關,永絕貢道。而巡撫許進等,又潛兵直搗哈密,走牙蘭,阿黑麻漸懼。其鄰邦不獲貢,胥怨阿黑麻。十年冬,送還陜巴,款關求貢,廷議許之。十二年,其使再求,命前使安置廣東者悉釋還。

十七年,阿黑麻死,諸子爭立,相仇殺。已而長子滿速兒嗣為速檀,修貢如故。明年,忠順王陜巴卒,子拜牙即襲,昏愚失道,國內益亂。而滿速兒桀點變詐踰於父,復有吞哈密之志。

正德四年,其弟真帖木兒在甘州,貢使乞放還。朝議不許,乃以甘州守臣奏送還。還即以邊情告其兄,共謀為逆。九年誘拜牙即叛,復據哈密。朝廷遣彭澤經略,贖還城印。其部下他只丁復據之,且導滿速兒犯肅州。自是,哈密不可復得,而患且中於甘肅。會中朝大臣自相傾陷,番酋覘知之,益肆讒構,賊腹心得侍天子,中國體大虧,賊氣焰益盛。

十五年復許通貢。甘肅巡按潘仿言:「番賊犯順,殺戮摽掠,慘不可勝言。今雖悔罪,果足贖前日萬一乎?數年以來,雖嘗閉關,未能問罪。今彼以困憊求通,且將窺我意向,探我虛實,緩我後圖,誘我重利。不於此時稍正其罪,將益啟輕慢之心,招反覆之釁,非所以尊中國馭外番也。況彼番文執難從之詞,示敢拒之狀,當悔罪求通之日,為侮慢不恭之語,其變詐已見。若曰來者不拒,馭戎之常,盡略彼事之非,納求和之使,必將叨冒恩禮,飽饜賞餼,和市私販,滿載而歸。所欲既足,驕志復萌,少不慊心,動則藉口,反復之釁,且在目前。叛則未嘗加罪,而反獲鈔掠之利,來則未必見拒,而更有賜賚之榮,何憚不為。臣謂宜乘窘迫之時,聊為懾伏之計,雖納其悔過之詞,姑阻其來貢之使,降敕責其犯順,仍索歸還未盡之人。其番文可疑者,詳加詰問,使彼知中國尊嚴,天威難犯,庶幾反側不萌,歸服可久。」時王瓊力主款議,不納其言。

明年,世宗立,賊腹心寫亦虎仙伏誅,失所恃,再謀犯邊。嘉靖三年寇肅州,掠甘州,四年復寇肅州,皆失利去,於是卑詞求貢。會璁、萼等起封疆之獄,遂陰庇滿速兒再許之貢,議已定。賊黨牙蘭者,本曲先人,幼為番掠,長而黠健,阿力以妹妻之,握兵用事,久為西陲患,至是獲罪其主,七年夏,率所部二千人來降。有帖木兒哥、土巴者,俱沙州番族,土魯番役屬之,歲徵婦女牛馬,不勝侵暴,亦率其族屬數千帳來歸。邊臣悉處之內地。

滿速兒怒,使其部下虎力納咱兒引瓦剌寇肅州,不勝,則復遣使求貢。總督王瓊請許之,詹事霍韜言:「番人攻陷哈密以來,議者或請通貢,或請絕貢,聖諭必有悔罪番文然後許。今王瓊譯進之文,皆其部下小醜之語,無印信足憑。我遽許之,恐戎心益驕,後難駕馭。可虞者一。哈密城池雖稱獻還,然無實據,何以興復。或者遂有棄置不問之議,彼愈得志,必且劫我罕東,誘我赤斤,掠我瓜、沙,外連瓦剌,內擾河西,而邊警無時息矣。可虞者二。牙蘭為番酋腹心,擁眾來奔,而彼云不知所向,安知非詐降以誘我。他日犯邊,曰納我叛臣也。我不歸彼叛臣,彼不歸我哈密。自是西陲益多事,而哈密終無興復之期。可虞者三。牙半之來,日給廩餼,所費實多,猶曰羈縻之策不獲已也。倘番酋擁眾叩關,索彼叛人,將予之耶,抑拒之耶?又或牙蘭包藏禍心,構變於內,內外協應,何以禦之?可虞者四。或曰今陜西饑困,甘肅孤危,哈密可棄也。臣則曰,保哈密所以保甘、陜也,保甘肅所以保陜西也。若以哈密難守即棄哈密,然則甘肅難守亦棄甘肅乎?昔文皇之立哈密也,因元遺孽力能自立,因而立之。彼假其名,我享其利。今忠順之嗣三絕矣,天之所廢,孰能興之?今於諸夷中,求其雄傑能守哈密者,即畀金印,俾和輯諸番,為我籓蔽,斯可矣,必求忠順之裔而立焉,多見其固也。」疏入,帝嘉其留心邊計,下兵部確議。尚書胡世寧等力言牙蘭不可棄,哈密不必興復,請專圖自治之策,帝深納其言。自是番酋許通貢,而哈密城印及忠順王存亡置不復問,河西稍獲休息,而滿速兒桀傲益甚矣。

十二年遣臣奏三事。一,請追治巡撫陳九疇罪。一,請遣官議和。一,請還叛人牙蘭。詞多悖慢,朝廷不能罪,但戒以修職貢無妄言。然自寫亦虎仙誅,他只丁陣歿,牙蘭又降,失其所倚賴,勢亦漸孤,部下各自雄長,稱王入貢者多至十五人,政權亦不一。

十五年,甘肅巡撫趙載陳邊事,言:「番酋屢服屢叛,我撫之太厚,信之太深,愈長其奸狡。今後入犯,宜戮其使臣,徙其從人於兩粵,閉關拒絕。即彼悔罪,亦但許奉貢,不得輒還從人。彼內有所牽,外有所畏,自不敢輕犯。」帝頗採其言。

二十四年,滿速兒死,長子沙嗣為速檀,其弟馬黑麻亦稱速檀,分據哈密。已而兄弟仇殺,馬黑麻乃結婚瓦剌以抗其兄,且墾田沙州,謀入犯。其部下來告,馬黑麻乃叩關求貢,復求內地安置。邊臣諭止之,乃還故土,與兄同處。總督張珩以聞,詔許其入貢。二十六年定令五歲一貢。其後貢期如令,而來使益多。逮世宗末年,番文至二百四十八道。朝廷重違其情,咸為給賜。

隆慶四年,馬黑麻嗣兄職,遣使謝恩。其弟瑣非等三人,亦各稱速檀,遣使來貢。禮官請裁其犒賜,許附馬黑麻隨從之數,可之。迄萬歷朝,奉貢不絕。


\end{pinyinscope}