\article{列傳第四十八}

\begin{pinyinscope}
王彰魏源金濂石璞王巹羅通羅綺張固張瑄張鵬李裕

王彰,字文昭,鄭人。洪武二十年舉於鄉,補國子生。使山東平糴,以廉乾稱,擢吏科源士。踰年,革源士,改給事中,累遷山西左參政。

永樂五年召為禮部侍郎。父喪,服除,改戶部。陜西大疫,奉使祀西嶽。新安民鬻子女償賦。彰奏為蠲除,贖還所鬻。改右副都御史。

陜西僉事馬英激肅州番為變,殺御史及都指揮。彰劾英,置極典。又劾御史陳孟旭受賕枉法、文獻盜銀課、及金吾指揮李嚴逐母不養,皆坐死。他所論劾甚眾。十一年從帝北巡。彰有母年八十餘矣,命歸省,賜其母冠服金幣。諭之曰:「君子居官不忘親,居家不忘君。凡所過,民安否,吏賢不肖,悉以聞。」彰還,奏事稱旨。久之,進右都御史。

十九年,帝遣廷臣二十六人巡撫天下,彰與給事中王勵往河南。終明世,大臣得撫鄉土者,彰與葉春而已。河南水災,民多流亡,長吏不加恤。彰奏黜貪刻者百餘人,罷不急之征十餘事。招復流民,發廩振貸,多所全活。還朝,命督餉北征。仁宗即位,河溢開封,命彰與都指揮李信往振恤。

宣德元年五月,命彰自良鄉抵南京巡撫軍民。尋以所言率常事,降敕切責,令詳具利病以聞。復諭侍臣曰:「兩京相距數千里,驛使往來為擾,或遘水旱,小民失所,朝使還及御史巡歷皆不以告,故遣彰往視。今所奏多細故。大臣如此,朕復何望!卿等當悉朕意,君臣同體,勿有所疑。」尋召還,命與都督山雲巡山海至居庸諸關隘。踰二月還,奏將士擅離者,帝命逮治。遂命兵部三月一遣御史、給事中點閱。明年四月卒於官。

彰嚴介自持,請托皆絕,然用法過刻。其母屢以為言,不能改。時劉觀為左都御史。人謂「彰公而不恕,觀私而不刻」云。

魏源,字文淵,建昌縣人。永樂四年進士。除監察御史。辨松江知府黃子威誣。奏減浙東瀕海漁課。巡按陜西。西安大疫,療活甚眾。奏言:「諸府倉粟積一千九十餘萬石,足支十年。今民疫妨農,請輸鈔代兩稅之半。」從之。涼州土寇將為變。亟請剿,亂遂息。兩遭喪,俱起復。洪熙元年出為浙江按察副使。

宣德三年召署刑部右侍郎。五年,河南旱荒,民多轉徙。帝以源廉正有為,命為左布政使,俾馳驛之任。時侍郎許廓往撫輯,廷議又起丁憂布政使李昌祺原官。源與廓、昌祺發倉廩,免逋賦雜役,流民漸歸。雨亦旋降,歲大豐。居三年,召還,授刑部左侍郎。明年,永豐民夏九旭等據大盤山為亂。帝以源江西人,命撫之,都督任禮帥兵隨其後。未至,官軍擒九旭,因命二人採木四川,兼飭邊務。

英宗即位,進尚書。正統二年五月命整飭大同、宣府諸邊,許便宜行事。源遣都督僉事李謙守獨石,楊洪副之,劾萬全衛指揮杜衡戍廣西。明年奏大同總兵官譚廣老,帝命黃真、楊洪充左右參將協鎮,諸將肅然。按行天城、朔州諸險要,令將吏分守。設威遠衛,增修開平、龍門城,自獨石抵宣府,增置墩堠。免屯軍租一年,儲火器為邊備,諸依權貴避役者悉括歸伍。尋以宣、大軍務久弛,請召還巡撫僉都御史盧睿,而薦兵部侍郎于謙為鎮守參贊。朝廷以謙方撫山西、河南,不聽。于是言官以臨邊擅易置大臣為源罪,合疏劾之。且言源為御史嘗犯贓,乃冒領誥命。帝以源有勞,置不問。事竣還朝,與都御史陳智相詈於直廬。智以聞,詔兩責之。

歲旱,錄上疑獄,且請推行於天下,報可。旋坐決獄不當,與侍郎何文淵俱下獄。得宥,復以上遼王貴烚罪狀,不言其內亂事,與三司官皆繫詔獄。累月,釋還職。

源在刑部久,議獄多平恕。陜西僉事計資言,武臣雜犯等罪,予半俸,謫極邊。源以所言深刻,奏寢之。郎中林厚言禁刁訟、告訐及擇理刑官、勘重囚務憑贓具四事,皆以源議得施行。六年以足疾命朝朔望。八年致仕,卒。

金濂,字宗瀚,山陽人。永樂十六年進士,授御史。宣德初,巡按廣東,廉能最。改按江西、浙江。捕巨盜不獲,坐免。盜就執,乃復官。嘗言郡縣吏貪濁,宜敕按察司、巡按御史察廉能者,如洪武間故事,遣使勞賚,則清濁分,循良勸。帝嘉納之。用薦遷陜西副使。

正統元年上書請補衛所缺官,益寧夏守兵,設漢中鎮守都指揮使,多議行。三年擢僉都御史,參贊寧夏軍務。濂有心計,善籌畫,西陲晏然。寧夏舊有五渠,而鳴沙洲、七星漢、伯石灰三渠淤。濂請濬之,溉蕪田一千三百餘頃。時詔富民輸米助邊,千石以上褒以璽書。濂言邊地粟貴,請並旌不及額者,儲由此充。六年詔僉都御史盧睿與濂更代。明年,睿召還,濂復出鎮。尋加右副都御史,與睿代者再。

八年秋拜刑部尚書,侍經筵。十一年,安鄉伯張安與弟爭祿,詔逮治。法司與戶部相諉,言官劾濂及戶部尚書王佐,右都御史陳鎰,侍郎丁鉉、馬昂,副都御史丁璿、程富等,俱下獄。數日,釋之。

福建賊鄧茂七等為亂,都督劉聚、都御史張楷征之,不克。十三年十一月大發兵,命寧陽侯陳懋等為將軍往討,以濂參軍務。比至,御史丁瑄已大破賊。茂七死,餘賊擁其兄子伯孫據九龍山,拒官軍。濂與眾謀,羸師誘之出,伏精兵,入其壘,遂擒伯孫。帝乃移楷討浙寇,而留濂擊平餘賊未下者。會英宗北狩,兵事棘,召還。言者交劾濂無功,景帝不問,加濂太子賓客,給二俸。尋改戶部尚書,進太子太保。

時四方用兵,需餉急,濂綜核無遺,議上撙節便宜十六事,國用得無乏。未幾,上皇還。也先請遣使往來如初,帝堅意絕之。濂再疏諫,不聽。初,帝即位,詔免景泰二年天下租十之三。濂檄有司,但減米麥,其折收銀布絲帛者徵如故。三年二月,學士江淵以為言,命部查理。濂內慚,抵無有。給事中李侃等請詰天下有司違詔故。濂恐事敗,乃言:「銀布絲帛,詔書未載,若概減免,國用何資?」於是給事中御史劾濂失信於民,為國斂怨,且訐其陰事。帝欲宥之,而侃與御史王允力爭,遂下都察院獄。越三日釋之,削宮保,改工部。吏部尚書何文淵言理財非濂不可,乃復還戶部。濂上疏自理,遂乞骸骨,帝慰留之。東宮建,復宮保。尋復條上節軍匠及僧道冗食共十事。五年卒官,以軍功追封沭陽伯,謚榮襄。

濂剛果有才,所至以嚴辦稱,然接下多暴怒。在刑部持法稍深。及為戶部,值兵興財詘,頗厚斂以足用云。

石璞,字仲玉,臨漳人。永樂九年舉於鄉,入國學。選授御史。

正統初,歷任江西按察使。三年坐逸囚,降副使。璞善斷疑獄。民娶婦,三日歸寧,失之。婦翁訟婿殺女,誣服論死。璞禱於神,夢神示以麥字。璞曰:「麥者,兩人夾一人也。」比明,械囚趣行刑。未出,一童子窺門屏間。捕入,則道士徒也。叱曰:「爾師令爾偵事乎?」童子首實,果二道士匿婦槁麥中。立捕,論如法。在江西數年,風紀整肅,雖婦豎無不知石憲使者。

七年遷山西布政使。明年,以朝廷歲用物料,有司科派擾民,請于折糧銀內歲存千兩,令官買辦,庶官用可完,民亦不擾。從之。

工部尚書王巹以不能屈意王振,十三年致仕去。璞為振所善,遂召為尚書。明年,處州賊葉宗留作亂,總兵官徐恭等往討,以璞參其軍事。師未至,宗留已為其黨陳鑒胡所殺。巡撫張驥招降鑒胡,賊勢稍息。璞等逗遛無功,為御史張洪等所劾,詔俟師旋以聞。

已而景帝嗣位,召還。論功,兼大理寺卿。尋出募天下義勇,還朝。會中官金英下獄,法司劾璞嘗賂英,遂并下璞獄,當斬,特宥之。出理大同軍餉。敵犯馬營,命提督宣府軍務。至則寇已退,還理部事。加太子太保,給二俸。

河決沙灣,命治之。璞以決口未易塞,別濬渠。自黑洋山至徐州,以通漕艘,而決口如故。乃命內官黎賢等偕御史彭誼助之。于沙灣築石堤以禦決河,開月河二,引水益運河以殺水勢,決乃塞。璞還言:「京師盜賊多出軍伍。間有獲者,輒云『糧餉虧減,妻孥饑凍故』。又聞兩畿、山東、河南被災窮民多事剽掠,不及今拊循,恐方來之憂甚於邊患。口外守軍,夜行晝伏,艱苦萬狀。今邊疆未靖,宜增餉以作士氣,乃反減其月糧,此實啟盜誤國之端,非節財足用之術。」帝深納其言。沙灣復決,璞再往治之。以母憂歸,起復。

六年改兵部尚書,與於謙協理部事。明年,湖廣苗亂,命璞總督軍務,與南和伯方瑛討之。天順元年以捷聞。召還,命致仕。既而論功,賜鈔幣。四年冬用李賢薦,召為南京左都御史。時璞已老聵,不能任事。七年為錦衣衛指揮僉事門達所劾罷,歸卒。

王巹,郿人。永樂中鄉薦,歷山東左布政使,所至有惠政。正統六年入為工部侍郎,代吳中為尚書。歸家十五年卒。

羅通,字學古,吉水人。永樂十年進士。授御史,巡按四川。都指揮郭贇與清軍御史汪琳中交通為奸利,通劾奏,逮治之。三殿災,偕同官何忠等極陳時政闕失。忤旨,出為交阯清化知州。

宣德元年,黎利反,王通戰敗,擅傳檄割清化迤南畀賊。賊方圍清化,通與指揮打忠堅守,乘間破賊,殺傷甚眾。賊將遁而檄至,通曰:「吾輩殺賊多,出城必無全理,與就縛,曷若盡忠死。」乃與忠益固守。賊久攻不下,令降將蔡福說降,通登陴大罵。賊知城不可拔,引去。及還京,宣宗大獎勞之。改戶部員外郎,出理宣府軍餉。奏言:「朝議儲餉開平,令每軍運一石,又當以騎士護行,計所費率二石七斗而致一石。今軍民多願輸米易鹽,請捐舊例五分之二,則人自樂輸,餉足而兵不疲。」帝可之。

正統初,遷兵部郎中,從尚書王驥整飭甘肅邊務。從破敵于兀魯乃還,以貪淫事為驥所覺。驥遣通奏邊情,即疏通罪。下獄,謫廣西容山閘官。已,調東莞河泊所官。九年,都督僉事曹儉薦其有文武才,乞收用。吏部執不可。

景帝監國,以於謙、陳循薦,起兵部員外郎,守居庸關。俄進郎中。帝即位,進右副都御史。也先犯京師,別部攻居庸甚急。天大寒,通汲水灌城,水堅不得近。七日遁走,追擊破之。

景泰元年召還。時楊洪督京營,命通參軍務兼理院事。言:「諸邊報警,率由守將畏徵調,飾詐以惑朝廷,遇賊數十輒稱殺敗數千。向者德勝等門外不知斬馘幾何,而獲官者至六萬六千餘人。輦下且然,何況塞外。且韓信起自行伍,穰苴拔於寒微,宜博搜將士中如信、苴者,與議軍事。若今腰玉珥貂,皆茍全性命保爵祿之人,憎賢忌才,能言而不能行,未足與議也。」意蓋詆謙與石亨輩。謙疏辨,言:「概責邊報不實,果有警,不奏必致誤事。德勝門外官軍升級,惟武清侯石亨功次冊當先者萬九千八百餘人,及陣亡三千餘人而已,安所得六萬之多?通以為濫,宜將臣及亨等升爵削奪。有如韓信、穰苴者,乞即命指薦,並罷臣營務,俾專治部事。」疏下廷議。廷臣共言謙及石亨、楊洪實堪其任;又謂通志在滅賊,無他。帝兩解之。尋敕謙錄功,不得如從前冒濫,蓋因通言而發也。給事中覃浩等言通本以知兵用,不宜理院事,乃解其兼職。

塞上軍民多為寇所掠。通請榜諸邊能自歸者,軍免戍守三年,民復徭役終身。又請懸封爵重賞,募能擒斬也先、伯顏帖木兒、喜寧者。已,又言:「古之將帥務搜拔眾才,如知山川形勢者可使導軍,能騰高越險者可使覘敵,能風角鳥占者可使備變。今軍中未見其人,乞敕廷臣各舉所知,命總兵官楊洪、副將孫鏜同臣考驗。」詔皆行之。

宣府有警,總兵官朱謙告急。廷推都督同知范廣帥兵往,以通提督軍務。寇退,駐師懷來、宣府,以邊儲不敷,召還。六月,於謙以山西近寇,請遣大臣往鎮,楊洪亦乞遣重臣從雁門關護餉大同。帝以命通。通不欲行,請得與謙、洪俱。謙言國家多難,非臣子辭勞之日,奏乞躬往。帝不允,卒命通。通本謙所舉,而每事牴牾,人由是不直通。

二年召還,仍贊軍務。東宮改建,加太子少保。上言:「貢使攜馬四萬餘匹,宜量增價酬之。價增則後來益眾,此亦強中國弱外裔之一策。」帝以所貢馬率不堪用,若增價正墮賊計,寢通奏。四年進右都御史,贊軍務如故。

通好大言,遇人輒談兵。自陳殺賊功,求世襲武職,為給事中王竑所劾。帝釋不罪。天順初,自陳預謀迎駕,恐為石亨等所掩,乃授其二子所鎮撫。三年致仕。成化六年卒。賜祭葬如例。

羅綺,磁州人。宣德五年進士。英宗即位,授御史,按直隸、福建,有能名。

正統九年參贊寧夏軍務。踰年當代,軍民詣鎮守都御史陳鎰乞留。以聞,命復任。尋擢大理右寺丞,參贊如故。常以事劾指揮任信、陳斌。二人皆王振黨。十一年四月,信、斌訐綺不法事,下總兵官黃真覆核。真謂綺常詈宦官為「老奴」,以激怒振。召還京。法司擬贖,振改令錦衣衛再鞫。指揮同知馬順鍛煉成獄,謫戍遼東。景帝立,綺訴冤,不聽。尋用尚書於謙、金濂薦,召復故官,進右少卿,副李實使瓦剌。

上皇還,以勞擢刑部左侍郎。明年二月,出督雲南、四川軍儲。已,代寇深鎮守松潘。賊首卓勞糾他寨阿兒結等頻為寇,綺擒斬之。土官王永、高茂林、董敏相仇殺,守將不能制。綺搗永巢誅之。又敗黑虎諸塞番,斬馘三百五十。在鎮七年,威名甚震。

天順初,召為左副都御史,以功賜二品祿。御史張鵬、楊瑄劾石亨。亨謂綺與右都御史耿九疇使之,並下獄,降廣東參政。綺鞅鞅未赴。明年閏二月,綺鄉人告磁州同知龍約自京還,與綺言天子仍寵宦官,刻香木為王振形以葬。綺微笑云:「朝廷失政,致吾輩降黜。」奏上,捕綺下吏,坐死。籍其家,陳所籍財賄于文華門示百官。家屬戍邊,婦女沒入浣衣局。憲宗立,赦為民,還其資產。

時與綺先後鎮四川者,張固,字公正,新喻人。宣德八年進士。正統初,授刑科給事中。改吏科,奉命撫裕州流民。景泰改元,給事中李實請於四川行都司設鎮守大臣,乃遷固大理右少卿,鎮建昌。有政績。三年還理寺事。山東盜起,奉命督捕。適霖潦災,流人載道,固盡心振釁,盜賊弭散。還,卒於官。固在諫職敢言,大臣多被彈劾,又劾都御史陳鎰等舉屬官出身掾吏者為知府。自是掾吏不得歷知府,著為例。英宗將北征,偕同官疏諫。復辟,追念之,已卒。遣使諭祭,官其一子。子黼,仕至廣西按察使。

張瑄,字廷璽,江浦人。正統七年進士。授刑部主事,歷郎中,有能聲。

景泰時,賜敕為吉安知府。俗尚巫,迎神無休日。瑄遇諸途,設神水中。俄遘危疾,父老皆言神為祟,請復之。瑄怒,不許,疾亦愈。歲大饑,陳牒上官,不俟報,輒發廩振貸。

居八年,用薦擢廣東右布政使。廣西賊莫文章等越境陷連山,瑄擊斬之。又破陽山賊周公轉、新興賊鄧李保等。既而大藤峽賊頻陷屬邑,瑄坐停俸。成化初,韓雍平賊,錄瑄轉餉勞,賜銀幣,給俸如初。瑄按行所部,督建預備倉六十二,修陂塘圩岸四千六百,增築廣州新會諸城垣一十二。民德瑄,惟恐其去。既轉左布政使,會滿九載,當赴京,軍民相率乞留。巡撫陳濂等為之請,乃仍故任。

八年始以右副都御史巡撫福建。平賊林壽六、魏懷三等。福安、壽寧諸縣鄰江、浙,賊首葉旺、葉春等負險。瑄捕誅之,餘盡解散。帝降敕勞之,改撫河南。議事入都,陳撫流民、振滯才十八事,所司多議行。黃河水溢,瑄請振,且移王府祿米於他所,留應輸榆林餉濟荒,石取直八錢輸榆林,民稱便。

還理院事。尋遷南京刑部侍郎。久之,進尚書。二十年,星變,被劾,帝弗問。居三年,給事御史復劾之,遂落職。孝宗立,復官,致仕。張鵬,字騰霄,淶水人。景泰二年進士。授御史。上疏言:「懷利事君,人臣所戒。比每遇聖節,或進羊馬錦綺,交錯殿廷。自非貪賄,安有餘財充進奉?且陛下富有四海,豈借是足國哉?宜一切停罷,塞諂諛奔競之途。」疏凡四事,帝頗採用。出按大同、宣府,奏:「兩鎮軍士敝衣菲食,病無藥,死無棺。乞官給醫藥、棺槥,設義塚,俾饗歷祭。死者蒙恩,則生者勸。」帝立報可,且命諸邊概行之。奏停淮、揚征賦,給牛種。

天順元年,同官楊瑄劾石亨、曹吉祥。鵬亦偕劉泰、魏瀚、康驥論劾。俱得罪,下詔獄。諸御史多謫官,而鵬、瑄戍遼東。頃之赦免,復戍南丹。憲宗立,廷臣交薦,召復原官。尋超擢福建按察使。

成化四年,以右僉都御史巡撫廣西,剿蠻寇有功。其冬罷巡撫官,命還理南京都察院事。改督漕運,兼撫淮、揚四府。尋解漕務,專理巡撫事。復還南院,進副都御史,巡撫寧夏。召還,歷兵部左、右侍郎。

十八年代陳鉞為兵部尚書。守珠池宦官韋助乞往來高、肇、瓊、廉,會守巡官捕寇。鵬執不可,帝竟許之。南北印馬,率遣勛臣、內侍,後以災傷止遣御史。是年,帝復欲遣內侍,鵬等執不可。帝勉從之,命俟後仍如故事。鎮守大同中官汪直言小王子將大舉,請發京兵援。鵬等言:「大同士馬四萬已足用,所請宜勿許。且京軍困營造,精力銷沮,猝有急,何以作威厲氣,請悉停其役。」詔可。尋加太子少保。

鵬初為御史,剛直尚氣節,有盛名。後揚歷中外,惟事安靜。群小竊柄,閣臣萬安、劉吉輩專營私,鵬循職而已,不能有所匡救。二十一年,星變,鵬偕僚屬言:「傳奉武職至八百餘人,乞悉令閒住,非軍功毋濫授。四方鎮守、監槍、守備內官,非正統間原設者,悉宜召還。」廷臣亦交以請,下兵部復核。鵬畏中官,不敢堅其議,帝遂盡留之。時論皆咎鵬。奸民章瑾獻珍寶,得為錦衣鎮撫。理刑缺,鵬所上不允。知帝意屬瑾,即推用焉。臺諫劾大臣不職者多及鵬,鵬力求去,遂賜敕給驛以歸。弘治四年卒。謚懿簡。

李裕,字資德,豐城人。景泰五年進士。授御史。天順中,巡按陜西,上安邊八事。石彪濫報首功,詔裕核實。彪從父亨以書抵裕,裕焚之,以實聞。亨亦旋敗。由是有強直聲。都御史寇深遇僚屬嚴,惟裕不為屈。

以才擢山東按察使。重囚二百餘人,或經十餘年未判,裕旬月間決遣殆盡。大峴山賊寨七十餘,裕捕戮其魁,縱脅從,除其逋負,亂遂平。

成化初,遷陜西左布政使,入為順天府尹。政聲大著。進右副都御史,總督漕運兼巡撫江北諸府。浚白塔、孟瀆二河以便漕。張秋南旺及淮安西湖舊編木捍衝激,勞費無已。裕與郎中楊恭等謀,易以石,遂為永利。淮、鳳方饑,而太僕征預備馬二萬匹。裕論罷之。在淮六歲,每歲入計事,陳利病,多施行。父憂歸,服除,留佐院事。

十九年代戴縉為右都御史。縉附汪直,嘗請復立西廠者也,在臺綱紀不立。裕欲振之。御史有過,或遭箠撻,由是得謗。汪直敗,偕副都御史屠滽請雪諸忤直得罪者。帝不悅,奪俸。又坐累,調南京都察院。考績赴都,留為工部尚書。

初,吏部尚書尹旻罷,耿裕代之。以持正不為萬安所喜。而李孜省方貴幸用事,欲引鄉人,乃協謀去耿裕,以裕代之。裕本廉介負時望,以孜省故,名頗損。其銓敘亦平。故事,考察目有四:曰老疾,曰罷軟,曰貪酷,曰不謹。裕言:「人材質不同。偏執類酷,遲鈍類軟。乞立『才力不及』一途,以寓愛惜人才之意。」帝善之,遂著為令。考宗立,言官交章劾裕進由孜省。裕不平,為《辨誣錄》,連疏乞休去。正德中卒,年八十八。

贊曰:王彰等或以性行未純,為時訾議。綜其生平,瑕瑜互見。然揚歷中外,勞績多有可紀。《書》稱「與人不求備」,《春秋》之義「善善長」,則諸人固不失為國家幹濟材歟。


\end{pinyinscope}