\article{列傳第一百十四 海瑞何以尚 丘橓 呂坤 郭正域}

\begin{pinyinscope}
海瑞,字汝賢,瓊山人。舉鄉試。入都,即伏闕上《平黎策》,欲開道置縣,以靖鄉土。識者壯之。署南平教諭。御史詣學宮,屬吏咸伏謁,瑞獨長揖,曰:「臺謁當以屬禮,此堂,師長教士地,不當屈。」遷淳安知縣。布袍脫粟,令老僕藝蔬自給。總督胡宗憲嘗語人曰:「昨聞海令為母壽,市肉二斤矣。」宗憲子過淳安,怒驛吏,倒懸之。瑞曰:「曩胡公按部,令所過毋供張。今其行裝盛,必非胡公子。」發雚金數千,納之庫,馳告宗憲,宗憲無以罪。都御史鄢懋卿行部過,供具甚薄,抗言邑小不足容車馬。懋卿恚甚。然素聞瑞名,為斂威去,而屬巡鹽御史袁淳論瑞及慈谿知縣霍與瑕。與瑕,尚書韜子,亦抗直不諂懋卿者也。時瑞已擢嘉興通判,坐謫興國州判官。久之,陸光祖為文選,擢瑞戶部主事。

時世宗享國日久,不親朝,深居西苑,專意齋醮。督撫大吏爭上符瑞,禮官輒表賀。廷臣自楊最、楊爵得罪後。,無敢言時政者。四十五年二月,瑞獨上疏曰:

臣聞君者天下臣民萬物之主也,其任至重。欲稱其任,亦惟以責寄臣工,使盡言而已。臣請披瀝肝膽,為陛下陳之。

昔漢文帝賢主也,賈誼猶痛哭流涕而言。非苛責也,以文帝性仁而近柔,雖有及民之美,將不免於怠廢,此誼所大慮也。陛下天資英斷,過漢文遠甚。然文帝能充其仁恕之性,節用愛人,使天下貫朽粟陳,幾致刑措。陛下則銳精未久,妄念牽之而去,反剛明之質而誤用之。至謂遐舉可得,一意修真,竭民脂膏,濫興土木,二十餘年不視朝,法紀弛矣。數年推廣事例,名器濫矣。二王不相見,人以為薄於父子。以猜疑誹謗戮辱臣下,人以為薄於君臣。樂西苑而不返,人以為薄於夫婦。吏貪官橫,民不聊生,水旱無時,盜賊滋熾。陛下試思今日天下,為何如乎?

邇者嚴嵩罷相,世蕃極刑,一時差快人意。然嵩罷之後,猶嵩未相之前而已,世非甚清明也,不及漢文帝遠甚。蓋天下之人不直陛下久矣。古者人君有過,賴臣工匡弼。今乃修齋建醮,相率進香,仙桃天藥,同辭表賀。建宮築室,則將作竭力經營;購香市寶,則度支差求四出。陛下誤舉之,而諸臣誤順之,無一人肯為陛下正言者,諛之甚也。然愧心餒氣,退有後言,欺君之罪何如!

夫天下者,陛下之家,人未有不顧其家者,內外臣工皆所以奠陛下之家而磐石之者也。一意修真,是陛下之心惑。過於苛斷,是陛下之情偏。而謂陛下不顧其家,人情乎?諸臣徇私廢公,得一官多以欺敗,多以不事事敗,實有不足當陛下意者。其不然者,君心臣心偶不相值也,而遂謂陛下厭薄臣工,是以拒諫。執一二之不當,疑千百之皆然,陷陛下於過舉,而恬不知怪,諸臣之罪大矣。《記》曰「上人疑則百姓惑,下難知則君長勞」,此之謂也。

且陛下之誤多矣,其大端在於齋醮。齋醮所以求長生也。自古聖賢垂訓,修身立命曰「順受其正」矣,未聞有所謂長生之說。堯、舜、禹、湯、文、武,聖之盛也,未能久世,下之亦未見方外士自漢、唐、宋至今存者。陛下受術於陶仲文,以師稱之。仲文則既死矣,彼不長生,而陛下何獨求之?至於仙桃天藥,怪妄尤甚。昔宋真宗得天書於乾祐山,孫奭曰:「天何言哉?豈有書也!」桃必採而後得,藥必製而後成。今無故獲此二物,是有足而行耶?曰天賜者,有手執而付之耶?此左右奸人,造為妄誕以欺陛下,而陛下誤信之,以為實然,過矣。

陛下將謂懸刑賞以督責臣下,則分理有人,天下無不可治,而修真為無害已乎?太甲曰:「有言逆于汝心,必求諸道;有言遜于汝志,必求諸非道。」用人而必欲其唯言莫違,此陛下之計左也。既觀嚴嵩,有一不順陛下者乎?昔為同心,今為戮首矣。梁材守道守官,陛下以為逆者也,歷任有聲,官戶部者至今首稱之。然諸臣寧為嵩之順,不為材之逆,得非有以窺陛下之微,而潛為趨避乎?即陛下亦何利於是。

陛下誠知齋齋無益,一旦翻然悔悟,日御正朝,與宰相、侍從、言官講求天下利害,洗數十年之積誤,置身於堯、舜、禹、湯、文、武之間,使諸臣亦得自洗數十年阿君之恥,置其身於皋、夔、伊、傅之列,天下何憂不治,萬事何憂不理。此在陛下一振作間而已。釋此不為,而切切於輕舉度世,敝精勞神,以求之於繫風捕影、茫然不可知之域,臣見勞苦終身,而終於無所成也。今大臣持祿而好諛,小臣畏罪而結舌,臣不勝憤恨。是以冒死,願盡區區,惟陛下垂聽焉。

帝得疏,大怒,抵之地,顧左右曰:「趣執之,無使得遁!」宦官黃錦在側曰:「此人素有癡名。聞其上疏時,自知觸忤當死,市一棺,訣妻子,待罪於朝,僮僕亦奔散無留者,是不遁也。」帝默然。少頃復取讀之,日再三,為感動太息,留中者數月。嘗曰:「此人可方比干,第朕非紂耳。」會帝有疾,煩懣不樂,召閣臣徐階議內禪,因曰:「海瑞言俱是。朕今病久,安能視事。」又曰:「朕不自謹惜,致此疾困。使朕能出御便殿,豈受此人詬詈耶?」遂逮瑞下詔獄,究主使者。尋移刑部,論死。獄上,仍留中。戶部司務何以尚者,揣帝無殺瑞意,疏請釋之。帝怒,命錦衣衛杖之百,錮詔獄,晝夜搒訊。越二月,帝崩,穆宗立,兩人並獲釋。

帝初崩,外庭多未知。提牢主事聞狀,以瑞且見用,設酒饌款之。瑞自疑當赴西市,恣飲啖,不顧。主事因附耳語:「宮車適晏駕,先生今即出大用矣。」瑞曰:「信然乎?」即大慟,盡嘔出所飲食,隕絕於地,終夜哭不絕聲。既釋,復故官。俄改兵部。擢尚寶丞,調大理。

隆慶元年,徐階為御史劉康所劾,瑞言:「階事先帝,無能救於神仙土木之誤,畏威保位,誠亦有之。然自執政以來,憂勤國事,休休有容,有足多者。康乃甘心鷹犬,捕噬善類,其罪又浮於高拱。」人韙其言。

歷兩京左、右通政。三年夏,以右僉都御史巡撫應天十府。屬吏憚其威,墨者多自免去。有勢家朱丹其門,聞瑞至,黝之。中人監織造者,為減輿從。瑞銳意興革,請浚吳淞、白茆,通流入海,民賴其利。素疾大戶兼并,力摧豪強,撫窮弱。貧民田入於富室者,率奪還之。徐階罷相里居,按問其家無少貸。下令飆發凌厲,所司惴惴奉行,豪有力者至竄他郡以避。而奸民多乘機告訐,故家大姓時有被誣負屈者。又裁節郵傳冗費。士大夫出其境率不得供頓,由是怨頗興。都給事中舒化論瑞,滯不達政體,宜以南京清秩處之,帝猶優詔獎瑞。已而給事中戴鳳翔劾瑞庇奸民,魚肉搢紳,沽名亂政,遂改督南京糧儲。瑞撫吳甫半歲。小民聞當去,號泣載道,家繪像祀之。將履新任,會高拱掌吏部,素銜瑞,并其職於南京戶部,瑞遂謝病歸。

萬曆初,張居正當國,亦不樂瑞,令巡按御史廉察之。御史至山中視,瑞設雞黍相對食,居舍蕭然,御史歎息去。居正憚瑞峭直,中外交薦,卒不召。十二年冬,居正已卒,吏部擬用左通政。帝雅重瑞名,畀以前職。明年正月,召為南京右僉都御史,道改南京吏部右侍郎,瑞年已七十二矣。疏言衰老垂死,願比古人尸諫之義,大略謂:「陛下勵精圖治,而治化不臻者,貪吏之刑輕也。諸臣莫能言其故,反借待士有禮之說,交口而文其非。夫待士有禮,而民則何辜哉?」因舉太祖法剝皮囊草及洪武三十年定律枉法八十貫論絞,謂今當用此懲貪。其他規切時政,語極剴切。獨勸帝虐刑,時議以為非。御史梅鵾祚劾之。帝雖以瑞言為過,然察其忠誠,為奪鵾祚俸。

帝屢欲召用瑞,執政陰沮之,乃以為南京右都御史。諸司素偷惰,瑞以身矯之。有御史偶陳戲樂,欲遵太祖法予之杖。百司惴恐,多患苦之。提學御史房寰恐見糾擿,欲先發,給事中鐘宇淳復慫恿,寰再上疏醜詆。瑞亦屢疏乞休,慰留不允。十五年,卒官。

瑞無子。卒時,僉都御史王用汲入視,葛幃敝籝,有寒士所不堪者。因泣下,醵金為斂。小民罷市。喪出江上,白衣冠送者夾岸,酹而哭者百里不絕。贈太子太保,謚忠介。

瑞生平為學,以剛為主,因自號剛峰,天下稱剛峰先生。嘗言:「欲天下治安,必行井田。不得已而限田,又不得已而均稅,尚可存古人遺意。」故自為縣以至巡撫,所至力行清丈,頒一條鞭法。意主於利民,而行事不能無偏云。

始救瑞者何以尚,廣西興業人,起家鄉舉。出獄,擢光祿丞。又以劾高拱坐謫。拱罷,起雷州推官,終南京鴻臚卿。

丘橓,字茂實,諸城人。嘉靖二十九年進士。由行人擢刑科給事中。三十四年七月,倭六七十人失道流劫,自太平直逼南京。兵部尚書張時徹等閉城不敢出,閱二日引去。給事御史劾時徹及守備諸臣罪,時徹亦上其事,詞多隱護。舜劾其欺罔,時徹及侍郎陳洙皆罷。帝久不視朝,嚴嵩專國柄。橓言權臣不宜獨任,朝綱不宜久弛,嚴嵩深憾之。已,劾嵩黨寧夏巡撫謝淮、應天府尹孟淮貪黷,謝淮坐免。是年,嵩敗,舜劾由嵩進者順天巡撫徐紳等五人,帝為黜其三。遷兵科都給事中。劾南京兵部尚書李遂、鎮守兩廣平江伯陳王謨、錦衣指揮魏大經咸以賄進,大經下吏,王謨革任。已,又劾罷浙江總兵官盧鏜。寇犯通州,總督楊選被逮。及寇退,橓偕其僚陳善後事宜,指切邊弊。帝以橓不早劾選,杖六十,斥為民,餘謫邊方雜職。橓歸,敝衣一篋,圖書一束而已。隆慶初,起任禮科,不至。尋擢南京太常少卿,進大理少卿。病免。神宗立,言官交薦。張居正惡之,不召。

萬曆十一年秋,起右通政。未上,擢左副都御史,以一柴車就道。既入朝,陳吏治積弊八事,言:

臣去國十餘年,士風漸靡,吏治轉汙,遠近蕭條,日甚一日。此非世運適然,由風紀不振故也。如京官考滿,河南道例書稱職。外吏給由,撫按官概與保留。以朝廷甄別之典,為人臣交市之資。敢徇私而不敢盡法,惡無所懲,賢亦安勸?此考績之積弊,一也。

御史巡方,未離國門,而密屬之姓名,已盈私牘。甫臨所部,而請事之竿牘,又滿行臺。以豸冠持斧之威,束手俯眉,聽人頤指。此請托之積弊,二也。

撫按定監司考語,必托之有司。有司則不顧是非,侈加善考,監司德且畏之。彼此結納,上下之分蕩然。其考守令也亦如是。此訪察之積弊,三也。

貪墨成風,生民塗炭,而所劾罷者大都單寒軟弱之流。茍百足之蟲,傅翼之虎,即贓穢狼籍,還登薦剡。嚴小吏而寬大吏,詳去任而略見任。此舉劾之積弊,四也。

懲貪之法在提問。乃豺狼見遺,狐狸是問,徒有其名。或陰縱之使去,或累逮而不行,或批駁以相延,或朦朧以幸免。即或終竟其事,亦必博長厚之名,而以盡法自嫌。苞苴或累萬金,而贓止坐之銖黍。草菅或數十命,而罰不傷其毫釐。此提問之積弊,五也。

薦舉糾劾,所以勸儆有司也。今薦則先進士而舉監,非有憑藉者不與焉。劾則先舉監而進士,縱有訾議者罕及焉。晉接差委,專計出身之途。於是同一官也,不敢接席而坐,比肩而行。諸人自分低昂,吏民觀瞻頓異。助成驕縱之風,大喪賢豪之氣。此資格之積弊,六也。

州縣佐貳雖卑,亦臨民官也,必待以禮,然後可責以法。今也役使譴訶,無殊輿隸。獨任其污黷害民,不屑禁治。禮與法兩失之矣。學校之職,賢才所關,今不問職業,而一聽其所為。及至考課,則曰「此寒官也」,概與上考。若輩知上官不我重也,則因而自棄;知上官必我憐也,又從而日偷。此處佐貳教職之積弊,七也。

科場取士,故有門生、座主之稱。若巡按,舉劾其職也。乃劾者不任其怨,舉者獨冒為恩。尊之為舉主,而以門生自居,筐篚問遺,終身不廢。假明揚之典,開賄賂之門,無惑乎清白之吏不概見於天下也。方今國與民俱貧,而官獨富。既以官而得富,還以富而市官。此餽遺之積弊,八也。

要此八者,敗壞之源不在於外,從而轉移亦不在於下也。昔齊威王烹一阿大夫,封一即墨大夫,而齊國大治。陛下誠大奮乾剛,痛懲吏弊,則風行草偃,天下可立治矣。

疏奏,帝稱善。敕所司下撫按奉行,不如詔者罪。

頃之,言:「故給事中魏時亮、周世選,御史張檟、李復聘以忤高拱見黜,文選郎胡汝桂以忤尚書被傾,宜賜甄錄。御史于應昌構陷劉臺與王宗載同罪,宗載遺戍而應昌止罷官。勞堪巡撫福建,殺侍郎洪朝選。御史張一鯤監應天鄉試,王篆子之鼎夤緣中式。錢岱監湖廣鄉試,先期請居正少子還就試,會居正卒不果,遂私中篆子之衡。曹一夔身居風憲,盛稱馮保為顧命大臣。硃璉則結馮保為父,游七為兄。此數人者,得罪名教,而亦止罷官。此綱紀所以不振,人心所以不服。臣初八臺,誓掃除積弊。今待罪三月,而大吏恣肆,小吏貪殘,小民怨咨,四方賂遺如故,臣不職可見。請罷斥以儆有位。」時已遷刑部右侍郎。帝優詔報之。召時亮、世選、檟、復聘、汝桂還,削慶昌、堪、一鯤、一夔、璉籍,貶岱三秩。未幾,偕中官張誠往籍張居正家。還,轉左侍郎,增俸一秩。尋拜南京吏部尚書,卒官。贈太子太保,謚簡肅。

橓彊直好搏擊,其清節為時所稱云。

呂坤,字叔簡,寧陵人。萬曆二年進士。為襄垣知縣,有異政。調大同,徵授戶部主事,歷郎中。遷山東參政、山西按察使、陜西右布政使。擢右僉都御史,巡撫山西。居三年,召為左僉都御史。歷刑部左、右侍郎。

二十五年五月,書疏陳天下安危,其略曰:

竊見元旦以來,天氣昏黃,日光黯淡,占者以為亂徵。今天下之勢,亂象已形,而亂勢未動。天下之人,亂心已萌,而亂人未倡。今日之政,皆播亂機使之動,助亂人使之倡者也。臣敢以救時要務,為陛下陳之。

自古幸亂之民有四。一曰無聊之民。飽溫無由,身家俱困,因懷逞亂之心,冀緩須臾之死。二曰無行之民。氣高性悍,玩法輕生,居常愛玉帛子女而不得,及有變則淫掠是圖。三曰邪說之民。白蓮結社,遍及四方,教主傳頭,所在成聚。倘有招呼之首,此其歸附之人。四曰不軌之民。乘釁蹈機,妄思雄長。惟冀目前有變,不樂天下太平。陛下約己愛人,損上益上,則四民皆赤子,否則悉為寇仇。

今天下之蒼生貧困可知矣。自萬曆十年以來,無歲不災,催科如故。臣久為外吏,見陛下赤子凍骨無兼衣,饑腸不再食,垣舍弗蔽,苫槁未完;流移日眾,棄地猥多;留者輸去者之糧,生者承死者之役。君門萬里,孰能仰訴?今國家之財用耗竭可知矣。數年以來,壽宮之費幾百萬,織造之費幾百萬,寧夏之變幾百萬,黃河之潰幾百萬,今大工、採木費,又各幾百萬矣。土不加廣,民不加多,非有雨菽湧金,安能為計?今國家之防禦疏略可知矣。三大營之兵以衛京師也,乃馬半羸敝,人半老弱。九邊之兵以禦外寇也,皆勇於挾上,怯於臨戎。外衛之兵以備徵調資守禦也,伍缺於役占,家累於需求,皮骨僅存,折衝奚賴?設有千騎橫行,兵不足用,必選民丁。以怨民鬥怨民,誰與合戰?

人心者,國家之命脈也。今日之人心,惟望陛下收之而已。關隴氣寒土薄,民生實艱。自造花絨,比戶困趣逼。提花染色,日夜無休,千手經年,不成一匹。他若山西之紬,蘇、松之錦綺,歲額既盈,加造不已。至饒州磁器,西域回青,不急之須,徒累小民敲骨。陛下誠一切停罷,而江南、陜西之人心收矣。

以採木言之。丈八之圍,非百年之物。深山窮谷,蛇虎雜居,毒霧常多,人煙絕少,寒暑饑渴瘴癘死者無論矣。乃一木初臥,千夫難移,倘遇阻艱,必成傷殞。蜀民語曰:「入山一千,出山五百」。哀可知也。至若海木,官價雖一株千兩,比來都下,為費何止萬金!臣見楚、蜀之人,談及採木,莫不哽咽。茍損其數,增其直,多其歲月,減其尺寸,而川、貴、湖廣之人心收矣。

以採礦言之。南陽諸府,比歲饑荒。生氣方蘇,菜色未變。自責報殷戶,是半已驚逃。自供應礦夫工食、官兵口糧,而多至累死。自都御史李盛春嚴旨切責,而撫按畏罪不敢言。今礦沙無利,責民納銀,而奸人仲春復為攘奪侵漁之計。朝廷得一金,郡縣費千倍。誠敕戒使者,毋散砂責銀,有侵奪小民若仲春者,誅無赦,而四方之人心收矣。

宮店租銀收解,自趙承勛造四千之說,而皇店開。自朝廷有內官之遣,而事權重。夫市井之地,貧民求升合絲毫以活身家者也,陛下享萬方之富,何賴於彼?且馮保八店,為屋幾何,而歲有四千金之課。課既四千,徵收何止數倍。不奪市民,將安取之?今豪家遣僕設肆,居民尚受其殃,況特遣中貴,賜之敕書,以壓卵之威,行竭澤之計,民困豈顧問哉?陛下撤還內臣,責有司輸課,而畿甸之人心收矣。

天下宗室,皆九廟子孫。王守仁、王錦襲蓋世神奸,藉隔數千里,而冒認王弼子孫;事隔三百年,而妄稱受寄財產。中間偽造絲綸,假傳詔旨,明欺聖主,暗陷親王,有如楚王銜恨自殺,陛下何辭以謝高皇帝之靈乎?此兩賊者,罪應誅殛,乃止令回籍,臣恐萬姓驚疑。誠急斬二賊以謝楚王,而天下宗籓之心收矣。

崇信伯費甲金之貧,十廂珠寶之誣,皆通國所知也。始誤於科道之風聞,嚴追猶未為過。今真知其枉,又加禁錮,實害無辜。請還甲金革去之祿,復五城廠衛降斥之官,而勛戚之人心收矣。

法者所以平天下之情。其輕其重,太祖既定為律,列聖又增為例。如輕重可以就喜怒之情,則例不得為一定之法。臣待罪刑部三年矣,每見詔獄一下,持平者多拂上意,從重者皆當聖心。如往年陳恕、王正甄、常照等獄,臣等欺天罔人,已自廢法,陛下猶以為輕,俱加大辟。然則律例又安用乎!誠俯從司寇之平,勉就祖宗之法,而囹圄之人心收矣。

自古聖明之君,豈樂誹謗之語。然而務求言賞諫者,知天下存亡,係言路通塞也。比來驅逐既多,選補皆罷。天閽邃密,法座崇嚴,若不廣達四聰,何由明照萬里?今陛下所聞,皆眾人之所敢言也,其不敢言者,陛下不得聞矣。一人孤立萬乘之上,舉朝無犯顏逆耳之人,快在一時,憂貽他日。陛下誠釋曹學程之繫,還吳文梓等官,凡建言得罪者,悉分別召用,而士大夫之心收矣。

朝鮮密邇東陲,近吾肘腋,平壤西鄰鴨綠,晉州直對登、萊。倘倭夷取而有之,籍眾為兵,就地資食,進則斷我漕運,退則窺我遼東。不及一年,京城坐困,此國家大憂也。乃彼請兵而二三其說,許兵而延緩其期;力窮勢屈,不折入為倭不止。陛下誠早決大計,并力東征,而屬國之人心收矣。

四方輸解之物,營辦既苦,轉運尤艱。及入內庫,率至朽爛,萬姓脂膏,化為塵土。倘歲一稽核,苦窳者嚴監收之刑,朽腐者重典守之罪。一整頓間,而一年可備三年之用,歲省不下百萬,而輸解之人心收矣。

自抄沒法重,株連數多。坐以轉寄,則並籍家資。誣以多贓,則互連親識。宅一封而雞豚大半餓死,人一出則親戚不敢藏留。加以官吏法嚴,兵番搜苦,少年婦女,亦令解衣。臣曾見之,掩目酸鼻。此豈盡正犯之家、重罪之人哉?一字相牽,百口難解。奸人又乘機恐嚇,挾取資財,不足不止。半年之內,擾遍京師,陛下知之否乎?願慎抄沒之舉,釋無辜之繫,而都下之人心收矣。

列聖在御之時,豈少宦官宮妾,然死於箠楚者,未之多聞也。陛下數年以來,疑深怒盛。廣廷之中,狼籍血肉,宮禁之內,慘戚啼號。厲氣冤魂,乃聚福祥之地。今環門守戶之眾,皆傷心側目之人。外表忠勤,中藏憸毒。既朝暮不能自保,即九死何愛一身。陛下臥榻之側,同心者幾人?暮夜之際,防患者幾人?臣竊憂之。願少霽威嚴,慎用鞭撲,而左右之人心收矣。

祖宗以來,有一日三朝者,有一日一朝者。陛下不視朝久,人心懈弛已極,奸邪窺伺已深,守衛官軍祇應故事。今乾清修造,逼近御前,軍夫往來,誰識面貌?萬一不測,何以應之?臣望發宮鑰於質明,放軍夫於日昃。自非軍國急務,慎無昏夜傳宣。章奏不答,先朝未有。至於今日,強半留中。設令有國家大事,邀截實封,揚言於外曰「留中矣」,人知之乎?願自今章疏未及批答者,日於御前發一紙,下會極門,轉付諸司照察,庶君臣雖不面談,而上下猶無欺蔽。

臣觀陛下昔時勵精為治,今當春秋鼎盛,曾無夙夜憂勤之意,惟孜孜以患貧為事。不知天下之財,止有此數,君慾富則天下貧,天下貧而君豈獨富?今民生憔悴極矣,乃採辦日增,誅求益廣,斂萬姓之怨於一言,結九重之仇於四海,臣竊痛之。使六合一家,千年如故,即宮中虛無所有,誰忍使陛下獨貧?今禁城之內,不樂有君。天下之民,不樂有生。怨讟愁歎,難堪入聽。陛下聞之,必有食不能咽,寢不能安者矣。臣老且衰,恐不得復見太平,籲天叩地,齋宿七日,敬獻憂危之誠。惟陛下密行臣言,翻然若出聖心警悟者,則人心自悅,天意自回。茍不然者,陛下他日雖悔,將何及耶!

疏入,不報。坤遂稱疾乞休,中旨許之。於是給事中戴士衡劾坤機深志險,謂石星大誤東事,孫鑛濫殺不辜,坤顧不言,曲為附會,無大臣節。給事中劉道亨言往年孫丕揚劾張位,位疑疏出坤手,故使士衡劾坤。位奏辨。帝以坤既罷,悉置不問。

初,坤按察山西時,嘗撰《閨範圖說》,內侍購入禁中。鄭貴妃因加十二人,且為製序,屬其伯父承恩重刊之。士衡遂劾坤因承恩進書,結納宮掖,包藏禍心。坤持疏力辨。未幾,有妄人為《閨範圖說》跋,名曰《憂危竑議》,略言:「坤撰《閨範》,獨取漢明德后者,后由貴人進中宮,坤以媚鄭貴妃也。坤疏陳天下憂危,無事不言,獨不及建儲,意自可見。」其言絕狂誕,將以害坤。帝歸罪於士衡等,其事遂寢。

坤剛介峭直,留意正學。居家之日,與後進講習。所著述,多出新意。初,在朝與吏部尚書孫丕揚善。後丕揚復為吏部,屢推坤左都御史未得命,言:「臣以八十老臣保坤,冀臣得親見用坤之效。不效,甘坐失舉之罪,死且無憾。」已,又薦天下三大賢,沈鯉、郭正域,其一即坤。丕揚前後推薦,疏至二十餘上,帝終不納。福王封國河南,賜莊田四萬頃。坤在籍,上言:「國初分封親籓二十有四,賜田無至萬頃者。河南已封周、趙、伊、徽、鄭、唐、崇、潞八王,若皆取盈四萬,占兩河郡縣且半,幸聖明裁減。」復移書執政言之。會廷臣亦力爭,得減半。卒,天啟初,贈刑部尚書。

郭正域,字美命,江夏人。萬曆十一年進士。選庶吉士,授編修,與修撰唐文獻同為皇長子講官。皆三遷至庶子,不離講帷。每講畢,諸內侍出相揖,惟二人不交一言。

出為南京祭酒。諸生納貲許充貢,正域奏罷之。李成梁孫以都督就婚魏國徐弘基家,騎過文廟門,學錄李維極執而抶之。李氏蒼頭數十人蹋邸門,弘基亦至。正域曰:「今天子尚皮弁拜先聖,人臣乃走馬廟門外乎?且公侯子弟入學習禮,亦國子生耳,學錄非抶都督也。」令交相謝而罷。

三十年,徵拜詹事,復為東宮講官。旋擢禮部右侍郎,掌翰林院。三十一年三月,尚書馮琦卒,正域還署部事。夏,廟饗,會日食,正域言:「《禮》,當祭日食,牲未殺,則廢。朔旦宜專救日,詰朝享廟。」從之。方澤陪祀者多托疾。正域謂祀事不虔,由上不躬祀所致。請下詔飭厲,冬至大祀,上必親行。帝然之,而不能用。

初,正域之入館也,沈一貫為教習師。後服闋授編修,不執弟子禮,一貫不能無望。至是,一貫為首輔,沈鯉次之。正域與鯉善,而心薄一貫。會臺官上日食占,曰:「日從上食,占為君知佞人用之,以亡其國。」一貫怒而詈之,正域曰:「宰相憂盛危明,顧不若瞽史邪?」一貫聞之怒。兩淮稅監魯保請給關防,兼督江南、浙江織造,鯉持不可,一貫擬予之,正域亦力爭。秦王以嫡子夭未生,請封其庶長子為世子,屢詔趣議。前尚書馮琦持不上,正域亦執不許。王復請封其他子為郡王,又不可。一貫使大璫以上命脅之,正域榜於門曰:「秦王以中尉進封,庶子當仍中尉,不得為郡王。妃年未五十,庶子亦不得為世子。」一貫無以難。及建議欲奪黃光昇、許論、呂本謚,一貫與朱賡皆本同鄉也,曰:「我輩在,誰敢奪者!」正域援筆判曰:「黃光昇當謚,是海瑞當殺也。許論當謚,是沈煉當殺也。呂本當謚,是鄢懋卿、趙文華皆名臣,不當削奪也。」議上,舉朝韙之,而卒不行。

正域既積忤一貫,一貫深憾之。會楚王華奎與宗人華勣等相訐,正域復與一貫異議,由此幾得危禍。先是,楚恭王得廢疾,隆慶五年薨,遺腹宮人胡氏孿生子華奎、華壁。或云內官郭綸以王妃兄王如言妾尤金梅子為華奎,妃族人如糸孛奴王玉子為華壁。儀賓汪若泉嘗訐奏之,事下撫按。王妃持甚堅,得寢。萬曆八年,華奎嗣王,華壁亦封宣化王。宗人華勣者,素強禦忤王。華勣妻,如言女也。是年遣人訐華奎異姓子也,不當立。一貫屬通政使沈子木格其疏勿上。月餘楚王劾華勣疏至,乃上之。命下部議。未幾,華勣入都訴通政司邀截實封及華奎行賄狀,楚宗與名者,凡二十九人。子木懼,召華勣令更易月日以上。旨并下部。正域請敕撫按公勘,從之。

初,一貫屬正域毋言通政司匿疏事。及華勣疏上,正域主行勘。一貫言親王不當勘,但當體訪。正域曰:「事關宗室,臺諫當亦言之。」一貫微笑曰:「臺諫斷不言也。」及帝從勘議,楚王懼,奉百金為正域壽,且屬毋竟楚事,當酬萬金,正域嚴拒之。已而湖廣巡撫趙可懷、巡按應朝卿勘上,言詳審無左驗,而王氏持之堅,諸郡主縣主則云「罔知真偽」,乞特遣官再問。詔公卿雜議於西闕門,日晏乃罷。議者三十七人,各具一單,言人人殊。李廷機以左侍郎代正域署部事,正域欲盡錄諸人議,廷機以辭太繁,先撮其要以上。一貫遂嗾給事中楊應文、御史康丕揚劾禮部壅閼群議,不以實聞。正域疏辨,且發子木匿疏、一貫阻勘及楚王餽遺狀。一貫益恚,謂正域遣家人導華勣上疏,議令楚王避位聽勘,私庇華勣。

當是時,正域右宗人,大學士沈鯉右正域,尚書趙世卿、謝傑、祭酒黃汝良則右楚王。給事中錢夢皋遂希一貫指論正域,詞連次輔鯉。應文又言正域父懋嘗笞辱於楚恭王,故正域因事陷之。正域疏辨,留中不報。一貫、鯉以楚事皆求去,廷機復請再問。帝以王嗣位二十餘年,何至今始發,且夫訐妻證,不足憑,遂罷楚事勿按。正域四疏乞休去。楚王既得安,遂奏劾正域,大略如應文言;且訐其不法數事,請褫正域官。詔下部院集議。廷機微刺正域,而謂其已去,可無苛求。給事中張問達則謂籓王欲進退大臣,不可訓,乃不罪正域,而令巡按御史勘王所訐以聞。

俄而妖書事起。一貫以鯉與己地相逼,而正域新罷,因是陷之,則兩人必得重禍,乃為帝言臣下有欲相傾者為之。蓋微引其端,以動帝意。亡何,錦衣衛都督王之禎等四人以妖書有名,指其同官周嘉慶為之。東廠又捕獲妖人皦生光。巡城御史康丕揚為生光訟冤,言妖書、楚事同一根柢,請少緩其獄,賊兄弟可授首闕下。意指正域及其兄國子監丞正位。帝怒,以為庇反賊,除其名。一貫力救始免。丕揚乃先後捕僧人達觀、醫者沈令譽等,而同知胡化則告妖書出教官阮明卿手。未幾,廠衛又捕可疑者一人曰毛尚文。數日間鋃鐺旁午,都城人人自危。嘉慶等皆下詔獄。嘉慶旋以治無驗,令革任回籍。令譽故嘗往來正域家,達觀亦時時游貴人門,嘗為正域所搒逐,尚文則正域僕也。一貫、丕揚等欲自數人口引正域,而化所訐阮明卿,則錢夢皋婿。夢皋大恚,上疏顯攻正域,言:「妖書刊播,不先不後,適在楚王疏入之時。蓋正域乃沈鯉門徒,而沈令譽者,正域食客,胡化又其同鄉同年,群奸結為死黨。乞窮治根本,定正域亂楚首惡之罪,勒鯉閒住。」帝令正域還籍聽勘,急嚴訊諸所捕者。達觀拷死,令譽亦幾死,皆不承。法司迫化引正域及歸德。歸德,鯉所居縣也。化大呼曰:「明卿,我仇也,故訐之。正域舉進士二十年不通問,何由同作妖書?我亦不知誰為歸德者。」帝知化枉,釋之。

都督陳汝忠掠訊尚文,遂發卒圍正域舟於楊村,盡捕媼婢及傭書者男女十五人,與生光雜治,終無所得。汝忠以錦衣告身誘尚文曰:「能告賊,即得之。」令引令譽,且以乳媼龔氏十歲女為徵。比會訊,東廠太監陳矩詰女曰:「汝見妖書版有幾?」曰:「盈屋。」矩笑曰:「妖書僅二三紙,版顧盈屋邪?」詰尚文曰:「令譽語汝刊書何日?」尚文曰:「十一月十六日。」戎政尚書王世揚曰;「妖書以初十日獲,而十六日又刊,將有兩妖書邪?」拷生光妻妾及十歲兒,以鍼刺指爪,必欲引正域,皆不應。生光仰視夢皋、丕揚,大罵曰:「死則死耳,奈何教我迎相公指,妄引郭侍郎乎?」都御史溫純等力持之,事漸解,然猶不能具獄。

光宗在東宮,數語近侍曰:「何為欲殺我好講官?」諸人聞之皆懼。詹事唐文獻偕其僚楊道賓等詣一貫爭之,李廷機亦力為之地,獄益解。刑部尚書蕭大亨具爰書,猶欲坐正域。郎中王述古抵稿於地,大亨乃止。遂坐生光極刑,釋諸波及者,而正域獲免。方獄急時,邏卒圍鯉舍及正域舟,鈴柝達旦。又聲言正域且逮,迫使自裁。正域曰:「大臣有罪,當伏尸都市,安能自屏野外?」既而幸無事,乃歸。歸三年,巡按御史史學遷勘上楚王所訐事,無狀。給事顧士琦因請召還正域,不報。

正域博通載籍,勇於任事,有經濟大略,自守介然,故人望歸之。扼於權相,遂不復起,家居十年卒。後四年,贈禮部尚書。光宗遺詔,加恩舊學,贈太子少保,謚文毅,官其子中書舍人。

贊曰:海瑞秉剛勁之性,戇直自遂,蓋可希風漢汲黯、宋包拯。苦節自厲,誠為人所難能。丘橓、呂坤,雖非瑞匹,而指陳時政,炳炳鑿鑿,鯁亮有足稱者。郭正域持楚獄,與執政異趣,險難忽發,慬而後免,危矣哉!以妖書事與坤相首尾,故並著焉。


\end{pinyinscope}