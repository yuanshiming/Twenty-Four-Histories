\article{列傳第一百六十六}

\begin{pinyinscope}
楊廷麟(彭期生等}}萬元吉楊文薦梁於水矣郭維經姚奇胤詹兆恒胡夢泰周定仍等陳泰來曹志明王養正夏萬亨等曾亨應弟和應子筠揭重熙傅鼎銓陳子壯麥而炫朱實蓮霍子衡張家玉陳象明等陳邦彥蘇觀生

楊廷麟,字伯祥,清江人。崇禎四年進士。改庶吉士,授編修,勤學嗜古,有聲館閣間,與黃道周善。十年冬,皇太子將出閣,充講官兼直經筵。延麟具疏讓道周,不許。明年二月,帝御經筵,問保舉考選何者為得人。廷麟言:「保舉當嚴舉主,如唐世濟、王維章乃溫體仁、王應熊所薦。今二臣皆敗,而舉主不問。是連坐之法先不行於大臣,欲收保舉效得乎?」帝為動色。

其冬,京師戒嚴。廷麟上疏劾兵部尚書楊嗣昌,言:「陛下有撻伐之志,大臣無禦侮之才,謀之不臧,以國為戲。嗣昌及薊遼總督吳阿衡內外扶同,朋謀誤國。與高起潛、方一藻倡和款議,武備頓忘,以至於此。今可憂在外者三,在內者五。督臣盧昇以禍國責樞臣,言之痛心。夫南仲在內,李綱無功;潛善秉成,宗澤殞命。乞陛下赫然一怒,明正向者主和之罪,俾將士畏法,無有二心。召見大小諸臣,咨以方略。諭象昇集諸路援師,乘機赴敵,不從中制。此今日急務也。」時嗣昌意主和議,冀紓外患,而廷麟痛詆之。嗣昌大恚,詭薦廷麟知兵。帝改廷麟兵部職方主事,贊畫象昇軍。象升喜,即令廷麟往真定轉餉濟師。無何,象升戰死賈莊。嗣昌意廷麟亦死,及聞其奉使在外,則為不懌者久之。

初,張若麒、沈迅官刑曹,謀改兵部,御史塗必泓沮之。必泓,廷麟同里也。兩人疑疏出廷麟指,因與嗣昌比而構廷麟。會廷麟報軍中曲折,嗣昌擬旨責以欺罔。事平,貶廷麟秩,調之外。黃道周獄起,詞連廷麟,當逮。未至而道周已釋,言者多薦廷麟。十六年秋,復授職方主事,未赴,都城失守,廷麟慟哭,募兵勤王。福王立,用御史祁彪佳薦,召為左庶子,辭不就。宗室朱統金類誣劾廷麟召健兒有不軌謀,以姜曰廣為內應。王不問,而廷麟所募兵亦散。

順治二年,南都破,江西諸郡惟贛州存。唐王手書加廷麟吏部右侍郎,劉同升國子祭酒。同升自雩都至贛,與廷麟謀大舉。乃偕巡撫李永茂集紳士於明倫堂,勸輸兵餉。九月,大兵屯泰和,副將徐必達戰敗,廷麟、同升乘虛復吉安、臨江。加兵部尚書兼東閣大學士,賜劍,便宜從事。十月,大兵攻吉安,必達戰敗,赴水死。會廣東援兵至,大兵退屯峽江。已而萬元吉至贛。十二月,同升卒。

三年正月,廷麟赴贛,招峒蠻張安等四營降之,號龍武新軍。廷麟聞王將由汀赴贛,將往迎王,而以元吉代守吉安。無何,吉安復失,元吉退保贛州。四月,大兵逼城下,廷麟遣使調廣西狼兵,而身往雩都趣新軍張安來救。五月望,安戰梅林,再敗,退保雩都。廷麟乃散其兵,以六月入贛,與元吉憑城守。未幾,援兵至,圍暫解,已,復合。八月,水師戰敗,援師悉潰。及汀州告變,贛圍已半年,守陴者皆懈。十月四日,大兵登城。廷麟督戰,久之,力不支,走西城,投水死。同守者郭維經、彭期生輩皆死。

期生,字觀我,海鹽人,御史宗孟子。登萬曆四十四年進士。崇禎初,為濟南知府,坐失囚謫布政司照磨,量移應天推官,轉南京兵部主事,進郎中。十六年,張獻忠亂江西,遷湖西兵備僉事,駐吉安。吉安不守,走贛州,偕廷麟招降張安等,加太常寺卿,仍視兵備事。城破,冠帶自縊死。

一時同殉者,職方主事周瑚,磔死。通判王明汲,編修兼兵科給事中萬發祥,吏部主事龔棻,戶部主事林琦,兵部主事王其狖、黎遂球、柳昂霄、魯嗣宗、錢謙亨,中書舍人袁從鶚、劉孟鍧、劉應試,推官署府事吳國球,監紀通判郭寧登,臨江推官胡縝,贛縣知縣林逢春,皆被戮。鄉官盧觀象盡驅男婦大小入水,乃自沉死。舉人劉日佺偕母妻弟婦子姪同日死。參將陳烈數力戰,眾以其弟已降,疑之,烈益奮勇疾鬥。及見執,不屈,顧謂贛人曰:「而後乃今知我無二心也。」遂就戮。

萬元吉,字吉人,南昌人。天啟五年進士。授潮州推官,補歸德。捕大盜李守志,散其黨。崇禎四年大計,謫官。十一年秋,用曾櫻薦,命以永州檢校署推官事。居二年,督師楊嗣昌薦其才,改大理右評事,軍前監紀。嗣昌倚若左右手,諸將亦悅服,馳驅兵間,未嘗一夕安枕。嗣昌卒,元吉丁內艱歸。十六年起南京職方主事,進郎中。

福王立,仍故官。四鎮不和,元吉請奉詔宣諭。又請發萬金犒高傑於揚州,諭以大義,令保江、淮。乃渡江詣諸將營。傑與黃得功、劉澤清方爭揚州,元吉與得功書,令共獎王室。得功報書如元吉指,乃錄其槁示澤清、傑,嫌漸解。廷議以元吉能輯諸鎮,擢太僕少卿,監視江北軍務。元吉身在外,不忘朝廷,數有條奏。請修建文實錄,復其尊稱,並還懿文追尊故號,祀之寢園,以建文配,而速褒靖難死事諸臣,及近日北都四方殉難者,以作忠義之氣。從之。又言:

先帝天資英武,銳意明作,而禍亂益滋。寬嚴之用偶偏,任議之途太畸也。

先帝初懲逆璫用事,委任臣工,力行寬大。諸臣狃之,爭意見之異同,略綢繆之桑土,敵入郊圻,束手無策。先帝震怒,宵小乘間,中以用嚴。於是廷杖告密,加派抽練,使在朝者不暇救過,在野者無復聊生,廟堂號振作,而敵強如故,寇禍彌張。十餘年來,小人用嚴之效如是。先帝亦悔,更從寬大,悉反前規,天下以為太平可致。諸臣復競賄賂,肆欺蒙,每趨愈下,再攖先帝之怒,誅殺方興,宗社繼歿。蓋諸臣之孽,每乘於先帝之寬;而先帝之嚴,亦每激於諸臣之玩。臣所謂寬嚴之用偶偏者此也。

國步艱難,於今已極。乃議者求勝於理,即不審勢之重輕;好伸其言,多不顧事之損益。殿上之彼己日爭,閫外之從違遙制,一人任事,眾口議之。如孫傳庭守關中,識者俱謂不宜輕出,而已有以逗撓議之者矣。賊既渡河,臣語史可法、姜曰廣急撤關、寧吳三桂兵,隨樞輔迎擊。先帝召對時,群臣亦曾及此,而已有以蹙地議之者矣。及賊勢燎原,廷臣或勸南幸,或勸皇儲監國南都,皆權宜善計,而已有以邪妄議之者矣。由事後而觀,咸追恨議者之誤國。倘事幸不敗,必共服議者之守經。大抵天下事,無全害亦無全利,當局者非樸誠通達,誰敢違眾獨行;旁持者競意氣筆鋒,必欲強人從我。臣所謂任議之途太畸者此也。

乞究前事之失,為後事之師,以寬為體,以嚴為用。蓋崇簡易、推真誠之謂寬,而濫賞縱罪者非寬;辨邪正、綜名實之謂嚴,而鉤距索隱者非嚴。寬嚴得濟,任議乃合。仍請於任事之人,嚴核始進,寬期後效,無令行間再踵藏垢,邊才久借然灰,收之以嚴,然後可任之以寬也。詔褒納之。

明年五月,南京覆,走福建,歸唐王。六月,我大清兵已取南昌、袁州、臨江、吉安。踰月,又取建昌。惟贛州孤懸上游,兵力單寡。會益府永寧王慈炎招降峒賊張安,所號龍武新軍者也,遣復撫州。南贛巡撫李永茂乃命副將徐必達扼泰和,拒大兵。未幾,戰敗,至萬安,遇永茂。永茂遂奔贛。

八月,叛將白之裔入萬安,江西巡撫曠昭被執,知縣梁於涘死之。於涘,江都人。崇禎十六年進士。時唐王詔適至贛,永茂乃與楊廷麟、劉同升同舉兵。未幾,王召永茂為兵部右侍郎,以張朝綖代。甫任事,擢元吉兵部右侍郎兼右副都御史,總督江西、湖廣諸軍,召朝綖還,以同升代。元吉至贛,同升已卒,遂以元吉兼巡撫。

順治三年三月,廷麟將朝王,元吉代守吉安。初,崇禎末,命中書舍人張同敞調雲南兵,至是抵江西,兩京已相繼失,因退還吉安。廷麟留與共守,用客禮待之。其將趙印選、胡一青頻立功,而元吉約束甚嚴,諸將漸不悅。時有廣東兵亦以赴援至。而新軍張安者,汀、贛間峒賊四營之一,驍勇善戰,既降,有復撫州功,且招他營盡降。元吉以新軍足恃也,蔑視雲南、廣東軍,二軍皆解體。然安卒故為賊,居贛淫掠,遣援湖西,所過殘破。及是,大兵逼吉安,諸軍皆內攜,新軍又在湖西。城中軍不戰潰,城遂破。元吉退屯皂口,檄諭贛州極言雲南兵棄城罪,其眾遂西去。四月,大兵逼皁口,元吉不能禦,入贛城。大兵乘勝抵城下。給事中楊文薦奉命湖南,過贛,入城共守禦,城中賴之。文薦,元吉門生也。

元吉素有才,蒞事精敏。及失吉安,士不用命,昏然坐城上,對將吏不交一言。隔河大營遍山麓,指為空營。兵民從大營中至,言敵勢盛,輒叱為間諜,斬之。江西巡撫劉遠生令張琮者,將兵趨湖東。及贛圍急,遠生自出城,召琮於雩都。贛人曰「撫軍遁矣」,怒焚其舟,拘遠生妻子。俄遠生率琮兵至,贛人乃大悔。琮軍渡河,抵梅林,中伏大敗,還至河,爭舟,多死於水。遠生憤甚,五月朔,渡河再戰,身先士卒,遇大兵,被獲,復逃歸。而新軍先往湖西者,聞吉安復失,仍還雩都。廷麟躬往邀之,與大兵戰梅林,再敗,乃散遣其軍,而身入城,與元吉同守。自遠生敗,援軍皆不敢前。六月望,副將吳之蕃以廣東兵五千至,圍漸解,未幾復合,城中守如初。

王聞贛圍久,獎勞之,賜名忠誠府,加元吉兵部尚書,文薦右僉都御史,使尚書郭維經來援。維經與御史姚奇胤沿途募兵,得八千人。元吉部將汪起龍率師數千,雲南援將趙印選、胡一青率師三千,大學士蘇觀生遣兵如之。兩廣總督丁魁楚亦遣兵四千。廷麟又收集散亡,得數千。先後至贛,營於城外。諸將欲戰,元吉待水師至並擊。而中書舍人來從諤募砂兵三千,吏部主事龔棻、兵部主事黎遂球募水師四千,皆屯南安,不敢下。主事王其狖謂元吉曰:「水師帥羅明受海盜也,桀驁難制,棻、遂球若慈母之奉驕子。且今水涸,臣舟難進,豈能如約。」不聽。及八月,大兵聞水師將至,即夜截諸江,焚巨舟八十,死者無算,明受遁還,舟中火藥戎器盡失。於是兩廣、雲南軍不戰而潰,他營亦稍稍散去。城中僅起龍、維經部卒四千餘人,城外僅水師後營二千餘人。參將謝志良擁眾萬餘雩都不進,廷麟調廣西狼兵八千人踰嶺,亦不即赴。會聞汀州破,人情益震懼。

十月初,大兵用向導夜登城,鄉勇猶巷戰。黎明,兵大至,城遂破,元吉死之。先是,元吉禁婦女出城。其家人潛載其妾縋城去,元吉遣飛騎追還,捶其家人,故城中無敢出者。及城破,部將擁元吉出城。元吉歎曰:「為我謝贛人,使闔城塗炭者我也,我何可獨存!」遂赴水死,年四十有四。

楊文薦,字幼宇,京山人。由進士為兵科給事中。城破時,病困不能起,執送南昌,絕粒而卒。

郭維經,字六修,江西龍泉人。天啟五年進士。授行人。崇禎三年遷南京御史,疏陳時弊,中有所舉刺。帝責令指實,乃極稱順天府尹劉宗周之賢,力詆吏部尚書王永光溪刻及用人顛倒罪,帝置不問。六年秋,溫體仁代周延儒輔政,維經言:「執政不患無才,患有才而用之排正人,不用之籌國事。國事日非,則委曰我不知,坐視盜賊日猖,邊警日急,止與二三小臣爭口舌,角是非。平章之地幾成聚訟,可謂有才邪?」帝切責之。憂去。久之,起故官。

北都變聞,南都諸臣有議立潞王者,維經力主福王。王立,進應天府丞,仍兼御史,巡視中城。俄上言:「聖明御極將二旬,一切雪恥除兇、收拾人心之事,絲毫未舉。令偽官縱橫於鳳、泗,悍卒搶攘於瓜、儀,焚戮剽掠之慘,漸逼江南,而廊廟之上不聞動色相戒,惟以慢不切要之務,盈庭而議。乞令內外文武諸臣洗滌肺腸,盡去刻薄偏私及恩怨報復故習,一以辦賊復仇為事。」報聞。尋遷大理少卿,左僉都御史。命專督五城御史,察非常,情輦轂。明年二月,隆平侯張拱日、保國公朱國弼相繼以他事劾罷維經,維經回籍。唐王召為吏部右侍郎。

順治三年五月,大兵圍贛州。王乃命維經為吏、兵二部尚書兼右副都御史,總理湖廣、江西、廣東、浙江、福建軍務,督師往援。維經與御史姚奇胤募兵八千人入贛州,與楊廷麟、萬元吉協守。及城破,維經入嵯峨寺自焚死,奇胤亦死之。

奇胤,字有僕,錢塘人。由進士授南海知縣。地富饒,多盜賊。奇胤絕苞苴,力以弭盜為事,政聲大起。入為兵部主事,改監察御史,巡按廣東。未任,與維經赴援,遂同死。

詹光恒,字月如,廣信永豐人。父士龍,順天府尹。光恒舉崇禎四年進士。由甄寧知縣徵授南京御史,疏陳盜鑄之弊,帝下所司察核。十四年夏,言燕、齊二千里間,寇盜縱橫,行旅阻絕,四方餉金滯中途者,至數百萬,請急發京軍剿滅。又言楚、豫之疆盡青燐白骨,新征舊逋,斷無從出,請多方蠲貸。帝並采納。明年,賊陷含山,犯無為,劾總督高斗光。又明年秋,賊陷廬州,臨江欲渡,陳內外合防策。再劾斗光,請以史可法代,斗光遂獲譴。時江北民避亂,盡走南京。光恒慮賊諜闌入,處之城外,為嚴保伍,察非常,奸宄無所匿。

福王立,擢光恒大理寺丞。馬士英薦阮大鋮,令冠帶陛見。光恒言:「先皇手定逆案,芟刈群兇,第一美政。今者大仇未報,乃忽召大鋮,還以冠帶,豈不上傷先皇靈,下短忠義氣哉!」疏奏,命取逆案進覽,光恒即上進。而士英亦以是日進《三朝要典》,大鋮竟起用。其秋,奉命祭告,尋進本寺少卿。使事竣,即旋里。

唐王立,拜光恒兵部左侍郎,佐黃道周協守廣信。廣信破,奔懷玉山,聚眾數千人自保。尋進攻衢州之開化縣,兵敗,歿於陣。

胡夢泰,字友蠡,廣信鉛山人。崇禎十年進士。除奉化知縣。邑人戴澳官順天府丞,怙勢不輸賦。夢泰捕治其子,其子走京師,醖澳,令劾去夢泰。澳念州民不當劾長吏,而劫於其子,姑出一疏,言天下不治由守令貪污,以陰詆夢泰。及得旨,令指實。其子即欲訐夢泰,而澳念夢泰無可劾,乃以嘉興推官文德翼、平遙知縣王凝命實之。給事中沈迅為兩人訴枉,發澳隱情。澳下詔獄,除名。夢泰聲益起。

十六年夏,吏部會廷臣舉天下賢能有司十人,夢泰與焉,行取入都。帝以畿輔州縣殘破,欲得廉能者治之,諸行取者悉出補。夢泰得唐縣。京師陷,南歸。

唐王時,授兵科給事中,奉使旋里。順治三年,大兵逼城下,夢泰傾家募士,與巡撫周定仍等守城。圍數月,城破,夫婦俱縊死。

定仍,南昌人。崇禎十六年進士。與萬文英、胡奇偉、胡甲桂舉兵保廣信,唐王即以為右僉都御史,巡撫其地。城破,死之。

文英,亦南昌人。初為鳳陽推官,以子元亨代死,得脫歸。福王時,起禮部主事,丁艱不赴。唐王授為兵部員外郎,監黃道周諸軍,協守廣信。諸軍敗於鉛山,文英舉家赴水死。

奇偉,進賢人。歷官兵部主事。唐王授為湖東副使,守廣信,兵敗,死之。

田桂,字秋卿,崑山人。崇禎十二年以鄉試副榜貢入國學,授南昌通判。遷永州同知,以道梗改廣信。至則南昌、袁州、吉安俱失。廣信止疲卒千人,士民多竄徙。會黃道周以募兵至,相與議城守。已而道周敗歿,勢益孤,甲桂效死不去。城破被執,諭降不從,幽別室,自經死。

有畢貞士者,貴溪人,舉於鄉。同守廣信,城破,赴水。家人救之,行至五里橋,望拜祖塋,觸橋柱死。

陳泰來,字剛長,江西新昌人。崇禎四年進士。由宣城知縣入為戶科給事中。十五年冬,都城戒嚴,泰來陳戰守數策。總督趙光抃言泰來與同官荊祚永素晰邊情,行間奏報,宜敕二臣參預,報可。泰來又自請假兵一萬,肅清輦轂。帝壯之,即改授兵科,出視諸軍戰守方略,召對中左門。至軍中,奏界嶺失事狀,劾副將柏永鎮論死。以功遷吏科右給事中,乞假歸。福王時,起刑科左給事中,不赴。唐王擢為太僕寺少卿,與萬元吉同守贛州。再擢右僉都御史,提督江西義軍。李自成敗走武昌,其部下散掠新昌境,泰來大破之。初,益王起兵建昌,泰來欲從之。同邑按察使漆嘉祉、舉人戴國士持不可。已而新昌破,國士出降,泰來惡之。會上高舉人曹志明等兵起。泰來與相結。十二月攻取上高、新昌、寧州,殺國士妻子,遂取萬載。已而大兵逼新昌,守將出降,泰來走界埠,志明等從上高移軍會之,進攻撫州,兵敗皆死。

王養正,字聖功,泗州人。崇禎元年進士。授海鹽知縣。遭父喪,服除,起官秀水,中大計,補河南按察司照磨,累遷南康知府。計殲巨寇鄧毛溪、熊高,一方賴之。福王時,進副使,分巡建昌。南都既覆,大兵下江西。巡撫曠昭棄南昌遁,走瑞州,列城望風潰。養正乃與布政夏萬亨、知府王域、推官劉允浩、南昌推官史夏隆起兵拒守。閱三日,有客兵內應,城即破。養正等被執,械至南昌,與萬亨等同死。其妻張氏聞之,絕粒九日而死。

萬亨,字元禮,崑山人,起家舉人。南昌失守,避建昌,與養正同死。妻顧、子婦陸及一孫、一孫女先赴井死。僕婢死者復十餘人。

域,字元壽,松江華亭人。舉於鄉,授宿州學正。流賊至,佐有司捍禦有功。屢遷工部主事,榷稅蕪湖。都城陷,諸榷稅者多以自入。域歎曰:「君父遭非常禍,臣子反因以為利邪!」悉歸之南京戶部。尋由郎中遷建昌知府。城破,械至南昌,與允浩、夏隆同日死。

允浩,掖縣人。夏隆,宜興人。皆崇禎十六年進士。時同死者六人,其一人失其姓名。建昌人哀其忠,裒而瘞之,表曰:「六君子之墓」。

初,建昌南城諸生有鄧思銘者,聞北都陷,集其儕數十人為庠兵,期朔望習射,學技擊,為國報仇。請於有司,有司笑曰:「庠可兵邪?」眾志遂懈。思銘鬱鬱不得志。明年,城破,死之。

建昌既破,新城知縣譚夢開迎降,民潛導守關兵殺之。夢開黨與民互相殘,彌月不靖。唐王以邵武貢生李翔為新城知縣。翔至,擒殺餘黨,眾遂散。然民習於亂,佃人以田主征租斛大,聚數千人,噪縣庭。翔潛遣義兵三百,詭稱鄭彩軍,殺亂民。明日復斬百餘級,亂乃靖。彩兵數萬駐新城,畏大兵,遁入關。獨監軍張家玉、新城人徐伯昌與翔共守。及大兵逼,家玉亦戰敗入關。翔率民兵千餘出城拒擊。大兵從間道入城,民兵皆散,翔與伯昌皆死之。伯昌,字子期,唐王時,由舉人授兵部主事,改御史者也。

時江西郡邑吏城守者,又有李時興、高飛聲。時興,福清人,舉於鄉,歷官袁州同知,攝府事。會城已降,時興力城守。無何,守將蒲纓兵潰,湖廣援將黃朝宣五營亦噪歸。時興度不能守,自縊於萍鄉官舍,一僕亦同死。飛聲,字克正,長樂人。崇禎中,由鄉舉授玉山知縣,遷同知,氣養去。唐王時,黃道周出督師,邀與偕,令攝撫州事。大兵至,遣家人懷印走謁王,而身守城死焉。

曾亨應,字子喜,臨川人。父棟,廣東布政使。亨應舉崇禎七年進士。歷官吏部文選主事。十五年秋,有詔起廢,亨應以毛士龍、李右讜、喬可聘等十人上。御史張懋爵劾其納賄行私,亨應疏辨。懋爵三疏力攻,遂被謫去。福王立之明年,江西列城皆不守。亨應命弟和應奉父入閩,而己與艾南英、揭重熙謀城守。會永寧王慈炎招連子峒土兵數萬復建昌,入撫州,寓書亨應。亨應募兵數百,與相掎角。一日,方置酒宴客,大兵至。亨應避右室,其從弟指示之,遂被執,并執其長子筠。亨應顧筠曰:「勉之,一日千秋,毋自負!」筠曰:「諾。」先受刑死。釋亨應縛,諭之降,不答,被戮。和應聞兄死,曰:「烈哉!兄為忠臣,兄子為孝子,復何憾!」既奉父入閩,又走避之肇慶,乃拜辭其父,投井死。先是,棟弟栻為蒲圻知縣,栻兄益為貴州僉事,並死難,人稱「曾氏五節」云。

始,亨應為懋爵所訐,朝士頗疑之。後亨應死節,而懋爵竟降李自成為直指使。

揭重熙,字祝萬,臨川人。崇禎十年以五經登進士,授福寧知州。福王時,擢吏部考功主事。外艱歸。撫州破,與同里曾亨應先後舉兵。唐王命以故官聯絡建昌兵,戰敗被劾。用大學士曾櫻薦,以考功員外郎兼兵科給事中,從大學士傅冠辦湖東兵事。瀘溪告警,冠不能救,重熙劾解冠任,兵事遂皆委重熙。江西巡撫劉廣胤戰敗被執,復用櫻薦,擢右僉都御史,代廣胤。攻撫州,不克而退。俄聞汀州失,解兵入山。永明王拜重熙兵部尚書兼右副都御史,總督江西兵,召募萬餘人,薄邵武,敗還。

金聲桓,左良玉將也,已降於大清,復乘間為亂,據南昌。大兵攻討之,聲桓死,諸軍盡散,獨張自盛眾數萬走閩。重熙入其軍,約廣信曹大鎬並進。自盛掠邵武,戰敗被執。重熙走依大鎬百丈霡。適大鎬還軍鉛山,惟空營在,眾就營炊食。大兵偵得之,率眾至,射重熙中項,執至建寧,下之獄。重熙日呼高皇帝,祈死不得。至冬十一月,昂首受刃,顏色不改。

傅鼎銓,字維新,重熙同邑人。崇禎十三年進士。除翰林檢討。李自成陷京師,鼎銓出謁,賊敗南還。唐王時,曾櫻薦鼎銓,命予知府銜,赴贛州軍自效,尋復其故官。贛州破,退隱山中。已,聞金聲桓叛,鼎銓舉兵以應。永明王命為兵部右侍郎兼翰林院侍讀學士。聲桓滅,鼎銓往來自盛、大鎬軍。順治八年,至廣信張村,為守將所執,繫南昌獄。諭之降,不從。令作書招重熙,亦不從。八月朔,乃從容就刑。

鼎銓自降流賊,為鄉人非笑,嘗欲求一死所。至是得死,鄉人更賢鼎銓。已,重熙、大鎬相繼敗,都昌督師餘應桂亦以是歲亡,江右兵遂盡。

陳子壯,字集生,南海人。萬曆四十七年以進士第三人授翰林編修。天啟四年典浙江鄉試,發策刺閹豎。魏忠賢怒,假他事削子壯及其父給事中熙昌籍。崇禎初,起子壯故官,累遷禮部右侍郎。流賊犯皇陵,帝素服召對廷臣。子壯言:「今日所急,在收人心。宜下罪己詔,激發忠義。」帝納之。乃會諸臣,列上蠲租、清獄、使過、宥罪等十二事。帝以海內多故,思廣羅賢才,下詔援《祖訓》,郡王子孫文武堪任用者,得考驗授職。子壯慮為民患,立陳五不可。會唐王上疏,歷引前代故事,詆子壯,遂除子壯名,下之獄,坐贖徒歸。久之,廷臣交薦,起故官,協理詹事府。未上,京師陷。

福王立,起禮部尚書。至蕪湖,南京亦失守,乃歸。唐王立福建,召相子壯。以前議宗室事,有宿憾,辭不行。

順治三年,汀州遘變,丁魁楚等擁立桂王子永明王由榔於肇慶。蘇觀生又議立唐王弟聿,子壯沮不得,退居邑之九江村。永明王授子壯東閣大學士兼兵部尚書,督廣東、福建、江西、湖廣軍務。會大兵入廣州,聿被執死,子壯止不行。

明年春,張家玉、陳邦彥及新會王興、潮陽賴其肖先後起兵,子壯亦以七月起兵九江村。兵多蜒戶番鬼,善戰。乃與陳邦彥約共攻廣州,結故指揮使楊可觀等為內應。事洩,可觀等死。子壯駐五羊驛,為大兵擊敗,走還九江村。長子上庸陣歿。會故御史麥而炫破高明,迎子壯,以故主事朱實蓮攝縣事。實蓮,子壯邑子也。九月,大兵克高明,實蓮戰死。子壯、而炫俱執至廣州,不降,被戮。子壯母自縊。永明王贈子壯番禺侯,謚文忠,廕子上圖錦衣衛指揮使。

而炫,字章闇,高明人。由進士歷上海、安肅知縣。唐王時,擢御史。

實蓮,字子潔。由舉人歷官刑部主事。

初,聿之自立於廣州也,召南海霍子衡為太僕卿。子衡,字覺商,舉萬曆中鄉試,歷袁州知府。及官太僕時,而廣州不守。子衡乃召妾莫氏及三子應蘭、應荃、應芷語之曰:「《禮》,『臨難毋茍免』,若輩知之乎?」三子皆應曰:「惟大人命!」子衡援筆大書「忠孝節烈之家」六字,懸中堂,易朝服,北向拜。又易緋袍,謁家廟。先赴井死。妾從之,應蘭偕妻梁氏及一女繼之,應荃、應芷偕其妻徐氏、區氏又繼之。惟三孫得存。有小婢見之,亦投井死。

張家玉,字元子,東莞人。崇禎十六年進士。改庶吉士。李自成陷京師,被執。上書自成,請旌己門為:「翰林院庶吉士張先生之廬」,而褒恤范景文、周鳳翔等,隆禮劉宗周、黃道周,尊養史可程、魏學濂。自稱殷人從周,願學孔子,稱自成大順皇帝。自成怒,召之入,長揖不跪。縛午門外三日,復脅之降,怵以極刑,卒不動。自成曰:「當磔汝父母!」乃跪。時其父母在嶺南,家玉遽自屈,人咸笑之。

賊敗南歸。阮大鋮等攻家玉薦宗周、道周於賊,令收人望,集群黨。家玉遂被逮。明年,南都失守,脫歸。從唐王入福建,擢翰林侍講,監鄭彩軍。出杉關,謀復江西,解撫州之圍。

順治三年,風聞大兵至,彩即奔入關,家玉走新城。大兵來攻,出戰,中矢,墮馬折臂,走入關。令以右僉都御史巡撫廣信。廣信已失,請募兵惠、潮,說降山賊數萬,將赴贛州急。會大兵克汀州,乃歸東莞。

四年,家玉與舉人韓如璜結鄉兵攻東莞城,知縣鄭霖降,乃籍前尚書李覺斯等貲以犒士。甫三日,大兵至,家玉敗走。奉表永明王,進兵部尚書。無何,大兵來擊,如璜戰死,家玉走西鄉。祖母陳、母黎、妹石寶俱赴水死。妻彭被執,不屈死,鄉人殲焉。西鄉大豪陳文豹奉家玉取新安,襲東莞,戰赤岡。未幾,大兵大至,攻數日,家玉敗走鐵岡,文豹等皆死。

覺斯怨家玉甚,發其先壟,毀及家廟,盡滅家玉族,村市為墟。家玉過故里,號哭而去。道得眾數千,取龍門、博羅、連平、長寧,遂攻惠州,克歸善,還屯博羅。大兵來攻,家玉走龍門,復募兵萬餘人。家玉好擊劍,任俠,多與草澤豪士游,故所至歸附。乃分其眾為龍、虎、犀、象四營,攻據增城。

十月,大兵步騎萬餘來擊。家玉三分其兵,掎角相救,倚深溪高崖自固。大戰十日,力竭而敗,被圍數重。諸將請潰圍出,家玉歎曰:「矢盡炮裂,欲戰無具;將傷卒斃,欲戰無人。烏用徘徊不決,以頸血濺敵人手哉!」因遍拜諸將,自投野塘中以死,年三十有三。明年,永明王贈家玉少保、武英殿大學士、吏部尚書、增城侯,謚文烈。其父兆龍猶在,以子爵封之。

陳象明,字麗南,家玉同邑人。崇禎元年進士。授戶部主事,榷稅淮安,以清操聞。屢遷饒州知府。忤巡按御史,被劾。謫兩浙鹽運副使,累遷湖南道副使。唐王時,總督何騰蛟令征餉廣西。會永明王立,廣東地盡失。象明徵調土兵,與陳邦傳連營,東至梧州榕樹潭,遇大兵,戰敗,死之。

廣東之失也,龍門破,里人廖翰標以二幼子託從父,從容自縊死。番禺破,里人梁萬爵曰「此志士盡節之秋也」,赴水死。翰標,天啟中舉人,官江西新城知縣,廉惠,民為建祠。萬爵,字天若,唐王時舉人。

陳邦彥,字令斌,順德人。為諸生,意氣豪邁。福王時,詣闕上政要三十二事,格不用,唐王聿鍵讀而偉之。既自立,即其家授監紀推官。未任,舉於鄉。以蘇觀生薦,改職方主事,監廣西狼兵,援贛州。至嶺,聞汀州變,勸觀生東保潮、惠,不聽。

會丁魁楚等已立永明王監國於肇慶,觀生遣邦彥入賀。王因贛州破,懼逼己,西走梧州。邦彥甫入謁,而觀生別立唐王聿於廣州,邦彥不知也。夜二鼓,王遣中使十餘輩召入舟中。王太后垂簾坐,王西向坐,魁楚侍,語以廣州事。邦彥請急還肇慶,正大位以繫人心。命南雄勍卒取韶,制粵東十郡之七,而委其三於唐王,代我受敵,從而乘其敝。王大悅,立擢兵科給事中,齎敕還諭觀生。抵廣州,聞使臣彭耀被殺,乃遣從人授觀生敕,而自以書曉利害。觀生猶豫累日,欲議和,會聞永明王兵大敗,不果。邦彥遂變姓名入高明山中。

順治三年冬十二月,大兵破廣州,觀生死,列城悉下,邦彥乃謀起兵。初,贛州萬元吉遣族人萬年募兵於廣,得餘龍等千餘人,未行而贛州失。龍等無所歸,聚甘竹灘為盜,他潰卒多附,至二萬餘人。總督朱治心間招降之,既而噪歸。四年春,大兵定廣州,克肇慶、梧州,敗走治心間,殺魁楚,前驅抵平樂。永明王方自梧道平樂,走桂林,勢危甚。邦彥乃說龍乘間圖廣州,而己發高明兵由海道入珠江與龍會。且遣張家玉書曰:「桂林累卵,但得牽制毋西,潯、平間可完葺,是我致力於此而收功於彼也。」家玉以為然。然龍卒故無紀律,大兵自桂林還救,揚言取甘竹灘,龍等顧其家,輒退,邦彥亦卻歸。既,乃遣門人馬應芳會龍軍取順德。無何,大兵至,龍戰敗,應芳被執,赴水死。四月,龍再戰黃連江,亦敗歿。大兵攻家玉於新安。邦彥乃棄高明,收餘眾,徇下江門據之。

初,廣州之圍,大兵知謀出邦彥,求其家,獲妾何氏及二子,厚遇之,為書招邦彥。邦彥判書尾曰:「妾辱之,子殺之。身為忠臣,義不顧妻子。」七月與陳子壯密約,復攻廣州。子壯先至,謀洩,將引退。邦彥軍亦至,謀伏兵禺珠洲側,伺大兵還救會城,而縱火以焚舟。子壯如其計,果焚舟數十。大兵引而西,邦彥尾之。會日暮,子壯不能辨旗幟,疑皆敵舟也,陣動。大兵順風追擊,遂大潰。子壯奔高明,邦彥奔三水。八月,清遠指揮白常燦以城迎邦彥。乃入清遠,與諸生朱學熙嬰城固守。

邦彥自起兵,日一食,夜則坐而假寐,與其下同勞苦,故軍最強,嘗分兵救諸營之敗者。至是精銳盡喪,外無援軍。越數日,城破,常燦死。邦彥率數十人巷戰,肩受三刃,不死,走朱氏園,見學熙縊,拜哭之。旋被執,饋之食,不食,繫獄五日,被戮。邦彥死,子壯被執。踰月,家玉亦自沉。永明王贈邦彥兵部尚書,謚忠愍,廕子錦衣指揮。

蘇觀生,字宇霖,東莞人。年三十始為諸生。崇禎中,由保舉授無極知縣。總督范志完薦其才,進永平同知,監紀軍事,尋遷戶部員外郎。十七年,京師陷,脫還南京,進郎中,催餉蘇州。明年五月,南京破,走杭州。會唐王聿鍵至,觀生謁王。王與語大悅,聯舟入福建。與鄭芝龍、鴻逵兄弟擁立王,擢為翰林學士,旋進禮部右侍郎兼學士。設儲賢館,分十二科,招四方士,令觀生領之。觀生矢清操,稍有文學,而時望不屬。王以故人,恩眷出廷臣右,乃超拜東閣大學士,參機務。

觀生數贊王出師。見鄭氏不足有為,事權悉為所握,請王赴贛州,經略江西、湖廣。王乃議觀生先行。明年,觀生赴贛州,大徵甲兵。餉不繼,竟不能出師。

時順治三年三月,大兵破吉安,總督萬元吉乞援,觀生遣二百人往。元吉令協守綿津灘,遇大兵,潰走。元吉乃退回贛州,大兵遂圍城。觀生走南康,贛人數告急,不敢援。六月,大兵退屯水西,觀生發三千人助贛守。久之,他將戰敗。九月,大兵再攻贛州,三千人皆引去。時觀生移駐南安,閩中急,不能救。聿鍵死於汀州,贛州亦破,觀生退入廣州。監紀主事陳邦彥勸觀生疾趨惠、潮,扼漳、泉、兩粵可自保。觀生不從。

會丁魁楚等議立永明王,觀生欲與共事。魁楚素輕觀生,拒不與議,呂大器亦叱辱之。適唐王弟聿與大學士何吾騶自閩至,南海關捷先、番禺梁朝鐘首倡兄終弟及議。觀生遂與吾騶及布政使顧元鏡,侍郎王應華、曾道唯等以十一月二日擁立王,就都司署為行宮。即日封觀生建明伯,掌兵部事,進吾騶等秩,擢捷先吏部尚書,旋與元鏡、應華、道唯並拜東閣大學士,分掌諸部。時倉卒舉事,治宮室、服御、鹵簿,通國奔走,夜中如晝。不旬日,除官數千,冠服皆假之優伶云。

永明王監國肇慶,遣給事中彭耀、主事陳嘉謨齎敕往諭。耀,順德人,過家拜先廟,託子於友人。至廣州,以諸王禮見,備陳天潢倫序及監國先後,語甚切至,因歷詆觀生諸人。觀生怒,執殺之,嘉謨亦不屈死。乃治兵日相攻,以番禺人陳際泰督師,與永明王總督林佳鼎戰於三水。兵敗,復招海盜數萬人,遣大將林察將。

十二月二日,戰海口,斬佳鼎。觀生意得,務粉飾為太平事,而委任捷先及朝鍾。

捷先,由進士歷官監司,小有才,便筆札。朝鍾舉於鄉,善談論,浹旬三遷至祭酒。有楊明競者,潮州人,好為大言,詭稱精兵滿惠、潮間,可十萬,即特授惠潮巡撫。朝鍾語人:「內有捷先,外有明競,強敵不足平矣。」觀生亦器此三人,事必咨之。又有梁鍙者,妄人也,觀生才之,用為吏科都給事中,與明競大納賄賂,日薦用數十人。

觀生本乏猷略,兼總內外任,益昏瞀。招海盜資捍禦,其眾白日殺人,縣肺腸於貴官之門以示威,城內外大擾。時大兵已下惠、潮,長吏皆降附,即用其印移牒廣州,報無警。觀生信之。

是月十五日,聿視學,百僚咸集,或報大兵已逼。觀生叱之曰:「潮州昨尚有報,安得遽至此。妄言惑眾,斬之!」如是者三。大兵已自東門入,觀生始召兵搏戰。兵精者皆西出,倉卒不能集。觀生走鍙所問計。曰:「死爾,復何言!」觀生入東房,鍙入西房,各拒戶自縊。觀生慮其詐,稍留聽之。鍙故扼其吭,氣湧有聲,且推幾僕地,久之寂然。觀生信為死,遂自經。明日,鍙獻其尸出降。朝鐘聞變赴池,為鄰人救出,自經死,聿方事閱射,急易服逾垣匿王應華家。俄縋城走,為追騎所獲。饋之食,不受,曰:「我若飲汝一勺水,何以見先人地下!」投繯而絕。吾騶、應華等悉降。

贊曰:自南都失守,列郡風靡。而贛以彈丸,獨憑孤城,誓死拒命。豈其兵力果足恃哉,激於義而眾心固也。迨汀、贛繼失,危近目睫,而肇慶、廣州日治兵相攻,自取兩敗。蓋天速其禍,如發蒙振槁,無煩驅除矣。


\end{pinyinscope}