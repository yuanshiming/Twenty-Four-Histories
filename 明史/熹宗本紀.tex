\article{熹宗本紀}


熹宗達天闡道敦孝篤友章文襄武靖穆莊勤悊皇帝,諱由校,光宗長子也。母選侍王氏。萬曆三十三年十一月,神宗以元孫生,詔告天下。

四十八年,神宗遺詔皇長孫及時冊立,未及行。九月乙亥,光宗崩,遺詔皇長子嗣皇帝位。群臣哭臨畢,請見皇長子於寢門,奉至文華殿行禮,還居慈慶宮。丙子,頒遺詔。時選侍李氏居乾清宮,吏部尚書周嘉謨等及御史左光斗疏請選侍移宮,御史王安舜疏論李可灼進藥之誤,「紅丸」、「移宮」二案自是起。己卯,選侍移仁壽殿。庚辰,即皇帝位。詔赦天下,以明年為天啟元年。己丑,以是年八月以後稱泰昌元年。辛卯,逮遼東總兵官李如柏。甲午,廕太監魏進忠兄錦衣衛千戶。封乳保客氏為奉聖夫人,官其子。冬十月丙午,葬顯皇帝、孝端顯皇后於定陵。戊申,遼東巡撫都御史袁應泰為兵部侍郎,經略遼東,代熊廷弼。辛酉,御經筵。壬戌,禮部尚書孫如游兼東閣大學士,預機務。丁卯,噦鸞宮災。十一月丙子,追謚皇妣孝元貞皇后,生母孝和皇太后。十二月辛酉,方從哲致仕。

天啟元年春正月庚辰,享太廟。壬辰,追謚伍文定等七十三人。壬寅,御史王心一請罷客氏香火土田,魏進忠陵工敘錄,不報。二月甲辰,言官請復當朝口奏及召對之典,從之。己未,御經筵。閏月乙酉,以風霾諭群臣修省。丁亥,孫如游致仕。丙申,除齊泰、黃子澄戚屬戍籍。戊戌,昭和殿災。三月乙卯,大清兵取沈陽,總兵官尤世功、賀世賢戰死。總兵官陳策、童仲揆、戚金、張名世帥諸將援遼,戰於渾河,皆敗沒。壬戌,大清兵取遼陽,經略袁應泰等死之。巡按御史張銓被執,不屈死。丙寅,諭兵部:「國家文武並用,頃承平日久,視武弁不啻奴隸,致令豪傑解體。今邊疆多故,大風猛士深軫朕懷,其令有司於山林草澤間慎選將材。」丁卯,京師戒嚴。夏四月壬申朔,日有食之。甲戌,禁抄發軍機。丙子,遼東巡撫僉都御史薛國用為兵部侍郎,經略遼東。參議王化貞為右僉都御史,巡撫廣寧。戊寅,募兵於通州、天津、宣府、大同。甲午,募兵於陜西、河南、山西、浙江。戊戌,冊皇后張氏。五月丁未,貴州紅苗平。甲寅,禁訛言。辛酉,陜西都指揮陳愚直以固原兵入援,潰於臨洺。未幾,寧夏援遼兵潰於三河。六月癸酉,何宗彥入閣。丙子,朱國祚入閣。熊廷弼為兵部尚書兼右副都御史,經略遼東。辛巳,兵部尚書王象乾總督薊、遼軍務。秋七月乙巳,沈飀入閣。八月丙子,擢參將毛文龍為副總兵,駐師鎮江城。戊子,杭州大火,詔停強造。癸巳,停刑。九月壬寅,葬貞皇帝於慶陵。乙卯,永寧宣撫使奢崇明反,殺巡撫徐可求,據重慶,分兵陷合江、納溪、瀘州。丁卯,陷興文,知縣張振德死之。冬十月戊辰,御史周宗建請出客氏於外,不聽。給事中倪思輝、朱欽相等相繼言,皆謫外任。丙子,史繼偕入閣。乙酉,奢崇明圍成都,布政使朱燮元固守。尋擢燮元僉都御史,巡撫四川。石砫宣撫使女土官秦良玉起兵討賊。壬辰,葉向高入閣。十二月丁丑,巡撫河南都御史張我續為兵部侍郎,提督川、貴軍務。陜西巡撫移駐漢中,鄖陽巡撫移駐夷陵。湖廣官軍由巫峽趨忠、涪討賊。庚辰,援遼浙兵嘩於玉田。辛卯,以熊廷弼、王化貞屢議戰守不合,遣使宣諭。是年,安南、土魯番、烏斯藏入貢。

二年春正月丁未,延綏總兵官杜文煥、四川總兵官楊愈懋討永寧賊。丁巳,大清兵取西平堡,副將羅一貴死之。鎮武營總兵官劉渠、祁秉忠逆戰於平陽橋,敗沒。王化貞走閭陽,與熊廷弼等俱入關。參政高邦佐留松山,死之。壬戌,振山東流徙遼民。癸亥,兵部尚書張鶴鳴視師遼東。乙丑,京師戒嚴。河套部犯延綏。永寧賊將羅乾象約降,與官軍共擊賊,成都圍解。二月癸酉,水西土同知安邦彥反,陷畢節、安順、平壩、沾益、龍里,遂圍貴陽,巡撫都御史李枟、巡按御史史永安固守。戊寅,免天下帶徵錢糧二年及北畿加派。禮部右侍郎孫承宗為兵部尚書兼東閣大學士,預機務。己丑,孫承宗兼理兵部事。三月丁酉朔,劉一燝致仕。甲辰,陽武侯薛濂管理募兵。兵部侍郎王在晉為尚書兼右副都御史,經略遼、薊、天津、登、萊軍務。甲寅,賜文震孟等進士及第、出身有差。丁巳,敕湖廣、雲南、廣西官軍援貴州。是春,舉內操。夏四月甲申,京師旱。五月戊戌,復張居正原官。己亥,錄方孝孺遺嗣,尋予祭葬及謚。丙午,山東白蓮賊徐鴻儒反,陷鄆城。癸亥,秦良玉、杜文煥破賊於佛圖關,官軍合圍重慶,復之。六月戊辰,徐鴻儒陷鄒縣、滕縣,滕縣知縣姬文胤死之。加毛文龍為總兵官。貴州總兵官張彥芳為平蠻總兵官,從巡撫都御史王三善討水西賊。己巳,前總兵官楊肇基、遊擊陳九德帥兵討山東賊。秋七月甲辰,松潘副使李忠臣約總兵官楊愈懋謀復永寧,不克,皆死之。賊攻大壩,遊擊龔萬祿戰死,遂陷遵義。癸丑,沈飀致仕。乙卯,神宗神主祔太廟。庚申,援黔兵潰於新添。癸亥,武邑賊于弘志作亂,尋伏誅。八月庚辰,孫承宗以原官督理山海關及薊、遼、天津、登、萊軍務。九月甲午朔,光宗神主祔太廟。壬寅,御史馮英請設州縣兵,按畝供餉,從之。乙卯,封皇弟由檢為信王。停刑。冬十月辛未,水西賊犯雲南,官軍擊敗之。辛巳,官軍復鄒縣,擒徐鴻儒等,山東賊平。壬午,總兵官魯欽代杜文煥為總理,援貴州。十一月癸丑,硃燮元總督四川軍務。十二月己巳,王三善、副總兵劉超敗賊於龍里,貴陽圍解。是年,暹羅入貢。

三年春正月己酉,禮部侍郎朱國禎,尚書顧秉謙,侍郎硃延禧、魏廣微,俱禮部尚書東閣大學士,預機務。乙卯,紅夷據澎湖。貴州官軍三路進討水西,副總兵劉超敗績於陸廣河。二月乙酉,贈恤鄒縣死難博士孟承光及母孔氏,子弘略。是月,停南京進鮮。三月癸卯,朝鮮廢其主李琿。是春,振山東被兵州縣。夏四月庚申朔,京師地震。己巳,朱國祚致仕。五月辛丑,四川官軍敗賊於永寧,奢崇明走紅崖。秋七月辛卯,南京大內災。壬辰,奢崇明走龍場,與安邦彥合。丁酉,安南寇廣西,巡撫都御史何士晉禦卻之。己亥,史繼偕致仕。九月癸巳,給事中陳良訓疏陳防微四事,下鎮撫司獄。冬十月乙亥,京師地震。丁丑,停刑。閏月壬寅,以皇子生,詔赦天下。是月,王三善剿水西,屢破賊,至大方。十一月丁巳朔,祀天於南郊。十二月癸巳,封李倧為朝鮮國王。戊戌,京師地震。庚戌,魏忠賢總督東廠。是年,暹羅、琉球入貢。

四年春正月丙辰朔,長興民吳野樵殺知縣石有恆、主簿徐可行,尋伏誅。乙丑,王三善自大方旋師遇伏,被執死之,諸官將皆死。庚午,何宗彥卒。二月丁酉,薊州、永平、山海關地震,壞城郭廬舍。甲寅,京師地震,宮殿動搖有聲。帝不豫。三月丁巳,疾愈。庚申,杭州兵變。是月,京師屢地震。夏五月甲寅朔,福寧兵變,有司撫定之。六月癸未,左副都御史楊漣劾魏忠賢二十四大罪,南北諸臣論忠賢者相繼,皆不納。丙申,大雨雹。杖殺工部郎中萬燝,逮杖御史林汝翥。秋七月辛酉,葉向高致仕。癸亥,河決徐州。振山東饑。冬十月,削吏部侍郎陳于廷、副都御史楊漣、僉都御史左光斗籍。十一月己巳,韓爌致仕。是月,貴州官兵敗賊於普定,進至織金,破之。十二月辛巳,逮內閣中書汪文言下鎮撫司獄。丙申,朱國禎致仕。癸卯,南京地震如雷。是月,兩當民變,殺知縣牛得用。

五年春正月癸亥,大清兵取旅順。戊寅,以慶陵工成,予魏忠賢等廕賚。是月,總理魯欽、劉超等自織金旋師,為賊所襲,諸營兵潰。三月甲寅。釋奠於先師孔子。丙寅,賜餘煌等進士及第、出身有差。甲戌,朱燮元總督雲、貴、川、湖、廣西軍務,討安邦彥。丁丑,讞汪文言獄,逮楊漣、左光斗、袁化中、魏大中、周朝瑞、顧大章,削尚書趙南星等籍。未幾,漣等逮至,下鎮撫司獄,相繼死獄中。夏四月己亥,削大學士劉一燝籍。五月癸亥,給事中楊所修請以「梃擊」、「紅丸」、「移宮」三案編次成書,從之。乙丑,祀地於北郊。庚午,行宗室限錄法。六月丙戌,朱延禧致仕。秋七月壬戌,毀首善書院。壬申,韓爌削籍。甲戌,追論萬曆辛亥、丁巳、癸亥三京察,尚書李三才、顧憲成等削籍。八月壬午,毀天下東林講學書院。削尚書孫慎行等籍。戊子,禮部尚書周如磐兼東閣大學士,侍郎丁紹軾、黃立極為禮部尚書,少詹事馮銓為禮部右侍郎,並兼東閣大學士,預機務。己亥,魏廣微罷。壬寅,熊廷弼棄市,傳首九邊。九月壬子,遼東副總兵魯之甲敗沒於柳河。冬十月己卯,兵部尚書高第經略遼、薊、登、萊、天津軍務。丙戌,停刑。庚寅,孫承宗致仕。丙申,逮中書舍人吳懷賢下鎮撫司獄,杖殺之。庚子,以皇子生,詔赦天下。十一月壬子,周如磐致仕。十二月乙酉,榜東林黨人姓名,頒示天下。戊子,戍前尚書趙南星。是年,琉球、烏斯藏入貢。

六年春正月戊午,修《三朝要典》。丁卯,大清兵圍寧遠,總兵官滿桂、寧前道參政袁崇煥固守。己巳,圍解。二月乙亥,袁崇煥為僉都御史,專理軍務,仍駐寧遠。戊戌,以蘇杭織造太監李實奏,逮前應天巡撫周起元,吏部主事周順昌,左都御史高攀龍,諭德繆昌期,御史李應昇、周宗建、黃尊素。攀龍赴水死,起元等下鎮撫司獄,相繼死獄中。己亥。祭日於東郊。三月丁未,設各邊鎮監軍內臣。太監劉應坤鎮守山海關,大學士丁紹軾、兵部尚書王永光等屢諫不聽。論寧遠解圍功,封魏忠賢從子良卿肅寧伯。庚戌,安邦彥犯貴州,官軍敗績,總理魯欽死之。壬子,袁崇煥巡撫遼東、山海。夏四月丁丑,命南京守備內臣搜括應天各府貯庫銀,充殿工、兵餉。戊戌,丁紹軾卒。五月戊申,王恭廠災,死者甚眾。己酉,以旱災敕群臣修省。癸亥,朝天宮災。六月丙子,京師地震,靈丘地震經月。壬午,河決廣武。辛卯,《三朝要典》成,刊布中外。閏月辛丑,巡撫浙江僉都御史潘汝楨請建魏忠賢生祠,許之。嗣是建祠幾遍天下。壬寅,馮銓罷。壬子,朱燮元以憂去,偏沅巡撫都御史閔夢得代之。是夏,京師大水,江北、山東旱蝗。秋七月辛未朔,日當食,陰雲不見。辛巳,下前揚州知府劉鐸詔獄,殺之。丙戌,禮部侍郎施鳳來、張瑞圖,詹事李國普,俱禮部尚書東閣大學士,預機務。八月,陜西流賊起,由保寧犯廣元。九月庚寅,顧秉謙致仕。壬辰,皇極殿成,停刑。己亥,魏良卿進封肅寧侯。是月,參將楊明輝齎敕招諭水西賊,被殺。是秋,江北大水,河南蝗。冬十月戊申,進魏忠賢爵上公,魏良卿寧國公,予誥券,加賜莊田一千頃。己酉,以皇極殿成詔天下,官匠雜流升授者九百六十五人。癸丑,改修《光宗實錄》。十一月庚寅,予魏良卿鐵券。十二月戊申,南京地震。甲子,潯州賊殺守備蔡人龍。是年,安南、烏斯藏、琉球入貢。

七年春正月辛未,振鳳陽饑。乙亥,太監塗文輔總督太倉銀庫、節慎庫,崔文昇、李明道提督漕運河道,核京師、通州諸倉。辛卯,免榷潼關、咸陽商稅。二月壬戌,修隆德殿。三月癸酉,豐城侯李承祚請開採珠池、銅礦,不許。戊子,澄城民變,殺知縣張斗耀。是春,大清兵征朝鮮。夏四月丁酉,下前侍郎王之寀鎮撫司獄,死獄中。五月己巳,監生陸萬齡請建魏忠賢生祠於太學旁,祀禮如孔子,許之。丙子,大清兵圍錦州。癸巳,攻寧遠。六月庚子,錦州圍解。秋七月乙丑朔,帝不豫。丙寅,罷袁崇煥。己卯,封魏忠賢孫鵬翼為安平伯。壬午,戍孫慎行。丁亥,海賊寇廣東。是月,浙江大水。八月丙申,加魏良卿太師,魏鵬翼少師。戊戌,中極、建極二殿成。乙巳,召見閣部、科道諸臣於乾清宮,諭以魏忠賢、王體乾忠貞可計大事。封忠賢姪良棟為東安侯。甲寅,大漸。乙卯,崩於乾清宮,年二十三。遺詔以皇第五弟信王由檢嗣皇帝位。冬十月庚子,上尊謚,廟號熹宗,葬德陵。

贊曰:明自世宗而後,綱紀日以陵夷,神宗末年,廢壞極矣。雖有剛明英武之君,已難復振。而重以帝之庸懦,婦寺竊柄,濫賞淫刑,忠良慘禍,億兆離心,雖欲不亡,何可得哉。

