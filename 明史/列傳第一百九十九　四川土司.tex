\article{列傳第一百九十九 四川土司}

\begin{pinyinscope}
四川土司諸境,多有去蜀遠,去滇、黔近者。如烏蒙、東川近於滇,烏撒、鎮雄、播州近於黔。明太祖略定邊方,首平蜀夏,置四川布政司,使招諭諸蠻,次第歸附。故烏蒙、烏撒、東川、芒部舊屬雲南者,皆隸於四川,不過歲輸貢賦,示以羈縻。然夷性獷悍,嗜利好殺,爭相競尚,焚燒劫掠,習以為恆。去省窵遠,莫能控制,附近邊民,咸被其毒。皆由規模草創,未嘗設立文武為之鈐轄,聽其自相雄長。雖受天朝爵號,實自王其地。以故終明之世,常煩撻伐。唯建昌、松、茂等處設立衛所,播州改遵義、平越二府以後,稍安戢云。

○四川土司一

烏蒙烏撒東川鎮雄四軍民府馬湖建昌衛寧番衛越巂衛鹽井衛會川衛茂州衛松潘衛天全六番招討司黎州安撫司

烏蒙、烏撒、東川、芒部,古為竇地、的巴、東川、大雄諸甸,皆唐烏蒙裔也。宋有封烏蒙王者。元初置烏蒙路,遂以東川、芒部皆隸於烏蒙、烏撒等處宣慰司。烏撒富盛甲諸部,元時嘗置軍民總管府,而於東川置萬戶府。地勢並在蜀之東南,與滇、黔壞土相接,皆據險阻深,與中土聲教隔離。

明太祖既平蜀,規取雲南,大師皆集於辰、沅,欲並剪諸蠻以通蜀道。洪武十四年遣內臣齎敕諭烏蒙、烏撒諸部長曰:「西南諸部,自古及今,莫不朝貢中國。朕受天命為天下主十有五年,而烏蒙、烏撒、東川、芒部、建昌諸部長猶桀驁不朝。朕已遣征南將軍潁川侯、左副將軍永昌侯、右副將軍西平侯率師往征。猶恐諸部長未喻朕意,故復遣內臣往諭。如悔罪向義,當即躬親來朝,或遣人入貢,亟攄誠款,朕當罷兵,以安黎庶。爾共省之。」時征南將軍傅友德已分遣都督胡海洋等帥師五萬,由永寧趨烏撒,復自率師由曲靖循格孤山而南,以通永寧之兵,搗烏撒。時元右丞實卜聞海洋兵至,乃聚兵赤水河以拒之。及聞大軍繼進,皆遁。友德令諸軍築城,版BE方具,蠻寇大集。友德屯兵山岡,持重以待。既知士勇可用,乃縱兵接戰。有芒部土酋率眾來援,實卜兵與合,鋒甚銳。大軍鼓噪而前,其酋長多中槊墜馬死。大軍益奮,蠻眾力不支,大潰,斬首三千,獲馬六百,實卜率眾遁。遂城烏撒,克七星關以通畢節,又克可渡河,於是東川、烏蒙、芒部諸蠻震讋,皆望風降附。

十五年置東川、烏撒、烏蒙、芒部諸衛指揮使司,詔諭諸部人民。以雲南已降附,宜益效順中國,以享昇平。復諭諸部長曰:「今置郵傳通云南,宣率土人,隨其疆界遠邇,開築道路,各廣十丈,準古法,以六十里為一驛。符至奉行。」又敕征南將軍友德等曰:「烏蒙、烏撒、東川、芒部諸酋長雖已降,恐大軍一還,仍復嘯聚。符到日,悉送其酋長入朝。」又諭以貴州已設都指揮使,然地勢偏東,今宜於實卜所居之地立司,以便控制,卿其審之。」已,烏撒諸蠻復叛,帝諭友德曰:「烏撒諸蠻伺官軍散處,即有此變,朕前已慮之,今果然。然雲南之地如曲靖、普安、烏撒、建昌,勢在必守,其東川、芒部、烏蒙,未可遽守也。且留屯大軍蕩埽諸蠻,戮其渠長,方可分兵守禦耳。」乃命安陸侯吳復為總兵,平涼侯費聚副之,征烏撒、烏蒙諸叛蠻。並諭勿與蠻戰於關索嶺上,當分兵掩襲,直搗其巢,使彼各奔救其家不暇,必不敢出以抗大師。俟三將軍至,破擒之。是月,副將軍西平侯沐英自大理還軍,會友德擊烏撒,大敗其眾,斬首三萬餘級,獲馬牛羊萬計,餘眾悉遁,復追擊破之。帝諭友德等,師捷後,必戮其渠魁,使之畏懼。搜其餘黨,絕其根株,使彼智窮力屈,誠心款附,方可留兵鎮守。又諭宜乘兵勢修治道途,令土酋諭其民,各輸糧一石以給軍,為持久計。

十六年以雲南所屬烏撒、烏蒙、芒部三府隸四川布政使司。烏蒙、烏撒、東川、芒部諸部長百二十人來朝,貢方物。詔各授以官,賜朝服、冠帶、錦騎、鈔錠有差。其烏撒女酋實卜,加賜珠翠。芒部知府發紹、烏蒙知府阿普病卒,詔賜綺衣並棺殮之具,遣官致祭,歸其柩於家。十七年割雲南東川府隸四川布政府司,并烏撒、烏蒙、芒部皆改為軍民府,而定其賦稅。烏撒歲輸二萬石,氈衫一千五百領;烏蒙、東川、芒部皆歲輸八千石,氈衫八百領。又定茶鹽布疋易馬之數,烏撒歲易馬六千五百匹,烏蒙、東川、芒部皆四千匹。凡馬一匹,給布三十疋,或茶一百斤,鹽如之。實卜復貢馬,賜綺鈔。十八年,烏蒙知府亦德言,蠻地刀耕火種,比年霜旱疾疫,民饑窘,歲輸之糧無從徵納。詔悉免之。二十年徵烏撒知府阿能赴京。

二十一年命西平侯沐英南征。英言,東川強盛,據烏山路作亂,罪狀已著,必先加兵。但其地重關衣復嶺,上下三百餘里,人跡阻絕,須以大兵臨之。帝命潁國公傅友德仍為征南將軍,英與陳醒為左桓副將軍,率諸軍進討。敕友德等曰:「東川、芒部諸夷,種類皆出於羅羅。厥後子姓蕃衍,各立疆場,乃異其名曰東川、烏撒、烏蒙、芒部、祿肇、水西。無事則互起爭端,有事則相為救援。若唐時閣羅鳳亡居大理,唐兵追捕,道經芒部諸境,君蠻聚眾據險設伏。唐將不備,遂墮其計,喪師二十萬,皆將帥無謀故也。今須預加防閑,嚴為之備。」烏撒軍民府葉原常獻馬三百匹、米四百石於征南將軍,以資軍用,且願收集士兵從征。英等以聞,從之。復命景川侯曹震、靖寧侯葉昇等分討東川,平之,捕獲叛蠻五千五百三十八人。

二十三年,烏撒土知府阿能,烏蒙、芒部土官,各遣子弟入監讀書。二十七年,烏撒知府卜穆奏,沾益州屢侵其地,命沐春諭之。二十八年,戶部言:「烏撒、烏蒙、芒部、東川歲賦氈衫不如數,詔已免徵。今有司仍追之,宜申明。」從之。二十九年,烏蒙軍民府知府實哲貢馬及氈衫。自是,諸土知府三年一入貢,以為常,或有恩賜,則進馬及方物謝恩。

宣德七年,兵部侍郎王驥言,烏蒙、烏撒土官祿昭、尼祿等,爭地仇殺,宜遣官按問。八年遣行人章聰、侯璉齎敕往諭,仍敕巡按與三司官往平之。設烏蒙儒學教授、訓導各一員。以通判黃甫越言,元時本府向有學校,今文廟雖存,師儒未建。乞除教官,選俊秀子弟入學讀書,以廣文治,從之。

正統七年裁烏撒軍民府通判、推官、知事、檢校各一員。十一年裁烏蒙、東川知事、檢校各一員,並革烏撒、烏蒙遞運所。景泰元年敕諭烏撒、烏蒙諸府土官普茂等,以貴州諸苗叛亂,恐滋蔓鄰近,宜戒嚴防守,毋聽賊眾誘惑,倘來逼犯,便當剿殺。時烏撒進萬壽表逾期,部議宜究,詔以遠人宥之。嗣後,朝貢過期及表箋不至者,朝廷率以土官多從寬貸,應賞者給其半。天順元年,鎮守四川中官陳清等奏,芒部所轄白江蠻賊千餘備作亂,攻圍筠連縣治,敕御史項愫會鎮巡官捕之。

成化十二年,烏撒知府隴舊等奏,同知剛正撫字有方,蠻民信服,今九年秩滿,乞再任三年,以慰群望。從之。弘治十四年,烏撒所轄可渡河巡檢司言:「自閏七月二十七日,大雷雨不止,至二十九日,水漲山崩地裂,山嗚如牛吼,地陷湧出清泉數十派,沖壞廬舍橋梁及壓死人口牲畜無算。又本府阿都地方,八月亦暴風雨,田土渰沒二百餘處,死者三百餘人。」

正德十五年討斬芒部僰蠻阿又磉等。初,芒部土舍隴壽,與庶弟隴政及兄妻支祿爭襲仇殺。所部僰蠻阿又磉等乘機倡亂流劫。事聞,命鎮守中官會撫按官捕治。至是,貴州參政傅習、都指揮許詔,督永寧宣撫司女土官奢爵等,討擒阿又磉等四十三人,斬一百十九級,事乃定。

嘉靖元年命芒部護印土舍隴壽襲知府,免赴京。故事,土官九品以上,皆保送至京乃襲。時壽、政等爭襲,不敢離任。朝廷以嫡故立壽,恐壽赴京而政等北隙為亂,故有是命。然政與支祿倚烏撒土舍安寧等兵力,仇殺如故。壩底參將何卿請於巡撫許廷光,發土兵二萬五千人,命貴州參將楊仁等將之,受何卿節制,相機進剿政、祿佯聽撫,乞緩師,而令賊黨阿黑等掠周泥站、七星關,復遣阿核等糾集諸苗,剽掠畢節諸處,殺傷官軍,毀官民房屋甚眾。兵部言賊勢猖獗,宜速征。於是可卿等進剿,斬首二百餘級,俘二十餘人,降其眾數百,政敗奔烏撒,卿檄烏撒土舍安寧、土婦奢勿擒之。安寧佯許諾,僅以阿核等屍獻,竟不出政,兵久不解。都御史湯沐以聞,詔切責諸將及守巡官罪,而革何卿冠帶,令剿賊自贖。

四年,政誘殺壽,奪其印。巡撫王軏、巡按劉黻各上其事。黻言從蠻情,立支祿便。軏以隴政、支祿怙終稔惡,戕朝廷命吏,罪不可赦。乃命鎮巡官諭安寧,縛政、祿及諸助惡者。時政已為官軍擒於水西,追獲芒部印信,前後斬首六百七十四級,生擒一百六十七人,招撫白烏石等四十九寨,以捷聞。貴州巡按劉廷簠言:「烏撒所獻阿核等尸,及水西所縛隴政,真偽未可信,恐首惡尚在,不無後慮,請核實。」五年,兵部奏:「芒部隴氏,釁起蕭牆,騷動兩省,王師大舉,始克蕩平。今其本屬親支已盡,無人承襲,請改為鎮雄府,設流官知府統之。分屬夷良、毋響、落角利之地,為懷德、歸化、威信、安靜四長官司,使隴氏疏屬阿濟、白壽、祖保、阿萬四人統之。如程番府例,令三年一入朝,貢馬十二匹,而以通判程洸為試知府。」

六年,芒部賊沙保等謀復瓏氏,擁隴壽子勝糾眾攻陷鎮雄城,執程洸,奪其印,殺傷數百人,洸奔畢節。事聞,兵科給事中鄭自壁等言:「鎮雄初設流官,蠻情未服,而有司失先事之防,不亟收遣裔隴勝,而令沙保得擁孺子,致煽禍一方。宜速遣總兵何卿並力剿寇。」於是兵部覆言:「隴勝非真隴壽子,故議設流官,有司撫循失策,遂生叛亂。沙保罪不容誅,當剿。何卿方守松潘,勢難相援,宜亟趣都御史王廷相之任,並敕總兵牛桓調兵速進。」時沙保出鎮雄府印乞降,然尚持兩端,欲立土官如故。四川撫按以保狡悍不可馴,檄瀘州守備丁勇擊之。又遣使勞賜芒部撫夷郤良佐,使計擒沙保。保怒,復叛。

七年,川、貴諸軍會剿,敗沙保等,擒斬三百餘級,招撫蠻羅舅婦以千計。捷聞,設鎮雄流官如舊。而芒部、烏撒、毋響苗蠻隴革等復起,攻劫畢節屯堡,殺掠士民,紛紛見告。兵部尚書李承勛以伍文定專主用兵為失計,疏及之。而御史楊彞復言芒部改土易流非長策,又時值荒饉,小民救死不贍,何能趣戰。時帝亦軫念災傷,令罷芒部兵,俟有秋再議征討。於是四川巡撫唐鳳儀言:「烏蒙、烏撒、東川諸土官,故與芒部為脣齒。自芒部改流,諸部內懷不安,以是反者數起。今懷德長官阿濟等雖自詭擒賊,其心固望隴勝得一職,以存隴後。臣請如宣德中復安南故事,俯順輿情,則不假兵而禍源自塞。」川、貴巡按戴金、陳講等奏如鳳儀言。金又以首惡如毋響、祖保等,宜剿誅以折其驕氣,始下撫處之令,許生獻沙保等,待阿濟以不死,然後復隴勝故職,或降為知州。其長官或因或革,或分隸,庶操縱得宜,恩威並著。章下部覆,乃革鎮雄流官知府,而以隴勝為通判,署鎮雄府事。令三年後果能率職奉貢,準復知府舊銜。時嘉靖九年四月也。

三十九年命勘東川阿堂之亂。初,東川土知府祿慶死,子位幼,妻安氏攝府事。有營長阿得革頗擅權,謀奪其官。因先求烝安氏不得,乃縱火焚府治,走武定州,為土官所殺。得革子堂奔水西,賄結烏撒土官安泰,入東川,囚安氏,奪其印。貴州宣慰安萬銓故與祿氏姻連,乃起兵攻阿堂所居寨,破之。堂妻阿聚攜幼子奔沾益州土官安九鼎。萬銓脅九鼎,取阿聚及幼子殺之。堂以是怨九鼎,時相攻擊。堂兵侵羅雄州境,九鼎及祿位與羅雄土官者浚等,各上書訟堂罪。詔下雲、貴、四川撫按官會勘。堂聽勘於車洪江,具服罪,願獻所劫府印並沾益、羅雄人口牲畜及侵地,乞貸死。時位及弟僎已前歿,官府因訊祿氏所當襲者,堂以己幼子詭名祿哲以報。據府印如故,復與九鼎治兵相攻。九鼎訴之云南巡撫游居敬,謂堂怙亂,請致討,且自詭當率所部為前鋒,必擒堂以獻。居敬信之,遂上疏言堂念惡不悛,請專意進剿,為地方除害。帝允部議,行川、貴撫按會勘具奏。居敬遽調土漢兵五萬餘進剿。雲南承平久,一旦兵動,費用不貲,賦斂百出,諸軍衛及有司土官舍等乘之為姦利,遠近騷動。巡按王大任言:「逆堂奪印謀官,法所必誅。第彼猶借朝廷之印以約土蠻,冒祿氏之宗以圖世職,而四川之差稅辦納以時,雲、貴之鄰壞未見侵越,此其非叛明矣。其與九鼎治兵相攻,彼此俱屬有罪。居敬乃信一偏之詭辭,違會勘之明旨,輕動大眾,恐生意外患。且外議籍籍,謂居敬入九鼎重賄,欲為雪怨,及受各土官賂,攘盜帑積,皆有實跡。請亟罷居敬,暫停征剿為便。」乃命逮居敬。時堂聞大兵至東川,逃深箐,諸將分兵於新舊諸城,窮搜不獲,地方民夷大遭屠掠。

四十年,營長者阿易謀於堂之心腹母勒阿濟等,掩殺堂於戛來矣石之地,其子阿哲就擒,哲時年八歲。事雖定,而府印不知所在。於是安萬銓取東川府經歷印,畀祿位妻寧著署之,以照磨印畀羅雄土官者浚,而以寧著女妻者浚子。仍留水西兵三千於東川,為寧著防衛。水西與東川鄰,萬銓本水西土官,故議者謂其有陰據東川之志。巡按王大任以誅阿堂聞,因言:「東川地方殘傷,該府三印悉為土官部置,請通敕川、貴總督及鎮巡官,按究各土官私擅標署之罪。並訪祿氏支派之宜立,與所以處阿哲者。」部覆報可。

四十一年鑄給四川東川府印。初,阿堂既誅,索府印不獲,人疑為安萬銓所匿,及是屢勘,印實亡失。而祿位近派悉絕,惟同六世祖有幼男阿采。撫按官雷賀、陳瓚請以采襲祿氏職,姑予同知銜,令寧著署掌,後果能撫輯其眾,仍進襲知府。其新印請更名,以防奸偽。有旨不必更,餘如議。先是,烏撒與永寧、烏蒙、沾益、水西諸土官,境土相連,世戚親厚,既而以各私所親,彼此構禍,奏訐紛紜,詳四川《永寧土司傳》中,當事者頗厭苦之。萬歷六年乃令照蠻俗罰牛例處分,務悔禍息爭,以保境安民,然終不能靖也。

三十八年詔東川土司並聽雲南節制。時巡按鄧水美疏稱:「蜀之東川逼處武定、尋甸諸郡,只隔一嶺,出沒無時,朝發夕至。其酋長祿壽、祿哲兄弟,安忍無親,日尋干戈。其部落以劫殺為生,不事耕作。蜀轄遼遠,法紀易疏。滇以非我屬內,號令不行。以是驕蹇成習,目無漢法。今惟改敕滇撫兼制東川。」因條三利以進,詔從之。

先是,四川烏撒軍民府,雲南沾益州,雖滇、蜀異轄,宗派一源。明初大軍南下,女土官實卜與夫弟阿哥二人,率眾歸順,授實卜以烏撒土知府,授阿哥以沾益土知州。其後,彼絕此繼,通為一家。萬曆元年,沾益女土官安素儀無嗣,奏以土知府祿墨次子繼本州,即安紹慶也。已,祿墨及長子安雲龍與兩孫俱歿,安紹慶奏以次子安效良歸宗,襲土知府。安雲龍之妻隴氏,即鎮雄女土官者氏之女也,以雲龍雖故,尚有遺孤,且挾外家兵力,與紹慶為敵。紹慶則以隴氏所出,明係假子,亦倚沾益兵力,與隴氏為難。彼此仇殺,流毒一方。士民連名上奏,事行兩省會勘,歷十有四年不結。是年,安雲翔奏稱:「隴氏有子官保,今已長成。效良倚父兵,強圖竊據,殺戮無辜。」因極言效良不可立者數事。

三十九年,廷臣議行川、貴大吏勘報。貴州撫臣以土官爭職在雲南,而為害在黔、蜀,必得三省會勘,始可定獄。帝命速勘,乃命隴鶴書承襲鎮雄土知府。鶴書,原名阿卜,自其始祖隴飛沙獻土歸順,授為世職知府,五傳而為庶魯卜,別居於果利地,又四傳而為庶祿姑,別居夷良、七欠頭地,又五傳而隴氏之正支斬矣。水西安堯臣贅於祿,欲奄有之,眾論不平,始有驅安立隴之奏,奉旨察立隴後。女官者氏以阿固應。阿固者,魯卜之六世孫,而易名隴正名者也。於是主立阿固,而先立其父阿章。章尋病死,阿固不為夷眾所服,往復察勘。者氏及四十八目、十五火頭等共推阿卜。阿卜者,祿姑之五世孫,咸以為長且賢,而者氏且以印獻,遂定立阿卜,而以阿固充管事,從巡撫喬應星之議也。

四十一年,烏撒土舍安效良初與安雲翔爭立,朝廷以嫡派立效良。雲翔數為亂,謀逐效良,焚劫烏撒。四川撫按上其事,以效良為雲龍親姪,雲翔乃其堂弟,親疏判然,效良自當立。雲翔擾害地方,欺岡朝廷,罪原難赦,但為奸人指使,情可原,姑準復冠帶。從之。

四十三年,雲南巡按吳應琦言:「東川土官祿壽、祿哲爭襲以來,各縱部眾,越境劫掠。擁眾千餘,剽掠兩府,浹旬之間,村屯並掃,荼毒未有如此之甚者。或撫或剿,毋令養禍日滋。」下所司勘奏。貴州巡按御史楊鶴言:「烏撒土官,自安雲龍物故,安咀與安效良爭官奪印,仇殺者二十年。夷民無統,盜寇蜂起,堡屯焚毀,行賈梗絕者亦二十年。是爭官奪印者蜀之土官,而蹂踐糜爛者黔之赤子。誠改隸於黔,則彈壓既便,干戈可戢。」又言:「《烏撒者,滇、蜀之咽喉要地。臣由普安入滇,七日始達烏撒。見效良之父安紹慶據沾益,當曲靖之門戶。效良據烏撒,又扼滇、蜀之咽喉。父子各據一方,且壞地相接,無他郡縣上司以隔絕鈐制之,將來尾大不掉,實可寒心。蓋黔有可制之勢,而無其權;蜀有遙制之名,而無其實。誠以為隸黔中便。」帝命所司速議。

泰昌元年,雲南撫按沈儆炌等言:蜀之東川,業奉朝命兼制,然事權全不相關。祿千鐘、祿阿伽縱賊披猖,為患不已。是東川雖隸蜀,而相去甚遠,雖不隸滇,而禍實震鄰。宜特敕蜀撫按,凡遇襲替,務合兩省會勘。蜀察其世次,滇亦按無侵犯,方許起送,亦羈縻綏靜之要術也。」詔下所司。時諸土司皆桀驁難制,烏撒、東川、烏蒙、鎮雄諸府地界,復相錯於川、滇、黔、楚之間,統轄既分,事權不一,往往軼出為諸邊害。故封疆大吏紛紛陳情,冀安邊隅,而中樞之臣動諉勘報,彌年經月,卒無成畫,以致疆事日壞。播州初平,永寧又叛,水西煽起,東川、烏蒙、鎮雄皆觀望騎牆,心懷疑二。於是安效良以烏撒首附逆於邦彥,並力攻陸廣,復合沾益賊圍羅平,陷沾益,為雲南巡撫閔洪學所敗。洪學以兵力不繼,好語招之,令擒賊自贖,效良亦佯為恭順。又見黔師出陸廣,滇師出沾益,水、烏之勢已成騎虎,遂合永寧、水西諸部三十六營,直抵沾益,對壘城下五日。副總兵袁善、宣撫使沙源等督將士力戰,出奇兵破之,效良敗死。妻安氏無子,妾設白生其爵、其祿。二婦素不相能,安氏居鹽倉,設白母子居抱渡。安氏遂代效良為土官,然亦未絕其爵,其爵亦以安氏為安位姐,不敢抗。

崇禎元年,四川巡撫差官李友芝齎冠帶獎賞其爵母子,令管烏撒。安氏惡分,始絕其爵。其爵夜襲安氏鹽倉,不克,與設白、其祿逃東川界,為東川所拒,而抱渡又失。李友芝為請於制府,發滇兵三千援其爵,滇撫不應。安氏懼,謀迎沾益土官安邊為婚,授之烏撒以拒其爵。安邊亦欲偶安氏以拒其祿,以催糧為名至建昌。安氏遂迎邊至鹽倉成婚。一時皇皇謂水西必糾沾、烏入犯。雲南巡撫謝存仁以聞,存仁因移鎮曲靖以觀變。安邊、安氏請復烏撒衛以自贖。

二年,總督硃燮元調集漢土兵,列營沾益,趣滇撫會兵進烏撒境。安邊、安氏逃避偏橋。大兵入鹽倉,拔難民一千餘人。師還,安邊、安氏復還鹽倉,遣人至軍前,請俟烏城克復,束身歸命,意實緩師。乃復發兵逐安邊、安氏,以鹽倉授其爵。兵至望城坡,遇賊哨騎百餘,麾兵奮擊,賊盡奔箐中,遂復烏撒城。安邊駐三十里外,擁兵求見,諭令束身歸誠。邊夜遁,遂棄鹽倉,入九龍囤。烏撒陷賊八年,至是始復。乃召其爵來鹽倉,令約束九頭目以守,且令圖獻安邊、安氏。其爵以鹽倉殘毀,乞移烏撒城,從之。時其爵署烏撒知府,其祿署沾益知州,雖懦稚頗忠順,其母亦頗有主持,能得眾。安邊屢乞降於總督朱燮元,用藉水西安位代申,以邊實紹慶嫡孫,宜襲知州,請罪其爵、其祿。燮元曲為調護,欲予以職銜,分烏撒安置之。雲南撫按堅執不可,以安邊令其黨勒兵於野馬川,復以千金誘其爵頭目,日為并吞沾、烏計。萬一其爵被襲,則烏撒失,而前功盡棄。烏撒失,沾益危,而全滇動搖,非但震鄰,實乃切膚。竟不行。安邊乃乞師於安位,納之沾益,而逐其祿,時安氏在也。既而安氏死。安位與之貳,其祿乃假手羅彩令者布發難,邊遄死。不移日,其祿率兵至,詭言為其叔報仇,士民歸者如流,於是其祿復有沾益。而廟堂之上方急流寇,不復能問云。

馬湖,漢牂牁郡內地也,有龍馬湖,因名焉。唐為羈縻州四,總名馬湖部。洪武四年冬,馬湖路總管安濟,遣其子仁來歸附,詔改馬湖路為馬湖府。領長官司四:「曰泥溪,曰平夷,曰蠻夷,曰沭川。以安濟為知府,世襲。六年,安濟以病告,乞以子安仁代職,詔從之。自是,三年一入貢。七年,馬湖知府氏德遣其弟阿穆上表貢馬,廷臣言:「洪武四年,大兵下蜀,氏德叔安濟遣子入朝,朝廷授以世襲知府,恩至渥矣。今氏德既襲其職,不自來朝而遣其弟,非奉上之道。」帝卻其所貢馬。十二年,氏德貢香楠木,詔賜衣鈔。十六年,氏德來朝,獻馬十八匹,賜衣一襲、米二十石、鈔三十錠。

永樂十二年,泥溪、平夷、蠻夷、沐川四長官司遣人貢方物,賜鈔幣。宣德八年,平夷長官司奏,比者火延公廨,凡朝廷頒降榜文、倉庫稅糧錢帛及案牘皆救免,乞宥罪,並獻馬二匹。帝曰:「遠蠻能恭謹畏法如此。」置不問。正統二年,泥溪土官醫學正科田璣盜官藏絲鈔,援永、宣時例,邊夷有犯,聽以馬贖,許之。三年,免馬湖府舉人王有學棄吏。先是,有學會試,過期不至,例充吏。有學原籍長官司,因遣通事貢馬,乞宥罪,仍肄習太學,許之。

弘治八年,土知府安鰲有罪,伏誅。鰲性殘忍虐民,計口賦錢,歲入銀萬計。土民有婦女,多淫之。用妖僧百足魘魅殺人。又令人殺平夷長官王大慶,大慶聞而逃,乃殺其弟。為橫二十年。巡按御史張鸞請治之,得實,伏誅,遂改馬湖府為流官知府。

建昌衛,本邛都地。漢武帝置越巂郡。隨、唐皆為巂州。至德初,沒於吐番。貞元中收復。懿宗時,為蒙詔所據,改建昌府,以烏、白二蠻實之。元至元間,置建昌路,又立羅羅斯宣慰司以統之。

洪武五年,羅羅斯宣慰安定來朝,而建昌尚未歸附,十四年遣內臣齎敕諭之,乃降。十五年置建昌衛指揮使司。元平章月魯帖木兒等自雲南建昌來貢馬一百八十匹,並上元所授符印。詔賜月魯帖木兒綺衣、金帶、靴襪,家人綿布一百六十疋、鈔二千四百四十錠。以月魯帖木兒為建昌衛指揮使,月給三品俸贍其家。十六年,建昌土官安配及土酋阿派先後來朝,貢馬及方物,皆賜織金文綺、衣帽、靴襪。十八年,月魯帖木兒舉家來朝,請遣子入學,厚賜遣之。二十一年,建昌府故土官安思正妻師克等來朝,貢馬九十九匹。詔授師克知府,賜冠帶、襲衣、文綺、鈔錠,因命師克討東川、芒部及赤水河叛蠻。二十三年,安配遣子僧保等四十二人入監讀書。二十五年,致仕指揮安配貢馬,詔賜配及其把事五十三人幣紗有差。

已而月魯帖木兒反,合德昌、會川、迷易、柏興、邛部並西番土軍萬餘人,殺官軍男婦二百餘口,掠屯牛,燒營屋,劫軍糧,率眾攻城。指揮使安的以所部兵出戰,敗之,斬八十餘級,擒其黨十餘人。賊退屯阿宜河,轉攻蘇州。指揮僉事魯毅率精騎出西門擊之,賊眾大集,毅且戰且卻,復入城拒守。賊圍城,毅乘間遣壯士王旱突入賊營,斫賊,賊驚遁。於是置建昌、蘇州二軍民指揮使司及會川軍民千戶所,調京衛及陜西兵萬五千餘人往戍之。仍諭將士互相應援,設伏出奇,並諭擒首獻者賞千金。復諭總兵官涼國公藍玉,以月魯帖木兒詭詐,不可信其降,致緩師養禍。四川都指揮使瞿能率各衛兵至雙狼寨,擒偽千戶段太平等,賊眾大潰,月魯帖木兒敗遁。能督兵追捕,攻托落寨,拔之。轉戰而前,進至打沖河三里所,與月魯帖木兒遇,大戰,又敗之。俘其眾五百餘人,溺死者千餘,獲牛馬無算。官軍入德昌,能遂調指揮同知徐凱分兵入普濟州搜捕。復駕橋於打沖河,遣指揮李華引兵追托落寨餘孽,進至水西,斬月魯帖木兒把事七人,其截路寨土蠻長沙、納的皆中矢死。能還攻天星、臥漂諸寨,皆克之,先後俘殺千八百餘人。月魯帖木兒遁入柏興州。

帝遣諭藍玉曰:「月魯帖木兒信其逆黨達達、楊把事等,或遣之先降,或親來覘我,不可不密為防。其柏興州賈哈喇境內麼些等部,更須留意。」賈哈喇者,麼些洞土酋也。初,王師克建昌,授以指揮之職,自是從月魯帖木兒叛。玉率兵至柏興州,遣百戶毛海以計誘致月魯帖木兒並其子胖伯,遂降其眾,送月魯帖木兒京師,伏誅。玉因奏:「四川地曠山險,控扼西番。松、茂、碉、黎當吐番出入之地,馬湖、建昌、嘉定俱為要道,皆宜增屯衛。」報可,命玉班師。

二十七年,麼些洞蠻寇打沖河西守堡,都督徐凱擊敗之。二十九年,威龍土知州普習叛。普習,月魯帖木兒妻兄也。官軍捕之,普習中流矢死。三十一年,徐凱等平卜木瓦寨,執賈哈喇,送京師,誅之。寨地峻險,三百陡絕,下臨大江,江流悍急,不可行舟,惟一道僅可通人行。官軍至,輒自上投石,不得進。凱乃斷其汲道困之,寇窮促,凱督將士抵其寨,力攻破之,遂就擒。因改建昌路為建昌衛,置軍民指揮使司。安氏世襲指揮使,不給印,置其居於城東郭外里許。所屬有四十八馬站,大頭土番、僰人子、白夷、麼些、作佫鹿、保羅、韃靼、回紇諸種散居山谷間。北至大渡,南及金沙江,東抵烏蒙,西訖鹽井,延袤千餘里。以昌州、普濟、威龍三州長官隸之,有把事四人,世轄其眾,皆節制於四川行都指揮使司。西南土官,安氏殆為稱首。配六世孫安忠無後,妻鳳氏管指揮使事。鳳氏死,族人安登繼襲,復無子,妻瞿氏管事,以族人世隆嗣。世隆復無子,繼妻祿氏管事。祿死,以族姪安崇業嗣。崇業與祿氏不相能,因養那固為假子,其奴祿祈從臾構難,歲仇殺。鎮巡官讞之,殺那固而戍祿祈,事遂平。安氏所轄四驛,曰祿馬、阿用、白水、瀘沽,各百里有差。其涼山拖郎、桐槽、熱水諸番,則以強弱為向背。所領昌州等三長官司,皆在衛東、西、南三百里內。洪武十八年,土官盧尼姑、吉撒加、白氐等歸附,皆令世襲為知州。月魯帖木兒之亂,諸州皆廢革。永樂元年復置,悉改為長官司,仍隸建昌。其千戶所之隸於衛者有三:曰禮州,曰打沖河,曰德昌。禮州,漢蘇示縣;打沖河,唐沙野城;德昌,元定昌路也。

寧番衛,元時立於邛都之野,曰蘇州。洪武間,土官怕兀它從月魯帖木兒為亂,廢州置衛。環而居者,皆西番種,故曰寧番。有冕山、鎮西、禮州中三千戶所。

越巂衛,漢邛都及闌二縣地。有奴諾城,即蜀漢時諸葛亮征蠻所築以憩軍者也。元置邛部安撫招討司,已,改邛部州。滿武中,嶺真伯以招討使來歸,因改為邛部軍民州。洪武二十五年置越巂軍民指揮使司於邛部州,命指揮僉事李質領謫戍軍士守之。二十六年置越巂衛。永樂元年改邛部為長官司,隸越巂衛。萬歷中,土官嶺柏死,孽子應升負印去,柏妾沙氏爭之不得。土目阿堆等擁沙氏,焚利濟站廬舍,擁兵臨城。總兵劉顯率兵往撫之,沙氏悔禍,殺阿堆等自贖,顯遂以印授之。後沙氏淫於族人阿祭,印復為昇所奪。祭死。其子嶺鳳起嗾他番刺殺應升。鎮守官因平蠻之師,誘鳳起縶之,收其印,而誅從鳳起為亂者百餘人。印無所歸,緘於庫。部眾無統,肆行為盜。普雄部眾姑咱等乘勢蜂起,郵傳不通,遠近震恐。十五年,鎮巡官會師討之,斬馘千數,鳳起病死,其眾爭歸附,因置平夷、歸化二堡以居之。有鎮西千戶所。

鹽井衛,古定笮縣也。元初為落蘭部。至元中,於黑、白鹽井置閏鹽縣,於縣置柏興府。洪武中,改為柏興千戶所,旋改鹽井衛,又於二井置鹽課司。永樂五年設馬刺長官司;其村落多白夷居之。長官民阿氏,洪武時歸附,授世職。地接雲南北勝州,稱庶富,人亦擾馴。

打沖河守御中左千戶所,其土千戶刺兀,於洪武二十五年征賈哈喇順來歸。其子馬刺非復貢馬赴京,授本所副千戶。永樂十一年陞正,以別於四所。地與麗江、永寧二府鄰,麗江土官木氏侵削其地幾半。

會川衛,越巂之會無縣也。唐上元中,移邛都縣於會川鎮,以川原並會故名。宋屬大理,為會川府。元置會川路,治武安州,隸羅羅斯宣慰司。洪武十七年,會川土同知馬誠來朝,復立會川府,領武安、永昌、麻龍等州。二十六年革會川府。初,月魯帖木兒反,土知府王春陷會川,毀民居府治,至是遂墮其城。尋改為會川衛軍民指揮使司,領迷易千戶所。土官賢姓,其先雲南景東僰種也,徙其屬來田種。洪武十六歸附,以隨征東川、芒部勞,授世襲副千戶。居所治城外,所轄僰蠻僅八百戶。

茂州,古冉龍國地。漢武帝置汶山郡,宣帝為北部都尉。隋為蜀州,尋改會州。唐貞觀改茂州。宋、元仍舊,治汶山縣。洪武六年,茂州權知州楊者七及隴木頭、靜州、岳希蓬諸土官來朝貢。十一年置茂州衛指揮使司。時四川都司遣兵修灌縣橋梁至陶關,汶川土酋孟道貴疑之,集部落阻陶關道。都司遣指揮胡淵、童勝等統兵分二道擊之,一由石泉,一由灌口。由灌口者進次陶關,蠻眾伏兩山間,投石崖下,兵不能進。適汶川土官來降,得其間道。乃選勇士卷旗甲,乘夜潛出兩山後,遲明從山頂張旗幟,發火砲,蠻驚潰。師進雁門關,道險,蠻復據之。乃駐平野,得小舟渡,至龍止鐵冶寨,擊破之。其由石泉者次泥池,蠻悉眾拒。千戶薛文突陣射卻之,士卒奮擊,大敗其眾。兩軍遂會於茂州,楊者七迎降,以者七仍領其州。乃詔立茂州衛,留指揮楚華將兵三千守之。十五年,者七陰結生番,約日伏兵陷城。有小校密告於官,遂發兵捕斬者七。生番不之覺,如期入寇,官軍掩擊敗之,於是盡徙羌民於城外。

正德二年,太監羅籥奏,茂州所轄卜南村、曲山等寨,乞為白人,願納糧差。其俗以白為善,以黑為惡。禮部覆,番人向化,宜令入貢給賞。從之。十四年,巡撫馬昊調松潘兵,攻小東路番寨,而茂州核桃溝上、下關番蠻懼,遂糾白石、羅打鼓諸寨生番,攻圍城堡,游擊張傑敗績。十五年,巡撫盛應期奏,綽頭番犯松州,總兵張傑克之,復犯雄溪屯,指揮杜欽敗之,煙崇等寨皆降。萬曆十九年,威、茂諸番作亂,攻破新橋,乘勢圍普安等堡。四川巡撫李尚忠檄諸路兵追剿過河,普安諸堡得以保全。

茂州地方數千里,自唐武德改郡會州,領羈縻州九,前後皆蠻族,向無城郭。宋熙寧中,范百常知茂州,民請築城,而蠻人來爭。百常與之拒,且戰且築,城乃得立。自宋迄元,皆為羌人所據,不置州縣者幾二百年。洪武十一年平蜀,置壘溪右千戶所,隸茂州衛。而置威茂道,開府茂州,分游擊以駐疊溪,規防始立。然東路生羌,白草最強,又與松潘黃毛韃相通,出沒為寇,相沿不絕云。其通西域要路,為桃坪,即古桃關也,有繩橋渡江。守桃坪者,為隴木司。

茂州長官司三:曰隴木,曰靜州,曰疊溪。隴木長官司,其長官即隴木里人也。洪武時歸附,授承直郎,世襲長官,歲貢馬二匹。所屬玉亭、神溪十二寨,俱為編氓,有保長統之。靜州長官司,其地即唐之悉唐縣,其長官亦靜州里人也。襲官貢馬,與隴木同。正德間,與岳希蓬、節孝為亂,攻茂城,斷水道七日。節孝弟車勺潛引水以濟我軍。事平,使車勺襲職,轄法虎、核桃溝八寨,俱編戶為氓,亦有保長統之。疊溪千戶所,永樂四年置。領長官司二:曰疊溪,在治北一里;曰鬱即,在治西十五里。疊溪郁氏,洪武十五年歸附,給印世襲,凡三年貢馬四匹。長官所轄河東熟番八寨,皆大姓,及馬路、小關七族。其土舍轄河西小姓六寨。地土廣遠。饒畜產,稞麥路積。人皆梟黠,名雖熟番,與生番等。鬱即長官啖保,萬曆十八年與黑水、松坪稱兵,攻新橋,明年伏誅。漢關墩附近諸小姓,舊屬鬱即,至是改屬疊溪。初,都督方政平曆日諸寨,設長寧安撫司,隸松潘。至正統元年,總兵蔣貴言其遼闊,亦改隸於疊溪守御千戶。

松潘,古氐羌地。西漢置護羌校尉於此。唐初置松州都督,廣德初,陷於吐蕃。宋時,吐蕃將潘羅支領之,名潘州。元置吐蕃宣慰司。

洪武十二年,命平羌將軍御史大夫丁玉定其地,敕之曰:「松潘僻在萬山,接西戎之境,朕豈欲窮兵遠討,但羌戎屢寇邊,征之不獲已也。今捷至,知松州已克,徐將資糧於容州,進取潘州。若盡三州之地,則疊州不須窮兵,自當來服。須擇士勇者守納都、疊溪路,其驛道無阻遏者,不可守也。來降諸戎長,必遣入朝,朕親撫諭之。」遂並潘州於松州,置松州衛指揮使司。丁玉遣寧州衛指揮高顯城其地。十三年,帝以松州衛遠在山谷,屯種不給,饋餉為難,命罷之。未幾,指揮耿忠經略其地,奏言松州為番蜀要害地,不可罷,命復置。

十四年置松潘等處安撫司,以龍州知州薛文勝為安撫使,秩從五品。又置十三族長官司,秩正七品:曰勒都,曰阿昔洞,曰北定,曰牟力結,曰蛒匝,曰祈命,曰山洞,曰麥匝,曰者多,曰占藏先結,曰包藏先結,曰班班,曰白馬路。棋後復隸松潘者,長官司四,曰阿思,曰思囊兒,曰阿用,曰潘斡寨;安撫司四,曰八郎,曰阿角寨,曰麻兒匝,曰芒兒者。後又以思曩日安撫司附焉。諸長官司每三年入貢,賞賜如例。十五年,占藏先結等土酋來朝,貢馬一百三匹,詔賜綺鈔有差。十六年,秋忠言:「臣所轄松潘等處安撫司屬各長官司,宜以其戶口之數,量其民力,歲令納馬置驛,而籍其民充驛夫,供徭役。」從之。既而松潘羌民作亂,官兵討平之。甃松州及疊溪城。

十七年,松潘八積族老虎等寨蠻亂。官兵擊破之,獲馬一百二十,゛牛三百,犛牛九十。景川侯曹震請擇良馬貢京師,餘給軍,其゛牛、犛牛非中國所畜,令易糧餉犒軍,從之。十八年,松州羌反。成都衛指揮成信等率兵攻其牟力等寨,破之。兵還,又遇賊三千人於道,復擊敗之,追至乞刺河乃還。

二十年改松州衛松潘等處軍民指揮使司,改松潘安撫司為龍州。二十一年,雜貢生番則路、南向等引草地生番千餘人寇潘州阿昔洞長官司,殺傷人口。指揮周助率馬步軍同松潘衛軍討之,番寇率眾迎戰,千戶劉德破之,斬首三十四級,獲馬三十餘匹。賊潰,渡河四十餘里,復收敗卒屯聚。指揮周能追擊之,斬首一百三十餘級,獲馬六十餘匹,溺死甚眾,群番遠遁。二十六年,西番思曩日等族來歸,進馬百三十匹,命給金銅信符並賜文綺襲衣。

宣德二年,麻兒匝順化,喇嘛著八讓卜來歸。置麻兒匝安撫司,以喇嘛著八讓卜為安撫。麻兒匝在阿樂地,去松潘七百餘里。初,著八讓卜時侵掠邊民及遮八郎安撫司朝貢路。松潘衛指揮吳瑋遣人招之,因遣其侄完卜來貢獻,言其地廣民眾,過於八郎,請置宣撫司以轄之。帝命置安撫,遣敕諭之。四川巡按等奏松潘衛所轄阿用等寨蠻寇,擁眾萬餘,傷敗官軍,請討之。帝意邊將必有激之者。既四川都司奏至,言並非番寇。實由千戶錢宏因調發松潘官軍往征交址,眾憚行,宏詭言番寇至,當追捕,冀免調。又領軍突入麥匝諸族,逼取牛馬,致番人忿怨。復以大軍將致討懾之,番眾驚潰,約黑水生番為亂。帝命逮宏等,而責諸司怠玩邊務,亟捕諸傷官軍者。遣都指揮僉事蔣貴往,同松潘衛指揮吳瑋招撫番寇,令調附近諸衛軍二萬人以行。時賊圍松潘、疊溪、茂州,斷索橋,官軍與戰皆敗,出掠綿竹諸縣,官署民居皆被焚毀,鎮撫侯璉死之。蜀王護衛官校七千人來援,命都督陳懷與指揮蔣貴等合師亟討之,而梟宏於松潘以徇,並竄諸將之貪淫玩寇者。三年,陳懷等率諸軍屢敗賊於圪答壩、葉棠關,奪永鎮等橋,復疊溪,撫定祁命等十族,又招降渴卓等二十餘寨,松潘平。

八年,八部安撫司及思囊兒十四族朝貢之使陛辭,令齎敕還諭其土官,俾約束所轄蠻民,安分循理,毋作過以取罪戾。九年,敕指揮僉事方政、蔣貴等撫剿松潘。政等至,榜諭禍福,威、茂諸衛俱聽命,惟松潘、疊溪所轄任昌、巴豬、黑虎等寨梗化。政令指揮趙得、宮聚等以次進兵,平龍溪等三十七寨,班師還。命蔣貴佩平蠻將軍印,鎮守松潘。十年,貴奏,比因番人不靖,松潘、疊溪諸處倉糧,銷殆盡,別無儲積。帝命戶部於四川歲運之數,量益二分給之。

正統三年,巖州長官司讓達作亂,侵雜道諸邊,要道長官安白訴於朝。帝命四川三司往諭之,皆歸服。四年,松潘指揮趙得奏:「祁命族番寇商巴作亂,官軍捕擒之。其弟小商巴復聚浦江、新塘等關,據險劫掠,乞發大軍剿除。」帝命李安棄總兵官,王翱參贊軍務,調成都左衛官軍及松潘土兵,合二萬人征之。已,翱知商巴為都指揮趙諒所陷,乃按誅諒而釋商巴等,事遂已。

九年,松潘指揮僉事王杲奏:「比者,黑虎等寨番蠻攻圍椒園、松溪等關堡,殺傷官民。欲行擒剿,恐各寨驚疑,應諭能擒賊者重賞之。」報可。十年,黑虎寨賊首多兒太伏誅。初,多兒太掠茂州境,為官軍所獲,誠而釋之。未幾,復糾諸寨入掠。帝命序班祁全往諭諸寨,擒多兒太至京,梟其首。十一年以寇深為僉都御史,提督松潘兵備。時松潘皆已向化,惟歪地骨鹿簇二十寨不服,命督高廣、王杲等剿之。設思曩日安撫司,以阿思觀為之使,隸松潘衛。先是,阿思觀父端葛,洪武中歸順,給金牌撫番,至阿思觀又能招撫,故有是命。

景泰三年,鎮守松潘刑部左侍郎羅綺等奏:「雪兒卜寨賊首卓時芳等,煙崇寨賊首阿兒結等,累年糾合於安化關劫掠。臣會師抵其巢穴,斬首不計其數,生擒卓時芳、阿兒結等,梟斬於市。」七年,提督松潘羅綺復奏:「松潘土番王永習性兒獷,嘗殺其土官高茂林男婦五百餘口,及故土官董敏子伯浩等二十餘人。今又糾合番蠻,攻劫地方。臣與指揮周貴等統領官軍,直抵桑坪,已將永等誅滅,邊境肅清。」降敕褒賞。天順五年,番眾入龍安、石泉等處,擾糧道。六年敕松潘總兵許貴曰:「敘州蠻賊出沒為患,比松潘尤甚,其馳往會剿。」貴聞命,會兵敘州,追討昔乖件、莫洞、都夜三寨,分兵兩哨,克硬寨四十餘,斬首一千一百餘級。

成化二年,鎮守太監閻禮奏:「松、茂、疊溪所轄白草壩等寨,番羌聚眾五百人,越龍州境剽掠。白草番者,唐吐蕃贊普遺種,上下凡十八寨。部曲素強,恃其險阻,往往剽奪為患。」四年,禮復奏:「白草諸番擁眾寇安縣、石泉諸處,因各軍俱調征山都掌蠻,致指揮王璟備禦不謹。命副總兵盧能剿之。能遣指揮閻斌巡邊至廟子溝,番賊三百突至,殺傷相當。斌以失機逮治。九年,巡撫夏塤奏:「黑虎寨賊首夜合等攻關堡,左參將宰用、兵備副使沈琮督兵馳詣松溪堡敗之,斬獲夜合等三十六級。」松潘指揮僉事堯彧奏:「臣與兵備沈琮分剿白馬路水土、茹兒等番寨,大克之。

弘治二年,松潘番寇殺傷平夷堡官軍,命逮指揮以下各官治之。三年免思曩日安撫等十六族明年朝覲,以守臣言其地方災傷也。七年,松潘空心寨番賊犯邊,都指揮僉事李鎬敗之。十三年,番賊入犯松潘壩州坡抵關,勢益獗。」命逮指揮湯綱等,而敕巡撫張瓚調漢、土官兵五萬,由東南二路分剿,破白羊嶺、鵝飲溪等三十一寨,斬四百餘級。商巴等二十六族皆納款。十四年復攻黃頭、青水諸寨,前後殺獲男婦七百餘人,赭其碉房九百,墜崖死者不可勝計,諸番稍靖。

正德元年,巡撫劉洪奏:「祈命族八長官司所攝番眾多至三十寨,少亦二十餘寨,環布松潘兩河。其土官已故子孫,自應承襲。今宜察勘,有原降印信者,方許襲。」報可。十六年,松潘衛熟番八大禳等作亂,同知杜欽平之。

嘉靖五年命都督僉事何卿鎮守松潘。時黑虎五寨及烏都、鵓鴿諸番叛,卿次第平之,降者日至。卿有威望,在鎮十七年,松潘以寧。二十三年以北警召卿入衛,繼之者李爵、高岡鳳,未幾皆為巡撫劾罷。二十六年復命卿往鎮。時白草番亂,卿會巡撫張時徹討擒渠惡數人,俘斬九百七十餘級,克營寨四十七,毀碉房四千八百,獲馬牛器械儲積無算。終嘉靖世,松潘鎮號得人,邊境安堵焉。初,龍州薛文勝於洪武六年來降,命仍知龍州。既置松潘安撫司,命文勝為安撫使。既置松州衛,仍以松潘為龍州。宣德七年升龍州為宣撫司,以土知州薛忠義為宣撫使。龍州者,漢陰平道也。宋景定間,臨邛進士薛嚴來守是州,捍衛有功,得世襲。自文勝歸附,其部長李仁廣、王祥皆輸糧餉有功,亦得世襲。及宣德中,以征松潘功,升州為宣撫使,仁廣為副使,祥為僉事,各統兵五百世守白馬、白草、木瓜番地。至嘉靖四十四年,宣撫薛兆乾與副使李蕃相仇訐,兆乾率眾圍執蕃父子,毆殺之。撫按檄兵備僉事趙教勘其事。兆乾懼,與母陳氏及諸左右糾白草番眾數千人,分據各關隘拒命,絕松潘餉道。脅僉事王華,不從,屠其家。居民被焚掠者無算。是年春,與官軍戰,不利,求救於上下十八族番蠻,皆不應。兆乾率其家屬奔至石壩,官軍追及之,就擒。四十五年,兆乾伏誅,籍其家,母及其黨二十二人皆以同謀論斬,餘黨悉平。遂改龍州宣撫司為龍安府,設立流官如馬湖,而割保寧之江油、成都之石泉二縣分隸之。

萬曆八年,雪山國師喇嘛等四十八寨,勾北邊部落為寇,圍漳臘,守備張良賢破之。犯鎮虜,百戶杜世仁力戰,城得全,世仁死焉。又犯制臺,良賢復擊之,追至思答弄,連戰大破之,火落赤之侄小王子死焉。十九年,巡按李化龍言:「松潘為四川屏蔽,疊、茂為松潘咽喉。番戎作梗,松潘力不能支,宜移四川總兵於松潘以備防禦。」是時疊、茂諸番眾糾結為亂,鎮巡官率兵剿之,俘馘八百餘級,番寇亦斬其部長黑卜、白什等,獻功贖罪。而松坪諸惡屯據大雪山頂,諸將卒搜討,亦有斬獲。以捷聞,遂設平武縣於龍安府。

松潘以孤城介絕域,寄一線饋運路於龍州,制守為難。洪武時欲棄者數,以形勝扼險,不可罷,乃內修屯務,外輯羌戎,因俗拊循,擇人為理,番眾相安者垂四十餘年。及宣德初,調兵啟釁,致動干戈,自是置鎮建牙,宿重兵以資彈壓,亦時服時叛。自漳臘以北即為大荒,斯籌邊者之所亟圖也。

天全,古氐羌地。五代孟蜀時,置碉門、黎、雅、長河西、魚通、寧遠六軍安撫司。宋因之,隸雅州。元置六安撫司,屬土番等處宣慰司,後改六番招討,又分置天全招討司。明初並為天全六番招討司,隸四川都司。

洪武六年,天全六番招討使高英遣子敬嚴等來朝,貢方物。帝賜以文綺龍衣。以英為正招討,楊藏卜為副招討,秩從五品,每三歲入貢,賜予甚厚。二十一年,楊藏卜來朝,言茶戶向與西番貿易,歲收其課。近在官收買,額遂虧,乞從民便,許之。先是,高敬嚴襲招討使,偕楊藏卜奏請簡土民為兵,以守邊境,詔許之。敬嚴等遂招選土民,教以戰陣,得馬步卒千餘人。至是藏卜來朝,奏其事,詔更天全六番招討司為武職,令戍守邊界,控制西番。三十一年,帝諭左都督徐增壽曰:「曩因碉門拒長河西口,道路險隘,以致往來跋涉艱難,市馬數少。今聞有路自碉門出枯木任場徑抵長河西口,通雜道長官司,道路平坦,往來徑直,可即檄所司開拓,以便往來。」

永樂二年,高敬讓來朝,并賀立皇太子,且遣其子虎入國子學,賜虎衣衾等物。十年,敬讓遣子虎貢馬。初,虎入國學讀書,以丁母憂去,至是服闋還監,皇太子命禮部賜予如例。

宣德五年,六番招討司奏:「舊額歲辦烏茶五萬斤,二年一次,運付碉門茶馬司易馬。今戶部令再辦芽茶二千二百斤,山深地瘠,艱於採辦,乞減其數。」帝令免烏茶只辦芽茶。十年命高鳳署天全六番招討司事。先是,敬讓以罪下獄死。至是,其子鳳乞襲父職。帝念其祖有撫綏功,命暫理招討事。正統四年命鳳襲。

正德十五年,招討高文林父子稱兵亂,副招討楊世仁亦助惡。命四川撫按官討之。初,文林等與蘆山縣民爭田構釁,知縣處置失宜,致叛亂。踰年,討斬文林,擒其子繼恩,擇其宗人承襲。

初,天全招討司治碉門城,元之碉門安撫司也,在雅州境。明初,宣慰餘思聰、王德貴歸附,始降司為州,設雅州千戶所,而設碉門百戶,近天全六番之界。又置茶課司以平互市。蓋其地為南詔咽喉,三十六番朝貢出入之路。三十六番者,皆西南諸部落,洪武初,先後至京,授職賜印。立都指揮使二:曰烏斯藏,曰朵甘。為宣慰司者三:曰朵甘,曰董卜韓胡,曰長河西魚通寧遠。為招討司者六,為萬戶府者四,為千戶所者十七,是為三十六種。或三年,或五年一朝貢,其道皆由雅州入,詳《西番傳》。

黎州,漢沈黎郡地。《史記》稱越巂以東北,君長以十數,筰都最大。自唐蒙通夜郎,邛、筰之君請為內臣,因置筰都縣,復曰旄牛縣。元鼎中,以為沈黎郡。唐割雅、巂二州置黎州。天寶初,改為洪源郡,尋改漢源。宋屬成都路。元屬土番等處宣慰司。

洪武八年省漢源縣,置黎州長官司,以芍德為長官。德,雲南人,馬姓。祖仕元,世襲邛部州六番招討使。明氏據蜀,德兄安復為黎州招討使。明氏亡,蠻民潰散,德奉母還居邛部。至是,四川布政司招之,德遂來朝貢馬,請置長官司。詔以德為黎州長官,賜印及衣服綺帛,十一年升為黎州安撫司,即以德為使。十四年,德遣使貢馬。詔賜德鈔五十四錠、文綺七疋。自是,三年一入貢。弘治十四年命黎州安撫隸四川都司。

萬曆十九年,安撫馬祥無後,妻瞿氏掌司事,取瞿姓子撫之,將有他志。祥侄上舍居松坪者,遂興兵攻城,奪印,番眾乘機剽掠。時參將吳文傑方有征東之役,移師剿平之。二十四年降黎州安撫司為千戶所,立所治於司南三十里大田山壩。分上七枝編戶,屬大渡河千戶所,下七枝仍屬松坪馬氏約束。松坪在司之東南,自炒米城直接峨眉,高山峻阪三百餘里,皆安撫族人居之。

黎、雅諸蠻,宋時屢為邊患。明興,以諸蠻皆天全六番諸部,散居於二州之境,遂於黎州設安撫,於天全六番設招討,以示羈縻。而雅州所屬,與招討所轄之蠻民,境土相連,時有爭訟。徼外大、小木瓜種分三枝,膩乃卜最強,世居西河。初屬馬湖土官安氏鈐轄,自馬湖改流,諸瓜叛入邛部,歸嶺氏。其地自西河至涼山、雪山諸處,周圍蟠據。嘉靖末,諸瓜畜牧蕃盛,時窺邊,邛部長官嶺柏不能制,嘉、峨、犍為諸邊皆為侵擾。鎮巡官督邛部兵捕之,瓜兵益熾,乃議大征,分建昌、越巂、馬湖三路兵進討。瓜部始惶駭請降,願歲貢馬方物,乃定。其地四千八百四十餘畝,徵糧四百四十餘石,輸峨眉縣。明初與安撫司同置者,有大渡河守御千戶所。唐時,河平廣可通漕,戍將一不守,則黎、雅、邛、嘉、成都皆動搖。宋建隆三年,王全斌平蜀,以圖來上。議者欲因兵威復越巂,藝祖以玉斧畫圖曰:「外此,吾不有也。」自是之後,河中流忽陷下五六十丈,水至此,洶湧如空中落,船筏不通,名為噎口,殆天設險以限內外雲。


\end{pinyinscope}