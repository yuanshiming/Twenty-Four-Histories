\article{輿服志}

大輅玉輅大馬輦小馬輦步輦大涼步輦板轎耕根車后妃車輿皇太子親王以下車輿公卿以下車輿傘蓋鞍轡

有虞氏御天下,車服以庸。夏則黻冕致美。商則大輅示儉。成周有巾車、典輅、弁師、司服之職,天子以之表式萬邦,而服車五乘,下逮臣民。漢承秦制,御金根為乘輿,服袀玄以承大祀。東都乃有九斿、雲罕、旒冕、絇屨之儀物,踵事增華,日新代異。江左偏安,玉輅棲寶鳳,采旄銜金龍。其服冕也,或飾翡翠、珊瑚、雜珠。豈古所謂法駕、法服者哉?唐武德間著車輿、衣服之制,上得兼下,下不得擬上。宋初,袞冕不綴珠玉。政和中詔修車輅,並建旂常,議禮局所釐定,用為成憲。元制,郊祀則駕玉輅,服袞冕;巡幸,或乘象轎,四時質孫之服,各隨其宜。明太祖甫有天下,考定邦禮,車服尚質。酌古通今,合乎禮意。迄於世宗,耤田造耕根,燕居服燕弁,講武用武弁,更為忠靖冠以風有位,為保和冠以親宗籓,亦一王之制也。若夫前代傘扇、鞍勒之儀,門戟、旌節之屬,咸別等威,至宋加密。明初儉德開基,宮殿落成,不用文石甃地。以此坊民,武臣猶有飾金龍於床幔,馬廄用九五間數,而豪民亦或熔金為酒器,飾以玉珠。太祖皆重懲其弊。乃命儒臣稽古講禮,定官民服舍器用制度。歷代守之,遞有禁例。茲更以朝家冊寶、中外符信及宮室器用之等差,附敘於後焉。

天子車輅:明初大朝會,則拱衛司設五輅於奉天門,玉居中,左金,次革,右象,次木。駕出則乘玉輅,後有腰輿,以八人載之。其後太祖考《周禮》五輅,以詢儒臣,曰:「玉輅太侈,何若祗用木輅?」博士詹同對曰:「孔子云『乘殷之輅』,即木輅也。」太祖曰:「以玉飾車,古惟祀天用之,常乘宜用殷輅。然祀天之際,玉輅未備,木輅亦未為不可。」參政張昶曰:「木輅,戎輅也,不可以祀天。」太祖曰:「孔子斟酌四代禮樂,以為萬世法,木輅寧不可祀?祀在誠敬,豈泥儀文。」洪武元年,有司奏乘輿服御,應以金飾,詔用銅。有司言費小不足惜。太祖曰:「朕富有四海,豈吝乎此?第儉約非身先無以率下,且奢泰之習,未有不由小而至大者也。」六年,命禮官考五輅制,為木輅二乘。一以丹漆,祭祀用之;一以皮鞔,行幸用之。是冬,大輅成。命更造大輅一,象輅十,中宮輅一,後宮車十,飾俱以鳳。以將幸中立府,故造之,非常制也。二十六年,始定鹵簿大駕之制。玉輅一,大輅一,九龍車一,步輦一。後罷九龍車。永樂三年更定鹵簿大駕,有大輅、玉輅、大馬輦、小馬輦、步輦、大涼步輦、板轎各一,具服、幄殿各一。

大輅,高一丈三尺九寸五分,廣八尺二寸五分。輅座高四尺一寸有奇,上平盤。前後車欞並雁翅及四垂如意滴珠板。轅長二丈二尺九寸有奇,紅髹。鍍金銅龍頭、龍尾、龍鱗葉片裝釘。平盤下方箱,四周紅髹,匡俱十二槅。內飾綠地描金,繪獸六,麟、狻猊、犀、象、天馬、天祿;禽六,鸞、鳳、孔雀、朱雀、翟、鶴。盤左右下有護泥板及車輪二,貫軸一。每輪輻十有八,其輞皆紅髹,抹金銅鈒花葉片裝釘。輪內車心,用抹金銅鈒蓮花瓣輪盤裝釘,軸中纏黃絨駕轅諸索。輅亭高六尺七寸九分,四柱長五尺八寸四分。檻座皆紅髹。前二柱戧金,柱首寶相花,中雲龍文,下龜文錦。前左右有門,高五尺一寸九分,廣二尺四寸九分,四周裝雕木沉香色描金香草板十二片。門旁槅各二及明栨,俱紅髹,以抹金銅鈒花葉片裝釘,槅編以黃線條。後紅髹屏風,上雕描金雲龍五,紅髹板戧金雲龍一。屏後地沉香色,上四槅雕描金雲龍四,其次雲板如之。下三槅雕描金雲龍三,其次雲板亦如之。俱抹金銅鈒花葉片裝釘。亭內黃線條編紅髹匡軟座,下蓮花墜石,上施花毯、紅錦褥席、紅髹坐椅。靠背上雕描金雲龍一,下雕雲板一,紅髹福壽板一,并褥。椅中黃織金椅靠坐褥,四圍椅裙,施黃綺帷幔。亭外青綺緣邊紅簾十扇。輅頂并圓盤,高三尺有奇,鍍金銅蹲龍頂,帶仰覆蓮座,垂攀頂黃線圓條。盤上以紅髹,其下外四面地沉香色,描金雲;內四角地青,繪五彩雲。以青飾輅蓋,亭內貼金斗拱,承紅髹匡寶蓋,斗以八頂,冒以黃綺,謂之黃屋;中并四周繡五彩雲龍九。天輪三層,皆紅髹,上安雕木貼金邊耀葉板八十一片,內綠地雕木貼金雲龍文三層,間繪五彩雲襯板八十一片。盤下四周,黃銅釘裝,施黃綺瀝水三層,每層八十一摺,間繡五彩雲龍文。四角垂青綺絡帶,各繡五彩雲升龍。圓盤四角連輅坐板,用攀頂黃線圓條,并貼金木魚。輅亭前有左右轉角闌幹二扇,後一字帶左右轉角闌幹一扇,皆紅髹,內嵌雕木貼金龍,間以五彩雲。三扇共十二柱,柱首雕木貼金蹲龍及線金五彩蓮花抱柱。闌干內四周布花毯。亭後樹太常旂二,以黃線羅為之,皆十有二斿,每斿內外繡升龍一。左旂腰繡日月北斗,竿首用鍍金銅龍首。右旂腰繡黻字,竿首用鍍金銅戟。各綴抹金銅鈴二,垂紅纓十二,纓上施抹金銅寶蓋,下垂青線帉錔。踏梯一,紅髹,以抹金銅鈒花葉片裝釘。行馬架二,紅髹,上有黃絨匾條,用抹金銅葉片裝釘。有黃絹幰衣、即遮塵。油絹雨衣、青氈衣及紅油合扇梯、紅油托叉各一。輅以二象駕之。

玉輅,亦駕以二象,制如大輅,而無平盤下十二槅之飾。輅亭前二柱,飾以搏換貼金升龍。屏風後無上四槅雲龍及雲板之飾。天輪內用青地雕木飾玉色雲龍文。而太常旗及踏梯、行馬之類,悉與大輅同。

大馬輦,古者輦以人挽之。《周禮·巾車》后五輅,其一「輦車,組挽」。然《縣師》有「車輦之稽」,《黍苗》詩云「我任我輦」,則臣民所乘亦名輦。至秦始去其輪,而制乃尊。明諸輦有輪者駕以馬,以別於步輦焉。其制,高一丈二尺五寸九分,廣八尺九寸五分,轅長二丈五寸有奇,輦座高三尺四寸有奇,餘同大輅。輦亭高六尺四寸有奇,紅髹四柱,長五尺四寸有奇。檻座高與輅同,四周紅髹條環板。前左右有門,高五尺有奇,廣二尺四寸有奇。門旁槅各二,後槅三及明栨,皆紅髹,抹金銅鈒花葉片裝釘。槅心編以黃線條。亭內制與大輅同,第軟座上不用花毯,而用紅毯。亭外用紅簾十二扇。輦頂并圓盤高二尺六寸有奇,上下俱紅髹,以青飾輦蓋。其銅龍、蓮座、寶蓋、黃屋及天輪、輦亭,制悉與大輅同。太常旗、踏梯、行馬之屬,亦同大輅。駕以八馬,備鞍韉、鞦轡、鈴纓之飾。

小馬輦,視大馬輦高廣皆減一尺,轅長一丈九尺有奇,餘同大馬輦。輦亭高五尺五寸有奇,紅髹四柱,長五尺四寸有奇。檻座紅髹,四周條環板,前左右有門,高五尺,廣二尺二寸有奇。門旁槅各二及明栨,後屏風壁板,俱紅髹,用抹金銅鈒花葉片裝釘。亭底紅髹,上施紅花毯、紅錦褥席。外用紅簾四扇,駕以四馬。餘同大馬輦。

步輦者,古之步挽。明制,高一丈三尺二寸有奇,廣八尺二寸有奇。輦座高三尺二寸有奇,四周雕木五彩雲渾貼金龍板十二片,間以渾貼金仰覆蓮座,下雕木線金五彩雲板十二片。轅四,紅髹。中二轅長三丈五尺九寸,左右二轅長二丈九尺五寸有奇,俱以鍍金銅龍頭、龍尾裝釘。輦亭高六尺三寸有奇,四柱長六尺二寸有奇。檻座紅髹,四周雕木沉香色描金香草板十二片,抹金銅鈒花葉片裝釘。前左右有門,高五尺七寸有奇,廣二尺四寸有奇。門旁紅髹十字槅各二扇,雕飾沉香色描金雲龍板八片,下雲板如其數。後紅髹屏風,上雕沉香色描金雲龍五。屏後雕沉香色描金雲龍板三片,又雲板如其數,俱用抹金銅鈒花葉片裝釘。餘同馬輦,惟紅簾用十扇。輦頂并圓盤高二尺六寸有奇,其蓮座、輦蓋、天輪、AW衣之屬,俱同馬輦。

大涼步輦,高一丈二尺五寸有奇,廣一丈二尺五寸有奇。四面紅髹匡,裝雕木五彩雲板二十片,間以貼金仰覆蓮座,下紅髹如意條環板,如其數。紅髹轅六:中二轅長四丈三尺五寸有奇,左右二轅長四丈有奇,外二轅長三丈六尺五寸有奇,前後俱飾以雕木貼金龍頭、龍尾。輦亭高六尺五寸有奇,廣八尺五寸有奇,四柱紅髹。前左右有門,高五尺八寸有奇,廣二尺五寸有奇,四周描金香草板十二片。門旁槅各二,後槅三及明栨皆紅髹,編以黃線條。亭底上施墊氈,加紅錦褥并席。紅髹坐椅一,四周雕木沉香色,描金寶相花,靠背、褥、裙、帷幔與馬輦同。內設紅髹桌二;紅髹闌干香桌一,闌干四,柱首俱雕木貼金蹲龍;鍍金銅龍蓋香爐一,并香匙、箸、瓶;紅錦墩二。外紅簾三扇。輦頂高二尺七寸有奇,又鍍金銅寶珠頂,帶仰覆蓮座,高一尺三寸有奇;垂攀頂黃線圓條四。頂用丹漆,上冒紅氈,四垂以黃氈為如意雲,黃氈緣條;四周施黃綺瀝水三層,每層百三十二摺,間繡五彩雲龍文。或用大紅羅冒頂,以黃羅為如意雲緣條,瀝水亦用黃羅。頂下四周以紅氈為帷,黃氈緣條,四角鍍金銅雲四。亭內寶蓋繡五龍,頂以紅髹木匡,冒以黃綺為黃屋,頂心四周繡雲龍各一。輦亭四角至輦座,用攀頂黃線圓條四,并貼金木魚。輦亭前左右轉角闌干二扇,後一字帶轉角闌干一扇,皆紅髹,雕木渾貼金龍,間以五彩雲板。闌干內四周布席。其闌干十二柱之飾及踏梯之屬,俱與馬輦同。

轎者,肩行之車。宋中興以後,皇后嘗乘龍肩輿。又以征伐,道路險阻,詔百官乘轎,名曰「竹轎子」,亦曰「竹輿」。元皇帝用象轎,駕以二象。至用紅板轎,則自明始也。其制,高六尺九寸有奇。頂紅髹。近頂裝圓匡蜊房窗,鍍金銅火燄寶,帶仰覆蓮座,四角鍍金銅雲朵。轎杠二,前後以鍍金銅龍頭、龍尾裝釘,有黃絨墜角索。四周紅髹板,左右門二,用鍍金銅釘鉸。轎內紅髹匡坐椅一,福壽板一并褥。椅內黃織金綺靠坐褥,四周椅裙,下鋪席並踏褥。有黃絹轎衣、油絹雨衣各一,青氈衣,紅氈緣條雲子。嘉靖十三年謁廟,帝及后妃俱乘肩輿出宮,至奉天門降輿升輅。隆慶四年設郊祀慶成宴,帝乘板輿由歸極門出,入皇極門,至殿上降輿。

車駕之出,有具服幄殿。按《周官》大小次,木架葦障,上下四旁周以幄壩,以象宮室。明鹵簿載具服幄殿,儀仗有黃帳房,仍元制也。帳并帷幕,以黃木棉布為之。上施獸吻,柱竿紅髹,竿首彩裝蹲獅,氈頂。

耕根車,世宗朝始造。漢有耕車,晉曰耕根車,俱天子親耕所用。嘉靖十年,帝將耕耤田,詔造耕根車。禮官上言:「考《大明集禮》,耕耤用宋制,乘玉輅,以耕根車載耒耜同行。今考儀注,順天府官奉耒耜及穜AL種置彩輿,先於祭前二日而出。今用耕根車以載耒耜,宜令造車,於祭祀日早進呈,置耒耜,先玉輅以行。第稽諸禮書,只有圖式,而無高廣尺寸。宜依今置車式差小,通用青質。」從之。

皇后輅:一,高一丈一尺三寸有奇,平盤。前後車欞並雁翅,四垂如意滴珠板。轅長一丈九尺六寸,皆紅髹。轅用抹金銅鳳頭、鳳尾、鳳翎葉片裝釘。平盤左右垂護泥板及輪二,貫軸一。每輪輻十有八,皆紅髹,輞以抹金鈒花銅葉片裝釘。輪內車轂,用抹金銅鈒蓮花瓣輪盤裝釘,軸中纏黃絨駕轅諸索。輅亭高五尺八寸有奇,紅髹四柱。檻座上沉香色描金香草板十二片。前左右有門,高四尺五寸有奇,廣二尺四寸有奇。門旁沉香色線金菱花槅各二,下條環板,有明栨,抹金銅鈒花葉片裝釘。後紅髹五山屏風,戧金鸞鳳雲文,屏上紅髹板,戧金雲文,中裝雕木渾貼金鳳一。屏後紅髹板,俱用抹金銅鈒花葉片裝釘。亭底紅髹,上施紅花毯、紅錦褥席、紅髹坐椅一。靠背雕木線金五彩裝鳳一,上下香草雲板各一,紅福壽板一并褥。椅中黃織金綺靠坐褥,四周有椅裙,施黃綺帷幔。或黃線羅。外用紅簾十二扇。前二柱,戧金,上寶相花,中鸞鳳雲文,下龜文錦。輅頂并圓盤,高二尺有奇,抹金銅立鳳頂,帶仰覆蓮座,垂攀頂黃線圓條四。盤上紅髹,下四周沉香色描金雲文,內青地五彩雲文,以青飾輅蓋。內寶蓋,紅髹匡,斗以八頂,冒以黃綺;頂心及四周繡鳳九,并五彩雲文。天輪三層,紅髹,上雕木貼金邊耀葉板七十二片,內飾青地雕木五彩雲鸞鳳文三層,間繪五彩雲襯板七十二片。下四周黃銅裝釘,上施黃綺瀝水三層,間繡鸞鳳文。四垂青綺絡帶,繡鸞鳳各一。圓盤四角連輅座板,用攀頂黃線圓條四。輅亭前後有左右轉角闌干各二扇,內嵌條環板,皆紅髹;計十二柱,柱首雕木紅蓮花,線金青綠裝蓮花抱柱。其踏梯、行馬之屬,與大馬輦同。

安車,本《周禮》后五輅之一。應劭《漢官鹵簿圖》有五色安車。晉皇后乘雲母安車。唐皇后安車,制如金輅。明皇后安車獨簡素。其制,高九尺七寸有奇,平盤,前後車欞並雁翅板。轅二,長一丈六尺七寸有奇,皆紅髹,用抹金銅鳳頭、鳳尾、鳳翎葉片裝釘。平盤左右垂護泥板及輪二,貫軸一。每輪輻十有八,皆紅髹,軸中纏黃絨駕轅諸索。車亭高四尺四寸,紅髹方柱四,上裝五彩花板十二片。前左右有門,高三尺七寸有奇,廣二尺二寸有奇。門旁紅髹十字槅各二。後三山屏鳳,屏後壁板俱紅髹,用抹金銅鈒花葉片裝釘。亭底紅髹板,上施紅花毯、紅錦褥,四周施黃綺帷幔,外用紅簾四扇。車蓋用紅髹抹金銅寶珠頂,帶蓮座,高六寸,四角抹金銅鳳頭,用攀條四,並紅髹木魚。蓋施黃綺瀝水三層,銷金鸞鳳文,鳳頭下垂紅帉錔。其踏梯、行馬、AW衣與輅同。

行障:坐障,自唐、宋有之。皇后重翟車後,皆有行障六,坐障三,左右夾車宮人執之。而《唐書》、《宋史》不載其制。《金史》:行障長八尺,高六尺;坐障長七尺,高五尺。明皇后用行障、坐障,皆以紅綾為之,繪升降鸞鳳雲文;行障繪瑞草於瀝水,坐障繪雲文於頂。

太皇太后、皇太后輅及安車、行障、坐障,制與皇后同。

皇妃車曰鳳轎,與歷代異名。其制,青頂,上抹金銅珠頂,四角抹金銅飛鳳各一,垂銀香圓寶蓋并彩結。轎身紅髹木匡,三面篾織紋簟,繪以翟文,抹金銅鈒花葉片裝釘。紅髹㧏,飾以抹金銅鳳頭、鳳尾。青銷金羅緣邊紅簾並看帶,內紅交床並坐踏褥。紅銷金羅轎衣一頂,用銷金寶珠文;瀝水,香草文;看帶並幃,皆鳳文。紅油絹雨轎衣一。

自皇后以下,皆用行障二,坐障一,第別以彩繪。皇妃行障、坐障,俱紅綾為之,繪雲鳳,而行障瀝水繪香草。

皇太子金輅,高一丈二尺二寸有奇,廣八尺九寸。轅長一丈九尺五寸。輅座高三尺二寸有奇。平盤、滴珠板、輪輻、輪輞悉同玉輅。輅亭高六尺四寸有奇,紅髹四柱,長五尺四寸。檻座上四周線金五彩香草板。前左右有門,高五尺有奇,廣二尺四寸有奇。門旁槅各二,編紅線條及明栨,皆紅髹。後五山屏鳳,青地上雕木貼金龍五,間以五彩雲文。屏後紅髹板,皆抹金銅鈒花葉片裝釘。紅髹匡軟座,紅絨墜座,大索四,下垂蓮花墜石,上施紅毯紅錦褥席。紅髹椅一,納板一并褥。椅中紅織金綺靠坐褥,四周有椅裙,施紅羅帷幔,外用青綺緣邊。紅簾十二扇。椅雕貼金龍彩雲,下線金彩雲板一。亭內編紅線條。輅頂并圓盤,高二尺五寸有奇,又鍍金銅寶珠頂,帶仰覆蓮座,高九寸,垂攀頂紅線圓條四。盤上丹漆,下內外皆青地繪雲文,以青飾輅蓋。亭內周圍青斗拱,承以丹漆匡,寶蓋鬥以八頂,冒以紅綺,頂心繡雲龍,餘繡五彩雲文。天輪三層皆紅髹,上雕木貼金邊耀葉板七十二片,內飾青地雕木貼金雲龍文三層,間繪五彩雲襯板七十二片,四周黃銅裝釘。上施紅綺瀝水三層,每層七十二摺,間繡五彩雲龍文。四角之飾與大輅同,第圓條用紅線。輅亭前一字闌干一扇,後一字帶轉角闌干一扇,左右闌干二扇,內嵌五彩雲板,皆丹漆。計十四柱,柱首制與大輅同。亭後建紅旗二,以紅羅為之。九斿,每斿內外繡升龍一。左旗腰繡日月北斗,竿用抹金銅龍首。右旗腰繡黻字,竿用抹金銅戟。綴抹金銅鈴二,垂紅纓。其踏梯、行馬之屬,與玉輅同。帳房用青木棉布,竿首青綠蹲猊,餘同乘輿帳房。

東宮妃車,亦曰鳳轎、小轎,制同皇妃。行障、坐障之制亦同。

親王象輅,其高視金輅減六寸,其廣減一尺。轅長視大輅減一尺。輅座高三尺有奇,餘飾同金輅。輅亭高五尺二寸有奇,紅髹四柱。檻座上四周紅髹條環板。前左右有門,高四尺五寸有奇,廣二尺二寸有奇。門旁槅各二及明栨、後五山屏風,皆紅髹,用抹金銅鈒花葉片裝釘。亭底紅髹,施紅花毯、紅錦褥席。其椅靠、坐褥、帷幔、紅簾之制,俱同金輅。輅頂并圓盤,高二尺四寸有奇,用抹金銅寶珠頂,餘同金輅。天輪三層,皆紅髹,上雕木貼金邊耀葉板六十三片,內飾青地雕木五彩雲文三層,間繪五彩雲襯板六十三片,四周黃銅裝釘。上施紅綺瀝水三層,每層八十一摺,繡瑞草文。前垂青綺絡帶二,俱繡升龍五彩雲文。圓盤四角連輅座板,用攀頂紅線圓條四,并紅髹木魚。亭前後闌幹同金輅,左右闌干各一扇,內嵌條環板,皆紅髹。計十四柱,柱首雕木紅蓮花,線金青綠裝蓮花抱柱,前闌干內布花毯。紅旗二,與金輅所樹同,竿上只垂紅纓五。其踏梯、行馬之屬,亦同金輅。帳房用綠色螭頭,餘與東宮同。

親王妃車,亦曰鳳轎、小轎,制俱同東宮妃。惟鳳轎衣用木紅平羅。小轎衣二:一用礬紅素紵絲,一用木紅平羅。行障、坐障,制同東宮妃。

公主車,宋用厭翟車,明初因之。其後定制,鳳轎、行障、坐障,如親王妃。

皇孫車,永樂中,定皇太孫婚禮儀仗如親王,降皇太子一等,而用象輅。

郡王無輅,只有帳房,制同親王。

郡王妃及郡主俱用翟轎,制與皇妃鳳轎同,第易鳳為翟。行障、坐障同親王妃,而繪雲翟文。

百官乘車之制:洪武元年令,凡車不得雕飾龍鳳文。職官一品至三品,用間金飾銀螭繡帶,青縵。四品五品,素獅頭繡帶,青縵。六品至九品,用素雲頭青帶,青縵。轎同車制。庶民車及轎,並用黑油,齊頭平頂,皁縵,禁用雲頭。六年令,凡車轎禁丹漆,五品以上車止用青縵。婦女許坐轎,官民老疾者亦得乘之。景泰四年令,在京三品以上得乘轎。弘治七年令,文武官例應乘轎者,以四人舁之。其五府管事,內外鎮守、守備及公、侯、伯、都督等,不問老少,皆不得乘轎,違例乘轎及擅用八人者,奏聞。蓋自太祖不欲勛臣廢騎射,雖上公,出必乘馬。永樂元年,駙馬都尉胡觀越制乘晉王濟熹朱尞棕轎,為給事中周景所劾。有詔宥觀而賜濟熹書,切責之。惟文職大臣乘轎,庶官亦乘馬。又文臣皆許乘車,大臣得乘安車。後久廢不用。正德四年,禮部侍郎劉機言,《大明集禮》,公卿大臣得乘安車,因請定轎扇傘蓋品級等差。帝以京城內安車傘蓋久不行,卻其請,而命轎扇俱如例行。嘉靖十五年,禮部尚書霍韜言:「禮儀定式,京官三品以上乘轎,邇者文官皆用肩輿,或乘女轎。乞申明禮制,俾臣下有所遵守。」乃定四品下不許乘轎,亦毋得用肩輿。隆慶二年,給事中徐尚劾應城伯孫文棟等乘轎出入,驕僭無狀。帝命奪文棟等俸。乃諭兩京武職非奉特恩不許乘轎,文官四品以下用帷轎者,禁如例。萬曆三年奏定勳戚及武臣不許用帷轎、肩輿并交床上馬。至若破格殊典,則宣德中少保黃淮陪遊西苑,嘗乘肩輿入禁中。嘉靖間,嚴嵩奉詔苑直,年及八旬,出入得乘肩輿。武臣則郭勛、朱希忠特命乘肩輿扈南巡蹕,後遂賜常乘焉。皆非制也。

傘蓋之制:洪武元年,令庶民不得用羅絹涼傘,但許用油紙雨傘。三年,令京城內一品二品用傘蓋,其餘用雨傘。十六年,令尚書、侍郎、左右都御史、通政使、太常卿、應天府尹、國子祭酒、翰林學士許張傘蓋。二十六年定一品、二品傘用銀浮屠頂,三品、四品用紅浮屠頂,俱用黑色茶褐羅表,紅絹裹,三簷;雨傘用紅油絹。五品紅浮屠頂,青羅表,紅絹裹,兩簷;雨傘同。四品、六品至九品,用紅浮屠頂,青絹表,紅絹裹,兩簷;雨傘俱用油紙。三十五年,官員傘蓋不許用金繡,朱丹裝飾。公、侯、駙馬、伯與一品、二品同。成化九年,令兩京官遇雨任用油傘,其涼傘不許張於京城。

鞍轡之制:洪武六年,令庶民不得描金,惟銅鐵裝飾。二十六年,定公、侯、一品、二品用銀AY,鐵事件,占用描銀。三品至五品,用銀AY,鐵事件,占用油畫。六品至九品,用擺錫,鐵事件,占用油畫。三十五年,官民人等馬頷下纓並鞦轡俱用黑色,不許紅纓及描金、嵌金、天青、硃紅裝飾。軍民用鐵事件,黑綠油占。

皇帝冕服后妃冠服皇太子親王以下冠服

皇帝冕服:洪武元年,學士陶安請製五冕。太祖曰:「此禮太繁。祭天地、宗廟,服袞冕。社稷等祀,服通天冠,絳紗袍。餘不用。」三年,更定正旦、冬至、聖節並服袞冕,祭社稷、先農、冊拜,亦如之。十六年,定袞冕之制。冕,前圓後方,玄表纁裏。前後各十二旒,旒五采,玉十二,珠五,采繅十有二就,就相去一寸。紅絲組為纓,黈纊充耳,玉簪導。袞,玄衣黃裳,十二章,日、月、星辰、山、龍、華蟲六章織於衣,宗彝、藻、火、粉米、黼、黻六章繡於裳。白羅大帶,紅裏。蔽膝隨裳色,繡龍、火、山文。玉革帶,玉佩。大綬六采,赤、黃、黑、白、縹、綠,小綬三,色同大綬。間施三玉環。白羅中單,黻領,青緣襈。黃襪黃舄,金飾。二十六年,更定袞冕十二章。冕版廣一尺二寸,長二尺四寸。冠上有覆,玄表朱裏,餘如舊制。圭長一尺二寸。袞,玄衣纁裳,十二章如舊制。中單以素紗為之。紅羅蔽膝,上廣一尺,下廣二尺,長三尺,織火、龍、山三章。革帶佩玉,長三尺三寸。大帶素表朱裏,兩邊用緣,上以朱錦,下以綠錦。大綬,六采黃、白、赤、玄、縹、綠織成,純玄質五百首。凡合單紡為一系,四系為一扶,五扶為一首。小綬三,色同大綬。間織三玉環。朱襪,赤舄。永樂三年定,冕冠以皁紗為之,上覆曰綖,桐板為質,衣之以綺,玄表朱裏,前圓後方。以玉衡維冠,玉簪貫紐,紐與冠武足前體下曰武,綏在冠之下,亦曰武。并繫纓處,皆飾以金。綖以左右垂黈纊充耳,用黃玉。繫以玄紞,承以白玉瑱朱紘。餘如舊制。玉圭長一尺二寸,剡其上,刻山四,以象四鎮之山,蓋周鎮圭之制,異於大圭不彖者也。以黃綺約其下,別以囊韜之,金龍文。袞服十有二章。玄衣八章,日、月、龍在肩,星辰、山在背,火、華蟲、宗彝在袖,每袖各三。皆織成本色領褾襈裾。褾者袖端。襈者衣緣。纁裳四章,織藻、粉米、黼、黻各二,前三幅,後四幅,前後不相屬,共腰,有辟積,本色綼裼。裳側有純謂之綼,裳下有純謂之裼,純者緣也。中單以素紗為之。青領褾襈裾,領織黻文十三。蔽膝隨裳色,四章,織藻、粉米、黼黻各二。本色緣,有紃,施於縫中。玉鉤二。玉佩二,各用玉珩一、瑀一、琚二、衝牙一、璜二;瑀下垂玉花一、玉滴二;彖飾雲龍文描金。自珩而下繫組五,貫以玉珠。行則衝牙、二滴與璜相觸有聲。金鉤二。有二小綬,六采黃、白、赤、玄、縹、綠纁質。大綬,六採黃、白、赤、玄、縹、綠纁質,三小綬,色同大綬。間施三玉環,龍文,皆織成。襪舄皆赤色,舄用黑絇純,以黃飾舄首。

嘉靖八年,諭閣臣張璁:「袞冕有革帶,今何不用?」璁對曰:「按陳祥道《禮書》,古革帶、大帶,皆謂之鞶。革帶以纛佩AX,然後加以大帶,而笏搢於二帶之間。夫革帶前繫AX,後繫綬,左右繫佩,自古冕弁恒用之。今惟不用革帶,以至前後佩服皆無所繫,遂附屬裳要之間,失古制矣。」帝曰:「冕服祀天地,享祖宗,若闕革帶,非齊明盛服之意。及觀《會典》載蔽膝用羅,上織火、山、龍三章,並大帶緣用錦,皆與今所服不合。卿可并革帶繫蔽膝、佩、綬之式,詳考繪圖以進。」又云:「衣裳分上下服,而今衣恒掩裳。裳制如帷,而今兩幅。朕意衣但當與裳要下齊,而露裳之六章,何如?」已,又諭璁以變更祖制為疑。璁對曰:「臣考禮制,衣不掩裳,與聖意允合。夫衣六章,裳六章,義各有取,衣自不容掩裳。《大明集禮》及《會典》與古制不異。今衣八章,裳四章,故衣常掩裳,然於典籍無所準。內閣所藏圖注,蓋因官司織造,循習訛謬。今訂正之,乃復祖制,非有變更。」帝意乃決。因復諭璁曰:「衣有六章,古以繪,今當以織。朕命織染局考國初冕服,日月各徑五寸,當從之。裳六章,古用繡,亦當從之。古色用玄黃,取象天地。今裳用纁,於義無取,當從古。革帶即束帶,後當用玉,以佩綬繫之於下。蔽膝隨裳色,其繡上龍下火,可不用山。卿與內閣諸臣同考之。」於是楊一清等詳議:「袞冕之服,自黃、虞以來,玄衣黃裳,為十二章。日、月、星辰、山、龍、華蟲,其序自上而下,為衣之六章;宗彝、藻、火、粉米、黼、黻,其序自下而上,為裳之六章。自周以後浸,變其制,或八章,或九章,已戾於古矣。我太祖皇帝復定為十二章之制,司造之官仍習舛訛,非制作之初意。伏乞聖斷不疑。」帝乃令擇吉更正其制。冠以圓匡烏紗冒之,旒綴七采玉珠十二,青纊充耳,綴玉珠二,餘如舊制。玄衣黃裳,衣裳各六章。洪武間舊制,日月徑五寸,裳前後連屬如帷,六章用繡。蔽膝隨裳色,羅為之,上繡龍一,下繡火三,繫於革帶,大帶素表朱裏,上緣以朱,下以綠。革帶前用玉,其後無玉,以佩綬繫而掩之。中單及圭,俱如永樂間制。朱襪,赤舄,黃條緣玄纓結。

皇帝通天冠服:洪武元年定,郊廟、省牲,皇太子諸王冠婚、醮戒,則服通天冠、絳紗袍。冠加金博山,附蟬十二,首施珠翠,黑介幘,組纓,玉簪導。絳紗袍,深衣製。白紗內單,皂領褾襈裾。絳紗蔽膝,白假帶,方心曲領。白襪,赤舄。其革帶、佩綬,與袞服同。

皇帝皮弁服:朔望視朝、降詔、降香、進表、四夷朝貢、外官朝覲、策士傳臚皆服之。嘉靖以後,祭太歲山川諸神亦服之。其制自洪武二十六年定。皮弁用烏紗冒之,前後各十二縫,每縫綴五采玉十二以為飾,玉簪導,紅組纓。其服絳紗衣,蔽膝隨衣色。白玉佩革帶。玉鉤苾,緋白大帶。白襪,黑舄。永樂三年定,皮弁如舊制,惟縫及冠武并貫簪繫纓處,皆飾以金玉。圭長如冕服之圭,有脊,并雙植文。絳紗袍,本色領褾襈裾。紅裳,但不織章數。中單,紅領褾衣巽裾。餘俱如冕服內制。

皇帝武弁服:明初親征遣將服之。嘉靖八年諭閣臣張璁云:「《會典》紀親征、類祃之祭,皆具武弁服。不可不備。」璁對:《周禮》有韋弁,謂以韎韋為弁,又以為衣裳。國朝視古損益,有皮弁之制。今武弁當如皮弁,但皮弁以黑紗冒之,武弁當以絳紗冒之。」隨具圖以進。帝報曰:「覽圖有韠形,但無繫處。冠制古象上尖,今皮弁則圓。朕惟上銳取其輕利,當如古制。又衣裳韠舄皆赤色,何謂?且佩綬俱無,於祭用之,可乎?」璁對:「自古服冕弁俱用革帶,以前系AX,後繫綬。韋弁之韠,正繫於革帶耳。武事尚威烈,故色純用赤。」帝復報璁:「冠服、衣裳、韠舄俱如古制,增革帶、佩綬及圭。」乃定制,弁上銳,色用赤,上十二縫,中綴五采玉,落落如星狀。韎衣、韎裳、韎韐,俱赤色。佩、綬、革帶,如常制。佩綬及韎韐,俱上繫於革帶。舄如裳色。玉圭視鎮圭差小,剡上方下,有篆文曰「討罪安民」。

皇帝常服:洪武三年定,烏紗折角向上巾,盤領窄袖袍,束帶間用金、琥珀、透犀。永樂三年更定,冠以烏紗冒之,折角向上,其後名翼善冠。袍黃,盤領,窄袖,前後及兩肩各織金盤龍一。帶用玉,靴以皮為之。先是,洪武二十四年,帝微行至神樂觀,見有結網巾者。翼日,命取網巾,頒示十三布政使司,人無貴賤,皆裹網巾,於是天子亦常服網巾。又《會典》載皇太孫冠禮有云:「掌冠跪加網巾」,而皇帝、皇太子冠服,俱闕而不載。

嘉靖七年,更定燕弁服。初,帝以燕居冠服,尚沿習俗,諭張璁考古帝王燕居法服之制。璁乃采《禮書》「玄端深衣」之文,圖注以進。帝為參定其制,諭璁詳議。璁言:「古者冕服之外,玄端深衣,其用最廣。玄端自天子達於士,國家之命服也。深衣自天子達於庶人,聖賢之法服也。今以玄端加文飾,不易舊制,深衣易黃色,不離中衣,誠得帝王損益時中之道。」帝因諭禮部曰:「古玄端上下通用,今非古人比,雖燕居,宜辨等威。」因酌古制,更名曰「燕弁」,寓深宮獨處、以燕安為戒之意。其制,冠匡如皮弁之制,冒以烏紗,分十有二瓣,各以金線壓之,前飾五采玉雲各一,後列四山,朱條為組纓,雙玉簪。服如古玄端之制,色玄,邊緣以青,兩肩繡日月,前盤圓龍一,後盤方龍二,邊加龍文八十一,領與兩祛共龍文五九。衽同前後齊,共龍文四九。襯用深衣之制,色黃。袂圓祛方,下齊負繩及踝十二幅。素帶,朱裏青表,綠緣邊,腰圍飾以玉龍九。玄履,硃緣紅纓黃結。白襪。

皇后冠服:洪武三年定,受冊、謁廟、朝會,服禮服。其冠圓匡,冒以翡翠,上飾九龍四鳳,大花十二樹,小花數如之。兩博鬢十二鈿。禕衣,深青繪翟,赤質,五色十二等。素紗中單,黻領,朱羅縠逯襈裾。蔽膝隨衣色,以緅為領緣,用翟為章三等。大帶隨衣色,朱裏紕其外,上以朱錦,下以綠錦,紐約用青組。玉革帶。青襪、青舄,以金飾。永樂三年定制,其冠飾翠龍九,金鳳四,中一龍銜大珠一,上有翠蓋,下垂珠結,餘皆口銜珠滴,珠翠雲四十片,大珠花、小珠花數如舊。三博鬢,飾以金龍、翠雲,皆垂珠滴。翠口圈一副,上飾珠寶鈿花十二,翠鈿如其數。托裏金口圈一副。珠翠面花五事。珠排環一對。皁羅額子一,描金龍文,用珠二十一。翟衣,深青,織翟文十有二等,間以小輪花。紅領褾襈裾,織金雲龍文。中單,玉色紗為之,紅領褾襈裾,織黻文十三。蔽膝隨衣色,織翟為章三等,間以小輪花四,以緅為領緣,織金雲龍文。玉穀圭,長七寸,剡其上,彖穀文,黃綺約其下,韜以黃囊,金龍文。玉革帶,青綺鞓,描金雲龍文,玉事件十,金事件四。大帶,表裏俱青紅相半,末純紅,下垂織金雲龍文,上朱緣,下綠緣,青綺副帶一。綬五采,黃、赤、白、縹、綠,纁質,間施二玉環,皆織成。小綬三,色同大綬。玉佩二,各用玉珩一、瑀一、琚二、衝牙一、璜二,瑀下垂玉花一、玉滴二;彖飾雲龍文描金;自珩而下,繫組五,貫以玉珠,行則衝牙二滴與二璜相觸有聲;上有金鉤,有小綬五采以副之,纁質,織成。青襪舄,飾以描金雲龍,皁純,每舄首加珠五顆。

皇后常服:洪武三年定,雙鳳翊龍冠,首飾、釧鐲用金玉、珠寶、翡翠。諸色團衫,金繡龍鳳文,帶用金玉。四年更定,龍鳳珠翠冠,真紅大袖衣霞帔,紅羅長裙,紅褙子。冠制如特髻,上加龍鳳飾,衣用織金龍鳳文,加繡飾。永樂三年更定,冠用皁縠,附以翠博山,上飾金龍一,翊以珠。翠鳳二,皆口銜珠滴。前後珠牡丹二,花八蕊,翠葉三十六。珠翠穰花鬢二,珠翠雲二十一,翠口圈一。金寶鈿花九,飾以珠。金鳳二,口銜珠結。三博鬢,飾以鸞鳳。金寶鈿二十四,邊垂珠滴。金簪二。珊瑚鳳冠觜一副。大衫霞帔,衫黃,霞帔深青,織金雲霞龍文,或繡或鋪翠圈金,飾以珠玉墜子,彖龍文。四衣癸襖子,即褙子。深青,金繡團龍文。鞠衣紅色,前後織金雲龍文,或繡或鋪翠圈金,飾以珠。大帶紅線羅為之,有緣,餘或青或綠,各隨鞠衣色。緣襈襖子,黃色,紅領褾衣巽裾,皆織金采色雲龍文。緣襈裙,紅色,綠緣襈,織金采色雲龍文。玉帶,如翟衣內制,第減金事件一。玉花采結綬,以紅綠線羅為結,玉綬花一,彖雲龍文。綬帶玉墜珠六,金垂頭花瓣四,小金葉六。紅線羅繫帶一。白玉雲樣玎榼二,如佩制,有金鉤,金如意雲蓋一,下懸紅組五貫,金方心雲板一,俱鈒雲龍文,襯以紅綺,下垂金長頭花四,中小金鐘一,末綴白玉雲朵五。青襪舄,與翟衣內制同。

皇妃、皇嬪及內命婦冠服:洪武三年定,皇妃受冊、助祭、朝會禮服。冠飾九翬、四鳳花釵九樹,小花數如之。兩博鬢九鈿。翟衣,青質繡翟,編次於衣及裳,重為九等。青紗中單,黻領,朱縠逯襈裾。蔽膝隨裳色,加文繡重雉,為章二等,以緅為領緣。大帶隨衣色。玉革帶。青韈舄、佩綬。常服:鸞鳳冠,首飾、釧鐲用金玉、珠寶、翠。諸色團衫,金繡鸞鳳,不用黃。帶用金、玉、犀。又定山松特髻,假鬢花鈿,或花釵鳳冠。真紅大袖衣,霞帔,紅羅裙,褙子,衣用織金及繡鳳文。永樂三年更定,禮服:九翟冠二,以皁縠為之,附以翠博山,飾大珠翟二,小珠翟三,翠翟四,皆口銜珠滴。冠中寶珠一座,翠頂雲一座,其珠牡丹、翠穰花鬢之屬,俱如雙鳳翊龍冠制,第減翠雲十。又翠牡丹花、穰花各二,面花四,梅花環四,珠環各二。其大衫、霞帔、燕居佩服之飾,俱同中宮,第織金繡彖,俱雲霞鳳文,不用雲龍文。

九嬪冠服:嘉靖十年始定,冠用九翟,次皇妃之鳳。大衫、鞠衣,如皇妃制。圭用次玉穀文。

內命婦冠服,洪武五年定,三品以上花釵、翟衣,四品、五品山松特髻,大衫為禮服。貴人視三品,以皇妃燕居冠及大衫、霞帔為禮服,以珠翠慶雲冠,鞠衣、褙子、緣襈襖裙為常服。

宮人冠服,制與宋同。紫色,團領,窄袖,遍刺折枝小葵花,以金圈之,珠絡縫金帶紅裙。弓樣鞋,上刺小金花。烏紗帽,飾以花,帽額綴團珠。結珠鬢梳。垂珠耳飾。

皇太子冠服:陪祀天地、社稷、宗廟及大朝會、受冊、納妃則服袞冕。洪武二十六年定,袞冕九章,冕九旒,旒九玉,金簪導,紅組纓,兩玉瑱。圭長九寸五分。玄衣纁裳,衣五章,織山、龍、華蟲、宗彞、火;裳四章,織藻、粉米、黼、黻。白紗中單,黻領。蔽膝隨裳色,織火、山二章。革帶,金鉤苾,玉佩。綬五采赤、白、玄、縹、綠織成,純赤質,三百三十首。小綬三,色同。間織三玉環。大帶,白表朱裏,上緣以紅,下緣以綠。白襪,赤舄。永樂三年定,冕冠,玄表朱裏,前圓後方,前後各九旒。每旒五采繅九就,貫五采玉九,赤、白、青、黃、黑相次。玉衡金簪,玄紞垂青纊充耳,用青玉。承以白玉瑱,朱紘纓。玉圭長九寸五分,以錦約其下,并韜。袞服九章,玄衣五章,龍在肩,山在背,火、華蟲、宗彞在袖,每袖各三。皆織成。本色領褾襈裾。纁裳四章,織藻、粉米、黼、黻各二,前三幅,後四幅,不相屬,共腰,有襞積,本色綼裼。中單以素紗為之,青領褾襈裾,領織黻文十一。蔽膝隨裳色,四章,織藻、粉米、黼、黻。本色緣,有紃,施於縫中。上玉鉤二。玉佩二,各用玉珩一、瑀一、琚一、衝牙一、璜二;瑀下垂玉花一、玉滴二。彖雲龍文,描金。自珩而下,繫組五,貫以玉珠。上有金鉤。小綬四采赤、白、縹、綠以副之,纁質。大帶,素表朱裏,在腰及垂,皆有綼,上綼以朱,下綼以綠。紐約用青組。大綬四采,赤、白、縹、綠。纁質。小綬三采。間施二玉環,龍文,皆織成。襪舄皆赤色,舄用黑絇純,黑飾舄首。朔望朝、降詔、降香、進表、外國朝貢、朝覲,則服皮弁。永樂三年定,皮弁,冒以烏紗,前後各九縫,每縫綴五采玉九,縫及冠武并貫簪繫纓處,皆飾以金。金簪朱纓。玉圭,如冕服內制。絳紗袍,本色領褾襈裾。紅裳,如冕服內裳制,但不織章數。中單以素紗為之,如深衣制。紅領褾襈裾,領織黻文十一。蔽膝隨裳色,本色緣,有紃,施於縫中;其上玉鉤二,玉佩如冕服內制,但無雲龍文;有小綬四采以副之。大帶、大綬、韈舄赤色,皆如冕服內制。其常服,洪武元年定,烏紗折上巾。永樂三年定,冠烏紗折角向上巾,亦名翼善冠,親王、郡王及世子俱同。袍赤,盤領窄袖,前後及兩肩各金織盤龍一。玉帶、靴,以皮為之。

皇太子妃冠服:洪武三年定,禮服與皇妃同。永樂三年更定,九翬四鳳冠,漆竹絲為匡,冒以翡翠,上飾翠翬九、金鳳四,皆口銜珠滴。珠翠雲四十片,大珠花九樹,小珠花數如之。雙博鬢,飾以鸞鳳,皆垂珠滴。翠口圈一副,上飾珠寶鈿花九,翠鈿如其數。托裏金口圈一副。珠翠面花五事。珠排環一對。珠皁羅額子一,描金鳳文,用珠二十一。翟衣,青質,織翟文九等,間以小輪花。紅領褾襈裾,織金雲龍文。中單玉色紗為之。紅領褾襈裾,領織黻文十一。蔽膝隨衣色,織翟為章二等,間以小輪花三,以緅為領緣,織金雲鳳文。其玉圭、帶綬、玉佩、襪舄之制,俱同皇妃。洪武三年又定常服。犀冠,刻以花鳳。首飾、釧鐲、衫帶俱同皇妃。四年定,冠亦與皇妃同。永樂三年定燕居冠,以皁縠為之,附以翠博山,上飾寶珠一座,翊以二珠翠鳳,皆口銜珠滴。前後珠牡丹二,花八蕊,翠葉三十六。珠翠穰花鬢二。珠翠雲十六片。翠口圈一副。金寶鈿花九,上飾珠九。金鳳一對,口銜珠結。雙博鬢,飾以鸞鳳。金寶鈿十八,邊垂珠滴。金簪一對。珊瑚鳳冠觜一副。其大衫、霞帔、燕居佩服之飾,俱同皇妃。

親王冠服:助祭、謁廟、朝賀、受冊、納妃服袞冕,朔望朝、降詔、降香、進表、四夷朝貢、朝覲服皮弁。洪武二十六年定,冕服俱如東宮,第冕旒用五采,玉圭長九寸二分五釐,青衣纁裳。永樂三年又定冕服、皮弁制,俱與東宮同,其常服亦與東宮同。

嘉靖七年,諭禮部:「朕仿古玄端,自為燕弁冠服,更制忠靜冠服,錫於有位,而宗室諸王制猶未備。今酌燕弁及忠靜冠之制,復為式具圖,命曰保和冠服。自郡王長子以上,其式已明。鎮國將軍以下至奉國中尉及長史、審理、紀善、教授、伴讀,俱用忠靜冠服,依其品服之。儀賓及餘官不許概服。夫忠靜冠服之異式,尊賢之等也。保和冠服之異式,親親之殺也。等殺既明,庶幾乎禮之所保,保斯和,和斯安,此錫名之義也。其以圖說頒示諸王府,如敕遵行。」保和冠制,以燕弁為準,用九衣取,去簪與五玉,後山一扇,分畫為四。服,青質青緣,前後方龍補,身用素地,邊用雲。襯用深衣,玉色。帶青表綠裏綠緣。履用皁綠結,白襪。

親王妃冠服:受冊、助祭、朝會服禮服。洪武三年定九翬四鳳冠。永樂三年又定九翟冠,制同皇妃。其大衫、霞帔、燕居佩服之飾,同東宮妃,第金事件減一,玉綬花,彖寶相花文。

公主冠服,與親王妃同,惟不用圭。

親王世子冠服:聖節、千秋節并正旦、冬至、進賀表箋及父王生日諸節慶賀,皆服袞冕。洪武二十六年定,袞冕七章,冕三采玉珠,七旒。圭長九寸。青衣三章,織華蟲、火、宗彝。纁裳四章,織藻、粉米、黼、黻。素紗中單,青領襈,赤AX。革帶,佩白玉,玄組綬。綬紫質,用三采紫、黃、赤織成,間織三白玉環。白襪,赤舄。永樂三年更定,冕冠前後各八旒,每旒五采繅八就,貫三采玉珠八,赤、白、青色相次。玉圭長九寸。青衣三章,火在肩,華蟲、宗彞在兩袖,皆織成。本色領褾襪裾。其纁裳、玉佩、帶、綬之制,俱與親王同,第領織黻文減二。皮弁用烏紗冒之,前後各八縫,每縫綴三采玉八,餘制如親王。其圭佩、帶綬、韈舄如冕服內制。常服亦與親王同。嘉靖七年定保和冠服,以燕弁為準,用八衣取,去簪玉,後山以一扇分畫為四,服與親王同。

世子妃冠服:永樂三年定,與親王妃同,惟冠用七翟。

郡王冠服:永樂三年定,冕冠前後各七旒,每旒五采繅七就,貫三采玉珠七。圭長九寸。青衣三章,粉米在肩,藻、宗彝在兩袖,皆織成。纁裳二章,織黼、黻各二。中單,領織黻文七,餘與親王世子同。皮弁,前後各七縫,每縫綴三採玉七,餘與親王世子同。其圭佩、帶綬、襪舄如冕服內制。常服亦與親王世子同。嘉靖七年定保和冠服,冠用七衣取,服與親王世子同。

郡王妃冠服:永樂三年定,冠用七翟,與親王世子妃同。其大衫、霞帔、燕居佩服之飾,俱同親王妃,第繡雲霞翟文,不用盤鳳文。

郡王長子朝服:七梁冠,大紅素羅衣,白素紗中單,大紅素羅裳及蔽膝,大紅素羅白素紗二色大帶,玉朝帶,丹礬紅花錦,錦雞綬,玉佩,象笏,白絹襪,皁皮雲頭履鞋。公服:皁縐紗襆頭,大紅素紵絲衣,玉革帶。常服:烏紗帽,大紅紵絲織金獅子開衣癸,圓領,玉束帶,皁皮銅線靴。其保和冠,如忠靜之制,用五衣取;服與郡王同,補子用織金方龍。

郡主冠服:永樂三年定,與郡王妃同。惟不用圭,減四珠環一對。

郡王長子夫人冠服:珠翠五翟冠,大紅紵絲大衫,深青紵絲金繡翟褙子,青羅金繡翟霞帔,金墜頭。

鎮國將軍冠服,與郡王長子同。鎮國將軍夫人冠服,與郡王長子夫人同。輔國將軍冠服,與鎮國將軍同,惟冠六梁,帶用犀。輔國將軍夫人冠服,與鎮國將軍夫人同,惟冠用四翟,抹金銀墜頭。奉國將軍冠服,與輔國將軍同,惟冠五梁,帶用金鈒花,常服大紅織金虎豹。奉國將軍淑人冠服,與輔國將軍夫人同,惟褙子、霞帔,金繡孔雀文。鎮國中尉冠服,與奉國將軍同,惟冠四梁,帶用素金,佩用藥玉。鎮國中尉恭人冠服,與奉國將軍淑人同。輔國中尉冠服,與鎮國中尉同,惟冠三梁,帶用銀鈒花,綬用盤雕,公服用深青素羅,常服紅織金熊羆。輔國中尉宜人冠服,與鎮國中尉恭人同,惟冠用三翟,褙子、霞帔,金繡鴛鴦文,銀墜頭。奉國中尉冠服,與輔國中尉同,惟冠二梁,帶用素銀,綬用練鵲,襆頭黑漆,常服紅織金彪。奉國中尉安人冠服,與輔國中尉宜人同,惟大衫用丹礬紅,褙子、霞帔金繡練鵲文。

縣主冠服:珠翠五翟冠,大紅紵絲大衫,深青紵絲金繡孔雀褙子,青羅金繡孔雀霞帔,抹金銀墜頭。郡君冠服,與縣主同,惟冠用四翟,褙子、霞帔金繡鴛鴦文。縣君冠服,與郡君同,惟冠用三翟。鄉君冠服,與縣君同,惟大衫用丹礬紅,褙子、霞帔金繡練鵲文。

文武官冠服命婦冠服內外官親屬冠服內使冠服侍儀以下冠服士庶冠服

樂工冠服軍隸冠服外蕃冠服僧道服色

群臣冠服:洪武元年命制公服、朝服,以賜百官。時禮部言:「各官先授散官,與見任職事高下不同。如御史董希哲前授朝列大夫澧州知州,而任七品職事;省司郎中宋冕前授亞中大夫黃州知府,而任五品職事。散官與見任之職不同,故服色不能無異,乞定其制。」乃詔省部臣定議。禮部復言:「唐制,服色皆以散官為準。元制,散官職事各從其高者,服色因之。國初服色依散官,與唐制同。」乃定服色準散官,不計見職,於是所賜袍帶亦並如之。三年,禮部言:「歷代異尚。夏黑,商白,周赤,秦黑,漢赤,唐服飾黃,旂幟赤。今國家承元之後,取法周、漢、唐、宋,服色所尚,於赤為宜。」從之。

文武官朝服:洪武二十六年定凡大祀、慶成、正旦、冬至、聖節及頒詔、開讀、進表、傳制,俱用梁冠,赤羅衣,白紗中單,青飾領緣,赤羅裳,青緣,赤羅蔽膝,大帶赤、白二色絹,革帶,佩綬,白襪黑履。一品至九品,以冠上梁數為差。公冠八梁,加籠巾貂蟬,立筆五折,四柱,香草五段,前後玉蟬。侯七梁,籠巾貂蟬,立筆四折,四柱,香草四段,前後金蟬。伯七梁,籠巾貂蟬,立筆二折,四柱,香草二段,前後玳瑁蟬。俱插雉尾。駙馬與侯同,不用雉尾。一品,冠七梁,不用籠巾貂蟬,革帶與佩俱玉,綬用黃、綠、赤、紫織成雲鳳四色花錦,下結青絲網,玉綬環二。二品,六梁,革帶,綬環犀,餘同一品。三品,五梁,革帶金,佩玉,綬用黃、綠、赤、紫織成雲鶴花錦,下結青絲網,金綬環二。四品,四梁,革帶金,佩藥玉,餘同三品。五品,三梁,革帶銀,鈒花,佩藥玉,綬用黃、綠、赤、紫織成盤雕花錦,下結青絲網,銀鍍金綬環二。一品至五品,笏俱象牙。六品、七品,二梁,革帶銀,佩藥玉,綬用黃、綠、赤織成練鵲三色花錦,下結青絲網,銀綬環二。獨御史服獬廌。八品、九品,一梁,革帶烏角,佩藥玉,綬用黃、綠織成鸂水鶒二色花錦,下結青絲網,銅綬環二。六品至九品,笏俱槐木。其武官應直守衛者,別有服色。雜職未入流品者,大朝賀、進表行禮止用公服。三十年令視九品官,用朝服。嘉靖八年,更定朝服之制。梁冠如舊式,上衣赤羅青緣,長過腰指七寸,毋掩下裳。中單白紗青緣。下裳七幅,前三後四,每幅三襞積,赤羅青緣。蔽膝綴革帶。綬,各從品級花樣。革帶之後佩綬,繫而掩之。其環亦各從品級,用玉犀金銀銅,不以織於綬。大帶表裏俱素,惟兩耳及下垂緣綠,又以青組約之。革帶俱如舊式。珮玉一如《詩傳》之制,去雙滴及二珩。其三品以上玉,四品以下藥玉,及襪履俱如舊式。萬曆五年,令百官正旦朝賀毋僭躡朱履。故事,十一月百官戴煖耳。是年朝覲外官及舉人、監生,不許戴煖耳入朝。

凡親祀郊廟、社稷,文武官分獻陪祀,則服祭服。洪武二十六年定,一品至九品,青羅衣,白紗中單,俱皁領緣。赤羅裳,皁緣。赤羅蔽膝。方心曲領。其冠帶、佩綬等差,並同朝服。又定品官家用祭服。三品以上,去方心曲領。四品以下,并去佩綬。嘉靖八年,更定百官祭服。上衣青羅,皁緣,與朝服同。下裳赤羅,皁緣,與朝服同。蔽膝、綬環、大帶、革帶、佩玉、襪履俱與朝服同。其視牲、朝日夕月、耕耤、祭歷代帝王,獨錦衣衛堂上官,大紅蟒衣,飛魚,烏紗帽,鸞帶,佩繡春刀。祭太廟、社稷,則大紅便服。

文武官公服:洪武二十六年定,每日早晚朝奏事及侍班、謝恩、見辭則服之。在外文武官,每日公座服之。其制,盤領右衽袍,用紵絲或紗羅絹,袖寬三尺。一品至四品,緋袍;五品至七品,青袍;八品九品,綠袍;未入流雜職官,袍、笏、帶與八品以下同。公服花樣,一品,大獨科花,徑五寸;二品,小獨科花,徑三寸;三品,散答花,無枝葉,徑二寸;四品、五品,小雜花紋,徑一寸五分;六品、七品,小雜花,徑一寸;八品以下無紋。襆頭:漆、紗二等,展角長一尺二寸;雜職官襆頭,垂帶,後復令展角,不用垂帶,與入流官同。笏依朝服為之。腰帶:一品玉,或花或素;二品犀;三品、四品,金荔枝;五品以下烏角。襪用青革,仍垂撻尾於下。靴用皁。其後,常朝止便服,惟朔望具公服朝參。凡武官應直守衛者,別有服色,不拘此制。公、侯、駙馬、伯服色花樣、腰帶與一品同。文武官花樣,如無從織造,則用素。百官入朝,雨雪許服雨衣。奉天、華蓋、武英諸殿奏事,必躡履鞋,違者御史糾之。萬曆五年,令常朝俱衣本等錦繡服色,其朝覲官見辭、謝恩,不論已未入流,公服行禮。

文武官常服:洪武三年定,凡常朝視事,以烏紗帽、團領衫、束帶為公服。其帶,一品玉,二品花犀,三品金鈒花,四品素金,五品銀鈒花,六品、七品素銀,八品、九品烏角。凡致仕及侍親辭閑官,紗帽、束帶。為事黜降者,服與庶人同。至二十四年,又定公、侯、伯、駙馬束帶與一品同,雜職官與八品、九品同。朝官常服禮鞋,洪武六年定。先是,百官入朝,遇雨皆躡釘靴,聲徹殿陛,侍儀司請禁之。太祖曰:「古者入朝有履,自唐始用靴。其令朝官為軟底皮鞋,冒於靴外,出朝則釋之。」

禮部言近奢侈越制。詔申禁之,仍參酌漢、唐之制,頒行遵守。凡職官,一品、二品用雜色文綺、綾羅、彩繡,帽頂、帽珠用玉;三品至五品用雜色文綺、綾羅,帽頂用金,帽珠除玉外,隨所用;六品至九品用雜色文綺、綾羅,帽頂用銀,帽珠瑪瑙、水晶、香木。一品至六品穿四爪龍,以金繡為之者聽。禮部又議:「品官見尊長,用朝君公服,於理未安。宜別製梁冠、絳衣、絳裳、革帶、大帶、大白襪、烏舄、佩綬,其衣裳去緣襈。三品以上佩綬,三品以下不用。」從之。

二十二年,令文武官遇雨戴雨帽,公差出外戴帽子,入城不許。二十三年定制,文官衣自領至裔,去地一寸,袖長過手,復回至肘。公、侯、駙馬與文官同。武官去地五寸,袖長過手七寸。二十四年定,公、侯、駙馬、伯服,繡麒麟、白澤。文官一品仙鶴,二品錦雞,三品孔雀,四品雲雁,五品白鷴,六品鷺鷥,七品鸂水鶒,八品黃鸝,九品鵪鶉;雜職練鵲;風憲官獬廌。武官一品、二品獅子,三品、四品虎豹,五品熊羆,六品、七品彪,八品犀牛,九品海馬。又令品官常服用雜色紵絲、綾羅、綵繡。官吏衣服、帳幔,不許用玄、黃、紫三色,并織繡龍鳳文,違者罪及染造之人。朝見人員,四時並用色衣,不許純素。三十年,令致仕官服色與見任同,若朝賀、謝恩、見辭,一體具服。

景泰四年,令錦衣衛指揮侍衛者,得衣麒麟。天順二年,定官民衣服不得用蟒龍、飛魚、斗牛、大鵬、像生獅子、四寶相花、大西番蓮、大雲花樣,并玄、黃、紫及玄色、黑、綠、柳黃、姜黃、明黃諸色。弘治十三年奏定,公、侯、伯、文武大臣及鎮守、守備,違例奏請蟒衣、飛魚衣服者,科道糾劾,治以重罪。正德十一年設東、西兩官廳,將士悉衣黃罩甲。中外化之。金緋盛服者,亦必加此於上。都督江彬等承日紅笠之上,綴以靛染天鵝翎,以為貴飾,貴者飄三英,次者二英。兵部尚書王瓊得賜一英,冠以下教場,自謂殊遇。其後巡狩所經,督餉侍郎、巡撫都御史無不衣罩甲見上者。十三年,車駕還京,傳旨,俾迎候者用曳撒大帽、鸞帶。尋賜群臣大紅紵絲羅紗各一。其服色,一品斗牛,二品飛魚,三品蟒,四、五品麒麟,六、七品虎、彪;翰林科道不限品級皆與焉;惟部曹五品下不與。時文臣服色亦以走獸,而麒麟之服逮於四品,尤異事也。

十六年,世宗登極詔云:「近來冒濫玉帶,蟒龍、飛魚、斗牛服色,皆庶官雜流并各處將領夤緣奏乞,今俱不許。武職卑官僭用公、侯服色者,亦禁絕之。」嘉靖六年復禁中外官,不許濫服五彩裝花織造違禁顏色。

七年既定燕居法服之制,閣臣張璁因言:「品官燕居之服未有明制,詭異之徒,競為奇服以亂典章。乞更法古玄端,別為簡易之制,昭布天下,使貴賤有等。」帝因復製《忠靜冠服圖》頒禮部,敕諭之曰:「祖宗稽古定制,品官朝祭之服,各有等差。第常人之情,多謹於明顯,怠於幽獨。古聖王慎之,制玄端以為燕居之服。比來衣服詭異,上下無辨,民志何由定。朕因酌古玄端之制,更名『忠靜』,庶幾乎進思盡忠,退思補過焉。朕已著為圖說,如式製造。在京許七品以上官及八品以上翰林院、國子監、行人司,在外許方面官及各府堂官、州縣正堂、儒學教官服之。武官止都督以上。其餘不許濫服。」禮部以圖說頒布天下,如敕奉行。按忠靜冠仿古玄冠,冠匡如制,以烏紗冒之,兩山俱列於後。冠頂仍方中微起,三梁各壓以金線,邊以金緣之。四品以下,去金,緣以淺色絲線。忠靜服仿古玄端服,色用深青,以紵絲紗羅為之。三品以上雲,四品以下素,緣以藍青,前後飾本等花樣補子。深衣用玉色。素帶,如古大夫之帶制,青表綠緣邊并裏。素履,青綠絳結。白襪。

十六年,群臣朝於駐蹕所,兵部尚書張瓚服蟒。帝怒,諭閣臣夏言曰:「尚書二品,何自服蟒?」言對曰:「瓚所服,乃欽賜飛魚服,鮮明類蟒耳。」帝曰:「飛魚何組兩角?其嚴禁之。」於是禮部奏定,文武官不許擅用蟒衣、飛魚、斗牛、違禁華異服色。其大紅紵絲紗羅服,惟四品以上官及在京五品堂上官、經筵講官許服。五品官及經筵不為講官者,俱服青綠錦繡。遇吉禮,止衣紅布絨褐。品官花樣,並依品級。錦衣衛指揮,侍衛者仍得衣麒麟,其帶俸非侍衛,及千百戶雖侍衛,不許僭用。

歷朝賜服:文臣有未至一品而賜玉帶者,自洪武中學士羅復仁始。衍聖公秩正二品,服織金麒麟袍、玉帶,則景泰中入朝拜賜。自是以為常。內閣賜蟒衣,自弘治中劉健、李東陽始。麒麟本公、侯服,而內閣服之,則嘉靖中嚴嵩、徐階皆受賜也。仙鶴,文臣一品服也,嘉靖中成國公朱希忠、都督陸炳服之,皆以玄壇供事。而學士嚴訥、李春芳、董份以五品撰青詞,亦賜仙鶴。尋諭供事壇中乃用,於是尚書皆不敢衣鶴。後敕南京織閃黃補麒麟、仙鶴,賜嚴嵩,閃黃乃上用服色也;又賜徐階教子升天蟒。萬曆中,賜張居正坐蟒;武清侯李偉以太后父,亦受賜。

儀賓朝服、公服、常服:俱視品級,與文武官同,惟笏皆象牙;常服花樣視武官。弘治十三年定,郡主儀賓鈒花金帶,胸背獅子。縣主儀賓鈒花金帶,郡君儀賓光素金帶,胸背俱虎豹。縣君儀賓鈒花銀帶,鄉君儀賓光素銀帶,胸背俱彪。有僭用者,革去冠帶,戴平頭巾,於儒學讀書習禮三年。

狀元及諸進士冠服:狀元冠二梁,緋羅圓領,白絹中單,錦綬,蔽膝,紗帽,槐木笏,光銀帶,藥玉佩,朝靴,氈襪,皆御前頒賜,上表謝恩日服之。進士巾如烏紗帽,頂微平,展角闊寸餘,長五寸許,系以垂帶,皁紗為之。深藍羅袍,緣以青羅,袖廣而不殺。槐木笏,革帶、青鞓,飾以黑角,垂撻尾於後。廷試後頒於國子監,傳臚日服之。上表謝恩後,謁先師行釋菜禮畢,始易常服,其巾袍仍送國子監藏之。

命婦冠服:洪武元年定,命婦一品,冠花釵九樹。兩博鬢,九鈿。服用翟衣,繡翟九重。素紗中單,黼領,朱縠逯襈裾。蔽膝隨裳色,以緅為領緣,加文繡重翟,為章二等。玉帶。青襪舄,佩綬。二品,冠花釵八樹。兩博鬢,八鈿。服用翟衣八等,犀帶,餘如一品。三品,冠花釵七樹。兩博鬢,七鈿。翟衣七等,金革帶,餘如二品。四品,冠花釵六樹。兩博鬢,六鈿。翟衣六等,金革帶,餘如三品。五品,冠花釵五樹。兩博鬢,五鈿。翟衣五等,烏角帶,餘如四品。六品,冠花釵四樹。兩博鬢,四鈿。翟衣四等,烏角帶,餘如五品。七品,冠花釵三樹。兩博鬢,三鈿。翟衣三等,烏角帶,餘如六品。自一品至五品,衣色隨夫用紫。六品、七品,衣色隨夫用緋。其大帶如衣色。四年,以古天子諸侯服袞冕,后與夫人亦服禕翟。今群臣既以梁冠、絳衣為朝服,不敢用冕,則外命婦亦不當服翟衣以朝。命禮部議之。奏定,命婦以山松特髻、假鬢花鈿、真紅大袖衣、珠翠蹙金霞帔為朝服。以朱翠角冠、金珠花釵、闊袖雜色綠緣為燕居之用。一品,衣金繡文霞帔,金珠翠妝飾,玉墜。二品,衣金繡雲肩大雜花霞帔,金珠翠妝飾,金墜子。三品,衣金繡大雜花霞帔,珠翠妝飾,金墜子。四品,衣繡小雜花霞帔,翠妝飾,金墜子。五品,衣銷金大雜花霞帔,生色畫絹起花妝飾,金墜子。六品、七品,衣銷金小雜花霞帔,生色畫絹起花妝飾,鍍金銀墜子。八品、九品,衣大紅素羅霞帔,生色畫絹妝飾,銀墜子。首飾,一品、二品,金玉珠翠。三品、四品,金珠翠。五品,金翠。六品以下,金鍍銀,間用珠。

五年,更定品官命婦冠服:一品,禮服用山松特髻,翠松五株,金翟八,口銜珠結。正面珠翠翟一,珠翠花四朵,珠翠雲喜花三朵;後鬢珠梭球一,珠翠飛翟一,珠翠梳四,金雲頭連三釵一,珠簾梳一,金簪二;珠梭環一雙。大袖衫,用真紅色。霞帔、褙子,俱用深青色。紵絲綾羅紗隨用。霞帔上施蹙金繡雲霞翟文,鈒花金墜子。褙子上施金繡雲霞翟文。常服用珠翠慶雲冠,珠翠翟三,金翟一,口銜珠結;鬢邊珠翠花二,小珠翠梳一雙,金雲頭連三釵一,金壓鬢雙頭釵二,金腦梳一,金簪二;金腳珠翠佛面環一雙;鐲釧皆用金。長襖長裙,各色紵絲綾羅紗隨用。長襖緣襈,或紫或綠,上施蹙金繡雲霞翟文。看帶,用紅綠紫,上施蹙金繡雲霞翟文。長裙,橫豎金繡纏枝花文。二品,特髻上金翟七,口銜珠結,餘同一品。常服亦與一品同。三品,特髻上金孔雀六,口銜珠結。正面珠翠孔雀一,後鬢翠孔雀二。霞帔上施蹙金雲霞孔雀文,鈒花金墜子。褙子上施金繡雲霞孔雀文,餘同二品。常服冠上珠翠孔雀三,金孔雀二,口銜珠結。長襖緣衣巽。看帶,或紫或綠,並繡雲霞孔雀文。長裙,橫豎襴並繡纏枝花文,餘同二品。四品,特髻上金孔雀五,口銜珠結,餘同三品。常服亦與三品同。五品,特髻上銀鍍金鴛鴦四,口銜珠結。正面珠翠鴛鴦一,小珠鋪翠雲喜花三朵;後鬢翠鴛鴦二,銀鍍金雲頭連三釵一,小珠簾梳一,鍍金銀簪二;小珠梳環一雙。霞帔上施繡雲霞鴛鴦文,鍍金銀鈒花墜子。褙子上施雲霞鴛鴦文,餘同四品。常服冠上小珠翠鴛鴦三,鍍金銀鴛鴦二,挑珠牌。鬢邊小珠翠花二朵,雲頭連三釵一,梳一,壓鬢雙頭釵二,鍍金簪二;銀腳珠翠佛面環一雙。鐲釧皆用銀鍍金。長襖緣襈,繡雲霞鴛鴦文。長裙,橫豎襴繡纏枝花文,餘同四品。六品,特髻上翠松三株,銀鍍金練鵲四,口銜珠結。正面銀鍍金練鵲一,小珠翠花四朵;後鬢翠梭球一,翠練鵲二,翠梳四,銀雲頭連三釵一,珠緣翠簾梳一,銀簪二。大袖衫,綾羅紬絹隨所用。霞帔施繡雲霞練鵲文,花銀墜子。褙子上施雲霞練鵲文,餘同五品。常服冠上鍍金銀練鵲三,又鍍金銀練鵲二,挑小珠牌;鐲釧皆用銀。長襖緣襈。看帶,或紫或綠,繡雲霞練鵲文。長裙,橫豎襴繡纏枝花文,餘同五品。七品,禮服、常服俱同六品。其八品、九品禮服,惟用大袖衫、霞帔、褙子。大衫同七品。霞帔上繡纏枝花,鈒花銀墜子。褙子上繡摘枝團花。通用小珠慶雲冠。常服亦用小珠慶雲冠,銀間鍍金銀練鵲三,又銀間鍍金銀練鵲二,挑小珠牌;銀間鍍金雲頭連三釵一,銀間鍍金壓鬢雙頭釵二,銀間鍍金腦梳一,銀間鍍金簪二。長襖緣襈、看帶並繡纏枝花,餘同七品。又定命婦團衫之制,以紅羅為之,繡重雉為等第。一品九等,二品八等,三品七等,四品六等,五品五等,六品四等,七品三等,其餘不用繡雉。

二十四年定制,命婦朝見君后,在家見舅姑并夫及祭祀則服禮服。公侯伯夫人與一品同。大袖衫,真紅色。一品至五品,紵絲綾羅;六品至九品,綾羅紬絹。霞帔、褙子皆深青段。公侯及一品、二品,金繡雲霞翟文;三品、四品,金繡雲霞孔雀文;五品,繡雲霞鴛鴦文;六品、七品,繡雲霞練鵲文。大袖衫,領闊三寸,兩領直下一尺,間綴紐子三,末綴紐子二,紐在掩紐之下,拜則放之。霞帔二條,各繡禽七,隨品級用,前四後三。墜子中鈒花禽一,四面雲霞文,禽如霞帔,隨品級用。笏以象牙為之。二十六年定,一品,冠用金事件,珠翟五,珠牡丹開頭二,珠半開三,翠雲二十四片,翠牡丹葉一十八片,翠口圈一副,上帶金寶鈿花八,金翟二,口銜珠結二。二品至四品,冠用金事件,珠翟四,珠牡丹開頭二,珠半開四,翠雲二十四片,翠牡丹葉一十八片,翠口圈一副,上帶金寶鈿花八,金翟二,口銜珠結二。一品、二品,霞帔、褙子俱雲霞翟文,鈒花金墜子。三品、四品,霞帔、褙子俱雲霞孔雀文,鈒花金墜子。五品、六品,冠用抹金銀事件,珠翟三,珠牡丹開頭二,珠半開五,翠雲二十四片,翠牡丹葉一十八片,翠口圈一副,上帶抹金銀寶鈿花八,抹金銀翟二,口銜珠結子二。五品,霞帔、褙子俱雲霞鴛鴦文,鍍金鈒花銀墜子。六品,霞帔、褙子俱雲霞練鵲文,鈒花銀墜子。七品至九品,冠用抹金銀事件,珠翟二,珠月桂開頭二,珠半開六,翠雲二十四片,翠月桂葉一十八片,翠口圈一副,上帶抹金銀寶鈿花八,抹金銀翟二,口銜珠結子二。七品,霞帔、墜子、褙子與六品同。八品、九品,霞帔用繡纏枝花,墜子與七品同,褙子繡摘枝團花。

內外官親屬冠服:洪武元年,禮部尚書崔亮奉詔議定。內外官父、兄、伯、叔、子、孫、弟、侄用烏紗帽,軟腳垂帶,圓領衣,烏角帶。品官祖母及母、與子孫同居親弟侄婦女禮服,合以本官所居官職品級,通用漆紗珠翠慶雲冠,本品衫,霞帔、褙子,緣襈襖裙,惟山松特髻子止許受封誥敕者用之。品官次妻,許用本品珠翠慶雲冠、褙子為禮服。銷金闊領、長襖長裙為常服。二十五年,令文武官父兄、伯叔、弟侄、子婿,皆許穿靴。

內使冠服:明初置內使監,冠烏紗描金曲腳帽,衣胸背花盤領窄袖衫,烏角帶,靴用紅扇面黑下樁。各宮火者,服與庶人同。洪武三年諭宰臣,內使監未有職名者,當別製冠,以別監官。禮部奏定,內使監凡遇朝會,依品具朝服、公服行禮。其常服,葵花胸背團領衫,不拘顏色;烏紗帽;犀角帶。無品從者,常服團領衫,無胸背花,不拘顏色;烏角帶;烏紗帽,垂軟帶。年十五以下者,惟戴烏紗小頂帽。按《大政記》,永樂以後,宦官在帝左右,必蟒服,製如曳撒,繡蟒於左右,系以鸞帶,此燕閑之服也。次則飛魚,惟入侍用之。貴而用事者,賜蟒,文武一品官所不易得也。單蟒面皆斜向,坐蟒則面正向,尤貴。又有膝衣闌者,亦如曳撒,上有蟒補,當膝處橫織細雲蟒,蓋南郊及山陵扈從,便於乘馬也。或召對燕見,君臣皆不用袍而用此;第蟒有五爪、四爪之分,襴有紅、黃之別耳。弘治元年,都御史邊鏞言:「國朝品官無蟒衣之制。夫蟒無角、無足,今內官多乞蟒衣,殊類龍形,非制也。」乃下詔禁之。十七年,諭閣臣劉健曰:「內臣僭妄尤多。」因言服色所宜禁,曰:「蟒、龍、飛魚、斗牛,本在所禁,不合私織。間有賜者,或久而敝,不宜輒自織用。玄、黃、紫、皁乃屬正禁,即柳黃、明黃、姜黃諸色,亦應禁之。」孝宗加意鉗束,故申飭者再,然內官驕姿已久,積習相沿,不能止也。初,太祖制內臣服,其紗帽與群臣異,且無朝冠、襆頭,亦無祭服。萬曆初,穆宗主入太廟,大榼冠進賢,服祭服以從,蓋內府祀中霤、灶井之神,例遣中官,因自創為祭服,非由廷議也。

侍儀舍人冠服:洪武二年,禮官議定。侍儀舍人導禮,依元制,展腳襆頭,窄袖紫衫,塗金束帶,皁紋靴。常服:烏紗唐帽,諸色盤領衫,烏角束帶,衫不用黃。四年,中書省議定,侍儀舍人並御史臺知班,引禮執事,冠進賢冠,無梁,服絳色衣,其蔽膝、履、襪、帶、笏,與九品同,惟不用中單。

校尉冠服:洪武三年定制,執仗之士,首服皆縷金額交腳襆頭,其服有諸色辟邪、寶相花裙襖,銅葵花束帶,皁紋靴。六年,令校尉衣只孫,束帶,襆頭,靴鞋。只孫,一作質孫,本元制,蓋一色衣也。十四年改用金鵝帽,黑漆戧金荔枝銅釘樣,每五釘攢就,四面稍起邊襴,鞓青緊束之。二十二年,令將軍、力士、校尉、旂軍常戴頭巾或榼腦。二十五年,令校尉、力士上直穿靴,出外不許。

刻期冠服:宋置快行親從官,明初謂之刻期。冠方頂巾,衣胸背鷹鷂,花腰,線襖子,諸色闊匾絲絳,大象牙雕花環,行縢八帶鞋。洪武六年,惟用雕刻象牙絳環,餘同庶民。

儒士、生員、監生巾服:洪武三年,令士人戴四方平定巾。二十三年,定儒士、生員衣,自領至裳,去地一寸,袖長過手,復回不及肘三寸。二十四年,以士子巾服,無異吏胥,宜甄別之,命工部制式以進。太祖親視,凡三易乃定。生員襴衫,用玉色布絹為之,寬袖皁緣,皁絳軟巾垂帶。貢舉入監者,不變所服。洪武末,許戴遮陽帽,後遂私戴之。洪熙中,帝問衣藍者何人,左右以監生對。帝曰:「著青衣較好。」乃易青圓領。嘉靖二十二年,禮部言士子冠服詭異,有凌雲等巾,甚乖禮制,詔所司禁之。萬曆二年禁舉人、監生、生儒僭用忠靜冠巾,錦綺鑲履及張傘蓋,戴煖耳,違者五城御史送問。

庶人冠服:明初,庶人婚,許假九品服。洪武三年,庶人初戴四帶巾,改四方平定巾,雜色盤領衣,不許用黃。又令男女衣服,不得僭用金繡、錦綺、紵絲、綾羅,止許紬、絹、素紗,其靴不得裁製花樣、金線裝飾。首飾、釵、鐲不許用金玉、珠翠,止用銀。六年,令庶人巾環不得用金玉、瑪瑙、珊瑚、琥珀。未入流品者同。庶人帽,不得用頂,帽珠止許水晶、香木。十四年令農衣紬、紗、絹、布,商賈止衣絹、布。農家有一人為商賈者,亦不得衣紬、紗。二十二年,令農夫戴斗笠、蒲笠出入市井不禁,不親農業者不許。二十三年,令耆民衣制,袖長過手,復回不及肘三寸;庶人衣長去地五寸,袖長過手六寸,袖樁廣一尺,袖口五寸。二十五年,以民間違禁,靴巧裁花樣,嵌以金線藍條,詔禮部嚴禁庶人不許穿靴,止許穿皮札翁,惟北地苦寒,許用牛皮直縫靴。正德元年,禁商販、僕役、倡優、下賤不許服用貂裘。十六年,禁軍民衣紫花罩甲,或禁門或四外遊走者,緝事人擒之。

士庶妻冠服:洪武三年定制,士庶妻,首飾用銀鍍金,耳環用金珠,釧鐲用銀,服淺色團衫,用紵絲、綾羅、紬絹。五年,令民間婦人禮服惟紫絁,不用金繡,袍衫止紫、綠、桃紅及諸淺淡顏色,不許用大紅、鴉青、黃色,帶用藍絹布。女子在室者,作三小髻,金釵,珠頭閟窄袖褙子。凡婢使,高頂髻,絹布狹領長襖,長裙。小婢使,雙髻,長袖短衣,長裙。成化十年,禁官民婦女不得僭用渾金衣服,寶石首飾。正德元年,令軍民婦女不許用銷金衣服、帳幔,寶石首飾、鐲釧。

協律郎、樂舞生冠服:明初,郊社宗廟用雅樂,協律郎襆頭,紫羅袍,荔枝帶;樂生緋袍,展腳襆頭;舞士襆頭,紅羅袍,荔枝帶,皁靴;文舞生紅袍,武舞生緋袍,俱展腳襆頭,革帶,皁靴。朝會大樂九奏歌工:中華一統巾,紅羅生色大袖衫,畫黃鶯、鸚鵡花樣,紅生絹襯衫,錦領,杏紅絹裙,白絹大口褲,青絲絳,白絹襪,茶褐鞋。其和聲郎押樂者:皁羅闊帶巾,青羅大袖衫,紅生絹襯衫,錦領,塗金束帶,皁靴。其三舞:

一、武舞,曰《平定天下之舞》。舞士皆黃金束髮冠,紫絲纓,青羅生色畫舞鶴花樣窄袖衫,白生絹襯衫,錦領、紅羅銷金大袖罩袍,紅羅銷金裙,皁生色畫花緣襈,白羅銷金汗褲,藍青羅銷金緣,紅絹擁項,紅結子,紅絹束腰,塗金束帶,青絲大絳,錦臂韝,綠雲頭皁靴。舞師,黃金束髮冠,紫絲纓,青羅大袖衫,白絹襯衫,錦領,塗金束帶,綠雲頭皁靴。

一、文舞,曰《車書會同之舞》。舞士皆黑光描金方山冠,青絲纓,紅羅大袖衫,紅生絹襯衫,錦領,紅羅擁項,紅結子,塗金束帶,白絹大口褲,白絹襪,茶褐鞋。舞師冠服與舞士同,惟大袖衫用青羅,不用紅羅擁項、紅結子。

一、文舞,曰《撫安四夷之舞》。舞士,東夷四人,椎髻於後,繫紅銷金頭繩,紅羅銷金抹額,中綴塗金博山,兩傍綴塗金巾環,明金耳環,青羅生色畫花大袖衫,紅生色領袖,紅羅銷金裙,青銷金裙緣,紅生絹襯衫,錦領,塗金束帶,烏皮靴。西戎四人,間道錦纏頭,明金耳環,紅紵絲細摺襖子,大紅羅生色雲肩,綠生色緣,藍青羅銷金汗褲,紅銷金緣繫腰合缽,十字泥金數珠,五色銷金羅香囊,紅絹擁項,紅結子,赤皮靴。南蠻四人,綰朝天髻,繫紅羅生色銀錠,紅銷金抹額,明金耳環,紅織金短襖子,綠織金細摺短裙,絨錦褲,間道紵絲手巾,泥金頂牌,金珠瓔珞綴小金鈴,錦行纏,泥金獅蠻帶,綠銷金擁項,紅結子,赤皮靴。北翟四人,戴單于冠,貂鼠皮簷,雙垂髻,紅銷金頭繩,紅羅銷金抹額,諸色細摺襖子,藍青生色雲肩,紅結子,紅銷金汗褲,繫腰合缽,皁皮靴。其舞師皆戴白捲簷氈帽,塗金帽頂,一撒紅纓,紫羅帽襻,紅綠金繡襖子,白銷金汗褲,藍青銷金緣,塗金束帶,綠擁項,紅結子,赤皮靴。

凡大樂工及文武二舞樂工,皆曲腳襆頭,紅羅生色畫花大袖衫,塗金束帶,紅絹擁項,紅結子,皁皮靴。四夷樂工,皆蓮花帽,諸色細摺襖子,白銷金汗褲,紅銷金緣,紅綠絹束腰,紅羅擁項,紅結子,花靴。

永樂間,定殿內侑食樂。奏《平定天下之舞》,引舞、樂工,皆青羅包巾,青、紅、綠、玉色羅銷金胸背襖子,渾金銅帶,紅羅褡愬,雲頭皁靴,青綠羅銷金包臀。舞人服色如之。奏《撫安四夷之舞》,高麗舞四人,皆笠子,青羅銷金胸背襖子,銅帶,皁靴;琉球舞四人,皆棉布花手巾,青羅大袖襖子,銅帶,白碾光絹間道踢褲,皁皮靴;北番舞四人,皆狐帽,青紅紵絲銷金襖子,銅帶;伍魯速回回舞四人,皆青羅帽,比里罕棉布花手巾,銅帶,皁靴。奏《車書會同之舞》,舞人皆皁羅頭巾,青、綠、玉色皂沿邊襴,茶褐線條皁皮四縫靴。奏《表正萬邦之舞》,引舞二人,青羅包巾,紅羅銷金項帕,紅生絹錦領中單,紅生絹銷金通袖襖子,青線絳銅帶,織錦臂韝,雲頭皁靴,各色銷金包臀,紅絹褡愬。舞人、樂工服色與引舞同。奏《天命有德之舞》,引舞二人,青幪紗如意冠,紅生絹錦領中單,紅生絹大袖袍,各色絹采畫直纏,黑角偏帶,藍絹綵雲頭皁靴,白布襪。舞人、樂工服色與引舞同。

洪武五年,定齋郎、樂生、文武舞生冠服:齋郎,黑介幘,漆布為之,無花樣;服紅絹窄袖衫,紅生絹為裏;皁皮四縫靴;黑角帶。文舞生及樂生,黑介幘,漆布為之,上加描金蟬;服紅絹大紬袍,胸背畫纏枝方葵花,紅生絹為裏,加錦臂韝二;皁皮四縫靴;黑角帶。武舞生,武弁,以漆布為之,上加描金蟬;服飾、靴、帶並同文舞生。嘉靖九年定文、武舞生服制:圜丘服青紵絲,方澤服黑綠紗,朝日壇服赤羅,夕月壇服玉色羅。

宮中女樂冠服:洪武三年定制。凡中宮供奉女樂、奉鑾等官妻,本色皪髻,青羅圓領。提調女樂,黑漆唐巾,大紅羅銷金花圓領,鍍金花帶,皁靴。歌章女樂,黑漆唐巾,大紅羅銷金裙襖,胸帶,大紅羅抹額,青綠羅彩畫雲肩,描金牡丹花皁靴。奏樂女樂,服色與歌章同。嘉靖九年,祀先蠶,定樂女生冠服。黑縐紗描金蟬冠,黑絲纓,黑素羅銷金葵花胸背大袖女袍,黑生絹襯衫,錦領,塗金束帶,白襪,黑鞋。

教坊司冠服:洪武三年定。教坊司樂藝,青字頂巾,繫紅綠褡愬。樂妓,明角冠,皁褙子,不許與民妻同。御前供奉俳長,鼓吹冠,紅羅胸背小袖袍,紅絹褡愬,皁靴。色長,鼓吹冠,紅青羅紵絲彩畫百花袍,紅絹褡愬。歌工,弁冠,紅羅織金胸背大袖袍,紅生絹錦領中單,黑角帶,紅熟絹錦腳褲,皁皮琴鞋,白棉布夾襪。樂工服色與歌工同。凡教坊司官常服冠帶,與百官同;至御前供奉,執粉漆笏,服黑漆襆頭,黑綠羅大袖襴袍,黑角偏帶,皁靴。教坊司伶人,常服綠色巾,以別士庶之服。樂人皆戴鼓吹冠,不用錦絳,惟紅褡愬,服色不拘紅綠。教坊司婦人,不許戴冠,穿褙子。樂人衣服,止用明綠、桃紅、玉色、水紅、茶褐色。俳、色長,樂工,俱皁頭巾,雜色絳。

王府樂工冠服:洪武十五年定。凡朝賀用大樂宴禮,七奏樂樂工俱紅絹彩畫胸背方花小袖單袍,有花鼓吹冠,錦臂韝,皁靴,抹額以紅羅彩畫,束腰以紅絹。其餘樂工用綠絹彩畫胸背方花小袖單袍,無花鼓吹冠,抹額以紅絹彩畫,束腰以紅絹。

軍士服:洪武元年令製衣,表裏異色,謂之鴛鴦戰襖,以新軍號。二十一年,定旂手衛軍士、力士俱紅袢襖,其餘衛所袢襖如之。凡袢襖,長齊膝,窄袖,內實以棉花。二十六年,令騎士服對襟衣,便於乘馬也。不應服而服者,罪之。

皁隸公人冠服:洪武三年定,皁隸,圓頂巾,皁衣。四年定,皁隸公使人,皁盤領衫,平頂巾,白褡愬,帶錫牌。十四年,令各衛門祗禁,原服皁衣改用淡青。二十五年,皁隸伴當不許著靴,止用皮札翁。

外國君臣冠服:洪武二年,高麗入朝,請祭服制度,命製給之。二十七年,定蕃國朝貢儀,國王來朝,如賞賜朝服者,服之以朝。三十一年,賜琉球國王併其臣下冠服。永樂中,賜琉球中山王皮弁、玉圭,麟袍、犀帶,視二品秩。宣德三年,朝鮮國王李濩言:「洪武中,蒙賜國王冕服九章,陪臣冠服比朝廷遞降二等,故陪臣一等比朝臣第三等,得五梁冠服。永樂初,先臣芳遠遣世子禔入朝,蒙賜五梁冠服。臣竊惟世子冠服,何止同陪臣一等,乞為定制。」乃命製六梁冠賜之。嘉靖六年,令外國朝貢入,不許擅用違制衣服。如違,賣者、買者同罪。

僧道服:洪武十四年定,禪僧,茶褐常服,青絳玉色袈裟。講僧,玉色常服,綠絳淺紅袈裟。教僧,皁常服,黑絳淺紅袈裟。僧官如之。惟僧錄司官袈裟,綠文及環皆飾以金。道士,常服青法服,朝衣皆赤,道官亦如之。惟道錄司官法服、朝服,綠文飾金。凡在京道官,紅道衣,金襴,木簡。在外道官,紅道衣,木簡,不用金襴。道士,青道服,木簡。

皇帝寶璽皇后冊寶皇妃以下冊印皇太子冊寶皇太子妃冊寶親王以下冊寶冊印鐵券印信符節宮室制度臣庶室屋制度器用

明初寶璽十七:其大者曰「皇帝奉天之寶」,曰「皇帝之寶」,曰「皇帝行寶」,曰「皇帝信寶」,曰「天子之寶」,曰「天子行寶」,曰「天子信寶」,曰「制誥之寶」,曰「敕命之寶」,曰「廣運之寶」,曰「皇帝尊親之寶」,曰「皇帝親親之寶」,曰「敬天勤民之寶」;又有「御前之寶」、「表章經史之寶」及「欽文之璽」。丹符出驗四方。洪武元年欲制寶璽,有賈胡浮海獻美玉,曰:「此出於闐,祖父相傳,當為帝王寶璽。」乃命製為寶,不知十七寶中,此玉製何寶也。成祖又製「皇帝親親之寶」、「皇帝奉天之寶」、「誥命之寶」、「敕命之寶」。

弘治十三年,鄠縣民毛志學於泥河濱得玉璽,其文曰「受命於天,既壽永昌」。色白微青,螭紐。陜西巡撫熊翀以為秦璽復出,遣人獻之。禮部尚書傅瀚言:「自有秦璽以來,歷代得喪真偽之跡具載史籍。今所進,篆文與《輟耕錄》等書摹載魚鳥篆文不同,其螭紐又與史傳所紀文盤五龍、螭缺一角、旁刻魏錄者不類。蓋秦璽亡已久,今所進與宋、元所得,疑皆後世摹秦璽而刻之者。竊惟璽之用,以識文書,防詐偽,非以為寶玩也。自秦始皇得藍田玉以為璽,漢以後傳用之,自是巧爭力取,謂得此乃足以受命,而不知受命以德,不以璽也。故求之不得,則偽造以欺人;得之則君臣色喜,以誇示於天下。是皆貽笑千載。我高皇帝自制一代之璽,文各有義,隨事而施,真足以為一代受命之符,而垂法萬世,何藉此璽哉!」帝從其言,卻而不用。

嘉靖十八年,新製七寶:曰「奉天承運大明天子寶」、「大明受命之寶」、「巡狩天下之寶」、「垂訓之寶」、「命德之寶」、討罪安民之寶」、敕正萬民之寶」。與國初寶璽共為御寶二十四,尚寶司官掌之。

皇后之冊:用金冊二片,依周尺長一尺二寸,廣五寸,厚二分五釐。字依數分行,鐫以真書。上下有孔,聯以紅絳,開闔如書帙,藉以紅錦褥。冊盝用木,飾以渾金瀝粉蟠龍,紅紵絲襯裏,內以紅羅銷金小袱裹冊,外以紅羅銷金夾袱包之,五色小絳縈於外。寶用金,龜紐,篆文曰「皇后之寶」,依周尺方五寸九分,厚一寸七分。寶池用金,闊取容。寶篋二副,一置寶,一置寶池。每副三重:外篋用木,飾以渾金瀝粉蟠龍,紅紵絲襯裏;中篋用金鈒蟠龍;內小篋飾如外篋,內置寶座,四角雕蟠龍,飾以渾金。座上用錦褥,以銷金紅羅小夾袱裹寶,其篋外各用紅羅銷金大夾袱覆之。臨冊之日,冊寶俱置於紅髹輿案,案頂有紅羅瀝水,用擔床舉之。

皇貴妃而下,有冊無寶而有印。妃冊,用鍍金銀冊二片,廣長與后冊同。冊盝飾以渾金瀝粉蟠鳳。其印用金,龜紐,尺寸與諸王寶同,文曰「皇妃之印」。篋飾以蟠鳳。宣德元年,帝以貴妃孫氏有容德,特請於皇太后,製金寶賜之,未幾即誕皇嗣。自是貴妃授寶,遂為故事。嘉靖十年,立九嬪,冊用銀,殺皇妃五分之一,以金飾之。

皇太子冊寶:冊用金,二片,其制及盝篋之飾與皇后冊同。寶用金,龜紐,篆書「皇太子寶」。其制及池篋之飾與後寶同。

皇太子妃冊寶:其冊用金,兩葉,重百兩,每葉高一尺二寸,廣五寸。藉冊以錦,聯冊以紅絲絳,墊冊以錦褥,裹冊以紅羅銷金袱。其盝飾以渾金瀝粉雲鳳,內有花銀釘鉸,嵌金絲鐵筦籥;外以紅羅銷金袱覆之。其金寶之制未詳。洪武二十八年更定,止授金冊,不用寶。

親王冊寶:冊制與皇太子同。其寶用金,龜紐,依周尺方五寸二分,厚一寸五分,文曰「某王之寶」。池篋之飾,與皇太子寶同。寶盝之飾,則雕蟠螭。

親王妃冊印:其金冊,高視太子妃冊減一寸,餘制悉同,冊文視親王。其金印之制未詳。洪武二十八年更定,止授金冊。

公主冊印:銀冊二片,鐫字鍍金,藉以紅錦褥。冊盝飾以渾金瀝粉蟠螭。其印同宋制,用金,龜紐,文曰「某國公主之印」。方五寸二分,厚一寸五分。印池用金,廣取容。印外篋用木,飾以渾金瀝粉盤鳳,中篋用金鈒蟠鳳,內小篋,飾如外篋。

親王世子金冊金寶:承襲止授金冊,傳用金寶。

世子妃亦用金冊。洪武二十三年鑄世子妃印,制視王妃,金印,龜紐,篆文曰「某世子妃印」。

郡王,鍍金銀冊、鍍金銀印,冊文視世子。其妃止有鍍金銀冊。

功臣鐵券:洪武二年,太祖欲封功臣,議為鐵券,而未有定制。或言台州民錢允一有家藏吳越王鏐唐賜鐵券,遂遣使取之,因其式而損益焉。其制如瓦,第為七等。公二等:一高尺,廣一尺六寸五分;一高九寸五分,廣一尺六寸。侯三等:一高九寸,廣一尺五寸五分;一高八寸五分,廣一尺五寸;一高八寸,廣一尺四寸五分。伯二等:一高七寸五分,廣一尺三寸五分;一高六寸五分,廣一尺二寸五分。外刻履歷、恩數之詳,以記其功;中鐫免罪、減祿之數,以防其過。字嵌以金。凡九十七副,各分左右,左頒功臣,右藏內府,有故則合之,以取信焉。三年,大封功臣,公六人,侯二十八人,並賜鐵券。公:李善長、徐達、李文忠、馮勝、鄧愈、常茂。侯:湯和、唐勝宗、陸仲亨、周德興、華雲龍、顧時、耿炳文、陳德、郭子興、王志、鄭遇春、費聚、吳良、吳楨、趙庸、廖永忠、俞通源、華高、楊璟、康鐸、朱亮祖、傅友德、胡美、韓政、黃彬、曹良臣、梅思祖、陸聚。二十五年,改製鐵券,賜公傅友德,侯王弼、耿炳文、郭英及故公徐達、李文忠,侯吳傑、沐英,凡八家。永樂初,靖難功臣亦有賜者。

百官印信:洪武初,鑄印局鑄中外諸司印信。正一品,銀印,三臺,方三寸四分,厚一寸。六部、都察院並在外各都司,俱正二品,銀印二臺,方三寸二分,厚八分。其餘正二品、從二品官,銀印二臺,方三寸一分,厚七分。惟衍聖公以正二品,三臺銀印,則景泰三年賜也。順天、應天二府俱正三品,銀印,方二寸九分,厚六分五釐。其餘正三品、從三品官,俱銅印,方二寸七分,厚六分。惟太僕、光祿寺並在外鹽運司,俱從三品,銅印,方減一分,厚減五釐。正四品、從四品,俱銅印,方二寸五分,厚五分。正五品、從五品,俱銅印,方二寸四分,厚四分五釐。惟在外各州從五品,銅印,方減一分,厚減五釐。正六品、從六品,俱銅印,方二寸二分,厚三分五釐。正七品、從七品,銅印,方二寸一分,厚三分。正從八品,俱銅印,方二寸,厚二分五釐。正從九品,俱銅印,方一寸九分,厚二分二釐。未入流者,銅條記,闊一寸三分,長二寸五分,厚二分一釐。以上俱直紐,九疊篆文。初,雜職亦方印,至洪武十三年始改條記。凡百官之印,惟文淵閣銀印,直紐,方一寸七分,厚六分,玉箸篆文,誠重之也。武臣受重寄者,征西、鎮朔、平蠻諸將軍,銀印,虎紐,方三寸三分,厚九分,柳葉篆文。洪武中,嘗用上公佩將軍印,後以公、侯、伯及都督充總兵官,名曰「掛印將軍」。有事征伐,則命總兵佩印以往,旋師則上所佩印於朝。此外,惟漕運總兵印同將軍。其在外鎮守總兵、參將掛印,則洪熙元年始也。有文臣掛將軍印者,王驥以兵部尚書征湖、貴苗,掛平蠻將軍印;王越以左都御史守大同,掛征西將軍印。其他文武大臣,有領敕而權重者,或給以銅關防,直紐,廣一寸九分五釐,長二寸九分,厚三分,九疊篆文,雖宰相行邊,與部曹無異。惟正德時,張永征安化王,用金鑄,嘉靖中,顧鼎臣居守,用牙鏤關防,皆特賜也。初,太祖重御史之職,分河南等十三道,每道鑄二印,文曰「繩愆糾繆」,守院御史掌其一,其一藏內府,有事則受以出,復命則納之。洪武二十三年,都御史袁泰言各道印篆相類。乃命改製某道監察御史,其奉差者,則曰「巡按某處監察御史」,銅印直紐,有眼,方一寸五分,厚三分,八疊篆文。成祖初幸北京,有一官署二三印者,夏原吉至兼掌九卿印,諸曹並於朝房取裁,其任重矣。

明初,賜高麗金印,龜紐,方三寸,文曰「高麗國王之印」,賜安南鍍金銀印,駝紐,方三寸,文曰「安南國王之印」。賜占城鍍金銀印,駝紐,方三寸,文曰「占城國王之印」。賜吐蕃金印,駝紐,方五寸,文曰「白蘭王印」。

符牌:凡宣召親王,必遣官齎金符以往。親王之籓及鎮守、巡撫諸官奏請符驗,俱從兵部奏,行尚寶司領之。洪武二十六年定制:凡公差,以軍情重務及奉旨差遣給驛者,兵部既給勘合,即赴內府,關領符驗,給驛而去,事竣則繳。嘉靖三十七年定制:南京、鳳陽守備內外官,並各處鎮守總兵、巡撫,及各守一方不受鎮守節制內外守備,並領符驗奏事。凡監槍、整飭兵備,並一城一堡守備官,不許關領符驗。其制,上織船馬之狀,起馬者用馬字號,起船者水字號,起雙馬者達字號,起單馬者通字號,起站船者信字號。洪武四年,始製用寶金牌。凡軍機文書,自都督府、中書省長官而外,不許擅奏。有詔調軍,中書省同都督府覆奏,乃各出所藏金牌,入請用寶。又造軍中調發符牌,用鐵,長五寸,闊半之,上鈒二飛龍,下鈒二麒麟,首為圜竅,貫以紅絲絳。嘗遣官齎金牌、信符詣西番,以茶易馬。其牌四十一,上號藏內府,下號降各番,篆文曰「皇帝聖旨」,左曰「合當差發」,右曰「不信者斬」。二十二年又頒西番金牌、信符。其後番官款塞,皆齎原降牌符而至。永樂二年製信符、金字紅牌給雲南諸蠻。凡歷代改元,則所頒外國信符、金牌,必更鑄新年號給之。此符信之達於四裔者也。

其武臣懸帶金牌,則洪武四年所造。闊二寸,長一尺,上鈒雙龍,下鈒二伏虎,牌首尾為圓竅,貫以紅絲絳。指揮佩金牌,雙雲龍,雙虎符。千戶佩鍍金銀牌,獨雲龍,獨虎符。百戶素雲銀牌符。太祖親為文鈒之曰:「上天祐民,朕乃率撫。威加華夏,實憑虎臣。賜爾金符,永傳後嗣。」天子祀郊廟,若視學、耤田,勛衛扈從及公侯、駙馬、五府都督日直、錦衣衛當直,及都督率諸衛千百戶夜巡內皇城,金吾諸衛各輪官隨朝巡綽,俱給金牌,有龍者、虎者、麒麟者、獅者、雲者,以官為差。

其扈駕金字銀牌,則洪武六年所造。尋改為守衛金牌,以銅為之,塗以金,高一尺,闊三寸,分字號凡五。仁字號,上鈒獨龍蟠雲花,公、侯、伯、都督佩之。義字號,鈒伏虎盤雲花,指揮佩之。禮字號,獬貂豸蟠雲花,千戶、衛鎮撫佩之。智字號,鈒獅子蟠雲花,百戶、所鎮撫佩之。信字號,鈒蟠雲花,將軍佩之。牌下鑄「守衛」二篆字,背鑄「凡守衛官軍懸帶此牌」等二十四字,牌首竅貫青絲。鎮撫及將軍隨駕直宿衛者佩之,下直則納之。凡夜巡官,於尚寶司領令牌,禁城各門、金吾等衛指揮、千戶,分領申字號牌,午門自一至四,長安左右門、東華門自五至八,西華門自九至十二,玄武門自十三至十六。五城兵馬指揮亦日領令牌,東西南北中城,分領木、金、火、水、土五字號。留守五衛、巡城官並金吾等衛守衛官,俱領銅符。留守衛指揮所領承字及東西北字號牌,俱左半字陽文,左比。金吾等衛,端門、承天門、東西北安門指揮千戶所領承字及東西北字號,俱右半字陰文,右比。銅符字號比對相同,方許巡行。內官、內使之出,亦須守門官比對銅符而後行。皇城九門守衛軍與圍子手,各領勇字號銅牌。錦衣校尉上直及光祿寺吏典廚役,遇大祀,俱佩雙魚銅牌。永樂六年駕幸北京,扈從官俱帶牙牌;五府、六部、都察院、大理寺、錦衣衛各鑄印信,通政司、鴻臚寺各鑄關防,謂之行在衛門印信關防。其後命內府印綬監收貯。嘉靖十八年南巡,禮部領出,以給扈從者焉。凡郊廟諸祭陪祀供事官及執事者,入壇俱領牙牌,洪武八年始也。圓花牌,陪祀官領之。長花牌,供事官領之。長素牌,執事人領之。又謂之祀牌。凡駕詣陵寢,扈從官俱於尚寶司領小牙牌。嘉靖九年,皇后行親蠶禮,文官四品以上、武官三品以上命婦及使人,俱於尚寶司領牙牌,有雲花圜牌、鳥形長牌之異。凡文武朝參官、錦衣衛當駕官,亦領牙牌,以防姦偽,洪武十一年始也。其制,以象牙為之,刻官職於上。不佩則門者卻之,私相借者論如律。牙牌字號,公、侯、伯以勳字,駙馬都尉以親字,文官以文字,武官以武字,教坊官以樂字,入內官以官字。正德十六年,禮科邢寰言:「牙牌惟常朝職官得懸。比來權姦侵柄,傳旨升官者輒佩牙牌,宜清核以重名器。」乃命文職不朝參者,毋得濫給牙牌;武官進御侍班、佩刀、執金爐者給與。嘉靖二十八年,內府供事匠作、武職官皆帶朝參牙牌,嘗奉旨革奪,旋復給之。給事中陳邦修以為言,禮部覆奏:「《會典》所載,文武官出入禁門帶牙牌,有執事、供事、朝參之別。執事、供事者,皆屆期而領,如期而繳。惟朝參牙牌,得朝夕懸之,非徒為關防之具,亦以示等威之辨也。虛銜帶俸、供事、執事者,不宜概領。第出入禁闥,若一切革奪,何由譏察?尚寶司所貯舊牌數百,上有『入內府』字號,請以給之。至於衛所武官,掌印、僉書侍衛之外,非屬朝參供役者,盡革奪之。其納粟、填註冒賜牙牌及罷退閑住官舊所關領不繳者,俱逮問。」報可。

洪武十五年,製使節,黃色三簷寶蓋,長二尺,黃紗袋籠之。又製丹漆架一,以節置其上。使者受命,則載以行;使歸,則持之以復命。二十三年,詔考定使節之制,禮部奏:「漢光武時,以竹為節,柄長八尺,其毛三重。而黃公紹《韻會》註:漢節柄長三尺,毛三重,以旄牛為之。」詔從三尺之制。

宮室之制:吳元年作新內。正殿曰奉天殿,後曰華蓋殿,又後曰謹身殿,皆翼以廊廡。奉天殿之前曰奉天門,殿左曰文樓,右曰武樓。謹身殿之後為宮,前曰乾清,後曰坤寧,六宮以次列。宮殿之外,周以皇城,城之門,南曰午門,東曰東華,西曰西華,北曰玄武。時有言瑞州文石可甃地者。太祖曰:「敦崇儉朴,猶恐習於奢華,爾乃導予奢麗乎?」言者慚而退。洪武八年,改建大內宮殿,十年告成。闕門曰午門,翼以兩觀。中三門,東西為左、右掖門。午門內曰奉天門,門內奉天殿,嘗御以受朝賀者也。門左右為東、西角門,奉天殿左、右門,左曰中左,右曰中右,兩廡之間,左曰文樓,右曰武樓。奉天殿之後曰華蓋殿,華蓋殿之後曰謹身殿,殿後則乾清宮之正門也。奉天門外兩廡間有門,左曰左順,右曰右順。左順門外有殿曰文華,為東宮視事之所。右順門外有殿曰武英,為皇帝齋戒時所居。制度如舊,規模益宏。二十五年改建大內金水橋,又建端門、承天門樓各五間,及長安東、西二門。永樂十五年,作西宮於北京。中為奉天殿,側為左右二殿,南為奉天門,左右為東、西角門。其南為午門,又南為承天門。殿北有後殿、涼殿、煖殿及仁壽、景福、仁和、萬春、永壽、長春等宮,凡為屋千六百三十餘楹。十八年,建北京,凡宮殿、門闕規制,悉如南京,壯麗過之。中朝曰奉天殿,通為屋八千三百五十楹。殿左曰中左門,右曰中右門。丹墀東曰文樓,西曰武樓,南曰奉天門,常朝所御也。左曰東角門,右曰西角門,東廡曰左順門,西廡曰右順門,正南曰午門。中三門,翼以兩觀,觀各有樓,左曰左掖門,右曰右掖門。午門左稍南,曰闕左門,曰神廚門,內為太廟。右稍南,曰闕右門,曰社左門,內為太社稷。又正南曰端門,東曰廟街門,即太廟右門也。西曰社街門,即太社稷壇南左門也。又正南曰承天門,又折而東曰長安左門,折而西曰長安右門。東後曰東安門,西後曰西安門,北後曰北安門。正南曰大明門,中為馳道,東西長廊各千步。奉天殿之後曰華蓋殿,又後曰謹身殿。謹身殿左曰後左門,右曰後右門。正北曰乾清門,內為乾清宮,是曰正寢。後曰交泰殿。又後曰坤寧宮,為中宮所居。東曰仁壽宮,西曰清寧宮,以奉太后。左順門之東曰文華殿。右順門之西曰武英殿。文華殿東南曰東華門,武英殿西南曰西華門。坤寧宮後曰坤寧門,門之後曰玄武門。其他宮殿,名號繁多,不能盡列,所謂千門萬戶也。皇城內宮城外,凡十有二門:曰東上門、東上北門、東上南門、東中門、西上門、西上北門、西上南門、西中門、北上門、北上東門、北上西門、北中門。復於皇城東南建皇太孫宮,東安門外東南建十王街。宣宗留意文雅,建廣寒、清暑二殿,及東、西瓊島,游觀所至,悉置經籍。正統六年重建三殿。嘉靖中,於清寧宮後地建慈慶宮,於仁壽宮故基建慈寧宮。三十六年,三殿門樓災,帝以殿名奉天,非題扁所宜用,敕禮部議之。部臣會議言:「皇祖肇造之初,名曰奉天者,昭揭以示虔爾。既以名,則是昊天監臨,儼然在上,臨御之際,坐以視朝,似未安也。今乃修復之始,宜更定,以答天庥。」明年重建奉天門,更名曰大朝門。四十一年更名奉天殿曰皇極,華蓋殿曰中極,謹身殿曰建極,文樓曰文昭閣,武樓曰武成閣,左順門曰會極,右順門曰歸極,大朝門曰皇極,東角門曰弘政,西角門曰宣治。又改乾清宮右小閣名曰道心,旁左門曰仁蕩,右門曰義平。世宗初,墾西苑隙地為田,建殿曰無逸,亭曰豳風,又建亭曰省耕,曰省斂,每歲耕獲,帝輒臨觀。十三年,西苑河東亭榭成,親定名曰天鵝房,北曰飛靄亭,迎翠殿前曰浮香亭,寶月亭前曰秋輝亭,昭和殿前曰澄淵亭,後曰AZ臺坡,臨漪亭前曰水雲榭,西苑門外二亭曰左臨海亭、右臨海亭,北閘口曰湧玉亭,河之東曰聚景亭,改呂梁洪之亭曰呂梁,前曰檥金亭,翠玉館前曰擷秀亭。

親王府制:洪武四年定,城高二丈九尺,正殿基高六尺九寸,正門、前後殿、四門城樓,飾以青綠點金,廊房飾以青黛。四城正門,以丹漆,金塗銅釘。宮殿窠栱攢頂,中畫蟠螭,飾以金,邊畫八吉祥花。前後殿座,用紅漆金蟠螭,帳用紅銷金蟠螭。座後壁則畫蟠螭、彩雲,後改為龍。立山川、社稷、宗廟於王城內。七年定親王所居殿,前曰承運,中曰圜殿,後曰存心;四城門,南曰端禮,北曰廣智,東曰體仁,西曰遵義。太祖曰:「使諸王睹名思義,以籓屏帝室。」九年定親王宮殿、門廡及城門樓,皆覆以青色琉璃瓦。又命中書省臣,惟親王宮得飾朱紅、大青綠,其他居室止飾丹碧。十二年,諸王府告成。其制,中曰承運殿,十一間,後為圜殿,次曰存心殿,各九間。承運殿兩廡為左右二殿,自存心、承運,周迴兩廡,至承運門,為屋百三十八間。殿後為前、中、後三宮,各九間。宮門兩廂等室九十九間。王城之外,周垣、西門、堂庫等室在其間,凡為宮殿室屋八百間有奇。弘治八年更定王府之制,頗有所增損。

郡王府制:天順四年定。門樓、廳廂、廚庫、米倉等,共數十間而已。

公主府第:洪武五年,禮部言:「唐、宋公主視正一品,府第並用正一品制度。今擬公主第,廳堂九間,十一架,施花樣獸脊,梁、棟、斗栱、簷桷彩色繪飾,惟不用金。正門五間,七架。大門,綠油,銅環。石礎、牆磚,鐫鑿玲瓏花樣。」從之。

百官第宅:明初,禁官民房屋不許雕刻古帝后、聖賢人物及日月、龍鳳、狻猊、麒麟、犀象之形。凡官員任滿致仕,與見任同。其父祖有官,身歿,子孫許居父祖房舍。洪武二十六年定制,官員營造房屋,不許歇山轉角,重簷重栱,及繪藻井,惟樓居重簷不禁。公侯,前廳七間、兩廈,九架。中堂七間,九架。後堂七間,七架。門三間,五架,用金漆及獸面錫環。家廟三間,五架。覆以黑板瓦,脊用花樣瓦獸,梁、棟、斗栱、簷桷彩繪飾。門窗、枋柱金漆飾。廊、廡、庖、庫從屋,不得過五間,七架。一品、二品,廳堂五間,九架,屋脊用瓦獸,梁、棟、斗栱、簷桷青碧繪飾。門三間,五架,綠油,獸面錫環。三品至五品,廳堂五間,七架,屋脊用瓦獸,梁、棟、簷桷青碧繪飾。門三間,三架,黑油,錫環。六品至九品,廳堂三間,七架,梁、棟飾以土黃。門一間,三架,黑門,鐵環。品官房舍,門窗、戶牖不得用丹漆。功臣宅舍之後,留空地十丈,左右皆五丈。不許那移軍民居止,更不許於宅前後左右多占地,構亭館,開池塘,以資遊眺。三十五年,申明禁制,一品、三品廳堂各七間,六品至九品廳堂梁棟祗用粉青飾之。

庶民廬舍:洪武二十六年定制,不過三間,五架,不許用鬥栱,飾彩色。三十五年復申禁飭,不許造九五間數,房屋雖至一二十所,隨基物力,但不許過三間。正統十二年令稍變通之,庶民房屋架多而間少者,不在禁限。

器用之禁:洪武二十六年定,公侯、一品、二品,酒注、酒盞金,餘用銀。三品至五品,酒注銀,酒盞金,六品至九品,酒注、酒盞銀,餘皆磁、漆。木器不許用硃紅及抹金、描金、雕琢龍鳳文。庶民,酒注錫,酒盞銀,餘用磁、漆。百官,床面、屏風、槅子,雜色漆飾,不許雕刻龍文,並金飾硃漆。軍官、軍士,弓矢黑漆,弓袋、箭囊,不許用硃漆描金裝飾。建文四年申飭官民,不許僭用金酒爵,其椅棹木器亦不許硃紅金飾。正德十六年定,一品、二品,器皿不用玉,止許用金。商賈、技藝家器皿不許用銀。餘與庶民同。
