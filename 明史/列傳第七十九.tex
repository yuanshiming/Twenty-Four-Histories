\article{列傳第七十九}

\begin{pinyinscope}
毛澄汪俊弟偉吳一鵬朱希周何孟春豐熙子坊徐文華薛蕙胡侍王祿侯廷訓

毛澄,字憲清,崑山人。舉弘治六年進士第一。授修撰。預修《會典》成,進右諭德,直講東宮。武宗為太子,以澄進講明晰稱之帝。帝大喜。方秋夜置宴,即彳育又以賜。武宗立,進左庶子,直經筵。以母憂歸。正德四年,劉瑾摘《會典》小疵貶諸纂修者秩,以澄為侍讀。服闋還朝,進侍講學士。再遷學士,掌院事,歷禮部侍郎。十二年六月拜尚書。

其年八月朔,帝微行。澄率侍郎王瓚、顧清等疏請還宮。既又出居庸,幸宣府,久留不返。澄等頻疏諫,悉不報。明年正月,駕旋,命百官戎服郊迎。澄等請用常服,不許。七月,帝自稱威武大將軍朱壽,統六師巡邊。遂幸宣府,抵大同,歷山西至榆林。澄等屢疏馳諫。至十二月,復偕廷臣上疏曰:「去歲正月以來,鑾輿數駕,不遑寧居。今茲之行,又已半歲。宗廟、社稷享祀之禮並係攝行,萬壽、正旦、冬至朝賀之儀悉從簡略。臘朔省牲,闕而不行,遂二年矣。歲律將周,郊禋已卜。皇祖之訓曰:『凡祀天地,精誠則感格,怠慢則禍生。』今六龍遐騁,旋軫無日。萬一冰雪阻違,道途梗塞,元正上日不及躬執玉帛於上帝前,陛下何以自安?且邊地荒寒,隆冬尤甚。臣等處重城,食厚祿,仰思聖體勞頓,根本空虛,遙望清塵,憂心如醉。伏祈趣駕速還,躬親稞享,宗社臣民幸甚。」不報。十四年二月,駕甫還京,即諭禮部:「總督軍務威武大將軍、總兵官、太師、鎮國公硃壽遣往兩畿,瞻東嶽,奉安聖像,祈福安民。」澄等駭愕,復偕廷臣上言:「陛下以天地之子,承祖宗之業,九州四海但知陛下有皇帝之號。今曰『總督軍務、威武大將軍、太師、鎮國公』者,臣等莫知所指。夫出此旨者,陛下也。加此號者,陛下也。不知受此號者何人?如以皇儲未建,欲遍告名山大川,用祈默相,則遣使走幣,足將敬矣。何必躬奉神像,獻寶香,如佛、老所為哉?」因歷陳五不可。亦不報。

宸濠反江西,帝南征示威武,駐蹕留都者踰歲。澄屢請回鑾。及駕返通州,用江彬言,將即賜宸濠死。澄據漢庶人故事,請還京告郊廟,獻俘行戮。不從。中官王堂鎮浙江,請建生祠;西番闡化王使者乞額外賜茶九萬斤。澄皆力爭,不聽。王瓊欲陷彭澤,澄獨白其無罪。

武宗崩,澄偕大學士梁儲、壽寧侯張鶴齡、駙馬崔元、太監韋霦等迎世宗於安陸。既至,將謁見,有議用天子禮者。澄曰:「今即如此,後何以加?豈勸進、辭讓之禮當遂廢乎?」世宗踐阼甫六日,有旨議興獻王主祀及尊稱。五月七日戊午,澄大會文武群臣,上議曰:「考漢成帝立定陶王為皇太子,立楚孝王孫景為定陶王,奉共王祀。共王者,皇太子本生父也。時大司空師丹以為恩義備至。今陛下入承大統,宜如定陶王故事,以益王第二子崇仁王厚炫繼興王後,襲興王主祀事。又考宋濮安懿王之子入繼仁宗後,是為英宗。司馬光謂濮王宜尊以高官大爵,稱王伯而不名。范鎮亦言:『陛下既考仁宗,若復以濮王為考,於義未當。』乃立濮王園廟,以宗樸為濮國公奉濮王祀。程頤之言曰:『為人後者,謂所後為父母,而謂所生為伯、叔父母,此生人之大倫也。然所生之義,至尊至大,宜別立殊稱。曰皇伯、叔父某國大王,則正統既明,而所生亦尊崇極矣。』今興獻王於孝宗為弟,於陛下為本生父,與濮安懿王事正相等。陛下宜稱孝宗為皇考,改稱興獻王為『皇叔父興獻大王』,妃為『皇叔母興獻王妃』。凡祭告興獻王及上箋於妃,俱自稱『姪皇帝』某,則正統、私親,恩禮兼盡,可以為萬世法。」議上,帝怒曰:「父母可更易若是耶!」命再議。

其月二十四日乙亥,澄復會廷臣上議曰:「《禮》為人後者為之子,自天子至庶人一也。興獻王子惟陛下一人,既入繼大統,奉祀宗廟,是以臣等前議欲令崇仁王厚炫主興獻王祀。至於稱號,陛下宜稱為『皇叔父興獻大王』,自稱『姪皇帝』名。以宋程頤之說為可據也。本朝之制,皇帝於宗籓尊行,止稱伯父、叔父,自稱皇帝而不名。今稱興獻王為『皇叔父大王』,又自稱名,尊崇之典已至,臣等不敢復有所議。」因錄程頤《代彭思永議濮王禮疏》進覽。帝不從,命博考前代典禮,再議以聞。澄乃復會廷臣上議曰:「臣等會議者再,請改稱興獻王為叔父者,明大統之尊無二也。然加『皇』字於『叔父』之上,則凡為陛下伯、叔諸父皆莫能與之齊矣。加『大』字於『王』之上,則天下諸王皆莫得而並之矣。興獻王稱號既定,則王妃稱號亦隨之,天下王妃亦無以同其尊矣。況陛下養以天下,所以樂其心,不違其志,豈一家一國之養可同日語哉。此孔子所謂事之以禮者。其他推尊之說,稱親之議,似為非禮。推尊之非,莫詳於魏明帝之詔。稱親之非,莫詳於宋程頤之議。至當之禮,要不出於此。」並錄上魏明帝詔書。當是時,帝銳意欲推崇所生,而進士張璁復抗疏極言禮官之謬。帝心動,持澄等疏久不下。至八月庚辰朔,再命集議。澄等乃復上議曰:「先王制禮,本乎人情。武宗既無子嗣,又鮮兄弟,援立陛下於憲廟諸孫之中。是武宗以陛下為同堂之弟,考孝宗,母慈壽,無可疑矣,可復顧私親哉?」疏入,帝不懌,復留中。

會給事中邢寰請議憲廟皇妃邵氏徽號,澄上言:「王妃誕生獻王,實陛下所自出。但既承大統,則宜考孝宗,而母慈壽太后矣。孝宗於憲廟皇妃宜稱皇太妃,則在陛下宜稱太皇太妃。如此,則彞倫既正,恩義亦篤。」疏入,報聞。其月,帝以母妃將至,下禮官議其儀。澄等請由崇文門入東安門,帝不可。乃議由正陽左門入大明東門,帝又不可。澄等執議如初。帝乃自定其儀,悉由中門入。

時尊崇禮猶未定,張璁復進《大禮或問》,帝益向之。至九月末,乃下澄等前疏,更令博採輿論以聞。澄等知勢不可已,謀於內閣,加稱興王為『帝』,妃為「后」,而以皇太后懿旨行之。乃疏言:「臣等一得之愚,已盡於前議。茲欲仰慰聖心,使宜於今而不戾乎情,合乎古而無悖乎義,則有密勿股肱在。臣等有司,未敢擅任。」帝遂於十月二日庚辰,以慈壽皇太后旨加興獻王號曰「興獻帝」,妃曰「興國太后」,皇妃邵氏亦尊為「皇太后」,宣示中外。顧帝雖勉從廷議,意猶慊之。十二月十一日己丑,復傳諭加稱皇帝。內閣楊廷和等封還御批,澄抗疏力爭,又偕九卿喬宇等合諫,帝皆不允。明年,嘉靖改元正月,清寧宮後三小宮災。澄復以為言,會朝臣亦多諫者,事獲止。

澄端亮有學行,論事侃侃不撓。帝欲推尊所生,嘗遣中官諭意,至長跪稽首。澄駭愕,急扶之起。其人曰:「上意也。上言『人孰無父母,奈何使我不獲伸』,必祈公易議。」因出囊金畀澄。澄奮然曰:「老臣悖耄,不能隳典禮。獨有一去,不與議已耳。」抗疏引疾至五六上,帝輒慰留不允。二年二月疾甚,復力請,乃許之。舟至興濟而卒。

先是,論定策功,加澄太子太傅,廕錦衣世指揮同知,力辭不受。帝雅敬憚澄,雖數忤旨,而恩禮不衰。既得疾,遣醫診視,藥物之賜時至。其卒也,深悼惜之。贈少傅,謚文簡。

汪俊,字抑之,弋陽人。父鳳,進士,貴州參政。俊舉弘治六年會試第一,授庶吉士,進編修。正德中,與修《孝宗實錄》,以不附劉瑾、焦芳,調南京工部員外郎。瑾、芳敗,召復原官。累遷侍讀學士,擢禮部右侍郎。嘉靖元年轉吏部左侍郎。

時議興獻王尊號,與尚書喬宇、毛澄輩力爭。澄引疾去,代者羅欽順不至,乃以俊為禮部尚書。是時獻王已加帝號矣,主事桂萼復請稱皇考。章下廷議。三年二月,俊集廷臣七十有三人上議曰:「祖訓『兄終弟及』,指同產言。今陛下為武宗親弟,自宜考孝宗明矣。孰謂與人為後,而滅武宗之統也。《儀禮》傳曰:『為人後者,孰後?後大宗也。』漢宣起民間,猶嗣孝昭。光武中興,猶考孝元。魏明帝詔皇后無子,擇建支子,以繼大宗。孰謂入繼之主與為人後者異也。宋范純仁謂英宗親受詔為子,與入繼不同,蓋言恩義尤篤,尤當不顧私親,非以生前為子者乃為人後,身後入繼者不為人後也。萼言『孝宗既有武宗為之子,安得復為立後。』臣等謂陛下自後武宗而上考孝宗,非為孝宗立後也。又言『武宗全神器授陛下,何忍不繼其統。』臣等謂陛下既稱武宗皇兄矣,豈必改孝宗稱伯,乃為繼其統乎?又言『禮官執者不過前宋《濮議》』。臣等愚昧,所執實不出此。蓋宋程頤之議曰:『雖當專意於正統,豈得盡絕於私恩。故所繼,主於大義;所生,存乎至情。至於名稱,統緒所繫,若其無別,斯亂大倫。』殆為今日發也。謹集諸章奏,惟進士張璁、主事霍韜、給事中熊浹與萼議同,其他八十餘疏二百五十餘人,皆如臣等議。」

議上,留中。而特旨召桂萼、張璁、席書於南京。越旬有五日,乃下諭曰:「朕奉承宗廟正統,大義豈敢有違。第本生至情,亦當兼盡。其再集議以聞。」俊不得已,乃集群臣請加「皇」字,以全徽稱。議上,復留十餘日。至三月朔,乃詔禮官,加稱興獻帝為本「生皇考恭穆獻皇帝」,興國太后為「本生母章聖皇太后」。擇日祭告郊廟,頒詔天下。而別諭建室奉先殿側,恭祀獻皇。俊等復爭曰:「陛下入奉大宗,不得祭小宗,亦猶小宗之不得祭大宗也。昔興獻帝奉籓安陸,則不得祭憲宗。今陛下入繼大統,亦不得祭興獻帝。是皆以禮抑情者也。然興獻帝不得迎養壽安皇太后於籓邸,陛下得迎興國太后於大內,受天下之養,而尊祀興獻帝以天子之禮樂,則人子之情獲自盡矣。乃今聖心無窮,臣等敢不將順,但於正統無嫌,乃為合禮。」帝曰:「朕但欲奉先殿側別建一室,以伸追慕之情耳。迎養籓邸,祖宗朝無此例,何容飾以為詞。其令陳狀。」俊具疏引罪。用嚴旨切責,而趣立廟益急。俊等乃上議曰:「立廟大內,有干正統。臣實愚昧,不敢奉詔。」帝不納,而令集廷臣大議。俊等復上議曰:「謹按先朝奉慈別殿,蓋孝宗皇帝為孝穆皇太后附葬初畢,神主無薦享之所而設也。當時議者,皆據周制特祀姜嫄而言。至為本生立廟大內,則從古未聞。惟漢哀帝為定陶恭王立廟京師。師丹以為不可,哀帝不聽,卒遺後世之譏。陛下有可以為堯、舜之資,臣等不敢導以衰世之事。請於安陸特建獻帝百世不遷之廟,俟他日襲封興王子孫世世獻饗,陛下歲時遣官持節奉祀,亦足伸陛下無窮至情矣。」帝仍命遵前旨再議,俊遂抗疏乞休。再請益力,帝怒,責以肆慢,允其去。召席書未至,令吳一鵬署事。《明倫大典》成,落俊職,卒於家。隆慶初,贈少保,謚文莊。

俊行誼修潔,立朝光明端介。學宗洛、閩。與王守仁交好,而不同其說。學者稱「石潭先生」。

弟偉,字器之。由庶吉士授檢討。與俊皆忤劉瑾,調南京禮部主事。瑾誅,復故官。屢遷南京國子祭酒。武宗以巡幸至,率諸生請幸學,不從。江彬矯旨取玉硯,偉曰:「有秀才時故硯,可持去。」俊罷官之歲,偉亦至吏部右侍郎,偕廷臣數爭「大禮」,又伏闕力爭。及席書、張璁等議行,猶持前說不變。轉官左侍郎,為陳水光劾罷,卒於家。

吳一鵬,字南夫,長洲人。弘治六年進士。遷庶吉士,授編修。戶部尚書周經以讒去,上疏乞留之。正德初,進侍講,充經筵講官。劉瑾出諸翰林為部曹,一鵬得南京刑部員外郎。遷禮部郎中。瑾誅,復為侍講。進侍講學士,歷國子祭酒、太常卿。並在南京。母喪除,起故官。

世宗踐阼,召拜禮部右侍郎。尋轉左。數與尚書毛澄、汪俊力爭「大禮」。俊去國,一鵬署部事,而帝趣建獻帝廟甚亟。一鵬集廷臣上議曰:「前世入繼之君,間有為本生立廟園陵及京師者。第歲時遣官致祀,尋亦奏罷。然猶見非當時,取議後代。若立廟大內而親享之,從古以來未有也。臣等寧得罪陛下,不欲陛下失禮於天下後世。今張璁、桂萼之言曰『繼統公,立後私』。又曰『統為重,嗣為輕』。竊惟正統所傳之謂宗,故立宗所以繼統,立嗣所以承宗,統之與宗初無輕重。況當我朝傳子之世,而欲仿堯、舜傳賢之例,擬非其倫。又謂『孝不在皇不皇,惟在考不考』,遂欲改稱孝宗為『皇伯考』。臣等歷稽前古,未有神主稱『皇伯考』者。惟天子稱諸王曰『伯叔父』則有之,非可加於宗廟也。前此稱本生皇考,實裁自聖心。乃謂臣等留一皇字以覘陛下,又謂『百皇字不足當父子之名』,何肆言無忌至此。乞速罷建室之議,立廟安陸,下璁、萼等法司按治。」帝報曰:「朕起親籓,奉宗祀豈敢違越。但本生皇考寢園,遠在安陸,於卿等安乎?命下再四,爾等欺朕沖歲,黨同執違,敗父子之情,傷君臣之義。往且勿問,其奉先殿西室亟修葺,盡朕歲時追遠之情。」時嘉靖三年四月也。

頃之,一鵬極陳四方災異,言:「自去年六月迄今二月,其間天鳴者三,地震者三十八,秋冬雷電雨雹十八,暴風、白氣、地裂、山崩、產妖各一,民饑相食二。非常之變,倍於往時。願陛下率先群工,救疾苦,罷營繕,信大臣,納忠諫,用回天意。」帝優詔報之。踰月,手敕名奉先殿西室為觀德殿,遂命一鵬偕中官賴義、京山侯崔元迎獻帝神主於安陸。一鵬等復上言:「歷考前史,並無自寢園迎主入大內者。此天下後世觀瞻所係,非細故也。且安陸為恭穆啟封之疆,神靈所戀,又陛下龍興之地,王氣所鐘。故我太祖重中都,太宗重留都,皆以王業所基,永修世祀。伏乞陛下俯納群言,改題神主,奉安故宮,為百世不遷。其觀德殿中別設神位香几以慰孝思,則本生之情既隆,正統之義亦盡。」奏入,不納。一鵬乃行。慮使者為道途患,疏請禁約,帝善其言而戒飭之。

比還朝,則廷臣已伏闕哭爭,朝事大變,而給事中陳水光言壽張尤甚。一鵬抗疏曰:「大禮之議斷自聖心,正統本生,昭然不紊。而水光妄謂陛下誕生於孝宗沒後三年,嗣位於武宗沒後二月,無從授受,其說尤為不經。謹按《春秋》以受命為正始,故魯隱公上無所承,內無所受,則不書即位。今陛下承武宗之遺詔,奉昭聖之懿旨,正合《春秋》之義。而水光謂孰從授受,是以陛下為不得正始也。洸本小人,不痛加懲艾,無以杜效尤之漸。」不聽。

其年九月,一鵬以本官入內閣專典誥敕兼掌詹事府事。《武宗實錄》成,進尚書,領職如故。尋以省墓歸,還朝仍典誥敕。未幾,出理部事。前此典內閣誥敕者,皆需次柄政。而張璁、桂萼新用事,素銜一鵬異己,乃出為南京吏部尚書,加太子少保。居二年,南京官劾諸大臣王瓊等不職,一鵬與焉,遂乞致仕。給廩如故事。卒贈太子太保,謚文端。子子孝,湖廣參政。

朱希周,字懋忠,崑山人,徙吳縣。高祖吉,戶科給事中。父文雲,按察副使。希周舉弘治九年進士。孝宗喜其姓名,擢為第一。授修撰,進侍講,充經筵講官。劉瑾摘修《會典》小疵,降修撰。《孝宗實錄》成,復官。久之,進侍讀學士,擢南京吏部右侍郎。閱五年,召為禮部右侍郎。

時方議「大禮」,數偕其長爭執。會左侍郎吳一鵬奉使安陸,尚書席書未至,希周獨理部事。而帝方營觀德殿,令協律郎崔元初習樂舞生於大內。太常卿汪舉劾之。帝遂令太常官一人同入內教習。希周上言:「太常樂舞有定數,不當更設。」帝不從。舉復爭,帝責其妄議。而是時張璁、桂萼已召至,益交章請去本生之號。帝悅從之,趣禮官具上冊儀。希周率郎中餘才、汪必東等疏諫曰:「陛下考孝宗、母昭聖三年矣,而更定之論忽從中出,則明詔為虛文,不足信天下,祭告為瀆禮,何以感神祇。且本生非貶詞也,不妨正統,而親之義寓焉。何嫌於此,而必欲去之,以滋天下之議。」時群臣諫者甚眾,疏皆留中,遂相率詣左順門跪伏。希周走告諸閣臣曰:「群臣伏闕,公等能坐視乎?」亦偕群臣跪伏以請。帝聞,大怒,命希周與何孟春等俱待罪,而盡繫庶僚於詔獄。明日,上章聖皇太后冊文,希周及尚書秦金、金獻民、趙鑑、趙璜,侍郎何孟春,都御史王時中,大理少卿張縉、徐文華俱不赴。帝怒,責陳狀。希周等伏罪,復嚴旨譙責乃已。而是時庶僚繫獄者猶未釋,希周上言:「諸臣狂率,固不可宥。但今獻皇帝神主將至,必百官齋迎,乃克成禮。乞早寬縲絏,用襄大典。」不納。「大禮」遂自此定矣。

其明年,由左侍郎遷南京吏部尚書。嘉靖六年,大計京官,南六科無黜者。桂萼素以議禮嗛希周,且惡兩京言官嘗劾己,因言希周畏勢曲庇。希周言:「南京六科止七人,實無可去者。臣以言路私之固不可,如避言路嫌,誅責之,尤不可。且使舉曹皆賢,必去一二人示公,設舉曹皆不肖,亦但去一二人塞責乎?」因力稱疾乞休。溫旨許之,仍敕有司歲給夫廩。

林居三十年,中外論薦者三十餘疏,竟不復起。性恭謹,不妄取予。卒年八十有四。贈太子少保。瀕歿,屬諸子曰:「他日倘蒙易名典,勿犯我家諱。」乃避「文」,謚恭靖。

何孟春,字子元,郴州人。祖俊,雲南按察司僉事。父說,刑部郎中。孟春少遊李東陽之門,學問該博。第弘治六年進士,授兵部主事。言官龐泮等下獄,疏救之。詔修萬歲山毓秀亭、乾清宮西室,役軍九千人,計費百餘萬。抗疏極諫。清寧宮災,陳八事,疏萬餘言。進員外郎、郎中,出理陜西馬政,條目畢張。還,上釐弊五事,並劾撫臣不職。正德初,請釐正孔廟祀典,不果行。出為河南參政,廉公有威。擢太僕少卿,進為卿。駕幸宣府,馳疏諫。尋以右副都御史巡撫雲南。討平十八寨叛蠻阿勿、阿寺等,奏設永昌府,增五長官司、五守禦所。錄功,廕一子,辭不受。

世宗即位,遷南京兵部右侍郎,半道召為吏部右侍郎。會蘇、松諸府旱潦相繼,而江、淮北河水大溢,漂沒田廬人畜無算。孟春仿漢魏相條奏八事,帝嘉納焉。尋進左侍郎。尚書喬宇罷,代署部事。

先是,「大禮」議起。孟春在雲南聞之,上疏言:

臣閱邸報,見進士屈儒奏中請尊聖父為「皇叔考興獻大王」,聖母為「皇叔母興獻大王妃」。得旨下部,知猶未奉俞命也。

臣惟前世帝王,自旁支入奉大統,推尊本生,得失之迹具載史冊。宣帝不敢加號於史皇孫,光武不敢加號於南頓君,晉元帝不敢加號於恭王,抑情守禮。宋司馬光所謂當時歸美,後世頌聖者也。哀、安、桓、靈乃追尊其父祖,犯義侵禮。司馬光所謂取譏當時,見非後世者也。《儀禮·喪服》:「為人後者」《傳》曰:「何以三年也?受重者,必以尊服服之」。「為人後者,為其父母報」,傳曰:「何以期也?不二斬也」,「重大宗者,降其小宗也」。夫父母,天下莫隆焉。至繼大宗則殺其服,而移於所後之親,蓋名之不可以二也。為人後者為之子,不敢復顧私親。聖人制禮,尊無二上,若恭敬之心分於彼,則不得專於此故也。

今者廷臣詳議,事獄未決,豈非皇叔考之稱有未當者乎?抑臣愚亦不能無疑。《禮》,生曰「父母」,死曰「考妣」,有「世父母」、「叔父母」之文,而無世叔考、世叔妣之說。今欲稱興獻王為皇叔考,古典何據?宋英宗時有請加濮王皇伯考者,宋敏求力斥其謬。然則皇叔考之稱,豈可加於興獻王乎?即稱皇叔父,於義亦未安也。經書稱伯父、叔父皆生時相呼,及其既歿,從無通親屬冠於爵位之上者。然則皇叔父之稱,其可復加先朝已謚之親王乎?臣伏睹前詔,陛下稱先皇帝為皇兄,誠於獻王稱皇叔,如宋王珪、司馬光所云,亦已愜矣。而議者或不然,何也?天下者,太祖之天下也。自太祖傳至孝宗,孝宗傳之先皇帝,特簡陛下,授之大業。獻王雖陛下天性至親,然而所以光臨九重,富有四海,子子孫孫萬世南面者,皆先皇帝之德,孝宗之所貽也。臣故願以漢宣、光武、晉元三帝為法,若非古之名,不正之號,非臣所願於陛下也。

及孟春官吏部,則已尊本生父母為「興獻帝」、「興國太后」。繼又改稱「本生皇考恭穆獻皇帝」、「本生聖母章聖皇太后」。孟春三上疏乞從初詔,皆不省。於是帝益入張璁、桂萼等言,復欲去本生二字。璁方盛氣,列上禮官欺妄十三事,且斥為朋黨。孟春偕九卿秦金等具疏,略曰:「伊尹謂『有言逆於心,必求諸道。有言遜於志,必求諸非道』。邇者,大禮之議,邪正不同。若諸臣匡拂,累千萬言,此所謂逆於心之言也,陛下亦嘗求諸道否乎?一二小人,敢託將順之說,招徠罷閒不學無恥之徒,熒惑聖聽,此所謂遜於志之言也,陛下亦嘗求諸非道否乎?何彼言之易行,而此言之難入也。」遂發十三難以辨折璁,疏入留中。

其時詹事、翰林、給事、御史及六部諸司、大理、行人諸臣各具疏爭,並留中不下,群情益洶洶。會朝方罷,孟春倡言於眾曰:「憲宗朝,百官哭文華門,爭慈懿皇太后葬禮,憲宗從之,此國朝故事也。」修撰楊慎曰:「國家養士百五十年,仗節死義,正在今日。」編修王元正、給事中張翀等遂遮留群臣於金水橋南,謂今日有不力爭者,必共擊之。孟春、金獻民、徐文華復相號召。於是九卿則尚書獻民及秦金、趙鑑、趙璜、俞琳、侍郎孟春及朱希周、劉玉,都御史王時中、張潤,寺卿汪舉、潘希曾、張九敘、吳祺,通政張瓚、陳霑,少卿徐文華及張縉、蘇民、金瓚,府丞張仲賢,通政參議葛禬,寺丞袁宗儒,凡二十有三人;翰林則掌詹事府侍郎賈詠,學士豐熙,侍講張璧,修撰舒芬、楊維聰、姚淶、張衍慶,編修許成名、劉棟、張潮、崔桐、葉桂章、王三錫、餘承勛、陸釴、王相、應良、王思,檢討金皋、林時及慎、元正,凡二十有二人;給事中則張翀、劉濟、安磐、張漢卿、張原、謝蕡、毛玉、曹懷、張嵩、王瑄、張、鄭一鵬、黃重、李錫、趙漢、陳時明、鄭自璧、裴紹宗、韓楷、黃臣、胡納,凡二十有一人;御史則王時柯、餘翱、葉奇、鄭本公、楊樞、劉潁、祁杲、杜民表、楊瑞、張英、劉謙亨、許中、陳克宅、譚纘、劉翀、張錄、郭希愈、蕭一中、張恂、倪宗枿、王璜、沈教、鐘卿密、胡瓊、張濂、何鰲、張曰韜、藍田、張鵬翰、林有孚,凡三十人;諸司郎官,吏部則郎中餘寬、黨承志、劉天民,員外郎馬理、徐一鳴、劉勛,主事應大猷、李舜臣、馬冕、彭澤、張鵾,司務洪伊,凡十有二人;戶部則郎中黃待顯、唐昇、賈繼之、楊易、楊淮、胡宗明、栗登、黨以平、何巖、馬朝卿,員外郎申良、鄭漳、顧可久、婁志德,主事徐嵩、張庠、高奎、安璽、王尚志、朱藻、黃一道、陳儒、陳騰鸞、高登、程旦、尹嗣忠、郭日休、李錄、周詔、戴亢、繆宗周、邱其仁、俎琚、張希尹,司務金中夫,檢校丁律,凡三十有六人;禮部則郎中餘才、汪必東、張、張懷,員外郎翁磐、李文中、張澯,主事張鏜、豐坊、仵瑜、丁汝夔、臧應奎,凡十有二人;兵部則郎中陶滋、賀縉、姚汝皋、劉淑相、萬潮。員外郎劉漳、楊儀、王德明,主事汪溱、黃嘉賓、李春芳、盧襄、華鑰、鄭曉、劉一正、郭持平、餘禎、陳賞,司務李可登、劉從學,凡二十人;刑部則郎中相世芳、張峨、詹潮、胡璉、范錄、陳力、張大輪、葉應驄、白轍、許路,員外郎戴欽、張儉、劉士奇,主事祁敕、趙廷松、熊宇、何鰲、楊濂、劉仕、蕭樟、顧鐸、王國光、汪嘉會、殷承敘、陸銓、錢鐸、方一蘭,凡二十有七人;工部則郎中趙儒、葉寬、張子衷、汪登、劉璣、江珊,員外郎金廷瑞、范金、龐淳,主事伍餘福、張鳳來、張羽、車純、蔣珙、鄭騮,凡十有五人;大理之屬則寺正母德純、蔣同仁,寺副王、劉道,評事陳大綱、鐘雲瑞、王光濟、張徽、王天民、鄭重、杜鸞,凡十有一人。俱跪伏左順門。帝命司禮中官諭退,眾皆曰:「必得俞旨乃敢退。」自辰至午,凡再傳諭,猶跪伏不起。

帝大怒,遣錦衣先執為首者。於是豐熙、張翀、餘翱、餘寬、黃待顯、陶滋、相世芳、母德純八人,並繫詔獄。楊慎、王元正乃撼門大哭,眾皆哭,聲震闕廷。帝益怒,命收繫五品以下官若干人,而令孟春等待罪。翼日,編修王相等十八人俱杖死,熙等及慎、元正俱謫戍,始下孟春等前疏,責曰:「朕嗣承大統,祗奉宗廟,尊崇大禮,自出朕心。孟春等毀君害政,變亂是非。且張璁等所上十三條尚留中未發,安得先知?其以實對。」於是孟春等具疏伏罪,言:「璁等所條者,於未進之日先以私稿示人,且有副本存通政司,故臣等知之。臣等忝從大臣後,得與議禮之末。竊以璁等欺罔,故昌言論辨,以瀆天聽,罪應萬死。惟望聖明加察,辨其孰正孰邪,則臣等雖死亦幸。」帝怒不已,責孟春倡眾逞忿,非大臣事君之道,法宜重治,姑從輕奪俸一月。旋出為南京工部左侍郎。故事,南部止侍郎一人,時已有右侍郎張琮,復以孟春為左,蓋賸員也。

孟春屢疏引疾,至六年春始得請。及《明倫大典》成,削其籍。久之,卒於家。隆慶初,贈禮部尚書,謚文簡。孟春所居有泉,用燕去來時盈涸得名,遂稱「燕泉先生」云。

豐熙,字原學,鄞人,布政司慶孫也。幼有異稟。嘗大書壁間曰:「立志當以聖人為的。遜第一等事於人,非夫也。」年十六喪母,水漿不入口數日,居倚廬三年。弘治十二年舉殿試第二。孝宗奇其策,賜第一人袍帶寵之。授編修,進侍講,遷右諭德。以不附劉瑾,出掌南京翰林院事。父喪闋,起故官。

世宗即位,進翰林學士。興獻王「大禮」議起,熙偕禮官數力爭。及召張璁、桂萼為學士,方獻夫為侍讀學士,熙昌言於朝曰:「此冷褒、段猶流也,吾輩可與並列耶?」抗疏請歸,不允。既而尊稱禮定,卜日上恭穆獻皇帝謚冊。熙等疏諫曰:「大禮之議頒天下三年矣,乃以一二人妄言,欲去本生之稱,專隆鞠育之報。臣等聞命,驚惶罔知攸措。竊惟陛下為宗廟神人之主,必宗廟之禮加隆,斯繼統之義不失。若乖先王之禮,貽後世之譏,豈不重累聖德哉。」不得命,相率伏哭左順門。遂下詔獄掠治,復杖之闕廷,遣戍。熙得福建鎮海衛。

既璁等得志,乃相率請釋謫戍諸臣罪,皆首及熙,帝不聽。最後謹身殿災,熙年且七十,給事中田濡復請矜宥,卒不聽。居十有三年,竟卒於戍所。隆慶初,贈官賜恤。

子坊,字存禮。舉鄉試第一。嘉靖二年成進士。出為南京吏部考功主事。尋謫通州同知。免歸。坊博學工文,兼通書法,而性狂誕。熙既卒,家居貧乏,思效張璁、夏言片言取通顯。十七年詣闕上書,言建明堂事,又言宜加獻皇帝廟號稱宗,從配上帝,世宗大悅。未幾,進號睿宗,配饗玄極殿。其議蓋自坊始,人咸惡坊畔父云。明年復進《卿雲雅詩》一章,詔付史館。待命久之,竟無所進擢,歸家悒悒以卒。晚歲改名道生。別為《十三經訓詁》,類多穿鑿語。或謂世所傳《子貢詩傳》,亦坊偽纂也。

徐文華,字用光,嘉定州人。正德三年進士。授大理評事。擢監察御史,巡按貴州。乖西苗阿雜等倡亂,偕巡撫魏英討之,破寨六百三十。璽書獎勞。

江西副使胡世寧坐論寧王宸濠繫詔獄,文華抗疏救曰:「世寧上為聖朝,下為宗室,竭誠發憤,言甫脫口,而禍患隨之,亦可哀也。寧王威焰日以張,隱患日以甚。失今不戢,容有紀極。顧又置世寧重法,杜天下之口,奪忠鯁之氣,弱朝廷之勢,啟宗籓之心,招意外之變,皆自今日始矣。」不納。

帝遣中官劉允迎佛烏斯藏,文華力諫。不報。馬昂納妊身女弟於帝,又疏諫曰:「中人之家不取再醮之婦。陛下萬乘至尊,乃有此舉,返之於心則不安,宣之於口則不順,傳之天下後世則可醜。誰為陛下進此者,罪可族也。萬一防閑闊略,不幸有李園、呂不韋之徒乘間投隙,豈細故哉。今昂兄弟子姪出入禁闥,陛下降絀等威,與之亂服雜坐,或同臥起,壞祖宗法,莫此為甚。馬姬專寵於內,昂等弄權於外,禍機竊發,有不可勝言者。乞早誅以絕禍源。」亦不報。文華既數進直言,帝及諸近倖皆銜之。會文華條上宗廟禮儀,祧廟、禘祫、特享、出主、祔食,凡五事。考證經義,悉可施行。帝怒,責其出位妄言,章下所司。禮官闇於經術,又阿帝意,遂奏文華言非是。命下詔獄,黜為民。時正德十一年十月也。

世宗即位,起故官,歷河南按察副使。嘉靖二年舉治行卓異,入為大理右少卿,尋轉左。時方議興獻帝「大禮」,文華數偕諸大臣力爭。明年七月復倡廷臣伏闕哭諫,坐停俸四月。已,席書、張璁、、桂萼、方獻夫會廷臣大議,文華與汪偉、鄭岳猶力爭。武定侯郭勛遽曰:「祖訓如是,古禮如是,璁等言當。書曰大臣事君,當將順其美。」議乃定。及改題廟主,文華諫曰:「孝宗有祖道焉,不可以伯考稱。武宗有父道焉,不可以兄稱。不若直稱曰『孝宗敬皇帝』、『武宗毅皇帝』,猶兩全無害也。」疏入,命再奪俸。

六年秋,李福達獄起。主獄者璁、萼、獻夫,以議禮故憾文華等,乃盡反獄詞,下文華與諸法官獄。獄具,責文華阿附御史殺人,遣戍遼陽。遇赦,卒於道。隆慶初,贈左僉都御史。

自大學士毛紀、侍郎何孟春去位,諸大臣前爭「大禮」者,或依違順旨,文華顧堅守前議不變。其被譴不以罪,士論深惜之。

薛蕙,字君采,亳州人。年十二能詩。舉正德九年進士,授刑部主事。諫武宗南巡,受杖奪俸。旋引疾歸。起故官,改吏部,歷考功郎中。

嘉靖二年,廷臣數爭「大禮」,與張璁、桂萼等相持不下。蕙撰《為人後解》、《為人後辨》及辨璁、萼所論七事,合數萬言上於朝。《解》有上下二篇,推明大宗義。其《辨》曰:

陛下繼祖體而承嫡統,合於為人後之義,坦然無疑。乃有二三臣者,詭經畔禮,上惑聖聰。夫經傳纖悉之指,彼未能睹其十一,遽欲恃小慧,騁夸詞,可謂不知而作者也。

其曰「陛下為獻帝不可奪之適嗣。」按漢《石渠議》曰:「大宗無後,族無庶子,己有一適子,當絕父嗣以後大宗否?」戴聖云:「大宗不可絕。《禮》言適子不為後者,不得先庶子耳。族無庶子,則當絕父以後大宗。」晉范汪曰:「廢小宗,昭穆不亂。廢大宗,昭穆亂矣。先王所以重大宗也。豈得不廢小宗以繼大宗乎?」夫人子雖有適庶,其親親之心一也。而《禮》適子不為後,庶子得為後者,此非親其父母有厚薄也,直繫於傳重收族不同耳。今之言者不知推本祖禰,惟及其父母而止,此弗忍薄其親,忍遺其祖也。

其曰「為人後者為之子,乃漢儒邪說」。按此踵歐陽修之謬也。夫「為人後者為之子」,其言出於《公羊》,固漢儒所傳者。然於《儀禮》實相表裏,古今以為折衷,未有異論者也。藉若修之說,其悖禮甚矣。《禮》「為人後者,斬衰三年」,此子於父母之喪也。以其父母之喪服之,非為之子而何?其言之悖禮一也。傳言「為所後者之祖父母妻,妻之父母昆弟,昆弟之子若子」。其若子者,由為之子故耳。傳明言「若子」,今顧曰「不為之子」,其言之悖禮二也。且為人後者不為之子,然則稱謂之間,將不曰父,而仍曰伯父、叔父乎?其言之悖禮三也。又立後而不為之子,則古立後者,皆未嘗實子之,而姑偽立是人也。是聖人偽教人以立後,而實則無後焉耳。其言之悖禮四也。夫無後者,重絕祖考之祀,故立後以奉之。今所後既不得而子,則祖考亦不得而孫矣。豈可以入其廟而奉其祀乎?其言之悖禮五也。由此觀之,名漢臣以邪說,無乃其自名耶?抑二三臣者亦自度其說之必窮也,於是又為遁辭以倡之曰:「夫統與嗣不同,陛下之繼二宗,當繼統而不繼嗣。」此一言者,將欲以廢先王為人後之義與?則尤悖禮之甚者也。然其牽合附會,眩於名實,茍不辨而絕之,殆將為後世禍矣。

夫《禮》為大宗立後者,重其統也。重其統不可絕,乃為之立後。至於小宗不為之後者,統可以絕,則嗣可以不繼也。是則以繼統故繼嗣,繼嗣所以繼統也。故《禮》「為人後」,言繼嗣也;「後大宗」,言繼統也。統與嗣,非有二也,其何不同之有?自古帝王入繼者,必明為人後之義,而後可以繼統。蓋不為後則不成子也。若不成子,夫安所得統而繼之。故為後也者,成子也,成子而後繼統,又將以絕同宗覬覦之心焉。聖人之制禮也,不亦善乎。抑成子而後繼統,非獨為人後者爾也。《禮》無生而貴者。雖天子諸侯之子,茍不受命於君父,亦不敢自成尊也。《春秋》重授受之義,以為為子受之父,為臣受之君。故穀梁子曰「臣子必受君父之命」。斯義也,非直尊君父也,亦所以自尊焉耳。蓋尊其君父,亦將使人之尊己也。如此則義禮明而禍亂亡。今說者謂『倫序當立斯立已』,是惡知《禮》與《春秋》之意哉!

若夫前代之君,間有弟終而兄繼,侄終而伯叔父繼者,此遭變不正者也。然多先君之嗣。先君於己則考也,己於先君則子也。故不可考後君,而亦無兩統二父之嫌,若晉之哀帝、唐之宣宗是也。其或諸王入嗣,則未有仍考諸王而不考天子者也。陛下天倫不先於武宗,正統不自於獻帝,是非予奪,至為易辨。而二三臣者猥欲比於遭變不正之舉,故曰悖禮之尤者也。

其他所辨七事,亦率仿此。

書奏,天子大怒,下鎮撫司考訊。已,貰出之,奪俸三月。會給事中陳洸外轉,疑事由文選郎夏良勝及蕙。良勝已被訐見斥,而蕙故在。時亳州知州顏木方坐罪,乃誣蕙與木同年相關通,疑有奸利。章下所司,蕙亦奏辨。帝不聽,令解任聽勘。蕙遂南歸。既而事白,吏部數移文促蕙起。蕙見璁、萼等用事,堅臥不肯起。十八年詔選宮僚,擬蕙春坊司直兼翰林檢討。帝猶以前憾故,報罷。而蕙亦卒矣。

蕙貌臒氣清,持己峻潔,於書無所不讀。學者重其學行,稱為「西原先生」。

當是時,廷臣力持「大禮」,而璁、萼建異議,舉朝非之。其不獲與廷議,而以璁、萼得罪者,又有胡侍、王祿、侯廷訓云。

胡侍,寧夏人。舉進士。歷官鴻臚少卿。張璁、桂萼既擢學士,侍劾二人越禮背經。因據所奏,反復論辨,凡千餘言。帝怒,命逮治。言官論救,謫潞州同知。沈府宗室勛注以事憾之,奏侍試諸生題譏刺,且謗「大禮」。逮至京,訊斥為民。

王祿,新城人。舉於鄉,為福建平和知縣。嘉靖九年,疏請建獻帝廟於安陸,封崇仁王以主其祀,不當考獻帝,伯孝宗,涉二本之嫌。宗籓子有幼而岐嶷者,當養之宮中,備儲貳選。疏奏,即棄官歸。命按臣逮治,亦斥為民。

侯廷訓,樂清人。與張璁同郡,同舉進士,而持論不合。初釋褐,即上疏請考孝宗,且言不當私籓邸舊臣,語最切直。除南京禮部主事。嘉靖三年冬,「大禮」定,廷訓心非之。私刊所著議禮書,潛寄京師,下詔獄拷訊。子一元,年十三,伏闕訟冤,得釋。後起官至漳南僉事。以貪虐,被劾為民。一元舉進士,官至江西布政使。

贊曰:「大禮」之議,楊廷和為之倡,舉朝翕然同聲,大抵本宋司馬光、程頤《濮園議》。然英宗長育宮中,名稱素定。而世宗奉詔嗣位,承武宗後,事勢各殊。諸臣徒見先賢大儒成說可據,求無得罪天下後世,而未暇為世宗熟計審處,準酌情理,以求至當。爭之愈力,失之愈深,惜夫。


\end{pinyinscope}