\article{附錄:張廷玉上明史表}

\begin{pinyinscope}
經筵日講官太保兼太子太保保和殿大學士兼管吏部尚書翰林院掌院學士事世襲三等伯臣張廷玉等上言:

臣等奉敕纂修《明史》告竣,恭呈睿鑒,臣等謹奉表恭進者。伏以瑤圖應運,丹綸繙竹素之遺;雒鼎凝庥,玉局理汗青之業。集百年之定論,裒一代之舊聞,歷纂輯於興朝,畢校紬於茲日。垂光冊府,煥採書林。竊惟論道首在尊經,紀事必歸攬史。興衰有自,七十二君之跡何稱;法戒攸關,《二十一史》之編具在。繼咸五登三之治,心源不隔於邃初;開萬方一統之模,典制必參諸近世。況乎歲時綿歷,載籍叢殘。執簡相先,合眾長而始定;含毫能斷,昭公道以無私。考獻徵文,用備酉山之秘;屬辭比事,上塵乙夜之觀。欽惟皇帝陛下,乘六御天,奉三出治。紹庭建極,綏蕩平正直之猷;典學傅心,綜忠敬質文之統。觀人文以化天下,鑒物惟公;考禮樂以等百王,折衷必當。

惟茲《明史》,職在儒臣。紀統二百餘年,傳世十有六帝。創業守成之略,卓乎可觀;典章文物之規,燦然大備。迨乎繼世,法弗飭於廟堂;降及末流,權或移於閹寺。無治人以行治法,既外釁而內訌;因災氛以啟寇氛,亦文衰而武弊。朝綱不振,天眷既有所歸;賊焰方張,明祚遂終其運。我國家丕承景命,肇建隆基,天戈指而掃欃槍,《王會》圖而陳玉帛。滌中原寇盜之孽,奠我民生;慰前朝諸帝之心,雪其國恥。迄今通侯備恪,俎豆相承;依然守戶衛陵,松楸勿翦。是則擴隆恩於覆載,既極優崇;因之徵故籍於《春秋》,絕無忌諱。

第以長編汗漫,抑且雜記舛訛。靖難從亡,傳聞互異;追尊議禮,聚訟紛拏。降及國本之危疑,釀為《要典》之決裂。兵符四出,功罪難明;黨論相尋,貞邪易貿。稗官野錄,大都荒誕無稽;家傳碑銘,亦復浮誇失實。欲以信今而傳後,允資博考而旁參。仰惟聖祖仁皇帝搜圖書於金石,羅耆俊於山林。創事編摩,寬其歲月。我世宗憲皇帝重申公慎之旨,載詳討論之功。

巨等於時奉敕充總裁官,率同纂修諸臣開館排緝。聚官私之紀載,核新舊之見聞。簽帙雖多,牴牾互見。惟舊臣王鴻緒之《史稿》,經名人三十載之用心。進在彤闈,頒來秘閣。首尾略具,事實頗詳。在昔《漢書》取裁於馬遷,《唐書》起本於劉昫。茍是非之不謬,詎因襲之為嫌。爰即成編,用為初稿。發凡起例,首尚謹嚴;據事直書,要歸忠厚。曰紀,曰志,曰表,曰傳,悉仍前史之體裁;或詳,或略,或合,或分,務核當時之心跡。文期共喻,掃艱深鄙穢之言;事必可稽,黜荒誕奇邪之說。十有五年之內,幾經同事遷流;三百餘卷之書,以次隨時告竣。勝國君臣之靈爽,實式憑之;累朝興替之事端,庶幾備矣。

臣等才謝宏通,學慚淹貫。幸際右文之代,獲尚論於先民;敢云稽古之勤,遠希風於作者。恭蒙睿鑒,俾授梓人。伏願金鏡高懸,璇樞廣運。參觀往跡,考證得失之源;懋建鴻猷,昭示張弛之度。無怠無荒而熙庶績,化阜虞紘;克寬克仁而信兆民,時存殷鑒。則冠百王而首出,因革可徵百世之常;邁千祀以前驅,政教遠追千古而上矣。謹將纂成本紀二十四卷,志七十五卷,表十三卷,列傳二百二十卷,目錄四卷,共三百三十六卷,刊刻告成,裝成一十二函,謹奉表隨進以聞。

乾隆四年七月二十五日

經筵日講官太保兼太子太保保和殿大學士兼管吏部尚書翰林院掌院學士事世襲三等伯臣張廷玉太子少保食尚書俸臣徐元夢戶部右侍郎加五級臣留保


\end{pinyinscope}