\article{曆志}

後世法勝於古,而屢改益密者,惟曆為最著。《唐志》謂天為動物,久則差忒,不得不屢變其法以求之。此說似矣,而不然也。《易》曰:「天地之道,貞觀者也。」蓋天行至健,確然有常,本無古今之異。其歲差盈縮遲疾諸行,古無今有者,因其數甚微,積久始著。古人不覺,而後人知之,而非天行之忒也。使天果久動而差忒,則必差參凌替而無典耍,安從修改而使之益密哉?觀傳志所書,歲失其次、日度失行之事,不見於近代,亦可見矣。夫天之行度多端,而人之智力有限,持尋尺之儀表,仰測穹蒼,安能洞悉無遺。惟合古今人心思,踵事增修,庶幾符合。故不能為一成不易之法也。

黃帝迄秦,曆凡六改。漢凡四改。魏迄隋,十五改。唐迄五代,十五改。宋十七改。金迄元,五改。惟明之《大統曆》,實即元之《授時》,承用二百七十餘年,未嘗改憲。成化以後,交食往往不驗,議改曆者紛紛。如俞正己、冷守中不知妄作者無論已,而華湘、周濂、李之藻、刑云路之倫頗有所見。鄭世子載堉撰《律曆融通》,進《聖壽萬年曆》,其說本之南部御史何瑭,深得《授時》之意,而能補其不逮。臺官泥於舊聞,當事憚於改作,並格而不行。崇禎中,議用西洋新法,命閣臣徐光啟、光祿卿李天經先後董其事,成《曆書》一百三十餘卷,多發古人所未發。時布衣魏文魁上疏排之,詔立兩愕扒驗。累年校測,新法獨密,然亦未及頒行。由是觀之,曆固未有行之久而差者,烏可不隨時修改,以求合天哉。

今采扣家論說,有裨於曆法者,著於篇端。而《大統曆》則述立法之原,以補《元志》之未備。《回回曆》始終隸於欽天監,與《大統》參用,亦附錄焉。

▲曆法沿革

吳元年十一月乙未冬至,太史院使劉基率其屬高翼上戊申《大統曆》。太祖諭曰:「古者季冬頒曆,太遲。今於冬至,亦未善。宜以十月朔,著為令。」洪武元年改院為司天監,又置回回司天監。詔徵元太史院使張佑、回回司天太監黑的兒等共十四人,尋召回回司天臺官鄭阿里等十一有至京,議曆法。三年改監為欽天,設四科:曰天文,曰漏刻,曰《大統曆》,曰《回回曆》。以監令、少監統之。歲造《大統民曆》、《御覽月令曆》、《七政躔度曆》、《六壬遁甲曆》、《四季天象占驗曆》、《御覽天象錄》,各以時上。其日月交食分秒時刻、起復方位,先期以聞。十年三月,帝與群臣論天與七政之行,皆以蔡氏旋之說對。帝曰:「朕自起以來,仰觀乾象,天左旋,七政右旋,曆家之論,確然不易。爾等猶守蔡氏之說,豈所謂格物致知學乎?」十五年九月,詔翰林李翀、吳伯宗譯《回回曆書》。

十七年閏十月,漏刻博士元統言:「曆以《大統》為名,而積分猶踵《授時》之數,非所以重始敬正也。況《授時》以元辛巳為曆元,至洪武甲子積一百四年,年遠數盈,漸差天度,合修改。七政運行不齊,其理深奧。聞有郭伯玉者,精明九數之理,宜徵令推算,以成一代之制。」報可。擢統為監令。統乃取《授時曆》,去其歲實消長之說,析其條例,得四卷,以洪武十七年甲子為曆元,命曰《大統曆法通軌》。二十二年改監令、丞為監正、副。二十六年,監副李德芳言:「監正統孜作洪武甲子曆元,不用消長之法,以考魯獻公十五年戊寅歲天正冬至,比辛巳為元,差四日半強。今當復用辛巳為元及消長之法。」疏入,元統奏辨。太祖曰:「二說皆難憑,但驗七政交會行度無差者為是。」自是《大統曆》元以洪武甲子,而推算仍依《授時》法。三十一年在罷回回欽天監,其《回回曆》科仍舊。

永樂遷都順天,仍用應天冬夏晝夜時刻,至正統十四年始改用順天之數。其冬,景帝即位,天文生馬軾奏,晝夜時刻不宜改。下廷臣集議。監正許惇等言:「前監正彭德清測驗得北京北極出地四十度,比南京高七度有奇,冬至晝三十八刻,夏至晝六十二刻。奏準改入《大曆》,永為定式。軾言誕妄,不足聽。」帝曰:「太陽出入度數,當用四方之中。今京師在堯幽都之地,寧可為準。此後造曆,仍用洪、永舊制。」

景泰元年正月辛卯,卯正三刻月食。監官誤推辰初初刻,致失救護。下法司,論徒。詔宥之。成化十年,以監官多不職,擢雲南提學童軒為太常寺少卿,掌監事。十五年十一月戊戍望,月食,監推又誤,帝以天象微渺,不之罪也。十七年,真定教論俞正己上《改曆議》,詔禮部及軒參考。尚書周洪謨等言:「正己止據《皇極經世書》及歷代天文、曆志推算氣朔,又以己意創為八十七年約法,每月大小相間。輕率狂妄,宜正其罪。」遂下正己詔獄。十九年,天文生張陞上言改曆。欽天監謂祖制不可變,陞說遂寢。弘治中,月食屢不應,日食亦舛。

正德十二、三年,連推日食起復,皆弗合。於是漏刻博士朱裕上言:「至元辛巳距今二百三十七年,歲久不能無差,若不量加損益,恐愈久愈舛。乞簡大臣總理其事,令本監官生半推古法,半推新法,兩相交驗,回回科推驗西域《九執曆法》。仍遣官至各省,候土圭以測節氣早晚。往復參較,則交食可正,而七政可齊。」部覆言:「裕及監官曆學未必皆精,今十月望月食,中官正周濂等所推算,與古法及裕所奏不同,請至期考驗。」既而濂等言:「日躔歲退之差一分五十秒。今正德乙亥,距至元辛巳二百三十五年,赤道歲差,當退天三度五十二分五十秒。不經改正,推步豈能有合。臣參較德驗,得正德丙子歲前天正冬至氣應二十七日四百七十五分,命得辛卯日丑初初刻,日躔赤道箕宿六度四十七五十秒,黃道箕宿五度九十六分四十三秒為曆元。其氣閏轉交四應,併周天黃赤道,諸類立成,悉從歲差,隨時改正。望敕禮臣併監正董其事。」部奏:「古法未可輕變,請仍舊法。別選精通曆學者,同濂等以新法參驗,更為奏請。」從之。

十五年,禮部員外郎鄭善夫言:「日月交食,日食最為難測。蓋月食分數,但論距交遠近,別無四時加減,且月小闇虛大,八方所見皆同。若日為月所掩,則日大而月小,日上而月下,日遠而月近。日行有四時之異,月行有九道之分。故南北殊觀,時刻亦異。必須據地定表,因時求合。如正德九年八月辛卯日食,曆官報食八分六十七秒,而閩、廣之地,遂至食既。時刻分秒,安得而同?今宜按交食以更曆元,時刻分秒,必使奇零剖析詳盡。不然,積以歲月,躔離朓朒,又不合矣。」不報。十六年以南京戶科給事中樂頀、工部主事華湘通曆法,俱擢光祿少卿,管監事。

嘉靖二年,湘言:「古今善治曆者三家,漢《太初》以鐘律,唐《大衍》以蓍策,元《授時》以晷景為近。欲正曆而不登臺測景,皆空言臆見也。望許臣暫朝參,督中官正周濂等,及冬至前詣觀象臺,晝夜推測,日記月書,至來年冬至,以驗二十四氣、分至合朔、日躔月離、黃赤二道、昏旦中星、七政四餘之度,視元辛巳所測,離合何如,差次錄聞。更敕禮部延訪精通理數者徵赴京師,令詳定歲差,以成一代之制。」下禮部集議,而護謂曆不可改,與湘頗異。禮部言:「湘欲自行測候,不為無識。請二臣各盡所見,窮極異同,以協天道。」從之。

七年,欽天監奏:「閏十月朔,《回回曆》推日食二分四十七秒,《大統曆》推不食。」已而不食。十九年三月癸巳朔,臺官言日當食,已而不食。帝喜,以為天眷,然實由推步之疏也。隆慶三年,掌監事順天府丞周相刊《大統曆法》,其曆原歷敘古今諸曆異同。萬曆十二年十一有癸酉朔《大統曆》推日食九十二秒,《回回曆》推不食,已而《回回曆》驗。禮科給事中侯先春因言:「邇年月食在酉而曰戌,月食將既而曰未九分,差舛甚矣。《回回曆》科推算日月交食,五星凌犯,最為精密,何妨纂人《大統曆》中,以備考驗。」詔可。二十年五月戌夜月食,監官推算差一日。

二十三年,鄭世子載堉進《聖壽萬年曆》、《律曆融通》二書。疏略曰:「高皇帝革命時,元曆未久,氣朔未差,故不改作,但討論潤色而已。積年既久,氣朔漸差。《後漢志》言『三百年斗曆改憲』。今以萬曆為元,而九年辛巳歲適當『斗曆改憲』之期,又協『乾元用九』之義,曆元正在是矣。臣嘗取《大統》與《授時》二曆較之,考古則氣差三日,推今則時差九刻。夫差雖九刻,處夜半之際,所差便隔一日。節氣差天一日,則置閏差一月。閏差一月,則時差一季。時差一季,則歲差一年。其失豈小小哉?蓋因《授時》減分太峻,失之先天;《大統》不減,失之後天。因和會兩家,酌取中數,立為新率,編撰成書,大旨出於許衡,而與曆不同。黃鐘乃律曆本原,而舊曆罕言之。新法則以步律呂爻象為首。堯時冬至日躔宿次,何承天推在須、女十度左右,一行推在女、虛間,元人曆議亦云在女、虛之交。而《授時曆》考之,乃在牛宿二度。《大統曆》考之,乃在危宿一度。相差二十六度,皆不與《堯典》合。新法上考堯元年甲辰歲,夏至午中,日在柳宿十二度左右,冬至午中,日在女宿十度左右,心昴昏中,各去午正不逾半次,與承天、一行二家之說合。此皆與舊曆不同大者,其餘詳見《曆議》。望敕大臣名儒參訂採用。」

其法首曰步發斂。取嘉靖甲寅歲為曆元,元紀四千五百六十,期實千四百六十一,節氣歲差一秒七十五忽,歲周氣策無定率,各隨歲差求而用之。律應即氣應五十五日六十刻八十九分,律總旬周六十日。次曰步朔閏。朔望弦策與《授時》同,閏應十九日三十六刻十九分。次曰步日躔。日平行一度,躔周即天周三百六十五度二十五分,躔中半之,象策又半之,辰策十二分躔周之一。黃、赤道歲差,盈初縮末限,縮初盈末限,俱與《授時》同,周應二百三十八度二十二分三十九秒。按《授時》求日度法,以周應加積度,命起虛七,其周應為自虛七度至箕寸十度之數。《萬年曆法》以周應減積度,命起角初,其周應為箕十度至角初度之數,當為二百八十六度四十五分。今數不合,似誤。次曰步晷漏。北極出地度分,冬、夏至中晷恒數,並二至晝夜長短刻數,俱以京師為準。參以岳臺,以見隨處里差之數。次曰步月離。月平行、轉周、轉中,與《授時》同。離周即遲疾限三百三十六限十六分六十秒,離中半之,離象又半之。轉差一日九十刻六十分。轉應七日五十刻三十四分。次曰步交道。正交、中交與《授時》同。距交十四度六十六分六十六秒。交周、交中、交差,與《授時》同。交應二十日四十七刻三十四分。次曰步交食。日食交外限六度,定法六十一,交內限八度定法八十一。月食限定法與《授時》同。次曰步五緯。合應:土星二百六十二日三千二十六分,木星三百一十一千八百三十七分,火星三百四十三日五千一百七十六分,金星二百三十八千三百四十七分,水星九十一日七千六百二十八分。曆應:土星八千六百四日五千三百三十八分,木星四千一十八日六千七十三分,火星三百一十四日四十九分,金星六十日一千九百七十五分,水星二百五十三日七千四百九十七分。周率、度率及晨夕伏見度,俱與《授時》同。

其議歲餘也,曰:「陰陽消長之理,以漸而積,未有不從秒起。《授時》考古,於百年之際頓加一分,於理未安。假如魯隱公三年酉歲,下距至元辛巳二千年,以《授時》本法算之,於歲實當加二十分,得庚午日六刻,為其年天正冬至。次年壬戌歲,下距至元辛巳一千九百九十年,本法當加十九分,得乙亥日五十刻四十四分,為其年天正冬至。兩冬至相減,得相距三百六十五日四十四刻四十四分,則是歲餘九分日之四,非四分日之一也。曆法之廖,莫甚於此。新法酌量,設若每年增損二秒,推而上之,則失昭公己丑;增損一秒至一秒半,則失僖公辛亥。今約取中數,其法置定距自相乘,七因八歸,所得百,約之為分,得一秒七十五忽,則辛亥、己丑皆得矣。」

其議日躔也,曰:「古曆見於《六經》,灼然可考者莫如日躔及中星。而推步家鮮有達者,蓋由不知夏時、周正之異也。大抵夏曆以節氣為主,周曆以中氣為主。何承天以正月甲子夜半合朔雨水為上元,進乖夏朔,退非周正。故近代推《月令》、《小正》者、皆不與古合。嘗以新法歲差,上考《堯典》中星,則所謂四仲月,蓋自節氣之始至於中氣之終,三十日內之中星耳後世執者於二分二至,是亦誤矣。」

其議侯極也,曰:「自漢至齊、梁,皆謂紐星即不動處。惟祖恆之測知紐星去極一度有餘。自唐至宋,又測紐星去極三度有餘。《元志》從三度,蓋未有說也。新法不測紐星,以日景驗之,於正方案上,周天度內權指一度為北極,自此度右旋,數至六十七度四十一分,為夏至日躔所在。復至一百一十五度二十一分,為冬至日躔所在。左旋,數亦如之。四處并中心五處,各識一鍼。於二至日午中,將案直立向南取景,使三鍼景合,然後縣繩界取中綿,又取方十字界之,視橫界上距極出地度分也,即極出地度分也。」

其議晷景也,曰:「何承天立表測景,始知自漢以來,冬至皆後天三日。然則推步晷景,乃治曆之耍也。《授時曆》亦憑晷景為本,而《曆經》不載推步晷景之術,是為缺略,今用北極出地度數,兼弧矢二術以求之,庶盡其原。又隨地形高下,立差以盡變,前此所未有也。」又曰:「《授時曆》議據《前漢志》魯獻公十五年戊寅歲正月甲寅朔旦冬至,引用為首。夫獻公十五年下距隱公元年己未,歲百六十一年,其非春秋時明矣。而《元志》乃云『自春秋獻公以來』,又云『昭公冬至,乃日度失行之驗』,誤矣。夫獻公甲寅冬至,別無所據,惟劉歆《三統曆》言之。豈左傳不足信,而歆乃可信乎?太初元年冬至在辛酉,歆乃以為甲子,差天三日,尚不能知,而能逆知上下數百年乎?故凡春秋前後千載之間,氣朔交食,《長曆》、《大衍》所推近是,劉歆、班固所說全非也。」又曰:「《大衍曆》議謂宋元嘉十三年一月甲戌,景長為日度變行,《授時曆》議亦云,竊以為過矣。茍日度失行,當如歲差,漸漸而移。今歲既已不合,來歲豈能復合耶?蓋前人所測,或未密耳。夫冬至之景一丈有餘,表高晷長,則景虛而淡,或設望筒、副表、景符之類以求實景。然望筒或一低昂,副表、景符或一前卻,所據之表或稍有傾欹,圭面或稍有斜側,二至前後數日之景,進退只在毫釐之間,耍亦難辨。況委託之人,未智當否。九服之遠,既非自摯,所報晷景,寧足信乎?」

其議漏刻也,曰:「日月帶食出入,五星晨昏伏見,曆家設法悉因晷漏為準。而晷漏則隨地勢南北,辰極高下為異焉。元人都燕,其《授時曆》七曜出沒之早晏,四時晝夜之永短,皆準大都晷漏。國初都金陵,《大統曆》晷漏改徒南京,冬夏至相差三刻有奇。今推交食分秒,南北東西等差及五星定伏定見,皆因元人舊法,而獨改其漏刻,是以互神舛誤也。故新法晷漏,照依元舊。」

其議日食也,曰:「日道與月道相交處有二,若正會於交,則食既,若但在交前後相近者,則食而不既。此天之交限也。又有人之交限,假令中國食既,戴日之下,所虧纔半,化外之地,則交而不食。易地反觀,亦如之。何則?日如大赤丸,月如小黑丸,共縣一綿,日上而下,即其下正望之,黑丸必掩赤丸,似食之既;及旁觀有遠近之差,則食數有多寡矣。春分已後,日行赤道北畔,交外偏多,交內偏少。秋分已後,日行赤道南畔,交外偏少,交內偏多。是故有南北差。冬至已後,日行黃道東畔,午前偏多,午後偏少。夏至已後,日行黃道西畔,午前偏少,午後偏多。是故有東西差。日中仰視則高,旦暮平視則低。是有距午差。食於中前見早,食於中後見遲。是故有時差,凡此諸差,唯日有之,月則無也。故推交食,惟日頗難。欲推九服之變,必各據其處,考晷景之短長,揆辰極之高下,庶幾得之。《曆經》推定之數,徒以燕都所見者言之耳。舊云:『月行內道,食多有驗。月行外道,食多不驗。』又云:『天之交限,雖係內道,若在人之交限之外,類同外道,日亦不食。』此說似矣,而未盡也。假若夏到前後,日食於寅卯酉戌之間,人向東北、西北觀之,則外道食分反多於內道矣。日體大於月,月不能盡掩之,或遇食既,而日光四溢,形如金環,故日無食十分之理。雖既,亦止九分八十秒。《授時曆》日食,陽曆限六度,定法六十,陰曆限八度,定法八十。各置其限度,如其定法而一,皆得十分。今於其定法下,各加一數以除限度,則得九分八十餘秒也。」

其議月食也,曰:「暗虛者,景也。景之蔽月,無早晚高卑之異,四時九服其之殊。譬如縣一黑丸於暗室,其左燃燭,其右縣一白丸,若燭光為黑丸所蔽,則白丸不受其光矣。人在四旁觀之,所見無不同也。故月食無時差之說。自《紀元曆》妄立時差,《授時》因之,誤矣。」

其議五緯也,曰:「古法推步五緯,不如變數之加減。北齊張子信仰觀歲久,知五緯有盈縮之變,當加減以求逐日之躔。蓋五緯出入黃道內外,各自有其道,視日遠近為遲疾,其變數之加減,如里路之徑直斜曲也。宋人有言曰:『五星行度,惟留退之際最多差。自內而進者,其退必向外,自外而進者,其退必由內。其迹臺循柳葉,兩末銳於中間,往還之道相去甚遠。故星行兩末度稍遲,以其斜行故也。中間行度稍速,以其徑捷故也。』前代修曆,止增損舊法而已,未嘗實考天度。其法須測驗每夜昏曉夜半,月及五星所在度秒,置簿錄之。滿五年,其間去陰雲晝見日數外,可行三年實行,然後可以算術綴之也。」

書上,禮部尚范謙奏:「歲差之法,自虞喜以來,代有差法之議,竟無晝一之規。所以求之者,大約有三:考月令之中星,測二至之日景,驗交食之分秒。考以衡管,測以臬表,驗以漏刻,斯亦危得之矣。曆家以周天三百六十五度四分度之一,紀七政之行,又析度為百分,分為百秒,可謂密矣。然渾象之體,徑僅數尺,布周天度,每度不及指許,安所置分秒哉?至於臬表之樹不過數尺,刻漏之籌不越數寸。以天之高且廣也,而以寸之物求之,欲其纖微不爽,不亦難乎?故方其差在公秒之間,無可驗者,至蹬踰一度,乃可以管窺耳。此所以窮古今之智七巧,不能盡其變歟?即如世子言,以《大統》、《授時》二曆相較,考古則氣差三日,推今則時差必刻。夫時差九刻,在亥子之間則移一日,在晦朔之交則移一月,此可驗之於近也。設移而前,則生明在二日之昏,設移而後,則生明在四日之夕矣。今似未至此也。其書應發欽天監參訂測驗。世子留心曆學,博通今古,宜賜獎諭。」從之。

河南僉事刑雲路上書言:「治曆之耍治曆之耍,無踰觀象、測景、候時、籌策四事。今丙申年日至,臣測得乙未日未正一刻,而《大統》推在申正二刻,相差九刻。且今年立春、夏至、立冬皆適直子半之交。臣推立春乙亥,而《大統》推丙子;夏至壬辰,而《大統》推癸巳;立冬巳酉,而《大統》推庚戌。相隔皆一日。若或直元日於子半,則當退履端於月窮,而朝賀大禮在月正二日矣。豈細故耶?閏八月朔,日食,《大統》推初虧巳正二刻,食幾既,而臣候初虧巳正一刻,食止七分餘。《大統》實後天幾二刻,則閏應及轉應、交應,各宜增損之矣。」欽天監見雲路疏,甚惡之。監正張應候奏詆,謂其僭妄惑世。禮部尚書范謙乃言:「曆為國家大事,士夫所當講求,非曆士之所得私。律例所禁,乃妄言妖祥者耳。監官拘守成法,不能修改合天。幸有其人,所當和衷共事,不宜妒忌。乞以雲路提叔欽天監事,督率官屬,精心測候,以成鉅典。」議上,不報。

三十八年,監推十一月壬寅朔日食分秒及虧圓之候,職方郎范守己疏駁其誤。禮官因請博求知曆學者,令與監官晝夜推測,庶幾曆法靡差。於是五官正周子愚言:「大西洋歸化遠臣龐迪峨、熊三撥等,攜有彼國曆法,多中國典籍所未備者。乞視洪中譯西域曆法例,取知曆儒臣率同監官,將諸書盡譯,以補典籍之缺。」先是,大西洋人利瑪竇進貢土物,而迪峨、三撥及能華同、鄧玉函、湯若望等先後至,俱精究天文曆法。禮部因奏:「精通曆法,如雲路、守己為時所推,請改授京卿,共理曆事。翰林院檢討徐光啟、南京工部員外郎李之藻亦皆精心曆理,可與迪峨、三撥等同譯西洋法,俾雲路等參訂修改。然曆法疏密,莫顯於交食,欲議修曆,必重測驗。乞敕所司修治儀器,以便從事。」疏入,留中。未幾雲路、之藻皆召至京,參預曆事。雲路據其所學,之藻則以西法為宗。

四十一年,之藻已改銜南京太僕少卿,奏上西洋曆法,略言臺監推算日月交食時刻虧分之謬。而力薦迪峨、三撥及華民、陽瑪諾等,言:「其所論天文曆數,有中國昔賢所未及者,不徒論其數,又能明其所以然之理。其所製窺天、窺日之器,種種精絕。今迪峨等年齡向衰,乞敕禮部開局,取其曆法,譯出成書。」禮科姚永濟亦以為言。時庶務因循,未暇開局也。

四十四年,雲路獻《七政真數》,言:「步曆之法,必以兩交相對。兩交正,而中間時刻分秒之度數,一一可按。日月之交食,五星之凌犯,皆日月五星之相交也。兩交相對,互相發明,七政之能事畢矣。」天啟元年春,雲路復詳述古今時刻,與欽天監所推互異。癥新法至密,章下禮部。四月壬申朔日食,雲路所推食分時刻,與欽天監所推互異。自言新法至密,至期考驗,皆與天下不合。雲路又嘗論《大統》宮度交界,當以歲差考定,不當仍用《授時》三百年前所測之數。又月建月關半杓所指,斗杓有歲差,而月建無改移。皆篤論也。

崇禎二年五月乙酉朔日食,禮部侍郎徐光啟依西法預推,順天府見食二分有奇,瓊州食既,大寧以北不食。《大統》、《回回》所推,順天食分時刻,與光啟妻異。已而光啟法驗,餘皆疏。帝切責監官。時五官正戈豐年等言:「《大統》乃國初所定,寮即郭守敬《授時曆》也,二百六十年毫未增損。自至元十八年造曆,越十八年為大德三年八月,已當食不食,六年六月又食而失推。是時守敬方知院事,亦付之無可奈佑,況斤斤守法者哉?今若循舊,向後不能無差。」於是禮部奏開局修改。乃以光啟督修曆法。光啟言:近世言曆諸家,大都宗郭守敬法,至若歲差環轉,歲實參差,天有緯度,地有經度,列宿有本行,月五星有本輪,日月有真會、視會,皆古所未聞,惟西曆有之。而舍此數法,則交食凌犯,終無密合理。宜取其法參互考訂,使與《大統》法會同歸一。」

已而光啟上曆法修正十事:其一,議歲差,每歲東行漸長短之數,以正古來百年、五十年、六十年多寡互異之說。其二,議歲實小餘,昔多今少,漸次改易,及日景長短歲歲不同之因,以定冬至,以正氣明朔。其三,每日測驗日行經度,以定盈縮加減真率,東西南北高下之差,以步月離。其四,夜測月行經緯度數,以定交轉遲疾真率,東西北高下之差,以步月離。其五,密測列宿以緯行度,以定七政盈縮、遲疾、順逆、違離、遠近之數。其六,密測五星經緯行度,以定小輪行度遲疾、留逆、伏見之數,東西南北高下之差,以推步凌犯。其七,推變黃道、赤道廣狹度數,密測二道距度,及月五星各道與黃道相距之度,以定交轉。其八,議日月去交遠近及真會、視會之因,以定距午時差之真率,以正交食。其九,測日行,考知二極出入地度數,以定周天緯度,以齊七政。因月食考知東西相距地輪經度,以定交食時刻。其十,依唐、元法,隨地測驗二極出入地度數,地輪經緯,以求晝夜晨昏永短,以正交食有無、先後、多寡之數。因舉南京太僕少卿李之藻、西洋人能華民、鄧玉涵。報可。九月癸卯開曆局。三年,玉函卒,又徵西洋人湯若望、羅雅谷譯書演算。光啟進本部尚書,仍督修曆法。

時巡按四御史馬如蚊薦資縣諸生冷守中精曆學以所呈曆書送局。光啟力駁其謬,並預推次年四月川食時刻,令其臨時比測。四年正月,光啟進《曆書》二十四卷。夏四月戊午,夜望月食,光啟預推分秒時刻方位。奏言:「日食隨地不同,則用地緯度算其食分多少,用地經度算其加時早晏。月食分秒,海內並同,止用地經度推求先後時刻。臣從輿地圖約略推步,開載各布政司月食初虧度分,蓋食分多少既天下皆同,則餘率可以類推,不若日食之經緯各殊,心須詳備也。又月體一十五分,則盡入闇虛亦十五分止耳。今推二十六分六十六十秒者,蓋闇虛體大於月,若食時去交稍遠,即月體不能全入闇虛,止從月體論其分數。是夕之食,極近於交,故月入闇虛十五分方為食既,更進一十一分有奇,乃得生光,故為二十六分有奇。如《回回曆》推十八分四十七秒,略同此法也。」已四川報次序守中所推月食實差二時,而新法密合。

光啟又進《曆書》二十一卷。冬十月辛丑朔日食,新法預順天見食二分一十二秒,應天以南下食,大漢以北食既,例以京師見食不及三分,不救護。光啟言:

月食在夜,加時早晚,若無定據。惟日食按晷定時,無可遷就。故曆法疏密,此為的癥。臣等纂輯新法,漸次就緒,而向生交食為期尚遠,此時不與監臣共見,至成曆後,將何徵信?且是食之必當測俟,更有說焉。

舊法食在正中,則無時差。今此食既在日中,而新法仍有時差者,蓋以七政運行皆依黃道,不由赤道。舊法所謂中乃道之午中,非黃道之正中也。黃赤道二道之中,獨冬夏至加時正午,乃得同度。今十月朔去冬至度數尚遠,兩中之差,二下三度有奇,豈可因加時近午,不加不減乎?適際此日,又值此時,足可驗時差之正術,二也。

本方之地經度,未得真率,則加時難定,其法心從交食時測驗數次,乃可較勘晝一。今此食依新術測候,其加時刻分,或後未合,當取從前所記地經度分,斟酌改定,此可以求里差之真率,二也。

時差一法,但知中無加減,而不知中分黃赤,今一經目見,人人知加時之因黃道,因此推彼,他術皆然,足以知學習之甚易,三也。

即分數甚少,宜詳加測候,以求顯驗。帝是其言。至期,光啟率監臣預點日晷,調壺漏,用測高儀器測食甚日晷高度。又於密室中斜開一隙,置窺筒、遠鏡以測虧圓,晝日體分板分數圖板以定食分,其時刻、高度悉合,惟食甚分數未及二分。於是光啟言:「今食甚之度分密合,則經度里差已無煩更定矣。獨食分未合,原推者蓋因太陽光大,能減月魄,必食及四五分以上,乃得與原推相合,然此測,用密室窺筒,故能得此分數,倘止憑目力,或水盆照映,則眩耀不定,恐少尚不止此也。」

時有滿城布衣魏文魁,著《曆元》、《曆測》二書,令其子象乾進《曆元》於朝,通政司送局考驗。光啟摘當極論者七事:其一,歲實自漢以來,代有減差,到《授時》減為二十四分二十五秒。依郭法百年消一,今當為二十一秒有奇。而《曆元》用趙知微三十六秒,翻覆驟加。其一,弧背求弦矢,宜用密率。今《曆測》中猶用徑一圍三之法,不合弧矢真數。其一,盈縮之限,不在冬夏至,宜在冬夏至後六度。今考日躔,春分迄夏至,夏至迄秋分,此兩限中,日時刻分不等。又立春迄立夏,立秋迄立冬,此兩限中,日時刻分亦不等。測量可見。其一,言太陰最高得疾,最低得遲,且以圭表測而得之,非也。太陰遲疾是入轉內事,表測高下是入交內事,豈容混推。而月行轉周之上,又復左旋,所以最高向西行極遲,最低向東行乃極疾,舊法正相反。其一,言日食正午無時差,非也。時差言距,非距赤道之午中,乃距黃道限東西各九十度之中也。黃道限之中,有距午前後二十餘度者,但依午正加減,焉能必合。其一,言交食定限,陰曆八度,陽曆六度,非也。日食,陰曆當十七度,陽曆當八度。月食則陰陽曆俱十二度。其一,《曆測》云:「宋文帝元嘉六年十一月己丑朔,日食不盡如鉤,晝星見。今以《授時》推之,止食六分九十六秒,郭曆舛矣。」夫月食天下皆同,日食九服各異。南宋都於金陵,郭曆造於燕地,北極出地差八度,時在十一月則食差當得二分弱,其云「不盡如鉤」,當在九分左右。郭曆推得七分弱,乃密合,非舛也。本局今定日食分數,首言交,次言地,次言時,一不可闕。已而文魁反覆論難,光啟更申前說,著為《學曆濁辨》。

其論歲實小餘及日食變差尤明晰。曰:「歲實小餘,自漢迄元漸次消減。今新法定用歲實,更減於元。不知者必謂不惟先天,更先《大統》。乃以推壬申冬至,《大統》得已亥寅正一刻,而新法得辰初一刻十八分。何也?蓋正歲年與步月離相似,冬至無定率,與定朔、定望無定率一也。朔望無定率,宜以平朔望加減之,冬至無定率,宜以平年加減之。故新法之平冬至,雖在《大統》前,而定冬至恒在《大統》後也。」又曰:「宋仁宗天聖二年甲子歲,五月丁亥朔,曆官推當食不食,諸曆推算皆云當食。夫於法則實當食,而於時則實不食。今當何以解之?蓋日食有變差一法,月在陰曆,距交十度強,於法當食。而獨此日此之南北差,變為東西差,故論天行,則地心與日月相參直,實不失食。而從人目所見,則日月相距近變為遠,實不得食。顧獨汴京為然,若從汴以東數千里,則漸見食,至東北萬餘里外,則全見食也。夫變差時不同,或多變為少,或少變為多,或有變為無,或無變為有。推曆之難,全在此等。」未幾,光啟入愉閣。

五年九月十五日,月食,監推初虧在卯初一刻,光啟等推在卯初三刻,回回科推在辰初初刻。三法異同,致奉詰問。至期測候,陰雲不見,無可徵驗。光啟具陳三法不同之故,言:

時刻之加減,由於盈縮、遲疾兩差。而盈縮差,舊法起冬夏至,新法起最高,最高有行分,惟宋紹興間與夏至同度。郭守敬後此百年,去離一度有奇,故未覺。今最高在夏至後六度。此兩法之盈縮差所不同也。遲疾差,舊法只用一轉周,新法謂之自行輪。自行之外,又有兩次輪。此兩法之遲疾差所以不同也。至於《回回曆》又異者,或由於四應,或由於里差,臣實未曉其故。總之,三家俱依本法推步,不能變法遷就也。

將來有宜講求者二端:一曰食分多寡。日食時,陽晶晃耀,每先食而後見。月食時,游氣紛侵,每先見而後食。其差至一分以上。今欲灼見實分,有近造窺筒,日食時,於密室中取其光景,映照尺素之上,初虧至復圓。分數真確,書然不爽。月食用以仰觀二體離合之際,鄞鄂著明。與目測迥異。此定分法也。一曰加時早晚。定時之術,壺漏為古法,輪鐘為新法,然不若求端於日星,晝則用日,夜則任用一星。皆以儀器測取經緯度數,推算得之。此定時法也。二法既立,則諸術之疏密,毫末莫遁矣。

古今月食,諸史不載。日食,自漢至隋,凡二百九十三,而食於晦者七十七,晦前一日者三,初二日者三,其疏如此。唐至五代凡一百一十,而食於晦者一,初二日者一,初三日者一,稍密矣。宋凡一百四十八人,無晦食者,更密矣。猶有推食而不食者一。至加時差至四五刻者,當其時已然。可知高速無窮之事,必積時累世,仍稍見其端兒。故漢至今千七百歲,立法者十有三家,而守敬為最優,尚不能無刻之差,而況於沿習舊法者,何能現其精密哉?

是年,光啟又進《曆書》三千卷。明年冬十月,光啟以病辭曆務,以山東參政李天經代之。逾月而光啟卒。七年,魏文魁上言,曆官所推交食節氣皆非是。於是命魁入京測驗。是時言曆者四家,《大統》、《回回》外、別立西洋為西局,文魁為炙局。言人人殊,紛若聚訟焉。

天經繕進《曆書》凡二十九卷,并星屏一具,俱故輔光啟督率西人所造也。天經預推五星凌犯會合行度,言:「閏八月二十四,木犯積履尸氣。九月初四昏初,火土同度。初七卯正,金土同度。十一昏初,金火同度。舊法推火土同度,在初七,是後天三日。金火同度在初三,是先天八日。」而文魁則言,天經所報,木星犯積尸不合。天經又言:「臣於閏八月二十五日夜及九月初一日夜,同體臣陳六韋等,用窺管測,見積尸為數十小星圍聚,木與積尸,共納管中。蓋窺圓徑寸許,兩星相距三十分內者,方得同見。如觜宿三星相距二十七分,則不能同見。而文魁但據臆算,未經實測。據云初二日木星已在柳前,則前此豈能越鬼宿而飛渡乎?」天經又推木星退行、順行,兩經鬼宿,其度分晷刻,已而皆驗,於是文魁說絀。

天經又進《曆書》三十二卷,并日晷、星晷、窺筒諸儀器。八年四月,又上《乙亥丙子七政行度曆》及《參訂曆法條議》二十六則。

某七政公說之議七:一曰諸曜之應宜改。蓋日月五星平行起算之根則為應,乃某曜某日某時躔某宮次之數。今新法改定諸應,悉從崇禎元年戊辰前,冬至後,己卯日子正為始。二曰測諸曜行度,應用黃道儀。蓋太陽由黃道行,月星各有本道,出入黃道內外,不行赤道。若用赤道儀測之,所得經緯度分,須通以黃、赤通率表,不如用黃道儀,即得七政之本度為便也。三曰諸方七政行度,隨地不等。蓋日月東西見食,其時各有先後,既無庸疑矣。則太陽之躔二十四節氣,與月五星之掩食凌犯,安得不與交食同一理乎?故新法水成諸表,雖以順天府為主,而推算諸方行度亦皆各有本法。四曰諸曜加減分,用平、立、定三差法,尚不足。蓋加減平行以求自行,乃曆家耍務。第天實圓體,與平行異類,舊所用三差法,俱從句股平行定者,於天體未合。即扣盈縮損益之數,未得其真。今新法加減諸表,乃以圓齊圓,始可合天。五曰隨時隨地可求諸曜之經度。舊法欲得某日曜經度,必先推各曜冬至日所行宮度宿次,後乃以各段日度比算始得。今法不拘時日方所,只簡本表推步即是。六曰徑一圍三,非弧矢真法。蓋古曆家以直綿測圓形,名曰弧矢法,而算用徑一圍三,廖也。今立割圓八綿表,其用簡而大。弧矢等綿,但乘除一次,使能得之。七曰球上三角三弧形,非句股可盡。蓋古法測天以句股為本,然句股能御直角,不能御斜角。且天為圓球,其面上與諸道相割生多三弧形,句股不足以盡之。

恒星之議四:一曰恒星本行,即所謂歲差,從黃道極起算。蓋各星距赤極度分,古今不同。其距赤道內外地也,亦古今不同。而距黃極或距黃道內外,則皆終古如一,所以知日月五星俱依黃道行。其恒星本行,應從黃極起算,以為歲差之率。二曰古今各宿度不同。蓋恒星以黃道極為極,故各宿距星行度,與赤道極時近時遠。行漸近極,即赤道所出過距星綿漸密,其本宿赤道弧則較小。漸遠極,即過距星綿漸疏,其本宿赤道弧則較大。此緣二道二極不同,非距星有異行,亦非距星有易位也。如觜宿距星,漢測距參二度,唐測一度,宋崇寧測半度,元郭守敬五分。今測之,不啻無分,且侵入參宿二十四分,非一癥乎?三曰夜中測星定時。蓋太陽依赤道左行,每十五度為一小時。今任測一星距子午圈前後度分,又以本星經行與太陽經行查加減,得太陽距子午圈度分,因以變為真時刻。四曰宋時所定十二宮次,在某宿度,今不能定於某宿度。蓋因恒星有本行,宿度已右移故也。

太陽之議四:一太陽盈縮之限,非冬、夏二至,所謂最高及最高衝出也。此限年年右行,今已過二至後六度有奇。二曰以圭表測冬夏二至,非法之善。蓋二至前後,太陽南北之行度甚微,計一丈之表,其一日之影差不過一分三十秒,則一秒得六刻有奇,若測差二三秒,即差幾二十刻,安所得準乎?今法獨用春、秋二分,蓋以此時太陽一日南北行二十四分,一日之景差一寸二分,即測差一二秒,算不得滿一刻,較二至為最密。三曰日出入分,應從順天府起算。蓋諸方北極出地不同,晨昏時刻亦因以異。《大統》依應天府算,上以晝夜長短,日月東刃西帶食,所推不準。今依天罕改定。四曰平節氣,非上天真節氣。蓋舊法氣策,乃歲周二十四分之一。然太陽之行有盈有縮,不得平分。如以平分,則春分後天二日,秋分先天二日矣。今悉改定。

太陰之議四:一曰朔望之外,別有損益分,一加減不足以盡之。蓋舊定太陰平行,算朔望加減,大率五度有奇,然兩弦時多寡不一,即《授時》亦言朔望外,平行數不定,明其理未著其法。今於加減外,再用一加減,名為二三均數。二曰緯度不能定於五度,時多時寡。古今曆家以交食分數及交泛等,測量定黃白二道相距約五度。然朔望外兩道距度,有損有益,大距計五度三公度之一。若一月有兩食,其弦時用儀求距黃道度五度,未能合天。三曰交行有損益分。蓋羅喉、計都即正交、中交行度,古今為平行。今細測之,月有時在交上,以平求之,必不合算。因設一加減,為交行均數。四曰天行無紫氣。舊謂生於閏餘,又為木之餘氣。今細考諸曜,無象可明,知為妄增。

交食之議四:一曰日月景徑分恒不一。蓋日月時行最高,有時行最高,有時行最卑,因相距有遠近,見有大小。又因遠近竿太陰過景,時有厚薄,所以徑分不能為一。二曰日食午正非中限,乃以黃道九十度限為中限。蓋南北東西差俱依黃道,則時差安得不從黃道道論其初末以求中限乎?且黃道出地平上,兩象限自有其高,亦自有其中。此理未明,或宜加反減,宜減反加,凡加進不合者由此也。三曰日食初虧復圓,時刻多寡恒不等,非二時折半之說。蓋視差能變實行為視行,則以視差較食甚前後,鮮有不參差者。夫視差既食甚前後不一,又安能令視行前後一乎?今以視行推變時刻,則初虧復圓,其不能相等也明矣。四曰諸方各依地經推算時刻及日食分。蓋地面上東西見日月出沒,各有前後不同即所得時刻亦不同。故見食雖一而時刻異,此日月食皆一理。若日食則因視差隨地不一,即太陰視距不一,所見食分亦異焉。

五緯之議三:一曰五星應用太陽視行,不得以段目定之。蓋五星皆以太陽為主,與太陽合則疾行,衝則退行。且太陽之行有遲疾,則五星合伏日數,時寡時多,自不可以段目定其度分。二曰五星應加緯行。蓋五星出入黃道,各有定距度。又木、土、火三星衝太陽緯大,合太陽緯小。金、水二星順伏緯小,逆伏緯大。三曰測五星,當用恒星為準則。蓋測星用黃道儀外,宜用弧矢等儀。以所測緯星視距二恒星若干度分,依法布算,方得本星真經緯度分。或繪圖亦可免算。

是時新法書器俱完,屢測交食凌犯俱密合,但魏文魁等多方陰撓,內官實左右之。以故帝意不能決,諭天經同監局虛心詳究,務祈書一。是年,天經推水星伏見及木星所在之度,皆與《大統》各殊,而新法為合。又推八月二十七日寅正二刻,木、火、月三曜同在張六度,而《大統》推木在張四度,火、月張三度。至期,果同在張六度。九年正月十五日辛酉,曉望月食。天經及《大統》、《回回》、東局,各頂推虧圓食甚分秒時刻。天經恐至期雲掩難見,乃按里差,推河南、山西所見時刻,奏遣官分行測驗。其日,天經與羅雅谷、湯若望、大理評事王應遴、禮臣李焻及監局守登、文魁等赴臺測驗,惟天經所推獨合。已而,河南所報盡合原推,山西則食時雲掩無從考驗。

帝以測驗月食,新法為近,但十五日雨水,而天經以十三日為雨水,令再奏明。天經覆言:

諭節氣有二法:一為平節氣,一為定節氣。平節氣者,以一歲之實,二十四平分之,每得一十五日有奇,為一節氣。故從歲前冬至起算,必越六十日八十七刻有奇為雨水。舊法所推十五日子正一刻者此也,定節氣者,以三百六十為周天度,而亦以二十四平分之,每得一十五度為一節氣。從歲前冬至起算,歷五十九日二刻有奇,而太陽行滿六十度為雨水。新法所推十三日卯初二刻八分者此也。太陽之行胡盈有縮,非用法加減之,必不合天,安得平分歲實為節氣乎?以春分癥之,其理更明。分者,黃赤相交之點,太陽行至此,乃晝夜平分。舊法於二月十四日下,註晝五十刻、夜五十刻是也。夫十四日書夜已平分,則新法推十四日春分者為合天,而舊法推十六日者,後天二日矣。知春分,則秋分及各節氣可知,而無疑於雨水矣。

已而天經於春分屈期,每午赴臺測午正太陽高度。二月十四日高五十度八分,十五日高五十度三十分。末經乃言:

京師北極出地三十九度五十五分,則赤道應高五十度五分,春分日太陽正當赤道上,其午正高度與赤道高度等,過此則太陽高度必漸多,今置十四日所測高度,加以地半經差二分,較赤道已多五分。蓋原推春分在卯正二五分弱,是時每日緯行二十四分弱,時差二十一刻五分,則緯行應加五分強。至十五日,并地半徑較赤道高度已多至三十分,況十六日乎?是春分當在十四,不當在十六也。秋風京然。又出《節氣圖》曰:

內規分三百六十五度四分度之一者,日度也。外規公三百六十度者,天度也。自冬至起算,越九十一日三十一刻六分,而始歷春分者,日為之限敢,乃在天則已踰二度餘矣。又越二百七十三日九十三刻,一十九分,而即交秋分者,亦日為之限也,乃在天不及二度餘。豈非舊法春分每後天二日,秋分先天二日耶?

十年正月辛丑朔,日食,天經等預推京師師見食一分一十秒,應天及各省分秒各殊,惟雲南、太原則不見食。其初虧、食甚、復圓時刻亦各異。《大統》推食一分六十三秒,《回回》推食三分七十秒,東局所推止游氣侵光三十餘秒。而食時推驗,惟天經為密。時將廢《大統》,用新法,於上管理另局曆務代州知州郭正中言:「中曆必不可盡廢,西曆必不可專行。四曆各有短長,當參合諸家,兼收西法。」十一年正月,乃詔仍行《大統曆》,如交食經緯,晦朔弦望,因年遠有差者,旁求參考新法與回回科並存。上年,進天經光祿寺卿,仍管曆務,十四年十二月,天經言:「《大統》置閏,但論月無中氣,新法尤視合朔後先。今所進十五年新曆,其十月、十二月中氣,適交次月合朔時刻之前,所以月內雖無中氣,而實非閏月。蓋氣在朔前,則此氣尚屬前月之晦也。至十六年第二月止有驚蟄一節,而春分中氣,交第三月合朔之後,則第二月為閏正月,第三月為第二月無疑。」時帝已深知西法之密。迨十六年三月乙丑朔日食,測又獨驗。八月,詔西法果密,即改為《大統曆法》,通行天下。未幾國變,竟未施行。本朝用為憲曆。

按明制,曆官皆世業,成、弘間尚能建修改之議,萬曆以後則皆專己守殘而已。其非曆官而知曆者,鄭世子而外,唐順之、周述學、陳壤、袁黃、雷宗皆有著述。唐順之未有成書,其議論散見周述學之《曆宗通議》、《曆宗中經》。袁黃著《曆法新書》,其天地人三元,則本之陳壤。而雷宗亦著《合壁連珠曆法》皆會通回回曆以入《授時》,雖不能如鄭世子之精微,其於中西曆理,亦有所發明。邢雲路《古今律曆考》,或言本出魏文魁手,文魁學本慮淺,無怪其所疏《授時》,皆不得其旨也。

西洋人之來中土者,皆自稱甌羅巴人。其曆法與回回同,而加精密。嘗考前代,遠國之人言曆法者多在西域,而東南北無聞。唐之《九執律》,元之《萬年曆》,及洪武間所譯《回回曆》,皆西域也。蓋堯命義、和仲叔分宅四方,義仲、義叔、和叔則以隅夷、南交、朔方為限,獨和仲但曰「宅西」,而不限以地,豈非當時聲教之西被者遠哉。至於周末,疇人子弟分散。西域、天方諸國,接壤西陲,百若東南有大海之阻,又無極北嚴寒之畏,則抱書器而西征,勢固便也。甌羅巴在回回西,其風俗相類,而好奇喜新競勝之習過之。故則曆法與回回同源,而世世增修,遂非回回所及,亦其好勝之欲為之也。義、和既失其守,古籍之可見者,僅有《周髀》範圍,亦可知其源流之所自矣。夫旁搜採以續千百年之墜緒,亦禮秀求野之意也,故備論也。


▲大統曆法一上法原

造曆者各有本原,史宜備錄,使後世有以考。如《太初》之起數鐘律,《大衍》之造端蓍策,皆詳本志。《授時曆》以測算術為宗,惟求合天,不牽合律呂、卦爻。然其法所以立,數之所從出,以及晷影、星度,皆有全書。郭守敬、齊履謙傳中,有書名可考。《元史》漫無采摭,僅存李謙之《議祿》、《曆經》之初稿。其後改三應率及立成之數,與夫割圓弧矢之法,平立定三差之原,盡削不載。使作者精意湮沒,識者憾焉。今據《大統因通軌》及《曆草》諸書,稍為編次,首法原,次立成,次推步。而法原之目七:曰句股測望,曰弧矢割圓,曰黃赤道內外度,曰白道交周,曰日月五星平立定三差,曰里差刻漏。

▲句股測望

北京立四丈表,冬至日午正,測得景辰七丈九尺八寸五分。隨以簡儀測到太陽南至地平二十六度四十六分半,為半弧背。求得矢度,五度九十一分半。置周天半徑,截矢餘五十四度九十六分為股,乃本地支戴日下之度。以弦股別句術,求得句二十六度一下七分六十六秒,為日出地半弧弦。

北京立四丈表,夏至日午正,測得景長一丈一尺七寸一分。隨以簡儀測到太陽南至地平七十四度二十六分半,為半弧背。求得矢度,四十三度七十四分少。置周天半徑,截矢餘一十七度一十三分二十五秒為句,乃本地去戴日下之度。以句弦別股術,求得股五十八度四十五分半,為日出地半弧弦。

以二至日度相併,得一百度七十三分,折半得五十度三十六分半,為北京赤道出地度。以赤道出地度轉減周天四之一,餘四十度九十四分九十三秒七十五微,為北京北極出地度。

▲弧矢割圓

周天經一百二十一度七十五分少。少不用。半徑六十零度八十七分半。又為黃赤道大弦。二至黃赤道內外半弧背二十四度。所測就整。二至黃赤道弧矢四度八十四分十二秒。黃赤道大句二十三度八十分七十秒。黃赤道大股五十六度零二分六十八秒。半徑內減去矢度之數。

割圓求矢術置半弧度自之,為半弧背幕,周天徑自之,為上廉。上廉乘半弧背幕,為正實。上廉乘徑,為益從方。半弧背倍之,乘徑,為下廉。以初商乘上廉,得數以減益從方,餘為從方。置初商自之以下廉,餘以初商乘之,為從廉。從方、從廉相並,為下法。下法乘初商,以減正實,實不足減,改初商。實有不盡,次第商除之。倍初商數,與次商相並以乘上廉,得數以減益從方,餘為從方。并初商次商而自之,又以初商自之,並二數以減下廉,餘以初商倍數並次商乘之,為從廉。從方、從廉相並,為下法。下法乘次商,以減餘實,而定次商。有不盡者,如法商之,皆以商得數為矢度之數。黃赤道同用。

如以半弧背一度求矢。術曰:置半弧背一度自之,得一度,為半弧幕。置周天徑一百二十一度太自之,得一萬四千八百二十三度零六分二十五秒,為上廉。上廉乘半弧背幕,得一萬四千八百二十三度零六分二五,為正實。上廉又乘徑,得一百八十零萬四千七百零七度八十五分九十三秒七五,為益從方。半弧背一度倍之,得二度,以乘徑得二百四十三度五十分,為下廉。初商八十秒。置初商八十秒乘上廉一萬四千八百二十三度零六二五,得一百一十八度五八四五,以減益從方一百八十零萬四千七百零七度八五九三七五,餘一百八十零萬四千五百八十九度二七四八七五,為從方。又置初商八十秒自之,得六十四微,以減下廉餘二百四十三度四九九三六。仍以八十秒乘之,得一度九四七九九九四八八,為從廉。以從廉、從方並之,共得一百八十零萬四千五百九十一度二二二八七四四八八,為下法。下法乘初商,得一萬四千四百三十六度七十二分九七八二九九五九零四,以減正實,余實三百八十六度三十三分二七一七零零四零九六。次商二秒。置初商八十秒倍之,得一分六十秒。加次商二委六十二秒,乘上廉一萬四千八百二十三度零六二五,得二百四十零度一三三六一二五,以減益從方,餘一百八十零萬四千四百六十七二五七六二五,為從方。又置初次商八十二秒自之,得六十七微。加初商八十秒自之之數,得一秒三十一微,以減下廉,餘二百四十三度四九九八六九。以前所得一分六十二秒乘之,得三度九十四分四六九七八七七八,為從廉。以從廉、從方並,得一百八十零萬四千四百七十一度六十七分零四六零三七八,為下法。下法乘次商,得三百六十零度八九四三三四零九二零七五五六,以減餘實,仍餘二十五度四三八三八二九一二零二零四四。不足一秒葉不用,下同。

凡求得矢度八十二秒,餘度各如上法,求到矢度,以為黃赤相求及其內外度之根。數詳後。

▲黃赤道差

求黃赤道各度下赤道積度術。置周天半徑內減去黃道矢度,餘為黃赤道小弦。置黃赤道小弦,以黃赤道大股乘之大股見割圓為實。黃赤道大弦半徑為法。實如法而一,為黃赤道小股。直黃道矢自乘為實,以周天全徑為法,實如法而一,為黃道半背弦差。以差去減黃赤道積度,即黃道半弧背。餘為黃道半弧弦。置黃赤道半弧弦自之為股幕,黃赤道小股自之為句幕,二幕並之,以開平方法除之,為赤道小弦。置黃赤道半弧弦,以周天半徑亦為赤道大弦乘之為實,以赤道小弦為法而一,為赤道半弧弦。置黃赤道小股,亦為赤道橫小句以赤道大弦即半徑乘之為實,以赤道小弦為法而一,為赤道橫大句,以減半徑,餘為赤道磺弧矢。橫弧矢自之為實,以全徑為法而一,為赤道半背弦差。以差加赤道半弧,為赤道積度。

如黃道半弧背一度,求赤道積度。術曰:「置半徑六十零度八十七分五十秒,即黃赤道大弦。內減黃道矢八十二秒餘六十零度八六六八,為黃赤道小弦。置黃赤道小弦,以黃赤道大股五十六度零二六八乘之,得三千四百一十零度一七二零三零二四為實,以黃赤道大弦六十零度八七五為法,實如法而一,得五十六度零一分九十二秒,為黃赤道小股。又為赤道小句。置矢度八十二秒自之,得六十七微,以全徑一百二十一度七五為法,除之得五十五纖,為黃道平半背弦差。置黃道半弧弦一度,內減黃道半背弦差,餘為半弧弦,因因差在微以下不減,即用一度為半弧弦。置黃道半弧弦一度自之,得一度為股幕。黃赤道小股五十六度零一矣二自之,得三千一百三十八度一五零七六八六四為句幕。二幕並得三千一百三十九度一五零七六八六四為弦實,平方開之,得五十六度零二八一,為赤道小弦。置黃道半弧弦一度,以半徑即赤道大弦乘之,得六十零度八七五為實,以赤道小股五十六度零二八一為法除之,得一度零八分六十五秒,為赤道半弧弦。置黃赤道小股五十六度零一九二,又為赤道小句。以赤道大弦半徑六十零度八七五乘之,得三千四百一十零度一六八八為實,以赤道小弦為法除之,得六十零度八十六分五十三秒,為赤道橫大句。置半徑六十零度八十七分五十秒,內減赤道大句六十零度八十六分五十三秒,餘九十七秒,為赤道橫弧矢。置赤道橫弧矢九十七秒自之,得九十四微零九,以全徑為法除之,得七十纖,為赤道背弦差。置赤道半弧弦一度零八分六十五秒,加赤道背弦差,為赤道積度,今差在微已下不加,即用半弧弦為積度。

凡求得赤道積度一度零八分六十五秒。餘度各如上法,求到各黃道度下赤道積,兩數相減,即得黃赤道差,乃至後之率。其分後,以赤道度求黃道,反此求之,其數並同。

▲黃赤道相求弧矢諸率立成上

表格略

▲黃赤道相求弧矢諸率立成下

表格略

按郭敬創法五端,內一曰黃道差,此其根率也。舊法以一百一度相減乘。《授時》立術,以句股、弧矢、方圓、斜直所容,求其數差,合於渾象之理,視古為密。顧《至元曆經》所載略,又誤以黃道矢度為積差,黃道矢差為率,今正之。

▲割圓弧矢圖

凡渾圓中剖,則成平圓。任割平圓之一分,成弧矢形,皆有弧背,有弧弦,有矢。剖弧矢形而半之,則有半弧背,有半弧弦,有矢。因弦矢句股形,以半弧弦為句,矢減半徑之餘為股,半徑為弦。句股內成小句股,則有小句、小股、小弦、而大小可互求,平側可互用,渾圓之理,斯為密近。

平者為赤道,斜者為黃道。因二至黃道赤之距,生大句股。因各度黃赤之距,生小句股。

外大圓為赤道。從北極平視,則黃道在赤道內,有赤道各度,即各有其半弧弦,以生大名股。又各有其相當之黃道半弧弦,以生小句股。此二者皆可互求。

按舊史無圖,然表亦圖之屬也。今句股割弧矢之法,實為曆家測算之本。非圖不明,因存其要者數端。

▲黃赤道內外度

推黃道各度,距赤道內外及去極遠近術。置半徑內減去赤道小弦,餘為赤道二弦差。又為黃赤道小弧矢,又為內外矢,又為股弦差。置半徑內外減去黃道矢度,餘為黃赤道小弦,以二至黃赤道內外半弧弦乘之為實,以黃赤道大弦為法,即半徑。除之為黃赤道小弧弦。即黃赤道內外半弧弦,又為黃赤道小句。置黃赤道小弧矢自之,即赤道二弦差。以全徑除之,為半背弦差。以差加黃赤道小弧弦為黃赤道小弧半背,即黃赤道內外度。置黃赤道內外度,視在盈初縮末限以加,在縮初盈天限以減,皆加減象限度,即各得太陽去北極度分。

如冬至後四十四度,求太陽去赤道內外及去極度。術曰:「置半徑六十零度八十七分半,內減黃道四十四度下赤道小弦五十八度三十五分六十九秒,餘二度五十一分八十一秒,為黃赤道小弧矢。即內外矢。置半徑六十零度八七五,內減黃道四十四度,矢一十六度五十六分八十二秒,餘四十四三十零分六十八秒,為黃赤道小弦。置黃赤道小弦,以二至黃赤道內外半弧弦二十三度七十一分乘之,得一千零五十零度五十一分四二三八為實,以黃赤道大弦六十零度八七五為法除之,得一十七度二十五分十九秒為黃赤道小弧弦。即內外半弧弦。置黃赤道小弧矢二度五十一分八十一秒自之為實,以全徑地百二十一度七十五分除之,得五分二十一秒為背弦差,以差加黃赤道小弧弦一十七度二十五分六十九秒,得一十七度三十零分八十九秒,為二至前後四十四度,太陽去赤道內外度。置象限九十一度三十一分四十三秒七五,以內外度一十七度三零八九加之,得一百零八度六十二分三十二秒七五,為冬至後四十四度太陽去北極度。

▲黃道每度去赤道內外及去北極立成

表格略

▲白道交周

推白赤道正交,距黃赤道正交北極數。術曰:「置實測白道出入黃道內外六度為半徑弧弦,又為大圖弧矢,又為股弦差。置半徑六十零度七五自之,得三千七百零五度七六五六二五,以矢六度而一,得六百一十七度六十三分為股弦和,加矢六度,共六百二十三度六十三分為大圓徑。依法求得容闊五度七十分,又為小句。又以二至出入半弧弦二十三度七十一分為大句。以大句為法,除大股五十六度零六分五十秒,得二度三十七分就整為度差。以度差乘小句,得小股一十三度四十七分八十二秒,為容半長。置半徑六十零度八七五為大弦,以乘小句五度七十分為實,以大句二十三度七十一分為法除之,得一十四度六十三分為小弦,又為白赤道正交,距黃赤道正交半弧弦。依法求行半弧背一十四度六十六分,為白赤道正交距黃赤道正交極婁數。

▲大統曆法一下法原

日月五星平定三差

太陽盈縮平立定三差之原。

冬至前後盈初縮末限,八十八日九十一刻,就整。離為六段,每段各得一十四日八十二刻。就整。各段實測日躔度數,與平行相較,以為積差。

積日積差

第一段一十四日八二七千零五十八分零二五

第二段二十九日六四一萬二千九百七十六三九二

第三段四十四日四六一萬七千六百九十三七四六二

第四段五十九日二八二萬一千一百四十八七三二八

第五段七十四日一零二萬三千二百七十九九九七

第六段八十八日九二二萬四千零二十六一八四

各置其段積差,以其段積日除之,為各段日平差。置各段日平差,與後段日平差相減,為一差。置一差,與後段一差相減,為二差。

日平差一差二差

第一段四百七十六分二五三十八分四五一分三八

第二段四百三十七分八零三十九分八三一分三八

第三段三百九十七分九七四十一分二一一分三八

第四段三百五十六分七六四十一分五九一分三八

第五段三百一十四分一七四十三分九七

第六段二百七十零分二零

置第一段日平差,四百七十六分二十五秒,為凡平積。以第二段二差一分三十八秒,去減第一段一差十八分四十五秒,餘三十七分零七秒,不凡平積差。另置第一段二差一分三十八秒,折半得六十九秒,為凡立積差。以凡平積差三十七分零七秒,加入凡平積四百七十六分二十五秒,共得五百一十三分三十二秒,為定差。

以凡立積差六十九秒,去減凡平積差三十七分零七秒,餘三十六分三十八秒為實,以段日一十四日八十二刻為法除之,得二分四十六秒為平差。置凡立積差六十九秒為實,以段日為法除二次,得三十一微,為立差。

夏至前後縮初盈末限,九十三日七十一刻,就整。離為六段,每段各得一十五日六十二刻。就整。各段實測日躔度數,與平行相較,以為積差。

積日積差

第一段一十五日六二七千零五十八分九九零四

第二段三十一日二四一萬二千九百七十八六五八

第三段四十六日八六一萬七千六百九十六六七九

第四段六十二日四八二萬萬一千一百五十零七二九六

第五段七十八日一零二萬三千二百七十八四八六

第六段九十三日七二二萬四千零百一十七六二四四

推日平差、一差、二差術,與盈初縮末同。

日平差一差二差

第一段四百五十一分九二三十六分四七一分三三

第二段四百一十五分四五三十七分八零一分三三

第三段三百七十七分六五三十九分一二一分三三

第四段三百三十八分五二四十零分四六一分三三

第五段二百九十八分零六四十一分七九

第六段二百五十六分二七

置第一段日平差,四百五十一分九十二秒,為凡平積。以第一段二差一分三十三秒,去減第一段一差三十六分四十七秒,餘三十一分一十四秒,為凡平積差。另置第一段二差一分三十三秒折半,得六十六秒五十微,為凡立積差。以凡平積差三十五分一十四秒,加入凡平積四百五十一分九十二秒,共四百八十七分零六秒,為定差。以凡『立積差六十六秒五十微,去減凡平差三十五分一十四秒,餘三十四分四十七秒五十微為實,以段日一十五日六二為法除之,得二分二十一秒,為平差。置凡立積差六十六秒五十微為實,以段日為法,除二次,得二十七微,為立差。

凡求盈縮,以入曆初末日乘立差,得數以加平差,再以初末日乘之,得數以減定差,餘數以初末日乘之,為盈縮積。

凡盈曆以八十日九零九二二五為限,縮曆以九十三日七一二零二五為限。在其限已下為初,以上轉減半歲周餘不末。盈初是人冬至後順推,縮末是從冬至前逆溯,其距冬至同,故其盈積同。縮初是從夏至後順推,盈末是從夏至前逆溯,其距夏至同,故其縮積同。

表格略

▲盈縮招差圖說

盈縮招生,本為一象限之法。如盈曆則以八十八日九十一刻為象限,縮曆則以九十三日七十一刻為象限。今止作九限者,舉此為例也。其空格九行定差本數,為實也。其斜綿以上平差立差之數,為法也。斜綿以下空格之定差,乃余實也。假如定差為一萬,平差為一百,立差為單一。今求九限法,以九限乘定差得九萬為實。另置平差,以九限乘二次,得八千一百。置立差,以九限乘三次,得七百二十九。並兩數得八百二十九為法。以法減實,餘八萬一千一百七十一,為九限積。又法,以九限乘平差行九百,又以九限乘立差二次得八十一,並兩數得九進八十一為法,定差一萬為實,以法減實,餘矣千零一十九,即九限末位所書之定差也。於是瑞以九限乘餘實,得八萬一千一百七十一,為九限積,與前所不所得不同。蓋前法是先乘後減,又法是先減後乘,其理一也。

按《授時曆》於七政盈縮,並以垛積招差立算,其污七巧合天行,與西人用小輪推步之法,殊途同歸。然世所傳《九章》諸書,不載其術,《曆草》載其術,而不言其故。宣城梅文鼎為之圖解,於平差、立差之理,垛積之法,皆有以發明其所以然。有專書行於世,不能備錄,謹錄《招生圖說》,以明立法之大意云。

盈初縮末置立差三十一微,以六因之,得一秒八十六微,為加分立差。置平差二分四十六秒,倍之,得四分九十二秒,加入加分立差,得四分九十二秒八十六微,為平立合差。

置定差五百一十三分三十二秒,內減平差二分四十六秒,再減立差三十一微,餘五百一十零分八十五秒六十九微,為加分。

縮初盈末置立差二十七微,以六因之,得一秒六十二微,為加分立差。置平差二分二十一秒,倍之,得四分四十二秒,加入加分立差,得四分四十三秒六十二微,為平立合差。

置定差四百八十七分零六秒,內減平差二分二十一秒,再減立差二十七微,餘四百八十四分八十四秒七十三微,為加分。

已上所推,皆初日之數。其推次日,皆以加分立差,累加平立合差,為次日平立合差。以平立合差減其日加分,為次日加分,盈縮並同。其加分累積之,即盈縮積,其數並見立成。

▲太陰遲疾平立三差之原

太陰轉周二十七日五十五刻四六。測分四象,象各七段,四象二十八段,每段十二限,每象八十四限,凡三百三十六限,而四象一周。以四象為法,除轉周日,得每象六日八八八六五,分為七段,每段下實測月行遲疾之數,與平行相較,以求積差。

積限積差

第一段一十二一度二十八分七一二

第二段二十四二度四十五分九六一六

第三段三十六三度四十八分三七九二

第四段四十八四度三十二分五九五二

第五段六十四度九十五分二四

第六段七十二五度三十二分九四四

第七段八十四五度四十二分三三七六

各置其段積差,以其段積限為法除之,為各段限平差。置各段限平差,與後段相減為一差。置一差,與後段一差相減為二差。

限平差一差二差

第一段一十零分七二六零四十七秒七六九秒三六

第二段一十零分二四八四五十七秒一二九秒本六

第三段九分六七七二六十六秒四八九秒三六

第四段九分零一二四七十五秒八四九秒三六

第五段八分二五四零八十五秒二零九秒三六

第六段七分四零二零九十四秒五六

第七段六分四五六四

置第一段限平差一十零分七二六為凡平積。置第一段一差四十七秒七六,以第一段二差九秒三六減之,餘三十八秒四十微,為凡平積差。另置第一段二差九秒三十六微折半,得四秒六十八微,為凡立積差。以凡平積差三十八秒四十微,加凡平積一十零分七二六,得一十一分一十一秒,為定差。置凡平積差三十八秒四十微,以凡立積差四秒六十八微減之,餘三十三秒七十二微為實,以十二限為法除之,得二秒八十一微,為平差。置凡立積差四秒六十八微為實,十二限為法,除二次,得三微二十五纖,為立差。

凡求遲疾,皆以入曆日乘十二限二十分,以在八十四限已下為初,已上轉減一百六十八限餘為末。各以初末限乘立差,得數以加平差,再以初末限乘之,得數以減定差,餘以初末限乘之,為遲疾積。其初限是從最遲最疾處順推至後,末限是從最遲最疾處逆溯至前,其距其距最遲疾處同,故其積度同。太陰與太陽立法同,但太陽以定氣立限,故盈縮異數。太陰以平行立限,故遲疾同原。

布立成法置立差三微二十五纖,以六因之,得一十九微五十纖,為損益立差。置平差二秒八十一微,倍之,得五秒六十二微,再加損益立差一十九微五十纖,共得五秒八十一微,為初限平立合差。自此以損益立差,累加之,即每限平立合差。至八十限下,積至二十一秒四一五,為平立合差之極。八十一限下差一秒七八零九,八十二限下一秒七八零八,至八十三限下,平立合差,與益分中分,為益分之終。八十四限下差,亦與損分中分,為損分之始。至八十六限下差,亦二十一秒四一五,自此以損益立差累減之,即每限平立合差,至末限與初限同。置定差一十一分一十一秒,內減平差二秒八十一微,再減立差三微二十五纖,餘一十一分零八秒一十五微七十五纖為加分定差,即初限損益分。置損益分,以其限平立合差益減損加之。即為次限損益分。以益分積之,損分減之,便為其下遲疾度。以八百二十分為一限日率,累加八百二十分為每限日率。以上俱詳立成。

五星平立定三差之原凡五星各以實測,分其行度為八段,以求積差,略如日月法。

木星立差加,平差減。

積日積差

第一段一十一日五十刻一度二一五二九七一一二

第二段二十三日二度三四零五二一四

第三段三十四日五十刻三度三五四一三七二六五

第四段四十六日四度二三四六零九一二

第五段五十七日五十刻四度九六零四零一三七五

第六段六十九日五度五零九九七八四四

第七段八十零日五十刻五度八六一八零四七二五

第八段九十二日五度九九四三四四六四

凡平差凡平較凡立較

第一段一十分五六七八零一三十九秒一六二一六秒二四二二

第二段一十分一七六一八四十五秒四零四三六秒二四二二

第三段九分七二二一三七五十一秒六四六五六秒二四二二

第四段九分二零五六七二五十七秒八八八七六秒二四二二

第五段八分六二六七八五六十四秒一三零九六秒二四二二

第六段七分九八五四七六七十零秒三七二一六秒二四二二

第七段七分二八一七四五七十六秒六一五三

第八段六分五一五五九二

各置其段所測積差度為實,以段日為法除之,為凡平差。各以凡平差與次段凡平差相較,為凡平較。又以凡平較與次段凡平較相較,為凡立較。置第一段凡平較三十九秒一六二一,減其下凡立較六秒二四二二,餘三十二秒九一九九,為初段平立較。加初段凡平差一十分五六七八零一,共得一十零分八十九秒七十零微,為定差。秒置萬位。置初段平立較差三十二秒九一九九,內減凡立較之半,三秒一二一一,餘二十九秒七九八八,以段日一十一日五十刻除之,得二秒五十九微一十二纖為平差。置凡立差之半,三秒一二一一,以段日為法除二次,得二微三十六纖為立差。

已上為木星平立定三差之原。

火星盈初縮末。立差減,平差減。

積日

第一段七日六十二刻五十分

第二段一十五日二十五刻

第三段二十二日八十七刻五十分

第四段三十零日五十零刻

第五段三十八日一十二刻五十分

第六段四十五日七十五刻

第七段五十三日三十七刻五十分

第八段六十一日

積差

第一段六度二六八二五一二二八一八五五九三七五

第二段一十一度六零零一七五七四三五九三七五

第三段一十六度零二五九六三七九二五一九五三一二五

第四段一十九度六六九零一三六二一二五

第五段二十二度二七九八九一四七六零七四二一八七五

第六段二十四度一六八二二八六零三二八一二五

第七段二十五度三三一五五六二四九二六零一五六二五

第八段二十五度六一九五一五六六

凡平差

第一段八十二分零六五七三四八四三七五

第二段七十六分零六六七二六一六七五

第三段七十零分零五八八五八一零九三七五

第四段六十四分一八二九六九二五

第五段五十八分四三九零五九六零九三七五

第六段五十二分八二七一二九一八七五

第七段四十七分三四七一七七九八四三七五

第八段四十一分九九九二零六

凡平較

第一段六分一三九八四七二九六八七五

第二段六分零零七八六八零七八一二五

第三段五分八七五八八八八五九三七五

第四段五分七四三九零九六四零六二五

第五段五分六一一九三零四二一八七五

第六段五分四七九九五一二零三一二五

第七段五分三四七九七一九八四三七五

凡立較

第一段一十三秒一九七九二一八七五

第二段一十三秒一九七九二一八七五

第三段一十三秒一九七九二一八七五

第四段一十三秒一九七九二一八七五

第五段一十三秒一九七九二一八七五

第六段一十三秒一九七九二一八七五

凡平較前多後少,應加凡立較。置初段下凡平較六分一三九八四七二九六八七五,加凡立較一十三秒一九七九二一八七五,得六分二七一八二六五一五六二五,為初日下平立較。置初段凡平差八十二分二十零秒六五七三四八四三七五,加初日下平立較六分二七一八二六五一五六二五,得八十八分四十七秒八十四微,為定差。置初日下平立較六分二七一八二六五一五六二五,加凡立較之半,六秒五九八九六零九三七五,得分三三七八一六一二五為實,以段日而一,得八十三秒一十一微八十九纖為平差。置凡立較之半,六秒五九八九六零九三七五,以段日七日六十二刻五十分為法除二次,得一十一微三十五纖為立差。

火星縮初盈末平差負減,立差減。

積日

第一段一十五日二十五刻

第二段三十零日五十刻

第三段四十五日七十五刻

第四段六十一日

第五段七十六日二十五刻

第六段九十一日五十刻

第七段一百零六日七十五刻

第八段一百二十二日

積差

第一段四度五三一二五一八五七九六八七五

第二段九度一零二九六一四五一二五

第三段一十三度五三一六七零九零一七七三七五

第四段一十七度四七八九七九零四

第五段二十零度八四三六六三零六六四零六二五

第六段二十三度四三一三三六二四一二五

第七段二十五度零九二四三五二八三四六八七五

第八段二十五度六一八三七四七二

凡平差

第一段二十九分七一三一二六九三七五

第二段二十九分八四五七七五二五

第三段二十九分五七八三五五零六二五

第四段二十八分六五四零六四

第五段二十七分三三三九五一五六二五

第六段二十五分六一八零一七七五

第七段二十三分五零六二六二五六二五

第八段二十零分九九八六八六

凡平較凡立較

第一段一十三秒二六四八三一二五一十三秒五七六九七七五

第二段二十六秒八四一八零八七五六十五秒五八七二九七五

第三段九十二秒四二九一零六二五三十九秒五八二一三七五

第四段一分三二零一一二四三七五三十九秒五八二一三七五

第五段一分七一五九三三八一二五三十九秒五八二一三七五

第六段二分一一一七五五一八七五三十九秒五八二一三七五

第七段二分五零七五七六二五

取凡立較停者,三十九秒五八二一三七五,以較一段下凡平較一十三秒二六四八三一二五,餘二十六秒三一七三零六二五為較較,以加一段下凡平差二十九分七一三一二六九三七五,得二十九分九十七秒六十三微,為定差。置較較二十六秒三一七三零六二五,以段日一十五日二十五刻而一,得一秒七二五七二五。再置凡立較之半一十九秒七九一零六八七五,以段日而一,得一秒二九七七七五。兩數並得三秒零二微三十五纖為平差。置凡立較之半一十九秒七九一零六八七五,以段日一十五日二五為法除二次,得八微五十一纖,為立差。

已上為火星平立定三差之原。

▲土星盈曆立差加,平差減。

積日積差

第一段一十一日五十刻一度六八三二四五八二八七五

第二段二十三日三度二三二一六四零一

第三段三十四日五十刻四度六二零九三零零八六二五

第四段四十六日五度八二三七一九六

第五段五十七日五十刻六度八一四七零八六六八七五

第六段六十九日七度五六八零七一一一

第七段八十零日五十刻八度零五七九八四一九一二五

第八段九十二日八度二五八六二二八八

凡平差凡平較凡立較第一段一十四分六三六九二零二五五十八秒四零三三二五七秒四八五三五第二段一十四分零五二八八七六十五秒八八八六七五七秒四八五三五第三段一十三分三九四零零零二五七十三秒三七四零二五七秒四八五三五第四段一十二分六六零二六八十零秒八五九三七五七秒四八五三五第五段一十一分八五一六六六二五八十八秒三四四七二五七秒四八五三五第六段一十一分九六八二一九九十五秒八三零零七五七秒四八五三五第七段一十零分零零九九一八二五一分零三秒三一五四二五第八段八分九七六七六四

置第一段下凡平較,內減其下凡立較,餘五十零秒九一七九七五,為平立較。以平立較,加本段凡平差,得一十五分一十四秒六十一微,為定差。置平立較,內減凡立較之半,三秒七四二六七五,餘四十七秒一七五三,以段日十一日五十刻而一,得四秒一十零微二十二纖,為平差。置凡立較之半,以段日除二次,得二微八十三纖,為立差。

▲土星縮曆立差加,平差減。

積日積差

第一段一十一日五十刻一度二四一九七四二六八七五

第二段二十三日二度四一三七三五六九

第三段三十四日五十刻三度四八五零七九六八六二五

第四段四十六日四度四二五八零一六八

第五段五十七日五十刻五度二零五六九七零九三七五

第六段六十九日五度七九四五六一三五

第七段八十零日五十刻六度一六二四一一零零四七五

第八段九十二日六度二七八三七八零八

凡平差凡平較凡立較第一段一十分七九九七七六二五三十零秒五二七三二五八秒七五四九五第二段一十分四九四五零三三十九秒二八二二七五八秒七五四九五第三段一十分一零一六八零二五四十八秒零三七二二五八秒七五四九五第四段九分六二一三零八五十六秒七九二一七五八秒七五四九五第五段九分零五三三八六二五六十五秒五四七一二五八秒七五四九五第六段八分三九七九一五七十四秒三零三零七五八秒七五四九五第七段七分六五四八九四二五八十三秒零五七零七五第八段六分八二四三二四

置一段凡平較,內減其下凡立較,餘二十一秒七七二三七五,為平立較。以平立較加入本段凡平差,得一十一分零一秒七十五微,為定差。置平立較,內減凡立較之半,四秒三七七四七五,餘一十七秒三九四九,以段日一十一日五十刻為法除之,得一秒五十一微二十六纖,為平差。置凡立較之半,以段日為法除二次,得三微三十一纖為立差。

已上為土星平定三差之原。

金星立差加,平差減。

積日積差

第一段一十一日五十刻空度四零二一三四零九八七五

第二段二十三日空度七九一三九三六六

第三段三十四日五十刻一度一五四九一二零八一二五

第四段四十六日一度七四九八二二七六

第五段五十七日五十刻一度七五三二五九零九三七五

第六段六十九日一度九六二三五四四八

第七段八十零日五十刻二度零九四二四二三一六二五

第八段九十二日二度一三六零五六

凡平差凡平較凡立較第一段三分四九六八一八二五五秒五九七六二五三秒七二九四五第二段三分四四零八四二零零九秒三二七零七五三秒七二九四五第三段三分三四七五七一二五一十三秒零六五五二五三秒七二九四五第四段三分二一七零零六一十六秒七八五九七五三秒七二九四五第五段三分零四九一四六二五二十零秒五一五四二五三秒七二九四五第六段

二分八四三九九二二十四秒二四四八七五三秒七二九四五第七段二分六零一五四三二五二十七秒九七四三二五第八段二分三二一八

置一段下凡平較,與其凡立較相減,餘一秒八六一七五為平立較,以加凡平差,得三分五十一秒五十五微,為定差。置平立較,與凡立較之半,一秒八六四七二五相減,餘三十四纖,以段日一十一日五十刻為法除之,得三纖,為平差。置凡立較之半,以段日為為法除二次,得一微四十一纖,為立差。

已上為金星平立定三差之原。

▲水星立差加,平差減。

積日積差

第一段一十一日五十刻空度四四零八四七三五三七五

第二段二十三日空度八六三一零一六八

第三段三十四日五十刻一度二五三八九六三七六二五

第四段四十六日一度六零零三六四八四

第五段五十七日五十刻一度八八九六三一零四三七五

第六段六十九日二度一零八八六六六

第七段八十零日五十刻二度二四五二九二一一三七五

第八段九十二日二度二八五六四四三二

凡平差凡平較凡立較

第一段三分八三三四五五二五八秒零八三九二五三秒七二九四五

第二段三分七五二六一六一十一秒八一三三七五三秒七二九四五

第三段三分六三四四八二二五一十五秒五四二八二五三秒七二九四五

第四段三分四七九零五四一十九秒二七二二七五三秒七二九四五

第五段三分二八六三三一二五二十三秒零零一七二五三秒七二九四五

第六段三分零五六三一四二十六秒七三二一七五三秒七二九四五

第七段二分七八九零零二二五三十零秒四六零六二五

第八段二分四八四三九六

術同金星,求得定差三分八十七秒九十微,平差二十一微六十五纖,立差一微四十一纖。

已上為水星平立定三差之原。

在五星,皆以立差為秒,平差為本,定差為總。五星各以段次因秒,木土金水四星併本,惟火星較本,各以積日而積,五星皆較總,又各以積日乘之,得各實測之度分。

五星積日,皆本度率,除周日得三百六十五度二十五分太。各以四分之一為象限,惟火星用象限三之一,減象限為盈初縮末限,加象限為縮初盈末限。其命度為日者,為各取盈縮曆乘除之便,其實積日之數,即積度也。

▲里差刻漏

求二至差股及出入差。術曰:置所測北極出地四十度九十五分為半弧背,以前割圓弧矢法,推得出地半弧弦三十九度二十六分,為大三斜中股。置測到二至黃赤道內外度二十三度九十分為半弧背,以前法推得內外半弧弦二十三度七十一分。又為黃赤道大句,又為小三斜弦。置內外半弧弦自之為句冪,半徑自之為弦冪,二冪相減,開方得股,以股轉減半徑,餘四度八十一分為二至出入矢,即黃赤道內外矢。夏至日,南至地平七十四度二十六分半為半弧背,求得日下至地半弧弦五十八度四十五分。半徑六十零度八十七分半,為大三斜中弦。置大三斜中股三十九度二十六分,以二至內外半弧弦二十三度七十一分乘之為實,以半徑六十零度八十七分半為法除之,得一十五度二十九分,為小三斜中股又為小股。置小三斜中股一十五度二十九分,去減日下至地半弧弦五十八度中十一分,餘四十三度一十六分,為大股。以出入矢四度八十一分,去減半徑六十零度八十七分半,餘五十六度零六分半,為大股弦。置大股弦,以小股一十五度二九乘之為實,大股四十三度一六為法除之,得一十九度八十七分為小弦,即為二至出入差半弧弦。置二至出入差半弧弦,依法求到二至出入差半弧背一十九度九十六分一十四秒。置二至出入差半弧背一十九度九十六一四秒,置二至出入半弧背一十九度九六一四,以二至黃赤道內外半弧弦二十三度七十一分除之,得八十四分一十九秒,為度差分。

求黃道每度書夜刻。術曰:置所求每度黃赤道內外半弧弦,以二至出入差半弧背乘之為實,二至黃赤道內外半弧弦為法除之,為每度出入差半弧背。又術:置黃赤道內外半弧弦,以度差八十四分一十九秒乘之,亦得出入差半弧背。置半徑內減黃赤道內外矢,即赤道二弦差,見前條立成。餘數倍之,又三因之,得數加一度,為日行百刻度。又術:以黃赤道內外矢倍之,以減全徑餘數,三因加一度,為日行百刻度,亦同。置每度出入半弧背,以百刻乘之為實,日行百刻為法除之,得數為出入差刻。置二十五刻,以出入差刻視黃道,在赤道內加之,在赤道外減之,得數為半晝刻,倍之為晝刻,以減百刻,為夜刻。

如求冬至後四度晝刻。術曰:置冬至後四十四度黃赤道內外半弧一十七度二十五分六十九秒,又為黃赤道小弧弦,前立成中取之。以二至出入差半弧背一十九度九十六分一十四秒乘之為實,以二至黃赤道內外半弧弦二十三度七十一分為法除之,得一十四度五十二分八十五秒,為出入半弧背。又法:置黃赤道內外半弧弦一十七度二五六九,以度差零度八四一九乘之,亦得一十四度五二八五,為出入半弧背。置半徑六十零度八七五,以四十四度黃赤道內外矢二度五十一分八十一秒又為赤道二弦差,前立成中取之。減之,餘五十八度三十五分六十九秒,即赤道小弦。倍之,得一百一十六度七十一分三十八秒,三因之,加一度,得三百五十一度一十四分一十四秒,為日行百刻度。又術:倍黃赤道內外矢得五度零三分六十二秒,以減全徑一百二十一度七十五分,亦得一百一十六度七十一分三十八秒,三因加一度,為日行百刻度,亦同。置出入半弧背一十四度五十二分八十五秒,以百刻乘之為實,以日行百刻度三百五十一度一十四分一十四秒為法除之,得四刻一十三分七十五秒,為出入差刻。置二十五刻,以出入差刻四刻一十三分七十五秒減之,因冬至後四十四度,黃道在赤道外,故減。餘二十零刻八十六分二十五秒,為半晝刻。倍之得四十一刻七十二分半,為晝刻。以晝刻減百刻,餘五十八刻二十七分半,為夜刻。又術:置出入差刻四刻一十三分七十五秒,倍之,得八刻二十七分半,以減春秋分晝夜五十刻,得四十一刻七十二分半,為晝刻。以倍刻加五十刻,得五十八刻二十七分半,為夜刻。晝減故廢加,餘仿此。

表格略

右《曆草》所載晝夜刻分,乃大都即燕京晷漏也。夏晝、冬夜極長,六十一刻八十四分,冬晝、夏夜極短,三十八刻一十六分。明既遷都於燕,不知遵用。惟正統己巳奏準頒曆用六十一刻,而群然非之。景泰初仍復用南京晷刻,終明之世未能改正也。


大統曆法二立成

立成者,以日月五星盈縮遲疾之數,預為排定,以便推步取用也。《元志》、《曆經》步七政盈縮遲疾,皆有二術。其一術以三差立算者,即布立成法也。其又術云,以其下盈縮分,乘入限分萬約之,以加其下盈縮積者,用立成法也。而遣立成未載,無從入算。今依《大統曆通軌》具錄之。其目四:曰太陽盈縮,曰晨昏分,曰太陰遲疾,曰五星盈縮。餘詳《法原》及《推步》卷中。按《元史》,至正十七年《授時曆》成。十九年王恂卒,時曆雖頒,然立成之數尚皆有定槁。郭守敬比類編次,整齊分秒,裁為二卷。而今欽天監本,載嘉議大夫太史令臣王恂奉敕撰。意者王先有槁,而郭卒成之歟?

太陽盈初縮末限立成冬至前後二象限同用

表格略

晨分加二百五十分,為日出分。日周一萬分,內減晨分為昏分。昏分減二百五十分,為日入分,又減五千分,為半晝分。故立成只列晨昏分,則出入及半晝分皆具,不必盡列也。

以下表格略

大統曆法三上推步

大統推步,悉本《授時》,惟去消長而已。然《通軌》諸捷法,實為布算所須,其間次序,亦有與《曆經》微別者。如氣朔發斂,《授時》原分二章,今古合為一。《授時》盈縮差在日躔,遲疾差在月離,定朔、經朔離為二處。今則經朔後,即求定朔,於用殊便。其目七:曰氣朔,曰日躔,曰月離,曰中星,曰交食,曰五星,曰四餘。

▲步氣朔發斂附

洪武十七年甲子歲為元。上距至元辛巳一百零四算。

歲周三百六十五萬二千四百二十五分,實測無消長。半之為歲周,四分之為氣象限,二十四分之為氣策。

日周一萬。即一百刻,刻有百分,分有百秒,以下微纖,皆以百遞析。

氣應五十五萬零三百七十五分。

置距算一百零四,求得中積三億七千六百一十九萬九千七百七十五分,加辛巳氣應五十五萬零六百分,得通積三億七千六百七十五萬零三百七十五分,滿紀法六十去之,餘為《大統》氣應。

開應一十八萬二千零百七十零分一十八秒。

置中積,加辛巳閏應二十零萬二千零五十分,得閏積三億七千六百四十零萬一千八百二十五分,滿朔實去之,餘為《大統》閏應。

轉應二十零萬九千六百九十零分。

置中積,加辛巳轉應一十三萬零二百零五分,共得三億七千六百三十二萬九千九百八十分,滿轉終去之,餘為《大統》轉應。

交應一十一萬五千一百零五分零八秒。

置中積加辛巳交應二十六萬零三百八十八分,共得三億七千六百四十六萬零一百六十三分,滿交終去之,餘為《大統》交應。

按《授時曆》既成之後,閏轉交三應數,旋有改定,故《元志》、《曆經》閏應二十零萬一千八百五十分,而《通軌》載閏應二十零萬二千零五十分,實加二百分,是當時經朔改早二刻也。《曆經》轉應一十三萬一千九百零四分,《通軌》載轉應一十三萬零二百零五分,實減一千六百九十九分,是入轉改遲一十七刻弱也。《曆經》交應二十六萬零一百八十七分八十六秒,《通軌》交應二十六萬零三百八十八分,實加二百分一十四秒,是正交改早二刻強也。或以《通軌》辛巳三應,與《元志》互異,目為元統所定,非也。夫改憲必由測驗,即當具詳始末,何反追改《授時曆》,自沒其勤乎?是故《通軌》所述者,乃《授時》續定之數,而《曆經》所存,則其未定之初槁也。

通餘五萬二千四百二十五分。

朔策二十九萬五千三百零五分九十三秒,一名朔寶。半之為望策,一名交望。又半之為弦策。

通閏一十零萬八千七百五十三分八十四秒。

月閏九千零百六十二分八十二秒。

閏限一十八萬六千五百五十二分零九秒。一名閏準。

盈初縮末限八十八萬九千零百九十二分二十五秒。

縮初盈末限九十三萬七千一百二十零分二十五秒。

轉終二十七萬五千五百四十六分,半之為轉中。

朔轉差一萬九千七百五十九分九十三秒。

日轉限一十二限二十。

轉中限一百六十八限零八三零六零。以日轉限乘轉中。一名限總。

朔轉限二十四限一零七一一四六。以日轉限乘朔轉差。

弦轉限九十零限零六八三零八六五。以日轉限乘弦策。一名限策。

交終二十七萬二千一百二十二分二十四秒。

朔交差二萬三千一百八十三分六十九秒。

氣盈二千一百八十四分三十七秒五十微。

朔虛四千六百九十四分零七秒。

沒限七千八百一十五分六十二秒五十微。

盈策九萬六千六百九十五分二十八秒。

虛策二萬九千一百零四分二十二秒。

土王策三萬零四百三十六分八十七秒五十微。

宿策一萬五千三百零五分九十三秒。

紀法六十萬。即旬周六十日。

推天正冬至置距洪武甲子積年減一,以歲周乘之為中積,加氣應為通積,滿紀法去之,至不滿之數,為天正冬至。以萬為日,命甲子算外,為冬至日辰。累加通餘,即得次年天正冬至。

推天正閏餘置中積,加閏應,滿朔策去之,至不滿之數,為天正閏餘。累加通閏,即得次年天正閏餘。

推天正經朔置冬至,減閏餘,遇不及減,加紀法減之,為天正經朔。無閏加五十四萬三六七一一六。十二朔策紀法。有閏,加二十三萬八九七七零九。十三朔實去紀法。滿紀法仍去之,即得次年天正經朔視天正閏餘在閏限已上,其年有閏月。

推天正盈縮置半歲周,內減其年閏餘全分,餘為所求天正縮曆。如徑求次年者,於天正縮曆內減通閏,即得。減後,視在一百五十三日零九已下者,復加朔實,為次年天正縮曆。

推天正遲疾置中積,加轉應,減去其年閏餘全分,餘滿轉終去之,即天正入轉。視在轉中已下為疾曆,已上去之為遲曆。如徑求次年者,加二十三萬七一一九一六,十二轉差之積。經閏再加轉差,皆滿轉終去之,遲疾各仍其舊。若滿轉中去之,為遲疾相代。

推天正入交置中積,減閏餘,加交應,滿交終去之,即天正入交凡日。如徑求次年者,加六千零八十二分零四秒,十二交差內去交終。經閏加二萬九千二百六十五分七十三秒,十三交差內去交終。皆滿交終仍去之,即得。

推各月經朔及弦望置天正經朔策,滿紀法去之,即得正月經朔。以弦策累加之,去紀法,即得弦望及次朔。

推各恒氣置天正冬至,加三氣策,滿紀法去之,即得立春恒日。以氣策累加之,去紀法,即得二十四氣恒日。

推閏在何月置朔策,以有閏之年之閏餘減之,餘為實,以月閏為法而一,得數命起天正次月算外,即得所閏之月。閏有進退,仍以定朔無中氣為定。如減餘不及月閏,或僅及一月閏者,為閏在年前。

推各月盈縮曆置天正縮曆,加二朔策,去半歲周,即得正月經朔下盈曆。累加弦策,各得弦望及次朔,如滿半歲周去之交縮,滿半周又去之即復交盈。

推初末限視盈曆在盈初縮末限已下,縮曆在縮初盈末限已下,各為初。已上用減半歲周為末。

推盈縮差置初末曆小餘,以立成內所有盈縮加之乘之為實,日周一萬為法除之,得婁數以加其下盈縮積,即盈縮差。

推各月遲疾曆置天正經朔遲疾曆,加二轉差,得正月經朔下遲疾曆。累加弦策,得弦望及次朔,皆滿轉中去之,為遲疾相代。

推遲疾限各置遲次曆,以日轉限乘之,即得限數。以弦轉限累加之,滿轉中限去之,即各弦望及次朔限。如徑求次月,以朔轉限加之,亦滿轉中去之,即得。又法:視立成中日率,有與遲疾曆較小布相近者以減之,餘在八百二十已下,即所用限。

求遲疾差置遲疾曆,以立成日率減之,如不及減,則退一位。餘以其下損益分乘之為實,八百二十分為法除之,得數以加其下遲疾積,即遲疾差。

推加減差視經朔弦望下所得盈縮差、遲疾差,以盈遇遲、縮遇疾為同相併,盈遇疾、縮遇遲為異相較,各以八百二十分乘之為實,再以遲疾限行度內減去八百於二十分,為定限度為法,法除實為加減差。盈遲為加,縮疾為減,異名相較者,盈多疾為加,疾多於盈為減,縮多於遲減,遲多於縮加。

推定朔望各置經朔弦望,以加減差加減之,即為定日。視定朔干名,與後朔同者月大,不同者月小,內無中氣者為閏月。其弦望在立成相同日日出分已下者,則退一日命之。

推各月入交置天正經朔入交凡日加二交差,得正月經朔下入交凡日。累加交望,滿交終去之,即得各月下入交凡日。徑求次月,加交差即得。

推土王用事置穀雨、大暑、霜降、大寒恒氣日,減土王策,如不及減,加紀法減之,即各得土王用事日。

推發斂加時各置所推定朔弦望及恒氣之小餘,以十二乘之,滿萬為時,命起子正。滿五千,又進一時,命起子初。算外得時不滿者,以一千二百除之為刻,命起初刻。初正時之刻,皆以初一二三四為好,於算外命之。其第四刻為畸零,得刻法三之一,凡三時成一刻,以足十二時百刻之數。

按古因及《授時》,皆以發斂為一章。發斂去者,日道發南斂北之細數也,而加時附焉,則又所以紀發斂之辰刻,故曰發斂加時也。《大統》取其便算,故合發斂與氣朔共為一章,或以乘除疏發斂,非其質矣。

推盈日視恒氣小餘,在沒限已上,為有盈之氣。置策餘一萬零一四五六二五,以十五日除氣策。以有盈之氣小餘減之,餘以六十八分六六以氣盈除十五日。乘之,得數以加恒氣大餘,滿紀法去之,命甲子算外,得盈日。求盈日及分秒,以盈策加之,又去紀法,即得。

推虛日視經朔小餘在朔虛已下,為有虛之朔。置有虛之朔小餘,以六十三分九一以朔虛除三十日。乘之,得數以加經朔大餘,滿紀法去之,命甲子算外為虛日。求次虛。置日及分秒,以虛策加之,又去紀法,即得。

推直宿置通積,以氣應加中積。減閏應,以宿會二十八萬累去之,餘命起翼宿算外,得天正經朔直宿。置天正經宿直宿,加兩宿策,為正月經朔直宿。以宿策累加,得各月經朔直宿。再以各月朔下加減差加減之,為定朔直宿。

▲步日躔

周天三百六十五度二十五分七十五秒,半之為半周天,又半之為象限。

歲差一分五十秒。

周應三百一十五度一十分七十五秒。

按此係至元辛巳之周應,乃自虛七度至箕十之度數也。洪武甲子相距一百四年,歲差已退天五十四分五十秒,而周應仍用舊數,殆傳習之誤耳。

推天正冬至日躔赤道宿次置中積,加周應,應減距曆元甲子以來歲差。滿周天去之,不盡,起虛七度,依各宿次去之,即冬至加時赤道日度。如求次年,累減歲差,即得。

表格略

推天正冬至日躔黃道宿次置冬至加時赤道日度,以至後赤道積度減之,餘以黃道率乘之。如赤道率而一,得數以加黃道積度,即冬至加時黃道日度。黃赤道積度及度率,俱見《法原》。

表格略

推定象限度以冬至加時赤道日度,與冬至加時黃道日度相減,為黃赤道差。以本年黃赤道差,與次年黃赤道相減,餘以四而一,加入氣象限內,為定象限度。

推四正定氣日置所推冬至分,即為冬正定氣,加盈初縮末限,滿紀法去之,餘為人正定氣。加縮初盈末限,去紀法,餘為秋正定氣。加縮初盈末限,去紀法,餘為次年冬正定氣。

推四正相距日以前正定氣大餘,減次正定氣大餘,加六十日,得相距日。如次正氣不及減者,加六十日減之,再加六十日,為相距日。

推四正加時黃道積度置冬至加時黃道日度,累加定象限,各得四正加時黃道積度。

推四正加時減分置四正定氣小餘,以其初日行度乘之,如日周而一,為各正加時減分。

冬正行一度零五一零八五。春正距夏正九十三日者,行零度九九九七零三,距九十四日者行一度。夏正行零度九五一五一六。秋正距冬正八十八日者,行一度零零零五零五,距八十九日者行一度。

推四正夜半積度置四正加時黃道積芭,減去其加時減分,即得。

推四正夜半黃道宿次置四正夜半黃道積度,滿黃道宿度去之,即得。

推四正夜半相距度置次正夜半黃道積度,以前正夜半黃道積度減之,餘為兩正相距度,遇不及減者,加周天減之。

推四正行度加減日差雙相距度與相距日下行積度相減,餘如相距日而一,為日差。從相距度人減去行積度者為加,從積度內減去相距度者為減。

秋正距冬至,冬至距春正八十八日,行積度九十度四零零九,八十九日行積度九十一度四零一四。春正距夏至,夏至距秋秋正九十三日,行積度九十度五九九零,九十四日行積十五九八七。

推每日夜度置四正後每日行度,在立成。以日差加減之,為每日行定度。置四正夜半日度,以行定度每日加之,滿黃道宿度去之,即每日夜半日度。

黃道十二次宿度

危十二度六四九一,入娵訾,辰在亥。

奎一度七三六二,入降婁,辰在戍。

奎度四五六,入大梁,辰在酉。

胃三七度七四五六,入大梁,辰在酉。

畢六度八八零五,入實沈,辰在申。

井八度三四九四,入鶉首,辰在未。

柳三度八六八零,入鶉火,辰在午。

張十五度二六零六,入鶉尾,辰在巳。

軫十度零七九七,入壽星,辰在辰。

氐一度一四五二,入大火,辰在卯。

尾三度一一五,入析木,辰在寅。

斗三度七六八五,入星紀,辰在丑。

女二度零六三八,入玄枵,辰在子。

推日躔黃道入十二次時刻置入次宿度,以入次日夜,以入次日夜半日度減之,餘以日周乘之,一分作百分。為實。以入次日夜半日度,與明日夜半日度相減,餘為法。實如法而一,各數,以發斂加時求之,即入次時刻。

▲步月離

月平行度一十三度三十六分八十七秒半。

周限三百三十六、半之為中限,又半之為初限。

限平行度零九分六十二秒。

太陽限行八分二十秒。

上弦九十一度三十一發四十三秒太。

望一百八十二度六十二分八十七秒半。

下弦二百七十三度九十四分三十一秒少。

交終度三百六十三度七十九分三十四秒一九六。

朔平行度三百九十四度七八七一一五一六八七五。

推朔後平交日置交終分,風氣朔曆。減天正經朔交凡分,為朔後平交日。如推次月,累減交差二日三一八六九,得次月朔平交日。不及減交差者,加交終減之,其交又在本月,為重交月朔後平交日。每歲必有重交之月。

推平交入轉遲疾曆置經朔遲疾曆,加入朔後平交日為平交入轉。在轉中已下,其遲疾與經朔同,已上減去轉中疾交遲,遲交疾。如推次月,累減交轉差三千四百二十三分七六,交差內減轉差數。即得。如不及減,加轉中減之,亦遲疾相代。

推平交入限遲疾差置平交入轉遲疾曆,依步氣朔內,推遲疾差,那得。

推平交加減定差置平交入限遲疾差,雙日率八百二十分乘之,以所入遲疾限下行度而一,即得。在遲為加,在疾為減。

推經朔加時積置經朔盈縮曆,見步氣朔內。在盈曆即為加時中積,在縮曆加半歲周。如推次月,累加朔策,滿歲周去之,即各朔加時中積,命日為度。若月內有二交,後交即注前交經朔加時中積。

推正交距冬至加時黃道積度及宿次置朔後平交日,以月平行乘之為距後度,加以經朔加時中積,為各月正交距冬至加時黃道積度。加冬至加時黃道日度,見日躔。以黃道積度鈐減之,至不滿宿次,即正交月離。如推次月,累減月平交朔差一度四六三一零二。以交終度減天周,其數宜為一度四六四零八零。遇重交月,同次朔。後仿此。

▲黃道積度鈐

表格略

推正交日辰時刻置朔後癥交日,加經朔,去紀法,以平交定差加減之,其日命甲子算外,小餘依發斂加時求之,即得正交日辰時刻。如推次月,累加交終,滿紀去之。如遇重交,再加交終。

推四正赤道宿次置冬至赤道日度,以氣象限累加之,滿赤道積度去之,為四正加時赤道日度。

▲赤道積度鈐

表格略

推正交黃道在二至後初末限置正交距冬至加時黃道積度,在半歲周已下為冬至後,已上減去半歲周,餘為夏至後。又視二至後度分,在氣象限已下為初限,已上用減半歲周,餘為末限。推次月者,若本月初限,則累減月平交朔差,餘為次月初限。不及減者,反減月平交朔差,餘為次月末限。若本月末限則累加月平交朔差,為次月天限,至滿氣象限,以減半歲周,餘為次月初限。

推定差度置初末限,以象極總差一分六零五五零八乘之,即為定差度。象極總差,是以象限除極差,其數宜為一十六分零五四四二。如推次月初限則累減,末限則累加,俱以極平差二十三分四九零二加減之。極平差,是以月平交朔差,乘象極總差,其數宜為二十三分五零四九。

推距差度置極差十四度六六,減去定差度,即得。求次月,以極平差加減之。初限加,末限減。

推定限度置定差度,以定極總差一分六三七一零七乘之,定極總差,是以極差除二十四度,其數宜為一度六三七一零七。所得視正交在冬至後為減,夏至後為加,皆置九十八度加減之,即得。

推月道與赤道正交宿度正交在冬至後,置春正赤道積度,以距差度初限加末限減之,在夏至後,置秋正赤道積度,以距差初限減末限加之。得數,滿赤道積度鈐去之,即得。

推月道與赤道正交後積度并入初末限視月道與赤道正交所入某宿次,即置本宿赤道全度,減去月道與赤道正交宿度,差為正後積度。以赤道各宿全度累中之,滿氣象限去之,為半交後。又滿去之,為中交後。再滿去之,為半交後。視各交積度,在半象限以焉為初限,以上覆減象限,餘為末限。

推定差置每交定限度,與初末限相乘,得數,千約之為度,即得。正交、中交後為加,半交後為減。

推月道定積度及宿次置月道與赤道各交後每宿積度,以定差加減之,為各交月道積度。加月道與赤道正交定宿度,共為正交後宿度。以前宿定積度減之,即得各交月道宿次。

▲活象限例

置正交後宿次,加前交後半交末宿定積度。為活象限。如正交後宿次度少,加前交不及數,卻置正交後宿次加氣象限即是。如遇換交之月,置正交後宿次,以前交前半交末宿定積度加之,為換交活象限。假如前交正交是軫,後交正交是角,其前交欠一軫。求活象限者,置正交後宿次,不從翼下取定積度加之,仍於軫下取定積度也。又如前交、正交是軫,後交、正交是翼,其前交多一翼。求活象限者,置正交後宿次,不從翼下取定積度加之,仍於張下取定積度也。

推相距日置定上弦大餘,減去定朔大餘,即得。上弦至望,望至下弦,下弦至朔仿此。不及減者,加紀法減之。

推定朔弦望入盈曆及盈縮定差置各月朔弦望入盈縮曆,以朔弦望加減差加減之,並在步氣朔內。為定盈縮曆。視盈曆在盈初限下為盈初已上用減半歲周,餘為盈末限。縮曆在縮初限已下為縮初限,已上用減半歲周,餘為縮末限。依步氣朔內求盈縮差,為盈縮定差。

推定朔弦望加時中積置定盈縮曆,如是盈曆在朔,便為加時中積,在上弦加氣象限,在望加半歲周,在下弦加三象限。如是縮曆在朔,加半歲周。在上弦加三象限,在望便為加時中積,在下弦加氣象限,加後滿周天去之。

推黃朔弦望加時中定積度置定朔弦望加時中積,以其下盈縮定差盈加縮之,即得。

推赤道加時積度及宿次置黃道加時定積度,在周天象限已下為至後,已上去之為分後,滿兩象限去之為至後,滿三象限去之為分後。置分至後黃道積度,以立成內分至後積度減之,餘以其下赤道度率乘之,如黃道度率而一,得數加入分至後積度,次以所去象限合之,為赤道加時定積度。置赤度加時定積度,加入天正冬至加時赤道日度,滿赤道積度鈐去之,得定朔弦望赤道加時宿次。

推正半合交後積度置定朔弦望加時赤道宿次,視朔弦望在何交後,正半、中半。即以交生積度,在朔望加時赤道宿前一宿者加之,即為正半中交後積度,滿氣象限去之,為正半中換交。

推初末限視正半中交後積度,在半象已下為初限,已上覆減氣象限,餘為末限。

推月道與赤道定差置其交定限度,與初末限相減相乘,所得,千約之為度,即定差。在正交、中交為加。在半交為減。

推定朔弦望加時月道宿次置定朔弦望加時月道定積度,取交後月道定積度,取交後月道定積度,在所置罕前一宿者減之,即得。遇轉交則前積度多,所置積度少為不及減。從半轉正,加其交活象限減之。從正轉半,從半轉中,從中轉半,皆加氣象限減之。

推夜半入轉日置經朔弦望遲疾曆,以定朔弦望加減差加減之。大疾曆,便為定朔弦望加時入轉日。在遲曆,用加轉中置定朔弦望加時入轉日,以定朔弦望小餘減之,為夜半入轉日,遇入轉日少不及減者,加轉終減之。

推加時入轉度置定朔弦望小餘,去秒,取夜半入轉日下轉定度乘之,萬約之為分,即得。

▲遲疾轉定度鈐

表格略

推定朔弦望夜半入轉積度及宿次置定朔弦望加時月道定積度,減去加時入轉度,為夜半積度。如朔弦望加時定積度初換交,則不及減,半正相接,用活象限,正半、中半相接,用氣象限加之,然後減加時入轉度,則正者為後年,後年為中,中為前半,前半為正。置朔弦望夜半月道定積度,依推定朔弦望加時月道宿次法減之,為夜半宿次。

推晨昏入轉日及轉度置夜半入轉日,以定盈縮歷檢立成日下晨分加之,為晨入轉日滿轉終去之。置其日晨分,取夜半入轉日下轉定度乘之,萬約為分,為晨轉度。如求昏轉日轉度,依法檢日下昏分,即得。

推晨昏轉積度及宿次置朔弦望夜半月道定積度,加晨轉度,為晨轉積度。如求昏轉積度,則加昏轉度,滿氣象限去之,則換交。若推夜半積度之時,因朔弦望加時定積不及減轉度,以半正相接,而加活象限之者,今復換正交,則以活象限減之。置晨轉積度,依前法減之,為晨分宿次。置昏轉積度,依法減之,為昏分宿次。

推相距度朔與上弦相距,上弦與望相距,用昏轉積度。望與下弦相距,下弦與朔相距,用晨轉積度。置後段晨昏轉積度,視與前段同交者,竟以前段晨昏轉積度減之,餘為相距度。若後段與前段接兩交者,從正入半,從半入中,從中入半,加氣象限。從半入正,加活象限。然後以前段晨昏轉積度減之。若後段與前段接三交者,其內無從半入正,則加二氣象限,其內有從半入正,則加一活象限,一氣象限,以前段晨昏轉積度減之。

推轉定積度置晨昏入轉日,朔至弦,弦至望,用昏。望至弦,弦至朔,用晨。以前段減後段,不及減者,加二十八日減之,為晨昏相距日。從前段下,於鈐內驗晨昏相距日同者,取其轉定積度。若朔弦望相距日少晨昏相距日一日者,則於晨昏相距日同者,取其轉積度,減去轉定極差一十四度七一五四,餘為前段至後段轉定積度。

▲轉定積度鈐

以下表格略

推加減差以相距度與轉定積度相減為實,以其朔弦望相距目為法除之,所得視相距度多為加差,少為減差。

推每日太陰行定度置朔弦望晨昏入轉日,視遲疾轉定度鈐日下轉定度,累日以加減差加減之,至所距日而止,即得。

推每日月離晨昏宿次置朔弦望晨昏宿次,以每日太陰行度加之,滿月道宿次減之,即得。

▲赤道十二宮界宿次

表格略

推月與赤道正交後宮界積度視月道與赤道正交後,各宿積度宮界,某宿次在後,即以加之,便為某宮正交後宮界積度。求次宮者,累加宮率二十度四三八一,滿氣象限去之,各得某宮下半產交後宮界積度。

推宮界定積度視宮界度在半象限已下為初限,已上覆減氣象限,餘為末限。置某交定限度,與初末限相減、相乘,所得,千約之為度,在正交、中交為加差,在半交為減差。置宮界正半中交後積度,以定差加減之,為宮界定積度。

推宮界宿次置宮界定積度,於月道內取其在所置前一宿者減之之不及減者,加氣象限減之。

推每月每日下交宮時刻置每月宮界宿次,減入交宮日下月離晨昏宿次。如不及減者,加宮界宿次前宿減之,餘以日周乘之,以其日太陰行定度而一,得數,又視定盈縮曆取立成日下晨昏分加之。晨加晨分,昏加昏分。

如滿日周交宮在次日,不滿在本日,依發斂推之,即交宮時刻。

▲步中星

推每日夜半赤道置推到每日夜半黃道,見日躔。依法以黃道積度減之,餘如黃道率而一,以加赤道積度。又以天正科至赤道加之,如在春正後,再加一象限,夏至後加半周天,秋正後加三象限,為每日夜半赤道積度。

推夜半赤道宿度置夜半赤道度,以赤道宿度挨次減之,為本日夜半赤道宿度。

推晨距度及更差度置立成內每日晨分,以三百六十六度二十五分七十五秒乘之為實,如日周而一,為晨距度。倍晨距度,以五除之,為更差度。

推每日夜半中星置推到每日夜半赤道宿度,加半周天,即夜半中唾積度。以赤道度挨次減之,為夜半中星宿度。

推昏旦中星置夜半中星積度,減晨距度,為昏中星積度。以更差度累加之,為遂更及旦中星積度。俱滿赤道宿度去之,即得。以晨分五之一,加倍為更率。更率五而一為點率。凡昏分,即一更一點,累加更率為各更。凡交更即為一點,累加點率為各點。

大統曆法三下推步

▲步交食

交周日二十七日二十一刻二二二四。半之為交中日。

交終度三百六十三度七九三四一九六。半之為交中日度。

正交度三百五十七度六四。

中交度一百八十八度零五。

前準一百六十六度三九六八。

後準一十五度五。

交差二日三一八三六九。

交望一十四日七六五二九六五。

日食陽曆限六度。定法六十。

日食陰曆限八度。定法八十。

月食十三度五分。定法八十七。

陽食限視定朔入交。

零日六零已下一十三日一零已上在一十四日,不問小餘,皆入食限。

一十五日二零已下二十五日六零已上在二十六日、二十七日,不問小餘,皆入食限。

▲陰食限視定望入交。

一日二零已下一十二日四零已上在零日一十三日,不問小餘,皆入食限。又視定朔小餘在日出前、日入後二十分已上者,日食在夜。定望小餘在日入前、日出後八刻二十分已上者,月食在晝。皆不必布算。

推日食用數

經朔盈縮曆盈縮差遲疾曆遲疾差加減差定朔入交凡分以上皆全錄之。定入遲疾曆以加減差,加減遲疾即是。遲疾定限置定入遲疾曆,以日轉限一十二限二十分乘之,小餘不用。定限行度以定限,取立成內行度,遲用遲,疾用疾,內減日行分八分二十秒,得之。日出分以盈縮曆,從立成內取之,下同。日入分半晝分取立成內昏分,減去五千二百五十分,得之。歲前冬至時黃道宿次

推交常度置有食之朔入交凡分,以月平行度乘之,即得。

推交定度置交常度,以朔下盈縮差盈加縮減之,即得。

推日食正交限度視交定度在七度已下,三百四十一度已上者,食在正交。在一百七十五度已上,二百零二度已下者,食在中交。不在限內不食。

推中前中後分視定朔小餘,在半日周已下,用減半日周,餘為中前分。在半日周已上,減去半日周,餘為中後分。

推時差置半日擊,以中前、中後分減之,餘以中後分乘之,所得以九千六百而一為時差。在中前為減,中後為加。

推食甚定分置定朔小餘,以時差加減之,即得。

推距午定分置中前、中後分,加時差即得。但加不減。

推食甚入盈縮曆置原得盈縮曆,加入定朔大餘及食甚定分,即得。

推食甚盈縮差依步氣朔求之。

推食甚入盈縮曆行定度置食甚入盈縮曆,盈縮差,盈加縮減之,即得。

推南北凡差視食甚人盈縮曆行定度,在周天象限已下為初限,已上與半歲周相減為末限。以初末限自之,如一千八百七十度而一,得數,置四度四十六分減之,餘為南北凡差。

推南北定差置南北凡差,以距午定分乘之,如半晝分而一,以減凡差,餘為南北定差。若凡差數少,即反減之。盈初縮末食在正交為減,中交為加。縮初盈末,食在正交為加,中交為減。如系凡差反減而得者,則其加減反是。

推東西凡差置半歲周,減去食甚入盈縮曆行定度,餘食甚入盈縮曆行定度乘之,以一千八百七十除之為度,即東西凡差。

推東西定差置東西凡差,以距午定分乘之,如二千五百度而一,視得數在東西凡差以下,即為東西定差。若在凡差已上,倍凡差減之,餘為定差。盈曆中前,縮曆後者,正交減,中交加。盈曆中後,縮中前者,正交加,中交減。

推正交中定限度視日食在正交者置正交度,在中交者置中交度,以南北東西二定差加減之,即得。

推日食入陰陽曆去閃前後度視交定在正交定限度已下,減去交定度,餘為陰曆交前度。已上,減去正交定限度,餘為陽曆交後度。在中交定限度已下,減去交定度,餘為陽曆閃前度。已上,減去中交定限度,餘為陰曆後度。若交定在七度已下者加交終度,減去正交定限度,餘為陽曆交後度。

推日食分秒在陽曆者,置陽食限六度,減去陽曆交前、交後度,不及減者,不食。陰曆同。餘以定法六十而一。在陰曆者,置陰食限八度,減去陰曆交前、交後度,餘以定法八十而一,即得。

推定用分置日食分秒與二十分相減相乘,為開方積。以平方法開之,為開方數。用五千七百四十分七因八百二十分也。乘之,如定限行度而一,即得。

推初虧復圓時刻置食甚定分,以定用分減為初虧,加為復圓。各依發斂加時,即時刻。

推日食起復方位陽曆初虧西南,甚於正南,復於東南。陰曆初虧西北,甚於正北,復於東北。若在八分以上,不分陰陽曆皆虧正西,復東位。據午地而論

推食甚日躔黃道宿次置食甚入盈縮曆行定度,在盈就為定積度,在縮加半歲周為定積度。置定積度,以歲前冬至加時黃道日度加之,滿黃道積度鈐去之,至不滿宿次即食甚日躔。

推日帶食視初虧食甚分,有在日出分已下,為晨刻帶食。食甚復圓分,有在日入分已上,為昏刻帶食。在晨置日出分,在昏昏置日入分,皆以食甚分與之相減,餘為帶食差。置帶帶差,以日食分秒乘之,以定用分而一,所得減日食分秒,餘為所見帶食分秒。

▲推月食用數

經望盈縮曆盈縮差遲疾曆

遲疾差加減差定望入交凡分

定入遲疾曆定限定限行度晨分

日出分昏分日入分限數

▲歲前冬至加時黃道宿次

推交常度置望下入交凡分,乘月平行,如日食法。

推交定度置交常度,以望下盈縮差盈加縮減之即得。不及減者,加交終度減之。

推食甚定分不用時差,即以定望分為食甚分。

推食甚入盈縮曆行定度法同推日食。

推月食入陰陽曆視交定度在交中度已下為陽曆,已上減去交中度,餘為隊曆。

推交前交後度視所得入陰陽曆,在後準已下為交後,在前準已上置交中度減之,餘為交前。

推月食分秒置月食限一十三度零五,減去前交後度,不及減者不食。餘以定法八十七分而一,即得。

推月食用分置三十分,與月食分秒相減相乘,為開方積。依平方法開之,為開方數。又以四千九百二十乃六因八百二十分數。分乘之,如定限行度而一,即得。

推月食三限初虧、食甚、復圓。時刻置食甚分定分,以用分減為初虧,加為復圓。依發斂得時刻如日食。

推月食五限時刻月食十分已上者,用五限推之,初虧、食既、食甚、生光、復圓也。置月食分秒,減去十分,餘與十分相減相乘,為開方積。平方開之,為開方數。又以四千九百二十分乘之,如定限行度而一為既內分。與定用分相減,餘為既外分。置食甚定分,減既內分為既分,又減既外分為初虧分。再置食甚定分,加既內分為生光分,又加既外分為復圓分。各依以斂得時刻。

推更點置晨分倍之,五分之為更法,又五分之為點法。

推月食入更點各置三限或五限,在昏分已上減去昏分,在晨分已下加入晨分,不滿更法為初更,不滿點法為一點,以次求之,各得更點之數。

推月食起復方位陽曆初虧東北,甚於正北,復於西北。陰曆初虧東南,甚於正南,復於西南。若食在八分已上者,皆初虧正東,復於正西。

推食甚月離黃道宿次置食甚入盈縮曆定度,在盈加半周天,在縮減去七十五秒為定積度。置定積度,加歲前冬至加時黃道日度,以黃道積度鈐去之,即得。

推月帶食視初虧、食甚、復圓等分,在日入分以下,為昏刻帶食。在日出分已上,為晨刻帶食。推法同日食。

▲步五星

曆度三百六十五度二五七五,半之為曆中,又半之為曆策。

△木星

合應二百四十三萬二三零一。置中積三億七千六百一十九萬七七五,加辛巳合應一百一十九七二六,得三億七行七百三十七萬九五零一,滿木星周率去之,餘為《大統》合應。

曆應五百三十八萬二五七七二二一五。置中積,加辛巳曆應一千八百九十九萬九四八一,得三億九千五百一十九萬娥二五六,滿木星曆率去之,餘為《大統》曆應。

周率三百九十八萬八八。

曆率四千三百三十一萬二九六四八六五。

度率一十一萬八五八二。

伏見一十三度。

段目段日平度限度初行率

合伏一十六日八六三度八六二度九三二十三分

晨疾初二十八日六度二一四度六四二十二分

晨疾末二十八日五度五一四度六四二十二分

晨遲初二十八日四度三一三度二八一十八分

晨遲末二十八日一度九一一度四五一十二分

晨留二十四日

晨退四十六日五八四度八八一二五零度三二八七五

夕退四十六日五八四度八八一二五零度三二八七五一十六分

夕留二十四日

夕遲初二十八日一度九一一度四五

夕遲末二十八日四度三一三度二八一十二分

夕疾初二十八日五度五一四度一九一十八分

夕疾末二十八日六度一一四度六四二十一分

夕伏一十六日八六三度八六二度九三二十二分

△火星

合應二百四十零萬一四。置中積,加辛巳合應五十六萬七五四五,得三億七千六百七十六萬七三二,滿火星周率去之,為《大統》合應。中積見木星,五星並同。

曆應三百八十四萬五七八九三五。置中積,加辛巳曆應五百四十七萬二九三八,得三億八千一百六十七萬二七一三,滿火星曆率去之。

周率七百七十九萬九二九。

曆率六百八十六萬九五八零四三。

度率一萬八八零七五。

伏見一十九度。

段目段日平度限度初行率

合伏六十九日五十度四十六度五零七十三分

晨疾初五十九日四十一度八零三十八度八七七十二分

晨疾末五十七日三十九度零八三十六度三四七十分

晨次疾初五十三日三十四度一六三十一度七七六十七分

晨次疾末四十七日二十七度零四二十五度一五六十二分

晨遲初三十九日一十七度七二一十六度四八五十三分

晨初末二十九日六度二零五度七七三十八分

晨留八日

晨退二十八日六九四五八度六五六七五六度四六三二五

夕退二十八日九六四五八度六五六七五六度四六三二五四十四分

夕留八日

夕遲初二十九日六度二零五度七七

夕遲末三十九日一十七度七二一十六度四八三十八分

夕次疾初四十七日二十七度零四二十五度一五五十三分

夕遲疾末五十三日三十四度一六三十一度七七六十二分

夕疾初五十七日三十九度零八三十六度三四六十七分

夕疾末五十九日四十一度八零三十八度八七七十分

夕伏六十九日五十度四十六度五零七十二分

△土星

合應二百零六萬四七三四。置中積,加辛巳合應一十七萬五六四三,得三億七千六百三十七萬五四一八,滿土星周率去之。

曆應一億零六百零零萬三七九九零二。置中積,加辛巳曆應五千二百二十四萬零五六一,得四億二千八百四十四萬零三三六,滿土星曆率去之。

周率三百七十八萬零九一六。

曆率一億零七百四十七萬八八四五六六。

度率二十九萬四二五五。

伏見一十八度。

段目段日平度限度初行率

合伏二十日四零二度四零一度四九一十二分

晨疾三十一日三度四零二度一一一十一分

晨次疾二十九日二度七五一度七一一十分

晨遲二十六日一度五零零度八三八分

晨留三十日

晨退五十二日六四五八三度六二五四五零度二八四五五

夕退五十二日六四五八三度六二五四五零度二八四五五一十分

夕留三十日

夕遲二十六日一度五零零度八三

夕次疾二十九日二度七五一度七一八分

夕疾三十一日三度四零二度一一一十分

夕伏二十日四零二度四零一度四九一十一分

△金星

合應二百三十七萬九四一五。置中積,加辛巳合應五百七十一萬六三三零,得三億八千一百九十一萬六一零五,滿金星周率去之。

曆應一十零萬四一八九。置中積,加辛巳曆應一十一萬九六三九,得三億七千六百三十一萬九四一四,滿金星曆率去之。

周率五百八十三萬九零二六。

曆率三百六十五萬二五七五。

度率一萬。

伏見一十度半

段目段日平度限度初行率

合伏三十九日四十九度五零四十七度六四一度二七五

夕疾初五十二日六十五度五零六十三度零四一度二七五

夕疾末四十九日六十一度五十八度七一一度二五五

夕次疾初四十二日五十度二五四十八度三六一度二三五

夕次疾末三十九日四十二度五零四十度九零一度一六

夕遲初三十三日二十七度二十五度九九一度零二

夕初末一十六日四度二五四度零九六十二分

夕留五日

夕退一十日九五三一三度六九八七一度五九一三

夕退伏六日四度三五一度六三六十一分

合退伏六日四度三五一度六三八十二分

晨退一十日九五三一三度六九八七一度五九一三六十一分

晨留五日

晨遲初一十六日四度二五四度零九

晨遲末三十三日二十七度二十五度九九六十二分

晨次疾初三十九日四十二度五零四十度九零一度零二

晨次疾末四十二日五十度二五四十八度三六一度一六

晨疾初四十九日六十一度五十八度七一一度二三五

晨疾末五十二日六十五度五零六十三度零四一度二五五

晨伏三十九日四十九度五零四十七度六四一度二六五

△水星

合應三十零萬三二一二。置中積,加辛巳合應七十零萬零四三七,得三億七千六百九十零萬零二一二,滿水星周率去之。

曆應二百零三萬九七一一。置中積,加辛巳曆應二百零五萬五一六一,得三億七千八百二十五萬四九三六,滿水星曆率去之。

周率一百一十五萬八七六。

曆率三百六十五萬二五七五。

度率一萬。

晨伏夕見一十六度半。

夕伏晨見一十九度。

段目段日平度限度初行率

合伏一十七日七五三十四度二五二十九度零八二度一五五八

夕疾一十五日二十一度三八一十八度一六一度七零三四

夕遲一十二日一十度一二八度五九一度一四七二

夕留二日

夕退伏一十一日一八八七度八一二二度一零八

合退伏一十一日一八八七度八一二二度一零八一度零三四六

晨留二日

晨遲一十二日一十度一二八度五九

晨疾一十五日二十一度三八一十八度一六一度一四七二

晨伏一十七日七五三十四度二五二十九度零八一度七零三四

推五星前後合置中積,加合應,滿周率去之,餘為前合。再置周率,以前合減之,於為後合。如滿歲周去之,即其年無後合分。

推五星中積日中星度置各星後合,既為合伏下中積中星。命為日,曰中積。命為度,曰中星。累加段日,為各段中積。皆滿歲周去之。以各段下平度,累加各段下平度,滿歲周去。退則減之,不及減,加歲周減之。次復累加之,為各段中星。

推五星盈縮曆置中積,加曆應及生合,滿曆率去之,餘以度率而一為度。在曆中已下為盈,已上減去曆中為縮。置各星合伏下盈縮曆,以段下限度累加之之滿曆中去之,盈交縮,縮交盈,即各段盈縮曆。

推五星盈縮差置各段盈縮曆,以曆策除之為策數,不盡,為策餘。以其下損益分見立成。乘之,以曆策而一,所得益加損減其盈縮積分,即盈縮差。金星倍之,水星三之。

推定積日置各段中積,以其段盈縮差盈加縮減之,即得。滿歲周去之,如中積不及減者,加歲周減之。本段原無差者,借前段差加之,則金水二星,亦只用所得盈縮差,不用三之倍之。

推加時定日置定積日,以歲前天正冬至分加之,滿紀法去之,餘命甲子算外,即為定日。視定積日會滿歲周去者,用本年冬至,會加歲周減者,用歲前冬至。

推所入月日置合伏下定積,以加天正閏餘滿朔策除之,為月數。起歲前十一月,其不滿朔策者,即入月已來日分也。視其月定朔甲子,與加時定日甲子相去即合伏日,累加相距日,滿各月大小去之,即各段所入月日。

推定星置各段中星,依推定積日法,以盈縮差加減之。

推加時定星置定星,以歲前冬至加時黃道日度加之,滿周歲天去之。若定積日會加歲周者,用歲前黃道日度。遇減歲周者,用本年黃道目度,如原無中星度,段下亦無定星星及加時定星度分。

推加減定分置定日小餘,以其段初行率乘之,滿萬為分,所得諸段為減分,退段為加分。

推夜半定星及宿次置加時定星,以加減定分加減之,為夜半定星。以黃道積度鈐減之,為夜半宿次。其留段即用時定星,為夜半一星。

推日度率置各段定日,與次段定日相減為日率。次段不及減,加紀法減之。置各段夜半-定星,與次段夜半定星相減為度漲。次段不及減,加周天減之。凡近留之段,皆用留段加時定星,與本段夜半定星相減。如星度逆者,以後段減前段,即各得度率。

推平行分置度率,以日率除之,即得。

推凡差及增減總差日差以本段前後之平行分相減,為本段凡差。凡五星之伏段及近留之遲段及退段,皆無凡差。倍凡差,退一位為增減差。倍增減差為總差。置總差,以日率減一日除之為日差。初日行分多,為減差。末日行分多,為加差。

推初日行分末日行分以增減差加減其段平行分,為初末日行分。視本段平行分與次段平行分相較,前多後少者,加為初,減為末。前少後多者,減為初,加為末。

推撫心差諸段為增減差總差日差合伏者,置次段初日行分,加其日差之半,亦次段日差。為末日行分。晨伏、夕伏者,置前段本段之前。末日行分,加其日差之半,亦前段日差。為二伏初日行分。置伏段呼得初末日行分,皆與本段平行分相減,餘為增減差。又以增差加減平行分,為初末日行分。視合伏末日行全較平行分,少則加,多則減,為初日行分。晨伏、夕伏初日行分較平行分,亦少加多減,為末日行分。木、火之晨遲末,土之晨遲,金之夕遲末,水之夕遲,皆置其前末日行分,銳其日差減之,即前段日差。餘為初日行分。木、火之夕遲初,土之夕遲,金之晨初,水之晨遲,皆置其後段初日行分,倍其日差減之,後段日差。餘為末日行分。木、火、土之夕伏,金、水之晨伏,皆置其前段末日行分,內加其前段日差之半,為鈦段初日行分,皆與平行分相減,餘為增減差。木、火之晨退、夕退,置其平行分,退一位、六因之,為增減差。晨退減為初,加為末。夕退加為初,減為末。晨加夕減,二段相比較。金之夕退伏合伏,置其平行分,退一位,三因之折半。水之夕退伏合退伏,以平行分折半,各為增減差。金之夕退,置其平分,退一位,三在之折半。水之夕退伏合退伏,以平行分折半,各為增減差。金之夕退,置其後段祿日行分,減日差,後段日差。為末日行分。金之晨退,置其前段末日行分,減日差,前段日差。為初日行分。皆與平行分相減,餘為增減差。凡增減差,倍之為總差,以相距日率減一除之,為日差。其初末日行分有其一者,以增減差加減,更求其一,如伏段法,餘依前後平行分相較增減之。金、火之夕遲末,晨遲初,置其段平行分,以相距日率下不倫分乘之,不倫分之秒,與平行之分對。即為增減差。置平行分,夕者以增減差,加為初日行分,減為末日行分。晨者反是。

不倫分金、火星之夕遲末,與晨遲初,其增減差,多於平行分者,為不倫分也。

十七日八十八秒八八五

十六日八十八秒二三一

十五日八十七秒四九六

十四日八十六秒七六一

推五星每日細行,置各段夜半宿次,以初日行分順加退減之,為次日宿次。又以日差加減其初日行分,為每日行分,亦順加退減於次日宿次,滿黃道宿次去之,至次段宿次而止,為每日夜半宿次。

推五星順逆交宮時刻視逐日五星細行,與黃道十二宮界宿次同名,其度分又相近者以相減。視其餘分,在本日行分以下者,為交宮在本日也。順行者,以本日夜半星行宿次度分減宮界度分。退行者,以宮界度分減本日夜半星行宿次度分。扣以日周乘之為實,以本日行分為法,法除實,得數,依發斂加時法,得交宮時刻。

推五星伏見凡取伏見,伏者要在已下,見者要在已上。晨見晨伏者,置其日太陽行度,內減各星行度。夕見夕伏者,置其日各星行度,內減太陽行度。即為其日晨昏伏見度。置本日伏見度,與次日伏見度相減,餘四而一,即得晨昏伏見分。視本日伏見度較次日伏見度為多者減,少者加。晨者,置本日伏見度,以伏見分加減之,為晨伏見度。夕者,三因伏見分,置伏見度加減之,為夕伏見度。視在各星伏見度上下取之。

△步四餘

紫氣周日一萬零二百二十七日一七九二。

紫氣度率二十八日,日行三分五七一四二九。

紫氣至後策八千一百九十四萬九六二三。

月孛周日三千二百三十一日九六八四。

月孛度率八日八四八四九二,日行十一分三零一三六一。

月孛至後策一千二百二十萬四六五九。

羅計周日六千七百九十三日四四三二。

羅計度率一十八日五九九一零七七六,日行五分三七六六零二。

羅至後策五千三百三十三萬六二一七。

計都至後策一千九百三十六萬九零零一。

推四餘至後策置中積,加各餘至後策,滿周日去之,即得。

推四餘周後策以至後策,減立成內各宿初末度積日,即得。

推四餘入各宿次初末度積日置各餘周後策,加入其年冬至分,滿紀法去之,即各餘末度積日。紫氣、月孛為各宿初,羅喉、計都為各宿末。氣孛順行,羅計逆行。

推四餘初末度積日所入月日置各餘周後策,加入天正閏餘滿期策減之,起十一月至不滿朔策,即所入月也。其初末度積日即滿紀法去者。命甲子算外,為日辰小餘,以發斂求之為時刻。視定朔某甲女,即知入月已來日也。

推四餘每日行度置各餘初末度積日,氣孛以度率日累加之,至末度加其宿零日及分,即次宿之初度。羅計先加其宿零日及分,後以度率日累加之,即次宿之末度。徊以其大餘,命甲子算外為日辰。其交次宿,以小餘以斂為時刻。

推四餘交宮以至後策減各宿交宮積日,餘為入某宮積中天正閏餘,滿朔策去之,起十一月至不滿朔策,即所入月。又置入宮積日,加冬至分,滿紀法去之,為日辰,小餘以斂為時刻。視定朔甲子,即知交宮及時刻。

▲紫氣宿次日分立成入箕初度。

以下表格略

至後策少者用前氐下積日,多者用後氐下積日。

▲回回曆法一

《回回曆法》,西域默狄納國王馬哈麻所作。其地北極高二十四度半,經度偏西一百零七度,約在雲南之西八千餘昊。其曆元用隋開皇己未,即其建國之年也。洪武初,得其書於元都。十五年秋,太祖謂西域推測天象最精,其五星緯度又中國所無。命翰林李翀、吳伯宗同回回大師馬沙亦黑等譯其書。其法不用閏月,以三百六十五日為一歲。歲十二宮,宮有閏日,凡百二十八年而宮閏三十一日。以三百五十四日為一周,周一十十月有閏日。凡有閏閏凡百二十八年而而宮閏三十一日,以三百五十四日為一周,周十二月,月有閏日。凡三十年月閏十一日,歷千九百四十一年,宮月日辰再會。此其立法之大概也。

按西域曆術見於史者,在唐有《九執曆》,元有札馬魯丁之《萬年曆》。《九執因》最疏,《萬年曆》行之未久。唯《回回曆》設科,隸欽天監,與《大統》參用二百七十餘年。雖於交食之有無深淺,時有出入,然勝於《九執》、《萬年》遠矣。但其書多脫誤。盜蓋其人之隸籍臺官者,類以土盤布算,仍用其本國之書。而明之習其術者,如唐順之、陳壤、袁黃輩之所論著又自成一家言。以故翻譯之本不行於世,其殘缺宜也。今為博訪專門之裔,考究其原書,以補其脫落,正其訛舛,為《回回曆尖》,著於篇。

積年起西域阿喇必年,隋開皇己未。下至洪武甲子,七百八十六年。

用數天周度三百六十。每度六十分,每分六十秒,微纖以下俱準此。宮十二。每宮三十度。目周分一千四百四十,時二十四,每時六十分。刻九十六。每刻十五分。宮度起白羊,節氣首春分,命時起午正。午初四刻屬前日。

七曜數日一,月二,火三,水四,木五,金六,土七。以七曜紀不用甲子。

宮數白羊初,金牛一,陰陽二,世蟹三,獅子四,變女五,天秤六,天蠍七,人馬八,磨羯九,實寶瓶十,變魚十一。

宮日白羊戌宮三十一日。金牛酉宮三十一日。陰陽申宮三十一日。巨蟹未宮三十二日。獅子午宮三十一日。孌女巳宮三十一日。天秤辰宮三十一日。天蠍卯宮三十日。人馬寅宮二十九日。磨羯丑宮二十九日。寶瓶子宮三十日。變魚亥宮三十日。已上十二宮,所謂不動之月,凡三百六十五日,乃歲周之日也。若遇宮分有閏之年,于變魚宮加一日,凡三百六十六日。

月分大小單月大,變月小。凡十二月,所謂動之月也。月大三十日,月小二十九日,凡三百五十四日,乃十二月之日也。遇月分有閏之處,於第十二月內增一日,凡三百五十五日。

太陽五星最高行度隋己未測定。太陽二宮二十九度二十一分。土星八宮十四度四十八分。木星六宮初度八分。火星四宮十五度四分。金星二宮十七度六分。水星七宮六度十七分。

求宮分閏日無之餘日。置西域歲前積年,減一,以一百五十九乘之,一百二十八年內,閏三十一日故以總數乘。內加一十五,閏應。以一百二十八屢減之,餘不滿之數,若在九十七已上,閏限。其年宮分有閏日,已下無閏日。於除得之數內加五,宮分立成起火三,故須加五。滿七去之,餘即所求年白羊宮一日七曜。有閏加一日,後同。

求月分閏日朔之餘日。置西域歲前積年,減一,以一百三十一年乘之,總數乘。內加一百九十四,閏應。以三十為法屢減之,餘在十九已上,閏限。其年月分有閏閏已下則無。於除得之數,滿七去之,餘即所求年第一月一日七曜。

加次法置積日,全積並宮閏所得數。減月閏內加三百三十一日,己未春正前日。以三百五十四一年數除之,餘數內減去所加三百三十一,又減二十三,足成一年日數。又減二十四,洪武甲子加次。又減一,改應所損之一日。為實距年己未至今得數。又法:以氣積宮閏並通閏為氣積內減月閏,置十一,以距年乘之,外加十四,以三十除之,得月閏數。以三百五十四除之,餘減洪武加次二十四,又減補日二十三,又減改應損日一,得數如前。求通閏,置十一日,以距年乘之。求宮閏前見。

▲太陽行度

求最高總度置西域歲前積年,入總年零年月分日期立成內,各取前年前月前日最高行度併之。如求十年,則取九年之類。蓋立成中行度,俱本年本月日足數也。如十年竟求十年,則逾數矣。月日義同。後仿此。

求最高行度置求到最高總度,加測定太陽最高行度,二宮二十九度二十一分。即年求年白羊宮最高行度。如求次宮,累加五秒零六微。求次月,加四秒五十六微。

求中心行度日平行度。置積年入總年零年月日立成內,各取日中心行度併之,取法同前。內減一分四秒,即所求白羊宮第一日中心行度。求各宮月日,按每日行度五十九分八秒累加之。內減一分四秒,或云西域中國裏差,非是,蓋係己未年之末日度應也。

求自行度置其日中心行度,減其宮最高行度,即得。即入盈縮曆度也。

求加減差。即盈縮差。以自行宮度為引數,入太陽加減立成內,照引數宮度取加減差。是名未定差。其度下小餘,用比例法,以本加減差,與後度加減差相減,餘數通為秒,如一分通為六十秒。與引數小餘亦通秒相乘,得數為纖,秒乘秒,得纖。以六十收之,為微、為秒、為分。如數多,先以六十收之為微,又以六十收之為秒,又以六十收之為分。視前所得未定加減差數較,少於後數者後度加減差加之,多於後數者減之,是為加減定差分。如無小餘,竟用未定差為定差。後準此。

求經度黃道度。置其日中心行度,以加減定差分加減之,視定差引數自行宮度,在初宮至五宮為減差,六宮至十一宮為加差。即得。

求七曜置積年入立成內,取總年零年月日下七曜數併之,累去七數,餘即所求白羊宮一日七曜。如求次宮者,內加各宮七曜數。如求逐日,累加一數,滿七去之。求太陰、五星、羅計七曜並準此。

▲太陰行度

求中心行度置積年入立成內,取總零年月日下中心行度併之,得數,內減一十四分,己未應轉。即所求年白羊宮一日中心行度。如求逐日,累加日行度。十三度一零三五。

求加倍相離度月體在小輪行度,合朔後,與日相離。置積年入立成內,取總年零年月日下加倍相離度併之,內減二十六分,即所求白羊宮一日度也。如逐日,累加倍離日行度。二十四度二二五三二二,半之,即小輪心離太陽數。

求本輪行度即月轉度。置積年入立成內,取總零年月日下本輪行度併之,內減一四分,即所求白羊宮一日度也。如求各日,累加本輪日行度。十三度三分五四。

求第一加減差又名倍離差。以加倍相離宮度為引數,入太陰第一加減立成內,取加減差。未定差。又與下差相減,餘乘引數小餘,得數為秒,分乘分以六十收之為分,用加減未定差,後差多加少減,同太陽。得第一分差。

求本輪行度置其日本輪行度,以第一差分加減之。視倍離度,前六宮加,後六宮減。

求第二加減差以本輪行定度度為引數,入太陰第二加減立成內,取未定差,依比例法,同前。求得零數加減之為第二加減差分。視引數,六宮已前為減差,後為加差。

求比數分以倍離宮度,入第一加減立成內,取比數分。如倍離零分在三十分已上者,取下度比敷分。

求遠近度以本輪行定宮度為引敷,入陰第二加減立成內,取遠近度分。其引數零分,亦依比例法取之。

求凡差定差置比敷分,以遠近度通分乘之,以六十約之為分,即凡差。以凡差加入第二加減差,即為定差。

求經度置其日太陰中心行度,以定差加減之,即太陰經度。視本輪行定前減,以後加。

▲太陰緯度

求計都與月相離度入交定度。置其日太限經度,內減其日計都行度,即計都與月相離度分。

求緯以計都與月相離宮度為引數,入太陰緯度立成,上宮用右行順度,下宮用左行逆度。取其度分,依比例法求得零分加減之,上六宮加,下六宮減。得緯度分。引數在六宮已前為黃道北,六宮後為黃道南。

求計羅行度置積年入總年零年月日立成內,取羅計中心行度併之,為其年白羊宮一日行度。求各宮一日,以各宮日行度加之,與十二宮相減,餘即所求宮一日計都行度。如求計都逐日細行,以前後二段行度相減,餘以相距日數除之,為日差。又置前段計都行度,以日差累減之。如求羅喉行度,置其日計都行度內。

▲五星經度

求最高總度數同太陽,依前太陽術求之。

求最高行度置所求本星最高總度,加測定本星最高行度,見前。為其年白羊最高行度。求扣宮各日,加各宮日行度。

求日中心行度依太陽術求之。

求自行度置積年入立成總零年月日下,各取自行度併之,得其年白羊宮一日自行度。土、木、金三星減一分,水星減三分,火星不減。如求各宮各日,照本星自行度累加之。水星如自行度遇三宮初度,作五日一段算,至九宮初度,作十日一段算緯度亦然。

求中心行度中輪心度即入曆度五星本輪。土、木、火三星,置太陽中心行度,減其星自行度,為三星中心行度。內又減最高行度,為三星小輪心度。金、水二星,其中心行度即太陽中心行度,內減其星最高行度,餘為其星小輪心度。不及減,加十二宮減之。

求第一加減差盈縮差。以其星小輪心宮度為引數,入本星第一加減立成,依比例法求之。法同太陽、太陰。

求自行定度及小輪心定度視第一加減差引數,在初宮至五宮,用加減差,加自行度,減小輪心度,各為定度。在六宮至一宮,用加減差,減自行度,加小輪心度,各為定度。

求第二加減差以其星自行定度,入本第二加減立成內,取其度分,用比例法加減之。同前。

求比敷分如土、木、金、水星,以本星小輪心一宮度,入第一加減立成內,取比敷分,如引數小餘在三十分已上,取手行經敷分。如火星,則必用比例法求之。

求遠近度以自行定宮度,入第二加減立成內,取遠近度,依比例法求之。

求凡差定差法同太陰。

求經度置小輪心定度,以定差加減之,視引數自行定度,在六宮已前加,已後減。內加其星最高行度。

求留段以其段小輪心,定宮諜為引數,即立成內各星入曆定限。入五星順退留立成內,於同宮近度,取本星度分,與前後行查減。若取得在初宮至六宮,本行與後行相減。六宮至初宮,本行與與前行相減。又以引數宮度,減立成內同宮近度,兩減,餘通分相乘,用六度除之,立成內每隔六度。六十分收之,順加逆減於前取度分,得數與其日自行定度同者,即本日留。如自行定度多者已過留日,少者未到留日。欲得細率,以所得數與其人日自行定度相減,餘以各星一日自行度約之,如土星一日自行五下七分有奇之類。即得留日在本日前後數也。土星留七日,其留日前三日,後三日,皆與留日數同。木星留五日,其留日前二日,後二二與留日數同。火、金、水三星不留,退而即退,但於行分極處留耳。

求細行分土、木、金、火四星,以前後兩段經度相減,以相距除之為日行分。水星以白羊宮初日經度,又與前一日經度相減,餘為初日行分。又置前後二段經度相減,餘以相距日除之,為平行分。與初日行分加減,倍之,以前段前一日與後段相距日數除之之為日差。以加減初日行分,初日行分少於平行分加,多減。為日行分。五星各置前段經度,以逐日行分順加退減之,為各星逐日經度。

求伏見視各星自行定度,在伏見立成內限度已上者,即五星晨夕伏見也。

五星緯度求最高總行度、中心行度、自行度、小輪度,並依五星比經度術求之。

求自行定度置自行宮度分,其宮以一十乘為度。如一宮,以十乘之得十度,此用約法折算,以造緯度立成。其度以二十乘之為分,滿六十約之為度。其分亦以二十乘之為秒,滿六十約之屬分。併之即得。

求小輪心定度置小輪心宮度分,其宮以五乘之為度。如一宮以五乘之,得五度。其度以一十乘之為分,滿六十約之迷度。其分亦以一十乘之為秒,滿六十約之為分。併之即得。

求緯度以小輪心定度及自行度,入本星緯度立成內兩取,一縱一橫。得數與後行相減。若遇交黃道者,與後行相併。又以小輪心定立成上小輪心定相減,上橫行。兩減餘相乘,以立成上小輪心度累加數除之。如土星上橫行小輪心度每隔三度,火星每隔二度之類。滿六十收之為分,用加減兩取數,多於後行減,少加。若遇交黃道者,即後行數多亦減。寄左。復以自行定度與立成上自行定度相減,首直行。又以兩取數,與下行相減,若遇交黃道埏,與下行併。兩減餘相乘,以立成上自行度累加數除之,如土星直行,自行度每隔十度,火星每隔四度之類。收之為分。與前寄左數相加減,如兩取數多於下行者減,少加。若遇交黃者,所得分多於寄左數,置所得分內,減寄左數,餘為交過黃道南北分也。即得黃道南北緯定分。

求緯度細行分置其星前段緯度,與後段緯度相減,餘以相距日除之,為日差。置前段緯度,以日差順加退減,即逐日緯度分。按緯度前段少於後段者,以日差順加退減。若前段多於後段者,宜以日差順減退加。非可一例也。若前後段南北不同者,置其星前後段緯度併之,以相距日除之,為日差。置前段緯度,以日差累減之,至不及減者,於日差內減之,餘以日差累加之,即得逐日緯度。

推日食法日食諸數,如午前合朔,用前一日數推,午後合朔,用次日數推。

辨日食限視合朔太陰緯度,在黃道南四十五分已下,黃道北九十分已下,為人食。若合朔為盡,則全見食。若膈朔在日未出三時及日已入十五分,一時四分之一。皆有帶食。若合朔在夜刻者不算。

求食甚凡時即合朔。置午正太陰行過太陽度,求法見後月食太陰逐時行過太陽分。通秒,以二十四乘之為實,置太陰日行度,減太陽日行度,通秒為法,除之為時。時下零數以六下通之為分,分下零數以六十通之為秒,三十秒已上收為一分,六十分收為一時,共為食甚凡時。

求各朔太陽經度以食甚凡時通分,以太陽日行度通秒乘之,以二十四除之為秒,滿六十約之為秒分,用加減午正太陽度,午前合朔減之,午後加之。得合朔時太陽經度。即食甚日躔黃道度。

求加減分視合朔時太陽宮度,入晝夜加減立成內,取加減分,依比例法求之。

求子正至合朔時分秒置食甚凡時,以加減分分加減之,午前合朔減,午後加。用加減十二時,午前合朔用減十二時,午後用加十二時。即子正至合朔時分秒。按命時起子正,乃變其術以劍《大統》,非其本法也。

求第一東西差經差。視合朔時,太陽宮在立成經緯時加減立成右七宮取上行時,順行。在左七宮取下行時,逆行。以子正至合朔時,取經差,依比例法求之。止用時下小餘求之。下同。第一東西差。

求第二東西差視合朔時,太陽宮在立成內,同上。取次宮子正至合朔時經差,依比例法求之,為第二東西差。

求第一南北差緯差。以合朔時,太陽宮及子正,至合朔時入立成內,同上。取緯差,依比例法求之,為第一南北差。

求第二南北差以合朔太陽宮,取次宮子正至合朔時緯差,依比例法求之,為第二南北差。

求第二時差以膈朔太陽宮及子正至膈朔時,入立成取時差,依比例法求之。

第二時差公合朔太陽宮,取次宮子正至合朔時時差差,依比例法求之。

求合朔時東西差以第一東西差與第二東西差相減,餘通秒,以乘合朔時太陽度分,亦通秒。以三十度除之為纖,以六十收之為微、為秒、為分、經加減第一東西差,視第一東西差數少於第二差者加已,多者減之,下同。為合朔時東西差。

求合朔時南北差以第一南北差與第南北差相減,餘通秒,以乘太陽度分,以三十除之為纖,依率收之為微、秒、分,以加減第一南北差,為各朔時南北差。

求合朔時差以第一第二兩時差相減,乘太陽度分,以三十除之,依率帳之,用加減第一時差,為合朔時差。

求合朔時本輪行度以本輪日行度一十三度四分通分,以乘食甚凡時,亦通分。以二十四除之為秒,依率收之為分、為度,以加減午正本輪行度,午前減,午後加。為合朔時行度。

求比敷分以本輪行度入立成,太陽、太陰時行影徑分立成。取同宮近度太陰比敷分,依比例法求之。

求東西定差置合朔時東西差通秒,以比敷分通秒乘之為纖以六十收之為微、為秒、為分,以加合朔東西差,有加、無減。為定差。

求南北定差法同東西定差。

求食甚定時即食甚定分。視其日合朔時,太陽度在立成經緯時加減立成左七宮,其時差,黑字減,白字加,在右七宮,白字減,黑字加,皆加減於子正至合朔時,得數命起正減之,得某時初正。餘通為秒,以一千乘之,以一百四下四除之,六十分為一時,每日一千四百四十分,故以千乘之,又以一四四除之。以六十約之,滿百為刻,即食甚定時。

求食甚太陰經度於合朔太陽經度內,加減東西定差,即得食甚太陰經度。其加減視食甚定時時差加減。

求合朔計都度置食甚凡時通分,以計都日行度三分一十一秒通秒乘之,以二十四除之為微,滿六十收之為秒、為分,以加減其日午時計都行度,羅計逆行,午前合朔加,午後減。為合朔時計都度。

求合朔太陰緯度食甚時,太降經度內加減合朔時計都度,餘為計都與月相離度,入太陰緯度立成取之。

求食甚太陰緯度南北定差內。加減合朔時太陰緯度,在黃道南加,北減。得食甚緯度。

求合朔時太陽自行度用太陽日行度五十九分八秒通秒,以乘食甚凡時,亦通分。用二十四除之,得數為微,滿六十收之為秒、為分,以加減其日午正自行度,午前合朔減,午後加。得合朔自行度。

求太陽徑分以合朔太陽自行度為引數,入立成影徑分立成內同宮近度,取太陽徑分,依比例法求之。

求太陰徑分以合朔時本輪行度為引數,入立成同上內取同宮近度太陰徑分,依比例法求之。

求二半徑分併太陽、太陰雨徑分,半之。

求太陽食限分置二半徑分,內減食甚太陰緯度,餘為太陽食限。如不及減者不食。如太陰無緯度者,食既。如太陰無緯度而日徑大於月徑者,食有金環。

求太陽食甚定分以太陽食限分通秒,以一千乘之為實,以太陽徑分通秒為法除之,以百約之為分,為太陽食甚定分。

求時差即定用分。食甚太陰緯度通秒自乘,二半徑分亦通秒自乘,兩自乘數相減,餘以平方開之,以二十四乘之為實,以其日太陰日行度內減太陽日行度通分為法。實如法而一,得數為分,滿六十分為一時,為時差。

求初虧置食甚定時,內減時差,餘時命起子正減之,得初正時。餘分通秒,以一千乘之,以一百四十四除之,以六十約之,滿百為刻,為初虧時刻。

求復圓置食甚定時,內加時差,命起子正,如初虧法,得復圓時刻。

求初虧食甚圓方位與《大統》法同。

推月食法月食諸數,午前望,用前一日推,午後望,用次一日推。

辨月食限視望日太陰經度與羅喉或計都度相離二十三度之內,太陽緯度在一度八分之下,為有食。又視合望在太陰未出二量,未入二時,其限有帶食。其在二時已上者不算。

求食甚凡時即經望。置其日太陰經度內減六宮,如不及減,加十二宮減以減其日午正太陽度為午前望。如太陽度不及減,加入六宮減之,為午後望。置相減餘數相通秒,以二十四乘之為實,置其日太陰經度,內減前一日太陰經度,若在午後望者,減後一日太陽度。餘為太陽日行度。兩日行度相減,餘通秒為法,除實得數為時。其時下餘數,以六十通之為分、秒,即所求食甚凡時。

求食甚月離黃道宮度置食甚凡時,與太陽日行度俱通秒相乘,以二十四除之,得數為纖潢六十收之為微、為秒、為分,以加減其日午正太陽度,午前望減,午後望加。為望時太陽度,加六宮,即得所求。

求晝夜加減差以望時太陽宮度為引數,入晝夜加減立成內,取加減分,依比例法求之。

求食甚定時置食甚凡時,以晝夜加減差法加減之。午前望減,午後望加。得數,用加減一十二時,如午後望加十二時,午前望與十二時相減。命起子正,得初正時。其小餘,如法收為刻,法詳日食。得定時。

求望時計都度置食甚凡時,通秒為實,以計都日行度三分一十一秒通秒乘之,以二十四除之,得數為纖以六十收之為微、為秒、為分,用加減其日午正計都行度,羅計逆行,午前望加,午後望減。即得。

求望時太陰緯度置食甚月離黃道度,內減望時計都度,如不及減,加十二宮減。餘為計都與月相離度,入太陰緯度立成取之。

求望時本輪行度即入遲疾曆。置太陰本輪日行度,十三四分。通分,以食甚凡時通秒乘之,以二十四除之為微,以六十收之為秒、為分、為度,用加減其日午正本輪行度,午前望減,午後加。即得。

求太陰徑分以望時本輪行宮度,入影徑分立成求之。法詳日食。

求太陰影徑分以望時本輪行宮度,放影徑分立成,取之。

求望時太陽自行度以太陽日行度五十九分八秒與食甚凡時俱通秒相乘,以二十四除之,得數為纖,滿六十收為微、為秒、為分,以減其日午正太陽自行度。法同日食求太陽經度。

求影徑減差以其日太陽自行範度為引數,入影徑立成內,於同宮近度取太陰影徑差分,依比例法求之。法詳前。

求影徑定分置太陰影徑分,內減影徑減差分。

求二半徑分置太陰徑分,加影徑定分,半之。

求太陰食限置二半徑分,內減望時太陰緯度。

求食甚定分置食限分,通秒,以一千乘之為實,以太陰徑分秒為法,除之,以百約之災分,為食甚定分。

求太陰逐時行過太陽分置太陰望時經度,減前一日太陰經度,又置望時太陽自行度,減前一日太陽自行度,以兩餘數相減,為太陰晝夜行過太陽度。通秒以二十四除之,滿六十收之,得逐時行過太陽分。

求時差以太陰緯度分,通秒自乘,又以二半徑分通秒自乘,兩數相減,餘開平方為實,以太陰行過太陽度通秒為法除之,得數即時刻差。即初虧至食甚定用分。

求初虧復圓時刻以時差減食甚定時,得初虧時刻。加食甚定時,得復圓時刻。其命時收刻之法,並同日食。

求食既至食甚時差置二半徑分,減太陰徑分,通秒自乘,又置太陰緯度亦通秒自乘,相減,平方開之為實。以太陰逐時行過太陽度通秒為法除之,得數即時差。

求食既生光時刻以食既至食甚時差,減食甚定時,為食既時刻。加食甚定時,為生光時刻。

求初虧食甚復圓方位與《大統》法同。

求日出入時以午正太陽經度為引數,入西域晝夜時立成,取其度分,依比例法求之,為未定分。又引於數相對宮度內,取其度分,如初宮三度,向六宮三度取之。亦依比例法求之,為後未定分。兩未定分相減,不及減,加三百六十度減。餘通秒,用十五除之,六十收之為分、為時,得其日晝時分秒。半之為其日半晝時分秒。以半晝時分秒減十二時,餘為日出時分秒,加十二時為日入時分秒。

求日月出入帶食分秒視其日日出時分秒,較多於初虧時分秒,少於食甚定時及復時分秒者,即有帶食。置其日日出時或日入時,與食甚定時分秒相減,餘為帶食差。置日月食甚定分,以帶食差通秒乘之,以時差通秒除之,得數為帶食分。於食甚定分內減帶食分,餘為日月帶食所見之分。

求月食更點置二十四時,內減晝時,又減晨昏時,七十二分,即中曆之五刻弱也。餘不夜時,通秒五約之為更法。寺分更法為點法。如食在子正以前者,置初虧食甚復圓等時,內減日入時,又減半晨昏時,三十六分。餘通就,以更法減之為更數。不滿更法者,以點法減之為點數。食在子正已後者,置夜時半之,加初虧食甚復圓等時,以更法減之為更數。不滿更法者,以點法減之為點數。皆命起初更、初點。更法減之,減一次為一更,其減餘不滿法者,亦虛命為一更。點法仿此。

▲太陰五星凌犯

求太陰晝夜行度以本日經度與次日經度相減,餘即本日晝夜行度。

求太陰晨昏刻度置其日午太陰經度,內加立成太陰出入晨昏加減立成其日昏刻加差,即為其日太陰昏刻經度。置其次日午正太陰經度,減立成其日晨刻減差,即為其日太陰晨刻經度。

求月出入度置其日午正太陰經度,加立成內即前立成其日月入加差,即為其日月入時太陰經度。加立成內其日月出加差,即其日月出時太陰經度。

耱太陰所犯星座朔後視昏刻度至月入度,望後視月出度至晨刻度,入黃道南北各像星立成內,經緯度相近在一度已下者,取之。

求時刻置其日午正太陰經度,與取到各像星經度相減,通分,以二十四乘之,以太陰晝夜行度亦通分除之,得初正時。其小餘,以六十通之為分,以一竿千乘之,一百國十四除之,以百約之為刻,即得所求時刻。

求上下相離分置太陰緯度與年犯星緯度相減,餘為上下相離分。若月星同在南,月多為下離。同在北,月多為上離,下為下離。若南北不同,月在北為上離,南為下離。

求五星凌犯各星相離置其日五星經緯度,入黃道立成內,視各像內外星經緯度,在一度已下和取之。其五星緯度與各星緯度相減,餘即上下相離分。

求月犯五星,五星相犯視太陰經緯度,五星經緯度相近在一度已下者,取之。

▲回回曆法二

日五星中心行度立成造法原本總年零年月分日期,及十二宮初日,凡五立成。每立成內,首列本信立成處月日宮各紀數,次刑七曜,次刑日中心行度,及土、木、火、金、水各自行度日五星最高行度,交多不祿。祿其造立成之法於左。

日中心行長日期立成一日行五十九分八秒,按日累加之,小月二十九日,得二十八度三十五分二秒,大月三十日,得二十九度三十四分一十秒。

月分立成。單月大,變月小,末置一閏日。大月,二十妨度三十四分十秒。小月二十八度三十五分二秒。按月累加之,十二月計十一宮十八度五十五分九秒,閏日加五十九分八秒。

宮分初日立成。於白羊宮初日起算,至金牛宮初日,凡三十一日,得一宮一度三十三分十八秒。五十九分八秒之積。視宮分日數多少,日數見前。累加積之,至變魚宮初日,得十一宮一度十一秒。自白羊至此凡三百三十五日之積。

零年立成。每年十一宮十八度五十五分九秒,三十年閏十一日,故二年、五年、七年、十年、十三年、十六年、十八年、二十一年、二十四年、二十六年、二十九年、皆閏日。約法,每一年減十一度四分五十一秒,閏年減十度五分四十三秒,三十年為一宮八度二十玢三十一秒。每年三百五十四日,計一萬六百二十日,加閏十一日,共一萬六百三十一日。

總年立成。第一年為三宮二十六度五分十九秒。每三十年加一宮八度二十五分三十一秒,至一千四百罩十年,得五宮十五分三十三秒。

▲五星自行度立成造法

土星日期立成一日五十七分,按日遞加。小月二十七度三十七分,大月二十八度三十四分。其五日、十二日、二十日、二十八日增一分者,乃秒數所積也。

月分立成。大月加二十八度三十四分,小月加二十七度三十革分。按月累加,十二月計十一宮七度四分,閏日加五十七分。

宮分初日立成。金牛宮初日為二十九度三十一分,自行三十一日之積。餘四星準此。視宮分日數累加之,至變魚宮祿日為十宮十八度五十八分。

零年立成。每年十一宮七度四分,其閏日有無,視日中行度,零年有加本星一日行分,下四星準此。至三十年,共一宮十二度一十六分。

總年立成。第一年十一宮二十九十度十八分,此隋己未測定根數,一云即洪武甲子年數,加次在內。下四星準此。六百年四宮四度四十四分。每三十年加一宮十二度二十七分,至一千四百四十年,計七宮十八度二十分。

木星日期立成一日五十四分,按日遞加。小月二十六度十分,大月二十七度五分。其四日、十一日、十七日、二十四日、三十日增一分者,秒數所積也。

月分立成按大、小月累加,十二月計十宮十九度二十九分,閏日加五十四分。

宮分初日立成金牛宮初日二十七度五十九分,至變魚宮初日為十宮二十九度二十六分。

零年立成每年十宮十九度,至三十年,計七宮二十四度三十九分。

總年立成第一年四宮二十五度十九分,六百年五宮八度二十七分。每三十年加七宮二四三九,至千四百四十年,計八呂八度五十分。

火星日期立成一日二十八分,按日遞加。小月十三度二十三分,大月十三度五十一分。其二日、五日、九日、十二日、十五日、十八日、二十二日、二十五日、二十八日各減一分。

月分立成按大小月累加,十二月計五宮十三度二十四分,閏日加二十八分。

宮分初日立成金牛宮初日十四度九分,至變魚宮初日五宮十八度二十九分。

零年立成每年五宮十三度二十四分,至三十年,計七宮十七度一分。

總年立成第一年八宮三十四度六分,六百年四宮四工三十三分。每三十年加七宮度一分,至一千四百四十年,計一度一十一分。

金星日期立成一日三十七分,按日遞加。小月十七度五十三分,大月十八度三十分。

月分立成按大小月累加,十二月計七宮八度十五分,閏日加三十七分。

宮分初日立成金牛宮初日十九度七分,至變魚宮初日七宮十五度二分。

零年立成每年七宮八度十五分,至三十年,計二宮十四度十五分。

總年立成第一年一宮十五度二十九分,六百年三宮零三十四分。每三十年加二宮十四度十五分,一百四十年,計九度五十一分。

水星日期立成一日三度六分,按日遞加。小月三宮初度六分,大月三宮三度十二分。其二日、四日、七日、九日、十二日、十四日、十七日、十九日、二十二日、二十四日、二十七日、二十九日各增一分。

月分立成按大小月累加,十二月計初宮十九度四十七分,閏月加三度六分。

宮分初日立成金牛宮初日三宮六度十九分,至變魚宮初日十宮二十度四十五分。

零年立成每年初宮十九度四十七分,至三十年,計八宮二十七度四十四分。

總年立成第一年二宮二十五度三十四分,六百年一宮十度九分。每三十年加八宮二十七度四十四分,至一千四百四十年,計十一宮六度三十五分。

▲日五星最高行度立成造法日五星同用。

最高行日分立成。一日一十微,按日遞加。其四日、十一日、十八日、二十五日。各減一微,大月四秒五十六微,小月四秒四十六微。

月分立成按大小月累加,十二月計五十八秒一十微。有閏日加十微。

宮分初日立成金牛宮初日五秒六微,至變魚宮初日五十五秒五微。

零年言成每年五十八秒,去二十微。按年遞加,三年積六十微加一秒,三十年計二十九分十秒。

總年立成一年初宮十度四十分二十八秒,洪武甲子加次。六百年五十八分十三秒。每三十年加二十九分七秒,一千四百四十年,計十二度三十六分五十五秒。

▲太陰經度立成造法

日期立成。中心行度,一日十三度十一分,按日累加。大月一宮五度十七分,小月初宮二十二度工分。內二日、四日、六日、九日、十一、十四、十六、十八、二十一、二十三、二十六、二十八、三十日。各減一分,共減十三分。加倍相離度,一日二十四度二十三分,按日遞加。大月初宮十一度二十七分,小月十一宮十七度四分。內五日、十四日、二十三日,各減一分。七輪行度,一日十三度四分,按日遞加。大月一宮一度五十七分,小月十八度五十三分。其中逢五,皆減一分。羅計中心行度,一日三分,按日遞加。大月一度三十五分,小月一度三十二分。內三日、九日、十五、二十、二十六日,各增一分。

月分立成。中心行度,大月一宮五度十七分,小月二十二度七分,按月加之,十二月計十一宮十四度二十七分。內三月、七月、十一月,各增一分。有閏日,加十三度十一分。加倍相離度,大月十一度二十七分,小月一宮十牙度四分,十二月計十一宮二十一度三分。內二、六、十月,各減一分。有閏日,加二十四度二十三分。本輪行度,大月一宮一度五十七分,小月十八度五十三分,十二月計十宮五度零分。有閏日,加十三行度四分。羅計行度,大月一度三十五分,小月一度三十二分,十二月十八度四十五分。內三、七、十一月,各增一分。有閏日,加三分。

協零年立成。中心行度,每年十一宮十四度二十七分,三十年一宮八度十五分。三十年閏十一日,與太陽零年同。下準此。銳離度,每年十一宮二十一度三分,三十年十一宮二十九度四十分。閏日,加二十四度二十三分。本輪行度,每年十宮五度,三十年九宮二十三度四十七分。閏日,加十三度四分。羅計行度,每年十八度四十五分,三十年六宮二十二度五十八分。閏日,加三分。

總年立成。中心行度,第一年四宮二十八度四十九分,六百年六宮八度四十二分,每三十加一宮八度十五分,一千四百四十年,五宮二十九度四十七分。倍離度,第一年,一宮二十五度二十八分,六百年一宮十八度三十三分。每三十年加十一宮二直九度四十分,一千四百四十年,一宮九度二十一分。本輪行度,第一年,四宮十二度二分,六百年八宮八度分。每三十年加二十三度六分,六百年十一宮十度三十四分。每三十年加六宮二十二度五十八分,一千寂百四十年,八宮十五度五十分。

▲總零年宮月日七曜立成造法

總年立成,第一年起金六,六百年起日一,每三十年加寺數。零年立成,起水四。宮分立成,金牛宮起火三。月分立成,起月二。日期立成,起日一。求法:有閏日,滿歲七曜。不滿歲,用月七曜。並之,得逐月末日七曜。

太陽加減立成自行宮度為引數。原本宮縱列首行,度橫列上行,每三宮順布三十度,內列加減差,又列加減分。其加減分,乃本度加減差與次度加減差之較也。今去之,止列加減差數,將引數宮列上橫行,度列首直行,用順逆查之,得數無異,而簡捷過之。月、五星加減立成,準此。

▲回回曆法三

土星黃道南北緯度立成上橫行,以小輪心定度為引數,起五十度,異累加三度。累加三度。首直行以自行定度為引數,累加十度。求法:簡兩引數近度,縱橫相遇度分,次各用比例法,得細率。

表格略
