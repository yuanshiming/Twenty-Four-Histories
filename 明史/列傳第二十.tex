\article{列傳第二十}

\begin{pinyinscope}
朱亮祖周德興王弼藍玉(曹震張翼張溫陳桓硃壽曹興謝成李新

朱亮祖,六安人。元授義兵元帥。太祖克寧國,擒亮祖,喜其勇悍,賜金幣,仍舊官。居數月,叛歸於元,數與我兵戰,為所獲者六千餘人,遂入宣城據之。太祖方取建康,未暇討也。已,遣徐達等圍之。亮祖突圍戰,常遇春被創而還,諸將莫敢前。太祖親往督戰,獲之,縛以見。問曰:「爾將何如?」對曰:「生則盡力,死則死耳!」太祖壯而釋之。累功授樞密院判。

從下南昌、九江,戰鄱陽湖,下武昌。進廣信衛指揮使。李文忠破李伯昇於新城,亮祖乘勝燔其營落數十,獲同僉元帥等六百餘人、軍士三千、馬八百匹,輜重鎧甲無算。伯升僅以數騎遁。太祖嘉其功,賜賚甚厚。胡深請會兵攻陳友定,亮祖由鉛山進取浦城,克崇安、建陽,功最多。會攻桐廬,圍餘杭。遷浙江行省參政,副李文忠守杭州。帥馬步舟師數萬討方國瑛。下天台,進攻台州。國瑛出走,追至黃巖,降其守將哈兒魯,徇下仙居諸縣。進兵溫州。方明善拒戰,擊敗之,克其城。徇下瑞安,復敗明善於盤嶼,追至楚門。國瑛及明善詣軍降。

洪武元年,副征南將軍廖永忠由海道取廣東。何真降,悉定其地。進取廣西,克梧州。元尚書普賢帖木兒戰死,遂定鬱林、潯、貴諸郡。與平章楊璟會師,攻克靖江。同廖永忠克南寧、象州。廣西平。班師,太子帥百官迎勞龍灣。三年封永嘉侯,食祿千五百石,予世券。

四年伐蜀。帝以諸將久無功,命亮祖為征虜右副將軍。濟師至蜀,而明昇已降。徇下未附州縣。師還,以擅殺軍校,不預賞。八年同傅友德鎮北平。還,又同李善長督理屯田,巡海道。十二年出鎮廣東。

亮祖勇悍善戰而不知學,所為多不法,番遇知縣道同以聞。亮祖誣奏同,同死,事見同傳。帝尋悟,明年九月召亮祖至,與其子府軍衛指揮使暹俱鞭死。御製壙志,仍以侯禮葬。二十三年追論亮祖胡惟庸黨,次子昱亦坐誅。

周德興,濠人。與太祖同里,少相得。從定滁、和。渡江,累戰皆有功,遷左翼大元帥。從取金華、安慶、高郵。援安豐,征廬州,進指揮使。從討贛州、安福、永新,拔吉安。再進湖廣行省左丞。同楊璟討廣西,攻永州。元平章阿思蘭及周文貴自全州來援,德興再擊敗之,斬硃院判。追奔至全州,遂克之。道州、寧州、藍山皆下。進克武岡州,分兵據險,絕靖江聲援。廣西平,功多。洪武三年封江夏侯,歲祿千五百石,予世券。

是歲,慈利土酋覃垕連茅岡諸寨為亂,長沙洞苗俱煽動。太祖命德興為征南將軍,帥師討平之。明年伐蜀,副湯和為征西左將軍,克保寧。先是,傅友德已克階、文,而和所帥舟師未進。及保寧下,兩路軍始合。蜀平,論功,帝以和功由德興,賞德興而面責和。且追數征蠻事,謂覃垕之役,楊璟不能克,趙庸中道返,功無與德興比者。復副鄧愈為征南左將軍,帥趙庸、左君弼出南寧,平婪鳳、安田諸州蠻,克泗城州,功復出諸將上。賞倍於大將,命署中立府,行大都督府事。德興功既盛,且恃帝故人,營第宅踰制。有司列其罪,詔特宥之。十三年命理福建軍務,旋召還。

明年,五溪蠻亂。德興已老,力請行,帝壯而遣之,賜手書曰:「趙充國圖征西羌,馬援請討交址,朕常嘉其事,謂今人所難。卿忠勤不怠,何忝前賢,靖亂安民,在此行也。」至五溪,蠻悉散走。會四川水盡源、通塔平諸洞作亂,仍命德興討平之。十八年,楚王楨討思州五開蠻,復以德興為副將軍。德興在楚久,所用皆楚卒,威震蠻中。定武昌等十五衛,歲練軍士四萬四千八百人。決荊州嶽山壩以溉田,歲增官租四千三百石。楚人德之。還鄉,賜黃金二百兩,白金二千兩,文綺百匹。

居無何,帝謂德興:「福建功未竟,卿雖老,尚勉為朕行。」德興至閩,按籍僉練,得民兵十萬餘人。相視要害,築城一十六,置巡司四十有五,防海之策始備。逾三年,歸第,復令節制鳳陽留守司,並訓練屬衛軍士。諸勳臣存者,德興年最高。歲時入朝,賜予不絕。二十五年八月,以其子驥亂宮,並坐誅死。

王弼,其先定遠人,後徙臨淮。善用雙刀,號「雙刀王」。初結鄉里,依三臺山樹柵自保。踰年,帥所部來歸。太祖知其才,使備宿衛。破張士誠兵於湖州,取池州石埭,攻婺源州,斬守將鐵木兒不花,拔其城,獲甲三千。擢元帥。下蘭溪、金華、諸暨。援池州,復太平,下龍興、吉安。大戰鄱陽,邀擊陳友諒於涇江口。從平武昌,還克廬州。拔安豐,破襄陽、安陸。取淮東,克舊館,降士誠將朱暹,遂取湖州。遷驍騎右衛親軍指揮使。進圍平江,弼軍盤門。士誠親帥銳士突圍,出西門搏戰,將奔常遇春軍。遇春分兵北濠,截其後,而別遣兵與戰。士誠軍殊死鬥。遇春拊弼臂曰:「軍中皆稱爾健將,能為我取此乎?」弼應曰:「諾。」馳騎,揮雙刀奮擊。敵小卻。遇春帥眾乘之,吳兵大敗,人馬溺死沙盆潭者甚眾。士誠馬逸墮水,幾不救,肩輿入城,自是不敢復出。吳平,賞賚甚厚。

從大軍征中原,下山東,略定河南北,遂取元都。克山西,走擴廓。自河中渡河,克陜西,進徵察罕腦兒,師還。洪武三年,授大都督府僉事,世襲指揮使。十一年副西平侯沐英征西番,降朵甘諸酋及洮州十八族,殺獲甚眾。論功,封定遠侯,食祿二千石。十四年從傅友德征雲南,至大理,土酋段世扼龍尾關。弼以兵由洱水趨上關,與沐英兵夾擊之,拔其城,擒段世,鶴慶、麗江諸郡以次悉平。加祿五百石,予世券。二十年,以副將軍從馮勝北伐,降納哈出。明年復以副將軍從藍玉出塞。深入不見敵,玉欲引還。弼持不可,玉從之。進至捕魚兒海,以弼為前鋒,直薄敵營。走元嗣主脫古思帖木兒,盡獲其輜重,語在玉傳。二十三年奉詔還鄉。二十五年從馮勝、傅友德練軍山西、河南。明年同召還,先後賜死。爵除。弼子六人,女為楚王妃。

藍玉,定遠人。開平王常遇春婦弟也。初隸遇春帳下,臨敵勇敢,所向皆捷。遇春數稱於太祖,由管軍鎮撫積功至大都督府僉事。洪武四年,從傅友德伐蜀,克綿州。五年從徐達北征,先出雁門,敗元兵於亂山,再敗之於土剌河。七年帥兵拔興和,獲其國公帖里密赤等五十九人。十一年同西平侯沐英討西番,擒其酋三副使,斬獲千計。明年,師還。封永昌侯,食祿二千五百石,予世券。

十四年,以征南左副將軍從潁川侯傅友德征雲南,擒元平章達里麻於曲靖,梁王走死,滇地悉平。玉功為多,益祿五百石。冊其女為蜀王妃。

二十年,以征虜左副將軍從大將軍馮勝徵納哈出,次通州。聞元兵有屯慶州者,玉乘大雪,帥輕騎襲破之,殺平章果來,擒其子不蘭溪還。會大軍進至金山,納哈出遣使詣大將軍營納欸,玉往受降。納哈出以數百騎至,玉大喜,飲以酒。納哈出酌酒酬玉,玉解衣衣之,曰:「請服此而飲。」納哈出不肯服,玉亦不飲。爭讓久之,納哈出覆酒於地,顧其下咄咄語,將脫去。鄭國公常茂在坐,直前砍傷之,都督耿忠擁以見勝。其眾驚潰,遣降將觀童諭降之。還至亦迷河,悉降其餘眾。會馮勝有罪,收大將軍印,命玉行總兵官事,尋即軍中拜玉為大將軍,移屯薊州。

時順帝孫脫古思帖木兒嗣立,擾塞上。二十一年三月,命玉帥師十五萬征之。出大寧,至慶州,諜知元主在捕魚兒海,間道兼程進至百眼井。去海四十里,不見敵,欲引還。定遠侯王弼曰:「吾輩提十餘萬眾,深入漠北,無所得,遽班師,何以復命?」玉曰:「然。」令軍士穴地而爨,毋見煙火。乘夜至海南,敵營尚在海東北八十餘里。玉令弼為前鋒,疾馳薄其營。敵謂我軍乏水草,不能深入,不設備。又大風揚沙,晝晦。軍行,敵無所覺。猝至前,大驚。迎戰,敗之。殺太尉蠻子等,降其眾。元主與太子天保奴數十騎遁去。玉以精騎追之,不及。獲其次子地保奴、妃、公主以下百餘人。又追獲吳王朵兒只、代王達里麻及平章以下官屬三千人,男女七萬七千餘人,並寶璽、符敕金牌、金銀印諸物,馬駝牛羊十五萬餘。焚其甲仗蓄積無算。奏捷京師,帝大喜,賜敕褒勞,比之衛青、李靖。又破哈剌章營,獲人畜六萬。師還,進涼國公。

明年命督修四川城池。二十三年,施南、忠建二宣撫司蠻叛,命玉討平之。又平都勻,安撫司散毛諸洞,益祿五百石,詔還鄉。二十四年命玉理蘭州、莊浪等七衛兵,以追逃寇祁者孫,遂略西番罕東之地。土酋哈昝等遁去。會建昌指揮使月魯帖木兒叛,詔移兵討之。至則都指揮瞿能等已大破其眾,月魯走柏興州。玉遣百戶毛海誘縛其父子,送京師誅之,而盡降其眾,因請增置屯衛。報可。復請籍民為兵,討朵甘、百夷。詔不許,遂班師。

玉長身赬面,饒勇略,有大將才。中山、開平既沒,數總大軍,多立功。太祖遇之厚。浸驕蹇自恣,多蓄莊奴、假子,乘勢暴橫。嘗佔東昌民田,御史按問,玉怒,逐御史。北征還,夜扣喜峰關。關吏不時納,縱兵毀關入。帝聞之不樂。又人言其私元主妃,妃慚自經死,帝切責玉。初,帝欲封玉梁國公,以過改為涼,仍鐫其過於券。玉猶不悛,侍宴語傲慢。在軍擅黜陟將校,進止自專,帝數譙讓。西征還,命為太子太傅。玉不樂居宋、潁兩公下,曰:「我不堪太師耶!」比奏事多不聽,益怏怏。

二十六年二月,錦衣衛指揮蔣瓛告玉謀反,下吏鞫訊。獄辭云:「玉同景川侯曹震、鶴慶侯張翼、舳艫侯硃壽、東莞伯何榮及吏部尚書詹徽、戶部侍郎傅友文等謀為變,將伺帝出耤田舉事。」獄具,族誅之。列侯以下坐黨夷滅者不可勝數。手詔布告天下,條列爰書為《逆臣錄》。至九月,乃下詔曰:「藍賊為亂,謀泄,族誅者萬五千人。自今胡黨、藍黨概赦不問。」胡謂丞相惟庸也。於是元功宿將相繼盡矣。凡列名《逆臣錄》者,一公、十三侯、二伯。葉昇前坐事誅,胡玉等諸小侯皆別見。其曹震、張翼、張溫、陳桓、朱壽、曹興六侯,附著左方。

曹震,濠人。從太祖起兵,累官指揮使。洪武十二年,以征西番功封景川侯,祿二千石。從藍玉征雲南,分道取臨安諸路,至威楚,降元平章閻乃馬歹等。雲南平,因請討容美、散毛諸洞蠻及西番朵甘、思曩日諸族。詔不許。又請以貴州、四川二都司所易番馬,分給陜西、河南將士。又言:「四川至建昌驛,道經大渡河,往來者多死瘴癘。詢父老,自眉州峨眉至建昌,有古驛道,平易無瘴毒,已令軍民修治。請以瀘州至建昌驛馬,移置峨眉新驛。」從之。二十一年,與靖寧侯葉升分道討平東川叛蠻,俘獲五千餘人。

尋復命理四川軍務,同藍玉核征南軍士。會永寧宣慰司言,所轄地有百九十灘,其八十餘灘道梗不利。詔震疏治之。震至瀘州按視,有支河通永寧,乃鑿石削崖,令深廣以通漕運。又闢陸路,作驛舍、郵亭,駕橋立棧。自茂州,一道至松潘,一道至貴州,以達保寧。先是行人許穆言:「松州地磽瘠,不宜屯種。戍卒三千,糧運不給,請移戍茂州,俾就近屯田。」帝以松州控制西番,不可動。至是運道既通,松潘遂為重鎮。帝嘉其勞。踰年復奏四事:一,請於雲南大寧境就井煮鹽,募商輸粟以贍邊。一,令商入粟雲南建昌,給以重慶、綦江市馬之引。一,請蠲馬湖逋租。一,施州衛軍儲仰給湖廣,溯江險遠,請以重慶粟順流輸之。皆報可。

震在蜀久,諸所規畫,並極周詳。蜀人德之。藍玉敗,謂與震及朱壽誘指揮莊成等謀不軌,論逆黨,以震為首,並其子炳誅之。

張翼,臨淮人。父聚,以前翼元帥從平江南、淮東,積功為大同衛指揮同知,致仕。翼隨父軍中,驍勇善戰,以副千戶嗣父職。從征陜西,擒叛寇。擢都指揮僉事,進僉都督府事。從藍玉征雲南,克普定、曲靖。取鶴慶、麗江,剿七百房山寨。搗劍川,擊石門。十七年論功封鶴慶侯,祿二千五百石,予世券。二十六年坐玉黨死。

張溫,不詳何許人。從太祖渡江,授千戶。積功至天策衛指揮僉事。從大軍收中原,克陜西,攻下蘭州,守之。元將擴廓偵大將軍南還,自甘肅帥步騎奄至。諸將請固守以待援。溫曰:「彼遠來,未知我虛實,乘幕擊之,可挫其銳。倘彼不退,固守未為晚也。」於是整兵出戰,元兵少卻。已而圍城數重,溫斂兵固守,敵攻不能下,乃引去。太祖稱為奇功,擢大都督府僉事。

已,又命兼陜西行都督府僉事。當蘭州之受圍也,元兵乘夜梯城而登。千戶郭佑被酒臥,他將巡城者擊退之。圍既解,溫將斬佑,天策衛知事朱有聞爭曰:「當賊犯城時,將軍斬佑以令眾,軍法也。賊既退,始追戮之,無及於事,且有擅殺名。」溫謝曰:「非君,不聞是言。」遂杖佑釋之。帝聞而兩善焉,並賞有聞綺帛。

其明年,以參將從傅友德伐蜀,功多。十一年,以副將會王弼等討西羌。明年論功封會寧侯,祿二千石。又明年命往理河南軍務。十四年從傅友德征雲南。二十年秋帥師討納哈出餘眾,從北伐,皆有功。後以居室器用僭上,獲罪,遂坐玉黨死。

陳桓,濠人。從克滁、和。渡江,克集慶先登。從取寧國、金華。戰龍江、彭蠡。收淮東、浙西。平中原。累功授都督僉事。洪武四年從伐蜀。十四年從征雲南,與胡海、郭英帥兵五萬,由永寧趨烏撒。道險隘,自赤河進師,與烏撒諸蠻大戰,敗走之。再破芒部士酋,走元右丞實卜,遂城烏撒。降東川烏蒙諸蠻,進克大理。略定汝寧、靖寧諸州邑。十七年封普定侯,祿二千五百石,予世券。二十年同靖寧侯葉昇征東川,俘獲甚眾。就令總制雲南諸軍。再平九溪洞蠻,立營堡,屯田。還,坐玉黨死。

硃壽,未詳何許人。以萬戶從渡江,下江東郡邑,進總管。收常、婺,克武昌。平蘇、湖,轉戰南北。積功為橫海衛指揮,進都督僉事。與張赫督漕運,有功。洪武二十年封舳艫侯,祿二千石,予世券。坐玉黨死。

曹興,一名興才,未詳何許人。從平武昌,授指揮僉事。取平江,進指揮使。克蘇九疇炭山寨。進都督僉事,兼太原衛指揮。進山西行省參政,領衛事,為晉王相。洪武十一年,從沐英討洮州羌,降朵甘酋,擒三副使等。師還,封懷遠侯,世襲指揮使。理軍務山西,從北征有功。後數年,坐玉黨死。

同時以黨連坐者,都督則有黃輅、湯泉、馬俊、王誠、聶緯、王銘、許亮、謝熊、汪信、蕭用、楊春、張政、祝哲、陶文、茆鼎凡十餘人,多玉部下偏裨。於是勇力武健之士芟夷略盡,罕有存者。

謝成,濠人。從克滁、和。渡江,定集慶,授總管。克寧國、婺州,進管軍千戶。戰鄱陽,平武昌,下蘇、湖,進指揮僉事。從大軍征中原,克元都,攻慶陽,搗定西。為都督僉事、晉王府相。從沐英征朵甘,降乞失迦,平洮州十八族。洪武十二年封永平侯,祿二千石,世指揮使。二十年同張溫追討納哈出餘眾,召還。二十七年坐事死,沒其田宅。

李新,濠州人。從渡江,數立功。戰龍灣,授管軍副千戶。取江陵,進龍驤衛正千戶。克平江,遷神武衛指揮僉事,調守茶陵衛,屢遷至中軍都督府僉事。十五年,以營孝陵,封崇山侯,歲祿千五百石。二十二年命改建帝王廟於雞鳴山。新有心計,將作官吏視成畫而已。明年遣還鄉,頒賜金帛田宅。時諸勳貴稍僭肆,帝頗嫉之,以黨事緣坐者眾。新首建言:公、侯家人及儀從戶各有常數,餘者宜歸有司。帝是之,悉發鳳陽隸籍為民,命禮部纂《稽制錄》,嚴公侯奢侈踰越之禁。於是武定侯英還佃戶輸稅,信國公和還儀從戶,曹國公景隆還莊田,皆自新發之。二十六年,督有司開胭脂河於溧水,西達大江,東通兩浙,以濟漕運。河成,民甚便之。二十八年以事誅。

贊曰:治天下不可以無法,而草昧之時法尚疏,承平之日法漸密,固事勢使然。論者每致慨於鳥盡弓藏,謂出於英主之猜謀,殊非通達治體之言也。夫當天下大定,勢如磐石之安,指麾萬里,奔走恐後,復何所疑忌而芟薙之不遺餘力哉?亦以介胄之士桀驁難馴,乘其鋒銳,皆能豎尺寸于疆場。迨身處富貴,志滿氣溢,近之則以驕恣啟危機,遠之則以怨望捍文網。人主不能廢法而曲全之,亦出於不得已,而非以剪除為私計也。亮祖以下諸人,既昧明哲保身之幾,又違制節謹度之道,駢首就僇,亦其自取焉爾。


\end{pinyinscope}