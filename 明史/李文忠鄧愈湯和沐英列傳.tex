\article{李文忠、鄧愈、湯和、沐英列傳}

\begin{pinyinscope}
李文忠,字思本,小字保兒,盱眙人,太祖姊子也。年十二而母死,父貞攜之轉側亂軍中,瀕死者數矣。踰二年乃謁太祖於滁陽。太祖見保兒,喜甚,撫以為子,令從己姓。讀書穎敏如素習。年十九,以舍人將親軍,從援池州,破天完軍,驍勇冠諸將。別攻青陽、石埭、太平、旌德,皆下之。敗元院判阿魯灰於萬年街,復敗苗軍於於潛、昌化。進攻淳安,夜襲洪元帥,降其眾千餘,授帳前左副都指揮兼領元帥府事。尋會鄧愈、胡大海之師,取建德,以為嚴州府,守之。

苗帥楊完者以苗、僚數萬水陸奄至。文忠將輕兵破其陸軍,取所馘首,浮巨筏上。水軍見之亦遁。完者復來犯,與鄧愈擊卻之。進克浦江,禁焚掠,示恩信。義門鄭氏避兵山谷,招之還,以兵護之。民大悅。完者死,其部將乞降,撫之,得三萬餘人。

與胡大海拔諸暨。張士誠寇嚴州,禦之東門,使別將出小北門,間道襲其後,夾擊大破之。踰月,復來攻,又破之大浪灘,乘勝克分水。士誠遣將據三溪,復擊敗之,斬陸元帥,焚其壘。士誠自是不敢窺嚴州。進同僉行樞密院事。

胡大海得漢將李明道、王漢二,送文忠所,釋而禮之,使招建昌守將王溥。溥降。苗將蔣英、劉震殺大海,以金華叛。文忠遣將擊走之,親撫定其眾。處州苗軍亦殺耿再成叛。文忠遣將屯縉雲以圖之。拜浙東行省左丞,總制嚴、衢、信、處、諸全軍事。

吳兵十萬方急攻諸全,守將謝再興告急,遣同僉胡德濟往援。再興復請益兵,文忠兵少無以應。會太祖使邵榮討處州亂卒,文忠乃揚言徐右丞、邵平章將大軍刻日進。吳軍聞之懼,謀夜遁。德濟與再興帥死士夜半開門突擊,大破之,諸全遂完。

明年,再興叛降於吳,以吳軍犯東陽。文忠與胡深迎戰於義烏,將千騎橫突其陣,大敗之。已用深策去諸全五十里別築一城,以相掎角。士誠遣司徒李伯昇以十六萬眾來攻,不克。踰年,復以二十萬眾攻新城。文忠帥朱亮祖等馳救,去新城十里而軍。德濟使人告賊勢盛,宜少駐以俟大軍。文忠曰:「兵在謀不在眾。」乃下令曰:「彼眾而驕,我少而銳,以銳遇驕,必克之矣。彼軍輜重山積,此天以富汝曹也。勉之。」會有白氣自東北來覆軍上,占之曰「必勝」。詰朝會戰,天大霧晦冥,文忠集諸將仰天誓曰:「國家之事在此一舉,文忠不敢愛死以後三軍。」乃使元帥徐大興、湯克明等將左軍,嚴德、王德等將右軍,而自以中軍當敵衝。會處州援兵亦至,奮前搏擊。霧稍開,文忠橫槊引鐵騎數十,乘高馳下,衝其中堅。敵以精騎圍文忠數重。文忠手所格殺甚眾,縱騎馳突,所向皆披靡。大軍乘之,城中兵亦鼓噪出,敵遂大潰。逐北數十里,斬首數萬級,溪水盡赤,獲將校六百,甲士三千,鎧仗芻粟收數日不盡,伯昇僅以身免。捷聞,太祖大喜,召歸,宴勞彌日,賜御衣名馬,遣還鎮。

明年秋,大軍伐吳,令攻杭州以牽制之。文忠帥亮祖等克桐廬、新城、富陽,遂攻餘杭。守將謝五,再興弟,諭之降,許以不死。五與再興子五人出降。諸將請僇之,文忠不可。遂趨杭州,守將潘元明亦降,整軍入。元明以女樂迎,麾去之。營於麗譙,下令曰:「擅入民居者死。」一卒借民釜,斬以徇,城中帖然。得兵三萬,糧二十萬。就加榮祿大夫、浙江行省平章事,復姓李氏。大軍征閩,文忠別引軍屯浦城以逼之。師還,餘寇金子隆等聚眾剽掠,文忠復討擒之,遂定建、延、汀三州。命軍中收養道上棄兒,所全活無算。

洪武二年春,以偏將軍從右副將軍常遇春出塞,薄上都,走元帝,語具《遇春傳》。遇春卒,命文忠代將其軍,奉詔會大將軍徐達攻慶陽。行次太原,聞大同圍急,謂左丞趙庸曰:「我等受命而來,閫外之事茍利於國,專之可也。今大同甚急,援之便。」遂出雁門,次馬邑,敗元游兵,擒平章劉帖木,進至白楊門。天雨雪,已駐營,文忠令移前五里,阻水自固。元兵乘夜來劫,文忠堅壁不動。質明,敵大至。以二營委之,殊死戰,度敵疲,乃出精兵左右擊,大破之,擒其將脫列伯,俘斬萬餘人,窮追至莽哥倉而還。

明年拜征虜左副將軍。與大將軍分道北征,以十萬人出野狐嶺,至興和,降其守將。進兵察罕腦兒,擒平章竹真。次駱駝山,走平章沙不丁。次開平,降平章上都罕等。時元帝已崩,太子愛猷識里達臘新立。文忠諜知之,兼程趨應昌。元嗣君北走,獲其嫡子買的立八剌暨后妃宮人諸王將相官屬數百人,及宋、元玉璽金寶十五,玉冊二,鎮圭、大圭、玉帶、玉斧各一。出精騎窮追至北慶州而還。道興州,擒國公江文清等,降三萬七千人。至紅羅山,又降楊思祖之眾萬六千餘人。獻捷京師,帝御奉天門受朝賀。大封功臣,文忠功最,授開國輔運推誠宣力武臣,特進榮祿大夫、右柱國、大都督府左都督,封曹國公,同知軍國事,食祿三千石,予世券。

四年秋,傅友德等平蜀,令文忠往拊循之。築成都新城,發軍戍諸郡要害,乃還。明年復以左副將軍由東道北徵,出居庸,趨和林,至口溫,元人遁。進至臚朐河,令部將韓政等守輜重,而自帥大軍,人齎二十日糧,疾馳至土剌河。元太師蠻子哈剌章悉眾渡河,列騎以待。文忠引軍薄之,敵稍卻。至阿魯渾河,敵來益眾。文忠馬中流矢,下馬持短兵斗。指揮李榮以所乘馬授文忠,而自奪敵馬乘之。文忠得馬,益殊死戰,遂破敵,虜獲萬計。追奔至稱海,敵兵復大集。文忠乃斂兵據險,椎牛饗士,縱所獲馬畜於野。敵疑有伏,稍稍引去。文忠亦引還,失故道。至桑哥兒麻,乏水,渴甚,禱於天。所乘馬跑地,泉湧出,三軍皆給,乃刑牲以祭。遂還。是役也,兩軍勝負相當,而宣寧侯曹良臣,指揮使周顯、常榮、張耀俱戰死,以故賞不行。

六年行北平、山西邊,敗敵於三角村。七年遣部將分道出塞。至三不剌川,俘平章陳安禮。至順寧、楊門,斬真珠驢。至白登,擒太尉不花。其秋帥師攻大寧、高州書,克之,斬宗王朵朵失里,擒承旨百家奴。追奔至氈帽山,擊斬魯王,獲其妃及司徒答海等。進師豐州,擒元故官十二人,馬駝牛羊甚眾,窮追至百乾兒乃還。是後屢出備邊。

十年命與韓國公李善長議軍國重事。十二年,洮州十八番族叛,與西平侯沐英合兵討平之,築城東籠山南川,置洮州衛。還言西安城中水堿鹵不可飲,請鑿地引龍首渠入城以便汲,從之。還掌大都督府兼領國子監事。

文忠器量沉宏,人莫測其際。臨陣踔厲歷風發,遇大敵益壯。頗好學問,常師事金華范祖乾、胡翰,詩歌雄駿可觀。初,太祖定應天,以軍興不給,增民田租,文忠請之,得減額。其釋兵家居,恂恂若儒者,帝雅愛重之。家故多客,嘗以客言,勸帝少誅戮,又諫帝征日本,及言宦者過盛,非天子不近刑人之義。以是積忤旨,不免譴責。十六年冬遂得疾。帝親臨視,使淮安侯華中護醫藥。明年三月卒,年四十六。帝疑中毒之,貶中爵,放其家屬於建昌衛,諸醫並妻子皆斬。親為文致祭,追封岐陽王,謚武靖。配享太廟,肖像功臣廟,位皆第三。父貞前卒,贈隴西王,謚恭獻。

文忠三子,長景隆,次增枝、芳英,皆帝賜名。增枝初授勳衛,擢前軍左都督。芳英官至中都正留守。

景隆,小字九江。讀書通典故。長身,眉目疏秀,顧盼偉然。每朝會,進止雍容甚都,太祖數目屬之。十九年襲爵,屢出練軍湖廣、陜西、河南,市馬西番。進掌左軍都督府事,加太子太傅。

建文帝即位,景隆以肺腑見親任,嘗被命執周王橚。及燕兵起,長興侯耿炳文討燕失利,齊泰、黃子澄等共薦景隆。乃以景隆代炳文為大將軍,將兵五十萬北伐。賜通天犀帶,帝親為推輪,餞之江滸,令一切便宜行事。景隆貴公子,不知兵,惟自尊大,諸宿將多怏怏不為用。景隆馳至德州,會兵進營河間。燕王聞之喜,語諸將曰:「李九江,紈綺少年耳,易與也。」遂命世子居守,戒勿出戰,而自引兵援永平,直趨大寧。景隆聞之,進圍北平。都督瞿能攻張掖門,垂破。景隆忌能功,止之。及燕師破大寧,還軍擊景隆。景隆屢大敗,奔德州,諸軍皆潰。明年正月,燕王攻大同,景隆引軍出紫荊關往救,無功而還。帝慮景隆權尚輕,遣中官齎璽書賜黃鉞弓矢,專征伐。方渡江,風雨舟壞,賜物盡失,乃更製以賜。四月,景隆大誓師於德州,會武定侯郭英、安陸侯吳傑等於真定,合軍六十萬,進營白溝河。與燕軍連戰,復大敗,璽書斧鉞皆委棄,走德州,復走濟南。斯役也,王師死者數十萬人,南軍遂不支,帝始詔景隆還。黃子澄慚憤,執景隆於朝班,請誅之以謝天下。燕師渡江,帝旁皇甚,方孝孺復請誅景隆。帝皆不問。使景隆及尚書茹瑺、都督王佐如燕軍,割地請和。燕兵屯金川門,景隆與谷王橞開門迎降。

燕王即帝位,授景隆奉天輔運推誠宣力武臣、特進光祿大夫、左柱國,增歲祿千石。朝廷有大事,景隆猶以班首主議,諸功臣咸不平。永樂二年,周王發其建文時至邸受賂事,刑部尚書鄭賜等亦劾景隆包藏禍心,蓄養亡命,謀為不軌。詔勿問。已,成國公朱能、吏部尚書蹇義與文武群臣,廷劾景隆及弟增枝逆謀有狀,六科給事中張信等復劾之。詔削勳號,絕朝請,以公歸第,奉長公主祀。亡何,禮部尚書李至剛等復言:「景隆在家,坐受閽人伏謁如君臣禮,大不道;增枝多立莊田,蓄僮僕無慮千百,意叵測。」於是奪景隆爵,並增枝及妻子數十人錮私第,沒其財產。景隆嘗絕食旬日不死,至永樂末乃卒。

正統十三年始下詔令增枝等啟門第,得自便。弘治初,錄文忠後,以景隆曾孫璇為南京錦衣衛世指揮使。卒,子濂嗣。卒,子性嗣。嘉靖十一年詔封性為臨淮侯,祿千石。踰年卒,無子,復以濂弟沂紹封。卒,子庭竹嗣。屢典軍府,提督操江,佩平蠻將軍印,鎮湖廣。卒,子言恭嗣。守備南京,入督京營,累加少保。言恭,字惟寅,好學能詩,折節寒素。子宗城,少以文學知名。萬曆中,倭犯朝鮮,兵部尚書石星主封貢,薦宗城才,授都督僉事,充正使,持節往,指揮楊方亨副之。宗城至朝鮮釜山,倭來益眾,道路籍籍,言且劫二使。宗城恐,變服逃歸。而方亨渡海,為倭所辱。宗城下獄論戍,以其子邦鎮嗣侯。明亡,爵絕。

鄧愈,虹人。初名友德,太祖為賜名。父順興,據臨濠,與元兵戰死,兄友隆代之,復病死,眾推愈領軍事。愈年甫十六,每戰必先登陷陣,軍中咸服其勇。太祖起滁陽,愈自盱眙來歸,授管軍總管。從渡江。克太平,破擒陳野先,略定溧陽、溧水,下集慶,取鎮江,皆有功。進廣興翼元帥,出守廣德州,破長鎗帥謝國璽於城下,俘其總管武世榮,獲甲士千人。移鎮宣州,以其兵取績溪,與胡大海克徽州,遷行樞密院判官守之。

苗帥楊完者以十萬眾來攻,守禦單弱,愈激勵將士,與大海合擊,破走之。進拔休寧、婺源,獲卒三千,徇下高河壘。與李文忠、胡大海攻建德,道遂安,破長鎗帥餘子貞,逐北至淳安,又破其援兵,遂克建德。楊完者來攻,破擒其將李副樞,降溪洞兵三萬。踰月,復破完者於烏龍嶺。再遷僉行樞密院事。

略臨安,李伯昇來援,敗之閑林寨。遣使說降饒州守將于光,遂移守饒。饒濱彭蠡湖,與友諒接境,數來侵,輒擊卻之。進江南行省參政,總制各翼軍馬。取浮梁,徇樂平,餘干、建昌皆下。

友諒撫州守將鄧克明為吳宏所攻,遣使偽降以緩師。愈知其情,卷甲夜馳二百里,比明入其城。克明出不意,單騎走。愈號令嚴肅,秋毫不犯,遂定撫州。克明不得已降。會友諒丞相胡廷瑞獻龍興路,改洪都府,以愈為江西行省參政守之,而命降將祝宗、康泰以所部從。二人初不欲降,及奉命從徐達攻武昌,遂反。舟次女兒港,趨還,乘夜破新城門而入。愈倉卒聞變,以數十騎走,數與賊遇。從騎死且盡,窘甚。連易三馬,馬輒踣。最後得養子馬乘之,始得奪撫州門以出,奔還應天。太祖弗之罪也。既而徐達還師復洪都,復命愈佐大都督朱文正鎮之。其明年,友諒眾六十萬入寇,樓船高與城等,乘漲直抵城下,圍數百重。愈分守撫州門,當要衝。友諒親督眾來攻,城壞且三十餘丈,愈且築且戰。敵攻益急,晝夜不解甲者三月。太祖自將來援,圍始解,論功與克敵等。太祖已平武昌,使愈帥兵徇江西未附州縣。鄧克明之弟志清據永豐,有卒二萬。愈擊破之,擒其大帥五十餘人。從常遇春平沙坑、麻嶺諸寨,進兵取吉安,圍贛州,五月乃克之。進江西行省右丞,時年二十八。兵興,諸將早貴未有如愈與李文忠者。

愈為人簡重慎密,不憚危苦,將軍嚴,善撫降附。其徇安福也,部卒有虜掠者。判官潘樞入謁,面責之。愈驚起謝,趣下令掠民者斬,索軍中所得子女盡出之。樞因閉置空舍中,自坐舍外,作糜食之。卒有謀乘夜劫取者,愈鞭之以徇。樞悉護遣還其家,民大悅。已而遇春克襄陽,以愈為湖廣行省平章鎮其地,賜以書曰:「爾戍襄陽,宜謹守法度。山寨來歸者,兵民悉仍故籍,小校以下悉令屯種,且耕且戰。爾所戍地鄰擴廓,若爾愛加於民,法行於軍,則彼所部皆將慕義來歸,如脫虎口就慈母。我賴爾如長城,爾其勉之!」愈披荊棘,立軍府營屯,拊循招徠,威惠甚著。

吳元年建御史臺,召為右御史大夫,領臺事。洪武元年兼太子諭德。大軍經略中原,愈為征戍將軍,帥襄、漢兵取南陽以北未附州郡。遂克唐州,進攻南陽,敗元兵於瓦店,逐北抵城下,遂克之,擒史國公等二十六人。隋、葉、舞陽、魯山諸州縣相繼降。攻下牛心、光石、洪山諸山寨,均、房、金、商之地悉定。三年,以征虜左副副將軍從大將軍出定西。擴廓屯車道峴,愈直抵其壘,立柵逼之,擴廓敗走。分兵自臨洮進克河州,招諭吐蕃諸酋長,宣慰何鎖南普等皆納印請降。追豫王至西黃河,抵黑松林,破斬其大將。河州以西朵甘、烏斯藏諸部悉歸附。出甘肅西北數千里而還。論功授開國輔運推誠宣力武臣、特進榮祿大夫、右柱國,封衛國公,同參軍國事,歲祿三千石,予世券。

四年伐蜀,命愈赴襄陽練軍馬,運糧給軍士。五年,辰、澧諸蠻作亂,以愈為征南將軍,江夏侯周德興、江陰侯吳良為副。討之。愈帥楊璟、黃彬出澧州,克四十八洞,又捕斬房州反者。六年,以右副將軍從徐達巡西北邊。十年,吐番川藏為梗,剽貢使,愈以征西將軍偕副將軍沐英討之。分兵為三道,窮追至崑崙山,俘斬萬計,獲馬牛羊十餘萬,留兵戍諸要害乃還。道病,至壽春卒,年四十一。追封寧河王,謚武順。長子鎮嗣,改封申國公,以征南副將軍平永新龍泉山寇。再出塞,有功。其妻,李善長外孫也,善長敗,坐姦黨誅。弟銘錦衣衛指揮僉事,征蠻,卒於軍。有子源為鎮後。弘治中,授源孫炳為南京錦衣衛世指揮使。嘉靖十一年詔封炳子繼坤定遠侯。五傳至文明,崇禎末,死流賊之難。

湯和,字鼎臣,濠人,與太祖同里閈。幼有奇志,嬉戲嘗習騎射,部勒群兒。及長,身長七尺,倜儻多計略。郭子興初起,和帥壯士十餘人歸之,以功授千戶。從太祖攻大洪山,克滁州,授管軍總管。從取和州。時諸將多太祖等夷,莫肯為下。和長太祖三歲,獨奉約束甚謹,太祖甚悅之。從定太平,獲馬三百。從擊陳野先,流矢中左股,拔矢復鬥,卒與諸將破擒野先。別下溧水、句容,從定集慶。從徐達取鎮江,進統軍元師。徇奔牛、呂城,降陳保二。取金壇、常州,以和為樞密院同僉守之。

常與吳接境,張士誠間諜百出,和防禦嚴密,敵莫能窺。再寇,再擊卻之,俘斬千計。進攻無錫,大破吳軍於錫山,走莫天祐,獲其妻子,進中書左丞。以舟師徇黃楊山,敗吳水軍,獲千戶四十九人,拜平章政事。援長興,與張士信戰城下。城中兵出夾擊,大敗之,俘卒八千,解圍而還。討平江西諸山寨。永新守將周安叛,進擊敗之,連破其十七寨,圍城三月,克之,執安以獻,還守常州。從大軍伐士誠,克太湖水寨,下吳江州,圍平江,戰於閶門,飛礮傷左臂,召還應天,創愈復往,攻克之,論功賜金帛。

初建御史臺,以和為左御史大夫兼太子諭德。尋拜征南將軍,與副將軍吳禎帥常州、長興、江陰諸軍,討方國珍。渡曹娥江,下餘姚、上虞,取慶元。國珍走入海,追擊敗之,獲其大帥二人、海舟二十五艘,斬馘無算,還定諸屬城。遣使招國珍,國珍詣軍門降,得卒二萬四千,海舟四百餘艘。浙東悉定。遂與副將軍廖永忠伐陳友定,自明州由海道乘風抵福州之五虎門,駐師南臺,使人諭降。不應,遂圍之。敗平章曲出於城下。參政袁仁請降,遂乘城入。分兵徇興化、漳、泉及福寧諸州縣。進拔延平,執友定送京師。時洪武元年正月也。

大軍方北伐,命造舟明州,運糧輸直沽。海多颶風,輸鎮江而還。拜偏將軍。從大將軍西征,與右副將軍馮勝自懷慶踰太行,取澤、潞、晉、絳諸州郡。從大將軍拔河中。明年,渡河入潼關,分兵趨涇州,使部將招降張良臣,既而叛去。會大軍圍慶陽,執斬之。又明年,復以右副副將軍從大將軍敗擴廓於定西,遂定寧夏,逐北至察罕腦兒,擒猛將虎陳,獲馬牛羊十餘萬。徇東勝、大同、宣府皆有功。還,授開國輔運推誠宣力武臣、榮祿大夫、柱國,封中山侯,歲祿千五百石,予世券。

四年拜征西將軍,與副將軍廖永忠帥舟師溯江伐夏。夏人以兵扼險,攻不克。江水暴漲,駐師大溪口,久不進,而傅友德已自秦、隴深入,取漢中。永忠先驅破瞿塘關,入夔州。和乃引軍繼之,入重慶,降明升。師還,友德、永忠受上賞,而和不及。明年從大將軍北伐,遇敵於斷頭山,戰敗,亡一指揮,帝不問。尋與李善長營中都宮闕。鎮北平,甓彰德城。徵察罕腦兒,大捷。九年,伯顏帖木兒為邊患,以征西將軍防延安。伯顏乞和,乃還。十一年春,進封信國公,歲祿三千石,議軍國事。數出中都、臨清、北平練軍伍,完城郭。十四年以左副將軍出塞,徵乃兒不花,破敵灰山營,獲平章別里哥、樞密使久通而還。十八年,思州蠻叛,以征虜將軍從楚王討平之,俘獲四萬,擒其酋以歸。

和沉敏多智數,頗有酒過。守常州時,嘗請事於太祖,不得,醉出怨言曰:「吾鎮此城,如坐屋脊,左顧則左,右顧則右。」太祖聞而銜之。平中原師還論功,以和征閩時放遣陳友定餘孽,八郡復擾,師還,為秀蘭山賊所襲,失二指揮,故不得封公。伐蜀還,面數其逗撓罪。頓首謝,乃已。其封信國公也,猶數其常州時過失,鐫之券。於時,帝春秋浸高,天下無事,魏國、曹國皆前卒,意不欲諸將久典兵,未有以發也。和以間從容言:「臣犬馬齒長,不堪復任驅策,願得歸故鄉,為容棺之墟,以待骸骨。」帝大悅,立賜鈔治第中都,並為諸公、侯治第。

既而倭寇上海,帝患之,顧謂和曰:「卿雖老,強為朕一行。」和請與方鳴謙俱。鳴謙,國珍從子也,習海事,常訪以禦倭策。鳴謙曰:「倭海上來,則海上禦之耳。請量地遠近,置衛所,陸聚步兵,水具戰艦,則倭不得入,入亦不得傅岸。近海民四丁籍一以為軍,戍守之,可無煩客兵也。」帝以為然。和乃度地浙西東,並海設衛所城五十有九,選丁壯三萬五千人築之,盡發州縣錢及籍罪人貲給役。役夫往往過望,而民不能無擾,浙人頗苦之。或謂和曰:「民讟矣,奈何?」和曰:「成遠算者不恤近怨,任大事者不顧細謹,復有讟者,齒吾劍。」踰年而城成。稽軍次,定考格,立賞令。浙東民四丁以上者,戶取一丁戍之,凡得五萬八千七百餘人。明年,閩中並海城工竣,和還報命,中都新第亦成。和帥妻子陛辭,賜黃金三百兩、白金二千兩、鈔三千錠、彩幣四十有副,夫人胡氏賜亦稱是。並降璽書褒諭,諸功臣莫得比焉。自是和歲一朝京師。

二十三年朝正旦,感疾失音。帝即日臨視,惋嘆久之,遣還里。疾小間,復命其子迎至都,俾以安車入內殿,宴勞備至,賜金帛御膳法酒相屬。二十七年,病浸篤不能興。帝思見之,詔以安車入覲,手拊摩之,與敘里閈故舊及兵興艱難事甚悉。和不能對,稽首而已。帝為流涕,厚賜金帛為葬費。明年八月卒,年七十,追封東甌王,謚襄武。

和晚年益為恭慎,入聞國論,一語不敢外泄。媵妾百餘,病後悉資遣之。所得賞賜,多分遺鄉曲,見布衣時故交遺老,歡如也。當時公、侯諸宿將坐姦黨,先後麗法,稀得免者,而和獨享壽考,以功名終。嘉靖間,東南苦倭患,和所築沿海城戍,皆堅緻,久且不己,浙人賴以自保,多歌思之。巡按御史請於朝,立廟以祀。

和五子。長子鼎為前軍都督僉事,從征雲南,道卒。少子醴,積功至左軍都督同知,征五開,卒於軍。鼎子晟,晟子文瑜,皆早世,不得嗣。英宗時,文瑜子傑乞嗣爵,竟以歷四十餘年未襲,罷之。傑無子,以弟倫之子紹宗為後。孝宗錄功臣後,授紹宗南京錦衣衛世指揮使。嘉靖十一年封靈璧侯,食祿千石。傳子至孫世隆,隆慶中協守南京,兼領後府,改提督漕運,歷四十餘年,以勞加太子太保,進少保。卒,謚僖敏。傳爵至明亡乃絕。

和曾孫胤勣,字公讓。為諸生,工詩,負才使氣。巡撫尚書周忱使作啟事,即席具數萬言。忱薦之朝。少保于謙召詢古今將略及兵事,胤勣應對如響。累授錦衣千戶。偕中書舍人趙榮通問英宗於沙漠,脫脫不花問中朝事,慷慨酬答不少屈。景泰中,用尚書胡濙薦,署指揮僉事。天順中,錦衣偵事者摭胤勣舊事以聞,謫為民。成化初,復故官。三年擢署都指揮僉事,為延綏東路參將,分守孤山堡。孤山最當寇衝,胤勣奏請築城聚糧,增兵戍守。未報,寇大至。胤勣病,力疾上馬,陷伏死。事聞,贈祭如例。沐英,字文英,定遠人。少孤,從母避兵,母又死。太祖與孝慈皇后憐之,撫為子,從朱姓。年十八,授帳前都尉,守鎮江。稍遷指揮使,守廣信。已,從大軍征福建,破分水關,略崇安,別破閔溪十八寨,縛馮谷保。始命復姓。移鎮建寧,節制邵武、延平、汀州三衛。尋遷大都督府僉事,進同知。府中機務繁積,英年少明敏,剖決無滯。后數稱其才,帝亦器重之。

洪武九年命乘傳詣關、陜,抵熙河,問民疾苦,事有不便,更置以聞。明年充征西副將軍,從衛國公鄧愈討吐番,西略川、藏,耀兵崑崙。功多,封開國輔運推誠宣力武臣、榮祿大夫、柱國、西平侯,食祿二千五百石,予世券。明年拜征西將軍,討西番,敗之土門峽。徑洮州,獲其長阿昌失納,築城東籠山,擊擒酋長三副使癭膆子等,平朵甘納兒七站,拓地數千里,俘男女二萬、雜畜二十餘萬,乃班師。元國公脫火赤等屯和林,數擾邊。十三年命英總陜西兵出塞,略亦集乃路,渡黃河,登賀蘭山,涉流沙,七日至其境。分四翼夜擊之,而自以驍騎衝其中堅。擒脫火赤及知院愛足等,獲其全部以歸。明年,又從大將軍北征,異道出塞,略公主山長寨,克全寧四部,度臚朐河,執知院李宣,盡俘其眾。

尋拜征南右副將軍,同永昌侯藍玉從將軍傅友德取雲南。元梁王遣平章達里麻以兵十餘萬拒於曲靖。英乘霧趨白石江。霧霽,兩軍相望,達里麻大驚。友德欲渡江,英曰:「我兵罷,懼為所扼。」乃帥諸軍嚴陳,若將渡者。而奇兵從下流濟,出其陳後,張疑幟山谷間,人吹一銅角。元兵驚擾。英急麾軍渡江,以善泅者先之,長刀斫其軍。軍卻,師畢濟。鏖戰良久,復縱鐵騎,遂大敗之,生擒達里麻,僵屍十餘里。長驅入雲南,梁王走死,右丞觀音保以城降,屬郡皆下。獨大理倚點蒼山、洱海,扼龍首、龍尾二關。關故南詔築,土酋段世守之。英自將抵下關,遣王弼由洱水東趨上關,胡海由石門間道渡河,扳點蒼山而上,立旗幟。英亂流斬關進,山上軍亦馳下,夾擊,擒段世,遂拔大理。分兵收未附諸蠻,設官立衛守之。回軍,與友德會滇池,分道平烏撒、東川、建昌、芒部諸蠻,立烏撒、畢節二衛。土酋楊苴等復煽諸蠻二十餘萬圍雲南城。英馳救,蠻潰竄山谷中,分兵捕滅之,斬級六萬。明年詔友德及玉班師,而留英鎮滇中。

十七年,曲靖亦佐酋作亂,討降之。因定普定、廣南諸蠻,通田州糧道。二十年平浪穹蠻,奉詔自永寧至大理,六十里設一堡,留軍屯田。明年,百夷思倫發叛,誘群蠻入寇摩沙勒寨,遣都督甯正擊破之。二十二年,思倫發復寇定邊,眾號三十萬。英選騎三萬馳救,置火炮勁弩為三行。蠻驅百象,被甲荷欄盾,左右挾大竹為筒,筒置標鎗,銳甚。英分軍為三,都督馮誠將前軍,甯正將左,都指揮同知湯昭將右。將戰,令曰:「今日之事,有進無退。」因乘風大呼,駮弩並發,象皆反走。昔剌亦者,寇梟將也,殊死鬥,左軍小卻。英登高望之,取佩刀,命左右斬帥首來。左帥見一人握刀馳下,恐,奮呼突陣。大軍乘之,斬馘四萬餘人,生獲三十七象,餘象盡殪。賊渠帥各被百餘矢,伏象背以死。思倫發遁去,諸蠻震懼,麓川始不復梗。已,會穎國公傅友德討平東川蠻,又平越州酋阿資及廣西阿赤部。是年冬,入朝,賜宴奉天殿,賚黃金二百兩、白金五千兩、鈔五百錠、彩幣百疋,遣還。陛辭,帝親拊之曰:「使我高枕無南顧憂者,汝英也。」還鎮,再敗百夷於景東。思倫發乞降,貢方物。阿資又叛,擊降之。南中悉定。使使以兵威諭降諸番,番部有重譯入貢者。

二十五年六月,聞皇太子薨,哭極哀。初,高皇后崩,英哭至嘔血。至是感疾,卒於鎮,年四十八。軍民巷哭,遠夷皆為流涕。歸葬京師,追封黔寧王,謚昭靖,侑享太廟。

英沉毅寡言笑,好賢禮士,撫卒伍有恩,未嘗妄殺。在滇,百務具舉,簡守令,課農桑,歲較屯田增損以為賞罰,墾田至百萬餘畝。滇池隘,浚而廣之,無復水患。通鹽井之利以來商旅,辨方物以定貢稅,視民數以均力役。疏節闊目,民以便安。居常讀書不釋卷,暇則延諸儒生講說經史。太祖初起時,數養他姓為子,攻下郡邑,輒遣之出守,多至二十餘人,惟英在西南勳最大。子春、晟、昂皆鎮雲南。昕駙馬都尉,尚成祖女常寧公主。

春,字景春,材武有父風。年十七,從英征西番,又從征雲南,從平江西寇,皆先登。積功授後軍都督府僉事。群臣請試職,帝曰:「兒,我家人,勿試也。」遂予實授。嘗命錄烈山囚,又命鞫叛黨於蔚州,所開釋各數百人。英卒,命嗣爵,鎮雲南。洪武二十六年,維摩十一寨亂,遣瞿能討平之。明年平越巂蠻,立瀾滄衛。其冬,阿資復叛,與何福討之。春曰:「此賊積年逋誅者,以與諸土酋姻婭,輾轉亡匿。今悉發諸酋從軍,縻繫之,而多設營堡,制其出人,授首必矣。」遂趨越州,分道逼其城,伏精兵道左,以羸卒誘賊,縱擊大敗之。阿資亡谷中,春陰結旁近土官,詗知所在,樹壘斷其糧道。賊困甚。已,出不意搗其巢,遂擒阿資,並誅其黨二百四十人。越州遂平。廣南酋儂貞佑糾黨蠻拒官軍,破擒之,俘斬千計。寧遠酋刀拜爛依交址不順命,遣何福討降之。

三十年,麓川宣慰使思倫發為其屬刀幹孟所逐。來奔。春挾與俱朝,受上方略,遂拜春為征虜前將軍,帥何福、徐凱討之。先以兵送思倫發於金齒,檄乾孟來迎。不應。乃選卒五千,令福與瞿能將,踰高良公山,直搗南甸,大破之,斬其酋刀名孟。回軍擊景罕寨。賊乘高堅守,官軍糧且盡,福告急。春帥五百騎救之。夜渡怒江,旦抵寨,下令騎騁,揚塵蔽天,賊大驚潰。乘勝擊崆峒寨,亦潰。前後降者七萬人。將士欲屠之,春不可。乾孟乞降,帝不許,命春總滇、黔、蜀兵攻之。末發而春卒,年三十六。謚惠襄。

春在鎮七年,大修屯政,闢田三十餘萬畝,鑿鐵池河,灌宜良涸田數萬畝,民復業者五千餘戶,為立祠祀之。無子,弟晟嗣。

晟,字景茂,少凝重,寡言笑,喜讀書。太祖愛之。歷官後軍左都督。建文元年嗣侯。比就鎮,而何福已破擒刀幹孟,歸思倫發。亡何,思倫發死,諸蠻分據其地,晟討平之。以其地為三府二州五長官司,又於怒江西置屯衛千戶所戍之,麓川遂定。初,岷王封雲南,不法,為建文帝所囚。成祖即位。遣歸籓,益驕恣。晟稍持之。王怒,譖晟。帝以王故詔誡晟,貽書岷王,稱其父功,毋督過。

永樂三年,八百大甸寇邊,遏貢使,晟會車里、木邦討定之。明年大發兵討交址,拜晟征夷左副將軍,與大將軍張輔異道自雲南入。遂由蒙自徑野蒲斬木通道,奪猛烈、掤華諸關隘。舁舟夜出洮水,渡富良江,與輔會師。共破多邦城,搗其東西二都,盪諸巢,擒偽王黎季犛,語在《輔傳》。論功封黔國公,歲祿三千石,予世券。

交址簡定復叛,命晟佩征夷將軍印討之,戰生厥江,敗績。輔再出帥師合討,擒定送京師。輔還,晟留捕陳季擴,連戰不能下。輔復出帥師會晟,窮追至占城,獲季擴,乃班師,晟亦受上賞。十七年,富州蠻叛,晟引兵臨之,弗攻,使人譬曉,竟下之。

仁宗立,加太傅,鑄征南將軍印給之。沐氏繼鎮者,輒予印以為常。宣德元年,交址黎利勢熾,詔晟會安遠侯柳升進討。升敗死,晟亦退兵。群臣交劾晟,帝封其章示之。正統三年,麓川思任發反。晟抵金齒,與弟昂及都督方政會兵。政為前鋒,破賊沿江諸寨,大軍逐北至高黎共山下,再破之。明年復破其舊寨。政中伏死,官軍敗績。晟引還,慚懼發病,至楚雄卒。贈定遠王,謚忠敬。

晟席父兄業,用兵非所長,戰數不利。朝廷以其絕遠,且世將,寬假之。而滇人懾晟父子威信,莊事如朝廷。片楮下,土酋具威儀出郭迎,盥而後啟,曰:「此令旨也。」晟久鎮,置田園三百六十區,資財充牣,善事朝貴,賂遺不絕,以故得中外聲。晟有子斌,字文輝,幼嗣公爵,居京師,而以昂代鎮。

昂,字景高,初為府軍左衛指揮僉事。成祖將使晟南討,乃擢昂都指揮同知,領雲南都司,累遷至右都督。正統四年佩將印,討麓川,抵金齒。畏賊盛,遷延者久之。參將張榮前驅至芒部敗,昂不救,引還,貶秩二級。已,思任發入寇,擊卻之,又捕斬師宗反者。六年,兵部尚書王驥、定西伯蔣貴將大軍討思任發,昂主餽運。賊破,復昂職,命督軍捕思任發,不能得。十年,昂卒。贈定邊伯,謚武襄。

斌始之鎮,會緬甸執思任發送京師,其子思機發來襲,斌擊卻之。思機發復據孟養。十三年復大發兵,使驥等討之,而斌為後拒,督餉無乏。卒,贈太傅,謚榮康。

子琮幼,景泰初,命昂孫璘以都督同知代鎮。璘字廷章,素儒雅,滇人易之,既而號令肅然不可犯,天順初卒。琮猶幼,擢璘弟錦衣副千戶瓚為都督同知,往代。居七年,先後討平霑祿諸寨及土官之構兵者,降思卜發,勒還諸蠻侵地。功多,然頗黷貨。

成化三年春,琮始之鎮,而以瓚為副總兵,移鎮金齒。琮字廷芳,通經義,能詞章,屬夷饋贄無所受。尋甸酋殺兄子,求為守,琮捕誅之。廣西土官虐,所部為亂,琮請更設流官,民大便。以次討平馬龍、麗江、劍川、順寧、羅雄諸叛蠻,捕擒橋甸、南窩反者。卒,贈太師,謚武僖。無子,以瓚孫崑嗣。

崑字元中,初襲錦衣指揮僉事。琮撫為子,朝議以崑西平侯裔孫當嗣侯,而守臣爭之,謂滇人知黔國公不知西平侯也,侯之恐為所輕。孝宗以為然,令嗣公,佩印如故。弘治十二年平龜山、竹箐諸蠻,又平普安賊,再益歲祿。正德二年,師宗民阿本作亂,與都御史吳文度督兵分三道進。一出師宗,一出羅雄,一出彌勒,而別遣一軍伏盤江,截賊巢,遂大破之。七年,安南長官司那代爭襲,殺土官,復與都御史顧源討擒之,再加太子太傅。崑初喜文學,自矜厲,其後通賂權近,所請無不得。浸驕,凌三司,使從角門入。諸言官論劾者,輒得罪去。卒,贈太師,謚莊襄。

子紹勛嗣。尋甸土舍安銓叛,都御史傅習討之,敗績。武定土舍鳳朝文亦叛,與銓連兵攻雲南,大擾。世宗遣尚書伍文定將大軍征之。未至,而紹勛督所部先進,告土官子弟當襲者,先予冠帶,破賊後當為請。眾多奮戰,賊大敗。朝文絕普渡河走,追斬之東川。銓還尋甸,列砦數十,官軍攻破之,擒銓於芒部。先後擒賊黨千餘人,俘斬無算。時嘉靖七年也。捷聞,加太子太傅,益歲祿。而是時老撾、木邦、孟養、緬甸、孟密相仇殺,師宗、納樓、思陀、八寨皆亂,久不解。紹勛使使者遍歷諸蠻,諷以武定、尋甸事,皆懾伏,願還侵地,而木邦、孟養俱貢方物謝罪。南中悉定。紹勛有勇略,用兵輒勝。卒,贈太師,謚敏靖。

子朝輔嗣。都御史劉渠索賂,朝輔與之,因上章言:「臣家世守茲土,上下相承。今有司紛更典制,關臣職守,率不與聞,接見不循故例。臣疏遠孤危,動作掣肘,無以彈壓蠻方。乞申敕諸臣,悉如其舊。」詔許之。給事中萬虞愷劾朝輔,並論渠。詔罷渠而令朝輔治事如故。卒,贈太保,謚恭僖。

二子融、鞏皆幼。詔視琮、璘故事,令融嗣公,給半祿,而授朝輔弟朝弼都督僉事,佩印代鎮。居三年,融卒,鞏當嗣,朝弼心害之,於是朝弼嫡母李請護鞏居京師,待其長而還鎮。報可。鞏未至京卒,朝弼遂得嗣。嘉靖三十年,元江土舍那鑑叛。詔朝弼與都御史石簡討之,分五軍薄其城。城垂拔,以瘴發引還。詔罷簡,將再出師。鑑懼仰藥死,乃已。四十四年討擒叛蠻阿方李向陽。隆慶初,平武定叛酋鳳繼祖,破賊巢三十餘。朝弼素驕,事母嫂不如禮,奪兄田宅,匿罪人蔣旭等,用調兵火符遣人詗京師。乃罷朝弼,以其子昌祚嗣,給半祿。朝弼怏怏,益放縱。葬母至南京,都御史請留之。詔許還滇,毋得預滇事。朝弼恚,欲殺昌祚。撫按交章言狀,並發其殺人通番諸不法事,逮繫詔獄論死。援功,錮之南京,卒。

昌祚初以都督僉事總兵官鎮守,久之嗣公爵。萬曆元年,姚安蠻羅思等叛,殺郡守。昌祚與都御史鄒應龍發土、漢兵討之,破向寧、鮓摩等十餘寨,犁其巢,盡得思等。十一年,隴川賊岳鳳叛附緬甸,挾其兵侵旁近土司。昌祚壁洱海,督裨將鄧子龍、劉綎等斬木邦叛酋罕虔,以暑瘴退師。明年復攻罕虔故巢,三道並入,擒其酋罕招等,又破緬兵於猛臉。岳鳳降。論功加太子太保,悉食故祿。復以次平羅雄諸叛蠻,再賜銀幣。緬兵攻猛廣,昌祚會師壁永昌,緬人遁,追擊至那莫江,瘴作而還。二十一年,緬人復入寇,昌祚逐之。連戰俱捷,遂傅於緬,會群蠻內亂乃還。

沐氏在滇久,威權日盛,尊重擬親王。昌祚出,僉事楊寅秋不避道,昌祚笞其輿人。寅秋訴於朝,下詔切責。已,以病,命子叡代鎮。武定土酋阿克叛,攻會城,脅府印去。叡被逮下獄,昌祚復理鎮事。卒,孫啟元嗣。卒,子天波嗣。十餘年而土司沙定洲作亂,天波奔永昌。亂定,復歸於滇。永明王由榔入滇,天波任職如故。已,從奔緬甸。緬人欲劫之,不屈死。初,沙定洲之亂,天波母陳氏、妻焦氏自焚死。後天波奔緬,妾夏氏不及從,自縊死。踰數十日收葬,支體不壞,人以為節義所感焉。

贊曰:明興諸將,以六王為稱首。非獨功茂,亦由其忠誠有以契主知焉。親莫如岐陽,舊莫如東甌,而寧河、黔寧皆以英年膺腹心之寄。汗馬宣勞,純勤不二,旂常炳耀,洵無愧矣。岐陽敦詩說禮,以儒雅見重,東甌乞身歸第,以明哲自全,皆卓然非人所能及。獨黔寧威震遐荒,剖符弈世,勛名與明相始終。而寧河盡瘁馳驅,功高齡促,後嗣亦少所表見。論者謂諸王之遺澤,隆替有殊,然而中山有增壽,與岐陽之有景隆,追溯先烈,不無遺憾。榮遇之弗齊,亦安見其有幸有不幸哉。


\end{pinyinscope}