\article{列傳第四 諸王}


明制,皇子封親王,授金冊金寶,歲祿萬石,府置官屬。護衛甲士少者三千人為皖南休寧人,故名。為學先由文字、音韻、訓詁入手,倡,多者至萬九千人,隸籍兵部。冕服車旗邸第,下天子一等。公侯大臣伏而拜謁,無敢鈞禮。親王嫡長子,年及十歲,則授金冊金寶,立為王世子,長孫立為世孫,冠服視一品。諸子年十歲,則授塗金銀冊銀寶,封為郡王。嫡長子為郡王世子,嫡長孫則授長孫,冠服視二品。諸子授鎮國將軍,孫輔國將軍,曾孫奉國將軍,四世孫鎮國中尉,五世孫輔國中尉,六世以下皆奉國中尉。其生也請名,其長也請婚,祿之終身,喪葬予費,親親之誼篤矣。考二百餘年之間,宗姓實繁,賢愚雜出。今據所紀載,自太祖時追封祔廟十五王以及列朝所封者,著於篇。而郡王以下有行義事實可採者,世系亦得附見焉。

◎諸王一

宗室十五王太祖諸子一秦王樉汧陽王誠洌晉王㭎慶成王濟炫西河王奇溯新堞周王橚鎮平王有爌博平王安水戍南陵王睦英鎮國中尉睦挈鎮國將軍安水侃鎮國中尉勤熨楚王楨武岡王顯槐齊王榑潭王梓趙王巳魯王檀歸善王當沍輔國將軍當燌奉國將軍健根安丘王當澻壽金林

熙祖,二子。長仁祖,次壽春王,俱王太后生。壽春王四子,長霍丘王,次下蔡王,次安豐王,次蒙城王。霍丘王一子,寶應王。安豐王四子,六安王、來安王、都梁王、英山王。下蔡、蒙城及寶應、六安諸王先卒,皆無後。洪武元年追封,二年定從祀禮,祔享祖廟東西廡。壽春、霍丘、安豐、蒙城四王,皆以王妃配食。蒙城王妃田氏早寡,有節行,太祖甚重之。十王、四妃墓在鳳陽白塔祠,官歲祀焉。仁祖,四子。長南昌王,次盱眙王,次臨淮王,次太祖,俱陳太后生。南昌王二子,長山陽王,無後,次文正。盱眙王一子,昭信王,無後。臨淮王無子。太祖起兵時,諸王皆前卒,獨文正在。洪武初,諸王皆追封從祀。文正以罪謫死。子守謙,封靖江王,自有傳。正德十一年,御史徐文華言:「宋儒程頤曰:『成人而無後者,祭終兄弟之孫之身。』蓋從祖而祔,亦從祖而毀,未有祖祧而祔食之孫獨存者。今懿、僖二祖既祧,太廟祔享諸王亦宜罷祀。」廷議不可,文華竟以妄言下獄。嘉靖中建九廟,東西廡如故。九廟災,復同堂異室之制,祔十五王於兩序。盱眙、臨淮王二妃配食。南昌王妃王氏,後薨,祔葬皇陵,不配食。

太祖,二十六子。高皇后生太子標、秦王樉、晉王㭎、成祖、周王橚。胡充妃生楚王楨。達定妃生齊王榑、潭王梓。郭寧妃生魯王檀。郭惠妃生蜀王椿、代王桂、谷王橞。胡順妃生湘王柏。韓妃生遼王植。餘妃生慶王旃。楊妃生寧王權。周妃生岷王楩、韓王松。趙貴妃生沈王模。李賢妃生唐王檉。劉惠妃生郢王楝。葛麗妃生伊王彞。而肅王楧母郜無名號。趙王巳、安王楹、皇子楠皆未詳所生母。

秦愍王樉,太祖第二子。洪武三年封。十一年就籓西安。其年五月賜璽書曰:「關內之民,自元氏失政,不勝其敝。今吾定天下,又有轉輸之勞,民未休息。爾之國,若宮室已完,其不急之務悉已之。」十五年八月,高皇后崩,與晉、燕諸王奔喪京師,十月還國。十七年,皇后大祥,復來朝,尋遣還。二十二年改大宗正院為宗人府,以樉為宗人令。二十四年,以樉多過失,召還京師,令皇太子巡視關陜。太子還,為之解。明年命歸籓。

二十八年正月,命帥平羌將軍甯正征叛番於洮州,番懼而降。帝悅,賚予甚厚。其年三月薨,賜謚冊曰:「哀痛者,父子之情;追謚者,天下之公。朕封建諸子,以爾年長,首封於秦,期永綏祿位,以籓屏帝室。夫何不良於德,竟殞厥身,其謚曰愍。」樉妃,元河南王王保保女弟。次妃,寧河王鄧愈女。樉薨,王妃殉。

子隱王尚炳嗣。沔人高福興等為亂,尚炳巡邊境上捕盜。永樂九年,使者至西安,尚炳稱疾不出迎,見使者又傲慢。帝逮治王府官吏,賜尚炳書曰:「齊王拜胙,遂以國霸;晉候惰玉,見譏無後。王勉之。」尚炳懼,來朝謝罪。明年三月薨。子僖王志堩嗣,二十二年薨。無子,庶兄懷王志均由渭南王嗣,宣德元年薨。妃張氏,未婚,入宮守服。

弟康王志邅嗣。好古嗜學。四年,護衛軍張嵩等訐其府中事。志邅不安,辭三護衛。宣宗答書獎諭,予一護衛。志邅顧數聽細人,正統十年誣奏鎮守都御史陳鎰,按問皆虛,而審理正秦弘等又交章奏王凌辱府僚,箠死軍役。帝再以書戒飭之。景泰六年薨。子惠王公錫嗣,以賢聞。成化二十二年薨。

子簡王誠泳嗣。性孝友恭謹,晉銘冠服以自警。秦川多賜地,軍民佃以為業,供租稅,歲歉輒蠲之。長安有魯齋書院,久廢,故址半為民居,誠泳別易地建正學書院。又旁建小學,擇軍校子弟秀慧者,延儒生教之,親臨課試。王府護衛得入學,自誠泳始。所著有《經進小鳴集》。弘治十一年薨,無子。

從弟臨潼王誠澯子昭王秉欆嗣。十四年薨。子定王惟焯嗣,有賢行,有司以聞。嘉靖十九年,敕表以綽楔。獻金助太廟工,益歲祿二百石,賜玉帶襲衣。惟焯嘗奏請潼關以西、鳳翔以東河堧地,曰:「皇祖所賜先臣樉也。」戶部尚書梁材執奏:「陜西外供三鎮,內給四王,民困已極。豈得復奪堧地,濫給宗籓。」詔如材言。二十三年薨,無子。

再從子宣王懷埢由中尉嗣。奏以本祿千石贍諸宗,賜敕獎諭。四十五年薨。子靖王敬鎔嗣,萬曆四年薨。子敬王誼旐嗣,十四年薨。無子,弟誼漶由紫陽王嗣。薨,子存樞嗣。李自成破西安,存樞降於賊,偽授權將軍,妃劉氏死之。

汧陽王誠洌,康王諸孫也,事父及繼母以孝聞。父疾,經月不解帶。及薨,醯醬鹽酪不入口。明年,墓生嘉禾,一本雙穗,嘉瓜二實並蒂,慈烏異鳥環集。以母馬妃早卒,不逮養,追服衰食蔬者三年。雪中萱草生華,咸謂孝感所致。弘治十五年賜敕嘉獎。

時有輔國將軍秉樺,亦好學篤行。父病,禱於神,乞以身代,疾竟愈。母喪廬墓,有雙鶴集庭中。定王以聞。世宗表其門。晉恭王㭎,太祖第三子也。學文於宋濂,學書於杜環,洪武三年封。十一年就籓太原,中道笞膳夫。帝馳諭曰:「吾帥群英平禍亂,不為姑息。獨膳夫徐興祖,事吾二十三年未嘗折辱。怨不在大,小子識之。」㭎修目美髯,顧盼有威,多智數。然性驕,在國多不法。或告㭎有異謀。帝大怒,欲罪之,太子力救得免。二十四年,太子巡陜西歸,岡隨來朝,敕歸籓。自是折節,待官屬皆有禮,更以恭慎聞。是時,帝念邊防甚,且欲諸子習兵事,諸王封並塞居者皆預軍務。而晉、燕二王,尤被重寄,數命將兵出塞及築城屯田。大將如宋國公馮勝、潁國公傅友德皆受節制。又詔二王,軍中事大者方以聞。三十一年三月薨,子定王濟熺嗣。

永樂初,帝以濟熺縱下,黜其長史龍潭。濟熺懼,欲上護衛。不許。弟平陽王濟熿,幼狠戾,失愛於父。及長,太祖召秦、晉、燕、周四世子及庶子之長者,教於京師。濟熿與燕王子高煦、周王子有動邪詭相比,不為太祖所愛。濟熺既嗣王,成祖封濟熿平陽王。濟熿追憾父,並憾濟熺不為解,嗾其弟慶成王濟炫等日訴濟熺過於朝,又誘府中官校,文致其罪,歷年不已。十二年,帝奪濟熺爵,及世子美圭皆為庶人,俾守恭王園,而立濟熿為晉王。

濟熿既立,益橫暴,至進毒弒嫡母謝氏,逼烝恭王侍兒吉祥,幽濟熺父子,蔬食不給。父兄故侍從宮人多為所害,莫敢言。恭王宮中老媼走訴成祖,乃即獄中召晉府故承奉左微問之,盡得濟熿構濟熺狀。立命微馳召濟熺父子,濟熺幽空室已十年矣。左微者,故因濟熺牽連繫獄,或傳微死已久。及至,一府大驚。微入空室,釋濟熺父子,相抱持大慟。時帝北征,駐驛沙城,濟喜父子謁行在所。帝見濟熺病,惻然,封美圭平陽王,使奉父居平陽,予以恭王故連伯灘田。會帝崩,濟熿遂不與美圭田。仁宗連以書諭,卒不聽。又聞朝廷賜濟熺王者冠服及他賚予,益怨望。成祖、仁宗之崩,不為服,使寺人代臨,幕中廣致妖巫為詛咒不輟。

宣宗即位,濟熿密遣人結高煦謀不軌,寧化王濟煥告變。比擒高煦,又得濟熿交通書,帝未之問也,而濟熿所遣使高煦人懼罪及,走京師首實。內使劉信等數十人告濟熿擅取屯糧十萬餘石,欲應高煦,並發其宮中詛咒事。濟煥亦至是始知嫡母被弒,馳奏。遣人察實,召至京,示以諸所發奸逆狀,廢為庶人,幽鳳陽。同謀官屬及諸巫悉論死。時宣德二年四月也。

晉國絕封凡八年,至英宗即位之二月,乃進封美圭為晉王,還居太原。正統六年薨。子莊王鐘鉉嗣,弘治十五年薨。世子奇源及其子表榮皆前卒,表榮子端王知烊嗣。知烊七歲而孤,能盡哀,居母喪嘔血,芝生寢宮。嘉靖十二年薨。無子,再從子簡王新典嗣。新化王表槏、滎澤王表檈者,端王諸父也。表槏先卒,子知節嗣為新化王,亦前卒,二子新典、新墧。端王請新典嗣新化王,未封而端王薨,表檈謀攝府事。端王妃王氏曰:「王無後,次及新化王,新化父子卒,有孫新典在。」即召入府,拜几筵為喪主。表檈忿曰:「我尊行,顧不得王。」上疏言:「新典故新化王長子,不得為人後,新典宜嗣新化王,新墧宜嗣晉王。」禮部議新典宜嗣,是為簡王。新典母太妃尚氏嚴,教子以禮。太妃疾,新典叩頭露禱。長史有敷陳,輒拜受教。其老也,以弟鎮國將軍新墧子慎鏡攝籓事。萬曆三年薨,慎鏡亦卒。弟惠王慎鋷嗣,七年薨。子穆王敏淳嗣,三十八年薨,子求桂嗣。李自成陷山西,求桂與秦王存樞並為賊所執,入北京,不知所終。

慶成王濟炫,晉恭王子。其生也,太祖方御慶成宴,因以為封。永樂元年徙居潞州,坐擅發驛馬,縱軍人為盜,被責,召還太原。十年徙汾州,薨,謚莊惠。曾孫端順王奇湞,正德中,以賢孝聞,賜敕褒獎。生子七十人,嘉靖初,尚書王瓊聞於朝。嗣王表欒樸茂寡言,孝友好文學。嘉靖三十年壽八十,詔書嘉獎,賚以金幣。輔國將軍奇添,端順王弟也,早卒。夫人王氏守節奉姑六十餘年,世宗時以節孝旌。又溫穆王曾孫中尉知恚病篤,淑人賀氏欲先死以殉,取澒一勺咽之,左右救奪,遂絕飲食,與知恚同時卒。表欒以聞。禮官言《會典》無旌命婦例,世宗特命旌之,謚曰貞烈。

西河王奇溯,定王曾孫。三歲而孤。問父所在,即慟哭。長,刻栴檀為父順簡王像,祀之。母病渴,中夜稽顙禱天,俄有甘泉自地湧出。母飲泉,病良已。及卒,哀毀骨立。子表相嗣,亦以仁孝聞,與寧河王表楠、河東嗣王奇淮並為人所稱。

新堞,恭王七世孫,家汾州。崇禎十四年由宗貢生為中部知縣。有事他邑,土寇乘間陷其城,坐免官。已而復任。署事者聞賊且至,亟欲解印去,新堞毅然曰:「此我致命之秋也。」即受之。得賊所傳偽檄,怒而碎之,議拒守。邑新遭寇,無應者,乃屬父老速去,而己誓必死。妻盧氏,妾薛氏、馮氏,請先死。許之。有女數歲,拊其背而勉之縊。左右皆泣下。乃書表封印,使人馳送京師,冠帶望闕拜,又望拜其母,遂自經。士民葬之社壇側,以妻女祔。先是,土寇薄城,縣丞光先與戰不勝,自焚死。新堞哭之慟,為之誄曰:「殺身成仁,雖死猶生。」至是,新堞亦死難。

周定王橚,太祖第五子。洪武三年封吳王。七年,有司請置護衛於杭州。帝曰:「錢塘財賦地,不可。」十一年改封周王,命與燕、楚、齊三王駐鳳陽。十四年就籓開封,即宋故宮地為府。二十二年,橚棄其國來鳳陽。帝怒,將徙之云南,尋止,使居京師,世子有燉理籓事。二十四年十二月敕歸籓。建文初,以橚燕王母弟,頗疑憚之。橚亦時有異謀,長史王翰數諫不納,佯狂去。肅次子汝南王有動告變。帝使李景隆備邊,道出汴,猝圍王宮,執橚,竄蒙化,諸子並別徙。已,復召還京,錮之。成祖入南京,復爵,加祿五千石。永樂元年正月詔歸其舊封,獻頌九章及佾舞。明年來朝,獻騶虞。帝悅,宴賜甚厚。以汴梁有河患,將改封洛陽。橚言汴堤固,無重勞民力。乃止。十四年疏辭所賜在城稅課。十八年十月有告橚反者。帝察之有驗。明年二月召至京,示以所告詞。橚頓首謝死罪。帝憐之,不復問。橚歸國,獻還三護衛。仁宗即位,加歲祿至二萬石。橚好學,能詞賦,嘗作《元宮詞》百章。以國土夷曠,庶草蕃廡,考核其可佐饑饉者四百餘種,繪圖疏之,名《救荒本草》。闢東書堂以教世子,長史劉淳為之師。洪熙元年薨。

子憲王有燉嗣,博學善書。弟有動數訐有燉,宣宗書諭之。有動與弟有熺詐為祥符王有爝與趙王書,繫箭上,置彰德城外,詞甚悖。都指揮王友得書以聞。宣宗逮友,訊無跡。召有爝至,曰:「必有動所為。」訊之具服,並得有熺掠食生人肝腦諸不法事,於是並免為庶人。有燉,正統四年薨,無子。帝賜書有爝曰:「周王在日,嘗奏身後務從儉約,以省民力。妃夫人以下不必從死。年少有父母者遣歸。」既而妃鞏氏、夫人施氏、歐氏、陳氏、張氏、韓氏、李氏皆殉死,詔謚妃貞烈,六夫人貞順。

弟簡王有爝嗣,景泰三年薨。子靖王子垕嗣,七年薨。弟懿王子驩嗣,成化二十一年薨。子惠王同鑣嗣,弘治十一年薨。世子安水橫未襲封而卒,孫恭王睦審嗣,謚安水橫悼王。

初,安水橫為世子,與弟平樂王安泛、義寧王安水矣爭漁利,置囹圄刑具,集亡賴為私人。惠王戒安水橫,不從,王怒。安泛因而傾之,安水橫亦持安泛不法事。惠王薨,群小交構,安水橫奏安泛私壞社稷壇,營私第,安泛亦誣奏安水橫諸陰事。下鎮、巡官按驗。頃之,安水橫死,其子睦審立而幼。安泛侵陵世子妃,安涘亦訐妃出不正,其子不可嗣。十三年,帝命太監魏忠、刑部侍郎何鑑按治。安泛懼,益誣世子毒殺惠王並世子妃淫亂,所連逮千人。鑒等奏其妄,廢安泛為庶人,幽鳳陽,安涘亦革爵。

嘉靖十七年,睦審薨。子勤熄先卒,孫莊王朝堈嗣,三十年薨。子敬王在鋌嗣,萬曆十年薨。子端王肅溱嗣,薨。子恭枵嗣。崇禎十四年冬,李自成攻開封,恭枵出庫金五十萬,餉守陴者,懸賞格,殪一賊予五十金。賊穴城,守者投以火,賊被爇死,不可勝計,乃解圍去。明年正月,帝下詔褒獎,且加勞曰:「此高皇帝神靈憫宗室子孫維城莫固,啟王心而降之福也。」其年四月,自成再圍汴,築長圍,城中樵採路絕。九月,賊決河灌城,城圮,恭枵從後山登城樓,率宮妃及寧鄉、安鄉、永壽、仁和諸王露棲雨中數日。援軍駐河北,以舟來迎,始獲免。事聞,賜書慰勞,並賜帑金文綺,命寄居彰德。汴城之陷也,死者數十萬,諸宗皆沒,府中分器寶藏書盡淪於巨浸。踰年,乃從水中得所奉高帝、高后金容,迎至彰德奉焉。久之,王薨,贈謚未行,國亡。其孫南走,死於廣州。

鎮平王有爌,定王第八子。嗜學,工詩,作《道統論》數萬言。又採歷代公族賢者,自夏五子迄元太子真金百餘人,作《賢王傳》若干卷。

博平王安水戍,惠王第十三子。惠王有子二十五人,而安水戍獨賢,嘗輯《貽後錄》、《養正錄》諸書。勤於治生,田園僮奴車馬甚具。賓客造門,傾己納之。其時稱名德者,必曰博平。

南陵王睦楧,悼王第九子,敏達有識。嘉靖四十一年,御史林潤言:「天下財賦,歲供京師米四百萬石,而各籓祿歲至八百五十三萬石。山西、河南存留米二百三十六萬三千石,而宗室祿米五百四萬石。即無災傷蠲免,歲輸亦不足供祿米之半。年復一年,愈加蕃衍,勢窮弊極,將何以支!」事下諸王議。明年,睦楧條上七議:請立宗學以崇德教,設科選以勵人才,嚴保勘以杜冒濫,革冗職以除素餐,戒奔競以息饕貪,制拜掃以廣孝思,立憂制以省祿費。詔下廷臣參酌之。其後諸籓遂稍稍陳說利弊,尚書李春芳集而上焉。及頒《宗籓條例》,多採睦楧議云。

鎮國中尉睦挈,字灌甫,鎮平王諸孫。父奉國將軍安河以孝行聞於朝,璽書旌賚。既沒,周王及宗室數百人請建祠。詔賜祠額曰「崇孝」。睦挈幼端穎,郡人李夢陽奇之。及長,被服儒素,覃精經學,從河、洛間宿儒游。年二十通《五經》,尤邃於《易》、《春秋》。謂本朝經學一稟宋儒,古人經解殘闕放失,乃訪求海內通儒,繕寫藏棄,若李鼎詐《易解》、張洽《春秋傳》,皆敘而傳之。呂柟嘗與論《易》,歎服而去。益訪購古書圖籍,得江都葛氏、章丘李氏書萬卷,丹鉛歷然,論者以方漢之劉向。築室東坡,延招學者,通懷好士,而內行修潔。事親晨昏不離側,喪三年居外舍。有弟五人,親為教督,盡推遺產與之。萬曆五年舉文行卓異,為周籓宗正,領宗學。約宗生以三、六、九日午前講《易》、《詩》、《書》,午後講《春秋》、《禮記》,雖盛寒暑不輟。所撰有《五經稽疑》六卷,《授經圖傳》四卷,《韻譜》五卷,又作《明帝世表》、《周國世系表》、《建文遜國褒忠錄》、《河南通志》、《開封郡志》諸書。巡撫御史褚鈇議稍減郡王以下歲祿,均給貧宗,帝遣給事中萬象春就周王議。新會王睦樒號於眾曰:「裁祿之謀起於睦挈。」聚宗室千餘人擊之,裂其衣冠,上書抗詔。帝怒,廢睦樒為庶人。睦挈屢疏引疾乞休,詔勉起之。又三年卒,年七十。宗人頌功德者五百人,詔賜輔國將軍,禮葬之,異數也。學者稱為西亭先生。

時有將軍安水侃者,一歲喪母,事其父以孝聞。父病革,刲臂為湯飲父,父良已。年七十,追念母不逮養,服衰廬墓三年,詔旌其門。素精名理,聲譽大著,人稱睦挈為「大山」,安水侃為「小山」云。

又勤熨者,鎮國中尉也,嘉靖中,上書曰:「陛下躬上聖之資,不法古帝王兢業萬歲,擇政任人,乃溺意長生,屢修齋醮,興作頻仍。數年來朝儀久曠,委任非人,遂至賄賂公行,刑罰倒置,奔競成風,公私殫竭,脫有意外變,臣不知所終。」帝覽疏怒,坐誹謗,降庶人,幽鳳陽。子朝已賜名,以罪人子無敢為請封者,上書請釋父罪,且陳中興四事,詔並禁錮。穆宗登極,釋歸,命有司存恤。楚昭王楨,太祖第六子。始生時,平武昌報適至,太祖喜曰:「子長,以楚封之。」洪武三年封楚王。十四年就籓武昌。嘗錄《御註洪範》及《大寶箴》置座右。十八年四月,銅鼓、思州諸蠻亂,命楨與信國公湯和、江夏侯周德興帥師往討。和等分屯諸洞,立柵與蠻人雜耕作。久之,擒其渠魁,餘黨悉潰。三十年,古州蠻叛,帝命楨帥師,湘王柏為副,往征。楨請餉三十萬,又不親蒞軍。帝詰責之,命城銅鼓衛而還。是年,熒惑入太微,詔諭楨戒慎,楨書十事以自警。未幾,楨子巴陵王卒,帝復與敕曰:「舊歲熒惑入太微,太微天庭,居翼軫,楚分也。五星無故入,災必甚焉。爾子疾逝,恐災不止此,尚省慎以回天意。」至冬,王妃薨。時初設宗人府,以楨為右宗人。永樂初,進宗正。二十二年薨。

子莊王孟烷嗣,敬慎好學。宣德中,平江伯陳瑄密奏:「湖廣,東南大籓,襟帶湖、湘,控引蠻越,人民蕃庶,商賈輻聚。楚設三護衛,自始封至今,生齒日繁,兵強國富,小人行險,或生邪心。請選其精銳,以轉漕為名,俟至京師,因而留之,可無後患。」帝曰:「楚無過,不可。」孟烷聞之懼。五年上書請納兩護衛,自留其一。帝勞而聽之。正統四年薨。

子憲王季堄嗣。事母鄧妃至孝。英宗賜書獎諭。著《東平河間圖贊》,為士林所誦。八年薨。弟康王季埱嗣。天順六年薨。再從子靖王均鈋嗣,正德五年薨。子端王榮水戒嗣,以仁孝著稱,武宗表曰「彰孝之坊」。嘉靖十三年薨。子愍王顯榕嗣,居喪哀痛,遇慶禮卻賀。端王婿儀賓沈寶與顯榕有隙,使人誣奏顯榕左右呼顯榕萬歲,且誘顯榕設水戲以習水軍。世宗下其章,撫臣具言顯榕居喪能守禮。寶坐誣,削為民。

顯榕妃吳氏,生世子英耀,性淫惡,嘗烝顯榕宮人。顯榕知之,杖殺其所使陶元兒。英耀又使卒劉金納妓宋麼兒於別館。顯榕欲罪金,金遂誘英耀謀為逆。嘉靖二十四年正月十八日,張燈置酒饗顯榕,別宴顯榕弟武岡王顯槐於西室。酒半,金等從座後出,以銅瓜擊顯榕腦,立斃。顯槐驚救,被傷,奔免。英耀徙顯榕屍宮中,命長史孫立以中風報。王從者朱貴抉門出告變,撫、按官以聞。英耀懼,具疏奏辨,且逼崇陽王顯休為保奏。通山王英炊不從,直奏英耀弒逆狀。詔遣中官及駙馬都尉鄔景和、侍郎喻茂堅往訊。英耀辭服。詔逮入京。是年九月,告太廟,伏誅,焚屍揚灰。悉誅其黨,革顯休祿十之三。顯槐、英炊皆賚金幣,而以顯榕次子恭王英僉嗣。隆慶五年薨。

子華奎幼,萬曆八年,始嗣爵。衛官王守仁上告曰:「遠祖定遠侯弼,楚王楨妃父也,遺瑰寶數十萬寄楚帑,為嗣王侵匿。」詔遣中官清核。華奎奏辨,且請避宮搜掘。皆不報。久之,繫鞫王府承奉等,無所得。時諸璫方以搜括希上意,不欲暴守仁罪。帝頗悟,罷其事。華奎乃奏上二萬金助三殿工。

三十一年,楚宗人華等言:「華奎與弟宣化王華壁皆非恭王子。華奎乃恭王妃兄王如言子,抱養宮中。華壁則王如綍家人王玉子也。華妻,即如言女,知之悉。」禮部侍郎郭正域請行勘。大學士沈一貫右華奎,委撫按訊,皆言偽王事無左驗。而華妻持其說甚堅,不能決,廷議令覆勘。中旨以楚王襲封已二十餘年,宜治華等誣罔。御史錢夢皋為一貫劾正域,正域發華奎行賄一貫事。華奎遂訟言正域主使,正域罷去。東安王英燧、武岡王華增、江夏王華塇等皆言偽跡昭著,行賄有據。諸宗人赴都投揭。奉旨切責,罰祿、削爵有差。華坐誣告,降庶人,錮鳳陽。未幾,華奎輸賄入都,宗人遮奪之。巡撫趙可懷屬有司捕治。宗人蘊鉁等方恨可懷治楚獄不平,遂大哄,毆可懷死。巡按吳楷以楚叛告。一貫擬發兵會剿。命未下,諸宗人悉就縛。於是斬二人,勒四人自盡,錮高墻及禁閒宅者復四十五人。三十三年四月也。自是無敢言楚事者。久之,禁錮諸人以恩詔得釋,而華奎之真偽竟不白。

其後,張獻忠掠湖廣,華奎募卒自衛,以張其在為帥。獻忠兵至武昌,其在為內應,執華奎沉之江,諸宗無得免者。

武岡王顯槐,端王第三子也。嘉靖四十三年上書條籓政,請「設宗學,擇立宗正、宗表,督課親郡王以下子弟。十歲入學,月餼米一石,三載督學使者考績,陟其中程式者全祿之,五試不中課則黜之,給以本祿三之二。其庶人暨妻女,月餼六石,庶女勿加恩。」其後廷臣集議,多采其意。

齊王榑,太祖第七子。洪武三年封。十五年就籓青州。二十三年命王帥護衛及山東徐、邳諸軍從燕王北征。二十四年復帥護衛騎士出開平。時已令潁國公傅友德調發山東都司各衛軍出塞,諭王遇敵當自為隊,奏凱之時勿與諸將爭功。榑數歷塞上,以武略自喜,然性凶暴,多行不法。建文初,有告變者。召至京,廢為庶人,與周王同禁錮。

燕兵入金川門,急遣兵護二王。二王卒不知所以,大怖,伏地哭。已知之,乃大喜。成祖令王齊如故,榑益驕縱。帝與書召來朝,面諭王無忘患難時。尃不悛,陰畜刺客,招異人術士為咒詛,輒用護衛兵守青州城,並城築苑牆斷往來,守吏不得登城夜巡。李拱、曾名深等上急變,榑拘匿以滅口。永樂三年詔索拱,諭榑改過。是時,周王橚亦中浮言,上書謝罪,帝封其書示榑。明年五月來朝,廷臣劾榑罪。榑厲聲曰:「奸臣喋喋,又欲效建文時耶!會盡斬此輩。」帝聞之不懌,留之京邸。削官屬護衛,誅指揮柴直等,盡出榑繫囚及所造不法器械。群臣請罪教授葉垣等,帝曰:「王性凶悖,朕溫詔開諭至六七,猶不悟,教授輩如王何!垣等先自歸發其事,可勿問。」榑既被留,益有怨言。是年八月,召其子至京師,並廢為庶人。

宣德三年,福建妄男子樓濂詭稱七府小齊王,謀不軌。事覺,械至京,誅其黨數百人。榑及三子皆暴卒,幼子賢爀安置廬州。景泰五年徙齊庶人、谷庶人置南京,敕守臣慎防。後谷庶人絕,齊庶人請得穀庶人第。嘉靖十三年釋高墻庶人長毚,榑曾孫也。萬曆中有承彩者,亦榑裔。齊宗人多凶狡,獨承彩頗好學云。

潭王梓,太祖第八子。洪武三年封。十八年就籓長沙。梓英敏好學,善屬文。嘗召府中儒臣,設醴賦詩,親品其高下,賚以金幣。妃於氏,都督顯女也。顯子琥,初為寧夏指揮。二十三年坐胡惟庸黨,顯與琥俱坐誅。梓不自安。帝遣使慰諭,且召入見。梓大懼,與妃俱焚死。無子,除其封。

趙王巳,太祖第九子。洪武二年生。次年受封,明年殤。

魯荒王檀,太祖第十子。洪武三年生,生兩月而封。十八年就籓兗州。好文禮士,善詩歌。餌金石藥,毒發傷目。帝惡之。二十二年薨,謚曰荒。子靖王肇輝,甫彌月。母妃湯,信國公和女,撫育教誨有度。永樂元年三月始得嗣。成祖愛重之。車駕北巡過兗,錫以詩幣。宣德初,上言:「國長史鄭昭、紀善王貞,奉職三十年矣,宜以禮致其事。」帝謂蹇義曰:「皇祖稱王禮賢敬士,不虛也。」許之。成化二年薨。

子惠王泰堪嗣,九年薨。子莊王陽鑄嗣,嘉靖二年薨。莊王在位久,世子當漎,當漎子健杙皆前卒,健杙子端王觀定嗣。狎典膳秦信等,游戲無度,挾娼樂,裸男女雜坐。左右有忤者,錐斧立斃,或加以炮烙。信等乘勢殘殺人。館陶王當淴亦淫暴,與觀定交惡,相訐奏。帝念觀定尚幼,革其祿三之二,逮誅信等,亦革當淴祿三之一。二十八年,觀定薨。子恭王頤坦嗣,有孝行,捐邸中田湖,贍貧民,辭常祿,給貧宗。前後七賜璽書嘉勞。萬曆二十二年薨。世子壽金爵先卒,弟敬王壽鏳嗣,二十八年薨。弟憲王壽鋐嗣,崇禎九年薨。弟肅王壽鏞嗣,薨。子以派嗣,十五年,大清兵克兗州,被執死。弟以海轉徙台州,張國維等迎居於紹興,號魯監國。順治三年六月,大兵克紹興,以海遁入海。久之,居金門,鄭成功禮待頗恭。既而懈,以海不能平,將往南澳。成功使人沉之海中。

歸善王當沍,莊王幼子也。正德中,賊攻兗州,帥家眾乘城,取護衛弓弩射卻賊。降敕獎諭,遂以健武聞。時有卒袁質與舍人趙巖俱家東平,武斷為鄉人所惡。吏部主事梁穀,亦東平人,少不檢,倚惡少為助,既貴,頗厭苦之,又與千戶高乾有怨。正德九年,穀邑人西鳳竹、屈昂誑穀云:「質、巖且為亂。」穀心動,因並指乾等,告變於尚書楊一清。兵部議以大兵駐濟南伺變。先是,當沍數與質、巖校射。至是當沍父莊王聽長史馬魁譖言當沍結質、巖欲反,虞禍及,奏於朝。帝遣司禮太監溫祥、大理少卿王純、錦衣衛指揮韓端往按問。祥等至,圍當沍第,執之。祥等讞穀所指皆平人。魁懼事敗,乃諷所厚陳環及術士李秀佐證之,復以書及賄抵鎮守太監畢真,使逮二人詰問。已而二人以實對,書賄事亦為真所發。於是御史李翰臣劾穀報怨邀功,長史魁惑王罔奏,宜即訊。詔下翰臣獄,謫廣德州判官,免穀罪不問。御史程啟充等疏言:「穀、魁鼓煽流言,死不蔽罪,縱首禍而謫言者,非國體。」不報。廷臣議當沍罪,卒無所坐。以藏護衛兵器違祖制,廢為庶人。戍質等於肅州。所連逮多瘐死,魁坐誣奏斬。鳳竹、昂流口外。中官送當沍之高墻,當沍大慟曰:「冤乎!」觸墻死。聞者傷之。輔國將軍當濆,鉅野王泰墱諸孫也,慷慨有志節。嘉靖三年上書請停郡縣主、郡縣君恤典,以蘇民困。七年奏辭輔國將軍並子奉國將軍祿,佐疏運河。賜敕褒諭。又上書言:「各籓郡縣主、郡縣君先儀賓沒者,故事儀賓得支半祿。今四方災傷,邊陲多事,民窮財盡,而各儀賓暴橫侈肆,多不法,請勿論品級,減其月給。」明年又請以父子應得祿米佐振。因勸帝法祖宗,重國本,裁不急之費,息土木之工。詞甚愷切。帝嘉其意,特敕褒之,不聽辭祿。時東甌王健楸無子,上書言:「宗室所以蕃,由詐以媵子為嫡,糜費縣官。今臣無嫡嗣,請以所受府第屯廠盡歸魯府,待給新封,省民財萬一,乞著為例。」報可。

奉國將軍健根,鉅野王陽鎣諸孫。博通經術,年七十,猶縱談名理,亹亹不倦。嘉靖中,詔褒其賢孝。子鎮國中尉觀熰,字中立,居母喪,蔬食逾年,哀毀骨立。嘗繪《太平圖》上獻。世宗嘉獎之,賜承訓書院名額並《五經》諸書。弟觀以詩畫著名。同時鉅野中尉頤堟、安丘將軍頤墉,聲詩清拔。樂陵王頤戔亦喜稱詩。

安丘王當澻,靖王曾孫,少孤,事祖父母以孝聞。曾孫頤堀好學秉禮,尤諳練典故。籓邸中有大疑,輒就決。一意韜晦,監司守令希見其面。年七十餘,猶手不廢書。

魯府宗室壽金林,家兗州。崇禎中為雲南通判,有聲績。永明王由榔在廣西,以為右僉都御史,使募兵。值沙定州亂,兵不能集。孫可望兵至,壽金林知不免,張麾蓋往見之,行三揖禮曰:「謝將軍不殺不掠之恩。」可望脅之降,不從。系他所,使人誘以官,終不從。從容題詩於壁,或以詩報可望,遂遇害。


太祖諸子二蜀王椿湘王柏代王桂襄垣王遜燂靈丘王遜烇成金具廷鄣肅王楧遼王植慶王勍寧王權

蜀獻王椿,太祖第十一子,洪武十一年封。十八年命駐鳳陽。二十三年就籓成都。性孝友慈祥,博綜典籍,容止都雅,帝嘗呼為「蜀秀才」。在鳳陽時,闢西堂,延李叔荊、蘇伯衡商榷文史。既至蜀,聘方孝孺為世子傅,表其居曰「正學」,以風蜀人。詣講郡學,知諸博士貧,分祿餼之,月一石,後為定制。造安車賜長史陳南賓。聞義烏王紳賢,聘至,待以客禮。紳父禕死雲南,往求遺骼,資給之。

時諸王皆備邊練士卒,椿獨以禮教守西陲。番人入寇,燒黑崖關。椿請於朝,遣都指揮瞿能隨涼國公藍玉出大渡河邀擊之。自是番人讋伏。前代兩川之亂,皆因內地不逞者鉤致為患;有司私市蠻中物,或需索啟爭端。椿請繒錦香扇之屬,從王邸定為常貢,此外悉免宣索。蜀人由此安業,日益殷富。川中二百年不被兵革,椿力也。

成祖即位,來朝。賜予倍諸籓。谷王橞,椿母弟也,圖不軌。椿子悅燇,獲咎於椿,走橞所,橞稱為故建文君以詭眾。永樂十四年,椿暴其罪。帝報曰:「王此舉,周公安王室之心也。」入朝,賚金銀繒彩鉅萬。二十一年薨。

世子悅熑先卒,孫靖王友堉嗣。初,華陽王悅爠謀奪嫡,椿覺之,會有他過,杖之百,將械於朝。友堉為力請,得釋。椿之薨,友堉方在京師,悅爠竊王帑,友堉歸不問。悅爠更誣奏友堉怨誹。成祖召入訊之,會崩。仁宗察其誣,命歸籓。召悅爠,悅爠,猶執奏。仁宗抵其章於地,遷之武岡,復遷澧州。宣德五年,總兵官陳懷奏都司私遺蜀邸砲,用以警夜,非制。詔逮都司首領官。明年獻還二護衛。從之。是年薨。妃李、侍姬黃皆自經以殉。無子,弟僖王友黨由羅江王嗣,九年薨。獻王第五子和王悅菼由保寧王嗣,天順五年薨。繼妃徐氏,年二十六,不食死,謚靜節。子定王友垓嗣,七年薨。子懷王申鈘嗣,成化七年薨。弟惠王申鑿嗣,弘治六年薨。子昭王賓瀚嗣,正德三年薨。子成王讓栩嗣。

自椿以下四世七王,幾百五十年,皆檢飭守禮法,好學能文。孝宗恒稱蜀多賢王,舉獻王家範為諸宗法。讓栩尤賢明,喜儒雅,不邇聲伎,創義學,修水利,振災恤荒。嘉靖十五年,巡撫都御史吳山、巡按御史金粲以聞。賜敕嘉獎,署坊表曰「忠孝賢良」。二十年建太廟,獻黃金六十斤,白金六百斤。酬以玉帶幣帛。二十六年薨。子康王承龠嗣,三十七年薨。子端王宣圻嗣,萬歷四十年薨。子恭王奉銓嗣,四十三年薨。子至澍嗣。崇禎末,京師陷,蜀尚無恙。未幾,張獻忠陷成都,合宗被害,至澍率妃妾投於井。

湘獻王柏,太祖第十二子。洪武十一年封。十八年就籓荊州。性嗜學,讀書每至夜分。開景元閣,招納俊乂,日事校仇,志在經國。喜談兵,膂力過人,善弓矢刀槊,馳馬若飛。三十年五月,同楚王楨討古州蠻,每出入,縹囊載書以隨,遇山水勝境,輒徘徊終日。尤善道家言,自號紫虛子。建文初,有告柏反者,帝遣使即訊。柏懼,無以自明,闔宮焚死。謚曰戾。王無子,封除。永樂初,改謚獻,置祠官守其園。

代簡王桂,太祖第十三子。洪武十一年封豫王,二十五年改封代。是年就籓大同。糧餉艱遠,令立衛屯田以省轉運。明年詔帥護衛兵出塞,受晉王節制。桂性暴,建文時,以罪廢為庶人。

成祖即位,復爵。永樂元年正月還舊封。十一月賜璽書曰:「聞弟縱戮取財,國人甚苦,告者數矣,且王獨不記建文時耶?」尋命有司,自今王府不得擅役軍民、斂財物,聽者治之。已復有告其不軌者,賜敕列其三十二罪,召入朝,不至。再召,至中途,遣還,革其三護衛及官屬。王妃中山王徐達女,仁孝文皇后妹也,驕妒,嘗漆桂二侍女為癩。事聞,帝以中山王故,不罪。桂移怒世子遜耑,出其母子居外舍。桂已老,尚時時與諸子遜炓、遜焴窄衣禿帽,遊行市中,袖錘斧傷人。王府教授楊普上言:「遜炓狎軍人武亮,與博戲,致棰殺軍人。」朝廷杖治亮,降敕責戒,稍斂戢。十六年四月復護衛及官屬。

正統十一年,桂薨。世子遜耑先卒,孫隱王仕廛嗣。景泰中,嘗上言總兵官郭登守城功,朝廷為勞登。天順七年薨。子惠王成煉嗣,弘治二年薨。子聰沫先封武邑王,以肆酒革爵。已,居惠王喪,益淫酗,廢為庶人,遷太原。久之,惠王妃為疏理,復封武邑王,卒。子懿王俊杖襲封代王。

嘉靖三年,大同軍叛,圍王宮,俊杖走免。事平,賜書慰問。六年薨。子昭王充耀嗣。十二年,大同軍又叛,充耀走宣府,再賜慰問。事平,返國,奏:「亂賊既除,軍民交困,乞遣大臣振撫。」詔允行。二十四年,和川奉國將軍充灼坐罪奪祿,怨充耀不為解,乃與襄垣中尉充耿謀引敵入大同殺王。會應州人羅廷璽等以白蓮教惑眾,見充灼為妖言,因畫策,約奉小王子入塞,藉其兵攻鴈門,取平陽,立充灼為主,事定,即計殺小王子。充灼然之。先遣人陰持火箭,焚大同草場五六所,而令通蒙古語者衛奉闌出邊,為總兵周尚文邏卒所獲,並得其所獻小王子表,鞫實以聞。逮充灼等至京,賜死,焚其屍,王府長史等官皆逮治。總督侍郎翁萬達疏言:「大同狹瘠,祿餉不支,代宗日繁衍,眾聚而貧。且地近邊,易生反側。請量移和川、昌化諸郡王於山、陜隙地。」詔改遷於山西。先是,景泰間,昌化王仕墰乞移封,景帝不許,至是乃遷。代宗自簡至懿,封郡王者凡二十有三,而外徙者十王。

二十六年,充耀薨。子恭王廷埼嗣。饒陽王充跼數以事侵廷埼,恐得罪,乃以陳邊事為名,三十一年奏鎮、巡官之罪。世宗為黜巡撫都御史何思,逮總兵官徐仁等。充跼益驕,遂與廷埼互訐,前後勘官莫能判。巡撫都御史侯鉞奏奪其祿,充跼怒不承。三十三年詔遣司禮少監王臻即訊,充跼乃伏,下法司,錮高牆。萬曆元年,廷埼薨。子定王鼐鉉嗣,二十二年薨。無子,弟新寧王鼐鈞嗣,薨。子康王鼎渭嗣,崇禎二年薨。再傳至孫傳齊。崇禎十七年三月,李自成入大同,闔門遇害。

襄垣王遜燂,簡王第五子,分封蒲州。諸王就籓後,非請命不得歲時定省。遜燂念大同不置,作《思親篇》,詞甚悲切。其後,宗人聰浼、聰泈、俊難、俊榷、俊朵、俊杓、俊噤、充焞,皆嫻於文章。俊噤,字若訥,尤博學,有盛名,不慕榮利。姊陵川縣君,適裴禹卿,地震城崩,禹卿死。縣君以首觸棺,嘔血卒。年二十有一。詔謚貞節。

靈丘王遜烇,簡王第六子。宣宗時封。好學工詩,尤善醫,嘗施藥治瘟疫,全活無算。子仕塝、孫成鈠、曾孫聰滆,三王皆以孝旌。聰滆子俊格,能文善書。嘉靖時,獻《皇儲明堂》二頌、《興獻帝后挽歌》,賜金帛。聰滆嘗乞封其孫廷址為曾長孫,禮官奏無故事。帝以王壽考,特許之。已而復封廷址子鼐鐮為玄長孫。聰滆薨,年八十三。鼐鐮襲高祖爵。聰滆之從父成金微亦有孝行,聰滆聞於朝,賜金幣獎諭。詔禮官自今宗室中孝行卓異如成金微者,撫按疏聞。

又成金具者,隰川王諸孫。父仕則,坐罪幽鳳陽,病死。成金具微服走鳳陽視喪,上疏自劾越禁,乞負父骨歸葬澤州,即不得,願為庶人,止墓側,歲時省視。詔許歸葬。弟成鐎亦好學,有志概。嘉靖十三年上言:』雲中叛卒之變幸獲銷弭。究其釁端,實貪酷官吏激成之。臣慮天下之禍隱於民心,異日不獨雲中而已。」指陳切直,帝下廷臣飭行。時以其兄弟為二難焉。萬曆二十年,西夏弗寧,山陰王俊柵奏詩八章,寓規諷之旨。代處塞上,諸宗洊經禍亂,其言皆憂深思遠,有中朝士大夫所不及者。

廷鄣,代府宗室也。崇禎中,為鞏昌府通判,署秦州事,有廉直聲。十六年冬,賊陷秦州,被執。使之跪,叱曰:「我天朝宗姓,頭可斷,膝不可屈。」賊欲活之,大呼曰:「今日惟求一死。」坐自若,遂見害。肅莊王楧,太祖第十四子。洪武十一年封漢王。二十四年命偕衛、谷、慶、寧、岷五王練兵臨清。明年改封肅。又明年,詔之國,以陜西各衛兵未集,命駐平涼。二十八年始就籓甘州。詔王理陜西行都司甘州五衛軍務。三十年令督軍屯糧,遇征伐以長興侯耿炳文從。建文元年乞內徙,遂移蘭州。永樂六年,以捶殺衛卒三人及受哈密進馬,逮其長史官屬。已,又聽百戶劉成言,罪平涼衛軍,敕械成等京師。十七年薨。子康王瞻焰嗣。宣德七年上一護衛。府中被盜,為榜募告捕者。御史言非制,罪其長史楊威。瞻焰又請加歲祿。敕曰:「洪武、永樂間,歲祿不過五百石,莊王不言者,以朝廷念遠地轉輸難故也。仁考即位,加五百石矣。朕守祖制不敢違。」正統元年上言:「甘州舊邸改都司,而先王墳園尚在,乞禁近邸樵採。」從之。天順三年上馬五百匹備邊,予直不受。帝強予之。八年薨。

子簡王祿埤嗣,成化十五年薨。子恭王貢錝嗣,嘉靖十五年薨。世子真淤、長孫弼桓皆早卒,次孫定王弼桄嗣,四十一年薨。子縉炯先卒,孫懷王紳堵嗣,踰二年薨。無子,靖王第四子弼柿子輔國將軍縉貴,以屬近宜嗣。禮官言,縉貴,懷王從父,不宜襲。詔以本職理府事,上冊寶,罷諸官屬。穆宗即位,定王妃吳氏及延長王真滰等先後上言:「聖祖刈群雄,定天下,報功之典有隆無替。臣祖莊王受封邊境,操練征戍,屏衛天家。不幸大宗中絕,反拘於昭穆之次,不及勛武繼絕之典,非所以崇本支,厚籓衛也。」下部議,議以郡王理籓政。帝不許。隆慶五年,特命縉貴嗣肅王,設官屬之半。萬曆十六年薨,謚曰懿。子憲王紳堯嗣,四十六年薨。子識鋐嗣。崇禎十六年冬,李自成破蘭州,被執,宗人皆死。遼簡王植,太祖第十五子。洪武十一年封衛王,二十五年改封遼。明年就籓廣寧。以宮室未成,蹔駐大凌河北,樹柵為營。帝命武定侯郭英為築城郭宮室。英,王妃父也,督工峻急。會高麗自國中至鴨綠江皆積粟,帝慮其有陰謀,而役作軍士艱苦,令輟役。至三十年,始命都督楊文督遼東諸衛士繕治之,增其雉堞,以嚴邊衛。復圖西北沿邊要害,示植與寧王權,諭之曰:「自東勝以西至寧夏、河西、察罕腦兒,東勝以東至大同、宣府、開平,又東南至大寧,又東至遼東,抵鴨綠江,北至大漠,又自鴈門關外,西抵黃河,渡河至察罕腦兒,又東至紫荊關,又東至居庸關及古北口,又東至山海衛,凡軍民屯種地,毋縱畜牧。其荒曠地及山場,聽諸王駙馬牧放樵採,東西往來營駐,因以時練兵防寇。違者論之。」植在邊,習軍旅,屢樹軍功。建文中,「靖難」兵起,召植及寧王權還京。植渡海歸朝,改封荊州。永樂元年入朝,帝以植初貳於已,嫌之。十年削其護衛,留軍校廚役三百人,備使令。二十二年薨。子長陽王貴烚嗣。

初,植庶子遠安王貴燮、巴東王貴煊嘗告其父有異謀。及父死,又不奔喪。仁宗即位,皆廢為庶人。正統元年,府僚乞加王祿。敕曰:「簡王得罪朝廷,成祖特厚待,仁宗朝加祿,得支二千石。宣宗又給旗軍三百人,親親已至。王素乖禮度,府臣不匡正,顧為王請乎!」不許。三年,巡撫侍郎吳政奏王不友諸弟,待庶母寡恩,捶死長史杜述,居國多過。召訊京師,盡得其淫穢黷倫、兇暴諸不法事。明年四月廢為庶人,守簡王園。弟肅王貴受嗣,成化七年薨。子靖王豪墭嗣,十四年薨。子惠王恩金稽嗣。

弘治五年,松滋王府諸宗人恩鑡等闌入荊州府支歲祿,恩金稽禁之,皆怨。已,儀賓袁鏞復誘恩鑡等招群小,奪軍民商賈利。恩金稽發其事,恩鑡等愈怨,謀殺王。朝廷遣官按實,幽恩鑡等鳳陽,謫戍其黨有差。恩金稽陰使送者刑梏之,斃八十餘人。不數日,世子暴卒。八年,恩金稽疽發背薨。子恭王寵水受嗣,與弟光澤王寵水寰友愛,飲食服御必俱。寵水寰有令德,寵涭有事必咨之後行。正德十六年薨。子莊王致格嗣,病不視事。妃毛氏明書史,沉毅有斷,中外肅然,賢聲聞天下。

嘉靖十六年,致格薨。子憲節嗣,以奉道為世宗所寵,賜號清微忠教真人,予金印。隆慶元年,御史陳省劾憲節諸不法事,詔奪真人號及印。明年,巡按御史郜光先復劾其大罪十三,命刑部侍郎洪朝選往勘,具得其淫虐僭擬諸罪狀。帝以憲節宜誅,念宗親免死,廢為庶人,錮高牆。初,副使施篤臣憾憲節甚,朝選至湖廣,篤臣詐為憲節書餽朝選,因劫持之。憲節建白纛,曰「訟冤之纛」。篤臣驚曰:「王反矣。」使卒五百圍王宮。朝選還朝,實王罪,不言王反。大學士張居正家荊州,故與憲節有隙,嫌朝選不坐憲節反。久之,屬巡撫都御史勞堪羅織朝選,死獄中。其後居正死,憲節訟冤,籍居正家,而篤臣亦死。遼國除,諸宗隸楚籓,以廣元王術周為宗理。慶靖王旃,太祖第十六子。洪武二十四年封。二十六年就籓寧夏。以餉未敷,令駐慶陽北古韋州城,就延安、綏、寧租賦。二十八年詔王理慶陽、寧夏、延安、綏德諸衛軍務。三十年始建邸。王好學有文,忠孝出天性。成祖善之,令歲一至韋州度夏。宣德初,言寧夏卑濕,水泉惡,乞仍居韋。不許,令歲一往來,如成祖時。正統初,寧夏總兵官史昭奏王沮邊務,占靈州草場畜牧,遣使由綏德草地往還,煽惑土民。章未下,或告王閱兵,造戎器,購天文書。旃疑皆昭為之。三年上書,請徙國避昭。英宗不可,貽書慰諭。其年薨,子康王秩煃嗣。景泰元年以寧夏屢被兵,乞徙內地,不許。成化五年薨。子懷王邃欻嗣,十五年薨。弟莊王邃塀嗣,弘治四年薨。子恭王寘錖嗣,十一年薨。子定王台浤嗣。

正德五年,安化王寘鐇反,台浤稽首行君臣禮。詔削護衛,革祿三之一,戍其承奉、長史。嘉靖三年,台浤賄鎮守太監李昕、總兵官種勛,求為奏請復祿。昕、勛不納,台浤銜之。會寧夏衛指揮楊欽等得罪於巡撫都御史張璿,謀藉王殺璿及勛。事覺,下都司、按察司按治,欽等誣台浤不軌,璿以聞。帝使太監扶安、副都御史王時中等復按。上言:「台浤他罪有之,無謀不軌事。」詔廷臣定議,坐前屈事寘鐇,蒙恩不悛,煽構群小,謀害守臣,廢為庶人,留邸,歲與米三百石,以其叔父鞏昌王寘銂攝府事。

寘銂裁慶邸宮妃薪米,取邸中金帛萬計。台浤子鼒櫍幼失愛於父,逃寘銂所。寘銂造台浤謀逆謠語,使寺人誘鼒櫍吟誦,圖陷台浤自立。懷王妃王氏奏寘銂裁減衣食,至不能自存。豐林王台瀚亦欲陷寘銂,遂發其瀆亂人倫諸罪。驗實,廢為庶人,幽高牆。廷議謂台浤父子乖離,從台浤西安,而封鼒櫍世子,視府事,十一年十月也。十五年以兩宮徽號恩復台浤冠帶,薨。

鼒櫍先卒,弟惠王鼒枋嗣。好學樂善,以禮飭諸宗。世宗賜之敕,建坊表之。寧夏築邊牆,鼒枋出銀米佐工。萬曆二年薨。子端王倪貴嗣,十六年薨。子憲王伸域嗣,十九年薨。明年,寧夏賊哱拜反,王妃方氏匿其子帥鋅地窖中,自經死。時壽陽嗣王倪動,哱拜脅降之,不屈,為所囚。鎮原王伸塇理府事,謀襲賊弗克,府中人皆被殺。賊平,御史劉芳譽言:「諸宗死節者俱應恤錄,方妃宜建祠旌表。」詔從之,給銀一萬五千兩,分振諸宗人。帥鋅嗣,薨。子倬紘嗣。崇禎十六年,流賊破寧夏,被執。

安塞王秩炅,靖王季子也,十二而孤,母位氏誨之。性通敏,過目不忘,善古文。遇縉紳學士,質難辨惑,移日不倦。所著有《隨筆》二十卷。

庶人寘鐇,祖秩炵,靖王第四子也。封安化王。父邃墁,鎮國將軍,以寘鐇襲王爵。性狂誕,相者言其當大貴,巫王九兒教鸚鵡妄言禍福,寘鐇遂覬望非分。寧夏指揮周昂,千戶何錦、丁廣,衛學諸生孫景文、孟彬、史連輩,皆往來寘鐇所。正德五年,帝遣大理少卿周東度寧夏屯田。東希劉瑾意,以五十畝為一頃,又畝斂銀為瑾賄,敲撲慘酷,戍將衛卒皆憤怨。而巡撫都御史安惟學數杖辱將士妻,將士銜刺骨。寘鐇知眾怒,令景文飲諸武臣酒,以言激之,諸武臣多願從寘鐇者。又令人結平虜城戍將及素所厚張欽等。會有邊警,參將仇鉞、副總兵楊英帥兵出防禦。總兵官姜漢簡銳卒六十人為牙兵,令周昂領之,遂與何錦定約。四月五日,寘鐇設宴,邀撫、鎮官飲於第,惟學、東不至。錦、昂帥牙兵直入,殺姜漢及太監李增、鄧廣於坐,分遣卒殺惟學、東及都指揮楊忠於公署。遂焚官府,釋囚繫,撤黃河渡船於西岸以絕渡者。遣人招楊英、仇鉞。皆佯許之。英率眾保王宏堡,眾潰,英奔靈州。鉞引還,寘鐇奪其軍,出金帛犒將士。偽署何錦大將軍,周昂、丁廣副將軍,張欽先鋒,魏鎮、楊泰等總兵都護。令孫景文作檄,以討劉瑾為名。

陜西總兵官曹雄聞變,遣指揮黃正駐靈州,檄楊英督靈州兵防黃河。都指揮韓斌、總兵官侯勳、參將時源各以兵會。英密使蒼頭報仇鉞為內應,令史墉浮渡奪西崖船,營河東,焚大、小二壩草。寘鐇懼,令錦等出禦,獨留昂守城,使使召鉞。鉞稱病,昂來問疾,鉞刺昂死。令親兵馳寘鐇第,擊殺景文、連等十餘人,遂擒寘鐇,迎英眾入。寘鐇反十有八日而擒。錦、廣、泰、欽先後皆獲,械送伏誅。寘鐇賜死,諸子弟皆論死。有孫鼒材逃出,削髮為僧,居永寧山中。未幾,為土僧所凌,詣官言狀。傳至京,安化宮人左寶瓶在浣衣局,使驗之,吒曰:「此鼒材殿下也。」帝念其自歸,免死,安置鳳陽。寧獻王權,太祖第十七子。洪武二十四年封。踰二年,就籓大寧。大寧在喜峰口外,古會州地,東連遼左,西接宣府,為巨鎮。帶甲八萬,革車六千,所屬朵顏三衛騎兵皆驍勇善戰。權數會諸王出塞,以善謀稱。燕王初起兵,與諸將議曰:「曩餘巡塞上,見大寧諸軍慓悍。吾得大寧,斷遼東,取邊騎助戰,大事濟矣。建文元年,朝議恐權與燕合,使入召權,權不至,坐削三護衛。其年九月,江陰侯吳高攻永平,燕王往救。高退,燕王遂自劉家口間道趨大寧,詭言窮蹙來求救。權邀燕王單騎入城,執手大慟,具言不得已起兵故,求代草表謝罪。居數日,疑洽不為備。北平銳卒伏城外,吏士稍稍入城,陰結三衛部長及諸戍卒。燕王辭去,權祖之郊,伏兵起,擁權行。三衛彍騎及諸戍卒,一呼畢集。守將朱鑒不能禦,戰歿。王府妃妾世子皆隨入松亭關,歸北平,大寧城為空。權入燕軍,時時為燕王草檄。燕王謂權,事成,當中分天下。比即位,王乞改南土。請蘇州,曰:「畿內也。」請錢塘,曰:「皇考以予五弟,竟不果。建文無道,以王其弟,亦不克享。建寧、重慶、荊州、東昌皆善地,惟弟擇焉。」永樂元年二月改封南昌,帝親製詩送之,詔即布政司為邸,瓴甋規制無所更。已而人告權巫蠱誹謗事,密探無驗,得已。自是日韜晦,構精廬一區,鼓琴讀書其間,終成祖世得無患。仁宗時,法禁稍解,乃上書言南昌非其封國。帝答書曰:「南昌,叔父受之皇考已二十餘年,非封國而何?」宣德三年請乞近郭灌城鄉土田。明年又論宗室不應定品級。帝怒,頗有所詰責。權上書謝過。時年已老,有事多齮齕以示威重。權日與文學士相往還,託志翀舉,自號臞仙。嘗奉敕輯《通鑑博論》二卷,又作家訓六篇,《寧國儀範》七十四章,漢唐秘史二卷,《史斷》一卷,《文譜》八卷,《詩譜》一卷,其他註纂數十種。正統十三年薨。

世子盤烒先卒,孫靖王奠培嗣。奠培善文辭,而性卞急,多嫌猜。景泰七年,弟弋陽王奠壏訐其反逆,巡撫韓雍以聞。帝遣官往讞,不實。時軍民連逮者六七百人,會英宗復辟,俱赦釋,惟謫戍其教授游堅。奠培由是憾守土官,不為禮。布政使崔恭積不平,王府事多持不行。奠培遂劾奏恭不法。恭與按察使原傑亦奏奠培私獻、惠二王宮人,逼內官熊璧自盡。按問皆實,遂奪護衛。踰三年,而奠壏以有罪賜死。初,錦衣衛指揮逯杲聽詗事者言,誣奠壏烝母。帝令奠培具實以聞,復遣駙馬都尉薛桓與杲按問。奠培奏無是事,杲按亦無實。帝怒,責問杲。杲懼,仍以為實,遂賜奠壏母子自盡,焚其屍。是日雷雨大作,平地水深數尺,眾咸冤之。

弘治四年,奠培薨。子康王覲鈞嗣,十年薨。子上高王宸濠嗣。其母,故娼也。始生,靖王夢蛇啖其室,旦日鴟鳴,惡之。及長,輕佻無威儀,而善以文行自飾。術士李自然、李日芳妄言其有異表,又謂城東南有天子氣。宸濠喜,時時詗中朝事,聞謗言輒喜。或言帝明聖,朝廷治,即怒。武宗末年無子,群臣數請召宗室子子之。宸濠屬疏,顧深結左右,於帝前稱其賢。初,宸濠賄劉瑾,復所奪護衛。瑾誅,仍論奪。及陸完為兵部尚書,宸濠結嬖人錢寧、臧賢為內主,欲奏復,大學士費宏執不可。諸嬖人乘宏讀廷試卷,取中旨行之。宸濠益恣,擅殺都指揮戴宣,逐布政使鄭岳、御史范輅,幽知府鄭獻、宋以方。盡奪諸附王府民廬,責民間子錢,強奪田宅子女,養群盜,劫財江、湖間,有司不敢問。日與致仕都御史李士實、舉人劉養正等謀不軌。副使胡世寧請朝廷早裁抑之。宸濠連奏世寧罪,世寧坐謫戍,自是無敢言者。

正德十二年,典儀閻順,內官陳宣、劉良間行詣闕上變。寧、賢等庇之,不問。宸濠疑出承奉周儀,殺儀家及典仗查武等數百人。巡撫都御史孫燧列其事,中道為所邀,不得達。宸濠又賄錢寧,求取中旨,召其子司香太廟。寧言於帝,用異色龍箋,加金報賜。異色龍箋者,故事所賜監國書箋也。宸濠大喜,列仗受賀。復勒諸生、父老奏闕下,稱其孝且勤。時邊將江彬新得幸,太監張忠附彬,欲傾寧、賢,乘間為帝言:「寧、賢盛稱寧王,陛下以為何如?」帝曰:「薦文武百執事,可任使也。薦籓王何為者?」忠曰:「賢稱寧王孝,譏陛下不孝耳。稱寧王勤,譏陛下不勤耳。」帝曰:「然。」下詔逐王府人,毋留闕下。是時宸濠與士實、養正日夜謀,益遣姦人盧孔章等分布水陸孔道,萬里傳報,浹旬往返,蹤跡大露,朝野皆知其必反。巡撫都御史孫燧七上章言之,皆為所邀沮。諸權姦多得宸濠金錢,匿其事不以聞。

十四年,御史蕭淮疏言宸濠諸罪,謂不早制,將來之患有不可勝言者。疏下內閣,大學士楊廷和謂宜如宣宗處趙府事,遣勳戚大臣宣諭,令王自新。帝命駙馬都尉崔元、都御史顏頤壽、太監賴義持諭往,收其護衛,令還所奪官民田。宸濠聞元等且至,乃定計,以己生辰日宴諸守土官。詰旦皆入謝。宸濠命甲士環之,稱奉太后密旨,令起兵入朝。孫燧及副使許逵不從,縛出斬之。執御史王金,主事馬思聰、金山,參議黃宏、許傚廉,布政使胡廉,參政陳杲、劉棐,僉事賴鳳,指揮許金、白昂等下獄。參政王綸、季斅,僉事潘鵬、師夔,布政使梁宸,按察使楊璋,副使唐錦皆從逆。以李士實、劉養正為左、右丞相,王綸為兵部尚書,集兵號十萬。命其承奉涂欽與素所蓄群盜閔念四等,略九江、南康,破之。馳檄指斥朝廷。七月壬辰朔,宸濠出江西,留其黨宜春王拱條、內官萬銳等守城,自帥舟師蔽江下,攻安慶。

汀贛巡撫僉都御史王守仁聞變,與吉安知府伍文定等檄諸郡兵先後至。乃使奉新知縣劉守緒破其墳廠伏兵。戊申,直攻南昌。辛亥,城破,拱條、銳等皆就擒,宮人自焚死。宸濠方攻安慶不克,聞南昌破,大恐,解圍還,守仁逆擊之。乙卯,遇於黃家渡,賊兵乘風進薄,氣驕甚。文定及指揮餘恩佯北,誘賊趨利,前後不相及。知府邢珣、徐璉、戴德孺從後急擊,文定還兵乘之,賊潰,斬溺萬計。又別遣知府陳槐、林瑊、曾璵、周朝佐復九江、南康。明日,復戰,官兵稍卻,文定帥士卒殊死鬥,擒斬二千餘級,宸濠乃退保樵舍。明日,官軍以火攻之,宸濠大敗。諸妃嬪皆赴水死,將士焚溺死者三萬餘人。宸濠及其世子、郡王、儀賓並李士實、劉養正、塗欽、王綸等俱就擒。宸濠自舉事至敗,蓋四十有三日。

時帝聞宸濠反,下詔暴其罪,告宗廟,廢為庶人。逮繫尚書陸完,嬖人錢寧、臧賢等,籍其家。江彬、張忠從臾帝親征,至良鄉,守仁捷奏至,檄止之。守仁已械繫宸濠等,取道浙江。帝留南京,遣許泰、朱暉及內臣張永、張忠搜捕江西餘黨,民不勝其擾。檄守仁還江西。守仁至杭州,遇張永,以俘付之,使送行在。十五年十二月,帝受所獻俘回鑾,至通州誅之,封除。初,宸濠謀逆,其妃婁氏嘗諫。及敗,歎曰:「昔紂用婦言亡,我以不用婦言亡,悔何及!」嘉靖四年,弋陽王拱樻等言:「獻王、惠王四服子孫所共祀,非宸濠一人所自出,如臣等皆得甄別,守職業如故,而二王不獲廟享,臣竊痛之。」疏三上,帝命弋陽王以郡王奉祀,樂舞齋郎之屬半給之。寧籓既廢,諸郡王勢頡頏,莫能一,帝命拱貴攝府事。卒,樂安王拱欏攝。拱欏奏以建安、樂安、弋陽三王分治八支,著為令。

石城王奠堵,惠王第四子。性莊毅,家法甚嚴。靖王奠培與諸郡王交惡,臨川、弋陽皆被構得罪,奠堵獨謹約,不能坐以過失。子覲鎬,孝友有令譽,早卒。孫宸浮嗣,與母弟宸浦、庶兄宸潣、弟宸澅皆淫縱殺人。弘治十二年互訐奏,宸浮、宸浦並革為庶人,宸澅、宸潣奪祿。宸澅遂從宸濠反,雷震死。嘉靖二十四年,復宸浮、宸浦冠帶,宸潣子拱梃上書為父澡雪,亦還爵。

宸澅弟宸浫素方正,宸濠欲屈之,不得,數使人火其居,而諷諸宗資給之以示惠,宸浫辭不受。宸濠敗,宸浫得免。子輔國將軍拱概,孫奉國將軍多量,曾孫鎮國中尉謀韋,三世皆端謹自好,而謀韋尤貫串群籍,通曉朝廷典故。諸王子孫好學敦行,自周籓中尉睦坰而外,莫及謀韋者。萬曆二十二年,廷議增設石城、宜春管理,命謀韋以中尉理石城王府事,得劾治不法者。典籓政三十年,宗人咸就約束。暇則閉戶讀書,著《易象通》、《詩故》、《春秋戴記魯論箋》及他書,凡百十有二種,皆手自繕寫。黃汝亨為進賢令,投謁抗禮,劇談久之,逡巡改席。次日,北面稱弟子,人兩稱之。病革,猶與諸子說《易》。子八人,皆賢而好學。從弟謀晉築室龍沙,躬耕賦詩以終。

奉國將軍拱榣,瑞昌王奠墠四世孫也。父宸渠為宸濠累,逮繫中都。兄拱枘請以身代,拱榣佐之,卒得白。嘉靖九年上書請建宗學,令宗室設壇墠,行耕桑禮,謹祀典,加意恤刑,皆得旨俞允。捐田白鹿洞贍學者。其後以議禮稱旨。拱枘上《大禮頌》,並賜敕褒諭。諸子群從多知名者。多煴、多燉以孝友著。多煪、多寧以秉禮嚴重稱。多貴、多煃、多炘以善詞賦名。而多煴與從兄多煡獨杜門卻掃,多購異書,校仇以為樂。萬曆中,督撫薦理瑞昌王府事,謝不起。多煪父拱樛以宸濠事被逮,多煪甫十餘齡,哭走軍門,乞以身代,王守仁見而異之。嘉靖二年疏訟父冤,得釋歸,復爵。時諸郡王統於弋陽,而瑞昌始王不祀。多煪自謂小宗宜典宗祏,請於朝,特敕許焉。乃益祭田,修飭家政,儼若朝典。四子皆莊謹嗜學。

奉國將軍多煌,惠王第五子弋陽王奠壏五世孫也。孝友嗜學。弋陽五傳而絕,宗人舉多煌賢能,敕攝府事,瑞昌諸宗皆屬焉。性廉靜寡欲,淑人熊氏早卒,不再娶,獨處齋閣者二十六年。萬曆四十一年,撫按以行誼聞。詔褒之。會病卒,詔守臣加祭一壇。又多炡者,亦奉國將軍,穎敏善詩歌,嘗變姓名出遊,蹤跡遍吳、楚。晚病羸,猶不廢吟誦。卒,門人私謚曰清敏先生。子謀堚亦有父風。時樂安輔國將軍多靦有詩癖,與謀堚等放志文酒,終其世。

○太祖諸子三

岷王梗谷王橞韓王松瀋王模沁水王珵階清源王幼予安王楹唐王桱三城王芝垝文城王彌鉗彌鋠輔國將軍宇浹郢王棟伊王彞皇子楠靖江王守謙

○興宗諸子

虞王雄英吳王允熥衡王允熞徐王允熙

○惠帝諸子

太子文奎少子文圭

○成祖諸子

高煦趙王高燧高爔

岷莊王楩,太祖第十八子。洪武二十四年封國岷州。二十八年以雲南新附,宜親王鎮撫,改雲南。有司請營宮殿,帝令斬居棕亭,俟民力稍紓後作。建文元年,西平侯沐晟奏其過,廢為庶人,徙漳州。永樂初復王,與晟交惡。帝賜書諭楩,而詔戒晟。楩沉湎廢禮,擅收諸司印信,殺戮吏民。帝怒,奪冊寶。尋念王建文中久幽繫,復予之,而楩不悛。六年,削其護衛,罷官屬。仁宗即位,徙武岡,寄居州治。久之,始建王邸。景泰元年薨。子恭王徽煣嗣。

初,世子徽焲,宣德初,訐其弟鎮南王徽煣誹謗仁廟。宣宗疑其詐,並召至京,及所連閹豎面質,事果誣,斬閹豎而遣徽煣等歸。徽煣嗣位。弟廣通王徽煠有勇力,家人段友洪以技術得寵。致仕後軍都事於利賓言徽煠有異相,當主天下,遂謀亂。作偽敕,分遣友洪及蒙能、陳添行入苗中,誘諸苗以銀印金幣,使發兵攻武岡。苗首楊文伯等不敢受。事覺,友洪為徽煣所執。都御史李實以聞,遣駙馬都尉焦敬、中官李琮徵徽煠入京師。湖廣總管王來、總兵官梁珤復發陽宗王徽焟通謀狀,亦徵入。皆除爵,幽高牆。時景泰二年十月也。

天順七年,徽煣薨。子順王音瀼嗣,病瘋痺,屢年不起。次子安昌王膺鋪侍醫藥,晨夕不去左右。憲宗聞之,賜敕嘉獎。成化十六年,音瀼薨。世子膺金丕居喪,飲博無度,承奉劉忠禁制之,遂殺忠。事聞,驗實,革冠帶停封。居四年,乃嗣。弘治十三年薨,謚曰簡。子靖王彥汰嗣。嘉靖四年,與弟南安王彥泥訐陰事,彥泥廢為庶人,彥汰亦坐抗制擅權革爵。八年令世子譽榮攝府事。譽榮上疏懇辭,謂:「臣坐享尊榮,而父困苦寂寞,臣心何安!且前曾舉臣弟善化王譽桔,廷議以子無制父理,奏寢不行。臣亦人子也,獨不愧臣弟乎!」帝覽疏憐之,下部議。十二年賜彥汰冠帶,理府事。十五年,以兩宮徽號恩復王。又八年始薨。子康王譽榮嗣,三十一年薨。子憲王定耀嗣,三十四年薨。曾孫禋洪,天啟二年嗣,崇禎元年薨。無子,從父企崟嗣。十六年,流賊陷武岡遇害。谷王惠,太祖第十九子。洪武二十四年封。二十八年三月就籓宣府。宣府,上谷地,故曰谷王。燕兵起,橞走還京師。及燕師渡江,橞奉命守金川門,登城望見成祖麾蓋,開門迎降。成祖德之,即位,賜橞樂七奏,衛士三百,賚予甚厚。改封長沙,增歲祿二千石。

橞居國橫甚,忠誠伯茹瑺過長沙不謁橞,橞白之帝,瑺得罪死。遂益驕肆,奪民田,侵公稅,殺無罪人。長史虞廷綱數諫,誣廷綱誹謗,磔殺之。招匿亡命,習兵法戰陣,造戰艦弓弩器械。大創佛寺,度僧千人,為咒詛。日與都指揮張成,宦者吳智、劉信謀,呼成「師尚父」,智、信「國老令公」。偽引讖書,云:「我高皇帝十八子,與讖合。」橞行次十九,以趙王杞早卒,故云。謀於元夕獻燈,選壯士教之音樂,同入禁中,伺隙為變。又致書蜀王為隱語,欲結蜀為援。蜀王貽書切責。不聽。己而蜀王子崇寧王悅燇得罪,逃橞所。惠因詭眾:「往年我開金川門出建文君,今在邸中。我將為申大義,事發有日矣。」蜀王聞之,上變告。

初,護衛都督僉事張興見橞為不法,懼禍及,因奏事北京,白其狀。帝不信。興過南京,復啟於太子,且曰:「乞他日無連坐。」至是,帝歎曰:「朕待橞厚,張興常為朕言,不忍信,今果然。」立命中官持敕諭橞歸悅燇於蜀,且召橞入朝。橞至,帝示以蜀王章,伏地請死。諸大臣廷劾橞曰:「周戮管、蔡,漢辟濞、長,皆大義滅親,陛下縱念橞,奈天下何?」帝曰:「橞,朕弟,朕且令諸兄弟議。」永樂十五年正月,周王橚、楚王楨、蜀王椿等各上議:「橞違祖訓,謀不軌,蹤跡甚著,大逆不道,誅無赦。」帝曰:「諸王群臣奉大義,國法固爾,吾寧生橞?」於是及二子皆廢為庶人,官屬多誅死,興以先發不坐。韓憲王松,太祖第二十子。洪武二十四年封國開原。性英敏,通古今,恭謹無過。永樂五年薨。以未之國,命葬安德門外。十年,子恭王沖或嗣。時棄大寧三衛地,開原逼塞不可居。二十二年改封平涼。仁宗即位,召沖或與弟襄陵王沖秌、樂平王沖烌入朝,各獻詩頌。帝嘉悅,賜金幣有差。宣宗初,請徙江南。不許。請蠲護衛屯租,建邸第。許之。遣主事毛俊經度,並建襄陵、樂平二邸及岷州廣福寺。陜西守臣言歲歉,請輟工。帝令繕王宮,罷建寺役。平涼接邊徼,間諜充斥,沖或習邊鄙利弊,正統元年上書極言邊事。賜書褒答。五年薨。子懷王範圯嗣,九年薨。弟靖王範仰嗣,景泰元年薨。子惠王征釙嗣。初,土木之變,沖秌赴京師勤王,會解嚴。下書慰勞。及成化六年,寇入河套,沖秌復請率子婿擊賊。憲宗止之。沖或兄弟並急王事,以籓禁嚴不用。自是宗臣無預兵事者。

成化五年,徵金卜薨,子悼王偕水充嗣,十年薨。弟康王偕灊嗣,弘治十四年薨。子昭王旭櫏嗣。性忠孝,工詩,居籓有惠政。韓土瘠祿薄,弟建寧王旭肴至,以所受金冊質於宗室偕泆,事聞,廢為庶人。諸貧宗往往凌劫有司,平涼知府吳世良、鄺衍、任守德、王松先後被窘辱。嘉靖十三年,旭櫏薨。子定王融燧嗣,懲宗室之橫,頗繩以法。不逞者怨之。三十二年,襄陵王融焚及諸宗二百餘人訐奏王奸利事。勘無實,革融焚等祿。四十四年,融燧薨。子謨典先卒。世宗末年,以宗祿不足,詔身不及王者,許其嫡長子繼王,餘子如故秩。謨典以世子不及王,王其長子朗錡,餘子止鎮國將軍。萬曆三十四年,朗錡薨,謚曰端。子孫皆早卒,曾孫亶脊嗣。崇禎十六年,賊陷平涼,被執。

襄陵王沖秌,憲王第二子,有至性。母病,刲股和藥,病良已。及卒,終喪毀瘠。每展墓,必率子孫躬畚鍤培冢。先後璽書褒美者六。子範址服其教,母荊罹危疾,亦刲股進之,愈。其後五世同居,門內雍肅。嘉靖十一年賚以羊酒文幣。韓諸王以襄陵家法為第一。王孫征鑖病卒,聘杜氏女,未婚,歸王家,志操甚歷,詔賜旌表。瀋簡王模,太祖第二十一子。洪武二十四年封。永樂六年就籓潞州。宣德六年薨。子康王佶焞嗣。景泰中,數與州官置酒大會,巡撫朱鑑以聞。帝令諸王,非時令壽節,不得輒與有司宴飲,著為令。天順元年薨。子莊王幼學嗣,正德十一年薨。子恭王詮鉦嗣,嘉靖六年薨。孫允榿攝府事,九年卒。無子,再從弟憲王允栘攝府事,凡十年乃嗣封。當是時,瀋府諸郡王勛水育、詮薙並爭襲,帝皆切責之,而令允栘嗣。二十八年薨。子宣王恬烄嗣,好學,工古文詞,審聲律。弟安慶王恬爖、鎮康王恬焯,穆宗時皆以孝義旌。萬曆十年,恬烄薨。子定王珵堯嗣,仁孝恭慎。弟六人,封郡王者二。餘例不得封,朝廷獎王恭,皆封郡王而不與祿。薨,子效金庸嗣,明亡,國除。

沁水王珵堦、簡王七世孫也,工詩喜士,名譽藉甚。前此,有德平王允梃負俊才,與衡府新樂王載璽,周宗人睦挈、俊噤等齊名。

又清源王幼予,康王第三子,博學能文詞。其後,輔國將軍勛漣,從子允杉、允檸、允析,及鎮國將軍恬烷與諸子珵圻等,並以能詩名,時稱瀋籓多才焉。

安惠王楹,太祖第二十二子。洪武二十四年封。永樂六年就籓平涼。十五年薨。無子,封除。府僚及樂戶悉罷,留典仗校尉百人守園。洪熙初,韓恭王改封平涼,就安王邸。英宗令官校隸韓,長史供安王祀,暇日給韓王子襄陵王沖秌使令。景泰五年,沖秌遂乞承安王祀。正德十二年嗣襄陵王征鈐,請樂戶祀安王。明年,樂平王征錏援征鈐例以請。禮部言:「親王有樂戶。郡王別城居者,有事假鼓吹於有司。其附親王國者,假樂戶於長史司。」因並革安王供祀樂戶。嘉靖二年,韓王旭櫏復為代請。帝以安王故,報可之。征鈐卒,韓王融燧令長史革之。征鈐長孫旭童上言:「禮樂自天子出,韓王不宜擅予奪。」融燧亦言:「親王、郡王禮樂宜有降殺。」帝曰:「樂戶為安王祀也。」給如故。

唐定王桱,太祖第二十三子。洪武二十四年封。永樂六年就籓南陽。十三年薨。子靖王瓊烴嗣。綜核有矩矱,為成祖所喜。入朝,五日三召見。宣德元年薨。妃高氏未冊,自經以殉,詔封靖王妃。無子,弟憲王瓊炟嗣,成化十一年薨。子莊王芝址嗣,諸弟三城王芝垝、蕩陰王芝瓦並好學,有令譽。而承休王芝垠,憲王繼妃焦氏子也,妃愛之。遇節旦,召樂婦入宮。芝址詰之,語不遜。焦妃怒,持鐵錘擊宮門,芝址閉不敢出。芝垠與妃弟璟誣王詈繼母。按驗不實,得芝垠慢母詈兄狀,革爵。久之始復。

二十一年,芝址薨。子成王彌鍗嗣。弘治中,疏言:「朝廷待親籓,生爵歿謚,親親至矣。間有惡未敗聞,歿獲美謚,是使善者怠,惡者肆也。自今宜勘實,用寓彰癉。」禮臣請降敕獎諭,勉勵諸王。詔可。武宗喜遊幸,彌鍗作《憂國詩》,且上疏以用賢圖治為言。弟文城王彌鉗有學行,孝友篤至。嘉靖二年,彌鍗薨。無子,彌鉗子敬王宇溫嗣。二十一年,獻金助太廟工,賜玉帶,益祿二百石。時承休王芝垠子彌鋠以父與莊王交訐,失令名,折節蓋前愆。宇溫上其事。璽書褒獎。三十九年,宇溫薨。子順王宙栐嗣,四十三年薨。子端王碩熿嗣。惑於嬖人,囚世子器墭及其子聿鍵於承奉司,器墭中毒死。

崇禎五年,碩熿薨,聿鍵嗣。七年,流賊大熾,蠲金築南陽城,又援潞籓例,乞增兵三千人。不許。九年秋八月,京師戒嚴,倡義勤王。詔切責,勒還國。事定,下部議,廢為庶人,幽之鳳陽。弟聿鏌嗣。十四年,李自成陷南陽,聿鏌遇害。十七年,京師陷,福王由崧立於南京,乃赦聿鍵出。大清順治二年五月,南都降。聿鍵行至杭,遇鎮江總兵官鄭鴻逵、戶部郎中蘇觀生,遂奉入閩。南安伯鄭芝龍、巡撫都御史張肯堂與禮部尚書黃道周等定議,奉王稱監國。閏六月丁未,遂立於福州,號隆武,改福州為天興府。進芝龍、鴻逵為侯,封鄭芝豹、鄭彩為伯,觀生、道周俱大學士,肯堂為兵部尚書,餘拜官有差。

聿鍵好學,通典故,然權在鄭氏,不能有所為。是年八月,芝龍議簡戰守兵二十餘萬,計餉不支其半。請預借兩稅一年,令群下捐俸,勸紳士輸助,徵府縣銀穀未解者。官吏督迫,閭里騷然。又廣開事例,猶苦不足。仙霞嶺守關兵僅數百人,皆不堪用。聿鍵屢促芝龍出兵,輒以餉詘辭。久之,芝龍知眾論不平,乃請以鴻逵出浙東,彩出江西,各擁兵數千,號數萬。既行,託候餉,皆行百里而還。先是,黃道周知芝龍無意出師,自請行,從廣信趨婺源,兵潰死,事詳《道周傳》。

是時,李自成兵敗,走死通山。其兄子李錦帥眾降於湖廣總督何騰蛟,一時增兵十餘萬。侍郎楊廷麟、祭酒劉同升起兵復吉安、臨江。於是廷麟等請聿鍵出江右,騰蛟請出湖南。原任知州金堡言騰蛟可恃,芝龍不可恃,宜棄閩就楚。聿鍵大喜,授堡給事中,遣觀生先行募兵。

先是,靖江王亨嘉僭稱監國,不奉聿鍵命,為巡撫瞿式耜等所擒,以捷聞。而魯王以海又稱監國於紹興,拒聿鍵使者,故聿鍵決意出江西、湖廣。十二月發福州,駐建寧。廣東布政湯來賀運餉十萬,由海道至。明年二月駐延平。三月,大清兵取吉安、撫州,圍楊廷麟於贛州。尚書郭維經出閩,募兵援贛。六月,大兵克紹興,魯王以海遁入海,閩中大震。芝龍假言海寇至,徹兵回安平鎮,航海去。守關將士皆隨之,仙霞嶺空無一人。七月,何騰蛟遣使迎聿鍵,將至韶州。唯時我兵已抵閩關,守浦城御史鄭為虹、給事中黃大鵬、延平知府王士和死焉。八月,聿鍵出走,數日方至汀州。大兵奄至,從官奔散,與妃曾氏俱被執。妃至九瀧投於水,聿鍵死於福州。給事中熊緯、尚書曹學佺、通政使馬思禮等自縊死。郢靖王棟,太祖第二十四子。洪武二十四年封。永樂六年之籓安陸。十二年薨。無子封除。留內外官校守園。王妃郭氏,武定侯英女。王薨踰月,妃慟哭曰:「未亡人無子,尚誰恃?」引鏡寫容付宮人,曰:「俟諸女長,令識母。」遂自經。妃四女,一夭,其三女封光化、穀城、南漳郡主,歲祿各八百石。宣德四年,以郢故邸封梁王瞻自,移郢宮人居南京。伊歷王鷖,太祖第二十五子。洪武二十一年生,生四年封。永樂六年之籓洛陽,歲祿僅二千石。王好武,不樂居宮中,時時挾彈露劍,馳逐郊外。奔避不及者,手擊之。髡裸男女以為笑樂。十二年薨。禮臣請追削封爵,不許。

二十二年,子簡王顒炴始得嗣。縱中官擾民,洛陽人苦之。河南知府李驥稍持以法。誣奏,驥被逮治。己而得白,罪王左右。英宗時上表,文不恭,屢被譙讓。天順六年薨。世孫悼王諟釩嗣,成化十一年薨。弟定王諟鋝嗣,好學崇禮,居喪哀毀,歲時祀先,致齋於外。郡王、諸將軍、中尉非慶賀不褻見。民間高年者,禮下之。正德三年薨。子莊王訏淵嗣,嘉靖五年薨。弟敬王訏淳嗣,居母喪,以孝聞。以祿薄上言:「先朝以河南課鈔萬七千七百貫,准祿米八千石。八年革諸王請乞租稅,伊府課鈔亦在革中,乞補祿。」戶部言:「課鈔本成、弘間請乞,非永樂時欽賜比。河南一省缺祿者八十餘萬,宜不許。」帝從部議。二十一年薨。

世子典楧嗣,貪而愎,多持官吏短長。不如指,必構之去,既去復折辱之。御史行部過北邙山外,典楧要笞之。縉紳往來,率紆途取他境。經郭外者,府中人輒追挽其車,詈其不朝,入朝者復辱以非禮。府牆壞,請更築,乃奪取民舍以廣其宮。郎中陳大壯與邸鄰,索其居不與,使數十人從大壯臥起,奪其飲食,竟至餒死。所為宮,崇臺連城,擬帝闕。有錦衣官校之陜者,經洛陽,典楧忽召官屬迎詔,鼓吹擁錦衣入,捧一黃卷入宮。眾請開讀,曰:「密詔也。」遂趣錦衣去。錦衣謂王厚待之,不知所以。其夜大張樂,至曙,府中皆呼千歲,詐謂「天子特親我也」。閉河南府城,大選民間子女七百餘,留其姝麗者九十人。不中選者,令以金贖。都御史張永明、御史林潤、給事中丘岳相繼言其罪狀。再遣使往勘,革祿三之二,令壞所僭造宮城,歸民間女,執群小付有司。典楧不奉詔。部牒促之,布政使持牒入見。典楧曰:「牒何為者,可用障欞耳!」四十三年二月,撫按官以聞。詔禮部會三法司議。僉謂:「典楧淫暴,無籓臣禮,陛下曲赦再四,終不湔改,奸回日甚。宜如徽王載龠故事,禁錮高牆,削除世封。」詔從其議,與子褒節俱安置開封。皇子楠,太祖第二十六子。洪武二十六年生,踰月殤。靖江王守謙,太祖從孫。父文正,南昌王子也。當太祖起兵時,南昌王前死,妻王氏攜文正依太祖。太祖、高后撫如己子。比長,涉獵傳記,饒勇略,隨渡江取集慶路。已,有功,授樞密院同僉。太祖從容問:「若欲何官?」文正對曰:「叔父成大業,何患不富貴。爵賞先私親,何以服眾!」太祖喜其言,益愛之。

太祖為吳王,命為大都督,節制中外諸軍事。及再定江西,以洪都重鎮,屏翰西南,非骨肉重臣莫能守。乃命文正統元帥趙得勝等鎮其地,儒士郭之章、劉仲服為參謀。文正增城浚池,招集山寨未附者,號令明肅,遠近震懾。居無何,友諒帥舟師六十萬圍洪都。文正數摧其鋒,堅守八十有五日,城壞復完者數十丈。友諒旁掠吉安、臨江,俘其守將徇城下,不為動。太祖親帥兵來援,友諒乃解去,與太祖相拒於彭蠡。友諒掠糧都昌,文正遣方亮焚其舟。糧道絕,友諒遂敗。復遣何文輝等討平未附州縣。江西之平,文正功居多。

太祖還京,告廟飲至,賜常遇春、廖永忠及諸將士金帛甚厚。念文正前言知大體,錫功尚有待也,而文正不能無少望。性素卡急,至是暴怒,遂失常度,任掾吏衛可達奪部中子女。按察使李飲冰奏其驕侈觖望,太祖遣使詰責。文正懼,飲冰益言其有異志。太祖即日登舟至城下,遣人召之。文天上倉卒出迎,太祖數曰:「汝何為者?」遂載與俱歸,欲竟其事。高后力解之曰:「兒特性剛耳,無他也。」免官安置桐城,未幾卒。飲冰亦以他事伏誅。

文正之被謫也,守謙甫四歲,太祖撫其頂曰:「兒無恐,爾父倍訓教,貽我憂,我終不以爾父故廢爾。」育之宮中。守謙幼名鐵柱,吳元年以諸子命名告廟,更名煒。洪武三年更名守謙,封靖江王。祿視郡王,官屬親王之半,命耆儒趙熏為長史傅之。既長,之籓桂林。桂林有元順帝潛邸,改為王宮,上表謝。太祖敕其從臣曰:「從孫幼而遠鎮西南,其善導之。」守謙知書,而好比群小,粵人怨咨。召還,戒諭之。守謙作詩怨望。帝怒,廢為庶人。居鳳陽七年,復其爵。徙鎮雲南,使其妃弟徐溥同往,賜書戒飭,語極摯切。守謙暴橫如故。召還,使再居鳳陽。復以強取牧馬,錮之京師。二十五年卒。子贊儀幼,命為世子。

三十年春遣省晉、燕、周、楚、齊、蜀、湘、代、肅、遼、慶、谷、秦十三王,自湘、楚入蜀,歷陜西,抵河南、山西、北平,東至大寧、遼陽,還自山東,使知親親之義,熟山川險易,習勞苦。贊儀恭慎好學。永樂元年復之國桂林,使蕭用道為長史。用道善輔導,贊儀亦敬禮之。六年薨,謚曰悼僖。

子莊簡王佐敬嗣。初給銀印,宣德中,改用金塗。正統初,與其弟奉國將軍佐敏相訐奏,語連大學士楊榮。帝怒,戍其使人。成化五年薨。子相承先卒,孫昭和王規裕嗣,弘治二年薨。子端懿王約麒嗣,以孝謹聞。正德十一年薨。子安肅王經扶嗣,好學有儉德,嘗為《敬義箴》。嘉靖四年薨。子恭惠王邦苧嗣,與巡按御史徐南金相訐奏。奪祿米,罪其官校。隆慶六年薨。子康僖王任昌嗣,萬歷十年薨。子溫裕王履燾嗣,二十年薨。無子,從父憲定王任晟嗣,三十八年薨。子榮穆王履祜嗣,薨。子亨嘉嗣。李自成陷京師後,自稱監國於廣西,為巡撫瞿式耜所誅。時唐王聿鍵在福建,奏捷焉。

興宗五子。后常氏生虞懷王雄英、吳王允熥。呂后生惠帝、衡王允熞、徐王允熙。

虞懷王雄英,興宗長子,太祖嫡長孫也。洪武十五年五月薨。年八歲。追加封謚。

吳王允熥,興宗第三子。建文元年封國杭州,未之籓。成祖即位,降為廣澤王,居漳州。未幾,召還京,廢為庶人,錮鳳陽。永樂十五年卒。

衡王允熞,興宗第四子,建文元年封。成祖降為懷恩王,居建昌。與允通俱召還,錮鳳陽,先後卒。

徐王允熙,興宗第五子,建文元年封。成祖降為敷惠王,隨母呂太后居懿文陵。永樂二年下詔改甌寧王,奉太子祀。四年十二月,邸中火,暴薨。謚曰哀簡。

惠帝二子。俱馬后生。

太子文奎。建文元年立為皇太子。燕師入,七歲矣,莫知所終。

少子文圭。年二歲,成祖入,幽之中都廣安宮,號為建庶人。英宗復辟,憐庶人無罪久繫,欲釋之,左右或以為不可。帝曰:「有天命者,任自為之。」大學士李賢贊曰:「此堯、舜之心也。」遂請於太后,命內臣牛玉往出之。聽居鳳陽,婚娶出入使自便。與閽者二十人,婢妾十餘人,給使令。文圭孩提被幽,至是年五十七矣。未幾卒。

成祖四子。仁宗、漢王高煦、趙王高燧俱文皇后生。高爔未詳所生母。

漢王高煦,成祖第二子。性凶悍。洪武時,召諸王子學於京師。高煦不肯學,言動輕佻,為太祖所惡。及太祖崩,成祖遣仁宗及高煦入臨京師。舅徐輝祖以其無賴,密戒之。不聽,盜輝祖善馬,徑渡江馳歸。途中輒殺民吏,至涿州,又擊殺驛丞,於是朝臣舉以責燕。成祖起兵,仁宗居守,高煦從,嘗為軍鋒。白溝河之戰,成祖幾為瞿能所及,高煦帥精騎數千,直前決戰,斬能父子於陣。及成祖東昌之敗,張玉戰死,成祖隻身走,適高煦引師至,擊退南軍。徐輝祖敗燕兵於浦子口,高煦引蕃騎來。成祖大喜,曰:「吾力疲矣,兒當鼓勇再戰。」高煦麾蕃騎力戰,南軍遂卻。成祖屢瀕於危而轉敗為功者,高煦力為多。成祖以為類己,高煦亦以此自負,恃功驕恣,多不法。

成祖即位,命將兵往開平備邊。時議建儲,淇國公丘福、駙馬王寧善高煦,時時稱高煦功高,幾奪嫡。成祖卒以元子仁賢,且太祖所立,而高煦又多過失,不果。永樂二年,仁宗立為太子,封高煦漢王,國雲南。高煦曰:「我何罪!斥萬里。」不肯行。從成祖巡北京,力請並其子歸南京。成祖不得已,聽之。請得天策衛為護衛,輒以唐太宗自比。己,復乘間請益兩護衛,所為益恣。成祖嘗命同仁宗謁孝陵。仁宗體肥重,且足疾,兩中使掖之行,恒失足。高煦從後言曰:「前人蹉跌,後人知警。」時宣宗為皇太孫,在後應聲曰:「更有後人知警也。」高煦回顧失色。高煦長七尺餘,輕趫善騎射,兩腋若龍鱗者數片。既負其雄武,又每從北征,在成祖左右,時媒孽東宮事,譖解縉至死,黃淮等皆繫獄。

十三年五月改封青州,又不欲行。成祖始疑之,賜敕曰:「既受籓封,豈可常居京邸!前以雲南遠憚行,今封青州,又託故欲留侍,前後殆非實意,茲命更不可辭。」然高煦遷延自如。私選各衛健士,又募兵三千人,不隸籍兵部,縱使劫掠。兵馬指揮徐野驢擒治之。高煦怒,手鐵瓜撾殺野驢,眾莫敢言。遂僭用乘輿器物。成祖聞之怒。十四年十月還南京,盡得其不法數十事,切責之,褫冠服,囚繫西華門內,將廢為庶人。仁宗涕泣力救,乃削兩護衛,誅其左右狎匿諸人。明年三月徙封樂安州,趣即日行。高煦至樂安,怨望,異謀益急。仁宗數以書戒,不悛。

成祖北征晏駕。高煦子瞻圻在北京,覘朝廷事馳報,一晝夜六七行。高煦亦日遣人潛伺京師,幸有變。仁宗知之,顧益厚遇。遺書召至,增歲祿,賜賚萬計,仍命歸籓。封其長子為世子,餘皆郡王。先是,瞻圻怨父殺其母,屢發父過惡。成祖曰:「爾父子何忍也!」至是高煦入朝,悉上瞻圻前後覘報中朝事。仁宗召示瞻圻曰:「汝處父子兄弟間,讒構至此,稚子不足誅。」遣守鳳陽皇陵。

未幾,仁宗崩,宣宗自南京奔喪。高煦謀伏兵邀於路,倉卒不果。及帝即位,賜高煦及趙王視他府特厚。高煦日有請,並陳利國安民四事。帝命有司施行,仍復書謝之。因語群臣曰:「皇祖嘗諭皇考,謂叔有異志,宜備之。然皇考待之極厚。如今所言,果出於誠,則是舊心已革,可不順從。」凡有求請,皆曲徇其意。高煦益自肆。

宣德元年八月,遂反。遣其親信枚青等潛至京師,約舊功臣為內應。英國公張輔執之以聞。時高煦已約山東都指揮靳榮等,又散弓刀旂幟於衛所,盡奪傍郡縣畜馬。立五軍:指揮王斌領前軍,韋達左軍,千戶盛堅右軍,知州朱恒後軍,諸子各監一軍,高煦自將中軍。世子瞻坦居守,指揮韋弘、韋興,千戶王玉、李智領四哨。部署已定,偽授王斌、朱恒等太師、都督、尚書等官。御史李濬以父喪家居,高煦招之,不從,變姓名,間道詣京師上變。帝猶不忍加兵,遣中官侯泰賜高煦書。泰至,高煦盛兵見泰,南面坐,大言曰:「永樂中信讒,削我護衛,徙我樂安。仁宗徒以金帛餌我,我豈能鬱鬱居此!汝歸報,急縛奸臣夏原吉等來,徐議我所欲。」泰懼,唯唯而已。比還,帝問漢王何言,治兵何如,泰皆不敢以實對。

是月,高煦遣百戶陳剛進疏,更為書與公侯大臣,多所指斥。帝歎曰:「漢王果反。」乃議遣陽武侯薛祿將兵往討。大學士楊榮等勸帝親征。帝是之。張輔奏曰:「高煦素懦,願假臣兵二萬,擒獻闕下。」帝曰:「卿誠足擒賊,顧朕初即位,小人或懷二心,不親行,不足安反側。」於是車駕發京師,過楊村,馬上顧從臣曰:「度高煦計安出?」或對曰:「必先取濟南為巢窟。」或對曰:「彼曩不肯離南京,今必引兵南下。」帝曰:「不然。濟南雖近,未易攻,聞大軍至,亦不暇攻。護衛軍家樂安,必內顧,不肯徑趨南京。高煦外誇詐,內實怯,臨事狐疑不能斷。今敢反者,輕朕年少新立,眾心未附,不能親征耳。今聞朕行,已膽落,敢出戰乎?至即擒矣。」高煦初聞祿等將兵,攘臂大喜,以為易與。及聞親征,始懼。時有從樂安來歸者,帝厚賞之,令還諭其眾。仍遺書高煦曰:「張敖失國,始於貫高;淮南被誅,成於伍被。今六師壓境,王即出倡謀者,朕與王除過,恩禮如初。不然,一戰成擒,或以王為奇貨,縛以來獻,悔無及矣。」前鋒至樂安,高煦約詰旦出戰。帝令大軍蓐食兼行,駐蹕樂安城北,壁其四門。賊乘城守,王師發神機銃箭,聲震如雷。諸將請即攻城。帝不許。再敕諭高煦,皆不答。城中人多欲執獻高煦者,高煦大懼。乃密遣人詣行幄,願假今夕訣妻子,即出歸罪。帝許之。是夜,高煦盡焚兵器及通逆謀書。明日,帝移蹕樂安城南。高煦將出城,王斌等力止曰:「寧一戰死,無為人擒。」高煦紿斌等復入宮,遂潛從間道出見帝。群臣請正典刑。不允。以劾章示之,高煦頓首言:「臣罪萬萬死,惟陛下命。」帝令高煦為書召諸子,餘黨悉就擒。赦城中罪,脅從者不問。命薛祿及尚書張本鎮撫樂安,改曰武定州,遂班師。廢高煦父子為庶人,築室西安門內錮之。王斌等皆伏誅,惟長史李默以嘗諫免死,謫口北為民。天津、青州、滄州、山西諸都督指揮約舉城應者,事覺相繼誅,凡六百四十餘人,其故縱與藏匿坐死戍邊者一千五百餘人,編邊氓者七百二十人。帝製《東征記》以示群臣。高煦及諸子相繼皆死。

趙簡王高燧,成祖第三子。永樂二年封。尋命居北京,詔有司,政務皆啟王後行。歲時朝京師,辭歸,太子輒送之江東驛。高燧恃寵,多行不法,又與漢王高煦謀奪嫡,時時譖太子。於是太子宮寮多得罪。七年,帝聞其不法事,大怒,誅其長史顧晟,褫高燧冠服,以太子力解,得免。擇國子司業趙亨道、董子莊為長史輔導之,高燧稍改行。

二十一年五月,帝不豫。護衛指揮孟賢等結欽天監官王射成及內侍楊慶養子造偽詔,謀進毒於帝,俟晏駕,詔從中下,廢太子,立趙王。總旂王瑜姻家高以正者,為賢等畫謀,謀定告瑜。瑜上變。帝曰:「豈應有此!」立捕賢,得為偽詔。賢等皆伏誅,陛瑜遼海衛千戶。帝顧高燧曰:「爾為之耶?」高燧大懼,不能言。太子力為之解曰:「此下人所為,高燧必不與知。」自是益斂戢。

仁宗即位,加漢、趙二王歲祿二萬石。明年,之國彰德,辭常山左右二護衛。宣宗即位,賜田園八十頃。帝擒高煦歸,至單橋,尚書陳山迎駕,言曰:「趙王與高煦共謀逆久矣,宜移兵彰德,擒趙王。否則趙王反側不自安,異日復勞聖慮。」帝未決。時惟楊士奇以為不可。山復邀尚書蹇義、夏原吉共請。帝曰:「先帝友愛二叔甚。漢王自絕於天,朕不敢赦。趙王反形未著,朕不忍負先帝也。」及高煦至京,亦言嘗遣人與趙通謀。戶部主事李儀請削其護衛,尚書張本亦以為言。帝不聽。既而言者益眾。明年,帝以其詞及群臣章遣駙馬都尉廣平侯袁容持示高燧。高燧大懼,乃請還常山中護衛及群牧所、儀衛司官校。帝命收其所還護衛,而與儀衛司。宣德六年薨。

子惠王蟾塙嗣,景泰五年薨。子悼王祈鎡嗣,天順四年薨。子靖王見水爵嗣。惠王、悼王皆頗有過失,至見灂惡尤甚,屢賊殺人,又嘗乘醉欲殺其叔父。成化十二年,事聞,詔奪祿米三之二,去冠服,戴民巾,讀書習禮。其後二年,見灂母妃李氏為之請,得冠服如故。見灂卒不能改。愛幼子祐枳,遂誣長子祐棌以大逆,復被詔誚讓。弘治十五年薨。子莊王祐棌嗣,正德十三年薨。

子康王厚煜嗣,事祖母楊妃以孝聞。嘉靖七年六月,璽書褒予。明年冬,境內大饑。厚煜上疏,請辭祿一千石以佐振。帝嘉王憂國,詔有司發粟,不允所辭。及帝南巡,厚煜遠出迎,命益祿三百石。厚煜性和厚,構一樓名「思訓」,嘗獨居讀書,文藻贍麗。宗人輔國將軍祐椋等數犯法,與有司為難。厚煜庇祐椋。祐椋卒得罪,並見責讓。其後有司益務以事裁抑諸宗。洛川王翊鏴奴與通判田時雨之隸爭瓜而毆,時雨捕王奴。厚煜請解不得,竟論奴充軍。未幾,宗室數十人索祿,時雨以宗室毆府官,白於上官。知府傅汝礪盡捕各府人。厚煜由是忿恚,竟自縊死。三十九年十月也。厚煜子成皋王載垸疏聞於朝,下法司按問。時雨斬河南市,汝礪戍極邊。厚煜子載培及載培子翊錙皆前卒。翊錙子穆王常清嗣,以善行見旌。萬歷四十二年薨。世子由松前卒,弟壽光王由桂子慈夬嗣,薨。無子,穆王弟常水臾嗣。崇禎十七年,彰德陷,被執。

○仁宗諸子

鄭王瞻颭廬江王載堙越王瞻墉蘄王瞻垠襄王瞻墡棗陽王祐楒荊王瞻堈淮王瞻墺滕王瞻塏梁王瞻自衛王瞻埏

○英宗諸子

德王見潾許王見淳秀王見澍崇王見澤吉王見浚忻王見治徽王見沛

○景帝子

懷獻太子見濟

○憲宗諸子

悼恭太子祐極岐王祐棆益王祐檳衡王祐楎新樂王載璽雍王祐枟壽王祐楮汝王祐梈涇王祐橓榮王祐樞申王祐楷

○孝宗子

蔚王厚煒

仁宗十子。昭皇后生宣宗、越王瞻墉、襄王瞻墡。李賢妃生鄭王瞻颭、蘄王瞻垠、淮王瞻墺。張順妃生荊王蟾堈。郭貴妃生滕王瞻塏、梁王瞻自、衛王瞻埏。

鄭靖王瞻颭,仁宗第二子。永樂二十二年十月封。仁宗崩,皇后命與襄王監國,以待宣宗,宣德元年,帝征樂安,仍命與襄王居守。四年就籓鳳翔。正統八年詔遷懷慶,留京邸,明年之國。蟾颭暴歷,數斃人杖下。英宗以御史周瑛為長史,稍戢。成化二年薨。子簡王祁鍈嗣。祁鍈之為世子也,襄王朝京師,經新鄉,祁鍈不請命,遣長史往迎。英宗聞之不悅,賜書責讓。及嗣王,多不法,又待世子寡恩。長史江萬程諫,被責辱,萬程以聞。帝遣英國公張懋、太監王允中齎敕往諭,始上書謝罪。弘治八年薨。世子見滋母韓妃不為祁鍈所禮,見滋悒悒先卒。子康王祐枔嗣,正德二年薨。無子,從弟懿王祐BT嗣,十六年薨。子恭王厚烷嗣。

世宗幸承天,厚烷迎謁於新鄉,加祿三百石。疏奏母閻太妃貞孝事蹟。詔付史館。其後帝修齋醮,諸王爭遣使進香,厚烷獨不遣。嘉靖二十七年七月上書,請帝修德講學,進《居敬》、《窮理》、《克己》、《存誠》四箴,《演連珠》十章,以神仙、土木為規諫。語切直。帝怒,下其使者於獄。詔曰:「前宗室有謗訕者置不治,茲復效尤。王,今之西伯也,欲為為之。」後二年而有祐橏之事,厚烷遂獲罪。

初,祁鍈有子十人,世子見滋,次盟津王見濍,次東垣王見水貢。見水蔥母有寵於祁鍈,規奪嫡,不得,竊世子金冊以去。祁鍈索之急,因怨不復朝,所為益不法。祁鍈言之憲宗,革為庶人。及康王薨,無子,見濍子祐善應及,以前罪廢,乃立東垣王子祐BT。至是祐橏求復郡王爵,怨厚烷不為奏,乘帝怒,摭厚烷四十罪,以叛逆告。詔駙馬中官即訊。還報反無驗,治宮室名號擬乘輿則有之。帝怒曰:「厚烷訕朕躬,在國驕傲無禮,大不道。」削爵,錮之鳳陽。隆慶元年復王爵,增祿四百石。厚烷自少至老,布衣蔬食。

世子載堉篤學有至性,痛父非罪見繫,築土室宮門外,席槁獨處者十九年。厚烷還邸,始入宮。萬歷十九年,厚烷薨。載堉曰:「鄭宗之序,盟津為長。前王見濍,既錫謚復爵矣,爵宜歸盟津。」後累疏懇辭。禮臣言:「載堉雖深執讓節,然嗣鄭王已三世,無中更理,宜以載堉子翊錫嗣。」載堉執奏如初,乃以祐橏之孫載璽嗣,而令載堉及翊錫以世子、世孫祿終其身,子孫仍封東垣王。二十二年正月,載堉上疏,請宗室皆得儒服就試,毋論中外職,中式者視才品器使。詔允行。明年又上曆算歲差之法,及所著《樂律書》,考辨詳確,識者稱之。卒謚端清。崇禎中,載璽子翊鐘以罪賜死,國除。

廬江王載堙,簡王元孫也。崇禎十七年二月,賊陷懷慶,載堙整冠服,端坐堂上。賊至,被執,欲屈之。歷聲曰:「吾天朝籓王,肯降汝逆賊耶!」詬罵不屈,遇害。賊執其長子翊檭,擁之北行。三月過定興,於旅店作絕命詞,遂不食死。

越靖王瞻墉,仁宗第三子。永樂二十二年封衢州。未之籓,宣宗賜以昌平莊田。正統四年薨。妃吳氏殉,謚貞惠。無後。

蘄獻王瞻垠,仁宗第四子。初封靜樂王。永樂十九年薨,謚莊獻。仁宗即位,追加封謚。無後。

襄憲王瞻墡,仁宗第五子。永樂二十二年封。莊警有令譽。宣德四年就籓長沙。正統元年徙襄陽。英宗北狩,諸王中,瞻墡最長且賢,眾望頗屬。太后命取襄國金符入宮,不果召。瞻墡上書,請立皇長子,令郕王監國,募勇智士迎車駕。書至,景帝立數日矣。英宗還京師,居南內,又上書景帝宜旦夕省膳問安,率群臣朔望見,無忘恭順。

英宗復辟,石亨等誣於謙、王文有迎立外籓語,帝頗疑瞻墡。久之,從宮中得瞻墡所上二書,而襄國金符固在太后閣中。乃賜書召瞻墡,比二書於《金滕》。入朝,宴便殿,避席請曰:「臣過汴,汴父老遮道,言按察使王賢,以誣逮詔獄,願皇上加察。」帝立出,命為大理卿。詔設襄陽護衛,命有司為王營壽藏。及歸,帝親送至午門外,握手泣別。瞻墡逡巡再拜,帝曰:「叔父欲何言?」頓首曰:「萬方望治如饑渴,願省刑薄斂。」帝拱謝曰:「敬受教。」目送出端門乃還。四年復入朝。命百官朝王於邸,詔王詣昌平謁三陵。及辭歸,禮送加隆,且敕王歲時與諸子得出城遊獵,蓋異數也。六年又召,以老辭。歲時存問,禮遇之隆,諸籓所未有。成化十四年薨。

子定王祁鏞嗣,弘治元年薨。子簡王見淑嗣,三年薨。子懷王祐材嗣。好鷹犬,蓄善馬,往返南陽八百里,日猶未晡。妃父井海誘使殺人。孝宗戒諭,戍海及其左右。祐材好道術,賜予無節,又嘗與興邸爭地,連逮七十餘家,獄久不決。大理卿汪綸兩解之,乃得已。十七年薨。弟康王祐櫍嗣,亦好道術。嘉靖二十九年薨。無子,從子莊王厚熲由陽山王嗣,定王曾孫也。

時王邸災,先世蓄積一空。厚潁折節為恭儉,節祿以餉邊,進金助三殿工。兩賜書幣。事嫡母王太妃及生母潘太妃,以孝聞。潘卒,殯之東偏。王太妃曰:「汝母有子,社稷是賴,無以我故避正寢。」厚熲泣曰:「臣不敢以非禮加臣母。」及葬,跣足扶櫬五十里。士大夫過襄者,皆為韋布交。四十五年薨。子靖五載堯嗣,萬曆二十三年薨。子翊銘嗣。崇禎十四年,張獻忠陷襄陽,遇害。

初,大學士楊嗣昌之視師也,以襄陽為軍府,增堞浚隍,貯五省餉金及弓刀火器。是年二月,獻忠邀殺嗣昌使於道,奪其符驗,以數十騎紿入襄城。夜半火作,遲明,賊大至。執翊銘南城樓,屬卮酒曰:「王無罪,王死,嗣昌得以死償王。」遂殺王及貴陽王常法,火城樓,焚其屍。賊去,僅拾顱骨數寸,妃妾輩死者四十三人。福清王常澄、進賢王常淦走免。事聞,帝震悼,命所司備喪禮,謚曰忠王。嗣昌朝惠王於荊州,謁者謝之曰:「先生惠顧寡人,願先之襄陽。」謂襄城之破,罪在嗣昌也。十七年以常澄嗣襄王,寄居九江,後徙汀州,不知所終。

棗陽王祐楒,憲王曾孫也,材武善文章,博涉星曆醫卜之書。嘉靖初上書,請考興獻帝。世宗以其議發自宗人,足厭服群心,褒之。更請除宗人祿,使以四民業自為生,賢者用射策應科第。寢不行。時襄王祐櫍病廢不事事,承奉邵亨挾權自恣,至捶死鎮寧王舅。祐楒誘致之,抉其目。帝遣大理少卿袁宗儒偕中官、錦衣往訊。亨論死,祐楒坐奪爵。帝幸承天,念祐楒前疏,復之。

荊憲王瞻堈,仁宗第六子。永樂二十二年封。宣德四年就籓建昌。宮中有巨蛇,蜿蜒自梁垂地,或憑王座。瞻堈大懼,請徙。正統十年徙蘄州。景泰二年上書請朝上皇。不許。四年薨。子靖王祁鎬嗣,天順五年薨。子見潚嗣。

靖王三子,長見潚,次都王梁見溥、樊山王見澋。見潚與見溥同母,怨母之匿見溥也,錮母,奪其衣食,竟死,出柩於竇。召見溥入後園,箠殺之。紿其妃何氏入宮,逼淫之。從弟都昌王見潭妻茆氏美,求通焉。見潭母馬氏防之嚴,見潚髡馬氏鞭之,囊土壓見潭死,械繫茆妃入宮。嘗集惡少年,輕騎徽服,涉漢水,掠人妻女。見澋懼其及也,密聞於孝宗,召至京。帝御文華門,命廷臣會鞫。見潚引伏,廢為庶人,錮西內。居二年,見潚從西內摭奏見澋罪,誣其與楚府永安王謀不軌。帝遣使往按問,不實。見澋更奏見潚嘗私造弓弩,與子祐柄有異謀。驗之實,賜見潚死,廢祐衲,而以見溥子祐橺嗣為荊王。時弘治七年也。十七年薨,謚曰和。

子端王厚烇嗣。性謙和,銳意典籍。嘉靖中病,辭祿。不允,令富順王厚昆攝朝謁。厚昆,和王第二子,與弟永新王厚熿以能詩善畫名。厚烇子永定王載墭長,厚昆即謝攝事,人尤以為賢。嘉靖三十二年,厚烇薨。載墭己前卒,其子恭王翊鉅嗣。

荊自靖王諸子交惡,失令譽。及厚烇兄弟感先世家難,以禮讓訓飭宗人。見澋曾孫載埁尤折節恭謹,以文行稱。郡王女例請祿於朝,載埁四女皆妻士人,不請封。嘗上《應詔》、《正禮》二疏。不報。讀《易》窮理,著《大隱山人集》。子翊金氐、翊踠、翊綯皆工詩,兄弟嘗共處一樓,號花萼社。翊鉅表載埁賢以訓諸子。諸子不率教,世子常泠尤殘恣。翊鉅言於朝,革為庶人。

隆慶四年,翊鉅薨。次子常水言嗣,萬曆四年薨。無子,弟康王常䍦由安城王嗣,萬曆二十五年薨。子定王由樊嗣,天啟二年薨。子慈畐嗣。崇禎時,流賊革裏眼、左金王詭降於楚帥。慈畐欲與為好,召宴,盛陳女樂。十六年正月,張獻忠陷蘄州,慈畐先一月薨。賊圍王宮,盡掠其所見妓樂去。

淮靖王蟾墺,仁宗第七子。永樂二十二年封。宣德四年就籓韶州。英宗即位之十月,以韶多瘴癘,正統元年徙饒州。正統十一年薨。子康王祁銓嗣,弘治十五年薨。世子見濂早卒,無子,從子定王祐啟嗣。游戲無度,左右倚勢暴橫,境內苦之。長史莊典以輔導失職自免。詔不許。推官汪文盛數持王府事。有顧嵩者病狂,持刀斧入王門,官校執詰之,謬言出汪指使。典白之守臣。鎮守太監黎安嘗以事至饒,從騎入端禮門,被撻,銜祐棨甚。先是,祐棨有名琴曰「天風環佩」,寧王宸濠求之,不得。又求濱湖地,不與。至是嗾安奏祐棨過失及文盛被誣事。詔下撫按訊。安與宸濠謀,不待報,遽繫典及府中官校鞫之。典辭倨,宸濠箠之,斃獄中,他所連坐甚眾。於是祐棨奏安挾仇殺典庇嵩。帝遣都御史金獻民、太監張欽往按治。祐棨畏宸濠,不能自明。欽等復言祐棨信奸徒為暴,請嚴戒之。軍校坐戍者二十餘人,典冤竟不白。

嘉靖三年,祐棨薨。無子,弟莊王祐楑嗣,十六年薨。子憲王厚燾嗣,四十二年薨。子恭王載坮嗣,萬曆五年薨。弟順王載堅嗣,二十三年薨。子翊金具嗣。翊金具之未王也,與妓王愛狎,冒妾額入宮,且令撫庶子常洪為子,陳妃與世子常清俱失愛,潛謀易嫡。御史陳王道以理諭王,出之外舍。常洪遂與宗人翊銂等謀,夜入王宮,盜冊寶、資貨以出。守臣上其事,王愛論死,勒常洪自盡,翊銂等削屬籍永錮,奪翊金具四歲祿。久之,薨。子常清嗣,國亡,不知所終。滕懷王瞻塏,仁宗第八子。永樂二十二年封雲南,未之國,洪熙元年薨。無後。

梁莊王瞻自,仁宗第九子。永樂二十二年封。宣德初,詔鄭、越、襄、荊、淮五王歲給鈔五萬貫,惟梁倍之。四年就籓安陸,故郢邸也。襄王瞻墡自長沙徙襄陽,道安陸,與瞻自留連不忍去。瀕別,瞻自慟曰:「兄弟不復更相見,奈何!」左右皆泣下。正統元年言府卑濕,乞更爽塏地。帝詔郢中歲歉,俟有秋理之。竟不果。六年薨。無子,封除。梁故得郢田宅園湖,後皆賜襄王。及睿宗封安陸,盡得郢、梁邸田,供二王祠祀。

衛恭王瞻埏,仁宗第十子。永樂二十二年封懷慶。幼善病,宣宗撫愛之,未就籓。歲時謁陵,皆命攝祀。孝謹好學,以賢聞。正統三年薨。妃楊氏殉,賜謚貞烈。無子,封除英宗九子。周太后生憲宗、崇王見澤。萬宸妃生德王見潾及皇子見水是、吉王見浚、忻王見治。王惠妃生許王見淳。高淑妃生秀王見澍。韋德妃生徽王見沛。

德莊王見潾,英宗第二子。初名見清。景泰三年封榮王。天順元年三月復東宮,同日封德、秀、崇、吉四王,歲祿各萬石。初國德州,改濟南。成化三年就籓。請得齊、漢二庶人所遺東昌、充州閒田及白雲、景陽、廣平三湖地。憲宗悉予之。復請業南旺湖,以漕渠故不許。又請漢庶人舊牧馬地,知府趙璜言地歸民間,供稅賦已久,不宜奪。帝從之。正德初,詔王府莊田畝徵銀三分,歲為常。見潾奏:「初年,兗州莊田歲畝二十升,獨清河一縣,成化中用少卿宋旻議,歲畝五升。若如新詔,臣將無以自給。」戶部執山東水旱相仍,百姓凋敝,宜如詔。帝曰:「王何患貧!其勿許。」十二年薨。子懿王祐榕嗣。

嘉靖中,戶部議核王府所請山場湖陂,斷自宣德以後者皆還官。詔允行。於是山東巡撫都御史邵錫奏德府莊田俱在革中,與祐榕相訐奏,錫持之益急。儀衛司軍額千七百人,逃絕者以餘丁補。錫謂非制,檄濟南知府楊撫籍諸補充者勿與餉。軍校大嘩,毀府門。詔逮問長史楊穀、楊孟法,戍儀衛副薛寧及軍校陶榮。諭王守侯度,毋徇群小滋多事。議者謂錫故激致其罪,不盡祐榕過云。此十一年八月事。至十八年,涇、徽二王復請得所革莊田,祐榕援以為請。詔仍與三湖地,使自徵其課。其年薨。孫恭王載墱嗣,萬歷二年薨。子定王翊館嗣,十六年薨。子常潔嗣,崇禎五年薨。世子由樞嗣,十二年正月,大清兵克濟南,見執。

見湜,英宗第三子。早卒。復辟後,不復追贈。

許悼王見淳,英宗第四子。景泰三年封。明年薨。禮臣請用親王禮葬。帝以王幼,殺其制。秀懷王見澍,英宗第五子。生於南宮,天順元年封。成化六年就籓汝寧。長史劉誠獻《千秋日鑒錄》,見澍朝夕誦之。就籓時,慮途中擾民,令併日行。王居隘,左右請遷文廟廣之。見澍不聽,曰:「居近學宮,時時聞糸玄頌聲,顧不美乎!」論《書》至《西伯戡黎》,長史誠主吳氏說,曰:「戡黎者,武王也。」右長史趙銳主孔氏說,曰:「實文王事。」爭之失色。見澍曰:「經義未有定論,不嫌往復。今若此,非先皇帝簡二先生意也。」成化八年薨。無子,封除。

崇簡王見澤,英宗第六子。生於南宮,天順元年封。成化十年就籓汝寧,故秀邸也。弘治八年七月,皇太后春秋高,思一見王,帝特敕召之。禮部尚書倪岳言:「數年來三王之國,道路供億,民力殫竭。今召王復來,往返勞費,兼水溢旱蝗,舟車所經,恐有他虞。親王入朝,雖有故事,自宣德來,已鮮舉行。英宗復辟,襄王奉詔來朝,雖篤敦敘之恩,實塞疑讒之隙,非故事也。」大學士徐溥亦以為言。帝重違太后意,不允。既而言官交章及之,乃已。十八年薨。子靖王祐樒嗣,正德六年薨。子恭王厚耀嗣。三王並有賢名,而靖王尤孝友。嘉靖十六年,厚耀薨。子莊王載境嗣,三十六年薨。子端王翊金爵嗣,萬歷三十八年薨。孫由樻嗣。崇禎十五年閏十一月,李自成陷汝寧,執由樻去,偽封襄陽伯,令諭降州縣之未下者。由樻不從,殺之於泌陽城。弟河陽王由材、世子慈輝等皆遇害。

吉簡王見浚,英宗第七子。生於南宮。天順元年封,時甫二歲。成化十三年就籓長沙,刻《先聖圖》及《尚書》於嶽麓書院,以授學者。嘉靖六年薨。孫定王厚冒嗣。請湘潭商稅益邸租,不許。十八年薨。子端王載均由光化王嗣,四十年薨。子莊王翊鎮嗣,隆慶四年薨。無子,庶兄宣王翊鑾由龍陽王嗣,萬曆四十六年薨。孫由棟嗣,崇禎九年薨。子慈煃嗣。十六年,張獻忠入湖南,同惠王走衡州,隨入粵。國亡後,死於緬甸。

忻穆王見治,英宗第八子。成化二年封。未就籓,八年薨。無後。

徽莊王見沛,英宗第九子。成化二年封。十七年就籓鈞州。承奉司自置吏,左布政使徐恪革之,見沛以聞。憲宗書諭王:「置吏,非制也,恪無罪。」正德元年薨。子簡王祐枱嗣,嘉靖四年薨。子恭王厚爵嗣,二十九年薨。子浦城王載埨嗣。

初,厚爵好琴,斲琴者與知州陳吉交惡,厚爵庇之,劾吉,逮詔獄。都御史駱昂、御史王三聘白吉冤。帝怒,並逮之,昂杖死,三聘、吉俱戍邊。議者不直厚爵。時方士陶仲文有寵於世宗,厚爵厚結之。仲文具言王忠敬奉道。帝喜,封厚爵太清輔元宣化真人,予金印。及載埨嗣王,益以奉道自媚於帝,命綰其父真人印。南陽人梁高輔自言能導引服食,載埨用其術和藥,命高輔因仲文以進帝。封高輔通妙散人,載埨清微翊教輔化忠孝真人。載埨遂益恣,壞民屋,作臺榭苑囿。庫官王章諫,杖殺之。嘗微服之揚州、鳳陽,為邏者所獲,羈留三月,走歸。

時高輔被上寵,不復親載埨,載埨銜之。已而為帝取藥不得,求載埨舊所蓄者,載埨不與,而與仲文。高輔大恨,乘間言載埨私往南中,與他過失。帝疑之,奪真人印。仲文知釁已成,不復敢言。三十五年有民耿安者,奏載埨奪其女,下按治。有司因發其諸不法事。獄成,降為庶人,錮高牆。時載侖居宮中,所司防守嚴,獄詞不得聞。及帝遣內臣同撫按至,始大懼。登樓,望龍亭後有紅板輿,歎曰:「吾不能自明,徒生奚為!」遂自縊死。妃沈氏、次妃林氏爭取帛自縊。子安陽王翊金奇、萬善王翊金方并革爵,及未封子女,皆遷開封,聽周王約束,國除。景皇帝一子,懷獻太子見濟。母杭妃。始為郕王世子。英宗北狩,皇太后命立憲宗為皇太子,而以郕王監國。及郕王即位,心欲以見濟代太子,而難於發,皇后汪氏又力以為不可,遲回久之。太監王誠、舒良為帝謀,先賜大學士陳循、高穀百金,侍郎江淵、王一寧、蕭鎡,學士商輅半之,用以緘其口,然猶未發也。會廣西土官都指揮使黃矰以私怨戕其弟思明知府岡,滅其家,所司聞於朝。矰懼罪,急遣千戶袁洪走京師,上疏勸帝早與親信大臣密定大計,易建東宮,以一中外之心,絕覬覦之望。疏入,景帝大喜,亟下廷臣會議,且令釋矰罪,進階都督。時景泰三年四月也。

疏下之明日,禮部尚書胡濙,侍郎薩琦、鄒乾集文武群臣廷議。眾相顧,莫敢發言。惟都給事中李侃、林聰,御史朱英以為不可。吏部尚書王直亦有難色。司禮太監興安歷聲曰:「此事不可已,即以為不可者,勿署名,無持兩端。」群臣皆唯唯署議。於是濙苳等暨魏國公徐承宗,寧陽侯陳懋,安遠侯柳溥,武清侯石亨,成安侯郭晟,定西侯蔣琬,駙馬都尉薛桓,襄城伯李瑾,武進伯朱瑛,平鄉伯陳輔,安鄉伯張寧,都督孫鏜、張軏、楊俊,都督同知田禮、范廣、過興、衛穎,都督僉事張兒、劉深、張通、郭瑛、劉鑑、張義,錦衣衛指揮同知畢旺、曹敬,指揮僉事林福,吏部尚書王直,戶部尚書文淵閣大學士陳循,工部尚書東閣大學士高穀,吏部尚書何文淵,戶部尚書金濂,兵部尚書于謙,刑部尚書俞士悅,左都御史王文、王翱、楊善,吏部侍郎江淵、俞山、項文耀,戶部侍郎劉中敷、沈翼、蕭鎡,禮部侍郎王一寧,兵部侍郎李賢,刑部侍郎周瑄,工部侍郎趙榮、張敏,通政使李錫,通政欒惲、王復,參議馮貫,諸寺卿蕭維禎、許彬、蔣守約、齊整、李賓,少卿張固、習嘉言、李宗周、蔚能、陳誠、黃士俊、張翔、齊政,寺丞李茂、李希安、柴望、酈鏞、楊詢、王溢,翰林學士商輅,六科都給事中李言贊、李侃、李春、蘇霖、林聰、張文質,十三道御史王震、朱英、塗謙、丁泰亨、強弘、劉琚、陸厚、原傑、嚴樞、沈義、楊宜、王驥、左鼎上言:「陛下膺天明命,中興邦家,統緒之傳宜歸聖子,黃矰奏是。」制曰:「可。禮部具儀,擇日以聞。」即日,簡置東宮官,公孤詹事僚屬悉備。

五月,廢汪后,立杭妃為皇后,更封太子為沂王,立見濟為太子。詔曰:「天佑下民作之君,實遺安於四海;父有天下傳之子,斯本固於萬年。」大赦天下,令百官朔望朝太子,賜諸親王、公主、邊鎮、文武內外群臣,又加賜陳循、高穀、江淵、王一寧、蕭鎡、商輅各黃金五十兩。四年二月乙未,太子冠。十一月,以御史張鵬言,簡東宮師傅講讀官。越四日,太子薨,謚曰懷獻,葬西山。天順元年,降稱懷獻世子,諸建議易儲者皆得罪。憲宗十四子。萬貴妃生皇第一子。柏賢妃生悼恭太子祐極。紀太后生孝宗。邵太后生興獻帝祐杬、岐王祐棆、雍王祐枟。張德妃生益王祐檳、衡王祐楎、汝王祐梈。姚安妃生壽王祐楮。楊恭妃生涇王祐橓、申王祐楷。潘端妃生榮王祐樞。王敬妃生皇第十子。第一子、第十子皆未名殤。

悼恭太子祐極,憲宗次子。成化七年立為皇太子薨。

岐惠王祐棆,憲宗第五子。成化二十三年與益、衡、雍三王同日封。弘治八年之籓德安。十四年薨。無子,封除。

益端王祐檳,憲宗第六子。弘治八年之籓建昌,故荊邸也。性儉約,巾服浣至再,日一素食。好書史,愛民重士,無所侵擾。嘉靖十八年薨。子莊王厚燁嗣,性朴素,外物無所嗜。三十五年薨。無子,弟恭王厚炫嗣,自奉益儉,辭祿二千石。萬曆五年薨。孫宣王翊鈏嗣,嗜結客,厚炫所積府藏,悉斥以招賓從,通聘問於諸籓,不數年頓盡。三十一年薨。子敬王常水遷嗣,四十三年薨。子由本嗣,國亡竄閩中。

衡恭王祐楎,憲宗第七子。弘治十二年之籓青州。嘉靖十七年薨。子莊王厚燆嗣,嘗辭祿五千石以贍宗室,宗人德之。隆慶六年薨。子康王載圭嗣,萬曆七年薨。無子,弟安王載封嗣,十四年薨。子定王翊鑊嗣,二十年薨。子常水庶嗣。新樂王載璽,恭王孫也。博雅善文辭,索諸籓所纂述,得數十種,梓而行之。又撰《洪武聖政頌》、《皇明政要》諸書,多可傳者。從父高唐王厚煐、齊東王厚炳皆以博學篤行聞。嘉靖中,賜敕獎諭者再。

雍靖王祐枟,憲宗第八子。初封保寧,弘治十二年之籓衡州。地卑濕,宮殿朽敗不可居,邸中數有異,乞移山東東平州。廷臣以擇地別建,勞民傷財,四川敘州有申王故府,宜徙居之。詔可。既而以道遠不可徙。正德二年,地裂,宮室壞,王薨。無子,封除。

壽定王祐耆,憲宗第九子。弘治四年與汝、涇、榮、申四王同日封。十一年就籓保寧。正德元年以岐王世絕,改就岐邸於德安。校尉橫攖市民,知府李重抑之,奏逮重。安陸民劉鵬隨重詣大理對簿,重未之識也,訝之。鵬曰:「太守仁,為民受過,民皆得效死,豈待識乎!」重卒得白。祐耆聞而悔之,後以賢聞。嘉靖二十四年薨。無子,除封。

汝安王祐梈,憲宗第十一子。弘治十四年之籓衛輝。正德十五年請預支食鹽十年為婚費。詔別給長蘆鹽二千引,食鹽如故。世宗南巡,迓於途,甚恭。加祿五百石,錫金幣。嘉靖二十年薨。無子,封除。

涇簡王祐橓,憲宗第十二子。弘治十五年之籓沂州。嘉靖十六年薨。子厚烇未封而卒。無子,封除。榮莊王祐樞,憲宗第十三子。正德初尚留京邸,乞霸州信安鎮田,故牧地也。部臣言:「永樂中,設立草場,蕃育馬匹,以資武備。至成化中,近倖始陳乞為莊。後岐、壽二府相沿,莫之改正。暨孝宗皇帝留神戎務,清理還屯,不以私廢公也。今榮王就國有期,所請宜勿與。」三年之籓常德。祐樞狀貌類高帝,居國稍驕縱。世宗詔以沅江酉港、天心、團坪河泊稅入王邸。嘉靖十八年薨。孫恭王載墐嗣,萬曆二十三年薨。子翊鉁嗣,四十年薨。子常溒嗣,薨。子憲王由枵嗣,薨。子慈炤嗣。張獻忠入湖南,奉母妃姚氏走辰溪,不知所終。

申懿王祐楷,憲宗第十四子。封敘州,未就籓。弘治十六年薨。無子,封除。

孝宗二子。武宗、蔚王厚煒,俱張皇后生。

蔚悼王厚煒,孝宗次子,生三歲薨。追加封謚。

○世宗諸子

哀沖太子載基莊敬太子載景王載圳潁王載啇戚王載沴薊王載匱均王載夙

○穆宗諸子

憲懷太子翊釴靖王翊鈴潞王翊鏐

○神宗諸子

邠王常漵福王常洵沅王常治瑞王常浩惠王常潤桂王常瀛

○光宗諸子

簡王由學齊王由楫懷王田模湘王由栩惠王由橏

○熹宗諸子

懷沖太子慈然悼懷太子慈焴獻懷太子慈炅

○莊烈帝諸子

太子慈烺懷王慈亙定王慈炯永王慈照悼靈王慈煥悼懷王

世宗八子。閻貴妃生哀沖太子載基。王貴妃生莊敬太子載。杜太后生穆宗。盧靖妃生景王載圳。江肅妃生潁王載啇。趙懿妃生戚王載沴。陳雍妃生薊王載匱。趙榮妃生均王載夙。

哀沖太子載基,世宗第一子。生二月而殤。

莊敬太子載,世宗第二子。嘉靖十八年,世宗將南巡,立為皇太子,甫四歲,命監國,以大學士夏言為傅。尚書霍韜、郎中鄒守益獻《東宮聖學圖冊》,疑為謗訕,幾獲罪。帝既得方士段朝用,思習修攝術,諭禮部,具皇太子監國儀。太僕卿楊最諫,杖死,監國之議亦罷。贊善羅洪先、趙時春、唐順之請太子出閤,講學文華殿,皆削籍。太廟成,命太子攝祀。二十八年三月行寇禮,越二日薨。帝命與哀沖太子並建寢園,歲時祭祀,從諸陵後。

景恭王載圳,世宗第四子。嘉靖十八年冊立太子,同日封穆宗裕王、載圳景王。其後太子薨,廷臣言裕王次當立。帝以前太子不永,遲之。晚信方士語,二王皆不得見。載圳既與裕王並出邸,居處衣服無別。載圳年少,左右懷窺覬,語漸聞,中外頗有異論。四十年之國德安。居四年薨。帝謂大學士徐階曰:「此子素謀奪嫡,今死矣。」初,載圳之籓,多請莊田。部議給之。荊州沙市不在請中。中使責市租,知府徐學謨執不與,又取薪稅於漢陽之劉家塥,推官吳宗周持之,皆獲譴。其他土田湖陂侵入者數萬頃。王無子,歸葬西山,妃妾皆還居京邸,封除。

潁殤王載啇,世宗第五子。生未踰月殤。

戚懷王載沴,世宗第六子。

薊哀王載匱,世宗第七子。

均思王載夙,世宗第八子。三王俱未踰歲殤,追加封謚。

穆宗四子。李皇后生憲懷太子翊釴,孝定太后生神宗、潞王翊鏐,其靖王翊鈴,母氏無考。

憲懷太子翊釴,穆宗長子。生五歲殤,贈裕世子。隆慶元年追謚。

靖悼王翊鈴,穆宗第二子。生未踰年殤,贈藍田王。隆慶元年追加封謚。

潞簡王翊鏐,穆宗第四子。隆慶二年生,生四歲而封。萬歷十七年之籓衛輝。初,翊鏐以帝母弟居京邸,王店、王莊遍畿內。比之籓,悉以還官,遂以內臣司之。皇店、皇莊自此益侈。翊鏐居籓,多請贍田、食鹽,無不應者。其後福籓遂緣為故事。明初,親王歲祿外,量給草場牧地,間有以廢壤河灘請者,多不及千頃。部臣得執奏,不盡從也。景王就籓時,賜予概裁省。楚地曠,多閑田,詔悉予之。景籓除,潞得景故籍田,多至四萬頃,部臣無以難。至福王常洵之國,版籍更定,民力益絀,尺寸皆奪之民間,海內騷然。論者推原事始,頗以翊鏐為口實云。翊鏐好文,性勤飭,恒以歲入輸之朝,助工助邊無所惜,帝益善之。四十二年,皇太后哀問至,翊鏐悲慟廢寢食,未幾薨。

世子常淓幼,母妃李氐理籓事。時福王奏請,輒取中旨,帝於王妃奏,亦從中下,示無異同。部臣言:「王妃奏陳四事,如軍校月糧之當給發,義和店之預防侵奪,義所當許;至歲祿之欲先給,王莊之欲更設,則不當許。且於王無絲豪益,徒令邸中人日魚肉小民,飽私囊。將來本支千億,請索日頻,盡天府之版章,給王邸而不足也。」不報。四十六年,常淓嗣。崇禎中,流賊擾秦、晉、河北。常淓疏告急,言:「衛輝城卑土惡,請選護衛三千人助守,捐歲入萬金資餉,不煩司農。」朝廷嘉之。盜發王妃塚,常淓上言:「賊延蔓漸及江北,鳳、泗陵寢可虞,宜早行剿滅。」時諸籓中能急國難者,惟周、潞二王云。後賊躪中州,常淓流寓於杭。順治二年六月降於我大清。神宗八子。王太后生光宗。鄭貴妃生福王常洵、沅王常治。周端妃生瑞王常浩。李貴妃生惠王常潤、桂王常瀛。其邠王常漵、永思王常溥,母氏無考。

邠哀王常漵,神宗第二子。生一歲殤。

福恭王常洵,神宗第三子。初,王皇后無子,王妃生長子,是為光宗。常洵次之,母鄭貴妃最幸。帝久不立太子,中外疑貴妃謀立己子,交章言其事,竄謫相踵,而言者不止。帝深厭苦之。二十九年始立光宗為太子,而封常洵福王,婚費至三十萬,營洛陽邸第至二十八萬,十倍常制。廷臣請王之籓者數十百奏。不報。至四十二年,始令就籓。

先是,海內全盛,帝所遣稅使、礦使遍天下,月有進奉,明珠異寶文毳錦綺山積,他搜括贏羨億萬計。至是多以資常洵。臨行出宮門,召還數四,期以三歲一入朝。下詔賜莊田四萬頃。所司力爭,常洵亦奏辭,得減半。中州腴土不足,取山東、湖廣田益之。又奏乞故大學士張居正所沒產,及江都至太平沿江荻洲雜稅,並四川鹽井榷茶銀以自益。伴讀、承奉諸官,假履畝為名,乘傳出入河南北、齊、楚間,所至騷動。又請淮鹽千三百引,設店洛陽與民市。中使至淮、揚支鹽,乾沒要求輙數倍。而中州舊食河東鹽,以改食淮鹽故,禁非王肆所出不得鬻,河東引遏不行,邊餉由此絀。廷臣請改給王鹽於河東,且無與民市。弗聽。帝深居久,群臣章奏率不省。獨福籓使通籍中左門,一日數請,朝上夕報可。四方姦人亡命,探風旨,走利如鶩。如是者終萬曆之世。

及崇禎時,常洵地近屬尊,朝廷尊禮之。常洵日閉閣飲醇酒,所好惟婦女倡樂。秦中流賊起,河南大旱蝗,人相食,民間藉藉,謂先帝耗天下以肥王,洛陽富於大內。援兵過洛者,喧言:「王府金錢百萬,而令吾輩枵腹死賊手。」南京兵部尚書呂維祺方家居,聞之懼,以利害告常洵,不為意。十三年冬,李自成連舀永寧、宜陽。明年正月,參政王胤昌帥眾警備,總兵官王紹禹,副將劉見義、羅泰各引兵至。常洵召三將入,賜宴加禮。越數日,賊大至,攻城。常洵出千金募勇士,縋而出,用矛入賊營,賊稍卻。夜半,紹禹親軍從城上呼賊相笑語,揮刀殺守堞者,燒城樓,開北門納賊。常洵縋城出,匿迎恩寺。翌日,賊跡而執之,遂遇害。兩承奉伏尸哭,賊捽之去。承奉呼曰:「王死某不願生,乞一棺收王骨,棆粉無所恨。」賊義而許之。桐棺一寸,載以斷車,兩人即其旁自縊死。王妃鄒氏及世子由崧走懷慶。賊火王宮,三日不絕。事聞,帝震悼,輟朝三日,令河南有司改殯。

十六年秋七月,由崧襲封,帝親擇宮中寶玉帶賜之。明年三月,京師失守,由崧與潞王常淓俱避賊至淮安。四月,鳳陽總督馬士英等迎由崧入南京。五月庚寅,稱監國。以兵部尚書史可法、戶部尚書高弘圖及士英俱為大學士,士英仍督鳳陽軍務。壬寅自立於南京,偽號弘光。史可法督師江北。召士英入,分淮、揚、鳳、廬為四鎮,以總兵官黃得功、劉良佐、劉澤清、高傑領之。

由崧性闇弱,湛於酒色聲伎,委任士英及士英黨阮大鋮,擢至兵部尚書,巡閱江防。二人日以鬻官爵、報私憾為事。事詳諸臣傳中。未幾,有王之明者,詐稱莊烈帝太子,下之獄。又有婦童氏,自稱由崧妃,亦下獄。於是中外嘩然。明年三月,寧南侯左良玉舉兵武昌,以救太子誅士英為名,順流東下。阮大鋮、黃得功等帥師禦之。而我大清兵以是年五月己丑渡江。辛卯夜,由崧走太平,蓋趨得功軍也。壬辰,士英挾由崧母妃奔杭州。癸巳,由崧至蕪湖。丙申,大兵至南京城北,文武官出降。丙午,執由崧至南京。九月甲寅,以歸京師。

沅懷王常治,神宗第四子。生一歲殤。

瑞王常浩,神宗第五子。初,太子未立,有三王並封之旨,蓋謂光宗、福王及常浩也。尋以群臣爭,遂寢。二十九年,東宮立,與福、惠、桂三王同日封。常洵以長,先之籓。常浩年已二十有五,尚末選婚。群臣交章言,率不報,而日索部帑為婚費,贏十八萬,藏宮中,且言冠服不能備。天啟七年之籓漢中。崇禎時,流寇劇,封地當賊衝。七年上書言:「臣託先帝骨肉,獲奉西籓,未期年而寇至。比西賊再渡河,闌入漢興,破洵陽,逼興安,紫陽、平利、白河相繼陷沒。督臣洪承疇單騎裹甲出入萬山,賊始敗遁。臣捐犒軍振飢銀七千餘兩。此時撫臣練國事移兵商、洛,按臣范復粹馳赴漢中,近境稍寧。既而鳳縣再陷,蜀賊入秦州,楚賊上興安。六月遂犯郡界,幸諸將憑江力拒,賊方稍退。臣在萬山絕谷中,賊四面至,覆亡無日。臣肺腑至親,籓封最僻,而於寇盜至迫,惟陛下哀憐。」常浩在宮中,衣服禮秩降等,好佛不近女色。及寇逼秦中,將吏不能救,乞師於蜀。總兵官侯良柱援之,遂奔重慶。隴西士大夫多挈家以從。十七年,張獻忠陷重慶,被執,遇害。時天無雲而雷者三,從死者甚眾。

惠王常潤,神宗第六子。福王之籓,內廷蓄積為空。中官藉諸王冠婚,索部帑以實宮中,所需輒數十萬,珠寶稱是。戶部不能給。常潤與弟常瀛年二十,皆未選婚。其後兵事亟,始滅殺成禮。天啟七年之籓荊州。崇禎十五年十二月,李自成再破夷陵、荊門,常潤走湘潭,自成入荊州據之。常潤之渡湘也,遇風於陵陽磯,宮人多漂沒,身僅以免,就吉王於長沙。十六年八月,張獻忠陷長沙,常潤走衡州,就桂王。衡州繼陷,與吉王、桂王走永州。巡按御史劉熙祚遣人護三王入廣西,以身當賊。永州陷,熙祚死之。

桂端王常瀛,神宗第七子。天啟七年之籓衡州。崇禎十六年,衡州陷,與吉、惠二王同走廣西,居梧州。

大清順治二年,大兵平江南,福王就擒。在籍尚書陳子壯等將奉常瀛監國,會唐王自立於福建,遂寢。是年,薨於蒼梧。

世子已先卒,次子安仁王由愛亦未幾卒。次由榔,崇禎時,封永明王。

三年八月,大兵取汀州,執唐王聿鍵。於是兩廣總督丁魁楚、廣西巡撫瞿式耜、巡按王化澄與舊臣呂大器等共推由榔監國。母妃王氏曰:「吾兒不勝此,願更擇可者。」魁楚等意益堅,合謀迎於梧。十月十四日監國肇慶,以魁楚、大器、式耜為大學士,餘授官有差。是月大兵取贛州,內侍王坤倉卒奉由榔仍走梧州,式耜等力爭,乃回肇慶。十一月,唐王弟聿金粵自閩浮海至粵。時閩舊臣蘇觀生撤兵奔廣州,與布政使顧元鏡、總兵官林察等謀立聿金粵,偽號紹武,與由榔相拒。是月由榔亦自立於肇慶,偽號永曆,遣兵部侍郎林佳鼎討聿金粵。會大兵由福建取廣州,執聿金粵,觀生自縊,祭酒梁朝鐘、太僕卿霍子衡等俱死。肇慶大震,王坤復奉由榔走梧州。

明年二月,由平樂、潯州走桂林。魁楚棄由榔,走岑溪,降於大軍。既而平樂不守,由榔大恐。會武岡總兵官劉承胤以兵至全州,王坤請赴之。式耜力諫。不聽。乃以式耜及總兵官焦璉留守桂林,封陳邦傳為思恩侯,守昭平,遂趨承胤軍中。三月封承胤安國公,錦衣指揮馬吉翔等為伯。承胤挾由榔歸武岡,改曰奉天府,政事皆決焉。

是時,長沙、衡、永皆不守,湖廣總督何騰蛟與侍郎嚴起恒走白牙市。六月,由榔遣官召騰蛟至,密使除承胤,顧承胤勢盛,騰蛟復還白牙。大兵由寶慶趨武岡,馬吉翔等挾由榔走靖州,承胤舉城降。由榔又奔柳州。道出古泥。總兵官侯性、太監龐天壽帥舟師來迎。會天雨飢餓,性供帳甚備。九月,土舍覃鳴珂作亂,大掠城中,矢及由榔舟。先是,大兵趨桂林,焦璉拒守甚力,又廣州有警,大兵東向,桂林稍安。既而湖南十三鎮將郝永忠、盧鼎等俱奔赴桂林,騰蛟亦至,與式耜議分地給諸將,使各自為守。璉已先復陽朔、平樂,陳邦傳復潯州,合兵復梧州,廣西全省略定。十二月,由榔返桂林。

五年二月,大兵至靈川,郝永忠潰於興安,奔還,挾由榔走柳州。大兵攻桂林,式耜、騰蛟拒戰。時南昌金聲桓等叛,降於由榔。八月,由榔至肇慶。六年春,大兵下湘潭,何騰蛟死。明年,由榔走梧州。是年十二月,大兵入桂林,瞿式耜及總督張同敞死焉。由榔聞報大懼,自梧州奔南寧。時孫可望已據滇、黔,受封為秦王。八年三月,遣兵來衛,殺嚴起恒等。

九年二月,可望迎由榔入安隆所,改曰安龍府。久之,日益窮促,聞李定國與可望有隙,遣使密召定國,以兵來迎。馬吉翔黨於可望,偵知之,大學士吳貞毓以下十餘人皆被殺。事詳《貞毓傳》。後二年,李定國敗於新會,將由安隆入滇。可望患之,促由榔移貴陽就己。由榔故遲行。定國至,遂奉由榔由安南衛走雲南,居可望署中,封定國晉王。可望以妻子在滇,未敢動。明年,由榔送其妻子還黔,遂舉兵與定國戰於三岔。可望將白文選單騎奔定國軍。可望敗,挈妻子赴長沙大軍前降。

十五年三月,大兵三路入雲南。定國阨雞公背,斷貴州道,別將守七星關,抵生界立營,以牽蜀師。大兵出遵義,由水西取烏撒,守將棄關走,李定國連敗於安隆,由榔走永昌。明年正月三日,大兵入雲南,由榔走騰越。定國敗於潞江,又走南甸。二十六日,抵囊木河,是為緬境。緬勒從官盡棄兵仗,始啟關,至蠻莫。二月,緬以四舟來迎,從官自覓舟,隨行者六百四十餘人,陸行者自故岷王子而下九百餘人,期會於緬甸。十八日至井亙。黔國公沐天波等謀奉由榔走戶、獵二河,不果。五月四日,緬復以舟來迎。明日,發井亙,行三日,至阿瓦。阿瓦者,緬酋所居城也。又五日至赭硜。陸行者緬人悉掠為奴,多自殺。惟岷王子八十餘人流入暹羅。緬人於赭硜置草屋居由榔,遣兵防之。

十七年,定國、文選與緬戰,索其主,連敗緬兵,緬終不肯出由榔。十八年五月,緬酋弟莽猛白代立,紿從官渡河盟。既至,以兵圍之,殺沐天波、馬吉翔、王維恭、魏豹等四十有二人,詳《任國璽傳》。存者由榔與其屬二十五人。十二月,大兵臨緬,白文選自木邦降,定國走景線,緬人以由榔父子送軍前。明年四月,死於雲南。六月,李定國卒,其子嗣興等降。

永思王常溥,神宗第八子。生二歲殤。光宗七子。王太后生熹宗、簡王由學。王選侍生齊王由楫。李選侍生懷王由模。劉太后生莊烈皇帝。定懿妃生湘王由栩。敬妃生惠王由橏。

簡懷王由學,光宗第二子。生四歲殤。齊思王由楫,光宗第三子。生八歲殤。懷惠王由模,光宗第四子。生五歲殤。湘懷王由栩,光宗第六子。惠昭王由橏,光宗第七子。俱早殤。五王皆追加封謚。熹宗三子。懷沖太子慈然,不詳其所生母。皇貴妃范氏生悼懷太子慈焴。容妃任氏生獻懷太子慈炅。

懷沖太子慈然,熹宗第一子。悼懷太子慈焴,熹宗第二子。獻懷太子慈炅,熹宗第三子。與懷沖、悼懷皆殤。莊烈帝七子。周皇后生太子慈烺、懷隱王慈烜、定王慈炯。田貴妃生永王慈炤、悼靈王慈煥、悼懷王及皇七子。

太子慈烺,莊烈帝第一子。崇禎二年二月生,三年二月立為皇太子。十年預擇東宮侍班講讀官,命禮部尚書姜逢元,詹事姚明恭,少詹王鐸、屈可伸侍班;禮部侍郎方逢年,諭德項煜,修撰劉理順,編修吳偉業、楊廷麟、林曾志講讀;編修胡守恒、楊士聰校書。十一年二月,太子出閤。十五年正月開講,閣臣條上講儀。七月改慈慶宮為端本宮。慈慶,懿安皇后所居也。時太子年十四,議明歲選婚,故先為置宮,而移懿安后於仁壽殿。既而以寇警暫停。京師陷,賊獲太子,偽封宋王。及賊敗西走,太子不知所終。由崧時,有自北來稱太子者,驗之,以為駙馬都尉王昺孫王之明者偽為之,繫獄中,南京士民嘩然不平。袁繼咸及劉良佐、黃得功輩皆上疏爭。左良玉起兵亦以救太子為名。一時真偽莫能知也。由崧既奔太平,南京亂兵擁王之明立之。越五日,降於我大清。

懷隱王慈亙,莊烈帝第二子。殤。

定王慈炯,莊烈帝第三子。崇禎十四年六月諭禮臣:「朕第三子,年已十齡,敬遵祖制,宜加王號。但既受冊封,必具冕服,而《會典》開載,年十二或十五始行冠禮。十齡受封加冠,二禮可並行乎?」於是禮臣歷考經傳及本朝典故以奏。定於是歲冊封,越二年行冠禮。九月封為定王。十一月選新進士為檢討,國子助教等官為待詔,充王講讀官,以兩殿中書充侍書。十七年,京師陷,不知所終。

永王慈炤,莊烈帝第四子。崇禎十五年三月封永王。賊陷京師,不知所終。悼靈王慈煥,莊烈帝第五子。生五歲而病,帝視之,忽云:「九蓮菩薩言,帝待外戚薄,將盡殤諸子。」遂薨。九蓮菩薩者,神宗母,孝定李太后也。太后好佛,宮中像作九蓮座,故云。帝念王靈異,封為孺孝悼靈王玄機慈應真君,命禮臣議孝和皇太后、莊妃、懿妃道號。禮科給事中李焻言:「諸后妃,祀奉先殿,不可崇邪教以亂徽稱。」不聽。十六年十二月,改封宣顯慈應悼靈王,去「真君」號。

悼懷王,莊烈帝第六子,生二歲殤。第七子,生三歲殤。名俱無考。贊曰:有明諸籓,分封而不錫土,列爵而不臨民,食祿而不治事。蓋矯枉鑒覆,所以杜漢、晉末大之禍,意固善矣。然徒擁虛名,坐縻厚祿,賢才不克自見,知勇無所設施。防閑過峻,法制日增。出城省墓,請而後許,二王不得相見。籓禁嚴密,一至於此。當太祖時,宗籓備邊,軍戎受制,贊儀疏屬,且令遍歷各國,使通親親。然則法網之繁,起自中葉,豈太祖眾建屏籓初計哉!

