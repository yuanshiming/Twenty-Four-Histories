\article{列傳第一百四}

\begin{pinyinscope}
吳山陸樹聲(子彥章}}瞿景淳子汝稷汝說田一俊?咍蝽鯈懋學從孫壽民)黃鳳翔韓世能餘繼登馮琦{從祖惟訥從父子咸王圖劉曰寧翁正春劉應秋子同升唐文獻楊道賓陶望齡李勝芳蔡毅中公鼐羅喻義姚希孟許士柔顧錫疇

吳山,字曰靜,高安人。嘉靖十四年進士及第,授編修。累官禮部左侍郎。三十五年,改吏部。尋代王用賓為禮部尚書。明年,加太子太保。山與嚴嵩鄉里。嵩子世蕃介大學士李本飲山,欲與為婚姻。山不可,世蕃不悅而罷。帝欲用山內閣,嵩密阻之。府丞朱隆禧者,考察罷官,獻方術,得加禮部侍郎。及卒請恤,山執不與。裕、景二邸並建,國本未定。三十九年冬,帝忽諭禮部,具景王之籓儀。嵩知帝激於郭希顏疏,欲覘人心,諷山留王。山曰:「中外望此久矣」,立具儀以奏,王竟之籓。司禮監黃錦嘗竊語山曰:「公他日得為編氓幸矣;王之籓,非帝意也。」

明年二月朔,日當食,微陰。曆官言:「日食不見,即同不食。」嵩以為天眷,趣部急上賀,侍郎袁煒亦為言。山仰首曰:「日方虧,將誰欺耶?」仍救護如常儀。帝大怒,山引罪。帝謂山守禮無罪,而責禮科對狀。給事中李東華等震懼,劾山,請與同罪。帝乃責山賣直沽名,停東華俸。嵩言罪在部臣。帝乃貰東華等,命姑識山罪。吏科梁夢龍等見帝怒山甚,又惡專劾山,乃并吏部尚書吳鵬劾之。詔鵬致仕,山冠帶閑住。時皆惜山而深快鵬之去。穆宗即位,召為南京禮部尚書,堅辭不赴,卒,贈少保,謚文端。

陸樹聲,字與吉,松江華亭人。初冒林姓,及貴乃復。家世業農。樹聲少力田,暇即讀書。舉嘉靖二十年會試第一。選庶吉士,授編修。三十一年,請急歸。遭父喪,久之,起南京司業。未幾,復請告去。起左諭德,掌南京翰林院。尋召還春坊,不赴。久之,起太常卿,掌南京祭酒事。嚴敕學規,著條教十二以勵諸生。召為吏部右侍郎,引病不拜。隆慶中,再起故官,不就。神宗嗣位,即家拜禮部尚書。

初,樹聲屢辭朝命,中外高其風節。遇要職,必首舉樹聲,唯恐其不至。張居正當國,以得樹聲為重,用後進禮先謁之。樹聲相對穆然,意若不甚接者,居正失望去。一日,以公事詣政府。見席稍偏,熟視不就坐,居正趣為正席。其介介如此。北部要增歲幣,兵部將許之,樹聲力爭。歲終,陳四方災異,請帝循舊章,省奏牘,慎賞賚,防壅蔽,納讜言,崇儉德,攬魁柄,別忠邪。詔皆嘉納。

萬歷改元,中官不樂樹聲,屢宣詣會極門受旨,且頻趣之。比趨至,則曹司常事耳。樹聲知其意,連疏乞休。居正語其弟樹德曰:「朝廷行相平泉矣。」平泉者,樹聲別號也。樹聲聞之曰:「一史官,去國二十年,豈復希揆席耶?且虛拘何益。」其冬,請愈力,乃命乘傳歸。辭朝,陳時政十事,語多切中,報聞而已。居正就邸舍與別,問誰可代者。舉萬士和、林燫。比出國門,士大夫傾城追送,皆謝不見。

樹聲端介恬雅,翛然物表,難進易退。通籍六十餘年,居官未及一紀。與徐階同里,高拱則同年生。兩人相繼柄國,皆辭疾不出。為居正所推,卒不附也。已,給廩隸如制,加太子少保,再遣存問。弟樹德,自有傳。子彥章,萬曆十七年進士。樹聲誡毋就館選,隨以行人終養。詔給月俸,異數也。樹聲年九十七卒。贈太子太保。謚文定。彥章有節概,官至南京刑部侍郎。

瞿景淳,字師道,常熟人。八歲能屬文。久困諸生間,教授里中自給。嘉靖二十三年,舉會試第一,殿試第二,授編修。鄭王厚烷以言事廢,徙鳳陽。景淳奉敕封其子載堉為世子,攝國事。世子內懼,贐重幣,景淳卻之。時恭順侯吳繼爵為正使,已受幣,慚景淳,亦謝不納。既而語景淳曰:「上遣使密詗狀,微公,吾幾中法。」滿九載,遷侍讀,請急歸。江南久苦倭,總督胡宗憲師未捷。景淳還京,謁大學士嚴嵩。嵩語之曰:「倭旦夕且平。胡總督才足辦,南中人短之,何也?」景淳正色曰:「相公遙度之耳。景淳自南來,目睹倭患。胡君坐擁十萬師,南中人不得一安枕臥。相公不欲聞,誰為言者?」嵩愕然謝之。歷侍讀學士,掌院事。改太常卿,領南京祭酒事,就遷吏部右侍郎。隆慶元年,召為禮部左侍郎。用總校《永樂大典》勞,兼翰林院學士,支二品俸,侍經筵,修《嘉靖實錄》。疾作,累疏乞骸骨歸。踰年卒。贈禮部尚書,謚文懿。

為編修時,典制誥。錦衣陸炳先後四妻,欲封最後者,屬景淳撰詞,不可。介嚴嵩為請,亦不應。橐金以投,卒笑謝之。

子汝稷、汝說。汝稷字元立。好學,工屬文,以陰補官。三遷刑部主事。扶溝知縣抶宗人,神宗令予重比。汝稷曰:「是微服至邑庭,官自抶扶溝民耳。」讞上,竟得釋。歷黃州知府,徙邵武,再守辰州。永順土司彭元錦助其弟保靖土司象坤,與酉陽冉躍龍相仇殺。汝稷馳檄元錦解兵去,三土司皆安。尋遷長蘆鹽運使,以太僕少卿致仕。尋卒。

汝說字星卿。五歲而孤。構文成,輒跪薦父木主前。萬曆中舉進士,官至湖廣提學僉事。亦以剛正聞。子式耜,別有傳。

田一俊,字德萬,大田人。隆慶二年會試第一。選庶吉士,授編修,進侍講。萬歷五年,吳中行攻張居正奪情,趙用賢等繼之,居正怒不測。一俊偕侍講趙志皋、修撰沈懋學等疏救,格不入。乃會王錫爵等詣居正,陳大義。一俊詞尤峻,居正心嗛之。未幾,志皋等皆逐,一俊先請告歸,獲免。居正歿,起故官。屢遷禮部左侍郎,掌翰林院。辭疾歸,未行卒。一俊禔身嚴苦,家無贏貲。贈禮部尚書。

懋學,字君典,宣城人。父寵,字畏思。嘉靖中舉鄉試,授行唐知縣。以民不諳織紝,置機杼教之。調獲鹿,徵授御史,官至廣西參議。師貢安國、歐陽德,又從王畿、錢德洪游。知府羅汝芳創講會,御史耿定向聘寵與梅守德共主其席。懋學少有才名。舉萬曆五年進士第一,授修撰。居正子嗣修,其同年生也。疏既格不入,乃三貽書勸嗣修諫,嗣修不能用。以工部尚書李幼滋與居正善,復貽書為言。幼滋報曰:「若所言,宋人腐語,趙氏所以不競也。張公不奔喪,與揖讓征誅,並得聖賢中道,賢儒安足知之。」幼滋初講學,盜虛名,至是縉紳不與焉。懋學遂引疾歸。居數年,卒。福王時,追謚文節。

從孫壽民,字眉生,為諸生有聲。崇禎九年,行保舉法,巡撫張國維以壽民應詔。甫入都,疏劾兵部尚書楊嗣昌奪情。復攻總督熊文燦,言:「嗣昌挈軍旅權,付文燦兵十二萬,餉二百八十餘萬。使賊面縛輿櫬,猶應宣布皇威,而後待以不死;今乃講盟結約,若與國然。天下有授柄於賊而能制賊者乎?」通政張紹先寢不上。壽民以書責,紹先乃請上裁,嗣昌皇恐待罪。帝以疏違式,命勿進。壽民遂隱括兩疏上之,留中。少詹事黃道周歎曰:「此何等事,在朝者不言而草野言之,吾輩媿死矣。」後道周及何楷等相繼抗疏,要自壽民發之。壽民名動天下。未幾移疾去,講學姑山,從游者數百人。福王時,阮大鋮用事,銜壽民劾嗣昌疏有「大鋮妄陳條畫,鼓煽豐芑」語,必欲殺之。壽民乃變姓名避之金華山。國變乃歸,不復出。

黃鳳翔,字鳴周,晉江人。隆慶二年進士及第,授編修。教習內書堂,輯前史宦官行事可為鑒戒者,令誦習之。《世宗實錄》成,進修撰。萬曆五年,張居正奪情,杖諸諫者。鳳翔不平,誦言於朝,編纂章奏,盡載諸諫疏。及居正二子會試,示意,鳳翔峻卻之。當主南畿試,以王篆欲私其子,復謝不往。屢遷南京國子祭酒。省母歸,起補北監。時方較刻《十三經註疏》,鳳翔言:「頃陛下去《貞觀政要》,進講《禮經》,甚善。陛下讀曾子論孝曰敬父母遺體,則當思珍護聖躬。誦《學記》言學然後知不足,則當思緝熙聖學。察《月令》篇以四時敷政、法天行健,則可見聖治之當勤勵。繹《世子》篇陳保傅之教、齒學之儀,則可見皇儲之當早建豫教。」疏入,報聞。

尋擢禮部右侍郎。洮、河告警,抗疏言:「多事之秋,陛下宜屏游宴,親政事,以實圖安攘。為今大計,惟用人、理財二端。宋臣有言:『平居無極言敢諫之臣,則臨難無敵愾致命之士。』鄒元標直聲勁節,銓司特擬召用。其他建言遷謫,如潘士藻、孫如法亦擬量移,而疏皆中寢。士氣日摧,言路日塞。平居只懷祿養交,臨難孰肯捐軀為國家盡力哉?昔宋藝祖欲積縑二百萬易遼人首,太宗移內藏上供物為用兵養士之資。今戶部歲進二十萬,初非舊額,積成常供。陛下富有四海,奈何自營私蓄!竊見都城寺觀,丹碧熒煌,梵剎之供奉,齋醮之祈禳,何一不糜內帑。與其要福於冥漠之鬼神,孰若廣施於孑遺之赤子。」帝不能用。廷臣爭建儲,久未得命,帝諭閣臣以明春舉行。大學士王家屏出語禮部,鳳翔與尚書于慎行、左侍郎李長春以冊立儀上。帝怒,俱奪俸,意復變。鳳翔又疏爭,不報,遂請告去。二十年,禮部左侍郎韓世能去,張一桂未任而卒,復起鳳翔代之。尋改吏部,拜南京禮部尚書。以養親歸。再起故官,力以親老辭。久之母卒,遂不出,卒於家。天啟初,謚文簡。

世能,字存良,長洲人。鳳翔同年進士。由庶吉士授編修。與修世宗、穆宗寶實《錄》,充經筵日講官。歷侍讀、祭酒、禮部侍郎、教習庶吉士。館閣文字,是科為最盛。世能嘗使朝鮮,贈遺一無所受。

餘繼登,字世用,交河人。萬曆五年進士。改庶吉士,授檢討。與修《會典》成,進修撰,直講經筵。尋進右中允,充日講官。時講筵久輟,侍臣無所納忠。繼登與同官馮琦共進《通鑑》講義,傅以時政缺失。歷少詹事兼侍讀學士,充正史副總裁。已,擢詹事,掌翰林院。兩宮災,偕諸講官引《洪範五行傳》切諫。不報。進禮部右侍郎。二十六年,以左侍郎攝部事。陜西、山西地震,南都雷火,西寧鐘自鳴,紹興地湧血。繼登於歲終類奏,因請罷一切誅求開採之害民者。時不能用。雷擊太廟樹,復請帝躬郊祀、廟享,冊立元子,停礦稅,撤中使。帝優詔報聞而已。

旋擢本部尚書。時將討播州楊應龍。繼登請罷四川礦稅,以佐兵食。復上言:「頃者星躔失度,水旱為沴,太白晝見,天不和也。鑿山開礦,裂地求砂,致狄道山崩地震,地不和也。閭閻窮困,更加誅求,帑藏空虛,復責珠寶,奸民蟻聚,中使鴟張,中外壅隔,上下不交,人不和也。戾氣凝而不散,怨毒結而成形,陵谷變遷,高卑易位,是為陰乘陽、邪干正、下叛上之象。臣子不能感動君父,言愈數愈厭,故天以非常之變,警悟陛下,尚可恬然不為意乎?」帝不省。繼登自署部事,請元子冊立冠婚。疏累上,以不得請,鬱鬱成疾。每言及,輒流涕曰:「大禮不舉,吾禮官死不瞑目!」病滿三月,連章乞休,不許。請停俸,亦不許。竟卒於官。贈太子少保,謚文恪。

繼登樸直慎密,寡言笑。當大事,言議侃侃。居家廉約。學士曾朝節嘗過其里,蓬蒿滿徑。及病革,視之,擁粗布衾,羊毳覆足而已。幼子應諸生試,夫人請為一言,終不可。

馮琦,字用韞,臨朐人。幼穎敏絕人。年十九,舉萬曆五年進士,改庶吉士,授編修。預修《會典》成,進侍講,充日講官,歷庶子。三王並封議起,移書王錫爵力爭之。進少詹事,掌翰林院事。遷禮部右侍郎,改吏部。蒞政勤敏,力抑營競,尚書李戴倚重之。

二十七年九月,太白、太陰同見於午;又狄道山崩,平地湧大小山五。琦草疏,偕尚書戴上言:

近見太陰經天,太白晝見,已為極異。至山陷成谷,地湧成山,則自開闢以來,惟唐垂拱中有之,而今再見也。竊惟上天無私,惟民是聽。欲承天意,當順民心。比來天下賦額,視二十年以前,十增其四。而民戶殷足者,則十減其五。東征西討,蕭然苦兵。自礦稅使出,而民間之苦更甚。加以水旱蝗災,流離載道,畿輔近地,盜賊公行,此非細故也。諸中使銜命而出,所隨奸徒,動以千百。陛下欲通商,而彼專困商;陛下欲愛民,而彼專害民。蓋近日神奸有二:其一工伺上意,具有成奏,假武弁上之;其一務剝小民,畫有成謀,假中官行之。運機如鬼蜮,取財盡錙銖。遠近同嗟,貧富交困。貧者家無儲蓄,惟恃經營。但奪其數錢之利,已絕其一日之生。至於富民,更蒙毒害。或陷以漏稅竊礦,或誣之販鹽盜木。布成詭計,聲勢赫然。及其得財,寂然無事。小民累足屏息,無地得容。利歸群奸,怨萃朝宁。夫以刺骨之窮,抱傷心之痛,一呼則易動,一動則難安。今日猶承平,民已洶洶,脫有風塵之警,天下誰可保信者?夫哱拜誅,關白死,此皆募民丁以為兵,用民財以為餉。若一方窮民倡亂,而四面應之,於何徵兵,於何取餉哉!陛下試遣忠實親信之人,採訪都城內外,閭巷歌謠,令一一聞奏,則民之怨苦,居然可睹。天心仁愛,明示咎徵,誠欲陛下翻然改悟,坐弭禍亂。乃禮部修省之章未蒙批答,而奸民搜括之奏又見允行。如納何其賢妄說,令遍解天下無礙官銀。夫四方錢穀,皆有定額,無礙云者,意蓋指經費羨餘。近者徵調頻仍,正額猶逋,何從得羨?此令一下,趣督嚴急,必將分公帑以充獻。經費罔措,還派民間,此事之必不可者也。又如仇世亨奏徐鼐掘墳一事,以理而論,烏有一墓藏黃金巨萬者?借使有之,亦當下撫按核勘。先正其盜墓之罪,而後沒墓中之藏。未有罪狀未明,而先沒入貲財者也。片紙朝入,嚴命夕傳,縱抱深冤,誰敢辨理?不但破此諸族,又將延禍多人。但有株連,立見敗滅。輦轂之下,尚須三覆,萬里之外,止據單詞,遂令狡猾之流,操生殺之柄。此風一倡,孰不效尤?已同告緡之令,又開告密之端。臣等方欲陳訴,而奸人之奏又得旨矣。五日之內,搜取天下公私金銀已二百萬。奸內生奸,例外創例。臣等前猶望其日減,今更患其日增,不至民困財殫激大亂不止。伏望陛下穆然遠覽,亟與廷臣共圖修弭,無令海內赤子,結怨熙朝,千秋青史,貽譏聖德。

不報。

尋轉左侍郎,拜禮部尚書。帝將冊立東宮,詔下期迫,中官掌司設監者以供費不給為詞。琦曰:「今日禮為重,不可與爭。」其弟戶部主事瑗適輦餉銀四萬出都,琦立追還,給費,事乃克濟。

三十年,帝有疾,諭停礦稅,既而悔之。琦與同列合疏爭,且請躬郊廟祭享,御殿受朝,不納。湖廣稅監陳奉以虐民撤還,會陜西黃河竭,琦言遼東高淮、山東陳增、廣東李鳳、陜西梁永、雲南楊榮,肆虐不減於奉,並乞征還,皆不報。南京守備中官邢隆請別給關防征稅,琦不可,乃以御前牙關防給之。

時士大夫多崇釋氏教,士子作文,每竊其緒言,鄙棄傳注。前尚書餘繼登奏請約禁,然習尚如故。琦乃復極陳其弊,帝為下詔戒厲。

琦明習典故,學有根柢。數陳讜論,中外想望豐采,帝亦深眷倚。內閣缺人,帝已簡用朱國祚及琦。而沈一貫密揭,言二人年未及艾,蓋少需之,先用老成者。乃改命沈鯉、朱賡。琦素善病,至是篤。十六疏乞休,不允。卒於官,年僅四十六。遺疏請厲明作,發章奏,補缺官,推誠接下,收拾人心。語極懇摯。帝悼惜之。贈太子少保。天啟初,謚文敏。

自琦曾祖裕以下,累世皆進士。裕,字伯順,以戍籍生於遼東。師事賀欽,有學行。終雲南副使。祖惟重,行人。父子履,河南參政。從祖惟健,舉人;惟訥,字汝言,江西左布政使,加光祿卿致仕。惟重、惟健、惟訥皆有文名,惟訥最著。

惟健子子咸,字受甫。少孤,事母孝。母疾,不解衣者踰年。母歿,哀毀骨立。萬曆元年舉於鄉。再會試不第,遂不復赴。講求濂、洛之學,嘗曰:「為學須剛與恆。不剛則隳,不恒則退。」治家宗《顏氏家訓》。鐘羽正稱「子咸信道忘仕則漆雕子,循經蹈古則高子羔」云。

王圖,字則之,耀州人。萬歷十一年進士。改庶吉士,授檢討,以右中允掌南京翰林院事。召充東宮講官。「妖書」事起,沈一貫欲有所羅織,圖其教習門生也,盡言規之。累遷詹事,充日講官,教習庶吉士。進吏部右侍郎,掌翰林院。兄國方巡撫保定,廷臣附東林及李三才者,往往推轂圖兄弟。會孫丕揚起掌吏部,孫瑋以尚書督倉場,皆陜西人,諸不悅圖者,目為秦黨。而是時郭正域、劉曰寧及圖並有相望。正域逐去,曰寧卒,時論益歸圖。葉向高獨相久,圖旦夕且入閣,忌者益眾。適將京察,惡東林及李三才、王元翰者,設詞惑丕揚,令發單咨是非,將陰為鉤黨計。圖急言於丕揚,止之,群小大恨。初,圖典庚戌會試。分校官湯賓尹欲私韓敬,與知貢舉吳道南盛氣相詬誶。比出闈,道南欲劾,以圖沮而止。王紹徽者,圖同郡人,賓尹門生也,極譽賓尹於圖,而言道南黨欲傾賓尹并及圖,宜善為計。圖正色卻之,紹徽怫然去。時賓尹已為祭酒,其先歷翰林京察,當圖注考,思先發傾之。乃與紹徽計。令御史金明時劾圖子寶坻知縣淑抃贓私巨萬。且謂國素疾李三才,圖為求解,國怒詈之,圖遂欲以拾遺去國。國兄弟抗章力辯,忌者復偽為淑抃劾國疏,播之邸抄。圖上疏言狀,帝為下詔購捕,乃已。及考察,卒注賓尹不謹,褫其官,明時亦被黜。由是其黨大噪。秦聚、奎朱一桂、鄭繼芳、徐兆魁、高節、王萬祚、曾陳易輩,連章力攻圖。圖亦連章求去,出郊待命。溫詔屢慰留,堅臥不起,九閱月始予告歸。國亦乞休去,未幾卒。四十五年京察,當事者多賓尹、紹徽黨,以拾遺落圖職。天啟三年,召起故官。進禮部尚書,協理詹事府。明年,魏忠賢黨劉弘先劾圖,遂削籍。尋卒。崇禎初,贈太子太保,謚文肅。淑抃終戶部郎中。

劉曰寧,字幼安,南昌人。萬歷十七年進士。改庶吉士,授編修。進右中允,直皇長子講幄。時冊立未舉,外議紛紜。曰寧旁慰曲喻,依於仁孝,光宗心識之。礦使四出,曰寧發憤上疏,陳六疑四患,極言稅監李道、王朝諸不法狀。疏入,留中。以母病歸。起右諭德,掌南京翰林院,就遷國子祭酒。奉母歸,吏進贏金數千,曰「例也」,曰寧峻卻之。尋起少詹事,母喪不赴。服闋,召為禮部右侍郎,協理詹事府。道卒。贈禮部尚書。天啟初,追謚文簡。

翁正春,字兆震,侯官人。萬曆中,為龍溪教諭。二十年,擢進士第一,授修撰,累遷少詹事。三十八年九月,拜禮部左侍郎,代吳道南署部事。十一月,日有食之,正春極言闕失,不報。明年秋,萬壽節,正春獻八箴:曰清君心,遵祖制,振國紀,信臣僚,寶賢才,謹財用,恤民命,重邊防。帝不省。吉王翊鑾請封支子常源為郡王。正春言翊鑾之封在《宗籓條例》已定之後,其支庶宜止本爵。乃授鎮國將軍。王貴妃薨,久不卜葬,正春以為言。命偕中官往擇地,得吉。中官難以煩費,正春勃然曰:「貴妃誕育元良,他日國母也,奈何以天下儉乎?」奏上,報可。代王欲廢長子鼎渭,立次子鼎莎,朝議持二十餘年。正春集眾議上疏,鼎渭卒得立。琉球中山王遣使入貢,正春言:「中山已入於倭,今使臣多倭人,貢物多倭器,絕之便;否亦宜詔福建撫臣量留土物,毋俾入朝。」帝是之。

四十年,進士鄒之麟分校鄉試,私舉子童學賢,為御史馬孟禎等所發。正春議黜學賢,謫之麟,而不及主考官。給事中趙興邦、亓詩教因劾正春徇私。正春求去,不許。頃之,言官發湯賓尹、韓敬科場事。正春坐敬不謹,敬黨大恨。詩教復劾正春,正春疏辯,益求去。帝雖慰留,然自是不安其位。尋改吏部,掌詹事府,以侍養歸。天啟元年,起禮部尚書,協理詹事府事。抗論忤魏忠賢,被旨譙責。明年,御史趙胤昌希指劾之,正春再疏乞歸。帝以正春嘗為皇祖講官,特加太子少保,賜敕馳傳,異數也。時正春年逾七十,母百歲,率子孫奉觴上壽,鄉閭艷之。未幾,卒。崇禎初,謚文簡。

正春風度峻整,終日無狎語。倦不傾倚,暑不裸裎,目無流視。見者肅然。明一代,科目職官冠廷對者二人;鼐以典史,正春以教諭云。

劉應秋,字士和,吉水人。萬曆十一年進士及第,授編修,遷南京司業。十八年冬,疏論首輔申時行言:「陛下召對輔臣,諮以邊事,時行不能抒誠謀國,專事蒙蔽。賊大舉入犯,既掠洮、岷,直迫臨、鞏,覆軍殺將,頻至喪敗,而時行猶曰『掠番』,曰『聲言入寇』,豈洮、河以內,盡皆番地乎?輔臣者,天子所與託腹心者也。輔臣先蒙蔽,何責庶僚?故近日敵情有按臣疏而督撫不以聞者,有督撫聞而樞臣不以奏者。彼習見執政大臣喜聞捷而惡言敗,故內外相蒙,恬不為怪。欺蔽之端,自輔臣始。夫士風高下,關乎氣運,說者謂嘉靖至今,士風三變。一變環境嚴嵩之黷賄,而士化為貪。再變於張居正之專擅,而士競於險。至於今,外逃貪黷之名,而頑夫債帥多出門下;陽避專擅之迹,而芒刃斧斤倒持手中。威福之權,潛移其向;愛憎之的,明示之趨。欲天下無靡,不可得也。」語并侵次輔王錫爵。時主事蔡時鼎、南京御史章守誠亦疏論時行。並留中。應秋尋召為中允,充日講官。歷右庶子、祭酒。

二十六年,有撰《憂危竑議》者,御史趙之翰以指大學士張位,并及應秋。所司言應秋非位黨,宜留。帝命調外,應秋遂辭疾歸。初,御史黃卷索珠商徐性善賕,不盡應,上章籍沒之。應秋詈卷啟天子好利之端。男子諸龍光奏訐李如松,至荷枷大暑中。應秋言一妄人上書,何必置死地。時詞臣率優游養望,應秋獨好議評時事,以此取忌,竟被黜。歸數年,卒。崇禎時,贈禮部侍郎,謚文節。

子同升,字晉卿。師同里鄒元標。崇禎十年,殿試第一。莊烈帝問年幾何,對曰:「五十有一。」帝曰:「若尚如少年,勉之。」授翰林修撰。楊嗣昌奪情入閣,何楷、林蘭友、黃道周言之俱獲罪,同升抗疏言:「日者策試諸臣,簡用嗣昌,良以中外交訌,冀得一效,拯我蒼生。聖明用心,亦甚苦矣。都人籍籍,謂嗣昌縗絰在身,且入閣非金革比。臣以嗣昌必且哀痛惻怛,上告君父,辭免綸扉;乃循例再疏,遽入辦事。夫人有所不忍,而後能及其所忍;有所不為,而後可以有為。臣以嗣昌所忍,覘其所為,知嗣昌心失智短,必不能為國建功,何也?成天下之事在乎志,勝天下之任在乎氣;志敗氣餒,而能任天下事,必無是理。伎倆已窮,茍且富貴。兼樞部以重綸扉之權,借綸扉為解樞部之漸。和議自專,票擬由己。與方一藻、高起潛輩扶同罔功,掩敗為勝。歲糜金繒,養患邊圉。立心如此,獨不畏堯、舜在上乎?曩自陛下切責議和,而嗣昌不可以為臣。今一旦忽易墨縗,而嗣昌不可以為子。若附和黨比,緘口全軀,嗣昌得罪名教,臣亦得罪名教矣。」疏入,帝大怒,謫福建按察司知事。移疾歸。廷臣屢薦,將召用,而京師陷。福王立,召起故官,不赴。明年五月,南都不守,江西郡縣多失。同升攜家將入福建,止雩都,與楊廷麟謀興復。唐王加同升祭酒。同升乃入贛州,偕廷麟籌兵食。取吉安、臨江,加詹事兼兵部左侍郎。同升已羸疾,日與士大夫講忠孝大節,聞者咸奮,以廷麟請,撫南、贛,十二,月卒於贛州。

唐文獻,字元徵,華亭人。萬曆十四年進士第一。授修撰,歷詹事。

沈一貫以「妖書」事傾尚書郭正域,持之急。文獻偕其僚楊道賓、周如砥、陶望齡往見一貫曰:「郭公將不免,人謂公實有意殺之。」一貫踞跼艴,酹地若為誓者。文獻曰:「亦知公無意殺之也,第臺省承風下石,而公不早訖此獄,何辭以謝天下。」一貫斂容謝之。望齡見朱賡不為救,亦正色責以大義,願棄官與正域同死。獄得稍解。然文獻等以是失政府意。久之,拜禮部右侍郎,掌翰林院事。初,文獻出趙用賢門,以名節相矜許。同年生給事中李沂劾張鯨被廷杖,文獻掖之出,資給其湯藥。荊州推官華鈺忤稅監逮下詔獄,文獻力周旋,得無死。掌翰林日,當考察,執政欲庇一人,執不許。卒官。贈禮部尚書,謚文恪。

楊道賓,字惟彥,晉江人。萬曆十四年進士第二,授編修。累遷國子祭酒,少詹事,禮部右侍郎,掌翰林院事。轉左,改掌部事。嘗因星變,請釋逮繫知縣滿朝薦等,又請亟舉朝講大典,皆不報。南京大水,疏陳時政,略言:「宮中夜分方寢,日旰未起,致萬幾怠曠。請夙興夜寐,以圖治功。時御便殿,與大臣面決大政。章疏及時批答,毋輒留中及從內降。」帝優旨報聞。皇太子輟講已四年,道賓極諫,引唐宦官仇士良語為戒。其冬,天鼓鳴,道賓言:「天之視聽在民。今民生顛躓,無所赴愬,天若代為之鳴。宜急罷礦使,更張闕政,以和民心。」帝不聽。踰年卒官。贈禮部尚書,謚文恪。

陶望齡,字周望,會稽人。父承學,南京禮部尚書。望齡少有文名。舉萬曆十七年會試第一,殿試一甲第三,授編修,歷官國子祭酒。篤嗜王守仁說,所宗者周汝登。與弟奭齡皆以講學名。卒謚文簡。

李騰芳,字子實,湘潭人。萬曆二十年進士。改庶吉士。好學,負才名。三王並封旨下,騰芳為書詣朝房投大學士王錫爵略言:「公欲暫承上意,巧借封王,轉作冊立。然恐王封既定,大典愈遲。他日公去而事壞,罪公始謀,何辭以解?此不獨宗社憂,亦公子孫禍也。」錫爵讀未竟,遽牽衣命坐,曰:「諸人詈我,我何以自明?如子言,我受教。但我疏必親書,謂子孫禍何也?」騰芳曰:「外廷正以公手書密揭,無由知其詳,公乃欲藉以自解。異日能使天子出公手書示天下乎?」錫爵憮然淚下,明日遂反並封之詔。

屢遷左諭德。騰芳與{山崑山顧天颭善。天颭險詖無行,為世所指名,被劾去,騰芳亦投劾歸。時遂有顧黨、李黨之目。詔論朝士擅去者罪,貶騰芳太常博士。三十九年京察,復以浮躁謫江西都司理問。稍遷行人司正,歷太常少卿,掌司業事。光宗立,擢少詹事,署南京翰林院。旋拜禮部右侍郎,教習庶吉士。御史王安舜劾騰芳驟遷。騰芳辭位,熹宗不許,竟以省母歸。天啟初,以故官協理詹事府,尋改吏部左侍郎。丁內艱,加禮部尚書以歸。魏忠賢惡騰芳與楊漣同鄉。御史王際逵因論騰芳被察驟起,丁憂進官,皆非制。遂削奪。崇禎初,再以尚書協理詹事府。京師戒嚴,條畫守禦,多稱旨,代何如寵掌部事。卒官。贈太子太保。蔡毅中,字宏甫,光山人。祖鳳翹,平陽同知。父光,臨洮同知。毅中五歲通《孝經》。父問:「讀書何為?」對曰:「欲為聖賢耳。」萬曆二十九年第進士,改庶吉士,授檢討。時礦稅虐民,毅中取《祖訓》、《會典》諸書禁戒礦稅者,集為二卷,注釋以上。大學士沈鯉於毅中為鄉先達,與首輔沈一貫不相能。而溫純參政河南,器毅中於諸生。至是為都御史,疏侵一貫。一貫疑出毅中手,為鯉地,銜之,遂用計典,鐫秩去。起麻城丞。旋以行人司副召擢尚寶丞。移疾歸。四十五年,以浮躁鐫秩。天啟初,大起廢籍,補長蘆鹽運判官。屢遷國子祭酒,擢禮部右侍郎,仍領祭酒事。楊漣劾魏忠賢得嚴旨,毅中率其屬抗疏言:

學校者,天下公議所從出也。臣正與諸生講「為君難」一書,忽接楊漣劾忠賢疏,合監師生千有餘人,無不鼓掌稱慶。乃皇上不下其奏於九卿,而謂一切朝政皆親裁,以奸璫為忠,代之受過,合監師生無不捫心悉歎不已也。臣惟三代以後,漢、隋、唐、宋諸君,其受權璫之害與處權璫之法,載在《通鑒》。我朝列聖受權璫之害與處權璫之法,載在實錄。臣皆不必多言。但取至近至親如武宗之處劉瑾、神宗之處馮保二事,願皇上遵之。瑾在武宗左右,言聽計從,一聞諸臣劾奏,夜半自起,擒而殺之。神宗臨御方十齡,保左右扶持,盡心竭力。既而少作威福,臺省劾奏,未聞舉朝公疏,神祖遂不動聲色而戍保於南京。今忠賢無保之功,而極瑾之惡。二十四罪,無一不當悉究。舉朝群臣欲於朝罷,跪以候旨,忠賢遂要皇上入宮,不禮群臣。今又欲於視學之日,群臣及太學諸生面叩陳請矣,而皇上漫不經意。數日以來,但有及忠賢者,留中不發,如此蒙蔽,其中寧可測哉!乞將漣疏發九卿科道從公究問,即不加劉瑾之誅,而以處馮保之法懲之,則恩威並著,與神祖媲美矣。

疏入,忠賢戟手大訽。毅中乃再疏乞歸,不許。已,嗾其黨劾罷之。

毅中有至性。四歲父病,籲天請代。公車時,聞母喪,一慟嘔血數升,終喪斷酒肉,不入內寢。方母病,盛夏思冰,盂水忽凍。廬居,有紫芝、白鳥、千鴉集墓之異。卒,贈禮部尚書。

公鼐,字孝與,蒙陰人。曾祖奎躋,湖廣副使。父家臣,翰林編修。鼐舉萬二十上九年進士,改庶吉士,授編修。屢遷左諭德,為東宮講官。進左庶子,引疾歸。光宗立,召拜祭酒。熹宗進鼐詹事,乃上疏曰:「近聞南北臣僚,論先帝升遐一事,跡涉怪異,語多隱藏。恐因委巷之訛傳,流為湘山之稗說,臣竊痛焉。皇祖在昔,原無立愛之心。只因大典遲回,於是繳還冊立之後,有三王並封之事,《憂危竑議》之後,有國本攸關之事。迨龐、劉之邪謀,張差之梃擊,而逆亂極矣。臣嘗備員宮僚,目睹狂謀孔熾,以歸向東宮者為小人,不向東宮者為君子,盡除朝士之清流,陰翦元良之羽翼,批根引蔓,干紀亂常。至今追想,猶為寒心。夫臣子愛君,存其真不存其偽。今實錄纂修在即,請將光宗事蹟,別為一錄。凡一月間明綸善政,固大書特書;其有聞見異詞及宮闈委曲之妙用,亦皆直筆指陳,勒成信史。臣雖不肖,竊敢任之。」疏入,不許。天啟元年,鼐以紀元甫及半載,言官獲譴者至十餘人,上疏切諫,并規諷輔臣。忤旨,譙責。尋遷禮部右侍郎,協理詹事府,充實錄副總裁。鼐好學博聞,磊落有器識。見魏忠賢亂政,引疾歸。

初,廷議李三才起用不決,鼐颺言曰:「今封疆倚重者,多遠道未至。三才猷略素優,家近輦轂,可朝發夕至也。」侍郎鄒元標趣使盡言,以言路相持而止。後御史葉有聲追論鼐與三才為姻,徇私妄薦,遂落職閒住。未幾卒。崇禎初,復官賜恤,謚文介。

羅喻義,字湘中,益陽人。萬曆四十一年進士。改庶吉士,授檢討。請假歸。天啟初還朝,歷官諭德,直經筵。六年擢南京國子祭酒。諸生欲為魏忠賢建祠,喻義懲其倡者,乃已。忠賢黨輯東林籍貫,湖廣二十人,以喻義為首。莊烈帝嗣位,召拜禮部右侍郎,協理詹事府。尋充日講官,教習庶吉士。

喻義性嚴冷,閉戶讀書,不輕接一客。後見中外多故,將吏不習兵,銳意講武事,推演陣圖獻之。帝為褒納。以時方用兵,而督撫大吏不立軍府,財用無所資,因言:「武有七德,豐財居其一。正餉之外,宜別立軍府,朝廷勿預知。饗士、賞功、購敵,皆取給於是。」又極陳車戰之利。帝下軍府議於所司,令喻義自製戰車。喻義復上言按畝加派之害,而以戰車營造職在有司,不肯奉詔。帝不悅,疏遂不行。

明年九月,進講《尚書》,撰《布昭聖武講義》。中及時事,有「左右之者不得其人」語,頗傷執政;末陳祖宗大閱之規,京營之制,冀有所興革。呈稿政府,溫體仁不懌,使正字官語喻義,令改。喻義造閣中,隔扉誚體仁。體仁怒,上言:「故事,惟經筵進規,多於正講,目講則正多規少。今喻義以日講而用經筵之制,及令刪改,反遭其侮,惟聖明裁察。」遂下吏部議。喻義奏辨曰:「講官於正文外旁及時事,亦舊制也。臣展轉敷陳,冀少有裨益。體仁刪去,臣誠恐愚忠不獲上達,致忤輔臣。今稿草具在,望聖明省覽。」吏部希體仁指,議革職閒住,可之。喻義雅負時望,為體仁所傾,士論交惜。瀕行乞恩,請乘傳,帝亦報可。家居十年,卒。

姚希孟,字孟長,吳縣人。生十月而孤,母文氏勵志鞠之。稍長,與舅文震孟同學,並負時名。舉萬曆四十七年進士,改庶吉士。座主韓爌、館師劉一景器之。兩人並執政,遇大事多所咨決。天啟初,震孟亦取上第,入翰林,甥舅並持清議,望益重。尋請假歸。四年冬還朝,趙南星、高攀龍等悉去位,黨禍大作,希孟鬱鬱不得志。其明年,以母喪歸。甫出都,給事中楊所修劾其為繆昌期死黨,遂削籍。魏忠賢敗,其黨倪文煥懼誅,使使持厚賄求解,希孟執而鳴之官。崇禎元年,起左贊善。歷右庶子,為日講官。三年秋,與諭德姚明恭主順天鄉試。有武生二人冒籍中式,給事中王猷論之,遂獲譴。希孟雅為東林所推。韓爌等定逆案,參其議。群小惡希孟,謀先之。及華允誠劾溫體仁、閔洪學,兩人疑疏出希孟手,體仁遂借冒籍事修隙,擬旨覆試,黜兩生下所司,論考官罪,擬停俸半年。體仁意未慊,令再擬。希孟時已遷詹事,乃貶二秩為少詹事,掌南京翰林院。尋移疾歸,家居二年,卒。

許士柔,字仲嘉,常熟人。天啟二年進士。改庶吉士,授檢討。崇禎時,歷遷左庶子,掌左春坊事。先是,魏忠賢既輯《三朝要典》,以《光宗實錄》所載與《要典》左,乃言葉向高等所修非實,宜重修,遂恣意改削牴牾《要典》者。崇禎改元,毀《要典》而所改《光宗實錄》如故。六年,少詹事文震孟言:「皇考實錄為魏黨曲筆,當改正從原錄。」時溫體仁當國,與王應熊等陰沮之,事遂寢。士柔憤然曰:「若是,則《要典》猶弗焚矣。」乃上疏曰:「皇考實錄總記,於世系獨略。皇上娠教之年,聖誕之日,不書也。命名之典,潛邸之號,不書也。聖母出何氏族,受何封號,不書也。此皆原錄備載,而改錄故削之者也。原錄之成,在皇上潛邸之日,猶詳慎如彼。新錄之進,在皇上御極之初,何以率略如此,使聖朝父子、母后、兄弟之大倫,皆暗而不明,缺而莫考。其於信史謂何?」疏上,不省。體仁令中書官檢穆宗總記示士柔,士柔具揭爭之曰:「皇考實錄與列聖條例不同。列聖在位久,登極後事,編年排纂,則總記可以不書。皇考在位僅一月,三后誕育聖躬皆在未登極以前,不書之總記,將於何書也?穆廟大婚之禮,皇子之生,在嘉靖中,故總記不載,至於冊立大典,編年未嘗不具載也。皇考一月易世,熹廟之冊立當書,皇上之冊封獨不當書乎?」體仁怒,將劾之,為同列沮止。士柔復上疏曰:「累朝實錄,無不書世系之例。臣所以抉擿改錄,正謂與累朝成例不合也。孝端皇后,皇考之嫡母也,原錄具書保護之功,而改錄削之,何也?當日國本幾危,坤寧調護,真孝慈之極則,顧復之深恩,史官不難以寸管抹摋之,此尤不可解也。」疏上,報聞。

體仁滋不悅。會體仁嗾劉孔昭劾祭酒倪元璐,因言士柔族子重熙私撰《五朝注略》,將以連士柔。士柔亟以《注略》進,乃得解。尋出為南京國子祭酒。

體仁去,張至發當國,益謀逐士柔。先是,高攀龍贈官,士柔草詔詞送內閣,未給攀龍家。故事,贈官誥,屬誥敕中書職掌。崇禎初,褒恤諸忠臣,翰林能文者或為之,而中書以為侵官。崇禎三年禁誥文駢儷語。至是攀龍家請給,去士柔草制時數年矣,主者仍以士柔前撰文進。中書黃應恩告至發誥語違禁,至發喜,劾士柔,降二級調用。司業周鳳翔抗疏辯曰:「詞林故事,閣臣分屬撰文,或手加詳定,或發竄改,未有徑自糾參者也。誥敕用寶,歲有常期,未有十年後用寶進呈,吹求當制者也。贈誥專屬中書,崇禎三年所申飭,未有追咎元年之史官,詆為越俎者也。」不報。士柔尋補尚寶司丞,遷少卿,卒。子琪詣闕辨誣,乃復原官。贈詹事兼侍讀學士。

顧錫疇,字九疇,崑山人。年十三,以諸生試南京,魏國公以女女之。第萬曆四十七年進士,改庶吉士,授檢討。天啟四年,魏忠賢勢大熾,錫疇偕給事中董承業典試福建,程策大有譏刺。忠賢黨遂指為東林,兩人並降調。已,更削籍。

崇禎初,召復故官。歷遷國子祭酒。疏請復積分法,禮官格不行。錫疇復申言之,且請擇監生為州縣長。已,請正從祀位次,進士為國子博士者得與考選。帝並允行。省親歸,乞在籍終養。母服除,起少詹事,進詹事,拜禮部左侍郎,署部事。帝嘗召對,問理財用人。錫疇退,列陳用人五失,曰銓敘無法,文網太峻,議論太多,資格太拘,鼓舞未至。請先令用人之地一清其源。「精心鑒別,隨才器使,一善也。赦小過而不終廢棄,二善也。省議論而專責成,三善也。拔異才而不拘常格,四善也。急獎勵而寬督責,五善也。」末極陳耗財之弊,仍歸本於用人。帝善其奏。

楊嗣昌疏請撫流寇,有「樂天者保天下」及「善戰服上刑」語。錫疇抗言此諸侯交鄰事,稱引不倫,與嗣昌大忤。嗣昌秉政,諸詞臣多攻之,嗣昌頗疑錫疇。會駙馬都尉王昺有罪,錫疇擬輕典,嗣昌構之,遂削其籍。十五年,廷臣交薦,召還。御史曹溶、給事中黃雲師復言其不當用。帝不聽,起為南京禮部左侍郎。

福王立,進本部尚書。時尊福恭王為恭皇帝,將議廟祀,錫疇請別立專廟。俄請補建文帝廟謚、景皇帝廟號及建文朝忠臣贈謚,並從之。東平伯劉澤清言:「宋高宗即位南京,即以靖康二年五月為建炎元年,從民望也。乞以今歲五月為弘光元年。」錫疇言明詔已頒,不可追改,乃已。時定大行皇帝廟號為思宗,忻城伯趙之龍言「思」非美稱,援證甚核,錫疇亦以為然,疏請改定。大學士高弘圖以前議自己出,力持之,遂寢。溫體仁之卒也,特謚文忠,而文震孟、羅喻義、姚希孟、呂維祺皆不獲謚。錫疇言:「體仁得君,行政最專且久,其負先帝,罪大且深,乞將文忠之謚,或削或改,而補震孟諸臣,庶天下有所勸懲。」報可。遂謚諸人,削體仁謚。吏部尚書張慎言去位,代者徐石麒未至,命錫疇攝之。時馬士英當國,錫疇雅不與合。給事中章正宸、熊汝霖劾之,遂乞祭南海去。明年春,御史張孫振力頌體仁功,請復故謚。遂勒錫疇致仕。南都失守,錫疇鄉邑亦破。時方遭父喪,間關赴閩。唐王命以故官,力辭不拜,寓居溫州江心寺。總兵賀君堯撻辱諸生,錫疇將論劾。君堯夜使人殺之,投屍於江。溫人覓之三日,乃得棺殮。

贊曰:吳山等雍容館閣,揚歷臺省,固所謂詞苑之鴻儒,廟堂之巋望也。要其守正自立,不激不爭,淳靜敦雅,承平士大夫之風流,概可想見矣。


\end{pinyinscope}