\article{藝文志}


明太祖定元都,大將軍收圖籍致之南京,復詔求四方遺書,設秘書監丞,尋改翰林典籍以掌之。永樂四年,帝御便殿閱書史,問文淵閣藏書。解縉對以尚多闕略。帝曰:「土庶家稍有餘資,尚欲積書,況朝廷乎?」遂命禮部尚書鄭賜遣使訪購,惟其所欲與之,勿較值。北京既建,詔修撰陳循取文淵閣書一部至百部,各擇其一,得百櫃,運致北京。宣宗嘗臨視文淵閣,親披閱經史,與少傅楊士奇等討論,因賜士奇等詩。是時,秘閣貯書約二萬餘部,近百萬卷,刻本十三,抄本十七。正統間,士奇等言:「文淵閣所貯書籍,有祖宗御製文集及古今經史子集之書,向貯左順門北廊,今移於文淵閣、東閣,臣等逐一點勘,編成書目,請用寶鈐識,永久藏弆。」制曰「可」。正德十年,大學士梁儲等請檢內閣並東閣藏書殘闕者,令原管主事李繼先等次第修補。先是,祕閣書籍皆宋、元所遺,無不精美,裝用倒摺,四周外向,蟲鼠不能損。迄流賊之亂,宋刻元鐫胥歸殘闕。至明御製詩文,內府鏤板,而儒臣奉敕修纂之書及象魏布告之訓,卷帙既夥,文藻復優,當時頒行天下。外此則名公卿之論撰,騷人墨客一家之言,其工者深醇大雅,卓卓可傳。即有怪奇駁雜出乎其間,亦足以考風氣之正變,辨古學之源流,識大識小,掌故備焉。挹其華實,無讓前徽,可不謂文運之盛歟!

四部之目,昉自荀勖,晉、宋以來因之。前史兼錄古今載籍,以為皆其時柱下之所有也。明萬曆中,修撰焦竑修國史,輯《經籍志》,號稱詳博。然延閣廣內之藏,竑亦無從遍覽,則前代陳編,何憑記錄,區區掇拾遺聞,冀以上承《隋志》,而贗書錯列,徒滋訛舛。故今第就二百七十年各家著述,稍為釐次,勒成一志。凡卷數莫考、疑信未定者,寧闕而不詳云。

經類十:一曰《易》類,二曰《書》類,三曰《詩》類,四曰《禮》類,五曰《樂》類,六曰《春秋》類,七曰《孝經》類,八曰諸經類,九曰《四書》類,十曰小學類。

朱升《周易旁注前圖》二卷、《周易旁注》十卷

梁寅《周易參義》十二卷

趙汸《大易文詮》八卷

鮑恂《大易舉隅》三卷又名《大易鉤玄》。

林大同《易經奧義》二卷

歐陽貞《周易問辨》三十卷

朱謐《易學啟蒙述解》二卷

張洪《周易傳義會通》十五卷

程汝器《周易集傳》十卷

永樂中敕修《周易傳義大全》二十四卷、《義例》一卷胡廣等纂。

楊士奇《周易直指》十卷

劉髦《石潭易傳撮要》一卷

林言志《周易集說》三卷

李賢《讀易記》一卷

劉定之《周易圖釋》三卷

王恕《玩易意見》二卷

羅倫《周易說旨》四卷

談綱《讀易愚慮》二卷,《易攷圖義》一卷,《卜筮節要》一卷,《易義雜言》一卷,《易指攷辨》一卷

蔡清《周易蒙引》二十四卷

朱綬《易經精蘊》二十四卷

何孟春《易疑初筮告蒙約》十二卷

胡世寧《讀易私記》四卷

陳鳳梧《集定古易》十二卷

劉玉《執齋易圖說》一卷

許誥《圖書管見》一卷

周用《讀易日記》一卷

崔銑《讀易餘言》五卷,《易大象說》一卷

湛若水《修復古易經傳訓測》十卷

張邦奇《易說》一卷

鄭善夫《易論》一卷

呂柟《周易說翼》三卷

王崇慶《周易義卦》二卷

唐龍《易經大旨》四卷

韓邦奇《易學啟蒙意見》四卷一名《周學疏原》、《易占經緯》四卷

鐘芳《學易疑義》三卷

王道《周易人意》四卷

梅鷟《古易考原》三卷

金賁亨《學易記》五卷

舒芬《易箋問》一卷

季本《易學四同》八卷、《圖文餘辨》一卷、《蓍法別傳》一卷、《古易辨》一卷

林希元《易經存疑》十二卷

陳琛《易經通典》六卷一名《淺說》。

方獻夫《周易約說》十二卷

餘誠《易圖說》一卷

黃芹《易圖識漏》一卷

李舜臣《易卦辱言》一卷

葉良珮《周易義叢》十六卷

豐坊《古易世學》十五卷坊雲家有《古易》,傳自遠祖豐稷。又有《古書世學》六卷,言得朝鮮、倭國二本,合於今文。古文《石經》、古本《魯詩世學》三十六卷,亦言豐稷所傳。錢謙益謂皆坊偽撰也。

唐樞《易修墨守》一卷

羅洪先《易解》一卷

楊爵《周易辨錄》四卷

薛甲《易象大旨》八卷

熊過《周易象旨決錄》七卷

胡經《易演義》十八卷

王畿《大象義述》一卷

盧翰《古易中說》四十四卷

陳言《易疑》四卷

陳士元《易象鉤解》四卷

《易象匯解》二卷

魯邦彥《圖書就正錄》一卷

李贄《九正易因》四卷贄自謂初著《易因》一書,改至八九次而後定,故有「九正」之名。

徐師曾《今文周易演義》十二卷

姜寶《周易補疑》十二卷

顧曾唯《周易詳蘊》十三卷

孫應BI《易談》四卷

鄧元錫《易經繹》五卷

顏鯨《易學義林》十卷

陳錫《易原》一卷

王世懋《易解》一卷

徐元氣《周易詳解》十卷

萬廷言《易說》四卷、《易原》四卷

楊時喬《周易古今文全書》二十一卷

來知德《周易集註》十六卷

任惟賢《周易義訓》十卷

張獻翼《讀易韻考》七卷

曾士傳《正易學啟蒙》一卷

葉山《八白易傳》十六卷

金瑤《六爻原意》一卷

李逢期《易經隨筆》三卷

方社昌《周易指要》三卷

孫從龍《周易參疑》十卷

沈一貫《易學》十二卷

馮時可《易說》五卷

唐鶴徵《周易象義》四卷

黃正憲《易象管窺》十五卷

郭子章《易解》十五卷

吳中立《易銓古本》三卷

周坦《易圖說》一卷

朱篁《易郵》七卷

朱謀韋《易象通》八卷

陳第《伏羲圖贊》二卷

鄧伯羔《古易詮》二十九卷,《今易詮》二十四卷

傅文兆《羲經十一翼》五卷

林兆恩《易外別傳》一卷

王宇《周易占林》四卷

彭好古《易鑰》五卷

方時化《易疑》一卷,《易引》九卷,《周易頌》二卷,《學易述談》四卷

章潢《周易象義》十卷

姚舜牧《易經疑問》十二卷

顏素《易研》六卷

曾朝節《易測》十卷

鄒元標《易彀通》一卷

徐三重《易義》一卷

蘇濬《周易冥冥篇》四卷,《易經兒說》四卷

沈孚聞《周易日鈔》十一卷

屠隆《讀易便解》四卷

楊啟新《易林疑說》二卷

鐘化民《讀易鈔》十四卷

李廷機《易經纂注》四卷、《易答問》四卷

鄒德溥《易會》八卷

錢一本《像象管見》七卷,《易象鈔》、《續鈔》共六卷,《四聖一心錄》四卷

潘士藻《洗心齋讀易述》十七卷

岳元聲《易說》三卷

顧允成《易圖說人意言》四卷

焦竑《易筌》六卷

高攀龍《大易易簡說》三卷,《周易孔義》一卷

郝敬《周易正解》二十卷,《周領》四卷,《周易補》七卷,《學易枝言》二卷

張納陛《學易飲河》八卷

吳炯《周易繹旨》八卷

萬尚烈《易贊測》一卷,《易大象測》一卷

吳默《易說》六卷

姚文蔚《周易旁注會通》十四卷

李本固《古易匯編意辭集》十七卷

楊廷筠《易顯》六卷

湯賓尹《易0經翼註》四卷

孫慎行《周易明洛義纂述》六卷,《不語易義》二卷

曹學牷《周易可說》七卷

張汝霖《周易因指》八卷

崔師訓《大成易旨》二卷

劉宗周《周易古文鈔》三卷,《讀易圖記》一卷

薛三省《易蠡》二卷

程汝繼《周易宗義》十二卷

王三善《周易象注》九卷

魏濬《周易古象通》八卷

樊良樞《易疑》一卷,《易象》二卷

高捷《易學象辭二集》十二卷

陸振奇《易芥》十卷

楊瞿崍《易林疑說》十卷

王納諫《周易翼註》三卷

陸夢龍《易略》三卷

文翔鳳《邵窩易詁》一卷

卓爾康《易學0全書》五十卷

繆昌期《周易會通》十二卷

羅喻義《讀易內篇》、《問篇》、《外篇》共七卷

程玉潤《周易演旨》六十五卷

錢士升《易揆》十二卷

錢繼登《易簣》三卷

吳極《易學》五卷

方孔炤《周易時論》十五卷

徐世淳《易就》六卷

汪邦柱《周易會通》十二卷

葉憲祖《大易玉匙》六卷

方鯤《易盪》二卷

鮑觀白《易說》二卷

張伯樞《易象大旨》三卷

吳桂森《像象述》五卷

鄭維嶽《易經意言》六卷

喻有功《周易懸鏡》七卷

潘士龍《演易圖說》一卷

洪守美《易說醒》四卷

餘叔純《周易讀》五卷

陸起龍《周易易簡編》四卷

徐奇《周易卦義》二卷

洪化昭《周易獨坐談》五卷

沈瑞鐘《周易廣筌》二卷

林有桂《易經觀理說》四卷

陳履祥《孔易彀》一卷

許順義《易經三注粹鈔》四卷

王祚昌《周易敝書》五卷

容若春《今易圖學心法釋義》十卷

張次仲《周易玩辭困學記》十二卷

顧樞《西疇易稿》三卷

陳仁錫《羲經易簡錄》八卷

黃道周《易象正》十四卷,《三易洞璣》十六卷

倪元璐《兒易內儀》六卷、《外儀》十五卷

龍文光《乾乾篇》三卷

文安之《易傭》十四卷

林胤昌《周易耨義》六卷

張鏡心《易經增註》十二卷

李奇玉《易義》四卷

朱之俊《周易纂》六卷

何楷《古周易訂詁》十六卷

侯峒曾《易解》三卷

黎遂球《周易爻物當名》二卷

鄭賡唐《讀易搜》十二卷

陳際泰《易經大意》七卷,《群經輔易說》一卷,《周易翼簡捷解》十六卷

秦鏞《易序圖說》二卷

金鉉《易說》一卷

黃端伯《易疏》五卷

來集之《讀易偶通》二卷

──右《易》類,二百二十二部,一千五百七十卷。

明太祖注《尚書洪範》一卷帝嘗命儒臣書《洪範》,揭於御座之右,因自為注。

仁宗《體尚書》二卷釋《尚書》中《皋陶謨》、《甘誓》、《盤庚》等十六篇,以講解更其原文。

世宗《書經三要》三卷帝以太祖有注《洪範》一篇,因注《無逸》,再注《伊訓》,分三冊,共為一書。已乃制《洪範序略》一篇,復將《皋陶謨》、《伊訓》、《無逸》等篇通加注釋,名曰《書經三要》。

洪武中敕修《書傳會選》六卷太祖以蔡沈《書傳》有得有失,詔劉三吾等訂正之。又集諸家之說,足其未備。書成頒刻,然世竟鮮行。永樂中,修《大全》,一依蔡《傳》,取便於士子舉業,此外不復有所考究也。

朱升《尚書旁注》六卷,

《書傳補正輯注》一卷

梁寅《書纂義》十卷

朱右《書集傳發揮》十卷,《禹貢凡例》一卷

徐蘭《書經體要》一卷

陳雅言《尚書卓躍》六卷

郭元亮《尚書該義》十二卷

永樂中敕修《書傳大全》十卷胡廣等纂。

張洪《尚書補得》十二卷

彭勖《書傳通釋》六卷

徐善述《尚書直指》六卷

陳濟《書傳補注》一卷

徐驥《洪範解訂正》一卷

章陬《書經提要》四卷

費希冉《尚書本旨》七卷

楊守陳《書私鈔》一卷

黃瑜《書經旁通》十卷

李承恩《書經拾蔡》二卷

楊廉《洪範纂要》一卷

熊宗立《洪範九疇數解》八卷

張邦奇《書說》一卷

吳世忠《洪範考疑》一卷

鄭善夫《洪範論》一卷

劉天民《洪範辨疑》一卷

馬明衡《尚書疑義》一卷

呂柟《尚書說疑》五卷

韓邦奇《禹貢詳略》二卷

王崇慶《書經說略》一卷

舒芬《書論》一卷

鄭曉《尚書考》二卷,《禹貢圖說》一卷

馬森《書傳敷言》十卷

張居正《書經直解》八卷

王樵《尚書日記》十六卷,《書帷別記》四卷

陳錫《尚書經傳別解》一卷

歸有光《洪範傳》一卷,《考定武成》一卷

程弘賓《書經虹臺講義》十二卷

屠本畯《尚書別錄》六卷

鄧元錫《尚書釋》二卷

章潢《尚書圖說》三卷

陳第《尚書疏衍》四卷

羅敦仁《尚書是正》二十卷

鐘庚陽《尚書傳心錄》七卷

王祖嫡《書疏叢鈔》一卷

瞿九思《書經以俟錄》六卷

姚舜牧《書經疑問》十二卷

劉應秋《尚書旨》十卷

郭正域《東宮進講尚書義》一卷

錢一本《範衍》十卷

袁宗道《尚書纂注》四卷

焦竑《禹貢解》一卷

吳炯《書經質疑》一卷

王肯堂《尚書要旨》三十一卷

郝敬《尚書辨解》十卷

盧廷選《尚書雅言》六卷

曹學牷《書傳會衷》十卷

謝廷贊《書經翼注》七卷

趙惟寰《尚書蠡》四卷

陸鍵《尚書傳翼》十卷

張爾嘉《尚書貫言》二卷

姜逢元《禹貢詳節》一卷

朱道行《尚書集思通》十二卷

史惟堡《尚書晚訂》十二卷

楊肇芳《尚書副墨》六卷

潘士遴《尚書葦籥》五十卷

徐大儀《書經補注》六卷

黃道周《洪範明義》四卷

鄭鄤《禹貢注》一卷

艾南英《禹貢圖注》一卷

傅元初《尚書撮義》四卷

袁儼《尚書百家匯解》六卷

江旭奇《尚書傳翼》二卷

朱朝瑛《讀書略記》二卷

茅瑞徵《虞書箋》二卷,《禹貢匯疏》十二卷

王綱振《禹貢逆志》一卷

張能恭《禹貢訂傳》一卷

黃翼登《禹貢注刪》一卷

夏允彞《禹貢古今合注》五卷

羅喻義《洪範直解》一卷,《讀範內篇》一卷

──右《書》類,八十八部,四百九十七卷。

周是修《詩小序集成》三卷

梁寅《詩演義》八卷,《詩考》四卷

朱升《詩旁注》八卷

汪克寬《詩集傳音義會通》三十卷

曾堅《詩疑大鳴錄》一卷

硃善《詩解頤》四卷

高頤《詩集傳解》二十卷

張洪《詩正義》十五卷

楊禹錫《詩義》二卷

鄭旭《詩經總旨》一卷

永樂中敕修《詩集傳大全》二十卷胡廣等纂。

範理《詩集解》三十卷

王逢《詩經講說》二十卷

孫鼎《詩義集說》四卷

李賢《讀詩紀》一卷

楊守陳《詩私鈔》四卷

易貴《詩經直指》十五卷

程楷《詩經講說》二十卷

陸深《儼山詩微》二卷

張邦奇《詩說》一卷

湛若水《詩釐正》二十卷

呂柟《毛詩序說》六卷

胡纘宗《胡氏詩識》三卷

王崇慶《詩經衍義》一卷

季本《詩說解頤》八卷、《總論》二卷

黃佐《詩傳通解》二十五卷

潘恩《詩經輯說》七卷

陸垹《詩傳存疑》一卷

薛應旂《方山詩說》八卷

陳錫《詩辨疑》一卷

勞堪《詩林伐柯》四卷

沈一貫《詩經纂注》四卷

馮時可《詩人意》二卷

郭子章《詩傳書例》四卷

朱得之《印古詩說》一卷

袁仁《毛詩或問》二卷

鄧元錫《詩繹》三卷

陳第《毛詩古音考》四卷

朱謀韋《詩故》十卷

凌濛初《聖門傳詩嫡塚》十六卷,《詩逆》四卷

陶其情《詩經注疏大全纂》十二卷

趙一元《詩經理解》十四卷

黃一正《詩經埤傳》八卷

馮復京《六家詩名物疏》五十五卷

吳雨《毛詩鳥獸草木疏》三十卷

唐汝諤《毛詩微言》二十卷

瞿九思《詩經以俟錄》六卷

姚舜牧《詩經疑問》十二卷

林兆珂《毛詩多識篇》七卷

汪應蛟《學詩略》一卷

徐常吉《毛詩翼說》五卷

吳炯《詩經質疑》一卷

郝敬《毛詩原解》三十六卷、《序說》八卷

張彩《詩原》三十卷

徐必達《南州詩說》六卷

劉憲寵《詩經會說》八卷

曹學牷《詩經質疑》六卷

沈萬鈳《詩經類考》三十卷

顧起元《爾雅堂詩說》四卷

蔡毅中《詩經補傳》四卷

沈守正《詩經說通》十四卷

樊良樞《詩商》五卷

徐光啟《毛詩六帖》六卷

趙琮《葩經約說》十卷

莊廷臣《詩經逢源》八卷

鄒忠胤《詩傳闡》二十四卷

陸化熙《詩通》四卷

胡胤嘉《讀詩錄》二卷

朱道行《詩經集思通》十二卷

何楷《毛詩世本古義》二十八卷

張次仲《待軒詩記》六卷

張睿卿《詩疏》一卷

唐達《毛詩古音考辨》一卷

張溥《詩經注疏大全合纂》三十四卷

高承埏《五十家詩義裁中》十二卷

朱朝瑛《讀詩略記》二卷

張星懋《詩采》八卷

高鼎熺《詩經存旨》八卷

韋調鼎《詩經考定》二十四卷

趙起元《詩權》八卷

喬中和《葩經旁意》一卷

胡紹曾《詩經胡傳》十二卷

范王孫《詩志》二十六卷

──右《詩》類,八十七部,九百八卷。

方孝孺《周禮考次目錄》一卷

何喬新《周禮集注》七卷,《周禮明解》十二卷

陳鳳梧《周禮合訓》六卷

魏校《周禮沿革傳》六卷、《官職會通》二卷

楊慎《周官音詁》一卷

舒蒼《周禮定本》十三卷

季本《讀禮疑圖》六卷

陳深《周禮訓雋》十卷、《周禮訓注》十八卷、《考工》《記句詁》一卷

唐樞《周禮因論》一卷

羅洪先《周禮疑》一卷

王圻《續定周禮全經集注》十四卷

李如玉《周禮會注》十五卷

柯尚遷《周禮全經釋原》十四卷

金瑤《周禮述注》六卷

王應電《周禮傳》十卷,《周禮圖說》二卷,《學周禮法》一卷,《非周禮辨》一卷

馮時行《周禮別說》一卷

施天麟《周禮通義》二卷

徐即登《周禮說》十四卷

焦竑《考工記解》二卷

陳與郊《考工記輯注》二卷

郝敬《周禮完解》十二卷

郭良翰《周禮古本訂注》六卷

孫攀古《周禮釋評》六卷

陳仁錫《周禮句解》六卷

張采《周禮合解》十八卷

林兆珂《考工記述注》二卷

徐昭慶《考工記通》二卷

王志長《周禮注疏刪翼》三十卷

郎兆玉《注釋古周禮》六卷

沈羽明《周禮匯編》六卷

已上《周禮》。

汪克寬《經禮補逸》九卷

黃潤玉《儀禮戴記附注》五卷

何喬新《儀禮敘錄》十七卷

陳鳳梧《射禮集要》一卷

湛若水《儀禮補逸經傳測》一卷

徐駿《五服集證》一卷

王廷相《昏禮圖》一卷,《鄉射禮圖注》一卷,《喪禮論》一卷,《喪禮備纂》二卷

舒芬《士相見禮儀》一卷

聞人詮《飲射圖解》一卷

朱縉《射禮集解》一卷

胡纘宗《儀禮鄭注附逸禮》二十五卷

郝敬《儀禮節解》十七卷

王志長《儀禮注疏刪翼》十七卷

已上《儀禮》。

連伯聰《禮記集傳》十六卷

朱右《深衣考》一卷

黃潤玉《考定深衣古制》一卷

永樂中敕修《禮記大全》三十卷胡廣等纂。

鄭節《禮傳》八十卷

岳正《深衣注疏》一卷

楊廉《深衣纂要》一卷

夏時正《深衣考》一卷

王廷相《夏小正集解》一卷,《深衣圖論》一卷

夏言《深衣考》一卷

王崇慶《禮記約蒙》一卷

楊慎《檀弓叢訓》二卷一名《附注》,《夏小正解》一卷

張孚敬《禮記章句》八卷

戴冠《禮記集說辨疑》一卷

柯尚遷《曲禮全經類釋》十四卷

李孝先《投壺譜》一卷

黃乾行《禮記日錄》四十九卷

聞人德潤《禮記要旨補》十六卷

丘橓《禮記摘訓》十卷

徐師曾《禮記集注》三十卷

戈九疇《禮記要旨》十六卷

陳與郊《檀弓輯註》二卷

姚舜牧《禮記疑問》十二卷

沈一中《禮記述注》十八卷

王荁《禮記纂註》四卷

郝敬《禮記通解》二十二卷

余心純《禮經搜義》二十八卷

劉宗周《禮經考次正集》十四卷、《分集》四卷

樊良樞《禮測》二卷

陳有元《禮記約述》八卷

朱泰禎《禮記意評》四卷

湯三才《禮記新義》三十卷

王翼明《禮記補注》三十卷

黃道周《月令明義》四卷,《坊記集傳》二卷,《表記集傳》二卷,《緇衣集傳》二卷

陳際泰《王制說》一卷

張習孔《檀弓問》四卷

盧翰《月令通考》十六卷

楊鼎熙《禮記敬業》八卷

閻有章《說禮》三十一卷

已上《禮記》。

夏時正《三禮儀略舉要》十卷

湛若水《二禮經傳測》六十八卷大略以《曲禮》、《儀禮》為經,《禮記》為傳。

吳嶽《禮考》一卷

劉績《三禮圖》二卷

貢汝成《三禮纂注》四十九卷

李黼《二禮集解》十二卷合《周禮》、《儀禮》為一,集諸家之說以解之

李經綸《三禮類編》三十卷

鄧元錫《三禮編釋》二十六卷

唐伯玉《禮編》二十八卷

已上通《禮》。

──右《禮》類,一百七部,一千一百二十一卷。

湛若水《古樂經傳全書》二卷

張敔《雅樂發微》八卷,《樂書雜義》七卷

韓邦奇《律呂新書直解》一卷、《苑洛志樂》二十卷

周瑛《律呂管鑰》一卷

劉績《六樂圖》二卷

黃佐《禮典》四十卷,《樂典》三十六卷

何瑭《樂律管見》一卷一名《律呂管見》。

呂柟《詩樂圖譜》十八卷

季本《樂律纂要》一卷,《律呂別書》一卷

李文利《大樂律呂元聲》六卷,《大樂律呂考證》四卷

張諤《大成樂舞圖譜》二卷,《古雅心談》一卷

李文察《樂記補說》二卷,《四聖圖解》二卷,《律呂新書補注》一卷,《典樂要論》三卷,《古樂筌蹄》九卷,《青宮樂調》三卷

劉濂《樂經元義》八卷,《九代樂章》二十三卷

鄧文憲《律呂解注》二卷

黃積慶《樂律管見》二卷正李文利之非。

唐順之《樂論》八卷

蔡宗兗《律同》二卷

楊繼盛《擬補樂經》一卷

潘巒《文廟樂編》二卷

李璧《宴饗樂譜》一卷

葛見堯《含少論略》一卷

呂懷《律呂古義》二卷,《韻樂補遺》二卷,《律呂廣義》三卷

孫應鰲《律呂分解發明》四卷

王邦直《律呂正聲》六十卷

朱載堉《樂律全書》四十卷

樂和聲《大成樂舞圖說》一卷

何棟如《文廟雅樂考》二卷

史記事《大成禮樂集》三卷

瞿九思《孔廟禮樂考》五卷

李之藻《BL宮禮樂疏》十卷

黃居中《文廟禮樂志》十卷

梅鼎祚《古樂苑》五十二卷,《衍錄》四卷,《唐樂苑》三十卷

黃汝良《樂律志》四卷

王朝璽《律呂新書私解》一卷

王思宗《黃鐘元統圖說》一卷,《八音圖注》一卷

葉廣《禮樂合編》三十卷

王正中《律書詳註》一卷

──右《樂》類,五十四部,四百八十七卷。

《春秋本末》三十卷洪武中,懿文太子命宮臣傅藻等編

趙汸《春秋集傳》十五卷,《附錄》二卷,《春秋屬辭》十五卷,《左傳補注》十卷

梁寅《春秋考義》十卷

張以寧《春秋尊王發微》八卷,《春秋春王正月考》一卷,《辨疑》一卷

汪克寬《春秋胡傳附錄纂疏》三十卷

徐尊生《春秋論》一卷

蔡深《春秋纂》十卷

李衡《春秋釋例集說》三卷

石光霽《春秋書法鉤玄》四卷

永樂中敕修《春秋集傳大全》三十七卷胡廣等纂

金幼孜《春秋直指》三十卷,《春秋要旨》三卷

張洪《春秋說約》十二卷

饒秉鑒《春秋會傳》十五卷,《提要》一卷

張復《春秋中的》一卷

童品《春秋經傳辨疑》一卷

餘本《春秋傳疑》一卷

郭登《春秋左傳直解》十二卷

邵寶《左觿》一卷

楊循吉《春秋經解摘錄》一卷

湛若水《春秋正傳》三十七卷

金賢《春秋紀愚》十卷

劉節《春秋列傳》五卷

劉績《春秋左傳類解》二十卷

張邦奇《春秋說》一卷

席書《元山春秋論》一卷

江曉《春秋補傳》十五卷

魏校《春秋經世書》二卷

蔡芳《春秋訓義》十一卷

呂柟《春秋說志》五卷

許誥《春秋意見》一卷

胡世寧《春秋志疑》十八卷

鐘芳《春秋集要》二卷

楊慎《春秋地名考》一卷

湯虺《春秋易簡發明》二十卷

季本《春秋私考》三十卷

王崇慶《春秋析義》二卷

王道《春秋人意》四卷

胡纘宗《春秋本義》十二卷

姜絅《春秋曲言》十卷

李濂《夏周正辨疑會通》四卷

陸粲《左傳附注》五卷,《春秋左氏觿》二卷,《胡傳辨疑》二卷

任桂《春秋質疑》四卷

黃佐《纘春秋明經》十二卷

石琚《左傳章略》三卷

唐順之《春秋論》一卷,《左氏始末》十二卷

趙恒《春秋錄疑》十七卷

魏謙吉《春秋大旨》十卷

詹萊《春秋原經》十七卷

林命《春秋訂疑》十二卷

姚咨《春秋名臣傳》十三卷

袁顥《春秋傳》三十卷

袁祥《春秋或問》八卷

袁仁《鍼胡篇》一卷

邵弁《春秋尊王發微》十卷《屬辭比事》八卷,《或問》一卷,《凡例輯略》一卷。

傅遜《春秋左傳屬事》二十卷,《春秋左傳注解辨誤》二卷

嚴訥《春秋國華》十七卷

高拱《春秋正旨》一卷

姜寶《春秋事義全考》二十卷,《春秋讀傳解略》十二卷疏胡傳之義意,以便學者。

王樵《春秋輯傳》十五卷,《凡例》三卷

馬森《春秋伸義辨類》二十九卷

許孚遠《左氏詳節》八卷

顏鯨《春秋貫玉》四卷

李攀龍《春秋孔義》十二卷

汪道昆《春秋左傳節文》十五卷

吳國倫《春秋世譜》十卷以《春秋》列國事實見於《史記》、他書者,分國為諸侯世家。

徐學謨《春秋人意》六卷

朱睦挈《春秋諸傳辨疑》四卷

王錫爵《左傳釋義評苑》二十卷

鄧元錫《春秋繹》一卷

黃洪憲《春秋左傳釋附》二十七卷

黃正憲《春秋翼附》二十卷

馮時可《左氏討》二卷,《左氏論》二卷,《左氏釋》二卷

穆文熙《國概》六卷

餘懋學《春秋蠡測》四卷

凌稚隆《左傳測義》七十卷

錢時俊《春秋胡傳翼》三十卷

冷逢震《周正考》一卷

徐即登《春秋說》十一卷

鄒德溥《春秋匡解》八卷

姚舜牧《春秋疑問》十二卷

郝敬《春秋直解》十二卷

鄭良弼《春秋或問》十四卷,《存疑》一卷,《續義》二卷

張事心《春秋左氏人物譜》一卷

陸曾曄《編春秋所見所聞所傳聞》三卷

施仁《左粹類纂》十二卷

陳可言《春秋左傳類事》三十六卷

曹宗儒《春秋序事本末》三十卷,《逸傳》三卷,《左氏辨》三卷

曹學牷《春秋闡義》十二卷,《春秋義略》三卷

錢世揚《春秋說》十卷

王衡《春秋纂注》四卷

魏靖國《三傳異同》三十卷

卓爾康《春秋辨義》四十卷

張國經《春秋比事》七卷

錢應奎《左記》十一卷

張銓《春秋補傳》十二卷

馮伯禮《春秋羅纂》十二卷

耿汝忞《春秋愍渡》十五卷

顧懋樊《春秋義》三十卷

王震《春秋左翼》四十三卷

徐允祿《春秋愚謂》四卷

馮夢龍《春秋衡庫》二十卷

林嗣昌《春秋易義》十二卷

張溥《春秋三書》三十一卷

餘颺《春秋存俟》十二卷

虞宗瑤《春秋提要》二卷

劉城《春秋左傳地名錄》二卷

孫范《左傳紀事本末》二十二卷

來集之《春秋志在》十二卷,《四傳權衡》一卷

賀仲軾《春秋歸義》三十二卷,《便考》十卷

──右《春秋》類,一百三十一部,一千五百二十五卷。

宋濂《孝經新說》一卷

孫賁《孝經集善》一卷

孫吾與《孝經注解》一卷

方孝孺《孝經誡俗》一卷

晏璧《孝經刊誤》一卷

曹端《孝經述解》一卷

劉實《孝經集解》一卷

薛瑄《定次孝經今古文》一卷

楊守陳《孝經私鈔》八卷

餘本《孝經集注》三卷

王守仁《孝經大義》一卷

陳深《孝經解詁》一卷

歸有光《孝經敘錄》一卷

李材《孝經疏義》一卷

楊起元《孝經外傳》一卷,《孝經引證》二卷

虞淳熙《孝經邇言》九卷,《孝經集靈》一卷

胡時化《注解孝經》一卷

吳捴謙《重定孝經列傳》七卷

朱鴻《孝經質疑》一卷,《集解》一卷

王元祚《孝經匯注》三卷

陳仁錫《孝經小學詳解》八卷

黃道周《孝經集傳》二卷

何楷《孝經集傳》二卷

張有譽《孝經衍義》六卷

江旭奇《孝經疏義》一卷

瞿罕《孝經貫注》二十卷,《孝經存餘》三卷,《孝經考異》一卷,《孝經對問》三卷

呂維祺《孝經本義》二卷,《孝經大全》二十八卷,《或問》三卷

──右《孝經》類,三十五部,一百二十八卷。

蔣悌生《五經蠡測》六卷

董彞《經疑》十卷

黃潤玉《經書補注》四卷,《經譜》一卷

周洪謨《經書辨疑錄》三卷

王恕《石渠意見》二卷,《拾遺》一卷,《補缺》一卷

章懋《諸經講義》二卷

邵寶《簡端錄》十二卷

王崇慶《五經心義》五卷

王守仁《五經臆說》四十六卷

呂柟《經說》十卷

楊慎《經說》八卷

詹萊《七經思問》三卷

鄭世威《經書答問》十卷

薛治《五經發揮》七十卷

丁奉《經傳臆言》二十八卷

唐順之《五經總論》一卷

胡賓《六經圖全集》六卷

陳深《十三經解詁》六十卷

穆相《五經集序》二卷

王覺《五經四書明音》八卷

蔡汝楠《說經劄記》八卷

朱睦挈《授經圖》二十卷,《五經稽疑》六卷,《經序錄》五卷

陳士元《五經異文》十一卷

王世懋《經子臆解》一卷

徐常吉《遺經四解》四卷,《六經類雅》五卷

周應賓《九經考異》十二卷,《逸語》一卷

郝敬《九部經解》一百六十五卷

陳禹謨《經言枝指》十卷

蔡毅中《六經注疏》四十三卷

卜大有《經學要義》五卷

杜質明《儒經翼》七卷

陳仁錫《六經圖考》三十六卷

楊聯芳《群經類纂》三十四卷

楊維休《五經宗義》二十卷

張瑄《五經研朱集》二十二卷

顧夢麟《十一經通考》二十卷

──右諸經類,四十三部,七百三十四卷。

陶宗儀《四書備遺》二卷

劉醇《四書解疑》四卷

周是修《論語類編》二卷

永樂中敕修《四書大全》三十六卷胡廣等纂。

孔諤《中庸補注》一卷

黃潤玉《學庸通旨》一卷

周洪謨《四書辨疑錄》三卷

丁璣《大學疑義》一卷

蔡清《四書蒙引》十五卷

王守仁《古本大學注》一卷

朱綬《四書補注》三卷

夏良勝《中庸衍義》十七卷

湛若水《中庸測》一卷

程嗣光《四書講義》十卷

呂柟《四書因問》六卷

魏校《大學指歸》一卷

王道《大學人意》一卷

穆孔暉《大學千慮》一卷

季本《四書私存》三十七卷

薛甲《四書正義》十二卷

梁格《集四書古義補》十卷

金賁亨《學庸義》二卷

蘇濂《四書通考補遺》六卷

硃潤《四書通解》十卷

馬森《學庸口義》三卷

廖紀《四書管窺》四卷

陳士元《論語類考》二十卷,《孟子雜記》四卷

許孚遠《論語學庸述》四卷

謝東山《中庸集說啟蒙》一卷

唐樞《四書問錄》二卷

楊時喬《四書古今文註發》九卷

李材《論語大意》十二卷

顧憲成《大學通考》一卷,《大學質言》一卷

管志道《論語訂釋》十卷,《中庸測義》一卷,《孟子訂釋》七卷

鄒元標《學庸商求》二卷

鄭維嶽《四書知新日錄》三十七卷

王肯堂《論語義府》二十卷

史記事《四書疑問》五卷

郝敬《四書攝提》十卷

姚舜牧《四書疑問》十二卷

李槃《中庸臆說》一卷

吳應賓《中庸釋論》十二卷

顧起元《中庸外傳》三卷

林茂槐《四書正體》五卷

陳禹謨《談經苑》四十卷,《漢詁纂》二十卷,《引經繹》五卷,《人物概》十五卷,《名物考》二十卷

陶廷奎《四書正學衍說》八卷

劉元卿《四書宗解》八卷

陳仁錫《四書語錄》一百卷,《四書析義》十卷,《四書備考》八十卷

張溥《四書纂注大全》三十七卷

──右《四書》類五十九部,七百十二卷

危素《爾雅略義》十九卷

朱睦挈《訓林》十二卷

朱謀韋《駢雅》七卷

李文成《博雅志》十三卷

張萱《匯雅前編》二十卷,《後編》二十卷

羅曰褧《雅餘》八卷

穆希文《蟫史集》十一卷

黃裳《小學訓解》十卷

朱升《小四書》五卷集宋元儒方逢辰《名物蒙求》、程若庸《性理字訓》、陳櫟《歷代蒙求》各一卷,黃繼善《史學提要》二卷

何士信《小學集成》十卷,《圖說》一卷

趙古則《學範》六卷,《童蒙習句》一卷

方孝孺《幼儀雜箴》一卷

張洪《小學翼贊詩》六卷

鄭真《學範》六卷

朱逢吉《童子習》一卷

吳訥《小學集解》十卷

劉實《小學集注》六卷

丘陵《嬰教聲律》二十卷

廖紀《童訓》一卷

陳選《小學句讀》六卷

王雲鳳《小學章句》四卷

湛若水《古今小學》六卷

鐘芳《小學廣義》一卷

黃佐《小學古訓》一卷

王崇文《蒙訓》一卷

王崇獻《小學撮要》六卷

朱載瑋《困蒙錄》一卷

耿定向《小學衍義》二卷

吳國倫《訓初小鑑》四卷

周憲王有燉《家訓》一卷

朱勤美《諭家邇談》二卷

鄭綺《家範》二卷

王士覺《家則》一卷

程達道《家教輯錄》一卷

周是修《家訓》十二卷

楊榮《訓子編》一卷

曹端《家規輯略》一卷

楊廉《家規》一卷

何瑭《家訓》一卷

程敏政《貽範錄》三十卷

周思兼《家訓》一卷

孫植《家訓》一卷

吳性《宗約》一卷,《家訓》一卷

楊繼盛《家訓》一卷

王祖嫡《家庭庸言》二卷

已上小學。

《女誡》一卷洪武中命儒臣編。

高皇后《內訓》一卷

文皇后《勸善書》二十卷

章聖太后《女訓》一卷獻宗為序,世宗為後序。

慈聖太后《女鑑》一卷,《內則詩》一卷嘉靖中命方獻夫等撰。

黃佐《姆訓》一卷

王敬臣《婦訓》一卷

王直《女教續編》一卷

已上女學。

《洪武正韻》十六卷

孫吾與《韻會訂正》四卷

謝林《字學源委》五卷

趙古則《聲音文字通》一百卷,《六書本義》十二卷

穆正《文字譜系》十二卷

蘭廷秀《韻略易通》二卷

章黻《韻學集成》十二卷,《直音篇》七卷

塗觀《六書音義》十八卷

黃諫《從古正文》六卷

顧充《字類辨疑》二卷

張穎《古今韻釋》五卷

梁倫《稽古葉聲》二卷

周瑛《書纂》五卷

《音釋》一卷

王應電《同文備考》九卷

楊慎《轉注古音略》五卷,《古音叢目》五卷,《古音獵要》五卷,《古音附錄》五卷,《古音餘》五卷,《古音略例》一卷,《六書練證》五卷,《六書索隱》五卷,《古文韻語》二卷,《韻林原訓》五卷,《奇字韻》五卷,《韻藻》四卷

方豪《韻補》五卷

龔時憲《玉篇鑑磻》四十卷

劉隅《古篆分韻》五卷

潘恩《詩韻輯略》五卷

張之象《四聲韻補》五卷

陳士元《古俗字略》七卷,《韻苑考遺》四卷

田藝蘅《大明同文集》五十卷

茅溱《韻譜本義》十六卷

焦竑《俗書刊誤》十二卷

方日升《古今韻會小補》三十卷

程元初《五經詞賦葉韻統宗》二十四卷

黃鐘《音韻通括》二卷

郝敬《讀書通》二十卷

林茂槐《讀書字考略》四卷

趙宧光《說文長箋》七十二卷,《六書長箋》十三卷

梅膺祚《字匯》十二卷

吳汝紀《古今韻括》五卷

吳繼仕《音聲紀元》六卷

呂維祺《音韻日月燈》六十卷

周宇《字考啟蒙》十六卷,《認字測》三卷

周伯殷《字義切略》二卷

楊昌文《篆韻正義》五卷

熊晦《類聚音韻》三十卷

楊廉《綴算舉例》一卷,《數學圖訣發明》一卷

顧應祥《測圓算術》四卷,《弧矢算術》二卷,《釋測圓海鏡》十卷

唐順之《句股等六論》一卷

硃載堉《嘉量算經》三卷

李瓚《同文算指通編》二卷,《前編》二卷

楊輝《九章》一卷

已上書數。

──右小學類,一百二十三部,一千六十四卷。


史類十:一曰正史類,編年在內。二曰雜史類,三曰史鈔類,四曰故事類,五曰職官類,六曰儀注類,七曰刑法類,八曰傳記類,九曰地理類,十曰譜牒類。

明《太祖實錄》二百五十七卷建文元年,董倫等修。永樂元年,解縉等重修。九年,胡廣等復修。起元至正辛卯,訖洪武三十一年戊寅,首尾四十八年。萬歷時,允科臣楊天民請,附建文帝元、二、三、四年事跡於後。

《日歷》一百卷、洪武中,詹同等編,具載太祖征討平定之績,禮樂治道之詳。《寶訓》十五卷《日歷》即成,詹同等又請分類更輯聖政為書,凡五卷。其後史官隨類增至十五卷。

《成祖實錄》一百三十卷,《寶訓》十五卷楊士奇等修。

《仁宗實錄》十卷,《寶訓》六卷蹇義等修。

《宣宗實錄》一百十五卷,《寶訓》十二卷楊士奇等修。

《英宗實錄》三百六十一卷、成化元年,陳文等修,起宣德十年正月,訖天順八年正月,首尾三十年。附景泰帝事跡於中,凡八十七卷。《寶訓》十二卷與《實錄》同修。

《憲宗實錄》二百九十三卷,《寶訓》十卷劉吉等修。

《孝宗實錄》二百二十四卷、正德元年,劉健、謝遷等修。未幾健、遷皆去位,焦芳等續修。《寶訓》十卷與《實錄》同修。

《武宗實錄》一百九十七卷,《寶訓》十卷費宏等修。

《睿宗實錄》五十卷,《寶訓》十卷嘉靖四年,大學士費宏言:「獻皇帝嘉言懿行,舊邸必有成書,宜取付史館纂修。」從之。

《世宗實錄》五百六十六卷,《寶訓》二十四卷隆慶中,徐階等修,未竣。萬歷五年,張居正等續修成之。

《穆宗實錄》七十卷,《寶訓》八卷張居正等修。

《神宗實錄》五百九十四卷,《寶訓》二十六卷溫體仁等修。

《光宗實錄》八卷天啟三年,葉向高等修成,有熹宗御製序。既而霍維華等改修,未及上,而熹宗崩。至崇禎元年,始進呈向高原本,並貯皇史宬。

《熹宗實錄》八十四卷溫體仁等修。

《洪武聖政記》二卷

《永樂聖政記》三卷

《永樂年表》四卷

《洪熙年表》二卷

《宣德年表》四卷

儲巏《皇明政要》二十卷

鄭曉《吾學編》六十九卷

雷禮《大政記》三十六卷

鄧元錫《明書》四十五卷

夏浚《皇明大紀》三十六卷

王世貞《國朝紀要》十卷,《天言匯錄》十卷。

陳建《皇明通紀》二十七卷,《續通紀》十卷

薛應旂《憲章錄》四十六卷

沈越《嘉隆聞見紀》十二卷

唐志大《高廟聖政記》二十四卷

孫宜《國朝事跡》一百二十卷

吳朴《洪武大政記》二十卷

吳瑞登《明繩武編》三十四卷,《嘉隆憲章錄》二十卷

黃翔鳳《嘉靖大政編年紀》一卷,《嘉靖大政類編》二卷

陳翼飛《史待》五十卷

何喬遠《名山藏》三十七卷

朱國禎《史概》一百二十卷,《輯皇明紀傳》三十卷

支大倫《永昭二陵編年信史》六卷

尹守衡《史竊》一百七卷

朱睦挈《聖典》三十四卷

茅維《嘉靖大政記》二卷

吳士奇《皇明副書》一百卷

譚希思《皇明大紀纂要》六十三卷

王大綱《皇明朝野紀略》一千二百卷

雷叔聞《國史》四十卷

周永春《政紀纂要》四卷

張銓《國史紀聞》十二卷

馮復京《明右史略》三十卷

陳仁錫《皇明世法錄》九十二卷

沈國元《天啟從信錄》三十五卷

江旭奇《通紀集要》六十卷

談遷《國榷》一百卷

已上明史。

《元史》二百十二卷洪武中宋濂等修。

《續宋元資治通鑑綱目》二十七卷成化中商輅等修

《歷代通鑑纂要》九十二卷弘治中李東陽等修。

周定王橚《甲子編年》十二卷

王禕《大事記續編》七十七卷

梁寅《宋史略》四卷,《元史略》四卷

朱右《元史補遺》十二卷

張九韶《元史節要》二卷

胡粹中《元史續編》七十七卷

丘浚《世史正綱》三十二卷

金濂《諸史會編》一百十二卷,《南軒資治通鑑綱目前編》二十五卷

柯維騏《宋史新編》二百卷

唐順之《史纂左編》一百四十二卷,《右編》四十卷

薛應旂《宋元資治通鑒》一百五十七卷,《甲子會紀》五卷

王宗沐《宋元資治通鑑》六十四卷

安都《十九史節定》一百七十卷

吳珫《史類》六百卷

鄧元錫《函史》上編九十五卷、下編二十卷

許誥《綱目前編》三卷

魏國顯《史書大全》五百十二卷

黃佐《通曆》三十六卷

姜寶《稽古編大政記綱目》八卷,《資治上編大政記綱目》四十卷,《下編大政記綱目》三十卷

邵經邦《學史會同》三百卷,《弘簡錄》二百五十卷

楊寅冬《曆代史匯》二百四十卷

饒伸《學海君道部》二百三十四卷

徐師曾《世統紀年》六卷

吳繼安《帝王曆祚考》八卷

馮琦《宋史紀事本末》二十八卷

張溥《宋史紀事本末》一百九卷,《元史紀事本末》二十七卷

陳邦瞻《元史紀事本末》六卷

湯桂禎《戰國紀年》四十六卷

嚴衍《資治通鑑補》二百七十卷

已上通史。

──右正史類一百十部,一萬二百三十二卷

劉辰《國初事蹟》一卷

俞本《記事錄》二卷

王禕《造邦勳賢略》一卷

劉基《禮賢錄》一卷,《翊運錄》二卷

楊儀《壟起雜事》一卷紀張士誠、韓林兒、徐壽輝事

楊學可《明氏實錄》一卷明玉珍事。

何榮祖《家記》一卷何真子,紀真事。

張紞《雲南機務鈔黃》一卷

夏原吉《萬乘肇基錄》一卷

卞瑺《興濠開基錄》一卷

陸深《平元錄》一卷

童承敘《平漢錄》一卷

黃標《平夏錄》一卷

姚淶《驅除錄》一卷

蔡於穀《開國事略》十卷

孫宜《明初略》二卷

邵相《皇明啟運錄》八卷

梁億《洪武輯遺》二卷

范守己《造夏略》二卷

戴重《和陽開天記》一卷

錢謙益《開國群雄事略》十五卷,《太祖實錄辨證》三卷

已上皆紀洪武時事。

袁祥《建文私記》一卷

孫交《國史補遺》六卷

姜清《祕史》一卷

黃佐《革除遺事》六卷

張芹《建文備遺錄》二卷

何孟春《續備遺錄》一卷

馮汝弼《補備遺錄》一卷

許相卿《革朝志》十卷

朱睦挈《遜國記》二卷

屠叔方《建文朝野匯編》二十卷

朱鷺《建文書法儗》四卷

陳仁錫《壬午書》二卷

曹參芳《遜國正氣紀》九卷

周遠令《建文紀》三卷

已上紀建文時事

都穆《壬午功臣爵賞錄》一卷,《別錄》一卷

袁褧《奉天刑賞錄》一卷

郁袞《順命錄》一卷

楊榮《北征記》一卷

金幼孜《北征前錄》一卷,《後錄》一卷

黃福《安南事宜》一卷

丘浚《平定交南錄》一卷

楊士奇《三朝聖諭錄》三卷,《西巡扈從紀行錄》一卷

已上紀永樂、洪熙、宣德時事。

袁彬《北征事跡》一卷

楊銘《正弦臨戎錄》一卷,《北狩事跡》一卷

李實《使北錄》一卷

劉定之《否泰錄》一卷

劉濟《革書》一卷塞外無楮,以羊皮書之,故名《革書》。

李賢《天順日錄》二卷

湯韶《天順實錄辨證》一卷

張楷《監國曆略》一卷

彭時《可齋筆記》二卷

已上紀正統、景泰、天順時事。

馬文升《西征石城記》一卷,《興復哈密記》一卷

宋端儀《立齋閒錄》四卷

梅純《損齋備忘錄》二卷

李東陽《燕對錄》二卷

劉大夏《宣召錄》一卷

陳洪謨《治世餘聞》四卷弘治、《繼世紀聞》四卷正德

許進《平番始末》一卷

硃國祚《孝宗大紀》一卷

費宏《武廟初所見事》一卷

楊廷和《視草餘錄》二卷

王鏊《震澤紀聞》一卷,《續紀聞》一卷,《震澤長語》二卷,《守溪筆記》二卷

王瓊《雙溪雜記》二卷

楊一清《西征日錄》一卷,《車駕幸第錄》二卷

胡世寧《桃源建昌徵案》、《東鄉撫案》共十卷

祝允明《九朝野記》四卷,《江海殲渠記》一卷紀劉六、劉七、趙風子事

夏良勝《東戍錄》一卷

謝蕡《後鑒錄》三卷

已上紀成化、弘治、正德時事。

世宗《大禮集議》四卷,《纂要》二卷,《明倫大典》二十四卷,《大狩龍飛錄》二卷

王之垣《承天大志紀錄事實》三十卷

費宏《宸章集錄》一卷

張孚敬《敕諭錄》三卷,《諭對錄》三十四卷,《大禮要略》二卷,《欽明大獄錄》二卷

李時《南城召對錄》一卷,《文華盛記》一卷

夏言《聖駕渡黃河記》一卷,《記召對廟廷事》一卷,《扈蹕錄》一卷

嚴嵩《嘉靖奏對錄》十二卷

毛澄《聖駕臨雍錄》一卷

陸深《聖駕南巡錄》一卷,《北還錄》一卷

韓邦奇《大同紀事》一卷

孫允中《雲中紀變》一卷

蘇祐《雲中事紀》一卷

張岳《交事紀聞》一卷

翁萬達《平交紀事》十卷

江美中《安南來威輯略》三卷

談愷《前後平粵錄》四卷

王軾《平蠻錄》二卷

萬表《前後海寇議》三卷

鄭茂《靖海紀略》一卷

徐宗魯《松寇紀略》一卷

李日華《倭變志》一卷

張鼐《吳淞甲乙倭變志》二卷

朱紈《茂邊紀事》一卷

趙汝謙《平黔三記》一卷

徐學謨《世廟識餘錄》二十六卷

高拱《邊略》五卷

劉應箕《款塞始末》一卷

方逢時《平惠州事》一卷

林庭機《平曾一本敘》一卷

查志隆《安慶兵變》一卷

曹子登《甘州紀變》一卷

王尚文《征南紀略》一卷

張居正《召對紀事》一卷

申時行《召見紀事》一卷

王錫爵《召見紀事》一卷

趙志皋《召見紀事》一卷

方從哲《乙卯召對錄》三卷

董其昌《萬曆事實纂要》三百卷

顧憲成《寤言寐言》一卷

陳惟之《乞停礦稅疏圖》一卷

郭子章《黔中止榷記》一卷,《西南三征記》一卷,《黔中平播始末》三卷

王禹聲《郢事紀略》一卷紀楚中稅監激變事。

郭正域《楚事妖書始末》一卷。

朱賡《勘楚始末》一卷

蔡獻臣《勘楚紀事》一卷

瞿九思《萬曆武功錄》十四卷

諸葛元聲《兩朝平攘錄》五卷

茅瑞徵《萬曆三大徵考》五卷哱氏、關白、楊應龍。

曾偉芳《寧夏紀事》一卷

宋應昌《朝鮮復國經略》六卷

蕭應宮《朝鮮征倭紀略》一卷

王士琦《封貢紀略》一卷

李化龍《平播全書》十五卷

楊寅秋《平播錄》五卷

沈德符《野獲編》八卷

李維楨《庚申紀事》一卷

張潑《庚申紀事》一卷

已上紀嘉靖、隆慶、萬歷時事

《三朝要典》二十四卷天啟中,顧秉謙等修。崇禎初,詔毀之。

葉茂才《三案記》一卷

蔡士順《傃庵野鈔》十一卷

李枟《全黔紀略》一卷

張鍵《平藺紀事》一卷

李遜之《三朝野記》七卷

蔣德璟《愨書》十卷

李日宣《枚卜始末》一卷

楊士聰《玉堂薈記》四卷

孫承宗《督師全書》一百卷

楊嗣昌《督師紀事》五十卷

夏允彞《幸存錄》一卷

夏完淳《續幸存錄》一卷

吳偉業《綏寇紀略》十二卷

文秉《先撥志始》六卷,《烈皇小識》四卷

彭孫貽《流寇志》十四卷

李清《南渡錄》二卷

以上紀天啟、崇禎時事。

黃瑜《雙槐歲鈔》十卷起洪武訖成化中事。

倫以訓《國朝彞憲》二十卷

孫宜《國朝事迹》一百二十卷

高岱《鴻奠錄》十六卷

鄭曉《今言》四卷,《徵吾錄》二卷,《吾學編餘》一卷

潘恩《美芹錄》二卷

袁帙《皇明獻實》二十卷

孫繼芳《磯園稗史》二卷

李先芳《安攘新編》三十卷

王世貞《弇山堂別集》一百卷,《識小錄》二十卷,《少陽叢談》二十卷,《明野史匯》一百卷萬曆中,董復表匯纂諸集為《弇州史料》,凡一百卷。

鄧球《泳化類編》一百三十六卷,《雜記》二卷

高鳴鳳《今獻匯言》二十八卷

何棟如《皇明四大法》十二卷

王禪《國朝史略》四十五卷、《別集》二卷

于慎行《穀山筆麈》十八卷

黃汝良《野紀矇搜》十二卷起洪、永,訖嘉、隆。

曹育賢《皇明類考》二十二卷

鄒德泳《聖朝泰交錄》八卷

張萱《西園聞見錄》一百六卷

吳士奇《徵信編》五卷,《考信編》二卷

項鼎鉉《名臣寧攘編》三十卷

范景文《昭代武功錄》十卷

已上統紀明代事。

寧獻王權《漢唐祕史》二卷洪武中奉敕編次。

吳源《至正近記》二卷

權衡《庚申外史》二卷

楊循吉《遼金小史》九卷

楊慎《滇載記》一卷

倪輅《南詔野史》一卷

包宗吉《古史補》二百卷

袁祥《新舊唐書折衷》二十四卷

程敏政《宋紀受終考》一卷

李維楨《韓范經略西夏紀》一卷

王士騏《苻秦書》十五卷

李廷機《宋賢事匯》二卷

姚士粦《後梁春秋》十卷

胡震亨《靖康盜鑒錄》一卷

陳霆《唐餘紀傳》二十一卷

錢謙益《北盟會編鈔》三卷

已上紀前代事

──右雜史類,二百十七部,二千二百四十四卷。

楊維楨《史義拾遺》二卷

范理《讀史備忘》八卷

陳濟《通鑒綱目集覽正誤》五十九卷

趙弼《雪航膚見》十卷

李裕《分類史鈔》二十二卷

呂柟《史約》三十七卷

許誥《宋元史闡幽》三卷

張寧《讀史錄》六卷

李浩《通鑑斷義》七十卷

邵寶《學史》十三卷

王峰《通鑒綱目發微》三十卷

張時泰《續通鑒綱目廣義》十七卷

卜大有《史學要義》四卷

周山《師資論統》一百卷

鄭曉《刪改史論》十卷

柯維騏《史記考要》十卷

王洙《宋元史質》一百卷

戴璟《漢唐通鑑品藻》三十卷

鐘芳《續古今紀要》十卷

歸有光《讀史纂言》十卷

李維楨《南北史小識》十卷

萬廷言《經世要略》二十卷

張之象《太史史例》一百卷

徐明勳《史衡》二十卷

于慎行《讀史漫錄》十四卷

李贄《藏書》六十八卷,《續藏書》二十七卷

馬惟銘《史書纂略》一百卷

趙惟寰《讀史快編》六十卷

謝肇淛《史鐫》二十一卷

吳無奇《史裁》二十六卷

張溥《史論二編》十卷

楊以任《讀史四集》四卷

馮尚賢《史學匯編》十二卷

──右史鈔類三十四部,一千四十三卷

太祖《御製永鑑錄》一卷訓親籓、《紀非錄》一卷訓周、齊、潭、魯諸王。

《祖訓錄》一卷洪武中編集,太祖制序,頒賜諸王。

《祖訓條章》一卷封建王國之制。

《宗籓昭鑑錄》五卷洪武中陶凱等編集。

《歷代公主錄》一卷洪武中編集。

《世臣總錄》二卷

《為政要錄》一卷

《醒貪簡要錄》二卷

《武士訓戒錄》一卷

《臣戒錄》一卷俱洪武中頒行。

《存心錄》十八卷吳沉等編集。

《省躬錄》三卷劉三吾等編集。

《精誠錄》三卷吳沉等編集。

《國朝制作》一卷王叔銘等編集。

宣宗《御製歷代臣鑒》三十七卷,《外戚事鑑》五卷

萬曆中重修《大明會典》二百二十八卷,《條例全文》三十卷,《增修條例備考》二十六卷

《大明會要》八十卷太祖開國時事,凡三十九則,不知撰人。

李賢《鑒古錄》一卷

夏寅《政鑑》三十卷

顧灒《稽古政要》十卷

王圻《續文獻通考》二百五十四卷

鄧球《續泳化編》十七卷

鄒泉《古今經史格要》二十八卷

黃光昇《昭代典則》二十八卷

周子義《國朝故實》二百卷一名《國朝典故備遺》。

張居正《帝鑑圖說》六卷

焦竑《養正圖解》二卷

勞堪《皇明憲章類編》四十二卷

徐學聚《國朝典匯》二百卷

唐瑤《歷代志略》四卷

張銓《鑒古錄》六卷

喬懋敬《古今廉鑒》八卷

馮應京《皇明經世實用編》二十八卷

鄧士龍《國朝典故》一百卷

黃溥《皇明經濟錄》十八卷

徐奮鵬《古今治統》二十卷

朱健《古今治平略》三十六卷

餘繼登《皇明典故紀聞》十八卷

《宗籓條例》二卷李春芳等輯。

戚元佐《宗籓議》一卷

馮柯《歷代宗籓訓典》十二卷

張志淳《謚法》二卷

何三省《帝后尊謚紀略》一卷

鮑應BI《皇明臣謚匯考》二卷

葉來敬《皇明謚考》三十八卷

郭良翰《皇明謚紀匯編》二十五卷

鄭汝璧《功臣封考》八卷

陸深《科場條貫》一卷

張朝瑞《皇明貢舉考》八卷,《明歷科殿試錄》七十卷,《歷科會試錄》七十卷

汪鯨《大明會計類要》十二卷

張學顏《萬曆會計錄》四十三卷

趙官《後湖志》十一卷,《後湖黃志》六卷

劉斯潔《太倉考》十卷

王儀《吳中田賦錄》五卷

徐民式《三吳均役全書》四卷

婁志德《兩浙賦役全書》十二卷

何士晉《廠庫須知》十二卷

邵寶《漕政錄》十八卷

席書《漕船志》一卷,《漕運錄》二卷

楊宏《漕運志》四卷

王在晉《通漕類編》九卷

陳仁錫《漕政考》二卷

崔旦《海運編》二卷

劉體仁《海道漕運記》一卷

王宗沐《海運志》二卷

梁夢龍《海運新考》三卷

史繼偕《皇明兵志考》三卷

侯繼高《全浙兵志考》四卷

王士琦《三雲籌俎考》四卷

何孟春《軍務集錄》六卷

閻世科《計遼始末》四卷

蔡鼎《邊務要略》十卷

周文郁《邊事小紀》六卷

王士騏《馭倭錄》八卷

方日乾《屯田事宜》五卷

楊守謙《屯田議》一卷

張抱赤《屯田書》一卷

沈摐《南船記》四卷

倪涷《船政新書》四卷

朱廷立《鹽政志》十卷

史啟哲《兩淮鹽法志》十二卷

王圻《兩浙鹽志》二十四卷

冷宗元《長蘆鹺志》七卷

李開先《山東鹽法志》四卷

詹榮《河東運司志》十七卷

謝肇淛《八閩鹺政志》十六卷

李涷《粵東鹽政考》二卷

陳善《黑白鹽井事宜》二卷

傅浚《鐵冶志》二卷

胡彥《茶馬類考》六卷

陳講《茶馬志》四卷

徐彥登《歷朝茶馬奏議》四卷

王宗聖《榷政記》十卷

薛僑《南關志》六卷

許天贈《北關志》十二卷

林希元《荒政叢言》一卷

賀燦然《備荒議》一卷

俞汝為《荒政要覽》十卷

──右故事類,一百六部,二千一百二十一卷。

《諸司職掌》十卷洪武中翟善等編。

《憲綱》一卷洪武中御史臺進。

《官制大全》十六卷

《品級考》五卷

宣宗《御製官箴》一卷

郭子章《官釋》十卷

李日華《官制備考》二卷

鄭曉《直文淵閣表》一卷,《典銓表》一卷

呂本《館閣類錄》二十二卷

雷禮《列卿表》一百三十九卷

王世貞《公卿表》二十四卷

李維楨《進士列卿表》二卷

徐鑑《續列卿表》十卷

徐鑒《續列卿表》十卷

許重熙《殿閣部院大臣表》十六卷

范景文《大臣譜》十六卷

黃尊素《隆萬列卿記》二卷

陳盟《崇禎閣臣年表》一卷,《內閣行略》一卷

廖道南《殿閣詞林記》二十二卷

黃佐《翰林記》二十卷

張位《詞林典故》一卷,《史職議》一卷

陳沂《翰林志》一卷

焦竑《詞林歷官表》三卷

董其昌《南京翰林志》十二卷

周應賓《舊京詞林志》六卷

劉昌《南京詹事府志》二十卷

李默《吏部職掌》四卷

張瀚《吏部職掌》八卷

鄭汝譬《封司典故》八卷

王士騏《銓曹紀要》十六卷

宋啟明《吏部志》四十卷

汪宗伊《南京吏部志》二十卷,《留銓志餘》二卷

徐大相《銓曹儀注》五卷

王崇慶《南京戶部志》二十卷

謝彬《南京戶部志》二十卷

宋端儀《祠部典故》四卷

李廷機《春官要覽》六卷

李化龍《邦政條例》十卷

譚綸《軍政條例類考》七卷

傅鶚《軍政類編》二卷

陳夢鶴《武銓邦政》二卷

李邦華《南樞新志》四卷

范景文《南樞志》一百七十卷

俞汝為《南京兵部車駕司職掌》八卷

張可大《南京錦衣衛志》二十卷

應廷育《刑部志》八卷

龐嵩《刑曹志》四卷

劉文徵《刑部事宜》十卷

陳公相《刑部文獻考》八卷

來斯行《刑部獄志》四十卷

江山麗《南京刑部志》二十六卷

曾同亨《工部條例》十卷

周夢暘《水部備考》十卷

劉振《工部志》一百三十九卷

王廷相《申明憲綱錄》一卷

劉宗周《憲綱規條》一卷

傅漢《風紀輯覽》四卷

符驗《西臺雜記》八卷

何出光《蘭臺法鑑錄》二十三卷

徐必達《南京都察院志》四十卷

朱廷益《通政司志》六卷

夏時正《太常志》十卷

陳慶《太常志》十六卷

盧維禎《太常志》十六卷

呂鳴珂《太常紀》二十二卷

倪嵩《太常典禮總覽》六卷

屠本畯《太常典錄》六卷

沈若霖《南京太常寺志》四十卷

顧存仁《太僕志》十四卷

楊時喬《馬政記》十二卷

李日宣《太僕志》二十二卷

雷禮《南京太僕寺志》十六卷

徐必達《光祿寺志》二十卷

韓鼎《尚寶司實錄》一卷

潘煥宿《南京尚寶司志》二十卷

周崑《六科仕籍》六卷

蕭彥《掖垣人鑑》十七卷

《國子監規》一卷錄洪武以來訓諭。

邢讓《國子監志》二十二卷

謝鐸《國子監續志》十一卷

吳節《南雍舊志》十八卷

黃佐《南雍志》二十四卷

王材《南雍申教錄》十五卷

崔銑《國子監條例類編》六卷

盧上銘《辟雍紀事》十五卷

汪俊《四夷館則例》二十卷,《四夷館考》二卷

楊樞《上林記》八卷

王象雲《上林匯考》五卷

焦竑《京學志》八卷

──右職官類九十三部,一千四百七十九卷。

《集禮》五十卷洪武中梁寅等纂修。初係寫本,嘉靖中,詔禮部校刊。

《孝慈錄》一卷宋濂等考定喪服古制為是書,太祖有序。

《行移繁減體式》一卷洪武中,以元季官府文移紛冗,詔廷臣減繁,著為定式。

《稽制錄》一卷編輯功臣服舍制度。

《禮制集要》一卷官民服舍器用等式。

《稽古定制》一卷頒示功臣。

《禮儀定式》一卷,《教民榜文》一卷,《鄉飲酒禮圖式》一卷俱洪武中頒行。

《祭祀禮儀》六卷,《郊壇祭享儀注》一卷皆明初定式

《巡狩事宜》一卷永樂中儀注。

《瑞應圖說》一卷永樂中編次。

憲宗《幸學儀注》一卷

世宗《御製忌祭或問》一卷,《祀儀成典》七十一卷嘉靖間更定儀文。

《郊祀通典》二十七卷夏言等編次。

《乘輿冕服圖說》一卷嘉靖間考古衣冠之制,張璁為注說

《武弁服制圖說》一卷親征冠服之制,張璁為注說。

《玄端冠服圖說》一卷燕居冠服之制,張璁為注說。

《保和冠服圖說》一卷宗室冠服之制,張璁為注說。

《圜丘方澤總圖》二卷

《圜丘方澤祭器樂器圖》二卷

《朝日夕月壇總圖》二卷

《朝日夕月壇祭器樂器圖》二卷

《神祇社稷雩壇總圖》三卷

《太廟總圖》一卷

《太廟供器祭器圖》一卷

《大享殿圖》一卷

《大享殿供器祭器圖》一卷

《天壽山諸陵總圖》一卷

《泰神殿圖》一卷

《帝王廟總圖》二卷

《皇史宬景神等殿圖》二卷

《圓明閣陽雷軒殿宇圖》一卷

《沙河行宮圖》一卷已上俱嘉靖間制式。

《皇明典禮》一卷萬歷中頒。

《朝儀》二卷,《車駕巡幸禮儀》一卷,《親王昏禮儀注》一卷,《昏禮傳制遣官圖》一卷,《陵寢儀式》一卷,《王國儀注》一卷,《儀注事例》一卷,《鴻臚儀注》二卷,《出使儀注》二卷,《射禮儀注》一卷已上俱萬歷間制式

《禮書》四十一卷不知撰人,凡十七冊。目錄一,吉禮五,軍禮、凶禮共一,喪禮三,制度一,考正一,宮制二,公式二,雜禮一。

《大明禮制》二十五卷不知撰人。

《嘉靖祀典》十七卷不知撰人。

朱國祚《冊立儀注》一卷

皇甫濂《籓府政令》一卷

郭正域《皇明典禮志》二十卷

朱勤美《王國典禮》八卷

謝鐸《祭禮儀注》二卷

羅青霄《儀注輯錄》一卷郡邑慶賀祭祀諸儀。

俞汝楫《禮儀志》一百卷

──右儀注類五十七部,四百二十四卷。

《大明律》三十卷洪武六年,命刑部尚書劉惟謙詳定。篇目皆準唐律,合六百有六條。九年復釐正十有三條,餘仍故。

《更定大明律》三十卷洪武二十八年,命詞臣同刑官參考比年律條,以類編附,凡四百六十條。

太祖《御製大誥》一卷、《大誥續編》一卷、《大誥三編》一卷、《大誥武臣》一卷、《武臣敕諭》一卷、《昭示姦黨錄》一卷、《第二錄》一卷、《第三錄》三卷已上三《錄》皆胡黨獄詞。

《逆臣錄》五卷藍黨獄詞。

《彰善癉惡錄》三卷,《癉惡續錄》一卷,《集犯諭》一卷,《戒敕功臣鐵榜》一卷已上皆洪武中頒。

何廣《律解辨疑》三十卷

鄭節《續真西山政經》二卷

薛瑄《從政錄》一卷

盧雍《祥刑集覽》二卷

何文淵《牧民備用》一卷,《司刑備用》一卷

陳廷璉《大明律分類條目》四卷

顧應祥《重修問刑條例》七卷

劉惟謙《唐律疏義》十二卷

張楷《大明律解》十二卷

應檟《大明律釋義》三十卷

高舉《大明律集解附例》三十卷

范永鑾《大明律例》三十卷

陳璋《比部招擬》二卷

段正《柏臺公案》八卷

應廷育《讀律管窺》十二卷

雷夢麟《讀律瑣言》三十卷

孫存《大明律讀法書》三十卷

王樵《讀律私箋》二十四卷

林兆珂《注大明律例》二十卷

王之垣《律解附例》八卷

舒化《問刑條例》七卷,《刑書會據》三十卷

王肯堂《律例箋解》三十卷

歐陽東鳳《闡律》一卷

熊鳴岐《昭代王章》十五卷

吳訥《祥刑要覽》二卷

鄒元標《筮仕要訣》一卷

蘇茂相《臨民寶鏡》十六卷

陳龍正《政書》二十卷

曹璜《治術綱目》十卷

──右刑法類,四十六部,五百九卷。

《開國功臣錄》三十一卷黃金編次,自徐達至指揮李觀,凡五百九十一人

謝鐸《名臣事略》二十卷洪武至成化時人。

彭韶《名臣錄贊》二卷

楊廉《名臣言行錄》四卷,《理學名臣言行錄》二卷

徐紱《名臣琬琰錄》五十四卷

徐咸《名臣言行錄前集》十二卷,《後集》十二卷

王道《名臣琬琰錄》二卷,《續錄》二卷

張芹《備遺錄》一卷

何孟春《續遺錄》一卷

何喬新《勳賢琬琰集》二卷

唐龍《康山群忠錄》一卷,《二忠錄》二卷紀王禕、吳雲事。

沈庭奎《名臣言行錄新編》三十四卷

楊豫孫《補輯名臣琬琰錄》一百十卷

雷禮《閣臣行實》八卷,《列卿記》一百六十五卷起洪武,訖嘉靖。禮子映補隆慶一朝。

王世貞《嘉靖以來首輔傳》八卷,《名卿紀蹟》六卷

吳伯與《內閣名臣事略》十六卷

薛應旂《皇明人物考》七卷鄭以偉注。

唐樞《國琛集》二卷

史繼偕《越章》六卷明代八閩人傳。

顧璘《國寶新編》一卷

項篤壽《今獻備遺》四十二卷

凌迪知《名臣類苑》四十六卷

錢薇《名臣事實》三十卷

耿定向《先進遺風》二卷

李廷機《閣臣錄》六卷

焦竑《國史獻徵錄》一百二十卷《經籍志》作三百六十卷、《遜國忠節錄》八卷

唐鶴徵《輔世編》六卷,《續編》五卷

徐即登《建文諸臣錄》二卷

童時明《昭代明良錄》二十卷

劉夢雷《名臣考》四卷

林墊《重輯名臣錄》二卷

朱謀韋《籓獻記》四卷

朱勤美《公族傳略》二卷

過庭訓《直省分郡人物考》一百十五卷

王兆雲《詞林人物考》十六卷

張璽《明尚友集》十六卷

江盈科《明臣小傳》十六卷

瞿汝說《臣略纂聞》十二卷

錢士升《皇明表忠錄》二卷

餘翹《池陽三忠傳》一卷紀黃觀、金焦、陳敬宗事。

馮復京《先賢事略》十卷

李裁《明臣論世》四卷

林之盛《應謚名臣傳》十二卷

杜瓊《紀善錄》一卷

陳沂《畜德錄》一卷

蘇茂相《名臣類編》二卷

史旌賢《維風編》二卷

鄒期禎《東林諸賢言行錄》五卷已上皆紀明代人物。

《相鑒》二十卷

洪武十三年罷中書省,詔儒臣採歷代史所載相臣,賢者自蕭何至文天祥八十二人,為傳十六卷;不肖者自田分至賈似道二十六人,為傳四卷。太祖製序。

《外戚傳》三卷永樂中,編輯漢以後可為法戒者。成祖製序。

《古今列女傳》三卷永樂中解縉等編。

宋濂《唐仲友補傳》一卷,《浦江人物記》二卷

胡廣《文丞相傳》一卷

朱右《李鄴侯傳》二卷

方槐生《莆陽人物志》三卷

謝應芳《懷古錄》三卷,《思賢錄》六卷

劉徵《金華名賢傳》三卷

丁元吉《陸丞相蹈海錄》一卷

賈斌《忠義集》四卷

尹真《南宋名臣言行錄》十六卷

楊循吉《吳中往哲記》一卷

謝鐸《尊鄉錄》十卷

董遵《金華淵源錄》二卷

金江《義烏人物志》二卷

金賁亨《台學源流》二卷

王佐《東嘉先哲錄》二十卷

南逢吉《越中述傳》四卷

周璟《昭忠錄》五卷

程敏政《宋遺民錄》十五卷

方鵬《崑山人物志》八卷

姜絅《漢名臣言行錄》八卷

魏顯國《歷代相臣傳》一百六十八卷,《守令傳》二十四卷,《儒林傳》二十卷

陳鎬《金陵人物志》六卷

王賓《吳下名賢紀錄》一卷

龔守愚《臨江先哲言行錄》二卷

劉元卿《江右歷代名賢錄》二卷

黃佐《廣州人物志》二十四卷

劉有光《麻沙劉氏忠賢傳》四卷

孫承恩《歷代聖賢像贊》六卷

楊時偉《諸葛武侯全書》十卷

王承裕《李衛公通纂》四卷

戴銑《朱子實紀》十二卷

祝允明《蘇材小纂》六卷

張褲《吳中人物志》十三卷

袁BM《吳中先賢傳》十卷

劉鳳《續吳先賢贊》十五卷

歐大任《百粵先賢志》四卷

耿定向《二孝子傳》一卷

楊俊民《河南忠臣集》八卷,《烈女集》五卷

桑喬《節義林》六卷

王冥《歷代忠義錄》十八卷

鄒泉《人物尚論編》二十卷

鄭瑄《唐忠臣睢陽錄》二卷

黃省曾《高士傳頌》二卷

皇甫濂《逸民傳》二卷

皇甫涍《續高士傳》十卷

薛應旂《隱逸傳》二卷,《高士傳》四卷

黃姬水《貧士傳》二卷

錢一本《遁世編》十四卷

李默《建寧人物志》三卷

呂維祺《節孝義忠集》四卷

徐標《忠孝廉節集》四十卷

顧憲成《桑梓錄》十卷

李廷機《漢唐宋名臣錄》五卷

王鴻儒《掾曹名臣錄》一卷

丁明登《古今長者錄》八卷

朱睦挈《中州人物志》十六卷

朱謀韋《豫章耆舊傳》三卷

朱常淓《古今宗籓懿行考》十卷

郭良翰《歷代忠義匯編》二十六卷

屠隆《義士傳》二卷

沈堯中《高士匯林》二卷

顧樞《古今隱居錄》三十卷

陳懋仁《壽者傳》三卷

陳繼儒《邵康節外紀》四卷,《逸民史》二十二卷

璩之璞《蘇長公外紀》十二卷

徐勃《蔡端明別紀》十卷

范明泰《米襄陽志林》十三卷

徐學聚《兩浙名賢錄》五十四卷,《外錄》八卷

曹學牷《蜀中人物記》六卷

郭凝之《孝友傳》二十四卷

王道隆《吳興名賢續錄》六卷

陳克仕《古今彤史》八卷

曹思學《內則類編》四卷

顧昱《至孝通神集》三十卷

張采《宋名臣言行錄》十六卷

夏樹芳《女鏡》八卷

潘振《古今孝史》十二卷

已上皆紀歷代人物。

──右傳記類,一百四十四部,一千九百九十七卷。

《大明志書》洪武三年詔儒士魏俊民等類編天下州郡地理形勢、降附顛末為書。卷亡。

《寰宇通志》一百十九卷景泰中修。

《一統志》九十卷天順中李賢等修。

《承天大志》四十卷嘉靖中,顧璘修《興都志》二十四卷。世宗以其載獻帝事實,於志體例不合,詔徐階等重修。

桂萼《歷代地理指掌》四卷,《明輿地指掌圖》一卷

羅洪先《增補朱思本廣輿圖》二卷

蔡汝楠《輿地略》十一卷

吳龍《郡縣地理沿革》十五卷

盧傳印《職方考鏡》六卷

張天復《皇輿考》十二卷

蔡文《職方鈔》十卷

曹嗣榮《輿地一覽》十五卷

郭子章《郡縣釋名》十六卷,《古今郡國名類》三卷

項篤壽《考定輿地圖》十卷

徐樞《寰宇分合志》八卷

曹學牷《一統名勝志》一百九十八卷

陸應陽《廣輿記》二十四卷

陳組綬《職方地圖》三卷

張元陽《方隅武備》十六卷一作《方隅武事考》。

龐迪我《海外輿圖全說》二卷

劉崧《北平八府志》三十卷,《北平事蹟》一卷

郭造卿《燕史》一百二十卷

劉侗《帝京景物略》八卷

孫國敉《燕都游覽志》四十卷

蔣一葵《長安客話》八卷

沈應文《順天府志》六卷

唐舜卿《涿州志》十二卷

汪浦《薊州志》九卷

張欽《保定府志》二十五卷

潘恩《祁州志》六卷

戴詵《易州志》三十卷

樊文深《河間府志》二十八卷

廖紀《滄州志》四卷

項喬《董子故里志》六卷

雷禮《真定府志》三十二卷

倪璣《定州志》四卷

曹安《冀州志》四卷

陳棐《廣平府志》十六卷

宋訥《東郡志》十六卷

唐錦《大名府志》二十八卷

王崇慶《開州志》十卷

張廷綱《永平府志》十一卷

陳士元《灤州志》十一卷

胡文璧《天津三衛志》十卷

馬中錫《宣府志》十卷

畢恭《遼東志》九卷

李輔《重修遼東志》十二卷

《洪武京城圖志》一卷

陳沂《南畿志》六十四卷,《金陵世紀》四卷,《金陵古今圖考》一卷

顧起元《客座贅語》十卷

王兆雲《烏衣佳話》八卷

周暉《金陵瑣事》八卷,《剩錄》八卷

《留都錄》五卷見國子監書目,不著撰人。

程嗣功《應天府志》三十二卷

柳瑛《中都志》十卷

袁又新《鳳陽新書》八卷

汪應軫《泗州志》十二卷

王浩《亳州志》十卷

呂景蒙《潁州志》二十卷

潘鏜《廬陽志》三十卷

楊循吉《廬陽客記》一卷

潘塤《淮郡文獻志》二十六卷

陳文燭《淮安府志》十六卷

高宗本《揚州府志》十卷

沈明臣《通州志》八卷

張珩《高郵州志》三卷

陳奇《泰州志》八卷

盧熊《吳邦廣記》五十卷

劉昌《蘇州續志》一百卷

王鏊《姑蘇志》六十卷

劉鳳《續吳錄》二卷,《吳郡考》二卷

桑悅《太倉州志》十一卷

錢岡《雲間通志》十八卷

顧清《松江府志》三十二卷

陳繼儒《松江府志》九十四卷

謝應芳《毘陵續志》十卷

王人與《毘陵志》四十卷

張愷《常州府志續集》八卷

唐鶴徵《常州府志》二十卷

沈敕《荊溪外紀》二十五卷

王樵《鎮江府志》三十六卷

胡纘宗《安慶府志》三十一卷

鐘城《太平府志》二十卷

李默《寧國府志》十卷

王崇《池州府志》九卷

朱同《新安志》十卷

程敏政《新安文獻志》一百卷

何東序《徽州府志》二十二卷

程一枝《鄣大事記》二卷

李德陽《廣德州志》十卷

陳璉《永陽志》二十六卷

胡松《滁州志》四卷

周斯盛《山西通志》三十三卷

張欽《大同府志》十八卷

呂柟《解州志》四卷

孔天胤《汾州府志》八卷

栗應麟《潞安府志》十二卷

周弘禴《代州志》二卷

陸釴《山東通志》四十卷

黃瓚《齊魯通志》一百卷

彭勖《山東郡邑勝覽》九卷

李錦《泰安州志》十卷

邢侗《武定州志》十五卷

于慎行《兗州府志》五十二卷

莫驄《濟寧州志》十三卷

舒祥《沂州志》四卷

李玨《東昌府志》九卷

鄧AX《濮州志》十卷

周禧《臨清州志》十八卷

馮惟訥《青州府志》十八卷

李時颺《少陽乘》二十卷

鐘羽正《青州風土記》四卷

任順《莒州志》六卷

潘滋《登州府志》十卷

楊循吉《寧海州志》二卷

胡杞忠《萊州府志》八卷

郭維洲《平度州志》二卷

胡諲《河南總志》十九卷

鄒守愚《河南通志》四十五卷

李濂《汴京遺迹志》二十四卷,《祥符文獻志》十七卷

朱睦挈《中州文獻志》四十卷,《開封府志》八卷

邵寶《許州志》三卷

馮相《陳州志》四卷

吳三樂《鄭州志》六卷

徐衍祥《禹州志》十卷萬歷中,鈞州改曰禹州。

李嵩《歸德府志》八卷

李孟暘《睢州志》一卷

程應登《睢州志》七卷

崔銑《彰德府志》八卷一名《鄴乘》。

郭朴《續志》三卷

劉湜《磁州志》四卷

李遇春《衛輝府志》七卷

何瑭《懷慶府志》十二卷

喬縉《河南郡志》四十二卷

程緒《陜州志》十卷

葉珠《南陽府志》十卷

張仙《鄧州志》六卷

牛孟耕《裕州志》六卷

陳鑾《汝寧府志》八卷

李本固《汝南新志》二十二卷

江貴《信陽州志》二卷

張輝《光州志》十卷

方應選《汝州志》四卷

伍福《陜西通志》三十五卷成化中修。

馬理《陜西通志》四十卷嘉靖中修。

何景明《雍大記》三十六卷

李應祥《雍勝略》二十四卷

南軒《關中文獻志》八十卷

宋廷佐《乾州志》二卷

喬世寧《耀州志》十一卷

任慶雲《商州志》八卷

周易《鳳翔府志》五卷

賈鳳翔《鳳翔府歷代事跡紀略》二卷

范文光《豳風考略》三卷

趙時春《平涼府志》十三卷

胡纘宗《漢中府志》十卷,《鞏郡記》三十卷,《秦州志》三十卷

熊爵《臨洮府志》十卷

韓鼎《慶陽府志》十卷

胡汝礪《寧夏新志》八卷

鄭汝璧《延綏鎮志》八卷

楊寧《固原州志》二卷

李泰《蘭州志》十二卷

張最《岷州衛志》一卷

李璣《洮州衛志》五卷

郭伸《甘州衛志》十卷

朱捷《河州志》四卷

包節《陜西行都司志》十二卷

孟秋《潼關衛志》十卷

王崇古《莊浪漫記》八卷

薛應旂《浙江通志》七十二卷

夏時正《杭州府志》六十四卷成化中修。

陳善《杭州府志》一百卷,《外志》一卷、全郡山川原委。《武林風俗略》一卷

吳瓚《武林紀事》八卷

柳琰《嘉興府志》三十二卷

李日華《韋李叢談》四卷

江翁儀《湖州府志》二十四卷

徐獻忠《吳興掌故集》十七卷

江一麟《安吉州志》八卷

李德恢《嚴州府志》二十三卷

吳堂《富春志》六卷

徐與泰《金華文獻志》二十二卷

吾哻《衢州府志》十四卷

何鏜《括蒼志》五十五卷,《括蒼匯紀》十五卷

樓公璩《括蒼志補遺》四卷

司馬相《越郡志略》十卷

張元忭《紹興府志》六十卷

李堂《四明文獻志》十卷

張時徹《寧波府志》四十二卷

范理《天台要略》八卷

謝鐸《赤城新志》二十三卷

王啟《赤城會通記》二十卷

李漸《三台文獻志》二十三卷

王瓚《溫州府志》二十三卷

林庭昂《江西通志》三十七卷

王宗沐《江西大志》八卷

趙秉忠《江西輿地圖說》一卷

王世懋《饒南九三郡輿地圖說》一卷

郭子章《註豫章古今記》一卷,《豫章雜記》八卷,《廣豫章災祥記》六卷

盧廷選《南昌府志》五十卷

江汝璧《廣信府志》二十卷

王時槐《吉安府志》二十六卷

郭子章《吉志補》二十卷

熊相《瑞州府志》十四卷

陳定《袁州府志》九卷

餘文龍《贛州府志》二十卷

虞愚《虔臺志》十二卷

談愷《虔臺續志》五卷

魏裳《湖廣通志》九十八卷

廖道南《楚紀》六十卷

陳士元《楚故略》二十卷

郭正域《武昌府志》六卷

朱衣《漢陽府志》三卷

曹璘《襄陽府志》二十卷

謝澭《均州志》八卷

顏木《隨州志》二卷

舒旌《黃州府志》十卷

甘澤《蘄州志》九卷

王寵懷《荊州府志》十二卷

張春《夷陵州志》十卷

劉璣《岳州府志》十卷

張治《長沙府志》六卷

陸東《寶慶府志》五卷

楊佩《衡州府志》九卷

朱麟《常德府志》二卷

胡靖《沅州志》七卷

姚昺《永州府志》十卷

林球《荊門州志》十卷

童承敘《沔陽州志》十八卷

周紹稷《鄖陽府志》二十一卷

王心《郴州志》六卷

黃仲昭《八閩通志》八十七卷,《邵武府志》二十五卷

王應山《閩大記》五十五卷,《閩都記》三十二卷

何喬遠《閩書》一百五十四卷

王世懋《閩部疏》一卷

陳鳴鶴《閩中考》一卷,《晉安逸志》三卷

林燫《福州府志》三十六卷

林材《福州府志》七十六卷

周瑛《興化府志》五十四卷

鄭岳《莆陽文獻志》七十五卷

黃鳳翔《泉州府志》二十四卷

何炯《清源文獻志》八卷

陳懋仁《泉南雜記》二卷

徐鑾《漳州府志》三十八卷

劉璵《建寧府志》六十卷

游居敬《延平府志》三十四卷

張大光《福寧州志》十六卷

王元正《四川總志》八十卷

楊慎《全蜀藝文志》五十四卷

杜應芳《補蜀藝文志》五十四卷

郭棐《四川通志》三十六卷,《夔州府志》十二卷,《夔記》四卷

曹學牷《蜀漢地理補》二卷,《蜀郡縣古今通鐸》四卷,《蜀中風土記》四卷,《方物記》十二卷

彭韶《成都志》二十五卷

周洪謨《敘州府志》十二卷

金光《涪州志》二卷

陳嘉言《嘉州志》十卷

餘承勛《西眉郡縣志》十卷

戴璟《廣東通志》七十二卷

郭棐《粵大記》三十二卷,《嶺南名勝志》十六卷

謝肇淛《百粵風土記》一卷

張邦翼《嶺南文獻志》十二卷,《補遺》六卷

馬焱《南粵概》四卷

黃佐《廣州府志》二十二卷,《香山志》八卷

鄭敬甫《惠大記》六卷

郭春震《潮州府志》八卷

郭子章《潮中雜記》十二卷

符錫《韶州府志》十卷

葉春及《肇慶府志》二十卷

王佐《瓊臺外紀》五卷,《珠崖錄》五卷

顧玠《海差餘錄》一卷

張詡《厓門新志》十八卷

周孟中《廣西通志》六十卷

魏浚《西事珥》八卷,《嶠南瑣記》二卷

陳璉《桂林志》三十卷

張鳴鳳《桂故》八卷,《桂勝》十四卷

謝少南《全州志》七卷

黨緒《思恩府志》四卷

田秋《思南府志》八卷

郭棐《右江大志》十二卷

《雲南志書》六十一卷洪武十四年既平雲南,詔儒臣考定為書。

李元陽《雲南通志》十八卷,《大理府志》十卷

陳善《滇南類編》十卷

楊慎《滇程記》一卷

彭汝實《六詔紀聞》一卷

楊鼐《南詔通志》十卷

諸葛元聲《滇史》十四卷

吳懋《葉榆檀林志》八卷

楊士雲《黑水集證》一卷,《郡大記》一卷

趙瓚《貴州新志》十七卷

謝東山《貴陽圖考》二十六卷

郭子章《黔記》六十卷,《黔小志》一卷

祁順《石阡府志》十卷

袁表《黎平府志》九卷

周瑛《興隆衛志》二卷

許論《九邊圖論》三卷

魏煥《九邊通考》十卷

霍冀《九邊圖說》一卷

范守己《籌邊圖說》三卷

劉效祖《四鎮三關志》十二卷

蘇祐《三關紀要》三卷

劉昌《兩鎮邊關圖說》二卷

翁萬達《宣大山西諸邊圖》一卷

楊守謙《大寧考》一卷,《紫荊考》一卷,《花馬池考》一卷

楊一葵《雲中邊略》四卷

楊時寧《大同鎮圖說》三卷,《大同分營地方圖》一卷

張雨《全陜邊政考》十二卷

劉敏寬《延鎮圖說》二卷

楊錦《朔方邊紀》五卷

詹榮《山海關志》八卷

莫如善《威茂邊政考》五卷

方孔炤《全邊略記》十二卷

胡宗憲《籌海圖編》十三卷

黃光昇《海塘記》一卷

仇俊卿《海塘錄》十卷

鄭若曾《萬里海防圖論》二卷,《江南經略》八卷

王在晉《海防纂要》十三卷

謝廷傑《兩浙海防類考》十卷

范淶《續編》十卷

李如華《溫處海防圖略》二卷

安國賢《南澳小記》十二卷,《南日寨小記》十卷

吳時來《江防考》六卷

洪朝選《江防信地》二卷

吳道南《國史河渠志》二卷

劉隅《治河通考》十卷

劉天和《問水集》六卷

吳山《治河通考》十卷

潘季馴《河防一覽》十四卷,《宸斷大工錄》十卷

潘大復《河防榷》十二卷

張光孝《西瀆大河志》六卷

黃克纘《疏治黃河全書》二卷

徐標《河患備考》二卷,《河防律令》二卷

王恕《漕河通志》十四卷

王瓊《漕河圖志》八卷

車璽《漕河總考》四卷

顧寰《漕河總錄》二卷

高捷《漕黃要覽》二卷

黃承玄《河漕通考》四十五卷,《安平鎮志》十一卷,《北河紀略》十四卷

秦金《通惠河志》二卷

謝肇淛《北河紀》八卷,《紀餘》四卷

朱國盛《南河志》十四卷

陳夢鶴《濟寧閘河類考》六卷

徐源《山東泉志》六卷

王寵《東泉志》四卷,《濟寧閘河志》四卷

張純《泉河紀略》八卷

胡瓚《泉河史》十五卷

張橋《泉河志》六卷

馮世雍《呂梁洪志》一卷

陳穆《徐州洪志》十卷

袁黃《皇都水利》一卷

伍餘福《三吳水利論》一卷

歸有光《三吳水利錄》四卷

許應夔《修舉三吳水利考》四卷

王道行《三吳水利考》二卷

王圻《東吳水利考》十卷

沈摐《吳江水利考》四卷

賈應璧《紹興水利圖說》二卷

何鏜《名山記》十七卷

慎蒙《名山一覽記》十五卷

都穆《遊名山記》六卷

黃以升《遊名山記》六卷

查志隆《岱史》十八卷

宋燾《泰山紀事》十二卷

燕汝靖《嵩獄古今集錄》二卷

李時芳《華獄全集》十卷

婁虛心《北嶽編》五卷

王浚和《恒嶽志》二卷

彭簪《衡岳志》八卷

孫存《岳麓書院圖志》一卷

《太岳太和山志》十五卷洪熙中道士任自垣編。

葛寅亮《金陵梵剎志》五十二卷

張萊《京口三山志》十卷

劉大彬《茅山志》十五卷

王鏊《震澤編》八卷

盧雍《石湖志》十卷

談修《惠山古今考》十卷

潘之恒《新安山水志》十卷,《黃海》二十九卷

方漢《齊雲山志》七卷

汪可立《九華山志》二卷

吳之鯨《武林梵剎志》十二卷

田汝成《西湖遊覽志》二十四卷

張元忭《雲門志略》五卷

周應賓《普陀山志》五卷

僧傳燈《天台山志》二十九卷

朱諫《雁山志》四卷

桑喬《廬山紀事》十二卷

劉俊《白鹿洞書院志》六卷

楊亙《武夷山志》六卷

黃天全《九鯉湖志》六卷

劉中藻《洞山九潭志》四卷

喬世寧《五臺山志》一卷

李應奇《崆峒志》二卷

僧德清《曹溪志》四卷

左宗郢《麻姑山志》十七卷

陳璉《羅浮志》十五卷

謝肇淛《支提山志》七卷,《鼓山志》十二卷

楊士奇《北京紀行錄》二卷

劉定之《代祀錄》一卷

陸深《停驂錄》二卷

李東陽《東祀錄》三卷

張寧《奉使錄》二卷

李思聰《百夷傳》一卷洪武中出使緬國所紀。

費信《星槎勝覽集》二卷,《天心紀行錄》一卷永樂中從鄭和使西洋所紀。

陳誠《西域行程記》二卷

馬歡《瀛涯勝覽》一卷

倪謙《朝鮮紀事》一卷,《遼海編》四卷

錢溥《朝鮮雜志》三卷,《使交錄》一卷

黃福《安南水程日記》二卷

龔用卿《使朝鮮錄》三卷

謝傑《使疏球錄》六卷

李文鳳《粵嶠書》二十卷經安南事。

黃省曾《西洋朝貢典錄》二卷

張燮《東西洋考》十二卷

李言恭《日本考》五卷

侯繼高《日本風土記》四卷

卜大同《備倭國記》四卷,《征苗圖記》一卷

田汝成《炎徼紀聞》四卷

寧獻王權《異域志》一卷

嚴從簡《殊域周咨錄》二十四卷

羅曰褧《咸賓錄》八卷

茅瑞徵《象胥錄》八卷

尹耕《譯語》一卷

艾儒略《職方外紀》五卷

右地理類,四百七十一部,七千四百九十八卷。

《天潢玉牒》一卷,《宗支》二卷男女各一冊、《宗譜》一卷,《主婿譜牒》一卷已上皆明初修。

朱睦挈《帝系世表》一卷,《周國世系表》一卷,《周乘》一卷,《鎮平世系錄》二卷

《周憲王年表》二卷

《周定王年表》一卷

《楚王宗支》一卷

《蜀府宗支圖譜》一卷

朱宙枝《統宗繩蟄錄》十二卷

吳震元《宋相譜》二百卷

朱右《邾子世家》一卷

盧熊《孔顏世系譜》二卷

楊廉《二程年譜》一卷

李默《硃子年譜》四卷

徐渤《蔡忠惠年譜》一卷

郭勛《三家世典》一卷輯徐達、沐英、郭英三家世系勛伐本末。

《中山徐氏世系錄》一卷

《李韓公家乘》一卷

李臨淮《遐思錄》八卷

吳沈《千家姓》一卷

楊信民《姓源珠璣》六卷

邢參《姓氏匯典》二卷

楊慎《希姓錄》五卷

王文翰《尚古類氏集》十二卷

李日華《姓氏譜纂》七卷

曹宗儒《郡望辨》二卷

陳士元《姓匯》四卷,《姓觿》二卷,《名疑》四卷

凌迪知《歷代帝王姓系統譜》六卷,《姓氏博考》十四卷,《萬姓統譜》一百四十卷

餘寅《同姓名錄》十二卷

鄧名世《古今姓氏書辨證》四十卷

──右譜牒類,三十八部,五百四卷。

子類十二:一曰儒家類,二曰雜家類,前代藝文志列名法諸家,然寥寥無幾,備數而已。今總附雜家。三曰農家類學及陽明學》等。,四曰小說家類,五曰兵書類,六曰天文類,七曰歷數類,八曰五行類,九曰藝術類,醫書附。十曰類書類,十一曰道家類,十二曰釋家類。

《聖學心法》四卷永樂中編,為類四:曰君道、臣道、父道、子道。成祖制序。

《性理大全》七十卷永樂中,既命胡廣等纂修《經書大全》,又以周、程、張、硃諸儒性理之書類聚成編。成祖制序。

《傳心要語》一卷,《孝順事實》十卷,《為善陰騭》十卷皆永樂中編

《五倫書》六十二卷宣宗採經傳子史嘉言善行為是書。正統中,英宗製序刊行。

憲宗《文華大訓》二十八卷綱四,目二十有四,成化中編。嘉靖中,世宗製序刊行。

世宗《敬一箴》一卷,《注程子四箴》、《注範浚心箴》共二卷

孫作《東家子》一卷

葉儀《潛書》一卷

留睿《留子》一卷

葉子奇《太玄本旨》九卷

硃右《性理本原》三卷

張九韶《理學類編》八卷

謝應芳《辯惑編》四卷

周是修《綱常彞範》十二卷

曹端《理學要覽》二卷,《夜行燭》一卷,《月川語錄》一卷

尤文《語錄》二卷

鮑寧《天原發微辨正》五卷

金潤《心學探微》十二卷

吳與弼《康齋日錄》一卷

薛瑄《讀書錄》十卷,《續錄》十卷

周洪謨《南皋子雜言》二卷,《箐齋讀書錄》二卷

胡居仁《居業錄》八卷

謝鐸《伊洛淵源續錄》六卷

程敏政《道一編》五卷

蔡清《性理要解》二卷

楊廉《伊洛淵源錄類增》十四卷,《畏軒劄記》三卷

張吉《陸學訂疑》二卷

章懋《楓山語錄》二卷

周木《延平問答續錄》一卷

楊守阯《困學寡聞錄》十卷

韓邦奇《性理三解》八卷

王鴻漸《讀書記》二卷

王輶《大儒心學錄》二十七卷

徐問《讀書劄記》八卷,《續記》八卷

方鵬《觀感錄》十二卷

魏校《莊渠全書》十卷

陳獻章《言行錄》十卷,《附錄》二卷

趙鶴《金華正學編》十卷

王守仁《傳習錄》四卷,《陽明則言》二卷

羅欽順《困知記》六卷,《附錄》二卷

陳建《學蔀通辨》十卷

許贊《性學編》一卷,《道統溯流錄》一卷

湛若水《甘泉明論》十卷,《遵道錄》十卷,《問辨錄》六卷

黃佐《泰泉庸言》十二卷

呂柟《涇野子內篇》三十三卷,《語錄》二十卷

鄒守益《道南三書》三卷,《明道錄》四卷

何瑭《柏齋三書》四卷

薛蕙《日錄》五卷

顧應祥《惜陰錄》十二卷

沈霽《語錄》四卷

邵經邦《弘道錄》五十七卷

唐順之《儒編》六十卷

薛應旂《考亭淵源錄》二十四卷,《薛子庸語》十卷

王艮《心齋語錄》二卷

周思兼《學道記言》六卷

胡直《胡子衡齊》八卷

陸樹聲《汲古叢語》一卷

金賁亨《道南錄》五卷,《台學源流集》七卷

尤時熙《擬學小記》八卷

劉元卿《諸儒學案》八卷

周琦《東溪日談》十八卷

羅汝芳《明道錄》八卷,《近溪集語》十二卷

耿定向《庸言》二卷,《雅言》一卷,《新語》一卷,《教學商求》一卷

李渭《先行錄》十卷

王樵《劄記》一卷,《筆記》一卷

許孚遠《語要》二卷

朱衡《道南源委錄》十二卷

孫應鰲《論學匯編》八卷

梁斗輝《聖學正宗》二十卷

管志道《問辨牘》八卷,《理學酬咨錄》八卷

王敬臣《俟後編》四卷

呂坤《呻吟語》四卷

鄒德溥《畏聖錄》二卷

鄧球《理學宗旨》二卷

李材《教學錄》十二卷,《南中問辨錄》十卷

曾朝節《臆言》八卷

鄒元標《仁文會語》四卷,《日新編》二卷

楊起元《證學編》二卷,《識仁編》二卷

徐即登《儒學明宗錄》二十五卷

黃時熠《知非錄》六卷

錢一本《黽記》四卷

顧憲成《劄記》十八卷,《東林商語》二卷,《證性編》八卷,《當下繹》一卷,《涇陽遺書》二十卷

李多見《學原前後編》八卷

塗宗浚《證學記》三卷

周子義《日錄見聞》十卷

吳仕期《大儒敷言》三十三卷

徐三重《信古餘論》八卷

來知德《日錄》十二卷

方學漸《心學宗》四卷

姚舜牧《性理指歸》二十八卷

馮從吾《元儒考略》四卷,《語錄》六卷

唐鶴徵《憲世編》六卷

曾鳳儀《明儒見道編》二卷

周汝登《聖學宗傳》十八卷

高攀龍《就正錄》二卷,《高子遺書》十二卷

孫慎行《困思抄》四卷

劉宗周《理學宗要》一卷,《證人要旨》一卷,《劉子遺書》四卷

葉秉敬《讀書錄鈔》八卷

黃道周《榕壇問業》十八卷

章世純《留書》十卷

黃淳耀《吾師錄》一卷,《語錄》一卷,《劄記》二卷

──右儒家類,一百四十部,一千二百三十卷。

太祖《資治通訓》一卷,凡十四章,首君道,次臣道,又次民用、士用、工用、商用,皆著勸導之意。《公子書》一卷,訓世臣。《務農技藝商賈書》一卷訓庶民子弟。

成祖《務本之訓》一卷採太祖創業事跡及往古興亡得失為書,以訓太孫

仁孝皇后《勸善書》二十卷

宋濂《燕書》一卷

王廉《迂論》十卷

葉子奇《草木子》八卷

王達《筆疇》二卷

曹安《讕言長語》二卷

趙弼《事物紀原刪定》二十卷

解延年《物類集說》三十四卷

羅頎《梅山叢書》二百卷,《物原》二卷

謝理《東岑子》四卷

潘府《南山素言》一卷

何孟春《餘冬序錄》六十五卷,《閒日分義》一百卷

戴鱀《經濟考略》二十卷

戴璟《博物策會》十七卷

陸深《同異錄》一卷,《傳疑錄》二卷

孫宜《遁言》二卷

祝允明《前聞記》一卷,《讀書筆記》一卷

蔡羽《太藪外史》五卷

劉繪《劉子通論》十卷

高岱《楚漢餘談》一卷

羅虞臣《原子》八卷

王傑《經濟總論》十卷

汪坦《日知錄》五卷

劉鳳《劉子雜組》十卷

王世貞《劄記》二卷,《宛委餘編》十九卷

王可大《國憲家猷》五十六卷萬歷中,御史言內閣絲綸簿猝無可考,惟是書載之。遂取以進。

沈津《百家類纂》四十卷

陳耀文《學圃萱蘇》六卷,《學林就正》四卷

陳絳《金罍子》四十四卷

方弘靜《千一錄》二十六卷

勞堪《史編始事》二卷

陳其力《芸心識餘》八卷

周祈《名義考》十二卷

詹景鳳《詹氏小辨》六十四卷

穆希文《說原》十六卷,《動植記原》四卷

王三聘《事物考》八卷

徐常吉《諸家要旨》二卷

徐伯齡《蟫精雋》二十卷

趙士登《省身至言》十卷

劉仕義《知新錄》二十四卷

屠隆《冥寥子》二卷,《鴻苞》四十八卷

閔文振《異物類苑》五卷

朱謀韋《玄覽》八卷

趙樞生《含玄子》十六卷,《別編》十卷

吳安國《纍瓦編》三十二卷

馮應京《經世實用編》二十八卷

柯壽愷《語叢》三十八卷

徐三重《鴻洲雜著》十八卷

王納諫《會心言》四卷

沈節甫《紀錄匯編》二百十六卷

祁承業《國朝徵信錄》二百十二卷,《淡生堂餘苑》六百四卷

董斯張《廣博物志》五十卷

鄭瑄《昨非庵日纂》六十卷

──右雜家類,六十七部,二千二百八十四卷。

劉基《多能鄙事》十二卷

周定王《救荒本草》四卷

寧獻王《臒仙神隱書》四卷

楊溥《水雲錄》二卷

周履靖《茹草編》四卷

鄺璠《便民圖纂》十六卷

顧清《田家月令》一卷

施大經《閱古農書》六卷

俞貞木《種樹書》三卷

溫純《齊民要書》一卷

王世懋《學圃雜疏》三卷

黃省曾《稻品》一卷,《蠶經》一卷

李德紹《樹藝考》二卷

袁黃《寶坻勸農書》二卷

陳鳴鶴《田家月令》一卷

宋公望《四時種植書》一卷

馮應京《月令廣義》二十四卷

王象晉《群芳譜》二十八卷

徐光啟《農政全書》六十卷,《農遺雜疏》五卷

張國維《農政全書》八卷

吳嘉言《四季須知》二卷

──右農家類,二十三部,一百九十一卷。

宋濂《蘿山雜言》一卷

葉子奇《草木子餘錄》三卷

陶宗儀《輟耕錄》三十卷,《說郛》一百二十卷又有《續說郛》四十六卷,明季人陶珽纂輯。

劉績《霏雪錄》二卷

陶輔《桑榆漫筆》一卷

瞿佑《香臺集》三卷

張綸《林泉隨筆》一卷

李賢《古穰雜錄》二卷

岳正《類博雜言》二卷

葉盛《水東日記》三十八卷

單宇《菊坡叢話》二十六卷

陸容《菽園雜記》十五卷

姚福《青谿暇筆》二十卷

張志淳《南園漫錄》十卷,《續錄》十卷

梅純《續百川學海》一百卷

王錡《寓圃雜記》十卷

羅鳳《漫錄》三十卷

李詡《漫筆》八卷

徐泰《玉池談屑》四卷

羅欽德《閒中瑣錄》二卷

王渙《墨池瑣錄》三卷

沈周《客坐新聞》二十二卷

都邛《三餘贅筆》二卷

都穆《奚囊續要》二十卷

徐禎卿《異林》一卷

唐錦《龍江夢餘錄》四卷

戴冠《筆記》十卷

侯甸《西樵野記》十卷

陸粲《庚巳編》十卷

陸深《儼山外集》四十卷

馬攀龍《株守談略》四卷

陸采《天池聲雋》四十卷

胡侍《野談》六卷

楊慎《丹鉛總錄》二十七卷,《續錄》十二卷,《餘錄》十七卷,《新錄》七卷,《閏錄》九卷,《卮言》四卷《談菀醍醐》九卷,《藝林伐山》二十卷,《墐戶錄》一卷,《清暑錄》二卷

陸楫《古今說海》一百四十二卷

陳霆《兩山墨談》十八卷

司馬泰《廣說郛》八十卷,《古今匯說》六十卷,《再續百川學海》八十卷,《三續三十卷,《史流十品》一百卷

王文祿《明世學山》五十卷

尤鏜《紅箱集》五十卷

朱應辰《漫鈔》十卷

李文鳳《月山叢談》十卷

何良俊《語林》三十卷,《叢說》三十八卷

沈儀《麈談錄》十卷

萬表《灼艾集》十卷

高鶴《見聞搜玉》八卷

項喬《甌東私錄》六卷

張時徹《說林》二十四卷

袁褧《前後四十家小說》八十卷,《廣四十家小說》四十卷

陸樹聲《清暑筆談》一卷,《長水日鈔》一卷,《耄餘雜識》一卷

徐伯相《畫暇叢記》二十卷

姚弘謨《錦囊瑣綴》八卷

陳師《筆談》十五卷

石磐《菊徑漫談》十四卷

郎瑛《七修類稿》五十一卷

朱國禎《涌幢小品》二十四卷

李豫亨《自樂編》十六卷

徐渭《路史》二卷

汪雲程《逸史搜奇》十卷

孫能傳《剡溪漫筆》六卷

王應山《風雅叢談》六十卷

陳禹謨《說麈》八卷

田藝蘅《留青日札》三十九卷,《西湖志餘》二十六卷

胡應麟《少室山房筆叢》三十二卷,《續》十六卷

林茂槐《說類》六十二卷

焦竑《筆乘》二十卷,《玉堂叢語》八卷,《明世說》八卷

黃汝良《筆談》十二卷

朱謀韋《異林》十六卷

湯顯祖《續虞初志》八卷

張鼎思《瑯琊代醉編》四十卷

屠本畯《山林經濟籍》二十四卷

顧起元《說略》六十卷

王肯堂《鬱岡齋筆麈》四卷

董其昌《畫禪室隨筆》二卷

商浚《稗海》三百六十八卷

謝肇淛《五雜組》十六卷,《麈餘》四卷,《文海披沙》八卷

徐勃《徐氏筆精》八卷

王兆雲《驚座新書》八卷,《王氏青箱餘》十二卷

張所望《閱耕餘錄》六卷

郭良翰《問奇類林》三十六卷

陳繼儒《秘笈》一百三十卷

潘之恆《亙史鈔》九十一卷

王學海《筠齋溫錄》十卷

李日華《六研齋筆記》十二卷,《日記》二十卷

包衡《清賞錄》十二卷

張重華《娛耳集》十二卷

馬應龍《藝林鉤微錄》二十四卷

李紹文《明世說新語》八卷

張大復《筆談》十四卷

徐應秋《談薈》三十六卷

楊崇吾《檢蠹隨筆》三十卷

來斯行《槎庵小乘》四十六卷

沈弘正《蟲天志》十卷

胡震亨《讀書雜錄》三卷

閔元京《湘煙錄》十六卷

茅元儀《雜記》三十二卷

華繼善《咫聞錄》五卷

王所《日格類鈔》三十卷

王勣《纂言鉤玄》十六卷

楊德周《隨筆》十二卷

吳之俊《獅山掌錄》二十八卷

──右小說家類,一百二十七部,三千三百七卷。

劉寅《七書直解》二十六卷,《集古兵法》一卷

寧獻王權《注素書》一卷

徐昌會《握機匯鑰》六卷

陳元素《古今名將傳》十七卷

劉畿《諸史將略》十六卷

何喬新《續百將傳》四卷五代訖宋、元。

何瑭《兵論》一卷

王芑《綱目兵法》六卷

穆伯寅《兵鑒撮要》七卷

劉濂《兵說》十二卷

吳從周《兵法匯編》十二卷

唐順之《武編》十二卷,《兵垣四編》五卷

何東序《益智兵書》一百卷,《武庫益智錄》六卷

陳禹謨《左氏兵法略》三十二卷

李材《將將紀》二十四卷,《兵政紀略》五十卷,《經武淵源》十五卷

顧其言《新續百將傳》四卷一名《明百將傳》。

馮孜《古今將略》四卷

尹商《閫外春秋》三十二卷

戚繼光《紀效新書》十四卷,《練兵實紀》九卷,《雜集》六卷,《將臣寶鑒》一卷

趙本學《韜鈐內篇》一卷

俞大猷《韜鈐續篇》一卷,《劍經》一卷

葉夢熊《運籌綱目》十卷

王鳴鶴《登壇必究》四十卷

何僎《讀史機略》十卷

鄭璧《古今兵鑒》三十二卷,《經世宏籌》三十六卷

王有麟《古今戰守攻圍兵法》六十卷

姚文蔚《省括編》二十二卷

趙大綱《方略摘要》十卷

高折枝《將略類編》二十四卷

施浚明《古今紆籌》十二卷

楊惟休《武略》十卷

孫承宗《車營百八扣》一卷

徐常《陣法舉要》一卷

龍正《八陣圖演注》一卷

瞿汝稷《兵略纂聞》十二卷

茅元儀《武備志》二百四十卷

孫元化《經武全編》十卷

顏季亨《明武功紀勝通考》八卷

徐標《兵機纂要》四卷

范景文《師律》十六卷

谷中虛《水陸兵律令操法》四卷

張燾《西洋火攻圖說》一卷

王應遴《備書》二十卷

冒起宗《守筌》五卷

《講武全書兵覽》三十二卷,《兵律》三十八卷,《兵占》二十四卷

《兵機備纂》十三卷

已上四部,不知撰人。

──右兵書類,五十八部,一千一百二十二卷。

《清類天文分野書》二十四卷洪武中編,以十二分野星次分配天下郡縣,又於郡縣之下詳載古今沿革之由

《天元玉曆祥異賦》七卷仁宗製序。

葉子奇《元理》一卷

劉基《天文秘略》一卷

《觀象玩占》十卷不知撰人,或云劉基輯。楊廉《星略》一卷

王應電《天文會通》一卷

周述學《天文圖學》一卷

吳珫《天文要義》二卷

范守己《天官舉正》六卷

陸侹《天文地理星度分野集要》四卷

王臣夔《測候圖說》一卷

黃履康《管窺略》三卷

黃鐘和《天文星象考》一卷

楊惟休《天文書》四卷

潘元和《古今災異類考》五卷

趙宦光《九圜史》一卷

餘文龍《祥異圖說》七卷,《史異編》十七卷

李之藻《渾蓋通憲圖說》二卷

利瑪竇《幾何原本》六卷,《勾股義》一卷,《表度說》一卷,《圜容較義》一卷,《測量法義》一卷,《天問略》一卷,《泰西水法》六卷

熊三拔《簡平儀說》一卷,《測量異同》一卷

李天經《渾天儀說》五卷

王應遴《乾象圖說》一卷,《中星圖》一卷

陳胤昌《天文地理圖說》二卷

李元庚《乾象圖說》一卷

陳藎謨《象林》一卷

馬承勛《風纂》十二卷

魏浚《緯談》一卷

吳雲《天文志雜占》一卷

艾儒略《幾何要法》四卷

《圖注天文祥異賦》十卷

《天文玉曆璇璣經》五卷

《天文鬼料竅》一卷

《天文玉曆森羅記》十二卷

《經史言天錄》二十六卷

《嘉隆天象錄》四十五卷

《雷占》三卷

《風雲寶鑑》一卷

《天文占驗》二卷

《物象通占》十卷

《白猿經》一卷

已上十一部,皆不知撰人。

──右天文類,五十部,二百六十三卷。

劉信《曆法通徑》四卷

馬沙亦黑《回回曆法》三卷

左贊《歷解易覽》一卷

呂柟《寒暑經圖解》一卷

顧應祥《授時曆法》二卷

曾俊《曆法統宗》二卷,《曆臺撮要》二卷

周述學《曆宗通議》一卷,《中經測》一卷,《歷草》一卷

貝琳《百中經》十卷起成化甲午訖嘉靖癸巳,凡六十年。後人又續至壬戌止。

戴廷槐《革節卮言》五卷

袁黃《曆法新書》五卷

何註《曆理管窺》一卷

郭子章《枝幹釋》五卷

朱載堉《律曆融通》四卷,《音義》一卷,《萬年曆》一卷,《萬年曆備考》二卷,《曆學新說》二卷萬歷二十三年編進。

蕭懋恩《監曆便覽》二卷

邢雲路《古今律曆考》七十二卷

徐光啟《崇禎曆書》一百二十六卷《歷書總目》一卷,《日躔歷指》四卷,《日躔表》二卷,《恆星歷指》三卷,《恆星圖》一卷,《恆星圖系》一卷,《恆星歷表》四卷,《恆星經緯表》二卷,《恆星出沒表》二卷,《月離曆指》四卷,《月離表》六卷,《交食歷指》七卷,《交食表》七卷,《五緯曆指》九卷,《五緯表》十卷,《測天約說》二卷。《大測》二卷,《割圓八線表》六卷,《黃道升度表》七卷,《黃赤道距度表》一卷,《通率表》二卷,《元史揆日訂訛》一卷,《通率立成表》一卷,《散表》一卷,《測圓八線立成長表》四卷,《黃道升度立成中表》四卷,《曆指》一卷,《測量全義》十卷,《比例規解》一卷,《南北高弧表》十二卷、《諸方半晝分表》一卷,《諸方晨昏分表》一卷,《曆學小辯》一卷,《曆學日辯》五卷。崇禎二年敕光啟與李之藻、王應遴及西洋人羅雅谷等陸續成書。

羅雅谷《籌算》一卷

王英《明曆體略》三卷

何三省《曆法同異考》四卷

賈信《臺曆百中經》一卷

《曆法統宗》十二卷

《曆法集成》四卷

《經緯曆書》八卷

《七政全書》四卷

已上四部,皆不知撰人。

──右曆數類,三十一部,二百九十一卷。

劉基《玉洞金書》一卷,《注靈棋經》二卷,《解皇極經世稽覽圖》十八卷

《選擇曆書》五卷洪武中,欽天監奉敕撰定。

馬貴《周易雜占》一卷

胡宏《周易黃金尺》一卷

盧翰《中庵簽易》一卷

季本《蓍法別傳》二卷

周瑞《文公斷易奇書》三卷

蔡元谷《神易數》一卷

張其堤《易卦類選大成》四卷

王宇《周易占林》四卷

錢春《五行類應》八卷

劉均《卜筮全書》八卷

趙際隆《卜筮全書》十四卷

張濡《先天易數》二卷

周視考《陰陽定論》三卷

楊向春《皇極心易發微》六卷

蔡士順《皇極祕數占驗》一卷

吳珫《皇極經世鈐解》二卷,《太乙統宗寶鑒》二十卷,《太乙淘金歌》一卷,《六壬金鑰匙》二卷

馮柯《三極通》二卷

張乾山《古今應驗異夢全書》四卷

陳士元《夢占逸旨》八卷

張鳳翼《夢占類考》十二卷

池本理《禽遁大全》四卷,《禽星易見》四卷

鮑世彥《奇門微義》四卷,《奇門陽遁》一卷,《陰遁》一卷

劉翔《奇門遁甲兵機書》二十卷

徐之鏌《選擇禽奇盤例定局》五卷

胡獻忠《八門神書》一卷

葉容《太乙三辰顯異經》十卷

李元灃《太乙九旗曆》三卷

邢雲路《太乙書》十卷

李克家《戎事類占》二十一卷

楊瓚《六壬直指捷要》二卷

蔣日新《開雲觀月歌》一卷

黃公達《鳳髓靈文》一卷

袁祥《六壬大全》三十三卷

徐常吉《六壬釋義》一卷

黃賓廷《六壬集應鈐》六十卷

寧獻王權《肘後神樞》二卷,《運化玄樞》一卷

《歷法通書》三十卷金谿何士泰景祥《歷法》,臨江宋魯珍輝山《通書》合編。

熊宗立《金精鰲極》六卷,《通書大全》三十卷

王天利《三元節要》三卷

徐瓘《陰陽捷徑》一卷

劉最《選擇類編》八卷

萬邦孚《匯選筮吉指南》十一卷,《日家指掌》二卷,《通書纂要》六卷

何瑭《陰陽管窺》一卷

劉黃裳《元圖符藏》二卷

已上卜筮陰陽。

劉基《三命奇談》、《滴天髓》一卷

吳天洪《造命宗鏡集》十二卷

洪理《曆府大成》二十二卷

歐陽忠《星命秘訣望斗真經》三卷

楊源《星學源流》二十卷

雷鳴夏《子平管見》二卷

李欽《淵海子平大全》六卷

萬民英《三命會通》十二卷,《星學大成》十八卷

陸位《星學綱目正傳》二十卷

張果《星宗命格》十卷,《文武星案》六卷

西窗老人《蘭臺妙選》三卷

袁忠徹《古今識鑒》八卷

鮑慄之《麻衣相法》七卷

李廷湘《人相編》十二卷

已上星相。

周繼《陽宅真訣》二卷

王君榮《陽宅十書》四卷

陳夢和《陽宅集成》九卷

李邦祥《陽宅真傳》二卷

周經《陽宅新編》二卷

《陽宅大全》十卷不知撰人。

劉基《金彈子》三卷,《披肝露膽》一卷,《一粒粟》一卷,《地理漫興》三卷

趙汸《葬說》一卷

瞿佑《葬說》一卷

謝昌《地理四書》四卷

謝廷柱《堪輿管見》二卷

周孟中《地理真機》十五卷

徐善繼《人子須知》三十五卷

董章《堪輿祕旨》六卷

徐國柱《地理正宗》八卷

趙祐《地理紫囊》八卷

郭子章《校定天玉經七注》七卷

陳時暘《堪輿真諦》三卷

王崇德《地理見知》四卷

李迪人《天眼目》九卷

徐之鏌《羅經簡易圖解》一卷,《地理琢玉斧》十三卷

《地理全書》五十一卷不知撰人。

《地理天機會元》三十五卷不知撰人。

李國本《理氣祕旨》七卷,《地理形勢真訣》三十卷

徐勃《堪輿辨惑》一卷

已上堪輿。

──右五行類,一百四部,八百六十一卷。

《格古要論》十四卷洪武中曹昭撰。天順間王均增輯。

沈津《欣賞編》十卷

茅一相《續欣賞編》十卷

吳繼《墨蛾小錄》四卷

周履靖《藝苑》一百卷,《繪林》十六卷,《畫藪》九卷

朱存理《鐵網珊瑚》二十卷

朱凱《圖畫要略》一卷

都穆《金薤琳瑯》二十卷,《寓意編》一卷

唐寅《畫譜》三卷

韓昂《明畫譜》一卷

楊慎《墨池瑣錄》一卷,《書品》一卷,《斷碑集》四卷

徐獻忠《金石文》一卷

周英《書纂》五卷

程士莊《博古圖錄》三十卷

朱觀熰《畫法權輿》二卷

劉璋《明書畫史》三卷

羅周旦《古今畫鑑》五卷

李開先《中麓畫品》一卷

王勣《畫史》二十卷

王世貞《畫苑》十卷,《補遺》二卷

莫是龍《畫說》一卷

劉世儒《梅譜》四卷

王稚登《吳郡丹青志》一卷

徐勃《閩畫記》一卷

曹學牷《蜀畫苑》四卷

李日華《畫媵》二卷,《書畫想像錄》四十卷

張丑《清河書畫舫》十二卷

寧獻王權《爛柯經》一卷,《琴阮啟蒙》一卷,《神奇祕譜三卷》

袁均哲《太古遺音》二卷

嚴澂《琴譜》十卷

楊表正《琴譜》六卷

林應龍《適情錄》二十卷,《棋史》二卷

葉良貴《歙硯志》四卷

方于魯《墨譜》六卷

程君房《墨苑》十卷

周應愿《印說》一卷

鄭履祥《印林》二卷

臧懋循《六博碎金》八卷

文震亨《長物志》十二卷

已上雜藝。

孝宗《類證本草》三十一卷

世宗《易簡方》一卷

趙簡王《補刊素問遺篇》一卷世傳《素問》王水注本,中有缺篇,簡王得全本,補之。

寧獻王權《乾坤生意》四卷,《壽域神方》四卷

周定王《普濟方》六十八卷

李絅《集解脈訣》十二卷

劉純《玉機微義》五十卷,《醫經小學》六卷

楊文德《太素脈訣》一卷

李恆《袖珍方》四卷

周禮《醫學碎金》四卷

俞子容《續醫說》十卷

徐子宇《致和樞要》九卷

劉均美《拔萃類方》二十卷一作四十卷。

胡濙《衛生易簡方》四卷永樂中,濙為禮部侍郎,出使四方,輯所得醫方進於朝。一作十二卷。

陶華《傷寒六書》六卷,《傷寒九種書》九卷,《傷寒全書》五卷

鄭達《遵生錄》十卷

楊慎《素問糾略》三卷

陰秉暘《內經類考》十卷

孫兆《素問注釋考誤》十二卷

張介賓《張氏類經》四十二卷

張世賢《圖注難經》八卷

吳球《諸證辨疑》四卷,《用藥玄機》二卷

方賢《奇效良方》六十九卷

錢原浚《集善方》三十六卷

鄒福《經驗良方》十卷

丁毅《醫方集宜》十卷

王鏊《本草單方》八卷

錢寶《運氣說》二卷

李言聞《四診發明》八卷

李時珍《瀕湖脈學》一卷,《奇經八脈考》一卷時珍《本草綱目》一書,用力深入,詳《方伎傳》。

虞摶《醫學正傳》八卷,《方脈發蒙》六卷

樓英《醫學綱目》四十卷

陳諫《藎齋醫要》十五卷

徐春甫《古今醫統》一百卷

方廣《丹溪心法附餘》二十四卷

傅滋《醫學集成》十二卷

薛己《家居醫錄》十六卷,《外科心法》七卷

王璽《醫林集要》八十八卷

錢萼《醫林會海》四十卷

方穀《脈經直指》七卷,《本草集要》十二卷

王肯堂《醫論》四卷肯堂著《證治準繩全書》,博通醫學,見《王樵傳》

黃承昊《折肱漫錄》六卷

萬全《保命活訣》三十五卷

李中梓《頤生微論》十卷

李濂《醫史》十卷

楊珣《針炙詳說》二卷

徐鳳《針炙大全》七卷

徐彪《本草證治辨明》十卷

繆希雍《本草經疏》二十卷,《方藥宜忌考》十二卷

熊宗立《傷寒運氣全書》十卷,《傷寒活人指掌圖論》十卷

趙原陽《外科序論》一卷

汪機《外科理論》八卷

吳倫《養生類要》二卷

王鑾《幼科類萃》二十八卷

薛鎧《保嬰撮要》二十卷

周子蕃《小兒推拿祕訣》一卷

吳洪《痘疹會編》十卷

以上醫術。

──右藝術類,一百十六部,一千五百六十四卷。

《永樂大典》二萬二千九百卷永樂初,解縉等奉敕編《文獻大成》既竣,帝以為未備,復敕姚廣孝等重修,四歷寒暑而成,更定是名。成祖製序。後以卷帙太繁,不及刊布,嘉靖中,復加繕寫。

張九韶《群書備數》十二卷

袁均哲《群書纂數》十二卷,《類林雜說》十五卷楊士奇《文籍志》云明初人所編。

沈易《博文編》四卷

吳相《滄海遺珠》十卷

楊循吉《奚囊手鏡》二十卷

《群書集事淵海》四十七卷《百川書志》云弘治時人編。

楊慎《升庵外集》一百卷焦竑編次。

王圻《三才圖說》一百六卷

司馬泰《文獻匯編》一百卷

凌瀚《群書類考》二十二卷

浦南金《修辭指南》二十卷

顧充《古雋考略》十卷

吳珫《經史文編》三十卷,《三才廣志》三百卷

唐順之《稗編》一百二十卷

李先芳《雜纂》四十卷

鄭若庸《類雋》三十卷

王世貞《類苑詳注》三十六卷

陳耀文《天中記》六十卷

凌迪知《文林綺繡》七十卷,《文選錦字》二十一卷,《左國腴詞》八卷,《太史華句》八卷

徐璉《群書纂要》一百九十六卷

曹大同《藝林華燭》一百六十卷

陳禹謨《駢志》二十卷,《補注北堂書鈔》一百六十卷

茅綯《學海》一百六十四卷

徐常吉《事詞類奇》三十卷

徐元泰《喻林》一百二十卷

馮琦《經濟類編》一百卷

章潢《圖書編》一百二十七卷

何三畏《類熔》二十卷

彭大翼《山堂肆考》二百四十卷

卓明卿《藻林》八卷

郭子章《黔類》十八卷

詹景鳳《六緯擷華》十卷

焦竑《類林》八卷

彭好古《類編雜說》六卷

王家佐《古今元屑》八卷

況叔祺《考古詞宗》二十卷

朱謀韋《金海》一百二十卷

林濂《詞叢類採》八卷,《續》八卷

俞安期《唐類函》二百卷

宋應奎《翼學編》十三卷

陳世寶《古今類腴》十八卷

陳懋學《事文類纂》十六卷

袁黃《群書備考》二十卷

徐鑒《諸書考略》四卷

凌以棟《五車韻瑞》一百六十卷

劉仲達《鴻書》一百八卷

劉胤昌《類山》十卷

黃一正《事物紺珠》四十六卷

汪宗姬《儒函數類》六十二卷

劉國翰《記事珠》十卷

吳楚材《強識略》二十四卷

彭儼《五侯鯖》十二卷

商浚《博聞類纂》二十卷

范泓《典籍便覽》八卷

楊淙《事文玉屑》二十四卷

徐袍《事典考略》六卷

朱東光《玉林摘粹》八卷

王光裕《客窗餘錄》二十二卷

劉業《古今事類通考》十卷

夏樹芳《詞林海錯》十六卷

王路清《珠淵》十卷

唐希言《事言要玄集》二十二卷

錢應充《史學璧珠》十八卷

胡尚洪《子史類語》二十四卷

沈夢熊《三才雜組》五卷

屠隆《漢魏叢書》六十卷

陳仁錫《潛確居類書》一百二十卷,《經濟八編類纂》二百五十五卷

林琦《倫史鴻文》二十四卷

程良孺《茹古略》八十卷

雷金科《文林廣記》三十一卷

徐應秋《駢字憑霄》二十卷

《枳記》二十八卷

胡震亨《祕冊匯函》二十卷

毛晉《津逮秘書》十五集

──右類書類,八十三部,二萬七千一百八十六卷。

《道藏目錄》四卷

《道經》五百十二函

太祖《注道德經》二卷,《周顛仙傳》一卷太祖制。

《神仙傳》一卷成祖制。

寧獻王權《庚辛玉冊》八卷,《造化鉗錘》一卷

陶宗儀《金丹密語》一卷

張三豐《金丹直指》一卷,《金丹祕旨》一卷

劉太初《金丹正惑》一卷

黃潤玉《道德經注解》二卷

楊慎《莊子闕誤》一卷

王道《老子人意》二卷

朱得之《老子通義》二卷,《莊子通義》十卷,《列子通義》八卷

薛蕙《老子集解》二卷

商廷試《訂注參同契經傳》三卷

徐渭《分釋古注參同契》三卷

皇甫濂《道德經輯解》三卷

孫應鰲《莊義要刪》十卷

王宗沐《南華經別編》二卷

田藝蘅《老子指玄》二卷

焦竑《老子翼》二卷,《考異》一卷,《莊子翼》八卷,《南華經餘事雜錄》二卷,《拾遺》一卷

龔錫爵《老子疏略》一卷

陶望齡《老子解》二卷《莊子解》五卷

郭良翰《南華經薈解》三十三卷

羅勉道《南華循本》三十卷

陸長庚《老子玄覽》二卷,《南華副墨》八卷,《陰符經測疏》一卷,《周易參同契測疏》一卷,《金丹就正篇》一卷,《張紫陽金丹四百字測疏》一卷,《方壺外史》八卷

李先芳《陰符經解》一卷,《蓬玄雜錄》十卷

沈宗霈《陰符釋義》三卷

尹真人《性命圭旨》四卷

桑喬《大道真詮》四卷

孫希化《真武全傳》八卷

池顯方《國朝仙傳》二卷

靳昂《龍砂一脈》一卷

朱多鱟《龍砂八百純一玄藻》二卷

朱載韋《葆真通》十卷

顧起元《紫府奇玄》十一卷

曹學牷《蜀中神仙記》十卷

傅兆際《寰有詮》六卷

楊守業《洞天玄語》五卷

徐成名《保合編》十二卷

──右道家類,五十六部,二百六十七卷

《釋藏目錄》四卷

《佛經》六百七十八函

太祖《集注金剛經》一卷成祖制序。

成祖《御製諸佛名稱歌》一卷,《普法界之曲》四卷,《神僧傳》九卷

仁孝皇后《夢感佛說大功德經》一卷,《佛說大因緣經》三卷

宋濂《心經文句》一卷

姚廣孝《佛法不可滅論》一卷,《道餘錄》一卷

克庵禪師《語錄》一卷

一如《三藏法數》十八卷

陳實《大藏一覽》十卷

大祐《凈土指歸》二卷

元瀞《三會語錄》二卷

溥洽《雨軒語錄》五卷

法聚《玉芝語錄》六卷,《內語》二卷

宗泐《心經注》一卷,《金剛經注》一卷

洪恩《金剛經解義》一卷,《心經說》一卷

楊慎《禪藻集》六卷,《禪林鉤玄》九卷

弘道《注解楞伽經》四卷

梵琦《楚石禪師語錄》二十卷

汪道昆《楞嚴纂注》十卷

交光法師《楞嚴正脈》十卷

陸樹聲《禪林餘藻》一卷

管志道《龍華懺法》一卷

王應乾《楞嚴圓通品》四卷

方允文《楞嚴經解》十二卷

曾鳳儀《金剛般若宗通》二卷,《心經釋》一卷,《楞嚴宗通》十卷,《楞伽宗通》八卷,《圓覺宗通》四卷

沈士榮《續原教論》二卷

楊時芳《心經集解》一卷

何湛之《金剛經偈論疏注》二卷

戚繼光《禪家六籍》十六卷

如愚《金剛筏喻》二卷

張有譽《金剛經義趣廣演》三卷

李通《華嚴疏鈔》四十卷

方澤《華嚴要略》二卷

劉璉《無隱集偈頌》三卷

古音《禪源諸詮》一卷

景隆《大藏要略》五卷

劉鳳《釋教編》六卷

陳士元《象教皮編》六卷,《釋氏源流》二卷

方晟《宗門崇行錄》四卷

一元《歸元直指》四卷

陶望齡《宗鏡廣刪》十卷

沈泰鴻《慈向集》十三卷

陸長庚《楞嚴述旨》十卷

王肯堂《參禪要訣》一卷

楊惟休《佛宗》一卷

張明弼《兔角詮》十卷

徐可求《禪燕》二十卷

瞿汝稷《指月錄》三十二卷

袁宏道《宗鏡攝錄》十二卷

姚希孟《佛法金湯文錄》十二卷

袁中道《禪宗正統》一卷

祩宏《彌陀經疏》四卷,《正訛集》一卷,《禪關策進》一卷,《竹窗三筆》三卷,《自知錄》二卷

真可《紫柏語錄》一卷

德清《華嚴法界境》一卷,《楞嚴通義》十卷,《法華通義》七卷,《觀楞伽記》四卷,《肇論略注》三卷,《長松茹退》二卷,《憨山緒言》一卷

李樹乾《竺乾宗解》四卷

蕭士瑋《起信論解》一卷

曹胤儒《華嚴指南》四卷

俞王言《金剛標指》一卷,《心經標指》一卷,《楞嚴標指》十二卷,《圓覺標指》一卷

鎮澄《楞嚴正觀疏》十卷,《般若照真論》一卷

傳燈《楞嚴玄義》四卷,《天台山方外志》三十卷

通潤《楞嚴合轍》十卷,《楞伽合轍》四卷,《法華大窾》七卷

石顯《西方合論》十卷

智順《善才五十三參論》一卷

仁潮《法界安立圖》六卷

如巹《禪宗正脈》十卷

章有成《金華分燈錄》十卷

鐘惺《楞嚴如說》十卷

沈宗霈《楞嚴約指》十二卷,《徵心百問》一卷

王正位《赤水玄珠》一卷,《栴檀林》一卷

曾大奇《通翼》四卷

曹學牷《蜀中高僧記》十卷

王應遴《慈無量集》四卷

林應起《全閩祖師語錄》三卷

夏樹芳《棲真志》四卷

祖心《冥樞會要》四卷

凈喜《禪林寶訓》四卷

凈喜《禪林寶訓》四卷

大艤《禪警語》一卷,《宗教答響》一卷,《歸正錄》一卷,《博山語錄》二十二卷

元賢《弘釋錄》三卷

宗林《寒燈衍義》二卷

──右釋家類,一百十五部,六百四十五卷。

集類三:一曰別集類,二曰總集類,三曰文史類。

《明太祖文集》五十卷,《詩集》五卷

《仁宗文集》二十卷,《詩集》二卷

《宣宗文集》四十四卷,《詩集》六卷,《樂府》一卷

《憲宗詩集》四卷

《孝宗詩集》五卷

世宗《翊學詩》一卷,《宸翰錄》一卷,《詠和錄》一卷,《詠春同德錄》一卷,《白鵲贊和集》一卷

神宗《勸學詩》一卷各籓及宗室自著詩文集,已見本傳,不載。

宋濂《潛溪文集》三十卷皆元時作。《潛溪文粹》十卷劉基選。《續文粹》十卷方孝孺鄭濟同選。《宋學士文集》七十五卷,《鑾坡前集》十卷,《後集》十卷,《續集》十卷,《別集》十卷,《芝園前集》十卷,《後集》十卷,《別集》十卷,《朝天集》五卷。《詩集》五卷

劉基《覆瓿集》二十四卷,《拾遺》二卷、皆元時作。《犁眉公集》四卷,《文成集》二十卷、匯編諸集及《鬱離子》、《春秋明經》諸書。詞四卷

危素《學士集》五十卷

葉儀《南陽山房稿》二十卷

王冕《竹齋詩集》三卷

范祖幹《柏軒集》四卷

戴良《九靈山房集》三十卷

王逢《梧溪詩集》七卷

梁寅《石門集》四卷

楊維楨《東維子集》三十卷,《鐵崖文集》五卷,《古樂府》十六卷,《詩集》六卷

陶宗儀《南村詩集》四卷

貢性之《南湖集》二卷

謝應芳《龜巢集》二十卷

《張昱詩集》二卷

楊芾《鶴崖集》二十卷

李祁《雲陽先生集》十卷裔孫李東陽傳其集。

塗幾《塗子類稿》十卷

張憲《玉笥集》十卷

吳復《雲槎集》十卷

華幼武《黃楊集》四卷

《陶振賦》一卷洪武初,振獻《紫金山》、《金水河》、及《飛龍在天》三賦。

《陶安文集》二十卷

李習《橄欖集》五卷

汪廣洋《鳳池吟稿》十卷

孫炎《左司集》四卷

劉炳《春雨軒集》十卷、詞一卷

《劉迪簡文集》五卷

郭奎《望雲集》五卷

王禕《忠文集》二十四卷

張以寧《翠屏集》五卷

《詹同文集》三卷

《劉崧文集》十八卷、詩八卷

魏觀《蒲山集》四卷

朱善《一齋集》十卷,《遼海集》五卷

顧輝《守齋類稿》三十卷

朱升《楓林集》十二卷

趙汸《東山集》十五卷

汪克寬《環谷集》八卷

唐桂芳《白雲集略》四十卷

李勝原《盤谷遺稿》五卷

《胡翰文集》十卷

蘇伯衡《蘇平仲集》十六卷

《朱廉文集》十七卷

陳謨《海桑集》十卷

周霆《震石初集》十卷

高啟《槎軒集》十卷,《大全集》十八卷、詞一卷

楊基《眉庵集》十二卷、詞一卷

徐賁《北郭集》六卷

張羽《靜居集》六卷

陳基《夷白齋集》二十卷

王彝《溈蜼子集》四卷

王行《半軒集》十二卷

袁凱《海叟詩集》四卷

孫作《滄螺集》六卷

朱右《白雲稿》十二卷

《徐尊生制誥》二卷,《懷歸稿》十卷,《還鄉稿》十卷

貝瓊《清江文集》三十卷、詩十卷

顧祿《經進集》二十卷

《答祿與權文集》十卷

杜斅《拙庵集》十卷

吳源《託素齋集》八卷

《劉駟文集》十卷

宋訥《西隱集》十卷

劉三吾《坦齋集》二卷一作《坦翁集》十二卷。

《張孟兼名丁,以字行。文集》六卷

王翰《敝帚集》五卷,《梁園寓稿》九卷

方克勤《愚庵集》二十卷

《吳伯宗集》二十四卷《南宮》、《使交》、《成均》、《玉堂》凡四種

杜隰《雙清集》十卷

鄭真《滎陽外史集》一百卷

吳玉林《松蘿吟稿》二十卷

方幼學《翬山集》十二卷

唐肅《丹崖集》八卷

謝肅《密庵集》十卷

謝徽《蘭庭集》六卷

邵亨貞《蛾術文集》十六卷

烏斯道《春草齋集》十卷

貝翱《舒庵集》十卷

葉顒《樵雲集》六卷

沈夢麟《花溪集》三卷

劉薦《盤谷集》十卷

《宋禧文集》三十卷、詩十卷

鄭淵《遂初齋稿》十卷

林靜《愚齋集》二十卷

劉永之《山陰集》五卷

龔斅《鵝湖集》六卷

王沂《徵士集》八卷

王祐《長江稿》五卷

《解開文集》四十卷

林鴻《鳴盛集》四卷鴻與唐泰、黃玄、周玄、鄭定、高棅、王偁、王褒、王恭、陳亮另有《閩中十才子詩》十卷。

孫蕡《西庵集》九卷蕡與王佐、黃哲、趙介、李德另有《廣中五先生詩》四卷。

《藍仁詩集》六卷

《藍智詩集》六卷

張適《樂圃集》六卷

浦源《舍人集》十卷

林弼《登州集》六卷

陸中《蒲棲集》二十卷

《林大同文集》九卷

丁鶴年《海巢集》三卷本西域人,後家武昌,永樂中始卒。楚憲王為刻其集。

方孝孺《遜志齋集》三十卷,《拾遺》十卷黃孔昭、謝鐸同輯。

卓敬《卓氏遺書》五十卷

練子寧《金川玉屑集》五卷

《茅大芳集》五卷

程本立《巽隱集》四卷

王艮吉水人,王充耘孫。《翰林集》十卷

王叔英《靜學集》二卷

周是修《芻蕘集》六卷

《鄭居貞集》五卷

《程通遺稿》十卷

梅殷《都尉集》三卷

《任亨泰遺稿》二卷

《王紳文集》三十卷

王稌《青巖類稿》十卷

《林右集》二卷

《王賓詩集》二卷

張紞《鶠庵集》一卷

樓璉《居夷集》五卷

龔詡《野古集》二卷

高遜志《嗇齋集》二卷

解縉《學士集》三十卷,《春雨集》十卷,《似羅隱集》二卷

已上洪武、建文時。

姚廣孝《逃虛子集》十卷,《外集》一卷

黃淮《省愆集》二卷、詞一卷

《胡廣集》十九卷

楊榮《兩京類稿》三十卷,《玉堂遺稿》十二卷

楊士奇《東里集》二十五卷、詩三卷

胡儼《頤庵集》三十卷

《金幼孜集》十二卷

《夏原吉集》六卷

王鈍《野莊集》六卷

鄭賜《聞一齋集》四卷

《趙羾集》三卷

《茹瑺詩》一卷

黃福《家集》三十卷,《使交文集》十七卷

鄒濟《頤庵集》九卷

王達《天游集》二十二卷

《曾棨集》十八卷

《林環文集》十卷、詩三卷

林志《蔀齋集》十五卷

《王汝玉詩集》八卷

《張洪集》二卷

《王紱詩集》五卷

梁潛《泊庵集》十二卷

劉髦《石潭集》五卷

鄒緝《素庵集》十卷

王偁《虛舟集》五卷

王褒《養靜齋集》十卷

《王恭詩集》七卷

高棅《嘯臺集》二十卷,《木天清氣集》十四卷

《黃壽生文集》十卷

《楊慈文集》五卷

蘇伯厚《履素集》十卷

鄭棠《道山集》二十卷

劉均《拙庵集》八卷

《徐永達文集》二十卷、詩十卷

王洪《毅齋集》八卷

《黃裳集》十卷

袁忠徹《符臺外集》五卷

陸顒《頤光集》二十卷

瞿佑《存齋樂全集》三卷、詞三卷

曾鶴齡《松臒集》三卷

陳叔剛《絅齋集》十卷

柯暹《東岡集》十二卷

《羅亨信集》十二卷

《劉鉉詩集》六卷

《金實文集》二十八卷

《王暹奏議》二十卷,《文集》四十卷

蘇鉦《竹坡吟稿》二十卷

周鳴《退齋稿》六十卷

方勉《怡庵集》十五卷

周敘《石溪集》十八卷

《楊溥文集》十二卷、詩四卷

胡濙《澹庵集》五卷

已上永樂時。

熊概《芝山集》四十卷,《公餘集》三十卷

《吳訥文集》二十卷,詩八卷

秦樸《抱拙集》六卷

陳繼《怡庵集》二十卷

《黃澤詩集》十四卷

羅紘《蘭坡集》十二卷

馬愉《淡軒文集》八卷

陳循《芳洲集》十六卷

《高穀集》十卷

廖莊《漁梁集》二卷

林文《澹軒稿》十二卷

龔錡《蒙齋集》十卷

《王訓文集》三十卷

《梁萼集》二十卷

姜洪《松岡集》十一卷

楊復《土苴集》五十卷

劉廣衡《雲庵集》三十卷

陳泰《拙庵集》二十五卷

李奎《九川集》六卷

《徐琦文集》六卷

已上洪熙、宣德時。

《孫原貞奏議》八卷,《歲寒集》二卷

王直《抑庵集》四十二卷

《王英文集》六卷,詩五卷

《錢習禮文集》十四卷,《應制集》一卷

《陳鎰文集》六卷

《魏驥摘稿》十卷

周忱《雙崖集》八卷

陳璉《琴軒稿》三十卷

《周旋文集》十卷

劉球《兩溪集》二十四卷

張楷《和唐音》二十八卷,《和李杜詩》十二卷

《李時勉文集》十一卷、詩一卷

陳敬宗《澹然集》十八卷

張倬《毅齋集》二十卷

鄭鯨《雲遨摘稿》八卷

《彭時奏疏》一卷,《文集》四卷

《商輅奏議》一卷,《文集》三十二卷

《蕭糸茲文集》二十卷、詩十卷

《于謙奏議》十卷,《文集》二十卷

郭登《聯珠集》二十二卷景泰初,登封定襄伯,有詩名。是集以其父玨兄武之作,與登詩合編。

《何文淵奏議》一卷,《文稿》四卷

章瑄《竹莊集》四十卷

吳宣《野庵集》十六卷

鄭文康《平橋集》十八卷

劉溥《草窗集》二卷溥與蔣主忠、王貞慶、晏鐸、蘇平、蘇正、湯胤勣、王淮、沈愚、鄒亮等稱景泰十才子,當時各有專稿。

桑琳《鶴溪集》二十卷

《錢洪詩集》四卷

《劉英詩集》六卷

徐有貞《武功集》八卷

《許彬文集》十卷、詩四卷

薛瑄《敬軒集》四十卷、詩八卷

李賢《古穰集》三十卷,《續集》二十卷

呂原《介軒集》十二卷

岳正《類博稿》十卷

《劉儼文集》三十二卷

吳與弼《康齋文集》十二卷

王宇《厚齋集》三卷

張穆《勿齋集》二十卷

劉昌《五臺集》二十二卷《胥臺》、《鳳臺》、《金臺》、《嵩臺》、《越臺》諸稿匯編。

蕭儼《竹軒集》二十卷

周瑩《郡齋稿》十卷

羅周《梅隱稿》十八卷

姚綬《雲東集》十卷

湯胤勣《東谷集》十卷

《易貴文集》十五卷

已上正統、景泰、天順時。

《劉定之存稿》二十一卷,《續稿》五卷

《劉珝文集》十六卷

《軒輗奏議》四卷

《彭華文集》十卷

尹直《澄江集》二十五卷

《姚夔奏議》三十卷,《文集》十卷

《李裕奏議》七卷,《文集》四卷

《楊鼎奏議》五卷,《文稿》二十卷

倪謙《玉堂》、《南宮》、《上谷》、《歸田》四稿共一百七十卷

《餘子俊奏議》六卷

周洪謨《箐齋集》五十卷,《南皋集》二十卷

《林聰奏議》八卷,《文集》十四卷

《張瑄奏議》八卷,《觀庵集》十五卷,《關洛紀巡錄》十七卷

《謝一夔文集》六卷

《韓雍奏議》一卷,《文集》十五卷

柯潛《竹巖集》八卷

陸釴《春雨堂稿》三十卷

《葉盛奏草》三十卷,《文稿》二卷、詩一卷

《楊守陳全集》三十卷

范理《丹臺稿》十卷

《林鶚文稿》十卷

羅倫《一峰集》十卷

莊昶《定山集》十卷

黃仲昭《未軒集》十三卷

陳獻章《白沙子》八卷,《文集》二十二卷,《遺編》六卷

《楊起元文編》六卷

《張弼文集》五卷、詩四卷

胡居仁《敬齋集》三卷

陳真晟《布衣存稿》九卷

《夏寅文集》四十卷,《備遺錄》二十三卷

《張寧文集》三十二卷

夏時正《留餘稿》三十五卷

陸容《式齋集》三十八卷

龍瑄《鴻泥集》二十卷

周瑛《翠渠摘稿》七卷

段正《介庵集》三十卷

《蔣琬文集》十卷

朱翰《石田稿》十四卷

張胄《西溪集》十五卷

《丁元吉文集》六十四卷

劉敔《鳳巢稿》六卷

桑悅《兩都賦》二卷、《古賦》三卷,《文集》十六卷

祁順《巽川集》二十卷

《徐溥文集》七卷

丘浚《瓊臺類稿》五十二卷、詩十二卷

李東陽《懷麓堂前後集》九十卷,《續稿》二十卷

謝遷《歸田稿》十卷

陸簡《龍皋稿》十九卷

程敏政《篁墩全集》一百二十卷

吳寬《匏庵集》七十八卷

《張元禎文集》二十四卷

《王恕奏稿》十五卷,《文集》九卷

《韓雍奏議》一卷

倪岳《青溪漫稿》二十四卷

《馬文升奏議》十六卷,《文集》一卷

王人與《思軒集》十二卷

楊守阯《碧川文鈔》二十九卷、詩二十卷

《張昇文集》二十二卷

童軒《枕肱集》二十卷

杭淮《雙溪詩集》八卷

黎淳《龍峰集》十三卷

《劉大夏奏議》一卷、詩二卷

《張悅集》五卷

《何喬新文集》三十二卷

《彭韶奏議》五卷,《文集》十二卷

《王珣奏稿》十卷、詩二卷

《閔珪文集》十卷

徐貫《餘力集》十二卷

《董越文集》四十二卷

《謝鐸奏議》四卷,《文稿》四十五卷、詩三十六卷

陳音《愧齋集》十二卷

張詡《東所集》十卷

鄒智《立齋遺文》四卷

李承箕《大崖集》二十卷

《錢福文集》六卷

《楊循吉遺集》五卷

邵珪《半江集》六卷

趙寬《半江集》六卷

《杭濟詩集》六卷

吳元應《詩集》十五卷

顧潛《靜觀堂集》十四卷

文林《溫州集》十二卷

呂翾《九柏集》六卷

沈周《石田詩鈔》十卷

史鑒《西村集》八卷

祝允明《祝氏集略》三十卷,《懷裏堂集》三十卷,《小集》七卷

《唐寅集》四卷

顧磐《海涯集》十卷

《王鏊文集》三十卷

《楊廷和奏議》一卷,《石齋集》八卷

梁儲《鬱洲集》九卷

《費宏文集》二十四卷

靳貴《戒庵集》二十卷

《楊一清奏議》三十卷,《石淙類稿》四十五卷、詩二十卷

蔣冕《湘皋集》三十三卷

毛紀《鰲峰類稿》二十六卷

韓文《質庵集》四卷

吳文度《交石集》十卷

《林瀚集》二十五卷

屠勳《東湖稿》十二卷

《羅奏議》一卷,《文集》十八卷,《續集》十四卷

《儲巏文集》十五卷

王鴻儒《凝齋集》九卷

邵寶《容春堂全集》六十一卷

《章懋文集》九卷

《楊廉奏議》四卷,《文集》六十二卷

喬宇《白巖集》二十卷

《黃瓚文集》十二卷

蔡清《虛齋文集》五卷

《魯鐸文集》十卷

王雲鳳《虎谷集》二十一卷

《毛澄類稿》十八卷

《王瓊奏議》四卷

彭澤《幸庵行稿》十二卷

《林俊文集》四十卷、詩十四卷

李夢陽《空同全集》六十六卷

康海《對山集》十九卷、《樂府》二卷

王九思《水美陂集》十九卷、《樂府》四卷

何景明《大復集》六十四卷

《鄭善夫奏議》一卷,《少谷全集》二十五卷

徐禎卿《迪功集》十一卷

朱應登《凌溪集》十九卷

王廷陳《夢澤集》三十八卷

景暘《前谿集》十四卷

《陳沂文集》十二卷、詩五卷

《田汝耔奏議》五卷,《永南集》十八卷

倫文敘《迂岡集》十卷,《白沙集》十二卷

顏木《燼餘稿》四卷

盧雍《古園集》十二卷

陳霆《水南集》十七卷

王守仁《陽明全書》三十八卷

陸完《水村集》二十卷

唐錦《龍江集》十四卷

《穆孔暉文集》三卷

史學《埭谿集》二十卷

許莊《康衢集》一百卷

汪循《仁峰文集》二十五卷

錢仁夫《水部詩曆》十二卷

徐璉《玉峰集》十五卷、五言詩五卷

黃省曾《五岳山人集》三十八卷

孫一元《太白山人稿》五卷

《謝承舉一名璿詩集》十五卷

王寵《雅宜山人集》十卷

傅汝舟《丁戊集》十二卷

高瀫《石門集》二卷

蕭雍《酌齋遺稿》四卷

已上成化、弘治、正德時。

《廖道南文集》五十卷、詩六卷

羅欽順《整庵稿》三十三卷

《何孟春疏議》十卷,《文集》十八卷

《顧清文集》四十二卷

劉瑞《五清集》十八卷

呂柟《涇野集》五十卷

《何瑭文集》十一卷

魏校《莊渠文錄》十六卷、詩四卷

陳察《虞山集》十三卷

《楊慎文集》八十一卷,《南中集》七卷、詩五卷、詞四卷

《胡世寧奏議》十卷

鄭岳《山齋稿》二十四卷

《陳洪謨文稿》二卷

《王時中奏議》十卷

《董文集》六卷

《秦金詩集》十卷

《潘希曾奏議》四卷,《竹澗集》八卷

《劉龍文集》四十八卷

《劉夔奏議》十卷

《陸深全集》一百卷,《續集》十卷

《張邦奇全集》五十卷

《馬中錫奏疏》三卷,《東田集》六卷

劉玉《執齋集》二十卷

周倫《貞翁稿》十二卷

劉節《梅國集》四十二卷

《章拯文集》八卷

邊貢《華泉集》四卷、詩八卷

《王廷相奏議》十卷,《家藏集》五十四卷顧璘《息園文稿》九卷、詩十四卷

《劉麟文集》十二卷

崔銑《洹詞》十二卷

王爌《南渠稿》十六卷

《陳鳳梧奏議》十卷,《修辭錄》六卷

《張翀文集》二十卷

夏良勝《東洲稿》十二卷詩八卷

《姚鏌文集》八卷

《王道文集》十二卷

《徐問文集》二十四卷

萬鏜《治齋文集》四卷

湛若水《甘泉前後集》一百卷

韓邦奇《苑洛集》二十二卷

劉訒《春岡集》六卷

黃衷《矩齋集》二十卷

《顧應祥文集》十四卷、樂府一卷

樂頀《木亭稿》三十六卷

石珤《熊峰集》四卷

賈言永《南隖集》十卷

崔桐《東洲集》四十卷

《毛伯溫奏議》二十卷,《東塘集》十卷

《王以旂奏議》十卷,《石岡集》四卷

《林廷昂集》十卷

《孫承恩集》三卷

黃佐《兩都賦》二卷,《泰泉集》六十卷

童承敘《內方集》十卷

貢汝成《三大禮賦》一卷嘉靖中獻。

林大輅《槐喑集》十六卷

《許宗魯全集》五十二卷

胡纘宗《鳥鼠山人集》十八卷,《擬古樂府》四卷、詩七卷

《方鵬文集》十八卷、詩八卷

王同祖《太史集》六十卷

鄒守益《東郭集》十二卷,《遺稿》十三卷

《顧鼎臣文集》二十四卷

張璧《陽峰集》二十六卷

《張治文集》十四卷

許贊《松皋集》二十六卷

王崇慶《端溪集》八卷

《王邦瑞文集》二十卷

聶豹《雙江集》十八卷

薛蕙《考功集》十卷

汪必東《南雋集》二十卷

孫存《豐山集》四十卷

《蕭鳴鳳文集》十五卷

周佐《北澗集》十卷

《金賁亨文集》四卷

蔣山卿《南泠集》十二卷

李濂《嵩渚集》一百卷

《林士元文集》十卷

林春澤《人瑞翁集》十二卷

《汪應軫文集》十四卷

《陳琛文集》十二卷

王漸逵《青蘿集》十六卷

《戴鱀文集》八卷

廖世昭《明一統賦》三卷

《許相卿全集》二十六卷

陸釴《少石子集》十三卷

邵經邦《弘藝錄》三十二卷

陳講《中川集》十三卷

丘養浩《集齋類稿》十八卷

《王用賓文集》十六卷

倫以訓《白山集》十卷

倫以諒《石溪集》十卷

倫以詵《穗石集》十卷

顧瀍《寒松齋稿》四卷

黃綰《石龍集》二十八卷

《費寀集》四卷

席書《元山文選》五卷

方獻夫《西樵稿》五卷

《霍韜集》十五卷

舒芬《內外集》十八卷

汪佃《東麓稿》十卷

戴冠《邃谷集》十二卷、詩二卷

唐龍《漁石集》四卷

《歐陽鐸集》二十二卷

夏言《桂洲集》二十卷

嚴嵩《鈐山堂集》二十六卷

《張孚敬詩集》三卷

歐陽德《南野集》三十卷

《許誥奏議》二卷

許論《默齋集》四卷

張時徹《芝園全集》八十五卷

呂禎《澗松稿》四卷

《鄭曉奏疏》十四卷,《文集》十二卷

潘恩《笠江集》二十四卷

陳儒《芹山集》四十卷

王艮《心齋文集》二十卷

王畿《龍谿文集》二十卷

錢德洪《緒山集》二十四卷

孫宜《洞庭山人集》五十三卷

高叔嗣《蘇門集》八卷

呂本《期齋集》十六卷

徐階《世經堂全集》五十卷

鄒守愚《俟知堂集》十三卷

《胡松奏疏》五卷,《文集》十卷

《袁煒詩集》八卷

《嚴訥表奏》二卷,《文集》十二卷

李春芳《詒安堂稿》十卷

《郭朴文集》五卷

《林庭機文集》十二卷

《茅瓚文集》十五卷

董份《泌園全集》三十七卷

《孫升文集》二十卷

李璣《西野集》十三卷

尹臺《洞麓堂集》三十八卷

范欽《天一閣集》十九卷

陳堯《梧岡文集》五卷、詩三卷

雷禮《鐔墟堂稿》二十卷

蔡汝楠《自知堂集》二十四卷

張岳《凈峰稿》四十六卷

蘇濂《伯子集》十三卷

蘇澹《仲子集》七卷

《陸垹文集》十二卷

《謝東山文集》四十卷

李舜臣《愚谷集》十卷

龔用卿《雲岡集》二十卷

《王維楨全集》四十二卷

《王材文集》六十五卷

《呂懷類稿》三十三卷

趙時春《浚谷集》十七卷

王慎中《遵巖文集》四十一卷

唐順之《荊川集》二十六卷

《陳束文集》二卷

熊過《南沙集》八卷

《任瀚逸稿》六卷

呂高《江峰稿》十二卷

李默《群玉樓稿》七卷

《馮恩奏疏》一卷,《芻蕘錄》四卷

馬一龍《游藝集》十九卷

陸粲《貞山集》十二卷

康太和《蠣峰集》二十四卷

餘光《兩京賦》二卷

楊爵《斛山稿》六卷

馮汝弼《祐山集》十六卷

包節《侍御集》六卷

錢薇《海石集》二十八卷

周怡《訥溪集》二十七卷

《羅洪先全集》二十五卷

唐樞《木鐘臺集》三十二卷

林春《東城集》二卷

柯維騏《藝餘集》十四卷

盧襄《五隖草堂集》十卷

薛甲《藝文類稿》十四卷

薛應旂《方山集》六十八卷

《唐音文集》二十卷

《劉繪奏議》二卷,《嵩陽集》十五卷

喬世寧《丘隅集》十九卷

《孔汝錫文集》十六卷、詩十四卷

袁BM《胥臺集》二十卷

袁尊尼《魯望集》十二卷

文徵明《甫田集》三十五卷

文彭《博士集》三卷

文嘉《和州集》一卷

蔡羽《林屋集》二十卷,《南館集》十三卷

陳淳《白陽詩集》八卷

湯珍《小隱堂詩集》八卷

彭年《隆池山樵集》三卷

田汝成《叔禾集》十二卷

屠應颭《蘭暉堂集》八卷

范言《菁陽集》五卷

楊本仁《少室山人集》二十四卷

沈愷《環溪集》二十六卷

李開先《中麓集》十二卷

皇甫沖《子浚集》六十卷

皇甫涍《少玄集》三十六卷

皇甫汸《司勳集》六十卷

皇甫濂《水部集》二十卷

周詩《虛巖山人集》六卷

黃姬水《淳父集》二十四卷

《駱文盛存稿》十五卷

崔廷槐《樓溪集》三十六卷

慄應宏《太行集》十六卷、詩六卷

莫如忠《崇蘭館集》二十卷

《陳昌積文集》三十四卷

何良俊《柘湖集》二十八卷

何良傅《禮部集》十卷

許穀《省中》、《二臺》、《武林》、《歸田》四稿共十七卷

華鑰《水西居士集》十二卷

張之象《剪綃集》二卷

徐獻忠《長谷集》十五卷

鄔紳《中憲集》六卷

《陳暹文集》四卷

瞿景淳《內制集》一卷,《文集》十六卷

王問《仲山詩選》八卷

侯一元《少谷集》十六卷

《俞憲詩集》二十四卷

南逢吉《姜泉集》十四卷

錢芹《永州集》五卷

《姚淶文集》八卷

華察《巖居稿》八卷

沈東《屏南集》十卷

《茅坤文集》三十六卷

吳維嶽《天目山齋稿》二十八卷

李嵩《存笥稿》十卷

馮惟健《陂門集》八卷

馮惟訥《光祿集》十卷

桑介《白厓詩選》十卷

李應元《蔡蒙山房稿》四卷

陳鳳《清華堂稿》六卷

吳珫《環山樓集》六卷

沈煉《鳴劍集》十二卷,《青霞山人集》五卷

金大車《子有集》二卷

金大輿《子坤集》二卷

楊繼盛《忠愍集》四卷

呂時中《潭西文集》十七卷

林懋和《雙臺詩選》九卷

王交《綠槐堂稿》二十二卷

《向洪邁詩文集》十卷

盧岐嶷《吹劍集》三十五卷

《周思兼文集》八卷

詹萊《招搖池館集》三十卷

謝江《岷陽集》八卷

傅夏器《錦泉集》六卷

朱曰籓《山帶閣集》三十三卷

岳岱《山居稿》三十卷

高岱《西曹集》九卷

陸楫《蒹葭堂集》七卷

李先芳《東岱山房稿》三十卷

陳宗虞《臥雲樓稿》十四卷

《黃伯善文稿》六卷、詩十五卷

胡瀚《今山文集》一百卷

蔡宗堯《龜陵集》二十卷

孫樓《百川集》十二卷

張世美《西谷集》十六卷

邵圭潔《北虞集》六卷

李攀龍《滄溟集》三十二卷,《白雪樓詩集》十卷

王世貞《弇州四部稿》一百七十四卷、四部者:一賦、二詩、三文、四說,以擬域中之四部州。汪道昆序之。《續稿》二百十八卷

王世懋《奉常集》五十四卷、詩十五卷

梁有譽《比部集》八卷

徐中行《天目山人集》二十一卷、詩六卷

《宗臣詩文集》十五卷

吳國倫《甔甀洞稿》五十四卷,《續稿》二十七卷、詩十五卷

謝榛《四溟山人集》二十卷、詩四卷

《盧柟賦》五卷,《蠛蠓集》五卷

《劉鳳文集》三十二卷

《陸弼詩集》二十六卷

汪道昆《太函集》一百二十卷,《南溟副墨》二十四卷

許邦才《梁園集》四卷

《魏學禮集》二十四卷

魏裳《雲山堂集》六卷

《張佳胤奏議》七卷,《崌崍文集》六十五卷

張九一《綠波樓集》十卷

《黎民表文集》十六卷

《歐大任虞部集》二十二卷

《俞允文詩文集》二十四卷

《餘曰德詩集》十四卷

萬表《玩鹿亭稿》八卷

高拱《獻忱集》五卷,《詩文集》四十四卷

《趙貞吉文集》二十三卷、詩五卷

《高儀奏議》十卷

楊巍《夢山存稿》四卷

殷士儋《金輿山房稿》十四卷

《諸大綬文集》八卷

楊博《獻納稿》十卷,《奏議》七十卷,《詩文集》十二卷

《張瀚詩文集》四十卷

《董傳策奏議》一卷,《採薇集》十四卷

《馬森文集》二十卷

洪朝選《靜庵稿》十五卷

《朱衡文集》二十卷

陳紹儒《司空集》二十卷

何維柏《天山堂集》二十卷

周詩《與鹿集》十二卷

郭汝霖《石泉山房集》十二卷

《王時槐存稿》十四卷

曹大章《含齋稿》二十卷

林大春《井丹集》十五卷

王叔果《半山藏稿》二十卷

王叔杲《玉介園稿》二十卷

徐師曾《湖上集》十四卷

張祥鳶《華陽洞稿》二十二卷

陳善《黔南類稿》八卷

穆文熙《逍遙園集》十卷

胡直《衡廬稿》三十卷

王格《少泉集》十卷

《姚汝循詩文集》二十四卷

張元忭《不二齋稿》十二卷

歸有光《震川集》三十卷,《外集》十卷錢謙益訂正。

《劉效祖詩稿》六卷

王叔承《吳越游》七卷

《沈明臣詩集》四十二卷

《陳鶴詩集》二十一卷

馮遷《長鋏齋稿》七卷

《朱邦憲詩文集》十五卷

《徐渭詩文全集》二十九卷

《王寅詩文集》八卷

郭造卿《海岳山房集》二十卷

俞汝為《缶音集》四卷

謝汝韶《天池稿》十六卷

《謝肇淛文集》二十八卷、詩三十卷

駱問禮《萬一樓集》六十一卷,《外集》十卷

王可大《三山匯稿》八卷

沈桐《觀頤集》二十卷

王養端《遂昌三賦》一卷

《黃謙詩文稿》十六卷

戴廷槐《錦雲集》十六卷

已上嘉靖、隆慶時。

張居正《奏對稿》十卷,《詩文集》四十七卷

張四維《條麓堂集》三十四卷

《馬自強文集》二十卷

《陸樹聲詩文集》二十六卷

《林燫文集》十六卷、詩六卷

汪鏜《餘清堂定稿》三十二卷

《徐學謨文集》四十二卷、詩二十二卷

《潘季馴奏疏》二十卷,《文集》五卷

《吳桂芳奏議》十六卷,《文集》十六卷

《譚綸奏議》十卷

俞大猷《正氣堂集》十六卷

戚繼光《橫槊稿》三卷

《海瑞文集》七卷

吳時來《悟齋稿》十五卷

《趙用賢奏議》一卷,《文集》三十卷、詩六卷

吳中行《賜餘堂集》十四卷

艾穆《熙亭集》十卷

《鄒元標奏疏》五卷,《文集》七卷,《續集》十二卷

沈思孝《陸沈漫稿》六卷

《蔡文範文集》十八卷

范槲明《蜀都賦》一卷

《王宗沐奏疏》四卷,《文集》三十卷

《王崇古奏議》五卷,《山堂匯稿》十七卷

王士性《五岳遊草》十二卷

陳士元《歸雲集》七十五卷

鄧元錫《潛學稿》十七卷

林偕春《雲山居士集》八卷

申時行《綸扉奏章》十卷,《賜閒堂集》四十卷

《餘有丁詩文集》十五卷

《許國文集》六卷

《王錫爵詩文集》三十二卷

《王家屏文集》二十卷

《趙志皋奏議》十六卷,《文集》四卷、詩五卷

《耿定向文集》二十卷

《姜寶文集》三十八卷、詩十卷

孫應鰲《匯稿》十六卷

《魏學曾文集》十卷

《沈節甫文集》十五卷

王樵《方麓居士集》十四卷

《宋儀望文集》十二卷、詩十四卷

《魏允貞文集》四卷

《魏允中文集》八卷

《顧憲成文集》二十卷

《孟化鯉文集》八卷

葉春及《絅齋集》六卷

《王稚登詩集》十二卷

盛時泰《城山堂集》六十八卷

張鳳翼《處實堂前後集》五十三卷

張獻翼《文起堂集》十六卷

莫是龍《石秀齋集》十卷

《曹子念詩集》十卷

顧大典《清音閣集》十卷

鄔佐卿《芳潤齋集》九卷

茅溱《四友齋集》四卷

《莫叔明詩》三卷

《田藝蘅詩文集》二十卷

胡應麟《少室山房類稿》一百二十卷

《陳文燭文集》十四卷、詩十二卷

李維楨《大泌山房全集》一百三十四卷

屠隆《由拳集》二十三卷,《白榆集》二十卷,《棲真館集》三十卷

《屠本畯詩草》六卷

馮時可《元成選集》八十三卷

沈鯉《亦玉堂稿》十八卷

《于慎行文集》十二卷、詩二十卷

《李廷機文集》十八卷

曾同亨《泉湖山房稿》三十卷

王圻《鴻洲類稿》十卷

謝傑《天靈山人集》二十卷

馮琦《宗伯集》八十一卷

曾朝節《紫園草》二十二卷

郭子章《粵草》、《蜀草》、《楚草》、《閩草》、《浙草》、《晉草》、《留草》共五十五卷

許孚遠《致和堂集》八卷

田一BN《鐘臺遺稿》十二卷

林景暘《玉恩堂集》十卷

鄧以贊同《定宇集》四卷

黃洪憲《碧山學士集》二十一卷

《王祖嫡文集》三十七卷

劉日升《慎修堂集》二十三卷

郭正域《黃離草》十卷

唐文獻《占星堂集》十六卷

《鄒德溥全集》五十卷

沈懋學《郊居稿》六卷

馮夢禎《快雪堂集》六十四卷

邢侗《來禽館集》二十八卷

餘寅《農丈人集》二十卷、詩八卷

虞淳熙《德園全集》六十卷

湯顯祖《玉茗堂文集》十五卷、詩十六卷

謝廷諒《薄遊草》二十四卷

謝廷贊《綠屋遊草》十五卷

陳第《寄心集》六卷

《羅大紱文集》十二卷

來知德《瞿塘日錄》三十卷

徐即登《正學堂稿》二十六卷

蘇浚《紫溪集》三十四卷

羅汝芳《近溪集》十二卷、詩二卷

潘士藻《闇然堂集》六卷

焦竑《澹園集》四十九卷,《續集》三十五卷

袁宗道《白蘇齋類稿》二十四卷

《袁宏道詩文集》五十卷

袁中道《珂雪齋集》二十四卷

陶望齡《歇庵集》十六卷

《瞿九思文集》七十五卷

《馮大受詩集》十卷

何三畏《漱六齋集》四十八卷

瞿汝稷《同鄉集》十四卷

郝敬《小山草》十卷

許樂善《適志齋稿》十卷

王納諫《初日齋集》七卷

《姚舜牧文集》十六卷

葉向高《綸扉奏草》三十卷,《文集》二十卷、詩八卷

《丁賓文集》八卷

《區大相詩集》二十七卷

《顧起元文集》三十卷、詩二十卷

湯賓尹《睡庵初集》六卷

王衡《緱山集》二十七卷

公鼐《問次齋集》三十卷

《丘禾實文集》八卷、詩四卷

南師仲《玄麓堂集》五十卷

張以誠《酌春堂集》十卷

《何喬遠集》八十卷

張燮《群玉樓集》八十四卷

張萱《西園全集》三十卷

李光縉《景璧集》十九卷

曹學牷《石倉詩文集》一百卷

徐熥《幔亭集》二十卷

徐勃《鰲峰集》二十六卷

黃汝亨《寓林集》三十二卷

趙宦光《寒山漫草》八卷

俞安期《翏拼集》二十八卷

歸子慕《陶庵集》四卷

《趙南星文集》二十四卷

《楊漣文集》三卷

《左光斗奏疏》三卷,《文集》五卷

魏大中《藏密齋集》二十五卷

魏學洢《茅薝集》八卷

繆昌期《從野堂存稿》八卷

李應昇《落落齋遺稿》十卷

《周宗建奏議》四卷

《黃尊素文集》六卷

《馮從吾疏草》一卷,《少墟文集》二十二卷

《孫慎行奏議》二卷,《玄晏齋集》十卷

曹於汴《抑節堂集》十四卷

陳於廷《定軒存稿》三卷

張鼐《寶日堂集》六卷

楊守勤《寧澹齋集》十卷

婁堅《學古緒言》二十六卷

唐時升《三易集》二十卷

李流芳《檀園集》十二卷

程嘉燧《松圓浪淘集》十八卷

朱國祚《介石齋集》二十卷

鐘惺《隱秀堂集》八卷

譚元春《嶽歸堂集》十卷

蔡復一《遁庵集》十七卷

《王思任文集》三十卷

董其昌《容臺集》十四卷,《別集》六卷

陳繼儒《晚香堂集》三十卷

王廷宰《緯蕭齋集》六卷

李日華《恬致堂集》四十卷

方應祥《青來閣集》三十五卷

《姚希孟文集》二十八卷

陳仁錫《無夢園集》四十卷

蕭士瑋《春浮園集》十卷

鄭懷魁《葵圃集》三十卷

《謝兆申詩文稿》二十四卷

顧正誼《詩史》十五卷

張采《知畏堂文存》十一卷,《詩存》四卷

張溥《七錄齋集》十二卷、詩三卷

唐汝詢《編篷集》十卷

曾異撰《紡授堂集》二十七卷

《孫承宗奏議》三十卷,《文集》十八卷

賀逢聖《文類》五卷

蔣德璟《敬日草》九卷

黃景昉《甌安館集》三十卷

《倪元璐奏牘》三卷,《詩文集》十七卷

《李邦華奏議》六卷,《文集》八卷

《王家彥奏議》五卷,《文集》五卷

《凌義渠文集》六卷

《馬世奇文集》六卷、詩三卷

《劉理順文集》十二卷

《金鉉文集》六卷

《鹿善繼文稿》四卷

《孫元化文集》一百卷

熊人霖《華川集》二十四卷

陳山毓《靖質居士集》六卷

陳龍正《幾亭集》六十四卷

陳際泰《太乙山房集》十四卷

《吳應箕文集》二十八卷

《呂維祺詩文集》二十卷

徐石麒《可經堂集》十二卷

黃道周《石齋集》十二卷

張肯堂《莞爾集》二十卷

袁繼咸《六柳堂集》三卷

黃端伯《瑤光閣集》八卷

《金聲文集》九卷

陳函輝《寒山集》十卷

艾南英《天慵子集》六卷

《黎遂球文集》二十一卷、詩十卷

《李日宣奏議》十六卷,《敬修堂集》三十卷

黃淳耀《陶庵集》七卷

《侯峒曾文集》四十卷

《侯岐曾文集》三十卷

已上萬歷、天啟、崇禎時。

宗泐《全室外集》十卷,《西游集》一卷洪武中,宗泐為右善世,奉使西域求遺經,往返道中之作。

來復《蒲庵集》十卷

法住《幻住詩》一卷

清簷《蘭江望雲集》二卷

廷俊《泊川文集》五卷

克新《雪廬稿》一卷

守仁《夢觀集》六卷

如蘭《支離集》七卷

德祥《桐嶼詩》一卷

子楩《水雲堂稿》二卷

宗衍《碧山堂集》三卷

妙聲《東皋錄》七卷

元極《圓庵集》十卷

溥洽《雨軒外集》八卷

善啟《江行倡和詩》一卷

大旟《竺庵集》二卷

覺澄《雨華詩集》二卷

明秀《雪江集》三卷

普泰《野庵詩集》三卷

宗林《香山夢BO集》一卷

方澤《冬谿內外集》八卷

真可《紫柏老人集》十五卷

德清《憨山夢游集》四十卷

弘恩《雪浪齋詩集》二卷

寬悅《堯山藏草》五卷

法杲《雪山詩集》八卷

一元《山居百詠》一卷

如愚《空華集》二卷,《飲河集》二卷,《四悉稿》四卷

智舷《黃山老人詩》六卷

慧秀《秀道人集》十三卷

傳慧《浮幻齋詩》三卷,《流雲集》二卷

圓復《三支集》二卷,《一葦集》二卷

元賢《禪餘集》四卷

張宇初《峴泉文集》二十卷

鄧羽《觀物吟》一卷

張友霖《鐵礦集》二卷

《邵元節集》四卷

汪麗陽《野懷散稿》一卷

張蚩蚩《適適吟》一卷

顏復膺《潛庵詠物詩》六卷

已上方外。

安福郡主《桂華詩集》一卷

周憲王宮人夏雲英《端清閣詩》一卷

《陳德懿詩》四卷

《楊夫人詞曲》五卷

孟淑卿《荊山居士詩》一卷

《朱靜庵詩集》十卷

《鄒賽貞詩》四卷

《楊文儷詩》一卷

金文貞《蘭莊詩》一卷

馬閒卿《芷居集》一卷

端淑卿《綠窗詩稿》四卷

王鳳嫻《焚餘草》五卷

張引元、張引慶《雙燕遺音》一卷

《董少玉詩》一卷

周玉如《雲巢詩》一卷

邢慈靜《非非草》一卷

沈天孫《留香草》四卷

屠瑤瑟《留香草》一卷

袁九淑《伽音集》一卷

姚青蛾《玉鴛閣詩》二卷

王虞鳳《罷繡吟》一卷

《劉苑華詩》一卷

陸卿子《考槃集》六卷,《雲臥閣稿》四卷,《玄芝集》四卷

徐媛《絡緯吟》十二卷

沈紉蘭《效顰集》一卷

項蘭貞《裁雲草》一卷,《月露吟》一卷

薄少君《嫠泣集》一卷

方孟式《紉蘭閣集》八卷

方維儀《清芬閣集》七卷

黃幼藻《柳絮編》一卷

桑貞白《香BP稿》二卷

已上閨秀。

──右別集類,一千一百八十八部,一萬九千八百九十六卷。

《歷代名臣奏議》三百五十卷永樂中黃淮等奉敕纂輯。

王恕《歷代諫議錄》一百卷

謝鐸《赤城論諫錄》十卷鐸與黃孔昭同輯天臺人文之有關治道者,宋十人,明六人。

張瀚《明疏議輯略》三十七卷

張國綱《明代名臣奏疏》二十卷

張鹵《嘉隆疏鈔》二十卷

吳亮《萬曆疏鈔》五十卷

孫甸《明疏議》七十卷

朱吾弼《明留臺奏議》二十卷

慶靖王旃《文章類選》四十卷

鄭淵《續文類》五十卷

鄭柏《續文章正宗》四十卷

王稌《國朝文纂》四十卷

趙友同《古文正原》十五卷

吳訥《文章辨體》五十卷,《外集》五卷

李伯璵《文翰類選大成》一百六十二卷

張洪《古今箴銘集》十四卷

程敏政《明文衡》九十八卷

楊循吉《明文寶》八十卷

姚福《明文苑通編》十卷

賀泰《唐文鑑》二十一卷

李夢陽《古文選增定》二十二卷

劉節《廣文選》八十二卷

李堂《正學類編》十五卷

謝朝宣《古文會選》三十卷

楊慎《古雋》八卷

林希元《古文類鈔》二十卷

唐順之《文編》六十四卷,《明文選》二十卷

張時徹《明文範》六十八卷

汪宗元《明文選》二十卷

張士水龠《明文纂》五十卷

慎蒙《明文則》二十二卷

薛甲《大家文選》二十二卷

王逢年《文統》一百卷

茅坤《唐宋八大家文鈔》一百四十四卷

徐師曾《文體明辨》八十四卷《正錄》六十卷,《附錄》二十四卷

褚鈇《匯古菁華》二十四卷

姚翼《歷代文選》五十卷

陳第《屈宋古音義》三卷

郭棐《名公玉屑錄》二十卷

胡時化《名世文宗》三十卷

李鐸《西漢菁華》十四卷

申用懋《西漢文苑》十二卷

湯紹祖《續文選》二十七卷

孫幰《今文選》二十五卷

馬繼銘《廣文選》十二卷

劉世教《賦紀》一百卷

潘士達《古文世編》一百卷

陳翼飛《文儷》六十卷

何喬遠《明文徵》七十四卷

汪瑗《楚辭集解》十五卷

陳仁錫《古文奇賞》二十二卷,《續》二十四卷,《三續》二十六卷,《四續》五十三卷,《明文奇賞》四十卷

王志堅《古文瀾編》二十卷,《續編》三十卷,《四六法海》十二卷

楊瞿崍《明文翼統》四十卷

張燦《擬離騷》二十卷

黃道周《續離騷》二卷

胡震亨《續文選》十四卷

方岳貢《古文國瑋集》五十二卷

俞王言《辭賦標義》十八卷

陳山毓《賦略》五十卷

陳子龍《明代經世文編》五百八卷

張溥《古文五刪》五十二卷,《漢魏百三名家集》

陳經邦《明館課》五十一卷

張陽《新安文粹》十五卷

趙鶴《金華文統》十三卷

阮元聲《金華文徵》二十卷

張應麟《海虞文苑》二十四卷

錢穀《續吳都文粹》六百卷

董斯張《吳興藝文補》七十卷

楊慎《尺牘清裁》十一卷,《古今翰苑瓊琚》十二卷

王世貞《增集尺牘清裁》二十八卷

梅鼎祚《書記洞詮》一百二十卷

俞安期《啟雋類函》一百卷

凌稚隆《名公翰藻》五十二卷

宋公傳《元時體要》十四卷南海鄧林序稱共嘗同修東觀書,蓋永樂初纂修《大典》者。

高棅《唐詩品匯》九十卷,《拾遺》十卷,《唐詩正聲》二十二卷

周敘《唐詩類編》十卷

蕭儼《明代風雅廣選》三十七卷

楊慎《風雅逸編》十卷,《選詩外編》九卷,《五言律祖》六卷,《近體始音》五卷,《詩林振秀》十一卷,《明詩鈔》七卷

何景明《校漢魏詩》十四卷

黃佐《明音類選》十八卷

徐泰《明代風雅》四十卷

程敏政《詠史詩選》十五卷

徐獻忠《六朝聲偶集》七卷,《百家唐詩》一百卷

黃德水《初唐詩紀》三十卷

李於鱗《古今詩刪》三十四卷,《唐詩選》七卷

何喬新《唐律群玉》十六卷

鄒守愚《全唐詩選》十八卷

謝東山《明近體詩鈔》二十九卷

馮惟訥《詩紀》一百五十六卷,《風雅廣逸》七卷

王宗聖《增補六朝詩匯》一百十四卷

張之象《古詩類苑》一百二十卷,《唐詩類苑》二百卷,《唐雅》二十六卷

卓明卿《唐詩類苑》一百卷

潘是仁《宋元名家詩選》一百卷

毛應宗《唐雅同聲》五十卷

俞安期《詩雋類函》一百五十卷

許學彝《詩源辨體》十六卷

俞憲《盛明百家詩》一百卷

盧純學《明詩正聲》六十卷

符觀《唐詩正體》七卷,《宋詩正體》四卷,《元詩正體》四卷,《明詩正體》五卷

鐘惺《古唐詩歸》四十七卷

臧懋循《古詩所》五十二卷,《唐詩所》四十七卷

李騰鵬《詩統》四十二卷

張可仕《補訂明布衣詩》一百卷

沈子來《唐詩三集合編》七十八卷

陳子龍《明詩選》十三卷

胡震亨《唐音統簽》一千二十四卷甲簽帝王詩七卷,乙簽初唐詩七十九卷,丙簽、盛唐詩一百二十五卷,丁簽中唐詩三百四十一卷,戊簽晚唐詩二百一卷,又餘閏六十四卷,己簽五唐雜詩四十六卷,庚簽僧詩三十八卷、道士詩六卷、宮閨詩九卷、外國詩一卷,辛簽樂章十卷、雜曲五捲、填詞十卷、歌一卷、謠一卷、諧謔四卷、諺一卷、語一卷、酒令一卷、題語判語一卷、讖記一卷、占辭一卷、蒙求一卷、章咒一卷、偈頌二十四卷、壬簽仙詩三卷、神詩一卷、鬼詩二卷、夢詩一卷、物怪詩一卷,癸簽體凡、發微、評匯、樂通、詁箋、談叢、集錄,凡三十六卷。

曹學牷《石倉十二代詩選》八百八十八卷古詩十三卷,唐詩一百十卷,宋詩一百七卷,元詩五十卷,明詩一集八十六卷,二集一百四十卷,三集一百卷,四集一百三十二卷,五集五十卷,六集一百卷。

徐獻忠《樂府原》十五卷

胡瀚《古樂府類編》四卷

陳耀文《花草粹編》十二卷

錢允治《國朝詩餘》五卷

沈際飛《草堂詩餘》十二卷

卓人月《古今詞統》十六卷

毛晉《宋六十家詞》六十卷

程明善《嘯餘譜》十卷

黎淳《國朝試錄》六百四十卷輯明成化已前試士之文。丘浚為序。

汪克寬《春秋作義要訣》一卷

楊慎《經義模範》一卷

梁寅《策要》六卷

劉定之《十科策略》八卷

張和《筱庵論鈔》一卷

黃佐《論原》十卷,《論式》三卷

戴鱀《策學會元》四十卷

唐順之《策海正傳》十二卷

茅維《論衡》六卷,《表衡》六卷,《策衡》二十二卷

陳禹謨《類字判草》二卷

《明狀元策》十二卷坊刻本。

《四書程文》二十九卷,《五經程文》三十二卷,《論程文》十卷,《詔誥表程文》五卷,《策程文》二十卷

已上五種,見葉盛《菉竹堂書目》,皆明初舉業程式。

──右總集類,一百六十二部,九千八百一十卷

《詩學梯航》一卷宣德中,周敘等奉敕編。

寧獻王《臒仙文譜》八卷,《詩譜》一卷,《詩格》一卷,《西江詩法》一卷

寧靖王奠培《詩評》一卷

宋元禧《文章緒論》一卷

唐之淳《文斷》四卷

溫景明《藝學淵源》四卷

閔文振《蘭莊文話》一卷,《詩話》一卷

張大猷《文章源委》一卷

王弘誨《文字談苑》四卷

朱荃宰《文通》二十卷

瞿佑《吟堂詩話》三卷

懷悅《詩家一指》一卷

葉盛《秋臺詩話》一卷

游潛《夢蕉詩話》二卷

李東陽《懷麓堂詩話》一卷

徐禎卿《談藝錄》一卷

《都穆詩話》二卷

強晟《汝南詩話》四卷

沈麟《唐詩世紀》五卷

楊慎《升庵詩話》四卷

程啟充《南溪詩話》三卷

安磐《頤山詩話》二卷

黃卿《編苕詩話》八卷

宋孟清《詩學體要類編》三卷

朱承爵《詩話》一卷

顧元慶《夷白齋詩話》一卷

陳霆《渚山堂詩話》三卷

皇甫循《解頤新語》八卷

黃省曾《詩法》八卷

梁格《冰川詩式》四卷

邵經邦《律詩指南》四卷

《謝東山詩話》四卷

王世懋《藝圃擷餘》一卷

謝榛《詩家直說》四卷

俞允文《名賢詩評》二十卷

胡應麟《詩藪》二十卷

凌云《續全唐詩話》十卷

郭子章《豫章詩話》六卷,《續》十二卷

謝肇淛《小草齋詩話》四卷

趙宧光《彈雅集》十卷

曹學牷《蜀中詩話》四卷

程元初《名賢詩指》十五卷

王昌會《詩話匯編》三十二卷右文史類,四十八部,二百六十卷。