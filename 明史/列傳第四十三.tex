\article{列傳第四十三}

\begin{pinyinscope}
○宋晟薛祿郭義金玉劉榮朱榮費瓛譚廣陳懷馬亮蔣貴孫琬任禮趙安趙輔劉聚

宋晟,字景陽,定遠人。父朝用,兄國興,並從渡江,皆積功至元帥。攻集慶,國興戰歿,晟嗣其職。既而朝用請老,晟方從鄧愈克徽州,召還,襲父官。累進都指揮同知,歷鎮江西、大同、陜西。洪武十二年坐法降涼州衛指揮使。十七年五月討西番叛酋,至亦集乃路,擒元海道千戶也先帖木兒、國公吳把都剌赤等,俘獲萬八千人,送酋長京師,簡其精銳千人補卒伍,餘悉放遣。召還,復為都指揮,進右軍都督僉事,仍鎮涼州。

二十四年充總兵官,與都督劉真討哈梅里。其地去肅州千餘里。晟令軍中多具糧糗,倍道疾馳,乘夜至城下。質明,金鼓聲震地,闔城股慄,遂克之。擒其王子別兒怯帖木兒,及偽國公以下三十餘人,收其部落輜重以歸。自是番戎慴服,兵威極於西域。明年五月從藍玉征罕東,徇阿真川,土酋哈昝等遁去。師還,調中軍都督僉事。

二十八年六月,從總兵官周興出開原,至忽剌江。部長西陽哈遁,追至甫答迷城,俘人畜而還。明年拜征南右副將軍,討廣西帡幪諸寨苗,擒斬七千餘人。又明年,總羽林八衛兵討平五開、龍里苗。三十一年出鎮開平,從燕王出塞,還城萬全諸衛。建文改元,仍鎮甘肅。

成祖即位,入朝,進後軍左都督,拜平羌將軍,遣還鎮。永樂三年招降把都帖木兒、倫都兒灰等部落五千人,獲馬駝牛羊萬六千。封西寧侯,祿千一百石,世指揮使。

晟凡四鎮涼州,前後二十餘年,威信著絕域。帝以晟舊臣,有大將材,專任以邊事,所奏請輒報可。御史劾晟自專。帝曰:「任人不專則不能成功,況大將統制一邊,寧能盡拘文法。」即敕晟以便宜從事。晟嘗請入朝。報曰:「西北邊務,一以委卿,非召命,毋輒來。」尋命營河西牧地,及圖出塞方略。會病卒,五年七月也。

晟三子。長瑄,建文中為府軍右衛指揮使,戰靈璧,先登,斬數級,力鬥死。

瑄弟琥,尚成祖女安成公主,得嗣侯,予世券。八年佩前將軍印,鎮甘肅。十年與李彬捕叛酋老的罕,俘斬甚眾。召還。洪熙元年坐不敬奪爵,並削駙馬都尉官。宣德中復都尉。

琥既廢,弟瑛嗣。瑛尚咸寧公主。正統中,歷掌左軍前府事。瓦剌也先入寇,瑛充總兵官,督大同守將朱冕、石亨等戰陽和,全軍敗沒,瑛及冕皆戰死。贈鄆國公,謚忠順。

子傑嗣。景泰中典禁兵宿衛,以謹慎稱。卒,子誠嗣。署右府事,復佩平羌將軍印,鎮甘肅。誠有材武,嘗出獵至涼州,遇寇掠牛馬北去。誠三矢殪三人,寇驚散,盡驅所掠還。九傳至孫裕德,死流寇難。

薛祿,膠人。行六,軍中呼曰「薛六」。既貴,乃更名「祿」。祿以卒伍從燕起兵,首奪九門。真定之戰,左副將軍李堅迎斗。鋒始交,祿持槊刺堅墜馬,擒之。擢指揮僉事。從援永平,下大寧、富峪、會州、寬河。還救北平,先驅敗南軍游騎。進指揮同知。攻大同,為先鋒。戰白溝河,追奔至濟南,遷指揮使。戰東昌,以五十騎敗南兵數百。時成祖為盛庸所敗,還走北平。庸檄真定諸將屯威縣、深州,邀燕歸路。祿皆擊走之。戰滹沱河,右軍卻。祿馳赴陣,出入數十戰,破之。追奔至夾河,斬馘無算。戰單家橋,為平安所執。奮脫縛,拔刀殺守卒,馳還復戰,大敗安軍。掠順德、大名、彰德。攻西水寨,生擒都指揮花英。乘勝下東阿、東平、汶上,連戰淝河、小河、靈璧,功最。入京師,擢都督僉事。

永樂六年進同知。八年充驃騎將軍,從北征,進右都督。十年上言:「自古用人,必資豫教。今武臣子弟閒暇不教,恐緩急無可使者。」帝韙其言。會四方送幼軍數萬至,悉隸祿操習之。十五年以行在後軍都督董營造。

十八年十二月定都北京,授奉天靖難推誠宣力武臣,封陽武侯,祿千一百石。二十一年將右哨從北征。還,討平長興盜。二十二年再領右哨從北征。

仁宗即位,命掌左府,加太子太保,予世券。洪熙元年充總兵官,備禦塞外。尋以獲寇功,益祿五百石。是年頒諸將軍印於各邊鎮,祿佩鎮朔大將軍印,巡開平,至大同邊。

宣宗即位,召還,陳備邊五事。尋復遣巡邊。宣德元年從征樂安,為前鋒。高煦就擒,留祿與尚書張本鎮撫之。明年春,奉詔巡視畿南諸府城池,嚴戒軍士毋擾民,違者以軍法論。是夏復佩大將軍印,北巡開平,還駐宣府。敵犯開平,無所得而退,去城三百餘里。祿帥精兵晝伏夜行,三夕至。縱輕騎蹂敵營,破之,大獲人畜。師還,敵躡其後,復奮擊敗之,敵由是遠遁。召還。三年從北征,破敵於寬河,留鎮薊州、永平。復數佩鎮朔印,巡邊護餉,出開平、宣府間。五年遇敵於鳳凰嶺,斬獲多,加太保。上言永寧衛團山及雕鶚、赤城、雲州、獨石宜築城堡,便守禦。詔發軍民三萬六千赴工,精騎一千五百護之,皆聽祿節制。臨行賜詩,以山甫、南仲為比。祿武人不知書,以問楊士奇。士奇曰:「上以古賢人待君也。」祿拊心曰;「祿安敢望前賢,然敢不勉圖報上恩萬一。」其年六月有疾,召還。踰月卒。贈鄞國公,謚忠武。

祿有勇而好謀,謀定後戰,戰必勝。紀律嚴明,秋毫無犯。善撫士卒,同甘苦,人樂為用。「靖難」諸功臣,張玉、朱能及祿三人為最,而祿逮事三朝,巋然為時宿將。

孫詵嗣。至曾孫翰卒,無子,族人爭襲,久之不得請,田宅並入官,世絕者三十餘年。萬曆五年乃復封翰族子鋹為侯。再傳至濂。崇禎末,京師陷,被害。

永樂中從起兵北平,後積功至大將,封侯伯不以「靖難」功者,薛祿及郭義、金玉、劉榮、朱榮凡五人,而義、玉與祿同日封云。

郭義,濟寧人。洪武時,累功為燕山千戶。從成祖入京師,累遷左都督。永樂九年坐曠職謫交阯,立功,已而宥之。數從出塞,有功,封安陽侯,祿千一百石,亦授奉天靖難武臣號。時義在南京,疾革,聞命而卒。

金玉,江浦人。襲父官為羽林衛百戶,調燕山護衛。從起兵有功,累遷河南都指揮使。永樂三年進都督僉事。八年充鷹揚將軍從北征。師旋,為殿。至長秀川,收敵所棄牛羊雜畜亙數十里。十四年討平山西妖賊劉子進。論前後功,封惠安伯,祿九百石。十九年卒。妾田氏自經以殉,贈淑人。

劉榮,宿遷人。初冒父名江。從魏國公徐達戰灰山、黑松林。為總旗,給事燕邸。雄偉多智略,成祖深器之,授密雲衛百戶。從起兵為前鋒,屢立戰功。徇山東,與朱榮帥精騎三千,夜襲南軍於滑口,斬數千人,獲馬三千,擒都指揮唐禮等。累授都指揮僉事。戰滹沱河,奪浮橋,掠館陶、曹州,大獲。還軍救北平,敗平安軍於平村。楊文以遼東兵圍永平,江往援,文引卻。江聲言還北平,行二十餘里,卷甲夜入永平。文聞江去,復來攻。江突出掩擊,大敗之。斬首數千,擒指揮王雄等七十一人。遷都指揮使。從至淝河,與白義、王真以輕騎誘致平安,敗之。

時南軍駐宿州,積糧為持久計。成祖患之,議絕其餉道。命江將三千人往,趑趄不行。成祖大怒,欲斬之。諸將叩首請,乃免。渡江策功,以前罪不封,止授都督僉事。遷中府右都督。

永樂八年從北征,以遊擊將軍督前哨。乘夜據清水源,敗敵斡難河,復敗阿魯台於靖虜鎮。師還為殿,即軍中進左都督,遣鎮遼東。敵闌入殺官軍。帝怒,命斬江,既而宥之。九年復鎮遼東。十二年再從北征,仍為前鋒,將勁騎偵敵於飲馬河。見敵騎東走,追至康哈里孩,擊斬數十人。復與大軍合擊馬哈木於忽失溫,下馬持短兵突陣,斬獲多,受上賞。復充總兵官,鎮遼東。

倭數寇海上,北抵遼,南訖浙、閩,瀕海郡邑多被害。江度形勢,請於金線島西北望海堝築城堡,設烽堠,嚴兵以待。十七年六月,尞者言東南海島中舉火。江急引兵赴堝上。倭三十餘舟至,泊馬雄島,登岸奔望海堝。江依山設伏,別遣將斷其歸路,以步卒迎戰,佯卻。賊入伏中,炮舉伏起,自辰至酉,大破賊。賊走櫻桃園空堡中,江開西壁縱之走。復分兩路夾擊,盡覆之,斬首千餘級,生擒百三十人。自是倭大創,不敢復入遼東。詔封廣寧伯,祿千二百石,予世券,始更名榮。尋遣還鎮。明年四月卒。

榮為將,常為軍鋒,所向無堅陣。馭士卒有紀律,恩信嚴明。諸款塞者,撫輯備至。既卒,人咸思之。贈侯,謚忠武。

子湍嗣。卒,無子,弟安嗣。正統十四年與郭登鎮大同。也先擁英宗至城下,邀登出見,登不可。安出見,伏哭帝前。景帝降敕切責。安馳至京師,言奉上皇命來告敵情,且言進己為侯。群臣交劾,下獄論死。會京師戒嚴,釋安充總兵官,陣東直門。寇退,進都督同知,守備白羊口,復伯爵。英宗復位,予世侯,再益祿三百石。曹欽反,安被創,加太子少傅。成化中卒。贈嶧國公,謚忠僖。傳爵至明亡。

朱榮,字仲華,沂人。洪武十四年,以總旗從西平侯沐英征雲南。累官副千戶。守大寧,降於成祖。襲孫霖於滑口,圍定州,斷南軍餉道,大小二十餘戰,論功授都督僉事。

永樂四年從新城侯張輔征交阯,破雞陵關,會沐晟於白鶴。輔等議於嘉林江上流濟師,遣榮陣下流十八里,日增其數以惑賊。又作舟筏為欲濟狀,以牽制之。賊果分兵渡江登岸。榮等奮擊,大破之。大軍進克多邦城,榮功為多。帝以榮嘗怠事,師還論功,僅擢右都督,賜白金鈔幣。七年復從輔討賊餘黨,平之。

明年督右掖,從征阿魯台,與劉榮並進左都督。十二年復從北征,與榮俱為前鋒。其冬充總兵官,鎮大同。修忙牛嶺、兔毛河、赤山、榆楊口、來勝諸城,寇不敢近。居三年,召還。

十八年,代劉榮鎮遼東。二十年復從北征,為前鋒。駐雕鶚詗寇,以五千騎視敵所向。大軍次玉沙泉。榮帥銳士三百人,人三馬,齎二十日糧深入。敵已棄牛羊馬駝北走,悉收之,焚其輜重,移師破兀良哈。師還,封武進伯,祿千二百石,仍鎮遼東。二十二年復從北征。已,還鎮。洪熙元年,佩征虜前將軍印,鎮如故。其年七月卒於鎮。贈侯,謚忠靖。

子冕嗣。以晉王濟熿新廢,命鎮山西,尋召還。六年命輸餉獨石,因巡其地。正統四年,佩征西將軍印,鎮大同。十四年從北征,戰於陽和,死之。謚忠愨。子瑛嗣。傳爵至明亡。

費瓛,定遠人。祖愚,洪武時為燕府左相,改授燕山中護衛指揮使。傳子肅。至瓛從成祖起兵有功,累進後軍都督僉事。

永樂八年春,涼州衛千戶虎保、永昌衛千戶亦令真巴等叛,眾數千,屯據驛路。新附伯顏帖木兒等應之。西鄙震動。都指揮李智擊之不勝。賊聲言攻永昌、涼州城。皇太子命瓛往討。至涼州,智及都指揮陳懷以師會,遂進兵鎮番。遇賊於雙城。瓛擊其左,懷等擊其右。賊大敗走,斬首三百餘級。追奔至黑魚海,獲賊千餘,馬駝牛羊十二萬。虎保等遠遁。乃班師。

十二年充總兵官,鎮甘肅。瓛以肅州兵多糧少,脫有調發,猝難措置,請以臨鞏稅糧付近邊軍丁轉運。又以涼州多閒田,請給軍屯墾。從之。洪熙元年予平羌將軍印。永樂時,諸邊率用宦官協鎮,恣睢專軍務,瓛亦為所制。仁宗知之,賜璽書責之曰:「爾以名臣後,受國重寄,乃俯首受制於人,豈大丈夫所為!其痛自懲艾,圖後效。」瓛得書陳謝。

宣宗嗣位,進右府左都督。元年七月入朝,封崇信伯,祿千一百石。從征高煦,次流河驛。帝念前鋒薛祿軍少,命瓛帥兵益之。還,予世券,復鎮甘肅。二年,沙州衛賊屢劫撒馬兒罕及亦力把里貢使,瓛討破之。明年卒於鎮。

瓛為人和易,善撫士。在鎮十五年,境內寧謐。子釗嗣。從征鄧茂七,還掌都督府。天順中,受武定侯郭英次孫昭賂,誣嫡孫昌不孝,欲奪其爵。法司請逮治,詔解府事。卒,子淮嗣爵。傳至明亡乃絕。

譚廣,字仲宏,丹徒人。洪武初,起卒伍,從征金山,為燕山護衛百戶。從成祖起兵,以百騎掠涿州,生得將校三十人。戰白溝、真定、夾河咸有功,屢遷指揮使,留守保定。都督韓觀帥師十二萬來攻。廣以孤軍力拒四十餘日,伺間破走之。

永樂九年進大寧都指揮僉事。董建北京。既而領神機營,從北征,充驍騎將軍。十一年練軍山西。明年從徵九龍口,為前鋒。賊數萬憑岸,廣命挽強士射之。萬矢齊發,死者無算。乘勝夾擊,賊大敗。論功,進都督僉事。

仁宗嗣位,擢左都督,佩鎮朔將軍印,鎮宣府。宣德三年,請軍衛如郡縣例,立風雲雷雨山川社稷壇。六年以宣府糧少,請如開平、獨石召商中鹽納粟,以足兵食。俱從之。明年,帝從戶部議,令他衛軍戍宣府者,悉遣還屯種。廣上言:「臣所守邊一千四百餘里,敵人窺伺,竊發無時。脫有警,徵兵數百里外,勢豈能及?屯種之議,臣愚未見其可。」帝以邊卒戍守有餘,但命永樂中調戍者勿遣。

正統初,朝議以脫歡雖款塞,狡謀未可測,命廣及他鎮總兵官陳懷、李謙、王彧圖上方略。廣等各上議,大要謂:「邊寇出沒不常,惟守禦為上策。宜分兵扼要害,而間遣精稅巡塞外,遇敵則量力戰守,間諜以偵之,輕兵以躡之。寇來無所得,去有所懼,則邊患可少弭。」帝納其言。六年十一月以禦敵功,封永寧伯,祿千二百石,仍鎮宣府。八年乞致仕。優詔不許。明年十月召還陛見。帝憫其老,免常朝。是月卒,年八十二。謚襄毅。

廣長身多力,奮跡行伍至大將,大小百餘戰,未嘗挫衄。在宣府二十年,修屯堡,嚴守備,增驛傳,又請頒給火器於各邊。將校失律,即奏請置罪,而撫士卒有恩。邊徼帖然,稱名將。嘗逞憤杖殺都司經歷,又以私憾杖百戶,並為言官所劾。置不問,既卒,吏部言非世券,授其子序指揮使。

陳懷,合肥人。襲父職為真定副千戶。永樂初,積功至都指揮僉事。從平安南,進都指揮使,涖山西都司事。再從張輔擒安南賊簡定,從都督費瓛征涼州叛人虎保,皆有功。仁宗立,進都督同知。

宣德元年,代梁銘為總兵官,鎮寧夏。時官軍征交阯者屢敗,詔發松潘軍援之,將士憚行。千戶錢宏與眾謀,詐言番叛,帥兵掠麥匝諸族。番人震恐,遂反。殺指揮陳傑等,陷松潘、疊溪,圍威、茂諸州。指揮吳玉、韓整、高隆相繼敗績,西鄙騷然。詔遣鴻臚丞何敏、指揮吳瑋往招之,而命懷統劉昭、趙安、蔣貴帥師數萬隨其後。瑋等至,賊不順命。瑋與龍州知州薛繼賢擊賊,復松潘。比懷至,仍用瑋前鋒,遂復疊溪,降二十餘寨。招撫復業者萬二千二百餘戶,歸所掠軍民二千二百餘人,事遂定。進左都督,厚賚金幣,而絀瑋功不錄。懷留鎮四川。在鎮驕縱不法,干預民事,受賕庇罪人,侵奪屯田,笞辱僉事柴震等,數為言官所劾。帝降敕責讓,復以御史王禮彈章示之。懷引罪。置不問。

六年,松潘勒都、北定諸族暨空郎、龍溪諸寨番復叛。懷遣兵戰敗,指揮安寧等死者三百餘人。懷乃親督兵深入,破革兒骨寨,進攻空郎乞兒洞。賊敗,斬首墜崖死者無算。革兒骨賊復聚生苗邀戰。擊破之,剿戮殆盡。於是任昌、牛心諸寨番聞風乞降,群寇悉平。久之,巡按御史及按察使復奏:「懷僭侈踰分。每旦,令三司官分班立,有事跪白。懷中坐,稱旨行遣。且日荒於酒,不飭邊備,致城寨失陷。」宣宗怒,召懷還,命文武大臣鞫之,罪當斬。下都察院獄,宥死落職。

正統二年以原官鎮大同。時北人來貢者日給廩餼,為軍民累。懷言於朝,得減省。居二年,以老召還,命理中府事。九年春,與中官但住出古北口,徵兀良哈。還與馬亮等同封,而懷得平鄉伯。十四年扈駕北征,死土木。贈侯,謚忠毅。

子輔乞襲爵,吏部言非世券,執不許。景帝以懷死事,許之。輔卒,子政請襲,吏部執如初,中旨許嗣。政鎮兩廣久,自陳軍功,乞世券,吏部復執不可,詔予之。政卒,子信嗣。弘治中卒,無子,弟俊嗣指揮使。

馬亮,淇人。以燕山衛卒從成祖起兵,累功至都指揮僉事。宣宗時官至左都督。兀良哈之役,偕中官劉永誠出劉家口,至黑山、大松林、流沙河諸處,遇賊勝之。還封招遠伯。是役也,王振主之,故諸將功少率得封。

亮善騎射,每戰身先士卒,所向克捷,時稱驍將。為伯三年卒。謚榮毅。

蔣貴,字大富,江都人。以燕山衛卒從成祖起兵。雄偉多力,善騎射,積功至昌國衛指揮同知。從大軍征交阯及沙漠,遷都指揮僉事,掌彭城衛事。

宣德二年,四川松潘諸番叛,充右參將,從總兵官陳懷討之。募鄉導,絕險而進,薄其巢。一日十數戰,大敗之。進都指揮同知,鎮守密雲。七年復命為參將,佐懷鎮松潘。明年進都督僉事,充副總兵,協方政鎮守。又明年,諸番復叛,政等分道進討。貴督兵四千,攻破任昌大寨。會都指揮趙得、宮聚兵以次討平龍溪等三十七寨,斬首一千七百級,投崖墜水死者無算。捷聞,進都督同知,充總兵官,佩平蠻將軍印,代政鎮守。

英宗即位,以所統皆極邊地,奏增軍士月糧。正統元年召還,為右都督。阿台寇甘、涼,邊將告急。命佩平虜將軍印,帥師討之。賊犯莊浪,都指揮江源戰死,亡士卒百四十餘人。侍郎徐晞劾貴,朝議以貴方選軍甘州,勢不相及。而莊浪及晞所統,責晞委罪。置貴不問。

明年春,諜報敵駐賀蘭山後。詔大同總兵官方政、都指揮楊洪出大同迤西,貴與都督趙安出涼州塞會剿。貴至魚兒海子,都指揮安敬言前途無水草,引還。鎮守陜西都御史陳鎰言狀,尚書王驥出理邊務,斬敬,責貴立功。貴感奮,會朵兒只伯懼罪,連遣使入貢,敵勢稍弱。貴帥輕騎敗之於狼山,追抵石城。已,聞朵兒只伯依阿台於兀魯乃地,貴將二千五百人為前鋒往襲。副將李安沮之,貴拔劍厲聲叱安曰:「敢阻軍者死!」遂出鎮夷。間道疾馳三日夜,抵其巢。阿台方牧馬,貴猝入馬群,令士卒以鞭擊弓韣驚馬,馬盡佚。敵失馬,挽弓步斗。貴縱騎蹂擊,指揮毛哈阿奮入其陣,大敗之。復分軍為兩翼,別遣百騎乘高為疑兵,轉戰八十里。會任禮亦追敵至黑泉,阿台與朵兒只伯以數騎遠遁,西邊悉平。三年四月,王驥以捷聞,論功封定西伯,食祿一千二百石,給世券。明年代任禮鎮甘肅。又明年冬,以征麓川蠻思任發,召還京。

六年命佩平蠻將軍印,充總兵官,與王驥帥師抵金齒。分路進搗麓川上江寨,破杉木籠山七寨及馬鞍山象陣,功皆第一。事詳《王驥傳》。明年,師還,進封侯,益祿三百石。

八年夏,復佩平蠻將軍印,與王驥討思任發子思機發,攻破其寨。明年,師還,賞賚甚渥,加歲祿五百石。是役也,貴子雄乘敵敗,帥三十人深入。敵扼其後,自刎沉於江。贈懷遠將軍、彭城衛指揮使。

十四年正月,貴卒,年七十。贈涇國公,謚武勇。

貴起卒伍,不識字,天性朴實。忘己下人,能與士卒同甘苦。出境討賊,衣糧器械常身自囊負,不役一人,臨陣輒身先之,以故所向有功。

子義,病不能嗣,以義子琬嗣侯。天順末,佩平羌將軍印,總兵甘肅,築甘州沙河諸屯堡。

成化八年召還,協守南京,兼督操江。十年入督十二團營,尋兼總神機營兵。上言:「太祖肇建南京,京城外復築土城以衛居民,誠萬世之業。今北京但有內城。己巳之變,敵騎長驅直薄城下,可以為鑒。今西北隅故址猶存。亟行勸募之令,濟以工罰,成功不難。」又言:「大同、宣府諸塞下,腴田無慮數十萬,悉為豪右所占。畿內八府,良田半屬勢要家。細民失業。脫使邊關有警,內郡何資?運道或梗,京師安給?請遣給事、御史按核塞下田,定其科額;畿內民田,嚴戢豪右毋得侵奪。庶兵民足食而內外有備。」章下所司。雖不盡行,時論韙之。十三年帥京軍防秋大同、宣府,陳機宜十餘事。皆報可。十五年偕汪直按遼東邊事。

二十年佩將軍印,出禦邊寇。寇退班師,累加太保兼太子太傅。卒,贈涼國公,謚敏毅。

子驥嗣,典京營兵。弘治中充總兵官,歷鎮薊州、遼東、湖廣。官中外二十年,家無餘資。再傳至孫傅。嘉靖中,累典軍府。佩征蠻將軍印,鎮兩廣。以平海賊及慶遠瑤功,加太子太保。明亡,爵絕。

任禮,字尚義,臨漳人。以燕山衛卒從成祖起兵,積功至山東都指揮使。永樂二十年擢都督僉事。從北征,前行偵敵,還受厚賞。仁宗即位,命掌廣西都司事。尋改遼東。宣宗立,進都指揮同知。從平樂安,又從征兀良哈,還為後拒。英宗立,進左都督。

正統元年佩平羌將軍印,充左副總兵鎮甘肅。阿台、朵兒只伯數犯肅州,璽書譙讓。二年復寇莊浪。都指揮魏榮擊卻之,擒朵兒只伯姪把禿孛羅。禮以聞。三年與王驥、蔣貴出塞,敗朵兒只伯於石城。復分道至梧桐林、亦集乃,進至黑泉而還。斬獲多,封寧遠伯,祿千二百石。明年還朝。又明年代貴鎮甘肅。八年,赤斤蒙古衛都督且旺失加苦也先暴橫,欲移駐也洛卜剌。禮以其地近肅州,執不許。已,奏請建寺於其地。禮復言許其建寺,彼必移居,遺後患,事竟寢。時邊將家僮墾塞上田者,每頃輸糧十二石。禮連請於朝,得減四石。是時邊塞無警,禮與巡撫曹翼屯田積粟,繕甲訓兵,邊備甚固。

十一年,沙州衛都督喃哥兄弟爭,部眾離貳。禮欲乘其饑窘,遷之內地。會喃哥亦請居肅州境內。禮因遣都指揮毛哈剌往撫其眾,而親帥兵繼其後。比至,喃哥復持兩端。其部下欲奔瓦剌,禮進兵逼之,遂收其全部千二百餘人以還。事聞,賜賚甚厚。時瓦剌也先方盛,封喃哥弟鎖南奔為祁王。禮以二寇合則勢益難制,遣人招之。鎖南奔欲從未決,禮潛師直抵罕東,縶之以歸。帝大喜,賜禮鐵券,令世襲。

十四年,也先分道入寇,抵肅州。禮遣裨將禦之,再戰再敗,失士馬萬計。徵還,以伯就第。景泰初,提督三千營,以老致仕。久之,復起守備南京,入掌中府。

禮自起卒伍,至大將,恪謹奉法。成化初卒。贈侯,謚僖武。子壽嗣,總兵鎮陜西。坐徵滿四失律,宥死戍邊。子弘,予世指揮使。

趙安,狄道人。從兄琦,土指揮同知,坐罪死,安謫戍甘州。永樂元年進馬,除臨洮百戶,使西域。從北征有功,累進都指揮同知。

宣德二年,松潘番叛。充左參將,從總兵陳懷討平之,進都督僉事。時議討兀良哈,詔安與史昭統所部赴京師。兀良哈旋來朝,命回原衛。使烏思藏,四年還。明年復以左參將從史昭討曲先,斬獲多。九年,中官宋成等使烏思藏,命安帥兵千五百人送之畢力術江。尋與侍郎徐晞出塞討阿台、朵兒只伯,敗之。正統元年進都督同知,充右副總兵官,協任禮鎮甘肅。明年與蔣貴出塞,剿寇無功。三年,復與王驥、任禮、蔣貴分道進師,至刁力溝執右丞、達魯花赤等三十人。以功封會川伯,祿千石。明年移鎮涼州。安家臨洮,姻黨廝養多為盜,副使陳斌以聞。在涼州又多招無賴為僮奴,擾民,復為御史孫毓所劾。詔皆不問。

安勇敢有將略,與貴、禮並稱西邊良將。九年十二月卒。子英為指揮使,立功,進都督同知。

趙輔,字良佐,鳳陽人。襲職為濟寧衛指揮使。景帝嗣位,尚書王直等以將才薦,擢署都指揮僉事,充左參將,守懷來。天順初,徵入右府涖事。

成化元年,以中府都督同知拜征夷將軍,與韓雍討兩廣蠻,克大藤峽,還封武靖伯。已而蠻入潯州,言官交劾。廣西巡按御史端宏謂:「賊流毒方甚,而輔妄言賊盡,冒封爵,不罪輔無以示戒。」輔乃自陳戰閥,委其罪於守將歐信。帝皆弗問。三年總兵征迤東,與都御史李秉從撫順深入,邊戰有功,進侯。

八年,廷議大舉搜河套,拜輔將軍,陜西、延綏、寧夏三鎮兵皆聽節制。輔至榆林,寇已深入大掠。輔不能制,與王越疏請罷兵。言官交論其罪。命給事中郭鏜往勘,還言:「寇於六月入平涼、鞏昌、臨洮,殺掠人畜。迨七月而縱橫慶陽境內。輔與越至榆林不進,宜治其弛兵玩寇罪。」帝不納。輔還,猶督京營。言者攻益力,詔姑置之。輔辭侯,乞世伯。帝許其世伯,侯如故,僅減祿二百石。言官力爭。不聽。輔復上疏暴功,言減祿無以贍老。又言上命內官盧永征南蠻,黃順、汪直征東北,皆莫大功,宜付史館。餘子俊等請置輔於法,卒不問。十二年解營務。家居十年卒。贈容國公,謚恭肅。

輔少俊辨有才,善詞翰,多交文士,又好結權幸。故屢遭論劾,卒無患。

子承慶嗣伯,協守南京。正德初,坐傳寫諫官劉郤疏,為劉瑾所惡,削半祿閒住。四傳至玄孫光遠,萬曆中鎮湖廣。明亡乃絕。

劉聚者,太監永誠從子也。為金吾指揮同知。以「奪門」功,進都指揮僉事,復超擢都督同知。與討曹欽,進右都督。

成化六年,以右副總兵從朱永赴延綏,追賊黃草梁。遇伏,鏖戰傷頦,麾下力捍以免。頃復與都督范瑾等擊寇青草溝,敗之。永等追寇牛家寨,聚亦據南山力攻。寇大敗,出境。論功進左都督,以內援特封寧晉伯。

八年冬代趙輔為將軍,總陜西諸鎮兵。寇入花馬池,率副總兵孫鉞、遊擊將軍王璽等擊卻之。還至高家堡,寇復至,敗之。追奔至漫天嶺,伏起夾擊,又敗之。鉞、璽亦別破賊於井油山。捷聞,予世券。

其冬,孛羅忽、滿都魯、加思蘭連兵深入,至秦州、安定、會寧諸州縣,縱橫數千里。賊退,適王越自紅鹽池還,妄以大捷聞,璽書嘉勞。頃之,紀功兵部員外郎張謹劾聚及總兵官范瑾等六將殺被掠者,冒功。部科及御史交章劾。詔遣給事中韓文往勘,還奏如謹言:所報首功百五十,僅十九級。帝以寇既遁,置不問。聚尋卒。贈侯,謚威勇。

傳子祿及福,福,弘治中掌三千營,加太子太保。卒,子岳嗣。卒,從子文請嗣。吏部言聚無大功,子孫不宜再襲。世宗不允,命文嗣。亦傳至明亡乃絕。

贊曰:宋晟在太祖時,即與開國諸元勛參迹戎行,其後四鎮涼州,威著西鄙。兩子尚主,世列徹侯,功名盛矣。薛祿以下諸人,皆與「靖難」。祿東昌、滹沱之戰,劉榮守永平,譚廣守保定,宣力最著。雖策勛之日,未即剖符,而各以積閥受封。其善撫士卒,慎固封守,恪謹奉職,有足尚者。趙輔、劉聚猷績遠遜前人,而帶礪之盟,與國終始,誠厚幸哉。諸人並以勛爵鎮禦邊陲,故類著於篇。


\end{pinyinscope}