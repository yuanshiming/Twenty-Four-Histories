\article{列傳第一百八十一 忠義五}

\begin{pinyinscope}
○武大烈徐日泰等錢祚徵盛以恆高孝志等顏日愉艾毓初等潘弘劉振世等陳豫抱許宣等劉振之杜邦舉費曾謀等李乘雲餘爵等關永傑侯君擢等張維世姚若時等王世琇顏則孔等許永禧高斗垣等李貞佐周卜歷等魯世任張信等劉禋陳顯元等何燮左相申等趙興基鄭元綬等

武大烈,臨潼人。舉天啟七年鄉試。崇禎中,授永寧知縣。奸人倚萬安郡王恣不法,大烈痛懲之。十三年十二月,李自成自南陽陷宜陽,知縣唐啟泰被害,遂攻永寧。大烈與鄉官四川巡撫張論協力捍御。論歿,子吏部郎中鼎延及從父治中言贊繼之。有獄囚勾賊入,都司馬有義棄城走。大烈、鼎延等固守三日,賊夜半登城,執大烈。自成以同鄉欲活之,大烈不屈,索印又不予,乃燔灼以死。鼎延匿眢井免。言贊及子國學生祚延死之。主簿魏國輔、教諭任維清、守備王正己、百戶孫世英並不屈死。萬安王采金輕亦被害。

賊移攻偃師,一日而陷。知縣徐日泰大罵不屈,為賊臠割死。啟泰,掖縣人。日泰,金谿人。並起家鄉舉。

明年正月,賊陷寶豐,知縣朱由椷死之。陷密縣,知縣朱敏汀及里居太僕卿魏持衡、舉人馬體健死之。由椷,益府鎮國將軍常澈子,敏汀亦宗室,並由貢生。敏汀妾張,一女一孫及臧獲數人俱死,與由椷並贈僉事。

是月,陷洛陽,鄉官來秉衡、劉芳奕、常克念、郭顯星、韓金聲、王明、楊萃、荀良翰等抗節死。秉衡,天啟四年舉於鄉,未仕。城陷,為賊將劉宗敏所執,令易服,欲官之,不可。羈南郊民舍,顧見其友,謂之曰:「賊勒我以官,我義不受辱,恨母老子幼,死不瞑目爾。」賊聞,燒鐵索加其脛,終不從,遂被殺,並其母劉、妾吳及幼子俱殺之。芳奕,慷慨負智略,與秉衡同舉於鄉,為昌樂知縣。解官歸,歲大歉,人相食,傾橐濟之。賊漸逼,集義士為干城社,佐有司保障。及城陷,縊死西城戍樓。克念舉進士,為平陽推官,有聲。顯星舉於鄉,為翰林待詔。金聲、明,皆進士。金聲官邯鄲知縣,明官行人。萃、良翰皆舉人。萃官辰州知府,良翰未仕。

錢祚徵,字錫吉,掖縣人。崇禎中,由鄉舉歷官汝州知州。汝為流賊往來孔道,土寇又竊據山中。祚徵欲先除土寇,募壯士千人訓練,而遣人為好言招撫,夜半取間道直搗其巢,寇大敗。乃令民千家立一大寨,有急鳴鉦相救,寇勢衰息,其魁遂降。十四年正月,李自成驟來犯,祚徵乘城守,身中流矢,守益力。月餘,大風霾,炮炸樓焚,城遂陷,罵賊而死。汝人立廟祀之。

盛以恒,潼關衛人。崇禎十三年舉人。知商城縣。視事月餘,流賊突至,卻之。明年,張獻忠陷襄陽,鄰境大恐。以恒已遷開封同知,將行,士民懇留之,乃登陴,與鄉官楊所修、洪胤衡、馬剛中、段增輝共城守。二月中,賊奄至,適雨雪,守者凍餒不能戰。以恒督家眾射賊十七人墜馬,賊怒,併力攻,矢中以恒右額,猶裹創拒敵。賊登北城,家眾巷戰死且盡,乃被執,罵賊不屈,為賊支解。孫覺及典史呂維顯、教諭曹維正皆死。

所修,故魏忠賢黨也。歷左副都御史,入逆案,贖徒為民,至是罵賊死。胤衡,萬曆中進士。歷官陽和兵備副使,分守北門,力戰死。剛中,字九如。崇禎七年進士。除大同知縣,行取授檢討,乞假歸。賊入,大罵,被磔死。增輝,字含素,為諸生,以學行稱。朝廷下保舉令,被薦,不樂為吏,擬除教授,未謁選歸。遇變,罵賊死。

賊既陷商城,即疾驅犯信陽。城陷,知州高孝誌,訓導李逢旭、程所聞,里居靜海知縣張映宿死之。其陷光山,典史魏光遠亦死之。所司請贈恤,未報。

十五年七月,帝下詔曰:「比州縣有司不設守備,賊至即陷,與衝鋒陷陣,持久力詘者殊科。若概援天啟間例,優予贈廕,何由旌勸勞臣。自今五品以下,止贈監司,四品及方面,始贈京卿。著為令。」乃贈以恒副使,孝志參議,維顯等贈恤有差。天啟中,州縣長吏殉難者,率贈京卿,廕錦衣世職,賜祭葬,有司建祠。崇禎初,改廕國子生,俾之出仕,而京卿之贈如故,至是始改贈外僚云。

顏日愉,字華陽,上虞人。萬曆中,舉於鄉。崇禎初,除知葉縣,有惠政,為上官所惡,劾罷。部民爭詣闕訟冤,乃獲敘用。後為靜寧知州。羅賊亂,馳請固鎮五道兵合剿。而先率敢死士數人招諭之,賊弛備,遂遣精卒搗其營,賊倉皇潰,斬數百級。黎明,五道兵繼至,復大破之。遷開封同知。流賊勢方熾,上官以南陽要衝,舉日愉為知府,大治守具,人心稍固。十四年五月,賊猝至,百餘人冒雨登城。日愉擊殺之幾盡,餘賊引去,城獲全。日愉手中一矢,頭項被二刃,死城上。事聞,贈太僕卿。賊既不得志去,遂縱掠旁近州縣。其冬再圍南陽,攻陷之,參議艾毓初死焉。

毓初,字孩如,米脂人,戶部侍郎希淳曾孫也。崇禎四年進士。授內鄉知縣。生長邊陲,習戰事。六年冬,流寇來犯。埋大炮名「滾地龍」者於城外,城中燃線發之,賊死無算,遂解去。內鄉與領邑淅川多深山邃谷,為盜窟,民居懍懍。毓初至,為設守備,民得少安。明年冬,唐王聿鍵上言:「祖制,親王所封地,有司早晚必謁見。今艾毓初等皆不謁。」帝怒,悉逮下法司,而敕禮部申典制。已而王被逮,毓初獲補官。屢遷至右參議,分守南陽,與日愉卻賊有功。自成用宋獻策計,欲取南陽以圖關中,復率大眾來寇。毓初偕總兵官猛如虎等堅守。賊攻入南門,會總督楊文岳援軍至,賊引退。文岳去,賊復攻之,食盡援絕,毓初題詩城樓,遂自縊。南陽知縣姚運熙、主簿門迎恩、訓導楊氣開亦死之。

明年十月,自成再陷南陽,知府丘懋素罵賊不屈,闔門被害。是月,賊過扶溝,眾議城守,舉人劉恩澤初嘗以策干當事,多見用。縣令騃不解事,恩澤痛哭曰:「吾不幸從木偶人死。」自題樓壁曰:「千古綱常事,男兒肯讓人。」明日,城陷,擲樓下以死。

潘弘,字若稚,淮安山陽人。起家貢生。崇禎十三年為舞陽知縣。時流賊披猖,土寇亦間發,弘數討敗之。明年十一月,李自成、羅汝才既陷南陽,縱兵覆所屬州縣,將攻舞陽,弘諭士民共拒。諸生慮賊屠城,請委曲紓禍,弘叱之去。賊薄城,發炮擊之,多斃。有小校善射,屢卻賊。諸生潛遣人約降,賊復至。弘作告先聖文,自誓必死。諸生潛開門,縛弘以獻。賊索印,弘不予。脅降,怒罵不屈,乃支解之。子澄瀾痛憤大哭,投井死。

時鄧州、鎮平、內鄉、沁陽、新野相繼陷。鄧州知州劉振世,吏目李國璽,千戶餘承廕、李錫,諸生丁一統、張五美、王鐘、王子章、海寬、傅彥皆抗節死。鎮平知縣成縣鐘其碩被執,罵賊死。內鄉知縣南昌龔新、新野知縣四川韓醇,並不屈死。

泌陽凡再陷。是年五月,張獻忠破信陽,獲左良玉旗幟,假之以登城。知縣雲南南寧王士昌懷印端坐,被縛,謾罵死。臨昌姚昌祚代之,甫數月,復陷。昌祚手斬數賊,力屈死。典史雷晉暹率捕卒戰死。又有武職王衍范、錢繼功、海成俱死難。而鄧州於十年春為張獻忠所破,知州孫澤盛、同知薛應齡皆戰死,至是亦再陷云。

陳豫抱,舞陽人。母段早寡,撫豫抱及其弟豫養、豫懷,皆為諸生,力田好學,善承母志。崇禎十四年,流賊陷舞陽,母先赴井,三子從之。豫抱妻黃攜其子默通,豫養妻馬攜子默恒、默言俱從之。三世九人,一時盡節。

時郡邑諸生死者甚眾,錄其著者。內鄉許宣及二弟寀、宮,慷慨好義。賊陷鄧州,宣兄弟結里中壯士,直入其城,擒偽官,堅守許家寨。賊怒,攻破之,寀從母常先投井死,宣、宮皆詈賊被殺,宮妻鐘、寀妻陳並自經,其妹亦罵賊被殺。時稱「許氏七烈」。

賊之攻偃師也,張毓粹率二子佐有司固守,城陷,大罵,俱被殺。妻藺與三女、二孫悉赴井死。賊殺武同芳母,同芳噴血大罵,支解而死。劉芳名、劉芳世、藺之粹、喬于昆、藺完馪、王光顯、喬國屏、王邦紀、藺相裔、張一鷺、張一鵬、牛一元皆抗節死。芳名、完馪妻皆張氏,與邦紀妻高並從死。一鷺、一鵬父亦罵賊死。

唐縣許曰琮,早喪父。母歿,廬墓三年。城破,遁居南山。賊徵之不出,脅以死,鐫其背曰:「誓不從賊」,遂嘔血而死。

劉振之,字而強,慈谿人。性剛方,敦學行,鄉人嚴重之。崇禎初,舉於鄉,以教諭遷鄢陵知縣。十四年十二月,李自成陷許州。知州王應翼被害,都司張守正,鄉官魏完真,諸生李文鵬、王應鵬皆死。自許以南無堅城。有奸人素通賊,倡言城小宜速降,振之怒叱退之。典史杜邦舉曰:「城存與存,亡與亡,人臣大義,公言是。」振之乃與集吏民共守。賊大至,城陷,振之秉笏坐堂上。賊索印,不與,縛置雪中三日夜,罵不絕口,亂刃交下乃死。初,振之書一小簡,藏篋中,每歲元旦取視,輒加紙封其上。及死,家人發篋,乃「不貪財、不好色、不畏死」三語也,其立志如此。贈光祿寺丞。邦舉,富平人。許被屠,鄢陵人恟懼,守者或遁走,邦舉捕得,斬以徇。及城陷,自成欲降之,邦舉罵曰:「朝廷臣子,豈為賊用!」賊抉其舌,含血噴之,遂遇害。

開封屬邑多陷,殉難者,有費曾謀、魏令望、柴薦禋、楊一鵬、劉孔暉、王化行、姚文衡之屬。

曾謀,鉛山人,少師宏裔也。由鄉舉知通許,甫四旬,賊猝至。曾謀召父老曰:「我死,若輩以城降,可免屠戮。」北向再拜,抱印投井死。令望,字於野,武鄉人。舉進士,授商丘知縣,調太康。寇至,固守不下。賊怒,攻破之,屠其城,令望闔門自焚。薦禋,江山舉人,知洧川,城陷,大罵死。一鵬,河津人。舉崇禎九年鄉試,為尉氏知縣,甫數月,政聲四起。城破,罵賊死。孔暉,邵陽人。舉天啟元年鄉試,知新鄭,固守不能支,遂死之。士民祀之子產祠。化行,知商水,城陷,被殺。代者文衡,蒞任數月,賊復至,攜印赴井死。其小吏,則臨潁千總賈廕序、長葛典史杜復春,鄉居則長葛舉人孟良屏、諸生張範孔等,汜水舉人張治載、馬德茂,皆死之。

李乘雲,高陽人,舉於鄉。崇禎初,知浮山縣。流賊數萬來寇,乘雲手發一矢斃其魁,眾遂遁。屢遷山西僉事。十四年秋,以才調河南大梁道,駐禹州。十二月,李自成連陷鄢陵、陳留諸縣,遂寇禹州。乘雲誓死固守,賊多斃於炮。俄以十萬眾攀堞登,執乘雲使跪,乘雲怒叱賊,賊捽而杖之,大罵不絕聲。縛諸樹攢射之,罵不已,斷其舌,亂刃交下而死。贈光祿卿。州先有徽王府,嘉靖時,王載埨有罪,爵絕,而延津等五郡王皆被難。

明年,賊犯開封,監軍主事餘爵、監軍僉事任棟先後戰死。棟,永壽人,由貢生為萊州通判。崇禎四年,李九成等叛,棟佐知府朱萬年共守。萬年與巡撫謝璉為賊所誘執,棟與同知寇化、掖縣知縣洪恩炤助大帥楊御蕃力拒。圍解,論功進秩,屢遷保定監軍僉事。十四年從總督楊文岳南征,鳴皋鎮之捷,與有功。尋與總兵虎大威破賊平峪,再破之鄧州。明年正月,從解開封圍。尋戰郾城,大捷。後從援開封,會左良玉大潰於朱仙鎮,賊來追,棟力戰,歿於陣。餘爵,禹州人。崇禎元年進士。歷知撫寧、章丘。遷職方主事,罷歸。楊嗣昌出督師,請爵以故官參謀軍事。嗣昌入蜀,命與張克儉同守襄陽。城陷,爵脫走,從督師丁啟睿於河南,破賊鄧州。十五年,開封圍急,監左良玉軍往援,戰敗被執,罵賊死。姪敦華亦遇害。棟贈太僕卿,爵太僕少卿。

關永傑,字人孟,鞏昌衛人。世官百戶。永傑好讀書,每遇忠義事,輒書之壁。狀貌奇偉,類世人所繪壯繆侯像。崇禎四年會試入都,與儕輩遊壯繆祠。有道士前曰:「昨夢神告:『吾後人當有登第者,後且繼我忠義,可語之。』」永傑愕然,頗自喜。已果登第,授開封推官,強植不阿,民畏愛之。憂歸,起官紹興。遷兵部主事,督師楊嗣昌薦其才,請用之軍前,乃擢睢陳兵備僉事,駐陳州。陳故賊衝,歲被蹂躪,永傑日夜為儆備。十五年二月,李自成數十萬眾來攻,永傑與知州侯君擢、鄉官崔泌之、舉人王受爵等率士民分堞守。賊遣使說降,斬其頭,懸之城上。賊怒,攻破之,永傑格殺數賊,身中亂刃而死。

君擢,字際明,成安人,起家舉人。城圍時,身先士卒,運木石擊賊,城濠皆滿。後被縛,罵不絕口死。泌之,鹿邑人。進士。知雄縣,調清苑,多所建豎。舊令黃宗昌為御史,劾周延儒,延儒屬保定知府摭宗昌罪。知府以屬泌之,泌之曰:「殺人媚人可乎!」知府愧且怒。會泌之遷戶部主事,知府謂其侵陷錢糧三萬,不聽行。御史行部至,泌之直前與知府角。御史以聞,下獄遣戍,久之釋還。至是,遭變,用鐵杖斃賊數人,自剄死。守備張鷹揚力戰被擒,不屈。受爵亦擊殺數賊,大罵。並死之。贈永傑光祿卿,君擢右參議,泌之復故官。受爵,宛平知縣。

有龔作梅者,年十七,父母俱亡,殯於舍。賊火民居,作梅跪柩前焚死。

張維世,太康人。萬曆四十四年進士。歷平陽知府,捕治絳州奸猾數十人,遷副使。累官右僉都御史,代陳新甲巡撫宣府,視事甫旬日,坐失防,削籍遣戍,已而釋還。崇禎十五年二月,李自成陷睢州,犯太康。維世佐知縣魏令望竭力拒守。城陷,抗節死。

時中州縉紳先後死難者甚眾。十三年,登封土寇李際遇因歲饑倡亂,旬日間眾數萬。前鳳陽通判姚若時居魯莊,被執,誘之降,大罵死。族諸生不顯亦死之。若時子諸生城,思報父仇,數請兵討賊。賊執之於路,亦抗罵死。陜州趙良棟,仕蓬萊教諭,罷歸,寓澠池。寇陷澠池,父子挺身罵賊死,子婦與孫亦赴井以殉。陜州之陷,平定知州梁可棟大罵而死,淮安同知萬大成投井死。商水陷,臨汾知縣張質抗賊死。西平陷,懷仁知縣楊士英死之,子婦王亦死。睢州陷,太平知府杜時髦不屈死。時髦,字觀生,崇禎七年進士。息縣陷,賊召前項城教諭王多福欲官之,堅拒不赴。賊逼之,投繯死。其後以國變死者,有洛陽阮泰,知廣靈,解職歸。聞京師陷,不食死,妻朱氏從之。睢州楊汝經,崇禎十年進士。授戶部主事,擢井陘兵備僉事。十七年,甘肅陷,巡撫林日瑞殉難,超拜汝經右僉都御史,代之。行次林縣,聞京師陷,將赴南京,至東明,率壯士百餘騎還討林縣偽官。遇賊,戰敗被執。偽官釋其縛,屢說之降,不從,斃之獄。

王世琇,字崑良,清苑人。崇禎十年進士。授歸德推官,遷工部主事。十五年二月,李自成陷陳州,乘勝犯歸德。世琇將行,僚屬邀共守,慨然曰:「久官其地,臨難而去之,非誼也。」遂與同知顏則孔、經歷徐一源、商丘知縣梁以樟、教諭夏世英、里居尚書周士樸等誓眾堅守。賊攻圍七日,總督侯恂家商丘,其子方夏率家眾斬關出,傷守者,眾遂亂。賊乘之入,世琇、則孔並遇害。則孔女聞之,即自縊。一源分守北城,殺賊多,城陷,巷戰,罵賊死。以樟中賊刃,久而復蘇,妻張及子女僕從皆死,以樟竟獲免。世英持刀罵賊,死於明倫堂,妻石亦自刎。同死者,尚書士樸,工部郎中沈試,主事硃國慶,中書侯忻,廣西知府沈仔,威縣知縣張儒及舉人徐作霖、吳伯胤、周士美等六人,官生沈佖、侯矣等三人,貢生侯恒、沈誠、周士貴等八人,國學生侯悰、沈倜等四人,諸生吳伯裔、張渭、劉伯愚等一百十餘人。試,商丘人,大學士鯉之孫。作霖、伯胤、伯裔、渭、伯愚,皆郡中名士。則孔,忻州人。一源,海鹽人。世英,祥符人。士樸自有傳。賊既破歸德,尋陷鹿邑,知縣紀懋勛死之。陷虞城,署縣事主簿孔亮死之。

許永禧,曲沃人。由鄉舉為上蔡知縣,多惠政。性耿介,嚬笑無所假。崇禎十五年春,李自成遣數騎抵城下,脅降,永禧即督吏民城守。賊大呼曰:「今日不降,明日屠!」眾懼,永禧嘆曰:「賊勢披猖,彈丸邑豈能守,吾一死盡職而已!」眾皆泣。明日,賊果大至,守者驚潰。永禧具袍笏,北面再拜,據案秉燭端坐。賊入,遂自剄。

時西平、遂平先後皆陷。西平知縣高斗垣,繁峙人。崇禎十二年由貢生授官。為人孤鯁,以清慎得名。城陷,被執不屈死。遂平知縣劉英,貴州貢生,誓眾死守。城陷,罵賊死。

上蔡既陷,有官篆者,以汝寧通判往攝縣事。城中民舍盡毀,篆廣招流亡,眾觀望不敢入。會左良玉駐城南,兵士恣淫掠,眾始入城依篆。村民遭難來醖,篆即入良玉營,責以大義,奪還之。悍卒挾弓刃相向,篆坦腹當之,不敢害,民獲完家室者甚眾。是年冬,汝寧陷,賊黨賀一龍掠地上蔡。訛傳土寇剽掠,篆出禦之,陷陣死。篆,膠州人,起家任子。

李貞佐,字無欲,安邑人。少受業同里曹於汴之門,以學行著,後舉於鄉。崇禎十四年除知郟縣。初,李自成焚掠至郟,土寇導之,害前令邵可灼。貞佐至,則練鄉兵,括土寇財充餉,時出郊勞耕者,月課士。邑有姊妹二人抗賊死,拜其冢,祀以少牢。民王錫胤有孝行,造廬禮之。士民大悅。明年二月,自成復來寇,貞佐集眾死守。汝州吏目顧王家,仁和人,撫賊有功,當遷,汝人乞留以助之。城陷,貞佐走拜其母曰:「兒不忠不孝,陷母至此。」有勸微服遁者,不可,賊執之去,大罵。見賊殺人,輒厲聲曰:「驅百姓固守者,我也,妄殺何為!」賊割其舌,支解而死,母喬亦死。友人王昱,相隨不去,賊義之。昱收葬貞佐於南郊。歲寒食,鄉人傾邑祭奠,廣其冢至二畝餘。贈河南僉事。王家亦大聲叱賊,賊亂刃斫死。子國誘賊發金墟墓間,用巨石擊殺之,賊遂盡殺郟人。

郟有陳心學者,授知縣,不謁選而歸。其友周卜曆舉鄉試,知內黃,以父喪歸里。自成陷郟,執兩人欲官之,心學不從被殺。自成謂卜曆曰:「為我執知縣來,可代汝死。」曰:「戕人以利己,仁者不為。」賊怒,并殺之。

汝所轄四邑並陷。寶豐知縣張人龍,遵化人。城陷,不屈死。妻年少,悍奴四人欲亂之。妻飲以酒俾極歡,潛遣婢告丞尉,捕殺奴,乃扶櫬旋里。魯山知縣楊呈芳,山海衛人,有惠政。練總詹思鸞與進士宗麟祥等謀不軌,呈芳捕斬之。城陷,死。伊陽知縣孔貞璞,曲阜人。賊薄城,以守禦堅,解圍去。他日有事汝陽,道遇賊,被執,亦不屈死。

寶豐之陷也,舉人李得笥短衣雜眾中,為所執。賊謀主牛金星者,故舉人也,勸賊重用舉人,賊所至獲舉人,即授以官。得笥終不自言,賊莫知其為舉人也,役使之,不肯,伺賊寐將刺之,賊覺,被殺。或告賊曰:「此舉人也。」賊懼,棄其屍而去。

時中州舉人盡節者,南陽張鳳翷、王明物,洛陽張民表,永城夏云醇,商城餘容善,光州王者琯,光山胡植,嵩縣王翼明,並罵賊死。

魯世任,字媿尹,垣曲人。性端方,事親孝。從安邑曹于汴學,又交絳州辛全,學日有聞。天啟末舉於鄉。崇禎十年知鄭州,建天中書院,集士子講肄其中,遠近從學者千人。十三年秋,給事中范士髦薦世任及臨城諸生喬己百、內丘太原通判喬中和於朝,稱為德行醇儒,堪繼薛瑄、陳獻章之後。乞召試平臺,置左右備顧問,不報。十五年,流賊來犯,世任勒民兵禦之河干,戰敗自剄死。士民祀之書院中。

其年正月,賊陷襄城,知縣曹思正被殺,訓導張信罵賊不屈死,典史趙鳳豸拒賊死。復陷西華,知縣劉伯驂懷印投井死。明年,汜水陷,知縣周騰蛟亦死焉。

伯驂,河間人。由歲貢生得官。賊信急,遣妻奉母歸。及城被困,有勸出降者,立斬之,登陴死守。賊驅其下為十覆,迭攻之,城遂陷,抗節死。

騰蛟,香河舉人。邑兵荒,撫字有術,以其間釐定徭役,民甚便之。城孤懸河畔,縣人吳邦清等於城南立七砦相掎角,摩天砦最險。土寇李際遇伺騰蛟往河北,急據之,遂攻縣城。騰蛟聞,力請於上官,救兵至,始解去。騰蛟念故城難守,遷縣治於摩天砦以扼賊衝。未幾,賊大至,持十餘日,勢且不支,砦臨河,可渡以免。騰蛟曰:「吾何忍舍眾獨生!」遂自投於河。賊退,人從河濱獲其屍,印懸肘間。

河南凡八郡,三在河北,自六年蹂躪後,賊未再犯。其南五郡十一州七十三縣,靡不殘破,有再破三破者。城郭丘墟,人民百不存一。朝廷亦不復設官。間有設者,不敢至其地,遙寄治他所。其遺黎僅存者,率結山寨自保,多者數千人,少者數百。最大者,洛陽則際遇,汝寧則沈萬登,南陽則劉洪起兄弟,各擁眾數萬,而諸小寨悉歸之。或附賊,或受朝命,陰陽觀望。獨洪起嘗官副總兵,頗恭順。其後諸人自相吞併,中原禍亂於是為極。至十六年四月,帝特下詔蠲五郡賦三年,諭諸人赦其罪,斬偽官者受職,捕賊徒者齎金,復城獻俘者不次擢用,然事已不可為矣。

劉禋,字誠吾,中部人。祖仕,刑部郎中,以諍大禮廷杖。後與定李福達獄,下吏遣戍。穆宗朝起太僕少卿,不就。父爾完,歷知商丘、名山,有學行。禋性孝,母歿于名山,四千里扶櫬,過劍閣雲棧,以肩任之。父少寐,愛聽《史記》,禋每夕朗誦,俟父熟寢乃已,崇禎四年,賊陷中部,禋負父走免。十四年由鄉舉授登封知縣。土寇為亂,禋練壯士,且守且戰,寇不敢近。十五年,李自成陷其城,禋被縛。自成以同郡故欲降之,禋叱曰:「豈有奕世清白吏肯降賊耶!」自成義之,遣賊將反覆說,禋執彌厲,乃見殺。贈僉事。

陳顯元者,由副榜授新安知縣。惡衣糲食,徒步咨疾苦。以城堞傾頹,寇至不能守,率士民入保闕門寨。賊檄降,立碎其檄。及來犯,死守月餘,力竭而陷。見賊怒罵。賊大殺寨中人,顯元叱曰:「守寨者,我也。百姓何辜,寧殺我!」賊怒,遂支解而死。

當是時,河南被賊尤酷,故死事者尤多,其傳隸未詳者,開封之陷,則同知蘇茂均,通判彭士奇,大使徐升、閻生白皆死之。士奇,高要人,由鄉舉。河南之陷,則先後知府亢孟檜、王蔭長,通判白守文,訓導張道脈,靈寶知縣朱挺,或被執不屈,或陷城自盡。孟檜,臨汾人。蔭長,吳橋人。並由鄉舉。南陽之陷,則葉縣知縣張我翼被害,新野先後知縣陳公、丘茂表皆死之。汝寧之陷,武臣則遊擊朱崇祖,千戶劉懋勳、楊紹祖、袁永基同子世蔭,百戶葉榮蔭、張承德、李衍壽、閻忠國,皆力戰死。崇祖妻孫、永基母王亦死之。歲貢生林景暘,國學生趙得庚、楊道臨等,諸生趙重明、費明棟、楊應禎、李士諤等,皆死之。巡按御史蘇京奉詔錄上,凡二百四十九人。後因國變,諸籍散佚。蓋武職及州縣末秩、舉貢諸生,所遺者幾什之五六。

何燮,字中理,晉江人。舉於鄉。崇禎中,知亳州。州自八年後,寇賊交橫,益以饑饉,民死徙過半。燮盡心拊循,營戰守具甚備。未幾,山東、河南土寇迭至,燮戰盧家廟,生擒賊魁二人,刳其腸示眾,撫降者數千人。十五年二月,李自成陷河南,居民望風逃竄,城空不能守。賊至,執燮欲降之,罵不屈,斷足剖胸而死,懸首市上三日,耳鼻猶動。賊遂縱兵四出,霍丘、靈璧、盱眙皆陷。

霍丘,八年春嘗陷,至是再陷。知縣左相申率兵巷戰,力屈死之。巡檢吳姓者,鬥死。靈璧知縣唐良銳,全州舉人。城陷,抗罵死。盱眙,先被陷,賊至,士民悉走,獨主簿胡淵不去。縣故無城,淵持戟至龜山寺力鬥,殪數人。賊駭欲遁,會馬蹶被執,奮罵而死。淵,永年人,起家貢生。

趙興基,雲南太和人。崇禎初,以鄉舉通判廬州。賀一龍、左金王等五部據英、霍二山,暑入秋出以為常。督師楊嗣昌遣監軍僉事楊卓然招之,受侮而返。十四年六月襲陷英山,知縣高在颻抗賊死。十二月陷潛山,知縣李胤嘉、典史沈所安素苛急,奸民導賊執之,並不屈死。所安子亦死焉。

十五年,張獻忠為左良玉所敗,走與諸部合,遂以三月攻舒城。踰月城陷,改為得勝州,據之。遣其黨分掠旁邑,游騎日抵廬州城下。興基與知府鄭履祥、經歷鄭元綬、合肥知縣潘登貴、指揮同知趙之璞、里居參政程楷分門守。監司蔡如蘅貪戾,民不附,賊諜滿城中不能知。五月,提學御史徐之垣以試士至,獻忠遣其徒偽為諸生,襲儒冠以入,夜半舉炮,城中大擾。之垣、如蘅及履祥、登貴並縋城走。興基時守水西門,聞變,挺刃下戍樓與鬥,斬數人,被創死。元綬、楷共守南薰門,元綬力鬥死,楷不屈死。之璞守東門,巷戰死。

賊乘勢連陷含山、巢縣、廬江及無為、六安,又陷太湖。知縣楊春芳、典史陳知訓、教諭沈鴻起、訓導婁懋履並死焉。

廬州城池高深。八年春,賊百方力攻,知府吳太樸堅守不下。後屢犯,終不得志,至是以計得之。履祥、登貴懼罪,委之興基。總督史可法察其冤以聞,乃治守令罪,而贈興基河南僉事,楷光祿卿,元綬亦贈恤。

方賊攻舒城,縣令適以憂去,里居編修胡守恆與遊擊孔廷訓督民兵共守。會遊擊縱所部淫掠,士民遂叛降賊。城將陷,悍卒殺守恒。事聞,贈少詹事,謚文節。


\end{pinyinscope}