\article{列傳第一百八十三 忠義七}

\begin{pinyinscope}
○何復邵宗元等張羅俊弟羅彥等金毓峒韓東明等湯文瓊範箴聽等許琰曹肅等王喬棟張繼孟陳其赤等劉士鬥沈云祚等王勵精劉三策等尹伸莊祖誥等高其勛王士傑等張耀吳子騏曾異撰等米壽圖耿廷籙馬乾席上珍孔師程等徐道興羅國瓛等劉廷標王運開王運閎

何復,字見元,平度人。邵宗元,字景康,碭山人。復,崇禎七年進士。知高縣,有卻賊功。忤上官,被劾謫戍。後廷臣多論薦,起英山知縣,累遷工部主事,進員外郎。十七年二月擢保定知府。宗元,由恩貢生歷保定同知,有治行。

李自成陷山西,遣偽副將軍劉方亮由固關東犯,畿輔震動。及真定游擊謝嘉福殺巡撫徐標反,遣使迎賊,人情益洶洶。宗元時攝府事,亟集通判王宗周,推官許曰可,清苑知縣硃永康,後衛指揮劉忠嗣及鄉官張羅彥、尹洗等,議城守。復聞,兼程馳入城,宗元授以印。復曰:「公部署已定,印仍佩之,我相與僇力可也。」乃謁文廟,與諸生講《見危致命章》,詞氣激烈。講畢,登城分守。

都城陷之次日,賊使投書誘降,宗元手裂之。明日,賊大至,絡繹三百里。有數十騎服婦人衣,言:「所過百餘城,皆開門遠迎,不降即屠。且京師已破,汝為誰守?」城上人聞之,發豎眥裂。賊環攻累日,宗元等守甚堅,賊稍稍引卻。

督師大學士李建泰率殘卒數百,輦餉銀十餘輛,叩城求入。宗元等不許。建泰舉敕印示之,宗元等曰:「荷天子厚恩,御門賜劍,酌酒餞別。今不仗鉞西征,乃叩關避賊耶?」建泰怒,厲聲叱呼,且舉尚方劍脅之。或請啟門,宗元曰:「脫賊詐為之,若何?」眾以御史金毓峒嘗監建泰軍,識建泰,推出視之信,乃納之。建泰入,賊攻益厲。建泰倡言曰:「勢不支矣,姑與議降。」書牒,迫宗元用印。宗元抵印厲聲曰:「我為朝廷守土,義不降,欲降者任為之。」大哭,引刀將自刎,左右急止之,皆雨泣。羅彥前曰:「邪說勿聽,速擊賊。」復自起巘西洋巨砲,火發,被燎幾死。賊攻無遺力,雉堞盡傾。俄賊火箭中城西北樓,復遂焚死。南郭門又焚,守者多散。南城守將王登洲縋城出降,賊蜂擁而上。建泰中軍副將郭中傑等為內應,城遂陷。宗元及中官方正化不屈死。建泰率曰可、永康出降。忠嗣分守東城,城將陷,召女弟適楊千戶者歸,與妻毛、子婦王同處一室,俱以弓弦縊殺之,復登城拒守。城破被執,怒詈,奪賊刀殺二賊。賊麇至,剜目劓鼻支解死。

一時武臣死事者,守備則張大同與子之坦力戰死。指揮則文運昌、劉洪恩、戴世爵、劉元靖、呂九章、呂一照、李一廣,中軍則楊儒秀,鎮撫則管民治,千戶則楊仁政、李尚忠、紀動、趙世貴、劉本源、侯繼先、張守道,百戶則劉朝卿、劉悅、田守正、王好善、強忠武、王爾祉,把總則郝國忠、申錫,皆殉城死。

有呂應蛟者,保定右衛人,歷官密雲副總兵,謝事歸。賊至,總監正化知其能,延與共守,晝夜戮力。城破,短兵鬥殺十餘賊而死。

張羅俊,字元美,清苑人。父純臣,由武進士歷官署參將、神機營左副將。生六子:羅俊、羅彥、羅士、羅善、羅喆、羅輔。

羅俊娶瞽女,終身不置妾。羅彥,字仲美,舉崇禎二年進士。累遷吏部文選郎中。楊嗣昌數借封疆事引用匪人,羅彥多駁正。帝疑吏部行私,廠卒常充庭,曹郎多罹譴者,羅彥獨無所染。秩滿,遷光祿少卿,被誣落職歸。羅俊以十六年秋舉進士,羅輔亦以是年舉武進士。而羅彥少從父塞上,習兵事。初官行人,奉使旋里,鄉郡三被兵,佐當事守御,三著功。給事中時敏奉使過其地,夜半欲入城,羅彥不許。敏劾其擅司鎖鑰,羅彥疏辯,帝不問。

十七年二月,賊逼京師,眾議守御。羅彥兄弟與同知邵宗元等歃血盟,誓死守。總兵官馬岱謁羅彥曰:「賊分兩道,一出固關,一趨河間。吾當出屯蠡縣扼其沖,先殺妻子而後往,其城守悉屬公。」羅彥曰:「諾。」詰旦,岱果殺妻孥十一人,率師去。羅彥等糾鄉兵二千分陴守。羅俊守東城,羅彥西北,羅輔為游兵。公廩不足,出私財佐之。賊遣騎呼降,羅俊顧其下曰:「欲降者,取我首去。」後衛指揮劉忠嗣挺劍曰:「有不從張氏兄弟死守者,齒此劍。」怒目,發上指。聞者咸憤厲,守益堅,賊為引卻。

已,聞京師變,眾皆哭,北向拜,又羅拜相盟誓。而賊攻益急,城中多異議。羅彥謂宗元曰:「小民無知,非鼓以大義,氣不壯。」乃下令人綴崇禎錢一枚於項,以示戴主意。賊謂羅彥主謀,呼其名大詬,且射書說降,羅彥不顧。賊死傷多,攻愈力。李建泰親軍為內應,城遂陷。羅俊猶持刀砍賊,刀脫,兩手抱賊齧其耳,血淋漓口吻間。賊至益眾,大呼「我進士張羅俊也」,遂遇害。羅彥見賊入,急還家,大書官階、姓名於壁,投繯死;子晉與羅俊子伸並赴井死。

羅善,字舜卿,為諸生,佐兩兄守城。城將陷,兩兄戒勿死,羅善曰:「有死節之臣,不可無死節之士。」妻高攜三女投井死,羅善亦投他井死。羅輔多力善射,晝夜乘城,射必殺賊。城破,與羅俊奪圍走,羅俊不可,羅輔連射殺數人,矢盡,持短兵殺數人乃死。

張氏兄弟六人,羅士早卒,其妻高守節十七年,至是自經死。惟羅喆從水門走免,其妻王亦縊死。羅俊伯母李罵賊死。羅彥妻趙、二妾宋、錢及晉妻師,當圍急時,並坐井傍以待。賊入,皆先羅彥投井死,獨趙不沈,家人出之。羅輔妻白在母家,聞變欲死,侍者止之,紿以汲井,推幼女先入,已從之。羅俊再從子震妻徐,巽妻劉,亦投井死,一門死者凡二十三人。

金毓峒,字稚鶴,保定衛人。父銓,戶部員外郎。毓峒舉崇禎七年進士。授中書舍人。十四年面陳漕務,稱旨,授御史。疏論兵部尚書陳新甲庸才誤國,戶部尚書李待問積病妨賢。又請渙發德音,自十五年始,蠲除繁苛,與海內更新。因言復社一案,其人盡縫掖,不可以一夫私怨開禍端。帝多採納。明年出按陜西。孫傳庭治兵關中,吏民苦征繕,日夜望出關,天子亦屢詔督趣。毓峒獨謂將驕卒悍,未可輕戰,抗疏爭。帝不納,師果敗。

十六年冬,期滿得代,甫出境,而賊入關。復還至朝邑,核上將吏功罪而後行。明年三月召對,命監李建泰軍。馳赴山西,抵保定,賊騎已逼,遂偕邵宗元等共守。毓峒分守西城,散家貲千餘金犒士,其妻王亦出簪珥佐之。京師變聞,賊射書說降,眾頗懈。毓峒厲聲曰:「正當為君父復仇,敢異議者斬!」懸銀牌,令擊賊者自取。眾爭奮,斃賊多。城陷,一賊挽毓峒往謁其帥,且罵且行,遇井。推賊仆地,自墮井死。妻聞,即自經。其從子振孫有勇力,以武舉佐守城。賊至,眾皆散,獨立城上,大呼曰:「我金振孫,前日殺數賊魁者,我也。」群賊支解之。振孫兄肖孫、子婦陳與侍兒桂春,亦投井死。肖孫匿毓峒二子,為賊搒掠無完膚,終不言,二孤獲免。

同時守城殉難者,邠州知州韓東明、武進士陳國政赴井死。平涼通判張維綱,舉人張爾翬、孫從範,不屈死。舉人高經負母避難,遇賊求釋母,母獲釋而經被執,乘間赴水死。貢生郭鳴世寢疾,聞城陷,整衣端坐。賊至,持棒奮擊而死。諸生王之珽,先城陷一日,置酒會家人,飲達旦。城破,偕妻齊及三子、二女入井死。諸生韓楓、何一中、杜日芳、王法等二十九人,布衣劉宗向、田仰名、劉自重等二十人,或自經,或溺,或受刃,皆不屈死。婦人盡節者一百十五人。他若都給事中尹洗、舉人劉會昌、貢生王聯芳,以城陷次日為賊收獲,亦不屈死。賊揭其首於竿,書曰:「據城抗節,惡官逆子。」見者飲泣。

湯文瓊,字兆鰲,石埭人。授徒京師,見國事日非,數獻策闕下,不報。京師陷,慨然語其友曰:「吾雖布衣,獨非大明臣子耶?安忍見賊弒君篡國。」乃書其衣衿曰:「位非文丞相之位,心存文丞相之心。」投繯而卒。福王時,給事中熊汝霖上疏曰:「北都之變,臣傳詢南來者,確知魏藻德為報名入朝之首,梁兆陽、楊觀光、何瑞徵為從逆獻謀之首,其他皆稽首賊庭,乞憐恐後。而文瓊以閭閻匹夫,乃能抗志捐生,爭光日月。賊聞其衣帶中語,以責陳演,即斬演於市。文瓊布衣死節,賊猶重之,不亟表章,何以慰忠魂,勵臣節。」乃贈中書舍人,祀旌忠祠。

時都城以布衣盡節者,又有範箴聽、楊鉉、李夢禧、張世禧輩。福王建國,喪亂益甚,且見聞不詳,未盡表章。

箴聽,端方有義行。高攀龍講學都下,受業其門。魏國公徐允禎延為館賓,數進規諫。允禎或倨見他客,箴聽至,輒斂容。賊入,置一棺,偃臥其上,絕食七日死。鉉,善寫真。京師陷,攜二子赴井死。夢禧,負志節,與妻杜、二子、二女、一婢俱縊死。世禧,儒士也,亦與二子懋賞、懋官俱縊死。

又有周姓者,悲憤槌胸,嘔血數升而死。而柏鄉人郝奇遇,居京師,聞變,謂妻曰:「我欲死難,汝能之乎?」妻曰:「能。」遂先死。奇遇瘞畢,服藥死。

許琰,字玉仲,吳縣人。幼有至性,嘗刲臂療父疾。為諸生,磊落不羈。聞京師陷,帝殉社稷,大慟,誓欲舉義兵討賊。走告里薦紳,皆不應。端午日過友人,出酒飲之,琰擲杯大詬曰:「今何日,我輩讀聖賢書,尚縱酒如平日耶!」拂衣徑去。已,聚哭明倫堂,琰衰杖擗踴,號泣盡哀。御史謁文廟,猶吉服。琰率諸生責以大義,御中惶悚謝罪去。及南都頒監國詔,而哀詔猶未頒。琰益憤慟,趨古廟自經,為人所解,乃步至胥門,投於河。潞王舟至,拯之出,詢其故,嗟嘆良久。識琰者掖以歸,家人旦夕守,不得死,遂絕粒。尋聞哀詔至,即庭中稽首號慟,并不復言,以六月三日卒。鄉人私謚曰潛忠先生。南中贈《五經》博士,祀旌忠祠。

是時諸生殉義者,京師則曹肅、藺衛卿、周讜、李汝翼,大同則李若葵,金壇則王明灝,丹陽則王介休,雞澤則殷淵,肥鄉則宋湯齊、郭珩、王拱辰。

肅,曾祖子登,仕為甘肅巡撫。賊入,肅與祖母姜、母張、嫂李及弟持敏、妹持順、弟婦鄧並自縊。衛卿止一幼女,托其友,亦自縊,讜被執,罵賊不屈死。汝翼,布政使本緯子。亦罵賊,被磔死。若蔡與親屬九人皆自縊,題曰一門完節。明灝聞變,日夕慟哭,家人解慰之。托故走二十里外,投水死。介休,不食七日死。

淵,字仲弘。父大白,官監軍副使,為楊嗣昌所殺。淵負奇氣。從父兵間,善技擊,嘗欲報父仇。及賊破雞澤,謀起兵恢復。俄聞京師陷,即同諸生黃祐等悲號發喪,約山中壯士,誅賊所置官。偽令秦植踉蹌走,乃入城,行哭臨禮,義聲大震。為奸人所乘,被殺,遠近悼之。湯齊、珩、拱辰亦起兵討賊,為賊將張汝行所害。

王喬棟,雄縣人。舉進士,授朝邑知縣。縣人王之寀為魏忠賢黨所惡,坐以贓,下喬棟嚴征。喬棟不忍,封印於庫而去。巡撫怒,將劾之。士民擁署號呼,乃止。崇禎初,起順天教授,累遷湖廣參政。楚中大亂,諸道監司多不至,喬棟兼綰數篆。乙酉夏,李自成據武昌,喬棟時駐興國州。城為賊陷,自經城樓上。

張繼孟,字伯功,扶風人。萬歷末年進士。知濰縣。天啟三年擢南京御史,未出都,奏籌邊六事,末言己被抑南臺,由錢神世界,公道無權,宜嚴禁餽遺。帝令實指,繼孟以風聞對,詔詰責之。左都御史趙南星言:「今天下進士重而舉貢輕,京官重而外官輕,在北之科道重而南都輕。乞因繼孟言,思偏重之弊。敕下吏部極力挽回,於用人不為無補。」於是忌者咸指目繼孟為東林。尋以不建魏忠賢祠,斥為邪黨,削奪歸。

崇禎二年起故官,上言:

近見塚臣王永光「人言踵至」一疏,語語謬戾。其曰「惠世揚等借題當議」。夫云借者,無其事而借名也。世揚與楊漣、左光鬥同事同心,但未同死耳。今楊、左業有定議,世揚方昭揭於天下後世,奈何以借名之,謬一。

又曰「高捷、史褷發奸已驗,特用宜先。」夫捷、褷之糾劉鴻訓也,為楊維垣等報仇耳。鴻訓輔政,止此一事快人意。其後獲罪以納賄,非以捷、褷劾也。今指護奸者為發奸,謬二。

又曰「諸臣所擁戴者,錢謙益、李騰芳、孫慎行。」夫謙益本末,陛下近亦洞然。至騰芳、慎行,天下共推服。會推之時,永光身主其議。乃指公論為擁戴,謬三。

又曰「欲諸臣疏一面網,息天下朋黨之局。」信斯言也,則部議漏張文熙等數十人,是為疏網,而陛下嚴核議罪,反開朋黨之局乎?謬四。

且永光先為御史李應昇所糾,今又為御史馬孟正、徐尚勛等所論。而推轂永光者先為崔呈秀、徐大化,今則霍維華、楊維垣、張文熙,其賢不肖可知矣。

後又劾南京兵部尚書胡應臺貪污。帝並不納。永光深疾之,出為廣西知府。土酋普名聲久亂未靖,繼孟設計鴆之,一方遂安。稍遷浙江鹽運使,忤視鹽內官崔璘,左遷保寧知府。尋進副使,分巡川西。

十七年八月,張獻忠寇成都,與陳其赤、張孔教、鄭安民、方堯相等佐巡撫龍文光協守,城陷被執。獻忠僭帝號,欲用諸人備百官。繼孟等不為屈,乃被殺,妻賈從之。

其赤,字石文,崇仁人。崇禎元年進士。歷兵備副使,轄成都。城陷,投百花潭死,家人同死者四十餘人。孔教,字魯生,會稽人。舉於鄉。歷四川僉事,不屈死。子以衡,奉母孔南竄,匿不使知。踰年母詣以衡書室,見副使周夢尹請孔教恤典疏,隕絕,罵以衡曰:「父死二載,我尚偷生,使我無顏見汝父地下!」遂取刀斷喉死。安民,浙江貢生,歷蜀府左長史。賊圍成都,分守南城,城陷,不屈死。堯相,字紹虞,黃岡人。官成都同知,監紀軍事,兵食不足,泣請於蜀王,王不允,自投於池,以救免。次日城陷,被殺於萬里橋下。總兵劉佳胤亦盡節。

劉士鬥,字瞻甫,南海人。崇禎四年進士。知太倉州,有政聲。忤上官,中許典,謫江西按察司知事,擢成都推官。十六年,御史劉之勃薦為建昌兵備僉事。明年八月,賊將入境,之勃促之行。士斗曰:「安危生死與公共,復何往。」城陷被執,見之勃與張獻忠語,大呼曰:「此賊也,公不可少屈!」獻忠怒,命捽以上,士斗又返顧之勃,語如前,遂闔門被殺。

同時沈云祚,字子凌,太倉人。崇禎十三年進士。知華陽縣。有奸民為搖、黃賊耳目,設策捕戮之。賊破夔門,成都大震,云祚走謁蜀王,陳守御策,不聽。聞內江王至淥賢,往說之曰:「成都危在旦夕,而王府貨財山積,不及今募士殺賊,疆場淪喪,誰為王守?」至淥言於王,不聽。賊迫成都,王始出財佐軍,已無及。城陷,獻忠欲用之,幽之大慈寺而遣其黨饋食,以刃脅降,不屈,遂遇害。

王勵精,蒲城人。崇禎中,由選貢生授廣西府通判,仁恕善折獄。歲兇,毀銀帶易粟,減價糶。富人聞之,爭出粟,價遂平。遷崇慶知州,多善政。十七年,張獻忠陷成都,州人驚竄。勵精朝服北面拜,又西向拜父母,從容操筆書文信國成仁取義四語於壁,登樓縛利刃柱間,而置火藥樓下,危坐以俟。俄聞賊騎渡江,即命舉火,火發,觸刃貫胸而死。賊歎其忠,斂葬之。其墨跡久逾新,滌之不滅。後二十餘年,州人建祠奉祀,祀甫畢,壁即頹,遠近嘆異。

先是,十三年賊犯仁壽,知縣鄱陽劉三策拒守,城陷不屈死,贈尚寶司丞。及是再陷,知縣顧繩貽遇害。賊陷郫縣,主簿山陰趙嘉煒守都江堰,賊誘降,不從,投江死。陷綿竹,典史卜大經與其僕俱縊死,鄉官戶部郎中刁化神亦死之。他若榮縣知縣漢陽秦民湯、蒲田知縣江夏朱蘊羅、興文知縣漢川艾吾鼎、南部知縣鄭夢眉、中江教諭攝劍州事單之賓,皆殉難。夢眉夫婦並縊。蘊羅、吾鼎闔家被難。宗室朱奉金尹,由進士歷御史,劾督師丁啟睿諸疏,為時所稱。時里居,并及於難。

尹伸,字子求,宜賓人。萬歷二十六年進士。授承天推官。屢遷南京兵部郎中、西安知府、陜西提學副使、蘇松兵備參政。公廉強直,不事媕阿,三任皆投劾去。天啟時,起故官,分守貴州威清道。貴陽圍解,巡撫王三善將深入,伸頗贊之,監軍西征。三善敗歿,伸突圍歸,坐奪官,戴罪辦賊。四年,賊圍普安,伸赴援,賊解去,遂移駐其地。賊復來攻,率參將範邦雄破走之,逐北至三岔河。總督蔡復一上其功,免戴罪,貶一秩視事。崇禎五年歷河南右布政使,以失御流賊,罷歸。伸所至與長吏迕,然待人有始終,篤分義,工詩善書,日課楷書五百字,寒暑不輟。張獻忠陷敘州,匿山中,搜得之,罵不肯行。賊重其名,不殺。至並研,罵益厲,遂攢殺之。福王時,起太常卿,伸已先死。

蜀中士大夫在籍死難者,成都則雲南按察使莊祖誥,廣元則戶科給事中吳宇英,資縣則工部主事蔡如蕙,郫縣則舉人江騰龍。而安岳進士王起峨、渠縣禮部員外郎李含乙,皆舉義兵討賊,不克死。

高其勛,字懋功。初襲千戶,後舉武鄉試,為黔國公標下中軍。吾必奎反,擢參將,守御武定。及沙定洲再反,分兵來攻。固守月餘,城陷,衣冠望北拜,服毒死。

時有陳正者,世為大理衛指揮,未嗣職。沙賊陷城,督眾巷戰,手馘數賊而死。

王承憲者,襲祖職為楚雄衛指揮,擢游擊,為副使楊畏知前鋒。定洲來攻,凡守禦備悉,畏知深倚之。賊去復至,承憲偕土官那籥等出城衝擊,賊皆披靡,俄為流矢所中死。弟承瑱力戰死,一軍盡歿。

賊進圍大理時,太和縣丞王士傑佐上官畢力捍禦,城陷,死城上。同死者,大理府教授段見錦、經歷楊明盛及子一甲、司獄魏崇治。而故永昌府同知蕭時顯,解任,以道阻,寓居大理,亦自經。

士人同死者,舉人則高拱極投池死,楊士俊同母妻妹自焚死。諸生則尹夢旗、夢符、馮大成倡義助守,罵賊死,楊憲偕妻女、子婦、姪女、孫女、弟婦一門自焚死。楊孫心既死復蘇,妻竟死。人稱太和節義為獨盛云。

單國祚者,會稽人,為通海典史。城陷,握印坐堂上,罵賊被殺,印猶在握。縣人葬之諸葛山下。

張耀,字融我,三原人。萬曆中,舉於鄉。知聞喜縣,慈惠撫民,民為立祠。崇禎中,歷官貴州布政使。張獻忠死,其部將孫可望、李定國等率眾奔貴州。耀急言於巡撫,請發兵民守御,巡撫以眾寡不敵難之。俄賊眾奄至,耀率家眾乘城拒擊。城陷被執,賊帥與耀皆秦人,說之曰:「公若降,當用為相。」耀怒詈不屈,賊執其妾媵述之曰:「降則免一家死。」耀詈益甚,賊殺之,並其家屬十三人。時鄉官吳子騏、劉琯、楊元瀛等率鄉兵敗賊,賊來益眾,戰敗被執,俱不屈死。

子騏,字九逵,貴陽人。萬曆中,舉於鄉,知興寧縣。天啟時,安邦彥圍貴陽,子騏以母在城內,倉皇棄官歸。崇禎十年,蠻賊阿烏謎叛,陷大方城,逐守將。總督朱燮元屬子騏詣六廣,走書召諸目,曉以利害,果乞降。燮元上其功,優旨獎賞。琯戶部主事,元瀛府同知,並起家鄉舉。同時譚先哲,平壩衛人,子騏同年生也。官戶部郎中。賊陷其城,與里人石聲和皆闔家殉難。聲和,天啟中,舉於鄉,官寧前兵備參議。

有顧人龍者,定番州人,嘗出仕,解職家居。流賊來犯,率士民拒守,殺賊甚眾。城破,大罵而死。可望寇安平,僉事臨川曾益集眾拒守,城陷死之。

曾異撰,榮昌人。舉於鄉,知永寧州。可望既陷貴州,將長驅入雲南。異撰與其客江津進士程玉成、貢生龔茂勛謀曰:「州據盤江天險,控扼滇、黔,棄之不守,事不可為矣。」遂集眾登陴守,城陷,自焚死。

米壽圖,宛平人。崇禎中,由舉人知新鄉縣。土寇來犯,督吏民破走之,斬首千二百餘級。以治行征授南京御史。十五年四月極論監軍張若騏罪,言:「若騏本不諳軍旅,諂附楊嗣昌,遂由刑曹調職方。督臣洪承疇孤軍遠出,若騏任意指揮,視封疆如兒戲。虛報大捷,躐光祿卿,冒功罔上,恃鄉人謝升為內援。升奸險小人,非與若騏駢斬,何以慰九廟之靈。」會廷臣多糾若騏,遂論死,升亦除名。初,嗣昌倡練兵之議,擾民特甚。壽圖疏陳十害,又言:「往時督撫多用京卿,今封疆不靖,遇卿貳則爭先,推督撫則引避,宜嚴加甄別,內外兼補。」因劾偏沅巡撫陳睿謨、廣西巡撫林贄貪黷。帝納其言。十七年五月,福王立,馬士英薦用阮大鋮,壽圖論劾。七月,出按四川。時川地已為張獻忠所據,命吏部簡堪任監司守令者從壽圖西行。至則與督師王應熊、總督樊一蘅等聯絡諸將,號召遠近,漸復川南郡縣。唐王立,擢右僉都御史,巡撫貴州。大清順治四年,獻忠遺黨孫可望等陷貴陽,壽圖出奔沅州。十一月,沅州亦陷,壽圖死之。

耿廷籙,臨安河西人。天啟四年舉於鄉。崇禎中,知耀州,有能聲。十五年夏,疏陳時政,言:「將多不若將良,兵多不若兵練,餉多不若餉核。」又言:「諸臣恩怨當忘,廉恥當勵。小怨必報,何不大用於斷頭飲血之元兇;私恩必酬,何不廣用於鵠面鳩形之赤子。」優旨褒納。擢山西僉事,改監宣府軍。十七年,京師陷,走南都。十一月以張獻忠亂四川,命加太僕少卿赴雲南監沙定洲軍,由建昌入川討賊。明年三月,四川巡撫馬乾罷,即拜廷籙右僉都御史代之。未赴,而定洲作亂,蜀地亦盡失,遂止不行。後李定國掠臨安,過河西,廷籙聞之赴水死。妻楊被執,亦不屈死。

馬乾者,昆明人。舉崇禎六年鄉試,為四川廣安知州。夔州告警,巡撫邵捷春檄乾攝府事。張獻忠攻圍二十餘日,固守不下。督師楊嗣昌兵至,圍始解。擢川東兵備僉事。成都陷,巡撫龍文光死,蜀人共推乾攝巡撫事。賊陷重慶,留其將劉廷舉戍守。乾擊走之,復其城。督師王應熊劾乾淫掠,奪職提訊。會蜀地大亂,詔命不至,乾行事如故。乃傳檄遠近,協力討賊。廷舉既敗去,賊遣劉文秀等以數萬眾來攻,乾固守。曾英等援兵至,賊敗還。及獻忠死,其黨孫可望等南奔,大清兵追至重慶,乾戰敗而死。

席上珍,姚安人。崇禎中,舉於鄉。磊落尚節義,聞孫可望、李定國等入雲南,與姚州知州何思、大姚舉人金世鼎據姚安城拒守。可望遣張虎攻陷之,世鼎自殺,上珍、思被執至昆明。可望呵之,上珍厲聲曰:「我大明忠臣,肯為若屈耶!」可望怒,命引出斬之,大罵不絕,遂磔於市。思亦不屈死。

有孔師程者,昆明人,以從事得官。至是糾合晉寧、呈貢諸州縣,起兵拒賊。定國率眾奄至,師程遁,晉寧知州石阡冷陽春、呈貢知縣嘉興夏祖訓並死之。晉寧舉人段伯美,諸生餘繼善、耿希哲助陽春城守,亦殉難。賊陷富民,貢生李開芳妻及二子俱赴井死。開芳走至松花壩自經,其友王朝賀掩埋訖,亦自經。在籍知縣陳昌裔不受偽職,為賊杖死。楚雄舉人杜天禎,初佐楊畏知拒沙賊,頻有功。後畏知督兵擊可望敗績,天禎聞之即自盡。臨安之陷,進士廖履亨赴水死。

徐道興,睢州人。崇禎末,官云南都司經歷,署師宗州事,廉潔愛民。孫可望等入雲南,破曲靖。巡按羅國瓛方按部其地,與知府焦潤生被執。可望欲降之,國瓛不屈,攜至昆明,自焚死。潤生亦不屈死。道興見賊逼,集士民諭之曰:「力薄兵寡,不能抗賊,吾死分也。若等可速去。」民請偕行,道興厲聲曰:「封疆之臣死封疆,吾將安之!」眾雨泣辭去。舍中止一僕,出俸金二錠授之曰:「一以賜汝,一買棺斂我。」僕大哭,請從死。道興曰:「爾死,誰收吾骨?」僕叩頭號泣乃去。及賊入署,令出迎其將。道興大罵,擲酒杯擊之,罵不絕口,遂被殺。

國瓛,嘉定州人,崇禎十六年進士。潤生,修撰竑子。同時張朝綱,廣通人,由貢生授渾源州同知,解職歸。可望等兵至,與共妻馮並縊死,子諸生耀葬親訖,亦縊死。

劉廷標,字霞起,上杭人。王運開,字子朗,夾江人。廷標由貢生歷永昌府通判。運開舉於鄉,授永昌推官。沙定洲之亂,黔國公沐天波走永昌。及孫可望等入雲南,馳檄諭天波降。時運開攝監司事,廷標攝府事,方發兵守瀾滄,而天波將遣子納款,諭兩人以印往。兩人堅不予,各遣家人走騰越。永昌士民聞賊所至屠戮,號泣請運開納款紓禍,運開不可,慰遣之。又詣廷標,廷標亦不可,眾大哭。廷標取毒酒將飲,乃散去。兩人相謂曰:「眾情如此,吾輩惟一死自靖耳。」是夕,運開先自經。廷標聞之曰:「我老當先死,王乃先我。」遂沐浴,賦詩三章,亦自經。兩家子弟自騰越來奔喪,厝畢復返。可望等重兩人死節,求其後,或以運開弟運閎對,即聘之。行至潞江,謂其僕曰:「吾兄弟可異趣耶!吾死,若收吾骨與兄合葬。」遂躍入江死。


\end{pinyinscope}