\article{列傳第一百二十七}

\begin{pinyinscope}
張臣子承廕孫應昌全昌德昌董一元王保杜桐弟松子文煥孫弘域蕭如薰達雲尤繼先官秉忠柴國柱李懷信

張臣,榆林衛人。起行伍,為隊長。蹻捷精悍,搏戰好陷堅。從千總劉朋守黃甫川。朋遇寇喪馬被圍,臣單騎馳救,射中其魁,奪馬載朋歸,由此知名。旋代朋職,屢戰跨馬梁、李家溝、高家堡、田家梁、西紅山,並有功,遷宣府膳房堡守備。寇嘗大入,環攻堡,欲生得臣。臣召麾下酌水為酒,歡呼歌飲,寇莫測所為,不敢登。臣夜決圍出,取他道以歸。上官壯之,擢延綏入衛遊擊將軍。

隆慶元年九月,土蠻大入昌黎、撫寧、樂亭、盧龍,遊騎至灤河。諸將莫敢戰,臣獨勒兵赴之。遼帥王治道曰:「敵眾我寡,往必無利。」臣不顧,率所部千人擐甲直馳,呼聲震山谷,寇以數騎嘗,奮前斬之。追至棒槌崖,斬首百十餘級,墜崖死傷者無算。事寧,薊鎮諸將悉獲罪,臣以功增秩二級。無何,寇潛入場子嶺,參將吳昂被殺,命臣代之。尋進副總兵,領總督標下事,改守薊鎮西協。

萬曆初,錄秋防功,進署都督僉事。炒蠻潛入古北口,參將範宗儒追至十八盤山,戰歿,餘眾被圍。臣急偕遊擊高廷禮等馳救,寇始去,坐鐫一秩。五年春,以總兵官鎮守寧夏。順義王俺答報怨瓦剌,欲取道賀蘭,臣不可,俺答恚,語不遜。臣夜決漢、唐二渠水,道不通,復陳兵赤水口,俺答乃從山後去。三歲互市,毋敢譁者。閱邊給事中以苛禮責望,劾罷之。

十一年,小阿卜戶犯黑峪關,守將陳文治以下俱逮繫。詔起臣副總兵,駐守馬蘭峪。會朵顏長昂屢擾邊,薊鎮總兵官楊四畏不能禦,乃調四畏保定,而徙臣代之。長昂雅憚臣,使其從母土阿、妻東桂款關乞降,乃撫賞如初。猛可真者,俺答弟老把都棄妾也,坐與小阿卜戶犯黑峪關,罷歲賞。既納款,復猖獗,以謾詞報邊臣,而令大嬖只代為謝罪。大嬖只者,順義王乞慶哈棄妾也。臣等測其詐,令將士出塞捕二十三人,繫之獄,令還我被掠人。猛可真以所愛者五人在俘中,許獻還所掠,親叩關索故賞。臣等并召大嬖只入演武場,譙責甚厲。兩婦叩頭請死,乃貸之。先後獻還八十餘人,中有被拘數十年者。臣以功紀錄優敘。尋進署都督同知,召僉書左府事。出為陜西總兵官,鎮守固原。

十八年春,移鎮甘肅。火落赤犯洮、河,卜失兔將往助之,其母泣沮,不從,遂攜妻女行,由永昌宋家莊穴牆入。臣逆戰水泉三道溝,手格殺數人,奪其坐纛。卜失兔及其黨炒胡兒並中流矢走,臣亦被創。將士斬級以百數,生獲其愛女及牛馬羊萬八百有奇。卜失兔仰天大慟曰:「傷哉我女,悔不用母言,以至此也。」自是不敢歸巢,與宰僧匿西海。已,屬宰僧謝罪,其母及順義王亦代為言,乃還其女,而使歸套。臣以功進秩為真。

時諸部長桀驁甚,經略鄭洛專主款。臣以為不足恃,上書陳八難、五要。大略云:邊薄兵寡,餉絀寇驕,諸部順逆難明,宜復額兵,嚴勾卒,足糧餉,分敵勢,明賞罰。且以創重乞歸,帝不許。後二年,謝病去。臣更歷四鎮,名著塞垣,為一時良將。

子承廕,由父廕積功至延綏副總兵。勇而有謀,尤精騎射,數鏖戰未嘗挫衄。萬曆三十七年,代王威為延綏總兵官。沙計及猛克什力數犯邊。是年冬,復犯波羅、神木。承廕邀卻之,追斬八十餘人。沙計欲修貢,守臣惡其反覆,拒之,益徙近邊,以數千騎犯雙山堡。承廕擊走之,俘斬百二十有奇。四十年,沙計復入塞。承廕遮擊之響水,斬首百七十餘級。積前功,進署都督同知,世廕本衛副千戶。是歲,遼東總兵官麻貴罷,敕承廕馳代之。蟒金諸部近寧前,守將祖天壽間出獵,被圍曹莊,將士死者二百三十人,被掠者六百餘人,天壽以數騎免。事聞,論死。承廕初抵任,獲免。敖克等犯中後所,拒斬其二長,餘走出塞。時虎墩兔、炒花、煖兔、宰賽逼處遼境,無歲不犯邊。承廕未至時,虎墩兔以三萬騎犯穆家堡,參將郎名忠等遏斬其四十餘騎。及再舉,守將梁汝貴襲破其營。已而乃蠻諸部連犯中後所、連山驛,副總兵李繼功等力戰,殪其魁,徐引去。自是虎墩兔所屬貴英哈等三十餘部悉奉約束,遼西得少安。承廕旋以病去。甫歲餘,起守薊鎮。未至,復改鎮遼東。

四十六年四月,我太祖高皇帝起兵,拔撫順,巡撫李維翰趣承廕赴援。承廕急率副將頗廷相、參將蒲世芳、遊擊梁汝貴等諸營並發,次撫順。承廕據山險,分軍三,立營浚濠,布列火器。甫交鋒,大清兵蹴之,大潰。承廕、世芳皆戰死。廷相、汝貴已潰圍出,見失主將,亦陷陣死。將士死者萬人,生還者十無一二,舉朝震駭。既而撫安、三岔兒、白家沖三堡連失,詔逮維翰,贈承廕少保,左都督,立祠曰精忠,世蔭指揮僉事。廷相以下,贈蔭有差。

承廕子應昌、全昌、德昌。應昌嗣祖臣職,當為指揮僉事。以父陣亡,增三秩為都司僉書,經略楊鎬用為左翼遊擊。四路出師,使從李如柏。天啟元年,遷大同井坪參將,調延綏。二年秋,河套入犯,不能禦,免歸。督師孫承宗召置麾下,命駐錦州。承宗去,高第盡撤松、錦守具,應昌亦歸。

崇禎二年,總督楊鶴檄應昌署定邊鎮將事。河套入寇,擊斬百二十餘級,擢昌平副總兵,鶴遂薦應昌以副將鎮定邊。四年春,神一元陷保安,應昌偕左光先破斬一元。其弟一魁代領其眾,圍慶陽。應昌及杜文煥趨戰,圍始解。不沾泥圍米脂,應昌偕王承恩擊走之。楊鶴撫一魁,處之寧塞,而殺其黨茹成名。賊黨張孟金、黃友才懼,挾一魁以叛。延綏巡撫張福臻令應昌及馬科擊之,斬首千七百餘級。友才走,一魁守不下。其冬,洪承疇代鶴,命參政戴君恩、總兵曹文詔同應昌討之。數敗賊,賊棄城走。文詔偕應昌擊敗之駙馬溝。明年春,應昌擒友才。混天猴陷宜君、鄜州,襲靖邊,應昌追敗之,射傷賊將白廣恩。八月,山西總兵官馬士麟病免,擢應昌都督僉事代之。言者謂寧武卒善逃,宜令應昌率所部三千人以從,報可。王之臣陷臨縣。其地倚黃雲山,榆林河水出焉,入於黃河。城三面峭壁,西阻水。巡撫許鼎臣、總督張宗衡督兵攻。賊與土寇田福、田科等相倚,久不拔。會王自用陷遼州,逼會城。鼎臣還,專以恢復責應昌。六年春,賊約福劫官軍,撫標中軍陳國威因偽稱之臣往逆,斬福頭懸城下,急擊,賊始降。

應昌在關中,威名甚著。及是選懦逗撓,務與賊相避。總督宗衡五檄之不赴,奏諸朝,限應昌與文詔三月平賊。應昌避賊不擊,殺良民冒功,為巡按御史李嵩、兵科祝世美所劾。帝乃遣近侍為應昌內中軍。七月,部卒潰鳴謙驛。監視中官劉允中劾其避賊,帝猶貸之,令會剿畿南賊。久之,擊賊平山,偽報首功,連為允中及巡按御史馮明玠、真定巡撫周堪賡所劾,帝令圖功自贖。七年春,追賊靈寶,稍有功。已,擊賊均州五嶺山,敗績。身中一矢,退還河南。其弟全昌為宣府總兵官。宣府有警,令應昌援,又無功,命解職候勘。

八年,洪承疇出師河南,令率私家士馬以從。三月,抵信陽。會賊大入秦,承疇命應昌及鄧、尤翟文防漢江南北。死,承疇以賊必由鳳縣棧道直入略陽,改命應昌、翟文自鄖陽轉赴興安、漢中,以會左光先、趙光遠諸軍。至六月,艾萬年、曹文詔相繼戰歿,賊盡趨西安,承疇急檄應昌及光先還救。八月,李自成陷咸陽。越二日,應昌、光先兵至,擊斬四百四十餘級,獲軍師一人。及全昌兵敗陷賊,其潰卒歸關中,掠沿河州縣。山西巡撫吳甡請令應昌收置麾下,應昌已得疾,不能軍。無何卒。

全昌由蔭敘,歷官靈州參將。崇禎四年,與同官趙大胤擊點燈子於中部,已,連戰合阜陽、韓城,首功多。巡撫練國事請加二將副將銜。大胤駐耀州、富平間,扼賊西路;全昌駐韓城、合阜陽間,扼賊東路。五年七月,代應昌為定邊副總兵。曹文詔追賊隴州、平、鳳界,全昌及馬科率千人應之,殄滅殆盡。

明年五月,擢署都督僉事,充總兵官,鎮守宣府。應昌方鎮山西,兄弟接壤為大帥。明年七月,大清兵西征插漢,旋師入其境。攻圍龍門、新城、赤城,克保安州,薄鎮城,全昌嬰城固守。已而大清兵西行,全昌進兵應州。帝以其孤軍,敕吳襄、尤世威赴援,不應。全昌至渾源,以捷聞,還軍葛峪、羊房口。襄等復不援。八月,大清兵再入其境。閏八月四日,克萬全右衛,他城堡多失守。既解嚴,兄應昌以罪解職,命全昌并將其軍。兵科常自裕言文臣張宗衡等重論,而武臣輕貸,非法。於是全昌與文詔並戍邊。用山西巡撫吳甡請,命全昌、文詔為援剿總兵官,與猛如虎等大破高加計。

八年春,會洪承疇於汝寧,擊敗汝州賊。俄西入關,與祖大弼敗賊涇陽。未幾,敗賊醴泉。五月,與賀人龍敗老回回於秦王嶺。尋解鳳翔圍,走賊秦州,敗之張家川。已而都司田應龍、張應龍戰死,艾萬年、曹文詔相繼歿,官軍益衰,賊盡趨西安。承疇急檄全昌及曹變蛟先赴渭、華格其前,親督軍尾其後,卻賊紅鄉溝,賊乃南入商、雒。承疇又命全昌及趙光遠提兵三千截潼關大峪口,部卒大嘩,闌入滎澤,劫庫殺人。河南巡撫玄默請急援盧氏,不聽。光遠擅歸關中,全昌迤邐至潁州。九月中,追蠍子塊於沈丘瓦店,戰敗被執,賊挾之攻蘄、黃。全昌因代賊求撫,總理盧象昇不許,責全昌喪師辱國,曰「賊果欲降,可滅其黨示信」。賊不聽命。久之,全昌脫歸,謁象昇陽和。象昇令募兵山、陜。尋薦之朝,令赴軍前立功,帝不許。十年四月,以楊嗣昌言逮付法司,謫戍邊衛。

德昌,崇禎初為清水營守備。三年夏,剿王嘉胤被傷,坐奪官。久之,起歷保定參將,連破土寇仁義王。十四年春,總督楊文岳命從虎大威以五千人援開封,不敢進。其冬,擢保定副總兵,仍從文岳,數有功。十六年卒。贈特進榮祿大夫,左都督。

董一元,宣府前衛人。父暘,嘉靖中為宣府遊擊將軍。俺答犯滴水崖,力戰死。贈官錫廕,春秋世祀。兄一奎,都督僉事。歷鎮山西、延綏、寧夏三邊,以勇敢著。一元勇如兄,而智略過之。嘉靖時,歷薊鎮遊擊將軍。土蠻、黑石炭等以萬餘騎犯一片石,總兵官胡鎮禦之,一元功最,超俸三級,遷石門寨參將。隆慶初,破敵棒槌崖,功復最。再進二級,遷副總兵,駐防古北口。移守宣府。萬曆十一年,以都督僉事為昌平總兵官,尋徙宣府。十五年,徙薊州。久之,劾罷。鄭洛經略洮、河,命一元練兵西寧。火落赤入犯,一元擊之西川,多所斬獲。尋以副總兵協守寧夏,擢延綏總兵官。哱拜之亂,套中諸部長悉助之。一元乘其西掠,輕騎搗土昧巢,獲首功百三十,驅其畜產而還,寇內顧引去。進署都督同知,入為中府僉事。

遼東自李成梁後,代以楊紹勛,一歲三失事。尤繼先繼之,半歲病去。廷議擇帥,乃以命一元。泰寧速把亥為官軍所殺,其次子把兔兒常欲復仇。從父炒花及姑婿花大助之,勢益強。西部卜言台周,故插漢土蠻子也,部眾十餘萬,與把兔兒東西相倚,數侵邊。至是卜言合一克灰正、腦毛大諸部,聲犯廣寧。而把兔兒以炒花、花大、煖兔、伯言兒之眾營舊遼陽,將入掠鎮武、錦、義。一元與巡撫李化龍策曰:「卜言雖眾,然去邊遠,我特患把兔兒及炒花耳。今其眾不過萬騎,破之則西部將不戰走。」乃遣副將孫守廉馳右屯禦西部,而親將大軍匿鎮武外,為空營待之。寇騎馳入營,大笑,以為怯,乃深入。官軍忽從中起,奮呼陷陣,自午至酉。寇大奔,逐北七十餘里,至白沙堝。俘斬五百四十有奇,獲馬駝二千計。伯言兒中矢死,把兔兒亦傷,餘眾終夜馳,天明駐馬環哭。其明日,卜言台周入右屯,攻五日夜。守廉等固守,乃引去。時二十二年十月也。捷聞,帝大喜,祭告郊廟,宣捷行賞,進一元左都督,加太子太保,廕本衛世指揮使。兵部尚書石星以下亦進秩有差。

伯言兒最心票悍,諸部倚以為強。嘗誘殺慶雲守備王鳳翔,坐革歲賞。至是被殲,諸部為奪氣,其部下遂納款。把兔兒、炒花及卜言台周、瓜兔兒、歹青復臨邊駐牧,期以明年正月略遼、沈東西。一元慮歲晏不備,為寇所乘,乃先西巡以遏其鋒。化龍亦留弱卒廣寧,數西發以疑寇。一元提健卒,踏冰渡河,監軍楊鎬與之俱。度墨山,天大雪,將士氣益奮。行四百里,三日夜乃抵其巢。斬首百二十級,獲牛馬甲仗無算,全師而還。把兔兒以鎮武創重,歎曰:「我竟不獲報父仇乎?」未幾死,其眾散亂,諸部悉遠遁。一元以功進世廕二秩。久之,以病歸,命王保代。

朝鮮再用師,詔一元隸總督邢玠麾下,參贊軍事。尋代李如梅為禦倭總兵官。時兵分四路。一元由中路,禦石曼子於泗州,先拔晉州,下望晉,乘勝濟江,連毀永春、昆陽二寨。賊退保泗州老營,攻下之,遊擊盧得功陣歿。前逼新寨。寨三面臨江,一面通陸,引海為濠,海艘泊寨下千計,築金海、固城為左右翼。一元分馬步夾攻。步兵遊擊彭信古用大棓擊寨,碎其數處。眾軍進逼賊濠,毀其柵。忽營中炮裂,煙焰漲天。賊乘勢衝擊,固城援賊亦至。騎兵諸將先奔,一元亦還晉州。事聞,詔斬遊擊馬呈文、郝三聘,落信古等職,充為事官;一元亦奪宮保,貶秩三等。會關白死,倭遁走。石曼子為陳璘所殲,一元得還故秩,賚銀幣。久之卒。一元歷鎮衝邊,並著勞績。與麻貴、張臣、杜桐、達雲為邊將選云。

王保,榆林衛人。驍勇絕倫,起行伍,積功為延綏參將。萬曆十六年,遷延綏、定邊副總兵。十九年冬,擢署都督僉事,充昌平總兵官,尋改山西。薊鎮總兵官張邦奇被劾,命保與易任。自嘉靖庚戌後,薊鎮重於他鎮。穆宗有詔,獲大小部長者破格酬,他鎮不得比。迨俺答款塞,宣、大、山西三鎮烽煙寂然。陜西四鎮以火落赤敗盟,始復用兵,然寇弱易禦。獨泰寧、插漢諸部時時犯遼東。而薊門密邇王畿,與遼帥俱慎選。以保有威望,用之。朵顏長昂當張臣鎮薊時納款。居五六年,復連寇石門路、木馬峪、花場谷,遂罷其市賞。後偕銀燈寇山海關。已,又馳喜峰口要賞。邦奇佯許增市,誘殺其通事二十五人。長昂益怒,犯大青山。頃之,遣其黨小郎兒等潛伏喜峰口,射殺偵卒。會保已至,遂擒之。長昂每資小郎兒籌策,懼而謝罪,獻還被掠人畜,保乃釋還小郎兒。長昂補五貢,邊吏始補二賞,互市如初。御史陳遇文援穆宗詔以請,進保署都督同知,副將張守愚以下皆進秩。

薊三協南營兵,戚繼光所募也,調攻朝鮮,撤還,道石門,鼓噪,挾增月餉。保誘令赴演武場,擊之,殺數百人,以反聞。給事中戴士衡、御史汪以時言南兵未嘗反,保縱意擊殺,請遣官按問。巡關御史馬文卿庇保,言南兵大逆有十,尚書石星附會之,遂以定變功進保秩為真,廕子。督撫孫幰、李頤等亦進官受賜,時論尤之。

二十三年冬,順義王撦力克弟趕兔率三軍犯白馬關及東西臺,為守備徐光啟、副將李芳春、戴延春所卻。明年秋,復偕部長倒布犯黑谷頂,敗而去。保度其再至,分營開連口及橫河兒。寇果馳橫河。官軍夜半疾抵石塘嶺,襲其營。寇大驚潰,乘勢追出塞。其冬,復犯羅文峪,敗去。詔代一元鎮遼東。朝鮮再用師,敕保防海,卒於海州。贈左都督。

子學書,宣府總兵官。學時、學禮並副總兵。學書既里居,守榆林城,拒李自成,不屈死。

杜桐,字來儀,崑山人,徙延安衛。萬曆初,由世廕累官清水營守備,以謀勇著。遷延綏入衛遊擊將軍,改古北口參將。用總督梁夢龍薦,擢延綏副總兵。十四年,就拜署都督僉事,充總兵官。

時卜失兔以都督同知為套中主,威令不行,其下各雄長,志常叵測。朔漠素無痘癥,自嘉靖庚戌深入石州,染此癥,犯者輒死。打兒漢者,卜失兔祖吉能部落也,數將命奉貢,累官指揮同知。一日,互市還,與其儕禿退台吉等俱染痘死。禿退子阿計疑邊吏其父,思亂。及卜失兔西助火落赤共擾河西,諸部遂蠢動。十九年冬,打兒漢子土昧與他部明安互市訖,復臨邊要賞,聲犯內地。桐與巡撫賈仁元計先出兵襲之。乃令參將張剛自神木,遊擊李紹祖自孤山,桐率輕騎自榆林,三道並出。遇寇力戰,大破之,斬首四百七十餘級,馘明安而還。延綏自吉能納款,塞上息肩二十年,自此兵端復開。明安子擺言太日思報復,寇鈔無已時矣。桐先被劾罷,以是役功,超授右都督,僉書後府。

二十一年,以總兵官鎮保定。二十四年,徙延綏。明年,再徙鎮寧夏。著力兔、宰僧入犯,逆戰水塘溝,俘斬百二十。寇益糾諸部連犯平虜、興武,桐督諸將馬孔英、鄧鳳、蕭如蕙等連破之,斬首二百餘級。而延綏將士亦數搗巢,諸部長懼,乞款,詞甚哀。三十年,二鎮撫臣孫維城、黃嘉善協謀撫之,乃復貢市。論功,文臣自內閣以下悉進官。桐以先去職,但賚銀幣,許復用而已。久之,卒於家。桐自偏裨至大帥,積首功一千八百,時服其勇。

弟松,字來清。有膽智,勇健絕倫。由舍人從軍,累功為寧夏守備。萬歷二十二年,卜失兔掠張春井,大入下馬關。松偕遊擊史見、李經以二千餘騎邀擊馬蓮井,小勝,誤入伏中,見戰死,松、經皆重傷,士卒死過半。麻貴援軍至,松復裹創力戰,寇始敗走。時松已進遊擊將軍,論功遷延綏參將。貴大舉搗巢,松以右軍出清平塞,多所斬獲,進副總兵。尋以本官改寧夏東路。松為將廉,尚氣不能容物。嘗因小忿,雉髮為僧,部議聽其歸。尋起孤山副總兵。三十三年,擢署都督僉事,代李如樟鎮延綏。明年,套寇犯安邊、懷遠,松大破之,改鎮薊州。

三十六年夏,代李成梁鎮遼東。十二月,敗敵連山驛。賴暈歹者,朵顏長昂子也。狡黠為邊患。與從父蟒金潛入薊鎮河流口,大掠去。復結黃臺吉謀犯喜峰口。松受總督王象乾指,潛搗黃臺吉帳,以牽薊寇。乃從寧遠中左所夜馳至哈流兔,掩殺拱兔部落百四十餘級。以大捷聞,邀重賞。副使馬拯謂拱兔內屬,不當剿,彼且復仇,與松相訐。松忿,邀賞愈急,詔予之。拱兔果以無罪見剿怒,小歹青又數激之,乃以五千騎攻陷大勝堡,執守將耿尚仁支解之。深入小凌河,肆焚掠。遊擊於守志遇於山口,大敗,死千餘人,守志亦重創。松駐大凌河,不敢救。遼人多咎松,朝議謂松前僅抵錦州邊十里,未嘗出塞,所殺乃保塞部落,悉縛殺之,非陣斬。松愈忿,言撫按諸臣附會馬拯,害其奇功。自提兵出塞,將搗巢以雪前恥,而所得止五級,士馬多陷大凌河。松益慚憤,數欲自經,盡焚其鎧胄器仗,置一切疆事弗問。兵部以聞,乃勒松歸里,而以王威代之。

松既廢,時多惜其勇,然惡其僨事,無推轂之者。至四十三年,河套寇大入,令松以輕騎搗火落赤營。獲首功二百有奇,復敘用。逾二年,薊、遼多事,特設總兵官鎮山海關,以松任之。四十六年,張承蔭戰歿,詔松馳援遼陽。明年二月,楊鎬議四路出師。以撫順最衝,令松以六萬兵當之,故總兵趙夢麟、保定總兵王宣為佐。期三月二日抵二道關,會李如柏等並進。松勇而無謀,則愎使氣。二十九日夜,出撫順關,日馳百餘里,抵渾河。半渡,河流急,不能盡渡。松醉趣之,將士多溺河中。松遂以前鋒進,連克二小砦,松喜。三月朔,乘勢趨撒爾湖谷口。時大清方築城界凡山上,役夫萬五千,以精騎四百護之。聞松軍至,精騎則盡伏谷口以待。松軍過將半,伏兵尾擊之,追至界凡渡口,與築城夫合據山旁吉林崖。明日,松引大軍圍崖,別遣將營撒爾湖山上。松軍攻崖,方戰,大清益千人助之,已又續遣二旗兵趨界凡以為援,而遣六旗兵攻松別將於撒爾湖山。明日,六旗兵大戰,破撒爾湖山軍,死者相枕藉。所遣助吉林崖者,自山馳下擊松軍,二旗兵亦直前夾擊,松兵大敗,松與夢麟、宣皆歿於陣。橫尸亙山野,流血成渠。大清兵逐北二十里,至勺琴山而還。時車營五百尚阻渾河,而松已敗。頃之,馬林、劉綎兩軍亦敗,獨李如柏一軍遁還。事聞,朝議多咎松輕進。天啟初,贈少保左都督,世廕千戶,立祠賜祭。宣亦贈官,立祠,世廕指揮僉事。宣,榆林人。夢麟,見父岢傳。

桐子文煥,字弢武。由廕敘,歷延綏遊擊將軍,累進參將、副總兵。四十三年,擢署都督僉事、寧夏總兵官。延綏被寇,文煥赴救,大破之。明年,遂代官秉忠鎮延綏。屢敗寇安邊、保寧、長樂,斬首三百有奇。西路火落赤、卜言太懼,相率降。沙計數盜邊,為文煥所敗,遂納款。既而復與吉能、明愛合,駐高家、柏林邊,要封王、補賞十事。文煥襲其營,斬首百五十。火落赤諸部落攢刀立誓,獻罰九九。九九者,部落中罰駝馬牛羊數也。已,沙計又伏兵沙溝,誘殺都指揮王國安,糾猛克什力犯雙山堡,復犯波羅。文煥擊破之,追奔二十餘里。當是時,套寇號十萬。然其眾分四十二枝,多者二三千,少不及千騎,屢不得志。沙計乃與吉能、明愛、猛克什力相繼納款,延綏遂少事。文煥尋以疾歸。

天啟元年,再鎮延綏。詔文煥援遼,文煥乃遣兵出河套,搗巢以致寇。諸部大恨,深入固原、慶陽,圍延安,揚言必縛文煥,掠十餘日始去。命解職候勘。奢崇明圍成都,總督張我續請令文煥赴救。至則圍已解,偕諸軍復重慶。崇明遁永寧,文煥頓不進。尋擢總理,盡統川、貴、湖廣軍。度不能制賊,謝病去。坐延綏失事罪,戍邊。七年,起鎮寧夏。寧、錦告警,詔文煥馳援,俄令分鎮寧遠。進右都督,調守關門。尋引疾去。

崇禎元年,錄重慶功,廕指揮僉事。三年,陜西群盜起,五鎮總兵並以勤王行。總督楊鶴請令文煥署延鎮事,兼督固原軍。數敗賊,賊亦日益多。會山西總兵王國梁擊王嘉胤於河曲,大敗,賊入據其城。部議設一大將,兼統山、陜軍協討。乃令文煥為提督,偕曹文詔馳至河曲,絕餉道以困之。神一元陷寧塞,文煥家破。遂留文詔,令文煥西還。四年,御史吳甡劾其殺延川難民冒功,給事中張承詔復劾之,下獄褫職。十五年,用總督楊文岳薦,以故官討賊。無功,復謝病歸。

子弘域,天啟初歷延綏副總兵。七年夏,文煥援遼,即擢總兵官,代鎮寧夏。積資至右都督。崇禎中,提督池河、浦口二營練兵,遏賊南渡,頗有功。十三年,移鎮浙江。尋謝病去。困變後,文煥父子歸原籍崑山,卒。

蕭如薰,字季馨,延安衛人。萬曆中,由世蔭百戶歷官寧夏參將,守平虜城。二十年春,哱拜、劉東暘據寧夏鎮城反,譴其黨四出略地。拜子承恩徇玉泉營,遊擊傅桓拒守,為其下所執。賊已徇中衛及廣武,參將熊國臣等棄城奔,列城皆風靡。賊黨土文秀徇平虜,獨如薰堅守不下。如薰妻楊氏,故尚書兆女也,賢而有智,贊夫死守,日具牛酒犒士。拜養子雲最驍勇,引河套著力兔急攻。如薰伏兵南關,佯敗,誘賊入,射雲死,餘眾敗去。又襲著力兔營,獲人畜甚多。著力兔憤,復來攻,為麻貴所卻,城獲全。初,帝聞如薰孤城抗賊,大喜,厚賚銀幣,擢官副總兵。六月,遂以都督僉事為寧夏總兵官,盡統延綏、甘肅、固原諸援軍。其秋,竟與李如松等共平賊,再進署都督同知,蔭錦衣世指揮僉事;妻楊氏亦被旌。

二十二年八月,卜失兔西犯定邊,闌入固原塞,副將姜直不能禦,遂由沙梁聵牆入,直抵下馬關,縱橫內地幾一月。如薰免官,直下吏。尋復以總兵官鎮守固原。套寇入犯,擊卻之。青海寇糾番族犯洮、岷,如薰及臨洮總兵孫仁禦之,擒斬三百四十有奇,撫叛番五千人,獲駝馬甲仗無算。再鎮寧夏。銀定、歹成數入犯,輒挫衄去。徙鎮薊州。久之,罷歸。再起故官,鎮延綏。

天啟初,廷議京軍不足用,召邊將分營訓練。如薰典神機營。陛見,帝賜食加獎勞焉。明年,出鎮徐州。俄召還京,復以總兵官鎮守保定。五年夏,魏忠賢黨劾其與李三才聯姻,遂奪職。祟禎初卒,賜恤如制。

如薰為將持重。更歷七鎮,所在見稱。自隆慶後,款市既成,烽燧少警,輦下視鎮帥為外府。山人雜流,乞朝士尺牘往者,無不饜所欲。薊鎮戚繼光有能詩名,尤好延文士,傾貲結納,取足軍府。如薰亦能詩,士趨之若鶩,賓座常滿。妻楊氏、繼妻南氏皆貴家女,至脫簪珥供客猶不給。軍中患苦之,如薰莫能卻也。一時風會所尚,諸邊物力為耗,識者嘆焉。

如薰祖漢,涼州副總兵、都督僉事。父文奎,京營副將、都督同知。兄如蘭,陜西副總兵、都督僉事,前府僉書;如蕙,寧夏總兵官、都督同知;如芷,提督南京教場、都督僉事。

達雲,涼州衛人。勇悍饒智略。萬曆中,嗣世職指揮僉事。擢守備,進肅州遊擊將軍。炒胡兒入犯,偕參將楊濬擊敗之,遷西寧參將。永邵卜者,順義王俺答從子也,部眾強盛。先嘗授都督同知,再進龍虎將軍。自以貢市在宣府,守臣遇己厚,不可逞,乃隨俺答西迎活佛,留據青海,與瓦剌他卜囊歲為西寧患。嘗誘殺副將李魁。邊臣不能報,益有輕中國心。二十三年九月九日,度將士必燕飲,擁勁騎直入南川。屬番偵告,雲設兵要害,令番人繞出朵爾硤口外,潛扼其背,而己提精卒二千與戰。方合,伏忽起,寇首尾不相顧,番人夾擊,大敗之。雲手馘其帥一人,斬首六百八十餘級。其走峽外者,又為番人所殲。獲駝馬戎器無算。為西陲戰功第一。所馘把都爾哈,即前殺李魁者,其地即魁陣亡處,時又皆九月也。先是,副將李聯芳為寇所殺,總兵尤繼先生獲其仇。邊人以此二事為快。

雲既勝,度寇必復至,厚集以待。踰月,寇果連真相、火落赤諸部,先圍番剌卜爾寨以誘官軍。番不能支,合於寇,寇遂逼西川。雲督諸軍營康纏溝,寇悉眾圍之,矢石如雨。雲左右衝擊,自辰至申,戰數十合。寇死傷無算,乃以長鎗鉤桿專犯西寧軍。西寧軍堅不可破,寇始遁,追奔數十里而還。捷聞,帝大喜,遣官告郊廟,宣捷。大學士趙志皋以下悉進官。雲擢都督同知,蔭本衛世指揮使。寇歲掠諸番,番不敵則折而入寇。及寇敗遠徙,雲急招番,復業者七千餘戶。永邵卜連犯明沙、上谷,雲並擊走之。初,南川奏捷,雲已進副總兵,至是命以總兵官鎮守延綏。未幾,鎮甘肅。二十六年,永邵卜復犯西寧,參將趙希雲等陣歿,雲坐停俸。

甘、寧間有松山,賓兔、阿赤兔、宰僧、著力兔等居之,屢為兩鎮患。巡撫田樂決策恢復。雲偕副將甘州馬應龍、涼州姜河、永昌王鐵塊等分道襲之。寇遠竄,盡拔其巢,攘地五百里。雲以功進右都督,廕世指揮僉事。無何,青海寇糾眾分犯河西,五道俱有備,獻首功百七十有奇。松山既復,為築邊垣,分屯置戍。錄功進左都督。寇戀其故巢,乘官軍撤防時潛兵入犯,雲據險邀擊之。寇大敗,斬首百六十。加雲太子少保。寇益糾其黨犯鎮番,雲及諸將葛賴等大破之,斬首三百七十餘級。帝為告廟,行賞,進雲世廕二秩。寇復入犯,雲破走之。是時,寇失松山,走據賀蘭山。後連青海諸部寇鈔不已,銀定、歹成尤桀驁。三十三年,連營犯鎮番。雲遣副將柴國柱擊之,寇大敗去。未幾,青海寇復大入,將士分道遮擊,生擒其長沙賴,餘敗奔。三十五年敘功,雲增勳廕。是年,松山、青海二寇復連兵犯涼州,雲逆戰紅崖,大獲,斬首百三十有奇。

雲為將,先登陷陣,所至未嘗挫衄,名震西陲,為一時邊將之冠。以秋防卒於軍。贈太子太保。子奇勛,萬曆末為昌平總兵官。

尤繼先,榆林衛人。萬歷中,積功為大同副總兵。十八年,火落赤、真相犯洮河,副總兵李聯芳等戰死。詔進署都督僉事,充總兵官,代劉承嗣鎮守固原。寇據莽剌、捏工二川,日蠶食番族,且擾西寧。聞官軍大集,卜失兔又敗於水泉,乃乘冰堅渡黃河北走,留其黨可卜列、宗塔兒等五百餘人牧莽剌川南山。南山即石門大山口,走烏思藏門戶也。屬番來告,繼先乃令番以八百人前導,與故總兵承嗣、遊擊原進學、吳顯等疾馳七百里,直抵南山。奮擊,大破之,斬首百五十有奇。生獲十二人。而拜巴爾的者,可卜列從子,前殺聯芳,至是被擒。師旋,寇尾至撒川。見有備,乃夜走。他寇犯鎮羌、西寧、石羊亦俱敗。火落赤遂徙帳西海。錄功,進秩為真,增世廕一秩。尋以病歸。起僉中軍府事。

二十一年冬,為遼東總兵官。炒花二千騎入韓家路,繼先督諸軍奮擊,寇乃去。再引疾歸。二十四年,起鎮薊州。自戚繼光鎮守十年,諸部雖叛服不常,然邊警頗稀。寇嘗一入青山口,輒敗去。最後,長昂導班、白二部長入犯,道石門,闞山海關,京東民盡逃入通州。繼先出關,寇已縱掠寧前去。總督蹇達怒繼先不追擊,而繼先方收召降丁八百人,欲倚為用。達乃疏言番情難馭,恐遺後憂,請調繼先別鎮,俾降丁隨往。部議以延綏杜松與易任,巡撫劉四科爭之。達復疏言:「守邊在自強,繼先獨言惟藉降丁。去歲出關,何竟不得降丁力?羽書狎至,邊隘虛實,久為所窺。呼吸變生,安所措手!」兵科宋一韓等力主達議,且劾繼先他事。繼先遂罷,卒於家。

繼先眇一目,習兵敢戰,時稱「獨目將軍」。

官秉忠,榆林衛人。萬曆中起世廕,歷官固原參將,擢寧夏、甘肅副總兵。嘗與主將達雲大破寇於紅崖,銀定、歹成屢被挫去。移守薊鎮東協,積功加署都督同知。四十年五月,擢總兵官,代張承蔭鎮延綏。套寇犯保寧,秉忠督參將杜文煥等敗之白土澗。一日再捷,俘斬二百五十,馘其長十二人。無何,旗牌撒勒犯長樂,秉忠將輕騎追襲之。大獲。猛克什力犯保寧,秉忠又破之。已而猛克挾賞不獲,再寇保寧及懷遠,秉忠隨所向以勁騎遮擊,先後斬首二百二十有奇。猛克及旗牌復以千餘騎犯波羅,遙見保寧軍,遂遁出塞。

吉能者,卜失兔子,為套中之主,士馬雄諸部,見卜失兔襲順義王,補其五年市賞,遂挾求封王,且還八年市賞。邊臣不許,則大怨。會他部鐵雷以痘瘡死,妄言邊吏毒殺之。而沙計盜邊,又被衄去。吉能遂合套中諸部。大舉入寇。東道高家、大柏油、神木、柏林,中道波羅,西道磚井、寧塞,諸城堡盡被蹂躪。副將孫洪謨禦之大柏油,中伏被圍。遊擊萬化孚等不救,士卒死傷過半,洪謨遂降。秉忠聞寇入,急遣遊擊張榜潛劫其營,又敗,死四百餘人。會故帥杜松、寧夏帥杜文煥援軍至,並破敵,而秉忠所部亦有斬獲,寇始退。然猶駐塞下,時鈔掠。秉忠亦屢出襲擊,多獲首功,竟以前負被劾去官。方候代,沙計謀從雙山、建安入犯,秉忠設伏待之。遂大敗去,斬其首二百有奇。

四十六年,與劉綎、柴國柱等同被召,令僉書前府,尋赴援遼東。楊鎬之四路出師也,令秉忠防守鎮城。無何,辭疾歸。久之卒。子撫民,亦為寧夏總兵官。

柴國柱,西寧衛人。萬曆中,由世廕歷西寧守備。驍猛善射。從參將達雲擊寇南川,勇冠軍。錄功,進都指揮僉事。寇盜邊,輒為國柱所挫。屢進涼州副總兵。松山既復,方建堡置堠,寇數來擾,國柱頻擊卻之。銀定、歹成連兵寇鎮番,國柱馳救,斬首二百有奇,獲馬駝甲仗無算。青海寇大掠鎮羌、黑古城諸堡,守備楊國珍不能御,國柱急率遊擊王允中等擊走之。銀定、歹成復犯河西,國柱邀擊,獲首功百二十。擢署都督僉事,陜西總兵官。三十六年春,改鎮甘肅。銀定、歹成屢不得志,益寇鈔永昌。國柱馳與大戰,敗之,追至麻山湖,斬首百六十有奇。其部落復入寇,守備鄭崇雅等戰歿,國柱坐奪俸一年。河套、松山諸部長合兵入寇,國柱檄諸將分道擊,復斬首百六十。屢加右都督,世廕指揮僉事。久之,罷官。四十六年夏,召僉書都督府事。無何,代杜松鎮山海關。松敗歿,虎墩兔乘機犯邊,國柱等力遏之。尋移鎮沈陽。謝病歸。天啟初,追錄邊功,加左都督。卒,賜恤如制。

李懷信,大同人。由世廕歷都指揮僉事,掌山西都司。廉勤,數被推薦。萬曆中,遷延綏中路參將,進定邊副總兵。卜失兔、火落赤、鐵雷、擺言太等歲擾邊。定邊居延綏西,被患尤棘。懷信勇敢有謀,寇入輒敗。其先後鎮帥杜松、王威、張承蔭、官秉忠又皆一時選,故邊患雖劇,而士氣不衰。四十三年,擢甘肅總兵官,延人為立生祠。松山寇入掠蘆溝墩諸處,懷信邀擊,大敗之。斬首三百有奇,獲駝馬甲仗無算。已,復分三道犯鎮番諸堡,懷信亦分遏之。寇引還,將士尾其後,獲首功百九十有奇。自後寇入多失利去,威名著河西。先是,陜西止設四鎮,自西寧多警,增設臨洮總兵官,遂為五鎮。然惟甘、延最當敵衝,故擇帥常慎。而甘肅北有松山,南臨青海,諸部落環居其外,尤難禦。懷信在鎮,邊人恃以無恐。四十七年,遼東急,詔充援剿總兵官,馳赴遼東。時熊廷弼為經略,令懷信偕柴國柱、賀世賢以四萬人守沈陽。煖兔、炒花謀入犯,廷弼急移懷信戍首山,寇不敢入。俄泛懿有警,檄懷信御卻之。遼事益急,諸老將多引避。廷弼復負氣凌諸將,懷信不能堪,亦堅臥引疾去。天啟二年,起鎮大同。明年罷。已,追錄邊功,進左都督。久之,卒於家。

贊曰:張臣諸人,勇略自奮,著效邊陲,均一時良將選也。董一元白沙堝、墨山之捷,奇偉不下王越。至承廕與松,以將門子捐軀報國,視世所稱「東李西麻」者,相去何等也!


\end{pinyinscope}