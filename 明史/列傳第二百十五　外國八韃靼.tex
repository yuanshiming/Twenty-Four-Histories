\article{列傳第二百十五 外國八韃靼}

\begin{pinyinscope}
韃靼,即蒙古,故元後也。太祖洪武元年,大將軍徐達率師取元,元主自北平遁出塞,居開平,數遣其將也速等擾北邊。明年,常遇春擊敗之,師進開平,俘宗王慶孫、平章鼎住。

時元主奔應昌,其將王保保據定西為邊患。三年春,以徐達為大將軍,使出西安搗定西;李文忠為左副將軍,馮勝為右副將軍,使出居庸搗應昌。文忠至興和,擒平章竹貞,復大破元兵於駱駝山,遂趨應昌。未至,知元主已殂,進圍其城,克之。獲元主孫買的里八剌及其妃嬪、大臣、寶玉、圖籍。太子愛猷識理達臘獨以數十騎遁去。而徐達亦大破王保保兵於沈兒峪口,走之。太祖封買的里八剌為崇禮侯,謚元主曰順帝。於是故元諸將江文清等、王子失篤兒等,先後歸附。獨王保保擁太子愛猷識理達臘居和林,屢詔諭之,不從。

五年春,命大將軍徐達、左副將軍李文忠、征西將軍馮勝率師三道征之。大將軍達由中路出鴈門,戰不利,守塞。勝軍西次蘭州。右副將軍傅友德先進,轉戰至埽林山,勝等兵合,斬其平章不花,降上都驢等所部吏民八千三百餘戶,遂由亦集乃路至瓜、沙州,復連敗之。文忠東出居庸至口溫,元將棄營遁,乃率輕騎自臚朐河疾馳,進敗蠻子哈剌章於土剌河,追及阿魯渾河,又追及稱海,獲其官屬子孫并軍士家屬千八百餘,送京師。達等尋召還。明年春,遣達、文忠等備西北邊。元兵入犯武、朔,達遣陳德、郭子興擊破之。未幾,達等復大破王保保兵於懷柔。時元兵先後犯白登、保德、河曲,輒為守將所敗,獨撫寧、瑞州被殘,太祖乃徙其民於內地。

七年夏,都督藍玉拔興和。文忠亦遣裨將擒斬其長,而自以大軍攻高州大石崖,克之,斬宗王、大臣朵朵失里等,至氈帽山斬魯王,獲其妃蒙哥禿。秋,太祖以故元太子流離沙漠,父子隔絕,未有後嗣,乃遣崇禮侯北歸,以書諭之。又二年,其部下九住等寇西邊,敗去。

洪武十一年夏,故元太子愛猷識理達臘卒,太祖自為文,遣使弔祭。子脫古思帖木兒繼立。其丞相驢兒、蠻子哈剌章,國公脫火赤,平章完者不花、乃兒不花,樞密知院愛足等,擁眾於應昌、和林,時出沒塞下。太祖屢賜璽書諭之,不從。十三年春,西平侯沐英師出靈州,渡黃河,歷賀蘭山,踐流沙,擒脫火赤、愛足等於和林,盡以其部曲歸。冬,完者不花亦就擒。明年春,徐達及副將軍湯和、傅友德徵乃兒不花,至河北,襲灰山,斬獲甚眾。

時王保保已先卒,諸巨魁多以次平定,或望風歸附,獨丞相納哈出擁二十萬眾據金山,數窺伺遼。二十年春,命宋國公馮勝為大將軍,率潁川侯傅友德、永昌侯藍玉等,將兵二十萬征之,還其先所獲元將乃剌吾。勝軍駐通州,遣藍玉乘大雪襲慶州,克之。夏,師踰金山,臨江侯陳鏞失道,陷敵死。乃剌吾歸,備以朝廷撫恤恩語其眾,於是全國公觀童來降。納哈出因聞乃剌吾之言已心悸,復為大軍所迫,乃陽使人至大將軍營納款,以覘兵勢。勝遣玉往受降。使者見勝軍還報,納哈出仰天嘆曰:「天弗使吾有此眾矣。」遂率數百騎詣玉納降。已,將脫去,為鄭國公常茂所傷不得去。都督耿忠遂以眾擁之見勝,勝重禮之,使忠與同寢食。先後降其部曲二十餘萬人,及聞納哈出傷,由是驚潰者四萬人,獲輜重畜馬亙百餘里。勝班師,都督濮英以三千騎殿,為潰卒所邀襲,死之。秋,勝等表上納哈出所部官屬二百餘人,將校三千三百餘人,金銀銅印一百顆,虎符牌面百二十五事,馬二百九十餘匹,稱賀。太祖封納哈出為海西侯,先後賜予甚厚,並授乃剌吾千戶。

納哈出既降,帝以故元遺寇終為邊患,乃即軍中拜藍玉為大將軍,唐勝、郭英副之,耿忠、孫恪為左、右參將,率師十五萬往征之。冬,元將脫脫等降於玉。明年春,玉以大軍由大寧至慶州,聞脫古思帖木兒在捕魚兒海,從間道馳進,至百眼井哨不見敵,欲引還。定遠侯王弼曰:「吾等奉聖主威德,提十萬餘眾,深入至此,無所得,何以復命?」玉乃穴地而爨,一夜馳至捕魚兒海。黎明,去敵營八十里。時大風揚沙,晝晦,軍行無知者,敵不設備。弼為前鋒,直薄之,遂大破其軍,斬太尉、蠻子數千人。脫古思帖木兒以其太子天保奴、知院捏怯來、丞相失烈門等數十騎遁去,獲其次子地保奴及妃主五十餘人、渠率三千、男女七萬餘,馬駝牛羊十萬,聚鎧仗焚之。又破其將哈剌章營,盡降其眾。於是漠北削平。捷奏至,太祖大悅,賜地保奴等鈔幣,命有司給供具。既有言玉私元主妃者,帝怒,妃慚懼自殺。地保奴出怨言,帝居之琉球。

脫古思帖木兒既遁,將依丞相咬住於和林,行至土剌河,為其下也速迭兒所襲,眾復散,獨與捏怯來等十六騎偕。適咬住來迎,欲共往依闊闊帖木兒,大雪不得發。也速迭兒兵猝至,縊殺之,并殺天保奴。於是捏怯來、失烈門等來降,置之全寧衛。未幾,捏怯來為失烈門所襲殺,眾潰,詔朵顏等衛招撫之,來降者益眾。二十三年春,命潁國公傅友德等以北平兵從燕王,定遠侯王弼等以山西兵從晉王,徵咬住及乃兒不花、阿魯帖木兒等。燕王出古北口,偵知乃兒不花營迤都,冒大雪馳進,去敵一磧,敵不知也。先遣指揮觀童往,觀童舊與乃兒不花善,一見相持泣。頃之,大軍壓其營,乃兒不花驚,欲遁,觀童止之,引見王,賜飲食慰諭遣還。乃兒不花喜過望,遂偕咬住等來降。久之,乃兒不花等以謀叛誅死,敵益衰。太祖亦封燕、晉諸王為邊籓鎮,更歲遣大將巡行塞下,督諸衛卒屯田,戒以持重,寇來輒敗之。而敵自脫古思帖木兒後,部帥紛拏,五傳至坤帖木兒,咸被弒,不復知帝號。有鬼力赤者篡立,稱可汗,去國號,遂稱韃靼云。

成祖即位,遣使諭之通好,賜以銀幣並及其知院阿魯台、丞相馬兒哈咱等。時鬼力赤與瓦剌相仇殺,數往來塞下,帝敕邊將各嚴兵備之。

永樂三年,頭目埽胡兒、察罕達魯花等先後來歸。久之,阿魯台殺鬼力赤,而迎元之後本雅失里於別失八里,立為可汗。

六年春,帝即以書諭本雅失里曰:「自元運既訖,順帝後愛猷識理達臘至坤帖木兒凡六傳,瞬息之間,未聞一人善終者。我皇考太祖高皇帝於元氏子孫,加意撫恤,來歸者輒令北還,如遣脫古思帖木兒歸,嗣為可汗,此南北人所共知。朕之心即皇考之心。茲元氏宗祧不絕如線,去就之機,禍福由分,爾宜審處之。」不聽。

明年,獲其部曲完者帖木兒等二十二人,帝因復使給事中郭驥齎書往。驥被殺,帝怒。秋,命淇國公丘福為大將軍,武城侯王聰、同安侯火真副之,靖安侯王忠、安平侯李遠為左、右參將,將精騎十萬北討,諭以毋失機,毋輕犯敵,一舉未捷,俟再舉。時本雅失裡已為瓦剌所襲破,與阿魯台徙居臚朐河。福率千騎先馳,遇游兵擊破之。軍未集,福乘勝渡河追敵,敵輒佯敗引去。諸將以帝命止福,福不聽。敵眾奄至,圍之,五將軍皆沒。帝益怒。

明年,帝自將五十萬眾出塞。本雅失里聞之懼,欲與阿魯台俱西,阿魯台不從,眾潰散,君臣始各為部。本雅失里西奔,阿魯台東奔。帝追及斡難河,本雅失里拒戰。帝麾兵奮擊,一呼敗之。本雅失里棄輜重孳畜,以七騎遁。斡難河者,元太祖始興地也。班師至靜虜鎮,遇阿魯台,帝使諭之降。阿魯台欲來,眾不可,遂戰。帝率精騎大呼衝擊,矢下如注,阿魯台墜馬,遂大敗,追奔百餘里乃還。冬,阿魯台使來貢馬,帝納之。

越二年,本雅失里為瓦剌馬哈木等所殺。阿魯台已數入貢,帝俱厚報之,并還其向所俘同產兄妹二人。至是,奏馬哈木等弒其主,又擅立答里巴,願輸誠內附,請為故主復仇。天子義之,封為和寧王。自是,歲或一貢,或再貢,以為常。

十二年,帝征瓦剌。阿魯台使部長以下來朝會。賜米五十石,乾肉、酒糗、彩幣有差。十四年,以戰敗瓦剌,使來獻俘。十九年,阿魯台貢使至邊,要劫行旅,帝諭使戒戢之,由是驕蹇不至。

阿魯台之內附,困于瓦剌,窮蹙而南,思假息塞外。帝納而封之,母妻皆為王太夫人、王夫人。數年生聚,畜牧日以蕃盛,遂慢我使者,拘留之。其貢使歸,多行劫掠,部落亦時來窺塞。二十年春,大人興和。於是詔親征之。阿魯台聞大軍出,懼,其母妻皆詈之曰:「大明皇帝何負爾,而必為逆!」于是盡棄其輜重馬畜于闊灤海側,以其孥直北徙。帝命焚其輜重,收其馬畜,遂班師。

明年秋,邊將言阿魯台將入寇。帝曰:「彼意朕必不復出,當先駐塞下待之。」遂部分寧陽侯陳懋為先鋒,至宿嵬山不見敵,遇王子也先土幹率妻子部屬來降。帝封為忠勇王,賜姓名曰金忠。忠勇王至京師,數請擊敵自效。帝曰:「姑待之。」二十二年春,開平守將奏阿魯台盜邊,群臣勸帝如忠勇王言。帝復親征,師次荅蘭納木兒河,得諜者,知阿魯台遠遁。帝意亦厭兵,乃下詔暴阿魯臺罪惡,而宥其所部來降者,止勿殺。車駕還,崩於榆木川。未幾,阿魯台使來貢馬,仁宗已登極,詔納之。自是,歲修職貢如永樂時。時阿魯台數敗于瓦剌,部曲離散。其屬把的等先後來歸,朝廷皆予官職,賜鈔幣,詔有司給供具。自後來歸者,悉如例。阿魯台日益蹙,乃率其屬東走兀良哈,駐牧遼塞。諸將請出兵掩擊之,帝不聽。

宣德九年,阿魯台復為脫脫不花所襲,妻子死,孳畜略盡,獨與其子失捏乾等徙居母納山、察罕腦剌等處。未幾,瓦剌脫懽襲殺阿魯台及失捏干,於是阿魯台子阿卜只俺及其孫妻速木答思等喪敗無依,來乞內附。帝憐而撫之。

阿魯台既死,其故所立阿台王子及所部朵兒只伯等復為脫脫不花所窘,竄居亦集乃路。外為納款,而數入寇甘、涼。正統元年,將軍陳懋敗朵兒只伯于平川,追及蘇武山,頗有斬獲。二年冬,命都督任禮為總兵官,蔣貴、趙安副之,尚書王驥督師,以便宜行事。明年夏,復敗朵兒只伯等于石城。阿台與朵兒合,復敗之兀魯乃地,追及黑泉,又及之刁力溝,出沙漠千里,東西夾擊,敵幾盡,先後獲其部長一百五十人。於是阿台、朵兒只伯等來歸。

未幾,脫脫不花捕阿台等殺之。脫脫不花者,故元後,韃靼長也。瓦剌脫心雚既擊殺阿魯台,悉收其部,兼並賢義、安樂二王之眾,欲自立為可汗。眾不可,乃立脫脫不花,以阿魯台眾屬之,自為丞相,陽推奉之,實不承其號令。

脫懽死,子也先嗣,益桀驁自雄,諸部皆下之,脫脫不花具可汗名而已。脫脫不花歲來朝貢,天子皆厚報之,比諸蕃有加,書稱之曰達達可汗,賜賚并及其妃。十四年秋,也先謀大舉入寇,脫脫不花止之曰:「吾儕服食,多資大明,何忍為此?」也先不聽,曰:「可汗不為,吾當自為。」遂分道,俾脫脫不花侵遼東,而自擁眾從大同入。帝親征之,駕於土木陷焉。景皇帝自監國即位,尊帝為太上皇帝。明年秋,上皇歸自也先所。事載《瓦剌傳》。

脫脫不花自上皇歸後,修貢益勤。嘗妻也先姊,生子,也先欲立之,不從。也先亦疑其與中國通,將害己,遂治兵相攻。也先殺脫脫不花,收其妻子孳畜,給諸部屬,而自立為可汗。時景皇帝二年也。朝廷稱也先為瓦剌可汗。

未幾,為所部阿剌知院所殺。韃靼部長孛來復攻破阿剌,求脫脫不花子麻兒可兒立之,號小王子。阿剌死,而孛來與其屬毛里孩等皆雄視部中,于是韃靼復熾。

景泰六年遣使入貢。英宗復辟,遣都督馬政往賜故伯顏帖木兒妻幣。孛來留之,而遣使入賀,欲獻璽。帝敕之曰:「璽已非真,即真,亦秦不祥物耳,獻否從爾便。第無留我使,以速爾禍。」時敵數寇威遠諸衛,夏,定遠伯石彪敗之於磨兒山。

天順二年,孛來大舉寇陜西,安遠侯柳溥禦之輒敗,而飾小捷以聞。明年春,敵入安邊營,石彪等破之,都督周賢、指揮李金監戰死。四年復寇榆林,彰武伯楊信拒卻之。再入,敗之於金雞峪。未幾,復大掠陜西諸邊,廷臣請治各守將罪,帝宥之。五年春,寇入平虜城,誘指揮許顒等入伏,殺之。邊報日亟,命侍郎白圭、都御史王竑往視師。秋,孛來求款,帝使詹昇齎敕往諭。孛來遣使隨昇來貢,請改大同舊貢道,而由陜西蘭縣入,許之。未幾,復糾其屬毛里孩等入河西。明年春,圭等分巡西邊,圭遇敵於固原川,竑遇敵於紅崖子川,皆破之。帝賜璽書獎勵,敕孛來使臣仍從大同入貢。

時麻兒可兒復與孛來相仇殺。麻兒可兒死,眾共立馬可古兒吉思,亦號小王子。自是,韃靼部長益各專擅。小王子稀通中國,傳世次,多莫可考。孛來等每歲入貢,數寇掠,往來塞下,以西攻瓦剌為辭,又數要劫三衛。七年冬,貢使及關,帝卻之,以大學士李賢言乃止。八年春,御史陳選言:「韃靼部落,孛來最強,又密招三衛諸蕃,相結屯住。去冬來朝,要我賞宴,窺我虛實,其犯邊之情已露。而我邊關守臣,因循怠慢,城堡不修,甲仗不利,軍士不操習,甚至富者納月錢而安閒,貧者迫饑寒而逃竄。邊備廢弛,緩急何恃?乞敕在邊諸臣,痛革前弊。其鎮守、備禦等官,亦宜以時黜陟,庶能者知奮,怠者知警。至阨塞要害之處,或益官軍,或設營堡,或用墩臺,咸須處置得宜,歲遣大臣巡視,庶邊防有備,寇氛可戢。」報聞。

成化元年春,孛來誘兀良哈九萬騎入遼河,武安侯鄭宏禦卻之。秋,散掠延綏。冬,復大入。命彰武伯楊信率山西兵,都御史項忠率陜西兵禦之,少卻。未幾,復渡河曲,圍黃甫川堡,官軍力戰,乃引去。

始,韃靼之來也,或在遼東、宣府、大同,或在寧夏、莊浪、甘肅,去來無常,為患不久。景泰初,始犯延慶,然部落少,不敢深入。天順間,有阿羅出者,率屬潛入河套居之,遂逼近西邊。河套,古朔方郡,唐張仁愿築三受降城處也。地在黃河南,自寧夏至偏頭關,延袤二千里,饒水草,外為東勝衛。東勝而外,土平衍,敵來,一騎不能隱,明初守之,後以曠絕內徙。至是,孛來與小王子、毛里孩等先後繼至,擄中國人為鄉導,抄掠延綏無虛時,而邊事以棘。

二年夏,大入延綏。帝命楊信充總兵官,都督趙勝為副,率京軍及諸邊卒二萬人討之。信先以議事赴闕,未至。敵散掠平涼,入靈州及固原,長驅寇靜寧、隆德諸處。冬,復入延綏,參將湯胤績戰死。

未幾,諸部內爭,孛來弒馬可古兒吉思,毛里孩殺孛來,更立他可汗。斡羅出者復與毛里孩相仇殺,毛里孩遂殺其所立可汗,逐斡羅出,而遣使入貢。尋渡河掠大同。三年春,帝命撫寧侯朱永等征之。會毛里孩再乞通貢,而別部長孛魯乃亦遣人來朝。帝許之,詔永等駐軍塞上。

四年秋,給事中程萬里上言:「毛里孩久不朝貢,窺伺邊疆,其情叵測。然臣度其有可敗者三。近我邊地才二三日程,彼客我主,一也。兼并諸部,馳驅不息,既驕且疲,二也。比來散逐水草,部落四分,兵力不一,三也。宜選精兵二萬,每三千人為一軍,統以驍將,嚴其賞罰,使探毛里孩所在,潛師搗之,破之必矣。」帝壯之,而不能用。冬寇延綏。明年春再入。守將許寧等輒擊敗之。冬復糾三衛入寇,延綏、榆林大擾。

六年春,大同巡撫王越遣遊擊許寧擊敗之;楊信等亦大破之于胡柴水冓。時孛魯乃與斡羅出合別部加思蘭、孛羅忽亦入據河套,為久居計。延綏告急,帝命永為將軍,以王越參贊軍務,使禦敵。永至,數以捷聞,越等皆升賞,論功永世侯,而敵據套自如。

七年春,永上戰守二策,廷議以糧匱馬乏,難於進剿,請命邊將慎守禦以圖萬全。于是吏部侍郎葉盛巡邊,偕延綏巡撫餘子俊及越議築邊牆,設立臺堡。冬,敵入塞,參將錢亮敗績,越等不能救。兵部尚書白圭請擇遣大將軍專事敵,會盛還,越亦赴京計事,乃集廷議,請大發兵搜套。帝以武靖侯趙輔為將軍,節制諸路,王越仍督師。敵大入延綏,輔不能禦,遂召還,以寧晉伯劉聚代之,聚亦未有功。而毛里孩、孛魯乃、斡羅出稍衰,滿都魯入河套稱可汗,加思蘭為太師。

九年秋,滿都魯等與孛羅忽並寇韋州。王越偵知敵盡行,其老弱巢紅鹽池,乃與許寧及遊擊周玉率輕騎晝夜疾馳至,分薄其營,前後夾擊,大破之。復邀擊于韋州。滿都魯等敗歸,孳畜廬帳蕩盡,妻孥皆喪亡,相顧悲哭去。自是不復居河套,邊患少弭;間盜邊,弗敢大入,亦數遣使朝貢。

初,加思蘭以女妻滿都魯,立為可汗。久之殺孛羅忽,并其眾,益專恣。滿都魯部脫羅干、亦思馬因謀殺之。尋滿都魯亦死,諸強酋相繼略盡,邊人稍得息肩。

時中官汪直怙恩用事,思以邊功自樹,王越、朱永附之。十六年春,邊將上言,傳聞敵將渡河,遽以永為將軍。直與越督師至邊,未及期,襲敵於威寧海子,大破之,又敗之于大同。永晉公爵,予世襲,越封威寧伯,直增祿至三百石。未幾,詔以越代永總兵。于是亦思馬因等益糾眾盜邊,延及遼塞。秋,敵三萬騎寇大同,連營五十里,殺掠人畜數萬。總兵許寧禦之,兵敗,以捷聞。敵既得利,長驅入順聖川,散掠渾源、朔諸州。宣府巡撫秦紘、總兵周玉力戰卻之。山西巡撫邊鏞,參將支玉等悉力捍禦,敵去輒復來,迄成化末無寧歲。

亦思馬因死,入寇者復稱小王子,又有伯顏猛可王。弘治元年夏,小王子奉書求貢,自稱大元大可汗。朝廷方務優容,許之。自是,與伯顏猛可王等屢入貢,漸往來套中,出沒為寇。八年,北部亦卜剌因王等入套駐牧。于是小王子及脫羅干之子火篩相倚日強,為東西諸邊患。其年,三入遼東,多殺掠。明年,宣、大、延綏諸境俱被殘。

十一年秋,王越既節制諸邊,乃率輕兵襲敵於賀蘭山後,破之。明年,敵擁眾入大同、寧夏境,遊擊王杲敗績,參將秦恭、副總兵馬升逗遛不進,皆論死。時平江伯陳銳為總兵,侍郎許進督師,久無功,被劾去,以保國公朱暉、侍郎史琳代之,太監苗逵監軍。

十三年冬,小王子復居河套。明年春,吏部侍郎王鏊上禦敵八策:一曰定廟算,二曰重主將,三曰嚴法令,四曰恤邊民,五曰廣招募,六曰用間,七曰分兵,八曰出奇。帝命所司知之。時敵以八千騎東駐遼塞下,攻入長勝堡,殺掠殆盡。秋,暉等以五路之師夜襲敵于河套,斬首三級,驅孳畜千餘歸,賞甚厚。小王子以十萬騎從花馬池、鹽池入,散掠固原、寧夏境,三輔震動,戕殺慘酷。

十五年,以戶部尚書秦紘總制陜西。夏,敵入遼東清河堡,至密雲,旋西掠偏頭關。秋,復以五千騎犯遼東長安堡,副總兵劉祥禦之,斬首五十一級,敵乃退。明年,稍靖。

十七年春,敵上書請貢,許之,竟不至;仍入大同殺墩軍,犯宣府及莊浪,守將衛勇、白玉等禦卻之。明年春,敵三萬騎圍靈州,復散掠內地,指揮仇鉞、總兵李祥擊走之。敵大舉入寇宣府,總兵張俊禦之,大敗,裨將張雄、穆榮戰歿。

武宗嗣位,復命暉、琳出禦。冬,敵入鎮夷所,指揮劉經死之。復自花馬池毀垣入,掠隆德、靜寧、會寧諸處,關中大擾,以楊一清為總制。時正德元年春也。

劉瑾用事,監軍皆閹人,一清不得職去,文貴、才寬相繼受事。二年,敵入寧夏、莊浪及定遼後衛諸境,守將皆逮問。

四年,敵數寇大同。冬,才寬禦敵於花馬池,中伏死。總兵馬昂與別部亦孛來戰于木瓜山,勝之,斬三百六十五級,獲馬畜六百餘,軍器二千九百餘。

明年,北部亦卜剌與小王子仇殺。亦卜剌竄西海,阿爾禿廝與合,逼脅洮西屬番,屢入寇。巡撫張翼、總兵王勛不能制,漸深入,邊人苦之。八年夏,擁眾來川,遣使詣翼所,乞邊地駐牧修貢。翼啖以金帛,令遠徙,亦卜剌遂西掠烏斯藏,據之。自是洮、岷、松潘無寧歲。

小王子數入寇,殺掠尤慘。復以五萬騎攻大同,趣朔州,掠馬邑。帝命咸寧侯仇鉞總兵禦之,戰于萬全衛,斬三級,而所失亡十倍,以捷聞。明年秋,敵連營數十,寇宣、大塞,而別遣萬騎掠懷安。總制叢蘭告急,命太監張永督宣、大、延綏兵,都督白玉為大將,協蘭守禦,京師戒嚴。已,敵踰懷安趣蔚州,至平虜城南,蘭等預置毒飯於田間如農家餉,而設伏以待。敵至,中毒,伏猝發,多死者。其年,小王子部長卜兒孩以內難復奔據西海,出沒寇西北邊。

十一年秋,小王子以七萬騎分道入,與總兵潘浩戰于賈家灣。浩再戰再敗,裨將朱春、王唐死之。張永遇於老營坡,被創走居庸。敵遂犯宣府,凡攻破城堡二十,殺掠人畜數萬。浩奪三官,諸將降罰有差。

十二年冬,小王子以五萬騎自榆林入寇,圍總兵王勛等於應州。帝幸陽和,親部署,督諸將往援,殊死戰,敵稍卻。明日復來攻,自辰至酉,戰百餘合,敵引而西,追至平虜、朔州,值大風黑霧,晝晦,帝乃還,命宣捷於朝。是後歲犯邊,然不敢大入。

嘉靖四年春,以萬騎寇甘肅。總兵姜奭禦之於苦水墩,斬其魁。明年犯大同及宣府,亦卜剌復駐牧賀蘭山後,數擾邊。明年春,小王子兩寇宣府。參將王經、關山先後戰死。秋,以數萬騎犯寧夏塞,尚書王憲以總兵鄭卿等敗之,斬三百餘級。明年春,掠山西。夏,入大同中路,參將李蓁禦卻之。冬,復寇大同,指揮趙源戰死。

十一年春,小王子乞通貢,未得命,怒,遂擁十萬騎入寇。總制唐龍請許之,帝不聽。龍連戰,頗有斬獲。

時小王子最富強,控弦十餘萬,多畜貨貝,稍厭兵,乃徙幕東方,稱土蠻,分諸部落在西北邊者甚眾。曰吉囊、曰俺答者,於小王子為從父行,據河套,雄黠喜兵,為諸部長,相率躪諸邊。

十二年春,吉囊擁眾屯套內,將犯延綏,邊臣有備,乃突以五萬騎渡河西,襲亦不剌、卜兒孩兩部,大破之。卜兒孩為莊、寧邊患久,亦郎骨、土魯番諸蕃皆苦之,嘗因屬番帖木哥求貢市,朝廷未之許。至是唐龍以卜兒孩衰敗遠徙,西海獲寧,請無更議款事。

吉囊等既破西海,旋竊入宣府永寧境,大掠而去。冬,犯鎮遠關,總兵王效、副總兵梁震敗之於柳門,又追敗之於蜂窩山,敵溺水死者甚眾。明年春,寇大同。秋,復由花馬池入犯,梁震及總兵劉文拒卻之。

十五年夏,吉囊以十萬眾屯賀蘭山,分兵寇涼州,副總兵王輔御之,斬五十七級。又入莊浪境,總兵姜奭遇之於分水嶺,三戰三勝之。又入延綏及寧夏邊。冬,復犯大同,入掠宣大塞,總制侍郎劉天和、總督尚書楊守禮及巡撫都御史楚書悉力禦之。

十九年秋,書以總兵白爵等三敗敵於萬全右衛境,斬百餘級。天和以總兵周尚文大破敵於黑水苑,斬吉囊子小十王。明年春,守禮以總兵李義禦敵於鎮朔堡,以總兵楊信禦敵於甘肅,皆勝之。

秋,俺答及其屬阿不孩遣使石天爵款大同塞,巡撫史道以聞,詔卻之。以尚書樊繼祖督宣大兵,懸賞格購俺答、阿不孩首。遂大舉內犯,俺答下石嶺關,趣太原。吉囊田平虜衛入掠平定、壽陽諸處。總兵丁璋、遊擊周宇戰死,諸將多獲罪,繼祖獨蒙賞。

二十一年夏,敵復遣天爵求貢。大同巡撫龍大有誘縛之,上之朝,詭言用計擒獲。帝悅,擢大有兵部侍郎,邊臣升賞者數十人,磔天爵於市。敵怒,入寇,掠朔州,抵廣武,由太原南下,沁、汾、襄垣、長子皆被殘;復從忻、崞、代而北,屯祁縣。參將張世忠力戰,敵圍之數重。自巳至申,所殺傷相當。已而世忠矢盡見殺,百戶張宣、張臣俱死,敵遂從鴈門故道去。秋,復入朔州。吉囊死,諸子狼臺吉等散處河西,勢既分,俺答獨盛,歲數擾延綏諸邊。

二十三年冬,小王子自萬全右衛入,至蔚州及完縣。京師戒嚴。

二十四年秋,俺答犯延綏及大同,總兵張達拒卻之。又犯鵓鴿峪,參將張鳳、指揮劉欽、千戶李瓚、生員王邦直等皆戰死。會總督侍郎翁萬達、總兵周尚文嚴兵備陽和,敵引去。明年夏,俺答復遣使詣大同塞,求貢,邊卒殺之。秋,復來請,萬達再疏以聞,帝不許。敵以十萬騎西入保安,掠慶陽、環縣而東,以萬騎寇錦、義。總督三邊侍郎曾銑率參將李珍等直搗敵巢於馬梁山後,斬百餘級,敵始退。

銑議復河套,大學士夏言主之。帝方嚮用言,令銑圖上方略,以便宜從事。明年夏,萬達復言:「敵自冬涉春屢求貢,詞恭,似宜許。」不聽,責萬達罔瀆。銑鳩兵繕塞,輒破敵。既而帝意中變,言與銑竟得罪,斬西市。敵益蓄忿思逞,廷臣不敢言復套事矣。

二十八年春,犯宣府滴水崖。把總指揮江瀚、董暘戰死,全軍覆,遂犯永寧、大同。總兵周尚文禦之於曹家莊,大敗之,斬其魁。會萬達自懷來赴援,宣府總兵趙國忠聞警,亦率千騎追擊,復連敗之。是歲,犯西塞者五。

二十九年春,俺答移駐威寧海子。夏,犯大同,總兵張達、林椿死之。敵引去,傳箭諸部大舉。秋,循潮河川南下至古北口,都御史王汝孝率薊鎮兵禦之。敵陽引滿內嚮,而別遣精騎從間道潰牆入。汝孝兵潰,遂大掠懷柔,圍順義,抵通州,分兵四掠,焚湖渠馬房。畿甸大震。

敵大眾犯京師,大同總兵咸寧侯仇鸞、巡撫保定都御史楊守謙等,各以勤王兵至。帝拜鸞為大將軍,使護諸軍。鸞與守謙皆軿懦不敢戰,兵部尚書丁汝夔心匡擾不知所為,閉門守。敵焚掠三日夜,引去。帝誅汝夔及守謙。敵將出白羊口,鸞尾之。敵猝東返,鸞出不意,兵潰,死傷千餘人。敵乃徐由古北口出塞。諸將收斬遺屍,得八十餘級,以捷聞。

方俺答薄都城時,縱所擄馬房內官楊增持書入城求貢。輔臣徐階等謂當以計款之,諭令退屯塞外,因邊臣以請。俺答歸,遣子脫脫陳款。時鸞方用事,乃議開馬市以中敵。兵部郎中楊繼盛上疏爭之,不得。明年春,以侍郎史道蒞其事,給白金十萬,開市大同,次及延、寧。叛人蕭芹、呂明鎮者,故以罪亡入敵,挾白蓮邪教,與其黨趙全、丘富、周原、喬源諸人導俺答為患。俺答市畢,旋入掠。邊臣責之,以芹等為詞。芹詭有術,能墮城。敵試之不驗,遂縛芹及明鎮,而全、富等竟匿不出。俺答復請以牛馬易粟豆,求職役誥敕,又潛約河西諸部內犯,墮諸邊垣。帝惡之,詔罷馬市,召道還。自是,敵日寇掠西邊,邊人大困。

三十一年春,敵二千騎寇大同,指揮王恭禦之於平川墩,戰死。夏,東入遼塞,圍百戶常祿,指揮姚大謨、劉棟、劉啟基等於三道水冓,四人皆戰沒。備禦指揮王相赴援,大戰於寺兒山,殺傷相當,敵舍去。千戶葉廷瑞率百人助相。明日,相裹創復邀敵於蠟黎山,殊死鬥,矢竭,遂與麾下將士三百人皆死之。廷瑞被創死復蘇,敵亦引退。其年,凡四犯大同,三犯遼陽,一犯寧夏。明年春,犯宣府及延綏。夏,犯甘肅及大同。守將禦之輒敗。秋,俺答復大舉入寇,下渾源、靈丘、廣昌,急攻插箭、浮圖等峪。固原遊擊陳鳳、寧夏遊擊朱玉率兵赴援,大戰卻之。敵分兵東犯蔚,西掠代、繁畤。已,駐鹿阜、延二十日,延慶諸城屠掠幾遍,乃移營中部,以瞰涇、原,會久雨乃去。時小王子亦乘隙為寇,犯宣府赤城。未幾,俺答復以萬騎入大同,縱掠至八角堡。巡撫趙時春禦之,遇敵於大蟲嶺,總兵李淶戰死,軍覆,時春僅以身免。

三十三年春,入宣府柴溝堡。夏,復犯寧夏,大同總兵岳懋中伏死。秋,攻薊鎮牆,百道並進。警報日數十至,京師戒嚴。總督楊博悉力拒守,募死士夜砍其營,敵驚擾乃遁。明年數犯宣、薊,參將趙傾葵、李光啟、丁碧先後戰死。朝廷再下賞格,購俺答首,賜萬金,爵伯;獲丘富、周原者三百金,授三品武階。時富等在敵,招集亡命,居豐州,築城自衛,構宮殿,墾水田,號曰板升。板升,華言屋也。趙全教敵,益習攻戰事。俺答愛之甚,每入寇必置酒全所問計。

三十五年夏,敵三萬騎犯宣府。遊擊張紘迎戰,敗死。冬,掠大同邊,繼掠陜西環、慶諸處,守將孫朝、袁正等卻之。其年,土蠻再犯遼東。

明年,敵以二萬騎分掠大同邊,殺守備唐天祿、把總汪淵。俺答弟老把都復擁眾數萬入河流口,犯永平及遷安,副總兵蔣承勛力戰死。夏,突犯宣府馬尾梁,參將祁勉戰死。秋,復入大同右衛境,攻毀七十餘堡,所殺擄甚眾。冬,俺答子辛愛有妾曰桃松寨,私部目收令哥,懼誅來降。總督楊順自詡為奇功,致之闕下。辛愛來索不得,乃縱掠大同諸墩堡,圍右衛數匝。順懼,乃詭言敵願易我以趙全、丘富。本兵許論以為便,乃遣桃松寨夜逸出塞,紿之西走,陰告辛愛,辛愛執而戮之。敵狎知順無能,圍右衛益急,更分兵犯宣、薊鎮。西鄙震動,右衛烽火斷絕者六閱月。大學士嚴嵩與許論議,欲棄右衛。帝不聽,詔諸臣發兵措餉,而以兵部侍郎江東代順。時故將尚表以饋餉入圍城,悉力捍禦,粟盡食牛馬,徹屋為薪,士卒無變志。表時出兵突戰,獲俺答孫及婿與其部將各一人。會帝所遣侍郎江東及巡撫楊選、總兵張承勳等各嚴兵進,圍乃解。復掠永昌、涼州及宣府赤城,圍甘州十四日始退。土蠻亦數寇遼東。

三十八年春,老把都、辛愛謀大舉入犯,駐會州,使其諜詭稱東下。總督王忬不能察,遽分兵而東,號令數易,敵遂乘間入薊鎮潘家口,忬得罪。夏,犯大同,轉掠宣府東西二城,駐內地旬日,會久雨乃退。

三十九年,敵聚眾喜峰口外,窺犯薊鎮。大同總兵劉漢出搗其帳於灰河,敵稍遠徙。秋,漢復與參將王孟夏等搗豐州,擒斬一百五十人,焚板升略盡。是歲,寇大同、延綏、薊、遼邊無虛日。明年春,敵自河西踏冰入寇,守備王世臣、千戶李虎戰死。秋,犯宣府及居庸。冬,掠陜西、寧夏塞。已,復分兵而東,陷蓋州。

四十一年夏,土蠻入撫順,為總兵黑春所敗。冬,復攻鳳凰城,春力戰二日夜,死之。海、金殺掠尤甚。冬,俺答數犯山西、寧夏塞。延綏總兵趙岢分部銳卒,令裨將李希靖等東出神木堡,搗敵帳於半坡山,徐執中等西出定邊營,擊敵騎於荍麥湖,皆勝之,斬一百十九級。

四十二年春,敵入宣府滴水崖,劉漢卻之。敵遂引而東,數犯遼塞。秋,總兵楊照敗死。時薊遼總督楊選囚縶三衛長通罕,令其諸子更迭為質。通罕者,辛愛妻父也,冀以牽制辛愛,三衛皆怨。冬,大掠順義、三河。諸將趙溱、孫臏戰死,京師戒嚴。大同總兵姜應熊禦之於密雲,敗之,敵退。詔誅選。明年,土蠻入遼東,都御史劉燾上諸將守禦功,言海水暴漲,敵騎多沒者。帝曰:「海若效靈。」下有司祭告,燾等皆有賞。冬,敵犯狹西,大掠板橋、響閘兒諸處。

四十四年春,犯遼東寧前小團山,參將線補袞、遊擊楊維籓死之。夏,犯肅州,總兵劉承業禦之,再戰皆捷。秋,俺答子黃台吉帥輕騎,自宣府洗馬林突入,散掠內地。把總姜汝棟以銳卒二百伏暗莊堡,猝遇台吉,搏之。台吉墮馬,為所部奪去。台吉受傷,越日始蘇。明年,俺答屢犯東西諸塞。夏,清河守備郎得功扼之張能峪口,勝之。冬,大同參將崔世榮禦敵於樊皮嶺,及子大朝、大賓俱戰死。時丘富死,趙全在敵中益用事,尊俺答為帝,治宮殿。期日上棟,忽大風,棟墜傷數人。俺答懼,不敢復居。兵部侍郎譚綸在薊鎮善治兵,全乃說俺答無輕犯薊,大同兵弱,可以逞。

隆慶元年,俺答數犯山西。秋,復率眾數萬分三道入井坪、朔州、老營、偏頭關諸處。邊將不能禦,遂長驅攻岢嵐及汾州,破石州,殺知州王亮采,屠其民,復大掠孝義、介休、平遙、文水、交城、太谷、隰州間,男女死者數萬。事聞,諸邊臣罰治有差。而三衛勾土蠻同時入寇,薊鎮、昌黎、撫寧、樂亭、盧龍,皆被蹂躪。遊騎至灤河,京師震動,三日乃引去。諸將追之,敵出義院口。會大霧,迷失道,墮棒槌崖中,人馬枕藉,死者頗眾,諸將乃趨割其首。

二年,敵犯柴溝,守備韓尚忠戰死。時兵部侍郎王崇古鎮西邊,總兵李成梁守遼東,數以兵邀擊於塞外。敵知有備,入寇稍稀。

四年秋,黃台吉寇錦州,總兵王治道、參將郎得功以十餘騎入敵死。冬,俺答有孫曰把漢那吉者,俺答第三子鐵背台吉子也,幼孤,育於俺答妻所。既長,娶婦比吉。把漢復聘襖兒都司女,即俺答外孫女,貌美,俺答奪之。把漢恚,遂率其屬阿力哥等十人來降。大同巡撫方逢時受之,以告總督王崇古。崇古上言:「把漢來歸,非擁眾內附者比,宜給官爵,豐館餼,飭輿馬,以示俺答。俺答急,則使縛送板升諸叛人;不聽,即脅誅把漢牽沮之;又不然,因而撫納,如漢置屬國居烏桓故事,使招其故部,徙近塞。俺答老且死,黃臺吉立,則令把漢還,以其眾與台吉抗,我按兵助之。」詔可,授把漢指揮使,阿力哥正千戶。

俺答方西掠吐番,聞之亟引還,約諸部入犯,崇古檄諸道嚴兵禦之。敵使來請命,崇古遣譯者鮑崇德往,言朝廷待把漢甚厚,第能縛板升諸叛人趙全等,旦送至,把漢即夕返矣。俺答大喜,屏人語曰:「我不為亂,亂由全等。若天子幸封我為王,長北方諸部,孰敢為患?即死,吾孫當襲封,彼衣食中國,忍倍德乎?」乃益發使與崇德來乞封,且請輸馬,與中國鐵鍋、布帛互市,隨執趙全、李自馨等數人來獻。崇古乃以帝命遣把漢歸,把漢猶戀戀,感泣再拜去。俺答得孫大喜,上表謝。

崇古因上言:「朝廷若允俺答封貢,諸邊有數年之安,可乘時修備。設敵背盟,吾以數年蓄養之財力,從事戰守,愈於終歲奔命,自救不暇者矣。」復條八事以請。一,議封號官爵。諸部行輩,俺答為尊,宜錫以王號,給印信。其大枝如老把都、黃台吉及吉囊長子吉能等,俱宜授以都督。弟姪子孫如兀慎打兒漢等四十六枝,授以指揮。其俺答諸婿十餘枝,授以千戶。一,定貢額。每歲一入貢,俺答馬十匹,使十人。老把都、吉能、黃臺吉八匹,使四人。諸部長各以部落大小為差,大者四匹,小者二匹,使各二人。通計歲貢馬不得過五百匹,使不得過百五十人。馬分三等,上駟三十進御,餘給價有差,老瘠者不入。其使,歲許六十人進京,餘待境上。使還,聽以馬價市繒布諸物。給酬賞,其賞額視三衛及西蕃諸國。一,議貢期、貢道。以春月及萬壽聖節四方來同之會,使人,馬匹及表文自大同左衛驗入,給犒賞。駐邊者,分送各城撫鎮驗賞。入京者,押送自居庸關入。一,立互市。其規如弘治初,北部三貢例。蕃以金、銀、牛馬、皮張、馬尾等物,商販以緞紬、布匹、釜鍋等物。開市日,來者以三百人駐邊外,宣府應於萬全右衛、張家口邊外,山西應於水泉營邊外。一,議撫賞。守市兵人布二匹,部長緞二匹、紬二匹。以好至邊者,酌來使大小,量加賞犒。一,議歸降。通貢後,降者不分有罪無罪,免收納。其華人被擄歸正者,查別無竊盜,乃許入。一,審經權。一,戒狡飾。

疏入,下廷臣議。帝終從崇古言,詔封俺答為順義王,賜紅蟒衣一襲;昆都力哈、黃台吉授都督同知,各賜紅獅子衣一襲、彩幣四表裏;賓兔台吉等十人,授指揮同知;那木兒臺吉等十九人,授指揮僉事;打兒漢台吉等十八人,授正千戶;阿拜台吉等十二人,授副千戶;恰台吉等二人,授百戶。昆都力哈,即老把都也。兵部採崇古議,定市令。秋市成,凡得馬五百餘匹,賜俺答等彩幣有差。西部吉能及其姪切盡等亦請市,詔予市紅山墩暨清水營。市成,亦封吉能為都督同知。已而俺答請金字經及剌麻僧,詔給之。崇古復請玉印,詔予鍍金銀印。俺答老佞佛,復請於海南建寺,詔賜寺額仰華。俺答常遠處青山,二子,曰賓兔,居松山,直蘭州之北,曰丙兔,居西海,直河州之西,並求互市,多桀驁。俺答諭之,亦漸馴。

自是約束諸部無入犯,歲來貢市,西塞以寧。而東部土蠻數擁眾寇遼塞。總兵李成梁敗之於卓山,斬五百八十餘級,守備曹簠復敗之於長勝堡。神宗即位,頻年入犯。

萬曆六年,成梁率遊擊秦得倚等擊敵於東昌堡,斬部長九人,餘級八百八十四,總督梁夢龍以聞。帝大悅,祭告郊廟,御皇極門宣捷。

七年冬,土蠻四萬騎入錦川營。夢龍、成梁及總兵戚繼光等已預受大學士張居正方略,併力備禦,敵始退。自是敵數入,成梁等數敗之,輒斬其巨魁,又時襲擊於塞外,多所斬獲。敵畏之,少戢,成梁遂以功封寧遠伯。

俺答既就市,事朝廷甚謹。部下卒有掠奪邊氓者,必罰治之,且稽首謝罪,朝廷亦厚加賞賚。十年春,俺答死,帝特賜祭七壇、彩緞十二表裏、布百匹,示優恤。其妻哈屯率子黃台吉等,上表進馬謝,復賜幣布有差。封黃台吉為順義王,改名乞慶哈。立三歲而死,朝廷給恤典如例。

十五年春,子撦力克嗣。其妻三娘子,故俺答所奪之外孫女而為婦者也,歷配三王,主兵柄,為中國守邊保塞,眾畏服之,乃敕封為忠順夫人,自宣大至甘肅不用兵者二十年。及撦力克西行遠邊,而套部莊禿賴等據水塘,卜失兔、火落赤等據莽剌、捏工兩川,數犯甘、涼、洮、氓、西寧間。他部落亡慮數十種,出沒塞下,順逆不常。帝惡之,十九年詔並停撦力克市賞。已而撦力克叩邊輸服,率眾東歸,獨莊禿賴、卜失兔等寇抄如故。其年冬,別部明安、土昧分犯榆林邊,總兵杜桐禦之,斬獲五百人,殺明安。

二十年,寧夏叛將哱拜等勾卜失兔、莊禿賴等,大舉入寇,總兵李如松擊敗之。二十二年,延綏巡撫李春光奏:「套部納款已久,自明安被戮而寇恨深,西夏黨逆而貢市絕,延鎮連年多事。今東西各部皆乞款,而卜失兔挾私叵測,邊長兵寡,制禦為難。宜察敵情,審時勢。敵入犯則血戰,偶或小失,應寬吏議。倘敵真心效順,相機議撫,不可忘戰備也。」帝命兵部傳飭各邊。秋,卜失兔入固原,遊擊史見戰死。延綏總兵麻貴禦之,閱月始退。全陜震動。其年,東部炒花犯鎮武堡,總兵董一元與戰,大破之。明年春,松部宰僧等犯陜西,總督葉夢熊督卻之。秋,海部永邵卜犯西寧,總督三邊李旼檄參將達雲、遊擊白澤暨馬其撒、卜爾加諸屬番,設伏邀擊,大敗之,斬六百八十三級。捷聞,帝大悅,且以屬番效命,追敘前總制鄭雒功,賞賚並及雒。

二十四年春,總督李釐以勁兵分三道出塞,襲卜失兔營,共斬四百九級,獲馬畜器械數千。火落赤部眾復窺伺洮州,釐遣參將周國柱等擊之於莽剌川腦,斬一百三十六級。秋,著力兔、阿赤兔、火落赤等合謀犯西邊,炒花亦擁眾犯廣寧,守將皆嚴兵卻之。二十五年秋,海部寇甘鎮,官軍擊走之。冬,炒花糾土蠻諸部寇遼東,殺掠無算。明年夏,復寇遼東,總兵李如松遠出搗巢,死之。冬,釐等分道出襲火落赤等於松山,走之,復其地。

二十七年詔復撦力克市賞。時釐等築松山,諸部紛叛,延、寧守臣共擊之,殺獲甲首幾三千。明年,著力兔、宰僧、莊禿賴等乞通款,不許。邊臣王見賓等復為請,詔復套部貢市。

三十一年,海部數入陜西塞,兵備副使李自實,總兵蕭如薰、達雲等擊走之。三十三年夏,東部宰賽誘殺慶雲堡守禦熊鑰,詔革其市賞。

三十五年夏,總督徐三畏言:「河套之部與河東之部不同。東部事統於一,約誓定,歷三十年不變。套部分四十二枝,各相雄長,卜失兔徒建空名於上。西則火落赤最狡,要挾最無厭;中則擺言太以父明安之死,無歲不犯;東則沙計爭為監市,與炒花朋逞。西陲搶攘非一日矣。然眾雖號十萬,分為四十二枝,多者不過二三千騎,少者一二千騎耳。宜分其勢,納其款,俾先順者獲賞,後至者拒剿。仍須主戰以張國威。」時已許宰賽及火落赤諸部復貢市矣。

未幾撦力克死,未有嗣,忠順夫人率所部仍效貢職。西部銀定、歹青數擁眾犯東西邊。延綏部猛克什力亦以挾賞故,常沿邊抄掠。卜失兔欲婚於忠順,忠順拒之。其所部素囊臺吉、五路台吉等,各不相下,封號久未定。四十一年,卜失兔始婚於忠順,東、西諸部長皆具狀為請封。忠順夫人旋卒,詔封卜失兔為順義王,而以把漢比吉素效恭順,封忠義夫人。卜失兔為撦力克孫,襲封時,已少衰,所制止山、大二鎮外十二部。其部長五路、素囊及兀慎台吉等,兵力皆與順義埒。朝廷因宣大總督塗宗浚言,各予升賞如例。

其年,炒花糾虎墩兔三犯遼東。虎墩兔者,居插漢兒地,亦曰插漢兒王子,元裔也。其祖打來孫始駐牧宣塞外,俺答方強,懼為所併,乃徙帳於遼,收福餘雜部,數入掠薊西,四傳至虎墩兔,遂益盛。明年夏,炒花復合宰賽、煖兔以三萬騎入掠,至平虜、大寧。既求撫賞,許之。

四十二年,猛克什力寇懷遠及保寧。延綏總兵官秉忠等破之。斬二百二十一級。明年,插部數犯遼東。已,掠義州,攻陷大安堡,兵民死者甚眾。

四十四年,總兵杜文煥數破套部猛克什力等於延綏邊,火落赤、擺言太及吉能、切盡、歹青、沙計東西諸部皆懼,先後來請貢市。

四十六年,我大清兵起,略撫順及開原,插部乘隙擁眾挾賞。西部阿暈妻滿旦亦以萬騎自石塘路入掠薊鎮白馬關及高家、馮家諸堡。遊擊朱萬良禦之,被圍。羽書日數十至,中外戒嚴。頃之,滿旦亦叩關乞通貢。

四十七年,大清兵滅宰賽及北關金台什、布羊古等。金台什孫女為虎墩兔婦,於是薊遼總督文球、巡撫周永春等以利啖之,俾聯結炒花諸部,以捍大清兵,給白金四千。明年,為泰昌元年,加賞至四萬。虎乃揚言助中國,邀索無厭。

天啟元年秋,吉能犯延綏邊,榆林總兵杜文煥擊敗之。明年春,復大掠延安黃花峪,深入六百里,殺掠居民數萬。三年春,銀定糾眾再掠西邊,官軍擊敗之。明年春,復謀入故巢,犯松山,為守臣馮任等所敗。夏,遂糾海西古六台吉等犯甘肅,總兵董繼舒擊之,斬三百餘級。其年,歹青以領賞嘩於邊,邊人格殺之。歹青,虎墩兔近屬也,邊臣議歲給償命銀一萬三千有奇,而虎怏怏,益思颺去。未幾,大清兵襲破炒花,所部皆散亡,半歸於插漢。時卜失兔益衰,號令不行於諸部,部長干兒罵等歲數犯延綏諸邊。七慶台吉及敖目比吉、毛乞炭比吉等,亦各擁眾往來窺伺塞下。

崇禎元年,虎墩兔攻哈喇嗔及白言台吉、卜失兔諸部,皆破之,遂乘勝入犯宣大塞。秋,帝御平臺,召總督王象乾,詢以方略,象乾對言:「禦插之道,宜令其自相攻。今卜失兔西走套內,白台吉挺身免,而哈喇嗔所部多被擄,不足用。永邵卜最強,約三十萬人,合卜失兔所部並聯絡朵顏三十六家及哈喇嗔餘眾,可以禦插漢。然與其構之,水如撫而用之。」帝曰:「插漢意不受撫,奈何?」對曰:「當從容籠絡。」帝曰:「如不款何?」象乾復密奏,帝善之,命往與督師袁崇煥共計。象乾至邊,與崇煥議合,皆言西靖而東自寧,虎不款,而東西並急,因定歲予插金八萬一千兩,以示羈縻。大同巡撫張宗衡上言:「插來宣大,駐新城,去大同僅二百里,三閱月未敢近前,飢餓窮乏,插與我等耳。插恃撫金為命,兩年不得,資用已竭,食盡馬乏,暴骨成莽。插之望款不啻望歲,而我遺之金繒、牛羊、茶果、米穀無算,是我適中其欲也。插炰烋悖慢,耳目不忍睹聞,方急款尚如是。使插士馬豐飽,其憑陵狂逞,可勝道哉。」象乾言:「款局垂成而復棼之,既示插以不信,亦非所以為國謀。」疏入,帝是象乾議,詔宗衡毋得異同。

明年秋,虎復擁眾至延綏紅水灘,乞增賞未遂,即縱掠塞外,總兵吳自勉禦卻之。既而東附大清兵攻龍門。未幾,為大清兵所擊。六年夏,插漢聞大清兵至,盡驅部眾渡河遠遁。是時,韃靼諸部先後歸附於大清。明年,大清兵遂大會諸部於兀蘇河南岡,頒軍律焉。而虎已卒,乃追至上都城,盡俘插漢妻孥部眾。

其後,套部歲入寧夏、甘、涼境,巡撫陳奇瑜、總兵馬世龍、督師洪承疇等輒擊敗之。套部乾兒罵,亦為總兵尤世祿所斬。迄明世,邊陲無寧,致中原盜賊蜂起。當事者狃與俺答等貢市之便,見插之恣於東也,謂歲捐金錢數十萬,冀茍安旦夕,且覬收之為用,而卒不得。迨其後也,明未亡而插先斃,諸部皆折入於大清。國計愈困,邊事愈棘,朝議愈紛,明亦遂不可為矣。

韃靼地,東至兀良哈,西至瓦剌。當洪、永、宣世,國家全盛,頗受戎索,然畔服亦靡常。正統後,邊備廢弛,聲靈不振。諸部長多以雄傑之姿,恃其暴強,迭出與中夏抗。邊境之禍,遂與明終始云。


\end{pinyinscope}