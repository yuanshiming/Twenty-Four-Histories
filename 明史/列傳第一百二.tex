\article{列傳第一百二}

\begin{pinyinscope}
楊博子俊民馬森劉體乾王廷毛愷葛守禮靳學顏弟學曾

楊博,字惟約,蒲州人。父瞻,御史,終四川僉事。博登嘉靖八年進士,除盩啡知縣,調長安。徵為兵部武庫主事,歷職方郎中。大學士翟鑾巡九邊,以博自隨。所過山川形勢,土俗好惡,士卒多寡強弱,皆疏記之。至肅州,屬番數百遮道邀賞。鑾慮來者益眾,不能給。博請鑾盛儀衛,集諸番轅門外,數以天子宰相至,不悉眾遠迎,將縛以屬吏。諸番羅拜請罪,乃稍賚其先至者,餘皆懼不復來。鑾還,薦博可屬大事。吉囊、俺答歲盜邊,尚書張瓚一切倚辦博。帝或中夜降手詔,博隨事條答,悉稱旨。毛伯溫代瓚,博當遷,特奏留之。已,遷山東提學副使,轉督糧參政。

二十五年,超拜右僉都御史,巡撫甘肅。大興屯利,請募民墾田,永不征租。又以暇修築肅州榆樹泉及甘州平川境外大蘆泉諸處墩臺,鑿龍首諸渠。初,罕東屬番避土魯番亂,遷肅州境上,時與居民戕殺。監生李時暘以為言,事下守臣。博為築金塔、白城七堡,召其長,令率屬徙居之。諸番徙七百餘帳,州境為之肅清。總兵官王繼祖卻寇永昌,鎮羌參將蔡勳等戰鎮番、山丹,三告捷,斬首百四十餘級。進博右副都御史。以母憂歸。仇鸞鎮甘肅,總督曾銑劾之,詔逮治。博亦發其貪罔三十事。鸞拜大將軍,數毀之,帝不聽。服闋,鸞已誅,召拜兵部右侍郎。轉左,經略薊州、保定。

初,俺答薄都城,由潮河川入,議者爭請為備。水湍悍,不可城。博緣水勢建石墩,置戍守,還督京城九門。時因寇警,歲七月分兵守陴。博曰:「寇至,須鎮靜,奈何先事自擾?」罷其令。尋遷總督薊、遼、保定軍務。博以薊逼京師,護畿甸陵寢為大,分布諸將,畫地為防。三十三年秋,把都兒及打來孫十餘萬騎犯薊鎮,攻牆。帝憂甚,數遣騎偵博。博擐甲宿古北口城上,督總兵官周益昌等力禦。帝大喜,馳賜緋豸衣,犒軍萬金。寇攻四晝夜不得入,乃并攻孤山口,登牆。官軍斷一人腕,乃退屯虎頭山。博募死士,夜以火驚其營,寇擾亂,比明悉去。進右都御史,廕子錦衣千戶。明年,打來孫復入益昌,擊卻之。遂擢博兵部尚書,錄防秋功,加太子少保。

嚴嵩父子招權利,諸司為所撓,博一切格不行。嵩恨博,會丁父憂去。兵部尚書許論罷,帝起博代之。博未終喪,疏辭。而帝以大同右衛圍急,改博總督宣、大、山西軍務。博墨縗馳出關。未至,侍郎江東等以大軍進,寇引去。時右衛圍六月,守將王德戰亡,城中芻粟且盡,士死守無二心。博厚撫恤,奏行善後十事。以給事中張學顏言,留博鎮撫。奏蠲被寇租,因僉其丁壯為義勇,分隸諸將。博以邊人不習車戰,寇入輒不支,請造箱車百輛,有警則右衛車東,左衛車西,使相聲援。又以大同牆圮,繕治為急;次則塞銀釵、驛馬諸嶺,以絕窺紫荊路;備居庸南山,以絕窺陵寢畿甸路;修陽神地諸牆塹,以絕入山西路。乃於大同牛心山諸處築堡九,墩臺九十二,接左衛高山站,以達鎮城。浚大濠二,各十八里,小濠六十有四。五旬訖功,賜敕獎賚。

帝數欲召博還,又虞邊,以問嵩。嵩雅不喜博,請令江東署部事,俟秋防畢徐議之,遂不召。秋防訖,加太子太保,留鎮如故。哱素把伶及叛人了都記等數以輕騎寇邊,博先後計擒之。又數出奇兵襲寇,寇稍徙帳。因議築故總督翁萬達所創邊牆,招還內地民為寇掠者千六百餘人。又請通宣、大荒田水利,薄其租。報可。改薊遼總督。秋防竣,廷議欲召博還,吏部尚書吳鵬不可。鄭曉署兵部,爭之曰:「博在薊、遼則薊、遼安,在本兵則九邊俱安。」乃如還,加少保。

帝憂邊甚,博每先事為防,帝眷倚若左右手。嘗語閣臣:「自博入,朕每憂邊,其語博預為謀。」博上言:「今九邊,薊鎮為重。請敕邊臣逐大同寇,使不得近薊,宣、大諸將從獨石偵情形,預備黃花、古北諸要害,使一騎不得入關,即首功也。」帝是之。

四十二年十月,寇擁眾窺薊州,聲言犯遼陽。總督楊選帥師東,博檄止之。又手書三往,卒不從。博拊几曰:「敗矣。」急徵兵入援,寇已潰牆子嶺,犯通州。帝嘆曰:「庚戌事又見矣。」諸路兵先後至。命宣大總督江東統文武大臣分守皇城、京城,鎮遠侯顧寰以京營兵分布城內外。寇解而東,躪順義、三河,飽掠去。援兵不發一矢,取道斃及零騎傷殘者報首功。帝怏怏,諭博曰:「賊復飽颺,何以懲後?」遂誅選。博懼及,徐階力保持之。帝念博前功,不罪。久之,改吏部尚書。

隆慶改元,請遵遺詔,錄建言諸臣,死者皆贈恤。時方計群吏,山西人無一被黜者。給事中胡應嘉劾博庇其鄉人,博連疏乞休。並慰留,且斥言者。一品滿三考,進少傅兼太子太傅。帝將遊南海子,博率同列諫。御史詹仰庇以直言罷,博爭之。屯鹽都御史龐尚鵬被論,博議留。懺旨,遂謝病歸。尚書劉體乾等交章乞留,不聽。大學士高拱掌吏部,薦博堪本兵。詔以吏部尚書理兵部事。陳薊、昌戰守方略,謂:「議者以守墻為怯,言可聽,實無少效。牆外邀擊,害七利三;牆內格鬥,利一害九。夫因牆守,所謂先處戰地而待敵。名守,實戰也。臣為總督,嘗拒打來孫十萬眾,以為當守牆無疑。」因陳明應援、申駐守、處京營、諭屬夷、修內治諸事,帝悉從之。

博魁梧豐碩,臨事安閒有識量。出入中外四十餘年,始終以兵事著。六年,高拱罷,乃改博吏部,進少師兼太子太師。明年秋,疾作,三疏乞致仕歸。逾年卒。贈太傅,謚襄毅。

拱柄國時,欲中徐階危禍,博造拱,力為解。拱亦心動,事獲已。其後張居正逐拱,將周內其罪,博毅然爭之。及興王大臣獄,博與都御史葛守禮詣居正力為解。居正憤曰:「二公謂我甘心高公耶?」博曰:「非敢然也,然非公不能回天。」會帝命守禮偕都督朱希孝會訊,博陰為畫計,使校尉怵大臣改供;又令拱僕雜稠人中,令大臣識別,茫然莫辨,事乃白。人以是稱博長者。

子俊民,字伯章,嘉靖四十一年進士。除戶部主事,歷禮部郎中。隆慶初,遷河南提學副使。萬曆初,歷太僕少卿。父博致政,侍歸。起故官,累遷兵部左侍郎署部事。時議撦力克嗣封。俊民言:「款未可遽罷。惟內修守備,而外勒西部,使盡還巢,申定市額,使無濫索而已。」議遂定。進戶部尚書,總督倉場。十九年,還理部事。河南大饑,人相食,請發銀米各數十萬。或議其稽緩,因自劾求罷。疏六上,不允。小人競請開礦,俊民爭不得,稅使乃四出。天下騷然,時以咎俊民。在事歷三考,累加太子太保。卒官,贈少保。後敘東征轉餉功,贈少傅兼太子太傅。

馬森,字孔養,懷安人。父俊,晚得子,家人抱之墜,殞焉。俊紿其妻曰「我誤也」,不之罪。踰年而舉森。嘉靖十四年成進士,授戶部主事,歷太平知府。民有兄弟訟者,予鏡令照曰:「若二人老矣,忍傷天性乎?」皆感泣謝去。再遷江西按察使。有進士嬖外婦而殺妻,撫按欲緩其獄,森卒抵之法。

歷左布政使,就擢巡撫右副都御史。入為刑部右侍郎,改戶部。初,森在江西薦布政使宋淳。淳後撫南、贛,以贓敗,森坐調大理卿。屢駁疑獄,與刑部尚書鄭曉、都御史周延稱為「三平」。病歸,起南京工部右侍郎。改戶部,督倉場,尋轉左。以右都御史總督漕運,兼巡撫鳳陽,遷南京戶部尚書。隆慶初,改北部。

是時,登極詔書蠲天下田租半。太倉歲入少,不能副經費,而京、通二倉積貯無幾。森鉤校搜剔,條行十餘事。又列上錢穀出入之數,勸帝節儉。帝手詔責令措置,森奏:「祖宗舊制,河、淮以南以四百萬供應京師,河、淮以北以八百萬供邊。一歲之入,足供一歲之用。後邊陲多事,支費漸繁,一變而有客兵之年例,再變而有主兵之年例。其初止三五十萬耳,後漸增至二百三十餘萬。屯田十虧七八,鹽法十折四五,民運十逋二三,悉以年例補之。在邊則士馬不多於昔,在太倉則輸入不益於前,而所費數倍。重以詔書蠲除,故今日告匱,視往歲有加。臣前所區畫,算及錙銖,不過紓目前急,而於國之大體,民之元氣,未暇深慮。願廣集眾思,令廷臣各陳所見。」又奏河東、四川、雲南、福建、廣東、靈州鹽課事宜。詔皆如所請。帝嘗命中官崔敏發戶部銀六萬市黃金。森持不可,且言,故事御札皆由內閣下,無司禮徑傳者,事乃止。即,又命購珠寶,森亦力爭,不聽。三年,以母老乞終養。賜馳驛歸,後屢薦不起。

森為考官時,夏言婿出其門,欲介之見言,謝不往。嚴嵩聞而悅之,森亦不附。為徐階所重,遂引用之。里居,贊巡撫龐尚鵬行一條鞭法,鄉人為立報功祠。萬曆八年卒。贈太子少保,謚恭敏。

劉體乾,字子元,東安人。嘉靖二十三年進士。授行人,改兵科給事中。司禮太監鮑忠卒,其黨李慶為其侄鮑恩等八人乞遷。帝已許之,以體乾言,止錄三人。轉左給事中。

帝以財用絀,詔廷臣集議。多請追宿逋,增賦額。體乾獨上奏曰:「蘇軾有言:『豐財之道,惟在去其害財者。』今之害最大者有二,冗吏、冗費是也。歷代官制,漢七千五百員,唐萬八千員,宋極冗至三萬四千員。本朝自成化五年,武職已逾八萬。合文職,蓋十萬餘。今邊功升授、勛貴傳請、曹局添設、大臣恩廕,加以廠衛、監局、勇士、匠人之屬,歲增月益,不可悉舉。多一官,則多一官之費。請嚴敕請曹,清革冗濫,減俸將不貲。又聞光錄庫金,自嘉靖改元至十五年,積至八十萬。自二十一年以後,供億日增,餘藏頓盡。進御果蔬,初無定額,止眎內監片紙,如數供御。乾沒狼籍,輒轉鬻市人。其他諸曹,侵盜尤多。宜著為令典,歲終使科道臣會計之,以清冗費。二冗既革,國計自裕。舍是而督逋、增賦,是揚湯止沸也。」於是部議請汰各監局人匠。從之。

累官通政使,遷刑部左侍郎。改戶部左侍郎,總督倉場。隆慶初,進南京戶部尚書。南畿、湖廣、江西銀布絹米積逋二百六十餘萬,鳳陽園陵九衛官軍四萬,而倉粟無一月儲。體乾再疏請責成有司,又條上六事,皆報可。

馬森去,召改北部。詔取太倉銀三十萬兩。體乾言:「太倉銀所存三百七十萬耳,而九邊年例二百七十六萬有奇,在京軍糧商價百有餘萬薊州、大同諸鎮例外奏乞不與焉。若復取以上供,經費安辦?」帝不聽。體乾復奏:「今國計絀乏,大小臣工所共知。即存庫之數,乃近遣御史所搜括,明歲則無策矣。今盡以供無益費,萬一變起倉卒,如國計何?」於是給事中李已、楊一魁、龍光,御史劉思問、蘇士潤、賀一桂,傅孟春交章乞如體乾言,閣臣李春芳等皆上疏請,乃命止進十萬兩。又奏太和山香稅宜如泰山例,有司董之,毋屬內臣。忤旨,奪俸半年。

帝嘗問九邊軍餉,太倉歲發及四方解納之數。體乾奏:「祖宗朝止遼東、大同、宣府、延綏四鎮,繼以寧夏、甘肅、薊州,又繼以固原、山西,今密雲、昌平、永平、易州俱列戍矣。各鎮防守有主兵。其後增召募,增客兵,而坐食愈眾。各鎮芻餉有屯田。其後加民糧,加鹽課,加京運,而橫費滋多。」因列上隆慶以來歲發之數。又奏:「國家歲入不足供所出,而額外陳乞者多。請以內外一切經費應存革者,刊勒成書。」報可。

詔市綿二萬五千斤,體乾請俟湖州貢。帝不從,趣之急。給事中李已言:「三月非用綿時,不宜重擾商戶。」體乾亦復爭,乃命止進萬斤。踰年,詔趣進金花銀,且購貓睛、祖母綠諸異寶。已上書力諫,體乾請從已言,不納。內承運庫以白答刂索部帑十萬。體乾執奏,給事中劉繼文亦言白答刂非體。帝報有旨,竟取之。體乾又乞承運庫減稅額二十萬,為中官崔敏所格,不得請。是時內供已多,數下部取太倉銀,又趣市珍珠黃綠玉諸物。體乾清勁有執,每疏爭,積忤帝意,竟奪官。給事中光懋、御史凌琯等交章請留,不聽。

神宗即位,起南京兵部尚書,奏言:「留都根本重地,故額軍九萬,馬五千餘匹。今軍止二萬二千,馬僅及半,單弱足慮。宜選諸衛餘丁,隨伍操練,發貯庫草場銀買馬。」又條上防守四事。並從之。萬歷二年致仕,卒。贈太子少保。

王廷,字子正,南充人。嘉靖十一年進士。授戶部主事,改御史。疏劾吏部尚書汪鋐,謫亳州判官。歷蘇州知府,有政聲。累遷右副都御史,總理河道。三十九年,轉南京戶部右侍郎,總督糧儲。南京督儲,自成化後皆以都御史領之,至嘉靖二十六年,始命戶部侍郎兼理。及振武營軍亂,言者請復舊制,遂以副都御史章煥專領,而改廷南京刑部。未上,復改戶部右侍郎兼左僉都御史,總督漕運,巡撫鳳陽諸府。

時倭亂未靖,廷建議以江南屬鎮守總兵官,專駐吳淞,江北屬分守副總兵,專駐狼山。遂為定制。淮安大饑,與巡按御史朱綱奏留商稅餉軍,被詔切讓。給事中李邦義因劾廷拘滯,吏部尚書嚴訥為廷辨,始解。轉左侍郎,還理部事。以通州禦倭功,加俸二級。遷南京禮部尚書,召為左都御史。奏行慎選授、重分巡、謹刑獄、端表率、嚴檢束、公舉劾六事。

隆慶元年六月,京師雨潦壞廬舍,命廷督御史分行振恤。會朝覲天下官,廷請嚴禁饋遺,酌道里費,以儆官邪,蘇民力。帝謁諸陵,詔廷同英國公張溶居守。中官許義挾刃脅人財,為巡城御史李學道所笞。群璫伺學道早朝,邀擊之左掖門外。廷上其狀,論戍有差。

御史齊康為高拱劾徐階,廷言:「康懷奸黨邪,不重懲無以定國是。」帝為謫康,諭留階。拱遂引疾去。而給事中張齊者,嘗行邊,受賈人金。事稍泄,陰求階子璠居間,璠謝不見。齊恨,遂摭康疏語復論階,階亦引疾去。廷因發齊奸利事,言:「齊前奉命賞軍宣大,納鹽商楊四和數千金,為言恤邊商、革餘鹽數事,為大學士階所格。四和抵齊取賄,蹤跡頗露。齊懼得罪,乃借攻階冀自掩。」遂下齊詔獄。刑部尚書毛愷當齊戍,詔釋為民。拱起再相,廷恐其修郤,而愷亦階所引,遂先後乞休以避之。給事中周芸、御史李純樸訟齊事,謂廷、愷阿階意,羅織不辜。刑部尚書劉自強覆奏:「齊所坐無實,廷、愷屈法徇私。」詔奪愷職,廷斥為民,宥齊,補通州判官。

萬曆初,齊以不謹罷,愷已前卒。浙江巡按御史謝廷傑訟愷狷潔有古人風,坐按張齊奪官,今齊已黜,足知愷守正。詔復愷官。於是巡撫四川都御史曾省吾言:「廷守蘇州時,人比之趙清獻。直節勁氣,始終無改。宜如毛愷例復官。」詔以故官致仕。十六年,給夫廩如制,仍以高年特賜存問。明年卒,謚恭節。

毛愷,字達和,江山人。嘉靖十四年進士。授行人,擢御史。坐論洗馬鄒守益不當投散地,為執政所惡,謫寧國推官。歷刑部尚書。太監李芳驟諫忤穆宗,命刑部置重辟。愷奏:「芳罪狀未明,非所以示天下公。」芳仍得貰死。愷贈太子少保,謚端簡。

葛守禮,字與立,德平人。嘉靖七年,舉鄉試第一。明年成進士,授彰德推官。巨盜誣富家,株連以百數,守禮盡出之。主獄者譖之御史。會籓府獄久不決,屬守禮,一訊即得,乃大驚服。冬至,趙王戒百官朝服賀,守禮獨不可。遷兵部主事。父喪服闋,補禮部。寧府宗人悉錮高牆,後稍得脫,因請封。禮部尚書夏言議量復中尉數人。未上,而言入閣,嚴嵩代之。守禮適遷儀制郎中,駁不行。故事,郡王絕,近支得以本爵理府事,不得繼封。交城、懷仁、襄垣近支絕,以繼封請,守禮持之堅。會以疾在告,三邸人乘間行賂,遂得請。旗校詗其事以聞。所籍記賂遺十餘萬,獨無守禮名,帝由是知守禮廉。遷河南提學副使,再遷山西按察使,進陜西布政使,擢右副都御史,巡撫河南。入為戶部侍郎,督餉宣、大。改吏部。自左侍郎遷南京禮部尚書。李本署吏部事,希嚴嵩指考察廷臣,署守禮下考,勒致仕。後帝問守禮安在,左右謬以老病對。帝為歎惜久之。

隆慶元年,起戶部尚書。奏言:「畿輔、山東流移日眾,以有司變法亂常,起科太重,徵派不均。且河南北,山東西,土地磽瘠,正供尚不能給,復重之徭役。工匠及富商大賈,皆以無田免役,而農夫獨受其困,此所謂舛也。乞正田賦之規,罷科差之法。又國初徵糧,戶部定倉庫名目及石數價值,通行所司,分派小民,隨倉上納,完欠之數了然可稽。近乃定為一條鞭法,計畝征銀。不論倉口,不問石數。吏書夤緣為奸,增減灑派,弊端百出。至於收解,乃又變為一串鈴法,謂之夥收分解。收者不解,解者不收,收者獲積餘之貲,解者任賠補之累。夫錢穀必分數明而後稽核審,今混而為一,是為那移者地也。願敕所司,酌復舊規。」詔悉舉行。於是奏定國計簿式,頒行天下。自嘉靖三十六年以後,完欠、起解、追徵之數及貧民不能輸納,備錄簿中。自府州縣達布政,送戶部稽考,以清隱漏那移侵欺之弊。又以戶部專理財賦,必周知天下倉庫盈虛,然後可節縮調劑。祖宗時令天下歲以文冊報部,乃請遣御史譚啟、馬明謨、張問明、趙巖分行天下董其事,並承敕以行。覃恩例嘗邊軍,或言士伍虛冒,宜乘給賞汰之。守禮言:「此朝廷曠典,乃以賈怨耶?」議乃止。

大學士高拱與徐階不相能,舉朝攻拱。侍郎徐養正、劉自強,拱所厚,亦詣守禮言。守禮不可,養正等遂論拱。守禮尋乞養母歸。及拱再相,深德守禮,起為刑部尚書。初,階定方士王金等獄,坐妄進藥物,比子殺父律論死。詔下法司會訊。守禮等議金妄進藥無事實,但習故陶仲文術,左道惑眾,應坐為從律編戍。給事中趙奮言:「法司為天下平。昔則一主於入,而不為先帝地;今則一主於出,而不恤後世議。罪有首而後有從,金等為從,孰為首?將以陶仲文為首,則仲文死已久。為法如此,陛下何賴哉!」疏入,報聞。

尋改守禮左都御史。奏言:「畿內地勢窪下,河道堙塞,遇潦則千里為壑。請仿古井田之制,浚治溝洫,使旱潦有備。」章下有司。又申明巡撫事宜,條列官箴、士節六事。守禮議王金獄,與拱合,然不附拱。後張居正欲以王大臣事手冓殺拱,守禮力為解,乃免。階、拱、居正更用事,交相軋。守禮周旋其間,正色獨立,人以為難。萬曆三年,以老乞休。詔加太子少保,馳驛歸。六年卒。贈太子太保,謚端肅。

靳學顏,字子愚,濟寧人。嘉靖十三年舉鄉試第一。明年成進士,授南陽推官,以廉平稱。歷吉安知府,治行高,累遷左布政使。隆慶初,入為太僕卿,改光祿。旋拜右副都御史,巡撫山西。應詔陳理財,凡萬餘言。言選兵、鑄錢、積穀最切。其略曰:

宋初禁軍十萬,總天下諸路亦不過十萬,其後慶曆、治平間增至百餘萬。然其時財用不絀。我朝邊兵四十萬。其後雖增兵益戍,而主兵多缺,不若宋人十倍其初也。然自嘉靖中即以絀乏告,何哉?宋雖增兵,而天下無養兵費。我朝以民養兵,而新軍又一切仰太倉。舊餉不減,新餉日增,費一也。周豐鎬、漢四都,率有其名而無實。我朝留都之設,建官置衛,坐食公帑,費二也。唐、宋宗親或通名仕版,或散處民間。我朝分封列爵,不農不仕,吸民膏髓,費三也。有此三者,儲畜安得不匱。而其間尤耗天下之財者,兵而已。夫陷鋒摧堅,旗鼓相當,兵之實也。今邊兵有戰時,若腹兵則終世不一當敵。每盜賊竊發,非陰陽、醫藥、雜職,則丞貳判簿為之將;非鄉民里保,則義勇快壯為之兵。在北則借鹽丁礦徒,在南則借狼土。此皆腹兵不足用之驗也。當限以輪番守戍之法。或遠不可徵,或弱不可任,則聽其耕商,而移其食以餉邊。如免班軍而征償,省充發而輸贖,亦變通一策也。慾京兵強,亦宜責以輪番戍守。夫京師去宣府、薊鎮纔數百里,京營九萬卒,歲以一萬戍二鎮,九年而一周,未為苦也,而怯者與邊兵同其勁矣。又以畿輔之卒填京戍之闕,其部伍、號令、月糧、犒賞亦與京卒同,而畿輔之卒皆親兵矣。夫京卒戍薊鎮,則延、固之費可省。戍宣府,則宣府、大同之氣自張。寇畏宣、大之力制其後,京卒之勁當其前,則仰攻深入之事鮮矣。

臣又睹天下之民皇皇以匱乏為慮者,非布帛五穀不足也,銀不足耳。夫銀,寒不可衣,饑不可食,不過貿遷以通衣食之用,獨奈何用銀而廢錢?錢益廢,銀益獨行。獨行則藏益深,而銀益貴,貨益賤,而折色之辦益難。豪右乘其賤收之,時其貴出之。銀積於豪右者愈厚,行於天下者愈少。更踰數十年,臣不知所底止矣。錢者,泉也,不可一日無。計者謂錢法之難有二:利不鸑本,民不願行。此皆非也。夫朝廷以山海之產為材,以億兆之力為工,以賢士大夫為役,何本之費?誠令民以銅炭贖罪,而匠役則取之營軍,一指麾間,錢遍天下矣。至不顧行錢者,獨奸豪爾。請自今事例、罰贖、征稅、賜賚、宗祿、官俸、軍餉之屬,悉銀錢兼支。上以是征,下以是輸,何患其不行哉。

臣又聞中原者,邊鄙之根本也。百姓者,中原之根本也,民有終身無銀,而不能終歲無衣,終日無食。今有司夙夜不遑者,乃在銀而不在穀,臣竊慮之。國家建都幽燕,北無郡國之衛,所恃為腹心股肱者,河南、山東、江北及畿內八府之人心耳。其人率鷙悍而輕生,易動而難戢,游食而寡積者也。一不如意,則輕去其鄉;往往一夫作難,千人響應,前事已屢驗矣。弭之之計,不過曰恤農以繫其家,足食以繫其身,聚骨肉以繫其心。今試核官廩之所藏,每府得數十萬,則司計者安枕可矣。得三萬焉,猶足塞轉徙者之望。設不滿萬,豈得無寒心?臣竊意不滿萬者多也。

臣近者疏請積穀,業蒙允行。第恐有司從事不力,無以塞明詔。敢即臣說申言之:

其一曰官倉,發官銀以糴也。一曰社倉,收民穀以充也。官倉非甚豐歲不能舉,社倉雖中歲皆可行。唐義倉之開,每歲自王公以下皆有入。宋則準民間正稅之數,取二十分之一以為社。誠仿而推之,就土俗,合人情,占歲候以通其變,計每歲二倉之入以驗其功,著為令,而歲歲修之,時其豐歉而斂散之。在官倉者,民有大饑則以振。在民倉者,雖官有大役亦不聽貸。借此藏富於民,即藏富於國也。今言財用者,不憂穀之不足,而憂銀之不足。夫銀實生亂,穀實弭亂。銀之不足,而泉貨代之;五穀不足,則孰可以代者哉?故曰明君不寶金玉,而寶五穀,伏惟聖明垂意。

疏入,下所司議,卒不能盡行也。

尋召為工部右侍郎,改吏部,進左侍郎。學顏內行修潔,見高拱以首輔掌銓,專恣甚,遂謝病歸,卒。弟學曾,山西副使。治績亦有聞。

贊曰:明之中葉,邊防墮,經費乏。當時任事之臣,能留意於此者鮮矣。若楊博、馬森、劉體乾、葛守禮、靳學顏之屬,庶幾負經濟之略者。就其設施與其所建白,究而行之,亦補苴一時而已,況言之不盡行,行之不能久乎!自時厥後,張居正始一整飭。居正歿,一切以空言從事,以迄於亡。蓋其壞非朝夕之積矣。


\end{pinyinscope}