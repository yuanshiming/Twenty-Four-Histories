\article{莊烈帝本紀}

\begin{pinyinscope}
莊烈愍皇帝,諱由檢,光宗第五子也,萬曆三十八年十二月生。母賢妃劉氏,早薨。天啟二年,封信王。六年十一月,出居信邸。

明年八月,熹宗疾大漸,召王入,受遺命。丁巳,即皇帝位。大赦天下,以明年為崇禎元年。九月甲申,追謚生母賢妃曰孝純皇后。丁亥,停刑。庚寅,冊妃周氏為皇后。冬十月甲午朔,享太廟。癸丑,南京地震。十一月甲子,安置魏忠賢於鳳陽。戊辰,撤各邊鎮守內臣。己巳,魏忠賢縊死。癸酉,免天啟時逮死諸臣贓,釋其家屬。癸巳,黃立極致仕。十二月,前南京吏部侍郎錢龍錫、禮部侍郎李標、禮部尚書來宗道、吏部侍郎楊景辰、禮部侍郎周道登、少詹事劉鴻訓俱禮部尚書兼東閣大學士,預機務。魏良卿、客氏子侯國興俱伏誅。

崇禎元年春正月辛巳,詔內臣非奉命不得出禁門。壬午,尊熹宗后為懿安皇后。丙戌,戮魏忠賢及其黨崔呈秀尸。二月乙未,禁章奏冗蔓。癸丑,御經筵。丁巳,戒廷臣交結內侍。三月己巳,葬悊皇帝於德陵。癸未,施鳳來、張瑞圖致仕。乙酉,贈恤冤陷諸臣。夏四月癸巳,賜劉若宰等進士及第、出身有差。甲午,袁崇煥為兵部尚書,督師薊、遼。庚戌,指揮卓銘請開礦,不許。五月己巳,李國普致仕。庚午,毀《三朝要典》。甲戌,裁各部添注官,辛巳,禱雨。乙酉,復外吏久任及舉保連坐之法,禁有司私派。六月,削魏忠賢黨馮銓、魏廣微籍。壬寅,許顯純伏誅。壬子,來宗道、楊景辰致仕。秋七月癸酉,召對廷臣及袁崇煥於平臺。壬午,浙江風雨,海溢,漂沒數萬人。癸未,海寇鄭芝龍降。甲申,寧遠兵變,巡撫都御史畢自肅自殺。八月乙未,詔非盛暑祁寒,日御文華殿與輔臣議政。九月丁卯,京師地震。冬十月戊戌,劉鴻訓罷,尋遣戍。十一月癸未,祀天於南郊。十二月丙申,韓爌復入閣。是年,革廣寧及薊鎮塞外諸部賞。諸部饑,告糴,不許。陜西饑民苦加派,流賊大起,分掠鄜州、延安。

二年春正月丙子,釋奠於先師孔子。丁丑,定逆案,自崔呈秀以下凡六等。二月戊子,祀社稷。庚寅,皇長子慈烺生,赦天下。三月戊寅,薊州兵變,有司撫定之。夏四月甲午,裁驛站。閏月癸亥,流賊犯三水,遊擊高從龍戰歿。癸未,祀地於北郊。五月乙酉朔,日有食之。庚子,議改曆法。六月戊午,袁崇煥殺毛文龍於雙島。癸亥,以久旱,齋居文華殿,敕群臣修省。秋八月甲子,總兵官侯良柱、兵備副使劉可訓擊斬奢崇明、安邦彥於紅土川,水西賊平。甲戌,熹宗神主祔太廟。九月丁未,楊鎬棄市。冬十月戊寅,大清兵入大安口。十一月壬午朔,京師戒嚴。乙酉,山海關總兵官趙率教戰沒於遵化。甲申,大清兵入遵化,巡撫都御史王元雅、推官何天球等死之。丁亥,總兵官滿桂入援。己丑,吏部侍郎成基命為禮部尚書兼東閣大學士,預機務。召前大學士孫承宗為兵部尚書中極殿大學士,視師通州。辛卯,袁崇煥入援,次薊州。戊子,宣、大、保定兵相繼入援。徵天下鎮巡官勤王。辛丑,大清兵薄德勝門。甲辰,召袁崇煥等於平臺,崇煥請入城休兵,不許。下兵部尚書王洽於獄。十二月辛亥朔,再召袁崇煥於平臺,下錦衣衛獄。甲寅,總兵官祖大壽兵潰,東出關。乙卯,孫承宗移駐山海關。庚申,諭廷臣進馬。丁卯,遣中官趨滿桂出戰,桂及前總兵官孫祖壽俱戰歿。總兵官馬世龍總理援軍。壬申,錢龍錫罷。癸酉,山西援兵潰於良鄉。丁丑,禮部侍郎周延儒、尚書何如寵、侍郎錢象坤俱禮部尚書兼東閣大學士,預機務。

三年春正月甲申,大清兵克永平,副使鄭國昌、知府張鳳奇等死之。丙戌,瘞城外戰士骸。戊子,大清兵克灤州。庚寅,逮總督薊遼都御史劉策下獄,論死。乙未,禁抄傳邊報。韓爌致仕。壬寅,兵部右侍郎劉之綸敗沒於遵化。是月,陜西諸路總兵官吳自勉等帥師入衛,延綏、甘肅兵潰西去,與群寇合。二月庚申,立皇長子慈烺為皇太子,大赦。三月壬午,李標致仕。戊申,流賊犯山西。夏四月乙卯,以久旱,齋居文華殿,諭百官修省。丁丑,流賊陷蒲縣。五月辛卯,馬世龍、祖大壽諸軍入灤州。壬辰,大清兵東歸,永平、遷安、遵化相繼復。六月癸丑,流賊王嘉胤陷府谷,米脂賊張獻忠聚眾應之。己未,授宋儒邵雍後裔《五經》博士。辛酉,禮部尚書溫體仁、吳宗達並兼東閣大學士,預機務。秋八月癸亥,殺袁崇煥。九月己卯,逮錢龍錫下獄。冬十月癸亥,停刑。丙寅,巡撫延綏副都御史洪承疇、總兵官杜文煥敗賊張獻忠於清澗。十一月壬辰,破賊於懷寧。甲午,山西總兵官王國梁追賊於河曲,敗績。十二月乙巳朔,增田賦充餉。戊午,流賊陷寧塞。是年,烏斯藏入貢。

四年春正月己卯,流賊陷保安。丁酉,御史吳甡振延綏饑民。己亥,召對內閣、九卿、科道及入覲兩司官於文華殿。命都察院嚴核巡按御史。二月壬子,流賊圍慶陽,分兵陷合水。三月丁丑,副將張應昌等擊敗之,慶陽圍解。癸未,總督陜西三邊軍務侍郎楊鶴招撫流賊於寧州,群賊偽降,尋復叛。己丑,賜陳於泰等進士及第、出身有差。夏四月庚戌,禱雨。辛酉,詔廷臣條時政。是月,延綏副將曹文詔擊賊於河曲,王嘉胤敗死。五月甲戌朔,步禱於南郊。庚辰,戍錢龍錫。六月丁未,錢象坤致仕。秋七月甲戌,總兵官王際恩敗賊於鄜州,降賊首上天龍。八月癸卯,總兵官賀虎臣擊斬賊劉六於慶陽。丁未,大清兵圍祖大壽於大凌城。丙辰,何如寵致仕。九月庚辰,內臣王應朝、鄧希詔等監視關、寧、薊鎮兵糧及各邊撫賞。甲午,逮楊鶴下獄,論戍。洪承疇總督三邊軍務。丁酉,太監張彞憲總理戶、工二部錢糧,給事中宋可久等相繼諫,不聽。戊戌,山海總兵官宋偉等援大凌,敗於長山,監軍太僕少卿張春被執。冬十月辛丑朔,日有食之。戊辰,祖大壽殺副將何可綱。己巳,大壽自大凌脫歸,入錦州。十一月丙戌,太監李奇茂監視陜西茶馬,吳直監視登島兵糧、海禁,群臣合疏諫,不聽。壬辰,孫承宗致仕。癸巳,召對廷臣於文華殿,歷詢軍國諸務。語及內臣,帝曰:「諸臣若實心任事,朕亦何需此輩。」己亥,流賊羅汝才犯山西。閏月乙丑,陜西降賊復叛,陷甘泉,殺參政張允登。丁卯,登州遊擊孔有德率師援遼,次吳橋反,陷陵縣,連陷臨邑、商河、齊東,屠新城。十二月丙子,濟南官軍禦賊於阮城店,敗績。丁丑,以大凌築城招釁奪孫承宗官。是冬,延安、慶陽大雪,民饑,盜賊益熾。

五年春正月辛丑,孔有德陷登州,遊擊陳良謨戰死,總兵官張可大死之。巡撫都御史孫元化、副使宋光蘭等被執,尋縱還。辛亥,孔有德陷黃縣。丙寅,總兵官楊御蕃、王洪率師討孔有德,敗績於新城鎮。二月己巳朔,孔有德圍萊州,巡撫都御史徐從治固守。辛巳,孔有德陷平度。三月壬寅,兵部侍郎劉宇烈督理山東軍務,討孔有德。夏四月甲戌,劉宇烈敗績於沙河。癸未,徐從治中傷卒。是月,總兵官曹文詔、楊嘉謨連破賊於隴安、靜寧,賊奔水落城,平涼、莊浪饑民附之,勢復熾。五月丙午,參政朱大典為僉都御史,巡撫山東。辛亥,禮部尚書鄭以偉、徐光啟並兼東閣大學士,預機務。六月,京師大雨水。壬申,河決孟津。秋七月辛丑,太監曹化淳提督京營戎政。癸卯,孔有德偽降,誘執登萊巡撫都御史謝璉,萊州知府朱萬年死之。己未,孫元化棄市。逮劉宇烈下獄,論戍。八月甲戌,洪承疇敗賊於甘泉,賊首白廣恩降。甲申,朱大典督軍救萊州,前鋒參將祖寬敗賊於沙河。乙酉,萊州圍解。癸巳,官軍大敗孔有德於黃縣,進圍登州。九月丁酉,海賊劉香寇福建。是秋,陜西賊入山西,連陷大寧、澤州、壽陽,分部走河北,犯懷慶,陷修武。冬十一月戊戌,劉香寇浙江。

六年春正月癸卯,曹文詔節制山、陜諸將討賊。丁未,副將左良玉破賊於涉縣,賊走林縣山中,饑民爭附之。庚申,遣使分督直省逋賦。是月,曹文詔擊山西賊,屢敗之。二月壬申,削左副都御史王志道籍。癸酉,流賊犯畿南。戊子,總兵官陳洪範等克登州水城。辛卯,孔有德遁入海,山東平。三月癸巳,敕曹文詔諸將限三月平賊。夏四月己巳,免延安、慶陽、平涼新舊遼餉。壬申,總兵官鄧巳、左良玉剿河南賊。五月乙巳,太監陳大金等分監曹文詔、張應昌、左良玉、鄧巳軍。壬子,孔有德及其黨耿仲明等航海降於我大清。癸丑,河套部犯寧夏,總兵官賀虎臣戰沒。六月辛酉朔,太監高起潛監視寧、錦兵餉。乙丑,鄭以偉卒。庚辰,周延儒致仕。甲申,延綏副將李卑援剿河南。庚寅,太監張彞憲請催逋賦一千七百餘萬,給事中范淑泰諫,不聽。秋七月甲辰,大清兵取旅順,總兵官黃龍死之。癸丑,改曹文詔鎮大同,山西巡撫都御史許鼎臣請留文詔剿賊,不許。八月己巳,曹文詔敗賊於濟源,又敗之於懷慶。九月庚戌,南京禮部侍郎錢士升為禮部尚書兼東閣大學士,預機務。冬十月戊辰,徐光啟卒。十一月癸巳,禮部侍郎王應熊、何吾騶俱禮部尚書兼東閣大學士,預機務。辛亥,詔保定、河南、山西會兵剿賊。壬子,賊渡河。乙卯,陷澠池。十二月,連陷伊陽、盧氏,分犯南陽、汝寧,遂逼湖廣。是年,安南入貢。

七年春正月己丑,廣鹿島副將尚可喜降於我大清。設河南、山、陜、川、湖五省總督,以延綏巡撫陳奇瑜兼兵部侍郎為之。庚寅,總兵官張應昌渡河,敗賊於靈寶。壬辰,賊自鄖陽渡漢。癸巳,犯襄陽,連陷紫陽、平利、白河,南入四川。二月戊寅,陷夔州,大寧諸縣皆失守。甲申,耕耤田。乙酉,張獻忠突商、雒,凡十三營流入漢南。是月,振登、萊饑,蠲逋賦。三月丁亥朔,日有食之。甲辰,賜劉理順等進士及第、出身有差。乙巳,張應昌擊賊於五嶺山,敗績。庚戌,賊自四川走湖廣,副將楊世恩追敗之於石河口。山西自去年不雨至於是月,民大饑。夏四月,賊自湖廣走盧氏、靈寶。癸酉,發帑振陜西、山西饑。五月丙申,副將賀人龍等敗賊於藍田。六月辛未,總督侍郎陳奇瑜、鄖陽撫治都御史盧象昇會師於上津,剿湖廣賊。甲戌,河決沛縣。是夏,官軍圍高迎祥、李自成諸賊於興安之車箱峽兩月。賊食盡,偽降。陳奇瑜受之,縱出險。復叛,陷所過州縣。張應昌自清水追賊,敗績。秋七月壬辰,大清兵入上方堡,至宣府。乙未,詔總兵官陳洪範守居庸,巡撫保定都御史丁魁楚等守紫荊、雁門。辛丑,京師戒嚴。庚戌,大清兵克保安,沿邊諸城堡多不守。八月,分遣總兵官尤世威等援邊。戊辰,宣大總督侍郎張宗衡節制各鎮援兵。閏月甲申,賊陷隆德、固原,參議陸夢龍赴援,敗沒。丁亥,大清兵克萬全左衛。庚寅,旋師出塞。壬寅,李自成圍賀人龍於隴州。九月庚申,盔甲廠災。庚辰,洪承疇解隴州圍。甲戌,以賊聚陜西,詔河南兵入潼、華,湖廣兵入商、雒,四川兵由興、漢,山西兵出蒲州、韓城,合剿。冬十月庚戌,湖廣兵援漢中,副將楊正芳戰死。十一月庚辰,逮陳奇瑜下獄,論戍。乙酉,洪承疇兼攝五省軍務。是冬,陜西賊分犯湖廣、河南,李自成陷陳州。是年,暹羅入貢。

八年春正月乙卯,賊陷上蔡,連陷汜水、滎陽、固始。己未,洪承疇出關討賊。辛酉,張獻忠陷潁州。丙寅,陷鳳陽,焚皇陵樓殿,留守朱國相等戰死。壬申,徐州援兵至鳳陽。張獻忠犯廬州,尋陷廬江、無為。李自成走歸德,與羅汝才復入陜西。二月,張獻忠陷潛山、羅田、太湖、新蔡,應天巡撫都御史張國維禦卻之。甲午,以皇陵失守,逮總督漕運尚書楊一鵬下獄,尋棄市。丁酉,總兵官鄧巳敗賊於羅山。是月,曹文詔敗賊於隨州。夏四月,張獻忠復走漢中,犯平涼、鳳翔。丁亥,鄭芝龍擊敗海賊劉香,香自殺,眾悉降。辛卯,洪承疇會師於汝州,分部諸將防豫、楚要害。乙巳,川兵變於樊城,鄧巳自殺。丙午,洪承疇西還,駐師靈寶。五月乙亥,吳宗達致仕。六月己丑,官軍遇賊於亂馬川,敗績。壬辰,副將艾萬年、柳國鎮擊李自成於寧州之襄樂,戰沒。丙午,曹文詔追賊至真寧之湫頭鎮,遇伏,力戰死之。秋七月甲戌,少詹事文震孟、刑部侍郎張至發俱禮部侍郎兼東閣大學士,預機務。是月,張獻忠突朱陽關,總兵官尤世威敗績,賊復走河南。八月,李自成陷咸陽,賊將高傑降。壬辰,詔撤監視總理內臣,惟京營及關、寧如故。辛丑,盧象昇總理直隸、河南、山東、湖廣、四川軍務。九月辛亥,洪承疇督副將曹變蛟等敗賊於關山鎮。李自成東走,與張獻忠合。壬戌,官軍敗績於沈丘之瓦店,總兵官張全昌被執。壬申,王應熊致仕。冬十月庚辰,下詔罪己,辟居武英殿,減膳撤樂,示與將士同甘苦。丙戌,戶部尚書侯恂請嚴征新舊逋賦,從之。辛卯,李自成陷陜州。十一月庚戌,何吾騶、文震孟罷。庚申,祀天於南郊。總兵官祖寬破賊於汝州。十二月戊寅,城鳳陽。乙酉,盧象昇、祖寬敗李自成於確山。戊子,左良玉敗賊於閿鄉。癸巳,賊犯江北,圍滁州。乙巳,老回回諸賊自河南犯陜西,洪承疇敗之於臨潼。是年,安南、暹羅、琉球入貢。

九年春正月甲寅,總理侍郎盧象昇、祖寬援滁,大敗賊於朱龍橋。丁卯,前禮部侍郎林釬以原官兼東閣大學士,預機務。二月人格化。,前副將湯九州及賊戰嵩縣,敗沒。山西大饑,人相食。乙酉,寧夏饑,兵變,殺巡撫都御史王楫,兵備副使丁啟睿撫定之。辛卯,以武舉陳起新為給事中。三月,盧象升、祖大樂剿河南賊。高迎祥、李自成分部入陜西,餘賊自光化走湖廣。振南陽饑,蠲山西被災州縣新舊二餉。夏四月戊子,錢士升致仕。五月壬子,詔赦協從諸賊。願歸者,護還鄉,有司安置;原隨軍自效者,有功一體敘錄。丙辰,延綏總兵官俞沖霄擊李自成於安定,敗績,死之。李自成犯榆林,賀人龍擊敗之。癸酉,免畿內五年以前逋賦。六月乙亥,林釬卒。甲申,吏部侍郎孔貞運,禮部尚書賀逢聖、黃士俊,俱禮部尚書兼東閣大學士,預機務。己亥,總兵官解進忠撫賊於淅川,被殺。秋七月甲辰,內臣李國輔等分守紫荊、倒馬諸關。庚戌,成國公朱純臣巡視邊關。癸丑,詔諸鎮星馳入援。己未,大清兵入昌平,巡關御史王肇坤等死之。壬戌,巡撫陜西都御史孫傳庭擊擒賊首高迎祥於盩厔,送京師伏誅。癸亥,諭廷臣助餉。甲子,兵部尚書張鳳翼督援軍,高起潛為總監。是月,大清兵入寶坻,連下近畿州縣。八月癸酉,括勛戚文武諸臣馬。乙未,盧象昇入援,次真定。丙申,唐王聿鍵起兵勤王,勒還國,尋廢為庶人。是月,大清兵出塞。九月辛酉,改盧象昇總督宣大、山西軍務。冬十月乙亥,工部侍郎劉宗周以論內臣及大學士溫體仁削籍。甲申,張獻忠犯襄陽。丙申,命開銀鐵銅鉛諸礦。十一月丁未,蠲山東五年以前逋賦。十二月,大清兵征朝鮮。是年,洪承疇敗賊於隴州,賊走慶陽、鳳翔。暹羅入貢。

十年春正月辛丑朔,日有食之。丙午,老回回諸賊趨江北,張獻忠、羅汝才自襄陽犯安慶,南京大震。二月甲戌,遣使督直省逋賦。丁酉,賊犯潛山,總兵官左良玉、副使史可法敗之於楓香驛。是月,朝鮮降於我大清。三月辛亥,振陜西災。丁巳,賜劉同升等進士及第、出身有差。甲子,官軍援安慶,敗績於酆家店。夏四月戊寅,大清兵克皮島,副總兵金日觀力戰死之,總兵官沈冬魁走石城島。癸巳旱,清刑獄。是月,洪承疇剿賊於漢南。閏月壬寅,敕群臣潔己愛民,以回天意。江北賊分犯河南,總督兩廣都御史熊文燦為兵部尚書,總理南京、河南、山、陜、川、湖軍務,駐鄖陽討賊。五月戊寅,李自成自秦州犯四川。六月戊申,溫體仁致仕。是夏,兩畿、山西大旱。秋七月,山東、河南蝗,民大饑。八月己酉,吏部侍郎劉宇亮、禮部侍郎傅冠俱禮部尚書,僉都御史薛國觀為禮部侍郎,並兼東閣大學士,預機務。庚申,閱城。九月丙子,左良玉敗賊於虹縣。辛卯,洪承疇敗賊於漢中。癸巳,李自成陷寧羌。冬十月丙申,自成自七盤關入西川。壬寅,陷昭化、劍州、梓潼,分兵趨潼川、江油、綿州,總兵官侯良柱戰死,遂陷彰明、鹽亭諸縣。庚戌,逼成都。十一月庚辰,以星變修省,求直言。十二月癸卯,黃士俊致仕。癸亥,洪承疇、曹變蛟援四川,次廣元。是年,安南、琉球入貢。

十一年春正月丁丑,洪承疇敗賊於梓潼,賊還走陜西。丁亥,裁南京冗官。二月甲辰,改河南巡按御史張任學為總兵官。三月戊寅,賀逢聖致仕。是月,李自成自洮州出番地,總兵官曹變蛟追破之,復入塞,走西和、禮縣。夏四月辛丑,張獻忠偽降於穀城,熊文燦受之。戊申,張至發致仕。己酉,熒惑逆行,諭廷臣修省。五月癸亥朔,策試考選官於中左門。六月癸巳,安民廠災,壞城垣,傷萬餘人。壬寅,孔貞運致仕。乙卯,兵部尚書楊嗣昌、戶部尚書程國祥、禮部侍郎方逢年、工部侍郎蔡國用俱禮部尚書,大理少卿范復粹為禮部侍郎,並兼東閣大學士,預機務。嗣昌仍掌兵部。是月,兩畿、山東、河南大旱蝗。秋七月乙丑,少詹事黃道周以論楊嗣昌奪情,謫按察司照磨。八月戊戌,以災異屢見,齋居永壽宮,諭廷臣修省。癸丑,傅冠致仕。戊午,停刑。流賊羅汝才等自陜州犯襄陽。九月,陜西、山西旱饑。辛巳,大清兵入牆子嶺,總督薊遼兵部侍郎吳阿衡死之。癸未,京師戒嚴。冬十月癸巳,盧象昇入援,召對於武英殿。甲午,括馬。盧象昇、高起潛分督援軍。是月,洪承疇、曹變蛟大破賊於潼關南原,李自成以數騎遁。十一月戊辰,大清兵克高陽,致仕大學士孫承宗死之。戊子,罷盧象升,戴罪立功。劉宇亮自請視師,許之。是月,羅汝才降。十二月庚子,方逢年罷。盧象昇兵敗於巨鹿,死之。戊申,孫傳庭為兵部侍郎督援軍。征洪承疇入衛。是年,土魯番、琉球入貢。

十二年春正月己未朔,以時事多艱,卻廷臣賀。庚申,大清兵入濟南,德王由樞被執「三統」、「性三品」等學說。注疏有清凌曙《春秋繁露注》、蘇,布政使張秉文等死之。戊辰,劉宇亮、孫傳庭會師十八萬於晉州,不敢進。丁丑,改洪承疇總督薊、遼,孫傳庭總督保定、山東、河北。二月乙未,劉宇亮罷。大清兵北歸。三月丙寅,出青山口。凡深入二千里,閱五月,下畿內、山東七十餘城。丙子,加上孝純皇太后謚,詔天下。夏四月戊申,程國祥致仕。是月,左良玉擊降賊首李萬慶。五月甲子,禮部侍郎姚明恭、張四知,兵部侍郎魏照乘,俱禮部尚書兼東閣大學士,預機務。乙丑,張獻忠叛於穀城,羅汝才等起應之,陷房縣。乙亥,削孫傳庭籍,尋逮下獄。六月,畿內、山東、河南、山西旱蝗。己酉,抽練各鎮精兵,復加征練餉。秋七月壬申,左良玉討張獻忠,敗績於羅猴山,總兵官羅岱被執死之。熊文燦削籍,尋逮下獄。八月癸巳,詔誅封疆失事巡撫都御史顏繼祖,總兵官倪寵、祖寬,內臣鄧希詔、孫茂霖等三十三人,俱棄市。己亥,免唐縣等四十州縣去年田租之半。壬子,大學士楊嗣昌督師討賊,總督以下並聽節制。冬十月甲申朔,楊嗣昌誓師襄陽。甲午,左良玉為平賊將軍。丙申,《欽定保民四事全書》成,頒布天下。十一月辛巳,祀天於南郊。十二月,羅汝才犯四川。丙午,下兵部尚書傅宗龍於獄。是年,琉球入貢。

十三年春閏正月乙酉,振真定饑。戊子,振京師饑民。癸卯,振山東饑。二月壬子朔,祀日於東郊。戊午,總督陜西三邊侍郎鄭崇儉大破張獻忠於太平縣之瑪瑙山,獻忠走歸州。戊寅,以久旱求直言。三月甲申,禱雨。丙戌,大風霾,詔清刑獄。戊子,罷各鎮內臣。丙申,賜魏藻德等進士及第、出身有差。戊戌,振畿內饑。丁未,免河北三府逋賦。夏四月戊午,逮江西巡撫僉都御史解學龍及所舉黃道周。己卯,吏部尚書謝陞為禮部尚書,禮部侍郎陳演以原官並兼東閣大學士,預機務。五月,羅汝才犯夔州,石砫女官秦良玉連戰卻之。甲申,祀地於北郊。庚戌,姚明恭致仕。六月辛亥朔,總兵官賀人龍等分道逐賊,敗之,羅汝才走大寧。庚午,蔡國用卒。辛未,薛國觀罷。秋七月庚辰朔,畿內捕蝗。己丑,發帑振被蝗州縣。辛卯,左良玉及京營總兵官孫應元等大破羅汝才於興山。汝才走巫山,與張獻忠合。八月甲戌,振江北饑。九月,陜西官軍圍李自成於巴西魚腹山中,自成走免。癸巳,張獻忠陷大昌,總兵官張令戰死。尋陷劍州、綿州。冬十月癸丑,熊文燦棄市。十一月,楊嗣昌進軍重慶。丁亥,祀天於南郊。戊子,南京地震。十二月丁未朔,嚴軍機抄傳之禁。辛亥,張獻忠陷瀘州。乙卯,逮薛國觀。是月,李自成自湖廣走河南,饑民附之,連陷宜陽、永寧,殺萬安王采崿,陷偃師,勢大熾。是年,兩畿、山東、河南、山、陜旱蝗,人相食。

十四年春正月辛巳,祈穀於南郊。己丑,總兵官猛如虎追張獻忠及於開縣之黃陵城,敗績,參將劉士傑等戰死聽言動「無一是我自家氣質,如此便是格物物格,致知知至,賊遂東下。丙申,李自成陷河南,福王常洵遇害,前兵部尚書呂維祺等死之。二月己酉,詔以時事多艱,災異疊見,痛自刻責,停今歲行刑,諸犯俱減等論。庚戌,張獻忠陷襄陽,襄王翊銘、貴陽王常法並遇害,副使張克儉等死之。戊午,李自成攻開封,周王恭枵、巡按御史高名衡拒卻之。乙丑,張獻忠陷光州。己巳,召閣臣、九卿、科道於乾清宮左室。命駙馬都尉冉興讓等齎帑金振恤河南被難宗室。三月丙子朔,楊嗣昌自四川還,至荊州卒。乙酉,禱雨。丙申,洪承疇會八鎮兵於寧遠。丁酉,逮鄭崇儉下獄,尋棄市。夏四月壬子,大清兵攻錦州,祖大壽拒守。己未,總督三邊侍郎丁啟睿為兵部尚書,督師討賊。五月庚辰,范復粹致仕。釋傅宗龍於獄,命為兵部侍郎,總督陜西三邊軍務,討李自成。戊子,祀地於北郊。六月,兩畿、山東、河南、浙江、湖廣旱蝗,山東寇起。秋七月己卯,李自成攻鄧州,楊文岳、總兵官虎大威擊敗之。壬寅,洪承疇援錦州,駐師松山。是月,臨清運河涸。京師大疫。八月乙巳,援兵戰於松山,陽和總兵官楊國柱敗沒。辛亥,賜薛國觀死。辛酉,重建太學成,釋奠於先師孔子。甲子,總兵官吳三桂、王樸自松山遁,諸軍夜潰。是月,左良玉大敗張獻忠於信陽。九月丁丑,傅宗龍帥師次新蔡,與總督保定侍郎楊文岳軍會。己卯,遇賊,賀人龍師潰,宗龍被圍,文岳走陳州。甲申,周延儒、賀逢聖復入閣。辛卯,封皇子慈炯為定王。壬辰,傅宗龍潰圍出,趨項城,被執死之。賊屠項城及商水、扶溝。戊戌,李自成、羅汝才陷葉縣,守將劉國能死之。是月,官軍破張獻忠於英山之望雲寨。冬十月癸卯朔,日有食之。十一月丙子,李自成陷南陽,唐王聿鏌遇害,總兵官猛如虎等死之。十二月,李自成連陷洧川、許州、長葛、鄢陵。甲子,戍解學龍、黃道周。李自成、羅汝才合攻開封,周王恭枵、巡撫都御史高名衡拒守。

十五年春正月癸未,孫傳庭為兵部侍郎,督京軍救開封。乙酉,楊文岳援開封,賊解去,南陷西華。戊子,免天下十二年以前逋賦。是月,山東賊陷張秋、東平,劫漕艘。太監王裕民、劉元斌帥禁兵會兗東官軍討平之。二月戊申,振山東就撫亂民。癸丑,總督陜西都御史汪喬年次襄城,遇賊,賀人龍等奔入關,喬年被圍。丁巳,城陷,被執死之。戊午,大清兵克松山,洪承疇降,巡撫都御史丘民仰,總兵官曹變蛟、王廷臣,副總兵江翥、饒勳等死之。是月,孫傳庭總督三邊軍務。三月,李自成陷陳州。丁丑,魏照乘致仕。己卯,祖大壽以錦州降於大清。辛卯,李自成陷睢州、太康、寧陵、考城。壬辰,封皇子慈炤為永王。丙申,李自成陷歸德。是春,江北賊陷含山、和州,南京戒嚴。夏四月癸亥,李自成復圍開封。乙丑,削謝升籍。五月己巳,孫傳庭入關,誅賀人龍。甲戌,張獻忠陷廬州。丁亥,王樸棄市。六月戊申,賀逢聖致仕。癸丑,張四知致仕。甲寅,詔天下停刑三年。己未,詹事蔣德璟、黃景昉,戎政侍郎吳甡,俱禮部尚書兼東閣大學士,預機務。庚申,詔孫傳庭出關。兵部侍郎侯恂督左良玉軍援開封。壬戌,以會推閣臣下吏部尚書李日宣六人於獄,謫戍有差。甲子,祀地於北郊。是月,築壇親祭死事文武大臣。山西總兵官許定國援開封,潰於沁水,寧武兵潰於覃懷。秋七月己巳,左良玉、虎大威、楊德政、方國安四鎮兵潰於朱仙鎮。八月庚戌,安慶兵變,殺都指揮徐良憲,官軍討定之。乙丑,釋黃道周於戍所,復其官。丁卯,兵部尚書陳新甲下獄,尋棄市。九月壬午,賊決河灌開封。癸未,城圮,士民溺死者數十萬人。己丑,孫傳庭帥師赴河南。辛卯,鳳陽總兵官黃得功、劉良佐大敗張獻忠於潛山。冬十月辛酉,孫傳庭敗績於郟縣,走入關。十一月丁卯,援汴總兵官劉超據永城反。庚午,發帑振開封被難宗室兵民。壬申,大清兵分道入塞,京師戒嚴。命勳臣分守九門,太監王承恩督察城守。詔舉堪督師大將者。戊寅,徵諸鎮入援。庚辰,大清兵克薊州。丁亥,薊鎮總督趙光抃提調援兵。戊子,張獻忠陷無為。己丑,遼東督師侍郎范志完入援。閏月癸卯,下詔罪己,求直言。壬寅,大清兵南下,畿南郡邑多不守。丁巳,起廢將。是月,李自成陷汝寧,前總督侍郎楊文岳、僉事王世琮不屈死。十二月,大清兵趨曹、濮,山東州縣相繼下,魯王以派自殺。己巳,李自成陷襄陽,據之。左良玉奔承天,尋走武昌。賊分兵下德安、彞陵、荊門,遂陷荊州。癸巳,焚獻陵。

十六年春正月丁酉,李自成陷承天,巡撫都御史宋一鶴、留守沈壽崇等死之。庚申,張獻忠陷蘄州。二月乙丑朔,日有食之。己巳,范志完、趙光抃會師於平原。三月庚子,李自成殺羅汝才,併其眾。壬寅,命大學士吳甡督師討賊。丁未,賊陷武岡,殺岷王企昪。張獻忠陷黃州。夏四月丁卯,周延儒自請督師,許之。辛卯,大清兵北歸,戰於螺山,總兵官張登科、和應薦敗沒,八鎮兵皆潰。是月,劉超平。五月癸巳朔,張獻忠陷漢陽。壬寅,周延儒還京師。丙午,修撰魏藻德為少詹事兼東閣大學士,預機務。戊申,吳甡罷。丁巳,周延儒罷。壬戌,張獻忠陷武昌,沈楚王華奎於江,在籍大學士賀逢聖等死之。六月癸亥,詔免直省殘破州縣三餉及一切常賦二年。己卯,逮范志完下獄。丙戌,雷震奉先殿獸吻,敕修省。秋七月丁酉,親鞫范志完於中左門。乙卯,親鞫前文選郎中吳昌時於中左門,徵周延儒聽勘。己未,戒廷臣私謁閣臣。京師自二月至於是月大疫,詔釋輕犯,發帑療治,瘞五城暴骸。八月壬戌朔,左良玉復武昌、漢陽。丙寅,張獻忠陷岳州。丙戌,陷長沙。庚寅,陷衡州。九月丙申,張獻忠陷寶慶。己亥,黃景昉致仕。辛丑,孫傳庭復寶豐,進次郟縣,李自成迎戰,擊敗之。庚戌,張獻忠陷永州,巡按御史劉熙祚死之。辛亥,賜楊廷鑑等進士及第、出身有差。壬子,孫傳庭兵以乏食引退,賊追及之,還戰大敗,傳庭以餘眾退保潼關。是月,鳳陽地屢震。冬十月辛酉朔,享太廟。丙寅,李自成陷潼關,督師尚書孫傳庭死之。賊連陷華州、渭南、臨潼。命有司以贖鍰充餉。戊辰,李自成屠商州。庚午,張獻忠陷常德。壬申,李自成陷西安,秦王存樞降,巡撫都御史馮師孔、按察使黃絅等死之。丁丑,張獻忠陷吉安。十一月甲午,李自成陷延安,尋屠鳳翔。壬寅,祀天於南郊。辛亥,吏部侍郎李建泰、副都御史方岳貢並兼東閣大學士,預機務。癸丑,范志完、趙光抃棄市,戍吳甡於金齒。丁巳,李自成陷榆林,兵備副使都任、在籍總兵官尤世威等死之。寧夏、慶陽相繼陷,韓王亶脊被執。十二月壬戌,張獻忠陷建昌。乙丑,周延儒有罪賜死。丁卯,張獻忠陷撫州。辛巳,賊渡河,陷平陽,山西州縣相繼潰降。甲申,賊陷甘州,巡撫都御史林日瑞、總兵官馬爌等死之。丙戌,左良玉復長沙。是年,暹羅,琉球、哈密入貢。十七年春正月庚寅朔,大風霾,鳳陽地震。庚子,李建泰自請措餉治兵討賊,許之。乙卯,幸正陽門樓,餞李建泰出師。南京地震。丙辰,工部尚書范景文、禮部侍郎丘瑜並兼東閣大學士,預機務。是月,張獻忠入四川。二月辛酉,李自成陷汾州,別賊陷懷慶。丙寅,陷太原,執晉王求桂,巡撫都御史蔡懋德等死之。壬申,下詔罪己。癸酉,潞安陷。乙亥,議京師城守。李自成攻代州,總兵官周遇吉力戰,食盡,退守寧武關。丁丑,賊別將陷固關,犯畿南。己卯,遣內臣高起潛、杜勳等十人監視諸邊及近畿要害。壬午,真定知府丘茂華殺總督侍郎徐標,檄所屬降賊。甲申,賊至彰德,趙王常水臾降。丁亥,詔天下勤王。命廷臣上戰守事宜。左都御史李邦華、右庶子李明睿請南遷及太子撫軍江南,皆不許。戊子,陳演致仕。李自成陷寧武,周遇吉力戰死之。三月庚寅,賊至大同,總兵官姜瑰降賊,代王傳齊遇害,巡撫都御史衛景瑗被執,自縊死。辛卯,李建泰疏請南遷。壬辰,召廷臣於平臺,示建泰疏,曰:「國君死社稷,朕將焉往?」李邦華等復請太子撫軍南京,不聽。蔣德璟致仕。癸巳,封總兵官吳三桂、左良玉、唐通、黃得功俱為伯。甲午,徵諸鎮兵入援。乙未,總兵官唐通入衛,命偕內臣杜之秩守居庸關。戊戌,太監王承恩提督城守。己亥,李自成至宣府,監視太監杜勳降,巡撫都御史朱之馮等死之。癸卯,唐通、杜之秩降於自成,賊遂入關。甲辰,陷昌平。乙巳,賊犯京師,京營兵潰。丙午,日晡,外城陷。是夕,皇后周氏崩。丁未,昧爽,內城陷。帝崩於萬歲山,王承恩從死。御書衣襟曰:「朕涼德藐躬,上干天咎,然皆諸臣誤朕。朕死無面目見祖宗,自去冠冕,以髮覆面。任賊分裂,無傷百姓一人。」自大學士范景文而下死者數十人。丙辰,賊遷帝、后梓宮於昌平。昌平人啟田貴妃墓以葬。明亡。是年夏四月,我大清兵破賊於山海關,五月,入京師,以帝體改葬,令臣民為服喪三日,謚曰莊烈愍皇帝,陵曰思陵。

贊曰:帝承神、熹之後,慨然有為。即位之初,沈機獨斷,刈除奸逆,天下想望治平。惜乎大勢已傾,積習難挽。在廷則門戶糾紛。疆埸則將驕卒惰。兵荒四告,流寇蔓延。遂至潰爛而莫可救,可謂不幸也已。然在位十有七年,不邇聲色,憂勸惕勵,殫心治理。臨朝浩歎,慨然思得非常之材,而用匪其人,益以僨事。乃復信任宦官,布列要地,舉措失當,制置乖方。祚訖運移,身罹禍變,豈非氣數使然哉。迨至大命有歸,妖氛盡掃,而帝得加謚建陵,典禮優厚。是則聖朝盛德,度越千古,亦可以知帝之蒙難而不辱其身,為亡國之義烈矣。

\end{pinyinscope}