\article{禮志}

《周官》、《儀禮》尚已,然書缺簡脫,因革莫詳。自漢史作《禮志》,後皆因之,一代之制,始的然可考。歐陽氏云:「三代以下,治出於二,而禮樂為虛名。」要其用之郊廟朝廷,下至閭里州黨者,未嘗無可觀也。惟能修明講貫,以實意行乎其間,則格上下、感鬼神,教化之成即在是矣。安見後世之禮,必不可上追三代哉。

明太祖初定天下,他務未遑,首開禮、樂二局,廣徵耆儒,分曹究討。洪武元年,命中書省暨翰林院、太常司,定擬祀典。乃歷敘沿革之由,酌定郊社宗廟儀以進。禮官及諸儒臣又編集郊廟山川等儀,及古帝王祭祀感格可垂鑒戒者,名曰《存心錄》。二年,詔諸儒臣修禮書。明年告成,賜名《大明集禮》。其書準五禮而益以冠服、車輅、儀仗、鹵簿、字學、音樂,凡升降儀節,制度名數,纖悉畢具。又屢敕議禮臣李善長、傅瓛、宋濂、詹同、陶安、劉基、魏觀、崔亮、牛諒、陶凱、朱升、樂韶鳳、李原名等,編輯成集。且詔郡縣舉高潔博雅之士徐一夔、梁寅、周子諒、胡行簡、劉宗弼、董彞、蔡深、滕公琰至京,同修禮書。在位三十餘年,所著書可考見者,曰《孝慈錄》,曰《洪武禮制》,曰《禮儀定式》,曰《諸司職掌》,曰《稽古定制》,曰《國朝制作》,曰《大禮要議》,曰《皇朝禮制》,曰《大明禮制》,曰《洪武禮法》,曰《禮制集要》,曰《禮制節文》,曰《太常集禮》,曰《禮書》。若夫釐正祀典,凡天皇、太乙、六天、五帝之類,皆為革除,而諸神封號,悉改從本稱,一洗矯誣陋習,其度越漢、唐遠矣。又詔定國恤,父母並斬衰,長子降為期年,正服旁服以遞而殺,斟酌古今,蓋得其中。永樂中,頒《文公家禮》於天下,又定巡狩、監國及經筵日講之制。後宮罷殉,始於英宗。陵廟嫡庶之分,正於孝宗。暨乎世宗,以制禮作樂自任。其更定之大者,如分祀天地,復朝日夕月於東西郊,罷二祖並配,以及祈穀大雩,享先蠶,祭聖師,易至聖先師號,皆能折衷於古。獨其排眾議,祔睿宗太廟躋武宗上,徇本生而違大統,以明察始而以豐暱終矣。當時將順之臣,各為之說。今其存者,若《明倫大典》,則御製序文以行之;《祀儀成典》,則李時等奉敕而修;《郊祀考議》,則張孚敬所進者也。至《大明會典》,自孝宗朝集纂,其於禮制尤詳。世宗、神宗時,數有增益,一代成憲,略具是焉。今以五禮之序,條為品式,而隨時損益者,則依類編入,以識沿革云。

壇壝之制神位祭器玉帛牲牢祝冊之數籩豆之實

祭祀雜議諸儀祭祀日期習儀齋戒遣官祭祀

○分獻陪祀

五禮,一曰吉禮。凡祀事,皆領於太常寺而屬於禮部。明初以圜丘、方澤、宗廟、社稷、朝日、夕月、先農為大祀,太歲、星辰、風雲雷雨、嶽鎮、海瀆、山川、歷代帝王、先師、旗纛、司中、司命、司民、司祿、壽星為中祀,諸神為小祀。後改先農、朝日、夕月為中祀。凡天子所親祀者,天地、宗廟、社稷、山川。若國有大事,則命官祭告。其中祀小祀,皆遣官致祭,而帝王陵廟及孔子廟,則傳制特遣焉。每歲所常行者,大祀十有三:正月上辛祈穀、孟夏大雩、季秋大享、冬至圜丘皆祭昊天上帝,夏至方丘祭皇地祇,春分朝日於東郊,秋分夕月於西郊,四孟季冬享太廟,仲春仲秋上戊祭太社太稷。中祀二十有五:仲春仲秋上戊之明日,祭帝社帝稷,仲秋祭太歲、風雲雷雨、四季月將及嶽鎮、海瀆、山川、城隍,霜降日祭旗纛於教場,仲秋祭城南旗纛廟,仲春祭先農,仲秋祭天神地祗於山川壇,仲春仲秋祭歷代帝王廟,春秋仲月上丁祭先師孔子。小祀八:孟春祭司戶,孟夏祭司灶,季夏祭中霤,孟秋祭司門,孟冬祭司井,仲春祭司馬之神,清明、十月朔祭泰厲,又於每月朔望祭火雷之神。至京師十廟、南京十五廟,各以歲時遣官致祭。其非常祀而間行之者,若新天子耕耤而享先農,視學而行釋奠之類。嘉靖時,皇后享先蠶,祀高禖,皆因時特舉者也。

其王國所祀,則太廟、社稷、風雲雷雨、封內山川、城隍、旗纛、五祀、厲壇。府州縣所祀,則社稷、風雲雷雨、山川、厲壇、先師廟及所在帝王陵廟,各衛亦祭先師。至於庶人,亦得祭里社、穀神及祖父母、父母并祀灶,載在祀典。雖時稍有更易,其大要莫能踰也。

至若壇壝之制,神位、祭器、玉帛、牲牢、祝冊之數,籩豆之實,酒齊之名,析其彼此之異同,訂其初終之損益,臚於首簡,略於本條,庶無缺遺,亦免繁復云爾。

○壇壝之制

明初,建圜丘於正陽門外,鐘山之陽,方丘於太平門外,鐘山之陰。圜丘壇二成。上成廣七丈,高八尺一寸,四出陛,各九級,正南廣九尺五寸,東、西、北八尺一寸。下成周圍壇面,縱橫皆廣五丈,高視上成,陛皆九級,正南廣一丈二尺五寸,東、西、北殺五寸五分。甃證磚闌盾,皆以琉璃為之。壝去壇十五丈,高八尺一寸,四面靈星門,南三門,東、西、北各一。外垣去壝十五丈,門制同。天下神祇壇東門外。神庫五楹,在外垣北,南向。廚房五楹祇,在外壇東北,西向。庫房五楹,南向。宰牲房三楹,天池一,又在外庫房之北。執事齋舍,在壇外垣之東南。坊二,在外門外橫甬道之東西,燎壇在內壝外東南丙地,高九尺,廣七尺,開上南出戶。方丘壇二成。上成廣六丈,高六尺,四出陛,南一丈,東、西、北八尺,皆八級。下成四面各廣二丈四尺,高六尺,四出陛,南丈二尺,東、西、北一丈,皆八級。壝去壇十五丈,高六尺,外垣四面各六十四丈,餘制同。南郊有浴室,瘞坎在內壝外壬地。

洪武四年,改築圜丘。上成廣四丈五尺,高五尺二寸。下成每面廣一丈六尺五寸,高四尺九寸。二成通徑七丈八尺。壇至內壝墻,四面各九丈八尺五寸。內壝墻至外壝墻,南十三丈九尺四寸,北十一丈,東、西各十一丈七尺。方丘,上成廣三丈九尺四寸,高三尺九寸。下成每面廣丈五尺五寸,高三尺八寸,通徑七丈四寸。壇至內壝墻,四面皆八丈九尺五寸。內壝墻至外壝墻,四面各八丈二尺。

十年,改定合祀之典。即圜丘舊制,而以屋覆之,名曰大祀殿,凡十二楹。中石臺設上帝、皇地祇座。東、西廣三十二楹。正南大祀門六楹,接以步廊,與殿廡通。殿後天庫六楹。瓦皆黃琉璃。廚庫在殿東北,宰牲亭井在廚東北,皆以步廊通殿兩廡,後繚以圍墻。南為石門三洞以達大祀門,謂之內壇。外周垣九里三十步,石門三洞南為甬道三,中神道,左御道,右王道。道兩旁稍低,為從官之地。齋宮在外垣內西南,東向。其後殿瓦易青琉璃。二十一年增修壇壝,壇後樹松柏,外壝東南鑿池二十區。冬月伐冰藏凌陰,以供夏秋祭祀之用。成祖遷都北京,如其制。

嘉靖九年,復改分祀。建圜丘壇於正陽門外五里許,大祀殿之南,方澤壇於安定門外之東。圜丘二成,壇面及欄俱青琉璃,邊角用白玉石,高廣尺寸皆遵祖制,而神路轉遠。內門四。南門外燎爐毛血池,西南望燎臺。外門亦四。南門外左具服臺,東門外神庫、神廚、祭器庫、宰牲亭,北門外正北泰神殿。正殿以藏上帝、太祖之主,配殿以藏從祀諸神之王。外建四天門:東曰泰元,南曰昭亭,西曰廣利。又西鑾駕庫,又西犧牲所,其北神樂觀。北曰成貞。北門外西北為齋宮,迤西為壇門,壇北,舊天地壇,即大祀殿也。十七年撤之,又改泰神殿曰皇穹宇。二十四年,又即故大祀殿之址建大享殿。方澤亦二成,壇面黃琉璃,陛增為九級,用白石圍以方坎。內,北門外西瘞位,東燈臺,南門外皇祇室。外,西門外迤西神庫、神廚、宰牲亭、祭器庫,北門外西北齋宮。又外建四天門,西門外北為鑾駕庫、遣官房、內陪祀官房。又外為壇門,門外為泰折街牌坊,護壇地千四百餘畝。

太社稷壇,在宮城西南,東西峙,明初建。廣五丈,高五尺,四出陛,皆五級。壇土五色隨其方,黃土覆之。壇相去五丈,壇南皆樹松。二壇同一壝,方廣三十丈,高五尺,甃磚,四門飾色隨其方。周坦四門,南靈星門三,北戟門五,東西戟門三。戟門各列戟二十四。洪武十年,改壇午門右,社稷共一壇,為二成。上成廣五丈,下成廣五丈三尺,崇五尺。外壝崇五尺,四面各十九丈有奇。外垣東西六十六丈有奇,南北八十六丈有奇。垣北三門,門外為祭殿,其北為拜殿。外復為三門,垣東、西、南門各一。永樂中,建壇北京,如其制。帝社稷壇在西苑,壇址高六寸,方廣二丈五尺,甃細磚,實以凈土。壇北樹二坊,曰社街。王國社稷壇,高廣殺太社稷十之三。府、州、縣社稷壇,廣殺十之五,高殺十之四,陛三級。後皆定同壇合祭,如京師。

朝日、夕月壇,洪武三年建。朝日壇高八尺,夕月壇高六尺,俱方廣四丈。兩壝,壝各二十五步。二十一年罷。嘉靖九年復建,壇各一成。朝日壇紅琉璃,夕月壇用白。朝日壇陛九級,夕月壇六級,俱白石。各建天門二。

先農壇,高五尺,廣五丈,四出陛。御耕耤位,高三尺,廣二丈五尺,四出陛。

山川壇,洪武九年建。正殿、拜殿各八楹,東西廡二十四楹。西南先農壇,東南具服殿,殿南耤田壇,東旗纛廟,後為神倉。周垣七百餘丈,垣內地歲種穀蔬,供祀事。嘉靖十年,改名天神地祇壇,分列左右。

太歲壇與嶽瀆同。嶽鎮海瀆山川城隍壇,據高阜,南向,高二尺五寸,方廣十倍,四出陛,南向五級,東西北三級。王國山川壇,高四尺,四出陛,方三丈五尺。天下山川所在壇,高三尺,四出陛,三級,方二丈五尺。

○神位祭器玉帛牲牢祝冊之數

神位圜丘。洪武元年冬至,正壇第一成,昊天上帝南向。第二成,東大明,星辰次之,西夜明,太歲次之。二年,奉仁祖配,位第一成,西向。三年,壇下壝內,增祭風雲雷雨。七年更定,內壝之內,東西各三壇。星辰二壇,分設於東西。其次,東則太歲、五嶽,西則風雲雨、五鎮。內壝之外,東西各二壇。東四海,西四瀆。次天下神祇壇,東西分設。

方丘。洪武二年夏至,正壇第一成,皇地祇,南向。第二成,東五嶽,次四海,西五鎮,次四瀆。三年,奉仁祖配,位第一成,西向。壇下壝內,增祭天下山川。七年更定,內壝之內,東西各二壇。東四海,西四瀆。次二壇,天下山川。內壝之外,東西各設天下神祇壇一。

十二年正月,合祀大祀殿。正殿三壇,上帝、皇地祇並南向。仁祖配位在東,西向。從祀十四壇。丹陛東一壇曰大明,西一壇曰夜明。兩廡壇各六:星辰二壇;次東,太歲、五嶽、四海,次西,風雲雷雨、五鎮、四瀆二壇;又次天下山川神祇二壇。俱東西向。二十一年,增修丹墀內石臺四,大明、夜明各一,星辰二。內壝外石臺二十:東十壇,北嶽、北鎮、東嶽、東鎮、東海、太歲、帝王、山川、神祇、四瀆;西十壇,北海、西嶽、西鎮、西海、中嶽、中鎮、風雲雷雨、南嶽、南鎮、南海。俱東西向。臺高三尺有奇,周以石欄,陟降為磴道。臺上琢石鑿龕,以置神位。建文時,撤仁祖,改奉太祖配,位第一成。西向。洪熙元年,增文皇帝於太祖下。

嘉靖九年,復分祀之典。圜丘則東大明,西夜明。次東,二十八宿、五星、周天星辰。次西,風雲雷雨。共四壇。方丘則東五嶽,基運、翊聖、神烈三山,西五鎮,天壽、純德二山。次東四海,次西四瀆。南北郊皆獨奉太祖配。太社稷配位別見。先農正位南向,后稷配位西向。

凡神位,天地、祖宗曰「神版」,餘曰「神牌」。圜丘神版長二尺五寸,廣五寸,厚一寸,趺高五寸,以栗木為之,正位題曰昊天上帝,配位題曰某祖某皇帝,並黃質金字。從祀風雲雷雨位版,赤質金字。神席,上帝用龍椅龍案,上施錦褥,配位同。從祀,位置於案,不設席。方丘正位曰皇地祇,配位及從祀,制並同圜丘。奉先殿帝后神主高尺二寸,廣四寸,趺高二寸,用木,飾以金,鏤以青字。龕高二尺,廣二尺,趺高四寸,朱漆鏤金龍鳳花版,開二窗,施紅紗,側用金銅環,內織金文綺為藉。社稷,社玉用石,高五尺,廣五尺,上微銳。立於壇上,半在土中,近南北向;稷不用主。洪武十年,皆設木主,丹漆之。祭畢,貯於庫,仍用石主埋壇中,微露其末。後奉祖配,其位製塗金牌座,如先聖櫝用架罩。嘉靖中,藏於寢廟。帝社稷神位以木,高一尺八寸,廣三寸,朱漆質金書。壇南置石龕,以藏神位。王府州縣社主皆用石,長二尺五寸,廣尺五寸。日月壇神位,以松柏為之。長二尺五寸,廣五寸,趺高五寸。朱漆金字。餘仿此。

祭器南郊。洪武元年定,正位,登一,籩豆各十二,簠簋各二,爵三;壇上,太尊二,著尊、犧尊、山罍各一;壇下,太尊一,山罍二。從祀位,登一,籩豆各下,簠簋各二,東西各設著尊二,犧尊二。北郊同。七年增圜丘從祀,共設酒尊六於壇西,大明,夜明位各三。天下神祇,鉶三,籩豆各八,簠簋各二,壝內外東西各設酒尊三,每位爵三。方丘、嶽鎮,各設酒尊三,壝內東西各設酒尊三,壝外東西各設酒尊三,每位爵三。神祇與圜丘同。八年,圜丘從祀,更設登一、鉶二。每位增酒睟,星辰、天下神祇各三十,太歲、風雲雷雨、嶽鎮海瀆各十五。方丘,從祀同。十年,定合祀之典,各壇陳設如舊,惟太歲、風雲雷雨酒盞各十,東西廡俱共設酒尊三、爵十八於壇南。

二十一年更定,正殿上三壇,每壇登一,籩豆各十二,簠簋各二,共設酒尊六、爵九於殿東南,西向。丹墀內四壇,大明、夜明各登一,籩豆十,簠簋二,酒尊三,爵三。星辰二壇,各登一,鉶二,酒盞三十,餘與大明同。壝外二十壇,各登一,鉶二,籩豆各十,簠簋各二,酒盞十,酒尊三,爵三。神祇壇,鉶三,籩豆各八。帝王、山川、四瀆、中嶽、風雲雷雨神祇壇,酒盞各三十,餘並同嶽鎮。

太廟時享。洪武元年定,每廟登一,鉶三,籩豆各十二,簠簋各二,共酒尊三、金爵八、瓷爵十六於殿東西向。二十一年更定,每廟登二,鉶二。弘治時,九廟通設酒尊九,祫祭加一,金爵十七,祫祭加二,瓷爵三十四,祫祭加四。親王配享,洪武三年定,登鉶各三,籩豆各十二,簠簋各二,酒尊三,酒注二。二十一年更定,登鉶各一,爵各三,籩豆各十,簠簋各二,共用酒尊三於殿東。功臣配享,洪武二年定,每位籩豆各二,簠簋各二。三年增定,共用酒尊二,酒注二。二十一年更定,十壇,每壇鉶一,籩豆各二,簠簋各一,爵三,共用酒尊於殿西。

太社稷。洪武元年定,鉶三,籩豆各十,簠簋各二,配位同。正配位皆設酒尊三於壇東。十一年更定,每位登一,鉶二,籩豆十二,正配位共設酒尊三,爵九。後太祖、成祖並配時,增酒尊一,爵三。府、州、縣社稷,鉶一,籩豆四,簠簋二。

朝日、夕月。洪武三年定,太尊、著尊、山罍各二,在壇上東南隅,北面。象尊、壺尊、山罍各二,在壇下,籩豆各十,簠簋各二,登鉶各三。

先農,與社稷同,加登一,籩豆減二。

神祇。洪武二年定,每壇籩豆各四,簠簋登爵各一。九年更定,正殿共設酒尊三,爵七,兩廡各設酒尊三,爵三,餘如舊。二十一年更定,每壇登一,鉶二,籩豆各十,簠簋各二,酒盞三十。星辰,正殿中登一,鉶二。餘九壇,鉶二。每壇籩豆十,簠簋各一,酒盞三十,爵一,共設酒尊三。太歲諸神,籩豆各八,簠簋各二,酒尊三。嶽瀆山川同。

歷代帝王。洪武四年定,登一,鉶二,籩豆各八,簠簋各一,俎一,爵三,尊三。七年更定,登、鉶、簠簋各一,籩豆各十,爵各三,共設酒尊五於殿西階,酒尊三於殿東階。二十一年增定,每位鉶二,簠簋各二,五室共設酒尊三,爵四十八。配位每壇籩豆各二,簠簋各一,饋盤一,每位鉶一,酒盞三。三皇,籩豆各八,簠簋各二,登、鉶各二,爵三,犧尊、象尊、山罍各一。配位,籩豆各四,簠簋各二,鉶一,爵三,犧尊、象尊各一。

至聖先師。洪武元年定,籩豆各六,簠簋各二,登一,鉶二,犧尊、象尊、山罍各一。四配位,籩豆各四,簠簋各一,登一。十哲,兩廡,籩豆二。四年更定,正位,籩豆各十,酒尊三,爵三,餘如舊。四配,每位酒尊一,餘同正位。十哲,東西各爵一,每位籩豆各四,簠簋各一,鉶一,酒盞一。兩廡,東西各十三壇,東西各爵一,每壇籩豆各四,簠簋各一,酒盞四。十五年更定,正位,酒尊一,爵三,登一,鉶二,籩豆各八,簠簋各二。四配位,共酒尊一,各爵三,登一,鉶二,籩豆各六,簠簋各一。十哲,共酒尊一,東西各爵五,鉶一,籩豆各四,簠簋各一。東西廡,每四位爵四,籩豆各二,簠簋各一。景泰六年增兩廡籩豆各二,簠簋各一。成化十二年,增正位籩豆為十二。嘉靖九年,仍減為十。

旗纛,與先農同。馬神,籩豆各四,簠簋、登、象尊、壺尊各一。

○玉帛牲牢

玉三等:上帝,蒼璧;皇地祇,黃琮;太社、太稷,兩圭有邸;朝日、夕月,圭璧五寸。帛五等:曰郊祀制帛,郊祀正配位用之。上帝,蒼;地祇,黃;配位,白。曰禮神制帛,社稷以下用之。社稷,黑;大明,赤;夜明、星辰、太歲、風雲雷雨、天下神祇俱白;五星,五色;嶽鎮、四海、陵山隨方色;四瀆,黑;先農,正配皆青;群神,白;帝王先師皆白;旗纛,洪武元年用黑,七年改赤,九年定黑二、白五。曰奉先制帛,太廟用之,每廟二。曰展親制帛,親王配享用之。曰報功制帛,功臣配享用之。皆白。每位各一。惟圜丘,嘉靖九年用十二,而周天星辰則共用十,孔廟十哲、兩廡東西各一云。又洪武十一年,上以小祀有用楮錢者為不經。禮臣議定,在京,大祀、中祀用制帛,有篚。在外,王國府州縣亦如之。小祀惟用牲醴。

牲牢三等:曰犢,曰羊,曰豕。色尚騂,或黝。大祀,入滌九旬;中祀,三旬;小祀,一旬。大祀前一月之朔,躬詣犧牲所視牲,每日大臣一人往視。洪武二年,帝以祭祀省牲,去神壇甚邇,於人心未安,乃定省牲之儀,去神壇二百步。七年定制,大祀,皇帝躬省牲;中祀、小祀,遣官。嘉靖十一年更定,冬、夏至,祈穀,俱祭前五日親視,後俱遣大臣。圜丘,蒼犢;方丘,黃犢;配位,各純犢。洪武七年,增設圜丘配位。星辰,牛一,羊豕三。太歲,牛羊豕一。風雲雷雨、天下神祇,羊豕各五。方丘配位,天下山川,牛一,羊豕各三。太廟禘,正配皆太牢,祫皆太牢。時享每廟犢羊豕各一。親王配位,洪武三年定,共牛羊豕一。二十一年更定,每壇犢羊豕各一。功臣配位,洪武二年定,每位羊豕體各一。二十一年更定,每壇羊豕一。太社稷,犢羊豕各一,配位同。府州縣社稷,正配位,共羊一、豕一。洪武七年增設,各羊一、豕一。朝日、夕月,犢羊豕各一。先農與太社稷同。神祇,洪武二年定,羊六、豕六。二十一年更定,每壇犢羊豕各一。嘉靖十年,天神左,地祇右,各牲五。星辰,每壇羊一、豕一。帝王,每室犢羊豕各一。配位,每壇羊豕各一。先師如帝王,四配如配位,十哲東西各豕一分五,兩廡東西各豕一,後增為三。府州縣學先師,羊一、豕一。四配。共羊一、豕一,解為四體。十哲東西各豕一,解為五體。兩廡豕一,解為百八分。旗纛,洪武九年定犢羊豕,永樂後,去犢。王國及衛所同。五祀馬神俱用羊豕。

○祝冊

南北郊,祝板長一尺一分,廣八寸,厚二分,用楸梓木。宗廟,長一尺二寸,廣九寸,厚一,用梓木,以楮紙冒之。群神帝王先師,俱有祝,文多不載。祝案設於西。

○籩豆之實

凡籩豆之實,用十二者,籩實以形鹽、AK魚、棗、栗、榛、菱、芡、鹿脯、白餅、黑餅、糗餌、粉餈。豆實以韭菹、醯醢,菁菹、鹿醢、芹菹、兔醢、筍菹、魚醢、脾析、豚胉、赩食、糝食。用十者,籩則減糗餌、粉餈,豆則減赩食、糝食。用八者,籩又減白、黑餅,豆又減脾析、豚胉。用四者,籩則止實以形鹽、AK魚、棗、慄,豆則止實以芹菹、兔醢、菁菹、鹿醢。各二者,籩實栗、鹿脯,豆實菁菹、鹿醢。簠簋各二者,實以黍稷、稻粱。各一者,實以稷粱。登實以太羹,鉶實以和羹。

洪武三年,禮部言:「《禮記·郊特牲》曰,『郊之祭也』,『器用陶匏』,尚質也。《周禮·籩人》,『凡祭祀供簠簋之實』,《疏》曰,『外祀用瓦簠』。今祭祀用瓷,合古意。惟盤盂之屬,與古簠璺簋登鉶異制。今擬凡祭器皆用瓷,其式皆仿古簠簋登豆,惟籩以竹。」詔從之。

酒齊仿周制,用新舊醅,以備齊三酒。其實於尊之名數,各不同。

○祭祀雜議諸儀

其祭祀雜議諸儀,凡版位,皇帝位,方一尺二寸,厚三寸,紅質金字。皇太子位,方九寸,厚二寸,紅質青字。陪祀官位,並白質黑字。

拜褥。初用緋。洪武三年定制,郊丘席為表,蒲為裏。宗廟、社稷、先農、山川,紅文綺為表,紅木棉布為裏。

贊唱。凡皇帝躬祀,入就位時,太常寺奏中嚴,奏外辦。盥洗、升壇、飲福、受胙,各致贊辭。又凡祀,各設爵洗位,滌爵拭爵。初升壇,唱再拜,及祭酒,唱賜福胙。洪武七年,禮部奏其煩瀆,悉刪去。

上香禮。明初祭祀皆行。洪武七年以翰林詹同言罷。嘉靖九年復行。

拜禮。初,每節皆再拜。洪武九年,禮臣奏:「《禮記》一獻三獻五獻七獻之文,皆不載拜禮。唐、宋郊祀,每節行禮皆再拜。然亞獻終獻,天子不行禮,而使臣下行之。今議大祀中祀,自迎神至飲福送神,宜各行再拜禮。」帝命節為十二拜,迎神、飲福受胙、送神各四拜云。

登壇脫舄。初未行。洪武八年詔翰林院臣考定大祀登壇脫舄之禮。學士樂韶鳳雜考漢、魏以來朝祭儀,議於郊祀廟享前期一日,有司以席藉地,設御幕於壇東南門外,設執事官脫履之次於壇門外西階側。祭日,大駕入幕次,脫舄升壇。其升壇執事、導駕、贊禮、讀祝并分獻陪祀官,皆脫舄於外,以次升壇供事。協律郎、樂舞生依前跣襪就位。祭畢,降壇納舄。從之。嘉靖十七年罷其禮。

○祭祀日期

欽天監選擇,太常寺預於十二月朔至奉天殿具奏。蓋古卜法不存,而擇干支之吉以代卜也。洪武七年,命太常卿議祭祀日期,書之於版,依時以祭,著為式。其祭日,遣官監祭,不敬失儀者罪之。

○習儀

凡祭祀,先期三日及二日,百官習儀於朝天宮。嘉靖九年更定,郊祀冬至,習儀於先期之七日及六日。

○齋戒

洪武二年,學士朱升等奉敕撰齋戒文曰:「戒者,禁止其外;齋者,整齊其內。沐浴更衣,出宿外舍,不飲酒,不茹葷,不問疾,不弔喪,不聽樂,不理刑名,此則戒也;專一其心,嚴畏謹慎,茍有所思,即思所祭之神,如在其上,如在其左右,精白一誠,無須臾間,此則齋也。大祀七日,前四日戒,後三日齋。」太祖曰:「凡祭祀天地、社稷、宗廟、山川等神,為天下祈福,宜下令百官齋戒。若自有所禱於天地百神,不關民事者,不下令。」又曰:「致齋以五日七日,為期太久,人心易怠。止臨祭,齋戒三日,務致精專,庶可格神明。」遂著為令。是年從禮部尚書崔亮奏,大祀前七日,部祀官詣中書省受誓戒。各揚其職,不共其事,國有常刑。宗廟社稷,致齋三日,不誓戒。三年,諭禮部尚書陶凱曰:「人心操舍無常,必有所警,而後無所放。」乃命禮部鑄銅人一,高尺有五寸,手執牙簡,大祀則書致齋三日,中祀則書致齋二日於簡上,太常司進置齋所。四年,定天子親祀齋五日,遣官代祀齋三日,降香齋一日。五年,命諸司各置木牌,以警褻慢,刻文其上曰:「國有常憲,神有鑒焉。」凡祭祀,則設之。又從陶凱奏,凡親祀,皇太子宮中居守,親王戎服侍從。皇太子親王雖不陪祀,一體齋戒。

六年,建陪祀官齋房於北郊齋宮之西南,後定齋戒禮儀。凡祭天地,正祭前五日午後,沐浴更衣,處外室,次早,百官於奉天門觀誓戒牌。次日,告仁祖廟,退處齋宮,致齋三日。享宗廟,正祭前四日午後,沐浴更衣,處外室。次日為始,致齋三日。祭社稷、朝日、夕月、周天星辰、太歲、風雲雷雨、嶽鎮海瀆、山川等神,致齋二日,如前儀。凡傳制降香,遣官代祀,先一日沐浴更衣,處外室。次日遣官。七年定制,凡大禮前期四日,太常卿至天下神祇壇奠告,中書丞相詣京師城隍廟發咨。次日,皇帝詣仁祖廟請配享。二十一年定制,齋戒前二日,太常司官宿於本司。次日,奏請致齋。又次日,進銅人,傳制諭文武百官齋戒。是日,禮部太常司官檄城隍神,遍請天下當祀神祇,仍於各廟焚香三日。

二十六年,定傳制誓戒儀。凡大祀前三日,百官詣闕,如大朝儀,傳制官宣制云:「某年月日,祀於某所,爾文武百官,自某日為始,致齋三日,當敬慎之。」傳制訖,四拜,奏禮畢。宣德七年,大祀南郊,帝御齋宮,命內官使飲酒食葷入壇唾地者,皆罪之,司禮監縱容者同罪。齋之日,御史檢視各官於齋次,仍行南京,一體齋戒。弘治五年,鴻臚少卿李燧言:「分獻陪祭等官,借居道士房榻,貴賤雜處,且宣召不便。乞於壇所隙地,仿天壽山朝房禮制,建齋房。」從之。嘉靖九年,定前期三日,帝御奉天殿,百官朝服聽誓戒。萬曆四年十一月,禮部以二十三日冬至祀天,十八日當奏祭,十九日百官受誓戒。是日,皇太后聖旦,百官宜吉服賀。一日兩遇禮文,服色不同,請更奏祭、誓戒皆先一日。帝命奏祭、誓戒如舊,而以十八日行慶賀禮。

○遣官祭祀

洪武二十六年,定傳制特遣儀。是日,皇帝升座如常儀,百官一拜。禮畢,獻官詣拜位四拜,傳制官由御前出宣制。如祭孔子,則曰:「某年月日,祭先師孔子大成至聖文宣王,命卿行禮。」祭歷代帝王,則曰:「某年月日,祭先聖歷代帝王,命卿行禮。」俯伏,興,四拜,禮畢出。其降香遣官儀,前祀一日清晨,皇帝皮弁服,陞奉天殿。捧香者以香授獻官。獻官捧由中陛降中道出,至午門外,置龍亭內。儀仗鼓吹,導引至祭所。後定祭之日,降香如常儀,中嚴以待。獻官祭畢後命,解嚴還宮。嘉靖九年大祀遣官,不行飲福禮。

○分獻陪祀

凡分獻官,太常寺豫請旨。洪武七年,太祖謂學士詹同曰:「大祀,終獻方行分獻禮,未當。」同乃與學士宋濂議以上,初獻奠玉帛將畢,分獻官即行初獻禮。亞獻、終獻皆如之。嘉靖九年,四郊工成,帝諭太常寺曰:「大祀分獻官豫定,方可習儀。」乃用大學士張璁等於大明、夜明、星辰、風雲雷雨四壇。舊制,分獻用文武大臣及近侍官共二十四人,今定四人,法司官仍舊例不興。

凡陪祀,洪武四年,太常寺引《周禮》及唐制,擬用武官四品、文官五品以上,其老疾瘡疥刑餘喪過體氣者不與。從之。後定郊祀,六科都給事中皆與陪祀,餘祭不與。又定凡南北郊,先期賜陪祀執事官明衣布,樂舞生各給新衣。制陪祀官入壇牙牌,凡天子親祀,則佩以入。其制有二,圓者與祭官佩之,方者執事人佩之。俱藏內府,遇祭則給,無者不得入壇。洪武二十九年,初祀山川諸神,流官祭服,未入流官公服。洪武二十九年,從禮臣言,未入流官,凡祭皆用祭服,與九品同。

郊祀郊祀配位郊祀儀注祈穀大雩大饗令節拜天

○郊祀之制

洪武元年,中書省臣李善長等奉敕撰進《郊祀議》,略言:

王者事天明,事地察,故冬至報天,夏至報地,所以順陰陽之義也。祭天於南郊之圜丘,祭地於北郊之方澤,所以順陰陽之位也。《周禮·大司樂》:「冬日至,禮天神,夏日至,禮地祇。」《禮》曰:「享帝於郊,祀社於國。」又曰:「郊所以明天道,社所以明地道。」《書》曰:「敢昭告於皇天后土。」按古者或曰地祇,或曰后土,或曰社,皆祭地,則皆對天而言也。此三代之正禮,而釋經之正說。自秦立四時,以祀白、青、黃、赤四帝。漢高祖復增北畤,兼祀黑帝。至武帝有雍五畤,及渭陽五帝、甘泉太乙之祠,而昊天上帝之祭則未嘗舉行,魏、晉以後,宗鄭玄者,以為天有六名,歲凡九祭。宗王肅者,以為天體惟一,安得有六?一歲二祭,安得有九?雖因革不同,大抵多參二家之說。自漢武用祠官寬舒議,立后土祠於汾陰脽上,禮如祀天。而後世因於北郊之外,仍祠后土。又鄭玄惑於緯書,謂夏至於方丘之上祭崑崙之祇,七月於泰折之壇祭神州之祇,析而為二。後世又因之一歲二祭。元始間,王莽奏罷甘泉泰畤,復長安南北郊。以正月上辛若丁,天子親合祀天地於南郊。由漢歷唐,千餘年間,皆因之合祭。其親祀北郊者,惟魏文帝、周武帝、隋高祖、唐玄宗四帝而已。宋元豐中,議罷合祭。紹聖、政和間,或分或合。高宗南渡以後,惟用合祭之禮。元成宗始合祭天地五方帝,已而立南郊,專祀天。泰定中,又合祭。文宗至順以後,惟祀昊天上帝。今當遵古制,分祭天地於南北郊。冬至則祀昊天上帝於圜丘,以大明、夜明、星辰、太歲從祀。夏至則祀皇地祇於方丘,以五嶽、五鎮、四海、四瀆從祀。

太祖如其議行之。建圜丘於鐘山之陽,方丘於鐘山之陰。三年,增祀風雲雷雨於圜丘,天下山川之神於方丘。七年,增設天下神祇壇於南北郊。九年,定郊社之禮,雖有三年喪,不廢。十年秋,太祖感齋居陰雨,覽京房災異之說,謂分祭天地,情有未安,命作大祀殿於南郊。是歲冬至,以殿工未成,乃合祀於奉天殿,而親製祝文,意謂人君事天地猶父母,不宜異處。遂定每歲合祀於孟春,為永制。十二年正月,始合祀於大祀殿,太祖親作《大祀文》並歌九章。永樂十八年,京都大祀殿成,規制如南京。南京舊郊壇,國有大事,則遣官告祭。

嘉靖九年,世宗既定《明倫大典》,益覃思制作之事,郊廟百神,咸欲斟酌古法,釐正舊章。乃問大學士張璁:「《書》稱燔柴祭天,又曰『類於上帝』,《孝經》曰:『郊祀后稷以配天,宗祀文王於明堂以配上帝』,以形體主宰之異言也。朱子謂,祭之於壇謂之天,祭之屋下謂之帝。今大祀有殿,是屋下之祭帝耳,未見有祭天之禮也。況上帝皇地祇合祭一處,亦非專祭上帝。」璁言:「國初遵古禮,分祭天地,後又合祀。說者謂大祀殿下壇上屋,屋即明堂,壇即圜丘,列聖相承,亦孔子從周之意。」帝復諭璁:「二至分祀,萬代不易之禮。今大祀殿擬周明堂或近矣,以為即圜丘,實無謂也。」璁乃備述《周禮》及宋陳襄、蘇軾、劉安世、程頤所議分合異同以對。且言祖制已定,無敢輕議。帝銳欲定郊制,卜之奉先殿太祖前,不吉。乃問大學士翟鑾,鑾具述因革以對。復問禮部尚書李時,時請少需日月,博選儒臣,議復古制。帝復卜之太祖,不吉,議且寢。

會給事中夏言請舉親蠶禮。帝以古者天子親耕南郊,皇后親蠶北郊,適與所議郊祀相表裏,因令璁諭言陳郊議。言乃上疏言:「國家合祀天地,及太祖、太宗之並配,諸壇之從祀,舉行不於長至而於孟春,俱不應古典。宜令群臣博考《詩》、《書》、《禮經》所載郊祀之文,及漢、宋諸儒匡衡、劉安世、朱熹等之定論,以及太祖國初分祀之舊制,陛下稱制而裁定之。此中興大業也。」禮科給事中王汝梅等詆言說非是,帝切責之。乃敕禮部令群臣各陳所見。且言:「汝梅等舉《召誥》中郊用二牛,謂明言合祭天地。夫用二牛者,一帝一配位,非天地各一牛也。又或謂天地合祀,乃人子事父母之道,擬之夫婦同牢。此等言論,褻慢已甚。又或謂郊為祀天,社稷為祭地。古無北郊,夫社乃祭五土之祇,猶言五方帝耳,非皇地祇也。社之名不同,自天子以下,皆得隨所在而祭之。故《禮》有『親地』之說,非謂祭社即方澤祭地也。」璁因錄上《郊祀考議》一冊。

時詹事霍韜深非郊議,且言分祀之說,惟見《周禮》,莽賊偽書,不足引據,於是言復上疏言:

《周禮》一書,於祭祀為詳。《大宗伯》以祀天神,則有禋祀、實柴、燎之禮,以祀地祇,則有血祭、薶沈、趯辜之禮。《大司樂》冬至日,地上圜丘之制,則曰禮天神,夏至日,澤中方丘之制,則曰禮地祇。天地分祀,從來久矣。故宋儒葉時之言曰:「郊丘分合之說,當以《周禮》為定。」今議者既以大社為祭地,則南郊自不當祭皇地祇,何又以分祭為不可也?合祭之說,實自莽始,漢之前皆主分祭,而漢之後亦間有之。宋元豐一議,元祐再議,紹聖三議,皆主合祭,而卒不可移者,以郊賚之費,每傾府藏,故省約安簡便耳,亦未嘗以分祭為禮也。今之議者,往往以太祖之制為嫌為懼。然知合祭乃太祖之定制,為不可改,而不知分祭固太祖之初制,為可復。知《大祀文》乃太祖之明訓,為不可背,而不知《存心錄》固太祖之著典,為可遵。且皆太祖之制也,從其禮之是者而已。敬天法祖,無二道也。《周禮》一書,朱子以為周公輔導成王,垂法後世,用意最深切,何可誣以莽之偽為耶?且合祭以后配地,實自莽始。莽既偽為是書,何不削去圜丘、方丘之制,天神地祇之祭,而自為一說耶?

於是禮部集上群臣所議郊禮,奏曰:「主分祭者,都御史汪鋐等八十二人,主分祭而以慎重成憲及時未可為言者,大學士張璁等八十四人,主分祭而以山川壇為方丘者,尚書李瓚等二十六人,主合祭而不以分祭為非者,尚書方獻夫等二百六人,無可否者,英國公張崙等一百九十八人。臣等祗奉敕諭,折衷眾論。分祀之義,合於古禮,但壇壝一建,工役浩繁。《禮》,屋祭曰帝,夫既稱昊天上帝,則當屋祭。宜仍於大祀殿專祀上帝,改山川壇為地壇,以專祀皇地祇。既無創建之勞,行禮亦便。」帝復諭當遵皇祖舊制,露祭於壇,分南北郊,以二至日行事。言乃奏曰:「南郊合祀,循襲已久,朱子所謂千五六百年無人整理。而陛下獨破千古之謬,一理舉行,誠可謂建諸天地而不悖者也。」

已而命戶、禮、工三部偕言等詣南郊相擇。南天門外有自然之丘,咸謂舊丘地位偏東,不宜襲用。禮臣欲於具服殿少南為圜丘。言復奏曰:「圜丘祀天,宜即高敞,以展對越之敬。大祀殿享帝,宜即清閟,以盡昭事之誠。二祭時義不同,則壇殿相去,亦宜有所區別。乞於具服殿稍南為大祀殿,而圜丘更移於前,體勢峻極,可與大祀殿等。」制曰「可」。於是作圜丘,是年十月工成。明年夏,北郊及東、西郊,亦以次告成,而分祀之制遂定。萬曆三年,大學士張居正等輯《郊祀新舊圖考》進呈。舊禮者,太祖所定。新禮者,世宗所定也。

○郊祀配位

洪武元年,始有事於南郊。有司議配祀。太祖謙讓不許,親為文告太廟曰:「歷代有天下者,皆以祖配天。臣獨不敢者,以臣功業有未就,政治有闕失。去年上天垂戒,有聲東南,雷火焚舟擊殿吻,早暮兢惕,恐無以承上帝好生之德,故不敢輒奉以配。惟祖神與天通,上帝有問,願以臣所行奏帝前,善惡無隱。候南郊竣事,臣率百司恭詣廟廷,告成大禮,以共享上帝之錫福。」明年夏至,將祀方丘,群臣復請。乃奉皇考仁祖淳皇帝配天於圜丘。明年祀方丘,亦如之。建文元年,改奉太祖配。洪熙改元,敕曰:「太祖受命上天,肇興皇業。太宗中興宗社,再奠寰區。聖德神功,咸配天地。《易》曰,『殷薦上帝,以配祖考』。朕崇敬祖考,永惟一心。正月十五日,大祀天地神祇,奉皇祖、皇考以配。」遂於郊祀前告太廟及几筵,請太祖、太宗並配。

嘉靖九年,給事中夏言上疏言:「太祖、太宗並配,父子同列,稽之經旨,未能無疑。臣謂周人郊祀后稷以配天,太祖足當之。宗祀文王於明堂以配上帝,太宗足當之。」禮臣集議,以為二祖配享,百有餘年,不宜一旦輕改。帝降敕諭,欲於二至日奉太祖配南、北郊,歲首奉太宗配上帝於大祀殿。於是大學士張璁、翟鑾等言,二祖分配,於義未協,且錄仁宗年撰敕諭並告廟文以進。帝復命集議於東閣,皆以為:「太廟之祀,列聖昭穆相向,無嫌並列。況太祖、太宗,功德並隆,圜丘、大祀殿所祀,均之為天,則配天之祖,不宜闕一。臣等竊議南、北郊及大祀殿,每祭皆宜二祖並配。」帝終以並配非禮,諭閣臣講求。璁等言:「《禮》曰:『有其舉之,莫敢廢也。』凡祭盡然,況祖宗配享大典?且古者郊與明堂異地,故可分配。今圜丘、大祀殿同兆南郊,冬至禮行於報而太宗不與,孟春禮行於祈而太祖不與,心實有所不安。」帝復報曰:「萬物本乎天,人本乎祖。天惟一天,祖亦惟一祖。故大報天之祀,止當以高皇帝配。文皇帝功德,豈不可配天?但開天立極,本高皇帝肇之耳。如周之王業,武王實成之,而配天止以后稷,配上帝止以文王,當時未聞爭辨功德也。」因命寢其議。已而夏言復疏言:「虞、夏、殷、周之郊,惟配一祖。後儒穿鑿,分郊丘為二,及誤解《大易》配考、《孝經》嚴父之義。以致唐、宋變古,乃有二祖並侑,三帝並配之事。望斷自宸衷,依前敕旨。」帝報曰:「禮臣前引太廟不嫌一堂。夫祀帝與享先不同,此說無當。」仍命申議。於是禮臣復上議:「南北郊雖曰祖制,實今日新創。請如聖諭,俱奉太祖獨配。至大祀殿則太祖所創,今乃不得侑享於中,恐太宗未安,宜仍奉二祖並配。」遂依擬行之。

○郊祀儀注

洪武元年冬至,祀昊天上帝於圜丘。先期,皇帝散齋四日,致齋三日。前祀二日,皇帝服通天冠、絳紗袍省牲器。次日,有司陳設。祭之日,清晨車駕至大次,太常卿奏中嚴,皇帝服袞冕。奏外辦,皇帝入就位,贊禮唱迎神。協律郎舉麾奏《中和之曲》。贊禮唱燔柴,郊社令升煙,燔全犢於燎壇。贊禮唱請行禮,太常卿奏有司謹具,請行事。皇帝再拜,皇太子及在位官皆再拜。贊禮唱奠玉帛,皇帝詣盥洗位。太常卿贊曰:「前期齋戒,今辰奉祭,加其清潔,以對神明。」皇帝搢圭,盥手,帨手。出圭,升壇。太常卿贊曰:「神明在上,整肅威儀。」升自午陛。協律郎舉麾奏《凝和之曲》。皇帝詣昊天上帝神位前跪,搢圭,三上香,奠玉帛,出圭,再拜復位。贊禮唱進俎,協律郎舉麾奏《凝和之曲》。皇帝詣神位前,搢圭奠俎,出圭,復位。贊禮唱行初獻禮。皇帝詣爵洗位,搢圭,滌爵,拭爵,以爵授執事者,出圭。詣酒尊年,搢圭,執爵,受泛齊,以爵授執事者,出圭。協律郎舉麾奏《壽和之曲》、《武功之舞》。皇帝詣神位前跪,搢圭,上香,祭酒,奠爵,出圭。讀祝官捧祝跪讀訖,皇帝俯伏,興,再拜,復位。亞獻,酌醴齊,樂奏《豫和之曲》、《文德之舞》。終獻,酌盎齊,樂奏《熙和之曲》、《文德之舞》。儀並同初獻,但不用祝。贊禮唱飲福受胙,皇帝升壇,至飲福位,再拜,跪,搢圭。奉爵官酌福酒跪進,太常卿贊曰:「惟此酒肴,神之所與,賜以福慶,億兆同沾。」皇帝受爵,祭酒,飲福酒,以爵置於坫。奉胙官奉胙跪進,皇帝受胙,以授執事者,出圭,俯伏,興,再拜,復位。皇太子以下在位官皆再拜。贊禮唱徹豆,協律郎舉麾奏《雍和之曲》,掌祭官徹豆。贊禮唱送神,協律郎舉麾奏《安和之曲》。皇帝再拜,皇太子以下在位官皆再拜。贊禮唱讀祝官奉祝,奉幣官奉幣,掌祭官取饌及爵酒,各詣燎所。唱望燎,皇帝至望燎位。半燎,太常卿奏禮畢,皇帝還大次,解嚴。

二年夏至,祀皇地祇於方丘,其儀並同。惟迎神後瘞毛血,祭畢,奉牲帛祝饌而埋之,與郊天異。其冬,奉仁祖配天於南郊,儀同元年。其奠玉帛、進俎、三獻,皆先詣上帝前,次詣仁祖神位前,行禮亦如之,惟不用玉。四年定,先祭六日,百官沐浴宿官署。翼日,朝服詣奉天殿丹墀,受誓戒。丞相以祀期遍告百神,後詣各祠廟行香三日。次日,駕詣仁祖廟,告請配享。禮畢,還齋宮。七年,去中嚴、外辦及贊唱上香之縟節,定十二拜禮。十年,改合祀之制,奠玉帛、進俎、三獻,俱先詣上帝神位前,次皇地祇,次仁祖,餘悉仍舊儀。

嘉靖八年,罷各廟焚香禮。九年,復分祀之制,禮部上大祀圓丘儀注:前期十日,太常寺題請視牲。次請命大臣三員看牲,四員分獻。前期五日,錦衣衛備隨朝駕,帝詣犧牲所視牲。其前一日,常服告於廟。前期四日,御奉天殿,太常寺進銅人如常儀。太常博士請太祖祝版於文華殿,候帝親填御名捧出。前期三日,帝具祭服,以脯醢酒果詣太廟,請太祖配。帝還易服,御奉天殿,百官朝服受誓戒。前期二日,太常光祿卿奏省牲,帝至奉天殿親填祝版。前期一日免朝,錦衣衛備法駕,設版輿於奉天門。常服告廟,乘輿詣南郊,由西天門入,至昭亨門外降輿。禮部太常官導由左門入,至內壝。太常卿導至圜丘,恭視壇位,次至神庫視籩豆,至神廚視牲畢,仍由左門出,升輿,至齋宮。分獻陪祀官叩首,禮部太常官詣皇穹宇,請皇天上帝神版、太祖神主、從祀神牌,奉安壇座。祭之日,三鼓,帝自齋宮乘輿至外壝神路之西,降輿至神路東大次。禮部、太常寺捧神位官復命訖,退。百官分列神路東西以候。帝具祭服出,導引官導由左靈星門入內。贊對引官導行至內壝,典儀唱樂舞生就位,執事官各司其事。帝至御拜位,自燔柴、迎神至禮畢,其儀悉如舊。至大次易服,禮部太常官奉神位安於皇穹宇。還齋宮,少憩。駕還,詣廟參拜畢。回宮。詔如擬。

明年,定方澤儀:先期一日,太常卿請太祖配位,奉安皇祇室。至期,禮部太常官同請皇地祇神版、太祖神主、從祀神牌,奉安壇座。祀畢,太常奉神版、神牌安皇祇室,奉神主還廟寢。餘皆如圜丘儀。

是年十月,帝將郊祀,諭禮部尚書夏言欲親行奉安禮。言乃擬儀注以聞:先期擇捧主執事官十一員,分獻配殿大臣二員,撰祝文,備脯醢、酒果、制帛、香燭。前一日行告廟禮,設神輿香案於奉天殿,神案二於泰神殿,神案二於東西配殿,香案一於丹墀正中,設大次於圜丘左門外。是日質明,帝常服詣奉天殿,行一拜三叩頭禮。執事官先後捧昊天上帝、太祖高皇帝及從祀神主,各奉安輿中,至圜丘泰神殿門外。帝乘輅至昭亨門,禮官導至泰神殿丹墀。執事官就神輿捧神主升石座,奉安於龕中。帝乃詣香案前,行三獻禮如儀。禮畢,出至大次升座,百官行一拜三叩頭禮畢,還宮。帝從之,而命行禮用祭服,導引用太常寺官一員,合禮部堂上官四員。十一年冬至,尚書言,前此有事南郊,風寒莫備。乃采《禮書》天子祀天張大次、小次之說,請「作黃氈御幄為小次。每大祭,所司以隨。值風雪,則設於圜丘下,帝就幄中對越,而陟降奠獻以太常執事官代之」。命著為令。

○祈穀

明初末嘗行。世宗時,更定二祖分配禮。因諸臣固請,乃許於大祀殿祈穀,奉二祖配。嘉靖十年,始以孟春上辛日行祈穀禮於大祀殿。禮畢,帝心終以為未當,諭張璁曰:「自古惟以祖配天,今二祖並配,決不可法後世。嗣後大報與祈穀,但奉太祖配。」尋親製祝文,更定儀注,改用驚蟄節,禮視大祀少殺。帛減十一,不設從壇,不燔柴,著為定式。十一年驚蟄節,帝疾,不能親,乃命武定侯郭勛代。給事中葉洪言:「祈穀、大報,祀名不同,郊天一也。祖宗無不親郊。成化、弘治間,或有故,寧展至三月。蓋以郊祀禮重,不宜攝以人臣,請俟聖躬痊,改卜吉日行禮。」不從。十八年,改行於大內之玄極寶殿,不奉配,遂為定制。隆慶元年,禮臣言:「先農親祭,遂耕耤田,即祈穀遺意。今二祀並行於春,未免煩數。且玄極寶殿在禁地,百官陪祀,出入非便。宜罷祈穀,止先農壇行事。」從之。

○大雩

明初,凡水旱災傷及非常變異,或躬禱,或露告於宮中,或於奉天殿陛,或遣官祭告郊廟、陵寢及社稷、山川,無常儀。嘉靖八年,春祈雨,冬祈雪,皆御製祝文,躬祀南郊及山川壇。次日,祀社稷壇。冠服淺色,鹵簿不陳,馳道不除,皆不設配,不奏樂。九年,帝欲於奉天殿丹陛上行大雩禮。夏言言:「按《左傳》『龍見而雩』。蓋巳月萬物始盛,待雨而大,故祭天為百穀祈膏雨也。《月令》:『雩帝用盛樂,乃命百縣雩祀,祀百辟卿士有益於民者,以祈穀實。』《通典》曰:『巳月雩五方上帝,其壇名雩,禜於南郊之傍。』先臣丘濬亦謂:『天子於郊天之外,別為壇以祈雨者也。後世此禮不傳,遇有旱,輒假異端之人為祈禱之事,不務以誠意感格,而以法術劫制,誣亦甚矣。』浚意欲於郊傍擇地為雩壇,孟夏後行禮。臣以為孟春既祈穀矣,茍自二月至四月,雨暘時若,則大雩之祭,可遣官攝行。如雨澤愆期,則陛下躬行禱祝。」乃建崇雩壇於圜丘壇外泰元門之東,為制一成,歲旱則禱,奉太祖配。

十二年,夏言等言:「古者大雩之祀,命樂正習盛樂、舞皇舞。蓋假聲容之和,以宣陰陽之氣。請於三獻禮成之後,九奏樂止之時,樂奏《雲門之舞》。仍命儒臣括《雲漢》詩詞,制《雲門》一曲,使文武舞士並舞而歌之。蓋《雲門》者,帝堯之樂,《周官》以祀天神,取雲出天氣,雨出地氣也。且請增鼓吹數番,教舞童百人,青衣執羽,繞壇歌《雲門之曲》而舞,曲凡九成。」因上其儀,視祈穀禮。又言:「大雩乃祀天禱雨之祭。凡遇亢旱,則禮部於春末請行之。」帝從其議。十七年,躬禱於壇,青服。用一牛,熟薦。

○大饗禮

明初無明堂之制。嘉靖十七年六月,致仕揚州府同知豐坊上疏言:「孝莫大於嚴父,嚴父莫大於配天。請復古禮,建明堂。加尊皇考獻皇帝廟號稱宗,以配上帝。」下禮部會議。尚書嚴嵩等言:

昔羲、農肇祀上帝,或為明堂。嗣是夏后氏世室,殷人重屋,周人作明堂之制,視夏、殷加詳焉。蓋聖王事天,如子事父,體尊而情親。故制為一歲享祀之禮,冬至圜丘,孟春祈穀,孟夏雩壇,季秋明堂,皆所以尊之也。明堂帝而享之,又以親之也。今日創制,古法難尋,要在師先王之意。明堂圜丘,皆所以事天,今大祀殿在圜丘之北,禁城東西,正應古之方位。明堂秋享,即以大祀殿行之為當。至配侑之禮,昔周公宗祀文王於明堂,詩傳以為物成形於帝,猶人成形於父。故季秋祀帝明堂,而以父配之,取其物之時也。漢孝武明堂之享,以景帝配,孝章以光武配,唐中宗以高宗配,明皇以睿宗配,代宗以肅宗配,宋真宗以太宗配,仁宗以真宗配,英宗以仁宗配,皆世以遞配,此主於親親也。宋錢公輔曰:「郊之祭,以始封之祖,有聖人之功者配焉。明堂之祭,以繼體之君,有聖人之德者配焉。」當時司馬光、孫抃諸臣執辨於朝,程、朱大賢倡議於下,此主於祖宗之功德也。今復古明堂大享之制,其所當配之帝,亦惟二說而已。若以功德論,則太宗再造家邦,功符太祖,當配以太宗。若以親親論,則獻皇帝陛下之所自出,陛下之功德,即皇考之功德,當配以獻皇帝。至稱宗之說,則臣等不敢妄議。

帝降旨:「明堂秋報大禮,於奉天殿行,其配帝務求畫一之說。皇考稱宗,何為不可?再會議以聞。」於是戶部左侍郎唐胄抗疏言:

三代之禮,莫備於周。《孝經》曰:「郊祀后稷以配天,宗祀文王於明堂以配上帝。」又曰:「嚴父莫大於配天,則周公其人也。」說者謂周公有聖人之德,制作禮樂,而文王適其父,故引以證聖人之孝,答曾子問而已。非謂有天下者皆必以父配天,然後為孝。不然,周公輔成王踐阼,其禮蓋為成王而制,於周公為嚴父,於成王則為嚴祖矣。然周公歸政之後,未聞成王以嚴父之故,廢文王配天之祭,而移於武王也。後世祀明堂者,皆配以父,此乃誤《孝經》之義,而違先王之禮。昔有問於朱熹曰:「周公之後,當以文王配耶,當以時王之父配耶?」熹曰:「只當以文王為配。」又曰:「繼周者如何?」熹曰:「只以有功之祖配,後來第為嚴父說所惑亂耳。」由此觀之,明堂之配,不專於父明矣。今禮臣不能辨嚴父之非,不舉文、武、成、康之盛,而乃濫引漢、唐、宋不足法之事為言,謂之何哉!雖然,豐坊明堂之議,雖未可從,而明堂之禮,則不可廢。今南、北兩郊皆主尊尊,必季秋一大享帝,而親親之義始備。自三代以來,郊與明堂各立所配之帝。太祖、大宗功德並盛,比之於周,太祖則后稷也,太宗則文王也。今兩郊及祈穀,皆奉配太祖,而太宗獨未有配。甚為缺典。故今奉天殿大享之祭,必奉配太宗,而後我朝之典禮始備。

帝怒,下胄詔獄。嵩乃再會廷臣,先議配帝之禮,言:「考季秋成物之指,嚴父配天之文,宜奉獻皇帝配帝侑食。」因請奉文皇帝配祀於孟春祈穀。帝從獻皇配帝之請,而卻文皇議不行。已復以稱宗之禮,集文武大臣於東閣議,言:「《禮》稱:『祖有功,宗有德。』釋者曰:『祖,始也。宗,尊也。』《漢書注》曰:『祖之稱始,始受命也。宗之稱尊,有德可尊也。』《孝經》曰:『宗祀文王於明堂,以配上帝。』王肅注曰:『周公於文王,尊而祀之也。』此宗尊之說也。古者天子七廟。劉歆曰:「七者正法,茍有功德則宗之,不可預為設數。宗不在數中,宗變也。』朱熹亦以歆之說為然。陳氏《禮書》曰:『父昭子穆,而有常數者,禮也。祖功宗德,而無定法者,義也。』此宗無數之說,禮以義起者。今援據古義,推緣人情,皇考至德昭聞,密佑穹旻,宗以其德可。聖子神孫,傳授無疆,皆皇考一人所衍布,宗以其世亦可。宜加宗皇考,配帝明堂,永為有德不遷之廟。」帝以疏不言祔廟,留中不下,乃設為臣下奏對之詞,作《明堂或問》,以示輔臣。大略言:「文皇遠祖,不應嚴父之義,宜以父配。稱宗雖無定說,尊親崇上,義所當行。既稱宗,則當祔廟,豈有太廟中四親不具之禮?」帝既排正議,崇私親,心念太宗永無配享,無以謝廷臣,乃定獻皇配帝稱宗,而改稱太宗號曰成祖。時未建明堂,迫季秋。遂大享上帝於玄極寶殿,奉睿宗獻皇帝配。殿在宮右乾隅,舊名欽安殿。禮成,禮部請帝升殿,百官表賀,如郊祀慶成儀。帝以大享初舉,命賜宴群臣於謹身殿。已而以足疾不御殿,命群臣勿行賀禮。禮官以表聞,並罷宴,令光祿寺分給。

二十一年,敕諭禮部:「季秋大享明堂,成周禮典,與郊祀並行。曩以享地未定,特祭於玄極寶殿,朕誠未盡。南郊舊殿,原為大祀所,昨歲已令有司撤之。朕自作制象,立為殿,恭薦名曰泰享,用昭寅奉上帝之意。」乃定歲以秋季大享上帝,奉皇考睿宗配享。行禮如南郊,陳設如祈穀。明年,禮部尚書費寀以大享殿工將竣,請帝定殿門名,門曰大享,殿曰皇乾。及殿成,而大享仍於玄極寶殿,遣官行禮以為常。隆慶元年,禮臣言:「我朝大享之禮,自皇考舉行,追崇睿宗,以昭嚴父配天之孝。自皇上視之,則睿宗為皇祖,非周人宗祀文王於明堂之義。」於是帝從其請,罷大享禮,命玄極寶殿仍為欽安殿。

○令節拜天

嘉靖初,沿先朝舊儀,每日宮中行拜天禮。後以為瀆,罷之。遇正旦、冬至、聖誕節,於奉天殿丹陛上行禮。既定郊祀,遂罷冬至之禮。惟正旦、聖誕節行禮於玄極寶殿。隆慶元年正旦,命宮中拜天,不用在外執事,祭品亦不取供於太常。

社稷朝日夕月先農先蠶高禖祭告祈報神祇星辰靈星壽星司中司命司民司祿太歲月將風雲雷雨嶽鎮海瀆山川城隍

○社稷

社稷之祀,自京師以及王國府州縣皆有之。其壇在宮城西南者,曰太社稷。明初建太社在東,太稷在西,壇皆北向。洪武元年,中書省臣定議:「周制,小宗伯掌建國之神位,右社稷,左宗廟。社稷之祀,壇而不屋。其制在中門之外,外門之內。尊而親之,與先祖等。然天子有三社。為群姓立者曰太社。其自為立者曰王社。又勝國之社屋之,國雖亡而存之,以重神也。後世天子惟立太社、太稷。漢高祖立官太社、太稷,一歲各再祀。光武立太社稷於洛陽宗廟之右,春秋二仲月及臘,一歲三祀。唐因隋制,並建社稷於含光門右,仲春、秋戊日祭之。玄宗升社稷為大祀,仍令四時致祭。宋制如東漢時。元世祖營社稷於和義門內,以春秋二仲上戊日祭。今宜祀以春秋二仲月上戊日。」是年二月,太祖親祀太社、太稷。社配以后土,西向。稷配以后稷,東向。帝服皮弁服,省牲;通天冠、絳紗袍,行三獻禮。初,帝命中書省翰林院議創屋,備風雨。學士陶安言:「天子太社必受風雨霜露。亡國之社則屋之,不受天陽也。建屋非宜。若遇風雨,則請於齋宮望祭。」從之。三年,於壇北建祭殿五間,又北建拜殿五間,以備風雨。

十年,太祖以社稷分祭,配祀未當,下禮官議,尚書張籌言:

按《通典》,顓頊祀共工氏子句龍為后土。后土,社也。烈山氏子柱為稷。稷,田正也。唐、虞、夏因之。此社稷所由始也。商湯因旱遷社,以后稷代柱。欲遷句龍,無可繼者,故止。然王肅謂社祭句龍,稷祭后稷,皆人鬼,非地祇。而陳氏《禮書》又謂社祭五土之祇,稷祭五穀之神。鄭康成亦謂社為五土總神,稷為原隰之神。句龍有平水土功,故配社,后稷有播種功,故配稷。二說不同。漢元始中,以夏禹配官社,后稷配官稷。唐、宋及元又以句龍配社,周棄配稷。此配祀之制,初無定論也。至社稷分合之義,《書召誥》言『社於新邑」,孔註曰:「社稷共牢。」《周禮》「封人掌設王之社壝」,註云:「不言稷者,舉社則稷從之。」陳氏《禮書》曰:「稷非土無以生,土非稷無以見生生之效,故祭社必及稷。」《山堂考索》曰:「社為九土之尊,稷為五穀之長,稷生於土,則社與稷固不可分。」其宜合祭,古有明證。請社稷共為一壇。至句龍,共工氏之子也,祀之無義。商湯欲遷未果。漢嘗易以夏禹,而夏禹今已列祀帝王之次,棄稷亦配先農。請罷句龍、棄配位,謹奉仁祖淳皇帝配享,以成一代盛典。遂改作於午門之右,社稷共為一壇。

初,社稷列中祀,及以仁祖配,乃升為上祀。具冕服以祭,行奉安禮。十一年春,祭社稷行新定儀。迎神、飲福、送神凡十二拜,餘如舊。建文時,更奉太祖配,永樂中。北京社稷壇成,制如南京。洪熙後,奉太祖、太宗同配。舊制,上丁釋奠孔子,次日上戊祀社稷。弘治十七年八月,上丁在初十日,上戊在朔日,禮官請以十一日祀社稷。御史金洪劾之,言如此則中戊,非上戊矣。禮部覆奏言:「洪武二十年嘗以十一日為上戊,失不始今日。」命遵舊制,仍用上戊。

嘉靖九年諭禮部:「天地至尊,次則宗廟,又次則社稷。今奉祖配天,又奉祖配社,此禮官之失也。宜改從皇祖舊制,太社以句龍配,太稷以后稷配。」乃以更正社稷壇配位禮,告太廟及社稷,遂藏二配位於寢廟,更定行八拜禮。其壇在西苑豳風亭之西者,曰帝社稷。東帝社,西帝稷,皆北向。始名西苑土穀壇。嘉靖十年,帝謂土穀壇亦社稷耳,何以別於太社稷?張璁等言:「古者天子稱王,今若稱王社、王稷,與王府社稷名同。前定神牌曰五土穀之神,名義至當。」帝採帝耤之義,改為帝社、帝稷,以上戊明日祭。後改次戊,次戊在望後,則仍用上巳。春告秋報為定制。隆慶元年,禮部言:「帝社稷之名,自古所無,嫌於煩數,宜罷。」從之。

中都亦有太社壇,洪武四年建。取五方土以築。直隸、河南進黃土,浙江、福建、廣東、廣西進赤土,江西、湖廣、陜西進白土,山東進青土,北平進黑土。天下府縣千三百餘城,各土百斤,取於名山高爽之地。

王國社稷,洪武四年定。十一年,禮臣言:「太社稷既同壇合祭,王國各府州縣亦宜同壇,稱國社國稷之神,不設配位。」詔可。十三年九月,復定制兩壇一壝如初式。十八年,定王國祭社稷山川等儀,行十二拜禮。

府州縣社稷,洪武元年頒壇制於天下郡邑,俱設於本城西北,右社左稷。十一年,定同壇合祭如京師。獻官以守禦武臣為初獻,文官為亞獻、終獻。十三年,溧水縣祭社稷,以牛醢代鹿醢。禮部言:「定制,祭物缺者許以他物代。」帝曰:「所謂缺者,以非土地所產。溧水固有鹿,是有司故為茍簡也。百司所以能理其職而盡民事者,以其常存敬懼之心耳。神猶忽之,於人事又何懼焉!」命論如律。乃敕禮部下天下郡邑,凡祭祀必備物,茍非地產、無從市鬻者,聽其缺。十四年,令三獻皆以文職長官,武官不與。

里社,每里一百戶立壇一所,祀五土五穀之神。

○朝日夕月

洪武三年,禮部言:

古者祀日月之禮有六。《郊特牲》曰:「郊之祭,大報天而主日,配以月」,一也。《玉藻》曰:「翰日於東門之外」,《祭義》曰:「祭日於東郊,祭月於西郊」,二也。《小宗伯》:「肆類於四郊,兆日於東郊,兆月於西郊」,三也。《月令》:孟冬「祈來年於天宗」,天宗,日月之類,四也。《覲禮》:「拜日於東門之外,反祀方明,禮日於南門之外,禮月於北門之外」,五也。「霜雪風雨之不時,則禜日月」,六也。說者謂因郊祀而祀之,非正祀也。類禜而祀之,與覲諸侯而禮之,非常祀也。惟春分朝之於東門外,秋分夕之於西門外者,祀之正與常也。蓋天地至尊,故用其始而祭以二至。日月次天地,春分陽氣方永,秋分陰氣向長,故祭以二分,為得陰陽之義。自秦祭八神,六曰月主,七曰日主,雍又有日月廟。漢郊太乙,朝日夕月改周法。常以郊泰畤,質明出行宮,東向揖日,西向揖月,又於殿下東西拜日月。宣帝於神山祠日,萊山祠月。魏明帝始朝日東郊,夕月西郊。唐以二分日,朝日夕月於國城東西。宋人因之,升為大祀。元郊壇以日月從祀,其二分朝日夕月,皇慶中議建立而未行。今當稽古正祭之禮,各設壇專祀。朝日壇宜築於城東門外,夕月壇宜築於城西門外。朝日以春分,夕月以秋分。星辰則祔祭於月壇。從之。其祀儀與社稷同。二十一年,帝以大明、夜明已從祀,罷朝日夕月之祭。嘉靖九年,帝謂「大報天而主日,配以月。大明壇當與夜明壇異。且日月照臨,其功甚大。太歲等神,歲有二祭,而日月星辰止一從祭,義所不安」。大學士張璁亦以為缺典。遂定額春秋分之祭如舊儀,而建朝日壇於朝陽門外,西向;夕月壇於阜城門外,東向。壇制有隆殺以示別。朝日,護壇地一百畝;夕月,護壇地三十六畝。朝日無從祀,夕月以五星、二十八宿、周天星辰共一壇,南向祔焉。春祭,時以寅,迎日出也。秋祭,時以亥,迎月出也。十年,禮部上朝日、夕月儀:朝日迎神四拜,飲福受胙兩拜,送神四拜;夕月迎神飲福受胙送神皆再拜。餘並如舊儀。隆慶元年,禮部議定,東郊以甲、丙、戊、庚、壬年,西郊以丑、辰、未、戌年,車駕親祭。餘歲遣文大臣攝祭朝日壇,武大臣攝祭夕月壇。三年,禮部上朝日儀,言:「正祭遇風雨,則設小次於壇前,駕就小次行禮。其升降奠獻,俱以太常寺執事官代。」制曰「可」。

○先農

洪武元年,諭廷臣以來春舉行耤田禮。於是禮官錢用壬等言:「漢鄭玄謂王社在耤田之中。唐祝欽明云:「先農即社。」宋陳祥道謂:「社自社,先農自先農。耤田所祭乃先農,非社也。至享先農與躬耕同日,禮無明文,惟《周語》曰:「農正陳耤禮。」而韋昭注云:「祭其神為農祈也。」至漢以耤田之日祀先農,而其禮始著。由晉至唐、宋相沿不廢。政和間,命有司享先農,止行親耕之禮。南渡後,復親祀。元雖議耕耤,竟不親行。其祀先農,命有司攝事。今議耕耤之日,皇帝躬祀先農。禮畢,躬耕耤田。以仲春擇日行事。」從之。

二年二月,帝建先農壇於南郊,在耤田北。親祭,以后稷配。器物祀儀與社稷同。祀畢,行耕耤禮。御耒耜二具,韜以青絹,御耕牛四,被以青衣。禮畢,還大次。應天府尹及上元、江寧兩縣令率庶人終畝。是日,宴勞百官耆老於壇所,十年二月,遣官享先農,命應天府官率農民耆老陪祀。二十一年,更定祭先農儀,不設配位。

永樂中,建壇京師,如南京制,在太歲壇西南。石階九級。西瘞位,東齋宮、鑾駕庫,東北神倉,東南具服殿,殿前為觀耕之所。護壇地六百畝,供黍稷及薦新品物地九十餘畝。每歲仲春上戊,順天府尹致祭。後凡遇登極之初,行耕耤禮,則親祭。

弘治元年,定耕耤儀:前期百官致齋。順天府官以耒耜及穜AL種進呈,內官仍捧出授之,由午門左出。置彩輿,鼓樂,送至耤田所。至期,帝翼善冠黃袍,詣壇所具服殿,服袞冕,祭先農。畢,還,更翼善冠黃袍。太常卿導引至耕耤位,南向立。三公以下各就位,戶部尚書北向跪進耒耜,順天府官北向跪進鞭。帝秉耒,三推三反訖,戶部尚書跪受耒耜,順天府官跪受鞭,太常卿奏請復位。府尹挾青箱以種子播而覆之。帝御外門,南向坐,觀三公五推,尚書九卿九推。太常卿奏耕畢,帝還具服殿,升座。府尹率兩縣令耆老人行禮畢,引上中下農夫各十人,執農器朝見,令其終畝。百官行慶賀禮,賜酒饌。三品以上丹陛上東西坐,四品以下臺下坐,並宴勞耆老於壇旁。宴畢,駕還宮。大樂鼓吹振作,農夫人賜布一匹。

嘉靖十年,帝以其禮過煩,命禮官更定。迎神送神止行二拜。先二日,順天府尹以耒耜穜AL種置綵輿,至耕耤所,並罷百官慶賀。後又議造耕根車載耒耜,府尹於祭日進呈畢,以耒耜載車內前玉輅行。其御門觀耕,地位卑下,議建觀耕臺一。詔皆可。後又命墾西苑隙地為田。建殿曰無逸,亭曰豳風,又曰省耕,曰省斂,倉曰恒裕。禮部上郊廟粢盛支給之數,因言:「南郊耤田,皇上三推,公卿各宣其力,較西苑為重。西苑雖農官督理,皇上時省耕斂,較耤田為勤。請以耤田所出,藏南郊圓廩神倉,以供圜丘、祈穀、先農、神祇壇、長陵等陵、歷代帝王及百神之祀。西苑所出,藏恒裕倉,以供方澤、朝日、夕月、太廟、世廟、太社稷、帝社稷、禘佩、先蠶及先師孔子之祀。」從之。十六年,諭凡遇親耕,則戶部尚書先祭先農。皇帝至,止行三推禮。三十八年,罷親耕,惟遣官祭先農。四十一年,並令所司勿復奏。隆慶元年,罷西苑耕種諸祀,皆取之耤田。

○先蠶

明初未列祀典。嘉靖時,都給事中夏言請改各宮莊田為親蠶廠公桑園。令有司種桑柘,以備宮中蠶事。九年,復疏言,耕蠶之禮,不宜偏廢。帝乃敕禮部:「古者天子親耕,皇后親蠶,以勸天下。自今歲始,朕親祀先農,皇后親蠶,其考古制,具儀以聞。」大學士張璁等請於安定門外建先蠶壇。詹事霍韜以道遠爭之。戶部亦言:「安定門外近西之地,水源不通,無浴蠶所。皇城內西苑中有太液、瓊島之水。考唐制在苑中,宋亦在宮中,宜仿行之。」帝謂唐人因陋就安,不可法。於是禮部尚書李時等言:「大明門至安定門,道路遙遠,請鳳輦出東華、玄武二門。」因條上四事:一、治繭之禮,二、壇壝之向,三、採桑之器,四、掌壇之官。帝從其言,命自玄武門出。內使陳儀衛,軍一萬人,五千圍壇所,五千護於道,餘如議。

二月,工部上先蠶壇圖式,帝親定其制。壇方二丈六尺,疊二級,高二尺六寸,四出陛。東西北俱樹桑柘,內設蠶宮令署。採桑臺高一尺四寸,方十倍,三出陛。鑾駕庫五間。後蓋織堂。壇圍方八十丈。禮部上皇后親蠶儀:蠶將生,欽天監擇吉巳日以聞。順天府具蠶母名數送北郊,工部以鉤箔筐架諸器物給蠶母。順天府以蠶種及鉤筐一進呈,內官捧出,還授之。出玄武右門,置彩輿中,鼓樂送至蠶室。蠶母受蠶種,浴飼以待。命婦文四品、武三品以上俱陪祀,攜一侍女執鉤筐。皇后齋三日,內執事並司贊、六尚等女官及應入壇者齋一日。先一日,太常寺具祝版,祭物,羊、豕、籩豆各六、黑帛,送蠶宮令。是日,分授執事女官。日未明。宿衛陳兵備,女樂司設監備儀仗及重翟車,俱候玄武門外。將明,內侍詣坤寧宮奏請。皇后服常服,導引女官導出宮門,乘肩輿,至玄武門。內侍奏請降輿,升重翟車。兵衛儀仗及女樂前導,出北安門,障以行帷,至壇內壝東門。內侍奏請降車,乘肩輿,兵衛、儀仗停東門外。皇后入具服殿,易禮服,出,至壇。司贊奏就位。公主、內外命婦各就拜位。祭先蠶,行三獻禮,女官執事如儀。迎神四拜,賜福胙二拜,送神四拜。凡拜跪興,公主、內外命婦皆同。禮畢,皇后還具服殿,更常服。司賓引外命婦先詣採桑壇東陛下,南北向。尚儀奏請,皇后詣採桑位,東向。公主以下位皇后位東,亦南北向,以西為上。執鉤者跪進鉤,執筐者跪奉筐受桑。皇后採桑三條,還至壇南儀門坐,觀命婦採桑。三公命婦採五條,列侯、九卿命婦採九條。訖,各授女侍。司賓引內命婦一人,詣桑室,尚功率執鉤筐者從。尚功以桑授蠶母。蠶母受桑,縷切之,以授內命婦。內命婦食蠶,灑一箔訖,還。尚儀奏禮畢,皇后還坐具服殿。司賓率蠶母等叩頭訖,司贊唱班齊。外名婦序立定,尚儀致詞云:「親蠶既成,禮當慶賀。」四拜畢,賜宴命婦,並賜蠶母酒食。公主及內命婦於殿內,外命婦文武二品以上於臺上,三品以下於丹墀,尚食進膳。教坊司女樂奏樂。宴畢,公主以下各就班四拜。禮畢,皇后還宮,導從前。詔如擬。

四月,蠶事告成,行治繭禮。選蠶婦善繅絲及織者各十人。卜日,皇后出宮,導從如常儀,至織堂。內命婦一人行三盆手禮,布於織婦,以終其事。蠶宮令送尚衣織染監局造祭服,其祀先蠶,止用樂,不用舞,樂女生冠服俱用黑。

十年二月,禮臣言:「去歲皇后躬行採桑,已足風勵天下。今先蠶壇殿工未畢,宜且遣官行禮。」帝初不可,令如舊行。已而以皇后出入不便,命改築先蠶壇於西苑。壇之東為採桑臺,臺東為具服殿,北為蠶室,左右為廂房,其後為從室,以居蠶婦。設蠶宮署於宮左,令一員,丞二員,擇內臣謹恪者為之。四月,皇后行親蠶禮於內苑。帝謂親耕無賀,此安得賀,第行叩頭禮,女樂第供宴,勿前導。三十八年罷,親蠶禮。四十一年,並罷所司奏請。

○高禖

嘉靖九年,青州儒生李時颺請祠高禖,以祈聖嗣。禮官覆以聞。帝曰:「高禖雖古禮,今實難行。」遂寢其議。已而定祀高禖禮。設木臺於皇城東,永安門北,震方。臺上,皇天上帝南向,騂犢,蒼壁。獻皇帝配,西向,牛羊豕各一。高禖在壇下西向,牲數如之,禮三獻。皇帝位壇下北向,後妃位南數十丈外,北向,用帷。壇下陳弓矢、弓韣如後妃嬪之數。祭畢,女官導后妃嬪至高禖前,跪取弓矢授后妃嬪,后妃嬪受而納於弓韣。

○祭告

明制,凡登極、巡幸及上謚、葬陵、冊立、冊封、冠婚等事,皆祭告天地、宗廟、社稷。凡營造宮室,及命將出師,歲時旱潦,祭告天地、山川、太廟、社稷、后土。凡即位之初,並祭告闕里孔廟及歷代帝王陵寢。

洪武二年,禮部尚書崔亮奏,圜丘、方丘、大祀,前期親告太廟,仍遣使告百神於天下神祇壇。六年,禮部尚書牛諒奏,太歲諸神,凡祈報,則設一十五壇,有事祭告,則設神位二十八壇。中,太歲、風雲雷雨、五嶽、五鎮、四海,凡五壇。東,四瀆、京畿、湖廣、山東、河南、北平、廣西、四川、甘肅山川,夏冬二季月將,京都城隍,凡十二壇。西,鐘山,江西、浙江、福建、山西、廣東、遼東山川,春秋二季月將,旗纛、戰船等神,凡十一壇。若親祀,皇帝皮弁服行一獻禮,每三壇行一次禮。八年,帝駐蹕中都,祭告天地於中都之圜丘。九年,以諸王將之籓,分日告祭太廟、社稷、嶽鎮海瀆及天下名山大川,復告祀天地於圜丘。初,諸王來朝還籓,祭真武等神於端門,用豕九、羊九、制帛等物,祭護衛旗纛於承天門,亦如之。二十六年,帝以其禮太繁,定制豕一、羊一,不用帛。尋又罷端門祭,惟用葷素二壇祭於承天門外。

永樂七年,巡狩北京,祭告天地、宗廟、社稷,嘉靖八年秋,以躬祭山川諸神,命先期不必遣官告太廟。凡出入,必親告祖考於內殿。聖誕前一日,以酒果告列聖帝后於奉先殿,至日,以酒脯告皇天上帝於玄極寶殿,遣官以牲醴祭神烈、天壽、純德諸陵山,及東嶽、都城隍,以素羞祭真武及靈濟宮,又告修齋於道極七寶帝尊。隆慶三年,以親祭朝日壇,預告奉先、弘孝、神霄殿。

○祈報

洪武二年,太祖以春久不雨,祈告諸神祇。中設風雲雷雨、嶽鎮海瀆,凡五壇。東設鐘山、兩淮、江西、兩廣、海南北、山東、燕南燕薊山川、旗纛諸神,凡七壇。西設江東、兩浙、福建、湖廣荊襄、河南北、河東、華州山川、京都城隍,凡六壇。中五壇奠帛。初獻,帝親行禮,兩廡命官分獻。三年夏,旱。六月朔,帝素服草履,步禱於山川壇。槁席露坐,晝曝於日,夜臥於地,凡三日。六年,從禮部尚書牛諒言,太歲諸神,春祈秋報,凡十五壇。中,太歲、風雲雷雨、五嶽、五鎮、四海。東,四瀆、京畿山川,春秋二季月將,京都各府城隍。西,鐘山、甘肅山川,夏冬二季月將,旗纛戰船等神。各五壇。時甘肅新附,故附其山川之祭於京師。其親祀之儀與祭告同。正統九年三月,雨雪愆期,遣官祭天地、社稷、太歲、風雲雷雨、嶽鎮海瀆。弘治十七年,畿內、山東久旱,命官祭告天壽山,分命各巡撫祭告北嶽、北鎮、東嶽、東鎮、東海。

嘉靖八年春,帝諭禮部:「去冬少雪,今當東作,雨澤不降,當親祭南郊社稷、山川。」尚書方獻夫等言:「《周禮·大宗伯》:『以荒禮哀凶札。』釋者謂:『君膳不舉,馳道不除,祭事不縣,皆所以示貶損之意。」又曰:『國有大故,則旅上帝及四望。』釋者曰:『故謂凶災。旅,陳也,陳其祭祀以禱焉,禮不若祀之備也。』今陛下閔勞萬姓,親出祈禱。禮儀務簡約,以答天戒。常朝官並從,同致省愆祈籲之誠。」隨具上儀注。二月,親禱南郊,山川同日,社稷用次日,不除道,冠服淺色,群臣同。文五品、武四品以上於大祀門外,餘官於南天門外,就班陪祀。是秋,帝欲親祀山川諸神。禮部尚書李時言:「舊例山川等祭,中夜行禮,先一日出郊齋宿。祭畢,清晨回鑾,兩日畢事,禮太重。宜比先農壇例,昧爽行禮。」因具儀以進。制可。祭服用皮弁,迎神、送神各兩拜。

十一年,大學士李時等以聖嗣未降,請廷臣詣嶽鎮名山祝禱。帝欲分遣道士齎香帛行,令所在守臣行禮,在廷大臣分詣地祇壇祈告。於是禮部尚書夏言言:「我朝建地祇壇,自嶽鎮海瀆以及遠近名山大川,莫不懷柔,即此而禱,正合古人望衍之義。但輔臣所請,止於嶽鎮。竊以山川海瀆,發祥效靈,與嶽鎮同功,況基運、翊聖、神烈、天壽、純德諸山,又祖宗妥靈之地,祈禱之禮,皆不可缺。」遂命大臣詣壇分祀。

○神祇壇

洪武二年,從禮部尚書崔亮言,建天下神祇壇於圓丘壝外之東,及方丘壝外之西。郊祀前期,帝躬詣壇,設神位,西向,以酒脯祭告。郊之日,俟分獻從祀將畢,就壇以祭。後定遣官預告。又建山川壇於正陽門外天地壇西,合祀諸神。凡設壇十有九,太歲、春夏秋冬四季月將為第一,次風雲雷雨,次五嶽,次五鎮,次四海,次四瀆,次京都鐘山,次江東,次江西,次湖廣,次淮東、淮西,次浙東、浙西、福建,次廣東、廣西、海南、海北,次山東、山西、河南、河北,次北平、陜西,次左江、右江,次安南、高麗、占城諸國山川,次京都城隍,次六纛大神、旗纛大將、五方旗神、戰船、金鼓、銃炮、弓弩、飛槍飛石、陣前陣後諸神,皆躬自行禮。先祭,禮官奏:「祝文,太歲以下至四海,凡五壇,稱臣者親署御名。其鐘山諸神,稱餘者請令禮官代署。」帝曰:「朋友書牘,尚親題姓名,況神明乎?」遂加親署。後又定驚蟄、秋分後三日,遣官祭山川壇諸神。七年令春、秋仲月上旬,擇日以祭。九年,復定山川壇制,凡十三壇。正殿,太歲、風雲雷雨、五嶽、五鎮、四海、四瀆、鐘山七壇。東西廡各三壇,東,京畿山川、夏冬二季月將。西,春秋二季月將、京都城隍。十年,定正殿七壇,帝親行禮,東西廡遣功臣分獻。二十一年,增修大祀殿諸神壇壝。乃敕十三壇諸神並停春祭,每歲八月中旬,擇日祭之。命禮部更定祭山川壇儀,與社稷同。永樂中,京師建山川壇,並同南京制,惟正殿鐘山之右,益以天壽山之神。嘉靖十一年,改山川壇名為天神地祇壇,改序雲師、雨師、風伯、雷師。天神壇在左,南向,雲、雨、雷,凡四壇。地祇壇在右,北向,五嶽、五鎮、基運翊聖神烈天壽純德五陵山、四海、四瀆,凡五壇。從祀,京畿山川,西向;天下山川,東向。以辰、戌、丑、未年仲秋,皇帝親祭,餘年遣大臣攝祭。其太歲、月將、旗纛、城隍,別祀之。十七年,加上皇天上帝尊稱,預告於神祇,遂設壇於圜丘外壝東南,親定神祇壇位,陳設儀式。禮部言:「皇上親獻大明壇,則四壇分獻諸臣,不敢並列。請先上香畢,命官代獻。」帝裁定,上香、奠帛、獻爵復位後,分獻官方行禮。亞、終二獻,執事官代,餘壇俱獻官三行。隆慶元年,禮臣言:「天神地祇已從祀南北郊,其仲秋神祇之祭不宜復舉。」令罷之。

○星辰壇

洪武三年,帝謂中書省臣:「日月皆專壇祭,而星辰乃祔祭於月壇,非禮也。」禮部擬於城南諸神享祭壇正南向,增九間,朝日夕月祭周天星辰,俱於是行禮。朝日夕月仍以春秋分祭,星辰則於天壽節前三日。從之。四年九月,帝躬祀周天星辰。正殿共十壇,中設周天星辰位,儀如朝日。二十一年,以星辰既從祀南郊,罷禜星之祭。

○靈星諸神

洪武元年,太常司奏:「《周禮》『以燎祀司中、司命、風師、雨師』。《天府》『若祭天則祀司民、司祿,而獻民數、穀數,受而藏之。』漢高帝命郡國立靈星祠。唐制,立秋後辰日祀靈星,立冬後亥日遣官祀司中、司命、司民、司祿,以少牢。宋祀如唐,而於秋分日祀壽星。今擬如唐制,分日而祀,為壇於城南。」從之。二年,從禮部尚書崔亮奏,每歲聖壽日祭壽星,同日祭司中、司命、司民、司祿,示與民同受其福也。八月望日祀靈星。皆遣官行禮。三年,罷壽星等祀。

○太歲月將風雲雷雨之祀

古無太歲、月將壇宇之制,明始重其祭。增雲師於風師之次,亦自明始。太祖既以太歲諸神從祀圜丘,又合祭群祀壇。已而命禮官議專祀壇壝。禮臣言:「太歲者,十二辰之神。按《說文》,歲字從步從戌。木星一歲行一次,歷十二辰而周天,若步然也。陰陽家說,又有十二月將,十日十二時所直之神,若天乙、天罡、太乙、功曹、太衝之類。雖不經見,歷代因之。元每有大興作,祭太歲、月將、日直、時直於太史院。若風師、雨師之祀,見於《周官》,後世皆有祭。唐天寶中,增雷師於雨師之次。宋、元因之。然唐制各以時別祭,失享祀本意。宜以太歲、風雲雷雨諸天神合為一壇,諸地祇為一壇,春秋專祀。」乃定驚蟄、秋分日祀太歲諸神於城南。三年後以諸神陰陽一氣,流行無間,乃合二壇為一,而增四季月將。又改祭期,與地祇俱用驚蟄、秋分後三日。

嘉靖十年,命禮部考太歲壇制。禮官言:「太歲之神,唐、宋祀典不載,元雖有祭,亦無常典。壇宇之制,於古無稽。太歲天神,宜設壇露祭,準社稷壇制而差小。」從之。遂建太歲壇於正陽門外之西,與天壇對。中,太歲殿。東廡,春、秋月將二壇。西廡,夏、冬月將二壇。帝親祭於拜殿中。每歲孟春享廟,歲暮祫祭之日,遣官致祭。王國府州縣亦祀風雲雷雨師,仍築壇城西南。祭用驚蟄、秋分日。

○嶽鎮海瀆山川之祀

洪武二年,太祖以嶽瀆諸神合祭城南,未有專祀。又享祀之所,屋而不壇,非尊神之道。禮官言:「虞舜祭四嶽,《王制》始有五嶽之稱。《周官》:「兆四望於四郊」,《鄭注》以四望為五嶽四鎮四瀆。《詩序》巡狩而禮四岳河海,則又有四海之祭。蓋天子方望之事,無所不通。而嶽鎮海瀆,在諸侯封內,則各祀之。奏罷封建,嶽瀆皆領於祠官。漢復建諸侯,則侯國各祀其封內山川,天子無與。武帝時,諸侯或分或廢,五嶽皆在天子之邦。宣帝時,始有使者持節祠嶽瀆之禮。由魏及隋,嶽鎮海瀆,即其地立祠,有司致祭。唐、宋之制,有命本界刺史、縣令之祀,有因郊祀而望祭之祀,又有遣使之祀。元遣使祀岳鎮海瀆,分東西南北中為五道。今宜以嶽鎮海瀆及天下山川城隍諸地祇合為一壇。與天神埒,春秋專祀。」遂定祭日以清明霜降。前期一日,皇帝躬省牲。至日,服通天冠絳紗袍,詣嶽鎮海瀆前,行三獻禮。山川城隍,分獻官行禮。是年,命官十八人,祭天下嶽鎮海瀆之神。帝皮弁御奉天殿,躬署御名,以香祝授使者。百官公服,送至中書省,使者奉以行。黃金合貯香,黃綺幡二,白金二十五兩市祭物。

三年,詔定嶽鎮海瀆神號。略曰:「為治之道,必本於禮。嶽鎮海瀆之封,起自唐、宋。夫英靈之氣,萃而為神,必受命於上帝,豈國家封號所可加?瀆禮不經,莫此為甚。今依古定制,並去前代所封名號。五嶽稱東嶽泰山之神,南嶽衡山之神,中嶽嵩山之神,西嶽華山之神,北嶽恒山之神。五鎮稱東鎮沂山之神,南鎮會稽山之神,中鎮霍山之神,西鎮吳山之神,北鎮醫無閭山之神。四海稱東海之神,南海之神,西海之神,北海之神。四瀆稱東瀆大淮之神,南瀆大江之神,西瀆大河之神,北瀆大濟之神。」帝躬署名於祝文,遣官以更定神號告祭。六年,禮官言:「四川未平,望祭江瀆於峽州。今蜀既下,當遣人於南瀆致祭。」從之。十年,命官十八人分祀嶽鎮海瀆,賜之制。

萬曆十四年,巡撫胡來貢請改祀北嶽於渾源州。禮官言:「《大明集禮》載,漢、唐、宋北嶽之祭,皆在定州曲陽縣,與史俱合。渾源之稱北嶽,止見州誌碑文,經傳無可考,仍祀曲陽是。」

其他山川之祀。洪武元年躬祀汴梁諸神,仍遣官祭境內山川。二年,以天下山川祔祭嶽瀆壇。帝又以安南、高麗皆臣附,其國內山川,宜與中國同祭。諭中書及禮官考之。安南之山二十一,其江六,其水六。高麗之山三,其水四。命著祀典,設位以祭。三年,遣使往安南、高麗、占城,祀其國山川。帝齋戒,親為祝文。仍遣官頒革正山川神號詔於安南、占城、高麗。六年,琉球諸國已朝貢,祀其國山川。八年,禮部尚書牛諒言:「京都既罷祭天下山川,其外國山川,亦非天子所當親祀。」中書及禮臣請附祭各省,從之。廣西附祭安南、占城、真臘、暹羅、鎖里,廣東附祭三佛齊、爪哇,福建附祭日本、琉球、渤泥,遼東附祭高麗,陜西附祭甘肅、朵甘、烏斯藏,京城不復祭。又從禮官言,各省山川居中南向,外國山川東西向,同壇共祀。其王國山川之祀,洪武十三年定制。十八年定王國祭山川。儀同社稷,但無瘞埋之文。凡嶽鎮海瀆及他山川所在,令有司歲二祭,以清明、霜降。

○城隍

洪武二年,禮官言:「城隍之祀,莫詳其始。先儒謂既有社,不應復有城隍。故唐李陽冰《縉雲城隍記》謂『祀殿無之,惟吳越有之。』然成都城隍祠,李德裕所建,張說有祭城隍之文,杜牧有祭黃州城隍文,則不獨吳越為然。又蕪湖城隍廟建於吳赤烏二年,高齊慕容儼、梁武陵王祀城隍,皆書於史,又不獨唐而已。宋以來其祠遍天下,或錫廟額,或頒封爵,至或遷就傅會,各指一人以為神之姓名。按張九齡《祭洪州城隍文》曰:『城隍是保,氓庶是依。』則前代崇祀之意有在也。今宜附祭於嶽瀆諸神之壇。」乃命加以封爵。京都為承天鑒國司民昇福明靈王,開封、臨濠、太平、和州、滁州皆封為王。其餘府為鑒察司民城隍威靈公,秩正二品。州為鑒察司民城隍靈佑侯,秩三品。縣為鑒察司民城隍顯佑伯,秩四品。袞章冕旒俱有差。命詞臣撰制文以頒之。

三年,詔去封號,止稱其府州縣城隍之神。又令各廟屏去他神。定廟制,高廣視官署廳堂。造木為主,毀塑像舁置水中,取其泥塗壁,繪以雲山。六年,制中都城隍神主成,遣官齎香幣奉安。京師城隍既附饗山川壇,又於二十一年改建廟。尋以從祀大禮殿,罷山川壇春祭。永樂中,建廟都城之西,曰大威靈祠。嘉靖九年,罷山川壇從祀,歲以仲秋祭旗纛日,並祭都城隍之神。凡聖誕節及五月十一日神誕,皆遣太常寺堂上官行禮。國有大災則告廟。在王國者王親祭之,在各府州縣者守令主之。

歷代帝王陵廟三皇聖師國先師孔子旗纛五祀國馬神南京神廟功臣廟京師九廟諸神祠厲壇

○歷代帝王陵廟

洪武三年,遣使訪先代陵寢,仍命各行省具圖以進,凡七十有九。禮官考其功德昭著者,曰伏羲,神農,黃帝,少昊,顓頊,唐堯,虞舜,夏禹,商湯、中宗、高宗,周文王、武王、成王、康王,漢高祖、文帝、景帝、武帝、宣帝、光武、明帝、章帝,後魏文帝,隋高祖,唐高祖、太宗、憲宗、宣宗,周世宗,宋太祖、太宗、真宗、仁宗、孝宗、理宗,凡三十有六。各製袞冕,函香幣。遣秘書監丞陶誼等往修祀禮,親製祝文遣之。每陵以白金二十五兩具祭物。陵寢發者掩之,壞者完之。廟敝者葺之。無廟者設壇以祭。仍令有司禁樵採。歲時祭祀,牲用太牢。

四年,禮部定議,合祀帝王三十五。在河南者十:陳祀伏羲、商高宗,孟津祀漢光武,洛陽祀漢明帝、章帝,鄭祀周世宗,鞏祀宋太祖、太宗、真宗、仁宗。在山西者一:滎河祀商湯。在山東者二:東平祀唐堯,曲阜祀少昊。在北平者三:內黃祀商中宗,滑祀顓頊、高辛。在湖廣者二:酃祀神農,寧遠祀虞舜。在浙江者二:會稽祀夏禹、宋孝宗。在陜西者十五:中部祀黃帝,咸陽祀周文王、武王、成王、康王、宣王,漢高帝、景帝,咸寧祀漢文帝,興平祀漢武帝,長安祀漢宣帝,三原祀唐高祖,醴泉祀唐太宗,蒲城祀唐憲宗,涇陽祀唐宣宗。歲祭用仲春、仲秋朔。于是遣使詣各陵致祭。陵置一碑,刊祭期及牲帛之數,俾所在有司守之。已而命有司歲時修葺,設陵戶二人守視。又每三年,出祝文、香帛,傳制遣太常寺樂舞生齎往所在,命有司致祭。其所祀者,視前去周宣王,漢明帝、章帝,而增祀媧皇於趙城,後魏文帝於富平,元世祖於順天,及宋理宗於會稽,凡三十六帝。後又增祀隋高祖於扶風,而理宗仍罷祀。又命帝王陵廟所在官司,以春秋仲月上旬,擇日致祭。

六年,帝以五帝、三王及漢、唐、宋創業之君,俱宜於京師立廟致祭,遂建歷代帝王廟於欽天山之陽。仿太廟同堂異室之制,為正殿五室:中一室三皇,東一室五帝,西一室夏禹、商湯、周文王,又東一室周武王、漢光武、唐太宗,又西一室漢高祖、唐太祖、宋太祖、元世祖。每歲春秋仲月上旬甲日致祭。已而以周文王終守臣服,唐高祖由太宗得天下,遂寢其祀,增祀隋高祖。七年,令帝王廟皆塑袞冕坐像,惟伏羲、神農未有衣裳之制,不必加冕服。八月,帝躬祀於新廟。已而罷隋高祖之祀。

二十一年,令每歲郊祀,附祭歷代帝王於大祀殿。仍以歲八月中旬,擇日遣官祭於本廟,其春祭停之。又定每三年遣祭各陵之歲,則停廟祭。是年,詔以歷代名臣從祀,禮官李原名奏擬三十六人以進。帝以宋趙普負太祖不忠,不可從祀。元臣四傑,木華黎為首,不可祀孫而去其祖,可祀木華黎而罷安童。既祀伯顏,則阿術不必祀。漢陳平、馮異,宋潘美,皆善始終,可祀。於是定風后、力牧、皋陶、夔、龍、伯夷、伯益、伊尹、傅說、周公旦、召公奭、太公望、召虎、方叔、張良、蕭何、曹參、陳平、周勃、鄧禹、馮異、諸葛亮、房玄齡、杜如晦、李靖、郭子儀、李晟、曹彬、潘美、韓世忠、岳飛、張浚、木華黎、博爾忽、博爾術、赤老溫、伯顏,凡三十七人,從祀於東西廡,為壇四。初,太公望有武成王廟。嘗遣官致祭如釋奠儀。至是,罷廟祭,去王號。

永樂遷都,帝王廟遣南京太常寺官行禮。嘉靖九年,罷歷代帝王南郊從祀。令建歷代帝王廟於都城西,歲以仲春秋致祭。後并罷南京廟祭。十年春二月,廟未成,躬祭歷代帝王於文華殿,凡五壇,丹陛東西名臣四壇。禮部尚書李時言:「舊儀有賜福胙之文。賜者自上而下之義,惟郊廟社稷宜用。歷代帝王,止宜云答。」詔可。十一年夏,廟成,名曰景德崇聖之殿。殿五室,東西兩廡,殿後祭器庫,前為景德門。門外神庫、神廚、宰牲亭、鐘樓。街東西二坊,曰景德街。用八月壬辰親祭。帝由中門入,迎神、受福胙、送神各兩拜。嗣後歲遣大臣一員行禮,四員分獻。凡子、午、卯、酉祭於陵寢之歲,則停秋祭。二十四年,以禮科陳棐言,罷元世祖陵廟之祀,及從祀木華黎等,復遷唐太宗與宋太祖同室。凡十五帝,從祀名臣三十二人。

○三皇

明初仍元制,以三月三日、九月九日通祀三皇。洪武元年,令以太牢祀。二年,命以句芒、祝融、風后、力牧左右配,俞跗、桐君、僦貸季、少師、雷公、鬼臾區、伯高、岐伯、少俞、高陽十大名醫從祀。儀同釋奠。四年,帝以天下郡邑通祀三皇為瀆。禮臣議:「唐玄宗嘗立三皇五帝廟於京師。至元成宗時,乃立三皇廟於府州縣。春秋通祀,而以醫藥主之,甚非禮也。」帝曰:「三皇繼天立極,開萬世教化之原,汨於藥師可乎?」命天下郡縣毋得褻祀。

正德十一年,立伏羲氏廟於秦州。秦州,古成紀地,從巡按御史馮時雄奏也。嘉靖間,建三皇廟於太醫院北,名景惠殿。中奉三皇及四配。其從祀,東廡則僦貸季、岐伯、伯高、鬼臾區、俞跗、少俞、少師、桐君、雷公、馬師皇、伊尹、扁鵲、淳于意、張機十四人,西廡則華陀、王叔和、皇甫謐、葛洪、巢元方、孫思邈、韋慈藏、王冰、錢乙、朱肱、李杲、利完素、張元素、朱彥修十四人。歲仲春、秋上甲日,禮部堂上官行禮,太醫院堂上官二員分獻,用少牢。復建聖濟殿於內,祀先醫,以太醫官主之。二十一年,帝以規制湫隘,命拓其廟。

○聖師

聖師之祭,始於世宗。奉皇師伏羲氏、神農氏、軒轅氏、帝師陶唐氏,有虞氏,王師夏禹王、商湯王、周文王武王,九聖南向。左先聖周公,右先師孔子,東西向。每歲春秋開講前一日,皇帝服皮弁,拜跪,行釋奠禮。用羹酒果脯帛祭於文華殿東室。

初,東室有釋像,帝以其不經,撤之,乃祀先聖先師。自為祭文,行奉安神位禮。輔臣禮卿及講官俟行禮訖,入拜。先是洪武初,司業宋濂建議欲如建安熊氏之說,以伏羲為道統之宗,神農、黃帝、堯、舜、禹、湯、文、武,以次列焉。秩祀天子之學,則道統益尊。太祖不從。至是,世宗仿其意行之。十六年,移祀於永明殿後,行禮如初。其後常遣官代祭。隆慶初,仍於文華殿東室行禮。

○至聖先師子孔廟祀

漢晉及隋或稱先師,或稱先聖、宣尼、宣父。唐謚文宣王,宋加至聖號,元復加號大成。明太祖入江淮府,首謁孔子廟。洪武元年二月,詔以太牢祀孔子於國學,仍遣使詣曲阜致祭。臨行諭曰:「仲尼之道,廣大悠久,與天地並。有天下者莫不虔修祀事。朕為天下主,期大明教化,以行先聖之道。今既釋奠成均,仍遣爾修祀事於闕里,爾其敬之。」又定制,每歲仲春、秋上丁,皇帝降香,遣官祀於國學。以丞相初獻,翰林學士亞獻,國子祭酒終獻。先期,皇帝齋戒。獻官、陪祀、執事官皆散齋二日,致齋一日。前祀一日,皇帝服皮弁服,御奉天殿降香。至日,獻官行禮。三年,詔革諸神封號,惟孔子封爵仍舊。且命曲阜廟庭,歲官給牲幣,俾衍聖公供祀事。四年,禮部奏定儀物。改初制籩豆之八為十,籩用竹。其簠簋登鉶及豆初用木者,悉易以瓷。牲易以熟。樂生六十人,舞生四十八人,引舞二人,凡一百一十人。禮部請選京民之秀者充樂舞生,太祖曰:「樂舞乃學者事,況釋奠所以崇師,宜擇國子生及公卿子弟在學者,豫教肄之。」五年,罷孟子配享。踰年,帝曰:「孟子辨異端,闢邪說,發明孔子之道,配享如故。」七年二月,上丁日食,改用仲丁。

十五年,新建太學成。廟在學東,中大成殿,左右兩廡,前大成門,門左右列戟二十四。門外東為犧牲廚,西為祭器庫,又前為靈星門。自經始以來,駕數臨視。至是落成,遣官致祭。帝既親詣釋奠,又詔天下通祀孔子,并頒釋奠儀注。凡府州縣學,籩豆以八,器物牲牢,皆殺于國學。三獻禮同,十哲兩廡一獻。其祭,各以正官行之,有布政司則以布政司官,分獻則以本學儒職及老成儒士充之。每歲春、秋仲月上丁日行事。初,國學主祭遣祭酒,後遣翰林院官,然祭酒初到官,必遣一祭。十七年,敕每月朔望,祭酒以下行釋菜禮,郡縣長以下詣學行香。二十六年,頒大成樂於天下。二十八年,以行人司副楊砥言,罷漢揚雄從祀,益以董仲舒。三十年,以國學孔子廟隘,命工部改作,其制皆帝所規畫。大成殿門各六楹,靈星門三,東西廡七十六楹,神廚庫皆八楹,宰牲所六楹。永樂初,建廟於太學之東。

宣德三年,以萬縣訓導李譯言,命禮部考正從祀先賢名位,頒示天下。十年,慈利教諭蔣明請祀元儒吳澄。大學士楊士奇等言當從祀,從之。正統二年,以宋儒胡安國、蔡沈、真德秀從祀。三年,禁天下祀孔子於釋、老宮。孔、顏、孟三氏子孫教授裴侃言:「天下文廟惟論傳道,以列位次。闕里家廟,宜正父子,以敘彞倫。顏子、曾子、子思,子也,配享殿廷。無繇、子曨、伯魚,父也,從祀廊廡。非惟名分不正,抑恐神不自安。況叔梁紇元已追封啟聖王,創殿於大成殿西崇祀,而顏、孟之父俱封公,惟伯魚、子曨仍侯,乞追封公爵,偕顏、孟父俱配啟聖王殿。」帝命禮部行之,仍議加伯魚、子曨封號。成化二年,追封董仲舒廣川伯,胡安國建寧伯,蔡沈崇安伯,真德秀浦城伯。十二年,從祭酒周洪謨言,增樂舞為八佾,籩豆各十二。弘治八年,追封楊時將樂伯。從祀,位司馬光之次。九年,增樂舞為七十二人,如天子之制。十二年,闕里孔廟毀,敕有司重建。十七年,廟成,遣大學士李東陽祭告,並立御製碑文。正德十六年,詔有司改建孔氏家廟之在衢州者,官給錢,董其役。令博士孔承義奉祀。

嘉靖九年,大學士張璁言:「先師祀典,有當更正者。叔梁紇乃孔子之父,顏路、曾曨、孔鯉乃顏、曾、子思之父,三子配享廟庭,紇及諸父從祀兩廡,原聖賢之心豈安?請於大成殿後,別立室祀叔梁紇,而以顏路、曾曨、孔鯉配之。」帝以為然。因言:「聖人尊天與尊親同。今籩豆十二,牲用犢,全用祀天儀,亦非正禮。其謚號、章服悉宜改正。」璁緣帝意,言:「孔子宜稱先聖先師,不稱王。祀宇宜稱廟,不稱殿。祀宜用木主,其塑像宜毀。籩豆用十,樂用六佾。配位公侯伯之號宜削,止稱先賢先儒。其從祀申黨、公伯寮、秦冉等十二人宜罷,林放、蘧瑗等六人宜各祀於其鄉,后蒼、王通、歐陽修、胡瑗、蔡元定宜從祀。」

帝命禮部會翰林諸臣議。編修徐階疏陳易號毀像之不可。帝怒,謫階官,乃御製《正孔子祀典說》,大略謂孔子以魯僭王為非,寧肯自僭天子之禮?復為《正孔子祀典申記》,俱付史館。璁因作《正孔子廟祀典或問》奏之。帝以為議論詳正,并令禮部集議。於是御史黎貫等言:「聖祖初正祀典,天下嶽瀆諸神皆去其號,惟先師孔子如故,良有深意。陛下疑孔子之祀上擬祀天之禮。夫子以不可及也,猶天之不可階而升,雖擬諸天,亦不為過。自唐尊孔子為文宣王,已用天子禮樂。宋真宗嘗欲封孔子為帝,或謂周止稱王,不當加帝號。而羅從彥之論,則謂加帝號亦可。至周敦頤則以為萬世無窮王祀孔子,邵雍則以為仲尼以萬世為王。其辨孔子不當稱王者,止吳澄一人而已。伏望博考群言,務求至當。」時貫疏中言:「莫尊於天地,亦莫尊於父師。陛下敬天尊親,不應獨疑孔子王號為僭。」帝因大怒,疑貫借此以斥其追尊皇考之非,詆為奸惡,下法司會訊,褫其職。給事中王汝梅等亦極言不宜去王號,帝皆斥為謬論。

於是禮部會諸臣議:「人以聖人為至,聖人以孔子為至。宋真宗稱孔子為至,宋真宗稱孔子為至聖,其意已備。今宜於孔子神位題至聖先師孔子,去其王號及大成、文宣之稱。改大成殿為先師廟,大成門為廟門。其四配稱復聖顏子、宗聖曾子、述聖子思子、亞聖孟子。十哲以下凡及門弟子,皆稱先賢某子。左丘明以下,皆稱先儒某子,不復稱公侯伯。遵聖祖首定南京國子監規制,製木為神主。仍擬大小尺寸,著為定式。其塑像即令屏撤。春秋祭祀,遵國初舊制,十籩十豆。天下各學,八籩八豆。樂舞止六佾。凡學別立一祠,中叔梁紇,題啟聖化孔氏神位,以顏無繇、曾點、孔鯉、孟孫氏配,俱稱先賢某氏,至從祀之賢,不可不考其得失。申黨即申棖,釐去其一。公伯寮、秦冉、顏何、荀況、戴聖、劉向、賈逵、馬融、何休、王肅、王弼、杜預、吳澄罷祀。林放、蘧瑗、盧植、鄭眾、鄭玄、服虔、范寧各祀於其鄉。后蒼、王通、歐陽修、胡瑗宜增入。」命悉如議行。又以行人薛侃議,進陸九淵從祀。

初,洪武時,司業宋濂請去像設主,禮儀樂章多所更定,太祖不允。成、弘間,少詹程敏政嘗謂馬融等八人當斥。給事中張九功推言之,并請罷荀況、公伯寮、蘧瑗等,而進后蒼、王通、胡瑗。為禮官周洪謨所卻而止。至是以璁力主,眾不敢違。毀像蓋用濂說,先賢去留,略如九功言。其進歐陽修,則以濮議故也。

明年,國子監建啟聖公祠成。從尚書李時言,春秋祭祀,與文廟同日。籩豆牲帛視四配,東西配位視十哲,從祀先儒程晌、硃松、蔡元定視兩廡。輔臣代祭文廟,則祭酒祭啟聖祠。南京祭酒於文廟,司業於啟聖祠。遂定制,殿中先師南向,四配東西向。稍後十哲:閔子損、冉子雍、端木子賜、仲子由、卜子商、冉子耕、宰子予、冉子求、言子偃、顓孫子師皆東西向。兩廡從祀:先賢澹臺滅明、宓不齊、原憲、公冶長、南宮適、高柴、漆雕開、樊須、司馬耕、公西赤、有若、琴張、申棖、陳亢、巫馬施、梁鱣、公曨哀、商瞿、冉孺、顏辛、伯虔、曹恤、冉季、公孫龍、漆雕哆、秦商、漆雕徒父、顏高、商澤、壤駟赤、任不齊、石作蜀、公良孺、公夏首、公肩定、后處、鄡單、奚容AM、罕父黑、顏祖、榮旂、秦祖、左人郢、句井疆、鄭國、公祖句茲、原亢、縣成、廉潔、燕人及、叔仲會、顏之僕、邽巽、樂欬、公西輿如、狄黑、孔忠、公西AM、步叔乘、施之常、秦非、顏噲,先儒左丘明、公羊高、穀梁赤、伏勝、高堂生、孔安國、毛萇、董仲舒、后蒼、杜子春、王通、韓愈、胡瑗、周敦頤、程顥、歐陽修、邵雍、張載、司馬光、程頤、楊時、胡安國、朱熹、張栻、陸九淵、呂祖謙、蔡沈、真德秀、許衡凡九十一人。

隆慶五年,以薛瑄從祀。萬曆中,以羅從彥、李侗從祀。十二年,又以陳獻章、胡居仁、王守仁從祀。二十三年,以宋周敦頤父輔成從祀啟聖祠。又定每歲仲春、秋上丁日御殿傳制,遣大臣祭先師及配位。其十哲以翰林官、兩廡以國子監官各二員分獻。每月朔,及每科進士行釋菜禮。司府州縣衛學各提調官行禮。牲用少牢,樂如太學。京府及附府縣學,止行釋菜禮。崇禎十五年,以左丘明親授經於聖人,改稱先賢。并改宋儒周、二程、張、硃、邵六子亦稱先賢,位七十子下,漢唐諸儒之上。然僅國學更置之,闕里廟廷及天下學宮未遑頒行也。

○旗纛

旗纛之祭有四。其一,洪武元年,禮官奏:「軍行旗纛所當祭者,旗謂牙旗。黃帝出軍訣曰:『牙旗者,將軍之精,一軍之形侯。凡始豎牙,必祭以剛日。』纛,謂旗頭也。《太白陰經》曰:『大將中營建纛。天子六軍,故用六纛。犛牛尾為之,在左騑馬首。』唐、宋及元皆有旗纛之祭。今宜立廟京師,春用驚蟄,秋用霜降日,遣官致祭。」乃命建廟於都督府治之後,以都督為獻官,題主曰軍牙之神、六纛之神。七年二月,詔皇太子率諸王詣閱武場祭旗纛,為壇七,行三獻禮。後停春祭,止霜降日祭於教場。其二,歲暮享太廟日,祭旗纛於承天門外。其三,旗纛廟在山川壇左。初,旗纛與太歲諸神合祭於城南。九年,別建廟。每歲仲秋,天子躬祀山川之日,遣旗手衛官行禮。其正祭,旗頭大將、六纛大將、五方旗神、主宰戰船正神、金鼓角銃炮之神、弓弩飛槍飛石之神、陣前陣後神祇五昌等眾,凡七位,共一壇,南向。皇帝服皮弁,御奉天殿降香。獻官奉以從事。祭物視先農,帛七,黑二白五。瘞毛血、望燎,與風雲雷雨諸神同。祭畢,設酒器六於地。刺雄雞六,瀝血以釁之。其四,永樂後,有神旗之祭,專祭火雷之神。每月朔望,神機營提督官祭於教場。牲用少牢。凡旗纛皆藏內府,祭則設之。

王國祭旗纛,則遣武官戎服行禮。天下衛所於公署後立廟,以指揮使為初獻官。僚屬為亞獻、終獻。儀物殺京都。

○五祀

洪武二年定制,歲終臘享,通祭於廟門外。八年,禮部奏:「五祀之禮,周、漢、唐、宋不一。今擬孟春祀戶,設壇皇宮門左,司門主之。孟夏祀灶,設壇御廚,光祿寺官主之。季夏祀中霤,設壇乾清宮丹墀,內官主之。孟秋祀門,設壇午門左,司門主之。孟冬祀井,設壇宮內大庖井前,光祿寺官主之。四孟於有事太廟之日,季夏於土旺之日,牲用少牢。」制可。從定中霤于奉天殿外文樓前。又歲暮合祭五祀于太廟西廡下,太常寺官行禮。

○馬神

洪武二年命祭馬祖、先牧、馬社、馬步之神,築壇後湖。禮官言:「《周官》春祭馬祖,天駟星也;夏祭先牧,始養馬者;秋祭馬社,始乘馬者;冬祭馬步,乃神之災害馬者。隋用周制,祭以四仲之月。唐、宋因之。今定春、秋二仲月,甲、戊、庚日,遣官致祀。為壇四,樂用時樂,行三獻禮。」四年,蜀明昇獻良馬十,其一白者,長丈餘,不可加韉勒。太祖曰:「天生英物,必有神司之。」命太常以少牢祀馬祖,囊沙四百斤壓之,令人騎而遊苑中,久之漸馴。帝乘之以夕月於清涼山。比還,大悅,賜名飛越峰。復命太常祀馬祖。五年,并諸神為一壇,歲止春祭。永樂十二年,立北京馬神祠於蓮花池。其南京馬神,則南太僕主之。

○南京神廟

初稱十廟。北極真武以三月三日、九月九日,道林真覺普濟禪師寶志以三月十八日,都城隍以八月祭帝王後一日,祠山廣惠張王渤以二月十八日,五顯靈順以四月八日、九月二十八日,皆南京太常寺官祭。漢秣陵尉蔣忠烈公子文、晉成陽卞忠貞公壼、宋濟陽曹武惠王彬、南唐劉忠肅王仁瞻、元衛國忠肅公福壽俱以四孟朔,歲除,應天府官祭。惟蔣廟又有四月二十六日之祭。并功臣廟為十一。後復增四:關公廟,洪武二十七年建於雞籠山之陽,稱漢前將軍壽亭侯。嘉靖十年訂其誤,改稱漢前將軍漢壽亭侯。以四孟歲暮,應天府官祭,五月十三日,南京太常寺官祭。天妃,永樂七年封為護國庇民妙靈昭應弘仁普濟天妃,以正月十五日、三月二十三日,南京太常寺官祭。太倉神廟,以仲春、秋望日,南京戶部官祭。司馬、馬祖、先牧神廟,以春、秋仲月中旬,擇日南京太僕寺官祭。諸廟皆少牢,真武與真覺禪師素羞。

○功臣廟

太祖既以功臣配享太廟,又命別立廟於雞籠山。論次功臣二十有一人,死者塑像,生者虛其位。正殿:中山武寧王徐達、開平忠武王常遇春、岐陽武靖王李文忠、寧河武順王鄧愈、東甌襄武王湯和、黔寧昭靖王沐英。羊二,豕二。西序:越國武莊公胡大海、梁國公趙德勝、巢國武壯公華高、虢國忠烈公俞通海、江國襄烈公吳良、安國忠烈公曹良臣、黔國威毅公吳復、燕山忠愍侯孫興祖。東序:郢國公馮國用、西海武壯公耿再成、濟國公丁德興、蔡國忠毅公張德勝、海國襄毅公吳楨、蘄國武義公康茂才、東海郡公茅成。羊二,豕二。兩廡各設牌一,總書「故指揮千百戶衛所鎮撫之靈」。羊十,豕十。以四孟歲暮,遣駙馬都尉祭。

初,胡大海等歿,命肖像於卞壼、蔣子文之廟。及功臣廟成,移祀焉。永樂三年,以中山王勛德第一,又命正旦、清明、中元、孟冬、冬至遣太常寺官祭於大功坊之家廟,牲用少牢。

○京師九廟

京師所祭者九廟。真武廟,永樂十三年建,以祀北極佑聖真君。正德二年改為靈明顯佑宮,在海子橋之東,祭日同南京。

東嶽泰山廟,在朝陽門外,祭以三月二十八日。

都城隍廟,祭以五月十一日。

漢壽亭侯關公廟,永樂間建。成化十三年,又奉敕建廟宛平縣之東,祭以五月十三日。皆太常寺官祭。

京都太倉神廟,建於太倉,戶部官祭。

司馬、馬祖、先牧神廟,太僕寺官祭。

宋文丞相祠,永樂六年從太常博士劉履節請,建於順天府學之西。元世祖廟,嘉靖中罷。皆以二月,八月中旬順天府官祭。

洪恩靈濟宮,祀徐知證、知諤。永樂十五年,立廟皇城之西,正旦、冬至聖節,內閣禮部及內官各一員祭。生辰,禮部官祭。弘治中,大學士劉健等請毋遣閣臣。嘉靖中,改遣太常寺官。

其榮國公姚廣孝,洪熙元年從祀太廟。嘉靖九年撤廟祀,移祀大興隆寺,在皇城西北隅。後寺毀,復移崇國寺。

東嶽、都城隍用太牢,五廟用少牢,真武、靈濟宮素羞。

○諸神祠

洪武元年,命中書省下郡縣,訪求應祀神祇。名山大川、聖帝明王、忠臣烈士,凡有功於國家及惠愛在民者,著於祀典,令有司歲時致祭。二年,又詔天下神祇,常有功德於民,事跡昭著者,雖不致祭,禁人毀撤祠宇。三年,定諸神封號,凡後世溢美之稱皆革去。天下神祠不應祀典者,即淫祠也,有司毋得致祭。弘治元年,禮科張九功言:「祀典正則人心正。今朝廷常祭之外,又有釋迦牟尼文佛、三清三境九天應元雷聲普化天尊、金玉闕真君元君、神父神母,諸宮觀中又有水官星君、諸天諸帝之祭,非所以法天下。」帝下其章禮部,尚書周洪謨等言:

釋迦牟尼文佛生西方中天竺國。宗其教者,以本性為法身,德業為報身,并真身為三,其實一人耳。道家以老子為師。朱熹有曰:「玉清元始天尊既非老子法身,上清太上道君又非老子報身,設有二像,又非與老子為一。而老子又自為上清太上老君,蓋仿釋氏而又失之者也。」自今凡遇萬壽等節,不令修建吉祥齋醮,或遇喪禮,不令修建薦揚齋醮。其大興隆寺、朝天宮俱停遣官祭告。

北極中天星主紫微大帝者,北極五星在紫微垣中,正統初,建紫微殿,設像祭告。夫幽禜祭星,古禮也。今乃像之如人,稱之為帝,稽之祀典,誠無所據。

雷聲普化天尊者,道家以為總司五雷,又以六月二十四日為天尊示現之日,故歲以是日遣官詣顯靈宮致祭。夫風雲雷雨,南郊合祀,而山川壇復有秋報,則此祭亦當罷免。

祖師三天扶教輔玄大法師真君者,傳記云:「漢張道陵,善以符治病。唐天寶,宋熙寧、大觀間,累號正一靖應真君,子孫亦有封號。國朝仍襲正一嗣教真人之封。」然宋邵伯溫云:「張魯祖陵、父衡,以符法相授受,自號師君。」今歲以正月十五日為陵生日,遣官詣顯靈宮祭告,亦非祀典。

大小青龍神者,記云:「有僧名盧,寓西山。有二童子來侍。時久旱,童子入潭化二青龍,遂得雨。後賜盧號曰感應禪師,建寺設像,別設龍祠於潭上。宣德中,建大圓通寺,加二龍封號,春秋祭之。」邇者連旱,祈禱無應,不足崇奉明矣。

梓潼帝君者,記云:「神姓張,名亞子,居蜀七曲山。仕晉戰沒,人為立廟。唐、宋屢封至英顯王。道家謂帝命梓潼掌文昌府事及人間祿籍,故元加號為帝君,而天下學校亦有祠祀者。景泰中,因京師舊廟闢而新之,歲以二月三日生辰,遣祭。」夫梓潼顯靈於蜀,廟食其地為宜。文昌六星與之無涉,宜敕罷免。其祠在天下學校者,俱令拆毀。

北極佑聖真君者,乃玄武七宿,後人以為真君,作龜蛇於其下。宋真宗避諱,改為真武。靖康初,加號佑聖助順靈應真君。圖志云:「真武為凈樂王太子,修煉武當山,功成飛昇。奉上帝命鎮北方。被髮跣足,建皂纛玄旗。」此道家附會之說。國朝御製碑謂,太祖平定天下,陰佑為多,當建廟南京崇祀。及太宗靖難,以神有顯相功,又於京城艮隅并武當山重建廟宇。兩京歲時朔望各遣官致祭,而武當山又專官督祀事。憲宗嘗範金為像。今請止遵洪武間例,每年三月三日、九月九日用素羞,遣太常官致祭,餘皆停免。

崇恩真君、隆恩真君者,道家以崇恩姓薩名堅,西蜀人,宋徽宗時嘗從王侍宸、林靈素輩學法有驗。隆恩,則玉樞火府天將王靈官也,又嘗從薩傳符法。永樂中,以道士周思得能傳靈官法,乃於禁城之西建天將廟及祖師殿。宣德中,改大德觀,封二真君。成化初改顯靈宮。每年換袍服,所費不訾。近今祈禱無應,亦當罷免。

金闕上帝、玉闕上帝者,志云:「閩縣靈濟宮祀五代時徐溫子知證、知諤。國朝御製碑謂太宗嘗弗豫,禱神輒應,因大新閩地廟宇,春秋致祭。又立廟京師,加封金闕真君、玉闕真君。正統、成化中,累加號為上帝。朔望令節俱遣官祀,及時薦新,四時換袍服。」夫神世系事跡,本非甚異,其僭號宜革正,妄費亦宜節省。神父聖帝、神母元君及金玉闕元君者,即二徐父母及其配也。宋封其父齊王為忠武真人,母田氏為仁壽仙妃,配皆為仙妃。永樂至成化間,屢加封今號,亦宜削號罷祀。

東嶽泰山之神者,泰山五嶽首,廟在泰安州山下。又每歲南郊及山川壇俱有合祭之禮。今朝陽門外有元東嶽舊廟,國朝因而不廢。夫既專祭封內,且合祭郊壇,則此廟之祭,實為煩瀆。

京師都城隍之神者,舊在順天府西南,以五月十一日為神誕辰,故是日及節令皆遣官祀。夫城隍之神,非人鬼也,安有誕辰?況南郊秋祀俱已合祭,則誕辰及節令之祀非宜,凡此俱當罷免。

議上,乃命修建齋醮,遣官祭告,并東嶽、真武、城隍廟、靈濟宮祭祀,俱仍舊。二徐真君及其父母妻革去帝號,仍舊封,冠袍等物換回焚毀,餘如所議行之。

按祀典,太祖時,應天祀陳喬、楊邦乂、姚興、王鉷,成都祀李冰、文翁、張詠,均州祀黃霸,密縣祀卓茂,松江祀陸遜、陸抗、陸凱,龍州祀李龍遷,建寧祀謝夷甫,彭澤祀狄仁傑,九江祀李黼,安慶祀余闕、韓建之、李宗可。宣宗時,高郵祀耿遇德。英宗時,豫章祀韋丹、許遜,無錫祀張巡。憲宗時,崖山祀張世傑、陸秀夫。孝宗時,新會祀宋慈元楊后,延平祀羅從彥、李侗,建寧祀劉子翬,烏撒祀潭淵,廬陵祀文天祥,婺源祀朱熹,都昌祀陳澔,饒州祀江萬里,福州祀陳文龍,興化祀陳瓚,湖廣祀李芾,廣西祀馬慨。武宗時,真定祀顏杲卿、真卿,韶州附祀張九齡子拯,沂州祀諸葛亮,蕭山祀游酢、羅從彥。皆歷代名臣,事跡顯著。守臣題請,禮官議覆,事載實錄,年月可稽。至若有明一代之臣抗美前史者,或以功勳,或以學行,或以直節,或以死事,臚于志乘,刻于碑版,匪一而足。其大者,鄱陽湖忠臣祠祀丁普郎等三十五人,南昌忠臣祠祀趙德勝等十四人,太平忠臣廟祀花雲、王鼎、許瑗,金華忠臣祠祀胡大海,皆太祖自定其典。其後,通州祀常遇春,山海關祀徐達,蘇州祀夏原吉、周忱,淮安祀陳瑄,海州衛祀衛青、徐安生,甘州祀毛忠,榆林祀餘子俊,杭州祀于謙,蕭山祀魏驥,汀州祀王得仁,廣州祀楊信民、毛吉,雲南祀沐英、沐晟,貴州祀顧成,廬陵祀劉球、李時勉,廣信祀鄧顒,寶慶祀賀興隆,上杭祀伍驥、丁泉,慶遠祀葉禎,雲南祀王禕、吳雲,青田祀劉基,平陽祀薛瑄,杭州祀鄒濟、徐善述,金華祀章懋,皆眾著耳目,炳然可考。其他郡縣山川龍神忠烈之士,及祈禱有應而祀者,《會典》所載,尤詳悉云。

○厲壇

泰厲壇祭無祀鬼神。《春秋傳》曰「鬼有所歸,乃不為厲。」此其義也。《祭法》:王祭泰厲,諸侯祭公厲,大夫祭族厲。《士喪禮》:「疾病禱於厲』,《鄭注》謂「漢時民間皆秋祠厲」,則此祀達于上下矣,然後世皆不舉行。洪武三年定制,京都祭泰厲,設壇玄武湖中,歲以清明及十月朔日遣官致祭。前期七日,檄京都城隍。祭日,設京省城隍神位於壇上,無祀鬼神等位於壇下之東西,羊三,豕三,飯米三石。王國祭國厲,府州祭郡厲,縣祭邑厲,皆設壇城北,一年二祭如京師。里社則祭鄉厲。後定郡邑厲、鄉厲,皆以清明日、七月十五日、十月朔日。


廟制禘佩時享薦新加上謚號廟諱

○宗廟之制

明初作四親廟於宮城東南,各為一廟。皇高祖居中,皇曾祖東第一,皇祖西第一,皇考東第二,皆南向。每廟中室奉神主。東西兩夾室,旁兩廡。三門,門設二十四戟。外為都宮。正門之南齋次,其西饌次,俱五間,北向。門之東,神廚五間,西向。其南宰牲池一,南向。

洪武元年,命中書省集儒臣議祀典,李善長等言:

周制,天子七廟。而《商書》曰:「七世之廟,可以觀德」,則知天子七廟,自古有之。太祖百世不遷。三昭三穆以世次比,至親盡而遷。此有天下之常禮。若周文王、武王雖親盡宜祧,以其有功當宗,故皆別立一廟,謂之文世室、武世室,亦百世不遷。

漢每帝輒立一廟,不序昭穆,又有郡國廟及寢園廟。光武中興,於洛陽立高廟,祀高祖及文、武、宣、元五帝。又於長安故高廟中,祀成、哀、平三帝。別立四親廟於南陽舂陵,祀父南頓君以上四世。至明帝,遺詔藏主於光烈皇后更衣別室。後帝相承,皆藏於世祖之廟。由是同堂異室之制,至於元莫之改。

唐高祖尊高曾祖考,立四廟於長安。太宗議立七廟,虛太祖之室。玄宗創制,立九室,祀八世。文宗時,禮官以景帝受封於唐,高祖、太宗創業受命,百代不遷。親盡之主,禮合祧遷,至禘佩則合食如常。其後以敬、文、武三宗為一代,故終唐之世,常為九世十一室。

宋自太祖追尊僖、順、翼、宣四祖,每遇禘,則以昭穆相對,而虛東向之位,神宗奉僖祖為太廟始祖,至徽宗時增太廟為十室,而不祧者五宗。崇寧中,取王肅說,謂二祧在七世之外,乃建九廟。高宗南渡,祀九世。至於寧宗,始別建四祖殿,而正太祖東向之位。

元世祖建宗廟於燕京,以太祖居中,為不遷之祖。至泰定中,為七世十室。

今請追尊高曾祖考四代,各為一廟。

於是上皇高祖考謚曰玄皇帝,廟號德祖,皇高祖妣曰裕玄皇后。皇曾祖考謚曰恒皇帝,廟號懿祖,皇曾祖妣曰恒皇后。皇祖考謚曰裕皇帝,廟號熙祖,皇祖妣曰裕皇后。皇考謚曰淳皇帝,廟號仁祖,皇妣陳氏曰淳皇后。

詔製太廟祭器。太祖曰:「禮順人情,可以義起。所貴斟酌得宜,隨時損益。近世泥古,好用古籩豆之屬,以祭其先。生既不用,死而用之,甚無謂也。孔子曰:『事死如事生,事亡如事存。』其製宗廟器用服御,皆如事生之儀。」於是造銀器,以金塗之。酒壺盂盞皆八,朱漆盤碗二百四十,及楎椸枕簟篋笥幃幔浴室皆具。後又詔器皿以金塗銀者,俱易以金。

二年,詔太廟祝文止稱孝子皇帝,不稱臣。凡遣皇太子行禮,止稱命長子某,勿稱皇太子。後稱孝玄孫皇帝,又改稱孝曾孫嗣皇帝。初,太廟每室用幣一。二年,從禮部議,用二白繒。又從尚書崔亮奏,作圭瓚。

八年,改建太廟。前正殿,後寢殿。殿翼皆有兩廡。寢殿九間,間一室,奉藏神主,為同堂異室之制。九年十月,新太廟成。中室奉德祖,東一室奉懿祖,西一室奉熙祖,東二室奉仁祖,皆南向。十五年,以孝慈皇后神主祔享太廟,其後皇后祔廟仿此。建文即位,奉太祖主祔廟。正殿神座次熙祖。東向。寢殿神主居西二室,南向。成祖遷都,建廟如南京制。

宣德元年七月,禮部進太宗神主祔廟儀:先期一日,遣官詣太廟行祭告禮。午後,於几筵殿行大祥祭。翼日昧爽,設酒果於几筵殿,設御輦二、冊寶亭四於殿前丹陛上。皇帝服淺淡服,行祭告禮畢,司禮監官跪請神主陞輦,詣太廟奉安。內使二員捧神主,內使四員捧冊寶,由殿中門出,安奉於御輦、冊寶亭。皇帝隨行至思善門,易祭服,升輅。至午門外,儀衛傘扇前導,至廟街門內,皇帝降輅。監官導詣御輦前奏,跪請神主奉安太廟,俯伏,興。內使捧神主冊寶,皇帝從,由中門入,至寢廟東第三室,南向奉安。皇帝叩頭,畢,祭祀如時祭儀。文武官具祭服行禮。其正殿神座,居仁祖之次,西向。二年五月,仁宗神主祔廟,如前儀。寢殿,西第三室,南向。正殿,居高祖之次,東向。其後大行祔廟仿此。正統七年十二月,奉昭皇后神主祔廟,神主詣列祖神位前謁廟。禮畢,太常寺官唱賜座,內侍捧衣冠,與仁宗同神位。唱請宣宗皇帝朝見,內侍捧宣宗衣冠置褥位上,行四拜禮訖,安奉於座上。

孝宗即位,憲宗將升祔。時九廟已備,議者咸謂德、懿、熙、仁四廟,宜以次奉祧。禮臣謂:「國家自德祖以上,莫推世次,則德祖視周后稷,不可祧。憲宗升祔,當祧懿祖。宜於太廟寢殿後,別建祧殿,如古夾室之制。歲暮則奉祧主合享,如古祫祭之禮。」吏部侍郎楊守陳言:「《禮》,天子七廟,祖功而宗德。德祖可比商報乙、周亞圉,非契、稷比。議者習見宋儒嘗取王安石說,遂使七廟既有始祖,又有太祖。太祖既配天,又不得正位南向,非禮之正。今請并祧德、懿、熙三祖,自仁祖下為七廟,異時祧盡,則太祖擬契、稷,而祧主藏於後寢,祫禮行於前殿。時享尊太祖,祫祭尊德祖,則功德並崇,恩義亦備。」帝從禮官議,建祧廟於寢殿後,遣官祭告宗廟。帝具素服告憲宗几筵,祭畢,奉遷懿祖神主衣冠於後殿,床幔、御座、儀物則貯於神庫。其後奉祧仿此。

嘉靖九年春,世宗行特享禮。令於殿內設帷幄如九廟,列聖皆南向,各奠獻,讀祝三,餘如舊。十年正月,帝以廟祀更定,告於太廟、世廟并祧廟三主。遷德祖神主於祧廟,奉安太祖神主於寢殿正中,遂以序進遷七宗神位。丁酉,帝詣太廟行特享禮。九月,諭大學士李時等,以「宗廟之制,父子兄弟同處一堂,於禮非宜。太宗以下宜皆立專廟,南向。」尚書夏言奏:「太廟兩傍,隙地無幾,宗廟重事,始謀宜慎。」未報。中允廖道南言:「太宗以下宜各建特廟於兩廡之地。有都宮以統廟,不必各為門垣。有夾室以藏主,不必更為寢廟。第使列聖各得全其尊,皇上躬行禮於太祖之廟,餘遣親臣代獻,如古諸侯助祭之禮。」帝悅,命會議。言等言:「太廟地勢有限,恐不能容,若小其規模,又不合古禮。且使各廟既成,陛下遍歷群廟,非獨筋力不逮,而日力亦有不給,古者宗伯代后獻之文,謂在一廟中,而代后之亞獻。未聞以人臣而代主一廟之祭者也。且古諸侯多同姓之臣,今陪祀執事者,可擬古諸侯之助祭者乎?先臣丘浚謂宜間日祭一廟,歷十四日而遍。此蓋無所處,而強為之說耳。若以九廟一堂,嫌於混同。請以木為黃屋,如廟廷之制,依廟數設之,又設帷幄於其中,庶得以展專奠之敬矣。」議上,不報。

十三年,南京太廟災。禮部尚書湛若水請權將南京太廟香火并於南京奉先殿,重建太廟,補造列聖神主。帝召尚書言與群臣集議。言會大學士張孚敬等言:「國有二廟,自漢惠始。神有二主,自齊桓始。周之三都廟,乃遷國立廟,去國載主,非二廟二主也。子孫之身乃祖宗所依,聖子神孫既親奉祀事於此,則祖宗神靈自當陟降於此。今日正當專定廟議,一以此地為根本。南京原有奉先殿,其朝夕香火,當合併供奉如常。太廟遺址當仿古壇墠遺意,高築牆垣,謹司啟閉,以致尊嚴之意。」從之。

時帝欲改建九廟。夏言因言:「京師宗廟,將復古制,而南京太廟遽災,殆皇天列祖佑啟默相,不可不靈承者。」帝悅,詔春和興工。諸臣議於太廟南,左為三昭廟,與文祖世室而四,右為三穆廟。群廟各深十六丈有奇,而世室殿寢稍崇,縱橫深廣,與群廟等。列廟總門與太廟戟門相並,列廟後垣與太廟祧廟後牆相並。具圖進。帝以世室尚當隆異,令再議。言等請增拓世室前殿,視群廟崇四尺有奇,深廣半之;寢殿視群廟崇二尺有奇,深廣如之。報可。十四年正月,諭閣臣:「今擬建文祖廟為世室,則皇考世廟字當避。」張孚敬言:「世廟著《明倫大典》,頒詔四方,不可改。文世室宜稱太宗廟。其餘群廟不用宗字,用本廟號,他日遞遷,更牌額可也。」從之。二月,盡撤故廟改建之。諸廟各為都宮,廟各有殿有寢。太祖廟寢後有祧廟,奉祧主藏焉。太廟門殿皆南向,群廟門東西向,內門殿寢皆南向。十五年十二月,新廟成,更創皇考廟曰睿宗獻皇帝廟。帝乃奉安德、懿、熙、仁四祖神主於祧廟,太祖神主於太廟,百官陪祭如儀。翌日,奉安太宗以下神主,列於群廟,命九卿正官及武爵重臣,俱詣太宗廟陪祭。文三品以上,武四品以上,分詣群廟行禮。又擇日親捧太祖神主,文武大臣捧七宗神主,奉安於景神殿。

二十年四月,太廟災,成祖、仁宗主毀,奉安列聖主於景神殿。遣大臣詣長陵、獻陵告題帝后主,亦奉安景神殿。二十二年十月,以舊廟基隘,命相度規制。議三上,不報。久之,乃命復同堂異室之舊,廟制始定。二十四年六月,禮部尚書費寀等以太廟安神,請定位次。帝曰:「既無昭穆,亦無世次,只序倫理。太祖居中,左四序成、宣、憲、睿,右四序仁、英、孝、武。皆南向。』七月,以廟建禮成,百官表賀,詔天下。新廟仍在闕左,正殿九間,前兩廡,南戟門。門左神庫,右神廚。又南為廟門,門外東南宰牲亭,南神宮監,西廟街門。正殿後為寢殿,奉安列聖神主,又後為祧廟,藏祧主,皆南向。

二十七年,帝欲祔孝烈皇后方氏於太廟,而祧仁宗。大學士嚴嵩、禮部尚書徐階等初皆持不可,既而不能堅其議。二十九年十一月,祧仁宗,遂祔孝烈於西第四室。隆慶六年八月,穆宗將祔廟,敕禮臣議當祧廟室。禮科陸樹德言:「宣宗於穆宗僅五世,請仍祔睿宗於世廟,而宣宗勿祧。」疏下禮部,部議宣宗世次尚近,祧之未安。因言:「古者以一世為一廟,非以一君為一世,故晉之廟十一室而六世,唐之廟十一室而九世。宋自太祖上追四祖至徽宗,始定為九世十一室之制,以太祖、太宗同為一世故也。其後徽宗祔以與哲宗同一世,高宗祔以與欽宗同一世,皆無所祧,及光宗升祔,增為九世十二室。今自宣宗至穆宗凡六世,上合二祖僅八世,準以宋制,可以無祧,但於寢殿左右各增一室,則尊祖敬宗,並行不悖矣。」帝命如舊敕行,遂祧宣宗。天啟元年七月,光宗將祔廟。太常卿洪文衡請無祧憲宗,而祧睿宗。不聽。

○禘佩

洪武元年祫饗太廟。德祖皇考妣居中。南向。懿祖皇考妣東第一位,西向。熙祖皇考妣西第一位,東向。仁祖皇考妣東第二位,西向。七年,御史答祿與權言:「皇上受命七年而禘祭未舉。宜參酌古今,成一代之典。」詔下禮部、太常司、翰林院議,以為:「虞、夏、商、周世系明白,故禘禮可行。漢、唐以來,莫能明其始祖所自出,當時所謂禘祭,不過祫已祧之祖而祭之,乃古之大祫,非禘也。宋神宗嘗曰:『禘者,所以審諦祖之所自出。』是則莫知祖之所自出,禘禮不可行也。今國家追尊四廟,而始祖所自出者未有所考,則禘難遽行。」太祖是其議。弘治元年,定每歲暮奉祧廟懿祖神座於正殿左,居熙祖上,行祫祭之禮。

嘉靖十年,世宗以禘祫義詢大學士張璁,令與夏言議。言撰《禘義》一篇獻之,大意謂:「自漢以下,譜牒難考,欲如虞夏之禘黃帝,商周之禘帝嚳,不能盡合。謹推明古典,采酌先儒精微之論,宜為虛位以祀。」帝深然之。會中允廖道南謂朱氏為顓頊裔,請以《太祖實錄》為據,禘顓頊。遂詔禮部以言、道南二疏,會官詳議。諸臣咸謂:「稱虛位者茫昧無據,尊顓頊者世遠難稽。廟制既定高皇帝始祖之位,當禘德祖為正。」帝意主虛位,令再議。而言復疏論禘德祖有四可疑,且言今所定太祖為太廟中之始祖,非王者立始祖廟之始祖。帝併下其章。諸臣乃請設虛位,以禘皇初祖,南向,奉太祖配,西向。禮臣因言,大祫既歲舉,大禘請三歲一行,庶疏數適宜。帝自為文告皇祖,定丙、辛歲一行,敕禮部具儀擇日。四月,禮部上大禘儀注。前期告廟,致齋三日,備香帛牲醴如時享儀。錦衣衛設儀衛,太常卿奉皇初祖神牌、太祖神位於太廟正殿,安設如圖儀。至日行禮,如大祀圜丘儀。及議祧德祖,罷歲除祭,以冬季中旬行大祫禮。太常寺設德祖神位於太廟正中,南向。懿祖而下,以次東西向。

十五年,復定廟饗制。立春犆享,各出主於殿。立夏、立秋、立冬出太祖、成祖七宗主,饗太祖殿,為時祫。季冬中旬,卜日出四祖及太祖、成祖七宗主,饗太祖殿,為大祫。祭畢,各歸主於其寢。十七年定大祫祝文。九廟帝后謚號俱全書,時祫止書某祖、某宗某皇帝。更定季冬大祫日,奉德、懿、熙、仁及太祖異室皆南向,成祖西向北上,仁宗以下七宗東西相向。二十年十一月,禮官議,歲暮大祫,當陳祧主,而景神殿隘,請暫祭四祖於後寢,用連几,陳籩豆,以便周旋。詔可。二十二年,定時享、大祫,罷出主、上香、奠獻等儀,臨期捧衣冠出納。太常及神宮監官奉行。二十四年,罷季冬中旬大祫,並罷告祭,仍以歲除日行大祫,禮同時享。二十八年,復告祭儀。穆宗即位,禮部以大行皇帝服制未除,請遵弘治十八年例,歲暮大祫、孟春時享兩祭,皆遣官攝事。樂設而不作,帝即喪次致齋,陪祀官亦在二十七日之內,宜令暫免。從之。

○時享

洪武元年,定宗廟之祭,每歲四孟及歲除,凡五享。學士陶安等言:「古者四時之祭,三祭皆合享於祖廟,惟春祭於各廟。自漢而下,廟皆同堂異室,則四時皆合祭。今宜仿近制,合祭於第一廟,庶適禮之中,無煩瀆也。」太祖命孟春特祭於各廟,三時及歲除則祫佩祭於德祖廟。二年,定時享之制,春以清明,夏以端午,秋以中元,冬以冬至。歲除如舊。三年,禮部尚書崔亮言:「孟月者,四時之首。因時變,致孝思,故備三牲黍稷品物以祭。至仲季之月,不過薦新而已。既行郊祀,則廟享難舉,宜改從舊制。其清明等節,各備時物以薦。」從之。九年,新建太廟。凡時享,正殿中設德祖帝后神座,南向。左懿祖,右熙祖,東西向。仁祖次懿祖。凡神座俱不奉神主,止設衣冠。禮畢,藏之。孟春擇上旬日,三孟用朔日,及歲除皆合享。自是五享皆罷特祭,而行合配之禮。二十一年,定時享儀。更前制,迎神四拜,飲福四拜,禮畢四拜。二十五年,定時享。若國有喪事,樂備而不作。

正統三年正月,享太廟。禮部言,故事,先三日,太常寺奏祭祀,御正殿受奏。是日,宣宗皇帝忌辰,例不鳴鐘鼓,第視事西角門。帝以祭祀重事,仍宜升殿,餘悉遵永樂間例行之。天順六年,閣臣以皇太后喪,請改孟冬時享於除服後。從之。成化四年,禮部以慈懿太后喪,請改孟秋享廟於初七日。不從。

嘉靖十一年,大學士張孚敬等言:「太廟祭祀,但設衣冠。皇上改行出主,誠合古禮。但遍詣群廟,躬自啟納,不免過勞。今請太祖神主,躬自安設。群廟帝后神主,則以命內外捧主諸臣。」帝從其請。十七年,定享祫禮,凡立春特享,親祭太祖,遣大臣八人分獻諸帝,內臣八人分獻諸后。立夏時祫,各出主於太廟。太祖南向,成祖西向,序七宗之上,仁、宣、英、憲、孝、睿、武宗東西相向。秋冬時祫,如夏禮。二十四年,新廟成,復定享祫止設衣冠,不出主。隆慶元年,孟夏時享,以世宗几筵未撤,遵正德元年例,先一日,帝常服祭告几筵,祗請諸廟享祀。其後,時享、祫祭在大祥內者,皆如之。

○薦新

洪武元年,定太廟月朔薦新儀物:正月,韭、薺、生菜、雞子、鴨子。二月,水芹、蔞蒿、臺菜、子鵝。三月,茶、筍、鯉魚、鮆魚。四月,櫻桃、梅、杏、鰣魚、雉。五月,新麥、王瓜、桃、李、來禽、嫩雞。六月,西瓜、甜瓜、蓮子、冬瓜。七月,菱、梨、紅棗、蒲萄。八月,芡、新米、藕、茭白、姜、鱖魚。九月,小紅豆、栗、柿、橙、蟹、扁魚。十月,木瓜、柑、橘、蘆菔、兔、雁。十一月,蕎麥、甘蔗、天鵝、鶿䳓、鹿。十二月,芥菜、菠菜、白魚、鯽魚。其禮皆天子躬行。未幾,以屬太常。二年詔,凡時物,太常先薦宗廟,然後進御。三年,定朔日薦新,各廟共羊一、豕一、籩豆八、簠簋登鉶各二、酒尊三,及常饌鵝羹飯。太常卿及與祭官法服行禮。望祭,止常饌鵝羹飯,常服行禮。又有獻新之儀,凡四方別進新物,在月薦外者,太常卿與內使監官常服獻於太廟。不行禮。其後朔望祭祀,及薦新、獻新,俱於奉先殿。

○加上謚號

洪武元年,追尊四廟謚號,冊寶皆用玉。冊簡長尺二寸,廣一寸二分,厚五分,簡數從文之多寡。聯以金繩,藉以錦褥,覆以紅羅泥金夾帕,冊匣朱漆鏤金,龍鳳文。其寶篆文,廣四寸九分,厚一寸二分,金盤龍紐,系以錦綬,裹以紅錦,加帕納於盝,盝裝以金。德祖冊文曰:「孝玄孫嗣皇帝臣某,再拜稽首上言:臣聞尊敬先世,人之至情,祖父有天下,傳之子孫,子孫有天下,追尊祖父,此古今通義也。臣遇天下兵起,躬披甲胄,調度師旅,戡定四方,以安人民,土地日廣。皆承祖宗之庇。臣庶推臣為皇帝,而先世考妣未有稱號。謹上皇高祖考府君尊號曰玄皇帝,廟號德祖。伏惟英爽。鑒此孝思。」其寶文曰「德祖玄皇帝之寶」。玄皇后、懿祖以下,帝后冊寶並同。建文時,追尊謚冊之典,以革除無考。

永樂元年五月,進高皇帝、高皇后謚議。前一日,於奉天殿中設謚議案。是日早,帝袞冕升殿,如常儀。文武官朝服四拜。禮部官奏進尊謚議。序班二員引班首升丹陛,捧謚議官以謚議文授班首,由中門入,至殿中。贊進尊謚議。駕興,詣謚議文案。班首進置於案,跪,百官皆跪。帝覽畢,復坐。班首與百官俯伏興,復位,再行四拜。禮畢。帝親舉謚議,付翰林院臣撰冊文。

六月,以上尊謚,先期齋戒,遣官祭告天地、宗廟、社稷。鴻臚寺設香案於奉天殿。是日早,內侍以冊寶置於案。太常寺於太廟門外丹陛上,皇考、皇妣神御前各設冊寶案。鴻臚寺設冊寶輿於奉天門外,鹵簿、樂懸如常儀。百官祭服詣太廟門外立俟,執事官并宣冊寶官,先從太廟右門入,序立殿右。帝袞冕御華蓋殿,捧冊寶官四員祭服,於奉天殿東西序立。鴻臚寺奏請行禮。帝出奉天殿冊寶案前,捧冊寶官各捧前行,置彩輿內,鹵簿大樂前導。帝乘輿,隨彩輿行。至午門外降輿,陞輅,至太廟門。百官跪俟綵輿過,興。帝降輅,隨綵輿至太廟中門外。捧冊寶官各捧前行,帝隨行,至丹陛上。捧冊寶置於案,典儀傳唱如常,內贊奏就位,典儀奏迎神奏樂。樂止,內贊奏帝四拜,百官同。典儀奏進冊寶,捧冊寶官前行,駕由左門入,至廟中,詣皇考神御前。奏跪,搢圭。奏進冊,捧冊官跪進於帝左,帝受冊以授執事官,置於案左,奏出圭,贊宣冊,宣冊官跪宣於帝左。冊文曰:「惟永樂元年,歲次癸未,六月丁未朔,越十一日丁巳,孝子嗣皇帝臣某,謹拜手稽首言:臣聞俊德贊堯,重華美舜,禹、湯、文、武,列聖相承,功德並隆,咸膺顯號。欽惟皇考皇帝,統天肇運,奮自布衣,戡定禍亂,用夏變夷,以孝治天下。四十餘年,民樂永熙,禮樂文章,垂憲萬世,德合乾坤,明同日月,功超千古,道冠百王。謹奉冊寶,上尊謚曰聖神文武欽明啟運俊德成功統天大孝高皇帝,廟號太祖。伏惟神靈陟降,陰騭下民,覆幬無極,與天常存。」宣冊訖,奏搢圭。奏進寶,捧寶官以寶跪進於帝左。帝受寶,以授執事官,置於案右。奏出圭。贊宣寶,宣寶官跪宣於帝右,寶文如謚號。宣寶訖,奏俯伏,興。帝詣皇妣神御前,進宣冊寶如前儀。冊文曰:「臣聞自古后妃,皆承世緒。媯汭嬪虞,發祥帝室,周姜輔治,肇基邦君。欽惟皇妣孝慈皇后,以聖輔聖,同起側微,弘濟艱難,化家為國。克勤克儉,克敬克誠,克孝克慈,以奉神靈之統,理萬物之宜。正位中宮十有五年,家邦承式,天下歸仁。謹奉冊寶,上尊謚曰孝慈昭憲至仁文德承天順聖高皇后。伏惟聖靈陟降,膺慈顯名,日月光華,照臨永世。」寶文如謚號。宣寶訖,帝復位。奏四拜,百官同。行祭禮如常儀。翌日,頒詔天下。以上謚禮成,賜陪祀執事官宴,餘官人賜鈔一錠。

仁宗即位,九月,禮部同諸臣進大行皇帝仁孝皇后謚議。仁宗立受之,覽畢,流涕不已,以付翰林院撰謚冊。禮部奏上謚儀,前期齋戒遣祭如常,內侍置冊寶輿於奉天門。厥明,捧冊寶置輿中。帝衰服詣奉天門,內侍舉冊寶輿,帝隨輿後行,降階,升輅。百官立金水橋南,北向跪。俟輿過,興。隨至思善門外,序立,北向。帝降輅。冊寶輿由中門入,至幾筵殿丹陛上。帝由左門入,就丹陛上拜位。捧冊寶官由殿左門入,至几筵前。導引官奏四拜,皇太孫、親王、皇孫陪拜丹陛上,百官陪拜思善門外。帝由殿左門入,詣大行皇帝位前,跪進冊、進寶。宣冊寶官跪宣畢,奏俯伏、興。帝詣仁孝皇后神位前,禮並同。奏復位四拜,皇太孫以下同。禮畢,行祭禮。是日,改題仁孝皇后神主,詔頒天下。是後,上皇帝及太皇太后、皇太后尊謚,皆仿此。

嘉靖十七年,世宗用豐坊奏,加獻皇帝廟號,稱宗配帝,乃改奉太宗為成祖。命製二聖神位於南宮,遂詣景神殿,奉冊寶。尊文皇帝曰成祖啟天弘道高明肇運聖武神功純仁至孝文皇帝,尊獻皇帝曰睿宗知天守道洪德淵仁寬穆純聖恭儉敬文獻皇帝。又上皇天上帝大號。十一月朔,帝詣南郊,恭進冊表。禮成,還詣太廟,奉冊寶,加上高皇帝尊號曰太祖開天行道肇紀立極大聖至神仁文義武俊德成功高皇帝,加上高皇后尊號曰孝慈貞化哲順仁徽成天育聖至德高皇后。是日,中宮捧高皇后主,助行亞獻禮,文武官命婦陪祀。復擇日詣太廟,行改題神主禮。

○廟諱

天啟元年正月,從禮部奏,凡從點水加各字者,俱改為「雒」,從木加交字者,俱改為「較」。惟督學稱較字未宜,應改為學政。各王府及文武職官,有犯廟諱御名者,悉改之。

奉先殿奉慈殿獻皇帝廟新王從饗功臣配饗王國宗廟群臣家廟

○奉先殿

洪武三年,太祖以太廟時享,未足以展孝思,復建奉先殿於宮門內之東。以太廟象外朝,以奉先殿象內朝。正殿五間,南向,深二丈五尺。前軒五間,深半之。製四代帝后神位、衣冠,定儀物、祝文。每日朝晡,帝及皇太子諸王二次朝享。皇后率嬪妃日進膳羞,諸節致祭,月朔薦新,其品物視元年所定。惟三月不用鮆魚,四月減鰣魚,益以王瓜彘,五月益以茄,九月減柿蟹,十月減木瓜蘆菔,益以山藥,十一月減天鵝鶿䳓,益以麞。皆太常奏聞,送光祿寺供薦。凡遇時新品物,太常供獻。又錄皇考妣忌日,歲時享祀以為常。成祖遷都北京,建如制。宣德元年,奉太宗祔廟畢,復遣鄭王瞻颭詣奉先殿,設酒果祭告,奉安神位。天順七年,奉孝恭皇后祔廟畢,帝還行奉安神位禮,略如祔廟儀。弘治十七年,吏部尚書馬文升言:「南京進鮮船,本為奉先殿設。挽夫至千人,沿途悉索。今揚、徐荒旱,願仿古兇年殺禮之意,減省以蘇民困。」命所司議行之。武宗即位,祧熙祖。奉先殿神位亦遷德祖之西,其衣冠、床幔、儀物貯於神庫。

嘉靖十四年,定內殿之祭并禮儀。清明、中元、聖誕、冬至、正旦,有祝文,樂如宴樂。兩宮壽旦,皇后并妃嬪生日,皆有祭,無祝文、樂。立春、元宵、四月八日、端陽、中秋、重陽、十二月八日,皆有祭,用時食。舊無祝文,今增告詞。舊儀,但一室一拜,至中室跪祝畢,又四拜,焚祝帛。今就位四拜,獻帛爵,祝畢,后妃助亞獻,執事終獻,撤饌又四拜。忌祭,舊具服作樂,今更淺色衣,去樂。凡祭方澤、朝日夕月,出告、回參,及冊封告祭,朔望行禮,皆在焉。十五年,禮部尚書夏言等奏:「悼靈皇后神主,先因祔於所親,暫祔奉慈殿孝惠太后之側。茲三后神主既擬遷於陵殿,則悼靈亦宜暫遷奉先殿旁室,享祀祭告,則一體設饌。」從之。隆慶元年,禮部言:「舊制,太廟一歲五享,而節序忌辰等祭,則行於奉先殿。今孝潔皇后既祔太廟,則奉先殿亦宜奉安神位。」乃設神座、儀物於第九室,遣官祭告如儀。萬曆三年,帝欲以孝烈、孝恪二后神位奉安於奉先殿。禮官謂世宗時,議祔陵祭,不議祔內殿。帝曰:「奉先殿見有孝肅、孝穆、孝惠三后神位,俱皇祖所定,宜遵行祔安。」蓋當時三后既各祔陵廟,仍并祭於奉先殿,而外廷莫知也。命輔臣張居正等入視。居正等言:「奉先殿奉安列聖祖妣,凡推尊為后者,俱得祔享內殿,比之太廟一帝一后者不同,今亦宜奉安祔享。」從之。

先是,冊封告祭,以太常寺官執事,仍題請遣官。到萬曆元年,帝親行禮,而遣官之請廢。二年,太常寺以內殿在禁地,用內官供事便。帝俞其請。凡聖節、中元、冬至、歲暮,嘉靖初俱告祭於奉先殿。十五年,罷中元祭。四十五年,罷歲暮祭。隆慶元年,罷聖節、冬至祭。其方澤、朝日、夕月,出告、回參,嘉靖中行於景神殿。隆慶元年,仍行於奉先殿。諸帝后忌辰,嘉靖以前行於奉先殿。十八年,改高皇帝、后忌辰於景神殿,文皇帝、后以下於永孝殿。二十四年,仍行於奉先殿。凡內殿祭告,自萬曆二年後,親祭則祭品告文執事,皆出內監。遣官代祭,則皆出太常。惟品用脯醢者,即親祭亦皆出太常。萬曆十四年,禮臣言:「近年皇貴妃冊封,祭告奉先殿,祝文執事出內庭,而祭品取之太常,事體不一。夫太常專主祀享,而光祿則主膳羞。內庭祭告,蓋取象於食時上食之義也。宜遵舊制,凡祭告內殿,無論親行、遣官,其祭品光祿寺供;惟告文執事人,親行則辦之內庭,遣官則暫用太常寺。」從之。

○奉慈殿

孝宗即位,追上母妃孝穆太后紀氏謚,祔葬茂陵。以不得祔廟,遂於奉先殿右別建奉慈殿以祀。一歲五享,薦新忌祭,俱如太廟奉先殿儀。弘治十七年,孝肅周太后崩。先是成化時,預定周太后祔葬、祔祭之議,至是召輔臣議祔廟禮。劉健等言:「議誠有之,顧當年所引唐、宋故事,非漢以前制也。」帝以事當師古,乃援孝穆太后別祭奉慈殿為言,而命廷臣議。健退,復疏論其事,以堅帝心。於是英國公張懋、吏部尚書馬文升等言:「宗廟之禮,乃天下公議,非子孫得以私之。殷、周七廟,父昭子穆,各有配座,一帝一后,禮之正儀。《春秋》書『考仲子之宮』,胡安國《傳》云:『孟子入惠公之廟,仲子無祭享之所。』以此見魯秉周禮,先王之制猶存,祖廟無二配故也。伏睹憲宗敕諭,有曰『朕心終不自安』。竊窺先帝至情,以重違慈意,因勉從並配之議。群臣欲權以濟事,亦不得已而為此也。據禮區處,上副先帝在天遺志,端有待於今日。稽之《周禮》,有祀先妣之文,《疏》云『姜嫄也』,《詩》所謂『閟宮』是已。唐、宋推尊太后,不配食祖廟者,則別立殿以享之,亦得閟宮之義。我朝祖宗迄今已溢九廟,配皆無二。今宜於奉先殿外建一新廟,如《詩》之閟宮,宋之別殿,歲時薦享,仍稱太皇太后,則情義兩盡。」議上,復召健等至素幄,袖出《奉先殿圖》,指西一區曰:「此奉慈殿也。」又指東一區曰:「此神廚也。」欲於此地別建廟,奉遷孝穆神主,併祭於此。健等皆對曰:「最當。」已而欽天監奏,年方有礙,廷議暫奉於奉慈殿正中,徙孝穆居左。

及孝宗崩,武宗即位,禮部始進奉安孝肅神主儀。前期致齋三日,告奉先殿及孝宗几筵。是日早,帝具黑翼善冠、淺淡色服、黑犀帶,告孝穆神座。禮畢,帝詣神座前,請神主降座。帝捧主立,內執事移神座於殿左間。帝奉安訖,行叩頭禮,至午,帝詣清寧宮孝肅几筵,行禮畢,內侍進神主輿於殿前,衣冠輿於丹陛上。帝詣拜位,親王吉服後隨,四拜,興。帝捧神主由殿中門出,奉安輿內。執事捧衣冠置輿後隨。帝率親王步從。至寶善門外,太皇太后、皇太后率宮妃迎於門內。先詣奉慈殿,序立於殿西。神主輿至奉先殿門外,少駐。帝詣輿前跪,請神主詣奉先殿,俯伏,興,捧神主由殿左門入,至殿內褥位,跪,置神主。帝行五拜三叩頭禮畢,捧神主,仍由左門出,安輿內。至奉慈殿門外,帝捧神主由中門入,奉安於神座訖,行安神禮,三獻如常儀。太皇太后以下四拜。禮畢,內侍官設褥位於殿正中之南。帝詣孝穆皇太后神座前,跪請神主謁孝肅太皇太后,跪置於褥位上,俯伏,興,行五拜三叩頭禮。畢,帝捧主興,仍安於神座訖,行安神禮如前,皇太后以下四拜。

嘉靖元年,世宗奉孝惠邵太后祔祀。八年二月,禮部尚書方獻夫等言:「悼靈皇后,禮宜祔享太廟,但今九廟之制已備。考唐、宋故事,后於太廟未有本室,則創別廟。故《曲臺禮》有別廟皇后禘祫於太廟之文。又《禮記·喪服小記》:『婦祔於祖姑,祖姑有三人,則祔於親者。』釋之者曰:『親者謂舅所生母也。』今孝惠太皇太后實皇考獻皇帝之生母,則悼靈皇后當祔於側。」詔可。三月,行祔廟禮。先期祭告諸殿。至期,請悼靈后主詣奉慈殿奉安。內侍捧神主、謚冊、衣冠隨帝至奉先殿謁見。帝就位,行五拜三叩頭禮。次詣崇先殿,次詣奉慈殿,謁三太后,內侍捧主安神座,皇妃以下四拜。

十五年,帝以三太后別祀奉慈殿,不若奉於陵殿為宜,廷臣議:「古天子宗廟,惟一帝一后,所生母,薦於寢,身歿而已。孝宗奉慈殿之祭,蓋子祀生母,以盡終身之孝焉耳。然《禮》『妾母不世祭』,《疏》曰:『不世祭者,謂子祭之,於孫則止。』明繼祖重,故不復顧其私祖母也。今陛下於孝肅,曾孫也;孝穆,孫屬也;孝惠,孫也。禮不世祭,議當祧。考宋熙寧罷奉慈廟故事,與今同。宜遷主陵廟,歲時祔享如故。」報可。奉慈殿遂罷。世宗孝烈后,隆慶時祀弘孝殿,萬曆三年遷祔奉先殿。穆宗母孝恪皇太后,隆慶初祀神霄殿,又祔孝懿后於其側。六年,孝懿祔太廟,萬歷三年,孝恪遷祔奉先殿,二殿俱罷。

○獻皇帝廟

嘉靖二年四月,始命興獻帝家廟亭祀,樂用八佾。初,禮官議廟制未決,監生何淵上書,請立世室於太廟東。禮部尚書汪俊等皆謂不可。帝諭奉先殿側別立一室,以盡孝思。禮官集議言:「奉慈之建,禮臣據姜嫄特廟而言。至為本生父立廟大內,古所未有,惟漢哀為定陶共王建廟京師,不可為法。」詹事石珤等亦言不可。不聽。葺奉慈殿後為觀德殿以奉之。四年四月,淵已授光祿寺署丞,復上書請立世室,崇祀皇考於太廟,禮部尚書席書等議:「天子七廟,周文、武並有功德,故立文、武世室於三昭穆之上。獻皇帝追稱帝號,未為天子。淵妄為諛詞,乞寢其奏。」帝令再議,書等言:「將置主於武宗上。則武宗君也,分不可僭。置武宗下,則獻皇叔也,神終未安。」時廷臣於稱考稱伯,異同相半,至議祔廟,無一人以為可者。學士張璁、桂萼亦皆以為不可,書復密疏爭之。帝不聽,復令會議。乃準漢宣故事,於皇城內立一禰廟,如文華殿制。籩豆樂舞,一用天子禮。帝親定其名曰世廟。五年七月,諭工部以觀德殿窄隘,欲別建於奉天殿左。尚書趙璜謂不可,不聽。乃建於奉先之東,曰崇先殿。十三年,命易承天家廟曰隆慶殿。十五年,以避渠道,遷世廟,更號曰獻皇帝廟,遂改舊世廟曰景神殿,寢殿曰永孝殿。

十七年,以豐坊請,稱宗以配明堂。禮官不敢違,集議者久之,言:「古者父子異昭穆,兄弟同世數。故殷有四君一世而同廟,宋太祖、太宗同居昭位。今皇考與孝宗當同一廟。」遂奉獻皇帝祔太廟。二十二年,更新太廟,廷議睿宗、孝宗並居一廟,同為昭。帝責諸臣不竭忠任事,寢其議。已而左庶子江汝璧請遷皇考廟於穆廟首,以當將來世室,與成祖廟並峙。右贊善郭希顏又欲於太祖廟文世室外,止立四親廟,而祧孝宗、武宗。以禮臣斥其妄而止。二十四年六月,新太廟成,遂奉睿宗於太廟之左第四,序躋武宗上,而罷特廟之祀。四十四年,以舊廟柱產芝,更號曰玉芝宮,定日供時享儀。穆宗初,因禮臣請,乃罷時享及節序、忌辰、有事奉告之祭,但進日供而已。隆慶元年,禮科王治請罷獻皇祔廟,而專祀之世廟,章下所司。萬曆九年,禮科丁汝謙請仍專祭玉芝宮,復奉宣宗帝后冠服於太廟。帝責汝謙妄議,謫外任。天啟元年,太常少卿李宗延奏祧廟宜議,言:「睿宗入廟,世宗無窮之孝思也,然以皇上視之,則遠矣。俟光宗升祔時,或從舊祧,或從新議。蓋在孝子固以恩事親,而在仁人當以義率祖。」章下禮部,卒不能從。

○親王從饗

洪武三年,定以皇伯考壽春王、王夫人劉氏為一壇;皇兄南昌王、霍丘王、下蔡王、安豐王、霍丘王夫人翟氏、安豐王夫人趙氏為一壇;皇兄蒙城王、盱眙王、臨淮王、臨淮王夫人劉氏為一壇,後定夫人皆改稱妃;皇姪寶應王、六安王、來安王、都梁王、英山王、山陽王、昭信王為一壇,凡一十九位。春夏於仁祖廟東廡,秋冬及歲除於德祖廟東廡,皇帝行初獻禮,時獻官詣神位分獻。四年,進親王於殿內東壁。九年,新太廟成,增祀蒙城王妃田氏、盱眙王妃唐氏。正德中,御史徐文華言:「族有成人而無後者,祭終兄弟之孫之身。諸王至今五六世矣,宜祧。」禮官議不可。嘉靖間,仍序列東廡。二十四年,新建太廟成,復進列東壁,罷分獻。萬曆十四年,太常卿裴應章言:「諸王本從祖祔食。今四祖之廟已祧,而諸王無所於祔,宜罷享,而祔之祧廟。」禮部言:「祧以藏毀廟之主,為祖非為孫。禮有祧,不聞有配祧者。請仍遵初制,序列東廡為近禮。」報可。

○功臣配饗

洪武二年,享太廟,以廖永安、俞通海、張得勝、桑世傑、耿再成、胡大海、趙德勝配。設青布幃六於太廟庭中,遣官分獻。俟皇帝亞獻將畢,行禮。每歲春秋享廟,則配食於仁祖廟之東廡。三年,定配享功臣常遇春以下凡八位。春夏於仁祖廟西廡,秋冬於德祖廟西廡,設位東向,遂罷幃次之設。更定三獻禮,皇帝初獻,時獻官即分詣行禮,不拜。四年,太祖謂中書省臣:「太廟之祭,以功臣配列廡間。今既定太廟合祭禮,朕以祖宗具在,使功臣故舊歿者得少依神靈,以同享祀,不獨朝廷宗廟盛典,亦以寓朕不忘功臣之心。」於是禮官議:「凡合祭時,為黃布幄殿,中祖考神位,旁設兩壁,以享親王及功臣,令大臣分獻。」制可。已而命去布幄。九年,新太廟成,以徐達、常遇春、李文忠、鄧愈、湯和、沐英、俞通海、張德勝、胡大海、趙得勝、耿再成、桑世傑十二位配於西廡,罷廖永安。建文時,禮部侍郎宋禮言:「功臣自有雞籠山廟,請罷太廟配享。」帝以先帝所定,不從。且令候太廟享畢,別遣官即其廟祭之。洪熙元年,以張玉、朱能、姚廣孝配享太廟。遣張輔、朱勇、王通及尚寶少卿姚繼各祭其父。嘉靖九年,以廖道南言,罷姚廣孝。十年,以刑部郎中李瑜議,進劉基,位次六王。十六年,以武定侯郭勛奏,進其祖英。初,二廟功臣,位各以爵,及進基位公侯上,至是復令禮官議合二廟功臣敘爵。於是列英於桑世傑上,張玉、朱能於沐英下,基於世傑下。二十四年,進諸配位於新太廟西壁,罷分獻。萬曆十四年,太常卿裴應章言:「廟中列后在上,異姓之臣禮當別嫌。且至尊拜俯於下,諸臣之靈亦必不安。」命復改西廡,遣官分獻。天啟元年,太常少卿李宗延言:「前代文臣皆有從祀。我朝不宜獨闕。」下禮部議,不行。

○王國宗廟

洪武四年,禮部尚書陶凱等議定,王國宮垣內,左宗廟,右社稷。廟制,殿五間,寢殿如之,門三間。永樂八年,建秦愍王享堂,命視晉恭王制,加高一尺。因定享堂七間,廣十丈九尺五寸,高二丈九尺,深四丈三尺五寸。弘治十三年,寧王宸濠奏廟祀禮樂未有定式,乞頒賜遵守。禮部議:「洪武元年,學士宋濂等奏定諸王國祭祀禮樂,用清字,但有曲名,而無曲辭,請各王府稽考。於是靖江王長史具上樂章,且言四孟上旬及除夕五祭所用品物、俎豆、佾舞,禮節悉遵國初定制。」從之。嘉靖八年,秦王充燿言:「代懿王當祔廟,而自始封至今,已盈五廟之數,請定祧廟之制。」禮臣言:「親王祧廟,古制未聞,宜推太廟祧祔之禮而降殺之。始封居中,百世不遷,以下四世,親盡而祧。但諸侯無祧廟,祧主宜祔始祖之室,置櫝藏之,每歲暮則出祧主合祭。」詔如議。

○群臣家廟

明初未有定制,權仿朱子祠堂之制,奉高曾祖禰四世神主,以四仲之月祭之,加臘月忌日之祭與歲時俗節之薦。其庶人得奉祖父母、父母之祀,已著為令。至時享於寢之禮,略同品官祠堂之制。堂三間,兩階三級,中外為兩門。堂設四龕,龕置一桌。高祖居西,以次而東,藏主櫝中。兩壁立櫃,西藏遺書衣物,東藏祭器。旁親無後者,以其班附。庶人無祠堂,以二代神主置居室中間,無櫝。

洪武六年,定公侯以下家廟禮儀。凡公侯品官,別為祠屋三間於所居之東,以祀高曾祖考,并祔位。祠堂未備,奉主於中堂享祭。二品以上,羊一豕一,五品以上,羊一,以下豕一,皆分四體熟薦。不能具牲者,設饌以享。所用器皿,隨官品第,稱家有無。前二日,主祭者聞於上,免朝參。凡祭,擇四仲吉日,或春、秋分,冬、夏至。前期一日,齋沐更衣,宿外舍。質明,主祭者及婦率預祭者詣祠堂。主祭者捧正祔神主櫝,置於盤,令子弟捧至祭所。主祭開櫝,捧各祖妣神主,以序奉安。子弟捧祔主,置東西壁。執事者進饌,讀祝者一人,就贊禮,以子弟親族為之。陳設神位訖,各就位,主祭在東,伯叔諸兄立於其前稍東,諸親立於其後,主婦在西,母及諸母立於其前稍西,婦女立於後。贊拜,皆再拜。主祭者詣香案前跪,三上香,獻酒奠酒,執事酌酒於祔位前。讀祝者跪讀訖,贊拜,主祭者復位,與主婦皆再拜。再獻終獻並如之,惟不讀祝。每獻,執事者亦獻於祔位。禮畢,再拜,焚祝并紙錢於中庭,安神主於櫝。

成化十一年,祭酒周洪謨言:「臣庶祠堂神主,俱自西而東。古無神道尚右之說,惟我太祖廟制,合先王左昭右穆之義。宜令一品至九品,皆立一廟,以高卑廣狹為殺。神主則高祖居左,曾祖居右,祖居次左,考居次右。」帝下禮臣參酌更定。嘉靖十五年,禮部尚書夏言言:「按三代有五廟、三廟、二廟、一廟之制者,以其有諸侯、卿、大夫上中下之爵也。後世官職既殊,無世封採邑,豈宜過泥於古。至宋儒程頤乃始約之而歸於四世,自公卿以及士庶,莫不皆然。謂五服之制,皆至高祖,則祭亦當如之。今定官自三品以上立五廟,以下皆四廟。為五廟者,亦如唐制。五間九架,廈旁隔板為五室,中祔五世祖,旁四室,祔高曾祖禰。為四廟者,三間五架,中一室祔高曾,左右二室祔祖禰。若當祀始祖,則如硃熹所云,臨祭時,作紙牌,祭訖焚之。其三品以上者,至世數窮盡,則以今之得立廟者為世世奉祀之祖,而不遷焉。四品以下,四世遞遷而已。」從之。

登極儀大朝儀常朝儀皇太子親王朝儀諸王來朝儀諸司朝覲儀中宮受朝儀朝賀東宮儀大宴儀上尊號徽號儀

二曰嘉禮。行於朝廷者,曰朝會,曰宴饗,曰上尊號、徽號,曰冊命,曰經筵,曰表箋。行於辟雍者,曰視學。自天子達於庶人者,曰冠,曰婚。行於天下者,曰巡狩,曰詔赦,曰鄉飲酒。舉其大者書之。儀之同者,則各附於其類云。

○登極儀

漢高帝即位氾水之陽,其時綿蕞之禮未備。魏、晉以降,多以受禪改號。元世祖履尊既久,一統後,但舉朝賀。明興,太祖以吳元年十二月將即位,命左相國李善長等具儀。善長率禮官奏。

即位日,先告祀天地。禮成,即帝位於南郊。丞相率百官以下及都民耆老,拜賀舞蹈,呼萬歲者三。具鹵簿導從,詣太廟,上追尊四世冊寶,告祀社稷。還,具袞冕,御奉天殿,百官上表賀。

先期,侍儀司設表案於丹墀內道之西北,設丞相以下拜位於內道東西,每等異位,重行北面。捧表、展表、宣表官位於表案西,東向。糾儀御史二人於表案南,東西向。宿衛鎮撫二人於東西陛下,護衛百戶二十四人於其南,稍後。知班二人,於文武官拜位北,東西向。通贊、贊禮二人於知班北,通贊西,贊禮東。引文武班四人於文武官拜位北,稍後,東西向。引殿前班二人於引文武班南。舉表案二人於引文武班北。舉殿上表案二人於西陛下,東向。丹陛上設殿前班指揮司官三人,東向。宣徽院官三人,西向。儀鸞司官於殿中門之左右,護衛千戶八人於殿東西門,俱東西向。鳴鞭四人於殿前班之南,北向。將軍六人於殿門左右,天武將軍四人於陛上四隅,俱東西向。殿上,尚寶司設寶案於正中,侍儀司設表案於寶案南。文武侍從兩班於殿上東西,文起居注、給事中、殿中侍御史、尚寶卿,武懸刀指揮,東西向。受表官於文侍從班南,西向。內贊二人於受表官之南,捲簾將軍二人簾前,俱東西向。

是日,拱衛司陳鹵簿,列甲士於午門外,列旗仗,設五輅於奉天門外,侍儀舍人二,舉表案入。鼓初嚴,百官朝服立午門外。通贊、贊禮、宿衛官、諸侍衛及尚寶卿侍從官入。鼓三嚴,丞相以下入。皇帝袞冕陞御座,大樂鼓吹振作。樂止,將軍捲簾,尚寶卿置寶於案。拱衛司鳴鞭,引班導百官入丹墀拜位。初行樂作,至位樂止。知班贊班,贊禮贊拜。樂作,四拜,興。樂止。捧表以下官由殿西門入。內贊贊進表。捧表官跪捧。受表官搢笏,跪受,置於案。出笏,興,退立,東向。內贊贊宣表。宣表官前,搢笏,跪,展表官搢笏,同跪。宣訖,展表官出笏,以表復於案,俱退。宣表官俯伏興。俱出殿西門,復位。贊禮贊拜。樂作,四拜,樂止。搢笏,鞠躬三,舞蹈。拱手加額,呼萬歲者三。出笏,俯伏興。樂作,四拜,賀畢。遂遣官冊拜皇后,冊立皇太子,以即位詔告天下。

成祖即位倉猝,其議不詳。仁宗即位,先期,司設監陳御座於奉天門,欽天監設定時鼓,尚寶司設寶案,教坊司設中和韶樂,設而不作。是日早,遣官告天地宗社,皇帝具孝服告几筵。至時,鳴鐘鼓,設鹵簿。皇帝袞冕,御奉天門。百官朝服,入午門。鴻臚寺導執事官行禮,請升御座。皇帝由中門出。升座,鳴鞭。百官上表,行禮,頒詔,俱如儀。宣宗以後,儲宮嗣立者並同。正德十六年,世宗入承大統。先期造行殿於宣武門外,南向。設帷幄御座,備翼善冠服及鹵簿大駕以候。至期,百官郊迎。駕入行殿,行四拜禮。明日,由大明門入。省詔草,改年號,素服詣大行几筵謁告。畢,設香案奉天殿丹陛上。皇帝袞冕,行告天地禮。詣奉先殿、奉慈殿謁告,仍詣大行几筵、慈壽皇太后、莊肅皇后前各行禮,遂御華蓋殿。百官朝服入。傳旨免賀,五拜三稽首。鴻臚寺官請升殿,帝由中門出御奉天殿。鳴鞭,贊拜,頒詔,如制。

○大朝儀

漢正會禮,夜漏未盡七刻,鐘鳴受賀。公卿以下執贄來庭,二千石以上升殿,稱萬歲,然後宴饗。晉《咸寧注》,有晨賀晝會之分。唐制,正旦、冬至、五月朔、千秋節,咸受朝賀。宋因之。

明太祖洪武元年九月,定正旦朝會儀,與登極略相仿。其後屢詔更定,立為中制。凡正旦冬至,先日,尚寶司設御座於奉天殿,及寶案於御座東,香案於丹陛南。教坊司設中和韶樂於殿內東西,北向。翌明,錦衣衛陳鹵簿、儀仗於丹陛及丹墀,設明扇於殿內,列車輅於丹墀。鳴鞭四人,左右北向。教坊司陳大樂於丹陛東西,北向,儀禮司設同文、玉帛兩案於丹陛東。金吾衛設護衛官於殿內及丹陛,陳甲士於丹墀至午門外,錦衣衛設將軍於丹陛至奉天門外,陳旗幟於奉天門外,俱東西列。典牧所陳仗馬犀象於文、武樓南,東西向。司晨郎報時位於內道東,近北。糾儀御史二,位於丹墀北,內贊二,位於殿內,外贊二,位於丹墀北,傳制、宣表等官位於殿內,俱東西向。鼓初嚴,百官朝服,班午門外。次嚴,由左、右掖門入,詣丹墀東西,北向立。三嚴,執事官詣華蓋殿,帝具袞冕陞座,鐘聲止。儀禮司奏執事官行禮,贊五拜,畢,奏請升殿。駕興,中和樂作。尚寶司捧寶前行,導駕官前導,扇開簾捲,寶置於案,樂止。鳴鞭報時,對贊唱排班,班齊。贊禮唱鞠躬,大樂作。贊四拜,興,樂上。典儀唱進表,樂作。給事中二人,詣同文案前,導引序班舉案由東門入,置殿中,樂止。內贊唱宣表目。宣表目官跪,宣訖,俯伏,興。唱宣表,展表官取表,宣表官至簾前,外贊唱,眾官皆跪。宣表訖,內外皆唱,俯伏,興。序班舉表案於殿東,外贊唱眾官皆跪。代致詞官跪丹陛中,致詞云:「具官臣某,茲遇正旦,三陽開泰,萬物咸新。」冬至則云:「律應黃種,日當長至。」「恭惟皇帝陛下,膺乾納祜,奉天永昌。」賀畢,外贊唱,眾官皆俯伏,興。樂作,四拜,興。樂止。傳制官跪奏傳制,由東門出,至丹陛,東向立,稱有制。贊禮唱,跪,宣制。正旦則云:「履端之慶,與卿等同之。」冬至則云:「履長之慶,與卿等同之。」萬壽聖節則致詞曰:「具官臣某,欽遇皇帝陛下聖誕之辰,謹率文武官僚敬祝萬歲壽。」不傳制。贊禮唱俯伏,興。樂止。贊搢笏,鞠躬三,舞蹈。贊跪唱山呼,百官拱手加額曰「萬歲」;唱山呼,曰「萬歲」;唱再山呼,曰「萬萬歲」。凡呼萬歲,樂工軍校齊聲應之。贊出笏,俯伏,興,樂作。贊四拜,興,樂止。儀禮司奏禮畢,中和樂作。鳴鞭,駕興。尚寶官捧寶,導駕官前導,至華蓋殿,樂止。百官以次出。

洪武三十年,更定同文、玉帛案俱進安殿中,宣表訖,舉置於寶案之南。嘉靖十六年,更定蕃國貢方物案入於丹陛中道左右,設定時鼓於文樓上,大樂陳奉天門內東西,北向。他儀亦略有增損。

立春日進春,都城府縣舉春案由東階升,跪置於丹陛中道,俯伏,興。贊拜,樂作。四拜,興,樂止。文武官北向立,致詞官詣中道之東,跪奏云:「新春吉辰,禮當慶賀。」贊拜,樂作。五拜三叩頭,興,樂止。儀禮司奏禮畢。正統十一年,正旦立春,禮部議順天府官進春後,百官即詣班行賀正旦禮。舊制,冬至日行賀禮。嘉靖九年,分祀二郊,以冬至大報,是日行慶成禮。次日,帝詣內殿,行節祭禮。又詣母后前行賀禮訖,始御奉天殿受賀。

○常朝儀

古禮,天子有外朝、內朝、燕朝。漢宣帝五日一朝。唐制,天子日御紫宸殿見群臣曰常參,朔望御宣政殿見群臣曰入閣。宋則侍從官日朝垂拱謂之常參,百司五日一朝紫宸為六參,在京朝官朔望朝紫宸為朔參、望參。

明洪武三年定制,朔望日,帝皮弁服御奉天殿,百官朝服於丹墀東西,再拜。班首詣前,同百官鞠躬,稱「聖躬萬福」。復位,皆再拜,分班對立。省府臺部官有奏,由西階升殿。奏畢降階,百官出。十七年,罷朔望起居禮。後更定,朔望御奉天殿,常朝官序立丹墀,東西向,謝恩見辭官序立奉天門外,北向。升座作樂。常朝官一拜三叩頭,樂止,復班。謝恩見辭官序立奉天門外,北向。升座作樂。常朝官一拜三叩頭,樂止,復班。謝恩見辭官於奉天門外,五拜三叩頭畢,駕興。

又凡早朝,御華蓋殿,文武官於鹿頂外東西立,鳴鞭,以次行禮訖。四品以上官入侍殿內,五品以下仍前北向立。有事奏者出班,奏畢,鳴鞭以次出。如御奉天殿,先於華蓋殿行禮。奏事畢,五品以下詣丹墀,北向立,五品以上及翰林院、給事中、御史於中左、中右門候鳴鞭,詣殿內序立,朝退出。凡百官於御前侍坐,有官奏事,必起立,奏畢復坐。後皇帝行丹墀,常北面,不南向,左右周旋不背北。皇帝升奉天門及丹陛,隨從官不得徑由中道並王道。二十四年,定侍班官:東則六部都察院堂上官、十三道掌印御史、通政司、大理寺、太常寺、太僕寺、應天府、翰林院、春坊、光祿寺、欽天監、尚寶司、太醫院、五軍斷事及京縣官,西則五軍都督、錦衣衛指揮、各衛掌印指揮、給事中、中書舍人。又令禮部置百官朝牌,大書品級,列丹墀左右木柵上,依序立。二十六年,令凡入殿必履鞋。

永樂初,令內閣官侍朝立金臺東,錦衣衛在西,後移御道,東西對立。四年諭六部及近侍官曰:「早朝多四方所奏事。午後事簡,君臣之間得從容陳論。自今有事當商榷者,皆於晚朝。」四年,諭行在禮部曰:「北京冬氣嚴凝,群臣早朝奏事,立久不勝。今後朝畢,於右順門內便殿奏事。」

景泰初,定午朝儀。凡午朝,御左順門,設寶案。執事奏事官候於左掖門外。駕出,以次入。內閣、五府、六部奏事官,六科侍班官,案西序立;侍班御史二,序班二,將軍四,案南面北立;鳴贊一,案東,西向立;錦衣衛、鴻臚寺東向立;管將軍官、侍衛官立於將軍西。府部奏事畢,撤案,各官退。有密事,赴御前奏。

嘉靖九年,令常朝官禮畢,內閣官由東陛、錦衣衛官由西陛升,立於寶座東西。有欽差官及外國人領敕,坊局官一人奉敕立內閣後,稍上,候領敕官辭,奉敕官承旨由左陛下,循御道授之。隆慶六年,詔以三六九日視朝。萬曆三年,令常朝日記注起居官四人,列於東班給事中上,稍前,以便觀聽。午朝,則列於御座西,稍南。

凡入朝次第,洪武二十四年,令朝參將軍先入,近侍次之,公、侯、駙馬、伯又次之,五府、六部又次之,應天府及在京雜職官員又次之。成化十四年令進士照辦事衙門次第,立見任官後。

○皇太子親王朝儀

前史多不載。明洪武元年十月定制,凡正旦等大朝,皇帝御奉天殿,先設皇太子、親王次於文樓,設拜位並拜褥於丹陛上正中。皇帝升座,殿前執事班起居訖。引進引皇太子及親王由奉天東門入,百官齊入。樂作,皇太子、親王升自東階,至丹陛拜位,樂止。贊四拜,樂作,興,樂止。引進導由殿東門入,樂作。內贊引至御座前位,樂止。唱跪,皇太子跪稱賀云:「長子某,茲遇履端之節」,冬至則云「履長」,「謹率諸弟某等,欽詣父皇陛下稱賀」。傳制如前,贊俯伏,興。皇太子、諸王由東門出,樂作。引進引復丹陛位,樂止。贊四拜,樂作,興,樂止。降自東階,樂作。至文樓,樂止。百官隨入賀。其朝皇后則於坤寧宮,略如朝皇帝儀。

二十六年,改定朝賀於乾清宮。其日,皇帝、皇后陞座,侍從導引如儀,引禮引皇太子及妃、親王及妃詣上位前,贊禮贊四拜,興。贊禮引皇太子詣前,贊跪,引禮贊太了妃、諸王及妃皆跪。皇太子致詞,同前,不傳制。贊禮贊皇太子俯伏,興,引禮贊諸王俯伏,興,太子妃、諸王妃皆興。贊禮引皇太子復位。贊拜,皇太子以下皆四拜。禮畢,引禮引至皇后前,其前後贊拜,皆如朝皇帝儀。致詞稱「母后殿下」。禮畢,出。七年更定,不致賀辭,止行八拜禮。朝賀皇太后禮皆同。

○諸王來朝儀

古者,六年五服一朝。漢法有四見儀。魏制,籓王不得入覲。晉泰始中,令王公以下入朝者,四方各為二番。唐以後,親籓多不就國。明代仿古封建,親王之籓不常入朝,朝則賜賚甚厚。

明初,凡來朝,先期陳御座於奉天殿,如常儀。諸王次於奉天門外東耳房。鼓三嚴,百官入就侍立位。引禮導王具袞冕,由東門入,升東陛,就位。王府從官就丹墀位。贊拜,樂作,王與從官皆四拜。興,樂止。王從殿東門入,樂作。內贊導至御前,樂止。王跪,王府官皆跪。王致辭曰:「第幾子某王某,茲遇某時入覲,欽詣父皇陛下朝拜。」贊俯伏,興。王由東門出。樂作,復拜位,樂止。贊拜,王興。從官皆四拜,興。樂作,駕興,王及各官以次出。

洪武二十六年定,凡諸王大朝,行八拜禮。常朝,一拜。凡伯叔兄見天子,在朝行君臣禮,便殿行家人禮。伯叔兄西向坐,受天子四拜。天子居中南面坐,以尚親親之義,存君臣之禮。凡外戚朝見,皇后父母見帝行君臣禮,后見父母行家人禮。皇太子見皇后父母,皇后父母西向立,皇太子東向行四拜禮,皇后父母立受兩拜,答兩拜。

○諸司朝覲儀

明制,天下官三年一入朝。自十二月十六日始,鴻臚寺以次引見。二十五日後,每日方面官隨常朝官入奉天門行禮,府州縣官及諸司首領官吏、土官吏俱午門外行禮。正旦大朝以後,方面官於奉天殿前序立,知府以下,奉天門金水橋南序立,如常朝儀。天順三年,令凡方面官入朝,遞降京官一等。萬曆五年,令凡朝覲,南京府尹、行太僕寺苑馬寺卿、布按二司,俱於十二月十六日朝見,外班行禮。由右掖門至御前,鴻臚寺官以次引見。其鹽運司及知府以下官吏,浙江、江西十七日,山東、山西十八日,河南、陜西十九日,湖廣、南直隸二十日,福建、四川二十一日,廣東、廣西二十二日,雲南、貴州二十三日,北直隸二十四日,各外班行禮,至御前引見。免朝則止,仍候御朝日引見。正旦朝賀,俱入殿前行禮。凡朝覲官見辭謝恩,具公服,正旦具朝服,不著朱履。常朝俱錦繡。

○中宮受朝儀

惟唐《開元禮》有朝皇太后及皇后受群臣賀,皇后會外命婦諸儀。明制無皇后受群臣賀儀,而皇妃以下,正旦、冬至朝賀儀,則自洪武元年九月詔定。

凡中宮朝賀,內使監設皇后寶座於坤寧宮。丹陛儀仗,內使執之,殿上儀仗,女使執之。陳女樂於宮門外。設皇貴妃幄次於宮門外之西,近北;設公主幄次於宮門外之東,稍南;設外命婦幄次於門外之南,東西向。皇后服禕衣出閤,仗動,樂作。陞座,樂止。司賓導外命婦由東門入內道,東西班侍立,訖。導皇貴妃、眾妃由東門入,至陛上拜位。贊拜,樂作,四拜興,樂止。導由殿東門入,樂作。內贊接引至殿上拜位,樂止。贊跪,妃皆跪。皇貴妃致祠曰,「妾某氏等,遇茲履端之節」,冬至則云「履長」,「恭詣皇后殿下稱賀」。致詞畢,皆俯伏,興,樂作,復位,樂止。贊拜,樂作,四拜興,樂止。降自東階出。司賓導公主由東門入,至陛下拜位,以次立,行禮如皇妃儀。司賓導外命婦入殿前中道拜位。贊拜如儀。班首由西陛陞,入殿西門,樂作。內贊接引至殿上拜位,班首及諸命婦皆跪。班首致詞曰:「某國夫人妾某氏等稱賀。」賀畢,出復位。司言跪承旨,由殿中門出,立露臺之東,南向,稱有旨。命婦皆跪,司言宣旨曰:「履端之慶,與夫人等共之。」贊興。司言奏宣旨畢。皇后興,樂作。入內閤門,樂止。諸命婦出。太皇太后、皇太后朝賀儀同。

洪武二十六年,重定中宮朝賀儀:先日,女官設御座香案。至日內官設儀仗、陳女樂於丹陛東西,北向,設箋案於殿東門。命婦至宮門,司賓引入就拜位,女官具服侍班。尚宮、尚儀等官詣內奉迎,皇后具服出,作樂,贊拜如前儀。女官舉箋案由殿東門入,樂作。至殿中,樂止。贊跪,命婦皆跪。贊宣箋目,女官宣訖,贊展箋,宣箋女官詣案前,展宣訖,舉案於殿東。命婦皆興,司賓引班首由東階升入殿東門,樂作。內贊引至殿中,樂止。贊跪,班首及諸命婦皆跪。班首致詞訖,皆興,由西門出。贊拜及司言宣旨,皆如儀,禮畢。千秋節致詞云:「茲遇千秋令節,敬詣皇后殿下稱賀。」不傳旨。凡朔望命婦朝參,是日設御座於宮中,陳儀仗女樂。皇后陞座,引禮女官引命婦入班,文東武西,各以夫品。贊拜,樂作,四拜。禮畢,出。陰雨、大寒暑則免。後命婦朝賀,俱於仁智殿。朝東宮妃,儀如朝中宮,不傳令。

○朝賀東宮儀

漢以前無聞。隋文帝時,冬至百官朝太子,張樂受賀。唐制,宮臣參賀皇太子,皆舞蹈。開元始罷其禮。故事,百官詣皇太子止稱名,惟宮臣稱臣。明洪武十四年,給事中鄭相同請如古制,詔下群臣議。編修吳沈等議曰:「東宮國之大本,所以繼聖體而承天位也。臣子尊敬之禮,不得有二。請凡啟事東宮者,稱臣如故。」從之。

凡朝東宮,前期,典璽官設皇太子座於文華殿,錦衣衛設儀仗於殿外,教坊司陳大樂於文華門內東西,北向,府軍衛列甲士旗幟於門外,錦衣衛設將軍十二人於殿中門外及文華門外,東西向,儀禮司官設箋案於殿東門外,設百官拜位於殿下東西,設傳令宣箋等官位於殿內東西。是日,百官詣文華門外。導引官啟外備,皇太子具冕服出,樂作。升座,樂止。百官入贊拜,樂作。四拜興,樂止。丞相升自西階,至殿內拜位,俱跪。丞相致詞曰:「某等茲遇三陽開泰,萬物維新。敬惟皇太子殿下,茂膺景福。」畢,俯伏,興,復位。舍人舉箋案入殿中,其捧箋、展箋、宣箋、傳令,略與皇后同。令曰:「履茲三陽,願同嘉慶。」餘俱如儀。冬至致詞,則易「律應黃鐘,日當長至」。傳令則易「履長之節。」千秋節致詞則云「茲遇皇太子殿下壽誕之辰,謹率文武群官,敬祝千歲壽。」不傳令。凡朔望,百官朝退,詣文華殿門外,東西立。皇太子升殿,樂作。百官行一拜禮。其謝恩見辭官亦行禮。

洪武元年十二月,帝以東宮師傅皆勳舊大臣,當待以殊禮,命議三師朝賀東宮儀。禮官議曰:「唐制,群臣朝賀東宮,行四拜禮,皇太子答後二拜。三公朝賀,前後俱答拜。近代答拜之禮不行,而三師之禮不可不重。今擬凡大朝賀,設皇太子座於大本堂,設答拜褥位於堂中,設三師、賓客、諭德拜位於堂前。皇太子常服升座,三師、賓客常服入就位,北向立。皇太子起立,南向。贊四拜,皇太子答後二拜。」

六年,詔百官朝見太子,朝服去蔽膝及佩。二十九年,詔廷臣議親王見東宮儀。禮官議,諸王來見,設皇太子位於正殿中,設諸王拜位於殿門外及殿內,設王府官拜位於庭中道上之東西,設百官侍立位於庭中,東西向。至日,列甲士,陳儀仗,設樂如常。諸王詣東宮門外幄次,皇太子常服出,樂作。陞座,樂止。引禮導諸王入就殿門外位。初行,樂作,就位,樂止。導詣殿東門入,樂作。內贊引至位,北向立,樂止。贊跪,王與王府官皆跪,致詞曰:「茲遇某節,恭詣皇太子殿下。」致詞畢,王與王府官皆俯伏,興,樂作。復位,樂止。贊拜,樂作,王與王府官皆四拜。興,樂止。禮畢,王及各官以次出。王至後殿,敘家人禮。東宮及王皆常服,王由文華殿東門入,至後殿。王西向,東宮南向。相見禮畢,敘坐,東宮正中,南面,諸王列於東西。

嘉靖二十八年,禮部奏,故事,皇太子受朝賀,設座文華殿中,今易黃瓦,似應避尊。帝曰:「東宮受賀,位當設文華門之左,南向。然侍衛未備,已之。」隆慶二年冊皇太子,詔於文華殿門東間設座受賀。

○大宴儀

漢大朝會,群臣上殿稱萬歲,舉觴。百官受賜宴饗,大作樂。唐大饗登歌,或於殿庭設九部伎。宋以春秋仲月及千秋節,大宴群臣,設山樓排場,窮極奢麗。明制,有大宴、中宴、常宴、小宴。

洪武元年,大宴群臣於奉天殿,三品以上升殿,餘列於丹墀,遂定正旦、冬至聖節宴謹身殿禮。二十六年,重定大宴禮,陳於奉天殿。永樂元年,以郊祀禮成,大宴。十九年,以北京郊社、宗廟及宮殿成,大宴。宣德、正統間,朝官不與者,給賜節錢。凡立春、元宵、四月八日、端午、重陽、臘八日,永樂間,俱於奉天門賜百官宴,用樂。其後皆宴於午門外,不用樂。立春日賜春餅,元宵日團子,四月八日不落莢,嘉靖中,改不落莢為麥餅。端午日涼糕粽,重陽日糕,臘八日面,俱設午門外,以官品序坐。宣德五年冬,久未雪,十二月大雪,帝示群臣《喜雪》詩,復賜賞雪宴。群臣進和章,帝擇其寓警戒者錄之,而為之序。皇太后聖誕,正統四年賜宴午門。東宮千秋節,永樂間,賜府部堂上、春坊、科道、近侍錦衣衛及天下進箋官,宴於文華殿。宣德以後,俱宴午門外。凡祀圜丘、方澤、祈穀、朝日夕月、耕耤、經筵日講、東宮講讀,皆賜飯。親蠶,賜內外命婦飯。纂修校勘書籍,開館暨書成,皆賜宴。閣臣九年考滿,賜宴於禮部,九卿侍宴。新進士賜宴曰恩榮。

凡大饗,尚寶司設御座於奉天殿,錦衣衛設黃麾於殿外之東西,金吾等衛設護衛官二十四人於殿東西。教坊司設九奏樂歌於殿內,設大樂於殿外,立三舞雜隊於殿下。光祿寺設酒亭於御座下西,膳亭於御座下東,珍羞醯醢亭於酒膳亭之東西。設御筵於御座東西,設皇太子座於御座東,西向,諸王以次南,東西相向。群臣四品以上位於殿內,五品以下位於東西廡,司壺、尚酒、尚食各供事。至期,儀禮司請升座。駕興,大樂作。升座,樂止。鳴鞭,皇太子親王上殿。文武官四品以上由東西門入,立殿中,五品以下立丹墀,贊拜如儀。光祿寺進御筵,大樂作。至御前,樂止。內官進花。光祿寺開爵注酒,詣御前,進第一爵。教坊司奏《炎精之曲》。樂作,內外官皆跪,教坊司跪奏進酒。飲畢,樂止。眾官俯伏,興,贊拜如儀。各就位坐,序班詣群臣散花。第二爵奏《皇風之曲》。樂作,光祿寺酌酒御前,序班酌群臣酒。皇帝舉酒,群臣亦舉酒,樂止。進湯,鼓吹響節前導,至殿外,鼓吹止。殿上樂作,群臣起立,光祿寺官進湯,群臣復坐。序班供群臣湯。皇帝舉箸,群臣亦舉箸,贊饌成,樂止。武舞入,奏《平定天下之舞》。第三爵奏《眷皇明之曲》。樂作,進酒如初。樂止,奏《撫安四夷之舞》。第四爵奏《天道傳之曲》,進酒、進湯如初,奏《車書會同之舞》。第五爵奏《振皇綱之曲》,進酒如初,奏《百戲承應舞》。第六爵奏《金陵之曲》,進酒、進湯如初,奏《八蠻獻寶舞》。第七爵奏《長楊之曲》,進酒如初,奏《採蓮隊子舞》。第八爵奏《芳醴之曲》,進酒、進湯如初,奏《魚躍於淵舞》。第九爵奏《駕六龍之曲》,進酒如初。光祿寺收御爵,序班收群臣盞。進湯,進大膳,大樂作,群臣起立,進訖復坐,序班供群臣飯食。訖,贊膳成,樂止。撤膳,奏《百花隊舞》。贊撤案,光祿寺撤御案,序班撤群臣案。贊宴成,群臣皆出席,北向立。贊拜如儀,群臣分東西立。儀禮司奏禮畢,駕興,樂止,以次出。其中宴禮如前,但進七爵。常宴如中宴,但一拜三叩頭,進酒或三或五而止。

凡宴命婦,坤寧宮設儀仗、女樂。皇后常服升座,皇妃、皇太子妃、王妃、公主亦常服隨出閤,入就位,大小命婦各立於座位後。丞相夫人率諸命婦舉御食案。丞相夫人捧壽花,二品外命婦各舉食案於皇妃、皇太子妃、王妃、公主前。大小命婦各就座位,奉御執事人分進壽花於殿內及東西廡。酒七行,上食五次,酌酒、進湯、樂作止,並如儀。

○上尊號徽號儀

子無爵父之道。漢高帝感家令之言而尊太公,荀悅非之。晉哀帝欲尊崇皇太妃,江AN以為宜告顯宗之廟,明事不在己。宋、元志俱載皇太后上尊號儀,而不行告廟,非禮也。明制,天子登極,奉母后或母妃為皇太后,則上尊號。其後或以慶典推崇皇太后,則加二字或四字為徽號。世宗時,上兩宮皇太后,增至八字。上徽號致詞,而上尊號則止進寶冊。

上尊號,自宣宗登極尊皇太后始。先期遣官祭告天地宗社,帝親告太宗皇帝、大行皇帝几筵。是日,鳴鐘鼓,百官朝服。奉天門設冊寶彩輿香亭。中和韶樂及大樂,設而不作。內官設皇太后寶座,陳儀仗於宮中。設冊寶案於寶座前,設皇帝拜位於丹陛正中,親王拜位於丹墀內。女樂設而不作。皇帝冕服御奉天門。奉冊寶官以冊寶置輿中,內侍舉輿,皇帝隨輿降階升輅。百官於金水橋南,北向立,輿至皆跪,過興。隨至思善門外橋南,北向立。皇帝至思善門內降輅。皇太后升座。輿至丹陛。皇帝由左門入,至陛右,北向立。親王冕服各就位。奏四拜,皇帝及王以下皆四拜。奉冊寶官以冊寶由殿中門入,立於左。皇帝由殿左門入,至拜位跪,親王百官皆跪。奏搢圭,奏進冊。奉冊官以冊跪進,皇帝受冊獻訖,執事官跪受,置案左。奏進寶,奉寶官以寶跪進。皇帝受寶,獻訖,執事官跪受,置案右。奏出圭,奏宣冊,執事官跪宣讀。皇帝俯伏,興,由左門出,至拜位。奏四拜,傳唱百官同四拜。禮畢,駕興。是日,皇帝奉皇太后謁奉先殿及几筵,行謁謝禮。禮畢,皇太后還宮,服燕居冠服,升座。皇帝率皇后、皇妃、親王、公主及六尚等女官行慶賀禮。翌日,外命婦四品以上行進表箋禮。宣德以後,儀同。正統初,尊太皇太后儀同。天順八年二月,增命婦致詞云:「某夫人妾某氏等,恭惟皇太后陛下尊居慈極,永膺福壽。」弘治十八年,上兩宮尊號,改皇太后致詞云:「尊居慈闈,茂隆福壽。」

嘉靖元年二月上尊號,以四宮行禮過勞,分為二日。又以武宗服制未滿,莊肅皇后免朝賀,命婦賀三宮,亦分日。

上徽號,自天順二年正月奉皇太后始。致詞云:「嗣皇帝臣,伏惟皇太后陛下,功德兼隆,顯崇徽號,永膺福壽,率土同歡。」命婦進表慶賀致詞云:「某夫人妾某氏等,恭惟皇太后陛下德同坤厚,允協徽稱,壽福無疆,輿情歡戴。」餘如常儀。後上徽號及加上徽號,仿此。成化二十三年,禮部具儀上,未及皇太子妃禮,特命增之。

冊皇后儀冊妃嬪儀附冊皇太子及皇太子妃儀冊親王及王妃儀冊公主儀附皇帝加元服儀冊皇太子皇子冠禮品官冠禮庶人冠禮

○冊皇后儀

古者立後無冊命禮。至漢靈帝立宋美人為皇后,始御殿,命太尉持節,奉璽綬,讀冊。皇后北面稱臣妾,跪受。其後沿為定制,而儀文代各不同。明儀注大抵參唐、宋之制而用之,太祖初,定制。

凡冊皇后,前期三日齋戒,遣官祭告天地、宗廟。前一日,侍儀司設冊寶案於奉天殿御座前,設奉節官位於冊案之東,掌節者位於其左,稍退,設承制官位於其南,俱西向。設正副使受制位於橫街於南,北向。設承制宣制官位於其北,設奉節奉冊奉寶官位於其東北,俱西向。設正副使受冊寶褥位於受制位之北,北向。典儀二人位丹陛上南,贊禮二人位正副使北,知班二人位贊禮之南,俱東西相向。百官及侍從位,如朝儀。

是日早,列鹵簿,陳甲士,設樂如儀。內官設皇后受冊位及冊節寶案於宮中,設香案於殿上,設權置冊寶案於香案前,設女樂於丹陛。質明,正副使及百官入。鼓三嚴,皇帝袞冕御奉天殿。禮部官奉冊寶,各置於案。諸執事官各人就殿上位立。樂作,四拜,興,樂止。承制官奏發皇后冊寶,承制訖,由中門出,降自中陛,至宣制位,曰「有制」。正副使跪,承制官宣制曰:「冊妃某氏為皇后,命卿等持節展禮。」宣畢,由殿西門入。正副使俯伏,興。執事者舉冊寶案,由中門出,降自中陛。奉節官率掌節者前導,至正副使褥位,以案置於北。掌節者脫節衣,以節授奉節官。奉節官以授正使,正使以授掌節者,掌節者跪受,興,立於正使之左。奉節官退。引禮引正使詣受冊位,奉冊官以冊授正使,正使跪受,置於案。退,復位。副使受寶亦如之。樂作,正副使四拜。興,樂止。正使隨冊,副使隨寶,掌節者前導,舉案者次之,樂作。出奉天門,樂止。侍儀奏禮畢,駕興,百官出。掌節者加節衣,奉冊寶官皆搢笏,取冊寶置龍亭內,儀仗大樂前導,至中宮門外,樂作。皇后具九龍四鳳冠,服禕衣,出閤,至殿上,南向立。樂止,正副使奉冊寶權置於門外所設案上。引禮引正副使及內使監令俱就位。正使詣內使監令前,稱冊禮使臣某,副使臣某,奉制授皇后冊寶。內使監令入告皇后,出,復位。引禮引內外命婦入就位。正使奉冊授內使監令,內使監令跪受,以授內官。副使授寶亦如之。各復位。內使監令率奉冊奉寶內官入,各置於案。尚儀引皇后降陛,詣庭中位立。內官奉冊寶立於皇后之東西。內使監令稱「有制」,尚儀奏拜。皇后拜,樂作。四拜,興,樂止。宣制訖,奉冊內官以冊授讀冊內官讀訖,以授內使監令。內使監令跪以授皇后,皇后跪受,以授司言。奉寶如前儀。受訖,以授司寶。尚儀奏拜,皇后拜如前。內使監令出,詣正副使前,稱「皇后受冊禮畢」。使者退詣奉天殿橫街南,北面西上立,給事中立於正副使東北,西向。正副使再拜復命曰:「奉制冊命皇后禮畢。」又再拜,給事中奏聞,乃退。皇后既受冊寶,升座。引禮引內命婦班首一人,詣殿中賀位跪,致詞曰:「茲遇皇后殿下膺受冊寶,正位中宮,妾等不勝懽慶,謹奉賀。」贊拜,樂作。再拜,興,樂止。退,復位。又引外命婦班首一人,入就殿上賀位,如內命婦儀。禮畢俱出。皇后降座,樂作。還閤,樂止。

次日,百官上表箋稱賀。皇帝御殿受賀,如常儀。遂卜日,行謁廟禮,先遣官用牲牢行事,告以皇后將祗見之意。前期,皇后齋三日,內外命婦及執事內官齋一日。設皇后拜位於廟門外及廟中,設內命婦陪祀位於廟庭南,外命婦陪祀位於內命婦之南。司贊位皇后拜位之東西,司賓位內命婦之北,司香位香案右。陳盥洗於階東,司盥洗官位其所。至日,內外命婦各翟衣集中宮內門外。皇后具九龍四鳳冠,服禕衣。出內宮門,升輿,至外門外降輿,升重翟車。鼓吹設而不作。尚儀陳儀衛,次外命婦,次內命婦,皆乘車前導。內使監扈從,宿衛陳兵仗前後導從。皇后至廟門,司賓引命婦先入。皇后降車,司贊導自左門入,就位,北向立。命婦各就位,北向立。司贊奏拜,司賓贊拜,皇后及命婦皆再拜,興。司贊請詣盥洗位,盥手帨抉手,由東陛升,至神位前。司贊奏上香者三,司香捧香於右,皇后三上訖,導復位,贊拜如前。司贊奏禮畢,皇后出自廟之左門,命婦以次出。皇后升車,命婦前導,如來儀。過廟,鼓吹振作,皇后入宮。是日,皇帝宴群臣於謹身殿,皇后宴內外命婦於中宮。皆如正旦宴會儀。

及成祖即位,冊皇后徐氏,其制小異。皇帝皮弁服御華蓋殿,翰林院官以詔書用寶訖,然後御奉天殿,傳制皇后受冊。禮畢,翰林官以詔書授禮部官,禮部官奉詔書於承天門開讀。皇帝還宮,率皇后具服詣奉先殿謁告畢。皇后具服於內殿,俟皇帝升座。贊引女官導詣拜位,行謝恩禮,樂作。八拜,興,樂止。禮畢。次日,皇帝皇后受賀宴會,如前儀。天順八年,增定親王於皇帝前慶賀,次詣皇太后慶賀,次詣皇后前八拜儀。嘉靖十三年,冊皇后方氏,禮臣具儀注,有謁告內殿儀,無謁告太廟世廟之禮,帝命議增。於是禮臣以儀上。先期齋三日,所司陳設如時祫儀。至日,皇帝御輅,皇后妃御翟車,同詣太廟。命官奉七廟主升神座訖。皇帝奉高皇帝主,皇后奉高皇后主,出升神座。迎神、上香、奠帛、祼獻,樂作止,皆如儀。次詣世廟行禮,同上儀。隆慶元年增定,頒詔次日,命婦行見皇后禮。

冊妃之儀。自洪武三年冊孫氏為貴妃,定皇帝不御殿,承制官宣制曰:「妃某氏,特封某妃,命卿等持節行禮。」但授冊,無寶,餘並如中宮儀。永樂七年,定冊妃禮。皇帝皮弁服御華蓋殿,傳制。至宣宗立孫貴妃,始授寶,憲宗封萬貴妃,始稱皇,非洪武之舊矣。嘉靖十年,帝冊九嬪,禮官上儀注。先日,所司陳設儀仗如朔望儀。至期,皇帝具袞冕,告太廟、世廟訖,易皮弁服,御華蓋殿。百官公服入行禮。正、副使朝服承制,舉節冊至九嬪宮。九嬪迎於宮門外,隨至拜位。女官宣冊,九嬪受冊,先後八拜。送節出宮門復命。九嬪隨具服候,皇后率詣奉先殿謁告,及詣皇帝、皇后前謝恩,俱如冊妃禮。惟圭用次玉,穀文、銀冊少殺於皇妃五分之一。二十年,冊德妃張氏,以妃將就室,而帝方靜攝,不傳制,不謁告內殿,餘並如舊。

○冊皇太子及皇太子妃儀

自漢代始稱皇太子,明帝始有臨軒、冊拜之儀。唐則年長者臨軒冊授,幼者遣使內冊。宋惟用臨軒。元惟用內冊,不以長幼。

明興,定制,冊皇太子,所司陳設如冊后儀。設皇太子拜位於丹陛上。中嚴,皇帝袞冕御謹身殿,皇太子冕服俟於奉天門。外辦,皇帝升奉天殿,引禮導皇太子入奉天東門。樂作,由東階升至丹陛位,樂止。百官各就丹墀位。樂作,皇太子再拜,興,樂止。承制官由殿中門出,立於門外,曰:「有制」。皇太子跪。宣制曰:「冊長子某為皇太子。」皇太子俯伏,興,樂作。再拜。樂止。引禮導皇太子由殿東門入,樂作。內贊導至御座前,樂止。內贊贊跪,贊宣冊。宣畢,贊搢圭,贊授冊。皇太子搢圭,跪受冊,以授內侍。復贊授寶,如授冊儀。贊出圭,皇太子出圭,俯伏,興,由殿東門出。執事官舉節冊寶隨出。皇太子復位,樂作。四拜興,樂止。由東階降,樂作。至奉天門,樂止。儀仗、鼓樂迎冊寶至文華殿,持節官持節復命,禮部官奉詔書赴午門開讀,百官迎詔至中書省,頒行。侍儀奏禮畢,駕興,還宮。皇太子詣內殿,候皇后升座,行朝謝禮,四拜,恭謝曰:「小子某,茲受冊命,謹詣母后殿下恭謝。」復四拜,禮畢。親王、世子、郡王俟於文華殿陛上。皇太子升座,親王以下由東陛升,就拜位四拜。長王恭賀曰:「小弟某,茲遇長兄皇太子榮膺冊寶,不勝欣忭之至,謹率諸弟詣殿下稱賀。」賀畢,皆四拜。皇太子興,以次出。諸王詣中宮四拜,長王致詞賀畢,皆四拜出。是日,皇太子詣武英殿見諸叔,行家人禮,四拜,諸叔西向坐受。見諸兄,行家人禮,二拜,諸兄西向立受。次日,百官進表箋慶賀,內外命婦慶賀中宮,如常儀。乃擇日,太子謁太廟。

洪武二十八年,皇太子、親王俱授金冊,不用寶。永樂二年定,先三日齋戒,遣官祭告天地、宗廟,受冊寶畢,先詣太廟謁告,後至奉天殿謝恩,乃入謝中宮。二十二年十月,冊東宮,以梓宮在殯,樂設而不作。奉先殿行禮畢,仍詣几筵謁告。宣德二年十一月,皇子生,群臣表請立太子,三年二月行禮,以太子尚幼,乃命正、副使授冊寶於文華門。成化十一年以冊立皇太子禮成,文武官分五等,賜彩緞有差。嘉靖十八年二,月冊東宮,帝詣南郊告上帝,詣太廟告皇祖,自北郊及列聖宗廟以下皆遣官。時太子方二歲,保姆奉之,迎冊寶於文華殿門,詣皇帝前謝恩,皇后、貴妃代太子八拜。詣皇后前,貴妃代八拜。詣貴妃前,保姆代四拜。餘如常儀。

其皇太子妃受冊,與皇太子同日傳制。節冊將至內殿,妃降自東階,迎置於案。贊就拜位,贊跪,妃跪。贊宣冊,女官跪取冊,立宣畢。贊授冊,贊搢圭。女官以冊授妃,妃搢圭,受冊訖,以授女官。女官跪受,捧立。贊出圭,興,四拜。禮畢,內官持節出,妃送至殿外,正副使持節復命。是日,妃具禮服詣奉先殿行謁告禮。隨詣宮門,俟皇帝、皇后升座,入謝恩,行八拜禮。又詣各宮皇妃前,行四拜禮。還宮,詣皇太子前,亦四拜。禮畢,升座,王妃、公主、郡主及外命婦,於丹墀拜賀如儀。

○冊親王及王妃儀

漢冊親王於廟。唐臨軒冊命,禮極詳備。宋有冊命之文,皆上表辭免,惟迎官誥還第。元亦降制命之,不行冊禮。

明洪武三年定制,冊命親王,先期告宗廟,所司陳設如冊東宮儀。至日,皇帝御奉天殿,皇太子、親王由奉天東門入。樂作,升自東陛。皇太子由殿東門入,內贊導至御前,侍立位。親王入至丹陛拜位,樂止。贊拜,樂作。再拜,興,樂止。承制官承制如儀,諸王皆跪,宣制曰:「封皇子某為某王,某為某王。」宣畢,諸王俯伏,興。贊拜,樂作。再拜,興,樂止。引禮導王由殿東外入,樂作。內贊引至御座前拜位,樂止。王跪。贊授冊,捧冊官以冊授讀冊官,讀訖,以授丞相。丞相授王,王搢圭受,以授內使。授寶如上儀。訖,王出圭,俯伏,興。引禮導王出,復位。以次引諸王入殿,授冊寶如儀。內使以冊寶置彩亭訖,贊拜,樂作。諸王皆四拜,興,樂止。內使舉亭前行,親王由東階降,樂作。出奉天東門,樂止。禮部尚書請詔書用寶,赴午門開讀。禮畢,皇帝還宮,皇太子出。王年幼,則遣官齎冊寶授之。丞相承制至王所,東北立,西南向,宣制。最幼者行保抱之禮。是日,親王朝謝皇后、太子,與東宮受冊朝謝同。親王各自行賀,幼者詣長者,行四拜禮。百官詣親王賀,亦行四拜禮。丞相至殿上跪,文武官於庭中。丞相致詞曰:「某官某等,茲遇親王殿下榮膺冊寶,封建禮成,無任欣忭之至。」賀畢,丞相及百官復四拜。次日,皇太子冕服於奉天殿朝賀皇帝,太子致詞曰:「長子某,茲遇諸弟某等受封建國,謹詣父皇陛下稱賀。」賀中宮,致詞曰:「謹詣母后殿下稱賀。」百官進表箋賀皇帝及中宮、東宮,如東宮受冊儀。內外命婦賀中宮,致詞曰:「妾某氏等,茲遇親王受封建國,恭詣皇后殿下稱賀。」是日,百官及命婦各賜宴。擇日,諸王謁太廟。時秦、晉、燕、楚、吳五王皆長,而齊、潭、趙、魯四王方幼,故兼具其制。靖江王則以親王封,故視秦、晉儀。

二十八年定制,親王嫡長子,年十歲,授金冊寶,立為王世子。次嫡及庶子皆封郡王。凡王世子必以嫡長,王年三十,正妃未有嫡子,其子止為郡王。待王與正妃年五十無嫡子,始立庶長子為王世子,襲封。朝廷遣人行冊命之禮。成化末,封興、岐、益、衡、雍五王,帝親告奉先殿,遣使就各王府冊之,罷臨軒禮。而諸王當襲封者,俱於歲終遣官冊封。嘉靖中,改於孟春,著為令。冊王妃與冊太子妃儀同。

冊公主儀。洪武九年七月,命使冊公主。設冊案於清乾宮御座之東南,冊用銀字鍍金。皇帝、皇后升御座,遣使捧冊傳制如儀。使者至華蓋殿,公主拜受,其儀略與冊太子妃同。凡皇姑曰大長公主,皇姊妹曰長公主,皇女曰公主,親王女曰郡主,郡王女曰縣主,孫女曰郡君,曾孫女曰縣君,玄孫女曰鄉君。郡主以下,受誥封,不冊命。

○皇帝加元服儀

古者冠必於廟,天子四加。魏以後始冠於正殿,又以天子至尊,禮惟一加,歷代因之。

明洪武三年定制。先期,太史院卜日,工部製冕服,翰林院撰祝文,禮部具儀注。中書省承制,命某官攝太師,某官攝太尉。既卜日,遣官告天地、宗廟。前一日,內使監令陳御冠席於奉天殿正中,其南設冕服案及香案寶案。侍儀司設太師、太尉起居位於文樓南,西向,設拜位於丹墀內道,設侍立位於殿上御席西,設盥洗位於丹陛西。其百官及諸執事位次如大朝儀。是日質明,鼓三嚴,百官入。皇帝服空頂幘、雙童髻、雙玉導、絳紗袍,御輿以出。侍衛警蹕奏如儀。皇帝升座。鳴鞭報時訖,通班贊各供事。太師太尉先入,就拜位,百官皆入。贊拜,樂作。四拜,興,樂止。引禮導太師先詣盥洗位,搢笏盥帨訖,出笏,由西陛升。內贊接引至御席西,東向立。引禮復導太尉盥搢訖,入立於太師南。侍儀奏請加元服。太尉詣皇帝前,少右,跪搢笏。脫空頂幘以授內使,置於箱。進櫛設AO畢,出笏,興,退立於西。太師前,北向立。內使監令取冕立於左,太師祝曰:「令月吉日,始加元服,壽考維祺,以介景福。」內使監令捧冕,跪授太師。太師搢笏,跪受冕。加冠、加簪纓訖,出笏,興,退立於西。御用監令奏請皇帝著袞服,皇帝興,著袞服。侍儀奏請就御座,內贊贊進醴,樂作。太師詣御前北面立,光祿卿奉酒進授太師,太師搢笏受酒,祝曰:「甘醴惟厚,嘉薦令芳。承天之休,壽考不忘。」祝訖,跪授內使。內使跪受酒,捧進。皇帝受,祭少許,啐酒訖,以虛盞授內使,樂止。內使受盞降,授太師。太師受盞興,以授光祿卿,光祿卿受盞退。太師出笏,退,復位。內贊導太師太尉出殿西門,樂作,降自西階。引禮導至丹墀拜位,樂止。贊拜,樂作。太師太尉及文武官皆四拜,興,樂止。三舞蹈,山呼,俯伏,興,樂作。復四拜,樂止。禮畢,皇帝興,鳴鞭,樂作。入宮,樂止。百官出。皇帝改服通天冠、絳紗袍,拜謁太后,如正旦儀。擇日謁太廟,與時祭同。明日,百官公服稱賀,賜宴謹身殿。

萬曆三年正月,帝擇日長髮,命禮部具儀。大學士張居正等言:「禮重冠婚,皇上前在東宮已行冠禮,三加稱尊,執爵而酳。巨禮既成,可略其細。不必命部臣擬議。第先期至奉先殿、弘孝殿、神霄殿以長髮告。禮畢,詣兩宮皇太后,行五拜三叩頭禮,隨御乾清宮受賀。」帝是之,遂著為令。

○皇太子皇子冠禮

《禮》曰:「冠於阼,以著代也。醮於客位,三加彌尊,加有成也。已冠而字之,成人之道也。」「雖天子之元子,猶士也。」其禮歷代用之。明皇太子加元服,參用周文王、成王冠禮之年,近則十二,遠則十五。嘉靖二十四年,穆宗在東宮,方十歲,欲行冠禮。大學士嚴嵩、尚書費寀初皆難之,後遂阿旨以為可行,而請稍簡煩儀,止取成禮。帝以冠當具禮,至二十八年始行之。

其儀洪武元年定。前期,太史監卜日,工部置袞冕諸服,翰林院撰祝文。中書省承制。命某官為賓,某官贊。既卜日,遣官告天地宗廟。前一日,陳御座香案於奉天殿,設皇太子次於殿東房,賓贊次於午門外。質明,執事官設罍洗於東階,設皇太子冠席於殿上東南,西向,設醴席於西階上,南向,張帷幄於東序內,設褥席於帷中,又張帷於序外。御用監陳服於帷內東,領北上,袞服九章、遠遊冠、絳紗袍、折上巾、緇AO犀簪在服南,櫛又在南。司尊實醴於側尊,加勺冪,設於醴席之南。設坫於尊東,置二爵。進饌者實饌,設於尊北。諸執事者各立其所。鼓三嚴,文武官入。皇帝服通天冠、絳紗袍,升座如常儀。賓贊就位,樂作。四拜興,樂止。侍儀司跪承制,降至東階,詣賓前,稱有敕。賓贊及在位官皆跪。宣制曰:「皇太子冠,命卿等行禮。」皆俯伏,興,四拜。文武侍從班俱就殿內位,賓贊執事官詣東階下位。東宮官及太常博士詣殿前東房,導皇太子入就冠席,二內侍夾侍,東宮官後從,樂作。即席西南向,樂止。賓贊以次詣罍洗,樂作。搢笏,盥帨,出笏,樂止。升自西階,執事者奉折上巾進,賓降一等受之。右執項,左執前,進太子席前,北面祝畢,跪冠,樂作。賓興,席南北面立。贊冠者進席前,北面跪,正冠,興,立於賓後。內侍跪進服,皇太子興,服訖,樂止。賓揖皇太子復坐。賓贊降,詣罍洗訖,贊進前跪,脫折上巾,置於箱,興,以授內侍。執事者奉遠遊冠進,賓降二等受之,樂作,進冠如前儀。贊進前,北面跪,簪結紘,內侍跪進服,樂止。賓揖皇太子復坐。又詣罍洗,贊脫冠,執事者奉袞冕進,賓降三等受之,樂作。進冠結紘,內侍跪進服,如前儀,樂止。太常博士導皇太子降自東階,樂作。由西階升,即醴席,南向坐,樂止。賓詣罍洗盥帨訖,贊冠者取爵、盥爵、帨爵,詣司尊所酌醴,授賓。賓受爵,跪進於皇太子。祝畢,皇太子搢圭,跪受爵,樂作。飲訖,奠爵,執圭。進饌者奉饌於前,皇太子搢圭,食訖,執圭,興,樂止。徹爵與饌。博士導皇太子降自西階,至殿東房,易朝服,詣丹墀拜位,北向。東宮官屬各復拜位。賓贊詣皇太子位稍東,西向。賓少進字之辭曰:「奉敕字某。」皇太子再拜,跪聽宣敕。復再拜,興。進御前跪奏曰:「臣不敏,敢不祗承。」奏畢,復位。侍立官並降殿復位,四拜禮畢,皇帝興。內給事導皇太子入內殿,見皇后,如正旦儀。明日謁廟,如時享禮。又明日,百官朝服詣奉天殿稱賀,退易公服,詣東宮稱賀,錫宴。

成化十四年,續定皇太子冠禮。先日,設幕次於文華殿東序,設節案、香案、冠席、醴席、盥洗、司尊所等,具如儀。內侍張帷幄,陳袍服、皮弁服、袞服、圭帶、舄、翼善冠、皮弁、九旒冕。質明,皇帝御奉天殿傳制,遣官持節。皇太子迎於文華殿門外,捧入,置於案,退。禮部官導皇太子詣香案前,樂作。四拜,樂止。行初加冠禮。內侍奉翼善冠,賓祝曰:「吉月令辰,乃加元服。懋敬是承,永介景福。」樂作。賓跪進冠,興,樂止。禮部官啟易服,皇太子入幄,易袍服出,啟復坐。行再加冠禮。內侍奉皮弁,賓祝曰:「冠禮申舉,以成令德。敬慎威儀,惟民之式。」冠畢,入幄,易皮弁服舄出,啟復坐。行三加冠禮。內侍奉冕旒,賓祝曰:「章服咸加,飭敬有虔。永固皇圖,於千萬年。」冠畢,入幄,易袞服出,啟復坐。行醮禮,皇太子詣醴席,樂作。即坐,樂止。光祿寺官舉醴案,樂作。贊酌醴授賓,賓執爵詣席前,樂止。賓祝曰:「旨酒孔馨,加薦再芳。受天之福,萬世其昌。」賓跪進爵,皇太子搢圭,受爵,置於案。教坊司作樂,奏《喜千春之曲》。次啟進酒,皇太子舉爵飲訖,奠爵於案,樂止。光祿寺官進饌,樂作。至案,樂止。饌訖,出圭,徹案,賓贊復位。鳴贊贊受敕戒。皇太子降階,樂作。至拜位,樂止。宣敕戒官詣皇太子前稍東,西向立,曰「有制」。皇太子跪,宣敕戒曰:「孝事君親,友于兄弟。親賢愛民,居由仁義。毋怠毋驕,茂隆萬世。」樂作。四拜興,樂止。持節官捧節出,樂作。皇太子送節至殿門外,還東序。內侍導還宮,樂止。賓贊等官持節復命,餘如舊儀。是日,皇太子詣皇太后、皇帝、皇后前謝,俱行五拜三叩頭禮,用樂。明日,皇帝及皇太子受群臣賀,如儀。

皇子冠禮。初加,進網巾,祝詞曰:「茲惟吉日,冠以成人。克敦孝友,福祿來駢。」再加,進翼善冠,祝詞曰:「冠禮斯舉,賓由成德。敬慎威儀,維民之則。」三加,進袞冕,祝詞曰:「冠至三加,命服用章。敬神事上,永固籓邦。」酌醴祝曰:「旨酒嘉薦,載芬載芳。受茲景福,百世其昌。」敕戒詞曰:「孝於君親,友於兄弟。親賢愛民,率由禮義。毋溢毋驕,永保富貴。」其陳設執事及傳制謁謝,並如皇太子儀。初,皇子冠之明日,百官稱賀畢,詣王府行禮。成化二十三年,皇子冠之次日,各詣奉天門東廡序坐,百官常服四拜。

萬曆二十九年,禮部尚書馮琦言:「舊制皇太子冠,設冠席、醴席於文華殿內。今文華殿為皇上臨御遣官之地,則皇太子冠醴席,應移於殿之東序。又親王冠,舊設席於皇極門之東廡。今皇太子移席於殿東序,則親王應移席於殿西序。」從之。

永樂九年十一月,命皇太子嫡長子為皇太孫,冠於華蓋殿,其儀與皇太子同。

○品官冠禮

古者男子二十而冠,大夫五十而後爵,故無大夫冠禮。唐制,三加,一品之子以袞冕,逮九品之子以爵弁,皆仿士禮而增益之。

明洪武元年定制,始加緇布冠,再加進賢冠,三加爵弁。其儀,前期擇日,主者告於家廟,乃筮賓。前二日,戒賓及贊冠者。明日,設次於大門外之右,南向。至日,夙興,設洗於阼階東南,東西當東霤,六品以下當東榮,南北以堂深。罍水在洗東,加勺冪。篚在洗西南。肆實巾一於篚,加冪。設席於東房西牖下,陳服於席東,領北上。莞筵四,加藻席四,在南。側尊AP醴在服北,加勺冪,設坫在尊北。四品以下,設篚無坫,饌陳於坫北。設先於東房,近北。罍在洗西,篚在洗東北,肆實以巾。質明,賓贊至門外,掌次者引之次。賓贊公服,諸行事者各服其服,就位。冠各一笥,人執之,侍於西階之西,東面北上。設主席於阼階上,西面;設賓席於西階,東面;冠者席於主者東北,西面。主者公服立於阼階下,當東序,西面。諸親公服立於罍洗東南,西面北上。尊者在別室。儐者公服立於門內道東,北面。冠者雙童髻、空頂幘、雙玉導、彩褶、錦紳、烏皮履,六品以下,導不以玉,立於房中,南面。主者、贊冠者公服立於房內戶東,西面。賓及贊冠者出次,立於門西,東面北上。儐者進受命,出立門東,西面,曰:「敢請事。」賓曰:「某子有嘉禮,命某執事。」儐者入告,主者迎賓於大門外之東,西面,再拜,賓答拜。主者揖贊冠者,贊冠者報揖。又揖賓,賓報揖。主者入,賓贊次入,及內門至階。主者請升,賓三辭,乃陞。主者自阼階,立於席東,西向;賓自西階,立於席西,東向。賓贊冠者及庭,盥於洗,陞自西階,入於東房,立於主贊冠者之南,西面。主贊冠者導冠者立於房外之西,南面。賓贊冠者取AO櫛簪,跪奠於筵南端,退立於席北,少東,西面。賓揖冠者,冠者進升席,西向坐。賓贊冠者進筵前,東西跪,脫雙童髻,櫛畢,設AO,興,復位立。賓降至罍,洗盥訖,詣西階。主者立於席後西面,賓立於西階上,東面。執緇布冠者升,賓降一等受之,右執項,左執前,進冠者筵前,東向立。祝用士禮祝詞,祝畢,跪冠。興,復位。賓贊冠者進筵前,東面跪,結纓,興,復位。冠者興,賓揖之適房,賓主皆坐。冠者衣青衣素裳出戶西,南面立,賓主俱興。賓揖冠者,冠者進升席,西向坐。賓贊冠者跪,脫緇布冠,櫛畢,設AO。賓進進賢冠,立祝,如初加禮。祝畢,跪冠,興,復位。賓贊冠者跪,脫進賢冠,櫛畢,設AO。賓進爵弁,立祝,如再加禮。賓贊冠者,設簪結纓如前。冠者適房,著爵弁之服出。主贊冠者徹AO櫛及筵,入於房。又設筵於室戶西,南向。冠者出房戶西,南面立。主贊洗觶於房,酌醴出,南面立。賓揖冠者就筵西,南面立。賓受醴,進冠者筵前,北面立。祝畢,冠者拜受觶,賓復西階上答拜。執饌者進饌於筵,冠者左執觶,右取脯,祭於籩豆間。贊者取胏一以授冠者,奠觶於薦西以祭。冠者坐取觶,祭醴,奠觶,再拜,賓答拜。冠者執觶興,賓主俱坐。冠者升筵,跪奠觶於薦東。興,進,北面跪取脯,降自西階。入見母,進奠脯於席前。退,再拜出。母不在,則使人受脯於西階下。

初,冠者入見母,賓主俱興。賓降,當西序東面立,主者降,當東序西面立。冠者出,立於西階東,南面。賓少進字之,辭同士禮。冠者再拜,跪曰:「某不敏,夙夜祗承。」賓出,主者送於內門外,西向,請禮從者。賓就次,主者入。

初,賓出,冠者東面見諸親,諸親拜之,冠者答拜。冠者西向拜賓贊,賓贊亦答拜。見諸尊於別室,亦如之。賓主既釋服,入醴席,一獻訖,賓與眾賓出次,立於門東,西面。主者出揖賓,賓報揖。主者先入,賓及眾賓從之。至階,賓立於西階上,主者立於東階上,眾賓立於西階下。主者授幣篚於賓贊,復位,還阼階上,北面拜送。賓贊降自西階,主者送賓於大門外,西面,再拜而入。孤子則諸父諸兄戒賓。冠之日,主者紒而迎賓,冠於阼階下,其儀亦如之。明日見廟,冠者朝服入南門中庭道西,北面再拜出。

○庶人冠禮

古冠禮之存者惟士禮,後世皆推而用之。明洪武元年詔定冠禮,下及庶人,纖悉備具。然自品官而降,鮮有能行之者,載之禮官,備故事而已。

凡男子年十五至二十,皆可冠。將冠,筮日,筮賓,戒賓,俱如品官儀。是日,夙興,張幄為房於廳事東,皆盛服。設盥於阼階下東南,陳服於房中西牖下。席二在南,酒在服北次。襆頭巾帽,各盛以盤,三人捧之,立於堂下西階之西,南向東上。主人立於阼階下,諸親立於盥東,儐者立於門外以俟賓。冠者雙紒袍,勒帛素履待於房。賓至,主人出迎,揖而入。坐定,冠者出於房,執事者請行事。賓之贊者取櫛總篦幧頭,置於席南端。賓揖冠者,即席西向坐。贊者為之櫛,合紒施總,加幧頭。賓降,主亦降,立於阼階下。賓盥,主人揖讓,升自西階,復位。執事者進巾,賓降一等受之,詣冠者席前,東向。祝詞同品官。祝訖,跪著巾。興,復位。冠者興,賓揖之入房,易服,深衣大帶,出就冠席。賓盥如初。執事者進帽,賓降二等受之。進祝,跪,冠訖,興,復位。揖冠者入房,易服,襴衫要帶,出就冠席。賓盥如初。執事者進襆頭,賓降三等受之。進祝,跪,冠訖,興,復位。揖冠者入房,易公服出。執事者徹冠席,設醴席於西階,南向。贊者酌醴出房,立於冠者之南。賓揖冠者即席,西向立。賓受醴,詣席前北面祝。冠者拜受,賓答拜。執事者進饌,冠者即席坐,飲食訖,再拜。賓答拜。冠者離席,立於西階之東,南向。賓字之,如品官詞。冠者拜,賓答拜。冠者拜父母,父母為之起。拜諸父之尊者,遂出見鄉先生及父之執友。先生執友皆答拜。賓退,主人請禮賓,固請,乃入,設酒饌。賓退,主人酬賓贊,侑以幣。禮畢,主人以冠者見於祠堂,再拜出。

天子納后儀皇太子納妃儀親王婚禮公主婚禮品官婚禮庶人婚禮皇帝視學儀經筵日講東宮出閣講學儀諸王讀書儀

○天子納后儀

婚禮有六,天子惟無親迎禮。漢、晉以來,皆遣使持節奉迎,其禮物儀文,各以時損益。明興,諸帝皆即位後行冊立禮。正統七年,英宗大婚,始定儀注。

凡納采問名,前期擇日,遣官告天地宗廟。至期,設御座、制案、節案、鹵簿、彩輿、中和大樂如儀。禮部陳禮物於丹陛上及文樓下。質明,皇帝冕服升座,百官朝服行禮訖,各就位。正副使朝服四拜,執事舉制案、節案,由中門出,禮物隨之,俱置丹陛中道。傳制官宣制曰:「茲選某官某女為皇后,命卿等持節行納采問名禮。」正副使四拜,駕興。舉制、節案由奉天門中門出。正副使取節及制書置綵輿中,儀仗大樂前導,出大明門。釋朝服,乘馬行,詣皇后第。第中設使者幕次於大門外左,南向,設香案於正堂,設制、節案於南,別設案於北。使者至,引禮導入幕次,執事官陳禮物於正堂。使者出次,奉制書於案。禮官先入,立於東;主婚朝服出,立於西。禮官曰:「奉制建后,遣使行納采問名禮。」引主婚者出迎。使者捧制書及節,主婚者隨至堂,置制書及節於案。正副使分立案左右。主婚者四拜,詣案前跪。正使取納采制,宣曰:「朕承天序,欽紹鴻圖。經國之道,正家為本。夫婦之倫,乾坤之義,實以相宗祀之敬,協奉養之誠,所資惟重。祗遵聖母皇太后命,遣使持節,以禮采擇。」宣訖,授主婚者。主婚者授執事者,置於北案上稍左。副使取問名制,宣曰:「朕惟夫婦之道,大倫之本。正位乎內,必資名家。特遣使持節以禮問名,尚佇來聞。」宣訖,授如前,置案上稍右。主婚者俯伏,興。執事舉表案,以表授主婚者。主婚者跪授正使,表曰:「臣某,伏承嘉命。正使某官某等,重宣制詔,問臣名族。臣女,臣夫婦所生,先臣某官某之曾孫,先臣某官某之孫,先臣某官某之外孫。臣女今年若干,謹具奏聞。」主婚者俯伏,興,退四拜。使者出,置表彩輿中。主婚者前曰:「請禮從者。」酒饌畢,主婚者捧幣以勞使者。使者出,主婚者送至大門外。使者隨彩輿入大明門左門,至奉天門外,以表節授司禮監,復命。

次納吉、納徵、告期,傳制遣使,並如前儀。但納徵用玄纁、束帛、六馬、穀圭等物,制詞曰:「茲聘某官某女為皇后,命卿等持節行納吉、納徵、告期禮。」皇后第,陳設如前,惟更設玉帛案。使者至,以制書、玉帛置案上,六馬陳堂下。執事先設皇后冠服諸物於正堂。禮官入,主婚者出迎,執事舉玉帛案,正使捧納吉、納徵制書,副使捧告期制書,執節者捧節,以次入,各置於案。主婚者四拜,詣案前跪。正使取制書,宣曰:「大婚之卜,龜筮師士協從。敬循禮典,遣使持節告吉。」又宣曰:「卿女有貞靜之德,稱母儀之選,宜共承天地宗廟。特遣使持節,以禮納徵。」宣訖,授主婚者。正副使又捧圭及玄纁以授主婚者,俱如前儀。副使取制書,宣曰:「歲令月良,吉日某甲子,大婚維宜。特遣使持節,以禮告期。」宣訖,授如前儀。主婚者四拜,使者持節出,主婚者禮使者,使者還,復命如初。

次發冊奉迎,所司陳設如前儀。禮部陳雁及禮物於丹陛上,內官監陳皇后鹵簿車輅於奉天門外。制詞曰:「茲冊某官某女為皇后,命卿等持節奉冊寶,行奉迎禮。」正副使以冊寶置綵輿中,隨詣皇后第。至門,取制書冊寶置案上。禮官先入,主婚者朝服出見。禮官曰:「奉制冊后,遣使持節奉冊寶,行奉迎禮。」主婚者出迎。執事者舉案前行,使者捧制書及節,執事者以雁及禮物從之。至堂中,各置於案。使者左右立,主婚者四拜,退立於西南。

女官以九龍四鳳冠禕衣進皇后。內官陳儀仗於中堂前,設女樂於堂下,作止如常儀。使者以節冊寶授司禮監官,內贊導入中堂。皇后具服出閣,詣香案前,向闕立,四拜。贊宣冊,皇后跪。宣冊官宣訖,以授皇后。皇后搢圭,受冊,以授女官。女官跪受,立於西。贊宣寶,如宣冊儀。贊出圭,贊興,四拜訖,皇后入閣。司禮監官持節出,授使者,報受冊寶禮畢。主婚者詣案前跪。正使取奉迎制宣訖,授主婚者。副使進雁及禮物。主婚者皆跪受,如前儀。主婚者興,使者四拜出。主婚者禮使者如初。女官奏請皇后出閣。自東階下,立香案前,四拜。升堂,南向立。主婚者進立於東,西向,曰:「戒之敬之,夙夜無違。」退立於東階。母進,立於西,東向,施衿結帨,曰:「勉之敬之,夙夜無違。」退立於西階。內執事請乘輿,皇后降階升輿。導從出,儀仗大樂前行,次綵輿,正副使隨,次司禮監官擁導,從大明門中門入。百官朝服於承天門外班迎,候輿入,乃退。皇后至午門外,鳴鐘鼓,鹵簿止。正副使以節授司禮監,復命。捧冊寶官捧冊寶,儀仗女樂前導,進奉天門。至內庭幕次,司禮監以冊寶授女官。皇后出輿,由西階進。皇帝由東階降迎於庭,揖皇后入內殿。帝詣更服處,具袞冕。后詣更服處,更禮服。同詣奉先殿,行謁廟禮。祭畢,還宮。合巹,帝更皮弁,升內殿。后更衣,從升。各升座,東西相向。執事者舉饌案於前,女官取四金爵,酌酒以進。既飲,進饌。復進酒、進飯訖,女官以兩巹酌酒,合和以進。既飲,又進饌畢,興,易常服。帝從者餕后之饌,后從者餕帝之饌。

次日早,帝后皆禮服,候太后升座。帝后進座前。宮人以腶脩盤立於后左,帝后皆四拜。執事舉案至,宮人以腶脩盤授后,后捧置於案。女官舉案,后隨至太后前,進訖,復位。帝后皆四拜。三日早,帝冕服,后禮服,同詣太后宮,行八拜禮。還宮,帝服皮弁,升座。女宮導后,禮服詣帝前,行八拜禮。后還宮,升座。引禮導在內親屬及六尚等女官,行八拜禮;次各監局內官內使,行八拜禮。是日,皇帝御奉天殿。頒詔如常儀。四日早,皇帝服袞冕御華蓋殿,親王八拜,次執事官五拜,遂升奉天殿,百官進表,行慶賀禮。是日,太后及皇后各禮服陞座。親王入,八拜出,次內外命婦慶賀及外命婦進表箋,皆如常儀。五日行盥饋禮,尚膳監具膳脩。皇后禮服詣太后前,四拜。尚食以膳授皇后,皇后捧膳進於案,復位,四拜,退立於西南。俟膳畢,引出。

○皇太子納妃儀

歷代之制與納后同。隋、唐以後,始親迎,天子臨軒醮戒。宋始行盥饋禮,明因之,洪武元年定制,凡行禮,皆遣使持節,如皇帝大婚儀。

納采、問名。制曰:「奉制納某氏女為皇太子妃,命卿等行納采問名禮。」至妃第,儐者出,詣使者前曰:「敢請事。」使者曰:「儲宮納配,屬於令德。邦有常典,使某行納采之禮。」儐者入告,主婚者曰:「臣某之子,昧於壼儀,不足以備采擇。恭承制命,臣某不敢辭。」儐者出告,使者入,陳禮物於庭,宣制曰:「某奉詔采擇。」奠雁禮畢,使者出。儐者復詣使者前曰:「敢請事。」使者曰:「儲宮之配,採擇既諧。將加卜筮,奉制問名。」儐者入告,主婚者曰:「制以臣某之女,可以奉侍儲宮,臣某不敢辭。儐者出告。使者復入,陳禮奠雁如儀,宣制曰:「臣某奉詔問名,將謀諸卜筮。」主婚者曰:「臣某第幾女,某氏出。」

次納吉。儐者請事如前,使者曰:「謀諸卜筮,其占協從,制使某告吉。」儐者入告,主婚者曰:「臣某之子蠢愚,懼弗克堪。卜筮云吉,惟臣之幸,臣謹奉典制。」儐者出告。使者入,陳禮奠雁如儀,宣制曰:「奉制告吉。」

又次納徵。儐者出告,使者入陳玉帛禮物,不奠雁,宣制曰:「某奉制告成。」

又次請期。辭曰:「詢於龜筮。某月某日吉,制使某告期。」主婚者曰:「敢不承命。」陳禮奠雁如儀。

又次告廟。遣使持節授冊寶儀注,悉見前。

又次醮戒。皇帝服通天冠、絳紗袍,御奉天殿,百官侍立。引進導皇太子至丹陛,四拜。入殿東門就席位,東向立。司爵以醆進,皇太子跪,搢圭,受醆祭酒。司饌以饌進,跪受亦如之。興,就席坐,飲食訖,導詣御座前跪。皇帝命之曰:「往迎爾相,承我宗事,勖帥以敬。」皇太子曰:「臣某謹奉制旨。」俯伏,興。出至丹陛,四拜畢,皇帝還宮,皇太子出。

又次親迎。前一日,有司設皇太子次於妃氏大門外,南向,東宮官次於南,東西相向。至日質明,東宮官具朝服陳鹵簿鼓吹於東宮門外。皇太子冕服乘輿出,侍衛導從如儀。至宮門降輿升輅,東宮官皆從至妃第,回轅南向,降輅升輿。至次,降輿入就次,東宮官皆就次。先是,皇太子將至,主婚者設會宴女。至期,妃服褕翟花釵,出就閣南面立,傅姆立於左右。主婚者具朝服立於西階之下。引進導皇太子出次,立於大門之東,西向。儐者朝服出,立於門東曰:「敢請事。」引進跪啟訖,皇太子曰:「某奉制親迎。」引進受命興,承傳於儐者。儐者入告,導主婚者出迎於大門外之西,東向再拜。皇太子答拜。引進導皇太子入門而左,執雁者從。儐者導主婚者入門而右。皇太子升東階進,立於閣門戶前,北向立。主婚者升西階,立於西,東向。引進啟奠雁,執雁者以雁進。皇太子受雁,以授主婚者。主婚者跪受,興,以授左右,退立於西。皇太子再拜,降自東階,出至次以伺。主婚者不降送。初,皇太子入門,妃母出,立於閣門外,奠雁位之西,南向。皇太子拜訖,宮人傅姆導妃出,立於母左。主婚者命之曰:「戒之戒之,夙夜恪勤,毋或違命。」母命之曰:「勉之勉之,爾父有訓,往承惟欽。」庶母申之曰:「恭聽父母之言。」宮人傅姆擎執導從,妃乘輿出門,降輿,乘鳳轎。皇太子揭簾訖,遂升輅,侍從如來儀。至東宮門外,降輅乘輿。至閣,降輿入,俟於內殿門外之東,西面。司閨導妃詣內殿門外之西,東面。皇太子揖妃入,行合巹禮,如中宮儀。

又次朝見。其日,妃詣內殿陛下,候皇帝升座。司閨導妃入,北面立,再拜,自西階升。宮人奉棗慄盤,進至御座前授妃。妃奠於御前,退復位,再拜。禮畢,詣皇后前,奉腶脩盤,如上儀。

又次醴妃,次盥饋,次謁廟,次群臣命婦朝賀,皆如儀。

四年,冊開平王常遇春女為皇太子妃。禮部上儀注,太祖覽之曰:「贄禮不用笄,但用金盤,翟車用鳳轎,雁以玉為之。古禮有親迎執綏御輪,今用轎,則揭簾是矣。其合巹,依古制用匏。妃朝見,入宮中,乘小車,以帷幕蔽之。謁廟,則皇太子俱往。禮成後三日,乃宴群臣命婦。」著為令。

成化二十二年,更定婚禮。凡節冊等案,俱由奉天左門出。皇太子親迎,由東長安門出。親迎日,妃服燕居服,隨父母家廟行禮。執事者具酒饌,妃飲食訖。父母坐堂上,妃詣前各四拜。父命之曰:「爾往大內,夙夜勤慎,孝敬無違。」母命之曰:「爾父有訓,爾當敬承。」合巹前,於皇太子內殿各設拜位。皇太子揖妃入就位,再拜,妃四拜,然後各升座。廟見後,百官朝賀,致詞曰:「某官臣某等,恭惟皇太子嘉禮既成,益綿宗社隆長之福。臣某等不勝欣忭之至,謹當慶賀。」帝賜宴如正旦儀。命婦詣太后皇后前賀,亦賜宴,致詞曰:「皇太子嘉聘禮成,益綿景福。」余大率如洪武儀。

○親王婚禮

唐制,皇子納妃,命親王主婚。宋皆皇帝臨軒醮戒,略與皇太子同。明因之。

其宣制曰:「冊某氏為某王妃。」納采,致詞曰:「某王之儷,屬於懿淑,使某行納采禮。」問名詞曰:「某既受命,將加諸卜筮協從,使某告吉。」主婚者曰:「制以臣某之子,可以奉侍某王,臣某不敢辭。」納吉詞曰:「卜筮協從,使某告吉。」主婚者曰:「臣某之子,愚弗克堪。卜貺之吉,臣與有幸,謹奉典制。」納徵詞曰:「某王之儷,卜既協吉,制使某以儀物告成。」主婚者曰:「奉制賜臣以重禮,臣某謹奉典制。」請期詞曰:「某月日涓吉,制使某告期。」主婚者曰:「謹奉命。」醮戒命曰:「往迎爾相,用承厥家,勖帥以敬。」其親迎、合巹、朝見、盥饋,並如皇太子。盥饋畢,王皮弁服,妃翟衣,詣東宮前,行四拜禮。東宮坐受,東宮妃立受二拜,答二拜。王與妃至妃家,妃父出迎。王先入,妃父從之。至堂,王立於東,妃父母立於西。王四拜,妃父母立受二拜,答二拜。王中坐,其餘親屬見王,四拜,王皆坐受。妃入中堂,妃父母坐,妃四拜。其餘序家人禮。

太祖之世,皇太子、皇子有二妃。洪武八年十一月,徵衛國公鄧愈女為秦王次妃,不傳制,不發冊,不親迎。正副使行納徵禮,冠服擬唐、宋二品之制,儀仗視正妃稍減。婚之日,王皮弁服,導妃謁奉先殿。王在東稍前,妃西稍後。禮畢入宮,王與正妃正坐,次妃詣王前四拜,復詣正妃前四拜。次妃東坐,宴飲成禮。次日朝見,拜位如謁殿。謁中宮,不用棗慄腶脩,餘並同。

○公主婚禮

古者天子嫁女,不自主婚,以同姓諸侯主之,故曰公主。唐猶以親王主婚。宋始不用,惟令掌婚者於內東門納表,則天子自為主矣。明因之。

凡公主出降,行納采問名禮,婿家備禮物表文於家庭,望闕再拜。掌婚者奉至內東門,詣內使前曰:「朝恩貺室於某官某之子,某習先人之禮,使臣某請納采。」以表跪授內使。內使跪受,奉進內殿,執雁及禮物者從入。內使出,掌婚者曰:「將加卜筮,使臣某問名。」進表如初,內使出曰:「有制。」掌婚者跪,內使宣曰;「皇帝第幾女,封某公主。」掌婚者俯伏,興。入就次,賜宴出。

納吉儀與納采同。掌婚者致詞曰:「加諸卜筮,占曰從吉,謹使臣某敢告納徵。」婿家具玄纁、玉帛、乘馬、表文如儀。掌婚者致詞曰:「朝恩貺室於某官某之子某,有先人之禮,使臣某以束帛、乘馬納徵。」請期詞曰:「某命臣某謹請吉日。」

親迎日,婿公服告廟曰:「國恩貺室於某,以某日親迎,敢告。」將行,父醮於廳,隨意致戒。婿再拜出,至內東門內。內使延入次,執雁及奉禮物者各陳於庭。其日,公主禮服辭奉先殿,詣帝后前四拜,受爵。帝后隨意訓戒。受命訖,又四拜。降階,內命婦送至內殿門外,公主升輦。至內東門,降輦。婿揭簾,公主升轎。婿出次立。執雁者以雁跪授婿,婿受雁,跪進於內使。內使跪受以授左右。婿再拜,先出,乘馬還。公主鹵簿車輅後發,公侯百官命婦送至府。婿先候於門。公主至,婿揭簾。公主降,同詣祠堂。婿東,公主西,皆再拜。進爵,讀祝,又再拜。出,詣寢室。婿公主相向再拜,各就坐,婿東,公主西。進饌合巹如儀,復相向再拜。明日,見舅姑。舅姑坐於東,西向。公主立於西,東向,行四拜禮。舅姑答二拜。第十日,駙馬朝見謝恩,行五拜禮。

初,洪武九年,太祖以太師李善長子祺為駙馬都尉,尚臨安公主。先期告奉先殿。下嫁前二日,命使冊公主。冊後次日,謁奉先殿。又定駙馬受誥儀,吏部官捧誥命置龍亭,至太師府,駙馬朝服拜受。次日,善長及駙馬謝恩。後十日,始請婚期。二十六年,稍更儀注。然儀注雖存,其拜姑舅及公主駙馬相向拜之禮,終明之世實未嘗行也。明年,又更定公主、郡主封號、婚禮,及駙馬、儀賓品秩。

弘治二年,冊封仁和長公主,重定婚儀。入府,公主駙馬同拜天地,行八拜禮。堂內設公主座於東,西向,駙馬東向座,餘如前儀。嘉靖二年,工科給事中安磐等言:「駙馬見公主,行四拜禮,公主坐受二拜。雖貴賤本殊,而夫婦分定,於禮不安。」不聽。崇禎元年,教習駙馬主事陳鐘盛言:「臣都習駙馬鞏永固,駙馬黎明於府門外月臺四拜,云至三月後,則上堂、上門、上影壁,行禮如前。始視膳於公主前,公主飲食於上,駙馬侍立於旁,過此,方議成婚。駙馬餽果肴書臣,公主答禮書賜,皆大失禮。夫既合巹,則儼然夫婦,安有跪拜數月,稱臣侍膳,然後成婚者?《會典》行四拜於合巹之前,明合巹後無拜禮也。以天子館甥,下同隸役,豈所以尊朝廷?」帝是其言,令永固即擇日成婚。

凡選駙馬,禮部榜諭在京官員軍民子弟年十四至十六,容貌齊整、行止端莊、有家教者報名,司禮內臣於諸王館會選。不中,則博訪於畿內、山東、河南。選中三人,欽定一人,餘二人送本處儒學,充廩生。自宣德時,駙馬始有教習,用學官為之。正統以後,令駙馬赴監讀書習禮。嘉請六年,始定禮部主事一人,專在駙馬府教習。

○品官婚禮

周制,凡公侯大夫士之婚娶者,用六禮。唐以後,儀物多以官品為降殺。明洪武五年詔曰:「古之婚禮,結兩姓之歡,以重人倫。近世以來,專論聘財,習染奢侈。其儀制頒行。務從節儉,以厚風俗。」故其時品節詳明,皆有限制,後克遵者鮮矣。

其制,凡品官婚娶,或為子聘婦,皆使媒氏通書。女氏許之,擇吉納采。主婚者設賓席。至日,具祝版告廟訖,賓至女氏第。主婚者公服出迎,揖賓及媒氏人。雁及禮物陳於廳。賓左主右,媒氏立於賓南,皆再拜。賓詣主人曰:「某官以伉儷之重施於某,某率循典禮,謹使某納采。」主婚者曰:「某之子弗嫻姆訓,既辱採擇,敢不拜嘉。」賓主西東相向坐,徹雁受禮訖,復陳雁及問名禮物。賓興,詣主婚者曰:「某官慎重婚禮,將加卜筮,請問名。」主婚者進曰:「某第幾女,妻某氏出。」或以紅羅,或以銷金紙,書女之第行年歲。賓辭,主婚者請禮從者。禮畢,送賓至門外。

納吉如納采儀。賓致詞曰:「某官承嘉命,稽諸卜筮,龜筮協從,使某告吉。」主婚者曰:「某未教之女,既以吉告,其何敢辭。」納徵如納吉儀,加玄纁束帛、函書,不用雁。賓致詞曰:「某官以伉儷之重,加惠某官,率循典禮。有不腆之幣,敢請納徵。」主婚者曰:「某官貺某以重禮,某敢不拜受。」賓以函書授主婚者,主婚者亦答以函書。請期,亦如納吉儀。

親迎日,婿父告於禰廟。婿北面再拜立,父命之曰:「躬迎嘉偶,釐爾內治。」婿進曰:「敢不承命。」再拜,媒氏導婿之女家。其日,女氏主婚者告廟訖,醴女如家人禮。婿至門,下馬,就大門外之次。女從者請女盛服,就寢門內,南向坐。婿出次,主婚者出迎於門外,揖而入。主婚者入門而右。婿入門而左,執雁者從,至寢戶前,北面立。主婚者立於戶東,西向。婿再拜,奠雁,出就次。主婚者不降送。婿既出,女父母南向坐,保母導女四拜。父命之曰:「往之女家,以順為正,無忘肅恭。」母命之曰:「必恭必戒,毋違舅姑之命。」庶母申之曰:「爾忱聽於訓言,毋作父母羞。」保姆及侍女翼女出門,升車。儀衛導前,送者乘車後。婿先還以俟。婦車至門,出迎於門內,揖婦入。及寢門,婿先升階,婦從升。入室,婿盥於室之東南,婦從者執巾進水以沃之;婦盥於室之西北,婿從者執巾進水以沃之。盥畢,各就坐,婿東,婦西。舉食案,進酒,進饌。酒食訖,復進如初。侍女以巹注酒,進於婿婦前。各飲畢,皆興,立於座南,東西相向,皆再拜。婿婦入室,易服。婿從者餕婦之餘,婦從者餕婿之餘。

明日見宗廟,設婿父拜位於東階下,婿於其後;主婦拜位於西階下,婦於其後。諸親各以序分立。其日夙興,婿父以下各就位,再拜。贊禮引婦至庭中,北面立。婿父升自東階,詣神位前,跪。三上香,三祭酒,讀祝,興,立於西。婦四拜,退,復位。婿父降自西階就拜位,婿父以下皆再拜,禮畢。次見舅姑。其日,婦立堂下,伺舅姑即座,就位四拜。保姆引婦升自西階,至舅前,侍女奉棗慄授婦。婦進訖,降階四拜。詣姑前,進腶脩,如前儀。次舅姑醴婦,如家人禮。次盥饋。其日,婦家備饌至婿家。舅姑即座,婦四拜。升自西階,至舅前。從者舉食案以饌授婦,婦進饌,執事者加匕箸。進饌於姑,亦如之。食訖,徹饌,婦降階就位,四拜,禮畢。舅姑再醴婦,如初儀。

○庶人婚禮

《禮》云:「婚禮下達。」則六禮之行,無貴賤一也。朱子《家禮》無問名、納吉,止納採、納幣、請期。洪武元年定制用之,下令禁指腹、割衫襟為親者。凡庶人娶婦,男年十六,女年十四以上,並聽婚娶。婿常服,或假九品服,婦服花釵大袖。其納采、納幣、請期,略仿品官之儀。有媒無賓,詞亦稍異。親迎前一日,女氏使人陳設於婿之寢室,俗謂之鋪房。至若告詞、醮戒、奠雁、合巹,並如品官儀。見祖禰舅姑,舅姑醴婦,亦略相準。

○皇帝視學儀

《禮》曰:「凡始立學者,必釋奠於先聖先師。」周末淪喪,禮廢不行。漢明帝始幸辟雍。唐以後,天子視學,始設講榻。洪武十五年,太祖將幸國子監。議者言,孔子雖聖,乃人臣,禮宜一奠而再拜。太祖不從,命禮部尚書劉仲質定其制。

前期設御幄於大成門東,南向,設御座於彞倫堂。至日,學官率諸生迎駕於成賢街左。皇帝入御幄,具皮弁服,詣先師神位,再拜。獻爵,復再拜。四配、十哲、兩廡分獻,如常儀。皇帝入御幄,易常服。升輿,至彞倫堂升座。學官諸生五拜叩頭,東西序立於堂下。三品以上及侍從官,以次入堂,東西序立。贊進講,祭酒、司業、博士、助教四人由西門入,至堂中。贊舉經案於御前,禮部官奏,請授經於講官。祭酒跪受。賜講官坐。及以經置講案,叩頭,就西南隅几榻坐講。賜大臣翰林儒臣坐,皆叩頭,序坐於東西,諸生圜立以聽。講畢,叩頭,退就本位。司業、博士、助教,各以次進講。出堂門,復位。贊宣制,學官諸生列班俱北面跪,聽宣諭,五拜叩頭。禮畢,學官諸生出成賢街送駕。明日,祭酒率學官上表謝恩。

永樂四年,禮部尚書鄭賜引宋制,請服靴袍,再拜。帝不從,仍行四拜禮。進講畢,賜百官茶。禮部請立視學之碑,帝親製文勒石。祭酒等表謝。帝御奉天門,賜百官宴,仍賜祭酒、司業糸寧絲羅衣各二襲,學官三十五人各糸寧絲衣一襲,監生三千餘人各鈔五錠。正統九年,帝幸國子監,如儀。禮畢,賜公、侯、伯、駙馬、武官都督以上、文官三品以上及翰林學士至檢討、國子監祭酒至學錄宴。

先是,視學祭先師,不設牲,不奏樂。至成化元年,始用牲樂。視學之日,樂設而不作。禮畢,百官慶賀,賜衣服,賜宴,皆及孔、顏、孟三氏子孫。弘治元年,定先期致齋一日,奠加幣,牲用太牢,改分獻官為分奠官。嘉靖元年,定衍聖公率三氏子孫,祭酒率學官諸生,上表謝恩,皆賜宴於禮部。十三年,以先師祀典既正,再視學,命大臣致奠啟聖公祠。萬曆四年,定次日行慶賀禮,頒賞如舊,免賜宴。

初,憲宗取三氏子孫赴京觀禮,又命衍聖公分獻。至世宗,命衍聖公及顏、孟二博士,孔氏老成者五人,顏、孟各二人,赴京陪祀。

○經筵

明初無定日,亦無定所。正統初,始著為常儀,以月之二日御文華殿進講,月三次,寒暑暫免。其制,勳臣一人知經筵事,內閣學士或知或同知。尚書、都御史、通政使、大理卿及學士等侍班,翰林院、春坊官及國子監祭酒二員進講,春坊官二員展書,給事中御史各二員侍儀,鴻臚寺、錦衣衛堂上官各一員供事,鳴贊一贊禮,序班四舉案,勳臣或駙馬一人領將軍侍衛。

禮部擇吉請,先期設御座於文華殿,設御案於座東稍南,設講案於案南稍東。是日,司禮監先陳所講《四書》、經、史各一冊置御案,一冊置講案,皆《四書》東,經、史西。講官各擇撰講章置冊內。帝升座,知經筵及侍班等官於丹陛上,五拜三叩頭。後每講止行叩頭禮。以次上殿,東西序立。序班二員,舉御案於座前,二員舉講案置御案南正中。鴻臚官贊進講。講官二員從東西班出,詣講案前,北向並立。東西展書官各至御案南銅鶴下,相向立。鴻臚官贊講拜,興。東班展書官詣御案前,跪展《四書》,退立於東鶴下。講官至講案前立,奏講某書,講畢退。展書官跪掩書,仍退立鶴下。西班展書官展經或史,講官進講,退,如初。鴻臚官贊講官拜,興。各退就東西班,展書官隨之,序班徹御案講案。禮畢,命賜酒飯。各官出至丹陛,行叩頭禮。至左順門,酒飯畢,入行叩頭禮。

隆慶元年,定先一日告奉先殿,告几筵。是日,帝詣文華殿左室,展禮先聖先師。講章於前兩日先進呈覽。萬曆二年,定春講以二月十二日起,至五月初二日止,秋講以八月十二日起,至十月初二日止,不必題請。

○日講

日講,御文華穿殿,止用講讀官內閣學士侍班,不用侍儀等官。講官或四或六。開讀初,吉服,五拜三叩首,後常服,一拜三叩首。閣臣同侍於殿內,候帝口宣「先生來」,同進,叩首,東西立。讀者先至御前一揖,至案展書,壓金尺,執牙簽。讀五過,掩書一揖退。先書,次經,次史,進講如讀儀。侍書官侍習書畢,各叩頭退。於文華殿賜茶,文華門賜酒飯。

午講,隆慶六年定。每日早講畢,帝進煖閣少憩,閱章奏。閣臣等退西廂房,久之,率講官再進午講,講《通鑒節要》及《貞觀政要》。講畢,帝還宮。凡三、六、九視朝日,暫免講讀。

又嘉靖六年定制,月三、八日,經筵日講官二員,講《大學衍義》。十年,定無逸殿講儀。質明,帝常服乘輦至殿門,眾官於門外迎候。帝降輦,乘板輿,至殿升座。各官於殿門外一拜三叩首,入內,東西序立。贊進講,講官大學士一員出班叩首。命賜坐,一叩首,乃坐。講畢,展書官跪掩講章,講官叩頭復班。又學士一員承旨坐講,如初禮畢。各官至豳風亭候駕至,亭內賜宴。

○東宮出閣講學儀

太祖命學士宋濂授皇太子、諸王經於大本堂,後於文華後殿。世宗改為便殿,遂移殿東廂。天順二年,定出閣儀。是日早,侍衛侍儀如常。執事官於文華後殿四拜,鴻臚官請皇太子升殿,師保等於丹陛上四拜。各官退出,內侍導皇太子至後殿升座,以書案進。侍班侍讀講官入,分班東西立。內侍展書,侍讀講官以次進讀講,叩頭而退。

其每日講讀儀,早朝退後,皇太子出閣升座,不用侍衛等官,惟侍班侍讀講官入,行叩頭禮。內侍展書,先讀《四書》,則東班侍讀官向前,伴讀十數遍,退復班。次讀經或史,則西班伴讀,亦如之。讀畢,各官退。至巳時,各官入,內侍展書,侍講官講早所讀《四書》畢,退班。次講經史亦然。講畢,侍書官侍習寫字。寫畢,各官叩頭退。凡讀書,三日後一溫,背誦成熟。溫書之日,不授新書。凡寫字,春夏秋日百字,冬日五十字。凡朔望節假及大風雨雪、隆寒盛暑,則暫停。

弘治十一年更定,三師三少并宮僚於丹陛四拜畢,從殿左右門入,東西立。候講讀畢,叩頭退。隆慶六年,改設皇太子座於文華殿之東廂,正中西向。每日講讀各官,先詣文華門外東西向,序立。候帝御日講經筵畢,皇太子出閣升座。凡東宮初講時,閣臣連侍五日,後每月三、八日一至,先拜出,然後各官入。崇禎十一年,署禮部事學士顧錫疇言:「東宮嘉禮告成,累朝錫賚有據。《實錄》載成化十五年,皇太子出閣講學,六卿皆加保、傅。弘治十年,皇太子出閣講學,內閣徐溥等四人、尚書馬文升等七人,俱加宮保。」帝命酌議行之。

○諸王讀書儀

書堂在皇極門右廂。講官選部曹或進士改授翰林官充之。天順二年定,初入書堂,其日早,王至右順門之北書堂,面東,中坐。提督講讀并講讀官行四拜禮。內官捧書展於案上,就案左坐。講讀官進立於案右。伴讀十遍,叩頭退。每日講讀,清晨,王至書堂,講讀官行叩頭禮,伴讀十遍,出。飯後,復詣堂伴看寫字。講書畢,仍叩頭退。萬曆六年定,書堂設中座,書案在左,寫字案在右。輔臣率講讀侍書官候於門外。王入書堂,傳令旨「先生進」。輔臣率各官入,四拜,分班侍立。講讀官以次授書各十遍訖,令旨「先生吃酒飯」,各官出,王暫入堂南間少憩。輔臣各率官入。令旨「先生進」,遂入分班侍立。侍書官看寫字,講讀以次進講畢,各官一拜出。

巡狩東宮監國皇長孫監國頒詔儀迎接詔赦儀進書儀進表箋儀鄉飲酒禮蕃王朝貢禮遣使之蕃國儀蕃國遣使進表儀品官相見禮庶人相見禮

○巡狩之制

永樂六年北巡,禮部行直省,凡有重事及四夷來朝與進表者,俱達行在所,小事達京師啟聞。車駕將發,奏告天地、社稷、太廟、孝陵,祭大江、旗纛等神,AQ祭於承天門。緣途當祭者,遣官祭。將至北京,設壇祭北京山川等神。車駕至,奏告天地,祭境內山川。扈從馬步軍五萬。侍從,五府都督各一,吏、戶、兵、刑四部堂上官各一,禮、工二部堂上官各二,都察院堂上官一,御史二十四,給事中十九,通政、大理、太常、光祿、鴻臚堂上官共二十,翰林院、內閣官三,侍講、修撰、典籍等官六,六部郎官共五十四,餘不具載。車駕將發,宴群臣,賜扈從官及軍校鈔。至北京,宴群臣、耆老,賜百官及命婦鈔。所過郡縣,官吏、生員、耆老朝見,分遣廷臣核守令賢否,即加黜陟。給事、御史存問高年,賜幣帛酒肉。

嘉靖十八年,幸承天。先期親告上帝於玄極寶殿。同日,告皇祖及睿宗廟,遣官分告北郊及成祖以下諸廟、社稷、日月、神祇。駕出正陽門,后妃輦轎從。錦衣衛設欽製武陣駕,衛卒八千,奉輿輦,執儀仗。衛指揮前驅。武重臣二員留守,兵部尚書參贊機務,各賜敕行事。分命文武重臣,出督宣大、薊州、山海關,行九邊,亦各賜敕。皇城及京城諸門,皆命文武大臣各一員坐守。設警備扈駕官軍六千。駕發,百官吉服送於彰義關外。扈從官軍,略如永樂時數。先發在途者免朝參,惟禮兵二部、鴻臚、太常、科道糾儀官及光祿寺從行。過真定,望祭北嶽。帝常服,從臣大臣及巡撫都御史吉服行禮。衛輝,遣官祭濟瀆。鈞州,望祭中嶽;滎澤,祭河,禮如北嶽。南陽,遣祭武當山。途次古帝王、聖賢、忠臣、烈士祠墓,遣官致祭。撫、按、三司迎於境上,至行宮,吉服朝見。生員耆老,俱三十里外迎。所過王府,親王常服候駕,隨至行宮,冕服朝見。賜宴,宗室不許出。至承天,詣獻皇帝廟謁告。越四日,行告天禮於龍飛殿丹陛上,奉獻皇帝配。更皮弁服,詣國社稷及山川壇行禮,次日,謁顯陵。次日,從駕官上表賀,遂頒詔如儀。回京,親謝上帝、皇祖、皇考,分遣官告郊、廟、社稷、群神,行禮如初。

○東宮監國

古制,太子出曰「撫軍」,守曰「監國」。三代而下,惟唐太子監國結雙龍符,而其儀不著。

永樂七年,駕幸北京,定制,凡常朝,皇太子於午門左視事。左右侍衛及各官啟事如常儀。若御文華殿,承旨召入者方入。凡內外軍機及王府急務,悉奏請。有邊警,即調軍剿捕,仍馳奏行在。皇城及各門守衛,皆增置官軍。遇聖節、正旦、冬至,皇太子率百官於文華殿前拜表,行十二拜禮。表由中門出,皇太子由左門送至午門,還宮。百官導至長安右門外,文五品、武四品以上,及近侍官、監察御史,俱乘馬導三山門外,以表授進奏官。至期,告天祝壽,行八拜禮。其正旦、冬至、千秋節,百官於文華殿慶賀如常儀。凡享太廟及社稷諸神之祭,先期敕皇太子攝祭。其祀典神祇,太常寺於行在奏聞,遣官行禮。凡四夷來朝,循例賜宴,命禮部遣送行在所。凡詔書至,設龍亭儀仗大樂,百官朝服,出三山門外奉迎。皇太子冕服迎於午門前,至文華殿,行五拜三叩頭禮,升殿展讀。使者捧詔置龍亭中,皇太子送至午門外。禮部官置詔書雲輿中,文武二品以上官迎至承天門,開讀如儀。以鼓樂送使者詣會同館。使者見皇太子,行四拜禮,賜宴於禮部。

十二年北征,復定制。常朝於文華殿視事,文武啟事,俱達北京。嘉靖十八年南巡,命皇太子監國。時太子幼,命輔臣一人居守,軍國機務悉聽啟行。

○皇太孫監國

永樂八年,帝自北京北征。時皇太子已監國於南,乃命皇長孫居北京監國。時宣宗未冠,及冠始加稱皇太孫云。

其制,每日皇長孫於奉天門左視事,侍衛如常儀。諸司有事,具啟施行。若軍機及王府要務,一啟皇太子處分,一奏聞行在所。聖節,設香案於奉天殿,行禮如常儀。天下諸司表文俱詣北京。四夷朝貢俱送南京,武選及官民有犯,大者啟皇太子,小者皇長孫行之。皇親有犯,啟皇太子所。犯情重及謀逆者,即時拘執,命皇親會問。不服,乃命公、侯、伯、五府、六部、都察院、大理寺會皇親再問,啟皇太子,候車駕回京,奏請處分。

○頒詔儀

凡頒命四方,有詔書,有赦書、有敕符、丹符,有制諭、手詔。詔赦,先於闕廷宣讀,然後頒行。敕符等,則使者齎付所授官,秘不敢發。開讀迎接,儀各不同。

洪武二十六年,定頒詔儀。設御座於奉天殿,設寶案於殿東,陳中和韶樂於殿內,設大樂於午門及承天門外,設宣讀案於承天門上,西南向。清晨,校尉擎雲蓋於殿內簾前,百官朝服班承天門外,公侯班午門外,東西向。皇帝皮弁服,升殿如儀。禮部官捧詔書詣案前,用寶訖,置雲蓋中。校尉擎雲蓋,由殿東門出。大樂作,自東陛降,由奉天門至金水橋南午門外,樂作,公侯前導,迎至承天門上。鳴贊唱排班,文武官就位,樂作。四拜,樂止。宣讀展讀官升案,稱有制,眾官跪。禮部官捧詔書,授宣讀官。宣訖,禮部官捧置雲蓋中。贊禮唱俯伏興,樂作。四拜,樂止。舞蹈山呼,又四拜。儀禮司奏禮畢,駕興。禮部官捧詔書分授使者,百官退。

嘉靖六年續定,鴻臚官設詔案,錦衣衛設雲蓋盤於奉天殿內東,別設雲盤於承天門上。設彩輿於午門外,鴻臚官設宣讀案於承天門上。百官入丹墀侍立,帝冕服升座,如朝儀。翰林院官捧詔書從,至御座前東立。百官入班,四拜,出至承天門外。贊頒詔,翰林院官捧詔書授禮部官,捧至雲盤案上。校尉擎雲蓋,俱從殿左門出,至午門外,捧詔置彩輿內。公侯伯三品以上官前導,迎至承天門上,宣讀贊拜,俱如上儀。禮部官捧詔書授錦衣衛官,置雲匣中,以彩索繫之龍竿,頒降。禮部官捧置龍亭內,鼓樂迎至禮部,授使者頒行。隆慶六年,詔出至皇極門,即奏禮畢,駕還。

○迎接詔赦儀

洪武中定。凡遣使開讀詔赦,本處官具龍亭儀仗鼓樂,出郭迎。使者下馬,奉詔書置龍亭中,南向,本處官朝服行五拜禮。眾官及鼓樂前導,使者上馬隨龍亭後,至公廨門。眾官先入,文武東西序立,候龍亭至,排班四拜。使者捧詔授展讀官,展讀官跪受,詣開讀案。宣讀訖,捧詔授朝使,仍置龍亭中。眾官四拜,舞蹈山呼,復四拜畢。班首詣龍亭前,跪問皇躬萬福,使者鞠躬答曰:「聖躬萬福。」眾官退,易服見使者,並行兩拜禮。復具鼓樂送詔於官亭。如有出使官在,則先守臣行禮。

○進書儀

進書儀惟《實錄》最重。皇帝具袞冕,百官朝服,進表稱賀。其餘纂修書成,則以表進。重錄書及玉牒,止捧進。茲詳載進《實錄》儀,餘可推見云。

建文時,《太祖實錄》成,其進儀無考。永樂元年,重修《太祖實錄》成。設香案於奉天殿丹陛正中,表案於丹陛之東,設寶輿於奉天門,設鹵簿大樂如儀。史官捧《實錄》置輿中,帝御殿如大朝儀。百官詣丹墀左右立,鴻臚官引寶輿至丹陛上,史官舉《實錄》置於案,遂入班。鴻臚官奏進《實錄》,序班舉《實錄》案,以次由殿中門入,班首由左門入。帝興,序班以《實錄》案置於殿中。班首跪於案前,贊史官皆跪。序班并內侍官舉《實錄》案入謹身殿,置於中。帝復座。贊俯伏,班首俯伏,興。復位,贊四拜。贊進表,序班舉表案,由左門入,置於殿中。贊宣表,贊眾官皆跪。宣訖,俯伏,興,四拜,進《實錄》官退於東班,百官入班。鴻臚官奏慶賀,各官四拜興。贊有制,史官仍入班。贊跪,宣制云:「太祖高皇帝、高皇后功德光華,纂述詳實。朕心懽慶,與卿等同之。」宣訖,俯伏興,三舞蹈,又四拜,禮畢。

萬歷五年,《世祖實錄》成,續定進儀。設寶輿、香亭、表亭於史館前,帝袞冕御中極殿,百官朝服侍班。監修、總裁、纂修等官,朝服至館前。監修官捧表置表亭中,纂修官捧《實錄》置寶輿中,鴻臚官導迎。用鼓樂傘蓋,由會極門下階,至橋南,由中道行。監修、總裁等官隨表亭後,由二橋行至皇極門。《實錄》輿由中門入,表亭由左門入,至丹墀案前。監修官捧表置於案,纂修官捧《實錄》置於案,俱侍立於石墀東。內殿百官行記訖,帝出御皇極殿。監修、總裁等官入,進《實錄》、進表俱如永樂儀。次日,司禮監官自內殿送《實錄》下殿,仍置寶輿中,用傘蓋,與監修總裁官同送皇史宬尊藏。

○進表箋儀

明初定制,凡王府遇聖節及冬至、正旦,先期陳設畢。王冕服就位四拜,詣香案前跪。進表訖,復位,四拜,三舞蹈,山呼,又四拜。百官朝服隨班行禮。進中宮箋儀如之,惟不舞蹈山呼。進皇太子箋,王皮弁服,行八拜禮,百官朝服隨班行禮。

凡進賀表箋,皇子封王者,於天子前自稱曰「第幾子某王某」,稱天子曰「父皇陛下」,皇后曰「母后殿下」。若孫,則自稱曰「第幾孫某王某」,稱天子曰「祖父皇帝陛下,」皇后曰「祖母皇后殿下」。若弟,則自稱曰「第幾弟某封某」,稱天子曰「大兄皇帝陛下」,皇后曰「尊嫂皇后殿下」。姪則自稱曰「第幾姪某封某」,稱天子曰「伯父皇帝陛下」,「叔父皇帝陛下」,皇后曰「伯母皇后殿下」,「叔母皇后殿下」。若尊屬,則自稱曰「某封臣某」,稱天子曰「皇帝陛下」,皇后曰「皇后殿下」。若從孫以下,則稱「從孫、再從孫、三從孫某封某」,皆稱皇帝皇后曰「伯祖、叔祖皇帝陛下」,「伯祖母、叔祖母皇后殿下」。至世宗時,始令各王府表箋,俱用聖號,不得用家人禮。

凡在外百官進賀表箋,前一日,結彩於公廨及街衢。文武官各齋沐,宿本署。清晨,設龍亭於庭中,設儀仗鼓樂於露臺,設表箋案於龍亭前,香案於表箋案前,設進表箋官位於龍亭東。鼓初嚴,各官具服。次嚴,班首具服詣香案前,滌印用印訖,以表箋置於案,退立幕次。三嚴,各官入班四拜,班首詣香案前。贊跪,眾官皆跪。執事者以表箋跪授班首,班首跪授進表官,進表官跪受,置龍亭中。班首復位,各官皆四拜,三舞蹈,山呼,四拜。金鼓儀仗鼓樂百官前導,進表官在龍亭後東。至郊外,置龍亭南向,儀仗鼓樂陳列如前,文武官侍立。班首取表箋授進表官,進表官就於馬上受表,即行,百官退。

○鄉飲酒禮

《記》曰:「鄉飲酒之禮廢,則爭鬥之獄繁矣。」故《儀禮》所記,惟鄉飲之禮達於庶民。自周迄明,損益代殊,而其禮不廢。洪武五年,詔禮部奏定鄉飲禮儀,命有司與學官率士大夫之老者,行於學校,民間里社亦行之。十六年,詔班《鄉飲酒禮圖式》於天下,每歲正月十五日、十月初一日,於儒學行之。

其儀,以府州縣長吏為主,以鄉之致仕官有德行者一人為賓。擇年高有德者為僎賓,其次為介,又其次為三賓,又其次為眾賓,教職為司正。贊禮、贊引、讀律,皆使能者。前期,設賓席於堂北兩楹之間,少西,南面;主席於阼階上,西面;介席於西階上,東面;僎席於賓東,南面;三賓席於賓西,南面。皆專席不屬。眾賓六十以上者,席於西序,東面北上。賓多則設席於西階,北面東上;僚佐席於東序,西面北上。設眾賓五十以下者位於堂下西階之西,當序,東面北上。賓多則又設位於西階之南,北面東上。司正及讀律者,位於堂下阼階之南,北面西上。設主之贊者位於阼階之東,西面北上。設主及僚佐以下次於東廊,賓介及眾賓次於庠門之外,僎次亦在門外。設酒尊於堂上東南隅,加勺冪,用葛巾;爵洗於阼階下東南;篚一於洗西,實以爵觶;盥洗在爵洗東。設卓案於堂上下席位前,陳豆於其上。六十者三豆,七十者四豆,八十者五豆,九十者六豆,堂下者二豆。主人豆如賓之數,皆實以菹醢。至期,賓將及門,執事者進報曰:「賓至。」主人率僚屬出迎於門外,主西面,賓以下皆東面。三揖三讓,而後升堂,相向再拜,升坐。執事者報僎至,迎坐如前儀。贊禮唱司正揚觶。司正詣盥洗位,次詣爵洗位,取觶於篚,洗觶。升自西階,詣尊所酌酒,進兩楹之間,北面立。在坐者皆起,司正揖,僎賓以下皆報揖。司正乃舉觶,言曰:「恭惟朝廷,率由舊章。敦崇禮教,舉行鄉飲,非為飲食。凡我長幼,各相勸勉。為臣竭忠,為子盡孝,長幼有序,兄友弟恭。內睦宗族,外和鄉里,無或廢墜,以忝所生。」言畢,贊禮唱司正飲酒。飲畢,揖報如初。司正復位,僎賓以下皆坐。贊禮唱讀律令,執事舉律令案於堂之中。讀律令者詣案前,北向立讀,皆如揚觶儀。有過之人俱赴正席立聽,讀畢復位。贊禮唱供饌,執事者舉饌案至賓前,次僎,次介,次主,三賓以下各以次舉訖。贊禮唱獻賓,主降詣盥洗及爵洗位,洗爵酌酒,至賓前,置於席。稍退,兩拜,賓答拜。又詣僎前,亦如之。主退復位。贊禮唱賓酬酒,賓起,僎從之,詣盥洗爵洗位如儀。至主前,置爵。賓、僎、主皆再拜,各就坐。執事者於介、三賓、眾賓以下,以次斟酒訖。贊禮唱飲酒,或三行,或五行。供湯三品畢。贊禮唱徹饌,在坐者皆興。僎、主、僚屬居東,賓、介、三賓、眾賓居西,皆再拜。贊禮唱送賓,以次下堂,分東西行,仍三揖出庠門而退。里中鄉飲略同。

二十二年,命凡有過犯之人列於外坐,同類者成席,不許雜於善良之中,著為令。

三曰賓禮,以待蕃國之君長與其使者。宋政和間,詳定五禮,取《周官·司儀》掌九儀賓客擯相,詔王南鄉以朝諸侯之義,故以朝會儀列為賓禮。按古之諸侯,各君其國,子其民,待以客禮可也,不可與後世之臣下等。茲改從其舊,而百官庶人相見之禮附焉。

○蕃王朝貢禮

蕃王入朝,其迎勞宴饗之禮,惟唐制為詳。宋時,蕃國皆遣使入貢,所接見惟使臣而已。

明洪武二年定制:凡蕃王至龍江驛,遣侍儀、通贊二人接伴。館人陳蕃王座於廳西北,東向。應天府知府出迎,設座於廳東南,西向。以賓主接見。宴畢,知府還,蕃王送於門外。明日,接伴官送蕃王入會同館,禮部尚書即館宴勞。尚書至,蕃王服其國服相見。宴饗迎送俱如龍江驛。酒行,用樂。明日,中書省奏聞,命官一員詣館,如前宴勞。侍儀司以蕃王及從官具服,於天界寺習儀三日,擇日朝見。設蕃王及從官次於午門外,蕃王拜位於丹墀中道,稍西,從官在其後。設方物案於丹墀中道東西。知班二,位於蕃王拜位北,引蕃王舍人二,位於蕃王北,引蕃王從官舍人二,位於蕃王從官北,俱東西相向。鼓三嚴,百官入侍。執事舉方物案,蕃王隨案由西門入,至殿前丹墀西,俟立。皇帝服通天冠、絳紗袍御殿。蕃王及從官各就拜位,以主物案置拜位前。贊四拜訖,引班導蕃王升殿。宣方物官以方物狀由西陛升,入殿西門,內贊引至御前。贊拜,蕃王再拜,跪,稱賀致詞。宣方物官宣狀。承制官宣制訖,蕃王俯伏,興,再拜,出殿西門,復位。贊拜,蕃王及其從官皆四拜。禮畢,皇帝興,蕃王以下出。樂作樂止皆如常。見皇太子於東宮正殿,設拜位於殿外。皇太子皮弁服升座,蕃王再拜,皇太子立受。蕃王跪稱賀,致詞訖,復位再拜,皇太子答拜。蕃王出,其從官行四拜禮。見親王,東西相向再拜,王答拜。俱就座,王座稍北。禮畢,揖而出。見丞相、三公、大都督、御史大夫皆鈞禮。蕃王陛辭,如朝見儀,不傳制。中書省率禮部官送龍江驛,宴如初。

二十七年四月,以舊儀煩,命更定。凡蕃國來朝,先遣禮部官勞於會同館。明日,各具其國服,如嘗賜朝服者則服朝服,於奉天殿朝見。行八拜禮畢,即詣文華殿朝皇太子,行四拜禮。見親王亦如之,王立受,答後二拜。從官隨蕃王後行禮。凡遇宴會,蕃王居侯伯之下。

凡蕃國遣使朝貢,至驛,遣應天府同知禮待。明日至會同館,中書省奏聞,命禮部侍郎於館中禮待如儀。宴畢,習儀三日,擇日朝見。陳設儀仗及進表,俱如儀。承制官詣使者前,稱有制。使者跪,宣制曰:「皇帝問使者來時,爾國王安否?」使答畢,俯伏,興,再拜。承制官稱有後制,使者跪。宣制曰:「皇帝又問,爾使者遠來勤勞。」使者俯伏,興,再拜。承制官復命訖,使者復四拜。禮畢,皇帝興,樂作止如儀。見東宮四拜,進方物訖,復四拜。謁丞相、大都督、御史大夫,再拜。獻書,復再拜。見左司郎中等,皆鈞禮。

凡錫宴,陳御座於謹身殿。設皇太子座於御座東,諸王座於皇太子下,西向,設蕃王座於殿西第一行,東向,設文武官座於第二、第三行,東西向。酒九行,上食五次,大樂、細樂間作,呈舞隊。蕃國從官坐於西廡下,酒數食品同,不作樂。東宮宴蕃王,殿上正中設皇太子座,設諸王座於旁,東西向;蕃王座於西偏,諸王之下,東向;三師、賓客、諭德位於殿上第二行,東西向;蕃王從官及東宮官位於西廡,東向北上。和聲郎陳樂,光祿寺設酒饌,俱如謹身殿儀。或宰相請旨宴勞,則設席於中書省後堂,賓西主東。設蕃王從官及左右司官坐於左司。教坊司陳樂於堂及左司南楹。蕃王至省門外,省官迎入,從官各從其後。升階就坐,酒七行,食五品,作樂,雜陳諸戲。宴畢,省官送至門外。都督府御史臺宴如之。其宴蕃使,禮部奉旨錫宴於會同館。館人設坐次及御酒案,教坊司設樂舞,禮部官陳龍亭於午門外。光祿寺官請旨取御酒,置龍亭,儀仗鼓樂前導。至館,蕃使出迎於門外。執事者捧酒由中道入,置酒於案。奉旨官立於案東,稱有制,使者望闕跪。聽宣畢,贊再拜。奉旨官酌酒授使者,北面跪飲畢,又再拜。各就坐,酒七行,湯五品,作樂陳戲如儀。宴畢,奉旨官出,使者送至門外。皇太子錫宴,則遣宮官禮待之。省府臺亦置酒宴會,酒五行,食五品,作樂,不陳戲。

○遣使之蕃國儀

凡遣使、賜璽綬及問遣慶弔,自漢始。唐使外國,謂之入蕃使,宋謂之國信使。明祖既定天下,分遣使者奉詔書往諭諸國,或降香幣以祀其國之山川。撫柔之意甚厚,而不傷國體,視前代為得。

,凡遣使,翰林院官草詔。至期,陳設如常儀。百官入侍,皇帝御奉天殿。禮部官捧詔書,尚寶司奏用寶,以黃銷金袱裹置盤中,置於案。使者就拜位四拜,樂作止如儀。承制官至丹陛稱有制,使者跪。宣制曰:「皇帝敕使爾某詔諭某國,爾宜恭承朕命。」宣訖,使者俯伏,興,四拜。禮部官奉詔降自中陛,以授使者。使者拜出午門,置龍亭內。駕興,百官出。

使者入蕃國境,先遣人報於王,王遣使遠接。前期於國門外公館設幄結綵,陳龍亭香案,備金鼓儀仗大樂。又於城內街巷結彩,設闕亭於王殿上,設香案於其前。設捧詔官位殿陛之東北,宣詔展詔官以次南,俱西向。詔使至,迎入館。王率國中官及耆老出迎於國門外,行五拜禮。儀仗鼓樂導龍亭入,使者隨之。至殿上,置龍亭於正中。使者立香案東,蕃王位殿庭中北向,眾官隨之。使者南向立,稱有制,蕃王以下皆四拜。蕃王升自西階,詣香案前跪。三上香,俯伏,興,眾官同。蕃王復位。使者詣龍亭前,取詔書授捧詔官。捧詔官捧詣開讀案,授宣詔官。宣詔官受詔,展詔官對展,蕃王以下皆跪聽。宣訖,仍以詔置龍亭。蕃王以下皆俯伏,興,四拜,三舞蹈,復四拜。凡拜皆作樂。禮畢,使者以詔書付所司頒行。蕃王與使者分賓主行禮。

其賜蕃王印綬及禮物,宣制曰:「皇帝敕使爾某,授某國王印綬,爾其恭承朕命。」至蕃國,宣制曰:「皇帝敕使某,持印賜爾國王某,并賜禮物。」餘如儀。

○蕃國遣使進表儀

洪武二年定。所司於王宮及國城街巷結彩,設闕庭於殿上正中。前設表箋案,又前設香案。使者位於香案東,捧表箋二人於香案西。設龍亭於殿庭南正中,儀仗鼓樂具備。清晨,司印者陳印案於殿中,滌印訖,以表箋及印俱置於案。王冕服,眾官朝服。詣案前用印畢,用黃袱裹表,紅袱裹箋,各置於匣中,仍各以黃袱裹之。捧表箋官捧置於案。引禮引王至殿庭正中,眾官位其後。贊拜,樂作。再拜,樂止。王詣香案前跪,眾官皆跪,三上香訖。捧表官取表東向跪進王,王授表以進於使者。使者西向跪受,興,置於案。贊興,王復位。贊拜,樂作,王與眾官皆四拜。樂止,禮畢。捧表箋官捧表前行。置於龍亭中,金鼓儀仗鼓樂前導。王送至宮門外,還;眾官朝服送至國門外。使者乃行。

○品官相見禮

凡官員揖拜,洪武二十年定,公、侯、駙馬相見,各行兩拜禮。一品官見公、侯、駙馬,一品官居右,行兩拜禮,公、侯、駙馬居左,答禮。二品見一品亦如之。三品以下仿此。若三呂見一品,四品見二品,行兩拜禮。一品二品答受從宜,餘品仿此。如有親戚尊卑之分,從行私禮。三十年令,凡百官以品秩高下分尊卑。品近者行禮,則東西對立,卑者西,高者東。其品越二、三等者,卑者下,尊者上。其越四等者,則卑者拜下,尊者坐受,有事則跪白。

凡文武官公聚,各依品級序坐。若資品同者,照衙門次第。若王府官與朝官坐立,各照品級,俱在朝官之次。成化十四年定,在外總兵、巡撫官位次,左右都督與左右都御史並,都督同知與副都御史並,都督僉事與僉都御史並,俱文東武西。伯以上則坐於左。十五年重定,都御史係總督及提督軍務者,不分左右副僉,俱坐於左。總兵官雖伯,亦坐於右。

凡官員相遇迴避,洪武三十年定,駙馬遇公侯,分路而行。一品、二品遇公、侯、駙馬,引馬側立,須其過。二品見一品,趨右讓道而行。三品遇公、侯、駙馬,引馬迴避,遇一品引馬側立,遇二品趨右讓道而行。四品遇一品以上官,引馬迴避,遇二品引馬側立,遇三品趨右讓道而行。五品至九品,皆視此遞差。其後盡遵行。文職雖一命以上,不避公、侯、勛戚大臣;而其相迴避者,亦論官不論品秩矣。

凡屬官見上司,洪武二十年定,屬官序立於堂階之上,總行一揖,上司拱手,首領官答揖。其公幹節序見上司官,皆行兩拜禮,長官拱手,首領官答禮。

凡官員公座,洪武二十年定,大小衙門官員,每日公座行肅揖禮。佐貳官揖長官,長官答禮。首領官揖長官、佐貳官,長官、佐貳官拱手。

○庶人相見禮

洪武五年令,凡鄉黨序齒,民間士農工商人等平居相見及歲時宴會謁拜之禮,幼者先施。坐次之列,長者居上。十二年令,內外官致仕居鄉,惟於宗族及外祖妻家序尊卑,如家人禮。若筵宴,則設別席,不許坐於無官者之下。與同致仕官會,則序爵;爵同,序齒。其與異姓無官者相見,不須答禮。庶民則以官禮謁見。凌侮者論如律。二十六年定,凡民間子孫弟侄甥婿見尊長,生徒見其師,奴婢見家長,久別行四拜禮,近別行揖禮。其餘親戚長幼悉依等第,久別行兩拜禮,近別行揖禮。平交同。

親征遣將禡祭受降奏凱獻俘論功行賞大閱大射救日伐鼓

四曰軍禮。親征為首,遣將次之。方出師,有禡祭之禮。及還,有受降、奏凱獻俘、論功行賞之禮。平居有閱武、大射之禮。而救日伐鼓之制,亦以類附焉。

○親征

洪武元年閏七月,詔定軍禮。中書省臣會儒臣言:古者天子親征,所以順天應人,除殘去暴,以安天下。自黃帝習用干戈以征不享,此其始也。周制,天子親征,則類於上帝,宜於大社,造於祖廟,禡於所征之地,及祭所過山川。師還,則奏凱獻俘於廟社。後魏有宣露布之制。唐仍舊典,宋亦間行焉。於是歷考舊章,定為親征禮奏之。前期,擇日祭告天地神祠行禡祭禮。凡所過山川嶽鎮海瀆用太牢,其次少牢,又次特牲。若行速,止用酒脯,祭器籩豆各一。前期,齋一日。皇帝服通天冠、絳紗袍,省牲視滌。祭之日,服武弁,行一獻禮。凱旋,告祭宗社,禮與出師同。獻俘廟社,以露布詔天下,然後論功行賞。永樂、宣德、正統間,率遵用之。

正德十四年,帝親征宸濠,禮部上祭告儀注如舊。帝令祭祀俱遣官代。及疏請遣官,有旨勿遣。其頒詔,亦如舊制。明年十一月將凱旋,禮臣言:「宸濠悖逆,皇上親統六師,往正其罪,與宣德間親征漢庶人高煦故事相同。但一切禮儀無從稽考。請於師還之日,聖駕從正陽門入,遣官告謝天地廟社。駕詣奉先殿、几筵殿,謁見畢,朝見皇太后。次日早,御午門樓,百官朝見,行獻俘禮。擇日詔告天下。」十二月,帝還京,百官迎於正陽門外,帝戎服乘馬入。

○遣將

洪武元年,中書省臣會官議奏,王者遣將,所以討有罪,除民害也。《書》稱大禹徂征,《詩》美南仲薄伐。《史記》引《兵書》曰:「古王者之遣將,跪而推轂。」漢高命韓信為將,設壇具禮。北齊親授斧鉞。唐則告於廟社,又告太公廟。宋則授旌節於朝堂,次告廟社,又禡祭黃帝。今定遣將禮,皇帝武弁服,御奉天殿。大將軍入就丹墀,四拜,由西陛入殿,再拜跪。承制官宣制,以節鉞授大將軍。大將軍受之,以授執事者,俯伏,興,再拜出。降陛,復位,四拜。駕還宮,大將軍出。至午門外勒所部將士,建旗幟,鳴金鼓,正行列,擎節鉞。奏樂前導,百官以次送出。造廟宜社之禮,即命大將軍具牲幣,行一獻禮,與遣官祭告廟社儀同。其告武成王廟儀,前二日,大將省牲。祭日,大將於幕次僉祝版,入就位,再拜。詣神位前上香、奠帛、再拜。進熟酌獻,讀祝,再拜。詣位,再拜。飲福受胙,復再拜。徹豆,望燎。其配位,亦大將行禮。兩廡陪祀,諸將分獻。

○禡祭

親征前期,皇帝及大將陪祭官皆齋一日。前一日,皇帝服通天冠、絳紗袍省牲,詣神廚,視鼎鑊滌溉。執事設軍牙六纛於廟中之北,軍牙東,六纛西,籩豆十二,簠簋各二,鉶登俎各三。設瘞坎位於神位西北,設席於坎前。上置酒碗五,雄雞五,餘陳設如常儀。祭日,建牙旗六纛於神位後。皇帝服武弁,自左南門入。至廟庭南,正中北向立。大將及陪祭官分文武重行班於後。迎神,再拜,奠幣。行初獻禮,先詣軍牙神位前,再詣六纛神位前,俱再拜。亞獻、終獻如之。惟初獻讀祝,詣飲福位,再拜飲福,受胙,又再拜。掌祭官徹豆,贊禮唱送神,復再拜。執事官各以祝幣,掌祭官取饌詣燎所,太常奏請望燎。執事殺雞,刺血於酒碗中,酹神。燎半,奏禮畢,駕還。若遣將,則於旗纛廟壇行三獻禮。大將初獻,諸將亞獻、終獻。

○受降

洪武四年七月,蜀夏明昇降表至京師,太祖命中書集議受降禮。省部請如宋太祖受蜀主孟昶降故事,擬明昇朝見日,皇帝御奉天門,昇等於午門外跪進待罪表。侍儀使捧表入,宣表官宣讀訖,承制官出傳制。升等皆俯伏於地,侍儀舍人掖昇起,其屬官皆起,跪聽宣制釋罪。昇等五拜,三呼萬歲。承制官傳制,賜衣服冠帶。侍儀舍人引昇入丹墀中四拜。侍儀使傳旨,升跪聽宣諭,俯伏四拜,三呼萬歲,又四拜出。百官行賀禮。帝以昶專治國政,所為奢縱,升年幼,事由臣下,免其叩頭伏地上表請罪禮,惟命昇及其官屬朝見,百官朝賀。

○奏凱獻俘

凡親征,師還,皇帝率諸將陳凱樂俘馘於廟南門外,社北門外。告祭廟社,行三獻禮,同出師儀。祭畢,以俘馘付刑部,協律郎導樂以退。皇帝服通天冠、絳紗袍,升午門樓,以露布詔天下,百官具朝服以聽,儀與開讀詔赦同。

大將奏凱儀。先期,大都督以露布聞。內使監陳御座於午門樓上前楹,設奏凱樂位於樓前,協律郎位於奏凱樂北,司樂位於協律郎南。又設獻俘位於樓前少南,獻俘將校位於其北,刑部尚書奏位於將校北,皆北向。又設刑部尚書受俘位於獻俘位西,東向。設露布案於內道正中,南向。受露布位於案東,承制位於案東北,俱西向。宣露布位於文武班南,北向。至日清晨,先陳凱樂俘馘於廟社門外,不奏歌曲。俟告祭禮畢,復陳樂於午門樓前,將校引俘侍立於兵仗之外,百官入侍立位。皇帝常服升樓,侍衛如常儀。大將於樓前就位,四拜。諸將隨之,退,就侍立位。贊奏凱樂,協律郎執麾引樂工就位,司樂跪請奏凱樂。協律郎舉麾,鼓吹振作,編奏樂曲。樂止,贊宣露布。承制官以露布付受露布官,引禮引詣案跪受,由中道南行,以授宣露布官。宣訖,付中書省頒示天下。將校引俘至位,刑部尚書跪奏曰:「某官某以某處所俘獻,請付所司。」奏訖,退復位。其就刑者立於西廂,東向,以付刑官。其宥罪者,樓上承制官宣旨,有敕釋縛。樓下承旨,釋訖,贊禮贊所釋之俘謝恩,皆四拜三呼,將校以所釋俘退。如有所賜,就宣旨賜之。大將以下就拜位,舞蹈山呼如常儀。班前稍前跪,稱賀致詞訖,百官復四拜,禮畢還宮。

洪武三年六月,左副將軍李文忠北征大捷,遣官送所俘元孫買的里八剌及寶冊至京師。百官請行獻俘禮。帝不許,事詳《本紀》。止令服本俗服,朝見畢,賜中國衣冠就謝。復謂省臣曰:「故國之妃朝於君者,元有此禮,不必效之。」亦令衣本俗服,入見中宮,賜中國服就謝。十一月,大將軍徐達及文忠等師還,車駕出勞於江上。明日,達率諸將上《平沙漠表》。帝御奉天殿,皇太子親王侍,百官朝服陪列,達、文忠奉表賀。禮成,退自西階。皇太子親王入賀。後定,凡大捷,擇日以宣,其日不奏事,百官吉服賀,即日遣官薦告郊廟。中捷以下,止宣捷,不祭告慶賀。

永樂四年定,凡捷,兵部官以露布奏聞,大將在軍則進露布官行禮,次日行開讀禮,第三日行慶賀禮,餘如前儀。武宗征宸濠還,禮部上獻俘儀,值帝弗豫,不果行。嘉靖二十三年十月,叛賊王三屢導吉囊入犯大同,官軍計擒之。遣官謝南北郊、景神殿、太社稷。擇日獻俘,百官表賀。天啟二年,四川獻逆犯樊友邦等,山東獻逆犯徐鴻儒等,俱遣官告祭郊廟,御樓獻俘。

○論功行賞

凡凱還,中書省移文大都督府,兵部具諸將功績,吏部具勳爵職名,戶、禮二部具賞格。中書集六部論定功賞,奏取上裁。前期,陳御座香案於奉天殿,設寶案詔書案於殿中,誥命案於丹陛正中之北,宣制案於誥命案之北。吏、戶、禮三部尚書位於殿上東南,大都督、兵部尚書位於殿上西南,應受賞官拜位於丹墀中,序立位於丹墀西南,受賞位於誥命案之南,受賞執事位於受賞官序立位之西。每官用捧誥命、捧禮物各一人,俱北向。餘陳設如朝儀。是日,鼓三嚴,執事官各就位。皇帝袞冕升座,皇太子諸王袞冕,自殿東門入侍立,受賞官入就拜位,四拜。承制官跪承制,由殿中門出,吏、戶、禮尚書由殿西門出,立於誥命案東。承制官南向稱有制,受賞官皆跪,宣制曰:「朕嘉某等為國建功,宜加爵賞。今授以某職,賜以某物,其恭承朕命。」宣畢,受賞官俯伏,興,再拜。唱行賞,受賞官第一人詣案前跪,吏部尚書捧誥命,戶部尚書捧禮物,各授受賞官。受賞官以授左右,俯伏,興,復位。餘官以次受賞訖,承制官、吏部尚書等俱至御前復命,退復位。受賞官皆再拜,三舞蹈,山呼。俯伏,興,復四拜。禮畢,皇帝還宮。各官出,至午門外,以誥命禮物置於龍亭,用儀仗鼓樂各送還本第。明日進表稱謝,如常儀。

○大閱

宣德四年十月,帝將閱武郊外,命都督府整兵,文武各堂上官一員、屬官一員扈從。正統間,或閱於近郊,於西苑,不著令。隆慶二年,大學士張居正言:「祖宗時有大閱禮,乞親臨校閱。」兵部引宣宗、英宗故事,請行之。命於明年八月舉行。及期,禮部定儀。

前期一日,皇帝常服告於內殿,行四拜禮,如出郊儀。司設監設御幄於將臺上,總協戎政大臣、巡視科道督率將領軍兵預肅教場。至日早,遣官於教場祭旗纛之神。三大營官軍具甲仗,將官四員統馬兵二千扈駕。文臣各堂上官,科道掌印官、禮兵二科、禮部儀制司、兵部四司官、糾儀監射御史、鴻臚寺供事官,武臣都督以上、錦衣衛堂上及南鎮撫司掌印僉書官,俱大紅便服,關領扈從,牙牌懸帶,先詣教場。是日免朝。錦衣衛備鹵簿。皇帝常服乘輦由長安左門出,官軍導從,鉦鼓振作。出安定門,至閱武門外。總協戎政官率大小將佐戎服跪迎,入將臺下,北向序立。駕進閱武門,內中軍舉號砲三,各營鉦鼓振作,扈從官序立於行宮門外。駕至門,降輦。兵部官導入行宮,鳴金止鼓,候升座。扈從官行一拜禮,傳賜酒飯。各官謝恩出,將臺下東西序立。兵部官奏請大閱。兵部、鴻臚寺官導駕登臺,舉炮三。京營將士叩頭畢,東西侍立。總協戎政官列於扈從官之北,諸將列從官之南。兵部尚書奏請,令各營整搠人馬。臺上吹號笛,麾黃旗,總協戎政及將佐等官各歸所部。兵部尚書請閱陣,舉砲三。馬步官軍演陣,如常法。演畢,復吹號笛,麾黃旗,將士俱回營。少頃,兵部尚書請閱射。總協戎政官以下及聽射公、侯、駙馬、伯、錦衣衛等官,俱於臺下較射。馬三矢,步六矢,中的者鳴鼓以報,御史、兵部官監視紀錄。把總以下及家丁、軍士射,以府部大臣并御史、兵部官於東西廳較閱。鎗刀火器等藝,聽總協戎政官量取一隊,於御前呈驗。兵部尚書奏大閱畢,臺下舉號旗。總協戎政官及諸將領俱詣臺下,北向序立。鴻臚寺官奏傳制,贊跪。宣制訖,贊叩頭。各官先退,出門外,贊扈從官行叩頭禮。禮畢,駕回行宮,少憩,扈從等官趨至門內立。皇帝升輦。中軍舉炮三,各營皆鼓吹,鹵簿及馬兵導從如來儀,鉦鼓與大樂相應振作。總協戎政以下候駕至,叩頭退。馬兵至長安左門外止。鹵簿、大樂至午門外止。駕還,仍詣內殿參謁,如前儀。百官不扈從者,各吉服於承天門外橋南序立恭送,駕還,迎如之。次日,總協戎政官以下表謝,百官侍班行稱賀禮,如常儀。兵部以將士優劣及中箭多寡、教練等第奏聞。越二日,皇帝御皇極門,賜敕勉勵將士。總協戎政官捧至彩輿,將士迎導至教場,開讀行禮如儀。是日,即行賞賚並戒罰有差。次日,總協戎政官率將佐復謝恩。

詔如議行。駕還,樂奏《武成之曲》。

萬曆九年大閱,如隆慶故事。

○大射

大射之禮,後世莫講,惟《宋史》列於嘉禮。至《明集禮》則附軍禮中,《會典》亦然。

其制洪武三年定。凡郊廟祭祀,先期行大射禮,工部製射侯等器。其射鵠有七。虎鵠五采,天子用之。熊鵠五採,皇太子用之。豹鵠五採,親王用之。豹鵠四採,文武一品、二品者用之。糝鵠三采,三品至五品用之。狐鵠二采,六品至九品用之。布鵠無采,文武官子弟及士民俊秀用之。凡射時,置乏於鵠右。乏又名容,見《周禮·大司馬》服不氏,職執旗及待獲者以蔽身。設楅及韋,當射時置於前,以齊矢。設射中五。皮樹中,天子大射用之。閭中,天子宴射用之。虎中,皇太子親王射用之。兕中,一品至五品文武官用之。鹿中,六品至九品及文武官子弟士民俊秀通用之。其職事,設司正官二,掌驗射者品級尊卑人力強弱而定耦,其中否則書於算,兵部官職之。司射二,掌先以強弓射鵠誘射,以鼓眾氣,武職官充之。司射器官二,掌辨弓力強弱,分為三等,驗人力強弱以授,工部官職之。舉爵者,掌以馬湩授中者飲,光祿寺官職之。請射者,掌定耦射。射畢,再請某耦射,侍儀司職之。待獲者、掌矢納於司射器者,以隸僕供其役。執旗者六人,掌於容後執五色旗。如射者中的,舉紅旗應之。中采,舉采旗應之。偏西,舉白旗。偏東,舉青旗。過於鵠舉黃旗。不及鵠,舉黑旗。軍士二人掌之。引禮二,掌引文武官進退,侍儀司舍人職之。

太祖又以先王射禮久廢,弧矢之事專習於武夫,而文士多未解,乃詔國學及郡縣生員皆令習射,頒儀式於天下。朔望則於公廨或閒地習之。其官府學校射儀,略仿大射之式而殺其禮。射位初三十步,自後累加至九十步。射四矢,以二人為耦。

永樂時有擊球射柳之制。十一年五月五日幸東苑,擊球射柳,聽文武群臣四夷朝使及在京耆老聚觀。分擊球官為兩朋,自皇太孫而下諸王大臣以次擊射,賜中者幣布有差。

○救日伐鼓

洪武六年二月,定救日食禮。其日,皇帝常服,不御正殿。中書省設香案,百官朝服行禮。鼓人伐鼓,復圓乃止。月食,大都督府設香案,百官常服行禮,不伐鼓,雨雪雲翳則免。

二十六年三月更定,禮部設香案於露臺,向日,設金鼓於儀門內,設樂於露臺下,各官拜位於露臺上。至期,百官朝服入班,樂作,四拜興,樂止,跪。執事者捧鼓,班首擊鼓三聲,眾鼓齊鳴,候復圓,復行四拜禮。月食,則百官便服於都督府救護如儀。在外諸司,日食則於布政使司、府州縣,月食則於都指揮使司、衛所,如儀。

隆慶六年,大喪。方成服,遇日食。百官先哭臨,後赴禮部,青素衣、黑角帶,向日四拜,不用鼓樂。


○山陵

次五曰凶禮。凡山陵、寢廟與喪葬、服紀及士庶喪制,皆以類編次。其謁陵、忌辰之禮,亦附載焉。

○山陵

太祖即位,追上四世帝號。皇祖考熙祖,墓在鳳陽府泗州蠙城北,薦號曰祖陵。設祠祭署,置奉祀一員,陵戶二百九十三。皇考仁祖,墓在鳳陽府太平鄉。太祖至濠,嘗議改葬,不果。因增土以培其封,令陵旁故人汪文、劉英等二十家守視。洪武二年薦號曰英陵,後改稱皇陵。設皇陵衛并祠祭署,奉祀一員、祀丞三員,俱勳舊世襲。陵戶三千三百四十二,直宿灑掃。禮生二十四人。四年,建祖陵廟。仿唐、宋同堂異室之制,前殿寢殿俱十五楹,東西旁各二,為夾室,如晉王肅所議。中三楹通為一室,奉德祖神位,以備袷祭。東一楹奉懿祖,西一楹奉熙祖。十九年,命皇太子往泗州修繕祖陵,葬三祖帝后冠服。

三十一年,太祖崩。禮部定議,京官聞喪次日,素服、烏紗帽、黑角帶,赴內府聽遺詔。於本署齋宿,朝晡詣几筵哭。越三日成服,朝晡哭臨,至葬乃止。自成服日始,二十七日除。命婦孝服,去首飾,由西華門入哭臨。諸王、世子、王妃、郡主、內使、宮人俱斬衰三年,二十七月除。凡臨朝視事,素服、烏紗帽、黑角帶,退朝衰服。群臣麻布員領衫麻布冠、麻絰、麻鞋。命婦麻布大袖長衫,麻布蓋頭。明器如鹵簿。神主用栗,制度依家禮。行人頒遺詔於天下。在外百官,詔書到日,素服、烏紗帽、黑角帶,四拜。聽宣讀訖,舉哀,再四拜。三日成服,每旦設香案哭臨,三日除。各遣官赴京致祭,祭物禮部備。孝陵設神宮監并孝陵衛及祠祭署。建文帝詔行三年喪,事在《本紀》。以遭革除,喪葬之制皆不傳。

文帝崩於榆木川,遺詔一遵太祖遺制。京師聞訃,皇太子以下皆易服。宮中設几筵,朝夕哭奠。百官素服,朝夕哭臨思善門外。禮部定喪禮,宮中自皇太子以下及諸王、公主,成服日為始,斬衰三年,二十七月除。服內停音樂、嫁娶、祭禮,止停百日。文武官聞喪之明日,詣思善門外哭,五拜三叩頭,宿本署,不飲酒食肉。四日衰服,朝夕哭臨三日,又朝臨十日。衰服二十七日。凡入朝及視事,白布裹紗帽、垂帶、素服、腰絰、麻鞋。退朝衰服,二十七日外,素服、烏紗帽、黑角帶,二十七月而除。聽選辦事等官衰服,監生吏典僧道素服,赴順天府,朝夕哭臨三日,又朝臨十日。命婦第四日由西華門入,哭臨三日,俱素服,二十七日除。凡音樂祭祀,並輟百日。婚嫁,官停百日,軍民停一月。軍民素服,婦人素服不妝飾,俱二十七日。在外以聞喪日為始,越三日成服,就本署哭臨,餘如京官。命婦素服舉哀三日,二十七日除。軍民男女皆素服十三日,餘俱如京師。凡京官服,給麻布一匹自製。四夷使臣,工部造與。諸王公主遣官及內外文武官詣几筵祭祀者,光祿寺備物,翰林院撰文,禮部引赴思善門外行禮。京城聞喪日為始,寺觀各鳴鐘三萬杵,禁屠宰四十九日。喪將至,文武官衰服,軍民素服赴居庸關哭迎。皇太子、親王及群臣皆衰服哭迎於郊。至大內,奉安於仁智殿,加斂,奉納梓宮。遣中官奉大行皇帝遺衣冠。作書賜漢王、趙王。禮臣言:「喪服已踰二十七日,請如遺命,以日易月。」帝以梓宮在殯,不忍易,素冠、麻衣、麻絰視朝,退仍衰服,群臣聽其便。

十二月,禮部進葬祭儀:發引前三日,百官齋戒。遣官以葬期告天地宗社,皇帝衰服告几筵,皇太子以下皆衰服隨班行禮。百官衰服朝一臨,至發引止。前一日,遣官祭金水橋、午門、端門、承天門、大明門、德勝門并所過河橋、京都應祀神祇及經過應祀神祠,儀用酒果肴饌。是夕,設辭奠,帝后太子以下皆衰服,以序致祭。司禮監、禮部、錦衣衛命執事者設大昇轝、陳葬儀於午門外并大明門外。將發,設啟奠。皇帝暨皇太子以下衰服四拜。奠帛、獻酒、讀祝,四拜。舉哀,興,哀止,望瘞。執事者升,徹帷幙,拂拭梓宮,進龍輴於几筵殿下。設神亭、神帛輿、謚冊寶輿於丹陛上,設祖奠如啟奠儀。皇帝詣梓宮前,西向立。皇太子、親王以次侍立。內侍於梓宮前奏,請靈駕進發,捧冊寶、神帛置輿中;次銘旌出;執事官升梓宮,內執事持翣左右蔽。降殿,內侍官請梓宮升龍輴,執事官以彩帷幕梓宮,內侍持傘扇侍衛如儀。舊御儀仗居前,冊寶、神帛、神亭、銘旌以次行。皇帝由殿左門出,后妃、皇太子、親王及宮妃後隨。至午門內,設遣奠,如祖奠儀。內侍請靈駕進發,皇帝以下哭盡哀,俱還宮。梓宮至午門外,禮官請梓宮升大昇轝。執事官奉升轝訖,禮官請靈駕進發。皇太子、親王以下哭送出端門外,行辭祖禮。執事官設褥位於太廟帛香案前。皇太子易常服,捧神帛,由左門入,至褥位跪,置神帛於褥,興,正立於神後跪。禮官跪於左,奏太宗體天弘道高明廣運聖武神功純仁至孝文皇帝謁辭。皇太子俯伏,興。贊五拜三叩頭畢,皇太子捧神帛興,以授禮官。禮官安輿中,請靈駕進發。皇太子仍喪服,親王以下隨行。梓宮由大明中門出,皇太子以下由左門出,步送至德勝門外,乘馬至陵,在途朝夕哭奠臨。諸王以下及百官、軍民耆老、四品以上命婦,以序沿途設祭。文武官不係山陵執事者悉還。至陵,執事官先陳龍輴於獻殿門外,俟大昇轝至。禮官請靈駕降轝,升龍輴詣獻殿。執事官奉梓宮入,皇太子、親王由左門入,安奉訖,行安神禮。皇太子四拜,興,奠酒,讀祝。俯伏,興,四拜,舉哀。親王以下陪拜,如常儀。遣官祀告后土並天壽山,設遷奠禮,如上儀。將掩玄宮,皇太子以下詣梓宮前跪。內侍請靈駕赴玄宮,執事官奉梓宮入皇堂。內侍捧冊寶置於前,陳明器,行贈禮。皇太子四拜興,奠酒,進贈。執事官捧玉帛進於右,皇太子受獻,以授內執事,捧入皇堂安置。俯伏,興,四拜,舉哀,遂掩玄宮。行饗禮,如遷奠儀。遣官祀謝后土及天壽山。設香案玄宮門外,設題主案於前,西向。設皇太子拜位於前,北向。內侍盥手奉主置案上,題主官盥手西向題畢,內侍奉主安於神座,藏帛箱中。內侍奏請太宗文皇帝神靈上神主。贊四拜,興,獻酒,讀祝。俯伏,興,四拜,舉哀。內侍啟櫝受主訖,請神主降座升輿。至獻殿,奏請神主降輿升座,行初虞禮。皇太子四拜,初獻,奠帛酒,讀祝,俯伏,興。亞獻、終獻,四拜,舉哀,望瘞。內官捧神帛箱埋於殿前,焚凶器於野。葬日初虞,柔日再虞,剛日三虞,後間日一虞,至九虞止。在途,皇太子行禮。還京,皇帝行禮。

神主將還,內侍請神主降座升輿,儀仗侍衛如儀。皇太子隨,仍朝夕奠。至京,先於城外置幄次,列儀衛,鼓吹備而不作。百官衰服候城外,主入幄次,百官序列,五拜三叩首。神主行,百官從。至午門外,皇帝衰服迎於午門內,舉哀,步導主升几筵殿。皇帝立殿上,內侍請神主降輿升座,行安神禮。皇帝四拜,興,奠酒,讀祝。俯伏,興,四拜,舉哀。皇太子以下陪拜。百官於思善門外行禮如儀。明日,百官行奉慰禮。

卒哭用虞祭後剛日,禮同虞祭,自是罷朝夕奠。祔饗用卒哭之明日,太常寺設醴饌於太廟,如時饗儀,樂設而不作。設儀衛傘扇於午門外,內侍進御輦於几筵殿前,皇帝衰服四拜,舉哀。興,哀止,立於拜位之東,西向。內侍請神主降座升輦,詣太廟祔饗。至思善門外,皇帝易祭服,升輅,隨至午門外,詣御輦前跪。太常卿奏請神主降輦,皇帝俯伏,興,捧主由左門入,至丹陛上。典儀唱「太宗文皇帝謁廟」。至廟前,內侍捧主至褥位,皇帝於後行八拜禮。每廟俱同。內侍捧主北向,太常卿立壇東,西向。唱「賜坐」,皇帝搢圭,奉神主安於座,詣拜位行祭禮,如時饗儀。太常卿奏請神主還几筵,皇帝捧主由廟左門出,安奉於御輦。皇帝升輅隨,至思善門降輅,易衰服,隨至几筵殿前。內侍請神主降輦,升座。皇帝由殿左門入,行安神禮畢,釋服還宮。明日,百官素服行奉慰禮。

大祥,奉安神主於太廟,禮詳廟制。皇帝祭告几筵殿,皇太后、皇后以下各祭一壇,王府遣官共祭一壇,在京文武官祭一壇。自神主出几筵殿,內侍即撤几筵、帷幄,焚於思善門外。禫祭,遣親王詣陵行禮。

洪熙元年,仁宗崩。皇太子還自南京,至良鄉,宮中始發喪,宣遺詔。文武官常服於午門外四拜。宣畢,舉哀,復四拜。易素服,迎皇太子於盧溝橋,橋南設幕次香案。皇太子至,常服,詣次四拜。聽宣遺詔,復四拜,哭盡哀。易素服至長安右門下馬,步哭至宮門外,釋冠服,披髮詣梓宮前,五拜三叩首,哭盡哀。宮中自皇后以下皆披髮哭。皇太子就喪次東,見母后。親王以次見皇太子畢,各居喪次,行祭告禮。喪儀俱如舊。惟改在京朝夕哭臨三日,後又朝臨止七日,在外止朝夕哭臨三日,無朝臨禮。文武官一品至四品命婦入哭臨。服除,禮臣請帝服淺淡色衣、烏紗翼善冠、黑角帶,於奉天門視事。百官皆淺淡色衣、烏紗帽、黑角帶,朝參如常儀。退朝,仍終太宗服制。帝曰:「朕心何能忍,雖加一日愈於已。」仍素服坐西角門,不鳴鐘鼓,令百日後再議。已百日,禮臣復請御奉天門。帝命候山陵事畢。

先是,詔營獻陵,帝召尚書蹇義、夏原吉諭曰:「國家以四海之富葬親,豈惜勞費。然古聖帝明王皆從儉制。孝子思保其親體魄於永久,亦不欲厚葬。況皇考遺詔,天下所共知,宜遵先志。」於是建寢殿五楹,左右廡神廚各五楹,門樓三楹。其制較長陵遠殺,皆帝所規畫也。吏部尚書蹇義等請祔廟後,素服御西角門視事。至孟冬歲暮,行時饗禮。鳴鐘鼓,黃袍御奉天門視朝。禫祭後,始釋素服。從之。

宣宗崩,喪葬如獻陵故事。惟改命婦哭臨,自三品以上。英宗崩,遺命東宮過百日成婚,不得以宮妃殉葬。憲宗即位,百日御奉天門視朝,禮儀悉用吉典。憲宗崩,孝宗既除服,仍素翼善冠、麻衣、腰絰視朝,不鳴鐘鼓,百官素服朝參,百日後如常。弘治元年正旦,時未及小祥,帝黃袍御殿受朝。次日,仍黑翼善冠,淺淡服、犀帶。及大祥,神主奉安太廟及奉先殿。至禫祭,免朝。擇日遣官詣陵致祭。

孝宗崩,工部言:「大行遺詔,惓惓以節用愛民為本。乞敕內府諸司,凡葬儀冥器并山陵殿宇,務從減省。」禮部言:「百日例應變服,但梓宮未入山陵,請仍素翼善冠、麻布袍服、腰絰,御西角門視事,不鳴鐘鼓,百官仍素服朝參。」從之。自辭靈至虞祔,榮王俱在陪列。既而王以疾奏免。禮部請以駙馬等官捧帛朝祖,帝曰:「朝祖捧帛,朕自行。」發引,親王止送至大明門外。其在途及至陵臨奠,俱護喪官行禮。後遂為例。

世宗崩,令旨免命婦哭臨。隆慶元年正月,未及二十七日,帝衰服御宣治門,百官素服、腰絰奉慰。發引,帝行遣奠禮。至朝祖,則遣官捧帛行禮。梓宮至順天府,皇親命婦及三品以上命婦祭,餘如舊制。光宗即位,禮部言:「喪服列代皆有制度,而斷自孝宗。蓋孝宗篤於親,喪禮詳且備,故武、世、穆三廟皆宗之。今遵舊制,以衰服御文華門視事,百官素服朝參,候梓宮發引除。」從之。

明自仁宗獻陵以後,規制儉約。世宗葬永陵,其制始侈。及神宗葬定陵,給事中惠世揚、御史薛貞巡視陵工,費至八百餘萬云。

皇后陵寢興宗帝后陵寢睿宗帝后陵寢皇妃等喪葬皇太子及妃喪葬諸王及妃公主喪葬

○皇后陵寢

洪武十五年,皇后馬氏崩。禮部引宋制為請。於是命在京文武官及聽除官,人給布一匹,令自製服,皆斬衰二十七日而除,服素服百日。凡在京官,越三日素服至右順門外,具喪服入臨畢,素服行奉慰禮,三日而止。武官五品以上、文官三品以上命婦,亦於第四日素服至乾清宮入臨。用麻布蓋頭,麻布衫裙鞋,去首飾脂粉。其外官服制與京官同。聞訃日於公廳成服,命婦服亦與在京命婦同,皆三日而除。軍民男女素服三日。禁屠宰,在京四十九日,在外三日。停音樂祭祀百日。嫁娶,官停百日,軍民一月。將發引,告太廟,遣官祭金水橋、午門等神及鐘山之神。帝親祭於几筵,百官喪服詣朝陽門外奉辭。是日,安厝皇堂。皇太子奠,玄纁玉璧,行奉辭禮。神主還宮,百官素服迎於朝陽門外,仍行奉慰禮。帝復以醴饌祭於几筵殿,自再虞至九虞,皆如之。遣官告謝鐘山之神。卒哭,以神主詣廟行祔享禮。喪滿百日,帝輟朝,祭几筵殿,致欽不拜。東宮以下奠帛爵,百官素服行奉慰禮。東宮、親王、妃、主以牲醴祭孝陵,公侯等從。命婦詣几筵殿祭奠。自後凡節序及忌日,東宮親王祭幾筵及陵。小祥,輟朝三日。禁在京音樂屠宰,設醮於靈谷寺、朝天宮各三日。帝率皇太子以下詣几筵殿祭。百官素服,詣宮門。進香訖,詣後右門奉慰。外命婦詣几筵殿進香。皇太子、親王熟布練冠九衣取,皇孫七衣取,皆去首絰。負版辟領衰。見帝及百官則素服、烏紗帽、烏犀帶。妃、主以下,熟布蓋頭,去腰絰。宗室駙馬練冠,去首絰。內尚衣、尚冠,以所釋服於几筵殿前丙位焚之。皇太子、親王復詣陵行禮。大祥,奉安神主於奉先殿,預期齋戒告廟。百官陪祀畢,行奉慰禮。

成祖皇后徐氏崩,自次日輟朝,不鳴鐘鼓。帝素服御西角門,百官素服詣思善門外哭臨畢,行奉慰禮。三日成服,哭臨如上儀。自次日為始,各就公署齋宿,二十七日止。文武四品以上命婦成服日為始,詣思善門內哭臨三日。聽選辦事官,俱喪服。人材監生、吏典、僧道、坊廂耆老各素服。自成服日始,赴應天府舉哀三日,餘悉遵高后時儀。又定諸王、公主等服制,世子郡王皆齊衰不杖期。世子郡王妃、郡主皆大功。周、楚諸王及寧國諸公主及郡王之子皆小功。遣中官訃告諸王府,造祔里,謁太廟。祭器、謚冊、謚寶悉用檀香。將冊,帝躬告天地於奉天殿丹陛上。御華蓋殿,鴻臚寺官引頒冊寶官入行禮,傳制曰:「永樂五年十月十四日,冊謚大行皇后,命卿行禮。」四拜畢,序班舉冊寶案至奉天殿丹陛上,置綵輿中,由中道出,入右順門至几筵殿,以冊寶置案,退俟於殿外。尚儀女官詣香案前,跪進曰:「皇帝遣某官冊謚大行皇后,謹告。」贊宣冊,女官捧冊宣於几筵之右,置冊於案,宣寶如之。尚儀奏禮畢,女官以冊寶案置几筵之左。內官出報禮畢,頒冊寶官復命。百日,禮部請御正門視朝,鳴鐘鼓,百官易淺淡色服。帝以梓宮未葬,不允。至周期,帝素服詣几筵致祭,百官西角門奉慰,輟朝三日。在京停音樂、禁屠宰七日。禮部官於天禧寺、朝天宮齋醮。其明日,帝吉服御奉天門視朝,鳴鐘鼓。百官服淺淡色衣、烏紗帽、黑角帶,退朝署事仍素服。遇朔望,朝見慶賀如常儀。几筵祭祀,熟布練冠。及發引,齋三日,遣官以葬期告郊廟社稷。帝素服祭告几筵,皇太子以下衰服行禮,遣官祭所過橋門及沿途祀典諸神。百官及命婦俱素服,以次路祭。梓宮至江濱,百官奉辭於江濱。皇太子送渡江,漢王護行,途中朝夕哭奠。官民迎祭者,皆素服。既葬,賜護送官軍及舁梓官軍士鈔米有差。

正統中,仁宗皇后張氏崩,禮部定大行太皇太后喪禮。皇帝成服三日後,即聽政。祀典皆勿廢,諸王以下內外各官及命婦哭臨如前儀,衰服二十七日而除,軍民男女素服十三日。諸王勿會葬,外官勿進香,臣民勿禁音樂嫁娶。及葬,遣官告太廟。帝親奉太后衣冠謁列祖帝、后及仁宗神位,又奉宣宗衣冠謁太后神位,其禮視時享。天順中,宣帝皇后孫氏崩,儀如故事,止改哭臨於清寧門。英宗皇后錢氏崩,禮如舊,惟屠宰止禁七日,外國使臣免哭臨。正德元年,景帝后汪氏薨。禮部會群臣言,宜如皇妃例,輟朝三日,祭九壇。太后、中宮、親王以下文武大臣命婦皆有祭。制可。

憲宗廢后吳氏,正德四年薨,以大學士李東陽等言,禮如英宗惠妃故事。憲宗皇后王氏,正德十三年崩。越三日,帝至自宣府,乃發喪。百官具素服,於清寧宮門外聽宣遺誥。及發引,先期結平臺,與順天府交衢相值。帝晨出北安門迎,皇太后及皇后御平臺候殯。復入至清寧宮,親奉梓宮朝祖。百官步送德勝門外,惟送喪官騎送。明日,帝奉神主還京,百官迎於德勝門。帝素服、腰絰御西角門,百官奉慰。卒哭,始釋服。孝宗母紀氏,憲宗妃也。成化中薨,輟朝如故事。自初喪及葬,帝及皇太后、中宮、妃、主、皇子皆致祭。遣皇子奉祝冊行禮,塋域、葬儀俱從厚。皇親百官及命婦送葬設祭,皆如儀。

世宗祖母邵氏,嘉靖元年崩。服除,部臣毛澄等請即吉視事。議再上,命考孝肅太皇太后喪禮。澄等言:「孝肅崩時,距葬期不遠,故暫持凶服,以待山陵事竣,與今不同。況當正旦朝元,亦不宜縞衣臨見萬國。若孝思未忘,第毋御中門及不鳴鐘鼓足矣。」從之,仍免朔望日升殿。既葬四日,帝御奉天門,百官行奉慰禮,始從吉。嘉靖中,孝宗皇后張氏崩,禮臣以舊制上。帝謂郊社不宜瀆,罷祭告。又謂躬行諸禮,前已諭代,亦罷謁廟禮。及太常寺以朝祖祔廟,請各廟捧主官,詔主俱不必出,蓋從殺也。

先是,武宗皇后夏氏崩,禮部上儀注,有素冠、素服、絰帶舉哀及群臣奉慰禮。帝曰:「朕於皇兄后無服,矧上奉兩宮,又迫聖母壽旦,忍用純素。朕青服視事,諸儀再擬。」於是尚書夏言等言:「莊肅皇后喪禮,在臣民無容議。惟是皇上以天子之尊,服制既絕,不必御西角門。群臣成服後,不當素服朝參。」及上喪葬儀,帝復諭:「毅皇后事宜與累朝元后不同,無几筵之奉,當即行祔廟,令皇后攝事於內殿。」言等議:「按禮,卒哭乃行祔裡告。蓋以新主當入,舊主當祧,故預以告也。此在常典則然,非今日議例。毅皇后神主誠宜即祔太廟,以妥神靈,而祔告之禮宜免。」因具上其儀。制可。

嘉靖七年,世宗皇后陳氏崩。禮部上喪祭儀,帝疑過隆。議再上,帝自裁定,概從減殺,欲九日釋服。閣臣張璁等言:「夫婦之倫,參三綱而立。人君乃綱常之主,尤不可不慎。《左傳》昭公十五年六月乙丑,周景王太子壽卒。秋八月戊寅,王穆后崩。叔向曰:『王一歲而有三年之喪二焉。』蓋古禮,父為子,夫為妻,皆服報服三年。後世,夫為妻,始制為齊衰杖期,父母在則不杖。《喪服》,自期以下,諸侯絕,然特為旁期言。若妻喪,本自三年報服,殺為期年,則固未嘗絕者。今皇上為后服期,以日易月,僅十二日。臣子為君母服三年,以日易月,僅二十七日。較諸古禮,已至殺矣。皇上宜服期,十二日,臣子素服,終二十七日。不然,則恩紀不明,典禮有乖。」禮臣方獻夫亦雜引《儀禮·喪服》等篇,反覆爭辨,并《三朝聖諭》所載仁孝皇后崩,太宗衰服後,仍服數月白衣冠故事以證之。帝言:「文皇后喪時,上無聖母,下有東宮,從重盡禮為宜。今不敢不更其制。」已,詹事霍韜言:「今百官遭妻喪,無服衰蒞事之禮。蓋妻喪內而不外,陰不可當陽也。聖諭云:『素服十日,仿輟朝之義。』於內廷行之則可。若對臨百官,總理萬幾,履當陽之位,行中宮之服則不可。百官為皇后服衰,為其母儀天下也。禮,父在為母,杖不上於堂,尊父也。於朝廷何獨不然?臣請陛下玄冠素服,御西角門十日,即玄冠玄服御奉天門,百官入左掖門則烏紗帽、青衣侍班。退出公署及私室,則仍素服白帽二十七日。若曰於禮猶有未慊,則山陵事畢而除。」帝從其言。

尋定進冊謚儀,禮部議:「先期,帝袞冕告奉先殿、崇先殿。至期,帝常服御奉天門,正副使常服,百官淺淡色衣、黑角帶,入班行禮如儀。節冊至右順門,內侍捧入正門,至几筵前置於案。內贊贊就位上香,宣冊官立宣訖,復置冊於案。內侍持節由正門出,以節授正副使,報禮畢,正副使持節復命。」次日,禮部謄黃頒示天下。

時中宮喪禮自文皇后而後,至是始再行。永樂時典禮毀於火,《會典》所載皆略,乃斷自帝心,著為令。梓宮將葬,帝新定諸儀,亦從減損。以思善門逼近仁智殿,命百宮哭臨止一日,亦罷辭祖禮,喪由左王門出。

二十六年,皇后方氏崩,即日發喪,諭禮部:「皇后嘗救朕危,其考元后喪禮行之。」禮部定儀:「以第四日成服,自後黑冠素服,十日後易淺色衣,俱西角門視朝。百官十日素服絰帶,自後烏紗帽、黑角帶、素服,通前二十七日。帝常服於奉天門視朝,百官淺色衣,鳴鐘鼓、鳴鞭如常,朔望不升殿。梓宮發引,百官始常服。帝於奉先等殿行禮,俱常服。於几筵祭則服其服。服滿日,命中官代祭。」從之。尋諭:「皇妃列太子後非禮,其改正。」及葬,部臣以舊儀請。詔梓宮由中道行,虞祭如制用九數。安玄宮居左,他日即配祀。部臣復上儀注,改席殿曰行享殿。又以孝潔皇后自發引至神主還京將半載,遇令節百官常服,今孝烈皇后初十日發引,十五日即還,事禮不同,以諸臣服制請。帝命隨喪往來者,仍制服。祭畢,烏紗帽素服入朝,素冠素服辦事。迎主仍制服,思善門外行安神禮,更素冠素服從事。先是,帝命孝烈居左,而遷孝潔。既而以孝潔久安,不宜妄動,罷不行。乃更命孝烈居右,而虛其左以自待。

穆宗母杜氏,三十三年薨。禮部言:「宜用成化中淑妃紀氏喪制。且裕王已成婚,宜持服主喪,送葬出城。」乃議輟朝五日,裕王遵《孝慈錄》斬衰三年。欽遣大臣題主,開塋掩壙,祠謝后土,并用工部官,送葬儀仗人數皆增於舊。帝謂非禮之正,令酌考賢妃鄭氏例。於是尚書歐陽德等復上儀注,輟朝二日,不鳴鐘鼓。帝服淺淡色衣,奉天門視事,百官淺色衣、烏紗帽、黑角帶朝參。命裕王主饋奠之事,王率妃入宮,素服哭盡哀,四拜視殮。成服後,朝夕哭臨三日。後每日一奠,通前二十七日而止。仍於燕居盡斬衰三年之制。冊謚焚黃日,陳祭儀,裕王詣靈前行禮。喪出玄武門,裕王步送至京城門外,路祭畢,還宮。帝謂焚黃乃制命,非王可行,仍如常儀。禮部覆奏:「皇妃焚黃儀,傳訛已久。皆拜獻酒,跪讀祝,乃參用上尊謚之儀,而未思賜謚為制命,其祭文稱皇帝遣諭,與上尊謚不同。今奉旨以常禮從事,當改議賜謚,如賜祭禮。讀祝、宣冊皆平立不拜。」報可,著為令。

穆宗皇后李氏,裕邸元妃也,先薨,葬西山。隆慶元年,加謚孝懿皇后,親告世宗几筵。御皇極門,遣大臣持節捧冊寶詣陵園上之。神宗母皇太后李氏,萬曆四十二年崩。帝諭禮部從優具儀,帝衰服行奠祭禮。穆廟皇妃、中宮妃嬪、太子、諸王、公主以下皆成服。百官詣慈寧宮門外哭臨。命婦入宮門哭臨。餘俱如大喪禮。

○興宗帝后陵寢

洪武二十五年,皇太子薨,命禮部議喪禮。侍郎張智等議曰:「喪禮,父為長子服齊衰期年。今皇帝當以日易月,服齊衰十二日,祭畢釋之。在內文武官公署齋宿。翌日,素服入臨文華殿,給衰麻服。越三日成服,詣春和門會哭。明日,素服行奉慰禮。其當祭祀及送葬者,仍衰絰以行。在京,停大小祀事及樂,至復土日而止。停嫁娶六十日。在外,文武官易服,於公署發哀。次日,成服行禮。停大小祀事及樂十三日,停嫁娶三十日。」其內外官致祭者,帝令光祿寺供具,百官惟致哀行禮。建文帝即位,追謚為興宗孝康皇帝,所薦陵號不傳。

元妃常氏,先興宗薨。太祖素服,輟朝三日。中宮素服哀臨,皇太子齊衰。葬畢,易常服。皇孫斬衰,祭奠則服之。諸王公主服如制。建文初,追謚曰孝康皇后。永樂初,皆追削。福王立南京,復帝后故號。

○睿宗帝后陵寢

睿宗帝後陵寢在安陸州。世宗入立,追謚曰睿宗獻皇帝。葺陵廟,薦號曰顯陵。既而希進之徒屢言獻皇帝梓宮宜改葬天壽山。帝不聽。嘉靖十七年,帝母蔣太后崩。禮部言:「歲除日,大行皇太后服制二十七日已滿,適遇正旦,請用黑冠、淺淡服受朝。」疏未下,帝諭大學士夏言:「元旦玄極殿拜天,仍具祭服,先期一日宜變服否?」禮部請「正旦拜天、受朝,及先一日俱青服,孟春時享,前三日齋,青服,臣下同之,餘仍孝貞皇太后喪禮例」。不從。於是定議,歲除日變服玄色吉衣,元旦祭服玄極殿行告祀禮,具翼善冠、黃袍御殿,百官公服致詞,鳴鐘鼓、鳴鞭,奏堂上樂。

是時議南北遷祔,久不決。帝親詣承天。及歸,乃定議梓宮南祔。禮部上葬儀,自常典外,帝復增定太廟辭謁、承天門辭奠、朝陽門遣奠、題主後降神饗神,及梓宮登舟、升岸等祭。梓宮發引,帝衰服行諸禮如儀。百官步送朝陽門外,奠獻,使行遣奠禮。至通州,題主官復命。神主回京,百官奉迎於門外,帝衰服率皇后以下哭迎午門內,奉安於几筵殿。梓宮所過河瀆江山神祇,俱牲醴致祭。勳臣青服行禮,梓宮升席殿。先詣睿宗舊陵,奉遷於祾恩殿,復奉梓宮至殿,合葬於新寢。

○皇妃等喪葬

洪武七年九月,貴妃孫氏薨。無子,太祖命吳王橚主喪事,服慈母服,斬衰三年。東宮諸王皆服期。由是作《孝慈錄》。

永樂中,貴妃王氏薨。輟朝五日,御祭一壇,皇后、皇妃、皇太子各祭一壇,親王共祭一壇,公主共祭一壇。七七、百日期、再期,皆祭贈謚冊,行焚黃禮。開塋域,遣官祠后土。發引前期,辭靈祭壇與初喪同,惟增六尚司及內官、內使各一壇。啟奠、祖奠、遣奠各遣祭一壇。發引日,百官送至路祭所,皇親駙馬共一壇,公侯伯文武共一壇,外命婦共一壇。所過城門祭祀,內門遣內官,外門遣太常寺官。下葬,遣奠、遣祭一壇。掩壙,遣官祀后土,迎靈轎至享堂,行安神禮,遣祭一壇。

天順七年,敬妃劉氏薨。輟朝五日,帝服淺淡黃衣於奉天門視事,百官淺淡色衣、烏紗帽、黑角帶朝參。冊文置靈柩前,皇太子以下行三獻禮。靈柩前儀仗,內使女樂二十四人,花幡、雪柳女隊子二十人,女將軍十一人。自初喪至期年辭靈,各於常祭外增祭一壇。

弘治十四年,憲廟麗妃章氏發引,輟朝一日。

凡陪葬諸妃,歲時俱享於殿內。其別葬金山諸處者,各遣內官行禮。嘉靖間,始命併入諸陵,從祭祾恩殿之兩旁,以紅紙牌書曰「某皇帝第幾妃之位」,祭畢,焚之。後改用木刻名號。嘉靖十三年,諭禮工二部:「世婦、御妻皆用九數。九妃同一墓,共一享殿,為定制。」

○皇太子及妃喪葬

自洪武中懿文太子後,至成化八年悼恭太子薨,年甫三歲。帝諭禮部,禮宜從簡,王府及文武官俱免進香帛。禮部具儀上。自發喪次日,輟朝三日。帝服翼善冠、素服,七日而除。又三日,御西角門視朝,不鳴鐘鼓,祭用素食。文武群臣,素服、麻布、絰帶、麻鞋、布裹紗帽,詣思善門哭臨,一日而除。第四日,素服朝西角門奉慰。在外王府并文武官,素服舉哀,二日而除。

嘉靖二十八年,莊敬太子薨。禮部上喪禮。帝曰:「天子絕期。況十五歲外方出三殤,朕服非禮,止輟朝十日。百官如制成服,十二日而除。詣停柩所行,罷詣門哭臨。葬遣戚臣行禮。」

萬曆四十七年二月,皇太子才人王氏薨,命視皇太子妃郭氏例。輟朝五日,不鳴鐘鼓。帝服淺淡色衣,百官青素服、黑角帶朝參,皇長孫主饋奠。

○諸王及妃公主喪葬諸儀

洪武二十八年,秦王樉詆薨,詔定喪禮。禮部尚書任亨泰言:「考宋制,宜輟朝五日。今遇時享,請暫輟一日。皇帝及親王以下,至郡主及靖江王宮眷服制,皆與魯王喪禮同。皇太子服齊衰期,亦以日易月,十二日而除,素服期年。」從之。

定制:親王喪,輟朝三日。禮部奏遣官掌行喪葬禮,翰林院撰祭文、謚冊文、壙志文,工部造銘旌,遣官造墳,欽天監官卜葬,國子監監生八名報訃各王府。御祭一,皇太后、皇后、東宮各一,在京文武官各一。自初喪至除服,御祭凡十三壇,封內文武祭一。其服制,王妃、世子、眾子及郡王、郡主,下至宮人,斬衰三年,封內文武官齊衰三日,哭臨五日而除。在城軍民素服五日。郡王、眾子、郡君,為兄及伯叔父齊衰期年,郡王妃小功。凡親王妃喪,御祭一壇,皇太后中宮、東宮、公主各祭一壇。布政司委官開壙合葬。繼妃、次妃祭禮同。其夫人則止御祭一壇。俱造壙祔葬。郡王喪,輟朝一日。行人司遣掌行喪葬禮,餘多與親王同,無皇太后、皇后祭。郡王妃與親王妃同,無公主祭。合葬郡王繼妃次妃喪禮,俱與正妃同。凡世子喪,御祭一,東宮祭一。遇七及百日、下葬、期年、除服,御祭各一。凡世孫喪禮,如世子,減七七及大祥祭。凡鎮國將軍,止聞喪、百日、下葬三祭,奉國將軍以下,御祭一。

初,洪武九年五月,晉王妃謝氏薨,命議喪服之制。侍講學士宋濂等議曰:「按唐制,皇帝為皇妃等舉哀。宋制,皇帝為皇親舉哀。今參酌唐、宋之制,皇帝及中宮服大功,諸妃皆服小功,南昌皇妃服大功,東宮、公主、親王等皆服小功,晉王服齊衰期,靖江王妃小功,王妃服緦麻,輟朝三日。既成服,皇帝素服入喪次,十五舉音。百官奉慰,皇帝出次釋服,服常服。」制曰「可」。其後,王妃喪視此。

正統十三年,定親王塋地五十畝,房十五間。郡王塋地三十畝,房九間。郡王子塋地二十畝,房三間,郡主、縣主塋地十畝,房三間。天順二年,禮部奏定,親王以下,依文武大臣例。或王、或妃先故者,合造其壙。後葬者,止令所在官司安葬。繼妃則祔葬其旁,同一享堂。

成化八年二月,忻王見治薨。發引日,帝不視朝。及葬,輟朝一日。十三年,四川按察使彭韶言:「親王郡王薨逝,皆遣官致祭,使臣絡繹,人夫勞擾。自後惟親王如舊,其郡王初喪遣官一祭,餘並遣本處官。凡王國母妃之喪,俱遣內官致祭。今宗婦眾多,其地有鎮守太監者,宜遣行禮。又王國塋葬,夫婦同穴。初造之時,遣官監修,開壙合葬,乞止命本處官司。」帝從禮部覆奏,王妃祭禮如舊,餘依議行。弘治十六年七月,申王祐楷薨。禮部言:「前沂穆王薨,未出府。申王已出府而未之國,擬依沂穆參以在外親王例行之。」

王妃葬地載於《會典》者,明初追封壽春等十王及妃,墳在鳳陽府西北二十五里白塔,設祠祭署、陵戶。南昌等五王及妃祔葬鳳陽皇陵,有司歲時祭祀,皆與享。懷獻世子以下諸王未之國者,多葬於西山,歲時遣內官行禮。

永樂十五年正月,永安公主薨。時初舉張燈宴,遂罷之。輟朝四日,賜祭,命有司治喪葬。二月,太祖第八女福清公主薨,輟朝三日。定制,凡公主喪聞,輟朝一日。自初喪至大祥,御祭凡十二壇。下葬,輟朝一日。儀視諸王稍殺,喪制同,惟各官不成服,其未下嫁葬西山者,歲時遣內官行禮。

謁祭陵廟忌辰受蕃國王訃奏儀為王公大臣舉哀儀臨王公大臣喪儀中宮為父祖喪儀遣使臨弔儀遣使冊贈王公大臣儀賜祭葬喪葬之制碑碣賜謚品官喪禮士庶人喪禮服紀

○謁祭陵廟

洪武元年三月,遣官致祭仁祖陵,二年,加號英陵。禮部尚書崔亮請下太常行祭告禮。博士孫吾與言:「山陵之制,莫備於漢,初未有祭告之禮。蓋廟號、陵號不同。廟號易大行之號,必上冊謚,告之神明,陵號則後嗣王所以識別先後而已,願罷英陵祭告。」亮言:「漢光武加先陵曰昌,宋太祖加高、曾、祖、考陵曰欽、康、定、安。蓋尊祖考由尊其陵,尊其制則必以告,禮緣人情,告之是。」廷議皆是亮。從之。熙祖陵,每歲正旦、清明、中元、冬至及每月朔望,本署官供祭行禮。又即其地望祭德祖、懿祖二陵。英陵後改稱皇陵,多孟冬一祭,俱署官行禮;朔望,中都留守司官行禮。

八年,詔翰林院議陵寢朔望節序祭祀禮。學士樂韶鳳等言:「漢諸廟寢園有便殿,日祭於寢,月祭於廟,時祭於便殿。後漢都洛陽,以關西諸陵久遠,但四時用特牲祀。每西幸,即親詣。歲正月祀郊廟畢,以次上洛陽諸陵。唐園陵之制,皇祖以上陵,皆朔望上食,元日、冬至、寒食、伏臘、社各一祭。皇考陵,朔望及節祭日進食,又薦新於諸陵。永徽二年,定獻陵朔望、冬夏至、伏臘、清明、社等節,皆上食。開元中,敕獻、昭、乾、定、橋、恭六陵,朔望上食,冬至、寒食各設一祭。宋每歲春秋仲月,遣太常宗正卿朝諸陵。我朝舊儀,每歲元旦、清明、七月望、十月朔、冬至日,俱用太牢,遣官致祭。白塔二處,則用少牢,中官行禮,今擬如舊儀,增夏至日用太牢,其伏臘、社、每月朔望,則用特羊,祠祭署官行禮。如節與朔望、伏臘、社同日,則用節禮。」從之。

十六年,孝陵殿成,命皇太子以牲醴致祭。清晨陳祭儀畢,皇太子、親王由東門入,就殿中拜位,皆四拜。皇太子少前,三上香,奠酒,讀祝曰:「園陵始營,祭享之儀未具。今禮殿既成,奉安神位,謹用祭告。」遂行亞獻、終獻禮,皇太子以下皆四拜,執事行禮皆內官。二十六年令,車馬過陵,及守陵官民入陵者,百步外下馬,違者以大不敬論。建文初,定孝陵每歲正旦、孟冬、忌辰、聖節,俱行香,清明、中元、冬至,俱祭祀。勳舊大臣行禮,文武官陪祀。若親王之籓,過京師者謁陵。官員以公事至,入城者謁陵,出城者辭陵。國有大事,遣官祭告。懿文太子陵在孝陵左,四孟、清明、中元、冬至、歲暮及忌辰,凡九祭。

永樂元年,工部以泗州祖陵黑瓦為言。帝命易以黃,如皇陵制。宣宗即位,遣鄭王謁祭孝陵。正統二年諭,天壽山陵寢,剪伐樹木者重罪,都察院榜禁,錦衣衛官校巡視,工部欽天監官環山立界,十年,謁三陵,諭百官具淺色衣服,如洪武、永樂例。南京司禮太監陳祖圭言:「魏國公徐俌每祭孝陵,皆由紅券門直入,至殿內行禮,僭妄宜改。」俌言:「入由紅券門者,所以重祖宗之祭,尊皇上之命。出由小旁門者,所以守臣下之分。循守故事,幾及百年,豈敢擅易。」下禮部議,言:「長陵及太廟,遣官致祭,所由之門與孝陵事體相同,宜如舊。」從之。

弘治元年,遣內官監護鳳陽皇陵,凡官員以公事經過者俱謁陵。十七年,更裕陵神座。初,議以孝肅太皇太后祔葬裕陵,已遣官分告諸陵及天壽山后土,而欽天監以為歲殺在北,方向不利。內官監亦謂英廟陵寢,難以輕動,遂議別建廟,奉安神主。帝心未慊,卒移英廟居中,孝莊居左,孝肅祔其右云。

正德間,定長陵以下諸陵,各設神宮監并衛及祠祭署。凡清明、中元、冬至,俱分遣駙馬都尉行禮,文武官陪祭。忌辰及正旦、孟冬、聖節,亦遣駙馬都尉行禮。親王之籓,詣諸陵辭謁。恭讓章皇后陵,清明、中元、冬至、忌辰內官行禮。西山景皇帝陵,祭期如上,儀賓行禮。

初,成祖易黃土山名天壽山。嘉靖十年,名祖陵曰基運山,皇陵曰翌聖山,孝陵曰神烈山,顯陵曰純德山,及天壽山,並方澤從祀,所在有司祭告各陵山祇。禮官因奏:「神祇壇每年秋祭,有鐘山、天壽山之神,今宜增基運等山。」從之。

十四年,諭禮部尚書夏言:「清明節既遣官上陵,內殿復祭,似涉煩復。」言因言:「我朝祀典,如特享、時享、祫享、禘祭,足應經義,可為世法。惟上陵及奉先殿多沿前代故事。上陵之祀,每歲清明、中元、冬至凡三。中元俗節,事本不經。往因郊祀在正首,故冬至上陵,蓋重一氣之始,伸報本之義。今冬至既行大報配天之禮,則陵事為輕。況有事南郊,乃輟陪祀臣僚,遠出山陵,恐於尊祖配天之誠未盡。可罷冬至上陵,而移中元於霜降,惟清明如舊。蓋清明禮行於春,所謂雨露既濡,君子履之,有怵惕之心者也。霜降禮行於秋,所謂霜露既降,君子履之,有悽愴之心者也。二節既遣官上陵,則內殿之祭,誠不宜復。」遂著為令。

十五年,諭言曰:「廟重於陵,其禮嚴。故廟中一帝一后,陵則二三后配葬。今別建奉慈殿,不若奉主於陵殿為宜。且梓宮配葬,而主乃別置,近於黜之,非親之也。」乃遷孝肅、孝穆、孝惠三后神主於陵殿。又諭言曰:「三后神主稱皇太后、太皇太后者,乃子孫所奉尊稱。今既遷陵殿,則名實不準。」言等議曰:「三后神主,禮不祔廟,義當從祧。遷奉陵殿,深合典禮。其稱皇太后、太皇太后者,乃子孫所上尊號。今已遷奉於陵,則當從夫婦之義,改題孝肅神主,不用睿字,孝穆、孝惠神主,俱不用純字,則嫡庶有別,而尊親併隆矣。」命如擬行。又諭:「祭告長陵等七陵俱躬叩拜,恭讓章皇后、景皇帝陵亦展拜一次,以慰追感之情。」十七年,改陵殿曰祾恩殿,門曰祾恩門。又建成祖聖蹟亭於平臺山,率從官行祭禮。二十一年,工部尚書顧璘請以帝所上顯陵聖製歌詩,製為樂章,享獻陵廟。禮部言:「天壽山諸陵,歲祀皆不用樂。」已而承天府守備太監傅霖乞增顯陵歲暮之祭。部議言:「諸陵皆無歲暮祀典。」詔並從部議。

隆慶二年,帝詣天壽山春祭。前一日,告世宗几筵及奉先、弘孝、神霄殿。駕至天壽山紅門降輿,由左門入,升輿,駐蹕感思殿。越二日,質明行禮。帝青袍,乘板輿至長陵門外,東降輿,由殿左門入,至拜位,上香,四拜。至神御前獻帛、獻爵訖,復位。亞獻、終獻,令執爵者代,復四拜。餘如常祭之儀。隨詣永陵行禮。是日遣官六員,俱青服,分祭六陵。

萬曆八年,謁陵禮如舊。十一年,復謁陵。禮部言:「宜遵世宗彞憲,酌分二日,以次展拜。」乃定長、永、昭三陵,上香,八拜,親奠帛。初獻,六陵二寢,上香,四拜。其奠帛三獻,俱執事官代。十四年,禮部言:「諸妃葬金山諸處者,嘉靖中俱配享各陵殿,罷本墳祭。今世廟諸妃安厝西山者,宜從其例。至陵祭品物,九陵、恭讓、恭仁之陵止於酒果,而越、靖諸王及諸王妃則又有牲果祝文,反從其厚者,蓋以九陵帝后,歲暮已祫祭於廟,旬日內且復有孟春之享,故元旦陵殿止用酒果,非儉也;諸王諸妃則祫祭春祭皆不與,元旦一祭不宜從簡,故用牲帛祝文,非豐也。特恭讓、恭仁既不與祫享於廟中,又不設牲帛於陵殿,是則禮文之缺,宜增所未備。而諸王諸妃祝文,尚仍安厝時所用,宜改敘歲時遣官之意,則情順禮安。」報可。

凡山陵規制,有寶城,長陵最大,徑一百一丈八尺。次永陵,徑八十一丈。各陵深廣丈尺有差。正前為明樓,樓中立帝廟謚石碑,下為靈寢門。惟永陵中為券門。左右墻門各一樓。明樓前為石几筵,又前為祾恩殿、祾恩門。殿惟長陵重簷九間,左右配殿各十五間。永陵重簷七間,配殿各九間。諸陵俱殿五間,配殿五間。門外神庫或一或二,神廚宰牲亭,有聖跡碑亭。諸陵碑俱設門外,率無字。長陵迤南有總神道,有石橋,有石像人物十八對,擎天柱四,石望柱二。長陵有《神功聖德碑》,仁宗御撰,在神道正南。南為紅門,門外石牌坊一。門內有時陟殿,為車駕更衣之所。永陵稍東有感思殿,為駐蹕之所。殿東為神馬廠。

○忌辰

洪武八年四月,仁祖忌日,太祖親詣皇陵致祭。永樂元年,禮部尚書李至剛等奏定,高皇帝忌辰前二日,帝服淺淡色衣,御西角門視事。不鳴鐘鼓,不行賞罰,不舉音樂,禁屠宰。百官淺淡色衣、黑角帶朝參。至日,親祀於奉先殿,仍率百官詣孝陵致祭。高皇后忌辰如之。

宣德四年令,凡遇忌辰,通政司、禮科、兵馬司勿引囚奏事。五年,敕百官朝參輟奏事儀。

英宗即位,召禮臣及翰林院議忌辰禮。大學士楊士奇、楊榮,學士楊溥議:「每歲高廟帝后、文廟帝后、仁宗忌辰,服淺淡色服,不鳴鐘鼓,於奉天門視事。宣宗忌辰,小祥之日,於西角門視事。」從之。

弘治十四年令,凡遇忌辰,朝參官不得服糸寧絲紗羅衣。景皇帝、恭讓皇后忌辰,遇節令,服青絲花樣。宣宗忌辰,遇祭祀,服紅。十六年八月,吏部尚書馬文升言:「宣德間,仁宗忌辰,諸司悉免奏事。自太祖至仁宗生忌,俱輟朝。其後不知何時,仁宗忌辰,依前奏事。惟太祖至憲宗忌辰,百官淺淡色服、黑角帶。朝廷亦出視朝,鳴鐘鼓,奏事。臣思自仁至憲,世有遠近,服有隆殺。請自仁宗忌辰、英宗生忌日,視朝,鳴鐘鼓。若遇憲宗及孝穆皇太后忌日,不視朝,著淺淡服,進素膳,不預他事。或遵宣宗時例,自太祖至憲宗生忌,俱輟朝一日。憲宗、孝穆忌日,如臣所擬。」帝下禮部議。部臣言:「經傳所載,忌日為親死之日。則死日為忌,非謂生辰也。其曰忌日不用,不以此日為他事也。曰忌日不樂,是不可舉吉事也,此日當專意哀思父母,餘事皆不舉。但先朝事例,迄今見行,未敢更易。」帝乃酌定以淺淡服色視事。

嘉靖七年令,忌辰只祭本位。十八年令,高廟帝后忌辰祭於景神殿,列聖帝后忌辰祭於永孝殿。二十四年令,仍祭於奉先殿。

○乘輿受蕃國王訃奏儀

凡蕃國王薨,使者訃奏至,於西華門內壬地設御幄,皇帝素服乘輿詣幄。太常卿奏:「某國世子遣陪臣某官某,奏某國王臣某薨。」承制官至使者前宣制曰:「皇帝致問爾某國王某,得何疾而逝。」使者答故。其儀大略如臨王公大臣喪儀,但不舉哀。

凡塞外都督等官訃至,永樂間遣官齎香鈔諭祭。後定例,因其奏請,給與表裏祭文,令攜歸自祭。來京病故者,遣官諭祭或賜棺賜葬。後定年終類奏,遣官祭之。若在邊歿於戰陣者,不拘此例。凡外國使臣病故者,令所在官司賜棺及祭,或欲歸葬者聽。

○乘輿為王公大臣舉哀儀

洪武二年,開平王常遇春卒於軍。訃至,禮官請如宋太宗為趙普舉哀故事。遂定制,凡王公薨,訃報太常司,示百官,於西華門內壬地設御幄,陳御座,置素褥。設訃者位於前,設百官陪哭位東西向,奉慰位於訃者位北,北向。贊禮二人,位於訃者位之北,引訃者二人,位於贊禮之南,引百官四人,位於陪位之北,皆東西向。其日,備儀仗於奉天門迎駕。皇帝素服乘輿詣幄,樂陳於幄之南,不作。太常卿奉:「某官來訃,某年月日,臣某官以某疾薨,請舉哀。」皇帝哭,百官皆哭。太常卿奏止哭,百官奉慰訖,分班立。訃者四拜退,太常卿奏禮畢。乘輿還宮,百官出。東宮為王公舉哀儀同,但設幄於東宮西門外,陪哭者皆東宮屬。

○乘輿臨王公大臣喪儀

凡王公大臣訃奏,太史監擇皇帝臨喪日期。拱衛司設大次於喪家大門外,設御座於正廳中。有司設百官次於大次之左右。侍儀司設百官陪立位於廳前左右,引禮四人位於百官之北,東西向。設喪主以下拜位於廳前,主婦以下哭位於殯北幔中。其日,鑾駕至大次,降輅,升輿,入易素服。百官皆易服,先入就廳前,分班侍立。御輿出次。喪主以下免絰去杖,衰服,出迎於大門外。望見乘輿,止哭,再拜,入於門內之西。乘輿入門,將軍四人前導,四人後從。入至正廳,降輿,升詣靈座前,百官班於後。皇帝哭,百官皆哭。太常卿奏止哭,三上香,三祭酒。出至正廳御座,主喪以下詣廳下拜位,再拜。承制官詣喪主前云「有制」。喪主以下皆跪。宣制訖,皆再拜,退立於廳西。太常卿奏禮畢,皇帝升輿,出就大次,易服。御輿出,喪主以下詣前再拜退。皇帝降輿升輅,喪主杖哭而入。諸儀衛贊唱,大略如常。

其公、侯卒葬輟朝禮,洪武二十三年定。凡公、侯卒於家者,聞喪輟朝三日。下葬,一日。卒於外者,聞喪,一日。柩至京,三日。下葬,仍一日。凡輟朝之日,不鳴鐘鼓,各官淺淡色衣朝參。初制,都督至都指揮卒,輟朝二日。永樂後更定,惟公、侯、駙馬、伯及一品官,輟朝一日。

○中宮為父祖喪儀

凡中宮父母薨,訃報太常寺,轉報內使監。前期,設薦於別殿東壁下,為皇后舉哀位及內命婦以下哭位。皇后出詣別殿,內使監令奏:「考某官以某月某日薨」,母則云「妣某夫人」,祖考、妣同。皇后哭,內命婦以下皆哭盡哀。皇后問故,又哭盡哀。乃素服,內命婦皆素服,止哭,還宮。

內使監令奏聞。得旨:「皇后奔喪。」喪家設薦席於喪寢之東,從臨內命婦哭位於其下,主喪以下哭位於喪寢之西,主婦以下哭位於喪寢之北幔下。至日,內使監進堊車,備儀仗導引。皇后素服出宮,升輿,三面白布行帷。至閣外,降輿,升堊車。至喪家大門內,降車哭入,仍以行帷圍護。從臨者皆哭入。喪主以下,降詣西階下立哭。皇后升自東階,進至尸東,憑尸哭。從臨者皆哭。喪主升自西階,俱哭於尸西。皇后至哭位,內使監令跪請止哭。應奉慰者詣皇后前,奉慰如常禮。如皇后候成服,則從臨命婦應還者先還。如本日未即奔喪,則是晡復哭於別殿。尚服製皇后齊衰及從臨命婦孝服,俟喪家成服日進之。詣靈前再拜,上香,復位,再拜。如為諸王外戚舉哀,仍於別殿南向,不設薦位。

○遣使臨弔儀

太常司奉旨遣弔。前期,設宣制位於喪家正廳之北,南向;喪主受弔位於南,北向;婦人立哭位於殯北幕下。其日,使者至。喪主去杖,免絰衰服,止哭,出迎於中門外。復先入,就廳前拜位。內外止哭,使者入,就位稱有制。喪主以下再拜跪。宣制曰;「皇帝聞某官薨,遣臣某弔。」喪主以下復再拜。禮畢,內外皆哭。使者出,喪主至中門外,拜送,杖哭而入。宮使則稱有令。至遣使賻贈及致奠,其儀節亦相仿云。賻贈之典,一品米六十石,麻布六十匹。二品以五,三品、四品以四,五品、六品以三,公侯則以百。永樂後定制,公、侯、駙馬、伯皆取上裁。凡陣亡者全支,邊遠守禦出征及出海運糧病故半支。

其遣百官會王公大臣喪儀。前期,有司於喪家設位次。其日,百官應會弔者素服至。喪主以下就東階哭位,主婦以下就殯北哭位。百官入,就殯前位哭,主喪主婦以下皆哭。止哭,再拜,主喪以下答拜。班首詣喪主前展慰畢,百官出,喪主拜送,杖哭而入。會葬儀同。

○遣使冊贈王公大臣儀

前期,禮部奏請製冊,翰林院取旨製文,中書省禮部奏請某官為使。其日,祠祭司設龍亭、香亭於午門前正中,執事於受冊者家設宣制官位於正廳之東北,南向;喪主代受冊命者位於廳前,北向。禮部官封冊文,以盝匣盛之,黃袱裹置龍亭中。儀仗、鼓樂前導,至其家。代受冊者出迎於大門外。執事舁龍亭置廳上正中,使者入,立於東北。代受冊者就拜位,再拜。使者稍前,稱「有制」。代受冊者跪。宣制曰:「皇帝遣臣某,冊贈故某官某為某勛某爵。」宣訖,代受冊者復再拜。使者取冊授之,代受冊者捧置靈座前。使者出,代受冊者送至大門外。如不用冊者,吏部用誥命,喪家以冊文錄黃,設祭儀於靈前。代受冊者再拜,執事者展黃立讀於左。喪主以下皆再拜,焚黃。

○賜祭葬

洪武十四年九月,衍聖公孔希學卒,遣官致祭。其後,群臣祭葬,皆有定制。太祖諭祭群臣文,多出御筆。嘉靖中,世宗為禮部尚書席書、兵部尚書李承勛親製祭文。皆特典,非常制也。

隆慶元年十二月,禮部議上恤典條例:凡官員祭葬,有無隆殺之等,悉遵《會典》。其特恩,如侍從必日侍講讀、軍功必躬履行陣、東宮官必出閣講授有勞者。據嘉靖中事例,祭葬加一等,無祭者與祭一壇,無葬者給半葬,半葬者給全葬。講讀官五品本身有祭,四品及父母,三品及妻。軍功四品得祭葬,三品未滿及父母。講讀年久、啟沃功多、軍旅身殲、勳勞茂著者,恩恤加厚,臨期請旨。

《會典》,凡一品官,祭九壇。父母妻加祭。或二壇、一壇,或妻止一壇者,恩難預擬,遇有陳乞,酌擬上請。二品,二壇。加東宮三少,或兼大學士贈一品者,至四壇,父母妻俱一壇,致仕加三少者加一壇,加太子太保者加三壇,妻未封夫人者不祭。三品祭葬,在任、致仕俱一壇,兼學士贈尚書者二壇,未及考滿病故者一壇減半。造葬悉如舊例。四、五品官不得重封。故四品官由六七品升者,父母有祭。由五品升者,以例不重封,遂不得祭。今定四品官,凡經考滿者,父母雖止授五品封,亦與祭一壇。四品以上官,本身及父母恩典,必由考滿而後得。然有二品、三品共歷四五年,父母未授三品封,終不得霑一祭者,宜並敘年資。二品、三品共歷三年以上者,雖未考三品滿,本身及父母俱與三品祭葬。三品四品,共歷三年以上者,雖未考四品滿,本身用三品未考滿例,祭一壇半,葬父母祭一壇。凡被劾閒住者,雖遇覃恩,復致仕,仍不給祭葬。

勳臣祭葬,皇親出自上裁。駙馬都尉祭十五壇。公、侯、伯在內掌府事坐營、在外總兵有殊勳加太子太保以上者,遵《會典》。公、侯十六壇,伯十五壇,掌府坐營總兵有勳勞者七壇,掌府坐營年勞者五壇,掌府坐營而政跡未著者四壇,管事而被劾勘明閒住者二壇,被劾未經勘實者一壇。勘實罪重者,并本爵應得祭葬皆削。又正德間,公、侯、伯本祭俱三壇,嘉靖間二壇。今遵嘉靖例,以復《會典》之舊。武臣祭葬,遵正德、嘉靖例,都督同知僉事、錦衣衛指揮祭三壇,署都督同知僉事一壇,餘推類行之。

帝從其議。萬曆六年更定,凡致仕養病終養聽用等官,祭葬俱與現任官同。十二年續定,被劾自陳致仕官,有日久論定原無可議者,仍給祭葬,父母妻視本身為差等。

○喪葬之制

洪武五年定。凡襲衣,三品以上三,四品、五品二,六品以下一。飯含,五品以上飯稷含珠,九品以上飯粱含小珠。銘旌、絳帛,廣一幅,四品以上長九尺,六品以上八尺,九品以上七尺。斂衣,品官朝服一襲,常服十襲,衾十番。靈座設於柩前,作白絹結魂帛以依神。棺槨,品官棺用油杉朱漆,槨用土杉。牆翣,公、侯六,三品以上四,五品以上二。明器,公、侯九十事,一品、二品八十事,三品、四品七十事,五品六十事,六品、七品三十事,八品、九品二十事。引者,引車之紼也;披者,以纁為之,擊於輀車四柱,在旁執之,以備傾覆者也;鐸者,以銅為之,所以節挽歌者。公、侯四引六披,左右各八鐸。一品、二品三引四披,左右各六鐸。三品、四品二引二披,左右各四鐸。五品以下,二引二披,左右各二鐸。羽幡竿長九尺,五品以上,一人執之以引柩,六品以下不用。功布,品官用之,長三尺。方相,四品以上四目,七品以上兩目,八品以下不用。柳車上用竹格,以彩結之,旁施帷幔,四角重流蘇。誌石二片,品官皆用之。其一為蓋,書某官之墓;其一為底,書姓名、鄉里、三代、生年、卒葬月日及子孫、葬地。婦人則隨夫與子孫封贈。二石相向,鐵束埋墓中。祭物,四品以上羊豕,九品以上豕。

初,洪武二年,敕葬開平王常遇春於鐘山之陰,給明器九十事,納之墓中。鉦二,鼓四,紅旗,拂子各二,紅羅蓋、鞍、籠各一,弓二,箭三,灶、釜、火爐各一,俱以木為之。水罐、甲、頭盔、臺盞、杓、壺、瓶、酒甕、唾壺、水盆、香爐各一,燭臺二,香盒、香匙各一,香箸二,香匙箸瓶、茶鐘、茶盞各一,箸二,匙二,匙箸瓶一,碗二,楪十二,橐二,俱以錫造,金裹之。班劍、牙仗各一,金裹立瓜、骨朵戟、響節各二,交椅、腳踏、馬杌各一,誕馬六,槍、劍、斧、弩、食桌、床、屏風、柱杖、箱、交床、香桌各一,凳二,俱以木為之。樂工十六,執儀伏二十四,控士六,女使十,青龍、白虎、朱雀、玄武神四,門神二,武士十,并以木造,各高一尺。雜物,翣六,璧一,筐、笥、楎、椸、衿、鞶各一,笣二,筲二,糧漿瓶二,油瓶一,紗廚、煖帳各一。束帛青三段,纁二段,每段長一丈八尺。後定制,公、侯九十事者準此行之。餘以次減殺。

○碑碣

明初,文武大臣薨逝,例請於上,命翰林官製文,立神道碑。惟太祖時中山王徐達、成祖時榮國公姚廣孝及弘治中昌國公張巒治先塋,皆出御筆。其制自洪武三年定。五品以上用碑,龜趺螭首。六品以下用碣,方趺圓首。五年,復詳定其制。功臣歿後封王,螭首高三尺二寸,碑身高九尺,廣三尺六寸,龜趺高三尺八寸。一品螭首,二品麟鳳蓋,三品天祿辟邪蓋,四品至七品方趺。首視功臣歿後封王者,遞殺二寸,至一尺八寸止。碑身遞殺五寸,至五尺五寸止。其廣遞殺二寸,至二尺二寸止。趺遞殺二寸,至二尺四寸止。

墳塋之制,亦洪武三年定。一品,塋地周圍九十步,墳高一丈八尺。二品,八十步,高一丈四尺。三品,七十步,高一丈二尺。以上石獸各六。四品,四十步。七品以下二十步,高六尺。五年重定。功臣歿後封王,塋地周圍一百步,墳高二丈,四圍墻高一丈,石人四,文武各二,石虎、羊、馬、石望柱各二。一品至六品塋地如舊制,七品加十步。一品墳高一丈八尺,二品至七品遞殺二尺。一品墳墻高九尺,二品至四品遞殺一尺,五品四尺。一品、二品石人二,文武各一,虎、羊、馬、望柱各二。三品四品無石人,五品無石虎,六品以下無。

當太祖時,盱眙揚王墳置守戶二百一十,宿州徐王墳置墳戶九十三,滁州滁陽王墳亦置墳戶。四年,又賜功臣李善長、徐達、常茂、馮勝墳戶百五十,鄧愈、唐勝宗、陸仲亨、華雲龍、顧時、陳德、耿炳文、吳楨、孫恪、郭興墳戶百。成化十五年,南京禮部言:「常遇春、李文忠等十四人勳臣墳墓,俱在南京城外,文忠曾孫萼等,以歲久頹壞為言,請命工修治。」帝可其奏,且令無子孫者,復墓旁一人守護之。

○賜謚

親王例用一字;郡王二字,文武大臣同。與否自上裁。若官品未高而侍從有勞,或以死勤事者,特賜謚,非常例。洪武初,有應得謚者,禮部請旨,令禮部行翰林院擬奏。弘治十五年定制,凡親王薨,行撫、按,郡王病故,行本府親王及承奉長史,核勘以奏,乃議謚。文武大臣請謚,禮部取旨,行吏兵部考實蹟。禮部定三等,行業俱優者為上,頗可者為中,行實無取者為下,送翰林院擬謚。有應謚而未得者,撫、按、科道官以聞。

按明初舊制,謚法自十七字至一字,各有等差。然終高帝世,文臣未嘗得謚,武臣非贈侯伯不可得。魯、秦二王曰荒、曰愍。至建文謚王禕,成祖謚胡廣,文臣始有謚。迨世宗則濫及方士,且加四字矣。定例,三品得謚,詞臣謚「文」。然亦有得謚不止三品,謚「文」不專詞臣者,或以勳勞,或以節義,或以望實,破格崇褒,用示激勸。其冒濫者,亦間有之。

萬曆元年,禮臣言:「大臣應得謚者,宜廣詢嚴核。應謚而未請者,不拘遠近,撫、按、科道舉奏,酌議補給。」十二年,禮臣言:「大臣謚號,必公論允服,毫無瑕疵者,具請上裁。如行業平常,即官品雖崇,不得概予。」帝皆從之。三十一年,禮部侍郎郭正域請嚴謚典。議奪者四人:許論、黃光昇、呂本、范廉;應奪而改者一人:陳瓚;補者七人:伍文定、吳悌、魯穆、楊繼宗、鄒智、楊源、陳有年。閣臣沈一貫、朱賡力庇呂本,不從其議。未幾,御史張邦俊請以呂柟從祀孔廟,而論應補謚者,雍泰、魏學曾等十四人。部議久之,共匯題先後七十四人,留中不發。天啟元年,始降旨俞允,又增續請者十人,而邦俊原請九人不與。正域所請伍文定等亦至是始定。凡八十四人。其官卑得謚者,鄒智、劉臺、魏良弼、周天佐、楊允繩、沈煉、楊源、黃鞏、楊慎、周怡、莊鷫、馮應京皆以直諫,孟秋、張元忭、曹端、賀欽、陳茂烈、馬理、陶望齡皆以學行,張銓以忠義,李夢陽以文章,魯穆、楊繼宗、張朝瑞、朱冠、傅新德、張允濟皆以清節,楊慎之文憲,莊鷫之文節,則又兼論文學云。

三年,禮部尚書林堯俞言:「謚典五年一舉,自萬曆四十五年至今,蒙恤而未謚者,九卿臺省會議與臣部酌議。」帝可之。然是時,遲速無定。六年,禮科給事中彭汝楠言:「耳目近則睹記真,宜勿逾五年之限。」又謂:「三品以上為當予謚,而建文諸臣之忠義,陶安等之參帷幄,葉琛等之殉行間,皆宜補謚。」事下禮部,以建文諸臣未易輕擬,不果行。至福王時,始從工科給事中李清言,追謚開國功臣李善長等十四人,正德諫臣蔣欽等十四人,天啟慘死諸臣左光斗等九人,而建文帝之弟允熥、允AR、允熙,子文奎,亦皆因清疏追補。

○品官喪禮

品官喪禮載在《集禮》、《會典》者,本之《儀禮·士喪》,稽諸《唐典》,又參以朱子《家禮》之編,通行共曉。茲舉大要,其儀節不具錄。

凡初終之禮,疾病,遷於正寢。屬纊,俟絕氣乃哭。立喪主、主婦,護喪以子孫賢能者。治棺訃告。設必尸床、帷堂,掘坎。設沐具,沐者四人,六品以下三人,乃含。置虛座,結魂帛,立銘旌。喪之明日乃小斂,又明日大斂,蓋棺,設靈床於柩東。又明日,五服之人各服其服,然後朝哭相弔。既成服,朝夕奠,百日而卒哭。乃擇地,三月而葬。告后土,遂穿壙。刻志石,造明器,備大舉,作神主。既發引,至墓所,乃窆。施銘旌誌石於壙內,掩壙復土,乃祠后土於墓。題主,奉安。升車,反哭。

凡虞祭,葬之日,日中而虞,柔日再虞,剛日三虞。若去家經宿以上,則初虞於墓所行之。墓遠,途中遇柔日,亦於館所行之。若三虞,必俟至家而後行。三虞後,遇剛日卒哭。

明日祔家廟。期而小祥。喪至此凡十三月,不計閏。古卜日祭,今止用初忌,喪主乃易練服。再期而大祥。喪至此凡二十五月,亦止用第二忌日祭。陳禫服,告遷於祠堂。改題神主,遞遷而西,奉神主入於祠堂。徹靈座,奉遷主埋於墓側。大祥後,間一月而禫。喪到此計二十有七月。卜日,喪主禫服詣祠堂,祗薦禫事。

其在遠聞喪者,始聞,易服,哭而行。至家,憑殯哭,四日而成服。若未得行,則設位,四日而變服。若既葬,則先哭諸墓,歸詣靈座前哭,四日成服。齊衰以下聞喪,為位而哭。若奔喪,則至家成服。若不奔喪,四日成服。凡有改葬者,孝子以下及妻、妾、女子子,俱緦麻服,周親以下素服。不設祖奠,無反哭,無方相魌頭,餘如常葬之儀。既葬,就吉帷靈座前一虞。孝子以下,出就別所,釋緦服素服而還。

洪武二十六年四月,除期服奔喪之制。先是百官聞祖父母、伯叔、兄弟喪,俱得奔赴。至是吏部言:「祖父母、伯叔、兄弟皆係期年服。若俱令奔喪守制,或一人連遭五六期喪,或道路數千里,則居官日少,更易繁數,曠官廢事。今後除父母、祖父母承重者丁憂外,其餘期喪不許奔,但遣人致祭。」從之。

○士庶人喪禮

《集禮》及《會典》所載,大略仿品官制,稍有損益。洪武元年,御史高元侃言:「京師人民,循習舊俗。凡有喪葬,設宴,會親友,作樂娛尸,竟無哀戚之情,甚非所以為治。乞禁止以厚風化。」乃令禮官定民喪服之制。

五年詔定:「庶民襲衣一稱,用深衣一、大帶一、履一雙,裙褲衫襪隨所用。飯用粱,含錢三。銘旌用紅絹五尺。斂隨所有,衣衾及親戚禭儀隨所用。棺用堅木,油杉為上,柏次之,土杉松又次之。用黑漆、金漆,不得用朱紅。明器一事。功布以白布三尺引柩。柳車以衾覆棺。誌石二片,如官之儀。塋地圍十八步。祭用豕,隨家有無。」又詔:「古之喪禮,以哀戚為本,治喪之具,稱家有無。近代以來,富者奢僭犯分,力不足者稱貸財物,誇耀殯送,及有惑於風水,停柩經年,不行安葬。宜令中書省臣集議定制,頒行遵守,違者論罪。」又諭禮部曰:「古有掩骼埋胔之令,近世狃元俗,死者或以火焚,而投其骨於水。傷恩敗俗,莫此為甚。其禁止之。若貧無地者,所在官司擇寬閒地為義塚,俾之葬埋。或有宦遊遠方不能歸葬者,官給力費以歸之。」

○服紀

明初頒《大明令》,凡喪服等差,多因前代之舊。洪武七年,《孝慈錄》成,復圖列於《大明令》,刊示中外。

先是貴妃孫氏薨,敕禮官定服制。禮部尚書牛諒等奏曰:「周《儀禮》,父在,為母服期年,若庶母則無服。」太祖曰:「父母之恩一也,而低昂若是,不情甚矣。」乃敕翰林院學士宋濂等曰;「養生送死,聖王大政。諱亡忌疾,衰世陋俗。三代喪禮散失於衰周,厄於暴秦。漢、唐以降,莫能議此。夫人情無窮,而禮為適宜。人心所安,即天理所在。爾等其考定喪禮。」於是濂等考得古人論服母喪者凡四十二人,願服三年者二十八人,服期年者十四人。太祖曰:「三年之喪,天下通喪。觀願服三年,視願服期年者倍,豈非天理人情之所安乎?」乃立為定制。子為父母,庶子為其母,皆斬衰三年。嫡子、眾子為庶母,皆齊衰杖期。仍命以五服喪制,并著為書,使內外遵寧。其制服五。曰斬衰,以至粗麻布為之,不縫下邊。曰齊衰,以稍粗麻布為之,縫下邊。曰大功,以粗熟布為之。曰小功,以稍粗熟布為之。曰緦麻,以稍細熟布為之。

其敘服有八。曰斬衰三年者:子為父母,庶子為所生母,子為繼母,謂母卒父命他妾養己者,子為養母,謂自幼過房與人者;女在室為父母,女嫁被出而反在室為父母;嫡孫為祖父母承重及曾高祖父母承重者;為人後者為所後父母,及為所後祖父母承重;夫為後則妻從服,婦為舅姑;庶子之妻為夫之所生母;妻妾為夫。

曰齊衰杖期者:嫡子眾子為庶母;嫡子眾子之妻為夫之庶母,為嫁母、出母、父卒繼母改嫁而已從之者;夫為妻。

曰齊衰不杖期者:父母為嫡長子及眾子,父母為女在室者,繼母為長子及眾子,慈母為長子及眾子;孫為祖父母,孫女雖適人不降,高曾皆然;為伯叔父母;妾為夫之長子及眾子,為所生子;為兄弟,為兄弟之子及兄弟之女在室者,為姑及姊妹在室者;妾為嫡妻;嫁母、出母為其子;女在室及雖適人而無夫與子者,為其兄弟及兄弟之子;繼母改嫁為前夫之子從己者;為繼父同居兩無大功之親者;婦人為夫親兄弟之子,婦人為夫親兄弟子女在室者;女出嫁為父母;妾為其父母;為人後者為其父母;女適人為兄弟之為父後者;祖為嫡孫;父母為長子婦。

曰齊衰五月者:為曾祖父母。

曰齊衰三月者:為高祖父母,為繼父昔同居而今不同者,為繼父雖同居而兩有大功以上親者。

曰大功九月者:為同堂兄弟及姊妹在室者,為姑及姊妹及兄弟之女出嫁者;父母為眾子婦,為女之出嫁者;祖為眾孫;為兄弟之子婦;婦人為夫之祖父母,為夫之伯叔父母,為夫之兄弟之子婦,為夫兄弟之女嫁人者;女出嫁為本宗伯叔父母,及為兄弟與兄弟之子,為姑姊妹及兄弟之女在室者;為人後者為其兄弟及姑姊妹在室者;妻為夫本生父母;為兄弟之子為人後者。

曰小功五月者:為伯叔祖父母,為同堂伯叔父母,為再從兄弟及再從姊妹在室者,為同堂兄弟之子,為祖姑在室者,為從祖姑在室者,為同堂兄弟之女在室者,為兄弟之妻;為人後者為其姑姊妹適人者;為嫡孫婦,為同堂姊妹之出嫁者,為孫女適人者,為兄弟之孫及兄弟之女孫在室者,為外祖父母,為母之兄弟姊妹,為同母異父之兄弟姊妹,為姊妹之子;婦人為夫之姑及夫之姊妹,為夫之兄弟及夫兄弟之妻,為夫兄弟之孫及夫兄弟之女孫在室者,為夫同堂兄弟之子及同堂兄弟之女在室者。

曰緦麻三月者:為族曾祖父母,為族伯叔祖父母,為族父母,為族兄弟及族姊妹在室者,為族曾祖姑在室者,為族祖姑及族姑在室者,為兄弟之曾孫,女在室同,為曾孫玄孫,為同堂兄弟之孫,女在室同,為再從兄弟之子,女在室同,為祖姑、從祖姑及從祖姊妹之出嫁者,為兄弟之孫女出嫁者,為同堂兄弟之女出嫁者,為乳母,為舅之子,為姑之子,為姨之子,為外孫,為婿,為妻之父母,為兄弟孫之婦,為同堂兄弟子之婦,為同堂兄弟之妻,為外孫婦,為甥婦;婦人為夫之曾祖、高祖父母,為夫之叔伯祖父母,為夫之同堂伯叔父母,為夫兄弟之曾孫,為夫之同堂兄弟,為夫同堂兄弟之孫,孫女同,為夫再從兄弟之子,為夫兄弟之孫婦,為夫同堂兄弟子之婦,為夫同堂兄弟之妻,為夫同堂姊妹,為夫之外祖父母,為夫之舅及姨,為夫之祖姑及從祖姑在室者;女出嫁為本宗叔伯祖父母,為本宗同堂叔伯父母,為本宗同堂兄弟之子女,為本宗祖姑及從祖姑在室者,為本宗同堂姊妹之出嫁者;為人後者為本生外祖父母。

嘉靖十八年正月,諭輔臣:「昨居喪理疾,閱《禮記·檀弓》等篇,其所著禮儀制度俱不歸一,又不載天子全儀。雖曰『三年之喪,通乎上下』,而今昔亦有大不同者。皇祖所定,未有全文,每遇帝後之喪,亦未免因仍為禮。至於冠裳衰絰,所司之制不一,其與禮官考定之。自初喪至除服,冠裳輕重之制具為儀節,俾歸至當。」於是禮部議喪服諸制奏之。帝令更加考訂,畫圖注釋,並祭葬全儀,編輯成書備覽。

