\article{列傳第一百二十一}

\begin{pinyinscope}
姜應麟從子思睿陳登雲羅大摐黃正賓李獻可舒弘緒陳尚象丁懋遜吳之佳葉初春楊其休董嗣成賈名儒張棟孟養浩朱維京王如堅王學曾塗傑張貞觀樊玉衡子鼎遇維城孫自一謝廷贊兄廷諒楊天民何選馮生虞任彥蘗

姜應麟,字泰符,慈谿人。父國華,嘉靖中進士。歷陜西參議,有廉名。應麟舉萬曆十一年進士,改庶吉士,授戶科給事中。貴妃鄭氏有殊寵,生子常洵,詔進封為皇貴妃。而王恭妃育皇長子已五歲,無所益封。中外籍籍,疑帝欲立愛。十四年二月,應麟首抗疏言:「禮貴別嫌,事當慎始。貴妃所生陛下第三子猶亞位中宮,恭妃誕育元嗣翻令居下。揆之倫理則不順,質之人心則不安,傳之天下萬世則不正,非所以重儲貳、定眾志也。伏請俯察輿情,收還成命。其或情不容已,請先封恭妃為皇貴妃,而後及於鄭妃,則禮既不違,情亦不廢。然臣所議者末,未及其本也。陛下誠欲正名定分,別嫌明微,莫若俯從閣臣之請,冊立元嗣為東宮,以定天下之本,則臣民之望慰,宗社之慶長矣。」疏入,帝震怒,抵之地,遍召大璫諭曰:「冊封貴妃,初非為東宮起見,科臣奈何訕朕!」手擊案者再。諸璫環跪叩首,怒稍解,遂降旨:「貴妃敬奉勤勞,特加殊封。立儲自有長幼,姜應麟疑君賣直,可降極邊雜職。」於是得大同廣昌典史。吏部員外郎沈璟、刑部主事孫如法繼言之,並得罪。兩京申救者疏數十上,皆不省。自後言者蜂起,咸執「立儲自有長幼」之旨,以責信於帝。帝雖厭苦之,終不能奪也。

應麟居廣昌四年,量移餘干知縣。以父憂歸。服闋,至京,會吏部數以推舉建言諸臣得重譴,應麟遂不復補。家居二十年。光宗立,起太僕少卿。給事中薛鳳翔劾應麟老病失儀,遂引疾去。崇禎三年卒,贈太常卿。

從子思睿,字顓愚。少孤,事母孝。舉天啟二年進士,授行人。崇禎三年擢御史。明年春,陳天下五大弊:曰加派病民,曰郵傳過削,曰搜剔愈精,頭緒愈亂,曰懲毖愈甚,頹廢愈多,曰督責愈急,蒙蔽愈深。忤旨,切責。其冬遣宦官監視邊務,抗疏切諫。已,劾首輔周延儒以家人周文郁為副將,弟素儒為錦衣,叔父人瑞為中書,受賕行私,請罷斥。已,論救給事中魏呈潤、御史李曰輔、王績燦。巡按雲南。陛辭,歷指諸弊政,而言:「舉朝拯焚救溺之精神,專用之摘抉細微,而以察吏詰戎予奪大柄僅付二三閹寺。厝火自安,不知變計,天下安望太平!」忤旨,切責。還朝,值帝撤還二部總理諸鎮監視內臣。思睿請並撤監視京營關、寧者。因詆向來秉政大臣阿承將順之罪,意指溫體仁也。體仁二子儼、伉數請囑提學僉事黎元寬。會元寬以文體險怪論黜,遂發其二子私書。思睿劾體仁縱子作奸,以元寬揭為據。體仁謂揭不出元寬手,思睿等群謀排陷。元寬上疏證明,思睿再劾體仁以「群謀」二字成陷人之阱,但知有子,不知有君。帝怒,奪俸五月。出視河東鹽政。安邑有故都御史曹於汴講學書院,思睿為置田構學舍,公餘親蒞講授。代還,乞假歸里。未幾卒。

陳登雲,字從龍,唐山人。萬曆五年進士。除鄢陵知縣。政最,徵授御史。出按遼東,疏陳安攘十策,又請速首功之賞。改巡山西。還朝,會廷臣方爭建儲。登雲謂議不早決,由貴妃家陰沮之。十六年六月,遂因災異抗疏,劾妃父鄭承憲,言:「承憲懷禍藏奸,窺覬儲貳。日與貂璫往來,綢繆杯酌,且廣結山人、術士、緇黃之流。曩陛下重懲科場冒籍,承憲妻每揚言事由己發,用以恐喝勛貴,簧鼓朝紳。不但惠安遭其虐焰,即中宮與太后家亦謹避其鋒矣。陛下享國久長,自由敬德所致,而承憲每對人言,以為不立東宮之效。干撓盛典,蓄隱邪謀,他日何所不至。茍不震奮乾剛,斷以大義,雖日避殿撤樂、素服停刑,恐天心未易格,天變未可弭也。」疏入,貴妃、承憲皆怒,同列亦為登雲危,帝竟留中不下。

久之,疏論吏部尚書陸光祖,又論貶四川提學副使馮時可,論罷應天巡撫李淶、順天巡撫王致祥,又論禮部侍郎韓世能、尚書羅萬化、南京太僕卿徐用檢。朝右皆憚之。時方考選科道,登雲因疏言:「近歲言官,壬午以前怵於威,則摧剛為柔;壬午以後暱於情,則化直為佞。其間豈無剛直之人,而弗勝齟齬,多不能安其身。二十年來,以剛直擢京卿者,百止一二耳。背公植黨,逐嗜乞憐,如所謂『七豺』、『八狗』者,言路顧居其半。夫臺諫為天下持是非,而使人賤辱至此,安望其抗顏直繩,為國家鉏大奸、殲巨蠹哉!與其誤用而斥之,不若慎於始進。」因條數事以獻。

出按河南。歲大饑,人相食。副使崔應麟見民啖澤中雁矢,囊示登雲,登雲即進之於朝。帝立遣寺丞鐘化民齎帑金振之。登雲巡方者三,風裁峻厲。以久次當擢京卿,累寢不下,遂移疾歸。尋卒。

羅大紘,字公廓,吉水人。萬曆十四年進士。授行人。十九年八月,遷禮科給事中。甫拜命,即上《定制書》數千言。已,復言視朝宜勤,語皆切直。先有詔以二十年春冊立東宮,至是工部主事張有德以預備儀物請。帝怒,命奪俸三月,更緩冊立事。尚書曾同亨請如前詔,忤旨,切讓。大紘復以為言,詔奪俸如有德。大學士許國、王家屏連署閣臣名,乞收新命,納諸臣請,帝益怒。首輔申時行方在告,聞帝怒,乃密揭言:「臣雖列名公疏,實不與知。」帝喜,手詔褒答,而揭與詔俱發禮科。故事,閣臣密揭無發科者。時行慚懼,亟謀之禮科都給事中胡汝寧,遣使取揭。時獨大紘守科,使者紿取之。及往索,時行留不發。大紘乃抗疏曰:「臣奉職無狀,謹席稿以待。獨念時行受國厚恩,乃內外二心,藏奸蓄禍,誤國賣友,罪何可勝言。夫時行身雖在告,凡翰林遷改之奏,皆儼然首列其名,何獨於建儲一事深避如此。縱陛下赫然震怒,加國等以不測之威,時行亦當與分過。況陛下未嘗怒,而乃沮塞睿聰,搖動國本,茍自獻其乞憐之術,而遏主上悔悟之萌,此臣之所大恨也。假令國等得請,將行慶典而恩澤加焉,時行亦辭之乎?蓋其私心妄意陛下有所牽繫,故陽附廷臣請立之議,而陰緩其事,以為自交宮掖之謀。使請之而得,則明居羽翼之功;不得,則別為集菀之計。其操此術以愚一世久矣,不圖今日乃發露之也。」疏入,帝震怒,命貶邊方雜職。俄以六科鐘羽正等論救,斥為民,羽正等奪俸。中書舍人黃正賓復抗疏力詆時行。帝怒,下獄拷訊,斥為民。時行亦不安,無何,竟引去。大紘志行高卓。鄉人以配里先達羅倫、羅洪先,號為「三羅」。天啟中,贈光祿少卿。

正賓,歙人。以貲為舍人,直武英殿。恥由貲入官,思樹奇節,至是遂見推清議。後李三才、顧憲成咸與遊,益有聲士大夫間。熹宗立,起故官。再遷尚寶少卿,引病歸。魏忠賢下汪文言獄,詞連正賓。坐贓千金,遣戍大同。莊烈帝嗣位,復官,致仕。崇禎元年六月,魏黨徐大化、楊維垣已罷官,猶潛居輦下,交通奄寺,正賓在都,抗疏發其奸。勒兩人歸田里,都人快之。而疏有「潛通宦寺」語,帝令指名。正賓以趙倫、於化龍對。帝以其妄,斥回籍。

李獻可,字堯俞,同安人。萬曆十一年進士。除武昌推官。課最,徵授戶科給事中。屢遷禮科都給事中。二十年正月,偕六科諸臣疏請豫教,言:「元子年十有一矣,豫教之典當及首春舉行。倘謂內庭足可誦讀,近侍亦堪輔導,則禁闥幽閒,豈若外朝之清肅;內臣忠敬,何如師保之尊嚴。」疏入,帝大怒,摘疏中誤書弘治年號,責以違旨侮君,貶一秩調外,餘奪俸半歲。大學士王家屏封還御批,帝益不悅。吏科都給事中鐘羽正言:「獻可之疏,臣實贊成之,請與同謫。」吏科給事中舒弘緒亦言「言官可罪,豫教必不可不行」。帝益怒,出弘緒南京,而羽正及獻可並以雜職徙邊方。大學士趙志皋論救,被旨譙讓。吏科右給事中陳尚象復爭之,坐斥為民。戶科左給事中孟養浩,御史鄒德泳,戶兵刑工四科都給事中丁懋遜、張棟、吳之佳、楊其休,禮科左給事中葉初春,各上疏救。帝益怒,廷杖養浩百,除其名。德泳、懋遜等六人並貶一秩,出之外。獻可、羽正、弘緒亦除名。

當是時,帝一怒而斥諫官十一人,朝士莫不駭歎,然諫者卒未已。禮悅員外郎董嗣成、御史賈名儒特疏爭之,御史陳禹謨、吏科左給事中李周策亦偕其僚論諫。帝怒加甚,奪嗣成職,名儒謫邊方,德泳、懋遜等咸削籍,禹謨等停俸有差。禮部尚書李長春等亦疏諫,帝復詰讓。獻可等遂廢於家。久之,吏部尚書蔡國珍、侍郎楊時喬先後請收敘,咸報寢。

天啟初,錄先朝言事諸臣。獻可已前卒,詔贈光祿卿。

弘緒、名儒皆獻可同年進士。尚象、懋遜、之佳、初春、其休、嗣成皆萬曆八年進士。

弘緒,通山人。由庶吉士改給事中。天啟中,贈光祿少卿。

尚象,都勻人。以中書舍人為給事中。嘗劾罷尚書沈鯉,為士論所非。至是以直言去,國人始稱焉。天啟中,贈官如弘緒。

懋遜,霑化人。為餘姚知縣,有治績,入為吏科給事中。既削籍,里居三十年。光宗立,起太僕少卿,累遷工部左侍郎。卒,贈尚書。

之佳,長洲人。初為襄陽知縣。初春,吳縣人。初為順德知縣。並以治行征。至是與張棟並斥,稱「吳中三諫」。天啟初,贈之佳太僕少卿,初春光祿少卿。之佳孫適,亦兵科給事中。敢言。

其休,青城人。由蘇州推官擢吏科給事中。內官張德毆殺人,帝令司禮按問,蔽罪其下。其休乞並付德法司,竟報許。帝數不視朝。十七年正月,其休以萬邦入覲,請臨御以風勵諸臣。他論奏甚眾。罷歸,卒,贈太常少卿。

嗣成,烏程人。祖份,禮部尚書。父道醇,南京給事中。仍世貴顯。嗣成以氣節著,士論多之。

名儒,真定人。贈官如初春。

棟,字伯任,崑山人。萬曆五年進士。除新建知縣。征授工科給事中。請盡蠲天下逋租,格不行。時蠲租例,相沿但蠲存留,不及起運。棟請無拘故事,從之。再遷刑科左給事中。吳中白糧為累,民承役輒破家,棟請令出貲助漕舟附載。申時行、王錫爵絀其議,棟遂移疾歸。起兵科都給事中。劾去南京戶部尚書張西銘、刑部侍郎詹仰庇。軍政拾遺,劾恭順侯吳繼爵、宣城伯衛國本、忻城伯趙泰修、宣府總兵官李迎恩。繼爵留,餘並罷。已,言邊臣敘功不宜及內閣、部、科,帝亦從焉。遣視固原邊備。時經略鄭洛方議和,棟言撦力克負固不歸,卜失兔傑黠如故,火落赤、真相雄據海上,不可使洛委責以去。因論兵部尚書王一鶚。會一鶚已卒,洛亦報撦力克東歸,遂寢其奏。棟又言:「洮、河失事,陛下赫然震怒。命洛視師,豈止欲其虛詞媚敵,博一順義東歸畢事耶?今火、真依海為窟,出沒自如,不宜敘將吏功。」報聞。母卒,棟年已六十,毀瘠廬墓,竟卒於墓所。天啟中,贈太常少卿。

德泳,祭酒守益孫。養浩、羽正自有傳。

孟養浩,字義甫,湖廣咸寧人。萬曆十一年進士。授行人。擢戶科給事中,遷左給事中。帝嚴譴李獻可,養浩疏諫曰:「人臣即至狂悖,未有敢於侮君者,陛下豈真以其侮而罪之耶?獻可甫躋禮垣,驟議巨典。一字之誤,本屬無心,乃遽蒙顯斥。臣愚以為有五不可。元子天下本,豫教之請,實為宗社計。陛下不惟不聽,且從而罰之,是坐忍元子失學,而敝帚宗社也。不可者一。長幼定序,明旨森嚴,天下臣民既曉然諒陛下之無他矣。然豫教、冊立?本非兩事。今日既遲回於豫教,安知來歲不游移於冊立,是重啟天下之疑。不可者二。父子之恩,根於天性,豫教之請,有益元子明甚。而陛下罪之,非所以示慈愛。不可者三。古者引裾折檻之事,中主能容之。陛下量侔天地,奈何言及宗社大計,反震怒而摧折之?天下萬世謂陛下何如主?不可者四。獻可等所論,非二三言官之私言,實天下臣民之公言也。今加罪獻可,是所罪者一人,而實失天下人之心。不可者五。祈陛下收還成命,亟行豫教。」帝大怒,言冊立已諭於明年舉行,養浩疑君惑眾,殊可痛惡。令錦衣衛杖之百,削籍為民,永不敘用。中外交薦,悉報寢。光宗立,起太常少卿。半歲中遷至南京刑部右侍郎。未之官,卒。

朱維京,字大可,工部尚書衡子也。舉萬歷五年進士,授大理評事,進右寺副。九年京察,謫汝州同知,改知崇德。入為屯田主事,再遷光祿丞。火落赤敗盟,經略鄭洛主和,督撫魏學曾、葉夢熊主戰。維京請召洛還,專委學曾等經理。及學曾以寧夏事被逮,復抗疏救之。

二十一年,三王並封詔下,維京首上疏曰:「往奉聖諭,許二十一年冊立,廷臣莫不延頸企踵。今忽改而為分封,是向者大號之頒,徒戲言也,何以示天下?聖諭謂立嗣以嫡,是已。但元子既長,欲少遲冊立,以待中宮正嫡之生,則祖宗以來,實無此制。考英宗之立,以宣德三年;憲宗之立,以正統十四年;孝宗之立,以成化十一年。少者止一二齡,多亦不過五六齡耳。維時中宮正位,嫡嗣皆虛,而祖宗曾不少待。即陛下冊立,亦在先帝二年之春。近事不遠,何不取而證之。且聖人為政,必先正名。今分封之典,三王並舉,冠服宮室混而無別,車馬儀仗雜而無章,府僚庶寀淆而無辨。名既不正,弊實滋多。且令中宮茍耀前星,則元子退就籓服,嫡庶分定,何嫌何疑。今預計將來,坐格成命,是欲愚天下,而實以天下為戲也。夫人臣以道事君,不可則止。陛下雖有並封之意,猶不遽行,必以手詔咨大學士王錫爵,錫爵縱不能如李沆引燭之焚,亦當為李泌造膝披陳,轉移聖心而後已。如其不然,王家屏之高蹤自在,陛下優禮輔臣,必無韓瑗、來濟之辱也。奈何噤無一語,若胥吏之承行,惟恐或後。彼楊素、李勣千古罪人,其初心豈不知有公論,惟是患得患失之心勝,遂至不能自持耳。」帝震怒,命謫戍極邊。錫爵力救,得為民。家居甫二年,卒。天啟時,贈太常少卿。

王如堅,字介石,安福人。萬曆十四年進士。授懷慶推官。入為刑科給事中,抗疏爭三王並封,其略曰:

謹按十四年正月聖諭「元子幼小,冊立事俟二三年舉行」,是明言長子之為元子也。又十八年正月詔旨「朕無嫡子,長幼自有定序」,是明示倫次之不可易也。已而十九年八月,奉旨「冊立之事,改於二十一年舉行」,此則陛下雖怒群臣激聒,輒更定期,未嘗遽寢冊立之事。乃今已屆期,忽傳並封為王,以待嫡嗣。臣始而疑,既而駭。陛下言猶在耳,豈忘之耶?曩者謂二三年舉行,已遲至二十年矣,二十年舉行又改至二十一年矣,今二十一年倏改為並封,是陛下前此灼然之命,尚不自堅,今日群臣,將何所取信?

夫立嫡之條,《祖訓》為廢嫡者戒也。今日有嫡可廢乎?且陛下欲待正嫡,意非真待也。古王者後宮無偏愛,故適后多後嗣。後世愛有所專,則天地之交不常泰,欲後嗣之繁難矣。我祖宗以來,中宮誕生者有幾?國本早定,惟元子是屬。或二三齡而立,或五六齡而立。即陛下春宮受冊時,止六齡耳,寧有待嫡之議與潞王並封之詔哉?今皇長子且十二齡矣,聞皇后撫育無間己出。元子早定一日,即早慰中宮一日之心。后素賢明,何有舍當前之冢嗣,而覬幸不可知之數耶?宮闈之內,衽席之間,左右近習之輩,見形生疑,未必不以他意窺陛下。即如昨歲冊立之旨,方待舉行,而宗室中已有並封之疏,安知非機事外洩,彼得量朝廷之淺深?

夫別名號,辨嫌疑,禮之善經也。元子與眾子,其間冠服之制,齒簿之節,恩寵之數,接見之儀,迥然不齊矣。一日並封而同號,則有並大之嫌,逼長之患。執狐疑而來讒賊,幾微之際,不可不慎。茍謂渙命新頒,難於遽改,則數年已定之明旨,尚可移易,今綸言初發,何不可中止也。

帝怒甚,命與朱維京皆戍極邊。王錫爵疏救,免戍為民。尋卒。天啟中,贈光祿少卿。

王學曾,字唯吾,南海人。萬曆五年進士。授醴陵知縣,調崇陽。擢南京御史。時吏民有罪,輒遣官校逮捕。學曾疏請止之,不納。十三年,慈寧宮成,諸督工內侍俱廕錦衣。學曾論其太濫,且劾工部尚書楊兆諛諂中官。兆惶恐,引罪。已,言龍江關密邇蕪湖,蕪湖已征稅,龍江不宜復徵,格不行。光山牛產一犢若麟,有司欲以聞,巡撫臧惟一不可。帝命禮部征之,尚書沈鯉諫,惟一亦疏論,不聽。學曾抗言:「麟生牛腹,次日即斃,則祥者已不祥矣。不祥之物,所司未嘗上聞,陛下何自聞之?毋亦左右小人以奇怪惑聖心也?今四方災旱,老稚流離,啼饑號寒之聲,陛下不聞;北敵梟張,士卒困苦,呻吟嗟怨之狀,陛下不聞;宗室貧窮,饔餐弗給,愁困涕洟之態,陛下不聞;而獨已斃之麟聞。彼為左右者,豈誠忠於陛下乎?願收還成命,內臣語涉邪妄者,即嚴斥之。」帝責其要名沽直,降興國判官。時御史蔡時鼎亦以言獲罪。南京御史王籓臣、給事中王嗣美等交章救兩人。帝怒,奪俸一級。

學曾累遷南京刑部主事,召為光祿丞。與少卿塗傑合疏爭三王並封,忤旨,皆削籍。後數年,吏部尚書蔡國珍疏請起用,不納。卒於家。傑,新建人。隆慶五年進士。由龍游知縣入為御史。擢官光祿。熹宗時,贈學曾太僕少卿,傑太常少卿。

張貞觀,字惟誠,沛人。萬曆十一年進士。除益都知縣,擢兵科給事中。出閱山西邊務。五臺奸人張守清招亡命三千餘人,擅開銀礦,又締姻潞城、新寧二王。帝納巡按御史言,敕守清解散徒黨,諭二王絕姻。守清乞輸課於官,開礦如故。貞觀力爭,乃已。前巡撫沈子木、李采菲皆貪。子木夤緣為兵部侍郎,貞觀並追劾之。子木坐貶,採菲奪職。還,進工科右給事中。泗州淮水大溢,幾齧祖陵。貞觀往視,定分黃道淮之策。

再遷禮科都給事中。三王並封制下,貞觀率同列力爭。沈王珵堯由郡王進封,其諸弟止應為將軍,珵堯為營得郡王。貞觀及禮部尚書羅萬化守故事極諫。不納。時郊廟祭享率遣官代行,貞觀力請帝親祀。俄秋享,復將遣官。貞觀再諫,不報。明年正月,有詔皇長子出閣講讀。而兵部請護衛,工部奏儀仗,禮部進儀注,皆留中。又止令預告奉先殿,朝謁兩宮,他禮皆廢。於是貞觀等上言:「禮官議,御門受賀、皇長子見群臣之禮,載在舊儀;即諸王加冠,亦以成禮而賀,賀畢謁見。元子初出,乃不當諸王一冠乎?且謁謝止兩宮,而缺然於陛下及中宮母妃之前,非所以教孝;賀靳於二皇子,而漠然於兄弟長幼之間,非所以序別。」疏入,忤旨,奪俸一年。

工科給事中黎道照上言:「元子初就外傅,陛下宜示之身教。乃採辦珠玉珍寶,費至三十六萬有奇,又取太僕銀十萬充賞,非作法於初之意。且貞觀等秉禮直諫,職也,不宜罰治。」給事中趙完璧等亦言之。帝怒,奪諸臣俸,謫貞觀雜職。大學士王錫爵等切救,乃貶三秩。頃之,都給事中許弘綱、御史陳惟芝等連章申論,帝竟除貞觀名,言官亦停俸。中外交薦,卒不起。天啟中卒,贈太常少卿。

樊玉衡,字以齊,黃岡人。萬曆十一年進士。由廣信推官征授御史。京察,謫無為判官。稍遷全椒知縣。二十六年四月,玉衡以冊立久稽,上言:「陛下愛貴妃,當圖所以善處之。今天下無不以冊立之稽歸過貴妃者,而陛下又故依違,以成其過。陛下將何以託貴妃於天下哉?由元子而觀則不慈,由貴妃而觀則不智,無一可者。願早定大計,冊立、冠婚諸典次第舉行,使天下以元子之安為貴妃功,豈不並受其福,享令名無窮哉!」疏奏,帝及貴妃怒甚。旨一日三四擬,禍且不測。大學士趙志皋等力救,言自帝即位未嘗殺諫臣。帝乃焚其疏,忍而不發。再踰月,以《憂危竑議》連及,遂永戍雷州。長子鼎遇伏闕請代者再,不許。光宗立,起南京刑部主事,以老辭。疏陳親賢、遠奸十事,優詔答之。尋命以太常少卿致仕,卒於家。

子維城,舉萬曆四十七年進士。除海鹽知縣,遷禮部主事。天啟七年,坐事謫上林苑典簿。莊烈帝即位,魏忠賢未誅,抗疏言:「高皇帝定律,人臣非有大功,朦朧奏請封爵者,所司及封受之人俱斬。今魏良卿、良棟、鵬翼,白丁乳臭兒,並叨封爵,皆當按律誅。忠賢所積財,半盜內帑,籍還太府,可裕九邊數歲之餉。」因請褒恤楊漣、萬璟等一十四人,召還賀逢聖、文震孟、孫必顯等三十二人,亟正張體乾、許顯純、楊寰等罪。其月,又言:「崔呈秀雖死,宜剖棺戮屍。『五虎』、『五彪』之徒,乃或賜馳驛,或僅令還鄉,何以服人心,昭國典。」末斥吏科陳爾翼請緝東林遺孽之非,乞釋御史方震孺罪。帝並採納之。

崇禎元年,遷戶部主事,進員外郎。歷泉州知府、福建副使。八年,以大計罷歸。十六年,黃州城南門哭五日夜。眾知禍必至,傾城走,婦女多不及行。三月二十四日,張獻忠破黃岡,知縣孫自一、縣丞吳文燮死之。賊欲屈維城,抗聲大罵,刃洞胸而死。賊遂驅婦女墮城,稍緩,輒斷其腕,血淋漓土石間。三日而城平,復殺之以實塹焉。自一,光山人。

謝廷讚,字曰可,金谿人。父相,由鄉舉為東安知縣。初,歲饑,吏偽增戶口冒振,繼者遂按籍征賦,民困甚。相為請,得減戶千三百。奸人殺四人,棄其尸,獄三年不決。相禱於神,得屍所在,獄遂成。廷贊舉萬曆二十六年進士。未授官,即極論礦稅之害。旋授刑部主事。先是,詔二十八年春舉行冊立、冠婚之禮。將屆期,都御史溫純、禮科給事中楊天民、御史馮應鳳相繼言,不報。廷贊上疏言閣員當補,臺省當選,礦稅當撤,冠婚、冊立當速,詔令當信。持疏跪文華門,候命踰時。帝震怒,遣中官田義詰責。趙數日,命大學士趙志皋、沈一貫擬敕諭,令禮部具儀。比擬諭進,竟不發。志皋、一貫趣之,帝乃言因廷贊出位邀功,以致少待,命示諸司靜俟。遂褫廷贊職為民,並奪尚書蕭大亨,侍郎邵傑、董裕俸一歲,貶郎中徐如珂、員外郎林耀,主事鐘鳴陛、曹文偉三秩,調極邊。是歲冊立之禮不行,廷贊歸。僑寓維揚,授徒自給。久之,卒。天啟中,贈尚寶卿。

兄廷諒,字友可。萬曆二十三年進士。授南京刑部主事。帝命李廷機入閣,又召王錫爵。廷諒言:「廷機才弱而闇,錫爵氣高而揚,均不宜用。」又曰:「儲君之立為王也,自錫爵始;舉人之有考察也,自廷機始;巡按之久任也,自趙世卿始;章疏之留中也,自申時行始;年例之不舉,考察之不下也,自沈一貫始。此皆亂人國者也。」疏入,留中。終順慶知府。

楊天民,字正甫,山西太平人。萬曆十七年進士。除朝城知縣。調繁諸城,有異政,擢禮科給事中。時方纂修國史,與御史牛應元請復建文年號,從之。二十七年,狄道山崩,下成池,山南湧大小山五。天民言:「平地成山,惟唐垂拱間有之,而唐遂易為周。今虎狼之使吞噬無窮,狗鼠之徒攘奪難厭。不市而征稅,無礦而輸銀。甚且毀廬壞冢,籍人貲產,非法行刑。自大吏至守令,每被譴逐。郡邑不肖者,反助虐交歡,藉潤私橐。嗷嗷之眾,益無所歸命,懷樂禍心,有土崩之勢。天心仁愛,亟示譴告,陛下尚不覺悟,翻然與天下更始哉!」不報。文選郎中梅守峻貪黷,將擢太常少卿,天民劾罷之。延綏總兵官趙夢麟潛師襲寇,以大捷聞,督撫李汶、王見賓等咸進秩予廕。寇乃大入,殺軍民萬計,汶等又妄奏捷。天民再疏論之,奪見賓職,夢麟戍邊,汶亦被譴。

天民尋進右給事中。冊立久稽,再疏請,不報。無何,貴妃弟鄭國泰疏請皇長子先冠婚後冊立,天民斥其非。國泰懼,委罪都指揮李承恩,奪其俸。順天、湖廣鄉試文多用二氏語,天民請罪考官楊道賓、顧天颭等,疏留中。二十九年五月,天民復偕同官上言,請早定國本。帝大怒,謫天民及王士昌雜職,餘奪俸一年,以士昌亦給事禮科也。時御史周盤等公疏請,亦奪俸。天民得貴州永從典史。至十月,帝迫廷議,始立東宮,而天民等卒不召。天民幽憤卒。天啟中,贈光祿少卿。

初,天民去諸城,民為立祠。其後長吏不職,父老率聚哭祠下。

何選,字靖卿,宛平人。萬曆十一年進士。除南昌知縣,徵授御史。廷臣爭國本多獲譴,選語鄭貴妃弟國泰,令以朝野公論、鄭氏禍福懇言於貴妃,俾妃自請。國泰猶豫,選厲色責之曰:「若不及今為身家計,吾儕群擊之,悔無及矣。」國泰懼,乃入告於妃,且疏請早定,以釋危疑。帝意不懌。已,知出選指,深銜之。未幾,吏部擬調驗封員外郎鄒元標於文選,疏六日不下,選以為言。帝憶前事,謫湖廣布政司照磨。稍遷南京通政司經歷。刑部缺員外郎,吏部擬用選。帝憾未釋,謂特降官不當推舉,切讓尚書孫丕揚等,謫文選郎中馮生虞、員外郎馮養志等極邊,而斥選為民。以閣臣言,稍寬生虞、養志等罰。南京給事中任彥蘗抗章論救,語侵閣臣。帝復怒,謫彥蘗於外,生虞仍以雜職調邊方。旋以言官論救,並斥彥蘗為民。於是御史許聞造上言:「陛下頃歲以來,謂公忠為比周,謂論諫為激擾;詘銓衡之所賢,撓刑官之所執。光祿太僕之帑,括取幾空;中外大小之官,縣缺不補。敲撲遍於宮闈,桁楊接於道路。論救忠良,則愈甚其罪;諫止貢獻,則愈增其額。奏牘沉閣而莫稽,奄寺縱橫而無忌。今欲摘陳一事,則慮陛下益甚其事;欲摘救一人,則慮陛下益罪其人。陛下執此以拒建言之臣,諸臣因此而塞進言之路。邇年以來,諸臣謇諤之風,視昔大沮矣。」不報。

生虞,大足人。彥蘗,任城人。天啟中,贈選光祿少卿,生虞太常少卿。

贊曰:野史載神宗金合之誓。都人子之說,雖未知信否,然恭妃之位久居鄭氏下,固有以滋天下之疑矣。姜應麟等交章力爭,不可謂無羽翼功。究之鄭氏非褒、驪之煽處,國泰亦無駟、鈞之惡戾,積疑召謗,被以惡聲。《詩》曰:「時靡有爭,王心載寧。」諸臣何其好爭也!


\end{pinyinscope}