\article{列傳第七十一}

\begin{pinyinscope}
○何喬新彭韶周經耿裕倪岳閔珪戴珊

何喬新,字廷秀,江西廣昌人。

父文淵,永樂十六年進士。授御史,歷按山東、四川。烏蒙奸民什伽私其知府祿昭妻,懼誅,誣昭反。詔發軍討。文淵檄止所調軍,而白其誣。宣德五年用顧佐薦,賜敕知溫州府。居六年,治最,增俸賜璽書。以胡瀅薦,擢刑部右侍郎,督兩淮鹽課。正統三年,兩議獄不當,與尚書魏源下獄,皆得釋。朝議征麓川,文淵疏諫曰:「麓川徼外彈丸地,不足煩大兵。若遣雲南守將屯金齒,令三司官撫諭之,遠人獲更生,而朝廷免調兵轉餉,策之善者也。」帝下其議,廷臣多主用兵。於是西南騷動,僅乃克之,而失亡多。其冬,以疾乞歸。景帝即位,起吏部左侍郎,尋進尚書,佐王直理部事。東宮建,加太子太保。災異見,給事中林聰等劾文淵憸邪。左庶子周旋疏言其枉,聰並劾旋。御史曹凱復廷爭之,遂與旋俱下獄。聰疏有「囑內臣」語,太監興安請詰主名。聰不敢堅對,乃釋文淵命致仕。英宗復位,削其加官。而景泰中易儲詔書「父有天下傳之子」,語出文淵,或傳朝命逮捕,懼而自縊。

時喬新已登景泰五年進士,官南京禮部主事,奔喪歸里。里人故侍郎揭稽嘗受業文淵,而與喬新兄弟不協,奏文淵死實諸子迫之自經,又逼嫁父所愛妾。喬新亦訐稽為巡撫時,嘗薦黃厷,且代草易儲疏。皆被徵比對簿。父妾斷指,為諸郎訟冤,獄得少解。帝亦以事經赦,釋不問。已,復丁母憂。服除,改刑部主事,歷廣東司郎中。錦衣衛卒犯法,捕治不少貸。都指揮袁彬有所囑,執不從。彬怒,使人捃摭無所得。由是名大起。

成化四年遷福建副使。所屬壽寧銀礦,盜採者聚眾千餘人,所過剽掠,募兵擊擒其魁。福寧豪尤氏殺人,出入隨兵甲,拒捕者二十年。福清薛氏時出諸番互市,事覺,謀作亂。皆捕殺之。福安、寧德銀礦久絕,有司責課,民多破產。喬新以為言,減三之二。興化民自洪武初受牛於官,至是猶歲課其租,奏免之。清流歸化里介沙縣、將樂間,恃險不供賦,白都御史置歸化縣,其民始奉要束。遷河南按察使。歲大饑,故事,振貸迄秋止,喬新曰:「止於秋,謂秋成可仰也,今秋可但已乎?」振至明年麥熟乃止。都御史原傑以招撫流民至南陽,引喬新自助。初,項忠驅流民過當,民聞傑至,益竄山谷。喬新躬往招之,附籍者六萬餘戶。遷湖廣右布政使。荊州民苦徭役,驗丁口貧富,列為九等,民便之。

十六年擢右副都御史,巡撫山西。邊地軍民每出塞伐木捕獸,喬新言:「此輩茍遇敵,必輸情求生,皆賊導也。宜毋聽闌出,犯者罪守將。」詔可。敵犯塞,偕參將支玉伏兵灰溝營,擊斬甚眾,進左副都御史。歲饑,奏免雜辦及戶口鹽鈔十之四。劾僉事尚敬、劉源稽獄,請敕天下斷獄官,淹半載以上者悉議罪。帝稱善,亟從之。召拜刑部右侍郎。山西大饑,人相食。命往振,活三十餘萬人,還流冗十四萬戶。還朝,會安寧宣撫使楊友欲奪嫡弟播州宣慰使愛爵,誣愛有異謀。喬新往勘,與巡撫劉璋共白愛誣。友奪官安置他府,播人遂安。

孝宗嗣位,萬安、劉吉等忌喬新剛正,出為南京刑部尚書。沿江蘆洲率為中官占奪,託言備進奉費,喬新奏還之民。初,喬新之出,中官懷恩不平。一日以事詣閣言:「新君踐阼,常用正人,胡為出何公?」安等默然。既而刑部尚書杜銘罷,群望屬喬新,而吉代安為首輔,終忌之,久不補。弘治改元,用王恕薦,始召喬新代銘。奏言:「舊制遣官勘事及逮捕,必齎精微批文,赴所在官司驗視乃行。近止用駕帖不合符,宜復舊制,以防矯詐。」帝立報許。時吉仇正人,頻興大獄,喬新率據法直之。吉愈銜恨,數摭他事奪俸。二年夏,京城大水,喬新請恤被災者家,又慮刑獄失平,條上律文當更議者數事,吉悉格不行。大理丞闕,御史鄒魯覬遷,而喬新薦郎中魏紳。會喬新外家與鄉人訟,魯即誣喬新受賕曲庇。吉取中旨下其外家詔獄,喬新乃拜疏乞歸。頃之,窮治無驗,魯坐停俸,喬新亦許致仕。

喬新性廉介,觀政工部時,嘗使淮西。巢令閻徽少學於文淵,以金幣饋。喬新卻之,閻曰:「以壽吾師耳。」喬新曰:「子欲壽吾親,因他人致之則可,因吾致之則不可。」卒不受。福建市舶中官死,鎮守者分其貲遺三司,喬新獨固辭。不得,輸之於庫。既家居,楊愛遣使厚致贈,且獻良材可為櫬者,喬新堅卻之。

喬新年十一時,侍父京邸。修撰周旋過之,喬新方讀《通鑑續編》。旋問曰:「書法何如《綱目》?」對曰:「呂文煥降元不書叛,張世傑溺海不書死節,曹彬、包拯之卒不書其官,而紀羲、軒多採怪妄,似未有當也。」旋大驚異。比長,博綜群籍,聞異書輒借鈔,積三萬餘帙,皆手較讎,著述甚富。與人寡合,氣節友彭韶,學問友邱濬而已。

罷歸後,巡按江西御史陳詮奏:「喬新始終全節,中間只以受親故餽遺之嫌,勒令致仕,進退黯昧,誠為可惜。乞行勘,本官如無疾則行取任用,有疾則加慰勞,以存故舊之恩,全進退之節。」不許。後中外多論薦,竟不復起。十五年卒,年七十六。

江西巡撫林俊為彭韶及喬新請謚,吏部覆從之。有旨令上喬新致仕之由,給事中吳世忠言:「喬新學行、政事莫不優,忠勤剛介,老而彌篤。御史鄒魯挾私誣劾,一辭不辨,恬然退歸。杜門著書,人事寡接,士大夫莫不高其行。若必考退身之由,疑旌賢之典,則如宋蔣之奇嘗誣奏歐陽修矣,胡紘輩嘗誣奏朱熹矣,未聞以一人私情廢萬世公論也。」事竟寢。正德十一年,廣昌知縣張傑復以為言,乃贈太子太保,予廕。明年賜謚文肅。

喬新五世孫源,萬曆初,為刑部右侍郎,亦有清節。

彭韶,字鳳儀,莆田人。天順元年進士。授刑部主事,進員外郎。成化二年疏論僉都御史張岐憸邪,宜召王竑、李秉、葉盛,忤旨,下詔獄。給事中毛弘等救之,不聽,卒輸贖。尋遷郎中。

錦衣指揮周彧,太后弟也,奏乞武強、武邑民田不及賦額者,籍為閒田。命韶偕御史季琮覆勘。韶等周視徑歸,上疏自劾曰:「真定田,自祖宗時許民墾種,即為恒產,除租賦以勸力農。功臣、戚里家與國咸休,豈當與民爭尺寸地。臣誠不忍奪小民衣食,附益貴戚,請伏奉使無狀罪。」疏入,詔以田歸民,而責韶等邀名方命,復下詔獄。言官爭論救,得釋。當是時,韶與何喬新同官,並有重名,一時稱「何彭」。

遷四川副使。安岳扈氏焚滅劉某家二十一人,定遠曹氏殺其兄一家十二人,所司以為疑獄,久不決。韶一訊得實,咸伏辜。進按察使,盡撤境內淫祠。王府祭葬舊遣內官,公私煩費,奏罷之。雲南鎮守太監錢能進金燈,擾道路,韶劾之,不報。

十四年春,遷廣東左布政使。中官奉使紛遝,鎮守顧恒、市舶韋眷、珠池黃福,皆以進奉為名,所至需求,民不勝擾。韶先後論奏。最後,梁芳弟錦衣鎮撫德以廣東其故鄉,歸採禽鳥花木,害尤酷。韶抗疏極論,語侵芳。芳怒,構於帝,調之貴州。

二十年擢右副都御史,巡撫應天。明年正月,星變,上言:「彗星示災,見於歲暮,遂及正旦。歲墓者,天道之終。正旦者,歲事之始。此天心仁愛,欲陛下善始善終也。陛下嗣位之初,家禮正,防微周,儉德昭,用人慎。乃邇年以來,進奉貴妃,加於嫡后,褒寵其家,幾與先帝后家埒,此正家之道未終也。監局內臣數以萬計,利源兵柄盡以付之,犯法縱奸,一切容貸,此防微之道未終也。四方鎮守中官,爭獻珍異,動稱敕旨,科擾小民,此持儉之道未終也。六卿並加師保,監寺兼領崇階,及予告而歸,廩食輿夫濫加庸鄙。爵賞一輕,人誰知勸,此用人之道未終也。惟陛下慎終如始,天下幸甚。」時方召為大理卿,帝得疏不悅,命仍故官巡撫順天、永平二府。均大興、宛平、昌平諸縣徭役,劾奏鎮守中官陶弘罪。

孝宗即位,召為刑部右侍郎。嘉興百戶陳輔緣盜販為亂,陷府城大掠,遁入太湖。遣韶巡視。韶至,賊已滅,乃命兼僉都御史,整理鹽法。尋進左侍郎。韶以商人苦抑配,為定折價額,蠲宿負。憫灶戶煎辦、征賠、折閱之困,繪八圖以獻,條利病六事,悉允行。弘治二年秋,還朝。明年,改吏部。與尚書王恕甄人才,核功實,仕路為清。彗星見,上言宦官太盛,不可不亟裁損。因請午朝面議大政,毋只具文。已,又言濫授官太多,乞嚴杜倖門,痛為釐正。帝是其言,然竟不能用。

四年秋,代何喬新為刑部尚書。故安遠侯柳景贓敗至數千兩,徵僅十一。以其母訴免。韶執奏曰:「昔唐宣宗元舅鄭光官租不入,京兆尹韋澳械其莊吏。宣宗欲寬之,澳不奉詔。景無元舅之親,贓非負租之比,獨蒙宥除,是臣等守法愧於澳也。」不從。御史彭程以論皇壇器下獄,韶疏救,因極陳光祿冗食濫費狀,乃命具歲辦數以聞。荊王見潚有罪,奏上,淹旬不下。內官王明、苗通、高永殺人,減死遣戍。昌國公張巒建墳踰制,役軍至數萬。畿內民冒充陵廟戶及勇士旗校,輒免徭役,致見戶不支,流亡日眾。韶皆抗疏極論,但下所司而已。

韶蒞部三年,昌言正色,秉節無私,與王恕及喬新稱三大老,而為貴戚、近習所疾,大學士劉吉亦不之善。韶志不能盡行,連章乞休,乃命乘傳歸。月廩、歲隸如制。明年,南京地震,御史宗彞等言韶、喬新、強珍、謝鐸、陳獻章、章懋、彭程俱宜召用,不報。又明年,卒,年六十六。謚惠安,贈太子少保。

韶嗜學,公暇手不釋書。正德初,林俊言韶謚不副行,乞如魏驥、吳訥、葉盛,改謚文。竟不行。

周經,字伯常,刑部尚書瑄子也。天順四年進士。改庶吉士,授檢討。成化中,歷侍讀、中允,侍孝宗於東宮。講《文華大訓》,太子起立,閣臣以為勞,議請坐聽。經與諸講官皆不可,乃已。

孝宗立,進太常少卿兼侍讀。弘治二年擢禮部右侍郎。中官請修黃村尼寺,奉祀孝穆太后。土魯番貢獅子不由甘肅,假道滿剌加,浮海至廣東。經倡議毀其寺,卻貢不與通。改吏部,進左侍郎。通政經歷沈祿者,皇后姑婿也。尚書王恕在告,中官傳旨擢祿本司參議。經言非面承旨,又無御札,不敢奉詔,復與恕疏爭之。事雖不能止,朝論韙焉。靈壽奸民獻地於中官李廣,戶部持不得。經倡九卿疏爭,卒罪獻地者。嘗上言:「外戚家無功求遷,無勞乞賞,兼齋醮遊宴,濫費無紀,致帑藏殫虛,宜大為撙節。近例,預備倉積粟多者,守令賜誥敕,不次遷官,遂致剝下干進。請如洪武間例,悉出官帑平糴,毋奪民財,考績毋專以積粟為能。至清軍之弊,洪熙以前在旗校,宣德以後在里胥。弊在旗校者,版籍猶存,若里胥則並版籍而淆亂之,宜考故冊洗奸弊。災傷民,乞省恤。惜薪司薪炭約支數年,災荒郡縣,宜盡與停免。四方顏料雜辦亦然。此救民急務也。」帝多採納之。

八年,文武大臣以災異陳時政,經為具奏草,而斥戲樂一事,語尤切直。帝密令中官廉草奏者,尚書耿裕曰:「疏首吏部,裕實具草。」經曰:「疏草出經手,即有罪,罪經。」世兩賢之。

明年,代葉淇為戶部尚書。時孝宗寬仁,而戶部尤奸蠹所萃,挾勢行私者不可勝紀。少不如意,讒毀隨之。經悉按祖宗成憲,無所顧。寬逋緩征,裁節冗濫。四方告災,必覆奏蠲除。每委官監稅課,入多者與下考,苛切之風為之少衰。

奉御趙瑄獻雄縣地為東宮莊。經等劾瑄違制,下詔獄。而帝復從鎮撫司言遣官勘實。經等復爭之曰:「太祖、太宗定制,閒田任民開墾。若因奸人言而籍之官,是土田予奪,盡出奸人口,小民無以為生矣。」既而勘者及巡撫高銓言閑田止七十頃,悉與民田錯。於是從經言仍賦之民,治瑄罪。中官何鼎劾外戚張鶴齡下獄,經疏救之,忤旨切責。雍王祐枟乞衡州稅課司及衡陽縣河泊所,經言不可許。帝納之,命自今四方稅課,王府不得請。中官織造者,請增給兩浙鹽課二萬引,經等言:「鹽筴佐邊,不宜濫給。且祖宗朝織染諸局供御有常數,若曰取用有加,則江南、兩浙已例外囑造,若曰工匠不足,則仰食公家不下千餘人,所為何事。是知供用未必缺,而徒導陛下以勞民傷財之事也。」帝不從。經恐歲以為常,再疏請斷其後,乃命歲予五千引。

先是,倉場監督內官依成化末年例裁減。十一年秋,帝復增用少監莫英等三人。經上疏力爭,帝以已遣不聽。內靈臺請錦衣餘丁百人供灑掃,經等諫,不納。經曰:「祖宗設內臺,其地至密。今一旦增百人,將必有漏洩妄言者。」帝悟,立已之。

崇王見澤乞河南退灘地二十餘里,經言不宜予。興王祐杬前後乞赤馬諸河泊所及近湖地千三百餘頃,經三疏爭之,竟不許。帝以肅寧諸縣地四百餘頃賜壽寧侯張鶴齡,其家人因侵民地三倍,且毆民至死,下巡撫高銓勘報。銓言可耕者無幾,請仍賦民,不許。時王府、勳戚莊田例畝征銀三分,獨鶴齡奏加征二分,且概加之沙堿地。經抗章執奏,命侍郎許進偕太監朱秀覆核。經言:「地已再勘,今復遣使,徒滋煩擾。昔太祖以劉基故減青田賦,徵米五合,欲使基鄉里子孫世世頌基。今興濟篤生皇后,正宜恤民減賦,俾世世戴德,何乃使小民銜怨無已也。」頃之,進等還言此地乃憲廟皇親柏權及民恒產,不可奪。帝竟予鶴齡,如其請加稅,而命償權直,除民租額。經等復諫曰:「東宮、親王莊田征稅自有例,鶴齡不宜獨優。權先帝妃家,亦戚畹也,名雖償直,實乃奪之。天下將謂陛下惟厚椒房親,不念先朝外戚。」帝終不納。

大同缺戰馬,馬文升請太倉銀以市。經言:「糧馬各有司存。祖訓六部毋相壓,兵部侵戶部權,非祖訓。」帝為改撥太僕銀給之。給事中魯昂請盡括稅役金錢輸太倉,經曰:「不節織造、賞賚、齋醮、土木之費,而欲括天下財,是舛也。」內官傳旨索太倉銀三萬兩為燈費,持不與。

經剛介方正,好強諫,雖重忤旨不恤。宦官、貴戚皆憚而疾之。太監李廣死,帝得朝臣與饋遺簿籍,大怒。科道因劾諸臣交通狀,有及經者。經上疏曰:「昨科道劾廷臣奔競李廣,闌入臣名。雖蒙恩不問,實含傷忍痛,無以自明。夫人奔競李廣,冀其進言左右,圖寵眷耳。陛下試思廣在時,曾言及臣否。且交結饋遺簿籍具在,乞檢曾否有臣姓名。更嚴鞫廣家人,臣但有寸金、尺帛,即治臣交結之罪,斬首市曹,以為奔競無恥之戒。若無干涉,亦乞為臣洗雪,庶得展布四體,終事聖明。若令含汙忍垢,即死填溝壑,目且不瞑。」帝慰答之。十三年,星變,自陳乞休。報許,賜敕馳驛,加太子太保,以侶鐘代。廷臣爭上章留之,中外論薦者至八十餘疏,咸報寢。

武宗即位,言官復薦,召為南京戶部尚書,遭繼母憂未任。正德三年,服闋。經婿兵部尚書曹元方善劉瑾,言經雖老尚可用,乃召為禮部尚書。固辭不許,強赴召。受事數月即謝病去。五年三月卒,年七十一。贈太保,謚文端。

子曾,進士。浙江右參政。

耿裕,字好問,刑部尚書九疇子也。景泰五年進士。改庶吉士,授戶科給事中,改工科。天順初,以九疇為右都御史,改裕檢討。九疇坐劾石亨貶,裕亦謫泗州判官。終父喪,補定州。

成化初,召復檢討,歷國子司業、祭酒。侯伯年幼者皆肄業監中,裕采古諸侯、貴戚言行可法者為書授之,帝聞而稱善。歷吏部左右侍郎。坐尚書尹旻累,停俸者再。已,代旻為尚書。大學士萬安與裕不協,而李孜省私其同鄉李裕,欲使代裕,相與謀中之。坐以事,調侍郎黎淳南京,而奪裕俸。言官復交劾,宥之。裕入謝,既出,帝怒曰:「吾再寬裕罪,當再謝。今一謝,以奪俸故,意鞅鞅耶?」孜省等因而傾之,遂調南京禮部,而以李裕代。踰年,孝宗嗣位,轉南京兵部參贊機務。

弘治改元,召拜禮部尚書。時公私侈靡,耗費日廣。裕隨事救正,因災異條上時事及申理言官,先後陳言甚眾,大要歸於節儉。給事中鄭宗仁疏節光祿供應,裕等請納其奏。巡視光祿御史田CO以供費不足累行戶,請借太倉銀償之。裕等言,疑有侵盜弊,請敕所司禁防,帝皆從之。南京守備中官請增奉先殿日供品物,裕等不可。帝方踐阼,斥番僧還本土,止留乳奴班丹等十五人。其後多潛匿京師,轉相招引,齋醮復興。言官以為言,裕等因力請驅斥。帝乃留百八十二人,餘悉逐之。禮部公廨火,裕及侍郎倪岳、周經等請罪,被劾下獄。已,釋之,停其俸。

初,撒馬兒罕及土魯番皆貢獅子,甘肅鎮守太監傅德先圖形以進,巡按御史陳瑤請卻之。裕等乞從瑤請,而治德違詔罪,帝不從。後番使再至,留京師,頻有宣召。裕等言:「番人不道,因朝貢許其自新。彼復潛稱可汗,興兵犯順。陛下優假其使,適遇倔強之時,彼將謂天朝畏之,益長桀驁。且獅子野獸,無足珍異。」帝即遣其使還。

尋代王恕為吏部尚書,加太子太保。御用監匠人李綸等以內降得官,裕言:「先有詔,文官不由臣部推舉傳乞除授者,參送法司按治。今除用綸等,不信前詔,不可。」給事中呂獻等皆論奏,裕亦再疏爭,終不聽。

裕為人坦夷諒直,諳習朝章。秉銓數年,無愛憎,亦不徇毀譽,銓政稱平。自奉澹泊。兩世貴盛,而家業蕭然,父子並以名德稱。九年正月卒,年六十七。贈太保,謚文恪。

倪岳,字舜咨,上元人。父謙,奉命祀北岳,母夢緋衣神入室,生岳,遂以為名。謙終南京禮部尚書,謚文僖。

岳,天順八年進士。改庶吉士,授編修。成化中,歷侍讀學士,直講東宮。二十二年擢禮部右侍郎,仍直經筵。弘治初,改左侍郎。岳好學,文章敏捷,博綜經世之務。尚書耿裕方正持大體,至禮文制度率待岳而決。六年,裕改吏部,岳遂代為尚書。詔召國師領占竹於四川,岳力諫,帝不從。給事中夏昂、御史張禎等相繼爭之,事竟寢。時營造諸王府,規制宏麗,踰永樂、宣德之舊。岳請頒成式。又以四方所報災異,禮部於歲終類奏,率為具文,乃詳次其月日,博引經史徵應。勸帝勤講學,開言路,寬賦役,慎刑罰,黜奸貪,進忠直,汰冗員,停齋醮,省營造,止濫賞。帝頗採納焉。

左侍郎徐瓊與后家有連,謀代岳。九年,南京吏部缺尚書,廷推瓊。詔加岳太子太保,往任之,而瓊果代岳。尋改岳南京兵部參贊機務。還,代屠滽為吏部尚書,嚴絕請托,不徇名譽,銓政稱平。

岳狀貌魁岸,風采嚴峻,善斷大事。每盈廷聚議,決以片言,聞者悅服。同列中,最推遜馬文升,然論事未嘗茍同。前後陳請百餘事,軍國弊政剔抉無遺。疏出,人多傳錄之。論西北用兵害尤切,其略云:

近歲毛里孩、阿羅忽、孛羅出、加思蘭大為邊患。蓋緣河套之中,水草甘肥,易於屯牧,故賊頻據彼地,擁眾入掠。諸將怯懦,率嬰城自守。茍或遇敵,輒至挫衄。既莫敢折其前鋒,又不能邀其歸路。敵進獲重利,退無後憂,致兵鋒不靖,邊患靡寧。命將徂征,四年三舉,絕無寸功。或高臥而歸,或安行以返。析圭擔爵,優游朝行,輦帛輿金,充牣私室。且軍旅一動,輒報捷音,賜予濫施,官秩輕授。甚至妄殺平民,謬稱首級。敵未敗北,輒以奔遁為辭。功賞所加,非私家子弟,即權門廝養。而什伍之卒,轉餉之民,則委骨荒城,膏血野草。天怒人怨,禍幾日深,非細故也。

京營素號冗怯。留鎮京師,猶恐未壯根本,顧乃輕於出禦,用褻天威。臨陣輒奔,反墮邊軍之功,為敵人所侮。且延綏邊也,去京師遠;宣府、大同亦邊也,去京師近。彼有門庭之喻,此無陛楯之嚴,可乎?頃兵部建議:令宣府出兵五千,大同出兵一萬,併力以援延綏,而不慮其相去既遠,往返不逮,人心苦於轉移,馬力疲於奔軼。夫聲東擊西者,賊寇之奸態也。搗虛批亢者,兵家之長策也。精銳既盡乎西,老弱乃留於北。萬一北或有警,而西未可離,首尾衡決,遠近坐困,其可為得計哉?至於延綏士馬屯集,糧糗不貲,乃以山西、河南之民任飛芻轉粟之役。徒步千里,夫運而妻供,父挽而子荷,道路愁怨,井落空虛。幸而得至,束芻百錢,斗粟倍直;不幸遇賊,身且斃矣,他尚何云。輸將不足則有輕齎,輕齎不足又有預征。水旱不可先知,豐歉未能逆卜,徵如何其可預也。又令民輸芻粟補官,而媚權貴私親故者,或出空牒以授,倉庾無升合之入。至若輸粟給鹽,則豪右請託,率占虛名鬻之,而商賈費且倍蓰。官爵日輕,鹽法日沮,而邊儲之不充如故也。

又朝廷出帑藏給邊,歲為銀數十萬。山西、河南輸輕齎於邊者,歲不下數十萬。銀日積而多則銀益賤,粟日散而少則粟益貴。而不知者,遂於養兵之中,寓養狙之術。或以茶鹽,或以銀布,名為準折糧價,實則侵剋軍需。故朝廷有糜廩之虞,軍士無果腹之樂。至兵馬所經,例須應付。居平,人日米一斗,馬日芻一束。追逐,一日之間或一二堡,或三四城,豈能俱給哉?而典守者巧為竊攘之謀,凡所經歷悉有開支,罔上行私,莫此為甚。

及訪禦敵之策,則又論議紛紜。有謂復受降之故險,守東勝之舊城,使聲援交接,犄角易制。夫欲復城河北,即須塞外屯兵。出孤遠之軍,涉荒漠之地,輜重為累,饋餉惟艱。彼或抄掠於前,躡襲於後。曠日持久,軍食乏絕。進不得城,退不得歸,一敗而聲威大損矣。又有謂統十萬之眾,裹半月之糧,奮揚武威,掃蕩窟穴,使河套一空。事非不善也。然帝王之兵,以全取勝;孫、吳之法,以逸待勞。今欲鼓勇前行,窮搜遠擊,乘危履險,覬萬一之倖。贏糧遠隨則重不及事,提兵深入則孤不可援。且其間地方千里,無城郭之居,委積之守。彼或往來遷徙,罷我馳驅。我則情見勢屈,為敵所困。既失坐勝之機,必蹈覆沒之轍。其最無策者,又欲棄延綏勿守,使兵民息肩,不知一民尺土皆受之祖宗,不可忽也。向失東勝,故今日之害萃於延綏,而關陜震動。今棄延綏,則他日之害鐘於關陜,而京師震動。賊愈近而禍愈大矣。

因陳重將權、增城堡、廣斥堠、募民壯、去客兵、明賞罰、嚴間諜、實屯田、復邊漕數事。時兵部方主用兵,不能盡用也。

十四年十月卒,年五十八。贈少保,謚文毅。明世父子官翰林,俱謚文,自岳始。

閔珪,字朝瑛,烏程人。天順八年進士。授御史。出按河南,以風力聞。成化六年擢江西副使,進廣東按察使。久之,以右僉都御史巡撫江西。南、贛諸府多盜,率強宗家僕。珪請獲盜連坐其主,法司議從之。尹直輩謀之李孜省,取中旨責珪不能弭盜,左遷廣西按察使。

孝宗嗣位,擢右副都御史,巡撫順天。入為刑部右侍郎,進右都御史,總督兩廣軍務,與總兵官毛銳討古田僮。副總兵馬俊、參議馬鉉自臨桂深入,敗死,軍遂退。詔停俸討賊。珪復進兵,連破七寨,他賊悉就撫。

弘治七年遷南京刑部尚書,尋召為左都御史。十一年,東宮出閣,加太子少保。十三年代白昂為刑部尚書,再加太子太保。以災異與都御史戴珊共陳時政八事,又陳刑獄四事,多報可。

珪久為法官,議獄皆會情比律,歸於仁恕。宣府妖人李道明聚眾燒香,巡撫劉聰信千戶黃珍言,株連數十家,謂道明將引北寇攻宣府。及逮訊無驗,珪乃止坐道明一人,餘悉得釋,而抵珍罪,聰亦下獄貶官。帝之親鞫吳一貫也,將置大辟,珪進曰:「一貫推案不實,罪當徒。」帝不允,珪執如初。帝怒,戴珊從旁解之。帝乃霽威,令更擬。珪終以原擬上,帝不悅,召語劉大夏。對曰:「刑官執法乃其職,未可深罪。」帝默然久之,曰:「朕亦知珪老成不易得,但此事太執耳。」卒如珪議。

正德元年六月,以年踰七十再疏求退,不允。及劉瑾用事,九卿伏闕固諫,韓文被斥,珪復連章乞休。明年二月詔加少保,賜敕馳傳歸。六年十月卒,年八十二。贈太保,謚莊懿。

從孫如霖,南京禮部尚書。如霖曾孫洪學,吏部尚書。洪學從弟夢得,兵部戎政尚書。他為庶僚者復數人。

戴珊,字廷珍,浮梁人。父哻,由鄉舉官嘉興教授,有學行。富人數輩遣其奴子入學,哻不可。賄上官強之,執愈堅,見忤,坐他事去。

珊幼嗜學,天順末,與劉大夏同舉進士。久之,擢御史,督南畿學政。成化十四年遷陜西副使,仍督學政。正身率教,士皆愛慕之。歷浙江按察使,福建左、右布政使,終任不攜一土物。

弘治二年,以王恕薦擢右副都御史,撫治鄖陽。蜀盜野王剛流劫竹山、平利。珊合川、陜兵,檄副使朱漢等討擒其魁,餘皆以脅從論,全活甚眾。入歷刑部左、右侍郎,與尚書何喬新、彭韶共事。晉府寧化王鐘鈵淫虐不孝,勘不得實,再遣珊等勘之,遂奪爵禁錮。進南京刑部尚書。久之,召為左都御史。十七年,考察京官,珊廉介不茍合。給事中吳蕣、王蓋自疑見黜,連疏詆吏部尚書馬文升,並言珊縱妻子納賄。珊等乞罷,帝慰留之。御史馮允中等言:「文升、珊歷事累朝,清德素著,不可因浮詞廢計典。」乃下蕣、蓋詔獄,命文升、珊即舉察事。珊等言:「兩人逆計當黜,故先劾臣等。今黜之,彼必曰是挾私也。茍避不黜,則負委任,而使詐諼者得志。」帝命上兩人事蹟,皆黜之。已,劉健等因召對,力言蓋罪輕,宜調用。帝方嚮用文升、珊,卒不納。

帝晚年召對大臣,珊與大夏造膝宴見尤數。一日,與大夏侍坐。帝曰:「時當述職,諸大臣皆杜門。如二卿者,雖日見客何害。」袖出白金賚之,曰:「少佐而廉。」且屬勿廷謝,曰:「恐為他人忌也。」珊以老疾數求退,輒優詔勉留,遣醫賜食,慰諭有加。珊感激泣下,私語大夏曰:「珊老病子幼,恐一旦先朝露,公同年好友,何惜一言乎?」大夏曰:「唯唯。」後大夏燕對畢,帝問珊病狀,言珊實病,乞憫憐聽其歸。帝曰:「彼屬卿言耶?主人留客堅,客則強留。珊獨不能為朕留耶?且朕以天下事付卿輩,猶家人父子。今太平未兆,何忍言歸!」大夏出以告珊,珊泣曰:「臣死是官矣。」帝既崩,珊以新君嗣位不忍言去,力疾視事。疾作,遂卒。贈太子太保,謚恭簡。

贊曰:孝宗之為明賢君,有以哉。恭儉自飭,而明於任人。劉、謝諸賢居政府,而王恕、何喬新、彭韶等為七卿長,相與維持而匡弼之。朝多君子,殆比隆開元、慶歷盛時矣。喬新、韶雖未究其用,而望著朝野。史稱宋仁宗時,國未嘗無嬖幸,而不足以累治世之體;朝未嘗無小人,而不足以勝善類之氣。孝宗初政,亦略似之。不然,承憲宗之季,而欲使政不旁撓,財無濫費,滋培元氣,中外乂安,豈易言哉。


\end{pinyinscope}