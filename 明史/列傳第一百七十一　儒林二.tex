\article{列傳第一百七十一 儒林二}

\begin{pinyinscope}
○陳獻章李承箕張詡婁諒夏尚樸賀欽陳茂烈湛若水蔣信等鄒守益子善等錢德洪徐愛等王畿王艮等歐陽德族人瑜羅洪先程文德吳悌子仁度何廷仁劉邦採魏良政等王時槐許孚遠尤時熙張後覺等鄧以贊張元心卞孟化鯉孟秋來知德鄧元錫劉元卿章潢

陳獻章,字公甫,新會人。舉正統十二年鄉試,再上禮部,不第。從吳與弼講學。居半載歸,讀書窮日夜不輟。築陽春臺,靜坐其中,數年無戶外跡。久之,復游太學。祭酒邢讓試和楊時《此日不再得》詩一篇,驚曰:「龜山不如也。」揚言於朝,以為真儒復出。由是名震京師。給事中賀欽聽其議論,即日抗疏解官,執弟子禮事獻章。獻章既歸,四方來學者日進。廣東布政使彭韶、總督朱英交薦。召至京,令就試吏部。屢辭疾不赴,疏乞終養,授翰林院檢討以歸。至南安,知府張弼疑其拜官,與與弼不同。對曰:「吳先生以布衣為石亨所薦,故不受職而求觀秘書,冀在開悟主上耳。時宰不悟,先令受職然後觀書,殊戾先生意,遂決去。獻章聽選國子生,何敢偽辭釣虛譽。」自是屢薦,卒不起。

獻章之學,以靜為主。其教學者,但令端坐澄心,於靜中養出端倪。或勸之著述,不答。嘗自言曰:「吾年二十七,始從吳聘君學,於古聖賢之書無所不講,然未知入處。比歸白沙,專求用力之方,亦卒未有得。於是舍繁求約,靜坐久之,然後見吾心之體隱然呈露,日用應酬隨吾所欲,如馬之御勒也。」其學灑然獨得,論者謂有鳶飛魚躍之樂,而蘭谿姜麟至以為「活孟子」云。

獻章儀幹修偉,右頰有七黑子。母年二十四守節,獻章事之至孝。母有念,輒心動,即歸。弘治十三年卒,年七十三。萬曆初,從祀孔廟,追謚文恭。

門人李承箕,字世卿,嘉魚人。成化二十二年舉鄉試。往師獻章,獻章日與登涉山水,投壺賦詩,縱論古今事,獨無一語及道。久之,承箕有所悟,辭歸,隱居黃公山,不復仕。與兄進士承芳,皆好學,稱嘉魚二李。卒年五十四。

張詡,字廷實,南海人,亦師事獻章。成化二十年舉進士,授戶部主事。尋丁憂,累薦不起。正德中,召為南京通政司參議,一謁孝陵即告歸。獻章謂其學以自然為宗,以忘己為大,以無欲為至。卒年六十。

婁諒,字克貞,上饒人。少有志絕學。聞吳與弼在臨川,往從之。一日,與弼治地,召諒往視,云學者須親細務。諒素豪邁,由此折節。雖掃除之事,必身親之。景泰四年舉於鄉。天順末,選為成都訓導。尋告歸,閉門著書,成《日錄》四十卷、《三禮訂訛》四十卷。謂《周禮》皆天子之禮,為國禮。《儀禮》皆公卿大夫士庶人之禮,為家禮。以《禮記》為二經之傳,分附各篇,如《冠禮》附《冠義》之類。不可附各篇者,各附一經之後。不可附一經者,總附二經之後。其為諸儒附會者,以程子論黜之。著《春秋本意》十二篇,不採三傳事實,言:「是非必待三傳而後明,是《春秋》為棄書矣。」其學以收放心為居敬之門,以何思何慮、勿忘勿助為居敬要旨。然其時胡居仁頗譏其近陸子,後羅欽順亦謂其似禪學云。

子忱,字誠善,傳父學。女為寧王宸濠妃,有賢聲,嘗勸王毋反。王不聽,卒反。諒子姓皆捕繫,遺文遂散軼矣。

門人夏尚樸,字敦夫,廣信永豐人。正德初,會試赴京。見劉瑾亂政,慨然歎曰:「時事如此,尚可干進乎?」不試而歸。六年成進士,授南京禮部主事。歲饑,條上救荒數事。再遷惠州知府,投劾歸。嘉靖初,起山東提學副使。擢南京太僕少卿,與魏校、湛若水輩日相講習。言官劾大學士桂萼,語連尚樸。吏部尚書方獻夫白其無私,尋引疾歸。早年師諒,傳主敬之學,常言「纔提起,便是天理。纔放下,便是人欲」。魏校亟稱之。所著有《中庸語》《東巖文集》。王守仁少時,亦嘗受業於諒。

賀欽,字克恭,義州衛人。少好學,讀《近思錄》有悟。成化二年以進士授戶科給事中。已而師事陳獻章。既歸,肖其像事之。

弘治改元,用閣臣薦,起為陜西參議。檄未至而母死,乃上疏懇辭,且陳四事。一,謂今日要務莫先經筵,當博訪真儒,以資啟沃。二,薦檢討陳獻章學術醇正,稱為大賢,宜以非常之禮起之,或俾參大政,或任經筵,以養君德。三,內官職掌,載在《祖訓》,不過備灑掃、司啟閉而已。近如王振、曹吉祥、汪直等,或參預機宜,干政令,招權納寵,邀功啟釁。或引左道,進淫巧,以蕩上心。誤國殃民,莫此為甚。宜慎飭將來,內不使干預政事,外不使鎮守地方掌握兵權。四,興禮樂以化天下。「陛下紹基之初,舉行朱子喪葬之禮,而頹敗之俗因仍不改,乞申明正禮,革去教坊俗樂,以廣治化。」疏凡數萬言。奏入,報聞。正德四年,劉瑾括遼東田,東人震恐,而義州守又貪橫,民變,聚眾劫掠。顧相戒曰:「毋驚賀黃門。」欽聞之,急諭禍福,以身任之,亂遂定。欽學不務博涉,專讀《四書》、《六經》、《小學》,期於反身實踐。謂為學不必求之高遠,在主敬以收放心而已。卒年七十四。子士諮,鄉貢士,嘗陳十二事論王政,不報。終身不仕。

陳茂烈,字時周,莆田人。年十八,作《省克錄》,謂顏之克己,曾之日省,學之法也。弘治八年舉進士。奉使廣東,受業陳獻章之門,獻章語以主靜之學。退而與張詡論難,作《靜思錄》。尋授吉安府推官,考績過淮,寒無絮衣,凍幾殆。入為監察御史,袍服朴陋,乘一疲馬,人望而敬之。以母老終養。供母之外,不辦一帷。治畦汲水,身自操作。太守聞其勞,進二卒助之,三日遣之還。吏部以其貧,祿以晉江教諭,不受。又奏給月米,上書言:「臣素貧,食本儉薄,故臣母自安於臣之家,而臣亦得以自逭其貧,非有及人之廉,盡己之孝也。古人行備負米,皆以為親,臣之貧尚未至是。而臣母鞠臣艱苦,今年八十有六,來日無多。臣欲自盡心力,尚恐不及,上煩官帑,心竊未安。」奏上不允。母卒,茂烈亦卒。

茂烈為諸生時,韓文問莆田人物於林俊,曰:「從吾。」謂彭時也。又問,曰:「時周。」且曰:「與時周語,沉痾頓去。」其為所重如此。

湛若水,字元明,增城人。弘治五年舉於鄉,從陳獻章游,不樂仕進。母命之出,乃入南京國子監。十八年會試,學士張元禎、楊廷和為考官,撫其卷曰:「非白沙之徒不能為此。」置第二。賜進士,選庶吉士,授翰林院編修。時王守仁在吏部講學,若水與相應和。尋丁母憂,廬墓三年。築西樵講舍,士子來學者,先令習禮,然後聽講。嘉靖初,入朝,上經筵講學疏,謂聖學以求仁為要。已復上疏言:「陛下初政,漸不克終。左右近侍爭以聲色異教蠱惑上心。大臣林俊、孫交等不得守法,多自引去,可為寒心。亟請親賢遠奸,窮理講學,以隆太平之業。」又疏言日講不宜停止,報聞。明年進侍讀,復疏言:「一二年間,天變地震,山崩川湧,人饑相食,殆無虛月。夫聖人不以屯否之時而後視賢之訓,明醫不以深錮之疾而廢元氣之劑,宜博求修明先王之道者,日侍文華,以裨聖學。」已,遷南京國子監祭酒,作《心性圖說》以教士。拜禮部侍郎。仿《大學衍義補》,作《格物通》,上於朝。歷南京吏、禮、兵三部尚書。南京欲尚侈靡,為定喪葬之制頒行之。老,請致仕。年九十五卒。

若水生平所至,必建書院以祀獻章。年九十,猶為南京之游。過江西,安福鄒守益,守仁弟子也,戒其同志曰:「甘泉先生來,吾輩當憲老而不乞言,慎毋輕有所論辨。」若水初與守仁同講學,後各立宗旨,守仁以致良知為宗,若水以隨處體驗天理為宗。守仁言若水之學為求之於外,若水亦謂守仁格物之說不可信者四。又曰:「陽明與吾言心不同。陽明所謂心,指方寸而言。吾之所謂心者,體萬物而不遺者也,故以吾之說為外。」一時學者遂分王、湛之學。

湛氏門人最著者,永豐呂懷、德安何遷、婺源洪垣、歸安唐樞。懷之言變化氣質,遷之言知止,樞之言求真心,大約出入王、湛兩家之間,而別為一義。垣則主於調停兩家,而互救其失。皆不盡守師說也。懷,字汝德,南京太僕少卿。遷,字益之,南京刑部侍郎。垣,字峻之,溫州府知府。樞,刑部主事,疏論李福達事,罷歸,自有傳。

蔣信,字卿實,常德人。年十四,居喪毀瘠。與同郡冀元亨善,王守仁謫龍場,過其地,偕元亨事焉。嘉靖初,貢入京師,復師湛若水。若水為南祭酒,門下士多分教。至十一年,舉進士,累官四川水利僉事。卻播州土官賄,置妖道士於法。遷貴州提學副使。建書院二,廩群髦士其中。龍場故有守仁祠,為置祠田。坐擅離職守,除名。信初從守仁游時,未以良知教。後從若水游最久,學得之湛氏為多。信踐履篤實,不事虛談。湖南學者宗其教,稱之曰正學先生。卒年七十九。時宜興周衝,字道通,亦游王、湛之門。由舉人授高安訓導,至唐府紀善。嘗曰:「湛之體認天理,即王之致良知也。」與信集師說為《新泉問辨錄》。兩家門人各相非笑,衝為疏通其旨焉。

鄒守益,字謙之,安福人。父賢,字恢才,弘治九年進士。授南京大理評事,數有條奏,歷官福建僉事,擒殺武平賊渠黃友勝。居家以孝友稱。

守益舉正德六年會試第一,出王守仁門。以廷對第三人授翰林院編修。踰年告歸,謁守仁,講學於贛州。宸濠反,與守仁軍事。世宗即位,始赴官。嘉靖三年二月,帝欲去興獻帝本生之稱。守益疏諫,忤旨,被責。踰月,復上疏曰:

陛下欲隆本生之恩,屢下群臣會議,群臣據禮正言,致蒙詰讓,道路相傳,有孝長子之稱。昔曾元以父寢疾,憚於易簀,蓋愛之至也。而曾子責之曰:「姑息」。魯公受天子禮樂,以祀周公,蓋尊之至也。而孔子傷之曰「周公其衰矣」。臣願陛下勿以姑息事獻帝,而使後世有其衰之嘆。且群臣援經證古,欲陛下專意正統,此皆為陛下忠謀,乃不察而督過之,謂忤且慢。臣歷觀前史,如冷褒、段猶之徒,當時所謂忠愛,後世所斥以為邪媚也。師丹、司馬光之徒,當時所謂欺慢,後世所仰以為正直也。後之視今,猶今之視古。望陛下不吝改過,察群臣之忠愛,信而用之,復召其去國者,無使姦人動搖國是,離間宮闈。

昔先帝南巡,群臣交章諫阻,先帝赫然震怒,豈不謂欺慢可罪哉。陛下在籓邸聞之,必以是為盡忠於先帝。今入繼大統,獨不容群臣盡忠於陛下乎。

帝大怒,下詔獄拷掠,謫廣德州判官。廢淫祠,建復初書院,與學者講授其間。稍遷南京禮部郎中,州人立生祠以祀。聞守仁卒,為位哭,服心喪,日與呂柟、湛若水、錢德洪、王畿、薛侃輩論學。考滿入都,即引疾歸。久之,以薦起南京吏部郎中,召為司經局洗馬。守益以太子幼,未能出閣,乃與霍韜上《聖功圖》,自神堯茅茨土階,至帝西苑耕稼蠶桑,凡為圖十三。帝以為謗訕,幾得罪,賴韜受帝知,事乃解。明年遷太常少卿兼侍讀學士,出掌南京翰林院,夏言欲遠之也。御史毛愷請留侍東宮,被謫。尋改南京祭酒。九廟災,守益陳上下交修之道,言:「殷中宗、高宗,反妖為祥,亨國長久。」帝大怒,落職歸。

守益天姿純粹。守仁嘗曰:「有若無,實若虛,犯而不校,謙之近之矣。」里居,日事講學,四方從遊者踵至,學者稱東廓先生。居家二十餘年卒。隆慶初,贈南京禮部右侍郎,謚文莊。

先是,守仁主山東試,堂邑穆孔暉第一,後官侍講學士,卒,贈禮部右侍郎,謚文簡。孔暉端雅好學,初不肯宗守仁說,久乃篤信之,自名王氏學,浸淫入於釋氏。而守益於戒懼慎獨,蓋兢兢焉。

子善,嘉靖三十五年進士。以刑部員外郎恤刑湖廣,矜釋甚眾。擢山東提學僉事,時與諸生講學。萬曆初,累官廣東右布政使,謝病歸。久之,以薦即家授太常卿,致仕。子德涵、德溥。德涵,字汝海,隆慶五年進士。歷刑部員外郎。張居正方禁講學,德涵守之自若。御史傅應禎、劉臺相繼論居正,皆德涵里人,疑為黨,出為河南僉事。御史承風指劾之,貶秩歸。善服習父訓,踐履無怠,稱其家學。而德涵從耿定理游,定理不答。發憤湛思,自覺有得,由是專以悟為宗,於祖父所傳,始一變矣。德溥,由萬曆十一年進士。歷司經局洗馬。善從子德泳,萬曆十四年進士。官御史。給事中李獻可請預教太子,斥為民。德泳偕同官救之,亦削籍。家居三十年,言者交薦。光宗立,起尚寶少卿,歷太常卿。魏忠賢用事,乞休歸。所司將為忠賢建祠,德泳塗毀其募籍,乃止。

錢德洪,名寬,字德洪,後以字行,改字洪甫,餘姚人。王守仁自尚書歸里,德洪偕數十人共學焉。四方士踵至,德洪與王畿先為疏通其大旨,而後卒業於守仁。嘉靖五年舉會試,徑歸。七年冬,偕畿赴廷試,聞守仁訃,乃奔喪至貴溪。議喪服,德洪曰:「某有親在,麻衣布絰弗敢有加焉。」畿曰:「我無親。」遂服斬衰。喪歸,德洪與畿築室於場,以終心喪。十一年始成進士。累官刑部郎中。郭勛下詔獄,移部定罪,德洪據獄詞論死。廷臣欲坐以不軌,言德洪不習刑名。而帝雅不欲勛死,因言官疏,下德洪詔獄。所司上其罪,已出獄矣。帝曰:「始朕命刑官毋梏勛,德洪故違之,與勛不領敕何異。」再下獄。御史楊爵、都督趙卿亦在繫,德洪與講《易》不輟。久之,斥為民。德洪既廢,遂周遊四方,講良知學。時士大夫率務講學為名高,而德洪、畿以守仁高第弟子,尤為人所宗。德洪徹悟不如畿,畿持循亦不如德洪,然畿竟入於禪,而德洪猶不失儒者矩矱云。

穆宗立,復官,進階朝列大夫,致仕。神宗嗣位,復進一階。卒年七十九。學者稱緒山先生。

初,守仁倡道其鄉,鄰境從游者甚眾,德洪、畿為之首。其最初受業者,則有餘姚徐愛,山陰蔡宗袞、朱節及應良、盧可久、應典、董涷之屬。

愛,字曰仁,守仁女弟夫也。正德三年進士。官至南京工部郎中。良知之說,學者初多未信,愛為疏通辨析,暢其指要。守仁言:「徐生之溫恭,蔡生之沉潛,朱生之明敏,皆我所不逮。」愛卒,年三十一,守仁哭之慟。一日講畢,歎曰:「安得起曰仁九泉聞斯言乎!」率門人之其墓所,酹酒告之。

蔡宗袞,字希淵。正德十二年進士。官至四川提學僉事。

朱節,字守中。正德八年進士。為御史,巡按山東。大盜起顏神鎮,蔓州縣十數。驅馳戎馬間,以勞卒。贈光祿少卿。

應良,字原忠,仙居人。正德六年進士。官編修。守仁在吏部,良學焉。親老歸養,講學山中者將十年。嘉靖初,還任,伏闕爭大禮,廷杖。張總黜翰林為外官,良得山西副使,謝病歸,卒。

盧可久,字一松。程粹,字養之。皆永康諸生。與同邑應典,皆師守仁。粹子正誼,歷順天府尹。

應典,字天彞。進士。官兵部主事。居家養母,不希榮利。通籍三十年,在官止一考。

可久傳東陽杜惟熙,惟熙傳同邑陳時芳、陳正道。惟熙以克己為要,嘗言:「學者一息不昧,則萬古皆通;一刻少寬,即終朝欠缺。」卒年八十餘。時芳博覽多聞,而歸於實踐。歲貢不仕。正道為建安訓導,年八十餘,猶徒步赴五峰講會。其門人呂一龍,永康人,言動不茍,學者咸宗之。

董涷,字子壽,海寧人。年六十八矣,游會稽,肩瓢笠詩卷謁守仁,卒請為弟子。子穀,官知縣,亦受業守仁。

王畿,字汝中,山陰人。弱冠舉於鄉,跌宕自喜。後受業王守仁,聞其言,無底滯,守仁大喜。嘉靖五年舉進士,與錢德洪並不就廷對歸。守仁征思、田,留畿、德洪主書院。已,奔守仁喪,經紀葬事,持心喪三年。久之,與德洪同第進士。授南京兵部主事,進郎中。給事中戚賢等薦畿。夏言斥畿偽學,奪賢職,畿乃謝病歸。畿嘗云:「學當致知見性而已,應事有小過不足累。」故在官弗免干請,以不謹斥。畿既廢,益務講學,足跡遍東南,吳、楚、閩、越皆有講舍,年八十餘不肯已。善談說,能動人,所至聽者雲集。每講,雜以禪機,亦不自諱也。學者稱龍谿先生。其後,士之浮誕不逞者,率自名龍谿弟子。而泰州王艮亦受業守仁,門徒之盛,與畿相埒,學者稱心齋先生。陽明學派,以龍谿、心齋為得其宗。

艮,字汝止。初名銀,王守仁為更名。七歲受書鄉塾,貧不能竟學。父灶丁,冬晨犯寒,役於官。艮哭曰:「為人子,令父至此,得為人乎!」出代父役,入定省,惟謹。艮讀書,止《孝經》、《論語》、《大學》,信口談說,中理解。有客聞艮言,詫言:「何類王中丞語。」艮乃謁守仁江西,與守仁辨久之,大服,拜為弟子。明日告之悔,復就賓位自如。已,心折,卒稱弟子。從守仁歸里,歎曰:「吾師倡明絕學,何風之不廣也!」還家,製小車北上,所過招要人士,告以守仁之道,人聚觀者千百。抵京師,同門生駭異,匿其車,趣使返。守仁聞之,不悅。艮往謁,拒不見,長跪謝過乃已。王氏弟子遍天下,率都爵位有氣勢。艮以布衣抗其間,聲名反出諸弟子上。然艮本狂士,往往駕師說上之,持論益高遠,出入於二氏。

艮傳林春、徐樾,樾傳顏鈞,鈞傳羅汝芳、梁汝元,汝芳傳楊起元、周汝登、蔡悉。

樾,字子直,貴溪人。舉進士。歷官雲南左布政使。元江土酋那鑑反,詐降。樾信之,抵其城下,死焉。詔贈光祿寺卿,予祭葬,任一子官。

春,字子仁,泰州人。聞良知之學,日以朱墨筆識臧否自考,動有繩檢,尺寸不踰。嘉靖十一年會試第一,除戶部主事,調吏部。縉紳士講學京師者數十人,聰明解悟善談說者,推王畿,志行敦實推春及羅洪先。進文選郎中,卒官,年四十四。發其篋,僅白金四兩,僚友棺斂歸其喪。

汝芳,字維德,南城人。嘉靖三十二年進士。除太湖知縣。召諸生論學,公事多決於講座。遷刑部主事,歷寧國知府。民兄弟爭產,汝芳對之泣,民亦泣,訟乃已。創開元會,罪囚亦令聽講。入覲,勸徐階聚四方計吏講學。階遂大會於靈濟宮,聽者數千人。父艱,服闋,起補東昌,移雲南屯田副使,進參政,分守永昌,坐事為言官論罷。初,汝芳從永新顏鈞講學,後鈞繫南京獄當死,汝芳供養獄中,鬻產救之,得減戍。汝芳既罷官,鈞亦赦歸。汝芳事之,飲食必躬進,人以為難。鈞詭怪猖狂,其學歸釋氏,故汝芳之學亦近釋。

楊起元、周汝登,皆萬曆五年進士。起元,歸善人。選庶吉士,適汝芳以參政入賀,遂學焉。張居正方惡講學,汝芳被劾罷,而起元自如,累官吏部左侍郎。拾遺被劾,帝不問。未幾卒。天啟初,追謚文懿。汝登,嵊人。初為南京工部主事,榷稅不如額,謫兩淮鹽運判官,累官南京尚寶卿。起元清修姱節,然其學不諱禪。汝登更欲合儒釋而會通之,輯《聖學宗傳》,盡採先儒語類禪者以入。蓋萬曆世士大夫講學者,多類此。

蔡悉,字士備,合肥人。嘉靖三十八年進士。授常德推官。築郭外六隄以免水患。擢南京吏部主事,累官南京尚寶卿,移署國子監。嘗請立東宮,又極論礦稅之害。有學行,恬宦情。仕五十年,家食強半。清操亮節,淮西人宗之。

歐陽德,字崇一,泰和人。甫冠舉鄉試。之贛州,從王守仁學。不應會試者再。嘉靖二年策問陰詆守仁,德與魏良弼等直發師訓無所阿,竟登第。除知六安州,建龍津書院,聚生徒論學。入為刑部員外郎。六年詔簡朝士有學行者為翰林,乃改德編修。遷南京國子司業,作講亭,進諸生與四方學者論道其中。尋改南京尚寶卿。召為太僕少卿。以便養,復改南京鴻臚卿。父憂,服闋,留養其母,與鄒守益、聶豹、羅洪先日講學。以薦起故官。累遷吏部左侍郎兼學士,掌詹事府。母憂歸,服未闋,即用為禮部尚書。喪畢之官,命直無逸殿。時儲位久虛,帝惑陶仲文「二龍不相見」之說,諱言建儲,德懇請。會有詔,二王出邸同日婚。德以裕王儲貳不當出外,疏言:「曩太祖以父婚子,諸王皆處禁中。宣宗、孝宗以兄婚弟,始出外府。今事與太祖同,請從初制。」帝不許。德又言:「《會典》醮詞,主器則曰承宗,分籓則曰承家。今裕王當何從?」帝不悅曰:「既云王禮,自有典制。如若言,何不竟行冊立耶?」德即具冊立儀上。帝滋不悅,然終諒其誠,婚亦竟不同日。裕王母康妃杜氏薨,德請用成化朝紀淑妃故事,不從。德遇事侃侃,裁制諸宗籓尤有執。或當利害,眾相顧色戰,德意氣自如。

當是時,德與徐階、聶豹、程文德並以宿學都顯位。於是集四方名士於靈濟宮,與論良知之學。赴者五千人。都城講學之會,於斯為盛。德器宇溫粹,學務實踐,不尚空虛。晚見知於帝,將柄用,而德遽卒。贈太子少保,謚文莊。

族人瑜,字汝重,亦學於守仁。守仁教之曰:「常舀然無自是而已。」瑜終身踐之。舉於鄉,不就會試,曰:「老親在,三公不與易也。」母死,廬墓側。虎環廬嗥,不為動。歷官四川參議,所至有廉惠聲。年近九十而卒。

羅洪先,字達夫,吉水人。父循,進士。歷兵部武選郎中。會考選武職,有指揮二十餘人素出劉瑾門,循罷其管事。瑾怒罵尚書王敞,敞懼,歸部趣易奏。循故遲之,數日瑾敗,敞乃謝循。循歷知鎮江、淮安二府,徐州兵備副使,咸有聲。

洪先幼慕羅倫為人。年十五,讀王守仁《傳習錄》好之,欲往受業,循不可而止。乃師事同邑李中,傳其學。嘉靖八年舉進士第一,授修撰,即請告歸。外舅太僕卿曾直喜曰:「幸吾婿成大名。」洪先曰:「儒者事業有大於此者。此三年一人,安足喜也。」洪先事親孝。父每肅客,洪先冠帶行酒、拂席、授几甚恭。居二年,詔劾請告踰期者,乃赴官。尋遭父喪,苫塊蔬食,不入室者三年。繼遭母憂,亦如之。

十八年簡宮僚,召拜春坊左贊善。明年冬,與司諫唐順之、校書趙時春疏請來歲朝正後,皇太子出御文華殿,受群臣朝賀。時帝數稱疾不視朝,諱言儲貳臨朝事,見洪先等疏,大怒曰:「是料朕必不起也。」降手詔百餘言切責之,遂除三人名。

洪先歸,益尋求守仁學。甘淡泊,煉寒暑,躍馬挽強,考圖觀史,自天文、地志、禮樂、典章、河渠、邊塞、戰陣攻守,下逮陰陽、算數,靡不精究。至人才、吏事、國計、民情,悉加意諮訪。曰:「茍當其任,皆吾事也。」邑田賦多宿弊,請所司均之,所司即以屬。洪先精心體察,弊頓除。歲饑,移書郡邑,得粟數十石,率友人躬振給。流寇入吉安,主者失措。為畫策戰守,寇引去。素與順之友善。順之應召,欲挽之出,嚴嵩以同鄉故,擢假邊才起用,皆力辭。

洪先雖宗良知學,然未嘗及守仁門,恒舉《易大傳》「寂然不動」、周子「無欲故靜」之旨以告學人。又曰:「儒者學在經世,而以無欲為本。惟無欲,然後出而經世,識精而力鉅。」時王畿謂良知自然,不假纖毫力。洪先非之曰:「世豈有現成良知者耶?」雖與畿交好,而持論始終不合。山中有石洞,舊為虎穴,葺茅居之,命曰石蓮。謝客,默坐一榻,三年不出戶。

初,告歸,過儀真,同年生主事項喬為分司。有富人坐死,行萬金求為地,洪先拒不聽。喬微諷之,厲聲曰:「君不聞志士不忘在溝壑耶?」江漲,壞其室,巡撫馬森欲為營之,固辭不可。隆慶初卒,贈光祿少卿,謚文莊。

程文德,字舜敷,永康人。初受業章懋,後從王守仁遊。登洪先榜進士第二,授翰林編修。坐同年生楊名劾汪鋐事,下詔獄,謫信宜典史。鋐罷,量移安福知縣,遷兵部員外郎。父憂,廬墓側,終喪不入內。起兵部郎中,擢廣東提學副使,未赴,改南京國子祭酒。母憂,服闋,起禮部右侍郎。俺答犯京師,分守宣武門,盡納鄉民避寇者。調吏部為左。已,改掌詹事府。三十三年,供事西苑。所撰青詞,頗有所規諷,帝銜之。會推南京吏部尚書,帝疑文德欲遠己,命調南京工部右侍郎。文德疏辭,勸帝享安靜和平之福。帝以為謗訕,除其名。既歸,聚徒講學。卒,貧不能殮。萬曆間,追贈禮部尚書,謚文恭。

吳悌,字思誠,金谿人。嘉靖十一年進士。除樂安知縣,調繁宣城,徵授御史。十六年,應天府進試錄,考官評語失書名,諸生答策多譏時政。帝怒,逮考官諭德江汝璧、洗馬歐陽衢詔獄,貶官,府尹孫懋等下南京法司,尋得還職,而停舉子會試。悌為舉子求寬,坐下詔獄,出視兩淮鹽政。海溢,沒通、泰民廬,悌先發漕振之而後奏聞。尋引疾歸,還朝,按河南。伊王典楧驕橫,憚悌,遺書稱為友。悌報曰:「殿下,天子親籓,非悌所敢友。悌,天子憲臣,非殿下所得友。」王愈憚之。夏言、嚴嵩當國,與悌鄉里。嘗謁言,眾見言新服宮袍,競前譽之,悌卻立不進。言問故,徐曰:「俟談少間,當以政請。」言為改容。及嵩擅政,悌惡之,引疾家居垂二十年。嵩敗,起故官,一歲中累遷至南京大理卿。時吳嶽、胡松、毛愷並以耆俊為卿貳,與悌稱「南都四君子」。隆慶元年就遷刑部侍郎。明年卒。

悌為王守仁學,然清修果介,反躬自得為多。萬曆中,子仁度請恤。吏部尚書孫丕揚曰:「悌,理學名臣,不宜循常格。」遂用黃孔昭例,贈禮部尚書,謚文莊。鄉人建祠,與陸九淵、吳澄、吳與弼、陳九川並祀,曰五賢祠,學者稱疏山先生。

仁度,字繼疏。萬曆十七年進士。授中書舍人。三王並封議起,抗疏爭之。久之,擢吏部主事,歷考功郎中。稽勳郎中趙邦清被劾,疑同官鄧光祚等嗾言路,憤激力辨。章下考功,仁度欲稍寬邦清罰,給事中梁有年遂劾仁度黨比。時光祚引疾去,而仁度代為文選,御史康丕揚復劾仁度傾光祚而代之,詔改調之南京。自邦清被論後,言路訐不已,都御史溫純恚甚,請定國是,以剖眾疑,而深為仁度惜。仁度尋補南京刑部郎中,擢太僕少卿,進右僉都御史,巡撫山西。砥廉隅,務慈愛,與魏允貞齊名。居四年,以疾歸。熹宗初,起大理卿,進兵部右侍郎,復稱疾去。再起工部左侍郎。天啟五年,魏忠賢以仁度與趙南星、楊漣等善,勒令致仕,尋卒。仁度,名父子,克自振勵,鄒元標亟稱之。

何廷仁,初名秦,以字行,改字性之。黃弘綱,字正之。皆雩都人。廷仁和厚,與人接,誠意盎溢。而弘綱難近,未嘗假色笑於人。然兩人志行相準。廷仁初慕陳獻章,後聞王守仁之學於弘綱。守仁征桶岡,詣軍門謁,遂師事焉。嘉靖元年舉於鄉,復從守仁浙東。廷仁立論尚平實,守仁歿後,有為過高之論者,輒曰:「此非吾師言也。」除新會知縣,釋菜獻章祠,而後視事。政尚簡易,士民愛之。遷南京工部主事,分司儀真,榷蕪湖稅,不私一錢。滿考,即致仕。弘綱由鄉舉官刑部主事。

守仁之門,從遊者恒數百,浙東、江西尤眾,善推演師說者稱弘綱、廷仁及錢德洪、王畿。時人語曰:「江有何、黃,浙有錢、王。」然守仁之學,傳山陰、泰州者,流弊靡所底極,惟江西多實踐,安福則劉邦采,新建則魏良政兄弟,其最著云。

邦采,字君亮。族子曉受業守仁,歸語邦采,遂與從兄文敏及弟姪九人謁守仁於里第,師事焉。父憂,蔬水廬墓。免喪,不復應舉。提學副使趙淵檄赴試,御史儲良才許以常服入闈,不解衣檢察,乃就試,得中式。久之,除壽寧教諭,擢嘉興府同知,棄官歸。邦采識高明,用力果銳。守仁倡良知為學的,久益敝,有以揣摩為妙悟,縱恣為自然者,邦采每極言排斥焉。

文敏,字宜充。父喪除,絕意科舉。嘗曰:「學者當循本心之明,時見己過,刮磨砥礪,以融氣稟,絕外誘,徵諸倫理、事物之實,無一不慊於心,而後為聖門正學,非困勉不可得入也。高談虛悟,炫未離本,非德之賊乎?」曉,字伯光。舉於鄉,後為新寧知縣,有善政。

良政,字師伊。守仁撫江西,與兄良弼,弟良器、良貴,咸學焉。提學副使邵銳、巡按御史唐龍持論與守仁異,戒諸生勿往謁,良政兄弟獨不顧,深為守仁所許。良政功尤專,孝友敦朴,燕居無惰容,嘗曰:「不尤人,何人不可處;不累事,何事不可為。」舉鄉試第一而卒。良弼嘗言,「吾夢見師伊,輒汗浹背」,其為兄憚如此。良器,字師顏。性超穎絕人,雖宗良知,踐履務平實。良弼,自有傳。良貴,官右副都御史。

王時槐,字子植,安福人。嘉靖二十六年進士。授南京兵部主事。歷禮部郎中、福建僉事。累官太僕少卿,降光祿少卿。隆慶末,出為陜西參政。張居正柄國,以京察罷歸。萬曆中,南贛巡撫張岳疏薦之。吏部言:「六年京察,祖制也。若執政有所驅除,非時一舉,謂之閏察。時槐在閏察中,群情不服,請召時槐,且永停閏察。」報可。久之,陸光祖掌銓,起貴州參政,旋擢南京鴻臚卿,進太常,皆不赴。

時槐師同縣劉文敏,及仕,遍質四方學者,自謂終無所得。年五十,罷官,反身實證,始悟造化生生之幾,不隨念慮起滅。學者欲識真幾,當從慎獨入。其論性曰:「孟子性善之說,決不可易。使性中本無仁義,則惻隱羞惡更何從生。且人應事接物,如是則安,不如是則不安,非善而何?」又曰:「居敬、窮理,二者不可廢一。要之,居敬二字盡之。自其居敬之精明了悟而言,謂之窮理,即考索討論,亦居敬中之一事。敬無所不該,敬外更無餘事也。」年八十四卒。

廬陵陳嘉謨,字世顯,與時槐同年進士。為給事中,不附嚴嵩,出之外。歷湖廣參政,乞休歸,專用力於學。凡及其門者,告之曰:「有塘南在,可往師之。」塘南,時槐別號也。年八十三卒。

許孚遠,字孟中,德清人,受學同郡唐樞。嘉靖四十一年成進士,授南京工部主事,就改吏部。已,調北部。尚書楊博惡孚遠講學,會大計京朝官,黜浙人幾半,博鄉山西無一焉。孚遠有後言,博不悅,孚遠遂移疾去。隆慶初,高拱薦起考功主事,出為廣東僉事,招大盜李茂、許俊美,擒倭黨七十餘輩以降,錄功,賚銀幣。旋移福建。神宗立,拱罷政,張居正議逐拱黨,復大計京官。王篆為考功,誣孚遠黨拱,謫兩淮鹽運司判官。歷兵部郎中,出知建昌府,暇輒集諸生講學,引貢士鄧元錫、劉元卿為友。尋以給事中鄒元標薦,擢陜西提學副使,敬禮貢士王之士,移書當路,並元卿、元錫薦之。後三人並得征,由孚遠倡也。遷應天府丞,坐為李材訟冤,貶二秩,由廣東僉事再遷右通政。二十年擢右僉都御史,巡撫福建。倭陷朝鮮,議封貢,孚遠請敕諭日本,擒斬平秀吉,不從。呂宋國酋子訟商人襲殺其父,孚遠以聞,詔戮罪人,厚犒其使。福州饑,民掠官府,孚遠擒倡首者,亂稍定,而給事中耿隨龍、御史甘士價等劾孚遠宜斥,帝不問。所部多僧田,孚遠入其六於官。又募民墾海壇地八萬三千有奇,築城建營舍,聚兵以守,因請推行於南日、彭湖及浙中陳錢、金塘、玉環、南麂諸島,皆報可。居三年,入為南京大理卿,就遷兵部右侍郎,改左,調北部。甫半道,被論。乞休,疏屢上,乃許。又數年,卒於家,贈南京工部尚書,後謚恭簡。

孚遠篤信良知,而惡夫援良知以入佛者。知建昌,與郡人羅汝芳講學不合。及官南京,與汝芳門人禮部侍郎楊起元、尚寶司卿周汝登,並主講席。汝登以無善無惡為宗,孚遠作《九諦》以難之,言:「文成宗旨,原與聖門不異,以性無不善,故知無不良。良知即是未發之中,立論至為明析。無善無惡心之體一語,蓋指其未發時,廓然寂然者而言之,止形容得一靜字,合下三語,始為無病。今以心意知物,俱無善惡可言者,非文成之正傳也。」彼此論益齟齠。而孚遠撫福建,與巡按御史陳子貞不相得,子貞督學南畿,遂密諷同列拾遺劾之。從孚遠游者,馮從吾、劉宗周、丁元薦,皆為名儒。

尤時熙,字季美,洛陽人。生而警敏不群,弱冠舉嘉靖元年鄉試。時王守仁《傳習錄》始出,士大夫多力排之,時熙一見嘆曰:「道不在是乎?向吾役志詞章,末矣。」已而以疾稍從事養生家。授元氏教諭,父喪除,改官章丘,一以致良知為教,兩邑士亦知新建學。入為國子博士,徐階為祭酒,命六館士咸取法焉。居常以不獲師事守仁為恨,聞郎中劉魁得守仁之傳,遂師事之。魁以直言錮詔獄,則書所疑,時時從獄中質問。尋以戶部主事榷稅滸墅,課足而止,不私一錢。念母老,乞終養歸,遂不出,日以修己淑人為事,足未嘗涉公府。齋中設守仁位,晨興必焚香肅拜,來學者亦令民謁。晚年,病學者憑虛見而忽躬行,甚且越繩墨自恣,故其論議切於日用,不為空虛隱怪之談。卒於萬曆八年,年七十有八,學者稱西川先生。其門人,孟化鯉最著,自有傳。

張後覺,字志仁,茌平人。父文祥,由鄉舉官廣昌知縣。後覺生有異質,事親考,居喪哀毀,三年不御內。早歲,聞良知之說於縣教諭顏鑰,遂精思力踐,偕同志講習。已而貴谿徐樾以王守仁再傳弟子來為參政,後覺率同志往師之,學益有聞。久之,以歲貢生授華陰訓導,會地大震,人多傾壓死,上官令署縣事,救災扶傷,人胥悅服。及致仕歸,士民泣送載道。

東昌知府羅汝芳、提學副使鄒善皆宗守仁學,與後覺同志。善為建願學書院,俾六郡士師事焉。汝芳亦建見泰書院,時相討論。猶以取友未廣,北走京師,南游江左,務以親賢講學為事,門弟子日益進。凡吏於其土及道經茌平者,莫不造廬問業。巡撫李世達兩詣山居,病不能為禮,乃促席劇談,飽蔬食而去。平生不作詩,不談禪,不事著述,行孚遠近,學者稱之為弘山先生。年七十六,以萬曆六年卒。

其門人,孟秋、趙維新最著。秋,自有傳。維新,亦茌平人,年二十,聞後覺講良知之學。遂師事之。次其問答語,為《弘山教言》。性純孝,居喪,五味不入口,柴毀骨立,杖而後起。鄉人欲舉其孝行,力辭之。喪偶,五十年不再娶。嘗築垣得金一篋,工人持之去,維新不問。家貧,或併日而食,超然自得。亦以歲貢生為長山訓導,年九十二,無疾而終。

鄧以贊,字汝德,新建人。張元忭,字子藎,紹興山陰人。二人皆生有異質,又好讀書。以贊幼,見父與人論學,輒牽衣尾之,間出語類夙儒。父閔其勤學,嘗扃之斗室。元忭素羸弱,母戒毋過勞,乃藏燈幕中,俟母寢始誦。十餘歲時以氣節自負,聞楊繼盛死,為文遙誄之,慷慨泣下。父天復,官雲南副使,擊武定賊鳳繼祖有功。已,賊還襲武定,官軍敗績,巡撫呂光洵討滅之。至隆慶初,議者追理前失亡狀,逮天復赴雲南對簿,元忭適下第還,萬里護行,髮盡白。已,復馳詣闕下白冤,當事憐之,天復得削籍歸。

隆慶五年,以贊舉會試第一,廷試第三,授編修,而元忭以廷試第一,授修撰。萬曆初,座主張居正枋國政,以贊時有匡諫,居正弗善也,移疾歸。久之,補原官,旋引退。詔起中允,至中途復以念母返。再起南京祭酒,就擢禮部右侍郎,復就轉吏部,再疏請建儲,且力斥三王並封之非,中言:「中宮鐘愛元子,其願早正春宮,視臣民尤切。陛下以厚中宮而緩冊立,殆未諒中宮心。況信者,國之大寶,建儲一事,屢示更移,將使詔令不信於天下,非所以重宗廟,安社稷也。」會廷臣多諫者,事竟寢。尋召為吏部右侍郎,力辭不拜。以贊登第二十餘年,在官僅滿一考。居母憂,不勝喪而卒,贈禮部尚書,謚文潔。

元忭嘗抗疏救御史胡涍,又請進講《列女傳》於兩宮,修《二南》之化,皆不省。萬曆十年奉使楚府還,過家省母,既行心動,輒馳歸,僅五日,母卒。元忭奉二親疾,湯藥非口嘗弗進,居喪毀瘠,遵用古禮,鄉人多化之。服闋,起故官,進左諭德,直經筵。先是,元忭以帝登極恩,請復父官,詔許給冠帶。至是復申前請,格不從。元忭泣曰:「吾無以下見父母矣。」遂悒悒得疾卒。天啟初,追謚文恭。

以贊、元忭自未第時即從王畿游,傳良知之學,然皆篤於孝行,躬行實踐。以贊品端志潔,而元忭矩矱儼然,無流入禪寂之弊。元忭子汝霖,江西參議。汝懋,御史。

孟化鯉,字叔龍,河南新安人。孟秋,字子成,茌平人。化鯉年十六,慨然以聖賢自期。而秋兒時受《詩》,至《桑中》諸篇,輒棄去不竟讀。化鯉舉萬曆八年進士。授戶部主事,時相欲招致之,辭不往。榷稅河西務,與諸生講學,河西人尸祝之。南畿、山東大饑,奉命往振,全活多。改吏部,歷文選郎中,佐尚書孫鑨黜陟,名籍甚。時內閣權重,每銓除必先白,化鯉獨否,中官請託復不應,以故多不悅。都給事中張棟先以建言削籍,化鯉奏起之,忤旨,奪堂官俸,謫化鯉及員外郎項復弘、主事姜仲軾雜職。閣臣疏救,命以原品調外。頃之,言官復交章救,帝益怒,奪言官俸,斥化鯉等為民。既歸,築書院川上,與學者講習不輟,四方從游者恒數百人。久之卒。

秋舉隆慶五年進士。為昌黎知縣,有善政。遷大理評事,去之日,老稚載道泣留。以職方員外郎督視山海關。關政久馳,奸人出入自擅,秋禁之嚴。中流言,萬曆九年京察坐貶,歸塗與妻孥共駕一牛車,道旁觀者咸歎息。許孚遠嘗過張秋,造其廬,見茆屋數椽,書史狼藉其中,歎曰:「孟我疆風味,大江以南未有也。」我疆者,秋別號也。後起官刑部主事,歷尚寶丞少卿,卒。秋既歿,廷臣為請謚者章數十上。天啟初,賜謚清憲。

化鯉自貢入太學,即與秋道義相勖,後為吏部郎,而秋官尚寶,比舍居,食飲起居無弗共者,時人稱「二孟」。化鯉之學得之洛陽尤時熙,而秋受業於邑人張後覺。時熙師曰劉魁,後覺則顏鑰、徐樾弟子也。

來知德,字矣鮮,梁山人。幼有至行,有司舉為孝童。嘉靖三十一年舉於鄉。二親相繼歿,廬墓六年,不飲酒茹葷。服除,傷不及祿養,終身麻衣蔬食,誓不見有司。其學以致知為本,盡倫為要。所著有《省覺錄》、《省事錄》、《理學辨疑》、《心學晦明解》諸書,而《周易集註》一篇用功尤篤。自言學莫邃於《易》。初,結廬釜山,學之六年無所得。後遠客求溪山中,覃思者數年,始悟《易》象。又數年始悟文王《序卦》、孔子《雜卦》之意。又數年始悟卦變之非。蓋二十九年而後書成。萬曆三十年,總督王象乾、巡撫郭子章合詞論薦,特授翰林待詔。知德力辭,詔以所授官致仕,有司月給米三石,終其身。

鄧元錫,字汝極,南城人。十五喪父,水漿不入口。十七行社倉法,惠其鄉人。已為諸生,遊邑人羅汝芳門,又走吉安,學於諸先達。嘉靖三十四年舉於鄉,復從鄒守益、劉邦采、劉陽諸宿儒論學。後不復會試,杜門著述,踰三十年,《五經》皆有成書,閎深博奧,學者稱潛谷先生。

休寧範淶知南城時,重元錫。後為南昌知府,萬曆十六年入覲,薦元錫及劉元卿、章潢於朝。南京祭酒趙用賢亦請徵聘,如吳與弼、陳獻章故事。得旨,有司起送部試,元錫固辭。明年,御史王道顯復以元錫、元卿並薦,且請仿祖宗征辟故事,無拘部試。詔令有司問病,痊可起送赴部,竟不行。二十一年,巡按御史秦大夔復並薦二人,詔以翰林待詔征之,有司敦遣上道,甫離家而卒。鄉人私謚文統先生。

元錫之學,淵源王守仁,不盡宗其說。時心學盛行,謂學惟無覺,一覺即無餘蘊,九容、九思、四教、六藝皆桎梏也。元錫力排之,故生平博極群書,而要歸於《六經》。所著《五經繹》、《函史上下編》、《皇明書》,並行於世。

元卿,字調父,安福人。舉隆慶四年鄉試,明年會試,對策極陳時弊,主者不敢錄。張居正聞而大怒,下所司申飭,且令人密詗之,其人反以情告,乃獲免。既歸,師同邑劉陽,王守仁弟子也。萬曆二年,會試不第,遂絕意科名,務以求道為事。既累被薦,乃召為國子博士。擢禮部主事,疏請早朝勤政,又請從祀鄒守益、王艮於文廟,釐正外蕃朝貢舊儀。尋引疾歸,肆力撰述,有《山居草》、《還山續草》、《諸儒學案》、《賢弈編》、《思問編》、《禮律類要》、《大學新編》、諸書。

潢,字本清,南昌人。居父喪,哀毀血溢。構此洗堂,聯同志講學。輯群書百二十七卷,曰《圖書編》。又著《周易象義》、《時經原體》、《書經原始》、《春秋竊義》、《禮記》答刂言》、《論語約言》諸書。從游者甚眾。數被薦,從吏部侍郎楊時喬請,遙授順天訓導,如陳獻章、來知德故事,有司月給米三石贍其家。卒於萬歷三十六年,年八十二。其鄉人稱潢自少迄老,口無非禮之言,身無非禮之行,交無非禮之友,目無非禮之書,乃私謚文德先生。自吳與弼後,元錫、元卿、潢並蒙薦闢,號「江右四君子」。


\end{pinyinscope}