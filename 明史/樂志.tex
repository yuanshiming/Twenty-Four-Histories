\article{樂志}


古先聖王,治定功成而作樂,以合天地之性,類萬物之情,天神格而民志協。蓋樂者心聲也,君心和,六合之內無不和矣。是以樂作於上,民化於下。秦、漢而降,斯理浸微,聲音之道與政治不相通,而民之風俗日趨於靡曼。明興,太祖銳志雅樂。是時,儒臣冷謙、陶凱、詹同、宋濂、樂韶鳳輩皆知聲律,相與究切釐定。而掌故闊略,欲還古音,其道無由。太祖亦方以下情偷薄,務嚴刑以束之,其於履中蹈和之本,未暇及也。文皇帝訪問黃鐘之律,臣工無能應者。英、景、憲、孝之世,宮縣徒為具文。殿廷燕享,郊壇祭祀,教坊羽流,慢瀆茍簡,劉翔、胡瑞為之深慨。世宗制作自任,張鶚、李文察以審音受知,終以無成。蓋學士大夫之著述止能論其理,而施諸五音六律輒多未協,樂官能紀其鏗鏘鼓舞而不曉其義,是以卒世莫能明也。稽明代之制作,大抵集漢、唐、宋、元人之舊,而稍更易其名。凡聲容之次第,器數之繁縟,在當日非不爛然俱舉,第雅俗雜出,無從正之。故備列於篇,以資考者。

太祖初克金陵,即立典樂官。其明年置雅樂,以供郊社之祭。吳元年,命自今朝賀,不用女樂。先是命選道童充樂舞生,至是始集。太祖御戟門,召學士朱升、范權引樂舞生入見,閱試之。太祖親擊石磬,命升辨五音。升不能審,以宮音為徵音。太祖哂其誤,命樂生登歌一曲而罷。是年置太常司,其屬有協律郎等官。元末有冷謙者,知音,善鼓瑟,以黃冠隱吳山。召為協律郎,令協樂章聲譜,俾樂生習之。取石靈璧以製磬,採桐梓湖州以製琴瑟。乃考正四廟雅樂,命謙較定音律及編鐘、編磬等器,遂定樂舞之制。樂生仍用道童,舞生改用軍民俊秀子弟。又置教坊司,掌宴會大樂。設大使、副使、和聲郎,左、右韶樂,左右司樂,皆以樂工為之。後改和聲郎為奉鑾。

洪武元年春,親祭太社、太稷。夏祫享於太廟。其冬祀昊天上帝於圜丘。明年,祀皇地祇於方丘,又以次祀先農、日月、太歲、風雷、嶽瀆、周天星辰、歷代帝王、至聖文宣王,皆定樂舞之數,奏曲之名。

圜丘:迎神,奏《中和之曲》。奠玉帛,奏《肅和之曲》。奉牲,奏《凝和之曲》。初獻,奏《壽和之曲》,《武功之舞》。亞獻,奏《豫和之曲》,終獻,奏《熙和之曲》,俱《文德之舞》。徹豆,奏《雍和之曲》。送神,奏《安和之曲》。望燎,奏《時和之曲》。方丘並同,曲詞各異,易望燎曰望瘞。太社太稷,易迎神曰《廣和》,省奉牲,餘並與方丘同,曲詞各異。

先農:迎神、奠帛,奏《永和之曲》。進俎,奏《雍和之曲》。初獻、終獻,並奏《壽和之曲》。徹豆、送神,並奏《永和之曲》。望瘞,奏《太和之曲》。

朝日:迎神,奏《熙和之曲》。奠玉帛,奏《保和之曲》。初獻,奏《安和之曲》,《武功之舞》。亞獻,奏《中和之曲》,終獻,奏《肅和之曲》,俱《文德之舞》。徹豆,奏《凝和之曲》。送神,奏《壽和之曲》。望燎,奏《豫和之曲》。夕月,迎神易《凝和》,奠帛以下與朝日同,曲詞各異。

太歲、風雷、嶽瀆:迎神,奏《中和》。奠帛,奏《安和》。初獻,奏《保和》。亞獻,奏《肅和》。終獻,奏《凝和》。徹豆,奏《壽和》。送神,奏《豫和》。望燎,奏《熙和》。

周天星辰,初附祀夕月,洪武四年別祀:迎神,奏《凝和》。奠帛、初獻,奏《保和》,《武功舞》。亞獻,奏《中和》,終獻,奏《肅和》,俱《文德舞》。徹豆,奏《豫和》。送神,奏《雍和》。

太廟:迎神,奏《太和之曲》。奉冊寶,奏《熙和之曲》。進俎,奏《凝和之曲》。初獻,奏《壽和之曲》,《武功之舞》。亞獻,奏《豫和之曲》,終獻,奏《熙和之曲》,俱《文德之舞》。徹豆,奏《雍和之曲》。送神,奏《安和之曲》。初獻則德、懿、熙、仁各奏樂舞,亞、終獻則四廟共之。

釋奠孔子:初用大成登歌舊樂。洪武六年,始命詹同、樂韶鳳等更製樂章。迎神,奏《咸和》。奠帛,奏《寧和》。初獻,奏《安和》。亞獻、終獻,奏《景和》。徹饌、送神,奏《咸和》。

歷代帝王:迎神,奏《雍和》。奠帛、初獻,奏《保和》,《武功舞》。亞獻,奏《中和》,終獻,奏《肅和》,俱《文德舞》。徹豆,奏《凝和》。送神,奏《壽和》。望瘞,奏《豫和》。

又定王國祭祀樂章:迎神,奏《太清之曲》。初獻,奏《壽清之曲》。亞獻,奏《豫清之曲》。終獻,奏《熙清之曲》。徹饌,奏《雍清之曲》。送神,奏《安清之曲》。其社稷山川,易迎神為《廣清》,增奉瘞曰《時清》。

此祭祀之樂歌節奏也。

洪武三年,又定朝會宴饗之制。

凡聖節、正旦、冬至、大朝賀,和聲郎陳樂於丹墀百官拜位之南,北向。駕出,仗動。和聲郎舉麾,奏《飛龍引之曲》,樂作,升座。樂止,偃麾。百官拜,奏《風雲會之曲》,拜畢,樂止。丞相上殿致詞,奏《慶皇都之曲》,致詞畢,樂止。百官又拜,奏《喜昇平之曲》,拜畢,樂止。駕興,奏《賀聖朝之曲》,還宮,樂止。百官退,和聲郎、樂工以次出。

凡宴饗,和聲郎四人總樂舞,二人執麾,立樂工前之兩旁;二人押樂,立樂工後之兩旁。殿上陳設畢,和聲郎執麾由兩階升,立於御酒案之左右;二人引歌工、樂工由兩階升,立於丹陛上之兩旁,東西向。舞師二人執旌,引武舞士立於西階下之南;又二人執翿,引文舞士立於東階下之南;又二人執幢,引四夷舞士立於武舞之西南;俱北向。武舞曰《平定天下之舞》,象以武功定禍亂也;文舞曰《車書會同之舞》,象以文德致太平也;四夷舞曰《撫安四夷之舞》,象以威德服遠人也。此大樂二人,執戲竹,引大樂工陳列於丹陛之西,文武二舞樂工列於丹陛之東,四夷樂工列於四夷舞之北,俱北向。駕將出,仗動,大樂作。升座,樂止。進第一爵,和聲郎舉麾,唱奏《起臨濠之曲》。引樂二人引歌工、樂工詣酒案前,北面,重行立定。奏畢,偃麾,押樂引眾工退。第二,奏《開太平之曲》。第三,奏《安建業之曲》。第四,奏《大一統之曲》。第五,奏《平幽都之典》。第六,奏《撫四夷之曲》。第七,奏《定封賞之曲》。第八,奏《大一統之曲》。第九,奏《守承平之曲》。其舉麾、偃麾,歌工、樂工進退,皆如前儀。進第一次膳,和聲郎舉麾,唱奏《飛龍引之樂》,大樂作。食畢,樂止,偃麾。第二,奏《風雲會之樂》。第三,奏《慶皇都之樂》。第四,奏《平定天下之舞》。第五,奏《賀聖朝之樂》。第六,奏《撫安四夷之舞》。第七,奏《九重歡之樂》。第八,奏《車書會同之舞》。第九,奏《萬年春之樂》。其舉麾、偃麾如前儀。九奏三舞既畢,駕興,大樂作。入宮,樂止,和聲郎執麾引眾工以次出。

宴饗之曲,後凡再更。四年所定,一曰《本太初》,二曰《仰大明》,三曰《民初生》,四曰《品物亨》,五曰《御六龍》,六曰《泰階平》,七曰《君德成》,八曰《聖道行》,九曰《樂清寧》。其詞,詹同、陶凱所製也。十五年所定,一曰《炎精開運》,二曰《皇風》,三曰《眷皇明》,四曰《天道傳》,五曰《振皇綱》,六曰《金陵》,七曰《長楊》,八曰《芳醴》,九曰《駕六龍》。

凡大朝賀,教坊司設中和韶樂於殿之東西,北向;陳大舞於丹陛之東西,亦北向。駕興,中和韶樂奏《聖安之曲》。陞座進寶,樂止。百官拜,大樂作。拜畢,樂止。進表,大樂作。進訖,樂止。宣表目,致賀訖,百官俯伏,大樂作。拜畢,樂止。宣制訖,百官舞蹈山呼,大樂作。拜畢,樂止。駕興,中和韶樂奏《定安之曲》,導駕至華蓋殿,樂止。百官以次出。

其大宴饗,教坊司設中和韶樂於殿內,設大樂於殿外,立三舞雜隊於殿下。駕興,大樂作。升座,樂止。文武官入列於殿外,北向拜,大樂作。拜畢,樂止。進御筵,樂作。進訖,樂止。進花,樂作。進訖,樂止。進第一爵,教坊司奏《炎精開運之曲》,樂作。內外官拜畢,樂止。散花,樂作。散訖,樂止。第二爵,教坊司奏《皇風之曲》。樂止,進湯。鼓吹饗節前導至殿外,鼓吹止,殿上樂作。群臣湯饌成,樂止。武舞入,教坊司請奏《平定天下之舞》。第三爵,教坊司請奏《眷皇明之曲》,進酒如前儀。樂止,教坊司請奏《撫安四夷之舞》。第四爵,奏《天道傳之曲》,進酒進湯如前儀。樂止,奏《車書會同之舞》。第五爵,奏《振皇綱之曲》,進酒如前儀。樂止,奏百戲承應。第六爵,奏《金陵之曲》,進酒進湯如前儀。樂止,奏八蠻獻寶承應。第七爵,奏《長楊之曲》,進酒如前儀。樂止,奏採蓮隊子承應。第八爵,奏《芳醴之曲》,進酒進湯如前儀。樂止,奏魚躍於淵承應。第九爵,奏《駕六龍之曲》,進酒如前儀。樂止,收爵。進湯,進大膳,樂作。供群臣飯食訖,樂止,百花隊舞承應。宴成徹案。群臣出席,北向拜,樂作。拜畢,樂止。駕興,大樂作、鳴鞭,百官以次出。

此朝賀宴饗之樂歌節奏也。

其樂器之制,郊丘廟社,洪武元年定。樂工六十二人,編鐘、編磬各十六,琴十,瑟四,搏拊四,柷敔各一,壎四,篪四,簫八,笙八,笛四,應鼓一;歌工十二;協律郎一從執麾以引之。七年復增籥四,鳳笙四,壎用六,搏拊用二,共七十二人。舞則武舞生六十二人,引舞二人,各執干戚;文舞生六十二人,引舞二人,各執羽籥;舞師二人執節以引之。共一百三十人。惟文廟樂生六十人,編鐘、編磬各十六,琴十,瑟四,搏拊四,柷敔各一,壎四,篪四,簫八,笙八,笛四,大鼓一;歌工十。六年鑄太和鐘。其制,仿宋景鐘。以九九為數,高八尺一寸。拱以九龍,柱以龍虡,建樓於圜丘齋宮之東北,懸之。郊祀,駕動則鐘聲作。升壇,鐘止,眾音作。禮畢,升輦,鐘聲作。俟導駕樂作,乃止。十七年改鑄,減其尺十之四焉。

朝賀。洪武三年定丹陛大樂:簫四,笙四,箜篌四,方響四,頭管四,龍笛四,琵琶四,闉六,杖鼓二十四,大鼓二,板二。二十六年又定殿中韶樂:簫十二,笙十二,排簫四,橫笛十二,壎四,篪四,琴十,瑟四,編鐘二,編磬二,應鼓二,柷一,敔一,捕拊二,丹陛大樂:戲竹二,簫十二,笙十二,笛十二,頭管十二,闉八,琵琶八,二十弦八,方響二,鼓二,拍板八,杖鼓十二。命婦朝賀中宮,設女樂:戲竹二,簫十四,笙十四,笛十四,頭管十四,闉十,琵琶八,二十弦八,方響六,鼓五,拍板八,杖鼓十二。正旦、冬至、千秋凡三節。其後太皇太后、皇太后並用之。朔望朝參:戲竹二,簫四,笙四,笛四,頭管四,闉二,琵琶二,二十弦二,方響一,鼓一,拍板二,杖鼓六。

大宴。洪武元年定殿內侑食樂:簫六,笙六,歌工四。丹陛大樂:戲竹二,簫四,笙四,琵琶六,闉六,箜篌四,方響四,頭管四,龍笛四,杖鼓二十四,大鼓二,板二。文武二舞樂器:笙二,橫管二,闉二,杖鼓二,大鼓一,板一。四夷舞樂:腰鼓二,琵琶二,胡琴二,箜篌二,頭管二,羌笛二,闉二,水盞一,板一。二十六年又定殿內侑食樂:祝一,敔一,搏拊一,琴四,瑟二,簫四,笙四,笛四,壎二,篪二,排簫一,鐘一,磬一,應鼓一。丹陛大樂:戲竹二,簫四,笙四,頭管二,琵琶二,闉二,二十弦二,方響二,杖鼓八,鼓一,板一。迎膳樂:戲竹二,笙二,笛四,頭管二,闉二,杖鼓十,鼓一,板一。進膳樂:笙二,笛二,杖鼓八,鼓一,板一。太平清樂:笙四,笛四,頭管二,闉四,方響一,杖鼓八,小鼓一,板一。

樂工舞士服色之制。郊廟,洪武元年定;朝賀,洪武三年定。文武兩舞:武舞士三十二人,左干右戚,四行,行八人,舞作發揚蹈厲坐作擊刺之狀,舞師二人執旌以引之;文舞士三十二人,左籥右翟,四行,行八人,舞作進退舒徐揖讓升降之狀,舞師二人執翿以引之。四夷之舞:舞士十六人,四行,行四人,舞作拜跪朝謁喜躍俯伏之狀,舞師二人執幢以引之。

此祭祀朝賀之樂舞器服也。

當太祖時,前後稍有增損。樂章之鄙者,命儒臣易其詞。二郊之作,太祖所親製。後改合祀,其詞復更。太社稷奉仁祖配,亦更製七奏。嘗諭禮臣曰:「古樂之詩,章和而正。後世之詩,章淫以誇。故一切諛詞艷曲,皆棄不取。」嘗命儒臣撰回鑾樂歌,所奏《神降祥》、《神貺》、《酣酒》、《色荒》、《禽荒》諸曲,凡三十九章,命曰《御鑾歌》,皆寓諷諫之意。然當時作者,惟務明達易曉,非能如漢、晉間詩歌,鏗鏘雅健,可錄而誦也。殿中韶樂,其詞出於教坊俳優,多乖雅道。十二月樂歌,按月律以奏,及進膳、迎膳等曲,皆用樂府、小令、雜劇為娛戲。流俗喧嘵,淫哇不逞。太祖所欲屏者,顧反設之殿陛間不為怪也。

永樂十八年,北京郊廟成。其合祀合享禮樂,一如舊制。更定宴饗樂舞:初奏《上萬壽之曲》,《平定天下之舞》;二奏《仰天恩之曲》,《撫四夷之舞》;三奏《感地德之曲》,《車書會同之舞》;四奏《民樂生之曲》,《表正萬邦之舞》;五奏《感皇恩之曲》,《天命有德之舞》;六奏《慶豐年之曲》;七奏《集禎應之曲》;八奏《永皇圖之曲》;九奏《樂太平之曲》。奏曲膚淺,舞曲益下俚。景泰元年,助教劉翔上書指其失。請敕儒臣推演道德教化之意,君臣相與之樂,作為詩章,協以律呂,如古《靈臺》、《辟雍》、《清廟》、《湛露》之音,以振勵風教,備一代盛典。時以襲用既久,卒莫能改。其後教坊司樂工所奏中和韶樂,且多不諧者。成化中,禮官嘗請三倍其額,博教而約取之。

弘治之初,孝宗親耕耤田,教坊司以雜劇承應,間出狎語。都御史馬文升厲色斥去。給事中胡瑞嘗言:「御殿受朝,典禮至大,而殿中中和韶樂乃屬之教坊司,嶽鎮海瀆,三年一祭,乃委之神樂觀樂舞生,褻神明,傷大體。望敕廷臣議,岳瀆等祭,當以縉紳從事。中和韶樂,擇民間子弟肆習,設官掌之。年久則量授職事。」帝以奏樂遣祭,皆國朝舊典,不能從也。馬文升為尚書,因災異陳言,其一訪名儒以正雅樂,事下禮官。禮官言:「高皇帝命儒臣考定八音,修造樂器,參定樂章。其登歌之詞,多自裁定。但歷今百三十餘年,不復校正,音律舛訛,釐正宜急。且太常官恐未足當製器協律之任。乞詔下諸司,博求中外臣工及山林有精曉音律者,禮送京師。會禮官熟議至當,然後造器正音,庶幾可以復祖制,致太和。」帝可其奏。末年詔南京及各王府,選精通樂藝者詣京師,復以禮官言而罷。

正德三年,武宗諭內鐘鼓司康能等曰:「慶成大宴,華夷臣工所觀瞻,宜舉大樂。邇者音樂廢缺,無以重朝廷。」禮部乃請選三院樂工年壯者,嚴督肄之,仍移各省司取藝精者赴京供應。顧所隸益猥雜,筋斗百戲之類日盛於禁廷。既而河間等府奉詔送樂戶,居之新宅。樂工既得幸,時時言居外者不宜獨逸,乃復移各省司所送技精者於教坊。於是乘傳續食者又數百人,俳優之勢大張。臧賢以伶人進,與諸佞倖角寵竊權矣。

嘉靖元年,御史汪珊請屏絕玩好,令教坊司毋得以新聲巧技進。世宗嘉納之。是時更定諸典禮,因亦有志於樂。建觀德殿以祀獻帝,如協律郎肄樂供祀事。後建世廟成,改殿曰崇先。乃親製樂章,命大學士費宏等更定曲名,以別於太廟。其迎神曰《永和之曲》。初獻曰《清和之曲》,亞獻曰《康和之曲》,終獻曰《沖和之曲》,徹饌曰《泰和之曲》,送神曰《寧和之曲》。宏等復議,獻皇生長太平,不尚武功,其三獻皆當用《文德舞》。從之。已而太常復請,乃命禮官會張璁議。璁言:「樂舞以佾數為降殺,不聞以武文為偏全。使八佾之制,用其文而去其武,則兩階之容,得其左而闕其右。是皇上舉天子禮樂,而自降殺之矣。」乃從璁議,仍用二舞。

九年二月,始祈穀於南郊。帝親製樂章,命太常協於音譜。是年,始祀先蠶,下禮官議樂舞。禮官言:「先蠶之祀,周、漢所同。其樂舞儀節,經史不載。唐開元先蠶儀注,大樂令設宮縣於北郊壇壝內,諸女工咸列於后,則祀先蠶用女樂可知。《唐六典》,宮縣之舞八佾,軒縣之舞六佾,則祀先蠶用八佾又可知。然止言舞生冠服,而不及舞女冠服。陳暘《樂書享先蠶圖》下,止有《宮架登歌圖》,而不及舞。夫有樂有舞,雖祀禮之常,然周、漢制度既不可考,宋祀先蠶,代以有司,又不可據。惟開元略為近古,而陳氏《樂書》考據亦明。前享先農,既以佾數不足,降八為六,則今祀先蠶,止用樂歌,不用樂舞,亦合古制。且以見少殺先農之禮。」帝以舞非女子事,罷不用。使議樂女冠服以聞。禮官言:「北郊陰方,其色尚黑。同色相感,事神之道。漢蠶東郊,魏蠶西郊,色皆尚青,非其色矣。樂女冠服宜黑。」乃用樂六奏,去舞。其樂女皆黑冠服,因定享先蠶樂章。

又以祀典方釐定南北郊,復朝日夕月之祭,命詞臣取洪武時舊樂歌,一切更改。禮官因請廣求博訪,有如宋胡瑗、李照者,具以名聞。授之太常,考定雅樂。給事中夏言乃以致仕甘肅行太僕寺丞張鶚應詔。命趣召之。既至,言曰:

大樂之正,乃先定元聲。元聲起自冥罔既覺之時,亥子相乘之際。積絲成毫,積毫成釐,積釐成分。一時三十分,一日十二時。故聲生於日,律起於辰。氣在聲先,聲從氣後。若拘於器以求氣,則氣不能致器,而反受制於器,何以定黃鐘、起曆元?須依蔡元定,多截竹以擬黃鐘之律,長短每差一分。冬至日按律而候,依法而取。如眾管中先飛灰者,即得元氣。驗其時刻,如在子初二刻,即子初一刻移於初二刻矣;如在正二刻,即子正一刻移於正二刻矣。顧命知歷官一人,同臣參候,庶幾元聲可得,而古樂可復。

又言:

古人製為十六編鐘,非徒事觀美,蓋為旋宮而設。其下八鐘,黃鐘、大呂、太簇、夾鐘、姑洗、仲呂、蕤賓、林鐘是已;其上八鐘,夷則、南呂、無射、應鐘、黃鐘、大呂、太簇是已。近世止用黃鐘一均,而不遍具十六鐘,古人立樂之方已失。況太常止以五、凡、工、尺、上、一、四、六、勾、合字眼譜之,去古益遠。且如黃鐘為合似矣,其以大呂為下四,太簇為高四,夾鐘為下一,姑洗為高一,夷則為下工,南呂為高工之類,皆以兩律兼一字,何以旋宮取律,止黃鐘一均而已。

且黃鐘、大呂、太族、夾鐘為上四清聲。蓋黃鐘為君,至尊無比。黃鐘為宮,則十一律皆從而受制,臣民事物莫敢凌犯焉。至於夾鐘為宮,則下生無射為徵,無射上生仲呂為商,仲呂下生黃鐘為羽。然黃鐘正律聲長,非仲呂為商三分去一之次。所以用黃鐘為羽,必用子聲,即上黃六之清聲,正為不敢用黃鐘全聲,而用其半耳。姑洗以下之均,大率若此。此四清聲之所由立也。編鐘十六,其理亦然。

宋胡瑗知此義,故四清聲皆小其圍徑以就之。然黃鐘、太簇二聲雖合,大呂、夾鐘二聲又非,遂使十二律、五聲皆不得正。至於李照、范鎮止用十二律,不用四清聲,其合於三分損益者則和矣。夷則以降,其臣民事物,安能尊卑有辨,而不相凌犯耶?

臣又考《周禮》,圜鐘、函鐘、黃鐘、天地人三宮之說,有薦神之樂,有降神之樂。所為薦神之樂者,乃奏黃鐘,歌大呂,子丑合也,舞《雲門》以祀天神。乃奏太簇,歌應鐘,寅亥合也,舞《咸池》以祀地祇。乃奏姑洗,歌南呂,辰酉合也,舞《大韶》以祭四望。乃奏蕤賓,歌林鐘,午未合也,舞《大夏》以祭山川。乃奏夷則,歌小呂,巳申合也,舞《大武》以享先祖,舞《大濩》以享先妣。所謂降神之樂者,冬至祀天圜丘,則以圜鐘為宮,黃鐘為角,太簇為徵,姑洗為羽,是三者陽律相繼。相繼者,天之道也。夏至祭地方丘,則以函鐘為宮,夾鐘為角,姑洗為徵,南呂為羽,是三者陰呂相生。相生者,地之功也。祭宗廟,以黃鐘為宮,大呂為角,太簇為徵,夾鐘為羽,是三者律呂相合。相合者,人之情也。

且圜鐘,夾鐘也。生於房心之氣,為天地之明堂,祀天從此起宮,在琴中角絃第十徽,卯位也。函鐘,林鐘也。生於坤位之氣,在井東輿鬼之外,主地祇,祭地從此起宮,在琴中徽絃第五徽,未位也。黃鐘,生於虛危之氣,為宗廟,祭人鬼從此起宮,在琴中宮絃第三徽,子位也。至若六變而天神降,八變而地祇格,九變而人鬼享,非有難易之分。蓋陽數起子而終於少陰之申,陰數起午而終於少陽之寅。圜鐘在卯,自卯至申六數,故六變而天神降。函鐘在未,自未至寅八數,故八變而地祇格。黃鐘在子,自子至申九數,故九變而人鬼享。此皆以本元之聲,召本位之神,故感通之理速也。或者謂自漢以來,天地鬼神聞新聲習矣,何必改作。不知自人觀天地,則由漢迄今千七百年;自天地觀,亦頃刻間耳。自今正之,猶可及也。

併進所著樂書二部。其一曰《大成樂舞圖譜》,自琴瑟以下諸樂,逐字作譜。其一曰《古雅心談》,列十二圖以象十二律。圖各有說。又以琴為正聲,樂之宗系。凡郊廟大樂,分註琴絃定徽,各有歸旨。且自謂心所獨契,斲輪之妙,有非口所能言者。

疏下禮部。禮官言:「音律久廢,太常諸官循習工尺譜,不復知有黃鐘等調。臣等近奉詔演習新定郊祀樂章,間問古人遺制,茫無以對。今鶚謂四清聲所以為旋宮,其註絃定徽,蓋已深識近樂之弊。至欲取知曆者,互相參考,尤為探本窮源之論。似非目前司樂者所及。」乃授鶚太常寺丞,令詣太和殿較定樂舞。

鶚遂上言:「《周禮》有郊祀之樂,有宗祀之樂。尊親分殊,聲律自別。臣伏聽世廟樂章,律起林鐘,均殊太廟。臣竊異之。蓋世廟與太廟同禮,而林鐘與黃鐘異樂。函鐘主祀地祇,位寓坤方,星分井鬼,樂奏八變,以報資生之功。故用林鐘起調,林鐘畢調也。黃鐘主祀宗廟,位分子野,星隸虛危,樂奏九成,以報本源之德。故用黃鐘起調,黃鐘畢調也。理義各有歸旨,聲數默相感通。況天地者父母之象,大君者宗子之稱。今以祀母之樂,奏以祀子,恐世廟在天之靈,必不能安且享矣。不知譜是樂者,何所見也。臣觀舊譜樂章,字用黃鐘,聲同太廟。但審聽七聲,中少一律,今更補正。使依奏格,則祖孫一氣相為流通,函黃二宮不失均調。尊親之分兩得,神人之心胥悅矣。」詔下禮官。

李時等覆奏,以為:「鶚所言,與臣等所聞於律呂諸書者,深有所合。蓋黃鐘一調,以黃鐘為宮,太簇為商,姑洗為角,蕤賓為變徵,林鐘為徵,南呂為羽,應鐘為變宮。舊樂章用合,用四,用一,用尺,用工。去蕤賓之勾,而越次用再生黃鐘之六,此舊樂章之失也。若林鐘一調,則以林鐘為宮,南呂為商,應鐘為角,大呂之半聲為變徵,太簇之半聲為徵,姑洗之半聲為羽,蕤賓之半聲為變宮。邇者沈居敬更協樂章,用尺,用合,用四,用一,用工,用六。夫合,黃鐘也;四,太簇之正聲也;一,姑洗之正聲也;六,黃鐘之子聲也。以林鐘為宮,而所用為角徵羽者,皆非其一均之聲,則謬甚矣。況林鐘一調,不宜用於宗廟,而太廟與世廟,不宜異調,鶚見尤真。自今宜用舊協音律,惟加以蕤賓勾聲,去再生黃鐘之六,改用應鐘之凡,以成黃鐘一均,庶於感格之義,深有所補。」

乃命鶚更定廟享樂音,而逮治沈居敬等。鶚尋譜定帝社稷樂歌以進。詔嘉其勤,晉為少卿,掌教雅樂。

夏言又引古者龍見而雩,命樂正習盛樂,舞皇舞。請依古禮,定大雩之制。當三獻禮成之後,九奏樂止之時,皛括《雲漢》詩辭,製為《雲門》一曲,使文武舞士並舞而合歌之。帝可其議。

時七廟既建,樂制未備,禮官因請更定宗廟雅樂,言:「德、懿、熙、仁四祖久祧,舊章弗協。太祖創業,太宗定鼎,列聖守成。當有頌聲,以對越在天,垂之萬蜺。若特享,若祫享,若大祫,詩歌頌美,宜命儒臣撰述,取自上裁。其樂器、樂舞、各依太廟成式,備為規制。」制可。已而尊獻帝為睿宗,祔享太廟。於是九廟春特、三時祫、季冬大祫樂章,皆更定焉。

十八年巡狩興都,帝親製樂章,享上帝於飛龍殿,奉皇考配。其後,七廟火,復同堂之制,四時歲祫,樂章器物仍如舊制。初增七廟樂官及樂舞生,自四郊九廟暨太歲神祇諸壇,樂舞人數至二千一百名。後稍裁革,存其半。

張鶚遷太常卿,復申前說,建白三事:一請設特鐘、特磬以為樂節;一請復宮縣以備古制;一請候元氣以定鐘律。事下禮官,言:「特鐘、特磬宜造樂懸,在廟廷中,周旋未便,不得更製。惟黃鐘為聲氣之元,候氣之法,實求中氣以定中聲,最為作樂本原。其說,若重室墐戶,截管實灰,覆緹,按曆氣至灰飛,證以累黍,具有成法可依。其法,築室於圜丘外垣隙地,選知歷候者往相其役,待稍有次第,然後委官考驗。」從之。仍詔取山西長子縣羊頭山黍,大小中三等各五斗,以備候氣定律。

明自太祖、世宗,樂章屢易,然鐘律為制作之要,未能有所講明。呂懷、劉濂、韓邦奇、黃佐、王邦直之徒著書甚備,職不與典樂,託之空言而已。張鶚雖因知樂得官,候氣終屬渺茫,不能準以定律。弘治中,莆人李教授文利,著《律呂元聲》,獨宗《呂覽》黃鐘三寸九分之說。世宗初年,御史范永鑾上其書,其說與古背,不可用。嘉靖十七年六月,遼州同知李文察進所著樂書四種,禮官謂於樂理樂書多前人所未發者。乃授文察為太常典簿,以獎勸之。而其所云:「按人聲以考定五音」者,不能行也。神宗時,鄭世子載堉著《律呂精義》、《律學新說》、《樂舞全譜》共若干卷,具表進獻。崇禎六年,禮部尚書黃汝良進《昭代樂律志》。宣付史館,以備稽考,未及施行。


○樂章一

洪武元年圜丘樂章。

迎神,《中和之曲》:昊天蒼兮穹窿,廣覆燾兮龐洪。建圜丘兮國之陽,合眾神兮來臨之同。念螻蟻兮微衷,莫自期兮感通。思神來兮金玉其容,馭龍鸞兮乘雲駕風。顧南郊兮昭格,望至尊兮崇崇。

奠玉帛,《肅和之曲》:聖靈皇皇,敬瞻威光。玉帛以登,承筐是將。穆穆崇嚴,神妙難量。謹茲禮祭,功徵是皇。

進俎,《凝和之曲》:祀儀祗陳,物不於大。敢用純犢,告於覆載。惟茲菲薦,恐未周完。神其容之,以享以觀。

初獻,《壽和之曲》:眇眇微躬,何敢請於九重,以煩帝聰。帝心矜兮,有感而通。既俯臨於几筵,神繽紛而景從。臣雖愚蒙,鼓舞歡容,乃子孫之親祖宗。酌清酒兮在鐘,仰至德兮玄功。

亞獻,《豫和之曲》:荷天之寵,眷駐紫壇。中情彌喜,臣庶均懽。趨蹌奉承,我心則寬。再獻御前,式燕且安。

終獻,《熙和之曲》:小子於茲,惟父天之思,惟恃天之慈,內外殷勤。何以將之?奠有芳齊,設有明粢。喜極而抃,奉神燕娭。禮雖止於三獻,情悠長兮遠而。

徹饌,《雍和之曲》:烹飪既陳,薦獻斯就。神之在位,既歆既右。群臣駿奔,徹茲俎豆。物倘未充,尚幸神宥。

送神,《安和之曲》:神之去兮難延,想遐袂兮翩翩。萬靈從兮後先,衛神駕兮回旋。稽首兮瞻天,雲之衢兮眇然。

望燎,《時和之曲》:焚燎於壇,燦爛晶熒。幣帛牲黍,冀徹帝京。奉神於陽,昭祀有成。肅然望之,玉宇光明。

洪武八年御製圜丘樂章。

迎神:仰惟兮昊穹,臣率百職兮迓迎。幸來臨兮壇中,上下護衛兮景從。旌幢繚繞兮四維,重悅聖心兮民獲年豐。

奠玉帛:民依時兮用工,感帝德兮大化成功。臣將兮以奠,望納兮微衷。

進俎:庖人兮列鼎,肴羞兮以成。方俎兮再獻,願享兮以歆。

初獻:靈兮皇皇,穆嚴兮金床。臣令樂舞兮景張,酒行初獻兮捧觴。

亞獻:載斟兮再將,百辟陪祀兮具張。感聖情兮無已,拜手稽首兮願享。

終獻:三獻兮樂舞揚,肴羞具納兮氣藹而芳。光朗朗兮上方,況日吉兮時良。

徹饌:粗陳菲薦兮神喜將,感聖心兮何以忘。民福留兮佳氣昂,臣拜手兮謝恩光。

送神:旌幢燁燁兮雲衢長,龍車鳳輦兮駕飛揚。遙瞻冉冉兮去上方,可見烝民兮永康。

望燎:進羅列兮詣燎方,炬焰發兮煌煌。神變化兮物全於上,感至恩兮無量。

洪武二年方丘樂章。

迎神,《中和之曲》:坤德博厚,物資以生。承天時行,光大且寧。穆穆皇祇,功化順成。來御方丘,嚴恭奉迎。

奠玉帛,《肅和之曲》:地有四維,大琮以方;土有正色,制幣以黃。敬存於中,是薦是將。奠之几筵,臨鑒洋洋。

進俎,《凝和之曲》:奉將純牡,其牡童犢。烹飪既嚴,登俎惟肅。升壇昭薦,神光下燭。眷佑邦家,報效惟篤。

初獻,《壽和之曲》:午為盛陽,陰德初萌。天地相遇,品物光榮。吉日令辰,明祀攸行。進以醇醴,展其潔清。

亞獻,《豫和之曲》:至廣無邊,道全持載。山嶽所憑,海瀆咸賴。民次水土,既安且泰。酌酒揭虔,功德惟大。

終獻,《熙和之曲》:庸眇之資,有此疆宇。匪臣攸能,仰承佑助。恩崇父母,臣懽鼓舞。八音宣揚,疊侑明醑。

徹饌,《雍和之曲》:牲牷在俎,籩豆有實。臨之蕣蚃,匪惟飲食。登歌乃徹,薦獻爰畢。執事奉承,一其嚴粟。

送神,《安和之曲》:神化無方,妙用難量。其功顯融,其禮攸長。飆輪云旋,龍探鸞翔。拜送稽首,瞻禮餘光。

望瘞,《時和之曲》:牲醴制幣,餕饌惟馨。瘞之於坎,以達坤靈。奉神於陰,典禮是程。企而望之,厚壤寬平。

洪武八年御製方丘樂章。

迎神:仰皇祇兮駕來,川嶽從迎兮威靈備開,香煙繚繞兮神臨御街。漸升壇兮穆穆,靄瑞氣兮應結樓臺。以微衷兮率職,幸望聖悅兮心諧。但允臣兮固請,願嘉烝民兮永懷。

奠玉帛:臣奉兮以筐,玉帛是進兮歲奠以常。百辟陪祀兮珮聲琅瑯。惟南薰兮解慍,映燎炎兮煌煌。

進俎:庖人兮凈湯,大烹牲兮氣靄而芳。以微衷兮獻上,曰享兮曰康。

初獻:初獻行兮捧觴,聖靈穆穆兮洋洋。為烝民兮永康,鑒豐年兮耿光。

亞獻:雜肴羞兮已張,法前王兮典章。臣固展兮情悃,用斟醴兮載觴。

終獻:爵三獻兮禮將終,臣心眷戀兮無窮。恐肴羞兮未具,將何報兮神功。

徹饌:俎豆徹兮神熙,鸞輿駕兮旋歸。百神翼翼兮雲衣。敬奉行兮弗敢違。

送神:祥風興兮悠悠,雲衢開兮民福留。歲樂烝民兮大有,想洋洋兮舉觴載酒。

望瘞:肴羞玉帛兮瘞坎中,遙瞻隱隱兮龍旗從。祀事成兮盡微衷,感厚德兮民福雍雍。

洪武十二年合祀天地樂章。

迎神,《中和之曲》:荷蒙天地兮君主華夷,欽承踴躍兮備筵而祭,誠惶無已兮寸衷微,仰瞻俯首兮惟願來期。想龍翔鳳舞兮慶雲飛,必昭昭穆穆兮降壇壝。

奠玉帛,《肅和之曲》:天垂風露兮雨澤霑,黃壤氤氳兮氣化全。民勤畝兮束帛鮮,臣當設宴兮奉來前。

進俎以後,咸同八年圜丘詞。

嘉靖九年復定分祀圜丘樂章。

迎神,《中和之曲》:仰惟玄造兮於皇昊穹,時當肇陽兮大禮欽崇。臣惟蒲柳兮螻蟻之衷,伏承春命兮職統群工。深懷愚昧兮恐負洪德,爰遵彞典兮勉竭微衷。遙瞻天闕兮寶輦臨壇,臣當稽首兮祗迓恩隆。百辟陪列兮舞拜於前,萬神翊衛兮以西以東。臣俯伏迎兮敬瞻帝御,願垂歆鑒兮拜德曷窮。

奠玉帛,《肅和之曲》龍輿既降兮奉禮先,爰有束帛兮暨瑤瑄。臣謹上獻兮進帝前,仰祈聽納兮荷蒼乾。

進俎,《凝和之曲》:肴羞珍饌兮薦上玄,庖人列鼎兮致精虔。臣盍祗獻兮馨醴牷,願垂歆享兮民福淵。

初獻,《壽和之曲》:禮嚴初獻兮奉觴,臣將上進兮聖皇。聖皇垂享兮穆穆,臣拜手兮何以忘。

亞獻,《豫和之曲》:禮觴再舉兮薦玉漿,帝顏歆悅兮民福昂。民生有賴兮感上蒼,臣惟鞠拜兮荷恩長。

終獻,《熙和之曲》:三獻兮禮告成,一念微衷兮露悃情。景張樂舞兮響鍠鋐,仰瞻聖容兮俯錫恩泓。

徹饌,《雍和之曲》:祀禮竣兮精意禋,三獻備兮誠已申。敬徹弗遲兮肅恭寅,恐多弗備兮惟賴洪仁。

送神,《清和之曲》:禋事訖終兮百辟維張,帝垂歆鑒兮沐澤汪洋。龍車冉冉兮寶駕旋雲,靈風鼓舞兮瑞露清瀼。洪恩浩蕩兮無以為酬,粗陳菲薦兮已感歆嘗。香氣騰芳兮上徹帝座,仰瞻聖造兮賜福群方。臣同率土兮載懽載感,祗回寶輦兮鳳嘯龍翔。誠惶誠恐兮仰戀彌切,願福生民兮永錫亨昌。

望燎,《時和之曲》:龍駕寶輦兮昇帝鄉,御羞菲帛兮奉燎方。環珮鏗鏘兮羅壇壝,炬焰特舉兮氣輝煌。生民蒙福兮聖澤霑,臣荷眷佑兮拜謝恩光。

嘉靖九年復定方丘樂章。

迎神,《中和之曲》:俯瞻兮鳳輦來,靈風兮拂九垓。川嶽從兮後先,百辟列兮襄陪。臣拜首兮迓迎,願臨享兮幸哉。

奠玉帛,《廣和之曲》:祀禮有嚴兮奉虔,玉帛在笥兮來前。皇靈垂享兮以納,烝民率土兮樂豐年。

進俎,《咸和之曲》:肴羞馨兮氣芳,庖人奉役兮和湯。奉進兮皇祗歆慰,臣稽首兮敬將。

初獻,《壽和之曲》:酒行初獻兮樂舞張,齊醴明潔兮氙香。願垂享兮以歆,生民安兮永康。

亞獻,《安和之曲》:載獻兮奉觴,神顏和懿兮以嘗。功隆厚載兮配天,民感德兮無量。

終獻,《時和之曲》:三進兮玉露清,百職奔繞兮佩環鳴。鳧鐘鷺鼓兮韻錚鍧,願留福兮群生。

徹饌,《貞和之曲》:禮告終兮徹敢違,深惟一念兮誠意微。神垂博容兮聽納,恐未備兮惟慈依。

送神,《寧和之曲》:禮成兮誠已伸,駕還兮法從陳。靈祇列兮以隨,百辟拜兮恭寅。望坤宮兮奉辭,願普福兮烝民。

望燎,曲同《寧和》。

洪武三年朝日樂章。二十一年罷。

迎神,《熙和之曲》:吉日良辰,祀典式陳。純陽之精,惟是大明。濯濯厥靈,昭鑒我心。以候以迎,來格來歆。

奠幣,《保和之曲》:靈旗蒞止,有赫其威。一念潛通,幽明弗違。有幣在篚,物薄而微。神兮安留,尚其享之。

初獻,《安和之曲》:神兮我留,有薦必受。享祀之初,奠茲醴酒。晨光初升,祥徵應候。何以侑觴,樂陳雅奏。

亞獻,《中和之曲》:我祀維何?奉茲犧牲,爰酌醴齊,貳觴載升。洋洋如在,式燕以寧。庶表微衷,交於神明。

終獻,《肅和之曲》:執事有嚴,品物斯祭,稷非馨,式將其意。薦茲酒醴,成我常祀。神其顧歆,永言樂只。

徹饌,《凝和之曲》:春祈秋報,率為我民。我民之生,賴於爾神。維神佑之,康寧是臻。祭祀云畢,神其樂忻。

送神,《壽和之曲》:三獻禮終,九成樂作。神人以和,既燕且樂。雲車風馭,靈光昭灼。瞻望以思,邈彼寥廓。

望燎,《豫和之曲》:俎豆既徹,禮樂已終。神之云旋,倏將焉從。以望以燎,庶幾感通。時和歲豐,維神之功。

嘉靖九年復定朝日樂章。

迎神,《熙和之曲》:仰瞻兮大明,位奠兮王宮。時當仲春兮氣融,爰遵祀禮兮報功。微誠兮祈神昭鑒,願來享兮迓神聰。

奠玉帛,《凝和之曲》:神靈壇兮肅其恭,有帛在篚兮赤琮。奉神兮祈享以納,予躬奠兮忻以顒。

初獻,《壽和之曲》:玉帛方奠兮神歆,酒行初獻兮舞呈。齊芳馨兮犧色騂,神容悅兮鑒予情。

亞獻,《時和之曲》:二齊昇兮氣芬芳,神顏怡和兮喜將。予令樂舞兮具張,願垂普照兮民康。

終獻,《保和之曲》:殷勤三獻兮告成,群職在列兮周盈。神錫休兮福民生,萬世永賴兮神功明。

徹饌,《安和之曲》:一誠盡兮予心懌,五福降兮民獲禧。仰九光兮誠已申,終三獻兮徹敢遲。

送神,《昭和之曲》:祀禮既周兮樂舞揚,神享以納兮還青鄉。予當拜首兮奉送,願恩光兮普萬方。永耀熹明兮攸賴,烝民咸仰兮恩光。

望燎之曲:睹六龍兮御駕,神變化兮鳳翥鸞翔。束帛肴羞兮詣燎方,佑我皇明兮基緒隆長。

洪武三年夕月樂章。周天星辰附。二十一年罷。

迎神,《凝和之曲》:吉日良辰,祀典式陳。太陰夜明,以及星辰。濯濯厥靈,昭鑒我心以候以迎,來格來歆。四年,星辰別祀,改「以及星辰」句為「惟德孔神」。

奠帛以下,咸同朝日。

嘉靖九年復定夕月樂章。

迎神,《凝和之曲》:陰曰配合兮承陽宗,式循古典兮齋以恭。睹太陰來格兮星辰羅從,予拜首兮迓神容。

初獻,《壽和之曲》:神其來止,有嚴其誠。玉帛在篚,清酤方盈。奉而奠之,願鑒微情。夫祀兮云何?祈佑兮群氓。

亞獻,《豫和之曲》:二觴載斟,樂舞雍雍。神歆且樂,百職惟供。願順軌兮五行,祈民福兮惟神必從。

終獻,《康和之曲》:一誠以申,三舉金觥。鐘鼓鍧鍧,環珮琤琤。鑒予之情,願永保我民生。

徹饌,《安和之曲》:禮樂肅具,精意用申。位坎居歆,納茲藻蘋。徹之弗遲,儀典肅陳。神其鑒之,佑我生民。

送神,《保和之曲》:禮備告終兮神喜旋,穹碧澄輝兮素華鮮。星辰從兮返神鄉,露氣清兮霓裳蹁躚。

望瘞之曲:肴羞兮束帛,薦之於瘞兮罔敢愆。予拜首兮奉送,願永貺兮民樂豐年。

嘉靖十年,定祈穀樂章。

迎神,《中和之曲》:臣惟穹昊兮民物之初,為民請命兮祀禮昭諸。備筵率職兮祈洪庥,臣衷微眇兮悃懇誠攄。遙瞻駕降兮霽色輝,歡迎鼓舞兮迓龍輿。臣愧菲才兮后斯民,願福斯民兮聖恩渠。

奠玉帛,《肅和之曲》:烝民勤職兮農事顓,蠶工亦慎兮固桑阡。玉帛祗奉兮暨豆籩,仰祈大化兮錫以豐年。

進俎,《咸和之曲》:鼎烹兮氣馨,香羞兮旨醽。帝垂享兮以歆,烝民蒙福兮以寧。

初獻,《壽和之曲》:禮嚴兮初獻行,百職趨蹌兮佩琤鳴。臣謹進兮玉觥,帝心歆鑒兮歲豐亨。

亞獻,《景和之曲》:二觴舉兮致虔,清醴載斟兮奉前。仰音容兮忻穆,臣感聖恩兮實拳拳。

終獻,《永和之曲》:三獻兮一誠微,禋禮告成兮帝鑒是依。烝民沐德兮歲豐禨,臣拜首兮竭誠祈。

徹饌,《凝和之曲》:三獻周兮肅乃儀,俎豆敬徹兮弗敢遲。願留福兮丕而,曰雨曰暘兮若時。

送神,《清和之曲》:祀禮告備兮帝鑒彰,臣情上達兮感昊蒼。雲程肅駕兮返帝鄉,臣荷恩眷兮何以忘。祥風瑞靄兮彌壇壝,烝民率土兮悉獲豐康。

望燎,《太和之曲》:遙睹兮天衢長,邈彼寥廓兮去上方。束帛薦火兮升聞,悃愊通兮沛澤長。樂終九奏兮神人以和,臣同率土兮咸荷恩光。

嘉靖十七年,定大饗樂章。

迎神,《中和之曲》:於皇穆清兮弘覆惟仁,既成萬寶兮惠此烝民。祗受厥明兮欲報無因,爰稽古昔兮式展明禋。肅肅廣庭兮遙遙紫旻,笙鏞始奏兮祥風導雲。臣拜稽首兮中心孔勤,爰瞻寶輦兮森羅萬神。庶幾昭格兮眷命其申,徘徊顧歆兮鑒我恭寅。

奠玉帛,《肅和之曲》:捧珪幣兮瑤堂,穆將愉兮聖皇。秉予心兮純一,荷帝德兮溥將。

進俎,《凝和之曲》:歲功阜兮庶類成,黍稷飶兮濡鼎馨。敬薦之兮慚菲輕,大禮不煩兮惟一誠。

初獻,《壽和之曲》:金風動兮玉宇澄,初獻觴兮交聖靈。瞻玄造兮懷鴻禎,曷以酬之心怦怦。

亞獻,《豫和之曲》:帝眷我兮居歆,紛繁會兮五音。再捧觴兮莫殫臣心,惟帝欣懌兮生民是任。

終獻,《熙和之曲》:綏萬邦兮屢豐年,眇眇予躬兮實荷昊天。酒三獻兮心益虔,帝命參輿兮勿遽旋。

徹饌,《雍和之曲》:祀禮既洽兮神人肅雍,享帝享親兮勉歆臣衷。惟洪恩兮罔極,儼連蜷兮聖容。

送神,《清和之曲》:《九韶》既成兮金玉鏗鏘,百辟森立兮戚羽期藏。皇天在上兮昭考在旁,嚴父配天兮祗修厥常。殷薦既終兮神去無方,玄雲上升兮鸞鵠參翔。靈光回照兮郁乎芬芳,載慕載瞻兮願錫亨昌。子孫庶民兮惟帝是將,於昭明德兮永懷不忘。

望燎,《時和之曲》:龍輿杳杳兮歸上方,金風應律兮燎斯揚,達精誠兮合靈光。帝廷納兮玉帛將,顧下土兮春不忘,願錫吾民兮長阜康。

嘉靖十八年,興都大饗樂章。

迎神,《中和之曲》:仰高高之在上兮皇穹,冒九圍之遍覆兮罔止西東。王者出王游衍兮必奉天顧,愚臣之此行兮亶荷帡幪。

初獻,《壽和之曲》:於昭帝庥兮臣感恩淵淵,巡省舊籓之地兮實止承天。下情思報兮此心拳拳,瓊卮蒼幣兮捧扣壇前。

亞獻,《敷和之曲》:樂奏兮三成,觴舉兮再呈。帝鑒幾微兮曰爾誠,小臣頓首兮敢不嚴於此精。

終獻,《承和之曲》:臣來茲土,本之思親。思親伊何?昌厥嗣人。嗣人克昌,菲戴帝之臨汝夫何因。

徹饌,《永和之曲》:肅其具兮祀禮行,備彼儀兮樂舞張。退省進止兮臣疏且狂,沐含仁兮何以量。

送神,《感和之曲》:王之狩兮典有禋望,於維柴祀兮首重上蒼。臣情罔殫兮夙夜惶惶,祗伸愚悃兮允賴恩光。遙瞻兮六龍騰翔,帝垂祉兮萬世永昌。

嘉靖十一年,定雩祀樂章,十七年罷。

迎神,《中和之曲》:於穆上帝,爰處瑤宮。咨爾黎庶,覆憫曷窮。旗幢戾止,委蛇雲龍。霖澤斯溥,萬寶有終。

奠幣,《肅和之曲》:神之格思,奠茲文纁。盛樂斯舉,香氣氤氳。精禋孔,徹於紫冥。懇祈膏澤,渥我嘉生。

進俎,《咸和之曲》:百川委潤,名山出雲。愆暘孔熾,膏澤斯屯。祈年於天,載牲於俎。神之格思,報以甘雨。

初獻,《壽和之曲》:有嚴崇祀,日吉辰良。酌彼罍洗,椒馨飶香。元功溥濟,時雨時暘。惟神是聽,綏以多穰。

亞獻,《景和之曲》:皇皇禋祀,孔惠孔明。瞻仰來歆,拜首欽承。有醴維醽,有酒維清,去韶侑獻,肅雍和鳴。聖靈有赫,鑒享精誠。

終獻,《永和之曲》:靈承無斁,駿奔有容。嘉玉以陳,酌鬯以供。禮三再稱,誠一以從。備物致志,申薦彌恭。神昭景貺,佑我耕農。

徹饌,《凝和之曲》:有赫旱,民勞瘁斯。於牲於醴,載舞載詩。禮成三獻,敬徹不遲。神之聽之,雨我公私。

送神,《清和之曲》:爰迪寅清,昭事昊穹。仰祈甘雨,惠我三農。既歆既格,言歸太空。式霑下土,萬方其同。

望燎,《太和之曲》:赤龍旋馭,禮洽樂成。燔燎既舉,昭格精禋。維帝降康,雨施雲行。登我黍稌,溥受厥明。

祭畢,樂舞童群歌《雲門之曲》:景龍精兮時見,測鶉緯兮宵懸。肆廣樂兮鏗鍧,列皇舞兮蹁躚。祈方社兮不莫,薦圭璧兮孔虔。需密雲兮六漠,霈甘澍兮九玄。慰我農兮既渥,錫明昭兮有年。

洪武元年,太社稷異壇同壝樂章。

迎神,《廣和之曲》:五土之靈,百穀之英。國依土而寧,民以食而生。基圖肇建,祀禮修明。神其來臨,肅恭而迎。

奠幣,《肅和之曲》:有國有人,社稷為重。昭事云初,玉帛虔奉。維物匪奇,敬實將之。以斯為禮,冀達明祗。

進俎,《凝和之曲》:崇壇北向,明禋方闡。有潔犧牲,禮因物顯。大房載設,中情以展。景運既承,神貺斯衍。

初獻,《壽和之曲》:太社云,高為山林,深為川澤。崇丘廣衍,亦有原隰。惟神所司,百靈效職。清醴初陳,顒然昭格。句龍配云,平治水土,萬世神功。民安物遂,造化攸同。嘉惠無窮,報祀宜豐。配食尊嚴,國家所崇。太稷云,黍稷稻粱,來牟降祥,為民之天。豐年穰穰,其功甚大,其恩正長。乃登芳齊,以享以將。後稷配云,皇皇后稷,克配於天。誕降嘉種,樹藝大田。生民粒食,功垂萬年。建壇於京,歆茲吉蠲。

亞獻,《豫和之曲》:太社云,廣厚無偏,其體弘兮。德侔坤順,萬物生兮。錫民地利,神化行兮。恭祀告虔,國之禎兮。句龍配云,周覽四方,偉烈昭彰。九州既平,五行有常。壇位以妥,牲醴之將。是崇是嚴,煥然典章。太稷云,億兆林林,所資者穀。雨暘應時,家給人足。倉庾坻京,神介多福。祗薦其儀,昭事維肅。後稷配云,躬勤稼穡,有相之道。不稂不莠,實堅實好。農事開國,王基永保。有年自今,常奉蘋藻。

終獻,《豫和之曲》,詞同亞獻。

徹豆,《雍和之曲》:禮展其勤,樂奏其節。庶品苾芬,神明是達。有嚴執事,俎豆乃徹。穆穆雍雍,均其欣悅。

送神,《安和之曲》:維壇潔清,維主堅貞。神之所歸,依茲以寧。土宇靖安,年穀順成。祀事昭明,永致昇平。

望瘞,《時和之曲》:晨光將發,既侑既歆。瘞茲牲幣,達於幽陰。神人和悅,實獲我心。永久禋祀,其始於今。

洪武十一年,合祭太社稷樂章。

迎神,《廣和之曲》:予惟土穀兮造化工,為民立命兮當報崇。民歌且舞兮朝雍雍,備筵率職兮候迓迎。想聖來兮祥風生,欽當稽首兮告年豐。

初獻,《壽和之曲》:氤氳氣合兮物遂蒙,民之立命兮荷陰功。予將玉帛兮獻微衷,初斟醴薦兮民福洪。

亞獻,《豫和之曲》:予令樂舞兮再捧觴,願神昭格兮軍民康。思必穆穆兮靈洋洋,感恩厚兮拜祥光。

終獻,《熙和之曲》:干羽飛旋兮酒三行,香煙繚繞兮雲旌幢。予今稽首兮忻且惶,神顏悅兮霞彩彰。

徹饌,《雍和之曲》:粗陳微禮兮神喜將,瑯然絲竹兮樂舞揚。願祥普降兮遐邇方,烝民率土兮盡安康。

送神,《安和之曲》:氤氳氤氳兮祥光張,龍車鳳輦兮駕飛揚。遙瞻稽首兮去何方,民福留兮時雨暘。

望瘞,《時和之曲》:捧肴羞兮詣瘞方,鳴鑾率舞兮聲鏗鏘。思神納兮民福昂,予今稽首兮謝恩光。

嘉靖十年,初立帝社稷樂章。

迎神,《時和之曲》:東風兮地脈以融,首務兮稼穡之工。秋祭云:「金風兮萬寶以充,忻成兮稼穡之工。祀神於此兮苑中,願來格兮慰予衷。

初獻,《壽和之曲》:神兮臨止,禮薦清醇。菲幣在笥,初獻式遵。神其鑒茲,享斯藻蘋。我祀伊何?祈報是因。神兮錫祉,則阜吾民。

亞獻。《雍和之曲》:二觴載舉,中此殷勤。神悅兮以納,祥靄兮氤氳。

終獻,《寧和之曲》:禮終兮酒三行,喜茂實兮黍稷粱。農事待兮豐康,予稽首兮以望。

徹饌,《保和之曲》:祀事告終,三獻既周。徹之罔遲,惠注田疇。迓以休貺,庇茲有秋。

送神,《廣和之曲》:耕耨伊首,秋祭云:「耕耨告就。」力事豆籩。粢盛賴之,於此大田。予將以祀,神其少延。願留嘉祉,副我潔虔。肅駕兮雲旋,普予兮有年。

望瘞,曲同。

洪武二年,分祀天神地祇樂章。

迎天神,奏《中和之曲》:吉目良辰,祀典式陳。太歲尊神,雷雨風雲。濯濯厥靈,昭鑒我心。以候以迎,來格來歆。

奠帛以後,咸同朝日。

迎地祇,奏《中和之曲》:吉日良辰,祀典式陳。惟地之祇,百靈繽紛。嶽鎮海瀆,山川城隍,內而中國,外及四方。濯濯厥靈,昭鑒我心。以候以迎,來格來歆。

奠帛以後,咸同朝日。

洪武六年,合祀天神地祗樂章。

迎神,《保和之曲》:吉日良辰,祀典式陳。太歲尊神,雷雨風雲,嶽鎮海瀆,山川城隍。內而中國,外及四方。濯濯厥靈,昭鑒我心。以候以迎。來格來歆。

奠帛以後,咸同朝日。

嘉靖九年,復分祀天神地祇樂章。

迎天神,《保和之曲》:吉日良辰,祀典式陳。景靈甘雨,風雷之神。赫赫其靈,功著生民。參贊玄化,宣布蒼仁。爰茲報祀,鑒斯藻蘋。

奠帛以後,俱如舊。

迎地祇,《保和之曲》:吉日良辰,祀典式陳。靈嶽方鎮,海瀆之神,京畿四方,山澤群真。毓靈分隅,福我生民。薦斯享報,鑒我恭寅。

奠帛以後,亦如舊。

洪武四年,祀周天星辰樂章。

迎神,《凝和之曲》:星辰垂象,布列玄穹,擇茲吉日,祀禮是崇。濯濯厥靈,昭鑒我心。謹候以迎,庶幾來歆。

奠帛,《保和之曲》,詞同朝日。

初獻,《保和之曲》:神兮既留,品物斯薦。奉禮之初,醴酒斯奠。仰惟靈耀,以享以歆。何以侑觴?樂奏八音。

亞獻,《中和之曲》:神既初享,亞獻再升。以酌醴齊,仰薦於神。洋洋在上,式燕以寧。庶表微衷,交於神明。

終獻,《肅和之曲》:神既再享,終獻斯備。不腆菲儀,式將其意。薦茲酒醴,成我常祀。神其顧歆,永言樂只。

徹豆,《豫和之曲》:祀事將畢,神既歆只。徹茲俎豆,以成其禮。惟神樂欣,無間始終。樂音再作,庶在微悰。

送神,《雍和之曲》,詞同朝日。

望燎,《雍和之曲》:神既享祀,靈馭今旋。燎煙既升,神帛斯焚。巍巍霄漢,倏焉以適。拳拳餘衷,瞻望弗及。

嘉靖八年,祀太歲月將樂章。

迎神:吉日良辰,祀典式陳。輔國佑民,太歲尊神。四時月將,功曹司辰。濯濯厥靈,昭鑒我心。以候以迎,來格來歆。

奠帛以後,俱同神祇。

洪武元年宗廟樂章。

迎神,《太和之曲》:慶源發祥,世德惟崇。致我眇躬,開基建功。京都之中,親廟在東。惟我子孫,永懷祖風。氣體則同,呼吸相通。來格來崇,皇靈顯融。

奉冊寶,《熙和之曲》:時享不用維水有源,維木有根。先世積善,福垂後昆。冊寶鏤玉,德顯名尊。祗奉禮文,仰答洪恩。

進俎,《凝和之曲》:時享不用明明祖考,妥神清廟。薦以牲牷,匪云盡孝。願通神明,願成治效。此帝王之道,亦祖考之教。

初獻,《壽和之曲》:德祖廟,初獻云:思皇高祖,穆然深玄。其遠歷年,其神在天。尊臨太室,餘慶綿綿。歆於几筵,有永其傳。懿祖廟初獻云:思皇曾祖,清勤純古。田里韜光,天篤其祜。祐我曾孫,弘開土宇。追遠竭虔,勉遵前矩。熙祖廟初獻云:維我皇祖,淑後貽謀。盛德靈長,與泗同流。發於孫枝,明禋載修。嘉潤如海,恩何以酬。仁祖廟初獻云:惟我皇考,既淳且仁。弗耀其身,克開嗣人。子有天下,尊歸於親。景運維新,則有其因。

亞獻,《豫和之曲》:對越至親,儼然如生。其氣昭明,感格在庭。如見其形,如聞其聲。愛而敬之,發乎中情。

終獻,《熙和之曲》:承先人之德,化家為國。毋曰予小子,基命成績。欲報其德,昊天罔極。殷勤三獻,我心悅懌。

徹豆,《雍和之曲》:樂奏具肅,神其燕嬉。告成於祖,亦右皇妣。敬徹不遲,以終祀禮。祥光煥揚,錫以嘉祉。

送神,《安和之曲》:顯兮幽兮,神運無迹。鸞馭逍遙,安其所適。其靈在天,其主在室。子子孫孫,孝思無斁。

二十一年,更定其初獻合奏,餘並同。

思皇先祖,耀靈於天。源衍慶流,由高逮玄。玄孫受命,追遠其先。明禋世崇,億萬斯年。

永樂以後,改迎神章「致我眇躬」句為「助我祖宗」。又改終獻章首四句為「惟前人之功,肇膺天曆。延及於小子,爰受方國」。餘並同。

嘉靖十五年,孟春九廟特享樂章。

太祖廟。迎神,《太和之曲》:於皇於皇兮仰我聖祖,乃武乃文,攘夷正華,為天下大君。比隆於古,越彼放勛。肇造王業,佑啟予子孫。功德超邁,大室攸尊。首稱春祀,誠敬用申。維神格思,萬世如存。

初獻。《壽和之曲》:薦帛於篚,潔牲於俎,嘉我黍稷,酌我清酤。愚孫毖祀,奠獻初舉。翼翼精誠,對越我皇祖。居然顧歆,永錫純祜。

亞獻,《豫和之曲》:籥舞既薦,八音洋洋,工歌喤喤。醇醴載羞,齊明其將之。永佑於子孫,歲事其承之。俾嗣續克承,百世其保之。

終獻,《寧和之曲》:三爵既崇,禮秩有終。盈溢孚顒,顯相肅雍。惟皇祖格哉,以繹以融,申錫無窮。暨於臣民,萬福攸同。

徹饌,《豫和之曲》:禮畢樂成,神悅人宜。籩豆靜嘉,敬徹不遲。穆穆有容,秩秩有儀。益祗以嚴,矧敢斁於斯。

還宮,《安和之曲》:於皇我祖,陟降在天。清廟翼翼,禋祀首虔。明神既留,寢祏靜淵。介福綏祿,錫胤綿綿。以惠我家邦,於萬斯年。

成祖廟。迎神,《太和之曲》:於惟文皇,重光是宣。克戡內難,轉坤旋乾。外讋百蠻,威行八埏。貽典則於子孫,不忘不愆。聖德神功,格於皇天。作廟奕奕,百世不遷。祀事孔明,億萬斯年。

初獻、亞獻、終獻、徹饌、還宮,俱與太祖廟同。

仁宗廟。迎神,《太和之曲》:明明我祖,盛德天成。至治訏謨,遹駿有聲。專奠致享,惟古經是程。春祀有嚴,以迓聖靈。惟陟降在庭,以賚我思成。

初獻,《壽和之曲》:幣牲在陳,金石在懸。清酒方獻,百執事有虔。明神洋洋,降歆自天。俾我孝孫,德音孔宣。

亞獻,《豫和之曲》:中誠方殷,明神如存。醴齊孔醇,再舉罍尊。福祿穰穰,攸介攸臻。追遠報酬,罔極之恩。

終獻,《寧和之曲》:樂比聲歌,佾舞婆娑。稱彼玉爵,酒旨且多。獻享維終,神聽以和。

徹饌,《雍和之曲》:牷牲在俎,稷黍在簠。孝享多儀,格我皇祖。稱歌進徹,髦士ASAS。孝孫受福,以敷錫於下土。

還宮,《安和之曲》:犆享孔明,物備禮成。於昭在天,以莫不聽。神明即安,維華寢是憑。肇祀迄今,百世祗承。

宣廟、英廟、憲廟俱與仁廟同。

孝廟。迎神,《太和之曲》:列祖垂統,景運重熙。於惟孝皇,敬德允持。用光於大烈,化被烝黎。專廟以享,經禮攸宜。俎豆式陳,庶幾來思。

初獻,《壽和之曲》:粢盛孔蠲,腯肥牲牷。考鼓雰雰,萬舞躚躚。清醑初酌,對越在天。明神居歆,式昭厥虔。

亞獻,《豫和之曲》:祀事孔勤,精意未分。樂感鳳儀,禮虔駿奔。醖齊挹清,載奠瑤尊。神其格思,福祿來臻。

終獻,《寧和之曲》:樂舞既成,獻享維終。明明對越,彌篤其恭。篤恭維何?明德是崇。神之聽之,萬福來同。

徹饌,《雍和之曲》:牲牢醴陳,我享我將,黍稷蘋藻,潔白馨香。徹以告成,降禧穰穰。神錫無疆,祐我萬方。

還宮,《安和之曲》:禮享既洽,神御聿興。廟寢煌煌,以憑以寧。維神匪遐,上下在庭。於寢孔安,永底我烝民之生。

武廟。迎神,《太和之曲》:列祖垂統,景運重熙。於惟武皇,昭德敕威。用剪除奸AT,大業弗隳。專廟以享,經禮攸宜。俎豆式陳,庶幾來思。

初獻、亞獻、終獻及徹饌、還宮,俱與孝廟同。

睿廟。迎神,《太和之曲》:於穆神皇,秉德凝道。仁厚積累,配於穹昊。流慶顯休,萃於眇躬。施於無窮,以似以續,以光紹我皇宗。惟茲氣始,俎豆是供。循厥典禮,式敬式崇。神其至止,以鑒愚衷。

初獻,《壽和之曲》:制帛牲牢,庶羞芬尞。玉戚朱干,協於韶簫。清醑在筵,中情纏綿。神之格思,儀形僾然。

亞獻,《豫和之曲》:瑤爵再陳,侑以工歌。籥舞蹌蹌,八音諧和。孝思肫肫,感格聖靈。致愨則存,如聞其聲。

終獻,《寧和之曲》:儀式弗踰,奠爵維三。樂舞雍容,以雅以南。仰仁源德澤,嶽崇海淵。願啟我子孫,緝熙光明,維兩儀是參。

徹饌,《雍和之曲》:嘉饌甘只,亦既歆只。登歌迅徹,敬終惟始。維神孔昭,賚永成於孝矣。

還宮,《安和之曲》:幽顯莫測,神之無方。祀事既成,神返諸帝鄉。申發休祥,俾胤嗣蕃昌。宜君兮宜王,歷世無疆。

○九廟時祫樂章

孟夏。迎神,《太和之曲》:序屆夏首兮風氣薰,禮嚴時祫兮拏擊鐘卉鼓。迎群主來合享交欣,於皇列聖正南面,以申崇報皇勳。

初獻,《壽和之曲》:瞻曙色方昕,仰列聖在上,奠金觥而捧幣紋。小孫執盈兮敢不懼慇。

亞獻,《豫和之曲》:思皇祖,仰聖神。來列主,會太宸。時祫修,循古倫。惟聖鑒歆,愚孫忱恂。

終獻,《寧和之曲》:齊醴清兮麥熟新,籩豆潔兮孝念申。仰祖功兮宗德,願降祐兮後人。

徹饌,《雍和之曲》:樂終兮禮成,告玉振兮訖金聲。徹之弗違,以肅精誠。

還宮,《安和之曲》:三獻就兮祖宗鑒享,一誠露兮念維長。思弗盡兮思弗忘,深荷德澤之啟佑,小孫惟賴以餘光。神返宮永安,保家國益昌。

孟秋。迎神:時兮孟秋火西流,感時毖祀兮爽氣回。喜金風兮飄來,仰祖宗兮永慕哉。秋祫是舉兮希鑒歆,小孫恭迓兮捧素裁。

初獻:皇祖降筵,列聖靈聯。執事恐蹎,樂舞蹁躚。小孫捧盈兮敢弗虔。

亞獻:再酌兮玉漿,潔凈兮馨香。祖宗垂享兮錫胤昌,萬歲兮此禮行。

終獻:進酒三觥,歌舞雍韺,鐘鼓轟錚。皇祖列聖,永享愚誠。

徹饌:秋嘗是舉,稌黍豐農。三獻既周,聖靈顯容。小孫時思恩德兮惟忡。

還宮:仰皇祖兮聖神功,祀典陳兮報莫窮。嘗祫告竣,鸞馭旋宮。皇靈在天主在室,萬祀陟降何有終。

孟冬。迎神:時兮孟冬凜以淒,感時毖祀兮氣潛回。溯朔風兮北來,仰祖宗兮永慕哉。冬祫是舉兮希鑒歆,小孫恭迓兮捧素裁。

初、亞、終獻,俱同孟秋。

撤饌:冬烝是舉,俎豆維豐。三獻既周,聖靈顯容。小孫時思,恩德兮惟忡。

還宮。同孟秋,惟改「嘗祫」為「烝祫。」

大祫樂章。

迎神:仰慶源兮大發祥,惟世德兮深長。時惟歲殘,大祫洪張。祖宗聖神,明明皇皇。遙瞻兮頓首,世德兮何以忘。

初獻:神之格兮慰我思,慰我思兮捧玉卮。捧來前兮慄慄,仰歆納兮是幸已而。

亞獻:再舉瑤漿,樂舞群張。小孫在位,陪助賢良。百工羅從,大禮肅將。惟我祖宗,顯錫恩光。

終獻:思祖功兮深長,景宗德兮馨香。報歲事之既成兮典則先王,惟功德之莫報兮何以量。

徹饌:三酌既終,一誠感通。仰聖靈兮居歆,萬示異是舉兮庶乎酬報之衷。

還宮:顯兮幽兮,神運無迹。神運無迹兮化無方,靈返天兮主返室。願神功聖德兮啟佑無終,玄孫拜送兮以謝以祈。

嘉靖十年大禘樂章。

迎神,《元和之曲》:於維皇祖,肇創丕基。鐘祥有自,曰本先之。奄有萬方,作之君師。追報宜隆,以申孝思。瞻望稽首,介我休禧。

初獻,《壽和之曲》:木有本兮水有源,人本祖兮物本天。思報德兮禮莫先,仰希鑒兮敢弗虔。

亞獻,《仁和之曲》:中觴載升,於此瑤觥。小孫奉前,願歆其誠。樂舞在列,庶職在庭。祖鑒孔昭,錫祐攸亨。

終獻,《德和之曲》:於維兮先祖,延慶兮深高。追報兮曷能,三進兮香醪。

徹饌,《太和之曲》:芬兮豆籩,潔兮黍粢。祖垂歆享,徹乎敢遲。禮云告備,以訖陳辭。永裕後人,億世丕而。

送神,《永和之曲》:禘祀兮具張,佳氣兮鬱昂。皇靈錫納兮喜將,一誠通兮萬載昌。祈鑒祐兮天下康,仰源仁浩德兮曷以量。小孫頓首兮以望,遙瞻冉冉兮聖靈皇皇。

洪武七年,御製祀歷代帝王樂章。

迎神,《雍和之曲》:仰瞻兮聖容,想鑾輿兮景從。降雲衢兮後先,來俯鑒兮微衷。荷聖臨兮蒼生有崇,眷諸帝兮是臨,予頓首兮幸蒙。

奠帛,《保和之曲》:秉微誠兮動聖躬,來列坐兮殿庭。予今願兮效勤,奉禮帛兮列酒尊。鑒予情兮忻享,方旋駕兮雲程。

初獻,《保和之曲》:酒行兮爵盈,喜氣兮雍雍。重荷蒙兮載瞻載崇,群臣忻兮躍從,願睹穆穆兮聖容。

亞獻,《中和之曲》:酒斟兮禮明,諸帝熙和兮悅情。百職奔走兮滿庭,陳籩豆兮數重,亞獻兮願成。

終獻,《肅和之曲》:獻酒兮至終,早整雲鸞兮將旋宮。予心眷戀兮神聖,欲攀留兮無從。躡雲衢兮緩行,得遙瞻兮達九重。

徹饌,《凝和之曲》:納肴羞兮領陳,烝民樂兮幸生。將何以兮崇報,惟歲時兮載瞻載迎。

送神,《壽和之曲》:幡幢繚繞兮導來蹤,鑾輿冉冉兮歸天宮。五雲擁兮祥風從,民歌聖佑兮樂年豐。

望燎,《豫和之曲》:神機不測兮造化功,珍羞禮帛兮薦火中。望瘞庭兮稽首,願神鑒兮寸衷。

洪武六年定祀先師孔子樂章。

迎神,《咸和之曲》:大哉宣聖,道德尊崇。維持王化,斯民是宗。典祀有常,精純益隆。神其來格,於昭聖容。

奠帛,《寧和之曲》:自生民來,誰底其盛?惟王神明,度越前聖。粢帛具成,禮容斯稱。黍稷非馨,惟神之聽。「惟王」,後改曰「惟師」。

初獻,《安和之曲》:太哉聖王,實天生德。作樂以崇,時祀無斁。清酤惟馨,嘉牲孔碩。薦羞神明,庶幾昭格。

亞、終獻,《景和之曲》:百王宗師,生民物軌。瞻之洋洋,神其寧止。酌彼金罍,惟清且旨。登獻惟三,於戲成禮。

徹饌,《咸和之曲》:犧象在前,豆籩在列,以享以薦,既芬既潔。禮成樂備,人和神悅。祭則受福,率遵無越。

送神,《咸和之曲》:有嚴學宮,四方來宗。恪恭祀事,威儀雍雍。歆格惟馨,神馭旋復。明禋斯畢,咸膺百福。

洪武二年享先農樂章。

迎神,《永和之曲》:東風啟蟄,地脈奮然。蒼龍掛角,燁燁天田。民命惟食,創物有先。圜鐘既奏,有降斯筵。

奠帛,《永和之曲》:帝出乎震,天發農祥。神降於筵,藹藹洋洋。禮神有帛,其色惟蒼。豈伊具物,誠敬之將。

進俎,《雍和之曲》:制帛既陳,禮嚴奉牲。載之於俎,祀事孔明。簠簋攸列,黍稷惟馨。民力普存,先穡之靈。

初獻,《壽和之曲》:九穀未分,庶草攸同。表為嘉種,實在先農。黍稌斯豐,酒醴是供。獻奠之初,以祈感通。配位云:厥初生民,粒食其天。開物惟智,邃古奚傳。思文后稷,農官之先。侑神作主,初獻惟蠲。

亞獻,《壽和之曲》:倬彼甫田,其隰其原。耒耜云載,驂馭之間。報本思享,亞獻惟虔。神其歆之,自古有年。配位云:后稷配天,興於有邰。誕降嘉種,有栽有培。俶載南畝,祗事三推。佑神再獻,歆我尊罍。

終獻,《壽和之曲》:帝耤之典,享祀是資,潔豐嘉慄,咸仰於斯。時維親耕,享我農師。禮成於三,以訖陳詞。配位云:嘉德之薦,民和歲豐。帝命率育,報本之功。陳常時夏,其德其功。齊明有格,惟獻之終。

徹饌,《永和之曲》:於赫先農,歆此潔修。於篚於爵,於饌於羞。禮成告徹,神惠敢留。餕及終畝,豐年是求。

送神,《永和之曲》:神無不在,於昭於天。曰迎曰送,於享之筵。冕衣在列,金石在懸。往無不之,其佩翩翩。

望瘞,《太和之曲》:祝帛牲醴,先農既歆。不留不褻,瘞之厚深。有幽其瘞,有赫其臨。曰禮之常,匪今斯今。

嘉靖九年定享先蠶樂章。

迎神,《貞和之曲》:於穆惟神,肇啟蠶桑。衣我萬民,保我家邦。茲舉曠儀,春日載陽。恭迎霞馭,靈氣洋洋。

奠帛,《壽和之曲》:神其臨只,有苾有芬。乃獻玉歊,乃奠文纁。仰祈昭鑒,淑氣氤氳。顧茲蠶婦,祁祁如雲。

初獻,曲同奠帛。

亞獻,《順和之曲》:載舉清觴,蠶祀孔明。以格以享,鼓瑟吹笙。陰教用彰,坤儀允貞。神之聽之,鑒此禋誠。

終獻,《寧和之曲》:神之格思,桑土是宜。三繅七就,惟此繭絲。獻禮有終,神不我遺。錫我純服,藻繪皇儀。

徹饌,《安和之曲》:俎豆具徹,式禮莫愆。既匡既敕,我祀孔虔。我思古人,葛覃惟賢。明靈歆只,永顧桑阡。

送神,《恆和之曲》:神之升矣,日霽霞蒸。相此女紅,杼軸其興。茲返玄宮,鸞鳳翔騰。瞻望弗及,永錫嘉徵。

望燎,曲同送神。

○樂章二

洪武三年定朝賀樂章。

陞殿,奏《飛龍引之曲》。百官行禮,奏《風雲會之曲》。丞相致詞,奏《慶皇都之曲》。復位,百官行禮,奏《喜昇平之曲》。還宮,奏《賀聖朝之曲》。俱見後宴饗九奏中。

二十六年更定。

陞殿,韶樂,奏《聖安之曲》:乾坤日月明,八方四海慶太平。龍樓鳳閣中,扇開簾捲帝王興。聖感天地靈,保萬壽,洪福增。祥光王氣生,升寶位,永康寧。

還宮,韶樂,奏《定安之曲》:九五飛聖龍,千邦萬國敬依從。鳴鞭三下同,公卿環珮響玎東,掌扇護御容。中和樂,音呂濃,翡翠錦繡,擁還華蓋赴龍宮。

公卿入門,奏《治安之曲》:忠良為股肱,昊天之德承主恩,森羅拱北辰。御爐煙繞奉天門,江山社稷興。安天下,軍與民,龍虎會風雲。後不用。

洪武二十六年定中宮正旦、冬至、千秋節朝賀樂章。

中宮《天香鳳韶之曲》:寶殿光輝晴天映,懸玉鉤珍珠簾櫳,瑤觴舉時簫韶動。慶大筵,來儀鳳,昭陽玉帛齊朝貢。贊孝慈賢助仁風,歌謠正在昇平中,謹獻上齊天頌。

宣德以後增定慈宮朝賀樂章。

《天香鳳韶之曲》:龍樓鳳閣彤雲曉,開繡簾天香芬馥,瑤階春暖千花簇。壽聖母,齊頌祝,御筵奏獻長生曲。坤道寧品類咸育,和氣四時調玉燭,享萬萬年太平福。

洪武三年定宴饗樂章。

一奏《起臨濠之曲》,名《飛龍引》:千載中華生聖主,王氣成龍虎。提劍起淮西,將勇師雄,百戰收強虜。驅馳鞍馬經寒暑,將士同甘苦。次第靜風塵,除暴安民,功業如湯武。

二奏《開太平之曲》,名《風雲會》:玉壘瞰江城,風雲繞帝營。駕樓船龍虎縱橫,飛炮發機驅六甲,降虜將,勝胡兵。談笑掣長鯨,三軍勇氣增。一戎衣,宇宙清寧。從此華夷歸一統,開帝業,慶昇平。

三奏《安建業之曲》。名《慶皇都》:虎踞龍蟠佳麗地,真主開基,千載風雲會。十萬雄兵屯鐵騎,臺臣守將皆奔潰。一洗煩苛施德惠,里巷謳歌,田野騰和氣。王業弘開千萬世,黎民咸仰雍熙治。

四奏《削群雄之曲》,名《喜昇平》:持黃鉞,削平荊楚清吳越。清吳越,暮秦朝晉,幾多豪傑。幽燕齊魯風塵潔,伊涼蜀隴人心悅。人心悅,車書一統,萬方同轍。

五奏《平幽都之曲》,名《賀聖朝》:天運推遷虜運移,王師北討定燕畿。百年禮樂重興日,四海風雲慶會時。除暴虐,撫瘡痍,漠南爭睹舊威儀。君王聖德容降虜,三恪衣冠拜玉墀。

六奏《撫四夷之曲》,名《龍池宴》:海波不動風塵靜,中國有真人。文身交阯,氈裘金齒,重譯來賓。奇珍異產,梯山航海,奉表稱臣。白狼玄豹,九苞丹鳳,五色麒麟。

七奏《定封賞之曲》,名《九重歡》:乾坤清廓,論功定賞,策勳封爵。玉帶金符,貂蟬簪珥,形圖麟閣。奉天洪武功臣,佐興運,文經武略。子子孫孫,尊榮富貴,久長安樂。

八奏《大一統之曲》,名《鳳凰吟》:大明天子駕飛龍,開疆宇,定王封。江漢遠朝宗,慶四海,車書會同。東夷西旅,北戎南越,都入地圖中。遐邇暢皇風,億萬載,時和歲豐。

九奏《守承平之曲》,名《萬年春》:風調雨順遍乾坤,齊慶承平時節。玉燭調和甘露降,遠近桑麻相接。偃武修文,報功崇德,率土皆臣妾。山河磐固,萬方黎庶歡悅。長想創業艱難,君臣曾共掃四方豪傑。露宿宵征鞍馬上,歷盡風霜冰雪。朝野如今,清寧無事,任用須賢哲。躬勤節儉,萬年同守王業。以上九奏,前三奏和緩,中四奏壯烈,後二奏舒長。其曲皆按月律。

十二月按律樂歌。

正月太簇,本宮黃鐘商,俗名大石,曲名《萬年春》:奏天承運秉黃麾,志在安民除慝。曾睹中天騰王氣,五色虹霓千尺。龍繞兜鍪,神迎艘艦,嘉應非人力。鳳凰山上,廢雲長繞峰石。天助神武成功,人心效順,所至皆無敵。手握乾符開寶祚,略定山河南北。飲馬江淮,列營河漢,四海風波息。師雄將猛,萬方齊仰威德。

二月夾鐘,本宮夾鐘宮,俗名中呂,曲名《玉街行》:山林豺虎,中原狐兔,四海英雄無數。大明真主起臨濠,震於赫戎衣一怒。星羅玉壘,雲屯鐵騎,一掃乾坤煙霧。黎民重睹太平年,慶萬里山河磐固。

三月姑洗,本宮太簇商,俗名大石,曲名《賀聖朝》:雲氣朝生芒碭間,虹光夜起鳳凰山。江淮一日真主出,華夏千年正統還。瞻日角,睹天顏,雲龍風虎競追攀。君臣勤苦成王業,王業汪洋被百蠻。

四月仲呂,本宮無射徵,俗名黃鐘正徵,曲名《喜昇平》:風雲密,濠梁千載真龍出。真龍出,鯨鯢豺虎,掃除無迹。江河從此波濤息,乾坤同慶承平日。承平日,華夷萬里,地圖歸一。

五月蕤賓,本宮姑洗商,俗名中管雙調,曲名《樂清朝》:中原鹿走英雄起,回首四郊多壘。英主倡兵淮水,將士皆雄偉。百靈護助人心喜,一呼萬人風靡。談笑掃除螻蟻,王業從茲始。

六月林鐘,本宮夾鐘角,俗名中呂角,曲名《慶皇都》:王氣呈祥飛紫鳳,虎嘯龍興,千里旌旗動。四海歡呼師旅眾,天戈一指風雲從。將士爭先民樂用,駕御英雄,聖德皆天縱。率土華夷歸職貢,詞臣拜獻河清頌。

七月夷則,本宮南呂商,俗名中管商角,曲名《永太平》:鳳凰佳氣好,王師起義,乾坤初曉。淮水西邊,五色慶雲繚繞。三尺龍泉似水,更百萬貔貅熊豹。軍令悄,魚麗鵝鸛,風雲蛇鳥。赳赳電掣鷹揚,在伐罪安民,去殘除暴。天與人歸,豪傑削平多少。萬里煙塵凈洗,正紅日一輪高照。膺大寶,王業萬年相保。

八月南呂,本宮南呂宮,俗名中管仙呂,曲名《鳳凰吟》:紫微翠蓋擁蓬萊,聖天子,帝圖開。曆數應江淮,看五色雲生上台。櫛風沐雨,攻堅擊銳,將士總英才。躍馬定塵埃,創萬古山河壯哉。

九月無射,本宮無射宮,俗名黃鐘,曲名《飛龍引》,詞同前《起臨濠》之曲。

十月應鐘,本宮姑洗徵,俗名中呂正徵,曲名《龍池宴》:大明英主承天運,倡義擁天戈。星辰旋繞,風雲圍護,龍虎麾訶。旌旗所指,羌夷納款,江海停波。從今平定,萬年疆宇,百二山河。

十一月黃鐘,本宮夷則角,俗名仙呂角,曲名《金門樂》:慶皇明聖主開寶祚,起臨濠。正汝潁塵飛,江淮浪捲,赤子呼號。天戈奮然倡義,擁神兵百萬總英豪。貔虎朝屯壁壘,虹霓夜繞弓刀。鳳凰同勢聳層霄,佳氣五雲高。愛士伍同心,君臣協力,不憚勤勞。風雲一時相會,看魚龍飛舞出波濤。靜掃八方氛祲,咸聽九奏簫韶。

十二月大呂,本宮大呂宮,俗名高宮,曲名《風雲會》:天眷顧淮西,真人起布衣,正乾剛九五龍飛。駕馭英雄收俊傑,承永命,布皇威。一劍立鴻基,三軍擁義旗,望雲霓四海人歸。整頓乾坤除暴虐,歌聖德,慶雍熙。

武舞曲,名《清海宇》:拔劍起淮土,策馬定寰區。王氣開天統,寶曆應乾符。武略文謨,龍虎風雲創業初。將軍星繞弁,勇士月彎弧。選騎平南楚,結陣下東吳,跨蜀驅胡,萬里山河壯帝居。

文舞曲,名《泰階平》:乾坤清寧,治功告成,武定禍亂,文致太平。郊則致其禮,廟則盡其誠。卿雲在天甘露零,風雨時若百穀登。禮樂雍和,政刑肅清。儲嗣既立,封建乃行。讒佞屏四海,賢俊立朝廷。玉帛鐘鼓陳兩楹,君臣賡歌揚頒聲。

四夷舞曲,其一,《小將軍》:大明君,定宇寰,聖恩寬,掌江山,東虜西戎,北狄南蠻,手高擎,寶貝盤。其二,《殿前歡》:五雲宮闕連霄漢,金光明照眼。玉溝金水聲潺潺,頫囟觀,趨蹌看,儀鑾嚴肅百千般,威人心膽寒。其三,《慶新年》:虎豹關,文武班,五綵間慶雲朝霞燦。黃金殿,喜氣增,丹墀內,仰聖顏。翠繞紅圍錦繡班,高樓十二欄。笙簫趁紫壇,仙音韻,瑤闉按,拜舞齊,歌謠纘,吾皇萬壽安。其四,《過門子》:定宇寰,定宇寰,掌江山,撫百蠻。謳歌拜舞仰祝纘,萬萬年,帝業安。

洪武十五年重定宴饗九奏樂章。

一奏《炎精開運之曲》:炎精開運,篤生聖皇。大明御極,遠紹虞唐。河清海宴,物阜民康。威加夷僚,德被戎羌。八珍有薦,九鼎馨香。鼓鐘鐄鐄,宮徵洋洋。怡神養壽,理陰順陽。保茲遐福,地久天長。

二奏《皇風之曲》:皇風被八表,熙熙聲教宣。時和景象明,紫宸開繡筵。龍袞曜朝日,金爐裊祥煙。濟濟公與侯,被服麗且鮮。列坐侍丹扆,磬折在周旋。羔豚升華俎,玉饌充方圓。初筵奏《南風》,繼歌賡載篇。瑤觴欣再舉,拜俯禮無愆。同樂及斯辰,於皇千萬年。

奏《平定天下之舞》,曲名《清海宇》。同前。

三奏《眷皇明之曲》:赫赫上帝,眷我皇明。大命既集,本固支榮。厥本伊何?育德春宮。厥支伊何?籓邦以寧。慶延百世,澤被群生。及時為樂,天祿是膺。千秋萬歲,永觀厥成。

奏《撫安四夷之舞》,曲名《小將軍》、《殿前歡》、《慶新年》、《過門子》。俱同前。

四奏《天道傳之曲》:馬負圖兮天道傳,龜載書兮人文宣。羲畫卦兮禹疇敘,皇極建兮合自然。綿綿曆數歸明主,祥麟在郊威鳳舞。九夷入貢康衢謠,聖子神孫繼祖武,垂拱無為邁前古。

奏《車書會同之舞》,曲名《泰階平》。同前

五奏《振皇綱之曲》:《周南》詠麟趾,《卷阿》歌鳳凰。藹藹稱多士,為楨振皇綱。赫赫我大明,德尊踰漢唐。百揆修庶績,公輔理陰陽。峨冠正襟佩,都俞在高堂,坐令八紘內,熙熙民樂康。氣和風雨時,田疇歲豐穰。獻禮過三爵,歡娛良未央。

六奏《金陵之曲》:鐘山蟠蒼龍,石城踞金虎。千年王氣都,於今歸聖主。六代繁華經幾秋,江流東去無時休。誰言天塹分南北,英雄豈但嗤曹劉。我皇昔住濠梁屋,神遊天錫真人服,提兵乘勢渡江來,語臣早獻《金陵曲》。歌金陵,進珍饌,皆八音,繼三歎。請觀漢祖用兵時,為嘗馮異滹沱飯。

七奏《長楊之曲》:長楊曳綠,黃鳥和鳴。菡萏呈鮮,紫燕輕盈。千花浥露,日麗風清。及時為樂,芳尊在庭。管音嘒嘒,絲韻泠泠,玉振金聲,各奏爾能。皤皤國老,載勸載懲。明德惟馨,垂之聖經。《唐風》示戒,永保嘉名。無已太康,哲人是聽。

八奏《芳醴之曲》:夏王厭芳醴,商湯遠色聲。聖人示深戒,千春垂令名。惟皇登九五,玉食保尊榮。日昃不遑餐,布德延群生。天庖具豐膳,鼎鼐事調烹,豈但資肥甘,亦足養遐齡。達人悟茲理,恆令五氣平。隨時知有節,昭哉天道行。

九奏《駕六龍之曲》:日麗中天漏下遲,公卿侍宴多令儀。簫韶九奏觴九獻,爐煙細逐祥風吹。群臣舞蹈天顏喜,歲熟民康常若此。六龍回駕鳳樓深,寶扇齊開扶玉幾。景星呈瑞慶雲多,兩曜增暉四序和。聖人道大如天地,歲歲年年奈樂何。

進膳曲,《水龍吟》:寶殿祥雲紫氣濛,聖明君,龍德宮。氤氳霧靄,檜柏間青松。龍樓鳳閣,雕梁畫棟,此是蓬萊洞。

太平清樂曲,《太清歌》:萬國來朝進貢,仰賀聖明主,一統華夷。普天下八方四海,南北東西。託聖德,勝堯王,保護家國太平,天下都歸一,將兵器銷為農器。旌旗不動酒旗招,仰荷天地。《上清歌》:一願四時風調雨順民心喜。攝外國,將寶貝;攝外國,將寶貝;見君王,來朝寶殿裏,珊瑚、瑪瑙、玻璃,進在丹墀。《開天門》:託長生,日月光天德,萬萬歲永固皇基。公卿文武來朝會,開玳筵,捧金盃。

迎膳,奏《水龍吟曲》,與進膳同。升座、還宮、百官行禮,奏《萬歲樂》、《朝天子》二曲,與朝賀同。

大祀慶成大宴,用《萬國來朝隊舞》、《纓鞭得勝隊舞》。

萬壽聖節大宴,用《九夷進寶隊舞》、《壽星隊舞》。冬至大宴,用《贊聖喜隊舞》、《百花聖朝隊舞》。

正旦大宴,用《百戲蓮花盆隊舞》、《勝鼓采蓮隊舞》。

永樂十八年定宴饗樂舞。

一奏《上萬壽之曲》:龍飛定萬方,受天命,振紀綱。彞倫攸敘四海康,普天率土盡來王。臣民舞蹈,嵩呼載揚,稱觴奉吾皇,聖壽天長。

《平定天下舞曲》,其一,《四邊靜》:威伏千邦,四夷來賓納表章。顯禎祥,承乾象,皇基永昌,萬載山河壯。其二,《刮地風》:聖主過堯、舜、禹、湯,立五常三綱。八蠻進貢朝今上,頓首誠惶。朝中宰相,變理陰陽。五穀收成,萬民歡暢。賀吾皇,齊贊揚,萬國來降。

二奏《仰天恩之曲》:皇天眷聖明,五辰順,四海寧,風調雨順百穀登,臣民鼓舞樂太平。賢良在位,邦家永禎。吾皇仰洪恩,夙夜存誠。

黃童白叟鼓腹謳歌承應曲,曰《豆葉黃》:雨順風調,五穀收成,倉廩豐盈,大利民生。託賴著皇恩四海清,鼓腹謳歌,白叟黃童,共樂咸寧。

四夷舞曲,其一,《小將軍》:順天心,聖德誠,化番邦,盡朝京。四夷歸伏,舞於龍廷。貢皇明,寶貝擎。其二,《殿前歡》:四夷率土歸王命,都來仰大明。萬邦千國皆歸正,現帝庭,朝仁聖。天階班列眾公卿,齊聲歌太平。其三,《慶豐年》:和氣增,鸞鳳鳴,紫霧生,祥雲朝霞映。爇金爐,香味馨,列丹墀,御駕盈。絃管簫韶五音應,龍笛間鳳笙。其四,《渤海令》:金盃中,酒滿盛。御案前,列群英。君德成,皇圖慶,嵩呼萬歲聲。其五,《過門子》:聖主興,聖主興,顯威靈,蠻夷靜。至仁至德至聖明,萬萬年,帝業成。

三奏《感地德之曲》:皇心感地靈,順天時,德厚生。含弘光大品物亨,鐘奇毓秀產俊英。河清海宴,麟來鳳鳴,陰陽永和平,相我文明。

《車書會同舞曲》,其一,《新水令》:錦衣花帽設丹墀,具公服百司同會。麟至舞,鳳來儀,文武班齊,朝賀聖明帝。其二,《水仙子》:八方四面錦華夷,天下蒼生仰聖德。風調雨順昇平世,遍乾坤,皆贊禮,託君恩民樂雍熙。萬萬年皇基堅固,萬萬載江山定體,萬萬歲洪福天齊。

四奏《民樂生之曲》:世間的萬民,荷天地,感聖恩。乾坤定位四海春,君臣父子正大倫。皇風浩蕩,人心載醇,熙熙樂天真,永戴明君。

《表正萬邦舞曲》,其一,《慶太平》:奸邪濁亂朝綱,構禍難,煽動戈AU。赫怒吾皇,親征灞上,指天戈,敵皆降。其二,《武士歡》:白溝戰場,旌旗雲合迷日光。令嚴氣張,三軍踴躍齊奮揚,掃除殘甲如風蕩,凱歌傳四方。仁聖不殺降,望河南,失欃槍。其三,《滾繡球》:肆旅拒,恃力強,一心構殃,築滄洲百尺城隍。騁蠆毒,恣虎狼,孰能禦當。順天心有德隆昌,倒戈斂甲齊歸降,撫將生還達故鄉,自此仁聞愈彰。其四,《陣陣贏》:不數孫吳兵法良,神謀睿算合陰陽,八陣堂堂行天上,虎略龍韜孰敢當。俘囚十萬皆疏放,感荷仁恩戴上蒼。其五,《得勝回》:兩傍四方,展鳥翼風雲雁行。出奇兵,敵難量,士強馬強。遍百里,眠旌臥槍。勝兵回,樂洋洋。其六,《小梁州》:敵兵戰敗神魂喪,擁貔貅,直渡長江。開市門,肆不移,宣聖恩,如天曠。綸音頒降,普天下,仰吾皇。

五奏《感皇恩之曲》:當今四海寧,頌聲作,禮樂興。君臣慶會躋太平,衣冠濟濟宴彤庭。文臣武將,共荷恩榮,忠心盡微誠,仰答皇明。

《天命有德舞曲》,其一,《慶宣和》:雨順風調萬物熙,一統華夷。四野嘉禾感和氣,一幹百穗,一幹百穗。其二,《窄磚兒》:梯航萬國來丹陛,太平年,永固洪基。正東西南北來朝會,洽寰宇,布春暉,四夷咸賓聲教美。自古明王在慎德,不須威武服戎狄。祥瑞集,鳳來儀。佳期萬萬歲,聖明君,主華夷。

六奏《慶豐年之曲》:萬方仰聖君,大一統,撫萬民。豐年時序雨露均,穰穰五穀貨財殷。酣歌擊壤,風清俗淳,四夷悉來賓,正統皇仁。

七奏《集禎應之曲》:皇天眷大明,五星聚,兆太平;騶虞出現甘露零,野蠶成繭嘉禾生,醴泉湧地河水清。乾坤萬萬年,四海永寧。

八奏《永皇圖之曲》:天心眷聖皇,正天位,撫萬邦。仁風宣布禮樂張,戎夷稽首朝明堂。皇圖鞏固,賢臣贊襄。太平日月光,地久天長。

九奏《樂太平之曲》:皇恩被八紘,三光明,四海清。人康物阜歲屢登,含哺鼓腹皆歡聲。民歌帝力,唐堯至仁。乾坤永清,共樂太平。

導膳、迎膳、進膳及升座、還宮、百官行禮諸曲,俱與洪武間同。

大祀慶成,用《纓鞭得勝蠻夷隊舞》;萬壽聖節,《九夷進寶隊舞》;冬至節,《贊聖喜隊舞》;正旦節,《百戲蓮盆隊舞》。

○嘉靖間續定慶成宴樂章

升座,樂曲《萬歲樂》:五百昌期嘉慶會,啟聖皇,龍飛天位。九州四海重華日,大明朝,萬萬世。

百官行禮,樂曲《朝天子》:滿前瑞煙,香繞蓬萊殿,風回韶律鼓淵淵,列陛上,旌旗絢,日至朱躔。陽生赤甸氣和融,徹上元。歷年萬千,長慶天宮宴。

上護衣、上花,樂曲《水龍吟》:寶殿金爐瑞靄浮,陳玉案,列珍羞。天花炫彩,照曜翠雲裘。鸞歌鳳舞,虞庭樂奏,萬歲君王壽。

一奏《上萬歲之曲》:聖主垂衣裳,興禮樂,邁虞唐。簫韶九成儀鳳凰,日月中天照八荒。民安物阜,時和歲康。上奉萬年觴,胤祚無疆。

奏《平定天下舞曲》,其一,《四邊靜》:天啟嘉祥,聖主中興振紀綱。頌洋洋,功蕩蕩,國運隆昌,萬歲皇圖壯。其二,《鳳鸞吟》:維皇上天佑聖明,景命宣,五雲輝,三台潤,七緯光懸。協氣生,嘉祥見。正萬民,用群賢。垂袞御經筵,宵衣勤政殿。禮圜丘大祀精虔,明水潔,蒼璧圓。秉周文,承殷薦,眷皇家億萬斯年。

二奏《仰天恩之曲》:皇穹啟聖神,欽乾運,祗郊禋。一陽初動靄先春,萬福來同仰至仁。祥開日月,瑞見星辰。禮樂協神人,宇宙咸新。

迎膳曲,《水龍吟》:春滿雕盤獻玉桃,葭管動,日輪高。熹微霽色,遙映袞龍袍。千官舞蹈,鈞韶迭奏,曲度昇平調。

進膳曲,《水龍吟》:紫禁瓊筵暖應冬,驂八螭,乘六龍,玉卮瓊斝,黼座獻重瞳。堯天廣運,舜雲飛動,喜聽賡歌頌。

進湯曲,其一,《太清歌》:長至日,開黃道,喜乾坤佳氣,陽長陰消。奏鈞韶,音調鳳軫,律協鸞簫。仰龍顏,天日表,如舜如堯。金爐煙暖御香飄,玉墀晴霽祥光繞。宮梅苑柳迎春好,燕樂蓬萊島。其二,《上清歌》:雲捧宸居,五星光映三台麗。仰日月,層霄霽;仰日月,層霄霽。中興重見唐虞際,太和元氣自陽回,兆姓歡愉。其三,《開天門》:九重霄,日轉皇州曉。宴天家,共歌《魚藻》。龍鱗雉尾高,祝聖壽,慶清朝。

奏黃童白叟鼓腹謳歌承應曲,《御鑾歌》:雅奏樂昇平,瞻絳闕,集瑤京。黃童白叟喜氣盈,謳歌鼓舞四海寧。金枝結秀,玉樹含英。聽康衢擊壤聲,帝力難名。

三奏《感昊德之曲》:帝德運光明,一陽動,萬物生。升中大報蒼璧陳,禮崇樂暢歆太清。星懸紫極,日麗璇庭,乾坤瑞氣盈,海宇安寧。

奏《撫安四夷舞曲》,其一,《賀聖朝》:華夷一統,萬國來同。獻方物,修庭貢,遠慕皇風,自南自北,自西自東。望天宮,佳氣鬱重重,四靈畢至,麟鳳龜龍。其二,《殿前歡》:瑞雲晴靄浮宮殿,一脈陽和轉。禮成交泰開周宴,鳳笙調,龍幄展,天心感格人歡忭,四海謳歌遍。其三,《慶豐年》:賴皇天,錫豐年,勤禹稼,力舜田,喜慰三農願。嘉禾秀,瑞麥鮮,賦九州,貢八埏。神倉御廩咸充滿,養民以養賢。其四,《新水令》:聖德精禋格昊穹,大一統。四夷來貢,玉帛捧。文軌同,世際昌隆,共聽輿人頌。其五,《太平令》:誕明禋,天鑒元后,光四表,惠澤周流。來四裔,趨前擁後,獻萬寶,充庭滿囿。稽首頓首,天高地厚,祝聖人,多男福壽。

四奏《民樂生之曲》:大報禮初成,象乾德,運皇誠。神州赤縣永清寧,靈雨和風樂太平。陰陽交暢,品物咸亨,元化自流行,允殖群生。

迎膳曲,《水龍吟》:五色祥雲捧玉皇,開閶闔,坐明光。鈞天樂奏,冬日御筵張。文恬武熙,太平氣象,人在唐虞上。

進膳曲,《水龍吟》:玉律陽回景運新,燕鎬京,藹皇仁。光昭雲漢,一氣沸韶韺。錦瑟和聲,瑤琴清韻,瞻仰天顏近。

進湯曲,《太清歌》:萬方民,樂時雍,鼓舞荷天工,雷行風動。喜今逢,南蠻北貊,東夷西戎,來朝貢。大明宮,星羅斗拱。九重天上六飛龍,五色雲間雙彩鳳,普天率土效華封,允協河清頌。

奏《車書會同舞曲》,其一,《新水令》:五雲深護九重城,感洪恩。一人有慶,陽初長,禮方行。帝德文明,表率邦家正。其二,《水仙子》:萬方安堵樂康寧,九域同仁荷聖明。千年撫運承天命,露垂甘,河獻清,見雙岐秀麥連莖,喜靈雪隨冬應,睹祥雲拂曙生,神與化並運同行。

五奏《感皇恩之曲》:雙闕五星光,霓旌樹,紫蓋張。璇臺玉曆轉新陽,鈞天廣樂諧宮商。恩深露湛,喜溢霞觴,日月煥龍章,地久天長。

奏《表正萬邦舞曲》,其一,《慶太平》:維天眷我聖明,禮圜丘,至德精誠。乾元永清,洪膺景命,休徵應,泰階平。其二,《千秋歲》:聖主乘龍御萬邦,慶雲翔化日重光。群臣拜舞稱壽觴,載歌《天保》章。其三,《滾繡球》:五雲車,度九重,利見飛龍。耀袞章,火藻華蟲。擊虎敔,考鳧鐘,鼉鼓逢逢。八珍列,九鼎豐隆。堯眉揚彩舜重瞳,萬國咸熙四海雍,齊歌頌聖德神功。其四,《殿前歡》:萬年禮樂中興日,大化睹重熙。河清海宴臻祥瑞,五行順,七政齊,超三邁五貞元會,既醉頌鳧鷖。其五,《天下樂》:萬靈朝拱接清都,享南郊,欽天法祖。願聖人,承乾納祜,中和位育,龜獻範,馬陳圖。其六,《醉太平》:禮樂萬年規,謳歌四海熙。衣冠蹈舞九龍墀,麗正仰南離。紫雲高捧唐虞帝,垂衣天下文明治。鎬鳥岐鳳呈嘉瑞,真個是人在成周世。

六奏《慶豐年之曲》:聖人懋承乾,綏萬邦,屢豐年。神倉御廩登天田,明粢鬱鬯祀孔虔。輿情咸豫,協氣用宣,萬古帝圖傳,璧合珠聯。

七奏《集禎應之曲》:天保泰階平,寶露降,渾河清,喜禾秀麥集休禎,遐陬絕域喜氣盈。一人有慶,百度惟貞,萬國頌咸寧,麗正重明。

八奏《永皇圖之曲》:鎬燕集天京,頌《魚藻》,歌《鹿鳴》。邊陲安堵萬國寧,重譯來庭四海清。咸池日曙,昧谷雲征,帝座仰前星,豫大豐亨。

九奏《樂太平之曲》:皇極永登祥,乾符啟,泰運昌。玉管回春動一陽,金鑾錫燕歌九章,虞庭獸舞,岐山鳳翊,日麗袞龍裳,主聖臣良。

迎膳曲,《水龍吟》:香霧氤氳紫閣重,仰天德,瞻帝容。星輝海潤,甘雨間和風。樂比鳶魚,瑞呈麟鳳,永獻《卷阿》頌。

進膳曲,其一,《水龍吟》:萬戶千門啟建章,台階峻,帝座張。三垣九道,北斗玉衡光。元氣調和,雅韻鏗鏘,昭代慶明良。其二,《太清歌》:萬方國,盡來庭。稽首歌帝仁,仰荷生成。振乾綱,陰陽順序,民物樂生。逢明聖,萬年春,永膺休命。華夷蠻僚咸歸正,蒼生至老不知兵,鼓腹含哺囿太平,九有享清寧。

奏《天命有德舞曲》,其一,《萬歲樂》:太平天子興隆日,履初長,陽回元吉。醴泉芝草休徵集,曾聞道五星聚室。其二,《賀聖朝》:一人元良,百度維新。握赤符,凝玄應,享太清。大禮方行,祀事孔明感天心,億載恆承慶。明王慎德,四裔咸賓。

奏《纓鞭得勝蠻夷隊舞承應曲》,其一,《醉太平》:星華紫殿高,雲氣彤樓繞。九夷重譯梯航到,皇圖光八表。玉宇無塵明月皎,銀河自轉扶桑曉,平平蕩蕩歸王道。百獸舞,鳳鳴簫韶。其二,《看花會》:普天下,都賴吾皇至聖。看玉關頻款,天山已定。四夷效順歸王命。《天保》歌,群黎百姓。其三,《天下樂》:九重樂奏萬花開,望龍樓,雲蒸霧靄。仰天工,雍熙帝載,臣民歡戴。溥仁恩,遍九垓。其四,《清江引》:黃鐘既奏陽和長,德盛天心貺。人文日月明,國勢山河壯,衢室民謠頻擊壤。

奏致語曲,其一,《清江引》:鈞天畢奏日方中,既醉歡聲動。雲章傍袞龍,飆勢翔威鳳,萬方安樂興嘉頌。其二,《千秋歲》:上下交歡燕禮成,一陽奮,萬匯咸亨。風雲會合開明運,紫極轉璇衡。

宴畢,百官行禮曲,《朝天子》:文班武班,歡動承明殿,禮成樂備頌聲喧。真咫尺,仰天顏,日照龍筵。風回雉扇翠蕤旋,奉仙鑾,雲間斗間,五色金章燦。

還宮曲,《萬歲樂》:天回北極雲成瑞,望層霄,重華日麗。九垓八極樂雍熙,祝聖壽,萬萬歲。

永樂間小宴樂章。

一奏《本太初之曲》,《朝天子》:混兮沌兮,水土成元氣,不分南北與東西,未辨天和地。萬象包涵,其中秘密,難窮造化機,是陰陽本體。乃為之太極,兩儀因而立。

二奏《仰大明之曲》,《歸朝歡》:太極分,混然方始見儀形,清浮濁偃乾坤定。日月齊興,照青霄,萬象明。陽須動,陰須靜,陰與陽,皆相應。流行二氣,萬物俱生。

三奏《民初生之曲》,其一,《沽美酒》:乾坤清,宇宙寧,六合凈,四維正,萬象原來一氣生。定三才五行,民與物,共成群。其二,《太平令》:為一類不分人品,竟生食豈曉庖烹,避寒暑巢居穴遁,披樹葉相尋趁,如何是愛親。世情治生。雖混然各安其性。

四奏《品物亨之曲》,《醉太平》:黎民生世間,萬物長塵寰,陰陽交運轉循環,久遠時庶繁。相傳氣候應無間,品物交錯憑誰鑒。望聖人出世整江山,主萬民得安。

五奏《御六龍之曲》,其一,《清江引》:人心久仰生聖君,天使人生聖。聖人受天機,體天居中正,御六龍,驛明登九重。其二,《碧玉簫》:君坐神京,海嶽共從新。民仰君恩,聖治有人倫。人品分,萬物增,聖承乾,百福臻。垂法明,尊天命,興後朝,皆從正。

六奏《泰階平之曲》,《十二月》:聖乃有言天,天是無言聖。聖人臨正,萬物亨通,恩威盛,社稷安,仁德感,江山定。選用英賢興王政,分善惡賞罰均平。三公九卿,左右股肱,庶事康寧。

七奏《君德成之曲》,其一,《十二月》:皇基以興,聖帝修身,奉天體道,聖德愈明。敬天地,勤勞萬民;立法度,上下咸寧。其二《堯民歌》:風俗禮樂厚彝倫,爰興學校進儒經,賢臣良將保朝廷,四野人民頌歡聲。用的是賢英,賢英定太平,寰海皆歸正。

八奏《聖道行之曲》,其一,《金殿萬年歡》:三綱既定,九疇復興。聖道如天,嘉禾齊秀,寒暑和平。聖威無邊皇基穩,勝磐石,慶雲生。景星長現,三光輝耀,百穀收成,萬姓安寧。其二,《得勝令》:聖德感皇乾,甘露降山川。萬邦來朝貢,奇珍擺布全,玉階下鳴鞭。仰聖主,升金殿,丹墀列英賢,贊吾皇,豐稔年。

九奏《樂清寧之曲》,其一,《普天樂》:萬邦寧,皇圖正。父君母后,天下咸欽。君治外,永聖明;后治內,長安靜。後聖承乾皆從正,德相傳,聖子神孫。天威浩蕩,江山永固,洪福無窮。其二,《沽美酒》:和氣生,滿玉京;祥煙起,映皇宮。明聖開基整萬民,風雲會帝庭。奏簫韶,九韻成。其三,《太平令》:紫霧隱金鸞彩鳳,祥光罩良將賢臣。玉案列珍羞美醞,寶鼎爇龍涎香噴。至尊永寧,儲嗣守成,賀萬萬歲一人有慶。

右二奏至八奏,俱奏百戲承應;第九奏,《魚躍于淵》承應。奉天門宴百官,止用《本太初》、《仰大明》、《民初生》三奏曲,其進酒、進膳樂同。惟百官叩頭禮,用《朝天子》曲。宴畢,導駕還宮,用《御鑾歌》。

嘉靖間仁壽宮落成宴饗樂章。

一奏《本太初之曲》,《朝天子》:帝誠帝明,寶位基昌命。仙苑開筵歌《鹿鳴》,亭殿天章映。我有嘉賓,鼓瑟吹笙。示周行,昭德音。日升月恆,萬載皇圖正。

二奏《仰大明之曲》,《殿前歡》:天保定聖人,多壽多男慶。修和禮樂協中興,麗正重明,如山阜,如岡陵,如川方至莫不增。協氣生,禎祥應,百神受命,萬國來庭。

三奏《民初生之曲》,其一,《沽美酒》:黃河清,寶露凝,瑞麥呈,靈鵲鳴,諸福來同仰聖明。喜萬寶告成,占景緯,泰階平。其二,《太平令》:念農桑,衣食之本;仰君德,獨厚民生。事耕鑿,群黎百姓。歌《鹿鳴》,神人胥慶。明主宴嘉賓,承筐鼓瑟吹笙,繼自今福增天定。

四奏《品物亨之曲》,《醉太平》:瑤宮怡聖顏,閬苑隔人寰。吹笙鼓瑟賓,旨酒天開宴,《鹿鳴》歌舞黃金殿,賴吾皇錫福萬民安,醉歌《天保》歡。

五奏《御六龍之曲》,其一,《清江引》:聖主有道樂昇平,宴會延休慶。務本軫民生,弘化凝天命。欣落成,萬載開鴻運。其二,《碧玉簫》:帝重農桑,法駕起明光。麟遊鳳翔,宴陳《天保》章。開玳筵,薦瑤觴,既醉頌洋洋。聖德巍,皇恩蕩,世際唐虞上。

進膳曲,其一,《水龍吟》:寶瑟瑤笙鼓吹喧,聖天子,御華筵。南山萬壽,瑞日正中天。百穀豐年,八方珍膳,人樂昇平宴。其二,《太清歌》:祥麥嘉瓜臻瑞,仰荷堯舜主,愛育群黎,感天意五風十雨。秋報春祈遍爾德,勸農桑,日用衣食。嘉賓和樂開筵地,紅雲捧雕盤珍味。山呼萬歲福無疆,日升川至。其三,《上清歌》:仰賴吾皇,參天兩地凝和氣。四三王,六五帝,四三王,六五帝,國家興,賢才為上瑞。養萬民,九域熙,百祿咸宜。其四,《開天門》:寶殿輝,龍虎風雲會。瞻丹陛,覲紫微,周詩歌《既醉》,《螽斯》《麟趾》開祥瑞,仰飛龍,在天位。

豳風亭宴講官樂章。

一奏《本太初之曲》,《朝天子》:九重詔傳,殿閣開秋宴。授衣時節肅霜天,禾稼登場遍。鼓瑟吹笙,昇平重見,工歌《七月》篇。春酒當筵獻,願吾皇萬年,歲歲臨西苑。

二奏《仰大明之曲》,《殿前歡》:鳳苑御筵開,黃花映玉階。《鹿鳴》《天保》歌三代,古調新裁,奉君王壽盃。日月明,乾坤大,看年年秋報賽。太平有象,元首明哉。

三奏《民初生之曲》,其一,《沽美酒》:熙春陽,化日長。執懿筐,採柔桑。拾繭繰絲有萬箱。染紅黃孔陽,為公子製衣裳。其二,《太平令》:勤樹藝,歲年豐穰,九十月禾黍登場。為春酒甕浮新釀,村田樂齊歌齊唱。饗公堂,殺羊舉觴,繼進著兕觥,祝聖壽,萬靈扶相。

四奏《品物亨之曲》,《醉太平》:納嘉禾滿場,釀御酒盈缸,公桑蠶績製玄黃,服龍衣袞裳。螽斯蟋蟀諧清唱,水光山色明仙仗,幽風亭殿進霞觴,祝聖壽無疆。

五奏《御六龍之曲》,其一,《清江引》:九月風光何處有?鳳苑在龍池右。農夫稼已登,公子衣方授,萬歲君王頻進酒。其二,《碧玉簫》:凡我生民,農桑最苦辛,終歲經營。氣候變冬春,田畯欣,婦子勤。言永豳詩,仰化鈞,場圃新,風雨順。宴御墀,龍顏近。

進膳曲,其一,《水龍吟》:養老休農敞御筵,瀉春酒,介耆年,刲羊剪韭,社鼓正闐闐。香粳米顆,升堂拜獻,此樂真堪羨。其二,《太清歌》:九月天,開西苑,宸居無逸殿,講幄張筵。集儒流,雲蒸星炫,璧緯珠躔。睹御製,煥天章,昭回雲漢。堯天舜日民安宴,御廩神倉百穀登,金輝玉燦休徵見,大有豐年。其三,《上清歌》:鳳苑宸居,公桑帝耤今方舉。躬耕蠶,勸士女,躬耕蠶,勸士女。獻羊羔,升堂奏樂舞,葵菽棗壺上珍廚,萬壽山呼。其四,《開天門》:豳風亭,共仰吾皇聖。百穀登,萬國咸寧。民康物阜禎祥應,仰乾運,俯坤靈。

隆慶三年大閱禮成回鑾樂章。

《武成之曲》:吾皇閱武成,簡戎旅,壯帝京。龍旗照耀虎豹營,六師雲擁甲胄明。威靈廣播,蠻夷震驚,稽首頌昇平,四海澄清。

嘉靖間皇后親蠶宴內外命婦樂章。

升座,奏《天香鳳韶之曲》:春雲繚繞芳郊曙,喜乾坤萬象感舒,蘭皋蕙圃迎仙馭。采桑條,攀茂樹。蠶宮繭館親臨御。璧月珠星照太虛,開筵還駐翠旓AV,萬載垂貞譽。

進膳曲,《沽美酒》:蠶禮成,鳳輦停,薦霞觴,列雲屏。宮妃世婦仰坤寧。祥雲映紫冥,同祝頌,耀前星。

回宮,《御鑾歌》:惟天啟聖皇,君耕耤,后躬桑,身先田織率萬邦。天清地寧民阜康,百穀用成,四夷來王。治化登虞唐,世發禎祥。

永樂間定東宮宴饗樂章。

一奏《喜千春之曲》,《賀聖朝》:開國承天,聖感極多,總一統,封疆闊。百姓快活,萬物榮光,共沐恩波。仙音韻,合贊升平詠歌。齊朝拜,千千歲東宮,滿國春和。

二奏《永南山之曲》,《水仙子》:洪基永固海波清,盛世明時禮樂興,華夷一統江山靜。民通和,樂太平。贊東宮仁孝賢明,秉鈞衡端正,順乾坤泰亨,坐中華萬世昌寧。

三奏《桂枝香之曲》,《蟾宮曲》:曉光融,宴饗春宮,日朗風和,喜氣蔥蔥。鎮領台樞,規宏綱憲,禮節至公。事聖上柔聲婉容,問安寧勤孝虔恭。果斷寬洪,剛健文明,聖德合同。

四奏《初春曉之曲》,《小梁州》:端拱嚴宸事紫微,秉運璇璣,四時百物總相宜。仰賴明君德,大業勝磐石。皇儲仁孝明忠義,美遐方順化朝儀。孝能歡慈愛心,敬篤上尊卑意,禮上和下睦民,鼓舞樂雍熙。

五奏《乾坤泰之曲》,《滿庭芳》:春和玳筵,安邦興國,欽聖尊賢,文英武烈於民便。禮樂成全,享大業中庸不偏,順天常節儉為先,達文獻嚴儀訓典,孝敬億千年。

六奏《昌運頌之曲》,《喜秋風》:文武安,軍民樂。宴文華,會班僚,五雲齊動鈞天樂。賀春宮,贊皇朝。

右二奏至六奏,俱奏百戲承應。

七奏《泰道開之曲》,《沽美酒》:布春風,滿畫樓,對嘉景,鳳凰洲。高捧金波碧玉甌,設威儀左右,分品從,列公侯。其二,《太平令》:效聖上誠心勤厚,主宗器嚴備《春秋》,諧律呂仙音齊奏,欽王政皇天保佑。拜舞頓首,讚祝進酒,千千歲康寧福壽。

迎膳樂曲,《水龍吟》:方響笙闉鼓樂喧,排寶器,開玳筵。鸞儀旌工,錦繡景相連。簪纓趨進,皆來朝見,春滿文華殿。

升座、還宮、百官行禮,奏《千秋歲曲》:堯年舜日勝禹周,慶雲生繚繞鳳樓。風調雨順五穀收,萬民暢歌謳。朔望朝參同。
