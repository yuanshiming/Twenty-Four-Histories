\article{列傳第一百五十六 曹文詔弟文耀 周遇吉 黃得功}

\begin{pinyinscope}
曹文詔,大同人。勇毅有智略。從軍遼左,歷事熊廷弼、孫承宗,積功至遊擊。崇禎二年冬,從袁崇煥入衛京師。明年二月,總理馬世龍畀所賜尚方劍,令率參將王承胤、張叔嘉、都司左良玉等伏玉田、枯樹、洪橋,鏖戰有功,遷參將。自大塹山轉戰逼遵化,又從世龍等克大安城及占魚諸關。以與復四城功,加都督僉事。七月,陜西賊熾,擢延綏東路副總兵。

賊渠王嘉胤入據河曲。四年四月,文詔克其城。嘉胤脫走,轉掠至陽城南山。文詔追及之,其下斬以降,以功擢臨洮總兵官。

點燈子自陜入山西。文詔追之,及於稷山,諭降七百人。點燈子遁,尋被獲,伏誅。

李老柴、獨行狼陷中部,巡撫練國事、延綏總兵王承恩圍之。五月,慶陽賊郝臨庵、劉道江援之。會文詔西旋,與榆林參政張福臻合剿,馘老柴及其黨一條龍,餘黨奔摩雲谷。副將張弘業、遊擊李明輔戰死。文詔乃與遊擊左光先、崔宗廕、李國奇分剿綏德、宜君、清澗、米脂賊,戰懷寧川、黑泉峪、封家溝、綿湖峪,皆大捷,掃地王授首。

紅軍友、李都司、杜三、楊老柴者,神一魁餘黨也,屯鎮原,將犯平涼。國事檄甘肅總兵楊嘉謨、副將王性善扼之,賊走慶陽。文詔從鄜州間道與嘉謨、性善合。五年三月,大戰西濠,斬千級,生擒杜三、楊老柴。餘黨糾他賊掠武安監,陷華亭,攻莊浪。文詔、嘉謨至,賊屯張麻村。官軍掩擊,賊走高山。遊擊曹變蛟、馮舉、劉成功、平安等噪而上,賊潰走。變蛟者,文詔從子也。會性善及甘肅副將李鴻嗣、參將莫與京等至,共擊斬五百二十餘級。追敗之咸寧關,又敗之關上嶺。追至隴安,嘉謨、變蛟夾擊,復敗之。賊餘眾數千欲走漢南,為遊擊趙光遠所遏,乃由長寧驛走張家川。其逸出清水者,副將蔣一陽遇之敗,都司李宮用被執。文詔乃縱反間,紿其黨,殺紅軍友,遂蹙敗之水落城。追至靜寧州,賊奔據唐毛山,變蛟先登,殄其眾。

可天飛、郝臨庵,劉道江為王承恩所敗,退保鐵角城。獨行狼、李都司走與合,可天飛、劉道江遂圍合水。文詔往救。賊匿精銳,以千騎逆戰,誘抵南原,伏大起。城上人言曹將軍已歿。文詔持矛左右突,匹馬縈萬眾中。諸軍望見,夾擊,賊大敗,僵屍蔽野,餘走銅川橋。文詔率變蛟、舉、嘉謨及參將方茂功等追及之,大戰陷陣,賊復大敗。尋與寧夏總兵賀虎臣、固原總兵楊麒破賊甘泉之虎兕凹。麒復追賊安口河、崇信窯、白茅山,皆大獲。總督洪承疇斬可天飛、李都司於平涼,降其將白廣恩,餘賊分竄。文詔追擊之隴州、平、鳳間。十月三戰三敗之,遂蹙賊耀州錐子山,其黨殺獨行狼、郝臨庵以降。承疇戮四百人,餘散遣。關中巨寇略平。

巡撫御史范復粹匯奏首功凡三萬六千六百有奇,文詔功第一,嘉謨次之,承恩、麒又次之。文詔在陜西,大小數十戰,功最多,承疇不為敘。巡按御史吳甡推獎甚至,復粹疏復上。兵部抑其功,卒不敘。

當是時,賊見陜兵盛,多流入山西,其魁紫金梁、混世王、姬關鎖、八大王、曹操、闖塌天、興加哈利七大部,多者萬人,少亦半之,蹂躪汾州、太原、平陽。御史張宸極言:「賊自秦中來。秦將曹文詔威名宿著,士民為之謠曰『軍中有一曹,西賊聞之心膽搖』。且嘗立功晉中,而秦賊滅且盡。宜敕令入晉協剿。」於是命陜西、山西諸將並受文詔節制。

六年正月抵霍州,敗賊汾河、盂系,追及於壽陽。巡撫許鼎臣遣謀士張宰先大軍嘗賊,賊驚潰。二月,文詔追擊之,斬混世王於碧霞村。餘黨為猛如虎逐走,遇文詔兵方山,復敗。五臺、盂、定襄、壽陽賊盡平。鼎臣命文詔軍平定,備太原東,張應昌軍汾州,備太原西。文詔連敗賊太谷、范村、榆社,太原賊幾盡。

帝以文詔功多,敕所過地多積糗糧以犒,並敕文詔速平賊。山西監視中官劉中允言文詔剿賊徐溝、盂、定襄,所司不給米,反以炮石傷士卒。帝即下御史按問。三月,賊從河內上太行,文詔大敗之澤州。賊走潞安,文詔至陽城遇賊不擊,自沁水潛師,還擊之芹地、劉村寨,斬首千餘。四月,賊屯潤城,其他部陷平順,殺知縣徐明揚。文詔至,賊走,乃夜半襲潤城,斬賊千五百。紫金梁、老回回自榆社走武鄉,過天星走高澤山,文詔皆擊敗。他賊圍涉縣,聞文詔破賊黎城,解去。

五月,帝遣中官孫茂霖為文詔內中軍。賊犯沁水,文詔大敗之,擒其魁大虎,又敗之遼城毛嶺山西。賊既屢敗,乃避文詔鋒,多流入河北。帝乃命文詔移師往討。而賊已敗鄧兵於林縣,文詔率五營軍夜襲破之。七月大敗懷慶賊柴陵村,馘其魁滾地龍,又追斬老回回於濟源。

文詔在洪洞時,與里居御史劉令譽忤。及是,令譽按河南,而四川石硅土官馬鳳儀軍敗沒於侯家莊,賴文詔馳退賊。甫解甲,與令譽,語復相失。文詔拂衣起,面叱之。令譽怒,遂以鳳儀之敗為文詔罪。部議文詔怙勝而騎,乃調之大同。

七年七月,大清兵西征插漢,還師入大同境,攻拔得勝堡、參將李全自經,遂攻圍懷仁縣及井坪堡、應州。文詔偕總督張宗衡先駐懷仁固守。八月,圍解,即移駐鎮城,挑戰敗還。已而靈丘及他屯堡多失陷,大清兵亦旋。十一月論罪,文詔、宗衡及巡撫胡沾恩並充軍邊衛。令甫下,山西巡撫吳甡薦文詔知兵善戰,請用之晉中。乃命為援剿總兵官,立功自贖。當是時,河南禍尤劇,帝已允兵部議,敕文詔馳剿河南賊。甡復抗疏力爭,請令先平晉賊,後入豫,帝不許。而文詔以甡有恩,竟取道太原,為甡所留。

會賊高加計已殲,而鳳陽告陷,遂整兵南,以八年三月會總督洪承疇於信陽。承疇大喜,即令擊賊隨州,文詔追斬賊三百八十有奇。四月,承疇次汝州。以賊盡入關中,議還顧根本。分命諸將扼要害,檄文詔入關,文詔乃馳至靈寶謁承疇。承疇以賊在商、雒,聞官兵至,必先走漢中,而大軍由潼關入,反在其後,乃令文詔由閿鄉取山路至雒南、商州,直搗賊巢,復從山陽、鎮安、洵陽馳入漢中,遏其奔軼。曰:「此行也,道路回遠,將軍甚勞苦,吾集關中兵以待將軍。」拊其背而遣之,文詔跌馬去。五月五日抵商州。賊去城三十里,營火滿山。文詔夜半率從子參將變蛟、守備鼎蛟、都司白廣恩等敗賊深林中,追至金嶺川。賊據險以千騎逆戰,變蛟大呼陷陣,諸軍並進,賊敗走。變蛟勇冠三軍,賊中聞大小曹將軍名,皆怖懾。

已而闖王、八大王諸賊犯鳳翔,趨水幵陽、隴州,文詔自漢中馳赴。賊盡向靜寧、泰安、清水、秦州間,眾且二十萬。承疇以文詔所部合張全昌、張外嘉軍止六千,眾寡不敵,告急於朝,未得命。六月,官軍遇賊亂馬川。前鋒中軍劉弘烈被執,俄副將艾萬年、柳國鎮復戰死。文詔聞之,瞋目大罵,亟詣承疇請行。承疇喜曰:「非將軍不能滅此賊。顧吾兵已分,無可策應者。將軍行,吾將由涇陽趨淳化為後勁。」文詔乃以三千人自寧州進,遇賊真寧之湫頭鎮。變蛟先登,斬首五百,追三十里,文詔率步兵繼之。賊伏數萬騎合圍,矢蝟集。賊不知為文詔也,有小卒縛急,大呼曰:「將軍救我!」賊中叛卒識之,惎賊曰:「此曹總兵也。」賊喜,圍益急。文詔左右跳蕩,手擊殺數十人,轉鬥數里。力不支,拔刀自刎死。遊擊平安以下死者二十餘人。承疇聞,拊膺大哭,帝亦痛悼,贈太子太保、左都督,賜祭葬,世蔭指揮僉事,有司建祠,春秋致祭。文詔忠勇冠時,稱明季良將第一。其死也,賊中為相慶。

弟文耀,從兄征討,數有功。河曲之戰,斬獲多。後擊賊忻州,戰死城下。詔予贈恤。從子變蛟,自有傳。

周遇吉,錦州衛人。少有勇力,好射生。後入行伍,戰輒先登,積功至京營遊擊。京營將多勛戚中官子弟,見遇吉質魯,意輕之。遇吉曰:「公等皆紈褲子,豈足當大敵。何不於無事時練膽勇,為異日用,而徒糜廩祿為!」同輩咸目笑之。

崇禎九年,都城被兵。從尚書張鳳翼數血戰有功,連進二秩,為前鋒營副將。明年冬,從孫應元等討賊河南,戰光山、固始,皆大捷。十一年班師,進秩受賚。明年秋,復出討賊,破胡可受於淅川,降其全部。楊嗣昌出師襄陽,遇吉從中官劉元斌往會。會張獻忠將至房縣,嗣昌策其必窺渡鄖灘,遣遇吉扼守槐樹關,張一龍屯光化,賊遂不敢犯。十二月,獻忠敗於興安,將走竹山、竹溪,遇吉復以嗣昌令至石花街、草店扼其要害,賊自是盡入蜀。遇吉乃從元斌駐荊門,專護獻陵。明年與孫應元等大破羅汝才於豐邑坪。又明年與黃得功追破賊鳳陽。已而旋師,敗他賊李青山於壽張,追至東平,殲滅幾盡,青山遂降。屢加太子少保、左都督。

十五年冬,山西總兵官許定國有罪論死,以遇吉代之。至則汰老弱,繕甲仗,練勇敢,一軍特精。明年十二月,李自成陷全陜,將犯山西。遇吉以沿河千餘里,賊處處可渡,分兵扼其上流,以下流蒲阪屬之巡撫蔡懋德,而請濟師於朝。朝廷遣副將熊通以二千人來赴。十七年正月,遇吉令通防河。會平陽守將陳尚智已遣使迎賊,諷通還鎮說降。遇吉叱之曰:「吾受國厚恩,寧從爾叛逆!且爾統兵二千,不能殺賊,反作說客邪!」立斬之,傳首京師。至二月七日,太原陷,懋德死之。賊遂陷忻州,圍代州。

遇吉先在代遏其北犯,乃憑城固守,而潛出兵奮擊。連數日,殺賊無算。會食盡援絕,退保寧武。賊亦踵至,大呼五日不降者屠其城。遇吉四面發大炮,殺賊萬人,火藥且盡,外圍轉急。或請甘言紿之,遇吉怒曰:「若輩何怯邪!今能勝,一軍皆忠義。即不支,縛我予賊。」於是設伏城內,出弱卒誘賊入城,亟下閘殺數千人。賊用炮攻城,圮復完者再,傷其四驍將。自成懼,欲退。其將曰:「我眾百倍於彼,但用十攻一,番進,蔑不勝矣。」自成從之。前隊死,後復繼。官軍力盡,城遂陷。遇吉巷戰,馬蹶,徒步跳蕩,手格殺數十人。身被矢如蝟,竟為賊執,大罵不屈。賊懸之高竿,叢射殺之,復臠其肉。城中士民感遇吉忠義,巷戰殺賊,不可勝計。其舍中兒,先從遇吉出鬥,死亡略盡。夫人劉氏素勇健,率婦女數十人據山巔公廨,登屋而射,每一矢斃一賊,賊不敢逼。縱火焚之,闔家盡死。

自成集眾計曰:「寧武雖破,吾將士死傷多。自此達京師,歷大同、陽和、宣府、居庸,皆有重兵。倘盡如寧武,吾部下寧有孑遺哉!不如還秦休息,圖後舉。」刻期將遁,而大同總兵姜瓖降表至,自成大喜。方宴其使者,宣府總兵王承廕表亦至,自成益喜。遂決策長驅,歷大同、宣府抵居庸。太監杜之秩、總兵唐通復開門延之,京師遂不守矣。賊每語人曰:「他鎮復有一周總兵,吾安得至此。」福王時,贈太保,謚忠武,列祀旌忠祠。

黃得功,號虎山,開原衛人,其先自合肥徙。早孤,與母徐居。少負奇氣,膽略過人。年十二,母釀酒熟,竊飲至盡。母責之,笑曰:「償易耳。」時遼事急,得功持刀雜行伍中,出斬首二級,中賞率得白金五十兩,歸奉母,曰:「兒以償酒也。」由是隸經略為親軍,累功至遊擊。

崇禎九年,遷副總兵,分管京衛管。十一年以禁軍從總督熊文燦擊賊於舞陽,鏖光、固間,最。八月又從擊賊馬光玉於淅川之吳村、王家寨,大破之。詔加太子太師,署總兵銜。十三年從太監盧九德破賊於板石畈,賊革裏眼等五營降。十四年以總兵與王憲分護鳳陽、泗州陵,得功駐定遠。張獻忠攻桐城,挾營將廖應登至城下誘降。得功與劉良佐合兵擊之於鮑家嶺,賊敗遁,追至潛山,擒斬賊將闖世王馬武、三鷂子王興國。三鷂子,獻忠養子,最號驍勇者也。得功箭傷面,愈自奮,與賊轉戰十餘日,所殺傷獨多。明年移鎮廬州。十七年封靖南伯。福王立江南,進封侯。旋命與劉良佐、劉澤清、高傑為四鎮。

初,督輔史可法慮傑跋扈難制,故置得功儀真,陰相牽制。適登萊總兵黃蜚將之任,蜚與得功同姓,稱兄弟,移書請兵備非常。得功率騎三百由揚州往高郵迎之,傑副將胡茂楨馳報傑。傑素忌得功,又疑圖己,乃伏精卒道中,邀擊之。得功行至土橋,方作食,伏起,出不意,上馬舉鐵鞭,飛矢雨集,馬踣,騰他騎馳。有驍騎舞槊直前,得功大呼,反斗,挾其槊而抶之,人馬皆糜。復殺數十人,跳入頹垣中,哮聲如雷,追者不敢進,遂疾馳至大軍,得免。方鬥時,傑潛師搗儀真,得功兵頗傷,而所俱行三百騎皆歿。遂訴於朝,願與傑決一死戰。可法命監軍萬元吉和解之,不可。會得功有母喪,可法來弔,語之曰:「土橋之役,無智愚皆知傑不義。今將軍以國故捐盛怒,而歸曲於高,是將軍收大名於天下也。」得功色稍和,終以所殺亡多為恨。可法令傑償其馬,復出千金為母賵。得功不得已,聽之。明年,傑欲趨河南,規取中原。詔得功與劉良佐守邳、徐。傑死,得功還儀真。傑家并將士妻子尚留揚州,得功謀襲之。朝廷急遣盧九德諭止,得功遂移鎮廬州。四月,左良玉東下,以請君側為名,至九江,病死,軍中立其子夢庚。命得功趨上江禦之,駐師荻港。得功破夢庚於銅陵,解其圍。命移家鎮太平,一意辦賊,論功加左柱國。

時大清兵已渡江,知福王奔,分兵襲太平。得功方收兵屯蕪湖,福王潛入其營。得功驚泣曰:「陛下死守京城,臣等猶可盡力,奈何聽奸人言,倉卒到此!且臣方對敵,安能扈駕?」王曰:「非卿無可仗者。」得功泣曰:「願效死。」得功戰荻港時,傷臂幾墮。衣葛衣,以帛絡臂,佩刀坐小舟,督麾下八總兵結束前迎敵。而劉良佐已先歸命,大呼岸上招降。得功怒叱曰:「汝乃降乎!」忽飛矢至,中其喉偏左。得功知不可為,擲刀拾所拔箭刺吭死。其妻聞之,亦自經。總兵翁之琪投江死,中軍田雄遂挾福王降。

得功粗猛不識文義。江南初立,王詔書指揮,多出群小。得功得詔紙或對使罵裂之。然忠義出天性,聞以國事相規誡者,輒屈己改不旋踵。北來太子之獄,得功抗疏爭曰:「東宮未必假冒,先帝子即上子,未有了無證明,混然雷同者。臣恐在廷諸臣,諂徇者多,抗顏者少,即明白識認,亦不敢抗詞取禍矣。」時太子真偽莫敢決,而得功忠憤不阿如此。得功每戰,飲酒數斗,酒酣氣益厲。喜持鐵鞭戰,鞭漬血沾手腕,以水濡之,久乃得脫,軍中呼為黃闖子。始為偏裨,隨大帥立功名,未嘗一當大敵。及專鎮封侯,不及一年餘而南北轉徙,主逃將潰,無所一用其力,束手就殪,與國俱亡而已。其軍行紀律嚴,下無敢犯,所至人感其德。廬州、桐城、定遠皆為立生祠。葬儀真方山母墓側。

贊曰:曹文詔等秉驍猛之資,所向摧敗,皆所稱萬人敵也。大命既傾,良將顛蹶。三人者忠勇最著,死事亦最烈,故別著於篇。


\end{pinyinscope}