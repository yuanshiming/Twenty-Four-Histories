\article{列傳第一百八十九 列女一}

\begin{pinyinscope}
婦人之行,不出於閨門,故《詩》載《關雎》、《葛覃》、《桃夭》、《芣苜》,皆處常履順,貞靜和平,而內行之修,王化之行,具可考見。其變者,《行露》、《柏舟》,一二見而已。劉向傳列女,取行事可為鑒戒,不存一操。範氏宗之,亦采才行高秀者,非獨貴節烈也。魏、隋而降,史家乃多取患難顛沛、殺身殉義之事。蓋挽近之情,忽庸行而尚奇激,國制所褒,志乘所錄,與夫里巷所稱道,流俗所震駭,胥以至奇至苦為難能。而文人墨客往往借俶儻非常之行,以發其偉麗激越跌宕可喜之思,故其傳尤遠,而其事尤著。然至性所存,倫常所係,正氣之不至於淪澌,而斯人之所以異於禽獸,載筆者宜莫之敢忽也。

明興,著為規條,巡方督學歲上其事。大者賜祠祀,次亦樹坊表,烏頭綽楔,照耀井閭,乃至僻壤下戶之女,亦能以貞白自砥。其著於實錄及郡邑志者,不下萬餘人,雖間有以文藝顯,要之節烈為多。嗚呼!何其盛也。豈非聲教所被,廉恥之分明,故名節重而蹈義勇歟!今掇其尤者,或以年次,或以類從,具著於篇,視前史殆將倍之。然而姓名湮滅者,尚不可勝計,存其什一,亦足以示勸云。

○月娥劉孝婦甄氏諸娥丁氏石氏楊氏張氏等貞女韓氏黃善聰姚孝女蔡孝女招遠孝女盧佳娘施氏吳氏畢氏石孝女湯慧信義婢妙聰徐孝女高氏孫義婦梁氏馬氏義姑萬氏陳氏郭氏幼谿女程氏王妙鳳唐貴梅張氏楊泰奴張氏陳氏秀水張氏歐陽金貞莊氏唐氏王氏易氏鐘氏四節婦宣氏孫氏徐氏義妾張氏龔烈婦江氏范氏二女丁美音成氏興安二女子章銀兒茅氏招囊猛凌氏杜氏義婦楊氏史氏林端娘汪烈婦竇妙善石門丐婦賈氏胡氏陳宗球妻史氏葉氏胡貴貞孫氏江氏嚴氏

月娥,西域人,元武昌尹職馬祿丁女也。少聰慧,聽諸兄誦說經史,輒通大義。長適蕪湖葛通甫,事上撫下,一秉禮法。長姒盧率諸婦女,悉受其教。太祖渡江之六年,偽漢兵自上游而下,盧曰:「太平有城郭,且嚴兵守,可恃。」使月娥挾諸婦女往避之。未幾,寇至,城陷,月娥歎曰:「吾生詩禮家,可失節於賊邪!」抱幼女赴水死。諸婦女相從投水者九人,方盛暑,屍七日不浮,顏色如生。鄉人為巨穴合葬之故居之南,題曰十女墓。娥弟丁鶴年,幼通經史,皆娥口授也。後通甫與盧皆死於寇。

劉孝婦,新樂韓太初妻。太初,元時為知印。洪武初,例徙和州,挈家行。劉事姑謹,姑道病,刺血和藥以進。抵和州,夫卒,劉種蔬給姑食。越二年,姑患風疾不能起,晝夜奉湯藥,驅蚊蠅不離側。姑體腐,蛆生席間,為齧蛆,蛆不復生。及姑疾篤,刲肉食之,少蘇,踰月而卒,殯之舍側。欲還葬舅塚,力不能舉喪,哀號五載。太祖聞之,遣中使賜衣一襲、鈔二十錠,命有司還其喪,旌門閭,復徭役。同時甄氏,欒城李大妻,事姑孝。姑壽九十一卒,甄廬墓三年,旦暮悲號,亦被旌。

孝女諸娥,山陰人。父士吉,洪武初為糧長。有黠而逋賦者,誣士吉於官,論死,二子炳、煥亦罹罪。娥方八歲,晝夜號哭,與舅陶山長走京師訴冤。時有令,冤者非臥釘板,勿與勘問。娥輾轉其上,幾斃,事乃聞,勘之,僅戍一兄而止。娥重傷卒,里人哀之,肖像配曹娥廟。

唐方妻,浙新昌丁氏女,名錦孥。洪武中,方為山東僉事,坐法死,妻子當沒為官婢。有司按籍取之,監護者見丁色美,借梳掠髮,丁以梳擲地,其人取掠之,持還丁。丁罵不受,謂家人曰:「此輩無禮,必辱我,非死無以全節。」肩輿過陰澤,崖峭水深,躍出赴水,衣厚不能沈,從容以手斂裙,隨流而沒,年二十八,時稱其處為夫人潭。

鄭煁妻石氏。煁,浦江鄭泳孫也。洪武初,李文忠薦諸朝,屢遷藏庫提點,坐法死。石當遣配,泣曰:「我義門婦也,可辱身以辱門乎!」不食死。

楊氏,慈谿人,字同邑鄭子琜。洪武中,子琜父仲徽戍雲南。明制,子成丁者隨遣,子琜亦在戍中。楊年甫十六,聞子琜母老弟幼,請於父母,適鄭養姑,以待子琜之返。子琜竟卒戍所,楊與姑撫諸叔成立,以夫從子孔武為嗣,苦節五十餘年。其後,鄭煥妻張氏,嫁未旬日;泰然妻嚴氏生子一蘭,方孩抱;栻妻王氏事夫癇病,狂不省人事,服勤八年弗怠;三人皆楊氏夫族,先後早寡,皆以節聞。萬曆中,知府鄒希賢題曰鄭氏節門,以比浦江鄭氏義門云。

貞女韓氏,保寧人。元末明玉珍據蜀,貞女慮見掠,偽為男子服,混迹民間。既而被驅入伍,轉戰七年,人莫知其處女也。後從玉珍破雲南還,遇其叔父贖歸成都,始改裝而行,同時從軍者莫不驚異。洪武四年嫁為尹氏婦。成都人以韓貞女稱。其後有黃善聰者,南京人。年十三失母,父販香廬、鳳間,令善聰為男子裝從遊數年。父死,善聰習其業,變姓名曰張勝。有李英者,亦販香,與為伴侶者踰年,不知其為女也。後偕返南京省其姊。姊初不之識,詰知其故,怒詈曰:「男女亂群,辱我甚矣。」拒不納。善聰以死自誓。乃呼鄰嫗察之,果處子也。相持痛哭,立為改裝。明日,英來,知為女,怏怏如失,歸告母求婚。善聰不從,曰:「若歸英,如瓜李何?」鄰里交勸,執益堅。有司聞之,助以聘,判為夫婦。

姚孝女,餘姚人,適吳氏。母出汲,虎銜之去,女追掣虎尾,虎欲前,女掣益力,尾遂脫,虎負痛躍去。負母還,藥之獲愈,奉其母二十年。後成化間,武康有蔡孝女,隨母入山採藥。虎攫其母,女折樹枝格鬥三百餘步。虎舍其母,傷女,血歕丈許,竹葉為赤,女亦獲全。後招遠有孝女,不知其姓。父採石南山,為蟒所吞。女哭之,願見父屍同死。俄頃大雷電擊蟒墮女前,腹裂見父屍。女負土掩埋,觸石而死。

盧佳娘,福清李廣妻。婚甫十月,廣暴卒,盧慟絕復蘇,見廣口鼻出惡血,悉餂食之。既殮,哭輒僵仆,積五六日,家人防懈,潛入寢室自經。後其縣有游政妻倪氏殉夫,亦然。又有施氏,滁州彭禾妻。正德元年,禾得疾不起,握手訣曰:「疾憊甚,知必死。汝無子,擇婿而嫁,毋守死,徒自苦也。」施泣曰:「君尚不知妾乎!願先君死。」禾固止之,因取禾所嘔血盡吞之,以見志。及禾歿,即自經。

吳氏,潞州廩生盧清妻。舅姑歿於臨洺,寄瘞旅次。清授徒自給,後失廩,充掾於汴,憤恥發狂死。吳聞訃,痛絕,哭曰:「吾舅姑委骨於北,良人死,忍令終不返乎!」乃寄幼孤於姊兄,鬻次女為資,獨抵臨洺,覓舅姑瘞處不得,號泣中野。忽一丈夫至,則清所授徒也,為指示,收二骸以歸。復冒暑之汴,負夫骨還。三喪畢舉,忍餓無他志。學正劉崧言於知州馬暾,贖其女,厚恤之。年七十五乃卒。後有畢氏,河間鄧節妻。年饑,攜家景州就食,舅姑相繼亡,節亦尋歿,俱槁葬景州。氏年三十三,無子女,獨歸里中,忍饑凍,晝夜紡織,積數年,市地城北八里莊,獨之景州,負舅姑及夫骨還葬。

石孝女,新昌人。襁褓時,父潛坐事籍沒,繫京獄。母吳以漏籍獲免,依兄弟為生。一日,父脫歸,匿吳家。吳兄弟懼連坐,殺置大窖中,母不敢言。及女長,問母曰:「我無父族何也?」母告之故,女大悲憤。永樂初,年十六,舅氏主婚配族子。女白母曰:「殺我父者,吳也。奈何為父仇婦?」母曰:「事非我主,奈何?」女頷而不答。嫁之日,方禮賓,女自經室中。母仰天哭曰:「吾女之死,不欲為仇人婦也。」號慟數日亦死。有司聞之,治殺潛者罪。湯慧信,上海人。通《孝經》、《列女傳》,嫁華亭鄧林。林卒,婦年二十五,一女七歲。鄧族利其居,迫使歸家,婦曰:「我鄧家婦,何歸乎?」族知不可奪,貿其居於巨室。婦泣曰:「我收夫骨於茲土,與同存亡,奈何棄之。」欲自盡,巨室義而去之。婦尋自計曰:「族利我財耳。」乃出家資,盡畀族人,躬績糸任以給。

歲大水,居荒野沮洳中。其女適人者,操舟來迎,不許。請暫憩舟中,亦不許,曰:「我守此六十年,因巨浸以從汝父,所甘心焉,復何往!」母女方相牽未捨,水至,湯竟溺死。

義婢妙聰,保安右衛指揮張孟喆家婢也。永樂中,調兵操宣府。孟喆在行。北寇入掠,妻李謂夫妹曰:「我命婦,與若皆宦門女,義不可辱。」相挈投井中,妙聰亦隨入,見二人俱未死,以李有娠,恐水冷有所害,遂負之於背。賊退,孟喆弟仲喆求三人井中,以索引嫂妹出,而婢則死矣。

徐孝女,嘉善徐遠女也。年六歲,母患臁瘡。女問母何以得愈,母謾曰:「兒吮之乃愈。」女遂請吮,母難之。女悲啼不已,母不得已聽之,吮數日,果愈。

高氏女,武邑人,適諸生陳和。和早卒,高獨持門戶,奉翁姑甚孝。及宣德時,翁姑並歿,氏以禮殯葬,時年五十矣。泣謂子剛曰:「我父,洪武間舉家客河南虞城。父死,旋葬城北,母以刺木小車輞識之。比還家,母亦死,弟懦不能自振。吾三十年不敢言者,以汝王母在堂,當朝夕侍養也。今大事已畢,欲舁吾父遺骸歸合葬。」剛唯唯,隨母至虞城,抵葬所,塚纍累不能辨。氏以髮繫馬鞍逆行,自朝及夕,至一小塚,鞍重不能前,即開其塚,所識車輞宛然。遠近觀者咸驚異,助之歸,啟母窆同葬。

孫義婦,慈谿人。歸定海黃誼昭,生子湑。未幾夫卒,孫育之成立,求兄女為配。甫三年,生二子,湑亦卒。時田賦皆令民自輸,孫姑婦相率攜幼子輸賦南京,訴尚書蹇義,言:「縣苦潮患,十年九荒,乞築海塘障之。」義見其孤苦,詰曰:「何為不嫁?」對曰:「餓死事極小,失節事極大。」義嗟歎久之,次日即為奏請,遣官偕有司相度成之,起自龍山,迄於觀海,永免潮患。慈谿人廟祀之塘上。

梁氏,大城尹之路妻。嫁歲餘,夫乏食出遊山海關,賣熟食為生。又娶馬氏,生子二,十餘年不通問。氏事翁姑,艱苦無怨言。夫客死,氏徒步行乞,迎夫喪,往返二千里,迄扶柩攜後妻二子以歸,里人嘆異。

餘人布妻馬氏,吳縣人。歸五年,夫死無子,家酷貧。姑欲奪其志,有田二畝半,得粟不以與婦,馬不為動。姑潛納他人聘,一夕鼓吹臨門,趣治妝,馬入臥室自經死,几上食器,糠籺尚存。

義姑萬氏,名義顓,字祖心,鄞人,寧波衛指揮僉事鐘女也。幼貞靜,善讀書。兩兄文、武,皆襲世職,戰死,旁無期功之親。繼母曹氏,兩嫂陳氏、吳氏,皆盛年孀居。吳遺腹僅六月,姑旦暮拜天哭告曰:「萬氏絕矣,願天賜一男,續忠臣後。我矢不嫁,共撫之。」已果生男,名之曰全。姑喜曰:「萬氏有後矣。」乃與諸嫠共守,名閥來聘,皆謝絕之,訓全讀書,迄底成立。全嗣職,傳子禧、孫椿,皆奉姑訓惟謹。姑年七十餘卒。姑之祖斌及父兄並死王事,母及二嫂守貞數十年,姑更以義著。鄉人重之,稱為四忠三節一義之門。

後有陳義姑者,沙縣陳穗女。年十八,父母相繼卒,遺二男,長七歲,次五歲。親族利其有,日眈眈於旁。姑矢志撫弟,居常置帚數十。族兄弟暮夜叩門,姑燃帚照之,亟啟戶具酒食款。叩者告曰:「吾輩夜行滅火,就求燭耳。」自此窺伺者絕意。及二弟畢婚,年四十五乃嫁,終無子。二弟迎歸,母事之。

郭氏,大田人。鄧茂七之亂,鄉人結寨東巖。寨破,郭褓幼兒走,且有身,為賊所驅。郭奮罵,投百尺巖下,與兒俱碎亂石間,胎及腸胃迸出,狼籍巖下。賊據高瞰之,皆歎曰:「真烈婦也!」瘞之去。同時有幼溪女,失其姓名。茂七破沙縣,匿草間,為二賊所獲。遇溪橋,貞女曰:「扶我過,當從一人而終。」二賊爭趨挽,至橋半,女視溪流湍急,拽二賊投水中,俱溺死。

程氏,揚州胡尚絅妻。尚絅嬰危疾,婦刲腕肉啖之,不能咽而卒。婦號慟不食二日。懷孕四月矣,或曰:「得男可延夫嗣,徒死何為?」答曰:「吾亦知之,倘生女,徒茍活數月耳。」因復食,彌月果生男。明年殤,即前語翁姑曰:「媳不能常侍奉,有娣姒在,無悲也。」復絕食,越二日其姑撫之曰:「爾父母家二百里內,若不俟面訣乎?」婦曰:「可急迎之。」日飲米沈一匙以待。逾十有二日,父母遣幼弟至,婦曰:「是可白吾志。」自是滴水不入口,徐簡DB中簪珥,令辦後事,以其餘散家人并鄰嫗嘗通問者,復自卜曰:「十八、九日皆良,吾當逝。向曾刲肉救夫,夫不可救,以灰和之置床頭,附吾左腕,以示全歸。」遂卒。

王妙鳳,吳縣人。適吳奎。姑有淫行。正統中,奎商於外。姑與所私飲,并欲污之,命妙鳳取酒,挈瓶不進。頻促之,不得已而入。姑所私戲糸其臂。妙鳳憤,拔刀斫臂不殊,再斫乃絕。父母欲訟之官,妙鳳曰:「死則死耳,豈有婦訟姑理邪?」逾旬卒。

唐貴梅者,貴池人。適同里朱姓。姑與富商私,見貴梅悅之,以金帛賄其姑,誨婦淫者,百端勿聽,加箠楚勿聽,繼以炮烙,終不聽。乃以不孝訟於官。通判某受商賂,拷之幾死者數矣。商冀其改節,復令姑保出之。親黨勸婦首實,婦曰:「若爾,妾之名幸全,如播姑之惡何?」夜易服,自經後園梅樹下。及旦姑起,且將撻之。至園中乃知其死,尸懸樹三日,顏如生。

其後,嘉靖二十三年,有嘉定張氏者,嫁汪客之子。其姑多與人私,諸惡少中有胡巖者,最桀黠,群黨皆聽其指使。於是與姑謀,遣其子入縣為卒,而巖等日夕縱飲。一日,呼婦共坐,不應。巖從後攫其梳,婦折梳擲地。頃之,巖徑入犯婦。婦大呼殺人,以杵擊巖。巖怒走出,婦自投於地,哭終夜不絕,氣息僅屬。詰旦,巖與姑恐事洩,縶諸床足守之。明日召諸惡少酣飲。二鼓共縛婦,槌斧交下。婦痛苦宛轉曰:「何不以利刃刺我。」一人乃前刺其頸,一人刺其脅,又豕其陰。舉尸欲焚之,尸重不可舉,乃火其室。鄰里救火者蹋門入,見赫然死人,驚聞於官。官逮小女奴及諸惡少鞫之,具得其實,皆以次受刑。婦死時年十九。邑故有烈婦祠,婦死前三日,祠旁人聞空中鼓樂聲,火炎炎從祠柱中出,人以為貞婦死事之征云。

楊泰奴,仁和楊得安女。許嫁未行。天順四年,母疫病不愈。泰奴三割胸肉食母,不效。一日薄幕,剖胸取肝一片,昏仆良久。及蘇,以衣裹創,手和粥以進,母遂愈。母宿有膝攣疾,亦愈。後有張氏,儀真周祥妻。姑病,醫百方不效。一方士至其門曰:「人肝可療。」張割左脅下,得膜如絮,以手探之沒腕,取肝二寸許,無少痛,作羹以進姑,病遂瘳。

陳氏,祥符人。字楊瑄,未嫁而瑄卒。女請死,父母不許,欲往哭,又不許。私剪髮,屬媒氏置瑄懷。汴俗聘女,以金書生年月日畀男家,號定婚帖。瑄母乃以帖裹其髮,置瑄懷以葬。女遂素服以居。亡何,父母謀改聘,女縊死。後五十三年,至正德中,瑄姪永康改葬瑄,求陳骨合焉。二骨朽矣,髮及定婚帖鮮完如故。葬三年,岐穀、丫瓜產墓上。

張氏,秀水人。年十四,受同邑諸生劉伯春聘。伯春負才名,必欲舉於鄉而後娶。未幾卒,女號泣絕髮,自為詩祭之。持服三年,不踰閫,不茹葷。服闋,即絕飲食,父母強諭之,終不食,旬日而卒。年二十,舅姑迎柩合葬焉。又有江夏歐陽金貞者,父梧,授《孝經》、《列女傳》。稍長,字羅欽仰,從梧之官柘城。梧艱歸,舟次儀真,欽仰墜水死。金貞年甫十四,驚哭欲赴水從之,父母持不許。又欲自縊,父母曰:「汝未嫁,何得爾?」對曰;「女自分無活理,即如父母言,願終身稱未亡人。」大聲哀號不止。及殮,剪發繫夫右臂以殉。抵家,告父母曰:「有婦,以事姑也。姑既失子,可并令無婦乎?願歸羅,以畢所事」。」父母從之。後父知廣元縣,姑病卒,女乃歸寧。有諷他適者,曰:「事姑畢矣,更何待?」女曰:「我昔殮羅郎時,有一束髮纏其手,誰能掘塚開棺,取髮還我,則易志矣。」遂止。生平獨臥一樓,年六十餘卒。

莊氏,海康吳金童妻。成化初,廣西流寇掠鄉邑,莊隨夫避新會,傭劉銘家。銘見莊美,欲犯之,屢誘不從。乃令黨梁狗同金童入海捕魚,沒水死。越三日不還,莊求之海賓,屍浮岸側,手足被縛,腫腐莫可辨。莊以衣識之,歸攜女赴水,抱夫屍而沒。翼日,三屍隨流繞銘門,去而復還。士人感異殯祭之,然莫知銘殺也,後梁狗漏言,有司並捕考,處以極刑。

唐氏,汝陽陳旺妻,隨其夫以歌舞逐食四方。正德三年秋,旺攜妻及女環兒、姪成兒至江夏九峰山。有史聰者,亦以傀儡為業。見婦、女皆艷麗,而旺且老,因紿旺至青山,夜殺之。明日,聰獨返,攜其婦、女、幼姪入武昌山吳王祠,持利刃脅唐。唐曰:「汝殺吾夫,吾不能殺汝以復仇,忍從汝亂邪?」遂遇害。賊裹以席,置荊棘中。明日,徙蓑衣園,賊又迫環兒,臨以刃。環兒哭且詈,聲振林木,賊亦殺之,瘞糞壤中而去。其年冬至,賊被酒,成兒潛出告官,擒於葛店市,伏誅。

王氏,慈谿人。聘於陳,而夫佳病,其父母娶婦以慰之。及門,即入侍湯藥。未幾,佳卒,王年甫十七,矢志不嫁。姑張氏曰:「未成禮而守,無名。」女曰:「入陳氏門,經事君子,何謂無名?」姑乃使其二女從容諷之。婦不答,截髮毀容。姑終欲強之,窘辱萬狀。二小姑陵之若婢,稍不順即爪其面,姑聞復加構楚。女口不出怨言,曰:「不逼嫁,為婢亦甘也。」夜寢處小姑床下,受濕得傴疾,私自幸曰:「我知免矣。」鞠從子梅為嗣,教之。成化初領鄉薦,卒昌其家。後有易氏,分宜人,嫁安福王世昌。時世昌已遘疾,奄奄十餘月,易事之,衣不解帶。世昌死,除喪猶縞素。姑憐之,謂:「汝猶處子,可終累乎?」跪泣曰:「是何言哉?父母許我王氏,即終身王氏婦矣。」自是獨處一樓,不窺外戶四十餘年。方世昌疾,所吐痰血,輒手一布囊盛之。卒後,用所盛囊為枕,枕之終身。

鐘氏,桐城陶鏞妻。鏞以罪被戍,卒於外。鐘年二十五,子繼甫在抱,負鏞骨四千餘里歸葬。乃斷髮杜門,年八十二以節終。繼亦早卒,妻方氏年二十七,子亮甫二歲。其兄憐之,微叩其意,方以死誓。景泰中,亮舉鄉試,業於太學,卒。妻王氏年二十八,妾吳氏二十二,皆無子,扶櫬歸葬。貧不能支,所親勸之嫁,兩人哭曰:「而不知我之為節婦婦乎!」乃共以紡績自給。越二十六年,縣令陳勉以聞,詔旌三代。人稱之曰四節里。

宣氏,嘉定張樹田妻。夫素狂悖,與宣不睦。夫病,宣晨夕奉事。及死,誓身殉。時樹田友人沈思道亦死,其婦孫與宣以死相要,各分尺帛。孫自經,或勸宣曰:「彼與夫相得,故以死報,汝何為效之?」宣歎曰:「予知盡婦道而已,安論夫之賢不賢。」卒縊死。

徐氏,慈谿人,定海金傑妻也。成化中,傑兄以罪逮入京,傑往請代。瀕行,徐已有身,傑謂曰:「予去,生死不可知,若生男善撫之,金氏鬼庶得食也。」已而悔曰:「我幾誤汝,吾去無還理,即死,善事後人。」徐泣曰:「君以義往,上必義君,君兄弟當同歸,無過苦也。即如君言,妾有死耳,敢忘付託乎?」已果生男,無何兄得還,傑竟瘐死。徐撫孤慟曰:「我本欲從汝父地下,奈金氏何?」強營葬事。服闋,父母勸他適,截髮斷指自誓,食澹茹苦六十餘年,視子孫再世成立,乃卒。

義妾張氏,南京人。松江楊玉山商南京,娶為妾。逾月以婦妒,遣之歸。張屏居自守,楊亦數往來,所贈千計。後二十餘年,楊坐役累,罄其產,怏怏失明。張聞之,直造楊廬,拜主母,捧楊袂大慟。乃悉出向所贈金珠,具裝,嫁其二女,并為二子娶婦,留侍湯藥。踰年楊死,守其柩不去。既免喪,父母強之歸,不從,矢志以歿,終身不見一人。

龔烈婦,江陰人。年十七嫁劉玉,家貧,力作養姑。姑亡,相夫營葬。夫又亡,無以為斂。里有羨婦色者,欲助以棺。龔覺其意,辭之。既又強之,龔恐無以自脫,乃以所生六歲男、三歲女寄食母家。是夜,積麥稿屋中,舉火自焚,抱夫屍死。又江氏,蒙城王可道妻。夫貧,負販糊口,死不能斂。比鄰諸生李雲蟾合錢斂之,卜日以葬。及期,率眾至其家,闃然無聲,廚下燈微明,趨視之飲食畢具,蓋以待舁棺者,婦已縊死灶旁矣。眾驚歎,復合錢并葬之。

會稽范氏二女,幼好讀書,並通《列女傳》。長適江,一月寡。次將歸傅,而夫亡。二女同守節,築高垣,圍田十畝,穿井其中,為屋三楹以居。當種獲,父啟圭竇率傭以入,餘日則塞其竇,共汲井灌田。如是者三十年。自為塋於屋後,成化中卒,竟合葬焉。族人即其田立祠以祀。

又有丁美音,漵浦丁正明女。幼受夏學程聘,年十八將嫁,學程死,美音誓不再嫁。父母曰:「未嫁守節,非禮也。何自苦如此?」美音齧指滴血,籲天自矢。當道交旌之,齎以銀幣約百金,乃構室獨居,鬻田自贍,事舅姑,養父母。鄉人名其田為貞女田。

成氏,無錫人,定陶教諭繒女,登封訓導尤輔妻也。輔游學靖江,成從焉。江水夜溢,家人倉卒升屋,成整衣欲上,問:「爾等衣邪?」眾謝不暇。成曰:「安有男女裸,而尚可俱生邪?我獨留死耳。」眾號哭請,不應。厥明,水退,坐死榻上。

後崇禎中,興安大水,漂沒廬舍。有結筏自救者,鄰里多附之。二女子附一朽木,倏沈倏浮,引筏救之,年皆十六七,問其姓氏不答。二女見筏上男子有裸者,歎曰:「吾姊妹倚木不死,冀有善地可存也,今若此,何用生為!」攜手躍入波中死。

章銀兒,蘭谿人。幼喪父,獨與母居。邑多火災,室盡毀,結茅以棲母。母方疾,鄰居又火,銀兒出視,眾呼令疾避。銀兒曰:「母疾不能動,何可獨避。」亟返入廬,欲扶母出,烈焰忽覆其廬,眾莫能救。火光中,遙見銀兒抱其母,宛轉同焚死,時弘治元年三月也。

義妹茅氏,慈谿人。年十四,父母亡,獨與兄嫂居。其兄病痿臥。值倭入縣,嫂出奔,呼與偕行。女曰:「我室女,將安之!且俱去,誰扶吾兄者!」賊至,縱火,女力扶其兄避於空室,竟被燔灼並死。

招囊猛,雲南孟璉長官司土官舍人刁派羅妻也。年二十五,夫死,守節二十八年。弘治六年九月,雲南都指揮使奏其事。帝曰:「朕以天下為家,方思勵名教以變夷俗。其有趨於禮義者,烏可不亟加獎勵。招囊猛貞節可嘉,其即令有司顯其門閭,使遠夷益知向化,無俟核報。」

張維妻凌氏,慈谿人。弘治中,維舉於鄉,卒。婦年二十五,子四歲亦卒。其兄諷之改圖,婦痛哭齧脣,噀血灑地,終身不歸寧。舅姑慰之曰:「不幸絕嗣,日計無賴,吾二人景逼矣,爾年尚遠,何以為活?」婦曰:「恥辱事重,餓死甘之。」乃出簪珥為舅納妾,果得子,喜曰:「張氏不絕,亡夫墓門且有寒食矣。」後舅病瘋,姑雙目瞽,婦紡績供養,二十年不衰。後有杜氏,貴池曹桂妻。年二十四,夫亡,遺腹生女,悲苦無計。日諷姑為舅納妾,果生一子。產後,妾死,杜以己女託於族母,而自乳其叔。逾年翁喪,勸者曰:「汝辛苦撫孤,寧能以叔後汝乎?」杜曰:「叔後吾翁,異日生二子,即以一子後我夫,吾志畢矣。」後卒如其言。

義婦楊氏,王世昌妻,臨漳人。弘治中,世昌兄坐事論死。世昌念兄為嫡子,請代其刑。時楊未笄,謀於父母宗族曰:「彼代兄死為義士,我顧不能為義婦邪?願訴於上代夫死。」遂入京陳情,敕法司議,夫妻並得釋。

史氏,杞縣人。字孔弘業,未嫁而夫卒。欲往殉之,母不許。女七日不食,母持茗逼之飲,雙蛾適墮杯中死,女指示曰:「物意尚孚我心,母獨不諒人邪!」母知不可奪,翌日製素衣縞裳,送之孔氏。及暮,辭舅姑,整衣自經死。白氣縷縷勝屋上,達旦始消。又有林端娘者,甌寧人,字陳廷策。聞廷策訃,寄聲曰:「勿殮,吾將就死。」父曰:「而雖許字,未納幣也。」對曰:「既詐矣,何幣之問?」父謹防之。曰:「女奚所不可死,顧死夫家韙耳。」父曰:「婿家貧,無以周身。」曰:「身非所恤。」又曰:「婿家貧,孰為標名?」曰:「名非所求。」遂往哭奠畢,自剋死期,理帛自經,三拱而絕。陳故家青陽山下,山下人言婦將盡時,山鳴三晝夜。

汪烈婦,晉江諸生楊希閔妻也。年二十三,夫死,無子,欲自經。家人防之謹,不得間。氏聞茉莉有毒能殺人,多方求之,家人不知也,日供數百朵。踰月,家人為亡者齋祭,婦自撰祭文,辭甚悲。夜五鼓,防者稍懈,取所積花煎飲之,天明死。

竇妙善,京師崇文坊人。年十五,為工部主事餘姚姜榮妾。正德中,榮以瑞州通判攝府事。華林賊起,寇瑞,榮出走。賊入城,執其妻及婢數人,問榮所在。時妙善居別室,急取府印,開後窗投荷池。衣鮮衣前曰:「太守統援兵數千,出東門捕爾等,旦夕授首,安得執吾婢?」賊意其夫人也,解前所執數人,獨輿妙善出城。適所驅隸中,有盛豹者父子被掠,其子叩頭乞縱父,賊許之。妙善曰:「是有力,當以舁我,何得遽縱。」賊從之。行數里,妙善視前後無賊,低語豹曰:「我所以留汝者,以太守不知印處,欲藉汝告之。今當令汝歸,幸語太守,自此前行遇井,即畢命矣。」呼賊曰:「是人不善舁,可仍縱之,易善舁者。」賊又從之。行至花塢遇井,妙善曰:「吾渴不可忍,可汲水置井傍,吾將飲。」賊如其言,妙善至井傍,跳身以入,賊驚救不得而去。豹入城告榮取印,引至花塢,覓井,果得妙善屍。越七年,郡縣上其事,詔建特祠,賜額貞烈。

石門丐婦,湖州人,莫詳其姓氏。正德中,湖大飢,婦隨其夫及姑走崇德石門市乞食。三人偶相失。婦有色,市人爭挑之。與之食不顧,誘之財亦不顧。寓東高橋上,不復乞食者二日。伺夫與姑皆不至,聚觀者益眾,婦乃從橋上躍入水中死。

賈氏,慶雲諸生陳俞妻。正德六年,兵變,值舅病卒,家人挽之避,痛哭曰:「舅尚未斂,婦何惜一死。」身服斬衰不解。兵至,縱火迫之出,罵不絕口,刃及身無完膚,與舅屍同燼。年二十五。

鄞縣諸生李珂妻胡氏,年十八歸珂。閱七年,珂死,遺男女各一,胡誓不踰閾。鄰火作,珂兄珮往救之,曰:「阿姆來,吾乃出。」珮使妻陳往,婦以七歲男自牖付之,屬曰:「幸念吾夫,善視之。」陳曰:「嬸將何如?」紿之曰:「取少首飾即出。」陳去,胡即累衣箱塞戶,抱三歲女端坐火中死。

陳宗球妻史氏,南安人。夫死將殉有期矣,尚為姑釀酒。姑曰:「婦已決死,生存豈多日,何辛苦為?」曰:「政為日短,故釀而奉姑。」將死,告舅曰:「婦有喪,幸毋髹棺。」遂縊。

葉氏,定海人。許聘慈谿翁姓,而父母俱歿,遂育於翁。年十四,翁資產日落,且失其姑,舅待之如奴,勞勩萬狀,略無怨色。舅以子幼,欲鬻之羅姓者,葉恚曰:「我非貨也,何輾轉貿易為?」日哽咽垂涕。既知不可免,偽為喜色,舅遂寬之。夜月上,紿諸姒曰:「月色甚佳,盍少猶夷乎?」趨門外良久。諸姒並勸曰:「夜既半矣,盍就寢。」遂入,及晨覓之,則氏已浮屍於河矣,起之色如生。

胡貴貞,樂平人。生時,父母欲不舉,其鄰曾媼救之歸,與子天福同乳,欲俟其長而配焉。天福年十八,父母繼亡,家甚落。貴貞父將奪以姻富家,女曰:「我鞠於曾,婦於曾,分姑媳,恩母子,可以飢寒棄之邪?」乃依從姑以居,蓽舍單淺,外人未嘗識其面。其兄乘天福未婚,曳以歸,出視求聘者金寶笄飾。女知不免,潛入房縊死。

孫氏,吳縣衛廷珪妻。隨夫商販,寓潯陽小江口。寧王陷九江,廷珪適他往,所親急邀孫共逃。孫謂兩女金蓮、玉蓮曰:「我輩異鄉人,汝父不在,逃將安之?今賊已劫鄰家矣,奈何?」女曰:「生死不相離,要當為父全此身耳。」於是母子共一長繩自束,赴河死。

江氏,餘干夏璞妻。正德間,賊至,抱方晬弟走,不得脫。賊將縛之,曰:「誠願與將軍俱,顧吾父年老,惟一弟,幸得全之。」賊以為信,縱令置所抱兒,出遂大聲罵賊,投橋下死。

後隆慶中,有高明嚴氏,賊掠其境,隨兄出避,遇賊,刃及其兄。女跪泣曰:「父早喪,孀母堅守,恃此一兄,殺之則祀殄矣,請以身代。」賊憫然為納刃。既而欲污之,則曰:「請釋吾兄即配汝。」及兄去,執不從,竟剖腹而死。


\end{pinyinscope}