\article{列傳第一百七十六 文苑四}

\begin{pinyinscope}
○李維楨郝敬徐渭屠隆王穉登俞允文王叔承瞿九思唐時升婁堅李流芳程嘉燧焦竑黃輝陳仁錫董其昌莫如忠邢侗米萬鐘袁宏道鐘惺譚元春王惟儉李日華曹學佺曾異撰王志堅艾南英章世純羅萬藻陳際泰張溥張採

李維楨,字本寧,京山人。父裕,福建布政使。維楨舉隆慶二年進士,由庶吉士授編修。萬曆時,《穆宗實錄》成,進修撰。出為陜西右參議,遷提學副使。浮沉外僚,幾三十年。天啟初,以布政使家居,年七十餘矣。會朝議登用耆舊,召為南京太僕卿,旋改太常,未赴。聞諫官有言,辭不就。時方修《神宗實錄》,給事中薛大中特疏薦之,未及用。四年四月,太常卿董其昌復薦之,乃召為禮部右侍郎,甫三月進尚書,並在南京。維楨緣史事起用,乃館中諸臣憚其以前輩壓己,不令入館,但超遷其官。維楨亦以年衰,明年正月力乞骸骨去。又明年卒於家,年八十。崇禎時,贈太子太保。

維楨弱冠登朝,博聞強記,與同館許國齊名。館中為之語曰:「記不得,問老許;做不得,問小李。」維楨為人,樂易闊達,賓客雜進。其文章,弘肆有才氣,海內請求者無虛日,能屈曲以副其所望。碑版之文,照耀四裔。門下士招富人大賈,受取金錢,代為請乞,亦應之無倦,負重名垂四十年。然文多率意應酬,品格不能高也。

邑人郝敬,字仲輿。父承健,舉於鄉,官肅寧知縣。敬幼稱神童,性跅弛,嘗殺人繫獄。維楨,其父執也,援出之,館於家。始折節讀書,舉萬曆十七年進士。歷知縉雲、永嘉二縣,並有能聲。征授禮科給事中,乞假歸養。久之,補戶科,數有所論奏。

山東稅監陳增貪橫,為益都知縣吳宗堯所奏,帝不罪。敬上言:「開採不罷,則陛下明旨不過為愚弄臣民之虛文。乞先停止,然後以宗堯所奏下撫按勘核,正增不法之罪。」不聽。頃之,山東巡撫尹應元亦極論增罪,帝怒,切責應元,斥完堯為民。敬再上言:「陛下處陳增一事,甚失眾心。」帝怒,奪俸一年。帝遣中官高寀榷稅京口,暨祿榷稅儀真,敬復力諫。宗堯之劾增也,增怒甚,誣訐其贓私,詞連青州一府官僚,旁引商民吳時奉等,請皆籍沒,帝輒可之。敬復力詆增,乞速寢其奏,亦不納。坐事,謫知江陰縣。貪污不檢,物論皆不予,遂投劾歸,杜門著書。崇禎十二年卒。

徐渭,字文長,山陰人。十餘歲仿揚雄《解嘲》作《釋毀》,長師同里季本。為諸生,有盛名。總督胡宗憲招致幕府,與歙餘寅、鄞沈明臣同憲書記。宗憲得白鹿,將獻諸朝,令渭草表,並他客草寄所善學士,擇其尤上之。學士以渭表進,世宗大悅,益寵異宗憲,宗憲以是益重渭。宗憲嘗宴將吏於爛柯山,酒酣樂作,明臣作《鐃歌》十章,中有云「狹巷短兵相接處,殺人如草不聞聲」。宗憲起,捋其鬚曰:「何物沈生,雄快乃爾!」即命刻於石,寵禮與渭埒。督府勢嚴重,將吏莫敢仰視。渭角巾布衣,長揖縱談。幕中有急需,夜深開戟門以待。渭或醉不至,宗憲顧善之。寅、明臣亦頗負崖岸,以侃直見禮。

渭知兵,好奇計,宗憲擒徐海,誘王直,皆預其謀。藉宗憲勢,頗橫。及宗憲下獄,渭懼禍,遂發狂,引巨錐剚耳,深數寸,又以椎碎腎囊,皆不死。已,又擊殺繼妻,論死繫獄,里人張元忭力救得免。乃游金陵,抵宣、遼,縱觀諸邊厄塞,善李成梁諸子。入京師,主元忭。元忭導以禮法,渭不能從,久之怒而去。後元忭卒,白衣往弔,撫棺慟哭,不告姓名去。

渭天才超軼,詩文絕出倫輩。善草書,工寫花草竹石。嘗自言:「吾書第一,詩次之,文次之,畫又次之。」當嘉靖時,王、李倡七子社,謝榛以布衣被擯。渭憤其以軒冕壓韋布,誓不入二人黨。後二十年,公安袁宏道游越中,得渭殘帙以示祭酒陶望齡,相與激賞,刻其集行世。

寅,字仲房。明臣,字嘉則。皆有詩名。

屠隆者,字長卿,明臣同邑人也。生有異才,嘗學詩於明臣,落筆數千言立就。族人大山、里人張時徹方為貴官,共相延譽,名大噪。舉萬曆五年進士,除潁上知縣,調繁青浦。時招名士飲酒賦詩,游九峰、三泖,以仙令自許,然於吏事不廢,士民皆愛戴之。遷禮部主事。

西寧侯宋世恩兄事隆,宴游甚歡。刑部主事俞顯卿者,險人也,嘗為隆所詆,心恨之。訐隆與世恩淫縱,詞連禮部尚書陳經邦。隆等上疏自理,並列顯卿挾仇誣陷狀。所司乃兩黜之,而停世恩俸半歲。隆歸,道青浦,父老為斂田千畝,請徙居。隆不許,歡飲三日謝去。

歸益縱情詩酒,好賓客,賣文為活。詩文率不經意,一揮數紙。嘗戲命兩人對案拈二題,各賦百韻,咄嗟之間二章並就。又與人對弈,口誦詩文,命人書之,書不逮誦也。

子婦沈氏,修撰懋學女,與隆女瑤瑟並能詩。隆有所作,兩人輒和之。兩家兄弟合刻其詩,曰《留香草》。

王穉登,字伯穀,長洲人。四歲能屬對,六歲善擘窠大字,十歲能詩,長益駿發有盛名。嘉靖末,遊京師,客大學士袁煒家。煒試諸吉士紫牡丹詩,不稱意。命穉登為之,有警句。煒召數諸吉士曰:「君輩職文章,能得王秀才一句耶?」將薦之朝,不果。隆慶初,復遊京師,徐階當國,頗修憾於煒。或勸穉登弗名袁公客,不從,刻《燕市》、《客越》二集,備書其事。

吳中自文徵明後,風雅無定屬。穉登嘗及徵明門,遙接其風,主詞翰之席者三十餘年。嘉、隆、萬曆間,布衣、山人以詩名者十數,俞允文、王叔承、沈明臣輩尤為世所稱,然聲華烜赫,穉登為最。申時行以元老里居,特相推重。王世貞與同郡友善,顧不甚推之。及世貞歿,其仲子士肅坐事繫獄,穉登為傾身救援,人以是重其風義。萬曆中,詔修國史,大學士趙志皋輩薦穉登及其同邑魏學禮、江都陸弼、黃岡王一鳴。有詔徵用,未上,而史局罷。卒年七十餘。子留,字亦房,亦以詩名。

俞允文,字仲蔚,崑山人。其父舉進士,官大理評事。允文年十五為《馬鞍山賦》,援據該博。年未四十,謝去諸生,專力於詩文書法。與王世貞善,而不喜李攀龍詩,其持論不茍同如此。

王叔承,字承父,吳江人。少孤,治經生業,以好古謝去。貧,贅婦家,為婦翁所遂,不予一錢,乃攜婦歸奉母,貧益甚。入都,客大學士李春芳所。性嗜酒,春芳有所撰述,覓之,往往臥酒樓,欠伸弗肯應。久之,乃謝歸。太倉王錫爵,其布衣交也。再召,會有三王並封之議,叔承遺書數千言,謂當引大義以去就力爭,不當依違兩端,負主恩,辜物望。錫爵得書歎服。其詩,極為世貞兄弟所許。卒於萬曆中。

瞿九思,字睿夫,黃梅人。父晟,嘉靖三十二年進士。歷官廣平知府。鑿長渠三百里,引水為四閘,得田數十萬畝。卒於官。九思十歲從父宦吉安,事羅洪先。十五作《定志論》。後從同郡耿定向游,學益進。舉萬曆元年鄉試。居二年,縣令張維翰違制苛派,民聚毆之,維翰坐九思倡亂。巡按御史向程劾維翰激變。吏部尚書張瀚言御史議非是,九思遂長流塞下。子甲,年十三,為書數千言,歷抵公卿,訟父冤。甲弟罕,亦伏闕上書求宥。屠隆作《訟瞿生書》,遍告中外,馮夢禎亦白於楚中當事,而張居正故才九思,乃獲釋歸。三十七年,以撫按疏薦,授翰林待詔,力辭不受。詔有司歲給米六十石,終其身。乃撰《樂章》及《萬歷武功錄》,遣罕詣闕上之。卒年七十一。九思學極奧博,其文章不雅馴,然一時嗜古篤志之士亦鮮其儔。甲,字釋之,年十九舉於鄉,早卒。罕,字曰有,七歲能文。白父冤時,往返徒步,不避寒餒,天下稱雙孝。崇禎時,辟舉知州。

唐時升,字叔達,嘉定人。父欽訓,與歸有光善,故時升早登有光之門。年未三十,謝舉子業,專意古學。王世貞官南都,延之邸舍,與辨晰疑義。時升自以出歸氏門,不肯復稱王氏弟子。及王錫爵枋國,其子衡邀時升入都,值塞上用兵,逆斷其情形虛實,將帥勝負,無一爽者。家貧,好施予,灌園藝蔬,蕭然自得。詩援筆成,不加點竄,文得有光之傳。與里人婁堅、程嘉燧並稱曰「練川三老」。卒於崇禎九年,年八十有六。

婁堅,字子柔。幼好學,其師友皆出有光門。堅學有師承,經明行修,鄉里推為大師。貢於國學,不仕而歸。工書法,詩亦清新。四明謝三賓知縣事,合時升、堅、嘉燧及李流芳詩刻之,曰《嘉定四先生集》。

流芳,字長蘅,萬歷三十四年舉於鄉。工詩善書,尤精繪事。天啟初,會試北上,抵近郊聞警,賦詩而返,遂絕意進取。

程嘉燧,字孟陽,休寧人,僑居嘉定。工詩善畫。與通州顧養謙善。友人勸詣之,乃渡江寓古寺,與酒人歡飲三日夜,賦《詠古》五章,不見養謙而返。崇禎中,常熟錢謙益以侍郎罷歸,築耦耕堂,邀嘉燧讀書其中。閱十年返休寧,遂卒,年七十有九。謙益最重其詩,稱曰松圓詩老。

焦竑,字弱侯,江寧人。為諸生,有盛名。從督學御史耿定向學,復質疑於羅汝芳。舉嘉靖四十三年鄉試,下第還。定向遴十四郡名士讀書崇正書院,以竑為之長。及定向里居,復往從之。萬曆十七年,始以殿試第一人官翰林修撰,益討習國朝典章。二十二年,大學士陳于陛建議修國史,欲竑專領其事,竑遜謝,乃先撰《經籍志》,其他率無所撰,館亦竟罷。翰林教小內侍書者,眾視為具文,竑獨曰:「此曹他日在帝左右,安得忽之。」取古奄人善惡,時與論說。

皇長子出閣,竑為講官。故事,講官進講罕有問者。竑講畢,徐曰:「博學審問,功用維均,敷陳或未盡,惟殿下賜明問。」皇長子稱善,然無所質難也。一日,竑復進曰:「殿下言不易發,得毋諱其誤耶?解則有誤,問復何誤?古人不恥下問,願以為法。」皇長子復稱善,亦竟無所問。竑乃與同列謀先啟其端,適講《舜典》,竑舉「稽於眾,舍己從人」為問。皇長子曰:「稽者,考也。考集眾思,然後舍己之短,從人之長。」又一日,舉「上帝降衷,若有恒性」。皇長子曰:「此無他,即天命之謂性也。」時方十三齡,答問無滯,竑亦竭誠啟迪。嘗講次,群鳥飛鳴,皇長子仰視,竑輟講肅立。皇長子斂容聽,乃復講如初。竑嘗採古儲君事可為法戒者為《養正圖說》,擬進之。同官郭正域輩惡其不相聞,目為賈譽,竑遂止。竑既負重名,性復疏直,時事有不可,輒形之言論,政府亦惡之,張位尤甚。二十五年主順天鄉試,舉子曹蕃等九人文多險誕語,竑被劾,謫福寧州同知。歲餘大計,復鐫秩,竑遂不出。

竑博極群書,自經史至稗官、雜說,無不淹貫。善為古文,典正馴雅,卓然名家。集名《澹園》,竑所自號也。講學以汝芳為宗,而善定向兄弟及李贄,時頗以禪學譏之。萬曆四十八年卒,年八十。熹宗時,以先朝講讀恩,復官,贈諭德,賜祭廕子。福王時,追謚文端。子潤生,見《忠義傳》。

黃輝,字平倩,一字昭素,南充人。竑同年進士。幼穎異,父子元,官湖廣,御史屬訊疑獄,輝檢律如老吏。御史聞而異之,命負以至,授錢穀集,一覽輒記。稍長,博極群書。年十五舉鄉試第一。久之,成進士,改庶吉士。館課文字多沿襲熟爛,目為翰林體,及李攀龍、王世貞之學行,則又改而從之。輝刻意學古,一以韓、歐為師,館閣文稍變。時同館中,詩文推陶望齡,書畫推董其昌,輝詩及書與齊名。至徵事,輝十得八九,竑以閎雅名,亦自遜不如也。

由編修遷右中允,充皇長子講官。時帝寵鄭貴妃,疏皇后、長子,長子生母王恭妃幾殆。輝從內豎徵知其狀,謂同里給事中王德完曰:「此國家大事,旦夕不測,書之史冊,謂朝廷無人,吾輩為萬世僇矣。」德完奮然,屬輝具草上之,下獄,廷杖瀕死。輝周旋橐饘,不避險阻,人或危之。輝曰:「吾陷人於禍,可坐視乎?」輝雅好禪學,多方外交,為言者所論。時已為庶子掌司經局,遂請告歸。已,起故官,擢少詹事兼侍讀學士,卒官。

陳仁錫,字明卿,長洲人。父允堅,進士。歷知諸暨、崇德二縣。仁錫年十九,舉萬曆二十五年鄉試。聞武進錢一本善《易》,往師之,得其指要。久不第。益究心經史之學,多所論著。天啟二年以殿試第三人授翰林編修。時第一為文震孟,亦老成宿學。海內咸慶得人。明年丁內艱,廬墓次。服闋,起故官,尋直經筵,典誥敕。魏忠賢冒邊功,矯旨錫上公爵,給世券。仁錫當視草,持不可,其黨以威劫之,毅然曰:「世自有視草者,何必我!」忠賢聞之怒。不數日,里人孫文豸以誦《步天歌》見捕,坐妖言鍛煉成獄,詞連仁錫及震孟,罪將不測。有密救者,得削籍歸。崇禎改元,召復故官。旋進右中允,署國子司業事,再直經筵。以預修神、光二朝實錄,進右諭德,乞假歸。越三年,即家起南京國子祭酒,甫拜命,得疾卒。福王時,贈詹事,謚文莊。仁錫講求經濟,有志天下事,性好學,喜著書,一時館閣中博洽者鮮其儔云。

董其昌,字玄宰,松江華亭人。舉萬曆十七年進士,改庶吉士。禮部侍郎田一俊以教習卒官,其昌請假,走數千里,護其喪歸葬。遷授編修。皇長子出閣,充講官,因事啟沃,皇長子每目屬之。坐失執政意,出為湖廣副使,移疾歸。起故官,督湖廣學政,不徇請囑,為勢家所怨,嗾生儒數百人鼓噪,毀其公署。其昌即拜疏求去,帝不許,而令所司按治,其昌卒謝事旭。起山東副使、登萊兵備、河南參政,並不赴。

光宗立,問:「舊講官董先生安在?」乃召為太常少卿,掌國子司業事。天啟二年擢本寺卿,兼侍讀學士。時修《神宗實錄》,命往南方採輯先朝章疏及遺事,其昌慶搜博徵,錄成三百本。又採留中之疏切於國本、籓封、人才、風俗、河渠、食貨、吏治、邊防者,別為四十卷。仿史贊之例,每篇系以筆斷。書成表進,有詔褒美,宣付史館。明年秋,擢禮部右侍郎,協理詹事府事,尋轉左侍郎。五年正月拜南京禮部尚書。時政在奄豎,黨禍酷烈。其昌深自引遠,踰年請告歸。崇禎四年起故官,掌詹事府事。居三年,屢疏乞休,詔加太子太保致仕。又二年卒,年八十有三。贈太子太傳。福王時,謚文敏。

其昌天才俊逸,少負重名。初,華亭自沈度、沈粲以後,南安知府張弼、詹事陸深、布政莫如忠及子是龍皆以善書稱。其昌後出,超越諸家,始以宋米芾為宗。後自成一家,名聞外國。其畫集宋、元諸家之長,行以己意,灑灑生動,非人力所及也。四方金石之刻,得其制作手書,以為二絕。造請無虛日,尺素短札,流布人間,爭購寶之。精於品題,收藏家得片語隻字以為重。性和易,通禪理,蕭閒吐納,終日無俗語。人儗之米芾、趙孟頫云。同時以善書名者,臨邑刑侗、順天米萬鐘、晉江張瑞圖,時人謂刑、張、米、董,又曰南董、北米。然三人者,不逮其昌遠甚。

莫如忠,字子良。嘉靖十七年進士。累官浙江布政使。潔修自好。夏言死,經紀其喪。善草書,詩文有體要。是龍,字雲卿,後以字行,更字廷韓。十歲能文,長善書。皇甫汸、王世貞輩亟稱之。以貢生終。刑侗,字子愿。萬曆二年進士。終陜西行太僕卿。家資鉅萬,築來禽館於古犁丘,減產奉客,遂致中落。妹慈靜,善仿兄書。米萬鐘,字友石。萬曆二十三年進士。歷官江西按察使。天啟五年,魏忠賢黨倪文煥劾之,遂削籍。崇禎初,起太僕少卿,卒官。張瑞圖者,官至大學士,逆案中人也。

袁宏道,字中郎,公安人。與兄宗道、弟中道並有才名,時稱「三袁」。宗道,字伯修。萬曆十四年會試第一。授庶吉士,進編修,卒官右庶子。泰昌時,追錄光宗講官,贈禮部右侍郎。

宏道年十六為諸生,即結社城南,為之長。閑為詩歌古文,有聲里中。舉萬歷二十年進士。歸家,下帷讀書,詩文主妙悟。選吳縣知縣,聽斷敏決,公庭鮮事。與士大夫談說詩文,以風雅自命。已而解官去。起授順天教授,歷國子助教、禮部主事,謝病歸。久之,起故官。尋以清望擢吏部驗封主事,改文選。尋移考功員外郎,立歲終考察群吏法,言:「外官三歲一察,京官六歲,武官五歲,此曹安得獨免?」疏上,報可,遂為定制。遷稽勳郎中,後謝病歸,數月卒。

中道,字小修。十餘歲,作《黃山》、《雪》二賦,五千餘言。長益豪邁,從兩兄宦游京師,多交四方名士,足跡半天下。萬曆三十一年始舉於鄉。又十四年乃成進士。由徽州教授,歷國子博士、南京禮部主事。天啟四年進南京吏部郎中,卒於官。

先是,王、李之學盛行,袁氏兄弟獨心非之。宗道在館中,與同館黃輝力排其說。於唐好白樂天,於宋好蘇軾,名其齋曰白蘇。至宏道,益矯以清新輕俊,學者多舍王、李而從之,目為公安體。然戲謔嘲笑,間雜俚語,空疏者便之。其後,王、李風漸息,而鐘、譚之說大熾。鐘、譚者,鐘惺、譚元春也。

惺,字伯敬,竟陵人。萬曆三十八年進士。授行人,稍遷工部主事,尋改南京禮部,進郎中。擢福建提學僉事,以父憂歸,卒於家。惺貌寢,羸不勝衣,為人嚴冷,不喜接俗客,由此得謝人事。官南都,僦秦淮水閣讀史,恒至丙夜,有所見即筆之,名曰《史懷》。晚逃於禪以卒。

自宏道矯王、李詩之弊,倡以清真,惺復矯其弊,變而為幽深孤峭。與同里譚元春評選唐人之詩為《唐詩歸》,又評選隋以前詩為《古詩歸》。鐘、譚之名滿天下,謂之竟陵體。然兩人學不甚富,其識解多僻,大為通人所譏。元春,字友夏,名輩後於惺,以《詩歸》故,與齊名。至天啟七年始舉鄉試第一,惺已前卒矣。

王惟儉,字損仲,祥符人。萬歷二十三年進士。授濰縣知縣,遷兵部職方主事。三十年春,遼東總兵官馬林以忤稅使高淮被逮,兵部尚書田樂等救之。帝怒,責職方不推代者,空司而逐,惟儉亦削籍歸。家居二十年,光宗立,起光祿丞。三遷大理少卿。

天啟三年八月擢右僉都御史,巡撫山東。值徐鴻儒之亂,民多逃亡,遼人避難來者,亦多失所,惟儉加意綏輯。五年三月擢南京兵部右侍郎,未赴。入為工部右侍郎,魏忠賢黨御史田景新劾之,落職閒住。

惟儉資敏嗜學。初被廢,肆力經史百家。苦《宋史》繁蕪,手刪定,自為一書。好書畫古玩。萬曆、天啟間,世所稱博物君子,惟儉與董其昌並,而嘉興李日華亞之。日華,字君實,嘉興人。萬歷二十年進士。官至太僕少卿。恬澹和易,與物無忤。惟儉則口多微詞,好抨擊道學,人不能堪。嘗與時輩宴集,征《漢書》一事,具悉本末,指其腹笑曰:「名下寧有虛士乎!」其自喜如此。

曹學牷,字能始,侯官人。弱冠舉萬曆二十三年進士,授戶部主事。中察典,調南京添注大理左寺正。居冗散七年,肆力於學。累遷南京戶部郎中,四川右參政、按察使。蜀府毀於火,估修資七十萬金,學牷以《宗籓條例》卻之。又中察典,議調。天啟二年起廣西右參議。初,梃擊獄興,劉廷元輩主瘋顛。學牷著《野史紀略》,直書事本末。至六年秋,學牷遷陜西副使,未行,而廷元附魏忠賢大幸,乃劾學牷私撰野史,淆亂國章,遂削籍,毀所鏤板。巡按御史王政新,以嘗薦學牷,亦勒閒住。廣西大吏揣學牷必得重禍,羈留以待。已,知忠賢無意殺之,乃得釋還。崇禎初,起廣西副使,力辭不就。

家居二十年,著書所居石倉園中,為《石倉十二代詩選》,盛行於世。嘗謂「二氏有藏,吾儒何獨無」,欲修儒藏與鼎立。采擷四庫書,因類分輯,十有餘年,功未及竣,兩京繼覆。唐王立於閩中,起授太常卿。尋遷禮部右侍郎兼侍講學士,進尚書,加太子太保。及事敗,走入山中,投繯而死,年七十有四。詩文甚富,總名《石倉集》。萬歷中,閩中文風頗盛,自學牷倡之,晚年更以殉節著云。

其同邑後起者,曾異撰,字弗人,晉江人,家侯官。父為諸生,早卒。母張氏,以遺腹生。家CI甚,紡績給晨夕。異撰起孤童,事母至孝。歲饑,採薯葉雜糠籺食之,母妻嘗負畚鋤乾草給爨。然性介甚,長吏知其貧,欲為地,不屑也。吳興潘曾紘督學政,上其母節行,獲旌於朝。及曾紘巡撫南、贛,得王惟儉所撰《宋史》,招異撰及新建徐世溥更定,未成而罷。異撰久為諸生,究心經世學,所為詩,有奇氣。崇禎十二年舉鄉試,年四十有九矣,再赴會試還,遂卒。

王志堅,字弱生,崑山人。父臨亨,進士。杭州知府。志堅舉萬歷三十八年進士,授南京兵部主事,遷員外郎、郎中。暇日要同舍郎為讀史社,撰《讀史商語》。遷貴州提學僉事,不赴,乞侍養歸。天啟二年起督浙江驛傳,奔母喪歸。崇禎四年復以僉事督湖廣學政,禮部推為學政第一。六年卒於官。

志堅少與李流芳同學,為詩文,法唐、宋名家。通籍後,卜居吳門古南園,杜門卻掃,肆志讀書,先經後史,先史後子、集。其讀經,先箋疏而後辨論。讀史,先證據而後發明。讀子,則謂唐、宋而後無子,當取說家之有裨經史者補之。讀集,則定秦、漢以後古文為五編,考核唐、宋碑志,援史傳,捃雜說,以參核其事之同異、文之純駁。其於內典,亦深辨性相之宗。作詩甚富,自選止七十餘首。

弟志長,字平仲,舉於鄉,亦深於經學。

艾南英,字千子,東鄉人。七歲作《竹林七賢論》。長為諸生,好學無所不窺。萬歷末,場屋文腐爛,南英深疾之,與同郡章世純、羅萬藻、陳際泰以興起斯文為任,乃刻四人所作行之世。世人翕然歸之,稱為章、羅、陳、艾。天啟四年,南英始舉於鄉。座主檢討丁乾學、給事中郝土膏發策詆魏忠賢,南英對策亦有譏刺語。忠賢怒,削考官籍,南英亦停三科。

莊烈帝即位,詔許會試。久之,卒不第,而文日有名。負氣陵物,人多憚其口。始王、李之學大行,天下談古文者悉宗之,後鐘、譚出而一變。至是錢謙益負重名於詞林,痛相糾駁。南英和之,排詆王、李不遺餘力。兩京繼覆,江西郡縣盡失,南英乃入閩。唐王召見,陳十可憂疏,授兵部主事,尋改御史。明年八月卒於延平。

章世純,字大力,臨川人。博聞強記。舉天啟元年鄉試。崇禎中,累官柳州知府,年已七十矣,聞京師變,悲憤,遘疾卒。

羅萬藻,字文止,世純同縣人。天啟七年舉於鄉。崇禎中行保舉法,祭酒倪元璐以萬藻應詔,辭不就。福王時為上杭知縣。唐王立於閩,擢禮部主事。南英卒,哭而殯之,居數月亦卒。

陳際泰,字大士,亦臨川人,父流寓汀州武平,生於其地。家貧,不能從師,又無書,時取旁舍兒書,屏人竊誦。從外兄所獲《書經》,四角已漫滅,且無句讀,自以意識別之,遂通其義。十歲,於外家藥籠中見《詩經》,取而疾走。父見之,怒,督往田,則攜至田所,踞高阜而哦,遂畢身不忘。久之,返臨川,與南英輩以時文名天下。其為文,敏甚,一日可二三十首,先後所作至萬首,經生舉業之富,無若際泰者。崇禎三年舉於鄉。又四年成進士,年六十有八矣。又三年除行人。居四年,護故相蔡國用喪南行,卒於道。

張溥,字天如,太倉人。伯父輔之,南京工部尚書。溥幼嗜學。所讀書必手鈔,鈔已朗誦一過,即焚之,又鈔,如是者六七始已。右手握管處,指掌成繭。冬日手皸,日沃湯數次。後名讀書之齋曰「七錄」,以此也。與同里張采共學齊名,號「婁東二張」。

崇禎元年以選貢生入都,採方成進士,兩人名徹都下。已而採官臨川。溥歸,集郡中名士相與復古學,名其文社日復社。四年成進士,改庶吉士。以葬親乞假歸,讀者若經生,無間寒暑。四方啖名者爭走其門,盡名為復社。溥亦傾身結納,交游日廣,聲氣通朝右。所品題甲乙,頗能為榮辱。諸奔走附麗者,輒自矜曰:「吾以嗣東林也。」執政大僚由此惡之。里人陸文聲者,輸貲為監生,求入社不許,采又嘗以事抶之。文聲詣闕言:「風俗之弊,皆原於士子。溥、采為主盟,倡復社,亂天下。」溫體仁方枋國事,下所司。遷延久之,提學御史倪元珙、兵備參議馮元揚、太倉知州周仲連言復社無可罪。三人皆貶斥,嚴旨窮究不已。閩人周之夔者,嘗為蘇州推官,坐事罷去,疑溥為之,恨甚。聞文聲訐溥,遂伏闕言溥等把持計典,己罷職實其所為,因及復社恣橫狀。章下,巡撫張國維等言之夔去官,無預溥事,亦被旨譙讓。

至十四年,溥已卒,而事猶未竟。刑部侍郎蔡奕琛坐黨薛國觀繫獄,未知溥卒也,訐溥遙握朝柄,己罪由溥,因言採結黨亂政。詔責溥、采回奏,采上言:「復社非臣事,然臣與溥生平相淬礪,死避網羅,負義圖全,誼不出此。念溥日夜解經論文,矢心報稱,曾未一日服官,懷忠入地。即今嚴綸之下,並不得泣血自明,良足哀悼。」當是時,體仁已前罷,繼者張至發、薛國觀皆不喜東林,故所司不敢復奏。及是,至發、國觀亦相繼罷,而周延儒當國,溥座主也,其獲再相,溥有力焉,故采疏上,事即得解。

明年,御史劉熙祚、給事中姜埰交章言溥砥行博聞,所纂述經史,有功聖學,宜取備乙夜觀。帝御經筵,問及二人,延儒對曰:「讀書好秀才。」帝曰:「溥已卒,采小臣,言官何為薦之?」延儒曰:「二人好讀書,能文章,言官為舉子時讀其文,又以其用未竟,故惜之耳。」帝曰:「亦未免偏。」延儒言:「誠如聖諭,溥與黃道周皆偏,因善讀書,以故惜之者眾。」帝頷之,遂有詔徵溥遺書,而道周亦復官。有司先後錄上三千餘卷,帝悉留覽。

溥詩文敏捷。四方征索者,不起草,對客揮毫,俄頃立就,以故名高一時。卒時,年止四十。

採,字受先,與溥善。溥性寬,泛交博愛。採特嚴毅,喜甄別可否,人有過,嘗面叱之。知臨川,摧強扶弱,聲大起。移疾歸,士民泣送載道。知州劉士斗、錢肅樂嚴重之,以奸蠹詢採,片紙報,咸置之法。福王時,起禮部主事,進員外郎,乞假去。南都失守,奸人素銜採者,群擊之死,復用大錐亂刺之。已而蘇,避之鄰邑,又三年卒。


\end{pinyinscope}