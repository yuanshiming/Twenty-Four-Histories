\article{列傳第五十三}

\begin{pinyinscope}
○陶成子魯陳敏丁瑄王得仁子一夔葉禎伍驥毛吉林錦郭緒姜昂子龍

陶成,字孔思,鬱林人。永樂中,舉於鄉,除交阯鳳山典史。尚書黃福知其賢,命署諒江府教授,交人化之。秩滿,遷山東按察司檢校,用薦擢大理評事。

正統中,以劉中敷薦,超擢浙江僉事。成有智略,遇事敢任。倭犯桃渚,成密布釘板海沙中。倭至,艤舟躍上,釘洞足背。倭畏之,遠去。秩滿,進副使。

處州賊葉宗留、陳鑒胡、陶得二等寇蘭谿,成擊斬數百人。進屯武義,立木城以守。誘賊黨為內應,前後斬首數百,生擒百餘人。又自抵賊巢,諭降者三千餘人。賊勢漸衰,惟得二尚在。久之,勢復熾,擁眾來犯。先遣其黨十餘輩偽為鄉民避賊者,以敝縕裹薪,闌入城。及成出戰,賊持薪縱火,焚木城。官軍驚潰,成與都指揮僉事崔源戰死。時景泰元年五月也。事聞,贈成左參政,錄其子魯為八品官。

魯,字自強,廕授新會丞。當是時,廣西瑤流劫高、廉、惠、肇諸府,破城殺吏無虛月。香山、順德間,土寇蜂起,新會無賴子群聚應之。魯召父老語曰:「賊氣吞吾城,不早備且陷,若輩能率子弟捍禦乎?」皆曰「諾」。乃築堡寨,繕甲兵,練技勇,以孤城捍賊衝。建郭掘濠,布鐵蒺藜刺竹于外,城守大固。賊來犯,輒擊破之。天順七年,秩滿,巡撫葉盛上其績,就遷知縣。尋以破賊功,進廣州同知,仍知縣事。

成化二年從總督韓雍征大藤峽。雍在軍嚴重,獨於魯未嘗不虛己。用其策,輒有功。雍請擢魯為僉事,專治新會、陽江、陽春、瀧水、新興諸縣兵。其冬會參將王瑛破劇賊廖婆保等於欽、化二州,大獲,璽書嘉勞。明年,賊首黃公漢等猖獗,偕參將夏鑑等連破之思恩、潯州。未幾,賊陷石康,執知縣羅紳。復偕鑑追擊至六菊山,敗之。兩廣自韓雍去,罷總督不設,帥臣觀望相推諉,寇盜滋蔓。魯奏請重臣仍開府梧州,遂為永制。秩滿,課最,進副使。兵部尚書餘子俊奏其撫輯勞,賚銀幣。

魯治兵久。賊剽兩粵,大者會剿,小者專征,所向奏捷。賊讎之次骨,劫其鬱林故居,焚誥命,發先塋,戕其族黨。魯聞大慟。詔徙籍廣東,補給封誥,慰勞有加,益奮志討賊。

二十年,以征荔浦瑤功,增俸一級。又九載,課最,進湖廣按察使,治兵兩廣如故。鬱林、陸川賊黃公定、胡公明等為亂,與參將歐磐分五路進討,大破之,毀賊巢一百三十。

弘治四年,總督秦紘遣平德慶瑤,進湖廣右布政使。魯言身居兩廣,而官以湖廣為名,於事體非便,乃改湖廣左布政使兼廣東按察副使,領嶺西道事。人稱之為「三廣公」。

十一年,總督鄧廷瓚請官其子,俾統魯所募健卒備征討。乃授其子荊民錦衣百戶。是年,魯卒。荊民復陳父功,遂進副千戶,世襲。

魯善撫士,多智計,謀定後戰。鑿池公署後,為亭其中,不置橋。夜則召部下計事。以版度一人,語畢,令退。如是凡數人,乃擇其長而參伍用之,故常得勝算而機不洩。羽書狎至,戎裝宿戒,聲色不動。審賊可乘,潛師出城,中夜合圍,曉輒奏凱。賊善偵,終不能得要領。歷官四十五年,始終不離兵事。大小數十戰,凡斬馘二萬一千四百有奇,奪還被掠及撫安復業者十三萬七千有奇,兩廣人倚之如長城。然魯將兵不專尚武,嘗言:「治寇賊,化之為先,不得已始殺之耳。」每平賊,率置縣建學以興教化。

魯初為丞,年纔弱冠,知縣王重勉之學。重故老儒,魯遂請執弟子禮。每晨,授經史講解而後視事。後重卒官,魯執喪如父禮,且資其二子。又敬事名儒陳獻章,獻章亦重之。宋陸秀夫、張世傑盡節崖山,未有廟祀,特為建祠,請祠額,賜名大忠。嘉靖初,魯歿三十載矣,新會人思其德,頌於朝,賜祠祀之。

陳敏,陜西華亭人。宣德時,為四川茂州知州。遭喪去官,所部諸長官司及番民百八十人詣闕奏言:「州僻處邊徼萬山中,與松籓、疊溪諸番鄰,歲被其患。自敏蒞州,撫馭有方,民得安業。今以憂去職,軍民失所依。乞矜念遠方,還此良牧。」帝立報可。

正統中,九載滿,軍民復請留。進成都府同知,視茂州事。都司徐甫言,敏及指揮孫敬在職公勤,群番信服。章下都御史王翱等核實,進敏右參議,仍視州事。以監司秩蒞州,前此未有也。

黑虎寨番掠近境,為官軍所獲。敏從其俗,與誓而遣之。既復出掠,為巡按御史陳員韜所劾。詔貰之。提督都御史寇深器其才,言敏往來撫恤番人,贊理軍政,乞別除知州,俾敏專戎務。吏部以敏蒞茂久,別除恐未悉番情,猝難馴服,宜增設同知一人佐之。報可。敏既以參議治州,其體儷監司。遂劾按察使陳泰無故杖死番人。泰亦訐敏,帝不問。而泰下獄論罪。

景泰改元,參議滿九載,進右參政,視州事如前。蒞州二十餘年,威信大行,番民胥悅。秩漸高,諸監司郡守反位其下,同事多忌之者。為按察使張淑所劾,罷去。

丁瑄,不知何許人。正統間為御史。初,福建多礦盜,命御史柳華捕之。華令村聚皆置望樓,編民為甲,擇其豪為長,得自置兵仗,督民巡徼。沙縣佃人鄧茂七素無賴,既為甲長,益以氣役屬鄉民。其俗佃人輸租外,例饋田主。茂七倡其黨令毋餽,而田主自往受粟。田主訴於縣,縣逮茂七,不赴。下巡檢追攝,茂七殺弓兵數人。上官聞,遣軍三百捕之。被殺傷幾盡,巡檢及知縣並遇害。茂七遂大剽略,偽稱「鏟平王」,設官屬,黨數萬人,陷二十餘縣。都指揮范真、指揮彭璽等先後被殺。時福建參政交阯人宋新,賄王振得遷左布政使,侵漁貪惡,民不能堪,益相率從亂。東南騷動。

十三年四月,茂七圍延平。刷卷御史張海登城撫諭。賊訴乞貰死,免三年徭役,即解散為良民。海以聞。命瑄往招討,以都督劉聚、僉都御史張楷大軍繼其後。瑄既至,先令人齎敕往撫。茂七不肯降,瑄馳赴沙縣圖之。賊首林宗政等萬餘人攻後坪,欲立寨。瑄令通判倪冕等率眾先據要害,而身與都指揮雍埜等邀其歸路,斬賊二百餘級,獲其渠陳阿巖。

明年二月,瑄誘賊復攻延平,督眾軍分道衝擊。賊大敗,遁走,指揮劉福追之,遂斬茂七,招脅從復業。未幾,復擒其黨林子得等。尤溪賊首鄭永祖率四千人攻延平。瑄偕埜等邀擊,擒之,斬首五百有奇,餘黨潰散。

楷之監大軍討賊也,至建寧頓不進,日置酒賦詩為樂。聞瑄破賊,則馳至延平攘其功。瑄被脅依違具奏。福不能平,愬之。詔責瑄具狀。楷等皆獲罪。瑄有功不問,功亦竟不錄。茂七雖死,其從子伯孫等復熾。朝廷更遣陳懋等以大軍討,瑄乃還朝。景泰初,出為廣東副使,卒。

當是時,浙、閩盜所在剽掠為民患。將帥率玩寇,而文吏勵民兵拒賊,往往多斬獲。閩則有張瑛、王得仁之屬。浙江則金華知府石瑁擒遂昌賊蘇才於蘭谿。處州知府張佑擊敗賊眾,擒斬千餘人。於是帝降敕,數詰讓諸將帥。都指揮鄧安等因歸咎於前御史柳華。時王振方欲殺朝士威眾,命逮華。華已出為山東副使,聞命,仰藥死。詔籍其家,男戍邊,婦女沒入浣衣局。而御史汪澄、柴文顯亦以是得罪。

初,澄按福建,以茂七亂,檄浙江、江西會討。尋以賊方議降,止兵毋進。既知賊無降意,復趣進兵,而賊已不可制。浙江巡按御史黃英恐得罪,具白澄止兵狀,兵部因劾澄失機。福建三司亦言,賊初起,按臣柴文顯匿不奏,釀成今患。遂俱下吏。獄成,詔磔文顯,籍其家。澄棄市。而宋新及按察使方冊等十人俱坐斬。遇赦,謫驛丞。天順初,復官。

論者謂華所建置未為過,澄、文顯罪不至死。武將不能滅賊,反罪文吏。華、文顯至與叛逆同科,失刑實由王振云。華,吳縣人。文顯,浙江建德人。澄仁和人。

王得仁,名仁,以字行,新建人。本謝姓,父避仇外家,因冒王氏。得仁五歲喪母,哀號如成人。初為衛吏,以才薦授汀州府經歷。廉能勤敏,上下愛之。秩滿當遷,軍民數千人乞留,詔增秩再任。居三年,推官缺,英宗從軍民請,就令遷擢。數辯冤獄,卻饋遺,抑鎮守內臣苛索,政績益著。

沙縣賊陳政景,故鄧茂七黨也。糾清流賊藍得隆等攻城。得仁與守將及知府劉能擊敗之,擒政景等八十四人,餘賊驚潰。諸將議窮搜,得仁恐濫及百姓,下令招撫,辨釋難民三百人。都指揮馬雄得通賊者姓名,將按籍行戮,得仁力請焚其籍。賊復寇寧化,率兵往援,斬首甚眾。民多自拔歸,賊勢益衰。

賊退屯將樂,得仁將追滅之,俄遘疾。眾欲輿歸就醫,得仁不可,曰:「吾一動,賊必長驅。」乃起坐帳中,諭將吏戮力平賊,遂卒。時正統十四年夏也。軍民哀慟。喪還,哭奠者道路相屬,多繪像祀之。天順末,吏民乞建祠。有司為請,詔如廣東楊信民故事,春秋致祭。

子一夔,天順四年舉進士第一。授修撰,進左諭德。成化七年,彗星見。應詔陳五事:請正宮闈,親大臣,開言路,慎刑獄,戒妄費。語極剴摯,被旨切責。累遷工部尚書。卒,贈太子少保。正德中,謚文莊。

葉禎,字夢吉,高要人。舉於鄉,授潯州府同知。補鳳翔,調慶遠。

兩廣瑤賊蜂起,列郡咸被害,將吏率縮朒觀望。禎誓不與賊俱生,募健兒日訓練。峒酋韋父強數敗官軍,禎生縶之,其黨忿,悉眾攻城。旗山守將擁兵不救。禎率健兒出戰,賊卻去。旋躡禎,戰相當,禎子公榮殲焉。

頃之,賊圍雞刺諸村,禎率三百人趨赴。道遇賊人頭山下,鏖戰,禎被數鎗,手刃賊一人,與從子官慶及三百人皆死。時天順三年正月晦也。嶺南素無雪,是夜大雷電,雪深尺許。賊釋圍去,諸村獲全。事聞,贈朝列大夫、廣西參議,守臣為立廟祀之。

伍驥,字德良,安福人。景泰五年進士。授御史。莊重寡言笑,見義敢為。

天順七年巡按福建。先是,上杭賊起,都指揮僉事丁泉,汶上人,善捍禦。賊屢攻城,皆為所卻。已而賊轉熾。驥聞,立馳入汀州,調援兵四集。驥單騎詣賊壘。賊不意御史猝至,皆擐甲露刃。驥從容立馬,諭以禍福。賊見其至誠,感悟泣下,歸附者千七百餘戶。給以牛種,俾復故業。

惟賊首李宗政負固不服,遂與泉深入破之。泉力戰,為賊所害。驥弔死恤傷,激以忠義,復與賊戰。連破十八寨,俘斬八百餘人,四境悉平。而驥冒瘴癘成疾,班師至上杭卒。軍民哀之如父母,旦夕臨者數千人,爭出財立祠。成化中以知縣蕭宏請,詔與泉並祀,賜祠名「褒忠」。

毛吉,字宗吉,餘姚人。景泰五年進士。除刑部廣東司主事。司轄錦衣衛。衛卒伺百官陰事,以片紙入奏即獲罪,公卿大夫莫不惴恐。公行請屬,狎侮官司,即以罪下刑部者,亦莫敢捶撻。吉獨執法不撓,有犯必重懲之。其長門達怙寵肆虐,百官道遇率避馬,吉獨舉鞭拱手過,達怒甚。吉以疾失朝,下錦衣獄。達大喜,簡健卒,用巨梃搒之。肉潰見骨,不死。

天順五年擢廣東僉事,分巡惠、潮二府。痛仰豪右,民大悅。及期當代,相率籲留之。

程鄉賊楊輝者,故劇賊羅劉寧黨也。已撫復叛,與其黨曾玉、謝瑩分據寶龍、石坑諸洞,攻陷江西安遠,剽閩、廣間。已,欲攻程鄉。吉先其未至,募壯士合官軍得七百人。抵賊巢。先破石坑,斬玉;次擊瑩,馘之。復生擒輝。諸洞悉破,凡俘斬千四百人。捷聞,憲宗進吉副使,璽書嘉勞。移巡高、雷、廉三府。時民遭賊躪,數百里無人煙,諸將悉閉城自守,或以賊告,反被撻。有自賊中逸歸者,輒誣以通賊,撲殺之。吉不勝憤,以平賊為己任。按部雷州。海康知縣王騏,雲南太和人也,日以義激其民,賊至輒奮擊。吉壯其勇節,獎勵之。適報賊掠鄉聚,吉與騏各率所部擊敗之。薦騏,遷雷州通判。未聞命,戰死。贈同知,蔭其子為國子生。

成化元年二月,新會告急。吉率指揮閻華、掌縣事同知陶魯,合軍萬人,至大磴破賊,乘勝追至雲岫山,去賊營十餘里。時已乙夜,召諸將分三哨,黎明進兵。會陰晦,眾失期。及進戰,賊棄營走上山。吉命潘百戶者據其營,眾競取財物。賊馳下,殺百戶。華亦馬躓,為賊所殺。諸軍遂潰。吉勒馬大呼止軍。吏勸吉避,吉曰:「眾多殺傷,我獨生可乎?」言未已,賊持鎗趨吉。古且罵且戰,手劍一人,斷其臂。力絀,遂被害。是日,雷雨大作,山谷皆震動。又八日,始得屍,貌如生。事聞,贈按察使,錄其子科入國子監。尋登進士,終雲南副使。

方吉出軍時,齎千金犒,委驛丞餘文司出入,已用十之三。吉既死,文憫其家貧,以所餘金授吉僕,使持歸治喪。是夜,僕婦忽坐中堂作吉語,顧左右曰:「請夏憲長來。」舉家大驚,走告按察使夏塤。塤至。起揖曰:「吉受國恩,不幸死於賊。今餘文以所遺官銀付吉家,雖無文簿可考,吉負垢地下矣。願亟還官,毋污我。」言畢,仆地,頃之始蘇。於是歸金於官。吉死時年四十,後賜謚忠襄。

林錦,字彥章,連江人。景泰初,由鄉貢授合浦訓導。瑤寇充斥,內外無備。錦條具方略,悉中機宜。巡撫葉盛異之,檄署靈山縣事。城毀於賊,錦因形便,為柵以守,廣設戰具,賊不敢逼。滿秩去官,民曰:「公去,賊復至,誰禦者?」悉逃入山。盛以狀聞,詔即以錦為知縣。馳驛之官,民復來歸。

適歲饑,諸瑤益剽掠無虛日。錦單騎詣壘,曉以禍福。瑤感悟,附縣二十五部咸聽命。其不服者則討之。天順六年破賊羅禾水,再破之黃姜嶺,又大破之新莊。先後斬獲千餘級,還所掠人口,賊悉平,乃去柵,築土城。

盛及監司屢薦其才。成化改元,會廉州為賊所陷,乃以錦為試知府。歲復大饑,賊四出劫掠。錦諭散千餘人,誅梗化者,而綏輯其流移。境內悉平。

四年,上官交薦,請改授憲職,令專備欽、廉群盜。乃以為按察使僉事,益勤於政。十年賜敕旌異。久之,進副使。錦以所部屢有盜警,思為經久計,乃設團河營於西,設新寮營於南,而別設洪崖營以杜諸寇出沒路。易靈山土城,更築高墉,亙五百丈,卒為巖邑。十四年,兵部上其撫輯功,被賚。

錦在兵間,以教化為務。靈山尚鬼,則禁淫祠,修學校,勸農桑。其治廉、欽,皆飭學宮,振起文教。為人誠實,洞見肺腑,瑤蠻莫不愛信。其行軍,與士卒同甘苦,有功輒推以與人,以故士多效死,所在祠祀。

郭緒,字繼業,太康人。成化十七年進士。使楚府,卻其饋。授戶部主事,督餉二十萬於陜西給軍。主者以羨告,悉還之。歷遷雲南參議。

初,孟密宣撫司之設也,實割木邦宣慰司地。既而孟密思揲復於界外侵木邦地二十七所。屬諭之還。不聽。乃調孟養宣撫思祿兵脅之。思揲始還所侵地,然多殺孟養兵。思祿仇之,發兵越金沙江奪木邦故割孟密地十有三所。兩酋構怨不已。

巡撫陳金承詔,遣緒與副使曹玉往諭之。旬餘抵金齒。參將盧和先統軍距所據地二程而舍,遣官馳驛往諭,皆留不報。和懼,還軍至乾崖遇緒,語故,戒勿進。緒不可。玉以疾辭。緒遂單騎從數人行,旬日至南甸,峻險不可騎,乃斬棘徒步引繩以登。又旬日至一大澤。土官以象輿來,緒乘之往。行毒霧中,泥沙𧾷甚踔。又旬日至孟賴,去金沙江僅二舍。手自為檄,使持過江,諭以朝廷招徠意。蠻人相顧驚曰:「中國使竟至此乎?」發兵率象馬數萬夜渡江,持長槊勁弩,環之數重。從行者懼,請勿進。緒拔刀叱曰:「明日必渡江,敢阻者斬!」思祿既得檄,見譬曉禍福甚備,又聞至者纔數人,乃遣酋長來聽令,且致饋。緒卻之,出敕諭宣示。思祿亦繼至。緒先敘其勞,次白其冤狀,然後責其叛。諸酋聞,咸俯伏呼萬歲,請歸侵地。緒詰前所留使人,乃盡出而歸之。和及玉聞報馳至,則已歸地納款矣。時弘治十四年五月也。

越三年,擢緒四川督儲參政。武宗即位,始以雲南功,加俸一級。明年致仕歸。

姜昂,字恒頫,太倉人。成化八年進士。除棗強知縣。授御史。偕同官劾方士李孜省,杖午門外。以母老乞改南,尋出為河南知府。吏白事畢,退闔門讀書,鞭箠懸不用。籓府人有犯,立決遣之。改知寧波,擢福建參政。請終養歸,服闋而卒。

昂在官,日市少肉供母,而自食菜茹。子弟學書,不聽用官紙筆,家居室不蔽風雨。

子龍,字夢賓,正德三年進士。歷禮部郎中。武宗南巡,率同官諫。罰跪五日,杖幾死。出為建寧同知,尋遷雲南副使,備兵瀾滄、姚安。滇故盜藪,龍讓土酋曰:「爾世官,縱盜寧非賄乎?」酋懼,撫諭群盜,悉聽命。巨盜方定者,既降而貧,為妻妾所詬,卒不忍負龍,竟仰藥死。南安大盜千人,御史欲徵兵,龍檄三日散盡。四川鹽井剌馬仁、雲南曬江和歌仲讎殺數十年,龍撫諭,遂解。大候州土官猛國恃險肆暴,龍擒之。在滇四年,番、漢大治。鄧川州立三正人祠,祀袁州郭紳、莆田林俊及龍。

贊曰:陶成、陳敏諸人,以監司守令著征剿功,而成及毛吉、葉禎身死王事,勞烈顯著,亦可以愧戎帥之畏懦𧾷戚蹜者矣。林錦威能臨制,材足綏懷,邊疆皆得斯人,何憂不治?郭緒單騎入險,諭服兩酋,令當洪、永間亦何至尚淹常調哉。平世秉國者,多抑邊功,謂恐生事。然大帥倚內援,敘祿又多逾等,適足以長武夫玩寇之心,而無以獎勞臣致死之節。國家以賞罰馭世,曷可不公乎!


\end{pinyinscope}