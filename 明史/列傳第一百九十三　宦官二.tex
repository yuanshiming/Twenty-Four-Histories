\article{列傳第一百九十三 宦官二}

\begin{pinyinscope}
○李芳馮保張鯨陳增陳奉高淮梁永楊榮陳矩王安魏忠賢王體乾李永貞等崔文昇張彞憲高起潛王承恩方正化

李芳,穆宗朝內官監太監也。帝初立,芳以能持正見信任。初,世宗時,匠役徐杲以營造躐官工部尚書,修盧溝橋,所侵盜萬計。其屬冒太僕少卿、苑馬卿以下職銜者以百數。隆慶元年二月,芳劾之。時杲已削官,乃下獄遣戍,盡汰其所冒冗員。又奏革上林苑監增設皁隸,減光祿歲增米鹽及工部物料,以是大為同類所嫉。而是時,司禮諸閹滕祥、孟沖、陳洪方有寵,爭飾奇技淫巧以悅帝意,作鰲山燈,導帝為長夜飲。芳切諫,帝不悅。祥等復媒孽之,帝遂怒,勒芳閒住。二年十一月復杖芳八十,下刑部監禁待決。尚書毛愷等言:「芳罪狀未明,臣等莫知所坐。」帝曰:「芳事朕無禮,其錮之。」芳錮,祥等益橫。前司禮太監黃錦已革廕,祥輒復予之。工部尚書雷禮劾祥:「傳造採辦器物及修補壇廟樂器,多自加徵,糜費巨萬。工廠存留大木,斬截任意。臣禮力不能爭,乞早賜罷。」帝不罪祥,而令禮致仕。沖傳旨下海戶王印於鎮撫司,論戍,法司不預聞。納肅籓輔國將軍縉貴賄,越制得嗣封肅王。洪尤貪肆,內閣大臣亦有因之以進者。三人所糜國帑無算。帝享太廟,三人皆冠進賢冠,服祭服以從,爵賞辭謝與六卿埒。廷臣論劾者,太常少卿周審怡以外補去,給事中石星、李已、陳吾德,御史詹仰庇,尚寶丞鄭履淳,皆廷杖削籍。三人各廕錦衣官至二十人,而芳獨久系獄。四年四月,刑科都給事中舒化等以熱審屆期,請宥芳,乃得釋,充南京凈軍。

馮保,深州人。嘉靖中,為司禮秉筆太監。隆慶元年提督東廠兼掌御馬監事。時司禮掌印缺,保以次當得之,適不悅於穆宗。大學士高拱薦御用監陳洪代,保由是疾拱。及洪罷,拱復薦用孟沖。沖故掌尚膳監者,例不當掌司禮。保疾拱彌甚,乃與張居正深相結,謀去之。會居正亦欲去拱專柄,兩人交益固。穆宗得疾,保密屬居正豫草遺詔,為拱所見,面責居正曰:「我當國,奈何獨與中人具遺詔。」居正面赤謝過。拱益惡保,思逐之。

穆宗甫崩,保言於后妃,斥孟沖而奪其位,又矯遺詔令與閣臣同受顧命。及帝登極,保升立寶座旁不下,舉朝大駭。保既掌司禮,又督東廠,兼總內外,勢益張。拱諷六科給事中程文、十三道御史劉良弼等,交章數其奸,而給事中雒遵、陸樹德又特疏論列,拱意疏下即擬旨逐保。而保匿其疏,亟與居正定謀,遂逐拱去。

初,穆宗崩,拱於閣中大慟曰:「十歲太子,如何治天下。。」保譖於后妃曰:「拱斥太子為十歲孩子,如何作人主。」后妃大驚,太子聞之亦色變。迨拱去,保憾猶未釋。萬曆元年正月,有王大臣者,偽為內侍服,入乾清宮,被獲下東廠。保欲緣此族拱,與居正謀,令家人辛儒飲食之,納刃其袖中,俾言拱怨望,遣刺帝。大臣許之。踰日,錦衣都督朱希孝等會鞫。大臣疾呼曰:「許我富貴,乃掠治我耶!且我何處識高閣老?」希孝懼,不敢鞫而罷。會廷臣楊博、葛守禮等保待之,居正亦迫眾議微諷保。保意稍解,乃以生漆酒喑大臣,移送法司坐斬,拱獲免。由是舉朝皆惡保,而不肖者多因之以進。

慈聖太后遇帝嚴。保倚太后勢,數挾持帝,帝甚畏之。時與小內豎戲,見保入,輒正襟危坐曰:「大伴來矣。」所暱孫海、客用為乾清宮管事牌子,屢誘帝夜游別宮,小衣窄袖,走馬持刀,又數進奇巧之物,帝深寵幸。保白太后,召帝切責。帝長跪受教,惶懼甚。保屬居正草帝罪己手詔,令頒示閣臣。詞過挹損,帝年已十八,覽之內慚,然迫於太后,不得不下。居正乃上疏切諫。又緣保意劾去司禮秉筆孫德秀、溫太及掌兵伏局周海,而令諸內侍俱自陳。由是保所不悅者,斥退殆盡,時八年十一月也。

保善琴能書。帝屢賜牙章曰「光明正大」,曰「爾惟鹽梅」,曰「汝作舟楫」,曰「魚水相逢」,曰「風雲際會」,所以待之甚隆。後保益橫肆,即帝有所賞罰,非出保口,無敢行者。帝積不能堪,而保內倚太后,外倚居正,帝不能去也。然保亦時引大體。內閣產白蓮,翰林院有雙白燕,居正以進。保使使謂居正曰:「主上沖年,不可以異物啟玩好。」又能約束其子弟,不敢肆惡,都人亦以是稱之。

居正固有才,其所以得委任專國柄者,由保為之左右也。然保性貪,其私人錦衣指揮徐爵、內官張大受,為保、居正交關語言。且數用計使兩人相疑,旋復相好,兩人皆在爵術中。事與籌畫,因恃勢招權利,大臣亦多與通。爵夜至禁門,守衛者不敢詰,其橫如此。居正之奪情及杖吳中行等,保有力焉。已而居正死,其黨益結保自固。居正以遺疏薦其座主潘晟入閣,保即遣官召之。御史雷士楨、王國,給事中王繼光相繼言其不可用,晟中途疏辭。內閣張四維度申時行不肯為晟下,擬旨允之,帝即報可。保時病起,詬曰:「我小恙,遽無我耶?」皇太子生,保欲封伯爵,四維以無故事難之,擬廕弟姪一人都督僉事。保怒曰:「爾由誰得今日,而負我!」御史郭惟賢請召用吳中行等,保責其黨護,謫之。吏部尚書王國光罷,保輒用其鄉人梁夢龍代。爵、大受等竊權如故。

然是時太后久歸政,保失所倚,帝又積怒保。東宮舊閹張鯨、張誠間乘陳其過惡,請令閒住。帝猶畏之,曰:「若大伴上殿來,朕奈何?」鯨曰:「既有旨,安敢復入。」乃從之。會御史李植、江東之彈章入,遂謫保奉御,南京安置,久之乃死。其弟佑、從子邦寧並官都督,削職下獄,瘐死。大受其黨周海、何忠等八人,貶小火者,司香孝陵。爵與大受子,煙瘴永戍。盡籍其家,保金銀百餘萬,珠寶瑰異稱是。

保之發南京也,太后問故。帝曰:「老奴為張居正所惑,無他過,行且召還。」時潞王將婚,所需珠寶未備,太后間以為言。帝曰:「年來無恥臣僚,盡貨以獻張、馮二家,其價驟貴。」太后曰:「已籍矣,必可得。」帝曰:「奴黠猾,先竊而逃,未能盡得也。」而其時,錦衣都督劉守有與僚屬張昭、龐清、馮昕等,皆以籍罪人家,多所隱沒,得罪。

張鯨,新城人,太監張宏名下也。內豎初入宮,必投一大當為主,謂之名下。馮保用事,鯨害其寵,為帝畫策害保。寵謂鯨曰:「馮公前輩,且有骨力,不宜去之。」鯨不聽。既譖逐保,宏遂代保掌司禮監,而鯨掌東廠。宏無過惡,以賢稱,萬曆十二年卒。張誠代掌司禮監。十八年,鯨罷東廠,誠兼掌之。二十四年春,以誠聯姻武清侯,擅作威福,降奉御,司香孝陵,籍其家,弟姪皆削職治罪。

鯨性剛果,帝倚任之。其在東廠兼掌內府供用庫印,頗為時相所憚。而其用事司房邢尚智,招權受賕。萬曆十六年冬,御史何出光劾鯨及其黨鴻臚序班尚智與錦衣都督劉守有相倚為奸,專擅威福,罪當死者八。帝命鯨策勵供事,而削尚智、守有職,餘黨法司提問。給事中陳尚象、吳文梓、楊文煥,御史方萬策、崔景榮復相繼論列,報聞。法司奏鯨等贓罪,尚智論死,鯨被切責。給事中張應登再疏論之,御史馬象乾并劾大學士申時行阿縱。帝皆不聽,命下象乾詔獄。以時行及同官許國、王錫爵等申救,象乾疏乃留中。給事中李沂至謂帝納鯨金寶,故寬鯨罪。帝大怒,言沂等為張居正、馮保報復,杖六十,削其官,鯨亦私家閒住。已而南京兵部尚書吳文華率南九卿請罪鯨而宥言者,帝亦不聽。尋復召鯨入。給事中陳與郊、御史賈希夷、南京吏部尚書陸光祖、給事中徐常吉、御史王以通等言益力,俱不報。最後大理評事雒于仁上酒色財氣四箴,指鯨以賄復進。帝怒甚,召申時行等於毓德宮,命治于仁罪,而召鯨,令時行等傳諭責訓之,鯨寵遂衰。尚智後減死充軍。

陳增,神宗朝礦稅太監也。萬曆十二年,房山縣民史錦奏請開礦,下撫按查勘,不果行。十六年,中使祠五臺山,還言紫荊關外廣昌、靈丘有礦砂,可作銀冶。帝聞之喜,以大學士申時行等言而止。十八年,易州民周言、張世才復言阜平、房山各產礦砂,請遣官開礦。時行等仍執不可。

至二十年,寧夏用兵,費帑金二百餘萬。其冬。朝鮮用兵,乎尾八年,費帑金七百餘萬。二十七年,播州用兵,又費帑金二三百萬。三大征踵接,國用大匱。而二十四年,乾清、坤寧兩宮災。二十五年,皇極、建極、中極三殿災。營建乏資,計臣束手,礦稅由此大興矣。其遣官自二十四年始,其後言礦者爭走闕下,帝即命中官與其人偕往,天下在在有之。真、保、薊、永則王亮,昌黎、遷安則田進,昌平、橫嶺、淶水、珠寶窩山則王忠,真定復益以王虎,并採山西平定、稷山,浙江則曹金,後代以劉忠,陜西則趙欽,山西則張忠,河南則魯坤,廣東則李鳳、李敬,雲南則楊榮,遼東則高淮,江西則潘相,福建則高寀,湖廣則陳奉,而增奉敕開採山東。通都大邑皆有稅監,兩淮則有鹽監,廣東則有珠監,或專遣,或兼攝。大當小監縱橫繹騷,吸髓飲血,以供進奉。大率入公帑者不及什一,而天下蕭然,生靈塗炭矣。其最橫者增及陳奉、高淮。

二十四年,增始至山東,即劾福山知縣韋國賢,帝為逮問削職。益都知縣吳宗堯抗增,被陷幾死詔獄。巡撫尹應元奏增二十大罪,亦罰俸。已,復命增兼徵山東店稅,與臨清稅監馬堂相爭。帝為和解,使堂稅臨清,增稅東昌。增益肆無忌,其黨內閣中書程守訓、中軍官仝治等,自江南北至浙江,大作奸弊。稱奉密旨DE金寶,募人告密。誣大商巨室藏違禁物,所破滅什伯家,殺人莫敢問。御史劉曰梧具以狀聞,鹽務少監魯保亦奏守訓等阻塞鹽課,帝俱弗省。久之,鳳陽巡撫李三才劾守訓奸贓。增懼,因搜得守訓違禁珍寶及賕銀四十餘萬,聞於朝。命械入京鞫治,乃論死。而增肆惡山東者十年,至三十三年始死。

陳奉,御馬監奉御也。萬曆二十七年二月命徵荊州店稅,兼採興國州礦洞丹砂及錢廠鼓鑄事。奉兼領數使,恣行威虐。每託巡歷,鞭笞官吏,剽劫行旅。商民恨刺骨,伺奉自武昌抵荊州,聚數千人噪於塗,競擲瓦石擊之。奉走免,遂誣襄陽知府李商畊黃州知府趙文煒、荊州推官華鈺、荊門知州高則巽、黃州經歷車任重等煽亂。帝為逮鈺、任重,而謫商畊等官。興國州奸人漆有光,訐居民徐鼎等掘唐相李林甫妻楊氏墓,得黃金巨萬。騰驤衛百戶仇世亨奏之,帝命奉括進內庫。奉因毒拷責償,且悉發境內諸墓。巡按御史王立賢言所掘墓乃元呂文德妻,非林甫妻。奸人訐奏,語多不仇,請罷不治,而停他處開掘,不報。

二十八年十二月,武昌民變。南京吏部主事吳中明奏言:「奉嚇詐官民,僭稱千歲。其黨至直入民家,奸淫婦女,或掠入稅監署中。王生之女、沈生之妻,皆被逼辱。以致士民公憤,萬餘人甘與奉同死,撫按三司護之數日,僅而得全。而巡撫支可大,曲為蒙蔽。天下禍亂,將何所底!」大學士沈一貫亦言:「陳奉入楚,始而武昌一變,繼之漢口、黃州、襄陽、武昌、寶慶、德安、湘潭等處,變經十起,幾成大亂。立乞撤回,以收楚民之心。」帝皆置不問。

奉復使人開穀城礦,不獲,脅其庫金,為縣所逐。武昌兵備僉事馮應京劾奉十大罪,奉隨誣奏,降應京雜職。奉又開棗陽礦,知縣王之翰以顯陵近,執不可。奉劾之翰及襄陽通判邸宅、推官何棟如,緹騎逮訊,并追逮應京。應素有惠政,民號哭送之。奉又榜列應京罪狀於衢。民切齒恨,復相聚圍奉署,誓必殺奉。奉逃匿楚王府,眾乃投奉黨耿文登等十六人於江,以巡撫可大護奉,焚其轅門。事聞,一貫及給事中姚文蔚等請撤奉,不報。而御馬監監丞李道方督理湖口船稅,亦奏奉水沮商舟,陸截販賈,徵三解一,病國剝民。帝始召奉歸,而用一貫請,革可大職。奉在湖廣二年,慘毒備至。及去,金寶財物巨萬計,可大懼為民所掠,多與徒衛,導之出疆,楚民無不毒恨者。奉至京師,給事中陳維春、郭如星復極言其罪。帝不懌,降二人雜職。三十二年始釋應京歸,之翰卒瘐死。

當奉劾商畊等時,臨清民亦噪而逐馬堂。馬堂者,天津稅監也,兼轄臨清。始至,諸亡命從者數百人,白晝手鋃鐺奪人產,抗者輒以違禁罪之。僮告主者,畀以十之三,中人之家破者大半,遠近為罷市。州民萬餘縱火焚堂署,斃其黨三十七人,皆黥臂諸偷也。事聞,詔捕首惡,株連甚眾。有王朝佐者,素仗義,慨然出曰:「首難者,我也。」臨刑,神色不變。知府李士登恤其母妻,臨清民立祠以祀。後十餘年,堂擅往揚州,巡鹽御史徐縉芳劾其九罪,不問。

高淮,尚膳監監丞也。神宗寵愛諸稅監,自大學士趙志皋、沈一貫而下,廷臣諫者不下百餘疏,悉寢不報。而諸稅監有所糾劾,朝上夕下,輒加重譴。以故諸稅監益驕,而淮及梁永尤甚。淮與陳奉同時採礦徵稅遼東。委官廖國泰,虐民激變,淮誣繫諸生數十人。巡按楊宏科救之,不報。參隨楊永恩婪賄事發,奉旨會勘,卒不問。淮又惡遼東總兵馬林不為己下,劾罷之。給事中候先春疏救,遂戍林而謫先春雜職。巡按何爾健與淮互訐奏,淮遣人邀於路,責其奏事人,錮之獄,匿疏不以聞。又請復遼東馬市,巡撫趙楫力爭,始得寢。

三十一年夏,淮率家丁三百餘,張飛虎幟,金鼓震天,聲言欲入大內謁帝,潛住廣渠門外。給事中田大益、孫善繼、姚文蔚等言:「淮搜括士民,取金至數十萬,招納諸亡命降人,意欲何為?」吏部尚書李戴、刑部尚書蕭大亨皆劾淮擅離信地,挾兵潛住京師,乃數百年未有之事。御史袁九皋、劉四科、孔貞一,給事中梁有年等,各疏劾淮,不報。巡撫楫劾淮罪惡萬端,且無故打死指揮張汝立,亦不報。淮因上疏自稱鎮守協同關務,兵部奏其妄。帝心護淮,謬曰:「朕固命之矣。」

淮自是益募死士,時時出塞射獵,發黃票龍旂,走朝鮮索冠珠、貂馬,數與邊將爭功,山海關內外咸被其毒。又扣除軍士月糧。三十六年四月,前屯衛軍甲而噪,誓食淮肉。六月,錦州、松山軍復變。淮懼內奔,誣同知王邦才、參將李獲陽逐殺欽使,劫奪御用錢糧。二人皆逮問,邊民益嘩。薊遼總督蹇達再疏暴淮罪,乃召歸,而以通灣稅監張曄兼領其事。獲陽竟死獄中,邦才至四十一年乃釋。

梁永,御馬監監丞也。萬曆二十七年二月命往陜西徵收名馬貨物。稅監故不典兵,永獨畜馬五百匹,招致亡命,用千戶樂綱出入邊塞。富平知縣王正志發其奸,并劾礦監趙欽。詔逮正志,瘐死詔獄中。渭南知縣徐斗牛,廉吏也。永責賂,箠斃縣吏卒,斗牛憤恨自縊死。巡撫賈待問奏之,帝顧使永會勘。永反劾西安同知宋賢,并劾待問有私,請皆勘。帝從之,而宥待問。永又請兼鎮守職銜。又請率兵巡花馬池、慶陽諸鹽池,徵其課。緣是帥諸亡命,具旌蓋鼓吹,巡行陜地。盡發歷代陵寢,搜摸金玉,旁行劫掠。所至,邑令皆逃。杖死縣丞鄭思顏、指揮劉應聘、諸生李洪遠等。縱樂綱等肆為淫掠,私宮良家子數十人。稅額外增耗數倍,藍田等七關歲得十萬。復用奸人胡奉言,索咸陽冰片五十斤、羊毛一萬斤、麝香二十斤。知縣宋時際怒,勿予。

咸寧人道行遇盜,跡之,稅使役也,知縣滿朝薦捕得之。永誣時際、朝薦劫稅銀,帝命逮時際,而以朝薦到官未久,鐫秩一級。陜西巡撫顧其志盡發其奸,且言秦民萬眾,共圖殺永。大學士沈鯉、朱賡請械永歸,以安眾心。帝悉置不報,而釋時際勿逮,復朝薦官。

會御史餘懋衡方按陜西,永懼,使綱鴆懋衡幾死。訟於朝,言官攻永者數十疏,永部下諸亡命乃稍稍散。其渠魁王九功、石君章等齎重寶,輜軿盈路,詐為上供物,持劍戟弓弩,結陣以行。而永所遣人解馬匹者,已乘郵傳先發。九功等急馳,欲追及與同出關。朝薦疑其盜,見九功等後至無驗,邏兵與格鬥,追至渭南,殺數人,盡奪其裝。御史懋衡以捕盜殺傷聞。永大窘,聽樂綱謀,使人繫疏髮中馳奏:「九功等各貢名馬、金珠、睛綠諸寶物,而咸寧知縣朝薦承餘御史指,伏兵渭南遮劫之,臠君章等,誣以盜。」帝怒曰:「御史鴆無恙,而朝薦代為報復,且劫貢物。」敕逮朝薦,而令撫按護永等還京。三十四年事也。

是年,楊榮為雲南人所殺。初,榮妄奏阿瓦、猛密諸番願內屬,其地有寶井,可歲益數十萬,願賜敕領其事。帝許之。既而榮所進不得什一,乃誣知府熊鐸侵匿,下法司。又請詔麗江土知府木增獻地聽開採。巡按御史宋興祖言:「太祖令木氏世守茲土,限石門以絕西域,守鐵橋以斷土蕃,奈何自撤籓蔽,生遠人心。」不報。榮由是愈怙寵,誣劾尋甸知府蔡如川、趙州知州甘學書,皆下詔獄。已,又誣劾雲南知府周鐸,下法司提問。百姓恨榮入骨,相率燔稅廠,殺委官張安民。榮弗悛,恣行威虐,杖斃數千人。至是怒指揮使樊高明後期,榜掠絕觔,枷以示眾。又以求馬不獲,繫指揮使賀瑞鳳,且言將盡捕六衛官。於是指揮賀世勛、韓光大等率冤民萬人焚榮第,殺之,投火中,并殺其黨二百餘人。事聞,帝為不食者數日,欲逮問守土官。大學士沈鯉揭爭,且密屬太監陳矩剖示。帝乃止誅世勛等,而用巡撫陳用賓議,令四川稅使丘乘雲兼攝雲南事。

當是時,帝所遣中官,無不播虐逞兇者。

湖口稅監李道劾降九江府經歷樊圃充,又劾逮南康知府吳寶秀、星子知縣吳一元,降臨江知府顧起淹。

山西稅監孫朝劾降夏縣知縣韓薰。給事中程紹以救薰鐫一級,給事中李應策等復救之,遂削紹、薰職。巡撫魏允貞以阻撓罷去。

廣東稅監李鳳劾逮鄉官通判吳應鴻等。鳳與珠池監李敬相仇,巡按李時華恃敬援劾鳳。給事中宋一韓言鳳乾沒五千餘萬,他珍寶稱是。吏部尚書李戴等言鳳釀禍,致潮陽鼓噪,粵中人爭欲殺之。帝不問。而敬惡亦不減於鳳,採珠七八年,歲得珠近萬兩。其後珠池盜起,敬乃請罷採。

山西礦監張忠劾降夏縣知縣袁應春,又劾逮西城兵馬戴文龍。

江西礦監潘相激浮梁景德鎮民變,焚燒廠房。饒州通判陳奇可諭散之,相反劾逮奇可。相檄上饒縣勘礦洞,知縣李鴻戒邑人敢以食物市者死。相竟日饑渴,憊而歸,乃螫鴻,罷其官。

橫嶺礦監王虎以廣昌民變,劾降易州知州孫大祚。

蘇、杭織造太監兼管稅務孫隆激民變,遍焚諸札委稅官家,隆急走杭州以免。

福建稅監高寀薦布政使陳性學,立擢巡撫。居閩十餘年,廣肆毒害。四十二年四月,萬眾洶洶欲殺寀,寀率甲士二百餘人入巡撫袁一驥署,露刃劫之,令諭眾退。復挾副使李思誠、僉事呂純如等至私署要盟,始釋一驥。復拘同知陳豸於署者久之。事聞,帝召寀還,命出豸,而一驥由此罷。

他若山東張曄、河南魯坤、四川丘乘雲輩,皆為民害。迨帝崩,始下遺詔罷礦稅,撤諸中使還京。

陳矩,安肅人。萬歷中,為司禮秉筆太監。二十六年提督東廠。為人平恕識大體。嘗奉詔收書籍,中有侍郎呂坤所著《閨範圖說》,帝以賜鄭貴妃,妃自為序,鋟諸木。時國本未定,或作《閨範圖說》跋,名曰《憂危竑議》,大指言貴妃欲奪儲位,坤陰助之,并及張養蒙、魏允貞等九人,語極妄誕。踰三年,皇太子立。

至三十一年十一月甲子昧爽,自朝房至勳戚大臣門,各有匿名書一帙,名曰《續憂危竑議》,言貴妃與大學士朱賡,戎政尚書王世揚,三邊總督李汶,保定巡撫孫瑋,少卿張養志,錦衣都督王之楨,千戶王名世、王承恩等相結,謀易太子,其言益妄誕不經。矩獲之以聞,大學士賡奏亦入。帝大怒,敕矩及錦衣衛大索,必得造妖書者。時大獄猝發,緝校交錯都下,以風影捕繫,所株連甚眾。之楨欲陷錦衣指揮周嘉慶,首輔沈一貫欲陷次輔沈鯉、侍郎郭正域,俱使人屬矩。矩正色拒之。已而百戶蔣臣捕皦生光至。生光者,京師無賴人也,嘗偽作富商包繼志詩,有「鄭主乘黃屋」之句,以脅國泰及繼志金,故人疑而捕之。酷訊不承,妻妾子弟皆掠治無完膚。矩心念生光即冤,然前罪已當死,且獄無主名,上必怒甚,恐輾轉攀累無已。禮部侍郎李廷機亦以生光前詩與妖書詞合。乃具獄,生光坐凌遲死。鯉、正域、嘉慶及株連者,皆賴矩得全。

三十三年掌司禮監,督廠如故。帝欲杖建言參政姜士昌,以矩諫而止。雲南民殺稅監楊榮,帝欲盡捕亂者,亦以矩言獲免。明年奉詔慮囚,御史曹學程以阻封日本酋關白事,繫獄且十年,法司請於矩求出,矩謝不敢。已而密白之,竟重釋,餘亦多所平反。又明年卒,賜祠額曰清忠。自馮保、張誠、張鯉相繼獲罪,其黨有所懲,不敢大肆。帝亦惡其黨盛,有缺多不補。迨晚年,用事者寥寥,東廠獄中至生青草。帝常膳舊以司禮輪供,後司禮無人,乾清宮管事牌子常雲獨辦,以故偵卒稀簡,中外相安。惟四方採榷者,帝實縱之,故貪殘肆虐,民心憤怨,尋致禍亂云。

王安,雄縣人,初隸馮保名下。萬曆二十二年,陳矩薦於帝,命為皇長子伴讀。時鄭貴妃謀立己子,數使人摭皇長子過。安善調護,貴妃無所得。「梃擊」事起,貴妃心懼。安為太子屬草,下令旨,釋群臣疑,以安貴妃。帝大悅。光宗即位,擢司禮秉筆太監,遇之甚厚。安用其客中書舍人汪文言言,勸帝行諸善政,發帑金濟邊,起用直臣鄒元標、王德完等,中外翕然稱賢。大學士劉一燝、給事中楊漣、御史左光斗等皆重之。

初,西宮李選侍怙寵陵熹宗生母王才人,安內忿不平。及光宗崩,選侍與心腹閹李進忠等謀挾皇長子自重,安發其謀於漣。漣偕一燝等入臨,安紿選侍抱皇長子出,擇吉即位,選侍移別宮去。事詳一燝等傳。熹宗心德安,言無不納。

安為人剛直而疏,又善病,不能數見帝。魏忠賢始進,自結於安名下魏朝,朝日夕譽忠賢,安信之。及安怒朝與忠賢爭客氏也,勒朝退,而忠賢、客氏日得志,忌安甚。天啟元年五月,帝命安掌司禮監,安以故事辭。客氏勸帝從其請,與忠賢謀殺之。忠賢猶豫未忍,客氏曰:「爾我孰若西李,而欲遺患耶?」忠賢意乃決,嗾給事中霍維華論安,降充南海子凈軍,而以劉朝為南海子提督,使殺安。劉朝者,李選侍私閹,故以移宮盜庫下獄宥出者。既至,絕安食。安取籬落中蘆菔啖之,三日猶不死,乃撲殺之。安死三年,忠賢遂誣東林諸人與安交通,興大獄,清流之禍烈矣。莊烈帝立,賜祠額曰昭忠。

魏忠賢,肅寧人。少無賴,與群惡少博,少勝,為所苦,恚而自宮,變姓名曰李進忠。其後乃復姓,賜名忠賢云。忠賢自萬曆中選入宮,隸太監孫暹,夤緣入甲字庫,又求為皇長孫母王才人典膳,諂事魏朝。朝數稱忠賢於安,安亦善遇之。長孫乳媼曰客氏,素私侍朝,所謂對食者也。及忠賢入,又通焉。客氏遂薄朝而愛忠賢,兩人深相結。

光宗崩,長孫嗣立,是為熹宗。忠賢、客氏並有寵。未踰月,封客氏奉聖夫人,廕其子侯國興、弟客光先及忠賢兄釗俱錦衣千戶。忠賢尋自惜薪司遷司禮秉筆太監兼提督寶和三店。忠賢不識字,例不當入司禮,以客氏故,得之。

天啟元年詔賜客氏香火田,敘忠賢治皇祖陵功。御史王心一諫,不聽。及帝大婚,御史畢佐周、劉蘭請遣客氏出外,大學士劉一燝亦言之。帝戀戀不忍舍,曰:「皇后幼,賴媼保護,俟皇祖大葬議之。」忠賢顓客氏,逐魏朝。又忌王安持正,謀殺之,盡斥安名下諸閹。客氏淫而狠。忠賢不知書,頗強記,猜忍陰毒,好諛。帝深信任此兩人,兩人勢益張,用司禮臨王體乾及李永貞、石元雅、塗文輔等為羽翼,宮中人莫敢忤。既而客氏出,復召入。御史周宗建、侍郎陳邦瞻、御史馬鳴起、給事中侯震暘先後力諍,俱被詰責。給事中倪思輝、朱欽相、王心一復言之,並謫外,尚未指及忠賢也。忠賢乃勸帝選武閹、煉火器為內操,密結大學士沈紘為援。又日引帝為倡優聲伎,狗馬射獵。刑部主事劉宗周首劾之,帝大怒,賴大學士葉向高救免。

初,神宗在位久,怠於政事,章奏多不省。廷臣漸立門戶,以危言激論相尚,國本之爭,指斥營禁。宰輔大臣為言者所彈擊,輒引疾避去。吏部郎顧憲成講學東林書院,海內士大夫多附之,「東林」之名自是始。既而「梃擊」、「紅丸」、「移宮」三案起,盈廷如聚訟。與東林忤者,眾目之為邪黨。天啟初,廢斥殆盡,識者已憂其過激變生。及忠賢勢成,其黨果謀倚之以傾東林。而徐大化、霍維華、孫傑首附忠賢,劉一燝及尚書周嘉謨並為傑劾去。然是時葉向高、韓爌方輔政,鄒元標、趙南星、王紀、高攀龍等皆居大僚,左光斗、魏大中、黃尊素等在言路,皆力持清議,忠賢未克逞。

二年敘慶陵功,廕忠賢弟姪錦衣衛指揮僉事。給事中惠世揚、尚書王紀論沈紘交通客、魏,俱被譴去。會初夏雨雹,周宗建言雹不以時,忠賢讒慝所致。修撰文震孟、太僕少卿滿朝薦相繼言之,亦俱黜。

三年春,引其私人魏廣微為大學士。令御史郭鞏訐宗建、一燝、元標及楊漣、周朝瑞等保舉熊廷弼,黨邪誤國。宗建駮鞏受忠賢指揮,御史方大任助宗建攻鞏及忠賢,皆不勝。其秋,詔忠賢及客氏子國興所廕錦衣官並世襲。兵部尚書董漢儒、給事中程註、御史汪泗論交諫,不從。忠賢益無忌,增置內操萬人,衷甲出入,恣為威虐。矯詔賜光宗選侍趙氏死。裕妃張氏有娠,客氏譖殺之。又革成妃李氏封。皇后張氏娠,客氏以計墮其胎,帝由此乏嗣。他所害宮嬪馮貴人等,太監王國臣、劉克敬、馬鑒等甚眾。禁掖事秘,莫詳也。是冬,兼掌東廠事。

四年,給事中傅櫆結忠賢甥傅應星為兄弟,誣奏中書汪文言,並及左光斗、魏大中。下文言鎮撫獄,將大行羅織。掌鎮撫劉僑受葉向高教,止坐文言。忠賢大怒,削僑籍,而以私人許顯純代。是時御史李應昇以內操諫,給事中霍守曲以忠賢乞祠額諫,御史劉廷佐以忠賢濫廕諫,給事中沈惟炳以立枷諫,忠賢皆矯旨詰責。於是副都御史楊漣憤甚,劾忠賢二十四大罪。疏上,忠賢懼,求解於韓廣不應,遂趨帝前泣訴,且辭東廠,而客氏從旁為剖析,體乾等翼之。帝懵然不辨也。遂溫諭留忠賢,而於次日下漣疏,嚴旨切責。漣既絀,魏大中及給事中陳良訓、許譽卿,撫寧侯朱國弼,南京兵部尚書陳道亨,侍郎岳元聲等七十餘人,交章論忠賢不法。向高及禮部尚書翁正春請遣忠賢歸私第以塞謗,不許。

當是時,忠賢憤甚,欲盡殺異己者。顧秉謙因陰籍其所忌姓名授忠賢,使以次斥逐。王體乾復昌言用廷杖,威脅廷臣。未幾,工部郎中萬燝上疏刺忠賢,立杖死。又以御史林汝翥事辱向高,向高遂致仕去,汝翥亦予杖。廷臣俱大讋。一時罷斥者,吏部尚書趙南星、左都御史高攀龍、吏部侍郎陳于廷及楊漣、左光斗、魏大中等先後數十人,已又逐韓爌及兵部侍郎李邦華。正人去國,紛紛若振槁。乃矯中旨召用例轉科道。以朱童蒙、郭允厚為太僕少卿,呂鵬雲、孫傑為大理丞,復霍維華、郭興治為給事中,徐景濂、賈繼春、楊維垣為御史,而起徐兆魁、王紹徽、喬應甲、徐紹吉、阮大鋮、陳爾翌、張養素、李應薦、李嵩、楊春懋等,為之爪牙。未幾,復用擬戍崔呈秀為御史。呈秀乃造《天鑒》、《同志》諸錄,王紹徽亦造《點將錄》,皆以鄒元標、顧憲成、葉向高、劉一燝等為魁,盡羅入不附忠賢者,號曰東林黨人,獻於忠賢。忠賢喜,於是群小益求媚忠賢,攘臂攻東林矣。

初,朝臣爭三案及辛亥、癸亥兩京察與熊廷弼獄事,忠賢本無預。其黨欲藉忠賢力傾諸正人,遂相率歸忠賢,稱義兒,且云:「東林將害翁。」以故,忠賢欲甘心焉。御史張訥、倪文煥,給事中李魯生,工部主事曹欽程等,競搏擊善類為報復。而御史梁夢環復興汪文言獄,下鎮撫司拷死。許顯純具爰書,詞連趙南星、楊漣等二十餘人,削籍遣戍有差。逮漣及左光斗、魏大中、周朝瑞、袁化中、顧大章等六人,至牽入熊廷弼案中,掠治死於獄。又殺廷弼,而杖其姻御史吳裕中至死。又削逐尚書李宗延、張問達,侍郎公鼐等五十餘人,朝署一空。而特召元詩教、劉述祖等為御史,私人悉不次超擢。於是忠賢之黨遍要津矣。

當是時,東廠番役橫行,所緝訪無論虛實輒糜爛。戚臣李承恩者,寧安大長公主子也,家藏公主賜器。忠賢誣以盜乘輿服御物,論死。中書吳懷賢讀楊漣疏,擊節稱歎。奴告之,斃懷賢,籍其家。武弁蔣應陽為廷弼訟冤,立誅死。民間偶語,或觸忠賢,輒被擒僇,甚至剝皮、刲舌,所殺不可勝數,道路以目。其年,敘門功,加恩三等,廕都督同知。又廕其族叔魏志德都督僉事。擢傅應星為左都督,且旌其母。而以魏良卿僉書錦衣衛,掌南鎮撫司事。

六年二月,鹵簿大駕成,廕都督僉事。復使其黨李永貞偽為浙江太監李實奏,逮治前應天巡撫周起元及江、浙里居諸臣高攀龍、周宗建、繆昌期、周順昌、黃尊素、李應昇等。攀龍赴水死,順昌等六人死獄中。蘇州民見順昌逮,不平,毆殺二校尉,巡撫毛一鷺為捕顏佩韋等五人悉誅死。刑部尚書徐兆魁治獄,視忠賢所怒,即坐大辟。又從霍維華言,命顧秉謙等修《三朝要典》,極意詆諸黨人惡。御史徐復陽請毀講學書院,以絕黨根。御史盧承欽又請立東林黨碑。海內皆屏息喪氣。霍維華遂教忠賢冒邊功矣。

遼陽男子武長春游妓家,有妄言,東廠擒之。許顯純掠治,故張其辭云:「長春敵間,不獲且為亂,賴廠臣忠智立奇勳。」詔封忠賢姪良卿為肅寧伯,賜宅第、莊田,頒鐵券。吏部尚書王紹徽請崇其先世,詔贈忠賢四代如本爵。忠賢又矯詔遣其黨太監劉應坤、陶文、紀用鎮山海關,收攬兵柄。再敘功,蔭都督同知,世襲錦衣衛指揮使,各一人。浙江巡撫潘汝楨奏請為忠賢建祠。倉場總督薛貞言草場火,以忠賢救,得無害。於是頌功德者相繼,諸祠皆自此始矣。

編修吳孔嘉與宗人吳養春有仇,誘養春僕告其主隱占黃山,養春父子瘐死。忠賢遣主事呂下問、評事許志吉先後往徽州籍其家,株蔓殘酷。知府石萬程不忍,削髮去,徽州幾亂。其黨都督張體乾誣揚州知府劉鐸代李承恩謀釋獄,結道士方景陽詛忠賢,鐸竟斬。又以睚眥怨,誣新城侯子錦衣王國興,論斬,並黜主事徐石麒。御史門克新誣吳人顧同寅、孫文豸誄熊廷弼,坐妖言律斬。又逮侍郎王之寀,斃於獄。凡忠賢所宿恨,若韓爌、張問達、何士晉、程註等,雖已去,必削籍,重或充軍,死必追贓破其家。或忠賢偶忘之,其黨必追論前事,激忠賢怒。

當此之時,內外大權一歸忠賢。內豎自王體乾等外,又有李朝欽、王朝輔、孫進、王國泰、梁棟等三十餘人,為左右擁護。外廷文臣則崔呈秀、田吉、吳淳夫、李夔龍、倪文煥主謀議,號「五虎」。武臣則田爾耕、許顯純、孫雲鶴、楊寰、崔應元主殺僇,號「五彪」。又吏部尚書周應秋、太僕少卿曹欽程等,號「十狗」。又有「十孩兒」、「四十孫」之號。而為呈秀輩門下者,又不可數計。自內閣、六部至四方總督、巡撫,遍置死黨。心忌張皇后,其年秋,誣后父張國紀縱奴不法,矯中宮旨,冀搖后。帝為致奴法,而誚讓國紀。忠賢未慊,復使順天府丞劉志選、御史梁夢環交發國紀罪狀,並言后非國紀女。會王體乾危言沮之,乃止。

其冬,三殿成。李永貞、周應秋奏忠賢功,遂進上公,加恩三等。魏良卿時已晉肅寧侯矣,亦晉寧國公,食祿如魏國公例,再加恩蔭錦衣指揮使一人,同知一人。工部尚書薛鳳翔奏給賜第。已而太監陶文奏築喜峰隘口成,督師王之臣奏築山海城,刑部尚書薛貞奏大盜王之錦獄,南京修孝陵工竣,甘鎮奏捷,蕃育署丞張永祚獲盜,並言忠賢區畫方略。忠賢又自奏三年緝捕功,詔書褒獎。半歲中,所蔭錦衣指揮使四人、同知三人、僉事一人。授其姪希孟世襲錦衣同知,甥傅之琮、馮繼先並都督僉事,而擢崔呈秀弟凝秀為薊鎮副總兵。名器僭濫,於是為極。其同類盡鎮薊、遼,山西宣、大諸阨要地。總兵梁柱朝、楊國棟等歲時賂名馬、珍玩絕。

七年春,復以崔文昇總漕運,李明道總河道,胡良輔鎮天津。文升故侍光宗藥,為東林所攻者也。海內爭望風獻諂,諸督撫大吏閻鳴泰、劉詔、李精白、姚宗文等,爭頌德立祠,洶洶若不及。下及武夫、賈豎、諸無賴子亦各建祠。窮極工巧。攘奪民田廬,斬伐墓木,莫敢控愬。而監生陸萬齡至請以忠賢配孔子,以忠賢父配啟聖公。

初,潘汝禎首上疏,御史劉之待會槁遲一日,即削籍。而薊州道胡士容以不具建祠文,遵化道耿如巳入祠不拜,皆下獄論死。故天下風靡,章奏無巨細,輒頌忠賢。宗室若楚王華煃、中書硃慎鑒,勛戚若豐城侯李永祚,廷臣若尚書邵輔忠、李養德、曹思誠,總督張我續及孫國楨、張翌明、郭允厚、楊維和、李時馨、汪若極、何廷樞、楊維新、陳維新、陳歡翼、郭如闇、郭希禹、徐溶輩,佞詞累牘,不顧羞恥。忠賢亦時加恩澤以報之。所有疏,咸稱「廠臣」不名。大學士黃立極、施鳳來、張瑞圖票旨,亦必曰「朕與廠臣」,無敢名忠賢者。山東產麒麟,巡撫李精白圖象以聞。立極等票旨云:「廠臣修德,故仁獸至。」其誣罔若此。前後賜獎敕無算,誥命皆擬九錫文。

是年自春及秋,忠賢冒款汪燒餅、擒阿班歹羅銕等功,積蔭錦衣指揮使至十有七人。其族孫希孔、希孟、希堯、希舜、鵬程,姻戚董芳名、王選、楊六奇、楊祚昌,皆至左、右都督及都督同知、僉事等官。又加客氏弟光先亦都督。魏撫民又從錦衣改尚寶卿。而忠賢志願猶未極,會袁崇煥奏寧遠捷,忠賢乃令周應秋奏封其從孫鵬翼為安平伯。再敘三大工功,封從子良棟為東安侯,加良卿太師,鵬翼少師,良棟太子太保。因遍賚諸廷臣。用呈秀為兵部尚書兼左都御史,獨絀崇煥功不錄。時鵬翼、良棟皆在襁褓中,未能行步也。良卿至代天子饗南北郊,祭太廟。於是天下皆疑忠賢竊神器矣。

帝性機巧,好親斧鋸髹漆之事,積歲不倦。每引繩削墨時,忠賢輩輒奏事。帝厭之,謬曰:「朕已悉矣,汝輩好為之。」忠賢以是恣威福惟己意。歲數出,輒坐文軒,羽幢青蓋,四馬若飛,鐃鼓鳴鏑之聲,轟隱黃埃中。錦衣玉帶靴褲握刀者,夾左右馳,廚傳、優伶、百戲、輿隸相隨屬以萬數。百司章奏,置急足馳白乃下。所過,士大夫遮道拜伏,至呼九千歲,忠賢顧盼未嘗及也。客氏居宮中,脅持皇后,殘虐宮嬪。偶出歸私第,騶從赫奕照衢路,望若鹵簿。忠賢故騃無他長,其黨日夜教之,客氏為內主,群凶煽虐,以是毒痡海內。

七年秋八月,熹宗崩,信王立。王素稔忠賢惡,深自儆備,其黨自危。楊所修、楊維垣先攻崔呈秀以嘗帝,主事陸澄原、錢元愨,員外郎史躬盛遂交章論忠賢。帝猶未發。於是嘉興貢生錢嘉徵劾忠賢十大罪:一並帝,二蔑后,三弄兵,四無二祖列宗,五剋削籓封,六無聖,七濫爵,八掩邊功,九朘民,十通關節。疏上,帝召忠賢,使內侍讀之。忠賢大懼,急以重寶啖信邸太監徐應元求解。應元,故忠賢博徒也。帝知之,斥應元。十一月,遂安置忠賢於鳳陽,尋命逮治。忠賢行至阜城,聞之,與李朝欽偕縊死。詔磔其屍。懸首河間。笞殺客氏於浣衣局。魏良卿、侯國興、客光先等並棄市,籍其家。客氏之籍也,於其家得宮女八人,蓋將效呂不韋所為,人尤疾之。

崇禎二年命大學士韓爌等定逆案,始蓋逐忠賢黨,東林諸人復進用。諸麗逆案者日夜圖報復。其後溫體仁、薛國觀輩相繼柄政,潛傾正人,為翻逆案地。帝亦厭廷臣黨比,復委用中當。而逆案中阮大鋮等卒肆毒江左,至於滅亡。

王體乾、李永貞、塗文輔,皆忠賢黨。體乾,昌平人,柔佞深險。熹宗初,為尚膳太監,遷司禮秉筆。王安之辭司禮掌印也,體乾急謀於客、魏奪之,而置安於死。用是,一意附忠賢,為之盡力。故事,司禮掌印者位東廠上。體乾避忠賢,獨處其下,故忠賢一無所忌。楊漣劾忠賢疏上,帝命體乾誦之,置疏中切要語不讀,漣遂得譴。萬燝之死,出體乾意。忠賢不識字,體乾與永貞等為之謀主,遇票紅文書及改票,動請御筆,體乾獨奏,忠賢默然也。及忠賢冒陵工、殿工、邊功等賞,體乾、永貞輩亦各蔭錦衣官數人。嘗疑選人受益、黃願素為錢謙益、黃尊素兄弟,欲並柰錮,其阿媚忠賢如此。及莊烈帝定逆案,革體乾職,籍其家。

永貞,通州人。萬曆中為內侍,犯法被繫者十八年,光宗立,得釋。忠賢用事,引其黨諸棟、史賓等為秉筆。永貞入棟幕,與忠賢掌班劉榮為死友。棟死,夤緣得通於忠賢,由文書房升秉筆太監,匝月五遷,與體乾、文輔及石元雅共為忠賢心腹。凡章奏入,永貞等先鈐識窾要,白忠賢議行。崔呈秀所獻諸錄,永貞等各置小冊袖中,遇有處分,則爭出冊告曰:「此某錄中人也。」故無得免者。永貞性貪,督三殿工,治信王邸,所侵沒無算。莊烈帝立,永貞陽引退,行十五萬金於體乾及司體王永祚、王本政求援。三人惡其反覆,首於帝。永貞懼,遂亡去。既而被獲,謫鳳陽,尋以偽草李實奏,逮至,伏誅。

文輔,初為客氏子侯國興授讀,諂附忠賢,由司禮秉筆歷掌御馬監,總督太倉、節慎二庫。奪寧安大長公主第為廨,署曰「戶工總部。」騶從常數百人,部郎以下皆庭參,勢焰出群閹上。莊烈帝立,復附徐應元,謫南京。

時有劉若愚者,故隸陳矩名下。善書,好學有文。天啟初,李永貞取入內直房,主筆札。永貞多密謀,若愚心識之,不敢與外廷通。忠賢敗,若愚為楊維坦所劾,充孝陵凈軍。已,御史劉重慶以李實誣高攀龍等七人事劾實。實疏辨言係空印紙,乃忠賢逼取之,令永貞填書者。帝驗疏,墨在硃上,遂誅永貞,坐若愚大辟。久之,得釋。若愚當忠賢時,祿賜未嘗一及,既幽囚,痛己之冤,而恨體乾、文輔輩之得漏網也,作《酌中志》以自明,凡四卷,見者鄰之。

崔文昇者,鄭貴妃宮中內侍也。光宗立,陞司禮秉筆,掌御藥房。時貴妃進帝美女四人,帝幸焉,既而有疾。文升用大黃藥,益劇,不視朝。外廷洶洶,皆言文昇受貴妃指,有異謀。給事中楊漣言:「陛下哀毀之餘,萬幾勞瘁。文升誤用伐藥,又構造流言,謂侍御蠱惑,損陛下令名。陛下奈何置賊臣於肘腋間哉!」然構造之說,漣疑文昇誤用藥,故為此以圖御罪,其實出於文昇果否,未知也。未幾,光宗服鴻臚丞李可灼紅丸,遂崩。言者交攻可灼及閣臣方從哲,惟御史鄭宗周等直指文昇。給事中魏大中言文昇之惡不下張差,御史吳甡亦謂其罪浮河灼。下廷議,可灼論戍,文昇謫南京。及忠賢用事,召文昇總督漕運兼管河道。莊烈帝即位,召回。御史吳煥復劾之。疏甫上,文昇即結同黨伏宮門號哭,聲徹御座。帝大怒,並其黨皆杖一百,充孝陵凈軍。

張彞憲,莊烈帝朝司禮太監也。帝初即位,鑒魏忠賢禍敗,盡撤諸方鎮守中官,委任大臣。既而廷臣競門戶,兵敗餉絀,不能贊一策,乃思復用近侍。崇禎四年九月,遣王應朝等監視關、寧,又遣王坤宣府,劉文忠大同,劉允中山西,監視軍馬。而以彞憲有心計,令鉤校戶、工二部出入,如塗文輔故事,為之建署,名曰戶工總理,其權視外總督,內團營提督焉。給事中宋可久、馮元飆等十餘人論諫,不納。吏部尚書閔洪學率朝臣具公疏爭,帝曰:「茍群臣殫心為國,朕何事乎內臣。」眾莫敢對。南京侍郎呂維祺疏責輔臣不能匡救,禮部侍郎李孫宸亦以召對力諫,俱不聽。彞憲遂按行兩部,踞尚書上,命郎中以下謁見。工部侍郎高弘圖不為下,抗疏乞歸,削籍去。彞憲益驕縱,故勒邊鎮軍器不發。管盔甲主事孫肇興恐稽滯軍事,因劾其誤國。帝命回奏,罪至遣戍。主事金鉉、周鑣皆以諫斥去。工部尚書周士樸以不赴彞憲期,被詰問,罷去。

是時,中璫勢復大振。王坤至宣府,甫踰月,即劾巡按御史胡良機。帝落良機職,命坤按治。給事中魏呈潤爭之,亦謫外。坤性狂躁敢言,朝中大吏有欲倚之相傾擠者。於是坤抗疏劾修撰陳於泰,謂其盜竊科名,語侵周延儒。給事中傅朝佑言坤妄干彈劾之權,且其文詞練達,機鋒擋激,必有陰邪險人主之,其意指溫體仁。帝置不問。左副都御史王志道言:「近者內臣舉動,幾於手握皇綱,而輔臣終不敢一問。至於身被彈擊,猶忍辱不言。何以副明主之知?」皆備責延儒,欲以動帝。帝怒,削其籍。時帝方一意用內臣,故言者多得罪。

到八年八月始下詔曰:「往以廷臣不職,故委寄內侍。今兵制麤立,軍餉稍清,盡撤監視總理。」又明年,命彞憲守備南京,尋死。然帝卒用高起潛輩典兵監鎮,馴至開關延賊,遂底滅亡。

高起潛,在內侍中,以知兵稱,帝委任之。五年命偕其儕呂直督諸將征孔有德於登州,明年凱旋。時流賊大熾,命太監陳大金、閻思印、謝文舉、孫茂霖等為內中軍,分入大帥曹文詔、左良玉、張應昌諸營,名曰監軍,在邊鎮者,悉名監視。而起潛得監視寧、錦諸軍。已而諸監多侵剋軍資,臨敵輒擁精兵先遁,諸將亦恥為之下,緣是皆無功。八年盡撤諸鎮內臣,惟起潛監視如故。

九年七月復遣太監李國輔、許進忠等分守紫荊、倒馬諸關,孫惟武、劉元斌防馬水河。時兵部尚書張鳳翼出督援軍,宣大總督梁廷棟亦引兵南,特命起潛為總監,給金三萬、賞功牌千,以司禮大璫張雲漢、韓贊周副之。然起潛實未嘗決一戰,惟割死人首冒功而已。明年,起潛行部視師,令監司以下悉用軍禮。永平道劉景耀、關內道楊於國疏爭,被黜。既而與兵部尚書楊嗣昌比,致宣大總督盧象昇孤軍戰歿,又匿不言狀,人多疾之。

十七年,李自成將犯闕,帝復命起潛監寧、前諸軍,而以杜勳鎮宣府。勛至鎮即降賊。事聞,廷臣請急撤城守太監,忽傳旨云:「杜勛罵賊殉難,予廕祠。」蓋為內臣蒙蔽也。未幾,勳從賊至,自成設黃幄坐廣寧門外,秦、晉二王左右席地坐,勳侍其下,呼城上請入見。守城諸璫縋之上,同入大內,盛稱賊勢,勸帝自為計。左右請留之,勛曰:「不返,則二王危。」乃縱之出,復縋下,語守城諸璫曰:「吾曹富貴固在也。」俄而城陷,諸璫皆降。及賊敗將遁,乃下令盡逐內豎,無貴賤老弱皆號哭徒跣,破面流血,走出京城門。賊遂捆載其金帛珠寶西去。

初,內臣奉命守城,已有異志,令士卒皆持白楊杖,朱其外,貫鐵環於端使有聲,格擊則折,至是賊即以其杖驅焉。廣寧門之啟,或日太監曹化淳獻之,或曰化淳實守東直門,而化淳入國朝,上疏奏辨甚力,時倉卒莫能明也。起潛赴寧、前,中道棄關走。福王召為京營提督,後亦降於我大清。

王承恩,太監曹化淳名下也,累官司禮秉筆太監。崇禎十七年三月,李自成犯闕,帝命承恩提督京營。是時,事勢已去,城陴守卒寥寥,賊架飛梯攻西直、平則、德勝三門。承恩見賊坎牆,急發炮擊之,連斃數人,而諸璫泄泄自如。帝召承恩,令亟整內官,備親征。夜分,內城陷。天將曙,帝崩於壽皇亭,承恩即自縊其下。福王時,謚忠愍。本朝賜地六十畝,建祠立碑旌其忠,附葬故主陵側。

方正化,山東人。崇禎時,為司禮太監。十五年冬,畿輔被兵,命總監保定軍務,有全城功,已而撤還。十七年二月復命出鎮,正化頓首辭,帝不許。又頓首曰:「奴此行萬無能為,不過一死報主恩爾。」帝亦垂涕遣之。既至,與同知邵宗元等登陴共守。有請事者,但曰:「我方寸已亂,諸公好為之。」及城陷,擊殺數十人,賊問:「若為誰?」厲聲曰:「我總監方公也!」賊攢刀斫殺之,其從奄皆死。時內臣殉難者,更有故司禮掌印太監高時明,司禮秉筆太監李鳳翔,提督諸監局太監褚憲章、張國元四人。督東廠太監王之心家最富,既降,賊勒其貲,拷死。南渡時,建旌忠祠祀諸死難者,以王承恩為正祀,內臣正化等附祀,而之心亦濫與焉。


\end{pinyinscope}