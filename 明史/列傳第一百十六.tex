\article{列傳第一百十六}

\begin{pinyinscope}
魏學曾葉夢熊梅國楨李化龍江鐸

魏學曾,字惟貫,涇陽人。嘉靖三十二年進士。除戶部主事,遷郎中。中官為商人請支芻糧銀巨萬,學曾持不可,乃已。尋擢光祿少卿,進右僉都御史,巡撫遼東。隆慶初,土蠻大入永平。學曾入駐山海,檄諸將王治道等追擊至義院口,大捷。進右副都御史。學曾乃易置將吏,招納降附,釐屯田二千餘頃,數破敵,被賞賚。以疾去。起兵部右侍郎,提督神樞營。旋改吏部,轉左侍郎。

穆宗崩,大學士高拱欲去馮保,屬言官論劾。學曾遺書大學士張居正曰:「外人皆言公與保有謀,遺詔亦出公手。今日之事,不宜復護此閹。」居正怒。及拱被逐,舉朝失色,學曾獨大言曰:「上踐阼伊始,輒逐顧命大臣,且詔出何人,不可不明示百官。」要諸大臣詣居正邸爭之。諸大臣多不往,居正亦辭以疾。自是益忤。出為南京右都御史。未上,給事中宗弘暹希居正指劾之。詔以故官候調,學曾遂歸。居正歿逾年,起南京戶部右侍郎。召為右都御史,督倉場。尋以南京戶部尚書致仕。

萬曆十八年,順義王扯撦力克西赴青海,火落赤、真相犯洮河,副總兵李奎、李聯芳先後被殺。朝命尚書鄭洛經略七鎮,兼領總督,洛固辭總督。明年春,閣臣王錫爵薦學曾。起兵部尚書,總督陜西、延、寧、甘肅軍務。時洛專主款,學曾至,與議不合,陜西巡撫葉夢熊助之。初,順義王封,夢熊以諫沮坐得罪,學曾亦為高拱言不便。至是,撦力克助叛,學曾、夢熊欲遂討之,詆洛玩寇。會撦力克東歸,火落赤諸部亦徙去,學曾奏撦力克雖歸,陰留精兵二萬於嘉峪,欲助火落赤、真相。其說本採諸道路,朝士乃爭附和之。錫爵意悔,具疏言狀,又遺書責夢熊。而兵部尚書石星以順義既東,宣、大事急,召洛還定撫議,置學曾疏不問。未幾,河套部長土昧明安入市畢,要請增賞。學曾令總兵官杜桐、神木參將張剛、孤山游擊李紹祖出不意擊斬明安,俘馘四百八十餘級,奪馬畜器械稱是。學曾以功加太子少保。而明安子擺言太聲言復仇,號召諸部。

明年,哱拜反,遂煽諸部為亂。拜,西部人也。嘉靖中得罪其部長,父兄皆見殺,拜跳脫來降,驍勇屢立戰功。前督撫王崇古、石茂華先後奏加副總兵,遂多畜亡命。子承恩,拜夢妖物入妻施脅而生,狼形梟啼,性狠戾。拜老,承恩襲父爵。十九年,洮、河告警,御史周弘禴舉承恩及指揮土文秀、拜義子哱雲等。巡撫黨馨檄文秀西援,拜謁經略鄭洛,願與子承恩從出師。馨惡其自薦,抑損之,拜以故心怨。至金城,見諸鎮兵皆出其下。比賊退,取道塞外還,寇騎遇之皆辟易,遂有輕中外心。馨數裁拜,且按承恩罪箠之二十,雲、文秀亦以他故怨馨。會戍卒請衣糧久弗給,拜遂嗾軍鋒劉東暘、許朝作亂。二十年三月,殺馨及副使石繼芳,逼總兵官張維忠縊死。雲、文秀殺游擊梁琦、守備馬承光,東暘稱總兵,奉拜為謀主,承恩、朝為左、右副總兵,雲、文秀為左、右參將。承恩遂陷玉泉營、中衛、廣武,河西望風靡。惟文秀徇平虜,參將蕭如薰堅守不下。賊既取河西四十七堡,且渡河,復誘河套著力兔、宰僧犯平虜、花馬池。全陜皆震動。

學曾檄副總兵李昫率游擊吳顯趨靈州,別遣游擊趙武趨鳴沙州,沿河扼賊南渡,而自駐花馬池,當賊衛。昫等渡河,賊將多遁去,四十七堡皆復,惟寧夏鎮城尚為賊據。著力兔等中外相呼應,拜、文秀攻趙武於玉泉。雲引著力兔攻平虜,如薰設伏射殺雲。昫救武,圍亦解。四月,昫引兵與故總兵牛秉忠抵鎮城下。帝已擢董一奎為總兵,李蕡副之,已,復擢如薰代一奎,而以麻貴代蕡。未至,昫等攻城。賊於東西二門各出驍騎三千搏戰,步卒列火車為營。官軍擊之,奪其車百輛,追奔入湖,賊溺死無算。副總兵王通戰尤力。家丁高益等乘勝入北門,後兵不繼被殺,通亦負傷,榆林游擊俞尚德戰死。翼日,朝、文秀脅慶王上東城,乞暫罷兵,詭言願獻首惡。會官軍糧盡,乃引退,休近堡。

學曾日夜趣芻餉,調延綏、莊浪、蘭、靖、榆林兵。道回遠,所治舟亦未具,乃駐花馬池,俟軍至移靈州。頃之,延綏游擊姜顯謨、都司蕭如蕙,甘州故總兵張傑及麻貴軍皆至,復抵鎮城攻之。賊計延綏、榆林兵出內虛,勾黃臺吉妻,令其子捨達大、從子火落赤、土昧鐵雷掠舊安邊、磚井堡以牽我兵。承恩復以間合寇兵,伏延漢渠,掠糧車二百。學曾自花馬池還靈州,被圍,救至而解。貴等數攻城不能克,賊殺慶王妃,盡掠其宮人金帛。牛秉忠戰傷右股,乃復退師。帝用尚書星言,賜學曾尚方劍督戰。會寧夏巡撫硃正色、甘肅巡撫葉夢熊、監軍御史梅國楨,諸大將劉承嗣、董一奎、李如松先後至軍,六月復攻城,連戰不下。

夢熊,字男兆,歸善人。嘉靖四十年進士。由福清知縣入為戶部主事,轉餉寧夏。改御史,以諫受把漢那吉降,貶鄖陽丞。累遷贛州知府,平黃鄉賊。遷浙江副使,改永平。萬曆十七年冬,由山東布政使擢右僉都御史,巡撫貴州。尋改陜西,進右副都御史。以請討撦力克,與經略洛議相左。廷議方右洛,絀其議不用。會撦力克東歸,洛亦還宣、大,乃移夢熊甘肅,與學曾共事。夢熊有膽決,敢任事。會拜反,上疏自請討賊,帝然之。以六月至靈州,與學曾合。

國楨,字克生,麻城人。少雄傑自喜,善騎射。舉萬曆十一年進士。除固安知縣。中官詣國楨請收責於民,國楨偽令民鬻妻以償。民夫婦哀慟,中官為毀券。擢御史,會拜反,學曾師久無功。時寧遠伯李成梁方被論,廷議欲遣為大將,未敢決,國楨獨疏保之。乃遣成梁子如松為提督,將遼東、宣、大、山西諸鎮兵以往。而國楨監其軍,遂與如松至寧夏。

初,學曾欲招東暘、朝,令殺拜父子贖罪,遣卒葉得新往。四人方約同死,折得新脛,置之獄。巡撫朱正色以賊詭請降,而張傑嘗總寧夏兵,故與拜善,遣傑入城招之。朝乃舁得新見傑,得新大罵賊,被殺,傑亦系不遣。而學曾以賊求撫為之請,帝切責。及是,城中百戶姚欽、武生張遐齡射書城外,約內應,夜半舉火。外兵不至,賊殺其黨五十人,欽縋城出,來奔。當是時,賊外以求撫緩兵,而陰結寇為助,然糧盡,勢且困。七月,學曾與夢熊、國楨定計,決黃河大壩水灌之,水抵城下。時套寇卜失兔、莊禿賴以三萬騎犯定邊、小鹽池,用土昧鐵雷為前鋒,而別遣宰僧以萬騎從花馬池西沙湃口入,為拜聲援。麻貴擊之右溝,寇稍挫,分趨下馬關及鳴沙洲。學曾令游擊龔子敬扼沙湃口,而檄延綏總兵官董一元搗土昧鐵雷巢,斬首百三十餘級,寇大驚引去。遇子敬,圍之十重,子敬死,寇亦去,賊援遂絕。學曾益決大壩水。八月,河決隄壞,復繕治之,城外水深八九尺,東西城崩百餘丈。著力兔、宰僧復入李剛堡。如松、貴等擊敗之,追奔至賀蘭山。賊益懼求款,未決,會學曾得罪罷。朝命以夢熊代,夢熊遂成功。

初,學曾之遣人招東暘、朝也,留固原十餘日以俟之,帝責其玩寇;李昫渡河又稍遲,松山、河套寇先入,官軍用是再失利。學曾嘗上疏令監軍無與兵事,帝為飭國楨如其言,國楨頗憾之。及至軍,劾諸將觀望,而頗以玩寇為學曾罪。給事中許子偉亦劾學曾惑於招撫,誤國事。國楨又言僉事隨府從城上躍下,賊令四人下取,我軍咫尺不敢前;又北寇數萬斷我糧道,殺戮無算,匿不以奏。帝遂大怒,逮學曾至京。然學曾逮未逾月,城壞而大軍入,賊竟以破滅。

夢熊既代學曾,亦賜尚方劍。時調度靈州,獨國楨監軍寧夏。賊被圍久,食盡無援,而城受水浸,益大崩。國楨挾諸將趨南關。秉忠先登,國楨大呼,諸將畢登。賊退據大城,攻數日不下。國楨使間紿東暘、朝、承恩互相殺,以降貰其罪。三人內猜疑,東暘、朝遂先誘殺承恩黨文秀。承恩亦與其黨周國柱誘東暘、朝殺之;盡懸東暘、朝、文秀首城上,開門降。如松率兵圍拜家。拜倉皇縊,闔室自焚死。夢熊自靈州馳至,下令盡誅拜黨及降人二千,慰問宗室士庶。寧夏平。夢熊、正色、國楨各上捷奏,而俘承恩獻京師。帝御門受賀,詔磔承恩於市,夢熊、正色、國楨各廕世官,如松功第一,如薰、貴、秉忠等加恩有差。學曾初奪職為民,敘功,以原官致仕。

學曾任事勞勩。灌城招降之策,本其所建。及宣捷,帝召見大學士趙志皋、張位,志皋、位力為學曾解,尚書星以下多白學曾無罪。國楨亦上疏言:「學曾應變稍緩,臣請責諸將以振士氣,而逮學曾之命,發自臣疏,竊自悔恨。學曾不早雪,臣將受萬世譏。」如松亦言:「學曾被逮時,三軍雨泣。」夢熊亦推功學曾。帝初不聽,既而復其官。居家數年卒。夢熊以功進右都御史。

初,卜失兔為都督,其部長切盡臺吉最用事。切盡臺吉死,卜失兔不能制諸部。經略鄭洛專事羈縻。學曾以洮河之變,惡諸部為逆,襲殺明安。會拜反,著力兔、宰僧遂聲言與拜為一家,而卜失兔、莊禿賴亦引兵助之。及拜誅,切盡臺吉之比吉率著力兔、宰僧、莊禿賴等頓首花馬池塞下,悔罪求款。夢熊為奏請。帝以夢熊初主學曾,責其前後異議,令要諸部縛叛贖罪。著力兔等求款益堅,夢熊乃與巡撫田樂奏上四鎮款戰機宜,俟朝議。中外相仗莫敢決,卜失兔遂率諸部大入定邊。總兵官麻貴等擊卻之,夢熊以功加太子少保。未幾,切盡臺吉從子青把都兒犯甘肅,總兵官楊浚、副總兵何崇德御之,斬首六百餘級。夢熊復加太子太保、兵部尚書。尋入為南京工部尚書,而以都御史李汶代。自洮河變後,寇頗輕中國。招撫議既絕,諸部數入犯,四鎮遂頻歲用兵云。夢熊雖功多,其品望遠出學曾下。卒官。

國楨既招降承恩,以夢熊貪功殺降,劾其罪。夢熊奏辨,言:「拜所畜家人皆死士,緩一二日,東暘、朝黨復集,必再亂。臣寧負殺降名,以絕禍本。」帝為下詔和解之。論功,擢國楨太僕少卿。逾年,遷右僉都御史,巡撫大同。久之,遷兵部右侍郎,總督宣、大、山西軍務。在鎮三年,節省市賞銀十五萬兩有奇。父喪歸,未起而卒。贈右都御史。

李化龍,字于田,長垣人。萬歷二年進士。除嵩縣知縣。年甫二十,胥吏易之。化龍陰察其奸,悉召置之法,縣中大治。遷南京工部主事,歷右通政使。

二十二年夏,擢右僉都御史,巡撫遼東。初,總兵官李成梁破殺泰寧速把亥,其子把兔兒弟炒花據舊遼陽以北,居兩河之中,益結土蠻為患。其年四月,把兔兒圍遼陽,朵顏小歹青、福餘伯言兒分犯錦、義,掠清細河,巡撫韓取善坐免。化龍受事甫兩月,把兔兒與伯言兒等寇鎮武,又約土蠻子卜言臺周犯右屯。把兔兒先至吳家墳。化龍與總兵官董一元定計先擊把兔、伯言兒,伯言兒中流矢死,把兔被傷。卜言臺周至,攻右屯不利,亦解去。於是把兔、小歹青、卜言台周益相結,謀復前恥。化龍與一元嚴備之。一元又出塞,搗巢有功,而把兔傷重竟死,邊塞襲服。詳具一元傳。化龍進兵部右侍郎。

明年,小歹青悔禍款塞,請開木市於義州,且告朵顏長昂將犯邊。已,長昂果犯錦、義,副總兵李如梅擊卻之。歹青言既信,化龍遂許其請。上疏曰:

環遼皆敵也,迤北土蠻種類多不可數。近邊者,直寧前則長昂,直錦、義則小歹青,直廣寧、遼、沈則把兔、炒花、花大,直開、鐵則伯言、燒兔,其在東邊海西則猛骨孛羅、那林孛羅、卜寨,皆與遼地項背相望。並墻圍獵,則刁鬥聲相聞,蓋肘腋憂也。自那卜被剿,數年東陲無事。去年把兔、伯言戰死,炒花、花大一敗塗地。今伯言子宰賽受罰,入市廣寧,遼、沈、開、鐵間警報漸希。所未馴伏者,惟小歹青與長昂耳。

小歹青素兇狡,雄長諸部。西助長昂,東助炒花。大舉動以數萬,小竊則飛騎出沒錦、義間。自周之望、柏朝翠戰歿,無敢以一矢加遺。凌河上下方數百里,野多暴骨,民無寧宇。遠慮者每以河西不保為虞。今乃叩關求市,臣遍詢將領及彼地居民,僉言木市開有五利。

河西無木,皆在邊外,叛亂以來,仰給河東,以邊警又不時至。故河西木貴於玉,市通則材木不可勝用。利一。所疑於歹青者,無信耳。彼重市為生路,當市時必不行掠。即今年市而明年掠,我已收今年不掠之利矣。利二。遼東馬市,成祖所開,無他賞,本聽商民與交易。木市與馬市等,有利於民,不費於官。利三。大舉之害酷而希,零竊之害輕而數。小歹青不掠錦、義,零竊少矣。又西不助長昂,東不助炒花,則敵勢漸分。即寧前、廣寧患亦漸減。且大舉先報,又得預為備。利四。零竊既希,邊人益得修備。利五。

疏入,從之。化龍尋以病去,木市亦停止。其後總兵官馬林復議開市,與巡撫李植相左,論久不決,小歹青遂復為寇雲。

二十七年三月,化龍起故官,總督湖廣、川、貴軍務兼巡撫四川,討播州叛臣楊應龍。應龍之先曰楊鏗。明初內附,授宣慰使。應龍性猜狠嗜殺。數從徵調,恃功驕蹇。知川兵脆弱,陰有據蜀志,間出剽州縣。嬖小妻田雌鳳,讒殺妻張氏,屠其家。用誅罰立威,所屬五司七姓不堪其虐,走貴州告變。巡撫葉夢熊疏請大征。詔不聽,逮繫重慶獄。應龍詭將兵征倭自效,得脫歸。復逮,不出。四川巡撫王繼光發兵討,覆於白石,應龍諉罪諸苗。朝廷命邢玠總督。值東西用兵,勢未能窮治,因招撫之。應龍益結生苗,奪五司七姓地,並湖廣四十八屯以畀之,歲出侵掠。是年二月,敗官軍於飛練堡,都司楊國柱、指揮李廷棟等皆死。已,復破殺綦江參將房嘉寵、游擊張良賢,投屍蔽江下。偽軍師孫時泰請直取重慶,搗成都,劫蜀王為質,而應龍遷延,聲言爭地界,冀曲赦如曩時。化龍至成都,徵兵未至,亦謬為好語縻之。

帝聞綦江破,大怒。追褫前四川、貴州巡撫譚希思、江東之職,而賜化龍劍,假便宜討賊。賊焚東坡、爛橋,梗湖、貴路,又焚龍泉,走都司楊惟忠。化龍劾諸大帥不用命者,沈尚文逮治,童元鎮、劉廷皆革職充為事官。諸軍大集,化龍先檄水西兵三萬守貴州,斷招苗路,乃移重慶,大誓文武。明年二月,分八道進兵。川師四路:總兵官劉綎由綦江,總兵官馬孔英由南川,總兵官吳廣由合江,副將曹希彬受廣節制,由永寧。黔師三路:總兵官童元鎮由烏江,參將硃鶴齡受元鎮節制,統宣慰使安疆臣由沙溪,總兵官李應祥由興隆。楚師一路分兩翼:總兵官陳璘由偏橋,副總兵陳良玭受璘節制,由龍泉。每路兵三萬,官兵三之,土司七之。貴州巡撫郭子章駐貴陽,湖廣巡撫支可大移沅州,化龍自將中軍策應。帝以楚地遼闊,又擢江鐸為僉都御史,巡撫偏、沅。湖廣設偏沅巡撫,自鐸始也。

推官高折枝先以南川兵進,據桑木鎮,綎復自綦江入。應龍以勁兵二萬屬其子朝棟曰:「爾破綦江,馳南川,盡焚積聚,彼無能為也。」比抗諸路兵,皆大敗,應龍頓足歎曰:「吾不用時泰計,今死矣!」或言水西佐賊,化龍詰之疆臣,斬賊使,二氏交遂絕。烏江兵敗績,逮下元鎮於理,諸將益奮。綎先入婁山關,直抵海龍囤,璘、疆臣兵亦至。賊勢急,上囤死守,遣使詐降。化龍檄諸將斬使,焚書。以綎與應龍有舊,諭無通賊,綎械其人以自明。八路兵皆會囤下,築長圍困之,更番迭攻。六月,綎破土、月二城,應龍窘,與二妾俱縊。明晨,官軍入城,七子皆被執。詔磔應龍屍並子朝棟於市。自出師至滅賊,凡百有十四日。播自唐乾符中入楊氏,二十九世,八百餘年,至應龍而絕,以其地置遵義、平越二府,分屬川、貴。

化龍初聞父喪,以金革起復,至是乞歸終制。三十一年四月,起工部右侍郎,總理河道,與淮、揚巡撫李三才奏開淤河,由直河入泇口抵夏鎮二百六十里,避黃河呂梁之險。再以憂去,未代。敘前平播功,晉兵部尚書,加少保,廕一子世錦衣指揮使。

三十五年夏,起戎政尚書。化龍以京營根本,奏陳十一濫、十二苦、十九宜,又上屯政十二事,皆置不理。兵部自二十七年後,左、右侍郎皆空署。未幾,尚書蕭大亨亦致仕,化龍掌部事。三十七年正月,京師訛言寇至,民爭避匿,邊民逃入都門者亦數萬,九門晝閉。輔臣言兵部尚書惟一人,何以應猝變,帝亦不報。遼戰士二萬餘皆老弱,而稅監高淮肆虐,遼人切齒。化龍請停稅課,且增兵萬人,又條上兵食款戰之策,帝皆不報。一品秩滿,加柱國、少傅兼太子太保。卒官,年七十。謚襄毅,贈少師,加贈太師。

化龍具文武才。播州之役,以劉綎驕蹇,先摧挫之而薦其才,故綎為盡力。開河之功,為漕渠永利,詳見《河渠志》。

江鐸,字士振,仁和人。高祖玭,景泰時為禮科給事中。劾石亨怙寵罔上,有直聲。官至山東參政。曾祖瀾,正德時南京禮部尚書。卒謚文昭。祖曉,嘉靖中工部侍郎。父圻,萬歷初廣西提學僉事。父母疾,嘗藥舐糞。居喪寢苫三年,經寢室必俯其首,妻經夫廬亦然。卒,門人私謚為孝端先生。自玭至鐸五世皆進士。而曉弟暉,正德中為庶吉士,與舒芬等諫南巡受杖。世宗時,由編修出為河南僉事。鐸登第在萬歷二年。授刑部主事。累官山西按察使,擢撫偏、沅。夾攻楊應龍有功,與郭子章皆廕一子世錦衣指揮。丁母艱去。奪情,命留討皮林諸洞蠻,平之。詳具《陳璘傳》。以勞疾歸。卒,贈兵部右侍郎。

贊曰:哱拜一降人耳,雖假以爵秩,而憑藉未厚。倉猝發難,據鎮城,聯外寇,邊鄙為之騷然,武備之弛,有由來矣。楊應龍惡稔貫盈,自速殄滅。然盤踞積久,地形險惡,非師武臣力,奏績豈易言哉!李化龍之功可與韓雍、項忠相埒,較寧夏之役,難易懸殊矣。


\end{pinyinscope}