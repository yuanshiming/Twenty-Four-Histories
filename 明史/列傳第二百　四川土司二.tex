\article{列傳第二百 四川土司二}

\begin{pinyinscope}
△播州宣慰司永寧宣撫司酉陽宣撫司石砫宣撫司

遵義府即播州。秦為夜郎且蘭地。漢屬牂牁。唐貞觀中,改播州。乾符初,南詔陷播,太原楊端應募復其城,為播人所懷服,歷五代,子孫世有其地。宋大觀中,楊文貴納土,置遵義軍。元世祖授楊邦憲宣慰使,賜其子漢英名賽因不花,封播國公。

洪武四年平蜀,遣使諭之。五年,播州宣慰使楊鏗、同知羅琛、總管何嬰、蠻夷總管鄭瑚等,相率來歸,貢方物,納元所授金牌、銀印、銅章。詔賜鏗衣幣,仍置播州宣慰使司,鏗、琛皆仍舊職。領安撫司二,曰草塘,曰黃平;長官司六,曰真州,曰播州,曰餘慶,曰白泥,曰容山,曰重安。以嬰等為長官。七年,中書省奏:「播州土地既入版圖,當收其貢賦,歲納糧二千五百石為軍儲。」帝以其率先來歸,田稅隨所入,不必以額。已,復置播州黃平宣撫。播州江渡蠻黃安作亂,貴州衛指揮張岱討平之。八年,鏗遣其弟錡來貢,賜衣幣。自是,每三歲一入貢。十四年遣使齎諭鏗:「比聞爾聽浮言,生疑貳。今大軍南征,多用戰騎,宜率兵二萬、馬三千為先鋒,庶表爾誠。」十五年城播州沙溪,以官兵一千人、土兵二千人戍之。改播州宣慰司隸貴州,改黃平衛為千戶所。十七年,鏗子震卒於京,命有司歸其喪。二十年征鏗入朝,貢馬十匹。帝諭以守土保身之道,賜鈔五百錠。二十一年,播州宣慰使司並所屬宣撫司官,各遣其子來朝,請入太學,帝敕國子監官善訓導之。

永樂四年免播州荒田租,設重安長官司,隸播州宣慰司,以張佛保為長官,以佛保嘗招輯重安蠻民響化故也。七年,宣慰使楊昇招諭草塘、黃平、重安所轄當科、葛雍等十二寨蠻人來歸。宣德三年,昇賀萬壽節後期,禮部議予半賞。帝以道遠,勿奪其賜。七年,草塘所屬穀人散等四十一寨蠻作亂,總兵陳懷剿撫之,旋定。

正統十四年,宣慰使楊綱老疾,以其子輝代。景泰三年,輝奏:「湖、貴所轄臻、剖、五坌等苗賊,糾合草塘、江渡諸苗黃龍、韋保等,殺掠人民,屢撫復叛,乞調兵征剿,以靖民患。」帝命總督王來、總兵梁珤等,會同四川巡撫剿之。七年,調輝兵征銅鼓、五開叛苗,賜敕頒賞。

成化十年以播州賊齎果等屢歲為患,敕責川、貴鎮巡官。正統末,苗蠻聚眾寇邊,土官同知羅宏奏,輝有疾,乞以其子愛代。帝命愛襲職,仍敕愛即率兵從總兵官剿賊。先是,輝奏所屬夭壩乾地五十三寨及重安所轄灣溪等寨,屢被苗蠻占據,乞令湖、貴會兵征之。命如輝言。部議以愛年幼,請仍起輝暫理軍事。又以輝難獨任,宜敕都御史張瓚親至播州督理,勵輝等振揚威武,以備徵調,其機宜悉聽瓚裁處。

十二年,瓚督諸軍及輝攻敗灣溪、夭壩乾地諸苗,凡破山寨十六,斬首四百九十六級,撫男婦九千八百餘口。事下兵部,以苗就撫者多,宜量為處分。瓚議設安寧宣撫司,並懷遠、宣化二長官司,建靖南、龍場二堡,命輝董其役。輝調兵民五千餘,立治所,委所屬黃平諸長官,分甓城垣。將竣,輝因奏:「各寨苗蠻,近頗知懼,但大軍還後,難保無虞。播州向設操守土兵一千五百人,今撥守懷遠、靖南、夭漂、龍場各二百人,宣化百人,安寧六百人,其家屬宜徙之同居,為固守計。其工之未畢者,宜命臣子愛董之,而聽臣致仕如故。」詔從之。時灣溪既立安寧宣撫,爛土諸蠻惡其逼,遂引齎果等攻陷夭漂、靖南城堡,圍安寧。愛新襲,力弗能支,求援於川、貴二鎮。兵部奏起輝再統兵剿之,又敕川、貴兵為助。十五年,貴州巡撫陳儼奏:「苗賊齎果轉橫,乞調川、湖等官軍五萬五千,剋期會貴州,聽儼節制。」兵部言:「賊作於四川,而貴州守臣自欲節制諸軍,恐有邀功之人主之。且興師五萬,以半年計,須軍儲十三萬五千石,山路險峻,諭運之夫須二十七萬眾,況天將暑,瘴癘可虞。」帝然其奏。

二十二年,愛兄宣撫楊友訐奏愛,帝命刑部侍郎何喬新往勘。二十三年,喬新奏:「輝在日,溺其庶子友,欲令承襲,長官張淵阿順之。安撫宋韜謂楊氏家法,立嗣以嫡,愛宜立。輝不得已立愛,又欲割地以授友,謀於淵,因以夭壩干乃本州懷遠故地,為生苗所據,請兵取之。容山長官韓瑄以土民安輯日久,不宜徵。淵與輝計執瑄,杖殺之。前巡撫張瓚受輝賂,以其地設安寧宣撫司,冒以友任宣撫。輝立券,以所有金玉、服用、莊田召諸子均分之。輝沒,淵乃與友潛謀刺愛,淵弟深亦與謀,不果,友遂奏愛居處器用僭擬朝廷,又通唐府,密書往來,私習兵法、天文,謀不軌,事皆誣。」帝命斬淵、深。以愛信讒薄兄,友因公擅殺,且謀嫡,盜官錢,皆有罪。愛贖復任,友遷保寧羈管,仍敕喬新從宜處治。

弘治元年增設重安守御千戶所,命播州歲調土兵一千助戍守。七年,以平苗功,賜敕勞愛。十四年,調播州兵五千徵貴州賊婦米魯等。

正德二年升播州宣慰使楊斌為四川按察使,仍理宣慰事。舊制,土官有功,賜衣帶,或旌賞部眾,無列銜方面者。斌狡橫,不受兩司節制,諷安撫羅忠等上其平普安等戰功,重賂劉瑾,得之。踰年,巡按御史俞緇言不宜授,乃裁之,仍原職。初,友既編置保寧,愛益恣,厚斂以賄中貴,徵取友向所居凱里地者獨苛。同知楊才居安寧,乘之,朘剝尤甚,諸苗憤怨。凱里民為友奏復官,弗得,乃潛入保寧,以友還,糾眾作亂,攻播州,焚愛居第及公私廨宇略盡,遂殺才,多所殘戮。愛屢奏於朝,帝命鎮巡官調兵征之。會友死,遂緩師。已而鎮巡官言:「友子弘能悔過自新,且善撫馭,蠻眾願聽其約束。其前為友所焚殺者,俱已隨土俗折償,且還所侵奪於官。乞授弘冠帶為土舍,協同播州經歷司撫輯諸蠻。其家眾置保寧者仍歸之,隸播州管轄。並諭斌與弘協和,不得再造釁端。」報可。未幾,播州安撫宋淮奏:「貴州凱口爛土苗婚於凱里草塘諸寨,陰相構結,誘出苗為亂。乞賜斌敕,令每年巡視邊境,會湖廣鎮巡官撫處。」部議,土官向無領敕出巡者。諭斌宜撫綏土眾,輯睦親族,以副朝廷優待之意。因授致仕宣慰愛為昭毅將軍,給誥命,賜麒麟服。時斌又為其父請進階及服色,禮科駁之,以服色等威所繫,不可假。兵部以愛舊有剿賊功,皆許之。斌復為其子相請入學,並得賜冠帶。

十二年,播州安撫羅忠、宋淮等奏:「斌有父喪,欲援文臣例守制,但邊防為重,乞仍令掌印理事。」初,楊弘既歸凱里,與重安土舍馮綸等有怨。弘卒,綸等誘苗蠻攻之,更相仇殺,侵軼貴州境。巡撫鄒文盛言狀,且請移文四川,會官撫處,踰歲不報。文盛乃遣參議蔡潮入播州,督致仕楊斌撫平之。因言:「宜復安寧宣撫,俾弘子弟襲之。斌未衰,宜仍起任事,以制諸蠻寨。潮有撫蠻勞,宜量擢。」兵部議:「安寧已革不可復,斌子既代,亦不可起。土官應襲與否,屬四川,非黔所得專。盛所請難行,而功不可誣。」十六年賜斌蟒衣玉帶。

嘉靖元年賜播州儒學《四書集注》,從宣慰楊相奏也。弘既死,其弟張求襲職不得,時盜邊,劫白泥司印信,復與相構兵。守臣乞改凱裏屬貴州,以張為土知州解釋之。兵部議:「張習父兄之惡,幸免於辜;敢肆然執印信以要挾,當命川、貴守臣按其前後爭產殺人諸罪,置於理。若張悔過輸情,還所獲印,尚可量授一官,聽調殺賊以自效。倘或怙終,必誅以為玩法戒。」既,遂許張襲宣撫,而改安寧為凱里,隸貴川。初,楊相之祖父皆以嫡庶相爭,梯禍數世。至是,相復寵庶子煦。嫡子烈母張,悍甚,與烈盜兵逐相,相走,客死水西。烈求父屍,宣慰安萬銓因要挾水煙、天旺故地,而後予屍,烈陽許之。及相喪還,烈靳地不予,遂與水西構難,又殺其長官王黻。時嘉靖二十三年也。烈既代襲,遂與黻黨李保治兵相攻,垂十年,總督馮岳調總兵石邦憲討平之。真州苗盧阿項者亦久稱亂,邦憲以兵七千擊敗之。有言賊求援於播者,邦憲曰:「吾方調水西兵,聲揚烈助逆罪,烈暇救人乎。」已,擒阿項父子,斬獲四百餘人。初,嘉靖初,議分凱里屬貴州,既,又以播地多在貴州境,並改屬思石兵備。及真州盜平,地方安靖,播人以為非便。川、貴守臣異議不決,命總督會勘。總督奏,仍以播歸四川,而貴州思石兵備仍兼制播、酉、平、邑諸土司事,報可。

隆慶五年,烈死,子應龍請襲,命予職。萬曆元年給應龍宣慰使敕書。八年賜故宣慰楊烈祭葬,從應龍請也。十四年,應龍獻大木七十,材美,賜飛魚服,又復引其祖斌賜蟒例。部議,以斌有軍功,且出特恩,未可為比。帝命以都指揮使銜授應龍。

十八年,貴州巡撫葉夢熊疏論應龍兇惡諸事,巡按陳效歷數應龍二十四大罪。時方防禦松潘,調播州土兵協守,四川巡按李化龍疏請暫免勘問,俾應龍戴罪圖功。由是,川、貴撫按疏辨,在蜀者謂應龍無可勘之罪,在黔者謂蜀有私暱應龍之心。於是給事中張希皋等,以事屬重大,兩省利害,豈漫不相關者,乞從公會勘,無執成心。十九年,夢熊主議,播州所轄五司改土為流,悉屬重慶,與化龍意復相左。化龍遂引嫌求斥。蓋應龍本雄猜,阻兵嗜殺,所轄五司七姓悉叛離。嬖妾田屠妻張氏,并及其母。妻叔張時照與所部何恩、宋世臣等上變,告應龍反。夢熊請發兵剿之,蜀中士大夫悉謂蜀三面鄰播,屬裔以什伯數,皆其彈壓,且兵驍勇,數徵調有功,剪除未為長策。以故,蜀撫按並主撫。朝議命勘,應龍願赴蜀,不赴黔。

二十年,應龍詣重慶對簿,坐法當斬,請以二萬金贖。御史張鶴鳴方駁問,會倭大入朝鮮,徵天下兵,應龍因奏辨,且願將五千兵征倭自贖,詔釋之。兵已啟行,尋報罷。巡撫王繼光至,嚴提勘結,應龍抗不出。張時照等復詣奏闕下,繼光用兵之議遂決。二十一年,繼光至重慶,與總兵劉承嗣等分兵三道進婁山關,屯白石口。應龍佯約降,而統苗兵據關沖擊。承嗣兵敗,殺傷大半。會繼光論罷,即撤兵,委棄輜重略盡。黔師協剿,亦無功。時四川新撫譚希忠與貴州鎮、撫再議剿,御史薛繼茂主撫。應龍上書自白,遣其黨攜金入京行間,執原奏何恩詣綦江縣。

二十二年,以兵部侍郎刑玠總督貴州。二十三年,玠至蜀,察永寧、酉陽皆應龍姻媾,而黃平、白泥久為仇仇,宜剪其枝黨。乃檄應龍,謂當待以不死。會水西宣慰安疆臣請父國亨恤典,兵部尚書石星手札示疆臣,趣應龍就吏得貰,疆臣奉札至播招應龍。時七姓恐應龍出得除罪,而四方亡命竄匿其間,又幸龍反,因以為利,驛傳文移,輒從中阻。玠檄重慶知府王士琦詣綦江,趣應龍安穩聽勘。應龍使弟兆龍至安穩,治郵舍,儲Я叩頭郊迎,致餼牽如禮,言:「應龍縛渠魁,待罪松坎。所不敢至安穩者,恐墮安穩仇民不測禍也,幸請至松坎受事。」士琦曰,「松坎亦曩奏勘地。」即單騎往。應龍果面縛道旁,泣請死罪,願執罪人,獻罰金,得自比安國亨。國亨者,曩亦被訐懼罪不出界,故應龍引之。士琦為請于玠,許之,應龍乃縛獻黃元等十二人。案驗,抵應龍斬,論贖,輸四萬金助採木,仍革職,以子朝棟代,次子可棟羈府追贖,黃元等斬重慶市,總督以聞。時倭氣未靖,兵部欲緩應龍,事東方,朝廷亦以應龍向有積勞,可其奏,於松坎設同知治焉,以士琦為川東兵備副使彈治之。應龍獲寬,益怙終不悛。尋可棟死於重慶,益痛恨。促喪歸不得,復檄完贖,大言曰:「吾子活,銀即至矣。」擁兵驅千餘僧招魂去。分遣土目,署關據險。厚撫諸苗,名其健者為硬手;州人稍殷厚者,沒入其貲以養苗。苗人咸願為出死力。

二十四年,應龍殘餘慶,掠大阡、都壩,焚劫草塘、餘慶二司及興隆、都勻各衛。又遣其黨圍黃平,戮重安長官家,勢復大熾。二十五年流劫江津及南川,臨合江,索其仇袁子升,縋城下,磔之。時兵備王士琦調徵倭,應龍益統苗兵,大掠貴州洪頭、高坪、新村諸屯。已,又侵湖廣四十八屯,阻塞驛站。詗原奏仇民宋世臣、羅承恩等挈家匿偏橋衛,襲破之。大索城中,戮其父母,淫其妻女,備極慘酷。

二十七年,貴州巡撫江東之令都司楊國柱部卒三千剿應龍,奪三百落。賊佯北,誘師殲焉,國柱等盡死。東之罷,以郭子章代,而起李化龍節制川、湖、貴州諸軍事,調東征諸將劉綎、麻貴、陳璘、董一元南征。時應龍乘大兵未集,勒兵犯綦江。城中新募兵不滿三千,賊兵八萬奄至,遊擊張良賢巷戰死,綦江陷。應龍盡殺城中人,投尸蔽江,水為赤。益結九股生苗及黑腳苗等為助,屯官壩,聲窺蜀。已,遂焚東坡、爛橋,楚、黔路梗。

二十八年,應龍五道並出,破龍泉司。時總督李化龍已移駐重慶,徵兵大集,遂以二月十二月誓師,分八路進。每路約三萬人,官兵三之,土司七之,旗鼓甲仗森列,苗大驚。總兵劉綎破其前鋒,楊朝棟僅以身免,賊膽落。遂連克桑木、烏江、河渡三關,奪天都、三百落諸囤。賊連敗,乃乘隙突犯烏江,詐稱水西隴澄會哨,誘永順兵,斷橋,淹死將卒無算。尋綎破九盤,入婁山關。關為賊前門,萬峰插天,中通一線。綎從間道攀藤毀柵入,陷焉。四月朔,師屯白石,應龍率諸苗決死戰。綎親勒騎沖中堅,分兩翼夾擊,敗之。追奔至養馬城,連破龍爪、海雲險囤,壓海龍囤,賊所倚天險,謂飛鳥騰猿不能逾者。時偏沅師已破青蛇囤,安疆臣亦奪落濛關,至大水田,焚桃溪莊。賊見勢急,父子相抱哭,上囤死守,每路投降文緩師。總兵吳廣入崖門關,營水牛塘,與賊力戰三日,卻之。賊詭令婦人於囤上拜表痛哭云:「田氏且降。」復許為應龍仰藥死報廣,廣輕信按兵。已,覘賊詐,益厲兵攻,燒二關,奪賊樵汲路。八路師大集海龍囤,遂築長圍,更番迭攻。賊知必死。會化龍聞父喪,詔以縗墨視師。化龍念賊前囤險不能越,令馬孔英率勍兵並力攻其後。天苦雨,將士馳泥淖中苦戰。六月四日,天忽霽,綎先士卒,克土城。應龍益迫,散金募死土拒戰,無應者。起,提刀巡壘,見四面火光燭天,大兵已登囤,破土城入。應龍倉皇同愛妾二闔室縊,且自焚。吳廣獲其子朝棟,急覓應龍屍,出焰中。賊平。計出師至滅賊,百十有四日,八路共斬級二萬餘,生獲朝棟等百餘人。化龍露布以聞,獻俘闕下彩應龍屍,磔朝棟、兆龍等於市。播州自唐入楊氏,傳二十九世,八百餘年,至應龍而亡。三十一年,播州餘逆吳洪、盧文秀等叛,總兵李應祥等討平之。分播地為二,屬蜀者曰遵義府,屬黔者為平越府。

永寧,唐蘭州地。宋為滬州江安、合江二縣境。元置永寧路,領筠連州及騰川縣,後改為永寧宣撫司。

洪武四年平蜀,永寧內附,置永寧衛。六年,筠連州滕大寨蠻編張等叛,詐稱雲南兵,據湖南長寧諸州縣,命成都衛指揮袁洪討之。洪引兵至敘州慶符縣,攻破清平關,擒偽千戶李文質等。編張遁走,復以兵犯江安諸縣。洪追及之,又敗其眾,焚其九寨,獲編張子偽鎮撫張壽。編張遁匿溪洞,餘黨散入雲南。帝聞之,敕諭洪曰:「南蠻叛服不常,不足罪。既獲其俘,宜編為軍。且駐境上,必以兵震之,使讋天威,無遺後患。」未幾,張復聚眾據滕大寨,洪移兵討敗之。追至小芒部,張遁去,遂取得花寨,擒阿普等。自是,張不敢復出,其寨悉平。遂降筠連州為縣,屬敘州,以九姓長官司隸永寧安撫司。

七年陞永寧等處軍民安撫司為宣撫使司,秩正三品。八年以祿照為宣撫使。十七年,永寧宣撫使祿照貢馬,詔賜鈔幣冠服,定三年一貢如例。十八年,祿照遣弟阿居來朝,言比年賦馬皆已輸,惟糧不能如數。緣大軍南征,蠻民驚竄,耕種失時,加以兵後疾疫死亡者多,故輸納不及。命蠲之。二十三年,永寧宣撫言,所轄地水道有一百九十灘,其江門大灘有八十二處,皆石塞其流。詔景川侯曹震往疏鑿之。二十四年,震至瀘州按視,有枝河通永寧,乃鑿石削崖,以通漕運。

二十六年,以祿照子阿聶襲職。先是,祿照坐事逮至京,得直,還卒於途。其子阿聶與弟智皆在太學,遂以庶母奢尾署司事。至是,奢尾入朝,請以阿聶襲,從之。永樂四年,免永寧荒田租。

宣德八年,故宣撫阿聶妻奢蘇朝貢。九年,宣撫奢蘇奏:「生儒皆土僚,朝廷所授官言語不通,難以訓誨。永寧監生李源資厚學通,乞如雲南鶴慶府例,授為儒學訓導。」詔從之。景泰二年,減永寧宣撫司稅課局鈔,以苗賊竊發,客商路阻,從布政司請也。

成化元年,山都掌大壩等寨蠻賊分劫江安等縣,兵部以聞。二年,國子學錄黃明善奏:「四川山都掌蠻屢歲出沒,殺掠良民。景泰元年招之復叛,天順六年撫之又反。近總兵李安令永寧宣撫奢貴赴大壩招撫,亦未效。恐開釁無已,宜及大兵之集,早為定計,毋釀邊患。」三年,明善復言:「宋時多剛縣蠻為寇,用白芀子兵破之。白芀子者,即今之民壯;多剛縣者,即今之都掌多剛寨也。前代用鄉兵有明效,宜急募民壯,以助官軍。都掌水稻十月熟,宜督兵先時取其田禾,則三月之內蠻必餒矣。軍宜分三路:南從金鵝池攻大壩,中從戎縣攻箐前,北從高縣攻都掌。小寨破,大寨自拔。又大壩南百餘里為芒部,西南二百里為烏蒙,令二府土官截其險要。更用火器自下而上,順鳳延熱,寨必可攻。且徵調土兵,須處置得宜,招募民壯,須賞罰必信。」詔總兵官參用之。時總督尚書程信亦奏:「都掌地勢險要,必得士兵響道。請敕東川、芒部、烏蒙、烏撒諸府兵,並速調湖廣永順、保靖兵,以備征遣。」又請南京戰馬一千應用。皆報可。四年,信奏:「永寧宣撫奢貴開通運道,擒獲賊首,宜降璽書獎齎。」從之。

十六年,白羅羅羿子與都掌大壩蠻相攻,禮部侍郎周洪謨言:「臣敘人也,知敘蠻情。戎、珙、筠、高諸縣,在前代皆土官,國朝始代以流,言語性情不相習,用激變。洪、永、宣、正四朝,四命將徂征,隨服隨叛。景泰初,益滋蔓,至今為梗。臣向嘗言仍立土官治之,為久遠計。而都御史汪浩儌幸邊功,誣殺所保土官及寨主二百餘人,諸蠻怨入骨髓,轉肆劫掠。及尚書程信統大兵,僅能克之。臣以謂及今順蠻人之情,擇其眾所推服者,許為大寨主,俾世襲,庶可相安。」又言:「白羅羅者,相傳為廣西流蠻,有眾數千,無統屬。景泰中,糾戎、珙苗,攻破長寧九縣,今又侵擾都掌。其所居,崖險箐深,既難剪滅,亦宜立長官司治之。地近芒部,宜即隸之。羿子者,永寧宣撫所轄。而永寧乃雲、貴要沖,南跨赤水、畢節六七百里,以一柔婦人制數萬強梁之眾,故每肆劫掠。臣以為宣撫土僚,仍令宣撫奢貴治之。其南境寨蠻近赤水、畢節要路者,宜立二長官司,仍隸永寧宣撫。夫土官有職無俸,無損國儲,有益邊備。」從之。二十五年,永寧宣撫司女土官奢祿獻大木,給誥如例。

萬歷元年,四川巡撫曾省吾奏:「都蠻叛逆,發兵征討,土官奢效忠首在調,但與貴州土官安國亨有仇。請並令總兵官劉顯節制,使不得藉口復仇,妄有騷動。」從之。初,烏撒與永寧、烏蒙、水西、沾益諸土官境相連,復以世戚親厚。既而安國亨殺安信,信兄智結永寧宣撫奢效忠報仇,彼此相攻。而安國亨部下吏目與智有親,恐為國亨所殺,因投安路墨。墨詐稱為土知府安承祖,赴京代奏。已而國亨亦令其子安民陳訴,與奢效忠俱奉命聽勘於川貴巡撫。議照蠻俗罰牛贖罪,報可。效忠死,妻世統無子,妾世續有幼子崇周。世統以嫡欲奪印,相仇殺。方奏報間,總兵郭成、參將馬呈文利其所有,遽發兵千餘,深入落紅。奢氏九世所積,搜掠一空。世續亦發兵尾其後。效忠弟沙卜出拒戰,且邀水西兵報仇。成兵敗績,乃檄取沙卜於世統,統不應,復殺把總三人,聚苗兵萬餘,欲攻永寧洩怨。巡按劾成等邀利起釁,宜逮;而議予二土婦冠帶,仍分地各管所屬,其宣撫司印俟奢崇周成立,赴襲理事。報可。十四年,奢崇周代職,未幾死。

奢崇明者,效忠親弟盡忠子也。幼孤,依世統撫養一十三年。至是,送之永寧,世續遺之氈馬,許出印給之。事已定,而諸奸閻宗傳等自以昔從世續逐世統,殺沙卜,懼崇明立,必復前恨,遂附水西,立阿利以自固。安疆臣陰陽其間,蠻兵四出,焚劫屯堡,官兵不能禁。總督以聞,朝議命奢崇明暫管宣撫事,冀崇明蠲夙恨,以收人心。而閻宗傳等攻掠永寧、普市、麾尼如故。崇明承襲幾一載,世續印竟不與,且以印私安疆臣妻弟阿利。巡撫遣都司張神武執世續索印,世續言印在鎮雄隴澄處。隴澄者,水西安堯臣也。隴氏垂絕,堯臣入贅,遂冒隴姓,稱隴澄。敘平播州、敘州功,澄與焉,中朝不知其為堯臣也。堯臣外怙播功,內仗水西,有據鎮雄制永寧心。蜀撫按以堯臣非隴氏種,無授鎮雄意。堯臣以是懷兩端,陰助世續。意世續得授阿利,則己據鎮雄益堅。又朝廷厭兵,宗傳、阿利等方驛騷,己可臥取隴氏也。而閻宗傳等每焚掠,必稱鎮雄兵,以怖諸部。川南道梅國樓所俘蠻醜者言,鎮雄遣將魯大功督兵五營屯大壩,水西兵已渡馬鈴堡,約攻永寧,普市遂潰,宗傳等以空城棄去。奢崇明又言,堯臣所遣目把彭月政、魯仲賢六大營助逆不退,聲言將抵敘南,攻永寧、滬州。於是總兵侯國弼等,皆歸惡於堯臣。都司張神武等所俘喚者、朗者,皆鎮雄土目,堯臣亦不能解。

黔中撫按以西南多事,兵食俱詘,無意取鎮雄。堯臣因以普市、摩尼諸焚掠,皆歸之蜀將。議者遂以貪功起釁,為蜀將罪。四川巡撫喬璧星言:「堯臣狡謀,欲篡鎮雄,垂涎蘭地有年矣。宗傳之背逆恃鎮雄,猶鎮雄之恃水西也。水西疆臣不助兵,臣已得其狀,宜乘逆孽未成,令貴州撫按調兵與臣會剿。倘堯臣稔惡如故,臣即移師擊之,毋使弗摧之虺復為蛇,弗窒之罅復為河也。」疏上,廷議無敢決用師者。久之,阿利死,印亦出,蜀中欲逐堯臣之論,卒不可解。時播州清疆之議方沸騰,黔、蜀各紛紛。至是,永寧議兵又如聚訟矣。時朝廷已一意休兵。三十五年,命釋奢世續,赦閻宗傳等罪,訪求隴氏子孫為鎮雄後。並令安疆臣約束堯臣歸本土司,聽遙授職銜,不許冒襲隴職。於是宗傳降,堯臣請避去,黔督遂請撤師。舊制,永寧衛隸黔,土司隸蜀。自水、蘭交攻,軍民激變,奢崇明雖立,而行勘未報。摩尼、普市千戶張大策等復請將永寧宣撫改土為流。兵部言,無故改流,置崇明何地,命速完前勘諸案。於是蜀撫擬張大策以失守城池罪,應斬,黔撫擬張神武以擅兵劫掠,罪亦應斬。策斬策,黔人,武,蜀人也。由是兩情皆不平,諸臣自相構訟,復紛結不解。會奢崇明子寅與水西已故土官妻奢社輝爭地,安兵馬十倍奢,而奢之兵精,兩相持。蜀、黔撫按不能制,以狀聞。四十八年,黔撫張鶴鳴以赤水衛白撒所屯地為永寧占據,宜清還,皆待勘未決。

天啟元年,崇明請調馬步兵二萬援遼,從之。崇明與子寅久蓄異志,借調兵援遼,遣其婿樊龍、部黨張彤等,領兵至重慶,久駐不發。巡撫徐可求移鎮重慶,趣永寧兵。樊龍等以增行糧為名乘機反,殺巡撫、道、府、總兵等官二十餘員,遂據重慶。分兵攻合江、納溪,破滬州,陷遵義,興文知縣張振德死之。興文,故九絲蠻地也。進圍成都,偽號大梁,布政使朱燮元、周著,按察使林宰分門固守。石砫土司女官秦良玉遣弟民屏、侄翼明等,發兵四千,倍道兼行,潛渡重慶,營南坪關。良玉自統精兵六千,沿江上趨成都。諸援兵亦漸集。時寅攻城急,陰納劉勳等為內應,事覺伏誅。復造雲梯及旱船,晝夜薄城,城中亦以砲石擊毀之。相持百日,會賊將羅乾象遣人輸款,願殺賊自效。是夜,乾象縱火焚營,賊兵亂,崇明父子倉皇奔,錢帛穀米委棄山積,窮民賴以得活。乾象因率其黨胡汝高等來降。時燮元已授巡撫,率川卒追崇明,江安、新都、遵義諸郡邑皆復。時二年三月也。樊龍收餘眾數萬,據重慶險塞。燮元督良玉等奪二郎關,總兵杜文煥破佛圖關,諸將迫重慶而軍。奢寅遣賊黨周鼎等分道來救,鼎敗走,為合江民所縛。官軍與平茶、酉陽、石砫三土司合圍重慶,城中乏食。燮元遂以計擒樊龍,殺之,張彤亦為亂兵所殺,生擒龍子友邦及其黨張國用、石永高等三十餘人,遂復重慶。

時安邦彥反於貴州,崇明遙倚為聲援。三年,川師復遵義,進攻永寧,遇奢寅於土地坎,率兵搏戰。大兵奮擊,敗之。寅被創遁,樊虎亦戰死。進克其城,降賊二萬。得進拔紅崖、天臺諸囤寨,降者日至。崇明勢益蹙,求救於水西,邦彥遣十六營過河援之。羅乾象急破蘭州,焚九鳳樓,覆其巢。崇明踉蹌走,投水西。邦彥與合兵,分犯遵義、永寧。川師敗之於芝麻塘,賊遁入青山。諸將逼渭河,鏖入龍場陣,獲崇明妻安氏及奢崇輝等,斬獲萬計。蘭州平。總督朱燮元請以赤水河為界,河東龍場屬黔,河西赤水、永寧屬蜀。永寧設道、府,與遵義、建武聲勢聯絡。未幾,貴州巡撫王三善為邦彥所襲死,崇明勢復張,將以逾春大舉寇永寧。會奢寅為其下所殺,而燮元亦以父喪去,崇明、邦彥得稽誅。崇明稱大梁王,邦彥號四裔大長老,諸稱元帥者不可勝計,合兵十餘萬,規先犯赤水。崇禎初,起燮元總督貴、湖、雲、川、廣諸軍務,大會師。燮元定計誘賊深入向永寧,邀之於五峰山桃紅壩,令總兵侯良柱大敗之,崇明、邦彥皆授首。是役也,掃蕩蜀、黔數十年巨憝,前後皆燮元功云。

酉陽,漢武陵郡酉陽縣地,宋為酉陽州。元屬懷德府。洪武五年,酉陽軍民宣慰司冉如彪遣弟如喜來朝貢。置酉陽州,以如彪為知州。八年改為宣撫司,仍以冉如彪為使。置平茶、邑梅、麻免、石耶四洞長官司,以楊底綱、楊金奉、冉德原、楊隆為之,每三年一入貢。石耶不能親至京,命附於酉陽。二十七年,平茶洞署長官楊再勝,謀殺兄子正賢及洞長楊通保等。正賢等覺之,逃至京師,訴其事,且言再勝與景川侯謀反。帝命逮再勝鞫之,再勝辭服,當族誅,正賢亦應緣坐。帝誅再勝,釋正賢,使襲長官。酉陽宣撫冉興邦以襲職來朝,命改隸渝州。

永樂三年,指揮丁能、杜福撫諭亞堅等十一寨生苗一百三十六戶,各遣子入朝,命隸酉陽宣撫司。四年免酉陽荒田租。五年,興邦遣部長龔俊等貢方物,並謝立儒學恩。

景泰七年調宣撫僉事冉廷璋兵,征五開、銅鼓叛苗,賜敕諭賞齎。天順十三年命進宣撫冉雲散官一階,以助討叛苗及擒石全州之功也。

弘治七年,宣撫冉舜臣以徵貴州叛苗功,乞升職。兵部以非例,請進舜臣階明威將軍,賜敕褒之。十二年,舜臣秦宋農寨蠻賊糾脅諸寨洞蠻,殺掠焚劫,乞剿捕。保靖、永順二宣慰亦奏,邑梅副長官楊勝剛父子謀據酉陽,結俊倍洞長楊廣震等,號召宋農、後溪諸蠻,聚兵殺掠,請並討。兵部議,酉陽溪洞連絡,易煽動,宜即撲滅,請行鎮巡官酌機宜。十四年調酉陽兵五千協剿貴州賊婦米魯。

正德三年,酉陽宣撫司護印舍人冉廷璽及邑梅長官司奏,湖廣鎮溪所洞苗聚眾攻劫,請兵剿捕。八年,宣撫冉元獻大木二十,乞免男維翰襲職赴京,從之。二十年,元再獻大木二十,詔量加服色酬賞。

萬歷十七年,宣撫再維屏獻大木二十,價逾三千。工部議,應加從三品服,以為土官輸誠之勸,從之。四十六年調酉陽兵四千,命宣撫冉躍龍將之援遼。四十七年,躍龍遣子天胤及文光等領兵赴遼陽,駐虎皮、黃山等處三載,解奉集之圍。再援沈陽,以渾河失利,冉見龍戰沒,死者千餘人。撤守遼陽,又以降敵縱火,冉文煥等戰沒,死者七百餘人。兵部尚書張鶴鳴言:「躍龍遣子弟萬里勤王,見龍既殺身殉國,躍龍又自捐金二千兩,運軍器至山海關,振困招魂,忠義可嘉。臣在貴州時,躍龍亦自捐餉征紅苗,屢建奇功。今又著節於邊,宜加優恤,以風諸邊。」

天啟元年授躍龍宣慰使,並妻舒氏,皆給誥命,仍恤陣亡千七百餘家。二年,奢崇明叛,躍龍率援師合圍重慶。及崇明誅,其土舍冉紹文與有功。四年,躍龍以東西赴調效命,為弟見龍及諸陣亡者請齎恤。命下所司。崇禎九年,宣慰使冉天麟疏言:「庶孽天胤假旨謀奪臣爵土,不遂,擅兵戕殺。」下撫按察勘。時蜀方憂盜,大吏自顧不暇,土官事多寢閣雲。

石砫,以石潼關、砫薄關而名。後周置施州。唐改青江郡。宋末,置石砫安撫司。元改石砫軍民府,尋仍為安撫司。

洪武七年,石砫安撫使馬克用遣其子付德與同知陳世顯入朝,貢方物。八年,改石砫安撫司為宣撫司,隸重慶府。十六年,石砫溪蠻寇施州,黔江守御官軍擊破之。十八年,石砫宣撫同知陳世顯遣子興潮等奉表貢方物,賀明年正旦。二十四年賜石砫宣撫同知陳興潮及其子文義白金百兩,以從征散毛洞有功故也。

宣德五年命宣撫馬應仁子鎮為宣撫。初,應仁有罪應死,貸謫戍。至是,帝念其祖克用嘗效力先朝,命求其子孫之良者用之,故有是命。

成化十八年,四川巡撫孫仁奏:「三月內盜三百人入石砫,殺宣撫馬澄及隸卒二十餘人,焚掠而去。以石砫地鄰酆都,互爭銀場相訐,有司不為區治,致相仇殺。」命責有司捕賊。仁奏:「石砫歲辦鉛課五千一百三十斤,正統後停之。鄰境軍民假以徵課,乘機竊取,釀成禍階。請除其課,閉其洞,仍移忠州臨江巡檢於酆都南賓里之姜池,以便防守。」從之。是年,命馬徽為宣撫。

萬歷二十二年,石砫女土官覃氏行宣撫事。土吏馬邦聘謀奪其印,與其黨馬斗斛、斗霖等,集眾數千,圍覃氏,縱火焚公私廬舍八十餘所,殺掠一空。覃氏上書言:「臣自從征疊、茂,擊賊大雪山,斬首捕寇,皆著有成勞,屢膺上官獎賞。今邦聘無故虔劉孤寡,臣豈不能出一旅與之角勝負,誠以非朝命,不敢也。今叛人斯在,請比先年楚金洞舍覃碧謀篡事,願與邦聘同就吏。」二十三年命四川撫,按讞其獄,事未決。會楊應龍反播州,覃與應龍為姻,而斗斛亦結應龍,兩家觀望,獄遂解。覃氏有智計,性淫,故與應龍通。長子千乘失愛,暱次子千駟,謂應龍可恃,因聘其女為千駟妻。千駟入播,同應龍反。千乘襲馬氏爵,應調,與酉陽冉禦同徵應龍。應龍敗。千駟伏誅,而千乘為宣撫如故。千乘卒,妻秦良玉以功封夫人,自有傳。


\end{pinyinscope}