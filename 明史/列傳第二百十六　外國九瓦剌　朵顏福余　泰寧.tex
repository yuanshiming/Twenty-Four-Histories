\article{列傳第二百十六 外國九瓦剌 朵顏福余 泰寧}

\begin{pinyinscope}
瓦剌,蒙古部落也,在韃靼西。元亡,其強臣猛可帖木兒據之。死,眾分為三,其渠曰馬哈木,曰太平,曰把禿孛羅。

成祖即位,遣使往告。永樂初,復數使鎮撫答哈帖木兒等諭之,并賜馬哈木等文綺有差。六年冬,馬哈木等遣暖答失等隨亦剌思來朝貢馬,仍請封。明年夏,封馬哈木為特進金紫光祿大夫、順寧王;太平為特進金紫光祿大夫、賢義王;把禿孛羅為特進金紫光祿大夫、安樂王;賜印誥。暖答失等宴賚如例。

八年春,瓦剌復貢馬謝恩。自是歲一入貢。

時元主本雅失里偕其屬阿魯台居漠北,馬哈木乃以兵襲破之。八年,帝既自將擊破本雅失里及阿魯台兵,馬哈木上言請得早為滅寇計。十年,馬哈木遂攻殺本雅失里。復上言欲獻故元傳國璽,慮阿魯台來邀,請中國除之;脫脫不花子在中國,請遣還;部屬多從戰有勞,請加賞賚;又瓦剌士馬強,請予軍器。帝曰:「瓦剌驕矣,然不足較。」賚其使而遣之。明年,馬哈木留敕使不遣,復請以甘肅、寧夏歸附韃靼者多其所親,請給還。帝怒,命中官海童切責之。冬,馬哈木等擁兵飲馬河,將入犯,而揚言襲阿魯台。開平守將以聞,帝詔親征。明年夏,駐蹕忽蘭忽失溫。三部埽境來戰,帝麾安遠侯柳升、武安侯鄭亨等先嘗之,而親率鐵騎馳擊,大破之,斬王子十餘人,部眾數千級。追奔,度兩高山,至土剌河。馬哈木等脫身遁,乃班師。明年春,馬哈木等貢馬謝罪,且還前所留使,詞卑。帝曰:「瓦剌故不足較。」受其獻,館其使者。明年,瓦剌與阿魯台戰,敗走。未幾,馬哈木死,海童歸言,瓦剌拒命由順寧,順寧死,賢義、安樂皆可撫。帝因復使海童往勞太平、把禿孛羅。

十六年春,海童偕瓦剌貢使來。馬哈木子脫懽請襲爵,帝封為順寧王。而海童及都督蘇火耳灰等以綵幣往賜太平、把禿孛羅及弟昂克,別遣使祭故順寧王。自是,瓦剌復奉貢。

二十年,瓦剌侵掠哈密,朝廷責之,遣使謝罪。二十二年冬,瓦剌部屬賽因打力來降,命為所鎮撫,賜彩幣、襲衣、鞍馬,仍令有司給供具。自後來歸者悉如例。

宣德元年,太平死,子捏烈忽嗣。時脫懽與阿魯台戰,敗之,遁母納山、察罕腦剌間。宣德九年,脫懽襲殺阿魯台,遣使來告,且請獻玉璽。帝賜敕曰:「王殺阿魯台,見王克復世仇,甚善。顧王言玉璽,傳世久近,殊不在此。王得之,王用之可也。」仍賜糸寧絲五十表裏。

正統元年冬,成國公朱勇言:「近瓦剌脫懽以兵迫逐韃靼朵兒只伯,恐吞併之,日益強大。乞敕各邊廣儲積,以備不虞。」帝嘉納之。未幾,脫懽內殺其賢義、安樂兩王,盡有其眾,欲自稱可汗,眾不可,乃共立脫脫不花,以先所併阿魯台眾歸之。自為丞相,居漠北,哈喇嗔等部俱屬焉。已,襲破朵兒只伯,復脅誘朵顏諸衛窺伺塞下。

四年,脫懽死,子也先嗣,稱太師淮王。於是北部皆服屬也先,脫脫不花具空名,不復相制。每入貢,主臣並使,朝廷亦兩敕答之;賜賚甚厚,並及其妻子、部長。故事,瓦使不過五十人。利朝廷爵賞,歲增至二千餘人。屢敕,不奉約。使往來多行殺掠,又挾他部與俱,邀索中國貴重難得之物。稍不饜,輒造釁端,所賜財物亦歲增。也先攻破哈密,執王及王母,既而歸之。又結婚沙州、赤斤蒙古諸衛,破兀良哈,脅朝鮮。邊將知必大為寇,屢疏聞,止敕戒防禦而已。

十一年冬,也先攻兀良哈,遣使抵大同乞糧,並請見守備太監郭敬。帝敕敬毋見,毋予糧。明年,復致書宣府守將楊洪。洪以聞,敕洪禮其使,報之。頃之,其部眾有來歸者,言也先謀入寇,脫脫不花止之,也先不聽,尋約諸番共背中國。帝詔問,不報。時朝使至瓦剌,也先等有所請乞,無不許。瓦剌使來,更增至三千人,復虛其數以冒廩餼。禮部按實予之,所請又僅得五之一,也先大媿怒。

十四年七月,遂誘脅諸番,分道大舉入寇。脫脫不花以兀良哈寇遼東,阿剌知院寇宣府,圍赤城,又遣別騎寇甘州,也先自寇大同。參將吳浩戰死貓兒莊,羽書踵至。太監王振挾帝親征,群臣伏闕爭,不得。大同守將西寧侯宋瑛、武進伯朱冕、都督石亨等與也先戰陽和,太監郭敬監軍,諸將悉為所制,失律,軍盡覆。瑛、冕死,敬伏草中免,亨奔還。車駕次大同,連日風雨甚,又軍中常夜驚,人恟懼,郭敬密言於振,始旋師。車駕還次宣府,敵眾襲軍後。恭順侯吳克忠拒之,敗歿。成國公朱勇、永順伯薛綬以四萬人繼往,至鷂兒嶺,伏發,盡陷。次日,至土木。諸臣議入保懷來,振顧輜重遽止,也先遂追及。土木地高,掘井二丈不得水,汲道已為敵據,眾渴,敵騎益增。明日,敵見大軍止不行,偽退,振遽令移營而南。軍方動,也先集騎四面沖之,士卒爭先走,行列大亂。敵跳陣而入,六軍大潰,死傷數十萬。英國公張輔,駙馬都尉井源,尚書鄺埜、王佐,侍郎曹鼐、丁鉉等五十餘人死之,振亦死。帝蒙塵,中官喜寧從。也先聞車駕至,錯愕未之信,及見,致禮甚恭,奉帝居其弟伯顏帖木兒營,以先所掠校尉袁彬來侍。也先將謀逆,會大雷雨震死也先所乘馬,復見帝寢幄有異瑞,乃止。也先擁帝至大同城,索金幣,都督郭登與白金三萬。登復謀奪駕入城,帝沮之不果,也先遂擁帝北行。

九月,郕王自監國即皇帝位,尊帝為太上皇帝。也先詭稱奉上皇還,由大同、陽和抵紫荊關,攻入之,直前犯京師。兵部尚書于謙督武清伯石亨、都督孫鏜等禦之。也先邀大臣出迎上皇,未果。亨等與戰,數敗之。也先夜走,自良鄉至紫荊,大掠而出。都督楊洪復大破其餘眾於居庸,也先仍以上皇北行。也先夜常於御幄上,遙見赤光奕奕若龍蟠,大驚異。也先又欲以妹進上皇,上皇卻之,益敬服,時時殺羊馬置酒為壽,稽首行君臣禮。

景泰元年,也先復奉上皇至大同,郭登不納,仍謀欲奪上皇,也先覺之,引去。初,也先有輕中國心,及犯京師,見中國兵強,城池固,始大沮。會中國已誘誅賊奄喜寧,失其間諜,而脫脫不花、阿剌知院復遣使與朝廷和,皆撤所部歸,也先亦決意息兵。秋,帝遣侍郎李實、少卿羅綺、指揮馬政等齎璽書往諭脫脫不花及也先。而脫脫不花、也先所遣皮兒馬黑麻等已至,帝因復使都御史楊善、侍郎趙榮率指揮、千戶等往。也先語實,兩國利速和,迎使夕至,大駕朝發,但當遣一二大臣來。實歸,善等至,致奉迎上皇意。也先曰:「上皇歸,當仍作天子邪?」善曰:「天位已定,不再更。」也先引善見上皇,遂設宴餞上皇行。也先席地彈琵琶,妻妾奉酒,顧善曰:「都御史坐。」善不敢坐,上皇曰:「太師著坐,便坐。」善承旨坐,即起,周旋其間。也先顧善曰:「有禮。」伯顏等亦各設餞畢,也先築土臺,坐上皇臺上,率妻妾部長羅拜其下,各獻器用、飲食物。上皇行,也先與部眾皆送約半日程,也先、伯顏乃下馬伏地慟哭曰:「皇帝行矣,何時復得相見!」良久乃去,仍遣其頭目七十人送至京。

上皇歸後,瓦剌歲來貢,上皇所亦別有獻。於是帝意欲絕瓦剌,不復遣使往。也先以為請,尚書王直、金濂、胡濙等相繼言絕之且起釁。帝曰:「遣使,有前事,適以滋釁耳。曩瓦剌入寇時,豈無使邪?」因敕也先曰:「前者使往,小人言語短長,遂致失好。朕今不復遣,而太師請之,甚無益。」也先與脫脫不花內相猜。脫脫不花妻,也先姊也,也先欲立其姊子為太子,不從。也先亦疑其通中國,將謀己,遂治兵相攻。脫脫不花敗走,也先追殺之,執其妻子,以其人畜給諸部屬;遂乘勝迫脅諸蕃,東及建州、兀良哈,西及赤斤蒙古、哈密。

三年冬,遣使來賀明年正旦,尚書王直等復請答使報之。下兵部議,兵部尚書于謙言:「臣職司馬,知戰而已,行人事非所敢聞。」詔仍毋遣使。明年冬,也先自立為可汗,以其次子為太師,來朝,書稱大元田盛大可汗,末曰添元元年。田盛,猶言天聖也。報書稱曰瓦剌可汗。未幾,也先復逼徙朵顏所部於黃河母納地。也先恃強,日益驕,荒於酒色。

六年,阿剌知院攻也先,發之。韃靼部孛來復殺阿剌,奪也先母妻並其玉璽。也先諸子火兒忽答等徙居乾趕河,弟伯都王、姪兀忽納等往依哈密。伯都王,哈密王母之弟也。英宗復辟三年,哈密為請封,詔授伯都王都督僉事,兀忽納指揮僉事。自也先死,瓦剌衰,部屬分散,其承襲代次不可考。

天順中,瓦剌阿失帖木兒屢遣使入貢,朝廷以其為也先孫,循例厚賚之。又撦力克者,常與孛來仇殺。又拜亦撒哈者,常偕哈密來朝。其長曰克捨,頗強,數糾韃靼小王子入寇。克捨死,養罕王稱雄,擁精兵數萬,克捨弟阿沙為太師。成化二十三年,養罕王謀犯邊,哈密罕慎來告。養罕不利去,憾哈密,兵還掠其大土剌。

弘治初,瓦剌中稱太師者,一曰火兒忽力,一曰火兒古倒溫,皆遣使朝貢。土魯番據哈密,都御史許進以金帛厚啖二部,令以兵擊走之。其部長卜六王者,屯駐把思闊。正德十三年,土魯番犯肅州。守臣陳九疇因遺卜六王彩幣,使乘虛襲破土魯番三城,殺擄以萬計。土魯番畏逼,與之和。嘉靖九年,復以議婚相仇隙。土魯番益強,瓦剌數困敗,又所部輒自殘,多歸中國,哈密復乘間侵掠。卜六王不支,亦求內附。朝廷不許,遣出關,不知所終。

朵顏、福餘、泰寧,高皇帝所置三衛也。其地為兀良哈,在黑龍江南,漁陽塞北。漢鮮卑、唐吐谷渾、宋契丹,皆其地也。元為大寧路北境。

高皇帝有天下,東蕃遼王、惠寧王、朵顏元帥府相率乞內附。遂即古會州地,置大寧都司營州諸衛,封子權為寧王使鎮焉。已,數為韃靼所抄。洪武二十二年置泰寧、朵顏、福餘三衛指揮使司,俾其頭目各自領其眾,以為聲援。自大寧前抵喜峰口,近宣府,曰朵顏;自錦、義歷廣寧至遼河,曰泰寧;自黃泥窪逾沈陽、鐵嶺至開原,曰福餘。獨朵顏地險而強。久之皆叛去。

成祖從燕起靖難,患寧王躡其後,自永平攻大寧,入之。謀脅寧王,因厚賂三衛說之來。成祖行,寧王餞諸郊,三衛從,一呼皆起,遂擁寧王西入關。成祖復選其三千人為奇兵,從戰。天下既定,徙寧王南昌,徙行都司於保定,遂盡割大寧地畀三衛,以償前勞。

帝踐阼初,遣百戶裴牙失里等往告。永樂元年復使指揮蕭尚都齎敕諭之。明年夏,頭目脫兒火察等二百九十四人隨尚都來朝貢馬。命脫兒火察為左軍都督府都督僉事,哈兒兀歹為都指揮同知,掌朵顏衛事;安出及土不申俱為都指揮僉事,掌福餘衛事;忽剌班胡為都指揮僉事,掌泰寧衛事;餘三百五十七人,各授指揮、千百戶等官。賜誥印、冠帶及白金、鈔幣、襲衣。自是,三衛朝貢不絕。三年冬,命來朝頭目阿散為泰寧衛掌衛事、都指揮僉事,其朵兒朵臥等各陞賞有差。

四年冬,三衛饑,請以馬易米。帝命有司第其馬之高下,各倍價給之。久之,陰附韃靼掠邊戍,復假市馬來窺伺。帝下詔切責,令其以馬贖罪。十二年春,納馬三千於遼東,帝敕守將王真,一馬各予布四匹。已,復叛附阿魯台。二十年,帝親征阿魯台還,擊之,大敗其眾於屈烈河,斬馘無算,來降者釋勿殺。

仁宗嗣位,詔三衛許自新。洪熙元年,安出奏其印為寇所奪,請更給,許之。冬,三衛頭目阿者禿來歸,授千戶,賜鈔幣、襲衣、鞍馬,仍命有司給供具。自後來歸者,悉如例。

宣宗初,三衛掠永平、山海間,帝將親討之,三衛頭目皆謝罪入貢,撫納之如初。七年更給泰寧衛印。秋,以朵顏頭目哈剌哈孫、福餘頭目安出、泰寧頭目脫火赤等恭事朝廷久,加賜織金綵幣表裏有差。

正統間,屢寇遼東、大同、延安境。獨石守備楊洪擊敗之,擒其頭目朵欒帖木兒。未幾,復附瓦剌也先,泰寧拙赤妻也先以女,皆陰為之耳目。入貢輒易名,且互用其印,又東合建州兵入廣寧前屯。帝惡其反覆,九年春,命成國公朱勇偕恭順侯吳克忠出喜峰,興安伯徐亨出界嶺,都督馬亮出劉家口,都督陳懷出古北,各將精兵萬人,分剿之。勇等捕其擾邊者致闕下,並奪回所掠人畜。

拙赤等拘肥河衛使人殺之,肥河衛頭目別里格與戰於格魯坤迭連,拙赤大敗。瓦剌復分道截殺,建州亦出兵攻之,三衛大困。

十二年春,總兵曹義、參將胡源、都督焦禮等分巡東邊,值三衛入寇,擊之,斬三十二級,擒七十餘人。其年,瓦剌賽刊王復擊殺朵顏乃兒不花,大掠以去。也先繼至,朵顏、泰寧皆不支,乞降,福餘獨走避腦溫江,三衛益衰。畏瓦剌強,不敢背,仍歲來致貢,止以利中國賜賚;又心銜邊將剿殺,故常潛圖報復。

十四年夏,大同參將石亨等復擊其盜邊者於箭溪山,擒斬五十人,三衛益怨。秋,導瓦剌大入,英宗遂以是役北狩。

景泰初,朝廷仍遣使撫諭。三衛受也先旨,數以非時入貢,多遣使往來伺察中國。既而也先虐使之,復逼徙朵顏所部於黃河母納地,三衛皆不堪,遂陰輸瓦剌情於中國,請得近邊屯駐。舊制,三衛每歲三貢,其貢使俱從喜峰口驗入,有急報則許進永平。時三衛使有自獨石及萬全右衛來者。邊臣以為言,敕止之。天順中,嘗乘間掠諸邊,復竊通韃靼孛來,每為之鄉導。所遣使與孛來使臣偕見。中國待韃靼厚,請加賞不得,大忿,遂益與孛來相結。

成化元年,頭目朵羅干等以兵從孛來,大入遼河。已,復西附毛里孩,東合海西兵,數入塞。又時獨出沒廣寧、義州間。九年,遼東總兵歐信以偏將韓斌等敗之於興中,追及麥州,斬六十二級,獲馬畜器械幾數千。其年,喜峰守將吳廣以貪賄失三衛心,三衛入犯,廣下獄死。明年復掠開原,慶雲參將周俊擊退之。

十四年詔復三衛馬市。初,國家設遼東馬市三,一城東,一廣寧,皆以待三衛。正統間,以其部眾屢叛,罷之。會韃靼滿都魯暴強,侵掠三衛,三衛頭目皆走避塞下。數饑困,請復馬市再四,不許。至是巡撫陳鉞為帝言,始許之。滿都魯死,亦思馬因主兵柄,三衛復數為所窘。

二十二年,韃靼別部那孩擁三萬眾入大寧、金山,涉老河,攻殺三衛頭目伯顏等,掠去人畜以萬計。三衛乃相率攜老弱,走匿邊圉。邊臣劉潺以聞,詔予芻糧優颻之。

弘治初,常盜掠古北、開原境,守臣張玉、總兵李杲等以計誘斬其來市者三百人,遂北結脫羅幹,請為復仇,數寇廣寧、寧遠諸處。時海西尚古者,以不得通貢叛中國,數以兵阻諸蕃入貢,諸蕃並銜之。朝廷旋許尚古納款,撫寧猛克帖木兒等皆以尚古為辭,入寇遼陽,殺掠甚眾。韃靼小王子屢掠三衛,三衛因各叩關輸罪,朝廷許之,然陽為恭順而已。

朵顏都督花當者,恃險而驕,數請增貢加賞,不許。正德十年,花當子把兒孫以千騎毀占魚關,入馬蘭谷大掠,參將陳乾戰死;復以五百騎入板場谷,千騎入神山嶺,又千餘騎入水開洞。事聞,命副總兵桂勇禦之。花當退去,屯駐紅羅山,匿把兒孫,使其子打哈等入朝請罪,詔釋不問。十三年,帝巡幸至大喜峰口,將征三衛頭目,使悉詣關下宴勞,不果。

當把兒孫犯邊時,朝廷詔削其職。把兒孫死,其子伯革入貢。嘉靖九年,詔予伯革父爵,而打哈自以花當子不得職,怒,遂先後掠冷口、擦崖、喜峰間。參將袁繼勛等失於防禦,皆逮治。十七年春,指揮徐顥誘殺泰寧部九人,其頭目把當亥率眾寇大清堡,總兵馬永擊斬之。其屬把孫以朵顏部眾復入,鎮守少監王永與戰,敗績。二十二年冬,攻圍墓田谷,殺守備陳舜,副總兵王繼祖等赴援,擊斬三十餘級。其年,詔罷舊設三衛馬市,並新設木市亦罷之。秋,三衛復導韃靼寇遼州,入沙河堡,守將張景福戰死。

三衛之迭犯也,實朵顏部哈舟兒、陳通事為之。二人者,俱中國人,被擄遂為三衛用。二十九年,韃靼俺答謀犯畿東,舟兒為指潮河川路。俺答移兵白廟,近古北,舟兒詐言敵已退,邊備緩,俺答遂由鴿子洞、曹榆溝入,直犯畿甸。已,俺答請開馬市,舟兒復往來誘阻之。三十年,薊遼總督何棟購捕至京,伏誅。

朵顏通罕者,俺答子辛愛妻父也。四十二年,古北哨卒出關,為朵顏所撲殺。俄通罕叩關索賞,副總兵胡鎮伏兵執之。總督楊選將為牽制辛愛計,乃拘縶通罕,令其諸子更迭為質。三衛恨甚,遂導俺答入掠順義及三河,選得罪。

萬曆初,朵顏長昂益強,挾賞不遂,數糾眾入掠,截諸蕃貢道。十二年秋,復導士蠻以四千騎分掠三山、三道溝、錦川諸處。守臣李松請急剿長昂等,朝議不從,僅革其月賞。未幾,復以千騎犯劉家口,官軍禦之,殺傷相當。於是長昂益跋扈自恣,東勾土蠻,西結婚白洪大,以擾諸邊。十七年合韃靼東西二部寇遼東,總兵李成梁逐之,官軍大敗,殲八百人。又二年大掠獨石路。二十二年復擁眾犯中後所,攻入小屯臺,副總兵趙夢麟、秦得倚等力戰卻之。明年潛入喜峰口,官軍擒其頭目小郎兒。

二十九年,長昂與董狐狸等皆納款,請復寧前木市,許之。三十四年冬,復糾韃靼班不什、白言台吉等,以萬騎迫山海關,總兵姜顯謨擊走之。長昂復以三千騎窺義院界,邊將有備,乃引去。旋詣喜峰,自言班、白入寇,己不預知。守臣具以聞,詔長昂復貢市,頒給撫賞如例。

長昂死,諸子稍衰,三衛皆靖。崇禎初,與插漢戰於早落兀素,勝之,殺獲萬計,以捷告。未幾,皆服屬於大清雲。


\end{pinyinscope}