\article{兵志}


明以武功定天下,革元舊制,自京師達於郡縣,皆立衛所。外統之都司,內統於五軍都督府泛神論否認有超自然的造物主的存在,把神融化在自然,而上十二衛為天子親軍者不與焉。征伐則命將充總兵官,調衛所軍領之,既旋則將上所佩印,官軍各回衛所。蓋得唐府兵遺意。文皇北遷,一遵太祖之制,然內臣觀兵,履霜伊始。洪、宣以後,狃於治平,故未久而遂有土木之難。于謙創立團營,簡精銳,一號令,兵將相習,其法頗善。憲、孝、武、世四朝,營制屢更,而威益不振。衛所之兵疲於番上,京師之旅困於占役。馴至末造,尺籍久虛,行伍衰耗,流盜蜂起,海內土崩。宦豎降於關門,禁軍潰於城下,而國遂以亡矣。今取其一代規制之詳,及有關於軍政者,著於篇。

京營侍衛上直軍皇城守衛京城巡捕四衛營

京軍三大營,一曰五軍,一曰三千,一曰神機。其制皆備於永樂時。

初,太祖建統軍元帥府,統諸路武勇,尋改大都督府。以兄子文正為大都督,節制中外諸軍。京城內外置大小二場,分教四十八衛卒。已,又分前、後、中、左、右五軍都督府。洪武四年,士卒之數,二十萬七千八百有奇。

成祖增京衛為七十二。又分步騎軍為中軍,左、右掖,左、右哨,亦謂之五軍。歲調中都、山東、河南、大寧兵番上京師隸之。設提督內臣一,武臣二,掌號頭官二,大營坐營官一,把總二,中營坐營官一,馬步隊把總各一。左右掖、哨官如之。又有十二營,掌隨駕馬隊官軍,設把總二。又有圍子手營,掌操練上直叉刀手及京衛步隊官軍,設坐營官一,統四司,以一、二、三、四為號,把總各二。又有幼官舍人營,掌操練京衛幼官及應襲舍人,坐營官一,四司把總各一。此五軍營之部分也。已,得邊外降丁三千,立營分五司。一,掌執大駕龍旗、寶纛、勇字旗、負御寶及兵仗局什物上直官軍。一,掌執左右二十隊勇字旗、大駕旗纛金鼓上直官軍。一,掌傳令營旗牌,御用監盔甲、尚冠、尚衣、尚履什物上直官軍。一,掌執大駕勇字旗、五軍紅盔貼直軍上直官軍。一,掌殺虎手、馬轎及前哨馬營上直明甲官軍、隨侍營隨侍東宮官舍、遼東備禦回還官軍。提督內臣二,武臣二,掌號頭官二,坐司官五,見操把總三十四,上直把總十六,明甲把總四。此三千營之部分也。已,征交阯,得火器法,立營肄習。提督內臣、武臣,掌號頭官,皆視三千營,亦分為五軍。中軍,坐營內臣一,武臣一。其下四司,各監鎗內臣一,把司官一,把總官二。左右掖、哨皆如之。又因得都督譚廣馬五千匹,置營名五千下,掌操演火器及隨駕護衛馬隊官軍。坐營內臣、武臣各一,其下四司,各把司官二。此神機營之部分也。居常,五軍肄營陣,三千肄巡哨,神機肄火器。大駕征行,則大營居中,五軍分駐,步內騎外,騎外為神機,神機外為長圍,周二十里,樵採其中。三大營之制如此。

洪熙時,始命武臣一人總理營政。宣德五年,以成國公朱勇言,選京衛卒隸五軍訓練。明年,命科道及錦衣官核諸衛軍數。帝之征高煦及破兀良哈,皆以京營取勝焉。正統二年,復因勇言,令錦衣等衛、守陵衛卒存其半,其上直旗校隸錦衣督操,餘悉歸三大營。土木之難,京軍沒幾盡。

景帝用于謙為兵部尚書,謙以三大營各為教令,臨期調撥,兵將不相習,乃請於諸營選勝兵十萬,分十營團練。每營都督一,號頭官一,都指揮二,把總十,領隊一百,管隊二百。於三營都督中推一人充總兵官,監以內臣,兵部尚書或都御史一人為提督。其餘軍歸本營,曰老家。京軍之制一變。英宗復辟,謙死,團營罷。

憲宗立,復之,增為十二。成化二年復罷。命分一等、次等訓練。尋選得一等軍十四萬有奇。帝以數多,令仍分十二營團練,而區其名,有奮、耀、練、顯四武營,取、果、效、鼓四勇營,立、伸、揚、振四威營。命侯十二人掌之,各佐以都指揮,監以內臣,提督以勳臣,名其軍曰選鋒。不任者仍為老家以供役,而團營之法又稍變。二十年,立殫忠、效義二營;練京衛舍人、餘丁。二營,永樂間設,後廢,至是復設。未幾,以無益罷。帝在位久,京營特注意,然缺伍至七萬五千有奇,大率為權貴所隱占。又用汪直總督團營,禁旅專掌於內臣,自帝始也。

孝宗即位,乃命都御史馬文升為提督。是時營軍久苦工役。成化末,餘子俊嘗言之,文升復力陳不可。又請於每營選馬步銳卒二千,遇警徵調。且遵洪、永故事,五日一操,以二日走陣下營,以三日演武。從之。時尚書劉大夏陳弊端十事,復奏減修乾清宮卒。內臣謂其不恤大工,大學士劉健曰:「愛惜軍士,司馬職也。」帝納之。會戶部主事李夢陽極論役軍之害,并及內臣主兵者。以語侵壽寧侯,下詔獄,遂格不行。

武宗即位,十二營銳卒僅六萬五百餘人,稍弱者二萬五千而已。給事中葛嵩請選五軍、三千營精銳歸團練,而存八萬餘人於營以供役。惠安伯張偉謬引舊制以爭,事遂已,隱占如故。BH鐇反,太監張永將京軍往討,中官權益重。及流寇起,邊將江彬等得幸,請調邊軍入衛。於是集九邊突騎家丁數萬人於京師。名曰外四家。立兩官廳,選團營及勇士、四衛軍於西官廳操練,正德元年所選官軍操於東官廳。自是兩官廳軍為選鋒。而十二團營且為老家矣。武宗崩,大臣用遺命罷之。當是時,工作浩繁,邊將用事,京營戎政益大壞。給事中王良佐奉敕選軍,按籍三十八萬有奇,而存者不及十四萬,中選者僅二萬餘。

世宗立,久之,從廷臣言,設文臣知兵者一人領京營。是時額兵十萬七千餘人,而存者僅半。專理京營兵部尚書李承勛請足十二萬之數。部議遵弘治中例,老者補以壯丁,逃、故者清軍官依期解補。從之。十五年,都御史王廷相提督團營,條上三弊:一,軍士多雜派,工作終歲,不得入操。雖名團營聽征,實與田夫無異。二,軍士替代,吏胥需索重賄,貧軍不能辦,老羸茍且應役,而精壯子弟不得收練。三,富軍憚營操徵調,率賄將弁置老家數中,貧者雖老疲,亦常操練。語頗切中。既而兩郊九廟諸宮殿之工起,役軍益多。兵部請分番為二,半團操,半放歸,而收其月廩雇役。詔行一年。自後邊警急,團營見兵少,僅選騎卒三萬,仍號東西官廳。餘者悉老弱,仍為營帥、中官私役。

二十九年,俺答入寇,兵部尚書丁汝夔核營伍不及五六萬人。驅出城門,皆流涕不敢前,諸將領亦相顧變色。汝夔坐誅。大學士嚴嵩乃請振刷以圖善後。吏部侍郎王邦瑞攝兵部,因言:「國初,京營勁旅不減七八十萬,元戎宿將常不乏人。自三大營變為十二團營,又變為兩官廳,雖浸不如初,然額軍尚三十八萬有奇。今武備積馳,見籍止十四萬餘,而操練者不過五六萬,支糧則有,調遣則無。比敵騎深入,戰守俱稱無軍。即見在兵,率老弱疲憊、市井遊販之徒,衣甲器械取給臨時。此其弊不在逃亡,而在占役;不在軍士,而在將領。蓋提督、坐營、號頭、把總諸官,多世胄紈褲,平時占役營軍,以空名支餉,臨操則肆集市人,呼舞博笑而已。先年,尚書王瓊、毛伯溫、劉天和常有意振飭。然將領惡其害己,陰謀阻撓,軍士又習於驕惰,競倡流言,事復中止,釀害至今。乞大振乾綱,遣官精核。」帝是其言,命兵部議興革。

於是悉罷團營、兩官廳,復三大營舊制。更三千曰神樞。罷提督、監鎗等內臣。設武臣一,曰總督京營戎政,以咸寧侯仇鸞為之;文臣一,曰協理京營戎政,即以邦瑞充之。其下設副參等官二十六員。已,又從部議,以四武營歸五軍營中軍,四勇營歸左右哨,四威營歸左右掖。各設坐營官一員,為正兵,備城守;參將二員,備征討。帝以營制新定,告於太廟行之。又遣四御史募兵畿輔、山東、山西、河南,得四萬人,分隸神樞、神機。各設副將一,而增能戰將六員,分領操練。大將所統三營之兵,居常名曰練勇,有事更定職名。五軍營:大將一員,統軍一萬,總主三營副、參、遊擊、佐擊及坐營等官;副將二員,各統軍七千;左右前後參將四員,各六千;遊擊四員,各三千。外備兵六萬六千六百六十人。神樞營:副將二員,各統軍六千;佐擊六員,各三千。外備兵四萬人。神機營亦如之。已,又定三大營官數:五軍營一百九十六員,神樞營二百八員,神機營一百八十二員,共五百八十六員。在京各衛軍,俱分隸三營。分之為三十營,合之為三大營。終帝世,其制屢更,最後中軍哨掖之名亦罷,但稱戰守兵兼立車營。

故事,五軍府皆開府給印,主兵籍而不與營操,營操官不給印,戎政之有府與印,自仇鸞始。鸞方貴幸,言於帝,選各邊兵六萬八千人,分番入衛,與京軍雜練,復令京營將領分練邊兵,於是邊軍盡隸京師。塞上有警,邊將不得徵集,邊事益壞。鸞死,乃罷其所置戎政廳首領官之屬,而入衛軍則惟罷甘肅者。

隆慶四年,大學士趙貞吉請收將權,更營制。極言戎政之設府鑄印,以數十萬眾統於一人,非太祖、成祖分府分營本意。請以官軍九萬分五營,營擇一將,分統訓練。詔下廷臣議。尚書霍冀言:「營制,世宗熟慮而後定,不宜更。惟大將不當專設,戎政不宜有印,請如貞吉言。」制曰「可」。於是三大營各設總兵一,副將二。其參佐等官,互有增損,各均為十人。而五軍營兵,均配二營,營十枝,屬二副將分統。以侯伯充總兵,尋改曰提督。又用三文臣,亦稱提督。自設六提督後,各持意見,遇事旬月不決。給事中溫純言其弊,乃罷,仍設總督、協理二臣。

萬曆二年,從給事中歐陽柏請,復給戎政印,汰坐營官二員。五年,巡視京營科臣林景暘請廣召募,立選鋒。是時,張居正當國,綜核名實,群臣多條上兵事,大旨在足兵、選將,營務頗飭。久之,帝厭政,廷臣漸爭門戶,習於偷惰,遂日廢弛。三十六年,尚書李化龍理戎政,條上京營積弊。敕下部議,卒無所振作。及兵事起,總督京營趙世新請改設教場城內,便演習。太常少卿胡來朝請調京軍戍邊,可變弱為強。皆無濟於用。

天啟三年,協理侍郎朱光祚奏革老家軍,補以少壯。老家怨,以瓦礫投光祚,遂不果革。是時,魏忠賢用事,立內操,又增內臣為監視及把牌諸小內監,益募健丁,諸營軍多附之。

莊烈帝即位,撤內臣,已而復用。戎政侍郎李邦華憤京營弊壞,請汰老弱虛冒,而擇材力者為天子親軍。營卒素驕,有疑其為變者。勳戚中官亦惡邦華害己,蜚語日聞。帝為罷邦華,代以陸完學,盡更其法。京營自監督外,總理捕務者二員,提督禁門、巡視點軍者三員,帝皆以御馬監、司禮、文書房內臣為之,於是營務盡領於中官矣。十年八月,車駕閱城,鎧甲旌旗甚盛,群臣悉鸞帶策馬從。六軍望見乘輿,皆呼萬歲。帝大悅,召完學入御幄獎勞,酌以金卮,然徒為容觀而已。

時兵事益亟。帝命京軍出防剿,皆監以中官。廩給優渥,挾勢而驕,多奪人俘獲以為功,輕折辱諸將士,將士益解體。周延儒再入閣,勸罷內操,撤諸監軍。京兵班師還。時營將率內臣私人,不知兵。兵惟注名支糧,買替紛紜,朝甲暮乙,雖有尺籍,莫得而識也。帝屢旨訓練,然日不過二三百人,未昏遂散。營兵十萬倖抽驗不及,玩愒佚罰者無算。帝嘗問戎政侍郎王家彥,家彥曰:「今日惟嚴買替之禁,改操練之法,庶可救萬一,然勢已晚。」帝不懌而罷。十六年,襄城伯李國禎總戎政,內臣王承恩監督京營。明年,流賊入居庸關,至沙河。京軍出禦,聞砲聲潰而歸。賊長驅犯闕,守陴者僅內操之三千人,京師遂陷。

大率京軍積弱,由於占役買閒。其弊實起於紈褲之營帥,監視之中官,竟以亡國云。

京營之在南者,永樂北遷,始命中府掌府事官守備南京,節制在南諸衛所。洪熙初,以內臣司守備。宣德末,設參贊機務官。景泰間,增協同守備官。成化末,命南京兵部尚書參贊機務,視五部特重。先是,京師立神機營,南京亦增設,與大小二教場同練。軍士常操不息,風雨方免。有逃籍者,憲宗命南給事御史時至二場點閱。成國公朱儀及太監安寧不便,詭言軍機密務,御史詰問名數非宜。帝為罪御史,仍令守備參贊官閱視,著為令。

嘉靖中,言者數奏南營耗亡之弊。二十四年冬,詔立振武營,簡諸營銳卒充之,益以淮揚趫〗捷者。江北舊有池河營,專城守,護陵寢。二營兵各三千,領以勳臣,別設場訓練。然振武營卒多無賴子。督儲侍郎黃懋官抑削之,遂嘩,毆懋官至死。詔誅首惡,以戶部尚書江東為參贊。東多所寬假,眾益驕,無復法紀。給事中魏元吉以為言,因舉浙直副總兵劉顯往提督。未至,池河兵再變,毆千戶吳欽。詔顯亟往,許以川兵五百自隨,事始定。隆慶改元,罷振武營,以其卒千餘仍隸二場及神機營。

萬歷十一年,參贊尚書潘季馴言:「操軍原額十有二萬,今僅二萬餘。祖軍與選充參半,選充例不補,營伍由是虛。請如祖軍收補。」已而王遴代季馴,言:「大小二場,新舊官軍二萬三千有餘。請如北京各邊,三千一百二十人為一枝,每枝分中、左、右哨,得兵七枝。餘置旗鼓下,備各營缺。」從之。巡視科臣阮子孝極論南營耗弊,言頗切中,然卒無振飭之者。已,從尚書吳文華請,增參贊旗牌,得以軍法從事,兼聽便宜調遣。三十一年,添設南中軍標營,選大教場卒千餘,設中軍參將統練。規制雖具,而時狃茍安,闒茸一如北京。及崇禎中,流寇陷廬、鳳,踞上流,有窺留都意。南中將士日夜惴惴,以護陵寢、守京城為名,倖賊不東下而已。最後,史可法為參贊尚書,思振積弊,未久而失,蓋無可言焉。

侍衛上直軍之制。太祖即吳王位,其年十二月設拱衛司,領校尉,隸都督府。洪武二年,改親軍都尉府,統中、左、右、前、後五衛軍,而儀鑾司隸焉。六年,造守衛金牌,銅塗金為之。長一尺,闊三寸。以仁、義、禮、智、信為號。二面俱篆文:一曰「守衛」,一曰「隨駕」。掌於尚寶司,衛士佩以上直,下直納之。十五年,罷府及司,置錦衣衛。所屬有南北鎮撫司十四所,所隸有將軍、力士、校尉,掌直駕侍衛、巡察緝捕。已又擇公、侯、伯、都督、指揮之嫡次子,置勛衛散騎舍人,而府軍前衛及旗手等十二衛,各有帶刀官。錦衣所隸將軍,初名天武,後改稱大漢將軍,凡千五百人。設千、百戶,總旗七員。其眾自為一軍,下直操練如制,缺至五十人方補。月糈二石,積勞試補千、百戶,亡者許以親子弟魁梧材勇者代,無則選民戶充之。

永樂中,置五軍、三千營。增紅盔、明甲二將軍及叉刀圍子手之屬,備宿衛。校尉、力士僉民間丁壯無惡疾、過犯者。力士先隸旗手衛,後改隸錦衣及騰驤四衛,專領隨駕金鼓、旗幟及守衛四門。校尉原隸儀鑾司,司改錦衣衛,仍隸焉。掌擎執鹵簿儀仗,曰鑒輿,曰擎蓋,曰扇手,曰旌節,曰旗幢,曰班劍,曰斧鉞,曰戈戟,曰弓矢,曰馴馬,凡十司,及駕前宣召差遣,三日一更直。設總旗、小旗,而領以勛戚官。官凡六:管大漢將軍及散騎舍人、府軍前衛帶刀官者一,管五軍營叉刀圍子手者一,管神樞營紅盔將軍者四。聖節、正旦、冬至及大祀、誓戒、冊封、遣祭、傳制用全直,直三千人,餘則更番,器仗衣服位列亦稍殊焉。凡郊祀、經筵、巡幸侍從各有定制,詳《禮志》中。居常,當直將軍朝夕分候午門外,夜則司更,共百人。而五軍叉刀官軍,悉於皇城直宿。掌侍衛官輸直,日一員。惟掌錦衣衛將軍及叉刀手者,每日侍。尤嚴收捕之令,及諸脫更離直者。共計錦衣衛大漢將軍一千五百七人,府軍前衛帶刀官四十,神樞營紅盔將軍二千五百,把總指揮十六,明甲將軍五百二,把總指揮二,大漢將軍八,五軍營叉刀圍子手三千,把總指揮八,勛衛散騎舍人無定員,旗手等衛帶刀官一百八十,此侍衛親軍大較也。

正統後,妃、主、公、侯、中貴子弟授官者,多寄祿錦衣中。正德時,奏帶傳升冒銜者,又不下數百人。武宗好養勇士,嘗以千、把總四十七人,注錦衣衛帶俸舍、餘千一百人充御馬監家將勇士,食糧騎操。又令大漢將軍試百戶,五年實授,著為令。倖竇開而恩澤濫,宿衛稍輕矣。至萬歷間,衛士多占役、買閒,其弊亦與三大營等。雖定離直者奪月糈之例,然不能革。

太祖之設錦衣也,專司鹵簿。是時方用重刑,有罪者往往下錦衣衛鞫實,本衛參刑獄自此始。文皇入立,倚錦衣為心腹。所屬南北兩鎮撫司,南理本衛刑名及軍匠,而北專治詔獄。凡問刑、奏請皆自達,不關白衛帥。用法深刻,為禍甚烈,詳《刑法志》。又錦衣緝民間情偽,以印官奉敕領官校。東廠太監緝事,別領官校,亦從本衛撥給,因是恒與中官相表裏。皇城守衛,用二十二衛卒,不獨錦衣軍,而門禁亦上直中事。京城巡捕有專官,然每令錦衣官協同。地親權要,遂終明之世云。初,太祖取婺州,選富民子弟充宿衛,曰御中軍。已,置帳前總制親兵都指揮使。後復省,置都鎮撫司,隸都督府,總牙兵巡徼。而金吾前後、羽林左右、虎賁左右、府軍左右前後十衛,以時番上,號親軍。有請,得自行部,不關都督府。及定天下,改都鎮撫司為留守,設左右前後中五衛,關領內府銅符,日遣二人點閱,夜亦如之,所謂皇城守衛官軍也。

二十七年,申定皇城門禁約。凡朝參,門始啟,直日都督、將軍及帶刀、指揮、千百戶、鎮撫、舍人入後,百官始以次入。上直軍三日一更番,內臣出入必合符嚴索,以金幣出者驗視勘合,以兵器雜藥入門者擒治,失察者重罪之。民有事陳奏,不許固遏。帝念衛士勞苦,令家有婚喪、疾病、產子諸不得已事,得自言情,家無餘丁,父母俱病者,許假侍養,愈乃復。

先是,新宮成,詔中書省曰:「軍士戰鬥傷殘,難備行伍,可於宮墻外造舍以居之,晝則治生,夜則巡警。」其後,定十二衛隨駕軍上直者,人給錢三百。二十八年,復於四門置舍,使恩軍為衛士執爨。恩軍者,得罪免死及諸降卒也。

永樂中,定制,諸衛各有分地。自午門達承天門左右,逮長安左右門,至皇城東西,屬旗手、濟陽、濟川、府軍及虎賁右、金吾前、燕山前、羽林前八衛。東華門左右至東安門左右,屬金吾、羽林、府軍、燕山四左衛。西華門左右至西安門左右,屬四右衛。玄武門左右至北安門左右,屬金吾、府軍後及通州、大興四衛。衛有銅符,頒自太祖。曰承,曰東,曰西,曰北,各以其門名也。巡者左半,守者右半。守官遇巡官至,合契而從事。各門守衛官,夜各領銅令申字牌巡警,自一至十六。內皇城衛舍四十,外皇城衛舍七十二,俱設銅鐸,次第循環。內皇城左右坐更將軍百,每更二十人,四門走更官八,交互往來,鈐印于籍以為驗。都督及帶刀、千百戶日各一人,領申字牌直宿,及點各門軍士。後更定都督府,改命侯、伯僉書焉。

洪熙初,更造衛士懸牌。時親軍缺伍,衛士不獲代。帝命選他衛軍守端、直諸門,尚書李慶謂不可。帝曰:「人主在布德以屬人心,茍心相屬,雖非親幸,何患焉。」宣德三年,命御史點閱衛卒。天順中,復增給事中一人。成化十年,尚書馬文升言:「太祖置親軍指揮使司,不隸五府。文皇帝復設親軍十二衛,又增勇士數千員,屬御馬監,上直,而以腹心臣領之。比者日廢弛,勇士與諸營無異,皇城之內,兵衛無幾,諸監門卒尤疲羸,至不任受甲。宜敕御馬監官,即見軍選練。仍敕守衛官常嚴步伍,譏察出入,以防微銷萌。」帝然其言,亦未能有所整飭。

正德初,嚴皇城紅鋪巡徼,日令留守衛指揮五員,督內外夜巡軍。而兵部郎中、主事各一人,同御史、錦衣衛稽閱,毋攝他務。嘉靖七年,增直宿官軍衣糧,五年一給。萬歷十一年,於皇城內外設把總二員,分東西管理。時門禁益弛,衛軍役於中官,每至空伍,賃市兒行丐應點閱。叉刀、紅盔日出始一入直,直廬虛無人。坐更將軍皆納月鏹於所轄。凡提號、巡城、印簿、走更諸事悉廢。十五年,再申門禁。久之,給事中吳文煒乞盡復舊制。不報。末年,有失金牌久之始覺者。梃擊之事,張差一妄男子,得闌入殿廷,其積弛可知。是後中外多事,啟、禎兩朝雖屢申飭,竟莫能挽,侵尋以至於亡。

京城巡捕之職,洪武初,置兵馬司,譏察奸偽。夜發巡牌,旗士領之,核城門扃鐍及夜行者。已改命衛所鎮撫官,而掌於中軍都督府。永樂中,增置五城兵馬司。宣德初,京師多盜,增官軍百人,協五城逐捕。已,復增夜巡候卒五百。成化中,始命錦衣官同御史督之。末年,撥給團營軍二百。弘治元年,令三千營選指揮以下四員,領精騎巡京城外,又令錦衣官五、旗手等衛官各一,分地巡警,巡軍給牌。五年,設把總都指揮,專職巡捕。正德中,添設把總,分畫京城外地,南抵海子,北抵居庸關,西抵盧溝橋,東抵通州。復增城內二員,而益以團營軍,定官卒賞罰例。末年,邏卒增至四千人,特置參將。

嘉靖元年,復增城外把總一員,並舊為五,分轄城內東西二路,城外西南、東南、東北三路,增營兵馬五千。又十選一,立尖哨五百騎,厚其月糈。命參將督操,而監以兵部郎。是時京軍弊壞積久,捕營亦然。三十四年,軍士僅三百餘。以給事中丘岳等言,削指揮樊經職,而禁以軍馬私役騎乘。萬歷十二年,從兵部議,京城內外盜發,自卯至申責兵馬司,自酉至寅責巡捕官,賊眾則協力捕剿。是後,軍額倍增,駕出及朝審、錄囚皆結隊駐巷口。籍伍雖具,而士馬實凋弊不足用。捕營提督一,參將二,把總十八,巡軍萬一千,馬五千匹。盜賊縱橫,至竊內中器物。獲其童索,竟不能得也。莊烈帝時,又以兵部左侍郎專督。然營軍半虛廩,馬多雇人騎,失盜嚴限止五日,玩法卒如故。

四衛營者,永樂時,以迤北逃回軍卒供養馬役,給糧授室,號曰勇士。後多以進馬者充,而聽御馬監官提調,名隸羽林,身不隸也。軍卒相冒,支糧不可稽。宣德六年,乃專設羽林三千戶所統之,凡三千一百餘人。尋改武驤、騰驤左右衛,稱四衛軍。選本衛官四員為坐營指揮,督以太監,別營開操,稱禁兵。器械、衣甲異他軍,橫於輦下,往往為中官占匿。弘治末,勇士萬一千七百八十人,旗軍三萬一百七十人,歲支廩粟五十萬。孝宗納廷臣言,核之。又令內臣所進勇士,必由兵部驗送乃給廩,五年籍其人數,著為令。省度支金錢歲數十萬。武宗即位,中官寧瑾乞留所汰人數。言官及尚書劉大夏持不可,不聽。後兩官廳設,遂選四衛勇士隸西官廳,掌以邊將江彬、太監張永等。

世宗入立,詔自弘治十八年存額外,悉裁之,替補必兵部查駁。而御馬監馬牛羊,令巡視科道核數。既而中旨免核,馬多虛增。後數年,御馬太監閔洪復矯旨選四衛官。給事中鄭自璧劾其欺蔽,不報。久之,兵部尚書李承勛請以選核仍隸本部,中官謂非便。帝從承勛言。十六年,又命收復登極詔書所裁者,凡四千人。後五年,內臣言,勇士僅存五千餘,請令子侄充選,以備邊警。部臣言:「故額定五千三百三十人。八年清稽,已浮其數,且此營本非為備邊設者。」帝從部議。然隱射、占役、冒糧諸弊率如故。萬曆二年,減坐營官二員。已,復定營官缺由兵部擇用。其後復為中官所撓,仍屬御馬監。廷臣多以為言,不能從。四十二年,給事中姚宗文點閱本營,言:「官勇三千六百四十七,僅及其半。馬一千四十三,則無至者。官旗七千二百四十,止四千六百餘。馬亦如之。乞下法司究治。」帝不能問。天啟末,巡視御史高弘圖請視三大營例,分弓弩、短兵、火器,加以訓練。至莊烈帝時,提督內臣曹化淳奏改為勇衛營,以周遇吉、黃得功為帥,遂成勁旅,出擊賊,輒有功。得功軍士畫虎頭於皁布以衣甲,賊望見黑虎頭軍,多走避,其得力出京營上云。


太祖下集慶路為吳王,罷諸翼統軍元帥,置武德、龍驤、豹韜、飛熊、威武、廣武、興武、英武、鷹揚、驍騎、神武、雄武、鳳翔、天策、振武、宣武、羽林十七衛親軍指揮使司。革諸將襲元舊制樞密、平章、元帥、總管、萬戶諸官號,而核其所部兵五千人為指揮,千人為千戶,百人為百戶,五十人為總旗,十人為小旗。天下既定,度要害地,係一郡者設所,連郡者設衛。大率五千六百人為衛,千一百二十人為千戶所,百十有二人為百戶所。所設總旗二,小旗十,大小聯比以成軍。其取兵,有從征,有歸附,有謫發。從征者,諸將所部兵,既定其地,因以留戍。歸附,則勝國及僭偽諸降卒。謫發,以罪遷隸為兵者。其軍皆世籍。此其大略也。

洪武三年,升杭州、江西、燕山、青州四衛為都衛,復置河南、西安、太原、武昌四都衛。四年,造用寶金符及調發走馬符牌。用寶符為小金牌二,中書省、大都督府各藏其一。有詔發兵,省府以牌入,內府出寶用之。走馬符牌,鐵為之,共四十,金字、銀字者各半,藏之內府。有急務調發,使者佩以行。尋改為金符。凡軍機文書,自都督府、中書省長官外,不許擅奏。有詔調軍,省、府同覆奏,然後納符請寶。五年,置親王護衛指揮使司,每府三護衛,衛設左、右、中、前、後五所;所,千戶二,百戶十。圍子手所二;所,千戶一。七年,申定兵衛之政,徵調則統於諸將,事平則散歸各衛。

八年,改在京留守都衛為留守衛指揮使司,在外都衛為都指揮使司,凡十三:北平、陜西、山西、浙江、江西、山東、四川、福建、湖廣、廣東、廣西、遼東、河南。又行都指揮使司二:甘州、大同。俱隸大都督府。九年,選公、侯、都督、各衛指揮嫡長次子為散騎、參侍舍人,隸都督府,充宿衛,或署各衛所事。十三年,丞相胡惟庸謀反誅,革中書省,因改大都督府為五,分統諸軍司衛所。明年,復置中都留守司及貴州、雲南都指揮使司。十五年三月,頒軍法定律。十六年,詔各都司上衛所城池水陸地里圖。二十年,置大寧都指揮使司。是年,命兵部置軍籍勘合,載從軍履歷、調補衛所年月、在營丁口之數,給內外衛所軍士,而藏其副於內府。三十年,定武官役軍之制:指揮、同知、僉事四,千戶三,百戶、鎮撫二,皆取正軍,三日一番上,下直歸伍操練。衛所直廳六,守門二,守監四,守庫一,皆任老軍,月一更。

建文帝嗣位,置河北都司、湖廣行都司。文皇入立,皆罷之,而升燕山三護衛為親軍,並建文時所立孝陵衛初學者又展開了名實之爭。,皆不隸五府。後諸陵設衛皆如之。移山西行都司所屬諸衛軍於北平,設衛屯種。永樂元年,罷北平都司,設留守行後軍都督府,遷大寧都司於保定。明年,更定衛所屯守軍士。臨邊險要者,守多於屯。在內平僻,或地雖險要而運輸難至者,皆屯多於守。七年,置調軍勘合,以勇、敢、鋒、銳、神、奇、精、壯、強、毅、克、勝、英、雄、威、猛十六字,編百號。制敕調軍及遣將,比號同,方准行。十八年,北京建,在南諸衛多北調。宣德五,年從平江伯陳瑄言,以衛官職漕運,東南之卒由是困。八年,減衛軍餘丁,正軍外每軍留一,餘悉遣歸。已,復以幼軍備操者不足,三丁至七八丁者選一,余聽治生,給軍裝。正軍有故,即令補伍,毋再勾攝。

當是時,都指揮使與布、按並稱三司,為封疆大吏。而專閫重臣,文武亦無定職,世猶以武為重實」、「實證」的事實為依據,獲得關於現象的知識,摒棄對,軍政修飭。正德以來,軍職冒濫,為世所輕。內之部科,外之監軍、督撫,疊相彈壓,五軍府如贅疣,弁帥如走卒。總兵官領敕於兵部,皆跽,間為長揖,即謂非禮。至於末季,衛所軍士,雖一諸生可役使之。積輕積弱,重以隱占、虛冒諸弊,至舉天下之兵,不足以任戰守,而明遂亡矣。

崇禎三年,范景文以兵部侍郎守通州,上言:「祖制,邊腹內外,衛所棋置目」兩種形式出現。「魏晉之際,天下多故,名士少有全者」,以軍隸衛,以屯養軍。後失其制,軍外募民為兵,屯外賦民出餉,使如鱗尺籍,不能為衝鋒之事,並不知帶甲之人。陛下百度振刷,豈可令有定之軍數付之不可問,有用之軍糈投之不可知?」因條上清核數事,不果行。

初,洪武二十六年定天下都司衛所,共計都司十有七,留守司一,內外衛三百二十九陸王心學即南宋陸九淵和明王守仁兩大學說的合稱。陸,守禦千戶所六十五。及成祖在位二十餘年,多所增改。其後措置不一,今區別其名於左,以資考鏡。

○上十二衛

金吾前衛金吾後衛羽林左衛羽林右衛府軍衛府軍左衛

府軍右衛府軍前衛府軍後衛虎賁左衛錦衣衛旂手衛

◎五軍都督府所屬衛所

◎左軍都督府

○在京凡本府在京屬衛,曾經永樂十八年調守北京者,各註其下曰「調北京」,其年月不重出。後四府同。

留守左衛調北京鎮南衛調北京水軍左衛驍騎右衛調北京龍虎衛調北京英武衛沈陽左衛調北京沈陽右衛調北京

○在外

△浙江都司

杭州前衛杭州右衛台州衛寧波衛處州衛紹興衛海寧衛昌國衛溫州衛臨山衛松門衛金鄉衛定海衛海門衛盤石衛觀海衛海寧千戶所衢州千戶所嚴州千戶所湖州千戶所

△遼東都司

定遼左衛定遼右衛定遼中衛定遼前衛定遼後衛鐵嶺衛東寧衛沈陽中衛海州衛蓋州衛金州衛復州衛義州衛遼海衛三萬衛廣寧左屯衛廣寧右屯衛廣寧前屯衛廣寧後屯衛廣寧中護衛後改為屯衛

△山東都司

青州左護衛後為天津右衛青州護衛革兗州護衛革兗州左護衛後為臨清衛登州衛青州左衛萊州衛寧海衛濟南衛平山衛德州衛後改屬後府樂安千戶所後改名武定,屬後府膠州千戶所諸城千戶所滕縣千戶所

◎右軍都督府

○在京

虎賁右衛調北京留守右衛調北京水軍右衛武德衛調北京廣武衛

○在外

△雲南都司

雲南左衛雲南右衛雲南前衛大理衛楚雄衛臨安衛景東衛曲靖衛金齒衛洱海衛蒙化衛馬隆衛改雲南右護衛,革平夷衛越州衛六涼衛鶴慶千戶所革

△貴州都司

貴州衛永寧衛普定衛平越衛烏撒衛普安衛層臺衛革赤水衛威清衛興隆衛新添衛清平衛平壩衛安莊衛龍里衛安南衛都勻衛畢節衛黃平千戶所

△四川都司

成都左護衛成都右護衛後為龍虎左衛,隸南京左府成都中護衛後為豹韜左衛,隸南京前府成都左衛革成都右衛成都前衛成都後衛成都中衛寧川衛茂州衛建昌衛後屬行都司重慶衛敘南衛蘇州衛後為寧番衛,屬行都司,革瀘州衛松潘軍民指揮使司巖州衛革青川千戶所威州千戶所大渡河千戶所

△陜西都司

西安左護衛後為神武右衛西安右護衛西安中護衛後為神武前衛西安左衛西安右衛改西安中護衛西安前衛西安後衛華山衛改西安左護衛,又改神武右衛泰山衛改西安右護衛延安衛綏德衛平涼衛慶陽衛寧夏衛臨洮衛鞏昌衛西寧衛後屬行都司漢中衛涼州衛後屬行都司莊浪衛後屬行都司蘭州衛秦州衛岷州軍民指揮使司洮州衛河州軍民指揮使司甘肅衛後為甘州後衛山丹衛後屬行都司永昌衛後屬行都司鳳翔千戶所金州千戶所寧夏中護衛西河中護衛後改雲南中護衛,革

△廣西都司

桂林左衛後為廣西護衛桂林右衛桂林中衛南寧衛柳州衛馴象衛梧州千戶所

○中軍都督府

○在京

留守中衛調北京神策衛調北京廣洋衛應天衛調北京和陽衛調北京牧馬千戶所調北京

○在外

△直隸

揚州衛和州衛後改為寧夏中屯衛,革高郵衛淮安衛鎮海衛滁州衛太倉衛泗州衛壽州衛邳州衛大河衛沂州衛金山衛新安衛蘇州衛儀真衛徐州衛安慶衛宿州千戶所

△中都留守司

鳳陽右衛鳳陽中衛皇陵衛鳳陽衛留守左衛留守中衛長淮衛懷遠衛洪塘千戶所

△河南都司

歸德衛後屬中府陳州衛弘農衛汝寧衛後改千戶所,屬中府潼關衛後屬中府河南衛睢陽衛宣武衛信陽衛彰德衛武平衛後屬中府南陽衛寧國衛後為涿鹿衛,後屬後府懷慶衛寧山衛後屬後府潁州衛安吉衛後為通州衛親軍潁上千戶所河南左護衛河南中護衛河南右護衛三護衛後并彭城衛

◎前軍都督府

○在京

天策衛後分為保安衛及保安右衛龍驤衛調北京豹韜衛調北京龍江衛後改為龍江左衛飛熊衛調北京

○在外

△直隸

九江衛

△湖廣都司

武昌衛武昌左衛黃州衛永州衛岳州衛蘄州衛施州衛長沙護衛革辰州衛安陸衛後屬行都司,改承天衛襄陽衛襄陽護衛後俱屬行都司常德衛沅州衛寶慶衛沔陽衛後屬興都留守司長沙衛茶陵衛衡州衛瞿塘衛後屬行都司鎮遠衛平溪衛清浪衛偏橋衛五開衛九溪衛荊州左護衛後為荊州左衛,屬行都司,改顯陵衛荊州中護衛革靖州衛永定衛郴州千戶所夷陵千戶所後屬行都司桂陽千戶所德安千戶所後改屬興都留守司忠州千戶所後屬行都司安福千戶所道州千戶所革大庸千戶所西平千戶所革麻寮千戶所枝江千戶所後屬行都司武岡千戶所崇山千戶所革長寧千戶所後屬行都司武昌左、右、中三護衛左改東昌衛,右改徐州左衛,中改武昌護衛。

△福建都司

福州中衛福州左衛福州右衛興化衛泉州衛漳州衛福寧衛鎮東衛平海衛永寧衛鎮海衛

△福建行都司

建寧左衛建寧右衛建陽衛革延平衛邵武衛汀州衛將樂千戶所

△江西都司

南昌左衛南昌前衛袁州衛贛州衛吉安衛後為千戶所饒州千戶所安福千戶所會昌千戶所永新千戶所南安千戶所建昌千戶所撫州千戶所鉛山千戶所廣信千戶所

△廣東都司

廣州前衛廣州左衛廣州右衛南海衛潮州衛雷州衛海南衛清遠衛惠州衛肇慶衛廣州後衛程鄉千戶所高州千戶所廉州千戶所後為廉州衛萬州千戶所儋州千戶所崖州千戶所南雄千戶所韶州千戶所德慶千戶所新興千戶所陽江千戶所新會千戶所龍川千戶所

◎後軍都督府

○在京

橫海衛鷹揚衛興武衛調北京江陰衛蒙古左衛革蒙古右衛革

○在外

△北平都司

燕山左衛燕山右衛燕山前衛大興左衛永清左衛永清右衛濟州衛濟陽衛彭城衛通州衛已上俱改為親軍薊州衛密雲衛後為密雲後衛,屬後府真定衛永平衛山海衛遵化衛居庸關千戶所後為隆慶衛已上俱屬後府

△北平行都司後為大寧都司大寧左衛大寧右衛二衛後為營州左、右護衛,改延慶左、右衛大寧中衛大寧前衛大寧後衛後為營州中護衛,改寬河衛會州衛俱改調京衛已上俱屬後府營州中護衛興州中護衛革

△山西都司

太原左衛太原右衛太原前衛振武衛平陽衛鎮西衛潞州衛蒲州千戶所廣昌千戶所沁州千戶所寧化千戶所雁門千戶所

△山西行都司

大同左衛大同右衛大同前衛蔚州衛朔州衛

△北平三護衛

燕山左護衛燕山右護衛燕山中護衛俱為親軍

△山西三護衛

太原左護衛太原右護衛太原中護衛俱革

後定天下都司衛所,共計都司二十一,留守司二,內外衛四百九十三,守禦屯田群牧千戶所三百五十九,儀衛司三十三,自儀衛司以下,舊無,後以次漸添設。宣慰使司二,招討使司二,宣撫司六,安撫司十六,長官司七十,原五十九。番邊都司衛所等四百七。後作四百六十三。

親軍上二十二衛,舊制止十二衛,後增設金吾左以下十衛,俱稱親軍指揮使司,不屬五府。又設騰驤等四衛,亦係親軍,并武功、永清、彭城及長陵等十五衛,俱不屬府。

金吾前衛金吾後衛羽林左衛羽林右衛府軍衛府軍左衛府軍右衛府軍前衛府軍後衛虎賁左衛錦衣衛旗手衛以上舊為上十二衛金吾右衛羽林前衛以上北平三護衛,洪武三十五年升燕山左衛燕山右衛燕山前衛大興左衛濟陽衛濟州衛通州衛舊為安吉衛已上北平都司七衛,永樂四年升,俱為親軍騰驤左衛騰驤右衛舊為神武前衛武驤左衛武驤右衛已上四衛,宣德八年以各衛養馬軍士及神武前衛官軍開設武功中衛洪武年間設武功左衛宣德二年設武功右衛宣德六年設永清左衛永清右衛彭城衛已上北平三衛,改常山三護衛,宣德初復為本衛,又併河南三護衛多餘官軍於彭城衛長陵衛舊為南京羽林右衛,永樂二十二年改獻陵衛舊武成左衛,宣德元年改景陵衛舊武成右衛,宣德十年改裕陵衛舊武成前衛,天順八年改茂陵衛舊武成後衛,成化二十三年改泰陵衛舊忠義左衛,弘治十八年改康陵衛舊義勇中衛,正德十六年改永陵衛舊義勇左衛,嘉靖二十七年改昭陵衛舊神武後衛,隆慶六年改定陵衛慶陵衛德陵衛奠靖千戶所嘉靖二十一年設犧牲千戶所屬太常寺轄已上俱不屬五府

◎五軍都督府所屬衛所

◎左軍都督府

○在京

留守左衛鎮南衛驍騎右衛龍虎衛沈陽左衛沈陽右衛俱南京舊制,永樂十八年分調

○在外

△浙江都司

杭州前衛杭州後衛臺州衛寧波衛處州衛紹興衛海寧衛昌國衛溫州衛臨山衛松門衛金鄉衛海門衛定海衛盤石衛觀海衛海寧千戶所衢州千戶所嚴州千戶所湖州千戶所金華千戶所澉浦千戶所以下各所,舊無,後添設乍浦千戶所三江千戶所定海後千戶所定海中左千戶所定海中中千戶所瀝海千戶所三山千戶所大嵩千戶所霩戺千戶所龍山千戶所石浦前千戶所石浦後千戶所爵谿千戶所錢倉千戶所水軍千戶所新河千戶所桃渚千戶所健跳千戶所隘頑千戶所楚門千戶所平陽千戶所瑞安千戶所海安千戶所蒲門千戶所壯士千戶所沙園千戶所蒲岐千戶所寧村千戶所新城千戶所舊有,後革

△遼東都司

定遼左衛定遼右衛定遼中衛定遼前衛定遼後衛鐵嶺衛東寧衛沈陽中衛海州衛蓋州衛金州衛復州衛義州衛遼海衛三萬衛廣寧左屯衛廣寧右屯衛廣寧中屯衛廣寧前屯衛廣寧後屯衛廣寧衛已下添設廣寧左衛廣寧右衛廣寧中衛寧遠衛撫順千戶所蒲河千戶所寧遠中左千戶所寧遠中右千戶所廣寧中前千戶所廣寧中後千戶所廣寧中左千戶所金州中左千戶所鐵嶺左右千戶所鐵嶺中左千戶所三萬前前千戶所三萬後後千戶所三萬中中千戶所遼海中中千戶所遼海右右千戶所遼海前前千戶所遼海後後千戶所東寧中左千戶所

△山東都司舊有青州左護衛,後改天津右衛。舊有貴州護衛,革

登州衛青州左衛萊州衛寧海衛濟南衛平山衛安東衛已下添設靈山衛鰲山衛大嵩衛威海衛成山衛靖海衛東昌衛臨清衛舊兗州左護衛,後改任城衛濟寧衛舊武昌左護衛,後改兗州護衛膠州千戶所諸城千戶所滕縣千戶所肥城千戶所已下添設海陽千戶所東平千戶所寧津千戶所雄崖千戶所浮山前千戶所福山中前千戶所奇山千戶所濮州千戶所金山左千戶所尋山後千戶所百尺崖後千戶所王徐寨前千戶所夏河寨前千戶所魯府儀衛司德府儀衛司涇府儀衛司衡府儀衛司德府群牧所涇府群牧所衡山群牧所

◎右軍都督府

○在京

留守右衛虎賁右衛武德衛俱南京舊衛,永樂十八年分調

○在外

△直隸

宣州衛舊無,後設

△陜西都司舊有階州衛、沙州衛、靈山千戶所,後俱革。

西安右護衛舊泰山衛改西安左衛西安前衛西安後衛延安衛漢中衛平涼衛綏德衛寧夏衛慶陽衛鞏昌衛臨洮衛蘭州衛秦州衛岷州衛舊軍民指揮使司,嘉靖二十四年添設岷州,四十年革,後存衛河州衛舊軍民指揮使司洮州衛寧夏中護衛甘州中護衛安東中護衛寧夏前衛已下各衛舊無,後設寧夏中衛寧夏中屯衛舊和州衛寧夏左屯衛寧夏右屯衛寧羌衛靖虜衛固原衛榆林衛寧夏後衛以花馬池千戶所改興安千戶所舊金州千戶所,萬歷十年改鳳翔千戶所禮店前千戶所以下各所舊設沔縣千戶所環縣千萬所文縣千戶所階州千戶所舊屬秦州衛,嘉靖二十二年改屬都司靈州千戶所西安千戶所西固城千戶所,歸德千戶所鎮羌千戶所安邊千戶所平虜千戶所興武營千戶所鎮戎千戶所寧夏平虜千戶所秦府儀衛司慶府儀衛司肅府儀衛司韓府儀衛司寧夏群牧所安東群牧所甘州群牧所

△陜西行都司洪武十二年添設

甘州左衛甘州右衛甘州中衛甘州前衛甘州後衛已上陜西甘肅衛分設永昌衛涼州衛莊浪衛西寧衛山丹衛已上舊屬陜西都司肅州衛鎮番衛鎮夷千戶所古浪千戶所高臺千戶所

△四川都司舊有浦江關軍民千戶所,後革

成都左護衛成都右衛成都中衛成都前衛成都後衛寧川衛茂州衛重慶衛敘南衛瀘州衛利州衛舊無,後設松潘衛舊為軍民指揮使司,後改青川千戶所保寧千戶所威州千戶所雅州千戶所大渡河千戶所廣安千戶所灌縣千戶所已下各所後設黔江千戶所疊溪千戶所建武千戶所小河千戶所蜀府儀衛司壽府儀衛司革壽府群牧所革

△土官

天全六番招討使司屬都司隴木頭長官司靜州長官司岳希蓬長官司已上屬茂州衛石砫宣撫司西陽宣撫司已上屬重慶衛石耶洞長官司邑梅洞長官司已上屬酉陽宣撫司占藏先結簇長官司蠟匝簇長官司白馬路簇長官司發山洞簇長官司阿昔洞簇長官司北定簇長官司麥匝簇長官司者多簇長官司牟力簇長官司班班簇長官司祈命簇長官司勒都簇長官司包藏簇長官司阿思簇長官司思曩兒簇長官司阿用簇長官司潘斡寨長官司八郎安撫司阿角寨安撫司麻兒匝安撫司芒兒者安撫司已上俱屬松潘衛疊溪長官司鬱即長官司已上屬疊溪千戶所

△四川行都司舊無,後設。舊有建昌前衛,後革

建昌衛舊屬四川都司寧番衛舊為蘇州衛,屬四川都司已下添設會川衛鹽井衛越巂衛禮州後千戶所禮州中中千戶所建昌打沖河中前千戶所德昌千戶所迷易千戶所鹽井打沖河中左千戶所冕山橋後千戶所鎮西後千戶所

△土官

昌州長官司威龍長官司普濟長官司俱屬建昌衛馬喇長官司屬鹽井衛邛部長官司屬越巂衛

△廣西都司

桂林右衛桂林中衛南寧衛柳州衛馴象衛南丹衛已下添設慶遠衛潯州衛奉議衛廣西護衛梧州千戶所懷集千戶所武緣千戶所古田千戶所貴縣千戶所賀縣千戶所全州千戶所太平千戶所象州千戶所平樂千戶所鬱林千戶所賓州千戶所來賓千戶所富川千戶所容縣千戶所融縣千戶所灌陽千戶所河池千戶所武宣千戶所向武千戶所五屯屯田千戶所遷江屯田千戶所靖江府儀衛司

△雲南都司舊有鶴慶、通海二千戶所,革

雲南左衛雲南右衛雲南前衛大理衛楚雄衛臨安衛景東衛曲靖衛洱海衛永昌衛舊為金齒軍民指揮使司蒙化衛平夷衛趙州衛六涼衛雲南中衛雲南後衛已下後設廣南衛大羅衛瀾滄衛以瀾滄軍民指揮使司改騰沖衛以騰衝軍民指揮使司改安寧千戶所宜良千戶所易門千戶所楊林堡千戶所十八寨千戶所通海前前千戶所通海右右千戶所定遠千戶所馬隆千戶所姚安千戶所姚安中屯千戶所武定千戶所木密關千戶所鎮安千戶所舊為金齒千戶所,萬曆十三年改,駐守猛淋鎮姚千戶所舊為永昌千戶所,萬歷十三年改,駐守老姚關永平前前千戶所永平後後千戶所騰沖千戶所新安千戶所鳳梧千戶所

△土官

茶山長官司潞江安撫司鳳溪長官司施甸長官司鎮道安撫司楊塘安撫司俱屬永昌衛蠻莫安撫司猛臉長官司猛養長官司俱萬曆十三年改設

△貴州都司舊有層臺、重安二千戶所,俱革。舊有平伐長官司,後隸貴陽府。舊有平浪、九名九姓獨山州二長官司,後隸都勻府。

貴州衛永寧衛普定衛平越衛烏撒衛普安衛赤水衛威清衛興隆衛新添衛清平衛平壩衛安莊衛龍里衛安南衛都勻衛畢節衛貴州前衛舊無,後設黃平千戶所普市千戶所重安千戶所安龍千戶所白撒千戶所摩泥千戶所關索嶺千戶所阿落密千戶所平夷千戶所安南千戶所樂民千戶所七星關千戶所

△土官

新添長官司小平伐長官司把平寨長官司丹平長官司丹行長官司已上屬新添衛楊義長官司屬平越衛大平伐長官司屬龍里衛

◎中軍都督府

○在京

留守中衛神策衛應天衛和陽衛俱南京舊衛,永樂十八年調牧馬千戶所南京舊所調蕃牧千戶所添設

○在外

△直隸

揚州衛高郵衛儀真衛淮安衛鎮海衛滁州衛徐州衛蘇州衛太倉衛金山衛新安衛泗州衛壽州衛邳州衛大河衛沂州衛安慶衛宿州衛舊為千戶所潼關衛已下舊屬河南都司歸德衛武平衛鎮江衛已下添設廬州衛六安衛徐州左衛建陽衛汝寧千戶所松江中千戶所青村中前千戶所南匯嘴中後千戶所嘉興中左千戶所在府吳淞江千戶所寶山千戶所劉河堡中千戶所崇明沙千戶所興化千戶所通州千戶所泰州千戶所鹽城千戶所東海中千戶所海州中前千戶所莒州千戶所

△中都留守司

鳳陽衛鳳陽中衛鳳陽右衛皇陵衛留守左衛留守中衛長淮衛懷遠衛洪塘千戶所

△河南都司舊有洛陽中護衛,後並汝州衛。

河南衛弘農衛陳州衛睢陽衛宣武衛信陽衛彰德衛南陽衛懷慶衛潁川衛南陽中護衛已下添設汝州衛潁上千戶所禹州千戶所舊名鈞州,後改嵩縣千戶所衛輝前千戶所林縣千戶所鄧州前千戶所唐縣右千戶所周府儀衛司唐府儀衛司伊府儀衛司趙府儀衛司鄭府儀衛司崇府儀衛司徽府儀衛司趙府群牧所鄭府群牧所崇府群牧所徽府群牧所

◎前軍都督府

○在京

留守前衛龍驤衛豹韜衛俱南京舊衛,永樂十八年分調

○在外

△直隸

九江衛

○湖廣都司舊有武昌右千戶所,革。

武昌衛武昌左衛黃州衛永州衛岳州衛蘄州衛施州衛辰州衛常德衛沅州衛寶慶衛沔陽衛長沙衛衡州衛茶陵衛鎮遠衛偏橋衛

清浪衛已上三衛在貴州境平溪衛五開衛九溪衛靖州衛永定衛寧遠衛已下添設銅鼓衛武昌護衛襄陽護衛郴州千戶所麻寮千戶所添平千戶所安福千戶所忠州千戶所在四川境大庸千戶所桂陽千戶所武岡千戶所澧州千戶所寧溪千戶所常寧千戶所鎮溪千戶所桃川千戶所枇杷千戶所錦田千戶所寧遠千戶所江華千戶所城步千戶所天柱千戶所汶溪千戶所宜章千戶所廣安千戶所大田千戶所黎平千戶所中潮千戶所新化千戶所新化亮寨千戶所隆裏千戶所已上五所在貴州境平茶千戶所平茶屯千戶所銅鼓千戶所楚府儀衛司荊府儀衛司雍府儀衛司榮府儀衛司岷府儀衛司吉府儀衛司荊府群牧所雍府群牧所榮府群牧所吉府群牧所

○土官

永順軍民宣慰使司屬都司臘惹洞長官司麥著黃洞長官司驢遲洞長官司施溶溪長官司白崖洞長官司田家洞長官司已上屬永順宣慰司保靖州軍民宣慰使司屬都司五寨長官司筸子坪長官司俱屬保靖宣慰司施南宣撫司屬施州衛東鄉五路安撫司屬施南宣撫司搖把洞長官司上愛茶峒長官司下愛茶峒長官司鎮遠蠻夷長官司隆奉蠻夷長官司俱屬東鄉五路安撫司忠孝安撫司屬施南忠路安撫司屬施南金峒安撫司屬施南劍南長官司屬忠路西坪蠻夷長官司屬金峒散毛宣撫司屬施州衛龍潭安撫司大旺安撫司俱屬散毛東流蠻夷長官司臘璧峒蠻夷長官司俱屬大旺忠建宣撫司屬施州衛忠峒安撫司高羅安撫司屬忠建木冊長官司屬高羅鎮南長官司唐崖長官司容美宣撫司俱屬施州衛椒山瑪瑙長官司五峰石寶長官司水盡源通塔平長官司石梁下峒長官司俱屬容美桑植安撫司屬九溪臻剖六洞橫波等處長官司屬鎮遠衛

○湖廣行都司以湖廣都司衛所改設

荊州衛荊州左衛荊州右衛瞿塘衛襄陽衛襄陽護衛安陸衛鄖陽衛夷陵千戶所德安千戶所枝江千戶所長寧千戶所遠安千戶所竹山千戶所均州千戶所房縣千戶所忠州千戶所遼府儀衛司襄府儀衛司興府儀衛司

○興都留守司

承天衛舊安陸衛,嘉靖十八年改沔陽衛舊屬都司,嘉靖二十一年改顯陵衛舊為荊州左衛,嘉靖十八年改德安千戶所舊屬行都司,嘉靖二十一年改

○福建都司

福州中衛福州左衛福州右衛興化衛泉州衛漳州衛福寧衛鎮東衛平海衛永寧衛鎮海衛大金千戶所巳下添設定海千戶所梅花千戶所萬安千戶所莆禧千戶所福全千戶所金門千戶所中左千戶所高浦千戶所浦城千戶所六鰲千戶所銅山千戶所玄鍾千戶所崇武千戶所南詔千戶所龍巖千戶所

○福建行都司

建寧左衛建寧右衛延平衛邵武衛汀州衛將樂千戶所武平千戶所已下添設永安千戶所上杭千戶所浦城千戶所

○江西都司

南昌衛正德十六年,以左、前二衛並改袁州衛贛州衛吉安千戶所舊為衛饒州千戶所安福千戶所會昌千戶所永新千戶所南安千戶所建昌千戶所撫州千戶所鉛山千戶所廣信千戶所信豐千戶所寧府儀衛司淮府儀衛司益府儀衛司淮府群牧所益府群牧所

○廣東都司

廣州前衛廣州後衛廣州左衛廣州右衛南海衛潮州衛雷州衛海南衛清遠衛惠州衛肇慶衛廣海衛已下添設碭石衛神電衛廉州衛舊千戶所新會千戶所韶州千戶所南雄千戶所龍川千戶所程鄉千戶所德慶千戶所新興千戶所陽江千戶所高州千戶所儋州千戶所新寧千戶所萬州千戶所崖州千戶所增城千戶所東莞千戶所已下添設大鵬千戶所香山千戶所連州千戶所河源千戶所長樂千戶所平海千戶所海豐千戶所捷勝千戶所甲子門千戶所大城千戶所海門千戶所靖海千戶所蓬州千戶所澄海千戶所廣寧千戶所四會千戶所陽春千戶所海朗千戶所雙魚千戶所寧川千戶所信宜千戶所石城千戶所永安千戶所欽州千戶所靈山千戶所海康千戶所樂民千戶所海安千戶所錦囊千戶所清瀾千戶所昌化千戶所南山千戶所瀧水千戶所從化千戶所封門千戶所函口千戶所富霖千戶所

◎後軍都督府

○在京

留守後衛鷹揚衛興武衛俱南京舊衛,永樂十八年分調大寧中衛大寧前衛會州衛俱北平行都司舊衛富峪衛已下添設,并北平山西等衛改調寬河衛舊大寧後衛神武左衛神武后衛改昭陵衛忠義左衛忠義右衛忠義前衛忠義後衛義勇中衛義勇左衛義勇右衛義勇前衛義勇後衛武成中衛蔚州左衛

○在外

△直隸舊為北平都司,有北平三護衛,後俱為親軍。其不係北平舊衛者,俱永樂以後添設。

薊州衛真定衛永平衛山海衛遵化衛已上北平舊衛密雲中衛密雲後衛以舊密雲分開平中屯衛興州左屯衛興州右屯衛興州中屯衛

興州前屯衛興州後屯衛延慶衛舊為北平都司居庸關千戶所,後改隆慶衛,後又改此東勝左衛東勝右衛鎮朔衛涿鹿衛舊為河南寧國衛,屬中府定邊衛神武右衛神武中衛忠義中衛盧龍衛武清衛撫寧衛德州衛寧山衛舊屬河南都司,屬中府大同中屯衛永樂初改調沈陽中屯衛定州衛已上舊為北平、山東、山西、河南等處衛所,永樂初改調天津衛已下添設天津左衛天津右衛舊青州左護衛通州左衛通州右衛涿鹿左衛涿鹿中衛河間衛潼關衛舊屬河南都司德州左衛梁城千戶所滄州千戶所已下添設倒馬關千戶所潮河千戶所白洋口千戶所渤海千戶所寬河千戶所鎮邊城千戶所順德千戶所武定千戶所舊樂安千戶所,改屬平定千戶所蒲州千戶所俱屬山西都司,後改

○大寧都司

保定左衛保定右衛保定中衛保定前衛保定後衛俱永樂元年設營州左屯衛營州右屯衛營州中屯衛營州前屯衛營州後屯衛俱洪武舊衛,永樂改屬茂山衛紫荊關千戶所

○萬全都司宣德五年,分直隸及山西等處衛所添設。

萬全左衛萬全右衛宣府前衛宣府左衛宣府右衛懷安衛開平衛延慶左衛舊屬北平行都司,後改延慶右衛舊屬北平都司,後改龍門衛保安衛舊屬前府,後改保安右衛舊屬前府,後改蔚州衛永寧衛懷來衛興和千戶所美峪千戶所廣昌千戶所舊屬山西都司,後改四海冶千戶所長安千戶所雲川千戶所龍門千戶所

○山西都司舊有太原三護衛,後革。蒲州千戶所,改屬直隸,廣昌千戶所,改屬萬全都司

太原左衛太原右衛太原前衛振武衛平陽衛鎮西衛潞州衛沈陽中護衛後設汾州衛後設沁州千戶所寧化千戶所雁門千戶所保德州千戶所已下添設偏頭關千戶所磁州千戶所寧武千戶所八角千戶所老營堡千戶所嘉靖十七年添設晉府儀衛司沈府儀衛司代府儀衛司晉府群牧所沈府群牧所代府群牧所

○山西行都司舊有蔚州衛,後改屬萬全都司

大同左衛大同右衛大同前衛大同後衛朔州衛已下俱山西大同等處衛所調改及添設鎮虜衛安東中屯衛陽和衛玉林衛高山衛雲川衛天城衛威遠衛平虜衛山陰千戶所馬邑千戶所井坪千戶所

◎南京衛所親軍衛

金吾前衛金吾後衛羽林左衛羽林右衛羽林前衛府軍衛府軍左衛府軍右衛府軍前衛府軍後衛虎賁左衛錦衣衛旂手衛金吾左衛金吾右衛江淮衛濟川衛孝陵衛犧牲千戶所

◎五軍都督府屬

○左軍都督府本府所屬衛,仍隸北京左府。

留守左衛鎮南衛水軍左衛驍騎右衛龍虎衛龍虎左衛舊為成都右護衛,宣德六年改英武衛沈陽左衛沈陽右衛龍江右衛

○右軍都督府本府所屬衛,仍隸北京右府。

虎賁右衛留守右衛水軍右衛武德衛廣武衛

○中軍都督府本府所屬衛,仍隸北京中府。

留守中衛神策衛廣洋衛應天衛和陽衛牧馬千戶所

○前軍都督府本府所屬衛,仍隸北京前府。

留守前衛龍江左衛龍驤衛飛熊衛天策衛豹韜衛豹韜左衛舊為成都中護衛,宣德六年改調

○後軍都督府本府所屬衛,仍隸北京後府。

留守後衛橫海衛鷹揚衛興武衛江陰衛

羈縻衛所,洪武、永樂間邊外歸附者,官其長,為都督、都指揮、指揮、千百戶、鎮撫等官,賜以敕書印記,設都司衛所。

◎都司一奴兒乾都司

○衛三百八十四

朵顏衛泰寧衛建州衛必里衛舊《會典》作兀里福餘衛已上洪武間置兀者衛兀者左衛兀者右衛兀者後衛赤不罕衛屯河衛安河衛已上永樂二年置毛憐衛虎兒文衛失里綿衛奴兒乾衛堅河衛舊《會典》有溫河撒力衛已上永樂三年置古賁河衛右城衛塔魯木衛蘇溫河衛斡灘河衛舊《會典》有灘納河兀者前衛卜顏衛亦罕河衛納憐河衛麥蘭河衛兀列河衛雙城衛撒剌兒衛亦馬剌衛斡蘭衛亦兒古里衛脫木河衛卜剌罕衛密陳衛脫倫衛嘉河衛塔山衛阿速江衛速平江衛木魯罕山衛馬英山衛土魯亭山衛木塔裏山衛朵林山衛兀也吾衛吉河衛劄竹哈衛舊《會典》有撒竹籃福山衛舊《會典》作福三肥河衛哈溫河衛舊《會典》作哈里河木束河衛撒兒忽衛罕答河衛舊《會典》作忽答河劄童衛已上永樂四年置阿古河衛喜樂溫河衛木陽河衛哈蘭城衛可令河衛兀的河衛哥吉河衛野木河衛納剌吉河衛亦里察河衛野兒定河衛卜魯丹河衛好屯河衛喜剌烏河衛舊《會典》作喜速烏考郎兀衛亦速里河衛阿剌山衛隨滿河衛撒禿河衛忽蘭山衛古魯渾山衛阿資河衛甫裏河衛答剌河衛舊《會典》作納剌河撒只剌河衛阿里河衛舊《會典》作阿吉河依木河衛亦文山衛木蘭河衛朵兒必河衛甫門河衛已上永樂五年置納木河衛童寬山衛兀魯罕河衛塔罕山衛者帖列山衛木興衛友帖衛牙魯衛益實衛剌魯衛乞忽衛兀里溪山衛希灘河衛弗朵禿河衛阿者迷河衛撒察河衛斡蘭河衛阿真河衛木忽剌河衛欽真河衛克默河衛察剌禿山衛嘔罕河衛阮里河衛列門河衛禿都河衛實山衛忽里急山衛莫溫河衛薛列河衛已上永樂六年置卜魯兀衛葛林衛把城衛劄肥河衛忽石門衛劄嶺上衛木里吉衛忽兒海衛伏里其衛乞勒尼衛愛河衛把河衛和屯吉衛失里木衛阿倫衛古里河衛塔麻速衛已上永樂七年置木興河衛木剌河衛舊《會典》作木束河衛喜申衛使防河衛舊《會典》作使方河甫兒河衛亦麻河衛兀應河衛法因河衛阿答赤河衛舊《會典》作阿答古木山衛葛稱哥衛已上永樂八年置督罕河衛建州左衛只兒蠻衛兀剌衛順民衛囊哈兒衛古魯衛舊《會典》作古魯山滿徑衛哈兒蠻衛塔亭衛也孫倫衛可木河衛弗思木衛弗提衛已上永樂十年置斡朵倫衛永樂十一年置哈兒分衛阿兒溫河衛速塔兒河衛兀屯河衛玄城衛和卜羅衛老哈河衛失兒兀赤衛卜魯禿河衛可河衛乞塔河衛兀剌忽衛已上永樂十二年置渚冬河衛劄真衛兀里哈里衛忽魯愛衛已上永樂十三年置吉灘河衛亦馬忽山衛已上永樂十四年置阿真同真衛亦東河衛亦迷河衛已上永樂十五年置建州右衛益實左衛阿答赤衛塔山左衛舊《會典》作塔山前城討溫衛舊《會典》作「成」,已上俱正統間置寄住毛憐衛此下正統已後續置可木衛失里衛失木魯河衛忽魯木衛塔馬速衛失烈木衛吉灘衛和屯衛禾屯吉河衛亦失衛亦力克衛納木衛弗納河衛忽失木衛兀也衛也速倫衛巴忽魯衛兀牙山衛塔木衛忽裏山衛罕麻衛木里吉河衛引門河衛亦里察衛只卜得衛塔兒河衛木忽魯衛木答山衛立山衛可吉河衛忽失河衛脫倫兀衛阿的納河衛兀力衛阿速衛速溫河衛納剌吉衛撒剌衛亦實衛弗朵脫河衛亦屯河衛兀討溫河衛甫河衛剌山衛阿者衛童山寬衛替里衛亦里察河衛哈黑分衛禿河衛好屯衛乞列尼衛撒里河衛忽思木衛兀里河衛忽魯山衛弗兒秀河衛沒脫倫衛阿魯必河衛咬里山衛亦文衛寫豬洛衛答里山衛古木河衛剌兒衛兀同河衛出萬山衛者屯衛喜辰衛海河衛蘭河衛朵州山衛者亦河衛納速吉河衛把忽兒衛鎮真河衛也速河衛者剌禿衛也魯河衛亦里河衛失里兀衛斡朵里衛禿屯河衛者林山衛波羅河衛朵兒平河衛散力衛密剌禿山衛甫門衛細木河衛沒倫河衛弗禿都河衛者列帖衛察札禿河衛出萬河衛者帖列衛兀失衛忽裏河衛失里綿河衛兀剌河衛愛河衛洽剌察衛卜忽禿河衛沒倫衛卜魯衛以哈阿哈衛速江平衛兀山衛弗力衛失郎山衛亦屯衛木河衛竹墩衛河木衛哈郎衛歲班衛失山衛考郎衛築屯衛黑里河衛右城衛弗河衛文東河衛阿古衛弗山衛兀答里衛納速河衛失列河衛朵兒玉衛兀魯河衛弗郎罕河衛赤卜罕山衛老河衛竹裏河衛吉答納河衛者不登衛也速脫衛阿木河衛顏亦衛已下添設山答衛塔哈衛弗魯納河衛行子衛兀勒阿城衛阿失衛吉真納河衛法衛薄羅衛塔麻所衛布兒哈衛亦思察河衛失剌衛卜忽禿衛撒里衛你實衛平河衛忽里吉山衛阿乞衛臺郎衛塞克衛拜苦衛所力衛巴里衛塔納衛木郎衛額克衛勒伏衛式木衛樹哈衛肥哈答衛蓋千衛

英禿衛乞忽衛阿林衛哈兒速衛巴答衛脫木衛忽把衛速哈兒衛馬失衛塔賽衛劄里衛者哈衛恨克衛哈失衛交枝衛葛衛艾答衛亦蠻衛哈察衛革出衛卜答衛蜀河衛禿里赤山衛賽因衛忙哈衛

○所二十四

兀者托溫千戶所哈魯門山千戶所兀者揆野木千戶所兀的罕千戶所兀者穩免赤千戶所得的河千戶所魚失千戶所五年千戶所兀者已河千戶所真河千戶所兀的千戶所屯河千戶所哈三千戶所兀者屯河千戶所古賁河千戶所五音千戶所鎖郎塔真河千戶所兀者揆野人千戶所敷答河千戶所兀禿河千戶所可里踢千戶所哈魯門千戶所兀討溫河千戶所兀者撒野人千戶所

△站七

別兒真站黑龍江地方莽亦帖站弗朵河站亦罕河衛忽把希站忽把希站弗答林站古代替站

○地面七

弗孫河地面木溫河地面埇坎河地面撒哈地面亦馬河咬東地面可木地面黑龍江地面

○寨一

黑龍江忽里平寨

西北諸部,在明初服屬,授以指揮等官,設衛給誥印。

○衛六

赤斤蒙古衛罕東衛安定衛阿端衛曲先衛哈密衛

西番即古吐番。洪武初,遣人招諭,又令各族舉舊有官職者至京,授以國師及都指揮、宣慰使、元帥、招討等官,俾因俗以治。自是番僧有封灌頂國師及贊善、闡化等王,大乘大寶法王者,俱給印誥,傳以為信。所設有都指揮使司、指揮司。

○都指揮使司二

烏思藏都指揮使司朵甘衛都指揮使司

○指揮使司一

隴答衛指揮使司

○宣尉使司三

朵甘宣慰使司董卜韓胡宣慰使司長河西魚通寧遠宣慰使司

○招討司六

朵甘思招討司朵甘隴答招討司朵甘丹招討司朵甘倉溏招討司朵甘川招討司磨兒勘招討司

○萬戶府四

沙兒可萬戶府乃竹萬戶府羅思端萬戶府別思麻萬戶府

○千戶所十七

朵甘思千戶所剌宗千戶所孛里加千戶所長河西千戶所多八三孫千戶所加八千戶所兆日千戶所納竹千戶所倫答千戶所果由千戶所沙裏可哈忽的千戶所孛里加思千戶所撒里土兒千戶所參卜郎千戶所剌錯牙千戶所泄里壩千戶所潤則魯孫千戶所

班軍者衛所之軍番上京師,總為三大營者也。初,永樂十三年詔邊將及河南、山東、山西、陜西各都司,中都留守司,江南、北諸衛官,簡所部卒赴北京,以俟臨閱。京操自此始。仁宗初,因英國公張輔等言,調直隸及近京軍番上操備,諭以畢農而來,先農務遣歸。既而輔言:「邊軍比悉放還,京軍少,請調山東、河南、中都、淮、揚諸衛校閱。」制曰「可」。又敕河南、山東、山西、大寧及中都將領,凡軍還取衣裝者,以三月畢務,七月至京,老弱者選代,官給之馬。歲春秋番上,共十六萬人:大寧七萬七百餘,中都、山東遞殺,河南最少,僅一萬四千有奇。定為例。後允成國公朱勇等請,罷鞏昌諸衛及階、文千戶所班軍,代以陜西內地卒。山東衛士沿海備倭,沿海衛士復內調,通州衛士漕淮安粟,安慶衛士赴京操,不便,皆更之。已,並放還陜西班軍。正統中,京操軍皆戍邊,乃遣御史於江北、山東、北直選卒,為京師備。景泰初,邊事棘,班軍悉留京,間歲乃放還取衣裝。于是于謙、石亨議三分之,留兩番操備。保定、河間、天津放五十日,河南、山東九十日,淮、揚、中都百日,紫荊、倒馬、白羊三關及保定諸城戍卒,屬山東、河南者,亦如之。逃者,官鐫秩三等,卒盡室謫邊衛。明年,謙又言:「班軍分十營團練,久不得休,請仍分兩番。」報可。

成化間,河南秋班軍二千餘不至,下御史趣之。海內燕安,外衛卒在京只供營繕諸役,勢家私占復半之。卒多畏苦,往往愆期,乃定遠限罪,輕者發居庸、密雲、山海關罰班六月。重者發邊衛罰班至年半。令雖具,然不能革也。

弘治中,兵部言占役之害,罰治如議。於是選衛兵八萬團操,內外各半。外衛四萬,兩番迭上。李東陽極言工作困軍,班軍逾期不至,大率坐此。帝然之。末年,歸大寧卒兩班萬人。正德中,宣府軍及京營互調,春秋番換如班軍例。迄世宗立乃已。

嘉靖初,尚書李承勛言:「永樂中調軍番上京師,後遂踵為故事,衛伍半空,而在京者徒供營造。不若省行糧之費,以募工作。」御史鮑象賢請分班軍為三,二入營操,一以赴役。通政司陳經復請半放之,收其糧募工。皆不行。久之,從翊國公郭勛言,寬河南因災不至班軍,而諭後犯者罪必如法。兵部因條議,軍士失期,治將領之罪,以多寡為差,重者至鐫秩戍邊。報可。其後邊警棘,乃併番上軍為一班,五月赴京,十一月放還,每歲秋防見兵十五六萬。仇鸞用事,抽邊卒入衛,凡選士六萬八千餘。又免大寧等衛軍京操,改防薊鎮,班軍遂耗減。豐城侯李熙核其數,僅四萬人,因請改征銀召募,而以見軍四萬歸營操練。嚴嵩議以「各衛兵雖有折幹之弊,然清核令下,猶凜凜畏罪。若奉旨征銀,恐借為口實,祖宗良法深意,一旦蕩然」。帝是之。折幹者,衛卒納銀,將弁以免其行,有事則召募以應。亡何,從平江伯陳圭奏,仍令中都、山東、河南軍分春秋兩班,別為一營,春以三月至,八月還,秋以九月至,來歲二月還,工作毋擅役。

隆慶初,大發治河,軍人憚久役,逃亡多。部議於見役軍中,簡銳者著伍,而以老弱供畚鍤。

萬歷二年,科臣言,班軍非為工作設。下兵部,止議以小工不得概派而已。時積弊已久,軍士苦役甚,多愆期不至。故事,失班脫逃者,罰工銀,追月糈。其後額外多征,軍益逃,中都尤甚。自嘉靖四十三年後,積逋工銀至五十餘萬兩。巡撫都御史張翀乞蠲額外工價,軍三犯者,不必罰工,竟調邊衛。而巡視京營給事中王道成則言:「凡軍一班不到,即係一年脫伍,盡扣月糧。本軍仍如例解京,罰補正班。三年脫班,仍調邊衛。」並報可。衛軍益大困。

後二十九年,帝以班軍多老弱雇倩,令嚴飭之。職方主事沈朝煥給班軍餉,皆傭諸丐,因言:「班軍本處有大糧,到京有行糧,又有鹽斤銀,所費十餘萬金,今皆虛冒。請解大糧貯庫,有警可召募,有工可雇役。」部議請先申飭,俟大工竣行之。是時專以班軍為役夫,番上之初意盡失矣。

又五年,內庭有小營繕,中官陳永壽請仍用班軍,可節省。給事中宋一韓爭之,謂:「班軍輸操即三大營軍,所係甚重。今邊鄙多事,萬一關吏不謹,而京師團練之軍多召募,游徼之役多役占,皇城宿衛多白徒,四衛扈從多廝役。即得三都司健卒三萬,猶不能無恐,況動以興作朘削,名存實亡,緩急何賴哉?」不聽。四十年,給事中麻僖請恤班操之苦。後六年,順天巡撫都御史劉曰梧言班軍無濟實用,因陳募兵十利。是時,法益弛,軍不營操,皆居京師為商販、工藝,以錢入班將。

啟、禎時,邊事洶洶,乃移班軍於邊。築垣、負米無休期,而糗糧缺,軍多死,班將往往逮革。特敕兵部右侍郎專督理,鑄印給之,然已無及。

○邊防海防江防民壯士兵鄉兵

元人北歸,屢謀興復。永樂遷都北平,三面近塞,正統以後,敵患日多。故終明之世,邊防甚重。東起鴨綠,西抵嘉峪,綿亙萬里,分地守御。初設遼東、宣府、大同、延綏四鎮,繼設寧夏、甘肅、薊州三鎮,而太原總兵治偏頭,三邊制府駐固原,亦稱二鎮,是為九邊。

初,洪武六年,命大將軍徐達等備山西、北平邊,諭令各上方略。從淮安侯華雲龍言,自永平、薊州、密雲迤西二千餘里,關隘百二十有九,皆置戍守。於紫荊關及蘆花嶺設千戶所守禦。又詔山西都衛於雁門關、太和嶺并武、朔諸山谷間,凡七十三隘,俱設戍兵。九年,敕燕山前、後等十一衛,分兵守古北口、居庸關、喜峰口、松亭關烽堠百九十六處,參用南北軍士。十五年,又於北平都司所轄關隘二百,以各衛卒守戍。詔諸王近塞者,每歲秋勒兵巡邊。十七年,命徐達籍上北平將校士卒。復使將核遼東、定遼等九衛官軍。是後,每遣諸公、侯校沿邊士馬,以籍上。二十年,置北平行都司於大寧。其地在喜峰口外,故遼西郡,遼之中京大定府也;西大同,東遼陽,南北平。馮勝之破納哈出,還師,城之,因置都司及營州五屯衛,而封皇子權為寧王,調各衛兵往守。先是,李文忠等取元上都,設開平衛及興和等千戶所,東西各四驛,東接大寧,西接獨石。二十五年,又築東勝城於河州東受降城之東,設十六衛,與大同相望。自遼以西,數千里聲勢聯絡。

建文元年,文帝起兵,襲陷大寧,以寧王權及諸軍歸。及即位,封寧王於江西。而改北平行都司為大寧都司,徙之保定。調營州五屯衛於順義、薊州、平谷、香河、三河,以大寧地畀兀良哈。自是,遼東與宣、大聲援阻絕,又以東勝孤遠難守,調左衛於永平,右衛於遵化,而墟其地。先是興和亦廢,開平徙於獨石,宣府遂稱重鎮。然帝於邊備甚謹。自宣府迤西迄山西,緣邊皆峻垣深濠,烽堠相接。隘口通車騎者百戶守之,通樵牧者甲士十人守之。武安侯鄭亨充總兵官,其敕書云:「各處煙墩,務增築高厚,上貯五月糧及柴薪藥弩,墩傍開井,井外圍墻與墩平,外望如一。」重門御暴之意,常凜凜也。

洪熙改元,朔州軍士白榮請還東勝、高山等十衛於故地。興州軍士範濟亦言:朔州、大同、開平、宣府大寧皆籓籬要地,其土可耕,宜遣將率兵,修城堡,廣屯種。皆不能用。

正統元年,給事中朱純請修塞垣。總兵官譚廣言:「自龍門至獨石及黑峪口五百五十餘里,工作甚難,不若益墩臺尞守。」乃增赤城等堡煙墩二十二。寧夏總兵官史昭言:「所轄屯堡,俱在河外,自河迤東至察罕腦兒,抵綏德州,沙漠曠遠,並無守備。請於花馬池築哨馬營。」大同總兵官方政繼以馬營請,欲就半嶺紅寺兒廢營修築。宣大巡撫都御史李儀以大同平衍,巡哨宜謹,請以副總兵主東路,參將主西路,而迤北則屬之總兵官都指揮。並如議行。後三年,詔塞紫荊關諸隘口,增守備軍。時瓦剌漸強,從成國公朱勇請也。既而也先入塞,英宗陷於土木。景帝即位,十餘年間,邊患日多,索來、毛里孩、阿羅出之屬,相繼入犯,無寧歲。

成化元年,延綏總兵官張傑言:「延慶等境廣袤千里,所轄二十五營堡,每處僅一二百人,難以應敵,宜選精銳九千為六哨,分屯府谷、神木二縣,龍州、榆林二城,高家、安邊二堡,庶緩急有備。」又請分布鄜、慶防秋軍二千餘人於沿邊要害。從之。七年,延綏巡撫都御史餘子俊大築邊城。先是,東勝設衛守在河外,榆林治綏德。後東勝內遷,失險,捐米脂、魚河地幾三百里。正統間,鎮守都督王禎始築榆林城,建緣邊營堡二十四,歲調延安、綏德、慶陽三衛軍分戍。天順中,阿羅出入河套駐牧,每引諸部內犯。至是,子俊乃徙治榆林。由黃甫川西至定邊營千二百餘里,墩堡相望,橫截套口,內復塹山堙谷,曰夾道,東抵偏頭,西終寧、固,風土勁悍,將勇士力,北人呼為橐駝城。十二年,兵部侍郎滕昭、英國公張懋條上邊備,言:「居庸關、黃花鎮、喜峰口、古北口、燕河營有團營馬步軍萬五千人戍守,請益軍五千,分駐永平、密雲以策應遼東。涼州鎮番、莊浪、賀蘭山迤西,從雪山過河,南通靖虜,直至臨、鞏,俱敵入犯之路,請調陜西官軍,益以甘、涼、臨、鞏、秦、平、河、洮兵,戍安定、會寧,遇警截擊;以涼州銳士五千,扼要屯駐,彼此策應。」詔可。二十一年,敕各邊軍士,每歲九月至明年三月,俱常操練,仍以操過軍馬及風雪免日奏報。邊備頗修飭。

弘治十四年,設固原鎮。先是,固原為內地,所備惟靖虜。及火篩入據河套,遂為敵沖。乃改平涼之開成縣為固原州,隸以四衛,設總制府,總陜西三邊軍務。是時陜邊惟甘肅稍安,而哈密屢為土魯番所擾,乃敕修嘉峪關。

正德元年春,總制三邊都御史楊一清請復守東勝:「因河為固,東接大同,西屬寧夏,使河套千里沃壤,歸我耕牧,則陜右猶可息肩」。因上修築定邊營等六事。帝可其奏。旋以忤中官劉瑾罷,所築塞垣僅四十餘里而已。武宗好武,邊將江彬等得幸,遼東、宣府、大同、延綏四鎮軍多內調,又以京軍六千與宣府軍六千春秋番換。十三年,頒定宣、大、延綏三鎮應援節度:敵不渡河,則延綏聽調於宣、大;渡河,則宣、大聽調於延綏。從兵部尚書王瓊議也。

初,大寧之棄,以其地畀朵顏、福餘、泰寧三衛,蓋兀良哈歸附者也。未幾,遂不靖。宣宗嘗因田獵,親率師敗之,自是畏服。故喜峰、密云止設都指揮鎮守。土木之變,頗傳三衛助逆,後因添設太監參將等官。至是,朵顏獨盛,情叵測。

嘉靖初,御史丘養浩請復小河等關於外地,以扼其要。又請多鑄火器,給沿邊州縣,募商糶粟,實各邊衛所。詔皆行之。初,太祖時,以邊軍屯田不足,召商輸邊粟而與之鹽。富商大賈悉自出財力,募民墾田塞下,故邊儲不匱。弘治時,戶部尚書葉淇始變法,令商納銀太倉,分給各邊。商皆撤業歸,邊地荒蕪,米粟踴貴,邊軍遂日困。十一年,御史徐汝圭條上邊防兵食,謂「延綏宜漕石州、保德之粟,自黃河而上,楚粟由鄖陽,汴粟由陜、洛,沔粟由漢中,以達陜右。宣、大產二麥,宜多方收糶。紫荊、倒馬、白羊等關,宜招商賃車運」。又請「以宣府游兵駐右衛懷來,以援大同。選補游兵於順聖西城為臨期應援,永寧等處游兵衛宣府,備調遣。直隸八府召募勇敢團練,赴邊關遠近警急。榆林、山、陜游兵,於本處策應」。報可,亦未能行也。

十八年,移三邊制府鎮花馬池。是時,俺答諸部強橫,屢深入大同、太原之境,晉陽南北,煙火蕭然。巡撫都御史陳講請「以兵六千戍老營堡東界之長峪,以山西兵守大同。三關形勢,寧武為中路,莫要於神池,偏頭為西路,莫要於老營堡,皆宜改設參將。雁門為東路,莫要於北樓諸口,宜增設把總、指揮。而移神池守備於利民堡,老營堡游擊於八角所,各增軍設備」。帝悉許之。規畫雖密,然兵將率怯弱,其健者僅能自守而已。

二十二年,詔宣府兵乘塞。舊制,總兵夏秋間分駐邊堡,謂之暗伏。至是,有司建議,入秋悉令赴邊,分地拒守,至九月中罷歸,犒以帑金。久之,以勞費罷。二十四年,巡按山西御史陳豪言:「敵三犯山西,傷殘百萬,費餉銀六十億,曾無尺寸功。請定計決戰,盡復套地。」明年,敵犯延安,總督三邊侍郎曾銑力主復套,條上十八事。帝嘉獎之。大學士嚴嵩窺帝意憚兵,且欲殺舊閣臣夏言,因劾銑,并言誅死,自是無敢言邊事者。

二十九年,俺答攻古北口,從間道黃榆溝入,直薄東直門,諸將不敢戰。敵退,大將軍仇鸞力主貢市之議。明年,開馬市於大同,然寇掠如故。又明年,馬市罷。

先是翁萬達之總督宣、大也,籌邊事甚悉。其言曰:「山西保德州河岸,東盡老營堡,凡二百五十四里。西路丫角山迤北而來,歷中北路,抵東路之東陽河鎮口臺,凡六百四十七里。宣府西路,西陽河迤東,歷中北路,抵東路之永寧四海冶,凡一千二十三里。皆逼臨巨寇,險在外者,所謂極邊也。老營堡轉南而東,歷寧武、雁門、北樓至平刑關盡境,約八百里。又轉南而東,為保定界,歷龍泉、倒馬、紫荊、吳王口、插箭嶺、浮圖峪至沿河口,約一千七十餘里。又東北為順天界,歷高崖、白羊,抵居庸關,約一百八十餘里。皆峻嶺層岡,險在內者,所謂次邊也。敵犯山西必自大同,入紫荊必自宣府,未有不經外邊能入內邊者。」乃請修築宣、大邊牆千餘里,烽堠三百六十三所。後以通市故,不復防,遂半為敵毀。至是,兵部請敕邊將修補。科臣又言,垣上宜築高臺,建廬以棲火器。從之。時俺答益強,朵顏三衛為之向道,遼、薊、宣、大連歲被兵。三十四年,總督軍務兵部尚書楊博,既解大同右衛圍,因築牛心諸堡,修烽堠二千八百有奇。宣、大間稍寧息,而薊鎮之患不已。

薊之稱鎮,自二十七年始。時鎮兵未練,因詔各邊入衛兵往戍。既而兵部言:「大同之三邊,陜西之固原,宣府之長安嶺,延綏之夾牆,皆據重險,惟薊獨無。渤海所南,山陵東,有蘇家口,至寨籬村七十里,地形平漫,宜築牆建臺,設兵守,與京軍相夾制。」報可。時兵力孱弱,有警徵召四集,而議者惟以據險為事,無敢言戰者。其後薊鎮入衛兵,俱聽宣、大督、撫調遣,防禦益疏。朵顏遂乘虛歲入。三十七年,諸鎮建議,各練本鎮戍卒,可省徵發費十之六。然戍卒選懦不任戰,歲練亦費萬餘,而臨事徵發如故。隆慶間,總兵官戚繼光總理薊、遼,任練兵事,因請調浙兵三千人以倡勇敢。及至,待命於郊,自朝至日中,天雨,軍士跬步不移,邊將大駭。自是薊兵以精整稱。

俺答已通貢,封順義王,其子孫襲封者累世。迨萬曆之季,西部遂不競,而土蠻部落虎燉兔、炒花、宰賽、爰兔輩,東西煽動,將士疲於奔命,未嘗得安枕也。

初,太祖沿邊設衛,惟土著兵及有罪謫戍者。遇有警,調他衛軍往戍,謂之客兵。永樂間,始命內地軍番戍,謂之邊班。其後占役逃亡之數多,乃有召募,有改撥,有修守民兵、土兵,而邊防日益壞。洪武時,宣府屯守官軍殆十萬。正統、景泰間,已不及額。弘治、正德以後,官軍實有者僅六萬六千九百有奇,而召募與士兵居其半。他鎮率視此。

正統初,山西、河南班軍守偏頭、大同、宣府塞,不得代。巡撫于謙言:「每歲九月至二月,水冷草枯,敵騎出沒,乘障卒宜多。若三月至八月,邊守自足。乞將兩班軍,每歲一班,如期放遣。」甘肅總兵官蔣貴又言:「沿邊墩臺,守了軍更番有例,惟坐事謫發者不許,困苦甚。乞如例踐更。」並從之。五年,山西總兵官李謙請偏頭關守備軍如大同例,半歲更番。部議,每番皆十月,而戍卒仍率以歲為期,有久而後遣者。弘治中,三邊總制秦紘言:「備御延綏官軍,自十二月赴邊,既周一歲,至次年二月始得代。在軍日多,請歲一更,上下俱在三月初。」邊軍便之。

嘉靖四十三年,巡撫延綏胡志夔請免戍軍三年,每軍征銀五兩四錢,為募兵用。至萬曆初,大同督、撫方逢時等請修築費。詔以河南應戍班軍,自四年至六年概免,盡扣班價發給,謂之折班,班軍遂耗。久之,所征亦不得。寧山、南陽、潁上三衛積逋延綏鎮折班銀至五萬餘兩。是後諸邊財力俱盡,敝極矣。

初,邊政嚴明,官軍皆有定職。總兵官總鎮軍為正兵,副總兵分領三千為奇兵,游擊分領三千往來防禦為游兵,參將分守各路東西策應為援兵。營堡墩臺分極衝、次沖,為設軍多寡。平時走陣、哨探、守尞、焚荒諸事,無敢惰。稍違制,輒按軍法。而其後皆廢壞云。

沿海之地,自樂會接安南界,五千里抵閩,又二千里抵浙,又二千里抵南直隸,又千八百里抵山東,又千二百里逾寶坻、盧龍抵遼東,又千三百餘里抵鴨綠江。島寇倭夷,在在出沒,故海防亦重。

吳元年,用浙江行省平章李文忠言,嘉興、海鹽、海寧皆設兵戍守。洪武四年十二月,命靖海侯吳禎籍方國珍所部溫、臺、慶元三府軍士及蘭秀山無田糧之民,凡十一萬餘人,隸各衛為軍。且禁沿海民私出海。時國珍及張士誠餘眾多竄島嶼間,勾倭為寇。五年,命浙江、福建造海舟防倭。明年,從德慶侯廖永忠言,命廣洋、江陰、橫海、水軍四衛增置多櫓快船,無事則巡徼,遇寇以大船薄戰,快船逐之。詔禎充總兵官,領四衛兵,京衛及沿海諸衛軍悉聽節制。每春以舟師出海,分路防倭,迄秋乃還。十七年,命信國公湯和巡視海上,築山東、江南北、浙東西沿海諸城。後三年,命江夏侯周德興抽福建福、興、漳、泉四府三丁之一,為沿海戍兵,得萬五千人。移置衛所於要害處,築城十六。復置定海、盤石、金鄉、海門四衛於浙,金山衛於松江之小官場,及青村、南匯嘴城二千戶所,又置臨山衛於紹興,及三山、瀝海等千戶所,而寧波、溫、臺並海地,先已置八千戶所,曰平陽、三江、龍山、霩戺、大松、錢倉、新河、松門,皆屯兵設守。二十一年,又命和行視閩粵,築城增兵。置福建沿海指揮使司五,曰福寧、鎮東、平海、永寧、鎮海。領千戶所十二,曰大金、定海、梅花、萬安、莆禧、崇武、福全、金門、高浦、六鰲、銅山、玄鐘。二十三年,從衛卒陳仁言,造蘇州太倉衛海舟。旋令濱海衛所,每百戶及巡檢司皆置船二,巡海上盜賊。後從山東都司周彥言,建五總寨於寧海衛,與萊州衛八總寨,共轄小寨四十八。已,復命重臣勛戚魏國公徐輝祖等分巡沿海。帝素厭日本詭譎,絕其貢使,故終洪武、建文世不為患。

永樂六年,命豐城侯李彬等緣海捕倭,復招島人、醿戶、賈豎、漁丁為兵,防備益嚴。十七年,倭寇遼東,總兵官劉江殲之於望海堝。自是倭大懼,百餘年間,海上無大侵犯。朝廷閱數歲一令大臣巡警而已。

至嘉靖中,倭患漸起,始設巡撫浙江兼管福建海道提督軍務都御史。已,改巡撫為巡視。未幾,倭寇益肆。乃增設金山參將,分守蘇、松海防,尋改為副總兵,調募江南、北徐、邳官民兵充戰守,而杭、嘉、湖亦增參將及兵備道。三十三年,調撥山東民兵及青州水陸槍手千人赴淮、揚,聽總督南直軍務都御史張經調用。時倭縱掠杭、嘉、蘇、松,踞柘林城為窟穴,大江南北皆被擾。監司任環敗之,經亦有王家涇之捷,乃遁出海,復犯蘇州。於是南京御史屠仲律言五事。其守海口云:「守平陽港、黃花澳,據海門之險,使不得犯溫、臺。守寧海關、湖頭灣,遏三江之口,使不得窺寧、紹。守鱉子門、乍浦峽,使不得近杭、嘉。守吳淞、劉家河、七丫港,使不得掩蘇、松。且宜修飭海舟,大小相比,或百或五十聯為一宗,募慣習水工領之,而充以原額水軍,於諸海口量緩急置防。」部是其議。未幾,兵部亦言:「浙、直、通、泰間最利水戰,往時多用沙船破賊,請厚賞招徠之。防禦之法,守海島為上,宜以太倉、崇明、嘉定、上海沙船及福倉、東莞等船守普陀、大衢。陳錢山乃浙、直分路之始,狼、福二山約束首尾,交接江洋,亦要害地,宜督水師固守。」報可。已,復令直隸吳淞江、劉家河、福山港、鎮江、圌山五總添設游兵,聽金山副總兵調度。

時胡宗憲為總督,誅海賊徐海、汪直。直部三千人,復勾倭入寇,閩、廣益騷。三十七年,都御史王詢請「分福建之福、興為一路,領以參將,駐福寧,水防自流江、烽火門、俞山、小埕至南日山,漳、泉為一路,領以參將,駐詔安,水防自南日山至浯嶼、銅山、玄鐘、走馬溪、安邊館。水陸兵皆聽節制。福建省城介在南北,去海僅五十里,宜更設參將,選募精稅部領哨船,與主客兵相應援」。部覆從之。廣東惠、潮亦增設參將,駐揭陽。福建巡撫都御史游震得言:「浙江溫、處與福寧接壤,倭所出沒,宜進戚繼光為副總兵,守之。而增設福寧守備,隸繼光。漳州之月港亦增設守備,隸總兵官俞大猷。延、建、邵為八閩上游,宜募兵以備緩急。」皆允行。既而宗憲被逮,罷總督官,以浙江巡撫趙炳然兼任軍事。炳然因請令定海總兵屬浙江,金山總兵屬南直,俱兼理水陸軍務,互相策應。其後,莆田倭寇平,乃復五水寨舊制。

五寨者,福寧之烽火門,福州之小埕澳,興化之南日山,泉州之浯嶼,漳州之西門澳,亦曰銅山。景泰三年,鎮守尚書薛希璉奏建者也,後廢。至是巡撫譚綸疏言:「五寨守扼外洋,法甚周悉,宜復舊。以烽火門、南日、浯嶼三宗為正兵,銅山、小埕二宗為游兵。寨設把總,分汛地,明斥堠,嚴會哨。改三路參將為守備。分新募浙兵為二班,各九千人,春秋番上。各縣民壯皆補用精悍,每府領以武職一人,兵備使者以時閱視。」帝皆是之。狼山故設副總兵,至是改為鎮守總兵官,兼轄大江南北。迨隆慶初,倭漸不為患,而諸小寇往往有之。

萬曆三年,設廣東南澳總兵官,以其據漳、泉要害也。久之,倭寇朝鮮,朝廷大發兵往援,先後六年。於是設巡撫官於天津,防畿甸。後十餘年,從南直巡按御史顏思忠言,分淮安大營兵六百守廖角嘴。從福建巡撫丁繼嗣言,設兵自浙入閩之三江及劉澳,而易海澄團練營土著軍以浙兵。

天啟中,築城於澎湖,設游擊一,把總二,統兵三千,築砲臺以守。先是,萬曆中,許孚遠撫閩,奏築福州海壇山,因及澎湖諸嶼,且言浙東沿海陳錢、金塘、玉環、南麂諸山俱宜經理,遂設南麂副總兵,而澎湖不暇及。其地遙峙海中,逶迤如修蛇,多岐港零嶼,其中空間可藏巨艘。初為紅毛所據,至是因巡撫南居益言,乃奪而守之。

自世宗世倭患以來,沿海大都會,各設總督、巡撫、兵備副使及總兵官、參將、游擊等員,而諸所防禦,於廣東則分東、中、西三路,設三參將;於福建則有五水寨;於浙則有六總,一金鄉、盤石二衛,一松門、海門二衛,一昌國衛及錢倉、爵溪等所,一定海衛及霩戺、大嵩等所,一觀海、臨山二衛,一海寧衛,分統以四參將;於南直隸則乍浦以東,金山衛設參將,黃浦以北,吳淞江口設總兵;於淮、揚則總兵駐通州,游擊駐廟灣,又於揚州設陸兵游擊,待調遣;於山東則登、萊、青三府設巡察海道之副使,管理民兵之參將,總督沿海兵馬備倭之都指揮,於薊、遼則大沽海口宿重兵,領以副總兵,而以密雲、永平兩游擊為應援。山海關外,則廣寧中、前等五所兵守各汛,以寧前參將為應援,而金、復、海、蓋諸軍皆任防海。三岔以東,九聯城外創鎮江城,設游擊,統兵千七百,哨海上,北與寬奠參將陸營相接,共計凡七鎮,而守備、把總、分守、巡徼會哨者不下數百員。以三、四、五月為大汛,九、十月為小汛。蓋遭倭甚毒,故設防亦最密雲。

日本地與閩相值,而浙之招寶關其貢道在焉,故浙、閩為最沖。南寇則廣東,北寇則由江犯留都、淮、揚,故防海外,防江為重。洪武初,於都城南新江口置水兵八千。已,稍置萬二千,造舟四百艘。又設陸兵於北岸浦子口,相掎角。所轄沿江諸郡。上自九江、廣濟、黃梅,下抵蘇、松、通、泰,中包安慶、池、和、太平,凡盜賊及販私鹽者,悉令巡捕,兼以防倭。永樂時,特命勛臣為帥視江操,其後兼用都御史。成化四年,從錦衣衛僉事馮瑤言,令江兵依地設防,於瓜、儀、太平置將領鎮守。後六年,守備定西侯蔣琬奏調建陽、鎮江諸衛軍補江兵缺伍。十三年,命擇武大臣一人職江操,毋攝營務。又五年,從南京都御史白昂言,敕沿江守備官互相應援,并給關防。著為令。弘治中,命新江口兩班軍如京營例,首班歇,即以次班操。嘉靖八年,江陰賊侯仲金等作亂,給事中夏言請設鎮守江、淮總兵官。已而寇平,總兵罷不設。十九年,沙賊黃艮等復起。帝詰兵部以罷總兵之故,乃復設,給旗牌符敕,提督沿江上下。後復裁罷。三十二年,倭患熾,復設副總兵於金山衛,轄沿海至鎮江,與狼山副總兵水陸相應。時江北俱被倭,於是量調九江、安慶官軍守京口、圌山等地。久之,給事中範宗吳言:「故事,操江都御史防江,應、鳳二巡撫防海。後因倭警,遂以鎮江而下,通常、狼、福諸處隸之操江,以故二撫臣得諉其責。操江又以向非本屬兵,難遙制,亦漠然視之,非委任責成意。宜以圌山、三江會口為操、撫分界。」報可。其後增上下兩江巡視御史,得舉劾有司將領,而以南京僉都御史兼理操江,不另設。

先是,增募水兵六千。隆慶初,以都御史吳時來請,留四之一,餘悉罷遣,并裁中軍把總等官。已,復令分汛設守,而責以上下南北互相策應。又從都御史宋儀望言,諸軍皆分駐江上,不得居城市。萬曆二十年,以倭警,言者請復設京口總兵。南京兵部尚書衷貞吉等謂既有吳淞總兵,不宜兩設。乃設兵備使者,每春汛,調備倭都督,統衛所水、陸軍赴鎮江。後七年,操江耿定力奏:「長江千餘里,上江列營五,兵備臣三;下江列營五,兵備臣二。宜委以簡閱訓練,即以精否為兵備殿最。」部議以為然。故事,南北總哨官五日一會哨於適中地,將領官亦月兩至江上會哨。其後多不行。崇禎中,復以勛臣任操江,偷惰成習,會哨巡徼皆虛名,非有實矣。

衛所之外,郡縣有民壯,邊郡有土兵。

太祖定江東,循元制,立管領民兵萬戶府。後從山西行都司言,聽邊民自備軍械,團結防邊。閩、浙苦倭,指揮方謙請籍民丁多者為軍。尋以為患鄉里,詔閩、浙互徙。時已用民兵,然非召募也。正統二年,始募所在軍餘、民壯願自效者,陜西得四千二百人。人給布二匹,月糧四斗。景泰初,遣使分募直隸、山東、山西、河南民壯,撥山西義勇守大同,而紫荊、倒馬二關,亦用民兵防守,事平免歸。

成化二年,以邊警,復二關民兵。敕御史往延安、慶陽選精壯編伍,得五千餘人,號曰土兵。以延綏巡撫盧祥言邊民驍果,可練為兵,使護田里妻子,故有是命。

弘治七年,立僉民壯法。州、縣七八百里以上,里僉二人,五百里三,三百里四,百里以上五。有司訓練,遇警調發,給以行糧,而禁役占放買之弊。富民不願,則上直於官,官自為募。或稱機兵,在巡檢司者稱弓兵。後以越境防冬非計,大同巡撫劉宇請免其班操,徵銀糧輸大同,而以威遠屯丁、舍、餘補役。給事中熊偉亦請編應募民於附近衛所。並從之。十四年,以西北諸邊所募士兵,多不足五千,遣使齎銀二十萬及太僕寺馬價銀四萬往募。指揮千百戶以募兵多寡為差,得遷級,失官者得復職,即令統所募兵。既而兵部議覆侍郎李孟暘請實軍伍疏,謂:「天下衛所官軍原額二百七十餘萬,歲久逃故,嘗選民壯三十餘萬,又核衛所舍人、餘丁八十八萬,西北諸邊召募士兵無慮數萬。請如孟暘奏,察有司不操練民壯、私役雜差者,如役占軍人罪。」報可。正德中,流賊擾山東,巡撫張鳳選民兵,令自買馬團操,民不勝其擾。兵部侍郎楊潭以為言。都御史寧杲所募多無賴子,為御史張璇所劾。

嘉靖二十二年增州縣民壯額,大者千人,次六七百,小者五百。二十九年,京師新被寇,議募民兵,以二萬為率。歲四月終,赴近京防禦。後五年,兵部尚書楊博請汰老弱,存精銳,在外者發各道為民兵,在京者隸之巡捕參將,逃者不補。帝以影占數多,耗糧無用,遣官核宜罷宜還者以聞。隆慶中,張居正、陳以勤復請籍畿甸民兵,謂:「直隸八府人多健悍,總按戶籍,除單丁老弱者,父子三人籍一子,兄弟三人籍一弟,州與大縣可得千六百人,小縣可得千人。中分之為正兵、奇兵,登名尺籍,隸撫臣操練,歲無過三月,月無過三次,練畢即令歸農,復其身。歲操外,不得別遣。」命所司議行。然自嘉靖後,山東、河南民兵戍薊門者,率征銀以充召募。至萬曆初,山東征銀至五萬六千兩,貧民大困。

治河之役,給事中張貞觀請益募士兵,捍淮、揚、徐、邳。畿南盜起,給事中耿隨龍請復民壯舊制,專捕賊盜。播州之亂,工部侍郎趙可懷請練土著,兵部因言:「天下之無兵者,不獨蜀也。各省官軍、民壯,皆宜罷老稚,易以健卒。軍操屬印官、操官,民操屬正官、捕官,郡守、監司不得牽制。立營分伍,以憑調發。」先後皆議行。

末年,募兵措餉益急。南京職方郎中鄒維璉陳調募之害。山西參政徐九翰尤極言民兵不可調。崇禎時,中原盜急,兵部尚書楊嗣昌議令責州縣訓練土著為兵。工部侍郎張慎言言其不便者數事,而御史米壽圖又言其害有十,謂不若簡練民兵,增民壯快手,備御地方為便。後嗣昌死,練兵亦不行。

鄉兵者,隨其風土所長應募,調佐軍旅緩急。其隸軍籍者曰浙兵,義烏為最,處次之,臺、寧又次之,善狼筅,間以叉槊。戚繼光製鴛鴦陣以破倭,及守薊門,最有名。曰川兵、曰遼兵,崇禎時,多調之剿流賊。其不隸軍籍者,所在多有。河南嵩縣曰毛葫蘆,習短兵,長於走山。而嵩及盧氏、靈寶、永寧並多礦兵,曰角腦,又曰打手。山東有長竿手。徐州有箭手。井陘有螞螂手,善運石,遠可及百步。閩漳、泉習鏢牌,水戰為最。泉州永春人善技擊。正統間,郭榮六者,破沙尤賊有功。商灶鹽丁以私販為業,多勁果。成化初,河東鹽徒千百輩,自備火砲、強弩、車仗,雜官軍逐寇。而松江曹涇鹽徒,嘉靖中逐倭至島上,焚其舟。後倭見民家有鹺囊,輒搖手相戒。粵東雜蠻蜑,習長牌、斫刀,而新會、東莞之產強半。延綏、固原多邊外土著,善騎射,英宗命簡練以備秋防。大滕峽之役,韓雍用之,以摧瑤、僮之用牌刀者。莊浪魯家軍,舊隸隨駕中,洪熙初,令土指揮領之。萬曆間,部臣稱其驍健,為敵所畏,宜鼓舞以儲邊用。西寧馬戶八百,嘗自備騎械赴敵,後以款貢裁之。萬歷十九年,經略鄭雒請復其故。又僧兵,有少林、伏牛、五臺。倭亂,少林僧應募者四十餘人,戰亦多勝。西南邊服有各土司兵。湖南永順、保靖二宣慰所部,廣西東蘭、那地、南丹、歸順諸狼兵,四川酉陽、石砫秦氏、冉氏諸司,宣力最多。末年,邊事急,有司專以調三省土司為長策,其利害亦恆相半云。

清理軍伍訓練賞功火器車船馬政

明初,垛集令行,民出一丁為軍,衛所無缺伍,且有羨丁。未幾,大都督府言,起吳元年十月,至洪武三年十一月,軍士逃亡者四萬七千九百餘。於是下追捕之令,立法懲戒。小旗逃所隸三人,降為軍。上至總旗、百戶、千戶,皆視逃軍多寡,奪俸降革。其從征在外者,罰尤嚴。十六年,命五軍府檄外衛所,速逮缺伍士卒,給事中潘庸等分行清理之。明年,從兵部尚書俞綸言,京衛軍戶絕者,毋冒取同姓及同姓之親,令有司核實發補,府衛毋特遣人。二十一年,詔衛所核實軍伍,有匿己子以養子代者,不許。其秋,令衛所著軍士姓名、鄉貫為籍,具載丁口以便取補。又置軍籍勘合,分給內外,軍士遇點閱以為驗。

成祖即位,遣給事等官分閱天下軍,重定垛集軍更代法。初,三丁已上,垛正軍一,別有貼戶,正軍死,貼戶丁補。至是,令正軍、貼戶更代,貼戶單丁者免;當軍家蠲其一丁徭。

洪熙元年,興州左屯衛軍範濟極言勾軍之擾。富峪衛百戶錢興奏言:「祖本涿鹿衛軍,死,父繼,以功授百戶。臣已襲父職,而本衛猶以臣祖為逃軍,屢行勾取。」帝謂尚書張本曰:「軍伍不清,弊多類此。」已而宣宗立,軍弊益滋,黠者往往匿其籍,或誣攘良民充伍。帝諭兵部曰:「朝廷於軍民,如舟車任載,不可偏重。有司宜審實毋混。」乃分遣吏部侍郎黃宗載等清理天下軍衛。三年敕給事、御史清軍,定十一條例,榜示天下。明年復增為二十二條。五年,從尚書張本請,令天下官吏、軍旗公勘自洪、永來勾軍之無蹤者,豁免之。六年,令勾軍有親老疾獨子者,編之近地,餘丁赴工逋亡者例發口外,改為罰工一年,示優恤焉。八年,免蘇州衛抑配軍百五十九人,已食糧止令終其身者,千二百三十九人。先是,蘇、常軍戶絕者,株累族黨,動以千計,知府況鐘言於朝,又常州民訴受抑為軍者七百有奇,故特敕巡撫侍郎周忱清理。

正統初,令勾軍家丁盡者,除籍;逃軍死亡及事故者,或家本軍籍,而偶同姓名,里胥挾讎妄報冒解,或已解而赴部聲冤者,皆與豁免。定例,補伍皆發極邊,而南北人互易。大學士楊士奇謂風土異宜,瀕於夭折,請從所宜發戍。署兵部侍郎鄺埜以為紊祖制,寢之。成化二年,山西巡撫李侃復請補近衛,始議行。十一年,命御史十一人分道清軍,以十分為率,及三分者最,不及者殿。時以罪謫者逃故,亦勾其家丁。御史江昂謂非「罰弗及嗣」之義,乃禁之。

嘉靖初,捕亡令愈苛,有株累數十家,勾攝經數十年者,丁口已盡,猶移覆紛紜不已。兵部尚書胡世寧請「屢經清報者免勾。又避役之人必緩急難倚,急改編原籍。衛所有缺伍,則另選舍餘及犯罪者充補。犯重發邊衛者,責賣家產,闔房遷發,使絕顧念。庶衛卒皆土著,而逃亡益鮮」。帝是其言。其後,用主事王學益議,制勾單,立法詳善。久之,停差清軍御史,寬管解逃軍及軍赴衛違限之科。清軍官日玩愒,文卷磨滅,議者復請申飭。

萬曆三年,給事中徐貞明言:「勾軍東南,資裝出於戶丁,解送出於里遞,每軍不下百金。大困東南之民,究無補於軍政。宜視班匠例,免其解補,而重徵班銀,以資召募,使東南永無勾補之擾,而西北之行伍亦充。」鄖陽巡撫王世貞因言有四便:應勾之戶,樂於就近,不圖避匿,便一;各安水土,不至困絕,便二;近則不逃,逃亦易追,便三;解戶不至破家,便四。而兵部卒格貞明議,不行。後十三年,南京兵部尚書郭應聘復請各就近地,南北改編。又言「應勾之軍,南直隸至六萬六千餘,株連至二三十萬人,請自天順以前竟與釋免」。報可,遠近皆悅。然改編令下,求改者相繼。明年,兵部言「什伍漸耗,邊鎮軍人且希圖脫伍」。有旨復舊,而應聘之議復不行。

凡軍衛掌於職方,而勾清則武庫主之。有所勾攝,自衛所開報,先核鄉貫居止,內府給批,下有司提本軍,謂之跟捕;提家丁,謂之勾捕。間有恩恤開伍者。洪武二十三年,令應補軍役生員,遣歸卒業。宣德四年,上虞人李志道充楚雄衛軍,死,有孫宗皋宜繼。時已中鄉試,尚書張本言於帝,得免。如此者絕少。戶有軍籍,必仕至兵部尚書始得除。軍士應起解者,皆僉妻;有津給軍裝、解軍行糧、軍丁口糧之費。其冊單編造皆有恆式。初定戶口、收軍、勾清三冊。嘉靖三十一年,又編四冊,曰軍貫,曰兜底,曰類衛、類姓。其勾軍另給軍單。蓋終明世,於軍籍最嚴。然弊政漸叢,而擾民日甚。

明太祖起布衣,策群力,取天下。即位後,屢命元勛宿將分道練兵,而其制未定。洪武六年,命中書省、大都督府、御史臺、六部議教練軍士律:「騎卒必善馳射槍刀,步兵必善弓弩槍。射以十二矢之半,遠可到,近可中為程。遠可到,將弁百六十步、軍士百二十步;近可中,五十步。彀弩以十二矢之五,遠可到,蹶張八十步,劃車一百五十步;近可中,蹶張四十步,劃車六十步。槍必進退熟習。在京衛所,以五千人為率,取五之一,指揮以下官領赴御前驗試,餘以次番試。在外都司衛所,每衛五千人,取五之一,千戶以下官領赴京驗試。餘以次番試。軍士步騎皆善,將領各以其能受賞,否則罰。軍士給錢六百為道里費。將領自指揮使以下,所統軍士三分至六分不中者,次第奪俸;七分以上,次第降官至為軍止。都指揮軍士四分以上不中,奪俸一年;六分以上罷職。」後十六年,令天下衛所善射者十選一,於農隙分番赴京較閱,以優劣為千百戶賞罰,邊軍本衛較射。二十年,命衛士習射於午門丹墀。明年復令:「天下衛所馬步軍士,各分十班,將弁以蔭敘久次升者統之,冬月至京閱試。指揮、千百戶,年深慣戰及屯田者免。仍先下操練法,俾遵行。不如法及不嫻習者,罰。」明年,詔五軍府:「比試軍士分三等賞鈔,又各給鈔三錠為路費,不中者亦給之。明年再試不如式,軍移戍雲南,官謫從征,總小旗降為軍。武臣子弟襲職,試騎步射不中程,令還衛署事,與半俸,二年後仍試如故者,亦降為軍。」

文皇即位,五駕北征,六師嘗自較閱。又嘗敕秦、晉、周、肅諸王,各選護衛軍五千,命官督赴真定操練,陜西、甘肅、寧夏、大同、遼東諸守將,及中都留守、河南等都司,徐、宿等衛,遣將統馬步軍分駐真定、德州操練,侯赴京閱視。

景泰初,立十團營。給事中鄧林進《軒轅圖》,即古八陣法也,因用以教軍。成化間,增團營為十二,命月二次會操,起仲春十五日,止仲夏十五日,秋、冬亦如之。弘治九年,兵部尚書馬文升申明洪、永操法,五日內,二日走陣下營,三日演武。武宗好武勇,每令提督坐營官操練,又自執金鼓演四鎮卒。然大要以恣馳騁、供嬉戲,非有實也。

嘉靖六年定,下營布陣,止用三疊陣及四門方營。又令每營選槍刀箭牌銃手各一二人為教師,轉相教習。及更營制,分兵三十枝,設將三十員,各統三千人訓練,擇精銳者名選鋒,厚其校藝之賞。總督大臣一月會操者四,餘日營將分練。協理大臣及巡視給事、御史隨意入一營,校閱賞罰,因以擇選鋒。帝又置內營於內教場,練諸內使。

隆慶初,命各營將領以教練軍士分數多寡為黜陟。全營教練者加都督僉事,以次減;全不教練者降祖職一級,革任回衛。三年內教練有成,操協大臣獎諭恩錄;無功績者議罰。規制雖立,然將卒率媮惰,操演徒為具文。

先是,浙江參將戚繼光以善教士聞,嘗調士兵,制鴛鴦陣破倭。至是已官總兵。穆宗從給事中吳時來請,命繼光練兵薊門。薊兵精整者數十年。繼光嘗著《練兵實紀》以訓士。一曰練伍,首騎,次步,次車,次輜重;先選伍,次較藝,總之以合營。二曰練膽氣,使明作止進退及上下統屬、相友相助之義。三曰練耳目,使明號令。四曰練手足,使熟技藝。五曰練營陣,詳布陣起行、結營及交鋒之正變。終之以練將。後多遵用之。

賞功之制,太祖時,大賞平定中原、征南諸將及雲南、越州之功。賞格雖具,然不豫為令。惟二十九年命沿海衛所指揮千百戶獲倭一船及賊者,升一級,賞銀五十兩,鈔五十錠,軍士水陸擒殺賊,賞銀有差。

永樂初,以將士久勞,命禮部依太祖升賞例,參酌行之。乃分奇功、首功、次功三等,其賞之輕重次第,率臨時取旨,亦不豫為令。十二年定:「凡交鋒之際,突出敵背殺敗賊眾者,勇敢入陣斬將搴旗者,本隊已勝、別隊勝負未決、而能救援克敵者,受命能任事、出奇破賊成功者,皆為奇功。齊力前進、首先敗賊者,前隊交鋒未決、後隊向前敗賊者,皆為首功。軍行及營中擒獲奸細者,亦準首功。餘皆次功。」又立功賞勘合,定四十字,曰:「神威精勇猛,強壯毅英雄。克勝兼超捷,奇功奮銳鋒。智謀宣妙略,剛烈效忠誠。果敢能安定,揚名顯大勛。」編號用寶,貯內府印綬監。當是時,稽功之法甚嚴。

正統十四年,造賞功牌,有奇功、頭功、齊力之分,以大臣主之。凡挺身突陣斬將奪旗者,與奇功牌。生擒瓦剌或斬首一級,與頭功牌。雖無功而被傷者,與齊力牌。蓋專為瓦剌入犯設也。是後,將士功賞視立功之地,準例奏行。北邊為上,東北邊次之,西番及苗蠻又次之,內地反賊又次之。世宗時,苦倭甚,故海上功比北邊尤為最。

北邊,自甘肅迤東,抵山海關。成化十四年例:「一人斬一級者,進一秩,至三秩止。二人共斬者,為首進秩同。壯男與實授,幼弱婦女與署職。為從及四級以上,俱給賞。領軍官部下五百人者,獲五級,進一秩。領千人者,倍之。」正德十年重定例:「獨斬一級者陞一秩。三人共者,首陞署一秩,從給賞。四五六人共者,首給賞,從量賞。二人共斬一幼敵者,首視三人例,從量賞。不願升者,每實授一秩,賞銀五十兩,署職二十兩。」嘉靖十五年定,領軍官千、把總,加至三秩止,都指揮以上,止升署職二級,餘加賞。

東北邊,初定三級當北邊之一。萬歷中,改與北邊同。

番寇苗蠻,亦三級進一秩,實授署職,視北邊。十級以上并不及數者給賞。萬曆三年,令陜西番寇功,視成化中例,軍官千總領五百人者,部下斬三十級,領千人者六十級,把總領五百人者十級,領千人者三十級,俱進一秩,至三秩止。南方蠻賊,宣德九年例,三級以上及斬獲首賊,俱升一秩,餘加賞。正德十六年,定軍官部下斬百級者升署一秩,三百級者實授一秩,四百級者升一秩,餘功加賞。

倭賊,嘉靖三十五年定:「斬倭首賊一級,升實授三秩,不願者賞銀百五十兩。從賊一級,授一秩。漢人脅從一級,署一秩。陣亡者,本軍及子實授一秩。海洋遇賊有功,均以奇功論。」萬曆十二年更定,視舊例少變,以賊眾及船之多寡,為功賞之差。復定海洋征戰,無論倭寇、海賊,勘是奇功,與世襲。雲南夷賊,擒斬功次視倭功。

內地反賊,成化十四年例,六級升一秩,至三秩止,幼男婦女及十九級以上與不及數者給賞。正德七年,定流賊例:「名賊一級,授一秩,世襲,為從者給賞。次賊一級,署一秩。從賊三級及陣亡者,俱授一秩,世襲。重傷回營死者,署一秩。」又以割耳多寡論功,最多者至升二秩,世襲。先是,五年寧夏功,後嘉靖元年江西功,俱視流賊例。崇禎中,購闖、獻以萬金,爵封侯,餘賊有差,以賊勢重,變常格也。

其俘獲人畜、器械,成化例,俱給所獲者。其論功升秩,成化十四年例,軍士升一秩為小旗,舍人升一秩給冠帶,以上類推。嘉靖四十三年定,都督等官無階可升者,所應襲男廕冠帶。萬歷十三年定,都指揮使陞秩者,不授都督,賞銀五十兩,升俸者半之。其有司民兵,隆慶六年定,視軍人例。

自洪、宣以後,賞格皆以斬級多少豫定。條例漸多,倖弊日啟。正德間,副使胡世寧言:「兩軍格鬥,手眼瞬息,不得差池,何暇割級?其獲級者或殺已降,或殺良民,或偶得單行之賊、被掠逃出之人,非真功也。宜選強明剛正之員,為紀功官,痛懲此弊。」時弗能行。故事,鎮守官奏帶,例止五名。後領兵官所奏有至三四百名者,不在斬馘之列,別立名目,曰運送神槍,曰齎執旗牌,曰衝鋒破敵,曰三次當先,曰軍前效勞。冒濫之弊,至斯極已。

古所謂砲,皆以機發石。元初得西域砲,攻金蔡州城,始用火。然造法不傳,後亦罕用。

至明成祖平交阯,得神機槍炮法,特置神機營肄習。製用生、熟赤銅相間,其用鐵者,建鐵柔為最,西鐵次之。大小不等,大者發用車,次及小者用架、用樁、用托。大利於守,小利於戰。隨宜而用,為行軍要器。永樂十年,詔自開平至懷來、宣府、萬全、興和諸山頂,皆置五砲架。二十年,從張輔請,增置於山西大同、天城、陽和、朔州等衛以禦敵。然利器不可示人,朝廷亦慎惜之。

宣德五年,敕宣府總兵官譚廣:「神銃,國家所重,在邊墩堡,量給以壯軍威,勿輕給。」正統六年,邊將黃真、楊洪立神銃局於宣府獨石。帝以火器外造,恐傳習漏洩,敕止之。正統末,邊備日亟,御史楊善請鑄兩頭銅銃。景泰元年,巡關侍郎江潮言:「真定藏都督平安火傘,上用鐵槍頭,環以響鈴,置火藥筒三,發之可潰敵馬。應州民師翱制銃,有機,頃刻三發,及三百步外。」俱試驗之。天順八年,延綏參將房能言麓川破賊,用九龍筒,一線然則九箭齊發,請頒式各邊。

至嘉靖八年,始從右都御史汪鋐言,造佛郎機砲,謂之大將軍,發諸邊鎮。佛郎機者,國名也。正德末,其國舶至廣東。白沙巡檢何儒得其制,以銅為之。長五六尺,大者重千餘斤,小者百五十斤,巨腹長頸,腹有修孔。以子銃五枚,貯藥置腹中,發及百餘丈,最利水戰。駕以蜈蚣船,所擊輒糜碎。二十五年,總督軍務翁萬達奏所造火器。兵部試之,言:「三出連珠、百出先鋒、鐵捧雷飛,俱便用。母子火獸、布地雷砲,止可夜劫營。」御史張鐸亦進十眼銅炮,大彈發及七百步,小彈百步;四眼鐵槍,彈四百步。詔工部造。

萬歷中,通判華光大奏其父所製神異火器,命下兵部。其後,大西洋船至,復得巨炮,曰紅夷。長二丈餘,重者至三千斤,能洞裂石城,震數十里。天啟中,錫以大將軍號,遣官祀之。

崇禎時,大學士徐光啟請令西洋人制造,發各鎮。然將帥多不得人,城守不固,有委而去之者。及流寇犯闕,三大營兵不戰而潰,槍炮皆為賊有,反用以攻城。城上亦發炮擊賊。時中官已多異志,皆空器貯藥,取聲震而已。

明置兵仗、軍器二局,分造火器。號將軍者自大至五。又有奪門將軍大小二樣、神機砲、襄陽炮、盞口砲、碗口砲、旋風砲、流星砲、虎尾砲、石榴炮、龍虎炮、毒火飛炮、連珠佛郎機炮、信炮、神炮、砲裏砲、十眼銅砲、三出連珠砲、百出先鋒砲、鐵捧雷飛炮、火獸布地雷砲、碗口銅鐵銃、手把銅鐵銃、神銃、斬馬銃、一窩鋒神機箭銃、大中小佛郎機銅銃、佛郎機鐵銃、木廂銅銃、筋繳樺皮鐵銃、無敵手銃、鳥嘴銃、七眼銅銃、千里銃、四眼鐵鎗、各號雙頭鐵鎗、夾把鐵手槍、快槍以及火車、火傘、九龍筒之屬,凡數十種。正德、嘉靖間造最多。又各邊自造,自正統十四年四川始。其他刀牌、弓箭、槍弩、狼筅、蒺藜、甲胄、戰襖,在內有兵仗、軍器、鍼工、鞍轡諸局,屬內庫,掌於中官,在外有盔甲廠,屬兵部,掌以郎官。京省諸司衛所,又俱有雜造局。軍資器械名目繁夥,不具載,惟火器前代所少,故特詳焉。

中原用車戰,而東南利舟楫,二者於兵事為最要。自騎兵起,車制漸廢。

洪武五年,造獨轅車,北平、山東千輛,山西、河南八百輛。永樂八年北征,用武剛車三萬輛,皆惟以供餽運。

至正統十二年,始從總兵官硃冕議,用火車備戰。自是言車戰者相繼。十四年,給事中李侃請以CA車千輛,鐵索聯絡,騎卒處中,每車翼以刀牌手五人,賊犯陣,刀牌手擊之,賊退則開索縱騎。帝命造成祭而後用。下車式於邊境,用七馬駕。寧夏多溝壑,總兵官張泰請用獨馬小車,時以為便。箭工周四章言,神機槍一發難繼,請以車載槍二十,箭六百,車首置五鎗架,一人推,二人扶,一人執爨。試可,乃造。

景泰元年,定襄伯郭登請仿古制為偏箱車。轅長丈三尺,闊九尺,高七尺五寸,箱用薄板,置銃。出則左右相連,前後相接,鉤環牽互。車載衣糧、器械並鹿角二。屯處,十五步外設為籓。每車鎗砲、弓弩、刀牌甲士共十人,無事輪番推挽。外以長車二十,載大小將軍銃,每方五輛,轉輸樵採,皆在圍中。又以四輪車一,列五色旗,視敵指揮。廷議此可以守,難於攻戰,命登酌行。蘭州守備李進請造獨輪小車,上施皮屋,前用木板,畫獸面,鑿口,置碗口銃四,槍四,神機箭十四,樹旗一。行為陣,止為營。二年,吏部郎中李賢請造戰車,長丈五尺,高六尺四寸,四圍箱板,穴孔置銃,上闢小窗,每車前後占地五步。以千輛計,四方可十六里,芻糧、器械輜重咸取給焉。帝令亟行。

成化二年,從郭登言,制軍隊小車。每隊六輛,輛九人,二人挽,七人番代,車前置牌畫猊首,遠望若城壘然。八年,寧都諸生何京上禦敵車式,上施鐵網,網穴發槍弩,行則斂之。五十車為一隊,用士三百七十五人。十二年,左都御史李賓請造偏箱車,與鹿角參用。兵部尚書項忠請驗閱,以登高涉險不便,已之。十三年,從甘肅總兵官王璽奏,造雷火車,中立樞軸,旋轉發炮。二十年,宣大總督餘子俊以車五百輛為一軍,每輛卒十人,車隙補以鹿角。既成,而遲重不可用,時人謂之鷓鴣軍。

弘治十五年,陜西總制秦紘請用隻輪車,名曰全勝,長丈四尺,上下共六人,可沖敵陣。十六年,閒住知府範吉獻先鋒霹靂車。

嘉靖十一年,南京給事中王希文請仿郭固、韓琦之制,造車,前銳後方,上置七槍,為櫓三層,各置九牛神弩,傍翼以卒。行載甲兵,止為營陣。下邊鎮酌行。十五年,總制劉天和復言全勝車之便,而稍為損益,用四人推挽,所載火器、弓弩、刀牌以百五十斤為準。箱前畫狻猊,旁列虎盾以護騎士。命從其制。四十三年,有司奏準,京營教演兵車,共四千輛,每輛步卒五人,神槍、夾靶槍各二。自正統以來,言車戰者如此,然未嘗一當敵。

至隆慶中,戚繼光守薊門,奏練兵車七營:以東西路副總兵及撫督標共四營,分駐建昌、遵化、石匣、密雲;薊、遼總兵二營,駐三屯;昌平總兵一營,駐昌平。每營重車百五十有六,輕車加百,步兵四千,騎兵三千。十二路二千里間,車騎相兼,可禦敵數萬。穆宗韙之,命給造費。然特以遏沖突,施火器,亦未嘗以戰也。是後,遼東巡撫魏學曾請設戰車營,仿偏箱之制,上設佛郎機二,下置雷飛砲、快鎗六,每車步卒二十五人。萬曆末,經略熊廷弼請造雙輪戰車,每車火砲二,翼以十卒,皆持火槍。天啟中,直隸巡按御史易應昌進戶部主事曹履吉所制鋼輪車、小沖車等式,以禦敵,皆罕得其用。大約邊地險阻,不利車戰。而舟楫之用,則東南所宜。

舟之制,江海各異。太祖於新江口設船四百。永樂初,命福建都司造海船百三十七,又命江、楚、兩浙及鎮江諸府衛造海風船。成化初,濟川衛楊渠獻《槳舟圖》,皆江舟也。

海舟以舟山之烏槽為首。福船耐風濤,且御火。浙之十裝標號軟風、蒼山,亦利追逐。廣東船,鐵栗木為之,視福船尤巨而堅。其利用者二,可發佛郎機,可擲火球。大福船亦然,能容百人。底尖上闊,首昂尾高,柁樓三重,帆桅二,傍護以板,上設木女牆及砲床。中為四層:最下實土石;次寢息所;次左右六門,中置水櫃,揚帆炊爨皆在是,最上如露臺,穴梯而登,傍設翼板,可憑以戰。矢石火器皆俯發,可順風行。海蒼視福船稍小。開浪船能容三五十人,頭銳,四槳一櫓,其行如飛,不拘風潮順逆。艟喬船視海蒼又小。蒼山船首尾皆闊,帆櫓並用。櫓設船傍近後,每傍五枝,每枝五跳,跳二人,以板閘跳上,露首於外,其制上下三層,下實土石,上為戰場,中寢處。其張帆下椗,皆在上層。戚繼光云:「倭舟甚小,一入裏海,大福、海蒼不能入,必用蒼船逐之,衝敵便捷,溫人謂之蒼山鐵也。」沙、鷹二船,相胥成用。沙船可接戰,然無翼蔽。鷹船兩端銳,進退如飛。傍釘大茅竹,竹間窗可發銃箭,窗內舷外隱人以蕩槳。先駕此入賊隊,沙船隨進,短兵接戰,無不勝。漁船至小,每舟三人,一執布帆,一執槳,一執鳥嘴銃。隨波上下,可掩賊不備。網梭船,定海、臨海、象山俱有之,形如梭。竹桅布帆,僅容二三人,遇風濤輒舁入山麓,可哨探。蜈蚣船,象形也,能駕佛朗機銃,底尖面闊,兩傍楫數十,行如飛。兩頭船,旋轉在舵,因風四馳,諸船無逾其速。蓋自嘉靖以來,東南日備倭,故海舟之制,特詳備云。

明制,馬之屬內廄者曰御馬監,中官掌之,牧於大壩,蓋仿《周禮》十有二閑意。牧於官者,為太僕寺、行太僕寺、苑馬寺及各軍衛,即唐四十八監意。牧於民者,南則直隸應天等府,北則直隸及山東、河南等府,即宋保馬意。其曰備養馬者,始於正統末,選馬給邊,邊馬足,而寄牧於畿甸者也。官牧給邊鎮,民牧給京軍,皆有孳生駒。官牧之地曰草場,或為軍民佃種曰熟地,歲征租佐牧人市馬。牧之人曰恩軍,曰隊軍,曰改編軍,曰充發軍,曰抽發軍。苑馬分三等,上苑萬,中七千,下四千。一夫牧馬十匹,五十夫設圉長一人。凡馬肥瘠登耗,籍其毛齒而時省之。三歲,寺卿偕御史印烙,鬻其羸劣以轉市。邊衛、營堡、府州縣軍民壯騎操馬,則掌於行寺卿。邊用不足,又以茶易於番,以貨市於邊。其民牧皆視丁田授馬,始曰戶馬,既曰種馬,按歲征駒。種馬死,孳生不及數,輒賠補。此其大凡也。

初,太祖都金陵,令應天、太平、鎮江、廬州、鳳陽、揚州六府,滁、和二州民牧馬。洪武六年,設太僕寺於滁州,統於兵部。後增滁陽五牧監,領四十八群。已,為四十監,旋罷,惟存天長、大興、舒城三監。置草場於湯泉、滁州等地。復令飛熊、廣武、英武三衛,五軍養一馬,馬歲生駒,一歲解京。既而以監牧歸有司,專令民牧。江南十一戶,江北五戶養馬一,復其身。太僕官督理,歲正月至六月報定駒,七月至十月報顯駒,十一、二月報重駒。歲終考馬政,以法治府州縣官吏。凡牡曰兒,牝曰騍。兒一、騍四為群,群頭一人。五群,群長一人。三十年,設北平、遼東、山西、陜西、甘肅行太僕寺,定牧馬草場。

永樂初,設太僕寺於北京,掌順天、山東、河南。舊設者為南太僕寺,掌應天等六府二州。四年,設苑馬寺於陜西、甘肅,統六監,監統四苑。又設北京、遼東二苑馬寺,所統視陜西、甘肅。十二年,令北畿民計丁養馬,選居閒官教之畜牧。民十五丁以下一匹,十六丁以上二匹,為事編發者七戶一匹,得除罪。尋以寺卿楊砥言,北方人戶五丁養一,免其田租之半,薊州以東至南海等衛,戍守軍外,每軍飼種馬一。又定南方養馬例:鳳、廬、揚、滁、和五丁一,應天、太、鎮十丁一。淮、徐初養馬,亦以丁為率。十八年,罷北京苑馬寺,悉牧之民。

洪熙元年,令民牧二歲徵一駒,免草糧之半。自是,馬日蕃,漸散於鄰省。濟南、兗州、東昌民養馬,自宣德四年始也。彰德、衛輝、開封民養馬,自正統十一年始也。已而也先入犯,取馬二萬,寄養近京,充團營騎操,而盡以故時種馬給永平等府。景泰三年,令兒馬十八歲、騍馬二十歲以上,免算駒。

成化二年,以南土不產馬,改征銀。四年,始建太僕寺常盈庫,貯備用馬價。是時,民漸苦養馬。六年,吏部侍郎葉盛言:「向時歲課一駒,而民不擾者,以芻牧地廣,民得為生也。自豪右莊田漸多,養馬漸不足。洪熙初,改兩年一駒,成化初,改三年一駒。馬愈削,民愈貧。然馬卒不可少,乃復兩年一駒之制,民愈不堪。請敕邊鎮隨俗所宜,凡可以買馬足邊、軍民交益者,便宜處置。」時馬文升撫陜西,又極論邊軍償馬之累,請令屯田卒田多丁少而不領馬者,歲輸銀一錢,以助賠償。雖皆允行,而民困不能舒也。繼文升撫陜者蕭禎,請省行太僕寺。兵部覆云:「洪、永時,設行太僕及苑馬寺,凡茶馬、番人貢馬,悉收寺、苑放牧,常數萬匹,足充邊用。正統以後,北敵屢入抄掠,馬遂日耗。言者每請裁革,是惜小費而忘大計。」於是敕諭禎,但令加意督察。而北畿自永樂以來,馬日滋,輒責民牧,民年十五者即養馬。太僕少卿彭禮以戶丁有限,而課駒無窮,請定種馬額。會文升為兵部尚書,奏行其請,乃定兩京太僕種馬,兒馬二萬五千,騍馬四之,二年納駒,著為令。時弘治六年也。

十五年冬,尚書劉大夏薦南京太常卿楊一清為副都御史,督理陜西馬政。一清奏言:「我朝以陜右宜牧,設監苑,跨二千餘里。後皆廢,惟存長樂、靈武二監。今牧地止數百里,然以供西邊尚無不足,但苦監牧非人,牧養無法耳。兩監六苑,開城、安定水泉便利,宜為上苑,牧萬馬;廣寧、萬安為中苑;黑水草場逼窄,清平地狹土瘠,為下苑。萬安可五千,廣寧四千,清平二千,黑水千五百。六苑歲給軍外,可常牧馬三萬二千五百,足供三邊用。然欲廣孳息,必多蓄種馬,宜增滿萬匹,兩年一駒,五年可足前數。請支太僕馬價銀四萬二千兩,於平、慶、臨、鞏買種馬七千。又養馬恩隊軍不足,請編流亡民及問遣回籍者,且視恩軍例,凡發邊衛充軍者,改令各苑牧馬,增為三千人。又請相地勢,築城通商,種植榆柳,春夏放牧,秋冬還廄,馬既得安,敵來亦可收保。」孝宗方重邊防,大夏掌兵部,一清所奏輒行。遷總制仍督馬政。

諸監草場,原額十三萬三千七百餘頃,存者已不及半。一清核之,得荒地十二萬八千餘頃,又開武安苑地二千九百餘頃。正德二年聞於朝。及一清去官,未幾復廢。時御史王濟言:「民苦養馬。有一孳生馬,輒害之。間有定駒,賂醫諱之,有顯駒墜落之。馬虧欠不過納銀二兩,既孳生者已聞官,而復倒斃,不過納銀三兩,孳生不死則饑餓。馬日瘦削,無濟實用。今種馬、地畝、人丁,歲取有定額,請以其額數令民買馬,而種馬孳生,縣官無與。」兵部是其言。自後,每有奏報,輒引濟言縣官無與種馬事,但責駒於民,遺母求子矣。

初,邊臣請馬,太僕寺以見馬給之。自改征銀,馬日少,而請者相繼,給價十萬,買馬萬匹。邊臣不能市良馬,馬多死,太僕卿儲巏以為言,請仍給馬。又指陳各邊種馬盜賣私借之弊。語雖切,不能從。而邊鎮給發日益繁。延綏三十六營堡,自弘治十一年始,十年間,發太僕銀二十八萬有奇,買補四萬九千餘匹,寧夏、大同、居庸關等處不與焉。至正德七年,遂開納馬例,凡十二條。九年,復發太僕銀市馬萬五千於山東、遼東、河南及鳳陽、保定諸府。

嘉靖元年,陜西苑馬少卿盧璧條上馬政,請督逋負、明印烙、訓醫藥、均地差,以救目前,而闢場廣蓄為經久計。帝嘉納之。自後言馬事者頗眾,大都因事立說,補救一時而已。二十九年,俺答入寇,太僕馬缺,復行正德納馬例。已,稍增損之。至四十一年,遂開例至捐馬授職。

隆慶二年,提督四夷館太常少卿武金言:「種馬之設,專為孳生備用。備用馬既別買,則種馬可遂省。今備用馬已足三萬,宜令每馬折銀三十兩,解太僕。種馬盡賣,輸兵部,一馬十兩,則直隸、山東、河南十二萬匹,可得銀百二十萬,且收草豆銀二十四萬。」御史謝廷傑謂:「祖制所定,關軍機,不可廢。」兵部是廷傑言。而是時,內帑乏,方分使括天下逋賦。穆宗可金奏,下部議。部請養、賣各半,從之。

太僕之有銀也,自成化時始,然止三萬餘兩。及種馬賣,銀日增。是時,通貢互市所貯亦無幾。及張居正作輔,力主盡賣之議。自萬曆九年始,上馬八兩,下至五兩,又折征草豆地租,銀益多,以供團營買馬及各邊之請。然一騸馬輒發三十金,而州縣以駑馬進,其直止數金。且仍寄養於馬戶,害民不減曩時。又國家有興作、賞賚,往往借支太僕銀,太僕帑益耗。十五年,寺卿羅應鶴請禁支借。二十四年詔太僕給陜西賞功銀。寺臣言:「先年庫積四百餘萬,自東西二役興,僅餘四之一。朝鮮用兵,百萬之積俱空。今所存者,止十餘萬。況本寺寄養馬歲額二萬匹,今歲取折色,則馬之派征甚少,而東徵調兌尤多。卒然有警,馬與銀俱竭,何以應之。」章下部,未能有所釐革也。

崇禎初,核戶兵工三部,借支太僕馬價至一千三百餘萬。蓋自萬曆以來,冏政大壞,而邊牧廢弛,愈不可問。既而遼東督師袁崇煥以缺馬,請於兩京州縣寄養馬內,折三千匹價買之西邊。太僕卿塗國鼎言:「祖宗令民養馬,專供京營騎操,防護都城,非為邊也。後來改折,無事則易馬輸銀,有警則出銀市馬,仍是為京師備御之意。今折銀已多給各鎮,如並此馬盡折,萬一變生,奈何?」帝是其言,卻崇煥請。

按明世馬政,法久弊叢。其始盛終衰之故,大率由草場興廢。太祖既設草場於大江南北,復定北邊牧地:自東勝以西至寧夏、河西、察罕腦兒,以東至大同、宣府、開平,又東南至大寧、遼東,抵鴨綠江又北千里,而南至各衛分守地,又自雁門關西抵黃河外,東歷紫荊、居庸、古北抵山海衛。荒閒平埜,非軍民屯種者,聽諸王駙馬以至近邊軍民樵採牧放,在邊籓府不得自占。永樂中,又置草場於畿甸。尋以順聖川至桑乾河百三十餘里,水草美,令以太僕千騎,令懷來衛卒百人分牧,後增至萬二千匹。宣德初,復置九馬坊於保安州。於是兵部奏,馬大蕃息,以色別而名之,其毛色二十五等,其種三百六十。其後莊田日增,草場日削,軍民皆困於孳養。弘治初,兵部主事湯冕、太僕卿王霽、給事中韓祐、周旋、御史張淳,皆請清核。而旋言:「香河諸縣地占於勢家,霸州等處俱有仁壽宮皇莊,乞罷之,以益牧地。」雖允行,而占佃已久,卒不能清。南京諸衛牧場亦久廢,兵部尚書張鎣請復之。御史胡海言恐遺地利,遂止。京師團營官馬萬匹,與旗手等衛上直官馬,皆分置草場。歲春末,馬非聽用者,坐營官領下場放牧,草豆住支,秋末回。給事御史閱視馬斃軍逃者以聞。後上直馬不出牧,而騎操馬仍歲出如例。嘉靖六年,武定侯郭勛以邊警為辭,奏免之,徵各場租以充公費,餘貯太僕買馬。於是營馬專仰秣司農,歲費至十八萬,戶部為詘,而草場益廢。議者爭以租佃取贏,侵淫至神宗時,弊壞極矣。

茶馬司,洪武中,立於川、陜,聽西番納馬易茶,賜金牌信符,以防詐偽。每三歲,遣廷臣召諸番合符交易,上馬茶百二十斤,中馬七十斤,下馬五十斤。以私茶出者罪死,雖勛戚無貸。末年,易馬至萬三千五百餘匹。永樂中,禁稍弛,易馬少。乃命嚴邊關茶禁,遣御史巡督。正統末,罷金牌,歲遣行人巡察,邊氓冒禁私販者多。成化間,定差御史一員,領敕專理。弘治間,大學士李東陽言:「金牌制廢,私茶盛,有司又屢以敝茶紿番族,番人抱憾,往往以羸馬應。宜嚴敕陜西官司揭榜招諭,復金牌之制,嚴收良茶,頗增馬直,則得馬必蕃。」及楊一清督理苑馬,遂命並理鹽、茶。一清申舊制,禁私販,種官茶。四年間易馬九千餘匹,而茶尚積四十餘萬斤。靈州鹽池增課五萬九千,貯慶陽、固原庫,以買馬給邊。又懼後無專官,制終廢也,於正德初,請令巡茶御史兼理馬政,行太僕、苑馬寺官聽其提調,報可。御史翟唐歲收茶七十八萬餘斤,易馬九千有奇。後法復弛。嘉靖初,戶部請揭榜禁私茶,凡引俱南戶部印發,府州縣不得擅印。三十年,詔給番族勘合,然初制訖不能復矣。

馬市者,始永樂間。遼東設市三,二在開原,一在廣寧,各去城四十里。成化中,巡撫陳鉞復奏行之。後至萬曆初不廢。嘉靖中,開馬市於大同,陜邊宣鎮相繼行。隆慶五年,俺答上表稱貢。總督王崇古市馬七千餘匹,為價九萬六千有奇。其價,遼東以米布絹,宣、大、山西以銀。市易外有貢馬者,以鈔幣加賜之。

初,太祖起江左,所急惟馬,屢遣使市於四方。正元壽節,內外籓封將帥皆以馬為幣。外國、土司、番部以時入貢,朝廷每厚加賜予,所以招攜懷柔者備至。文帝勤遠略,遣使絕域;外國來朝者甚眾,然所急者不在馬。自後狃於承平,駕馭之權失,馬無外增,惟恃孳生歲課。重以官吏侵漁,牧政荒廢,軍民交困矣。蓋明自宣德以後,祖制漸廢,軍旅特甚,而馬政其一云。

