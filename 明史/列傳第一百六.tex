\article{列傳第一百六}

\begin{pinyinscope}
申時行(子用懋用嘉孫紹芳}}王錫爵弟鼎爵子衡沈一貫方從哲沈水隺弟演

申時行,字汝默,長洲人。嘉靖四十一年進士第一。授修撰。歷左庶子,掌翰林院事。萬厲五年,由禮部右侍郎改吏部。時行以文字受知張居正,蘊藉不立崖異,居正安之。六年三月,居正將歸葬父,請廣閣臣,遂以左侍郎兼東閣大學士入預機務。已,進禮部尚書兼文淵閣,累進少傅兼太子太傅、吏部尚書、建極殿。張居正攬權久,操群下如束濕,異己者率逐去之。及居正卒,張四維、時行相繼柄政,務為寬大。以次收召老成,布列庶位,朝論多稱之。然是時內閣權積重,六卿大抵徇閣臣指。諸大臣由四維、時行起,樂其寬,多與相厚善。四維憂歸,時行為首輔。餘有丁、許國、王錫爵、王家屏先後同居政府,無嫌猜。而言路為居正所遏,至是方發舒。以居正素暱時行,不能無諷刺。時行外示博大能容人,心故弗善也。帝雖樂言者訐居正短,而頗惡人論時事,言事者間謫官。眾以此望時行,口語相詆諆。諸大臣又皆右時行拄言者口,言者益憤,時行以此損物望。

十二年三月,御史張文熙嘗言前閣臣專恣者四事,請帝永禁革之。時行疏爭曰:「文熙謂部院百執事不當置考成簿,送閣察考;吏、兵二部除授,不當一一取裁;督撫巡按行事,不當密揭請教;閣中票擬,當使同官知。夫閣臣不職當罷黜,若並其執掌盡削之,是因噎廢食也。

至票擬,無不與同官議者。」帝深以為然,絀文熙議不用。御史丁此呂言侍郎高啟愚以試題勸進居正,帝手疏示時行。時行曰:「此呂以暖昧陷人大辟,恐讒言接踵至,非清明之朝所宜有。」尚書楊巍因請出此呂於外,帝從巍言。而給事御史王士性、李植等交章劾巍阿時行意,蔽塞言路。帝尋亦悔之,命罷啟愚,留此呂。時行、巍求去。有丁、國言:「大臣國體所繫,今以群言留此呂,恐無以安時行、巍心。」國尤不勝憤,專疏求去,詆諸言路。副都御史石星、侍郎陸光祖亦以為言。帝乃聽巍,出此呂於外,慰留時行、國,而言路群起攻國。時行請量罰言者,言者益心憾。既而李植、江東之以大峪山壽宮事撼時行不勝,貶去,閣臣與言路日相水火矣。

初,御史魏允貞、郎中李三才以科場事論及時行子用懋,貶官。給事中鄒元標劾罷時行姻徐學謨,時行假他疏逐之去。已而占物情,稍稍擢三人官,三人得毋廢。世以此稱時行長者。時行欲收人心,罷居正時所行考成法;一切為簡易,亦數有獻納。嘗因災異,力言催科急迫,徵派加增,刑獄繁多,用度侈靡之害。又嘗請止撫按官助工贓罰銀,請減織造數,趣發諸司章奏。緣尚寶卿徐貞明議,請開畿內水田。用鄧子龍、劉綎平隴川,薦鄭洛為經略,趣順義王東歸,寢葉夢熊奏以弭楊應龍之變。然是時天下承平,上下恬熙,法紀漸不振。時行務承帝指,不能大有建立。帝每遇講期,多傳免。時行請雖免講,仍進講章。自後為故事,講筵遂永罷。評事雒于仁進《酒色財氣四箴》,帝大怒,召時行等條分析之,將重譴。時行請毋下其章,而諷于仁自引去,于仁賴以免。然章奏留中自此始。

十四年正月,光宗年五歲,而鄭貴妃有寵,生皇三子常洵,頗萌奪嫡意。時行率同列再請建儲,不聽。廷臣以貴妃故,多指斥宮闈,觸帝怒,被嚴譴。帝嘗詔求直言。郎官劉復初、李懋檜等顯侵貴妃。時行請帝下詔,令諸曹建言止及所司職掌,聽其長擇而獻之,不得專達。帝甚悅,眾多咎時行者。時行連請建儲。十八年,帝召皇長子、皇三子,令時行入見毓德宮。時行拜賀,請亟定大計。帝猶豫久之,下詔曰:「朕不喜激聒。近諸臣章奏概留中,惡其離間朕父子。若明歲廷臣不復瀆擾,當以後年冊立,否則俟皇長子十五歲舉行。」時行因戒廷臣毋激擾。明年八月,工部主事張有德請具冊立儀注。帝怒,命展期一年。而內閣中亦有疏入。時行方在告,次輔國首列時行名。時行密上封事,言:「臣方在告,初不預知。冊立之事,聖意已定。有德不諳大計,惟宸斷親裁,勿因小臣妨大典。」於是給事中羅大紘劾時行,謂陽附群臣之議以請立,而陰緩其事以內交。中書黃正賓復論時行排陷同官,巧避首事之罪。二人皆被黜責。御史鄒德泳疏復上,時行力求罷。詔馳驛歸。歸三年,光宗始出閣講學,十年始立為皇太子。

四十二年,時行年八十,帝遣行人存問。詔書至門而卒。先以雲南岳鳳平,加少師兼太子太師、中極殿大學士,詔贈太師,謚文定。

子用懋、用嘉。用懋,字敬中,舉進士。累官兵部職方郎中。神宗擢太僕少卿,仍視職方事。再遷右僉都御史,巡撫順天。崇禎初,歷兵部左、右侍郎,拜尚書,致仕歸。卒,贈太子太保。用嘉,舉人。歷官廣西參政。孫紹芳,進士,戶部左侍郎。王錫爵,字元馭,太倉人。嘉靖四十一年舉會試第一,廷試第二,授編修。累遷至祭酒。萬曆五年,以詹事掌翰林院。張居正奪情,將廷杖吳中行、趙用賢等。錫爵要同館十餘人詣居正求解,居正不納。錫爵獨造喪次,切言之,居正徑入不顧。中行等既受杖,錫爵持之大慟。明年,進禮部右侍郎。居正甫歸治喪,九卿急請召還,錫爵獨不署名。旋乞省親去。居正以錫爵形己短,益銜之,錫爵遂不出。十二年冬,即家拜禮部尚書兼文淵閣大學士,參機務。還朝,請禁諂諛、抑奔競、戒虛浮、節侈靡、闢橫議、簡工作。帝咸褒納。

初,李植、江東之與大臣申時行、楊巍等相構,以錫爵負時望,且與居正貳,力推之。比錫爵至,與時行合,反出疏力排植等,植等遂去。時時行為首輔,許國次之,三人皆南畿人,而錫爵與時行同舉會試,且同郡,政府相得甚。然時行柔和,而錫爵性剛負氣。十六年,子衡舉順天試第一,郎官高桂、饒伸論之。錫爵連章辨訐,語過忿,伸坐下詔獄除名,桂謫邊方。御史喬璧星請帝戒諭錫爵,務擴其量,為休休有容之臣,錫爵疏辨。以是積與廷論忤。

時群臣請建儲者眾,帝皆不聽。十八年,錫爵疏請豫教元子,錄用言官姜應麟等,且求宥故巡撫李材,不報。嘗因旱災,自陳乞罷。帝優詔留之。火落赤、真相犯西陲,議者爭請用兵,錫爵主款,與時行合。未幾,偕同列爭冊立不得,杜門乞歸。尋以母老,連乞歸省。乃賜道里費,遣官護行。歸二年,時行、國及王家屏相繼去位,有詔趣召錫爵。二十一年正月,還朝,遂為首輔。

先是有旨,是年春舉冊立大典,戒廷臣毋瀆陳。廷臣鑒張有德事,咸默默。及是,錫爵密請帝決大計。帝遣內侍以手詔示錫爵,欲待嫡子,令元子與兩弟且並封為王。錫爵懼失上指,立奉詔擬諭旨。而又外慮公論,因言「漢明帝馬后、唐明皇王后、宋真宗劉后皆養諸妃子為子,請令皇后撫育元子,則元子即嫡子,而生母不必崇位號以上壓皇貴妃」,亦擬諭以進。同列趙志皋、張位咸不預聞。帝竟以前諭下禮官,令即具儀。於是舉朝大嘩。給事中史孟麟、禮部尚書羅萬化等,群詣錫爵第,力爭。廷臣諫者,章日數上。錫爵偕志皋、位力請追還前詔,帝不從。已而諫者益多,而岳元聲、顧允成、張納陛、陳泰來、于孔兼、李啟美、曾鳳儀、鐘化民、項德禎等遮錫爵於朝房,面爭之。李騰芳亦上書錫爵。錫爵請下廷議,不許。請面對,不報。乃自劾三誤,乞罷斥。帝亦迫公議,追寢前命,命少俟二三年議行。錫爵旋請速決,且曰:「曩元子初生,業為頒詔肆赦,詔書稱『祗承宗社』,明以皇太子待之矣。今復何疑而弗決哉?」不報。

七月,彗星見,有詔修省。錫爵因請延見大臣。又言:「彗漸近紫微,宜慎起居之節,寬左右之刑,寡嗜欲以防疾,散積聚以廣恩。」踰月,復言:「慧已入紫微,非區區用人行政所能消弭,惟建儲一事可以禳之。蓋天王之象曰帝星,太子之象曰前星。今前星既耀而不早定,故致此災。誠速行冊立,天變自弭。」帝皆報聞,仍持首春待期之說。錫爵答奏復力言之,又連章懇請。十一月,皇太后生辰,帝御門受賀畢,獨召錫爵煖閣,勞之曰:「卿扶母來京,誠忠孝兩全。」錫爵叩頭謝,因力請早定國本。帝曰:「中宮有出,奈何?」對曰:「此說在十年前猶可,今元子已十三,尚何待?況自古至今,豈有子弟十三歲猶不讀書者?」帝頗感動。錫爵因請頻召對,保聖躬。退復上疏力請,且曰:「外廷以固寵陰謀歸之皇貴妃,恐鄭氏舉族不得安。惟陛下深省。」帝得疏,心益動,手詔諭錫爵:「卿每奏必及皇貴妃,何也?彼數勸朕,朕以祖訓后妃不得與外事,安敢輒從。」錫爵上言:「今與皇長子相形者,惟皇貴妃子,天下不疑皇貴妃而誰疑?皇貴妃不引為己責而誰責?祖訓不與外事者,不與外廷用人行政之事也。若冊立,乃陛下家事,而皇三子又皇貴妃親子,陛下得不與皇貴妃謀乎?且皇貴妃久侍聖躬,至親且賢,外廷紛紛,莫不歸怨,臣所不忍聞。臣六十老人,力捍天下之口,歸功皇貴妃,陛下尚以為疑。然則必如群少年盛氣以攻皇貴妃,而陛下反快於心乎?」疏入,帝頷之。志皋、位亦力請。居數日,遂有出閣之命。而帝令廣市珠玉珍寶,供出閣儀物,計直三十餘萬。戶部尚書楊俊民等以故事爭,給事中王德完等又力諫。帝遂手詔諭爵,欲易期。錫爵婉請,乃不果易。明年二月,出閣禮成,俱如東宮儀,中外為慰。

錫爵在閣時,嘗請罷江南織造,停江西陶器,減雲南貢金,出內帑振河南饑,帝皆無忤,眷禮逾前後諸輔臣。其救李沂,力爭不宜用廷杖,尤為世所稱。特以阿並封指被物議。既而郎中趙南星斥,侍郎趙用賢放歸,論救者咸遭譴謫,眾指錫爵為之。雖連章自明,且申救,人卒莫能諒也。錫爵遂屢疏引疾乞休。帝不欲其去,為出內帑錢建醮祈愈。錫爵力辭,疏八上乃允。先累加太子太保,至是命改吏部尚書,進建極殿,賜道里費,乘傳,行人護歸。歸七年,東宮建,遣官賜敕存問,賚銀幣羊酒。

三十五年,廷推閣臣。帝既用於慎行、葉向高、李廷機,還念錫爵,特加少保,遺官召之。三辭,不允。時言官方厲鋒氣,錫爵進密揭力詆,中有「上於章奏一概留中,特鄙夷之如禽鳥之音」等語。言官聞之大憤。給事中段然首劾之,其同官胡嘉棟等論不已。錫爵亦自闔門養重,竟辭不赴。又三年,卒於家,年七十七。贈太保,謚文肅。

子衡,字辰玉,少有文名。為舉首才,自稱因被論,遂不復會試。至二十九年,錫爵罷相已久,始舉會試第二人,廷試亦第二。授編修,先父卒。

錫爵弟鼎爵,進士。累官河南提學副使。

沈一貫,字肩吾,鄞人。隆慶二年進士。選庶吉士,授檢討,充日講官。進講高宗諒陰,拱手曰:「託孤寄命,必忠貞不二心之臣,乃可使百官總己以聽。茍非其人,不若躬親聽覽之為孝也。」張居正以為刺己,頗憾一貫。居正卒,始遷左中允。歷官吏部左侍郎兼侍讀學士,加太子賓客。假歸。

二十二年起南京禮部尚書,復召為正史副總裁,協理詹事府,未上。王錫爵、趙志皋、張位同居內閣,復有旨推舉閣臣。吏部舉舊輔王家屏及一貫等七人名以上。而帝方怒家屏,譙責尚書陳有年。有年引疾去。一貫家居久,故有清望,閣臣又力薦之。乃詔以尚書兼東閣大學士,與陳于陛同入閣預機務,命行人即家起焉。會朝議許日本封貢。一貫慮貢道出寧波,為鄉郡患,極陳其害,貢議乃止。未幾,錫爵去,于陛位第三,每獨行己意。一貫柔而深中,事志皋等惟謹。其後于陛卒官,志皋病痺久在告,位以薦楊鎬及《憂危竑議》事得罪去,一貫與位嘗私致鎬書,為贊畫主事丁應泰所劾。位疏辨,激上怒罷。一貫惟引咎,帝乃慰留之。

時國本未定,廷臣爭十餘年不決。皇長子年十八,諸請冊立冠婚者益迫。帝責戶部進銀二千四百萬,為冊立、分封諸典禮費以困之。一貫再疏爭,不聽。二十八年,命營慈慶宮居皇長子。工竣,諭一貫草敕傳示禮官,上冊立、冠婚及諸王分封儀。敕既上,帝復留不下。一貫疏趣,則言:「朕因小臣謝廷言贊乘機邀功,故中輟。俟皇長子移居後行之。」既而不舉行。明年,貴妃弟鄭國泰迫群議,請冊立、冠婚並行。一貫因再草敕請下禮官具儀,不報。廷議有欲先冠婚後冊立者,一貫不可,曰:「不正名而茍成事,是降儲君為諸王也。」會帝意亦頗悟,命即日舉行。九月十有八日漏下二鼓,詔下。既而帝復悔,令改期。一貫封還詔書,言「萬死不敢奉詔」,帝乃止。十月望,冊立禮成,時論頗稱之。會志皋於九月卒,一貫遂當國。初,志皋病久,一貫屢請增閣臣。及是乃簡用沈鯉、朱賡,而事皆取決於一貫。尋進太子太保、戶部尚書、武英殿大學士。

自一貫入內閣,朝政已大非。數年之間,礦稅使四出為民害。其所誣劾逮繫者,悉滯獄中。吏部疏請起用建言廢黜諸臣,並考選科道官,久抑不下,中外多以望閣臣。一貫等數諫,不省。而帝久不視朝,閣臣屢請,皆不報。一貫初輔政面恩,一見帝而已。東征及楊應龍平,帝再御午門樓受俘。一貫請陪侍,賜面對,皆不許。上下否隔甚,一貫雖小有救正,大率依違其間,物望漸減。

迨三十年二月,皇太子婚禮甫成,帝忽有疾。急召諸大臣至仁德門,俄獨命一貫入啟祥宮後殿暖西閣。皇后、貴妃以疾不侍側,皇太后南面立稍北,帝稍東,冠服席地坐,亦南面,太子、諸王跪於前。一貫叩頭起居訖,帝曰:「先生前。朕病日篤矣,享國已久,何憾。佳兒佳婦付與先生,惟輔之為賢君。礦稅事,朕因殿工未竣,權宜採取,今可與江南織造、江西陶器俱止勿行,所遣內監皆令還京。法司釋久繫罪囚,建言得罪諸臣咸復其官,給事中、御史即如所請補用。朕見先生止此矣。」言已就臥。一貫哭,太后、太子、諸王皆哭。一貫復奏:「今尚書求去者三,請定去留。」帝留戶部陳渠、兵部田樂,而以祖陵衝決,削工部楊一魁籍。一貫復叩首,出擬旨以進。是夕,閣臣九卿俱直宿朝房。漏三鼓,中使捧諭至,具如帝語一貫者。諸大臣咸喜。翼日,帝疾,廖悔之。中使二十輩至閣中取前諭,言礦稅不可罷,釋囚、錄直臣惟卿所裁。一貫欲不予,中使輒搏顙幾流血,一貫惶遽繳入。時吏部尚書李戴、左都御史溫純期即日奉行,頒示天下,刑部尚書蕭大亨則謂弛獄須再請。無何,事變。太僕卿南企仲劾戴、大亨不即奉帝諭,起廢釋囚。帝怒,并二事寢不行。當帝欲追還成命,司禮太監田義力爭。帝怒,欲手刃之。義言愈力,而中使已持一貫所繳前諭至。後義見一貫唾曰:「相公稍持之,礦稅撤矣,何怯也!」自是大臣言官疏請者日相繼,皆不復聽。礦稅之害,遂終神宗世。

帝自疾瘳以後,政益廢弛。稅監王朝、梁永、高淮等所至橫暴,奸人乘機虐民者愈眾。一貫與鯉、賡共著論以風,又嘗因事屢爭,且揭陳用人行政諸事。帝不省。顧遇一貫厚,嘗特賜敕獎之。一貫素忌鯉,鯉亦自以講筵受主眷,非由一貫進,不為下,二人漸不相能。禮部侍郎郭正域以文章氣節著,鯉甚重之。都御史溫純、吏部侍郎楊時喬皆以清嚴自持相標置,一貫不善也。會正域議奪呂本謚,一貫、賡與本同鄉,寢其議。由是益惡正域,並惡鯉及純、時喬等,而黨論漸興。浙人與公論忤,由一貫始。

三十一年,楚府鎮國將軍華勣訐楚王華奎為假王。一貫納王重賄,令通政司格其疏月餘,先上華奎劾華勣欺罔四罪疏。正域,楚人,頗聞假王事有狀,請行勘虛實以定罪案。一貫持之。正域以楚王饋遺書上,帝不省。及撫按臣會勘並廷臣集議疏入,一貫力右王,嗾給事中錢夢皋、楊應文劾正域,勒歸聽勘,華勣等皆得罪。正域甫登舟,未行,而「妖書」事起。一貫方銜正域與鯉,其黨康丕揚、錢夢皋等遂捕僧達觀、醫生沈令譽等下獄,窮治之。一貫從中主其事,令錦衣帥王之禎與丕揚大索鯉私第三日,發卒圍正域舟,執掠其婢僕乳媼,皆無所得。乃以皦生光具獄。二事錯見正域及楚王傳中。

始,都御史純劾御史于永清及給事中姚文蔚,語稍涉一貫。給事中鐘兆斗為一貫論純,御史湯兆京復劾兆斗而直純。純十七疏求去,一貫佯揭留純。至歲乙巳,大察京朝官。純與時喬主其事,夢皋、兆斗皆在黜中。一貫怒,言於帝,以京察疏留。久之,乃盡留給事、御史之被察者,且許純致仕去。於是主事劉元珍、龐時雍、南京御史朱吾弼力爭之,謂二百餘年計典無特留者。時南察疏亦留中,後迫眾議始下。一貫自是積不為公論所與,彈劾日眾,因謝病不出。三十上四年七月,給事中陳嘉訓、御史孫居相復連章劾其奸。一貫憤,益求去。帝為黜嘉訓,奪居相俸,允一貫歸,鯉亦同時罷。而一貫獨得溫旨,雖賡右之,論者益訾其有內援焉。

一貫之入閣也,為錫爵、志皋所薦。輔政十有三年,當國者四年。枝拄清議,好同惡異,與前後諸臣同。至楚宗、妖書、京察三事,獨犯不韙,論者醜之,雖其黨不能解免也。一貫歸,言者追劾之不已,其鄉人亦多受世詆諆云。一貫在位,累加少傅兼太子太傅、吏部尚書、建極殿大學士。家居十年卒。贈太傅,謚文恭。

方從哲,字中涵,其先德清人。隸籍錦衣衛,家京師。從哲登萬曆十一年進士,授庶吉士,屢遷國子祭酒。請告家居,久不出,時頗稱其恬雅。大學士葉向高請用為禮部右侍郎,不報。中旨起吏部左侍郎。為給事中李成名所劾,求罷,不允。

四十一年,拜禮部尚書兼東閣大學士,與吳道南並命。時道南在籍,向高為首輔,政事多決於向高。向高去國,從哲遂獨相。請召還舊輔沈鯉,不允。御史錢春劾其容悅,從哲乞罷。帝優旨慰留。未幾,道南至。會張差梃擊事起,刑部以瘋癲蔽獄。王之寀鉤得其情,龐保、劉成等跡始露。從哲偕道南斥之寀言謬妄,帝納之。道南為言路所詆,求去者經歲,以母憂歸。從哲復獨相,即疏請推補閣臣。自後每月必請。帝以一人足辦,迄不增置。

從哲性柔懦,不能任大事。時東宮久輟講,瑞王婚禮逾期,惠王、桂王未擇配,福府莊田遣中使督賦,又議令鬻鹽,中旨命呂貴督織造,駙馬王昺以救劉光復褫冠帶,山東盜起,災異數見,言官翟鳳翀、郭尚賓以直言貶,帝遣中使令工部侍郎林如楚繕修咸安營,宣府缺餉數月,從哲皆上疏力言,帝多不聽。而從哲有內援,以名爭而已,實將順帝意,無所匡正。

向高秉政時,黨論鼎沸。言路交通銓部,指清流為東林,逐之殆盡。及從哲秉政,言路已無正人,黨論漸息。丁巳京察,盡斥東林,且及林居者。齊、楚、浙三黨鼎立,務搏擊清流。齊人亓詩教,從哲門生,勢尤張。從哲暱群小,而帝怠荒亦益甚。畿輔、山東、山西、河南、江西及大江南北相繼告災,疏皆不發。舊制,給事中五十餘員,御史百餘員。至是六科止四人,而五科印無所屬;十三道止五人,一人領數職。在外巡按率不得代。六部堂上官僅四五人,都御史數年空署,督撫監司亦屢缺不補。文武大選、急選官及四方教職,積數千人,以吏、兵二科缺掌印不畫憑,久滯都下,時攀執政輿哀訴。詔獄囚以理刑無人不決遣,家屬聚號長安門。職業盡弛,上下解體。

四十六年四月,大清兵克撫順,朝野震驚。帝初頗憂懼,章奏時下,不數月泄泄如故。從哲子世鴻殺人,巡城御史劾之。從哲乞罷,不允。長星見東南,長二丈,廣尺餘,十有九日而滅。是日京師地震。從哲言:「妖象怪徵,層見疊出,除臣奉職無狀痛自修省外,望陛下大奮乾綱,與天下更始。」朝士雜然笑之。帝亦不省。御史熊化以時事多艱、佐理無效劾從哲,乞用災異策免。從哲懇求罷,堅臥四十餘日,閣中虛無人。帝慰留再三,乃起視事。明年二月,楊鎬四路出師,兵科給事中趙興邦用紅旗督戰,師大敗。禮部主事夏嘉遇謂遼事之壞,由興邦及從哲庇李維翰所致,兩疏劾之。眾哲求罷,不敢入閣,視事於朝房。帝優旨懇留,乃復故,而反擢興邦為太常少卿。未幾,大清兵連克開原、鐵嶺。廷臣於文華門拜疏,立請批發,又候旨思善門,皆不報。從哲乃叩首仁德門跪俟俞旨,帝終不報。俄請帝出御文華殿,召見群臣,面商戰守方略。亦不報。請補閣臣疏十上,情極哀,始命廷推。及推上,又不用。從哲復連請,乃簡用史繼偕、沈紘,疏仍留中,終帝世寢不下。御史張新詔劾從哲諸所疏揭,委罪群父,誑言欺人,祖宗二百年金甌壞從哲手。御史蕭毅中、劉蔚、周方鑑、楊春茂、王尊德、左光斗,山西參政徐如翰亦交章擊之。從哲連疏自明,且乞罷。帝皆不問。自劉光復繫獄,從哲論救數十疏。帝特釋為民,而用人行政諸章奏終不發。帝有疾數月。會皇后崩,從哲哭臨畢,請至榻前起居。召見弘德殿,跪語良久,因請補閣臣,用大僚,下臺諫命。帝許之,乃叩頭出。帝素惡言官,前此考選除授者,率候命二三年,及是候八年。從哲請至數十疏,竟不下。帝自以海宇承平,官不必備,有意損之。及遼左軍興,又不欲矯前失,行之如舊。從哲獨秉國成,卒無所匡救。又用姚宗文閱遼東,齮經略熊廷弼去,遼陽遂失。論者謂明之亡,神宗實基之,而從哲其罪首也。

四十八年七月丙子朔,帝不豫,十有七日大漸。外廷憂危,從哲偕九卿臺諫詣思善門問安。越二日,召從哲及尚書周嘉謨、李汝華、黃嘉善、黃克纘等受顧命。又二日,乃崩。八月丙午朔,光宗嗣位。鄭貴妃以前福王故,懼帝銜之,進珠玉及侍姬八人啖帝。選侍李氏最得帝寵,貴妃因請立選侍為皇后,選侍亦為貴妃求封太后。帝已於乙卯得疾,丁巳力疾御門,命從哲封貴妃為皇太后,從哲遽以命禮部。侍郎孫如游力爭,事乃止。辛酉,帝不視朝,從哲偕廷臣詣宮門問安。時都下紛言中官崔文昇進洩藥,帝由此委頓,而帝傳諭有「頭目眩暈,身體軟弱,不能動履」語,群情益疑駭。給事中楊漣劾文升,並及從哲。刑部主事孫朝肅、徐儀世、御史鄭宗周並上書從哲,請保護聖體,速建儲貳。從哲候安,因言進藥宜慎。帝褒答之。戊辰,新閣臣劉一燝、韓爌入直,帝疾已殆。辛未,召從哲、一燝、爌,英國公張惟賢,吏部尚書周嘉謨,戶部尚書李汝華,禮部侍郎署部事孫如游,刑部尚書黃克纘,左都御史張問達,給事中范濟世、楊漣,御史顧慥等至乾清宮。帝御東煖閣憑几,皇長子、皇五子等皆侍。帝命諸臣前,從哲等因請慎醫藥。帝曰:「十餘日不進矣。」遂諭冊封選侍為皇貴妃。甲戌,復召諸臣,諭冊封事。從哲等請速建儲貳。帝顧皇長子曰:「卿等其輔為堯、舜。」又語及壽宮,從哲等以先帝山陵對。帝自指曰;「朕壽宮也。」諸臣皆泣。帝復問:「有鴻臚官進藥者安在?」從哲曰:「鴻臚寺丞李可灼自云仙方,臣等未敢信。」帝命宣可灼至,趣和藥進,所謂紅丸者也。帝服訖,稱「忠臣」者再。諸臣出俟宮門外。頃之,中使傳上體平善。日晡,可灼出,言復進一丸。從哲等問狀,曰:「平善如前。」明日九月乙亥朔卯刻,帝崩。中外皆恨可灼甚,而從哲擬遺旨賚可灼銀幣。時李選侍居乾清宮,群臣入臨,諸閹閉宮門不許入。劉一燝、楊漣力拄之,得哭臨如禮,擁皇長子出居慈慶宮。從哲委蛇而已。初,鄭貴妃居乾清宮侍神宗疾,光宗即位猶未遷。尚書嘉謨責貴妃從子養性,乃遷慈寧宮。及光宗崩,而李選侍居乾清宮。給事中漣及御史左光斗念選侍嘗邀封后,非可令居乾清,以沖主付託也。於是議移宮,爭數日不決。從哲欲徐之。至登極前一日,一燝、爌邀從哲立宮門請,選侍乃移噦鸞宮。明日庚辰,熹宗即位。

先是,御史王安舜劾從哲輕薦狂醫,又賞之以自掩。從哲擬太子令旨,罰可灼俸一年。御史鄭宗周劾文昇罪,請下法司,從哲擬令旨司禮察處。及御史郭如楚、馮三元、焦源溥,給事中魏應嘉,太常卿曹珖,光祿少卿高攀龍,主事呂維祺,先後上疏言:「可灼罪不容誅,從哲庇之,國法安在!」而給事中惠世揚直糾從哲十罪、三可殺。言:「從哲獨相七年,妨賢病國,罪一。驕蹇無禮,失誤哭臨,罪二。梃擊青宮,庇護奸黨,罪三。恣行胸臆,破壞絲綸,罪四。縱子殺人,蔑視憲典,罪五。阻抑言官,蔽壅耳目,罪六。陷城失律,寬議撫臣,罪七。馬上催戰,覆沒全師,罪八。徇私罔上,鼎鉉貽羞,罪九。代營榷稅,蠹國殃民,罪十。貴妃求封后,舉朝力爭,從哲依違兩可,當誅者一。李選侍乃鄭氏私人,抗凌聖母,飲恨而沒。從哲受劉遜、李進忠所盜美珠,欲封選侍為貴妃,又聽其久據乾清,當誅者二。崔文升用洩藥傷損先帝,諸臣論之,從哲擬脫罪,李可灼進劫藥,從哲擬賞賚,當誅者三。」疏入,責世楊輕詆。從哲累求去,皆慰留。已而張潑、袁化中、王允成等連劾之,皆不聽。其冬,給事中程註復劾之,從哲力求去,疏六上。命進中極殿大學士,賚銀幣、蟒衣,遣行人護歸。

天啟二年四月,禮部尚書孫慎行追論可灼進紅丸,斥從哲為弒逆。詔廷臣議。都御史鄒元標主慎行疏。從哲疏辨,自請削官階,投四裔。帝慰諭之。給事中魏大中以九卿議久稽,趣之上。廷臣多主慎行,罪從哲,惟刑部尚書黃克纘,御史王志道、徐景濂,給事中汪慶百右從哲,而詹事公鼐持兩端。時大學士爌述進藥始末,為從哲解。於是吏部尚書張問達會戶部尚書汪應蛟合奏言:「進藥始末,臣等共聞見。輔臣視皇考疾,急迫倉皇,弒逆二字何忍言。但可灼非醫官,且非知脈知醫者。以藥嘗試,先帝龍馭即上昇。從哲與臣等九卿未能止,均有罪,乃反賚可灼。及御史安舜有言,止令養病去,罰太輕,何以慰皇考,服中外。宜如從哲請,削其官階,為法任咎。至可灼罪不可勝誅,而文升當皇考哀感傷寒時,進大黃涼藥,罪又在可灼上。法皆宜顯僇,以洩公憤。」議上,可灼遣戍,文升放南京,而從哲不罪。無何,慎行引疾去。五年,魏忠賢輯「梃擊」、「紅丸」、「移宮」三事為《三朝要典》,以傾正人,遂免可灼戍,命文升督漕運。其黨徐大化請起從哲,從哲不出。然一時請誅從哲者貶殺略盡矣。崇禎元年二月,從哲卒。贈太傅,謚文端。三月,下文升獄,戍南京。

沈紘,字銘縝,烏程人。父節甫,字以安。嘉靖三十八年進士。授禮部儀制主事,厲祠祭郎中。詔建祠禁內,令黃冠祝釐,節甫持不可。尚書高拱恚甚,遂移疾歸。起光祿丞。會拱掌吏部,復移疾避之。萬曆初,屢遷至南京刑部右侍郎。召為工部左侍郎,攝部事。御史高舉言節甫素負難進之節,不宜一歲三遷。吏部以節甫有物望,絀其議。節甫連上疏請省浮費,核虛冒,上興作,減江、浙織造,停江西瓷器,帝為稍減織造數。中官傳奉,節甫持不可,且上疏言之。又嘗獻治河之策,語鑿鑿可用。父憂歸,卒。贈右副都御史。天啟初,紘方柄用,得賜謚端清。

紘與弟演同登萬曆二年進士。紘改庶吉士,授檢討。累官南京禮部侍郎,掌部事。西洋人利瑪竇入貢,因居南京,與其徒王豐肅等倡天主教,士大夫多宗之。紘奏:「陪京都會,不宜令異教處此。」識者韙其言。然紘素乏時譽。與大學士從哲同里閈,相善也。神宗末,從哲獨當國,請補閣臣,詔會推。亓詩教等緣從哲意,擯何宗彥、劉一燝輩,獨以紘及史繼偕名上。帝遂用之。或曰由從哲薦也。疏未發,明年,神宗崩,光宗立,乃召紘為禮部尚書兼東閣大學士。未至,光宗復崩。天啟元年六月,紘始至。

故事,詞臣教習內書堂,所教內豎執弟子禮。李進忠、劉朝皆紘弟子。李進忠者,魏忠賢始名也。紘既至,密結二人,乃奏言:「遼左用兵亟,臣謹於東陽、義烏諸邑及揚州、紘安募材官勇士二百餘,請以勇士隸錦衣衛,而量授材官職。」進忠、朝方舉內操,得淮奏大喜。詔錦衣官訓練募士,授材官王應斗等遊擊以下官有差。紘又奏募兵後至者復二百餘人,請發遼東、四川軍前。詔從之。尋加太子太保,進文淵閣,再進少保兼太子太保、戶部尚書、武英殿大學士。

禁中內操日盛,駙馬都尉王昺亦奉詔募兵,願得帷幄重臣主其事。廷臣皆言紘與朝陰相結,於是給事中惠世揚、周朝瑞等劾紘陽託募兵,陰藉通內。劉朝內操,紘使門客誘之。王昺疏,疑出紘教。閹人、戚畹、姦輔內外弄兵,長安片土,成戰場矣。紘疏辨,因請疾求罷。帝慰留之。世揚等遂盡發紘通內狀,刑部尚書王紀再疏劾紘,比之蔡京。紘亦劾紀保護熊廷弼、佟卜年、劉一獻等。詔兩解之。未幾,紀以卜年獄削籍,議者益側目紘。大學士葉向高言「紀、紘交攻,均失大臣體。今以讞獄斥紀,如公論何?」硃國祚至以去就爭,帝皆弗聽。紘不自安,乃力求去。命乘傳歸。逾年卒。贈太保,謚文字。

淮弟演,由工部主事歷官南京刑部尚書。

+贊曰:神宗之朝,於時為豫,於象為蠱。時行諸人有鳴豫之兇,而無斡蠱之略。外畏清議,內固恩寵,依阿自守,掩飾取名,弼諧無聞,循默避事。《書》曰「股肱惰哉,萬事隳哉」,此孔子所為致嘆於「焉用彼相」也。


\end{pinyinscope}