\article{列傳第七十七}

\begin{pinyinscope}
○李文祥孫磐徐珪胡爟周時從王雄羅僑葉釗劉天麒戴冠黃鞏陸震夏良勝萬潮等何遵劉校等

李文祥,字天瑞,麻城人。祖正芳,山西布政使。父,陜西參政。文祥自幼俊異。弱冠舉於鄉,成化末登進士。萬安當國,重其才。以孫弘璧與同榜,款於家,文祥意弗慊也。屬題畫鳩,語含刺,安深銜之。未幾,孝宗嗣位,即上封事,略曰:

祖宗設內閣、六部,贊萬幾,理庶務,職至重也。頃者,在位多匪人,權移內侍。賞罰任其喜怒,禍福聽其轉移。仇視言官,公行賄賂。阿之則交引驟遷,忤之則巧讒遠竄。朝野寒心,道路側目。望陛下密察渠魁,明彰國憲,擇謹厚者供使令。更博選大臣,諮諏治理,推心委任,不復嫌疑,然後體統正而近習不得肆也。

祖宗定律,輕重適宜。頃法司專徇己私,不恤國典。豪強者雖重必寬,貧弱者雖輕必罪。惠及奸宄,養成玩俗。兼之風尚奢麗,禮制蕩然。豪民僭王者之居,富室擬公侯之服。奇技淫巧,上下同流。望陛下申明舊章,俾法曹遵律令,臣庶各守等威,然後禮法明而人心不敢玩也。

然國無其人,誰與共理?致仕尚書王恕、王竑,孤忠自許,齒力未衰;南京主事林俊、思南通判王純,剛方植躬,才品兼茂。望陛下起列朝端,資其議論,必有裨益,可翊明時。且賢才難得,自古為然。習俗移人,豪傑不免。惟茲臣庶,不盡庸愚。能知自愧,即屬名流;樂其危菑,乃為猥品。願陛下明察群倫,罷其罔上營私違天蠹物者,餘則勉以自新。既開改過之路,必多遷善之人。

臣見登極詔書,不許風聞言事。古聖王懸鼓設木,自求誹謗。言之縱非其情,聽者亦足為戒,何害於國,遽欲罪之?昔李林甫持此以禍唐,王安石持此以禍宋。遠近驟聞,莫不驚駭。願陛下再頒明詔,廣求直言,庶不墮奸謀,足彰聖德。大率君子之言決非小人之利,諮問倘及,必肆中傷。如有所疑,請試面對。

疏奏,宦官及執政萬安、劉吉、尹直等咸惡之,數日不下。忽詔詣左順門,以疏內有「中興再造」語,傳旨詰責。文祥從容辨析而出。謫授陜西咸寧丞。南京主事夏崇文論救,不納。工部主事莆田林沂復請召文祥及湯鼐,納崇文言,且召陳獻章、謝鐸等。時安已去,吉、直激帝怒,嚴旨切責之。廷臣多薦文祥,率為吉、直所沮。

弘治二年以王恕薦召為兵部主事,監司以下餽贐皆不納。到官未踰月,復以吉人事下獄,貶貴州興隆衛經歷。都御史鄧廷瓚征苗,咨以兵事,大奇之,欲薦為監司。文祥曰:「昔以言事出,今以軍功進,不可。」固辭不得,乃請齎表入都,固乞告歸。疏再上,不許。還經商城,渡冰陷,死焉,年僅三十。

孫磐,遼陽人。弘治九年進士。觀政在部時,刑部典吏徐珪以滿倉兒事劾中官楊鵬得罪,磐上疏曰:「近諫官以言為諱,而排寵倖觸權奸者乃在胥吏,臣竊羞之。請定建言者為四等。最上不避患害,抗彈權貴者。其次揚清激濁,能補闕拾遺。又其次,建白時政,有裨軍國。皆分別擢敘。而粉飾文具、循默不言者,則罷黜之。庶言官知警,不至曠鳷。」時不見用。

徐珪者,應城人。先是,千戶吳能以女滿倉兒付媒者鬻於樂婦張,紿曰:「周皇親家也。」後轉鬻樂工袁璘所。能歿,妻聶訪得之。女怨母鬻己,詭言非己母。聶與子劫女歸。璘訟於刑部,郎中丁哲、員外郎王爵訊得情。璘語不遜,哲笞璘,數日死。御史陳玉、主事孔琦驗璘屍,瘞之。東廠中官楊鵬從子嘗與女淫,教璘妻訴冤於鵬而令張指女為妹,又令賈校尉屬女亦如張言。媒者遂言聶女前鬻周皇親矣。奏下鎮撫司,坐哲、爵等罪。復下法司、錦衣衛讞,索女皇親周彧家,無有。復命府部大臣及給事、御史廷訊,張與女始吐實。都察院奏,哲因公杖人死,罪當徒。爵、玉、琦及聶母女當杖。獄上,珪憤懣,抗疏曰:「聶女之獄,哲斷之審矣。鵬拷聶使誣服,鎮撫司共相蔽欺。陛下令法司、錦衣會問,懼東廠莫敢明,至鞫之朝堂乃不能隱。夫女誣母僅擬杖,哲等無罪反加以徒。輕重倒置如此,皆東廠威劫所致也。臣在刑部三年,見鞫問盜賊,多東廠鎮撫司緝獲,有稱校尉誣陷者,有稱校尉為人報仇者,有稱校尉受首惡贓而以為從、令傍人抵罪者。刑官洞見其情,無敢擅更一字。上干天和,災異迭見。臣願陛下革去東廠,戮鵬叔姪並賈校尉及此女於市,謫戍鎮撫司官極邊,進哲、爵、琦、玉各一階,以洗其冤,則天意可回,太平可致。如不罷東廠,亦當推選謹厚中官如陳寬、韋泰者居之,仍簡一大臣與共理。鎮撫司理刑亦不宜專用錦衣官。乞推選在京各衛一二人及刑部主事一人,共蒞其事。或三年、六年一更,則巡捕官校,當有作奸擅刑,誣及無辜者矣。臣一介微軀,左右前後皆東廠鎮撫司之人,禍必不免。顧與其死於此輩,孰若死於朝廷。願斬臣頭,以行臣言。給臣妻子送骸骨歸,臣雖死無恨。」帝怒,下都察院考訊。都御史閔珪等抵以奏事不實,贖徒還役。帝責具狀,皆上疏引罪,奪俸有差。珪贖徒畢,發為民。既而給事中龐泮等言:「哲等獄詞覆奏已餘三月,繫獄者凡三十八人,乞早為省釋。」乃杖滿倉兒,送浣衣局。哲給璘理葬資,發為民。爵及琦、玉俱贖杖還職。時弘治九年十二月也。

磐尋擢吏部主事。正德元年,宦官漸用事,磐復上疏曰:「今日弊政,莫甚於內臣典民。夫臣以內稱,外事皆不當預,矧可使握兵柄哉。前代盛時,未嘗有此。唐、宋季世始置監軍,而其國遂以不永。今九邊鎮守、監槍諸內臣,恃勢專恣,侵剋百端。有警則擁精卒自衛,克敵則縱部下攘功。武弁藉以夤緣,憲司莫敢訐問。所攜家人頭目,率惡少無賴。吞噬爭攫,勢同狼虎,致三軍喪氣,百職灰心。乞盡撤還京,專以邊務責將帥,此今日修攘要務也。」不從。及劉瑾得志,斥磐為奸黨,勒之歸。瑾誅,起河南僉事,坐累罷。

珪以刑部主事陳鳳梧薦,授桐鄉丞。正德中,歷贛州通判。招降盜魁何積玉。已,復叛,下珪獄,尋釋之。後以平盜功擢知州。

胡爟,字仲光,蕪湖人。弘治六年進士。改庶吉士,授戶部主事。十年三月,災異求言。爟應詔,疏言:「中官李廣、楊鵬引左道劉良輔輩惑亂聖聰,濫設齋醮,耗蠹國儲。而不肖士大夫方昏暮乞憐於其門,交通請託。陰盛陽微,災何由弭?」因極陳戚畹、方士、傳奉冗員之害。疏留中。未幾,廣死,故爟得無罪。

當成化時,宦官用事。孝宗嗣位,雖間有罷黜,而勢積重不能驟返。忤之者必結黨排陷,不勝不止。前後庶僚以忤璫被陷者,如弘治元年戶部員外郎周時從疏請置先朝遺奸汪直、錢能、蔡用輩於重典,而察核兩京及四方鎮守中官。諸宦官摘其奏中「宗社」字不越格,命法司逮治。已而釋之。

十三年秋,大同有警,命保國公朱暉禦之。行人永清王雄極言暉不足任,且請罷中官監督,以重將權。苗逵方督暉軍,謂雄阻軍,乃下詔獄,謫雲南浪穹丞。

羅僑,字維升,吉水人。性純靜,寡嗜慾。受業張元禎,講學里中。舉弘治十二年進士,除新會知縣,有惠愛。

正德初,入為大理右評事。五年四月,京師旱霾,上疏曰:「臣聞人道理則陰陽和,政事失則災沴作。頃因京師久旱,陛下特沛德音,釋逋戍之囚,弛株連之禁,而齋禱經旬,雨澤尚滯。臣竊以為天心仁愛未已也。陛下視朝,或至日昃,狎侮群小,號呶達旦,其何以承天心基大業乎!文網日密,誅求峻急。盜賊白晝殺人,百姓流移載道,元氣索然。科道知之而不敢言,內閣言之而不敢盡,此壅蔽之大患也。古者進退大臣,必有體貌,黥劓之罪不上大夫。邇來公卿去不以禮。先朝忠藎如劉大夏者,謫戍窮邊,已及三載,陛下置之不問,非所以待耆舊、敬大臣也。本朝律例,參酌古今,足以懲奸而蔽罪。近者法司承望風旨,巧中善類。傳曰:『賞僭則及淫人,刑濫則及善人。不幸而過,寧僭無濫。』今之刑罰,濫孰甚焉。願陛下慎逸游,屏玩好,放棄小人,召還舊德,與在廷臣工,宵旰圖治,并敕法司慎守成律。即有律輕情重者,亦必奏請裁決,毋擅有輕重。庶可上弭天變,下收人心。」時朝士久以言為諱。僑疏上,自揣必死,輿櫬待命。劉瑾大怒,矯中旨詰責數百言,令廷臣議罪。大學士李東陽力救,得改原籍教職。其秋,瑾敗,僑尋召復官,引病去。宸濠反,王守仁起兵吉安,僑首赴義。

世宗即位,即家授台州知府。建忠節祠,祀方孝孺。延布衣張尺,詢民間疾苦。歲時循行阡陌,課農桑,講明冠婚喪祭禮,境內大治。嘉靖二年舉行卓異。都御史姚鏌上書訟僑曰:「人臣犯顏進諫,自古為難。曩『八黨』弄權,逆瑾亂政,廷臣結舌,全軀自保。而給事中劉掞、評事羅僑殉國忘身,發摘時弊,幸存餘息。遭遇聖朝,謂宜顯加獎擢,用厲具臣。乃僑知台州,掞知長沙,使懷忠竭節之士淹於常調,臣竊為朝廷惜之。」帝納其言,擢僑廣東左參政,僑辭。部牒敦趣,不得已之官。踰年,遂謝病歸。

僑敦行誼,動則古人。羅洪先居喪,不廢講學,僑以為非禮,遺書責之。其峭直如此。

葉釗,字時勉,豐城人。弘治十五年進士。除南京刑部主事。獄囚久淹,悉按法出之。守備中官侵蘆洲,判歸之民。應天諸府災,上荒政四事。尋進員外郎。

武宗立,應詔陳八事,中言:「宣、大被寇,殺卒幾千人。監督中官苗逵妄報首功,宜召還候勘。宦官典兵,於古未見。唐始用之,而宗社丘墟;我正統朝用之,而鑾輿北狩。自今軍務勿遣監督,鎮守者亦宜撤還。且國初宦官悉隸禮部,秩不過四品,職不過掃除。今請仍隸之部,易置司禮,俾供雜役。罷革東廠,移為他署。斯左右不得擅權,而後天下可安也。」又乞召還劉大夏,宥諫官戴銑等。劉瑾怒,坐斷獄詿誤,逮下詔獄,削籍歸。講學西江。瑾誅,起禮部員外郎,未聞命卒。學者祀之石鼓書院。

時又有工部主事劉天麒者,臨桂人,釗同年進士。分司呂梁。奄人過者不為禮,愬之瑾,逮下詔獄,謫貴州安莊驛丞卒。嘉靖初,復官予祭。

戴冠,信陽人。正德三年進士。為戶部主事。見寵倖日多,廩祿多耗,乃上疏極諫,略曰:「古人理財,務去冗食。近京師勢要家子弟僮奴茍竊爵賞,錦衣官屬數至萬餘,次者繫籍勇士,投充監局匠役,不可數計,皆國家蠹也。歲漕四百萬,宿有嬴餘。近絀水旱,所入不及前,而歲支反過之,計為此輩耗三之一。陛下何忍以赤子膏血,養無用之蠹乎!兵貴精,不貴多。邊軍生長邊士,習戰陣,足以守禦。今遇警輒發京軍,而宣府調入京操之軍,累經臣下論列,堅不遣還。不知陛下何樂於邊軍,而不為關塞慮也。天子藏富天下,務鳩聚為帑藏,是匹夫商賈計也。逆瑾既敗,所籍財產不歸有司,而貯之豹房,遂創新庫。夫供御之物,內有監局,外有部司,此庫何所用之。」疏入,帝大怒,貶廣東烏石驛丞。

嘉靖初,起官,歷山東提學副使,以清介聞。

黃鞏,字仲固,莆田人。弘治十八年進士。正德中,由德安推官入為刑部主事,掌諸司奏牘。歷職方武選郎中。十四年三月,有詔南巡,鞏上疏曰:

陛下臨御以來,祖宗之綱紀法度一壞於逆瑾,再壞於佞倖,又再壞於邊帥,蓋蕩然無餘矣。天下知有權臣,不知有天子,亂本已成,禍變將起。試舉當今最急者陳之。

一,崇正學。臣聞聖人主靜,君子慎動。陛下盤遊無度,流連忘返,動亦過矣。臣願陛下高拱九重,凝神定慮,屏紛華,斥異端,遠佞人,延故老,訪忠良。可以涵養氣質,薰陶德性,而聖學維新,聖政自舉。

二,通言路。言路者,國家之命脈也。古者明王導人以言,用其言而顯其身。今則不然。臣僚言及時政者,左右匿不以聞。或事關權臣,則留中不出,而中傷以他事。使其不以言獲罪,而以他事獲罪。由是,雖有安民長策,謀國至計,無因自達。雖必亂之事,不軌之臣,陛下亦何由知?臣願廣開言路,勿罪其出位,勿責其沽名,將忠言日進,聰明日廣,亂臣賊子亦有所畏而不敢肆矣。

三,正名號。陛下無故降稱大將軍、太師、鎮國公,遠近傳聞,莫不驚嘆。如此,則誰為天子者?天下不以天子事陛下,而以將軍事陛下,天下皆為將軍之臣矣。今不削去諸名號,昭上下之分,則體統不正,朝廷不尊。古之天子亦有號稱「獨夫」,求為匹夫而不得者,竊為陛下懼焉。

四,戒遊幸。陛下始時遊戲,不出大庭,馳逐止於南內,論者猶謂不可。既而幸宣府矣,幸大同矣,幸太原、榆林矣。所至費財動眾,郡縣騷然,至使民間夫婦不相保。陛下為民父母,何忍使至此極也?近復有南巡之命。南方之民爭先挈妻子避去,流離奔踣,怨讟煩興。今江、淮大饑,父子兄弟相食。天時人事如此,陛下又重蹙之,幾何不流為盜賊也。奸雄窺伺,侍時而發。變生在內,則欲歸無路;變生在外,則望救無及。陛下斯時,悔之晚矣。彼居位大臣,用事中官,親暱群小,夫豈有毫髮愛陛下之心哉?皆欲陛下遠出,而後得以擅權自恣,乘機為利也。其不然,則亦袖手旁觀,如秦、越人不相休戚也。陛下宜翻然悔悟,下哀痛罪己之詔。罷南巡,撤宣府離宮,示不復出。發內帑以振江、淮,散邊軍以歸卒伍。雪已往之謬舉,收既失之人心。如是,則尚可為也。

五,去小人。自古未有小人用事不亡國喪身者也。今之小人簸弄威權、貪溺富貴者,實繁有徒。至於首開邊事,以兵為戲,使陛下勞天下之力,竭四海之財,傷百姓之心者,則江彬之為也。彬,行伍庸流,兇狠傲誕,無人臣禮。臣但見其有可誅之罪,不聞其有可賞之功。今乃賜以國姓,封以伯爵,託以心腹,付以京營重寄。使其外持兵柄,內蓄逆謀,以成騎虎之勢,此必亂之道也。天下切齒怒罵,皆欲食彬之肉。陛下亦何惜一彬,不以謝天下哉!

六,建儲貳。陛下春秋漸高,前星未耀,祖宗社稷之託搖搖無所寄。方且遠事觀遊,屢犯不測;收養義子,布滿左右。獨不能豫建親賢以承大業,臣以為陛下殆倒置也。伏望上告宗廟,請命太后,旁諏大臣,擇宗室親賢者一人養於宮中,以繫四海之望。他日誕生皇子,仍俾出籓,實宗社無疆之福也。

員外郎陸震草疏將諫,見鞏疏稱嘆,因毀己稿,與鞏連署以進。帝怒甚,下二人詔獄,復跪午門。眾謂天子且出,鞏曰:「天子出,吾當牽裾死之。」跪五日,期滿,仍繫獄。越二十餘日,廷杖五十,斥為民。彬使人沿途刺鞏,有治洪主事知而匿之,間行得脫。

既歸,潛心著述。或米盡,日中未爨,晏如也。嘗歎曰:「人生至公卿富貴矣,然不過三四十年。惟立身行道,千載不朽。世人顧往往以此易彼,何也?」

世宗立,召為南京大理丞。疏請稽古正學,敬天勤民,取則堯、舜,保全君子,辯別小人。明年入賀,卒於京師。行人張岳訟其直節,贈大理少卿,賜祭葬。天啟初,追謚忠裕。

陸震,字汝亭,蘭谿人。受業同縣章懋,以學行知名。正德三年進士。除泰和知縣。時劉瑾擅政。以逋鹽課責縣民償者連數百人,震力白之上官,得免。鎮守中官歲徵貢絺,為減其額。增築學舍居諸生,毀淫祠祀忠節。浮糧累民,稽賦籍,得詭寄隱匿者萬五千石以補之。建倉縣左,儲穀待振。親行鄉落,勸課農桑。立保伍法,使民備盜。甓城七里,外為土城十里周之。時發狼兵討賊,所至擾民。震言於總督,令毋聽檥舟,官具糧糗,以次續食,兵行肅然。督捕永豐、新淦賊,以功受賞。撫按交薦,徵為兵部主事。泰和人生祠之。

在部,主諸司章奏,與中人忤,改巡紫荊諸關。又以論都御史彭澤、副使胡世寧無罪,忤尚書王瓊、陸完。

孝貞皇后崩,武宗至自宣府。既發喪數日,復欲北出。震抗疏曰:「日者,昊天不弔,威降大戚。車駕在狩,群情惶惶。陛下單騎衝雪還宮,百官有司莫不感愴,以為陛下前蔽而今明也。乃者梓宮在殯,遽擬遊巡,臣知陛下之心必有蹙然不安者。且陛下即位十有二年矣,十者乾之終,十有二者支之終。當氣運周會,正修德更新時,顧乃營宣府以為居,縱騎射以為樂,此臣所深懼也。古人君車馬遊畋之好,雖或有之,至若以外為主,以家為客,挈天下大器、賞罰大柄付之於人,漠然不關意念,此古今所絕無者。伏望勉終喪制,深戒盤游。」不報。

進武選員外郎。已,偕黃鞏諫南巡,遂下詔獄。獄中與鞏講《易》九卦,明憂患之道。同繫者率處分後事,震獨無一言。既杖,創甚,作書與諸子,「吾雖死,汝等當勉為忠孝。吾筆亂,神不亂也」,遂卒。世宗立,贈太常少卿。予祭。

方震等繫獄,江彬必欲致之死,絕其飲食。震季子體仁,年十五,變服為他囚親屬,職納橐饘焉。後有詔錄一子官,諸兄讓體仁,為漳州通判,有政聲。孫可教,由進士歷南京禮部侍郎。

夏良勝,字于中,南城人。少為督學副使蔡清所知,曰「子異日必為良臣,當無有勝子者」,遂名良勝。正德二年舉鄉試第一。明年,成進士,授刑部主事,調吏部,進考功員外郎。

南巡詔下,良勝具疏,與禮部主事萬潮、太常博士陳九川連署以進,言:「方今東南之禍,不獨江、淮;西北之憂,近在輦轂。廟祀之鬯位,不可以久虛;聖母之孝養,不可以恒曠。宮壺之孕祥,尚可以早圖;機務之繁重,未可以盡委。『鎮國』之號,傳聞海內,恐生覬覦之階;邊將之屬,納於禁近,詎忘肘腋之患。巡遊不已,臣等將不知死所矣。」時舒芬、黃鞏、陸震疏已前入。吏部郎中張衍瑞等十四人、刑部郎中陸俸等五十三人繼之,禮部郎中姜龍等十六人、兵部郎中孫鳳等十六人又繼之。而醫士徐鏊亦以其術諫,略言:「養身之道,猶置燭然,室閉之則堅,風暴之則淚。陛下輕萬乘,習嬉娛,躍馬操弓,捕魚玩獸。邇復不憚遠遊,冒寒暑,涉關河,饍飲不調,餚蔌無擇,誠非養生道也。況南方卑濕,尤易致病。乞念宗廟社稷之重,勿事鞍馬,勿過醉飽,喜無傷心,怒無傷肝,慾無傷腎,勞無傷脾,就密室之安,違暴風之禍。臣不勝至願。」諸疏既入,帝與諸倖臣皆大怒,遂下良勝、潮、九川、鞏、震、鏊詔獄,芬及衍瑞等百有七人罰跪午門外五日。而大理寺正周敘等十人,行人司副餘廷瓚等二十人,工部主事林大輅、何遵、蔣山卿連名疏相繼上。帝益怒,並下詔獄。俄令敘、廷瓚、大輅等,與良勝等六人,俱跪闕下五日,加梏鋋焉。至晚,仍繫獄。諸臣晨入暮出,纍纍若重囚,道途觀者無不泣下。而廷臣自大學士楊廷和、戶部尚書石玠疏救外,莫有言者。士民咸憤,爭擲瓦礫詬詈之。諸大臣皆恐,入朝不待辨色,請下詔禁言事者,通政司遂格不受疏。

是時,天連曀晝晦,禁苑南海子水涌四尺餘,橋下七鐵柱皆折如斬。金吾衛都指揮僉事張英曰:「此變徵也,駕出必不利。」乃肉袒戟刃於胸,囊土數升,持諫疏當蹕道跪哭,即自刺其胸,血流滿地。衛士奪其刃,縛送詔獄。問囊土何為?曰:「恐污帝廷,灑土掩血耳。」詔杖之八十,遂死。

芬等百有七人,跪既畢,杖各三十。以芬、衍瑞、俸、龍、鳳為倡首,謫於外。余奪俸半歲。良勝等六人及敘、廷瓚、大輅各杖五十,餘三十人四十。鞏、震、良勝、潮、九川除名。他貶黜有差。鏊戍邊。而車駕亦不復出矣。

良勝既歸,講授生徒。世宗立,召復故官。尚書喬宇賢之,奏為文選郎中,公廉多所振拔。「大禮」議起,數偕僚長力爭。及席書、張璁、桂萼、方獻夫用中旨超擢,又執不可。由是為議禮者所切齒。以久次遷南京太常少卿,未赴,外轉。給事中陳洸上書,傅會張璁等議,斥良勝與尚書宇等群結朋黨,任情擠排。遂謫良勝茶陵知州。及《明倫大典》成,詔責前郎中良勝脅持庶官,釀禍特深,黜為民。初,良勝輯其部中章奏,名曰《銓司存稿》,凡議禮諸疏具在。為仇家所發,再下獄。論杖當贖,特旨謫戍遼東三萬衛。踰五年,卒於戍所。穆宗立,贈太常卿。舒芬等自有傳。

萬潮,字汝信,進賢人。正德六年進士。由寧國推官入為儀制主事,與芬、良勝、九川稱「江西四諫」。世宗立,起故官,歷浙江提學副使。久之遷參政,以忤權貴調廣西。屢遷陜西左布政使、右副都御史巡撫延綏,所至著聲。

陳九川,字惟濬,臨川人。正德九年進士。從王守仁遊。尋授太常博士。既削籍,復從守仁卒業。世宗嗣位,召復故官,再遷主客郎中。正貢獻名物,節貢使犒賞費數萬。會天方國貢玉石,九川簡去其不堪者。所求蟒衣,不為奏覆,復怒罵通事胡士紳等。士紳恚,假番人詞訐九川及會同館主事陳邦偁。帝怒,下二人詔獄。而是時張璁、桂萼欲傾費宏奪其位,乃屬士紳再訐九川盜貢玉饋宏製帶,詞連兵部郎中張、錦衣指揮張潮等。帝益怒,并下等詔獄。指揮駱安請攝士紳質訊,給事中解一貫等亦以為言,帝不許。獄成,九川戍鎮海衛,邦偁等削籍有差。久之,遇赦放還,卒。

張衍瑞,字元承,汲人。弘治十八年進士。為清豐知縣。以執法忤劉瑾,逮下詔獄,幾死。瑾誅,得釋,官吏部文選郎中。既杖,謫平陽同知。嘉靖初,召還,擢太常少卿。尋卒,贈太僕卿。

姜龍,太倉人,見父《昂傳》。孫鳳,洛陽人。陸俸,吳縣人。周敘,九谿衛人。林大輅,莆田人。蔣山卿,儀真人。皆由進士。山卿遊顧璘門,以詩名於時。既杖,鳳、俸並謫府同知,敘縣丞,大輅州判官,山卿前府都事。世宗立,悉召復故官。鳳終副使,俸知府,敘工部尚書,大輅右副都御史巡撫湖廣,山卿廣西參政。

徐鏊,嘉定人,本高氏子。少孤,依舅京師,冒徐姓,從其業為醫,供事內殿。既杖,謫戍烏撒。世宗即位,召還,尋擢御醫。鏊性耿介,時朝士多新貴,不知鏊,鏊亦不言前事。一官垂三十年不調。年七十,求致仕。值同縣徐學謨為禮部郎中,引見尚書吳山。山閱牘,有諫南巡事,瞿然曰:「此武廟時徐先生耶?何淹也!」兩侍郎嫌其老,學謨抗聲曰:「鏊雖老,然少與舒狀元同患難,為可敬耳。」又久之,始遷院判。自引歸,卒年八十三。

時同受杖者,吏部則姚繼巖,行人則陶滋、巴思明、李錫、顧可久、鄧顯麒、熊榮、楊秦、王懋、黃國用、李儼、潘銳、劉黻、張岳,大理寺則寺正金罍,寺副孟庭柯、張士鎬、郝鳳升、傅尚文、郭五常,評事姚如皋、蔡時,並謫官。世宗立,召還。張英亦得贈官予祭,授弟雄都指揮僉事。

姚繼巖,南通州人,張衍瑞同年生也。當遷文選郎中,讓衍瑞。嘉靖初,歷太常少卿,伏闕爭「大禮」。甘貧約,遠權勢。及卒,不能成喪。

何遵,字孟循,江寧人。家貧,父命之賈,不願也,去為儒。舉正德九年進士。吏部尚書陸完聞其名,使子弟從學。及選臺諫,遵引疾曰:「不可因人進也。」授工部主事,榷木荊州。下令稅自百金以下減三之一,風濤敗貲者勿算。入算者手實其數自識之,藏於郡帑,數日一會所入。比去,不私一錢。

帝將南巡,以進香東嶽為詞。遵抗言:「淫祠無福。萬一宗籓中借口奉迎,潛懷不軌,則福未降而禍已隨。」蓋指宸濠也。諸權倖見疏,遏勿進。時黃鞏等已得罪,遵復與同官林大輅、蔣山卿上疏乞罷南巡,極言江彬怙權倡亂。鞏等無罪,願特寬宥,毋使後世有殺諫臣名。帝怒,下詔獄,廷杖四十。創甚,肢體俱裂,越二日遂卒,年三十四。家貧,僚友助而殮之。

當遵草疏時,家僮前,抱持哭曰:「主縱不自計,獨不念老親幼子乎?」遵執筆從容曰:「:為我謝大人,兒子勿令廢學足矣。」死之日,其父方與家人祭墓歸,有鳥悲鳴,心異之。或傳工部有以言獲罪者,父長號曰:「遵死矣!」已而果然。

時先遵受杖死者,刑部主事郾城劉校、照磨汲人劉玨。與遵同死杖下者,陸震而外,大理評事長樂林公黼,行人司副鄱陽餘廷瓚,行人盱眙李紹賢、澤州孟陽、玉山詹軾、安陸劉概、祥符李惠。

劉校,字宗道。性至孝。母胡教子嚴,偶不悅,輒長跪請罪,母悅乃起。正德六年與詹軾、劉概同舉進士,授刑部主事。迎父就養,卒於途。校奔赴,抱屍痛哭幾絕。面有塵,以舌舐而拭之。及起故官,帝將南巡,刑曹諫疏,校所草也。杖將死,大呼曰:「校無恨,恨不見老母耳!」子元婁,年十一,哭於旁。校曰:「爾讀書不多,獨不識事君致身義乎?善事祖母及母,毋媿而父。」遂絕。劉玨,由貢士。

林公黼,字質夫。父母喪,三年蔬粥,不入內。正德十二年與李紹賢、李惠同舉進士。諸曹諫南巡者,皆罰跪闕前,諸奸又日以危言恫喝,聞者惴惴。以故,戶曹不敢出疏,工曹諫者止三人。獨大理闔署諫,故帝怒加甚。公黼夜草疏,時聞暗中泣歎聲,不顧。比入獄,黃鞏與語,歎曰:「吾取友遍天下,乃近遺質夫。古人謂入險不驚,殆斯人乎!」公黼體羸,竟不勝杖而卒。

餘廷瓚,字伯獻。與孟陽皆正德九年進士。當禮、兵二曹之進諫也,廷瓚亦率其僚陳巡遊十不可,通政司獨留之。居數日,諸曹已罰跪,疏始上。帝愈怒,掠治尤嚴。

李紹賢,字崇德。嘗頒詔至徐州,監倉中使席班首,紹賢立命撤其席,中使愕然去。比逮繫,見中官猶奴視之。

孟陽,字子乾。吏部侍郎春之子。為行人,久不遷,或諷之見當路,陽不可。及是,語諸僚:「此舉繫社稷安危,一命之士皆與有憂,豈必言官乃當效死?」父春,前巡撫宣府,有軍功,忤中官張永罷歸。聞子死諫,哭之以詩,語甚悲壯,人爭傳之。

詹軾,字敬之。為人開爽磊落,善談論。從父瀚,字汝約,與公黼同舉進士。時方為刑部主事,亦以諫受杖。軾死,為經紀其喪以歸。嘉靖中,瀚爭「大禮」,再受杖。每陰雨創痛,曰:「吾無愧敬之地下,足矣。」積官刑部侍郎。

劉概,字平甫。李惠,字德卿,尚書鉞之子。世宗立,贈遵、校尚寶卿,玨刑部主事,公黼、廷瓚太常丞,紹賢御史。各賜祭,錄一子入國學。

其以創死稍後者,禮部員外郎慈谿馮涇,驗封郎中吳江王鑾,行人昌黎王瀚。

馮涇,字伯清,與瀚皆正德九年進士。涇以孝友稱。既卒,家貧不能還喪。世宗立,吏部以狀聞,賜米二十斛,命有司厚恤其家。

王鑾,字汝和。正德六年進士。試政吏部,為尚書楊一清所知,擢文選主事。朝夕扃戶,人罕得見。再遷驗封郎中。被創,踰年卒。王瀚亦前卒。世宗立,贈御史,賜祭。

當諸曹連章迭諫,江彬怒甚。陰屬典詔獄者重其杖,以故諸臣多死。哭聲徹禁掖,帝亦為感動,竟罷南巡,諸臣之力也。

嘉靖初,主事仵瑜上疏曰:「正德間,給事、御史挾勢凌人,趨權擇便,凡朝廷大闕失,群臣大奸惡,緘口不言。一時犯顏敢諍,視死如歸,或拷死闕廷,或流竄邊塞,皆郎中、員外、主事、評事、行人、照磨、庶吉士,非有言責者。張英本一武夫,抗言就死,行道悲傷。今幸聖皇御極,褒恤忠良,諸給事、御史更何顏復立清明之朝?請加黜罰,以示創懲。」章下吏部。瑜後以爭「大禮」杖死,自有傳。

贊曰:李文祥、孫磐甫釋褐觀政,未列庶位;胡爟以下率諸曹尚書郎,或冗散卑末。非司風憲,當言路,以諫諍為盡職也。抗言極論,竄謫接踵,而來者愈多;死相枕籍,而赴蹈恐後。其抵觸權倖,指斥乘輿,皆切於安危之至計。若張英陷胸以悟主,徐鏊託術以諷諭,誠心出於忠愛,抑尤人所難能者矣。


\end{pinyinscope}