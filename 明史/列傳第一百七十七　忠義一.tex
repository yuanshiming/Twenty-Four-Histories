\article{列傳第一百七十七 忠義一}

\begin{pinyinscope}
從古忠臣義士,為國捐生,節炳一時,名垂百世,歷代以來指出神是人幻想的產物,是人按照自己的形象、特性加以美,備極表章,尚已。明太祖創業江左,首褒余闕、福壽,以作忠義之氣。至從龍將士,或功未就而身亡,若豫章、康郎山兩廟及溪籠山功臣廟所祀諸人,爵贈公侯,血食俎豆,侑享太廟,恤錄子孫,所以褒厲精忠,激揚義烈,意至遠也。建文之變,群臣不憚膏鼎鑊,赤姻族,以抗成祖之威棱,雖《表忠》一錄出自傳疑,亦足以知人心天性之不泯矣。仁宣以降,重熙累洽,垂二百餘載,中間如交阯之徵,土木之變,宸濠之叛,以暨神、熹兩朝,邊陲多故,沉身殉難者,未易更僕數。而司勳褒恤之典,悉從優厚。或所司失奏,後人得自陳請。故節烈之績,咸得顯暴於時。迨莊烈之朝,運丁陽九,時則內外諸臣,或隕首封疆,或致命闕下,蹈死如歸者尤眾。今就有明一代死義死事之臣,博採旁搜,匯次如左。同死者,各因事附見。其事實繁多及國家興亡所繫,或連屬他傳,本末始著,與夫直諫死忠,疏草傳誦人口,概具前帙。至若抒忠勝國,抗命興朝,稽諸前史,例得並書。我太祖、太宗忠厚開基,扶植名教,獎張銓之守義,釋張春而加禮,洪量同天地,大義懸日月,國史所載,煥若丹青。諸臣之遂志成仁,斯為無忝,故備列焉。

○花云硃文遜許瑗等王愷孫炎王道同朱文剛牟魯裴源朱顯忠王均諒等王綱子彥達王禕吳雲熊鼎易紹宗琴彭陳汝石等皇甫斌子弼吳貴等張瑛熊尚初等王禎萬琛王祐周憲子幹楊忠李睿等吳景王源馮傑孫璽等霍恩段豸張汝舟等孫燧許逵黃宏馬思聰宋以方萬木鄭山趙楠等

花雲,懷遠人。貌偉而黑,驍勇絕倫。至正十三年癸巳,杖劍謁太祖於臨濠。奇其才,俾將兵略地,所至輒克。破懷遠,擒其帥。攻全椒,拔之。襲繆家寨,群寇散走。太祖將取滁州,率數騎前行,雲從。猝遇賊數千,雲舉鈹翼太祖,拔劍躍馬衝陣而進。賊驚曰:「此黑將軍勇甚,不可當其鋒。」兵至,遂克滁州。甲午從取和州,獲卒三百,以功授管勾。乙未,太祖渡江,雲先濟。既克太平,以忠勇宿衛左右。從下集慶,獲卒三千,擢總管。徇鎮江、丹陽、丹徒、金壇,皆克之。過馬馱沙,劇盜數百遮道索戰。雲且行且鬥三日夜,皆擒殺之,授前部先鋒。從拔常州,守牛塘營。太祖立行樞密院於太平,擢雲院判。丁酉克常熟,獲卒萬餘。命趨寧國,兵陷山澤中八日,群盜相結梗道。雲操矛鼓噪出入,斬首千百計,身不中一矢。還駐太平。庚子閏五月,陳友諒以舟師來寇。雲與元帥朱文遜、知府許瑗、院判王鼎結陣迎戰,文遜戰死。賊攻三日不得入,以巨舟乘漲,緣舟尾攀堞而上。城陷,賊縛雲。雲奮身大呼,縛盡裂,起奪守者刀,殺五六人,罵曰:「賊非吾主敵,盍趣降!」賊怒,碎其首,縛諸檣叢射之,罵賊不少變,至死聲猶壯,年三十有九。瑗、鼎亦抗罵死。太祖即吳王位,追封雲東丘郡侯,瑗高陽郡侯,鼎太原郡侯,立忠臣祠,並祀之。

方戰急,雲妻郜祭家廟,挈三歲兒,泣語家人曰:「城破,吾夫必死,吾義不獨存,然不可使花氏無後,若等善撫之。」雲被執,郜赴水死。侍兒孫瘞畢,抱兒行,被掠至九江。孫夜投漁家,脫簪珥屬養之。及漢兵敗,孫復竊兒走渡江,遇僨軍奪舟棄江中,浮斷木入葦洲,採蓮實哺兒,七日不死。夜半,有老父雷老挈之行,踰年達太祖所。孫抱兒拜泣,太祖亦泣,置兒膝上,曰:「將種也。」賜雷老衣,忽不見。賜兒名煒,累官水軍衛指揮僉事。其五世孫為遼復州衛指揮,請於世宗,贈郜貞烈夫人,孫安人,立祠致祭。

文遜者,太祖養子也。嘗與元帥秦友諒攻克無為州。瑗,字栗夫,樂平人。元末,兩舉鄉第一。太祖駐婺州,瑗謁曰:「足下欲定天下,非延攬英雄,難以成功。」太祖喜,置幕中,參軍事。已,命守太平。鼎,儀征人。初為趙忠養子。忠為總管,克太平,授行樞密院判,鎮池州。趙普勝來寇,忠陣歿。鼎嗣職,復故姓,駐太平。至是,三人皆死之。

時有劉齊者,以江西行省參政守吉安。守將李明道開門納友諒兵,殺參政曾萬中、陳海,執齊及知府宋叔華,脅之降,皆不屈。又破臨安,執同知趙天麟,亦不屈,並送友諒所。友諒方攻洪都,殺三人徇城下。及陷無為州,執知州董曾,曾抗罵不屈,沉之江。

王愷,字用和,當塗人。通經史,為元府吏。太祖拔太平,召為掾。從下京口,撫定新附民。及建中書省,用為都事。杭州苗軍數萬降,待命嚴州境。愷馳諭之,偕其帥至。太祖克衢州,命總制軍民事。愷增城浚濠,置遊擊軍,籍丁壯,得萬餘人。常遇春屯兵金華,部將擾民,愷械而撻諸市。遇春讓愷,愷曰:「民者國之本,撻一部將而民安,將軍所樂聞也。」乃謝愷。時饑疫相仍,愷出倉粟,修惠濟局,全活無算。學校毀,與孔子家廟之在衢者,並新之。設博士弟子員,士翕然悅服。開化馬宣、江山楊明並為亂,先後討擒之。遷左司郎中,佐胡大海治省事。苗軍作亂,害大海。其帥多德愷,欲擁之而西。愷正色曰:「吾守士,議當死,寧從賊邪!」遂並其子行殺之。年四十六。

愷善謀斷,嘗白事,未聽,卻立戶外,抵暮不去。太祖出,怪問之,愷諫如初,卒從其議。後贈奉直大夫、飛騎尉,追封當塗縣男。

孫炎,字伯融,句容人。面鐵色,跛一足。談辨風生,雅負經濟。與丁復、夏煜遊,有詩名。太祖下集慶,召見,請招賢豪成大業。時方建行中書省,用為首掾。從征浙東,授池州同知,進華陽知府,擢行省都事。克處州,授總制。太祖命招劉基、章溢、葉琛等,基不出。炎使再往,基遺以寶劍。炎作詩,以為劍當獻天子,斬不順命者,人臣不敢私,封還之。遺基書數千言,基始就見,送之建康。時城外皆賊,城守無一兵。苗軍作亂,殺院判耿再成,執炎及知府王道同、元帥朱文剛,幽空室,脅降,不屈。賊帥賀仁德燖雁斗酒啖炎,炎且飲且罵。賊怒,拔刀叱解衣,炎曰:「此紫綺裘,主上所賜,吾當服以死。」遂與道同、文剛皆見害,時年四十。追贈丹陽縣男,建像再成祠。

道同,由中書省宣使在處州,贈太原郡侯。

文剛,太祖養子,小字柴舍。變起,欲與再成聚兵殺賊,不及,遂被難。贈鎮國將軍,附祭功臣廟。

牟魯,烏程人,為莒州同知。洪武三年秋,青州民孫古朴為亂,襲州城,執魯欲降之。魯曰:「國家混一海字,民皆樂業。若等悔過自新,可轉禍為福。不然,官軍旦夕至,無遺種矣。我守土臣,義唯一死。」賊不敢害,擁至城南。魯大罵,遂殺之。賊破,詔恤其家。

又有白謙、裴源、朱顯忠、王均諒、王名善、黃里、顧師勝、陳敬、吳得、井孚之屬。

謙,婺源知州。信州盜蕭明來寇,謙力不能禦,懷印出北門,赴水死。

源,肇慶府經歷。以公事赴新興,遇山賊陳勇卿,被執,勒令跪。源大罵曰:「我命官,乃跪賊邪!」遂被殺。洪武三年贈官二等。

顯忠,如皋人。為張士誠將,來降。以指揮僉事從鄧愈下河州,抵吐番。從傅友德克文州,遂留守之。洪武四年,蜀將丁世珍召番數萬來攻。食盡無援,或勸走避,顯忠叱不聽。攻益急,裹創力戰,城破,為亂兵所殺。均諒時為千戶,被執不屈,磔死。事聞,贈恤有差。

名善,義烏人,高州通判。有海寇何均善曾被戮,洪武四年,其黨羅子仁率眾潛入城,執名善,不屈死。

里,雲內州同知。洪武五年秋,蒙古兵突入城。里率兵蒼戰,死之。

師勝,興化人,峨眉知縣。洪武十三年率民兵討賊彭普貴,戰死。詔褒恤。

敬,增城人。洪武十四年舉賢良,為曲靖府經歷,署劍川州事。鄰寇來攻,敬禦之。官兵寡,欲退,敬瞋目大呼,力戰死。命恤其家。

得,全椒人,龍里守禦所千戶。洪武三十年,古州上婆洞蠻作亂,得與鎮守將井孚守城。賊燒門急攻,二人開門奮擊,得中毒弩死,孚戰死。贈得指揮僉事,孚正千戶,子孫世襲。

王綱,字性常,餘姚人。有文武才。善劉基,常語曰:「老夫樂山林,異時得志,勿以世緣累我。」洪武四年以基薦徵至京師,年七十,齒髮神色如少壯。太祖異之,策以治道,擢兵部郎。潮民弗靖,除廣東參議,督兵餉,嘆曰:「吾命盡此矣。」以書訣家人,攜子彥達行,單舸往諭,潮民叩首服罪。還抵增城,遇海寇曹真,截舟羅拜,願得為帥。綱諭以禍福,不從,則奮罵。賊舁之去,為壇坐綱,日拜請。綱罵不絕聲,遂遇害。彥達年十六,罵賊求死,欲並殺之。其酋曰:「父忠子孝,殺之不詳。」與之食,不顧,令綴羊革裹父屍而出。御史郭純以聞,詔立廟死所。彥達以蔭得官,痛父,終身不仕。

王禕,字子充,義烏人。幼敏慧,及長,身長嶽立,屹有偉度。師柳貫、黃溍,遂以文章名世。睹元政衰敝,為書七八千言上時宰。危素、張起巖並薦,不報。隱青巖山,著書,名日盛。太祖取婺州,召見,用為中書省掾史。征江西,禕獻頌。太祖喜曰:「江南有二儒,卿與宋濂耳。學問之博,卿不如濂。才思之雄,濂不如卿。」太祖創禮賢館,李文忠薦禕及許元、王天錫,召置館中。旋授江南儒學提舉司校理,累遷侍禮郎,掌起居注。同知南康府事,多惠政,賜金帶寵之。太祖將即位,召還,議禮。坐事忤旨,出為漳州府通判。

洪武元年八月,上疏言:「祈天永命之要,在忠厚以存心,寬大以為政,法天道,順人心。雷霆霜雪,可暫不可常。浙西既平,科斂當減。」太祖嘉納之,然不能盡從也。明年修《元史》,命禕與濂為總裁。禕史事擅長,裁煩剔穢,力任筆削。書成,擢翰林待制,同知制誥兼國史院編修官。奉詔預教大本堂,經明理達,善開導。召對殿廷,必賜坐,從容宴語。未久,奉使吐蕃,未至,召還。

五年正月議招諭雲南,命禕齎詔往。至則諭梁王,亟宜奉版圖歸職方,不然天討旦夕至。王不聽,館別室。他日,又諭曰:「朝廷以雲南百萬生靈,不欲殲於鋒刃。若恃險遠,抗明命,龍驤鷁艫,會戰昆明,悔無及矣。」梁王駭服,即為改館。會元遣脫脫征餉,脅王以危言,必欲殺禕。王不得已出禕見之,脫脫欲屈禕,禕叱曰:「天既訖汝元命,我朝實代之。汝爝火餘燼,敢與日月爭明邪!且我與汝皆使也,豈為汝屈!」或勸脫脫曰:「王公素負重名,不可害。」脫脫攘臂曰:「今雖孔聖,義不得存。」禕顧王曰:「汝殺我,天兵繼至,汝禍不旋踵矣。」遂遇害,時十二月二十四日也。梁王遣使致祭,具衣冠斂之。建文中,禕子紳訟禕事,詔贈翰林學士,謚文節。正統中,改謚忠文。成化中,命建祠祀之。

紳,字仲縉。禕死時,年十三,鞠於兄綬,事母兄盡孝友。長博學,受業宋濂。濂器之曰:「吾友不亡矣。」蜀獻王聘紳,待以客禮。紳啟王往雲南求父遺骸,不獲即死所致祭,述《滇南慟哭記》以歸。建文帝時,用薦召為國子博士,預修《太祖實錄》,獻《大明鐃歌鼓吹曲》十二章。與方孝孺友善,卒官。

子稌,字叔豐。師方孝孺。孝孺被難,與其友鄭珣輩潛收遣骸,禍幾不測,自是絕意仕進。初,紳痛父亡,食不兼味。稌守之不變,居喪,不飲酒,不食肉者三年,門人私謚曰孝莊先生。

子汶,字允達。成化十四年進士。授中書舍人。謝病歸,讀書齊山下。弘治初,言者交薦,與檢討陳獻章同召,未抵京卒。

禕死雲南之三年,死事者又有吳雲。雲,宜興人。元翰林待制,仕太祖,為湖廣行省參政。洪武八年九月,太祖議再遣使招諭梁王,召雲至,語之曰:「今天下一家,獨雲南未奉正朔,殺我使臣,卿能為我作陸賈乎?」雲頓首請行。時梁王遣鐵知院輩二十餘人使漠北,為大將軍所獲,送京師,太祖釋之,令與雲偕行。既入境,鐵知院等謀曰:「吾輩奉使被執,罪且死。」乃誘雲,令詐為元使,改制書,共紿梁王。雲誓死不從,鐵知院等遂殺雲。梁王聞其事,收雲骨,送蜀給孤寺殯之。

雲子黻,上雲事於朝。詔馳傳返葬,以黻為國子生。弘治四年五月贈雲刑部尚書,謚忠節,與禕並祠,改祠額曰二忠。

熊鼎,字伯潁,臨川人。元末舉於鄉,長龍溪書院。江西寇亂,鼎結鄉兵自守。陳友諒屢脅之,不應。鄧愈鎮江西,數延見,奇其才,薦之。太祖欲官之,以親老辭,乃留愈幕府贊軍事。母喪除,召至京師,授德清縣丞。松江民錢鶴皋反,鄰郡大驚,鼎鎮之以靜。

吳元年召議禮儀,除中書考功博士。遷起居注,承詔搜括故事可懲戒者,書新宮壁間。舍人耿忠使廣信還,奏郡縣官違法狀,帝遣御史廉之。而時已頒赦書,丞相李善長再諫不納,鼎偕給事中尹正進曰:「朝廷布大信於四方,復以細故煩御史,失信,且褻威。」帝默然久之,乃不遣物史。

洪武改元,新設浙江按察司,以鼎為僉事,分部台、溫。台、溫自方氏竊據,偽官捍將二百人,暴橫甚。鼎盡遷之江、淮間,民始安。平陽知州梅鎰坐贓,辨不已,民數百咸訴知州無罪。鼎將聽之,吏白鼎:「釋知州,如故出何?」鼎歎曰:「法以誅罪,吾敢畏譴,誅無罪人乎!」釋鎰,以情聞,報如其奏。寧海民陳德仲支解黎異,異妻屢訴不得直。鼎一日覽牒,有青蛙立案上,鼎曰:「蛙非黎異乎?果異,止勿動。」蛙果勿動,乃逮德仲,鞫實,立正其罪。是秋,山東初定,設按察司,復以鼎為僉事。鼎至,奏罷不職有司數十輩,列部肅清。鼎欲稽官吏利弊,乃令郡縣各置二曆,日書所治訟獄錢粟事,一留郡縣,一上憲府,遞更易,按歷鉤考之,莫敢隱者。尋進副使,徙晉王府右傅。坐累左遷,復授王府參軍,召為刑部主事。

八年,西部朵兒只班率部落內附,改鼎岐寧衛經歷。既至,知寇偽降,密疏論之。帝遣使慰勞,賜裘帽,復遣中使趙成召鼎。鼎既行,寇果叛,脅鼎北還。鼎責以大義,罵之,遂與成及知事杜寅俱被殺。帝聞,悼惜,命葬之黃羊川,立祠,以所食俸給其家。

易紹宗,攸人。洪武時,從軍有功,授象山縣錢倉所千戶。建文三年,倭登岸剽掠。紹宗大書於壁曰:「設將禦敵,設軍衛民。縱敵不忠。棄民不仁。不忠不仁,何以為臣!為臣不職,何以為人!」書畢,命妻李具牲酒生奠之,訣而出,密令遊兵間道焚賊舟。賊驚救,紹宗格戰,追至海岸,陷淖中,手刃數十賊,遂被害。其妻攜孤奏於朝,賜葬祭,勒碑旌之。

琴彭,交阯人。永樂中,以乂安知府署茶籠州事,有善政。宣德元年,黎利反,率眾圍其城。彭拒守七月,糧盡卒疲,諸將無援者,巡按御史飛章請救。宣宗馳敕責榮昌伯陳智等曰:「茶籠守彭被困孤城,矢死無貳,若等不援,將何以逃責!急發兵解圍,無干國憲。」敕未至而城陷,彭死之。詔贈交阯左布政使,送一子京師官之。

時交阯人陳汝石、朱多蒲、陶季容、陳汀亦皆以忠節著。汝石初為陳氏小校,大軍南征,率先歸附,積功至都指揮僉事。永樂十七年,四忙士官車綿子等叛。汝石從方政討之,深入賊陣,中流矢墜馬,與千戶朱多蒲皆死。多蒲,亦交址人。事聞,遣行人賜祭,賻其家,官為置塚。

皇甫斌,壽州人。先為興州右屯衛指揮同知,以才調遼海衛。忠勇有智略,遇警,輒身先士卒。宣德五年十月勒兵禦寇,至密城東峪,自旦及晡力戰,矢盡援絕,子弼以身衛父,俱戰死。千戶吳貴,百戶吳襄、毛觀並驍勇,出必衝鋒,至是皆死。斌等雖死,殺傷過當,寇亦引退。事聞,詔有司褒恤。

張瑛,字彥華,浙江建德人。永樂中,舉於鄉,歷刑部員外郎。正統時,擢建寧知府。鄧茂七作亂,賊二千餘迫城結砦,四出剽掠。瑛率建安典史鄭烈會都指揮徐信軍,分三路襲之,斬首五百餘,遂拔其砦。進右參議,仍知府事。烈亦遷主簿。茂七既誅,其黨林拾得等轉掠城下,瑛與從父敬禦之。賊敗,乘勝逐北,陷伏中,敬死,瑛被執,大罵不屈死。詔贈福建按察使,賜祭,官其子。弘治中,建寧知府劉璵請於朝,立祠致祭。

時泉州守熊尚初亦以拒賊被執死。尚初,南昌人。初為吏,以才擢都察院都事,進經歷。正統中,用都御史陳鎰薦,擢泉州知府。盜起,上官檄尚初監軍,不旬日降賊數百。已而賊逼城下,守將不敢禦。尚初憤,提民兵數百,與晉江主簿史孟常、陰陽訓術楊仕弘分統之,拒於古陵坡。兵敗,皆遇害。郡人哀之,為配享忠臣廟。

王禎,字維禎,吉水人。祖省,死建文難,自有傳。成化初,禎由國子生授夔州通判。二年,荊、襄石和尚流劫至巫山,督盜同知王某者怯不救。禎面數之,即代勒所部民兵,晝夜行。至則城已陷,賊方聚山中。禎擊殺其魁,餘盡遁,乃行縣撫傷殘,招潰散,久乃得還。甫三日,賊復劫大昌。禎趣同知行,不應。指揮曹能、柴成與同知比,激禎曰:「公為國出力,肯復行乎?」禎即請往,兩人偽許相左右。禎上馬,挾二人與俱,夾水陣。既渡,兩人見賊即走。禎被圍半日,誤入淖中,賊執欲降之,禎大罵。賊怒,斷其喉及右臂而死。從行者奉節典史及部卒六百餘人皆死。

自死所至府三百餘里,所乘馬奔歸,血淋漓,毛盡赤。眾始知禎敗,往覓屍,面如生。子廣鬻馬為歸貲,王同知得馬不償直。櫬既行,馬夜半哀鳴。同知起視之,馬驟前齧項,搗其胸,翼日嘔血死,人稱為義馬。事聞,贈禎僉事,錄一子。

萬琛,字廷獻,宣城人。慷慨負氣節,舉於鄉。弘治中,知瑞金縣。十八年正月,劇盜大至,縣人洶洶逃竄。有勸琛急去者,琛斥之,率民兵數十人迎敵,殺賊二十餘人。相持至明日,力屈被執,罵不絕口,賊攢刺之,乃死。贈光祿少卿,賜祭葬,予廕。

時有王祐者,為廣昌知縣,賊至,民盡逃,援兵又不至。祐拔刀自刲其腹曰:「有城不能守,何生為!」左右奔奪其刀。後援兵集,賊稍退。越七日復突至,祐倉皇赴敵,死之。

周憲,安陸人。弘治六年進士。除刑部主事,進員外郎。十七年坐事下詔獄,謫袞州通判。正德初,復故官,歷江西副使。華林、馬腦賊方熾,總督陳金檄憲剿之,平馬腦砦及仙女、雞公嶺諸寨,先後斬獲千餘人。華林賊窘,遣諜者詭言饑困狀。憲信之,移檄會師夾擊。他將多觀望,憲攻北門,三戰,賊稍卻,與子乾先登逼之。賊下木石如雨,軍潰,憲中槍,乾前救,力戰墮崖死。憲創重被執,罵不絕口,賊支解之。事始聞,贈按察使,予祭葬,謚節愍,廕一子,旌乾門曰孝烈。嘉靖二年,江西巡撫盛應期請與黃宏、馬思聰並旌,詔附禮忠烈祠。後從給事中李鐸言,命有司歲給其家米二石,帛二匹。

楊忠,寧夏人。世官中衛指揮,以功進都指揮僉事,廉介有謀勇。正德五年,安化王寘鐇反,其黨丁廣將殺巡撫安惟學,忠在側,罵曰:「賊狗敢犯上邪!」廣怒,殺之,迄死罵益厲。忠同官李睿聞變,馳至寘鐇所。門閉不得入,大罵,為賊所殺。百戶張欽不從逆,走至雷福堡,亦被殺。皆贈官予蔭,表忠、睿曰忠烈之門,欽曰忠節之門。

吳景,南陵人。弘治九年進士。正德中,歷官四川僉事,守江津。重慶人曹弼亡命播州,糾眾寇川南,謀與大盜藍廷瑞合。六年正月逼江津。御史俞緇遁去,屬景及都指揮龐鳳禦之。鳳邀景俱走,景不可,率典史張俊迎擊,手殺三賊,矢被面。急收兵入保,城已陷,大呼曰:「寧殺我,毋殺士民!」賊強之跪,不屈,遂被殺,俊亦死。巡撫林俊上其事,詔贈景副使,賜祭葬,立祠江津,予世蔭。

是月,僉事王源行部川北,會藍廷瑞、鄢本恕等掠通、巴至營山,源率典史鄧俊禦之,皆被殺。贈源副使,蔭其子。源,五臺人,弘治十二年進士。

明年正月,賊麻六兒將逼川東。副使馮傑追擊於蒼溪,俘斬頗眾。日晡,移營鐵山關,賊乘夜衝突,傑死之。贈按察使,賜祭葬,謚恪愍,世廕百戶。

是時,略陽知縣孫璽、劍州判官羅明、梁山主簿時植亦皆死於賊。

璽,字廷信,代州人。舉於鄉,知扶風縣。都御史藍章以略陽漢中要地,舊無城,檄璽往城之。工未畢,賊至,縣令嚴順欲去,璽拔刀斫坐几曰:「欲去者視此!」乃率僚屬堅守,數日城陷,璽被執,大罵不屈,賊臠殺之。順逃去,誣璽俱逃,滋於江,以他人屍斂。璽子啟視,非是,訟於朝。勘得死節狀,贈光祿少卿,賜祭予廕,抵順罪。

明,以吏起家。鄢本恕逼其城,與子介拒守。城陷,父子皆罵賊死。

植,字良材,通許人。由國子生授官,時攝縣事。賊方四等略地,植拒卻之,斬獲數十級。踰月復至,相拒數日,城陷,說之降,不屈。脅取其印,不予,大罵被殺。妻賈聞變即自縊,女九歲,赴火死。明、植皆贈恤如制,而表植妻女為貞烈。

其時,士民冒死殺賊者,有趙趣、徐敬之、雷應通、袁璋之屬。

趣,梁山諸生。賊攻城,同友人黃甲、李鳳、何璟、蕭銳、徐宣、楊茂寬、趙采誓死拒守。城陷,皆死。都物史林俊嘉其義,立祠祀之。

敬之,亦梁山人。眾推為部長,以拒賊陷陣死。

應通,嘉州人。賊衝百丈關,父子七人倡義死戰。被執,俱慷慨就殺。

璋,江南人。素以勇俠聞。巡撫林俊委剿賊所在有功。後為所執,其子襲挺身救之,連殺七賊,亦被執,俱死。襲死三日,兩目猶瞠視其父。林俊表其門曰父子忠節。總制彭澤為勒石城隍廟,祀於忠孝祠。

霍恩,字天錫,易州人。弘治十五年進士。正德中,歷知上蔡縣。六年,賊四起,中原郡邑多殘破。畿內則棗強知縣段豸、大城知縣張汝舟,河南則恩及典史梁逵,西平知縣王佐、主簿李銓,葉縣知縣唐天恩,永城知縣王鼎,裕州同知郁采、都指揮詹濟、鄉官任賢,固始丞曾基,夏邑丞安宣,息縣主簿刑祥,睢寧主簿金聲、丘紳,西華教諭孔環,山東則萊蕪知縣熊驂,萊州衛指揮僉事蔡顯,南畿則靈譬主簿蔣賢,皆抗節死,而恩、佐、採、環死尤烈。

恩與梁逵共守,當賊至時,語妻劉曰:「脫有急,汝若何?」劉願同死,乃築臺廨後,約曰:「見我下城,即賊入矣。」及城陷,恩拔刀下城,劉臺上見之,即縊,未絕,以簪刺心死。恩被執,賊脅之跪。罵曰:「吾此膝肯為賊屈乎!」賊日殺人以懾之,罵益厲。賊以刀抉其口,支解之。逵自經死。

豸,字世高,澤洲人。起家進士。正德中,授兵科都給事中,謫棗強令。賊至,連戰卻之。及城陷,中四矢一鎗,瞋目大呼,殺賊而死,賊屠其城。汝舟官大城時,與主簿李銓迎戰,皆被殺。

佐,字汝弼。潞州舉人,授西平令。手殺賊數十人,矢斃其渠帥。賊忿,急攻三日,佐力屈被執,罵不絕口。賊懸諸竿,殺而支解之。天恩知葉縣,賊至,與父政等七人俱死。鼎知永城,城陷,繫印於肘,端坐待賊,不屈死。

採,字亮之,浙江山陰人,進士。由主事謫教諭,遷裕州同知。與濟、賢共堅守,斬獲多,城陷被執。采罵不輟,賊碎其輔頰而死。濟亦不屈死。賢嘗為御史,方里居,招邑子三千人拒守,罵賊死,一家死者十三人。基為固始丞,被執,使馭馬不從,被害。宣,初授夏邑丞。賊楊虎逼其境,或勸毋往,宣兼程進。抵任七日,賊大至,拒守有功。城陷,死之。祥已致仕,城陷,罵賊死。聲、紳與義士朱用之迎戰死。

環,南宮人。由歲貢生授來安知縣,為劉瑾黨所陷,左遷西華教諭。被執,賊曰:「呼我王,即釋汝。」厲聲曰:「我恨不得碎汝萬段,肯媚汝求活耶!」遂被殺。驂為賊所執,與主簿韓塘俱不屈死。顯與三子淇、英、順俱御盜力戰死。

諸人死節事聞,皆贈官賜祭予廕立祠如制。恩妻劉贈宜人,建忠節坊旌之。天恩、鼎、基、宣、祥諸人,里貫無考。

時有鄭寶,為鬱林州同知,署北流縣事。妖賊李通寶犯北流,寶與子宗珪出戰,皆死。

王振者,為福建黃崎鎮巡檢。海寇大至,率三子臣、朝、實迎戰競日。伏兵起,振被殺,屍僵立。三子救之,臣重傷,朝、實皆死。亦予恤有差。

孫燧,字德成,餘姚人。弘治六年進士。歷刑部主事,再遷郎中。正德中,歷河南右布政使。寧王宸濠有逆謀,結中官幸臣,日夜詗中朝事,幸有變。又劫持群吏,厚餌之,使為己用。惡巡撫王哲不附己,毒之,得疾,踰年死。董傑代哲,僅八月亦死。自是,官其地者惴惴,以得去為幸。代傑者任漢、俞諫,皆歲餘罷歸。燧以才節著治聲,廷臣推之代。

十年十月擢右副都御史,巡撫江西。燧聞命歎曰:「是當死生以之矣。」遣妻子還鄉,獨攜二僮以行。時宸濠逆狀已大露,南昌人洶洶,謂宸濠旦暮得天子。燧左右悉宸濠耳目,燧防察密,左右不得窺,獨時時為宸濠陳說大義,卒不悛。陰察副使許逵忠勇,可屬大事,與之謀。先是,副使胡世寧暴宸濠逆謀,中官幸臣為之地,世寧得罪去。燧念訟言於朝無益,乃托禦他寇預為備。先城進賢,次城南康、瑞州。患建昌縣多盜,割其地,別置安義縣,以漸弭之。而請復饒、撫二州兵備,不得復,則請敕湖東分巡兼理之。九江當湖衝,最要害,請重兵備道權,兼攝南康、寧州、武寧、瑞昌及湖廣興國、通城,以便控制。廣信橫峰、青山諸窯,地險人悍,則請設通判駐弋陽,兼督旁五縣兵。又恐宸濠劫兵器,假討賊,盡出之他所。宸濠瞷燧圖己,使人賂朝中幸臣去燧,而遣燧棗梨姜芥以示意,燧笑卻之。逵勸燧先發後聞,燧曰:「奈何予賊以名,且需之。」

十三年,江西大水,宸濠素所蓄賊凌十一、吳十三、閔念四等出沒鄱陽湖,燧與逵謀捕之。三賊遁沙井,燧自江外掩捕,夜大風雨,不克濟。三賊走匿宸濠祖墓間,於是密疏白其狀,且言宸濠必反。章七上,輒為宸濠遮獄,不得達。宸濠恚甚,因宴毒燧,不死。燧乞致仕,又不許,憂懼甚。

明年,宸濠脅鎮巡官奏其孝行,燧與巡按御史林潮冀藉是少緩其謀,乃共奏於朝。朝議方降旨責燧等,會御史蕭淮盡發宸濠不軌狀,詔重臣宣諭,宸濠聞,遂決計反。

六月乙亥,宸濠生日,宴鎮巡三司。明日,燧及諸大吏入謝,宸濠伏兵左右,大言曰:「孝宗為李廣所誤,抱民間子,我祖宗不血食者十四年。今太后有詔,令我起兵討賊,亦知之乎?」眾相顧愕眙,燧直前曰:「安得此言!請出詔示我。」宸濠曰:「毋多言,我往南京,汝當扈駕。」燧大怒曰:「汝速死耳。天無二日,吾豈從汝為逆哉!」宸濠怒叱燧,燧益怒,急起,不得出。宸濠入內殿,易戎服出,麾兵縛燧。逵奮曰:「汝曹安得辱天子大臣!」因以身翼蔽燧,賊并縛逵。二人且縛且罵,不絕口,賊擊燧,折左臂,與逵同曳出。逵謂燧曰:「我勸公先發者,知有今日故也。」燧、逵同遇害惠民門外。巡按御史王金、布政使梁宸以下,咸稽首呼萬歲。

宸濠遂發兵,偽署三賊為將軍,首遣婁伯徇進賢,為知縣劉源清所斬。招窯賊,賊畏守吏,不敢發。大索兵器於城中,不得,賊多持白梃。伍文定起義兵,設兩人木主於文天祥祠,率吏民哭之。南贛巡撫王守仁與共平賊。諸逋賊走安義,皆見獲,無脫者。人於是益思燧功。

燧生有異質,兩目爍爍,夜有光。死之日,天忽陰慘,烈風驟起凡數日,城中民大恐。走收兩人屍,屍未變,黑雲蔽之,蠅蚋無近者。明年,守臣上其事於朝,未報。世宗即位,贈禮部尚書,謚忠烈,與逵並祀南昌,賜祠名旌忠,各廕一子。燧子堪聞父訃,率兩弟墀、升赴之,會宸濠已擒,扶柩歸。兄弟廬墓蔬食三年,有芝一莖九葩者數本產墓上。服除,以父死難,更墨衰三年,世稱三孝子。

堪,字志健。為諸生,能文,善騎射。既廕錦衣,中武會試第一,擢署指揮同知。善用強弩,教弩卒數千人以備邊。歷都督僉事。事母楊至孝,母年九十餘,歿京師。堪年亦七十,護喪歸,在道,以毀卒。巡按御史趙炳然上堪孝行,得旌。堪子鈺,亦舉武會試,官都督同知。鈺子如津,都督僉事。

墀,字仲泉,以選貢生歷官尚寶卿。升,官尚書。墀孫如游,大學士。如游孫嘉績,僉事。升子,鑨、鑛皆尚書,鋌侍郎,錝太僕卿。鑨子,如法主事,如洵參政。並以文章行誼世其家。升、鑨、鑛、如游、如法、嘉績,事皆別見。

許逵,字汝登,固始人。正德三年進士。長身巨口,猿臂燕頷,沈靜有謀略。授樂陵知縣。六年春,流賊劉七等屠城邑,殺長吏。諸州縣率閉城守,或棄城遁,或遺之芻,粟弓馬乞賊毋攻。逵之官,慨然為戰守計。縣初無城,督民版築,不踰月,城成。令民屋外築牆,牆高過簾,啟圭竇,才容人。家選一壯者執刃伺竇內,餘皆入隊伍,日視旗為號,違者軍法從事。又募死士伏巷中,洞開城門。賊果至,旗舉伏發,竇中人皆出,賊大驚竄,斬獲無遺。後數犯,數卻之,遂相戒不敢近。事聞,進秩二等。

時知縣能抗賊者,益都則牛鸞,郯城則唐龍,汶上則左經,浚則陳滯,亦所當賊少。而逵屢禦大賊有功,遂與鸞俱超擢兵備僉事。逵駐武定州,州城圮濠平,不能限牛馬。逵築城鑿池,設樓櫓,置巡卒。明年五月,賊楊寡婦以千騎犯濰縣,指揮喬剛禦之,賊少卻。逵追敗之高苑,令指揮張勛邀之滄州,先後俘斬二百七十餘餘人。未幾,賊別部掠德平,逵盡殲之,咸名大著。十二年遷江西副使。時宸濠黨暴橫,逵以法痛繩之。嘗言於孫燧曰:「寧王敢為暴者,恃權臣也。權臣左右之者,貪重賄也。重賄由於盜藪,今惟翦盜則賄息,賄息則黨孤。」燧深然之,每事輒與密議。及宸濠縛燧,逵爭之。宸濠素忌逵,問許副使何言,逵曰:「副使惟赤心耳。」宸濠怒曰:「我不能殺汝邪?」逵罵曰:「汝能殺我,天子能殺汝。汝反賊,萬段磔汝,汝悔何及!」宸濠大怒,並縛之,曳出斫其頸,屹不動。賊眾共推抑令跪,卒不能,遂死,年三十六。

初,逵以文天祥集貽其友給事中張漢卿而無書。漢卿語人曰:「寧邸必反,汝登其為文山乎?」逵父家居,聞江西有變,殺都御史及副使,即為位,易服哭。人怪問故。父曰:「副使,必吾兒也。」世宗即位,贈左副都御史,謚忠節,蔭一子。又錄山東平賊功,復廕一子。嘉靖元年詔逵死事尤烈,改贈逵禮部尚書,進廕指揮僉事。

長子易,好學有器識。既葬父,日夜號泣,六年而後就廕。人或趣之,易曰:「吾父死,易乃因得官。」痛哭不能仰視。易子安阜,事親孝。隆慶中舉於鄉,數試禮部不第。有試官與瑒婚姻,慕安阜才,欲收羅之。安阜曰:「若此,何以見先忠節地下?」許氏子孫不如孫氏貴顯,亦能傳其家。

黃宏,字德裕,鄞人。弘治十五年進士。知萬安縣。民好訟,訟輒禱於神,宏毀其祠曰:「令在,何禱也。」訟者至,輒片言折之。累遷江西左參議,按湖西、嶺北二道。王守仁討橫水、桶岡賊,宏主餉有功。賊閔念四既降,復恃宸濠勢,剽九江上下。宏發兵捕之,走匿宸濠祖墓中,盡得其輜重以歸。宸濠逆節益露,士大夫以為憂,宏正色曰:「國家不幸有此,我輩守士,死而已。」有持大義不從宸濠黨者,宏每陰左右之。宸濠反,宏被執,憤怒,以手梏向柱擊項,是夕卒,賊義而棺斂之。子紹文奔赴,求得其棺,以偽命治斂,非父志,亟易之,扶歸。

時主事馬思聰亦抗節死。思聰,字懋聞,莆田人。弘治末舉進士,為象山知縣,復二十六渠,溉田萬頃。累遷南京戶部主事,督糧江西,駐安仁。值宸濠反,被執繫獄,不屈,絕食六日死。

世宗立,贈宏太常少卿,思聰光祿少卿,並配享旌忠祠。時有謂宏、思聰死節非真者。給事中毛玉勘江西逆黨,復請表章宏、思聰及承奉周儀,而宏子紹武訴於朝。巡按御史穆相列上二人死節狀甚悉,遂無異議。

宋以方,字義卿,靖州人。弘治十八年進士。歷戶部郎中。正德十年遷瑞州知府。時華林大盜甫平,瘡痍未復,以方悉心撫字,吏民愛之。宸濠逆謀萌,而瑞故無城郭,以方築城繕守具,募兵三千,日夕訓練。宸濠深忌之,有徵索又不應,遂迫鎮守劾繫南昌獄。明日,宸濠反,出以方,脅之降,不可,械舟中。至安慶,兵敗,問地何名,舟子云「黃石磯」,江西人音,則「王失機」也。宸濠以為不祥,斬以方祭江。後賊平,其子崇學求遺骸不得,斂衣冠歸葬。嘉靖六年,巡撫陳洪謨上其事,詔贈光祿卿,廕一子,立祠瑞州。

方宸濠之謀為變也,江西士民受害者不可勝紀。初遣閹校四出,籍民田廬,收縛豪強不附者。有萬木、鄭山,俱新建人,集鄉人結砦自固。賊黨謝重一馳入村,二人執之,積葦張睢陽廟前,縛人馬,生焚之,濠黨不敢犯。二人飲江上,為盜凌十一所逼,趣見宸濠,烙而椎之,皆罵賊死。

趙楠,南昌諸生。兄模,嘗捐粟佐振。宸濠捕模索金,楠代往,脅之,不屈,被掠死。同邑辜增見迫,抗節不從,一家百口皆死。諸生劉世倫、儒士陳經官、義士李廣源,皆被掠,不屈死。

葉景恩者,以俠聞,族居吳城。宸濠將作難,捕景恩,脅降之,不從,死獄中。宸濠兵過吳城,景恩弟景允以三百人邀擊賊。賊分兵焚劫景允家,其族景集、景修等四十九人皆死。

又有閻順者,為寧府典寶副。宸濠將反,順與典膳正陳宣、內使劉良微言不可,為典寶正塗欽所譖,三人懼誅,潛詣京師上變。群小庇宸濠,下之獄,搒掠備至。宸濠聞三人赴都,慮事洩,誣奏其罪,且嗾群小必殺之,會已遣戍孝陵,乃免。世宗立,復官。


\end{pinyinscope}