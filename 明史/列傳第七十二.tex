\article{列傳第七十二}

\begin{pinyinscope}
○周洪謨楊守陳弟守阯子茂元茂仁張元禎陳音傅瀚張昇吳寬傅珪劉春吳儼顧清劉瑞

周洪謨,字堯弼,長寧人。正統十年,進士及第。授編修。博聞強記,善文詞,熟國朝典故,喜談經濟。

景泰元年,疏勸帝親經筵,勤聽政,因陳時務十二事。再遷侍讀。天順二年掌南院事。憲宗嗣位,復陳時務,言人君保國之道有三:曰力聖學,曰修內治,曰攘外侮。力聖學之目一:曰正心。修內治之目五:曰求真才,去不肖,旌忠良,罷冗職,恤漕運。攘外侮之目六:曰選將帥,練士卒,講陳法,治兵器,足饋餉,靖邊陲。帝嘉納焉。

成化改元,廷議討四川山都掌蠻,洪謨上方略六事,詔付軍帥行之。進學士。尋為南京祭酒。上言南監有紅板倉二十間,高皇后積粟以養監生妻孥者,宜修復。帝允行之。母喪服闋,改北監。十一年,言士風澆浮,請復洪武中學規。帝嘉納,命禮部榜諭。崇信伯費淮入監習禮,久不至。洪謨劾之,奪冠帶,以儒巾赴監,停歲祿之半,學政肅然。先聖像用冕旒十二,而舞佾豆籩數不稱,洪謨請備天子制。又言:「古者鳴球琴瑟為堂上之樂,笙鏞柷敔為堂下之樂,而干羽則舞於兩階。今舞羽居上,樂器居下,非古制,當改。」尚書鄒乾駁止之,洪謨再疏爭。帝竟俞其議。

遷禮部右侍郎。久之,轉左。以蔡《傳》所釋璇璣玉衡,後人遵用其制,考驗多不合,宜改製,帝即屬洪謨。洪謨易以木,旬日而就。十七年進尚書。二十年加太子少保。二十一年,星變,有所條奏,帝多採納。

弘治元年四月,天壽山震雷風雹,樓殿瓦獸多毀。洪謨復力勸修省,帝深納之。洪謨矜莊寡合,與萬安同鄉,安居政府時頗與之善。至是,言官先後論奏,致仕歸。又三年卒,年七十二。謚文安。

洪謨嘗言:「士人出仕,或去鄉數千里,既昧土俗,亦拂人情,不若就近選除。王府官終身不遷,乖祖制,當稍變更。都掌蠻及白羅羅羿子數叛,宜特設長官司,就擇其人任之,庶無後患。」將歿,猶上安中國、定四裔十事。其好建白如此。

楊守陳,字維新,鄞人。祖範,有學行,嘗誨守陳以精思實踐之學。舉景泰二年進士,改庶吉士,授編修。成化初,充經筵講官,進侍講。《英宗實錄》成,遷洗馬。尋進侍講學士,同修《宋元通鑑綱目》。母憂服闋,起故官。孝宗出閣,為東宮講官。時編《文華大訓》,事涉宦官者皆不錄。守陳以為非,備列其善惡得失。書成,進少詹事。

孝宗嗣位,宮僚悉遷秩,執政擬守陳南京吏部右侍郎,帝舉筆去「南京」字。左右言劉宣見為右侍郎,帝乃改宣左,而以守陳代之。修《憲宗實錄》,充副總裁。弘治改元正月,上疏曰:

孟子言「我非堯舜之道不敢陳於王前。」夫堯舜之道何道?《書》曰:「人心惟危,道心惟微,惟精惟一,允執厥中」,此堯、舜之得於內者深,而為出治之本也。詢四岳,闢四門,明四目,達四聰,此堯、舜之資於外者博,而為致治之綱也。臣昔忝宮僚,伏睹陛下朗讀經書,未嘗勤睿問以究聖賢奧旨。儒臣略陳訓詁,未嘗進詳說以極帝王要道。是陛下得於內者未深也。今視朝,所接見者,大臣之豐采而已。君子、小人之情狀,小臣、遠臣之才行,何由識?退朝所披閱者,百官之章奏而已。諸司之典則,群吏之情弊,何由見?宮中所聽信者,內臣之語言而已。百官之正議,萬姓之繁言,何由聞?恐陛下資於外者未博也。

願遵祖宗舊制,開大小經筵,日再御朝。大經筵及早朝,但如舊儀。若小經筵,必擇端方博雅之臣,更番進講。凡所未明,輒賜清問。凡聖賢經旨,帝王大道,以及人臣賢否,政事得失,民情休戚,必講之明而無疑,乃可行之篤而無弊。若夫前朝經籍,祖宗典訓,百官章奏,皆當貯文華殿後,陛下退朝披覽。日令內閣一人、講官二人居前殿右廂,有疑則詢,必洞晰而後已。一日之間,居文華殿之時多,處乾清宮之時少,則欲寡心清,臨政不惑,得於內者深而出治之本立矣。午朝則御文華門,大臣臺諫更番侍直。事已具疏者用揭帖,略節口奏,陛下詳問而裁決之。在外文武官來覲,俾條列地方事,口陳大要,付諸司評議。其陛辭赴任者,隨其職任而戒諭之。有大政則御文華殿,使大臣各盡其謀,勿相推避。不當則許言官駁正。其他具疏進者,召閣臣面議可否,然後批答。而於奏事、辭朝諸臣,必降詞色,詳詢博訪,務竭下情,使賢才常接於目前,視聽不偏於左右,合天下之耳目以為聰明,則資於外者博而致治之綱舉矣。

若如經筵、常朝只循故事,凡百章奏皆付內臣調旨批答,臣恐積弊未革,後患滋深。且今積弊不可勝數。官鮮廉恥之風,士多浮競之習。教化凌夷,刑禁馳懈。俗侈而財滋乏,民困而盜日繁。列衛之城池不修,諸郡之倉庫鮮積。甲兵朽鈍,行伍空虛。將驕惰而不知兵,士疲弱而不習戰。一或有警,何以禦之?此臣所以朝夕憂思,至或廢寢忘食者也。

帝深嘉納。後果復午朝,召大臣面議政事,皆自守陳發之。尋以史事繁,乞解部務。章三上,乃以本官兼詹事府,專事史館。二年卒。謚文懿,贈禮部尚書。

弟守阯。子茂元、茂仁。守阯,字維立。成化初,鄉試第一,入國學。祭酒邢讓下獄,率六館生伏闕訟冤。十四年,進士及第。授編修。秩滿,故事無遷留都者。會從兄守隨為李孜省所逐,欲并逐守阯,乃以為南京侍讀。

弘治初,召修《憲宗實錄》,直經筵,再遷侍講學士。給事中龐泮等以救知州劉遜悉下獄,吏部尚書屠滽奏遣他官攝之。守阯貽書,極詆滽失。十年大計京官。守阯時掌院事,言:「臣與掌詹事府學士王鏊,俱當聽部考察。但臣等各有屬員。進與吏部會考所屬,則坐堂上,退而聽考,又當候階下。我朝優假學士,慶成侍宴,班四品上,車駕臨雍,坐彞倫堂內,視三品,此故事也。今四品不與考察,則學士亦不應與。臣等職講讀擇述,稱否在聖鑒,有不待考察者。」詔可。學士不與考察,自守阯始。修《會典》,充副總裁。尋遷南京吏部右侍郎。嘗署兵部,陳時弊五事。改署國子監。考績入都,《會典》猶未成,仍留為總裁。事竣,遷左侍郎還任,進二秩。武宗立,引年乞休,不待報竟歸,詔加尚書致仕。劉瑾亂政,奪其加官,瑾敗乃復,久之卒。

守阯博極群書,師事兄守陳,學行相埒。其為解元、學士、侍郎,皆與兄同。又對掌兩京翰林院,人尤艷稱之。守陳卒,守阯為位哭奠者三年。

茂元,字志仁。成化十一年進士。授刑部主事。歷郎中,出為湖廣副使,改山東。弘治七年,河決張秋,詔都御史劉大夏治之,復遣中官李興、平江伯陳銳繼往。興威虐,縶辱按察使。茂元攝司事,奏言:「治河之役,官多而責不專。有司供億,日費百金。諸臣初祭河,天色陰晦,帛不能燃。所焚之餘,宛然人面,具耳目口鼻,觀者駭異。鬼神示怪,夫豈偶然?乞召還興、銳等,專委大夏,功必可成。且水者陰象,今后戚家威權太盛,假名姓肆貪暴者,不可勝數,請加禁防,以消變異。畫工、藝士,宜悉放遣。山東既有內臣鎮守,復令李全鎮臨清,宜撤還。」疏入,下山東撫、按勘,奏言:「焚帛之異誠有之,所奏供億,多過其實。」於是興、銳連章劾茂元妄,詔遣錦衣百戶胡節逮之。父老遮道愬節,乞還楊副使。及陛見,茂元長跪不伏,帝怒,置之詔獄。節遍叩中官,備言父老愬冤狀,中官多感動。會言者交論救,部擬贖杖還職,特謫長沙同知。謝病歸。久之,起安慶知府,遷廣西左參政。正德四年,劉瑾遣御史孫迪校勘錢穀,索賄不予。瑾又惡茂元從父守隨,遂勒致仕。瑾誅,起官江西,尋遷雲南左布政使。以右副都御史巡撫貴州,改蒞南京都察院,終刑部右侍郎。

茂仁,字志道,成化末進士。歷刑部郎中。遼東鎮守中官梁巳被劾,偕給事中往按,盡發其罪。終四川按察使。

張元禎,字廷祥,南昌人。五歲能詩,寧靖王召見,命名元征。巡撫韓雍器之曰「人瑞也」,乃易元禎。舉天順四年進士,改庶吉士,授編修。

憲宗嗣位,疏請行三年喪,不省。其年五月,疏陳三事:「一,勤講學。願不廢寒暑,所講必切於修德為治之實,不必以亂亡忌觸為諱。講退,更凝神靜味,驗之於身心政化。講官,令大臣公舉剛明正大之人,不拘官職大小。一,公聽政。請日御文華殿,午前進講,午後聽政。天下章奏,命諸臣詳議面陳可否,陛下親臨決其是非。暇則召五品以下官,隨意問以時事得失利病,令下情得以畢達。一,廣用賢。請命給事中、御史,各陳兩京堂上官賢否。如有不盡,亦許在京五品官指陳之,以為進退。又令共薦有德望者,以代所去之位,則大臣皆得其人。於是命之各言其所屬及方面郡縣官之賢否,付內閣吏部升黜之。中外群臣,有剛正改言者,舉為臺諫,不必論其言貌、官職、出身。但不宜委之堂上官,恐憚其剛方,而薦柔媚者以充數,所舉之人感其推薦,不敢直斥其非。是以古者大臣不舉臺諫。」疏入,以言多窒礙難行,寢之。預修《英宗實錄》,與執政議不合,引疾家居,講求性命之學。閱二十年,中外交薦,皆不赴。

弘治初,召修《憲宗實錄》,進左贊善。上言:「人君不以行王道為心,非大有為之主也。陛下毓德青宮,已負大有為之望。邇者頗崇異端,嬖近習,以蠱此心;殖貨利,耽玩好,以荒此心;開倖門,塞言路,以昧此心。則不能大有為矣。願定聖志,一聖學,廣聖智。」疏反復累萬言,帝頗納之。《實錄》成,遷南京侍講學士,以養母歸。久之,召為《會典》副總裁。至則進學士,充經筵日講官,帝甚傾向。元禎體清臒,長不踰中人,帝特設低几聽之。數月,以母憂去。服闋,遷南京太常卿。已,修《通鑑纂要》,復召為副總裁。以故官兼學士,改掌詹事府。帝晚年德益進。元禎因請講筵增講《太極圖》、《通書》、《西銘》諸書。帝亟取觀之,喜曰:「天生斯人,以開朕也。」欲大用之,未幾晏駕。

武宗立,擢吏部左侍郎兼學士入東閣,專典誥敕。元禎素有盛譽。林居久,晚乃復出。館閣諸人悉後輩,見元禎言論意態,以為迂闊,多姍笑之。又名位相軋,遂騰謗議,言官交章劾元禎。元禎七疏乞休,劉健力保持之。健去,元禎亦卒。天啟初,追謚文裕。

陳音,字師召,莆田人。天順末進士。改庶吉士,授編修。成化六年三月,以災異陳時政,言:「講學莫先於好問。陛下雖間御經筵,然勢分嚴絕,上有疑未嘗問,下有見不敢陳。願引儒臣賜坐便殿,從容咨論,仰發聖聰。異端者,正道之反,法王、佛子、真人,宜一切罷遣。」章下禮部。越數日,又奏:「國家養士百年,求其可用,不可多得。如致仕尚書李秉,在籍修撰羅倫、編修張元禎、新會舉人陳獻章皆當世人望,宜召還秉等,而置獻章臺諫。言官多緘默,願召還判官王徽、評事章懋等,以開言路。」忤旨切責。

司禮太監黃賜母死,廷臣皆往弔,翰林不往。侍講徐瓊謀於眾,音大怒曰:「天子侍從臣,相率拜內豎之室,若清議何!」瓊愧沮。秩滿,進侍講。汪直黨韋瑛夜帥邏卒入兵部郎中楊士偉家,縛士偉,考掠及其妻子。音與比鄰,乘墉大呼曰:「爾擅辱朝臣,不畏國法耶!」其人曰:「爾何人,不畏西廠!」音厲聲曰:「我翰林陳音也。」久之,遷南京太常少卿。劉吉父喪起復,音貽書勸其固辭,吉不悅。後吏部擬用音,吉輒阻之曰「腐儒」,以故十年不得調。嘗與守備中官爭事,為所劾,事卒得直。弘治五年,吉罷,始進本寺卿。越二年卒。

音負經術,士多遊其門者。然性健忘,世故瑣屑事皆不解。世多以不慧事附之以為笑,然不盡實也。

傅瀚,字曰川,新喻人。天順八年進士。選庶吉士,除檢討。嗜學強記,善詩文。再遷左諭德,直講東宮。孝宗嗣位,擢太常少卿兼侍讀,歷禮部左、右侍郎。尋命兼學士入東閣,專典誥敕,兼掌詹事府事。

弘治十三年代徐瓊為禮部尚書。保定獻白鵲,疏斥之。陜西巡撫熊翀以鄠縣民所得玉璽來獻,以為秦璽復出也。瀚率同列言:「秦璽完毀,具載簡冊。今所進璽,形色、篆紐皆不類,蓋後人仿為之。且帝王受命在德不在璽,太祖製六璽,列聖相承,百三十餘載,天休滋至,受命之符不在秦璽明矣。請姑藏內府。」帝是其言,薄賞得璽者。

京師星變、地震、雨雹,四方多變異。瀚條上軍民所不便進者,請躬行節儉以先天下。光祿寺逋行戶物價至四萬餘兩。瀚言由供億之濫,願敦儉素,俾冗費不生。所條奏,率傅正議。十五年卒,贈太子太保,謚文穆。

張昇,字啟昭,南城人。成化五年進士第一。授修撰,歷諭德。弘治改元,遷庶子。

大學士劉吉當國,昇因天變,疏言:「陛下即位,言者率以萬安、劉吉、尹直為言,安、直被斥,吉獨存。吉乃傾身阿佞,取悅言官,昏暮款門,祈免糾劾,許以超遷。由是諫官緘口,奸計始遂。貴戚萬喜依憑宮壺,凶焰熾張,吉與締姻。及喜下獄,猶為營救。父存則異居各爨,父歿則奪情起官。談笑對客,無復戚容。盛納艷姬,恣為淫黷。」且歷數其納賄、縱子等十罪。吉憤甚,風科道劾昇誣詆,調南京工部員外郎。吉罷,復故官,歷禮部左、右侍郎。十五年代傅瀚為尚書。

孝宗崩,真人陳應衣盾、西番灌頂大國師那卜堅參等以祓除,率其徒入乾清宮,昇請置之法。詔奪真人、國師、高士等三十餘人名號,逐之。昇在部五年,遇災異,輒進直言。亦數為言者所攻,然自守謹飭。

武宗嬉遊怠政,給事中胡煜、楊一渶、張襘皆以為言,章下禮部。昇因上疏,請親賢遠佞,克謹天戒。帝是之而不能用,昇遂連疏乞休,不允。正德二年,秦府鎮國將軍誠漖請襲封保安王,昇執不可。忤劉瑾,謝病。詔加太子太保,乘傳歸,月米、歲夫如制。卒於家。

吳寬,字原博,長洲人。以文行有聲諸生間。成化八年,會試、廷試皆第一,授修撰。侍孝宗東宮,秩滿進右諭德。孝宗即位,以舊學遷左庶子,預修《憲宗實錄》,進少詹事兼侍讀學士。

弘治八年擢吏部右侍郎。丁繼母憂,吏部員缺,命虛位待之。服滿還任,轉左,改掌詹事府,入東閣,專典誥敕,仍侍武宗東宮。宦豎多不欲太子近儒臣,數移事間講讀。寬率其僚上疏曰:「東宮講學,寒暑風雨則止,朔望令節則止,一年不過數月,一月不過數日,一日不過數刻。是進講之時少,輟講之日多,豈容復以他事妨誦讀。古人八歲就傅,即居宿於外,欲離近習,親正人耳。庶民且然,矧太子天下本哉?」帝嘉納之。

十六年進禮部尚書,餘如故。先是,孝莊錢太后崩,廷議孝肅周太后萬歲後,並葬裕陵,祔睿廟,禮皆如適。至是,孝肅崩,將祔廟,帝終以並祔為疑,下禮官集議。寬言《魯頌·閟宮》、《春秋》考仲子之宮皆別廟,漢、唐亦然。會大臣亦多主別廟,帝乃從之。時詞臣望重者,寬為最,謝遷次之。遷既入閣,嘗為劉健言,欲引寬共政,健固不從。他日又曰:「吳公科第、年齒、聞望皆先於遷,遷實自愧,豈有私於吳公耶。」及遷引退,舉寬自代,亦不果用。中外皆為之惜,而寬甚安之,曰:「吾初望不及此也。」年七十,數引疾,輒慰留,竟卒於官。贈太子太保,謚文定。授長子奭中書舍人,補次子奐國子生,異數也。

寬行履高潔,不為激矯,而自守以正。於書無不讀,詩文有典則,兼工書法。有田數頃,嘗以周親故之貧者。友人賀恩疾,遷至邸,旦夕視之。恩死,為衣素一月。

傅珪,字邦瑞,清苑人。成化二十三年進士。改庶吉士。弘治中,授編修,尋兼司經局校書。與修《大明會典》成,遷左中允。武宗立,以東宮恩,進左諭德,充講官,纂修《孝宗實錄》。時詞臣不附劉瑾,瑾惡之。謂《會典》成於劉健等,多所糜費,鐫與修者官,降珪修撰。俄以《實錄》成,進左中允,再遷翰林學士,歷吏部左、右侍郎。

正德六年,代費宏為禮部尚書。禮部事視他部為簡,自珪數有執爭,章奏遂多。帝好佛,自稱「大慶法王」。番僧乞田百頃為法王下院,中旨下部,稱大慶法王與聖旨並。珪佯不知,執奏:「孰為大慶法王?敢與至尊並書,大不敬。」詔勿問,田亦竟止。

珪居閒類木訥者。及當大事,毅然執持,人不能奪,卒以此忤權倖去。教坊司臧賢請易牙牌,製如朝士,又請改鑄方印。珪格不行。賢日夜騰謗於諸閹間,冀去珪。流寇擾河南,太監陸訚謀督師,下廷議,莫敢先發。珪厲聲曰:「師老民疲,賊日熾,以冒功者多,僨事者漏罰,失將士心。先所遣已無功,可復遣耶?今賊橫行郊圻肘腋間,民囂然思亂,禍旦夕及宗社。吾儕死不償責,諸公安得首鼠兩端。」由是議罷。疏上,竟遣訚,而中官皆憾珪。御史張羽奏雲南災。珪因極言四方災變可畏。八年五月,復奏四月災,因言:「春秋二百四十二年,災變六十九事。今自去秋來,地震天鳴,雹降星殞,龍虎出見,地裂山崩,凡四十有二,而水旱不與焉,災未有若是甚者。」極陳時弊十事,語多斥權倖,權倖益深嫉之。會戶部尚書孫交亦以守正見忤,遂矯旨令二人致仕。兩京言官交章請留,不聽。

珪歸三年,御史盧雍稱珪在位有古大臣風,家無儲蓄,日給為累,乞頒月廩、歲隸,以示優禮。又謂珪剛直忠讜,當起用。吏部請如雍言,不報。而珪適卒,年五十七。遣命毋請恤典。撫、按以為言,詔蔭其子中書舍人。嘉靖元年錄先朝守正大臣,追贈太子少保,謚文毅。

劉春,字仁仲,巴人。成化二十三年進士及第。授編修,屢遷翰林學士。正德六年擢吏部右侍郎,進左。八年代傅珪為禮部尚書。淮王祐棨、鄭王祐BT皆由旁支襲封,而祐棨稱其本生為考,祐BT并欲追封入廟。交城王秉杋由鎮國將軍嗣爵,而進其妹為縣主。春皆據禮駁之,遂著為例。

帝崇信西僧,常襲其衣服,演法內廠。有綽吉我些兒者,出入豹房,封大德法王。遣其徒二人還烏思藏,請給國師誥命如大乘法王例,歲時入貢,且得齎茶以行。春持不可。帝命再議,春執奏曰:「烏思藏遠在西方,性極頑獷。雖設四王撫化,其來貢必有節制,使不為邊患。若許其齎茶,給之誥敕,萬一假上旨以誘羌人,妄有請乞,不從失異俗心,從之則滋害。」奏上,罷齎茶,卒與誥命。春又奏:「西番俗信佛教,故祖宗承前代舊,設立烏思藏諸司,及陜西洮、岷,四川松潘諸寺,令化導番人,許之朝貢。貢期、人數皆有定制。比緣諸番僻遠,莫辨真偽。中國逃亡罪人,習其語言,竄身在內,又多創寺請額。番貢日增,宴賞繁費,乞嚴其期限,酌定人數,每寺給勘合十道,緣邊兵備存勘合底簿,比對相同,方許起送。并禁自後不得濫營寺宇。」報可。廣東布政使羅榮等入覲,各言鎮守內臣入貢之害。春列上累朝停革貢獻詔旨,且言四方水旱盜賊,軍民困苦狀,乞罷諸鎮守臣。不納。

春掌禮三年,慎守彞典。宗籓請封、請婚及文武大臣祭葬、贈謚,多所裁正。遭憂,服闋起南京吏部尚書。尋以禮部尚書專典誥敕,掌詹事府事。十六年卒。贈太子太保,謚,文簡。

劉氏世以科第顯。春父規,御史。弟台,雲南參政。子彭年,巡撫貴州右副都御史。彭年子起宗,遼東苑馬寺卿。起宗子世賞,廣東左布政使。台子鶴年,雲南布政使,以清譽聞。鶴年孫世曾,巡撫雲南右副都御史,有徵緬功。皆由進士。

吳儼,字克溫,宜興人。成化二十三年進士。改庶吉士,授編修,歷侍講學士,掌南京翰林院。正德初,召修《孝宗實錄》,直講筵。劉瑾竊柄,聞儼家多資,遣人啖以美官。儼峻拒之,瑾怒。會大計群吏,中旨罷儼官。瑾誅,復職歷禮部左、右侍郎,拜南京禮部尚書。

十二年,武宗北巡,儼抗疏切諫。明年復偕諸大臣上疏曰:「臣等初聞駕幸昌平,曾具疏極論,不蒙採納。既聞出居庸,幸宣、大,宰輔不及知,群臣不及從,三軍之士不及衛,京師內外人心動搖。徐、淮以南,荒饉千里,去冬雨雪為災,民無衣食,安保其不為盜。所禦之寇尚遠隔陰山,而不虞之禍或猝起於肘腋,臣所大懼也。」不報。

十四年卒官。贈太子少保,謚文肅。

顧清,字士廉,松江華亭人。弘治五年舉鄉試第一。明年,成進士,改庶吉士,授編修。與同年生毛澄、羅欽順、汪俊相砥以名節。進侍讀。

正德初,劉瑾竊柄,清邑子張文冕為謀主,附者立尊顯。清絕不與通,瑾銜之。四年摘《會典》小誤,挫諸翰林,清降編修,。又以諸翰林未諳政事,調外任及兩京部屬,清得南京兵部員外郎。會父憂,不赴。瑾誅,還侍讀,擢侍讀學士掌院事。尋遷少詹事,充經筵日講官,進禮部右侍郎。時澄已為尚書,清協恭守職,前後請建儲宮,罷巡幸,疏凡十數上。世宗嗣位,為御史李獻所劾,罷歸。

清學端行謹,恬於進取。家居,薦者相繼,悉報寢。嘉靖六年,詔舉老成堪用內閣者,廷推及清,乃以為南京禮部右侍郎。上言:「錦衣職侍衛,祖宗朝非機密不遣。正德間,營差四出,海內騷然,陛下所親見。近乃遣千戶勘揚州高瀹爭私財事,囚其女婦,憯毒備加。請自今悉付所司,停旂校無遣。」從之。

屢疏引疾,詔進尚書致仕。時方進表入都,道卒。謚文僖。

劉瑞,字德符,內江人。父時斅,官山東僉事,以廉惠稱。瑞舉弘治九年進士,選庶吉士,授檢討。好學潔修,遇事輒有論建。清寧宮災,請罷醮壇。時召內閣講官延訪治道,又言:「故閹李廣門下內臣,宜悉治罪。前太監汪直,先帝罪人,今來覬用,當斥遠之。副使楊茂元、郎中王雲鳳以直言獲罪,宜召復其官。京師之萬春宮,興濟真武廟、壽寧侯第,在外之興、岐、衡、雍、汝、涇諸府,土木繁興,宜悉罷不急者。都勻之捷,鄧廷瓚冒其功。賀蘭之徵,王越啟其釁。請追正欺罔之罪。」報聞。闕里廟成,遣大學士李東陽祭告。瑞請更定先師封謚,不果行。

武宗即位,疏陳端治本九事。請召祭酒章懋,侍郎王鏊,都御史林俊、雍泰;而超擢參政王綸、副使王雲鳳、僉事胡獻、知府楊茂元、照靡餘濂。由是,諸臣多獲進用。

劉瑾用事,瑞即謝病。貧不能還鄉,依從母子李充嗣於澧州。瑾榜瑞為奸黨,又以前薦雍泰除其名,罰米輸塞上。坐是益困,授徒自給。

瑾誅,以副使督浙江學校,召為南京太僕少卿。嘉靖二年,由南太常卿就遷禮部右侍郎。因災變偕同官條上六事,且言齋醮無益且妨政,織造多費且病民。帝多粕用之。大禮議起,瑞偕九卿合疏。極言大宗、小宗之義,凡數千言。四年卒官。贈尚書。隆慶初,謚文肅。

贊曰:周洪謨等以詞臣歷卿貳。或職事拳拳,或侃侃建白,進講以啟沃為心,守官以獻替自效。於文學侍從之選,均無愧諸。


\end{pinyinscope}