\article{列傳第七十八}

\begin{pinyinscope}
○楊廷和梁儲蔣冕毛紀石珤兄玠

楊廷和,字介夫,新都人。父春,湖廣提學僉事。廷和年十二舉於鄉。成化十四年,年十九,先其父成進士。改庶吉士,告歸娶,還朝授檢討。

廷和為人美風姿,性沉靜詳審,為文簡暢有法。好考究掌故、民瘼、邊事及一切法家言,鬱然負公輔望。弘治二年進修撰。《憲宗實錄》成,以預纂修進侍讀。改左春坊左中允,侍皇太子講讀。修《會典》成,超拜左春坊大學士,充日講官。正德二年由詹事入東閣,專典誥敕。以講筵指斥佞幸,忤劉瑾,傳旨改南京吏部左侍郎。五月遷南京戶部尚書。又三月召還,進兼文淵閣大學士,參預機務。明年加少保兼太子太保。瑾摘《會典》小誤,奪廷和與大學士李東陽等俸二級。尋以成《孝宗實錄》功還之。明年加光祿大夫、柱國,遷改吏部尚書、武英殿大學士。

時瑾橫益甚,而焦芳、張糸採為中外媾。廷和與東陽委曲其間,小有劑救而已。安化王寘鐇反,以誅瑾為名。廷和等草赦詔,請擢邊將仇鉞,以離賊黨。鉞果執寘鐇。會張永發瑾罪,瑾伏誅,廷和等乃復論功,進少傅兼太子太傅、謹身殿大學士,予一子中書舍人。

流賊劉六、劉七、齊彥名反,楊一清薦馬中錫討之。廷和言:「中錫,文士也,不任此。」時業已行,果不能平賊。廷和請逮中錫下獄,以陸完代之,而斬故受賕縱賊者參將桑玉。已,又用學士陳霽言,調諸邊兵討河南賊趙鐩等,而薦彭澤為總制。賊平論功,錄廷和一子錦衣衛千戶。辭,特加少師、太子太師、華蓋殿大學士。東陽致政,廷和遂為首輔。

張永既去瑾而驕,捕得男子臂龍文者以為功,援故太監劉永誠例,覬封侯。廷和言「永誠從子聚自以戰功封伯耳,且非永誠身受之也」,乃止。彭澤將西討鄢本恕,問計廷和。廷和曰:「以君才,賊不足平,所戒者班師早耳。」澤後破誅本恕等即班師,而餘黨復蝟起不可制。澤既發復留,乃歎曰:「楊公先見,吾不及也。」

乾清宮災,廷和請帝避殿,下詔罪己,求直言。因與其僚上疏,勸帝早朝晏罷,躬九廟祭祀,崇兩宮孝養,勤日講。復面奏開言路、達下情、還邊兵、革宮市、罷皇店、出西僧、省工作、減織造,凡十餘條,皆切至。帝不省。尋以父卒乞奔喪,不許。三請乃許。遣中官護行。旋復起之,三疏辭,始許。閣臣之得終父母喪者,自廷和始也。服甫闋,即召至。帝方獵宣府,使使賜廷和羊酒、銀幣。廷和疏謝,因請迴鑾,不報。復與大學士蔣冕馳至居庸,欲身出塞請。帝令谷大用扼關門,乃歸。帝命迴鑾日群臣各製旗帳迎,廷和曰:「此里俗以施之親舊耳。天子至尊,不敢瀆獻。」帝再使使諭意,執不從,乃已。

當廷和柄政,帝恒不視朝,恣游大同、宣府、延綏間,多失政。廷和未嘗不諫,俱不聽。廷和亦不能執奏。以是邑邑不自得,數移疾乞骸骨,帝亦不聽。中官谷大用、魏彬、張雄,義子錢寧、江彬輩,恣橫甚。廷和雖不為下,然亦不能有所裁禁,以是得稍自安。

御史蕭淮發寧王宸濠反謀,錢寧輩猶庇之,詆淮離間。廷和請如宣宗諭趙王故事,遣貴戚大臣齎敕往諭,收其護衛屯田。於是命中官賴義、駙馬都尉崔元等往,未至而宸濠反。帝欲帥師親征,廷和等力阻之。帝乃自稱總督軍務、威武大將軍、總兵官、後軍都督府、太師、鎮國公朱壽,統各京邊將士南討。而安邊伯許泰為威武副將軍、左都督劉暉為平賊將軍前驅,鎮守、撫、按悉聽節制。命廷和與大學士毛紀居守。以乾清、坤寧二宮工成,推恩錄一子錦衣衛副千戶,辭。時廷和當草大將軍征南敕諭,謝弗肯,帝心恚。會推南京吏部尚書劉春理東閣誥敕,以廷和私其鄉人,切責之。廷和謝罪,乞罷,不許。少師梁儲等請與俱罷,復不許。廷和方引疾不入,帝遂傳旨行之。時十四年八月也。帝既南,兩更歲朔。廷和頗以鎮靜持重,為中外所推服。凡請迴鑾者數十疏,皆不復省。帝歸,駐蹕通州。廷和等舉故事,請帝還大內御殿受俘,然後正宸濠等誅,而帝已不豫。趨召廷和等至通州受事,即行在執宸濠等僇之,駕乃旋。

明年正月,帝郊祀,嘔血輿疾歸,逾月益篤。時帝無嗣。司禮中官魏彬等至閣言:「國醫力竭矣,請捐萬金購之草澤。」廷和心知所謂,不應,而微以倫序之說風之,彬等唯唯。三月十四日丙寅,谷大用、張永至閣,言帝崩於豹房。以皇太后命,移殯大內,且議所當立。廷和舉《皇明祖訓》示之曰:「兄終弟及,誰能瀆焉!興獻王長子,憲宗之孫,孝宗之從子,大行皇帝之從弟,序當立。」梁儲、蔣冕、毛紀咸贊之。乃令中官入啟皇太后,廷和等候左順門下。頃之,中官奉遺詔及太后懿旨,宣諭群臣,一如廷和請,事乃定。

廷和遂以遺詔令太監張永、武定侯郭勛、安邊伯許泰、尚書王憲選各營兵,分布皇城四門、京城九門及南北要害,廣衛御史以其屬扦掫。傳遺命罷威武營團練諸軍,各邊兵入衛者俱重賚散歸鎮,革皇店及軍門辦事官校悉還衛,哈密、土魯番、佛郎機諸貢使皆給賞遣還國,豹房番僧及少林僧、教坊樂人、南京快馬船、諸非常例者,一切罷遣。又以遺詔釋南京逮擊囚,放遣四方進獻女子,停京師不急工務,收宣府行宮金寶歸諸內庫。中外大悅。時平虜伯江彬擁重兵在肘腋間,知天下惡之,心不自安。其黨都督僉事李琮尤狠黠,勸彬乘間以其家眾反,不勝則北走塞外。彬猶豫未決。於是廷和謀以皇太后旨捕誅彬,遂與同官蔣冕、毛紀及司禮中官溫祥四人謀。張永伺知其意,亦密為備。司禮魏彬者,故與彬有連。廷和以其弱可脅也,因題大行銘旌,與彬、祥及他中官張銳、陳嚴等為詳言江彬反狀,以危語怵之。彬心動,惟銳力言江彬無罪,廷和面折之。冕曰:「今日必了此,乃臨。」嚴亦從旁贊決,因俾祥、彬等入白皇太后。良久未報,廷和、冕益自危。頃之,嚴至曰:「彬已擒矣。」彬既誅,中外相慶。

廷和總朝政幾四十日,興世子始入京師即帝位。廷和草上登極詔書,文書房官忽至閣中,言欲去詔中不便者數事。廷和曰:「往者事齟齬,動稱上意。今亦新天子意耶?吾儕賀登極後,當面奏上,問誰欲削詔草者!」冕、紀亦相繼發危言,其人語塞。已而詔下,正德中蠹政釐抉且盡。所裁汰錦衣諸衛、內監局旗校工役為數十四萬八千七百,減漕糧百五十三萬二千餘石。其中貴、義子、傳升、乞升一切恩倖得官者大半皆斥去。中外稱新天子「聖人」,且頌廷和功。而諸失職之徒銜廷和次骨,廷和入朝有挾白刃伺輿旁者。事聞,詔以營卒百人衛出入。帝御經筵,廷和知經筵事。修《武宗實錄》,充總裁。廷和先已加特進,一品滿九載,兼支大學士俸,賜敕旌諭。至是加左柱國。帝召對者三,慰勞備至。廷和益欲有所發攄,引用正人,布列在位。

給事、御史交章論王瓊罪狀,下詔獄。瓊迫,疏訐廷和以自解。法司當瓊奸黨律論死,瓊力自辨,得減戍邊。或疑法司承廷和指者。會石珤自禮部尚書掌詹事府,改吏部,廷和復奏改之掌詹事司誥敕。人或謂廷和太專。然廷和以帝雖沖年,性英敏,自信可輔太平,事事有所持諍。錢寧、江彬雖伏誅,而張銳、張忠、于經、許泰等獄久不決。廷和等言:「不誅此曹,則國法不正,公道不明,九廟之靈不安,萬姓之心不服,禍亂之機未息,太平之治未臻。」帝乃籍沒其資產。廷和復疏請敬天戒,法祖訓,隆孝道,保聖躬,務民義,勤學問,慎命令,明賞罰,專委任,納諫諍,親善人,節財用。語多剴切,皆優詔報可。

及議「大禮」,廷和持論益不撓,卒以是忤帝意。先是,武宗崩,廷和草遺詔。言皇考孝宗敬皇帝親弟興獻王長子某,倫序當立。遵奉《祖訓》兄終弟及之文,告於宗廟,請於慈壽皇太后,迎嗣皇帝位。既令禮官上禮儀狀,請由東安門入居文華殿。翼日,百官三上箋勸進,俟令旨俞允,擇日即位。其箋文皆循皇子嗣位故事。世宗覽禮部狀,謂:「遺詔以吾嗣皇帝位,非為皇子也。」及至京,止城外。廷和固請如禮部所具儀,世宗不聽。乃御行殿受箋,由大明門直入,告大行几筵,日中即帝位。詔草言「奉皇兄遺詔入奉宗祧」,帝遲回久之,始報可。越三日,遣官往迎帝母興獻妃。未幾,命禮官議興獻王主祀稱號。廷和檢漢定陶王、宋濮王事授尚書毛澄曰:「是足為據,宜尊孝宗曰『皇考』,稱獻王為『皇叔考興國大王』,母妃為『皇叔母興國太妃』,自稱『姪皇帝』名,別立益王次子崇仁王為興王,奉獻王祀。有異議者即奸邪,當斬。」進士張璁與侍郎王瓚言,帝入繼大統,非為人後。瓚微言之,廷和恐其撓議,改瓚官南京。五月,澄會廷臣議上,如廷和言。帝不悅。然每召廷和從容賜茶慰諭,欲有所更定,廷和卒不肯順帝指。乃下廷臣再議。廷和偕蔣冕、毛紀奏言:「前代入繼之君,追崇所生者,皆不合典禮。惟宋儒程頤《濮議》最得義理之正,可為萬世法。至興獻王祀,雖崇仁王主之,他日皇嗣繁衍,仍以第二子為興獻王後,而改封崇仁王為親王,則天理人情,兩全無失。」帝益不悅,命博考典禮,務求至當。廷和、冕、紀復言:「三代以前,聖莫如舜,未聞追崇其所生父瞽瞍也。三代以後,賢莫如漢光武,未聞追崇其所生父南頓君也。惟皇上取法二君,則聖德無累,聖孝有光矣。」澄等亦再三執奏。帝留中不下。

七月,張璁上疏謂當繼統,不繼嗣。帝遣司禮太監持示廷和,言此議遵祖訓,據古禮,宜從。廷和曰「秀才安知國家事體」,復持入。無何,帝御文華殿召廷和、冕、紀,授以手敕,令尊父母為帝、后。廷和退而上奏曰:「《禮》謂為所後者為父母,而以其所生者為伯叔父母,蓋不惟降其服而又異其名也。臣不敢阿諛順旨。」仍封還手詔。群臣亦皆執前議。帝不聽。迨九月,母妃至京,帝自定儀由中門入,謁見太廟,復申諭欲加稱興獻帝、后為「皇」。廷和言:「漢宣帝繼孝昭後,謚史皇孫、王夫人曰悼考、悼后,光武上繼元帝,鉅鹿、南頓君以上立廟章陵,皆未嘗追尊。今若加皇字,與孝廟、慈壽並,是忘所後而重本生,任私恩而棄大義,臣等不得辭其責。」因自請斥罷。廷臣諍者百餘人。帝不得已,乃以嘉靖元年詔稱孝宗為「皇考」,慈壽皇太后為「聖母」,興獻帝、后為本生父母,不稱「皇」。

當是時,廷和先後封還御批者四,執奏幾三十疏,帝常忽忽有所恨。左右因乘間言廷和恣無人臣禮。言官史道、曹嘉遂交劾廷和。帝為薄謫道、嘉以安廷和,然意內移矣。尋論定策功,封廷和、冕、紀伯爵,歲祿千石,廷和固辭。改廕錦衣衛指揮使,復辭。帝以賞太輕,加廕四品京職世襲,復辭。會滿四考,超拜太傅,復四辭而止。特賜敕旌異,錫宴於禮部,九卿皆與焉。

帝頗事齋醮。廷和力言不可,引梁武、宋徽為喻,優旨報納。江左比歲不登,中官請遣官督織造。工部及給事、御史言之,皆不聽,趣內閣撰敕。廷和等不奉命,因極言民困財竭,請毋遣。帝趣愈急,且戒毋瀆擾執拗。廷和力爭,言:「臣等與舉朝大臣、言官言之不聽,顧二三邪佞之言是聽,陛下能獨與二三邪佞共治祖宗天下哉?且陛下以織造為累朝舊例,不知洪武以來何嘗有之,創自成化、弘治耳。憲宗、孝宗愛民節財美政非一,陛下不取法,獨法其不美者,何也?即位一詔,中官之倖路絀塞殆盡,天下方傳誦聖德,今忽有此,何以取信?」因請究擬旨者何人,疑有假御批以行其私者。帝為謝不審,俾戒所遣官毋縱肆而已,不能止也。

廷和先累疏乞休,其後請益力。又以持考獻帝議不合,疏語露不平。三年正月,帝聽之去。責以因辭歸咎,非大臣道。然猶賜璽書,給輿廩郵護如例,申前蔭子錦衣衛指揮使之命。給事、御史請留廷和,皆不報。廷和去,始議稱孝宗為「皇伯考」。於是,廷和子修撰慎率群臣伏闕哭爭,杖謫雲南。既而王邦奇誣訐廷和及其次子兵部主事惇、婿修撰金承勛、鄉人侍讀葉桂章與彭澤弟沖交關請屬,俱逮下詔獄。鞫治無狀,乃得解。七年,《明倫大典》成,詔定議禮諸臣罪。言廷和謬主《濮議》,自詭門生天子、定策國老,法當僇市,姑削職為民。明年六月卒,年七十一。居久之,帝問大學士李時:「太倉所積幾何?」時對曰:「可支數年。由陛下初年詔書裁革冗員所致。」帝慨然曰;「此楊廷和功,不可沒也。」隆慶初,復官,贈太保,謚文忠。

初,廷和入閣,東陽謂曰:「吾於文翰,頗有一日之長,若經濟事須歸介夫。」及武宗之終,卒安社稷者,廷和力也,人以東陽為知言。

弟廷儀,兵部右侍郎。子慎、惇,孫有仁,皆進士。慎自有傳。

梁儲,字叔厚,廣東順德人。受業陳獻章。舉成化十四年會試第一,選庶吉士,授編修,尋兼司經局校書。弘治四年,進侍講。改洗馬,侍武宗於東宮。冊封安南,卻其饋。久之,擢翰林學士,同修《會典》,遷少詹事,拜吏部右侍郎。正德初,改左,進尚書,專典誥敕,掌詹事府。劉瑾摘《會典》小疵,儲坐降右侍郎。《孝宗實錄》成,復尚書,尋加太子少保,調南京吏部。瑾誅,以吏部尚書兼文淵閣大學士,入參機務。屢加少傅、太子太傅,進建極殿。十年,楊廷和遭喪去,儲為首輔。進少師、太子太師、華蓋殿大學士。時方建乾清、坤寧宮,又營太素殿、天鵝房、船塢,儲偕同官靳貴、楊一清切諫。明年春,以國本未定,請擇宗室賢者居京師,備儲貳之選,皆不報。其秋,一清罷,蔣冕代之。至明年,貴亦罷,毛紀入閣。

帝好微行,嘗出西安門,經宿返。儲等諫,不聽,然猶慮外廷知。是春,從近倖言召百官至左順門,明告以郊祀畢,幸南海子觀獵。儲等暨廷臣諫,皆不納。八月朔,微服從數十騎幸昌平。次日,儲、冕、紀始覺,追至沙河不及,連疏請回鑾。越十有三日乃旋。儲等以國無儲副,而帝盤游不息,中外危疑,力申建儲之請,亦不報。九月,帝馳出居庸關,幸宣府,命谷大用守關,無縱廷臣出。遂由宣府抵大同,遇寇於應州,幾殆。儲等憂懼,請回鑾益急。章十餘上,帝不為動,歲除竟駐宣府。當是時,帝失德彌甚。群小竊權,濁亂朝政,人情惶惶。儲懼不克任,以廷和服闋,屢請召之。廷和還朝,儲遂讓而處其下。鳳陽守備中官丘德及鎮守延綏、寧夏、大同、宣府諸中官皆乞更敕書兼理民事,帝許之。儲等極言不可,弗聽。

十三年七月,帝從江彬言,將遍游塞上。託言邊關多警,命總督軍務、威武大將軍、總兵官硃壽統六師往征,令內閣草敕。閣臣不可,帝復集百官左順門面諭。廷和、冕在告,儲、紀泣諫,眾亦泣,帝意不可回。已而紀亦引疾。儲獨廷爭累日,帝竟不聽。踰月,帝以「大將軍壽」肅清邊境,令加封「鎮國公」。儲、紀上言:「公雖貴,人臣耳。陛下承祖宗業,為天下君,奈何謬自貶損。既封國公,則將授以誥券,追封三代。祖宗在天之靈亦肯如陛下貶損否?況鐵券必有免死之文,陛下壽福無疆,何甘自菲薄,蒙此不祥之辭。名既不正,言自不順。臣等斷不敢阿意茍從,取他日戮身亡家之禍也。」不報。帝遂歷宣府、大同,直抵延綏。儲等疏數十上,悉置不省。

秦王請關中閑田為牧地,江彬、錢寧、張忠等皆為之請。帝排群議許之,命閣臣草制。廷和、冕引疾,帝怒甚。儲度不可爭,乃上制草曰:「太祖高皇帝著令,茲土不畀籓封。非吝也,念其土廣饒,籓封得之,多蓄士馬,富而且驕,奸人誘為不軌,不利宗社。王今得地,宜益謹。毋收聚奸人,毋多蓄士馬,毋聽狂人謀不軌,震及邊方,危我社稷,是時雖欲保親親不可得已。」帝駭曰:「若是其可虞!」事遂寢。明年,帝將南巡。言官伏闕諫,儲、冕、紀亦以為言。會諸曹多諫者,乃止。寧王宸濠反,帝南征,儲、冕扈從。在道聞賊滅,連疏請駕旋。抵揚州,帝議南京行郊禮。儲、冕計此議行,則回鑾益無日,極陳不可,疏三上始得請。帝以宸濠械將至,問處置之宜。儲等請如宣宗征高煦故事,罪人既得,即日班師。又因郊期改卜,四方災異、邊警,乞還乘輿。疏八九上,帝殊無還意。是秋,行在有物若豕首墮帝前,色碧,又進御婦人室中,若懸人首狀。人情益驚。儲、冕危言諫,帝頗心動。而群小猶欲導帝游浙西,泛江、漢。儲、冕益懼,手疏跪泣行宮門外,歷未至酉。帝遣人取疏入,諭之起。叩頭言:「未奉俞旨,不敢起也。」帝不得已,許不日還京,乃叩頭出。

帝崩,楊廷和等定策迎興世子。故事,當以內閣一人與中貴勛戚偕禮官往。廷和欲留蔣冕自助,而慮儲老或憚行,乃佯惜儲憊老,阻其行。儲奮曰:「事孰有大於此者,敢以憊辭!」遂與定國公徐光祚等迎世子安陸邸。既即位,給事中張九敘等劾儲結納權奸,持祿固寵。儲三疏求去,命賜敕馳傳,遣行人護行,歲給廩隸如制。卒,子鈞奏請贈謚。吏部侍郎桂萼等言,儲立身輔政,有干公議,因錄上兩京言官彈章。帝念先朝舊臣,特贈太師,謚文康。

先是,儲子次攄為錦衣百戶。居家與富人楊端爭民田,端殺田主,次攄遂滅端家二百餘人。事發,武宗以儲故,僅發邊衛立功。後還職,累冒功至廣東都指揮僉事。

蔣冕,字敬之,全州人。兄昇,南京戶部尚書,以謹厚稱。冕舉成化二十三年進士,選庶吉士,授編修。弘治十三年,太子出閣,兼司經局校書。正德中,累官吏部左侍郎,改掌詹事府,典誥敕,進禮部尚書,仍掌府事。

冕清謹有器識,雅負時望。十一年命兼文淵閣大學士,預機務。明年改武英殿,加太子太傅。近倖冒邊功,大行陞賞,冕及梁儲亦廕錦衣世千戶。兩人力辭,乃改文廕。

帝之以「威武大將軍」行邊也,冕時病在告,疏諫曰:「陛下自損威重,下同臣子,倘所過諸王以大將軍禮見,陛下何辭責之?曩睿皇帝北征,六軍官屬近三十萬,猶且陷於土木。今宿衛單弱,經行邊徼,寧不寒心?請治左右引導者罪。」不報。十四年扈帝南征還,加少傅兼太子太傅、戶部尚書、謹身殿大學士。帝崩,與楊廷和協誅江彬。

世宗即位,議定策功,加伯爵,固辭。改蔭錦衣世指揮,又辭。乃廕五品文職,仍進一階。御史張鵬疏評大臣賢否,請罷冕。御史趙永亨詆石珤不可掌銓衡。冕、珤遂求去。朝議不平,諸給事、御史皆言其不可去。帝乃命鴻臚諭留,再下優詔,始起視事。

嘉靖三年遣官織造江南,命冕草敕。冕以江南被災,具疏請止,帝不從,敕亦久不進。帝責其違慢,冕引罪而止。

「大禮」議起,冕固執為人後之說,與廷和等力爭之。帝始而婉諭,繼以譙讓,冕執議不回。及廷和罷政,冕當國,帝愈欲尊崇所生。逐禮部尚書汪俊以怵冕,而用席書代之,且召張璁、桂萼。物情甚沸,冕乃抗疏極諫曰:「陛下嗣承丕基,固因倫序素定。然非聖母昭聖皇太后懿旨與武宗皇帝遺詔,則將無所受命。今既受命於武宗,自當為武宗之後。特兄弟之名不容紊,故但兄武宗,考孝宗,母昭聖。而於孝廟、武廟皆稱嗣皇帝,稱臣,稱御名,以示繼統承祀之義。今乃欲為本生父母立廟奉先殿側,臣雖至愚,斷斷知其不可。自古人君嗣位謂之承祧踐阼,皆指宗祀而言。《禮》為人後者惟大宗,以大宗尊之統也,亦主宗廟祭祀而言。自漢至今,未有為本生父母立廟大內者。漢宣帝為叔祖昭帝後,止立所生父廟於葬所。光武中興,本非承統平帝,而止立四親廟於章陵。宋英宗父濮安懿王,亦止即園立廟。陛下先年有旨,立廟安陸,與前代適同,得其當矣。豈可既奉大宗之祀,又兼奉小宗之祀?夫情既重於所生,義必不專於所後,將孝、武二廟之靈安所托乎!竊恐獻帝之靈亦將不能安,雖聖心亦自不能安也。邇者復允汪俊之去,趣張璁、桂萼之來,人心益駭。是日廷議建廟,天本晴明,忽變陰晦,至暮風雷大作。天意如此,陛下可不思變計哉?」因力求去。帝得疏不悅,猶以大臣故,優詔答之。未幾,復請罷建廟之議,且乞體,疏中再以天變為言。帝益不悅,遂令馳傳歸,給月廩、歲夫如制。

冕當正德之季,主昏政亂,持正不撓,有匡弼功。世宗初,朝政雖新,而上下扞格彌甚,冕守之不移。代廷和為首輔僅兩閱月,卒齟齬以去,論者謂有古大臣風。《明倫大典》成,落職閒住,久之卒。隆慶初復官,謚文定。

毛紀,字維之,掖縣人。成化末,舉鄉試第一,登進士,選庶吉士。弘治初,授檢討,進修撰,充經筵講官,簡侍東宮講讀。《會典》成,遷侍讀。武宗立,改左諭德。坐《會典》小誤,降侍讀。《孝宗實錄》成,擢侍講學士,為講官。正德五年進學士,遷戶部右侍郎。

十年,由吏部左侍郎拜禮部尚書。烏思藏入貢,其使言有活佛能前知禍福。帝遣中官劉允迎之。攜錦衣官百三十,衛卒及私僕隸數千人,芻糧、舟車費以百萬計。紀等上言:「自京師至烏思藏二萬餘里,公私煩費,不可勝言。且自四川雅州出境,過長河西行數月而後至。無有郵驛、村市。一切資費,取辦四川。四川連歲用兵,流賊甫平,蠻寇復起。困竭之餘,重加此累,恐生意外變。」疏再上,內閣梁儲、靳貴、楊一清皆切諫,不報。郊祀畢,請勤朝講,又以儲嗣未建,乞早定大計,亦不聽。尋改理誥敕,掌詹事府。十二年兼東閣大學士入預機務。其秋加太子太保,改文淵閣。帝南征,紀佐楊廷和居守。駕旋,晉少保、戶部尚書、武英殿大學士。

世宗即位,錄定策功,加伯爵,再疏辭免。嘉靖初,帝欲追尊興獻帝,閣臣執奏,忤旨。三年,廷和、冕相繼去國。紀為首輔,復執如初。帝欲去本生之稱,紀與石珤合疏爭之。帝召見平臺,委曲諭意,紀終不從。朝臣伏闕哭爭者,俱逮繫,紀具疏乞原。帝怒,傳旨責紀要結朋奸,背君報私。紀乃上言曰:「曩蒙聖諭,國家政事商隺可否,然後施行。此誠內閣職業也,臣愚不能仰副明命。邇者大禮之議,平臺召對,司禮傳諭,不知其幾似乎商隺矣。而皆斷自聖心,不蒙允納,何可否之有。至於笞罰廷臣,動至數百,乃祖宗來所未有者,亦皆出自中旨,臣等不得與聞。宣召徒勤,扞格如故。慰留雖切,詰責隨加。臣雖有體國之心,不能自盡。宋司馬光告神宗曰:『陛下所以用臣,蓋察其狂直,庶有補於國家,若徒以祿位榮之而不取其言,是以官私非其人也。臣以祿位自榮,而不能救正,是徒盜竊名器以私其身也。』臣於陛下,敢舉以為告。夫要結朋奸,背君報私,正臣平日所痛憤而深疾者。有一於此,罪何止罷黜!今陛下以之疑臣,尚可一日靦顏朝寧間哉。乞賜骸骨歸鄉里,以全終始。尤望陛下法祖典學,任賢納諫,審是非,辨忠邪,以養和平之福。」帝銜紀亢直,允其去,馳驛給夫廩如故事。

紀有學識,居官廉靜簡重。與廷和、冕正色立朝,並為縉紳所倚賴。其代冕亦僅三月。後《明倫大典》成,追論奪官。久之,廷和、冕皆淪喪,紀以恩詔敘復,帝亦且忘之。二十一年,年八十,撫按以聞。詔遣官存問,再賜夫廩。又三年卒。贈太保,謚文簡。子渠,進士,太僕卿。

石珤,字邦彥,槁城人。父玉,山東按察使。珤與兄玠同舉成化末年進士,改庶吉士,授檢討,數謝病居家。孝宗末,始進修撰。正德改元,擢南京侍讀學士。歷兩京祭酒,遷南京吏部右侍郎。召改禮部,進左侍郎。武宗始遊宣府,珤上疏力諫,不報。改掌翰林院事。廷臣諫南巡,禍將不測,珤疏救之。十六年拜禮部尚書,掌詹事府。

世宗立,代王瓊為吏部尚書。自群小竊柄,銓政混濁。珤剛方,謝請託,諸犯清議者多見黜,時望大孚,而內閣楊廷和有所不悅。甫二月,復改掌詹事府,典誥敕。嘉靖元年遣祀闕里及東嶽。事竣還家,屢乞致仕。言官以珤望重,交章請留,乃起赴官。

三年五月,詔以吏部尚書兼文淵閣大學士入參機務。帝欲以奉先殿側別建一室祀獻帝,珤抗疏言其非禮。及廷臣伏闕泣爭,珤與毛紀助之。無何,「大禮」議定,紀去位。珤復諫曰:「大禮一事已奉宸斷,無可言矣。但臣反復思之,終有不安於心者。心所不安而不以言,言恐觸忤而不敢盡,則陛下將焉用臣,臣亦何以仰報君父哉?夫孝宗皇帝與昭聖皇太后,乃陛下骨肉至親也。今使疏賤讒佞小人輒行離間,但知希合取寵,不復為陛下體察。茲孟冬時享在邇,陛下登獻對越,如親見之,寧不少動於中乎?夫事亡如事存。陛下承列聖之統,以總百神,臨萬方,焉得不加慎重,顧聽細人之說,乾不易之典哉?」帝得奏不悅,戒勿復言。

明年建世廟於太廟東。帝欲從何淵言,毀神宮監,伐林木,以通輦道。給事中韓楷,御史楊秦、葉忠等交諫,忤旨奪俸。給事中衛道繼言之,貶秩調外。珤復抗章,極言不可,弗聽。及世廟成,帝欲奉章聖皇太后謁見,張璁、桂萼力主之。禮官劉龍等爭不得,諸輔臣以為言,帝不報,趣具儀。珤乃上疏曰:「陛下欲奉皇太后謁見世廟,臣竊以為從令固孝,而孝有大於從令者。臣誠不敢阿諛以誤君上。竊惟祖宗家法,后妃已入宮,未有無故復出者。且太廟尊嚴,非時享祫祭,雖天子亦不輕入,況后妃乎?璁輩所引廟見之禮,今奉先殿是也。聖祖神宗行之百五十年,已為定制,中間納后納妃不知凡幾,未有敢議及者,何至今日忽倡此議?彼容悅佞臣豈有忠愛之實,而陛下乃欲聽之乎?且陰陽有定位,不可侵越。陛下為天地百神之主,致母后無故出入太廟街門,是坤行乾事,陰侵陽位,不可之大者也。臣豈不知君命當承,第恐上累聖德,是以不敢順旨曲從,以成君父之過,負覆載之德也。」奏入,帝大慍。

珤為人清介端亮,孜孜奉國。數以力行王道,清心省事,辨忠邪,敦寬大,毋急近效為帝言。帝見為迂闊,弗善也。議「大禮」時,帝欲援以自助,而珤據禮爭,持論堅確,失帝意,璁、萼輩亦不悅。璁、萼朝夕謀輔政,攻擊費宏無虛日,以珤行高,不能有所加。至明年春,奸人王邦奇訐楊廷和,誣珤及宏為奸黨,兩人遂乞歸。帝許宏馳驛,而責珤歸怨朝廷,失大臣誼,一切恩典皆不予。歸,裝襆被車一輛而已。都人歎異,謂自來宰臣去國,無若珤者。自珤及楊廷和、蔣冕、毛紀以強諫罷政,迄嘉靖季,密勿大臣無進逆耳之言者矣。

珤加官,自太子太保至少保。七年冬卒,謚文隱。隆慶初,改謚文介。

玠,字邦秀。弘治中,由汜水知縣召為御史。出核大同軍儲,按甘肅及陜西,所條上邊務,悉中機宜,為都御史戴珊所委寄。嘗因災異劾南京刑部尚書翟瑄以下二十七人。

正德中,累官右副都御史,巡撫大同,召拜兵部右侍郎。海西部長數犯邊,泰寧三衛與別部相攻,久缺貢市,遣玠以左侍郎兼僉都御史往遼東巡視。出關撫諭,皆受約束。帝大喜,璽書嘉勞,召還。左都御史陸完遷,廷推代者,三上悉不用,最後推玠,乃以為右都御史掌院事。御史李隱劾玠夤緣,不報。十年拜戶部尚書。中官史大鎮雲南,請獨領銀場務。杜甫鎮湖廣,請借鹽船稅銀為進貢資。劉德守涼州,請帶食茶六百引。玠皆執不可。西僧闡教王請船三百艘販載食鹽,玠極言其害。帝初出居庸,玠切諫。及在宣府,需銀百萬兩,玠持不可。帝弗從,乃進其半。王瓊欲以哈密事害彭澤,玠獨廷譽之。奸民欲牟鹽利者,賄朱寧為請,玠不可,連章執奏。廷臣諫南巡跪闕下,諸大臣莫敢言,玠獨論救。群小激帝怒,嚴旨責令自陳,遂引疾去。賜敕馳傳給廩隸如故事。家居二年卒,贈太子少傅。

玠有操行,居官亦持正。其為都御史時,胡世寧論寧王,玠與李士實請罪世寧,以是為人所譏。

贊曰:武宗之季,君德日荒,嬖倖盤結左右。廷和為相,雖無能改於其德,然流賊熾而無土崩之虞,宗籓叛而無瓦解之患者,固賴廟堂有經濟之遠略也。至其誅大奸,決大策,扶危定傾,功在社稷,即周勃、韓琦殆無以過。儲雖蒙物議,而大節無玷。蔣冕、毛紀、石珤,清忠鯁亮,皆卓然有古大臣風。自時厥後,政府日以權勢相傾。或脂韋淟涊,持祿自固。求如諸人,豈可多得哉。


\end{pinyinscope}