\article{列傳第二百九 外國二}

\begin{pinyinscope}
○安南

安南,古交阯地。唐以前皆隸中國。五代時,始為土人曲承美竊據。宋初,封丁部領為交阯郡王,三傳為大臣黎桓所篡。黎氏亦三傳為大臣李公蘊所篡。李氏八傳,無子,傳其婿陳日炬。元時,屢破其國。

洪武元年,王日煃聞廖永忠定兩廣,將遣使納款,以梁王在雲南未果。十二月,太祖命漢陽知府易濟招諭之。日煃遣少中大夫同時敏,正大夫段悌、黎安世等,奉表來朝,貢方物。明年六月達京師。帝喜,賜宴,命侍讀學士張以寧、典簿牛諒往封為安南國王,賜駝紐塗金銀印。詔曰:「咨爾安南國王陳日煃,惟乃祖父,守境南陲,稱籓中國,克恭臣職,以永世封。朕荷天地之靈,肅清華夏,馳書往報。卿即奉表稱臣,專使來賀,法前人之訓,安遐壤之民。眷茲勤誠,深可嘉尚。是用遣使齎印,仍封爾為安南國王。於戲!視廣同仁,思效哲王之盛典;爵超五等,俾承奕葉之遺芳。益茂令猷,永為籓輔,欽哉。」賜日煃《大統曆》、織金文綺紗羅四十匹,同時敏以下皆有賜。

以寧等至,日煃先卒,姪日熞嗣位。遣其臣阮汝亮來迎,請誥印,以寧等不予。日熞乃復遣杜舜欽等請命於朝,以寧駐安南俟命。時安南、占城構兵,帝命翰林編修羅復仁、兵部主事張福諭令罷兵,兩國皆奉詔。明年,舜欽等至告哀。帝素服御西華門引見,遂命編修王廉往祭,賻白金五十兩、帛五十匹。別遣吏部主事林唐臣封日熞為王,賜金印及織金文綺紗羅四十匹。廉既行,帝以漢馬援立銅柱鎮南蠻,厥功甚偉,命廉就祀之。尋頒科舉詔於其國,且以更定嶽瀆神號及廓清沙漠,兩遣官詔告之。日熞遣上大夫阮兼、中大夫莫季龍、下大夫黎元普等謝恩,貢方物。兼卒於道,詔賜其王及使臣,而送兼柩歸國。頃之,復仁等還,言卻其贐不受,帝嘉之,加賜季龍等。

四年春,遣使貢象,賀平沙漠,復遣使隨以寧等來朝。其冬,日熞為伯父叔明逼死。叔明懼罪,貢象及方物。踰年至京,禮官見署表非日熞名,詰得其實,詔卻之。叔明復朝貢謝罪,且請封。其使者抵言日熞實病死,叔明遜避於外,為國人所推。帝命國人為日熞服,而叔明姑以前王印視事。七年,叔明遣使謝恩,自稱年老,乞命弟耑攝政,從之。耑遣使謝恩,請貢期。詔三年一貢,新王世見。尋復遣使貢,帝令所司諭卻,且定使者毋過三四人,貢物無厚。

十年,耑侵占城,敗沒。弟煒代立,遣使告哀,命中官陳能往祭。時安南怙強,欲滅占城,反致喪敗。帝遣官諭前王叔明毋構釁貽禍,以叔明實主國事也,叔明貢方物謝罪。廣西思明土官訴安南犯境,安南亦訴思明擾邊。帝移檄數其奸誑罪,敕守臣勿納其使。煒懼,遣使謝罪,頻年貢奄豎、金銀、紫金盤、黃金酒尊、象馬之屬。帝命助教楊盤往使,令饋雲南軍餉,煒即輸五千石於臨安。二十一年,帝復命禮部郎中邢文偉齎敕及幣往賜。煒遣使謝,復進象。帝以其頻煩,且貢物侈,命仍三歲一貢,毋進犀象。時國相黎季BX竊柄,廢其主煒,尋弒之,立叔明子日焜昆主國事,仍假煒名入貢。朝廷不知而納之,越數年始覺,命廣西守臣絕其使。季BX懼,二十七年遣使由廣東入貢。帝怒,遣宜詰責,卻其貢。季BX益懼,明年復詭詞入貢。帝雖惡其弒逆,不欲勞師遠征,乃納之。大軍方討龍州趙宗壽,命禮部尚書任亨泰、御史嚴震直諭日焜,毋自疑。季BX聞言,稍自安。帝又遣刑部尚書楊靖諭令輸米八萬石,餉龍州軍。季BX輸一萬石,饋金千兩、銀二萬兩,言龍州陸道險,請運至憑祥洞。靖不可,令輸二萬石於沲海江,江距龍州止半日。靖因言:「日焜年幼,國事皆決季BX父子,乃敢觀望如此。」時帝以宗壽納款,移兵征向武諸蠻,遂諭靖令輸二萬石給軍,而免其所饋金銀。明年,季BX告前王叔明之訃。帝以叔明本篡弒,弔祭則獎亂,止不行,移檄使知之。

思明土官黃廣成言:「自元設思明總管府,所轄左江州縣,東上思州,南銅柱為界。元征交阯,去銅柱百里立永平寨萬戶府,遣兵戍守,令交人給其軍。元季喪亂,交人攻破永平,越銅柱二百餘里,侵奪思明所屬丘溫、如嶅、慶遠、淵、脫等五縣地,近又告任尚書置驛思明洞登地。臣嘗具奏,蒙遣楊尚書勘實。乞敕安南以五縣地還臣,仍畫銅柱為界。」帝命行人陳誠、呂讓往諭,季BX執不從。誠自為書諭日焜,季BX貽書爭,且為日焜書移戶部。誠等復命,帝知其終不肯還,乃曰:「蠻夷相爭,自古有之。彼恃頑,必召禍,姑俟之。」建文元年,季BX弒日焜,立其子顒。又弒顒,立其弟案,方在襁褓中,復弒之。大殺陳氏宗族而自立,更姓名為胡一元,名其子蒼曰胡,謂出帝舜裔胡公後,僭國號大虞,年號元聖,尋自稱太上皇,傳位,朝廷不知也。

成祖既承大統,遣官以即位詔告其國。永樂元年,自署權理安南國事,遣使奉表朝貢,言:「高皇帝時安南王日煃率先輸誠,不幸早亡,後嗣絕。臣陳氏甥,為眾所推,權理國事,於今四年。望天恩賜封爵,臣有死無二。」事下禮部,部臣疑之,請遣官廉訪。乃命行人楊渤等齎敕諭其陪臣父老,凡陳氏繼嗣之有無,胡推戴之誠偽,具以實聞。賚使者遣還,復命行人呂讓、丘智賜絨錦、文綺、紗羅。既而使隨渤等還,進陪臣父老所上表,如所以誑帝者,乞即賜封爵。帝乃命禮部郎中夏止善封為安南國王。遣使謝恩,然帝其國中自若也。

思明所轄祿州、西平州、永平寨為所侵奪,帝諭令還,不聽。占城訴安南侵掠,詔令修好。陽言奉命,侵掠如故,且授印章逼為屬,又邀奪天朝賜物。帝惡之,方遣官切責,而故陪臣裴伯耆詣闕告難,言:「臣祖父皆執政大夫,死國事。臣母,陳氏近族。故臣幼侍國王,官五品,後隸武節侯陳渴真為裨將。洪武末,代渴真御寇東海。而賊臣黎季BX父子弒主篡位,屠戮忠良,滅族者以百十數,臣兄弟妻孥亦遭害。遣人捕臣,欲加誅醢。臣棄軍遁逃,伏處山谷,思詣闕庭,披瀝肝膽,展轉數年,始睹天日。竊惟季BX乃故經略使黎國髦之子,世事陳氏,叨竊寵榮,及其子蒼,亦蒙貴任。一旦篡奪,更姓易名,僭號改元,不恭朝命。忠臣良士疾首痛心,願興弔伐之師,隆繼絕之義,蕩除奸凶,復立陳氏後,臣死且不朽。敢效申包胥之忠,哀鳴闕下,惟皇帝垂察。」帝得奏感動,命所司周以衣食。會老撾送陳天平至,言:「臣天平,前王日烜孫,奣子,日煃弟也。黎賊盡滅陳族,臣越在外州獲免。臣僚佐激於忠義,推臣為主以討賊。方議招軍,賊兵見迫,倉皇出走,竄伏巖谷,萬死一生,得達老撾。恭聞皇帝陛下入正大統,臣有所依歸。匍匐萬里,哀愬明庭。陳氏後裔止臣一人,臣與此賊不共戴天。伏祈聖慈垂憐,迅發六師,用章天討。」帝益感動,命所司館之。

方遣使賀正旦,帝出天平示之,皆錯愕下拜,有泣者。伯耆責使者以大義,惶恐不能答。帝諭侍臣:「父子悖逆,鬼神所不容,而國中臣民共為欺蔽。一國皆罪人也,朕烏能容。」三年命御史李琦、行人王樞齎敕責,令具篡弒之實以聞。雲南寧遠州復訴侵奪七寨,掠其婿女。遣其臣阮景真從琦等入朝謝罪,抵言未嘗僭號改元,請迎天平歸,奉為主,且退還祿州、寧遠地。帝不虞其詐,許之。命行人聶聰齎敕往諭,言:「果迎還天平,事以君禮,當建爾上公,封以大郡。」復遣景真從聰等還報,迎天平。聰力言誠可信,帝乃冬天平還國,敕廣西左、右副將軍黃中、呂毅將兵五千送之。

四年,天平陛辭,帝厚加賚,敕封順化郡公,盡食所屬州縣。三月,中等護天平入雞陵關,將至芹站,伏兵邀殺天平,中等敗還。帝大怒,召成國公朱能等謀,決意討之。七月命能佩征夷將軍印充總兵官,四平侯沐晟佩征夷副將軍印為左副將軍,新城侯張輔為右副將軍,豐城侯李彬、雲陽伯陳旭為左、右參將,督師南征。能至龍州病卒,輔代將其軍。入安南坡壘關,傳檄數一元父子二十大罪,諭國人以輔立陳氏子孫意。師次芹站,遂造浮橋於昌江以濟。前鋒抵富良江北嘉林縣,而輔由芹站西取他道至北江府新福縣,諜晟、彬軍亦自雲南至白鶴,乃遣驃騎將軍朱榮往會之。時輔等分道進兵,所至皆克。賊乃緣江樹柵,增築土城於多邦隘,城柵連九百餘里,大發江北民二百餘萬守之。諸江海口皆下木樁,所居東都,嚴守備,水陸兵號七百萬,欲持久以老官軍。輔等乃移營三帶州個招市江口,造戰艦。帝慮賊緩師以待瘴癘,敕輔等必以明年春滅賊。十二月,晟次洮江北岸,與多邦城對壘。輔遣旭攻洮江州,造浮橋濟師,遂俱抵城下,攻拔之。賊所恃惟此城,既破,膽裂。大軍循富良江南下,遂搗東都。賊棄城走,大軍入據之,薄西都。賊大燒宮室,駕舟入海。郡縣相繼納款,抗拒者輒擊破之。士民上書陳黎氏罪惡,日以百數。

五年正月大破季BX於木丸江,宣詔訪求陳氏子孫。於是耆老千一百二十餘人詣軍門,言:「陳氏為黎賊殺盡,無可繼者。安南本中國地,乞仍入職方,同內郡。」輔等以聞。尋大破賊於富良江,季BX父子以數舟遁去。諸軍水陸並追,次茶籠縣,知季BX走乂安,遂循舉厥江,追至日南州奇羅海口,命柳升出海追之。賊數敗,不能軍。五月獲季BX及偽太子於高望山,安南盡平。群臣請如耆老言,設郡縣。

六月朔,詔告天下,改安南為交阯,設三司:以都督僉事呂毅掌都司事,黃中副之,前工部侍郎張顯宗、福建布政司左參政王平為左、右布政使,前河南按察使阮友彰為按察使,裴伯耆授右參議,又命尚書黃福兼掌布、按二司事。設交州、北江、諒江、三江、建平、新安、建昌、奉化、清化、鎮蠻、諒山、新平、演州、乂安、順化十五府,分轄三十六州,一百八十一縣。又設太原、宣化、嘉興、歸化、廣威五州,直隸布政司,分轄二十九縣。其他要害,咸設衛所控制之。乃敕有司,陳氏諸王被弒者咸予贈謚,建祠治塚,各置灑掃二十戶。宗族被害者贈官,軍民死亡暴露者瘞埋之。居官者仍其舊,與新除者參治。黎氏苛政一切蠲除,遭刑者悉放免。禮待高年碩德。鰥寡孤獨無告者設養濟院。懷才抱德之彥敦遣赴京。又詔訪求山林隱逸、明經博學、賢良方正、孝弟力田、聰明正直、廉能幹濟、練達吏事、精通書算、明習兵法及容貌魁岸、詔言便利、膂力勇敢、陰陽術數、醫藥方脈諸人,悉以禮敦致,送京錄用。於是張輔等先後奏舉九千餘人。九月,季BX、蒼父子俘至闕下,與偽將相胡杜等悉屬吏。赦蒼弟衛國大王澄、子芮,所司給衣食。

六年六月,輔等振旅還京,上交阯地圖,東西一千七百六十里,南北二千八百里。安撫人民三百一十二萬有奇,獲蠻人二百八萬七千五百有奇,象、馬、牛二十三萬五千九百有奇,米粟一千三百六十萬石,船八千六百七十餘艘,軍器二百五十三萬九千八百。於是大行封賞,輔進英國公,晟黔國公,餘敘賚有差。

時中朝所置吏,務以寬厚輯新造,而蠻人自以非類,數相驚恐。陳氏故官簡定者,先降,將遣詣京師,偕其黨陳希葛逃去,與化州偽官鄧悉、阮帥等謀亂。定乃僭大號,紀元興慶,國曰大越。出沒乂安、化州山中,伺大軍還,即出攻盤灘鹹子關,扼三江府往來孔道,寇交州近境。慈廉、威蠻、上洪、天堂、應平、石室諸州縣皆嚮應,守將屢出討,皆無功。事聞,命沐晟為征夷將軍,統雲南、貴州、四川軍四萬人,由雲南征討。而遣使齎敕招降者予世官。賊不應,晟與戰生厥江,大敗,呂毅及參贊尚書劉俊死之。

七年,敗書聞,益發南畿、浙江、江西、福建、湖廣、廣東、廣西軍四萬七千人,從英國公輔征之。輔以賊負江海,不利陸師,乃駐北江仙游,大造戰艦,而撫諸遭寇逋播者,遂連破慈廉、廣威諸營柵。偵其黨鄧景異扼南策州盧渡江太平橋,乃進軍鹹子關。偽金吾將軍阮世每眾二萬,對岸立寨柵,列船六百餘艘,樹樁東南以扞蔽。時八月,西北風急,輔督陳旭、朱廣、俞讓、方政等舟齊進,炮矢飆發,斬首三千級,生擒偽監門將軍潘低等二百餘人,獲船四百餘艘。遂進擊景異,景異先走,乃定交州、北江、諒江、新安、建昌、鎮蠻諸府。追破景異太平海口,獲其黨范必栗。

時阮帥等推簡定為太上皇,別立陳季擴為帝,紀元重光。乃遣使自稱前安南王孫,求封爵。輔叱斬之,由黃江、阿江、大安海口至福成江,轉入神投海口,盡去賊所樹樁柵。十餘日抵清化,水陸畢會。定已奔演州,季擴走乂安,帥、景異等亦散亡。於是駐軍,捕餘黨。定走美良縣吉利柵,輔等窮追及之。定走入山,大索不得,遂圍之,并其偽將相陳希葛、阮汝勵、阮晏等俱就擒。

先是,賊黨阮師檜僭王,與偽金吾上將軍杜元措等據東潮州安老縣之宜陽社,眾二萬餘人。八年正月,輔進擊之,斬首四千五百餘級,擒其黨範支、陳原卿、阮人柱等二千餘人,悉斬之,築京觀。輔將班師,言:「季擴及黨阮帥、胡具、鄧景異等尚在演州、晙安,逼清化。而鄧熔塞神投福成江口,據清化要路,出沒乂安諸處。若諸軍盡還,恐沐晟兵少不敵。請留都督江浩,都指揮俞讓、花英、師祐等軍,佐晟守禦。」從之。五月,晟追季擴至虞江,賊棄柵遁。追至古靈縣及會潮、靈長海口,斬首三千餘級,獲偽將軍黎弄。季擴大蹙,奉表乞降。帝心知其詐,姑許之,詔授交阯布政使,阮帥、胡具、鄧景異、鄧熔並都指揮,陳原樽右參政,潘季祐按察副使。詔既下,念賊無悛心,九年復命輔督軍二萬四千,合晟軍討之。賊據月常江,樹樁四十餘丈,兩崖置柵二三里,列船三百餘艘,設伏山右。秋,輔、晟等水陸並進,阮帥、胡具、鄧景異、鄧金容等來拒。輔令朱廣等連艦拔樁以進,自率方政等以步隊剿其伏兵,水陸夾攻。賊大敗,帥等皆散走。生擒偽將軍鄧宗稷、黎德彞、阮忠、阮軒等,獲船百二十艘。輔乃督水軍剿季擴,聞石室、福安諸州縣偽龍虎將軍黎蕊等斷銳江浮橋阻生厥江交州後衛道路,遂往征之。蕊及范慷來拒,蕊中矢死。斬偽將軍阮陀,獲偽將軍楊汝梅、防禦使馮翕,斬首千五百級,追殺餘賊殆盡。慷及杜個旦、鄧明、阮思瑊等亦就擒。

十年,輔督方政等擊賊舟於神投海,大敗之,擒偽將軍陳磊、鄧汝戲等。阮帥等遠遁,追之不及。輔軍至乂安土黃,偽少保潘季祐等請降,率偽官十七人上謁。輔承制授季祐按察副使,署乂安府事。於是偽將軍、觀察、安撫、招討諸使陳敏、阮士勤、陳全勖、陳全敏等相繼降。明年,輔及晟合軍至順州。阮帥等設伏愛子江,而據昆傳山險,列象陣迎敵。諸軍大破之,生擒偽將軍潘徑、阮徐等五十六人,追至愛母江。賊潰散,鄧金容弟偽侯鐵及將軍潘魯、潘勤等盡降。明年春,進軍政和。賊帥胡同降,言偽大將軍景異率黨黎蟾等七百人逃暹蠻昆蒲柵。遂進羅蒙江,舍騎步行,比至,賊已遁。追至叱蒲捺柵,又遁。昏夜行二十餘里,聞更鼓聲,輔率政等銜枚疾趨,黎明抵叱蒲乾柵,江北賊猶寨南岸。官軍渡江圍之,矢中景異脅,擒之。鎔及弟鈗亡走,追擒之,盡獲其眾。別將朱廣追偽大將軍阮帥於暹蠻,大搜暹人關諸山,獲帥及季擴等家屬。帥逃南靈州,依土官阮茶匯。指揮薛聚追獲帥,斬茶匯。初,鄧鎔之就執也,季擴逃乂安竹排山。輔遣都指揮師祐襲之,走老撾。祐踵其後,老撾懼官軍躪其地,請自縛以獻。輔檄索之,令祐深入,克三關,抵金陵個,賊黨盡奔,遂獲季擴及其弟偽相國驩國王季揝,他賊盡平。明年二月,輔、晟等班師入京。四月復命輔佩征夷將軍印,出鎮。十四年召還。明年命豐城侯李彬代鎮。

交人故好亂。中官馬騏以採辦至,大索境內珍寶,人情騷動,桀黠者鼓煽之,大軍甫還,即並起為亂。陸那阮貞,順州黎核、潘強與土官同知陳可論、判官阮昭、千戶陳忷、南靈州判官阮擬、左平知縣范伯高、縣丞武萬、百戶陳已律等一時並反。彬皆遣將討滅之,而反者猶不止。俄樂巡檢黎利、四忙故知縣車綿之子三、乂安知府潘僚、南靈州千戶陳順慶、乂安衛百戶陳直誠,亦乘機作亂。其他奸宄,范軟起俄樂,武貢、黃汝曲起偈江,儂文歷起丘溫,陳木果起武定,阮特起快州,吳巨來起善誓,鄭公證、黎姪起同利,陶強起善才,丁宗老起大灣,范玉起安老,皆自署官爵,殺將吏,焚廬舍。有楊恭、阮多者,皆自稱王,署其黨韋五、譚興邦、阮嘉為太師、平章,與群寇相倚,而潘僚、范玉尤猖獗。僚者,故乂安知府季祐子也,嗣父職,不堪馬騏虐,遂反。土官指揮路文律、千戶陳苔等從之。玉為塗山寺僧,自言天降印劍,遂僭稱羅平王,紀元永寧,與范善、吳中、黎行、陶承等為亂,署為相國、司空、大將軍,攻掠城邑。彬東西征剿,日不暇給。中朝以賊久未平,十八年命榮昌伯陳智為左參將,助之。又降敕責彬曰:「叛寇潘僚、黎利、車三、儂文歷等迄今未獲,兵何時得息,民何時得安。宣廣為方略,速奏蕩平。」彬皇恐,督諸將追剿。明年秋,賊悉破滅,惟黎利不能得。

利初仕陳季擴為金吾將軍,後歸正,用為清化府俄樂縣巡檢,邑邑不得志。及大軍還,遂反,僭稱平定王,以弟石為相國,與其黨段莽、范柳、范晏等放兵肆掠。官軍討之,生擒晏等,利遁去。久之,出據可藍柵行劫。諸將方政、師祐剿獲其偽將軍阮箇立等,利逃匿老撾。及政等還,利潛出,殺玉局巡檢。已,復出掠磊江,每追擊輒遁去。及群盜盡滅,利益深匿。彬奏言:「利竄老撾,老撾請官軍毋入,黨盡發所部兵捕利。今久不遣,情叵測。」帝疑老撾匿賊,令彬送其使臣至京詰問,老撾乃逐利。二十年春,彬卒,詔智代彬。二十一年,智追利於寧化州車來縣,敗之,利復遠竄。明年秋,智奏利初逃老撾,後被逐歸瑰縣。官軍進擊,其頭目范仰等已率男婦千六百人降,利雖求撫,願以所部來歸,而止俄樂不出,造國器未已,必當進兵。奏至,會仁宗以踐阼大赦天下,因敕智善撫之,而利已寇茶籠州,敗方政軍,殺指揮伍雲。

利未叛時,與鎮守中官山壽善。至是壽還朝,力言利與己相信,今往諭之,必來歸。帝曰:「此賊狡詐,若為所紿,則其勢益熾,不易制也。」壽叩頭言:「如臣往諭,而利不來,臣當萬死。」帝頷之,遣壽齎敕授利清化知府,慰諭甚至。敕甫降,利已寇清化,殺都指揮陳忠。利得敕,無降意,即借撫愚守臣,佯言俟秋涼赴官,而寇掠不已。時洪熙改元,鑄將軍印分頒邊將,智得征夷副將軍印,又命安平伯李寧往佐之。智素無將略,憚賊,因借撫以愚中朝,且與方政迕,遂頓兵不進。賊益無所忌,再圍茶籠,智等坐視不救。閱七月,城中糧盡,巡按御史以聞,奏至而仁宗崩。宣宗初即位,敕責智及三司官。智等不為意,茶籠遂陷,知州琴彭死之。尚書掌布按二司陳洽言:「利雖乞降,內攜貳,既陷茶籠,復結玉麻土官、老撾酋長與之同惡。始言俟秋涼,今秋已過,復言與參政梁汝笏有怨,乞改授茶籠州,而遣逆黨潘僚、路文律等往嘉興、廣威諸州招集徒眾,勢日滋蔓。乞命總兵者速行剿滅。」奏上,為降敕切責,期來春平賊。智始懼,與政薄可留關,敗還,至茶籠又敗。政勇而寡謀,智懦而多忌,素不相能,而山壽專招撫,擁兵晙安不救,是以屢敗。

宣德元年春,事聞,復降敕切責。時渠魁未平,而小寇蜂起,美留潘可利助逆,宣化周莊、太原黃庵等結雲南寧遠州紅衣賊大掠。帝敕沐晟剿寧遠,又發西南諸衛軍萬五千、弩手三千赴交阯,且敕老撾不得容叛人。四月,命成山侯王通為征夷將軍,都督馬瑛為參將,往討黎利。削陳智、方政職,充為事官。通未至,賊犯清化。政不出戰,都指揮王演擊敗之。詔大赦交阯罪人,黎利、潘僚降亦授職;停採辦金銀、香貨,冀以弭賊,而賊無悛心。政督諸軍進討,李安及都指揮于瓚、謝鳳、薛聚、朱廣等先奔,政由此敗,俱謫為事官,立功贖罪。未幾,智遣都指揮袁亮擊賊黎善於廣威州,欲渡河,土官何加伉言有伏。亮不從,遣指揮陶森、錢輔等渡河,中伏並死,亮亦被執。善遂分兵三道犯交州,其攻下關者為都督陳濬所敗,攻邊江小門者為李安所敗,善夜走。通聞之,亦分兵三道出擊。馬瑛敗賊清威,至石室與通會,俱至應平寧橋。士卒行泥濘中,遇伏兵,大敗。尚書陳洽死焉,通亦中脅還。利在乂安聞之,鼓行至清潭,攻北江,進圍東關。通素無戰功,以父真死事封。朝廷不知其庸劣,誤用之。一戰而敗,心膽皆喪,舉動乖張,不奉朝命,擅割清化以南地予賊,盡撤官吏軍民還東關。惟清化知州羅通不從,利移兵攻之不下。賊分兵萬人圍隘留關,百戶萬琮奮擊,乃退。帝聞通敗,大駭,命安遠侯柳升為總兵官,保定伯梁銘副之,督師赴討,又命沐晟為征南將軍,興安伯徐亨、新寧伯譚忠為左、右副將軍,從雲南進兵,兩軍共七萬餘人。復敕通固守,俟升。

二年春,利犯交州。通與戰,斬偽太監黎秘及太尉、司徒、司空等官,獲首級萬計。利破膽奔遁,諸將請乘勢追之,通逗留三日。賊知其怯,復立寨濬濠,四出剽掠。三月復發三萬三千人,從柳升、沐晟征討。賊分兵圍丘溫,都指揮孫聚力拒之。先是,賊以昌江為大軍往來要道,發眾八萬餘人來攻,都指揮李任等力拒,殺賊甚眾。閱九月,諸將觀望不救,賊懼升大軍至,攻益力。夏四月,城陷,任死之。時賊圍交州久,通閉城不敢出,賊益易之,致書請和。通欲許之,集眾議,按察使楊時習曰:「奉命討賊,與之和,而擅退師,何以逃罪!」通怒,厲聲叱之,眾不敢言,遂以利書聞。

升奉命久,俟諸軍集,九月始抵隘留關。利既與通有成言,乃詭稱陳氏有後,率大小頭目具書詣升軍,乞罷兵,立陳氏裔。升不啟封,遣使奏聞。無何,升進薄倒馬坡,陷歿,後軍相繼盡歿。通聞,懼甚,大集軍民官吏,出下哨河,立壇與利盟誓,約退師。遂遣官偕賊使奉表及方物進獻。沐晟軍至水尾,造船將進,聞通已議和,亦引退,賊乘之,大敗。

鴻臚寺進賊與升書,略言:「高皇帝龍飛,安南首朝貢,特蒙褒賞,錫以玉章。後黎賊篡弒,太宗皇帝興師討滅,求陳氏子孫。陳族避禍方遠竄,故無從訪求。今有遺嗣皓,潛身老撾二十年,本國人民不忘先王遺澤,已訪得之。倘蒙轉達黼宸,循太宗皇帝繼絕明詔,還其爵土,匪獨陳氏一宗,實蠻邦億萬生民之幸。」帝得書頷之。明日,皓表亦至,稱「臣皓,先王暊三世嫡孫」,其詞與利書略同。帝心知其詐,欲藉此息兵,遂納其言。初,帝嗣位,與楊士奇、楊榮語交阯事,即欲棄之。至是,以表示廷臣,諭以罷兵息民意。士奇、榮力贊之,惟蹇義、夏原吉不可。然帝意已決,廷臣不敢爭。十一月朔,命禮部左寺郎李琦、工部右侍郎羅汝敬為正使,右通政黃驥、鴻臚卿徐永達為副使,齎詔撫諭安南人民,盡赦其罪,與之更新,令具陳氏後人之實以聞。因敕利以興滅繼絕之意,並諭通及三司官,盡撤軍民北還。詔未至,通已棄交阯,由陸路還廣西,中官山壽、馬騏及三司守令,由水路還欽州。凡得還者止八萬六千人,為賊所殺及拘留者不可勝計。天下舉疾通棄地殃民,而帝不怒也。

三年夏,通等至京,文武諸臣合奏其罪,廷鞫具服,乃與陳智、馬瑛、方政、山壽、馬騏及布政使弋謙,俱論死下獄,籍其家。帝終不誅,長繫待決而已。騏恣虐激變,罪尤重,而謙實無罪,皆同論,時議非之。廷臣復劾沐晟、徐亨、譚忠逗留及喪師辱國罪,帝不問。

琦等還朝,利遣使奉表謝恩,詭言皓於正月物故,陳氏子孫絕,國人推利守其國,謹俟朝命。帝亦知其詐,不欲遽封,復遣汝敬、永達諭利及其下,令訪陳氏,並盡還官吏人民及其眷屬。明年春,汝敬等還,利復言陳氏無遺種,請別命。因貢方物及代身金人。又言:「臣九歲女遭亂離散,後知馬騏攜歸充宮婢,臣不勝兒女私,冒昧以請。」帝心知陳氏即有後,利必不言,然以封利無名,復命琦、汝敬敕諭再訪,且以利女病死告之。

五年春,琦等還,利遣使貢金銀器方物,復飾詞具奏,並具頭目耆老奏請令利攝國政。使臣歸,帝復以訪陳氏裔,還中國遺民二事諭之,詞不甚堅。明年夏,利遣使謝罪,以二事飾詞對,復進頭目耆老奏,仍為利乞封。帝乃許之,命禮部右侍郎章敞、右通政徐琦齎敕印,命利權署安南國事。利遣使齎表及金銀器方物,隨敞等入貢。七年二月達京師,比還,利及使臣皆有賜。明年八月來貢,命兵部侍郎徐琦等與其使偕行,諭以順天保民之道。是年,利卒。利雖受敕命,其居國稱帝,紀元順天,建東、西二都,分十三道:「曰山南、京北、山西、海陽、安邦、諒山、太原、,明光、諒化、清華、晙安、,順化、廣南。各設承政司、憲察司、總兵使司,擬中國三司。東都在交州府,西都在清華府。置百官,設學校,以經義、詩賦二科取士,彬彬有華風焉。僭位六年,私謚太祖。子麟繼,麟一名龍。自是其君長皆有二名,以一名奏天朝,貢獻不絕如常制。麟遣使告訃,命侍郎章敞、行人侯璡敕麟權署國事。明年遣使入貢謝恩。

正統元年四月以宣宗賓天,遣使進香。又以英宗登極及尊上太皇太后、皇太后位號,並遣使表賀,貢方物。閏六月復貢。帝以陳氏宗支既絕,欲使麟正位,下廷議,咸以為宜。乃命兵部右侍郎李郁、左通政李亨齎敕印,封麟為安南國王。明年遣使入貢謝恩。時安南思郎州土官攻掠廣西安平、思陵二州,據二峒二十一村。帝命給事中湯鼐、行人高寅敕麟還侵地。麟奉命,遣使謝罪,而訴安平、思陵土官侵掠思郎。帝令守臣嚴飭。七年,安南貢使還,令齎皮弁冠服、金織襲衣賜其王。是歲,麟卒,私謚太宗。改元二:紹平六年,大寶三年。子浚繼,一名基隆,遣使告訃。命光祿少卿宋傑、兵科都給事中薛謙持節冊封為國王。濬遣將侵占城,奪新州港,擄其王摩訶賁該以歸。帝為立新王摩訶貴來,敕安南使,諭濬歸其故王。浚不奉詔,侵掠人口至三萬三千餘,占城入訴。

景泰元年賜敕戒濬,迄不奉詔。四年遣使賀冊立皇太子。天順元年遣使入貢,乞賜袞冕,如朝鮮例,不從。其使者乞以土物易書籍、藥材,從之。二年遣使賀英宗復辟。三年十月,其庶兄諒山王琮弒之而自立。濬改元二:大利十一年,延寧六年。私謚仁宗。琮,一名宜民,篡位九月,改元天與,為國人所誅,貶厲德侯,以濬弟灝繼。灝,一名思誠。初,琮弒濬,以游湖溺死奏。天朝不知,將遣官弔祭。琮恐天使至覺其情,言禮不弔溺,不敢煩天使,帝即已之。使者言浚無子,請封琮。命通政參議尹旻、禮科給事中王豫往封。未入境,聞琮已誅,灝嗣位,即卻還。灝連遣使朝貢請封,禮官疑其詐,請命廣西守臣核實奏請,從之。使臣言:「禮,生有封,死有祭。今浚死既白,請賜祭。」乃命行人往祭。六年二月命侍讀學士錢溥、給事中王豫封灝為國王。

憲宗踐阼,命尚寶卿凌信、行人邵震賜王及妃綵幣。灝遣使來貢,因請冕服,不從,但賜皮弁冠服及紗帽犀帶。成化元年八月以英宗賓天,遣使進香,命赴裕陵行禮。

灝雄桀,自負國富兵強,輒坐大。四年侵據廣西憑祥。帝聞,命守臣謹備之。七年破占城,執其王盤羅茶全,逾三年又破之,執其王盤羅茶悅,遂改其國為交南州,設兵戍守。安南貢道,故由廣西。時雲南鎮守中官錢能貪恣,遣指揮郭景齎敕取其貨。灝素欲窺雲南,遂以解送廣西龍州罪人為詞,隨景假道雲南入京,索夫六百餘,且發兵繼其後,雲南大擾。兵部言雲南非貢道,龍州罪人宜解廣西,不必赴京。乃令守臣檄諭,且嚴邊備。灝既得憑祥,滅占城,遂侵廣東瓊、雷,盜珠池。廣西之龍州、右平,雲南之臨安、廣南、鎮安,亦數告警。詔守臣詰之,輒詭詞對。廟堂務姑息,雖屢降敕諭,無厲詞。灝益玩侮無畏忌,言:「占城王盤羅茶全侵化州道,為其弟盤羅茶悅所弒,因自立。及將受封,又為子茶質苔所弒。其國自亂,非臣灝罪。」中朝知其詐,不能詰,但勸令還其土宇。灝奏言:「占城非沃壤,家鮮積貯,野絕桑麻,山無金寶之收,海乏魚鹽之利,止產象牙、犀角、烏木、沉香。得其地不可居,得其民不可使,得其貨不足富,此臣不侵奪占城故也。明詔令臣復其土宇,乞遣朝使申畫郊圻,俾兩國邊陲休息,臣不勝至願。」時占城久為所據,而其詞誕如此。

先是,安南入貢,多攜私物,道憑祥、龍州,乏人轉運,輒興仇釁。會遣使賀冊立皇太子,有詔禁飭之。十五年冬,灝遣兵八百餘人,越雲南蒙自界,聲言捕盜,擅結營築室以居。守臣力止之,始退。灝既破占城,志意益廣,親督兵九萬,開山為三道,攻破哀牢,侵老撾,復大破之,殺宣慰刀板雅、蘭、掌父子三人,其季子怕雅賽走八百以免。灝復積糧練兵,頒偽敕於車里,徵其兵合攻八百。將士暴死者數千,咸言為雷霆所擊。八百乃遏其歸路,襲殺萬餘人,灝始引還。帝下廷議,請令廣西布政司檄灝斂兵,雲南、兩廣守臣戒邊備而已。既而灝言未侵老撾,且不知八百疆宇何在,語甚誑誕。帝復慰諭之,迄不奉命。十七年秋,滿剌加亦以被侵告,帝敕使諭令睦鄰保國。未幾,使臣入貢,請如暹羅、爪哇例賜冠帶。許之,不為例。

孝宗踐阼,命侍讀劉戩詔諭其國。其使臣來貢,以大喪免引奏。弘治三年,時占城王古來以天朝力得還國,復愬安南見侵。兵部尚書馬文升召安南使臣曰:「歸諭爾主,各保疆土享太平。不然,朝廷一旦赫然震怒,天兵壓境,如永樂朝事,爾主得無悔乎?」安南自是有所畏。十年,灝卒,私謚聖宗。其改元二:光順十年,洪德二十八年。子暉繼,一名鏳,遣使告訃,命行人徐鈺往祭。尋賜暉皮弁服、金犀帶。其使臣言,國主受王封,賜服與臣下無別,乞改賜。禮官言:「安南名為王,實中國臣也。嗣王新立,必賜皮弁冠服,使不失主宰一國之尊,又賜一品常服,俾不忘臣事中國之義。今所請,紊亂祖制,不可許。然此非使臣罪,乃通事者導之妄奏,安懲。」帝特宥之。十七年,暉卒,私謚憲宗,其改元曰景統。子水牽繼,一名敬甫,七月而卒,私謚肅宗。弟誼繼,一名璿。

武宗踐阼,命修撰倫文敘、給事中張弘至詔諭其國。誼亦遣使告訃,命官致祭如常儀。正德元年冊為王。誼寵任母黨阮種、阮伯勝兄弟,恣行威虐,屠戮宗親,鴆殺祖母。種等怙寵竊權,四年逼誼自殺,擁立其弟伯勝,貶誼為厲愍王。國人黎廣等討誅之,立灝孫晭,改謚誼威穆帝。誼在位四年,改元端慶。晭,一名瀅,七年受封,多行不義。十一年,社堂燒香官陳皓與二子昺、昇作亂,殺晭而自立。詭言前王陳氏後,仍稱大虞皇帝,改元應天,貶晭為靈隱王。晭臣都力士莫登庸初附皓,後與黎氏大臣阮私裕等起兵討之。皓敗走,獲昺及其黨陳璲等。皓與昇奔諒山道,據長寧、太原、清節三府自保。登庸等乃共立晭兄灝之子譓,改謚晭襄翼帝。晭在位七年,改元洪順。譓將請封,因國亂不果。以登庸有功,封武川伯,總水陸諸軍。既握兵柄,潛蓄異志。黎氏臣鄭綏,以譓徒擁虛位,別立其族子酉榜,發兵攻都城。譓出走,登庸擊破綏兵,捕酉榜殺之,益恃功專恣,遂逼妻譓母,迎譓歸,自為太傅仁國公。十六年率兵攻陳皓,皓敗走死。

嘉靖元年,登庸自稱安興王,謀弒譓。譓母以告,乃與其臣杜溫潤間行以免,居於清華。登庸立其庶弟廣,遷居海東長慶府。世宗踐阼,命編修孫承恩、給事中俞敦詔諭其國。至龍州,聞其國大亂,道不通,乃卻還。四年夏,譓遣使間道通貢,並請封,為登庸所阻。明年春,登庸賂欽州判官唐清,為廣求封。總督張嵿逮清,死於獄。六年,登庸令其黨危范嘉謨偽為廣禪詔,篡其位,改元明德,立子方瀛為皇太子。旋鴆殺■,謚為恭皇帝。踰年,遣使來貢,至諒山城,被攻而還。九年,登庸禪位於方瀛,自稱太上皇,移居都齋、海陽,為方瀛外援,作《大誥》五十九條,頒之國中。方瀛改元大正。其年九月,黎譓卒於清華,國亡。

十五年冬,皇子生,當頒詔安南。禮官夏言言:「安南不貢已二十年,兩廣守臣謂黎譓、黎■均非黎晭應立之嫡,莫登庸陳皓俱彼國纂逆之臣,宜遣官按問,求罪人主名。且前使既以道阻不通,今宜暫停使命。帝以安南叛逆昭然,宜急遣官往勘,命言會兵部議征討。言及本兵張瓚等力言逆臣篡主奪國,朝貢不修,決宜致討。乞先遣錦衣官二人往核其實,敕兩廣、雲南守臣整兵積餉,以俟師期,制可。乃命千戶陶鳳儀、鄭璽等,分往廣西、雲南,詰罪人主名,敕四川、貴州、湖廣、福建、江西守臣,預備兵食,候徵調。戶部侍郎唐胄上疏,力陳用兵七不可,語詳其傳中,末言:「安南雖亂,猶頻奉表箋,具方物,款關求入。守臣以其姓名不符,拒之。是彼欲貢不得,非負固不貢也。」章下兵部,亦以為然,命俟勘官還更議。

十六年,安南黎寧遣國人鄭惟僚等赴京,備陳登庸篡弒狀,言:「寧即譓子。譓卒,國人立寧為世孫,權主國事。屢馳書邊臣告難,俱為登庸邀殺。乞興師問罪,亟除國賊。」時嚴嵩掌禮部,謂其言未可盡信,請羈之,待勘官回奏,從之。尋召鳳儀等還,命禮、兵二部會廷臣議,列登庸十大罪,請大振宸斷,剋期徂征。乃起右都御史毛伯溫於家,參贊軍務,命戶部侍郎胡璉、高公韶先馳雲、貴、兩廣調度軍食,以都督僉事江桓、牛桓為左、右副總兵,督軍征討,其大將需後命。兵部復奉詔,條用兵機宜十二事。獨侍郎潘珍持不可,抗疏切諫。帝怒,褫其職。兩廣總督潘旦亦馳疏請停前命,言:「朝廷方興問罪之師,登庸即有求貢之使,宜因而許之,戒嚴觀變,以待彼國之自定。」嚴嵩、張瓚窺帝旨,力言不可宥,且言黎寧在清都圖恢復,而旦謂彼國俱定,上表求貢,決不可許。旦疏遂寢。五月,伯溫至京,奏上方略六事,以旦不可共事,請易之,優旨褒答。及兵部議上,帝意忽中變,謂黎寧誠偽未審,令三方守臣從宜撫剿,參贊、督餉大臣俱暫停,旦調用,以張經代之。時御史徐九皋、給事中謝廷■以修省陳言,亦請罷征南之師。八月,雲南巡撫汪文盛以獲登庸間諜及所撰偽《大誥》上聞。帝震怒,命守臣仍遵前詔征討。時文盛招納黎氏舊臣武文淵得其進兵地圖,謂登庸以可破,遂上之朝。廣東按臣餘光言:「莫之篡黎,猶黎之篡陳,不足深較。但當罪其不庭,責以稱臣修貢,不必遠征,疲敝中國。臣已遣使宣諭,彼如來歸,宜因以撫納。」帝以光輕率,奪祿一年。文盛即傳檄安南,登庸能束身歸命,籍上輿圖,待以不死。於是登庸父子遣使奉表乞降,且投牒文盛及黔國公沐朝輔,具述黎氏衰亂,陳皓叛逆,己與方瀛有功,為國人歸附,所有土地,已載《一統志》中,乞貰其罪,修貢如制。朝輔等以十七年三月奏聞,而黎寧承前詔,懼天朝竟納其降,備以本國篡弒始末及軍馬之數、水陸進兵道里來上。俱下兵部,集廷臣議。僉言莫氏罪不可赦,亟宜進師。請以原推咸寧侯仇鸞總督軍務,伯溫仍為參贊,從之。張經上言:「安南進兵之道有六,兵當用三十萬,一歲之餉當用百六十萬,造舟、市馬、制器、犒軍諸費又須七十餘萬。況我調大眾,涉炎海,與彼勞逸殊勢,不可不審處也。」疏方上,欽州知州林希元又力陳登庸可取狀。兵部不能決,復請廷議。及議上,帝不悅曰:「朕聞卿士大夫私議,咸謂不當興師。爾等職司邦政,漫無主持,悉委之會議。既不協心謀國,其已之。鸞、伯溫別用。」

十八年冊立皇太子,當頒詔安南。特起黃綰為禮部尚書,學士張治副之,往使其國。命甫下,方瀛遣使上表降,並籍其土地、戶口,聽天朝處分,凡為府五十有三,州四十有九,縣一百七十有六。帝納之,下禮、兵二部協議。至七月,綰猶未行,以忤旨落職,遂停使命。初,征討之議發自夏言,帝既責綰,因發怒曰:「安南事,本一人倡,眾皆隨之。乃訕上聽言計,共作慢詞。此國應棄應討,宜有定議,兵部即集議以聞。」於是瓚及廷臣惶懼,請如前詔,仍遣鸞、伯溫南征。如登庸父子束手歸命,無異心,則待以不死,從之。登庸聞,大喜。

十九年,伯溫等抵廣西,傳檄諭以納款宥罪意。時方瀛已卒,登庸即遣使請降。十一月率從子文明及部目四十二人入鎮南關,囚首徒跣,匍匐叩頭壇上,進降表,伯溫稱詔赦之。復詣軍門匍匐再拜,上土地軍民藉,請奉正朔,永為籓臣。伯溫等宣示威德,令歸國俟命。疏聞,帝大喜,命削安南國為安南都統使司,授登庸都統使,秩從二品,銀印。舊所僭擬制度悉除去,改其十三道為十三宣撫司,各設宣撫、同知、副使、僉事,聽都統黜陟。廣西歲給《大統曆》,仍三歲一貢以為常。更令核黎寧真偽,果黎氏後,割所據四府奉其祀事,否則已之。制下,登庸悚惕受命。

二十二年,登庸卒,方瀛子福海嗣,遣宣撫同知阮典敬等來朝。二十五年,福海卒,子宏瀷嗣。初,登庸以石室人阮敬為義子,封西寧侯。敬有女嫁方瀛次子敬典,因與方瀛妻武氏通,得專兵柄。宏瀷立,方五歲,敬益專恣用事。登庸次子正中及文明避之都齋,其同輩阮如桂、范子儀等亦避居田里。敬舉兵逼都齋,正中、如桂、子儀等禦之,不勝。正中、文明率家屬奔欽州,子儀收殘卒遁海東。敬詭稱宏瀷歿,以迎立正中為詞,犯欽州,為參將俞大猷所敗,誅死。宏瀷初立時,遣使黎光賁來貢,至南寧,守臣以聞。禮官以其國內亂,名分未定,止來使勿進,而令守臣核所當立者。至三十年事白,命授宏瀷都統使,赴關領牒。會部目黎伯驪與黎寧臣鄭檢合兵來攻,宏瀷奔海陽,不克赴。光賁等留南寧且十五年,其偕來使人物故大半。宏瀷祈守臣代請,詔許入京,其都統告身,仍俟宏瀷赴關則給。四十三年,宏瀷卒,子茂洽嗣。萬曆元年授都統使。三年遣使謝恩,賀即位,進方物,又補累年所缺之貢。

時莫氏漸衰,黎氏復興,互相構兵,其國益多故。始黎寧之據清華也,仍僭帝號,以嘉靖九年改元元和。居四年,為登庸所攻,竄占城界。國人立其弟憲,改元光照。十五年廉知寧所在,迎歸清華,後遷於漆馬江。寧卒,其臣鄭檢立寧子寵。寵卒,無子,國人共立黎暉四世孫維邦。維邦卒,檢子松立其子維潭,世居清華,自為一國。

萬曆十九年,維潭漸強,舉兵攻茂洽,茂洽敗奔嘉林縣。明年冬,松誘土人內應,襲殺茂洽,奪其都統使印,親黨多遇害。有莫敦讓者,奔防城告難,總督陳蕖以聞。松復擒敦讓,勢益張。茂洽子敬恭與宗人履遜等奔廣西思陵州,莫履機奔欽州。獨莫敬邦有眾十餘萬,起京北道,擊走黎黨范拔萃、范百祿諸軍,敦讓得復歸。眾乃推敬邦署都統,諸流寓思陵、欽州者悉還。黎兵攻南策州,敬邦被殺,莫氏勢益衰。敬恭、敬用屯諒山高平,敬璋屯東海新安,懼黎兵追索,竄至龍州、憑祥界,令土官列狀告當事。維潭亦叩關求通貢,識以國王金印。

二十一年,廣西巡撫陳大科等上言:「蠻邦易姓如弈棋,不當以彼之叛服為順逆,止當以彼之叛我服我為順逆。今維潭雖圖恢復,而茂洽固天翰外臣也,安得不請命而手間然戮之。竊謂黎氏擅興之罪,不可不問。莫氏孑遺之緒,亦不可不存。倘如先朝故事,聽黎氏納款,而仍存莫氏,比諸漆馬江,亦不翦其祀,於計為便。」廷議如其言。明年,大科方遣官往察,敬用即遣使叩軍門告難,且乞兵。明年秋,維潭亦遣使謝罪,求款。時大科已為兩廣總督,與廣西巡撫戴耀並以屬左江副使楊寅秋,寅秋竊計曰:「不拒黎,亦不棄莫,吾策定矣。」兩遣官往問,以敬恭等願居高平來告,而維潭求款之使亦數至。寅秋乃與之期,具報督撫。會敬璋率眾赴永安,為黎氏兵擊敗,海東、新安地盡失,於是款議益決。時維潭圖恢復名,不欲以登庸自處,無束身入關意。寅秋復遣官諭之,其使者來報如約,至期忽言於關吏曰:「士卒饑病,款儀未備。且莫氏吾仇也,棲之高平,未敢聞命。」遂中宵遁去。大科等疏聞,謂其臣鄭松專權所致。維潭復遣使叩關,自己非遁。大科等再遣官諭之,維潭聽命。

二十五年遣使請期,寅秋示以四月。郕期,維潭至關外,譯者詰以六事。首擅殺茂洽,曰:「復仇急,不遑請命。」次維潭宗派,曰:「世孫也,祖暉,天朝曾錫命。」次鄭松,曰:「此黎氏世臣,非亂黎氏也。」然則何宵遁,曰:「以儀物之不戒,非遁也。」何以用王章,曰:「權仿為之,立銷矣。」惟割高平居莫氏,猶相持不絕。復諭之曰:「均貢臣也,黎昔可棲漆馬江,莫獨不可棲高平乎?」乃聽命。授以款關儀節,俾習之。維潭率其下入關謁御幄,一如登庸舊儀。退謁寅秋,請用賓主禮,不從,四拜成禮而退。安南復定。詔授維潭都統使,頒曆奉貢。一如莫氏故事。先是,黎利及登庸進代身金人,皆囚首面縛,維潭以恢復名正,獨立而肅容。當事嫌其倨,令改製,乃為俯伏狀,鐫其背曰:「安南黎氏世孫,臣黎維潭不得蒲伏天門,恭進代身金人,悔罪乞恩。」自是,安南復為黎氏有,而莫氏但保高平一郡。

二十七年,維潭卒,子維新嗣,鄭松專其柄。會叛酋潘彥構亂,維新與松移保清化。三十四年遣使入貢,命授都統使。時莫氏宗黨多竄處海隅,往往僭稱公侯伯名號,侵軼邊境,維新亦不能制。守臣檄問,數發兵夾剿,雖應時破滅,而邊方頗受其害。維新卒,子維祺嗣。天啟四年,發兵擊莫敬寬,克之,殺其長子,掠其妻妾及少子以歸。敬寬與次子逃入山中,復回高平,勢益弱。然迄明之世,二姓分據,終不能歸一云。

安南都會在交州,即唐都護治所。其疆域東距海,西接老撾,南渡海即占城,北連廣西之思明、南寧,雲南之臨安、元江。土膏腴,氣候熱,穀歲二稔。人性獷悍。驩、演二州多文學,交、愛二州多倜儻士,較他方為異。


\end{pinyinscope}