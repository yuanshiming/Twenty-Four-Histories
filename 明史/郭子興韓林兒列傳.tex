\article{郭子興、韓林兒列傳}



郭子興,其先曹州人。父郭公,少以日者術遊定遠,言禍福輒中。邑富人有瞽女無所歸,郭公乃娶之,家日益饒。生三子,子興其仲出。始生,郭公卜之吉。及長,任俠,喜賓客。會元政亂,子興散家資,椎牛釃酒,與壯士結納。至正十二年春,集少年數千人,襲據濠州。太祖往從之。門者疑其諜,執以告子興。子興奇太祖狀貌,解縛與語,收帳下。為十夫長,數從戰有功。子興喜,其次妻小張夫人亦指目太祖曰:「此異人也。」乃妻以所撫馬公女,是為孝慈高皇后。

始,子興同起事者孫德崖等四人,與子興而五,各稱元帥不相下。四人者粗而戇,日剽掠,子興意輕之。四人不悅,合謀傾子興。子興以是多家居不視事。太祖乘閑說曰:「彼日益合,我益離,久之必為所制。」子興不能從也。

元師破徐州,徐帥彭大、趙均用帥餘眾奔濠。德崖等以其故盜魁有名,乃共推奉之,使居己上。大有智數,子興與相厚而薄均用。於是德崖等譖諸均用曰:「子興知有彭將軍耳,不知有將軍也。」均用怒,乘間執子興,幽諸德崖家。太祖自他部歸,大驚,急帥子興二子訴於大。大曰:「吾在,孰敢魚肉而翁者!」與太祖偕詣德崖家,破械出子興,挾之歸。元師圍濠州,乃釋故憾,共城守五閱月。圍解,大、均用皆自稱王,而子興及德崖等為元帥如故。未幾,大死,子早住領其眾。均用專狠益甚,挾子興攻盱眙、泗州,將害之。太祖已取滁,乃遣人說均用曰:「大王窮迫時,郭公開門延納,德至厚也。大王不能報,反聽細人言圖之,自剪羽翼,失豪傑心,竊為大王不取。且其部曲猶眾,殺之得無悔乎?」均用聞太祖兵甚盛,心憚之,太祖又使人賂其左右,子興用是得免,乃將其所部萬餘就太祖於滁。

子興為人梟悍善鬥,而性悻直少容。方事急,輒從太祖謀議,親信如左右手。事解,即信讒疏太祖。太祖左右任事者悉召之去,稍奪太祖兵柄。太祖事子興愈謹。將士有所獻,孝慈皇后輒以貽子興妻。子興至滁,欲據以自王。太祖曰:「滁四面皆山,舟楫商旅不通,非可旦夕安者也。」子興乃已。及取和州,子興命太祖統諸將守其地。德崖饑,就食和境,求駐軍城中,太祖納之。有讒於子興者。子興夜至和,太祖來謁,子興怒甚,不與語。太祖曰:「德崖嘗困公,宜為備。」子興默然。德崖聞子興至,謀引去。前營已發,德崖方留視後軍,而其軍與子興軍斗,多死者。子興執德崖,太祖亦為德崖軍所執。子興聞之,大驚,立遣徐達往代太祖,縱德崖還。德崖軍釋太祖,達亦脫歸。子興憾德崖甚,將甘心焉,以太祖故強釋之,邑邑不樂。未幾,發病卒,歸葬滁州。

子興三子。長子前戰死,次天敘、天爵。子興死,韓林兒檄天敘為都元帥,張天祐及太祖副之。天祐,子興婦弟也。太祖渡江,天敘、天祐引兵攻集慶,陳野先叛,俱被殺。林兒復以天爵為中書右丞。已而太祖為平章政事。天爵失職怨望,久之謀不利於太祖,誅死,子興後遂絕。有一女,小張夫人出者,事太祖為惠妃,生蜀、谷、代三王。

洪武三年追封子興為滁陽王,詔有司建廟,用中牢祀,復其鄰宥氏,世世守王墓。十六年,太祖手書子興事跡,命太常丞張來儀文其碑。滁人郭老舍者,宣德中以滁陽王親,朝京師。弘治中,有郭琥自言四世祖老舍,滁陽王第四子,予冠帶奉祀。已,為宥氏所訐。禮官言:「滁陽王祀典,太祖所定,曰無後,廟碑昭然,老舍非滁陽王子。」奪奉祀。



韓林兒,欒城人,或言李氏子也。其先世以白蓮會燒香惑眾,謫徙永年。元末,林兒父山童鼓妖言,謂「天下當大亂,彌勒佛下生」。河南、江、淮間愚民多信之。潁州人劉福通與其黨杜遵道、羅文素、盛文郁等復言「山童,宋徽宗八世孫,當主中國」。乃殺白馬黑牛,誓告天地,謀起兵,以紅巾為號。至正十一年五月,事覺,福通等遽入潁州反,而山童為吏所捕誅。林兒與母楊氏逃武安山中。福通據朱皋,破羅山、上蔡、真陽、確山,犯葉、舞陽,陷汝寧、光、息,眾至十餘萬,元兵不能禦。時徐壽輝等起蘄、黃,布王三、孟海馬等起湘、漢,芝麻李起豐、沛,而郭子興亦據濠應之。時皆謂之「紅軍」,亦稱「香軍」。

十五年二月,福通物色林兒,得諸碭山夾河;迎至亳,僭稱皇帝,又號小明王,建國曰宋,建元龍鳳。拆鹿邑太清宮材,治宮闕於亳。尊楊氏為皇太后,遵道、文郁為丞相,福通、文素平章政事,劉六知樞密院事。劉六者,福通弟也。遵道寵用事。福通嫉之,陰命甲士撾殺遵道,自為丞相,加太保,事權一歸福通。既而元師大敗福通於太康,進圍亳,福通挾林兒走安豐。未幾,兵復盛,遣其黨分道略地。

十七年,李武、崔德陷商州,遂破武關以圖關中,而毛貴陷膠、萊、益都、濱州,山東郡邑多下。是年六月,福通帥眾攻汴梁,且分軍三道:關先生、破頭潘、馮長舅、沙劉二、王士誠趨晉、冀;白不信、大刀敖、李喜喜趨關中;毛貴出山東北犯。勢銳甚。田豐者,元鎮守黃河義兵萬戶也,叛附福通,陷濟寧,尋敗走。其秋,福通兵陷大名,遂自曹、濮陷衛輝。白不信、大刀敖、李喜喜陷興元,遂入鳳翔,屢為察罕帖木兒、李思齊所破,走入蜀。

十八年,田豐復陷東平、濟寧、東昌、益都、廣平、順德。毛貴亦數敗元兵,陷清、滄,據長蘆鎮,尋陷濟南;益引兵北,殺宣慰使董搏霄於南皮,陷薊州,犯漷州,略柳林以逼大都。順帝征四方兵入衛,議欲遷都避其鋒,大臣諫乃止。貴旋被元兵擊敗,還據濟南。而福通出沒河南北,五月攻下汴梁,守將竹貞遁去,遂迎林兒都焉。關先生、破頭潘等又分其軍為二,一出絳州,一出沁州。踰太行,破遼、潞,遂陷冀寧;攻保定不克,陷完州,掠大同、興和塞外諸郡,至陷上都,毀諸宮殿,轉掠遼陽,抵高麗。十九年陷遼陽,殺懿州路總管呂震。順帝以上都宮闕盡廢,自此不復北巡。李喜喜餘黨復陷寧夏,略靈武諸邊地。

是時承平久,川郡皆無守備。長吏聞賊來,輒棄城遁,以故所至無不摧破。然林兒本起盜賊,無大志,又聽命福通,徒擁虛名。諸將在外者率不遵約束,所過焚劫,至啖老弱為糧,且皆福通故等夷,福通亦不能制。兵雖盛,威令不行。數攻下城邑,元兵亦數從其後復之,不能守。惟毛貴稍有智略。其破濟南也,立賓興院,選用元故官姬宗周等分守諸路。又於萊州立屯田三百六十所,每屯相距三十里,造挽運大車百輛,凡官民田十取其二。多所規畫,故得據山東者三年。及察罕帖木兒數破賊,盡復關、隴,是年五月大發秦、晉之師會汴城下,屯杏花營,諸軍環城而壘。林兒兵出戰輒敗,嬰城守百餘日,食將盡。福通計無所出,挾林兒從百騎開東門遁還安豐,後宮官屬子女及符璽印章寶貨盡沒於察罕。時毛貴已為其黨趙均用所殺,有續繼祖者,又殺均用,所部自相攻擊。獨田豐據東平,勢稍強。

二十年,關先生等陷大寧,復犯上都。田豐陷保定,元遣使招之,被殺。王士誠又躪晉、冀。元將孛羅敗之於臺州,遂入東平與豐合。福通嘗責李武、崔德逗撓,將罪之。二十一年夏,兩人叛去,降於李思齊。時李喜喜、關先生等東西轉戰,已多走死,餘黨自高麗還寇上都,孛羅復擊降之。而察罕既取汴梁,遂遣子擴廓討東平,脅降田豐、王士誠,乘勝定山東。惟陳猱頭者,獨守益都不下,與福通遙為聲援。

二十二年六月,豐、士誠乘閒刺殺察罕,入益都。元以兵柄付擴廓,圍城數重,猱頭等告急。福通自安豐引兵赴援,遇元師於火星埠,大敗走還。元兵急攻益都,穴地道以入,殺豐、士城,而械送猱頭於京師,林兒勢大窘。明年,張士誠將呂珍圍安豐,林兒告急於太祖。太祖曰:「安豐破則士誠益強。」遂親帥師往救,而珍已入城殺福通。太祖擊走珍,以林兒歸,居之滁州。明年,太祖為吳王。又二年,林兒卒。或曰太祖命廖永忠迎林兒歸應天,至瓜步,覆舟沉於江云。

初,太祖駐和陽,郭子興卒,林兒牒子興子天敘為都元帥,張天祐為右副元帥,太祖為左副元帥。時太祖以孤軍保一城,而林兒稱宋後,四方響應,遂用其年號以令軍中。林兒歿,始以明年為吳元年。其年,遣大將軍定中原,順帝北走,距林兒亡僅歲餘。林兒僭號凡十二年。

贊曰:元之末季,群雄蜂起。子興據有濠州,地偏勢弱。然有明基業,實肇於滁陽一旅。子興之封王祀廟,食報久長,良有以也。林兒橫據中原,縱兵蹂躪,蔽遮江、淮十有餘年。太祖得以從容締造者,藉其力焉。帝王之興,必有先驅者資之以成其業,夫豈偶然哉!
