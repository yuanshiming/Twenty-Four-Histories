\article{仁宗本紀}

\begin{pinyinscope}
仁宗敬天體道純誠至德弘文欽武章聖達孝昭皇帝,諱高熾,成祖長子也。母仁孝文皇后,夢冠冕執圭者上謁。寤而生帝。幼端重沉靜,言動有經。稍長習射,發無不中。好學問,從儒臣講論不輟。

洪武二十八年,冊為燕世子。嘗命與秦、晉、周三世子分閱衛士,還獨後。問之。對曰:「旦寒甚,俟朝食而後閱,故後。」又命分閱章奏,獨取切軍民利病者白之。或文字謬誤,不以聞。太祖指示之曰:「兒忽之耶?」對曰:「不敢忽,顧小過不足瀆天德。」又嘗問:「堯、湯時水旱,百姓奚恃?」對曰:「恃聖人有恤民之政。」太祖喜曰:「孫有君人之識矣。」

成祖舉兵,世子守北平,善拊士卒,以萬人拒李景隆五十萬眾,城賴以全。先是郡王高煦、高燧俱以慧黠有寵於成祖。而高煦從軍有功,宦寺黃儼等復黨高燧,陰謀奪嫡,譖世子。會朝廷賜世子書,為離間。世子不啟緘,馳上之。而儼先潛報成祖曰:「世子與朝廷通,使者至矣。」無何,世子所遣使亦至。成祖發書視之,乃歎曰:「幾殺吾子。」成祖踐阼,以北平為北京,仍命居守。

永樂二年二月,始召至京,立為皇太子。成祖數北征,命之監國,裁決庶政。四方水旱饑饉,輒遣振恤,仁聞大著。而高煦、高燧與其黨日伺隙讒構。或問太子:「亦知有讒人平?」曰:「不知也,吾知盡子職而已。」

十年,北征還,以太子遣使後期,且書奏失辭,悉徵宮僚黃淮竺下獄。十五年,高煦以罪徙安樂。明年,黃儼等復譖太子擅赦罪人,宮僚多坐死者。侍郎胡濙奉命察之,密疏太子誠敬孝謹七事以聞,成祖意乃釋。其後黃儼等謀立高燧,事覺伏誅,高燧以太子力解得免,自是太子始安。

二十二年七月,成祖崩於榆木川。八月甲辰,遺詔至,遣皇太孫迎喪開平。丁未,出夏原吉等於獄。丁巳,即皇帝位。大赦天下,以明年為洪熙元年。罷西洋寶船、迤西市馬及雲南、交阯採辦。戊午,復夏原吉、吳中官。己未,武安侯鄭亨鎮大同,保定侯孟瑛鎮交阯,襄城伯李隆鎮山海,武進伯朱榮鎮遼東。復設三公、三孤官,以公、侯、伯、尚書兼之。進楊榮太常寺卿,金幼孜戶部侍郎,兼大學士如故,楊士奇為禮部左侍郎兼華蓋殿大學士,黃淮通政使兼武英殿大學士,俱掌內制,楊溥為翰林學士。辛酉,鎮遠侯顧興祖充總兵官,討廣西叛蠻。甲子,汰冗官。乙丑,召漢王高煦赴京。戊辰,官吏謫隸軍籍者放還鄉。己巳,詔文臣年七十致仕。九月癸酉,交阯都指揮方政與黎利戰於茶籠州,敗績,指揮同知伍雲力戰死。丙子,召尚書黃福於交阯。庚辰,河溢開封,免稅糧,遣右都御史王彰撫恤之。壬午,敕自今官司所用物料於所產地計直市之,科派病民者罪不宥。癸未,禮部尚書呂震請除服,不許。乙酉,增諸王歲祿。丙戌,以風憲官備外任,命給事中蕭奇等三十五人為州縣官。丁亥,黎利寇清化,都指揮同知陳忠戰死。戊子,始設南京守備,以襄城伯李隆為之。乙未,散畿內民所養官馬於諸衛所。戊戌,賜吏部尚書蹇義及楊士奇、楊榮、金幼孜銀章各一,曰「繩愆糾繆」,諭以協心贊務,凡有闕失當言者,用印密封以聞。

冬十月壬寅,罷市民間金銀,革兩京戶部行用庫。癸卯,詔天下都司衛所修治城池。戊申,通政使請以四方雨澤章奏送給事中收貯。帝曰:「祖宗令天下奏雨澤假相即「偶像」。,欲知水旱,以施恤民之政。積之通政司,既失之矣,今又令收貯,是欲上之人終不知也。自今奏至即以聞。」己酉,冊妃張氏為皇后。壬子,立長子瞻基為皇太子。封子瞻埈為鄭五,瞻墉越王,瞻墡襄王,瞻堈荊王,瞻墺淮王,瞻塏滕王,瞻垍梁王,瞻埏衛王。乙卯,詔中外官舉賢才,嚴舉主連坐法。丁巳,令三法司會大學士、府、部、通政、六科於承天門錄囚,著為令。庚申,增京官及軍士月廩。丁卯,擢監生徐永潛等二十人為給事中。十一月壬申朔,詔禮部:「建文諸臣家屬在教坊司、錦衣衛、浣衣局及習匠、功臣家為奴者,悉宥為民,還其田土。言事謫戍者亦如之。」癸酉,詔有司:「條政令之不便民者以聞,凡被災不即請振者,罪之。」阿魯台來貢馬。甲戌,詔群臣言時政闕失。乙亥,赦兀良哈罪。始命近畿諸衛官軍更番詣京師操練。丙子,遣御史巡察邊衛。癸未,遣御史分巡天下,考察官吏。丙戌,賜戶部尚書夏原吉「繩愆糾繆」銀章。己丑,禮部奏冬至節請受賀,不許。庚寅,敕諸將嚴邊備。辛卯,禁所司擅役屯田軍士。壬辰,都督方政同榮昌伯陳智鎮交阯。是月,諭蹇義、楊士奇、夏原吉、楊榮、金幼孜曰:「前世人主,或自尊大,惡聞直言,臣下相與阿附,以至於敗。聯與卿等當用為戒。」又諭士奇曰:「頃群臣頗懷忠愛,朕有過方自悔,而進言者已至,良愜朕心。」十二月癸卯,宥建文諸臣外親全家戍邊者,留一人,餘悉放還。辛亥,揭天下三司官姓名於奉天門西序。癸丑,免被災稅糧。庚申,葬文皇帝於長陵。丙寅,鎮遠侯顧興祖破平樂、潯州蠻。

是年,丁闐、琉球、占城、哈密、古麻剌朗、滿剌加、蘇祿、瓦剌入貢。

洪熙元年春正月壬申朔,御奉天門受朝,不舉樂。乙亥,敕內外群臣修舉職業。己卯,享太廟。建弘文閣,命儒臣入直,楊溥掌閣事。癸未,以時雪不降,敕群臣修省。丙戌,大祀天地於南郊。奉太祖、太宗配。壬辰,朝臣予告歸省者賜鈔有差,著為令。己亥,布政使周乾、按察使胡概、參政葉春巡視南畿、浙江。二月辛丑,頒將軍印於諸邊將。戊申,祭社稷。命太監鄭和守備南京。丙辰,耕耤田。丙寅,太宗神主祔太廟。是月,南京地屢震。三月壬申,前光祿署丞權謹以孝行擢文華殿大學士。丁丑,求直言。戊子,隆平饑,戶部請以官麥貸之。帝曰:「即振之,何貸為。」己丑,詔曰:「刑者所以禁暴止邪,導民於善,非務誅殺也。吏或深文傅會,以致冤濫,朕深憫之。自今其悉依律擬罪。或朕過於嫉惡,法外用刑,法司執奏,。五奏不允,同三公、大臣執奏,必允乃已。諸司不得鞭囚背及加入宮刑。有自宮者以不孝論。非謀反。勿連坐親屬。古之盛世,採聽民言,用資戒儆。今奸人往往摭拾,誣為誹謗,法吏刻深,鍛練成獄。刑之不中,民則無措,其餘誹謗禁,有告者一切勿治。」庚寅,陽武侯薛祿為鎮朔大將軍,率師巡開平、大同邊。辛卯,參將安平伯李安與榮昌伯陳智同鎮交阯。戊戌,將還都南京,詔北京諸司悉稱行在,復北京行部及行後軍都督府。是月,南京地屢震。

夏四月壬寅,帝聞山東及淮、徐民乏食,有司徵夏稅方急,乃御西角門詔大學士楊士奇草詔,免今年夏稅及科糧之半。士奇言:「上恩至矣,但須戶、工二部預聞。」帝曰:「救民之窮當如救焚拯溺,不可遲疑。有司慮國用不足,必持不決之意。」趣命中官具楮筆,令士奇就門樓書詔。帝覽畢,即用璽付外行之。顧士奇曰:「今可語部臣矣。」設北京行都察院。壬子,命皇太子謁孝陵,遂居守南京。戊午,如天壽山,謁長陵。己未,還宮。是月,振河南及大名饑。南京地屢震。五月己卯,侍讀李時勉、侍講羅汝敬以言事改御史,尋下獄。庚辰,帝不豫,遣使召皇太子於南京。辛巳,大漸,遺詔傳位皇太子。是日,崩於飲安殿,年四十有八。

秋七月己巳,上尊謚,廟號仁宗,葬獻陵。

贊曰:「當靖難師起,仁宗以世子居守,全城濟師。其後成祖乘輿,歲出北征,東宮監國,朝無廢事。然中遘媒孽,瀕於危疑者屢矣,而終以誠敬獲全。善乎其告人曰「吾知盡子職而已,不知有讒人也」,是可為萬世子臣之法矣。在位一載。用人行政,善不勝書。使天假之年,涵濡休養,德化之盛,豈不與文、景比隆哉。


\end{pinyinscope}