\article{列傳第一百七十九 忠義三}

\begin{pinyinscope}
○潘宗顏竇永澄等張銓何廷魁徐國全高邦佐顧頤崔儒秀陳輔堯段展鄭國昌張鳳奇盧成功等黨還醇安上達任光裕等李獻明何天球徐澤武起潛張春閻生斗李師聖等王肇坤王一桂上官藎等孫士美白慧元李禎寧等喬若雯李崇德等張秉文宋學朱等彥胤紹趙珽等吉孔嘉王端冕等刑國璽馮守禮等張振秀劉源清等鄧籓錫王維新等張焜芳

潘宗顏,字士瓚,保安衛人。善詩賦,曉天文、兵法。舉萬歷四十一年進士,歷戶部郎中。數上書當路言遼事,當路不能用。以宗顏知兵,命督餉遼東。旋擢開原兵備僉事。四十六年,馬林將出師,宗顏上書經略楊鎬曰:「林庸懦,不堪當一面,乞易他將,以林為後繼,不然必敗。」鎬不從。宗顏監林軍,出三岔口,營稗子峪,夜聞杜松敗,林軍遂嘩。及旦,大清兵大至。林恐甚,一戰而敗,策馬先奔。守顏殿後,奮呼衝擊,膽氣彌厲。自辰至午,力不支,與遊擊竇永澄、守備江萬春、贊理通判董爾礪等皆死焉。事聞,賜祭葬,贈光祿卿,再贈大理卿,廕錦衣世百戶,謚節愍,立祠奉祀。永澄等亦賜恤如制。

張銓,字宇衡,沁水人。萬曆三十二年進士。授保定推官,擢御史,巡視陜西茶馬。以憂歸,起按江西。時遼東總兵官張承蔭敗歿,而經略楊鎬方議四道出師。銓馳奏言:「敵山川險易,我未能悉知,懸軍深入,保無抄絕?且突騎野戰,敵所長,我所短。以短擊長,以勞赴逸,以客當主,非計也。昔臚朐河之戰,五將不還,奈何輕出塞。為今計,不必徵兵四方,但當就近調募,屯集要害以固吾圉,厚撫北關以樹其敵,多行間諜以攜其黨,然後伺隙而動。若加賦選丁,騷擾天下,恐識者之憂不在遼東。」因請發帑金,補大僚,宥直言,開儲講,先為自治之本。又言:「李如柏、杜松、劉綎以宿將並起,宜責鎬約束,以一事權。唐九節度相州之潰,可為明鑒。」又言:「廷議將恤承蔭,夫承蔭不知敵誘,輕進取敗,是謂無謀。猝與敵遇,行列錯亂,是謂無法。率萬餘之眾,不能死戰,是謂無勇。臣以為不宜恤。」又論鎬非大帥才,而力薦熊廷弼。

四十八年夏復上疏言:「自軍興以來,所司創議加賦,畝增銀三釐,未幾至七釐,又未幾至九釐。辟之一身,遼東,肩背也,天下,腹心也。肩背有患,猶藉腹心之血脈滋灌。若腹心先潰,危亡可立待。竭天下以救遼,遼未必安,而天下已危。今宜聯人心以固根本,豈可朘削無已,驅之使亂。且陛下內廷積金如山,以有用之物,置無用之地,與瓦礫糞土何異。乃發帑之請,叫閽不應,加派之議,朝奏夕可。臣殊不得其解。」銓疏皆關軍國安危,而帝與當軸卒不省。綎、松敗,時謂銓有先見云。

熹宗即位,出按遼東,經略袁應泰下納降令,銓力爭,不聽,曰:「禍始此矣。」天啟元年三月,沈陽破,銓請令遼東巡撫薛國用帥河西兵駐海州,薊遼總督文球帥山海兵駐廣寧,以壯聲援。疏甫上,遼陽被圍,軍大潰。銓與應泰分城守,應泰令銓退保河西,以圖再舉,不從。守三日,城破,被執不屈,欲殺之,引頸待刃,乃送歸署。銓衣冠向闕拜,又遙拜父母,遂自經。事聞,贈大理卿,再贈兵部尚書,謚忠烈。官其子道浚錦衣指揮僉事。

銓父五典,歷官南京大理卿,時侍養家居。詔以銓所贈官加之,及卒,贈太子太保。初,五典度海內將亂,築所居竇莊為堡,堅甚。崇禎四年,流賊至,五典已歿,獨銓妻霍氏在,眾請避之。曰:「避賊而出,家不保。出而遇賊,身更不保。等死耳,盍死於家。」乃率僮僕堅守。賊環攻四晝夜,不克而去。副使王肇生名其堡曰「夫人城」。鄉人避賊者多賴以免。

道浚既官錦衣,以忠臣子見重,屢加都指揮僉事,僉書衛所。顧與閹黨楊維垣等相善,而受王永光指,攻錢龍錫、成基命等,為公論所不予。尋以納賄事敗,戍鴈門。流賊起,山西巡撫宋統殷檄道浚軍前贊畫。道浚家多壯丁,能禦賊。

崇禎五年四月,賊犯沁水,寧武守備猛忠戰死。道浚遣遊擊張瓚馳援,賊乃退。八月,紫金樑、老回回、八金剛等以三萬眾圍竇莊,謀執道浚以脅巡撫。道浚屢敗賊,賊乃欲因道浚求撫。紫金樑請見,免胄前曰:「我王自用也,誤從王佳胤至此。」又一人跽致辭曰:「我宜川廩生韓廷憲,為佳胤所獲,請誓死奉約束。」道浚勞遣之,而陰使使啗廷憲圖賊。賊至舊縣,守約不動,廷憲日惎紫金樑就款,未決。官軍襲之,賊怒,尤廷憲,遂敗約,南突濟源,陷溫陽。

九月,廷憲知紫金樑疑己,思殺之以歸,約道浚伏兵沁河以待。道浚遣所部劉偉佐之。是夕,賊攻諸生蓋汝璋樓,掘地深丈餘,樓不毀。賊怒,誓必拔。雞鳴不得間,廷憲知事且洩,偕偉倉卒奔。賊追之及河,伏起,殺追者滾山虎等六人,皆賊腹心也。賊臨沁河,索廷憲。竇莊東面河,道浚潛渡上流,繞賊後大噪,賊駭遁去。未幾,官軍扼賊陵川,師潰,道浚據九仙臺以免。十二月,廷憲知紫金樑、亂世王有隙,縱諜遣書間之。亂世王果疑,遣其弟混世王就道浚乞降。時統殷以失賊罷,許鼎臣來代,主進討。道浚權詞難之曰:「斬紫金樑以來,乃得請。」混世王怏怏去,賊眾遂分部掠諸郡縣。

明年三月,官軍躡賊,自陽城而北。道浚設伏三纏凹,擒賊渠滿天星等,巡撫鼎臣奏道浚功第一。八月,賊陷沁水。沁水當賊衝,去來無時,道浚倡鄉人築堡五十四以守,賊五犯皆卻去,至是乃陷。道浚率家眾三百人馳赴擊賊,賊退徙十五里。道浚收散亡,捕賊眾,傾家囷以餉。副使王肇生列狀上道浚功。道浚故得罪清議,冀用軍功自湔祓,而言者劾其離伍冒功。巡按御史馮明玠覆劾,謂沁城既失,不可言功,乃更戍海寧衛。

何廷魁,字汝謙,山西威遠衛人。萬曆二十九年進士。授涇縣知縣,調寧晉,遷刑部主事,歷歸德、衛輝、河南知府,西寧副使。坐考功法,復為黎平知府。會遼事棘,遷副使,分巡遼陽。袁應泰納降,廷魁爭,不聽。及沈陽破,同事者遣孥歸,廷魁曰:「吾不敢為民望。」大清兵渡濠,廷魁請乘半濟急擊之。俄薄城,圍未合,又請盡銳出禦。應泰並不從。遼陽破,廷魁懷印率其妾高氏、金氏投井死,婢僕從死者六人。都司徐國全聞之,亦自經公署。事聞,贈光祿卿,再贈大理卿,賜祭葬,謚忠愍,世廕錦衣百戶。國全贈恤如制。

高邦佐,字以道,襄陵人。萬曆二十三年進士。授壽光知縣,教民墾荒,招集流亡三千家。歷戶部主事、員外郎。遷永平知府,浚濼河,築長堤。裁抑稅使高淮,不敢大橫。遷天津兵備副使,平巨盜董時耀。轉神木參政,屢破套寇沙計。以嫡母憂歸,補薊州道,坐調兵忤主者意,被劾歸。天啟元年,遼陽破,起參政,分守廣寧。以母年八十餘,涕泣不忍去,母責以大義乃行。熊廷弼、王化貞構隙,邦佐知遼事必敗,累乞歸。方報允,而化貞棄廣寧逃。眾謂邦佐既請告,可入關。邦佐叱曰:「吾一日未去,則一日封疆臣也,將安之!」夜作書訣母,策騎趨右屯謁廷弼,言:「城中雖亂,敵尚未知。亟提兵入城,斬一二人,人心自定。公即不行,請授邦佐兵赴難。」廷弼不納,偕化貞並走。邦佐仰天長嘆,泣語從者曰:「經、撫俱逃,事去矣。松山吾守地,當死此。汝歸報太夫人。」遂西向拜闕,復拜母,解印緩自經官舍。僕高永曰:「主死,安可無從者。」亦自經於側。事聞,賜祭葬,贈光祿卿,再贈太僕卿,謚忠節,世蔭錦衣百戶。邦佐與張銓、何廷魁皆山西人,詔建祠宣武門外,顏曰三忠。

同時顧頤,以右參政分守遼海道。廣寧之變,力屈自經。贈太僕少卿,世廕本衛副千戶。

崔儒秀,字儆初,陜州人。萬歷二十六年進士。歷戶部郎中,遷開原兵備僉事。時開原已失,儒秀募壯士,攜家辭墓行。經略袁應泰以兵馬甲仗不足恃為憂,儒秀曰:「恃人有必死之心耳。」應泰深然之。遼陽被圍,分守東城,矢集如雨,不少卻。會兵潰,儒秀痛哭,戎服北向拜,自經。事聞,賜恤視何廷魁,賜祠曰愍忠,以陳輔堯、段展配祀。

輔堯,揚州人。萬曆中舉於鄉。歷永平同知,轉餉出關,與自在知州段展駐沈陽。天啟元年,日暈異常。展牒應泰言天象示警,宜豫防。踰月,沈陽破,展死之。輔堯方奉命印烙,左右以無守土責,勸之去。輔堯曰:「孰非封疆臣,何去為。」望闕拜,拔刀自剄,與展並贈按察僉事。輔堯官膠州時,有饋山繭者,受而懸之公幣中。展,涇陽舉人。

鄭國昌,邠州人。萬曆三十五年進士。歷山西參政。崇禎元年以按察使治兵永平,遷山西右布政使,上官奏留之。三年正月,大清兵自京師東行,先使人伏文廟承塵上,主者不覺也。初四日黎明登城,有守將左右之,國昌覺其異,捶之至死。須臾,北樓火發,城遂破。國昌自縊城上,中軍守備程應琦從之。應琦妻奔告國昌妻,與之偕死。

知府張鳳奇,推官盧成功,盧龍教諭趙允殖,副總兵焦延慶,東勝衛指揮張國翰及里居中書舍人廖汝欽,武舉唐之俊,諸生韓洞原、周祚新、馮維京、胡起鳴、胡光奎、田種玉等十數人皆死。國昌、鳳奇一門盡死。事聞,贈國昌太常卿,鳳奇光祿卿,並賜祭葬,廕一子。成功等贈恤有差。鳳奇,陽曲人,起家鄉舉。

黨還醇,字子貞,三原人。天啟五年進士。授休寧知縣,有善政,以父憂歸。崇禎二年服闋,起官良鄉。十二月,大清兵薄城,督吏民乘城拒守。或言縣小無兵,盍避去。還醇毅然曰:「吾守土吏也,去將安之!」救兵不至,力屈城破,與教諭安上達、訓導李廷表、典史史之棟、驛丞楊其禮並死焉。事定,父老覓還醇屍,得之草間,赤身面縛,體被數鎗,群哭而殮之。上達,貴州安順人。萬歷末年舉於鄉,謁選得教諭,至日闔門死難。事聞,贈還醇光祿丞,予祭葬,有司建祠,官其一子。之棟等亦贈恤,給驛歸其喪。已而吏科上言:「還醇城亡與亡,之死靡貳,猶曰有守土責也。上達、之棟等,微員末秩,亦能致命遂志,有死無隕。宜破格褒崇,以為世勸。朝廷必不惜今日之虛名,作將來之忠義,乃僅贈國學教職、良鄉主簿,於聖主憂恤之典謂何!」帝感其言,下部更議,乃贈上達、廷表《五經》博士,與之棟等及千戶蕭如龍、何秉忠,百戶李廕並配祀還醇祠。武舉陳蠡測、諸生梅友松等十五人,烈婦朱氏等十七人,並建坊旌表。順天府尹劉宗周以上達得死難之正,請贈翰苑宮坊,不報。

是時,列城以死事聞者,更有香河知縣任光裕、濼州知州楊燫。光裕贈恤如還醇,燫贈光祿少卿,並任一子。

李獻明,字思皇,壽光人。崇禎元年進士。授保定推官。明年十一月,大清兵臨遵化,巡撫王元雅與推官何天球、遵化知縣徐澤及先任知縣武起潛等憑城拒守。時獻明以察核官庫駐城中。或謂此邑非君所轄,去無罪。獻明正色曰:「莫非王土,安敢見危避難。」請守東門,城破死之。

元雅,太原人。為巡撫數月即遇變,自縊死。天球以永平推官理遵化軍餉。澤,字兌若,襄陽人,獻明同年進士。涖任七日,與天球、起潛並殉難。

起潛,字用潛,進賢人。天啟五年進士。初為武清知縣,有諸生為人所訐,納金酒甕以獻。起潛召學官及諸生貧者數人,置甕庭中,謂之曰:「美酒不可獨享,與諸生共之。」酒盡,金見,其人惶恐請罪,即以金分畀貧者。治縣一年,有聲,調繁遵化。坐事被劾,解官候代,遂及於難。

巡撫方大任論畿輔諸臣功罪,因言元雅有失城罪,而一死節概凜然,足以蓋愆。樞輔孫承宗請恤殉難諸臣,亦首元雅。帝贈獻明、天球光祿少卿,澤光祿丞,俱廕一子。元雅以大吏失城,贈恤不及。

張春,字泰宇,同州人。萬曆二十八年舉於鄉。歷刑部主事,勵操行,善談兵。天啟二年,遼東西盡失,廷議急邊才,擢山東僉事,永平、燕建二路兵備道。時大軍屯山海關,永平為孔道,士馬絡繹,關外難民雲集。春運籌有方,事就理而民不病。累轉副吏、參政,仍故官。七年,哈刺慎部長汪燒餅者,擁眾窺桃林口,春督守將擒三人。燒餅叩關願受罰,春等責數之,誓不敢叛。

崇禎元年改關內道。兵部尚書王在晉惑浮言,劾春嗜殺,一日梟斬十二人。春具揭辯,關內民亦為訟冤。在晉復劾其通奄剋餉,遂削籍,下法司治。督師袁崇煥言春廉惠,不聽。御史李炳言:「春疾惡過甚,為人中傷。夫殺之濫否,一勘即明,乞免提問。」不從。明年,法司言春被劾無實,乃釋之。

三年正月,永平失守,起春永平兵備參議。春言:「永平統五縣一州,今郡城及濼州、遷安並失,昌黎、樂亭、撫寧又關內道所轄。臣寄跡無所,當駐何城?臣以兵備名官,而實無一兵,操空拳入虎穴,安能濟事。乞於赴援大將中,敕一人與臣同事,臣亦招舊日義勇率之自效。臣身已許此城,不敢少規避。但必求實濟封疆,此臣區區之忠,所以報聖明而盡臣職也。」因言兵事不可預洩,乞賜陛見,面陳方略,帝許之。既入對,帝數稱善,進春參政。已而偕諸將收復永平諸城,論功加太僕少卿,仍涖兵備事,候巡撫缺推用。時乙榜起家者多授節鉞,而春獨需後命,以無援於朝也。永平當兵燹之餘,閭閻困敝,春盡心撫恤,人益懷之。

四年八月,大清兵圍大凌河新城,命春監總兵吳襄、宋偉軍馳救。九月二十四日渡小凌河。越三日次長山,距城十五里,大清兵以二萬騎來逆戰。兩軍交鋒,火器競發,聲震天地。春營被衝,諸軍遂敗,襄先敗,春復收潰眾立營。時風起,黑雲見,春命縱火,風順,火甚熾,天忽雨反風,士卒焚死甚眾。少頃雨霽,兩軍復鏖戰,偉力不支亦走。春及參將張洪謨、楊華征,遊擊薛大湖等三十三人俱被執,部卒死者無算。諸人見我太宗文皇帝皆行臣禮,春獨植立不跪。至晚,遣使賜以珍饌。春曰:「忠臣不事二君,禮也。我若貪生,亦安用我。」遂不食。越三日,復以酒饌賜之,春仍不食,守者懇勸,感太宗文皇帝恩,始一食。令薙髮,不從。居右廟,服故衣冠,迄不失臣節而死。

初,襄等敗書聞,以春守志不屈,遙遷右副都御史,恤其家。春妻翟聞之,慟哭,六日不食,自縊死。當春未死時,我大清有議和意,春為言之於朝,朝中嘩然詆春。誠意伯劉孔昭遂劾春降敵不忠,乞削其所授憲職。朝議雖不從,而有司繫其二子死於獄。

閻生斗,字文瀾,汾西人。由歲貢生,歷保安知州。大清兵入保安,生斗集吏民固守。城破,被執死之。判官李師聖、吏目王本立、訓導張文魁亦同死,時崇禎七年七月也。八月入靈丘,知縣蔣秉採募兵堅守,力屈眾潰,投繯死,合門殉之。守備於世奇,把總陳彥武、馬如豸,典史張標,教諭路登甫並鬥死。事聞,贈生斗太僕少卿,餘贈恤如制。秉采,字衷白,全州舉人。

王肇坤,字亦資,蘭谿人。崇禎四年進士。除刑部主事,改御史。初,流賊破鳳陽,疏言兵驕將悍之弊,請假督撫重權,大將犯軍令者,便宜行戳。得旨申飭而已。出巡山海、居庸二關。九年七月,大清兵入喜峰口,肇坤激眾往禦,不敵,退保昌平。被圍,與守陵太監王希忠,總兵官巢丕昌,戶部主事王一桂、趙悅,攝知州事保定通判王禹佐分門守。有降丁二千為內應,城遂破,肇坤被四矢兩刃而死。丕昌出降。一桂、悅、禹佐、希忠及判官胡惟忠、吏目郭永、學正解懷亮、訓導常時光、守備咸貞吉皆死之。禹佐子亦從父死。

一桂,黃岡舉人,督餉昌平,以南城最衝,身往扼之。俄西城失守,被執死。妻妾子女暨家眾二十七人悉赴井死。悅以公事赴昌平,遂遇難。未幾,大清兵攻順義。知縣上官藎,字忠赤,曲沃人。起家鄉舉,廉執有聲,在官三年,薦章十餘上。與遊擊治國器、都指揮蘇時雨等拒守。城破,藎自經。國器、時雨及訓導陳所蘊皆死。尋破寶坻,知縣趙國鼎、主簿樊樞、典史張六師、訓導趙士秀皆死。國鼎,山西樂平人。鄉試第一,崇禎七年進士。破定興,教諭濼州熊嘉志殉節死。破安肅,知縣臨清鄭延任與妻同殉。教諭靈壽耿三麟亦死之。事聞,贈肇坤大理卿,予祭葬,官一子。一桂、悅並贈太僕少卿,廕子祭葬,餘贈恤如制。

孫士美,青浦人。由鄉舉授舒城教諭。崇禎八年春,賊來犯,縣令以公事出,士美代守七十餘日,城以全。明年擢知深州。十一年冬,大清兵至,力守三日,城破,自剄於角樓。父訥亦自縊,一家死者十三人。贈太僕少卿,訥亦被旌。

是時,畿輔諸郡悉被兵,長吏多望風遁,失城四十有八。任丘白慧元、慶都黃承宗、靈壽馮登鰲、文安王鑰、蠡縣王采、新河崔賢、鹽山陳志、故城王九鼎,皆以殉難聞。他若青縣張文煥、興濟錢珍、慶雲陳緘,城破被殺。教官死難者則有劉廷訓、張純儒、唐一中。鄉官則喬若雯、李禎寧最著。而棄城者,吳橋知縣李綦隆等十人,皆坐死。

白慧元,青澗人。崇禎七年進士。居官善祛蠹,吏民畏之。九年以守城功,命減俸行取。會與大閹有隙,摭其罪於帝,逮治之,未行,大兵已抵城下,乃與代者李廉仲共守。無何,廉仲縋城遁,慧元躬擐甲胄,防禦甚力。及城破,一門俱死,贈僉事。

鄉官李禎寧,萬曆三十八年進士。歷山西按察使,罷歸,佐慧元拒守。城破,率家眾格鬥,身中數槊而死,一門從死者數人。承宗,未詳何許人。馮登鰲,膚施舉人,其從父大緯為蠡縣訓導,亦死。王鑰,武功舉人。王采,澤州人,進士。崔賢,弋州舉人。志、九鼎,亦未詳何許人,志自經死,九鼎戰死城上,各贈恤有差。

劉廷訓,順天通州人。歲貢生,為吳橋訓導。崇禎十一年,大清兵入畿內,知縣李綦隆欲遁,廷訓止之,與共守。外圍將合,綦隆縋城走。廷訓急趨城上,語守者曰:「守死,逃亦死,盍死於守,為忠義鬼乎!」眾泣諾,乃堅拒三晝夜。廷訓中流矢,束胸力戰,又中六矢乃死。踰月,其子啟棺更殮,面如生。

張純儒,新安人,為臨城訓導,率諸生共城守,城破死之。唐一中,全州人,為鉅鹿教諭,抗節死。

喬若雯,臨城人。萬曆四十七年進士。授中書舍人,遷禮部主事。崇禎元年春,廷臣爭擊魏忠賢黨,若雯亦兩疏劾兵部侍郎秦士文,御史張訥、智鋌,備列其傾邪狀。尋言:「故輔魏廣微,罪惡滔天,致先帝冒桓、靈之名,罪不下忠賢。其徒陳九疇、張訥、智鋌為之鷹犬,專噬善類,罪不下彪、虎。乞死者削其官階,生者投之荒裔。」帝責其詆毀先帝,而九疇等下所司行遣。若雯尋改吏部,遷員外郎。出為袞州知府,剔除積弊,豪猾斂手,以疾歸,士民遮道泣送。及城陷,若雯端坐按劍以待,遂被殺。

時鄉官李崇德、董祚、魏克家並以城亡殉難。崇德,青縣人。祚,隆平人。克家,高陽人。皆舉人。崇德歷戶部員外郎。祚未仕。克家為鄒平知縣,有善政。若雯贈太常少卿,餘贈恤有差。

張秉文,字含之,桐城人。祖淳,官參政,事具《循吏傳》。秉文舉萬曆三十八年進士,歷福建右參政,與平海寇李魁奇。崇禎中,歷廣東按察使,右布政使,調山東為左。十一年冬,大清兵自畿輔南下。本兵楊嗣昌檄山東巡撫顏繼祖移師德州,於是濟南空虛,止鄉兵五百,萊州援兵七百,勢弱不足守。巡按御史宋學朱方行部章丘,聞警馳還,與秉文及副使周之訓、翁鴻業,參議鄧謙,鹽運使唐世熊等議守城,連章告急於朝。嗣昌無以應,督師中官高起潛擁重兵臨清不救,大將祖寬、倪寵等亦觀望。大清兵徇下州縣十有六,遂臨濟南。秉文等分門死守,晝夜不解甲,援兵竟無至者。明年正月二日,城潰,秉文擐甲巷戰,已被箭,力不能支,死之。妻方、妾陳,並投大明湖死。學硃、之訓、謙、世熊及濟南知府茍好善、同知陳虞胤、通判熊烈獻、歷城知縣韓承宣皆死焉,德王由樞被執。秉文贈太常寺卿,之訓、謙光祿卿,承宣光祿少卿,皆建特祠,餘贈恤如制。學朱死,不得屍,疑未實,獨格不予,福王時,贈大理卿。鴻業及推官陸粲不知所終,贈恤亦不及。

學朱,字用晦,長洲人。崇禎四年進士。為御史,嘗抗疏劾楊嗣昌、田維嘉,時論壯之。之訓,黃岡人,進士。累官浙江按察使,坐事貶官,被薦未擢而遘難。望闕再拜,與妻劉偕死,闔門殉之。謙,孝感人,進士。戰於城上,與季父有正偕死,母莫氏匿民間不食死,族戚傔從,死者四十餘人。世熊,灌陽舉人,分守西門,被殺。好善,醴泉人,進士。虞胤,未詳。烈獻,黃陂貢生,城破,與二子俱死。承宜,大學士爌孫,進士,與妻妾同死。有劉大年者,江西廣昌人。官兵部主事,奉使南京,還朝,道歷城,城破抗節死。贈光祿少卿。

時大清兵所破州縣,守令失城者,皆論死。而臨邑宋希堯、博平張列宿、茌平黃建極、武城李承芳、丘縣高重光,皆以死節蒙贈恤。重光,字秀恒,保定人。由貢生為柏鄉訓導,率蒼頭擊盜以全城,遂擢為令。及大軍至,吏民欲負之逃,重光不可,抱印赴井死。

其縉紳殉難者,恩縣李應薦,天啟時,官御史。以附魏忠賢,麗名逆案。至是,捐貲募士,佐有司力守城,城破,身被數刃而死。歷城劉化光與子漢儀先後舉於鄉,父子俱守城力戰死,贈恤有差。

顏胤紹,字賡明,曲阜人,復聖六十五代孫也。崇禎四年進士。歷知鳳陽、江都、邯鄲,遷真定同知,守城剿寇有功。十五年擢河間知府,比歲大饑,死亡載道,寇盜充斥,拊循甚至。閏十一月,大清兵至,與參議趙廷、同知姚汝明、知縣陳三接等堅守。援兵雲集,率逗遛。胤紹知城必破,豫集一家老稚於室中,積薪繞之,而身往城上策戰守。城破,趨歸官舍,舉火焚室,衣冠北向再拜,躍入火中同死。

廷,字秉珪,慈谿人。崇禎元年進士。知南安、侯官二縣,屢遷河間兵備僉事,一門十四人悉被難。

汝明,夏縣人。天啟初,舉於鄉。性孝友。崇禎間歲大寢,傾廩振濟,立義塚,瘞暴骨。授蠡縣知縣,聞鄉邑又饑,貽書其子,令振救如初。後官河間,與妾任同死。

三接,文水人。舉崇禎六年鄉試,知河間縣。歲旱饑,人相食。三接至,雨即降。有疑獄,數年不決,至即決之。妻武氏賢,三接見封疆多故,遣之歸,答曰:「夫死忠,妻死節,分也。」三接巷戰死,武從之。

廷贈太僕卿,胤紹光祿卿,汝明、三接並僉事。

有周而淳者,掖縣人。由進士拜兵科給事中,與同官六人分督畿輔諸郡城守事。而淳甫至河間,城即被圍,遂與諸臣同死,贈太常少卿。

先是,大兵入霸州,兵備副使趙輝偕知州丁師羲、里居參政李時茪等督士民固拒。援軍不至,城遂破。輝整冠帶自盡,子琬同死。師羲、時茪皆死之。煇,字黃如,河津人,崇禎七年進士,贈光祿卿。師羲,字象先,楚雄人。選貢生,贈參議。時茪,進士,累官參政,贈太常卿。

吉孔嘉,洋縣人。幼時愬父冤於巡按御史,獲釋,以孝稱。舉崇禎三年鄉試。授寧津知縣,蠲繁苛,除寇賊,闔邑頌德。累遷順德知府。十五年冬,大清兵臨城,與鄉官知府傅梅,中書舍人孟魯缽、張鳳鳴募兵,悉力拒守,力屈城破,孔嘉與妻張、長子惠迪、次子婦王俱死。贈太僕少卿,妻子皆獲旌。梅,刑臺人。萬曆十九年舉於鄉。除知登封,有善政。遷刑部主事,治張差梃擊案,事別見。死,贈太常少卿。魯缽,工部主事。

時以守城殉難者,有王端冕,字服先,江陵舉人。知趙州,以廉惠得民。城破,被執死之。教諭陳廣心,元城人,起家乙榜。城將破,衣冠危坐,諸子環泣請避,厲聲曰:「吾平生所學何事,豈為兒女戀戀耶!」遂被殺。訓導王一統,成安人。居家多義行,死節明倫堂。唐鉉,字節玉,睢州人。崇禎七年進士。歷定州知州,死之。高維岱,昌邑人。舉於鄉,知永清縣,視事甫旬餘即遇變,一門死之。典史李時正、教諭邸養性、鄉官劉維蕙同死。清豐破,教諭曹一貞、訓導董調元皆死。鄉官吏部郎中李其紀、黃州推官侶鶴舉、富陽知縣杜斗愚亦死之。而南樂監生鄭獻書、河間襄陽知縣賈太初、永年山東副使申為憲皆抗節死。鉉贈右參議。維岱僉事,餘贈恤有差。

邢國璽,長葛人。崇禎七年進士。授濰縣知縣,改建石城,盡心民事。時帝以修城郭、練民兵、儲糗糧、備戎器四事課天下,有司率視為具文,惟國璽奉行如詔。上官交薦,遷戶部主事。運道梗於盜,有議開膠萊河者,國璽力陳其便。擢登萊兵備劍事,經度河道。十五年,畿輔戒嚴,部檄徵山東兵入衛。國璽監督至龍岡,猝遇大清兵。部卒驚懼欲沖,國璽叱止之,身先搏戰,矢刃交加,墮馬死。撫按不奏,帝降旨嚴責,乃具聞,贈恤如制。

時大兵下山東,直抵海州、贛榆、汱陽、豐、沛,列城將吏,或遁或降。其身死封疆者,有馮守禮、張百新、張予卿、朱迥添、任萬民等。

守禮,猗氏人,舉於鄉。縣令有疑獄,語訴者得馮孝廉一剌,獄即解。其人懷金以告,拒不聽。選平定州學正,諸生兄弟爭彥相訐,餽以金,守禮嚴卻之,勸以友悌,感悟去。歷遷知萊蕪縣。城破,與二子攄奇、拱奇並自殺。

日新,浙江建德人。由歲貢為訓導,造齊東教諭。見海內寇起,與諸生講藝習射,招土寇安守夏降之。及齊東被圍,與守夏登陴守,力屈及子光裔死之,妻方氏自刎,守夏亦從死。予卿知陽信,城陷殉難。迥添者,沈陽宗室也,居潞安。由宗學貢生為鄒平知縣,城失,全節以死。萬民,陽曲諸生。見鄉郡被寇,草救時八議、守城十二策,獻之當事,果得其用。以保舉授武城知縣,在職三年,有能聲,竟殉城死。

又文昌時,全州舉人。知臨淄縣,以廉慎得民。及大清兵東下,城受圍,與訓導申周輔共守。城破,舉家自焚,周輔亦殉難。同時,壽光知縣李耿,大興人。崇禎中進士,自縊城上。吳良能,遼東蓋州人。舉於鄉,知滕縣,城將破,盡殺家屬,拜母出,力戰死。吳汝宗,寧洋人。知東阿,城失守,死之。周啟元,黃岡舉人,知高苑縣,城破,朱衣坐堂上,死之。

劉光先,未詳里居,知豐縣。大兵二千騎營西城外,不攻。夜一人自營逸出,語城上人曰:「得梯即攻。」不信。又有逸者曰:「梯成,立攻矣。」婦人亦自營出曰:「盡甲矣。」昧爽突攻西南陬,方力禦,已登西北陬,光先殉焉。劉士璟,亦不知何許人,知沐陽,有強幹聲。竭力捍城,城破死之。贈山東僉事。

張振秀,臨清人。萬曆三十八年進士。知肥鄉、永平,遷兵部主事。泰昌元年改吏部,更歷四司,至文選員外郎,乞假歸。崇禎改元,起驗封郎中,歷考功、文選,擢太常少卿,坐事落職歸。崇禎十五年,大清兵圍河間,遠近震恐。臨清總兵官劉源清偕榷關主事陳興言、同知路如瀛、判官徐應芳、吏目陳翔龍、在籍兵部侍郎張宗衡員外郎刑泰吉、臨汾如縣尹任及振秀等合力備禦。未幾,城被圍,力拒數日,援不至,城破,並死之。興言,南靖人。如瀛,陵川人。應芳,臨川人。翔龍,蕭山人。泰吉、任皆進士。宗衡自有傳。源清,澤清弟,贈太子少保。

其時,城破殉難者,壽張王大年、曹州楚煙、滕縣劉弘緒數人。大年舉進士,歷御史,加太僕少卿,以附魏忠賢名持逆案,至是盡節死。煙舉進士,歷戶部主事,解職歸。及城失守,力抗,子鳳苞以身翼之,皆被殺。妻趙觸柱死。弘緒歷車駕郎中,遇變死。

鄧籓錫,字晉伯,金壇人。崇禎七年進士。歷南京兵部主事。十五年遷袞州知府,甫抵任,已聞大清兵入塞,亟繕守具。未幾,四萬騎薄城下,籓錫走告魯王曰:「郡有吏,國有王,猶同舟也。列城失守,皆由貴家惜金錢,而令窶人、餓夫列陴捍禦。夫城郭者,我之命也。財賄者,人之命也。我不能畀彼以命,而望彼畀我以命乎?王誠散積儲以鼓士氣,城猶可存。不然,大事一去,悔無及矣。」王不能從。籓錫與監軍參議王維新,同知譚絲、曾文蔚,通判閻鼎,推官李昌期,滋陽知縣郝芳聲,副將丁文明,長史俞起蛟及里居給事中范淑泰等分門死守。至十二月八日,力不支,城破,維新猶力戰,被二十一創乃死。籓錫受縛不降,被殺,其妾攜稚子投井死。魯王以派亦被殺。

昌期,永年人。芳聲,忻州人。並起家進士。昌期嘗監軍破土寇萬,眾推其才。芳聲治縣有聲。至是皆死。

起蛟,錢塘人。由貢生歷官魯府左長史,相憲王。及惠王立,欲易世子,起蛟力諫乃已。世子嗣位,值歲凶,勸王振貸,自出粟二千石佐之。大盜李青山率眾來犯,偕淑泰出擊,大破其眾。及王被難,起蛟率親屬二十三人殉之。文明亦戰死。

事聞,贈維新光祿卿,籓錫太僕少卿,昌期僉事,餘贈恤有差。

有樊吉人者,元城人。由進士知滋陽,累擢山東兵備僉事。未行遇變,自刎死。淑泰自有傳。

張焜芳,會稽人。崇禎元年進士。歷南京戶科給事中。十一年春,疏薦黃道周、惠世揚、陳子壯、金光辰,而為舊撫文震孟請恤。帝以沽名市恩,切責之。又糾太僕少卿史涘,為涘所訐,遂罷職,事具《薛國觀傳》。十六年正月,焜芳北上,抵臨清,遇大清兵,與諸生馬之騆,之駉俱被執死之。其妻妾聞之,赴井死。

時又有天津參將賀秉鉞者,泰寧左衛人。崇禎四年第武科一甲第三,亦以扶父柩至臨清,巷戰終日,矢盡,被執死。


\end{pinyinscope}