\article{列傳第六十六 項忠韓雍餘子俊阮勤硃英秦紘}

\begin{pinyinscope}
○項忠韓雍餘子俊阮勤朱英秦紘

項忠,字藎臣,嘉興人。正統七年進士。授刑部主事,進員外郎。從英宗陷於瓦剌,令飼馬,乘間挾二馬南奔。馬疲,棄之,徒跣行七晝夜,始達宣府。

景泰中,由郎中遷廣東副使。按行高州,諜報賊攜男女數百剽村落。忠曰:「賊無攜家理,必被掠良民也。」戒諸將毋妄殺。已,訊所俘獲,果然,盡釋之。從征瀧水瑤有功,增俸一秩。

天順初,歷陜西按察使。母憂歸,部民詣闕乞留,詔起復。時陜西連歲災傷,忠發廩振,且請輕罪納米,民賴以濟。

七年以大理卿召,民乞留如前,遂改右副都御史,巡撫其地。洮、岷羌叛,忠疏言:「羌志在劫掠,盡誅則傷仁,遽撫則不威,請聽臣便宜從事。」報可。乃發兵據險,揚聲進討,眾盡降。西安水泉鹵不可飲,為開龍首渠及皁河,引水入城。又疏鄭、白二渠,溉涇陽、三原、醴泉、高陵、臨潼五縣田七萬餘頃,民祠祀之。

陜西數苦兵。成化元年上言:「三邊大將遇敵逗留,雖云才怯,亦由權輕。士卒畏敵不畏將,是以戰無成功。宜許以軍法從事。廟堂舉將才,踰年不聞有一人應詔。陜西風土強勁,古多名將,豈無其人?但格於不能答策耳。今天下學校生徒善答策者百不一二,奈何責之武人。」帝善其言,而所司守故事不能用。

毛里孩寇延綏,詔忠偕彰武伯楊信禦之,無功。明年,信議大舉搜河套,敕忠提督軍務。忠方赴延綏,而寇復陷開城,深入靜寧、隆德六州縣,大掠而去。兵部劾忠,帝特宥之,搜套師亦不出。又明年,召理院事。

四年,滿俊反。滿俊者,亦名滿四。其祖巴丹,自明初率所部歸附,世以千戶畜牧為雄長。仍故俗,無科徭。其地在開城縣之固原里,接邊境。俊獷悍,素藏匿姦盜,出邊抄掠。會有獄連俊,有司跡逋至其家,多要求。俊怒,遂激眾為亂。守臣遣俊姪指揮璹往捕。俊殺其從者,劫璹叛,入據石城。石城,即唐吐番石堡。城稱險固,非數萬人不能克者也。山上有城寨,四面峭壁,中鑿五石井以貯水,惟一徑可緣而上。俊自稱招賢王,有眾四千。都指揮邢端等禦之,敗績。不再月,眾至二萬,關中震動。乃命忠總督軍務,與監督軍務太監劉祥、總兵官都督劉玉帥京營及陜西四鎮兵討之。師未行,而巡撫陳價等先以兵三萬進討,復大敗。賊因官軍器甲,勢益張。朝議欲益兵。忠慮京軍脆弱不足恃,且更遣大將撓事權,因上言:「臣等調兵三萬三千餘人,足以滅賊。今秋深草寒,若更調他軍,恐往復需時,賊得遠遁。且邊兵不能久留,益兵非便。」大學士彭時、商輅主其議,京軍得毋遣。

忠遂與巡撫都御史馬文升分軍七道,抵石城下,與戰,斬獲多。伏羌伯毛忠乘勝奪其西北山,幾破,忽中流矢死。玉亦被圍。諸軍欲退,忠斬一千戶以徇。眾力戰,玉得出,乃列圍困之。適有星孛於台斗,中朝多言「占在秦分,師不利」。忠曰:「李晟討朱泚,熒惑守歲,此何害。」日遣兵薄城下,焚芻草,絕汲道。賊窘欲降,邀忠與文升相見。忠偕劉玉單騎赴之,文升亦從數十騎至,呼俊、璹諭以速降。賊遙望羅拜,忠直前挾璹以歸。俊氣沮,猶豫不出。忠命縛木為橋,人負土囊填濠塹,擊以銅炮,死者益眾。賊倚愛將楊虎貍為謀主,夜出汲被擒。忠貰其死,諭以購賊賞格。示之金,且賜金帶鉤。縱歸,使誘俊出戰,伏兵擒焉。急擊下石城,盡獲餘寇。毀其城,鑿石紀功。增一衛於固原西北西安廢城,留兵戍之而還。

初,石城未下,天甚寒,士卒頗困。忠慮賊奔突,乘凍渡河與套寇合,日夜治攻具。身當矢石不少避,大小三百餘戰。彭時、商輅知忠能辦賊,不從中制,卒用殄賊。論功,進右都御史,與林聰協掌院事。

白圭既平劉通,荊、襄間流民屯結如故。通黨李胡子者名原,偽稱平王,與小王洪、王彪等掠南漳、房、內鄉、渭南諸縣。流民附賊者至百萬。六年冬,詔忠總督軍務,與湖廣總兵官李震討之。忠乃奏調永順、保靖土兵。而先分軍列要害,多設旗幟鉦鼓,遣人入山招諭。流民歸者四十餘萬,彪亦就擒。時白圭為兵部,遣錦衣百戶吳綬贊參將王信軍。綬欲攘功,不利賊瓦解。縱流言,圭信之,止土兵毋調。忠疏爭,且劾綬罪,帝為召綬還,而聽調土兵如故。合二十五萬,分八道逼之,流民歸者又數萬。賊潛伏山寨,伺間出劫。忠命副使餘洵、都指揮李振擊之,遇於竹山。乘溪漲半渡截擊,擒李原、小王洪等,賊多溺死。忠移軍竹山,捕餘孽。復招流民五十萬,斬首六百四十,俘八百有奇,家口三萬餘人。戶選一丁,戍湖廣邊衛,餘令歸籍給田。疏陳善後十事,悉允行。

忠之下令逐流民也,有司一切驅逼。不前,即殺之。民有自洪武中占籍者,亦在遺中。戍者舟行多疫死。給事中梁璟因星變求言,劾忠妄殺。白圭亦言流民既成業者,宜隨所在著籍,又駁忠所上功次互異。帝皆不聽。進忠左都御史。廕子綬錦衣千戶,諸將錄功有差。

忠上疏言:「臣先後招撫流民復業者九十三萬餘人,賊黨遁入深山,又招諭解散自歸者五十萬人。俘獲百人,皆首惡耳。今言皆良家子,則前此屢奏猖獗難禦者,伊誰也?賊黨罪固當死,正因不忍濫誅,故令丁壯謫發遣戍。其久附籍者,或乃占山四十餘里,招聚無賴千人,爭鬥劫殺。若此者,可以久居故不遣乎?臣揭榜曉賊,謂已殺數千,蓋張虛勢怵之,非實事也。且圭固嘗身任其事,今日之事又圭所遺。先時,中外議者謂荊、襄之患何日得寧。今幸平靖,而流言沸騰,以臣為口實。昔馬援薏苡蒙謗,鄧艾檻車被徵。功不見錄,身更不保。臣幸際聖明,願賜骸骨,勿使臣為馬、鄧之續。」帝溫詔答之。

八年召還,與李賓協掌院事。後二年拜刑部尚書,尋代圭為兵部。

汪直開西廠,恣橫,忠屢遭侮不能堪。會大學士商輅等劾直,忠亦倡九卿劾之。奏留中,而西廠遂罷,直深恨之。未幾,西廠復設,直以吳綬為腹心,綬挾前憾,伺忠益急。忠不自安,乞歸治病。未行,而綬嗾偵事者誣忠罪。給事中郭鏜、御史馮貫等復交章劾忠,事連其子經、太監黃賜、興寧伯李震、彰武伯楊信等。詔法司會錦衣衛廷鞫,忠抗辯不少屈。然眾知出直意,無敢為之白者,竟斥為民,賜與震等亦得罪。直敗,復官,致仕。家居二十六年,至弘治十五年乃卒,年八十二。贈太子太保,謚襄毅。

忠倜儻多大略,練戎務,彊直不阿,敏於政事,故所在著稱。

子經,經子錫,錫子治元,皆舉進士。經,江西參政。錫,南京光祿寺卿。治元,員外郎。

韓雍,字永熙,長洲人。正統七年進士。授御史。負氣果敢,以才略稱。錄囚南畿。碭山教諭某笞膳夫,膳夫逃匿,父訴教諭殺其子,取他尸支解以證。既誣服,雍蹤跡得之,白其冤。出巡河道。已,巡按江西,黜貪墨吏五十七人。廬陵、太和盜起,捕誅之。

十三年冬,處州賊葉宗留自福建轉犯江西。官軍不利,都督僉事陳榮、指揮劉真遇伏死。詔雍及鎮守侍郎楊寧督軍民協守。會福建巡按御史汪澄牒鄰境會討賊鄧茂七,俄以賊議降,止兵。雍曰:「賊果降,退未晚也。」趨進,賊已叛,澄坐得罪死。人以是服雍識。

景泰二年擢廣東副使。大學士陳循薦為右僉都御史,代楊寧巡撫江西。歲饑,奏免秋糧。劾奏寧王不法事,王府官皆得罪。時雍年甫三十,赫然有才望,所規畫措置,咸可為後法。

天順初,罷天下巡撫官,改山西副使。寧王以前憾劾其擅乘肩輿諸事,下獄,奪官。起大理少卿。尋復為右僉都御史,佐寇深理院事。石亨既誅,錦衣指揮劉敬坐飯亨直房,用朋黨律論死。雍言:「律重朋黨,謂阿比亂朝政也。以一飯當之,豈律意?且亨盛時大臣朝夕趨門,不坐,獨坐敬何也?」深歎服,出之。母憂,起復。四年,巡撫宣府、大同。七年議事入覲,帝壯其貌,留為兵部右侍郎。

憲宗立,坐學士錢溥累,貶浙江左參政。廣西瑤、僮流剽廣東,殘破郡邑殆遍。成化元年正月大發兵,拜都督趙輔為總兵官,以太監盧永、陳瑄監其軍。兵部尚書王竑曰:「韓雍才氣無雙,平賊非雍莫可。」乃改雍左僉都御史,贊理軍務。

雍馳至南京,集諸將議方略。先是,編修邱濬上書大學士李賢,言賊在廣東者宜驅,在廣西者宜困。欲宿兵大藤峽,扼其出入,蹂其禾稼,期一二年盡賊。賢善之,獻於朝,詔錄示諸將。諸將主其說,請令遊擊將軍和勇率番騎趨廣東,而大軍直趨廣西,分兵撲滅。雍曰:「賊已蔓延數千里,而所至與戰,是自敝也。當全師直搗大藤峽。南可援高、肇、雷、廉;東可應南、韶;西可取柳、慶;北可斷陽峒諸路。首尾相應,攻其腹心。巢穴既傾,餘迎刃解耳。舍此不圖,而分兵四出,賊益奔突,郡邑益殘,所謂救火而噓之也。」眾曰「善。」輔亦知雍才足辦賊,軍謀一聽雍。

雍等遂倍道趨全州。陽峒苗掠興安,擊破之。至桂林,斬失機指揮李英等四人以徇。按地圖與諸將議曰:「賊以修仁、荔浦為羽翼,當先收二縣以孤賊勢。」乃督兵十六萬人,分五道,先破修仁賊,窮追至力山。擒千二百餘人,斬首七千三百級。荔浦亦定。

十月至潯州,延問父老,皆曰:「峽,天險,不可攻,宜以計困。」雍曰:「峽延廣六百餘里,安能使困?兵分則力弱,師老則財匱,賊何時得平?吾計決矣。」遂長驅至峽口。儒生、里老數十人伏道左,願為嚮導。雍見即罵曰:「賊敢紿我!」叱左右縛斬之,左右皆愕,既縛,而袂中利刃出。推問,果賊也。悉支解刳腸胃,分挂林箐中,纍纍相屬。賊大驚曰:「韓公天神也!」雍令總兵官歐信等為五哨,自象州、武宣攻其北;身與輔督都指揮白全等為八哨,自桂平、平南攻其南;參將孫震等為二哨,從水路入;而別分兵守諸隘口。賊魁侯大狗等大懼,先移其累重於桂州橫石塘,而立柵南山,多置滾木、礧石、鏢鎗、藥弩拒官軍。

十二月朔,雍等督諸軍水陸並進,擁團牌登山,殊死戰。連破石門、林峒、沙田、古營諸巢,焚其室廬積聚,賊皆奔潰。伐木開道,直抵橫石塘及九層樓諸山。賊復立柵數重,憑高以拒。官軍誘賊發矢石,度且盡,雍躬督諸軍緣木攀藤上。別遣壯士從間道先登,據山頂舉炮。賊不能支,遂大敗。先後破賊三百二十四寨,生擒大狗及其黨七百八十人,斬首三千二百有奇,墜溺死者不可勝計。峽有大藤如虹,橫亙兩厓間。雍斧斷之,改名斷藤峽,勒石紀功而還。分兵擊餘黨,鬱林、陽江、洛容、博白次第皆定。

帝大喜,賜敕嘉勞,召輔等還,遷雍左副都御史,提督兩廣軍務。雍乃散遣諸軍,以省饋餉。而遺孽侯鄭昂等遂乘虛陷潯州及洛容、北流二縣。雍被劾引罪,帝宥之。雍益發兵撲討。時諸賊所在蜂起,思恩、潯、賓、柳城悉被擾掠。流劫至廣東,欽、化二州皆應時破殄。

四年春,雍以兩廣地大事殷,請東西各設巡撫,帝可之。命陳濂撫廣東,張鵬撫廣西,而雍專理軍事。尋以憂歸。明年,兩廣盜復起,僉事陶魯言:「兩廣地勢錯互,當如臂指相使,不可離析。近賊犯廣西,臣與廣東三司議調兵,匝月未決,盜賊無所憚。乞仍命大臣總督便。」會僉事林錦、巡按御史龔晟亦以為請。乃罷兩巡撫,而起復雍右都御史,總督如故。又明年正月,雍疏辭新命,乞終制,不許。雍抵任,遣參將張壽、遊擊馮昇等分道討賊,忻州八寨蠻及諸山瑤、僮掠州縣者,皆摧破之。蠻民素懾雍威,寇盜浸息。

九年,柳、潯諸蠻復叛,參將楊廣等俘斬九百人。方更進,而賊破懷集縣。兵部劾雍奏報不實。廣西鎮守中官黃沁素憾雍抑己,因訐雍,且言其貪欲縱酒,濫賞妄費。帝遣給事中張謙等往勘。而廣西布政使何宜、副使張斅銜雍素輕己,共醞釀其罪。謙還奏,事虛實交半,竟命致仕去。

雍洞達闓爽,重信義。撫江西時,請追謚文天祥、謝枋得。詔謚天祥忠烈、枋得文節。有雄略,善斷,動中事機。臨戰,率躬親矢石,不目瞬。自奉尊嚴,三司皆長跪白事。軍門設銅鼓數十,儀節詳密。裨將以下,繩柙無所假。兩地鎮守宦官素驕恣,亦惕息無敢肆。疾惡嚴,坦中不為崖岸,揮斥財帛不少惜。故雖令行禁止,民得安堵,而謗議亦易起。為中官所齮齕,公論皆不平。兩廣人念雍功,尤惜其去,為立祠祀焉。家居五年卒,年五十七。正德間,謚襄毅。

初以軍功予一子錦衣百戶,雍以授其弟睦。至是,錄一子國子生。

餘子俊,字士英,青神人。父祥,戶部郎中。子俊舉景泰二年進士,授戶部主事,進員外郎。在部十年,以廉乾稱。出為西安知府。歲饑,發廩十萬石振貸。區畫以償,官不損而民濟。

成化初,所司上治行當旌者,知府十人,而子俊為首。以林聰薦,為陜西右參政,歲餘擢右布政使。六年轉左,調浙江。甫半載,拜右副都御史,巡撫延綏。

先是,巡撫王銳請沿邊築牆建堡,為久遠計,工未興而罷。子俊上疏言:「三邊惟延慶地平易,利馳突。寇屢入犯,獲邊人為導,徑入河套屯牧。自是寇顧居內,我反屯外,急宜於沿邊築牆置堡。況今舊界石所在,多高山陡厓。依山形,隨地勢,或鏟削,或壘築,或挑塹,綿引相接,以成邊牆,於計為便。」尚書白圭以陜民方困,奏緩役。既而寇入孤山堡,復犯榆林,子俊先後與硃永、許寧擊敗之。

是時,寇據河套,歲發大軍征討,卒無功。八年秋,子俊復言:「今征套士馬屯延綏者八萬,芻茭煩內地。若今冬寇不北去,又須備來年軍資。姑以今年之數約之,米豆需銀九十四萬,草六十萬。每人運米豆六斗、草四束,應用四百七萬人,約費行資八百二十五萬。公私煩擾至此,安得不變計。臣前請築牆建堡,詔事寧舉行。請於明年春夏寇馬疲乏時,役陜西運糧民五萬,給食興工,期兩月畢事。」圭猶持前議阻之。帝是子俊言,命速舉。

子俊先用軍功進左副都御史。明年,又用紅鹽池搗巢功,進右都御史。寇以搗巢故遠徙,不敢復居套。內地患稍息,子俊得一意興役。東起清水營,西抵花馬池,延袤千七百七十里,鑿崖築牆,掘塹其下,連比不絕。每二三里置敵臺崖寨備巡警。又於崖寨空處築短牆,橫一斜二如箕狀,以尞敵避射。凡築城堡十一,邊墩十五,小墩七十八,崖寨八百十九,役軍四萬人,不三月而成。牆內之地悉分屯墾,歲得糧六萬石有奇。十年閏六月,子俊具上其事,因以母老乞歸,慰留不許。

初,延綏鎮治綏德州,屬縣米脂、吳堡悉在其外。寇以輕騎入掠,鎮兵覺而追之,輒不及,往往得利去。自子俊徙鎮榆林,增衛益兵,拓城置戍,攻守器畢具,遂為重鎮,寇抄漸稀,軍民得安耕牧焉。十二年十二月移撫陜西。子俊知西安時,以居民患水泉鹹苦,鑿渠引城西潏河入灌,民利之。久而水溢無所洩。至是,乃於城西北開渠洩水,使經漢故城達渭。公私益便,號「余公渠」。又於涇陽鑿山引水,溉田千餘頃。通南山道,直抵漢中,以便行旅。學校、公署圮者悉新之。奏免岷、河、洮三衛之戍南方者萬有奇。易置南北之更戍者六千有奇,就戍本土。岷州慄林羌為寇,子俊潛師設伏擊走之。

十三年召為兵部尚書。奏申明條例十事,又列上軍功賞格,由是中外有所遵守。緬甸酋卜剌浪欲奪思洪發貢章地,設詞請於朝。子俊言不宜許,乃諭止之。貴州巡撫陳儼等以播州苗竊發,請調湖廣、廣西、四川兵五萬,合貴州兵會剿。子俊言賊在四川,而貴州請討,是邀功也,奏寢其事。初,子俊論陳鉞掩殺貢夷罪,帝以汪直故宥之。鉞多方構子俊於直,會母憂歸,得免。

子俊之築邊牆也,或疑沙土易傾,寇至未可恃。至十八年,寇入犯,許寧等逐之。寇扼於牆塹,散漫不得出,遂大衄,邊人益思子俊功。

服闋,拜戶部尚書,尋加太子太保。二十年命兼左副都御史,總督大同、宣府軍務。其冬還朝。明年正月,星變,陳時弊八事,帝多采納。未幾,復出行邊。

初,子俊巡歷宣、大,請以延綏邊牆法行之兩鎮,因歲歉而止。比復出,銳欲行之。言東起四海冶,西抵黃河,延袤千三百餘里,舊有墩百七十,應增築四百四十,墩高廣皆三丈,計役夫八萬六千,數月可成。詔明年四月即工。然是時,歲比不登,公私耗敝,驟興大役,上下難之。子俊又欲責成於邊臣,而己不親其事。謗議由是起。至冬,疏請還京。帝入蜚語,命改左都御史,巡撫大同。中官韋敬讒子俊假修邊多侵耗,又劾子俊私恩怨,易將帥。兵部侍郎阮勤等為白。帝怒,讓勤等。而給事、御史復交章劾,中朝多欲傾子俊。工部侍郎杜謙等往勘,平情按之。還奏易置將帥如勤等言,所費無私。然為銀百五十萬,米菽二百三十萬,耗財煩民,不得無罪。遂落太子太保,致仕去,時二十二年二月也。

明年正月,兵部缺尚書。帝悟子俊無罪,復召任之,仍加太子太保。孝宗嗣位,以先朝老臣,待之彌厚。弘治元年疏陳十事,已,又上邊防七事,帝多允行。明年,疾亟,猶手削奏稿,陳救荒弭盜之策,甫得請而卒,年六十一。贈太保,謚肅敏。

子俊沉毅寡言,有偉略。凡奏疏公移,必自屬草,每夜分方寢。嘗曰:「大臣謀國,當身任利害,豈得遠怨市恩為自全計。」故榆林始事,怨讟叢起,子俊持之益堅,竟以成功,為數世利。性孝友,居母憂時,令子寘毋會試,曰:「雖無律令,吾心不忍也。」嘗廕子,移以蔭弟。

子寰,舉進士,終戶部員外郎。寘,就武廕為錦衣千戶,終指揮同知。曾孫承勛、承業,皆進士。承勛,翰林修撰。承業,雲南僉事。

阮勤,本交阯人,其父內徙,占籍長子。勤舉景泰五年進士。歷台州知府。清慎有惠政,賜誥旌異。以右副都御史巡撫陜西。築墩臺十四所,治垣塹三十餘里。歲饑,奏免七府租四十餘萬石。入為侍郎,調南京刑部。蠻邦人著聲中國者,勤為最。

朱英,字時傑,桂陽人。五歲而孤。力學,舉正統十年進士,授御史。浙、閩盜起,簡御史十三人與中官分守諸府,英守處州。而葉宗留黨四出剽掠,處州道梗。英間道馳至,撫降甚眾,戮賊首周明松等,賊散去乃還。

景泰初,御史王豪嘗以勘陳循爭地事,忤循,為所訐。至是,循草詔,言風憲官被訐者,雖經赦宥,悉與外除。於是豪當改知縣,英言:「若如詔書,則凡遭御史抨擊之人,皆將挾仇誣訐,而御史愈緘默不言矣。」章下法司,請如英言,乃復豪職。未幾,出為廣東右參議。過家省母,橐中惟賜金十兩。抵任,撫凋瘵流亡。立均徭法,十歲一更,民稱便。

天順初,兩廣賊愈熾,諸將多濫殺冒功。巡撫葉盛屬英督察。參將范信誣宋泰、永平二鄉民為賊,屠戮殆盡,又欲屠進城鄉。英馳訊,悉縱去。信忿,留師不還。英密請於盛,檄信班師,一方始靖。潮州賊羅劉寧等流劫遠近,屢挫官兵。英會師破滅之。還所掠人口數千,別置一營以處婦女,人莫敢犯。

官參議十年,進右參政。遭母憂。成化初服闋,補陜西。大軍討滿四,英主饋餉有功。歷福建、陜西左、右布政使,皆推行均徭法。十年以右副都御史巡撫甘肅,先後陳安邊二十八事。其請徙居戎、安流離、簡貢使,於時務尤切。明年冬,兩廣總督吳琛卒,廷議以英前在廣東有威信,遂以代琛。

自韓雍大征以來,將帥喜邀功,利俘掠,名為「雕剿」。英至,鎮以寧靜,約飭將士。毋得張賊聲勢,妄請用師。招撫瑤、僮效順者,定為編戶,給復三年。於是馬平、陽朔、蒼梧諸縣蠻悉望風附。而荔波賊李公主有眾數萬,久負固,亦遣子納款。為置永安州處之,俾其子孫世吏目。自是歸附日眾,凡為戶四萬三千有奇,口十五萬有奇。帝甚嘉之。

鎮守中官與督撫、總兵官坐次,中官居中,總督居總兵官左。時總兵官陳政以伯爵欲抑英居右,英不可,奏乞裁定。命解英總督,止為巡撫,居政下。尚書餘子俊言英招徠功多,當增秩褒賞,乃反削其事權,恐無以鎮諸蠻。乃擢英右都御史仍總督,位次如故。

田州酋黃明烝其知府岑溥祖母,欲殺溥。溥出走思恩,明因肆屠戮。英將進討,檄溥族人恩城知州岑欽殺明雪恥。欽遂誅明并其族屬,傳首軍門。

英淳厚,然持法無所假借。與市舶中官韋眷忤,眷摭奏英專權玩賊。潯州知府史芳以事見責,亦訐英奸貪欺罔。按皆無驗,乃鐫芳二官,諭眷協和共事。

十六年,交阯攻老撾,議者恐其內寇,詔問英處置之宜。英對言:「彼不過爭甌脫耳,諭之當自悔懼。」帝從其言,果上表謝。潯、梧、高、廉賊起,偕政等分道擊之。再戰,俘斬甚眾。十九年,桂林平樂蠻攻城殺將,英、政復分兵十二道擊破之。

明年入掌都察院事,尋加太子少保。又明年正月,星變,疏陳八事:請禁邊將節旦獻馬;鎮守中官、武將不得私立莊田,侵奪官地;燒丹符咒左道之人,當置重典;四方分守監槍內官勿進貢品物;罷撤倉場、馬房、上林苑增設內侍;召還建言得罪諸臣;清內府收白糧積弊;治奸民投獻莊田及貴戚受獻者罪。權倖皆不便,執政多持之不行。英造內閣力爭,竟不能盡從也。時流民集京師者多,英請人給米月三斗,幼者半之,報許。其年秋卒。贈太子太保。

英為總督承韓雍、吳琛後。雍雖有大功,恢廓自奉,贈遺過侈,有司困供億,公私耗竭;而琛務謹廉;至英益持清節,僅攜一蒼頭之官。先後屢賜璽書、金幣,英藏璽書,貯金幣於庫。其威望不及雍,而惠澤過之。在甘肅積軍儲三十萬兩,廣四十餘萬,皆不以聞。或問之,答曰:「此邊臣常分,何足言。」人服其知大體。正德中,追謚恭簡。

子守孚,進士,刑部郎中。

秦紘,字世纓,單人。景泰二年進士。授南京御史。劾治內官傅鎖兒罪,諫止江南採翠毛、魚鷿等使。權貴忌之,蜚語聞。會考察,坐謫湖廣驛丞。

天順初,以御史練綱薦,遷雄縣知縣。奉御杜堅捕天鵝暴橫,紘執杖其從者,坐下詔獄。民五千詣闕訟,乃調知府谷。憲宗即位,遷葭州知州,調秦州。母喪去官,州人乞借紘,服闋還故任。尋擢鞏昌知府,改西安,遷陜西右參政。岷州番亂,提兵三千破之,進俸一級。

成化十三年擢右僉都御史,巡撫山西,奏鎮國將軍奇澗等罪。奇澗父慶成王鐘鎰為奏辯,且誣紘。帝重違王意,逮紘下法司治。事皆無驗,而內官尚亨籍紘家,以所得敝衣數事奏。帝歎曰:「紘貧一至此耶?」賜鈔萬貫旌之。於是奪奇澗等三人爵,王亦削祿三之一,而改紘撫河南。尋復調宣府。

小王子數萬騎寇大同,長驅入順聖川,掠宣府境。紘與總兵官周玉等邀擊,遁去。尋入掠興寧口,連戰卻之,追還所掠,璽書勞焉。進左僉都御史,巡撫如故。未幾,召還理院事,遷戶部右侍郎。萬安逐尹旻,誣紘旻黨,降廣西右參政。進福建左布政使。

弘治元年以王恕薦,擢左副都御史,督漕運。明年三月進右都御史,總督兩廣軍務。奏言:「中官、武將總鎮兩廣者,率縱私人擾商賈,高居私家。擅理公事,賊殺不辜,交通土官為奸利。而天下鎮守官皆得擅執軍職,受民訟,非制,請嚴禁絕。總鎮府故有賞功所,歲儲金錢數萬,費出無經,宜從都御史勾稽。廣、潮、南、韶多盜,當設社學,編保甲,以絕盜源。」帝悉從其請。恩城知州岑欽攻逐田州知府岑溥,與泗城知州岑應分據其地。紘入田州逐走欽,還溥於府,留官軍戍之,亂遂定。復遣將討平黎賊陵水,瑤賊德慶。

紘之初蒞鎮也,劾總兵官安遠侯柳景貪暴,逮下獄。景亦訐紘,勘無左證,法司當景死。景連姻周太后家,有奧援,訐紘不已。詔并逮紘,廷鞫卒無罪。詔宥景死,奪爵閒住,而紘亦罷歸。大臣王恕等請留紘,不納。廷臣復連章言紘可大用。居數月,起南京戶部尚書。十一年引疾去。

十四年秋,寇大入花馬池,敗官軍孔壩溝,直抵平涼。言者謂紘有威名,雖老可用。詔起戶部尚書兼右副都御史,總制三邊軍務。紘馳至固原,按行敗所。躬祭陣亡將士,掩其骼。奏錄死事指揮朱鼎等五人,恤軍士戰歿者家。劾治敗將楊琳等四人罪,更易守將。練壯士,興屯田,申明號令,軍聲大振。

初,寇未入河套,平涼、固原皆內地,無患。自孛來往牧後,固原當兵衝,為平、慶、臨、鞏門戶。而城隘民貧,兵力單弱,商販不至。紘乃拓治城郭,招徠商賈,建改為州,而身留節制之。奏言:「固原主、客兵止萬八千人,散守城堡二十四。勢分力弱,宜益兵。舊臨、鞏、秦州諸軍歲赴甘、涼備禦。及他方有警,又調兵甘、涼,或發京軍征討。夫京師天下本,邊將手握重兵,而一遇有事輒請京軍,非強幹弱枝之道。請自今京兵毋輕發,臨、鞏、甘、涼諸軍亦宜各還本鎮。但選知兵宿將一二人各守其地,人以戍為家,軍以將為命,自樂趨役,而有戰心,計之得者也。」紘見固原迤北延袤千里,閑田數十萬頃,曠野近邊,無城堡可依。議於花馬池迤西至小鹽池二百里,每二十里築一堡,堡周四十八丈,役軍五百人。固原迤北諸處亦各築屯堡,募人屯種,每頃歲賦米五石,可得五十萬石。規畫已定,而寧夏巡撫劉憲為梗。紘乃奏曰:「竊見三邊情形,延綏、甘、涼地雖廣,而士馬精強。寧夏怯弱矣,然河山險阻。惟花馬池至固原,軍既怯弱,又墩臺疏遠,敵騎得長驅深入,故當增築墩堡。韋州、豫望城諸處亦然。今固原迤南修築將畢,惟花馬池迤北二百里當築十堡。而憲危言阻眾,且廢垂成之功。乞令憲制三邊,而改臣撫寧夏,俾得終邊防,於事為便。」帝下詔責憲,憲引罪,卒行紘策。修築諸邊城堡一萬四千餘所,垣塹六千四百餘里,固原屹為重鎮。紘又以意作戰車,名「全勝車」,詔頒其式於諸邊。在事三年,四鎮晏然,前後經略西陲者莫及。

十七年加太子少保,召還視部事。以年老連章力辭,乞致仕。詔賜敕乘傳歸,月廩歲隸如制。明年九月卒,年八十。贈少保,謚襄毅。

紘廉介絕俗,妻孥菜羹麥飯常不飽。性剛果,勇於除害,不自顧慮,士大夫識與不識稱為偉人。在兩廣被逮時,方議討後山賊。治軍事畢,從容就道,儀衛騶從不貶損。既踰嶺,始囚服就繫。謂官校曰:「兩廣蠻夷雜處,總制體尊,遽就拘執,損國威。今既踰嶺,真囚矣。」其嚴重得體如此。正德五年,劉瑾亂政。紘家奴憾紘婦弟楊瑾,以紘所遺火炮投緝事校尉,誣瑾畜違禁軍器。劉瑾怒,歸罪於紘。籍其家,無所得。言官張九敘、塗敬等復希瑾意劾紘,士類嗤之。

贊曰:項忠、韓雍皆以文學通籍,而親提桴鼓,樹勛戎馬之場。其應機決勝,成畫遠謀,雖宿將殆無以過,豈不壯哉!賞不酬勞,謠諑繼起,文法吏從而繩其後,功名之士所為發憤而太息也。餘子俊盡心邊計,數世賴之。朱英廉威名粵嶠,秦紘經略著西陲,文武兼資,偉哉一代之能臣矣!


\end{pinyinscope}