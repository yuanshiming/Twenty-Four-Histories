\article{列傳第一百六十四}

\begin{pinyinscope}
朱大典王道焜等張國維張肯堂李向中吳鐘巒硃永佑等曾櫻朱繼祚湯芬等餘煌陳函輝王瑞栴路振飛何楷林蘭友熊汝霖錢肅樂劉中藻鄭遵謙沈宸荃邑子履祥

朱大典,字延之,金華人。家世貧賤。大典始讀書,為人豪邁。登萬歷四十四年進士,除章丘知縣。天啟二年擢兵科給事中。中官王體乾、魏忠賢等十二人及乳嫗客氏,假保護功,廕錦衣世襲,大典抗疏力諫。五年出為福建副使,進右參政,以憂歸。

崇禎三年,起故官,蒞山東,尋調天津。五年四月,李九成、孔有德圍萊州。山東巡撫徐從治中炮死,擢大典右僉都御史代之,詔駐青州,調度兵食。七月,登萊巡撫謝璉復陷於賊,總督劉宇烈被逮。乃罷總督及登萊巡撫不設,專任大典,督主、客兵數萬及關外勁旅四千八百餘人合剿之。以總兵金國奇將,率副將靳國臣、劉邦域,參將祖大弼、祖寬、張韜,遊擊柏永福及故總兵吳襄、襄子三桂等,以中官高起潛監護軍餉,抵德州。賊復犯平度,副將牟文綬、何維忠等救之,殺賊魁陳有時,維忠亦被殺。八月,巡按監軍御史謝三賓至昌邑,請斬王洪、劉國柱,詔逮治之。兵部尚書熊明遇亦坐主撫誤國,罷去。三賓復抗疏請絕口勿言撫事。

國奇等至昌邑,分三路。國奇等關外兵為前鋒,鄧步兵繼之,從中路灰埠進。昌平總兵陳洪範,副將劉澤清、方登化,從南路平度進。參將王之富、王文緯等從北路海廟進。檄遊擊徐元亨等率萊陽師來會,以牟文綬守新河。諸軍皆攜三日糧,盡抵新河東岸,亂流以濟。祖寬至沙河,有德迎戰。寬先進,國臣繼之,賊大敗,諸軍乘勝追至城下。賊夜半東遁,圍始解。守者疑賊誘,炮拒之。起潛遣中使入諭,闔城相慶。明日,南路兵始至。國奇等遂擊賊黃縣,斬首萬三千,俘八百,逃散及墜海死者數萬。

賊竄歸登州,國臣等築長圍守之。城三面距山,一面距海,牆三十里而遙,東西俱抵海。分番戍,賊不能出,發大炮,官軍多死傷。李九成出戰相當。十一月,九成搏戰,降者洩其謀。官軍合擊之,馘於陣,賊乃曉夜哭。賊渠魁五,九成、有德、有時、耿仲明、毛承祿也,及是殺其二。帝嘉解圍功,進大典右副都御史,將吏升賞有差。是月,國奇卒,以襄代。攻圍既久,賊糧絕,恃水城可走,不降。及王之富、祖寬奪其水門外護牆,賊大懼。

六年二月中旬,有德先遁,載子女財帛出海。仲明以水城委副將王秉忠,已亦以單舸遁,官軍遂入大城。攻水城,未下。遊擊劉良佐獻轟城策,匿人永福寺中,穴城置火藥,發之,城崩,官軍入。賊退保蓬萊閣,大典招降,始釋甲,俘千餘人,獲秉忠及偽將七十五人,自縊及投海死者不可勝計,賊盡平。有德等走旅順,島帥黃龍邀擊,生擒其黨毛承祿、陳光福、蘇有功,斬李應元。惟有德、仲明逸去。乃獻承祿等於朝。磔之先一日,有功脫械走。帝震怒,斬監守官,刑部郎多獲罪。未幾被執,伏誅。敘功,進大典兵部右侍郎,世蔭錦衣百戶,巡撫如故。

八年二月,流賊陷鳳陽,毀皇陵,總督楊一鵬被逮。詔大典總督漕運兼巡撫廬、鳳、淮、揚四郡,移鎮鳳陽。時江北州縣多陷。明年正月,賊圍滁州,連營百餘里,總兵祖寬大破之。大典會總理盧象昇追襲,復破之。急還兵遏賊眾於鳳陽,賊始退。十一年,賊復入江北,謀竄茶山。大典與安慶巡撫史可法提兵遏之,賊乃西遁。大典先坐失州縣,貶秩視事。是年四月以平賊踰期,再貶三秩。尋敘援剿及轉漕功,盡復其秩。

十三年,河南賊大入湖廣。大典遣將救援,屢有功,進左侍郎。明年六月命大典總督江北及河南、湖廣軍務,仍鎮鳳陽,專辦流賊,而以可法代督漕運。賊帥袁時中眾數萬,橫潁、亳間。大典率總兵劉良佐等擊破之,敘賚有差。大典有保障功,然不能持廉,屢為給事中方士亮、御史鄭崑貞等所劾,詔削籍侯勘。事未竟,而東陽許都事發。

許都者,諸生,負氣,憤縣令苛斂,作亂,圍金華。大典子萬化募健兒禦之,賊平而所募者不散。大典聞,急馳歸。知縣徐調元閱都兵籍有萬化名,遂言大典縱子交賊。巡按御史左光先聞於朝,得旨逮治,籍其家充餉,且令督賦給事中韓如愈趣之。

已而京師陷,福王立。有白其誣者,而大典亦自結於馬士英、阮大鋮,乃召為兵部左侍郎。踰月,進尚書,總督上江軍務。左良玉興兵,命監黃得功軍禦之。福王奔太平,大典與大鋮入見舟中,誓力戰。得功死,王被擒,兩人遂走杭州。會潞王亦降,大典乃還鄉郡,據城固守。唐王聞,就加東閣大學士,督師浙東。踰年,城破,闔門死之。

其時浙東西郡縣前後失守死事者,抗州則有同知王道焜、錢塘知縣顧咸建、臨安知縣唐自彩,紹興則有兵部主事高岱、葉汝厓,衢州則有巡按王景亮、知府伍經正、推官鄧巖忠、江山知縣方召。若夫諸生及布衣殉義者,會稽潘集、周卜年,山陰朱瑋,諸暨傅日炯,鄞縣趙景麟,浦江張君正,瑞安鄒欽堯,永嘉鄒之琦,其尤著云。

王道焜,字昭平,錢塘人。以天啟元年舉於鄉。崇禎時,為南平知縣,遷南雄同知。會光澤寇發,其父老言非道焜不能平。撫按為請,詔改邵武同知,知光澤縣事。撫剿兼施,境內底定。莊烈帝破格求賢,盡徵天下賢能吏,撫按以道焜名聞。方待命而都城陷,微服南還。及杭州失守,遂投繯死。

顧咸建,字漢石,崑山人,大學士鼎臣曾孫也。崇禎十六年進士。授錢塘知縣。甫之官,聞京師陷,人情恟恟。咸建戢奸宄,嚴警備。巡按御史彭遇颽以貪殘激變,賴咸建調護,事寧而民免株連。及南都失守,鎮江守將鄭彩等率眾還閩,緣道劫掠。咸建出私財迎犒,乃斂威去。亡何,馬士英擁兵至。頃之,大將方國安兵亦至。咸建謀於上官,先期遣使行賂,兵乃不入城。四鄉多被淫掠,城中得無擾。時監司及郡縣長吏悉逋竄,咸建散遣妻子,獨守官不去。潞王既降,咸建不至。尋被執,死之。

唐自彩,達州人。為臨安知縣。杭州失守,自彩與從子階豫逃山中。有言其受魯王敕,陰部署為變,遂被捕獲。自彩麾階豫走,不從,竟同死。

高岱,字魯瞻,會稽人。崇禎中,以武學生舉順天鄉試,魯王授為職方主事。及紹興失守,即絕粒祈死。子朗知父意不可回,先躍入海中死。岱聞之曰:「兒果能先我乎!」自是不復言,數日亦卒。

葉汝厓,字衡生,岱同邑人,由舉人為兵部主事。聞變,與妻王氏出居桐塢墓所,並赴水死。

王景亮,字武侯,吳江人。崇禎末登進士。仕福王為中書舍人。唐王立,擢御史,巡撫金、衢二府,兼視學政。伍經正,安福人。由貢生為西安知縣,唐王超擢知府事。鄧巖忠,江陵人。由鄉舉為推官。衢州破,經正赴井死,景亮、巖忠皆自縊死。魯王所遣鎮將張鵬翼亦死之。

方召,宣城人。署江山縣事。金華被屠,集父老告之曰:「兵且至,吾義不當去。然不可以一人故,致闔城被殃。」遂封其印,冠帶向北拜,赴井死。士民為收葬,立祠祀焉。

張國維,安玉笥,東陽人。天啟二年進士。授番禺知縣。崇禎元年,擢刑科給事中,劾罷副都御史楊所修、御史田景新,皆魏忠賢黨也。已,陳時政五事,言:「陛下求治太銳,綜核太嚴。拙者跼艴以避咎,巧者委蛇以取容,誰能展布四體,為國家營職業者。故治象精明,而腹心手足之誼實薄,此英察宜斂也。祖宗朝,閣臣有封還詔旨者,有疏揭屢上而爭一事者。今一奉詰責,則俯首不遑;一承改擬,則順旨恐後。倘處置失宜,亦必不敢執奏,此將順宜戒也。召對本以通下情,未有因而獲罪者。今則惟傳天語,莫睹拜揚。臣同官熊奮渭還朝十日,旁措一詞,遂蒙譴謫。不可稍加薄罰,示優容之度乎?此上下宜洽也。」其二條,請平刑罰,溥膏澤。帝不能盡用。進禮科都給事中。京師地震,規弊政甚切,遷太常少卿。

七年,擢右僉都御史,巡撫應天、安慶等十府。其冬,流賊犯桐城,官軍覆沒。國維方壯年,一夕鬚髮頓白。明年正月率副將許自強赴援,遊擊潘可大、知縣陳爾銘等守桐不下。賊乃攻潛山,知縣趙士彥重傷卒。攻太湖、知縣金應元、訓導扈永寧被殺。國維至,解桐圍,遣守備朱士胤趨潛山,把總張其威趨太湖。士胤戰死,自強遇賊宿松,殺傷相當。安慶山民桀石以投賊,賊多死,乃越英山、霍山而遁。九月,賊復由宿松入潛山、太湖,他賊掃地王亦陷宿松等三縣。國維乃募土著二千人戍之,而以兵事屬監軍史可法。明年正月,賊圍江浦,遣守備蔣若來、陳于王戰卻之。十二月,賊分兵犯懷寧,可法及左良玉、馬爌遏之。復犯江浦,副將程龍及若來、於王等拒守。諸城並全。又圍望江,遣兵援之,亦解去。

十年三月,國維率龍等赴安慶,禦賊酆家店,龍軍數千悉沒。賊東陷和州、含山、定遠,攻陷六合,知縣鄭同元潰走,賊遂攻天長。國維見賊勢日熾,請於朝,割安慶、池州、太平,別設巡撫,以可法任之。安慶不隸江南巡撫,自此始也。議者欲并割江浦、六合,俾國維專護江南,不許。

國維為人寬厚,得士大夫心。屬郡災傷,輒為請命。築太湖、繁昌二城,建蘇州九里石塘及平望內外塘、長洲至和等塘,修松江捍海堤,濬鎮江及江陰漕渠,並有成績。遷工部右侍郎兼右僉都御史,總理河道。歲大旱,漕流涸,國維浚諸水以通漕。山東饑,振活窮民無算。

十四年夏,山東盜起,改兵部右侍郎兼督淮、徐、臨、通四鎮兵,護漕運。大盜李青山眾數萬,據梁山濼,遣其黨分據韓莊等八閘,運道為梗。周延儒赴召北上,青山謁之,言率眾護漕,非亂也。延儒許言於朝,授以職。而青山竟截漕舟,大焚掠,迫臨清。國維合所部兵擊降之,獻俘於朝,磔諸市。兵部尚書陳新甲下獄,帝召國維代之。乃定戰守賞罰格,列上嚴世職、酌推升、慎咨題等七事,帝皆報可。會開封陷,河北震動,條防河數策,帝亦納之。

十六年四月,我大清兵入畿輔,國維檄趙光抃拒螺山,八總兵之師皆潰。言者詆國維,乃解職,尋下獄。帝念其治河功,得釋。召對中左門,復故官,兼右僉都御史,馳赴江南、浙江督練兵輸餉諸務。出都十日而都城陷。

福王召令協理戎政。尋敘山東討賊功,加太子太保,廕錦衣僉事。吏部尚書徐石麒去位,眾議歸國維。馬士英不用,用張捷。國維乃乞省親歸。

南都覆,踰月,潞王監國於杭州,不數日出降。閏六月,國維朝魯王於台州,請王監國。即日移駐紹興,進國維少傅兼太子太傅、兵部尚書、武英殿大學士,督師江上。總兵官方國安亦自金華至。馬士英素善國安,匿其軍中,請入朝。國維劾其十大罪,乃不敢入。連復富陽、於潛,樹木城緣江要害,聯合國安及王之仁、鄭遵謙、熊汝霖、孫嘉績、錢肅樂諸營,為持久計。順治三年五月,國安等諸軍乏餉潰,王走台州航海,國維亦還守東陽。六月知勢不可支,作絕命詞三章,赴水死,年五十有二。

張肯堂,字載寧,松江華亭人。天啟五年進士。授浚縣知縣。崇禎七年,擢御史。明年春,賊陷鳳陽,條上滅賊五事。俄以皇陵震驚,疏責輔臣不宜作秦、越之視,帝不問。出按福建,數以平寇功受賚。還朝,言:「監司營競紛紜,意所欲就,則保留久任;意所欲避,則易地借才。今歲燕、秦,明歲閩、粵,道路往返,動經數千,程限稽遲,多踰數月。加一番更移,輒加一番擾害。」帝是其言。十二年十月,楊嗣昌出督師。肯堂奏言:「從古戡亂之法,初起則解散,勢成則剪除,未有專任撫者。今輔臣膺新命而出,賊必仍用故技,佯搖尾乞憐。而失事諸臣,冀掩從前敗局,必多方熒惑,仍進撫議。請特申一令,專務剿除。有進招撫說者,立置重典。」帝以偏執臆見責之。

十四年四月言:「流寇隳城破邑,往來縱橫,如入無人之境,此督師嗣昌受事前所未有。目前大計,在先釋嗣昌之權。」疏入而嗣昌已死。十二月復言:「今討賊不可謂無人,巡撫之外更有撫治,總督之上又有督師。位號雖殊,事權無別。今楚自報捷,豫自報敗,甚至南陽失守,禍中親籓,督師職掌安在。試問今為督師者,將居中而運,以發蹤指示為功乎,抑分賊而辦,以焦頭爛額為事乎?今為秦、保二督者,將兼顧提封,相為掎角之勢乎,抑遇賊追剿,專提出境之師乎?今為撫者,將一稟督師之令,進退惟其指揮乎,抑兼視賊勢之急,戰守可以擇利乎?凡此肯綮,一切置不問,中樞冥冥而決,諸臣瞆瞆而任。至失地喪師,中樞糾督撫以自解,督撫又互相委以謝愆,而疆事不可問矣。」帝納其言,下所司詳議。十五年請召還建言譴謫諸臣,乃復給事中陰潤、李清、劉昌,御史周一敬官。肯堂遷大理丞,旋擢右僉都御史,巡撫福建。

總兵鄭鴻逵擁唐王聿鍵入閩,與其兄南安伯芝龍及肯堂勸進,遂加太子少保、吏部尚書。曾櫻至,言官請令櫻掌吏部,乃令肯堂掌都察院。肯堂請出募舟師,由海道抵江南,倡義旅,而王由仙霞趨浙東,與相聲援。乃加少保,給敕印,便宜從事。芝龍懷異心,陰沮之,不成行。

順治三年,王敗死,肯堂飄泊海外。六年至舟山,魯王用為東閣大學士。八年,大清兵乘天霧集螺頭門。定西侯張名振奉王航海去,屬肯堂城守。城中兵六千,居民萬餘,堅守十餘日。城破,肯堂衣蟒玉南向坐,令四妾、一子婦、一女孫先死,乃從容賦詩自經。

時同死者,兵部尚書李向中、禮部尚書吳鍾巒、吏部侍郎朱永佑、安洋將軍劉世勛、左都督張名揚。又有通政使會稽鄭遵儉,兵科給事中鄞縣董志寧,兵部郎中江陰朱養時,戶部主事福建林瑛、蘇州江用楫,禮部主事會稽董元,兵部主事福建朱萬年、長洲顧珍、臨山衛李開國,工部主事長洲顧中堯,中書舍人蘇州蘇兆人,工部所正鄞縣戴仲明,定西侯參謀順天顧明楫,諸生福建林世英,錦衣指揮王朝相,內官監太監劉朝。凡二十一人。

李向中,鍾祥人。崇禎十三年進士。授長興知縣,調秀水。福王時,歷車駕郎中,蘇松兵備副使。唐王以為尚寶卿。閩事敗,避海濱。魯王監國,召為右僉都御史,從航海,進兵部尚書,從至舟山。及是破,大帥召向中,不赴。發兵捕之,以衰糸至見。大帥呵之曰:「聘汝不至,捕即至,何也。?」向中從容曰:「前則辭官,今就戮耳。」

吳鐘巒,字巒稚,武進人。崇禎七年進士。授長興知縣。以旱潦,徵練餉不中額,謫紹興照磨。踰年,移桂林推官。聞京師變,流涕曰:「馬君常必能死節。」已而世奇果死。福王立,遷禮部主事。抵南雄,聞南都失,轉赴福建,痛陳國計。魯王起兵,以鐘巒為禮部尚書,往來普陀山中。大清兵至寧波,鐘巒慷慨謂人曰:「昔仲達死璫禍,吾以諸生不得死。君常死賊難,吾以遠臣不得從死。今其時矣!」乃急渡海,入昌國衛之孔廟,積薪左廡下,抱孔子木主自焚死。仲達者,江陰李應昇,鍾巒弟子,忤魏忠賢死黨禍者也。

朱永佑,字爰啟。崇禎七年進士。授刑部主事,改吏部,罷歸。事唐王,後至舟山。城破被執,願為僧,不許,乃就戮。

名揚,名振弟。城破,母范以下自焚者數十人。

朝相聞城失守,護王妃陳氏、貴嬪張氏、義陽王妃杜氏入井,用巨石覆之,自刎其旁。開國母,瑛、明楫妻皆自盡。

曾櫻,字仲含,峽江人。萬曆四十四年進士。授工部主事,歷郎中。天啟二年,稍遷常州知府。諸御史巡鹽、侖、江、漕及提學、屯田者,皆操舉劾權,文牒日至。櫻牒南京都察院曰:「他方守令,奔命一巡按,獨南畿奔命數巡按。請一切戒飭,罷鉤訪取贖諸陋習。」都御史熊明遇為申約束焉。

櫻持身廉,為政愷悌公平,不畏強禦。屯田御史索屬吏應劾者姓名,櫻不應。御史危言恐之,答曰:「僚屬已盡,無可糾,止知府無狀。」因自署下考,杜門待罪。撫按亟慰留,乃起視事。織造中官李實迫知府行屬禮,櫻不從。實移檄以「爾」「汝」侮之,櫻亦報以「爾」「汝」,卒不屈。無錫高攀龍,江陰繆昌期、李應昇被逮,櫻助昌期、應昇貲,而經紀攀龍死後事,為文祭之,出其子及僮僕於獄。宜興毛士龍坐忤魏忠賢遣戍,櫻諷士龍逃去。上官捕其家人,賴櫻以免。武進孫慎行忤忠賢,當戍,櫻緩其行。忠賢敗,事遂解。

崇禎元年以右參政分守漳南。九蓮山賊犯上杭,櫻募壯士擊退之,夜搗其巢,殲馘殆盡。士民為櫻建祠。母憂歸。服闋,起故官,分守興、泉二郡。進按察使,分巡福寧。先是,紅夷寇興、泉,櫻請巡撫鄒維璉用副總兵鄭芝龍為軍鋒,果奏捷。及劉香寇廣東,總督熊文燦欲得芝龍為援,維璉等以香與芝龍有舊,疑不遣。櫻以百口保芝龍,遂討滅香,芝龍感櫻甚。

十年冬,帝信東廠言,以櫻行賄謀擢官,命械赴京。御史葉初春嘗為櫻屬吏,知其廉,於他疏微白之。有詔詰問,因具言櫻賢,然不知賄所從至。詔至閩,巡撫沈猶龍、巡按張肯堂閱廠檄有奸人黃四臣名。芝龍前白曰:「四臣,我所遣。我感櫻恩,恐遷去,令從都下訊之。四臣乃妄言,致有此事。」猶龍、肯堂以入告,力白櫻冤,芝龍亦具疏請罪。士民以櫻貧,為醵金辦裝,耆老數千人隨至闕下,擊登聞鼓訟冤。帝命毋入獄,俟命京邸。削芝龍都督銜,而令櫻以故官巡視海道。

尋以衡、永多寇,改櫻湖廣按察使,分守湖南,給以敕。故事,守道無敕,帝特賜之。時賊已殘十餘州縣,而永州知府推官咸不任職。櫻薦蘇州同知晏日曙、歸德推官萬元吉才。兩人方坐事罷官,以櫻言並起用。櫻乃調芝龍剿賊,賊多降,一方遂安。遷山東右布政使,分守登、萊。

十四年春,擢右副都御史,代徐人龍巡撫其地。明年遷南京工部右侍郎,乞假歸。山東初被兵,巡撫王永吉所部濟、兗、東三府州縣盡失,匿不以聞。兵退,以恢復報。而櫻所部青、登、萊三府失州縣無幾,盡以實奏。及論罪,永吉反擢總督,而櫻奪官,逮下刑部獄。不十日而京師陷,賊釋諸囚,櫻乃遁還。

其後唐王稱號於福州。芝龍薦櫻起工部尚書兼東閣大學士。無何,令掌使部,尋進太子太保、吏部尚書、文淵閣。王駐延平,令櫻留守福州。大清兵破福州,櫻挈家避海外中左衛。越五年,其地被兵,遂自縊死。

朱繼祚,莆田人。萬曆四十七年進士。改庶吉士,授編修。天啟中,與修《三朝要典》,尋罷去。崇禎初,復官。累遷禮部右侍郎,充實錄總裁。給事中葛樞言繼祚嘗纂修《要典》,得罪清議,不可總裁國史,不聽。繼祚旋謝病去。起南京禮部尚書,又以人言罷去。福王時起故官,未赴。南都失,唐王召為東閣大學士,從至汀州。王被擒,繼祚奔還其鄉。魯王監國,繼祚舉兵應王,攻取興化城。既而大清兵至,城復破。繼祚及參政湯芬、給事中林嵋、知縣都廷諫並死之。

芬,字方侯,嘉善人。崇禎十六年進士。福王時,為史可法監紀推官。唐王以為御史。尋以監司分守興泉道。城破,緋衣坐堂上,被殺。嵋,字小眉,繼祚同邑人。由進士為吳江知縣。蘇州失,歸仕唐王。至是自縊死。廷諫,杭州人,莆田知縣。

王自監國二年正月至長垣,迨次年正月,連克建寧、邵武、興化三府,福寧一州,漳浦、海澄、連江、長樂等二十七縣,軍聲頗振。及是得者復失。海澄失,知縣洪有文死之。永福失,邑人給事中鄢正畿、御史林逢經俱投水死。長樂失,邑人御史王恩及服毒死,妻李氏同死。建寧失,守將王祈巷戰不勝,自焚死。

餘煌,字武貞,會稽人。天啟五年進士第一。授翰林修撰,與修《三朝要典》。崇禎時,以內艱歸。服闋,起左中允,歷左諭德、右庶子,充經筵講官。給事中韓源劾禮部侍郎吳士元、御史華琪芳及煌皆與修《要典》,宜斥,帝置不問。煌疏辯,帝復溫旨慰諭之。戶部崇尚書程國祥請借京城房租,煌爭,乞假歸。遂丁外艱。服除,久不起。魯王監國紹興,起禮部右侍郎,再起戶部尚書,皆不就。明年以武將橫甚,拜煌兵部尚書,始受命。時諸臣競營高爵,請乞無厭。煌上言:「今國勢愈危,朝政愈紛,尺土未復,戰守無資。諸臣請祭,則當思先帝烝嘗未備;請葬,則當思先帝山陵未營;請封,則當思先帝宗廟未享;請廕,則當思先帝子孫未保;請謚,則當思先帝光烈未昭。」時以為名言。大清兵過江,王航海遁。六月二日,煌赴水,舟人拯起之。居二日,復投深處,乃死。

陳函輝,字木叔,臨海人。崇禎七年進士。授靖江知縣,為御史左光先劾罷。北都陷,誓眾倡義。會福王立,不許草澤勤王,乃已。尋起職方主事,監軍江北。事敗歸,魯王擢為禮部右侍郎。從王航海,已而相失,哭入雲峰山,作絕命詞十章,投水死。

王瑞栴,字聖木,永嘉人。天啟五年進士。授蘇州推官,兼理兌運。軍民交兌,桓相軋啟釁。瑞梅調劑得宜,歲省浮費三萬金,上官為勒石著令。貴人弟奸法,執問如律。其人中之當道,將議調,遂歸。崇禎七年,起河間推官,遷工部主事,調兵部,轉職方員外郎,擢湖廣兵備僉事,駐襄陽。十一年春,張獻忠據穀城乞撫,總理熊文燦許之。瑞栴以為非計,謀於巡按林銘球、總兵官左良玉,將俟其至,執之。文燦固執以為不可。瑞栴言:「賊以計愚我,我不可為所愚。今良玉及諸將賈一選、周仕鳳之兵俱在近境,誠合而擊之,何患不捷。」文燦怒,責以撓撫局。瑞栴曰:「賊未創而遽撫,彼將無所懼。惟示以必剿之勢,乃心折不敢貳。非相撓,實相成也。」文燦不從。瑞栴乃列上從征、歸農、解散三策,文燦亦不用。瑞栴自為檄諭獻忠,獻忠恃文燦庇己,不聽。明年,獻忠叛,瑞栴先己丁憂歸。獻忠留書於壁,言己之叛,總理使然。具列上官姓名及取賄月日,而題其末曰:「不納我金者,王兵備一人耳。」由是瑞栴名大著。服闋,未及用而都城陷。福王時,乃為太僕少卿,極陳有司虐民之狀,旋告歸。唐王召赴福建,仍故官,未幾復歸。及閩地盡失,溫州亦不守,避之山中。有欲薦令出者,乃拜辭家廟,從容入室自經死。

路振飛,字見白,曲周人。天啟五年進士。除涇陽知縣。大吏陷魏忠賢,將建祠涇陽,振飛執不從。邑人張問達忤奄,坐追贓十萬。振飛故遷延,奄敗事解。流賊入境,擊卻之。崇禎四年,徵授御史。疏劾周延儒卑污奸險,黨邪醜正,祈立斥以清揆路,被旨切責。未幾,陳時事十大弊,曰務苛細而忘政體,喪廉恥而壞官方,民愈窮而賦愈亟,有事急而無事緩,知顯患而忘隱憂,求治事而鮮治人,責外重而責內輕,嚴於小而寬於大,臣日偷而主日疑,有詔旨而無奉行。疏入,詔付所司。山東兵叛,劾巡撫余大成、孫元化,且論延儒曲庇罪,帝不問。已,劾吏部尚書閔洪學結權勢,樹私人,秉銓以來,吏治日壞,洪學自引去。廷推南京吏部尚書謝升為左都御史,振飛歷詆其醜狀,升遂不果用。六年,巡按福建。海賊劉香數勾紅夷入犯,振飛懸千金勵將士,遣游擊鄭芝龍等大破之,詔賜銀幣。俸滿,以京卿錄用。初,振飛論海賊情形,謂巡撫鄒維璉不能辦,語侵之。維璉罷去,命甫下,數奏捷,振飛乃力暴其功,維璉復召用。

八年夏,帝將簡輔臣。振飛言:「枚卜盛典,使夤緣者竊附則不光。如向者周延儒、溫體仁等公論俱棄,宅揆以後,民窮盜興,辱己者必不能正天下。」時延儒已斥,而體仁方居首揆,銜之。已而振飛按蘇、松,請除輸布、收銀、白糧、收兌之四大患,民困以蘇。會常熟錢謙益、瞿式耜為奸民張漢儒所訐,體仁坐振飛失糾,擬旨令陳狀。振飛白謙益無罪,語刺體仁。體仁恚,激帝怒,謫河南按察司檢校。入為上林丞,屢遷光祿少卿。

十六年秋,擢右僉都御史、總督漕運,巡撫淮、揚。明年正月,流賊陷山西。振飛遣將金聲桓等十七人分道防河,由徐、泗、宿遷至安東、沭陽。且團練鄉兵,犒以牛酒,得兩淮間勁卒數萬。福、周、潞、崇四王避賊,同日抵淮。大將劉澤清、高傑等亦棄汛地南下。振飛悉延接之。四月初,聞北都陷,福王立於南京。河南副使呂弼周為賊節度使來代振飛,進士武愫為賊防禦使招撫徐、沛,而賊將董學禮據宿遷。振飛擊擒弼周、愫,走學禮。竿弼周法場,命軍士人射三矢,乃解磔之。縛愫徇諸市,鞭八十,檻車獻諸朝,伏誅。五月,馬士英欲用所親田仰,乃罷振飛。振飛亦遭母喪,家無可歸,流寓蘇州。尋錄功,即家加右副都御史。

振飛初督漕,謁鳳陽皇陵。望氣者言高牆有天子氣。唐王聿鍵方以罪錮守陵,中官虐之。振飛上疏乞概寬罪宗,竟得請。順治二年,大兵破南京,聿鍵自立於福州,拜為左都御史。募能致振飛者官五宮,賜二千金。振飛乃赴召,道拜太子太保、吏部尚書兼文淵閣大學士。至則大喜,與宴,抵夜分,撤燭送歸,解玉帶賜之,官一子職方員外郎。又錄守淮功,蔭錦衣世千戶。王每責廷臣怠玩,振飛因進曰:「上謂臣僚不改因循,必致敗亡。臣謂上不改操切,亦未必能中興也。上有愛民之心,而未見愛民之政;有聽言之明,而未收聽言之效。喜怒輕發,號令屢更。見群臣庸下而過於督責,因博鑒書史而務求明備,凡上所長,皆臣所甚憂也。」其言曲中王短云。三年,大清兵進仙霞關,聿鍵走汀州,振飛追赴不能及。汀州破,走居海島,明年赴永明王召,卒於途。

何楷,字元子,漳州鎮海衛人。天啟五年進士。值魏忠賢亂政,不謁選而歸。崇禎時,授戶部主事,進員外郎,改刑科給事中。流賊陷鳳陽,毀皇陵。楷劾巡撫楊一鵬、巡按吳振纓罪,而刺輔臣溫體仁、王應熊,言:「振纓,體仁私人;一鵬,應熊座主也。逆賊犯皇陵,神人共憤。陛下輟講避殿,感動臣民。二輔臣獨漫視之,欲令一鵬、振纓戴罪自贖。情面重,祖宗陵寢為輕;朋比深,天下譏刺不恤。」忤旨,鐫一秩視事。又言:「應熊、體仁奏辯,明自引門生姻婭。刑官瞻徇,實由於此。乞宣諭輔臣,毋分別恩仇,以國事為戲。」應熊復奏辯。楷言:「臣疏未奉旨,應熊先一日摭引臣疏詞,必有漏禁中語者。」帝意動,令應熊自陳,應熊竟由是去。吏部尚書謝CD言登、萊要地,巡撫陳應元引疾,宜允其去。及推勞永嘉代應元,則言登萊巡撫本贅員。楷亦疏駁之。楷又請給贈都御史高攀龍官,誥賜左光斗諸臣謚,召還惠世揚。疏多見聽。屢遷工科都給事中。

十一年五月,帝以火星逆行,減膳修省。兵部尚書楊嗣昌方主款議,歷引前史以進。楷與南京御史林蘭友先後言其非。楷言:「嗣昌引建武款塞事,欲借以申市賞之說,引元和田興事,欲借以申招撫之說,引太平興國連年兵敗事,欲借以申不可用兵之說,徒巧附會耳。至永平二年馬皇后事,更不知指斥安在。」帝方護嗣昌,不聽。踰月,嗣昌奪情入閣,楷又劾之,忤旨,貶二秩為南京國子監丞。母憂歸。服闋,廷臣交薦,召入京,都城已陷。

福王擢楷戶部右侍郎,督理錢法,命兼工部右侍郎。連疏請告,不許。順治二年,南都破,楷走杭州。從唐王入閩,擢戶部尚書。鄭芝龍、鴻逵兄弟橫甚,郊天時,稱疾不出,楷言芝龍無人臣禮。王獎其鳳節,命掌都察院事。鴻逵扇殿上,楷呵止之,兩人益怒。楷知不為所容,連請告去。途遇賊,截其一耳,乃芝龍所使部將楊耿也。漳州破,楷遂抑鬱而卒。

楷博綜群書,寒暑勿輟,尤邃於經學。

林蘭友,字翰荃,仙游人。崇禎四年進士。授臨桂知縣。擢南京御史。疏劾大學士張至發、薛國觀,吏部尚書田惟嘉等,因論嗣昌忠孝兩虧。貶浙江按察司照磨,與楷及黃道周、劉同升、趙士春稱「長安五諫」。遷光祿署丞。京師陷,薙髮自匿。為賊所執,拷掠備至。賊敗,南還。唐王用為太僕少卿,遷僉都御史。事敗,挈家遁海隅,十餘年卒。

熊汝霖,字雨殷,餘姚人。崇禎四年進士。授同安知縣。擢戶科給事中。疏陳用將之失,言:「自偏裨至副將,歷任有功,方可授節鉞。今足未履行陣,幕府已上首功。胥吏提虎旅,紈褲子握兵符,何由奮敵愾。若大將之選,宜召副將有功者,時賜面對,擇才者用之。廷臣推擇有誤,宜用文吏保舉連坐法。」帝納其言。已,言:「楊嗣昌未罪,盧象昇未褒,殊挫忠義氣。至為嗣昌畫策練餉、驅中原萬姓為盜者,原任給事中沈迅也。為嗣昌運籌、以三千人駐襄陽、城破輒走者,監紀主事餘爵也。為嗣昌援引、遭襄籓之陷、重賂陳新甲、嫁禍鄖撫袁繼咸者,今解任侯代之宋一鶴也。皆誤國之臣,宜罪。」不報。

京師戒嚴,汝林分守東直門。嘗召對,言:「將不任戰。敵南北往返,謹隨其後,如廝隸之於貴官,負弩前驅,望塵靡及。何名為將,何名為督師。」帝深然之。已,言:「有司察處者,不得濫舉邊才;監司察處者,不得遽躐巡撫。庶封疆重任,不為匪人借途。」又言:「自戒嚴以來,臣疏凡二十上。援剿機宜,百不行一。而所揣敵情,不幸言中矣。比者外縣難民紛紛入都,皆云避兵,不云避敵。霸州之破,敵猶不多殺掠,官軍繼至,始無孑遺。朝廷歲費數百萬金錢以養兵,豈欲毒我赤子。」帝惡其中有「飲泣地下」語,謫為福建按察司照磨。

福王立,召還。上疏言:「臣自丹陽來,知浙兵為邊兵所擊,火民居十餘里。邊帥有言,四鎮以殺掠獲封爵,我何憚不為。臣意四鎮必毅然北征,一雪此恥,今戀戀淮、揚,何也?況一鎮之餉多至六十萬,勢必不能供。即仿古籓鎮法,亦當在大河以北開屯設府,曾奧窔之內,而遽以籓籬視之。」頃之,言:「臣竊觀目前大勢,無論恢復未能,即偏安尚未可必。宜日討究兵餉戰守,乃專在恩怨異同。勛臣方鎮,舌鋒筆鍔是逞,近且以匿名帖逐舊臣,以疏遠宗人劾宰輔,中外紛紛,謂將復廠衛。夫廠衛樹威牟利,小民雞犬無寧日,先帝止此一節,未免府怨。前事不遠,後事之師。且先帝篤念宗籓,而聞寇先逃,誰死社稷;先帝隆重武臣,而叛降跋扈,肩背相踵;先帝委任勳臣,而京營銳卒徒為寇藉;先帝倚任內臣,而開門延敵,眾口喧傳;先帝不次擢用文臣,而邊才督撫,誰為捍禦,超遷宰執,羅拜賊庭。知前日之所以失,即知今日之所以得。及今不為,將待何時。」疏奏,停俸。尋補吏科右給事中。

初,馬士英薦阮大鋮,汝霖爭不可。及大鋮起佐兵部,汝霖又言:「大鋮以知兵用,當置有用地,不宜處中朝。」不聽。踰月,以奉使陛辭,言:「朝端議論日新,宮府揣摩日熟。自少宰樞貳悉廢廷推,四品監司竟晉詹尹。蹊徑疊出,謠諑繁興。一人未用,便目滿朝為黨人;一官外遷,輒訾當事為可殺。置國恤於罔聞,逞私圖而得志。黃白充庭,青紫塞路,六朝佳麗,復見今時。獨不思他日稅駕何地耶?」不報。

未幾,南京破,士英竄杭州。汝霖責其棄主,士英無以應。杭州亦破,與孫嘉績同起兵。魯王監國,擢右僉都御史,督師防江,戰屢敗。入海寧募兵萬人,進兵部右侍郎。唐王立閩中,遣劉中藻頒詔,汝霖出檄嚴拒之。順治三年進兵部尚書,從魯王泛海。明年以本官兼東閣大學士。又明年春,鄭彩憾汝霖,遣兵潛害之,并其幼子投海中。

錢肅樂,字希聲,鄞縣人。臨江知府若賡孫,寧國知府敬忠兄子也。崇禎十年成進士,授太倉知州。豪家奴與黠吏為奸,而兇徒結黨殺人,焚其屍。肅樂痛懲,皆斂手。又以朱白榜列善惡人名,械白榜者階下,予大杖。久之,杖者日少。嘗攝崑山、崇明事,兩縣民皆立碑頌德。遷刑部員外郎,尋丁內外艱。

順治二年,大兵取杭州,屬郡多迎降。閏六月,寧波鄉官議納款,肅樂建議起兵。諸生華夏、董志寧等遮拜肅樂倡首,士民集者數萬人,肅樂乃建牙行事。郡中監司守令皆逃,惟一同知治府事。肅樂索取倉庫籍,繕完守具,與總兵王之仁締盟共守。聞魯王在台州,遣舉人張煌言奉表請監國。會紹興、餘姚亦舉兵,王乃赴紹興行監國事。召肅樂為右僉都御史,畫錢塘而守。尋進右副都御史。當是時,之仁及大將方國安並加封爵,其兵食用寧波、紹興、臺州三郡田賦,不能繼,恒缺食。已,加兵部右侍郎。明年五月,軍食盡,悉散去。魯王航海,肅樂亦之舟山。唐王召之,甫入境,王已沒。遂隱海壇山,採山薯為食。明年,魯王次長垣,召為兵部尚書,薦用劉沂春、吳鐘巒等。明年拜肅樂東閣大學士。

唐王雖歿,而其將徐登華為守富寧,魯王遣大學士劉中藻攻之。登華欲降,疑未決,曰:「海上豈有天子?舟中豈有國公?」肅樂致書:「將軍獨不聞南宋之末二帝並在舟中乎?」登華遂降。鄭彩專柄,連殺熊汝霖、鄭遵謙。肅樂憂憤卒於舟,故相葉向高曾孫進晟葬之福清黃檗山。

劉中藻,福安人。由進士官行人。賊陷京師,薙髮,被搒掠。賊敗南還,事唐王。既事魯王,攻降福寧守之,移駐福安。大清兵破城,冠帶坐堂上,為文自祭,吞金屑死。

鄭遵謙,會稽人。為諸生。潞王以杭州降大清,遵謙倡眾起兵,事魯王,崎嶇浙、閩間。從王航海,與汝霖並為彩害。

沈宸荃,慈谿人。崇禎十三年進士。授行人,奉使旋里。福王立,復命。擢御史,疏陳五事,皆切時病。已,論群臣醜正黨邪,請王臥薪嘗膽,為雪恥報仇之計。尋薦詞臣黃道周、劉同升、葛世俊、徐水幵、吳偉業等。又言:「經略山東、河南者,王永吉、張縉彥也。永吉失機,先帝拔為總督,擁兵近甸,不救國危。縉彥官部曹,先帝驟擢典中樞,乃率先從賊。即加二人極刑,不為過。陛下屈法用之,而永吉觀望逗遛,縉彥狼狽南竄。死何以對先帝,生何以對陛下。昌平巡撫何謙失陷諸陵,罪亦當按。都城既陷,守土臣宜皆厲兵秣馬,以報國仇,乃賊塵未揚,輒先去以為民望。如河道總督黃希憲、山東巡撫丘祖德,尚可容偃臥家園乎!」疏入,謙、祖德等皆命逮治,永吉、縉彥不罪。時朝政大亂,宸荃獨持正,要人多疾之。明年以年例出為蘇松兵備僉事。未赴,南都破,宸荃舉兵邑中。魯王監國,擢右僉都御史。已而事敗,宸荃棄家從王海外。王次長垣,連擢至大學士。從王於舟山,又從泛海抵廈門、金門。後艤舟南日山,遭風,沒於海。

其邑子沈履祥嘗為知縣,監國時,以御史督餉臺州。城破,避山中,被獲死之。

贊曰:自甲申以後,明祚既終,不逾年而南都亦覆,勢固無可為矣。硃大典、張國維等抱區區之義,徒假名號於海濱,以支旦夕。而上替下陵,事無統紀,欲以收偏安之效,何可得乎。


\end{pinyinscope}