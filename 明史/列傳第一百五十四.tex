\article{列傳第一百五十四}

\begin{pinyinscope}
馬世奇吳麟徵周鳳翔劉理順汪偉吳甘來王章陳良謨陳純德申佳允成德許直金鉉

馬世奇,字君常,無錫人。祖濂,進士,桂林知府。世奇幼穎異,嗜學有文名。登崇禎四年進士,改庶吉士,授編修。十一年,帝遣詞臣分諭諸籓。世奇使山東、湖廣、江西諸王府,所至卻饋遺。還,進左諭德。父憂歸。

久之還朝,進左庶子。帝數召廷臣問禦寇策。世奇言:「闖、獻二賊,除獻易,除闖難。人心畏獻而附闖,非附闖也,苦兵也。今欲收人心,惟敕督撫鎮將嚴束部伍,使兵不虐民,民不苦兵,則亂可弭。」帝善其言,為下詔申飭。時寇警日亟,每召對,諸大臣無能畫一策。世奇歸邸,輒太息泣下,曰:「事不可為矣。」十七年三月,城陷。世奇方早食,投箸起,問帝安在、東宮二王安在,或言帝已出城,或言崩,或又言東宮二王被執。世奇曰:「嗟乎,吾不死安之!」其僕曰:「如太夫人何?」世奇曰:「正恐辱太夫人耳。」將自經,二妾朱、李盛飾前。世奇訝曰:「若以我死,將辭我去耶?」對曰:「聞主人盡節,我二人來從死耳。」世奇曰:「有是哉!」二妾並自經,世奇端坐,引帛自力縊乃死。先是,兵部主事成德將死,貽書世奇,以慷慨從容二義質焉。世奇曰:「勉哉元升。吾人見危授命,吾不為其難,誰為其難者!與君攜手黃泉,預訂斯盟,無忘息壤矣。」世奇修頤廣顙,揚眉大耳,砥名行,居館閣有聲,好推獎後進。為人廉,父死,蘇州推官倪長圩以贖鍰三千助喪。世奇辭曰:「蘇饑,留此可用振。」座主周延儒再相,世奇同郡遠嫌,除服不赴都。及還朝,延儒已賜死,親暱者率避去,世奇經紀其喪。其好義如此。贈禮部右侍郎,謚文忠。本朝賜謚文肅。

吳麟征,字聖生,海鹽人。天啟二年進士。除建昌府推官,擒豪猾,捕劇盜,治聲日聞。父憂歸。補興化府,廉公有威,僚屬莫敢以私進。

崇禎五年,擢吏科給事中,請罷內遣,言:「古用內臣以致亂,今用內臣以求治。君之於臣,猶父之於子,未有信僕從,舍其子,求家之理者。」又言:「安民之本在守令。郡守廉,縣令不敢貪;郡守慈,縣令不敢虐;郡守精明,縣令不敢叢脞。宜仿宣宗用況鐘故事,精擇而禮遣之,重以璽書,假便宜久任。民生疾苦,吏治臧否,使得自達天子。」時不能行。麟徵在諫垣,直聲甚著。尋上疏乞假葬父。既去,貽言路公揭,謂:「自言官積輕,廟堂之上往往反其言而用之。奸人窺見此旨,明告君父,目為朋黨,自稱孤立,下背公論,上竊主權。諸君子宜盡化沾沾之意,毋落其彀中,使清流之禍再見明時。」居久之,還朝。劾吏部尚書田唯嘉贓污,唯嘉罷去。再遷刑科給事中,丁繼母憂。服闋,起吏科都給事中,時貨賂公行,銓曹資格盡廢。麟徵上言:「限年平配,固銓政之弊,然舍此無以待中才。今遷轉如流,不循資格,巧者速化,拙者積薪,開奔競之門,無益軍國之計。」帝深然之。

十七年春,推太常少卿。未幾,賊薄京師。麟徵奉命守西直門。門當賊衝,賊詐為勤王兵求入。中官欲納之,麟征不可,以土石堅塞其門,募死士縋城襲擊之,多所斬獲。賊攻益急,麟徵趨入朝,欲見帝白事。至午門,魏藻德引麟征手曰:「國家如天之福,必無他虞。旦夕兵餉集,公何遽為?」引之出,遂還西直門。明日城陷,欲還邸,已為賊所據。乃入道旁祠,作書訣家人曰:「祖宗二百七十餘年宗社,一旦至此,雖上有亢龍之悔,下有魚爛之殃,而身居諫垣,無所匡救,法當褫服。殮用角巾青衫,覆以單衾,以志吾哀。」解帶自經。家人救之蘇,環泣請曰:「待祝孝廉至,一訣可乎?」許之。祝孝廉名淵,嘗救劉宗周下獄,與麟徵善者也。明日,淵至。麟征慷慨曰:「憶登第時夢隱士劉宗周吟文信國《零丁洋詩》,今山河碎矣,不死何為!」酌酒與淵別,遂自經,淵為視含殮而去。贈兵部右侍郎,謚忠節。本朝賜謚貞肅。

方賊之陷山西也,薊遼總督王永吉請撤寧遠吳三桂兵守關門,選士卒西行遏寇,即京師警,旦夕可援。天子下其議,麟征深然之。輔臣陳演、魏藻德不可,謂:「無故棄地二百里,臣不敢任其咎。」引漢棄涼州為證。麟徵復為議數百言,六科不署名,獨疏昌言,弗省。及烽煙徹大內,帝始悔不用麟徵言,旨下永吉,永吉馳出關,徙寧遠五十萬眾,日行數十里,十六日入關,二十日抵豐潤,而京師已陷矣。城破,八門齊啟,惟西直門堅塞不能通。至五月七日,集民夫發掘乃開。

周鳳翔,字儀伯,浙江山陰人。崇禎元年進士。改庶吉士,授編修。遷南京國子司業。靈璧侯奴辱諸生,鳳翔執付法司。歷中允、諭德,為東宮講官。嘗召對平臺,陳滅寇策,言論慷慨,帝為悚聽。軍需急,議稅間架錢。鳳翔曰:「事至此,急宜收人心,尚可括民財搖國勢耶!」亡何,京師陷,莊烈帝殉社稷,有訛傳駕南幸者。鳳翔不知帝所在,趨入朝。見魏藻德、陳演、侯恂、宋企郊等群入,而賊李自成據御坐受朝賀。鳳翔至殿前大哭,急從左掖門趨出,賊亦不問。歸至邸,作書辭二親,題詩壁間自經。詩曰:「碧血九原依聖主,白頭二老哭忠魂。」天下悲之,去帝崩纔兩日也。後贈禮部右侍郎,謚文節。本朝賜謚文忠。

劉理順,字復禮,杞縣人。萬曆中舉於鄉。十赴會試,至崇禎七年始中式。及廷對,帝親擢第一,還宮喜曰:「朕今日得一耆碩矣。」拜修撰。益勤學,非其人不與交。

十二年春,畿輔告警,疏陳作士氣、矜窮民、簡良吏、定師期、信賞罰、招脅從六事。歷南京司業、左中允、右諭德,入侍經筵兼東宮講官。楊嗣昌奪情入閣,理順昌言於朝,嗣昌奪其講官。開封垂陷,理順建議河北設重臣,練敢死士為後圖,疏格不行。嗣昌、薛國觀、周延儒迭用事,理順一無所附麗。出溫體仁門,言論不少徇。

賊犯京師急,守卒缺餉,陰雨饑凍。理順詣朝房語諸執政,急請帑,眾唯唯。理順太息歸,捐家貲犒守城卒。僚友問進止,正色曰:「存亡視國,尚須商酌耶!」城破,妻萬、妾李請先死。既絕,理順大書曰:「成仁取義,孔、孟所傳。文信踐之,吾何不然!」書畢投繯,年六十三。僕四人皆從死。群盜多中州人,入唁曰:「此吾鄉杞縣劉狀元也,居鄉厚德,何遽死?」羅拜號泣而去。後贈詹事,謚文正。本朝賜謚文烈。

汪偉,字叔度,休寧人,寄籍上元。崇禎元年進士。十一年,由慈谿知縣行取。帝以國家多故,朝臣詞苑起家,儒緩不習吏事,無以理紛禦變,改舊例,擇知推治行卓絕者入翰林。偉擢檢討,給假歸。還朝,充東宮講官。

十六年,賊陷承天、荊、襄。偉以留都根本之地,上《江防綢繆疏》,言:「金陵城周圍百二十里,雖十萬眾不能守。議者謂無守城法,有防江法。賊自北來,淮安為要;自上游來,九江為要;禦淮所以禦江,守九江所以守金陵也。淮有史可法,屹然保障;九江一郡,宜設重臣鎮之。自是而上之至於武昌,下之至於太平、采石、浦口,命南京兵部大臣建牙分閫,以接聲援,而金陵之門戶固矣。南京兵部有重兵而無用,操江欲用兵而無人,宜使緩急相應。而府尹、府丞之官,重其權,久其任,聯百萬士民心,以分兵部操江之責。」帝嘉納之,乃設九江總督。又言:「兵額既虧,宜以衛所官舍餘丁補伍操練,修治兵船,以資防禦。額餉不足,暫借鹽課、漕米給之。」所條奏皆切時務。

明年三月,賊兵東犯。偉語閣臣:「事急矣,亟遣大僚守畿郡。都中城守,文自內閣,武自公侯伯以下,各率子弟畫地守。庶民統以紳士,家自為守。而京軍分番巡徼,以待勤王之師。」魏藻德笑曰:「大僚守畿輔,誰肯者?」偉曰:「此何等時,猶較尊卑、計安危耶?請以一劇郡見委。」藻德哂其早計。未幾,真定遊擊謝加福殺巡撫徐標迎賊。偉泣曰:「事至此乎!」作書寄友人曰:「賊據真定,奸人滿都城,外郡上供絲粟不至,諸臣無一可支危亡者,如聖主何!平時誤國之人,終日言門戶而不顧朝廷,今當何處伸狂喙耶!」賊薄都城,守兵乏餉,不得食,偉市餅餌以饋。已而城陷,偉歸寓,語繼室耿善撫幼子。耿泣曰:「我獨不能從公死乎!」因以幼子屬其弟,衣新衣,上下縫,引刀自剄不殊,復投繯遂絕,時年二十三。偉欣然曰:「是成吾志。」移其屍於堂,貽子觀書,勉以忠孝,乃自經。贈少詹事,謚文烈。本朝賜謚文毅。

吳甘來,字和受,江西新昌人。父之才,西安府同知。甘來與兄泰來同舉鄉試。崇禎改元,甘來成進士,授中書舍人。後三年,泰來亦成進士,授南京太常博士。

五年,甘來擢刑科給事中。七年,西北大旱,秦、晉人相食,疏請發粟以振,而言:「山西總兵張應昌等半殺難民以冒功,中州諸郡畏曹變蛟兵甚於賊。陛下生之而不能,武臣殺之而不顧,臣實痛之。」又言:「賞罰者,將將大機權也。陛下加意邊陲,賞無延格。乃紅夷獻俘,黔、蜀爭功,昌黎死守,功猶待勘,急則用其死綏,緩則束以文法。且封疆之罰,武與文二,內與外二,士卒與將帥二。受命建牙,或逮或逐,以封疆罪罪之;而跋扈將帥,罪狀已暴,止於戴罪。偏裨不能令士卒,將帥不能令偏裨,督撫不能令將帥,將聽賊自來自去,誰為陛下翦凶逆者?」憂歸。服闋,起吏科,進兵科右給事中,乞假歸。

十五年,起歷戶科都給事中。中外多故,荊、襄數郡,賊未至而撫道諸臣率稱護籓以去。甘來曰:「若爾,則是棄地方而逃也。城社人民,誰與守者?」乃上疏曰:「天子眾建親親,將使屏籓帝室,故曰『宗子維城』。乃烽火纔傳,一朝委去以為民望,而諸臣猶嘵嘵以擁衛自功,掩其失地之罪。是維城為可留可去之人,名都為可守可棄之土,撫道為可有可無之官。功罪不明,賞罰不著,莫此為甚!」疏入,帝大嘉歎。一日,帝詰戶部尚書倪元璐餉額,甘來曰:「臣科與戶曹表裏,餉可按籍稽。臣所慮者,兵聞賊而逃,民見賊而喜,恐非無餉之患,而無民之患。宜急輕賦稅,收人心。」帝頷之。

甘來遘疾,連請告。會帝命編修陳名夏掌戶科,甘來喜得代。不數日,賊薄都城。時泰來官禮部員外郎矣,甘來屬兄歸事母,而自誓必死。明日,城陷,有言駕南幸者,甘來曰:「主上明決,必不輕出。」乃疾走皇城,不得入。返檢几上疏草曰:「當賊寇縱橫,徒持議論,無益豪末。」盡取焚之,毋釣後世名,遂投繯死。贈太常卿,謚忠節。本朝賜謚莊介。

王章,字漢臣,武進人。崇禎元年進士。授諸暨知縣。少孤,母訓之嚴。及為令,祖帳歸少暮,母訶跪予杖,曰:「朝廷以百里授酒人乎!」章伏地不敢仰視。親友為力解,乃已。治諸暨有聲。甫半歲,以才調鄞縣。諸暨民與鄞民爭挽章,至相嘩。治鄞益有聲,數注上考。

十一年,行取入都。時有考選翰林之命,行取者爭奔競,給事中陳啟新論之。帝怒,命吏部上訪冊,罪廷臣濫徇者。尚書姜逢元、王業浩,給事中傅元初,御史禹好善等六人閒住;給事中孫晉、御史李右讜等三人降調;給事中劉含輝、御史劉興秀等十一人貶二秩視事。吏部尚書田維嘉等乃請先推部曹,凡推二十二人,章與焉,授工部主事。章及任濬、塗必泓、李嗣京欲疏辨,憚為首獲罪。李士淳者耄矣,四人不告而首其名,士淳知之,懼且怒,與章等大詬。而帝知維嘉有私,詔許與考。又以為首者必良士也,擢士淳編修,章等皆御史。章上疏請罷內操,寬江南逋賦。

明年出按甘肅,持風紀,飭邊防。西部寇莊浪,巡撫急徵兵。章曰:「貧寇索食耳。」策馬入其帳,眾羅拜乞降,乃稍給之食。兩河旱,章檄城隍神:「御史受錢或戕害人,神殛御史,毋虐民。神血食茲土,不能請上帝蘇一方,當奏天子易爾位。」檄焚,雨大注。邊卒貸武弁金,償以賊首,武弁以冒功,坐是數召邊釁。章著令,非大舉毋得以零級冒功。劾罷巡撫劉鎬貪惰。又所部十道監司,劾罷其四。母憂歸。服闋,還朝,巡視京營,按籍額軍十一萬有奇。喜曰:「兵至十萬,猶可為也。」及閱視,半死者,餘冒伍,憊甚,矢折刀缺,聞炮聲掩耳,馬未馳輒墮。而司農缺餉,半歲不發。章屢疏請帑,不報。

踰月,賊陷真定,京師大震。襄城伯李國禎發營卒五萬營城外,章與給事中光時亨守阜成門。城內外堞凡十五萬四千有奇,三堞一卒。三月初登陴,閱十日始一還邸,櫛沐易新衣冠。家人大駭,章不應。賊傅城下,章手發二炮,賊少卻。頃之,各門炮聲絕。時亨攝章走,章厲聲曰:「事至此,猶惜死耶!」時亨曰:「死此與士卒何別?入朝訪上所在,不獲則死,死未晚也。」章從之,與時亨並馬行。俄賊突至,呼下馬。時亨倉皇下馬跪,章持鞭不顧,叱曰:「吾視軍御史也,誰敢犯!」賊刺章股,墮。章罵曰:「逆賊!勤王兵且至。我死,爾滅不旋踵矣。」賊怒,攢槊刺殺章而去。抵暮,家人覓屍,猶一手據地坐,張口怒目,勃勃如叱賊狀。妻姜在籍,聞之,一慟而絕。贈大理寺卿,謚忠烈。本朝賜謚節愍。次子之栻仕閩為職方主事,亦死難。

陳良謨,字士亮,鄞人。崇禎四年進士,授大理推官。初名天工。莊烈帝虔事上帝,詔群臣名「天」者悉改之,乃改良謨。在職六年,兩注上考。行取陛見,擢御史。

十二年,出按四川。期滿當代,再留任。時流寇大入蜀,詔良謨專護蜀王,巡撫邵捷春專辦賊。良謨飭守具,堅壁清野。賊犯成都,遣將據要害為掎角。一再戰,賊潰奔。帝聞賊擾蜀,下詔責良謨,已聞其善守禦,乃優旨賜銀幣。及還朝,賊勢益迫,所規畫率不行,而京師陷矣。

良謨嘗夢拜文文山於堂下,文山揖之上:「公與予先後一揆,何下拜為?」覺而異之。及是城陷,良謨方移疾臥邸中,一慟幾絕,自是水漿不入口。或勸良謨無死,不答。謂邑子李天葆曰:「吾為國死,義不顧家。惟是母老,先君莫葬,繼嗣未定,須一言耳。」因賦詩付天葆。未幾,聞帝崩煤山,大慟曰:「主上不冕服,臣子敢具冠帶乎!吾巾褻,安所得明巾。」天葆以巾進。良謨著巾,藍便服,起入戶。妾時氏隨之,遂與妾俱自縊死。時氏,京師人,年十八。良謨踰五十無子,以禮納之,侍良謨百三日耳。良謨既卒,其族人以其兄之子久樞為之後。未幾,久樞亦卒,良謨竟無後。贈太僕卿,謚恭愍。本朝賜謚恭潔。

陳純德,字靜生,零陵人。為諸生,以學行稱。嘗夜泊洞庭,為盜窘,躍出墮水,再躍入洲渚。比曉,坐蘆葦中,去泊舟數十丈。

崇禎十三年成進士,年已六十矣。莊烈帝召諸進士,咨以時事。純德奏稱旨,立擢御史,巡按山西。七月,部內嚴霜,民凍餒。純德上疏請恤,因陳抽練之弊,言:「兵抽則人失故居,無父母妻子之依,田園丘壟之戀,思歸則逃,逢敵則潰。抽餘者即以餉薄而安於無用,抽去者又以遠調而不樂為用。伍虛而餉仍在,不歸主帥,則歸偏裨,樂其逃而利其餉,凡藉以營求遷秩,皆是物也。精神不以束伍,而以侵餉;厚餉不以養士,而以求官。伍虛則無人,安望其練;餉糜則愈缺,安望其充。此今日行間大弊也。」帝不能用。

還朝,督畿輔學政。將出按部,都城陷。賊下令百官以某日入見,眾攝純德入,還邸慟哭,遂自經。京山人秦嘉系買地葬之永定門外,立石表墓焉。贈太僕卿,謚恭節。

申佳允,字孔嘉。永年人。崇禎四年進士。授儀封知縣。縣故多盜,佳允嚴保甲法,盜無所容。霪雨河決,艤舟怒濤中,塞其口。捕大猾置之法。以才調杞縣。八年,賊掃地王率萬人來攻,城土垣多圮。佳允募死士擊走賊,因甓其城。唐王聿鍵勤王,將抵開封。諸大吏惴恐,集議曰:「留之,不聽。行,守土者且得罪。」佳允曰:「惟周王可留之。」眾稱善,用其計。

治行卓異,擢吏部文選主事,上備邊五策。進考功員外郎,佐京察。大學士薛國觀傾少詹事文安之。安之,佳允座主也,事連佳允,左遷南京國子博士。

久之,遷大理評事,進太僕丞,閱馬近畿。聞李自成破居庸,歎曰:「京師不守矣!君父有難,焉逃死?」馳入都,遍謁大臣為畫戰守策,皆不省。貽子涵光書曰:「行己曰義,順數曰命;義不可背也,命不可違也。天下事莫不壞於貪生而畏死。死於疾,死於利,死於刑戮,於房幃,於鬥戰,均死也,死數者不死君父,蓋亦不善用死矣。今日之事,君父之事,死義也,猶命也,我則行之。」京師陷,冠帶辭母,策馬至王恭廠,從者請易服以避賊。佳允曰:「吾起微賤,食祿十三年。國事至此,敢愛死乎!」兩僕環守不去,紿之曰:「吾不死也,我將擇善地焉。」下馬,旁見灌畦巨井,急躍入。僕號呼,欲出之。佳允亦呼曰:「告太安人,有子作忠臣,勿過傷也。」遂死,年四十二,贈太僕少卿,謚節愍。本朝賜謚端愍。

成德,字元升,霍州人,依舅氏占籍懷柔。崇禎四年進士。除滋陽知縣。性剛介,清操絕俗,疾惡若仇。文震孟入都,德郊迎,執弟子禮,語刺溫體仁,體仁聞而恨之。兗州知府增餉額,德固爭,又嘗捕治其牙爪吏。知府怒,讒於御史禹好善。好善,體仁客也,誣德貪虐,逮入京。滋陽民詣闕訟冤。震孟在閣,亦為之稱枉。德道中具疏極論體仁罪,,而震孟已被體仁擠而去之。好善再劾德,言其疏出震孟手,帝不之究。德母張伺體仁長安街,繞輿大罵,拾瓦礫擲之。體仁恚,疏聞於朝。詔五城御史驅逐,移德鎮撫獄掠治。杖六十午門外,戍邊。坐贓六千有奇。而給體仁校尉五十人護出入。

德居戍所七年,用御史詹兆恒薦,起如皋知縣。尋擢武庫主事。以母老辭,不允,乃就道。至則上言:「年來中外多故,居官者爵祿迷心,廉恥道喪。陛下御極十七年,何仗節死義之寥寥也!宋臣張栻有言:『仗節死義之臣,當於犯顏諫諍中求之。』夫犯顏諫諍何難,在朝廷養之而已。表厥宅里,所以伸忠臣孝子於生前;殊厥井疆,所以誅亂臣賊子於未死。茍死敵者無功,則媚敵者且無罪;死賊者褒揚不亟,則從賊者恬而不知畏也。」未幾,城破,不知帝所在,旁皇廳事。已,趨至午門,見兵部尚書張縉彥自賊所出。德以頭觸縉彥胸,且詈之,俄聞帝崩,痛哭。持雞酒奔東華門,奠梓宮於茶棚之下,觸地流血。賊露刃脅之,不為動。奠畢歸家,有妹年二十餘未嫁,德顧之曰:「我死,汝何依?」妹曰:「兄死,妹請前。」德稱善,哭而視其縊。入別其母,哭盡哀,出而自縊。母見子女皆死,亦投繯死。先是,懷柔城破,德父文桂遇害,家屬盡沒。妻劉在京,以征德贓急,憂悸死。至是,又闔門死難,惟幼子先寄友人家獲存。贈德光祿卿,謚忠毅。本朝賜謚介愍。

許直,字若魯,如皋人。崇禎七年成進士。出文震孟之門,以名節自砥,除義烏知縣。母憂歸,哀毀骨立,終喪蔬食,寢柩旁。補廣東惠來縣。用清望,徵授吏部文選主事,進考功員外郎。

賊薄都城,約同官出貲饗士,為死守計。城陷,賊令百官報名。直曰:「身可殺,志不可奪。」有傳帝南狩者,直將往從。見賊騎塞道,出門輒返,曰:「四方兵戈,駕焉往?國亂不匡,君危無濟,我何生為!」已,知帝崩,一慟幾絕。客以七十老父為解,直曰:「不死,辱及所生。」賦絕命詩六章,闔戶自經。越旦視之,神氣如生。贈太僕卿,謚忠節。本朝賜謚忠愍。

直有族子德溥者,在南,聞莊烈帝崩,大哭數日。揚州陷,又哭數日。每獨坐輒慟哭,食必以崇禎錢一枚置几上,祭而後食,食已復哭。又刺其兩臂曰:「生為明臣,死為明鬼。」事發,死西市。

金鉉,字伯玉,武進人,占籍順天之大興。祖汝升,南京戶部郎中。父顯名,汀州知府。鉉少有大志,以聖賢自期許。年十八舉鄉試第一。明年,崇禎改元,成進士。不習為吏,改揚州府教授,日訓諸生闡濂、洛正學。燕居言動,俱有規格,諸生嚴憚之。歷國子博士、工部主事。

帝方銳意綜核,疑廷臣朋黨營私。度支告匱,四方亟用兵,餉不敷,遣中官張彝憲總理戶、工二部,建專署,檄諸曹謁見,禮視堂官。鉉恥之,再疏爭,不納。乃約兩部諸僚,私謁者眾唾其面,彞憲慍甚。鉉當榷稅杭州,辭疾請假。彞憲摭火器不中程,劾鉉落職。鉉杜門謝客,躬爨以養父母。

十七年春,始起兵部主事,巡視皇城。聞大同陷,疏曰:「宣、大,京師北門。大同陷則宣府危,宣府危,大事去矣。請急撤回監宣中官杜勛,專任巡撫朱之馮。勛二心僨事,之馮忠懇,可屬大事。」不報,未幾,勛以宣府下賊,賊殺之馮,烽火偪京師。鉉奔告母:「母可且逃匿。兒受國恩,義當死。」鉉母章時年八十餘矣,呵曰:「爾受國恩,我不受國恩乎!廡下井,是我死所也。」鉉哭而去。

城破,趨入朝,宮人紛紛出。知帝已崩,解牙牌拜授家人,即投金水河。家人爭前挽之,鉉怒,口嚙其臂,得脫,遂躍入水。水淺,濡首泥中乃絕。母聞即投井,妾王隨之,皆死。賊踞大內,踰月始去。金水河冠袍泛泛見水上,內官群指之曰:「此金兵部也。」弟錝辨其屍,驗網巾環,得鉉首歸,合以木身,如禮而殮。事竣,錝自經。後贈鉉太僕少卿,謚忠節。本朝賜謚忠潔。

右范景文至鉉二十有一人,皆自引決。其他率委蛇見賊。賊以大僚多誤國,概囚縶之。庶官則或用或否,用者下吏政府銓除,不用者諸偽將搒掠取其貲,大氐降者十七,刑者十三。福王時,以六等罪治諸從逆者。而文武臣殉難並予贈蔭祭葬,且建旌忠祠於都城焉。曰正祀文臣,祀景文以下二十人,及大同巡撫衛景瑗、宣府巡撫朱之馮、布衣湯文瓊、諸生許琰四人。曰正祀武臣,祀新樂侯劉文炳、惠安伯張慶臻、襄城伯李國楨、駙馬都尉鞏永固、左都督劉文耀、山西總兵官周遇吉、遼東總兵官吳襄七人。曰正祀內臣,祀太監王承恩一人。曰正祀婦人,祀烈婦成德母張氏,金鉉母章氏,汪偉妻耿氏,劉理順妻萬氏、妾李氏,馬世奇妾朱氏、李氏,陳良謨妾時氏,吳襄妻祖氏九人。曰附祀文臣,祀進士孟章明及郎中徐有聲,給事中顧鋐、彭琯,御史俞志虞,總督徐標,副使朱廷煥七人。曰附祀武臣,祀成國公朱純臣、鎮遠侯顧肇迹、定遠侯鄧文明、武定侯郭培民、陽武侯薛濂、永康侯徐錫登、西寧侯宋裕德、懷寧侯孫維籓、彰武伯楊崇猷、宣城伯衛時春、清平伯吳遵周、新建伯王先通、安鄉伯張光燦、右都督方履泰、錦衣衛千戶李國祿十五人。曰附祀內臣,祀太監李鳳翔、王之心、高時明、褚憲章、方正化、張國元六人。有司春秋致祭。然顧鋐、彭琯、俞志虞輩,特為賊拷死,諸侯伯亦大半以兵死。而郎中周之茂、員外郎寧承烈、中書宋天顯、署丞于騰雲、兵馬指揮姚成、知州馬象乾皆以不屈死,顧未邀贈恤也。

徐有聲,字聞復,金壇人。登鄉薦,崇禎十三年特擢戶部主事,歷員外郎、郎中。督餉大同。城陷,被執不屈死。福王時,贈太僕少卿。

徐標,字準明,濟寧人。天啟五年進士。崇禎時,歷官淮徐道參議。十六年二月,超擢右僉都御史,巡撫保定。陛見,請重邊防,擇守令,用車戰禦敵,招流民墾荒。帝深嘉之。李自成陷山西,警日逼,加標兵部侍郎,總督畿南、山東、河北軍務,仍兼巡撫,移駐真定以遏賊。無何,賊遣使諭降,標毀檄戮其使。賊別將掠畿輔,真定知府邱茂華移妻孥出城,標執茂華下之獄。中軍謝加福伺標登城畫守禦策,鼓眾殺之,出茂華於獄。數日而賊至,以城降。福王時,贈標兵部尚書。

朱廷煥,單縣人。崇禎七年進士。除工部主事,歷知廬州、大名二府,即以兵備副使分巡大名。十七年,賊逼畿輔,廷煥嚴守備。賊傳檄入城,怒而碎之。三月四日,賊來攻,軍民皆走,城遂陷。被執不屈死。福王時,贈右副都御史。

周之茂,字松如,黃麻人。崇禎七年進士。歷官工部郎中。服闋,需次都下。賊搜得之,迫使跪,不屈,折其臂而死。

寧承烈,字養純,大興人。舉於鄉,歷魏縣教諭,戶部司務,進本部員外郎,管太倉銀庫。城陷,自經於官廨。

宋天顯,松江華亭人。由國子生官內閣中書舍人。為賊所獲,自經。

于騰雲,順天人。為光祿置丞。賊至,語其妻曰:「我朝臣,汝亦命婦,可污賊耶!」夫婦並服命服,從容投繯死。

姚成,字孝威,餘姚人。由禮部儒士為北城兵馬司副指揮。城陷,自縊死。

馬象乾,京師人。舉於鄉,官濮州知州。方里居,賊入,率妻及子女五人並自縊。

至若御史馮垣登、兵部員外郎鄭逢蘭、行人謝於宣皆拷死,郎中李逢甲,拷掠久之,逼令縊死。與鋐、琯、志虞皆獲贈太僕少卿,而垣登、于宣至謚忠節。行取知縣鄒逢吉拷死,贈太僕寺丞。時南北阻絕,皆未能核實也。湯文瓊、許琰事載《忠義傳》。

贊曰:《傳》云「君子居其位,則思死其官」。夫忠貞之士,臨危授命,豈矯厲一時,邀名身後哉!分誼所在,確然有以自持而不亂也。馬世奇等皆負貞亮之操,勵志植節,不欺其素,故能從容蹈義,如出一轍,可謂得其所安者矣。


\end{pinyinscope}