\article{穆宗本紀}

\begin{pinyinscope}
穆宗契天隆道淵懿寬仁顯文光武純德弘孝莊皇帝,諱載垕,世宗第三子也。母杜康妃。嘉靖十八年二月封裕王,與莊敬太子、景恭王同日受冊。已而莊敬薨,世宗以王長且賢,繼序已定,而中外危疑,屢有言者,乃令景王之國。

四十五年十二月庚子,世宗崩。壬子,即皇帝位。以明年為隆慶元年,大赦天下。先朝政令不便者,皆以遺詔改之。召用建言得罪諸臣,死者恤錄。方士悉付法司治罪。罷一切齋醮工作及例外採買。免明年天下田賦之半,及嘉靖四十三年以前逋賦。釋戶部主事海瑞於獄。是年,土魯番入貢。

隆慶元年春正月丙寅,罷睿宗明堂配享。戊辰,復鄭王厚烷爵。丁丑,追尊母康妃為孝恪皇太后。二月戊子,祭大社大稷。乙未,冊妃陳氏為皇后。吏部侍郎陳以勤為禮部尚書兼文淵閣大學士,禮部侍郎張居正為吏部左侍郎兼東閣大學士,預機務。三月壬申,葬肅皇帝於永陵。乙酉,土蠻犯遼陽,指揮王承德戰歿。夏四月丙戌朔,享太廟。丙午,禁屬國毋獻珍禽異獸。丁未,御經筵。五月己未,黃河決口工成。辛酉,祀地於北郊。丁丑,高拱罷。六月戊戌,以霪雨修省,素服避殿,御皇極門視事。是月,新河復決。秋七月辛巳,招撫山東、河南被災流民,復五年。八月癸未朔,釋奠於先師孔子。九月乙卯,俺答寇大同,詔嚴戰守。癸亥,俺答陷石州,殺知州王亮采,掠交城、文水。壬申,土蠻犯薊鎮,掠昌黎、盧龍,至於灤河。詔宣大總督侍郎王之誥還駐懷來,巡撫都御史曹亨駐兵通州。甲戌,郭朴致仕。免襄陽、鄖陽被災秋糧。乙亥,總兵官李世忠援永平,與敵戰於撫寧,京師戒嚴。冬十月丙戌,寇退,京師解嚴。甲辰,諭群臣議邊防事宜。寧夏總兵官雷龍出塞邀擊河套部,敗之。十一月癸亥,祀天於南郊。是年,廣東賊大起。琉球入貢。

二年春正月己卯,給事中石星疏陳六事,杖闕下,斥為民。二月丁酉,寇犯柴溝堡,守備韓尚忠戰死。己亥,耕耤田。丁未,如天壽山,謁長陵、永陵。庚戌,還宮,免所過田租有差。三月辛酉,立皇子翊鈞為皇太子,詔赦天下。乙丑,廣西總兵官俞大猷討廣東賊。戊辰,賜羅萬化等進士及第、出身有差。丙子,幸南海子。戊寅,京師地震,命百官修省。夏六月庚辰,遣使兩畿錄囚。己丑,廣東賊曾一本寇廣州,殺知縣劉師顏。秋七月己酉,賊入廉州。丙寅,徐階致仕。冬十月戊寅,免南畿被災秋糧,振淮、徐饑。己亥,廢遼王憲節為庶人。甲辰,免畿內、河南被災秋糧,十一月壬子,宣府總兵官馬芳襲俺答於長水海子,又敗之鞍子山。辛酉,免江西被災稅糧,戊辰,祀天於南郊。己巳,命廣東、福建督撫將領會剿曾一本。十二月庚寅,世宗神主祔太廟。丁酉,限勛戚莊田。是年,琉球入貢。

三年春正月壬子,大同總兵官趙岢敗俺答於弘賜堡。二月庚辰,免陜西被災秋糧。三月戊辰,曾一本陷碣石衛,裨將周雲翔殺參將耿宗元叛,附於賊。夏四月己丑,總兵官雷龍出塞襲河套部,敗之。五月庚戌,總兵官郭成等破賊於平山,周雲翔伏誅。甲寅,御史詹仰庇請罷靡費,斥為民。秋七月壬午,河決沛縣。乙酉,詔天下有司實修積穀備荒之政。壬辰,遣使振沿河被災州縣。八月癸丑,廣東賊平,曾一本伏誅。壬戌,禮部尚書趙貞吉兼文淵閣大學士,預機務。丁卯,振南畿、浙江、山東水災。九月丙子,俺答犯大同,掠山陰、應州、懷仁、渾源。辛卯,大閱。冬十一月甲戌,祀天於南郊。庚辰,京師地震有聲,敕修省。十二月己亥,命廠衛密訪部院政事。庚申,召高拱復入閣。乙丑,尚寶寺丞鄭履淳以言事廷杖下獄。是冬,免兩畿、山東、浙江、河南、湖廣稅糧。是年,陜西賊起。琉球、土魯番入貢。

四年春正月己巳朔,日有食之,免朝賀。辛未,避殿修省。是月,倭入廣海衛城。二月乙丑,分設三大營文武提督六人。夏四月戊戌,京師地震。丙午,俺答寇大同、宣府,官兵拒卻之。是月,陜西賊寇四川。五月癸酉,給事中李己諫買金寶,廷杖下獄。秋七月己巳,禁章奏浮冗。命撫、按官嚴禁有司酷刑。戊子,陳以勤致仕。乙未,免四川被災稅糧。八月庚戌,宣、大告警,敕邊備。九月癸酉,陜西水災,蠲振有差。甲戌,河決邳州。壬午,免北畿、湖廣被災稅糧。癸未,寇犯大同,副總兵錢棟戰死。戊子,犯錦州,總兵官王治道等戰死。甲午,罷京營文武提督,置總督協理大臣。冬十月癸卯,俺答孫把漢那吉來降。丁未,以把漢那吉為指揮使。壬戌,考察給事中、御史。十一月丁丑,俺答乞封。己卯,祀天於南郊。乙酉,趙貞吉罷。己丑,禮部尚書殷士儋兼文淵閣大學士,預機務。十二月丁酉,俺答執叛人趙全等九人來獻,詔遣把漢那吉歸,厚賜之。乙卯,受俘,磔趙全等於市。

五年春二月甲午,廷臣及朝覲官謁皇太子於文華左門。己未,封皇子翊鏐為潞王。三月己卯,賜張元忭等進士及第、出身有差。己丑,封俺答為順義王。夏四月甲午反復的過程。,河復決邳州。五月壬戌,古田僮賊平。戊寅,李春芳致仕。六月辛卯,京師地震者三,敕修省。甲辰,授河套部長吉能為都督同知。甲寅,順義王俺答貢馬,告廟受賀。丙辰,俺答執趙全餘黨十三人來獻。秋八月癸卯,許河套部互市。九月癸未,三鎮貢市成。冬十月己亥,河南、山東大水,申飭河防。十一月己巳,殷士儋致仕。是年,琉球、土魯番入貢。

六年春正月辛未,築徐州至宿遷隄三百七十里。二月丙申,倭寇廣東,陷神電衛,大掠。山寇復起。閏月丁卯恭寬信敏惠儒家倡導的賢者的五種品德。恭,指恭敬、肅,御皇極殿門,疾作,遽還宮。乙亥,倭寇高、雷,官軍擊敗之。夏四月戊辰,禮部尚書高儀兼文淵閣大學士,預機務。五月壬辰,免廣東用兵諸郡逋賦。己酉,大漸,召大學士高拱、張居正、高儀受顧命。庚戌,崩於乾清宮,年三十有六。七月丙戌,上尊謚,廟號穆宗,葬昭陵。

贊曰:穆宗在位六載,端拱寡營,躬行儉約,尚食歲省巨萬。許俺答封貢,減賦息民,邊陲寧謐。繼體守文,可稱令主矣。第柄臣相軋,門戶漸開,而帝未能振肅乾綱,矯除積習,蓋亦寬恕有餘,而剛明不足者歟!


\end{pinyinscope}