\article{列傳第一百八十五 孝義二}

\begin{pinyinscope}
○王俊劉準楊敬石鼐任鏜史五常周敖鄭榮瑄葉文榮傅檝楊成章謝用何競王原黃璽歸鉞族子繡何麟孫清宋顯章李豫劉憲羅璋等容師偃劉靜溫鉞俞孜張震孫文崔鑒唐儼丘緒張鈞張承相等王在復王抃等夏子孝阿寄趙重華謝廣王世名李文詠王應元等孔金子良楊通照弟通傑浦邵等張清雅白精忠等

王俊,城武人。父為順天府知事。母卒於官舍,俊扶櫬還葬,刈草萊為茇舍,寢處塋側。野火延巘將及,俊叩首慟哭,火及塋樹而止。正統三年被旌。

劉準者,唐山諸生。父喪,廬墓。冬月野火將及塚樹,準悲號告天,火遂息。正統六年旌表。

楊敬者,歸德人。父歿於陣,為木主招魂以葬。每讀書至戰陣之事,輒隕涕不止。母歿,柩在堂。鄰家失火,烈焰甚迫,敬撫柩哀號,風止火滅。正統十三年旌表。

石鼐,渾源諸生。父歿,廬墓。墓初成,天大雨,山水驟漲。鼐仰天號哭,水將及墓,忽分兩道去,墓獲全。弘治五年旌表。

任鏜,夏邑人。嫡母卒,廬於墓。黃河沖溢,將嚙塋域。鏜伏地號哭,河即南徙。嘉靖二十五年旌表。

史五常,內黃人。父萱,官廣東僉事。卒,葬南海和光寺側。五常方七歲,母攜以歸。比長,奉母至孝,常恨父不得歸葬。母語之曰:「爾父杉木櫬內,置大錢十,爾謹志之。」母歿,廬墓致毀,既終喪,往迎父櫬。時相去已五十年,寺沒於水久矣。五常泣禱,有老人以杖指示寺址。發地,果得父櫬,內置錢如母言,乃扶歸,與母合葬,復廬墓側。正統六年旌表。

周敖,河州衛軍家子也。正統末,聞英宗北狩,大哭,不食七日而死。其子諸生路方讀書別墅,聞父死,慟哭奔歸,以頭觸庭槐亦死。鄉人異之,聞於州。知州躬臨其喪,賻麥四十斛、白金一斤。路妻方氏,厲志守節,撫子堂成立,後為知縣。

鄭韺,石康人。父賜,舉人,兄頀,進士。天順中,母為瑤賊所掠。韺年十六,挺身入賊壘,紿之曰:「吾欲丐吾母,豈惜金,第金皆母所瘞,願代母歸取之。」賊遂拘韺而釋母,然其家實無金也,韺遂被殺。廉州知府張岳建祠祀之。

榮瑄,瓊州人。三歲而孤,與兄琇並以孝聞。天順四年,土賊據瓊城,瑄兄弟扶母走避。遇賊,琇謂瑄曰:「我以死衛母,汝急去。」瑄從之,琇與母遂陷賊中。官軍至,琇被執。主將將殺琇,瑄趨至,叩頭流血,泣請曰:「兄以母故陷賊,母老家貧,恃兄為命,願殺瑄存兄養母。」主將不察,竟殺瑄。

後有葉文榮,海寧人。弟殺人論死,母日悲泣不食。文榮謂母曰:「兒年已長,有子,請代弟死。」遂詣官服殺人罪,弟得釋,而文榮坐死。

傅檝,字定濟,泉州南安人。祖凱,父浚,並進士。為部郎。檝年十六舉鄉試,二十成進士。弘治中,授行人,出行襄府。半道聞母病,請入京省視再往竣事。禮部尚書劉春曰:「無害於若,而可教孝。」奏許之。浚後遷山東鹽運司同知。娶繼妻,私其二奴。浚聞將治之,遂暴卒。檝心疑未發,奴遽亡去。久之,偵一奴逃德化縣,傭巨姓家。檝微行往伺奴出,袖鐵椎擊殺之,而其一不可跡矣。檝不欲見繼母,葬父畢,號慟曰:「父仇尚在,何以為人!」乃裂衣冠,屏妻子,出宿郊墟間,蓬首垢面,饑寒風雨,不知就避。親戚故人率目之為狂,檝終不自明也。子燾卒,不哭。或詰之,則垂涕曰:「我不能為子,敢為父乎!」繼母卒,乃歸。蓋自廢自罰者三十五年,又十五年而卒。

楊成章,道州人。父泰,為浙江長亭巡檢。妻何氏無出,納丁氏女為妾,生成章。甫四歲,泰卒。何將扶櫬歸,丁氏父予之子,而奪其母。母乃剪銀錢與何別,約各藏其半,俟成章長授之。越六年,何臨歿,授成章半錢,告之故。成章嗚咽受命。既冠,娶婦月餘,即執半錢之浙中尋母。母先已適東陽郭氏,生子曰氏,而成章不知也。遍訪之,無所遇而還。弘治十一年,東陽典史李紹裔以事宿氏家。氏母知為道州人,遣氏問成章存否,知成章已為諸生,乃令氏執半錢覓其兄。會有會稽人官訓導者,嘗設教東陽,為氏師,與成章述氏母憶子狀。成章亦往尋母,遇氏於江西舟次。兄弟悲且喜,各出半錢合之,益信,遂俱至東陽,母子始相聚。自是成章三往迎母不遂,棄月廩,赴東陽侍養。及母卒,廬墓三載始返。至嘉靖十年,成章以歲貢入都,氏亦以事至,乃述成章尋親事,上之吏部,請進一官。部臣言:「成章孝行,兩地已勘實,登之朝覲憲綱,氏言非謬。昔硃壽昌棄官尋母,宋神宗詔令就官。今所司知而不能薦,臣等又拘例而不請旌,真有愧於古誼。請量授成章國子學錄,賜氏花紅羊酒。」制曰:「可。」

謝用,字希中,祁門人。父永貞。生母馬氏方妊,永貞客外,嫡母汪氏妒而嫁之,遂生用。永貞還,大恨,抱用歸,寄乳鄰媼。汪氏收而自鞠之,逾年亦生子,均愛無厚薄。用既冠,始知所生。密訪之,則又改適,不知其所矣。用遍覓幾一載。一夕宿休寧農家,有寡嫗出問曰:「若為誰?」用告以姓名,及尋母之故。曰:「若母為誰?」曰:「馬氏。」曰:「若非永貞之子乎?」曰:「然。」媼遂抱用曰:「我即汝母也。」於是母子相持而哭,時弘治十五年四月也。用歸告父,并其同母弟迎歸,居別室。孝養二母,曲盡其誠。後汪感悔,令迎馬同居,訖無間言。永貞卒,用居喪以孝聞。鄰人失火,延數十家,將至用舍,風反火息。用時為諸生,督學御史廉其孝,列之德行優等,月廩之。

何競,字邦植,蕭山人。父舜賓,為御史,謫戍廣西慶遠衛,遇赦還。好持吏短長。有鄒魯者,當塗人。亦以御史謫官,稍遷蕭山知縣,貪暴狡悍。舜賓求魯陰事訐之,兩人互相猜。縣中湘湖為富人私據,舜賓發其事於官,奏核之。富人因奏舜賓以戍卒潛逃,擅自冠帶。章並下所司核治。魯隱其文牒,詭言舜賓遇赦無驗,宜行原衛查核。上官不可,駁之。會舜賓門人訓導童顯章為魯所陷論死,下府覆驗,道經舜賓家,入與謀。魯聞之,大詬曰:「舜賓乃敢竄重囚。」發卒圍其門,輒捕舜賓,徑解慶遠。又令爪牙吏屏其衣服。至餘干,宿昌國寺,夜以濕衣閉其口,壓殺之。魯復捕舜賓妻子。競與母逃常熟,匿父友王鼎家。—已而魯遷山西僉事,將行。競乃潛歸與族人謀,召親黨數十人飲之酒,為舜賓稱冤。中坐,競出叩首哭以請,皆踴躍願效命。乃各持器伏道旁,伺魯過,競袖鐵捶奮擊,騶從駭散。僕其輿,裸之,杖齊下,矐兩目,須發盡拔。競拔佩刀砍其左股,必欲殺之,為眾所止。乃與魯連鎖赴按察司,而預令族父澤走闕下訴冤。僉事蕭翀故黨魯,嚴刑訊競。競大言曰:「必欲殺我,我非畏死者。顧人孰無父母,且我已訟於朝,非公輩所得擅殺。」噬臂肉擲案上,含血噀翀面,一堂皆驚。

會競疏已上,遣刑部郎中李時、給事中李舉,會巡按御史鄧璋雜治。諸人持兩端,擬魯故屏人衣食至死,競部民毆本屬知縣篤疾,律俱絞,餘所逮數百人,擬罪有差。競母朱氏復撾登聞鼓訴冤,魯亦使人馳訴,乃命大理寺正曹廉會巡按御史陳銓覆治。廉曰:「爾等何毆縣官?」競曰:「競知父仇,不知縣官,但恨未殺之耳。」廉以致死無據,遣縣令揭棺驗之。驗者報傷,而解役任寬慷慨首實,且出舜賓臨命所付血書。於是眾皆辭伏,改擬魯斬,競徒三年。法司議競遣戍,且曰:「魯已成篤疾,競為父報仇,律意有在,均俟上裁。」帝從其議,戍競福寧衛,時弘治十四年二月也。後武宗登極肆赦,魯免死,競赦歸,又九年卒。競自父歿至死,凡十六年,服衰終其身。

王原,文安人。正德中,父珣以家貧役重逃去。原稍長,問父所在。母告以故,原大悲慟。乃設肆於邑治之衢,治酒食舍諸行旅。遇遠方客至,則告以父姓名、年貌,冀得父蹤跡。久之無所得。既娶婦月餘,跪告母曰:「兒將尋父。」母泣曰:「汝父去二十餘載,存亡不可知。且若父氓耳,流落何所,誰知名者?無為父子相繼作羈鬼,使我無依。」原痛哭曰:「幸有婦陪母,母無以兒為念,兒不得父不歸也。」號泣辭母去,遍歷山東南北,去來者數年。

一日,渡海至田橫島,假寐神祠中,夢至一寺,當午,炊莎和肉羹食之。一老父至,驚覺。原告之夢,請占之。老父曰:「若何為者?」曰:「尋父。」老父曰:「午者,正南位也。莎根附子,肉和之,附子膾也。求諸南方,父子其會乎?」原喜,謝去,而南𧾷俞洺、漳,至輝縣帶山,有寺曰夢覺,原心動。天雨雪,寒甚,臥寺門外。及曙,一僧啟門出,駭曰:「汝何人?」曰:「文安人,尋父而來。」曰:「識之乎?」曰:「不識也。」引入禪堂,憐而予之粥。珣方執爨灶下,僧素知為文安人,謂之曰:「若同里有少年來尋父者,若倘識其人。」珣出見原,皆不相識,問其父姓名,則王珣也。珣亦呼原乳名。相抱持慟哭,寺僧莫不感動。珣曰:「歸告汝母,我無顏復歸故鄉矣。」原曰:「父不歸,兒有死耳。」牽衣哭不止。寺僧力勸之,父子相持歸,夫妻子母復聚。後原子孫多仕宦者。

黃璽,字廷璽,餘姚人。兄伯震,商十年不歸。璽出求之,經行萬里,不得蹤跡。最後至衡州,禱南岳廟,夢神人授以「纏綿盜賊際,狼狽江漢行」二句。一書生告之曰:「此杜甫《舂陵行》詩也,舂陵今道州,曷往尋之。」璽從其言,既至,無所過。一日入廁,置傘道旁。伯震適過之曰:此吾鄉之傘也。」循其柄而觀,見有「餘姚黃廷璽記」六字。方疑駭,璽出問訊,則其兄也,遂奉以歸。

歸鉞,字汝威,嘉定縣人。早喪母。父娶繼妻,有子,鉞遂失愛。父偶撻鉞,繼母輒索大杖與之,曰:「毋傷乃翁力也。」家貧,食不足,每炊將熟,即諓諓數鉞過,父怒而逐之,其母子得飽食。鉞饑困,匍匐道中。比歸,父母相與言曰:「有子不居家,在外作賊耳。」輒復杖之,屢瀕於死。及父卒,母益擯不納,因販鹽市中,時私其弟,問母飲食,致甘鮮焉。正德三年,大饑,母不能自活。鉞涕泣奉迎,母內自慚不欲往,然以無所資,迄從之。鉞得食,先母弟,而己有饑色。弟尋卒,鉞養母終其身,嘉靖中卒。族子繡,亦販鹽,與二弟紋、緯友愛。緯數犯法,繡輒罄貲護之,終無慍色。繡妻朱,制衣必三襲,曰:「二叔無室,豈可使郎君獨暖耶?」里人稱為歸氏二孝子。

何麟,沁水人,為布政司吏。武宗微行,由大同抵太原,城門閉,不得入。怒而還京,遣中官逮守臣不啟門者,巡撫以下皆大懼。麟曰:「朝廷未知主名。請厚賄中官,麟與俱往。即聖怒不測,麟一身獨當之。」及抵京,上疏曰:「陛下巡幸晉陽,司城門者實臣麟一人,他官無預也。臣不能啟門迎駕,罪當萬死。但陛下輕宗廟社稷而事巡游,且易服微行,無清道警蹕之詔,白龍魚服,臣下何由辨焉。昔漢光武夜獵,至上東門,守臣郅惲拒弗納,光武以惲能守法而賞之。今小臣欲守郅惲之節,而陛下乃有不敬之誅。臣恐天下後世以為臣之不幸不若郅惲,陛下寬仁之量亦遠遜光武也。」疏入,帝怒稍解,廷杖六十,釋還,餘不問。巡撫以下郊迎,禮敬之。

孫清,睢陽諸生也。幼孤,事母孝。母歿未葬,流賊入其境,居民盡逃,清獨守柩不去。賊兩經其門,皆不入,里人多賴以全。正德九年四月,河南巡按御史江良貴奏聞,并言:「清同邑徐儀女雪梅、嚴清女銳兒皆不受賊污,憤罵見殺。沭陽諸生沈麟以知府劉祥、縣丞程儉為賊所執,挺身詣賊,開陳利害,願以身代。賊義之,二人獲釋。凡此義烈,有關風化,宜如制旌表。」章下禮官。先是,八年二月,山東巡按御史張璿奏,賊所過州縣,有子救父,婦衛夫,罹賊兵刃者,凡百十九人,皆宜旌表。時傅珪代費宏為禮部,言:「所奏人多費廣。宜準山西近例,於所在旌善亭側,建二石碑,分書男婦姓名、邑里及其孝義、貞烈大略,以示旌揚,有司量給殯殮費。厥後地方有奏,悉以此令從事。」帝可之。至是,良貴奏下,劉春代珪為禮部,竟不請旌,但用珪前議,并給銀建坊之令亦不復行,而旌善之意微矣。

當是時,濮州諸生宋顯章、淅川諸生李豫,皆以孝行著聞,流賊過其門不敢犯,里人亦多賴以全。而顯章之死也,其妻辛氏自縊以殉。知州李緝為建孝節坊,並祠祀。嘉靖七年,豫獨被旌。

劉憲,靈石諸生也。父先亡。母年七十餘,兩目俱瞽,憲奉事惟謹。正德六年,流賊入城,憲負母避之城外。賊追至,欲殺母,憲哀告曰:「寧殺我,毋害我母。」賊乃釋之,行至嶺後,憲竟為他賊所殺。賊縱火焚民居,獨憲宅隨爇隨滅。同時羅璋,遂寧諸生。大盜亂蜀中,母為賊所獲,璋手挺長槍,連斃三賊,賊舍母去。後賊追至,璋力捍賊,久之力疲,竟被執。賊憤甚,剜心剖肝,裂其屍。並正德中旌表。有李壯丁者,安定縣人。嘉靖中,北寇入犯,從父母奔避山谷。遇賊縛母去,壯丁取石奮擊,母得脫。前行復遇五賊,一賊縛其母,母大呼曰:「兒速去,毋顧我!」壯丁憤,手提鐵器擊仆賊,母得逃,而壯丁竟為賊所殺。正德中,賊掠巨鹿,執趙智、趙慧之母,將殺之。智追至,跪告曰:「母年老,願殺我。」慧亦至,泣曰:「兄年長,願留養母而殺我。」智方與爭死,而母復請曰:「吾老當死,乞留二子。」群賊笑曰:「皆好人也。」並釋之。

容師偃,香山人。父患癱疾,扶持不離側。正德十二年,寇掠其鄉,師偃負父而逃。追者急,父麾使遁,泣曰:「父子相為命,去將安之。」俄被執,賊灼其父,師偃號泣請代。賊從之,父得釋,而師偃焚死。後有劉靜者,萬安諸生。嘉靖間,流賊陷其縣,負母出奔。遇賊,將殺母,靜以身翼蔽求代死。賊怒,攢刃殺之,猶抱母不解,尸閱七日不變。萬歷元年旌表。又有溫鉞者,大同人。父景清有膽力。嘉靖三年,鎮兵叛,殺巡撫張文錦。其後,巡撫蔡天佑令景清密捕首惡,戮數人,其黨恨之。十二年復叛,殺總兵李瑾,因遍索昔年為軍府效命者。景清深匿不出,遂執鉞及其母王氏以去,令言景清所在。鉞曰:「爾欲殺我父,而使我言其處,是我殺父也。如仇不可解,則殺我舒憤足矣。」賊不聽,逼母使言,母大罵不輟。賊怒,支解以怵鉞。鉞大哭且罵,並被殺。事平,母子並獲旌。

俞孜,字景修,浙江山陰人。為諸生,敦行誼。嘉靖初,父華充里役,解流人徐鐸至口外。鐸毒殺華,亡走。孜扶櫬歸,誓必報仇,縱跡數十郡不可得。後聞已還鄉,匿其甥楊氏家。乃結力士十數人,佯為賣魚,往來偵伺,且謁知府南大吉乞助。大吉義之,遣數健卒與俱,夜半驟率卒入楊氏家,呼鐸出見,縛送於官,置諸法。孜自是不復應舉,養繼母以終。

有張震者,餘姚農家子也。生周歲,父為人所陷將死,齧震指語曰:「某,吾仇也,汝勿忘。」震長而指瘡不愈,母告以故,震誓必報。其友謂曰:「汝力弱,吾為汝殺之。」未幾,仇乘馬出,友以田器擊之,即死。震喜,走告父墓。已而事發,有司傷其志,減死論戍,遇赦歸。孫文,亦餘姚人也。幼時,父為族人時行箠死。長欲報之,而力不敵,乃偽與和好,共武斷鄉曲。時行坦然不復疑。一日,值時行於田間,即以田器擊殺之。坐戍,未幾,遇赦獲釋。

崔鑒,京師人。父嗜酒狎娼,召與居。娼恃寵,時時陵鑒母,父又被酒,數侵辱之。一日,娼惡言詈母,母復之,娼遂擊敗母面。母不勝憤,入室伏床而泣,將自盡。鑒時年十三,自學舍歸,問之,母告以故。鑑曰:「母無死。」即走至學舍,挾刃還。娼適掃地,且掃且詈。鑑拔刃刺其左脅,立斃,乃匿刃牖下,亡走數里,忽自念曰:「父不知我殺娼,必累我母。」急趨歸,父果訴於官,將縶其母矣。鑒至,告捕者曰:「此我所為,非母也。」眾見其幼,不信。鑑曰:「汝等不信,請問兇器安在?」自出刃示之,眾乃釋母,縶鑒置獄。事聞,下刑部讞。尚書聞淵等議,鑒志在救母,且年少可矜,難拘常律。帝亦貸其罪。

唐儼,全州諸生也。父蔭,郴州知州,歸老得危疾。儼年十二,潛割臂肉進之,疾良已。及父歿,哀毀如成人。其後游學於外,嫡母寢疾。儼妻鄧氏年十八,奮曰:「吾婦人,安知湯藥。昔夫子以臂肉療吾舅,吾獨不能療吾姑哉?」於是割脅肉以進,姑疾亦愈。儼聞母疾,馳歸,則無恙久矣,拜其妻曰:「此吾分也,當急召我,何自苦如此!」妻曰:「子事父,婦事姑,一也。方危急時,召子何及。且事必待子,安用婦為。」儼益歎異。嫡母歿二十年,而生母歿,儼廬墓三年。嘉靖四年貢至京,有司奏旌其門。

丘緒,字繼先,鄞縣諸生他。生母黃,為嫡餘所逐,適江東包氏。未幾轉適他所,遂不復相聞。緒年十五,父歿,事餘至孝。余疾,謹奉湯藥,不解衣帶者數月。餘重感其孝,病革,與訣曰:「我即死,汝無忘若母。」時母被逐已二十年矣。一夕,夢人告曰:「若母在臺州金鰲寺前。」覺而識之。次日,與一人憩於途,詰之,則包氏故養馬廝也。叩以母所向,曰:「有周平者曾悉其事,今已戍京衛矣。」緒姊婿謁選在京,遺書囑訪平,久之未得。一日,有避雨於邸門者,其聲類鄞人,叩之,即周平也,言黃已適臺州李副使子。緒得報,即之臺,而李已歿,其嗣子漫不知前事。緒徬徨掩泣於道,有傷之者,導謁老媒妁王四,曰已再適仙居吳義官。吳,仙居巨族也。緒至,歷瞷數十家,無所遇。已而抵一儒生吳秉朗家,語之故。生感其意,留止焉。有叔母聞所留者異鄉人也,恚而咻之。生告以緒意。叔母者,黃故主母也,頗憶前事,然不詳所往。呼舊蒼頭問之,云金鰲寺前,去歲經之,棺已殯寺旁矣。緒以其言與夢合,信之,行且泣,牛觸之墜於溝,則輿夫馬長之門也。駭而出,問所從來。緒以情告。長曰:「吾前輿一婦至縉雲蒼嶺下,殆是也。」輿緒至其處。緒遍物色,無所遇,倀倀行委巷中。一媼立門外,探之,知為鄞人,告以所從來。嫗亦轉詢丘氏耗,則緒母也。抱持而哭,閭里皆感動。寺旁棺者,蓋其姒氏云。所適陳翁,貧而無子,且多負。緒還取金償之,並迎翁以歸,備極孝養。嘉靖十四年,知縣趙民順入覲,疏聞於朝,獲旌表。

張鈞,石州人。父赦,國子生。以二親早亡,矢志不仕,隱居城北村。鈞,正德末舉於鄉。以親老亦不仕,讀書養親,遠近皆稱其孝。嘉靖二十年,俺答犯石州。鈞慮父遭難,自城中馳一騎號泣赴救。寇射中其肩,裹瘡疾馳,至則父已被殺。鈞隕絕,盡餂父血,水漿不入口三日,不勝悲痛而卒。越二年,有司上其狀,獲旌。是時殺掠甚慘,石州為親死者十一人,而張承相、于博、張永安尤著。承相少孤,及長為諸生,養母二十餘年,以孝聞。寇至,負母出逃,為所得,叩頭號泣,乞免其母。寇怒,並殺之,抱母首死。博二歲而孤,奉母盡孝。寇抵城下,博方讀書城中。母居村舍,亟下城號泣求母。母已被執,遇諸途,博取石奮擊寇。寇就剖其心,母得逸去,年止十有八。永安,石州吏也。父為寇所逐,永安持梃追擊之,傷二賊,趣父逸去,而身自後衛之,被數十創死。與鈞同被旌。有溫繼宗者,沁州諸生。父卒,不能葬,日守柩哀泣。嘉靖二十一年,寇入犯,或勸出城避難,以父殯不肯去。寇至,與叔父淵等力禦,擊傷一賊,中矢死柩旁,淵等皆死。亦與鈞同被旌。

王在復,太倉人。年二十一,從父讀書城外。倭寇入犯,父子亟奔入城。父體肥不能速行,中道遇賊,遂相失。在復走二里許,展轉尋父。聞父被執,急趨賊所,叩頭求免。賊不聽,拔刃擬其父,在復以身蔽之,痛哭哀求。賊怒,并殺之,兩首墜地,而手猶抱父不釋。時嘉靖三十三年五月也。當是時,倭亂東南,孝子以衛父母見殺者甚眾,其得旌於朝者,在復及黃巖王蒐、慈谿向敘、無錫蔡元銳、丹徒殷士望。蒐隨父顯避賊。顯被執,將殺之。蒐亟趨前請代,賊遂殺蒐而釋顯。敘為慈谿諸生。倭入寇,以縣無城,掖母出避。遇賊,踣敘而斫其母,敘急起抱母頸,大呼曰:「寧殺我,毋殺我母!」賊如其言,母獲全。俱嘉靖三十五年旌表。元銳,無錫人,與弟元鐸並孝友。倭犯無錫,入元銳家,兄弟急扶父升屋避匿。而元銳為賊執,令言父所在,堅不從,遂見殺。元鐸不知兄死,明日持重貲往贖,并見殺。嘉靖三十八年旌表。士望,丹徒人,事親孝。倭犯京口,父被掠,士望請代死。賊笑而試之,火炙刀刺,受之怡然,賊兩釋之。嘉靖四十三年旌表。其他未及旌表者,又有陳經孚、龔可正、伍民憲。經孚,平陽人。倭至,負母出逃,遇賊索母珥環,欲殺之。經孚以身翼蔽,賊怒,揮刃截耳及肩而死,手猶抱母頸不解。可正,嘉定諸生。負祖母避賊,天雨泥濘,猝遇賊。賊惡見婦人,欲殺其祖母,叱可正去。可正跪泣請代,賊不從。可正以身覆祖母,賊並殺之。民憲,晉江人。扶父避難,遇賊,長跪哀告曰:「勿驚我父,他物任取之。」賊不聽,竟殺其父。民憲憤,挺身殺二賊,傷數賊。賊至益多,斷民憲右手。臥草中,猶一手執戈,呼其父三日而絕。

夏子孝,字以忠,桐城人。六歲失母,哀哭如成人。九歲父得危疾,禱天地,刲股六寸許,調羹以進,父食之頓愈。翌日,子孝痛創,父詰其故,始知之。里老以聞於官,知府胡麟先夢王祥來謁,詰旦而縣牒至,詫曰:「孺子其祥後身耶?」召見,易其舊名「恩」曰「子孝」。督學御史胡植即令入學為諸生,月廩之。麟復屬貢士趙簡授之經。嘉靖末,父卒,廬墓,獨居荒山,身無完衣,形容槁瘁。後歷事王畿、羅汝芳、史桂芳、耿定向,獲聞聖賢之學。定向為督學御史,將疏聞於朝,固辭曰:「不肖不忍以亡親賈名。」乃止。將死,命其子曰:「葬我父墓側。」

阿寄者,淳安徐氏僕也。徐氏昆弟析產而居,伯得一馬,仲得一牛,季寡婦得阿寄,時年五十餘矣。寡婦泣曰:「馬則乘,牛則耕,老僕何益。」寄嘆曰:「主謂我不若牛馬耶!」乃畫策營生,示可用狀。寡婦盡脫簪珥,得白金十二兩,畀寄。寄入山販漆,期年而三倍其息,謂寡婦曰:「主無憂,富可致矣。」歷二十年,積資巨萬,為寡婦嫁三女,婚二子,齎聘皆千金。又延師教二子,輸粟為太學生。自是,寡婦財雄一邑。及寄病且死,謂寡婦曰:「老奴牛馬之報盡矣。」出枕中二籍,則家鉅細悉均分之,曰:「以此遺兩郎君,可世守也。」既歿,或疑其有私,竊啟其篋,無一金蓄。所遺一嫗一兒,僅敝縕掩體而已。

趙重華,雲南太和人。七歲時,父廷瑞游江湖間,久不返。重華長,謁郡守請路引,榜其背曰:「萬里尋親。」別書父年貌、邑里數千紙,所歷都會州縣遍張之。西禱武當山,經太子巖,巖陰有字曰:「嘉靖四十四年十二月十二日,趙廷瑞朝山至此。」重華讀之,慟曰:「吾父果過此,今吾之來月日正同,可卜相逢矣。」遂書其後曰:「萬歷六年十二月十二日,趙廷瑞之子重華,尋父至此。」久之竟無所遇。過丹陽,盜攫其資,所遺獨路引。且行且乞,遇一老僧呼問其故,笑曰:「汝父客無錫南禪寺中。」語訖忽不見。重華急趨至寺,果其父,出路引示之,相與慟哭。留數日,乃還雲南。

是時,有謝廣者,祁門人。父求仙不返,廣娶婦七日即別母求父,遇於開封逆旅中。父乘間復脫去。廣跋涉四方者垂二十年,終不得父,聞者哀之。

王世名,字時望,武義人。父良,與族子俊同居爭屋,為俊毆死。世名年十七,恐殘父屍,不忍就理,乃佯聽其輸田議和。凡田所入,輒易價封識。俊有所饋,亦佯受之。而潛繪父像懸密室,繪己像於旁,帶刀侍,朝夕泣拜,且購一刃,銘「報仇」二字,母妻不知也。服闋,為諸生。及生子數月,謂母妻曰:「吾已有後,可以死矣。」一日,俊自外醉歸,世名挺刃迎擊之,立斃。出號於眾,入白母,即取前封識者詣吏請死。時萬歷九年二月,去父死六年矣。知縣陳某曰:「此孝子也,不可置獄。」別館之,而上其事於府。府檄金華知縣汪大受來訊。世名請死,大受曰:「檢屍有傷,爾可無死。」曰:「吾惟不忍殘父尸,以至今日。不然,何待六年。乞放歸辭母乃就死。」許之。歸,母迎而泣。世名曰:「身者,父之遺也。以父之遺為父死,雖離母,得從父矣,何憾。」頃之,大受至,縣人奔走直世名者以千計。大受乃令人舁致父棺,將開視之。世名大慟,以頭觸階石,血流殷地。大受及旁觀者咸為隕涕,乃令舁柩去,將白上官免檢屍,以全孝子。世名曰:「此非法也,非法無君,何以生為。」遂不食而死。妻俞氏,撫孤三載,自縊以殉,旌其門曰孝烈。

李文詠,崑山諸生。父大經,沂水知縣。萬曆二十七年,父寢室被火。文詠突入,將父抱出,而榱棟盡覆,父子俱焚死。火息,入視,尸猶覆其父,父存全體,文詠但餘一股。

王應元,武隆人。力農養父。父醉臥,家失火。應元自外趨烈焰中,竟不能出,抱父死。

唐治,黃岡人。父柩在堂,鄰居火,治盡出資財募人舁柩,人各自顧,無應者。或挽之出,泣曰:「父柩在此,我死不出。」火息,後堂巋然獨存,柩亦無恙,而治竟熏灼伏柩死。萬歷中旌表。

許恩,蘄水人。夜半鄰家失火,恩驚出,遍尋母不得,復突入,遂與母俱焚。

馮象臨,慈谿諸生。家被火,遍覓父母,煙焰彌空,迷失庭戶。象臨大呼,初得母,即從火中負出。再入負父,並挾一弟以出,半體已焦爛。聞妹尚留臥內,母號呼,將自入,亟止之,觸烈焰攜妹出,竟灼爛而死。事聞,賜旌。

後有龔作梅者,陳州人。年十七,父母俱亡,殯於舍。闖賊火民居,作梅跪柩前焚死。

孔金,山陽人。父早亡,母謝氏,遺腹三月而生金。母為大賈杜言逼娶,投河死。金長,屢訟於官,不勝。言行賄欲斃金,金乃乞食走闕下,擊登聞鼓訴冤,不得達。還墓所,晝夜號泣。里人劉清等陳其事於府,知府張守約異之,召閭族媒氏質實,坐言大辟。未幾守約卒,言夤緣免。金復號訴不已,被箠無完膚。已而撫按理舊牘,仍坐言大闢,迄死獄中。金子良亦有孝行,父病,刲股為羹以進,旋愈。比卒,廬墓哀毀。萬歷四十三年,父子並得旌。

楊通照、通傑,銅仁人。母周氏有疾,兄弟爭拜禱,求以身代。閱三年,不入內室。萬歷三十六年,群苗流劫,至其家,母被執去。二人追斗數十里,被傷不顧。至鬼空溪,見賊縶母,大罵,聲震山谷,橫擊萬眾中,為賊所磔死。通照年二十五,通傑年二十二。泰昌元年,巡撫李枟、巡按史永安上其事,旌曰雙孝之門。

時無錫民浦邵,賊縛其父虞,將殺之。邵以首迎刃而死,父得免。寧化民林上元,賊掠其繼母李氏出城,上元從城上持槍一躍而下,直奔賊壘,刺死二人。賊避其鋒,立出李氏,因引去,城賴以全。皆萬歷四十三年旌。

崇禎七年,流賊陷竹谿,執知縣餘霄將殺之。子諸生伯麟請代,乃免。

張清雅,潛山人。家貧,力學養親。崇禎十年,張獻忠來犯。清雅以父年老臥病,守之不去。無何,父卒。斂甫畢,賊入其家,疑棺內藏金銀,欲剖視之。清雅據棺哀泣,賊斷其手,僕地。幼子超藝年十六,號哭求代。賊復砍之,父子俱死,而棺得不剖。僕雲滿,具兩棺斂之,亦不食死。

時有白精忠者,潁州人。五歲而孤,母袁氏撫之。家貧,母食糠核,而以精者哺兒。精忠知之,每餐必先啖其惡者。天啟中,舉於鄉。崇禎八年,流賊陷潁州,家人勸逃匿。精忠以母年老,不忍獨去,遂遇害。

州有檀之槐者,護母柩下去。與賊格鬥,殺數人,被磔死。

又有李心唯,素敦孝行。賊至,泣守母喪。賊掠其室,將縛之,不出,被殺。子果,見父死,厲聲罵賊,賊又殺之。

有餘承德者,無為人。崇禎十五年,流賊突至,掖其祖母劉氏、母魏氏及妻楊氏、妹玉女出避。祖母、母行遲,為盜所獲,欲刃之。承德號呼救護,並遇害。楊氏見之,急投河死。賊將犯玉女,玉女大罵,堅不從,寸磔而死。


\end{pinyinscope}