\article{英宗后紀}

\begin{pinyinscope}
天順元年春正月壬午,昧爽,武清侯石亨,都督張輗、張軏,左都御史楊善,副都御史徐有貞,太監曹吉祥以兵迎帝於南宮,御奉天門,朝百官。徐有貞以原官兼翰林學士,入閣預機務。日中,御奉天殿即位。下兵部尚書于謙、大學士王文錦衣衛獄。太常寺卿許彬、大理寺卿薛瑄為禮部侍郎兼翰林學士,入閣預機務。丙戌,詔赦天下,改景泰八年為天順元年。論奪門迎復功,封石亨忠國公,張軏太平侯,張輗文安伯,楊善興濟伯,曹吉祥嗣子欽都督同知。丁亥,殺於謙、王文,籍其家。陳循、江淵、俞士悅謫戍,蕭鎡、商輅除名。己丑,復論奪門功,封孫鏜懷寧伯,董興海寧伯,欽天監正湯序禮部右侍郎,官舍旂軍晉級者凡三千餘人。辛卯,罷巡撫提督官。壬辰,榜于謙黨人示天下。甲午,殺昌平侯楊俊。二月乙未朔,廢景泰帝為郕王。庚子,高穀致仕。湯序請除景泰年號,不許。癸卯,吏部侍郎李賢兼翰林學士,入閣預機務。殺都督范廣。戊申,柳溥破廣西蠻。癸丑,郕王薨。戊午,方瑛、石璞大破湖廣苗。召璞還。壬戌,免南畿被災秋糧。三月己巳,復立長子見深為皇太子,封皇子見潾為德王,見澍秀王,見澤崇王,見浚吉王。癸酉,封徐有貞武功伯。乙亥,大賚文武軍民。庚辰,賜黎淳等進士及第、出身有差。石亨為征虜副將軍,剿寇延綏。丁亥,振山東饑。

夏四月甲午朔,以災異數見求直言。乙未,免浙江被災稅糧。丁酉,方瑛攻銅彭藕洞苗,悉平之。丁未,錄囚。癸丑,罷團營。乙卯,孛來寇寧夏,參將種興戰死。五月辛未,安遠侯柳溥備宣、大邊。是月,以石亨言下御史楊瑄、張鵬獄。六月甲午,下右都御史耿九疇、副都御史羅綺錦衣衛獄。己亥,下徐有貞、錦衣衛獄。是日,大風雨雹,壞奉天門鴟吻,敕修省。庚子,徐有貞、李賢、羅綺、耿九疇謫外任,楊瑄、張鵬戍邊。通政司參議兼侍講呂原入閣預機務。壬寅,薛瑄致仕。癸卯,修撰岳正人閣預機務。甲辰,復李賢為吏部侍郎。乙巳,巡撫貴州副都御史蔣琳坐于謙黨棄市。

秋七月乙丑,復下徐有貞於獄。丙寅,承天門災。丁卯,躬禱於南郊。戊辰,敕修省。庚午,李賢復入閣。改許彬南京禮部侍郎。辛未,出岳正為欽州同知,尋下獄,謫戍。癸酉,大赦。癸未,放徐有貞於金齒。辛卯,大賚諸邊軍士。八月甲午,以彗星屢見,躬禱於上帝。九月甲子,太常少卿彭時兼翰林學士,入閣預機務。

冬十月丁酉,賜王振祭葬,立祠曰:「旌忠」。壬寅,徵江西處士吳與弼。丙辰,釋建文帝幼子文圭及其家屬,安置鳳陽。十一月甲戌,廣西總兵官朱瑛討田州叛蠻。己丑,免山東被災夏稅。十二月壬辰,封曹欽昭武伯。辛丑,安遠侯柳溥充總兵官,禦孛來於甘、涼。

是年,琉球中山、安南、暹羅、占城、哈密、烏斯藏入貢。

二年春正月辛酉,兵部尚書陳汝言有罪下獄。乙丑,享太廟。甲戌,太祀天地於南郊。己卯,上皇太后尊號。二月戊申,開雲南、福建、浙江銀場。中官市雲南珍寶。閏月己卯,瘞土木暴骸。

夏四月,復設巡撫官。五月壬寅,授處士吳與弼左諭德,辭不拜,尋送還鄉。

秋七月癸卯,定遠伯石彪為平夷將軍,充總兵官,禦寇寧夏。八月戊辰,孛來寇鎮番。

冬十月甲子,獵南海子。壬午,武平伯陳友為征夷將軍,充總兵官,剿寇寧夏。十一月甲寅,免山東秋糧。

是年,安南、烏斯藏、占城、哈密入貢。

三年春正月乙未,大祀天地於南郊。甲辰,定遠伯石彪、彰武伯楊信敗孛來於安邊營,都督僉事周賢、都指揮李鑒戰死。進彪為侯。二月丁卯,遣御史及中官採珠廣東。

夏四月壬子,巡撫兩廣僉都御史葉盛破瀧水瑤。己巳,南和侯方瑛克貴州茵。六月辛酉,復命巡撫官以八月集京師議事。

秋八月庚戌,石彪有罪,下錦衣衛獄。己未,禁文武大臣、給事中、御史、錦衣衛官往來交通,違者依鐵榜例論罪。乙亥,免湖廣被災秋糧。

冬十月己未,幸南海子。庚午,石亨以罪罷。諸奪門冒功者許自首改正。是月,命法司會廷臣,每歲霜降錄囚,後以為常。十一月癸巳,振湖廣饑。

是年哈密、琉球中山、錫蘭山、滿剌加入貢。

四年春正月丁亥,太祀天地於南郊。癸卯,石亨有罪下獄,尋死。二月壬子,僮陷梧州。丁卯,石彪棄市。三月庚辰,賜王一夔等進士及第、出身有差。戊戌,免南畿被災秋糧。

夏四月己酉,分遣內臣督浙江、雲南、福建、四川銀課。壬子,襄王瞻墡來朝。五月壬午,免畿內、浙江被災秋糧。己亥,罷中官督蘇、杭織造。六月癸亥,免湖廣被災稅糧。

秋七月乙亥朔,日有食之。辛卯,自五月雨至是月,淮水決,沒軍民田廬,遣使振恤。八月甲子,孛來三道入寇,大同總兵官李文、宣府總兵官楊能禦之。癸酉,孛來入雁門,掠忻、代、朔諸州。九月庚辰,孛來圍大同右衛。庚寅,撫寧伯朱永,都督白玉、鮑政備宣府邊。甲午,免江西被災秋糧。

冬十月甲子,閱京營將領騎射於西苑。戊幸南海子。十一月丁酉,閱隨操武臣騎射於西苑。閏月己未,幸鄭村壩,閱甲仗軍馬。

是年,琉球中山、安南、占城、爪哇、哈密、烏斯藏入貢。

五年春正月庚戌,大祀天地於南郊。二月己卯,免山東被災稅糧。丙申,都督僉事顏彪為征夷將軍,充總兵官,討兩廣瑤賊。三月壬子,免蘇、松、常、鎮被災稅糧。甲寅,湖廣、貴州總兵官李震會廣西軍剿瑤、僮,悉破之。

夏四月癸巳,兵部侍郎白圭督陜西諸邊。討孛來。五月丁未,免河南被滅秋糧。六月丙子,孛來寇河西,官軍敗績。壬午,兵部尚書馬昂總督軍務,懷寧伯孫鏜充總兵官,帥京營軍禦之。

秋七月庚子,總督京營太監曹吉祥及昭武伯曹欽反,左都御史寇深、恭順侯吳瑾被殺,懷寧伯孫鏜師兵討平之。癸卯,磔吉祥於市,夷其族,其黨湯序等悉伏誅。丁未,免南畿被災稅糧。庚戌,大赦,求直言。丁巳,河決開封,侍郎薛遠往治之。戊午,都督馮宗充總兵官,禦寇於河西,兵部侍郎白圭、副都御史王參贊軍務。辛酉,孛來上書乞和。九月壬戌,京師地震有聲。

冬十月壬申,以西邊用兵,令河南、山西、陜西士民納馬者予冠帶。十一月丁酉朔,日有食之。壬戌,幸南海子。

是年,安南、流球中山、哈密、亦力把裡入貢。

六年春正月丁未,大祀天地於南郊。戊申,孛來遣使入貢。二月癸酉,諭孛來。三月癸丑,召馮宗等還。

夏四月壬申,免河南被災秋糧。五月庚子,顏彪討平兩廣諸瑤。己未,免陜西被災秋糧。六月戊辰,淮王祁銓來朝。

秋七月,淮安海溢。九月乙未,皇太后崩。

冬十一月甲午,葬孝恭章皇后。

是年,琉球中山、哈密、烏斯藏、暹羅入貢。

七年春正月丙午,大祀天地於南郊。二月壬戌,詹事陳文為禮部侍郎兼翰林學士,入閣預機務。三月壬寅,旱,詔行寬恤之政,停各處銀場。

夏四月壬午,逮宣、大巡按御史李蕃,荷校於長安門,尋死。丙戌,復遣中官督蘇、杭織造。五月己丑朔,日有食之。甲寅,遼東巡按御史楊璡以擅撻軍職逮治。六月丁卯,逮山西巡按御史韓祺,荷校於長安門,數日死。

秋七月庚戌,免陜西被災稅糧。閏月甲戌,上宣宗廢后胡氏尊謚。戊寅,命湖廣、貴州會師討洪江叛苗。九月甲戌,敕廣東總兵官歐信會廣西兵討瑤賊。

冬十月丁酉,振西安諸府饑。丁未,巡撫廣西僉都御史吳楨節制兩廣諸軍,諸瑤賊。十一月癸酉,賊陷梧州,致仕布政使宋欽死之。壬午,下右都御史李賓、副都御史林聰於錦衣衛獄。十二月辛卯,下刑部尚書陸瑜,侍郎周瑄、程信於錦衣衛獄,尋釋之。

是年,琉球中山、哈密、安南、烏斯茂入貢。

八年春正月乙卯,帝不豫。己未,皇太子攝事於文華殿。己巳,大漸,遺詔罷宮妃殉葬。庚午,崩,年三十有八。二月乙未,上尊謚,廟號英宗,葬裕陵。

贊曰:英宗承仁、宣之業,海內富庶,朝野清晏。大臣如三楊、胡濙、張輔,皆累朝勛舊,受遺輔政,綱紀未弛。獨以王振擅權開釁,遂至乘輿播遷。乃復闢而後,猶追念不巳,抑何其感溺之深也。前後在位二十四年,無甚稗政。至於上恭讓后謚,釋建庶人之系,罷宮妃殉葬,則盛德之事可法後世者矣。


\end{pinyinscope}