\article{列傳第一百四十八}

\begin{pinyinscope}
楊鶴從弟鶚陳奇瑜元默熊文燦洪雲蒸練國事丁啟睿從父魁楚鄭崇儉方孔炤楊一鵬邵捷春餘應桂高斗樞張任學

楊鶴,字修齡,武陵人。萬曆三十二年進士。授雒南知縣,調長安。

四十年擢御史,上疏請東宮講學。且言:「頃者,愛女被躪於宮奴,館甥受撻於朝市,叩閽不聞,上書不達,壅蔽極矣。」時壽寧主婿冉興讓為掌家宮人梁盈女、內官彭進朝等毆辱,公主三奏不達,興讓掛冠長安門去,故鶴言及之。

尋出督兩淮鹽法,巡按貴州。貴州接壤烏撒,去川南敘州千里,節制難。土官安雲龍死,其族人與霑益安效良爭印,構兵三十年,後竟為效良所據,其父紹慶又據霑益州,皆川、雲、貴咽喉地。鶴請割烏撒隸貴州,地近節制便,可弭後患,朝議不決。未幾,效良為亂,如其言。貴州土官以百數,水西安氏最大,而土地、戶口、貢賦之屬,無籍可稽。鶴乃檄宣慰安位盡著之籍,並首領目把主名、承襲源委,悉列上有司。自是簿牒始明,奸弊易核。事竣,不候命徑歸。久之,還朝。

楊鎬四路師敗,鶴薦熊廷弼、張鶴鳴、李長庚、薛國用、袁應泰,言:「遼事之失,不料彼己,喪師辱國,誤在經略;不諳機宜,馬上催戰,誤在輔臣;調度不聞,束手無策,誤在樞部。至尊優柔不斷,又至尊自誤。」當事惡其直,將假他事逐之,乃引疾去。丁外艱。天啟初,起太僕少卿,擢右僉都御史,巡撫南、贛。未任,丁內艱,而廣寧又敗。魏忠賢以鶴黨護廷弼,除鶴名。

崇禎元年,召拜左僉都御史,進左副都御史。鶴上言:「圖治之要,在培元氣。自大兵大役,加派頻仍,公私交罄,小民之元氣傷;自遼左、黔、蜀喪師失律,暴骨成丘,封疆之元氣傷;自搢紳構黨,彼此相傾,逆奄乘之,誅鋤善類,士大夫之元氣傷。譬如重病初起,百脈未調,風邪易入,道在培養。」時以為名言。

先是,遼左用兵,逃軍憚不敢歸伍,相聚剽虜。至是,關中頻歲昆,有司不恤下。白水王二者,鳩眾,墨其面,闖入澄城,殺知縣張耀採。由是府谷王嘉允、漢南王大梁、階州周大旺群賊蜂起,三邊饑軍應之,流氛之始也。當是時,承平久,卒被兵,人無固志。大吏惡聞賊,曰:「此饑氓,徐自定耳。」明年,總督武之望死。久之,廷臣莫肯往者,群推鶴。帝召見鶴,問方略。對曰:「清慎自持,撫恤將卒而已。」遂拜鶴兵部右侍郎,代之望總督陜西三邊軍務。至則大梁、大旺、王二已前誅滅,而繼起者益眾。鶴素有清望,然不知兵。其冬,京師戒嚴,延綏、寧夏、甘肅、固原、臨洮五鎮總兵官悉以勤王行。延綏兵中道逃歸,甘肅兵亦嘩,懼誅,並合於賊,賊益張。

三年正月,王左掛等攻宜川,為知縣成材所卻,轉攻韓城。軍中無帥,鶴命參政洪承疇禦之。俘斬三百餘人,圍解,賊走清澗。鶴連疏請諸將還鎮,不果,起故將杜文煥任之。二月,延安知府張輦、都司艾穆蹙賊延川,降其魁王子順、張述聖、姬三兒。別賊王嘉允掠延安、慶陽,鶴匿不奏,而給降賊王虎、小紅狼、一丈青、掠地虎、混江龍等免死牒,安置延綏、河曲間。賊淫掠如故,有司不敢問。寇患成於此矣。

七月,嘉允陷黃甫、清水、木瓜,遂陷府谷,文煥擊走之,賊流入山西。已撫王左掛以白汝學攻綏德州,謀內應。事覺,巡按李應期與承疇計誅左掛等綏德,五十七人皆死。十二月,賊神一元攻陷新安、寧塞、柳樹澗等堡。寧塞,文煥所居,宗人多死。

明年正月,賊棄寧塞,陷保安。一元死,弟一魁圍慶陽,陷合水,鶴聞,移駐寧州。一魁求撫,送還合水知縣蔣應昌,別賊拓先齡、金翅鵬、過天星、田近庵、獨頭虎、上天龍等亦先後降。鶴設御座於城樓,賊跪拜呼萬歲。鶴宣聖諭,令設誓,或歸伍,或歸農,賊佯應之,則立赦其罪,群盜自是視總督如兒戲矣。鶴又以一魁最強,致其婿帳中,同臥起,而一魁果至。數以十罪,則稽首謝。即宣詔赦之,畀以官,處其眾四千餘人於寧塞,使守備吳弘器護焉。文煥聞之,嘆曰:「寧塞之役,賊畏我而逃。今者賊偽降,楊公信之,借名城為盜資。我宗人,可與賊逼處此土乎!」遂以其族行。

五月,鶴移駐耀州。賊攻破金鎖關,殺都司王廉。七月,別賊李老柴、獨行狼攻陷中部,田近庵以六百人守馬欄山應之。而降渠一魁之黨茹成名者,尤桀驁,鶴令一魁誘殺之於耀州,其黨猜懼,挾一魁以叛。御史謝三賓言:「鶴謂慶陽撫局既畢,賊散遣俱盡。中部之賊,寧自天降?」疏下巡按御史吳甡核奏,甡奏鶴主撫誤國。帝怒,逮鶴下獄,戍袁州。

七年秋,子嗣昌擢宣大山西總督,疏辭,言:「臣父鶴以總督蒙譴已三年,臣何心復居此職。」帝優詔答之,而不赦鶴罪。八年冬,鶴卒於戍所,嗣昌請恤。帝復鶴官,而不予恤。鶴初以尤世祿寧夏大捷功,進兵部尚書、太子少保,世蔭錦衣千戶。十年,敘賀虎臣寧夏破賊功,追加太子少傅。十三年,又以甘肅敘功,任一子官。

從弟鶚,崇禎四年進士。官御史,有才名,擢順天巡撫。京師陷,南歸,福王以為兵部右侍郎,總督川、湖軍務。

陳奇瑜,字玉鉉,保德州人。萬曆四十四年進士。除洛陽知縣。天啟二年,擢禮科給事中。楊漣劾魏忠賢,奇瑜亦抗疏力詆。六年春,由戶科左給事中出為陜西副使,遷右參政,分守南陽。

崇禎改元,加按察使職,尋歷陜西左右布政使。五年,擢右僉都御史,代張福臻巡撫延綏。時大盜神一魁、不沾泥等已殲,而餘黨猶眾。歲大凶,民多從賊。明年五月,奇瑜上疏,極言鄜、延達鎮城千餘里饑荒盜賊狀,詔免延安、慶陽田租。奇瑜乃遣副將盧文善討斬截山虎、柳盜跖、金翅鵬等。尋遣遊擊常懷德斬薛仁貴,參政戴君恩斬一條龍、金剛鑽、開山鷂、黑煞神、人中虎、五閻王、馬上飛,都司賀思賢斬王登槐,巡檢羅聖楚斬馬紅狼、滿天飛,參政張伯鯨斬滿鵝,擒黃參耀、隔溝飛,守備閻士衡斬張聰、樊登科、樊計榮、一塊鐵、青背狼、穿山甲、老將軍、二將軍、滿天星、上山虎,把總白士祥斬掃地虎,守備郭金城斬扒地虎、括天飛,守備郭太斬跳山虎、新來將、就地滾、小黃鶯、房日兔,遊擊羅世勛斬賈總管、逼上天、小紅旗,他將斬草上飛、一隻虎、一翅飛、雲裏手、四天王、薛紅旗、獨尾狼,諸渠魁略盡。奇瑜乃上疏曰:「流寇作難,始於歲饑,而成於元兇之煽誘,致兩郡三路皆盜藪。今未頓一兵,未絕一弦,擒斬頭目百七十七人,及其黨千有奇。頭目既除,餘黨自散,向之斬木揭竿者,今且荷鋤負耒矣。」帝嘉之,令錄有功將士以聞。

延綏群賊多解,獨鑽天哨、開山斧據永寧關。永寧在鎮城東,前阻山,下臨黃河,數年不下。奇瑜謂是不可以力取,乃陰簡銳士,陽言總制檄發兵,令賀人龍將之而西,身為後勁,直抵延川。俄策馬東,曰:「視吾馬首所向。」潛師疾走入山,賊不虞大兵至,驚潰。焚其巢,斬首千六百有奇,二賊俱馘。分兵擊斬金翅鵬、一座城,獲首五百五十。延水群盜盡平,奇瑜威名著關陜。於是群盜盡萃於山西,流突河北、畿南。冬冰堅,從澠池渡,躪河南、湖廣,窺四川。

明年,廷議諸鎮撫事權不一,宜設大臣統之,多推薦洪承疇。以承疇方督三邊,不可易,乃擢奇瑜兵部右侍郎兼右僉都御史,總督陜西、山西、河南、湖廣、四川軍務,專辦流賊。奇瑜檄諸將會兵陜州。先是,老回回、過天星、滿天星、闖塌天、混世王五大營自楚入蜀,陷夔州。阻險,復走還楚,分為三:一犯均州,往河南;一犯鄖陽,往淅川;一犯金漆坪,渡河犯商南。奇瑜乃馳至均州,檄四巡撫會討。陜西練國事駐商南,遏其西北;鄖陽盧象昇駐房、竹,遏其西;河南元默駐盧氏,遏其東北;湖廣唐暉駐南漳,遏其東南。奇瑜乃偕象升督將士由竹谿至平利之烏林關,十餘戰,斬賊千七百餘級。越七日,大破之乜家溝,斬千八十餘級,總兵鄧功為多。已,設伏蚋谿,連戰,斬三百餘級。至獅子山,斬七百二十餘級。別將楊化麟、楊世恩、周任鳳、楊正芳等分道擊殺賊,擒其魁闖王、翻山虎等。

奇瑜上言:「楚中屢捷,一時大盜幾盡,其竄伏深山者,臣督鄉兵為嚮道,無穴不搜,楚中漸有寧宇。」帝嘉勞之。乃督副將劉遷等搜竹谿、平利賊,追至五狼河,擒其魁十二人。遣參將賀人龍等追八晝夜至紫陽,賊死者萬餘人。

先是,賊入蜀,復自蜀入秦,由陽平關奔鞏昌,承疇禦之秦州。賊遂越兩當,襲破鳳縣,分為二:一向漢中,取間道犯城固、洋縣;一由鳳縣奔寶雞、汧陽。於是賊在平利、洵陽間者數萬,自四川入西鄉者二三萬。犯城固、洋縣者,又東下石泉、漢陰,會漢、興,窺商、雒。當是時,奇瑜以湖廣賊盡,鼓行而西,謂賊不足平也。乃遣遊擊唐通防漢中,以護籓封;遣參將賀人龍、劉遷、夏鎬扼略陽、沔縣,防賊西遁;遣副將楊正芳、餘世任扼褒城,防賊北遁;自督副將楊化麟、柳國鎮等駐洋縣,防賊東遁;又檄練國事、盧象升、元默各守要害,截賊奔逸。

賊見官軍四集,大懼,悉遁入興安之車廂峽,諸渠魁李自成、張獻忠等咸在焉。峽四山巉立,中亙四十里,易入難出。賊誤入其中,山上居民下石擊,或投以炬火,山口累石塞,路絕,無所得食,困甚。又大雨二旬,弓矢盡脫,馬乏芻,死者過半。當是時,官軍蹙之,可盡殲,自成等見勢絀,用其黨顧君恩謀以重寶賄奇瑜左右及諸將帥,偽請降。奇瑜無大計,遽許之,先後籍三萬六千餘人,悉勞遣歸農。每百人以安撫官一護之,檄所過州縣具糗糧傳送,諸將無邀撓撫事。諸賊未大創,降非實也,既出棧道,遂不受約束,盡殺安撫官五十餘人,攻掠諸州縣,關中大震。

奇瑜悔失計,乃委罪他人以自解。賊初叛,猝至鳳翔,誘開城,守城知其詐,紿以縋城上,殺其先登者三十六人,餘噪而去。其犯寶雞,亦為知縣李嘉彥所挫。奇瑜遂劾嘉彥及鳳翔鄉官孫鵬等撓撫局,撫按官亦異心。帝怒,切責撫按,逮嘉彥、鵬及士民五十餘人。奇瑜又請敕陜西、鄖陽、湖廣、河南、山西五巡撫各守要害,有失則治諸臣罪,冀以分己過。又委罪巡撫練國事,國事亦被逮。給事中顧國寶劾奇瑜誤封疆,詔解任候勘。御史傅永淳復劾奇瑜解隴州圍報首功不實,詔除名,錦衣官逮訊。九年六月謫戍邊。

初,奇瑜官南陽,唐王殺其世子,欲并廢世子子聿鍵。賴奇瑜力,聿鍵得為世孫。後聿鍵自立於閩,召奇瑜為東閣大學士。道遠,未聞命,卒於家。

元默,字中象,靜海人。萬歷四十七年進士。除懷慶推官,擢吏科給事中。魏忠賢焰方熾,以鄉里欲招致之,默謝不可。言路承忠賢意,劾罷歸。

崇禎初,復官,歷遷太常卿。六年春,以僉都御史巡撫河南。流賊由均州犯河內,默率左良玉、湯九州、李卑、鄧兵待境上;復率九州乘雪夜薄吳城賊營,大破之。嵩、雒以北名城數十,賊避勿敢攻。奇瑜既失李自成於車箱峽,默自汝州移駐盧氏,檄良玉、九州各陳兵守要害,得稍寧者數月。當是時,賊勢張,良玉等承督師檄,守備尚固。默率諸將斬獲多,賊多趨秦、楚境。已,分為三,自潁州犯鳳陽皇陵,中州所在告急。八年夏,默被逮去。久之,得釋歸,八年卒。

熊文燦,貴州永寧衛人。萬曆三十五年進士。授貴州推官,遷禮部主事,歷郎中。出封琉球還,擢山東左參政、山西按察使、山東右布政使。憂歸,自是徙家蘄水。

崇禎元年,起福建左布政使。三月,就拜右僉都御史,巡撫其地。海上故多劇盜,袁進、李忠既降,楊六、楊七及鄭芝龍繼起。總兵官俞咨皋招六、七降,芝龍猖獗如故。然芝龍常敗都司洪先春,釋不追;獲一遊擊,不殺;咨皋戰敗,縱之走。當事知其可撫,遣使諭降之。文燦至,善遇芝龍,使為己用。其黨李魁奇再降,再叛去,芝龍擊擒之。海警漸息,而鐘斌又起。斌初亦就撫,後復叛,寇福州。文燦誘斌往泉州,令芝龍擊敗之。既而蹙之大洋,斌投海死。閩中屢平巨寇,皆芝龍力,文燦亦敘功增秩焉。

五年二月,擢文燦兵部右侍郎兼右僉都御史,總督兩廣軍務,兼巡撫廣東。先是,海寇鐘靈秀既降復叛,為芝龍所擒,其黨潰入長汀,轉掠江西屬邑,文燦檄芝龍屢敗賊。而福建有紅夷之患,海盜劉香乘之,連犯閩、廣沿海邑,帝以責文燦。文燦不能討,乃議招撫,賊佯許之。參政洪雲蒸,長沙人,初官廣西參政,嘗搜靈秀餘黨,斬三十餘級,盡毀其巢。文燦乃令雲蒸與副使康承祖,參將夏之本、張一傑入賊舟宣諭,俱被執。文燦懼罪,奏諸臣信賊自陷。給事中朱國棟劾之,詔貶秩,戴罪自效。八年,芝龍合廣東兵擊香於田尾遠洋。香脅雲蒸止兵,雲蒸大呼曰:「我矢死報國,急擊勿失!」遂遇害。香勢蹙,自焚溺死,承祖等脫還。賊黨千餘人詣浙江歸款,海盜盡平。

文燦官閩、廣久,積貲無算,厚以珍寶結中外權要,謀久鎮嶺南。會帝疑劉香未死,且不識文燦為人,遣中使假廣西采辦名,往覘之。既至,文燦盛有所贈遺,留飲十日。中使喜,語及中原寇亂,文燦方中酒,擊案罵曰:「諸臣誤國耳。若文燦往,詎令鼠輩至是哉!」中使起立曰:「吾非往廣西采辦也,銜上命覘公。公信有當世才,非公不足辦此賊。」文燦出不意,悔失言,隨言有五難四不可。中使曰:「吾見上自請之,若上無所吝,即公不得辭矣。」文燦辭窮,應曰「諾」。中使還朝,果言之帝。初,文燦徙蘄水,與邑人姚明恭為姻妮,明恭官詹事,又與楊嗣昌善。嗣昌握兵柄,承帝眷,以帝急平賊,冀得一人自助,明恭因薦文燦,且曰:「此有內援可引也。」嗣昌喜,遂薦之。

十年四月,拜文燦兵部尚書兼右副都御史,代王家禎總理南畿、河南、山西、陜西、湖廣、四川軍務。文燦拜命,即請左良玉所將六千人為己軍,而大募粵人及烏蠻精火器者一二千人以自護,弓刀甲胄甚整。次廬山,謁所善僧空隱。僧迎謂曰:「公誤矣。」文燦屏人問故,僧曰:「公自度所將兵足制賊死命乎?」曰:「不能。」曰:「諸將有可屬大事、當一面、不煩指揮而定者乎?」曰:「未知何如也。」曰:「二者既不能當賊,上特以名使公,厚責望,一不效,誅矣。」文燦卻立良久,曰:「撫之何如?」僧曰:「吾料公必撫。然流寇非海寇比,公其慎之。」文燦去,抵安慶,帝所遣中官劉元斌、盧九德監勇衛營軍者亦至。良玉宿將桀驁,不受文吏節制,會其下與粵軍不和,大詬。文燦不得已,遣還南兵,然良玉軍實不為用。嗣昌言於帝,乃以邊將馮舉、苗有才兵五千人隸焉。有才敗於真陽,而京營將黃得功連破賊兵,威甚振。

當是時,嗣昌建「四正六隅」之策,增兵餉大半,期滅賊,賊頗懼。及文燦至,京軍屢捷,益懼。文燦顧決計招降。初抵安慶,即遣人招張獻忠、劉國能,二人聽命。乃益刊招降檄,布通都。又請盡遷民與粟閉城中,賊無所掠,當自退。帝怒,譙讓文燦。嗣昌亦心非之,既已任之,則曲為文燦解。因其請,畀以畿輔、山西兵各三千。明年,國能果降,而獻忠襲據穀城。會得功又大破賊舞陽,馬士秀、杜應金夜半降信陽城下。獻忠為左良玉所創,幾被擒,其下饑困多散去。獻忠窮蹙,亦因陳洪範以降。於是嗣昌議功罪,絀洪承疇、曹變蛟等,而稱文燦功焉。

已而京軍解遂平圍,斬獲三千有奇。時文燦在裕州,馬進忠、羅汝才十三家賊聚南陽,文燦下令,殺賊者償死。賊不肯從,則齎金帛酒牢犒之,名曰「求賊」。帝詗得狀,曰:「文燦大言無實。」文燦恐。孫傳庭出關擊賊,文燦不救,而嗣昌已入政府掌中樞矣。九月,文燦次襄陽,賊分踞鄖、襄諸險。諸將請戰,文燦議分兵。盧九德曰:「兵分則力弱,一失利,全軍搖矣。莫若厚集其力而合擊之。」眾曰:「善。」乃以僉事張大經監大將左良玉、陳洪範軍,以通判孔貞會監副將龍在田軍,戰於雙溝,大破之,斬首二千餘級。羅汝才、惠登相率九營走均州,李萬慶率三營走光、固。

十一月,京師戒嚴,召洪承疇、孫傳庭入衛。汝才等以為討己也,懼而叩太和山提督中官,求撫於文燦,許之。處汝才及一丈青、小秦王、一條龍四營於鄖縣,處登相及王國寧、常德安、楊友賢、王光恩五營於均州。上言:「臣於李萬慶、賀一龍、馬光玉及順天王主剿,他皆主撫。請赦汝才等罪,授之官。」可之。時京軍、良玉軍皆以入衛行,馬士秀、杜應金遂叛於許州。初,士秀等降,良玉以其眾處許之郊外。許,大州也,良玉諸將寄孥與賄焉。良玉久征不歸,士秀、應金在文燦軍中,偽請急,假良玉軍號入城。夜半,兵從府第出,燒城南樓,劫庫,殺官吏,挈其貲投萬慶。萬慶者,賊魁射塌天也。

十二年三月,良玉還,破降馬進忠,使劉國能擊降萬慶,士秀、應金亦再降。順天王已前死,其黨順義王為其下所殺。文燦遂上言:「臣兵威震懾,降者接踵。十三家之賊,惟革、左及馬光玉三部尚稽天誅,可歲月平也。」帝優詔報之。

初,張獻忠之降也,擁兵萬人踞穀城,索十萬人餉,文燦及中外要人曰與之。為請官、請地、請關防矣,獻忠列軍狀曰請備遣,既而三檄其兵不應,朝野知獻忠必叛也。其後,汝才降,不肯釋甲。及進忠、萬慶等並降,文燦以為得策,謂天下且無賊也。五月,獻忠遂反於穀城,劫汝才於房縣,於是九營俱反。初,均州五營懼見討,自疑,相與歃血拒獻忠,無何亦叛去。帝聞變,大驚,削文燦官,戴罪視事。七月,良玉擊獻忠羅英山,敗績。帝大怒,命嗣昌來代。嗣昌已至軍,即遣使逮文燦下獄,坐大辟,所親姚明恭柄國而不能救也。十三年十月,文燦竟棄市。

練國事,字君豫,永城人。萬曆四十四年進士。授沛縣知縣,調山陽。

天啟二年,徵授御史。廣寧失守,國事請薊州、宣府、大同及山東、山西、河南撫臣各練兵萬,以壯山海聲援。又請捕誅殺大同妖人。又疏論魏忠賢使群閹辱尚書鐘羽正,索冬衣,傷國體。國事在諫垣,匡救多。給事中趙興邦,忠賢私人也,以國事為趙南星黨,劾之,削籍。

崇禎元年復官,擢太僕少卿,進右僉都御史,巡撫陜西。關中頻歲饑,盜賊蜂起。四年正月,神一元陷保安。國事遣賀虎臣援延安,而身率副將張全昌連破點燈子於中部、合阜陽、韓城,又破別部於宜君、雒川,降其魁李應鰲。諸將張全昌、趙大允、王承恩、杜文煥、賀虎臣等分剿賊澄城、宜川、耀州、白水、合阜陽,斬首千九百有奇。總督楊鶴既受群賊降,已,復相繼叛,田近庵、李老柴陷中部。國事偕承恩攻圍五月,克之,而所部亦頻失事,楊鶴被徵,國事亦戴罪自贖。

五年,紅軍友、李都司等將犯平涼。國事自涇趨固原,檄大帥楊嘉謨殺賊塘馬,斷其偵探。賊乃走慶陽西壕,嘉謨、曹文詔邀擊,大敗之。自三月至五月,大小數十戰,賊迄破滅。國事免戴罪。

當是時,關中五鎮,大帥曹文詔、楊嘉謨、王承恩、楊麟、賀虎臣各督邊軍協討,總督洪承疇尤善調度。賊魁多殲,餘盡走山西,關中稍靖。

六年冬,賊既從澠池渡,入盧氏。明年,賊遂由河南、湖廣入漢南。總督陳奇瑜檄國事駐商州,協剿商南、盧氏賊。漢南賊遂由寧羌至兩當,掠鳳縣,出棧道,陷寶雞,關中賊復熾。既而奇瑜受賊降,檄諸軍勿擊。賊出險,遂大掠鳳翔、麟游、寶雞、扶風、汧陽、乾州、涇陽、醴泉。奇瑜委罪國事以自解,國事上言:「漢南賊盡入棧道,奇瑜檄止兵,臣未知所撫實數。及見奇瑜疏,八大王部萬三千餘人,蠍子塊部萬五百餘人,張妙手部九千一百餘人,八大王又一部八千三百餘人,臣不覺仰天長歎。夫一月內,撫強寇四萬餘,盡從棧道入內地,食飲何自出,安得無剽掠?且一大帥將三千人,而一賊魁反擁萬餘眾,安能受紀律?即藉口回籍,延安州縣驟增四萬餘人,安集何所?合諸征剿兵不滿二萬,而降賊踰四萬,豈內地兵力所能支,宜其連陷名城而不可救也。若咎臣不堵剿,則先有止兵檄矣;若云賊已受撫,因誤殺使人致然,則未誤殺之先,何為破麟游、永壽。今事已至此,惟急調大軍致討,若仍以願回原籍,禁兵勿剿,三秦之禍安所終極哉!」疏入,事已不可為,遂逮下獄。九年正月遣戍廣西。久之,敘前功,赦還,復冠帶。

福王時,召為戶部左侍郎,尋改兵部。十二月加尚書,仍蒞侍郎事。明年二月致仕,未幾卒。

丁啟睿,永城人。萬曆四十七年進士。崇禎初,歷山東右參政,坐事謫陜西副使。九年,寧夏兵變,啟睿捕斬殺巡撫王楫者首惡六人,軍中大定。再遷右布政使,分守關南,從巡撫孫傳庭討賊。

十一年冬,就拜右僉都御史,代傳庭巡撫陜西。歲頻旱,民益為盜,長武、環、白水、長安、臨潼、咸陽賊起如蝟毛。十三年,用督師楊嗣昌薦,擢兵部右侍郎兼右僉都御史,代鄭崇儉總督陜西三邊軍務討賊。明年,嗣昌死,加啟睿兵部尚書,改稱督師,代嗣昌盡督陜西、湖廣、河南、四川、山西及江南、北諸軍,仍兼總督陜西三邊軍務,賜劍、敕、印如嗣昌。

啟睿自謫河西副使,數遷皆在陜西,然實庸才。為督、撫,奉督師期會,謹慎無功過;及督師任重專制,即莫知為計。啟睿已受命出潼關,將由承天赴嗣昌軍於荊州。湖廣巡按汪承詔言大寇在河南,荊、襄幸息警,無煩大軍,盡匿漢津船。啟睿至,五日不得渡,折而向鄧州,州人閉門詬;過內鄉,長吏閉之糴。軍行荒山間,割馬騾,燎以野草,士啖不得飽。是時李自成已陷洛陽,圍開封,有眾七十萬,啟睿憚不敢援。聞張獻忠在光山、固始間,少弱,乃謀於諸將曰:「上命我剿豫賊,此亦豫賊也。」遂檄左良玉破之於麻城,斬首千二百。開封日告急,則曰:「我方有事於獻忠,不赴矣。」聞傅宗龍將入關督秦師,啟睿曰「三邊已置總督矣」,乞帝更敕書,乃更敕書宗龍辦自成。九月,宗龍敗歿於項城,啟睿不能救。賊乘勝陷南陽,殺唐王,開、汝二郡望風下。十二月,自成再圍開封。河南巡撫高名衡飛檄至,啟睿督兵赴之,避賊入城,部下大淫掠。總兵陳永福射自成,中其左目。明年正月,賊解圍去。

啟睿之在許州也,畏賊逼,始赴開封。離城三十里,而城即破。其抵開封,啟門入,賊乘之,幾陷。四月,自成合群賊復攻開封。六月,帝釋侯恂於獄,命督援剿諸軍救開封。未至,開封圍益急。帝數詔切責啟睿。啟睿不得已,乃大集良玉、虎大威、楊德政、方國安之軍,偕保定總督楊文岳,以七月會於朱仙鎮,與賊壘相望。賊眾百萬,啟睿欲戰,良玉曰:「賊鋒銳,未可擊也。」啟睿曰:「圍已急,必擊之。」諸將皆懼。良玉歸營即先走,諸營俱走,啟睿、文岳聯騎奔汝寧。賊渡河逐之,追奔四百里。喪馬騾七千,將士數萬,啟睿敕書、印、劍俱失。事聞,詔褫職候代。九月,賊決馬家口河灌開封,開封遂陷,乃徵下吏,久之釋歸。自嗣昌死二年而啟睿敗,啟睿敗又二年而明亡矣。

福王時,啟睿夤緣馬士英充為事官,督河南勸農、剿寇諸務。尋以擒斬歸德偽官,拜兵部尚書,加太子太保,官其一子。事敗,脫身旋里,久之卒。

從父魁楚,崇禎四年春,以右僉都御史巡撫保定。七年,擢兵部右侍郎,代傅宗龍總督薊、遼、保定軍務。九年七月,畿輔被兵,魁楚坐下吏,久之放還。福王時,起故官,總督河南、湖廣,兼巡撫承天、德安、襄陽。未赴,會兩廣總督沈猶龍入為侍郎,魁楚竟代其任。尋加兵部尚書。唐王自立於福州,命以故官協理戎政。靖江王亨嘉反桂林,下梧州,執巡撫瞿式耜。魁楚檄思恩參將陳邦傳等襲走之,獲於桂林。封魁楚平粵伯,仍留鎮兩廣。閩中事敗,與式耜擁立桂王於肇慶,進東閣大學士,兼理戎政。大清兵下廣州,漸迫肇慶。魁楚奉王走梧州,復棄之,走岑溪。輜重多,舳艫相屬,為大將李成棟追獲,魁楚遂降。成棟與有隙,錄其家數百人殺之,魁楚乞一子,成棟笑曰:「汝身且莫保,尚求活人耶?」并殺之。

鄭崇儉,字大章,寧鄉人。萬曆四十四年進士。授河南府推官,歷濟南兵備副使。崇禎初,遷陜西右參政。屢遷右僉都御史,巡撫寧夏。數敗套寇,賚銀幣,世廕錦衣副千戶。

十二年正月,擢兵部右侍郎,代洪承疇總督陜西三邊軍務。五月,張獻忠反穀城,羅汝才等九營皆反,興安告警。總理熊文燦請敕楚撫方孔炤防荊門、當陽,鄖撫王鰲永防江陵、遠安,陜撫丁啟睿、蜀撫邵捷春各嚴兵於其境。而崇儉主提兵合擊,時固原、臨洮、寧夏三總兵左光先、曹變蛟、馬科隨承疇人衛,柴時華中道還甘肅,征之不應,崇儉乃檄副將賀人龍、李國奇等軍發西安。國奇至洛陽,卒大噪,剽瑞王租。國奇已擢陜西總兵官,坐停新命,崇儉亦貶一秩。

獻忠既叛,大敗左良玉軍於房縣之羅英山,謀入陜。崇儉率人龍、國奇軍扼之興安,賊還走興山、太平,處楚、蜀交。是時,楊嗣昌已出師,入文燦軍而代之矣。先是,尚書傅宗龍議令崇儉兼督蜀軍,而嗣昌亦檄秦軍入蜀。崇儉乃以十三年二月率人龍、國奇會良玉大敗賊於瑪瑙山,獲首功千三百三十有三,降賊將二十有五人,獲馬騾、甲仗無算。是役也,崇儉身在行,而嗣昌遠處襄陽。及論功,所賜半嗣昌,但增一秩,復先所降一秩而已。

獻忠既敗,竄柯家坪,蜀將張令追之,被圍。崇儉遣兵擊走賊,人龍、國奇等復追敗之寒溪寺、鹽井,先後斬首千五百級,其黨順天王、一條龍、一只龍皆降。崇儉軍五日三捷,威名甚振。以年衰乞骸骨,不許,令率總兵鄭家棟還關中,留人龍、國奇討賊。

當是時,獻忠竄伏興、歸山中。秦、楚師俱集於夔,諸將協心窮搜深箐,千餘殘寇可盡殲。崇儉既去,未幾,人龍軍亦自開縣噪而西歸,楚師遂敗績於土地嶺,蜀中由是大亂。嗣昌因言崇儉撤兵太早,致賊猖獗。帝初以崇儉不能馭軍,不悅,及是命削籍,以啟睿赴軍前代理,而疑崇儉託疾,令按臣核實。明年春,獻忠陷襄陽,嗣昌死,帝益恨崇儉不掎角平賊也,逮下獄,責以縱兵擅還,失誤軍律。不俟秋後,以五月棄市。

帝自即位以來,誅總督七人,崇儉及袁崇煥、劉策、楊一鵬、熊文燦、范志完、趙光抃也。帝憤寇日熾,用法益峻,功罪不假貸,而疆事寢壞,卒至於亡。福王時,給事中李清言:「崇儉未失一城、喪一旅,因他人巧卸,遂服上刑。群臣微知其冤,無敢訟言者,臣甚痛之。」崇儉冤始白。

方孔炤,字潛夫,桐城人。萬曆四十四年進士。天啟初,為職方員外郎。忤崔呈秀,削籍。

崇禎元年,起故官。憂歸。定桐城民變,還朝。十一年,以右僉都御史巡撫湖廣,擊賊李萬慶、馬光玉、羅汝才於承天,八戰八捷。時文燦納獻忠降,處之穀城。孔炤條上八議,言主撫之誤,不聽,而陰厲士馬備戰守。已而賊果叛,如孔炤言。賊故畏孔炤,不敢東,文燦乃檄孔炤防荊門、當陽,鰲永防江陵、遠安,秦、蜀各嚴兵。崇儉主合擊,孔炤乃請專斷德、黃,守承天,護獻陵;而江、漢以南責鰲永。會嗣昌代文燦,令孔炤仍駐當陽。惠王常潤言:「孔炤遏獻忠,有來家河、神通堡之捷,射中賊魁馬光玉,陵寢得無虞。請增秩久任。」章下部,未奏,而部將楊世恩、羅安邦奉調,會川、沅兵剿竹山寇。兩將深入,至香油坪而敗。嗣昌既以孔炤撫議異己也,又忮其言中,遂因事獨劾孔炤,逮下詔獄。子檢討以智,國變後,棄家為僧,號無可者也,伏闕訟父冤,膝行沙塸者兩年。帝為心動,下議,孔炤護陵寢功多,減死戍紹興。久之,用薦復官,以右僉都御史屯田山東、河北。馳至濟南,復命兼理軍務,督大名、廣平二監司禦賊。命甫下而京師陷,孔炤南奔。馬、阮亂政,歸隱十餘年而終。

先是,有以陵寢失守獲重譴者,為楊一鵬。一鵬,臨湘人。歷官大理寺丞,削籍。崇禎六年,以兵部左侍郎拜戶部尚書兼右僉都御史,總督漕運,巡撫江北四府。鳳陽軍民素疾守陵太監楊澤貪虐,引賊來寇。八年正月,賊遂攻陷鳳陽,焚皇陵,燒龍興寺,燔公私邸舍二萬二千六百五十,戮中都留守朱國相、指揮使程永寧等四十有一員,殺軍民數萬人。

先是,賊漸逼江北,兵部尚書張鳳翼請敕一鵬移鎮鳳陽,溫體仁格其議。賊驟至,一鵬在淮安,遠不及救。帝聞變大驚,素服避殿,親祭告太廟,遂逮一鵬及巡按御史吳振纓、守陵官澤。澤先自殺,一鵬棄市,振纓戍邊。

邵捷春,字肇復,侯官人。萬曆四十七年進士。累官稽勛郎中。

崇禎二年,出為四川右參政,分守川南,撫定天全六番高、楊二氏。遷浙江按察使。大計,坐貶。久之,起四川副使,以十年秋抵成都。時秦賊已入蜀,巡撫王維章、總兵侯良柱悉眾北拒,城中惟屯田軍及蜀府護衛軍,人情恇懼,捷春啟門納鄉民避賊者。中尉奉鐕勾賊抵城下,捷春與御史陳廷謨擒擊奉鐕,而募市人、起廢將固守。賊去,蜀王疏其功。會維章罷,傅宗龍代,命捷春監軍,偕總兵羅尚文擊賊。明年,尚文及安錦副使吳麟征大破賊過天星等。捷春進右參政,仍監軍。

十二年五月,宗龍入掌中樞,即擢捷春右僉都御史代之。時張獻忠、羅汝才已叛,謀入秦。秦兵扼之興安,乃犯興山及蜀太平,遂窺大寧。捷春遣副將王之綸、方國安分道扼之。國安連破賊,賊遂還入秦、楚。十月朔,楊嗣昌誓師襄陽,檄蜀軍受節度。嗣昌以楚地廣衍,賊難制,驅使入蜀,蜀險阻,賊不得逞,蹙之可全勝,又慮蜀重兵扼險,賊將還毒楚,調蜀精銳萬餘為己用,蜀中卒自是益疲弱不足支矣。捷春憤曰:「令甲失一城,巡撫坐。今以蜀委賊,是督師殺我也。」爭之,不能得。於是汝才、惠登相遂自興山、遠安犯大寧、大昌,獻忠亦西至太平。明年二月,左良玉大破獻忠瑪瑙山,他將張應元、張令等復數敗之。獻忠乃逃興、歸山中。久之復振,由汝才入寧昌故道走而西。

初,汝才在寧昌阻江,為諸將劉貴、秦良玉、秦翼明、楊茂選等所拒,不得渡。會獻忠西,遂與合。貴等戰皆卻,賊乃渡江,營萬頃山、苦桃灣,其別部營紅茨崖、青平砦,歸、巫間大震。嗣昌乃上夷陵,而檄捷春扼夔門。蜀大寧、大昌界楚竹溪、房縣,有三十二隘口,嗣昌欲厚集兵力專守夔,棄寧、昌啖賊,官軍環攻之。捷春曰:「棄隘口不守,是延賊入戶也。」乃遣茂選及覃思岱等出關分守。二將不相得,思岱潛殺茂選,捷春令兼統其眾,其眾相率去。賊入隘,守者潰,賊夜斬夔關,將士大驚潰,新寧、大竹皆陷。而汝才、登相越巴霧河,陷開縣,為鄭嘉棟、賀人龍所破。汝才乃與小秦王、混世王東奔,而登相獨過開縣西。人龍及李國奇又西追之,汝才等遁還興山,屢挫。會嗣昌下招降令,小秦王、混世王皆降,惟汝才逸去。嗣昌見楚地無賊,以八月終率師入蜀,於是群賊盡萃蜀中。

當是時,捷春提弱卒二萬守重慶,所倚惟秦良玉、張令軍。無何,秦師噪而西歸,楚將張應元等敗績於夔州之土地嶺。於是捷春以大昌上、中、下馬渡水淺地平,難與持久,乃扼水寨觀音巖為第一隘,以部將邵仲光守之。而夜叉巖、三黃嶺、磨子巖、魚河洞、下涌諸處,各分兵三四百人以守。萬元吉以兵分力弱為憂,捷春不聽。九月,獻忠突敗仲光軍,破上馬渡。元吉急檄諸將分邀之,復令張奏凱屯凈壁,捷春遣二將羅洪政、沈應龍為助。十月,獻忠突凈壁,遂陷大昌,屯開縣。良玉、令兩軍皆覆。賊行則哨探,止則息馬抄糧。關隘偵候不明,防軍或遠離戍所,賊乘隙而過無人之境。嗣昌遂收斬仲光,上疏劾捷春失事。捷春收兵扼梁山。時登相已歸正,而汝才復與獻忠合,以梁山河深不能渡,乃自開縣西走達州。捷春退保綿州,扼涪江。賊疾走,陷劍州,遂趨廣元,將由間道入漢中,為秦兵所扼,乃復走巴西。應元諸軍邀之梓潼,戰小利,既而衄,蜀將曹志耀等力戰卻之。降將張一川、張載福陷陣死,涪江師遂潰,賊屠綿州。捷春歸成都,賊逼成都。十一月,逮捷春使者至,遂以軍事付代者廖大亨而去。

捷春為人清謹,治蜀有惠政。士民哭送者載道,舟不得行,競逐散官旗。蜀王為疏救,不聽。敕巡按御史遣官送京師,下獄論死。捷春知不可脫,明年八月仰藥死獄中。福王時,復官,贈兵部右侍郎。

餘應桂,字二磯,都昌人。萬歷四十七年進士。歷知武康、龍巖、海澄三縣。

崇禎四年,徵授御史。劾戶部尚書畢自嚴朋比,殿試讀卷,取陳于泰第一。于泰者,首輔周延儒姻也。劾延儒納孫元化參、貂,受楊鶴重賂。帝方眷延儒,責應桂。未幾,賊陷登州,元化被執,應桂再疏劾延儒。帝怒,貶三秩視事,應桂引疾歸。

七年還朝,出按湖廣,居守承天。捐贖鍰十餘萬募壯士,繕城治器,賊不敢逼獻陵。帝聞而嘉之。期滿,命再巡一年。貽贖鍰萬五千助盧象昇軍需,而奏報屬城失事,具以實聞。帝以是知巡撫王夢尹詐,而益信應桂。期滿,命再巡一年。十年,即擢應桂右僉都御史,代夢尹。

當是時,諸監司袁繼咸、包鳳起、高斗樞輩已削平湖南群賊,而江北賊勢日熾,諸將雖奏捷,不能大創也。帝命熊文燦為總理,文燦主撫。明年,降其渠劉國能、張獻忠。馬進忠西走潼關,馬光玉、賀一龍、李萬慶、順義王、九條龍眾十餘萬萃麻城、黃安。應桂諭降光玉、一龍,未至,而遣將擊順天王等於黃福店,賊遂走黃安。會文燦至麻城,應桂請協擊,不從。賊復東走江北,為左良玉所遏,折而走廣濟、蘄水。文燦檄諸道兵合擊賊於茶山,賊逸於應桂所分地,文燦遂劾其後期誤軍。兵部尚書楊嗣昌以應桂曾劾其父鶴也,奏逮之。應桂乃陳撫剿始末,白己無罪,而詆文燦,言:

正月初,議撫劉國能,其黨李萬慶等諸大賊盡走泌陽、棗陽。時文燦、良玉並在德安。臣以為兵勢方盛,宜乘此追剿,而文燦調良玉諸軍盡赴信陽剿馬進忠。臣謂進忠小寇,勝之不武,文燦不聽。自此機一失,賊走西,而文燦東,致張獻忠攻陷穀城以要撫,李萬慶五部收餘燼,勢復振。而豫、楚之患,遂自文燦之愎諫貽之矣。迨賊西潰之後,遮飾上聞,妄報斬級。其自恃所長惟火炮火攻,經過州縣用夫至八百,死亡載道,未見其一試也。

且文燦辦賊之策曰「先撫後剿」。乃茶山不效,麻城又不效,第見招撫之旗絡繹於道。一遣使招賀一龍,而使者被殺;一遣使招李萬慶,而饋鹽椒運魚肉與通市,賊反因之焚掠,未見一賊歸順也。天下有如是撫法乎!其一切軍需,悉取於所歷之有司,名曰「借辦」,致城市空虛,孑遺盡絕。三月至麻城,民不堪淫掠,欲焚其署,始踉蹌而走。麻城,文燦婿家也,戚里如是,餘可知矣。三月在蘄水,其兵殺鄉民報捷,民家環哭,竟不敢治一兵。蘄水,文燦家園也,鄉里如是,餘可知矣。是以捷報日張,寇勢愈熾。十三家之賊蹂躪南陽、汝寧,如履無人之境。文燦駐宛、汝已久,調度不聞,天下有如是剿法乎!獻忠在穀城招納亡命,買馬置器,人人知其叵測。文燦顧欲借之為前茅,遣官調之,非惟不應,復留解餉之官,求總兵湖廣。今已造浮橋跨漢水矣。文燦前既誇張而敘功,後復掩匿而不報,可不謂之欺君乎!以總理之大柄畀之顛蹶之耄夫,臣不知其可也。

帝不納。逮至,下獄。

初,應桂貽書文燦,言獻忠必反,可先未發圖之。其書為獻忠邏者所得,獻忠騰牒鄖陽巡撫戴東旻,言「撫軍欲殺我」,東旻聞之文燦,文燦再糾應桂。應桂再疏辨,帝亦不納。應桂竟遣戍。無何,獻忠果反,廷臣交章薦應桂。

十六年,起應桂兵部右侍郎。十月,潼關陷,帝召問大臣。陳演言:「賊入關中,必戀子女玉帛,猶虎入陷阱。」應桂叱之曰:「壯士健馬咸出關西。賊得之,必長驅橫行,大臣安得面謾!」演股栗失色。十一月,督師孫傳庭戰歿,命應桂兼右僉都御史往代之。應桂以無兵無餉,入見帝而泣。帝但遣京軍千人護行,給御用銀萬兩、銀花四百、銀牌二百、蟒幣二百、雜幣倍之,為軍前賞功之用而已。應桂既受命,日夜悲疑,將至山西,則偽官充斥,逡巡不得前。帝責以逗遛,奪職,命新擢陜西巡撫李化熙代之,化熙亦不能進也。未幾,京師陷。應桂家居不出。久之,死於難。

高斗樞,字象先,鄞人。崇禎元年進士。授刑部主事。坐議巡撫耿如杞獄,與同列四人下詔獄。尋復官,進員外郎。

五年,遷荊州知府。久之,擢長沙兵備副使。楚郡之在湖北者,盡罹賊禍,勢且及湖南,臨、藍、湖、湘間土寇蜂起。長沙止老弱衛卒五百,又遣二百戍攸縣,城庫雉堞盡圮。斗樞至,建飛樓四十,大修守具。臨、藍賊艘二百餘,由衡、湘抵城下,相拒十餘日乃卻去,轉攻袁州。遣都司陳上才躡其後,賊亦解去。尋擊殺亂賊劉高峰等,撫定餘眾。詔錄其功。巡撫陳睿謨大征臨、藍寇,斗樞當南面,大小十餘戰,賊盡平。詔賚銀幣。

十四年六月進按察使,移守鄖陽。鄖被寇且十載,屬邑有六,居民不四千,數百里荊榛。撫治王永祚以襄陽急,移師鎮之。斗樞至甫六日,張獻忠自陜引而東。斗樞與知府徐啟元遣遊擊王光恩及弟光興分扼之,戰頻捷,賊不敢犯。光恩者,均州降渠小秦王也。初與張獻忠、羅汝才輩為賊,獻忠、汝才降而復叛,均州五營懼見討自疑。又以獻忠強,慮為所併,光恩斂眾據要害以拒獻忠。居久之,乃有稍稍颺去者,光恩亦去,已而復降。光恩善用其下,下亦樂為之用。斗樞察其誠,招入郡守。當是時,斗樞、啟元善謀,光恩善戰,鄖城危而復全。

十五年冬,李自成陷襄陽、均州,攻鄖陽四日而去。明年春,復來攻,十餘日不克,乃退屯楊溪。五月,斗樞召遊擊劉調元入城,旬日間殺賊三千餘。自成將來攻,卒不克而去。乃令光恩復均州,調元下光化,躬率將士復穀城。將襲襄陽,聞孫傳庭敗,旋師,均州復為賊有。

十七年正月,自成遣將路應標等以三萬人攻鄖。斗樞遣人入均州,燒其蓄積,賊乏食而退。當是時,湖南、北十四郡皆陷,獨鄖在。自十五年冬撫治王永祚被逮,連命李乾德、郭景昌代之,路絕不能至,中朝謂鄖已陷,不復設撫治。十六年夏,斗樞上請兵疏,始知鄖存,眾議即任斗樞。而陳演與之有隙,乃擢啟元右僉都御史任之,加斗樞太僕少卿,路阻亦不能達。是年二月,朝議設漢中巡撫,兼督川北軍務,擢斗樞右副都御史以往,朝命亦不達。至三月始聞太僕之命,即以軍事付啟元。七月而北都變聞,並聞漢中之命,地已失,不可往。

福王立,移斗樞巡撫湖廣,代何騰蛟。復以道路不通,改用王驥,斗樞皆不聞也。國變後數年卒。啟元、光恩亦皆以功名終。

張任學,安岳人,天啟五年進士。授太原知縣,以才調榆次。

崇禎四年,舉治行卓異入為御史。陳蜀中私稅、催科、訟獄三大苦,帝為飭行。出視兩浙鹽法,數條奏利弊。八年,流賊陷鳳陽,詔逮巡按吳振纓,命任學往代。還朝,復按河南,監軍討賊。時群盜縱橫,而諸將縮朒不敢擊。任學慨然曰:「事不辭難,臣職也。賊勢如此,我輩可雍容坐鎮耶!」十一年二月,遂上疏極詆諸將。請易武階,親執干戈,為國平賊。帝壯之,下吏、兵二部及都察院議。諸臣以文吏無改武職者,請仍以監軍御史兼總兵事。帝不從,命授署都督僉事,為河南總兵官。河南舊無總兵,左良玉、陳永福並以客兵備援剿,至是大將特設,而麾下無一官,兵部乃以署鎮許定國兵授之,使參將羅岱為中軍。岱,健將,屢著戰功,任學倚以自強。時熊文燦專主撫,劉國能、張獻忠俱降,羅汝才、馬進忠、李萬慶等躪中原如故。河南人據塢壁自保者數十,賊悉摧破之,踞息縣、光州,磔人投汝水,水為赤,任學不能大創也。進忠勢衰,佯求撫,文燦及巡撫常道立許之,乘間逸去。事聞,任學與文燦、道立並鐫秩。

七月,任學督岱等赴羅山,合左良玉軍擊汝才、萬慶及紫微星、順義王,大敗之,追奔五十里,斬首一千四百有奇,獲黑虎狼、滿天星,賊奔遂平。九月,進忠寇開封,至瓦子坡。岱奮擊,賊盡棄輜重遁入大隗山,獲其妻子。

其冬,京師戒嚴,任學入衛,道謁文燦,言:「獻忠狼子野心,終為國患,我以勤王為名,出其不意,可立縛也。」文燦不能用。抵畿南,有詔卻還。巡撫道立調良玉兵於陜州,賊乘盧氏虛,遁入內鄉、淅川,為文燦所劾。明年除道立名,任學亦鐫一秩。遊擊宋懷智、都司孔道興再破賊陳州,部將王應龍、尤之龍等破賊襄城,五戰皆勝。副將岱與應龍、懷智等復破賊葉縣,十日奏八捷,帝詔所司核實。已,又挫賊裕州。而是時總兵孫應元、黃得功統京軍討賊,屢奏大捷。凱旋錄功,任學亦敘復二秩。

尋與左良玉、陳洪範蹙李萬慶於內鄉。萬慶方降,獻忠已叛,文燦盡調河南軍援剿,獨任學留汝南。川貴總督李若星論文燦主撫之謬,請復任學原官,攝行大將,督察軍事。不從。七月,獻忠合汝才自房縣西走,岱偕良玉追之。良玉令岱為前鋒,己隨其後。至羅犬英山,軍乏食。賊伏兵要害,岱與副將劉元捷鼓勇直上,伏四起。岱馬足掛於藤,抽刀斷之,蹶而復進,乃棄馬步斗,久之矢盡,陷於賊,良玉軍亦大敗。事聞,任學坐褫職。十五年,言官請起廢,任學與焉,未及用而卒。

贊曰:流賊之肆毒也,禍始於楊鶴,成於陳奇瑜,而熾於熊文燦、丁啟睿。然練國事、鄭崇儉先罹其罰,而邵捷春、餘應桂亦或死或戍。疆場則剿撫乖方,廟堂則賞罰不當,僨師玩寇,賊勢日張,謂非人謀不臧實使之然乎!


\end{pinyinscope}