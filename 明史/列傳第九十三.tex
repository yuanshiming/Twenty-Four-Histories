\article{列傳第九十三}

\begin{pinyinscope}
硃紈張經李天寵周珫楊宜彭黯等胡宗憲阮鶚宗禮曹邦輔任環吳成器李遂弟逢進唐順之子鶴征

硃紈,字子純,長洲人。正德十六年進士。除景州知州,調開州。嘉靖初,遷南京刑部員外郎。歷四川兵備副使。與副總兵何卿共平深溝諸砦番。五遷至廣東左布政使。二十五年擢右副都御史,巡撫南、贛。明年七月,倭寇起,改提督浙、閩海防軍務,巡撫浙江。

初,明祖定制,片板不許入海。承平久,奸民闌出入,勾倭人及佛郎機諸國入互市。閩人李光頭、歙人許棟踞寧波之雙嶼為之主,司其質契。勢家護持之,漳、泉為多,或與通婚姻。假濟渡為名,造雙桅大船,運載違禁物,將吏不敢詰也。或負其直,棟等即誘之攻剽。負直者脅將吏捕逐之,洩師期令去,期他日償。他日至,負如初。倭大怨恨,益與棟等合。而浙、閩海防久墜,戰船、哨船十存一二,漳、泉巡檢司弓兵舊額二千五百餘,僅存千人。剽掠輒得志,益無所忌,來者接踵。

紈巡海道,採僉事項高及士民言,謂不革渡船則海道不可清,不嚴保甲則海防不可復,上疏具列其狀。於是革渡船,嚴保甲,搜捕奸民。閩人資衣食於海,驟失重利,雖士大夫家亦不便也,欲沮壞之。紈討平覆鼎山賊。明年將進攻雙嶼,使副使柯喬、都指揮黎秀分駐漳、泉、福寧,遏賊奔逸,使都司盧鏜將福清兵由海門進。而日本貢使周良違舊約,以六百人先期至。紈奉詔便宜處分。度不可卻,乃要良自請,後不為例。錄其船,延良入寧波賓館。奸民投書激變,紈防範密,計不得行。夏四月,鏜遇賊於九山洋,俘日本國人稽天,許棟亦就擒。棟黨汪直等收餘眾遁,鏜築塞雙嶼而還。番舶後至者不得入,分泊南麂、礁門、青山、下八諸島。

勢家既失利,則宣言被擒者皆良民,非賊黨,用搖惑人心。又挾制有司,以脅從被擄予輕比,重者引強盜拒捕律。紈上疏曰:「今海禁分明,不知何由被擄,何由協從?若以入番導寇為強盜,海洋敵對為拒捕,臣之愚暗,實所未解。」遂以便宜行戮。

紈執法既堅,勢家皆懼。貢使周良安插已定,閩人林懋和為主客司,宣言宜發回其使。紈以中國制馭諸番,宜守大信,疏爭之強。且曰:「去外國盜易,去中國盜難。去中國瀕海之盜猶易,去中國衣寇之盜尤難。」閩、浙人益恨之,竟勒周良還泊海嶼,以俟貢期。吏部用御史閩人周亮及給事中葉鏜言,奏改紈巡視,以殺其權。紈憤,又明年春上疏言:「臣整頓海防,稍有次第,亮欲侵削臣權,致屬吏不肯用命。」既又陳明國是、正憲體、定紀綱、扼要害、除禍本、重斷決六事,語多憤激。中朝士大夫先入浙、閩人言,亦有不悅紈者矣。

紈前討溫、盤、南麂諸賊,連戰三月,大破之,還平處州礦盜。其年三月,佛郎機國人行劫至詔安。紈擊擒其渠李光頭等九十六人,復以便宜戮之。具狀聞,語復侵諸勢家。御史陳九德遂劾紈擅殺。落紈職,命兵科都給事杜汝禎按問。紈聞之,慷慨流涕曰:「吾貧且病,又負氣,不任對簿。縱天子不欲死我,閩、浙人必殺我。吾死,自決之,不須人也。」制壙志,作絕命詞,仰藥死。二十九年,給事汝禎、巡按御史陳宗夔還,稱奸民鬻販拒捕,無僭號流劫事,坐紈擅殺。詔逮紈,紈已前死。柯喬、盧鏜等並論重闢。

紈清強峭直,勇於任事。欲為國家杜亂源,乃為勢家構陷,朝野太息。自紈死,罷巡視大臣不設,中外搖手不敢言海禁事。浙中衛所四十一,戰船四百三十九,尺籍盡耗。紈招福清捕盜船四十餘,分布海道,在臺州海門衛者十有四,為黃巖外障,副使丁湛盡散遣之,撤備馳禁。未幾,海寇大作,毒東南者十餘年。

張經,字廷彞,侯官人。初冒蔡姓,久之乃復。正德十二年進士。除嘉興知縣。嘉靖四年召為吏科給事中,歷戶科都給事中,數有論劾。言官指為張、桂黨,吏部言經行修,不問。擢太僕少卿,歷右副都御史,協理院事。十六年進兵部右侍郎,總督兩廣軍務。斷藤峽賊侯公丁據弩灘為亂。經與御史鄒堯臣等定計,以軍事屬副使翁萬達,誘執公丁。參議田汝成請乘勢進討。命副總兵張經將三萬五千人為左軍,萬達監之,指揮王良輔等六將分六道會南寧;都指揮高乾將萬六千人為右軍,副使梁廷振監之,指揮馬文傑等四將分四道會賓州,抵賊巢夾擊。賊奔林峒而東。良輔等邀之,賊中斷,復西奔,斬首千二百級。其東者遁入羅運山,萬達等移師攻之。檄右軍沿江而東,繞出其背。賊刊巨木塞隘口,布蒺藜菰簽,伏機弩毒鏢,懸石樹杪,急則撼其樹,石皆墜,官軍並以計破之。右軍愆期,田州土酋盧受乃縱賊去。俘其眾四百五十,招降者二千九百有奇。土人言,祖父居羅運八世矣,未聞官軍涉茲土也。捷聞,進經左侍郎,加秩一級。

尋與毛伯溫定計,撫定安南,再進右都御史。平思恩九土司及瓊州黎,進兵部尚書。副使張瑤等討馬平瑤屢敗,帝罪瑤等而宥經。給事中周怡劾經,經乞罷,不允。以憂歸。服闋,起三邊總督。給事中劉起宗言經在兩廣克餉銀,寢前命。

三十二年起南京戶部尚書,就改兵部。明年五月,朝議以倭寇猖獗,設總督大臣。命經解部務,總督江南、江北、浙江、山東、福建、湖廣諸軍,便宜行事。經征兩廣狼土兵聽用。其年十一月,用兵科言改經右都御史兼兵部右侍郎,專辦討賊。倭二萬餘據柘林川沙窪,其黨方踵至。經日選將練兵,為搗巢計。以江、浙、山東兵屢敗,欲俟狼土兵至用之。明年三月,田州瓦氏兵先至,欲速戰,經不可。東蘭諸兵繼至。經以瓦氏兵隸總兵官俞大猷,以東蘭、那地、南丹兵隸遊擊鄒繼芳,以歸順及思恩、東莞兵隸參將湯克寬,分屯金山衛、閔港、乍浦,掎賊三面,以待永順、保靖兵之集。會侍郎趙文華以祭海至,與浙江巡按胡宗憲比,屢趨經進兵。經曰:「賊狡且眾,待永、保兵至夾攻,庶萬全。」文華再三言,經守便宜不聽。文華密疏經糜餉殃民,畏賊失機,欲俟倭飽颺,剿餘寇報功,宜亟治,以紓東南大禍。帝問嚴嵩,嵩對如文華指,且謂蘇、松人怨經。帝怒,即下詔逮經。三十四年五月也。

方文華拜疏,永、保兵已至,其日即有石塘灣之捷。至五月朔,倭突嘉興,經遣參將盧鏜督保靖兵援,以大猷督永順兵由泖湖趨平望,以克寬引舟師由中路擊之,合戰於王江涇,斬賊首一千九百餘級,焚溺死者甚眾。自軍興來稱戰功第一。給事中李用敬、閻望雲等言:「王師大捷,倭奪氣,不宜易帥。」帝大怒曰:「經欺誕不忠,聞文華劾,方一戰。用敬等黨奸。杖於廷,人五十,斥為民。」已而帝疑之,以問嵩。嵩言:「徐階、李本江、浙人,皆言經養寇不戰。文華、宗憲合謀進剿,經冒以為功。」因極言二人忠。帝深入其言。經既至,備言進兵始末,且言:「任總督半載,前後俘斬五千,乞賜原宥。」帝終不納,論死繫獄。其年十月,與巡撫李天寵俱斬。天下冤之。

天寵,孟津人。由御史遷徐州兵備副使,卻倭通州、如皋。三十三年六月擢右僉都御史,代王忬巡撫浙江。倭掠紹興,殲焉,賚銀幣。頃之,賊犯嘉善,圍嘉興,劫秀水、歸安,副使陳宗夔戰不利,百戶賴榮華中炮死,嘉善知縣鄧植棄城走。入城大掠。賊復陷崇德,攻德清,殺裨將梁鄂等。文華謗天寵嗜酒廢事,帝遂除天寵名,而擢宗憲以代。未幾,御史棄恩以倭躪北新關,劾天寵,宗憲亦言其縱寇。帝怒,逮下獄,遂與經同日死。

代經者應城周珫、衡水楊宜。節制不行,狼土兵肆焚掠。東南民既苦倭,復苦兵矣。隆慶初,復經官,謚襄愍。

珫為戶科給事中,坐諫世宗南幸,謫鎮遠典史。累官右僉都御史,巡撫蘇、松諸府。疏陳禦倭有十難,有三策。經既得禍,即擢珫兵部右侍郎代之,無所展。會宗憲已代天寵,因欲奪珫位。文華遂劾珫,薦宗憲。帝為奪珫俸,尋勒為民。珫在官僅三十有四日,而楊宜代。

宜撫河南,平劇賊師尚詔。遷南京戶部右侍郎,未幾代珫。時倭勢猶盛。宜為總督,而文華督察軍務,威出宜上。易置文武大吏,惟其愛憎。宜懲經、天寵禍,曲意奉之。文華視之蔑如也。倭據陶宅,官軍久無功,文華遂劾宜。宜以狼兵徒剽掠不可用,請募江、浙義勇,山東箭手,益調江、浙、福建、湖廣漕卒,河南毛兵。比客兵大集,宜不能馭。川兵與山東兵私鬥,幾殺參將。酉陽兵潰於高橋,奪舟徑歸蘇州。明年正月,文華還朝,請罷宜,以宗憲代。會御史邵惟中上失事狀,遂奪宜職閒住。宜在事僅踰半歲,以諂事文華,故得禍輕。

倭之躪蘇、松也,起嘉靖三十二年,訖三十九年,其間為巡撫者十人。安福彭黯,遷南京工部尚書。畏賊,不俟代去,下獄除名。黃岡方任、上虞陳洙皆未抵任。任丁憂,洙以才不足任別用。而代以鄞人屠大山,使提督軍務。蘇、松巡撫之兼督軍務,自大山始。閱半歲,以疾免。尋坐失事下詔獄,為民。繼之者珫。繼珫者曹邦輔。以文華譖,下詔獄,謫戍。次眉州張景賢,以考察奪職。次盩厔趙忻,坐金山軍變,下獄貶官。次江陵陳錠,數月罷去。次翁大立。當大立時,倭患已息,而坐惡少年鼓噪為亂,竟罷職。無一不得罪去者。

胡宗憲,字汝貞,績溪人。嘉靖十七年進士。歷知益都、餘姚二縣。擢御史,巡按宣、大。詔徙大同左衛軍於陽和、獨石,卒聚而嘩。宗憲單騎慰諭,許勿徙,乃定。

三十三年,出按浙江。時歙人汪直據五島煽諸倭入寇,而徐海、陳東、麻葉等巢柘林、乍浦、川沙窪,日擾郡邑。帝命張經為總督,李天寵撫浙江,又命侍郎趙文華督察軍務。文華恃嚴嵩內援,恣甚。經、天寵不附也,獨宗憲附之。文華大悅,因相與力排二人。倭寇嘉興,守憲中以毒酒,死數百人。及經破王江涇,宗憲與有力。文華盡掩經功歸宗憲,經遂得罪。尋又陷天寵,即超擢宗憲右僉都御史代之。時柘林諸倭移屯陶宅,勢稍殺。會蘇、松巡撫曹邦輔殲倭滸墅,文華欲攘功不得,大恨,遂進剿陶宅殘寇。宗憲與共,將銳卒四千,營磚橋,約邦輔夾擊。倭殊死戰,宗憲兵死者千餘。文華令副使劉燾攻之,復大敗。而倭犯浙東諸州縣,殺文武吏甚眾。宗憲乃與文華定招撫計。文華還朝,盛毀總督楊宜,而薦宗憲,遂以為兵部右侍郎代宜。

初,宗憲令客蔣洲、陳可願諭日本國王,遇汪直養子滶於五島,邀使見直。直初誘倭入犯,倭大獲利,各島由此日至。既而多殺傷,有全島無一歸者,死者家怨直。直乃與滶及葉碧川、王清溪、謝和等據五島自保。島人呼為老船主。宗憲與直同鄉里,欲招致之,釋直母妻於金華獄,資給甚厚。洲等諭宗憲指。直心動,又知母妻無恙,大喜曰:「俞大猷絕我歸路,故至此。若貸罪許市,吾亦欲歸耳。但日本國王已死,各島不相攝,須次第諭之。」因留洲而遣滶等護可願歸。宗憲厚遇滶,令立功。滶遂破倭舟山,再破之列表。宗憲請於朝,賜滶等金幣,縱之歸。滶大喜,以徐海入犯來告。亡何,海果引大隅、薩摩二島倭分掠瓜洲、上海、慈谿,自引萬餘人攻乍浦,陳東、麻葉與俱。宗憲壁塘棲,與巡撫阮鶚相犄角。會海趨皂林,鶚遣游擊宗禮擊海於崇德三里橋,三戰三捷。既而敗死,鶚走桐鄉。

禮,常熟人,由世千戶歷署都督僉事。驍健敢戰。練卒三千連破倭,至是敗歿。贈都督同知,謚忠壯,賜祠皂林。

鶚既入桐鄉,賊乘勝圍之。宗憲計曰:「與鶚俱陷無益也。」遂還杭州,遣指揮夏正等持滶書要海降。海驚曰:「老船主亦降乎?」時海病創,意頗動,因曰:「兵三路進,不由我一人也。」正曰:「陳東已他有約,所慮獨公耳。」海遂疑東。而東知海營有宗憲使者,大驚,由是有隙。正乘間說下海。海遣使來謝,索財物,宗憲報如其請。海乃歸俘二百人,解桐鄉圍。東留攻一日,亦去,復巢乍浦。鶚知不能當海,乃東渡錢塘禦他賊。

初,海入犯,焚其舟,示士卒無還心。至是,宗憲使人語海曰:「若已內附,而吳淞江方有賊,何不擊之以立功?且掠其舸,為緩急計。」海以為然,逆擊之朱涇,斬三十餘級。宗憲令大猷潛焚其舟。海心怖,以弟洪來質,獻所戴飛魚冠、堅甲、名劍及他玩好。宗憲因厚遇洪,諭海縛陳東、麻葉,許以世爵。海果縛葉以獻。宗憲解其縛,令以書致東圖海,而陰泄其書於海。海怒。海妾受宗憲賂,亦說海。於是海復以計縛東來獻,帥其眾五百人去乍浦,別營梁莊。官軍焚乍浦巢,斬首三百餘級,焚溺死稱是。海遂刻日請降,先期猝至,留甲士平湖城外,率酋長百餘,胄而入。文華等懼,欲勿許,宗憲強許之。海叩首伏罪,宗憲摩海頂,慰諭之。海自擇沈莊屯其眾。沈莊者東西各一,以河為塹。宗憲居海東莊,以西莊處東黨。令東致書其黨曰:「督府檄海,夕擒若屬矣。」東黨懼,乘夜將攻海。海挾兩妾走,間道中槊。明日,官軍圍之,海投水死。會盧鏜亦擒辛五郎至。辛五郎者,大隅島主弟也。遂俘洪、東、葉、五郎及海首獻京師。帝大悅,行告廟禮,加宗憲右都御史,賜金幣加等。海餘黨奔舟山。宗憲令俞大猷雪夜焚其柵,盡死。兩浙倭漸平。

三十六年正月,阮鶚改撫福建,即命宗憲兼浙江巡撫事。蔣洲在倭中,諭山口、豐後二島主源義長、源義鎮還被掠人口,具方物入貢。宗憲以聞。詔厚賚其使,遣還。至十月,復遣夷目善妙等隨汪直來市,至岑港泊焉。浙人聞直以倭船至,大驚。巡按御史王本固亦言不便,朝臣謂宗憲且釀東南大禍。直遣滶詣宗憲曰:「我等奉詔來,將息兵安境。謂宜使者遠迎,宴犒交至。今盛陳軍容,禁舟楫往來,公紿我耶?」宗憲解諭至再,直不信。乃令其子以書招之,直曰:「兒何愚也。汝父在,厚汝。父來,闔門死矣。」因要一貴官為質。宗憲立遣夏正偕滶往。宗憲嘗預為赦直疏,引滶入臥內,陰窺之。滶語直,疑稍解,乃偕碧川、清溪入謁。宗憲慰藉之甚至,令至杭見本固。本固下直等於獄。宗憲疏請曲貸直死,俾戍海上,繫番夷心。本固爭之彊,而外議疑宗憲納賊賂。宗憲懼,易詞以聞。直論死,碧川、清溪戍邊。滶與謝和遂支解夏正,柵舟山,阻岑港而守。官軍四面圍之,賊死鬥,多陷歿者。

至明年春,新倭復大至,嚴旨責宗憲。宗憲懼得罪,上疏陳戰功,謂賊可指日滅。所司論其欺誕。帝怒,盡奪諸將大猷等職,切讓宗憲,令剋期平賊。時趙文華已得罪死,宗憲失內援,見寇患未已,思自媚於上,會得白鹿於舟山,獻之。帝大悅,行告廟禮,厚賚銀幣。未幾,復以白鹿獻。帝益大喜,告謝玄極寶殿及太廟,百官稱賀,加宗憲秩。既而岑港之賊徙巢柯梅,官軍屢攻不能克。御史李瑚劾宗憲誘汪直啟釁。本固及給事中劉堯誨亦劾其老師縱寇,請追奪功賞。帝命廷議之,咸言宗憲功多,宜勿罷。帝嘉其擒直功,令居職如故。

賊之徙柯梅也,造巨艦為遁計。及艦成,宗憲利其去,不擊。賊揚帆泊浯嶼,縱掠閩海州縣。閩人大噪,謂宗憲嫁禍。御史瑚再劾宗憲三大罪。瑚與大猷皆閩人,宗憲疑大猷漏言,劾大猷不力擊,大猷遂被逮。

當是時,江北、福建、廣東皆中倭。宗憲雖盡督東南數十府,道遠,但遙領而已,不能遍經畫。然小勝,輒論功受賚無虛月。即敗衄,不與其罪。三十八年,賊大掠溫、台,別部復寇濱海諸縣。給事中羅嘉賓、御史龐尚鵬奉詔勘之。言宗憲養寇,當置重典,帝不問。明年,論平汪直功,加太子太保。

宗憲多權術,喜功名,因文華結嚴嵩父子,歲遺金帛子女珍奇淫巧無數。文華死,宗憲結嵩益厚,威權震東南。性善賓客,招致東南士大夫預謀議,名用是起。至技術雜流,豢養皆有恩,能得其力。然創編提均徭之法,加賦額外,民為困敝,而所侵官帑、斂富人財物亦不貲。嘉賓、尚鵬還,上宗憲侵帑狀,計三萬三千,他冊籍沉滅。宗憲自辯,言:「臣為國除賊,用間用餌,非小惠不成大謀。」帝以為然,更慰諭之。尋上疏,請得節制巡撫及操江都御史,如三邊故事。帝即晉兵部尚書,如其請。復獻白龜二、五色芝五。帝為謝玄告廟如前,賚宗憲加等。

明年,江西盜起,又兼制江西。未至,總兵官戚繼光已平賊。九月奏言:「賊屢犯寧、台、溫,我師前後俘斬一千四百有奇,賊悉蕩平。」帝悅,加少保。兩廣平巨盜張璉,亦論宗憲功。時嵩已敗,大學士徐階曰:「兩廣平賊,浙何與焉?」僅賜銀幣。未幾,南京給事中陸鳳儀劾其黨嚴嵩及奸欺貪淫十大罪,得旨逮問。及宗憲至,帝曰:「宗憲非嵩黨。朕拔用八九年,人無言者。自累獻祥瑞,為群邪所疾。且初議獲直予五等封,今若加罪,後誰為我任事者?其釋令閒住。」

久之,以萬壽節獻秘術十四。帝大悅,將復用矣。會御史汪汝正籍羅龍文家,上宗憲手書,乃被劾時自擬旨授龍文以達世蕃者,遂逮下獄。宗憲自敘平賊功,言以獻瑞得罪言官,且訐汝正受贓事。帝終憐之,並下汝正獄。宗憲竟瘐死,汝正得釋。萬曆初,復官,謚襄懋。

阮鶚者,桐城人,官浙江提學副使。時倭薄杭州,鄉民避難入城者,有司拒不許入。鶚手劍開門納之,全活甚眾。以附文華、宗憲得超擢右僉都御史,代宗憲巡撫浙江。又以文華言,特設福建巡撫,即以命鶚。初在浙不主撫,自桐鄉被圍,懼甚。寇犯福州,賂以羅綺、金花及庫銀數萬,又遺巨艦六艘,俾載以走。不能措一籌,而斂括民財動千萬計,帷帟盤盂率以錦綺金銀為之。御史宋儀望等交章劾,逮下刑部。嚴嵩為屬法司,僅黜為民。所侵餉數,浮於宗憲,追還之官。

曹邦輔,字子忠,定陶人。嘉靖十一年進士。歷知元城、南和,以廉乾稱。擢御史,巡視河東鹽政。巡按陜西,劾總督張珩等冒功,皆謫戍。出為湖廣副使,補河南。

柘城賊師尚詔反,陷歸德。檢校董綸率民兵巷戰,手刃數賊,與其妻賈氏俱死之。又陷柘城,劫舉人陳聞詩為帥。不聽,斬從者脅之。聞詩紿曰:「必欲我行,毋殺人,毋縱火。」賊許諾,擁上馬。不食三日,至鹿邑自縊。賊圍太康,都指揮尚允紹與戰鄢陵,敗績。允紹復擊賊於霍山,賊圍之,兵無敢進。邦輔斬最後者,士卒競進。賊大潰,擒斬六百餘人。尚詔走莘縣,被擒。賊起四十餘日,破府一,縣八,殺戳十餘萬。邦輔亟戰,殲之。詔賚銀幣,擢山西右參政,遷浙江按察使。

三十四年拜右僉都御史,巡撫應天。倭聚柘林。其黨自紹興竄,轉掠杭、嚴、徽、寧、太平,遂犯南京,破溧水,抵宜興。為官軍所迫,奔滸墅。副總兵俞大猷、副使任環數邀擊之,而柘林餘賊已進據陶宅。邦輔督副使王崇古圍之,僉事董邦政、把總婁宇協剿。賊走太湖,追及之,盡殲其眾。副將何卿師潰,邦輔援之。以火器破賊舟,前後俘斬六百餘人。侍郎趙文華欲攘其功,邦輔捷書先奏,文華大恨。既而與浙江巡按御史胡宗憲會邦輔攻陶宅賊,諸營皆潰。賊退,邦輔進攻之,復敗,坐奪俸。文華奏邦輔避難擊易,致師後期,總督楊宜亦奏邦輔故違節制。給事中夏栻、孫浚爭之,得無罪。文華還京,奏餘賊且盡,而巡按御史周如斗又奏失事狀,帝頗疑文華。文華因言:「賊易滅,督撫非人,致敗。臣昔論邦輔,栻殼、浚遂媒孽臣。東南塗炭何時解?」乃逮繫邦輔,謫戍朔州。

隆慶元年,楊博為吏部,起邦輔左副都御史,協理院事。進兵部右侍郎,理戎政。尋以左侍郎兼右僉都御史,總督薊、遼、保定軍務。言修治邊牆非上策,宜急練兵;兵練而後邊事可議。以給事中張鹵言,召為右都御史,掌院事。帝以京營事重,更協理為閱視,令付大臣知兵者,遂以左都御史召還,任之。已,從恭順侯吳繼爵言,復改閱視為提督。未幾,轉南京戶部尚書。奏督倉主事張振選不奉約束。吏部因言:「往昔執政喜人悅己,屬吏恃為奧援。構陷堂上官,至屈體降意,倒置名分。在外巡按御史亦曲庇進士推知,監司賢不肖出其口吻。害政無甚於此。」穆宗深然其言,為黜振選,飭內外諸司,然迄不能變。邦輔累乞骸骨,不聽。萬曆元年給由赴闕,復以病求去,且言辛愛有窺覦志,宜慎防之。遂致仕去。居三年,卒。贈太子少保。

邦輔廉峻。自吳中被逮時,有司上所儲俸錢,揮之去。歷官四十年,家無餘貲。撫、按奏其狀,詔遣右評事劉叔龍為營墳墓。

任環,字應乾,長治人。嘉靖二十三年進士。歷知黃平、沙河、滑縣,並有能名。遷蘇州同知。倭患起,長吏不嫻兵革。環性慷慨,獨以身任之。三十二年閏三月禦賊寶山洋,小校張治戰死。環奮前搏賊,相持數日,賊遁去。尋犯太倉,環馳赴之。嘗遇賊,短兵接,身被三創幾殆。宰夫捍環出,死之,賊亦引去。已而復至,裹瘡出海擊之。怒濤作,操舟者失色。環意氣彌厲,竟敗賊,俘斬百餘。復連戰陰沙、寶山、南沙,皆捷。擢按察僉事,整飭蘇、松二府兵備。倭剽掠厭,悉歸,惟南沙三百人舟壞不能去,環與總兵官湯克寬列兵守之。數月,賊大至,與舊倭合,掠華亭、上海。環等被劾,得宥。踰年,賊犯蘇州。城閉,鄉民繞城號。環盡納之,全活數萬計。副將解明道擊退賊,論前後功,進環右參政。賊掠常熟,環率知縣王鈇破其巢,焚舟二十七。未幾,賊掠陸涇壩,都督周于德敗績。環偕總兵官俞大猷擊敗之,焚舟三十餘。賊犯吳江,環、大猷擊敗之鶯脰湖,賊奔嘉興。頃之,三板沙賊奪民舟出海,環、大猷擊敗之馬蹟山。其別部屯嘉定者,火爇之,盡死。論功,蔭一子副千戶。母憂奪哀。賊屯新場,環與都司李經等率永順、保靖兵攻之,中伏,保靖土舍、彭翅等皆死,環停俸戴罪。賊平,乞終制,許之。踰二年卒、年四十。給事中徐師曾頌其功,詔贈光祿卿,再蔭一子副千戶,建祠蘇州,春秋致祭。

環在行間,與士卒同寢食,所得賜予悉分給之。軍事急,終夜露宿,或數日絕餐。嘗書姓名於肢體曰:「戰死,分也。先人遺體,他日或收葬。」將士皆感激,故所向有功。

時休寧吳成器由小吏為會稽典史。倭三百餘劫會稽,為官軍所逐,走登龕山。成器遮擊,盡殪之。未幾,又破賊曹娥江,擢浙江布政司經歷。遭喪,總督胡宗憲奏留之。擢紹興通判。論功,進秩二級。成器與賊大小數十戰皆捷。身先士卒,進止有方略,所部無秋毫犯。士民率於其戰處立祠祀之。

李遂,字邦良,豐城人。弱冠,從歐陽德學。登嘉靖五年進士,授行人。歷刑部郎中。錦衣衛送盜十三人,遂惟抵一人罪,餘皆辨釋。東宮建,赦天下。遂請列「大禮」大獄諸臣於赦令中,尚書聶賢懼不敢,乃與同官盧蕙請於都御史王廷相,廷相從之。事雖報罷,議者嘉焉。俄調禮部,忤尚書夏言。因事劾之,下詔獄,謫湖州同知。三遷衢州知府,擢蘇、松兵備副使。屢遷廣東按察使。釋囚八百餘人。進山東右布政使。江洋多盜,遂遷右僉都御史提督操江。軍政明,盜不敢發。俺答犯京師,召遂督蘇州軍餉。未謝恩,請關防符驗用新銜。帝怒,削其籍。

三十六年,倭擾江北。廷議以督漕都御史兼理巡撫不暇辦寇,請特設巡撫,乃命遂以故官撫鳳陽四府。時淮、揚三中倭,歲復大水,且日役民輓大木輸京師。遂請餉增兵,恤民節用,次第畫戰守計。三十八年四月,倭數百艘寇海門。遂語諸將曰:「賊趨如皋,其眾必合。合則侵犯之路有三:由泰州逼天長、鳳、泗,陵寢驚矣;由黃橋逼瓜、儀,以搖南都,運道梗矣;若從富安沿海東至廟灣,則絕地也。」乃命副使劉景韶、遊擊丘陞扼如皋,而身馳泰州當其衝。時賊勢甚盛,副將鄧城之敗績,指揮張谷死焉。賊知如皋有備,將犯泰州,遂急檄景韶、陞遏賊。連戰丁堰、海安、通州,皆捷。賊沿海東掠,遂喜曰:「賊無能為矣。」令景韶、升尾之,而致賊於廟灣。復慮賊突淮安,乃夜半馳入城。賊尋至,遂督參將曹克新等禦之姚家蕩。通政唐順之、副總兵劉顯來援,賊大敗走,以餘眾保廟灣。景韶亦敗賊印莊,追奔至新河口,焚斬甚眾。廟灣賊據險不出,攻之月餘不克。遂令景韶塞塹、夷木壓壘陳,火焚其舟,賊乘夜雨潛遁。官軍據其巢,追奔至蝦子港,江北倭悉平。帝大喜,璽書獎勵。賊駐崇明三沙者,將犯揚州。景韶戰連勝,圍之劉莊。會劉顯來援,遂檄諸軍盡屬顯。攻破其巢,追奔白駒場,賊盡殄。時遂已遷南京兵部侍郎。論功,予一子官,賚銀幣。御史陳志勘上遂平倭功,前後二十餘戰,斬獲三千八百有奇。再予一子世千戶,增俸二級。

蒞南京甫數月,振武營軍變。振武營者,尚書張鏊募健兒以禦倭。素驕悍。舊制,南軍有妻者,月糧米一石;無者,減其四;春秋二仲月,米石折銀五錢。馬坤掌南戶部,奏減折色之一,督儲侍郎黃懋官又奏革募補者妻糧,諸軍大怨。代坤者蔡克廉方病,諸軍以歲饑求復折色故額於懋官。懋官不可,給餉又踰期。三十九年二月都肄日,振武卒鼓噪懋官署。懋官急招鏊及守備太監何綬、魏國公徐鵬舉、臨淮侯李庭竹及遂至,諸營軍已甲而入。予之銀,爭攫之。懋官見勢洶洶,越垣投吏舍,亂卒隨及。鵬舉、鏊慰解不聽,竟戕懋官,裸其屍於市。綬、鵬舉遣吏持黃紙,許給賞萬金,卒輒碎之。至許犒十萬金,乃稍定。明日,諸大臣集守備廳,亂卒亦集。遂大言曰:「黃侍郎自越墻死,諸軍特不當殘辱之。吾據實奏朝廷,不以叛相誣也。」因麾眾退,許復妻糧及故額,人畀之一金補折價,始散。遂乃托病閉閣,給免死券以慰安之,而密諭營將掩捕首惡二十五人,繫獄。詔追褫懋官及克廉職,罷綬、庭竹、鏊,任鵬舉如故,遂以功議擢。止誅叛卒三人,餘戍邊衛,而三人已前死。遂歎曰:「兵自此益驕矣。」未幾,江東代鏊為尚書。江北池河營卒以千戶吳欽革其幫丁,毆而縛之竿。幫丁者,操守卒給一丁,資往來費也。遂已召拜兵部左侍郎,以言官薦擢南京參贊尚書,鎮撫之。營卒惑妖僧繡頭,復倡訛言。遂捕斬繡頭,申嚴什伍,書其名籍、年貌,繫牌腰間,軍乃戢。既又奏調鎮武軍護陵寢,一日散千人,留都自是無患。越四年,以老致仕。

遂博學多智,長於用兵,然亦善逢迎。帝將重建三殿,遂奏五河縣泗水中湧大杉一,此川澤效靈,為聖主鼎新助,帝大喜。又進白兔,帝為遣官告廟。由此益眷遇。卒,贈太子太保,謚襄敏。

弟逢,字邦吉。由進士為吏科給事中。侍郎劉源清下吏,逢救之,並繫,得釋。進戶科左給事中。偕同官諫南巡,下詔獄,謫永福典史。終德安知府。遂子材,自有傳。

唐順之,字應德,武進人。祖貴,戶科給事中。父寶,永州知府。順之生有異廩。稍長,洽貫群籍。年三十二,舉嘉靖八年會試第一,改庶吉士。座主張璁疾翰林,出諸吉士為他曹,獨欲留順之。固辭,乃調兵部主事。引疾歸。久之,除吏部。十二年秋,詔選朝官為翰林,乃改順之編修,校累朝實錄。事將竣,復以疾告,璁持其疏不下。有言順之欲遠璁者,璁發怒,擬旨以吏部主事罷歸,永不復敘。至十八年選宮僚,乃起故官兼春坊右司諫。與羅洪先、趙時春請朝太子,復削籍歸。卜築陽羨山中,讀書十餘年。中外論薦,並報寢。

倭躪江南北。趙文華出視師,疏薦順之。起南京兵部主事。父憂未終,不果出。免喪,召為職方員外郎,進郎中。出核薊鎮兵籍,還奏缺伍三萬有奇,見兵亦不任戰,因條上便宜九事。總督王忬以下俱貶秩。

尋命往南畿、浙江視師,與胡宗憲協謀討賊。順之以禦賊上策,當截之海外,縱使登陸,則內地咸受禍。乃躬泛海,自江陰抵蛟門大洋,一晝夜行六七百里。從者咸驚嘔,順之意氣自如。倭泊崇明三沙,督舟師邀之海外。斬馘一百二十,沉其舟十三。擢太僕少卿。宗憲言順之權輕,乃加右通政。順之聞賊犯江北,急令總兵官盧鏜拒三沙,自率副總兵劉顯馳援,與鳳陽巡撫李遂大破之姚家蕩。賊窘,退巢廟灣。順之薄之,殺傷相當。遂欲列圍困賊,順之以為非計,麾兵薄其營,以火炮攻之,不能克。三沙又屢告急,順之乃復援三沙,督鏜、顯進擊,再失利。順之憤,親躍馬布陣。賊構高樓望官軍,見順之軍整,堅壁不出。顯請退師,順之不可,持刀直前,去賊營百餘步。鏜、顯懼失利,固要順之還。時盛暑,居海舟兩月,遂得疾,返太倉。李遂改官南京,即擢順之右僉都御史,代遂巡撫。順之疾甚,以兵事棘,不敢辭。渡江,賊已為遂等所滅。淮、揚適大饑,條上海防善後九事。三十九年春,汛期至。力疾泛海,度焦山,至通州卒,年五十四。訃聞,予祭葬。故事:四品但賜祭。順之以勞得賜葬云。

順之於學無所不窺。自天文、樂律、地理、兵法、弧矢、勾股、壬奇、禽乙,莫不究極原委。盡取古今載籍,剖裂補綴,區分部居,為《左》、《右》、《文》、《武》、《儒》、《稗》六《編》傳於世,學者不能測其奧也。為古文,洸洋紆折有大家風。生平苦節自厲,輟扉為床,不飾示因褥。又聞良知說於王畿,閉戶兀坐,匝月忘寢,多所自得。晚由文華薦,商出處於羅洪先。洪先曰:「向已隸名仕籍,此身非我有,安得侔處士?」順之遂出,然聞望頗由此損。崇禎中,追謚襄文。

子鶴徵,隆慶五年進士。歷官太常卿。亦以博學聞。

贊曰:朱紈欲嚴海禁,以絕盜源,其論甚正。顧指斥士大夫,令不能堪,卒為所齕齬,憤惋以死。氣質之為累,悲夫!當寇患孔熾,撲滅惟恐不盡,便宜行誅,自其職爾,而以為罪,則任法之過也。張經功不賞,而以冤戮,稔倭毒而助之攻,東南塗炭數十年。讒賊之罪,可勝誅哉!宗憲以奢黷蒙垢。然令徐海、汪直之徒不死,貽患更未可知矣。曹邦輔、任環戰功可紀,李遂、唐順之捍禦得宜。而邦輔之平師尚詔,李遂之靖亂卒,其功尤著。以其始終倭事,故並列焉。


\end{pinyinscope}