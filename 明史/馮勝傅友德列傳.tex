\article{馮勝、傅友德列傳}

\begin{pinyinscope}
馮勝,定遠人。初名國勝,又名宗異,最後名勝。生時黑氣滿室,經日不散。及長,雄勇多智略,與兄國用俱喜讀書,通兵法,元末結寨自保。太祖略地至妙山,國用偕勝來歸,甚見親信。太祖嘗從容詢天下大計,國用對曰:「金陵龍蟠虎踞,帝王之都,先拔之以為根本。然後四出征伐,倡仁義,收人心,勿貪子女玉帛,天下不足定也。」太祖大悅,俾居幕府,從克滁、和,戰三叉河、板門寨、雞籠山,皆有功。從渡江,取太平,遂命國用典親兵,委以心腹。太祖既擒陳野先,釋之,令招其部曲。國用策其必叛,不如弗遣。尋果叛,為其下所殺,其從子兆先復擁眾屯方山。蠻子海牙扼采石,國用與諸將攻破海牙水寨,又破擒兆先,盡降其眾三萬餘人。眾疑懼,太祖擇驍勇者五百人為親軍,宿衛帳中。悉屏舊人,獨留國用侍榻側,五百人者始安。即命國用將之,以攻集慶,爭效死先登。與諸將下鎮江、丹陽、寧國、泰興、宜興,從征金華,攻紹興,累擢親軍都指揮使。卒於軍,年三十六。太祖哭之慟。洪武三年追封郢國公,肖像功臣廟,位第八。

國用之卒,子誠幼,勝先已積功為元帥,遂命襲兄職,典親軍。

陳友諒逼龍灣。太祖禦之,戰石灰山。勝攻其中堅,大破之,又追究破之采石,遂復太平。從征友諒,破安慶水寨,長驅至江州,走友諒。進親軍都護。從解安豐圍,遷同知樞密院事。從戰鄱陽,下武昌,克廬州,移兵取江西諸路。與諸將收淮東,克海安壩,取泰州。徐達圍高郵未下,還師援宜興,以勝督軍。高郵守將詐降,勝令指揮康泰帥數百人先入城,敵閉門盡殺之。太祖怒,召勝決大杖十,令步詣高郵。勝慚憤,攻甚力。達亦自宜興還,益兵攻克之,遂取淮安。安豐破,擒吳將呂珍於舊館。下湖州,克平江,功次平章常遇春,再遷右都督。從大將軍達北征,下山東諸州郡。

洪武元年兼太子右詹事。坐小法貶一官,為都督同知。引兵溯河,取汴、洛,下陜州,趨潼關。守將宵遁,遂奪關,取華州。還汴,謁帝行在。授征虜右副將軍,留守汴梁。尋從大將軍征山西,由武陟取懷慶,踰太行,克碗子城,取澤、潞,擒元右丞賈成於猗氏。克平陽、絳州,擒元左丞田保保等,獲將士五百餘人。帝悅,詔右副將軍勝居常遇春下,偏將軍湯和居勝下,偏將軍楊璟居和下。

二年渡河趨陜西,克鳳翔。遂渡隴,取鞏昌,進逼臨洮,降李思齊。還從大將軍圍慶陽。擴廓遣將攻原州,為慶陽聲援。勝扼驛馬關敗其將,遂克慶陽,執張良臣。陜西悉平。

九月,帝召大將軍還,命勝駐慶陽,節制諸軍。勝以關陜既定,輒引兵還。帝怒,切責之。念其功大,赦勿治。而賞賚金幣,不能半大將軍。

明年正月復以右副將軍同大將軍出西安,搗定西,破擴廓帖木兒,獲士馬數萬。分兵自徽州南出一百八渡,徇略陽,擒元平章蔡琳,遂入沔州。遣別將自連雲棧取興元,移兵吐番,徵哨極於西北。凱旋,論功授開國輔運推誠宣力武臣、特進榮祿大夫、右柱國、同參軍國事,封宋國公,食祿三千石,予世券。誥詞謂勝兄弟親同骨肉,十餘年間,除肘腑之患,建爪牙之功,平定中原,佐成混一。所以稱揚之者甚至。五年,以勝宣力四方,與魏國公達、曹國公文忠各賜彤弓。

擴廓在和林,數擾邊。帝患之,大發兵三道出塞。命勝為征西將軍,帥副將軍陳德、傅友德等出西道,取甘肅。至蘭州,友德以驍騎前驅,再敗元兵,勝復敗之掃林山。至甘肅,元將上都驢迎降。至亦集乃路,守將卜顏帖木兒亦降。次別篤山,岐王朵兒只班遁去,追獲其平章長加奴等二十七人及馬駝牛羊十餘萬。是役也,大將軍達軍不利,左副將軍文忠殺傷相當,獨勝斬獲甚眾,全師而還。會有言其私匿駝馬者,賞不行。自後數出練兵臨清、北平,出大同征元遺眾,鎮陜西及河南。冊其女為周王妃。

久之,大將軍達、左副將軍文忠皆卒,而元太尉納哈出擁眾數十萬屯金山,數為遼東邊害。二十年命勝為征虜大將軍,穎國公傅友德、永昌侯藍玉為左右副將軍,帥南雄侯趙庸等以步騎二十萬征之。鄭國公常茂、曹國公李景隆、申國公鄧鎮等皆從。帝復遣故所獲納哈出部將乃剌吾者奉璽書往諭降。勝出松亭關,分築大寧、寬河、會州、富峪四城。駐大寧踰兩月,留兵五萬守之,而以全師壓金山。納哈出見乃剌吾驚曰:「爾尚存乎!」乃剌吾述帝恩德。納哈出喜,遣其左丞、探馬赤等獻馬,且覘勝軍。勝已深入,逾金山,至女直苦屯,降納哈出之將全國公觀童。大軍奄至,納哈出度不敵,因乃剌吾請降。勝使藍玉輕騎受之。玉飲納哈出酒,歡甚,解衣衣之。納哈出不肯服,顧左右咄咄語,謀遁去。勝之婿常茂在坐,遽起砍其臂。都督耿忠擁以見勝。納哈出將士妻子十餘萬屯松花河,聞納哈出傷,驚潰。勝遣觀童諭之乃降,得所部二十餘萬人,牛羊馬駝輜重互百餘里。還至亦迷河,復收其殘卒二萬餘、車馬五萬。而都督濮英殿後,為敵所殺。師還,以捷聞,並奏常茂激變狀,盡將降眾二十萬人入關。帝大悅,使使者迎勞勝等,械繫茂。會有言勝多匿良馬,使閽者行酒於納哈出之妻求大珠異寶,王子死二日強娶其女,失降附心,又失濮英三千騎,而茂亦訐勝過。帝怒,收勝大將軍印,命就第鳳陽,奉朝請,諸將士亦無賞。勝自是不復將大兵矣。

二十一年奉詔調東昌番兵征曲靖。番兵中道叛,勝鎮永寧撫安之。二十五年命籍太原、平陽民為軍,立衛屯田。皇太孫立,加太子太師,偕潁國公友德練軍山西、河南,諸公、侯皆聽節制。

時詔列勛臣望重者八人,勝居第三。太祖春秋高,多猜忌。勝功最多,數以細故失帝意。藍玉誅之月,召還京。踰二年,賜死,諸子皆不得嗣。而國用子誠積戰功雲南,累官至右軍左都督。

納哈出者,元木華黎裔孫,為太平路萬戶。太祖克太平被執,以名臣後,待之厚。知其不忘元,資遣北歸。元既亡,納哈出聚兵金山,畜牧蕃盛。帝遣使招諭之,終不報。數犯遼東,為葉旺所敗。勝等大兵臨之,乃降,封海西侯。從傅友德征雲南,道卒。子察罕,改封沈陽侯,坐藍玉黨死。

傅友德,其先宿州人,後徙碭山。元末從劉福通黨李喜喜入蜀。喜喜敗,從明玉珍,玉珍珍不能用。走武昌,從陳友諒,無所知名。

太祖攻江州,至小孤山,友德帥所部降。帝與語,奇之,用為將。從常遇春援安豐,略廬州。還,從戰鄱陽湖,輕舟挫友諒前鋒。被數創,戰益力,復與諸將邀擊於涇江口,友諒敗死。從征武昌,城東南高冠山下瞰城中,漢兵據之,諸將相顧莫前。友德帥數百人,一鼓奪之。流矢中頰洞脅,不為沮。武昌平,授雄武衛指揮使。從徐達拔廬州,別將克夷陵、衡州、襄陽。攻安陸,被九創,破擒其將任亮。從大軍下淮東,破張士誠援兵於馬騾港,獲戰艘千,復大破元將竹貞於安豐。同陸聚守徐州,擴廓遣將李二來攻,次陵子村。友德度兵寡不敵,遂堅壁不戰。詗其眾方散掠,以二千人溯河至呂梁,登陸擊之,單騎奮槊刺其將韓乙。敵敗去。度且復至,亟還,開城門而陣於野,臥戈以待,約聞鼓即起。李二果至,鳴鼓,士騰躍搏戰,破擒二。召還,進江淮行省參知政事,撤御前麾蓋,鼓吹送歸第。

明年從大將軍北征,破沂州,下青州。元丞相也速來援,以輕騎誘敵入伏,奮擊敗走之。遂取萊陽、東昌。明年從定汴、洛,收諸山寨。渡河取衛輝、彰德,至臨清,獲元將為嚮導,取德州、滄州。既克元都,偵邏古北隘口,守盧溝橋,略大同,還下保定、真定,守定州。從攻山西,克太原。擴廓自保安來援,萬騎突至。友德以五十騎衝卻之,因夜襲其營。擴廓倉卒遁去,追至土門關,獲其士馬萬計。復敗賀宗哲於石州,敗脫列伯於宣府,遂西會大將軍,圍慶陽,以偏師駐靈州,遏其援兵,遂克慶陽。還,賜白金文綺。

洪武三年從大將軍搗定西,大破擴廓。移兵伐蜀,領前鋒出一百八渡,奪略陽關,遂入沔。分兵自連雲棧合攻漢中,克之。以饋餉不繼,還軍西安。蜀將吳友仁寇漢中。友德以三千騎救之,攻斗山寨,令軍中人燃十炬布山上,蜀兵驚遁。是冬,論功授開國輔運推誠宣力武臣、榮祿大夫、柱國、同知大都督府事,封穎川侯,食祿千五百石,予世券。

明年充征虜前將軍,與征西將軍湯和分道伐蜀。和帥廖永忠等以舟師攻瞿塘,友德帥顧時等以步騎出秦、隴。太祖諭友德曰:「蜀人聞我西伐,必悉精銳東守瞿塘,北阻金牛,以抗我師。若出不意,直搗階、文,門戶既隳,腹心自潰。兵貴神速,患不勇耳。」友德疾馳至陜,集諸軍聲言出金牛,而潛引兵趨陳倉,攀援巖谷,晝夜行。抵階州,敗蜀將丁世珍,克其城。蜀人斷白龍江橋。友德修橋以渡,破五里關,遂拔文州。渡白水江,趨綿州。時漢江水漲,不得渡,伐木造戰艦。欲以軍聲通瞿塘,乃削木為牌為千,書克階、文、綿日月,投漢水,順流下。蜀守者見之,皆解體。

初,蜀人聞大軍西征,丞相戴壽等果悉眾守瞿塘。及聞友德破階、文,搗江油,始分兵援漢州,以保成都。未至,友德已破其守將向大亨於城下,謂將士曰:「援師遠來,聞大亨破,己膽落,無能為也。」迎擊,大敗之。遂拔漢州,進圍成都。壽等以象戰。友德令強弩火器衝之,身中流矢不退,將士殊死戰。象反走,躪藉死者甚眾。壽等聞其主明昇已降,乃籍府庫倉廩面縛詣軍門。成都平。分兵徇州邑未下者,克保寧,執吳友仁送京師,蜀地悉定。友德之攻漢州也,湯和尚頓軍大溪口。既於江流得木牌,乃進師。而戴壽等撤其精兵西救漢州,留老弱守瞿塘,故永忠等得乘勝搗重慶,降明升,於是太祖製《平西蜀文》,盛稱友德功為第一,廖永忠次之。師還,受上賞。

五年副征西將軍馮勝徵沙漠,敗失剌罕於西涼,至永昌,敗太尉朵兒只巴,獲馬牛羊十餘萬。略甘肅,射殺平章不花,降太尉鎖納兒等。至瓜沙州,獲金銀印及雜畜二萬而還。是時師出三道,獨友德全勝。以主將勝坐小法,賞不行。明年復出雁門,為前鋒,獲平章鄧孛羅帖木兒。還鎮北平,陳便宜五事。皆從之。召還,從太子講武於荊山,益歲祿千石。九年破擒伯顏帖木兒於延安,降其眾。帝將征雲南,命友德巡行川、蜀、雅、播之境,修城郭,繕關梁,因兵威降金築、普定諸山寨。

十四年副大將軍達出塞,討乃兒不花,渡北黃河,襲灰山,斬獲甚眾。其年秋充征南將軍,帥左副將軍藍玉、右副將軍沐英,將步騎三十萬征雲南。至湖廣,分遣都督胡海等將兵五萬由永寧趨烏撒,而自帥大軍由辰、沅趨貴州。克普定、普安,降諸苗蠻。進攻曲靖,大戰白石江,擒元平章達里麻。遂擊烏撒,循格孤山而南,以通永寧之兵,遣兩將軍趨雲南。元梁王走死。友德城烏撒,群蠻來爭,奮擊破之,得七星關以通畢節。又克可渡河,降東川、烏蒙、芒部諸蠻。烏撒諸蠻復叛,討之,斬首三萬餘級,獲牛馬十餘萬,水西諸部皆降。十七年論功進封潁國公,食祿三千石,予世券。

十九年帥師討平雲南蠻。二十年副大將軍馮勝,徵納哈出於金山。二十一年,東川蠻叛,復為征南將軍,帥師討平之。移兵討越州叛酋阿資,明年破之於普安。二十三年從晉王、燕王征沙漠,擒乃兒不花,還駐開平,復徵寧夏。明年為征虜將軍,備邊北平。復從燕王征哈者舍利,追元遼王。軍甫行,遽令班師。敵不設備,因潛師深入至黑嶺,大破敵眾而還。再出,練兵山、陜,總屯田事。加太子太師,尋遣還鄉。

友德喑啞跳盪,身冒百死。自偏裨至大將,每戰必先士卒。雖被創,戰益力,以故所至立功,帝屢敕獎勞。子忠,尚壽春公主,女為晉世子濟熺妃。

二十五年,友德請懷遠田千畝。帝不悅曰:「祿賜不薄矣,復侵民利何居?爾不聞公儀休事耶?」尋副宋國公勝分行山西,屯田於大同、東勝,立十六衛。是冬再練軍山西、河南。明年,偕召還。又明年賜死。以公主故,錄其孫彥名為金吾衛千戶。弘治中,晉王為友德五世孫瑛援六王例,求襲封。下禮官議,不許。嘉靖元年,雲南巡撫都御史何孟春請立祠祀友德。詔可,名曰「報功」。

廖永忠,巢人,楚國公永安弟也。從永安迎太祖於巢湖,年最少。太祖曰:「汝亦欲富貴乎?」永忠曰:「獲事明主,掃除寇亂,垂名竹帛,是所願耳。」太祖嘉焉。副永安將水軍渡江,拔采石、太平,擒陳野先,破蠻子海牙及陳兆先,定集慶,克鎮江、常州、池州,討江陰海寇,皆有功。

永安陷於吳,以永忠襲兄職,為樞密僉院,總其軍。攻趙普勝柵江營,復池州。陳友諒犯龍江,大呼突陣,諸軍從其後,大敗之。從伐友諒,至安慶,破其水寨,遂克安慶。從攻江州,州城臨江,守備甚固。永忠度城高下,造橋於船尾,名曰天橋,以船乘風倒行,橋傅於城,遂克之。進中書省右丞。

從下南昌,援安豐,戰鄱陽湖,決圍殊死戰。敵將張定邊直犯太祖舟,常遇春射走之。永忠乘飛舸追且射,定邊被百餘矢,漢卒多死傷。明日,復與俞通海等以七舟載葦荻,乘風縱火,焚敵樓船數百。又以六舟深入搏戰,復旋繞而出,敵驚為神。又邀擊之涇江口,友諒死。從征陳理,分兵柵四門,於江中連舟為長寨,絕其出入,理降。還京,太祖以漆牌書「功超群將,智邁雄師」八字賜之,懸於門。已,從徐達取淮東,張士誠遣舟師薄海安,太祖令永忠還兵水寨禦之,達遂克淮東諸郡。從伐士誠,取德清,進克平江,拜中書平章政事。

尋充征南副將軍,帥舟師自海道會湯和,討降方國珍,進克福州。洪武元年兼同知詹事院事。略定閩中諸郡,至延平,破執陳友定。尋拜征南將軍,以朱亮祖為副,由海道取廣東。永忠先發書諭元左丞何真,曉譬利害。真即奉表請降。至東莞,真帥官屬出迎。至廣州,降盧左丞。擒海寇邵宗愚,數其殘暴斬之。廣人大悅。馳諭九真、日南、朱崖、儋耳三十餘城,皆納印請吏。進取廣西,至梧州,降元達魯花赤拜住,潯、柳諸路皆下。遣亮祖會楊璟收未下州郡。永忠引兵克南寧,降象州。兩廣悉平。永忠善撫綏,民懷其惠,為之立祠。明年九月還京師,帝命太子帥百官迎勞於龍江。入見,仍命太子送還第。復出,撫定泉、漳。三年從大將軍徐達北征,克察罕腦兒。還,封德慶侯,食祿一千五百石,予世券。

明年,以征西副將軍從湯和帥舟師伐蜀。和駐大溪口,永忠先發。及舊夔府,破守將鄒興等兵。進至瞿塘關,山峻水急,蜀人設鐵鎖橋,橫據關口,舟不得進。永忠密遣數百人持糗糧水筒,舁小舟踰山渡關,出其上流。蜀山多草木,令將士皆衣青蓑衣,魚貫走崖石間。度已至,帥精銳出墨葉渡,夜五鼓,分兩軍攻其水陸寨。水軍皆以鐵裹船頭,置火器而前。黎明,蜀人始覺,盡銳來拒。永忠已破其陸寨,會將士舁舟出江者,一時並發,上下夾攻,大破之,鄒興死。遂焚三橋,斷橫江鐵索,擒同僉蔣達等八十餘人。飛天張、鐵頭張等皆遁去,遂入夔府。明日,和始至,乃與和分道進,期會於重慶。永忠帥舟師直搗重慶,次銅鑼峽。蜀主明昇請降,永忠以和未至辭。俟和至,乃受降,承制撫慰。下令禁侵掠。卒取民七茄,立斬之。慰安戴壽、向大享等家,令其子弟持書往成都招諭。壽等已為傅友德所敗,及得書,遂降。蜀地悉平。帝製《平蜀文》旌其功,有「傅一廖二」之語,褒賚甚厚。明年北征,至和林。六年督舟師出海捕倭,尋還京。

初,韓林兒在滁州,太祖遣永忠迎歸應天,至瓜步覆其舟死,帝以咎永忠。及大封功臣,諭諸將曰:「永忠戰鄱陽時,忘軀拒敵,可謂奇男子。然使所善儒生窺朕意,徼封爵,故止封侯而不公。」及楊憲為相,永忠與相比。憲誅,永忠以功大得免。八年三月坐僭用龍鳳諸不法事,賜死,年五十三。

子權,十三年嗣侯,從傅友德征雲南,守畢節及瀘州,召還。十七年卒。子鏞不得嗣,以嫡子為散騎舍人,累官都督。建文時與議兵事,宿衛殿廷。與弟銘皆嘗受學於方孝孺。孝孺死,鏞、銘收其遺骸,葬到處寶門外山上。甫畢,亦見收,論死。弟鉞及從父指揮僉事昇俱戍邊。

初,廖永忠等之歸太祖也,趙庸兄弟亦俱降,後亦有過不得封公,與永忠類。

庸,廬州人,與兄仲中聚眾結水寨,屯巢湖,歸太祖。仲中累功為行樞密院僉事,守安慶。陳友諒陷安慶,仲中棄城走還龍江,法當誅。常遇春請原之。太祖不許,曰:「法不行,無以懲後。」遂誅仲中,而以其官授庸。從復安慶,徇江西諸路,進參知政事。從戰康郎山,與俞通海、廖永忠等以六舟深入敗敵。平武昌,克廬州,援安豐,皆有功。大軍取淮東,庸與華高帥舟師克海安、泰州,進國平江。吳平,擢中書左丞。從大將軍取山東。洪武元年命兼太子副詹事。河南平,命庸留守。復分兵渡河,徇下河北州縣,進克河間,守之。尋移守保定,并收未復山寨。又從大軍克太原,下關。陜。從常遇春北追元帝。師還,遇春卒,命庸為副將軍,同李文忠攻慶陽。行至太原,元兵攻大同急,文忠與庸謀,以便宜援大同,再敗元兵於馬邑,擒其將脫列伯。論功,賞賚亞於大將軍。三年復從文忠北伐,出野狐嶺,克應昌。師還,論功最,以在應昌私納奴婢,不得封公,封南雄侯,食祿一千五百石,予世券。已,從伐蜀,中途還。

十四年,閩、粵盜起,命庸討之。踰年悉平諸盜及陽山、歸善叛蠻,戮其魁,散遣餘眾,民得復業。奏籍蜒戶萬人為水軍。又平廣東盜號鏟平王者,獲賊黨萬七千八百餘人,斬首八千八百餘級,降其民萬三千餘戶。還,賜彩幣、上尊、良馬。其冬出理山西軍務,巡撫北邊。二十年,以左參將從傅友德討納哈出。二十三年,以左副將軍從燕王出古北口,降乃兒不花。還,坐胡惟庸黨死。爵除。

楊璟,合肥人。本儒家子。以管軍萬戶從太祖下集慶,進總管。下常州,進親軍副都指揮使。從下婺州,遷樞密院判官。再從伐漢,以功擢湖廣行省參政,移鎮江陵。進攻湖南蠻寇,駐師三江口。復以招討功遷行省平章政事。帥左丞周德興、參政張彬將武昌諸衛軍,取廣西。

洪武元年春進攻永州。守將鄧祖勝迎戰敗,斂兵固守。璟進圍之。元兵來援,駐東鄉,倚湘水列七營,軍勢甚盛。璟擊敗之,俘獲千餘人。全州守將平章阿思蘭及周文貴再以兵來援,輒遣德興擊敗之。遣千戶王廷取寶慶,德興、彬取全州,略定道州、藍山、桂陽、武岡諸州縣。而永州久不下,令裨將分營諸門,築壘困之,造浮橋西江上,急攻之。祖勝力盡,仰藥死。百戶夏昇約降。璟兵踰城入,參政張子賢巷戰,軍潰被執,遂克永州。而征南將軍廖永忠、參政朱亮祖亦自廣東取梧州,定潯、貴、鬱林。亮祖以兵來會。進攻靖江不下,璟謂諸將曰:「彼所恃西濠水耳。決其隄岸,破之必矣。」乃遣指揮丘廣攻叚口關,殺守隄兵,盡決濠水,築土隄五道,傅於城。城中猶固守。急攻二月,克之,執平章也兒吉尼。先是張彬攻南關,為守城者所詬,怒欲屠其民。璟甫入,立下令禁止之,民乃安。復移師徇郴州,降兩江土官黃英岑、伯顏等,而永忠亦定南寧、象州。廣西悉平。

還,與偏將軍湯和從徐達取山西,至澤州,及元平章韓扎兒戰於韓店,敗績。還,捕唐州亂卒,留鎮南陽。未幾,詔璟往使於夏。是時夏主昇幼,母彭及諸大臣用事。璟既至。數諭升以禍福,俾從入覲。昇集其下共議。而諸大臣方專恣,不利昇歸朝,皆持不可,升亦莫能決。璟還,再以書諭升,終不聽。踰二年而夏亡。璟遷湖廣行省平章。

慈利土官覃垕構諸洞蠻為亂,命帥師往討,連敗之。垕詐降,璟使部卒往報,為所執。太祖遣使讓璟。璟督戰,士力攻,賊乃遁。

三年大封功臣,封璟營陽侯,祿千五百石,予世券。

四年從湯和伐夏,戰於瞿塘,不利。明年充副將軍,從鄧愈討定辰、沅蠻寇。再從大將軍徐達鎮北平,練兵遼東。十五年八月卒,追封芮國公,謚武信。子通嗣,二十年帥降軍戍雲南,多道亡,降普定指揮使。二十三年,詔書坐璟胡惟庸黨,謂以瞿塘之敗被責,有異謀云。

胡美,沔陽人。初名廷瑞,避太祖字,易名美。初仕陳友諒,為江西行省丞相,守龍興。太祖既下江州,遣使招諭美。美遣使鄭仁傑詣九江請降,且請無散部曲。太祖初難之,劉基蹴所坐胡床。太祖悟,賜書報曰:「鄭仁傑至,言足下有效順之誠,此足下明達也;又恐分散所部,此足下過慮也。吾起兵十年,奇才英士,得之四方多矣。有能審天時,料事機,不待交兵,挺然委身來者,嘗推赤心以待,隨其才任使之,兵少則益之以兵,位卑則隆之以爵,財乏則厚之以賞,安肯散其部曲,使人自危疑,負來歸之心哉?且以陳氏諸將觀之,如趙普勝驍勇善戰,以疑見戮。猜忌若此,竟何所成。近建康龍灣之役,予所獲長張、梁鉉、彭指揮諸人,用之如故,視吾諸將,恩均義一。長張破安慶水寨,梁鉉等攻江北,並膺厚賞。此數人者,其自視無復生理,尚待之如此,況如足下不勞一卒,以完城來歸者耶?得失之機,間不容髮,足下當早為計。」美得書,乃遣康泰至九江來降。太祖遂如龍興,至樵舍。美以陳氏所授丞相印及軍民糧儲之數來獻,迎謁於新城門。太祖慰勞之,俾仍舊官。

美之降也,同僉康泰、平章祝宗不欲從,美微言於太祖。太祖命將其兵,從徐達征武昌。二人果叛,攻陷洪都。達等還兵擊定之。祝宗走死,執康泰歸於建康。太祖以泰為美甥,赦勿誅。美從征武昌,復與達等帥馬步舟師取淮東,進伐張士誠,下湖州,圍平江,別將取無錫,降莫天祐。師還,加榮祿大夫。

其冬,命為征南將軍,帥師由江西取福建,諭之曰:「汝以陳氏丞相來歸,事吾數年,忠實無過,故命汝總兵取閩。左丞何文輝為爾副,參政戴德聽調發,二人雖皆吾親近,勿以其故廢軍法。聞汝嘗攻閩中,宜深知其地利險易。今總大軍攻圍城邑,必擇便宜可否為進退,無失機宜。」美遂渡杉關,下光澤,邵武守將李宗茂以城降。次建陽,守將曹復疇亦降。進圍建寧,守將同僉達里麻、參政陳子琦謀堅守以老我師。美數挑戰,不出,急攻之,乃降。整軍入城,秋毫無所犯。執子琦等送京師,獲將士九千七百餘人,糧糗馬畜稱是。會湯和等亦取福州、延平、興化,美遂遣降將諭降汀、泉諸郡。福建悉平。美留守其地。尋召還,從幸汴梁。

太祖即位,以美為中書平章、同知詹事院事。洪武三年命赴河南,招集擴廓故部曲。是年冬論功,封豫章侯,食祿千五百石,予世券,誥詞以竇融歸漢為比。十三年改封臨川侯,董建潭漂府於長沙。太祖榜列勛臣,謂持兵兩雄間,可觀望而不觀望來歸者七人。七人者,韓政、曹良臣、楊璟、陸聚、梅思祖、黃彬及美,皆封侯。美與璟有方面勛,帝遇之尤厚。

十七年坐法死。二十三年,李善長敗,帝手詔條列奸黨,言美因長女為貴妃,偕其子婿入亂宮禁,事覺,子婿刑死,美賜自盡云。贊曰:馮勝、傅友德,百戰驍將也。考當日功臣位次,與太祖褒美之詞,豈在湯和、鄧愈下哉。廖永忠智勇超邁,功亞宋、潁,皆不得以功名終,身死爵除,為可慨矣。江夏侯周德興之得罪也,太祖宥之,因誡諭公、侯,謂多粗暴無禮,自取敗亡。又謂永忠數犯法,屢宥不悛。然則洪武功臣之不獲保全者,或亦有以自取歟。楊璟、胡美功雖不逮,然嘗別將,各著方面勛,故次列之云。


\end{pinyinscope}