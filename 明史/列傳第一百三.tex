\article{列傳第一百三}

\begin{pinyinscope}
王治歐陽一敬胡應嘉周弘祖岑用賓鄧洪震詹仰庇駱問禮楊松張應治鄭履淳陳吾德李已胡涍汪文輝劉奮庸{{曹大埜

王治,字本道,忻州人。嘉靖三十二年進士。除行人,遷吏科給事中。寇屢盜邊,邊臣多匿不奏;小勝,文臣輒冒軍功。治請臨陣斬獲,第錄將士功;文臣及鎮帥不親搏戰者止賜賚。從之,再遷禮科左給事中。

隆慶元年,偕御史王好問核內府諸監局歲費。中官崔敏請止之,為給事中張憲臣所劾。得旨:「詔書所載者,自嘉靖四十一年始,聽治等詳核。不載者,已之。」治等力爭,不許。事竣,劾中官趙廷玉、馬尹乾沒罪,詔下司禮監按問。尋上疏陳四事:「一、定宗廟之禮以隆聖孝。獻皇雖貴為天子父,未嘗南面臨天下;雖親為武宗叔,然嘗北面事武宗。今乃與祖宗諸帝並列,設位於武宗右,揆諸古典,終為未協。臣以為獻皇祔會太廟,不免遞遷。若專祀世廟,則億世不改。乞敕廷臣博議,務求至當。一、謹燕居之禮以澄化源。人主深居禁掖,左右便佞窺伺百出,或以燕飲聲樂,或以遊戲騎射。近則損敝精神,疾病所由生。久則妨累政事,危亂所由起。比者人言籍籍,謂陛下燕閒舉動,有非諒闇所宜者。臣竊為陛下慮之。」其二,請勤朝講、親輔弼。疏入,報聞。

進吏科都給事中。劾薊遼總督都御史劉燾、南京督儲都御史曾于拱不職,于拱遂罷。山西及薊鎮並中寇,治以罪兵部尚書郭乾、侍郎遲鳳翔,偕同官歐陽一敬等劾之。詔罷乾,貶鳳翔三秩視事。部議恤光祿少卿馬從謙。帝不許,治疏爭。帝謂從謙所犯,比子罵父律,終不允。治又請追謚何瑭,雪夏言罪,且言大理卿朱廷立、刑部侍郎詹瀚共鍛成夏言、曾銑獄,宜追奪其官。咸報可。明年,左右有言南海子之勝者,帝將往幸。治率同官諫,大學士徐階、尚書楊博、御史郝傑等並阻止,皆不聽。至則荒莽沮濕,帝甚悔之。治尋擢太僕少卿,改大理,進太僕卿。憂歸,卒。

歐陽一敬,字司直,彭澤人。嘉靖三十八年進士。除蕭山知縣。征授刑科給事中。劾太常少卿晉應槐為文選郎時劣狀,而南京侍郎傅頤、寧夏巡撫王崇古、湖廣參政孫弘軾由應槐進,俱當罷。吏部為應槐等辨,獨罷頤官。未幾,劾罷禮部尚書董份。三遷兵科給事中。言廣西總兵當用都督,不當用勳臣。因劾恭順侯吳繼爵,罷之,以俞大猷代。寇大入陜西,劾總督陳其學、巡撫戴才,俱奪官。又以軍政劾英國公張溶,山西、浙江總兵官董一奎、劉顯,掌錦衣衛都督李隆等九人不職。溶留,餘俱貶黜。

自嚴嵩敗,言官爭發憤論事,一敬尤敢言。隆慶元年正月,吏部尚書楊博掌京察,黜給事中鄭欽、御史胡維新,而山西人無下考者。吏科給事中胡應嘉劾博挾私憤,庇鄉里。應嘉先嘗劾高拱,拱修郤,將重罪之。徐階等重違拱意,且以應嘉實佐察,初未言,今黨同官妄奏,擬旨斥為民。言路大嘩。一敬為應嘉訟,斥博及拱。詆拱奸險橫惡,無異蔡京,且言:「應嘉前疏臣與聞,黜應嘉不若黜臣。」會給事中辛自修、御史陳聯芳疏爭,階乃調應嘉建寧推官。一敬尋劾拱威制朝紳,專柄擅國,亟宜罷。不聽。踰月,御史齊康劾階。諸給事御史以康受拱指,群集闕下,詈而唾之。一敬首劾康,康亦劾一敬。時康主拱,一敬主階,互指為黨。言官多論康,康竟坐謫。

已,陳兵政八事,部皆議行。南京振武營兵由此罷。湖廣巡按陳省劾太和山守備中官呂祥,詔徵祥還,罷守備官。未幾,復遣監丞劉進往代。一敬言:「進故名俊,守顯陵無狀。肅皇帝下之獄,充孝陵衛凈軍,今不宜用。」從之。中官呂用等典京營,一敬力諫,事寢。黔國公沐朝弼殘恣,屢抗詔旨。一敬請治其罪,報可。俄擢太常少卿。拱再起柄政,一敬懼,即日告歸,半道以憂死。時應嘉已屢遷參議,憂歸,聞拱再相,亦驚怖而卒。

應嘉,沐陽人。由宜春知縣擢吏科給事中。三遷都給事中。論侍郎黃養蒙、李登雲及布政使李磐、侯一元不職,皆罷去。登雲者,大學士高拱姻也。應嘉策拱必害己,遂并劾拱,言:「拱輔政初,即以直廬為隘,移家西安門外,夤夜潛歸。陛下近稍違和,拱即私運直廬器物於外。臣不知拱何心。」疏入,拱大懼,亟奏辯。會帝崩,得不竟。拱以此銜應嘉。穆宗嗣位,應嘉請帝御文華殿與輔臣面議大政,召訪諸卿顧問侍從,令科臣隨事駁議。帝納焉。應嘉居諫職,號敢言。然悻悻好搏擊,議者頗以傾危目之。

周弘祖,麻城人。嘉靖三十八年進士。除吉安推官。征授御史,出督屯田、馬政。隆慶改元,司禮中貴及籓邸近侍廕錦衣指揮以下至二十餘人。弘祖馳疏請止賚金幣,或停世襲,且言:「高皇帝定制,宦侍止給奔走掃除,不關政事。孝宗召對大臣,宦侍必退去百餘武,非惟不使之預,亦且不使之聞。願陛下勿與謀議,假以嚬笑,則彼無亂政之階,而聖德媲太祖、孝宗矣。臣又聞先帝初載,欲廕太監張欽義子錦衣,兵部尚書彭澤執奏再四。今趙炳然居澤位,不能效澤忠,無所逃罪。」報聞。已,請汰內府監局、錦衣衛、光祿寺、文思院冗員,復嘉靖初年之舊,又請仿行古社倉制。詔皆從之。

明年春,言:「近四方地震,土裂成渠,旂竿數火,天鼓再鳴,隕星旋風,天雨黑豆,此皆陰盛之徵也。陛下嗣位二年,未嘗接見大臣,咨訪治道。邊患孔棘,備禦無方。事涉內庭,輒見撓沮,如閱馬、核庫,詔出復停。皇莊則親收子粒,太和則榷取香錢,織造之使累遣,糾劾之疏留中。內臣爵賞謝辭,溫旨遠出六卿上,尤祖宗朝所絕無者。」疏入,不報。其冬詔市珍寶,魏時亮等爭,不聽。弘祖復切諫。尋遷福建提學副使。大學士高拱掌吏部,考察言官,惡弘祖及岑用賓等,謫弘祖安順判官,用賓宜川縣丞。

用賓,廣東順德人。官南京給事中,多所論劾。又嘗論拱很愎,以故拱憾之,出為紹興知府。既中以察典,遂卒於貶所。而弘祖謫未幾,拱罷,量移廣平推官,萬歷中,屢遷南京光祿卿。坐朱衣謁陵免。

當隆慶初,以地震言事者,又有鄧洪震,宣化人。時為兵部郎中,上疏曰:「入夏以來,淫雨彌月。又京師去冬地震,今春風霾大作,白日無光。近大同又報雨雹傷物,地震有聲。陛下臨御甫半年,災異疊見。傳聞後宮游幸無時,嬪御相隨,後車充斥。左右近習,濫賜予。政令屢易,前後背馳,邪正混淆,用舍猶豫。萬一奸宄潛生,寇戎軼犯,其何以待之?」帝納其言,下禮官議修省。洪震尋以疾歸。萬曆改元,督撫交章論薦,竟不起。

詹仰庇,字汝欽,安溪人。嘉靖四十四年進士。由南海知縣徵授御史。隆慶初,穆宗詔戶部購寶珠,尚書馬森執奏,給事中魏時亮、御史賀一桂等繼爭,皆不聽。仰庇疏言:「頃言官諫購寶珠,反蒙詰讓。昔仲虺戒湯不邇聲色,不殖貨利;召公戒武王玩人喪德,玩物喪志。湯、武能受二臣之戒,絕去玩好,故聖德光千載。若侈心一生,不可復遏,恣情縱欲,財耗民窮。陛下玩好之端漸啟,弼違之諫惡聞,群小乘隙,百方誘惑,害有不勝言者。況寶石珠璣,多藏中貴家,求之愈急,邀直愈多,奈何以有用財,耗之無用之物。今兩廣需餉,疏請再三,猶靳不予,何輕重倒置乎!」不報。三年正月,中官製煙火,延燒禁中廬舍,仰庇請按治。左右近習多切齒者。

帝頗耽聲色,陳皇后微諫,帝怒,出之別宮。外庭皆憂之,莫敢言。仰庇入朝,遇醫禁中出。詢之,知后寢疾危篤,即上疏言:「先帝慎擇賢淑,作配陛下,為宗廟社稷內主。陛下宜遵先帝命,篤宮闈之好。近聞皇后移居別宮,已近一載,抑鬱成疾,陛下略不省視。萬一不諱,如聖德何?臣下莫不憂惶,徒以事涉宮禁,不敢頌言。臣謂人臣之義,知而不言,當死;言而觸諱,亦當死。臣今日固不惜死,願陛下采聽臣言,立復皇后中宮,時加慰問,臣雖死賢於生。」帝手批答曰:「后無子多病,移居別宮,聊自適,以冀卻疾。爾何知內庭事,顧妄言。」仰庇自分得重譴,同列亦危之。及旨下,中外驚喜過望,仰庇益感奮。

亡何,巡視十庫,疏言:「內官監歲入租稅至多,而歲出不置籍。按京城內外園廛場地,隸本監者數十計,歲課皆屬官錢,而內臣假上供名,恣意漁獵。利填私家,過歸朝宁。乞備核宜留宜革,并出入多寡數,以杜奸欺。再照人主奢儉,四方係安危。陛下前取戶部銀,用備緩急。今如本監所稱,則盡以創鰲山、修宮苑、製鞦遷、造龍鳳艦、治金櫃玉盆。群小因乾沒,累聖德,虧國計。望陛下深省,有以玩好逢迎者,悉屏出罪之。」宦官益恨。故事,諸司文移往還及牧民官出教,用「照」字,言官上書無此體。宦官因指「再照人主」語,為大不敬。帝怒,下詔曰:「仰庇小臣,敢照及天子,且狂肆屢不悛。」遂廷杖百,除名,并罷科道之巡視庫藏者。南京給事中駱問禮、御史余嘉詔等疏救,且言巡視官不當罷。不納。仰庇為御史僅八月,數進讜言,竟以獲罪。

神宗嗣位,錄先朝直臣。以仰庇在京時嘗為商人居間,不得內召,除廣東參議。尋乞歸。家居十餘年,起官江西。再遷南京太僕少卿。入為左僉都御史,進左副都御史。仰庇初以直節負盛名,至是為保位計,頗不免附麗。饒伸以科場事劾大學士王錫爵、左都御史吳時來,仰庇即劾伸。進士薛敷教劾時來及南京右都御史耿定向,仰庇未及閱疏,即論敷教排陷大臣,敷教坐廢。及吏部侍郎趙煥、兵部侍郎沈子木相繼去,仰庇謀代之,蹤跡頗著。給事中王繼光、主事姜士昌、員外郎趙南星、南京御史王麟趾等交章論列。仰庇不自安,屢求去。帝雖慰留,而眾議籍籍不止。稍遷刑部右侍郎。移疾歸,久之卒。

駱問禮,諸暨人。嘉靖末進士。歷南京刑科給事中。隆慶三年,陳皇后移別宮,問禮偕同官張應治等上言:「皇后正位中闈,即有疾,豈宜移宮。望亟返坤寧,毋使後世謂變禮自陛下始。」不報。給事張齊劾徐階,為廷臣所排,下獄削籍。問禮獨言齊贓可疑,不當以糾彈大臣實其罪。張居正請大閱,問禮謂非要務,而請帝日親萬幾,詳覽奏章。未幾,劾誠意伯劉世延、福建巡撫塗澤民不職,帝並留之。

帝初納言官請,將令諸政務悉面奏於便殿,問禮遂條上面奏事宜。一言:「陛下躬攬萬幾,宜酌用群言,不執己見,使可否予奪,皆合天道,則有獨斷之美,無自用之失。」二言:「陛下宜日居便殿,使侍從官常在左右,非嚮晦不入宮闈,則涵養薰陶,自多裨益。」三言:「內閣政事根本,宜參用諸司,無拘翰林,則講明義理,通達政事,皆得其人。」四言「詔旨必由六科,諸司始得奉行,脫有未當,許封還執奏。如六科不封駁,諸司失檢察者,許御史糾彈。」五言:「頃詔書兩下,皆許諸人直言。然所採納者,除言官與一二大臣外,盡付所司而已。宜益廣言路,凡臣民章奏,不惟其人惟其言,令匹夫皆得自效。」六言:「陛下臨朝決事,凡給事左右,如傳旨、接奏章之類,宜用文武侍從,毋使中官參與,則窺竊之漸,無自而生。」七言:「士習傾危,稍或異同,輒加排陷。自今凡議國事,惟論是非,不徇好惡。眾人言未必得,一人言未必非,則公論日明,士氣可振。」八言:「政令之出,宜在必行。今所司題覆,已報可者未見修舉,因循玩心妻,習為故常。陛下當明作於上,敕諸臣奮勵於下,以挽頹惰之風」。九言:「面奏之儀,宜略去繁文,務求實用,俾諸臣入而敷奏,退而治事,無或兩妨,斯上下之交可久。」十言:「修撰、編檢諸臣,宜令更番入直,密邇乘輿,一切言動,執簡侍書。其耳目所不及者,諸司或以月報,或以季報,令得隨事纂緝,以垂勸戒。」

疏奏,帝不悅。宦侍復從中構之,謫楚雄知事。明年,吏部舉雜職官當遷者,問禮及御史楊松在舉中。帝曰:「此兩人安得遽遷,俟三年後議之。」萬歷初,屢遷湖廣副使,卒。

楊松,河南衛人。歷官御史,巡視皇城。尚膳少監黃雄徵子錢與民哄,兵馬司捕送松所。事未決,而內監令校尉趣雄入直,詭言有駕帖。松驗問無有,遂劾雄詐稱詔旨。帝令黜兵馬司官,而鐫松三秩,謫山西布政司照磨。神宗立,擢廬州推官,終山西副使。

張應治,秀水人。在垣中抗疏,多可稱。為高拱所惡,出為九江知府。終山東副使。

鄭履淳,字叔初,刑部尚書曉子也。舉嘉靖四十年進士,除刑部主事,遷尚寶丞。隆慶三年冬,疏言:

頃年以來,萬民失業,四方多故,天鳴地震,災害洊臻,正陛下宵旰憂勤時也。夫饑寒迫身,易為衣食,嗷嗷赤子,聖主之所以為資。不及今定周家桑土之謀,切虞廷困窮之懼,則上天所以警動海內者,適足以資他人矣。今最急莫如用賢。陛下御極三祀矣,曾召問一大臣,面質一講官,賞納一諫士,以共畫思患豫防之策乎?高亢暌孤,乾坤否隔,忠言重折檻之罰,儒臣虛納牖之功,宮闈違脫珥之規,朝陛拂同舟之義。回奏蒙譴,補牘奚從?內批徑出,封還何自?紀綱因循,風俗玩心妻。功罪罔核,文案徒繁。閽寺潛為厲階,善類漸以短氣。言涉宮府,肆撓多端。梗在私門,堅持不破。萬眾惶惶,皆謂群小侮常,明良疏隔,自開闢以來,未有若是而永安者。伏願奮英斷以決大計,勿為小故之所淆;弘浚哲以任君子,勿為嬖暱之所惑。移美色奇珍之玩而保瘡痍,分昭陽細務之勤而和庶政。以蠻裔為關門勁敵,以錢穀為黎庶脂膏。拔用陸樹聲、石星之流,嘉納殷士儋、翁大立諸疏。經史講筵,日親無倦。臣民章奏,與所司面相可否。萬幾之裁理漸熟,人才之邪正自知。察變謹微,回天開泰,計無踰於此。

疏入,帝大怒,杖之百,繫刑部獄數月。刑科舒化等以為言,乃釋為民。神宗立,起光祿少卿,卒。

陳吾德,字懋修,歸善人。嘉靖四十四年進士。授行人。隆慶三年,擢工科給事中。兩廣多盜,將吏率虛文罔上。吾德列便宜八事,皆允行。明年正月朔,日有食之,已而月復食。吾德言:「歲首日月並食,天之大災,陛下宜屏斥一切玩好,應天以實。」詔遣中官督織造,吾德偕同官嚴用和切諫,報聞。帝從中官崔敏言,命市珍寶,戶部尚書劉體乾、戶科都給事中李已執奏,不從。吾德復偕已上疏曰:「伏睹登極詔書,罷採辦,蠲加派,且云『各監局以缺乏為名,移文苛取,及所司阿附奉行者,言官即時論奏,治以重典』,海內聞之,歡若更生。比者左右近習,干請紛紜,買玉市珠,傳帖數下。人情惶駭,咸謂詔書不信,無所適從。邇時府庫久虛,民生困瘁,司度支者日夕憂危。陛下奈何以玩好故,費數十萬貲乎!敏等獻諂營私,罪不可宥。乞亟譴斥,以全詔書大信。」帝震怒,杖已百,錮刑部獄,斥吾德為民。

神宗嗣位,起吾德兵科。萬歷元年,進右給事中。張居正柄國,諫官言事必先請,吾德獨不往。禮部主事宋儒與兵部主事熊敦朴不相能,誣敦朴欲劾居正,屬尚書譚綸劾罷之。既而誣漸白,吾德遂劾儒,亦謫之外。居正以吾德不白己,嗛之。未幾,爭成國公朱希忠贈定襄王爵,益忤居正。及慈寧宮後室災,吾德力爭,出為饒州知府。有盜建昌王印章者,遁之南京見獲。居正客操江都御史王篆坐吾德部下失盜,謫馬邑典史。御史又劾其蒞饒時違制講學,用庫金市學田,遂除名為民。居正死,薦起思州推官,移寶慶同知,皆以親老不赴。後終湖廣僉事。

李已,字子復,磁人。嘉靖四十四年進士。除太常博士,擢禮科給事中。隆慶中,頻詔戶部有所徵索。尚書劉體乾輒執奏,已每助之,以是積失帝意。及爭珍寶事,遂得禍。未幾,刑科給事中舒化等請釋已,刑部尚書葛守禮等因言:「朝審時,重囚情可矜疑者,咸得末減。已及內犯張恩等十人,讞未定,不列朝審中。茍瘐死犴狴,將累深仁。」帝乃釋已,恩等繫如故。法司以恩等有內援,欲借以脫已。及已獨釋,眾翕然稱帝仁明。

神宗立,薦起兵科都給事中。奏言:「陛下初基,弊端盡去,傳奉一事,豈可尚踵故常。內臣即有勤勞,當優以金帛,名器所在,不容濫設。」帝嘉納之。御史胡涍建言得罪,已首論救。尋劾兵部尚書譚綸去取邊將不當。平江伯陳王謨罪廢,復夤緣出鎮湖廣,已力爭得寢。擢順天府丞,遷大理右少卿。疏請改父母誥命,日已暮,逼禁門守者投入。帝怒,謫常州同知。

初,已與吾德並敢言,已尤以直著。兩遭摧抑,頗事營進。後為南京考功郎中。九年京察,希張居正指,與尚書何寬置司業張位、長史趙世卿察典,遂得擢南京尚寶卿。三遷右僉都御史,巡撫保定六府。踰年,罷歸,卒。

胡涍,字原荊,無錫人。嘉靖末舉進士。歷知永豐、安福二縣,擢御史。神宗即位之六日,命馮保代孟沖掌司禮監,召用南京守備張宏。涍請嚴馭近習,毋惑諂諛,虧損聖德。保大怒,思傾之。其冬,妖星見,慈寧宮後延燒連房。水孝乞遍察掖廷中曾蒙先朝寵幸者,體恤優遇,其餘無論老少,一概放遣。奏中有「唐高不君,則天為虐」語。帝怒,問輔臣,二語所指為誰。張居正對曰:「水孝言雖狂悖,心無他。」帝意未釋,嚴旨譙讓。涍惶恐請罪,斥為民。踰年,巡按御史李學詩薦涍。詔自後有薦者,并逮治涍。久之,卒。

汪文輝,字德充,婺源人。嘉靖四十四年進士。授工部主事。隆慶四年,改御史。高拱以內閣掌吏部,權勢烜赫。其門生韓楫、宋之韓、程文、塗夢桂等並居言路,日夜走其門,專務搏擊。文輝亦拱門生,心獨非之。明年二月,疏陳四事,專責言官。其略曰:

先帝末年所任大臣,本協恭濟務,無少釁嫌。始於一二言官見廟堂議論稍殊,遂潛察低昂、窺所向而攻其所忌。致顛倒是非,熒惑聖聽,傷國家大體。茍踵承前弊,交煽並構,使正人不安其位,恐宋元祐之禍,復見於今,是為傾陷。

祖宗立法,至精密矣,而卒有不行者,非法敝也,不得其人耳。今言官條奏,率銳意更張。部臣重違言官,輕變祖制,遷就一時,茍且允覆。及法立弊起,又議復舊。政非通變之宜,民無畫一之守,是為紛更。

古大臣坐事退者,必為微其詞;所以養廉恥,存國體。今或掇其已往,揣彼未形,逐景循聲,爭相詬病,若市井哄瘩然。至方面重臣,茍非甚奸慝,亦宜棄短錄長,為人才惜。今或搜抉小疵,指為大蠹,極言醜詆,使決引去。以此求人,國家安得全才而用之?是為苛刻。

言官能規切人主,糾彈大臣。至言官之短,誰為指之者?今言事論人或不當,部臣不為奏覆,即憤然不平;雖同列明知其非,亦莫與辨,以為體貌當如是。夫臣子且不肯一言受過,何以責難君父哉?是為求勝。

此四弊者,今日所當深戒。然其要在大臣取鑒前失,勿用希指生事之人。希指生事之人進,則忠直貞諒之士遠,而頌成功、譽盛德者日至於前。大臣任己專斷,即有闕失,孰從聞之?蓋宰相之職,不當以救時自足,當以格心為本。願陛下明飭中外,消朋比之私,還淳厚之俗,天下幸甚。

疏奏,下所司。拱惡其刺己,甫三日,出為寧夏僉事。修屯政,蠲浮糧,建水閘,流亡漸歸。御史富平孫丕揚忤拱,為希指者所劾。方行勘,文輝抗言曰:「毛舉細故,齮晷正人,以快當路之私,我固不肯為,諸君亦不可也。」於是緩其事。未幾,劾者先得罪去,丕揚竟獲免。神宗嗣位,拱罷政,召為尚寶卿。尋告歸。久之,有詔召用。未赴卒。

劉奮庸,洛陽人。嘉靖三十八年進士。授兵部主事。尋改禮部兼翰林待詔。侍穆宗裕邸。進員外郎。穆宗即位,以舊恩,擢尚寶卿。已,籓邸舊臣相繼柄用,獨奮庸久不調。大學士高拱亦故官也,再起任事,頗專恣,奮庸疾之。隆慶六年三月,上疏曰:

陛下踐阼六載,朝綱若振飭,而大柄漸移;仕路若肅清,而積習仍故。百僚方引領以睹勵精之治,而陛下精神志意漸不逮初。臣念潛邸舊恩,誼不忍默。謹條五事,以俟英斷。

一、保聖躬。人主一身,天地人神之主,必志氣清明,精神完固,而後可以御萬幾。望凝神定志,忍性抑情,毋逞旦夕之娛,毋徇無涯之慾,則無疆之福可長保也。

二、總大權。今政府所擬議,百司所承行,非不奉詔旨,而其間從違之故,陛下曾獨斷否乎?國事之更張,人才之用舍,未必盡出忠謀,協公論。臣願陛下躬攬大權,凡庶府建白,閣臣擬旨,特留清覽,時出獨斷,則臣下莫能測其機,而政柄不致旁落矣。

三、慎儉德。陛下嗣位以來,傳旨取銀不下數十萬,求珍異之寶,作鰲山之燈,服御器用,悉鏤金雕玉。生財甚難,靡敝無紀。願察內帑之空虛,思小民之艱苦,不作無益,不貴異物,則國用充羨,而民樂其生矣。

四、覽章奏。人臣進言,豈能皆當。陛下一切置不覽,非惟虛忠良獻納之誠,抑恐權奸蔽壅,勢自此成。望陛下留神章奏,曲垂容納。言及君德,則反己自修;言及朝政,則更化善治。聽言者既見之行事,而進言者益樂於效忠矣。

五、用忠直。邇歲進諫者,或以勤政,或以節用,或以進賢退不肖,此皆無所利而為之;非若承望風旨,肆攻擊以雪他人之憤,迎合權要,交薦拔以樹淫朋之黨者比也。

願恕狂愚之罪,嘉批鱗之誠,登之有位,以作士氣,則讜規日聞,裨益非鮮。

疏入,帝但報聞,不怒也。而附拱者謂奮庸久不徙官,怏怏風刺,相與詆訾之。給事中塗夢桂遂劾奮庸動搖國是。會給事中曹大埜亦劾拱十罪,帝斥之。給事中程文因奏拱竭忠報國,萬世永賴,奮庸與大埜漸構姦謀,傾陷元輔,罪不可勝誅。章並下吏部。拱方掌部事,陽為二臣祈寬。帝不許,竟謫大埜乾州判官,奮庸興國知州。夢桂、文皆拱門生。夢桂極詆奮庸,文則盛稱頌拱,又盡舉大埜奏中語代拱剖析,士論非之。奮庸謫官兩月,會神宗即位,遂擢山西提學僉事。再遷陜西提學副使。以病乞歸,卒。

大埜,巴縣人。其劾拱,張居正實使之。萬曆中,累遷右副都御史,巡撫江西。以貪劾免。

贊曰:世宗之季,門戶漸開。居言路者,各有所主,故其時不患其不言,患其言之冗漫無當,與其心之不能無私;言愈多,而國是愈益淆亂也。汪文輝所陳四弊,有旨哉!論明季言路諸臣,而考其得失,當於是觀之。


\end{pinyinscope}