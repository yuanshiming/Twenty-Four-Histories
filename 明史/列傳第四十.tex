\article{列傳第四十}

\begin{pinyinscope}
董倫王景儀智子銘鄒濟徐善述王汝玉梁潛周述弟孟簡陳濟陳繼楊翥俞山俞綱潘辰王英錢習禮周敘劉儼柯潛羅璟孔公恂司馬恂

董倫,字安常,恩人。洪武十五年以張以寧薦,授贊善大夫,侍懿文太子,陳說剴切。太祖嘉之,進左春坊大學士。太子薨,出為河南左參議。肇州吏目蘭溪諸葛伯衡廉,倫薦之。帝遽擢為陜西參議。又言儒學訓導宜與冠帶,別於士子。劉導始注選。三十年坐事謫雲南教官。雲南初設學校,倫以身教,人皆嚮學。

建文初,召拜禮部侍郎兼翰林學士,與方孝孺同侍經筵。御書「怡老堂」額寵之,又賜髹几、玉鳩杖。解縉謫河州,以倫言得召還。倫質直敦厚,嘗勸帝睦親籓,不聽。成祖即位,倫年已八十,命致仕,尋卒。

其與倫同時為禮部侍郎者,有王景,字景彰,松陽人。洪武初,為懷遠教諭。以博學應詔。命作朝享樂章,定籓王朝覲儀。累官山西參政,與倫先後謫雲南。建文初,召入翰林,修《太祖實錄》。用張紞薦,除禮部侍郎兼翰林侍講。成祖即位,擢學士。帝問葬建文帝禮,景頓首言:「宜用天子禮。」從之。永樂六年卒於官。

儀智,字居真,高密人。洪武末,舉耆儒,授高密訓導,遷莘縣教諭。擢知高郵州,課農興學,吏民愛之。

永樂元年遷寶慶知府。土人健悍,獨畏智,相戒不敢犯。召為右通政兼右中允。未幾,遷湖廣右布政使。坐事謫役通州。六年冬,湖廣都指揮使龔忠入見。帝問湖湘間老儒,忠以智對。即日召之。既至,拜禮部左侍郎。十一年元旦,日當食,尚書呂震請朝賀如常,智持不可。會左諭德楊士奇亦以為言,乃免賀如智議。

十四年詔吏部、翰林院擇耆儒侍太孫。士奇及蹇義首薦智。太子曰:「吾嘗舉李繼鼎,大誤,悔無及。智誠端士,然老矣。」士奇頓首言:「智起家學官,明理守正。雖耄,精神未衰。廷臣中老成正大,無踰智者。」是日午朝,帝顧太子曰:「侍太孫講讀得人未?」太子對曰:「舉禮部侍郎儀智,議未決。」帝喜曰:「智雖老,能直言,可用也。」遂命輔導皇太孫。每進講書史,必反復啟迪,以正心術為本。十九年,年八十,致仕,卒於家。洪熙元年贈太子少保,謚文簡。

季子銘,字子新。宣宗即位,以侍郎戴綸薦,授行在禮科給事中。九年秩滿,帝念智舊勞,改銘修撰。正統三年預修宣廟《實錄》成,遷侍講,後改郕府長史。

郕王監國,視朝午門。廷臣劾王振,叫號莫辨人聲。銘獨造膝前,免冠敷奏。下令旨族振,眾嘩始息。景帝即位,力贊征伐諸大事。尋以潛邸恩,授禮部右侍郎。明年兼經筵官。帝每臨講幄,輒命中官擲金錢於地,任講官遍拾之,號「恩典」。文臣與者,內閣高穀等外,惟銘與俞山、俞綱、蕭鎡、趙琬數人而已。尋進南京禮部尚書。懷獻太子立,加太子太保,召為兵部尚書兼詹事。

蘇州、淮安諸郡積雪,民凍餓死相枕。沙灣築河,役山東、河南九萬人,責民間鐵器數萬具。銘請於帝,多所寬恤。因災異,言消弭在敬天法祖,省刑薄斂,節用愛人。錄《皇明祖訓錄》以進,深見獎納。卒,謚忠襄。

銘少學於吳訥。天性孝友,易直有父風。長子海,錦衣衛百戶。季子泰,舉於鄉,為禮科給事中。並以父恩授云。

鄒濟,字汝舟,餘杭人。事母以孝聞。博學強記,尤長《春秋》。為餘杭訓導,師法嚴。累遷國子學錄、助教,以薦知平度州。永樂初,預修《太祖實錄》成,除禮部郎中。征安南,從幕府司奏記。還為廣東右參政,再遷左春坊左庶子,授皇孫經。

濟為人和易坦夷,無貴賤皆樂親之。秩滿,進少詹事。當是時,宮僚多得罪,徐善述、王汝玉、馬京、梁潛輩被讒,相繼下獄死。濟積憂得疾。皇太子以書慰曰:「卿善自攝。即有不諱,當提攜卿息,不使墜蓬蒿也。」卒,年六十八。洪熙元年贈太子少保,謚文敏。命有司立祠墓側,春秋祀之。

子幹,字宗盛,濟卒時尚幼。仁宗監國,命為應天府學生,月賜鈔米。舉正統四年進士。景帝初,由兵部郎中超擢本部右侍郎,以才為于謙所倚。也先入寇,九門皆閉。百姓避兵者,號城下求入,幹開門納之。尋改禮部,兼庶子,考察山西官吏,黜布政使侯復以下五十餘人。巡視河南、鳳陽水災,與王竑請振。又請令諸生輸粟入監讀書。納粟入監自此始。成化二年振畿內饑,再遷禮部尚書,加太子少保。被劾乞休,卒,謚康靖。

徐善述,字好古,天台人。洪武中,行歲貢法,善述首貢入太學。授桂陽州學正。永樂初,以國子博士擢春坊司直郎。見重於皇太子,每稱為「先生」,嘗致書賜酒及詩。遷左贊善,坐累死。與鄒濟同日贈太子少師,謚文肅。立祠,春秋祀亦如濟。

王汝玉,名璲,以字行,長洲人。穎敏強記。少從楊維楨學。年十七,舉於鄉。永樂初,由應天府學訓導擢翰林五經博士,歷遷右春坊右贊善,預修《永樂大典》。仁宗在東宮,特被寵遇。群臣應制撰《神龜賦》,汝玉第一,解縉次之。七年坐修《禮書》紊制度,當戍邊。皇太子監國,宥之,以為翰林典籍。尋進左贊善,坐解縉累,瘐死。洪熙初,贈太子賓客,謚文靖,遣官祭其家。

梁潛,字用之,泰和人。洪武末,舉鄉試。授四川蒼溪訓導。以薦除知四會縣,改陽江、陽春,皆以廉平稱。永樂元年召修《太祖實錄》。書成,擢修撰。尋兼右春坊右贊善,代鄭賜總裁《永樂大典》。帝幸北京,屢驛召赴行在。十五年復幸北京,太子監國。帝親擇侍從臣,翰林獨楊士奇,以潛副之。有陳千戶者,擅取民財,令旨謫交阯。數日後念其有軍功,貸還。或讒於帝曰:「上所謫罪人,皇太子曲宥之矣。」帝怒,誅陳千戶,事連潛及司諫周冕,逮至行在,親詰之。潛等具以實對。帝謂楊榮、呂震曰:「事豈得由潛!」然卒無人為白者,俱繫獄。或毀冕放恣,遂併潛誅。潛妻楊氏痛潛非命,不食死。

子楘,由進士為刑部主事,善辨冤獄。用薦擢廣西副使,進布政使。將士多殺良民報功,楘諭其帥,生致難民一人,準功一級,全活無算。田州土官岑鑒兄弟相仇,楘為解之,卻其厚餽。撫服梗化女土官,民夷服其信義。終浙江布政使。

周述,字崇述,吉水人。永樂二年與從弟孟簡並進士及第。帝手題二人策,獎賞之,並授翰林編修。尋詔解縉選曾棨等二十八人讀書文淵閣,述、孟簡皆與焉。司禮監給紙筆,光祿給朝暮饌,禮部月給膏燭鈔人三錠,工部擇近宅居之,一時以為榮。

述嘗扈北巡,累進左春坊諭德。仁宗即位,命從皇太子謁陵南京。召至榻前,問所以匡弼儲君者,對稱旨。宣宗時,進左庶子。正統初,卒官。

孟簡在翰林二十年,始遷詹事府丞,出為襄王府長史。有言宜留備顧問者,帝曰:「輔朕弟,尤勝於輔朕也。」述溫厚簡靜,未嘗有疾言遽色,文章雅贍。孟簡謙退不伐,生平無睚眥於人。並為世所重云。

陳濟,字伯載,武進人。讀書過目成誦。嘗以父命如錢塘,家人齎貨以從。比還,以其貲之半市書,口誦手鈔。十餘年,盡通經史百家之言。成祖詔修《永樂大典》,用大臣薦,以布衣召為都總裁,修撰曾棨等為之副。詞臣纂修者,及太學儒生數千人,繙秘庫書數百萬卷,浩無端倪。濟與少師姚廣孝等數人,發凡起例,區分鉤考,秩然有法。執筆者有所疑,輒就濟質問,應口辨析無滯。書成,授右贊善。謹慎無過,皇太子甚禮重之。凡稽古纂集之事,悉以屬濟。隨事敷奏,多所裨益。五皇孫皆從受經。居職十五年而卒。年六十二。

濟少有酒過,母戒之,終其身未嘗至醉。弟洽為兵部尚書,事濟如父。濟深懼盛滿,彌自謙抑。所居蓬戶葦壁,裁蔽風雨,終日危坐,手不釋卷。為文根據經史,不事葩藻。嘗云:「文貴如布帛菽粟,有益於世爾。」其後有陳繼、楊翥者,亦以布衣通經。用楊士奇薦,繼由博士入翰林。而翥竟用景帝潛邸恩,與俞山、俞綱等皆至大官。自天順後,始漸拘資格。編修馬升、檢討傅宗不由科目,李賢皆出之為參議。布衣無得預館閣者,而弘治間潘辰獨以才望得之,一時詫異數焉。

陳繼,字嗣初,吳人。幼孤,母吳氏,躬織以資誦讀。比長,貫穿經學,人呼為「陳五經」。奉母至孝,府縣交薦,以母老不就。母卒,哀毀過人。永樂中,復舉孝行,旌其母曰「貞節」。仁宗即位,開弘文閣。帝臨幸,問:「今山林亦有名士乎?」楊士奇初不識繼。夏原吉治水蘇、松,得其文,歸以示士奇,士奇心識之。及帝問,遂以繼對。召為國子博士,尋改翰林《五經》博士,直弘文閣。宣宗初,遷檢討。引疾歸,卒。

楊翥,字仲舉,亦吳人。少孤貧,隨兄戍武昌,授徒自給。楊士奇微時,流寄窘乏,翥輒解館舍讓之,而己教授他所。士奇心賢之。及貴,薦翥經明行修。宣宗詔試吏部,稱旨,授翰林院檢討,歷修撰。正統中,詔簡郕王府僚。諸翰林皆不欲行,乃出侍講儀銘及翥為左右長史。久之,引年歸。王即大位,入朝,拜禮部右侍郎。景泰三年進尚書,給祿致仕。明年卒,年八十五。翥篤行絕俗,一時縉紳厚德者,翥為最。既沒,景帝念之,召其子珒入覲,授本邑主簿。

俞山,字積之,秀水人。由鄉舉為郕府伴讀。景帝時,拜吏部右侍郎。而嘉興俞綱由諸生繕寫《實錄》,試中書舍人,授郕府審理。景帝時,以兵部右侍郎入閣預機務。居三日,固辭,守本官。景帝將易東宮,山密疏諫。不聽。懷獻太子立,加太子少傅,山意不自安,致仕去。綱加太子少保。英宗復辟,山以致仕得免。而綱當景泰時,能周旋二帝間,故得調南京禮部。成化初致仕,卒。

潘辰,字時用,景寧人。少孤,隨從父家京師,以文學名。弘治六年詔天下舉才德之士隱於山林者。府尹唐恂舉辰,吏部以辰生長京師,寢之。恂復奏,給事中王綸、夏昂亦交章薦,乃授翰林待詔。久之,掌典籍事。預修《會典》成,進五經博士。正德中,劉瑾摘《會典》小疵,復降為典籍,俄還故官。南京缺祭酒,吏部推石珤及辰。帝以命珤,而擢辰編修。居九年,超擢太常少卿,致仕歸,卒。特賜祭葬。辰居官勤慎,晨入夜歸。典制誥時,有以幣酬者,堅卻之。士大夫重其學行,稱為「南屏先生」。

王英,字時彥,金谿人。永樂二年進士。選庶吉士,讀書文淵閣。帝察其慎密,令與王直書機密文字。與修《太祖實錄》,授翰林院修撰,進侍讀。

二十年,扈從北征。師旋,過李陵城。帝聞城中有石碑,召英往視。既至,不識碑所。而城北門有石出土尺餘。發之,乃元時李陵臺驛令謝某德政碑也,碑陰刻達魯花赤等名氏。具以奏。帝曰:「碑有蒙古名,異日且以為己地,啟爭端。」命再往擊碎之。沉諸河,還奏。帝喜其詳審,曰:「爾是二十八人中讀書者,朕且用爾。」因問以北伐事。英曰:「天威親征,彼必遠遁,願勿窮追。」帝笑曰:「秀才謂朕黷武邪?」因曰:「軍中動靜,有聞即入奏。」且諭中官勿阻。立功官軍有過,命勿與糧,相聚泣。以英奏,復給予。仁宗即位,累進右春坊大學士,乞省親歸。

宣宗立,還朝。是時海內宴安,天子雅意文章,每與諸學士談論文藝,賞花賦詩,禮接優渥。嘗謂英曰:「洪武中,學士有宋濂、吳沉、朱善、劉三吾。永樂初,則解縉、胡廣。汝勉之,毋俾前人獨專其美。」修太宗、仁宗《實錄》成,遷少詹事,賜麒麟帶。母喪,特與葬祭,遣中官護歸。尋起復。正統元年命侍經筵,總裁《宣宗實錄》,進禮部侍郎。八年命理部事。浙江民疫,遣祭南鎮。時久旱,英至,大雨,民呼「侍郎雨」。年七十,再乞休。不許。十二年,英子按察副使裕坐事下獄。英上疏待罪。宥不問。明年進南京禮部尚書,俾就閒逸。居二年卒,年七十五。賜祭葬,謚文安。

英端凝持重,歷仕四朝。在翰林四十餘年,屢為會試考官,朝廷制作多出其手,四方求銘志碑記者不絕。性直諒,好規人過,三楊皆不喜,故不得柄用。裕後累官四川按察使。

錢習禮,名幹,以字行,吉水人。永樂九年進士。選庶吉士,尋授檢討。習禮與練子寧姻戚。既仕,鄉人以奸黨持之,恒惴惴。楊榮乘間言於帝,帝笑曰:「使子寧在,朕猶當用之,況習禮乎。」仁宗即位,遷侍讀,知制誥,以省親歸。

宣德元年修兩朝《實錄》,與侍講陳敬宗、陳循同召還,進侍讀學士。英宗開經筵,為講官。《宣宗實錄》成,擢學士,掌院事。七年以故鴻臚寺為翰林院。落成,諸殿閣大學士皆至,習禮不設楊士奇、楊溥座,曰:「此非三公府也。」士奇等以聞。帝命具座。後遂為故事。

正統九年乞致仕。不許。明年,六部侍郎多闕,帝命吏部尚書王直會大臣推舉,而特旨擢習禮於禮部。習禮力辭。不允。王振用事,達官多造其門,習禮恥為屈。十二年六月復上章乞骸骨,乃得歸。習禮篤行誼,好古秉禮,動有矩則。家居十五年卒,年八十有九。謚文肅。

周敘,字公敘,吉水人。年十一能詩。永樂十六年進士。選庶吉士,作《黃鸚鵡賦》,稱旨,授編修。歷官侍讀,直經筵。正統六年上疏言事,帝嘉納焉。八年夏又上言:「比天旱,陛下責躬虔禱,而臣下不聞效忠補過之言,徒陳情乞用而已。掌銓選者罔論賢否,第循資格。司國計者不問耕桑,惟勤賦斂。軍士困役作,刑罰失重輕,風憲無激揚,言官務緘默。僧道數萬,日耗戶口,流民眾多,莫為矜恤。」帝以章示諸大臣。王直等皆引罪求罷。十一年遷南京侍講學士。

郕王監國,馳疏言:「君父之仇不共戴天,殿下宜臥薪嘗膽,如越之報吳。使智者獻謀,勇者效力,務掃北庭,雪國恥。先遣辯士,卑詞重幣乞還鑾輿,暫為君父屈。」因條上勵剛明、親經史、修軍政、選賢才、安民心、廣言路、謹微漸、修庶政八事。王嘉納之。景泰二年又請復午朝,日接大臣,咨諏治道。經筵之餘,召文學從臣講論政事,并詔天下臣民直言時政缺失。帝因詔求言。

敘負氣節,篤行誼。曾祖以立,在元時以宋、遼、金三史體例未當,欲重修。敘思繼先志,正統末,請於朝。詔許自撰,銓次數年,未及成而卒。

同邑劉儼,字宣化。正統七年進士第一。歷官太常少卿。景泰中,典順天鄉試,黜大學士陳循、王文子,幾得危禍。詳《高穀傳》。天順初,改掌翰林院事,卒官。贈禮部侍郎,謚文介。儼立朝正直,居鄉亦有令德云。

柯潛,字孟時,莆田人。景泰二年舉進士第一。歷洗馬。天順初,遷尚寶少卿,兼修撰。憲宗即位,以舊宮僚擢翰林學士。《英宗實錄》成,進少詹事。慈懿太后之喪,潛與修撰羅璟上章,請合葬裕陵。廷臣相繼爭。未報。潛曰:「朝廷大事,臣子大節,舍是奚所用心。」與璟皆再疏爭,竟得如禮。連遭父母喪,詔起為祭酒,固乞終制。許之。未幾卒。

潛邃於文學,性高介。為學士時,即院中後圃構清風亭,鑿池蒔芙蓉,植二柏於後堂,人稱其亭為「柯亭」,柏為「學士柏」。院中有井,學士劉定之所浚也。柯亭、劉井,翰林中以為美談云。

羅璟,字明仲,泰和人。天順末,進士及第。授編修,進修撰。預修《宋元通鑑綱目》。累官洗馬。孝宗為太子,簡侍講讀。母喪歸。璟與尚書尹旻子侍講龍同娶於孔氏。旻得罪,李孜省指璟為旻黨,調南京禮部員外郎。孝宗嗣位,王恕等言璟才,乃授福建提學副使。弘治五年召為南京祭酒。久之,卒。

孔公恂,字宗文,先聖五十八世孫也。景泰五年舉會試,聞母疾,不赴廷對。帝以問禮部,具言其故,乃遣使召之。日且午,不及備試卷,命翰林院給以筆札。登第,即丁母憂歸。

衍聖公孔彥縉卒,孫弘緒幼弱,詔遣禮部郎治喪,公恂理其家事。天順初,授禮科給事中。弘緒已襲封,大學士李賢妻以女,公恂因得交於賢。賢言:「公恂,大聖人後;贊善司馬恂,宋大賢溫國公光後。宜輔導太子。」帝喜。同日超拜少詹事,侍東宮講讀。入語孝肅皇后曰:「吾今日得聖賢子孫為汝子傅。」孝肅皇后者,憲宗生母,方以皇貴妃有寵。於是具冠服拜謝,宮中傳以為盛事云。

憲宗嗣位,改公恂大理左少卿。公恂言不通法律,乃復少詹事。成化二年上章言兵事,諸武臣嘩然,給事御史交章駁之。下獄,謫漢陽知府。未至,丁父憂。服闋,商輅請復建言得罪者官,乃還故秩,涖南京詹事府。久之,卒。

司馬恂,字恂如,浙江山陰人。正統末,由舉人擢刑科給事中,累遷少詹事。憲宗立,命兼國子祭酒。卒,贈禮部左侍郎。恂強記、敦厚,與物無忤,居官無所表見。

贊曰:「建文之初,修尊賢敬老之節。董倫以宿儒見重,雖寡所表見,當非茍焉已也。儀智父子仍世以儒術進,從容輔導,蓋其賢哉。鄒濟諸人,以宮僚被遇而讒構不免。陳濟輩起布衣,列禁近而善始終,固有幸不幸歟。二周、王英、錢習禮、周敘、柯潛謙和直諒,各著其美,蓋皆異於浮華博習之徒矣。


\end{pinyinscope}