\article{列傳第一百七}

\begin{pinyinscope}
張四維子泰徵甲徵馬自強子怡慥許國趙志皋張位硃賡子敬循

張四維,字子維,蒲州人。嘉靖三十二年進士。改庶吉士,授編修。隆慶初,進右中允,直經筵,尋遷左諭德。四維倜儻有才智,明習時事。楊博、王崇古久歷邊陲,善談兵。四維,博同里而崇古姊子也,以故亦習知邊務。高拱深器之。拱掌吏部,超擢翰林學士。甫兩月,拜吏部右侍郎。俺答封貢議起,朝右持不決。四維為交關於拱,款事遂成。拱益才四維,四維亦干進不已,朝士頗有疾之者。御史郜永春視鹽河東,言鹽法之壞由勢要橫行,大商專利,指四維、崇古為勢要,四維父、崇古弟為大商。四維奏辨,因乞去。拱力護之,溫詔慰留焉。

初,趙貞吉去位,拱欲援四維入閣,而殷士儋夤緣得之,諸人遂相構。及御史趙應龍劾士儋,士儋未去,言路復有劾四維者。四維已進左侍郎,不得已引去,無何士儋亦去。東宮出閣,召四維充侍班官。給事中曹大埜言四維賄拱得召,四維馳疏辨,求罷。帝不許,趣入朝。未至而穆宗崩,拱罷政,張居正當國,復移疾歸。

四維家素封,歲時餽問居正不絕。武清伯李偉,慈聖太后父也,故籍山西,四維結為援。萬曆二年,復召掌詹事府。明年三月,居正請增置閣臣,引薦四維,馮保亦與善,遂以禮部尚書兼東閣大學士入贊機務。當是時,政事一決居正。居正無所推讓,視同列蔑如也。四維由居正進,謹事之,不敢相可否,隨其後,拜賜進官而已。居正卒,四維始當國。累加至少師、吏部尚書、中極殿大學士。

初,四維曲事居正,積不能堪,擬旨不盡如居正意,居正亦漸惡之。既得政,知中外積苦居正,欲大收人心。會皇子生,頒詔天下,疏言:「今法紀修明,海宇寧謐,足稱治平。而文武諸臣,不達朝廷勵精本意,務為促急煩碎,致徵斂無藝,政令乖舛,中外囂然,喪其樂生之心。誠宜及此大慶,蕩滌煩苛,弘敷惠澤,俾四海烝黎,咸戴帝德,此固人心培國脈之要術也。」帝嘉納之。自是,朝政稍變,言路亦發舒,詆居正時事。於是居正黨大懼。王篆、曾省吾輩,厚結申時行以為助。而馮保欲因兩宮徽號封己為伯,惡四維持之。篆、省吾知之,厚賄保,數短四維;而使所善御史曹一夔劾吏部尚書王國光媚四維,拔其中表弟王謙為吏部主事。時行遂擬旨罷國光,並謫謙。四維以帝慰留,復起視事。命甫下,御史張問達復劾四維。四維窘,求保心腹徐爵、張大受賄保,保意稍解。時行乃謫問達於外,以安四維。四維以時行與謀也,卒銜之。已而中官張誠譖保,保眷大衰,四維乃授意門生李植輩發保奸狀。保及篆、省吾皆逐,朝事一大變。於是四維稍汲引海內正人為居正所沉抑者。雖未即盡登用,然力反前事,時望頗屬焉。雲南貢金後期,帝欲罪守土官,又詔取雲南舊貯礦銀二十萬,皆以四維言而止。尋以父喪歸。服將闋,卒。贈太師,謚文毅。

子泰徵、甲徵皆四維柄政時舉進士。泰徵累官湖廣參政,甲徵工部郎中。

馬自強,字體乾,同州人。嘉靖三十二年進士。改庶吉士,授檢討。隆慶中,歷洗馬,直經筵。遷國子祭酒,振飭學政,請寄不行。遷少詹事兼侍讀學士,掌翰林院。

神宗為皇太子出閣,充講官。敷陳明切,遂受眷。及即位,自強已遷詹事,教習庶吉士,乃擢禮部右侍郎,為日講官。尋以左侍郎掌詹事府,直講如故。丁繼母憂歸。服闋,詔以故官協理詹事府。至則遷吏部左侍郎,仍直經筵。甫兩月,遷推禮部尚書。帝遣使詢居正尚書得兼講官否,居正言事繁不得兼。乃用為尚書,罷日講,充經筵講官。

禮官所掌,宗籓事最多,先後條例,自相牴牾,黠吏得恣為奸利。自強擇其當者俾僚吏遵守,諸不可用者悉屏之。每籓府疏至,應時裁決,榜之部門,明示行止,吏無所牟利。龍虎山正一真人,隆慶時已降為提點,奪印敕。至是,張國祥求復故號。自強寢其奏。國祥乃重賄馮保固求復,自強力持不可,卒以中旨許之。初,俺答通貢市,賞有定額,後邊臣徇其求,額漸溢。自強請申故約,濫乞者勿與,歲省費不貲。《世宗實錄》成,加太子少保。

六年三月,居正將歸葬父。念閣臣在鄉里者,高拱與己有深隙,殷士儋多奧援,或乘間以出,惟徐階老易與,擬薦之自代。已遣使報階,既念階前輩,已還,當位其下,乃請增置閣臣。帝即令居正推擇,遂以人望薦自強及所厚申時行。詔加自強太子太保兼文淵閣大學士,與時行並參機務。自強初以救吳中行、趙用賢忤居正,自分不敢望,及制下,人更以是多居正。時呂調陽、張四維先在閣。調陽衰,數寢疾不出,小事四維代擬旨,大事則馳報居正於江陵,聽其裁決。自強雖持正,亦不能有為,守位而已。已,居正還朝,調陽謝政,自強亦得疾卒。詔贈少保,謚文莊,遣行人護喪還。

子怡,舉人,終參議;慥,進士,尚寶卿。

關中人入閣者,自自強始。其後薛國觀繼之。終明世,惟二人。

許國,字維楨,歙縣人。舉鄉試第一,登嘉靖四十四年進士。改庶吉士,授檢討。神宗為太子出閣,兼校書。及即位,進右贊善,充日講官。歷禮部左、右侍郎,改吏部,掌詹事府。十一年四月,以禮部尚書兼東閣大學士入參機務。國與首輔申時行善。以丁此呂事與言者相攻,語侵吳中行、趙用賢,由是物議沸然。已而御史陳性學復摭前事劾國,時行右國,請薄罰性學。國再疏求去,力攻言者。帝命鴻臚宣諭,始起視事。南京給事中伍可受復劾國,帝為謫可受官。國復三疏乞休,語憤激,帝不允。性學旋出為廣東僉事。先是,帝考卜壽宮,加國太子太保,改文淵閣,以雲南功進太子太傅。國以父母未葬,乞歸襄事。帝不允,命其子代。御史馬象乾以劾中官張鯨獲罪,國懇救。帝為霽威受之。十七年,進士薛敷教劾吳時來,南京御史王麟趾、黃仁榮疏論臺規,辭皆侵國。國憤,連疏力詆,並及主事饒伸。伸方攻大學士王錫爵,公議益不直國。國性木強,遇事輒發。數與言者為難,無大臣度,以故士論不附。明年秋,火落赤犯臨洮、鞏昌,西陲震動,帝召對輔臣暖閣。時行言款貢足恃,國謂渝盟犯順,桀驁已極,宜一大創之,不可復羈縻。帝心然國言,而時行為政,不能奪。無何,給事中任讓論國庸鄙。國疏辨,帝奪讓俸。國、時行初無嫌,而時行適為國門生萬國欽所論,讓則時行門生也,故為其師報復云。福建守臣報日本結琉球入寇,國因言:「今四裔交犯,而中外小臣爭務攻擊,致大臣紛紛求去,誰復為國家任事者?請申諭諸臣,各修職業,毋恣胸臆。」帝遂下詔嚴禁。國始終忿疾言者如此。

廷臣爭請冊立,得旨二十年春舉行。十九年秋,工部郎張有德以儀注請,帝怒奪俸。時行適在告,國與王有屏慮事中變,欲因而就之,引前旨力請。帝果不悅,責大臣不當與小臣比。國不自安,遂求去。疏五上,乃賜敕馳傳歸。踰一日,時行亦罷,而冊立竟停。人謂時行以論劾去,國以爭執去,為二相優劣焉。國在閣九年,謙慎自守,故累遭攻擊,不能被以污名。卒,贈太保,謚文穆。

趙志皋,字汝邁,蘭谿人。隆慶二年進士及第,授編修。萬歷初,進侍讀。張居正奪情,將廷杖吳中行、趙用賢。志皋偕張位、習孔教等疏救,格不上,則請以中行等疏宜付史館,居正恚。會星變,考察京朝官,遂出志皋為廣東副使。居三年,再以京察謫其官。居正歿,言者交薦,起解州同知。旋改南京太僕丞,歷國子監司業、祭酒,再遷吏部右侍郎,並在南京。尋召為吏部左侍郎。十九年秋,申時行謝政,薦志皋及張位自代。遂進禮部尚書兼東閣大學士,入參機務。明年春,王家屏罷,王錫爵召未到,志皋暫居首輔。會寧夏變起,兵事多所咨決。主事岳元聲疏論錫爵,中言當事者變亂傾危,為主事諸壽賢、給事中許弘綱所駁。志皋再辨,帝皆不問。二十一年,錫爵還朝,明年五月遂歸,志皋始當國。

遼東失事,詔褫巡撫韓取善職,逮副使馮時泰詔獄,而總兵官楊紹勳止下御史問。給事中吳文梓等論其失平,志皋亦言:「封疆被寇,武臣罪也。今寬紹勛而深罪文吏,武臣益恣,文吏益喪氣。」帝不從,時泰竟謫戍。皇太后誕辰,帝受賀畢,召見輔臣暖閣,志皋論宥御史彭應參。言官乞減織造,志皋等因合詞請。尋極論章奏留中之弊,請盡付諸曹議行。帝惡中官張誠黨霍文炳,以言官不舉發,貶黜者三十餘人。志皋等連疏諫,皆不納。累進少傅,加太子太傅,改建極殿。時兩宮災,彗星見,日食九分有奇,三殿又災,連歲間變異迭出。志皋請下罪己詔,因累疏陳時政缺失。而其大者定國本、罷礦稅諸事,凡十一條。優詔報聞而已。皇長子年十六時,志皋嘗請舉冠婚禮。帝命禮官具儀。及儀上,不果行。二十六年三月,志皋等復以為言,終不允。

張居正柄國,權震主。申時行繼之,勢猶盛。王錫爵性剛負氣,人亦畏之。志皋為首輔,年七十餘,耄矣,柔而懦,為朝士所輕,詬誶四起。其始為首輔也,值西華門災,御史趙文炳論之。無何,南京御史柳佐、給事中章守誠言,吏部郎顧憲成等空司而逐志皋,實激帝怒。已而給事中張濤、楊洵,御史冀體、況上進,南京評事龍起雷相繼披詆。而巡按御史吳崇禮劾其子兩淮運副鳳威,鳳威坐停俸。未幾,工部郎中岳元聲極言志皋宜放,給事中劉道亨詆尤力。志皋憤言:「同一閣臣也,往日勢重而權有所歸,則相率附之以媒進。今日勢輕而權有所分,則相率擊之以博名。」因求退益切。帝慰諭之。

初,日本封貢議起,石星力主之。志皋亦冀無事,相與應和。及封事敗,議者蜂起,凡劾星者必及志皋。志皋每被言,輒疏辨求退,帝悉勉留。先嘗譴言者以謝之,後言者益眾,則多寢不下,而留志皋益堅。迨封事大壞,星坐欺罔下獄論死,位亦以楊鎬故褫官,而志皋終不問。然志皋已病不能視事,乞休疏累上,御史于永清、給事中桂有根復疏論之。志皋身在床褥,於罷礦、建儲諸大政,數力疾草疏爭,帝歲時恩賜亦如故。志皋疾轉篤。在告四年,疏八十餘上。二十九年秋,卒於邸舍。贈太傅,謚文懿。

張位,字明成,新建人。隆慶二年進士。改庶吉士。授編修,預修《世宗實錄》。

萬曆元年,位以前代皆有起居注,而本朝獨無,疏言:「臣備員纂修,竊見先朝政事,自非出於詔令,形諸章疏,悉湮沒無攷。鴻猷茂烈,鬱而未章,徒使野史流傳,用偽亂真。今史官充位,無以自效。宜日分數人入直,凡詔旨起居,朝端政務,皆據見聞書之,待內閣裁定,為他年實錄之助。」張居正善其議,奏行焉。後以救吳中行、趙用賢忤居正意。時已遷侍講,抑授南京司業。未行,復以京察,謫徐州同知。居正卒之明年,用給事中馮景隆、御史孫維城薦,擢南京尚寶丞。俄召為左中允,管司業事,進祭酒。疏陳六事,多議行。以禮部右侍郎。教習庶吉士,引疾歸。詔起故官,協理詹事府,辭不赴。久之,以申時行薦,拜吏部左侍郎兼東閣大學士,與趙志皋並命。

王錫爵還朝,帝適降諭三王並封,以待嫡為辭。而志皋、位遽請帝篤修交泰,早兆高禖,議者竊哂之。趙南星以考察事褫官,朝士詆錫爵者多及位。錫爵去,志皋為首輔。位與志皋相厚善。志皋衰,位精悍敢任,政事多所裁決。時黜陟權盡還吏部,政府不得侵撓。位深憾之,事多掣其肘。以故孫鑨、陳有年、孫丕揚、蔡國珍皆不安其位而去。

二十四年,兩宮災,礦稅議起,位等不能沮。及奸人請稅煤炭,開臨清皇店,位與沈一貫乃執奏不可,不報。明年春,偕一貫陳經理朝鮮事宜。請於開城、平壤建置重鎮,練兵屯田,通商惠工,省中國輸挽。且擇人為長帥,分署朝鮮八道,為持久計。事下朝鮮議。其國君臣慮中國遂并其土,疏陳非便,乃寢。頃之,日本封事壞,位力薦參政楊鎬才,請付以朝鮮軍務。鎬遭父喪,又請奪情視事,且薦邢玠為總督。帝皆從之。位已進禮部尚書,改文淵閣,以甘肅破賊敘功,加太子太保,復以延鎮功,進少保、吏部尚書,改武英殿。

三殿災,志皋適在告,位偕同列請面慰,不許。乃請帝引咎頒赦,勤朝講,發章奏,躬郊廟,建皇儲,錄廢棄,容狂直,寡細過,補缺官,減織造,停礦使,徹稅監,釋繫囚。帝優詔報之,不能盡行。位又言:「臣等請停礦稅,非遽停之也,蓋欲責成撫按,使上不虧國,下不累民耳。」於是給事中張正學劾位逢迎遷就,宜斥。帝亦不省。

位初官翰林,聲望甚重,朝士冀其大用。及入政府,招權示威,素望漸衰。給事中劉道亨劾位奸貪數十事。位憤,力辨,遂落道享三官。呂坤、張養蒙與孫丕揚交好,而沈思孝、徐作、劉應秋、劉楚先、戴士衡、楊廷蘭則與位善,各有所左右。丕揚嘗劾位,指道亨為其黨。道亨恥之,劾位以自解。已而贊畫主事丁應泰劾楊鎬喪師,言位與鎬密書往來,朋黨欺罔,鎬拔擢由賄位得之。帝怒,下廷議。位惶恐奏辨,帝猶慰留。給事中趙完璧、徐觀瀾復交章論。位窘,亟奏:「群言交攻,孤忠可憫。臣心無纖毫愧,惟上矜察。」帝怒曰:「鎬由卿密揭屢薦,故奪哀授任。今乃朋欺隱慝,辱國損威,猶云無媿。」遂奪職閒住。無何,有獲妖書名《憂危竑議》者,御史趙之翰言位實主謀。帝亦疑位怨望有他志,詔除名為民,遇赦不宥。其親故右都御史徐作、侍郎劉楚先、祭酒劉應秋、給事中楊廷蘭、主事萬建崑皆貶黜有差。

位有才,果於自用,任氣好矜。其敗也,廷臣莫之救。既卒,亦無湔雪之者。天啟中,復官,贈太保,謚文莊。

朱賡,字少欽,浙江山陰人。父公節,泰州知州。兄應,刑部主事。賡登隆慶二年進士,改庶吉士,授編修。萬曆六年,以侍讀為日講官。宮中方興土木,治苑囿。賡因講宋史,極言「花石綱」之害,帝為悚然。歷禮部左、右侍郎。帝營壽宮於大峪山,命賡往視。中官示帝意欲仿永陵制,賡言:「昭陵在望,制過之,非所安。」疏入,久不下。已,竟如其言。累官禮部尚書,遭繼母喪去。

二十九年秋,趙志皋卒,沈一貫獨當國,請增置閣臣。帝素慮大臣植黨,欲用林居及久廢者。詔賡以故官兼東閣大學士,參預機務,遣行人召之。再辭,不允。明年四月詣闕,即捐一歲俸助殿工。其秋極陳礦稅之害,帝不能用。既而與一貫及沈鯉共獻守成、遣使、權宜三論,大指為礦稅發,賡手筆也。賡於已邸門獲妖書,而書辭誣賡動搖國本,大懼。立以疏聞,乞避位。帝慰諭有加。一貫倡群小窮治不已,賡在告,再貽書一貫,請速具獄,無株連,事乃得解。

三十三,年大計京官。帝留被察者錢夢皋輩,及南京察疏上,亦欲有所留。賡力陳不可,曰:「北察之留,旨從中出,人猶咎臣等。今若出自票擬,則二百餘年大典,自臣壞之,死不敢奉詔。」言官劾溫純及鯉,中使傳帝意欲去純。賡言大臣去國必採公論,豈可於劾疏報允。帝下南察疏,而純竟去。其冬,工部請營三殿。時方浚河、繕城,賡力請俟之異日。帝皆納之,不果行。

三十四年,一貫、鯉去位,賡獨當國,年七十有二矣。朝政日弛,中外解體。賡疏揭月數上,十不能一下。御史宋壽首諷切賡,給事中汪若霖繼之。賡緣二人言,力請帝更新庶政,於增閣臣、補大僚、充言路三事語尤切。帝優詔答之而不行。賡乃素服詣文華門懇請,終不得命。賡以老,屢引疾,閣中空無人。帝諭簡閣臣,而廷臣慮帝出中旨如往年趙志皋、張位故事。賡力疾請付廷推,乃用于慎行、李廷機、葉向高,而召王錫爵於家,以為首輔。給事中王元翰、胡忻以廷機之用,賡實主之,疏詆廷機,並侵賡。賡疏辭,帝為切責言者。既而姜士昌及燾被謫,言路謂出賡意,益不平。禮部主事鄭振先遂劾賡十二大罪,且言賡與一貫、錫爵為過去、見在、未來三身。帝怒,貶振先三秩。俄以言官論救,再貶二秩。

先,考選科道,吏部擬上七十八人。候命踰年,不下,賡連疏趣之。三十六年秋,命始下。諸人列言路,方欲見風采,而給事中若霖先嘗忤賡,及是見黜,適當賡病起入直時。眾謂賡修郤,攻訐四起,先後疏論至五十餘人。給事中喻安性者,賡里人,為賡上疏言:「今日政權不由內閣,盡移於司禮。」言者遂交章劾安性,復侵賡。是時賡已寢疾,乞休疏二十餘上。言者慮其復起,攻不已,而賡以十一月卒於官。遺疏陳時政,語極悲切。賡先加少保兼太子太保,進吏部尚書、文華殿大學士。及卒,贈太保,謚文懿。御史鼓端吾復疏詆賡,給事中胡忻請停其贈謚,帝不聽。

賡醇謹無大過,與沈一貫同鄉相比,暱給事中陳治則、姚文蔚等,以故蒙詬病云。

子敬循,官禮部郎中,改稽勛。前此無正郎改吏部者,自敬循始。終右通政。

贊曰:四維等當軸處中,頗滋物議。其時言路勢張,恣為抨擊。是非瞀亂,賢否混淆,群相敵仇,罔顧國是。詬誶日積,又烏足為定論乎。然謂光明磊落有大臣之節,則斯人亦不能無愧辭焉。


\end{pinyinscope}