\article{河渠志}


○黃河上

黃河,自唐以前,皆北入海。宋熙寧中,始分趨東南,一合泗入淮,一合濟入海。金明昌中,北流絕,全河皆入淮,元潰溢不時,至正中受害尤甚,濟寧、曹、鄆間,漂沒千餘里。賈魯為總制,導使南,匯淮入海。

明洪武元年決曹州雙河口,入魚臺。徐達方北征,乃開塌場口,引河入泗以濟運,而徙曹州治於安陵。塌場者,濟寧以西、耐牢坡以南直抵魚臺南陽道也。八年,河決開封太黃寺堤。詔河南參政安然發民夫三萬人塞之。十四年決原武、祥符、中牟,有司請興築。帝以為天災,令護舊堤而已。十五年春,決朝邑。七月決滎澤、陽武。十七年決開封東月堤,自陳橋至陳留橫流數十里。又決杞縣,入巴河。遣官塞河,蠲被災租稅。二十二年,河沒儀封,徙其治於白樓村。二十三年春,決歸德州東南鳳池口,逕夏邑、永城。發興武等十衛士卒,與歸德民併力築之。罪有司不以聞者。其秋,決開封西華諸縣,漂沒民舍。遣使振萬五千七百餘戶。二十四年四月,河水暴溢,決原武黑洋山,東經開封城北五里,又東南由陳州、項城、太和、潁州、潁上,東至壽州正陽鎮,全入於淮。而賈魯河故道遂淤。又由舊曹州、鄆城兩河口漫東平之安山,元會通河亦淤。明年復決陽武,汜陳州、中牟、原武、封丘、祥符、蘭陽、陳留、通許、太康、扶溝、杞十一州縣,有司具圖以聞。發民丁及安吉等十七衛軍士修築。其冬大寒,役遂罷。三十年八月決開封,城三面受水。詔改作倉庫於滎陽高阜,以備不虞。冬,蔡河徙陳州。先是,河決,由開封北東行,至是下流淤,又決而之南。

永樂三年,河決溫縣堤四十丈,濟、澇二水交溢,淹民田四十餘里,命修堤防。四年修陽武黃河決岸。八年伙,河決開封,壞城二百餘丈,民被患者萬四千餘戶,沒田七千五百餘頃。帝以國家籓屏地,特遣侍郎張信往視。信言:「祥符魚王口至中灤下二十餘里,有舊黃河岸,與今河面平。濬而通之,使循故道,則水勢可殺。」因繪圖以進。時尚書宋禮、侍郎金純方開會通河。帝乃發民丁十萬,命興安伯徐亨、侍郎蔣廷瓚偕純相治,併令禮總其役。九年七月,河復故道,自封丘金龍口,下魚臺塌場,會汶水,經徐、呂二洪南入於淮。是時,會通河已開,黃河與之合,漕道大通,遂議罷海運,而河南水患亦稍息。已而決陽武中鹽堤,漫中牟、祥符、尉氏。工部主事蘭芳按視,言:「堤當急流之衝,夏秋泛漲,勢不可驟殺。宜捲土樹椿以資捍禦,無令重為民患而已。」又言:「中灤導河分流,使由故道北入海,誠萬世利。但緣河堤埽,止用蒲繩泥草,不能持久。宜編木為囤,填石其中,則水可殺,堤可固。」詔皆從其議。十四年決開封州縣十四,經懷遠,由渦河入於淮。二十年,工部以開封土城堤數潰,請濬其東故道。報可。

宣德元年霪雨,溢開封州縣十。三年,以河患,徙靈州千戶所於城東。六年從河南布政使言,濬祥符抵儀封黃陵岡淤道四百五十里。是時,金龍口漸淤,而河復屢溢開封。十年從御史李懋言,浚金龍口。

正統二年築陽武、原武、滎澤決岸。又決濮州、范縣。三年,河復決陽武及邳州,灌魚臺、金鄉、嘉祥。越數年,又決金龍口、陽穀堤及張家黑龍廟口,而徐、呂二洪亦漸淺,太黃寺巴河分水處,水脈微細。十三年方從都督同知武興言,發卒疏濬。而陳留水夏漲,決金村堤及黑潭南岸。築垂竣,復決。其秋,新鄉八柳樹口亦決,漫曹、濮,抵東昌,衝張秋,潰壽張沙灣,壞運道,東入海。徐、呂二洪遂淺澀。命工部侍郎王永和往理其事。永和至山東,修沙灣未成,以冬寒停役。且言河決自衛輝,宜敕河南守臣修塞。帝切責之,令山東三司築沙灣,趣永和塞河南八柳樹,疏金龍口,使河由故道。明年正月,河復決聊城。至三月,永和浚黑洋山西灣,引其水由太黃寺以資運河。修築沙灣堤大半,而不敢盡塞,置分水閘,設三空放水,自大清河入海。且設分水閘二空於沙灣西岸,以泄上流,而請停八柳樹工。從之。是時,河勢方橫溢,而分流大清,不耑向徐、呂。徐、呂益膠淺,且自臨清以南,運道艱阻。

景泰二年特敕山東、河南巡撫都御史洪英、王暹協力合治,務令水歸漕河。暹言:「黃河自陜州以西,有山峽,不能為害;陜州以東,則地勢平緩,水易泛溢,故為害甚多。洪武二十四年改流,從汴梁北五里許,由鳳陽入淮者為大黃河。其支流出徐州以南者為小黃河,以通漕運。自正統十三年以來,河復故道,從黑洋山後徑趨沙灣入海明,但存小黃河從徐州出。岸高水低,隨浚隨塞,以是徐州之南不得飽水。臣自黑洋山東南抵徐州,督河南三司疏浚。臨清以南,請以責英。」未幾,給事中張文質劾暹、英治水無績,請引塌場水濟徐、呂二洪,浚潘家渡以北支流,殺沙灣水勢。且開沙灣浮橋以西河口,築閘引水,以灌臨清,而別命官以責其成。詔不允,仍命暹、英調度。

時議者謂:「沙灣以南地高,水不得南入運河。請引耐牢坡水以灌運,而勿使經沙灣,別開河以避其衝決之勢。」或又言:「引耐牢坡水南去,則自此以北枯澀矣。」甚者言:「沙灣水湍急,石鐵沉下若羽,非人力可為。宜設齊醮符咒以禳之。」帝心甚憂念,命工部尚書石璞往治,而加河神封號。

璞至,濬黑洋山至徐州以通漕,而沙灣決口如故。乃命中官黎賢、阮洛,御史彭誼協治。璞等築石堤於沙灣,以禦決河,開月河二,引水以益運河,且殺其決勢。三年五月,河流漸微細,沙灣堤始成。乃加璞太子太保,而於黑洋山、沙灣建河神二新廟,歲春秋二祭。六月,大雨浹旬,復決沙灣北岸,掣運河之水以東,近河地皆沒。命英督有司修築。復敕中官黎賢、武艮,工部侍郎趙榮往治。四年正月,河復決新塞口之南,詔復加河神封號。至四月,決口乃塞。五月,大雷雨,復決沙灣北岸,掣運河水入鹽河,漕舟盡阻。帝復命璞往。乃鑿一河,長三里,以避決口,上下通運河,而決口亦築壩截之,令新河、運河俱可行舟。工畢奏聞。帝恐不能久,令璞且留處置,而命諭德徐有貞為僉都御史耑治沙灣。

時河南水患方甚,原武、西華皆遷縣治以避水。巡撫暹言:「黃河舊從開封北轉流東南入淮,不為害。自正統十三年改流為二。一自新鄉入柳樹,由故道東經延津、封丘入沙灣。一決滎澤,漫流原武,抵祥符、扶溝、通許、洧川、尉氏、臨潁、郾城、陳州、商水、西華、項城、太康。沒田數十萬頃,而開封患特甚。雖嘗築大小堤於城西,皆三十餘里,然沙土易壞,隨築隨決,小堤已沒,大堤復壞其半。請起軍民夫協築,以防後患。」帝可其奏。太僕少卿黃仕人雋亦言:「河分兩派,一自滎澤南流入項城,一自新鄉八柳樹北流,入張秋會通河,並經六七州縣,約二千餘里。民皆蕩析離居,而有司猶徵其稅。乞敕所司覆視免徵。」帝亦可其奏。巡撫河南御史張瀾又言:「原武黃河東岸嘗開二河,合黑洋山舊河道引水濟徐、呂。今河改決而北,二河淤塞不通,恐徐、呂乏水,必妨漕運,黑洋山北,河流稍紆迴,請因決口改挑一河以接舊道,灌徐、呂。」帝亦從之。

有貞至沙灣,上治河三策:「一置水閘門。臣聞水之性可使通流,不可使堙塞。禹鑿龍門,闢伊闕,為疏導計也。故漢武堙瓠子終弗成功,漢明疏汴河踰年著績。今談治水者甚眾,獨樂浪王景所述制水門之法可取。蓋沙灣地土皆沙,易致坍決,故作壩作閘皆非善計。請依景法損益其間,置閘門於水,而實其底,令高常水五尺。小則拘之以濟運,大則疏之使趨海,則有通流之利,無堙塞之患矣。一開分水河。凡水勢大者宜分,小者宜合。今黃河勢大恒衝決,運河勢小恒乾淺,必分黃水合運河,則有利無害。請度黃河可分之地,開廣濟河一道,下穿濮陽、博陵及舊沙河二十餘里,上連東、西影塘及小嶺等地又數十里,其內則有古大金堤可倚以為固,其外有八百里梁山泊可恃以為泄。至新置二閘亦頗堅牢,可以宣節,使黃河水大不至泛溢為害,小亦不至乾淺以阻漕運。」其一挑深運河。帝諭有貞,如其議行之。

有貞乃踰濟、汶,沿衛、沁,循大河,道濮、范,相度地形水勢,上言:「河自雍而豫,出險固而之夷斥,水勢既肆。由豫而兗,土益疏,水益肆。而沙灣之東,所謂大洪口者,適當其衝,於是決焉,而奪濟、汶入海之路以去。諸水從之而洩,堤以潰,渠以淤,澇則溢,旱則涸,漕道由此阻。然驟而堰之,則潰者益潰,淤者益淤。今請先疏其水,水勢平乃治其決,決止乃浚其淤。」於是設渠以疏之,起張秋金堤之首,西南行九里至濮陽濼,又九里至博陵陂,又六里至壽張之沙河,又八里至東、西影塘,又十有五里至白嶺灣,又三里至李鞬,凡五十里。由李鞬而上二十里至竹口蓮花池,又三十里至大瀦潭,乃踰范暨濮,又上而西,凡數百里,經澶淵以接河、沁,築九堰以禦河流旁出者,長各萬丈,實之石而鍵以鐵。六年七月,功成,賜渠名廣濟。沙灣之決垂十年,至是始塞。亦會黃河南流入淮,有貞乃克奏功。凡費木鐵竹石累數萬,夫五萬八千有奇,工五百五十餘日。自此河水北出濟漕,而阿、鄄、曹、鄆間田出沮洳者,百數十萬頃。乃浚漕渠,由沙灣北至臨清,南抵濟寧,復建八閘於東昌,用王景制水門法以平水道,而山東河患息矣。

七年夏,河南大雨,河決開封、河南、彰德。其秋,畿輔、山東大雨,諸水並溢,高地丈餘,堤岸多衝決。仍敕有貞修築。未幾,事竣,還京入見。獎勞甚至,擢副都御史。

天順元年修祥符護城大堤。五年七月,河決汴梁土城,又決磚城,城中水丈餘,壞官民舍過半。周王府宮人及諸守土官皆乘舟筏以避,軍民溺死無算。襄城亦決縣城。命工部侍郎薛遠往視,恤災戶、蠲田租,公廨民居以次修理。明年二月,開祥符曹家溜,河勢稍平。

七年春,河南布政司照磨金景輝考滿至京,上言:「國初,黃河在封丘,後徙康王馬頭,去城北三十里,復有二支河:一由沙門注運河,一由金龍口達徐、呂入海。正統戊辰,決滎澤,轉趨城南,并流入淮,舊河、支河俱堙,漕河因而淺澀。景泰癸酉,因水迫城,築堤四十里,勞費過甚,而水發輒潰,然尚未至決城壕為人害也。至天順辛巳,水暴至,土城磚城並圮,七郡財力所築之堤,俱委諸無用,人心惶惶,未知所底。夫河不循故道,併流入淮,是為妄行。今急宜疏導以殺其勢。若止委之一淮,而以堤防為長策,恐開封終為魚鱉之區。乞敕部檄所司,先疏金龍口寬闊以接漕河,然後相度舊河或別求泄水之地,挑浚以平水患,為經久計。」命如其說行之。

成化七年命王恕為工部侍郎,奉敕總理河道。總河侍郎之設,自恕始也。時黃河不為患,恕耑力漕河而已。

十四年,河決開封,壞護城堤五十丈。巡撫河南都御史李衍言:「河南累有河患,皆下流壅塞所致。宜疏開封西南新城地,下抵梁家淺舊河口七里壅塞,以洩杏花營上流。又自八角河口直抵南頓,分導散漫,以免祥符、鄢陵、睢、陳、歸德之災。乃敕衍酌行之。明年正月遷滎澤縣治以避水,而開封堤不久即塞。

弘治二年五月,河決開封及金龍口,入張秋運河,又決埽頭五所入沁。郡邑多被害,汴梁尤甚,議者至請遷開封城以避其患。布政司徐恪持不可,乃止。命所司大發卒築之。九月命白昂為戶部侍郎,修治河道,賜以特敕,令會山東、河南、北直隸三巡撫,自上源決口至運河,相機修築。

三年正月,昂上言:「臣自淮河相度水勢,抵河南中牟等縣,見上源決口,水入南岸者十三,入北岸者十七。南決者,自中牟楊橋至祥符界析為二支:一經尉氏等縣,合潁水,下塗山,入於淮;一經通許等縣,入渦河,下荊山,入於淮。又一支自歸德州通鳳陽之亳縣,亦合渦河入於淮。

北決者,自原武經陽武、祥符、封丘、蘭陽、儀封、考城,其一支決入金龍等口,至山東曹州,衝入張秋漕河。去冬,水消沙積,決口已淤,因併為一大支,由祥符翟家口合沁河,出丁家道口,下徐州。此河流南北分行大勢也。合潁、渦二水入淮者,各有灘磧,水脈頗微,宜疏濬以殺河勢。合沁水入徐者,則以河道淺隘不能受,方有漂沒之虞。況上流金龍諸口雖暫淤,久將復決,宜於北流所經七縣,築為堤岸,以衛張秋。但原敕治山東、河南、北直隸,而南直隸淮、徐境,實河所經行要地,尚無所統。」於是併以命昂。

昂舉郎中婁性協治,乃役夫二十五萬,築陽武長堤,以防張秋。引中牟決河出滎澤陽橋以達淮,浚宿州古汴河以入泗,又浚睢河自歸德飲馬池,經符離橋至宿遷以會漕河,上築長堤,下修減水閘。又疏月河十餘以洩水,塞決口三十六,使河流入汴,汴入睢,睢入泗,泗入淮,以達海。水患稍寧。昂又以河南入淮非正道,恐卒不能容,復於魚臺、德州、吳橋修古長堤;又自東平北至興濟鑿小河十二道,入大清河及古黃河以入海。河口各建石堰,以時啟閉。蓋南北分治,而東南則以疏為主云。

六年二月以劉大夏為副都御史,治張秋決河。先是,河決張秋戴家廟,掣漕河與汶水合而北行,遣工部侍郎陳政督治。政言:「河之故道有二:一在滎澤孫家渡口,經朱仙鎮直抵陳州;一在歸德州飲馬池,與亳州地相屬。舊俱入淮,今已淤塞,因致上流衝激,勢盡北趨。自祥符孫家口、楊家口、車船口,蘭陽銅瓦廂決為數道,俱入運河。於是張秋上下勢甚危急,自堂邑至濟寧堤岸多崩圮,而戴家廟減水閘淺隘不能洩水,亦有衝決。請濬舊河以殺上流之勢,塞決河以防下流之患。」政方漸次修舉,未幾卒官。帝深以為憂,命廷臣會薦才識堪任者。僉舉大夏,遂賜敕以往。

十二月,巡按河南御史塗昇言:「黃河為患,南決病河南,北決病山東。昔漢決酸棗,復決瓠子;宋決館陶,復決澶州;元決汴梁,復決蒲口。然漢都關中,宋都大梁,河決為患,不過瀕河數郡而已。今京師專藉會通河歲漕粟數百萬石,河決而北,則大為漕憂。臣博採與論,治河之策有四:

「一曰疏浚。滎、鄭之東,五河之西,飲馬、白露等河皆黃河由渦入淮之故道。其後南流日久,或河口以淤高不洩,或河身狹隘難容,水勢無所分殺,遂泛濫北決。今惟𧾷麗上流東南之故道,相度疏浚,則正流歸道,餘波就壑,下流無奔潰之害,北岸無衝決之患矣。二曰扼塞。既殺水勢於東南,必須築堤岸於西北。黃陵岡上下舊堤缺壞,當度下流東北形勢,去水遠近,補築無遺,排障百川悉歸東南,由淮入海,則張秋無患,而漕河可保矣。」三曰用人,薦河南僉事張鼐。四曰久任,則請專信大夏,且於歸德或東昌建公廨,令居中裁決也。帝以為然。

七年五月命太監李興、平江伯陳銳往同大夏共治張秋。十二月築塞張秋決口工成。初,河流湍悍,決口闊九十餘丈,大夏行視之,曰:「是下流未可治,當治上流。」於是即決口西南開越河三里許,使糧運可濟,乃浚儀封黃陵岡南賈魯舊河四十餘里,由曹出徐,以殺水勢。又浚孫家渡口,別鑿新河七十餘里,導使南行,由中牟、潁川東入淮。又浚祥符四府營淤河,由陳留至歸德分為二。一由宿遷小河口,一由亳渦河,俱會於淮。然後沿張秋兩岸,東西築臺,立表貫索,聯巨艦穴而窒之,實以土。至決口,去窒沉艦,壓以大埽,且合且決,隨決隨築,連晝夜不息。決既塞,繚以石堤,隱若長虹,功乃成。帝遣行人齎羊酒往勞之,改張秋名為安平鎮。

大夏等言:「安平鎮決口已塞,河下流北入東昌、臨清至天津入海,運道已通,然必築黃陵岡河口,導河上流南下徐淮,庶可為運道久安之計。」廷議如其言。乃以八年正月築塞黃陵岡及荊隆等口七處,旬有五日而畢。蓋黃陵岡居安平鎮之上流,其廣九十餘丈,荊隆等口又居黃陵岡之上流,其廣四百三十餘丈。河流至此寬漫奔放,皆喉襟重地。諸口既塞,於是上流河勢復歸蘭陽、考城,分流逕徐州、歸德、宿遷,南入運河,會淮水,東注於海,南流故道以復。而大名府之長堤,起胙城,歷滑縣、長垣、東明、曹州、曹縣抵虞城,凡三百六十里。其西南荊隆等口新堤起於家店,歷銅瓦廂、東橋抵小宋集,凡百六十里。大小二堤相翼,而石壩俱培築堅厚,潰決之患於是息矣。帝以黃陵岡河口功成,敕建黃河神祠以鎮之,賜額曰昭應。其秋,召大夏等還京。荊隆即金龍也。

十一年,河決歸德。管河工部員外郎謝緝言:黃河一支,先自徐州城東小浮橋流入漕河,南抵邳州、宿遷。今黃河上流於歸德州小壩子等處衝決,與黃河別支會流,經宿州、睢寧,由宿遷小河口流入漕河。於是小河口北抵徐州水流漸細,河道淺阻。且徐、呂二洪,惟賴沁水接濟,自沁源、河內、歸德至徐州小浮橋流出,雖與黃河異源,而比年河、沁之流合而為一。今黃河自歸德南決,恐牽引沁水俱往南流,則徐、呂二洪必至淺阻。請亟塞歸德決口,遏黃水入徐以濟漕,而挑沁水之淤,使入徐以濟徐、呂,則水深廣而漕便利矣。」帝從其請。

未幾,河南管河副使張鼐言:「臣嘗請修築侯家潭口決河,以濟徐、呂二洪。今自六月以來,河流四溢,潭口決齧彌深,工費浩大,卒難成功。臣嘗行視水勢,荊隆口堤內舊河通賈魯河,由丁家道口下徐、淮,其迹尚在。若於上源武陟木欒店別鑿一渠,下接荊隆口舊河,俟河流南遷,則引之入渠,庶沛然之勢可接二洪,而糧運無所阻矣。」帝為下其議於總漕都御史李蕙。

越二歲,兗州知府龔弘上言:』副使鼐見河勢南行,欲自荊隆口分沁水入賈魯河,又自歸德西王牌口上下分水亦入賈魯河,俱由丁家道口入徐州。但今秋水從王牌口東行,不由丁家口而南,顧逆流東北至黃陵岡,又自曹縣入單,南連虞城。乞令守臣亟建疏浚修築之策。」於是河南巡撫都御史鄭齡言:「徐、呂二洪藉河、沁二水合流東下,以相接濟。今丁家道口上下河決堤岸者十有二處,共闊三百餘丈,而河淤三十餘里。上源奔放,則曹、單受害,而安平可虞;下流散溢,則蕭、碭被患,而漕流有阻。浚築誠急務也。」部覆從之,乃修丁家口上下堤岸。

初,黃河自原武、滎陽分而為三:一自亳州、鳳陽至清河口,通淮入海;一自歸德州過丁家道口,抵徐州小浮橋;一自窪泥河過黃陵岡,亦抵徐州小浮橋,即賈魯河也。迨河決黃陵岡,犯張秋,北流奪漕,劉大夏往塞之,仍出清河口。十八年,河忽北徙三百里,至宿遷小河口。正德三年又北徙三百里,至徐州小浮橋。四年六月又北徙一百二十里,至沛縣飛雲橋,俱入漕河。

是時,南河故道淤塞,水惟北趨,單、豐之間河窄水溢,決黃陵岡、尚家等口,曹、單田廬多沒,至圍豐縣城郭,兩岸闊百餘里。督漕及山東鎮巡官恐經鉅野、陽穀故道,則奪濟寧、安平運河,各陳所見以請。議未定。明年九月,河復衝黃陵岡,入賈魯河,泛溢橫流,直抵豐、沛。御史林茂達亦以北決安平鎮為虞,而請濬儀封、考城上流故道,引河南流以分其勢,然後塞決口,築故堤。

工部侍郎崔巖奉命修理黃河,浚祥符董盆口、滎澤孫家渡,又濬賈魯河及亳州故河各數十里,且築長垣諸縣決口及曹縣外堤、梁靖決口。功未就而驟雨,堤潰。巖上疏言:「河勢衝蕩益甚,且流入王子河,亦河故道,若非上流多殺水勢,決口恐難卒塞。莫若於曹、單、豐、沛增築堤防,毋令北徙,庶可護漕。」且請別命大臣知水利者共議。於是帝責巖治河無方,而以侍郎李堂代之。堂言:「蘭陽、儀封、考城故道淤塞,故河流俱入賈魯河,經黃陵岡至曹縣,決梁靖、楊家二口。侍郎巖亦嘗修浚,緣地高河澱,隨浚隨淤,水殺不多,而決口又難築塞。今觀梁靖以下地勢最卑,故眾流奔注成河,直抵沛縣,藉令其口築成,而容受全流無地,必致迴激黃陵岡堤岸,而運道妨矣。至河流故道,堙者不可復疏,請起大名三春柳至沛縣飛雲橋,築堤三百餘里,以障河北徙。」從之。六年二月,功未竣,堂言:「陳橋集、銅瓦廂俱應增築,請設副使一人耑理。」會河南盜起,召堂還京,命姑已其不急者。遂委其事於副使,而堤役由此罷。

八年六月,河復決黃陵岡。部議以其地界大名、山東、河南,守土官事權不一,請耑遣重臣,乃命管河副都御史劉愷兼理其事。愷奏,率眾祭告河神,越二日,河已南徙。尚書李鐩因請祭河,且賜愷羊酒。愷於治河束手無策,特歸功於神。曹、單間被害日甚。

世宗初,總河副都御史龔弘言:「黃河自正德初載,變遷不常,日漸北徙。大河之水合成一派,歸入黃陵岡前乃折而南,出徐州以入運河。黃陵歲初築三埽,先已決去其二,懼山、陜諸水橫發,加以霖潦,決而趨張秋,復由故道入海。臣嘗築堤,起長垣,由黃陵岡抵山東楊家口,延袤二百餘里。今擬距堤十里許再築一堤,延袤高廣如之。即河水溢舊堤,流至十里外,性緩勢平,可無大決。」從之。自黃陵岡決,開封以南無河患,而河北徐、沛諸州縣河徙不常。

嘉靖五年,督漕都御史高友璣請濬山東賈魯河、河南鴛鴦口,分洩水勢,毋偏害一方。部議恐害山東、河南,不允。其冬,以章拯為工部侍郎兼僉都御史治河。

先是,大學士費宏言:「河入汴梁以東分為三支,雖有衝決,可無大害。正德末,渦河等河日就淤淺,黃河大股南趨之勢既無所殺,乃從蘭陽、考城、曹、濮奔赴沛縣飛雲橋及徐州之溜溝,悉入漕河,泛溢彌漫,此前數年河患也。近者,沙河至沛縣浮沙湧塞,官民舟楫悉取道昭陽湖。春夏之交,湖面淺涸,運道必阻,渦河等河必宜亟濬。」御史戴金言:「黃河入淮之道有三:自中牟至荊山合長淮曰渦河;自開封經葛岡小壩、丁家道口、馬牧集鴛鴦口至徐州小浮橋口曰汴河;自小壩經歸德城南飲馬池抵文家集,經夏邑至宿遷曰白河。弘治間,渦、白上源堙塞,而徐州獨受其害。宜自小壩至宿遷小河併賈魯河、鴛鴦口、文家集壅塞之處,盡行疏通,則趨淮之水不止一道,而徐州水患殺矣。」御史劉欒言:「曹縣梁靖口南岸,舊有賈魯河,南至武家口十三里,黃沙淤平,必宜開浚。武家口下至馬牧集鴛鴦口百十七里,即小黃河舊通徐州故道,水尚不涸,亦宜疏通。」督漕總兵官楊宏亦請疏歸德小壩、丁家道口、亳州渦河、宿遷小河。友璣及拯亦屢以為言。俱下工部議,以為濬賈魯故道,開渦河上源,功大難成,未可輕舉,但議築堤障水,俾入正河而已。

是年,黃河上流驟溢,東北至沛縣廟道口,截運河,注雞鳴臺口,入昭陽湖。汶、泗南下之水從而東,而河之出飛雲橋者漫而北,淤數十里,河水沒豐縣,徙治避之。

明年,拯言:「滎澤北孫家渡、蘭陽北趙皮寨,皆可引水南流,但二河通渦,東入淮,又東至鳳陽長淮衛,經壽春王諸園寢,為患叵測。惟寧陵北坌河一道,通飲馬池,抵文家集,又經夏邑至宿州符離橋,出宿遷小河口,自趙皮寨至文家集,凡二百餘里,浚而通之,水勢易殺,而園寢無患。」乃為圖說以聞。命刻期舉工。而河決曹、單、城武楊家、梁靖二口、吳士舉莊,衝入雞鳴臺,奪運河,沛地淤填七八里,糧艘阻不進。御史吳仲以聞,因劾拯不能辦河事,乞擇能者往代。其冬,以盛應期為總督河道右都御史。

是時,光祿少卿黃綰、詹事霍韜、左都御史胡世寧、兵部尚書李承勛各獻治河之議。綰言:「漕河資山東泉水,不必資黃河,莫若浚兗、冀間兩高中低之地,道河使北,至直沽入海。」

韜言:「議者欲引河自蘭陽注宿遷。夫水溢徐、沛,猶有二洪為之束捍,東北諸山亙列如垣,有所底極,若道蘭陽,則歸德、鳳陽平地千里,河勢奔放,數郡皆壑,患不獨徐、沛矣。按衛河自衛輝汲縣至天津入海,猶古黃河也。今宜於河陰、原武、懷、孟間,審視地形,引河水注於衛河,至臨清、天津,則徐、沛水勢可殺其半。且元人漕舟涉江入淮,至封丘北,陸運百八十里至淇門,入御河達京師。御河即衛河也。今導河注衛,冬春溯衛河沿臨清至天津,夏秋則由徐、沛,此一舉而運道兩得也。」

世寧言:「河自汴以來,南分二道:一出汴城西滎澤,經中牟、陳、潁,至壽州入淮;一出汴城東祥符,經陳留、亳州,至懷遠入淮。其東南一道自歸德、宿州,經虹縣、睢寧,至宿遷出其東,分五道:一自長垣、曹、鄆至陽穀出;一自曹州雙河口至魚臺塌場口出;一自儀封、歸德至徐州小浮橋出;一自沛縣南飛雲橋出;一自徐、沛之中境山、北溜溝出。六道皆入漕河,而南會於淮。今諸道皆塞,惟沛縣一道僅存。合流則水勢既大,河身亦狹不能容,故溢出為患。近又漫入昭陽湖,以致流緩沙壅。宜因故道而分其勢,汴西則浚孫家渡抵壽州以殺上流,汴東南出懷遠、宿遷及正東小浮橋、溜溝諸道,各宜擇其利便者,開濬一道,以洩下流。或修武城南廢堤,抵豐、單接沛北廟道口,以防北流。此皆治河急務也。至為運道計,則當於湖東滕、沛、魚臺、鄒縣間獨山、新安社地別鑿一渠,南接留城,北接沙河,不過百餘里。厚築西岸以為湖障,令水不得漫,而以一湖為河流散漫之區,乃上策也。」

承勛言:「黃河入運支流有六。自渦河源塞,則北出小黃河、溜溝等處,不數年諸處皆塞,北併出飛雲橋,於是豐、沛受患,而金溝運道遂淤。然幸東面皆山,猶有所障,故昭陽湖得通舟。若益徙而北,則徑奔入海,安平鎮故道可慮,單縣、穀亭百萬生靈之命可虞。又益北,則自濟寧至臨清運道諸水俱相隨入海,運何由通?臣愚以為相六道分流之勢,導引使南,可免衝決,此下流不可不疏浚也。欲保豐、沛、單縣、穀亭之民,必因舊堤築之,堤其西北使毋溢出,此上流不可不堤防也。」

其論昭陽湖東引水為運道,與世寧同。乃下總督大臣會議。

七年正月,應期奏上,如世寧策,請於昭陽湖東改為運河。會河決,於廟道口三十餘里,乃別遣官浚趙皮寨,孫家渡,南、北溜溝以殺上流,隄武城迤西至沛縣南,以防北潰。會旱災修省,言者請罷新河之役,乃召應期還京,以工部侍郎潘希曾代。希曾抵官,言:「邇因趙皮寨開浚未通,疏孫家渡口以殺河勢,請敕河南巡撫潘塤督管河副使,刻期成功。」帝從其奏。希曾又言:「漕渠廟道口以下忽淤數十里者,由決河西來橫衝口上,並掣閘河之水東入昭陽湖,致閘水不南,而飛雲橋之水時復北漫故也。今宜於濟、沛間加築東堤,以遏入湖之路,更築西堤以防黃河之衝,則水不散緩,而廟道口可永無淤塞之虞。」帝亦從之。

八年六月,單、豐、沛三縣長堤成。九年五月,孫家渡河堤成。逾月,河決曹縣。一自胡村寺東,東南至賈家壩入古黃河,由丁家道口至小浮橋入運河。一自胡村寺東北,分二支:一東南經虞城至碭山,合古黃河出徐州;一東北經單縣長堤抵魚臺,漫為坡水,傍穀亭入運河。單、豐、沛三縣長堤障之,不為害。希曾上言:「黃由歸德至徐入漕,故道也。永樂間,浚開封支河達魚臺入漕以濟淺。自弘治時,黃河改由單、豐出沛之飛雲橋,而歸德故道始塞,魚臺支河亦塞。今全河復其故道,則患害已遠,支流達於魚臺,則淺涸無虞,此漕運之利,國家之福也。」帝悅,下所司知之,乃召希曾還京。自是,豐、沛漸無患,而魚臺數溢。

十一年,總河僉都御史戴時宗請委魚臺為受水之地,言:「河東北岸與運道鄰,惟西南流者,一由孫家渡出壽州,一由渦河口出懷遠,一由趙皮寨出桃源,一由梁靖口出徐州小浮橋。往年四道俱塞,全河南奔,故豐、沛、曹、單、魚臺以次受害。今患獨鐘於魚臺,宜棄以受水,因而道之,使入昭陽湖,過新開河,出留城、金溝、境山,乃易為力。至塞河四道,惟渦河經祖陵,未敢輕舉,其三支河頗存故迹,宜乘魚臺壅塞,令開封河夫捲埽填堤,逼使河水分流,則魚臺水勢漸減,俟水落畢工,并前三河共為四道,以分洩之,河患可已。」

明年,都御史朱裳代時宗,條上治河二事,大略言:「三大支河宜開如時宗計,而請塞梁靖口迤東由魚臺入運河之岔口,以捍黃河,則穀亭鎮迤南二百餘里淤者可浚,是謂塞黃河之口以開運河。黃河自穀亭轉入運河,順流而南,二日抵徐州,徐州逆流而北,四日乃抵穀亭,黃水之利莫大於此。恐河流北趨,或由魚臺、金鄉、濟寧漫安平鎮,則運河堤岸衝決;或三支一有壅淤,則穀亭南運河亦且衝決。宜繕築堤岸,束黃入運,是謂借黃河之水以資運河。」詔裳相度處置。

十三年正月,裳復言:

「今梁靖口、趙皮寨已通,孫家渡方濬。惟渦河一支,因趙皮寨下流睢州野雞岡淤正河五十餘里,漫於平地,注入渦河。宜挑濬深廣,引導漫水歸入正河,而於睢州張見口築長堤至歸德郭村,凡百餘里,以防汎溢。更時疏梁靖口下流,且挑儀封月河入之,達於小浮橋,則北岸水勢殺矣。

夫河過魚臺,其流漸北,將有越濟寧、趨安平、東入於海之漸。嘗議塞岔河之口以安運河,而水勢洶湧,恐難遽塞。塞亦不能無橫決,黃陵岡、李居莊諸處不能無患。徐州迤上至魯橋泥沙停滯,山東諸泉水微,運道必澀。請創築城武至濟寧縷水大堤百五十餘里,以防北溢。而自魯橋至沛縣東堤百五十餘里修築堅厚,固之以石。自魚臺至穀亭開通淤河,引水入漕,以殺魚臺、城武之患,此順水之性不與水爭地者也。

孫家渡、渦河二支俱出懷遠,會淮流至鳳陽,經皇陵及壽春王陵至泗州,經祖陵。皇陵地高無慮,祖陵則三面距河,壽春王陵尤迫近。祖陵宜築土堤,壽春王陵宜砌石岸,然事體重大,不敢輕舉也。清江浦口正當黃、淮會合之衝,二河水漲漫入河口,以致淤塞滯運,宜浚深廣。而又築堤以防水漲,築壩以護行舟,皆不可緩。往時淮水獨流入海,而海口又有套流,安東上下又有澗河、馬邏諸港以分水入海。今黃河匯入於淮,水勢已非其舊,而諸港套俱已堙塞,不能速洩,下壅上溢,梗塞運道。宜將溝港次第開浚,海口套沙,多置龍爪船往來爬盪,以廣入海之路,此所謂殺其下流者也。

河出魚臺雖借以利漕,然未有數十年不變者也。一旦他徙,則徐、沛必涸。宜大濬山東諸泉以匯於汶河,則徐、沛之渠不患乾涸,雖岔河口塞亦無虞矣。」工部覆如其議,帝允行之。未幾,裳憂去,命劉天和為總河副都御史代裳。

是歲,河決趙皮寨入淮,穀亭流絕,廟道口復淤。天和役夫十四萬濬之。已而河忽自夏邑大丘、回村等集衝數口,轉向東北,流經蕭縣,下徐州小浮橋。天和言:「黃河自魚、沛入漕河,運舟通利者數十年,而淤塞河道、廢壞閘座、阻隔泉流、衝廣河身,為害亦大。今黃河既改衝從虞城、蕭、碭,下小浮橋,而榆林集、侯家林二河分流入運者,俱淤塞斷流,利去而害獨存。宜浚魯橋至徐州二百餘里之淤塞。」制可。

十四年從天和言,自曹縣梁靖口東岔河口築壓口縷水堤,復築曹縣八里灣至單縣侯家林長堤各一道。是年冬,天和條上治河數事,中言:「魯橋至沛縣東堤,舊議築石以禦橫流,今黃河既南徙,可不必築。孫家渡自正統時全河從此南徙,弘治間淤塞,屢開屢淤,卒不能通。今趙皮寨河日漸衝廣,若再開渡口,併入渦河,不惟二洪水澀,恐亦有陵寢之虞,宜仍其舊勿治。舊議祥符盤石、蘭陽銅瓦廂、考城蔡家口各添築月堤。臣以為黃河之當防者惟北岸為重,當擇其去河遠者大隄中隄各一道,修補完築,使北岸七八百里間聯屬高厚,則前勘應築諸堤舉在其中,皆可罷不築。」帝亦從之。

十五年,督漕都御史周金言:「自嘉靖六年後,河流益南,其一由渦河直下長淮,而梁靖口、趙皮寨二支各入清河,匯於新莊閘,遂灌裏河。水退沙存,日就淤塞。故老皆言河自汴來本濁,而渦、淮、泗清,新莊閘正當二水之口,河、淮既合,昔之為沛縣患者,今移淮安矣。因請於新莊更置一渠,立閘以資蓄洩。」從之。

十六年冬從總河副都御史于湛言,開地丘店、野雞岡諸口上流四十餘里,由桃源集、丁家道口入舊黃河,截渦河水入河濟洪。十八年,總河都御史胡纘宗開考城孫繼口、孫祿口黃河支流,以殺歸、睢水患,且灌徐、呂,因於二口築長堤,及修築馬牧集決口。

二十年五月命兵部侍郎王以旂督理河道,協總河副都御史郭持平計議。先一歲,黃河南徙,決野雞岡,由渦河經亳州入淮,舊決口俱塞。其由孫繼口及考城至丁家道口,虞城入徐、呂者,亦僅十之二。持平久治弗效,降俸戴罪。以旂至,上言:「國初,漕河惟通諸泉及汶、泗,黃河勢猛水濁,遷徙不常,故徐有貞、白昂、劉大夏力排之,不資以濟運也。今幸黃河南徙,諸閘復舊,宜浚山東諸泉入野雞岡新開河道,以濟徐、呂;而築長堤沛縣以南,聚水如閘河制,務利漕運而已。」明年春,持平請浚孫繼口及扈運口、李景高口三河,使東由蕭、碭入徐濟運。其秋,從以旂言,於孫繼口外別開一渠洩水,以濟徐、呂。凡八月,三口工成,以旂、持平皆被獎,遂召以旂還。未幾,李景高口復淤。

先是,河決豐縣,遷縣治於華山,久之始復其故治。河決孟津、夏邑,皆遷其城。及野雞岡之決也,鳳陽沿淮州縣多水患,乃議徙五河、蒙城避之。而臨淮當祖陵形勝不可徙,乃用巡按御史賈太亨言,敕河撫二臣亟濬碭山河道,引入二洪,以殺南注之勢。

二十六年秋,河決曹縣,水入城二尺,漫金鄉、魚臺、定陶、城武,衝穀亭。總河都御史詹瀚請於趙皮寨諸口多穿支河,以分水勢。詔可。

三十一年九月,河決徐州房村集至邳州新安,運道淤阻五十里。總河副都御史曾鈞上治河方略,乃浚房村至雙溝、曲頭,築徐州高廟至邳州沂河。又言:「劉伶臺至赤晏廟凡八十里,乃黃河下流,淤沙壅塞,疏濬宜先。次則草灣老黃河口,衝激淹沒安東一縣,亦當急築,更築長堤磯嘴以備衝激。又三里溝新河口視舊口水高六尺,開舊口有沙淤之患,而為害稍輕;開新口未免淹沒之虞,而漕舟頗便。宜暫閉新口,建置閘座,且增築高家堰長堤,而新莊諸閘甃石以遏橫流。」帝命侍郎吳鵬振災戶,而悉從鈞奏。

三里溝新河者,督漕都御史應檟以先年開清河口通黃河之水以濟運。今黃河入海,下流澗口、安東俱漲塞,河流壅而漸高,瀉入清河口,沙停易淤,屢浚屢塞。溝在淮水下流黃河未合之上,故閉清河口而開之,使船由通濟橋溯溝出淮,以達黃河者也。

時浚徐、邳將訖工,一夕,水湧復淤。帝用嚴嵩言,遣官祭河神。而鵬、鈞復共奏請急築浚草灣、劉伶臺,建閘三里溝,迎納泗水清流;且於徐州以上至開封浚支河一二,令水分殺。其冬,漕河工竣,進鈞秩侍郎。

三十七年七月,曹縣新集淤。新集地接梁靖口,歷夏邑、丁家道口、馬牧集、韓家道口、司家道口至蕭縣薊門出小浮橋,此賈魯河故道也。自河患亟,別開支河出小河以殺水勢,而本河漸澀。至是遂決,趨東北段家口,析而為六,曰大溜溝、小溜溝、秦溝、濁河、胭脂溝、飛雲橋,俱由運河至徐洪。又分一支由碭山堅城集下郭貫樓,析而為五,曰龍溝、母河、梁樓溝、楊氏溝、胡店溝,亦由小浮橋會徐洪,而新集至小浮橋故道二百五十餘里遂淤不可復矣。自後,河忽東忽西,靡有定向,水得分瀉者數年,不至壅潰。然分多勢弱,淺者僅二尺,識者知其必淤。

至四十四年七月,河決沛縣,上下二百餘里運道俱淤。全河逆流,自沙河至徐州以北,至曹縣棠林集而下,北分二支:南流者繞沛縣戚山楊家集,入秦溝至徐;北流者繞豐縣華山東北由三教堂出飛雲橋。又分而為十三支,或橫絕,或逆流入漕河,至湖陵城口,散漫湖坡,達於徐州,浩渺無際,而河變極矣。乃命朱衡為工部尚書兼理河漕,又以潘季馴為僉都御史總理河道。明年二月,復遣工科給事中何起鳴往勘河工。

衡巡行決口,舊渠已成陸,而盛應期所鑿新河故跡尚在,地高,河決至昭陽湖不能復東,乃定計開濬。而季馴則以新河土淺泉湧,勞費不貲,留城以上故道初淤可復也。由是二人有隙。起鳴至沛,還,上言:「舊河之難復有五。黃河全徙必殺上流,新集、龐家屯、趙家圈皆上流也,以不貲之財,投於河流已棄之故道,勢必不能,一也。自留城至沛,莽為巨浸,無所施工,二也。橫亙數十里,褰裳無路,十萬之眾何所棲身,三也。挑浚則淖隱,築岸則無土,且南塞則北奔,四也。夏秋淫潦,難保不污,五也。新河開鑿費省,且可絕後來潰決之患。宜用衡言開新河,而兼採季馴言,不全棄舊河。」廷臣議定,衡乃決開新河。

時季馴持復故道之議,廷臣又多以為然。遂勘議新集、郭貫樓諸上源地。衡言:

「河出境山以北,則閘河淤;出徐州以南,則二洪涸;惟出境山至小浮橋四十餘里間,乃兩利而無害。自黃河橫流,碭山郭貫樓支河皆已淤塞,改從華山分為南北二支:南出秦溝,正在境山南五里許,運河可資其利;惟北出沛縣西及飛雲橋,逆上魚臺,為患甚大。

朝廷不忍民罹水災,拳拳故道,命勘上源。但臣參考地形有五不可。自新集至兩河口皆平原高阜,無尺寸故道可因,郭貫樓抵龍溝頗有河形,又係新淤,無可駐足,其不可一也。黃河所經,鮮不為患,由新集則商、虞、夏邑受之,由郭貫樓則蕭、碭受之,今改復故道,則魚、沛之禍復移蕭、碭,其不可二也。河西注華山,勢若建瓴,欲從中鑿渠,挽水南向,必當築壩橫截,遏其東奔,於狂瀾巨浸之中,築壩數里,為力甚難,其不可三也。役夫三十萬,曠日持久,騷動三省,其不可四也。大役踵興,工費數百萬,一有不繼,前功盡隳,其不可五也。惟當開廣秦溝,使下流通行,修築南岸長堤以防奔潰,可以蘇魚、沛昏墊之民。」

從之。衡乃開魚臺南陽抵沛縣留城百四十餘里,而浚舊河自留城以下,抵境山、茶城五十餘里,由此與黃河會。又築馬家橋堤三萬五千二百八十丈,石堤三十里,遏河之出飛雲橋者,趨秦溝以入洪。於是黃水不東侵,漕道通而沛流斷矣。方工未成,河復決沛縣,敗馬家橋堤。論者交章請罷衡。未幾,工竣。帝大喜,賦詩四章志喜,以示在直諸臣。

隆慶元年五月加衡太子少保。始河之決也,支流散漫遍陸地,既而南趨濁河。迨新河成,則盡趨秦溝,而南北諸支河悉併流焉。然河勢益大漲。三年七月決沛縣,自考城、虞城、曹、單、豐、沛抵徐州俱受其害,茶城淤塞,漕船阻邳州不能進。已雖少通,而黃河水橫溢沛地,秦溝、濁河口淤沙旋疏旋壅。朱衡已召還,工部及總河都御史翁大立皆請於梁山之南別開一河以漕,避秦溝、濁河之險,後所謂泇河者也。詔令相度地勢,未果行。

四年秋,黃河暴至,茶城復淤,而山東沙、薛、汶、泗諸水驟溢,決仲家淺運道,由梁山出戚家港,合於黃河。大立復請因其勢而浚之。是時,淮水亦大溢,自泰山廟至七里溝淤十餘里,而水從諸家溝傍出,至清河縣河南鎮以合於黃河。大立又言:「開新莊閘以通回船,復陳瑄故道,則淮可無虞。獨黃河在睢寧、宿遷之間遷徙未知所定,泗州陵寢可虞。請浚古睢河,由宿遷歷宿州,出小浮橋以洩二洪之水。且規復清河、魚溝分河一道,下草灣,以免衝激之患,則南北運道庶幾可保。」時大立已內遷,方受代,而季馴以都御史復起總理河道。部議令區畫。

九月,河復決邳州,自睢寧白浪淺至宿遷小河口,淤百八十里,糧艘阻不進。大立言:「比來河患不在山東、河南、豐、沛,而專在徐、邳,故先欲開泇河口以遠河勢、開蕭縣河以殺河流者,正謂浮沙壅聚,河面增高,為異日慮耳。今秋水洊至,橫溢為災。權宜之計,在棄故道而就新衝;經久之策,在開泇河以避洪水。」乞決擇於二者。部議主塞決口,而令大立條利害以聞。大立遂以開泇口、就新沖、復故道三策並進,且言其利害各相參。會罷去,策未決,而季馴則主復故道。

時茶城至呂梁,黃水為兩崖所束,不能下,又不得決。至五年四月,乃自靈璧雙溝而下,北決三口,南決八口,支流散溢,大勢下睢寧出小河,而匙頭灣八十里正河悉淤。季馴役丁夫五萬,盡塞十一口,且浚匙頭灣,築縷堤三萬餘丈,匙頭灣故道以復。旋以漕船行新溜中多漂沒,季馴罷去。

六年春,復命尚書衡經理河工,以兵部侍郎萬恭總理河道。二人至,罷泇河議,專事徐、邳河,修築長堤,自徐州至宿遷小河口三百七十里,併繕豐、沛大黃堤,正河安流,運道大通。衡乃上言:「河南屢被河患,大為堤防,今幸有數十年之安者,以防守嚴而備禦素也。徐、邳為糧運正道,既多方以築之,則宜多方以守之。請用夫每里十人以防,三里一鋪,四鋪一老人巡視。伏秋水發時,五月十五日上堤,九月十五日下堤,願攜家居住者聽。」詔如議。六月,徐、邳河堤工竣,遂命衡回部,賞衡及總理河道都御史萬恭等銀幣有差。

是歲,御史吳從憲言:「淮安而上清河而下,正淮、泗、河、海衝流之會。河潦內出,海潮逆流,停蓄移時,沙泥旋聚,以故日就壅塞。宜以春夏時濬治,則下流疏暢,汎溢自平。」帝即命衡與漕臣勘議。而督理河道署郎中事陳應薦挑穵海口新河,長十里有奇,闊五丈五尺,深一丈七尺,用夫六千四百餘人。

衡之被召將還也,上疏言:「國家治河,不過濬淺、築堤二策。浚淺之法,或爬或澇,或逼水而衝,或引水而避,此可人力勝者。然茶城與淮水會則在清河,茶城、清河無水不淺。蓋二水互為勝負,黃河水勝則壅沙而淤,及其消也,淮漕水勝,則衝沙而通。水力蓋居七八,非專用人力也。築堤則有截水、縷水之異,截水可施於閘河,不可施於黃河。蓋黃河湍悍,挾川潦之勢,何堅不瑕,安可以一堤當之?縷水則兩岸築堤,不使旁潰,始得遂其就下入海之性。蓋以順為治,非以人力勝水性,故至今百五六十年為永賴焉。清河之淺,應視茶城,遇黃河漲落時,輒挑河、潢,導淮水沖刷,雖遇漲而塞,必遇落而通,無足慮也。惟清江浦水勢最弱,出口處所適與黃河相值。宜於黃水盛發時,嚴閉各閘,毋使沙淤。若口則自隆重慶三年海嘯,壅水倒灌低窪之地,積瀦難洩。宜時加疏浚,毋使積塞。至築黃河兩岸堤,第當縷水,不得以攔截為名。」疏上,報聞而已。


○黃河下

萬歷元年,河決房村,築堤AH子頭至秦溝口。明年,給事中鄭岳言:「運道自茶城至淮安五百餘里,自嘉靖四十四年河水大發,淮口出水之際,海沙漸淤,今且高與山等。自淮而上,河流不迅,泥水愈淤。於是邳州淺,房村決,呂、梁二洪平,茶城倒流,皆坐此也。今不治海口之沙,乃日築徐、沛間堤岸,桃、宿而下,聽其所之。民之為魚,未有已時也。」因獻宋李公義、王令圖浚川爬法。命河臣勘奏,從其所言。而是年秋,淮、河並溢。明年八月河決碭山及邵家口、曹家莊、韓登家口而北,淮亦決高家堰而東,徐、邳、淮南北漂沒千里。自此桃、清上下河道淤塞,漕艘梗阻者數年,淮、揚多水患矣。總河都御史傅希摯改築碭山月隄,暫留三口為洩水之路。其冬,並塞之。

四年二月,督漕侍郎吳桂芒言:「淮、揚洪潦奔衝,蓋緣海賓汊港久堙,入海止雲梯一徑,致海擁橫沙,河流泛溢,而鹽、安、高、寶不可收拾。國家轉運,惟知急漕,而不暇急漕,而不暇急民,故朝廷設官,亦主治河,而不知治海。請設水利僉事一員,專疏海道,審度地利,如草灣及老黃河皆可趨海,何必專事雲梯哉?」帝優詔報可。

桂芳復言:「黃水抵清河與淮合流,經清江浦外河,東至草灣,又折而西南,過淮安、新城外河,轉入安東縣前,直下雲梯關入海。近年關口多壅,河流日淺,惟草灣地低下,黃河衝決,駸駸欲奪安東入海,以縣治所關,屢決屢塞。去歲,草灣迤東自決一口,宜於決口之西開挑新口,以迎埽灣之溜,而於金城至五港岸築堤束水。語云:「救一路哭,不當復計一家哭。」今淮、揚、鳳、泗、邳、徐不啻一路矣。安東自眾流匯圍,只文廟、縣署僅存椽瓦,其勢垂陷,不如委之,以拯全淮。」帝不欲棄安東,而命開草灣如所請。八月,工竣,長萬一千一百餘丈,塞決口二十二,役夫四萬四千。帝以海口開浚,水患漸平,賚桂芳等有差。

未幾,河決韋家樓,又決沛縣縷水堤,豐、曹二縣長堤,豐、沛、徐州、睢寧、金鄉、魚臺、單、曹田廬漂溺無算,河流齧宿遷城。帝從桂芳請,遷縣治、築土城避之。於是御史陳世寶請復老黃河故道,言:「河自桃源三義鎮歷清河縣北,至大河口會淮入海。運道自淮安天妃廟亂淮而下,十里至大河口,從三義鎮出口向桃源大河而去,凡七十餘里,是為老黃河。至嘉靖初,三義鎮口淤,而黃河改趨清河縣南與淮會,自此運道不由大河口而徑由清河北上矣。近者,崔鎮屢決,河勢漸趨故道。若仍開三義鎮口引河入清河北,或令出大河口與淮流合,或從清河西別開一河,引淮出河上游,則運道無恐,而淮、泗之水不為黃流所漲。」部覆允行。

桂芳言:「淮水向經清河會黃河趨海。自去秋河決崔鎮,清江正河淤澱,淮口梗塞。於是淮弱河強,不能奪草灣入海之途,而全淮南徙,橫灌山陽、高、寶間,向來湖水不踰五尺,堤僅七尺,今堤加丈二,而水更過之。宜急護湖堤以殺水勢。」部議以為必淮有所歸,而後堤可保,請令桂芳等熟計。報可。

開河、護堤二說未定,而河復決崔鎮,宿、沛、清、桃兩岸多壞,黃河日淤墊,淮水為河所迫,徙而南,時五年八月也。希摯議塞決口,束水歸漕。桂芳欲衝刷成河,以為老黃河入海之路。帝令急塞決口,而俟水勢稍定,乃從桂芳言。時給事中湯聘尹議導淮入江以避黃,會桂芳言:「黃水向老黃河故道而去,下奔如駛,淮遂乘虛湧入清口故道,淮、揚水勢漸消。」部議行勘,以河、淮既合,乃寢其議。

管理南河工部郎中施天麟言:

「淮、泗之水不下清口而下山陽,從黃浦口入海。浦口不能盡洩,浸淫高、寶邵伯諸湖,而湖堤盡沒,則以淮、泗本不入湖,而今入湖故也。淮、泗之入湖者,又緣清口向未淤塞,而今淤塞故也。清口之淤塞者,又緣黃河淤塞日高,淮水不得不讓河而南徙也。蓋淮水併力敵黃,勝負或亦相半,自高家堰廢壞,而清口內通濟橋、朱家等口淮水內灌,於是淮、泗之力分,而黃河得以全力制其敝,此清口所以獨淤於今歲也。下流既淤,則上流不得不決。

每歲糧艘以四五月畢運,而堤以六七月壞。水發之時不能為力,水落之後方圖堵塞。甫及春初,運事又迫,僅完堤工,於河身無與。河身不挑則來年益高。上流之決,必及於徐、呂,而不止於邳、遷;下流之涸,將盡乎邳、遷,而不止於清、桃。須不惜一年糧運,不惜數萬帑藏,開挑正河,寬限責成,乃為一勞永逸。

至高家堰、朱家等口,宜及時築塞,使淮、泗併力足以敵黃,則淮水之故道可復,高、寶之大患可減。若興、鹽海口堙塞,亦宜大加疏浚。而湖堤多建減水大閘,堤下多開支河。要未有不先黃河而可以治淮,亦未有不疏通淮水而可以固堤者也。」事下河漕諸臣會議。

淮之出清口也,以黃水由老黃河奔注,而老黃河久淤,未幾復塞,淮水仍漲溢。給事中劉鉉請亟開通海口,而簡大臣會同河漕諸臣往治。乃命桂芳為工部尚書兼理河漕,而裁總河都御史官。桂芳甫受命而卒。

六年夏,潘季馴代。時給事中李淶請多濬海口,以導眾水之歸。給事中王道成則請塞崔鎮決口,築桃、宿長堤,修理高家堰,開復老黃河。並下河臣議。季馴與督漕侍郎江一麟相度水勢,言:

「海口自雲梯關四套以下,闊七八里至十餘里,深三四丈。欲別議開鑿,必須深闊相類,方可注放,工力甚難。且未至海口,乾地猶可施工,其將入海之地,潮汐往來,與舊口等耳。舊口皆係積沙,人力雖不可浚,水力自能衝刷,海無可浚之理。惟當導河歸海,則以水治水,即浚海之策也。河亦非可以人力導,惟當繕治堤防,俾無旁決,則水由地中,沙隨水去,即導河之策也。

頻年以來,日以繕堤為事,顧卑薄而不能支,迫近而不能容,雜以浮沙而不能久。是以河決崔鎮,水多北潰,為無隄也。淮決高家堰、黃浦口,水多東潰,隄弗固也。不咎制之未備,而咎築堤為下策,豈通論哉!上流既旁潰,又岐下流而分之,其趨雲梯入海口者,譬猶強弩之末耳。水勢益分則力益弱,安能導積沙以注海?

故今日濬海急務,必先塞決以導河,尤當固堤以杜決,而欲堤之不決,必真土而勿雜浮沙,高厚而勿惜鉅費,讓遠而勿與爭地,則堤乃可固也。沿河堤固,而崔鎮口塞,則黃不旁決而衝漕力專。高家堰築,朱家口塞,則淮不旁決而會黃力專。淮、黃既合,自有控海之勢。又懼其分而力弱也,必暫塞清江浦河,而嚴司啟閉以防其內奔。姑置草灣河,而專復雲梯以還其故道。仍接築淮安新城長堤,以防其末流。使黃、淮力全,涓滴悉趨於海,則力強且專,下流之積沙自去,海不浚而闢,河不挑而深,所謂固堤即以導河,導河即以浚海也。」

又言:「黃水入徐,歷邳、宿、桃、清,至清口會淮而東入海。淮水自洛及鳳,歷盱、泗,至清口會河而東入海。此兩河故道也。元漕江南粟,則由揚州直北廟灣入海,未嘗溯淮。陳瑄始堤管家諸湖,通淮為運道。慮淮水漲溢,則築高家堰堤以捍之,起武家墩,經大、小澗至阜寧湖,而淮不東侵。又慮黃河漲溢,則堤新城北以捍之,起清江浦,沿缽池山、柳浦灣迤東,而黃不南侵。

其後,堤岸漸傾,水從高堰決入,淮郡遂同魚鱉。而當事者未考其故,謂海口壅閉,宜亟穿支渠。詎知草灣一開,西橋以上正河遂至淤阻。夫新河闊二十餘丈,深僅丈許,較故道僅三十之一,豈能受全河之水?下流既壅,上流自潰,此崔鎮諸口所由決也。今新河復塞,故河漸已通流,雖深闊未及原河十一,而兩河全下,沙隨水刷,欲其全復河身不難也。河身既復,闊者七八里,狹亦不下三四百丈,滔滔東下,何水不容?匪惟不必別鑿他所,即草灣亦可置勿浚矣。

故為今計,惟修復陳瑄故蹟,高築南北兩堤,以斷兩河之內灌,則淮、揚昏墊可免。塞黃浦口,築寶應堤,浚東關等淺,修五閘,復五壩,則淮南運道無虞。堅塞桃源以下崔鎮口諸決,則全河可歸故道。黃、淮既無旁決,並驅入海,則沙隨水刷,海口自復,而桃、清淺阻,又不足言。此以水治水之法也。若夫爬撈之說,僅可行諸閘河,前入屢試無功,徒費工料。」

於是條上六議:曰塞決口以挽正河,曰築堤防以杜潰決,曰復閘壩以防外河,曰創滾水壩以固隄岸,曰止浚海工程以省糜費,曰寢開老黃河之議以仍利涉。帝悉從其請。

七年十月,兩河工成,賚季馴、一麟銀幣,而遣給事中尹瑾勘實。八年春進季馴太子太保工部尚書,蔭一子。一麟等遷擢有差。是役也,築高家堰堤六十餘里,歸仁集堤四十餘里,柳浦灣堤東西七十餘里,塞崔鎮等決口百三十,築徐、睢、邳、宿、桃、清兩岸遙堤五萬六千餘丈,碭、豐大壩各一道,徐、沛、豐、碭縷堤百四十餘里,建崔鎮、徐昇、季泰、三義減水石壩四座,遷通濟閘於甘羅城南,淮、揚間堤壩無不修築,費帑金五十六萬有奇。其秋擢季馴南京兵部尚書。季馴又請復新集至小浮橋故道,給事中王道成、河南巡撫周鑑等不可而止。自桂芳、季馴時罷總河不設,其後但以督漕兼理河道。高堰初築,清口方暢,流連數年,河道無大患。

至十五年,封丘、偃師、東明、長垣屢被衝決。大學士申時行言:「河所決地在三省,守臣畫地分修,易推委。河道未大壞,不必設都御史,宜遣風力老成給事中一人行河。」乃命工科都給事中常居敬往。居敬請修築大社集東至白茅集長堤百里。從之。

初,黃河由徐州小浮橋入運,其河深且近洪,能刷洪以深河,利於運道。後漸徙沛縣飛雲橋及徐州大、小溜溝。至嘉靖末,決邵家口,出秦溝,由濁河口入運,河淺,迫茶城,茶城歲淤,運道數害。萬曆五年冬,河復南趨,出小浮橋故道,未幾復堙。潘季馴之塞崔鎮也,厚築堤岸,束水歸漕。嗣後水發,河臣輒加堤,而河身日高矣。於是督漕僉都御史楊一魁欲復黃河故道,請自歸德以下丁家道口濬至石將軍廟,令河仍自小浮橋出。又言:「善治水者,以疏不以障。年來堤上加堤,水高凌空,不啻過顙。濱河城郭,決水可灌。宜測河身深淺,隨處挑浚,而於黃河分流故道,設減水石門以洩暴漲。」給事中王士性則請復老黃河故道。大略言:

「自徐而下,河身日高,而為堤以束之,堤與徐州城等。束益急,流益迅,委全力於淮而淮不任。故昔之黃、淮合,今黃強而淮益縮,不復合矣。黃強而一啟天妃、通濟諸閘,則灌運河如建瓴。高、寶一梗,江南之運坐廢。淮縮則退而侵泗。為祖陵計,不得不建石堤護之。堤增河益高,根本大可虞也。河至清河凡四折而後入海。淮安、高、寶、鹽、興數百萬生靈之命託之一丸泥,決則盡成魚暇矣。

紛紛之議,有欲增堤泗州者,有欲開顏家、灌口、永濟三河,南甃高家堰、北築滾水壩者。總不如復河故道,為一勞永逸之計也。河故道由三義鎮達葉家衝與淮合,在清河縣北別有濟運河,在縣南蓋支河耳。河強奪支河,直趨縣南,而自棄北流之道,然河形固在也。自桃源至瓦子灘凡九十里,AH下不耕,無室廬填墓之礙,雖開河費鉅,而故道一復,為利無窮。」

議皆未定。居敬及御史喬璧星皆請復專設總理大臣。乃命潘季馴為右都御史總督河道。

時帝從居敬言,罷老黃河議,而季馴抵官,言:「親集故道,故老言『銅幫鐵底』,當開,但歲儉費繁,未能遽行。」又言:「黃水濁而強,汶、泗清且弱,交會茶城。伏秋黃水發,則倒灌人漕,沙停而淤,勢所必至。然黃水一落,漕即從之,沙隨水去,不浚自通,縱有淺阻,不過旬日。往時建古洪、內華二閘,黃漲則閉閘以遏濁流,黃退則啟閘以縱泉水。近者居敬復增建鎮口閘,去河愈近,則吐納愈易。但當嚴閘禁如清江浦三閘之法,則河渠永賴矣。」帝方委季馴,即從其言,罷故道之議。未幾,水患益甚。

十七年六月,黃水暴漲,決獸醫口月堤,漫李景高口新堤,衝入夏鎮內河,壞田廬,沒人民無算。十月,決口塞。十八年,大溢,徐州水積城中者逾年。眾議遷城改河。季馴濬魁山支河以通之,起蘇伯湖至小河口,積水乃消。十九年九月,泗州大水,州治淹三尺,居民沉溺十九,浸及祖陵。而山陽復河決,江都、邵伯又因湖水下注,田廬浸傷。工部尚書曾同亨上其事,議者紛起。乃命工科給事中張貞觀往泗州勘視水勢,而從給事中楊其休言,放季馴歸,用舒應龍為工部尚書總督河道。

二十年三月,季馴將去,條上辨惑者六事,力言河不兩行,新河不當開,支渠不當浚。又著書曰河防一覽,大旨在築堤障河,束水歸漕;築堰障淮,逼淮注黃。以清刷濁,沙隨水去。合則流急,急則蕩滌而河深;分則流緩,緩則停滯而沙積。上流既急,則海口自闢而無待於開。其治堤之法,有縷堤以束其流,有遙堤以寬其勢,有滾水壩以洩其怒。法甚詳,言甚辯。然當是時,水勢橫潰,徐、泗、淮、揚間無歲不受患,祖陵被水。季馴謂當自消,已而不驗。於是季馴言詘,而分黃導淮之議由此起矣。

貞觀抵泗州言:「臣謁祖陵,見泗城如水上浮盂,盂中之水復滿。祖陵自神路至三橋、丹墀,無一不被水。且高堰危如累卵,又高、寶隱禍也。今欲洩淮,當以闢海口積沙為第一義。然洩淮不若殺黃,而殺黃於淮流之既合,不若殺於未合。但殺於既合者與運無妨,殺於未合者與運稍礙。別標本,究利害,必當殺於未合之先。至於廣入海之途,則自鮑家口、黃家營至魚溝、金城左右,地勢頗下,似當因而利導之。」貞觀又會應龍及總漕陳于陛等言:「淮、黃同趨者惟海,而淮之由黃達海者惟清口。自海沙開浚無期,因而河身日高;自河流倒灌無已,因而清口日塞。以致淮水上浸祖陵,漫及高、寶,而興、泰運堤亦衝決矣。今議闢清口沙,且分黃河之流於清口上流十里地,去口不遠,不至為運道梗。分於上,復合於下,則衝海之力專。合必於草灣之下,恐其復衝正河,為淮城患也。塞鮑家口、黃家營二決,恐橫衝新河,散溢無歸。兩岸俱堤,則東北清、沭、海、安AH下地不虞潰決。計費凡三十六萬有奇。若海口之塞,則潮汐莫窺其涯,難施畚鍤。惟淮、黃合流東下,河身滌而漸深,海口刷而漸闢,亦事理之可必者。」帝悉從其請。乃議於清口上流北岸,開腰鋪支河達於草灣。

既而淮水自決張福堤。直隸巡按彭應參言:「祖陵度可無虞,且方東備倭警,宜暫停河工。」部議令河臣熟計。應龍、貞觀言:「為祖陵久遠計,支河實必不容已之工,請候明春倭警寧息舉行。」其事遂寢。

二十一年春,貞觀報命,議開歸、徐達小河口,以救徐、邳之溢;導濁河入小浮橋故道,以紓鎮口之患。下總河會官集議,未定。五月,大雨,河決單縣黃堌口,一由徐州出小浮橋,一由舊河達鎮口閘。邳城陷水中,高、寶諸湖堤決口無算。明年,湖堤盡築塞,而黃水大漲,清口沙墊,淮水不能東下,於是挾上源阜陵諸湖與山溪之水,暴浸祖陵,泗城淹沒。二十三年,又決高郵中堤及高家堰、高良澗,而水患益急矣。

先是,御史陳邦科言:「固堤束水未收刷沙之利,而反致衝決。法當用濬,其方有三。冬春水涸,令沿河淺夫乘時撈淺,則沙不停而去,一也。官民船往來,船尾悉繫鈀犁,乘風搜滌,則沙不寧而去,二也。仿水磨、水碓之法,置為木機,乘水滾盪,則沙不留而去,三也。至淮必不可不會黃,故高堰斷不可棄。湖溢必傷堤,故周家橋潰處斷不可開。已棄之道必淤滿,故老黃河、草灣等處斷不可。」疏下所司議。戶部郎中華存禮則請復黃河故道,並濬草灣。而是時腰鋪猶未開,工部侍郎沈節甫言:「復黃河未可輕議,至諸策皆第補偏救弊而已,宜概停罷。」乃召應龍還工部,時二十二年九月也。

既而給事中吳應明言:「先因黃河遷徙無常,設遙、縷二堤束水歸漕,及水過沙停,河身日高,徐、邳以下居民盡在水底。今清口外則黃流阻遏,清口內則淤沙橫截,強河橫灌上流約百里許,淮水僅出沙上之浮流,而瀦蓄於盱、泗者遂為祖陵患矣。張貞觀所議腰鋪支河歸之草灣,或從清河南岸別開小河至駱家營、馬廠等地,出會大河,建閘啟閉,一遇運淺,即行此河,亦策之便者。」至治泗水,則有議開老子山,引淮水入江者。宜置閘以時啟閉,拆張福堤而堤清口,使河水無南向。部議下河漕諸臣會勘。直隸巡按牛應元因謁祖陵,目擊河患,繪圖以進,因上疏言:

「黃高淮壅,起於嘉靖末年河臣鑿徐、呂二洪巨石,而沙日停,河身日高,潰決由此起。當事者計無復之,兩岸築長堤以束,曰縷堤。縷堤復決,更於數里外築重堤以防,曰遙堤。雖歲決歲補,而莫可誰何矣。

黃、淮交會,本自清河北二十里駱家營,折而東至大河口會淮,所稱老黃河是也。陳瑄以其迂曲,從駱家營開一支河,為見今河道,而老黃河淤矣。萬曆間,復開草灣支河,黃舍故道而趨,以致清口交會之地,二水相持,淮不勝黃,則竄入各閘口,淮安士民於各閘口築一土埂以防之。嗣後黃、淮暴漲,水退沙停,清口遂淤,今稱門限沙是也。當事者不思挑門限沙,乃傍土埂築高堰,橫亙六十里,置全淮正流之口不事,復將從旁入黃之張福口一并築堤塞之,遂倒流而為泗陵患矣。前歲,科臣貞觀議闢門限沙,裁張福堤,其所重又在支河腰鋪之開。

總之,全口淤沙未盡挑闢,即腰鋪工成,淮水未能出也,況下流鮑、王諸口已決,難以施工。豈若復黃河故道,盡闢清口淤沙之為要乎?且疏上流,不若科臣應明所議,就草灣下流浚諸決口,俾由安東歸五港,或於周家橋量為疏通,而急塞黃堌口,挑蕭、碭渠道,濬符離淺阻。至宿遷小河為淮水入黃正路,急宜挑闢,使有所歸。」

應龍言:「張福堤已決百餘丈,清口方挑沙,而腰鋪之開尤不可廢。」工部侍郎沈思孝因言:「老黃河自三義鎮至葉家衝僅八千餘丈,河形尚存。宜亟開浚,則河分為二,一從故道抵顏家河入海,一從清口會淮,患當自弭。請遣風力科臣一人,與河漕諸臣定畫一之計。」乃命禮科給事中張企程往勘。而以水患累年,迄無成畫,遷延糜費,罷應龍職為民,常居敬、張貞觀、彭應參等皆譴責有差。御史高舉請「疏周家橋,裁張福堤,闢門限沙,建滾水石壩於周家橋、大小澗口、武家墩、綠楊溝上下,而壩外浚河築岸,使行地中。改塘埂十二閘為壩,灌閘外十二河,以闢入海之路。浚芒稻河,且多建濱江水閘,以廣入江之途。然海口日壅,則河沙日積,河身日高,而淮亦不能安流。有灌口者,視諸口頗大,而近日所決蔣家、鮑家、畀家三口直與相射,宜挑浚成河,俾由此入海。」工部主事樊兆程亦議闢海口,而言:「舊海口決不可濬,當自鮑家營至五港口挑浚成河,令從灌口入海。」俱下工部。請並委企程勘議。

是時,總河工部尚書楊一魁被論,乞罷,因言:「清口宜濬,黃河故道宜復,高堰不必修,石堤不必砌,減水閘壩不必用。」帝不允辭,而詔以盡心任事。御史夏之臣則言:「海口沙不可劈,草灣河不必浚,腰鋪新河四十里不必開,雲梯關不必闢,惟當急開高堰,以救祖陵。」且言:「歷年以來,高良澗土堤每遇伏秋即衝決,大澗口石堤每遇洶湧即崩潰。是高堰在,為高、寶之利小;而高堰決,則為高、寶之害大也。孰若明議而明開之,使知趨避乎?」給事中黃運泰則又言:「黃河下流未洩,而遽開高堰、周橋以洩淮水,則淮流南下,黃必乘之,高、寶間盡為沼,而運道月河必衝決矣。不如濬五港口,達灌口門,以入於海之為得也。」詔並行勘議。

企程乃上言:「前此河不為陵患,自隆慶末年高、寶、淮、揚告急,當事狃於目前,清口既淤,又築高堰以遏之,堤張福以束之,障全淮之水與黃角勝,不虞其勢不敵也。迨後甃石加築,堙塞愈堅,舉七十二溪之水匯於泗者,僅留數丈一口出之,出者什一,停者什九。河身日高,流日壅,淮日益不得出,而瀦蓄日益深,安得不倒流旁溢為泗陵患乎?今議疏淮以安陵,疏黃以導淮者,言人人殊。而謂高堰當決者,臣以為屏翰淮、揚,殆不可少。莫若於其南五十里開周家橋注草子湖,大加開濬,一由金家灣入芒稻河注之江,一由子嬰溝入廣洋湖達之海,則淮水上流半有宣洩矣。於其北十五里開武家墩,注永濟河,由窯灣閘出口直達涇河,從射陽湖入海,則淮水下流半有歸宿矣。此急救祖陵第一義也。」會是時,祖陵積水稍退,一魁以聞,帝大悅,仍諭諸臣急協議宣洩。

於是企程、一魁共議欲分殺黃流以縱淮,別疏海口以導黃。而督漕尚書褚鈇則以江北歲祲,民不堪大役,欲先洩淮而徐議分黃。御史應元折衷其說,言:「導淮勢便而功易,分黃功大而利遠。顧河臣所請亦第六十八萬金,國家亦何靳於此?」御史陳煃嘗令寶應,慮周家橋既開,則以高郵、邵伯為壑,運道、民產、鹽場交受其害,上疏爭之,語甚激,大旨分黃為先,而淮不必深治。且欲多開入海之路,令高、寶諸湖之水皆東,而後周家橋、武家墩之水可注。而淮安知府馬化龍復進分黃五難之說。潁州兵備道李弘道又謂宜開高堰。鈇遂據以上聞。給事中林熙春駁之,言:「淮猶昔日之淮,而河非昔日之河,先是河身未高,而淮尚安流,今則河身既高,而淮受倒灌,此導淮固以為淮,分黃亦以為淮。」工部乃覆奏云:「先議開腰鋪支河以分黃流,以倭儆、災傷停寢,遂貽今日之患。今黃家壩分黃之工若復沮格,淮壅為害,誰職其咎?請令治河諸臣導淮分黃,亟行興舉。」報可。

二十四年八月,一魁興工未竣,復條上分淮導黃事宜十事。十月,河工告成,直隸巡按御史蔣春芳以聞,復條上善後事宜十六事。乃賞賚一魁等有差。是役也,役夫二十萬,開桃源黃河壩新河,起黃家嘴,至安東五港、灌口,長三百餘里,分洩黃水入海,以抑黃強。闢清口沙七里,建武家墩、高良澗、周家橋石閘,洩淮水三道入海,且引其支流入江。於是泗陵水患平,而淮、揚安矣。

然是時一魁專力桃、清、淮、泗間,而上流單縣黃堌口之決,以為不必塞。鈇及春芳皆請塞之。給事中李應策言:「漕臣主運,河臣主工,各自為見。宜再令析議。」一魁言:「黃堌口一支由虞城、夏邑接碭山、蕭縣、宿州至宿遷,出白洋河,一小支分蕭縣兩河口,出徐州小浮橋,相距不滿四十里。當疏浚與正河會,更通鎮口閘裏湖之水,與小浮橋二水會,則黃堌口不必塞,而運道無滯矣。」從之。於是議浚小浮橋、沂河口、小河口以濟徐、邳運道,以洩碭、蕭漫流,掊歸仁堤以護陵寢。

是時,徐、邳復見清、泗運道不利,鈇終以為憂。二十五年正月,復極言黃堌口不塞,則全河南徙,害且立見。議者亦多恐下齧歸仁,為二陵患。三月,小浮橋等口工垂竣,一魁言:

「運道通利,河徙不相妨,已有明驗。惟議者以祖陵為慮,請徵往事折之。洪武二十四年,河決原武,東南至壽州入淮。永樂九年,河北入魚臺。未幾復南決,由渦河經懷遠入淮。時兩河合流,歷鳳、泗以出清口,未聞為祖陵患。正統十三年,河北衝張秋。景泰中,徐有貞塞之,復由渦河入淮。弘治二年,河又北衝,白昂、劉大夏塞之,復南流,一由中牟至潁、壽,一由亳州至渦河入淮,一由宿遷小河口會泗。全河大勢縱橫潁、亳、鳳、泗間,下溢符離、睢、宿,未聞為祖陵慮,亦不聞堤及歸仁也。

正德三年後,河漸北徙,由小浮橋、飛雲橋、谷亭三道入漕,盡趨徐、邳,出二洪,運道雖濟,而泛溢實甚。嘉靖十一年,朱裳始有渦河一支中經鳳陽祖陵未敢輕舉之說。然當時猶時濬祥符之董盆口、寧陵之五里鋪、滎澤之孫家渡、蘭陽之趙皮寨,又或決睢州之地丘店、界牌口、野雞岡,寧陵之楊村鋪,俱入舊河,從亳、鳳入淮,南流未絕,亦何嘗為祖陵患?

嘉靖二十五年後,南流故道始盡塞,或由秦溝入漕,或由濁河入漕。五十年來全河盡出徐、邳,奪泗入淮。而當事者方認客作主,日築堤而窘之,以致河流日壅,淮不敵黃,退而內瀦,遂貽盱、泗祖陵之患。此實由內水之停壅,不由外水之衝射也。萬曆七年,潘季馴始慮黃流倒灌小河、白洋等口,挾諸河水衝射祖陵,乃作歸仁堤為保障計,復張大其說,謂祖陵命脈全賴此堤。習聞其說者,遂疑黃堌之決,下齧歸仁,不知黃堌一決,下流易洩,必無上灌之虞。況今小河不日竣工,引河復歸故道,雲歸仁益遠,奚煩過計為?」報可。

一魁既開小浮橋,築義安山,濬小河口,引武沂泉濟運。及是年四月,河復大決黃堌口,溢夏邑、永城,由宿州府離橋出宿遷新河口入大河,其半由徐州入舊河濟運。上源水枯,而義安束水橫壩復衝二十餘丈,小浮橋水脈微細,二洪告涸,運道阻澀。一魁因議挑黃堌口迤上埽灣、淤嘴二處,且大挑其下李吉口北下濁河,救小浮橋上流數十里之涸。復上言:「黃河南旋至韓家道、盤岔河、丁家莊,俱岸闊百丈,深踰二丈,乃銅幫鐵底故道也。至劉家AH,始強半南流,得山西坡、永涸湖以為壑,出溪口入符離河,亦故道也。惟徐、邳運道淺涸,所以首議開小浮橋,再加挑闢,必大為運道之利。乃欲自黃堌挽回全河,必須挑四百里淤高之河身,築三百里南岸之長堤,不惟所費不貲,竊恐後患無已。」御史楊光訓等亦議挑埽灣直渠,展濟濁河,及築山西坡歸仁堤,與一魁合,獨鈇異議。帝命從一魁言。

一魁復言:「歸仁在西北,泗州在東南,相距百九十里,中隔重岡疊嶂。且歸仁之北有白洋河、朱家溝、周家溝、胡家溝、小河口洩入運河,勢如建瓴,即無歸仁,祖陵無足慮。濁河淤墊,高出地上,曹、單間闊一二百丈,深二三丈,尚不免橫流,徐、邳間僅百丈,深止丈餘,徐西有淺至二三尺者,而夏、永、韓家道口至符離,河闊深視曹、單,避高就下,水之本性,河流所棄,自古難復。且運河本籍山東諸泉,不資黃水,惟當仿正統間二洪南北口建閘之制,於鎮口之下,大浮橋之上,呂梁之下洪,邳州之沙坊,各建石閘,節宣汶、泗,而以小浮橋、沂河口二水助之,更於鎮口西築壩截黃,開唐家口而注之龍溝,會小浮橋入運,以杜灌淤鎮口之害,實萬全計也。」報可。

二十六年春,從楊光訓等議,撤鈇,命一魁兼管漕運。六月,召一魁掌部事,命劉東星為工部侍郎,總理河漕。

二十七年春,東星上言:「河自商、虞而下,由丁家道口抵韓家道口、趙家圈、石將軍廟、兩河口,出小浮橋下二洪,乃賈魯故道也。自元及我朝行之甚利。嘉靖三十七年北徙濁河,而此河遂淤。潘季馴議復開之,以工費浩繁而止。今河東決黃堌,由韓家道口至趙家圈百餘里,衝刷成河,即季馴議復之故道也。由趙家圈至兩河口,直接三仙臺新渠,長僅四十里,募夫五萬浚之,踰月當竣,而大挑運河,小挑濁河,俱可節省。惟李吉口故道嘗挑復淤,去冬已挑數里,前功難棄,然至鎮口三百里而遙,不若趙家圈至兩河口四十里而近。況大浮橋已建閘蓄汶、泗之水,則鎮口濟運亦無藉黃流。」報可。十月,功成,加東星工部尚書,一魁及餘官賞賚有差。

初,給事中楊廷蘭因黃堌之決,請開泇河,給事中楊應文亦主其說。既而直隸巡按御史佴祺復言之。東星既開趙家圈,復採眾說,鑿泇河,以地多沙石,工未就而東星病。河既南徙,李吉口淤水殿日高,北流遂絕,而趙家圈亦日就淤塞,徐、邳間三百里,河水尺餘,糧艘阻塞。

二十九年秋,工科給事中張問達疏論之。會開、歸大水,河漲商丘,決蕭家口,全河盡南注。河身變為平沙,商賈舟膠沙上。南岸蒙墻寺忽徙置北岸,商、虞多被淹沒,河勢盡趨東南,而黃堌斷流。河南巡撫曾如春以聞,曰:「此河徙,非決也。」問達復言:「蕭家口在黃堌上流,未有商舟不能行於蕭家口而能行於黃堌以東者,運艘大可慮。」帝從其言,方命東星勘議,而東星卒矣。問達復言:「運道之壞,一因黃堌口之決,不早杜塞;更因并力泇河,以致趙家圈淤塞斷流,河身日高,河水日淺,而蕭家口遂決,全河奔潰入淮,勢及陵寢。東星已逝,宜急補河臣,早定長策。」大學士沈一貫、給事中桂有根皆趣簡河臣。

御史高舉獻三策。請浚黃堌口以下舊河,引黃水注之東,遂塞黃堌口,而遏其南,俟舊河衝刷深,則並塞新決之口。其二則請開泇河及膠萊河,而言河漕不宜並於一人,當選擇分任其事。江北巡按御史吳崇禮則請自蒙墻寺西北黃河灣曲之所,開浚直河,引水東流。且浚李吉口至堅城集淤道三十餘里,而盡塞黃堌以南決口,使河流盡歸正漕。工部尚書一魁酌舉崇禮之議,以開直河、塞黃堌口、濬淤道為正策,而以泇河為旁策,膠萊為備策。帝命急挑舊河,塞決口,且兼挑泇河以備用。下山東撫按勘視膠萊河。

三十年春,一魁覆河撫如春疏言:「黃河勢趨邳、宿,請築汴堤自歸德至靈、虹,以障南徙。且疏小河口,使黃流盡歸之,則彌漫自消,祖陵可無患。」帝喜納之。已而言者再疏攻一魁。帝以一魁不塞黃堌口,致衝祖陵,斥為民。復用崇禮議,分設河漕二臣,命如春為工部侍郎,總理河道。如春議開虞城王家口,挽全河東歸,須費六十萬。

三十一年春,山東巡撫黃克纘言:「王家口為蒙墻上源,上流既達,則下流不可旁洩,宜遂塞蒙墻口。」從之。時蒙墻決口廣八十餘丈,如春所開新河未及其半,塞而注之,慮不任受。有獻策者言:「河流既回,勢若雷霆,藉其勢衝之,淺者可深也。」如春遂令放水,水皆泥沙,流少緩,旋淤。夏四月,水暴漲,衝魚、單、豐、沛間,如春以憂卒。乃命李化龍為工部侍郎,代其任。

給事中宋一韓言:「黃河故道已復,陵、運無虞。決口懼難塞,宜深濬堅城以上淺阻,而增築徐、邳兩岸,使下流有所容,則舊河可塞。」給事中孟成己言:「塞舊河急,而濬新河尤急。」化龍甫至,河大決單縣蘇家莊及曹縣縷堤,又決沛縣四鋪口太行堤,灌昭陽湖,入夏鎮,橫衝運道。化龍議開泇河,屬之邳州直河,以避河險。給事中侯慶遠因言:「泇河成,則他工可徐圖,第毋縱河入淮。淮利則洪澤水減,而陵自安矣。」

三十二年正月,部覆化龍疏,大略言:「河自歸德而下,合運入海,其路有三:由蘭陽道考城至李吉口,過堅城集,入六座樓,出茶城而向徐、邳,是名濁河,為中路;由曹、單經豐、沛,出飛雲橋,汎昭陽湖,入龍塘,出秦溝而向徐、邳,是名銀河,為北路;由潘家口過司家道口,至何家堤,經符離,道睢寧,入宿遷,出小河口入運,是名符離河,為南路。南路近陵,北路近運,惟中路既遠於陵,且可濟運,前河臣興役未竣,而河形尚在。」因奏開泇有六善。帝從其議。

工部尚書姚繼可言:「黃河衝徙,河臣議於堅城集以上開渠引河,使下流疏通,復分六座樓、苑家樓二路殺其水勢,既可移豐、沛之患,又不至沼碭山之城。開泇分黃,兩工并舉,乞速發帑以濟。」允之。八月,化龍奏分水河成。事具《泇河志》中。加化龍太子少保兵部尚書。會化龍丁艱候代,命曹時聘為工部侍郎,總理河道。是秋,河決豐縣,由昭陽湖穿李家港口,出鎮口,上灌南陽,而單縣決口復潰,魚臺、濟寧間平地成湖。

三十三年春,化龍言:「豐之失,由巡守不嚴,單之失,由下埽不早,而皆由蘇家莊之決。南直、山東相推諉,請各罰防河守臣。至年來緩堤防而急挑浚,堤壞水溢,不咎守堤之不力,惟委浚河之不深。夫河北岸自曹縣以下無入張秋之路,南岸自虞城以下無入淮之路,惟由徐、邳達鎮口為運道。故河北決曹、鄆、豐、沛間,則由昭陽湖出李家口,而運道溢;南決虞、夏、徐、邳間,則由小河口及白洋河,而運道涸。今泇河既成,起直河至夏鎮,與黃河隔絕,山東、直隸間,河不能制運道之命。獨硃旺口以上,決單則單沼,決曹則曹魚,及豐、沛、徐、邳、魚、碭皆命懸一線堤防,何可緩也?至中州荊隆口、銅瓦廂皆入張秋之路,孫家渡、野雞岡、蒙墻寺皆入淮之路,一不守,則北壞運,南犯陵,其害甚大。請西自開封,東至徐、邳,無不守之地,上自司道,下至府縣,無不守之人,庶幾可息河患。」乃敕時聘申飭焉。

其秋,時聘言:「自蘇莊一決,全河北注者三年。初泛豐、沛,繼沼單、魚,陳燦之塞不成,南陽之堤盡壞。今且上灌全濟,旁侵運道矣。臣親詣曹、單,上視王家口新築之壩,下視朱旺口北潰之流,知河之大可憂者三,而機之不可失者二。河決行堤,泛溢平地,昭陽日墊,下流日淤,水出李家口者日漸微緩,勢不得不退而上溢。溢於南,則孫家渡、野雞岡皆入淮故道,毋謂蒙墻已塞,而無憂於陵。溢於北,則芝麻莊、荊隆口皆入張秋故道,毋謂泇役已成,而無憂於運。且南之夏、商,北之曹、濮,其地益插,其禍益烈,其挽回益不易,毋謂災止魚、濟,而無憂於民。顧自王家口以達朱旺,新導之河在焉。疏其下流以出小浮橋,則三百里長河暢流,機可乘者一。

自徐而下,清黃並行,沙隨水刷,此數十年所未有,因而導水歸徐,容受有地,機可乘者二。臣與諸臣熟計,河之中路有南北二支:北出濁河,嘗再疏再壅;惟南出小浮橋,地形卑下,其勢甚順,度長三萬丈有奇,估銀八十萬兩。公諸虛耗,乞多方處給。」疏上留中。時聘乃大挑朱旺口。十一月興工,用夫五十萬。三十四年四月,工成,自朱旺達小浮橋延袤百七十里,渠廣堤厚,河歸故道。

六月,河決蕭縣郭煖樓人字口,北支至茶城、鎮口。三十五年,決單縣。三十九年六月,決徐州狼矢溝。四十年九月,決徐州三山,衝縷堤二百八十丈,遙堤百七十餘丈,梨林鋪以下二十里正河悉為平陸,邳、睢河水耗竭。總河都御史劉士忠開韓家壩外小渠引水,由是壩以東始通舟楫。四十二年,決靈璧陳鋪。四十四年五月,復決狼矢溝,由蛤鰻、周柳諸湖入泇河,出直口,復與黃會。六月,決開封陶家店、張家灣,由會城大堤下陳留,入亳州渦河。四十七年九月,決陽武脾沙堽,由封丘、曹、單至考城,復入舊河。時朝政日馳,河臣奏報多不省。四十二年,劉士忠卒,總河閱三年不補。四十六年閏四月,始命工部侍郎王佐督河道。河防日以廢壞,當事者不能有為。

天啟元年,河決靈譬雙溝、黃鋪,由永姬湖出白洋、小河口,仍與黃會,故道湮涸。總河侍郎陳道亨役夫築塞。時淮安霪雨連旬,黃、淮暴漲數尺,而山陽裏外河及清河決口匯成巨浸,水灌淮城,民蟻城以居,舟行街市。久之始塞。三年,決徐州青田大龍口,徐、邳、靈、睢河並淤,呂梁城南隅隱,沙高平地丈許,雙溝決口亦滿,上下百五十里悉成平陸。四年六月,決徐州魁山堤,東北灌州城,城中水深一丈三尺,一自南門至雲龍山西北大安橋入石狗湖,一由舊支河南流至鄧二莊,歷租溝東南以達小河,出白洋,仍與黃會。徐民苦淹溺,議集貲遷城。給事中陸文獻上徐城不可遷六議。而勢不得已,遂遷州治於雲龍,河事置不講矣。六年七月,河決淮安,逆入駱馬湖,灌邳、宿。

崇禎二年春,河決曹縣十四鋪口。四月,決睢寧,至七月中,城盡圮。總河侍郎李若星請遷城避之,而開邳州壩洩水入故道,且塞曹家口匙頭灣,逼水北注,以減睢寧之患。從之。四年夏,河決原武湖村鋪,又決封丘荊隆口,敗曹縣塔兒灣大行堤。六月黃、淮交漲,海口壅塞,河決建義諸口,下灌興化、鹽城,水深二丈,村落盡漂沒。逡巡踰年,始議築塞。興工未幾,伏秋水發,黃、淮奔注,興、鹽為壑,而海潮復逆衝,壞范公堤。軍民及商灶戶死者無算,少壯轉徙,丐江、儀、通、泰間,盜賊千百嘯聚。至六年,鹽城民徐瑞等言其狀。帝憫之,命議罰河曹官。而是時,總河朱光祚方議開高堰三閘。淮、揚在朝者合疏言:「建義諸口未塞,民田盡沉水底。三閘一開,高、寶諸邑蕩為湖海,而漕糧鹽課皆害矣。高堰建閘始於萬曆二十三年,未幾全塞。今高堰日壞,方當急議修築,可輕言開浚乎?」帝是其言,事遂寢。又從御史吳振纓請,修宿、寧上下西北舊堤,以捍歸仁。七年二月,建義決口工成,賜督漕尚書楊一鵬、總河尚書劉榮嗣銀幣。

八年九月,榮嗣得罪。初,榮嗣以駱馬湖運道潰淤,創挽河之議,起宿遷至徐州,別鑿新河,分黃水注其中,以通漕運。計工二百餘里,金錢五十萬。而其所鑿邳州上下,悉黃河故道,濬尺許,其下皆沙,挑掘成河,經宿沙落,河坎復平,如此者數四。迨引黃水入其中,波流迅急,沙隨水下,率淤淺不可以舟。及漕舟將至,而駱馬湖之潰決適平,舟人皆不願由新河。榮嗣自往督之,欲繩以軍法。有入者輒苦淤淺,弁卒多怨。巡漕御史倪于義劾其欺罔誤工,南京給事中曹景參復重劾之,逮問,坐贓,父子皆瘐死。郎中胡璉分工獨多,亦坐死。其後駱馬湖復潰,舟行新河,無不思榮嗣功者。

當是時,河患日棘,而帝又重法懲下,李若星以修濬不力罷官,朱光祚以建義蘇嘴決口逮繫。六年之中,河臣三易。給事中王家彥嘗切言之。光祚亦竟瘐死。而繼榮嗣者周鼎修泇利運頗有功,在事五年,竟坐漕舟阻淺,用故決河防例,遣戍煙瘴。給事中沈胤培、刑部侍郎惠世揚、總河侍郎張國維各疏請寬之,乃獲宥免雲。

十五年,流賊圍開封久,守臣謀引黃河灌之。賊偵知,預為備。乘水漲,令其黨決河灌城,民盡溺死。總河侍郎張國維方奉詔赴京,奏其狀。山東巡撫王永吉上言:「黃河決汴城,直走睢陽,東南注鄢陵、鹿邑,必害亳、泗,侵祖陵,而邳、宿運河必涸。」帝令總河侍郎黃希憲急往捍禦,希憲以身居濟寧不能攝汴,請特設重臣督理。命工部侍郎周堪賡督修汴河。

十六年二月,堪賡上言:「河之決口有二:一為朱家寨,寬二里許,居河下流,水面寬而水勢緩;一為馬家口,寬一里餘,居河上流,水勢猛,深不可測。兩口相距三十里,至汴堤之外,合為一流,決一大口,直衝汴城以去,而河之故道則涸為平地。怒濤千頃,工力難施,必廣濬舊渠,遠數十里,分殺水勢,然後畚鍤可措。顧築浚並舉,需夫三萬。河北荒旱,兗西兵火,竭力以供,不滿萬人,河南萬死一生之餘,未審能應募否,是不得不借助於撫鎮之兵也。」乃敕兵部速議,而令堪賡刻期興工。至四月,塞朱家寨決口,修堤四百餘丈。馬家口工未就,忽衝東岸,諸埽盡漂沒。堪賡請停東岸而專事西岸。帝令急竣工。

六月,堪賡言:「馬家決口百二十丈,兩岸皆築四之一,中間七十餘丈,水深流急,難以措手,請俟霜降後興工。」已而言:「五月伏水大漲,故道沙灘壅涸者刷深數丈,河之大勢盡歸於東,運道已通,陵園無恙。」疏甫上,決口再潰。帝趣鳩工,未奏績而明亡。

○運河上

明成祖肇建北京,轉漕東南,水陸兼輓,仍元人之舊,參用海運。逮會通河開,海陸並罷。南極江口,北盡大通橋,運道三千餘里。綜而計之,自昌平神山泉諸水,匯貫都城,過大通橋,東至通州入白河者,大通河也。自通州而南至直沽,會衛河入海者,白河也。自臨清而北至直沽,會白河入海者,衛水也。自汶上南旺分流,北經張秋至臨清,會衛河,南至濟寧天井閘,會泗、沂、洸三水者,汶水也。自濟寧出天井閘,與汶合流,至南陽新河,舊出茶城,會黃、沁後出夏鎮,循泇河達直口,入黃濟運者,泗、洸、小沂河及山東泉水也。自茶城秦溝,南歷徐、呂,浮邳,會大沂河,至清河縣入淮後,從直河口抵清口者,黃河水也。自清口而南,至於瓜、儀者,淮、揚諸湖水也。過此則長江矣。長江以南,則松、蘇、浙江運道也。淮、揚至京口以南之河,通謂之轉運河,而由瓜、儀達淮安者,又謂之南河,由黃河達豐、沛曰中河,由山東達天津曰北河,由天津達張家灣曰通濟河,而總名曰漕河。其踰京師而東若薊州,西北若昌平,皆嘗有河通,轉漕餉軍。

漕河之別,曰白漕、衛漕、閘漕、河漕、湖漕、江漕、浙漕。因地為號,流俗所通稱也。淮、揚諸水所匯,徐、兗河流所經,疏瀹決排,繄人力是繫,故閘、河、湖於轉漕尤急。

閘漕者,即會通河。北至臨清,與衛河會,南出茶城口,與黃河會,資汶、洸、泗水及山東泉源。泉源之派有五。曰分水者,汶水派也,泉百四十有五。曰天井者,濟河派也,泉九十有六。曰魯橋者,泗河派也,泉二十有六。曰沙河者,新河派也,二十有八。曰邳州者,沂河派也,泉十有六。諸泉所匯為湖,其浸十五。曰南旺,東西二湖,周百五十餘里,運渠貫其中。北曰馬蹋,南曰蜀山,曰蘇魯。又南曰馬場。又南八十里曰南陽,亦曰獨山,周七十餘里。北曰安山,周八十三里。南曰大、小昭陽,大湖袤十八里,小湖殺三之一,周八十餘里。由馬家橋留城閘而南,曰武家,曰赤山,曰微山,曰呂孟,曰張王諸湖,連注八十里,引薛河由地濱溝出,會於赤龍潭,並趨茶城。自南旺分水北至臨清三百里,地降九十尺,為閘二十有一;南至鎮口三百九十里,地降百十有六尺,為閘二十有七。其外又有積水、進水、減水、平水之閘五十有四。又為壩二十有一,所以防運河之洩,佐閘以為用者也。其後開泇河二百六十里,為閘十一,為壩四。運舟不出鎮口,與黃河會於董溝。

河漕者,即黃河。上自茶城與會通河會,下至清口與淮河會。其道有三:中路曰濁河,北路曰銀河,南路曰符離河。南近陵,北近運,惟中路去陵遠,於運有濟。而河流遷徙不常,上流苦潰,下流苦淤。運道自南而北,出清口,經桃、宿,溯二洪,入鎮口,陟險五百餘里。自二洪以上,河與漕不相涉也。至泇河開而二洪避,董溝闢而直河淤,運道之資河者二百六十里而止,董溝以上,河又無病於漕也。

湖漕者,由淮安抵揚州三百七十里,地卑積水,匯為澤國。山陽則有管家、射陽,寶應則有白馬、汜光,高郵則有石臼、甓社、武安、邵伯諸湖。仰受上流之水,傍接諸山之源,巨浸連亙,由五塘以達於江。慮淮東侵,築高家堰拒其上流,築王簡、張福二堤禦其分洩。慮淮侵而漕敗,開淮安永濟、高郵康濟、寶應弘濟三月河以通舟。至揚子灣東,則分二道:一由儀真通江口,以漕上江湖廣、江西;一由瓜洲通西江嘴,以漕下江兩浙。本非河道,專取諸湖之水,故曰湖漕。

太祖初起大軍北伐,開蹋場口、耐牢坡,通漕以餉梁、晉。定都應天,運道通利:江西、湖廣之粟,浮江直下;浙西、吳中之粟,由轉運河;鳳、泗之粟,浮淮;河南、山東之粟,下黃河。嘗由開封運粟,溯河達渭,以給陜西,用海運以餉遼卒,有事於西北者甚鮮。淮、揚之間,築高郵湖堤二十餘里,開寶應倚湖直渠四十里,築堤護之。他小修築,無大利害也。

永樂四年,成祖命平江伯陳瑄督轉運,一仍由海,而一則浮淮入河,至陽武,陸挽百七十里抵衛輝,浮於衛,所謂陸海兼運者也。海運多險,陸挽亦艱。九年二月乃用濟寧州同知潘叔正言,命尚書宋禮、侍郎金純、都督周長浚會通河。會通河者,元轉漕故道也,元末已廢不用。洪武二十四年,河決原武,漫安山湖而東,會通盡淤,至是復之。由濟寧至臨清三百八十五里,引汶、泗入其中。泗出泗水陪尾山,四泉並發,西流至兗州城東,合於沂。汶河有二:小汶河出新泰宮山下;大汶河出泰安仙臺嶺南,又出萊蕪原山陰及寨子村。俱至靜豐鎮合流,饒徂徠山陽,而小汶河來會。經寧陽北堈城,西南流百餘里,至汶上。其支流曰洸河,出堈城西南,流三十里,會寧陽諸泉,經濟寧東,與泗合。元初,畢輔國始於堈城左汶水陰作斗門,導汶入洸。至元中,又分流北入濟,由壽張至臨清,通漳、御入海。

南旺者,南北之脊也。自左而南,距濟寧九十里,合沂、泗以濟;自右而北,距臨清三百餘里,無他水,獨賴汶。禮用汶上老人白英策,築壩東平之戴村,遏汶使無入洸,而盡出南旺,南北置閘三十八。又開新河,自汶上袁家口左徙五十里至壽張之沙灣,以接舊河。其秋,禮還,又請疏東平東境沙河淤沙三里,築堰障之,合馬常泊之流入會通濟運。又於汶上、東平、濟寧、沛縣並湖地設水櫃、陡門。在漕河西者曰水櫃,東者曰陡門,櫃以蓄泉,門以洩漲。純復濬賈魯河故道,引黃水至塌場口會汶,經徐、呂入淮。運道以定。

其後宣宗時,嘗發軍民十二萬,濬濟寧以北自長溝至棗林閘百二十里,置閘諸淺,濬湖塘以引山泉。正統時,濬滕、沛淤河,又於濟寧、勝三州縣疏泉置閘,易金口堰土壩為石,蓄水以資會通。景帝時,增置濟寧抵臨清減水閘。天順時,拓臨清舊閘,移五十丈。憲宗時,築汶上、濟寧決堤百餘里,增南旺上、下及安山三閘。命工部侍郎杜謙勘治汶、泗、洸諸泉。武宗時,增置汶上袁家口及寺前鋪石閘,浚南旺淤八十里,而閘漕之治詳。惟河決則挾漕而去,為大害。

陳瑄之督運也,於湖廣、江西造平底淺般三千艘。二省及江、浙之米皆由江以入,至淮安新城,盤五壩過淮。仁、義二壩在東門外東北,禮、智、信三壩在西門外西北,皆自城南引水抵壩口,其外即淮河。清江浦者,直淮城西,永樂二年嘗一修閘。其口淤塞,則漕船由二壩,官民商船由三壩入淮,輓輸甚勞苦。瑄訪之故老,言:「淮城西管家湖西北,距淮河鴨陳口僅二十里,與清江口相值,宜鑿為河,引湖水通漕,宋喬維嶽所開沙河舊渠也。」瑄乃鑿清江浦,導水由管家湖入鴨陳口達淮。十三年五月,工成。緣西湖築堤亙十里以引舟。淮口置四閘,曰移風、清江、福興、新莊。以時啟閉,嚴其禁。並濬儀真、瓜洲河以通江湖,鑿呂梁、百步二洪石以平水勢,開泰州白塔河以達大江。築高郵河堤,堤內鑿渠四十里。久之,復置呂梁石閘,並築寶應、汜光、白馬諸湖堤,堤皆置涵洞,互相灌注。是時淮上、徐州、濟寧、臨清、德州皆建倉轉輸。濱河置舍五百六十八所,舍置淺夫。水澀舟膠,俾之導行。增置淺船三千餘艘。設徐、沛、沽頭、金溝、山東、穀亭、魯橋等閘。自是漕運直達通州,而海陸運俱廢。

宣德六年用御史白圭言,濬金龍口,引河水達徐州以便漕。末年至英宗初再浚,並及鳳池口水,徐、呂二洪,西小河,而會通安流,自永、宣至正統間凡數十載。至十三年,河決滎陽,東衝張秋,潰沙灣,運道始壞。命廷臣塞之。

景泰三年五月,堤工乃完。未匝月而北馬頭復決,掣漕流以東。清河訓導唐學成言:「河決沙灣,臨清告涸。地卑堤薄,黃河勢急,故甫完堤而復決也。臨清至沙灣十二閘,有水之日,其勢甚陡。請於臨清以南浚月河通舟,直抵沙灣,不復由閘,則水勢緩而漕運通矣。」帝即命學成與山東巡撫洪英相度。工部侍郎趙榮則言:「沙灣抵張秋岸薄,故數決。請於決處置減水石壩,使東入鹽河,則運河之水可蓄。然後厚堤岸,填決口,庶無後患。」

明年四月,決口方畢工,而減水壩及南分水墩先敗,已,復盡衝墩岸橋梁,決北馬頭,掣漕水入鹽河,運舟悉阻。教諭彭塤請立閘以制水勢,開河以分上流。御史練綱上其策。詔下尚書石璞。璞乃鑿河三里,以避決口,上下與運河通。是歲,漕舟不前者,命漕運總兵官徐恭姑輸東昌、濟寧倉。及明年,運河膠淺如故。恭與都御史王竑言:「漕舟蟻聚臨清上下,請亟敕都御史徐有貞築塞沙灣決河。」有貞不可,而獻上三策,請置水閘,開分水河,挑運河。

六年三月,詔群臣集議方略。工部尚書江淵等請用官軍五萬以濬運。有貞恐役軍費重,請復陳瑄舊制,置撈淺夫,用沿河州縣民,免其役。

五月,浚漕工竣。七月,沙灣決口工亦竣,會通復安。都御史陳泰一浚淮、揚漕河,築口置壩。黃河嘗灌新莊閘至清江浦三十餘里,淤淺阻漕,稍稍濬治,即復其舊。英宗初,命官督漕,分濟寧南北為二,侍郎鄭辰治其南,副都御史賈諒治其北。

成化七年,又因廷議,分漕河沛縣以南、德州以北及山東為三道,各委曹郎及監司專理,且請簡風力大臣總理其事。始命侍郎王恕為總河。二十一年敕工部侍郎杜謙浚運道,自通州至淮、揚,會山東、河南撫按相度經理。

弘治二年,河復決張秋,衝會通河,命戶部侍郎白昂相治。昂奏金龍口決口已淤,河並為一大支,由祥符合沁下徐州而去。其間河道淺隘,宜於所經七縣,築堤岸以衛張秋。下工部議,從其請。昂又以漕船經高郵甓社湖多溺,請於堤東開衣復河西四十里以通舟。越四年,河復決數道入運河,壞張秋東堤,奪汶水入海,漕流絕。時工部侍郎陳政總理河道,集夫十五萬,治未效而卒。

六年春,副都御史劉大夏奉敕往治決河。夏半,漕舟鱗集,乃先自決口西岸鑿月河以通漕。經營二年,張秋決口就塞,復築黃陵岡上流。於是河復南下,運道無阻。乃改張秋曰安平鎮,建廟賜額曰顯惠神祠,命大學士王鏊紀其事,勒於石。而白昂所開高郵衣復河亦成,賜名康濟,其西岸以石甃之。又甃高郵堤,自杭家閘至張家鎮凡三十里。高郵堤者,洪武時所築也。陳瑄因舊增築,延及寶應,土人相沿謂之老堤。正統三年易土以石。成化時,遣官築重堤於高郵、邵伯、寶應、白馬四湖老堤之東。而王恕為總河,修淮安以南諸決堤,且濬淮、揚漕河。重湖需民盜決溉田之罰,造閘達以儲湖水。及大夏塞張秋,而昂又開康濟,漕河上下無大患者二十餘年。

十六年,巡撫徐源言:「濟寧地最高,必引上源洸水以濟,其口在堈城石瀨之上。元時治閘作堰,使水盡入南旺,分濟南北運。成化間,易土以石。夫土堰之利,水小則遏以入洮,水大則閉閘以防沙壅,聽其漫堰西流。自石堰成,水遂橫溢,石堰既壞,民田亦衝。洸河沙塞,雖有閘門,壓不能啟。乞毀石復土,疏洸口壅塞以至濟寧,而築堈城迤西春城口子決岸。」帝命侍郎李遂往勘,言:「堈城石堰,一可遏淤沙,不為南旺湖之害,一可殺水勢,不慮戴村壩之衝,不宜毀。近堰積沙,宜浚。堈城稍東有元時舊閘,引洸水入濟寧,下接徐、呂漕河。東平州戴村,則汶水入海故道也。自永樂初,橫築一壩,遏汶入南旺湖,漕河始通。今自分水龍王廟至天井閘九十里,水高三丈有奇,若洸河更濬而深,則汶流盡向濟寧而南,臨清河道必涸。洸口不可濬。堈城口至柳泉九十里,無關運道,可弗事。柳泉至濟寧,汶、泗諸水會流處,宜疏者二十餘里。春城口,外障汶水,內防民田,堤卑岸薄,宜與戴村壩並修築。」從之。正德四年十月,河決沛縣飛雲橋,入運。尋塞。

世宗之初,河數壞漕。嘉靖六年,光祿少卿黃綰論泉源之利,言:「漕河泉源皆發山東南旺、馬場、樊村、安山諸湖。泉水所鐘,亟宜修濬,且引他泉並蓄,則漕不竭。南旺、馬場堤外孫村地窪,若瀦為湖,改作漕道,尤可免濟寧高原淺澀之苦。」帝命總河侍郎章拯議。而拯以黃水入運,運船阻沛上,方為御史吳仲所劾。拯言:「河塞難遽通,惟金溝口迤北新衝一渠可令運船由此入昭陽湖,出沙河板橋。其先阻淺者,則西歷雞塚寺,出廟道北口通行。下部併議,未決。給事中張嵩言:「昭陽湖地庳,河勢高,引河灌湖,必致彌漫,使湖道復阻。請罷拯,別推大臣。」部議如嵩言。拯再疏自劾,乞罷,不許。卒引運船道湖中。其冬,詔拯還京別敘,而命擇大臣督理。

諸大臣多進治河議。詹事霍韜謂:「前議役山東、河南丁夫數萬,疏浚淤沙以通運。然沙隨水下,旋濬旋淤。今運舟由昭陽湖入雞鳴臺至沙河,迂回不過百里。若沿湖築堤,浚為小河,河口為閘,以待蓄洩,水溢可避風濤,水涸易為疏浚。三月而土堤成,一年而石堤成,用力少,取效速。黃河愈溢,運道愈利,較之役丁夫以浚淤土,勞逸大不侔也。」尚書李承勛謂:「於昭陽湖左別開一河,引諸泉為運道,自留城沙河為尤便。」與都御史胡世寧議合。七年正月,總河都御史盛應期奏如世寧策,請於昭陽湖東鑿新河,自汪家口南出留城口,長百四十里,刻期六月畢工。工未半,而應期罷去,役遂已。其後三十年,朱衡始循其遺跡,浚而成之。是年冬,總河侍郎潘希會加築濟、沛間東西兩堤,以拒黃河。

十九年七月,河決野雞岡,二洪涸。督理河漕侍郎王以旂請浚山東諸泉以濟運,且築長堤聚水,如閘河制。遂清舊泉百七十八,開新泉三十一。以旂復奏四事。一請以諸泉分隸守土官兼理其事,毋使堙塞。一請於境山鎮、徐、呂二洪之下,各建石閘,蓄水數尺以行舟,旁留月河以洩暴汛;築四木閘於武家溝、小河口、石城、匙頭灣,而置方船於沙坊等淺,以備撈浚。一言漕河兩岸有南旺、安山、馬場、昭陽四湖,名為水櫃,所以匯諸泉濟漕河也。豪強侵占,蓄水不多,而昭陽一湖淤成高地,大非國初設湖初意。宜委官清理,添置閘、壩、斗門,培築堤岸,多開溝渠,浚深河底,以復四櫃。一言黃河南徙,舊閘口俱塞,惟孫繼一口獨存。導河出徐州小浮橋,下徐、呂二洪,此濟運之大者。請於孫繼口多開一溝,及時疏瀹,庶二洪得濟。帝可其奏,而以管泉專責之部曹。

徐、呂二洪者,河漕咽喉也。自陳瑄鑿石疏渠,正統初,復浚洪西小河。漕運參將湯節又以洪迅敗舟,於上流築堰,逼水歸月河,河南建閘以蓄水勢。成化四年,管河主簿郭昇以大石築兩堤,錮以鐵錠,鑿外洪敗船惡石三百,而平築裏洪堤岸,又甃石岸東西四百餘丈。十六年增甃呂梁洪石堤、石壩二百餘丈,以資牽挽。及是建閘,行者益便之。

四十四年七月,河大決沛縣,漫昭陽湖,由沙河至二洪,浩渺無際,運道淤塞百餘里。督理河漕尚書朱衡循覽盛應期所鑿新河遺跡,請開南陽、留城上下。總河都御史潘季馴不可。衡言:「是河直秦溝,有所束隘。伏秋黃水盛,昭陽受之,不為壑也。」乃決計開浚,身自督工,重懲不用命者。給事中鄭欽劾衡故興難成之役,虐民倖功。朝廷遣官勘新舊河孰利。給事中何起鳴勘河還,言:「舊河難復有五,而新河之難成者亦有三。顧新河多舊堤高阜,黃水難侵,脧而通之,運道必利。所謂三難者,一以夏村迤北地高,恐難接水,然地勢高低,大約不過二丈,一視水平加深,何患水淺。一以三河口積沙深厚,水勢湍急,不無阻塞,然建壩攔截,歲一挑浚之,何患沙壅。一以馬家橋築堤,微山取土不便,又恐水口投埽,勢必不堅,然使委任得人,培築高厚,無必不可措力之理。開新河便。」下廷臣集議,言新河已有次第,不可止。況百中橋至留城白洋淺,出境山,疏浚補築,亦不全棄舊河,群議俱合。帝意乃決。時大雨,黃水驟發,決馬家橋,壞新築東西二堤。給事中王元春、御史黃襄皆劾衡欺誤,起鳴亦變其說。會衡奏新舊河百九十四里俱已流通,漕船至南陽出口無滯。詔留衡與季馴詳議開上源、築長堤之便。

隆慶元年正月,衡請罷上源議,惟開廣秦溝,堅築南長堤。五月,新河成,西去舊河三十里。舊河自留城以北,經謝溝、下沽頭、中沽頭、金溝四閘,過沛縣,又經廟道口、湖陵城、孟陽、八里灣、穀亭五閘,而至南陽閘。新河自留城而北,經馬家橋、西柳莊、滿家橋、夏鎮、楊莊、硃梅、利建七閘,至南陽閘合舊河,凡百四十里有奇。又引占魚諸泉及薛河、沙河注其中,而設壩於三河之口,築馬家橋堤,遏黃水入秦溝,運道乃大通。未幾,占魚口山水暴決,沒漕艘。帝從衡請,自東邵開支河三道以分洩之,又開支河於東邵之上,歷東滄橋以達百中橋,鑿豸裏溝諸處為渠,使水入赤山湖,由之以歸呂孟湖,下境山而去。

衡召入為工部尚書,都御史翁大立代,上言:「漕河資泉水,而地形東高西下,非湖瀦之則涸,故漕河以東皆有櫃;非湖洩之則潰,故漕河以西皆有壑。黃流逆奔,則以昭陽湖為散漫之區;山水東突,則以南陽湖為瀦蓄之地。宜由回回墓開通以達鴻溝,令穀亭、湖陵之水皆入昭陽湖,即浚鴻溝廢渠,引昭陽湖水沿渠東出留城。其湖地退灘者,又可得田數千頃。」大立又言:「薛河水湍悍,今盡注赤山湖,入微山湖以達呂孟湖,此尚書衡成績也。惟呂孟之南為邵家嶺,黃流填淤,地形高仰,秋水時至,翕納者小,浸淫平野,奪民田之利。微山之西為馬家橋,比草創一堤以開運道,土未及堅而時為積水所撼,以尋丈之址,二流夾攻,慮有傾圮。宜鑿邵家嶺,令水由地濱溝出境山以入漕河,則湖地可耕,河堤不潰。更於馬家橋建減水閘,視旱澇為啟閉,乃通漕長策也。」並從之。

三年七月,河決沛縣,茶城淤塞,糧艘二千餘皆阻邳州。大立言:「臣按行徐州,循子房山,過梁山,至境山,入地濱溝,直趨馬家橋,上下八十里間,可別開一河以漕。」即所謂泇河也。請集廷議,上即命行之。未幾,黃落漕通,前議遂寢。時淮水漲溢,自清河至淮安城西淤三十餘里,決禮、信二壩出海,寶應湖堤多壞。山東諸水從直河出邳州。大立以聞。其冬,自淮安板閘至清河西湖嘴開浚垂成,而裏口復塞。督漕侍郎趙孔昭言:「清江一帶黃河五十里,宜築堰以防河溢;淮河高良澗一帶七十餘里,宜築堰以防淮漲。」帝令亟浚里口,與大立商築堰事宜,并議海口築塞及寶應月河二事。

四年六月,淮河及鴻溝境山疏濬工竣。大立方奏聞,諸水忽驟溢,決仲家淺,與黃河合,茶城復淤。未幾,自泰山廟至七里溝,淮河淤十餘里,其水從朱家溝旁出,至清河縣河南鎮以合於黃河。大立請開新莊閘以通回船,兼浚古睢河,洩二洪水,且分河自魚溝下草灣,保南北運道。帝命新任總河都御史潘季馴區畫。頃之,河大決邳州,睢寧運道淤百餘里。大立請開泇口、蕭縣二河。會季馴築塞諸決,河水歸正流,漕船獲通。大立、孔昭皆以遲誤漕糧削籍,開泇之議不果行。

五年四月,河復決邳州王家口,自雙溝而下,南北決口十餘,損漕船運軍千計,沒糧四十萬餘石,而匙頭灣以下八十里皆淤。於是膠、萊海運之議紛起。會季馴奏邳河功成。帝以漕運遲,遣給事中雒遵往勘。總漕陳炌及季馴俱罷官。

六年,從雒遵言,修築茶城至清河長堤五百五十里,三里一鋪,鋪十夫,設官畫地而守。又接築茶城至開封兩岸堤。從朱衡言,繕豐、沛大黃堤。衡又言:「漕河起儀真訖張家灣二千八百餘里,河勢凡四段,各不相同。清江浦以南,臨清以北,皆遠隔黃河,不煩用力。惟茶城至臨清,則閘諸泉為河,與黃相近。清河至茶城,則黃河即運河也。茶城以北,當防黃河之決而入;茶城以南,當防黃河之決而出。防黃河即所以保運河,故自茶城至邳、遷,高築兩堤,宿遷至清河,盡塞缺口,蓋以防黃水之出,則正河必淤,昨歲徐、邳之患是也。自茶城秦溝口至豐、沛、曹、單,創築增築以接縷水舊堤,蓋以防黃水之入,則正河必決,往年曹、沛之患是也。二處告竣,故河深水束,無旁決中潰之虞。水市縣之窯子頭至秦溝口,應築堤七十里,接古北堤。徐、邳之間,堤逼河身,宜於新堤外別築遙堤。」詔如其議,以命總河侍郎萬恭。

萬曆元年,恭言:「祖宗時造淺船近萬,非不知滿載省舟之便,以閘河流淺,故不敢過四百石也。其制底平、倉淺,底平則入水不深,倉淺則負載不滿。又限淺船用水不得過六拏,伸大指與食指相距為一拏,六拏不過三尺許,明受水淺也。今不務遵行,而競雇船搭運。雇船有三害,搭運有五害,皆病河道。請悉遵舊制。」從之。

恭又請復淮南平水諸閘,上言:「高、寶諸湖周遭數百里,西受天長七十餘河,徒恃百里長堤,若障之使無疏洩,是潰堤也。以故祖宗之法,偏置數十小閘於長堤之間,又為令曰:「但許深湖,不許高堤」,故設淺船淺夫取湖之淤以厚堤。夫閘多則水易落而堤堅,濬勤則湖愈深而堤厚,意至深遠也。比年畏修閘之勞,每壞一閘即堙一閘,歲月既久,諸閘盡堙,而長堤為死障矣。畏濬淺之苦,每湖淺一尺則加堤一尺,歲月既久,湖水捧起,而高、寶為盂城矣。且湖漕勿堤與無漕同,湖堤勿閘與無堤同。陳瑄大置減水閘數十,湖水溢則瀉以利堤,水落則閉以利漕,最為完計。積久而減水故迹不可復得,湖且沉堤。請復建平水閘,閘欲密,密則水疏,無漲懣患;閘欲狹,狹則勢緩,無齧決虞。」尚書衡覆奏如其請。於是儀直、江都、高郵、寶應、山陽設閘二十三,浚淺凡五十一處,各設撈淺船二,淺夫十。

恭又言:「清江浦河六十里,陳瑄浚至天妃祠東,注於黃河。運艘出天妃口入黃穿清特半餉耳。後黃漲,逆注入口,清遂多淤。議者不制天妃口而遽塞之,令淮水勿與黃值。開新河以接淮河,曰「接清流勿接濁流,可不淤也」。不知黃河非安流之水,伏秋盛發,則西擁淮流數十里,并灌新開河。彼天妃口,一黃水之淤耳。今淮、黃會於新閘開河口,是二淤也。防一淤,生二淤,又生淮、黃交會之淺。歲役丁夫千百,浚治方畢,水過復合。又使運艘迂八里淺滯而始達於清河,孰與出天妃口者之便且利?請建天妃閘,俾漕船直達清河。運盡而黃水盛發,則閉閘絕黃,水落則啟天妃閘以利商船。新河口勿浚可也。」。乃建天妃廟口石閘。

恭又言:「由黃河入閘河為茶城,出臨清板閘七百餘里,舊有七十二淺。自創開新河,汶流平衍,地勢高下不甚相懸,七十淺悉為通渠。惟茶、黃交會間,運盛之時,正值黃河水落之候,高下不相接,是以有茶城黃家閘之淺,連年患之。祖宗時,嘗建境山閘,自新河水平,閘沒泥淖且丈餘。其閘上距黃家閘二十里,下接茶城十里,因故基累石為之,可留黃家閘外二十里之上流,接茶城內十里之下流,且挾二十里水勢,衝十里之狹流,蔑不勝矣。」乃復境山舊閘。

恭建三議,尚書衡覆行之,為運道永利。而是時,茶城歲淤,恭方報正河安流,回空船速出。給事中朱南雍以回空多阻,劾恭隱蔽溺職。帝切責恭,罷去。

三年二月,總河都御史傅希摯請開泇河以避黃險,不果行。希摯又請浚梁山以下,與茶城互用,淤舊則通新而挑舊,淤新則通舊而挑新,築壩斷流,常通其一以備不虞。詔從所請。工未成,而河決崔鎮,淮決高家堰,高郵湖決清水潭、丁志等口,淮城幾沒。知府邵元哲開菊花潭,以洩淮安、高、寶三城之水,東方芻米少通。

明年春,督漕侍郎張翀以築清水潭堤工鉅不克就,欲令糧船暫由圈子田以行。巡按御史陳功不可。河漕侍郎吳桂芳言:「高郵湖老堤,陳瑄所建。後白昂開月河,距湖數里,中為土堤,東為石堤,首尾建閘,名為康濟河。其中堤之西,老堤之東,民田數萬畝,所謂圈子田也。河湖相去太遠,老堤缺壞不修,遂至水入圈田,又成一湖。而中堤潰壞,東堤獨受數百里湖濤,清水潭之決,勢所必至。宜遵弘治間王恕之議,就老堤為月河,但修東西二堤,費省而工易舉。」帝命如所請行之。是年,元哲修築淮安長堤,又疏鹽城石達口下流入海。

五年二月,高郵石堤將成,桂芳請傍老堤十數丈開挑月河。因言:「白昂康濟月河去老堤太遠,人心狃月河之安,忘老堤外捍之力。年復一年,不加省視,老、中二堤俱壞,而東堤不能獨存。今河與老堤近,則易於管攝。」御史陳世寶論江北河道,請於寶應湖堤補石堤以固其外,而於石堤之東復築一堤,以通月河,漕舟行其中。並議行。其冬,高郵湖土石二堤、新開漕河南北二閘及老堤加石、增護堤木城各工竣事。桂芳又與元哲增築山陽長堤,自板閘至黃浦亙七十里,閉通濟閘不用,而建興文閘,且修新莊諸閘,築清江浦南堤,創板閘漕堤,南北與新舊堤接。板閘即故移風閘也。堤、閘並修,淮、揚漕道漸固。

六年,總理河漕都御史潘季馴築高家堰,及清江浦柳浦灣以東加築禮、智二壩,修寶應、黃清等八淺堤,高、寶減水閘四,又拆新莊閘而改建通濟閘於甘羅城南。明初運糧,自瓜、儀至淮安謂之裏河,自五壩轉黃河謂之外河,不相通。及開清江浦,設閘天妃口,春夏之交重運畢,即閉以拒黃。歲久法馳,閘不封而黃水入。嘉靖末,塞天妃口,於浦南三里溝開新河,設通濟閘以就淮水。已又從萬恭言,復天妃閘。未幾又從御史劉光國言,增築通濟,自仲夏至季秋,隔日一放回空漕船。既而啟閉不時,淤塞日甚,開朱家口引清水灌之,僅通舟。至是改建甘羅城南,專向淮水,使河不得直射。

十年,督漕尚書凌雲翼以運船由清江浦出口多艱險,乃自浦西開永濟河四十五里,起城南窯灣,歷龍江閘,至楊家澗出武家墩,折而東,合通濟閘出口。更置閘三,以備清江浦之險。是時漕河就治,淮、揚免水災者十餘年。初,黃河之害漕也,自金龍口而東,則會通以淤。迨塞沙灣、張秋閘,漕以安,則徐、水市間數被其害。至崔鎮高堰之決,黃、淮交漲而害漕,乃在淮、揚間,湖潰則敗漕。季馴以高堰障洪澤,俾堰東四湖勿受淮侵,漕始無敗。而河漕諸臣懼湖害,日夜常惴惴。

十三年從總漕都御史李世達議,開寶應月河。寶應汜光湖,諸湖中最湍險者也,廣百二十餘里。槐角樓當其中,形曲如箕,瓦店翼其南,秤鉤灣翼其北。西風鼓浪,往往覆舟。陳瑄築堤湖東,蓄水為運道。上有所受,下無所宣,遂決為八淺,匯為六潭,興、鹽諸場皆沒。而淮水又從周家橋漫入,溺人民,害漕運。武宗末年,郎中楊最請開月河,部覆不從。嘉靖中,工部郎中陳毓賢、戶部員外范韶、御史聞人詮、運糧千戶李顯皆以為言,議行未果。至是,工部郎中許應逵建議,世達用其言以奏,乃決行之。浚河千七百餘丈,置石閘三,減水閘二,築堤九千餘丈,石堤三之一,子堤五千餘丈。工成,賜名弘濟。尋改石閘為平水閘。應逵又築高郵護城堤。其後,弘濟南北閘,夏秋淮漲,吞吐不及,舟多覆者。神宗季年,督漕侍郎陳薦於南北各開月河以殺河怒,而溜始平。

十五年,督漕侍郎楊一魁請修高家堰以保上流,砌范家口以制旁決,疏草灣以殺河勢,修禮壩以保新城。詔如其議。一魁又改建古洪閘。先是,汶、泗之水由茶城會黃河。隆慶間,濁流倒灌,稽阻運船,郎中陳瑛移黃河口於茶城東八里,建古洪、內華二閘,漕河從古洪出口。後黃水發,淤益甚。一魁既改古洪,帝又從給事中常居敬言,令增築鎮口閘於古洪外,距河僅八十丈,吐納益易,糧運利之。

工部尚書石星議季馴、居敬條上善後事宜,請分地責成:接築塔山縷堤,清江浦草壩,創築寶應西堤,石砌邵伯湖堤,疏浚裏河淤淺,當在淮、揚興舉;察復南旺、馬踏、蜀山、馬場四湖,建築坎河滾水壩,加建通濟、永通二閘,察復安山湖地,當在山東興舉。帝從其議。未幾,眾工皆成。

十九年,季馴言:「宿遷以南,地形西AH,請開縷堤放水。沙隨水入,地隨沙高,庶水患消而費可省。」又請易高家堰土堤為石,築滿家閘西攔河壩,使汶、泗盡歸新河。設減水閘於李家口,以洩沛縣積水。從之。十月,淮湖大漲,江都淳家灣石堤、邵伯南壩、高郵中堤、朱家墩、清水潭皆決。郎中黃日謹築塞僅竣,而山陽堤亦決。

二十一年五月,恒雨。漕河汎溢,潰濟寧及淮河諸堤岸。總河尚書舒應龍議:築堽城壩,遏汶水之南,開馬踏湖月河口,導汶水之北。開通濟閘,放月河土壩以殺洶湧之勢。從其奏。數年之間,會通上下無阻,而黃、淮並漲,高堰及高郵堤數決害漕。應龍卒罷去。建議者紛紛,未有所定。

楊一魁代應龍為總河尚書,力主分黃導淮。治逾年,工將竣,又請決湖水以疏漕渠,言:「高、寶諸湖本沃壤也,自淮、黃逆壅,遂成昏墊。今入江入海之路即濬,宜開治涇河、子嬰溝、金灣河諸閘及瓜、儀二閘,大放湖水,就湖疏渠,與高、寶月河相接。既避運道風波之險,而水涸成田,給民耕種,漸議起科,可充河費。」命如議行。時下流既疏,淮水漸帖,而河方決黃堌口。督漕都御史褚鈇恐洩太多,徐、邳淤阻,力請塞之。一魁持不可,濬兩河口至小浮橋故道以通漕。然河大勢南徙,二洪漕屢涸,復大挑黃堌下之李吉口,挽黃以濟之,非久輒淤。

一魁入掌部事。二十六年,劉東星繼之,守一魁舊議,李吉口淤益高。歲冬月,即其地開一小河,春夏引水入徐州,如是者三年,大抵至秋即淤。乃復開趙家圈以接黃,開泇河以濟運。趙家圈旋淤,泇河未復,而東星卒。於是鳳陽巡撫都御史李三才建議自鎮口閘至磨兒莊仿閘河制,三十里一閘,凡建六閘於河中,節宣汶、濟之水,聊以通漕。漕舟至京,不復能如期矣。東星在事,開邵伯月河,長十八里,闊十八丈有奇,以避湖險。又開界首月河,長千八百餘丈。各建金門石閘二,漕舟利焉。

三十二年,總河侍郎李化龍始大開泇河,自直河至李家港二百六十餘里,盡避黃河之險。化龍憂去,總河侍郎曹時聘終其事,疏敘泇河之功,言:「舒應龍創開韓家莊以洩湖水,而路始通。劉東星大開良城、侯家莊以試行運,而路漸廣。李化龍上開李家港,鑿都水石,下開直河口,挑田家莊,殫力經營,行運過半,而路始開,故臣得接踵告竣。」因條上善後六事,運道由此大通。其後每年三月開泇河壩,由直河口進,九月開召公壩入黃河,糧艘及官民船悉以為準。

四十四年,巡漕御史朱堦請修復泉湖,言:「宋禮築壩戴村,奪二汶入海之路,灌以成河,復導洙、泗、洸、沂諸水以佐之。汶雖率眾流出全力以奉漕,然行遠而竭,已自難支。至南旺,又分其四以南迎淮,六以北赴衛,力分益薄。況此水夏秋則漲,冬春而涸,無雨即夏秋亦涸。禮逆慮其不可恃,乃於沿河昭陽、南旺、馬踏、蜀山、安山諸湖設立斗門,名曰水櫃。漕河水漲,則瀦其溢出者於湖,水消則決而注之漕。積泄有法,盜決有罪,故旱澇恃以無恐。及歲久禁馳,湖淺可耕,多為勢豪所占,昭陽一湖已作籓田。比來山東半年不雨,泉欲斷流,按圖而索水櫃,茫無知者。乞敕河臣清核,亟築堤壩斗門以廣蓄儲。」帝從其請。

方議浚泉湖,而河決徐州狼矢溝,由蛤鰻諸湖入泇河,出直口,運船迎溜艱險。督漕侍郎陳薦開武河等口,洩水平溜。後二年,決口長淤沙,河始復故道。總河侍郎王佐加築月壩以障之。至泰昌元年冬,佐言:「諸湖水櫃已復,安山湖且復五十五里,誠可利漕。請以水櫃之廢興為河官殿最。」從之。

天啟元年,淮、黃漲溢,決里河王公祠,淮安知府宋統殷、山陽知縣練國事力塞之。三年秋,外河復決數口,尋塞。是年冬,浚永濟新河。自凌雲翼開是河,未幾而閉。總河都御史劉士忠嘗開壩以濟運,已復塞。而淮安正河三十年未濬。故議先挑新河,通運船回空,乃浚正河,自許家閘至惠濟祠長千四百餘丈,復建通濟月河小閘,運船皆由正河,新河復閉。時王家集、磨兒莊湍溜日甚,漕儲參政朱國盛謀改浚一河以為漕計,令同知宋士中自泇口迤東抵宿遷陳溝口,復水斥駱馬湖,上至馬頰河,往回相度。乃議開馬家洲,且疏馬頰河口淤塞,上接水加流,下避劉口之險,又疏三汊河流沙十三里,開滔莊河百餘丈,浚深小河二十里,開王能莊二十里,以通駱馬湖口,築塞張家等溝數十道,束水歸漕。計河五十七里,名通濟新河。五年四月,工成,運道從新河,無劉口、磨兒莊諸險之患。明年,總河侍郎李從心開陳溝地十里,以竟前工。

崇禎二年,淮安蘇家嘴、新溝大壩並決,沒山、鹽、高、泰民田。五年,又決建義北壩。總河尚書朱光祚濬駱馬湖,避河險十三處,名順濟河。六年,良城至徐塘淤為平陸,漕運愆期,奪光祚官,劉榮嗣繼之。

八年,駱馬湖淤阻,榮嗣開河徐、宿,引注黃水,被劾,得重罪。侍郎周鼎繼之,乃專力於泇河,浚麥河支河,築王母山前後壩、勝陽山東堤、馬蹄厓十字河攔水壩,挑良城閘抵徐塘口六千餘丈。九年夏,泇河復通,由宿遷陳溝口合大河。鼎又修高家堰及新溝漾田營堤,增築天妃閘石工,去南旺湖彭口沙礓,浚劉呂莊至黃林莊百六十里。而是時黃、淮漲溢日甚,倒灌害漕。鼎在事五年,卒以運阻削職。繼之者侍郎張國維,甫蒞任,即以漕涸被責。

十四年,國維言:「濟寧運道自棗林閘溯師家莊、仲家淺二閘,歲患淤淺,每引泗河由魯橋入運以濟之。伏秋水長,足資利涉。而挾沙注河,水退沙積,利害參半。旁自白馬河匯鄒縣諸泉,與泗合流而出魯橋,力弱不能敵泗,河身半淤,不為漕用。然其上源寬處正與仲家淺閘相對,導令由此入運,較魯橋高下懸殊,且易細流為洪流,又減沙滲之患,而濟仲家淺及師莊、棗林,有三便。」又言:「南旺水本地脊,惟藉泰安、新泰、萊蕪、寧陽、汶上、東平、平陰、肥城八州縣泉源,由汶入運,故運河得通。今東平、平陰、肥城淤沙中斷,請亟浚之。」復上疏運六策:一復安山湖水櫃以濟北閘,一改挑滄浪河從萬年倉出口以利四閘,一展浚汶河、陶河上源以濟邳派,一改道沂河出徐塘口以並利邳、宿,其二即開三州縣淤沙及改挑白馬湖也。皆命酌行。國維又浚淮、揚漕河三百餘里。當是時,河臣竭力補苴,南河稍寧,北河數淺阻。而河南守臣壅黃河以灌賊。河大決開封,下流日淤,河事益壞,未幾而明亡矣。


○運河下海運

江南運河,自杭州北郭務至謝村北,為十二里洋,為塘棲,德清之水入之。踰北陸橋入崇德界,過松老抵高新橋,海鹽支河通之。繞崇德城南,轉東北,至小高陽橋東,過石門塘,折而東,為王灣。至皁林,水深者及丈。過永新,入秀水界,踰陡門鎮,北為分鄉鋪,稍東為繡塔。北由嘉興城西轉而北,出杉青三閘,至王江涇鎮,松江運艘自東來會之。北為平望驛,東通鶯脰湖,湖州運艘自西出新興橋會之。北至松陵驛,由吳江至三里橋,北有震澤,南有黃天蕩,水勢澎湃,夾浦橋屢建。北經蘇州城東占魚口,水由BC塘入之。北至楓橋,由射瀆經滸墅關,過白鶴鋪,長洲、無錫兩邑之界也。錫山驛水僅浮瓦礫。過黃埠,至洛社橋,江陰九里河之水通之。西北為常州,漕河舊貫城,入東水門,由西水門出。嘉靖末防倭,改從南城壕。江陰,順塘河水由城東通丁堰,沙子湖在其西南,宜興鐘溪之水入之。又西,直瀆水入之,又西為奔牛、呂城二閘,常、鎮界其中,皆有月河以佐節宣,後並廢。其南為金壇河,溧陽、高淳之水出焉。丹陽南二十里為陵口,北二十五里為黃泥壩,舊皆置閘。練湖水高漕河數丈,一由三思橋,一由仁智橋,皆入運。北過丹徒鎮有豬婆灘,多軟沙。丹徒以上運道,視江潮為盈涸。過鎮江,出京口閘,閘外沙堵延袤二十丈,可藏舟避風,由此浮於江,與瓜步對。自北郭至京口首尾八百餘里,皆平流。歷嘉而蘇,眾水所聚,至常州以西,地漸高仰,水淺易洩,盈涸不恒,時浚時壅,往往兼取孟瀆、德勝兩河,東浮大江,以達揚泰。

洪武二十六年嘗命崇山侯李新開溧水胭脂河,以通浙漕,免丹陽輸挽及大江風濤之險。而三吳之粟,必由常、鎮。三十一年濬奔牛、呂城二壩河道。

永樂間,修練湖堤。即命通政張璉發民丁十萬,浚常州孟瀆河,又浚蘭陵溝,北至孟瀆河閘,六千餘丈,南至奔牛鎮,千二百餘丈。已,復濬鎮江京口、新港及甘露三港,以達於江。漕舟自奔牛溯京口,水涸則改從孟瀆右趨瓜洲,抵白塔,以為常。

宣德六年從武進民請,疏德勝新河四十里。八年,工竣。漕舟自德勝北入江,直泰興之北新河。由泰州壩抵揚子灣入漕河,視白塔尤便。於是漕河及孟瀆、德勝三河並通,皆可濟運矣。

正統元年,廷臣上言:「自新港至奔牛,漕河百五十里,舊有水車捲江潮灌注,通舟溉田。請支官錢置車。」詔可。然三河之入江口,皆自卑而高,其水亦更迭盈縮。八年,武進民請浚德勝及北新河。浙江都司蕭華則請浚孟瀆。巡撫周忱定議濬兩河,而罷北新築壩。白塔河之大橋閘以時啟閉,而常、鎮漕河亦疏浚焉。

景泰間,漕河復淤,遂引漕舟盡由孟瀆。三年,御史練綱言:「漕舟從夏港及孟瀆出江,逆行三百里,始達瓜洲。德勝直北新,而白塔又與孟瀆斜直,由此兩岸橫渡甚近,宜大疏淤塞。」帝命尚書石璞措置。會有請鑿鎮江七里港,引金山上流通丹陽,以避孟瀆險者。鎮江知府林鶚以為迂道多石,壞民田墓多,宜浚京口閘、甘露壩,道里近,功力省。乃從鶚議。浙江參政胡清又欲去新港、奔牛等壩,置石閘以蓄泉。亦從其請。而濬德勝河與鑿港之議俱寢。然石閘雖建,蓄水不能多,漕舟仍入孟瀆。

天順元年,尚寶少卿凌信言,糧艘從鎮江裏河為便。帝以為然,命糧儲河道都御史李秉通七里港口,引江水注之,且浚奔牛、新港之淤。巡撫崔恭又請增置五閘。至成化四年,閘工始成。於是漕舟盡由裏河,其入二河者,回空之艘及他舟而已。定制,孟瀆河口與瓜、儀諸港俱三年一濬。孟瀆寬廣不甚淤,裏河不久輒涸,則又改從孟瀆。

弘治十七年,部臣復陳夏港、孟瀆遠浮大江之害,請亟濬京口淤,而引練湖灌之。詔速行。正德二年復開白塔河及江口、大橋、潘家、通江四閘。十四年從督漕都御史臧鳳言,浚常州上下裏河,漕舟無阻者五十餘載。

萬曆元年又漸涸,復一浚之。歲貢生許汝愚上言:「國初置四閘:曰京口,曰丹徒,防三江之涸;曰呂城,曰奔牛,防五湖之洩。自丹陽至鎮江蓄為湖者三:曰練湖,曰焦子,曰杜墅。歲久,居民侵種,焦、杜二湖俱涸,僅存練湖,猶有侵者。而四閘俱空設矣。請濬三湖故址通漕。」總河傅希摯言:「練湖已浚,而焦子、杜墅源少無益。」其議遂寢。未幾,練湖復淤淺。

五年,御史郭思極、陳世寶先後請復練湖,濬孟瀆。而給事中湯聘尹則請於京口旁別建一閘,引江流內注,潮長則開,縮則閉。御史尹良任又言:「孟瀆渡江入黃家港,水面雖闊,江流甚平,由此抵泰興以達灣頭、高郵僅二百餘里,可免瓜、儀不測之患。,至如京口北渡金山而下,中流遇風有漂溺患,宜挑甘露港夾岸洲田十餘里,以便回泊。」御史林應訓又言:「自萬緣橋抵孟瀆,兩厓陡峻,雨潦易圮,且江潮湧沙,淤塞難免。宜於萬緣橋、黃連樹各建閘以資蓄洩。」又言:「練湖自西晉陳敏遏馬林溪,引長山八十四溪之水以溉雲陽,堤名練塘,又曰練河,凡四十里許。環湖立涵洞十三。宋紹興時,中置橫埂,分上下湖,立上、中、下三閘。八十四溪之水始經辰溪衝入上湖,復由三閘轉入下湖。洪武間,因運道澀,依下湖東堤建三閘,借湖水以濟運,後乃漸堙。今當盡革侵占,復浚為湖。上湖四際夾阜,下湖東北臨河,原埂完固,惟應補中間缺口,且增築西南,與東北相應。至三閘,惟臨湖上閘如故,宜增建中、下二閘,更設減水閘二座,界中、下二閘間。共革田五千畝有奇,塞沿堤私設涵洞,止存其舊十三處,以宣洩湖水。冬春即閉塞,毋得私啟。蓋練湖無源,惟藉瀦蓄,增堤啟閘,水常有餘,然後可以濟運。臣親驗上湖地仰,八十四溪之水所由來,懼其易洩;下湖地平衍,僅高漕河數尺,又常懼不盈。誠使水裕堤堅,則應時注之,河有全力矣。」皆下所司酌議。

十三年,鎮江知府吳捴謙復言:「練湖中堤宜飭有司春初即修,以防衝決,且禁勢豪侵占。」從之。十七年浚武進橫林漕河。

崇禎元年,浚京口漕河。五年,太常少卿姜志禮建《漕河議》,言:「神廟

初,先臣寶著《漕河議》,當事采行,不開河而濟運者二十餘年。後復佃湖妨運,歲累畚鍤。故老有言,「京口閘底與虎丘塔頂平」,是可知挑河無益,蓄湖為要也。今當革佃修閘,而高築上下湖圍埂,蓄水使深。且漕河閘座非僅京口、呂城、新閘、奔牛數處而已,陵口、尹公橋、黃泥壩、新豐、大犢山節節有閘,皆廢去,並宜修建。而運道支流如武進洞子河、連江橋河、扁擔河,丹陽簡橋河、陳家橋河、七里橋河、丁議河、越瀆河,勝村溪之大壩頭,丹陽甘露港南之小閘口,皆應急修整。至奔牛、呂城之北,各設減水閘。歲十月實以土,商民船盡令盤壩。此皆舊章所當率由。近有欲開九曲河,使運船竟從泡港閘出江,直達揚子橋,以免瓜洲啟閘稽遲者,試而後行可也。回空糧艘及官舫,宜由江行,而於河莊設閘啟閉。數役並行,漕事乃大善矣。」議不果行。

江漕者,湖廣漕舟由漢、沔下潯陽,江西漕舟出章江、鄱陽而會於湖口,暨南直隸寧、太、池、安、江寧、廣德之舟,同浮大江,入儀真通江閘,以溯淮、揚入閘河。瓜、儀之間,運道之咽喉也。洪武中,餉遼卒者,從儀真上淮安,由鹽城汎海;餉梁、晉者,亦從儀真赴淮安,盤壩入淮。江口則設壩置閘,凡十有三。浚揚子橋河至黃泥灣九千餘丈。永樂間,濬儀真清江壩、下水港及夾港河,修沿江堤岸。洪熙元年濬儀真壩河,後定制儀真壩下黃泥灘、直河口二港及瓜洲二港、常州之孟瀆河皆三年一濬。宣德間,從侍郎趙新、御史陳祚請,濬黃泥灘、清江閘。成化中,建閘於儀真通江河港者三,江都之留潮通江者二。已而通江港塞。弘治初,復開之,既又於總港口建閘蓄水。儀真、江都二縣間,有官塘五區,築閘蓄水,以溉民田,豪民占以為業,真、揚之間運道阻梗。嘉靖二年,御史秦鉞請復五塘。從之。萬曆五年,御史陳世寶言:「儀真江口,去閘太遠,請於上下十數丈許增建二閘,隨湖啟閉,以截出江之船,盡令入閘,庶免遲滯。」疏上,議行。

白塔河者,在泰州。上通邵伯,下接大江,斜對常州孟瀆河與泰興北新河,皆浙漕間道也。自陳瑄始開。宣德間,從趙新、陳祚請,命瑄役夫四萬五千餘人濬之,建新閘、潘家莊、大橋、江口四閘。正統四年,水潰閘塞,都督武興因閉不用,仍自瓜洲盤壩。瓜洲之壩,洪武中置,凡十五,列東西二港間。永樂間,廢東壩為廠,以貯材木,止存西港七壩。漕舟失泊,屢遭風險。英宗初年,乃復濬東港。既而巡撫周忱築壩白塔河之大橋閘,以時啟閉,漕舟稍分行。自鎮江裏河開濬,漕舟出甘露、新港,徑渡瓜洲;而白塔、北新,皆以江路險遠,捨而不由矣。

衛漕者,即衛河。源出河南輝縣,至臨清與會通河合,北達天津。自臨清以北皆稱衛河。詳具本《志》。

白漕者,即通濟河。源出塞地,經密雲縣霧靈山,為潮河川。而富河、罾口河、七渡河、桑乾河、三里河俱會於此,名曰白河。南流經通州,合通惠及榆、渾諸河,亦名潞河。三百六十里,至直沽會衛河入海,賴以通漕。楊村以北,勢若建瓴,底多淤沙。夏秋水漲苦潦,冬春水微苦澀。衝潰徙改頗與黃河同。耎兒渡者,在武清、通州間,尤其要害處也。自永樂至成化初年,凡八決,輒發民夫築堤。而正統元年之決,為害尤甚,特敕太監沐敬、安遠侯柳溥、尚書李友直隨宜區畫,發五軍營卒五萬及民夫一萬築決堤。又命武進伯朱冕、尚書吳中役五萬人,去河西務二十里鑿河一道,導白水入其中。二工並竣,人甚便之,賜河名曰通濟,封河神曰通濟河神。先是,永樂二十一年築通州抵直沽河岸,有衝決者,隨時修築以為常。迨通濟河成,決岸修築者亦且數四。萬曆三十一年從工部議,挑通州至天津白河,深四尺五寸,所挑沙土即築堤兩岸,著為令。

大通河者,元郭守敬所鑿。由大通橋東下,抵通州高麗莊,與白河合,至直沽,會衛河入海,長百六十里有奇。十里一閘,蓄水濟運,名曰通惠。又以白河、榆河、渾河合流,亦名潞河。洪武中漸廢。

永樂四年八月,北京行部言:「宛平昌平西湖、景東牛欄莊及青龍華家𠂅山三閘,水衝決岸。」命發軍民修治。明年復言:「自西湖、景東至通流,凡七閘,河道淤塞。自昌平東南白浮村至西湖、景東流水河口一百里,宜增置十二閘。」從之。未幾,閘俱堙,不復通舟。

成化中,漕運總兵官楊茂言:「每歲自張家灣舍舟,車轉至都下,雇值不貲。舊通惠河石閘尚存,深二尺許,修閘瀦水,用小舟剝運便。」又有議於三里河從張家灣煙墩橋以西疏河泊舟者。下廷臣集議,遣尚書楊鼎、侍郎喬毅相度。上言:「舊閘二十四座,通水行舟。但元時水在宮牆外,舟得入城內海子灣。今水從皇城金水河出,故道不可復行。且元引白浮泉往西逆流,今經山陵,恐妨地脈。又一畝泉過白羊口山溝,兩水衝截難引。若城南三里河舊無河源,正統間修城壕,恐雨多水溢,乃穿正陽橋東南AH下地,開壕口以洩之,始有三里河名。自壕口八里,始接渾河。舊渠兩岸多廬墓,水淺河窄,又須增引別流相濟。如西湖草橋源出玉匠局、馬跑等地,泉不深遠。元人曾用金口水,洶湧沒民舍,以故隨廢。惟玉泉、龍泉及月兒、柳沙等泉,皆出西北,循山麓而行,可導入西湖。請浚西湖之源,閉分水清龍閘,引諸泉水從高梁河,分其半由金水河出,餘則從都城外壕流轉,會於正陽門東。城壕且閉,令勿入三里河併流。大通橋閘河隨旱澇啟閉,則舟獲近倉,甚便。」帝從其議。方發軍夫九萬修浚,會以災異,詔罷諸役。所司以漕事大,乃命四萬人浚城壕,而西山、玉泉及抵張家灣河道,則以漸及焉。越五年,乃敕平江伯陳銳,副都御史李裕,侍郎翁世資、王詔督漕卒濬通惠河,如鼎、毅前議。明年六月,工成,自大通橋至張家灣渾河口六十餘里,濬泉三,增閘四,漕舟稍通。然元時所引昌平三泉俱遏不行,獨引一西湖,又僅分其半,河窄易盈涸。不二載,澀滯如舊。正德二年嘗一浚之,且修大通橋至通州閘十有二,壩四十有一。

嘉靖六年,御史吳仲言:「通惠河屢經修復,皆為權勢所撓。顧通流等八閘遺跡俱存,因而成之,為力甚易,歲可省車費貲二十餘萬。且歷代漕運皆達京師,未有貯國儲於五十里外者。」帝心以為然,命侍郎王軏、何詔及仲偕相度。軏等言:「大通橋地形高白河六丈餘,若濬至七丈,引白河達京城,諸閘可盡罷,然未易議也。計獨濬治河閘,但通流閘在通州舊城中,經二水門,南浦、土橋、廣利三閘皆闤闠衢市,不便轉挽。惟白河濱舊小河廢壩西,不一里至堰水小壩,宜修築之,使通普濟閘,可省四閘兩關轉搬力。」而尚書桂萼言不便,請改修三里河。帝下其疏於大學士楊一清、張璁。一清言:「因舊閘行轉搬法,省運軍勞費,宜斷行之。」璁亦言:「此一勞永逸之計,萼所論費廣功難。」帝乃卻萼議。

明年六月,仲報河成,因疏五事,言:「大通橋至通州石壩,地勢高四丈,流沙易淤,宜時加濬治。管河主事宜專委任,毋令兼他務。官吏、閘夫以罷運裁減,宜復舊額。慶豐上閘、平津中閘今已不用,宜改建通州西水關外。剝船造費及遞歲修艌,俱宜酌處。」帝以先朝屢勘行未即功,仲等四閱月工成,詔予賞,悉從其所請。仲又請留督工郎中何棟專理其事,為經久計。從之。九年擢棟右通政,仍管通惠河道。是時,仲出為處州知府,進所編《通惠河志》。帝命送史館,採入《會典》,且頒工部刊行。自此漕艘直達京師,迄於明末。人思仲德,建祠通州祀之。

薊州河者,運薊州官軍餉道也。明初,海運餉薊州。天順二年,大河衛百戶閔恭言:「南京並直隸各衛,歲用旗軍運糧三萬石至薊州等衛倉,越大海七十餘里,風濤險惡。新開沽河,北望薊州,正與水套、沽河直,袤四十餘里而徑,且水深,其間阻隔者僅四之一,若穿渠以運,可無海患。」下總兵都督宋勝、巡按御史李敏行視可否。勝等言便,遂開直沽河。闊五丈,深丈五尺。成化二年一浚,二十年再浚,并浚鴉鴻橋河道,造豐潤縣海運糧儲倉。正德十六年,運糧指揮王瓚言:「直沽東北新河,轉運薊州,河流淺,潮至方可行舟。邊關每匱餉,宜浚使深廣。」從之。初,新河三歲一濬。嘉靖元年易二歲,以為常。十七年濬殷留莊大口至舊倉店百十六里。

豐潤環香河者,濬自成化間,運粟十餘萬石以餉薊州東路者也。後堙廢,餉改薊州給,大不便。嘉靖四十五年從御史鮑承廕請,復之,且建三閘於北濟、張官屯、鴉鴻橋以瀦水。

昌平河,運諸陵官軍餉道也。起鞏華城外安濟橋,抵通州渡口。袤百四十五里,其中淤淺三十里難行。隆慶六年大浚,運給長陵等八衛官軍月糧四萬石,遂成流通。萬曆元年復疏鞏華城外舊河。

海運,始於元至元中。伯顏用朱清、張瑄運糧輸京師,僅四萬餘石。其後日增,至三百萬餘石。初,海道萬三千餘里,最險惡,既而開生道,稍徑直。後殷明略又開新道,尤便。然皆出大洋,風利,自浙西抵京不過旬日,而漂失甚多。

洪武元年,太祖命湯和造海舟,餉北征士卒。天下既定,募水工運萊州洋海倉粟以給永平。後遼左及迤北數用兵,於是靖海侯吳禎、延安侯唐勝宗、航海侯張赫、舳艫侯朱壽先後轉遼餉,以為常。督江、浙邊海衛軍大舟百餘艘,運糧數十萬。賜將校以下綺帛、胡椒、蘇木、錢鈔有差,民夫則復其家一年,溺死者厚恤。三十年,以遼東軍餉贏羨,第令遼軍屯種其地,而罷海運。

永樂元年,平江伯陳瑄督海運糧四十九萬餘石,餉北京、遼東。二年,以海運但抵直沽,別用小船轉運至京,命於天津置露囤千四百所,以廣儲蓄。四年定海陸兼運。瑄每歲運糧百萬,建百萬倉於直沽尹兒灣城。天津衛籍兵萬人戍守。至是,命江南糧一由海運,一由淮、黃,陸運赴衛河,入通州,以為常。陳瑄上言:「嘉定瀕海,當江流之衝,地平衍,無大山高嶼。海舟停泊,或值風濤,觸堅膠淺輒敗。宜於青浦築土為山,立堠表識,使舟人知所避,而海險不為患。」詔從之。十年九月,工成。方百丈,高三十餘丈。賜名寶山。御製碑文紀之。

十三年五月復罷海運,惟存遮洋一總,運遼、薊糧。正統十三年減登州衛海船百艘為十八艘,以五艘運青、萊、登布花鈔錠十二萬餘斤,歲賞遼軍。

成化二十三年,侍郎丘濬進大學衍義補,請尋海運故道與河漕並行,大略言:「海舟一載千石,可當河舟三,用卒大減。河漕視陸運費省什三,海運視陸省什七,雖有漂溺患,然省牽卒之勞、駁淺之費、挨次之守,利害亦相當。宜訪素知海道者,講求勘視。」其說未行。弘治五年,河決金龍口,有請復海運者,朝議弗是。

嘉靖二年,遮洋總漂糧二萬石,溺死官軍五十餘人。五年停登州造船。二十年,總河王以旂以河道梗澀,言:「海運雖難行,然中間平度州東南有南北新河一道,元時建閘直達安東,南北悉由內洋而行,路捷無險,所當講求。」帝以海道迂遠,卻其議。三十八年,遼東巡撫侯汝諒言:「天津入遼之路,自海口至右屯河通堡不及二百里,其中曹泊店、月坨桑、姜女墳、桃花島皆可灣泊。」部覆行之。四十五年,順天巡撫耿隨朝勘海道,自永平西下海,百四十五里至紀各莊,又四百二十六里至天津,皆傍岸行舟。其間開洋百二十里,有建河、糧河、小沽、大沽河可避風。初允其議,尋以御史劉翾疏沮而罷。是年,從給事中胡應嘉言,革遮洋總。

隆慶五年,徐、邳河淤,從給事中宋良佐言,復設遮洋總,存海運遺意。山東巡撫梁夢龍極論海運之利,言:「海道南自淮安至膠州,北自天津至海倉,島人商賈所出入。臣遣卒自淮、膠各運米麥至天津,無不利者。淮安至天津三千三百里,風便,兩旬可達。舟由近洋,島嶼聯絡,雖風可依,視殷明略故道甚安便。五月前風順而柔,此時出海可保無虞。」命量撥近地漕糧十二萬石,俾夢龍行之。

六年,王宗沐督漕,請行海運。詔令運十二萬石自淮入海。其道,由雲梯關東北歷鷹游山、安東衛、石臼所、夏河所、齊堂島、靈山衛、古鎮、膠州、鰲山衛、大嵩衛、行村寨,皆海面。自海洋所歷竹島、寧津所、靖海衛,東北轉成山衛、劉公島、威海衛,西歷寧海衛,皆海面。自福山之罘島至登州城北新海口沙門等島,西歷桑島、母已島,自母已西歷三山島、芙蓉島、萊州大洋、海倉口;自海倉西歷淮河海口、魚兒鋪,西北歷侯鎮店、唐頭塞;自侯鎮西北大清河、小清河海口,乞溝河入直沽,抵天津衛。凡三千三百九十里。

萬曆元年,即墨福山島壞糧運七艘,漂米數千石,溺軍丁十五人。給事、御史交章論其失,罷不復行。二十五年,倭寇作,自登州運糧給朝鮮軍。山東副使于仁廉復言:「餉遼莫如海運,海運莫如登、萊。蓋登、萊度金州六七百里,至旅順口僅五百餘里,順風揚帆一二日可至。又有沙門、鼉磯、皇城等島居其中,天設水遞,止宿避風。惟皇城至旅順二百里差遠,得便風不半日可度也。若天津至遼,則大洋無泊;淮安至膠州,雖僅三百里,而由膠至登千里而遙,礁礙難行。惟登、萊濟遼,勢便而事易。」時頗以其議為然,而未行也。四十六年,山東巡撫李長庚奏行海運,特設戶部侍郎一人督之,事具《長庚傳》。

崇禎十二年,崇明人沈廷揚為內閣中書,復陳海運之便,且輯《海運書》五卷進呈。命造海舟試之。廷揚乘二舟,載米數百石,十三年六月朔,由淮安出海,望日抵天津。守風者五日,行僅一旬。帝大喜,加廷揚戶部郎中,命往登州與巡撫徐人龍計度。山東副總兵黃廕恩亦上海運九議,帝即令督海運。先是,寧遠軍餉率用天津船赴登州,候東南風轉粟至天津,又候西南風轉至寧遠。廷揚自登州直輸寧遠,省費多。尋命赴淮安經理海運,為督漕侍郎硃大典所沮,乃命易駐登州,領寧遠餉務。十六年加光祿少卿。福王時,命廷揚以海舟防江,尋命兼理糧務。南都既失,廷揚崎嶇唐、魯二王間以死。

當嘉靖中,廷臣紛紛議復海運,漕運總兵官萬表言:「在昔海運,歲溺不止十萬。載米之舟,駕船之卒,統卒之官,皆所不免。今人策海運輒主丘浚之論,非達於事者也。」

○淮河泇河衛河漳河沁河滹沱河桑乾河膠萊河

淮河,出河南平氏胎簪山。經桐伯,其流始大。東至固始,入南畿潁州境,東合汝、潁諸水。經壽州北,肥水入焉。至懷遠城東,渦水入焉。東經鳳陽、臨淮,濠水入焉。又經五河縣南,而納澮、沱、漴、潼諸水,勢盛流疾。經泗州城南,稍東則汴水入焉。過龜山麓,益折而北,會洪澤、阜陵、泥墩、萬家諸湖。東北至清河,南會於大河,即古泗口也,亦曰清口,是謂黃、淮交會之衝。淮之南岸,漕河流入焉,所謂清江浦口。又東經淮安北、安東南而達於海。

永樂七年,決壽州,泛中都。正統三年,溢清河。天順四年,溢鳳陽。皆隨時修築,無鉅害也。正德十二年,復決漕堤,灌泗州。泗州,祖陵在焉,其地最下。初,淮自安東雲梯關入海,無旁溢患。迨與黃會,黃水勢盛,奪淮入海之路,淮不能與黃敵,往往避而東。陳瑄鑿清江浦,因築高家堰舊堤以障之。淮、揚恃以無恐,而鳳泗間數為害。嘉靖十四年用總河都御史劉天和言,築堤衛陵,而高堰方固,淮暢流出清口,鳳、泗之患弭。隆慶四年,總河都御史翁大立復奏浚淮工竣,淮益無事。

至萬曆三年三月,高家堰決,高、寶、興、鹽為巨浸。而黃水躡淮,且漸逼鳳、泗。乃命建泗陵護城石堤二百餘丈,泗得石堤稍寧。於是,總漕侍郎吳桂芳言:「河決崔鎮,清河路淤。黃強淮弱,南徙而灌山陽、高、寶,請急護湖堤。」帝令熟計其便。給事中湯聘尹議請導淮入江。會河從老黃河奔入海,淮得乘虛出清口。桂芳以聞,議遂寢。

六年,總河都御史潘季馴言:「高堰,淮、揚之門戶,而黃、淮之關鍵也。欲導河以入海,必藉淮以刷沙。淮水南決,則濁流停滯,清口亦堙。河必決溢,上流水行平地,而邳、徐、鳳、泗皆為巨浸。是淮病而黃病,黃病而漕亦病,相因之勢也。」於是築高堰堤,起武家墩,經大小澗、阜陵湖、周橋、翟壩,長八十里,使淮不得東。又以淮水北岸有王簡、張福二口洩入黃河,水力分,清口易淤淺,且黃水多由此倒灌入淮,乃築堤捍之。使淮無所出,黃無所入,全淮畢趨清口,會大河入海。然淮水雖出清口,亦西淫鳳、泗。

八年,雨澇,淮薄泗城,且至祖陵墀中。御史陳用賓以聞。給事中王道成因言:「黃河未漲,淮、泗間霖雨偶集,而清口已不容洩。宜令河臣疏導堵塞之。」季馴言:「黃、淮合流東注,甚迅駛。泗州岡阜盤旋,雨潦不及宣洩,因此漲溢。欲疏鑿,則下流已深,無可疏;欲堵塞,則上流不可逆堵。」乃令季馴相度,卒聽之而已。十六年,季馴復為總河,加泗州護堤數千丈,皆用石。

十九年九月,淮水溢泗州,高於城壕,因塞水關以防內灌。於是,城中積水不洩,居民十九淹沒,侵及祖陵。疏洩之議不一,季馴謂當聽其自消。會嘔血乞歸,言者因請允其去。而帝遣給事中張貞觀往勘,會總河尚書舒應龍等詳議以上,計未有所定。連數歲,淮東決高良澗,西灌泗陵。帝怒,奪應龍官,遣給事中張企程往勘。議者多請拆高堰,總河尚書楊一魁與企程不從,而力請分黃導淮。乃建武家墩經河閘,洩淮水由永濟河達涇河,下射陽湖入海。又建高良澗及周橋減水石閘,以洩淮水,一由岔河入涇河,一由草子湖、寶應湖下子嬰溝,俱下廣洋湖入海。又挑高郵茆塘港,通邵伯湖,開金家灣,下芒稻河入江,以疏淮漲,而淮水以平。其後三閘漸塞。

崇禎間,黃、淮漲溢,議者復請開高堰。淮、揚在朝者公疏力爭,議遂寢。然是時建義諸口數決,下灌興、鹽,淮患日棘矣。

泇河,二源。一出費縣南山谷中,循沂州西南流,一出嶧縣君山,東南與費泇合,謂之東、西二泇河。南會彭河水,從馬家橋東,過微山、赤山、呂孟等湖,踰葛墟嶺,而南經侯家灣、良城,至泇口鎮,合蛤鰻、連汪諸湖。東會沂水,從周湖、柳湖,接邳州東直河。東南達宿遷之黃墩湖、駱馬湖,從董、陳二溝入黃河。引泗合沂濟運道,以避黃河之險,其議始於翁大立,繼之者傅希摯,而成於李化龍、曹時聘。

隆慶四年九月,河決邳州,自睢寧至宿遷淤百八十里。總河侍郎翁大立請開泇河以避黃水,未決而罷。明年四月,河復決邳州,命給事中雒遵勘驗。工部尚書朱衡請以開泇口河之說下諸臣熟計。帝即命遵會勘。遵言:「泇口河取道雖捷,施工實難。葛墟嶺高出河底六丈餘,開鑿僅至二丈,硼石中水泉湧出。侯家灣、良城雖有河形,水中多伏石,難鑿,縱鑿之,湍激不可漕。且蛤鰻、周柳諸湖,築堤水中,功費無算。微山、赤山、呂孟等湖雖可築堤,然須鑿葛墟嶺以洩正派,開地濱溝以散餘波,乃可施工。」請罷其議。詔尚書朱衡會總河都御史萬恭等覆勘。衡奏有三難,大略如遵指。且言漕河已通,徐、邳間堤高水深,不煩別建置。乃罷。

萬曆三年,總河都御史傅希摯言:「泇河之議嘗建而中止,謂有三難。而臣遣錐手、步弓、水平、畫匠,於三難處核勘。起自上泉河口,開向東南,則起處低窪,下流趨高之難可避也。南經性義村東,則葛墟嶺高堅之難可避也。從陡溝河經郭村西之平坦,則良城侯家灣之伏石可避也。至泇口上下,河渠深淺不一,湖塘聯絡相因,間有砂礓,無礙挑挖。大較上起泉河口,水所從入也,下至大河口,水所從出也。自西北至東南,長五百三十里,比之黃河近八十里,河渠、河塘十居八九,源頭活水,脈絡貫通,此天子所以資漕也。誠能捐十年治河之費,以成泇河,則黃河無慮壅決,茶城無慮填淤,二洪無慮艱險,運艘無慮漂損,洋山之支河可無開,境山之閘座可無建,徐、呂之洪夫可盡省,馬家橋之堤工可中輟。今日不貲之費,他日所省抵有餘者也。臣以為開泇河便。」乃命都給事中侯于趙往會希摯及巡漕御史劉光國,確議以聞。于趙勘上泇河事宜:「自泉河口至大河口五百三十里內,自直河至清河三百餘里,無賴於泇,事在可已。惟徐、呂至直河上下二百餘里,河衝蕭、碭則涸二洪,衝睢寧則淤邳河,宜開以避其害,約費百五十餘萬金。特良城伏石長五百五十丈,開鑿之力難以逆料。性義嶺及南禹陵俱限隔河流,二處既開,則豐、沛河決,必至灌入。宜先鑿良城石,預修豐、沛堤防,可徐議興功也。」部覆如其言,而謂開泇非數年不成,當以治河為急。帝不閱,責于趙阻擾,然議亦遂寢。

二十年,總河尚書舒應龍開韓莊以洩湖水,泇河之路始通。至二十五年,黃河決黃堌口南徙,徐、呂而下幾斷流。方議開李吉口、小浮橋及鎮口以下,建閘引水以通漕,而論者謂非永久之計。於是工科給事中楊應文、吏科給事中楊廷蘭皆謂當開泇河,工部覆議允行。帝命河漕官勘報,不果。二十八年,御史佴祺復請開泇河。工部覆奏云:「用黃河為漕,利與害參用;泇河為漕,有利無害。但泇河之外,由微山、呂孟、周柳諸湖,伏秋水發虞風波,冬春水涸虞淺阻,須上下別鑿漕渠,建閘節水。」從之。總河尚書劉東星董其事,以地多沙石,工艱未就。工科給事中張問達以為言。御史張養志復陳開泇河之說有四:

一曰開黃泥灣以通入泇之徑。邳州沂河口,入泇河門戶也。進口六七里,有湖名連二汪,其水淺而闊,下多淤泥。欲挑濬則無岸可修,欲為壩埽則無基可築。湖外有黃泥灣,離湖不遠,地頗低。自沂口至湖北崖約二十餘里,於此開一河以接泇口,引湖水灌之,運舟可直達泇口矣。

一曰鑿萬家莊以接泇口之源。萬家莊,泇口迤北地也。與臺家莊、侯家灣、良城諸處,皆山岡高阜,多砂礓石塊,極難為工。東星力鑿成河。但河身尚淺,水止二三尺,宜更鑿四五尺,俾韓莊之水下接泇口,則運舟無論大小,皆沛然可達矣。

一曰濬支河以避微口之險。微山湖在韓莊西,上下三十餘里,水深丈餘。必探深淺,立標為嚮導,風正帆懸,頃刻可過,突遇狂飆,未免敗沒。今已傍湖開支河四十五里,上通西柳莊,下接韓莊,牽挽有路。當再疏濬,庶無漂溺之患。

其一則以萬莊一帶勢高,北水南下,至此必速。請即其地建閘數座,以時蓄洩。詔速勘行。而東星病卒。御史高舉獻河漕三策,復及泇河。工部尚書楊一魁覆言:「泇河經良城、彭河、葛墟嶺,石礓難鑿,故口僅丈六尺,淺亦如之,當大加疏鑿。其韓莊渠上接微山、呂孟,宜多方疏導,俾無淤淺。順流入馬家橋、夏鎮,以為運道接濟之資。」帝以泇河既有成績,命河臣更挑浚。

三十年,工部尚書姚繼可言泇河之役宜罷,乃止不治。未幾,總河侍郎李化龍復議開泇河,屬之直河,以避河險。工科給事中侯慶遠力主其說,而以估費太少,責期太遠,請專任而責成之。三十二年正月,工部覆化龍疏,言:「開泇有六善,其不疑有二。泇河開而運不借河,河水有無聽之,善一。以二百主十里之泇河,避三百三十里之黃河,善二。運不借河,則我為政得以熟察機宜而治之,善三。估費二十萬金,開河二百六十里,視朱衡新河事半功倍,善四。開河必行召募,春荒役興,麥熟人散,富民不擾,窮民得以養,善五。糧船過洪,必約春盡,實畏河漲,運入泇河,朝暮無妨,善六。為陵捍患,為民禦災,無疑者一。徐州向苦洪水,泇河既開,則徐民之為魚者亦少,無疑者二。」帝深善之,令速鳩工為久遠之計。八月,化龍報分水河成,糧艘由泇者三之二。會化龍丁艱去,總河侍郎曹時聘代,上言頌化龍功。然是時,導河、浚泇,兩工並興,役未能竟。而黃河數溢,壞漕渠。給事中宋一韓遂詆化龍開泇之誤,化龍憤,上章自辯。時聘亦力言泇可賴,因畫善後六事以聞。部覆皆從其議。且言:「泇開於梗漕之日,固不可因泇而廢黃;漕利於泇成之後,亦不可因黃而廢泇。兩利俱存,庶幾緩急可賴。」因請築郗山堤,削頓莊嘴,平大泛口湍溜,濬貓兒窩等處之淺,建鉅梁吳衝閘,增三市徐塘壩,以終泇河未就之功。詔如議。越數年,泇工未竟,督漕者復舍泇由黃。舟有覆者,遷徙黃、泇間,運期久踰限。

三十八年,御史蘇惟霖疏陳黃、泇利害,請專力於泇,略言:「黃河自清河經桃源,北達直河口,長二百四十里。此在泇下流,水平身廣,運舟日行僅十里。然無他道,故必用之。自直河口而上,歷邳、徐達鎮口,長二百八十餘里,是謂黃河。又百二十里,方抵夏鎮。其東自貓窩、泇溝達夏鎮,止二百六十餘里,是謂泇河。東西相對,舍此則彼。黃河三四月間淺與泇同。五月初,其流洶湧,自天而下,一步難行。由其水挾沙而來,河口日高。至七月初,則淺涸十倍。統而計之,無一時可由者。溺人損舟,其害甚劇。泇河計日可達,終鮮風波,但得實心任事之臣,不三五年缺略悉補,數百年之利也。」工科給事中何士晉亦言:「運道最險無如黃河。先年水出昭陽湖,夏鎮以南運道衝阻,開水加之議始決。避淺澀急溜二洪之險,聚諸泉水,以時啟閉,通行無滯者六年。乃今忽欲舍泇由黃,致倉皇損壞糧艘。或改由大浮橋,河道淤塞,復還由泇。以故運抵灣遲,汲汲有守凍之慮,由黃之害略可見矣。顧泇工未竟,闊狹深淺不齊。宜拓廣浚深,與會通河相等。重運空回,往來不相礙,迴旋不相避,水常充盛,舟無留行。歲捐水衡數萬金,督以廉能之吏,三年可竣工。然後循駱馬湖北岸,東達宿遷,大興畚鍤,盡避黃河之險,則泇河之事訖矣。或謂泉脈細微,太闊太深,水不能有。不知泇源遠自蒙、沂,近挾徐塘、許池、文武諸泉河,大率視濟寧泉河略相等。呂公堂口既塞,則山東諸水總合全收,加以閘壩堤防,何憂不足?或謂直抵宿遷,此功迂而難竟,是在任用得人,綜理有法耳。」疏入,不報。

明年,部覆總河都御史劉士忠《泇黃便宜疏》,言:「泇渠春夏間,沂、武、京河山水衝發,沙淤潰決,歲終當如南旺例修治。顧別無置水之地,勢不得不塞泇河壩,令水復歸黃流。故每年三月初,則開泇河壩,令糧艘及官民船由直河進。至九月內,則開召公壩,入黃河,以便空回及官民船往來。至次年二月中塞之。半年由泇,半年由黃,此兩利之道也。因請增驛設官。又覆惟霖疏,言:「直隸貓窩淺,為沂下流,河廣沙淤,不可以閘,最為泇患。宜西開一月河,以通沂口。凡水挾沙來,沙性直走,有月河以分之,則聚於洄伏之處,撈刷較易,而泇患少減矣。」俱報可,其後,無河遂為永利,但需補葺而已。然水加勢狹窄,冬春糧艘回空仍由黃河焉。

四十八年,巡漕御史毛一鷺言:「無河屬夏鎮者有閘九座,屬中河者止藉草壩。分司官議於直口等處建閘,請舉行之。」詔從其議。崇禎四年,總漕尚書楊一鵬浚泇河。九年,總河侍郎周鼎奏重浚泇河成。久之,鼎坐決河防遠戍。給事中沈胤培訟其修泇利運之功,得減論。

衛河,源出河南輝縣蘇門山百門泉。經新鄉、汲縣而東,至畿南浚縣境,淇水入焉,謂之白溝,亦曰宿胥瀆。宋、元時名曰御河。由內黃東出,至山東館陶西,漳水合焉。東北至臨清,與會通河合。北歷德、滄諸州,至青縣南,合滹沱河。北達天津,會白河入海。所謂衛漕也。其河流濁勢盛,運道得之,始無淺澀虞。然自德州下漸與海近,卑窄易衝潰。

初,永樂元年,沈陽軍士唐順言:「衛河抵直沽入海,南距黃河陸路纔五十里。若開衛河,而距黃河百步置倉廒,受南運糧餉,至衛河交運,公私兩便。」乃命廷臣議,未行。其冬,命都督僉事陳俊運淮安、儀真倉糧百五十萬餘石赴陽武,由衛河轉輸北京。五年,自臨清抵渡口驛決堤七處,發卒塞之。後宋禮開會通河,衛河與之合。時方數決堤岸,遂命禮並治之。禮言:「衛輝至直沽,河岸多低薄,若不究源析流,但務堤築,恐復潰決,勞費益甚。會通河抵魏家灣,與土河連,其處可穿二小渠以洩於土河。雖遇水漲,下流衛河,自無橫溢患。德州城西北亦可穿一小渠。蓋自衛河岸東北至舊黃河十有二里,而中間五里故有溝渠,宜開道七里,洩水入舊黃河,至海豐大沽河入海。」詔從之。

英宗初,永平縣丞李祐請閉漳河以防患,疏衛河以通舟。從之。正統四年築青縣衛河堤岸。十三年從御史林廷舉請,引漳入衛。十四年,黃河決臨清四閘,御史錢清請濬其南撞圈灣河以達衛。從之。

景泰四年,運艘阻張秋之決。河南參議豐慶請自衛輝、胙城洎於沙門,陸挽三十里入衛,舟運抵京師。命漕運都督徐恭覆報,如其策。山東僉事江良材嘗言:「通河於衛有三便。古黃河自孟津至懷慶東北入海。今衛河自汲縣至臨清、天津入海,則猶古黃河道也,便一。三代前,黃河東北入海,宇宙全氣所鐘。河南徙,氣遂遷轉。今於河陰、原武、懷、孟間導河入衛,以達天津,不獨徐、沛患息,而京師形勝百倍,便二。元漕舟至封丘,陸運抵淇門入衛。今導河注衛,冬春水平,漕舟至河陰,順流達衛。夏秋水迅,仍從徐、沛達臨清,以北抵京師。且修其溝洫,擇良有司任之,可以備旱澇,捍戎馬,益起直隸、河南富強之勢,便三。」詹事霍韜大然其畫,具奏以聞。不行。

萬曆十六年,總督河漕楊一魁議引沁水入衛,命給事中常居敬勘酌可否。居敬言:「衛小沁大,衛清沁濁,恐利少害多。」乃止。泰昌元年十二月,總河侍郎王佐言:「衛河流塞,惟挽漳、引沁、闢丹三策。挽漳難,而引沁多患。丹水則雖勢與沁同,而丹口既闢,自修武而下皆成安流,建閘築堰,可垂永利。制可,亦未能行也。

崇禎十三年,總河侍郎張國維言:「衛河合漳、沁、淇、洹諸水,北流抵臨清,會閘河以濟運。自漳河他徙,衛流遂弱,挽漳引沁之議,建而未行。宜導輝縣泉源,且酌引漳、沁,闢丹水,疏通滏、洹、淇三水之利害得失,命河南撫、按勘議以聞。」不果行。

漳河,出山西長子曰濁漳,樂平曰清漳,俱東經河南臨漳縣,由畿南真定、河間趨天津入海。其分流至山東館陶西南五十里,與衛河合。洪武十七年,河決臨漳,敕守臣防護。復諭工部,凡堤塘堰壩可禦水患者,皆預修治。有司以黃、沁、漳、衛、沙五河所決堤岸丈尺,具圖計工以聞。詔以軍民兼築之。永樂七年,決固安縣賀家口。九年,決西南張固村河口,與滏陽河合流,下田不可耕。臨漳主簿趙永中乞令災戶於漳河旁墾高阜荒地。從之。是年築沁州及大名等府決堤。十三年,漳、滏並溢,漂沒磁州田稼。二十二年,溢廣宗。洪熙元年,漳、滏並溢,決臨漳三塚村等堤岸二十四處,發軍民修築。宣德八年復築三塚村堤口。

正統元年,漳、滏並溢,壞臨漳杜村西南堤。三年,漳決廣平、順德。四年,又決彰德。皆命修築。十三年,御史林廷舉言:「漳河自沁州發源,七十餘溝會而為一,至肥鄉,堤巖逼隘,水勢激湍,故為民患。元時分支流入衛河,以殺其勢。永樂間堙塞,舊跡尚存,去廣平大留村十八里。宜發丁夫鑿通,置閘,遏水轉入之,而疏廣肥鄉水道。則漳河水減,免居民患,而衛河水增,便漕。」從之。漳水遂通於衛。

正德元年浚滏陽河。河舊在任縣新店村東北,源出磁州。經永年、曲周、平鄉,至穆家口,會百泉等河北流。永樂間,漳河決而與合,二水每並為患。至景泰間,又合漳,衝曲周諸縣,沿河之地皆築堤備之。成化間,舊河淤,衝新店西南為新河,合沙、洺等河入穆家口,亦築堤備之。英宗時,漳已通衛。弘治初,益徙入御河,遂棄滏堤不理。其後,漳水復入新河,兩岸地皆沒。任縣民高暘等以為言,下巡撫官勘奏,言:「穆家口乃眾河之委,當從此先,而併濬新舊河,令分流。漳、滏缺堤,以漸而築。」從之。自此漳、滏匯流,而入衛之道漸堙矣。

萬曆二十八年,給事中王德完言:「漳河決小屯,東經魏縣、元城,抵館陶入衛,為一變,其害小。決高家口,析二流於臨漳之南北,俱至成安東呂彪河合流,經廣平、肥鄉、永年,至曲周入滏水,同流至青縣口方入漕河,為再變,其害大。滏水不勝漳,而今納漳,則狹小不能束巨浪,病溢而患在民。衛水昔仰漳,而今舍漳,則細緩不能捲沙泥,病涸而患在運。塞高家河口,導入小屯河,費少利多,為上策。仍回龍鎮至小灘入衛,費鉅害少,為中策。築呂彪河口,固堤漳水,運道不資利,地方不罹害,為下策。」命河漕督臣集議行之。直隸巡按佴祺亦請引漳河。並下督臣,急引漳會衛,以圖永濟。不果行。

沁河,出山西沁源縣綿山東谷。穿太行山,東南流三十里入河南境。饒河內縣東北,又東南至武陟縣,與黃河會而東注,達徐州以濟漕。其支流自武陟紅荊口,經衛輝入衛河。元郭守敬言:「沁餘水引至武陟,北流合御河灌田。」此沁入衛之故跡也。

明初,黃河自滎澤趨陳、潁,徑入於淮,不與沁合。乃鑿渠引之,令河仍入沁。久之,沁水盡入黃河,而入衛之故道堙矣。武陟者,沁、黃交會處也。永樂間,再決再築。宣德九年,沁水決馬曲灣,經獲嘉至新鄉,水深成河,城北又匯為澤。築堤以防,猶不能遏。新鄉知縣許宣請堅築決口,俾由故道。遣官相度,從之。沁水稍定,而其支流復入於衛。正統三、四年間,武陟沁堤復再決再築。十三年,黃河決滎澤,背沁而去。乃從武陟東寶家灣開渠三十里,引河入沁,以達淮。自後,沁、河益大合,而沁之入衛者漸淤。

景泰三年,僉事劉清言:「自沁決馬曲灣入衛,沁、黃、衛三水相通,轉輸頗利。今決口已塞,衛河膠淺。運舟悉從黃河,嘗遇險阻。宜遣官濬沁資衛,軍民運船視遠近之便而轉輸之。」詔下巡撫集議。

明年,清復言:「東南漕舟,水淺弗能進。請自滎澤入沁河,濬岡頭百二十里以通衛河。且張秋之決,由沁合黃,勢遂奔急。若引沁入衛,則張秋無患。」行人王晏亦言:「開岡頭置閘,分沁水,使南入黃,北達衛。遇漲則閉閘,漕可永無患。」並下督漕都御史王竑等核實以聞。明年,給事中何升言:「沁河有漏港已成河。臨清屯聚膠淺之舟,宜使從此入黃,度二旬可達淮。」詔竑及都御史徐有貞閱之。既而罷引沁河議。初,王晏請漕沁,有司多言弗利。晏固爭。吏部尚書王直請遣官行河,命侍郎趙榮同晏往。榮亦言不利,議乃寢。天順八年,都察院都事金景輝復請濬陳橋集古河,分引沁水,北通長垣、曹州、鉅野,以達漕河。詔按實以聞,未能行也。

弘治二年夏,黃河決埽頭五處,入沁河。其冬,又決祥符翟家口,合沁河,出丁家道口。十一年,員外郎謝緝以黃河南決,恐牽沁水南流,徐、呂二洪必涸。請遏黃河,堤沁水,使俱入徐州。方下所司勘議,明年漕運總兵官郭鋐上副使張鼐《引沁河議》,請於武陟木欒店鑿渠抵荊隆口,分沁水入賈魯河,由丁家道口以下徐、淮。倘河或南徙,即引沁水入渠,以濟二洪之運。帝即令鼐理之。而曹縣知縣鄒魯又駁鼐議,謂引沁必塞沁入河之口,沁水無歸,必漫田廬。若俟下流既通而始塞之,水勢搗虛,千里不折,其患更大,甚於黃陵。且起木欒店至飛雲橋,地以千里計,用夫百萬,積功十年,未能必其成也。兗州知府龔弘主其說,因上言:「鼐見河勢南行,故建此議。但今秋水逆流東北,亟宜濬築。」乃從河臣撫臣議,修丁家口上下堤岸,而鼐議卒罷。

至萬歷十五年,沁水決武陟東岸蓮花池、金屹當,新鄉、獲嘉盡淹沒。廷議築堤障之。都御史楊一魁言:「黃河從沁入衛,此故道也。自河徙,而沁與俱南,衛水每涸。宜引沁入衛,不使助河為虐。」部覆言:「沁入黃,衛入漕,其來已久。頃沁水決木欒蓮花口而東,一魁因建此議。而科臣常居敬往勘,言:『衛輝府治卑於河,恐有衝激。且沁水多沙,入漕反為患,不如堅築決口,廣闢河身』。」乃罷其議。

三十三年,茶陵知州范守己復言:「嘉靖六年,河決豐、沛。胡世寧言:『沁水自紅荊口分流入衛,近年始塞。宜擇武陟、陽武地開一河,北達衛水,以備徐、沛之塞。』會盛應期主開新渠,議遂不行。近者十年前,河沙淤塞沁口,沁水不得入河,自木欒店東決岸,奔流入衛,則世寧紅荊口之說信矣。彼時守土諸臣塞其決口,築以堅堤,仍導沁水入河。而堤外河形直抵衛滸,至今存也。請建石閘於堤,分引一支,由所決河道東流入衛。漕舟自邳溯河而上,因沁入衛,東達臨清,則會通河可廢。」帝命總河及撫、按勘議,不行。

滹沱河,出山西繁峙泰戲山。循太行,掠晉、冀,逶迤而東,至武邑合漳。東北至青縣岔河口入衛,下直沽。或云九河中所稱徒駭是也。

明初,故道由槁城、晉州抵寧晉入衛,其後遷徙不一。河身不甚深,而水勢洪大。左右旁近地大率平漫,夏秋雨潦,挾眾流而潰,往往成巨浸。水落,則因其淺淤以為功。修堤浚流,隨時補救,不能大治也。洪武間一浚。建文、永樂間,修武強、真定決岸者三。至洪熙元年夏,霪雨,河水大漲,晉、定、深三州,槁城、無極、饒陽、新樂、寧晉五縣,低田盡沒,而滹沱遂久淤矣。宣德六年,山水復暴泛,衝壞堤岸,發軍民濬之。正統元年溢獻縣,決大郭黿窩口堤。四年溢饒陽,決醜女堤及獻縣郭家口堤,淹深州田百餘里,皆命有司修築。十一年復疏晉州故道。

成化七年,巡撫都御史楊璇言:「霸州、固安、東安、大城、香河、寶坻、新安、任丘、河間、肅寧、饒陽諸州縣屢被水患,由地勢平衍,水易瀦積。而唐、滹沱、白溝三河上源堤岸率皆低薄,遇雨輒潰。官吏東西決放,以鄰為壑。宜求故跡,隨宜濬之。」帝即命璇董其事,水患稍寧。至十八年,衛、漳、滹沱並溢,潰漕河岸,自清平抵天津決口八十六。因循者久之。

弘治二年修真定縣白馬口及近城堤三千九百餘丈。五年又築護城堤二道。後復比年大水,真定城內外俱浸。改挑新河,水患始息。

嘉靖元年築束鹿城西決口,修晉州紫城口堤。未幾,復連歲被水。十年冬,巡按御史傅漢臣言:「滹沱流經大名,故所築二堤衝敗,宜修復如舊。」乃命撫、按官會議。其明年,敕太僕卿何棟往治之,棟言:「河發渾源州,會諸山之水,東趨真定,由晉州紫城口之南入寧晉泊,會衛河入海,此故道也。晉州西高南下,因衝紫城東溢,而束鹿、深州諸處遂為巨浸。今宜起槁城張村至晉州故堤,築十八里,高三丈,廣十之,植椿榆諸樹。乃浚河身三十餘里,導之南行,使歸故道,則順天、真、保諸郡水患俱平矣。」又用郎中徐元祉言,於真定浚滹沱河以保城池,又導束鹿、武強、河間、獻縣諸水,循滹沱以出。皆從之。自後數十年,水頗戢,無大害。

萬曆九年,給事中顧問言:「臣令任丘,見滹沱水漲,漂沒民田不可勝紀。請自饒陽、河間以下水占之地,悉捐為河,而募夫深浚河身,堅築堤岸,以圖永久。」命下撫、按官勘議。增築雄縣橫堤八里,任丘東堤二十里。

桑乾河,盧溝上源也。發源太原之天池,伏流至朔州馬邑雷山之陽,有金龍池者渾泉溢出,是為桑乾。東下大同古定橋,抵宣府保安州,雁門、應州、雲中諸水皆會。穿西山,入宛平界。東南至看舟口,分為二。其一東由通州高麗莊入白河。其一南流霸州,合易水,南至天津丁字灃入漕河,曰盧溝河,亦曰渾河。河初過懷來,束兩山間,不得肆。至都城西四十里石景山之東,地平土疏,沖激震盪,遷徙弗常。《元史》名盧溝曰小黃河,以其流濁也。上流在西山後者,盈涸無定,不為害。

嘉靖三十三年,御史宋儀望嘗請疏鑿,以漕宣、大糧。三十九年,都御史李文進以大同缺邊儲,亦請「開桑乾河以通運道。自古定橋至盧溝橋務里村水運五節,七百餘里,陸運二節,八十八里。春秋二運,可得米二萬五千餘石。且造淺船由盧溝達天津,而建倉務里村、青白口八處,以備撥運。」皆不能行。下流在西山前者,泛溢害稼,畿封病之,堤防急焉。

洪武十六年濬桑乾河,自固安至高家莊八十里,霸州西支河二十里,南支河三十五里。永樂七年,決固安賀家口。十年,壞盧溝橋及堤岸,沒官田民廬,溺死人畜。洪熙元年,決東狼窩口。宣德三年,潰盧溝堤。皆發卒治之。六年,順天府尹李庸言:「永樂中,運河決新城,高從周口遂致淤塞。霸州桑圓里上下,每年水漲無所洩,漫湧倒流,北灌海子凹、牛欄佃,請亟修築。」從之。七年,侍郎王佐言:「通州至河西務河道淺狹,張家灣西舊有渾河,請疏濬。」帝以役重止之。九年,決東狼窩口,命都督鄭銘往築。正統元年復命侍郎李庸修築,並及盧溝橋小屯廠潰岸,明年工竣。越三年,白溝、運河二水俱溢,決保定縣安州堤五十餘處。復命庸治之,築龍王廟南石堤。七年築渾河口。八年築固安決口。

成化七年,霸州知州蔣愷言:「城北草橋界河,上接渾河,下至小直沽注於海。永樂間,渾河改流,西南經固安、新城、雄縣抵州,屢決為害。近決孫家口,東流入河,又東抵三角澱。小直沽乃其故道,請因其自然之勢,修築堤岸。」詔順天府官相度行之。十九年命侍郎杜謙督理盧溝河堤岸。弘治二年,決楊木廠堤,命新寧伯譚祐、侍郎陳政、內官李興等督官軍二萬人築之。正德元年築狼窩決口。久之,下流支渠盡淤。

嘉靖十年從郎中陸時雍言,發卒浚導。三十四年修柳林至草橋大河。四十一年命尚書雷禮修盧溝河岸。禮言:「盧溝東南有大河,從麗莊園入直沽下海,沙澱十餘里。稍東岔河,從固安抵直沽,勢高。今當先濬大河,令水歸故道,然後築長堤以固之。決口地下水急,人力難驟施。西岸故堤綿亙八百丈,遺址可按,宜併築。」詔從其請。明年旋工,東西岸石堤凡九百六十丈。

萬曆十五年九月,神宗幸石景山,臨觀渾河。召輔臣申時行至幄次,諭曰:「朕每聞黃河衝決,為患不常,欲觀渾河以知水勢。今見河流洶湧如此,知黃河經理倍難。宜飭所司加慎,勿以勞民傷財為故事。至選用務得人,吏、工二部宜明喻朕意。」

膠萊河,在山東平度州東南,膠州東北。源出高密縣,分南北流。南流自膠州麻灣口入海,北流經平度州至掖縣海倉口入海。議海運者所必講也。元至元十七年,萊人姚演獻議開新河,鑿地三百餘里,起膠西縣東陳村海口,西北達膠河,出海倉口,謂之膠萊新河。尋以勞費難成而罷。

明正統六年,昌邑民王坦上言:「漕河水淺,軍卒窮年不休。往者江南常海運,自太倉抵膠州。州有河故道接掖縣,宜浚通之。由掖浮海抵直沽,可避東北海險數千里,較漕河為近。。部覆寢其議。

嘉靖十一年,御史方遠宜等復議開新河。以馬家墩數里皆石岡,議復寢。十七年,山東巡撫胡纘宗言:「元時新河石座舊跡猶在,惟馬壕未通。已募夫鑿治,請復浚淤道三十餘里。」命從其議。

至十九年,副使王獻言:「勞山之西有薛島、陳島,石砑林立,橫伏海中,最險。元人避之,故放洋走成山正東,踰登抵萊,然後出直沽。考膠萊地圖,薛島西有山曰小竺,兩峰夾峙。中有石岡曰馬壕,其麓南北皆接海崖,而北即麻灣,又稍北即新河,又西北即萊州海倉。由麻灣抵海倉纔三百三十里,由淮安踰馬壕抵直沽,纔一千五百里,可免鐃海之險。元人嘗鑿此道,遇石而止。今鑿馬壕以趨麻灣,浚新河以出海倉,誠便。」獻乃於舊所鑿地迤西七丈許鑿之。其初土石相半,下則皆石,又下石頑如鐵。焚以烈火,用水沃之,石爛化為燼。海波流匯,麻灣以通,長十有四里,廣六丈有奇,深半之。由是江、淮之舟達於膠萊。踰年,復濬新河,水泉旁溢,其勢深闊,設九閘,置浮梁,建官署以守。而中間分水嶺難通者三十餘里。時總河王以旂議復海運,請先開平度新河。帝謂妄議生擾,而獻亦適遷去,於是工未變而罷。

三十一年,給事中李用敬言:「膠萊新河在海運舊道西,王獻鑿馬家壕,導張魯、白、現諸河水益之。今淮舟直抵麻灣,即新河南口也,從海倉直抵天津,即新河北口也。南北三百餘里,潮水深入。中有九穴湖、大沽河,皆可引濟。其當疏浚者百餘里耳,宜急開通。」給事中賀涇、御史何廷鈺亦以為請。詔廷鈺會山東撫、按官行視。既而以估費浩繁,報罷。

隆慶五年,給事中李貴和復請開濬,詔遣給事中胡檟會山東撫、按官議。檟言:「獻所鑿渠,流沙善崩,所引白河細流不足灌注。他若現河、小膠河、張魯河、九穴、都泊皆潢汙不深廣。膠河雖有微源,地勢東下,不能北引。諸水皆不足資。上源則水泉枯涸,無可仰給;下流則浮沙易潰,不能持久。擾費無益。」巡撫梁夢龍亦言:「獻誤執元人廢渠為海運故道,不知渠身太長,春夏泉涸無所引注,秋冬暴漲無可蓄洩。南北海沙易塞,舟行滯而不通。」乃復報罷。

萬曆三年,南京工部尚書劉應節、侍郎徐栻復議海運,言:「難海運者以放洋之險,覆溺之患。今欲去此二患,惟自膠州以北,楊家圈以南,濬地百里,無高山長阪之隔,楊家圈北悉通海潮矣。綜而計之,開創者什五,通濬者什三,量浚者什二。以錐探之,上下皆無石,可開無疑。」乃命栻任其事。應節議主通海。而栻往相度,則膠州旁地高峻,不能通潮。惟引泉源可成河,然其道二百五十餘里,鑿山引水,築堤建閘,估費百萬。詔切責栻,謂其以難詞沮成事。會給事中光懋疏論之,且請令應節往勘。應節至,謂南北海口水俱深闊,舟可乘潮,條悉其便以聞。

山東巡撫李世達上言:「南海麻灣以北,應節謂沙積難除,徙古路溝十三里以避之。又慮南接鴨綠港,東連龍家屯,沙積甚高,渠口一開,沙隨潮入,故復有建閘障沙之議。臣以為閘閉則潮安從入?閘啟則沙又安從障也?北海倉口以南至新河閘,大率沙淤潮淺。應節挑東岸二里,僅去沙二尺,大潮一來,沙壅如故,故復有築堤約水障沙之議。臣以為障兩岸之沙則可耳,若潮自中流衝激,安能障也?分水嶺高峻,一工止二十丈,而費千五百金。下多岡句石,掣水甚難。故復有改挑王家丘之議。臣以為吳家口至亭口高峻者共五十里,大概多岡句石,費當若何?而舍此則又無河可行也。夫潮信有常,大潮稍遠,亦止及陳村閘、楊家圈,不能更進。況日止二潮乎?此潮水之難恃也。河道紆曲二百里,張魯、白、膠三水微細,都泊行潦,業已乾涸。設遇亢旱,何泉可引?引泉亦難恃也。元人開浚此河,史臣謂其勞費不貲,終無成功,足為前鑒。」巡按御史商為正亦言:「挑分水嶺下,方廣十丈,用夫千名。纔下數尺為岡句石,又下皆沙,又下盡黑沙,又下水泉湧出,甫挑即淤,止深丈二尺。必欲通海行舟,更須挑深一丈。雖二百餘萬,未足了此。」給事中王道成亦論其失。工部尚書郭朝賓覆請停罷。遂召應節、栻還說,罷其役。嗣是中書程守訓,御史高舉、顏思忠,尚書楊一魁相繼議及之,皆不果行。

崇禎十四年,山東巡撫曾櫻、戶部主事邢國璽復申王獻、劉應節之說。給內帑十萬金,工未舉,櫻去官。十六年夏,尚書倪元璐請截漬糧由膠萊河轉餉,自膠河口用小船抵分水嶺,車盤嶺脊四十里達於萊河,復用小船出海,可無島礁漂損之患。山東副總兵黃廕恩獻議略同。皆未及行。


○直省水利

三代疆理水土之制甚詳。自井田廢,溝遂堙,水常不得其治,於是穿鑿渠塘井陂,以資灌溉。明初,太祖詔所在有司,民以水利條上者,即陳奏。越二十七年,特諭工部,陂塘湖堰可蓄洩以備旱潦者,皆因其地勢修治之。乃分遣國子生及人材,遍詣天下,督修水利。明年冬,郡邑交奏。凡開塘堰四萬九百八十七處,其恤民者至矣。嗣後有所興築,或役本境,或資鄰封,或支官料,或採山場,或農隙鳩工,或隨時集事,或遣大臣董成。終明世水政屢修,可具列云。

洪武元年修和州銅城堰閘,周迴二百餘里。四年修興安靈渠,為陡渠者三十六。渠水發海陽山,秦時鑿,溉田萬頃。馬援葺之,後圮。至是始復。六年發松江、嘉興民夫二萬開上海胡家港,自海口至漕涇千二百餘丈,以通海船,且浚海鹽澉浦。八年開登州蓬萊閣河。命耿炳文浚涇陽洪渠堰,溉涇陽、三原、醴泉、高陵、臨潼田二百餘里。九年修彭州都江堰。十二年,李文忠言:「陜西病鹹鹵,請穿渠城中,遙引龍首渠東注。」從其請,甃以石。十四年築海鹽海塘。十七年築磁州漳河決堤。決荊州嶽山壩以灌民田。十九年築長樂海堤。二十三年修崇明、海門決堤二萬三千九百餘丈,役夫二十五萬人。四川永寧宣慰使言:「所轄水道百九十灘,江門大灘八十二,皆被石塞。」詔景川侯曹震往疏之。二十四年修臨海橫山嶺水閘,寧海、奉化海堤四千三百餘丈。築上虞海堤四千丈,改建石閘。浚定海、鄞二縣東錢湖,灌田數萬頃。二十五年鑿溧陽銀墅東壩河道,由十字港抵沙子河胭脂壩四千三百餘丈,役夫三十五萬九千餘人。二十七年浚山陽支家河,鬱林州民言:「州南北二江相去二十餘里,乞鑿通,設石陡諸閘。」從之。二十九年修築河南洛堤。復興安靈渠。時尚書唐鐸以軍興至其地,圖渠狀以聞。請浚深廣,通官舟以餉軍。命御史嚴震直燒鑿陡澗之石,餉道果通。三十一年,洪渠堰圮,復命耿炳文修治之。且浚渠十萬三千餘丈。建文四年疏吳淞江。

永樂元年,修安陸京山漢水塌岸,章丘漯河東堤,高密、濰決岸,安陽河堤,福山護城決堤,浙江赭山江塘,餘干龍窟壩塘岸,臨潁褚河決口,濰縣白浪河堤,潛山、懷寧陂堰,高要青岐、羅婆圩,通州徐灶、食利等港,平遙廣濟渠,句容楊家港、王旱圩等堤,肇慶、鳳翔遙頭岡決岸,南陽高家、屯頭二堰及沙、澧等河堤,夏縣古河決口三十餘里。修築和州保大等圩百二十餘里,蓄水陡門九。浚昌邑河渠五所,鑿嘉定小橫瀝以通秦、趙二涇,浚崑山葫蘆等河。

命夏原吉治蘇、松、嘉興水患,浚華亭、上海運鹽河,金山衛閘及漕涇分水港。原吉言:「浙西諸郡,蘇、松最居下流,嘉、湖、常頗高,環以太湖,綿亙五百里。納杭、湖、宣、歙溪澗之水,散注澱山諸湖,以入三泖。頃為浦港堙塞,漲溢害稼。拯治之法,在浚吳淞諸浦。按吳淞江袤二百餘里,廣百五十餘丈,西接太湖,東通海,前代常疏之。然當潮汐之衝,旋疏旋塞。自吳江長橋抵下界浦,百二十餘里,水流雖通,實多窄淺。從浦抵上海南倉浦口,百三十餘里,潮汐淤塞,已成平陸,水艷沙游泥,難以施工。嘉定劉家港即古婁江,徑入海,常熟白茆港徑入江,皆廣川急流。宜疏吳淞南北兩岸、安亭等浦,引太湖諸水入劉家、白茆二港,使其勢分。松江大黃浦乃通吳淞要道,今下流遏塞難浚。旁有范家濱,至南倉浦口徑達海。宜浚深闊,上接大黃浦,達泖湖之水,庶幾復《禹貢》「三江入海「之舊。水道既通,乃相地勢,各置石閘,以時啟閉。每歲水涸時,預修圩岸,以防暴流,則水患可息。」帝命發民丁開浚。原吉晝夜徙步,以身先之,功遂成。

二年,修泰州河塘萬八千丈,興化南北堤、泰興沿江圩岸、六合瓜步等屯。浚丹徒通潮舊江,又修象山茭湖塘岸,海康、徐聞二縣那隱坡、調黎等港堤岸,黃嚴混水等十五閘、六陡門,孟津河堤,分宜湖塘,武陟馬田堤岸,香山竹徑水陂,復興安分水塘。興安有江,源出海陽山。江中橫築石埭,分南北渠,溉民田甚溥。埭上疊石如鱗,以防衝溢。嚴震直撤石增埭,水迫無所洩,衝塘岸,盡趨北渠,南渠淺澀,民失利。至是修復如舊。

海門民請發淮安、蘇、常民丁協修張墩港、東明港百餘里潰堤。帝曰:「三郡民方苦水患,不可重勞。」遣官行視,以揚州民協築之。當塗民言:「慈湖瀕江,上通宣、歙,東抵丹陽湖,西接蕪湖。久雨浸淫,潮漲傷農,宜遣勘修築。」帝從其請,且諭工部,安、徽、蘇、松,浙江、江西、湖廣,凡湖泊卑下,圩岸傾頹,亟督有司治之。夏原吉復奉命治水蘇、松,盡通舊河港。又浚蘇州千墩浦、致和塘、安亭、顧浦、陸皎浦、尤涇、黃涇共二萬九千餘丈,松江大黃浦、赤雁浦、范家濱共萬二千丈,以通太湖下流。

先是,修含山崇義堰。未幾,和州民言:「銅城閘上抵巢湖,下揚子江,決圩岸七十餘處,乞修治。」其吏目張良興又言:「水淹麻、澧二湖田五萬餘頃,宜築圩埂,起桃花橋,訖含山界三十里。」俱從之。

三年,修上虞曹娥江壩埂,溫縣馱塢村堤堰四千餘丈,南海衛蓮塘、四會縣鴉鵲水等堤岸,無為州周興等鄉及鷹揚衛烏江屯江岸。築昌黎及歷城小清河決堤,應天新河口北岸,從大勝關抵江東驛三千三百丈。浚海州北舊河,上通高橋,下接臨洪場及山陽運鹽河十八里。

四年,修築宣城十九圩,豐城穆湖圩岸,石首臨江萬石堤,溧水決圩。修懷寧斗潭河、彭灘圩岸,順天固安,保定荊岱,樂亭魯家套、社河口,吉水劉家塘、雲陂,江都劉家圩港。築湖廣廣濟、武家穴等江岸。新建石頭岡圩岸、江浦沿江堤。開泰州運鹽河、普定秦潼河、西溪南儀阡三處河口,導流興化、鹽城界入海。浚常熟福山塘三十六里。

五年,修長洲、吳江、崑山、華亭、錢塘、仁和、嘉興堤岸,餘姚南湖壩,築高要銀岡、金山等潰堤,溉田五百餘頃。治杭州江岸之淪者。六年浚浙江平陽縣河。七年修安陸州渲馬灘決岸、海鹽石堤,築泰興攔江堤三千九百餘丈。且浚大港北淤河,抵縣南,出大江,四千五百餘丈。八年修丹陽練湖塘,汝陽汝河堤岸,南陵野塘圩、蚌蕩壩,松滋張家坑、何家洲堤岸,平度州濰水、浮糠河決口百十二,堤堰八千餘丈,吳江石塘官路橋梁。

九年,修安福丁陂等塘堰,安仁鐃家陂、壽光堤,安陸京山景陵圩岸,長樂官塘,長洲至嘉興石土塘橋路七十餘里,泄水洞百三十一處,監利車水堤四千四百餘丈,高安華陂屯陂堤,仁和、海寧、海鹽土石塘岸萬餘丈。築沂州沭河口決岸,並瀹述陽述河。。築直隸新城張村等口決堤,仁和黃濠塘岸三百餘丈,孫家圍塘岸二十餘里。浚濰縣干丹河、定襄故渠六十三里,引滹沱水灌田六百餘頃。疏福山官渠,浚江陰青陽河道,鄒平白條溝河三十餘里。

麗水民言:「縣有通濟渠,截松陽、遂昌諸溪水入焉。上、中、下三源,流四十八派,溉田二千餘頃。上源民洩水自利,下源流絕,沙壅渠塞。請修堤堰如舊。」部議從之。齊東知縣張昇言:「小清河洪水衝決,淹沒諸鹽場及青州田。請浚上流,修長堤,使水行故道。」皇太子遣官經理之。鄜州民言:「洛水橫決而西,衝塌州城東北隅。請浚故道,循州東山麓南流。」從之。

十年,修浙江平陽捍潮堤岸,黃梅臨江決岸百二十餘里,海門捍潮堤百三十里。築新會圩岸二千餘丈,獻縣、饒陽恭儉等岸,安丘紅河決岸,安州直亭等河決口八十九,華容、安津等堤決口四十六。浚上海蟠龍江、濰縣白浪河。北京行太僕卿楊砥言:「吳橋、東光、興濟、交河及天津等衛屯田,雨水決堤傷稼。德州良店驛東南二十五里有黃河故道,與州南土河通。穿渠置閘,分殺水勢,大為民便。」命侍郎蘭芳往理之。

十一年,修蕪湖陶辛、政和二圩,保定、文安二縣河口決岸五十四,應天新河圩岸,天長福勝、戚家莊二塘,滎澤大濱河堤。浚崑山太平河。十二年修鳳陽安豐塘水門十六座及牛角壩、新倉鋪塌岸,武陟郭村、馬曲堤岸,聊城龍灣河,濮州紅船口,范縣曹村河堤岸。築三河決堤。濬海州官河二百四十里。解州民言:「臨晉涑水河逆流,決姚暹渠堰,入砂地,淹民田,將及鹽池。」尋又言:「硝池水溢,決豁口,入鹽池。」以涑水渠、姚暹渠併流,故命官修築如其請。

十三年,修興濟決岸、南京羽林右衛刁家圩屯田堤。吳江縣丞李升言:「蘇、松水患,太湖為甚,急宜洩其下流。若常熟白茆諸港,崑山千墩等河,長洲十八都港汊,吳縣、無錫近湖河道,皆宜循其故迹,浚而深之。乃修蔡涇等閘,候潮來往,以時啟閉。則泛濫可免,而民獲耕種之利。」從之。十五年修固安孫家口及臨漳固塚堤岸。十六年,修魏縣決岸。

十七年,蕭山民言:「境內河渠四十五里,溉田萬頃,比年淤塞。乞疏濬,仍置閘錢清小江壩東,庶旱潦無憂。」山東新城民言:「縣東鄭黃溝源出淄川,下流壅沮,霖潦妨農。陳家莊南有乾河,上與溝接,下通烏江,乞濬治。」並從之。十八年,海寧諸縣民言:「潮沒海塘二千六百餘丈,延及吳家等壩。」通政岳福亦言:「仁和、海寧壞長降等壩,淪海千五百餘丈。東岸赭山、嚴門山、蜀山舊有海道,淤絕久,故西岸潮愈猛。乞以軍民修築。」並從之。明年修海寧等縣塘岸。

二十一年,修嘉定抵松江潮圮圩岸五千餘丈、交恥順化衛決堤百餘丈。文水民言:「文谷山常稔渠分引文谷河流,袤三十餘里,灌田。今河潰洩水。」從其奏,葺治之。二十二年,修臨海廣濟河閘。

洪熙元年修黃巖濱海閘壩。視永樂初,增府判一員,專其事。修獻縣、鐃陽恭儉堤及窯堤口。

宣德二年,浙江歸安知縣華嵩言:「涇陽洪渠堰溉五縣田八千四百餘頃。洪武時,長興侯耿炳文前後修浚,未久堰壞。永樂間,老人徐齡言於朝,遣官修築,會營造不果。乞專命大臣起軍夫協治。」從之。三年修灌縣都江等堰四十四。臨海民言:「胡巉諸閘瀦水灌田,近年閘壞而金鰲、大浦、湖淶、舉嶼等河遂皆壅阻,乞為開築。」帝曰:「水利急務,使民自訴於朝,此守令不得人爾。」命工部即飭郡縣秋收起工。仍詔天下:「凡水利當興者,有司即舉行,毋緩視。」

巡按江西御史許勝言:「南昌瑞河兩岸低窪,多良田。洪武間修築,水不為患。比年水溢,岸圮二十餘處。豐城安沙繩灣圩岸三千六百餘丈,永樂間水衝,改修百三十餘丈。近者久雨,江漲堤壞。乞敕有司募夫修理。」中書舍人陸伯倫言:「常熟七浦塘東西百里,灌常熟、崑山田,歲租二十餘萬石。乞聽民自浚之。」皆詔可。

四年,修獻縣柳林口堤岸。潛江民言:「蚌湖、陽湖皆臨襄河,水漲岸決,害荊州三衛、荊門、江陵諸州縣官民屯田無算。乞發軍民築治。從之。福清民言:「光賢里官民田百餘頃,堤障海水。堤壞久,田盡荒。永樂中,嘗命修治,迄今未舉,民不得耕。」帝責有司亟治,而諭尚書吳中嚴飭郡邑,陂池堤堰及時修浚,慢者治以罪。

五年,巡撫侍郎成均言:「海鹽去海二里,石嵌土岸二千四百餘丈,水齧其石,皆已刓敝。議築新石於岸內,而存其舊者以為外障。乞如洪武中令嘉、嚴、紹三府協夫舉工。」從之。

六年,修瀏陽、廣濟諸縣堤堰,豐城西北臨江石堤及西南七圩壩,石首臨江三堤。浚餘姚舊河池。巡撫侍郎周忱言:「溧水永豐圩周圍八十餘里,環以丹陽、石臼諸湖。舊築埂壩,通陟門石塔,農甚利之。今頹敗,請葺治。」教諭唐敏言:「常熟耿涇塘,南接梅里,通昆承湖,北達大江。洪武中,濬以溉田。今壅阻,請疏導。」並從之。

七年,修眉州新津通濟堰。堰水出彭山,分十六渠,溉田二萬五千餘畝。河東鹽運使言:「鹽池近地姚暹河,流入五星湖轉黃流河,兩岸窪下。比歲雨溢水漲,衝至解州。浪益急,遂潰南岸,沒民田三十餘里,鹽池護堤皆壞。復因下流涑水河高,壅淤逆流,姚暹以決。乞起民夫疏瀹。」從之。蘇州知府況鐘言:「蘇、松、嘉、湖湖有六,曰太湖、龐山、陽城、沙湖、昆承、尚湖。永樂初,夏原吉浚導,今復淤。乞遣大臣疏浚。」乃命周忱與鐘治之。是歲,汾河驟溢,敗太原堤。鎮守都司李謙、巡按御史徐傑以便宜修治,然後馳奏。帝嘉獎之。

八年,葺湖廣偏橋衛高陂石洞,完縣南關舊河。復和州銅城堰閘。修安陽廣惠等渠,磁州滏陽河、五爪濟民渠。九年修江陵枝江沿江堤岸。築薊州決岸。毀蘇、松民私築堤堰。十年築海鹽潮決海塘千五百餘丈。主事沈中言:「山陰西小江,上通金、嚴,下接三江海口,引諸暨、浦江、義烏諸湖水以通舟。江口近淤,宜築臨浦戚堰障諸湖水,俾仍出小江。」詔部覆奪。

正統元年,修吉安沿江堤。築海陽、登雲、都雲、步村等決堤。濬陜西西安灞橋河。二年築蠡縣王家等決口。修新會鸞臺山至瓦塘浦頹岸,江陵、松滋、公安、石首、潛江、監利近江決堤。又修湖廣老龍堤,以為漢水所潰也。三年疏泰興順德鄉三渠,引湖溉田;潞州永祿等溝渠二十八道,通於漳河。四年修容城杜村口堤。設正陽門外減水河,并疏城內溝渠。荊州民言:「城西江水高城十餘丈,霖潦壞堤,水即灌城。請先事修治。」寧夏巡撫都御史金濂言:「鎮有五渠,資以行溉,今明沙州七星、漢伯、石灰三渠久塞。請用夫四萬疏浚,溉蕪田千三百餘頃。」並從之。

五年,修太湖堤,海鹽海岸,南京上中下新河及濟川衛新江口防水堤,漷縣、南宮諸堤。築順天、河間及容城杜村口、郎家口決堤。塞海寧蠣巖決堤口。濬鹽城伍祐、新興二場運河。初,溧水有鎮曰廣通,其西固城湖入大江,東則三塔堰河入太湖。中間相距十五里,洪武中鑿以通舟。縣地稍AH,而湖納寧國、廣德諸水,遇潦即溢,乃築壩於鎮以禦之,而堰水不能至壩下。是歲,改築壩於葉家橋。胭脂河者,溧水入秦淮道也。蘇、松船皆由以達,沙石壅塞,因并浚之。山陽涇河壩,上接漕河,下達鹽城,舊置絞關以通舟,歲久且敝,又恐盜洩水利,遂築塞河口。是歲,從民請,修壩并復絞關。

六年,造宣武門東城河南岸橋。修江米巷玉河橋及堤,并浚京城西南河。築豐城沙月諸河堤、蕪湖陶辛圩新埂。浚海寧官河及花塘河、硤石橋塘河,築瓦石堰二所。疏南京江洲,殺其水勢,以便修築塌岸。高郵知州韓簡言:「官河上下二閘皆圮,河亦不通,且子嬰溝塞,減水陰洞閉,致旱澇無所濟。俱乞濬治。」詔部核實以行。

七年,修江西廣昌江岸、蕭山長山浦海塘、彭山通濟堰。築南京浦子口、大勝關堤,九江及武昌臨江塌岸。浚江陵、荊門、潛江淤沙三十餘里。八年修蘭溪卸橋浦口堤,弋陽官陂三所。浚南京城河。

九年,修德州耿家灣等堤岸、杞縣離溝堤。築容城杜村堤決口。易上虞菱湖土壩為石閘。挑無錫里谷、蘇塘、華港、上村、李走馬塘諸河,東南接蘇州苑山湖塘,北通揚子江,西接新興河,引水灌田。浚杞縣牛墓岡舊河,武進太平、永興二河。疏海鹽永安河,茶市院新涇、陶涇塘諸河。都御史陳鎰言:「朝邑多沙鹻,難耕。縣治洛河,與渭水通,請穿渠灌之。」新安民言:「城南長溝河,西通徐、漕二水,東連雄縣直沽,沙土淤塞,請發丁夫疏浚。」海陽民蕭瑤言:「縣有長溪,源出山麓,流抵海口,周袤潮郡,故登隆等都俱置溝通溉。惟隆津等都陸野絕水,歲旱無所賴。乞開溝如登隆。」長樂民劉彥梁言:「嚴湖二十餘里,南接稠菴溪,西通倒流溪,可備旱溢。又有張塘涵、塘前涵、大塘涵、陳塘港,其利如嚴湖。乞令有司疏浚。」廣濟民言:「縣與鄰邑黃梅,歲運糧三萬石於望牛墩。小車盤剝,不堪其勞。連城湖港廖家口有溝抵墩前,淤淺不能行船。請與黃梅合力浚通,以便水運。」並從之。

十一年,修洞庭湖堤。築登州河岸。浚通州金沙場八里河,以通運渠。任丘民言:「凌城港去縣二十五里,內有定安橋河,北十八里通流,東七里沙塞。宜疏通與港相接。入直沽張家灣。」巡撫周忱言:「應天、鎮江、太平、寧國諸府,舊有石臼等湖。其中溝港,歲辦魚課。其外平圩淺灘,聽民牧放孳畜、採掘菱藕,不許種耕。故山溪水漲,有所宣洩。近者富豪築圩田,遏湖水,每遇泛溢,害即及民,宜悉禁革。」並從之。

十二年,疏平度州大灣口河道,荊州公安門外河,以便公安、石首諸縣輸納。浙江聽選官王信言:「紹興東小江,南通諸暨七十二湖,西通錢塘江。近為潮水湧塞,江與田平,舟不能行,久雨水溢,鄰田輒受其害。乞發丁夫疏浚。」從之。

十三年,築寧夏漢、唐壩決口。疏山西涑水河、南海縣通海泉源。鑿宣府城濠,引城北山水入南城大河。湖廣五開衛言:「衛與苗接,山路峻險。去衛三十里有水通靖州江,亂石沙灘,請疏以便輸運。」雲南鄧川州言:「本州民田與大理衛屯田接壤湖畔,每歲雨水沙土壅淤,禾苗淹沒。乞命州衛軍民疏治。」並從之。

十四年,浚南海潘埇堤岸,置水閘。和州民言:「州有姥鎮河,上通麻、澧二湖,下接牛屯大河,長七十里許,廣八丈。又有張家溝,連銅城閘,通大江,長減姥鎮之半,廣如之,灌溉降福等七十餘圩及南京諸衛屯田,近年河潰閘圮,率皆淤塞。請興役疏浚,仍於姥鎮、豐山嘴、葉公坡各建閘以備旱澇。」從之。

景泰元年,築丹陽甘露等壩。二年修玉河東、西堤。浚安定門東城河,永嘉三十六都河,常熟顧新塘,南至當湖,北至揚子江。三年修泰和信豐堤。築延安、綏德決河,綿州西岔河通江堤岸。浚常熟七浦塘,劍州海子。疏孟瀆河濱涇十一。工部言:「海鹽石塘十八里,潮水衝決,浮土修築,不能久。」詔別築石塘捍之。

四年,浚江陰順塘河十餘裏,東接永利倉大河,西通夏港及揚子江。雲南總兵官沐璘言:「城東有水南流,源發邵甸,會九十九泉為一,抵松花壩分為二支:一繞金馬山麓,入滇池;一從黑窯村流至雲澤橋,亦入滇池。舊於下流築堰,溉軍民田數十萬頃,霖潦無所洩。請令受利之家,自造石閘,啟閉以時。」報可。五年疏靈寶黎園莊渠,通鴻瀘澗,溉田萬頃。六年濬華容杜預渠,通運船入江,避洞庭險。修容城白溝河杜村口、固安楊家等口決堤。

七年,尚書孫原貞言:「杭州西湖舊有二閘,近皆傾圮,湖遂淤塞。按宋蘇軾云『杭本江海故地,水泉堿苦。自唐李泌引湖水入城為六井,然後井邑日富,不可許人佃種。』周淙亦言:『西湖貴深闊。』因招兵二百,專一撈湖。其後,豪戶復請佃,湖日益填塞,大旱水涸。詔郡守趙與亹開濬,芰荷茭蕩悉去,杭民以利。此前代經理西湖大略也。其後,勢豪侵占無已,湖小淺狹,閘石毀壞。今民田無灌溉資,官河亦澀阻。乞敕有司興濬,禁侵占以利軍民。」從之。

天順二年,修彭縣萬工堰,灌田千餘頃。五年,僉事李觀言:「涇水出涇陽仲山谷,道高陵,至櫟陽入渭,袤二百里。漢開渠溉田,宋、元俱設官主之。今雖有瓠口鄭、白二渠,而堤堰摧決,溝洫壅瀦,民弗蒙利。」乃命有司濬之。

八年,永平民言:「漆河繞城西南流入海,城趾皆石,故水不能決。其餘則沙土易潰,前人於東北築土堤,西南甓岸。今歲久日塌,宜作堤於東流,橫以激之,使合西流,庶無蕩析患。」都御史項忠言:「涇陽之瓠口鄭、白二渠,引涇水溉田數萬頃,至元猶溉八千頃。其後渠日淺,利因以廢。宣德初,遣官修鑿,畝收四三石。無何復塞,渠旁之田,遇旱為赤地。涇陽、醴泉、三原、高陵皆患苦之。昨請於涇水上源龍潭左側疏浚,訖舊渠口,尋以詔例停止。今宜畢其役。西安城西井泉堿苦,飲者輒病。龍首渠引水七十里,修築不易,且利止及城東。西南皁河去城一舍許,可鑿,令引水與龍首渠會,則居民盡利。」邳州知州孟琳言:「榆行諸社俱臨沂河,久雨岸崩二十八處,低田盡淹。乞與修築。並從之。

成化二年,修壽州安豐塘。四年,疏石州城河。六年,修平湖周家涇及獨山海塘。七年,潮決錢塘江岸及山陰、會稽、蕭山、上虞,乍浦、瀝海二所,錢清諸場。命侍郎李顒修築。八年,堤襄陽決岸。十年,廷臣會議:江浦北城圩古溝,北通滁河浦子口;城東黑水泉古溝,南入大江。二溝相望,岡壟中截。宜鑿通成河,旱引澇洩。從之。

十一年,濬杭州錢塘門故渠,左屬湧金門,建橋閘以蓄湖水。巡撫都御史牟俸言:「山東小清河,上接濟南趵突諸泉,下通樂安沿海高家港鹽場。大清河,上接東平坎河諸泉,下通濱州海豐、利津,沿海富國鹽場。淤塞,苦盤剝,雨水又患淹沒。勸農參政唐虞浚河造閘,請令兼治水利。」詔可。

十二年,巡按御史許進言:「河西十五衛,東起莊浪,西抵肅州,綿亙幾二千里,所資水利多奪於勢豪。宜設官專理。」詔屯田僉事兼之。

十四年,俸言:「直隸蘇、松與浙西各府,頻年旱澇,緣周環太湖,乃東南最窪地,而蘇、松尤最下之衝。故每逢積雨,眾水奔潰,湖泖漲漫,淹沒無際。按太湖即古震澤,上納嘉、湖、宣、歙諸州之水,下通婁、東、吳淞三江之流,東江今不復見,婁、淞入海故跡具存。其地勢與常熟福山、白茆二塘俱能導太湖入江海,使民無墊溺,而土可耕種,歷代開浚具有成法。本朝亦常命官修治,不得其要。而濱湖豪家盡將淤灘栽蒔為利。治水官不悉利害,率於泄處置石梁,壅土為道,或慮盜船往來,則釘木為柵。以致水道堙塞,公私交病。請擇大臣深知水利者專理之,設提督水利分司一員隨時修理,則水勢疏通,東南厚利也。」帝即令俸兼領水利,聽所浚築。功成,乃專設分司。

十五年,修南京內外河道。十八年,浚雲南東西二溝,自松華壩黑龍潭抵西南柳壩南村,灌田數萬頃。修居庸關水關、城券及隘口水門四十九,樓鋪、墩臺百二。二十年,修嘉興等六府海田堤岸,特選京堂官往督之。二十二年,浚南京中下二新河。

弘治三年,從巡撫都御史丘鼐言,設官專領灌縣都江堰。六年,敕撫民參政朱瑄浚河南伊、洛,彰德高平、萬金,懷慶廣濟,南陽召公等渠,汝寧桃陂等堰。

七年,浚南京天、潮二河,備軍衛屯田水利。七月命侍郎徐貫與都御史何鑑經理浙西水利。明年四月告成。貫初奉命,奏以主事祝萃自隨。萃乘小舟究悉源委。貫乃令蘇州通判張旻疏各河港水,瀦之大壩。旋開白茆港沙面,乘潮退,決大壩水衝激之,沙泥刷盡。潮水蕩激,日益闊深,水達海無阻。又令浙江參政周季麟修嘉興舊堤三十餘里,易之以石,增繕湖州長興堤岸七十餘里。貫乃上言:「東南財賦所出,而水患為多。永樂初,命夏原吉疏濬。時以吳淞江水艷沙浮蕩,未克施工。迨今九十餘年,港浦愈塞。臣督官行視,浚吳江長橋,導太湖散入水殿山、陽城、昆承等湖泖。復開吳淞江並大石、趙屯等浦,洩水殿山湖水,由吳淞江以達於海。開白茆港白魚洪、占魚口,洩昆承湖水,由白茆港以注於江。開斜堰、七鋪、鹽鐵等塘,洩陽城湖水,由七丫港以達於海。下流疏通,不復壅塞。乃開湖州之漊涇,洩西湖、天目、安吉諸山之水,自西南入於太湖。開常州之百瀆,洩溧陽、鎮江、練湖之水,自西北入於太湖。又開諸陡門,洩漕河之水,由江陰以入於大江。上流亦通,不復堙滯。」是役也,修浚河、港、涇、瀆、湖、塘、陡門、堤岸百三十五道,役夫二十餘萬,祝萃之功多焉。

巡撫都御史王珣言:「寧夏古渠三道,東漢、中唐並通。惟西一渠傍山,長三百餘里,廣二十餘丈,兩岸危峻,漢、唐舊跡俱堙。宜發卒濬鑿,引水下流。即以土築東岸,建營堡屯兵以遏寇衝。請帑銀三萬兩,并靈州六年鹽課,以給其費。」又請於靈州金積山河,開渠灌田,給軍民佃種。並從之。

十八年,修築常熟塘壩,自尚湖口抵江,及黃、泗等浦,新莊等沙三十餘處。浚杭州西湖。

正德七年,修廣平滏陽河口堤岸。十四年浚南京新江口右河。十五年,御史成英言:「應天等衛屯田在江北滁、和、六合者,地勢低,屢為水敗。從金城港抵濁河達烏江三十餘里,因舊跡浚之,則水勢洩而屯田利。」詔可。

嘉靖元年,築浚束鹿、肥鄉、獻、魏堤渠。初,蘇、松水道盡為勢家所據。巡撫李充嗣畫水為井地,示開鑿法,戶占一區,計工刻日。造浚川爬,用巨筏數百,曳木齒,隨潮進退,擊汰泥沙。置小艇百餘,尾鐵帚以導之。浚故道,穿新渠,巨浦支流,罔不灌注。帝嘉其勞,賚以銀幣。二年,修德勝門東、朝陽門北城垣河道,築儀真、江都官塘五區。

十年,工部郎中陸時雍言:「良鄉盧溝河,涿州琉璃、胡良二河,新城、雄縣白溝河,河間沙河,青縣滹沱河,下流皆淤。宜以時浚,使達於海。」詔巡撫議之。

十一年,太僕卿何棟勘畿封河患有二。一論滹沱河。其一言:「真定鴨、沙、磁三河,俱發源五臺。會諸支水,抵唐河蘭家圈,合流入河間。東南經任丘、霸州、天津入海,此故道也。河間東南高,東北下,故水決蘭家口,而肅寧、新安皆罹其害。宜築決口,濬故道。涿州胡良河,自拒馬分流,至州東入渾河。良鄉琉璃河,發源磁家務,潛入地中,至良鄉東入渾河。比者渾河壅塞,二河不流。然下流淤沙僅四五里,請亟浚之。」部覆允行。

郎中徐元祉受命振災,上言:「河本以洩水,今反下壅;澱本以瀦水,今反上溢。故畿輔常苦水,順天利害相半,真定利多於害,保定害多於利,河間全受其害。弘、正間,嘗築長堤,排決口,旋即潰敗。今惟疏浚可施,其策凡六。一濬本河,俾河身寬邃。九河自山西來者,南合滹沱而不侵真定諸郡,北合白溝而不侵保定諸郡。此第一義也。一浚支河。令九河之流,經大清河,從紫城口入;經文都村,從涅槃口入;經白洋澱,從蘭家口入;經章哥AH,從楊村河入。直遂以納細流,水力分矣。一濬決河。九河安流時,本支二河可受,遇漲則岸口四衝。宜每衝量存一口,復濬令合成一渠,以殺湍急,備淫溢。一浚澱河。令澱澱相通,達於本支二河,使下有所洩。一濬淤河。九河東逝,悉由故道,高者下,下者通。占據曲防者抵罪。一濬下河。九河一出青縣,一出丁字沽,二流相匝於苑家口。故施工必自苑家口始,漸有成效,然後次第舉行,庶減諸郡水害。」帝嘉納之。

明年,香河郭家莊自開新河一道,長百七十丈,闊五十丈,近舊河十里餘。詔河官亟繕治。

十三年,巡撫都御史周金言:「蘭家圈決口,塞之則東溢,病河間;不塞則東流漸淤,病保定。宜存決口而濬廣新河,使水東北平流,無壅涸患。」從之。

二十四年,浚南京後湖。初,胡體乾按吳,以松江泛溢,進六策:「曰開川,曰浚湖,曰殺上流之勢,曰決下流之壑,曰排潮漲之沙,曰立治田之規。是年,呂光洵按吳,復奏蘇、松水利五事:

一曰廣疏浚以備瀦洩。三吳澤國,西南受太湖諸澤,水勢尤卑。東北際海,岡隴之地,視西南特高。高苦旱,卑苦澇。昔人於下流疏為塘浦,導諸湖水北入江,東入海,又引江潮流衍於岡隴外。瀦洩有法,水旱無患。比來縱浦橫塘,多堙不治,惟黃浦、劉河二江頗通。然太湖之水源多勢盛,二江不足以洩之。岡隴支河又多壅絕,無以資灌溉。於是高下俱病,歲常告災。宜先度要害,於水殿山等茭蘆地,導太湖水散入陽城、昆承、三泖等湖。又開吳淞江及大石、趙屯等浦,洩澱山之水以達於海。濬白茆、占魚諸口,洩昆承之水以注於江。開七浦、鹽鐵等塘,洩陽城之水以達於江。又導田間之水,悉入小浦,以納大浦,使流者皆有所歸,瀦者皆有所洩。則下流之地治,而澇無所憂矣。乃浚艾祁、通波以溉青浦,濬顧浦、吳塘以溉嘉定,濬大瓦等浦以溉崑山之東,浚許浦等塘以溉常熟之北,濬臧村等港以溉金壇,浚澡港等河以溉武進。凡隴岡支河堙塞不治者,皆浚之深廣,使復其舊。則上流之地亦治,而旱無所憂矣。此三吳水利之經也。

一曰修圩岸以固橫流。蘇、松、常、鎮東南下流,而蘇、松又常、鎮下流,易瀦難洩。雖導河浚浦引注江海,而秋霖泛漲,風濤相薄,則河浦之水逆行田間,衝嚙為患。宋轉運使王純臣嘗令蘇、湖作田塍禦水,民甚便之。司農丞郟亦云:「治河以治田為本。」故老皆云,前二三十年,民間足食,因餘力治圩岸,田益完美。近皆空乏,無暇修繕,故田圩漸壞,歲多水災。合敕所在官司專治圩岸。岸高則田自固,雖有霖澇,不能為害。且足制諸湖之水咸歸河浦中,則不待決洩,自然湍流。而岡隴之地,亦因江水稍高,又得畝引以資灌溉,不特利於低田而已。

一曰復板閘以防污水殿。河浦之水皆自平原流入江海,水慢潮急,以故沙隨浪湧,其勢易淤。昔人權其便宜,去江海十里許夾流為閘,隨潮啟閉,以禦淤沙。歲旱則長閉以蓄其流,歲澇則長啟以宣其溢,所謂置閘有三利,蓋謂此也。近多堙塞,惟常熟福山閘尚存。故老以為河浦入海之地,誠皆置閘,自可歷久不壅。

一曰量緩急以處工費。

一曰重委任以責成功。

詔悉如議。光洵因請專委巡撫歐陽必進。從之。二十六年,給事中陳斐請仿江南水田法,開江北溝洫,以祛水患,益歲收。報可。

三十八年,總督尚書楊博請開宣、大荒田水利。從之。巡撫都御史翁大立言:「東吳水利,自震澤浚源以注江,三江導流以入海,而蘇州三十六浦,松江八匯,毘陵十四瀆,共以節宣旱澇。近因倭寇衝突,汊港之交,率多釘柵築堤以為捍禦,因致水流停瀦,淤滓日積。渠道之間,仰高成阜。且具區湖泖,並水而居者雜蒔茭蘆,積泥成蕩,民間又多自起圩岸。上流日微,水勢日殺。黃浦、婁江之水又為舟師所居,下流亦淤。海潮無力,水利難興,民田漸磽。宜於吳淞、白茆、七浦等處造成石閘,啟閉以時。挑鎮江、常州漕河深廣,使輸挽無阻,公私之利也。」詔可。

四十二年,給事中張憲臣言:「蘇、松、常、嘉、湖五郡水患疊見。請浚支河,通潮水;築圩岸,禦湍流。其白茆港、劉家河、七浦、楊林及凡河渠河蕩壅淤沮洳者,悉宜疏導。」帝以江南久苦倭患,民不宜重勞,令酌濬支河而已。四十五年,參政凌雲翼請專設御史督蘇、松水利。詔巡鹽御史兼之。

隆慶三年,開湖廣竹筒河以洩漢江。巡撫都御史海瑞疏吳淞江下流上海淤地萬四千丈有奇。江面舊三十丈,增開十五丈,自黃渡至宋家橋長八十里。明年春,瑞言:「三吳入海之道,南止吳淞,北止白茆,中止劉河。劉河通達無滯,吳淞方在挑疏。土人請開白茆,計浚五千餘丈,役夫百六十四萬餘。」又言:「吳淞役垂竣,惟東西二壩未開。父老皆言崑山夏駕口、吳江長橋、長洲寶帶橋、吳縣胥口及凡可通流下吳淞者,逐一挑畢,方可開壩。」並從之。是年築海鹽海塘。越四年,從巡撫侍郎徐栻議,復開海鹽秦駐山,南至澉浦舊河。

萬曆二年,築荊州采穴,承天泗港、謝家灣諸決堤口。復築荊、岳等府及松滋諸縣老垸堤。

四年,巡撫都御史宋儀望言:「三吳水勢,東南自嘉、秀沿海而北,皆趨松江,循黃浦入海;西北自常、鎮沿江而東,皆趨江陰、常熟。其中太湖瀦蓄,匯為巨浸,流注龐山、瀆墅、澱山、三泖,陽城諸湖。乃開浦引湖,北經常熟七浦、白茆諸港入於江,東北經崑山、太倉穿劉家河,東南通吳淞江、黃浦,各入於海。諸水聯絡,四面環護,中如仰盂。杭、嘉湖、常、鎮勢繞四隅,蘇州居中,松江為諸水所受,最居下。乞專設水利僉事以裨國計。」部議遣御史董之。

六年,巡撫都御史胡執禮請先浚吳淞江長橋、黃浦。先是,巡按御史林應訓言:

「蘇、松水利在開吳淞江中段,以通入海之勢。太湖入海,其道有三:東北由劉河,即古婁江故道;東南由大黃浦,即古東江遺意;其中為吳淞江,經崑山、嘉定、青浦、上海,乃太湖正脈。今劉河、黃浦皆通,而中江獨塞者,蓋江流與海潮遇,海潮渾濁,賴江水迅滌之。劉河獨受巴、陽諸湖,又有新洋江、夏駕浦從旁以注;大黃浦總會杭、嘉之水,又有澱山、泖蕩從上而灌。是以流皆清駛,足以敵潮,不能淤也。

惟吳淞江源出長橋、石塘下,經龐山、九里二湖而入。今長橋、石塘已堙,龐山、九里復為灘漲,其來已微。又有新洋江、夏駕浦掣其水以入劉河,勢乃益弱,不能勝海潮洶湧之勢而滌濁渾之流,日積月累,淤塞僅留一線。水失故道,時致淫濫。支河小港,亦復壅滯。舊熟之田,半成荒畝。

前都御史海瑞力破群議,挑自上海江口宋家橋至嘉定艾祁八十里,幸尚通流。自艾祁至崑山慢水港六十餘里,則俱漲灘,急宜開濬,計淺九千五百餘丈,闊二十丈。此江一開,太湖直入於海,濱江諸渠得以引流灌田,青浦積荒之區俱可開墾成熟矣。」

並從之。至是,工成。應訓又言:

「吳江縣治居太湖正東,湖水由此下吳淞達海。宋時運道所經,畏風阻險,乃建長橋、石塘以通牽挽。長橋百三十丈,為洞六十有二。石塘小則有竇,大則有橋,內外浦涇縱橫貫穿,皆為洩水計也。石塘涇竇半淤,長橋內外俱圮,僅一二洞門通水。若不疏浚,雖開吳淞下流,終無益也。宜開龐山湖口,由長橋抵吳家港。則湖有所洩,江有所歸,源盛流長,為利大矣。

松江大黃浦西南受杭、嘉之水,西北受水殿、泖諸蕩之水,總會於浦,而秀州塘、山涇港諸處實黃浦來源也。水殿山湖入黃浦道漸多淤淺,宜為疏瀹。而自黃浦、橫澇、洙涇,經秀州塘入南泖,至山涇港等處,萬四千餘丈,待浚尤急。

他如蘇之茜涇、楊林、白茆、七浦諸港,松之蒲匯、官紹諸塘,常、鎮之澡港、九曲諸河,併宜設法開導,次第修舉。」

八年,又言:「蘇、松諸郡乾河支港凡數百,大則洩水入海,次則通湖達江,小則引流灌田。今吳淞江、白茆塘、秀州塘、蒲匯塘、孟瀆河、舜河、青暘港俱已告成,支河數十,宜盡開濬。」俱從其請。

久之,用儀望議,特設蘇、松水利副使,以許應逵領之。乃浚吳淞八十餘晨,築塘九十餘處,開新河百二十三道,浚內河百三十九道,築上海李家洪老鴉嘴海岸十八里,發帑金二十萬。應逵以其半訖工。三十七、八年間,霪雨浸溢,水患日熾。越數年,給事中歸子顧言:「宋時,吳淞江闊九里。元末淤塞。正統間,周忱立表江心,疏而浚之。崔恭、徐貫、李充嗣、海瑞相繼浚者凡五,迄今四十餘年,廢而不講。宜使江闊水駛,塘浦支河分流四達。」疏入留中。巡按御史薛貞復請行之,下部議而未行。至天啟中,巡撫都御史周起元復請浚吳淞、白茆。崇禎初,員外郎蔡懋德、巡撫都御史李待問皆以為請。久之,巡撫都御史張國維請疏吳江長橋七十二谼及九里、石塘諸洞。御史李謨復請浚吳淞、白茆。俱下部議,未能行也。

十年,增築雄縣橫堤八里,禦滹沱暴漲。

十三年,以尚寶少卿徐貞明兼御史,領墾田使。貞明為給事中,嘗請興西北水利如南人圩田之制,引水成田。工部覆議:「畿輔諸郡邑,以上流十五河之水洩於貓兒一灣,海口又極束隘,故所在橫流。必多開支河,挑濬海口,而後水勢可平,疏浚可施。然役大費繁,而今以民勞財匱,方務省事,請罷其議。」乃已。後貞明謫官,著《潞水客譚》一書,論水利當興者十四條。時巡撫張國彥、副使顧養謙方開水利於薊、永有效,於是給事中王敬民薦貞明,特召還,賜敕勘水利。貞明乃先治京東州邑,如密雲燕樂莊,平谷水峪寺、龍家務莊,三河塘會莊、順慶屯地。薊州城北黃厓營,城西白馬泉、鎮國莊,城東馬伸橋,夾林河而下別山鋪,夾陰流河而下至於陰流。遵化平安城,夾運河而下沙河鋪西,城南鐵廠、湧珠湖以下韭菜溝、上素河、下素河百餘里。豐潤之南,則大寨、剌榆坨、史家河、大王莊,東則榛子鎮,西則鴉紅橋,夾河五十餘里。玉田青莊塢、後湖莊、三里屯及大泉、小泉,至於瀕海之地,自水道沽關、黑嚴子墩至開平衛南宋家營,東西百餘里,南北百八十里。墾田三萬九千餘畝。至真定將治滹沱近堧地,御史王之棟言:「滹沱非人力可治,徒耗財擾民。」帝入其言,欲罪諸建議者。申時行言:「墾田興利謂之害民,議甚舛。顧為此說者,其故有二。北方民遊惰好閑,憚於力作,水田有耕耨之勞,胼胝之苦,不便一也。貴勢有力家侵占甚多,不待耕作,坐收蘆葦薪芻之利;若開墾成田,歸於業戶,隸於有司,則已利盡失,不便二也。然以國家大計較之,不便者小,而便者大。惟在斟酌地勢,體察人情,沙堿不必盡開,黍麥無煩改作,應用夫役,必官募之,不拂民情,不失地利,乃謀國長策耳。」於是貞明得無罪,而水田事終罷。

巡撫都御史梁問孟築橫城堡邊墻,慮寧夏有黃河患,請堤西岔河,障水東流。從之。十九年,尚寶丞周弘禴言:「寧夏河東有漢、秦二壩,請依河西漢、唐壩築以石,於渠外疏大渠一道,北達鴛鴦諸湖。」詔可。

二十三年,黃、淮漲溢,淮、揚昏墊。議者多請開高家堰以分淮。寶應知縣陳煃為御史,慮高堰既開,害民產鹽場,請自興、鹽迤東,疏白塗河、石達口、廖家港為數河,分門出海;然後從下而上,浚清水、子嬰二溝,且多開瓜、儀閘口以洩水。給事中祝世祿亦言:「議者欲放淮從廣陽、射陽二湖入海。廣陽闊僅八里,射陽僅二十五丈,名為湖,實河也。且離海三百里,迂迴淺窄,高、寶七州縣水惟此一線宣洩之,又使淮注焉,田廬鹽場,必無幸矣。廣陽湖東有大湖,方廣六十里,湖北口有舊官河,自官蕩至鹽城石達口,通海僅五十三里,此導淮入海一便也。」下部及河漕官議,俱格不行。既而總河尚書楊一魁言:「黃水倒灌,正以海口為阻。分黃工就,則石達口、廖家港、白駒場海口,金灣、芒稻諸河,急宜開刷。」乃命如議行之。

三十年,保定巡撫都御史汪應蛟言:「易水可溉金臺,滹水可溉恒山,溏水可溉中山,滏水可溉襄國,漳水可溉鄴下,而瀛海當眾河下流,故號河中,視江南澤國不異。至於山下之泉,地中之水,所在皆有,宜各設壩建閘,通渠築堤,高者自灌,下則車汲。用南方水田法,六郡之內,得水田數萬頃,畿民從此饒,永無旱澇之患。不幸濱河有梗,亦可改折於南,取糴於北。此國家無窮利也。」報可。應蛟乃於天津葛沽、何家圈、雙溝、白塘,令防海軍丁屯種,人授田四畝,共種五千餘畝,水稻二千畝,收多,因上言:「墾地七千頃,歲可得穀二百餘萬石,此行之而效者也。」

是年,真定知府郭勉濬大鳴、小鳴泉四十餘穴,溉田千頃。邢臺達活、野狐二泉流為牛尾河,百泉流為澧河,建二十一閘二堤,灌田五百餘頃。

天啟元年,御史左光斗用應蛟策,復天津屯田,令通判盧觀象管理屯田水利。明年,巡按御史張慎言言:「自枝河而西,靜海、興濟之間,萬頃沃壤。河之東,尚有鹽水沽等處為膏腴之田,惜皆蕪廢。今觀象開寇家口以南田三千餘畝,溝洫蘆塘之法,種植疏浚之方,皆具而有法,人何憚而不為?大抵開種之法有五:一官種。謂牛、種、器具、耕作、雇募皆出於官,而官亦盡收其田之入也。一佃種。謂民願墾而無力,其牛、種、器具仰給於官,待納稼之時,官十而取其四也。一民種。佃之有力者,自認開墾若干,迨開荒既熟,較數歲之中以為常,十一而取是也。一軍種。即令海防營軍種葛沽之田,人耕四畝,收二石,緣有行、月糧,故收租重也。一屯種。祖宗衛軍有屯田,或五十畝,或百畝。軍為屯種者,歲入十七於官,即以所入為官軍歲支之用。國初兵農之善制也。四法已行,惟屯種則今日兵與軍分,而屯僅存其名。當選各衛之屯餘,墾津門之沃土,如官種法行之。」章下所司,命太僕卿董應舉管天津至山海屯田,規畫數年,開田十八萬畝,積穀無算。

崇禎二年,兵部侍郎申用懋言:「永平濼河諸水,逶迤寬衍,可疏渠以防旱潦。山坡隙地,便栽種。宜令有司相地察源,為民興利。」從之。

