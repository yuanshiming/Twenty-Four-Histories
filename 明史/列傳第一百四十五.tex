\article{列傳第一百四十五}

\begin{pinyinscope}
張鶴鳴弟鶴騰董漢儒汪泗論趙彥王洽王在晉高第梁廷棟熊明遇張鳳翼陳新甲馮元飆兄元颺

張鶴鳴,字元平,潁州人。中萬曆十四年會試,父病,馳歸。越六年,始成進士。除歷城知縣,移南京兵部主事。累官陜西右參政,分巡臨、鞏,以才略聞。

再遷右僉都御史,巡撫貴州。自楊應龍平後,銷兵太多,苗仲所在為寇。鶴鳴言:「仲賊乃粵西瑤種,流入黔中。自貴陽抵滇,人以三萬計,砦以千四百七十計,分即為民,合即為盜。又有紅苗,環銅仁、石阡、思州、思南四郡,數幾十萬,而鎮遠、清平間,大江、小江、九股諸種,皆應龍遺孽,眾萬餘。臣部卒止萬三千,何以禦賊?」因列上增兵增餉九議。合諸土兵剿洪邊十二馬頭,大破紅苗,追剿猱坪。賊首老蠟雞據峰巔仰天窩,窩有九井,地平衍,容數千人,下通三道,各列三關,老蠟雞僭王號。鶴鳴奪其關,老蠟雞授首,撫降餘眾而還。尋發兵擊平定廣、威平、安籠諸賊,威名甚著。遷兵部右侍郎,總督陜西三邊軍務。未上,轉左侍郎,佐理部事。時兵事亟,兵部增設二侍郎,而鶴鳴與祁伯裕、王在晉並臥家園不赴。

至天啟元年,遼陽破,兵事益亟。右侍郎張經世督援師出關,部中遂無侍郎。言官請趣鶴鳴等,章數十上,帝乃剋期令兵部馬上督催,鶴鳴等始履任。至則論平苗功,進本部尚書,視侍郎事。尚書王象乾出督薊、遼軍務,鶴鳴遂代其位。給事中韋蕃請留象乾,出鶴鳴督師。忤旨,謫外。時熊廷弼經略遼東,性剛負氣,好謾罵,凌轢朝士。鶴鳴與相失,事多齟齬,獨喜巡撫王化貞。化貞本庸才,好大言,鶴鳴主之,所奏請無不從,令無受廷弼節度。中外皆知經、撫不和,必誤封疆,而鶴鳴信化貞愈篤,卒致疆事大壞。

二年正月,廷議經、撫去留。給事中惠世揚、周朝瑞議以鶴鳴代廷弼,其他多言經、撫宜並任,鶴鳴獨毅然主撤廷弼,專任化貞。議甫上,化貞已棄廣寧遁。鶴鳴內慚,且懼罪,乃自請行邊,詔加太子太保,賜蟒玉及尚方劍。鶴鳴憚行,逗遛十七日,始抵山海關。至則無所籌畫,日下令捕間諜,厚啖蒙古炒花、宰賽諸部而已。

初,廣寧敗書聞,廷臣集議兵事。鶴鳴盛氣詈廷弼自解。給事中劉弘化首論之,坐奪俸。御史江秉謙、何薦可繼劾,並貶官。廷臣益憤。御史謝文錦,給事中惠世揚、周朝瑞、蕭良佐、侯震暘、熊德陽等交章極論,請用世宗戮丁汝夔、神宗逮石星故事,與化貞並按。鶴鳴抵言廷弼僨疆事,由故大學士劉一燝、尚書周嘉謨黨庇不令出關所致,因詆言者為一燝鷹犬。且曰:「祖宗故事,大司馬不以封疆蒙功罪。」於是朝瑞等復合疏劾之,御史周宗文亦列其八罪。帝不問。鶴鳴遷延數月,謝病歸。

六年春,魏忠賢勢大熾,起鶴鳴南京工部尚書。尋以安邦彥未滅,鶴鳴先有平苗功,改兵部尚書,總督貴州、四川、雲南、湖廣、廣西軍務,賜尚方劍。功未就,莊烈帝嗣位。給事中瞿式耜、胡永順、萬鵬以鶴鳴由忠賢進,連章擊之。鶴鳴求去,詔加太子太師,乘傳歸。崇禎八年,流賊陷潁州,執鶴鳴,例懸於樹,罵賊死,年八十五。

弟鶴騰,字元漢,舉萬歷二十三年進士。歷官雲南副使。行誼醇篤,譽過其兄。城陷被執,罵不絕口而死。

董漢儒,開州人。萬曆十七年進士。授河南府推官,入為戶部主事。疏陳減織造、裁冒濫諸事。且曰:「邇來九閽三殿間,惟聞縱酒、淫刑、黷貨。時事可憂,不止國計日絀已也。」不報。朝鮮再用兵,以郎中出理餉務。

尋遷山東僉事,進副使,歷湖廣左右布政使,所在有聲。四十年,就拜右副都御史,巡撫其地。帝賜福王莊田,責湖廣四千四百餘頃,漢儒以無所得田,請歲輸萬金代租,不聽。楚宗五十餘人,訐假王事獲罪,囚十載,漢儒力言,王,假也,請釋繫者。又為滿朝薦、卞孔時等乞宥。俱不報。憂歸。

光宗立,召拜工部右侍郎。旋改兵部,總督宣府、大同、山西軍務。天啟改元,遼陽失,簡精卒二千入衛,詔褒之。明年秋,以左侍郎協理戎政。未上,擢兵部尚書。時遼地盡亡,漢儒請逮治諸降將劉世勛等二十九人家屬,立誅逃將蔡汝賢等,報可。毛文龍居海外,屢以虛言誑中朝,登萊巡撫袁可立每代為奏請。漢儒言文龍計畫疏,虛聲未可長恃;又請誅逃將管大籓、張思任、孟淑孔等,語甚切。帝命逮治思任等,而大籓卒置不問。諸鎮援遼軍多逃逸,有出塞投插部者,漢儒請捕獲立誅,同伍相擒捕者重賞;且給餉以時,則逃者自少。帝亦嘉納。

奄人王體乾、宋晉、魏忠賢等十二人有舊勞,命所蔭錦衣官皆予世襲。漢儒據祖制力爭,帝不從。給事中程註、御史汪泗論等合疏諫,給事中朱大典、周之綱,御史宋師襄、胡良機特疏繼之,卒不納。漢儒旋以母喪歸。後忠賢大橫,漢儒服闋,遂不召。追敘甘肅功,即家進太子太保,蔭子錦衣百戶。卒贈少保,謚肅敏。

汪泗論,字自魯,休寧人。祖自,嘉靖中進士,歷官福建兵備僉事,分守福寧。倭犯同安,自釋重囚七人為軍鋒,擊倭卻之。捷聞,賚金幣。

泗論中萬曆三十八年進士。授漳浦知縣,調福清,有惠政,清屯田,繕城堡。征擢御史,首請杜內批以嚴履霜之漸,又請召還科臣楊漣等以作士氣。巡按江西,敦重持大體,奸宄肅然。宗人祿不給,疏以橋稅贖鍰存留接濟。歷太僕寺少卿。嘗識黃道周於諸生中,人服其精鑒。

趙彥,膚施人。萬曆十一年進士。授行人,屢遷山西左布政使。光宗嗣位,以右僉都御史巡撫山東。遼陽既失,彥請增兵戍諸島,特設大將登州,登、萊設鎮,自此始。天啟二年,廣寧復失。彥以山東南北咽喉,列上八事,詔多允行。

先是,薊州人王森得妖狐異香,倡白蓮教,自稱聞香教主。其徒有大小傳頭及會主諸號,蔓延畿輔、山東、山西、河南、陜西、四川。森居灤州石佛莊,徒黨輸金錢稱朝貢,飛竹籌報機事,一日數百里。萬曆二十三年,有司捕繫森,論死,用賄得釋。乃入京師,結外戚中官,行教自如。後森徒李國用別立教,用符咒召鬼。兩教相仇,事盡露。四十二年,森復為有司所攝,越五歲,斃於獄。其子好賢及鉅野徐鴻儒、武邑于弘志輩踵其教,徒黨益眾。至是,好賢見遼東盡失,四方奸民思逞,與鴻儒等約是年中秋並起兵。會謀洩,鴻儒遂先期反,自號中興福烈帝,稱大成興勝元年,用紅巾為識。五月戊申陷鄆城,俄陷鄒、滕、嶧,眾至數萬。

時承平久,郡縣無守備,山東故不置重兵。彥任都司楊國棟、廖棟,而檄所部練民兵,增諸要地守卒。請留京操班軍及廣東援遼軍,以備徵調。薦起故大同總兵官楊肇基為山東總兵官,討賊。賊乘肇基未至,襲兗州,為滋陽知縣楊炳所卻。棟等擊敗賊,復鄆城。其別部犯鉅野,知縣趙延慶固守不下,國棟兵至,敗之,又敗其犯兗州者。遂偕棟等合攻鄒縣。兵潰,遊擊張榜戰死,賊遂圍曲阜、郯城。旋敗去,遂復嶧縣。

七月,彥視師兗州。甫出城,遇賊萬餘,彥縋入城。肇基急迎戰,而令國棟及棟夾擊,大敗之橫河。時賊精銳聚鄒、滕中道,彥欲攻鄒、滕。副使徐從治曰:「攻鄒、滕難下,不如搗其中堅,兩城可圖也。」彥乃與肇基令遊兵綴賊鄒城,而以大軍擊賊精銳於黃陰、紀王城,大敗賊,蹙而殪之嶧山,遂圍鄒。大小數十戰,城未下,令天津僉事來斯行及國棟等乘間復滕縣。國棟又大破賊沙河,乃築長圍以攻鄒。鴻儒抗守三月,食盡,賊黨盡出降;鴻儒單騎走,被擒。撫其眾四萬七千餘人。彥乃紀績,告廟獻俘,磔鴻儒於市。鴻儒躪山東二十年,徒黨不下二百萬,至是始伏誅。

于弘志亦於是年六月據武邑白家屯,將取景州應鴻儒。斯行方赴援山東,還軍討之。弘志突圍走,為諸生葉廷珍所獲,凡舉事七日而滅。好賢亦捕得伏誅。

彥已加兵部侍郎,論功,進尚書兼右副都御史,再加太子太保,廕子錦衣世僉事,賚銀幣加等。奏請振濟,且捐鄒、滕賦三年,鄆城、嶧、滋陽、曲阜一年,鉅野半之,皆報許。

三年八月,召代董漢儒為兵部尚書,極陳邊將剋餉、役軍、虛伍、占馬諸弊,因條列綜核事宜。帝稱善,立下諸邊舉行。參將王楹行邊,為哈剌慎部襲殺,彥請核實論罪,并敕諸邊撫賞毋增故額。有傳我大清兵欲入喜峰口者,彥憂之,畫上八事,帝皆褒納。楊漣劾魏忠賢二十四罪,彥亦抗疏劾之,自是為忠賢所惡。貴州征苗兵屢敗,彥列八策以獻,詔頒示軍中。

彥有籌略,曉暢兵事。然征妖賊時,諸將多殺良民冒功,而其子官錦衣,頗招搖都市。給事、御史交劾之。彥三疏乞罷,忠賢挾前憾,令乘傳歸,子削籍。初,妖賊興,遼東經略王在晉遣兵助討,彥敘功不及在晉,在晉憾之,至是為南京吏部,數詆彥。給事中袁玉佩遂劾彥冒功濫廕,且言京觀不當築。詔削其世廕,並京觀毀之。尋追敘兵部時邊功,即家進太子太傅。未幾卒。

王洽,字和仲,臨邑人。萬曆三十二年進士。歷知東光、任丘。服闋,補長垣。治儀表頎偉,危坐堂上,吏民望之若神明。其廉能為一方最。

擢吏部稽勳主事,歷考功文選郎中。天啟初,諸賢匯進,洽有力焉。遷太常少卿。三年冬,以右僉都御史巡撫浙江。洽本趙南星所引,及魏忠賢逐南星,洽乞罷,不許。五年四月,御史李應公希忠賢指劾洽,遂奪職閒住。

崇禎元年,召拜工部右侍郎,攝部事。兵部尚書王在晉罷,帝召見群臣,奇洽狀貌,即擢任之。上疏陳軍政十事,曰嚴債帥,修武備,核實兵,衡將材,核欺蔽,懲朘削,勤訓練,釐積蠹,舉異才,弭盜賊,帝並褒納。宣大總督王象乾與大同巡撫張宗衡爭插漢款戰事,帝召諸大臣平臺,詰問良久,洽及諸執政並主象乾策,定款議,詳見《象乾》、《宗衡傳》。

尋上言:「祖宗養兵百萬,不費朝廷一錢,屯田是也。今遼東、永平、天津、登、萊沿海荒地,及寶坻、香河、豐潤、玉田、三河、順義諸縣閒田百萬頃。元虞集有京東水田之議,本朝萬曆初,總督張佳允、巡撫張國彥行之薊鎮,為豪右所阻。其後,巡撫汪應蛟復行之河間。今已墾者荒,未墾者置不問,遺天施地生之利,而日講生財之術,為養軍資,不大失策乎!乞敕諸道監司,遵先朝七分防操、三分屯墾之制,實心力行,庶國計有裨,軍食無缺。」帝稱善,即命行之。嘗奏汰年深武弁無薦者四十八人,以邊才舉監司楊嗣昌、梁廷棟,後皆大用。

二年十月,我大清兵由大安口入,都城戒嚴。洽急征四方兵入衛,督師袁崇煥,巡撫解經傳、郭之琮,總兵官祖大壽、趙率教、滿桂、侯世祿、尤世威、曹鳴雷等先後至,不能拒,大清兵遂深入。帝憂甚,十一月召對廷臣。侍郎周延儒言:「本兵備禦疏忽,調度乖張。」檢討項煜繼之,且曰:「世宗斬一丁汝夔,將士震悚,強敵宵遁。」帝頷之,遂下洽獄,以左侍郎申用懋代。明年四月,洽竟瘐死。尋論罪,復坐大辟。

洽清修伉直,雅負時望,而應變非所長。驟逢大故,以時艱見絀。遵化陷,再日始得報。帝怒其偵探不明,又以廷臣玩祇,擬用重典,故於洽不少貸。厥後都城復三被兵,樞臣咸獲免,人多為洽惜之。

在晉,字明初,太倉人。萬曆二十年進士。授中書舍人。自部曹歷監司,由江西布政使擢巡撫山東右副都御史,進督河道。泰昌時,遷添設兵部左侍郎。天啟二年署部事。三月,遷兵部尚書兼右副都御史,經略遼東、薊鎮、天津、登萊,代熊廷弼。八月,改南京兵部尚書,尋請告歸。五年,起南京吏部尚書,尋就改兵部。崇禎元年,召為刑部尚書,未幾,遷兵部。坐張慶臻改敕書事,削籍歸,卒。

高第,字登之,灤州人。萬曆十七年進士。歷官兵部尚書,經略薊、遼。未數月,以恇怯劾罷去。崇禎二年冬,大清兵破灤州,第竄免。

梁廷棟,鄢陵人。父克從,太常少卿。廷棟舉萬曆四十七年進士。授南京兵部主事,召改禮部,歷儀制郎中。天啟五年,遷撫治西寧參議。七年,調永平兵備副使。督撫以下為魏忠賢建祠,廷棟獨不往,乞終養歸。

崇禎元年起故官,分巡口北道。明年加右參政。十一月,大清兵克遵化,巡撫王元雅自縊,即擢廷棟右僉都御史代之。廷棟請賜對,面陳方略,報可。未幾,督師袁崇煥下獄,復擢廷棟兵部右侍郎兼故官,總督薊、遼、保定軍務及四方援軍。廷棟有才知兵,奏對明爽,帝心異之。

三年正月,兵部尚書申用懋罷,特召廷棟掌部事。時京師雖解嚴,羽書旁午,廷棟剖決無滯。而廷臣見其驟用,心嫉之。給事中陳良訓首刺廷棟,同官陶崇道復言:「廷棟數月前一監司耳,倏而為巡撫、總督、本兵,國士之遇宜何如報。乃在通州時,言遵、永易復,良、固難破,自以為神算。今何以難者易,易者難?且嘗請躬履行間,隨敵追擊,以為此報主熱血。今偃然中樞,熱血何銷亡也?謂制敵不專在戰,似矣,而伐謀用間,其計安在?」帝不聽崇道言。廷棟疏辨,乞一巖疆自效,優詔慰留之。未幾,工部主事李逢申劾廷棟虛名,崇道又言廷棟輕於發言,致臨洮、固原入衛兵變。帝皆不納。五月,永平四城復,賞廷棟調度功,加太子少保,世蔭錦衣僉事。

其秋,廷棟以兵食不足,將加賦,因言:「今日閭左雖窮,然不窮於遼餉也。一歲中,陰為加派者,不知其數。如朝覲、考滿、行取、推升,少者費五六千金,合海內計之,國家選一番守令,天下加派數百萬。巡按查盤、訪緝、餽遺、謝薦,多者至二三萬金,合天下計之,國家遣一番巡方,天下加派百餘萬,而曰民窮於遼餉,何也?臣考九邊額設兵餉,兵不過五十萬,餉不過千五百三十餘萬,何憂不足。故今日民窮之故,惟在官貪。使貪風不除,即不加派,民愁苦自若;使貪風一息,即再加派,民懽忻亦自若。」疏入,帝俞其言,下戶部協議。戶部尚書畢自嚴阿廷棟意,即言今日之策,無踰加賦,請畝加九釐之外,再增三釐。於是增賦百六十五萬有奇,海內並咨怨。已,陳釐弊五事:曰屯田,曰鹽法,曰錢法,曰茶馬,曰積粟。又極陳陜西致寇之由,請重懲將吏貪汙者以紓軍民之憤,塞叛亂之源。帝皆褒納。

廷棟居中樞歲餘,所陳兵事多中機宜,帝甚倚任。然頗挾數行私,不為朝論所重。給事中葛應斗劾御史袁弘勛納參將胡宗明金,請囑兵部;廷棟亦劾弘勛及錦衣張道濬通賄狀。兩人遂下獄。兩人者,吏部尚書王永光私人也。廷棟謀并去永光,以己代之,得釋兵事,永光遂由此去。御史水佳允者,弘勛郡人也,兩疏力攻廷棟,發其所與司官手書,且言其縱奸人沈敏交關薊撫劉可訓,納賄營私。廷棟疏辯求去,帝猶慰留。有安國棟者,初以通判主插漢撫賞事,廷棟薦其才,特擢職方主事,仍主撫賞,頗為奸利,廷棟庇之。後佳允坐他事左遷行人司副,復上疏發兩人交通狀,并列其賄鬻將領數事,事俱有迹。廷棟危甚,賴中人左右之,得閒住去,以熊明遇代。八年冬,召拜兵部右侍郎兼右都御史,代楊嗣昌總督宣、大、山西軍務。明年七月,我大清兵由間道踰天壽山,克昌平,逼京師。山後地,乃廷棟所轄也,命戴罪入援。兵部尚書張鳳翼懼罪,自請督師。兩人心匡怯不敢戰,近畿地多殘破,言官交章論劾。兩人益懼,度解嚴後必罹重譴,日服大黃藥求死。八月十九日,大清兵出塞。至九月朔,鳳翼卒。踰旬日,廷棟亦卒。已,法司定罪,廷棟坐大辟,以既死不究云。

廷棟既歿,其父克從尚在。後賊破鄢陵,避開封。及開封被淹,死於水。

熊明遇,字良孺,進賢人。萬曆二十九年進士。知長興縣。四十三年,擢兵科給事中,旋掌科事。上疏極陳時弊,言:

今春以來,天鼓兩震於晉地,流星晝隕於清豐,地震二十八,天火九,石首雨菽,河內女妖,遼東兵端吐火,即春秋二百四十年間,未有稠於今日者。且山東大昆,人相食,黃河水稽天,兼以太白經天,輔星湛沒,熒惑襲月,金水愆行,或日光無芒,日月同暈,為恒風,為枯旱。天譴愈深,而陛下所行皆誣天拂經之事,此誠禽息碎首、賈生痛哭之時也。敢以八憂、五漸、三無之說進。

今內庫太實,外庫太虛,可憂一。餉臣乏餉,邊臣開邊,可憂二。套部圖王,插部覬賞,可憂三。黃河泛濫,運河膠淤,可憂四。齊苦荒天,楚苦索地,可憂五。鼎鉉不備,棟梁常撓,可憂六。群嘩盈衢,訛言載道,可憂七。吳民喜亂,冠履倒置,可憂八。

八憂未已,五漸繼之。太阿之柄,漸入中涓。魁壘之人,漸如隕籜。制科之法,漸成奸藪。武庫之器,漸見銷亡。商旅之途,漸至梗塞。

五漸未已,三無繼之。匹夫可熒惑天子,小校可濫邀絲綸,是朝廷無紀綱。滇、黔之守令皆途窮,揚、粵之監司多規避,是遠方無吏治。讒構之口甚於戈戟,傾危之禍慘於蘇、張,是士大夫無人心。天下事可不寒心哉!

帝不省。亓詩教等以明遇與東林通,出為福建僉事,遷寧夏參議。

天啟元年,以尚寶少卿進太僕少卿,尋擢南京右僉都御史,提督操江。建營伏虎山,選練蒼頭軍,以資守禦。永樂中,齊王榑以罪廢,其子孫居南京,號齊庶人。有睿爁者,自負異表,與奸人謀不軌,明遇捕獲之,置其黨十餘人於法。魏忠賢黨謀盡逐東林,以明遇嘗救御史游士任,五年三月,給事中薛國觀遂劾其黨庇徇私,忠賢即矯旨革職。未幾,坐汪文言獄,追贓千二百金,謫戍貴州平溪衛。

莊烈帝即位,釋還。崇禎元年,起兵部右侍郎。明年進左,遷南京刑部尚書。四年,召拜兵部尚書,疏陳四司宿弊,悉見採納。楊鶴被逮,明遇言:「秦中流寇,明旨許撫剿並行。臣謂渠魁乞降亦宜撫,脅從負固亦宜剿。今鶴以撫賊無功就逮,倘諸臣因鶴故欲盡戮無辜,被脅之人絕其生路。宜急敕新督臣洪承疇,諭賊黨殺賊自效,即神一魁、劉金輩,果立奇功,亦一體敘錄。而諸將善撫馭如吳弘器等,仍與升擢,庶賊黨日孤。」帝亦納之。

五年正月,山東叛將李九成等陷登州,明遇過信巡撫余大成言,力主撫議,久愈猖獗,萊城被圍幾陷,乃調關外軍討定之。語詳《徐從治傳》。當是時,我大清兵入宣府,巡撫沈棨與中官王坤等遣使議和,饋金帛牢醴,師乃旋。事聞,帝惡棨專擅,召對明遇等於平臺。明遇曲為棨解,帝不悅,逮棨下吏。於是給事中孫三傑力詆明遇、棨交關誤國,同官陳贊化、呂黃鐘,御史趙繼鼎連劾之。明遇再疏乞罷,帝責以疏庸僨事,命解任候勘。尋以故官致仕。久之,用薦起南京兵部尚書,改工部,引疾歸。國變後卒。

張鳳翼,代州人。萬曆四十一年進士。授戶部主事。歷廣寧兵備副使,憂歸。

天啟初,起右參政,飭遵化兵備。三年五月,遼東巡撫閻鳴泰罷,擢鳳翼右僉都御史代之。自王化貞棄廣寧後,關外八城盡空,樞輔孫承宗銳意修復,而版築未興。鳳翼聞命,疑承宗欲還朝,以遼事委之己,甚懼,即疏請專守關門。其座主葉向高、鄉人韓爌柄政,抑使弗上。既抵關,以八月出閱前屯、寧遠諸城,上疏極頌承宗經理功,且曰:「八城畚插,非一年可就之工;六載瘡痍,非一時可起之疾。今日議剿不能,言戰不得,計惟固守。當以山海為根基,寧遠為門戶,廣寧為哨探。」其意專主守關,與承宗異議。

時趙率教駐前屯,墾田、練卒有成效。及袁崇煥、滿桂守寧遠,關外規模略定。忽有傳中左所被兵者,永平吏民洶洶思竄,鳳翼心動,亟遣妻子西歸。承宗曰:「我不出關,人心不定。」遂於四年正月東行。鳳翼語人曰:「樞輔欲以寧前荒塞居我,是殺我也。國家即棄遼左,猶不失全盛,如大寧、河套,棄之何害?今舉世不欲復遼,彼一人獨欲復耶?」密令所知居言路者詆馬世龍貪淫及三大將建閫之非,以撼承宗。承宗不悅,舉其言入告。適鳳翼遭內艱,遂解去。承宗復上疏為世龍等辨,因詆鳳翼才鄙而怯,識闇而狡,工於趨利,巧於避患。廷議以既去不復問。

六年秋,起故官,巡撫保定。明年冬,薊遼總督劉詔罷,進鳳翼右都御史兼兵部右侍郎代之。崇禎元年二月,御史甯光先劾鳳翼前撫保定,建魏忠賢生祠。鳳翼引罪乞罷,不許。未幾,謝病去。諸建祠者俱入逆案,鳳翼以邊臣故獲宥。

三年起故官,代劉策總督薊、遼、保定軍務。既復遵、永四城,敘功,進太子少保、兵部尚書,世蔭錦衣僉事。鳳翼以西協單弱,條奏增良將、宿重兵、備火器、預軍儲、遠哨探數事,從之。已,復謝病去。久之,召為兵部尚書。

明年二月,召對平臺,與吏部尚書李長庚同奉「為國任事,潔己率屬」之諭。尋以宣、大兵寡,上言:「國初額軍,宣府十五萬一千,今止六萬七千。大同十三萬五千,今止七萬五千。乞兩鎮各增募萬人,分營訓練。且月餉止給五錢,安能致赳桓之士,乞一人食二餉。」帝並從之。給事中周純修、御史葛征奇等以兵事日棘,劾鳳翼溺職。鳳翼連疏乞休,皆不許。

七年以恢復登州功,加太子少保。七月,我大清西征插漢,師旋,入山西、大同、宣府境。帝怒守臣失機,下兵部論罪。部議巡撫戴君恩、胡沾恩、焦源清革職贖杖,總督張宗衡閒住。帝以為輕,責鳳翼對狀。於是總督、巡撫及三鎮總兵睦自強、曹文詔、張全昌俱遣戍,監視中官劉允中、劉文中、王坤亦充凈軍。時討賊總督陳奇瑜以招撫僨事,給事中顧國寶劾鳳翼舉用非人,帝亦不問。奇瑜既罷,即命三邊總督洪承疇兼督河南、山西、湖廣軍務,剿中原群盜。言官以承疇勢難兼顧,請別遣一人為總督,鳳翼不能決,既而承疇竟無功。及賊將南犯,請以江北巡撫楊一鵬鎮鳳陽,防護皇陵,溫體仁不聽,鳳翼亦不能再請。八年正月,賊果毀鳳陽皇陵。言官交章劾鳳翼,鳳翼亦自危,引罪乞罷。帝不許,令戴罪視事。

初,賊之犯江北也,給事中桐城孫晉以鄉里為憂。鳳翼曰:「公南人,何憂賊?賊起西北,不食稻米,賊馬不飼江南草。」聞者笑之。事益急,始令朱大典鎮鳳陽。尋推盧象昇為總理,與洪承疇分討南北賊,而賊已蔓延不可制矣。給事中劉昌劾鳳翼推總兵陳壯猷,納其重賄。鳳翼力辯,昌貶秩調外。

已而鳳翼言:「剿賊之役,原議集兵七萬二千,隨賊所向,以殄滅為期。督臣承疇以三萬人分布豫、楚數千里,力薄,又久戍生疾,故尤世威、徐來朝俱潰。以二萬人散布三秦千里內,勢分,又孤軍無援,故艾萬年、曹文詔俱敗。今既益以祖寬、李重鎮、倪寵、牟文綬兵萬二千,又募楚兵七千,合九萬有奇,兵力厚矣。請以賊在關內者屬承疇,在關外者屬象昇,倘賊盡出關,則承疇合剿於豫,盡入關,則象昇合剿於秦。臣更有慮者,賊號三四十萬,更迭出犯,勢眾而力合;我零星四應,勢寡而力分。賊所至因糧於我,人皆宿飽;我所至樵蘇後爨,動輒呼庚。賊馬多行疾,一二日而十舍可至;我步多行緩,三日而重繭難馳。眾寡、饑飽、勞逸之勢,相懸如此,賊何日平。乞嚴敕督、理二臣,選將統軍,軍各一二萬人,俾前茅、後勁、中權聯絡相貫,然後可制賊而不為賊制。今賊大勢東行,北有黃河,南有長江,東有漕渠,彼無舟楫,豈能飛越?我兵從西北窮追,猶易為力。此防河扼險,目前要策,所當申飭者也。」帝稱善,命速行之。鳳翼自請督師討賊,帝優詔不允。

九年二月,給事中陳昌文上言:「將在軍,君命有所不受。今既假督、理二臣以便宜,則行軍機要不當中制。若今日議不許斬級,明日又議必斬級,今日議徵兵援鳳,明日又議撤兵防河,心至無所適從。願樞臣自今凡可掣督、撫之肘者,俱寬之文法,俾得展布可也。兵法:守敵所不攻,攻敵所不守,奇正錯出,滅賊何難。今不惟不能滅,乃今日破軍殺將,明日又陷邑殘州,止罪守令而不及巡撫,豈法之平?願樞臣自今凡可責諸撫之成者,勿寬文法,俾加磨礪可也。」帝納其言。

江北之賊,自滁州、歸德兩敗後,盡趨永寧、盧氏、內鄉、淅川大山中,關中賊亦由閿鄉、靈寶與之合。鳳翼請敕河南、鄖陽、陜西三巡撫各督將吏扼防,毋使軼出,四川、湖廣兩巡撫移師近界,聽援剿,而督、理二臣以大軍入山蹙之,且嚴遏米商通販,賊可盡殄。帝深然之,剋期五月蕩平,老師費財,督撫以下罪無赦。鳳翼雖建此策,象昇所部多騎軍,不善入山,賊竟不能滅。

至七月,我大清兵自天壽山後入昌平,都城戒嚴。給事中王家彥以陵寢震驚,劾鳳翼坐視不救。鳳翼懼,自請督師。賜尚方劍,盡督諸鎮勤王兵。以左侍郎王業浩署部事,命中官羅維寧監督通、津、臨、德軍務,而宣大總督梁廷棟亦統兵入援。三人相掎角,皆退怯不敢戰,於是寶坻、順義、文安、永清、雄、安肅、定興諸縣及安州、定州相繼失守。言官劾疏五六上,鳳翼憂甚。

己巳之變,尚書王洽下獄死,復坐大辟。鳳翼知不免,日服大黃藥,病已殆,猶治軍書不休。至八月末,都城解嚴,鳳翼即以九月朔卒。已而議罪奪其官。十一年七月,論前剿寇功,有詔敘復。

帝在位十七年間,易中樞十四人,皆不久獲罪。鳳翼善溫體仁,獨居位五載。其督師也,意圖逭責,乃竟以畏法死。

陳新甲,長壽人。萬曆時舉於鄉,為定州知州。崇禎元年,入為刑部員外郎,進郎中。遷寧前兵備僉事。寧前,關外要地,新甲以才能著。四年,大凌新城被圍,援師雲集,征繕悉倚賴焉。及城破,坐削籍。巡撫方一藻惜其才,請留之,未報。監視中官馬雲程亦以為言,乃報可。新甲言:「臣蒙使過之恩,由監視疏下,此心未白,清議隨之,不敢受。」不許。尋進副使,仍蒞寧遠。

七年九月,擢右僉都御史,代焦源清巡撫宣府。新甲以戎備久弛,親歷塞垣,經前人足跡所不到,具得士馬損耗、城堡傾頹、弓矢甲仗朽敝狀。屢疏請於朝,加整飭,邊防賴之。楊嗣昌為總督,與新甲共事,以是知其才。九年五月,內艱歸。

十一年六月,宣大總督盧象昇丁外艱,嗣昌方任中樞,薦新甲堪代。詔擢兵部右侍郎兼右僉都御史,奪情任之。會大清兵深入內地,詔新甲受代,即督所部兵協御。未幾,象昇戰歿,孫傳庭代統其軍,新甲與相倚仗,終不敢戰。明年春,畿輔解嚴。順天巡按劉呈瑞劾其前後逗撓。新甲歷陳功狀,且言呈瑞挾仇,帝不問。既赴鎮,列上編隊伍、嚴哨探、明訓練、飭馬政、練火器、禁侵漁諸事,報可。麾下卒夜嘩,新甲請罪,亦不問。給事中戴明說嘗劾之,帝以輕議重臣,停其俸。

十三年正月,召代傅宗龍為兵部尚書。自弘治初賈俊後,乙榜無至尚書者。兵事方亟,諸大臣避中樞,故新甲得為之。陛見畢,陳保邦十策,多廷臣所嘗言。惟言天壽山後宜設總兵,徐州亦宜設重鎮,通兩京咽喉,南護鳳陵,中防漕運,帝並採用之。復陳樞政四要及兵事四失,帝即命飭行。

十四年三月,賊陷雒陽、襄陽,福、襄二王被難,鐫新甲三秩視事。舊制,府、州、縣城郭失守者,長吏論死。宛平知縣陳景建言村鎮焚掠三所者,長吏當戍邊。新甲主其議,言:「有司能兼顧鄉城,即與優敘。若四郊被寇,與失機並論。」帝即從之。然是時中原皆盜,其法亦不能行也。楊嗣昌卒於軍中,新甲舉丁啟睿往代,議者尤其失人。然傅宗龍、孫傳庭並以微罪繫獄,新甲於召對時稱其才,退復上章力薦,兩人獲用,亦新甲力也。尋論秋防功,復所鐫秩。

時錦州被圍久,聲援斷絕。有卒逸出,傳祖大壽語,請以車營逼,毋輕戰。總督洪承疇集兵數萬援之,亦未敢決戰。帝召新甲問策,新甲請與閣臣及侍郎吳甡計之,因陳十可憂、十可議,而遣職方郎張若麒面商於承疇。若麒未返,新甲請分四道夾攻,承疇以兵分力弱,意主持重以待。帝以為然,而新甲堅執前議。若麒素狂躁,見諸軍稍有斬獲,謂圍可立解,密奏上聞。新甲復貽書趣承疇,承疇激新甲言,又奉密敕,遂不敢主前議。若麒益趣諸將進兵。諸將以八月次松山,為我大清兵所破,大潰,士卒死亡數萬人。若麒自海道遁還,言官請罪之,新甲力庇,復令出關監軍。錦州圍未解,承疇又被圍於松山,帝深以為憂,新甲不能救。十五年二月,御史甘惟爃劾新甲寡謀誤國,請速令舉賢自代,不納。三月,松山、錦州相繼失,若麒復自寧遠遁還。言官劾若麒者,悉及新甲。新甲屢乞罷,皆不從。

新甲雅有才,曉邊事,然不能持廉,所用多債帥。深結中貴為援,與司禮王德化尤暱,故言路攻之不能入。當是時,闖賊蹂躪河南,開封屢被圍,他郡縣失亡相踵,總督傅宗龍、汪喬年出關討賊,先後陷歿,賊勢愈張。言官劾新甲者,章至數十。新甲請罪章亦十餘上,帝輒慰留。

初,新甲以南北交困,遣使與大清議和,私言於傅宗龍。宗龍出都日,以語大學士謝升。升後見疆事大壞,述宗龍之言於帝。帝召新甲詰責,新甲叩頭謝罪。升進曰:「倘肯議和,和亦可恃。」帝默然,尋諭新甲密圖之,而外廷不知也。已,言官謁升。升言:「上意主和,諸君幸勿多言。」言官駭愕,交章劾升,升遂斥去。帝既以和議委新甲,手詔往返者數十,皆戒以勿洩。外廷漸知之,故屢疏爭,然不得左驗。一日,所遣職方郎馬紹愉以密語報,新甲視之置几上。其家僮誤以為塘報也,付之抄傳,於是言路嘩然。給事中方士亮首論之,帝慍甚,留疏不下。已,降嚴旨,切責新甲,令自陳。新甲不引罪,反自詡其功,帝益怒。至七月,給事中馬嘉植復劾之,遂下獄。新甲從獄中上書乞宥,不許。新甲知不免,遍行金內外。給事中廖國遴、楊枝起等營救於刑部侍郎徐石麒,拒不聽。大學士周延儒、陳演亦於帝前力救,且曰:「國法,敵兵不薄城不殺大司馬。」帝曰:「他且勿論,戮辱我親籓七,不甚於薄城耶?」遂棄新甲於市。

新甲為楊嗣昌引用,其才品心術相似,軍書旁午,裁答無滯。旁初甚倚之,晚特惡其洩機事,且彰主過,故殺之不疑。厥後給事中沈迅力詆其失,帝曰:「令爾作新甲,恐更不如。」迅慚而退。新甲初自陽和入都門,黃霧四塞,識者以為不祥,及是果應。

馮元飆,字爾韜,慈谿人。父若愚,南京太僕少卿。天啟元年,元飆與兄元颺同舉於鄉。明年,元飆成進士,歷知澄海、揭陽。

崇禎四年,徵授戶科給事中。帝遣中官出鎮,元飆力爭。時元颺亦疏論中官,兄弟俱有直聲。無何,上疏力詆周延儒,被切責。尋論山東總督劉宇烈縱寇主撫罪。又言禮部侍郎王應熊無大臣體,宜罷。復薦詞臣姚希孟孤忠獨立,不當奪講官;科臣趙東曦正詞讜論,不當奪言路。皆不納。應熊謀改吏部,元飆復摭劾其貪穢數事。被旨譙責,遂乞假歸。

八年春還朝。時鳳陽皇陵毀,廷臣交論溫體仁、王應熊朋比誤國。元飆上言:「政本大臣,居實避名,受功辭罪。平時養威自重,遇天下有事,輒曰:『昭代本無相名,吾儕止供票擬。』上委之聖裁,下委之六部,持片語,叢百欺。夫中外之責,孰大於票擬?有漢、唐宰相之名而更代天言,有國初顧問之榮而兼隆位號,地親勢峻,言聽志行,柄用專且重者莫如今日,猶可謝天下責哉?」遷禮科右給事中,再遷刑科左給事中。數言部囚多輕罪,請帝寬宥,並採納之。詔簡東宮講官,左諭德黃道周為首輔張至發所扼,且疏詆之。元飆言:「道周至清無徒,忠足以動人主,惟不能得執政歡。」至發恚,兩疏詆元飆,帝皆置不問。由戶科都給事中擢太常少卿,改南京太僕卿,就遷通政使。

十五年六月召拜兵部右侍郎,轉左。元飆多智數,尚權譎,與兄元颺並好結納,一時翕然稱「二馮」。然故與馮銓通譜誼。初在言路,詆周延儒,及為侍郎,延儒方再相,元飆因與善。延儒欲以振饑為銓功,復冠帶,憚眾議,元飆令引吳甡入閣助之,既而甡背延儒議。熊開元欲盡發延儒罪,元飆沮止之,開元以是獲重譴。兵部尚書陳新甲棄市,元飆署部事。一日,帝召諸大臣遊西苑,賜宴明德殿,因論兵事。良久,出御馬佳者百餘匹,及內製火箭,次第示元飆,元飆為辨其良楛。帝曰:「大司馬缺久,無踰卿者。」元飆以多病辭,乃用張國維。

十六年五月,國維下獄,遂以元飆為尚書。帝倚之甚至,元飆顧不能有所為。河南、湖廣地盡陷,關、寧又日告警。至八月,以病劇乞休。帝慰留之,賜瓜果食物,遣醫診視。請益堅,乃允其去。

元飆頗能料事。孫傳庭治兵關中,元飆謂不可輕戰。廷臣多言不戰則賊益張,兵久易懦。元飆謂將士習懦,未經行陣,宜致賊而不宜致於賊。乃於帝前爭之曰:「請先下臣獄,俟一戰而勝,斬臣謝之。」又貽書傳庭,戒毋輕鬥,白、高兩將不可任。傳庭果敗。將歸,薦李邦華、史可法自代。帝不用,用兵科都給事中張縉彥,都城遂不守。福王時,元飆卒,其家請恤。給事中吳適言:「元飆身膺特簡,莫展一籌,予以祭葬,是使誤國之臣生死皆得志也。」部議卒如所請。

元颺,字爾賡,舉崇禎元年進士,授都水主事。帝遣中官張彞憲總理戶、工二部事。元飆抗疏謂:「內臣當別立公署,不當踞二部堂,二部司屬亦不得至彞憲門,犯交結禁。」帝責以沽名,彞寧亦慍,元颺請告歸。尋起禮部主事,進員外郎中,遷蘇松兵備參議。溫體仁當國,唐世濟為都御史,皆烏程人,其鄉人盜太湖,以兩家為奧主。元颺捕得其渠魁,則世濟族子也,置之法。遷福建提學副使,巡撫張國維奏留之。太倉人陸文聲訐其鄉官張溥、張采倡復社,亂天下。巡按倪元珙以屬元颺,元颺盛稱溥等,元珙據以入告。體仁庇文聲,兩人並獲譴,元颺謫山東鹽運司判官。十一年,濟南被兵,攝濟寧兵備事。十四年,遷天津兵備副使。十月,擢右僉都御史,代李繼貞巡撫天津,兼督遼餉。明年敘軍功,廕一子錦衣衛。時元飆已掌中樞。帝顧其兄弟厚,嘗賜宮參療元颺疾。而元颺以衰老乞休。詔遣李希沆代,未至而京城陷,元颺乃由海道脫歸。是秋九月卒。

贊曰:明季疆場多故,則重本兵之權,而居是位者乃多庸闇闒冗之輩。若張鶴鳴之任王化貞,陳新甲之舉丁啟睿,皆闇於知人。至松山之役,其誤國可勝言哉!梁廷棟謂民窮之故在官貪,似矣。而因以售其加派之說,是所謂亡國之言也。


\end{pinyinscope}