\article{列傳第六十一}

\begin{pinyinscope}
○楊洪子俊從子能信石亨從子彪從孫後郭登朱謙子永孫暉等孫鏜趙勝范廣

楊洪,字宗道,六合人。祖政,明初以功為漢中百戶。父璟,戰死靈璧。洪嗣職,調開平。善騎射,遇敵輒身先突陣。初,從成祖北征,至斡難河,獲人馬而還。帝曰:「將才也。」令識其名,進千戶。宣德四年命以精騎二百,專巡徼塞上。繼命城西貓兒峪,留兵戍之。敗寇於紅山。

英宗立,尚書王驥言邊軍怯弱,由訓練無人,因言洪能。詔加洪遊擊將軍。洪所部才五百,詔選開平、獨石騎兵益之,再進都指揮僉事。時先朝宿將已盡,洪後起,以敢戰著名。為人機變敏捷,善出奇搗虛,未嘗小挫。雖為偏校,中朝大臣皆知其能,有毀之者,輒為曲護,洪以是得展其才。

尚書魏源督邊事,指揮杜衡、部卒李全皆訐奏洪罪。帝從源言,謫衡廣西,執全付洪自治。尋命洪副都督僉事李謙守赤城、獨石。謙老而怯,故與洪左。洪每調軍,謙輒陰沮之。洪嘗勵將士殺敵,謙笑曰:「敵可盡乎?徒殺吾人耳。」御史張鵬劾罷謙,因命洪代,洪益自奮。朝廷亦厚待之,每奏捷,功雖微必敘。

洪初敗兀良哈兵,執其部長朵欒帖木兒。既代謙任,復敗其兵於西涼亭。帝賜敕嘉獎。又敕宣大總兵官譚廣等曰:「此即前寇延綏,為指揮王禎所敗者,去若軍甚邇,顧不能撲滅,若視洪等愧不?」三年春,擊寇於伯顏山。洪馬蹶傷足,戰益力,擒其部長也陵台等四人。追至寶昌州,又擒阿台答剌花等五人。寇大敗,遁去。璽書慰勞,遣醫視,進都指揮同知,賜銀幣。尋以譚廣老,命充右參將佐之。洪建議加築開平城,拓龍門所,自獨石至潮河川,增置堠臺六十。尋進都指揮使。與兀良哈兵戰三岔口。又嘗追寇至亦把禿河。再遷都督同知。九年,兀良哈寇延綏,洪與內臣韓政等出大同,至黑山迤北,邀破之克列蘇。進左都督,軍士蒙賞者九千九百餘人。洪嘗請給旗牌,不許。乃自製小羽箭、木牌令軍中。有司論其專擅,帝不問。

十二年充總兵官,代郭竑鎮宣府。自宣德以來,迤北未嘗大舉入寇。惟朵顏三衛眾乘間擾邊,多不過百騎,或數十騎。他將率巽軿,洪獨以敢戰至大將。諸部亦憚之,稱為「楊王」。瓦剌可汗脫脫不花、太師也先皆嘗致書於洪,並遺之馬。洪聞於朝,敕令受之而報以禮。嗣後數有贈遺,帝方倚任洪,不責也。帝既北狩,道宣府,也先傳帝命趣開門。城上人對曰:「所守者主上城池。天已暮,門不敢開。且洪已他往。」也先乃擁帝去。景帝監國,論前後功,封昌平伯。也先復令帝為書遺洪,洪封上之。時景帝已即位,馳使報洪:「上皇書,偽也。自今雖真書,毋受。」於是洪一意堅守。也先逼京師,急詔洪將兵二萬入衛。比至,寇已退。敕洪與孫鏜、范廣等追擊餘寇。至霸州破之,獲阿歸等四十八人,還所掠人畜萬計。及關,寇返斗,殺官軍數百人,洪子俊幾為所及。寇去,以功進侯,命率所部留京師,督京營訓練,兼掌左府事。朝廷以洪宿將,所言多采納。嘗陳禦寇三策,又奏請簡汰三千諸營將校,不得以貧弱充伍,皆從之。

景泰元年,于謙以邊警未息,宜令洪等條上方略。洪言四事,命兵部議行。都督宮聚、王喜、張斌先坐罪繫獄,洪與石亨薦三人習戰,請釋令立功。詔已許,而言官劾其黨邪撓政。帝以國家多事,務得人,置不問。上皇還,洪與石亨俱授奉天翊衛宣力武臣,予世券。明年夏,佩鎮朔大將軍印,還鎮宣府。從子能、信充左右參將,其子俊為右都督,管三千營。洪自以一門父子官極品,手握重兵,盛滿難居,乞休致,請調俊等他鎮。帝不許。八月,以疾召還京,踰月卒。贈潁國公,謚武襄。妾葛氏自經以殉,詔贈淑人。

洪久居宣府,御兵嚴肅,士馬精強,為一時邊將冠。然未嘗專殺,又頗好文學,嘗請建學宣府,教諸將子弟。

子傑嗣,上言:「臣家一侯三都督,蒼頭得官者十六人,大懼不足報稱。乞停蒼頭楊釗等職。」詔許之,仍令給俸。傑卒,無子,庶兄俊嗣。

俊,初以舍人從軍。正統中累官署都指揮僉事,總督獨石、永寧諸處邊務。景帝即位,給事中金達奉使獨石,劾俊貪侈,乃召還。也先犯京師,俊敗其別部於居庸,進都督僉事。尋充右參將,佐朱謙鎮宣府。太監喜寧數誘敵入寇,中朝患之,購擒斬寧者賞黃金千兩,白金二萬兩,爵封侯。寧為都指揮江福所獲,而俊冒其功。廷臣請如詔。帝以俊邊將,職所當為,不允。加右都督,賜金幣。

俊恃父勢橫恣,嘗以私憾杖都指揮陶忠至死。洪懼,奏俊輕躁,恐誤邊事,乞令來京,隨臣操練。許之。既至,言官交劾,下獄論斬。詔令隨洪立功。未幾,冒擒喜寧功事覺,詔追奪冒升官軍,別賞福等,而降俊官,令剿賊自效。俄充遊擊將軍,巡徼真、保、涿、易諸城,還督三千營訓練。

景泰三年,俊上疏曰:「也先既弒其主,併其眾,包藏禍心,窺伺邊境,直須時動耳。聞其妻孥輜重去宣府纔數百里。我緣邊宿兵不下數十萬,宜分為奇正以待,誘使來攻。正兵列營大同、宣府,堅壁觀變,而出奇兵倍道搗其巢。彼必還自救,我軍夾攻,可以得志。」疏下廷議,于謙等以計非萬全,遂寢。團營初設,命俊分督四營。

明年復充遊擊將軍,送瓦剌使歸。至永寧,被酒,杖都指揮姚貴八十,且欲斬之。諸將力解而止。貴訴於朝,宣府參政葉盛亦論俊罪。以俊嘗潰於獨石,斥為敗軍之將。俊上疏自理,封還所賜敕書,以明己功。言官劾其跋扈,論斬,錮之獄。會傑卒,傑母魏氏請暫釋俊營傑葬事。乃宥死,降都督僉事。旋襲洪職。家人告俊盜軍儲,再論死,輸贖還爵。久之,又以陰事告俊。免死奪爵,命其子珍襲。

俊初守永寧、懷來,聞也先欲奉上皇還,密戒將士毋輕納。既還,又言是將為禍本。及上皇復位,張軏與俊不協,言於朝,遂徵下詔獄,坐誅。奪珍爵,戍廣西。憲宗立,授龍虎衛指揮使。

能,字文敬。沈毅善騎射。從洪屢立功,為開平衛指揮使,進都指揮僉事。景泰元年進同知,充遊擊將軍,沿邊巡徼。寇犯蔚州,畏不進,復與紀廣禦寇野狐嶺,敗傷右膝,為御史張昊所劾。宥之。尋命與石彪各統精兵三千,訓練備調遣。再加都督僉事,累進左副總兵,協守宣府。巡撫李秉劾其貪惰,弗問。五年召還,總神機營。天順初,以左都督為宣府總兵官,與石彪破寇磨兒山,封武強伯。也先已死,孛來繼興,能欲約兀良哈共襲劫之,與以信砲。兵部劾其非計。帝以能志在滅賊,置不罪。寇犯宣府,能失利,復為兵部所劾,帝亦宥之。是年卒。無子,弟倫襲羽林指揮使。

信,字文實。幼從洪擊敵興州。賊將方躍馬出陣前,信直前擒之,以是知名。累功至指揮僉事。正統末,進都指揮僉事,守柴溝堡。也先犯京師,入衛,進都指揮同知。

景泰改元,守懷來,寇入不能禦。護餉永寧,聞炮聲奔還,皆被劾。朝議以方用兵,不問。累進都督僉事,代能為左副總兵,協鎮宣府。上言:「鹿角之制,臨陣可捍敵馬,結營可衛士卒,每隊宜置十具。遇敵團牌拒前,鹿角列後,神銃弓矢相繼迭發,則守無不固,戰無不克。」從之。

天順初,移鎮延綏,進都督同知。明年破寇青陽溝,大獲。封彰武伯,佩副將軍印,充總兵官,鎮守如故。延綏設總兵官佩印,自信始也。頃之,破寇高家堡。三年與石彪大破寇於野馬澗。明年,寇二萬騎入榆林,信擊卻之。追奔至金雞峪,斬平章阿孫帖木兒,還所掠人畜萬計。其冬,代李文鎮大同。

憲宗即位,信自陳前後戰功,予世券。成化元年冬禦寇延綏無功,召還,督三千營。毛里孩據河套,命佩將軍印,總諸鎮兵往禦。寇既渡河北去,已,復還據套,分掠水泉營及朔州,信等屢卻之。寇遂東入大同。因詔信還鎮大同。六年,信與副將徐恕、參將張瑛分道出塞,敗寇於胡柴溝,獲馬五百餘匹。璽書獎勵。

信在邊三十年,鎮以安靜,人樂為用。然性好營利。代王嘗奏其違法事,詔停一歲祿。十三年冬卒於鎮。贈侯,謚武毅。

洪父子兄弟皆佩將印,一門三侯伯。其時稱名將者,推楊氏。昌平侯既廢,能以流爵弗世。而信獨傳其子瑾,弘治初領將軍宿衛。三傳至曾孫炳。隆慶時,協守南京。召掌京營戎政,屢加少師。卒,謚恭襄。傳子至孫崇猷。李自成陷京師,被殺。

石亨,渭南人。生有異狀,方面偉軀,美髯及膝。其從子彪魁梧似之,鬚亦過腹。就飲酒肆,相者曰:「今平也,二人何乃有封侯相?」亨嗣世父職,為寬河衛指揮僉事。善騎射,能用大刀,每戰輒摧破。

正統初,以獲首功,累遷都指揮僉事。敗敵黃牛坡,獲馬甚眾。三年正月,敵三百餘騎飲馬黃河,亨追擊至官山下,多所斬獲。進都指揮同知。尋充左參將,佐武進伯朱冕守大同。六年上言:「邊餉難繼,請分大同左右、玉林、雲川四衛軍,墾凈水坪迤西曠土,官給牛種,可歲增糧萬八千石。」明年又言:「大同西路屯堡,皆臨極邊。玉林故城去右衛五十里,與東勝單于城接,水草便利。請分軍築壘,防護屯種。」詔皆允行。尋以敗敵紅城功,進都指揮使。敵犯延安,追至金山敗之,再遷都督僉事。亨以國制搜將才未廣,請仿漢、唐制,設軍謀宏遠、智識絕倫等科,令人得自陳,試驗擢用,不專保舉。報可。

十四年,與都督僉事馬麟巡徼塞外。至箭豁山,敗兀良哈眾,進都督同知。是時,邊將智勇者推楊洪,其次則亨。亨雖偏將,中朝倚之如大帥,故亨亦盡力。其秋,也先大舉寇大同,亨及西寧侯宋瑛、武進伯朱冕等戰陽和口。瑛、冕戰沒,亨單騎奔還。降官,募兵自效。

郕王監國,尚書于謙薦之。召掌五軍大營,進右都督。無何,封武清伯。也先逼京師,命偕都督陶瑾等九將,分兵營九門外。德勝門當敵衝,特以命亨。于謙以尚書督軍。寇薄彰義門,都督高禮等卻之。轉至德勝門外,亨用謙令,伏兵誘擊,死者甚眾。既而圍孫鏜西直門外,以亨救引卻。相持五日,寇斂眾遁。論功,亨為多,進侯。

景泰元年二月命佩鎮朔大將軍印,帥京軍三萬人,巡哨大同。遇寇,敗之。其秋,予世襲誥券。易儲,加亨太子太師。於謙立團營,命亨提督,充總兵官如故。

八年,帝將郊,宿齋宮,疾作不能行禮,召亨代。亨受命榻前,見帝病甚,遂與張軏、曹吉祥等謀迎立上皇。上皇既復辟,以亨首功,進爵忠國公。眷顧特異,言無不從。其弟姪家人冒功錦衣者五十餘人,部曲親故竄名「奪門」籍得官者四千餘人。兩京大臣,斥逐殆盡。納私人重賄,引用太僕丞孫弘,郎中陳汝言、蕭璁、張用瀚、郝璜、龍文、朱銓,員外郎劉本道為侍郎。時有語曰「朱三千,龍八百」。勢焰熏灼,嗜進者競走其門。既以私憾殺于謙、范廣等,又以給事中成章、御史甘澤等九人嘗攻其失,貶黜之。數興大獄,構陷耿九疇、岳正,而戍楊瑄、張鵬,謫周斌、盛顒等。又惡文臣為巡撫,抑武臣不得肆,盡撤還。由是大權悉歸亨。

亨無日不進見,數預政事。所請或不從,艴然見於辭色。即不召,必假事以入,出則張大其勢,市權利。久之,帝不能堪,嘗以語閣臣李賢。賢曰:「惟獨斷乃可。」帝然之。一日語賢曰:「閣臣有事,須燕見。彼武臣,何故頻見?」遂敕左順門,非宣召毋得納總兵官。亨自此稀燕見。

亨嘗白帝立碑於其祖墓。工部希亨指,請敕有司建立,翰林院撰文。帝以永樂以來,無為功臣祖宗立碑故事,責部臣,而令亨自立。初,帝命所司為亨營第。既成,壯麗踰制。帝登翔鳳樓見之,問誰所居。恭順侯吳瑾謬對曰:「此必王府。」帝曰:「非也。」瑾曰:「非王府,誰敢僭踰若此?」帝頷之。亨既權侔人主,而從子彪亦封定遠侯,驕橫如亨。兩家蓄材官猛士數萬,中外將帥半出其門。都人側目。

三年秋,彪謀鎮大同,令千戶楊斌等奏保。帝覺其詐,收斌等拷問得實,震怒,下彪詔獄。亨懼,請罪。帝慰諭之。亨請盡削弟姪官,放歸田里。帝亦不許。及鞫彪,得繡蟒龍衣及違式寢床諸不法事,罪當死。遂籍彪家,命亨養病。亨嘗遣京衛指揮裴瑄出關市木,遣大同指揮盧昭追捕亡者。至是事覺,法司請罪亨,帝猶置不問。法司再鞫彪,言彪初為大同遊擊,以代王增祿為己功,王至跪謝。自是數款彪,出歌妓行酒。彪凌侮親王,罪亦當死。因劾亨招權納賕,肆行無忌,與術士鄒叔彞等私講天文,妄談休咎,宜置重典。帝命錮彪於獄,亨閒住,罷朝參。時方議革「奪門」功,窮治亨黨,由亨得官者悉黜,朝署一清。

明年正月,錦衣指揮逯杲奏亨怨望,與其從孫後等造妖言,蓄養無賴,專伺朝廷動靜,不軌迹已著。廷臣皆言不可輕宥。乃下亨詔獄,坐謀叛律斬,沒其家貲。踰月,亨瘐死,彪、後並伏誅。

彪驍勇敢戰,善用斧。初以舍人從軍。正統末,積功至指揮同知。也先逼京師,既退,追襲餘寇,頗有斬獲,進署都指揮僉事。

景泰改元,詔予實授,充遊擊將軍,守備威遠衛。敵圍土城,彪用炮擊死百餘人,遁去。塞上日用兵,彪勇冠流輩,每戰必捷,以故一歲中數遷,至都督僉事。

恃亨勢,多縱家人占民產,又招納流亡五十餘戶,擅越關置莊墾田,為給事中李侃、御史張奎所劾,請並罪亨。景帝皆宥不問,但令給還民產,遣流亡戶復業而已。

三年冬,充右參將,協守大同。嘗憾巡撫年富抑己不得逞。及英宗復辟,召彪還。亨方得志,彪遂誣奏富罪,致之獄。未幾,進都督同知,再以遊擊將軍赴大同備敵。與參將張鵬等哨磨兒山。寇千餘騎來襲,彪率壯士衝擊,斬把禿王,搴其旗,俘斬百二十人。追至三山墩,又斬七十二人。以是封定遠伯,遊擊如故。

天順二年命偕高陽伯李文赴延綏禦寇,以疾召還,尋充總兵官。明年,寇二萬騎入掠安邊營。彪與彰武伯楊信等禦之,連戰皆捷。斬鬼力赤,追出塞轉戰六十餘里,生擒四十餘人,斬首五百餘級,獲馬駝牛羊二萬餘,為西北戰功第一。捷聞,進侯。彪本以戰功起家,不藉父兄廕,然一門二公侯,勢盛而驕,多行不義。謀鎮大同,與亨表裏握兵柄,為帝所疑。遂及於禍。

後,天順元年進士,助亨籌畫。都督杜清出亨門下,後造妖言,有「土木掌兵權」語,蓋言杜也。事覺,後伏誅,清亦流金齒。

郭登,字元登,武定侯英孫也。幼英敏。及長,博聞強記,善議論,好談兵。洪熙時,授勳衛。

正統中,從王驥征麓川有功,擢錦衣衛指揮僉事。又從沐斌徵騰衝,遷署都指揮僉事。十四年,車駕北征,扈從至大同,超拜都督僉事,充參將,佐總兵官廣寧伯劉安鎮守。朱勇等軍覆,倉猝議旋師。登告學士曹鼐、張益曰「車駕宜入紫荊關」,王振不從,遂及於敗。當是時,大同軍士多戰死,城門晝閉,人心洶洶。登慷慨奮勵,修城堞,繕兵械;拊循士卒,弔死問傷,親為裹創傅藥。曰:「吾誓與此城共存亡,不令諸君獨死也。」八月,也先擁帝北去,經大同,使袁彬入城索金幣。登閉城門,以飛橋取彬入。登與安及侍郎沈固、給事中孫祥、知府霍瑄等出謁,伏地慟哭。以金二萬餘及宋瑛、朱冕、內臣郭敬家資進帝,以賜也先等。是夕,敵營城西。登謀遣壯士劫營迎駕,不果。明日,也先擁帝去。

景帝監國,進都督同知,充副總兵。尋令代安為總兵官。十月,也先犯京師,登將率所部入援,先馳蠟書奏。奏至,敵已退。景帝優詔褒答,進右都督。登計京兵新集,不可輕用,上用兵方略十餘事。

景泰元年春,偵知寇騎數千,自順聖川入營沙窩。登率兵躡之,大破其眾,追至栲栳山,斬二百餘級,得所掠人畜八百有奇。邊將自土木敗後,畏縮無敢與寇戰。登以八百人破敵數千騎,軍氣為之一振,捷聞,封定襄伯,予世券。

四月,寇騎數千奄至,登出東門戰。佯北,誘之入士城。伏起,敵敗走。登度敵且復至,令軍士齎毒酒、羊豕、楮錢,偽為祭塚者,見寇即棄走。寇至,爭飲食之,死者甚眾。六月,也先復以二千騎入寇,登再擊卻之。越數日,奉上皇至城外,聲言送駕還。登與同守者設計,具朝服候駕月城內,伏兵城上,俟上皇入,即下月城閘。也先及門而覺,遂擁上皇去。

時鎮守中官陳公忌登。會有發公奸贓者,公疑登使之,遂與登構。帝謂于謙曰:「大同,吾籓籬也。公與登如是,其何以守!」遣右監丞馬慶代公還,登愈感奮。初,也先欲取大同為巢穴,故數來攻。及每至輒敗,有一營數十人不還者。敵氣懾,始有還上皇意。上皇既還,代王仕廛頌登功,乞降敕獎勞。兵部言登已封伯,乃止。

二年,登以老疾乞休,舉石彪自代,且請令其子嵩宿衛。帝以嵩為散騎舍人,不聽登辭。是時邊患甫息,登悉心措置,思得公廉有為者與俱。遂劾奏沈固廢事,而薦尚書楊寧、布政使年富。又言大同既有御史,又有巡按御史,僉都御史任寧宜止巡撫宣府。帝悉從之,以年富代固,而徵還固及寧。其秋,以疾召還。登初至大同,士卒可戰者纔數百,馬百餘匹。及是馬至萬五千,精卒數萬,屹然成巨鎮。登去,大同人思之。

初,英宗過大同,遣人謂登曰:「朕與登有姻,何拒朕若是?」登奏曰:「臣奉命守城,不知其他。」英宗銜之。及復辟,登懼不免,首陳八事,多迎合。尋命掌南京中府事。明年召還。言官劾登結陳汝言獲召,鞫實論斬。宥死,降都督僉事,立功甘肅。

憲宗即位,詔復伯爵,充甘肅總兵官。奏邊軍償馬艱甚,至鬻妻子。乞借楚、慶、肅三王府馬各千匹,官酬其直。從之。用硃永等薦,召掌中府事,總神機營兵。成化四年復設十二團營,命登偕朱永提督。八年卒。贈侯,謚忠武。

登儀觀甚偉,髯垂過腹。為將兼智勇,紀律嚴明,料敵制勝,動合機宜。嘗以意造「攪地龍」、「飛天網」。鑿深塹,覆以土木如平地。敵入圍中,發其機,自相撞擊,頃刻皆陷。又仿古製造偏箱車、四輪車,中藏火器,上建旗幟,鉤環聯絡,布列成陣,戰守皆可用。其軍以五人為伍,教之盟於神祠,一人有功,五人同賞,罰亦如之。十伍為隊,隊以能挽六十斤弓者為先鋒。十隊領以一都指揮,令功無相撓,罪有專責,一時稱善。

登事母孝,居喪秉禮。能詩,明世武臣無及者。無子,以兄子嵩為子。登謫甘肅,留家京師,嵩窘其衣食。登妾縫紉自給,幾殆,弗顧。登還,欲黜之,以其婿於會昌侯,侯嘗活己,隱忍不發。及卒,嵩遂襲爵。後以非登嫡嗣,止嵩身。子參降錦衣衛指揮使。

朱謙,夏邑人。永樂初,襲父職,為中都留守左衛指揮僉事。洪熙時,隸陽武侯薛祿,征北有功,進指揮使。宣德元年進萬全都指揮僉事。

正統六年,與參將王真巡哨至伯顏山,遇寇擊走之。次閔安山,遇兀良哈三百騎,又敗之。追至莽來泉,寇越山澗遁去,乃還。時謙已遷都指揮同知,乃以為都指揮使。

八年充右參將,守備萬全左衛。明年與楊洪破兀良哈兵於克列蘇,進都督僉事。所部發其不法事,帝以方防秋,宥之。復以北征功,進都督同知。

帝北狩,也先擁至宣府城下,令開門。謙與參將紀廣、都御史羅亨信不應,遂去。進右都督。與楊洪入衛,會寇已退,追襲之近畿。戰失利,洪劾之。兵部並劾洪不救。景帝俱弗問。洪入總京營,廷議欲得如洪者代之,咸舉謙。乃進左都督,充總兵官,鎮守宣府。

景泰元年四月,寇三百騎入石烽口,復由故道去,降敕切責。踰月,復入犯。謙率兵禦之,次關子口。寇數千騎突至,謙拒以鹿角,發火器擊之,寇少卻,如是數四。謙軍且退,寇復來追。都督江福援之,亦失利。謙卒力戰,寇不得入。六月復有二千騎南侵。謙遣都指揮牛璽等往禦,戰南坡。謙見塵起,率參將紀廣等馳援。自巳至午,寇敗遁。論功,封撫寧伯。是時,寇氣甚驕,屢擾宣府、大同,意二城且旦夕下。而謙守宣府,郭登守大同,數挫其眾。也先知二人難犯,始一意歸上皇。八月,上皇還。道宣府,謙率子永出見,厚犒其使者。既而謙謬報寇五千騎毀牆入。察之,則也先貢使也。詔切責之,謙惶恐謝。明年二月,卒於鎮。贈侯。子永襲。

謙在邊久,善戰。然勇而寡謀,故其名不若楊洪、石亨、郭登之著。成化中,謚武襄。

永,字景昌。偉軀貌,顧盼有威。初見上皇於宣府,數目屬焉。景泰中,嗣爵奉朝請。英宗復辟,睹永識之曰:「是見朕宣府者耶?」永頓首謝,即日召侍左右,分領宣威營禁軍。天順四年,宣、大告警,命帥京軍巡邊。七年統三千營,尋兼神機營。憲宗立,改督團營,領三千營如故。

成化元年,荊、襄盜劉通作亂。命永與尚書白圭往討。進師南漳,擊斬九百有奇。會疾留南漳,而圭率大軍破賊。永往會,道遇餘賊,俘斬數百人。其秋復進討石龍、馮喜,皆捷。論功,進侯。

毛里孩犯邊,命佩將軍印,會彰武伯楊信禦之。會遣使朝貢,乃班師。六年,阿羅出寇延綏。復拜將軍,偕都御史王越,都督劉玉、劉聚往討,擊敗之蘇家寨。寇萬騎自雙山堡分五道至,戰於開荒川。寇少卻,乘勢馳之,皆棄輜重走。至牛家寨,遇都指揮吳瓚兵少,寇圍之。指揮李鎬、滕忠至,復力戰。聚及都指揮范瑾、神英分據南山夾擊,寇乃大敗。斬首一百有六,獲馬牛數千,阿羅出中流矢遁。時斬獲無多,然諸將咸力戰追敵,邊人以為數十年所未有。論功,予世侯。

阿羅出雖少挫,猶據河套。明年正月,寇屢入,永所部屢有斬獲。三月復以萬餘騎分掠懷遠諸堡。永與越等分兵為五,設伏敗之,追至山口及滉忽都河,寇敗走。而遊擊孫鉞、蔡瑄別破他部於鹿窖山。捷聞,璽書獎勞。永等再請班師,皆不許。寇復以二萬餘騎入掠,擊退之。歲將盡,乃召永還,留越總制三邊。

十四年加永太子太保。明年冬,拜靖虜將軍,東伐,以中官汪直監督軍務。還,進爵保國公。又明年正月,延綏告警。命永為將軍,越提督軍務,直仍監督,分道出塞。越與直選輕騎出孤店關,俘寇於威寧海子。而永率大軍由南路出榆林,不見寇,道回遠,費兵食巨萬,馬死者五千餘匹。於是越得封伯,直廕錫踰等,而永無功,賞不行。久之。進太子太傅。十七年二月,復偕直、越出師大同,禦亦思馬,獲首功百二十,遂賜襲世公。

十九年秋,小王子入邊,宣、大告急。越與直已得罪,以永為鎮朔大將軍,中官蔡新監其軍,督諸將周玉、李璵等擊敗之。還,仍督團營。或投匿名書言永圖不軌。永乞解兵柄,不許。其冬,手敕加太傅、太子太師。弘治四年監修太廟成,進太師。

永治軍嚴肅,所至多奏功。前後八佩將軍印,內總十二團營兼掌都督府,列侯勛名無與比。九年卒。追封宣平王,謚武毅,子暉嗣。給事中王廷言永功不當公,朝議止予襲一世,後皆侯。詔可。

暉,字東陽。長身美髯,人稱其威重類父。又屢從父塞下,歷行陣,時以為才。弘治五年授勳衛。年垂五十,始嗣爵,分典神機營。十三年更置京營大帥,命暉督三千營兼領右府事。

火篩入大同,平江伯陳銳等不能禦,命暉佩大將軍印代之。比至,寇已退,乃還。明年春,火篩連小王子,大入延綏、寧夏。右都御史史琳請濟師。復命暉佩大將軍印,統都督李俊、李澄、楊玉、馬儀、劉寧五將往,而以中官苗逵監其軍。至寧夏,寇已飽掠去,乃與琳、逵率五路師搗其巢於河套。寇已徙帳,僅斬首三級,獲馬駝牛羊千五百以歸。未幾,寇入固原,轉掠平涼、慶陽,關中大震。兩鎮將嬰城不敢戰,而暉等畏怯不急赴。比至,斬首十二人,還所掠生口四千,遂以捷聞。

是役也,大帥非制勝才,師行紆迴無紀律,邊民死者遍野。諸郡困轉輸餉軍,費八十餘萬。他徵發稱是,先後僅獲首功十五級。廷臣連章劾三人罪,帝不問。已而上搗巢有功將士萬餘人,尚書馬文升、大學士劉健持之,帝先入逵等言,竟錄二百十人,署職一級,餘皆被賚。及班師,帝猶遣中官齎羊酒迎勞。言官極論暉罪,終不聽,以暉總督團營,領三千營右府如故。

武宗即位,寇大入宣府,復命暉偕逵、琳帥師往。寇轉掠大同,參將陳雄擊斬八十餘級,還所掠人口二千七百有奇。暉等奏捷,列有功將士二萬餘人,兵部侍郎閻仲宇、大理丞鄧璋往勘,所報多不實。終以逵故,眾咸給賜。劉瑾用事,暉等更奏錄功太薄,請依成化間白狐莊例。兵部力爭,不納,竟從暉言,得擢者千五百六十三人,暉加太保。正德六年卒。

子麒,襲侯,嘗充總兵官,鎮兩廣。與姚鏌平田州,誅岑猛,加太子太保。嘉靖初,召還。久之,守備南京,卒。子岳嗣,亦守備南京。隆慶中卒。四傳至孫國弼。天啟中,楊漣劾魏忠賢,國弼亦乞速賜處分。忠賢怒,停其歲祿。崇禎時,總督京營。溫體仁柄國,國弼抗疏劾之。詔捕其門客及繕疏者下獄,停祿如初。及至南京,進保國公。乃與馬士英、阮大鋮相結,以訖明亡。

孫鏜,字振遠,東勝州人。襲濟陽衛指揮同知。用朱勇薦,進署指揮使。正統末,擢指揮僉事,充左參將,從總兵官徐恭討葉宗留。敗賊金華,復破之烏龍嶺。

英宗北狩,景帝召鏜還,超擢都督僉事,典三千營。也先將入犯,進右都督,充總兵官,統京軍一萬禦之紫荊關。將發,寇已入,遂營都城外。寇薄德勝門,為於謙等所卻,轉至西直門。鏜與大戰,斬其前鋒數人。寇稍北,鏜逐之,寇益兵圍鏜。鏜力戰不解。高禮、毛福壽來援,禮中流矢。會石亨兵至,寇乃退。詔鏜副楊洪追之,戰於涿州深溝,頗有斬獲。師還,仍典營務。

景泰初,楊洪劾鏜下獄。石亨請赦鏜,江淵亦言城下之役,惟鏜戰最力,乃釋之。

三年冬充副總兵,協郭登鎮大同。登節制嚴,鏜不得逞,欲與分軍,且令子百戶宏侮登。帝械宏,竟以鏜故貰之。召還,典三千營如故。英宗復辟,以「奪門」功封懷寧伯,尋予世券。

天順初,甘肅告警,詔鏜充總兵官,帥京軍往討。將陛辭,病宿朝房。夜二鼓,太監曹吉祥、昭武伯曹欽反。其部下都指揮馬亮告變於恭順侯吳瑾,瑾趨語鏜。鏜草奏,叩東長安門,自門隙投入內廷,始得集兵縛吉祥,守皇城諸門。鏜走太平侯張瑾家,邀兵擊賊,瑾不敢出。鏜倉猝復走宣武街,急遣二子輔、軏呼征西將士,紿之曰:「刑部囚反獄,獲者重賞。」眾稍聚至二千人,始語之故。時已黎明,遂擊欽。欽方攻東長安門,不得入,轉攻東安門。鏜兵追及,賊稍散。軏斫欽中膊,軏亦被殺。欽知事不成,竄歸其家,猶督眾拒鏜力戰,至晡始定。論功第一,進爵世侯,仍典三千營。贈軏百戶,世襲。

鏜麤猛善戰,然數犯法。初賄太監金英,得遷都督。事覺,論斬,景帝特宥之。天順末,以受將士賄,屢被劾。不自安,求退。詔解營務及府軍前衛事,猶掌左府。

憲宗即位,中官牛玉得罪。鏜坐與玉婚,停祿閒住。尋陳情,予半祿。已,復自陳功狀,給祿如故。成化七年卒。贈淶國公,謚武敏。

子輔請嗣,吏部言「奪門」功,例不得世傳。帝以鏜捕反者,予之。傳子至孫應爵,正德中總督團營。四傳至曾孫世忠。萬曆中鎮守湖廣,總督漕運凡二十年。又三傳至孫維籓。流賊陷京師,被殺。

鏜之冒「奪門」功封伯爵也,都督董興及曹義、施聚、趙勝等皆乘是時冒封,予世券。興、義、聚自有傳。

趙勝,字克功,遷安人。襲職為永平衛指揮使。正統末,禦寇西直門,進都指揮僉事。天順初,與孫鏜等預「奪門」功,超遷都督僉事。又與鏜擊反者曹欽,進同知。孛來犯甘肅,勝與李杲充左右參將,從白圭西征至固原,擊寇,卻之。憲宗立,典鼓勇營訓練。成化改元,山西告警,拜將軍。次雁門,寇已退,乃還。明年復出延綏禦寇。會方納款,遂旋師。尋典耀武營。四年充總兵官,鎮遼東。七年召典五軍營,已,改三千營。加思蘭犯宣府,詔勝為將軍,統京兵萬人禦之,亦以寇遁召還。久之,進左都督,加太子太保。十九年封昌寧伯。

勝初與李杲並有名。後屢督大師,未見敵,無功,夤緣得封,名大損。後加太保,營萬貴妃塋,墮崖石間死。贈侯,謚壯敏。弘治初,孫鑑乞襲爵。吏部言勝無功,不當傳世,乃授錦衣衛指揮使。

范廣,遼東人。正統中嗣世職,為寧元衛指揮僉事,進指揮使。十四年,積功遷遼東都指揮僉事。

廣精騎射,驍勇絕倫。英宗北狩,廷議舉將材,尚書于謙薦廣。擢都督僉事,充左副總兵,為石亨副。也先犯京師,廣躍馬陷陣,部下從之,勇氣百倍。寇退,又追敗之紫荊關。錄功,命實授。俄進都督同知,出守懷來。尋召還。

景泰元年二月,亨出巡邊。時都督衛穎統大營,命廣協理。三月,寇犯宣府。敕兵部會諸營將遴選將材,僉舉廣。命充總兵官偕都御史羅通督兵巡哨,駐居庸關外。數月還京,副石亨提督團營軍馬。

亨所為不法,其部曲多貪縱,廣數以為言。亨銜之,譖罷廣,止領毅勇一營。廣又與都督張軏不相能。及英宗復辟,亨、軏恃「奪門」功,誣廣黨附于謙,謀立外籓,遂下獄論死。子升戍廣西,籍其家,以妻孥第宅賜降丁。明年春,軏早朝還,途中為拱揖狀。左右怪問之,曰:「范廣過也。」遂得疾不能睡,痛楚月餘而死。成化初,廷臣訟廣冤。命子昇乃襲世職。

廣性剛果。每臨陣,身先士卒,未嘗敗衄。一時諸將盡出其下,最為於謙所信任,以故為儕輩所忌。

贊曰:楊洪、石亨輩,遭時多事,奮爪牙之力,侯封世券,照耀一門,酬庸亦過厚矣。洪知盛滿可懼,而亨邪狠粗傲,怙寵而驕,其赤族宜哉。朱謙勇略不及郭登,登乃無後,而謙子永,進爵上公,子孫世侯勿絕。孫鏜、范廣善戰略相等,而廣以冤死。所遇有幸有不幸,相去豈不遠哉!


\end{pinyinscope}