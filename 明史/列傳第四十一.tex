\article{列傳第四十一}

\begin{pinyinscope}
○宋禮藺芳陳瑄王瑜周忱

宋禮,字大本,河南永寧人。洪武中,以國子生擢山西按察司僉事,左遷戶部主事。建文初,薦授陜西按察僉事,復坐事左遷刑部員外郎。成祖即位,命署禮部事,以敏練擢禮部侍郎。永樂二年拜工部尚書。嘗請給山東屯田牛種,又請犯罪無力準工者徙北京為民,並報可。七年丁母憂,詔留視事。

九年命開會通河。會通河者,元至元中,以壽張尹韓仲暉言,自東平安民山鑿河至臨清,引汶絕濟,屬之衛河,為轉漕道,名曰「會通」。然岸狹水淺,不任重載,故終元世海運為多。明初輸餉遼東、北平,亦專用海運。洪武二十四年,河決原武,絕安山湖,會通遂淤。永樂初,建北京,河海兼運。海運險遠多失亡,而河運則由江、淮達陽武,發山西、河南丁夫,陸輓百七十里入衛河,歷八遞運所,民苦其勞。至是濟寧州同知潘叔正上言:「舊會通河四百五十餘里,淤者乃三之一,濬之便。」於是命禮及刑部侍郎金純、都督周長往治之。禮以會通之源,必資汶水。乃用汶上老人白英策,築堽城及戴村壩,橫亙五里,遏汶流,使無南入洸而北歸海。匯諸泉之水,盡出汶上,至南旺,中分之為二道,南流接徐、沛者十之四,北流達臨清者十之六。南旺地勢高,決其水,南北皆注,所謂水脊也。因相地置閘,以時蓄洩。自分水北至臨清,地降九十尺,置閘十有七,而達於衛;南至沽頭,地降百十有六尺,置閘二十有一,而達於淮。凡發山東及徐州、應天、鎮江民三十萬,蠲租一百一十萬石有奇,二十旬而工成。又奏浚沙河入馬常泊,以益汶。語詳《河渠志》。是年,帝復用工部侍郎張信言,使興安伯徐亨、工部侍郎蔣廷瓚會金純,濬祥符魚王口至中灤下,復舊黃河道,以殺水勢,使河不病漕,命禮兼董之。八月還京師,論功第一,受上賞。潘叔正亦賜衣鈔。

明年,以御史許堪言衛河水患,命禮往經畫。禮請自魏家灣開支河二,泄水入土河,復自德州西北開支河一,泄水入舊黃河,使至海豐大沽河入海。帝命俟秋成後為之。禮還言:「海運經歷險阻,每歲船輒損敗,有漂沒者。有司修補,迫於期限,多科斂為民病,而船亦不堅。計海船一艘,用百人而運千石,其費可辦河船容二百石者二十,船用十人,可運四千石。以此而論,利病較然。請撥鎮江、鳳陽、淮安、揚州及袞州糧,合百萬石,從河運給北京。其海道則三歲兩運。」已而平江伯陳瑄治江、淮間諸河功,亦相繼告竣。於是河運大便利,漕粟益多。十三年遂罷海運。

初,帝將營北京,命禮取材川蜀。禮伐山通道,奏言:「得大木數株,皆尋丈。一夕,自出谷中抵江上,聲如雷,不偃一草。」朝廷以為瑞。及河工成,復以採木入蜀。十六年命治獄江西。明年造番舟。自蜀召還,以老疾免朝參,有奏事令侍郎代。二十年七月卒於官。

禮性剛,馭下嚴急,故易集事,以是亦不為人所親。卒之日,家無餘財。洪熙改元,禮部尚書呂震請予葬祭如制。弘治中,主事王寵始請立祠。詔祀之南旺湖上,以金純、周長配。隆慶六年贈禮太子太保。

藺芳,夏縣人。洪武中舉孝廉。累遷刑部郎中。永樂中,出為吉安知府。寬厚廉潔,民甚德之。吉水民詣闕言縣有銀礦,遣使覆視。父老遮芳訴曰:「聞宋季嘗有言此者,卒以妄得罪。今皆樹藝地,安所得銀礦?」芳詰告者,知其誣。獄具,同官不敢署名,芳請獨任之。奏上,帝曰:「吾固知妄也。」得寢。已,坐事謫辦事官,從宋禮治會通河,復為工部都水主事。

十年,河決陽武,灌中牟、祥符、尉氏,遣芳按視。芳言:「中鹽隄當暴流之衝,請加築塞。」又言:「自中灤分導河流,使由故道北入海,誠萬世利。」又言:「新築岸埽,止用草索,不能堅久。宜編木成大囷,貫樁其中,實以瓦石,復以木橫貫樁表,牽築隄上,則殺水固隄之長策也。」詔悉從之。其後築隄者遵用其法。以宋禮薦,擢工部右侍郎。亡何,行太僕卿楊砥言:「吳橋、東光、興濟、交河及天津屯田,雨水決隄傷稼。乞開德州良店東南黃河故道,以分水勢。」復命芳往治之。所經郡邑,有不便民者輒疏以聞。事竣還。十五年十一月卒於官。

芳自奉約,布衣蔬食。事母至孝。母甚賢。芳所治事,暮必告母。有不當,輒加教誡。芳受命唯謹,由是為良吏云。

陳瑄,字彥純,合肥人。父聞,以義兵千戶歸太祖,累官都指揮同知。瑄代父職。父坐事戍遼陽,瑄伏闕請代,詔並原其父子。瑄少從大將軍幕,以射雁見稱。屢從征南番,又征越巂,討建昌叛番月魯帖木兒,踰梁山,平天星寨,破寧番諸蠻。復徵鹽井,進攻卜木瓦寨。賊熾甚。瑄將中軍,賊圍之數重。瑄下馬射,傷足,裹創戰。自巳至酉,全師還。又從征賈哈剌,以奇兵涉打沖河,得間道,作浮梁渡軍。既渡,撤梁,示士卒不返,連戰破賊。又會雲南兵征百夷有功,遷四川行都司都指揮同知。

建文末,遷右軍都督僉事。燕兵逼,命總舟師防江上。燕兵至浦口,瑄以舟師迎降,成祖遂渡江。既即位,封平江伯,食祿一千石,賜誥券,世襲指揮使。

永樂元年命瑄充總兵官,總督海運,輸粟四十九萬餘石,餉北京及遼東。遂建百萬倉於直沽,城天津衛。先是,漕舟行海上,島人畏漕卒,多閉匿。瑄招令互市,平其直,人交便之。運舟還,會倭寇沙門島。瑄追擊至金州白山島,焚其舟殆盡。

九年命與豐城侯李彬統浙、閩兵捕海寇。海溢隄圮,自海門至鹽城凡百三十里。命瑄以四十萬卒築治之,為捍潮隄萬八千餘丈。明年,瑄言:「嘉定瀕海地,江流衝會。海舟停泊於此,無高山大陵可依。請於青浦築土山,方百丈,高三十餘丈,立堠表識。」既成,賜名寶山,帝親為文記之。

宋禮既治會通河成,朝廷議罷海運,仍以瑄董漕運。議造淺船二千餘艘,初運二百萬石,浸至五百萬石,國用以饒。時江南漕舟抵淮安,率陸運過壩,踰淮達清河,勞費其鉅。十三年,瑄用故老言,自淮安城西管家湖,鑿渠二十里,為清江浦,導湖水入淮,築四閘以時宣洩。又緣湖十里築隄引舟,由是漕舟直達於河,省費不訾。其後復浚徐州至濟寧河。又以呂梁洪險惡,於西別鑿一渠,置二閘,蓄水通漕。又築沛縣刁陽湖、濟寧南旺湖長隄,開泰州白塔河通大江。又築高郵湖隄,於隄內鑿渠四十里,避風濤之險。又自淮至臨清,相水勢置閘四十有七,作常盈倉四十區於淮上,及徐州、臨清、通州皆置倉,便轉輸。慮漕舟膠淺,自淮至通州置舍五百六十八,舍置卒,導舟避淺。復緣河隄鑿井樹木,以便行人。凡所規畫,精密宏遠,身理漕河者三十年,舉無遺策。

仁宗即位之九月,瑄上疏陳七事。一曰南京國家根本,乞嚴守備。二曰推舉宜核實,無循資格,選朝臣公正者分巡天下。三曰天下歲運糧餉,湖廣、江西、浙江及蘇、松諸府並去北京遠,往復踰年,上逋公租,下妨農事。乞令轉至淮、徐等處,別令官軍接運至京。又快船、馬船所載不過五六十石,每船官軍足用,有司添差軍民遞送,拘集聽候,至有凍餒,請革罷。四曰教職多非其人,乞考不職者黜之,選俊秀補生員,而軍中子弟亦令入學。五曰軍伍竄亡,乞核其老疾者,以子弟代,逃亡者追補,戶絕者驗除。六曰開平等處,邊防要地,兵食虛乏,乞選練銳士,屯守兼務。七曰漕運官軍,每歲北上,歸即修船,勤苦終年。該衛所又於其隙,雜役以重困之,乞加禁絕。帝覽奏曰:「瑄言皆當。」令所司速行。遂降敕獎諭,尋賜券,世襲平江伯。

宣宗即位,命守淮安,督漕運如故。宣德四年言:「濟寧以北,自長溝至棗林淤塞,計用十二萬人疏浚,半月可成。」帝念瑄久勞,命尚書黃福往同經理。六年,瑄言:「歲運糧用軍十二萬人,頻年勞苦。乞於蘇、松諸郡及江西、浙江、湖廣別僉民丁,又於軍多衛所僉軍,通為二十四萬人,分番迭運。又江南之民,運糧赴臨清、淮安、徐州,往返一年,失誤農業,而湖廣、江西、浙江及蘇、松、安慶軍士,每歲以空舟赴淮安載糧。若令江南民撥糧與附近衛所,官軍運載至京,量給耗米及道里費,則軍民交便。」帝命黃福及侍郎王佐議行之。更民運為兌運,自此始也。八年十月卒於官,年六十有九。追封平江侯,贈太保,謚恭襄。

初,瑄以濬河有德於民,民立祠清河縣。正統中,命有司春秋致祭。

孫豫,字立卿,讀書修謹。正統末,福建沙縣賊起,以副總兵從寧陽侯陳懋分道討平之,進封侯。也先入犯,出鎮臨清,建城堡,練兵撫民,安靜不擾。明年召還,父老詣闕請留。從之。景泰五年,山東饑,奉詔振恤。尋守備南京。天順元年召還,益歲祿百石。七年卒。贈黟國公,謚莊敏。

子銳嗣伯。成化初,分典三千營及團營。尋佩平蠻將軍印,總制兩廣。移鎮淮陽,總督漕運。建淮河口石閘及濟寧分水南北二閘。築隄疏泉,修舉廢墜。總漕十四年,章數十上。日本貢使買民男女數人以歸,道淮安。銳留不遣,贖還其家。淮、揚饑疫,煮糜施藥,多所存濟。弘治六年,河決張秋,奉敕塞治。還,增祿二百石,累加太傅兼太子太傅。十三年,火篩寇大同,銳以總兵官佩將軍印往援。既至,擁兵自守,為給事中御史所劾,奪祿閑住。其年卒。

子熊嗣。正德三年出督漕運。劉瑾索金錢,熊不應,銜之。坐事,逮下詔獄,謫戍海南衛,奪誥券。熊故黷貨,在淮南頗殃民。雖為瑾構陷,人無惜之者。瑾誅,赦還復爵。卒,無子。

再從子圭嗣。以薦出鎮兩廣。封川寇起,圭督諸將往討,擒其魁,俘斬數千,加太子太保。復平柳慶及賀連山賊,加太保,廕一子。安南范子儀等寇欽、廉,黎岐賊寇瓊厓,相犄角。圭移文安南,曉以利害,使縛子儀,而急出兵攻黎岐,敗走之。論功,復蔭一子,加歲祿四十石。圭能與士卒同甘苦,聞賊所在,輒擐甲先登。深箐絕壑,衝冒瘴毒,無所避,以故所向克捷。在粵且十年,殲諸小賊不可勝數。召還,掌後軍府。圭妻仇氏,咸寧侯鸞女弟也。圭深嫉鸞,鸞數短圭於世宗,幾得罪。鸞敗,帝益重圭,命總京營兵。寇入紫荊關,圭請出戰,營於盧溝,寇退而止。明年,寇復入古北口,或議列營九門為備,圭以徒示弱無益,寇亦尋退。董築京師外城,加太子太傅。卒,贈太傅,謚武襄。

子王謨嗣。僉書後軍,出鎮兩廣。賊張璉反,屠掠數郡。王謨會提督張臬討平之,擒斬三萬餘。論功加太子太保,蔭一子。萬曆中出鎮淮安,總漕運,入掌前軍府事。卒,贈少保,謚武靖。傳至明亡,爵絕。

王瑜,字廷器,山陽人。以總旗隸趙王府。永樂末,常山護衛指揮孟賢等與宦官黃儼結,謀弒帝,廢太子而立趙王。其黨高正者,瑜舅也,密告瑜。瑜大驚曰:「奈何為此族滅計。」垂涕諫,不聽。正懼謀泄,將殺瑜,瑜遂詣闕告變。按治有驗,賢等盡伏誅,而授瑜遼海衛千戶。仁宗即位,擢錦衣衛指揮同知,厚賜之,并戒同官,事必白瑜乃行。瑜持大體,不為苛細,廷中稱其賢。

宣德八年進都指揮僉事,充左副總兵,代陳瑄鎮淮安,董漕運,累進左軍都督僉事。淮安,瑜故鄉也,人以為榮。在淮數年,守瑄成法不變,有善政。民有親在與弟訟產者。瑜曰:「訟弟不友,無親不孝。」杖而斥之。又有負金不能償,至翁婿兄弟相訟者。瑜曰:「奈何以財故傷恩!」即代償,勸其敦睦。二卒盜敗舟一板,有司以盜官物,坐卒死。瑜曰:「兩卒之命,抵敗舟一板耶?」竟得末減。歲凶,發官廩以振。然性好貨,為英宗切責,而前所發不軌事有枉者。正統四年,議事入京。得疾,束兩手如高懸狀,號救求解而卒。

周忱,字恂如,吉水人。永樂二年進士。選庶吉士。明年,成祖擇其中二十八人,令進學文淵閣。忱自陳年少乞預。帝嘉其有志,許之。尋擢刑部主事,進員外郎。

忱有經世才,浮沉郎署二十年,人無知者,獨夏原吉奇之。洪熙改元,稍遷越府長史。宣德初,有薦為郡守者。原吉曰:「此常調也,安足盡周君?」五年九月,帝以天下財賦多不理,而江南為甚,蘇州一郡,積逋至八百萬石,思得才力重臣往釐之。乃用大學士楊榮薦,遷忱工部右侍郎,巡撫江南諸府,總督稅糧。

始至,召父老問逋稅故。皆言豪戶不肯加耗,并徵之細民,民貧逃亡,而稅額益缺。忱乃創為平米法,令出耗必均。又請敕工部頒鐵斛,下諸縣準式,革糧長之大入小出者。舊例,糧長正副三人,以七月赴南京戶部領勘合。既畢,復齎送部。往反資費,皆科斂充之。忱止設正副各一人,循環赴領。訖事,有司類收上之部。民大便。忱見諸縣收糧無團局,糧長即家貯之,曰:「此致逋之由也。」遂令諸縣於水次置囤,囤設糧頭、囤戶各一人,名「轄收」。至六七萬石以上,始立糧長一人總之,名「總收」。民持貼赴囤,官為監納,糧長但奉期會而已。置撥運、綱運二簿。撥運記支撥起運之數,預計所運京師、通州諸倉耗,以次定支。綱運聽其填注剝淺諸費,歸以償之。支撥羨餘,存貯在倉,曰「餘米」。次年餘多則加六徵,又次年加五徵。

初,太祖平吳,盡籍其功臣子弟莊田入官,後惡富民豪并,坐罪沒入田產,皆謂之官田。按其家租籍征之,故蘇賦比他府獨重。官民田租共二百七十七萬石,而官田之租乃至二百六十二萬石,民不能堪。

時宣宗屢下詔減官田租,忱乃與知府況鍾曲算累月,減至七十二萬餘石,他府以次減,民始少蘇。七年,江南大稔,詔令諸府縣以官鈔平糴備振貸,蘇州遂得米二十九萬石。故時公侯祿米、軍官月俸皆支於南戶部。蘇、松民轉輸南京者,石加費六斗。忱奏令就各府支給,與船價米一斗,所餘五斗,通計米四十萬石有奇,并官鈔所糴,共得米七十萬餘石,遂置倉貯之,名曰「濟農」。振貸之外,歲有餘羨。凡綱運、風漂、盜奪者,皆借給於此,秋成,抵數還官。其修圩、築岸、開河、浚湖所支口糧,不責償。耕者借貸,必驗中下事力及田多寡給之,秋與糧並賦,凶歲再振。其姦頑不償者,後不復給。定為條約以聞。帝嘉獎之。終忱在任,江南數大郡,小民不知凶荒,兩稅未嘗逋負,忱之力也。

時漕運,軍民相半。軍船給之官,民則僦舟,加以雜耗,率三石致一石,往復經年失農業。忱與平江伯陳瑄議,民運至淮安或瓜洲水次交兌,漕軍運抵通州。淮安石加五斗,瓜洲又益五升。其附近并南京軍未過江者,即倉交兌,加與過江米二斗。襯墊蘆席,與折米五合。兌軍或後期阻風,則令州縣支贏米。設CC於瓜洲水次,遷米貯之,量支餘米給守者。由是漕費大省。

民間馬草歲運兩京,勞費不訾。忱請每束折銀三分,南京則輕齎即地買納。京師百官月俸,皆持俸帖赴領南京。米賤時,俸貼七八石,僅易銀一兩。忱請檢重額官田、極貧下戶兩稅,準折納金花銀,每兩當米四石,解京兌俸,民出甚少,而官俸常足。嘉定、崑山諸縣歲納布,疋重三斤抵糧一石。比解,以縷粗見斥者十八九。忱言:「布縷細必輕,然價益高。今既貴重,勢不容細。乞自今不拘輕重,務取長廣如式。」從之。各郡驛馬及一切供帳,舊皆領於馬頭。有耗損,則馬頭橫科補買。忱令田畝出米升九合,與秋糧俱徵,驗馬上中下直給米。

正統初,淮、揚災,鹽課虧,敕忱巡視。奏令蘇州諸府,撥餘米一二萬石連揚州鹽場,聽抵明年田租,灶戶得納鹽給米。時米貴鹽賤,官得積鹽,民得食米,公私大濟。尋敕兼理松江鹽課。華亭、上海二縣逋課至六十三萬餘引,灶丁逃亡。忱謂田賦宜養農夫,鹽課宜養灶丁。因上便宜四事,命速行之。忱為節灶戶運耗,得米三萬二千餘石。亦仿濟農倉法,置贍鹽倉,益補逃亡缺額。由是鹽課大殖。浙江當造海船五十艘,下忱計度。忱召問都匠,言一艘須米千石。忱以成大事不宜惜費,第減二十石,奏於朝,竟得報可。以九載秩滿,進左侍郎。六年命兼理湖州、嘉興二府稅糧,又命同刑科都給事中郭瑾錄南京刑獄。

忱素樂易。先是,大理卿胡為巡撫,用法嚴。忱一切治以簡易,告訐者輒不省。或面訐忱:「公不及胡公。」忱笑曰:「胡卿敕旨,在祛除民害;朝廷命我,但云安撫軍民。委寄正不同耳。」既久任江南,與吏民相習若家人父子。每行村落,屏去騶從,與農夫餉婦相對,從容問所疾苦,為之商略處置。其馭下也,雖卑官冗吏,悉開心訪納。遇長吏有能,如況鍾及松江知府趙豫、常州知府莫愚、同知趙泰輩,則推心與咨畫,務盡其長,故事無不舉。常詣松江相視水利,見嘉定、上海間,沿江生茂草,多淤流,乃浚其上流,使崑山、顧浦諸所水迅流駛下,壅遂盡滌。暇時以匹馬往來江上,見者不知其為巡撫也。歷宣德、正統二十年間,朝廷委任益專。兩遭親喪,皆起復視事。忱以此益發舒,見利害必言,言無不聽。

初,欲減松江官田額,依民田起科。戶部郭資、胡濙奏其變亂成法,請罪之,宣宗切責資等。忱嘗言:「吳淞江畔有沙塗柴場百五十頃,水草茂盛,蟲蜢多生其中。請募民開墾,可以足國課,消蟲災。」又言:「丹徒、丹陽二縣田沒入江者,賦尚未除。國初蠲稅之家,其田多併於富室,宜徵其租,沒於江者除之,則額不虧而貧富均。無錫官田賦白米太重,請改征租米。」悉報可。其因災荒請蠲貸,及所陳他利病無算。小者用便宜行之,無所顧慮。久之見財賦充溢,益務廣大。修葺廨舍學校、先賢祠墓、橋梁道路,及崇飾寺觀,贈遺中朝官,資餉過客,無稍吝惜。胥吏漁蠹其中,亦不甚訾省。以故屢召人言。

九年,給事中李素等劾忱妄意變更,專擅科斂。忱上章自訴。帝以餘米既為公用,置不問。先是,奸民尹崇禮欲撓忱法,奏忱不當多征耗米,請究問倉庫主者,忱因罷前法。既而兩稅復逋,民無所賴,咸稱不便。忱乃奏按崇禮罪,舉行前法如故。再以九載滿,進戶部尚書。尋以江西人不得官戶部,乃改工部,仍巡撫。

景泰元年,溧陽民彭守學復訐忱如崇禮言,戶部遂請遣御史李鑒等往諸郡稽核。明年又以給事中金達言,召忱還朝。忱乃自陳:「臣未任事之先,諸郡稅糧無歲不逋。自臣蒞任,設法刬弊,節省浮費,於是歲無逋租,更積贏羨。凡向之公用所須、科取諸民者,悉於餘米隨時支給。或振貸未還,遇赦宥免,或未估時值,低昂不一。緣奉宣宗皇帝并太上皇敕諭,許臣便宜行事,以此支用不復具聞。致守學訐奏,戶部遣官追徵,實臣出納不謹,死有餘罪。」禮部尚書楊寧言:「妄費罪乃在忱,今估計餘值,悉徵於民間,至有棄家逃竄者,乞將正統以前者免追。」詔許之,召鑒等還。既而言官猶交章劾忱,請正其罪。景帝素知忱賢,大臣亦多保持之,但令致仕。

然當時言理財者,無出忱右。其治以愛民為本。濟農倉之設也,雖與民為期約,至時多不追取。每歲徵收畢,踰正月中旬,輒下檄放糧,曰:「此百姓納與朝廷剩數,今還與百姓用之,努力種朝廷田,秋間又納朝廷稅也。」其所弛張變通,皆可為後法。諸府餘米,數多至不可校,公私饒足,施及外郡。景泰初,江北大饑,都御史王竑從忱貸米三萬石。忱為計至來年麥熟,以十萬石畀之。

性機警。錢穀鉅萬,一屈指無遺算。嘗陰為冊記陰晴風雨。或言某日江中遇風失米,忱言是日江中無風,其人驚服。有奸民故亂其舊案嘗之。忱曰:「汝以某時就我決事,我為汝斷理,敢相紿耶?」三殿重建,詔徵牛膠萬斤,為彩繪用。忱適赴京,言庫貯牛皮,歲久朽腐,請出煎膠,俟歸市皮償庫。土木之變,當國者議,欲焚通州倉,絕寇資。忱適議事至,言倉米數百萬,可充京軍一歲餉,令自往取,則立盡,何至遂付煨燼。頃之,詔趣造盔甲數百萬。忱計明盔浴鐵工多,令且沃錫,數日畢辦。

忱既被劾,帝命李敏代之,敕無輕易忱法。然自是戶部括所積餘米為公賦,儲備蕭然。其後吳大饑,道殣相望,課逋如故矣。民益思忱不已,即生祠處處祀之。景泰四年十月卒。謚文襄。況鍾等自有傳。

贊曰:宋禮、陳瑄治河通運道,為國家經久計,生民被澤無窮。周忱治財賦,民不擾而廩有餘羨。此無他故,殫公心以體國,而才力足以濟之。誠異夫造端興事,徼一時之功,智籠巧取,為科斂之術者也。然河渠之利,世享其成,而忱之良法美意,未幾而澌滅無餘,民用重困。豈非成功之有跡者易以循,而用法之因人者難其繼哉。雖然,見小利而樂紛更,不能不為當日之嘵嘵者惜也。


\end{pinyinscope}