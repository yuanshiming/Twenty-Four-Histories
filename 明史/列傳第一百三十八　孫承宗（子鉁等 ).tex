\article{列傳第一百三十八 孫承宗(子鉁等 )}

\begin{pinyinscope}
孫承宗,字稚繩,高陽人。貌奇偉,須髯戟張。與人言,聲殷墻壁。始為縣學生,授經邊郡。往來飛狐、拒馬間,直走白登,又從紇干、清波故道南下。喜從材官老兵究問險要厄塞,用是曉暢邊事。

萬歷三十二年,登進士第二人,授編修,進中允。「梃擊」變起,大學士吳道南以諮承宗。對曰:「事關東宮化。參見「普遍規律」。,不可不問;事連貴妃,不可深問。龐保、劉成而下,不可不問也;龐保、劉成而上,不可深問也。」道南如其言,具揭上之,事遂定。出典應天鄉試,發策著其語。攖黨人忌,將以大計出諸外,學士劉一景保持,乃得免。歷諭德、洗馬。

熹宗即位,以左庶子充日講官。帝每聽承宗講,輒曰「心開」,故眷注特殷。天啟元年進少詹事。時沈、遼相繼失,舉朝洶洶。御史方震孺請罷兵部尚書崔景榮等,以承宗代。廷臣亦皆以承宗知兵,遂推為兵部添設侍郎,主東事。帝不欲承宗離講筵,疏再上不許。二年擢禮部右侍郎,協理詹事府。

未幾,大清兵逼廣寧,王化貞棄城走,熊廷弼與俱入關。兵部尚書張鶴鳴懼罪,出行邊。帝亦急東事,遂拜承宗兵部尚書兼東閣大學士,入直辦事。越數日,命以閣臣掌部務。承宗上疏曰:「邇年兵多不練,餉多不核。以將用兵,而以文官招練;以將臨陣,而以文官指發;以武略備邊,而日增置文官於幕;以邊任經、撫,而日問戰守於朝;此極弊也。今天下當重將權,擇一沉雄有氣略者,授之節鉞,得自闢置偏裨以下,勿使文吏用小見沾沾陵其上。邊疆小勝小敗,皆不足問,要使守關無闌入,而徐為恢復計。」因列上撫西部、恤遼民、簡京軍、增永平大帥、修薊鎮亭障、開京東屯田數策,帝褒納焉。時邊警屢告,閣部大臣幸旦暮無事,而言路日益紛呶。承宗乃請下廷弼於理,與化貞並讞,用正朝士黨護。又請逮給事中明時舉、御史李達,以懲四川之招兵致寇者。又請詰責遼東巡按方震孺、登萊監軍梁之垣、薊州兵備邵可立,以警在位之骫骳者。諸人以次獲譴,朝右聳然,而側目怨咨者亦眾矣。

兵部尚書王在晉代廷弼經略遼東,與總督王象乾深相倚結。象乾在薊門久,習知西部種類情性,西部亦愛戴之,然實無他才,惟啖以財物相羈縻,冀得以老解職而已。在晉謀用西部襲廣寧,象乾惎之曰:「得廣寧,不能守也,獲罪滋大。不如重關設險,衛山海以衛京師。」在晉乃請於山海關外八里鋪築重關,用四萬人守之。其僚佐袁崇煥、沈棨、孫元化等力爭不能得,奏記於首輔葉向高。向高曰:「是未可臆度也。」承宗請身往決之。帝大喜,加太子太保,賜蟒玉、銀幣。抵關,詰在晉曰:「新城成,即移舊城四萬人以守乎?」在晉曰:「否,當更設兵。」曰:「如此,則八里內守兵八萬矣。一片石西北不當設兵乎?且築關在八里內,新城背即舊城趾,舊城之品坑地雷為敵人設,抑為新兵設乎?新城可守,安用舊城?如不可守,則四萬新兵倒戈舊城下,將開關延入乎,抑閉關以委敵乎?」曰:「關外有三道關可入也。」曰:「若此,則敵至而兵逃如故也,安用重關?」曰:「將建三寨於山,以待潰卒。」曰:「兵未潰而築寨以待之,是教之潰也。且潰兵可入,敵亦可尾之入。今不為恢復計,畫關而守,將盡撤籓籬,日哄堂奧,畿東其有寧宇乎?」在晉無以難。承宗乃議守關外。監軍閻鳴泰主覺華島,袁崇煥主寧遠衛,在晉持不可,主守中前所。舊監司邢慎言、張應吾逃在關,皆附和之。

初,化貞等既逃,自寧遠以西五城七十二堡悉為哈喇慎諸部所據,聲言助守邊。前哨遊擊左輔名駐中前,實不出八里鋪。承宗知諸部不足信,而寧遠、覺華之可守,已決計將自在晉發之,推心告語凡七晝夜,終不應。還朝,言:「敵未抵鎮武而我自燒寧、前,此前日經、撫罪也;我棄寧、前,敵終不至,而我不敢出關一步,此今日將吏罪也。將吏匿關內,無能轉其畏敵之心以畏法,化其謀利之智以謀敵,此臣與經臣罪也。與其以百萬金錢浪擲於無用之版築,曷若築寧遠要害?以守八里鋪之四萬人當寧遠衝,與覺華相犄角。敵窺城,令島上卒旁出三岔,斷浮橋,繞其後而橫擊之。即無事,亦且收二百里疆土。總之,敵人之帳幕必不可近關門,杏山之難民必不可置膜外。不盡破庸人之論,遼事不可為也。」其他制置軍事又十餘疏。帝嘉納。無何,御講筵,承宗面奏在晉不足任,乃改南京兵部尚書,並斥逃臣慎言等,而八里築城之議遂熄。

在晉既去,承宗自請督師。詔給關防敕書,以原官督山海關及薊、遼、天津、登、萊諸處軍務,便宜行事,不從中制,而以鳴泰為遼東巡撫。承宗乃辟職方主事鹿善繼、王則古為贊畫,請帑金八十萬以行。帝特御門臨遣,賜尚方劍、坐蟒,閣臣送之崇文門外。既至關,令總兵江應詔定軍制,僉事崇煥建營舍,廢將李秉誠練火器,贊畫善繼、則古治軍儲,沈棨、杜應芳繕甲仗,司務孫元化築炮臺,中書舍人宋獻、羽林經歷程崙主市馬,廣寧道僉事萬有孚主採木,而令遊擊祖大壽佐金冠於覺華,副將陳諫助趙率教於前屯,遊擊魯之甲拯難民,副將李承先練騎卒,參將楊應乾募遼人為軍。

是時,關上兵名七萬,顧無紀律,冒餉多。承宗大閱,汰逃將數百人反對無產階級專政的學說。列寧在《無產階級革命和叛徒考,遣還河南、真定疲兵萬餘,以之甲所救難民七千發前屯為兵。應乾所募遼卒出戍寧遠,咨朝鮮使助聲援。犒毛文龍於東江,令復四衛。檄登帥沈有容進據廣鹿島。欲以春防躬詣登、萊商進取,而中朝意方急遼,弗許也。應詔被劾,承宗請用馬世龍代之,以尤世祿、王世欽為南北帥,聽世龍節制,且為世龍請尚方劍。帝皆可之。世龍既受事,承宗為築壇,拜行授鉞禮。率教已守前屯,盡驅哈喇慎諸部,撫場猶在八里鋪。象乾議開水關,撫之關內,承宗不可,乃定於高臺堡。

時大清兵委廣寧去,遼遺民入居之。插漢部以告有孚,有孚謀挾西部乘間殲之,冒恢復功。承宗下檄曰:「西部殺我人者,致罰如盟言。」是役也,全活千餘人。帝好察邊情,時令東廠遣人詣關門,具事狀奏報,名曰「較事」。及魏忠賢竊政,遣其黨劉朝、胡良輔、紀用等四十五人齎內庫神炮、甲仗、弓矢之屬數萬至關門,為軍中用,又以白金十萬,蟒、麒麟、獅子、虎、豹諸幣頒賚將士,而賜承宗蟒服、白金慰勞之,實覘軍也。承宗方出關巡寧遠,中路聞之,立疏言:「中使觀兵,自古有戒。」帝溫旨報之。使者至,具杯茗而已。

鳴泰之為巡撫也,承宗薦之。後知其無實,軍事多不與議。鳴泰怏怏求去,承宗亦引疾。言官共留承宗,詆鳴泰,巡關御史潘雲翼復論劾之。帝乃罷鳴泰,而以張鳳翼代。鳳翼怯,復主守關議。承宗不悅,乃復出關巡視。抵寧遠,集將吏議所守。眾多如鳳翼指,獨世龍請守中後所,而崇煥、善繼及副將茅元儀力請守寧遠,承宗然之,議乃定。令大壽興工,崇煥、滿桂守之。先是,虎部竊出盜掠,率教捕斬四人。象乾欲斬率教謝虎部,承宗不可。而承宗所遣王楹戍中右,護其兵出採木,為西部朗素所殺。承宗怒,遣世龍剿之。象乾恐壞撫局,令郎素縛逃人為殺楹者以獻,而增市賞千金。承宗方疏爭,而象乾以憂去。

承宗患主款者撓己權,言督師、總督可勿兼設,請罷己,不可,則弗推總督。并請以遼撫移駐寧遠。帝命止總督推,而鳳翼謂置己死地也,因大恨。與其鄉人雲翼、有孚等力毀世龍,以撼承宗。無何,有孚為薊撫岳和聲所劾,益疑世龍與崇煥構陷,乃共為浮言,撓出關計。給事中解學龍遂極論世龍罪。承宗憤,抗疏陳守禦策,言:「拒敵門庭之中,與拒諸門庭外,勢既辨。我促敵二百里外,敵促我二百里中,勢又辨。蓋廣寧,我遠而敵近;寧遠,我近而敵遠。我不進逼敵,敵將進而逼我。今日即不能恢遼左,而寧遠、覺華終不可棄。請敕廷臣雜議主、客之兵可否久戍,本折之餉可否久輸,關外之土地人民可否捐棄,屯築戰守可否興舉,再察敵人情形果否坐待可以消滅。臣不敢為百年久計,只計及五年間究竟何如。倘臣言不當,立斥臣以定大計,無紆迴不決,使全軀保妻子之臣附合眾喙,以殺臣一身而誤天下也。」復為世龍辯,而發有孚等交構狀。

有孚者,故侍郎世德子也,為廣寧理餉同知。城陷逃歸,象乾題為廣寧道僉事,專撫插漢,乾沒多。至是以承宗言被斥。鳳翼亦以憂歸,喻安性代。而廷臣言總督不可裁,命吳用先督薊、遼,代象乾。承宗惡本兵越彥多中制,稱疾求罷,舉彥自代以困之,廷議不可而止。

時寧遠城工竣,關外守具畢備。承宗圖大舉,奏言:「前哨已置連山大凌河,速畀臣餉二十四萬,則功可立奏。」帝命所司給之。兵、工二部相與謀曰:「餉足,渠即妄為,不如許而不與,文移往復稽緩之。」承宗再疏促,具以情告。帝為飭諸曹,而師竟不果出。

初,方震孺、游士任、李達、明時舉之譴,承宗實劾之,後皆為求宥。復稱楊鎬、熊廷弼、王化貞之勞,請免死遣戍,朝端嘩然。給事中顧其仁、許譽卿,御史袁化中交章論駁,帝皆置弗省。會承宗敘五防效勞諸臣,且引疾乞罷,乃遣中官劉應坤等齎帑金十萬犒將士,而賜承宗坐蟒、膝襴,佐以金幣。

當是時,忠賢益盜柄。以承宗功高,欲親附之,令應坤等申意。承宗不與交一言,忠賢由是大憾。會忠賢逐楊漣、趙南星、高攀龍等舊兼學。四書五經,中國史事、政書、地圖為舊學,西政、西,承宗方西巡薊、昌。念抗疏帝未必親覽,往在講筵,每奏對輒有入,乃請以賀聖壽入朝面奏機宜,欲因是論其罪。魏廣微聞之,奔告忠賢:「承宗擁兵數萬將清君側,兵部侍郎李邦華為內主,公立齏粉矣!」忠賢悸甚,繞御床哭。帝亦為心動,令內閣擬旨。次輔顧秉謙奮筆曰:「無旨離信地,非祖宗法,違者不宥。」夜啟禁門召兵部尚書入,令三道飛騎止之。又矯旨諭九門守閹,承宗若至齊化門,反接以入。承宗抵通州,聞命而返。忠賢遣人偵之,一襆被置輿中,後車鹿善繼而已,意少解。而其黨李蕃、崔呈秀、徐大化連疏詆之,至比之王敦、李懷光。承宗乃杜門求罷。

五年四月,給事中郭興治請令廷臣議去留,論冒餉者復踵至,遂下廷臣雜議。吏部尚書崔景榮持之,乃下詔勉留,而以簡將、汰兵、清餉三事責承宗。奏報,承宗方遣諸將分戍錦州、大小凌河、松、杏、右屯諸要害,拓地復二百里,罷大將世欽、世祿,副將李秉誠、孫諫,汰軍萬七千餘人,省度支六十八萬。而言官論世龍不已。至九月,遂有柳河之敗,死者四百餘人,語詳《世龍傳》。於是臺省劾世龍并及承宗,章疏數十上。承宗求去益力,十月始得請。先已屢加左柱國、少師、太子太師、中極殿大學士,遂加特進光祿大夫,蔭子中書舍人,賜蟒服、銀幣,行人護歸。而以兵部尚書高第代為經略。無何,安性亦罷,遂廢巡撫不設。

初,第力扼承宗,請撤關外以守關內。承宗駁之,第深憾。明年,寧遠被圍,乃疏言關門兵止存五萬,言者益以為承宗罪。承宗告戶部曰:「第初蒞關,嘗給十一萬七千人餉,今但給五萬人餉足矣。」第果以妄言引罪。後忠賢遣其黨梁夢環巡關,欲傅致承宗罪,無所得而止。承宗在關四年,前後修復大城九、堡四十五,練兵十一萬,立車營十二、水營五、火營二、前鋒後勁營八,造甲胄、器械、弓矢、砲石、渠答、鹵楯之具合數百萬,拓地四百里,開屯五千頃,歲入十五萬。後敘寧遠功,廕子錦衣世千戶。

莊烈帝即位,在晉入為兵部尚書,恨承宗不置,極論世龍及元儀熒惑樞輔壞關事,又嗾臺省交口詆承宗,以沮其出。二年十月,大清兵入大安口,取遵化,將薄都城,廷臣爭請召承宗。詔以原官兼兵部尚書守通州,仍入朝陛見。承宗至,召對平臺。帝慰勞畢,問方略。承宗奏:「臣聞袁崇煥駐薊州,滿桂駐順義,侯世祿駐三河,此為得策。又聞尤世威回昌平,世祿駐通州,似未合宜。」帝問:「卿欲守三河,何意?」對曰:「守三河可以沮西奔,遏南下。」帝稱善,曰:「若何為朕保護京師?」承宗言:「當緩急之際,守陴人苦饑寒,非萬全策。請整器械,厚犒勞,以固人心。」所條畫俱稱旨。帝曰:「卿不須往通,其為朕總督京城內外守禦事務,仍參帷幄。」趣首輔韓爌草敕下所司鑄關防。承宗出,漏下二十刻矣,即周閱都城,五鼓而畢,復出閱重城。明日夜半,忽傳旨守通州。時烽火遍近郊,承宗從二十七騎出東便門,道亡其三,疾馳抵通,門者幾不納。既入城,與保定巡撫解經傳、御史方大任、總兵楊國棟登陴固守。而大清兵已薄都城,乃急遣遊擊尤岱以騎卒三千赴援。旋遣副將劉國柱督軍二千與岱合,而發密雲兵三千營東直門,保定兵五千營廣寧門。以其間遣將復馬蘭、三屯二城。

至十二月四日,而有祖大壽之變。大壽,遼東前鋒總兵官也,偕崇煥入衛。見崇煥下吏,懼誅,遂與副將何可綱等率所部萬五千人東潰,遠近大震。承宗聞,急遣都司賈登科齎手書慰諭大壽,而令遊擊石柱國馳撫諸軍。大壽見登科,言:「麾下卒赴援,連戰俱捷,冀得厚賞。城上人群詈為賊,投石擊死數人。所遣邏卒,指為間諜而殺之。勞而見罪,是以奔還。當出搗朵顏,然後束身歸命。」柱國追及諸軍,其將士持弓刀相向,皆垂涕,言:「督師既戮,又將以大炮擊斃我軍,故至此。」柱國復前追,大壽去已遠,乃返。承宗奏言:「大壽危疑已甚,又不肯受滿桂節制,因訛言激眾東奔,非部下盡欲叛也。當大開生路,曲收眾心。遼將多馬世龍舊部曲,臣謹用便宜,遣世龍馳諭,其將士必解甲歸,大壽不足慮也。」帝喜從之。承宗密札諭大壽急上章自列,且立功贖督師罪,而己當代為剖白。大壽諾之,具列東奔之故,悉如將士言。帝優詔報之,命承宗移鎮關門。諸將聞承宗、世龍至,多自拔來歸者。大壽妻左氏亦以大義責其夫,大壽斂兵待命。

當潰兵出關,關城被劫掠,閉門罷市。承宗至,人心始定。關城故十六里,衛城止二里。今敵在內,關城無可守,衛城連關,可步屟而上也。乃別築牆,橫互於關城,穴之使炮可平出。城中水不足,一晝夜穿鑿百井。舊汰牙門將僑寓者千人,窮而思亂,皆廩之於官,使巡行街衢,守臺護倉,均有所事。內間不得發,外來者輒為邏騎所得,由是關門守完。乃遣世龍督步騎兵萬五千入援,令遊擊祖可法等率騎兵四營西戍撫寧。三年正月,大壽入關謁承宗,親軍五百人甲而候於門。承宗開誠與語,即日列其所統步騎三萬於教場,行誓師禮,群疑頓釋。

時我大清已拔遵化而守之。是月四日拔永平。八日拔遷安,遂下灤州。分兵攻撫寧,可法等堅守不下。大清兵遂向山海關,離三十里而營,副將官惟賢等力戰。乃還攻撫寧及昌黎「物理」、「體育」中的「運動」。,俱不下。當是時,京師道梗,承宗、大壽軍在東,世龍及四方援軍在西。承宗募死士沿海達京師,始知關城尚無恙。關西南三縣:曰撫寧、昌黎、樂亭,西北三城:曰石門、臺頭、燕河。六城東護關門,西繞永平,皆近關要地。承宗飭諸城嚴守,而遣將戍開平,復建昌,聲援始接。

方京師戒嚴,天下勤王兵先後至者二十萬,皆壁於薊門及近畿,莫利先進。詔旨屢督趣,諸將亦時戰攻物,是對客觀存在的反映。一般說來,它反映了革命階級和,然莫能克復。世龍請先復遵化,承宗曰:「不然,遵在北,易取而難守,不如姑留之,以分其勢,而先圖灤。今當多為聲勢,示欲圖遵之狀以牽之。諸鎮赴豐潤、開平,聯關兵以圖灤。得灤則以開平兵守之,而騎兵決戰以圖永。得灤、永則關、永合,而取遵易易矣。」議既定,乃令東西諸營並進,親詣撫寧以督之。五月十日,大壽及張春、邱禾嘉諸軍先抵灤城下,世龍及尤世祿、吳自勉、楊麒、王承恩繼至,越二日克之,而副將王維城等亦入遷安。我大清兵守永平者,盡撤而北還,承宗遂入永平。十六日,諸將謝尚政等亦入遵化。四城俱復。帝為告謝郊廟,大行賞賚,加承宗太傅,賜蟒服、白金,世襲錦衣衛指揮僉事。力辭太傅不受,而屢疏稱疾乞休,優詔不允。

朵顏束不的反覆,承宗令大將王威擊敗之,復賚銀幣。先以冊立東宮,加太保;及《神宗實錄》成,加官亦如之。並辭免,而乞休不已。帝命閣臣議去留,不能決。特遣中書齎手詔慰問,乃起視事。四年正月出關東巡,抵松山、錦州,還入關,復西巡,遍閱三協十二路而返。條上東西邊政八事,帝咸採納。五月以考滿,詔加太傅兼食尚書俸,廕尚寶司丞,賚蟒服、銀幣、羊酒,復辭太傅不受。

初,右屯、大凌河二城,承宗已設兵戍守。後高第來代,盡撤之,二城遂被毀。至是,禾嘉巡撫遼東,議復取廣寧、義州、右屯三城。承宗言廣寧道遠,當先據右屯,築城大凌河,以漸而進。兵部尚書梁廷棟主之,遂以七月興工,工甫竣,我大清兵大至,圍數周。承宗聞,馳赴錦州,遣吳襄、宋偉往救。禾嘉屢易師期,偉與襄又不相能,遂大敗於長山。至十月,城中糧盡援絕,守將祖大壽力屈出降,城復被毀。廷臣追咎築城非策也,交章論禾嘉及承宗,承宗復連疏引疾。十一月得請,賜銀幣乘傳歸。言者追論其喪師辱國,奪官閒住,並奪寧遠世廕。承宗復列上邊計十六事,而極言禾嘉軍謀牴牾之失,帝報聞而已。家居七年,中外屢請召用,不報。

十一年,我大清兵深入內地。以十一月九日攻高陽,承宗率家人拒守。大兵將引去,繞城納喊者三,守者亦應之三,曰「此城笑也,於法當破」,圍復合。明日城陷,被執。望闕叩頭,投繯而死,年七十有六。

子舉人鉁,尚寶丞鑰,官生鈰,生員鋡、鎬,從子煉,及孫之沆、之滂、之澋、之潔、之水憲,從孫之澈、之水美、之泳、之澤、之渙、之瀚,皆戰死。督師中官高起潛以聞。帝嗟悼,命所司優恤。當國者楊嗣昌、薛國觀輩陰扼之,但復故官,予祭葬而已。福王時,始贈太師,謚文忠。

贊曰:承宗以宰相再視師,皆粗有成效矣,奄豎斗筲,後先齮扼,卒屏諸田野,至闔門膏斧金質,而恤典不加。國是如此,求無危,安可得也。夫攻不足者守有餘,度彼之才,恢復固未易言,令專任之,猶足以慎固封守;而廷論紛呶,亟行翦除。蓋天眷有德,氣運將更,有莫之為而為者夫。


\end{pinyinscope}