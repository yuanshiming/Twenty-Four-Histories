\article{列傳第九十一}

\begin{pinyinscope}
鄭岳劉玉子愨汪元錫邢寰寇天敘唐胄潘珍族子旦餘光李中李楷歐陽鐸陶諧孫大順大臨潘塤呂經歐陽重朱裳陳察孫懋王儀子緘王學夔曾鈞

鄭岳,字汝華,莆田人。弘治六年進士。授戶部主事,改刑部主事。董天錫偕錦衣千戶張福決囚,福坐天錫上,岳言其非體。且言:「糾劾非鎮監職,而董讓行之。太常本禮部屬,而崔志端專之。內外效尤,益無忌憚」。忤旨,繫獄。尚書周經、侍郎許進等救,不聽。贖杖還職。尋進員外郎。許進督師大同,貴近惡其剛方,議代之。罷職總兵官趙褲謀起用,京軍屢出無功。岳言進不可代,褲不可用,京軍不可出。朝論韙之。

遷湖廣僉事,歸宗籓侵地於民。施州夷民相仇殺者,有司以叛告。岳擒治其魁,餘悉縱遣。荊、岳饑,勸富民出粟,馳河泊禁。屬縣輸糧遠衛,率二石致一石。岳以其直給衛,而留粟備振,民乃獲濟。

正德初,擢廣西副使。土官岑猛當徙福建,據田州不肯徙。岳許為奏改近地,猛乃請自效。尋改廣東。遷江西按察使,就遷左布政使。宸濠奪民田億萬計,民立砦自保。宸濠欲兵之,岳持不可。會提學副使李夢陽與巡按御史江萬實相訐,岳承檄按之。夢陽執岳親信吏,言岳子澐受賕,欲因以脅岳。宸濠因助夢陽奏其事,囚掠澐。巡撫任漢顧慮不能決,帝遣大理卿燕忠會給事中黎奭按問。忠等奏勘岳子私有迹,而夢陽挾制撫、按,俱宜斥。岳遂奪官為民。宸濠敗,中外交薦,起四川布政使。以憂不赴。

世宗初,擢右副都御史,巡撫江西。甫兩月,召為大理卿。嘉靖元年冬,上言內臣有犯,宜聽部院問理,毋從中決,不能從。帝數不豫,岳請遵聖祖寡欲勤治之訓,宮寢有制,進御以時,而退朝即御文華,裁決章奏,日暮還宮,以養壽命之源。報聞。出按甘肅亂卒事,總兵官李隆等皆伏罪。還朝,以災異陳刑獄失平八事。尋遷兵部右侍郎。時「大禮」未定。岳言若以兩考為嫌,第稱孝宗廟號,毋稱伯考,以稍存正統。大學士石珤請從之。帝切責珤,奪岳俸兩月。轉左侍郎。請罷山海關稅,弗許。中官崔文欲用其兄子為副將,岳持不可。寧夏總兵官仲勛行賄京師,御史聶豹以風聞論岳。岳自白,因乞休。歸十五年而卒。

劉玉,字咸栗,萬安人。祖廣衡,永樂末進士。正統間,以刑部郎中出修浙江荒政,積粟數百萬,督治陂塘為旱澇備。景泰初,歷左副都御史,鎮守陜西。請遇災傷,毋俟勘報,即除其賦,庶有司不得借覆核陰行科率,從之。還治院事。福建、浙江盜起,命往督兵捕。議創壽寧縣於官臺山,以清盜窟。討平處州賊。已,復巡撫遼東。居官以廉節稱。終刑部尚書。父喬,成化初進士。累官湖廣左布政使。玉登弘治九年進士,授輝縣知縣。發粟振饑,奏蠲虛稅,復業者千家。擢御史。初,孫伯堅、金琦、王寧皆以傳奉得官,已,又以指揮胡震為都指揮,分守通州。玉抗疏言:「傳奉不已,繼之內批,累聖德,乞皆罷之。」不納。

武宗即位,甫四月,災異迭見,玉陳修省六事。出按京畿,中官吳忠奉命選后妃,肆貪虐。玉奏。不問。劉健、謝遷罷,玉馳疏言:「劉瑾等佞幸小臣,巧戲弄,投陛下一笑。顧讒邪而棄輔臣,此亂危所自起。況今白虹貫日,彗見紫微宮,星搖天王之位。民窮財殫,所在空虛,陛下不改圖,天下將殆。乞置瑾等於理,仍留健、遷輔政。」不報。玉遂引疾歸。後瑾榜玉奸黨,復誣構之。罰輸粟塞下者三,最後逮繫詔獄,削籍放歸。瑾誅,起河南僉事,遷福建副使,皆董學政。正德十五年,累擢南京右僉都御史,提督江防。宸濠反,攻安慶,玉以舟師赴援。事定,改撫鄖陽。

世宗即位,召為左僉都御史。論遏亂功,進右副都御史。嘉靖元年改左。歷刑部左、右侍郎。初,偕九卿爭興獻帝不宜稱皇,及帝欲考獻帝,又偕廷臣伏闕哭爭。六年秋坐李福達獄削籍,卒於家。

玉所居僅庇風雨。天文、地理、兵制、刑律皆有論著。隆慶初,贈刑部尚書,謚端毅。

子愨,南京工部右侍郎。歷官亦有聲。

汪元錫,字天啟,婺源人。正德六年進士。授兵科給事中。三遷都給事中。陜西鎮守中官廖鸞族子鎧,冒功為錦衣千戶,隨鸞於陜。元錫爭之,言鎧父鵬已亂中州,勿使鎧復亂陜右。乞征還鸞,置鎧父子於理。偏頭關之捷,錄功太濫,偕同官言太監張忠、總兵官劉暉等不宜賞。湖廣鎮守太監杜甫請巡歷所部,帝許之,元錫等據祖制力爭。帝幸昌平、宣府、大同,元錫偕同官邢寰累疏諫;復言宣府守將朱振等皆扈從西巡,寇乘虛入塞,何以禦之?已,聞帝將選禁軍親征四海治部寇,復極陳不可。安遠侯柳文鎮湖廣,奏攜參隨七十餘人,元錫乞寢所奏。車駕還京,以應州之捷大賚文武群臣。元錫等言:「是役殺邊民無算,六軍多傷。今君臣欣喜交賀,而軍民繫賊庭,南向號哭,臣等何忍受賜?」中旨以納粟都指揮馬昊守備儀真,復遣內官分守潼關、山海關,駕又幸大喜峰口,欲招三衛花當、把兒孫,元錫等皆抗章諫。

帝欲南幸,舒芬、黃鞏切諫得罪,給事御史遂不敢爭。及帝將親征宸濠,元錫復諫沮。宸濠就執,元錫、寰偕六科馳疏請迴鑾。十五年,帝在南京,元錫等復屢申前請,且言:「供億繁費,使牒旁午。奸宄冒官校,少女充離宮。陛下不以宗社為重,專事逸遊,豈能長保天下。」語甚危切。

中旨以內官晁進、楊保分守蘭州、肅州,元錫等言:「二州逼強寇,不可增官守,累居民。」群小不悅,矯旨責之。詔改團營西官廳為威武團練營,以江彬、許泰等提督之,別擇地為團營教場。元錫言:「拓地則擾居民,興工則費財力,以朝廷自將之軍而彬等概加提督,則僭名分。」不從。會帝崩,事已。

世宗即位,疏言:「都督郤永以附江彬下獄,宜釋而用之。錦衣都指揮郭鰲等十人皆彬黨,宜下獄治。」咸報可。張銑、許泰繫獄,帝忽宥其死。元錫爭,不聽。屢遷至太僕卿。嘉靖六年,帝以李福達獄下三法司於理。元錫不能平,有後言,聞於張璁,並下獄奪職。後用薦起故官。歷戶部左、右侍郎,致仕,卒。

邢寰,黃梅人。正德三年進士。數言事,有直聲。

寇天敘,字子惇,榆次人。由鄉舉入太學。與崔銑、呂柟善。登正德三年進士,除南京大理評事,進寺副。累遷應天府丞。武宗駐南京,從官衛士十餘萬,日費金萬計,近幸求索倍之。尹齊宗道憂懼卒,天敘攝其事,日青衣皁帽坐堂上。江彬使者至,好語之曰:「民窮官帑乏,無可結歡,丞專待譴耳。」彬使累至皆然,彬亦止。他權幸有求,則曰:「俟若奏即予。」禁軍攫民物,天敘與兵部尚書喬宇選拳勇者與搏戲。禁軍卒受傷,慚且畏,不敢橫。其隨事禁制多類此。駕駐九月,南京不大困者,天敘與宇力也。

嘉靖三年,以右僉都御史巡撫宣府。未行,改鄖陽。甫二月,又改甘肅。回賊犯山丹,督將士擒其長脫脫木兒。西域貢獅子、犀牛、西狗,天敘請卻之,不聽。進右副都御史,巡撫陜西。寇入固原,擊敗之,斬首百餘。又討平大盜王居等,累賜銀幣。織造太監至,有司議奏罷之。天敘曰:「甫至遽請罷,即不罷,焰且益張。」會歲祲,乃請蠲租稅,發粟振饑民;因言織造非儉歲所宜設,帝立召還。歷兵部右侍郎,卒。家貧,喪事不具。天敘在太學時,嘗聞父疾,馳六晝夜抵家,父疾亦廖。

唐胄,字平侯,瓊山人。弘治十五年進士。授戶部主事。以憂歸。劉瑾斥諸服除久不赴官者,坐奪職。瑾誅,召用,以母老不出。嘉靖初,起故官。疏諫內官織造,請為宋死節臣趙與珞追謚立祠。進員外郎,遷廣西提學僉事。令土官及瑤、蠻悉遣子入學。擢金騰副使。土酋莽信虐,計擒之。木邦、孟養構兵,胄遣使宣諭,木邦遂獻地。屢遷廣西左布政使。官軍討古田賊,久無功,胄遣使撫之其魁曰:「是前唐使君令吾子入學者。」即解甲。擢右副都御史,巡撫南、贛,移山東。遷南京戶部右侍郎。十五年改北部,進左侍郎。帝以安南久不貢,將致討,郭勛復贊之。詔遣錦衣官問狀,中外嚴兵待發。胄上疏諫曰:

今日之事,若欲其修貢而已,兵不必用,官亦無容遣。若欲討之,則有不可者七,請一一陳之:

古帝王不以中國之治治蠻夷,故安南不征,著在《祖訓》。一也。

太宗既滅黎季筼,求陳氏後不得,始郡縣之。後兵連不解,仁廟每以為恨。章皇帝成先志,棄而不守,今日當率循。二也。

外夷分爭,中國之福。安南自五代至元,更曲、劉、紹、吳、丁、黎、李、陳八姓,迭興迭廢,而嶺南外警遂稀。今紛爭,正不當問,奈何殃赤子以威小醜,割心腹以補四肢,無益有害。三也。

若謂中國近境,宜乘亂取之。臣考馬援南征,深歷浪泊,士卒死亡幾半,所立銅柱為漢極界,乃近在今思明府耳。先朝雖嘗平之,然屢服屢叛,中國士馬物故者以數十萬計,竭二十餘年之財力,僅得數十郡縣之虛名而止。況又有征之不克,如宋太宗、神宗,元憲宗、世祖朝故事乎?此可為殷鑒。四也。

外邦入貢,乃彼之利。一則奉正朔以威其鄰,一則通貿易以足其國。故今雖兵亂,尚累累奉表箋、具方物,款關求入,守臣以姓名不符卻之。是彼欲貢不得,非抗不貢也。以此責之,詞不順。五也。

興師則需餉。今四川有採木之役,貴州有凱口之師,而兩廣積儲數十萬,率耗於田州岑猛之役。又大工頻興,所在軍儲悉輸將作,興師數十萬,何以給之?六也。

然臣所憂,又不止此。唐之衰也,自明皇南詔之役始。宋之衰也,自神宗伐遼之役始。今北寇日強,據我河套。邊卒屢叛,毀我籓籬。北顧方殷,更啟南征之議,脫有不測,誰任其咎?七也。

錦衣武人,暗於大體。倘稍枉是非之實,致彼不服,反足損威。即令按問得情,伐之不可,不伐不可,進退無據,何以為謀?且今嚴兵待發之詔初下,而徵求騷擾之害已形,是憂不在外夷,而在邦域中矣。請停遣勘官,罷一切徵調,天下幸甚。

章下兵部,請從其議。得旨,待勘官還更議。明年四月,帝決計征討。侍郎潘珍、兩廣總督潘旦、巡按御史餘光相繼諫,皆不納。後遣毛伯溫往,卒撫降之。

郭勛為祖英請配享,胄疏爭。帝欲祀獻皇帝明堂,配上帝,胄力言不可。帝大怒,下詔獄拷掠,削籍歸。遇赦復冠帶,卒。隆慶初,贈右都御史。

胄耿介孝友,好學多著述,立朝有執持,為嶺南人士之冠。

潘珍,字玉卿,婺源人。弘治十五年進士。正德中,歷官山東僉事,分巡袞州。賊劉七等猝至,有備不敢攻,引去,掠曲阜。珍奏徙縣治而城之。遷福建副使,湖廣左布政使。嘉靖七年以右副都御史巡撫遼東。累遷兵部左侍郎。時議諫討安南,珍上疏諫曰:「陳暠、莫登庸皆殺逆之賊,黎寧與其父譓不請封入貢亦二十年,揆以大義,皆所當討,何獨徇寧請為左右?且其地不足郡縣置,叛服無與中國。今北敵曰蕃,聯帳萬里,烽警屢聞,顧釋門庭防,遠事瘴蠻,非計之得。宜遣大臣有文武才者,聲言進討。檄數登庸罪,赦其脅從,且令黎寧合剿。賊父子不擒則降,何必勞師?」帝責珍撓成命,褫職歸。尋以恩詔復官,致仕。珍廉直有行誼,中外十餘薦,皆報寢。卒,贈右都御史。

珍族子旦,字希周。弘治十八年進士。知漳州邵武。三遷浙江左布政使。斥羨金不取。嘉靖八年擢右副都御史,撫治鄖陽。數平巨寇。累遷刑部右侍郎。十五年冬,以兵部左侍郎提督兩廣軍務。詔起復毛伯溫討安南。旦行過其里,語之曰:「安南非門庭寇。公宜以終喪辭。往來之間,少緩師期。俟其聞命求款,因撫之,可百全也。」旦抵廣,適安南使至,馳疏言:「莫登庸之篡黎氏,猶黎氏之篡陳氏也。朝廷將興問罪師,登庸即有求貢之使,何嘗不畏天威?乞容臣等觀變,待彼國自定。若登庸奉表獻琛,於中國體足矣,豈必窮兵萬里哉。」

章下禮、兵二部。族父珍適以言得罪,尚書嚴嵩、張瓚絀旦議不用。會伯溫人都,見旦疏不悅。言總督任重,宜擇知兵者。遂改旦南京兵部,以張經代之。未行,引疾乞休,語侵伯溫。帝怒,勒致仕。將還,吏白例支庫金為道里費。旦笑曰:「吾不以妄取為例。」卒,贈工部尚書。

旦上書半歲,廣東巡按御史餘光亦言:「黎氏魚肉國君,在陳氏為賊子;抗拒中國,在我朝為亂魁。今失國,或天假手登庸以報之也。自宋以來,丁移於李,李奪於陳,陳篡於黎,今黎又轉於莫。欲興黎氏,勢必不能。臣已遣官責其修貢。道里懸遠,往復陳請,必失事機。乞令臣便宜從事。」帝以光疏中引五季、六朝事,下之兵部。咎光輕率,奪其俸。無何,光進鄉試錄。禮部尚書嚴嵩摘其誤,奏之,被逮削籍。光,江寧人。

李中,字子庸,吉水人,正德九年進士。楊一清為吏部,數召中應言官試,不赴。及授工部主事,武宗自稱大慶法王,建寺西華門內,用番僧住持,廷臣莫敢言。中拜官三月即抗疏曰:「曩逆瑾竊權,勢焰薰灼。陛下既悟,誅之無赦,聖武可謂卓絕矣。今大權未收,儲位未建,義子未革,紀綱日馳,風俗日壤,小人日進,君子日退,士氣日靡,言路日閉,名器日輕,賄賂日行,禮樂日廢,刑罰日濫,民財日殫,軍政日弊。瑾既誅矣,而善治一無可舉者,由陛下惑異端故也。夫禁掖嚴邃,豈異教所得雜居?今乃建寺西華門內,延止番僧,日與聚處。異言日沃,忠言日遠,用舍顛倒,舉錯乖方。政務廢馳,職此之故。伏望陛下翻然悔悟,毀佛寺,出番僧,妙選儒臣,朝夕勸講,攬大權以絕天下之奸,建儲位以立天下之本,革義子以正天下之名,則所謂振紀綱、勵風俗、進君子、退小人諸事,可次第舉矣。」帝怒。罪將不測,以大臣救得免。踰日,中旨謫廣東通衢丞。王守仁撫贛州,檄中參其軍事。預平宸濠。

世宗踐阼,復故官。未任,擢廣東僉事。再遷廣西提學副使,以身為教。擇諸生高等聚五經書院,五日一登堂講難。三遷廣東右布政使。忤總督及巡撫御史,坐以不稱職,當罷。霍韜署吏部事,稱中素廉節有才望,當留。會政府有不悅者,降四川右參政。十八年擢右僉都御史,巡撫山東。歲歉,令民捕蝗者倍予穀,蝗絕而饑者濟。擒劇盜關繼光,鄰境攘其功,中不與辯。進副都御史,總督南京糧儲。御史金燦按四川時,嘗薦中。中不謝,燦憾之,至是摭他事誣劾。方議調用而中卒。光宗時,追謚莊介。

中守官廉。自廣西歸,欲飯客,貸米鄰家。米至,又乏薪,將以浴器爨。會日已暮,竟不及飯而別。少學於同里楊珠,既而擴充之,沉潛邃密,學者稱谷平先生。門人羅洪先、王龜年、周子恭皆能傳其學。中族人楷,又傳洪先之學。

楷,字邦正。由舉人授湯溪知縣。母艱服闋,補青田。時倭躪東南,楷積穀資守禦。青田故無城。倭至,楷禦於沙埠,倭不得渡,乃以間築城。倭又至,登陴守,日殺賊數人,倭遁去。改知昌樂,亦以治行聞。

歐陽鐸,字崇道,泰和人。正德三年進士。授行人。上書極論時政,不報。使蜀府,王厚遺之,不受。歷工部郎中,改南兵部。出為延平知府。毀淫祠數十百所,以其材葺學宮。司禮太監蕭敬家奴殺人,置之法。調福州,議均徭曰:「郡多士大夫,其士大夫又多田產。民有產者無幾耳,而徭則盡責之民。請分民半役。」士大夫率不便。巡按御史汪珊力持之,議乃行。嘉靖三年擢廣東提學副使。累遷南京光祿卿,歷右副都御史,巡撫應天十府。蘇、松田不甚相懸。下者畝五升,上者至二十倍。鐸令賦最重者減耗米,派輕齎;最輕者徵本色,增耗米。陰輕重之,賦乃均。諸推收田,從圩不從戶,詭寄無所容。州縣荒田四千四百餘頃,歲勒民償賦。鐸以所清漏賦及他奇羨補之。議徭役及裁郵置費凡數十百條,民皆稱便。遷南京兵部侍郎,進吏部右侍郎。九廟災,自陳去。

鐸有文學,內行修潔。仕雖通顯,家具蕭然。卒,贈工部尚書,謚恭簡。

陶諧,字世和,會稽人。弘治八年鄉試第一。明年成進士,選庶吉士,授工科給事中。請命儒臣日講《大學衍義》,孝宗嘉納之。

正德改元,劉瑾等亂政。諧請以瑾等誤國罪告先帝,罪之勿赦。瑾摘其譌字令對狀,伏罪乃宥之。帝命中官崔杲等往江南、浙江織造,杲等復乞長蘆鹽引。諧再疏爭,皆不聽。諧當出理邊儲,以工科掌印無人,請俟行日遣官代署。瑾遂中諧,下詔獄廷杖,斥為民。旋榜為奸黨。又誣以巡視十庫時缺布不奏,復械至闕下杖之,謫戍肅州。瑾誅,釋還鄉,其黨猶用事,竟不獲召。

嘉靖元年復官。未至,除江西僉事,轉河南管河副使。命沿河植柳,傍藝葭葦,有事採以為埽。總理都御史請推行之諸道,歲省費鉅萬。遷參政,歷左、右布政使,皆在河南。久之,擢右副都御史,提督南、贛、汀、漳軍務。疏言:「守令遷太驟,宜以六年為期。言官忤旨,當優容。養病官才力堪任者,毋終棄。」時南京御史馬等劾王瓊被逮,而新例養病久者率不復收敘,故諧以為言。又奏:「今天下差徭煩重。既有河夫、機兵、打手、富戶、力士諸役,乃編審里甲,復徵曠丁課及供億諸費。乞皆罷免。」帝採納之。

尋遷兵部右侍郎,總督兩廣軍務。海寇陳邦瑞、許折桂等突入波羅廟,欲犯廣州,為指揮李簹所蹙。邦瑞投水死,折桂還所執指揮二人,乞就撫。諧居折桂等東莞,編為總甲,使約束其黨五百人為新民。兵部以降賊群聚,恐乘隙為變,令解散其黨。已,陽春賊趙林花等攻城,與德慶賊鳳二全相倚為患,諧討破百二十五砦。帝曰:「諧功足錄,第前縱患者誰?」乃僅賚銀幣。瓊山沙灣洞賊黎佛二等殺典史,諧復剿平。為總督三年,俘斬累萬。母憂歸。起兵部左侍郎。九廟災,自陳致仕歸。卒,贈兵部尚書。隆慶初,謚莊敏。

孫大順,字景熙。嘉靖四十五年進士。歷官福建右布政使。司帑失銀,吏卒五十人皆坐繫。大順言於左使曰:「盜者兩三人耳,何盡繫之為?請為公治之。」乃縱囚令跡盜,果得真者。終右副都御史,廣西巡撫。

弟大臨,字虞臣。嘉靖三十五年進士及第,授編修。吳時來劾嚴嵩,大臨為定疏草。時來下詔獄,詰所共謀。大臨不顧,日餉之藥物,時來亦忍死無一言。萬曆初,累官吏部侍郎。卒,贈吏部尚書,謚文僖。大臨少應舉杭州,鄰婦夜奔,拒之,旦遂徙舍。為人寬然長者,而內持貞介,不以勢利易。

大順子允淳,與父同登進士。終尚寶丞。

潘塤,字伯和,山陽人。正德三年進士。授工科給事中。性剛決,彈劾無所避。論諸大寮王鼎、劉機、甯杲、陳天祥等,多見納。

乾清宮災,塤上疏曰:「陛下蒞阼九年,治效未臻,災祥迭見。臣願非安宅不居,非大道不由,非正人不親,非儒術不崇,非大閱不觀兵,非執法不成獄,非骨肉之親不干政,非汗馬之勞不濫賞。臣聞陛下好戲謔矣。臣以為入而內庭琴瑟鐘鼓人倫之樂,不必遊離宮以為懽,狎群小以為快也;出而外廷華裔一統莫非臣妾,不必收朝官為私人,集遠人為勇士也。聞陛下好佛矣。臣以為南郊有天地,太廟有祖宗。錫祉迎庥,佛於何有?番僧可逐而度僧可止也。聞陛下好勇、好貨、好土木矣。臣以為誅奸遏亂,大勇也,不須馳馬試劍以自勞。三軍六師,大武也,不須邊將邊軍以自擁。任土作貢,皇店奚為?闤闠駢闐,內市安用?阿房壯麗,古以為金塊珠礫也,況養豹乎!金碧熒煌,古以為塗膏畔血也,況供佛乎!是數者之好皆可已而不已者也。」疏入,報聞。

十一年正月,上書言:「陛下始者血氣未定,禮度或踰。今春秋已盛,更絃易轍,此其時也。昔太甲居桐,處仁遷義,不失中興。漢武下輪臺之詔,年已七十,猶為令主。況陛下過未浮於太甲,悔又早於武帝,何愆不可蓋,何治不可建乎?」時欲毀西安門外民居,有所興作。塤與御史熊相、曹雷復切諫,皆不報。

三遷至兵科都給事中。右都督毛倫以附劉瑾論死,削世廕。倫嘗有德於錢寧,恃為內援,其子求復襲。塤等力爭,寧從中主之,寢其奏。忽中旨命塤與吏科給事中呂經各進一階,外調,舉朝大駭。給事中邵錫、御史王金等交章請留,不報。遂添注塤開州同知。

嘉請七年,累官右副都御史,巡撫河南。潞州巨盜陳卿據青陽山為亂,山西巡撫江潮、常道先後討賊無功,乃敕塤會剿。塤謀於道曰:「賊守險,難以陣。合諸路夾攻,出不意奪其險,乃可擒也。」遂分五哨三路入,募土人為導。首攻奪井腦,賊悉眾爭險。官軍奮擊,大破之,追奔至莎草嶺,毀安陽諸巢。山東副使牛鸞由潞城入,破賊李莊泉。其夕,河南副使翟瓚搗卿巢,卿敗走。瓚追敗之欒莊山,又敗之神河。山西僉事陳大綱亦屢蹙賊,先後降二千三百餘人。自進兵至搜滅賊巢,凡二十九日。捷聞,帝將大賚,遣給事中夏言往核,未報。河南大饑,塤不以時振,而河南知府范璁不待報,輒開倉發粟,民德而頌之。塤怨聲大起,流聞禁中。帝切責撫、按匿災狀。塤惶恐引罪,且歸罪於璁,遂為給事中蔡經等所劾。詔罷塤,永不敘用。言核上平賊功,塤為首。桂萼惡之,但賚銀幣。年八十七卒。

呂經,字道夫,陜西寧州人。正德三年進士。授禮科給事中。九年,乾清宮災,經上疏極論義子、番僧、邊帥之害。屢遷吏科都給事中,復極論馬昂女弟入宮事,又劾方面最貪暴者四人。群小咸惡,遂謫蒲州同知。又以事忤中官黃玉,誣劾繫獄。

世宗即位,擢山東參政。嘉靖十三年累官右副都御史,巡撫遼東。故事,每軍一,佐以餘丁三;每馬一,給牧地五十畝。經損餘丁之二編入均徭冊,盡收牧地還官。又役軍築邊牆,督趣過當。諸軍詣經乞罷役,都指揮劉尚德叱之不退,經呼左右榜訴者。卒遂爭毆尚德,經竄苑馬寺幽室中。亂卒毀府門,火均徭冊,搜得經,裂其冠裳,幽之都司署。帝詔經還朝。都指揮袁璘將克諸軍草價為辦裝,卒復執經,裸而置之獄,虐辱之,脅鎮守中官王純等奏經十一罪。帝逮經。亂卒復置官校於獄,久之始解。經下詔獄,謫戍茂州。數年釋還。隆慶初,復官,卒。亂卒為曾銑所定,見《銑傳》。

歐陽重,字子重,廬陵人。正德三年進士。殿試對策,歷詆闕政。授刑部主事。劉瑾兄死,百官往弔,重不往。張銳、錢寧掌廠衛,連構搢紳獄,重皆力與爭。銳等假他事繫之獄,贖杖還職,仍停俸。再遷郎中。歷四川、雲南提學副使。遷浙江按察使,未上。嘉靖六年春拜右僉都御史,巡撫應天。會尋甸土酋安銓、鳳朝文反,廷議以重諳滇事,乃改雲南。初,武定土知府鳳詔母子坐事留雲南,朝文紿其眾,言詔已戮,官軍將盡滅其部黨,以故諸蠻悉從為亂,攻圍會城。重督兵擊敗之,而遣詔母子還故地。其黨愕,相率歸之。朝文計窮,絕普渡河走。追兵至,殲焉。銓逃尋甸故巢。官軍攻破其砦,執銓,賊盡平。乃散其黨二萬人,遷尋甸府於鳳梧山下,更設守禦千戶所。重推功於前撫臣傅習,並進秩任子。緬甸、木邦、隴川、孟密、孟養諸酋相仇殺,各訐奏於朝,下重等勘覆。遣參政王汝舟、知府嚴時泰等遍歷諸蠻,譬以禍福。皆還侵地,供貢如故。重列善後數事,悉報可,賜璽書褒諭。重乃恤創殘,振貧乏,輕徭賦,規畫鹽鐵商稅、屯田諸務。民咸便之。

雲南歲貢金千兩,費不貲。大理太和蒼山產奇石,鎮守中官遣軍匠攻鑿。山崩,壓死無算。重皆疏罷之,浮費大省。當是時,鎮守太監杜唐、黔國公沐紹勛相比為奸利,長吏不敢問,群盜由此起。重疏言:盜率唐、紹勛莊戶,請究主者。又奏紹勛任千戶何經廣誘奸人,奪民產;唐役占官軍,歲取財萬計。因極言鎮守中官宜革。帝頗納其言,頻下詔飭紹勛,命唐還京待勘。二人懼且怒,遣人結張璁,謀去重。會重奉命清異姓冒軍弊,都司久未報,給餉後期。唐等遂嗾六衛軍華於軍門。巡按御史劉臬以聞。劾重及唐、紹勛處置失當。璁從中主之,解重職,責臬黨庇,調外任,唐、紹勛不問。都給事中夏言等抗章曰:「以軍士噪罪撫、按,紀綱謂何?況重奉詔非生事。臬言唐、紹勛罪與重等,今處分失宜,無以服天下。頃年士卒驕悍,相效成風,類以月糧借口。如甘肅、大同、福州、保定,事變屢見。失今不治,他日當事之臣以此為諱,專務姑息,孰肯為陛下任事哉!願曲宥二臣,全朝廷之體。」帝怒,奪言等俸。重罷歸在道,聞御史王化劾其為桂萼黨,不勝忿,抗疏陳辨,請錄「大禮」大獄被逐諸臣,而自乞褫職。又言得紹勛所遣百戶丁鎮私書,知行賄張璁,乞其覆護;璁奸佞,不宜在左右。璁疏辨。帝以重失職怨望,黜為民。重以臬被謫,言等奪俸,皆由己致之,復疏乞重譴代言官罪。帝益怒,以已除名,置不問。重家居二十餘年,言者屢薦,竟不復召。

朱裳,字公垂,沙河人。年十四為諸生,讀書黌舍,躬執爨。提學御史顧潛俾受學於崔銑。登正德九年進士,擢御史,巡鹽河南。錢寧遣人牟鹽利,裳禁不予。巡按山東。前御史王相忤鎮守中官黎監,被誣下詔獄。裳抗疏直相,劾監八罪。帝還自宣府,裳請下罪己詔,新庶政,以結人心。不報。山東大水,淹城武、單二城。以裳言,命相地改築。帝幸南都久,裳極陳小人熒惑之害。出為鞏昌知府。嘉靖二年舉治行卓異,遷浙江副使。日啜菜羹,妻操井臼,迎父就養。同列知其貧,製衣一襲為壽,父亦拒不納。三遷至浙江左布政使,以右副都御史總理河道,數條上方略。外艱歸,久不起。帝南巡,謁行在,命以故官總理河道。迎章聖太后梓宮,冒暑卒。隆慶中,追贈戶部右侍郎,謚端簡。

陳察,字元習,常熟人。弘治十五年進士。授南昌推官。正德初,擢南京御史。尋改北。劉瑾既誅,武宗猶日狎群小。察偕同官請務講學,節嗜欲,勤視朝,語甚切直。以養親歸。家居九年,始赴補。會帝將親征宸濠。察請無行,而亟下罪己詔。忤旨,奪俸一年。諭群臣更諫,必置極典。俄巡按雲南。助巡撫何孟春討定彌勒州,以功增秩。世宗即位,疏言金齒、騰衝地極邊徼,既統以巡撫總兵,又有監司守備分轄,無事鎮守中官。因劾太監劉玉、都督沐崧罪。詔並罷還。

嘉靖初,按四川。請罷鎮守中官,不聽。帝親鞫楊言,落其一指。察大呼曰:「臣願以不肖軀易言命,不忍言獨死。」帝目攝之,察不為動。退具疏申理,且請下王邦奇於獄,直聲震朝野。巡視京營,與給事中王科極陳武定侯郭勛貪橫狀。擢南京太僕少卿。疏辭,因請召前給事中劉世賢等二十餘人。帝怒,責以市恩要名,貶遠方雜職。給事中王俊民、鄭一鵬論救,皆奪俸。察補海陽教諭。累遷山西左布政使,入為光祿卿。十二年,以僉都御史巡撫南、贛。居二年,乞休,因薦前都御史萬鏜、大理卿董天錫等十四人可用。吏部請從其言。帝奪部臣俸,責察徇私妄舉,斥為民。察居官廉,既歸,敝衣糲食而已。

孫懋,字德夫,慈谿人。正德六年進士。授浦城知縣,擢南京吏科給事中。御史張經、寧波知府翟唐忤奄人被逮,懋偕同官論救。織造太監史宣誣主事王鑾、知縣胡守約,下之詔獄。懋言:「宣妄言御賜黃棍,聽撻死官吏,脅主簿孫錦死,今又誣守職臣。乞治宣罪,還鑾、守約故任。」未幾,復偕諸給事言:「臣等屢建白,不擇可否,一概留中。萬一奸人陰結黨類,公行阻遏,朝有大事,陛下不聞,大臣不知,禍可勝言!」皆不報。已,又劾罷鹽法侍郎薛章,請黜太僕少卿馬陟,留御史徐文華,召還謝遷、韓文、孫交、張原、周廣、高公韶、王思等,罷游畋射獵,復御朝常儀,還久留邊兵,汰錦衣冗官,諸疏皆侃侃。江彬導帝巡幸。懋言:「彬梟桀憸邪,挾至尊出居庸,無大臣保護,獨處沙漠將半載。兩宮違養,郊廟不親,四方災異迭見,盜賊蜂起。留彬一日,為宗社一日憂,乞立置重典。」時中外章奏,帝率不省視。規主闕者,往往得無罪。一觸權倖,禍立至,人皆為懋危。而彬方日侍帝娛樂,亦不之見也。請回鑾,諫南幸,懋皆與。宸濠反,帝在南都,懋從行。請急定平賊功賞,既又數請還京,率同官伏闕,皆不省。

世宗即位,疏薦建言貶謫諸臣周廣、茫輅等二十人,皆召用。劾南京祭酒陳霽、太常卿張道榮,皆罷。未幾,言:「謝遷、韓文起用,乞仿宋起文彥博故事,不煩職務,大禮大政,時令參預,必有裨新政。」帝雖善之,不能用。

出為廣東參議,遷副使。嘉靖四年,有錦衣官校偵事廣東,懋與按察使張祐疑其偽,執之。事聞,逮下詔獄,謫藤縣典史。屢遷至廣西布政使。十六年入為應天府尹。坐所進鄉試錄忤旨,致仕,卒。

王儀,字克敬,文安人。嘉靖二年進士。除靈璧知縣。以能,調嘉定。七年擢御史,巡按陜西。秦府豪占民產,儀悉奪還民。延綏大饑,朝命陜西布政使胡忠為巡撫,儀論罷之。已,巡按河南。趙府輔國將軍祐椋招亡命殺人劫奪積十餘年,莫敢發。儀偕巡撫吳山奏之,奪爵禁錮。會儀出為蘇州知府,甫三月,祐椋潛入都,奏儀捃摭,並訐都御史毛伯溫以私憾入己罪。且言:「臣嘗建醮祈皇嗣,為知府王天民訕笑」,請並按問。帝心知祐椋罪,而悅其建醮語。為遣使覆按,解儀、伯溫任,下天民獄。使者奏儀不誣,第祐椋罪在赦前,宜輕坐。帝終憐祐椋愛己,竟復其爵,除儀名,伯溫、山、天民皆得罪。終嘉靖世,多以誹謗齋醮獲重禍,由祐椋訐奏始。

儀去蘇州,士民走闕下乞留,帝不許。既而薦起知撫州。蘇州士民復走闕下乞還儀,至再,不報。歸愬於巡撫侯位。位以聞,帝乃許之。至則歎曰:「蘇賦當天下什二,而田額淆無可考,何以定賦?」乃履畝丈之,使縣各為籍。以八事定田賦,以三條核稅課,徭役、雜辦維均。治為知府第一,進浙江副使,飭蘇、松、常、鎮兵備。時巡撫歐陽鐸均田賦,儀佐之,以治蘇者推行於旁郡。坐與操江王學夔討賊敗績,停俸戴罪。未幾,殪賊江中,進秩一等,遷山西右參政,分守冀、寧。寇抵清源城,儀洞開城門,寇疑引去。按行所部,築城郭,積糗糧,榆次、平定間遂皆有城。

二十一年擢右僉都御史,巡撫宣府。冠入龍門,總兵官郤永等敗之。儀進右副都御史。尋以築邊垣,賚銀幣。寇自萬全右衛入,游騎犯完、唐。奪俸二級。考察拾遺,貶一官。已,勘上失事罪,貶秩如初。久之,除肅州兵備副使,協巡撫楊博徙哈密遺種於境外。稍遷右參政,復拜右僉都御史,巡撫甘肅。未行,俺答犯京師,詔儀馳鎮通州。仇鸞部卒掠民貲,捕笞之,枷市門外。鸞訴於帝,逮訊斥為民,卒。隆慶初,子緘訟冤,復官賜恤。

緘,官按察使,分巡遼陽,以知兵名。

王學夔,安福人。正德時,以吏部主事諫南巡,跪闕下,受杖。嘉靖初,奏請裁戚畹,又申救言官。歷考功、文選郎中,廉謹為時所稱。嘗撫治鄖陽。有偽稱皇子者,諸司議用兵。學夔曰:「妄豎子耳。」密捕致之辟。累遷南京吏、禮、兵三部尚書。隆慶、萬曆間,存問者再。年九十四卒。贈太子少保。

曾鈞,字廷和,進賢人。嘉靖十一年進士。授行人。擢南京禮科給事中。時四方銀場得不償費,且為盜窟,鈞奏罷之。

鈞剛廉疾俗。首劾罷參贊尚書劉龍。已,劾翊國公郭勛、禮部尚書嚴嵩。未幾劾工部侍郎蔣淦、延綏巡撫趙錦。最後劾罷操江都御史柴經。直聲震一時。

出為雲南副使。兩司詣黔國公率廷謁,鈞始正其禮,且釐還所侵麗江民地。遷四川參政。黔寇亂,撫定之。屢遷河南左布政使。三十一年以右副都御史總理河道。徐、邳等十七州縣連被水患,帝憂之,趣上方略。鈞請浚劉伶臺至赤晏廟八十里,築草灣老黃河口,增高家堰長堤,繕新莊等舊閘。閱數月,工成。進工部右侍郎。治河四年,入為南京刑部右侍郎。久之,乞歸。家居十餘年卒。贈刑部尚書,謚恭肅。

贊曰:鄭岳等居官,歷著風操。箴主闕,抑近幸,本末皆有可觀。斤斤奉職,所至以治辦聞,殆列卿之良歟!唐胄論安南,切於事理。歐陽鐸之均田賦,惠愛在民;令久於其任,幾與周忱比矣。


\end{pinyinscope}