\article{列傳第六十八}

\begin{pinyinscope}
○張寧王徽王淵等毛弘邱弘李森魏元康永韶等強珍王瑞張稷李俊汪奎從子舜民崔升等湯鼐吉人劉董傑姜綰餘濬等姜洪歐陽旦暢亨曹璘彭程龐泮呂獻葉紳胡獻武衢等張弘至屈伸王獻臣吳一貫余濂

張寧,字靖之,海鹽人。景泰五年進士。授禮科給事中。七年夏,帝從唐瑜等奏,考核南京大小諸臣。寧言:「京師尤根本地,不可獨免。」又言:「京衛帶俸武職,一衛至二千餘人,通計三萬餘員。歲需銀四十八萬,米三十六萬,並他折俸物,動經百萬。耗損國儲,莫甚於此。而其間多老弱不嫻騎射之人。莫若簡可者,補天下都司、衛所缺官,而悉汰其餘。」議格不行。

帝得疾,適遇星變,詔罷明年元會,百官朝參如朔望。寧言:「四方來覲,不得一睹天顏,疑似之際,必至訛言相驚,願勉循舊典,用慰人心。」帝疾不能從,而「奪門」之變作。

天順中,曹、石竊柄。事關禮科者,寧輒裁損,英宗以是知寧。朝鮮與鄰部毛憐衛仇殺,詔寧同都指揮武忠往解。寧辭義慷慨,而忠驍健,張兩弓折之,射鴈一發墜,朝鮮人大驚服,兩人竟解其仇而還。中官覃包邀與相見,不往。尋擢都給事中。

憲宗初御經筵,請日以《大學衍義》進講。是年十月,皇太后生辰,禮部尚書姚夔仍故事,設齋建醮,會百官赴壇行香。寧言無益,徒傷大體,乞禁止。帝嘉納之。未幾,給事中王徽以牛玉事劾大學士李賢,得罪。寧率六科論救,由是浸與內閣忤。會王竑等薦寧堪僉都御史清軍職貼黃,與岳正並舉。得旨,會舉多私,皆予外任。寧出為汀州知府,以簡靜為治,期年善政具舉。

寧才高負志節,善章奏,聲稱籍甚。英宗嘗欲重用之,不果。久居諫垣,不為大臣所喜。既出守,益鬱鬱不得志,以病免歸。家居三十年,言者屢薦,終不復召。

無子。有二妾。寧沒,剪髮誓死,樓居不下者四十年。詔旌為「雙節」。

王徽,字尚文,應天人。天順四年進士。除南京刑科給事中。憲宗即位數月,與同官王淵、朱寬、李翔、李鈞疏陳四事。末言:「自古宦官賢良者少,奸邪者多。若授以大權,致令敗壞,然後加刑,是始愛而終殺之,非所以保全之也。願法高皇帝舊制,毋令預政典兵,置產立業。家人義子,悉編原籍為民。嚴禁官吏與之交接。惟厚其賞賚,使得豐足,無復他望。此國家之福,亦宦官之福也。」

其冬,帝入萬妃譖,廢吳后,罪中官牛玉擅易中宮,謫之南京,徽復與淵等劾之曰:

陛下冊立中宮,此何等事,而賊臣牛玉乃大肆奸欺!中宮既退,人情咸謂玉必萬死。顧僅斥陪京,猶全首領,則凡侍陛下左右者將何所忌憚哉?內閣大臣,身居輔弼,視立后大事漠然不以加意。方玉欺肆之初,婚禮未成,禮官畏權,輒為阿附。及玉事發之後,國法難貸,刑官念舊,竟至茍容。而李賢等又坐視成敗,不出一言。黨惡欺君,莫此為甚。請並罪賢等,為大臣不忠者戒。

臣等前疏請保全宦官,正欲防患於未萌。乃處置之道未聞,牛玉之禍果作。然往不可諫,來猶可追。臣等不敢遠引,請以近事征之。正統末,有王振矣,詎意復有曹吉祥。天順初,有吉祥矣,詎意復有牛玉。若又不思預防,安知後不有甚於牛玉者哉?夫宦者無事之時似乎恭慎,一聞國政,即肆奸欺。將用某人也,必先賣之以為己功;將行某事也,必先泄之以張己勢。迨趨附日眾,威權日盛,而禍作矣。此所以不可預聞國政也。內官在帝左右,大臣不識廉恥,多與交結。餽獻珍奇,伊優取媚,即以為賢,而朝夕譽之。有方正不阿者,即以為不肖,而朝夕讒謗之,日加浸潤,未免致疑。由是稱譽者獲顯,讒謗者被斥。恩出於內侍,怨歸於朝廷,此所以不可許其交結也。內官弟侄授職任事,倚勢為非,聚奸養惡。廣營財利,奸弊多端。身雖居內,心實在外。內外交通,亂所由起,此所以不可使其子侄在外任職營立家產也。

臣等職居言路,不為茍容,雖死無悔,惟陛下裁察。

詔謂「妄言邀譽」,欲加罪。諸給事、御史交章論救,乃並謫州判官。徽得貴州普安,淵茂州,寬潼川,翔寧州,鈞綏德。奏蓋鈞筆也。侍郎葉盛、編修陳音相繼請留,不納。最後御史楊瑯言尤切,幾得罪。

微至普安,興學校教士,始有舉於鄉者。卻土官隴暢及白千戶賄,治甚有聲。居七年,棄官歸,言者屢薦,終以宦官惡之不復錄。徽嘗曰:「今仕者以剛方為刻,怠緩為寬。學者以持正為滯,恬軟為通。為文以典雅為膚淺,怪異為古健。」其論治,嘗誦張宣公語「無求辦事之人,當求曉事之人」,時皆服其切中。

弘治初,吏部尚書王恕薦起陜西左參議。踰年,謝病還,卒,年八十三。子韋,見《文苑傳》。

王淵,浙江山陰人。天順元年進士,除南京吏科給事中。素伉直,終順天府治中。

朱寬,莆田人,李翔,大足人,皆天順元年進士。李鈞,永新人,景泰二年進士。寬為南京禮科給事中,翔兵科,鈞工科。既被謫,寬進表入京,道卒。翔、鈞皆以判官終。

毛弘,字士廣,鄞人。登天順初進士。六年授刑科給事中。成化三年夏,偕六科諸臣上言:「比塞上多事,正陛下宵衣旰食時。乃聞退朝之暇,頗事逸遊。砲聲數聞於外,非禁城所宜有。況災變頻仍,兩畿水旱,川、廣兵草之餘,公私交困。願省遊戲宴飲之娛,停金豆、銀豆之賞。日御經筵,講求正學,庶幾上解天怒,下慰人心。」御史展毓等亦以為言,皆嘉納。

帝從學士商輅請,改元後建言罷官者悉錄用。弘請斷自踐阼而後,召還給事中王徽等,不許。慈懿太后崩,詔別葬。弘偕魏元等疏諫,未得請。朝罷,弘倡言曰:「此大事,吾輩當以死諫,請合大小臣工伏闕固爭。」眾許諾。有退卻者,給事中張賓呼曰:「君輩獨不受國恩乎,何為首鼠兩端。」乃伏哭文華門,竟得如禮。

弘在垣中所論列最多,聲震朝寧。帝頗厭苦之,嘗曰:「昨日毛弘,今日毛弘。」前後所陳,或不見聽,而弘慷慨論議無所屈。欽天監正谷濱受賕當除名,命輸贖貶秩。正一真人張元吉有罪論死,詔繫獄。弘等皆固爭,終不聽。三遷至都給事中。得疾,暴卒。

邱弘,字寬叔,上杭人。天順末進士。授戶科給事中。數陳時政。成化四年春,偕同官上言:「洪武、永樂間,以畿輔、山東土曠人稀,詔聽民開墾,永不科稅。邇者權豪怙勢,率指為閒田,朦朧奏乞。如嘉善長公主求文安諸縣地,西天佛子札實巴求靜海縣地,多至數十百頃。夫地踰百頃,古者百家產也。豈可徇一人之私情而奪百家恒產哉?」帝納其言,詔自今請乞,皆不許,著為令。札實巴所乞地,竟還之民。弘再遷,至都給事中。

六年夏,山東、河南大旱,弘請振。因言:「四方告災,部臣拘成例,必覆實始免。上雖蠲租,下鮮實惠。請自今遇災,撫按官勘實,即與蠲除。」從之。

萬貴妃有寵,中官梁芳、陳喜爭進淫巧;奸人屠宗順輩日獻奇異寶石,輒厚酬之,糜帑藏百萬計。有因以得官者。都人仿傚,競尚侈靡,僭擬無度。弘偕同官疏論宗順等罪,請追還帑金,嚴禁侈俗。事下刑部,尚書陸瑜因請置宗順等於理,沒其貲以振饑民。帝不許,但命僭侈者罪無赦,然竟不能禁也。

京師歲歉米貴,而四方游僧萬數,弘請驅逐,以省冗食。又請發太倉米,減價以糶,給貧民最甚者。帝悉從之。復言:「在京百獸房及清河寺諸處,所育珍禽野獸,日飼魚肉米菽,乞並縱放,以省冗費。」報聞。明年使琉球,道卒。

弘與毛弘同居言路,皆敢言,人稱「二弘」云。

李森,字時茂,歷城人。天順元年進士。授戶科給事中。負氣敢言。

憲宗立,上疏請禁朝覲官科斂徵求為民害者。吏部尚書王翱請從其言,帝為下詔禁止。頃之,言:「近有無功而晉侯、伯、都督者;有無才德而位九列者;有以畫、弈、彈琴、醫、卜技能而得官職者。名爵日輕,廩祿日費,是玩天下之公器,棄國家之大柄也。自今宜擇人授,毋令匪才競進。」且請嚴軍官黜陟,核逃伍虛糧。皆報可。御史謝文祥以劾姚夔下獄,森偕同官救之,不納。

明年夏,日食,瓊山縣地震,森疏陳十事。未幾,以貴倖侵奪民產,率諸給事言:「昔奉先帝敕,皇親強占軍民田者,罪毋赦,投獻者戍邊。一時貴戚莫敢犯。比給事中丘弘奏絕權貴請乞,陛下亦既俯從。乃外戚錦衣指揮周彧求武強、武邑田六百餘頃,翊聖夫人劉氏求通州、武清地三百餘頃,詔皆許之,何其與前敕悖也!彼谿壑難厭,而畿內膏腴有限,小民衣食皆出於此,一旦奪之,何以為生。且本朝百年來戶口日滋,安得尚有閒田不耕不稼?名為奏求,實豪奪而已。」帝善其言,而已賜者仍不問。山西災,山東及杭、紹、嘉、湖大水,森等請蠲振,帝並從之。

時帝未有儲嗣,而萬貴妃專寵,後宮莫得進。言者每勸上普恩澤,然未敢顯言妃妒也。惟森抗章為言,帝心慍。森已再遷左給事中,會戶科都給事中缺,吏部列森名上,詔予外任。部擬興化知府,不允,乃出為懷慶通判。未幾,投劾歸,不復出。

魏元,字景善,朝城人。天順元年進士。授禮科給事中。成化初,萬貴妃兄弟驕橫,元疏劾之。四年,慈懿太后崩,將別葬。元偕同官三十九人抗章極諫,御史康永韶亦偕同官四十一人爭之,伏哭文華門,竟得如禮。

其年九月,彗星見。元率諸給事上言:

入春以來,災異疊至,近又彗星見東方,光拂台垣,皆陰盛陽微之證。臣聞君之與後,猶天之與地,不可得而參貳也。傳聞宮中乃有盛寵,匹耦中宮。尚書姚夔等向嘗言之,陛下謂「內事朕自裁置」。屏息傾聽,將及半載,而昭德宮進膳未聞少減,中宮未聞少增。夫宮闈雖遠,而視聽猶咫尺,衽席之微,謫見玄象,不可不懼。且陛下富有春秋,而震位尚虛。豈可以宗社大計一付之愛專情一之人,而不求所以固國本安民心哉。願明伉儷之義,嚴嫡妾之防。俾尊卑較然,各安其分。本支百世之基,實在於此。

四方旱澇相仍,民困日棘,荊、襄流民告變。陛下作民父母,初無人敬惕,僅循故事,付部施行。而戶部尚書馬昂,凡有奏報,遇上意喜,則曰「移所司處置」;遇上意怒,則曰「事窒難行」;微有利害,即乞聖裁。首鼠依違,民更何望。惟亟罷征稅,發內帑,遣官振贍,庶可少慰人心。

陛下崇信異教,每遇生愍之辰,輒重糜資財,廣建齋醮。而西僧札實巴等,至加法王諸號,賜予駢蕃。出乘棕輿,導用金吾仗,縉紳避道,奉養過於親王。悖理亂紀,孰甚於此。乞革奪名號,遣還其國,追錄橫賜,用振饑民。仍敕寺觀,永不得再講齋醮,以橐國用。

天下之財,不在官則在民。今公私交困,由玩好太多,賞賚無節。或營立塔寺,或購市珍奇。一物之微,累價巨萬,國帑安得不絀?願屏絕淫巧,停罷宴遊,諸銀場及不急務悉為禁止。

至兩京文武大臣,不乏奸貪,爭為蒙蔽。陛下勿謂其位高而不忍遽去,勿謂其舊臣而姑且寬容。宜令各自陳免,用全大體。其貪位不去者,則言官糾劾。而臣等濫居言路,無補於時,亦望罷歸,為不職戒。

帝優詔褒答之,然竟不能用。

元屢遷都給事中,出為福建右參政。巡視海道,嚴禁越海私販。巨商以重寶賂,元怒叱出之。母憂歸,廬墓三年,服除,起江西參政,卒。

康永韶,字用和,祁門人。舉於鄉,入國學,選授御史。成化初,巡按畿輔,劾尚書馬昂抑市民地。四年偕同官胡深、鄭己等爭慈懿太后山陵事。彗星見,復偕同官上言八事,大旨與元前疏相類。兩京大臣考察庶寮,去留多不當。永韶等復劾大臣行私,且摘刑部主事餘志等十二人罪,為志所訐,俱下詔獄。永韶謫順昌知縣,再調福清、惠安。久之,有薦其知天文者,中旨召還,授欽天監正,進太常少卿,掌監事。永韶為御史有直聲,及是乃更迎合取寵,占候多隱諱,甚者以災為祥。陜西大饑,永韶言:「今春星變當有大咎,賴秦民饑死,足當之,誠國家無疆福。」帝甚悅,中旨擢禮部右侍郎,仍掌監事。坐曆多訛字,落職歸。

胡深,定遠衛人。天順未進士。既爭慈懿太后山陵事,復與同官陳宏、鄭己、何純、方昇、張進祿上疏請斥奸邪,痛詆學士商輅、尚書程信、姚夔、馬昂。帝不納。翌日給事中董旻、陳鶴、胡智亦劾輅等,疏呈御前。故事,諫官彈章非大廷宣讀則封進,未有不讀而面呈者。帝不悅,曰:「大臣進退有體,旻等敢不循舊章亂朝儀耶?」輅等乞休,帝惟聽昂去。夔憤甚,連疏求去。深、旻等復合辭攻,而詆夔甚力。帝怒,下深等九人獄。先是,御史林誠亦嘗劾輅,不納,引病去,帝並屬誠吏。毛弘等皆論救,輅亦請寬之,乃各杖二十,復其官。未幾,深坐按陜時杖殺訴冤者,謫黔陽丞,稍遷鬱林知州,卒。

鄭己,山海衛人。成化二年進士。巡按陜西,請蠲邊地逋賦,分別邊兵,命壯者戰守,老弱耕牧,章下所司。定西侯蔣琬鎮甘肅,己欲按其罪,語洩,為所劾,戍宣府。己性矜傲,時論不甚惜。

董旻,樂平人。成化二年進士。歷吏科都給事中。為吏所訐,下詔獄,謫石臼知縣。孝宗時,卒官四川參議。

強珍,字廷貴,滄州人。成化二年進士。除涇縣知縣。請減額賦,民德之。擢御史。

初,遼東巡撫陳鉞啟釁召敵,敵至,務為蔽欺。巡按御史王崇之劾鉞,鉞大恐。謀之汪直,誣逮崇之下詔獄,輸贖,調延安推官。及直、鉞用兵,方論功而敵大入,中官韋朗、總兵官緱謙等匿不以聞。珍往巡按,請正鉞罪。兵部尚書餘子俊等奏鉞累犯重辟,不當貸。帝弗從。未幾,指揮王全等誘殺朵顏衛人,珍發其狀,全等俱獲罪。直方自矜有大功,聞珍疏怒。適巡邊還,鉞郊迎五十里,訴珍誣已,直益怒,奏珍所劾皆妄。詔遣錦衣千戶蕭聚往勘,械赴京。比至,直先榜掠,然後奏聞,坐奏事不實,當輸贖。詔特謫戍遼東,而責兵部及言官先嘗劾鉞者。居三年,直敗,復珍官,致仕。

弘治初,起山東副使,擢大理少卿。明年,以右僉都御史巡撫宣府。時緱謙已罷,珍奏留謙才力可用。給事中言謙數失機,珍不應奏保,遂改南京右通政。尋以母老乞休,久之卒。

王瑞,字良璧,望江人。成化五年進士。授吏科給事中。嘗於文華殿抗言內寵滋甚,詞氣鯁直。帝震怒,同列戰慄,瑞無懼色。十五年疏請天下進表官各陳地方利病,帝惡其紛擾,杖之。

湖廣、江西撫、按官以所部災傷盜起,請免有司朝覲。瑞等言:「歲侵民困,由有司不職,正當加罪,乃為請留。正官既留,則人才進退,何由審辨?是朝覲、考察兩大典,皆從此廢壞矣。」帝然其言,即命吏部禁之。進都給事中,言:「三載黜陟,朝廷大典。今布、按二司賢否,由撫、按牒報,其餘由布、按評覆。任情毀譽,多至失真。舉劾謬者,請連坐。」十九年冬,瑞以傳奉冗員淆亂仕路,率同官奏曰:「祖宗設官有定員,初無倖進之路,近始有納粟冠帶之制,然止榮其身,不任以職。今倖門大開,鬻販如市。恩典內降,遍及吏胥。武階蔭襲,下逮白丁。或選期未至,超越官資;或外任雜流,驟遷京職。以至廝養賤夫、市井童稚,皆得攀援。妄竊名器,踰濫至此,有識寒心。伏睹英廟復辟,景泰倖用者卒皆罷斥。陛下臨御,天順冒功者一切革除。乞斷自宸衷,悉皆斥汰,以存國體。」御史寶應張稷等亦言:「比來末流賤伎妄廁公卿,屠狗販繒濫居清要。文職有未識一丁,武階亦未挾一矢。白徒驟貴,間歲頻遷,或父子並坐一堂,或兄弟分踞各署。甚有軍匠逃匿,易姓進身;官吏犯贓,隱罪希寵。一日而數十人得官,一署而數百人寄俸。自古以來,有如是之政令否也?」帝得疏,意頗動。居三日,貶李孜省、凌中等四人秩,奪黃謙、錢通等九人官。人心快之。

明年正月,太監尚銘罷斥,而其黨李榮、蕭敬等猶用事。瑞等復奏劾之,不從。

瑞居諫垣十餘年,遷湖廣右參議,謝病歸,卒。

李俊,字子英,岐山人。成化五年進士。除吏科給事中,屢遷都給事中。十五年,帝以李孜省為太常寺丞,俊偕同官言:「孜省本贓吏,不宜玷清班,奉郊廟百神祀。」會御史亦有言,乃改上林監副。

時汪直竊柄,陷馬文升、牟俸遣戍。帝責言官不糾,杖俊及同官二十七人,御史王濬等二十九人。當是時,帝耽於燕樂,群小亂政,屢致災譴。至二十一年正月朔申刻,有星西流,化白氣,聲如雷。帝頗懼,詔求直言,俊率六科諸臣上疏曰:

今之弊政最大且急者,曰近倖干紀也,大臣不職也,爵賞太濫也,工役過煩也,進獻無厭也,流亡未復也。天變之來,率由於此。

夫內侍之設,國初皆有定制。今或一監而叢一二十人,或一事而參五六七輩;或分布籓郡,享王者之奉;或總領邊疆,專大將之權;或依憑左右,援引憸邪;或交通中外,投獻奇巧。司錢穀則法外取財,貢方物則多端責賂,兵民坐困,官吏蒙殃。殺人者見原,僨事者逃罪。如梁芳、韋興、陳喜輩,不可枚舉。惟陛下大施剛斷,無令干紀,奉使於外者悉為召還,用事於內者嚴加省汰;則近倖戢而天意可回矣。

今之大臣,其未進也,非夤緣內臣則不得進;其既進也,非依憑內臣則不得安。此以財貿官,彼以官鬻財,無怪其漁獵四方,而轉輸權貴也。如尚書殷謙、張鵬、李本,侍郎艾福、杜銘、劉俊,皆既老且懦。尚書張鎣、張瑄,侍郎尹直,大理卿田景暘,皆清論不愜。惟陛下大加黜罰,勿為姑息,則大臣知警而天意可回矣。

夫爵以待有德,賞以待有功也。今或無故而爵一庸流,或無功而賞一貴倖。祈雨雪者得美官,進金寶者射厚利。方士獻煉服之書,伶人奏曼延之戲。掾史胥徒皆叨官祿,俳優僧道亦玷班資。一歲而傳奉或至千人,數歲而數千人矣。數千人之祿,歲以數十萬計。是皆國之命脈,民之脂膏,可以養賢士,可以活饑民,誠可惜也。方士道流如左通政李孜省、太常少卿鄧常恩輩,尤為誕妄,此招天變之甚者。乞盡罷傳奉之官,毋令汙玷朝列,則爵賞不濫而天意可回矣。

今都城佛剎迄無寧工,京營軍士不復遺力。如國師繼曉假術濟私,糜耗特甚,中外切齒。願陛下內惜資財,外惜人力,不急之役姑賜停罷,則工役不煩而天意可回矣。

近來規利之徒,率假進奉以耗國財。或錄一方書,市一玩器,購一畫圖,製一簪珥,所費不多,獲利十倍。願陛下洞燭此弊,留府庫之財為軍國之備,則進獻息而天意可回矣。

陜西、河南、山西赤地千里。屍骸枕籍,流亡日多,萑苻可慮。願體天心之仁愛,憫生民之困窮,追錄貴倖鹽課,暫假造寺資財,移振饑民,俾茍存活,則流亡復而天意可回矣。

夫天下譬之人身。人主,元首也;大臣,股肱也;諫官,耳目也;京師,腹心也;籓郡,軀幹也。大臣不職則股肱痿痺,諫官緘默則耳目塗塞,京師不職則腹心受病,籓郡災荒則軀幹削弱,元首豈能宴然而安哉?伏望陛下聽言必行,事天以實。疏斥群小,親近賢臣。咨治道之得失,究前代之興亡。以聖賢之經代方書,以文學之臣代方士。則必有正誼足以廣聖學,讜論足以究天變。而手足便利,耳目聰明,腹心安泰,軀幹強健,元首於是乎大明矣。

帝優詔答之。降孜省上林丞,常恩本寺丞,繼曉革國師為民,令巡按御史追其誥敕。制下,舉朝大悅。五月,俊出為湖廣布政司參議。弘治中,屢官山西參政,卒。

汪奎,字文燦,婺源人。成化二年進士。為秀水知縣,擢御史。

二十一年,星變,偕同官疏陳十事,言:

建言貶謫諸臣,效忠於國,宜復其職。妖僧繼曉結中官梁芳,耗竭內藏,乞治芳罪,斬繼曉都市。傳奉官顧賢等皆中官恒從子而冒錦衣,李孜省小吏而授通政,宜盡斥以清仕路。尚書殷謙、李本,侍郎杜銘、尹直,皆素乏清譽,尚書張鵬、張鎣、張瑄,侍郎杜謙、艾福、馬顯、劉俊,大理卿宋欽,巡撫都御史魯能、馬馴,皆老懦無能,侍郎談倫奔競無恥,巡撫趙文博粗鄙妄為,大理卿田景暘素行不謹,宜令致仕。鎮守、守備內官視天順間逾數倍,作威福,凌虐有司。浙江張慶、四川蔡用得逮治四品以下官,尤傷國體,宜悉撤還。內外坐營、監鎗內官增置過多,皆私役軍士,辦納月錢,多者至二三百人。武將亦皆私役健丁,行伍惟存老弱。勳戚、內官奏乞鹽利,滿載南行,所至張欽賜黃旗,商旅不行,邊儲虧損,並宜嚴禁。陜西、山西、河南頻年水旱,死徙大半,山、陜之民僅存無幾。宜核被災郡縣,概與蠲除。給事張善吉先坐罪謫官,考績至京,昏夜乞憐,得授茲職,大玷清班,宜罷斥。山、陜、河、洛饑民多流鄖、襄,至骨肉相啖。請大發帑庾振濟,消弭他變。」當是時,帝以災變求言,奎疏入,雖觸帝忌,未加譴。無何,有御史失儀,奎當面糾,退朝乃奏。帝以其怠緩,杖之於廷。居數月,復出為夔州通判,討平雲陽劇賊。

孝宗立,量移敘州同知。以薦,擢成都知府。歲饑多盜,振救多復業。三遷廣西左布政使。弘治十四年以右副都御史巡撫貴州。未浹歲,普安賊婦米魯作亂,被劾致仕。正德六年卒。

從子舜民,字從仁。成化十四年進士。授行人,擢御史,出按甘肅。劾中官將帥失事,陳邊計,章數十上。先是,奎杖闕下,舜民扶掖之,帝聞而怒。至是,奏獄情詞不當,貶蒙化衛經歷。

弘治初,遷知東莞,未上,擢江西僉事。善讞獄,剖析如流。其清軍法,後人遵守之。改雲南屯田副使。田為勢要奪者,厘而歸之官。麓川遺孽思祿渡金沙江,據孟密,承檄撫定之。母憂歸。服除,適淮、揚大饑,以故官奉命振濟。用便宜發粟,奏停不急務,活饑民百二十萬人,流冗復業者八千餘戶。進福建按察使。盜竊福清縣庫,或誣其怨家,已成獄。舜民廉得真盜,脫三十人於死,抵誣者罪。歲旱,禱不應。躬蒞福州獄,釋枉繫輕罪者,所部有司皆清獄,遂大雨。歷河南左、右布政使。正德二年以右副都御史撫治鄖陽。甫一月,罷天下巡撫官,改蒞南京都察院,道卒。

奎性簡靜,不茍取與,以篤實見稱。而舜民好學砥行,矯矯持風節,尤負時望。

方星變求言時,九卿各條奏數事,率有所避,無甚激切者,唯奎與李俊等言最直。而武選員外郎崔升、彭綱,主事蘇章,戶部主事周軫,刑部主事李旦皆有言。升、章言宦官妖僧罪,請亟誅竄,而尚書王恕今伊、傅,不宜置南京。綱斥李孜省、繼曉,請誅之以謝天下。軫亦請誅梁芳、李孜省,並汰內侍,罷方書。旦陳十事,且言:「神仙、佛老、外戚、女謁,聲色貨利,奇技淫巧,皆陛下素所惑溺,而左右近習交相誘之。」言甚切。帝以方修省,皆不罪。後以吏盜鬻舊賜外蕃故敕事,下綱、章吏,貶之外。而密諭吏部尚書尹旻出旦等,且書六十人姓名於屏,俟奏遷則貶遠惡地。旦乃與給事中盧瑀、秦昇、童柷同日俱謫。部臣見遠謫者多,有應遷者輒故遲之。升、軫遂得免。

崔升,字廷進,本樂安人。父為彰德庫大使,因家焉。成化五年進士。由工部主事改兵部。稍遷延安知府,四川參政。守官廉,居常服布袍,家童拾馬矢給爨。家居三十年,年八十八卒。子銑,自有傳。

彭綱,清江人。與蘇章、周軫、秦昇、童柷皆成化十一年進士。貶永寧知州,改汝州。鑿渠溉田數千畝。再遷雲南提學副使。

蘇章,餘干人。貶姚安通判,再遷延平知府。有政績。終浙江參政。

周軫,莆田人,副使瑛從子。後進郎中,終山東運使。

李旦,字啟東,獻縣人。成化十七年進士。貶鎮遠通判,未幾卒。

盧瑀,鄞縣人。成化五年進士。為刑科給事中,疏蠲淮、揚逋課十餘萬,清西北勒市戰馬宿弊。嘗觸帝怒,杖之。遷工科都給事中,與昇、柷皆因星變陳言,獲譴。瑀貶長沙通判,終廣平知府。

秦昇,南昌人,貶廣安州同知。

童柷,蘭谿人,貶興國州同知,終袁州知府。

是時,崔陞以請召王恕忤旨,而工部主事王純亦以諫罷王恕被杖謫官。純,仙居人。成化十七年進士。貶思南推官。弘治中,屢遷湖廣提學僉事。

湯鼐,字用之,壽州人。成化十一年進士。授行人,擢御史。

孝宗嗣位,首劾大學士萬安罔上誤國。明日,宣至左順門。中官森列,令跪。鼐曰:「令鼐跪者,旨耶,抑太監意耶?」曰:「有旨。」鼐始跪。及宣旨,言疏已留中。鼐大言:「臣所言國家大事,奈何留中?」已而安斥,鼐亦出畿輔印馬,馳疏言:「陛下視朝之餘,宜御便殿,擇侍臣端方謹厚若劉健、謝遷、程敏政、吳寬者,日與講學論道,以為出治之本。至如內閣尹直、尚書李裕、都御史劉敷、侍郎黃景,奸邪無恥,或夤緣中官進用,或依附佞倖行私。不早驅斥,必累聖明。司禮中官李榮、蕭敬曩為言官劾罷,尋夤緣復入。遂摭言官過,貶竄殆盡,致士氣委靡。宜亟正典刑,勿為姑息。諸傳奉得官者,請悉編置瘴鄉,示天下戒。且召致仕尚書王恕、王竑,都御史彭韶,僉事章懋等,而還建言得罪諸臣,以厲風節。」報聞。

弘治元年正月,鼐又劾禮部尚書周洪謨,侍郎倪岳、張悅,南京兵部尚書馬文升,因言:「少傅劉吉,與萬安、尹直奸貪等耳。安、直斥,而吉獨進官,不以為恥。請大申黜陟,明示勸懲。」又劾李榮、蕭敬,而薦謫降進士李文祥為臺諫。尚書王恕以盛暑請輟經筵,鼐極言不可,語侵恕。

當是時,帝更新庶政,言路大開。新進者爭,欲以功名自見。封章旁午,頗傷激訐,鼐意氣尤銳。其所抨擊,間及海內人望,以故大臣多畏之,而吉尤不能堪。使人啖御史魏璋曰:「君能去鼐,行僉院事矣。」璋欣然,日夜伺鼐短。未幾,而吉人之獄起。

吉人者,長安人。成化末進士,為中書舍人。四川饑,帝遣郎中江漢往振。人言漢不勝任,宜遣四使分道振,且擇才能御史為巡按,庶荒政有裨。因薦給事中宋琮、陳璚、韓鼎,御史曹璘,郎中王沂、洪鐘,員外郎東思誠,評事王寅,理刑知縣韓福及壽州知州劉概可使,而巡按則鼐足任之。璋遂草疏,偽署御史陳景隆等名,言吉人抵抗成命,私立朋黨。帝怒,下人詔獄,令自引其黨。人以鼐、璘、思誠、概、福對。璋又嗾御史陳璧等言:「璘、福、思誠非其黨,其黨則鼐、概及主事李文祥、庶吉士鄒智、知州董人桀是也。概嘗饋鼐白金,貽之書,謂夜夢一人騎牛幾墮,鼐手挽之得不仆,又見鼐手執五色石引牛就道。因解之曰:『人騎牛謂朱,乃國姓。意者國將傾,賴鼐扶之,而引君當道也。』鼐、概等自相標榜,詆毀時政,請並文祥、智、人桀逮治。」疏上,吉從中主之,悉下詔獄,欲盡置之死。

刑部尚書何喬新、侍郎彭韶等持之,外議亦洶洶不平。乃坐概妖言律斬;鼐受賄,戍肅州;人欺罔,削籍;智、文祥、人桀皆謫官。吏部尚書王恕奏曰:「律重妖言,謂造作符讖類耳。概書詞雖妄,良以鼐數言事不避利害,因推詡之。今當以妖言,設有如造亡秦讖者,更何以罪之?」帝得疏意動,命姑繫獄。既而熱審,喬新等言:「概本不應妖言律。且概五歲而孤,無兄弟,母孫氏守節三十年,曾被旌,老病且貧。概死,母必不全,祈聖恩矜恤。」乃減概死,戍海州。

,濟寧人。成化二十年進士。除壽州知州,毀境內淫祠幾盡,三年教化大行。弘治初上言:「刑賞予奪,人主大柄,後世乃有為女子、小人、強臣、外戚所攘竊者,由此輩心險術巧,人主稍加親信,輒墮計中。愛者,乘君之喜而游言以揚之;惡者,乘君之怒而微言以中之,使賢人君子卒受暖昧而去。卿相缺人,則遷延餌引,待有交通請屬軟美易制之人,然後薦用。其剛正不阿者,輒媒孽而放棄之,俟其氣衰慮易,不至大立異同,乃更收錄。巧計既行,刑賞予奪雖名人主獨操,實一出於其所簸弄。迨黨立勢成,復恐一旦敗露,則又極意以排諫諍之士。務使其君孤立於上,耳無聞,目無見,以圖便其私,不至其身與國俱敗不止。故夫刑賞予奪,必由大臣奏請、臺諫集議而後可行。或有矯誣,窮治不輕貸,則讒佞莫能間,而權不下移矣。」考績赴都,遂遇禍,竟卒於戍所。

鼐既戍,無援之者,久之始釋歸。

董人桀,涇縣人。成化末進士。鼐之論暑月輟講也,人桀方謁選,亦抗疏爭,由是知名。授沔陽知州,甫數月,逮繫詔獄,謫四川行都司知事,歷遷河南左布政使。所在盡職業,為民所懷。正德六年,江西盜起,巡撫王哲兵敗召還,擢傑右副都御史代之。未幾卒。

璋既為吉心腹,果擢大理寺丞。坐事下獄,黜為九江同知,悒悒死。

姜綰,字玉卿。弋陽人。成化十四年進士。由景陵知縣擢南京御史。弘治初,陳治道十事。又言午朝宜論大政,毋泛陳細故,皆報聞。

二年二月,南京守備中官蔣琮以蘆場事下綰覆按,琮囑綰求右己。綰疏言:「琮以守備重臣與小民爭利,假公事以適私情。用揭帖而抗詔旨,揚言陰中,脅以必從。其他變亂成法,厥罪有十。以內官侵言官職,罪一。妒害大臣,妄論都御史秦紘,罪二。怒河閘官失迎候,欲奏罷之,罪三。受民詞不由通政,罪四。分遣腹心,侵漁國課,罪五。按季收班匠工銀,罪六。擅收用罷閑都事,罪七。官僚忤意,輒肆中傷,罪八。妄奏主事周琦罪,欺罔朝廷,罪九。保舉罷斥內臣,竊天子威柄,罪十。」事下南京三法司。既,復特遣官覆治以奏。

先是,御史餘濬劾中官陳祖生違制墾後湖田,湖為之淤。奏下南京主事盧錦勘報。錦故與祖生有隙。而給事中方向嘗率同官繆樗等劾祖生及文武大臣不職狀,又因雷震孝陵柏,劾大學士劉吉等十一人,而詆祖生益力。祖生銜向切骨。時向方監後湖黃冊,祖生遂揭向、錦實侵湖田。詔下法司勘。勘未上,而琮為綰所劾。於是琮、祖生及吉合謀削錦籍,謫向官,復逮綰及同官孫紘、劉遜、金章、紀傑、曹玉、譚肅、徐禮、餘濬,給事中繆樗,赴京論鞫,皆謫為州判官。

綰謫判桂陽,量移寧國同知,遷慶遠知府。斬劇賊韋七旋、韋萬妙。其黨糾賊數萬攻城,綰堅守,檄民兵夾擊,破走之。東蘭諸州蠻悉歸侵地。總督劉大夏奇其材,薦為右江兵備副使。思恩知府岑濬逐田州知府岑猛,綰獻策總督潘蕃。蕃令與都指揮金堂合諸路兵大破賊,思恩平。綰條二府形勢,請改設流官,比中土,廷議從之。綰引疾還。俄起河南按察使,尋復以疾歸,卒於家。

餘濬,慈谿人。成化十七年進士。孝宗初,疏請永除納粟入監令。又劾浙江鎮守中官張慶、廣東鎮守中官韋眷,因薦王恕堪內閣,馬文升、彭韶、張悅、阮勤、黃孔昭堪吏部。後湖之勘,自濬啟之。貶平度州判官,終知府。

方向,字與義,桐城人。成化十七年進士。謫雲南多羅驛丞,歷官瓊州知府。入覲時,僕私市一珠,索而投諸海。

繆樗,字全之,溧陽人。成化十一年進士。孝宗初,陳時政八事。因劾大學士尹直等,時號「敢言」。終營州判官。

孫紘,字文冕,鄞人。成化十四年進士。謫膠州判官,遷廣德知州,卒官。紘少貧,人庸書市肉以養母。既通籍,終身不食肉。

劉遜,安福人。成化十四年進士。謫澧州判官,遷武岡知州。岷王不檢下,遜裁抑之,又欲損其歲祿。王怒,奏於朝,徵下詔獄,貶四川行都司斷事,歷湖廣副使。劉瑾徵賄不得,坐缺軍儲被逮,已而釋之。再坐斷獄稽延,罰米百石。先是,榮王乞辰州、常德田二千頃、山場八百里、民舍市廛千餘間,遜與巡撫韓重持勿予。至是,瑾悉予之。部議補遜瓊州副使,瑾勒令致仕。瑾誅,起官,歷福建按察使。

金章等無他表見。

姜洪,字希範,廣德人。成化十四年進士。除盧氏知縣。單騎勸農桑。民姜仲禮願代父死罪,洪奏免之。徵拜御史。

孝宗即位,陳時政八事。歷詆太監蕭敬,內閣萬安、劉吉,學士尹直,侍郎黃景、劉宣,都御史劉敷,尚書李裕、李敏、杜銘,大理丞宋經,而薦致仕尚書王恕、王竑、李秉,去任侍郎謝鐸,編修張元禎,檢討陳獻章,僉事章懋,評事黃仲昭,御史強珍、徐鏞、于大節,給事中王徽、蕭顯、賀欽,員外林俊,主事王純及現任尚書餘子俊、馬文升,巡撫彭韶,侍郎張悅,詹事楊守陳。且言指揮許寧、內官懷恩,並拔出曹輩,足副任使。他所陳,多斥近倖,疏辭幾萬言。帝嘉納之。為所斥者憾不置。

弘治元年,出按湖廣,與督漕都御史秦紘爭文移,被劾。所司白洪無罪。劉吉欲中之,再下禮部會議,遂貶夏縣知縣。御史歐陽旦請召還洪及暢亨等,不納。遷桂林知府。瑤、僮侵擾古田,請兵討平之,擢雲南參政。土官陶洪與八百媳婦約為亂,洪乘間翦滅。歷山東左參政。正德二年遷山西布政使。劉瑾索賀印錢,不應。四年二月,中旨令致仕。瑾誅,起山東左布政使。七年以右副都御史巡撫山西,未滿歲卒。

洪性廉直,身後喪不能舉。天啟初,追謚莊介。

歐陽旦,安福人。成化十七年進士。由休寧知縣擢御史。嘗請逐劉吉,罷皇莊。歷湖廣僉事、浙江副使,終南京右副都御史。

暢亨,字文通,河津人。成化十四年進士。由長垣知縣擢御史,巡按浙江。歲饑,奏罷上供綾紗等物。弘治元年二月,景寧縣屏風山異獸萬餘,大如羊,白色,銜尾浮空去。亨請罷溫、處銀課,而置鎮守中官張慶於法。章下所司,銀課得減,責慶陳狀。慶因訐亨考察不公,停亨俸三月。亨又劾僉事鄒滂,滂亦訐亨。慶等構之,逮亨,謫涇陽知縣。給事中龐泮上疏爭,不聽。

曹璘,字廷暉,襄陽人。成化十四年進士。授行人。久之,選授御史。

孝宗嗣位,疏言:「梓宮發引,陛下宜衰絰杖履送至大明門外,拜哭而別,率宮中行三年喪。貴妃萬氏有罪,宜告於先帝,削其謚,遷葬他所。」帝納其奏,而戒勿言貴妃事。頃之,請進用王恕等諸大臣,復先朝言事于大節等諸臣官,放遣宮中怨女,罷撤監督京營及鎮守四方太監。又言:「梁芳以指揮袁輅獻地建寺,請令襲廣平侯爵。以數畝地得侯,勛臣誰不解體,宜亟為革罷。」疏奏,帝頗采焉。

弘治元年七月上言:「近日星隕地震,金木二星晝現,雷擊禁門,皇陵雨雹,南京內園災,狂夫叫閽,景寧白氣飛騰,而陛下不深求致咎之由,以盡弭災之實。經筵雖御,徒為具文。方舉輒休,暫行遽罷,所謂『一日暴之,十日寒之』者。願日御講殿與儒臣論議,罷斥大學士劉吉等,以消天變。臣昨冬曾請陛下墨衰視政,今每遘節序,輒漸御黃袞,從官朱緋。三年之間,為日有幾,宜但御淺服。且陛下方諒陰,少監郭鏞乃請選妃嬪。雖拒勿納,鏞猶任用,何以解臣民疑。祖宗嚴自宮之禁,今此曹干進紛紜,當論罪。朝廷特設書堂,令翰林官教習內使,本非高皇帝制。詞臣多夤緣以干進,而內官亦且假儒術以文奸,宜速罷之。諸邊有警,輒命京軍北征,此輩驕惰久,不足用。乞自今勿遣,而以出師之費賞邊軍。」帝得疏,不喜,降旨譙讓。

已,出按廣東,訪陳獻章於新會,服其言論,遂引疾歸。居山中讀書,三十年不入城市。

彭程,字萬里,鄱陽人。成化末進士。弘治初,授御史,巡視京城。降人雜處畿甸多為盜,事發則投戚里、奄豎為窟穴。程每先機制之,有發輒得。巡鹽兩浙,代還,巡視光祿。

五年上疏言:「臣適見光祿造皇壇器。皇壇者,先帝修齋行法之所。陛下即位,此類廢斥盡,何復有皇壇煩置器?光祿金錢,悉民膏血。用得其當,猶恐病民,況投之無用地。頃李孜省、繼曉輩倡邪說,而先帝篤信之者,意在遠希福壽也。今二人已伏重辟,則禍患之來,二人尚不能自免,豈能福壽他人。倘陛下果有此舉。宜遏之將萌。如無,請治所司逢迎罪。」帝初無皇壇造器之命,特光祿姑為備。帝得程奏大怒,以為暴揚先帝過,立下錦衣獄。給事中叢蘭亦巡視光祿,繼上疏論之。帝宥蘭,奪光祿卿胡恭等俸,付程刑部定罪。尚書彭韶等擬贖杖還職。帝欲置之死,命系之。韶等復疏救,程子尚三上章乞代父死,終不聽。

是時巡按陜西御史嵩縣李興亦坐酷刑繫獄。及朝審,上興及程罪狀。詔興斬,程及家屬戍隆慶。文武大臣英國公張懋等合疏言:「興所斃多罪犯,不宜當以死。程用諫為職,坐此戍邊,則作奸枉法者何以處之?」尚書王恕又特疏救。乃減興死,杖之百,偕妻子戍賓州,程竟無所減。程母李氏年老無他子,叩闕乞留侍養。南京給事中毛珵等亦奏曰:「昔劉禹錫附王叔文當竄遠方,裴度以其母老為請,得改連州。陛下聖德,非唐中主可比,而程罪亦異禹錫。祈少矜憐,全其母子。」不許。子尚隨父戍所,遂舉廣西鄉試。明年,帝念程母老,放還。其後,劉瑾亂政,追論程巡鹽時稍虧額課,勒其家償。程死久矣,止遣一孫女。罄產不足,則並女鬻之,行道皆為流涕。

龐泮,字元化,天台人。成化二十年進士。授工科給事中。弘治中,中旨取善擊銅鼓者,泮疏諫。屢遷刑科都給事中。副使楊茂元被逮,泮率同列救之,茂元得薄譴。

九年四月,帝以岷王劾武岡知州劉遜,命逮之。泮率同官呂獻等言:「錦衣天子視軍,非不軌及妖言重情不可輕遣。遜所坐微,而王奏牽左證百人,勢難盡逮。宜敕撫、按官體勘。」疏入,忤旨,下泮等四十二人及御史劉紳等二十人詔獄。六科署空,吏部尚書屠滽請令中書代收部院封事。御史張淳奉使還,恥獨不與,抗疏論之。考功郎中儲巏亦諫,滽等復率九卿救之。帝乃釋泮等,皆停俸三月。

中官何鼎以直言下獄,楊鵬、戴禮夤緣入司禮監。泮等言:「鼎狂直宜容。鵬等得罪先朝,俾參機密,害非小。」會御史黃山、張泰等亦以為言。帝怒,詰外廷何由知內廷事,令對狀,停泮等俸半歲。威寧伯王越謀起用,中官蔣琮、李廣有罪,外戚周彧、張鶴齡縱家奴殺人,泮皆極論,直聲甚著。

十一年擢福建右參政。中官奪宋儒黃幹宅為僧庵,泮改為書院以祀幹。遷河南右布政使。中旨取洛陽牡丹,疏請罷之。轉廣西左布政使,致仕。

呂獻,浙江新昌人。成化二十年進士。授刑科給事中。坐事,杖闕廷。弘治時,詔選駙馬。李廣受富人金,陰為地,為獻所發,有直聲。正德中,終南京兵部右侍郎。

葉紳,字廷縉,吳江人。成化末進士。除戶科給事中,改吏科,歷禮科左給事中。

弘治十年,太子年七歲,猶未出閤,紳請擇講官教諭。尋以修省,陳八事。斥中官李廣,又劾尚書徐瓊、童軒、侯瓚,侍郎鄭紀、王宗彞,巡撫都御史劉瓛、張誥、張岫等二十人,乞賜罷斥。而末言「去大奸」,則專劾李廣八大罪:「誑陛下以燒煉,而進不經之藥,罪一。為太子立寄壇,而興煖疏之說,罪二。撥置皇親,希求恩寵,罪三。盜引玉泉,經繞私第,罪四。首開倖門,大肆奸貪,罪五。太常崔志端、真人王應裿輩稱廣為教主真人,廣即代求善官,乞賜玉帶,罪六。假果戶為名,侵奪畿民土地,幾至激變,罪七。四方輸納上供,威取勢逼,致民破產,罪八。內而皇親駙馬事之如父,外而總兵鎮守稱之為公。陛下奈何養此大奸於肘腋,而不思驅斥哉!」御史張縉等亦以為言。帝曰:「姑置之。」踰數月,廣竟得罪飲酖死。

紳又極陳大臣恩蔭葬祭之濫。下所司議,頗有減損。擢尚寶少卿,卒。

胡獻,字時臣,揚州興化人。弘治九年進士。改庶吉士,授御史。踰月,即極論時政數事,言:「屠滽為吏部尚書,王越、李蕙為都御史,皆交通中官李廣得之。廣得售奸,由陛下議政不任大臣,而任廣輩也。祖宗時,恒御內閣商決章奏,經筵日講悉陳時政得失,又不時接見儒臣,願陛下追復舊制。京、通二倉總督、監督內臣,每收米萬石勒白金十兩。以歲運四百萬石計之,人四千兩。又各占斗級二三百人,使納月錢。夫監督倉儲,自有戶部,焉用中官?願賜罷遣。京操軍士自數千里至,而總兵、坐營等官各使分屬辦納月錢,乞嚴革以蘇其困。陛下遇災修省,去春求言,諫官及郎中王雲鳳、主事胡爟皆有論奏,留中不報,雲鳳尋得罪。如此,則與不修省何異?願斷自聖心,凡利弊當興革者,即見施行。東廠校尉,本以緝奸,邇者但為內戚、中官泄憤報怨。如御史武衢忤壽寧侯張鶴齡及太監楊鵬,主事毛廣忤太監韋泰,皆為校尉所發,推求細事,誣以罪名。舉朝皆知其枉,無敢言者。臣亦知今日言之,異日必為所陷,然臣弗懼也。」疏入,鶴齡與泰各疏辨。會給事中胡易劾監庫中官賀彬貪黷八罪,彬亦訐易。帝遂下獻、易詔獄,謫獻藍山丞。久之,釋易。獻未赴官,遷宜陽知縣。馬文升數薦於朝,遷南都察院經歷。武宗即位,擢廣西提學僉事,遷福建提學副使,未任卒。

武衢,沂水人,成化二十年進士,以御史謫雲南通海主簿,終汾州知州。毛廣,平湖人。成化二十年進士。其事蹟無考。胡易,寧都人。弘治三年進士。為吏科給事中。華昶劾程敏政,法司白昂、閔珪據舊章令六科共鞫。東廠劾易等皆昶同僚,不當與訊。得旨下詔獄。昂、珪請罪,皆停俸。比昶獄成,易等猶被繫,大臣以為言,始令復職。

當弘治時,言官以忤內臣得罪者,又有任儀、車梁。

任儀,閬中人。成化二十三年進士,為御史。弘治三年秋,詔修齋於大興隆寺。理刑知縣王嶽騎過之,中使捽辱嶽,使跪於寺前。儀不平,劾中使罪。姓名偶誤,乃並儀下吏。出為中部知縣,終山西參政。

車梁,山西永寧人。弘治三年進士,為御史。十五年條列時政,中言東廠錦衣衛所獲盜,先嚴刑具成案,然後送法司,法司不敢平反。請自今徑送法司,毋先刑訊。章下,未報。主東廠者言梁從父郎中霆先以罪為東廠所發,挾私妄言,遂下梁詔獄。給事御史交章論救,乃得釋,終漢陽知府。

張弘至,字時行,華亭人,南安知府弼子也。舉弘治九年進士,改庶吉士,授兵科給事中。

十二年冬,陳初政漸不克終八事:「初汰傳奉官殆盡;近匠官張廣寧等一傳至百二十餘人,少卿李綸、指揮張已等再傳至百八十餘人。異初政者一。初追戮繼曉,逐番僧、佛子;近齋醮不息。異初政者二。初去萬安、李裕輩,朝彈夕斥;近被劾數十疏,如尚書徐瓊者猶居位。異初政者三。初聖諭有大政召大臣面議;近上下否隔。異初政者四。初撤增設內官;近已還者復去,已革者復增。異初政者五。初慎重詔旨,左右不敢妄干;近陳情乞恩率俞允。異初政者六。初令兵部申舊章,有妄乞升武職者奏治;近乞陞無違拒。異初政者七。初節光祿供億;近冗食日繁,移太倉銀賒市廛物。異初政者八。」帝下所司。

邊將王杲、馬昇、秦恭、陳瑛失機論死,久繫。弘至請速正典刑。親王之籓者,所次舍率營蓆殿,並從官幕次,俱飾絨毯、錦帛,因弘至言多減省。孝宗晚年,從廷臣請,遣官核騰驤四衛虛冒弊,以太監寧瑾言而止。弘至抗章爭,會兵部亦以為言,乃卒核之。

武宗立,以戶科右給事中奉使安南。還遷都給事中,母憂歸卒。

屈伸,字引之,任丘人。成化末進士。選庶吉士,授禮科給事中。

弘治九年詔度僧,禮部爭不得。伸極陳三不可,不納。京師民訛言寇近邊,兵部請榜諭。伸言:「若榜示,人心愈驚。昔漢建始中,都人訛言大水至,議令吏民上城避之。王商不從,頃之果定。今當以為法。」事遂寢。寇犯大同,游擊王杲匿敗績狀。伸率同官發之,並劾罪總兵官王璽等。

屢遷兵科都給事中。泰寧衛部長大掠遼陽,部議令守臣遺書,稱朝廷寬大不究已往,若還所掠,則予重賞。伸等言:「在我示怯弱之形,在彼無創艾之意,非王者威攘之道。前日犯邊不以為罪,今日歸俘反以為功。誨以為盜之利,啟無賴心,又非王者懷柔之道。」帝悟,書不果遣。

已,劾鎮守中官孫振、總兵官蔣驥、巡撫陳瑤僨事罪,帝不問。廣寧復失事,瑤等以捷聞。伸及御史耿明等交章劾其欺罔,乃按治之。

太監苗逵、成國公朱暉等搗巢獲三級,及寇大入固原,不敢救,既而斬獲十二級。先後以捷聞。伸等數劾之。及班師,又極論曰:「暉等西討無功,班師命甫下,將士已入國門,不知奉何詔書。且此一役糜京帑及邊儲共一百六十餘萬兩,而首功止三級。是以五十萬金易一無名之首也,乃所上有功將士至萬餘人。假使馘一渠魁如火篩,或斬級至千百,將竭天下財不足供費,而報功者不知幾萬萬也。暉、逵及都御史史琳、監軍御史王用宜悉置重典。」帝不聽。

雲南有鎮守中官,復遣監丞孫敘鎮金騰,伸等極言不可。錦衣指揮孫鑾坐罪閑住,中旨復之,令掌南鎮撫事。伸等力爭,乃命止帶俸。中旨令指揮胡震分守天津,伸力爭,不聽。鎮守河南中官劉郎乞皂隸,帝命予五十人。故事,尚書僅十二人,伸等力爭,詔止減二十人。自後中官咸援例陳乞,祖制遂壞。

伸居諫垣久,持議侃侃不撓,未及遷而卒。

王獻臣,字敬止,其先吳人,隸籍錦衣衛。弘治六年舉進士。授行人,擢御史。巡大同邊,請亟正諸將姚信、陳廣閉營避寇及馬昇、王杲、秦恭喪師罪,悉蠲大同、延綏旱傷逋賦,以寬軍民。帝多從之。嘗令部卒導從遊山,為東廠緝事者所發,並言其擅委軍政官。徵下詔獄,罪當輸贖。特命杖三十,謫上杭丞。

十七年,復以張天祥事被逮。天祥者,遼東都指揮僉事斌孫也。斌以罪廢,天祥入粟得祖官。有泰寧衛部十餘騎射傷海西貢使,天祥出毛喇關掩殺他衛三十八人以歸,指為射貢使者。巡撫張鼐等奏捷,獻臣疑之。方移牒駁勘,會斌婦弟指揮張茂及子欽與天祥有郤,詐為前屯衛文書呈獻臣,具言劫營事。獻臣即以聞。未報,而獻臣被徵。帝命大理丞吳一貫、錦衣指揮楊玉會新按臣餘水廉勘之,盡得其實。斌等皆論死,天祥斃於獄。

天祥叔父洪屢訟冤,帝密令東廠廉其事,還奏所勘皆誣。帝信之,欲盡反前獄,召內閣劉健等,出東廠揭帖示之,命盡逮一貫等會訊闕下。健等言東廠揭帖不可行於外。既退,復爭之。帝再召見,責健等。健對曰:「獄經法司讞,皆公卿士大夫,言足信。」帝曰:「法司斷獄不當,身且不保,言足信乎?」謝遷曰:「事當從眾,若一二人言,安可信?」健等又言眾證遠,不可悉逮。帝曰:「此大獄,逮千人何恤。茍功罪不明,邊臣孰肯效力者?」健等再四爭執,見帝聲色厲,終不敢深言東廠非。一貫等既至,帝親御午門鞫之,欲抵一貫死。閔珪、載珊力救,乃謫嵩明州同知,獻臣廣東驛丞,水廉雲南布政司照磨,茂父子論死,而斌免,洪反得論功。武宗立,獻臣遷永嘉知縣。

吳一貫,字道夫,海陽人。成化十七年進士。由上高知縣擢御史。弘治中,歷按浙江、福建、南畿,以強幹聞。擢大理右寺丞。畿輔、河南饑,請發粟二十萬石以振,又別請二萬石給京邑及昌平民。既謫官,正德初,遷江西副使。討華林賊有功,進按察使。行軍至奉新卒,士民立忠節祠祀焉。

餘濂,字宗周,都昌人。弘治六年進士。武宗時,終雲南副使。

孝宗勵精圖治,委任大臣,中官勢稍絀。而張天祥及滿倉兒事皆發自東廠,廷議猶為所撓云。滿倉兒事,具《孫磐傳》。

贊曰:御史為朝廷耳目,而給事中典章奏,得爭是非於廷陛間,皆號稱「言路」。天順以後居其職者,振風裁而恥緘默。自天子、大臣、左右近習無不指斥極言。南北交章,連名列署。或遭譴謫,則大臣抗疏論救,以為美談。顧其時門戶未開,名節自勵,未嘗有承意指於政府,效搏噬於權璫,如末季所為者。故其言有當有不當,而其心則公。上者愛國,次亦愛名。然論國事而至於愛名,則將惟其名之可取,而事之得失有所不顧,於匡弼之道或者其未善乎。


\end{pinyinscope}