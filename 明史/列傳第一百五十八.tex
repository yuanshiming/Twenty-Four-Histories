\article{列傳第一百五十八}

\begin{pinyinscope}
馬世龍楊肇基賀虎臣子贊誠沈有容張可大弟可仕魯欽子宗文秦良玉龍在田

馬世龍,字蒼元,寧夏人。由世職舉武會試,歷宣府遊擊。

天啟二年抉永平副總兵。署兵部孫承宗奇其才,薦授署都督僉事,充三屯營總兵官。承宗出鎮,薦為山海總兵,俾領中部,調總兵王世欽、尤世祿分領南北二部。明年正月賜尚方劍,實授府銜。承宗為築壇拜大將,代行授鉞禮,軍馬錢穀盡屬之。尋定分地,世龍居中,駐衛城,世欽南海,世祿北山,並受世龍節制,兵各萬五千人。世龍感承宗知已,頗盡力,與承宗定計出守關外諸城。四年,偕巡撫喻安性及袁崇煥東巡廣寧,又與崇煥、世欽航海抵蓋套,相度形勢而還。敘勞,加右都督。

當是時,承宗統士馬十餘萬,用將校數百人,歲費軍儲數百萬。諸有求於承宗者,率因世龍,不得則大恚。而世龍貌偉,中實怯,忌承宗者多擊世龍以撼之。承宗抗辯於朝曰:「人謂其貪淫朘削,臣敢以百口保其必無。」帝以承宗故,不問。

五年九月,世龍誤信降人劉伯漒言、遣前鋒副將魯之甲、參將李承先率師襲取耀州,敗沒。言官交章劾奏,嚴旨切責,令戴罪圖功。時魏忠賢方以清君側疑承宗,其黨攻世龍者並及承宗。承宗不安其位去,以兵部尚書高第來代。職方主事徐日久者,先佐第撓遼事,及從第贊畫,力攻世龍。世龍陰結忠賢,反削日久籍。其冬,世龍亦謝病去。

崇禎元年,王在晉為尚書。世龍上疏極論其罪,有詔逮世龍,久不至。在晉罷,始詣獄。二年冬,都城戒嚴。刑部尚書喬允升薦世龍才,詔圖功自贖。會祖大壽師潰,京師大震。承宗再起督師,以便宜遣世龍馳諭大壽聽命。及滿桂戰死,遂令世龍代為總理,賜尚方劍,盡統諸鎮援師。

三年三月進左都督。時遵化、永平,遷安、灤州四城失守已三月。承宗、大壽隔關門,與世龍諸軍聲息斷絕。帝急詔四方兵勤王,昌平尤世威、薊鎮楊肇基、保定曹鳴雷、山海宋偉、山西王國樑、固原楊麒、延綏吳自勉、臨洮王承恩、寧夏尤世祿、甘肅楊嘉謨,所將皆諸邊銳卒;內地則山東、河南、南都、湖廣、浙江、江西、福建、四川諸軍,亦先後至。并壁薊門,觀望不進。給事中張第元上言:「世龍在關數載,績效無聞,非若衛、霍之儔,功名足以服人也。諸帥宿將,非世龍偏裨,欲驅策節制,誰能甘之。師老財匱,銳氣日消,延及夏秋,將有不可言者。」帝以世龍方規進取,不納其言。時大壽於五月十日薄灤州。明日,世龍等以師會。又明日復其城。十三日,遊擊靳國臣復遷安。明日,副將何可綱復永平。又二日,別將復遵化。閱五月,四城始復。論功,大壽最,世祿次之。世龍加太子少保,廕本衛世千戶。八月復謝病歸。

六年五月,插漢虎墩兔合套寇犯寧夏,總兵賀虎臣戰歿,詔起世龍代之。世龍生長寧夏,習其形勢,大修戰備。明年正月,二部入犯,遣參將卜應第大破之,斬首二百有奇。踰月,套寇犯賀蘭山。世龍遣降丁潛入其營,馘其長撒兒甲,斬級如前。未幾,插部大舉入寇。世龍遣副將婁光先等分五道伏要害,而己中道待之,夾擊,斬首八百有奇。巡撫王振奇亦斬三百餘級。寇復犯河西玉泉宮,世龍復邀斬五百餘。其年七月犯棗園堡,世龍又大敗之,俘斬一千有奇。世龍半歲中屢奏大捷,威名震西塞。無何,卒於官,年四十餘。後論功,贈太子太傅,世錦衣僉事,賜恤如制。

楊肇基,沂州衛人。起家世職,積官至大同總兵。天啟二年,妖賊徐鴻儒反山東,連陷鄆、鉅野、鄒、滕、嶧,眾至數萬。巡撫趙彥任都司楊國棟、廖棟檄所部練民兵,增諸要地守卒。時肇基方家居,彥因即家薦起之,為山東總兵官討賊。未至,棟及國棟等攻鄒,兵潰,遊擊張榜戰死。彥方視師兗州,遇賊。肇基至,急迎戰,而令國棟及棟夾擊,大敗之橫河。時賊精銳聚鄒、滕中道,肇基令遊兵綴賊鄒城,而以大軍擊賊黃陰、紀王城,大敗賊,蹙而殪之嶧山,遂圍鄒。國棟等亦先後收復鄆、鉅野、嶧、滕諸縣,又大破之於沙河。乃築長圍攻鄒。圍三月,賊食盡,其黨出降,遂擒鴻儒。獻俘,磔於市,賊平。肇基由署都督僉事進右都督,廕本衛世千戶。尋代沈有容鎮登、萊。改延綏,以擊套寇功,進左都督,廕錦衣千戶,屢加太子太保。崇禎元年移薊鎮西協。二年冬,大清兵克三屯營。肇基乘間收復,困守數月,卒全孤城。廕錦衣世千戶。已,錄恢復四城功,加太子太師,改廕錦衣僉事。明年卒官。子御蕃,見《徐從治傳》。

賀虎臣,保定人。天啟初,歷天津海防遊擊,登萊參將,移兗州。六年遷延綏副總兵。河套寇大舉入犯,從帥楊肇基協擊,大破之。加署都督僉事。崇禎二年,捕誅階州叛卒周大旺等。擢總兵官,鎮守寧夏。關中賊大起,王嘉胤陷清水營,殺遊擊李顯宗,遂陷府谷。其黨李老柴應之,嘯聚三千餘人,攻合水。總督楊鶴檄虎臣往討,擊之盤谷,俘馘六百有奇。已,擊斬慶陽賊渠劉六。四年,神一元陷保安。延安告急,延綏撫鎮皆東援陜西。巡撫練國事檄虎臣及副將李卑援剿。虎臣等遂進圍保安,賊引河套數千騎挫虎臣軍。會張應昌擊敗之,賊眾棄城去。虎臣等前後獲首功一千九百。明年,可天飛、郝臨庵、劉道江、李都司再圍合水。虎臣偕臨洮曹文詔、甘肅楊嘉謨、固原楊麒合擊,大破賊甘泉之虎兕凹,斬首七百有奇,賊大困。

六年五月,插漢虎墩兔合套寇五萬騎自清水、橫城分道入。守備姚之夔等不能禦,沙井驛副將史開先、臨河堡參將張問政、岳家樓守備趙訪皆潰逃。寇遂進薄靈州,虎臣急領千騎入守。旋盡勒城中兵出擊,次大沙井。寇從漢伯堡突至,虎臣軍未及布陳,且眾寡不敵,遂戰死。子贊挾五十騎突重圍出。事聞,贈虎臣都督僉事,賜祭葬,世廕指揮僉事。尋錄先後剿寇功,再贈都督同知,世廕錦衣副千戶。

贊,勇敢有父風。既承廕,尋舉武進士。積官至京營副將。崇禎十七年三月,李自成薄京師,京軍六大營分列城外,皆不敢戰,或棄甲降。贊獨率部卒迎擊,中矢死。

弟誠,身長七尺,美鬚髯,為諸生,以忠義自許。兄誡襲副千戶,早卒,無子,誠當襲,以讓其弟詮。及賊陷保定,家人勸易衣遁。叱曰:「吾忠臣子,偷生而逃,何以見先將軍地下!」遂偕妻女投井死。

沈有容,字士弘,宣城入。僉事寵之孫也。幼走馬擊劍,好兵略。舉萬曆七年武鄉試。薊遼總督梁夢龍見而異之,用為昌平千總。復受知總督張佳胤,調薊鎮東路,轄南兵後營。十二年秋,朵顏長昂以三千騎犯劉家口。有容夜半率健卒二十九人迎擊,身中二矢,斬首六級,寇退乃還,由是知名。遼東巡撫顧養謙召隸麾下,俾練火器。十四年從李成梁出塞,抵可可毋林,斬馘多。明年再出,亦有功。成梁攻北關,有容陷陣,馬再易再斃,卒拔其城。錄功,世廕千戶。遷都司僉書,守浮屠谷。

從宋應昌援朝鮮,乞歸。日本封事壞,福建巡撫金學曾欲用奇搗其穴,起有容守浯嶼、銅山。二十九年,倭掠諸寨,有容擊敗之。逾月,與銅山把總張萬紀敗倭彭山洋。倭據東番。有容守石湖,謀盡殲之,以二十一舟出海,遇風,存十四舟。過彭湖,與倭遇,格殺數人,縱火沈其六舟,斬首十五級,奪還男婦三百七十餘人。倭遂去東番,海上息肩者十年。捷聞,文武將吏悉敘功,有容賚白金而已。

三十二年七月,西洋紅毛番長韋麻郎駕三大艘至彭湖,求互市,稅使高寀召之也。有容白當事,自請往諭。見麻郎,指陳利害。麻郎悟,呼寀使者,索還所賂寀金,揚帆去。改僉書浙江都司。由浙江遊擊調天津,遷溫處參將,罷歸。四十四年,倭犯福建。巡撫黃承元請特設水師,起有容統之,擒倭東沙。尋招降巨寇袁進、李忠,散遣其眾。

泰昌元年,遼事棘,始設山東副總兵,駐登州,以命有容。天啟改元,遼、沈相繼覆。熊廷弼建三方布置之議,以陶朗先巡撫登、萊,而擢有容都督僉事,充總兵官,登、萊遂為重鎮。八月,毛文龍有鎮江之捷。詔有容統水師萬,偕天津水師直抵鎮江策應。有容歎曰:「率一旅之師,當方張之敵,吾知其不克濟也。」無何,鎮江果失,水師遂不進。明年,廣寧覆,遼民走避諸島,日望登師救援。朗先下令,敢渡一人者斬。有容爭之,立命數十艘往,獲濟者數萬人。時金、復、蓋三衛俱空無人,有欲據守金州者。有容言金州孤懸海外,登州、皮島俱遠隔大洋,聲援不及,不可守。迨文龍取金州,未幾復失。四年,有容以年老乞骸骨,歸,卒。贈都督同知,賜祭葬。

張可大,字觀甫,應天人。世襲南京羽林左衛千戶,舉萬曆二十九年武會試,授建昌守備。遷浙江都司僉書,分守瓜洲、儀真,江洋大盜斂迹。稅監魯保死,淮撫李三才令可大錄其貲。保家饋重賄,卻不受。葉向高赴召過儀,見而異之,曰:「此不特良將,且良吏也。」遷劉河遊擊,改廣東高肇參將。調浙江舟山。奉命征黎,與總兵王鳴鶴用黑番為導,搗其巢,黎乃滅。

舟山,宋昌國城也,居海中,有七十二墺,為浙東要害。可大條上八議,皆碩畫。倭犯五罩湖、白沙港、茶山,潭頭,連敗之,加副總兵。城久圮,可大與副使蔡獻臣築之,兩月工竣。城內外田數千畝,海潮害稼。可大築碶蓄淡水,遂為膏腴。民稱曰:「張公碶」。天啟元年以都指揮使掌南京錦衣衛。六年擢都督僉事,僉書南京右府。崇禎元年出為登萊總兵官。會議裁登、萊撫鎮,乃命以總兵官視登州副總兵事,而巡撫遂罷不設。可大盡心海防,親歷巡視,圖沿海地形、兵力強弱,為《海防圖說》上之。二年冬,白蓮賊餘黨圍萊陽,可大擊破之,焚其六砦,斬偽國公二人,圍遂解。京師被兵,可大入衛,守西直、廣寧諸門。明年,以勤王功,升都督同知。

劉興治反東江,遂奉詔還鎮。已而四城並復,朝議復設登萊巡撫,以孫元化為之。元化率關外八千人至,強半皆遼人。可大慮有變,屢言於元化,不聽。

四年七月,錄前守城功,進右都督。十月,僉書南京左府,兼督池河、浦口二軍,登人泣留之。未行而孔有德反吳橋,東陷六城。可大急往剿,元化檄止之,不聽。次萊州,遇元化,復為所阻,乃還鎮。歲將晏,有德暮薄城。可大請擊之,元化持撫議,不許。可大陳利害甚切,元化期明歲元日發兵合擊。至期,元化兵不發。明日,合兵戰城東,可大兵屢勝。元化部卒皆遼人,親黨多,無鬥志。其將張燾先走,可大兵亦敗。中軍管維城,遊擊陳良謨,守備盛洛、姚士良皆戰死。燾兵半降有德,遣歸為內應。元化開門納之,可大諫,不聽。夜半賊至,城遂陷。可大時守水城,撫膺大慟。解所佩印付旗鼓,間道走濟南上之。還家辭母,令弟可度、子鹿徵奉母航海趨天津。而以佩劍付部將,盡斬諸婢妾,遂投繯死。事聞,贈特進榮祿大夫、太子少傅,謚莊節,賜祭葬,予世廕,建祠曰:「旌忠」。

可大好學能詩,敦節行,有儒將風。為南京錦衣時,歐陽暉由刑部主事謫本衛知事,嘗賦詩有「陰霾國事非」句,揚州知府劉鐸書之扇,贈一僧。惡鐸者譖之魏忠賢,暉、鐸俱被逮。可大約束旗尉,捐奉助之,卜室處其妻子。其尚義類如此。

弟可仕,字文峙,以字行。隱居博學,嘗輯明布衣詩一百卷。

魯欽,長清人。萬曆中,歷山西副總兵。天啟元年遷神機營左副將。尋擢署都督僉事,充保定總兵官。奢崇明、安邦彥並反,貴州總兵張彥方在圍中,而總理杜文煥稱病。明年十月用欽代文煥,命總川、忠、湖廣漢土軍刻期解圍。未至,圍已解,欽馳赴貴陽。三年正月,巡撫王三善大敗於陸廣,群苗宋萬化、何中尉等蜂起。欽佐三善防剿,率諸將擒中尉、萬化,遂進營紅崖。紅崖者,崇明敗走處也。三善謀大舉深入,欽及總兵官馬炯、張彥方,諸道監司尹伸、岳具仰、向日升、楊世賞各以兵從,五戰,斬首萬八千,直抵大方。四年正月,三善敗歿於內莊,欽等以殘卒還。命戴罪辦賊。

都勻凱里土司者,運道咽喉也,邦彥結諸蠻困其城,長官楊世蔚不能守。總督蔡復一遣欽及總兵官劉超救之,拔賊巖頭寨,遂移師克平茶。已而邦彥盡驅羅鬼,結四十營於斑鳩灣後寨,互二十餘里,分犯普定。復一令欽與總兵官黃鉞分道禦之。欽率部將張雲鵬、劉志敏、鄧等大敗賊汪家沖。鉞及參政陸夢龍、副使楊世賞亦大敗賊蔣義寨。合追至河,斬首千五百餘級。搜山,復斬六百餘級。尹伸守普定,亦敗賊兵,與大軍會,共剪水外逆苗。邦彥勢窘,渡河西奔。欽、鉞督諸將窮追,夢龍等分駐三岔河岸為後勁。前鋒雲鵬、等深入織金,先後斬首千餘級。

復一上其功,言:「欽廉勇。雖名總理,權力不當一偏裨。舊撫臣三善及諸監軍,人人為大帥,內莊失律,欽不當獨任大帥罪。臣至黔,以諸道監軍兵盡屬欽,每戰身先士卒。欽敗可原,勝足錄。當免其戴罪,仍以功論。」從之。明年正月,欽等渡河還,中伏,敗死者數千人。充為事官,立功自贖。

自平越至興隆、清平二衛,苗二百餘寨盤踞其間,以長田之天保、阿秧為魁。邦彥初反,授二酋都督,使通下六衛聲息。是年春,寇石阡、餘慶。監軍按察使來斯行啖阿秧,使圖天保,阿秧反以情告。斯行乃誘斬阿秧,議討天保,會以疾去。復一令貴陽同知周鴻圖代為監軍,阿秧弟阿買與天保請兵邦彥,復兄仇。復一以兵事屬鴻圖及欽,而遣參將胡從儀、楊明楷等佐之。欽等三道進,大戰米墩山,生擒天保及阿買,先後斬賊魁五十四人,獲首功二千三百五十,破焚百七十四寨。盛夏興師,將士冒暑雨,衝嵐瘴。劇寇盡除,土人謂二百年所未有。復一既奏功,未報而卒。監軍御史傅宗龍復以為言,乃命欽總理如故,鴻圖授平越知府。

六年三月,邦彥復大舉入寇。欽禦之河上,連戰數日,殺傷相當。夜半,賊直逼欽壘。將十逃竄,欽遂自刎。諸營盡潰,賊勢復張。

欽勇敢善戰,為西南大將之冠。莊烈帝嗣位,贈少保、左都督,世廕指揮僉事,賜祭葬,建祠曰:「旌忠」。

子宗文承廕。崇禎中,以薊鎮副總兵為總督吳阿衡中軍。十一年冬,牆子嶺失事,與阿衡並力戰死。

秦良玉,忠州人,嫁石砫宣撫使馬千乘。萬曆二十七年,千乘以三千人從征播州,良玉別統精卒五百裹糧自隨,與副將周國柱扼賊鄧坎。明年正月二日,賊乘官軍宴,夜襲。良玉夫婦首擊敗之,追入賊境,連破金築等七寨。已,偕酉陽諸軍直取桑木關,大敗賊眾,為南川路戰功第一。賊平,良玉不言功。其後,千乘為部民所訟,瘐死雲陽獄,良玉代領其職。良玉為人饒膽智,善騎射,兼通詞翰,儀度嫻雅。而馭下嚴峻,每行軍發令,戎伍肅然。所部號白桿兵,為遠近所憚。

泰昌時,徵其兵援遼。良玉遣兄邦屏、弟民屏先以數千人往。朝命賜良玉三品服,授邦屏都司僉書,民屏守備。天啟元年,邦屏渡渾河戰死,民屏突圍出。良玉自統精卒三千赴之,所過秋毫無犯。詔加二品服,即予封誥。子祥麟授指揮使。良玉陳邦屏死狀,請優恤。因言:「臣自征播以來,所建之功,不滿讒妒口,貝錦高張,忠誠孰表。」帝優詔報之。兵部尚書張鶴鳴言:「渾河血戰,首功數千,實石砫、酉陽二土司功。邦屏既歿,良玉即遣使入都,製冬衣一千五百,分給殘卒,而身督精兵三千抵榆關。上急公家難,下復私門仇,氣甚壯。宜錄邦屏子,進民屏官。」乃贈邦屏都督僉事,錫世廕,與陳策等合祠;民屏進都司僉書。

部議再徵兵二千。良玉與民屏馳還,抵家甫一日,而奢崇明黨樊龍反重慶,齎金帛結援。良玉斬其使,即發兵率民屏及邦屏子翼明、拱明溯流西上,度渝城,奄至重慶南坪關,扼賊歸路。伏兵襲兩河,焚其舟。分兵守忠州,馳檄夔州,令急防翟塘上下。賊出戰,即敗歸。良玉上其狀,擢民屏參將,翼明、拱明守備。」

已而奢崇明圍成都急,巡撫朱燮元檄良玉討。時諸土司皆貪賊賂,逗遛不進。獨良玉鼓行而西,收新都,長驅抵成都,賊遂解圍去。良玉乃還軍攻二郎關,民屏先登,已,克佛圖關,復重慶。良玉初舉兵,即以疏聞。命封夫人,錫誥命,至是復授都督僉事,充總兵官。命祥麟為宜慰使,民屏進副總兵,翼明、拱明進參將。良玉益感奮,先後攻克紅崖墩、觀音寺、青山墩諸大巢,蜀賊底定。復以援貴州功,數賚金幣。

三年六月,良玉上言:「臣率翼明、拱明提兵裹糧,累奏紅崖墩諸捷。乃行間諸將,未睹賊面,攘臂誇張,及乎對壘,聞風先遁。敗於賊者,唯恐人之勝;怯於賊者,唯恐人之強。如總兵李維新,渡河一戰,敗衄歸營,反閉門拒臣,不容一見。以六尺軀鬚眉男子,忌一巾幗婦人,靜夜思之,亦當愧死。」帝優詔報之,命文武大吏皆以禮待,不得疑忌。

是年,民屏從巡撫王三善抵陸廣,兵敗先遁。其冬,從戰大方,屢捷。明年正月,退師。賊來襲,戰死。二子佐明、祚明得脫,皆重傷。良玉請恤,贈都督同知,立祠賜祭,官二子。而是時翼明、拱明皆進官至副總兵。

崇禎三年,永平四城失守。良玉與翼明奉詔勤王,出家財濟餉。莊烈帝優詔褒美,召見平臺,賜良玉綵幣羊酒,賦四詩旌其功。會四城復,乃命良玉歸,而翼明駐近畿。明年築大凌河城。翼明以萬人護築,城成,命撤兵還鎮。七年,流賊陷河南,加翼明總兵官,督軍赴討。明年,鄧死,以所部皆蜀人,命翼明將之,連破賊於青崖河、吳家堰、袁家坪,扼賊走鄖西路。翼明性恇怯,部將連敗,不以實聞,革都督銜,貶二秩辦賊。已,從盧象升逐賊穀城。賊走均州,翼明敗之青石鋪。賊入山自保,翼明攻破之。連破賊界山、三道河、花園溝,擒黑煞神、飛山虎。賊出沒鄖、襄間,撫治鄖陽苗胙土遣使招降,翼明贊其事,為賊所紿,卒不絳。翼明、胙土皆被劾。已而賊犯襄陽,翼明連戰得利,屯兵廟灘,以扼漢江之淺。而羅汝才、劉國能自深水以渡,遂大擾蘄、黃間。帝以鄖、襄屬邑盡殘,罷胙土,切責翼明,尋亦被劾解官。而良玉自京師還,不復援剿,專辦蜀賊。

七年二月,賊陷夔州,圍太平,良玉至乃走。十三年扼羅汝才於巫山。汝才犯夔州,良玉師至乃去。已,邀之馬家寨,斬首六百,追敗之留馬埡,斬其魁東山虎。復合他將大敗之譚家坪北山,又破之仙寺嶺。良玉奪汝才大纛,擒其渠副塌天,賊勢漸衰。

當是時,督師楊嗣昌盡驅賊入川。川撫邵捷春提弱卒二萬守重慶,所倚惟良玉及張令二軍。綿州知州陸遜之罷官歸,捷春使按營壘。見良玉軍整,心異之。良玉為置酒。語遜之曰:「邵公不知兵。吾一婦人,受國恩,誼應死,獨恨與邵公同死耳。」遜之問故,良玉曰:「邵公移我自近,去所駐重慶僅三四十里,而遣張令守黃泥窪,殊失地利。賊據歸、巫萬山巔,俯瞰吾營。鐵騎建瓴下,張令必破。令破及我,我敗尚能救重慶急乎?且督師以蜀為壑,無愚智知之。邵公不以此時爭山奪險,令賊無敢即我,而坐以設防,此敗道也。」遜之深然之。已而捷春移營大昌,監軍萬元吉亦進屯巫山,與相應援。其年十月,張獻忠連破官軍於觀音巖、三黃嶺,遂從上馬渡過軍。良玉偕張令急扼之竹箘坪,挫其鋒。會令為賊所殪,良玉趨救不克,轉鬥復敗,所部三萬人略盡。乃單騎見捷春請曰:「事急矣,盡發吾溪峒卒,可得二萬。我自廩其半,半餼之官,猶足辦賊。」捷春見嗣昌與己左,而倉無見糧,謝其計不用。良玉乃歎息歸。時搖、黃十三家賊橫蜀中。有秦纘勳者,良玉族人也,為賊耳目,被擒,殺獄卒遁去。良玉捕執以獻,無脫者。

張獻忠盡陷楚地,將復入蜀。良玉圖全蜀形勢上之巡撫陳士奇,請益兵守十三隘,士奇不能用。復上之巡按劉之勃,之勃許之,而無兵可發。十七年春,獻忠遂長驅犯夔州。良玉馳援,眾寡不敵,潰。及全蜀盡陷,良玉慷慨語其眾曰:「吾兄弟二人皆死王事,吾以一孱婦蒙國恩二十年,今不幸至此,其敢以餘年事逆賊哉!」悉召所部約曰:「有從賊者,族無赦!」乃分兵守四境。賊遍招土司,獨無敢至石砫者。後獻忠死,良玉竟以壽終。

翼明既罷,崇禎十六年冬,起四川總兵官。道梗,命不達。而拱明值普名聲之亂,與賊鬥死,贈恤如制。

龍在田,石屏州土官舍人也。天啟二年,雲南賊安效良、張世臣等為亂。在田與阿迷普名聲、武定吾必奎等征討,數有功,得為土守備。新平賊剽石屏,安效良攻沾益,在田俱破走之。巡撫閔洪學上其功,擢坐營都司。

崇禎二年與必奎收復烏撒。八年,流賊犯鳳陽,詔征雲南土兵。在田率所部應詔,擊賊湖廣、河南,頻有功,擢副總兵。總理盧象昇檄討襄陽賊,至則象升已奉詔勤王,命屬熊文燦。十年三月擊擒大盜郭三海。十一年九月大破賀一龍、李萬慶於雙溝,進都督同知。明年三月大破賊固始,斬首三千五百有奇。張獻忠之叛也,文燦命在田駐穀城,遏賊東突。諸將多忌在田,讒言日興。及文燦被逮,在田亦罷歸,還至貴州,擊平叛賊安隴壁。

十五年夏,中原盜益熾。在田上疏曰:「臣以石屏世弁,因流氛震陵,奮激國難,捐貲募精卒九千五百,戰象四,戰馬二千,入楚、豫破賊。賊不敢窺江北陵寢,滇兵有力焉。五載捷二十有八,忌口中阻,逼臣病歸。自臣罷,親籓辱,名城屢陷。臣妄謂討寇必須南兵。蓋諸將所統多烏合,遇寇即逃,乏餉即噪。滇兵萬里長驅,家人父子同志,非若他軍易潰也。且一歲中,秋冬氣涼,賊得馳騁。春夏即入山避暑,養銳而出,故其氣益盛。夫平原戰既不勝,山蹊又莫敢攖,師老財殫,蕩平何日。滇兵輕走遠跳,善搜山。臣願整萬眾,力掃秦、楚、豫、皖諸寇,不滅不止。望速給行糧,沿途接濟。臣誓捐軀報國,言而不效,甘伏斧金質。」帝壯之,下兵部議,寢不行。

踰二載,乙酉八月,吾必奎叛。黔國公沐天波檄在田及寧州土知州祿永命協討,擊擒之。未幾,沙定洲作亂,據雲南府,在田不敢擊。明年,定州攻在田不下,移攻寧州,尋陷嶍峨,在田走大理。又明年,孫可望等至貴州,在田說令攻定洲,定洲迄破滅。在田歸,卒於家。

贊曰:馬世龍等值邊陲多事,奮其勇略,著績戎行,或捐軀力戰,身膏原野,可謂無忝爪牙之任矣。夫摧鋒陷敵,宿將猶難,而秦良玉一土舍婦人,提兵裹糧,崎嶇轉鬥,其急公赴義有足多者。彼仗鉞臨戎,縮朒觀望者,視此能無愧乎!


\end{pinyinscope}