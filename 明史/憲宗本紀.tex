\article{憲宗本紀}

\begin{pinyinscope}
憲宗繼天凝道誠明仁敬崇文肅武宏德聖孝純皇帝,諱見深,英宗長子也。母貴妃周氏。初名見浚。英宗留瓦剌,皇太后命立為皇太子。景泰三年,廢為沂王。天順元年,復立為皇太子,改名見深。

天順八年正月,英宗崩。乙亥,即皇帝位。以明年為成化元年,大赦天下。免明年田租三之一。浙江、江西、福建、陜西、臨清鎮守內外官,諸邊鎮守內官,正統間所無者悉罷之。下番使者、緝事官校皆召還。二月庚子,始以內批授官。三月甲寅朔,尊皇后為慈懿皇太后,貴妃周氏為皇太后。戊午,放宮人。丙寅,毀錦衣衛新獄。庚午,賜彭教等進士及第、出身有差。癸酉,詔內閣九卿考核天下方面官。戊寅,復立團營。

夏四月癸未朔,日當食,不見。五月丁巳,大風雨雹,敕群臣修省。庚申,葬睿皇帝於裕陵。

秋七月壬申,立吳氏為皇后。八月癸未,御經筵。甲申,命儒臣日講。癸卯,廢皇后吳氏。下太監牛玉於獄。

冬十月壬辰,立王氏為皇后。甲辰,立武舉法。十二月甲辰,免京官雜犯罪。

是年,兩畿、川、廣、荊、襄盜賊大起。道路不通。安南、烏斯藏入貢。

成化元年春正月乙卯,享太廟。己未,大祀天地於南郊。甲子,都督同知趙輔為征夷將軍,充總兵官,僉都御史韓雍贊理軍務,討廣西叛瑤。二月戊子,祭社稷。甲午,耕耤田。三月庚戌,四川山都掌蠻亂。丁巳,釋奠於先師孔子。

夏五月辛酉,大雨雹。壬戌,避正殿減膳,敕群臣修省。

秋七月己酉,免天下軍衛屯糧十之三。甲子,振兩畿、浙江、河南饑。八月丁丑,工部侍郎沈義、僉都御史吳琛振撫兩畿饑民。辛巳,瘞暴骸。庚寅,毛里孩犯延綏,總兵官房能敗之。

冬十二月癸卯,撫寧伯朱永為靖虜將軍,充總兵官,太監唐慎監軍,工部尚書白圭提督軍務,討荊、襄賊。是月,韓雍大破大藤峽瑤,改名峽曰「斷藤」。

是年,琉球、哈密、爪哇、烏斯藏入貢。

二年春正月戊申,罷團營。乙卯,大祀天地於南郊。辛酉,英宗神主祔太廟。二月癸未,禮部侍郎鄒乾巡視畿內饑民。三月甲辰,賜羅倫等進士及第、出身有差。己酉,李賢父卒,乞終制,不許。乙卯,朱永大破荊、襄賊劉通於南漳。閏月癸,振南畿饑。乙未,朱永擊擒劉通,其黨石龍遁,轉掠四川。

夏五月癸酉,修撰羅倫以論李賢起復謫福建市舶司提舉。己卯,禁侵損古帝王、忠臣、烈士、名賢陵墓。六月甲辰,趙輔師還。乙巳,免今年天下屯糧十之三。壬子,楊信為平虜將軍,充總兵官,太監裴當監督軍務,禦寇延綏。

秋七月辛巳,封弟見治為忻王,見沛徽王。戊戌,毛里孩犯固原。八月丁巳,犯寧夏,都指揮焦政戰死。丁卯,諭祭于謙,復其子冕官。

冬十月丁未,朱永擊擒石龍,賊平,進永爵為侯。十二月甲寅,李賢卒。丙辰,太常寺少卿兼翰林院侍讀學士劉定之入閣預機務。是月,斷藤峽賊復起。

是年,哈密、琉球、安南、烏斯藏、瓦剌入貢。

三年春正月己卯,大祀天地於南郊。丙申,撫寧侯朱永為平胡將軍,充總兵官,會楊信討毛里孩。二月丁酉朔,日有食之。丁巳,湖廣總兵官李震討破靖州苗。三月戊辰,召商輅為兵部侍郎,復入閣。己巳,毛里孩犯大同。辛巳,復開浙江、福建、四川、雲南銀場,以內臣領之。

夏四月,四川地屢震,自去年六月至於是月。乙巳,錄囚。癸丑,復立團營。六月戊申,雷震南京午門,敕群臣修省。辛酉,襄城伯李瑾為征夷將軍,充總兵官,兵部尚書程信提督軍務,太監劉恒監軍,討山都掌蠻。

秋七月乙酉,停河南採辦。九月辛未,振湖廣、江西饑。

冬十二月庚子,左庶子黎淳追論景泰廢立事,帝曰:「景泰事已往,朕不介意,且非臣下所當言。」切責之。辛丑,杖編修章懋、黃仲昭,檢討莊,謫官有差。是月,程信破山都掌蠻,平之。

是年,琉球、哈密、占城、烏斯藏入貢。朝鮮獻海青、白鵲,諭毋獻。

四年春正月甲戌,大祀天地於南郊。三月甲子,免湖廣被災秋糧。甲申,詔中外勢家毋得擅請田土。

夏四月丁巳,錄囚。陳文卒。五月癸未,遣使錄天下囚。六月丙午,免江西被災秋糧。辛亥,開城賊滿俊反,陜西總兵官寧遠伯任壽、巡撫都御史陳價討之。甲寅,慈懿皇太后崩。

秋七月癸酉,都督同知劉玉為平虜副將軍,充總兵官,太監劉祥監軍,副都御史項忠總督軍務,討滿俊。八月癸巳,京師地震。乙卯,朱永代劉玉為總兵官。是月,任壽、陳價、寧夏總兵官廣義伯吳琮及滿俊戰,敗績,都指揮蔣泰、申澄被殺。九月庚申,葬孝莊睿皇后於裕陵。辛酉,振陜西饑。壬申,以地震、星變下詔自責,敕群臣修省。甲申,給事中董旻、御史胡深等九人請罷商輅及禮部尚書姚夔,下獄,杖之。

冬十月乙未,項忠敗賊於石城,伏羌伯毛忠戰死。十一月,項忠擊擒滿俊,送京師,伏誅。壬戌,毛里孩犯遼東,指揮胡珍戰沒。十二月己酉,遼東總兵官趙勝奏:「十一月初六日,虜賊千餘攻指揮傅斌營,指揮胡珍率軍來援,被賊射死。」毛里孩犯延綏,都指揮僉事許寧擊敗之。

是年,琉球、烏斯藏、哈密、日本、滿剌加入貢。、五年春正月乙丑,大祀天地於南郊。三月辛丑,賜張升等進士及第、出身有差。夏五月辛丑,禮部侍郎萬安兼翰林院學士,入閣預機務。六月癸丑朔,日有食之。辛酉,錄囚。

秋八月辛酉,劉定之卒。

冬十一月乙未,毛里孩犯延綏。

是年冬,阿羅出入居河套。琉球、哈密、烏斯藏、滿剌加、安南、土魯番入貢。

六年春正月己丑,大祀天地於南郊。己亥,大同總兵官楊信敗毛里孩於胡柴溝。二月辛未,大理寺少卿宋旻,侍郎曾翬、原傑、黃琛,副都御史滕昭巡視畿南、浙江、河南、四川、福建,考察官吏,訪軍民疾苦。其餘直省有巡撫等官者,命亦如之。丁丑,禱雨於郊壇。戊寅,振廣西饑。三月甲申,免湖廣、山東被災稅糧。壬寅,詔延綏屯田。朱永為平虜將軍,充總兵官,太監傅恭、顧恒監軍,王越參贊軍務,備阿羅出於延綏。

夏五月丙申,振畿內、山東、河南饑。丁酉,王越敗阿羅出於延綏東路。六月戊申朔,日有食之。

秋七月壬午,朱永敗阿羅出於雙山堡。丙戌,都御史項忠、侍郎葉盛振畿輔饑民。都督李撫治屯營。甲辰,總兵官房能敗阿羅出於開荒川。是月,免南畿、四川被災稅糧,八月辛亥,振山西饑。癸丑,以水旱相仍,下詔寬恤。

冬十月,免畿內、河南、山東被災稅糧。十一月癸未,荊、襄流民作亂,項忠總督河南、湖廣、荊、襄軍務討之。是月,孛羅忽渡河與阿羅出合。十二月庚戌,遣使十四人分振畿輔。

是年,琉球、哈密、烏斯藏入貢。

七年春正月辛巳,命京官五品以上及給事中、御史各舉堪州縣者一人。丙戌,大祀天地於南郊。

夏四月己巳,錄囚。五月辛巳,瘞京師暴骸。

秋八月甲辰,振山東、浙江水災。閏九月己未,浙江潮溢,漂民居、鹽場,遣工部侍郎李顒往祭海神,修築堤岸。

冬十月乙亥,王恕為刑部侍郎,總理河道。十一月甲寅,立皇子祐極為皇太子,大赦。己未,荊、襄賊平,流民復業者一百四十餘萬人。十二月甲戌,彗星見,下詔自責,敕群臣修省,條時政得失。壬午,彗星入紫微垣,避正殿,撤樂,御奉天門聽政。癸未,召朱永還,王越總督延綏軍務。辛卯,減死罪以下。

是年,加思蘭入居河套,與阿羅出合。安南黎灝攻占城,破之。琉球、安南入貢。

八年春正月庚戌,大祀天地於南郊。癸亥,皇太子薨。是月,延綏參將錢亮禦毛里孩於安邊營,敗績,都指揮柏隆、陳英戰死。加思蘭犯固原、平涼。三月癸丑,賜吳寬等進士及第、出身有差。

夏四月,京師久旱,運河水涸。癸酉,遣使禱於郊社、山川、淮瀆、東海之神。乙酉,錄囚。丁亥,遣使錄天下囚。五月癸丑,武靖侯趙輔為平虜將軍,充總兵官,節制各邊軍馬,同王越禦加思蘭。

秋九月丙午,諭安南黎灝還占城侵地。

冬十一月己酉,寧晉伯劉聚代趙輔為將軍,屯延綏。十二月癸酉,振京師饑民。是年,孛羅忽、加思蘭屢入安邊營、花馬池,犯固原、寧夏、平涼、臨鞏、環慶,南至通渭。琉球、哈密、安南入貢。

九年春正月丁未,大祀天地於南郊。壬子,劉聚、王越敗加思蘭於漫天嶺。是月,土魯番速檀阿力破哈密,據之。

夏四月辛酉朔,日有食之。甲子,福餘三衛寇遼東,總兵官歐信擊敗之。戊辰,盡免山東稅糧。瘞京畿暴骸。壬午,閱武臣騎射於西苑。

秋七月壬辰,巡撫延綏都御史餘子俊敗加思蘭於榆林澗。九月辛卯,鎮守浙江中官李義杖殺寧波衛指揮馬璋,詔勿問。庚子,王越襲滿都魯、孛羅忽、加思蘭於紅鹽池,大破之。諸部漸出河套。

冬十一月丁酉,復閱騎射於西苑。

是年,免湖廣、畿內、山西、南畿、陜西被災稅糧。振畿內、陜西饑,振山西者再,山東者三。哈密、琉球、暹羅入貢。

十年春正月丁亥朔,振京師貧民。丁酉,大祀天地於南郊。癸卯,王越總制延綏、甘肅、寧夏三邊,駐固原。丙午,召劉聚還。三月,免南畿、湖廣被災秋糧。

夏五月戊申,申藏妖書之禁。是月,免山西、陜西被災秋糧。閏六月乙巳,築邊墻自紫城砦至花馬池。

秋七月甲寅,免江西被災秋糧。八月辛卯,都督同知趙勝為平虜將軍,充總兵官,太監劉恒、覃平監軍。討加思蘭。九月癸丑朔,日有食之。乙卯,免南畿水災秋糧。

冬十一月丙子,免河南被災稅糧。十二月己丑,罷寶慶諸府採金。甲午,錄妖書名示天下。

是年,琉球、烏斯藏、土魯番入貢。

十一年春正月癸亥,大祀天地於南郊。二月甲申,禁酷刑。三月壬子,賜謝遷等進士及第、出身有差。辛未,彭時卒。

夏四月乙酉,吏部侍郎劉珝、禮部侍郎劉吉並兼翰林學士,入閣預機務。壬辰,乾清門災。己亥,錄囚。五月癸酉,免湖廣被災秋糧。

秋八月辛巳,浚通惠河。丁亥,滿都魯、加思蘭遣使來朝。九月丁未朔,日有食之。

冬十一月癸丑,立皇子祐樘為皇太子,大赦。十二月戊子,復郕王帝號。丁酉,申自宮之禁。

是年,土魯番、琉球、暹羅、滿剌加、安南入貢。命琉球貢使二年一至。

十二年春正月辛亥,南京地震有聲。戊午,大祀天地於南郊。二月乙亥朔,日有食之。甲午,敕群臣修省。三月壬子,減內府供用物。壬戌,李震大破靖州苗。

夏五月丁卯,副都御史原傑撫治荊、襄流民。庚申,錄囚。

秋七月庚戌,黑眚見。乙丑,躬禱天地於禁中,以用度不節、工役勞民、忠言不聞、仁政不施四事自責。戊辰,遣使錄天下囚。

冬十月辛巳,京師地震。十一月,巡撫四川都御史張瓚討灣溪苗,破之。十二月己丑,置鄖陽府,設行都司衛所,處流民。

是年,土魯番、撒馬兒罕、琉球、烏斯藏入貢。

十三年春正月庚戌,大祀天地於南郊。己巳,置西廠,太監汪直提督官校刺事。

夏四月,汪直執郎中武清、樂章,太醫院院判蔣宗武,行人張廷綱,浙江布政使劉福下西廠獄。五月甲戌,執左通政方賢下西廠獄。丙子,大學士商輅、尚書項忠請罷西廠,從之。六月甲辰,罷項忠為民。庚戌,復設西廠。丁巳,商輅致仕。

秋八月壬戌,錦衣衛官校執工部尚書張文質繫獄,帝知而釋之。

冬十月戊申,復立哈密衛於苦峪谷,給士田牛種。十一月,張瓚破松潘疊溪苗。

是年,免浙江、山東、河南、江西、福建被災稅糧。振山東、南畿州縣饑。安南、琉球、烏斯藏、暹羅、日本入貢。滿者魯、加思蘭各遣使貢馬。

十四年春正月甲戌,大祀天地於南郊。三月戊辰,免浙江被災秋糧。己卯,賜曾彥等進士及第、出身有差。辛巳,罷烏撒衛銀場。丙戌,復開遼東馬市。丁亥,以浙江饑罷採花木。

夏四月丁酉,免南畿、山東被災秋糧。六月癸卯,太監汪直行遼東邊。

秋七月丁丑,遣使振畿南、山東饑。八月癸巳,以直隸、山東災傷,詔六部條恤民事宜。南京刑部侍郎金紳巡視江西水災。庚戌,免湖廣被災秋糧。甲寅,下巡撫蘇、松副都御史牟俸於錦衣衛獄,謫戍。十二月甲午,免畿內被災秋糧。

是年,占城、烏斯藏、撒馬兒罕入貢。

十五年春正月丁卯,大祀天地於南郊。辛巳,振山東饑。免秋糧。二月,免湖廣被災秋糧。甲寅,詔修開國勛臣墓,無後者置守塚一人。

夏四月丙午,免南畿被災稅糧。壬子,下駙馬都尉馬誠於錦衣衛獄。五月壬戌,汪直劾侍郎馬文升,下文升獄,謫戍。癸酉,以馬文升、牟俸事,杖給事中李俊、御史王濬五十六人於闕下。己卯,免湖廣、河南被災稅糧。

秋七月癸酉,汪直行大同、宣府邊。

冬十月丁亥,撫寧侯硃永為靖虜將軍,充總兵官,汪直監軍,禦伏當加。十二月辛未,論功封朱永保國公,加汪直歲祿,陞賞者二千六百餘人。是月,免四川、江西被災稅糧。

是年,琉球、安南、烏斯藏入貢。

十六年春正月甲午,大祀天地於南郊。丁酉,保國公朱永為平虜將軍,充總兵官,王越提督軍務,汪直監軍,禦亦思馬因於延綏。二月癸酉,免湖廣被災稅糧。戊寅,王越襲亦思馬因於威寧海子,破之。三月戊子,以歲歉減光祿寺供用物。

夏六月癸丑,禁勢家侵占民田。

秋八月辛酉,申存恤孤老之令。

冬十二月庚申,亦思馬因犯大同。丙寅,朱永、汪直、王越帥京軍禦之。是月,總督兩廣軍務都御史朱英、總兵官平鄉伯陳政討廣西瑤,破之。

是年,免兩畿、湖廣、河南、山東、雲南被災稅糧。琉球、暹羅、蘇門答剌、土魯番、撒馬兒罕入貢。

十七年春正月丙戌,大祀天地於南郊。二月壬戌,核天下庫藏出納之數。是月,免浙江、山西被災稅糧。三月辛卯,賜王華等進士及第、出身有差。

夏四月庚申,以久旱風霾敕群臣修省。戊辰,諭法司慎刑獄。太監懷恩同法司錄囚,自是每五歲遣內臣審錄以為常。癸酉,亦思馬因犯宣府。五月己亥,汪直監督軍務,王越為平胡將軍。充總兵官,禦之。

秋七月甲戌,免南畿披災秋糧。甲午,命所在鎮守總兵、巡撫聽汪直、王越節制。

冬十月壬戌,振河南饑。十一月戊子,取太倉銀三分之一入內庫。

是年,安南、占城、滿剌加、烏斯藏入貢。安南黎灝侵老撾宣慰司,賜敕諭之。

十八年春正月壬午,太祀天地於南郊。庚寅,劉吉起復。三月己巳朔,振南畿饑。壬申,罷西廠。

夏四月癸丑,罕慎復哈密城。甲子,免山西被災夏稅。五月,免山東、南畿被災稅糧。六月壬寅,亦思馬因犯延綏,汪直、王越調兵禦敗之。

秋八月癸丑,遣使振畿內、山東饑。辛酉,免河南被災稅糧。閏月壬申,倉副使應時用請罷饒州燒造御器內臣,下獄,贖還職。

冬十一月,免畿內、陜西、遼東被災秋糧。十二月庚午,御製《文華大訓》成。

是年,琉球、哈密、暹羅、土魯番、烏斯藏入貢。

十九年春正月丙午,大祀天地於南郊。三月丙辰,免湖廣被災稅糧。

夏四月丁丑,免河南被災稅糧。六月乙亥,汪直有罪,調南京御馬監。丁丑,陳政破廣西瑤。

秋七月辛丑,迤北小王子犯大同。癸卯,總兵官許寧禦之。敗績。己未,朱永為鎮朔大將軍,充總兵官,帥京軍禦之。八月甲子,犯宣府,巡撫都御史秦紘、總兵官周玉禦卻之。乙丑,戶部侍郎李衍、刑部侍郎何喬新巡視邊關。壬申,謫汪直為奉御,其黨王越、戴縉等貶黜有差。是月,朱永敗寇於大同、宣府。

冬十月壬申,召朱永還。

是年,撒馬兒罕貢獅子。

二十年春正月庚寅,京師地震。壬辰,敕群臣修省。詔減貢獻,飭備邊,罷營造,理冤獄,寬銀課、工役、馬價,恤大同陣亡士卒。丁酉,大祀天地於南郊。三月庚寅,賜李旻等進士及第、出身有差。己酉,太監張善監督軍務,定西侯蔣琬充總兵官,同總督尚書餘子俊備大同、宣府。

夏四月戊午,錄囚。五月甲午,再錄囚,減死罪以下。六月,免南畿、陜西被災稅糧。

秋九月乙酉朔,日有食之。是月,寇復入居河套。是秋,陜西、山西大旱饑。人相食。停歲辦物料,免稅糧,發帑轉粟。開納米事例振之。

冬十月丁巳,杖刑部員外郎林俊、都督府經歷張黻,並謫官。癸酉,罷雲南元江諸府銀坑。十二月,免山西、河南被災夏稅。

是年,安南、日本、琉球、哈密、土魯番入貢。

二十一年春正月甲申朔,星變。丙戌,詔群臣極言時政。庚寅,赦天下。乙未,大祀天地於南郊。乙巳,遣侍郎李賢、何喬新、賈俊振陜西、山西、河南饑。二月己未,放免傳奉文武官五百六十餘人。丁丑,免陜西被災稅糧。

夏四月戊午,以泰山屢震遣使祭告。壬戌,轉漕四十萬石,振陜西饑。是月,免南畿、山東被災稅糧。五月壬戌,京師地震。丙子,振京師饑民。六月辛巳,令武臣納粟襲職。癸未,詔盛暑祁寒廷臣所奏毋得過五事。

秋八月己卯朔,日有食之。九月甲子,劉珝致仕。

冬十月,免山東、山西、河南、陜西、四川被災稅糧。十一月丙寅,京師地震。十二月甲申,詹事彭華為吏部左侍郎兼翰林學士,入閣預機務。甲午,振南畿饑。是冬,小王子犯蘭州、莊浪、鎮番、涼州。

是年,哈密、烏斯藏入貢。

二十二年春正月己未,太祀天地於南郊。乙丑,免河南被災秋糧。二月庚辰,免畿南及湖廣被災秋糧。

夏四月乙未,清畿內勳戚莊田。六月,免南畿、陜西被災稅糧。乙亥,敕群臣修舉職業。甲午,諭法司慎刑。

秋七月,小王子犯甘州,指揮姚英等戰死。九月,免河南、廣東被災稅糧。丁卯,兵部左侍郎尹直為戶部侍郎兼翰林學士,入閣預機務。

冬十一月癸丑,占城為安南所侵,王子古來來奔。十二月,免江西、廣西被災稅糧。

是年,哈密、琉球入貢。

二十三年春正月,免陜西、湖廣被災稅糧。庚戌,大祀天地於南郊。二月乙酉,副都御史邊鏞、通政司參議田景賢巡視大同諸邊。三月丁未,彭華致仕。丁巳,賜費宏等進士及第、出身有差。癸亥,免山東被災稅糧。

夏四月乙亥,免浙江被災秋糧。五月乙卯,旱,遣使分禱天下山川。丙辰,敕群臣修省。是月,朵顏三衛避那孩入遼東,令駐牧近邊。給米布。六月,免陜西、南畿被災秋糧。

秋七月戊申,封皇子祐杬為興王,祐棆岐王,祐檳益王,祐楎衡王,祐枟雍王。八月庚辰,帝不豫。甲申,皇太子攝事於文化殿。己丑,崩,年四十有一。九月乙卯,上尊謚,廟號憲宗,葬茂陵。

贊曰:「憲宗早正儲位,中更多故,而踐阼之後,上景帝尊號,恤於謙之冤,抑黎淳而召商輅,恢恢有人君之度矣。時際休明,朝多耆彥,帝能篤於任人,謹於天戒,蠲賦省刑,閭里日益充足,仁、宣之治於斯復見。顧以任用汪直,西廠橫恣,盜竊威柄,稔惡弄兵。夫明斷如帝而為所蔽惑,久而後覺,婦寺之禍固可畏哉。

\end{pinyinscope}