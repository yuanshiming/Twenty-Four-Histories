\article{列傳第八十一}

\begin{pinyinscope}
費宏弟寀從子懋中子懋賢世父瑄翟鑾李時顧鼎臣嚴訥(袁煒李春芳孫思誠等陳以勤趙貞吉殷士儋高儀

費宏,字子充,鉛山人。甫冠,舉成化二十三年進士第一,授修撰。弘治中,遷左贊善,直講東宮,進左諭德。武宗立,擢太常少卿,兼侍講讀。預修《孝宗實錄》。充日講官。正德二年拜禮部右侍郎,尋轉左。五年進尚書。帝耽於逸樂,早朝日講俱廢。宏請勤政、務學、納諫,報聞。魯府鄒平王子當潩當襲父爵,為弟當涼所奪且數年矣。宏因當潩奏辨,據法正之。當涼怒,誣宏受賂,宏不為動。明年冬十二月,命宏兼文淵閣大學士參預機務。尋加太子太保、武英殿大學士,進戶部尚書。

倖臣錢寧陰黨宸濠,欲交歡宏,饋彩幣及他珍玩。拒卻之。寧慚且恚。宸濠謀復護衛、屯田,輦白金巨萬,遍賂朝貴,寧及兵部尚書陸完主之。宏從弟編修寀,其妻與濠妻,兄弟也,知之以告宏。宏入朝,完迎問曰:「寧王求護衛,可復乎?」宏曰:「不知當日革之者何故?」完曰:「今恐不能不予。」宏峻卻之。及中官持奏至閣,宏極言不當予,詔卒予之。於是宸濠與寧合,而恚宏。寧數偵宏事無所得。以御史餘珊嘗劾寀不當留翰林,即指為宏罪。中旨責陳狀,宏乞休。命并寀致仕。寧遣騎伺宏後,抵臨清,焚其舟,資裝盡毀。宏歸,杜門謝客。宸濠復求與通,宏謝絕之,益怒。會宏族人與邑奸人李鎮等訟,宸濠陰令鎮賊宏。鎮等遂據險作亂,率眾攻費氏。索宏不得,執所與訟者支解之,發宏先人塚,毀其家,劫掠遠近,眾至三千人。宏馳使愬於朝。下巡撫孫燧按狀,始遣兵剿滅。宸濠敗,言者爭請召宏。世宗即位,遣行人即家起宏,加少保,入輔政。

宏持重識大體,明習國家故事。與楊廷和、蔣冕、毛紀同心協贊,數勸帝革武宗弊政。「大禮」之議,諸臣力與帝爭,帝不能堪。宏頗揣知帝旨,第署名公疏,未嘗特諫,以是帝心善之。及廷和等去位,宏為首輔。加少師兼太子太師、吏部尚書、謹身殿大學士,委任甚至。

戶部議督正德時逋賦,宏偕石珤、賈詠請斷自十年以後。從之。帝以四方災異,敕群臣修省。宏等因言:「陛下用度無節,工役不休。畿內土地半成莊田,內庫收納要求踰倍。太倉無三年之積而冗食日增,京營無十萬之兵而赴工不已。直臣得罪未見原,言官舉職乃被詰。律所當行者數經讞不誅,罪無可辨者遽傳旨獲免。乾和召怨,自非一端。」帝引咎褒答,然不能用也。大同兵變,張璁請討之。宏曰:「討而勝,玉石俱焚;不勝,彼將據城守,損威重多矣。莫若觀變而徐圖之。」事果旋定。

宏為人和易,好推轂後進。其於「大禮」不能強諫,亦未嘗附離。而是時席書、張璁、桂萼用事。書弟檢討春,故由他曹改用。及《武宗實錄》成,宏議出為僉事,書由是憾宏。璁、萼由郎署入翰林,驟至詹事,舉朝惡其人。宏每示裁抑,璁、萼亦大怨。帝嘗御平臺,特賜御製七言一章,命輯倡和詩集,署其銜曰「內閣掌參機務輔導首臣」。其見尊禮,前此未有也。

璁、萼滋害宏寵。萼言:「詩文小技,不足勞聖心,且使宏得馮寵靈,凌壓朝士。」帝置不省。萼遂與璁毀宏於帝,言宏納郎中陳九川所盜天方貢玉,受尚書鄧璋賕謀起用,并及其居鄉事。宏上書乞休,略曰:「萼、璁挾私怨臣屢矣。不與經筵講官則怨,不與修獻皇帝實錄則怨,不為兩京鄉試考官則怨,不為教習則又怨。萼、璁疑內閣事屬臣操縱,抑知臣下采物望,上稟聖裁,非可專擅。萼、璁日攘袂搤掔,覬覦臣位。臣安能與小人相齮齕?祈賜骸骨。」不允。及璁居兵部,宏欲用新寧伯譚綸掌奮武營,璁遂劾宏劫制府部。無何,又因宏子懋良坐罪下吏,攻之益力,復錄前後劾疏上之。不得請,則力求罷,詆宏尤切,章數上。宏亦連疏乞休,帝輒下優詔慰留,然終不以譴璁、萼。於是奸人王邦奇承璁、萼指,上書污故大學士廷和等,并誣宏。宏竟致仕去。時六年二月也。十月,璁遂以尚書、大學士入直內閣。間一歲萼亦入矣。

十四年,萼既前死,璁亦去位,帝始追念宏。四月,再遣行人即家起官如故。七月至京師。使中使勞以上尊御饌,面諭曰:「與卿別久,卿康健無恙,宜悉心輔導稱朕意。」宏頓首謝。自是眷遇益厚。偕李時召入無逸殿,與周覽殿廬,從容笑語,移時始出。賜銀章曰「舊輔元臣」。數有咨問,宏亦竭誠無隱。承璁、萼操切之後,易以寬和,朝士皆慕樂之。未幾卒,年六十有八。帝嗟悼,賻恤加等,贈太保,謚文憲。

宏三入內閣,佐兩朝殆十年。中遭讒構,訖以功名終。其自少保入也,弟寀為贊善,從子懋中由進士及第為編修,宏長子懋賢方改庶吉士,父子兄弟並列禁近。寀官至少保、禮部尚書,謚文通。懋中終湖廣提學副使。懋賢歷兵部郎中。

宏世父瑄,成化十一年進士。弘治時為兵部員外郎。貴州巡撫謝昶、總兵官吳經等奏爛土苗反,僭稱王,乞發大軍征討。以兵部尚書馬文升請,令瑄與御史鄧庠往按。白苗無反狀,撫定之。劾昶、經及鎮守中官張成罪。遷貴州參議以終。

翟鑾,字仲鳴,其先諸城人。曾祖為錦衣衛校尉,因家京師。舉弘治十八年進士,改庶吉士。正德初,授編修。劉瑾改翰林於他曹,以鑾為刑部主事。旋復官,進侍讀。嘉靖中,累遷禮部右侍郎。六年春,廷推閣臣。帝意在張孚敬,弗與。命再推,乃及鑾。中貴人多譽鑾者,帝遂踰次用之。楊一清以鑾望輕,請用吳一鵬、羅欽順。帝不許,命鑾以吏部左侍郎兼學士入直文淵閣。尋賜銀章曰「清謹學士。」

鑾初入閣,一清、謝遷輔政,既而孚敬與桂萼入,鑾皆謹事之。孚敬、萼皆以所賜銀章密封言事,鑾獨無所言。詰之,則頓首謝曰:「陛下明聖,臣將順不暇,何獻替之有。」帝心愛之。一清、萼、孚敬先後罷,鑾留獨秉政者兩月。其後李時、方獻夫入,位皆居鑾上,鑾亦無所怫。帝數召時、鑾入見,嘗問:「都察院擬籍谷大用資產,當乎?」時、鑾皆北人,與中貴合。時曰:「所擬不中律。」鑾曰:「按律,籍沒止三條,謀反、叛逆及奸黨耳。不合三尺法,何以信天下。」帝曰:「大用亂政先朝,正奸黨也。」鑾曰:「陛下,即天也。春生秋殺,何所不可。」帝卒從重擬。丁生母憂歸。服闋,久不召。夏言、顧鼎臣居政府,鑾與謀召己。會帝將南巡,慮塞上有警,議遣重臣巡視,言等因薦鑾充行邊使。十八年二月改兵部尚書兼右都御史,諸邊文武將吏咸受節制。且齎帑金五十萬犒邊軍,東西往返三萬餘里。明年春入京,遂命以原官入閣。在大同與總督毛伯溫議築五堡,過甘肅與總督劉天和議拓嘉峪關,皆受廕敘。

二十一年,言罷,鑾為首輔。時已加少保、武英殿大學士,進少傅、謹身殿。嚴嵩初入,鑾以資地居其上,權遠出嵩下,而嵩終惡鑾,不能容。御史趙大佑劾鑾私同年,吏部尚書許讚亦發鑾請屬私書,帝皆不問。會鑾子汝儉、汝孝與其師崔奇勛所親焦清同舉二十三年進士,嵩遂屬給事中王交、王堯日劾其有弊。帝怒,下吏部、都察院。鑾疏辨,引西苑入直自解。帝益怒,勒鑾父子、奇勛、清及分考官編修彭鳳、歐陽奐為民,而下主考少詹事江汝璧及鄉試主考諭德秦鳴夏、贊善浦應麒詔獄,並杖六十,褫其官。

鑾初輔政,有修潔聲。中持服家居,至困頓不能自給。其用行邊起也,諸邊文武大吏俱櫜鞬郊迎,恒恐不得當鑾意,饋遺不貲。事竣,歸裝千輛,用以遺貴近,得再柄政,聲譽頓衰。又為其子所累,訖不復振。踰三年卒,年七十。穆宗即位,復官,謚文懿。

李時,字宗易,任丘人。父棨,進士,萊州知府。時舉弘治十五年進士,改庶吉士,授編修。正德中,歷侍讀、右諭德。世宗嗣位,為講官,尋遷侍讀學士。

嘉靖三年,擢禮部右侍郎。俄以憂歸。服除,為戶部右侍郎。復改禮部,尋代方獻夫為尚書。帝既定尊親禮,慨然有狹小前人之志,欲裁定舊章,成一朝制作。張孚敬、夏言用事,咸好更張。所建諸典禮,咸他人發端,而時傅會成之。或廷議不合,率具兩端,待帝自擇,終未嘗顯爭。以故帝愛其恭順。四方上嘉瑞,輒拜疏請賀。帝謙讓,時必再請。由是益以時為忠。賜銀章曰「忠敏安慎」,俾密封言事。久而失之,請罪,帝再賜焉。十年七月,四郊成,加太子太保。雷震午門,彗星見東井,時請敕臣工修省,令言官指陳利害興革。帝以建言乃科道專責,寢不行。光祿寺廚役王福、錦衣衛千戶陳昇請遷顯陵於天壽山,時等力陳不可。巡檢徐震奏於安陸建京師,時等駁其非制,遂議改州為承天府。

其秋,桂萼卒,命時兼文淵閣大學士入參機務。時張孚敬已罷,翟鑾獨相。時後入,以宮保官尊,反居鑾上。兩人皆謙遜,無齟齬。帝御無逸殿,召時坐講《無逸篇》,鑾講《豳風·七月》詩,武定侯郭勛及九卿翰林俱入侍。講畢,帝退御豳風亭賜宴。自是,數召見,諮謀政務。

明年春,孚敬還內閣,事取獨裁,時不敢有所評議。未幾,方獻夫入,與時亦相得。彗星復出,帝召見時等,諭以引咎修省之意,從容語及乏才。時等退,條上務安靜、惜人才、慎刑獄三事,頗及「大禮」大獄廢斥諸臣。帝優詔褒答之,然卒不能用也。給事中魏良弼、御史馮恩先後劾吏部尚書汪鋐,觸帝怒,時皆為論救。十二年,孚敬復入,鑾以憂去,獻夫致仕。時隨孚敬後,拱手唯諾而已,以故孚敬安之。孚敬謝政,費宏再入,未幾卒,時遂獨相。時素寬平,至是益鎮以安靜。帝亦恒召對便殿,接膝咨詢。時雖無大匡救,而議論恒本忠厚,廷論咸以時為賢。客星見天棓旁,帝問所主事應。對曰:「事應之說起漢京房,未必皆合。惟在人君修德以弭之。」帝稱善。扈蹕謁陵,道沙河,帝見居民蕭索,愴然曰:「七陵在此,宜加守護。」時對曰:「昔邱濬建議,京師當設四輔,以臨清為南,昌平為北,蘇州、保定為東西,各屯兵一二萬。今若於昌平增一總兵,可南衛京師,北護陵寢。」帝乃下廷臣勘議,於沙河築鞏華城,為置戍焉。屢加少傅、太子太師、吏部尚書、華蓋殿大學士。會夏言入輔,時不與抗,每事推讓言,言亦安之。帝待時不如孚敬、言,然少責辱,始終不替。孚敬、言亦不敢望也。十七年十二月卒官,贈太傅,謚文康。

顧鼎臣,字九和,崑山人。弘治十八年進士第一。授修撰。正德初,再遷左諭德。嘉靖初,直經筵。進講范浚《心箴》,敷陳剴切。帝悅,乃自為注釋,而鼎臣特受眷。累官詹事。給事中劉世揚、李仁劾鼎臣汙佞。帝下世揚等獄,以鼎臣救,得薄譴。拜禮部右侍郎。帝好長生術,內殿設齋醮。鼎臣進《步虛詞》七章,且列上壇中應行事。帝優詔褒答,悉從之。詞臣以青詞結主知,由鼎臣倡也。

改吏部左侍郎,掌詹事府。請令曾子後授《五經》博士,比三氏子孫,從之。大同軍變,張孚敬主用兵,鼎臣言不可,帝嘉納。十三年孟冬,享廟,命鼎臣及侍郎霍韜捧主。二人有期功服,當辭。乃上言:「古禮,諸侯絕期。今公卿即古諸侯,請得毋避。」禮部尚書夏言極詆其非,乃已。尋進禮部尚書,仍掌府事。京師淫雨,四方多水災,鼎臣請振饑弭盜,報可。

十七年八月,以本官兼文淵閣大學士入參機務。尋加少保、太子太傅、進武英殿。初,李時為首輔,夏言次之,鼎臣又次之。時卒,言當國專甚,鼎臣素柔媚,不能有為,充位而已。帝將南巡,立皇太子,命言扈行,鼎臣輔太子監國。御史蕭祥曜劾吏部侍郎張潮受鼎臣屬,調刑部主事陸昆為吏部。潮言:「兵部主事馬承學恃鼎臣有聯,自詭必得銓曹,臣故抑承學而用昆。」帝下承學詔獄,鼎臣不問。十九年十月卒官,年六十八。贈太保,謚文康。

鼎臣官侍從時,憫東南賦役失均,屢陳其弊,帝為飭撫按。巡撫歐陽鐸釐定之。崑山無城,言於當事為築城。後倭亂起,崑山獲全,鄉人立祠祀焉。

嚴訥,字敏卿,常熟人。舉鄉試,以主司試錄觸忌,一榜皆不得會試。嘉靖二十年成進士,改庶吉士,授編修,遷侍讀。三吳數中倭患,歲復大祲,民死徙幾半,有司徵斂益急。訥疏陳民困,請蠲貸。帝得疏感動,報如其請。尋與李春芳入直西苑。撰青詞,超授翰林學士。歷太常少卿,禮部左、右侍郎,改吏部,皆兼學士,仍直西苑。所撰青詞皆稱旨。禮部尚書郭朴遷吏部,遂以訥代之。朴遭父喪,復代為吏部尚書。嚴嵩當國,吏道汙雜。嵩敗,朴典銓猶未能盡變。訥雅意自飭,徐階亦推心任之。訥乃與朝士約,有事白於朝房,毋謁私邸。慎擇曹郎,務抑奔競,振淹滯。又以資格太拘,人才不能盡,仿先朝三途並用法,州縣吏政績異者破格超擢,銓政一新。尋錄供奉勞,加太子太保。

四十四年,袁煒罷,命兼武英殿大學士入參機務。以代者郭朴未至,仍掌銓政。帝齋居西苑,侍臣直廬皆在苑中。訥晨出理部事,暮宿直廬,供奉青詞,小心謹畏,至成疾久不愈。其年冬十一月,遂乞歸。踰年,世宗崩,遂不復出。

訥既歸里,父母皆在。晨夕潔餐孝養,人以為榮。訥嘗語人曰:「銓臣與輔臣必同心乃有濟。吾掌銓二年,遷華亭當國,事無阻。且所任選郎賢,舉無失人。」華亭謂徐階,選郎則陸光祖也。家居二十年卒,年七十有四。贈少保,謚文靖。

袁煒,字懋中,慈溪人。嘉靖十七年會試第一,殿試第三,授編修。煒性行不羈,為御史包孝所劾,帝宥不罪。進侍讀。久之,簡直西苑。撰青詞,最稱旨。三十五年,閣臣推修撰全元立掌南京翰林院,帝特用煒。煒疏辭,願以故官供奉。帝大喜,立擢煒侍講學士。甫兩月,手詔拜禮部右侍郎。明年,加太子賓客兼學士,賜一品服。三十九年,復以供奉恩加俸二等,俄進左侍郎。明年二月調吏部,兼官供奉如故。踰月遷禮部尚書,加太子少保,仍命入直。煒自供奉以後,六年中進宮保、尚書,前未有也。

先是二月朔,日食微陰,煒言不當救護。禮部尚書吳山不從,得譴去。帝聞煒言善之,遂以代山。及七月朔,又日食。曆官言食止一分五杪,例免救護。煒乃阿帝意上疏言:「陛下以父事天,以兄事日,群陰退伏,萬象輝華。是以太陽晶明,氛祲銷爍,食止一分,與不食同。臣等不勝欣忭。」疏入,帝益喜。其冬,遂命以戶部尚書兼武英殿大學士入閣典機務。累加少傅兼太子太傅、建極殿大學士。四十四年春,疾篤,請假歸,道卒,年五十八。贈少師,謚文榮。

煒才思敏捷。帝中夜出片紙,命撰青詞,舉筆立成。遇中外獻瑞,輒極詞頌美。帝畜一貓死,命儒臣撰詞以醮。煒詞有「化獅作龍」語,帝大喜悅。其詭詞媚上多類此。以故帝急枋用之,恩賜稠疊,他人莫敢望。

自嘉靖中年,帝專事焚修,詞臣率供奉青詞。工者立超擢,卒至入閣。時謂李春芳、嚴訥、郭朴及煒為「青詞宰相」。而煒貴倨鮮淟,故出徐階門,直以氣凌之。與階同總裁《承天大志》,諸學士呈稿,煒竄改殆盡,不以讓階。諸學士不平,階第曰任之而已。其後煒死,階亦盡竄改之。煒自負能文,見他人所作,稍不當意,輒肆詆誚。館閣士出其門者,斥辱尤不堪,以故人皆畏而惡之。

李春芳,字子實,揚州興化人。嘉靖二十六年舉進士第一,除修撰。簡入西苑撰青詞,大被帝眷,與侍讀嚴訥超擢翰林學士。尋遷太常少卿,拜禮部右侍郎,俱兼學士,直西苑如故。佐理部事,進左侍郎,轉吏部,代訥為禮部尚書。時宗室蕃衍,歲祿苦不繼。春芳考故事,為書上之。諸吉凶大禮及歲時給賜,皆嚴為之制。帝嘉之,賜名《宗籓條例》。尋加太子太保。四十四年命兼武英殿大學士,與訥並參機務。世宗眷侍直諸臣厚,凡遷除皆出特旨。春芳自學士至柄政,凡六遷,未嘗一由廷推。

春芳恭慎,不以勢凌人。居政府持論平,不事操切,時人比之李時;其才力不及也,而廉潔過之。時徐階為首輔,得君甚。春芳每事必推階,階亦雅重之。隆慶元年春,有詔修翔鳳樓,春芳曰:「上新即位,而遽興土木,可乎?」事遂止。

齊康之劾徐階也,語侵春芳。春芳疏辨求去,帝慰留之。及代階為首輔,益務以安靜,稱帝意。時同列者陳以勤、張居正。以勤端謹,而居正恃才凌物,視春芳蔑如也。始階以人言罷,春芳歎曰:「徐公尚爾,我安能久?容旦夕乞身耳。」居正遽曰:「如此,庶保令名。」春芳愕然,三疏乞休,帝不允。既而趙貞吉入代以勤,剛而負氣。及高拱再入直,凌春芳出其上,春芳不能與爭,謹自飭而已。俺答款塞求封,春芳偕拱、居正即帝前決之。會貞吉為拱逐,拱益張,修階故怨。春芳嘗從容為階解,拱益不悅。時春芳已累加少師兼太子太師,進吏部尚書,改中極殿,度拱輩終不容己,兩疏請歸養,不允。南京給事中王禎希拱意,疏詆之,春芳求去益力。賜敕乘傳,遣官護行,有司給夫廩如故事。閱一歲,拱復為居正所擠,幾不免。而春芳歸,父母尚無恙,晨夕置酒食為樂,鄉里艷之。父母歿數年乃卒,年七十五,贈太師,謚文定。

孫思誠,天啟六年官禮部尚書,尋罷。崇禎初,坐頌榼閒住。

思誠孫清,字映碧。崇禎四年進士。由寧波推官擢刑科給事中。熊文燦撫張獻忠,清論其失策。以久旱請寬刑,忤旨,貶浙江按察司照磨。未赴,憂歸。起吏科給事中。俄出封淮府,國變得不與。福王時,請追謚開國名臣及武、熹兩朝忠諫諸臣,於是李善長等十四人、陸震等十四人、左光斗等九人並得謚。

春芳曾孫信,廣東平和知縣。城破,與二子泓遠、淑遠同時死。

陳以勤,字逸甫,南充人。嘉靖二十年進士。選庶吉士,授檢討。久之,充裕王講官,遷修撰,進洗馬。時東宮位號未定,群小多構釁。世宗於父子素薄,王歲時不得燕見。常祿外,例有給賜,王亦不敢請。積三歲,邸中窘甚。王左右以千金賄嚴世蕃,世蕃喜,以屬戶部,得并給三歲資。然世蕃常自疑,一日屏人語以勤及高拱曰:「聞殿下近有惑志,謂家大人何?」拱故為謔語,以勤正色曰:「國本默定久矣。生而命名,從后從土,首出九域,此君意也。故事,諸王講官止用檢討,今兼用編修,獨異他邸,此相意也。殿下每謂首輔社稷臣,君安從受此言?」世蕃默然去。裕邸乃安。

為講官九年,有羽翼功,而深自晦匿,王嘗書「忠貞」二字賜之。父喪除,還為侍讀學士,掌翰林院。進太常卿,領國子監。擢禮部右侍郎,尋轉左,改吏部,掌詹事府。

穆宗即位,以勤自以潛邸舊臣,條上謹始十事,曰定志、保位、畏天、法祖、愛民、崇儉、攬權、用人、接下、聽言。其言攬權、聽言尤切。詔嘉其忠懇。隆慶元年春,擢禮部尚書兼文淵閣大學士,入參機務。累加少傅兼太子太傅,改武英殿。穆宗朝講希御,政無所裁決,近倖多緣內降得厚恩。以勤請勵精修政。帝心動,欲有所舉措,卒為內侍所阻,疏亦留中。四年,條上時務因循之弊,請慎擢用:酌久任、治贓吏、廣用人、練民兵、重農穀。帝嘉之,下所司議。高拱掌吏部,惡所言侵己職,寢其奏,惟都察院議行贓吏一事而已。

初,以勤之入閣也,徐階為首輔,而拱方嚮用,朝士各有所附,交相攻。以勤中立無所比,亦無私人,竟階與拱去,無訾及之者。及拱再入,與趙貞吉相軋,張居正復中構之。以勤與拱舊僚,貞吉其鄉人,而居正則所舉士也,度不能為解,恐終不為諸人所容,力引疾求罷。遂進兼太子太師、吏部尚書,賜敕馳傳歸,詔其子編修于陛侍行。後二年,拱被逐,倉皇出國門,嘆曰:「南充,哲人也。」以勤歸十年,年七十。復頒上方銀幣,命于陛馳歸賜之,且敕有司存問。又六年卒。贈太保,謚文端。于陛別有傳。

趙貞吉,字孟靜,內江人。六歲日誦書一卷。及長,以博洽名。最善王守仁學。舉嘉靖十四年進士,選庶吉士,授編修。時方士初進用,貞吉請求真儒贊大業。執政不懌,因請急歸。還朝遷中允,掌司業事。

俺答薄都城,謾書求貢。詔百官廷議,貞吉奮袖大言曰:「城下之盟,《春秋》恥之。既許貢則必入城,倘要索無已,奈何?」徐階曰:「君必有良策。」貞吉曰:「為今之計,請至尊速御正殿,下詔引咎。錄周尚文功以勵邊帥,出沈束於獄以開言路;輕損軍之令,重賞功之格;遣官宣諭諸將,監督力戰,退敵易易耳。」時帝遣中使瞷廷臣,日中莫發一語。聞貞吉言,心壯之,諭嚴嵩曰:「貞吉言是,第不當及周尚文、沈束事耳。」召入左順門,令手疏便宜。立擢左諭德兼監察御史,奉敕宣諭諸軍。給白金五萬兩,聽隨宜勞賞。初,貞吉廷議罷,盛氣謁嚴嵩。嵩辭不見,貞吉怒叱門者。適趙文華至,貞吉復叱之。嵩大恨。及撰敕,不令督戰,以輕其權,且不與一卒護行。時敵騎充斥,貞吉馳入諸將營,散金犒士,宣諭德意,明日即復命。帝大怒,謂貞吉漫無區畫,徒為尚文、束游說。下之詔獄,杖於廷,謫荔波典史。稍遷徽州通判,進南京吏部主事。

四十年,遷至戶部右侍郎。廷議遣大臣赴薊州督餉練兵,嵩欲用貞吉,召飲示之意。貞吉曰:「督餉者,督京運乎,民運乎?若二運已有職掌,添官徒增擾耳。況兵之不練,其過宜不在是,即十戶侍出,何益練兵?」嵩怫然罷。會嵩請告,吏部用倉場侍郎林應亮。比嵩出,益怒。令都給事中張益劾應亮,調之南京,而改用僉都御史霍冀。益又言:「督餉戶部專職,今貞吉與左侍郎劉大賓廷推不及,是不職也,宜罷。」於是二人皆奪官。

隆慶初,起禮部左侍郎,掌詹事府。穆宗幸太學,祭酒胡傑適論罷,以貞吉攝事。講《大禹謨》稱旨,命充日講官。貞吉年踰六十,而議論侃直,進止有儀,帝深注意焉。尋遷南京禮部尚書。既行,帝念之,仍留直講。三年秋,命兼文淵閣大學士參預機務。貞吉入謝,奏:「朝綱邊務一切廢弛,臣欲捐軀任事,惟陛下主之。」帝益喜。會寇入大同,總兵官趙岢失事,總督陳其學反以捷聞,為御史燕如宦所發。貞吉欲置重罰,兵部尚書霍冀僅議貶秩。貞吉與同官爭不得,因上言:「邊帥失律,祖宗法具在。今當事者屈法徇人,如公論何?臣老矣,效忠無術,乞賜罷。」不許。俄加太子太保。貞吉以先朝禁軍列三大營,營各有帥,今以一人總三營,權重難制。因極言其弊,請分五營,各統以大將,稍復祖宗之舊。帝善之,命兵部會廷臣議。尚書霍冀前與貞吉議不合,頗不然其言。廷臣亦多謂強兵在擇將,不在變法。冀等乃上議三大營宜如故。惟以一人為總督,權太重,宜三營各設一大將,而罷總督,以文臣為總理。報可。

初,給事中楊鎔劾冀貪庸。帝已留冀,冀以鎔貞吉鄉人,疑出貞吉意,疏辨乞罷,且詆貞吉。貞吉亦疏辨求去。詔留貞吉,褫冀官。其後營制屢更,未踰年即復其舊,貞吉亦不能爭也。俺答款塞求封,貞吉力贊其議。

先是,高拱再入閣即掌吏部。貞吉言於李春芳,亦得掌都察院。拱以私憾欲考察科道。貞吉與同事上言:「頃因御史葉夢熊言事忤旨,陛下嚴諭考核言官,并及升任在籍者。應考近二百人,其中豈無懷忠報主謇諤敢言之士?今一以放肆奸邪罪之,竊恐所司奉行過當,忠邪不分,致塞言路,沮士氣,非國家福也。」帝不從。拱以貞吉得其情,憾甚。及考察,拱欲去貞吉所厚者,貞吉亦持拱所厚以解。於是斥者二十七人,而拱所惡者咸與。拱猶以為憾也,嗾門生給事中韓楫劾貞吉庸橫,考察時有私。貞吉疏辨乞休,且言:「臣自掌院務,僅以考察一事與拱相左。其他壞亂選法,縱肆作奸,昭然耳目者,臣噤口不能一言,有負任使,臣真庸臣也。若拱者,斯可謂橫也已。臣放歸之後,幸仍還拱內閣,毋令久專大權,廣樹眾黨。」疏入,竟允貞吉去,而拱握吏部權如故。

貞吉學博才高。然好剛使氣,動與物迕。九列大臣,或名呼之,人亦以是多怨。高拱、張居正名輩出貞吉後,而進用居先。咸負才好勝不相下,竟齟齬而去。萬曆十年卒,贈少保,謚文肅。

殷士儋,字正甫,歷城人。嘉靖二十六年進士。選庶吉士,授檢討。久之,充裕王講官。凡關君德治道,輒危言激論,王為動色。遷右贊善,進洗馬,直論如故。隆慶元年擢侍讀學士,掌翰林院事,進禮部右侍郎,未幾改吏部。明年春,拜禮部尚書,掌詹事府事。其冬,還理部事。四年正月朔望,日月俱食。士儋疏請布德、緩刑、納諫、節用,飭內外臣工講求民瘼。報聞。以舊恩,進太子太保。時寒暑皆罷講,士儋請如故事,四時無輟,并進講《祖訓》及《大學衍義》、《貞觀政要》。帝嘉納之。

始世宗定宗籓條例,親王無後,以兄弟及兄弟之子嗣,不得以旁繼。嘉靖末,肅懷王薨,無子。其大母定王妃請以輔國將軍縉𤏳嗣,禮部議縉𤏳實懷王從叔,不可承祧。詔許以將軍攝府事。及帝即位,王妃復請,前尚書高儀執不可。縉𤏳重賄中官,屬宗人為奏,祈必得。士儋持之甚力。帝以肅籓越在遠塞,不王無以鎮之,遂許縉𤏳嗣。士儋爭曰:「肅府自甘州徙蘭州,實內地。且請別選郡王賢者理府事,毋遂私請,壞條例。」而帝意堅不可奪。士儋乃請封為郡王,諸宗率以此令從事,帝終不許。故事,郊畢,舉慶成宴。自世宗倦勤,典禮久廢。帝即位三載,猶未舉行,士儋始考定舊儀行之。十一月,命以本官兼文淵閣大學士入閣辦事。俄俺答封事成,進少保,改武英殿。

始士儋與陳以勤、高拱、張居正並為裕邸僚,三人皆柄用,士儋仍尚書,不能無望。拱素善張四維,欲引共政,而惡士儋不親己,不為援。士儋遂藉太監陳洪力,取中旨入閣,以故怨拱及四維。四維父擅鹽利,為御史郜永春所劾。事已解,他御史復及之。拱、四維疑出士儋指,益相構。御史趙應龍遂劾士儋進由陳洪,不可以參大政。士儋再辨求去,不允。而拱門生都給事中韓楫復揚言脅之,士儋亦疑出拱指。故事,給事中朔望當入閣會揖。士儋面詰楫曰:「聞君有憾於我,憾自可耳,毋為他人使。」拱曰:「非體也。」士儋勃然起,詬拱曰:「若逐陳公,逐趙公,復逐李公,今又為四維逐我,若能常有此座耶?」奮臂欲毆之。居正從旁解,亦誶而對。御史侯居良復劾士儋始進不正,求退不勇。士儋再疏請益力,乃賜道里費,乘傳歸,有司給稟隸如故事。家居十一年卒。時居正垂沒,四維為政,怨士儋。贈太保,謚文通。久之,改謚文莊。

高儀,字子象,錢塘人。嘉靖二十年進士。選庶吉士,授編修。歷侍講學士,掌南京翰林院。召為太常卿,掌國子監事。擢禮部右侍郎,改吏部,教習庶吉士。四十五年代高拱為禮部尚書。穆宗即位,諸大典禮皆儀所酌定。世宗遺命,郊社及祔享祔葬諸禮,悉稽祖制更定。儀乃會廷臣議:天地分祀不必改;既祭先農,不當復祈穀西苑;帝社、帝稷、睿宗明堂配天與玉芝宮專祀,當廢;孝潔皇后當祔廟,別祀孝烈於他所。帝皆報可。既而中官李芳復請天地合祀如洪武制,御史張檟請易皇極諸殿名,盡復其舊,儀皆持不可。帝踐阼四月,未召對大臣,儀屢請。隆慶二年正月饗太廟,帝將遣代,儀偕僚屬諫,閣臣亦以為言,乃親祀如禮。慶府輔國將軍縉貴請襲王爵,儀執不從。太子生七齡,儀疏請出閣,帝命待十齡行之。詔取光祿銀二十萬兩,儀力爭。初,世宗崇道教,太常多濫員,儀奏汰四十八人。寺卿陳慶奏供事乏缺,儀堅持不可。掌禮部四年,每歲暮類奏四方災異,遇事秉禮循法,居職甚稱。引疾章六上,卒見留。會御史傅寵以先帝時撰文叩壇事劾儀,儀四疏求去,乃加太子少保馳傳歸。

歸二年,用高拱薦,命以故官侍東宮講讀,掌詹事府。六年四月詔兼文淵閣大學士入閣辦事。踰月,帝崩,預顧命。及拱為張居正所逐,儀已病,太息而已。未幾卒。贈太子太保,謚文端。

儀性簡靜,寡嗜慾,室無妾媵。舊廬毀於火,終身假館於人。及沒,幾無以殮。

贊曰:費宏等皆起家文學,致位宰相。宏卻錢寧,拒宸濠,忤張、桂,再躓再起,終亦無損清譽。李時、翟鑾皆負才望,而鑾晚節不振。貞吉負氣自高,然處傾軋之勢,即委蛇,庸得免乎?顧鼎臣等雍容廟堂,可謂極遭逢之盛。而陳以勤誠心輔導,獻納良多。後賢濟美,繼登相位。終明之世,稱韋、平者,數以勤父子。天之報之,何其厚哉。


\end{pinyinscope}