\article{列傳第二百八 外國一}

\begin{pinyinscope}
○朝鮮

朝鮮,箕子所封國也。漢以前曰朝鮮。始為燕人衛滿所據,漢武帝平之,置真番、臨屯、樂浪、玄菟四郡。漢末,有扶餘人高氏據其地,改國號曰高麗,又曰高句麗,居平壤,即樂浪也。已,為唐所破,東徙。後唐時,王建代高氏,兼併新羅、百濟地,徙居松岳,曰東京,而以平壤為西京。其國北鄰契丹,西則女直,南曰日本,元至元中,西京內屬,置東寧路總管府,盡慈嶺為界。

明興,王高麗者王顓。太祖即位之元年遣使賜璽書。二年送還其國流人。顓表賀,貢方物,且請封。帝遣符璽郎偰斯齎詔及金印誥文封顓為高麗國王,賜曆及錦綺。其秋,顓遣總部尚書成惟得、千牛衛大將軍金甲兩上表謝,并賀天壽節,因請祭服制度,帝命工部製賜之。惟得等辭歸,帝從容問:「王居國何為?城郭修乎?兵甲利乎?宮室壯乎?」頓首言:「東海波臣,惟知崇信釋氏,他未遑也。」遂以書諭之曰:「古者王公設險,未嘗去兵。民以食為天,而國必有出政令之所。今有人民而無城郭,人將何依?武備不修,則威弛;地不耕,則民艱於食;且有居室,無廳事,無以示尊嚴。此數者朕甚不取。夫國之大事,在祀與戎。茍闕斯二者,而徒事佛求福,梁武之事,可為明鑒。王國北接契丹、女直,而南接倭,備禦之道,王其念之。」因賜之《六經》、《四書》、《通監》。自是貢獻數至,元旦及聖節皆遣使朝賀,歲以為常。

三年正月命使往祀其國之山川。是歲頒科舉詔於高麗,顓表謝,貢方物,并納元所授金印。中書省言:「高麗貢使多齎私物入貨,宜徵稅;又多攜中國物出境,禁之便。」俱不許。五年表請遣子弟入太學,帝曰:「入學固美事,但涉海遠,不欲者勿強。」貢使洪師範、鄭夢周等一百五十餘人來京,失風溺死者三十九人,師範與焉。帝憫之,遣元樞密使延安答里往諭入貢毋數。而顓復遣其門下贊成事姜仁裕來貢馬,其賀正旦使金湑等已先至,帝悉遣還。謂中書省臣曰:「高麗貢獻繁數,既困敝其民,而涉海復虞覆溺。宜遵古諸侯之禮,三年一聘。貢物惟所產,毋過侈。其明諭朕意。」

六年,顓遣甲兩等貢馬五十匹,道亡其二,甲兩以聞。及進,以私馬足之。帝惡其不誠,卻之。七年遣監門護軍周誼、鄭庇等來貢,表請每歲一貢,貢道從陸,由定遼,毋涉海,其貢物稱「送太府監」。中書省言:「元時有太府監,本朝未嘗有,言涉不誠。」帝命卻其貢。是歲顓為權相李仁人所弒。顓無子,以寵臣辛肫之子禑為子,於是仁人立禑。

八年,禑遣判宗簿事崔原來告哀,且言前有貢使金義殺朝使蔡斌,今嗣王禑已誅義,籍其家。帝疑其詐,拘原而遣使往祭弔。十年,使來請故王顓謚號,帝曰:「顓被殺已久,今始請謚,將假吾朝命,鎮撫其民,且掩其弒逆之跡,不可許。前所留使者,其遣之。」於是釋原歸。其夏,復遣周誼貢馬及方物,卻不受。冬,又遣使賀明年正旦。帝曰:「高麗王顓被弒,奸臣竊命,《春秋》之義,亂臣必誅,夫又何言。第前後使者皆稱嗣王所遣,中書宜遣人往問嗣王如何,政令安在。若政令如前,嗣王不為羈囚,則當依前王言,歲貢馬千匹,明年貢金百斤、銀萬兩、良馬百、細布萬,仍悉送還所拘遼東民,方見王位真而政令行,朕無惑已。否則弒君之賊,必討無赦。」

十一年四月,禑復命誼來貢。十二年敕遼東守將潘敬、葉旺等謹飭邊備。其冬,禑遣李茂芳等來貢,以不如約卻之。十三年,遼東送高麗使誼至京師,帝敕敬等曰:「高麗弒君,又殺朝使,前堅請入貢又不如期,今遣誼來,以虛文飾詐,他日必為邊患。自今來者,其絕勿通。」因留誼於京師。十六年來貢,卻之,命禮部責其朝貢過期、陪臣侮慢之罪;誠欲聽約者,當以前五歲違約不貢之物並至。十七年六月,禑遣司僕正崔涓、禮儀判書金進宜貢馬二千匹。且言金非地所產,願以馬代輸,餘皆如約。遼東守將唐勝宗為之請,帝許之。然請顓謚號,襲王爵,未允也。

十八年正月,貢使至。帝諭禮臣曰:「高麗屢請約束,朕數不允,而其請不已,故索歲貢以試其誠偽,非以此為富也。今既聽命,宜損其貢數,令三年一朝,貢馬五十匹。至二十一年正旦乃貢。」七月,禑上表請襲爵,並請故王謚。命封禑為高麗國王,賜故王顓謚恭愍。

十九年二月遣使貢布萬匹、馬千匹。九月,表賀,貢方物。其後貢獻輒踰常額,且未嘗至三年也。冬,詔遣指揮僉事高家奴以綺布市馬於高麗。二十年三月,高家奴還,陳高麗表辭馬直,帝敕如數償之。先是,元末遼、沈兵起,民避亂,轉徙高麗。至是因市馬,帝令就索之,遂以遼、沈流民三百餘口來歸。十二月命戶部咨高麗王:「鐵嶺北,東西之地,舊屬開元者,遼東統之。鐵嶺之南,舊屬高麗者,本國統之。各正疆境,毋侵越。」

二十一年四月,禑表言,鐵嶺之地實其世守,乞仍舊便。帝曰:「高麗舊以鴨綠江為界,今飾辭鐵嶺,詐偽昭然。其以朕言諭之,俾安分,毋生釁端。」八月,高麗千戶陳景來降,言:「是年四月,禑欲寇遼東,使都軍相崔瑩、李成桂繕兵西京。成桂使陳景屯艾州,以糧不繼退師。王怒,殺成桂之子。成桂還兵攻破王城,囚王及瑩。」景懼及,故降。帝敕遼東嚴守備,仍遣人偵之。十月,禑請遜位於其子昌。帝曰:「前聞其王被囚,此必成桂之謀,姑俟之以觀變。」

二十二年,權國事昌奏乞入朝,帝不許。是歲,成桂廢昌,而立定昌國院君瑤。二十三年正月遣使來告。二十四年三月詔市馬高麗。八月,權國事瑤進所市馬千五百匹。帝曰;「三韓君臣悖亂,二紀於茲。今王瑤嗣立,乃王氏苗裔,宜遣使勞之。」十二月,瑤遣其子奭朝賀明年正旦。奭未歸而成桂自立,遂有其國,瑤出居原州。王氏自五代傳國數百年,至是絕。

二十五年九月,高麗知密直司事趙胖等持國都評議司奏言:「本國自恭愍王薨,無嗣,權臣李仁人以辛肫子禑主國事,昏暴好殺,至欲興師犯邊,大將李成桂以為不可而回軍。禑負罪惶懼,遜位於子昌。國人弗順,啟請恭愍王妃安氏擇宗親瑤權國事。已及四年,昏戾信讒,戕害勛舊,子奭癡騃不慧,國人謂瑤不足主社稷。今以安氏命,退瑤於私第。王氏子姓無可當輿望者,中外人心咸繫成桂。臣等與國人耆老共推主國事,惟聖主俞允。」帝以高麗僻處東隅,非中國所治,令禮部移諭:「果能順天道,合人心,不啟邊釁,使命往來,實爾國之福,我又何誅。」冬,成桂聞皇太子薨,遣使表慰,並請更國號。帝命仍古號曰朝鮮。

二十六年二月遣使進馬九千八百餘匹,命運糸寧絲綿布一萬九千七百餘匹酬之。六月表謝,貢馬及方物,並上前恭愍王金印,請更己名曰旦。從之。是月,遼東都指揮使司奏,朝鮮國招引女直五百餘人,潛渡鴨綠江,欲入寇。乃遣使敕諭,示以禍福。旦得敕,惶懼陳謝,上貢,並械送逋逃軍民三百八十餘人至遼東。

二十七年,旦遣子入貢。二十八年遣使柳珣賀明年正旦。帝以表文語慢,詰責之。珣言表文乃門下評理鄭道傳所撰,遂命逮道傳,釋珣歸。二十九年送撰表人鄭總等三人至,云表實總等所撰,道傳病不能行。帝以總等亂邦構釁,留不遣。三十年冬,復以表涉譏訕,拘其使。建文初,旦表陳年老,以子芳遠襲位。許之。

成祖立,遣官頒即位詔。永樂元年正月,芳遠遣使朝貢。四月復遣陪臣李貴齡入貢,奏芳遠父有疾,需龍腦、沈香、蘇合、香油諸物,齎布求市。帝命太醫院賜之,還其布。芳遠表謝,因請冕服書藉。帝嘉其能慕中國禮,賜金印、誥命、冕服、九章、圭玉、珮玉,妃珠翠七翟冠、霞帔、金墜,及經籍彩幣表裏。自後貢獻,歲輒四五至焉。

二年十二月詔立芳遠子禔為世子,從其請也。五年十二月,貢馬三千匹至遼東,命戶部運絹布萬五千匹償之。六年,世子禔來朝,賜織金文綺。及歸,帝親製詩賜之。時朝鮮納女後宮,立為妃嬪者四人。其秋,遣陪臣鄭擢來告其父旦之喪。命官弔祭,賜謚康獻。

十六年奏世子禔不肖,第三子祹孝弟力學,國人所屬,請立為嗣,詔聽王所擇。因上表謝,並陳己年老,請以祹理國事。命光祿少卿韓確、鴻臚丞劉泉封祹為朝鮮國王。時帝已遷北都,朝鮮益近,而事大之禮益恭,朝廷亦待以加禮,他國不敢望也。

二十年,芳遠卒,賜謚恭定。二十一年七月,祹請立嫡子珦為世子,從之。先是,敕祹貢馬萬匹,至是如數至,賜白金綺絹。

宣德二年三月遣中官賜白金糸寧紗,別敕進馬五千匹,資邊用。九月如數至。四年祹賜書:「珍禽異獸,非朕所貴,其勿獻。」後又敕祹:「金玉之器,非爾國所產,宜止之,土物效誠而已。」八年,祹奏遣子弟詣太學或遼東學,帝不許,賜《五經》、《四書》、《性理》、《通鑑綱目》諸書。

正統元年三月放朝鮮婦女金黑等五十三人還其國。金黑等自宣德初至京師,至是遣中官送回。三年八月賜祹遠游冠、絳紗袍、玉佩、赤舄。先是,建州長童倉避居朝鮮界,已復還建州。朝鮮言:「昔以窮歸臣,臣遇之善。今負恩還建州李滿住所,慮其同謀擾邊。」建州長言,所部為朝鮮追殺,阻留一百七十餘家。五年詔祹還之。七年五月諭祹曰:「鴨綠江一帶東寧等衛,密邇王境,中多細人逃至王國,或被國人誘脅去者,無問漢人、女直,至即解京。」初,瓦剌密令女直諸部誘朝鮮,使背中國。祹拒之,白其事於朝。帝嘉其忠,敕獎之,並賜彩幣。九年春,倭寇犯邊,祹命將擒獲五十餘人,械送京師。十年又獲餘黨來獻。帝連敕獎諭,賜賚加等。十三年冬,命使調發朝鮮及野人女直兵會遼東,征北寇。時英宗北狩,郕王即位,遣官頒詔於其國。

景泰元年貢馬五百匹。奏稱奉敕辦馬二三萬匹,比因鄰寇構釁,馬畜踣斃,一時未能。詔曰:「寇今少息。馬已至者,償其直。未至者,止勿貢。」是年夏,祹卒,賜弔祭,謚莊憲,封子珦為國王。會遼東奏報開原、沈陽有寇入境,掠人畜,係建州、海西、野人女直頭目李滿住等為嚮導,因諭珦相為掎角截殺之。其秋,續貢馬千五百餘匹。賜冕服,並償其直。冬又賜珦及妃權氏誥命,封其子弘為世子。二年冬,以建州頭目潛與朝鮮通,戒珦絕其使。三年秋,珦卒,來告哀。遣中官往弔祭,賜謚恭順,命子弘嗣立。弘立三年,以年幼且嬰夙疾,請以叔瑈權國事。七年上表遜位,乃封瑈為國王。瑈請立子暲為世子,從之。

天順三年,邊將奏,有建州三衛都督私與朝鮮結,恐為中國患。因敕瑈毋作不靖,貽後悔。瑈疏辨,復諭曰:「宣德、正統年間,以王國與彼互相侵掠,敕解怨息兵,初不令交通給賞授官也。彼既受朝廷官職,王又加之,是與朝廷抗也。王素秉禮義,何爾文過飾非?後宜絕私交,以全令譽。」四年復諭瑈曰:「王奏毛憐衛都督郎卜兒哈通謀煽亂,已置之法。夫法止可行於國中,豈得加於鄰境。郎卜兒哈有罪,宜奏朝廷區處。今輒行殺害,何怪其子阿比車之思復仇也。聞阿比車之母尚在,宜急送遼東都司,令阿比車領回,以解仇怨。」五年,建州衛野人至義州殺掠,瑈奏乞朝命還所掠。兵部議:「朝鮮先嘗誘殺郎卜兒哈,繼又誘致都指揮兀克,縱兵掠其家屬。今野人實係復仇,宜諭朝鮮,寇盜之來皆自取,惟守分安法,庶弭邊釁。」從之。

成化元年冬,陪臣李門炯來朝,卒於道。命給棺賜祭,並賜彩幣慰其家。時朝鮮頻貢異物,三年春,敕諭瑈修常貢,勿事珍奇。是時朝廷用兵征建州,敕瑈助兵進剿。瑈遣中樞府知事康純統眾萬餘渡鴨綠、潑豬二江,攻破九獮府諸寨,斬獲多。

四年正月遣官來獻俘。詔從厚賚,敕獎諭之。是年,瑈卒,賜謚惠莊。遣太監鄭同、崔安封世子晄為王,給妃韓氏誥命。既行,巡按遼東御史侯英奏曰:「遼東連年被寇,瘡痍未起,今復禾稼不登,軍民乏食。太監鄭同等隨從人員所過驛騷。臣考先年曾於翰林院中,選有學行文望者出使。今同、安俱朝鮮人,墳墓宗族皆在,見其國王,不免屈節,殊褻中國體。乞寢成命,或翰林,或給事中及行人內推選一員,往使為便。」帝曰:「英所言良是。自後賞賚遣內臣,其冊封正副使,選廷臣有學行者。」

六年,晄病篤,以所生子幼,命其兄故世子暲女子娎子權國事,遣陪臣以聞。及卒,賜謚襄悼,命娎嗣位,娎妻韓氏封王妃。十年追贈娎父世子暲為國王,謚懷簡,母韓氏為王妃,從所請也。

十一年四月,娎奏建州野人糾聚毛憐等衛侵擾邊境不已,乞朝命戒飭。十二年十月,娎為繼妻尹氏請封,賜誥命冠服。時禁外國互市兵器,娎奏:「小邦北連野人,南鄰倭島,五兵之用,不可缺一。而弓材所需牛角,仰於上國。高皇帝時嘗賜火藥、火炮,今望特許收買弓角,不與外番同禁。」兵部議歲市弓角五十,後以不足於用,請無限額,詔許倍市。

十五年十月命娎出兵夾擊建州女直。娎遂遣右贊成魚有沼率兵至滿浦江,以水泮後期。復遣左議政尹弼商、節度使金嶠等渡江進剿。十六年春遣陪臣來獻捷,帝命內官齎敕獎其能繼先烈,賜金幣,領兵官賞賚如例。後使還,遣其臣許熙伴送。熙歸至開州,建州騎二千邀之,掠其從卒三十餘人,馬二百三十餘匹,他所亡物稱是。奏聞,英國公張懋、吏部尚書尹旻等以遼東連年用兵,未可輕動,宜以此意諭娎。敕遼東守臣整飭邊備,更令譯者窮究所掠,期在必得,仍賜熙白金綵幣慰安之。

十七年,娎奏繼妃尹氏失德,廢置,乞更封副室尹氏。從之。十九年四月封娎長子心隆為世子。

弘治七年十二月,娎卒,賜謚康靖。明年四月,封心隆為國,妻慎氏為王妃。十二年,心隆奏:「本國人屢有違禁匿海島,誘引軍民,漸至滋蔓。乞許本國自行搜刷。其係上國地方,請敕官追捕。」時遼東守臣亦奏如心隆言,報可。十五年冬,封心隆長子為世子。

正德二年,心隆以世子夭亡,哀慟成疾,奏請以國事付其弟懌,其國人復奏請封懌。禮部議命懌權理國事,俟心隆卒乃封。既,陪臣盧公弼等以朝貢至京,復請封懌,廷議不允。十二月,心隆母妃奏懌長且賢,堪付重寄。於是禮部奏:「心隆以痼疾辭位,懌以親弟承托,接受既明,友愛不失。通國臣民舉無異詞,宜順其請。」上乃允懌嗣位,遣中官敕封,並賜其妃尹氏誥命。初,成桂之自立也,與宰相李仁人本異族。永樂間,降祭海嶽祝文,稱成桂為仁人子,而《祖訓》亦載仁人子成桂更名旦。後成桂子芳遠奏辨,太宗許令改正。至是修《大明會典》,仍列《祖訓》於朝鮮國。貢使市以歸,懌上疏備陳世系,辨先世無弒逆事,乞改正。禮部議:「《會典》詳載本朝制度,事涉外國,疑似之際,在所略。況成桂得國出皇祖命,其不繫仁人後,太宗詔可徵,宜從其請。」詔曰:「可。」

十五年冬,命內官封懌子峼為世子,賜懌金帛珠玉,令括取異物及童男女以進。十六年,世宗即位,禮官言:「天子初踐祚,宜正中國之體,絕外裔狎侮之端。請諭懌非朝廷意,召內臣還,毋有所索取。」帝從之。

嘉靖二年八月,以俘獲倭夷來獻,並送還中國被掠八人。賜白金錦糸寧。

八年八月,陪臣柳溥上言:國祖李旦係本國全州人。二十八世祖瀚仕新羅為司空。新羅亡,六世孫兢休入高麗。十三世孫安社仕元為南京千戶所達魯花赤。元季兵興,安曾孫子春與男成桂避地東遷。至正辛丑,當恭愍王之十年,有紅巾賊入境,成桂擊賊有功,授武班職事,時尚未知名。恭愍無嗣,陰蓄寵臣辛肫之子禑為子,晚為嬖臣洪倫、內豎崔萬生所弒。權臣李仁人誅倫、萬生而立禑,擢成桂為門下侍中。禑遣成桂侵遼東,成桂不從,返兵。禑懼,遜位於子昌。昌以偽姓見黜,復立王氏裔定昌君瑤,竄仁人於外。瑤復不道,國人戴成桂,請於高皇帝,立為王,更名旦,贍瑤別邸,終其身,實未嘗為弒。前永樂、正德間屢經奏請,俱蒙俞允,而迄未改正。今遇重修《會典》,乞賜昭雪。」詔送史館編纂。

十八年二月,睿宗祔太廟,配享明堂禮成,懌表賀。帝特御奉天門引見,賜宴禮部。

二十三年冬,懌卒。二十四年正月來訃,賜謚恭僖。詔立其子峼。峼未踰年卒,賜謚榮靖。九月,峼弟權國事峘遣使謝祭謚,並請襲封,詔許之。

二十五年,峘遣使送下海番人六百餘至邊,賜金幣。二十六年正月,峘咨稱:「福建人從無泛海至本國者,因往日本市易,為風所漂,前後共獲千人以上,皆挾軍器貨物,致中國火炮亦為倭有,恐起兵端。」詔:「頃年沿海奸民犯禁,福建尤甚,往往為外國所獲,有傷國體。海道官員令巡按御史察參。仍賜王銀幣,以旌其忠。」

三十一年冬,以洪武、永樂間所賜樂器敝壞,奏求律管,更乞遣樂官赴京校習,許之。

三十五年五月有倭船四自浙、直敗還,漂入朝鮮境。峘遣兵擊殲之,得中國被俘及助逆者三十餘人來獻,因賀冬至節,帝賜璽書褒諭。三十八年十一月奏:「今年五月,有倭寇駕船二十五隻來抵海岸,臣命將李鐸等剿殺殆盡,獲中國民陳春等三百餘人,內招通倭嚮導陳得等十六人,俱獻闕下。」復降敕獎勵,厚賚銀幣,並賜鐸等有差。

四十二年九月,峘復上書辨先世非李仁人後,今修《會典》雖蒙釐正,乞著始祖旦、父子春之名,帝令附錄《會典》。

隆慶元年六月遣官頒即位詔。時帝將幸太學,來使乞留觀禮,許之。是年冬,峘卒,賜謚恭憲,命其姪昖襲封。

萬歷元年正月上穆宗尊謚、兩宮徽號禮成,昖表賀,獻方物馬匹。時昖屢請賜《皇明會典》,為其先康獻王旦雪冤。十六年正月,《會典》成,適貢使愈泓在京,請給前書,以終前命。許之。十七年十一月,陪臣奇芩等入賀冬至,奏稱本年六月,大琉球國船遭風至海岸,所有男婦合解京,給文放歸。從之。

十九年十一月奏,倭酋關白平秀吉聲言明年三月來犯,詔兵部申飭海防。平秀吉者,薩摩州人,初隨倭關白信長。會信長為其下所弒,秀吉遂統信長兵,自號關白,劫降六十餘州。朝鮮與日本對馬島相望,時有倭夷往來互市。二十年夏五月,秀吉遂分渠帥行長、清正等率舟師逼釜山鎮,潛渡臨津。時朝鮮承平久,兵不習戰,昖又湎酒,弛備,猝島夷作難,望風皆潰。昖棄王城,令次子琿攝國事,奔平壤。已,復走義州,願內屬。七月,兵部議令駐札險要,以待天兵;號召通國勤王,以圖恢復。而是時倭已入王京,毀墳墓,劫王子、陪臣,剽府庫,八道幾盡沒,旦暮且渡鴨綠江,請援之使絡繹於道。廷議以朝鮮為國籓籬,在所必爭。遣行人薛潘諭昖以興復大義,揚言大兵十萬且至。而倭業抵平壤,朝鮮君臣益急,出避愛州。遊擊史儒等率師至平壤,戰死。副總兵祖承訓統兵渡鴨綠江援之,僅以身免。中朝震動,以宋應昌為經略。八月,倭入豐德等郡,兵部尚書石星計無所出,議遣人偵探之,於是嘉興人沈惟敬應募。惟敬者,市中無賴也。是時秀吉次對馬島,分其將行長等守要害為聲援。惟敬至平壤,執禮其卑。行長紿曰:「天朝幸按兵不動,我不久當還。以大同江為界,平壤以西盡屬朝鮮耳。」惟敬以聞。廷議倭詐未可信,乃趣應昌等進兵。而星頗惑於惟敬,乃題署遊擊,赴軍前,且請金行間。十二月,以李如松為東征提督。明年正月,如松督諸將進戰,大捷於平壤。行長渡大同江,遁還龍山。所失黃海、平安、京畿、江原四道並復,清正亦遁還王京。如松既勝,輕騎趨碧蹄館,敗,退駐開城。事具《如松傳》。

初,如松誓師,欲斬惟敬,以參軍李應試言而止。至是敗,氣縮,而應昌急圖成功,倭亦乏食有歸志,因而封貢之議起。應昌得倭報惟敬書,乃令遊擊周弘謨同惟敬往諭倭,獻王京,返王子,如約縱歸。倭果於四月棄王城遁。時漢江以南千有餘里朝鮮故土復定,兵部言宜令王還國居守,我各鎮兵久疲海外,以次撤歸為便。詔可。應昌疏稱:「釜山雖瀕海南,猶朝鮮境,有如倭覘我罷兵,突入再犯,朝鮮不支,前功盡棄。今撥兵協守為第一策,即議撤,宜少需,俟倭盡歸,量留防戍。」部議留江浙兵五千,分屯要害,仍諭昖搜練軍實,毋恃外援。已而沈惟敬歸自釜山,同倭使來請款,而倭隨犯咸安、晉州,逼全羅,聲復漢江以南,以王京、漢江為界。如松計全羅饒沃,南原府尤其咽喉,乃命諸將分守要害。已,倭果分犯,我師並有斬獲。兵科給事中張輔之、遼東都御史趙耀皆言款貢不可輕受。七月,倭從釜山移西生浦,送回王子、陪臣。時師久暴露,聞撤,勢難久羈。應昌請留劉綎川兵,吳惟忠、駱尚志等南兵,合薊、遼兵共萬六千,聽綎分布尚之大丘,月餉五萬兩,資之戶兵二部。先是,發帑給軍費,已累百萬。廷臣言虛內實外非長策,請以所留川兵命綎訓練,兵餉令本國自辦。於是詔撤惟忠等兵,止留綎兵防守。諭朝鮮世子臨海君珒居全慶,以顧養謙為經略。九月,昖以三都既復,疆域再造,上表謝恩。然時倭猶據釜山也,星益一意主款。九月,兵部主事曾偉芳言:「關白大眾已還,行長留待。知我兵未撤,不敢以一矢加遺。欲歸報關白捲土重來,則風不利,正苦冬寒。故款亦去,不款亦去。沈惟敬前於倭營講購,咸安、晉州隨陷,而俗恃款冀來年不攻,則速之款者,正速之來耳。故款亦來,不款亦來。宜令朝鮮自為守,弔死問孤,練兵積粟,以圖自強。」帝以為然,因敕諭昖者甚至。

二十二年正月,昖遣金晬等進方物謝恩。禮部郎中何喬遠奏:「晬涕泣言倭寇猖獗,朝鮮束手受刃者六萬餘人。倭語悖慢無禮,沈惟敬與倭交通,不云和親,輒曰乞降。臣謹將萬曆十九年中國被掠人許儀所寄內地書、倭夷答劉綎書及歷年入寇處置之宜,乞特敕急止封貢。」詔兵部議。時廷臣交章,皆以罷封貢、議戰守為言。八月,養謙奏講貢之說,貢道宜從寧波,關白宜封為日本王,諭行長部倭盡歸,與封貢如約。九月,昖請許保國。帝乃切責群臣阻撓,追褫御史郭實等。詔小西飛入朝,集多官面議,要以三事:一,勒倭盡歸巢;一,既封不與貢;一,誓無犯朝鮮。倭俱聽從,以聞。帝復諭於左闕,語加周復。十二月,封議定,命臨淮侯李宗城充正使,以都指揮楊方亨副之,同沈惟敬往日本,王給金印,行長授都督僉事。

二十三年九月,昖奏立次子琿為嗣。先是,昖庶長子臨海君珒陷賊中,驚憂成疾,次子光海君琿收集流散,頗著功績,奏請立之。禮部尚書范謙言繼統大義,長幼定分,不宜僭差,遂不許。至是復奏,引永樂間恭定王例上請,禮臣執奏,不從。二十四年五月,昖復疏請立琿,禮部仍執不可,詔如議。時國儲未建,中外恫疑,故尚書范謙於朝鮮易封事三疏力持云。

九月,封使至日本。先是,沈惟敬抵釜山,私奉秀吉蟒玉、翼善冠、地圖、武經、良馬。而李宗城以貪淫為倭守臣所逐,棄璽書夜遁。事聞,逮問。乃以方亨充正使,加惟敬神機營銜副之。及是奉冊至,關白怒朝鮮王子不來謝,止遣二使奉白土綢為賀,拒其使不見,語惟敬曰:「若不思二子、三大臣、三都、八道悉遵天朝約付還,今以卑官微物來賀,辱小邦邪?辱天朝邪?且留石曼子兵於彼,候天朝處分,然後撤還。」翌日奉貢,遣使齎表文二道,隨冊使渡海至朝鮮。廷議遣使於朝鮮,取表文進驗,其一謝恩,其一乞天子處分朝鮮。

初,方亨詭報去年從釜山渡海,倭於大版受封,即回和泉州。然倭方責備朝鮮,仍留兵釜山如故,謝表後時不發,方亨徒手歸。至是,惟敬始投表文,案驗潦草,前折用豐臣圖書,不奉正朔,無人臣禮。而寬奠副總兵馬楝報清正擁二百艘屯機張營。方亨始直吐本末,委罪惟敬,並呈石星前後手書。帝大怒,命逮石星、沈惟敬案問。以兵部尚書邢玠總督薊、遼;改麻貴為備倭大將軍,經理朝鮮;僉都御史楊鎬駐天津,申警備;楊汝南、丁應泰贊畫軍前。

五月,玠至遼。行長建樓,清正布種,島倭窖水,索朝鮮地圖,玠遂決意用兵。麻貴望鴨綠江東發,所統兵僅萬七千人,請濟師。玠以朝鮮兵惟嫻水戰,乃疏請募兵川、浙,並調薊、遼、宣、大、山、陜兵及福建、吳淞水師,劉綎督川、漢兵聽剿。貴密報候宣、大兵至,乘倭未備,掩釜山,則行長擒,清正走。玠以為奇計,乃檄楊元屯南原,吳惟忠屯忠州。

六月,倭數千艘泊釜山,戮朝鮮郡守安弘國,漸逼梁山、熊川。惟敬率營兵二百,出入釜山。玠陽為慰藉,檄楊元襲執之,縛至貴營,惟敬執而嚮導始絕。七月,倭奪梁山、三浪,遂入慶州,侵閑山。統制元均兵潰,遂失閑山。閑山島在朝鮮西海口,右障南原,為全羅外籓,一失守則沿海無備,天津、登、萊皆可揚帆而至。而我水兵三千甫抵旅順,閑山破,經略檄守王京西之漢江、大同江,扼倭西下,兼防運道。

八月,清正圍南原,乘夜猝攻,守將楊元遁。時全州有陳愚衷,去南原僅百里,南原告急,愚衷不敢救,聞已破,棄城走。麻貴遣遊擊牛伯英赴援,與愚衷合兵,屯公州。倭遂犯全慶,逼王京。王京為朝鮮八道之中,東阻烏嶺、忠州,西則南原、全州,道相通。自二城失,東西皆倭,我兵單弱,因退守王京,依險漢江。麻貴請於玠欲棄王京退守鴨綠江。海防使蕭應宮以為不可,自平壤兼程趨王京止之。麻貴發兵守稷山,朝鮮亦調都體察使李元翼由烏嶺出忠清道遮賊鋒。玠既身赴王京,人心始定。玠召參軍李應試問計,應試請問廟廷主畫云何。玠曰:「陽戰陰和,陽剿陰撫,政府八字密畫,無泄也。應試曰:「然則易耳。倭叛以處分絕望,其不敢殺楊元,猶望處分也。直使人諭之曰沈惟敬不死,則退矣。」因請使李大諫於行長,馮仲纓於清正,玠從之。九月,倭至漢江,楊鎬遣張貞明持惟敬手書往,責其動兵,有乖靜候處分之實。行長、正成亦尤清正輕舉,乃退屯井邑。麻貴遂報青山、稷山大捷。蕭應宮揭言:「倭以惟敬手書而退,青山、稷山並未接戰,何得言功?」玠、鎬怒,遂劾應宮恇怯,不親解惟敬,並逮。

十一月,玠徵兵大集,帝發帑金犒軍,賜玠尚方劍,而以御史陳效監其軍。玠大會諸將,分三協。鎬同貴率左右協,自忠州、烏嶺向東安,趨慶州,專攻清正。使李大諫通行長,約勿往援。復遣中協屯宜城,東援慶州,西扼全羅。以餘兵會朝鮮合營,詐攻順天等處,以牽制行長東援。十二月,會慶州。麻貴遣黃慶賜賄清正約和,而率大兵奄至其營。時倭屯蔚山,城依山險,中一江通釜寨,其陸路由彥陽通釜山。貴欲專攻蔚山,恐釜倭由彥陽來援,乃多張疑兵,又遣將遏其水路,遂進逼倭壘。遊擊擺寨以輕騎誘倭入伏,斬級四百餘,獲其勇將,乘勝拔兩柵。倭焚死者無算,遂奔島山,連築三寨。翌日,遊擊茅國器統浙兵先登,連破之,斬獲甚多,倭堅壁不出。島山視蔚山高,石城堅甚,我師仰攻多損傷。諸將乃議曰:「倭艱水道,餉難繼,第坐困之,清正可不戰縛也。」鎬等以為然,分兵圍十日夜,倭饑甚,偽約降緩攻。俄行長援兵大至,將繞出軍後。鎬不及下令,策馬西奔,諸軍皆潰。遂撤兵還王京,士卒物故者二萬。上聞之,震怒。乃罷鎬聽勘,以天津巡撫萬世德代。事詳《鎬傳》。

二十六年正月,邢玠以前役乏水兵無功,乃益募江南水兵,議海運,為持久計。二月,都督陳璘以廣兵,劉綎以川兵,鄧子龍以浙、直兵先後至。玠分兵三協,為水陸四路,路置大將。中路如梅,東路貴,西路綎,水路璘,各守汛地,相機行剿。時倭亦分三窟。東路則清正,據蔚山。西路則行長,據粟林、曳橋,建砦數重。中路則石曼子,據泗州。而行長水師番休濟餉,往來如駛。我師約日並進,尋報遼陽警,李如松敗沒,詔如梅還赴之,中路以董一元代。

當應泰之劾鎬也,昖請回乾斷,崇勵鎮撫,以畢征討。上不許。又應泰曾以築城之議為鎬罪案,謂堅城得志,啟朝鮮異日之患,於是昖奏辨。帝曰:「連年用兵發餉,以爾國素效忠順故也,毋以人言自疑。」

九月,將士分道進兵,劉綎進逼行長營,約行長為好會。翌日,攻城,斬首九十二。陳璘舟師協堵擊,毀倭船百餘。行長潛出千餘騎扼之,綎不利,退,璘亦棄舟走。麻貴至蔚山,頗有斬獲,倭偽退誘之。貴入空壘,伏兵起,遂敗。董一元進取晉州,乘勝渡江,連毀二寨。倭退保泗州老營,鏖戰下之,前逼新寨。寨三面臨江,一面通陸,引海為濠,海艘泊寨下千計,築金海、固城為左右翼。十月,董一元遣將四面攻城,用火器擊碎寨門,兵競前拔柵。忽營中火藥崩,煙焰漲天。倭乘勢衝擊,固城倭亦至,兵遂大潰,奔還晉州。帝聞,命斬二遊擊以徇,一元等各帶罪立功。是月,福建都御史金學曾報七月九日平秀吉死,各倭俱有歸志。十一月,清正發舟先走,麻貴遂入島山、酉浦,劉綎攻奪曳橋。石曼子引舟師救行長,陳璘邀擊敗之。諸倭揚帆盡歸。自倭亂朝鮮七載,喪師數十萬,糜餉數百萬,中朝與屬國迄無勝算,至關白死而禍始息。

二十七年閏四月,以平倭詔告天下,又敕諭昖曰:「倭奴平秀吉肆為不道,蹂躪爾邦。朕念王世篤忠貞,深用憫惻。七年之中,日以此賊為事。始行薄伐,繼示包容,終加嚴討。蓋不殺乃天之心,而用兵非予得已。安疆靖亂,宜取蕩平。神惡凶盈,陰殲魁首,大師乘之,追奔逐北,鯨鯢盡戮,海隅載清,捷書來聞,憂勞始釋。惟王雖還舊物,實同新造,振凋起敝,為力倍艱。倭雖遁歸,族類尚在。茲命邢玠振旅歸京,量留萬世德等分布戍守。王宜臥薪嘗膽,無忘前恥,惟忠惟孝,纘紹前休。」五月,玠條陳東征善後事宜十事。一,留戍兵,馬步水陸共計三萬四千有奇,馬三千匹。一,定月餉,每年計銀九十一萬八千有奇。一,定本色,合用米豆,分派遼東、天津、山東等處,每年十三萬石。一,留中路海防道。一,裁餉司。一,重將領。一,添巡捕。一,分汛地。一,議操練。一,責成本國。廷臣議:「數年疲耗,今始息肩,自宜內固根本,不當更為繁費。況彼國兵荒之後,不獨苦倭之擾,兼苦我兵。故今日善後事宜,仍當商之彼國,先量彼餉之贏絀,始可酌我兵之去留。至於增買馬匹,添補標兵,創立巡捕,及至管餉府佐,悉宜停止。」帝命督撫會同國王酌奏。八月,昖獻方物,助大工,褒賞如例。十月,請留水兵八千,以資戍守。其撤回官兵,駐札遼陽備警。二十八年四月請將義州等倉遺下米豆運回遼陽。戶部議:「輸運維艱,莫若徑與彼國,振其彫敝,以昭皇仁。」詔曰:「可。」

二十九年二月,兵部覆奏經督條陳七事:「一,練兵士。麗人鷙悍耐寒苦,而長衫大袖,訓練無方,宜以束伍之法教之。一,守衝要。朝鮮三面距海,釜山與對馬相望,巨濟次之,宜各守以重兵,並蔚山、開山等處皆宜戍守。一,修險隘。王京北倚叢山,南環滄海。忠州左右烏、竹二嶺,羊腸繞曲,有一夫當關之險。今營壘遺址尚存,亟宜修葺。一,建城池。朝鮮八道,十九無城。平壤西北鴨、浿二江,俱南通海。倘倭別遣一旅占據平壤,則王京聲援斷絕,皆應修築屯聚。一,造器械。倭戰便陸不便海,以船制重大,不利攻擊。今準福唬造百十艘為奇兵,並添造神機百子火箭。一,訪異材。朝鮮貴世官,賊世役,一切禁錮,往往走倭走敵,為本國患,宜破格搜採。一,修內治。國家東南臨海,以登、旅為門戶,鎮江為咽喉,應援之兵,不宜盡撤。我自固,亦所以固朝鮮也。」詔朝鮮刻勵以行。九月,奏所頒誥命冕服遭變淪失,祈補給,從之。

時倭國內亂,對馬島主平義智悉遣降人還朝鮮,遺書乞和,且揚言秀吉將家康將輸糧數十萬石為軍興資,以脅朝鮮。朝鮮與對馬島一水相望,島地不產五穀,資米於朝鮮。兵興後,絕開市,因百計脅款。秀吉死,我軍盡撤,朝鮮畏倭滋甚。欲與倭通款,又懼開罪中國。十二月,昖以島倭求款來請命。兵部以事難遙度,令總督世德酌議,詔可。三十年十一月,昖言倭使頻來要挾和款,兵端漸露,乞選將率兵,督同本國訓練修防。帝曰:「曾留將士教習,成法具在,無容再遣。因命其使臣齎敕誡勵。三十三年九月,昖復請封琿為世子,禮部仍執立長之議。三十五年四月,昖以家康求和來告,兵部議聽王自計而已。由是和款不絕,後三年始畫開市之事。

三十六年,昖卒。光海君琿自稱署國事,追陪臣來訃,且請謚。帝惡其擅,不允,令該國臣民公義以聞。時我大清兵征服各部,漸近朝鮮。兵部議令該王大修武備,整飭邊防,並請敕遼左督撫鎮臣,遣官宣達毋相侵犯之意。從之。十月,封琿為國王,從其臣民請也。三十七年二月,謚昖曰昭敬,遣官賜琿及妃柳氏誥命。

初,朝鮮失守,賴中國力得復,倭棄釜山遁。然陰謀啟疆,為患不已。於是海上流言倭圖釜山,朝鮮與之通。四十一年九月,總兵官楊宗業以聞。琿疏辨,詔慰解之。

四十二年四月,奏請追封生母金氏。禮部按《會典》,嫡母受封而生母先亡者得追贈,乃命封為國王次妃。四十三年十一月,表賀冬至,因奏買回《吾學編》、《弇山堂別集》等書,載本國事與《會典》乖錯,乞改正。禮部言:「野史不足憑。今所請恥與逆黨同譏,宜憫其誠,宣付史館。」報可。初,琿為生母已得封,至是復祈給冠服。禮臣以金氏側室,禮有隆殺,執不可。四十五年正月,琿請至再,帝以琿屢次懇陳,勉從之。

四十七年,楊鎬督馬林、杜松、劉綎等出師,為我大清兵所敗。朝鮮助戰兵將,或降或戰死。琿告急,詔加優恤。十一月,兵部覆:朝鮮入貢之道,宜添兵防守。詔鎮江等處設兵將,令經略熊廷弼調委。四十八年正月,琿奏:「敵兵八月中攻破北關,金台吉自焚,白羊出降。鐵嶺之役,蒙古宰賽亦為所滅。聞其國謀議以朝鮮、北關、宰賽皆助兵南朝,今北關、宰賽皆滅,不可使朝鮮獨存。又聞設兵牛毛寨、萬遮嶺,欲略寬奠、鎮江等處。寬奠、鎮江與昌城、義州諸堡隔水相望,孤危非常。敵若從靉陽境上鴉鶻關取路繞出鳳凰城裏,一日長驅,寬鎮、昌城俱莫自保。內而遼左八站,外而東江一城,彼此隔斷,聲援阻絕,可為寒心。望速調大兵,共為掎角,以固邊防。」時遼鎮塘報稱朝鮮與大清講和,朝議遂謂琿陽衡陰順,宜遣官宣諭,或命將監護,其說紛拿。琿疏辨:「二百年忠誠事大,死生一節。」詞極剴摯。禮、兵二部請降敕令曉諭,以安其心。帝是其議,然敕令陪臣往,不遣官也。

天啟元年八月,改朝鮮貢道,自海至登州,直達京師。時毛文龍以總兵鎮皮島,招集逃民為兵,而仰給於朝鮮。十一月,琿奏力難饋餉,乞循萬歷東征例,發運山東粟,從之。三年四月,國人廢琿而立其姪綾陽君倧,以昭敬王妃之命權國事,令議政府移文督撫轉奏,文龍為之揭報。登州巡撫袁可立上言:「琿果不道,宜聽太妃具奏,以待中國更立。」疏留中。八月,王妃金氏疏請封倧,禮部尚書林堯俞言:「朝鮮廢立之事,內外諸臣抒忠發憤,有謂宜聲罪致討者,有謂勿遽討且受方貢核顛末者,或謂當責以大義,察輿情之向背者,或謂當令人宗討敵自洗者,眾論咸有可采。其謂琿實悖德,倧討叛臣以赤心奉朝廷者,惟文龍一人耳。皇上奉天討逆,扶植綱常,此正法也。毋亦念彼素稱恭順,迥異諸裔,則更遣貞士信臣,會同文龍,公集臣民,再四詢訪。勘辨既明,再請聖斷。」報可。十二月,禮部復上言:「臣前同兵部移咨登撫,並札毛師,遣官往勘。今據申送彼國公結十二道,自宗室至八道臣民共稱倧為恭順。且彼之陪臣相率哀籲,謂當此危急之秋,必須君國之主。乞先頒敕諭,令倧統理國事,仍令發兵索賦,同文龍設伏出奇,俟漸有次第,始遣重臣往正封典。庶幾字小之中,不失固圉之道。」從之。四年四月,封倧為國王。

五年十二月,文龍報:「朝鮮逆黨李適、韓明璉等起兵昌城,直趨王京,被臣擒獲。餘孽韓潤、鄭梅等竄入建州,有左議府尹義立約為內應,期今冬大舉犯朝鮮。臣已咨國王防守,暫移鐵山之眾就雲從島柴薪。」登萊巡撫武之望奏:「毛帥自五月以來,營室於須彌,所謂雲從島是也。今十月又徙兵民商賈以實之,而鐵山之地空矣。故朝鮮各道疑其有逼處之嫌,甚至布兵以防禦之。今鎮臣所稱李適等之叛,尹義立之內應,臣等微聞之,而未敢遽信焉。信之則益重鮮人之疑,不信則恐貽後來之患。」兵部言:「牽制敵國者,朝鮮也;聯屬朝鮮者,毛鎮也;駕馭毛鎮者,登撫也。今撫臣與鎮臣不和,以至鎮臣與屬國不和,大不利。」帝乃飭勉鎮撫同心,而韓潤、尹義立等令朝鮮自處。倧又請撤遼民安插中土,兵部言:「遼人去留,文龍是視。文龍一日不去,則遼人一日不離。鮮人驅之入島可也,驅之離島不可也。宜令鎮臣將遼民盡刷過島,登撫刻期運糧朝鮮,量行救振,以資屯牧。」帝是之。

六年十月,倧上疏曰:

皇朝之於小邦,覆幬之恩,視同服內。頃遭昏亂,潛通敵國,皇天震怒,降黜厥命。臣自權署之初,不敢遑寧,即命陪臣張晚為帥,李適副之,付以國中精銳,進屯寧邊,一聽毛鎮節制,以候協剿之期。而適重兵在握,潛蓄覬覦,遂與龜城府使明璉舉兵內叛,直犯京城。晚收餘兵躡其後,與京輔官兵表裏夾攻,賊皆授首,而西邊軍實及列鎮儲偫罄於是役矣。

毛鎮當全遼淪沒之後,孤軍東渡,寄寓海上,招集遼民前後數十萬,亦小邦所仰藉也。顧以封疆多故,土瘠民貧,內供本國之軍需,外濟鎮兵之待哺,生穀有限,支給實難。遼民迫於饑餒,散布村落,強者攫奪,弱者丐乞。小邦兵民被撓不堪,拋棄鄉邑,轉徙內地。遼民逐食,亦隨而入。自昌、義以南,安、肅以北,客居六七,主居三四。向者將此情形具奏,見兵部題覆處分已定,何敢再幹。

至韓潤及弟潭係逆賊明璉子姪,亡命潛逃,因而勾引來寇。賊既叛國而去,制命已不在臣。尹義立曾任判書,本非議政。頃年差為毛鎮接伴官,不稱任使,褫職歸家,並無怨叛之事。毛鎮據王仲保等所訴,都無實事。意必有讒邪之臣,欺妄督撫,以售其交構之計者。

毛帥久鎮海外,臣與周旋已近十稔。雖餼牽將竭,彼此俱困,而情誼之殷,實無少損。且其須彌之遷,直為保護累重,將以就便芻薪。一進一退,兵家常事。訛言噂沓,本不介意。竊見部撫移咨曰「虞其逼處」,曰「驅其民,驅其帥」,甚至有「布兵以防,屬國攜貳」之語,似海外情事,未盡諒悉。臣之請刷遼民,因力不足濟,初非慮及逼處也。臣方與毛鎮同心一力,建功報主,豈敢有一毫猜防意乎。

帝報曰:「王和協東鎮,愛戴中朝,忠貞之忱,溢乎言表。鎮軍久懸,鮮、遼雜處。久客累主,生寡食多。微王言,朕有不坐照萬里之外者。然毛帥在中朝為牽制之師,在王國則脣齒之形也。海上芻輓,已令該部區畫,刻期運濟。逃難邊民,亦令毛帥悉心計處,俾無重為王累。傳訛之言,未足介懷,並力一心,王其勉之。」

七年三月,兵部上文龍揭言:「麗官、麗人招敵攻鐵山,傷我兵千人,殺麗兵六萬,焚糧百餘萬,敵遂移兵攻麗矣。」帝敕文龍速相機應援。登撫李嵩奏:「朝鮮叛臣韓潤等引敵入安州,節度使南以興自焚死,中國援兵都司王三桂等俱陣亡。」既復奏:「義州及郭山、凌漢、山城俱破,平壤、黃州不戰自潰,敵兵直抵中和,遊騎出入黃、鳳之間,又分向雲從,攻掠毛帥,國王及士民遷於江華以避難。」時大清兵所至輒下,朝鮮列城望風奔潰,乃遣使諭倧。倧輸款,遂班師。九月,倧奏被兵情形。時熹宗崩,莊烈帝嗣位,優詔勵勉焉。

崇禎二年,改每歲兩貢為一貢。先是,遼路阻絕,貢使取道登、萊,已十餘年矣。自袁崇煥督師,題改覺華,迂途冒險,其國屢請復故。至是遣戶曹判書鄭斗源從登海來,移書登撫孫元化,屬其陳請。元化委官伴送,仍疏聞。帝以水路既有成命,改途嫌於自便,不許。是年六月,督師袁崇煥殺平遼將軍左都督毛文龍於雙島。六年六月,倧遺書總兵黃龍言:「文龍舊將孔有德、耿仲明率士卒二萬投順大清,向朝鮮徵糧。本國以有德等曩在皮島為本國患,故未之應。」龍以聞。十年正月,太宗文皇帝親征朝鮮,責其渝盟助明之罪,列城悉潰。朝鮮告急,命總兵陳洪範調各鎮舟師赴援。三月,洪範奏官兵出海。越數日,山東巡撫顏繼祖奏屬國失守,江華已破,世子被擒,國王出降。今大治舟艦,來攻皮島、鐵山,其鋒甚銳。宜急敕沈世魁、陳洪範二鎮臣,以堅守皮島為第一義。帝以繼祖不能協圖匡救,切責之。亡何,皮島並為大清兵所破,朝鮮遂絕,不數載而明亦亡矣。朝鮮在明雖稱屬國,而無異域內。故朝貢絡繹,錫賚便蕃,殆不勝書,止著其有關治亂者於篇。至國之風土物產,則具載前史,茲不復錄。


\end{pinyinscope}