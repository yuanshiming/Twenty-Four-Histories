\article{列傳第一百二十九}

\begin{pinyinscope}
周嘉謨張問達陸夢龍傅梅汪應蛟)王紀楊東明孫瑋鐘羽正陳道亨(子弘緒

周嘉謨,字明卿,漢川人。隆慶五年進士。除戶部主事,歷韶州知府。

萬歷十年遷四川副使,分巡瀘州。窮治大猾楊騰霄,置之死。建武所兵燔總兵官沈思學廨,單車諭定之。尋撫白草番。督兵邛州、灌縣,皆有方略。居五年,進按察使,移疾歸。久之,起故官。榷稅中官邱乘雲播虐,逮繫相屬。嘉謨檄所司拒絕,而搒殺奸民助虐者,乘雲為戢。

就遷左布政使。擢右副都御史,巡撫雲南。隴川宣撫多安民叛,入緬,據蠻灣。嘉謨討擒之,立其弟安靖而還。進兵部右侍郎,巡撫如故。黔國公沐昌祚侵民田八千餘頃,嘉謨劾治之,復劾其孫啟元罪狀。久之,改督兩廣軍務兼巡撫廣東。滿考,加右都御史。廣西土酋引交址兵內犯,官軍拒退之,嘉謨為增兵置戍。南海、三水、高要、四會、高明諸邑大水,壞圩岸,留贖鍰築之。

遷南京戶部尚書,尋召拜工部尚書。孝定后喪,內廷宣索不貲。嘉謨言喪禮有中制,不當信左右言,妄耗國帑,不納。俄改吏部尚書。

四十八年七月,神宗崩。八月丙午朔,光宗即位。鄭貴妃據乾清宮,且邀封皇太后。嘉謨從言官楊漣、左光斗等言,以大義責貴妃從子養性,示以利害。貴妃乃移慈寧宮,封后事亦寢。外廷皆言貴妃進侍姬八人,致帝得疾。二十六日,嘉謨因召見,以寡欲進規,帝注視久之,令皇長子諭外廷:「傳聞不可信。」諸臣乃退。二十九日,帝疾大漸,嘉謨偕大學士方從哲、劉一燝、韓爌等受顧命。其夕,帝崩。質明,九月乙亥朔,光宗遺詔皇長子嗣位,而李選侍專制宮中,勢頗張,廷臣慮不測。既入臨,請見皇長子,呼萬歲,奉至文華殿受朝,送居慈慶宮。嘉謨奏言:「殿下之身,社稷是託,出入不宜輕脫。大小殮,朝暮臨,須臣等至乃發。」皇長子頷之。諸大臣定議:皇長子以九月六日即位。選侍居乾清自如,且欲挾皇長子同居。嘉謨亟草疏率廷臣請移宮,光斗、漣繼之。五日,選侍始移噦鸞宮。時大故頻仍,國勢杌隉,首輔從哲首鼠兩端,一燝、爌又新秉政,嘉謨正色立朝,力持大議,中外倚以為重。神宗末,齊、楚、浙三黨為政,黜陟之權,吏部不能主。及嘉謨秉銓,惟才是任。光、熹相繼踐祚,嘉謨大起廢籍,耆碩滿朝。向稱三黨之魁及朋奸亂政者,亦漸自引去,中朝為清。已,極陳吏治敝壞,請責成撫、按、監司。上官注考,率用四六儷語,多失實,嘉謨請以六事定官評:一曰守,二曰才,三曰心,四曰政,五曰年,六曰貌。各注其實,毋飾虛詞。帝稱善,行之。

天啟元年,御史賈繼春得罪,其同官張慎言、高弘圖疏救,帝欲並罪之。嘉謨等力為解,乃奪慎言、弘圖俸而止。朱欽相、倪思輝被謫,嘉謨亦申救。給事中霍維華希魏忠賢指劾王安,置之死。嘉謨惡之,出維華於外。忠賢怒,嗾給事中孫傑劾嘉謨受劉一燝屬為安報仇,且以用袁應泰、佟卜年等為嘉謨罪。嘉謨求退,忠賢矯旨許之。大學士葉向高等請留嘉謨竣大計事,不聽。明年,廣寧陷,嘉謨憂憤,馳疏劾兵部尚書張鶴鳴主戰誤國罪。五年秋,忠賢黨周維持復劾嘉謨曲庇王安,遂削籍。

崇禎元年,薦起南京吏部尚書,加太子太保。明年,卒官,年八十四。贈少保。

張問達,字德允,涇陽人。萬歷十一年進士。歷知高平、濰二縣,有惠政。征授刑科給事中。寧夏用兵,請盡蠲全陜逋賦,從之。父喪除,起故官,歷工科左給事中。帝方營建兩宮,中官利乾沒,復興他役,問達力請停止,不納。俄陳礦稅之害,言:「閹尹一朝銜命,輒敢糾彈郡守,甚且糾撫按重臣。而孫朝所攜程守訓、陳保輩,至箠殺命吏,毀室廬,掘墳墓。不一按問,若萬方怨恫何!」典試山東,疏陳道中饑饉流離狀,請亟罷天下礦稅,皆不報。已,巡視廠庫。故事,令商人辦內府器物,僉名以進,謂之僉商。而諸高貲者率賄近幸求免,帝輒許之。問達兩疏爭執,又極論守訓罪,並寢不行。進禮科都給事中。劾晉江李贄邪說惑眾,逮死獄中。贄事具《耿定向傳》。

三十年十月,星變,復請盡罷礦稅。時比年日食皆在四月,問達以純陽之月其變尤大,先後疏請修省,語極危切,帝終不納。尋遷太常少卿,以右僉都御史巡撫湖廣。所部水災,數請蠲貸。帝方營三殿,採木楚中,計費四百二十萬有奇,問達多方拮據,民免重困。久之,召拜刑部右侍郎,署部事兼署都察院事。

四十三年五月,讞問張差梃擊事。問達從員外郎陸夢龍言,令十三司會訊,詞連鄭貴妃宮監龐保、劉成。中外籍籍,疑貴妃弟國泰為之。問達等奏上差獄。帝見保、成名,留疏不下。尋召方從哲、吳道南及問達等於慈寧宮,命並磔二人。甫還宮,帝意復變,乃先戮差,令九卿三法司會訊保、成於文華門。保、成供原姓名曰鄭進、劉登雲,而不承罪。方鞫時,東宮傳諭曰:「張差情實風癲,誤入宮門,擊傷內侍,罪不赦。後招保、成係內官,欲謀害本宮。彼何益,當以仇誣,從輕擬罪。」問達等以鞫審未盡,上疏曰:「奸人闖宮,事關宗社。今差已死,二囚易抵飾。文華門尊嚴之地,臣等不敢刑訊,何由得情?二囚偏詞,何足為據?差雖死,所供詞故在,其同謀馬三道等亦皆有詞在案,孰得而滅之?況慈寧召對,面諭並決。煌煌天語,通國共聞。若不付之外庭,會官嚴鞫,安肯輸情?既不輸情,安從正法?祖宗二百年來,未有罪囚不付法司,輒令擬罪者。且二人係內臣。法行自近,陛下尤當嚴其銜轡,而置之重辟。奈何任彼展辨,不與天下共棄之也。」帝以二囚涉鄭氏,付外庭,議益滋,乃潛斃之於內,言皆以創重身死。而馬三道等五人,命予輕比坐流配。其事遂止。是年解都察院事。久之,遷戶部尚書,督倉場。尋兼署刑部,拜左都御史。光宗疾大漸,同受顧命。

天啟元年冬,代周嘉謨為吏部尚書。連掌內外大計,悉葉公論。當是時,萬曆中建言詿誤獲譴諸臣棄林下久,死者已過半。問達等定議:以廷杖、繫獄、遣戍者為一等,贈官蔭子;貶竄、削籍者為一等,但贈官。獲恤者七十五人。

會孫慎行、鄒元標追論「紅丸」,力攻方從哲。詔廷臣集議,與議者百十餘人。問達既集眾議,乃會戶部尚書汪應蛟等上疏曰:

按慎行奏,首罪李可灼進紅丸。可灼先見從哲,臣等初未知。及奉召進乾清宮,候於丹墀,從哲與臣等共言李可灼進藥,俱慎重未決。俄宣臣等至宮內跪御前,先帝自言「朕躬虛弱」,語及壽宮,並諭輔陛下為堯、舜,因問「可灼安在」。可灼趨入,和藥以進,少頃又進。聖躬安舒就寢。此進藥始末,從哲及文武諸臣所共見者。是時群情倉惶,悽然共切,弒逆二字,何可忍言。在諸臣固諒從哲無是心,即慎行疏中亦已相諒。若可灼輕易進藥,非但從哲未能止,臣與眾人亦未能止,臣等均有罪焉。及御史王安舜等疏論可灼,從哲自應重擬,乃先止罰俸,繼令養疾,則失之太輕。今不重罪可灼,何以慰先帝而服中外之心?宜提付法司,正以刑辟。若崔文昇妄投涼藥,罪亦當誅。請並下法司,與可灼並按。從哲則應如其自請,削去官階,為法任咎,此亦大臣引罪之道宜然,而非臣等所敢議也。

至選侍欲垂簾聽政,群臣初入臨,閽者阻不容入,群臣排闥而進。哭臨畢,奉聖躬至文華殿,行朝謁嵩呼禮,復奉駕還慈慶宮。因議新主登極,選侍不當復居乾清。九卿即公疏請移,言官繼之,從哲始具揭奏請,選侍遂即日移宮。然輿論猶憾從哲之奏,不毅然為百僚倡。倘非諸臣共挾大義,連章急趨,則乾清何地,猶然混居,令得假竊魁柄,將如陛下登極還宮何!

疏入,帝謂從哲心跡自明,不當輕議,止逮可灼下吏。文升已安置南京,弗問。

問達歷更大任,「梃擊」、「紅丸」、「移宮」三大案並經其手。持議平允,不激不隨。先以秩滿,加太子太保,至是乞休,疏十三上。詔加少保,乘傳歸。

五年,魏忠賢擅國。御史周維持劾問達力引王之寀植黨亂政,遂削奪。御史牟志夔復誣問達贓私,請下吏按問。命捐貲十萬助軍興。頃之,問達卒。以巡撫張維樞言,免其半,問達家遂破。崇禎初,贈太保,予一子官。維持、志夔咸名挂逆案。

陸夢龍,字君啟,會稽人。萬曆三十八年進士。授刑部主事,進員外郎。

張差獄起,引凡向宮殿射箭、放彈、投磚石等律當以斬。獄具,提牢主事王之寀奏差口詞甚悉,乞敕會問,大理丞王士昌亦上疏趣之。時夢龍以典試廣東杜門,主事邢臺傅梅過之曰:「人情庇奸,而甘心儲皇。吾雖恤刑山右,當上疏極論,君能共事乎?」夢龍曰:「張公遇我厚,遽上疏,若張公何?當力爭之耳。」乃偕見問達。時郎中胡士相等不欲再鞫,趣問達具疏請旨,以疏入必留中,其事可遂寢。夢龍得其情,止勿復請。眾曰:「提馬三爺、李外父輩,非得旨不可。」夢龍曰:「堂堂法司,不能捕一編氓,須天子詔耶?差所供,必當訊實。」問達以為然。

明日,會訊,士相、永嘉、會禎、夢龍、梅、之寀及鄒紹先凡七人,惟之寀、梅與夢龍合。將訊,眾咸囁嚅。夢龍呼刑具三,無應者,擊案大呼,始具。差長身駢脅,睨視傲語,無風癲狀。夢龍呼紙筆,命畫所從入路。梅問:「汝何由識路?」差言:「我薊州人,非有導者,安得入?」問:「導者誰?」曰:「大老公龐公,小老公劉公。」且曰:「豢我三年矣,予我金銀壺各一。」夢龍曰:「何為?」曰:「打小爺。」於是士相立推坐起曰:「此不可問矣。」遂罷訊。夢龍必欲得內豎名。越數日,問達再令十三司會審,差供逆謀及龐保、劉成名,一無所隱。士相主筆,躊躇不敢下,郎中馬德灃趣之,永嘉復以為難。夢龍咈然曰:「陸員外不肯匿,誰敢匿?」獄乃具。給事中何士晉遂疏詆鄭國泰。帝於是斃保、成於內,而棄差市,梅慮其潛易,躬請監刑。當是時,自夢龍、之寀、梅、德灃外,鮮不為鄭氏地者。已而之寀、德灃悉被罪,梅以京察罷官。夢龍賴問達力獲免,由郎中歷副使。

天啟四年,貴州賊未靖,總督蔡復一薦夢龍知兵,改右參政,監軍討賊,安邦彥犯普定。夢龍偕總兵黃鉞以三千人禦之。曉行大霧中,直前薄賊,賊大敗。三山苗叛,思州告急。夢龍夜遣中軍吳家相進搗賊巢,撾苗鼓,聲振山谷,苗大奔潰,焚其巢而還。尋改湖廣監軍,遷廣東按察使。上官建忠賢祠,列夢龍名,亟遣使鏟去之。

崇禎元年大計,忠賢黨猶用事,鐫二級調任。三年起副使,以故官分巡東兗道。盜起曹、濮間,討斬其魁,餘眾悉降。遷右參政,守固原。夢龍慷慨好談兵,以廓清群盜自負。七年夏,賊來犯,擊卻之。閏八月,賊陷隆德,殺知縣費彥芳,遂圍靜海州。夢龍率遊擊賀奇勛、都司石崇德禦之,抵老虎溝。賊初不滿千,已而大至。夢龍所將止三百餘人,被圍數重,賊矢石如雨,突圍不得出。二將抱夢龍泣,夢龍揮之曰:「何作此婦孺態!」大呼奮擊,手馘數人,與二將俱戰死。事聞,贈太僕卿。

而傅梅,崇禎中歷台州知府,解職歸。十五年冬,捐金佐知府吉孔嘉守城。城破殉難,贈太常少卿。

汪應蛟,字潛夫,婺源人。萬曆二年進士。授南京兵部主事,歷南京禮部郎中。給由入都,值吏部侍郎陸光祖與御史江東之等相訐,應蛟不直光祖,抗疏劾之,於政府多所譏切。

累遷山西按察使。治兵易州,陳礦使王虎貪恣狀,不報。朝鮮再用兵,移應蛟天津。及天津巡撫萬世德經略朝鮮,即擢應蛟右僉都御史代之,屢上兵食事宜,扼險列屯,軍聲甚振。稅使王朝死,帝將遣代。應蛟疏請止之,忤旨,切責。朝鮮事寧,移撫保定。歲旱蝗,振恤甚力。已,極言畿民困敝,請盡罷礦稅。會奸人柳勝秋等妄言括畿輔稅可得銀十有三萬,應蛟三疏力爭,然僅得減半而已。三十年春,帝命停礦稅,俄中止。應蛟復力爭,不納。

應蛟在天津,見葛沽、白塘諸田盡為汙萊,詢之土人,咸言斥鹵不可耕。應蛟念地無水則堿,得水則潤,若營作水田,當必有利。乃募民墾田五千畝,為水田者十之四,畝收至四五石,田利大興。及移保定,乃上疏曰:「天津屯兵四千,費餉六萬,俱斂諸民間。留兵則民告病,恤民則軍不給,計惟屯田可以足食。今荒土連封,蒿萊彌望,若開渠置堰,規以為田,可七千頃,頃得穀三百石。近鎮年例,可以兼資,非獨天津之餉足取給也。」因條畫墾田丁夫及稅額多寡以請,得旨允行。

已,請廣興水利。略言:「臣境內諸川,易水可以溉金臺,滹水可以溉恒山,溏水可以溉中山,滏水可以溉襄國,漳水來自鄴下,西門豹嘗用之,瀛海當諸河下流,視江南澤國不異。其他山下之泉,地中之水,所在而有,咸得引以溉田。請通渠築防,量發軍夫,一準南方水田之法行之。所部六府,可得田數萬頃,歲益穀千萬石,畿民從此饒給,無旱潦之患。即不幸漕河有梗,亦可改折於南,取糴於北。」工部尚書楊一魁亟稱其議,帝亦報許,後卒不能行。召為工部右侍郎,未上,予告去。已,進兵部左侍郎,以養親不出。親沒,竟不召。

光宗立,起南京戶部尚書,天啟元年改北部。東西方用兵,驟加賦數百萬。應蛟在道,馳疏言:「漢高帝稱蕭何之功曰:『鎮國家,撫百姓,給餉饋不絕,吾不如蕭何。』夫給饋餉而先以撫百姓,故能興漢滅楚,如運諸掌也。今國家多難,經費不支,勢不得緩催科,然弗愛養民力,而徒竭其脂膏,財殫氓窮,變亂必起,安得不預為計?」因列上愛養十八事,帝嘉納焉。熊廷弼建三方布置之策,需餉千二百萬,應蛟力阻之。廷議「紅丸」事,請置崔文昇、李可灼於法,而斥方從哲為編氓。

應蛟為人,亮直有守,視國如家。謹出納,杜虛耗,國計賴之。帝保母客氏求墓地踰制,應蛟持不予,遂見忤。會有言其老不任事者,力乞骸骨。詔加太子少保,馳傳歸。陛辭,疏陳聖學,引宋儒語,以宦官、宮妾為戒。久之,卒於家。應蛟學主誠敬,其出處辭受一軌於義。里居,謝絕塵事,常衣縕枲。

王紀,字惟理,芮城人。萬曆十七年進士。授池州推官。入為祠祭主事,歷儀制郎中。秉禮持正,時望蔚然。二十九年,帝將冊立東宮,數遷延不決,紀抗疏極論。其冬,禮成,擢光祿少卿,引疾去。

四十一年,自太常少卿擢右僉都御史,巡撫保定諸府。連歲水旱,紀設法救荒甚備。稅監張曄請征恩詔已蠲諸稅,紀兩疏力爭,曄竟取中旨行之。紀劾曄抗違詔書,沮格成命,皆不報。居四年,部內大治,遷戶部右侍郎,總督漕運兼巡撫鳳陽諸府。歲大凶,振救如畿輔。光宗立,召拜戶部尚書,督倉場。

天啟二年,代黃克纘為刑部尚書。時方會議「紅丸」事,紀偕侍郎楊東明署議,言:「方從哲知有貴妃,不知有君父。李可灼進藥駕崩,反慰以恩諭,賚之銀幣,國典安在?不逮可灼,無以服天下;不逮崔文昇,無以服可灼;不削奪從哲官階祿蔭,無以洩天地神人之憤。」議出,群情甚竦。

主事徐大化者,素無賴,日走魏忠賢門,構陷善類,又顯劾給事中周朝瑞、惠世揚。紀憤甚,劾大化溺職狀,因言:「大化誠為朝廷擊賊,則大臣中有交結權璫,誅鋤正士,如宋蔡京者,何不登彈文,而與正人日尋水火。」其言大臣,指大學士沈紘也。大化由此罷去,而紘及忠賢深憾之。御史楊維垣與大化有連,且素附紘,遂助紘詆紀,言紀所劾大臣無主名,請令指實。紀遂直攻紘,言:「紘與京,生不同時,而事實相類。其結納魏忠賢,與京之契合童貫同也;乞哀董羽宸,與京之懇款陳瓘同也;要盟死友邵輔忠、孫傑,與京之固結吳居厚同也;逐顧命元臣劉一燝、周嘉謨,與安置呂大防、蘇軾同也;斥逐言官江秉謙、熊德陽、侯震暘,與貶謫安常民、任伯雨同也。至於賄交婦寺,竊弄威權,中旨頻傳而上不悟,朝柄陰握而下不知,此又京迷國罔上,百世合符者。」客、魏聞之怒,為紘泣愬帝前。帝謂紀煩言,加譙責焉。

初,李維翰、熊廷弼、王化貞下吏,紀皆置之重辟。而與都御史、大理卿上廷弼、化貞爰書,微露兩人有可矜狀,而言不測特恩,非法官所敢輕議。有千總杜茂者,齎登萊巡撫陶郎先千金,行募兵,金盡而兵未募,不敢歸,返薊州僧舍,為邏者所獲,詞連佟卜年。卜年,遼陽人,舉進士,歷知南皮、河間,遷夔州同知,未行,經略廷弼薦為登萊監軍僉事。邏者搒掠。茂言嘗客於卜年河間署中三月,與言謀叛,因挾其二僕往通李永芳。行邊尚書張鶴鳴以聞。鶴鳴故與廷弼有隙,欲藉卜年以甚其罪。朝士皆知卜年冤,莫敢言。及鎮撫既成獄,移刑部,紀疑之,以問諸曹郎。員外郎顧大章曰:「茂既與二僕往來三千里,乃拷訊垂斃,終不知二僕姓名,其誣服何疑,卜年雖非間諜,然實佟養真族子,流三千里可也。」紀議從之。邏者又獲奸細劉一獻,忠賢疑劉一燝昆弟,欲立誅一讞與卜年,因一讞以株連一燝。紀皆執不可。紘遂劾紀護廷弼,緩卜年等獄,為二大罪。帝責紀陳狀,遂斥為民。以侍郎楊東明署部事,坐卜年流二千里。獄三上三卻。給事中成明樞、張鵬雲、沈惟炳,卜年同年生也,為發憤,摭他事連劾東明。卜年獲長系,瘐死,而東明遂引疾去。

紀既斥,大學士葉向高、何宗彥、史繼偕論救,皆不聽。後閹黨羅織善類,紀先卒,乃免。崇禎元年復官,贈少保,蔭一子,謚莊毅。

楊東明,字啟修,虞城人。官給事中。請定國本,出閣豫教,早朝勤政,酌宋應昌、李如松功罪之平。上《河南饑民圖》,薦寺丞鐘化民往振。掌吏科,協孫丕揚主大計。後以劾沈思孝,思孝與相詆,貶三官為陜西布政司照磨。里居二十六年。光宗立,起太常少卿。天啟中,累遷刑部右侍郎。既歸,遂卒。崇禎初,贈刑部尚書。

孫瑋,字純玉,渭南人。萬曆五年進士。授行人,擢兵科給事中。劾中官魏朝及東廠辦事官鄭如金罪,如金坐下詔獄。二人皆馮保心腹也。

初,張居正以刑部侍郎同安洪朝選輕遼王罪,銜之。後勞堪巡撫福建,希居正意,諷同安知縣金枝捃摭朝選事,堪飛章奏之。命未下,捕置之獄,絕其飯食三日,死,禁勿殮,屍腐獄中。堪尋召為左副都御史,未至京而居正卒。朝選子都察院檢校競訴冤闕下,堪復飛書抵馮保,削競籍,廷杖遣歸。至是,瑋白發其事,並及堪諸貪虐狀,堪免官。未幾,朝選妻訴冤,邱橓亦為訟,競復援胡賈、王宗載事,請與堪俱死,乃遣堪戍。當是時,廠衛承馮保餘威,濫受民訟;撫按訪察奸猾,多累無辜;有司斷獄,往往罪外加罰;帝好用立枷,重三百餘斤,犯者立死。瑋皆極陳其害。詔立枷如故,餘從瑋言。以母病,不候命擅歸,坐謫桃源主簿。久之,歷遷太常卿。

三十年,以右副都御史巡撫保定。朝鮮用兵,置軍天津,月餉六萬,悉派之民間。先任巡撫汪應蛟役軍大治水田,以所入充餉。瑋踵行之,田益墾,遂免加派,歲比不登,旱蝗、大水相繼,瑋多方振救,帝亦時出內帑佐之。所條荒政,率報允。畿輔礦使倍他省,礦已竭而搜鑿不已,至歲責民賠納。瑋累疏陳其害,且列天津稅使馬堂六大罪,皆不省。

就進兵部侍郎,召為右都御史,督倉場。進戶部尚書,督倉場如故。大僚多缺,命署戎政。已,又兼署兵部。瑋言:「陛下以纍纍三印悉畀之臣,豈真國無人耶?臣所知大僚則有呂坤、劉元震、汪應蛟,庶僚則有鄒元標、孟一脈、趙南星、姜士昌、劉九經,臺諫則有王德完、馮從吾輩,皆德立行修,足備任使。茍更閱數年,陛下即欲用之,不可得矣。」弗聽。

都御史自溫純去後,八年不置代。至四十年十二月,外計期迫,始命瑋以兵部尚書掌左都御史事。瑋素負時望,方欲振風紀,而是時朋黨勢成,言路大橫。會南畿巡按御史荊養喬與提學御史熊廷弼相訐,瑋議廷弼解職候勘。廷弼黨官應震、吳亮嗣輩遂連章攻瑋。瑋累疏乞休,帝皆慰留。無何,吏部以年例出兩御史于外,不關都察院。瑋以失職,求去益力,疏十餘上。明年七月稽首文華門,出郭候命。至十月,始予告歸。

天啟改元,起南京吏部尚書,改兵部,參贊機務。三年,召拜刑部尚書。囚繫眾,獄舍至不能容,瑋請近畿者就州縣分繫。內使王文進殺人,下司禮議罪,其餘黨付法司。瑋言一獄不可分兩地,請并文進下吏,不聽。其冬,以吏部尚書再掌左都御史事,累以老疾辭,不允。明年秋,疾篤,上疏曰:「今者天災迭見,民不聊生。內而城社可憂,外而牖戶未固。法紀凌遲,人心瓦解。陛下欲圖治平,莫如固結人心;欲固結人心,莫如登用善類。舊輔臣劉一燝,憲臣鄒元標,尚書周嘉謨、王紀、孫慎行、盛以弘、鐘羽正等,侍郎曹于汴,詞臣文震孟,科臣侯震暘,臺臣江秉謙,寺臣滿朝薦,部臣徐大相,並老成蹇諤,跧伏草野,良可歎惜。倘蒙簡擢,必能昭德塞違,為陛下收拾人心。尤望寡欲以保聖躬,勤學以進主德,優容以廣言路,明斷以攬大權。臣遘疾危篤,報主無期,敢竭微忱,用當屍諫。」遂卒,贈太子太保。魏忠賢用事,陜西巡撫喬應甲劾瑋素黨李三才、趙南星,不當叨冒恩恤。詔追誥命,奪其廕。崇禎初,復之。後謚莊毅。

鐘羽正,字叔濂,益都人。萬曆八年進士。除滑縣知縣。甫弱冠,多惠政,徵授禮科給事中。疏言朝講不宜輟,張鯨不宜赦,不報。

遷工科左給事中,出視宣府邊務。哈剌慎老把都諸部挾增市賞二十七萬有奇,羽正建議裁之。與參政王象乾讋以利害,莫敢動。兵部左侍郎許守謙先撫宣府,以賄聞,羽正劾去之。又劾罷副總兵張充實等,而悉置諸侵盜軍資者於理。

還為吏科都給事中。劾禮部侍郎韓世能,薊遼總督蹇達,大理少卿楊四知、洪聲遠不職,四知、聲遠坐貶謫。時當朝覲,請禁饋遺,言:「臣罪莫大於貪。然使內臣貪而外臣不應,外臣貪而內臣不援,則尚相顧畏莫敢肆。今內以外為府藏,外以內為窟穴,交通賂遺,比周為奸,欲仕路清、世運泰,不可得也。」帝善其言,敕所司禁之。且命閣部大臣公事議於朝房,毋私邸接賓客。吏部推孟一脈應天府丞,蔡時鼎江西提學,副以呂興周、馬猶龍。帝惡一脈、時鼎嘗建言,皆用副者。羽正率同列上言:「陛下不用一脈、時鼎,中外謂建白之臣,不惟一時見斥,而且復進無階,銷忠直之氣,結諫諍之舌,非國家福。」疏入,忤旨,奪俸有差。

二十年正月,偕同官李獻可等請皇長子出閣豫教。帝怒,謫獻可官。羽正以己實主議,請與同謫,竟斥為民。杜門讀書,士大夫往來其地,率辭不見。林居幾三十年。光宗立,起太僕少卿。未至,進本寺卿。

天啟二年,吏部將用為左副都御史,羽正辭曰:「馮公從吾僉院已久,吾後入先之,是長競也。西臺何地,可以是風有位乎?」乃受僉都御史而讓從吾為副。甫入署,即言:「方從哲進藥議謚,封后移宮,無謀鮮斷,似佞似欺,宜免其官秩,使為法受過。沈紘結內援,招權賄,宜遄決其去。」群小多不悅。熊廷弼、王化貞之獄,眾議紛呶。羽正言:「向者開原、鐵嶺之罪不明,致失遼陽;遼陽之罪不明,致失廣寧。朝廷疆土,堪幾番敗壞!」由是二人皆坐大辟。會朱童蒙以講學擊鄒元標及從吾,羽正言書院之設,實為京師首善勸,不當議禁,因自劾乞休。頃之,代從吾為左副都御史,俄改戶部右侍郎,督倉場。

明年春,拜工部尚書。故事,奄人冬衣隔歲一給。是夏六月,群奄千餘人請預給,蜂擁入署,碎公座,毆掾吏,肆罵而去。蓋忌羽正者嗾奄使發難也。羽正疏聞,因求罷。詔司禮太監杖謫群奄,而諭羽正出視事。羽正求去益堅,因言:「今帑藏殫虛,九邊壯士日夜荷戈寢甲,弗獲一飽;慶陵工卒負重乘高,暴炎風赤日中,求傭錢不得;而獨內官請乞,朝至夕從。此輩聞之,其誰不含憤?臣奉職不稱,義當罷黜。」復三疏自引歸。

踰年,逆黨霍維華追理三案,言羽正委身門戶,遂削奪。崇禎初,復官。久之,卒。贈太子太保。

陳道亨,字孟起,新建人。萬曆十四年進士。除刑部主事,歷南京吏部郎中。同里鄧以贊、衷貞吉亦官南都,人號「江右三清」。遭母喪,家毀于火,僦屋以居。窮冬無幃,妻御葛裳,與子拾遺薪爇以禦寒,或有贈遺,拒弗受。由湖廣參政遷山東按察使、右布政使,轉福建為左,所至不私一錢。以右副都御史提督操江。光宗立,進工部右侍郎,總督河道。

天啟二年,妖賊徐鴻儒作亂。道亨守濟寧,扼諸要害,以衛漕舟。事平,增俸賜銀幣。尋拜南京兵部尚書,參贊機務。楊漣等群擊魏忠賢,被譙責。道亨憤,偕九卿上言:「高皇帝定令,內臣止供掃除,不得典兵預政。陛下徒念忠賢微勞,舉魁柄授之,恣所欲為,舉朝忠諫皆不納。何重視宦豎輕天下士大夫至此?」疏入,不納。道亨遂連疏求去,詔許乘傳歸。踰年卒。

道亨貞亮有守。自參政至尚書,不以家累自隨,一蒼頭執爨而已。崇禎初,贈太子少保,謚清襄。

子弘緒,字士業。為晉州知州,以文名。

贊曰:光、熹之際,朝廷多故,又承神宗頹廢之餘,政體怠弛,六曹罔修厥職。周嘉謨、張問達諸人,懇懇奉公,《詩》所稱「不懈于位」者,蓋庶幾焉。汪應蛟持國計,謹出納,水田之議,鑿鑿可見施行。孫瑋請登用善類,鐘羽正請禁饋遺,韙哉,救時之良規也。


\end{pinyinscope}