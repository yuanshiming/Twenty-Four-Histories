\article{列傳第二百六 廣西土司二}

\begin{pinyinscope}
△太平思明思恩鎮安田州恩城上隆都康

太平,漢屬交阯,號麗江。唐為羈縻州,隸邕州都督府。宋平嶺南,於左、右二江溪峒立五寨。其一曰太平,與古萬、遷隆、永平、橫山四寨各領州、縣、峒,屬邕州建武軍節度。元仍為五寨。後廢,乃置太平路於麗江。

洪武元年,征南將軍廖永忠下廣西,左江太平土官黃英衍等遣使齎印詣平章楊璟降。璟還自廣海,帝問黃、岑二氏所轄情形。璟言:「蠻僚頑獷,散則為民,聚則為盜,難以文治,當臨之以兵,彼始畏服。」帝曰:「蠻瑤性習雖殊,然其好生惡死之心,未嘗不同。若撫之以安靖,待之以誠,諭之以理,彼豈有不從化者哉。」遣中書照磨蘭以權齎詔,往諭左、右兩江溪峒官民曰:「朕惟武功以定天下,文德以化遠人,此古先哲王威德並施,遐邇咸服者也。眷茲兩江,地邊南徼,風俗質樸。自唐、宋以來,黃、岑二氏代居其間,世亂則保境土,世治則修職貢,良由其審時知幾,故能若此。頃者,朕命將南征,八閩克靖,兩廣平定。爾等不煩師旅,奉印來歸,嚮慕之誠,良足嘉尚。今特遣使往諭,爾其克慎乃心,益懋厥職,宣布朕意,以安居民。」以權至廣西衛,鎮撫彭宗、萬戶劉維善以兵護送。將抵兩江,適來賓洞蠻寇掠楊家寨居民。以權謂彭宗等曰:「奉詔遠來,欲以安民,今見賊不擊,何以庇民?」乃督宗等擊之。賊敗走,遂安輯其地,兩江之民由是懾服。二年,黃英衍遣使奉表貢馬,乃改為太平府。以英衍為知府,世襲。

宣德元年,崇善縣土知縣趙暹謀廣地界,遂招納亡叛,攻左州,執故土官,奪其印,殺其母,大肆擄掠,占據村洞四十餘所。造火器,建旗幟,僭稱王,署偽官,流劫州縣。事聞,帝命總兵官顧興祖會廣西三司剿捕。興祖等招之,不服,遣千戶胡廣率兵進。暹扼寨拒守,廣進圍之,紿出所奪各州印,撫諭脅從官民,使復職業。暹計窮,從間道遁。伏兵邀擊,及其黨皆就擒。時左州土官黃榮亦奏:「蠻人李圓英劫掠居民,偽稱官爵,乞發兵剿捕。」帝謂兵部曰:「蠻民愚獷,或挾私仇忿爭戕殺,來告者必欲深致其罪,未可遽信。其令鎮遠侯並廣西三司勘實,先遣人招撫,如叛逆果彰,發兵未晚也。」二年斬南寧百戶許善。初,善知趙暹謀逆,與之交通。及總兵官遣善追暹,又受暹馬十匹、銀百兩,故延緩之,冀幸免。事覺,下御史,鞫問得實,斬之,餘黨皆伏誅。

太平領州縣以十數。明初,皆以世職授土官,而設流官佐之。

太平州,舊名瓠陽,為西原、農峒地。唐為波州,宋隸太平寨,元隸太平路。洪武元年,土官李以忠歸附,授世襲知州,設流官吏目佐之。

鎮遠州,舊名古隴,宋置,隸邕州。元隸太平路。洪武初,土官趙勝昌歸附,授世襲知州,設流官吏目佐之。

茗盈州,宋置,隸邕州。元隸太平路。洪武初,土官李鐵釘歸附,授世襲知州,設流官吏目佐之。

安平州,舊名安山,亦西原、農峒地。唐置波州,宋析為安平州,元隸太平路。洪武初,土官李郭佑歸附,授世襲知州,設流官吏目佐之。

思同州,舊名永寧,為西原地,唐置,隸邕州。宋隸太平寨。洪武元年,土官黃克嗣歸附,授世襲知州,設流官吏目佐之,屬太平府。萬曆二十八年,省入永康州。

養利州,元屬太平路。洪武初,土官趙日泰歸附,授知州,以次傳襲。宣德間,稍侵其鄰境,肆殺掠。萬曆三年討平之,改流官。

萬承州,舊名萬陽。唐置萬承、萬形二州。宋省萬形,隸太平寨。元隸太平路。洪武初,土官許郭安歸附,授世襲知州,設流官吏目佐之。永樂間,郭安從征交阯,死於軍,子永誠襲。

全茗州,舊名連岡,為西原地,宋置,隸邕州。元隸太平路。洪武初,土官李添慶歸附,授世襲知州,設流官吏目佐之。

結安州,舊名營周,亦西原、農峒地。宋置結安峒,隸太平寨。元改州,屬太平路。洪武元年,土官張仕榮歸附,授世襲知州,設流官吏目佐之。

龍英州,舊名英山,宋為峒。元改州,屬太平路。洪武元年,土官李世賢歸附,授世襲知州,割上懷地益其境,設流官吏目佐之。

結倫州,舊名邦兜,亦西原、農峒地。宋置結安峒,隸太平寨。元改州,屬太平路。洪武二年,峒長馮萬傑歸附,授世襲知州,設流官吏目佐之。

都結州,元屬太平路,土官農姓。洪武初內附,授世襲知州,設流官吏目佐之。

上、下凍州,舊名凍江。宋置凍州。元分上、下凍二州,尋合為一,屬龍州萬戶府。洪武元年,土官趙貼從歸附,授世襲知州,設流官吏目佐之,屬太平府。貼從死,子福瑀襲。永樂四年從征交阯,死於軍。

思城州,亦西原、農峒地,唐置州。宋分為上、下思城二州,隸太平寨。元至正間,並為一,屬太平路。洪武元年,土官趙雄傑歸附,授世襲知州,設流官吏目佐之。

永康州,宋置縣,隸遷隆寨。元隸太平路,土官楊姓。成化八年,其裔孫楊雄傑糾合峒賊二千餘人,入宣化縣劫掠,且偽署官職。總兵官趙輔捕誅之,因改流官。萬曆二十八年升為州。

左州,舊名左陽,唐置,隸邕州。宋隸古萬寨。元屬太平路。洪武初,土官黃勝爵歸附,授世襲知州。再傳,子孫爭襲,相仇殺。成化十三年改流官。

羅陽縣,舊名福利,陀陵縣,舊名駱陀,皆宋置。元隸太平。洪武初,土官黃宣、黃富歸附,並授世襲知縣,設流官典史佐之。

思明,唐置州,隸邕州。宋隸太平寨。元改思明路。洪武初,改為府。二年,土官黃忽都遣使貢馬及方物。詔以忽都為思明府知府,世襲。十五年,忽都復遣其弟祿政奉表來貢,詔賜鈔錠。二十三年,忽都子黃廣平遣思州知州黃志銘率屬部,偕十五州土官李圓泰等來朝。明年,廣平以服闋,遣知州黃忠奉表貢馬及方物。詔廣平襲職,賜冠帶襲衣,及文綺十匹、鈔百錠。二十五年,憑祥洞巡檢高祥奏,思明州知州門三貴謀殺思明府知府黃廣平,廣平覺而殺之,乃以病死聞於朝,所言不實。詔逮廣平鞫之。既至,帝謂刑部曰:「蠻寇相殺,性習固然,獨廣平不以實言,故繩以法。今姑宥之,使其改過。」命給道里費遣還,是後朝貢如例。

二十九年,土官黃廣成遣使入貢,因奏言:「本府自故元改思明路軍民總管所,轄左江一路州縣峒寨,東至上思州,南至銅柱。元兵征交阯,去銅柱百里,設永平寨軍民萬戶府,置兵戍守,命交人供其軍餉。元季擾亂,交人以兵攻破永平寨,遂越銅柱二百餘里,侵奪思明屬地丘溫、如嶅、慶遠、淵、脫等五縣,逼民附之,以是五縣歲賦皆土官代輸。前者本府失理於朝,遂致交人侵迫益甚。及告禮部,任尚書立站於洞登,洞登實思明地,而交阯乃稱屬銅柱界。臣嘗具奏,蒙朝廷遣刑部尚書楊靖核其事,《建武志》尚可考。乞敕安南,俾還舊封,庶疆域復正,歲賦不虛。」帝令戶部錄所奏,遣行人陳誠、呂讓往諭安南。三十年,誠、讓至安南,諭其王陳日焜,令還思明地。議論往復,久而不決。以譯者言不達意,復為書曉之。安南終辨論不已,出黃金二錠、白金西錠及沉檀等香以賄,誠卻之。安南復咨戶部,無還地意。廷臣議其抗命當誅,帝曰:「蠻人怙頑不悛,終必取禍,姑待之。」

永樂二年,憑祥巡檢李升言,其地瀕安南,百姓樂業,生齒日繁,請改為縣,以便撫輯,從之。以升為知縣,設流官典史一員。三年,昇以新設縣治來朝,貢馬及方物謝恩。廣成奏安南侵奪其祿州、西平州永平寨地,請遣使諭還,從之。九年,免思明稅糧,以廣成言去秋雨水傷稼也。

宣德元年,思明賀天壽節奉表踰期,禮部請罪之。帝以遠蠻既至,毋問。土官知府黃岡奏憑祥歲凶民饑,命發龍州官倉糧振之。正統七年,岡遺使入貢。九年,貢解毒藥味,賜鈔錦。

景泰三年,岡致仕,以子鈞襲。岡庶兄都指揮矰欲殺鈞,代以己子。矰守備潯州,託言徵兵思明府,令其子糾眾結營於府三十里外,馳至府,襲殺岡一家,支解岡及鈞,甕葬後圃,仍歸原寨。明日,乃入城,詐發哀,遣人報矰捕賊,以掩其迹。方殺岡時,岡僕福童得免,走憲司訴其事,且以徵兵檄為證。郡人亦言殺岡一家者,矰父子也。副總兵武毅以聞,將逮治之。矰自度禍及,及謀迎合朝廷意,遣千戶袁洪奏永固國本事,請易儲。奏入,帝曰:「此天下國家重事,多官其會議以聞。」矰為此舉,眾皆驚愕,謂必有受其賂而教之者,或疑侍郎江淵云。事成,矰得釋罪,且進秩。英宗復辟,矰聞自殺。帝命發棺戮其屍,其子震亦為都督韓雍捕誅。

成化十八年,土知府黃道奏所轄思明州土官孫黃義為族人黃紹所殺,乞發兵捕剿。帝命兩廣守臣區處以聞。

弘治十年,況村賊黃紹侵占思明、上石、下石三州,復謀殺知府黃道父子。道妻趙氏累訴於朝,且謂屢經委官勘問,俱被賂免,乞發兵誅之。十一年,紹集眾數千人焚劫鄉村,據三州,屢撫不下,總鎮請發兵捕剿。嘉靖四十一年,以剿平瑤、僮功,命土官知州男黃承祖暫襲本職。隆慶四年,忠州土官黃賢相等據南寧府屬四都地作亂,永康典史李材計誘其黨,縛賢相以降。萬曆十六年,思明州土官黃拱聖謀奪襲,殺其母兄拱極等五人。而思明知府黃承祖乘亂掠村寨,為之援。按臣請以拱聖及諸凶正法,思明州改屬流府,革承祖冠帶,立功自贖,而追其所掠;更令族人黃恩護拱極妻許氏撫遺孤世延,待其長官之。

三十三年,總督戴耀奏:「思明叛目已擒,土官黃應雷縱僕起釁,棄印而逃,斷難復官。黃應宿爭地,殺戮六哨成仇,且係義子,不應襲職。黃應聘係承祖幼子,人心推戴,似應承襲知府,以存黃氏宗祀。但年甫七歲,暫令流官同知署府事,待至十五歲,交印接管。應雷既廢,不宜同城,應降為土舍,其後永襲土舍,給田養贍,制其出入。應宿仍管故業,俱屬思明府節制。於府治設教授一員,量給廩生六名,其寄附太平府者,悉歸本學,嗣後續增其祭祀廩餼之用,則地方可安,文教可興。」詔悉從之。

崇禎十一年,總督張鏡心疏報土官殺職官思明州黃日章、黃德志等,鼓眾叛逆。帝令速擒首惡以靖地方。論者以黃矰神奸,身逭大盩,世濟其兇,傳及四世,猶併思明州而有之,王綱隳矣。然骨肉相屠,至是四見,蓋天道云。

思明州,東抵思明府,西抵交阯界,南抵西平州,北抵龍英州。土官黃姓,與思明府同族。洪武初,黃君壽歸附,授世襲知州,屬思明府,後為黃矰所并。萬曆十六年,黃拱聖之亂,改屬太平。

上石西州,宋屬永平寨,元屬思明路。明初屬思明府,至萬曆三十八年改屬太平府。州更土官趙氏、何氏、黃氏凡三姓,皆絕,始改流官。下石西州,宋分石西州置,元屬思明路。洪武二年,土官閉賢歸附。授世襲知州,設流官吏目佐之。

忠州,宋置,隸邕州。元屬思明路。洪武初,土官黃威慶率子中謹歸附,授威慶江州知州,中謹忠州知州,皆世襲,設流官同知吏目佐之。其鄰地有四峒者,界於南寧、思明、忠江之間,思明、忠州屢肆侵奪。副使翁萬達議改峒名四都,隸之南寧,地方稍定。隆慶三年冬,思明府土官黃承祖奏取四都地,忠州土官黃賢相爭之,遂擅立總管諸名目,分兵數千戍守,因縱令剽掠,為禍甚烈。僉事譚惟鼎調永康典史李材以計擒賢相,斃之於獄。議改流官,不果,遂改隸州於南寧,仍以州印予賢相子有瀚,俾襲職。

憑祥,宋為憑祥洞,屬永平寨,元屬思明路。洪武十八年,土蠻李昇歸附。置憑祥鎮,授昇巡檢,屬思明府。永樂二年置縣,以昇為知縣。成化八年升為州,以昇孫廣寧為知州,直隸布政司。廣寧有十子,廣寧死,諸子爭立不決,凡三四年,乃以孫珠襲知州職。嘉靖十年,珠死,族弟珍、玨爭立,珍挈印走況村,玨攝州事。十四年,州目李清、趙琪等謀納珍,許思明府黃朝以州屬之。朝遂以兵納珍於憑祥,玨奔罄柳。既珍悔屬思明,與朝隙,朝乃以外婦所生子時芳,詭云廣寧孫,以兵千人納之。時珍淫縱,為部民所怨,於是廣寧季子寰以尊屬謀廢立。十七年,寰遂殺珍而附於安南,莫登庸藉為嚮導。總督蔡經屬副使翁萬達擒之,論死。於是玨與時芳復爭立,時芳倚思明勢,州民皆右之。萬達黜玨而論時芳死,更立李佛嗣珍為知州,憑祥遂定。

思恩,漢屬交阯。唐為思恩州,屬邕,乃澄州止戈縣地。宋開寶間,廢澄州,以止戈、賀水、無虞三縣省入上林。治平間,以上林之止戈入武緣,隸邕。無屬田州路。歷代羈縻而已。

明洪武二十二年,田州府知府岑堅遣其子思恩州知州永昌貢方物。二十八年,歸德州土官黃碧言,思恩州知州岑永昌既匿五縣民,不供賦稅,仍用故元印章。帝以不奉朝命,命左都督楊文相機討之,既以荒遠不問。永樂初,改屬布政司,時居民僅八百戶。永昌死,子瑛襲。宣德二年,瑛遣弟璥貢馬。正統三年進瑛職為知府,仍掌州事。瑛有謀略,善治兵,從征蠻寇,屢有功,故有是命。因與知府岑紹交惡,各具奏,下總兵官及三司議。於是安遠侯柳溥等請升思恩為府,俾瑛、紹各守疆土,以杜侵爭,從之。六年,瑛受屬挾詐事覺,帝以土蠻宥不問,令法司移文戒之。瑛以府治僻隘,橋利堡正當瑤寇出沒之所,且有城垣公廨,乞徙置,許之。以思恩府為思恩軍民府。十二年設儒學,置教授一員,訓導四員,俱從瑛請也。

景泰四年,總兵官陳旺奏:「思恩土兵調赴桂林哨守者,離本府遼遠,不便耕種,稅糧宜暫免。」從之。六月,以瑛親率本部狼兵韋陳威等赴城操練,協助軍威,敕授奉議大夫,賜彩緞,韋陳威等俱給冠帶。五年從瑛請建廟學,造祭祀樂器。又以瑛征剿瑤寇功,免土軍今年應輸田糧之半,進瑛從二品散官。瑛屢領兵隨征,以子鑌代為知府。鑌招集無賴,肆為不法。瑛舉發其事,請於總兵,回府治之。鑌聞其父將至,自縊死。事聞,嘉其能割愛效忠,降敕慰諭。又以柳溥奏,免思恩調用土軍千五百人、秋糧二千三百餘石。

天順元年,戶部奏:「思恩存留廣西操練軍一千五百人,有誤種田納糧。乞分為三班,留五百人操練,免其糧七百七十餘石。放回千人耕種,徵其糧千五百四十餘石,俟寧靖日放回全徵。」從之。三年,鎮守中官朱祥奏請量遷瑛都司軍職。帝以瑛歷練老成,累有軍功,改授都指揮同知,仍聽總兵官鎮守調用,以其子鐩為知府。

成化元年遣兵科給事中王秉彞齎敕獎諭瑛父子,并賜銀幣。二年命給瑛父母妻誥命,從總兵趙輔請也。十四年,瑛卒。瑛自襲父職,頻年領兵於外,多所斬獲。歷升知府、參政、都指揮使。年且八十,尚在軍中。既卒,鐩以誥請,帝念其勞,特賜之。十六年,田州府土目黃明作亂,知府岑溥避入思恩,鐩會鎮守等官討平之。巡撫朱英請獎鐩功。鐩死,子浚襲。

弘治十二年,田州土官岑溥為子猇所殺,猇亦死。次子猛幼,頭目黃驥、李蠻構難,督府命濬調眾護猛。驥厚賂濬,并獻其女,且約分地與浚。浚以兵屬驥,送猛至田州。不得入,猛遂久留浚所。及總鎮諸官攝浚,乃出猛襲知府。浚從索故分地,不得,怒,約泗城、東蘭二州攻劫田州,殺掠萬計,城郭為墟。浚兵二萬據舊田州,劫龍州印,納故知府趙源妻岑氏。及總兵官詣田州勘治,黃驥懼,匿浚所。先是,濬築石城於丹良莊,屯兵千餘人,截江道以括商利,官命毀之,不聽。會官軍自田州還,乘便毀其城。浚兵來拒,殺官軍二十餘人。官軍敗之,俘其目兵九人。總鎮及巡按等官請治浚罪,而參政武清納浚賂,曲護之。

浚從弟業少從中官京師,仕為大理寺副三司。總鎮請敕業往諭,兵部以濬稔惡,非業所能諭責,宜敕鎮巡召浚至軍門,諭以朝廷威德,罪其首惡,反侵地,納所劫印,并官私財物,乃可赦。總督鄧廷瓚奏:「濬屢撫不服,請調官軍土兵分哨逐捕按問。如集兵拒敵,相機剿殺,并將田州土官岑猛一并區處,以靖邊疆。」十六年,總督潘蕃奏:「濬僭叛,當用兵誅剿。今浚從弟業以山東布政司參議在內閣制敕房辦事,禁密之地,恐有泄漏。」吏部擬改調,而業亦奏乞養去。十七年,濬掠上林、武緣等縣,死者不可勝計。又攻破田州,猛僅以身免,掠其家屬五十人。總鎮以聞,兵部請調三廣兵剿之。十八年,總督潘蕃、太監韋經、總兵毛銳調集兩廣、湖廣官軍土兵十萬八千餘人,分六哨。副總兵毛倫、右參政王璘由慶遠,右參將王震、左參將王臣及湖廣都指揮官纓由柳州,左參將楊玉、僉事丁隆由武緣,都指揮金堂、副使姜綰由上林,都指揮何清、參議詹璽由丹良,都指揮李銘、泗城州土舍岑接由工堯,各取道共抵巢寨。賊分兵阻險拒敵,官軍奮勇直前,援崖而進。濬勢蹙,遁入舊城,諸軍圍攻之。濬死,城中人獻其首,思恩遂平。前後斬捕四千七百九十級,俘男女八百人,得思恩府印二,向武州印一。自進兵及班師僅踰月。捷聞,帝以蕃等有功,璽書勞之。兵部議濬既伏誅,不宜再錄其後,改設流官,擇其可者。以雲南知府張鳳陞廣西右參政,掌思恩府事,賜敕。

正德七年增設鳳化縣治。時初設流官,諸蠻未服,相繼作亂。嘉靖四年,都御史盛應期遣官軍平之。六年,土目王受與田州盧蘇謀煽亂,勢復熾。新建伯王守仁受命至,一意招撫,而檄受等破八寨賊,因列思恩地為九土巡檢司,管以頭目,授王受白山司巡檢,得比於世官。又以思恩舊治瘴霧昏塞,宜更之爽塏。於是擇地荒田建新郡,割武緣止戈二里益之;又議割上林三里,而移鳳化縣治於其處。蓋寓犬牙相錯之意。巡撫林富謂遷郡及割止戈里應如守仁議,至以三里當設衛,而并鳳化縣裁之,遂令府治益孤。其後九司頭目日恣,所轄蠻民不堪,知府陳璜曲加綏戢。目把劉觀、盧回以復土為名,鼓眾作亂。副使翁萬達因有事安南,計擒盧回殺之,招回從亂者三十餘人。最後東蘭岑瑄詐稱岑浚子起雲,謀復土官,為九司頭目所縛。萬曆七年,督撫吳文華謂九司日以驕黠,編氓甚少,緩急難恃,奏割南寧武緣縣屬思恩,自是思恩稱巨鎮云。

思恩府土巡檢九司,皆嘉靖七年設,曰興隆,曰那馬,曰白山,曰定羅,曰舊城,曰下旺,曰安定,曰都陽,曰古零。

鎮安,宋時於鎮安峒建右江軍民宣撫司,元改鎮安路。明洪武元年,鎮安歸附。以舊治僻遠,移建廢凍州,改為府。授土官岑添保知府,朝貢如例。二十七年,添保上言:「往者征南將軍傅友德令郡民歲輸米三千石,運雲南普安衛。鎮安僻處溪洞,南接交阯,孤立一方,且無所屬。州縣人民鮮少,舟車不通,陸行二十五日始到普安。道遠而險,一夫負米三斗,給食餘所存無幾,往往以耕牛及他物至其地易米輸納。而普安荒遠,米不易得,民甚病之。又歲輸本衛米四百石,尤極艱難。舊以白金一兩,折納一石。今願依前例,以蘇民困。」從之。

永樂中,向武知州黃世鐵侵奪鎮安高寨等地,朝廷遣兵討平之,以其地屬鎮安。成化八年,知府岑永壽姪宗紹糾集土兵,攻破府治,殺傷嫡母,流劫鄉村,有司撫諭不服,都指揮岑瑛擒斬之。嘉靖十四年,田州盧蘇作亂,糾歸順州土官岑瓛攻毀鎮安府,目兵遇害者以萬計。按臣曾守約以聞,帝命守臣治之。時蘇倡亂,田州無主,鎮安府土官男岑真寶以兵納岑邦佐於田州。歸順州岑瓛,蘇婿也,及向武州黃仲金皆與真寶隙,乘真寶入田州,蘇遣瓛及仲金襲破鎮安。真寶聞亂,走還。蘇會目兵追圍之武陵寨,瓛等遂發真寶父母墓,焚其骸,分兵占據諸洞寨。真寶訴之軍門,督諭瓛等不退。久之乃解,官軍歸真寶,於是瓛與真寶互相訐。巡按御史言,土蠻自相仇,非有所侵犯,從末減。於是蘇、瓛、仲金各降罰有差,真寶亦革冠帶,許立功自贖。二十二年以瑤、僮作亂,防禦需人,免真寶諸土官來朝。

鎮安所屬有上映洞、湖潤寨。巡檢皆土人,世官。

田州,古百粵地。漢屬交阯郡。唐隸邕州都督府。宋始置田州,屬邕州橫山寨。元改置田州路軍民總管府。明興,改田州府,省來安府入焉。後改田州,領縣一,曰上林。

洪武元年,大兵下廣西,右江田州府土官岑伯顏遣使齎印詣平章楊璟降。二年,伯顏遣使奉表貢馬及方物,詔以伯顏為田州知府,世襲,自是朝貢如制。六年,田州溪峒蠻賊竊發,伯顏討平之。伯顏請振安州、順龍州、侯州、陽縣、羅博州、龍威寨人民,詔有司各給牛米,仍蠲其稅二年。十六年,伯顏死,子堅襲。十七年,都指揮使耿良奏:「田州知府岑堅、泗州知州岑善忠率其土兵,討捕瑤寇,多樹功績。臣欲令選取壯丁各五千人,立二衛,以善忠之子振,堅之子永通為千戶,統眾守禦,且耕且戰,此古人以蠻攻蠻之術也。」詔行其言。二十年,堅遣子思恩知州永昌朝貢,如例給賜。

永樂元年,堅死,子永通襲。永通,上隆州知州也,州以瓊代,而己襲父職。正統八年,賜知府岑紹誥命,并封贈其父母妻。

天順元年,田州頭目呂趙偽稱敵國大將軍,張旗幟,鳴鉦鼓,率眾劫掠南丹州,又據向武州。武進伯朱瑛以聞,兵部請命瑛及土官岑瑛剿捕。三年,巡撫葉盛奏:「田州叛目呂趙勢愈獗,殺知府岑鑑,占據地方,偽稱太平王,圖謀岑氏宗族,冒襲知府職事。」帝命總兵速討。四年,巡按御史吳禎奏:「奉敕剿捕反賊呂趙,選調官軍土兵,攻破功饒、婪鳳二關,直搗府城。呂趙攜妻子,挾知州岑鐸等宵遁。官軍追至雲南富州,奪回鐸等及其子若婿。斬首四十九級,賊眾悉降。趙以數騎走鎮安府,追及之,斬趙及其子四人,從賊十八人,獲其妻孥及偽太平王木印、無敵將軍銅印,並鳳旗盔甲等物。復委知府岑鏞仍掌府事,撫安人民。」田州平,帝遣使齎敕獎諭禎等,並敕鏞謹守法度,保全宗族。

成化元年,遣兵科給事中王秉彞齎敕諭鏞,并賜銀幣,以兵部言其所部土官狼兵,屢調剿有勞,且有事於大藤峽也。二年,總兵官趙輔奏鏞從征有功,請給誥命,旌其父母并妻,從之。五年,復以輔言,予鏞官誥。十六年,田州頭目黃明聚眾為亂,知府岑溥走避思恩。總督朱英調參將馬義率軍捕明,明敗走,為恩城知州岑欽所執,並族屬誅之。已,溥復與欽交惡。欽攻奪田州,逐溥,殺五十餘家。時泗城州岑應方恃兵強,復黨欽,殺擄人民二萬六千餘,與欽分割田州而據其地。

弘治三年,總制遣官護溥之子猇入田,為欽所遏,居潯州。按察使陶魯率官軍次南寧,欽拒敵,敗走。而應復援之入城,陳兵以備。巡撫秦紘請合貴州、湖廣及兩廣兵剿之,欽勢蹙,乞兵於應,遂匿應所,總鎮官因檄應捕欽。欽從應飲,殺應父子於坐,收其兵以拒官軍。已而應弟岑接佯以兵送欽至田州界,亦殺其父子以報。事聞,廷議仍命溥還田州。九年,總督鄧廷瓚言溥前以罪革職,比隨征有功,乞復其冠帶,領土兵赴梧州聽調,從之。十二年,溥為子猇所弒,猇亦自殺。次子猛方四歲,溥母岑氏及頭目黃驥護之,赴制府告襲。歸至南寧,頭目李蠻來迎。驥慮蠻奪己權,殺其使。蠻率兵至舊田州,驥懼,誣蠻將為變,乞以兵納,乃調思恩岑濬率兵衛猛。浚受驥賂,納其女,挾猛,約分其六甲地。比至田州,蠻拒不納,驥復以猛奔思恩,幽之。事覺,廷瓚檄副總兵歐磐等攝濬,久乃出猛,置於會城。得奏,命猛襲知府。驥、濬怒其事之不由己出也,要泗城岑接、東蘭韋祖鋐各起兵攻蠻。接兵二萬先入田州,殺掠男女八百餘人,驅之溺水死者無算,括府庫,放兵大掠,城郭為墟。浚兵二萬攻舊田州,據之,殺掠男女五千三百餘人,蠻逃去。副總兵歐磐、參政武清等詣田州府勘治,遣兵送猛還府。驥懼罪,匿浚家,有司請治浚罪。

初,蠻之迎猛也,無他念,及猛在外,蠻守土以待其歸。驥爭權首亂,浚、接、祖鋐黨惡,以致茲變。清受濬賂,曲右之,且誣蠻占據府治,阻兵弄權,事竟不直。於是廷瓚言思恩岑浚罪惡,正在逐捕,而田州岑猛亦宜乘此區畫,降府為州,毋基異日尾大之患,從之。十八年,廷議以思、田既平,宜設流官;岑猛世濟兇惡,致陷府治,宜降授千戶,而遴選才望者假以方面職銜,守田州,仍賜敕以重其權。帝然之,於是以平樂知府謝湖為右參政,掌府事。

時岑猛已降福建平海衛千戶,遷延不行。及湖至,復陳兵自衛,令祖母岑氏奏乞於廣西極邊率部下立功,以便祭養,詔總鎮官詳議以聞。總督陳金奏:「猛據舊巢,要求府佐,不赴平海衛。參政謝湖不即赴任,為猛所拒,納饋遺而徇其要求,宜逮間。」時猛遣人重賂劉瑾,得旨,留猛而褫湖,并及前撫潘蕃、劉大夏,猛竟得以同知攝府事。猛撫輯遺民,兵復振,稍復侵旁郡自廣。嘗言督撫有調發,願立功,冀復舊職。會江西盜起,都御史陳金檄猛從征,猛所至剽掠。然以賊平故論功,遷指揮同知。非猛初意,頗犯望。

正德十五年,猛奏:「田州土兵每徵調,輒許戶留一二丁耕種,以供常稅。其久勞於外者,乞量振給,免其輸稅。」從之。

嘉靖二年,猛率兵攻泗城,拔六寨,遂克州治。岑接告急於軍門,言猛無故興兵攻寨。猛言接非岑氏後,據其祖業,欲得所侵地。時方有上思州之役,徵兵皆不至,總督張嵿以狀聞。四年,提督盛應期、巡按謝汝儀議大征猛,條徵調事宜,詔報可。而應期以他事去,詔以都御史姚鏌代,命懸金購猛。然鏌知猛無反心,猛方奏辯,鏌亦欲緩師。而巡按謝汝儀與鏌卻,乃誣鏌之子淶納猛萬金,廉得淶書獻之。鏌惶恐,乃再疏請徵。於是部趣鏌剋期進,鏌偕總兵官朱麒發兵八萬,以都指揮沈希儀、張經等統之,分道並入。猛聞大兵至,令其下毋交兵,裂帛書冤狀,陳軍門乞憐察。鏌不聽,督兵益急,沈希儀斬猛長子邦彥於工堯隘。猛懼,謀出奔,而歸順州知州岑璋,猛婦翁也,其女失愛,璋欲藉此報猛,乃甘言誘猛走歸順,鴆殺之,斬首以獻。

六年,鏌以田州平,告捷京師,乃請改田州為流官,並陳善後七事,詔俱從之。

鏌留參議汪必東、僉事申惠、參將張經以兵萬人鎮其地,知府王熊兆署府事。會必東、惠皆移疾他駐,惟經、熊兆在府,兵勢分,防守稍懈。於是逆黨盧蘇、王受等乃為偽印,誑言猛在,且借交阯兵二十萬,以圖興復。蠻民信之,聚眾薄府城。經出擊,兵少不敵,欲引還,而城中陰為內應,呼噪四出,官軍腹背受攻,力戰不支,突圍渡江走,賊逼其後,爭舟溺死者甚眾。賊沿江置闌索,伏藥弩,夾岸並起。官軍且戰且行,抵向武,失士卒三四百人。賊遂入據府城,燒倉粟以萬計。御史石金上其事,頗委罪前撫盛應期生事召釁,而給事中鄭自璧因請仍檄湖廣永順、保靖兵併力剿賊。帝以四方兵數萬方歸休,豈可復調,命再計機宜以聞。

時盧蘇等雖據府叛,佯聽撫,遣人迎署府事王熊兆。而其黨王受等糾眾萬餘,攻據思恩城,執知府吳期英、守備指揮門祖蔭等。已而釋期英等,亦投牒上官,願聽招撫。都御史姚鏌以兵未集,姑受之以緩其謀。遣諜者檄東蘭、歸順、鎮安、泗城、向武諸土官,各勒兵自效,且責失事守巡參將等官立功自贖。復疏調湖廣永、保土兵,江西汀、贛畬兵,俱會於南寧,併力進剿。帝以蠻亂日久,鎮巡官受命大征,未及殄絕,輒奏捷散兵,使餘孽復滋,罪不容逭。姑赦前過,益圖新功。乃起原任兵部尚書新建伯王守仁總督軍務,同鏌討之。

時受既入思恩,封府庫,以賊兵守之,而自攻武緣。守巡官鄒輗等率兵至思恩,思恩千夫長韋貴、徐伍等遣壯士由間道入城為內應,夜引官兵奪門,殺賊二十餘人,收府印及庫物,護送期英於賓州,因招撫城中未下者。時受攻武緣甚急,參將張經堅壁拒守。鎮守頭目許用與戰,斬其渠帥一人。賊見援兵大集,乃遁去。鏌以聞。

帝以田州、思恩賊鋒雖挫,首惡未擒,仍令守仁亟督兵剿撫。守仁威名素重,及督軍務,調兵數萬人至,諸蠻心懾。守仁至南寧,道中見受等勢盛,度亦未可卒滅,上疏極陳用兵利害。兵部議以守仁所見未確,復陳五事,令守仁詳計其宜,於是守仁又疏云:

臣奉命於去年十二月至廣西平南縣,與巡按御史石金及籓臬諸將領等會議。思、田禍結兩省,已踰二年。今日必欲窮兵盡剿,則有十患。若罷兵行撫,則有十善。臣與諸臣,攄心極論,今日之局,撫之為是。

臣抵南寧,遂下令盡撤調集防守之兵。數日內解歸者數萬,惟湖兵數千,道阻遠,不易即歸,仍使分留南寧,解甲休養,待間而動。而盧蘇、王受先遣其頭目黃富等訴告,願得歸境投生,乞宥一死。臣等諭以朝廷威德,令齎飛牌,歸巢曉諭,期以速降無死。蘇、受等得牌,皆羅拜踴躍,歡聲雷動。

尋率眾至南寧城下,分屯四營。蘇、受等囚首自縛,與頭目數百人赴軍門請命。臣等復諭之曰:「朝廷既赦爾罪,爾等擁眾負固,騷動一方。若不示罰,何以雪憤?」於是下蘇、受於軍門,各杖一百,乃解其縛。又諭之曰:「今日宥爾死者,朝廷好生之德;必杖爾者,人臣執法之義。」眾皆叩首悅服,願殺賊立功。臣隨至其營,撫定其眾七萬餘人,復委布政使林富等安插,於二月二十六日悉命歸業。是皆皇上至孝達順之德,神武不殺之威,未期月而蠻民率服,不折一矢,不傷一人;而全活數萬生靈,即古舞干之化,奚以加焉。

疏聞,帝嘉之,遣行人齎敕獎賚。於是守仁復疏言:

思、田久構禍,荼毒兩省,已踰二年。兵力盡於哨守,民脂竭於轉輸,官吏疲於奔走。地方臲卼,如破壞之舟,漂泊風浪,覆溺在目,不待智者而知之矣。必欲窮兵雪憤,以殲一隅,無論不克,縱使克之,患且不測。況田州外捍交阯,內屏各郡,深山絕谷,瑤、僚盤據。使盡誅其人,異日雖欲改土為流,誰為編戶?非惟自撤其籓籬,而拓土開疆以資鄰敵,非計之得也。

今岑氏世效邊功,猛獨詿誤觸法,雖未伏誅,聞已病死。臣謂治田州非岑氏不可,請降田州府為田州,而官其子,以存岑氏之後。查猛有二子,長邦佐,自幼出繼為武靖州知州。武靖當瑤賊之衝,邦佐才足制馭,宜仍舊職。而今所建州,請以猛幼子邦相授吏目,署州事,俟後遞升為知州,以承岑氏之祀。設土巡檢諸司,即以盧蘇、王受等九人為之,以殺其勢。添設田寧府,統以流官知府,以總其權。

從之。惟以守仁所奏岑猛子,與撫按所報異,令再覆。

於是守仁言:「臣初議立岑氏後,該府土目及耆老俱言岑猛本有四子:長邦佐,妻張氏出;次邦彥,妾林氏出;次邦輔,外婢所生;次邦相,妾韋氏出。猛嬖溺林氏而張失愛,故邦佐自幼出繼武靖。邦彥既死,邦佐得武靖民心,更代亦難其人。欲立邦輔,土目謂外婢所生,名實不正。惟邦相係猛正派,質貌厚重,堪繼岑氏。故當時直謂猛子存者二人,亦所以正名慎始,杜後日之爭也。」疏上,如議行。

八年,守仁於思、田既議設流官,又議移南丹衛於八寨,改思恩府城於荒田,改設鳳化縣治於三里,添設流官縣於思龍,增築五鎮城堡於五屯。及侍郎林富繼之,又言:「田州界居南寧、泗城,交通雲、貴、交阯,為備非一,不宜改設流官。南丹衛設在賓州,既不足以遙制八寨,遷八寨又不得以還護賓州。為今日計,獨上林之三里,守仁所議設縣者,可遷南丹衛於此。夫設縣則割賓州之地以益思恩,是顧彼而失此也。遷衛則扼八寨之吭以還護賓州,是一舉而兩得也。然不宜屬田州,而仍屬南寧為便。」其議與守仁頗有異同,詔從富言。

初,邦相兄邦彥有子芝,依大母林氏、瓦氏居,官給養田。其後邦相惡蘇專擅,密與頭目盧玉等謀誅蘇及芝。蘇知之,會邦相又侵削二氏原食莊田,二氏遂與蘇合謀,以芝奔梧州,赴軍門告襲,蘇又為芝疏請。尋令人剌邦相,邦相覺,殺行剌者。而蘇遂伏兵殺盧玉等,以兵圍邦相宅,誘邦相出,乘夜與瓦氏縊殺之。巡按御史曾守約以聞,帝命守臣亟為勘處。蘇之殺邦相也,歸順、鎮安、泗城、向武諸土官群起構難,互相訐奏。當事者謂以岑芝承襲未定,田州無主,致令鄰封覬覦,當給札付令芝管事。蘇又請早給芝冠帶,以撫田州,而自悔罪,願里糧立功,及追補累年所逋糧賦。巡按御史諸演疏聞,部議以土蠻自相仇殺,當從末減,皆令立功,方準贖罪復官。

三十二年,芝死,子大壽方四歲。土人莫葦冒岑姓,及土官岑施,相煽構亂,提督郎檟奏令思恩守備張啟元暫駐田州鎮之,報可。三十四年,田州土官婦瓦氏以狼兵應調至蘇州剿倭,隸於總兵俞大猷麾下。以殺賊多,詔賞瓦氏及其孫男岑大壽、大祿銀幣,餘令軍門獎賞。四十二年以平廣西瑤、僮功,準岑大祿實受知州職。

泰昌元年,總督許弘綱奏:「田州土官岑懋仁肆惡起釁,窺占上林,納叛人黃德隆等,糾眾破城,擅殺土官黃德勳,擄其妻女印信,乞正其罪。」詔令岑懋仁速獻印,執送諸犯,聽按臣分別正法,違則進剿。天啟二年,巡撫何士晉請免懋仁逮問,各率土兵援剿,有功優敘,從之。

田州世岑氏,改流者再,而終不果。盧蘇再叛弒主,終逸於罰,論者以為失刑云。

上林在田州東,宋置,隸橫山寨。元屬田州路。洪武二年,土官黃嵩歸附,授世襲知縣,流官典史佐之。

恩城州,唐置,宋、元仍舊。明初因之,隸廣西布政司,朝貢如例。成化十九年,知州岑欽,田州土官岑溥叔也,相仇殺。溥敗,欽入田州,焚府治,大肆殺掠。溥愬於制府,下三司官鞫理。弘治三年,欽復入田州,與泗城土官岑應分據其地。巡撫秦紘請調兵剿之。兵部言兵不可輕動,惟令守臣諭令應縛欽自贖。五年,欽走岑應所借兵,總鎮檄應捕之,欽遂殺應父子。已而應弟接佯以兵送欽,亦殺欽父子。有司以恩城宜裁革,從之,州遂廢。

上隆州,宋置,隸橫山寨。元屬田州路,明因之。後改隸布政司。洪武十九年,上隆知州岑永通遣從子岑安來貢,賜綺帛鈔錠。洪熙元年,土官知州岑瓊母陳氏來朝,貢馬,賜鈔幣。宣德四年以陳氏為知州。時瓊已卒,無子,土人訴於朝,願得陳氏襲職,故有是命。

都康州,宋置,隸橫山寨。元屬田州路。洪武間,為蠻僚所據。三十二年復置,隸布政司。土官馮姓。其界東南抵龍英,西至鎮安,北至向武。


\end{pinyinscope}