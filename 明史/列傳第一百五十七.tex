\article{列傳第一百五十七}

\begin{pinyinscope}
艾萬年李卑湯九州楊正芳楊世恩陳于王程龍等侯良柱子天錫張令汪之鳳猛如虎劉光祚等虎大威孫應元姜名武王來聘等鄧祖禹尤世威王世欽等侯世祿子拱極劉國能李萬慶

艾萬年,米脂人。由武學生從軍,積功至神木參將。崇禎四年從曹文詔復河曲。點燈子入山西,萬年從文詔連敗之桑落鎮、花地窊、霧露山。都司王世虎、守備姚進忠戰死。賊退屯石樓之康家山,西距河三十里。綏德知州周士奇、守備孫守法伏兵含峪,渡河襲殺之。五年從參政樊一蘅討平不沾泥。山西告警,隸文詔東討,與李卑一月奏五捷。又與賀人龍敗八大王、掃地王兵。明年,賊將東遁,連破之延家山、亢義村、賈寨村,擢副總兵。

初,山西既中賊,其土寇亦乘間起,三關王剛、孝義通天柱、臨縣王之臣皆殘破城邑。後見賊衰,相繼歸順,然陰結黨不散。巡撫戴君恩新視事,謀誅之。七年正月迎春,召王剛宴,殺之,并殺通天柱於他所,而萬年亦捕殺王豹五與其黨領兵王,生擒翻山動,姬關鎖、掌世王,獻俘京師,晉中巨盜略靖。豹五即之臣也。有議君恩殺降者,給事中張第元力言諸賊蹂躪之慘,請錄萬年功。萬年適遘疾告歸,尋加署都督僉事。

八年二月,上疏言:

臣仗劍從戎七載,復府谷,解孤山圍,救清水、黃甫、木瓜十一營堡。轉戰高山,設伏河曲,有馬鎮、虎頭巖、石臺山、西川之捷。戰平陽、汾州、太原,復臨縣及𧾷虒亭驛。大小數十戰,精力盡耗。與臣共事者李卑,溘先朝露。臣病勢奄奄,猶力戰冀北。又撫剿王剛、豹五、領兵王、通天柱,解散賊一萬三千有奇。蒙恩許臣養病,而督臣洪承疇檄又至,臣不敢不力疾上道。但念滅賊之法,不外剿撫,今剿撫俱未合機宜,臣不得不極言。

夫剿賊不患賊多,患賊走。蓋疊嶂重巒,皆其淵藪,兵未至而賊先逃,所以難滅,其故則兵寡也。當事非不知兵寡,因糗糧不足,為茍且計,日引月長,以至於今,雖多措餉,多設兵,而已不可救矣。宜合計賊眾多寡,用兵若干,餉若干,度其足用,然後審察地利,用正用奇,用伏用間,或擊首尾,或衝左右,有不即時殄滅者,臣不信也。

次則行堅壁清野之法,困賊於死地,然後可言撫。蓋群賊攜妻挈子,無城柵,無輜重,暮楚朝秦,傳食中土,以剽掠為生。誠令附近村屯移入城郭,儲精兵火器以待之,賊衣食易盡,生理一絕,鳥驚鼠竄。然後選精銳,據要害以擊之;或體陛下好生之心,誅厥渠魁,宥其協從,不傷仁,不損威,乃撫剿良策。

帝深嘉之,下所司議行,然卒不能用其策也。

尋授孤山副總兵,戍平涼。當是時,總督洪承疇迫六月滅賊之期,急進戰。諸將見賊眾兵寡,咸自揣不敵,而勢不可止。萬年及副將劉成功、柳國鎮,遊擊王錫命合兵三千,以六月十四日至寧州之襄樂,遇賊大戰,斬首數百。伏兵驟起,圍之數重。萬年、國鎮力戰不支,皆戰歿。成功、錫命負重傷歸。士卒死者千餘人。事聞,贈恤如制。

李卑,字侍平,榆林人。由千總擢守備。天啟初,總督王象乾設薊鎮車營五,以卑為都司僉書,統西協後車營。遷山海關遊擊,坐事罷歸。

崇禎二年,陜西巡撫劉廣生議討延慶回賊,三道進兵,命卑與遊擊伍維籓等由西路入。卑簡精騎二百,追擊兩晝夜,行四百里抵保安寧塞,連破之,共獲首功一千有奇。旋起延安參將。時群盜蜂起,延發尤甚,卑連敗之富家灣、松樹屯。四年,神一元陷保安,卑與寧夏總兵賀虎臣守延安,賊不敢犯。尋擢孤山副總兵。譚雄陷安塞,據其城,卑與王承恩擊降雄,戮之,斬首五百三十餘級。五年春,混天猴陷宜君、鄜州,其夏攻合水。卑及參將馬科追至甘泉山。七月破之延水關,斬首六百二餘級。其地東限黃河,賊溺死者無算,科部卒斬混天猴以獻。初,卑及遊擊吳國俊等斬賊魁三人於甘泉橋子溝,尋剿賊固原,斬其魁薛仁貴等三人。

時陜西賊多流入山西。詔卑及賀人龍各率部卒千,隸總督張宗衡麾下。會王自用陷遼州,聞官兵至,棄城走。六年春,諸軍入城,多殺良民冒功,卑獨嚴戢其下無所擾。已,敗賊陽城之郎家山,又與艾萬年連敗之南獨泉土河村,復敗之芃塸村。賊入濟源山中,巡撫許鼎臣檄卑、萬年合剿,卑破之天井關。七月,臨洮總兵曹文詔改大同,命卑代署其事,協討河北賊,加都督僉事,數有功。其冬,賊盡走河南,命卑援剿。七年春,敗賊內鄉,馳至光化,與楚兵敗賊蓮花坪、白溝坪,實授臨洮總兵官,討賊湖廣,賊多聚鄖、襄,總理盧象昇方倚卑辦賊,六月卒於官。

卑善持紀律,所至軍民安堵。為人有器度,當倉猝,鎮靜如常。贈右都督,賜祭葬。

湯九州,石埭人。崇禎時,為昌平副總兵。六年夏,流賊大擾河北、畿南。命九州協剿,與左良玉等屢破賊兵,賊悉渡河而南。其冬,大敗過天星於吳城鎮,斬首四百二十級。追賊闖天王等五華集,又敗之,斬首六百四十餘級。七年擊賊嵩縣之潭頭,斬首三百二十級。賊駐商、雒,謀再入山西。左良玉迎擊於商南,九州遣部將趙柱、周爾敬逆之雒南。賊至商州返。已,復侵閿鄉。九州病,遣部將凌元機、胡良翰等搜山,悉敗歿。九州尋赴援山西。未幾,以河南剿賊功,加署都督僉事。八年春,被劾褫官,從軍自效。洪承疇入關,以吳村、瓦屋為商南賊走內鄉、淅川要地,令九州偕良玉扼之。尋移駐洛陽。九年二月,賊走登封石陽關,與伊、嵩賊合。九州期良玉夾擊,良玉半道歸。九州以孤軍千二百人由嵩縣深入。賊屢敗,窮追四十餘里,誤入深崖。遇賊數萬,據險攻圍。九州勢不敵,夜移營,為賊所乘,遂戰歿。從孫文瓊伏闕三上書請恤,不報。文瓊後亦殉難。

時有楊正芳者,天啟間以小校從軍,屢剿貴州賊,積功至副總兵。敘桃紅壩功,加署都督同知。崇禎三年擊破定番叛苗。七年,賊陷當陽,正芳以鎮筸兵敗賊班鳩灘,復其城。湖廣巡撫唐暉以獻陵、惠籓為重,令正芳及總兵許成名專護荊州、承天。正芳連奏金沙鋪、蓮花坪、白溝坪、官田、石門山之捷。陳奇瑜出師鄖陽,正芳偕成名、鄧從竹山、竹溪、白河分道進,皆大獲。至十月,正芳督筸兵千餘援雒南,戰敗,及部將張上選皆死焉,一軍盡歿。贈太子少師、左都督,世廕指揮同知,再廕一子守備,賜祭葬,有司建祠。

又有楊世恩者,崇禎時,歷官湖廣副總兵。七年春,敗賊竹溪。大雨,山水驟發,賊多漂溺死,餘潰走。世恩疾擊,斬鎮山虎等四十餘人。已,追賊石河口,連戰康家坪、蚋溪,功最。八年冬,敗賊孝感。九年春,祖寬大破賊滁州。世恩從盧象升馳至,復大破之。十年春,與秦翼明破劉國能於細石嶺,獲其魁新來虎。賊陷隨州,責戴罪自贖。十二年冬,督師楊嗣昌令守宜城。會賊羅汝才、惠登相分屯興山、遠安,夷陵告急。嗣昌檄世恩及荊門守將羅安邦赴救。至洋坪猴兒洞,道險甚,嗣昌再檄召還,而安邦由祚峪,世恩由重陽坪已兩道深入,期至馬良坪合兵。汝才、登相圍之香油坪,嗣昌連發數道兵往援,皆以道遠不能進。世恩等被困久,突圍走黃連坪,絕地無水,士饑渴甚。賊至,兩軍盡覆,世恩、安邦並死。

陳于王,字丹衷,吳縣人。世為蘇州衛千戶。既襲職,兩舉武鄉試,授奇兵營守備。以捕獲海盜功,遷都司僉書,守崇明蛇山。盜王一爵等亂海濱,於王率戰船數十擊之羊山,持刀跌入其舟,生擒一爵,殲其黨殆盡。上官交薦,遂知名。天啟初,經略熊廷弼用為標下參將。代者至,餘于王酒暴卒。其子訴于王毒殺之,逮繫久不釋。

崇禎二年,京師有警。巡撫曹文衡貰其罪,署前鋒遊擊,將兵勤王。既至,兵事已解,遂南還。久之,巡撫張國維用為中軍守備。九年,賊入江北,圍廬州,陷和州。國維遣于王守六合,守備蔣若來守江浦。賊方圍江浦,若來急入與知縣李維越固守。賊登城,若來拒卻之。縋下角賊,矢著其頰,左臂傷,裹血還戰,賊乃退。六合無城,若來與于王掎角捍賊,二邑賴以全。賊犯宿松,于王弟國計偕指揮包文達等以二千人往救。文達敗歿,于王驟馬入,拔其弟而出。

十年正月,賊分犯江浦、六合及安慶。國維遣部將張載賡等援安慶,而以新募兵二千令副將程龍及於王、若來分戍二邑。已而賊不至,國維議赴安慶,城太湖,乃提龍等三將兵西上。三月,賊犯太湖,副將潘可大將安慶兵九百,龍等三將將吳中兵三千六百,禦之酆家店。賊先犯可大營,龍等至,夾擊之,賊多死。夜復至,中伏,亦敗去。監軍史可法欲退扼要害,諸將不從,掘塹守二十四日。羅汝才、劉國能等七營數萬眾齊至,圍數重。諸將突擊,頗有殺傷。可法偕副將許自強馳救,扼於賊,鳴大炮遙為聲援,諸將亦呼噪突圍。會天雨,甲重不得出。明日日中,賊四面入,將士短兵接戰。可大戰死,龍引火自焚死。於王手執大刀,左右殺賊,傷重力竭,北面叩頭自刎死,閱十日面色如生。若來服圉人衣以免。同死者,武舉詹兆鵬首觸石死。陸王猷殺賊過當,賊臠分其肉死。莫是驊、唐世龍及千戶王定遠皆力戰死。百戶王弘猷為賊所執,鋸齒斷足,罵不絕聲死。士卒脫者僅千餘人。事聞,贈於王昭勇將軍、指揮使,世廕副千戶。餘贈廕有差。

侯良柱,字朝石,永寧衛人。天啟初,累官四川副總兵。討奢崇明父子,復遵義城。又與參議趙邦清招降奢寅黨安鑾。六年五月代李維新為四川總兵官,鎮永寧。時崇明敗奔水西,與安邦彥合,貴州兵數討不克。

崇禎二年,總督朱燮元遣貴州總兵許成名復赤水衛,崇明、邦彥以十餘萬眾來爭。成名還永寧,賊追之銳甚。良柱偕監軍副使劉可訓出戰小卻,成名等來援,賊乃據五峰山桃紅壩。越數日,良柱乘賊不備,與副將鄧等侵早霧迫之,賊大潰。成名聞山上呼噪聲亦出。賊奔鵝項嶺,徑長而狹,人馬不能容。良柱、軍至,賊復大敗,死者數萬人。崇明、邦彥與邦彥黨偽都督莫德並授首,俘其黨楊作等數千人。積年巨寇平,時稱西南奇捷。

四川巡撫張論上其功不及黔將。成名等怒,言邦彥、德乃己部將趙國璽所斬,且崇明猶未死。燮元信之,奏於朝。兵部不能決,賞久不行。御史孫征蘭言:「訊俘囚阿癡、楊作等,咸云邦彥即時授首,灼然非黔兵力。」帝即命獻俘告廟,傳首九邊。川中撫按及御史毛羽健皆訟良柱、可訓功,詆燮元。燮元疏辯且求去,賞遂格不行。良柱怨燮元,不為用,至與相訐奏,解職侯勘。久之,御史劉宗祥列上功狀。七年八月,始錄前功,進良柱左都督,世廕錦衣指揮僉事;成名等亦優敘。未幾,復為四川總兵官。

八年夏,總督洪承疇大舉討賊,令良柱扼賊入川路。戰鳳縣三江口,斬首三百七十有奇。明年冬,賊犯漢中,瑞王遣使乞師。良柱督兵援,與他將同卻賊。十年四月,川中地震者七,地鳴者一,占主兵。賊果入犯,陷南江、通江。帝切責良柱及巡撫王維章。時良柱駐廣元,盡召諸地兵九千有奇,分防扼險,止餘二千人。賊知其勢弱,五月復寇川北。維章告急於朝。會賊轉掠他所,良柱乃撤還守隘兵,專守廣元。維章以為非計,上章言之。十月,李自成、過天星、混天星等陷寧羌,分三道入寇。良柱急拒戰於綿州,眾寡不敵,陣亡。賊直逼成都,維章方守保寧,反在外,連失三十餘州縣。帝大怒,命逮二人下詔獄,猶未知良柱死。獄成,維章遣戍,追奪良柱官。

十三年,良柱子指揮天錫伏闕言:「臣與賊不共戴天。願捐貲繕甲,選募勁旅及臣父舊將,自當一隊,與賊血戰,下雪父恥,上報國恩。」帝深嘉之,命授遊擊,赴嗣昌軍立功。已,嗣昌言天錫所將親丁二百六十人及召募精卒五六百人皆剽悍敢戰。帝益嘉之,再增一秩。

張令,永寧宣撫司人。天啟元年,奢崇明反,令為偽總兵,從攻成都。令雖為賊用,非其志。崇明敗歸永寧,令結宋武等乘間擒其偽丞相何若海,率眾以降。崇明怒,殺令一家,夷其先墓。巡撫朱燮元言令等為國忘家,請優擢示勸,命與武並授參將。後屢從大軍征討,頻有功,加副總兵,仍視參將事,後實授建武遊擊。崇禎中,屢遷副總兵,鎮川北。七年,流賊入犯,總兵張爾奇以令為先鋒,副將陳一龍、武聲華為左右翼,拒之員山。令追至龍潭,一龍等不至,面中三矢,斬賊百餘級而還。賊犯略陽,令又擊敗之,扼保寧、漢中諸要害,秦賊不敢犯。十年冬,李自成等陷四川三十餘州縣,總兵侯良柱陣亡,令獲免。楊嗣昌之督師也,張獻忠等悉奔興安,為令所扼,不得入漢中,乃轉寇夔州。十三年二月大敗瑪瑙山,走岔溪千江河,令復與副將方國安大破之。令時年七十餘,馬上用五石弩,中必洞胸,軍中號「神弩將」。

獻忠轉入柯家坪,其地亂峰錯峙,箐深道險。令率眾追及之,分其下為五,鼓勇爭利。賊眾官軍寡,國安為後拒,他道逸去。令獨深入,被圍,居絕阪中,屢射賊營,應弦斃者甚眾。水遠士渴,賴天雨以濟,圍終不解。襄陽監軍僉事張克儉言於總督鄭崇儉曰:「張令健將,奈何棄之!」急令參將張應元、汪之鳳從八台山進,總兵賀人龍從滿月嶆進。三月八日,應元等先至。令方與賊鬥,呼聲動山谷。應元等應之,內外夾擊,賊乃敗去。令與賊萬餘相持十三日,所殺傷過當,其卒僅五千耳。時巡撫邵捷春駐重慶,遣守黃泥窪,倚令及秦良玉為左右手。後捷春移大昌,以令守竹箘坪,防賊逸。九月,獻忠兵大至。令力戰,中矢死,軍遂敗。

之鳳既解柯家坪圍,後與應元同守夔州之土地嶺,部卒多新募。獻忠盡銳來攻,之鳳、應元力戰。賊分兵從後山下,突入其營。應元突圍出。之鳳走他道免,山行道渴,飲斗水臥,血凝臆而死。踰月,令亦戰死。軍中失二將,為奪氣。

猛如虎,本塞外降人,家榆林,積功至遊擊。崇禎五年,擊邢紅狼於高平,解其圍。明年敗賊壽陽黑山,覆姬關鎖軍。已,從曹文詔追賊西偃、碧霞村,斬混世王。與頗希牧逐賊壽陽東。又與陳國威、馬傑破來遠寨。從文詔大破賊范村。國威以步卒三百夜劫賊紅山嶺,如虎、傑及虎大威、和應詔擊殺九條龍。尋以巡撫許鼎臣命,由文水入山剿賊。又與大威、應詔、傑由皋落山剿東犯之賊,並有功。賊流入畿南,山西警漸息,如虎仍隸鼎臣。七年剿賊沁源,馘五條龍。

如虎驍勇善戰,與虎大威齊名。戴君恩、吳甡相繼為巡撫,並委任之。以功進參將。其年冬,賊在河南,欲乘冰北渡,如虎、大威扼之河濱。八年二月與大威、國威斬劇賊高加計。山西賊盡平,用甡薦加副總兵。其冬以防河功,加署都督僉事。連歲防河及援剿河南賊,勞績甚著。十一年冬,京師有警,如虎督兵勤王。明年四月擢薊鎮中協總兵官。

十三年坐事落職,發邊方立功。督師楊嗣昌請於朝,令從入蜀。十一月,監軍萬元吉大饗將士於保寧。以諸軍進止不一,擢如虎為正總統,張應元副之,率軍趨綿州。分遣諸將屯要害。而元吉自間道走射洪,扼蓬溪以待賊。賊方屯安岳界,偵官軍且至,宵遁,抵內江。如虎簡驍騎追之。元吉、應元營安岳城下,以扼其歸路。十二月,張獻忠陷瀘州,其地三面阻江,惟立石站可北走。元吉以賊居絕地,將遣大兵南搗其老巢,而伏兵旁塞玉蟾寺,蹙賊北竄永川,逆而擊之,可盡殄。永川知縣已先遁,城中止丞簿一二人。如虎覓嚮導不可得,夜宿西關空舍。及抵立石,賊已先渡南溪返走。關中將賀人龍軍隔水不擊,賊遂越成都,走漢川、德陽,渡綿河入巴州。

明年正月,嗣昌親統舟師下雲陽,檄諸將陸追賊,諸軍乃盡躡賊後。賊折而東返,歸路悉空,不可復遏。如虎所將止六百騎,餘皆左良玉部兵,驕悍不可制,所過肆焚掠,惟參將劉士傑勇敢思立功。諸軍從良玉,多優閒不戰。改隸如虎,馳逐山谷風雪中,咸怨望。謠曰:「想殺我左鎮,跑殺我猛鎮。」時賀人龍兵已大噪西歸,所恃止如虎,元吉深憂之。賊自巴州至開縣,官軍追之,遇諸黃陵城。日晡雨作,諸將疲乏,請詰朝戰。士傑奮曰:「四旬逐賊,今始及之。舍弗擊,我不能也。」執戈先,如虎激諸軍繼之。士傑所當,輒摧陷。獻忠登高望官軍,見無後繼,密抽壯騎潛行箐谷中,乘高大呼馳下。良玉兵先潰,士傑及游擊郭開、如虎子先捷並戰死。如虎率親兵力戰,部將挾上馬,潰圍出,旗纛軍符盡失。乃收殘卒從嗣昌下荊州。及嗣昌死,率所部扼德安、黃州。會疽發背,不能戰,退屯承天,尋移駐南陽。

十一月,李自成覆傅宗龍兵,乘勢來攻。如虎與劉光祚憑城固守,用計殺賊精卒數千。已而城破,如虎持短兵巷戰,大呼衝擊,血盈袍袖。過唐府門,北面叩頭謝上恩,自稱力竭,為賊揕死。光祚及分守參議艾毓初、南陽知縣姚運熙並死之,唐王亦遇害。

光祚,字鴻其,榆林衛人。初為諸生,棄去。承祖廕,歷官延綏游擊。崇禎三年奉詔勤王,與何可綱等戰灤州有功,遷汾州參將。五年與遊擊王尚義敗賊張有義於臨縣。賊還兵犯之,軍盡覆,光祚僅以身免。被徵,未行,偕諸將復臨縣,詔除其罪。六年,賊犯石樓,光祚分三道擊,大敗之,斬隔溝飛、撲天虎等六人,獲首功三百七十。又數敗賊於臨縣、永寧。撲天飛等詐降,光祚設伏斬之。已,擊敗賊魏家灣、黑茶山。七年剿敗王剛餘黨,斬四百餘級,加署都督僉事,為山西副總兵。敗賊崞縣,復其城。八年,賊渠賀宗漢號活地草者,見其黨劉浩然、高加計破滅,偽乞降。光祚伏兵斬之。晉中群盜皆盡,乃移光祚於宣府。久之,命率兵援剿河南。十一年連敗賊白果園、襄城。已,擢保定總兵官,仍協討河南賊。其冬,畿輔有警,馳還鎮。大清兵薄保定,以光祚堅守,不攻而去。光祚尋從總督孫傳庭南下。明年二月,大清兵還至渾河,值水漲,輜重難渡,諸將王樸、曹變蛟等相顧不敢擊,光祚恇怯尤甚。視師大學士劉宇亮劾之,詔即軍前正法。光祚適報武清捷,宇亮乃繫之武清獄,而拜疏請寬。帝怒罷宇亮,論光祚死。十四年,大學士范復粹錄囚,力言光祚才武,命充為事官,戴罪辦賊。光祚舉廢將尤翟文等,帝亦從之。

當是時,賊已陷河南、襄陽,中原郡縣大抵殘破。光祚士馬無幾,督師丁啟睿尤怯,光祚雖少有克捷,而賊勢轉盛。及傅宗龍敗歿於項城,南陽震恐。光祚適經其地,唐王邀與共守,城陷遂死。

虎大威,榆林人。本塞外降卒,勇敢嫻將略,從軍有功,累官山西參將。崇禎三年冬,從總兵尤世祿擊王嘉胤於河曲,力戰被傷。五年從總督張宗衡剿賊臨川、潞安、陽城、沁水,連勝之。明年從巡撫許鼎臣擊賊介休,殲其魁九條龍。時賊去山西,遁據輝林、武陟山中,約二萬餘。鼎臣令曹文詔自黎城入,大威、猛如虎諸將自皋落山入,賊屢敗。尋移大威守平陽。七年,巡撫吳甡至,察諸將中惟大威、如虎沈毅可屬兵事,委任之。其冬與如虎扼賊渡河。高加計據岢嵐,四出剽掠。明年三月,二將追至忻、代山中。加計馬上舞三十斤長梃突陣,大威射殺之,追斬其眾五百人,餘黨悉平。甡薦二人忠勇,進大威副總兵。其冬以扼賊功,加署都督僉事。

九年八月,畿輔被兵,率師入援。明年春,命援剿陜西賊,遂代王忠為山西總兵官。上疏言諸將討賊,零級不可取,生口不可貪,封域不可限。帝採納之。十一年詔兵部甄別諸大將,大威以稱職增秩。其年冬,京師戒嚴。命總督盧象昇統大威及宣府總兵楊國柱、大同總兵王樸入衛。尋從象昇轉戰至鉅鹿賈莊,被圍數匝,象昇死焉,大威等潰圍出。督師劉宇亮、總督孫傳庭皆言大威、國柱敢勇,身入重圍,視他將異,乞令立功自贖。大威亦上章請罪。帝不從,卒解其任。尋令從軍辦賊。

十四年正月,李自成圍開封。總督楊文岳遣大威及副將張德昌先率五千人渡河。會賊已解圍去,乃會河南巡撫李仙風於偃師,以兵少未敢擊賊。待文岳軍至,與賊戰鳴皋,大破之,又與監軍道任棟挫賊平峪。七月,自成及張獻忠、羅汝才攻鄧州,大威從文岳擊破之,斬首千餘級。陜西總督傅宗龍出關討賊,文岳、大威會之。九月次新蔡,抵孟家莊。將戰,秦帥賀人龍軍先潰,大威軍亦潰,遂奔沈丘。賊連陷河南鄧、許,再圍開封。大威從文岳援之,賊引去。明年二月,師次郾城。督師丁啟睿、總兵左良玉方與賊鏖戰,文岳督大威及馮大棟、張鵬翼等合擊,賊大敗。相持十一晝夜,俘斬數千。賊遂東陷陳州、歸德,已,復圍開封。七月朔,啟睿、文岳、大威及良玉、楊德政、方國安之師畢會。啟睿欲急擊,良玉不從,先走。大威諸軍亦走。帝大怒,立誅德政,黜譴啟睿諸人。大威時奔汝寧,出攻賊寨,中炮死,乃免其罪。

大威為偏裨,最有聲。及為大帥,值賊勢益張,所將止數千人,不能大有所挫。然身經數十戰,卒死王事,論者賢之。

孫應元,不知何許人。歷官京營參將,督勇衛營。勇衛營即騰驤、武驤四衛也,其先隸御馬監,專牧馬。莊烈帝銳意修武備,簡應元及黃得功、周遇吉等訓練,遂成勁旅。崇禎九年秋,從張鳳翼軍畿輔,有功,進副總兵。再以功增秩一等。明年,河南賊熾,應元、得功慷慨請行。帝壯之。發卒萬人,監以中官劉元斌、盧九德,戒毋擾民。諸將奉命,軍行肅然。十二月大破賊鄭州,再破之密縣,先後斬首千七百。明年正月大破之舞陽、光山、固始。四日三捷,斬首二千九百有奇,賊乃謀犯江北。元斌、九德南趨潁州,護鳳陵,密遣應元、得功督騎兵扼賊前。自南而北,破之方家集。賊遂由固始走商城。錄功,加都督僉事。已,復破之新野,又大破之遂平。熊文燦方主撫不戰。而賊憚應元等,多降,降者亦不遽叛。文燦以此擅撫賊功。已而京師有警,召應元等還,賊遂無所忌。帝初聞禁軍屢破賊,大喜,累加應元都督同知,賜銀幣蟒服,至是論功,遂進左都督,加銜總兵官,世蔭錦衣副千戶。

十二年五月,張獻忠、羅汝才復叛,仍命元斌、九德監應元、得功軍南征。應元等馳至南陽。會馬光玉屯淅川之吳村,偽乞撫,規渡漢江應獻忠。淅川知縣郭守邦說降其黨許可變、胡可受。可變即賊改世王,可受則安世王也。可變夜至,處之東關。可受為光玉所持,約未定。應元、得功趨內鄉掩其背,令副將周遇吉等分道別擊之。文燦所遣陳洪範亦至。八月至小黃河口,參將馬文豸等力戰,可受敗,呼曰:「始與許王約降者我也,今歸命。」遇吉駐馬受之。應元、得功遂進兵王家寨。賊分屯南北兩山,用木石塞道。應元率文豸戰其南,得功率副將林報國戰其北,河南兵又扼華陽關,賊遂大敗,光玉遁免。元斌至軍,檄除可變、可受罪,授以官,報先後首功三千人。

及楊嗣昌督師襄陽,令元斌、應元戍荊門,護獻陵。十三年七月與副將王允成、王之綸、監軍僉事孔貞會等大破羅汝才於豐邑坪,斬首二千三百,生擒五百有奇。混世王、小秦王皆降。時稱荊楚第一功。十五年春,擊賊羅山,力戰。孤軍無援,遂陣歿。贈恤如制。

應元善戰,在行間多與黃得功偕。應元死,得功勛益顯,故其名尤震於世。

姜名武,字我揚,保德州人。舉天啟二年武會試,授大同威遠守備。崇禎初,遷大水峪遊擊。築杏山城有功,遷宣府西城參將,擊斬大盜王科。移守宣府右衛,擢通州副總兵。護諸陵有功,以故官典保定總督楊文岳中軍,兼忠勇營團練事。

十五年,李自成圍開封急,名武從文岳往援。時諸軍壁朱仙鎮者十餘萬,左良玉最強。一夕,其軍大噪,突諸營,諸營驚潰。其軍遂乘亂掠諸營馬騾以去,於是諸營悉奔,獨名武一軍堅壁不動。侵晨,賊大至,督麾下血戰。殺數百人,力竭被執,大罵,為賊磔死。贈特進榮祿大夫、右都督,廕外衛世襲總旗。其子援王來聘、甄奇傑例,乃議贈特進光祿大夫、左都督,世襲錦衣百戶。疏上,踰月而都城陷,不果行。

來聘,京師人。崇禎四年,中武會試。時帝銳意重武,舉子運百斤大刀者止來聘及徐彥琦二人,而彥琦不與選。帝下考官及監試御史獄,悉貶兵部郎二十二人。遣詞臣倪元璐等覆閱,取百人,視文榜例,分三甲傳臚錫宴,以前三十卷進呈,欽定一甲三人,來聘居首,即授副總兵。武榜有狀元,自來聘始也。來聘既拜命,泫然流涕曰:「上重武若此,欲吾儕效命疆場爾,不捐軀殺賊,何以報上恩!」明年,孔有德據登州叛,官軍攻之久不下。又明年二月以火藥轟城,城壞。將士踴入,輒為賊擊退。來聘復先登,中傷而死。天子惜之,贈廕有加。奇傑亦官副總兵,隸楊文岳麾下,從擊賊河南,戰死。

先是,又有鄧祖禹者,蘄水人,舉萬歷四十七年武會試,授沈陽守備。嘗出戰,中矢死,夜半復蘇,創甚告歸。崇禎初,起宣府遊擊,入衛京師。副將申甫軍歿,祖禹力戰盧溝橋,擢涿州參將。疏請召對,不許。入朝上書,聲甚厲,為御史所糾下獄,然帝頗採其言。久之赦出,為辰沅參將,擒苗酋飛天王、張五保,斬首千五百級,夷其巢。擢副總兵,轄德安、黃州。攻賊土壁山,盡掩所獲為己有。當事將劾之,請剿寇自贖。乃令援應城,將七百人入城。賊大至,圍數重。祖禹突圍保西城外,賊復圍之,軍敗被執。賊說降,怒罵不屈。賊言之再三,復罵曰:「若此,須換卻心肝。」賊笑曰:「換不難。」遂剖心剜肝而死。

尤世威,榆林衛人。與兄世功、弟世祿並勇敢知名。天啟中,世威積官建昌營參將,調守牆子路。七年遷山海中部副總兵。寧遠告警,從大帥滿桂赴援,力戰城東有功,增秩受賜。崇禎二年擢總兵官,鎮守居庸、昌平。其冬,京師戒嚴,命提兵五千防順義。俄命還鎮,防護諸陵。四年代宋偉為山海總兵官,積資至左都督。七年命偕寧遠總兵官吳襄馳援宣府。坐擁兵不進,褫職論戍。未行,會流賊躪河南,詔世威充為事官,與副將張外嘉統關門鐵騎五千往剿。

明年正月,賊陷鳳陽。世威以二千五百騎赴之,抵亳州。會總督洪承疇出關討賊,次信陽,命世威趨汝州。甫二日,承疇亦至。時賊見河南兵盛,悉奔入關中。承疇將入關征討,乃大會諸將,令分防汝、雒諸要害。以世威部下皆勁旅,令與參將徐來朝分駐永寧、盧氏山中,以扼雒南蘭草川、朱陽關之險。戒之曰:「靈、陜,賊所出入,汝勿懈!」及承疇既入關,賊避之而南,復由藍田走盧氏。扼於世威,仍入商、雒山中。來朝所部三千人不肯入山,大噪。賊至,來朝逃,一軍盡歿。世威軍暴露久,大疫,與賊戰失利。世威及遊擊劉肇基、羅岱俱負重傷,軍大潰。賊遂越盧氏,走永寧。事聞,命解任侯勘。十年,宣大總督盧象升言:「世威善撫士卒,曉軍機,徒以數千客旅久戍荒山,疾作失利。今當用兵時,棄之可惜。」乃命赴象升軍自效。及象昇戰歿,自免歸。

十五年以廷臣薦,命與弟世祿赴京候調。召對中左門,復告歸。明年十月,李自成陷西安,傳檄榆林招降。總兵官王定懼,率所部精兵棄城走。時巡撫張鳳翼未至,城中士馬單弱,人心洶洶。布政使都任亟集副將惠顯、參將劉廷傑等與里居將帥世威及王世欽、王世國、侯世祿、侯拱極、王學書、故延綏總兵李昌齡議城守。眾推世威為主帥。無何,賊十萬眾陷延安,下綏德,復遣使說降。廷傑大呼曰:「長安雖破,三邊如故。賊皆中州子弟,殺其父兄而驅之戰,必非所願。榆林天下勁兵,一戰奪其氣,然後約寧夏、固原為三師迭進,賊可平也。」眾然其言,乃歃血誓師,簡卒乘,繕甲仗,各出私財佐軍。守具未備,賊已抵城下。

延傑募死士,乞師套部。師將至,賊分兵卻之,攻城甚力。官軍力戰,殺賊無算。賊益眾來攻,起飛樓逼城中,矢石交至,世威等戰益厲。守七晝夜,賊乃穴城,置大炮轟之,城遂破。世威等猶督眾巷戰,婦人豎子亦發屋瓦擊賊,賊屍相枕藉。既而力不支,任死之,侯世祿父子及學書俱不屈死。賊怒廷傑勾套部,磔之,至死罵不絕口。世威、世欽、世國、昌齡並被執,縛至西安。自成坐秦王府欲降之,四人不屈膝。自成曰:「諸公皆名將,助我平天下,取封侯,可乎?」眾罵曰:「汝驛卒,敢大言侮我!」自成笑,前解其縛,世欽唾曰:「驛卒毋近前,汙將軍衣!」自成怒,皆殺之。時顯亦被執,大罵賊。賊惜其勇,繫至神木,服毒死。

王世欽,大將威子,歷山海左部總兵官,謝病去。崇禎八年,洪承疇起之家,擊李自成有功,即謝歸。十六年召對中左門,未及用而歸,遂死於賊。世國,威弟,保定總兵官繼子,由柳溝總兵官罷歸。甫數日,竟拒賊以死。

世威弟世祿,為寧夏總兵官,累著戰功,至是與世威同死。世威從弟翟文為靖邊營副將。嘗從洪承疇敗闖賊於鳳翔之官亭,斬首七百餘級。至是,率敢死士出南門奮擊,殺傷甚眾,中矢死。

又有尤岱者,由步卒起家,至山海鐵騎營參將,數有功。忤上官,棄職歸,守水西門,城陷自殺。

廷傑既死,其父副使彞鼎聞之不哭,曰:「吾有子矣。」其弟廷夔收兄屍,亦自投閣死。

昌齡,字玉川,鎮番衛人。為延綏總兵官,數有功,以剛直罷,徙居榆林。賊至,或勸之去,昌齡曰:「賊至而遁,非勇也。見難而避,非義也。」起偕世威等同守城,卒同死。

侯世祿,榆林人。由世職累官涼州副總兵。遼事亟,詔擢總兵官,提兵赴援。世祿勇敢精悍,為經略熊廷弼所知。及袁應泰代廷弼,亦倚任之。天啟元年,應泰議復撫順、清河。以世祿及姜弼、梁仲善各將兵一萬駐清河。未行,遼陽破,仲善陣亡,世祿、弼俱負重傷,潰圍出。世祿以傷重,命立功自效。尋用為固原總兵官。六年以軍政拾遺罷。明年,寧、錦告警,命率家丁赴關聽調。旋命出守前屯,甫至,令以故官鎮山海。崇禎元年,移鎮宣府。明年冬,京師戒嚴,率師入衛。兵再潰,世祿被創。部卒剽民間,奔還鎮。事聞,當重坐,以勤王先至,減死戍邊。九年八月,京師被兵。率子弟從軍,敘功免戍,還籍。廷臣多推薦,卒不復用。十六年,李自成圍榆林,世祿與子拱極固守新添門。城陷,父子被執,俱不屈死。

拱極歷官參將,常從總兵尤世祿破賊河曲有功。九年冬,任山海總兵官,尋謝病歸。後以廷臣薦,應詔入都,與王洪、王世欽、尤世威召對中左門,未用遣歸。卒與父同死。

劉國能,延安人。始與李自成、張獻忠輩同為盜,自號闖塌天。崇禎三年大亂陜西。已,渡河而東,寇山西,轉掠畿南、河北。六年冬,入河南,遂由內鄉、淅川犯湖廣鄖、襄,破數縣。明年正月入四川,陷夔州。折而東,入鄖陽境,為總督陳奇瑜所蹙。走漢南,同困車箱峽。已得出,復大亂陜西,再入河南,躪江北。官軍逼之,與整齊王屯商、雒山間。九年復偕闖王、蠍子塊等由鄖、襄趨興安、漢中,總督洪承疇奔命不暇。尋南走荊、襄,與總兵秦翼明數戰。其冬,與蠍子塊等十七營窺潼關,巡撫孫傳庭扼之,引而南。明年聞馬光玉等將犯蘄、黃,率眾會之,直趨江北。官軍數道邀擊,乃不敢東。還走黃陂,入木蘭山,轉寇河南,敗參將李春貴兵,將迫開封。詔諸將發兵援,乃南走黃、麻。

當是時,總理熊文燦新至,賊憚之。見其下招降令,頗有欲歸正者。國能先與張獻忠有隙,慮為所并,後又與左良玉戰敗,乃以十一年正月四日率先就撫於隨州,頓首文燦前曰:「愚民陷不義且十載,賴公湔洗更生。願悉眾入軍籍,身隸麾下盡死力。」文燦大喜,慰撫之,署為守備,令隸良玉軍。國能受約束,無異志。已而張獻忠、羅汝才亦降,皆據邑自固。獨國能從軍征剿,數有功。明年二月從良玉勤王。有詔,還討賊,獎勵之。命兵部授官,錄其部下將士,曰:「獻忠能立功,視此。」遂授國能副總兵。四月,良玉會師南陽,擊李萬慶。國能分擊,賊潰奔,遂招萬慶降。其秋,獻忠、汝才並反。文燦遣國能率萬慶兵會討,遂並守鄖陽。既而李自成擾河南,復移守葉縣。

初,國能為盜時,與自成、汝才輩結為兄弟。及國能歸正,自成輩深恨之。十四年九月圍其城,四面力攻,國能不能支,城遂陷,被執。賊猶好謂之曰:「若,我故人也,何不降?」國能瞋目罵曰:「我初與若同為賊,今則王臣也,何故降賊!」遂殺之。事聞,贈左都督,特進榮祿大夫,建祠。

李萬慶,延安人。崇禎初,與張獻忠、羅汝才等並反,賊中所稱射塌天者是也。起陜西,蔓山西、畿南、河北,渡河殘河南,出沒湖廣、四川,還趨鄖陽,入興安,困於車箱峽。出險,益大肆。八年春,賊七十二營會滎陽,議分兵隨所向,令萬慶及許可變助馬進忠、橫天王西當陜兵。已而諸路之賊盡萃於陜,總督洪承疇彌歲不能定,益恣,出沒於河南、湖廣,凡十五家。

迨十一年春,國能、獻忠降,萬慶等大噪而去,改稱十三家,勢頗衰。而文燦擁兵德安,不敢擊,萬慶等復大振。李自成向關中,萬慶及馬光玉、馬進忠、羅汝才、惠登相、賀一龍、藺養成、順天王、順義王九家最著。八月,進忠、光玉大挫於潼關。九月,鄖、襄賊又大敗於雙溝,汝才率九營走均州,萬慶率三營走光、固。十一月,汝才亦降,自成又大敗關內,勢益衰,惟萬慶、光玉、一龍、順天王最勁。而萬慶得馬士秀、杜應金所劫左良玉賄,富且強,營麻城,徙信陽。

十二年正月戰敗,徙應山、德安。會光玉、進忠等皆大敗,進忠懼而降,而順天王已死。一龍、養成伏深山,登相遠掠秦、蜀,萬慶勢益孤。文燦檄良玉擊之唐縣姚梁,分三營肄賊,逐入三山,裨將王修政趨利戰死。文燦收二營卒,令良玉蹙之內鄉。萬慶等在赤眉城四平岡,依山結壘請降。良玉慮其詐,謀之文燦,益調諸將陳永福、羅岱、金聲桓之兵會於賈宋,大剿萬慶及光玉、可變。副將國能亦至,由張家林、七里河分擊,賊大奔。良玉遣國能以二十騎往偵,且諭萬慶降。萬慶馳見,輸情國能,遂執許州叛黨於汝虎以降,處內鄉城下者四千人。士秀、應金見進忠、萬慶降而懼,復來歸。有劉喜才者,夜取順義王首以獻,餘黨推可變為主,與胡可受皆降。自是群盜大衰。至五月,獻忠復叛,汝才率其黨九營應之,復大熾。而萬慶、進忠以徒眾既散,無二心。萬慶願從征自效,比國能給餉。遂授為副總兵,與國能守鄖陽。獻忠等方大亂蜀中,鄖境得無事。

十四年,獻忠突陷襄陽,鄖守如故。明年正月,總督汪喬年討賊,以萬慶從。至襄城,軍潰,入城。賊攻圍之,固守五日。城陷,喬年死,萬慶亦不屈死。事聞,贈都督同知、榮祿大夫,立祠襄城。

贊曰:明至末季,流寇蔓延,國勢坐困,雖有奮威禦敵之臣,而兵孱餉絀,徒使賊乘其敝,潰陷相屬,無救亂亡。如艾萬年等之捐軀盡節,其可悲者矣。此非其勇不具,略不嫻也。兵力耗頓,加以統馭失宜,應援不及,求無敗衄,得乎!


\end{pinyinscope}