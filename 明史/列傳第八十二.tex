\article{列傳第八十二}

\begin{pinyinscope}
喬宇孫交子元林俊子達張黻金獻民秦金孫柱趙璜鄒文盛梁材劉麟蔣瑤王廷相

喬宇,字希大,山西樂平人。祖毅,工部左侍郎。父鳳,職方郎中。皆以清節顯。宇登成化二十年進士,授禮部主事。弘治初,王恕為吏部,調之文選,三遷至郎中。門無私謁。擢太常少卿。武宗嗣位,遣祀中鎮、西海。還朝,條上道中所見軍民困苦六事。已,遷光祿卿,歷戶部左、右侍郎。劉瑾敗,大臣多以黨附見劾,宇獨無所染。拜南京禮部尚書。乾清宮災,率同列言視朝不勤,經筵久輟,國本未建,義子猥多,番僧處禁寺,優伶侍起居,立皇店,留邊兵,習戰鬥,土木繁興,織造不息,凡十事。帝不省。久之,改兵部,參贊機務。以帝遠遊塞上,而監國無人,請早建儲貳。帝將自擊寇,宇復率同列諫。皆不報。

未幾,寧王宸濠反,揚言旦夕下南京。宇嚴為警備,而談笑自如。時攜客燕城外,密察地險易,置戍守。綜理周密,內外宴然。指揮楊銳有才略,署為安慶守備。鎮守中官劉郎與濠通,為預伏死士。宇刺得其情,詰郎用事者,郎懼不敢動。宇乃大索城中,斬所伏壯士三百人,懸首江上。宸濠失內應,且知有備,不敢東。攻安慶,銳固守不得下。未幾敗。

帝至南京,詔百官戎服朝明年正旦。宇不可,率諸臣朝服賀。江彬索城門諸鑰,都督府問宇。宇曰:「守備者,所以謹非常。禁門鎖鑰,孰敢索?亦孰敢予?雖天子詔不可得。」都督府以宇言復,乃已。彬矯旨有所求,日數十至,宇必廷白之,彬亦稍稍止。彬欲譖去宇。守備太監王偉者,初為帝伴讀,帝信之,每從中調護,故彬謀不行。帝駐南京九月,宇倡諸臣三請回鑾,又自伏闕請。駕旋,扈至揚州。明年加太子太保。論保障功,復加少保。

世宗即位,召為吏部尚書。宇自為選郎,有人倫鑒,及是銓政一清。帝求治銳甚。宇與林俊、鼓澤、孫交,皆海內重望,帝亦委任之。凡為權倖所黜者,皆起列庶位,天下欣欣望治。帝性剛,好自用,宇所執漸不見聽。興府需次官六十三人,乞遷敘。宇言此輩虛隸名籍,與見供事者不同。黜罰之有差,皆怨宇。帝欲封駙馬都尉崔元為侯,外戚蔣輪、邵喜為伯,宇不可。無何,詔進壽寧侯張鶴齡為公,封后父陳萬言為伯,授萬言子紹祖尚寶丞。宇言:「累朝太后戚屬無生封公者,張巒亦歿後贈,今奈何以父贈為子封。萬言封伯視巒更驟,而子授尚寶非制。願陛下守典章,以垂萬世。」帝並不從。史道訐楊廷和,宇言道挾私,遂下之詔獄。曹嘉助道劾宇,宇求罷,帝命鴻臚趨視事。

宇遇事不可,無不力爭,而爭「大禮」尤切。帝欲加興獻帝皇號,宇言加皇於本生之親,則干正統,非所以重宗廟,正名分。及禮官請稱獻帝為本生考,帝改稱本生皇考,又詔建獻帝廟於大內,宇等復連章諫。特旨用席書為禮部尚書,宇又偕九卿言:「陛下罷汪俊,用席書;謫馬明衡、季本、陳逅,召張璁、桂萼、霍韜。舉措乖違,人心駭愕。夫以一二人邪說,廢天下萬世公議,內離骨肉,外間君臣,名為效忠,實累聖德。且書不繇廷推,特出內降,此祖宗來所未有。乞令俊與書各仍舊職,宥明衡等,止璁、萼毋召。」尋復請罷璁、萼、書,而出爭「大禮」者呂柟、鄒守益於獄。會璁、萼至京,詔皆用為學士。宇等又言:「內降恩澤,先朝率施於佞倖小人。若士大夫一預其間,即不為清議所齒。況學士最清華,而俾萼等居之,誰復肯與同列哉?」帝怒,切責。宇遂乞休,許之。馳傳給夫廩,猶如故事。御史許中、劉隅等請留宇,帝曰:「朕非不用宇,宇自以疾求去耳。」後《明倫大典》成,追論前議,奪官。楊一清卒,宇渡江弔之。南都父老皆出迎,舉手加額曰:「活我者,公也。」

宇幼從父京師,學於楊一清。成進士後,復從李東陽遊。詩文雄雋,兼通篆籀。性好山水,嘗陟太華絕頂。遇虎,僕夫皆驚仆,宇端坐不動,虎徐帖尾去。家居澹泊,服御若寒士。身歿,二妾劉、許皆從死。穆宗即位,復官,贈少傅,謚莊簡。

孫交,字志同,陸安人。成化十七年進士。授南京兵部主事,為尚書王恕所知。弘治初,怒入吏部,薦授稽勳員外郎,歷文選郎中。居吏部十四年,於善類多所推引。遷太常少卿,提督四夷館。大同有警,命經略黃花鎮諸邊。增垣塹,廣樹藝,制敵騎馳突。永樂時,歲遣隆慶諸衛軍採薪炭。其後罷之,令歲輸銀二萬兩,軍重困。交奏免之。正德初,擢光祿卿。三年進戶部右侍郎,提督倉場,改吏部。尚書張彩附劉瑾,交數規切。彩怒,調之南京。瑾敗,召拜戶部尚書。時征討流寇,調度煩急,仍歲凶,正賦不足,交區畫適宜。四方告饑,輒請蠲租遣振,以故民不至甚敝,而小人用事者皆不便之。帝欲以太平倉賜倖臣裴德,雲南鎮守中官張倫請採銀礦,南京織造中官吳經奏費乏,交皆力爭。八年六月,中旨與禮部尚書傅珪並致仕。言官多請留,不報。

世宗在潛邸知交名,甫即位,召復故官。首請帝日讀《祖訓》,言動悉取準則,經筵日講寒暑勿輟。帝褒納焉。或議遷顯陵天壽山,交言:「山陵事重,太祖欲遷仁祖於鐘山,慮泄靈氣而止,具載《皇陵碑》。」事乃止。武宗侈汰之後,庫藏殫虛。交裁冗食,定經制,宿弊為清。然事涉中官者,帝亦不能盡從也。嘗會廷臣議發內帑給軍廩官俸,已報可,為中官梁諫等所沮。交言:「宮府異同,令出復反,非新政所宜。」不聽。中官監督倉場者,初止數人,正德中增至五十五人。以交言罷撤過半,其後復漸增。帝已罷三十七人,交欲盡去之,並臨清、徐、淮諸倉,一切勿遣。帝令自今毋更加而已。守珠池中官,詔毋得預守土事,而安川夤緣復故。交劾川,命如前詔。正德中,上林苑內臣至九十九人,侵奪公私地無算。帝即位,命留十八人,如弘治時。已復傳奉至六十二人,交乞汰如初,且盡歸侵奪地。報許。又論御馬監內臣宜如祖制,毋監收芻豆,並令戶部通知馬數,杜其侵耗。不從。錦衣百戶張瑾率校尉支俸通倉,橫取狼藉,主事羅洪載欲按之。瑾紿請受杖,奏洪載擅笞禁衛官。帝怒,逮下詔獄謫外。交與林俊、喬宇先後論救,不納。御馬監閻洪乞外豹房地,交言:「先帝以豹房故,貽禍無窮。洪等欲修復以開游獵之端,非臣等所敢聞。」詔以地十頃給豹房,餘令百戶趙愷等佃如故。奉詔上各宮莊田數,視舊籍不同,帝詰其故。交言:「舊籍多以奏請投獻,數多妄報也。新籍少,以奉命清核,田多除豁也。」帝意稍解,令考成、弘間籍以聞。

交年已七十,連章乞罷。帝輒慰留,遣醫視療。請益力,乃許之。手詔加太子太保,馳驛。令子編修元侍行,有司時存問,給食米輿隸,復賜道里費。卒年八十,謚榮僖。

交言論恂恂,不以勢位驕人。清慎恬愨,終始一致。初在南京,僚友以事簡多暇,相率談諧飲弈為樂,交默處一室,讀書不輟。或以為言,交曰:「對聖賢語,不愈於賓客、妻妾乎!」興獻王素愛重交,嘗割陽春臺東偏地益其宅。後中官言孫尚書侵地,世宗曰:「此先皇所賜,吾敢奪耶?」

元,進士,終四川副使。謹厚有父風。

林俊,字待用,莆田人。成化十四年進士。除刑部主事,進員外郎。性侃直,不隨俗浮湛。事涉權貴,尚書林聰輒屬俊治之。上疏請斬妖僧繼曉並罪中貴梁芳,帝大怒,下詔獄考訊。後府經歷張黻救之,並下獄。太監懷恩力救,俊得謫姚州判官,黻師宗知州。時言路久塞,兩人直聲震都下,為之語曰:「御史在刑曹,黃門出後府。」尋以正月朔星變,帝感悟,復俊官,改南京。弘治元年用薦擢雲南副使。鶴慶玄化寺稱有活佛,歲時集士女萬人,爭以金塗其面。俊命焚之,得金悉以償民逋。又毀淫祠三百六十區,皆撤其材修學宮。乾崖土舍刀怕愈欲奪從子宣撫官,劫其印數年。俊檄諭之,遂歸印。進按察使。五年調湖廣。以雨雪災異上疏陳時政得失。又言德安、安陸建王府及增修吉府,工役浩繁,財費巨萬,民不堪命。乞循寧、襄、德府故事,一切省儉,勿用琉璃及白石雕闌,請著為例。不從。九年引疾,不待報徑歸。

久之,薦起廣東右布政使,不拜。起南京右僉都御史,督操江。十四年正月朔,陜西、山西地震水涌。疏述古宮闈、外戚、內侍、柄臣之禍;乞罷齋醮,減織造,清役占,汰冗員,止工作,省供應,應賞賜,戒逸欲,遠佞幸,親賢人。又請豫教皇儲,恩薦侍郎謝鐸,少卿儲瓘、楊廉,致仕副使曹時中,處士劉閔堪輔導。報聞。已,屢疏乞休,薦時中自代。不許。江西新昌民王武為盜,巡撫韓邦問不能靖,命俊巡視。身入武巢,武請自效,悉擒賊黨。詔即以俊代邦問,俊引朱熹代唐仲友、包拯代宋祁事,力辭。不允。乃更定要約,庶務一新。王府征歲祿,率倍取於民,以俊言大減省。寧王宸濠貪暴,俊屢裁抑之。王請易琉璃瓦,費二萬。俊言宜如舊,毋涉叔段京鄙之求,吳王幾杖之賜。王怒,伺其過,無所得。會俊以聖節按部,遂劾奏之,停俸三月。尋以母憂歸。

武宗即位,言官交薦,江西人在朝者合疏乞還俊。乃進右副都御史,再撫江西,遭父憂不果。正德四年起撫四川。眉州人劉烈倡亂,敗而逃,諸不逞假其名剽掠。俊繪形捕,莫能得。會保寧賊藍廷瑞、鄢本恕、廖惠等繼起,勢益張,轉寇巴州。猝遇之華壟,單輿抵其營,譬曉利害,賊羅拜約降。淫雨失期,復叛去,攻陷通江。俊擊敗之龍灘河,遣知府張敏等追敗之門鎮子,遂擒廖惠。而廷瑞奔陜西西鄉,越漢中三十六盤,至大巴山。官軍追及,復大破之。遂移師擊瀘州賊曹甫,且遣人招諭。甫佯聽令,使弟琯劫如故。指揮李蔭斬琯首,賊遂移江津。分七營,將攻重慶。俊發酉陽、播州土兵助蔭,以元日掩破其四營。賊遁入民家,焚之盡斃。乘勝搗老營,指揮汪洋等中伏死。蔭復進,去賊十五里。甫以數十騎出,遇蔭兵,敗走。官軍乘勝進圍之,俘及焚死者二千有奇。已,本恕、廷瑞為永順土舍彭世麟所擒。俊論功進右都御史。甫黨方四亡命思南,復攻南川、綦江,以窺瀘州。俊益發士兵,令副使何珊、李鉞等敗之去。捷聞,璽書獎勵。俊在軍,與總督洪鐘議多左。中貴子弟欲冒從軍功,輒禁止。御史俞緇走避賊,而僉事吳景戰歿。緇慚,欲委罪俊,遂劾俊累報首功,賊終不滅;加鑿井毀寺,逐僧徒,迫為賊。於是俊前後被切責。比方四敗,賊且盡,俊辭加秩及賞,乞以舊職歸田。詔不許辭秩,聽其致仕。言官交請留,不報。俊歸,士民號哭追送。時正德六年十一月也。

世宗即位,起工部尚書,改刑部。在道數引疾,不許。因請帝親近儒臣,正其心以出號令,用渾樸為天下先。初詔所革,無遷就以廢公議。既抵京師,會暑月經筵輟講,舉祖宗勤學故事以諫。俊時年已七十,寓止朝房,示無久居意。數為帝言親大臣,勤聖學,辨異端,節財用。朝有大政,必侃侃陳論,中外想望其風採。中官葛景等奸利事覺,為言官所糾,詔下司禮監察訊。俊言內臣犯法,法司不得訊,是宮府異體也。乞下法司公訊,以詔平明之治。都督劉暉下獄,俊當以交結朋黨律,言與許泰同罪,請斬以謝天下。廖鵬、廖鎧、齊佐、王瓛論死,屢詔緩刑,俊乞亟行誅。又劾谷大用占民田萬餘頃。皆不聽。中官崔文家人李陽鳳索匠師宋鈺賄不獲,嗾文杖之幾死,下刑部治未決,而中旨移鎮撫司。俊留不遣,力爭不納。明日又奏,帝怒責陳狀。俊言:「祖宗以刑獄付法司,以緝獲奸盜付鎮撫。訊鞫既得,猶必付法司擬罪。未有奪取未定之囚,反付推問者。文先朝漏奸,罪不容誅,茲復干內降。臣不忍朝廷百五十年紀綱,為此輩壞亂。」帝憚其言直,乃不問。

俊以耆德起田間,持正不避嫌,既屢見格,遂乞致仕。詔加太子太保,給驛賜隸廩如制。

俊數爭「大禮」,與楊廷和合。嘗上言推尊所生有不容已之情,有不可易之禮,因輯堯、舜至宋理宗事凡十條,以上。及「大禮」議定,得罪者或杖死。四年秋,俊從病中上書言:「古者鞭撲之刑,辱之而已,非欲糜爛其體膚而致之死也,又非所以加於士大夫也。成化時,臣及見廷杖二三臣,率容厚棉底衣,重氈疊裹,然且沉臥,久乃得痊。正德朝,逆瑾竊權,始令去衣,致末年多杖死。臣又見成化、弘治時,惟叛逆、妖言、劫盜下詔獄,始命打問。他犯但言送問而已。今一概打問,亦非故事。自去歲舊臣斥逐殆盡,朝署為空。乞聖明留念,既去者禮致,未去者慰留。碩德重望如羅欽順、王守仁、呂柟、魯鐸輩,宜列置左右。臣衰病待盡,無復他望,敢效古人遺表之意,敬布犬馬之心。」帝但下所司而已。又明年,疾革,復上書請懋學隆孝,任賢納諫,保躬導和,且預辭身後恤典,遂卒。年七十六。

後一年,《明倫大典》成,追論俊附和廷和,削其官,其子達以士禮葬之。

俊歷事四朝,抗辭敢諫,以禮進退,始終一節。隆慶初,復官,贈少保,謚貞肅。

達,正德九年進士。官至南京吏部郎中。工篆籀,能古文。

張黻,吉水人。成化八年進士。歷知涪州、宿州,介特不避權貴。弘治中,俊蒙顯擢,而黻老不用。王恕為之請,特予誥命。

金獻民,字舜舉,綿州人。成化二十年進士。除行人。弘治初,選授御史,按雲南、順天,並著風裁。出為天津副使,歷湖廣按察使。正德初,劉瑾亂政,追坐獻民勘天津地不實,與巡撫柳應辰等械繫詔獄,斥為民。未幾,又坐湖廣事,再下獄,罰贖歸。踰年,又以瀏陽民劉道隆獄讞不實,罰米輸塞下。瑾誅,起貴州按察使。擢僉都御史,巡撫延綏,歷南京刑部尚書。

世宗即位,召為左都御史。李鳳陽下刑部,程貴下都察院,皆改詔獄,獻民力爭。已,遷刑部尚書。執奏奸黨王欽、王銓不宜貸死。皆不納。尋代彭澤為兵部尚書。五星聚營室,其占主兵。獻民因請敕天下鎮巡官預守戰之備,且請用賢納諫,罷土木,屏玩好。帝頗采納。獻民性伉直,有執持,帝或不能從,卒無所徇。帝初即位,盡斥先朝傳奉官。已,太監邱福、潘傑等死,詔官其弟侄錦衣。及司禮太監張欽死,以家人李賢承廕,賢死復欲官其子儒。獻民先後執奏,帝皆不從。土魯番速檀滿速兒寇肅州,命獻民兼右都御史總制陜西四鎮軍務。比至蘭州,巡撫陳九疇已破敵,獻民再以捷聞。還京,仍理部事。論功,廕錦衣世百戶。

錦衣百戶俞賢,中官泰養子也,以中旨管事,諫官爭之。獻民言:「祖宗有舊制,孝廟有禁例,陛下登極有明詔。賢無公家庸,又非泰子姓,猥以廝養竊名器,紊棨典章,不可之大者。宜納諫官言。」弗聽。錦衣副千戶李全、王邦奇等以冒濫汰去,至是奏辨不已,下部覆議。獻民言:「全等足不履行陣而坐論首功,身不隸公家而躐躋顯秩。陛下登極,汰去者三百餘人,人心稱快。萬一倖端再啟,則前詔皆虛,將來奏擾,有何紀極。」帝竟授全等試百戶。獻民復奏曰:「令出惟行勿惟反。今以小人奏辨,一旦復官九十餘人,徇左右私,壞祖宗法,竊為陛下惜之。明旨不許夤緣管事,而奔競已成風矣;不許比例陳乞,而奏擾已踵至矣。誰生厲階,至今為梗。望仍斥全等,以息人言,消天變。」言官任洛等亦以為言,不聽。

會寧夏總兵官種勛行賂京師,偵事者獲其籍,獻民名在焉。給事蔡經、御史高世魁等交章劾之,獻民因引疾歸。居二年,邦奇訐前尚書彭澤,詞連獻民,逮下刑部獄。法司劾獻民奉命專征,未至其地,掠功妄報。失大臣體,宜奪職閒住,削其世廕。詔可。

初,「大禮」議起,獻民數偕廷臣疏爭。及左順門哭諫,又與徐文華倡之。帝由此不悅,卒得罪。隆慶初,贈恤如制。

秦金,字國聲,無錫人。弘治六年進士。授戶部主事,歷郎中。正德初,遷河南提學副使,改右參政。守開封,破趙鐩於陳橋。歷山東左、右布政使。承寇躪後,與巡撫趙璜共拊循,瘡痍始起。九年擢右副都御史,巡撫湖廣。諸王府所據山場湖蕩,皆奏還之官。降盜賀璋、羅大洪復叛,討平之。郴州桂陽瑤龔福全稱王,金先後破寨八十餘,斬首二千級,擒福全及其黨劉福興等。錄功,增俸一級,蔭錦衣世百戶,力辭得請。入為戶部右侍郎。

世宗即位,改吏部。言官論金無人倫鑒,復改戶部,轉左,署部事。外戚邵喜乞莊田,金述祖制,請按治。帝宥喜,命都察院禁如制。中旨各宮仍置皇莊,遣官校分督。金言:「西漢盛時以苑囿賦貧民,今奈何剝民以益上。乞勘正德間額外侵占者,悉歸其主,而盡撤管莊之人。」帝稱善,即從其議。

嘉靖二年擢南京禮部尚書,率諸臣上疏曰:「陛下繼統以來,昭德塞違,勵精圖治,動無過舉,宜召天和,而災眚頻告者,何也?《詩》曰:『靡不有初,鮮克有終。』陛下登極一詔,百度咸貞,天下拭目望至治。比來多與詔違,百司罔遵,萬民失仰,此詔令不能如初也。即位之初,逐庸回,任耆舊。比內閣擬旨輒中改,至疏請,徒答溫語,此任賢不能如初也。即位之初,聽言如流,朝請暮報。比來事涉戚畹、宦寺,雖九卿執奏,科道交章,皆曰『業經有旨』。此聽納不能如初也。即位之初,凡先朝傳升、乞升等官,一切釐革。比來恩澤過濫,封拜頻煩。此慎名器不能如初也。即位之初,凡奸黨巨惡俱付三法司。比來輒下鎮撫。此謹國法不能如初也。即位之初,首命戶部減馬房糧芻之半,且令科道官備核馬數。乃因太監閻洪等言,遂寢前詔。此恤民瘼不能如初也。即位之初,遣斥法王、佛子、國師、禪師。比來於禁地設齋醮。此崇正道不能如初也。即位之初,精明充盛。比來聖躬弗豫,天顏未復。此嗇精神不能如初也。夫初政所以清明者,政出公朝,而左右不預也;今政所以混溷者,政在左右,而外廷不知也。惟政不可一日不在朝廷,惟權不可一日移於左右。所謂政在朝廷者,非必皆獨運也。股肱有託,耳目有寄,即主威重於九鼎,國勢安於泰山。自古帝王制御天下,操此術而已。不則宮府之勢隔而信任有所偏,婦寺之情親而聽受有所蔽。名曰總攬,而太阿之鐏實移於下矣。」章下禮部,尚書汪俊力勸帝採納,報聞。

尋就改兵部。孫交去,召為戶部尚書。帝欲考興獻帝,金偕廷臣伏闕爭,又與何孟春等條張璁建議之非。及上聖母冊,金及趙璜等復不至,帝頻詰讓。

金為人樂易。及居官,一以廉正自持。在戶部,尤孜孜為國。永福長公主乞寶坻、武清地,以金言頗減。撫寧、山海莊地賜魏國公徐達者,達卒仍歸之官,定國公光祚請之,金執不可。給事中黃重、御史張珩等先後爭,金等復以為言,始報許。內府諸監局軍匠至數千人,中官梁諫請下部採金玉珠石,金皆執奏。不聽。奸人逯俊等乞兩淮鹽引三十萬,帝許之。金力爭不可,積失帝旨。

六年春以考察自陳致仕,馳驛給夫廩如制。歸五年,薦者不已,乃起南京戶部,疏陳利民六事。尋召為工部尚書,加太子少保。帝與張孚敬、李時評諸大臣,以金為賢,頗嫌其老。居數月,加太子太保,改南京兵部。踰歲致仕歸。二十三年卒,年七十八。贈少保,謚端敏。

孫柱,以諸生授中書舍人。大學士高拱得罪,倉黃去京師,門生皆避匿,柱獨追送百里外。吳中行疏論張居正奪情,被杖下詔獄。柱挾醫視湯藥,遂忤居正,遷魯府審理。尋假考察罷之。

趙璜,字廷實,安福人。少從父之官,墜江中不死。稍長,行道上,得遺金,悉還其主。登弘治三年進士,授工部主事。改兵部,歷員外郎。出為濟南知府。猾吏舞文,積歲為蠹。璜擇愿民教之律令,得通習者二十餘人,逐吏而代之。漢庶人牧場久籍於官,募民佃。德王府奏乞之,璜勘還之民。閱七年,政績大著。正德初,擢順天府丞,未上,劉瑾惡璜,坐巡撫朱欽事,逮下詔獄,除名。瑾誅,復職。遷右僉都御史,巡撫宣府。尋調山東。河灘地數百里,賦流民墾而除其租。番僧乞征以充齋糧,帝許之,璜力爭得免。曲阜為賊破,闕里林廟在曠野,璜請移縣就闕里,從之。擢工部右侍郎,總理河道。以邊警改理畿輔戎備。事定,命振順天諸府譏,還佐部事。

世宗即位,進左侍郎,掌部事。裁宦官賜葬費及御用監料價,革內府酒醋面局歲征鐵磚價銀歲巨萬。嘉靖元年進尚書。劉瑾創玄明宮,糜財數十萬,瑾死,奸人獻為皇莊。帝即位,斥以予民,既而中旨令仍舊。璜言,詔下數月而忽更,示天下不信。帝即報許。會方修仁壽、清寧宮,費不繼。璜因請與石景山諸房舍並斥賣以資用,可無累民,帝可之。給事中徐景嵩等謂,詔書許還民,官不當自鬻,劾璜。璜疏辨,並發景嵩他事。御史張鵬翰言璜摭言官,無大臣誼。帝責鵬翰黨庇景嵩,竟斥。其同官陳江亦以劾璜被責,求去。給事中章僑言璜一舉逐兩諫官,甚損國體。尚書彭澤復奏僑非是,僑再辨,帝兩解之。詔營后父陳萬言第,估工值六十萬,璜持之。萬言愬於帝,下郎中、員外二人詔獄。璜言:「二臣無與,乞罪臣。」帝不聽。其後論救踵至,萬言不自安,再請貸。二人獲釋,工價亦大減。

三年,顯陵司香內官言陵制陜小,請改營,視天壽山諸陵。璜言陵制與山水相稱,難概同,帝納其言。已,帝欲遷顯陵,璜不可,乃寢。詔建玉德殿,景福、安喜二宮,璜請俟仁壽宮成,徐議其事,帝不許。頃之,以災異申前請,帝始從之,并罷仁壽役。江西建真人府,陜西督織造,皆遣中使,璜皆疏爭。營建世廟,中官所派物料,戶部多裁省。帝以問璜,璜言曩造乾清、坤寧兩宮所積餘資,足移用,帝遂報可。

璜為尚書六年,值帝初政,銳意釐剔,中官不敢撓,故得舉其職。後論執不已,諸權倖嫉者眾,帝意亦浸疏。璜素與秦金齊名。考察自陳,與金俱致仕。廷臣乞留,不許,馳驛給夫廩如故事。

璜有幹局,多智慮。事棼錯,他人相顧愕眙,璜立辦。既去,人爭薦之。十一年召復故官,未上卒。贈太子太保,謚莊靖。

鄒文盛,字時鳴,公安人。弘治六年進士。除吏科給事中。遼東巡撫韓重劾鎮守中官廖,文盛偕郎中楊茂仁勘實其罪,謫長陵司香。雜顏三衛屢擾邊,文盛還奏制馭六策。尚書劉大夏深善之,下之邊吏。尋出核兩廣糧儲。思恩土官岑濬與田州岑猛構兵,文盛言:「田州廣西之籓蔽,李蠻田州之干城,參政武清受濬重賂,以計殺蠻釀成禍亂。制敕房供事參議岑業,濬懿親,為彌縫於中,漏我機事。請先誅二人,而後行討。」業有內援,帝不聽。清尋以考察罷。

正德初,歷戶科都給事中,出為保定知府,累遷福建左布政使。十一年以右副都御史巡撫貴州。清平苗阿旁、阿階、阿革稱王,巡撫曹祥調永順、保靖土兵討之,尋被劾罷。阿旁等據香爐山,興隆、偏橋、平越、新添、龍里諸衛咸被其患。文盛至,檄川、湖兵協剿,以貴州兵搗炮木寨,擒阿革。川、湖兵至,抵山下。山壁立,惟小徑五,賊皆樹柵。仰攻不能克,乃製戰樓與崖齊,乘夜雨附崖登,拔柵焚廬舍。賊奔後山,據絕頂。官軍乘間梯滕木以上,遂擒阿旁,餘賊盡平。移師討平龍頭、都黎、都蘭、都蓬、密西、大支、馬羅諸寨黑苗,先後斬降無算。錄功,增俸一等,廕子錦衣世百戶。力辭免。芒部陳聰等為亂,討破之。四川土舍重安馮綸與凱里楊弘有怨。弘卒,綸糾諸苗相仇殺,侵軼貴州境。文盛遣參議蔡潮詣播州,督宣慰楊斌撫定之。請復設安寧宣撫司,以弘子襲,而錄潮功。尚書王瓊以專擅為潮罪,不敘。頃之,改蒞南京都察院。

世宗即位,召為戶部左、右侍郎,遷南京右都御史,就改戶部尚書。嘉靖六年,戶部尚書秦金罷,召文盛代之。首疏鹽政、錢法十一事。文盛為人廉謹,踆々若無能。與孫交、秦金、趙璜咸稱長者。歲餘,以年至,再疏乞歸。卒贈太子少保,謚莊簡。

梁材,字大用,南京金吾右衛人。弘治十二年進士。授德清知縣,勤敏有異政。正德初,遷刑部主事,改御史。出為嘉興知府,調杭州。田租例參差,材為酌輕重,立畫一之法。遷浙江右參政,進按察使。鎮守中官畢真與宸濠通,將舉城應之。材與巡按張縉劫持真,奪其兵衛。尋以憂去。嘉靖初,起補雲南。土官相仇殺累年,材召其酋曰:「汝罪當死。今貰汝,以牛羊贖。」御史訝其輕,材曰:「如是足矣,急之變生。」諸酋衷甲待變,聞無他乃止。歷貴州、廣東左、右布政使。吏民輸課,令自操權衡,吏不得預。時天下布政使廉名最著者二人,材與姚鏌也。六年拜右副都御史,巡撫江西。甫兩月,召為刑部左侍郎。

尋改戶部,遂代鄒文盛為尚書。自外僚登六卿,不滿二載。自以受恩深,益盡職。上言:「臣考去年所入止百三十萬兩,而所出至二百四十萬。加催征不前,邊費無節,凶荒又多奏免,國計安所辦?詳求弊端:一宗籓,二武職,三冗食,四冗費,五逋負。乞集廷臣計畫條請。」於是宗籓、武職各議上三事,其他皆嚴為節。帝悉報可。惟武職閒住者議停半俸,帝不納。經費大省,國用亦充。中官麥福請盡徵牧馬草場租,材不可。侍郎王軏清勳戚莊田,言宜量等級為限。材奏:「成周班祿有土田,祿由田出,非常祿外復有土田。今勳戚祿已踰分,而陳乞動千萬,請申禁之。自特賜外,量存三之一,以供祀事。」帝命並清已賜者,額外侵據悉還之民,勢豪家乃不敢妄請乞。畿輔屯田,御史督理,正統間易以僉事,權輕,屯政日弛。材請仍用御史。御史郭弘化言天下土田視國初減半,宜通行清丈。材恐紛擾,請但敕所司清釐,籍難稽者始履畝而丈。帝悉可之。母喪去。服除,起故官。大同巡撫樊繼祖請益軍餉,材言:「大同歲餉七十七萬有奇,例外解發又累萬,較昔已數倍。日益月增,太倉銀不足供一鎮,無論九邊也。」繼祖數請不得,議開事例,下戶、兵二部行之。時修建兩宮、七陵,役京軍七萬,郭勛請給月糧冬衣。材言非故事,如所請,當歲費銀四十五萬;且冬衣例取內庫,非部事。勛怒,劾材誤公。帝詰責材,竟如勛奏。勛復建言三事:請開礦助工,餘鹽盡輸邊,漕卒得攜貨物。材議,不盡行,勛益怒。

材初為戶部,值帝勤政,力祛宿弊,多見從。及是屢忤權倖,不得志,乃乞改南。為給事中周珫所劾,下吏部,尚書許贊等請留之。帝不悅,令與材俱對狀。材引罪得宥,而讚等坐奪俸。材由此失帝意。考尚書六年滿,遂令致仕。初,徽王守莊者與佃人訟,材請革守莊者,令有司納租於王,報可。王奏不便,帝又從之。材已去,侍郎唐胄等執初詔。帝大怒,并責材。令以右侍郎閒住,而奪胄俸,下郎官詔獄。

明年,戶部尚書李廷相罷。帝念材廉勤,大臣亦多薦者,乃召復故官,加太子少保。三掌國計,砥節守公如一日,帝眷亦甚厚。其秋,考察京官,特命監之。有大獄不能決,又命兼掌刑部事。帝歎曰:「尚書得如材者十二人,吾無憂天下矣。」大工頻興,役外衛班軍四萬六千人。郭勛籍其不至得,責輸銀雇役,廩食視班軍。廷相嘗量給之,材堅持不予。勛劾材,帝命補給。勛又以軍不足,籍逃亡軍布棉折餉銀募工。材言:「今京班軍四萬餘,已足用,不宜借口耗國儲。」帝從其奏。勛益怒,劾材變亂舊章。無是,醮壇須龍涎香,材不以時進,帝銜之。遂責材沽名誤事,落職閒住。歸,旋卒,年七十一。隆慶初,贈太子太保,謚端肅。

當嘉靖中歲,大臣或阿上取寵,材獨不撓,以是終不容。自材去,邊儲、國用大窘。世宗乃歎曰:「材在,當不至此。」

劉麟,字元瑞,本安仁人。世為南京廣洋衛副千戶,因家焉。績學能文,與顧璘、徐禎卿稱「江東三才子」。弘治九年成進士。言官龐泮等下獄,麟偕同年生陸崑抗疏救。除刑部主事,進員外郎。錄囚畿內,平反三百九十餘人。正德初,進郎中,出為紹興府知府。劉瑾銜麟不謁謝,甫五月,摭前錄囚細故,罷為民。士民醵金贐不受,為建小劉祠以配漢劉寵,因寓湖州。與吳琬、施侃、孫一元、龍霓為「湖南五隱」。瑾誅,起補西安。遭父憂,樂吳興山水,奉父柩葬焉,遂居湖州。起陜西左參政,督糧儲。都御史鄧璋督師,議加賦充餉,麟力爭。會陜民詣闕愬,得寢。尋遷雲南按察使,謝病歸。

嘉靖初,召拜太僕卿。進右副都御史,巡撫保定六府。中官耿忠守備紫荊多縱,麟劾奏之。請捐天津三衛屯田課,及出庫儲給河間三衛軍月餉,徵逋課以償,皆報可。帝因諭戶部,中外軍餉未給者,悉補給之。再引疾歸。起大理卿,拜工部尚書。侍衛軍不給衣履,錦衣帥駱安援紅盔軍例以請,麟執不可。詔量給銀自製,後五載一給為常。四司財物悉貯後堂大庫,司官出納多侵漁,麟請特除一郎官主之。帝稱善,因賜名「節慎庫」。已,上節財十四事,汰內府諸監局冒破錢,中貴大恨。及顯陵工竣,執役者咸覬官。麟止擬賚,群小愈怨。會帝納諫官言,停中外雜派工役,麟牒停浙江、蘇、松織造,而上供袍服在停中。中官吳勛以為言,遂勒麟致仕。久之,顯陵殿閣雨漏,追論麟,落職。

麟清修直節,當官不撓。居工部,為朝廷惜財謹費,僅踰年而罷。居郊外南坦,賦詩自娛。守為築一臺,令為構堂,始有息游之所。家居三十餘年,廷臣頻論薦。晚好樓居,力不能構,懸籃輿於梁,曲臥其中,名曰神樓。文徵明繪圖遺之。年八十七卒。贈太子少保,謚清惠。

蔣瑤,字粹卿,歸安人。弘治十二年進士。授行人。正德時,歷兩京御史。陳時弊七事,中言:「內府軍器局軍匠六千,中官監督者二人,今增至六十餘人,人占軍匠三十。他局稱是,行伍安得不耗。」并言:「傳奉官及濫收校尉勇士並宜釐革。劉瑾雖誅,權猶在宦豎。」有旨詰問,且言「自今如瑤議者,毋復奏。」尋出為荊州知府。築黃潭隄。

調揚州。武宗南巡至揚,瑤供御取具而已,無所贈遺。諸嬖倖皆怒。江彬欲奪富民居為威武副將軍府,瑤執不可。彬閉瑤空舍挫辱之,脅以帝所賜銅瓜,不為懾。會帝漁獲一巨魚,戲言直五百金,彬即畀瑤責其直。瑤懷其妻簪珥、袿服以進,曰:「庫無錢,臣所有惟此。」帝笑而遣之。府故有瓊花觀,詔取瓊花。瑤言自宋徽、欽北狩,此花已絕,今無以獻。又傳旨徵異物,瑤具對非揚產。帝曰:「苧白布,亦非揚產耶?」瑤不得已,為獻五百疋。當是時,權倖以揚繁華,要求無所不至。微瑤,民且重困。駕旋,瑤扈至寶應。中官邱得用鐵糸亙繫瑤,數日始釋,竟扈至臨清而返。揚人見瑤,無不感泣。迨遷陜西參政,爭出資建祠祀之,名自此大震。

嘉靖初,歷湖廣、江西左、右布政使,以右副都御史巡撫河南。帝命桂萼等核巡撫官去留,令瑤歸候調。已,累遷工部尚書。四郊工竣,加太子少保。西苑宮殿成,帝置宴。見瑤與王時中席在外,命移殿內,而移皇親於殿右以讓瑤,曰:「親親不如尊賢。」其重瑤如此。時土木繁興,歲費數百萬計。瑤規畫咸稱帝意,數有賚予。以憂去。久之,自南京工部尚書,召改北部。帝幸承天,瑤扈從。京師營建,率役京軍,多為豪家占匿。至是大工頻仍,歲募民充役,費二百餘萬。瑤以為言,因請停不急者。豪家所匿軍畢出,募直大減。以老致仕去。

瑤端亮清介。既歸,僻處陋巷。與尚書劉麟、顧應祥輩結文酒社,徜徉峴山間。卒年八十九。贈太子太保,謚恭靖。

王廷相,字子衡,儀封人。幼有文名。登弘治十五年進士,選庶吉士,授兵科給事中。以憂去。正德初,服闋至京。劉瑾中以罪謫亳州判官,量移高淳知縣。召為御史,疏言:「大盜四起,將帥未能平。由將權輕,不能禦敵;兵機疏,不能扼險也。盜賊所至,鄉民奉牛酒,甚者為效力。盜有生殺權,而將帥反無之,故兵不用命。宜假便宜,退卻者必斬。河南地平曠,賊易奔,山西地險阻,亦縱深入,將帥罪也。若陳兵黃河之津,使不得西,分扼井陘、天井,使不得東,而主將以大軍蹙之,則賊進退皆窮,可不戰擒矣。」帝切責總督諸臣,悉從其議。已,出按陜西,裁抑鎮守中官廖堂,被誣。時已改督京畿學校,逮繫詔獄,謫贛榆丞。屢遷四川僉事,山東副使,皆提督學校。嘉靖二年舉治行卓異,再遷山東右布政使。以右副都御史巡撫四川,討平芒部賊沙保。

尋召理院事。歷兵部左、右侍郎,遷南京兵部尚書,參贊機務。初有詔,省進貢快船。守備太監賴義復求增,廷相請酌物輕重以定船數,而大減宣德以後傳旨非祖制者。龍江、大勝、新江、浦子、江淮五關守臣借稽察榷利,安慶、九江借春秋閱視索賂,廷相皆請革之。草場、蘆課銀率為中官楊奇、卜春及魏國公徐鵬舉所侵蝕。以廷相請,逮問奇、春,奪鵬舉祿。三月入為左都御史,疏言南京守備權太重,不宜令魏國世官。給事中曾忭亦言之,遂解鵬舉兵柄。

居二年,加兵部尚書兼前官,提督團營,仍理院事。兩考滿,加太子少保。畿民盜天壽山陵樹,巡按楊紹芳引盜大祀神御物,律斬。廷相言:「大祀神御物者,指神御在內祭器帷帳之物而言。律文,盜陵木者,止杖一百,徒三年。今舍本律,非刑之平。」忤旨,罰俸一月。帝將幸承天,廷相與諸大臣諫,不納。扈從還,以九年滿,加太子太保。雷震奉先殿,廷相言:「人事修而後天道順,大臣法而後小臣廉。今廉隅不立,賄賂盛行,先朝猶暮夜之私,而今則白日之攫。大臣污則小臣悉傚,京官貪則外臣無畏。臣職憲紀,不能絕其弊,乞先罷斥。」用以刺尚書嚴嵩、張瓚輩。帝但諭留而已。

初,廷相請以六條考察差還御史。帝令疏其所未盡,編之憲綱。乃取張孚敬、汪鋐所奏列,及新所定凡十五事以進,悉允行之。及九廟災,下詔修省,因敕廷相曰:「御史巡方職甚重。卿總憲有年,自定六條後,不考黜一人,今宜痛修省。」廷相惶恐謝。

廷相掌內臺最久,有威重。督團營,與郭勳共事,逡巡其間,不能有所振飭。給事中李鳳來等論權貴奪民利,章下都察院,廷相檄五城御史核實,遲四十餘日。給事中章允賢遂劾廷相徇私慢上。帝方詰責,而廷相以御史所核聞,惟郭勛侵最多。帝令勛自奏,於是劾勛者群起。勛復以領敕稽留觸帝怒,下獄。責廷相朋比阿黨,斥為民。越三年卒。廷相博學好議論,以經術稱。於星歷、輿圖、樂律、河圖、洛書及周、邵、程、張之書,皆有所論駁,然其說頗乖僻。隆慶初,復官,贈少保,謚肅敏。

贊曰:喬宇守南京,從容鎮靜,內嚴警備,可謂能當大事者矣。觀宇與孫交等砥節奉公,懇懇廷諍,意在杜塞倖門,裨益國是。雖得君行政,未能媲美蹇、夏,要其清嚴不茍,行無瑕尤,於前人亦不多讓。蔣瑤為尚書,功名損於治郡,王廷相掌內臺,風力未著,是殆其時為之歟。


\end{pinyinscope}