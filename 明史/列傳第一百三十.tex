\article{列傳第一百三十}

\begin{pinyinscope}
陳邦瞻畢懋康兄懋良蕭近高白瑜程紹翟鳳翀郭尚賓洪文衡何喬遠陳伯友李成名董應舉林材朱吾弼林秉漢張光前

陳邦瞻,字德遠,高安人。萬曆二十六年進士。授南京大理寺評事。歷南京吏部郎中,出為浙江參政。進福建按察使,遷右布政使。改補河南,分理彰德諸府。開水田千頃,建滏陽書院,集諸生講習。士民祠祀之。就改左布政使,以右副都御史巡撫陜西。

上林土官黃德勛弟德隆及子祚允叛德勛,投田州土酋岑茂仁。茂仁納之,襲破上林,殺德勛,掠妻子金帛。守臣問狀,詭言德勛病亡,乞以祚允繼。邦瞻請討於朝。會光宗嗣位,即擢邦瞻兵部右侍郎,總督兩廣軍務兼巡撫廣東,遂移師討擒之。海寇林莘老嘯聚萬餘人侵掠海濱,邦瞻扼之,不得逞。澳夷築室青州,奸民與通,時侵內地,邦瞻燔其巢。召拜工部右侍郎。未上,改兵部,進左。

天啟二年五月疏陳四事,中言:「客氏既出復入,乃陛下過舉。輔臣不封還內降,引義固爭,致罪謫言者,再蹈拒諫之失,其何解於人言?」疏入,忤旨譙讓。尋兼戶、工二部侍郎,專理軍需。明年卒官,詔贈尚書。

邦瞻好學,敦風節。服官三十年,吏議不及。

畢懋康,字孟侯,歙人。萬曆二十六年進士。以中書舍人授御史。言內閣不當專用詞臣,邊臣失律者宜重按,部郎田大年、賀盛瑞,中書舍人丁元薦以忤權要廢,當雪,疏留中。視鹽長蘆。

畿輔多河渠,湮廢不治。懋康言:「保定清河,其源發於滿城,抵清苑而南十里,則湯家口為上閘,又十里則清楊為下閘,順流東下,直抵天津。旁近易、安諸州,新安、雄、完、唐、慶都諸縣,並通舟楫仰其利。二閘創自永樂初,日久頹圮,急宜修復,歲漕臨、德二倉二十萬石餉保定、易州、紫荊諸軍,足使士卒宿飽。往者,密雲、昌平故不通漕,萬曆初,總督劉應節、楊兆疏潮、白二河,陵泉諸水,漕粟以餉二鎮,二鎮之軍賴之。此可仿而行也。」詔從之。巡按陜西,疏陳邊政十事,劾罷副總兵王學書等七人。請建宗學如郡縣學制,報可。改按山東,擢順天府丞,以憂去。天啟四年起右僉都御史,撫治鄖陽。

懋康雅負器局,揚歷中外,與族兄懋良並有清譽,稱「二畢」。

懋良,字師皋。先懋康舉進士。由萬載知縣擢南京吏部主事。歷副使,至左布政使,俱在福建。振饑民,減加派,撫降海寇,以善績稱。懋康為巡撫之歲,懋良亦自順天府尹擢戶部右侍郎,督倉場。魏忠賢以懋康為趙南星所引,欲去之。御史王際逵劾其附麗邪黨,遂削籍。而懋良亦以不附忠賢,為御史張訥所論,落職閒住。兄弟相繼去國,士論更以為榮。

崇禎初,起懋康南京通政使。越二年,召拜兵部右侍郎,尋罷。而懋良亦起兵部左侍郎。會京師戒嚴,尚書張鳳翔以下皆獲罪,懋良得原,致仕去。懋康再起南京戶部右侍郎,督糧儲。旋引疾歸。兄弟皆卒於家。

蕭近高,字抑之,廬陵人。萬曆二十三年進士。授中書舍人。擢禮科給事中。甫拜官,即上疏言罷礦稅、釋繫囚、起廢棄三事,明詔已頒,不可中止。帝怒,奪俸一年。頃之,論江西稅使潘相擅刑宗人罪,不報。既而停礦分稅之詔下,相失利,擅移駐景德鎮,請專理窯務。帝即可之,近高復力爭。後江西撫按並劾相,相以為近高主之,疏詆甚力。近高疏辨,復劾相。疏雖不行,相不久自引去。

屢遷刑科都給事中。知縣滿朝薦、諸生王大義等皆忤中使,繫獄三年。近高請釋之,不報。遼東稅使高淮激民變,近高劾其罪,請撤還,帝不納。又以淮誣奏逮同知王邦才、參將李獲陽,近高復論救。會廷臣多劾淮者,帝不得已徵還,而邦才等繫如故。無何,極陳言路不通、耳目壅蔽之患。未幾,又言王錫爵密揭行私,宜止勿召;朱賡被彈六十餘疏,不當更留。皆不報。故事,六科都給事中內外遞轉。人情輕外,率規避,近高自請外補。吏部侍郎楊時喬請亟許以成其美。乃用為浙江右參政,進按察使。以病歸。起浙江左布政使。所至以清操聞。

泰昌元年召為太僕卿。廷議「紅丸」之案,近高言崔文昇、李可灼當斬,方從哲當勒還故里,張差謀逆有據,不可蔽以瘋癲。歷工部左、右侍郎。天啟二年冬,引疾去。御史黃尊素因言近高暨侍郎餘懋衡、曹于汴、饒伸,太僕少卿劉弘謨、劉宗周並辭榮養志,清風襲人,亟宜褒崇,風勵有位。詔許召還。五年冬,起南京兵部,添注左侍郎。力辭,不允。時魏忠賢勢張,諸正人屏斥已盡。近高不欲出,遷延久之。給事中薛國觀劾其玩命,遂落職。崇禎初,乃復。卒於家。

白瑜,字紹明,永平人。萬曆二十三年進士。選庶吉士,授兵科給事中。帝既冊立東宮,上太后徽號,瑜請推廣孝慈,以敦儉、持廉、惜人才、省冤獄四事進,皆引《祖訓》及先朝事以規時政,辭甚切。三十年,京師旱,陜西河南黃河竭。禮官請修省,瑜言:「修省宜行實政。今逐臣久錮,纍臣久羈,一蒙矜釋,即可感格天心。」末言礦稅之害。皆不報。

累遷工科都給事中。帝於射場營乾德臺,瑜抗疏力諫,又再疏請斥中官王朝、陳永壽,帝不能無憾。會瑜論治河當專任,遂責其剿拾陳言,謫廣西布政使照磨。以疾歸。光宗立,起光祿少卿,三遷太常卿。給事中倪思輝、朱欽相,御史王心一以直言被謫,瑜抗疏論救。

天啟二年,由通政使拜刑部右侍郎,署部事。鄭貴妃兄子養性奉詔還籍,逗遛不去,其家奴張應登訐其通塞外。永寧伯王天瑞者,顯皇后弟也,以后故銜鄭氏,遂偕其弟錦衣天麟交章劾養性不軌。瑜以鄭氏得罪先朝,而交通事實誣,乃會都御史趙南星、大理卿陳于廷等讞上其獄,請抵奴誣告罪,勒養性居遠方。制可。明年進左侍郎。卒官。贈尚書。

程紹,字公業,德州人。祖瑤,江西右布政使。紹舉萬曆十七年進士。除汝寧推官,徵授戶科給事中。巡視京營。副將佟養正等五人行賄求遷,皆劾置於理。帝遣使採礦河南,紹兩疏言宜罷,皆不報。

再遷吏科左給事中。會大計京官,御史許聞造訐戶部侍郎張養蒙等,語侵吏部侍郎裴應章。紹言聞造挾吏部以避計典,且附會閣臣張位,聞造乃貶邊方。主事趙世德考察貶官,廷議征楊應龍,兵部舉世德知兵,紹駁止之。又劾文選郎楊守峻,守峻自引去。饒州通判沈榜貶官,夤緣稅監潘相得留,紹極言非法。山西稅使張忠以夏縣知縣韓薰忤己,奏調之僻地,紹又爭之,帝怒,斥為民。以沈一貫救,詔鐫一秩,出之外。給事中李應策、御史李炳等爭之,帝益怒,并薰斥為民,而奪應策等俸。紹家居二十年。光宗即位,起太常少卿。

天啟四年,歷右副都御史,巡撫河南。宗室居儀封者為盜窟,紹列上其狀,廢徙高牆。臨漳民耕地漳濱,得玉璽,龍紐龜形,方四寸,厚三寸,文曰:「受命於天,既壽永昌」,以獻紹。紹聞之於朝,略言:「秦璽不足徵久矣。今璽出,適在臣疆,既不當復埋地下,又不合私秘人間。欲遣官恭進闕廷,跡涉貢媚。且至尊所寶,在德不在璽,故先馳奏聞,候命進止。昔王孫圉不寶玉珩,齊威王不寶照乘,前史美之。陛下尊賢愛士,野無留良。尚有一代名賢,如鄒元標、馮從吾、王紀、周嘉謨、盛以弘、孫慎行、鐘羽正、餘懋衡、曹于汴等皆憂國奉公,白首魁艾。其他詞林臺諫一錮不起者,並皇國禎祥,盛朝珍寶。臣不能汲致明廷,徒獻符貢瑞,臣竊羞之。願陛下惟賢是寶。在朝之忠直,勿事虛拘;在野之老成,亟圖登進。彼區區秦璽之真偽,又安足計哉。」魏忠賢方斥逐耆碩,見之不悅。後忠賢勢益張,紹遂引疾歸。

崇禎六年,薦起工部右侍郎。越二年,以年老,四疏乞休去。卒,贈本部尚書。

翟鳳翀,字凌元,益都人。萬曆三十二年進士。歷知吳橋、任丘,有治聲,徵授御史。疏薦鐘羽正、趙南星、鄒元標等,因言:「宋季邪諂之徒,終日請禁偽學,信口詆諆。近年號講學者,不幸類此。」出按遼東。宰賽、煖兔二十四營環開原而居,歲為邊患。宰賽尤桀驁,數敗官軍,殺守將,因挾邊吏增賞。慶雲參將陳洪範所統止羸卒二千,又恇怯不任戰。鳳翀奏請益兵,易置健將,開原始有備。又請所在建常平倉,括贖鍰,節公費,易粟備荒。帝善其議,命推行於諸邊。故遼陽參將吳希漢失律聽勘,以內援二十年不決,且謀復官,鳳翀一訊成獄,置之大辟,邊人快之。

帝因「梃擊」之變,召見廷臣於慈寧宮。大學士方從哲、吳道南無所言,御史劉光復方發口,遽得罪。鳳翀上言:「陛下召對廷臣,天威開霽,千載一時。輔臣宜舉朝端大政,如皇太子、皇長孫講學,福府莊田鹺引,大僚空虛,考選沉閣,以及中旨頻降,邊警時聞,水旱盜賊之相仍,流移饑殍之載道,一一縷奏於前,乃緘默不言,致光復以失儀獲罪。光復一日未釋,輔臣未可晏然也。」忤旨,切責。山東大饑,以鳳翀疏,遣御史過庭訓齎十六萬金振之。

中官呂貴假奸民奏,留督浙江織造。冉登提督九門,誣奏市民毆門卒,下兵馬指揮歐相之吏。邢洪辱御史凌漢翀於朝,給事中郭尚賓等劾之,帝釋洪不問。漢翀為廢將凌應登所毆,洪復曲庇應登。鳳翀抗疏極論貴、登、洪三人罪,且曰:「大臣造膝無從,小臣叩閽無路。宦寺浸用,政令多違,實開群小假借之端,成太阿倒持之勢。」帝大怒,謫山西按察使經歷。而是時,尚賓亦上疏極言:「比來擬旨不由內閣,託以親裁。言官稍涉同類,輒云黨附,將使大臣不肯盡言,小臣不敢抗論,天下事尚可為哉?乞陛下明詔閣臣,封還內降,容納直諫,以保治安。」忤旨,謫江西布政使檢校。閣臣及言官論救,皆不納。帝於章疏多不省,故廷臣直諫者久不被譴。至是二人同日謫官,時稱「二諫」。

鳳翀既謫,三遷。天啟初,為南京光祿少卿。四年,以大理少卿進右僉都御史,巡撫延綏。魏忠賢黨御史卓邁、汪若極連章論之,遂削籍。崇禎二年起兵部右侍郎,尋出撫天津。以疾歸。卒,贈兵部尚書。

尚賓,字朝諤,南海人,鳳翀同年進士。自吉安推官授刑科給事中。遇事輒諫諍,尤憤中官之橫。嘗因事論稅使李鳳、高寀、潘相,頗稱敢言。已,竟謫官。光宗時乃復起,累官刑部右侍郎,亦以不附忠賢削籍。崇禎初,為兵部右侍郎。卒,贈尚書。

洪文衡,字平仲,歙人。萬曆十七年進士。授戶部主事。帝將封皇長子為王,偕同官賈巖合疏爭。尋改禮部。與郎中何喬遠善,喬遠坐詿誤被謫,文衡已遷考功主事,竟引病歸。

起補南京工部,歷郎中。力按舊章,杜中貴橫索,節冗費為多。官工部九年,進光祿少卿。改太常,督四夷館。中外競請起廢,帝率報寢。久之,乃特起顧憲成。憲成已辭疾,忌者猶憚其進用,御史徐兆魁首疏力攻之。文衡慮帝惑兆魁言,抗章申雪,因言:「今兩都九列,強半無人,仁賢空虛,識者浩歎。所堪選擇而使者,只此起廢一途。今憲成尚在田間,已嬰羅罔,俾聖心愈疑。連茹無望,貽禍賢者,流毒國家,實兆魁一疏塞之矣。」尋進大理少卿。以憂去。

泰昌元年起太常卿。光宗既崩,議升祔。文衡請祧睿宗,曰:「此肅宗一時崇奉之情,不合古誼。且睿宗嘗為武宗臣矣,一旦加諸其上,禮既不合,情亦未安。當時臣子過於將順,因循至今。夫情隆於一時,禮垂於萬世,更定之舉正在今時。」疏格不行。未幾卒,贈工部右侍郎。

文衡天性孝友。居喪,斷酒肉不處內者三年。生平不妄取一介。

喬遠,字稚孝,晉江人。萬曆十四年進士。除刑部主事,歷禮部儀制郎中。神宗欲封皇長子為王,喬遠力爭不可。同官陳泰來等言事被謫,抗疏救之。石星主封倭,而朝鮮使臣金晬泣言李如松、沈惟敬之誤,致國人束手受刃者六萬餘人。喬遠即以聞,因進累朝馭倭故事,帝頗心動。而星堅持己說,疏竟不行。坐累謫廣西布政使經歷,以事歸。里居二十餘年,中外交薦,不起。

光宗立,召為光祿少卿,移太僕。王化貞駐兵廣寧,主戰。喬遠畫守禦策,力言不宜輕舉。無何,廣寧竟棄。天啟二年進左通政。鄒元標建首善書院,朱童蒙等劾之,喬遠言:「書院上梁文實出臣手,義當并罷。」語侵童蒙。進光祿卿、通政使。五疏引疾,以戶部右侍郎致仕。崇禎二年,起南京工部右侍郎。給事中盧兆龍劾其衰庸,自引去。

喬遠博覽,好著書。嘗輯明十三朝遺事為《名山藏》,又纂《閩書》百五十卷,頗行於世,然援据多舛云。

陳伯友,字仲恬,濟寧人。萬曆二十九年進士。授行人。擢刑科給事中。甫拜命,即罷河南巡撫李思孝。俄論鄒之麟科場弊宜勘;奄豎辱駙馬冉興讓,宜置之法;楚宗英憔、蘊鈁,良吏滿朝薦、王邦才等宜釋。已,又言:「陛下清明之心,不幸中年為利所惑,皇皇焉若不足,以致財匱民艱,家成徹骨之貧,人抱傷心之痛。今天下所以杌隉傾危而不可救藥者,此也。」又言:「李廷機去國,操縱不出上裁。至外而撫按,內而庶僚,去留無所斷決。士大夫意見分岐,議論各異,陛下漫無批答。曷若盡付外廷公議,於以平曲直、定國是乎?」帝皆不省。熊廷弼為荊養喬所訐,伯友與李成名等力主行勘。

既又陳時政四事,言:「擬旨必由內閣。昨科臣曾六德之處分,閣臣葉向高之典試,悉由內降。而福王之國之旨,亦於他疏批行。非獨褻天言,抑且貽陰禍。法者天下所共,黔國公沐昌祚請令其孫啟元代鎮,已非法矣。乃撫按據法請勘,而以內批免之,疑中有隱情。御史呂圖南改提學,此爭為賢,彼爭為不肖,盍息兩家戈矛,共圖軍國大計?福王久應之國,今春催請不下數百疏,何以忽易期?」疏亦留中。尋以艱去。及服除,廷議多排東林,遂不出。

至四十六年,以年例,即家除河南副使。天啟四年,屢遷太常寺卿,治少卿事。楊漣劾魏忠賢,伯友亦偕卿胡世賞等抗疏極論。明年十二月,御史張樞劾其倚附東林,遂削奪。莊烈帝即位,詔復官,未及用而卒。

成名,字寰知,太原衛人。祖應時,南京戶部員外郎,以清白著。成名舉萬曆三十二年進士,授中書舍人。擢吏科給事中。疏陳銓政失平,語侵尚書趙煥。俄請釋累臣滿朝薦,言朝薦不釋則諸璫日肆,國家患無已。吏部侍郎方從哲,中旨起官,成名抗疏劾之,并及其子恣橫狀。從哲求去,帝不許。是時,黨人日攻東林,成名遂移疾歸。

家居五年,起山東副使。天啟初,遷湖廣參政,入為太僕少卿。四年春,擢右僉都御史,巡撫南、贛。魏忠賢以成名為趙南星所用,因所屬給由,犯御諱,除其名。為巡撫止八月,士民祠祀焉。崇禎改元,召拜戶部右侍郎,以左侍郎專理邊餉。京師戒嚴,改兵部。帝召對平臺,區畫兵事甚悉。數月而罷,卒於家。

董應舉,字崇相,閩縣人。萬歷二十六年進士。除廣州教授。與稅監李鳳爭學傍需地,鳳舍人馳騎文廟前,縶其馬,用是有名。

遷南京國子博士,再遷南京吏部主事。召為文選主事。歷考功郎中,告歸。起南京大理丞。四十六年閏四月,日中黑子相鬥。五月朔,有黑日掩日,日無光。時遼東撫順已失,應舉言:「日生黑眚,乃強敵侵凌之徵。亟宜勤政修備,以消禍變。」因條上方略。帝置不省。

天啟改元,再遷太常少卿,督四夷館。二年春,陳急務數事,極言天下兵耗民離,疆宇日蹙,由主威不立,國法不行所致。帝以為應舉知兵,令專任較射演武。

已,上言保衛神京在設險營屯。遂擢應舉太僕卿兼河南道御史,經理天津至山海屯務。應舉以責太重,陳十難十利,帝悉敕所司從之。乃分處遼人萬三千餘戶於順天、永平、河間、保定,詔書褒美。遂用公帑六千買民田十二萬餘畝,合閒田凡十八萬畝,廣募耕者,畀工廩、田器、牛種,浚渠築防,教之藝稻,農舍、倉廨、場圃、舟車畢具,費二萬六千,而所收黍麥穀五萬五千餘石。廷臣多論其功,就進右副都御史。天津葛沽故有水陸兵二千,應舉奏令屯田,以所入充歲餉,屯利益興。

五年六月,朝議以屯務既成,當廣鼓鑄。乃改應舉工部右侍郎,專領錢務,開局荊州。尋議給兩淮鹽課為鑄本,命兼戶部侍郎,並理鹽政。應舉至揚州,疏請釐正鹽規,議商人補行積引,增輸銀視正引之半,為部議所格。應舉方奏析,而巡鹽御史陸世科惡其侵官,劾之,魏忠賢傳旨詰讓,御史徐揚先遂希指再劾,落職閒住。崇禎初,復官。

應舉好學善文。其居官,慷慨任事;在家,好興利捍患。比沒,海濱人祠祀之。

林材,字謹任,閩縣人。萬曆十一年進士。授舒城知縣。擢工科給事中。吏部推鄭洛戎政尚書,起張九一貴州巡撫,材極言兩人不當用,九一遂罷。王錫爵赴召,材疏論,并及趙志皋、張位。再請建儲豫教,又爭三王並封之謬。

屢遷吏科都給事中。劾罷南京尚書郝傑、徐元泰。經略宋應昌惑沈惟敬,力請封貢,材乞斬應昌、惟敬,不報。志皋、位擬旨失當,材抗疏駁之。二十二年夏六月,西華門災,材偕同官上言,切指時政缺失。帝慍甚,以方修省不罪。吏部推顧養謹總理河道,材論止之。兵部將大敘平壤功,材力詆石星罔上,星乃不敢濫敘。其冬,復率同官言成憲不當為祭酒,馮夢禎不當為詹事,劉元震不當為吏部侍郎。帝積前怒,言材屢借言事誣謗大臣,今復暗傷善類,乃貶三官,餘停俸一歲。會御史崔景榮等論救,再貶程鄉典史。材遂歸里不出。

光宗即位,始起尚寶丞,再遷太僕少卿。還朝未幾,即乞歸。天啟中,起南京通政使,卒。崇禎初,贈右都御史。

朱吾弼,字諧卿,高安人。萬曆十七年進士。授寧國推官。征授南京御史。

大學士趙志皋弟學仕為南京工部主事,以贓敗。南京刑部因志皋故,輕其罪,議調饒州通判。吾弼疏論,竟謫戍之。奏請建國本,簡閣臣,補言官,罷礦稅,不報。山西巡撫魏允貞為稅使孫朝所訐,吾弼乞治朝欺罔罪。廣東稅使李鳳乾沒,奸人王遇桂請稅江南田契,吾弼皆疏論其罪。時無賴子蜂起言利,廷臣輒連章力爭,帝雖不盡從,亦未嘗不容其切直。雷震皇陵,吾弼請帝廷見大臣,講求祖宗典制,次第舉行,與天下更始。尋復言:「陛下孝敬疏於郊廟,惕厲弛於朝講;土木盛宮苑,榛蕪遍殿廷,群小橫中外,正士困囹圄;閭閻以礦稅竭,郵傳以輸輓疲,流亡以水旱增,郡縣以徵求困;草澤生心,衣冠喪氣;公卿不能補牘,臺諫無從引裾。不可不深察而改圖也。」末言禮部侍郎郭正域疾惡嚴,居己峻,不可以楚事棄。

先是,楚假王議起,首輔沈一貫陰左右王,以正域請行勘,嗾其黨錢夢皋阜輩逐之去。舉朝無敢留正域及言楚事者,吾弼獨抗章申理,而御史林秉漢以楚宗人戕殺巡撫,亦請詳勘。且言:「王既非假,何憚於勘?」吾弼、秉漢遂為一貫等所惡。會夢皋京察將黜,遂訐秉漢為正域鷹犬,語侵沈鯉、楊時喬、溫純。秉漢坐貶貴州按察司檢校,而夢皋得留。郎中劉元珍論之,反獲譴。吾弼復疏直元珍,請黜夢皋,因力詆一貫,亦忤旨,停俸一年,遂移疾去。居三年,起南京光祿少卿,召為大理右丞。齊、楚、浙三黨用事,吾弼復辭疾歸。熹宗立,召還。屢遷南京太僕卿。天啟五年為御史吳裕中劾罷。

秉漢,字伯昭,長泰人。按廣東,亦再疏劾李鳳。既謫,尋移疾歸,卒於家。天啟中,贈太僕少卿。

張光前,字爾荷,澤州人。萬曆三十八年進士。授蒲圻知縣,補安肅。甫四月,擢吏部驗封主事。歷文選員外郎、稽勳郎中。乞假去。

天啟四年,趙南星為尚書,起為文選郎中。甫視事,魏忠賢欲逐南星,假廷推謝應祥事矯旨切責。南星時與推應祥者,員外郎夏嘉遇,非光前也。光前抗疏爭之,曰:「南星人品事業昭灼人耳目,忽奉嚴旨責以不公忠,臣竊惑之。選郎,諸曹領袖,尚書臂指,南星所甄別進退,臣實佐之。功罪與共,乞先賜罷斥。」亦被旨切責。未幾,以推喬允升等代南星,忤忠賢意,削侍郎陳于廷及楊漣、左光斗籍。光前又抗疏曰:「會推尚書,于廷主議,臣執筆,謹席稿待罪。」遂貶三秩,調外任。

光前操行清嚴,峻卻請謁。知縣石三畏贓私狼籍,得奧援,將授臺諫,光前出之為王官,其黨咸側目。明年,光前兄右布政使光縉治兵遵化,為奄黨門克新所劾,亦削籍。兄弟並以忤奄去,見稱於世。崇禎元年,起光祿少卿,不赴。三年,起太常。已,進大理少卿。累疏乞休,及家而卒。

贊曰:朝政弛,則士大夫騰空言而少實用。若陳邦瞻、畢懋康、翟鳳翀、董應舉,尚思有所建立,惜不逢明作之朝,故所表見止此耳。蕭近高、洪文衡諸人皆以清素自矢,白瑜論鄭氏獄能持平,固卿貳之錚錚者歟。


\end{pinyinscope}