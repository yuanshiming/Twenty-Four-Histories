\article{列傳第一百六十一 左良玉鄧賀人龍高傑劉澤清祖寬}

\begin{pinyinscope}
左良玉鄧賀人龍高傑劉澤清祖寬

左良玉,字崑山,臨清人。官遼東車右營都司。崇禎元年,寧遠兵變,巡撫畢自肅自經死,良玉坐削職回衛。已,復官。總理馬世龍令從遊擊曹文詔援玉田、豐潤,連戰洪橋、大塹山,直抵遵化。論恢復四城功,與文詔等俱進秩,隸昌平督治侍郎侯恂麾下。大凌河圍急,詔昌平軍赴援,總兵尤世威護陵不得行,薦良玉可代率兵往。已,恂薦為副將,戰松山、杏山下,錄功第一。

良玉少孤,育於叔父。其貴也,不知其母姓。長身赬面,驍勇,善左右射。目不知書,多智謀,撫士卒得其歡心,以故戰輒有功。時陜西賊入河南,圖懷慶。廷議令良玉將昌平兵往剿,大指專辦河南。會賊寇修武、清化者竄入平陽,因檄良玉入山西禦之,頗有斬獲。河南巡撫樊尚璟以良玉駐澤州,扼豫、晉咽侯,可四面為援兵。詔從之。時曹文詔將陜西兵,帝令良玉受尚璟節制,與文詔同心討賊,有急則秦兵東,豫兵西,良玉兵從中橫擊。

六年正月,賊犯隰州,陷陽城。良玉敗之於涉縣之西陂。二月,良玉兵與賊戰武安,大敗。尚璟罷,以太常少卿玄默代之。三月,賊再入河內,良玉自輝縣逐之。賊奔修武,殺遊擊越效忠,追參將陶希謙,希謙墜馬死。良玉擊之萬善驛,至柳樹口大敗之,擒賊首數人,賊遂西奔。河南額兵僅七千,數被賊,折亡殆盡。良玉將昌平兵二千餘,數戰,雖有功,勢孤甚。總兵鄧方立功萊州,乃命將川兵益以石砫土司馬鳳儀兵馳赴良玉,與共角賊。已而鳳儀以孤軍戰沒於侯家莊。

當是時,賊勢已大熾,縱橫三晉、畿輔、河北間。諸將曹文詔、李卑、艾萬年、湯九州、鄧、良玉等先後與賊戰,勝負略相當。良玉、辦河南,屢破之於官村,於沁河,於清化,於萬善。良玉又扼之武安八德,斬獲尤多。會帝命倪寵、王樸為總兵,將京營兵六千赴河南,以中官楊進朝、盧九德監其軍,而別遣中官監良玉等軍。職方郎中李繼貞曰:「良玉、李卑身經百戰,位反在寵、樸下,恐聞而解體。」乃令良玉、卑署都督僉事,為援剿總兵官,與寵、樸體相敵。京營兵至,共擊賊,數有功。良玉敗賊濟源、河內,又敗之永寧青山嶺銀洞溝,又自葉縣追至小武當山,皆斬賊魁甚眾。然諸將以中官監軍,意弗善也。

其冬,賊西奔者復折而東。良玉、九州扼其前,京營兵尾其後,賊大困,官軍連破之柳泉、猛虎村。賊張妙手、賀雙全等三十六家詭詞乞撫於分巡布政司常道立,因監軍進朝以請。諸將俟朝命,不出戰。會天寒河冰合,賊遂從澠池徑渡,巡撫默率良玉、九州、卑、兵待之境上。賊乃竄盧氏山中,由此自鄖、襄入川中,折而掠秦隴,復出沒川中、湖北,以犯河南,中原益大殘破,而三晉、畿輔獨不受賊禍者十年。

賊既渡河去,良玉與諸將分地守。陳奇瑜、盧象升方角賊秦、楚,七年春夏間,中州幸無事。既而奇瑜失李自成於車箱,廷議合晉、豫、楚、蜀兵四面剿之。賊乃分軍三:一向慶陽,一趨鄖陽,而一出關趨河南。趨河南者又分為三,郡邑所在告急。良玉扼新安、澠池,他將陳治邦駐汝州,陳永福扼南陽,皆坐甲自保而已,不能大創賊也。賊每營數萬,兵番進,皆因糧宿飽;我兵寡備多,饋餉不繼。賊介馬馳,一日夜數百里;我步兵多,騎少,行數十里輒疲乏,以故多畏賊。而良玉在懷慶時,與督撫議不合,因是生心,緩追養寇,多收降者以自重。督撫檄調,不時應命,稍稍露跋扈端矣。十二月遇賊於磁山,大戰數十,追奔百餘里。

八年正月,河南賊破潁州,毀鳳陽皇陵。其陷鹿邑、柘城、寧陵、通許者,良玉在許州不能救。四月,督師洪承疇在汝州,令諸將分地遮賊。尤世威守雒南,陳永福控盧氏、永寧,鄧、尤翟文、張應昌、許成名遏湖廣。以吳村、瓦屋乃內鄉、淅川要地,令良玉與湯九州以五千人扼之。未幾,鄧圯以兵嘩死,而曹文詔討陜賊,敗沒於真寧。賊益張,遂超盧氏,奔永寧。巡撫默被逮未去,檄良玉自內鄉與陳治邦、馬良文等援盧氏。八月敗賊於鄢陵。九月躡賊於郟之神垕山。賊連營數十里,番休更戰,以疲我兵,良玉收其軍而止。賊再攻密,良玉自郟援之,乃去。十月,良玉抵靈寶,合遼東總兵祖寬兵剪賊於澗口、焦村。焦村,朱陽關地也。十一月,李自成出朱陽關,張獻忠久據靈寶,闖王高迎祥亦與合。良玉、寬禦之靈寶,不能支,陜州陷。賊東下攻洛陽,良玉、寬從巡撫陳必謙救洛陽,賊乃去。迎祥、自成走偃師、鞏。獻忠走嵩、汝。良玉出雒追迎祥、自成。寬分擊獻忠救汝。會總理盧象昇至自湖廣,與寬大敗賊汝西,令裨將破賊於宜陽黃澗口。

九年二月,賊敗於登封郜城鎮,走石陽關,與伊、嵩之賊合。故總兵九州由嵩縣深入,與良玉夾剿。良玉中道遁歸,九州乘勝窮追四十里,無援敗歿,良玉反以捷聞。五月,象昇遣祖寬、李重鎮隨陜西總督洪承疇西行。良玉軍最強,又率中州人,故獨久留之。而以其驕亢難用,用孔道興代其偏將趙柱駐靈寶,防雒西;良玉與羅岱駐宜、永,防雒東。七月,良玉兵抵開封,由登封之唐莊深入擊賊,自辰鏖至申,賊不支西走。陳永福方敗賊於唐河,賊至田家營,良玉渡河擊之,斬獲頗眾。九月,巡撫楊繩武劾良玉避賊,責令戴罪自贖。

十年正月,賊老回回合曹操、闖塌天諸部沿流東下,安慶告警,詔良玉從中州救之。良玉道剿殺南陽土寇楊四、侯馭民、郭三海,急抵六安,與賊遇。部將岱、道興乘勝連戰,大破賊。賊走霍、潛山。會馬爌、劉良佐亦屢敗賊於桐城、廬州、六安,賊在滁、和者亦西遁,江北警少息。應天巡撫張國維三檄良玉入山搜剿,不應,放兵掠婦女。屯舒城月餘,河南監軍太監力促之,始北去,賊已飽掠入山矣。已,淅川陷,良玉擁兵不救。以六安破賊功,詔落職戴罪,尋復之。賊東下襲六合,攻天長,分掠瓜洲、儀真,破盱眙。良玉堅不肯救,令中州士大夫合疏留己。帝知出良玉意,不能奪也。十月,總理熊文燦至安慶,部檄以良玉軍隸焉,良玉輕文燦不為用。

十一年正月,良玉與總兵陳洪範大破賊於鄖西。張獻忠假官旗號襲南陽,屯於南關。良玉適至,疑而急召之,獻忠逸去。追及,發兩矢,中其肩,復揮刀擊之,面流血。其部下救以免,遂逃之穀城。未幾,請降,良玉知其偽,力請擊之,文燦不許。九月,文燦剿鄖、襄諸賊,良玉與洪範及副將龍在田擊破之雙溝營,斬首二千餘級。十二月,河南巡撫常道立調良玉於陜州。賊乘盧氏虛,遁入內、淅。是月,許州兵變,良玉家在許,殲焉。

十二年二月,良玉率降將劉國能入援京師,詔還討河南賊。兵過灞頭、吳橋,大掠,太監盧九德疏聞,詔令戴罪。已而破賊馬進忠於鎮平關。進忠降。又與國能再破賊李萬慶於張家林、七里河,萬慶亦降。七月,獻忠叛去,良玉與羅岱追之,使岱為前鋒,己隨其後。逾房縣八十里,至羅猴山,軍乏食。伏起,岱馬掛於藤,抽刀斷之,蹶而復進,棄馬登山,賊圍急,矢盡被獲。良玉大敗奔還,軍符印信盡失,棄軍資千萬餘,士卒死者萬人。事聞,以輕進貶三秩。

十三年春,督師楊嗣昌薦良玉雖敗,有大將才,兵亦可用,遂拜平賊將軍。當是時,賊分為三:西則張獻忠,踞楚、蜀郊;東則革裏眼、左金王等四營,豕突隨、應、麻、黃;南則曹操、過天星等十營,伏漳、房、興、遠間。閏正月,良玉合諸軍擊賊於枸坪關,獻忠敗走,良玉乃請從漢陽、西鄉入蜀追之。嗣昌謀以陜西總督鄭崇儉率賀人龍、李國奇從西鄉入蜀,而令良玉駐兵興平,別遣偏將追剿,良玉不從。嗣昌檄良玉曰:「賊勢似不能入川,仍當走死秦界耳。將軍從漢陽、西鄉入川,萬一賊從舊路疾趨平利,仍入竹、房,將何以禦?不則走寧昌,入歸、巫,與曹操合,我以大將尾追,促賊反楚,非算也。」良玉報曰:「蜀地肥衍,賊渡險任其奔軼,後難制。且賊入川則有糧可因,回鄖則無地可掠,其不復竄楚境明矣。夫兵合則強,分則弱。今已留劉國能、李萬慶守鄖,若再分三千人入蜀,即駐興平,兵力已薄,賊來能遏之耶?今當出其不意疾攻之,一大創自然瓦解,縱折回房、竹間,人跡斷絕,彼從何得食?況鄖兵扼之於前,秦撫在紫、興扼之於右,勢必不得逞。若寧昌、歸、巫險且遠,曹操、獻忠不相下。倘窮而歸曹,必內相吞,其亡立見。」良玉已於二月朔涉蜀界之漁溪渡矣。嗣昌度力不能制,而其計良是,遂從之。

時獻忠營太平縣大竹河,良玉駐漁溪渡。未幾,總督崇儉引其兵來會。賊移軍九滾坪,見瑪瑙山峻險,將據之。良玉始抵山下,賊已踞山顛,乘高鼓噪。良玉下馬周覽者久之,曰:「吾知所以破賊矣。」分所進道為三,己當其二,秦兵當其一。令曰:「聞鼓聲而上。」兩軍夾擊,賊陣堅不可動。鏖戰久之,賊大潰,墜崖澗者無算。追奔四十里,良玉兵斬掃地王曹威、白馬鄧天王等渠魁十六人。獻忠妻妾亦被擒,遁入興山、歸州之山中,尋自鹽井竄興、歸界上。是役也,良玉功第一。事聞,加太子少保。四月,良玉進屯興安、平利諸山,連營百里。諸軍憚山險,圍而不攻。久之,獻忠自興、房走白羊山而西,與羅汝才合。七月,良玉乘勝擊過天星,降之。過天星者,名惠登相,既降,遂始終為良玉部將。

初,良玉受平賊將軍印,浸驕,不肯受督師約束。而賀人龍屢破賊有功,嗣昌私許以人龍代良玉。及良玉奏瑪瑙山捷,嗣昌語人龍須後命。人龍大恨,具以前語告良玉,良玉亦內恨。當獻忠之敗走也,追且及,遣其黨馬元利操重寶啖良玉曰:「獻忠在,故公見重。公所部多殺掠,而閣部猜且專。無獻忠,即公滅不久矣。」良玉心動,縱之去。監軍萬元吉知良玉跋扈不可使,勸嗣昌令前軍躡賊,後軍繼之,而身從間道出梓潼扼歸以俟濟師,嗣昌不用。賊既入蜀之巴州,人龍兵噪而西歸。召良玉兵合擊,九檄皆不至。

十四年正月,諸軍追賊開縣之黃陵城。參將劉士傑深入,所當披靡。獻忠登高望,見無秦人旗幟,而良玉兵前部無鬥志,獨士傑孤軍。乃密選壯士潛行箐谷中,乘高大呼馳下,良玉兵先潰,總兵猛如虎潰圍出。嗣昌方悔不用元吉言,而獻忠已席卷出川,西絕新開驛置,楚、蜀消息中斷,遂以計紿入襄陽城。襄王被執,嗣昌不食卒。賊瀕死復縱,迄以亡國者,以良玉素驕蹇不用命故也。二月,詔良玉削職戴罪,平賊自贖。五月,獻忠陷南陽,即攻沁陽破之。良玉至南陽,賊遁去。良玉不戢士,沁人脫於賊者,遇官軍無噍類。既而獻忠陷鄖西,掠地至信陽,屢勝而驕。良玉乃從南陽進兵,復大破之,降其眾數萬。獻忠中股,負重傷夜遁。而是時,李自成方殘襄城,圍良玉於郾城,幾陷。會陜西總督汪喬年出關,自成乃輟圍,與喬年戰襄陽城外。喬年軍盡覆,良玉不能救。帝既斬賀人龍以肅軍攻,專倚良玉辦賊。

十五年四月,自成復圍開封,乃釋故尚書初薦良玉者侯恂於獄,起為督師,發帑金十五萬犒良玉營將士,激勸之。良玉及虎大威、楊德政會師朱仙鎮,賊營西,官軍營北。良玉見賊勢盛,一夕拔營遁,眾軍望見皆潰。自成戒士卒待良玉兵過,從後擊之。官軍幸追者緩,疾馳八十里。賊已於其前穿塹深廣各二尋,環繞百里,自成親率眾遮於後。良玉兵大亂,下馬渡溝,僵仆溪谷中,趾其顛而過。賊從而蹂之,軍大敗,棄馬騾萬匹,器械無算,良玉走襄陽。帝聞良玉敗,詔恂拒河圖賊,而令良玉以兵來會。良玉畏自成,遷延不至。九月,開封以河決而亡。帝怒恂,罷其官,不能罪良玉也。開封既亡,自成無所得,遽引兵西,謀拔襄陽為根本。

時良玉壁樊城,大造戰艦,驅襄陽一郡人以實軍,諸降賊附之,有眾二十萬。然親軍愛將大半死,而降人不奉約束,良玉亦漸衰多病,不復能與自成角矣。自成乘勝攻良玉,良玉退兵南岸,結水寨相持,以萬人扼淺洲。賊兵十萬爭渡,不能遏。良玉乃宵遁,引其舟師,左步右騎而下。至武昌,從楚王乞二十萬人餉,曰:「我為王保境。」王不應,良玉縱兵大掠,火光照江中。宗室士民奔竄山谷,多為土寇所害。驛傳道王揚基奪門出,良玉兵掠其貲,并及其子女。自十二月二十四日抵武昌,至十六年正月中,兵始去。居人登蛇山以望,叫呼更生,曰:「左兵過矣!」良玉既東,自成遂陷承天,傍掠諸州縣。

當是時,降兵叛卒率假左軍號恣剽掠,蘄州守將王允成為亂首,破建德,劫池陽,去蕪湖四十里,泊舟三山、荻港,漕艘鹽舶盡奪以載兵。聲言諸將寄帑南京,請以親信三千人與俱。南京諸文武官及操江都御史至陳師江上為守禦。士民一夕數徙,商旅不行。都御史李邦華被召,道湖口,草檄告良玉,以危詞動之。而令安慶巡撫發九江庫銀十五萬兩,補六月糧,軍心乃定。邦華入見帝,論良玉潰兵之罪,請歸罪於王允成。帝乃令良玉誅允成,而獎其能定變。良玉卒留允成於軍中,不誅也。良玉留安慶久之,徐溯九江上。聞獻忠破湖廣,沉楚王於江,坐視不救。

八月乃入武昌,立軍府招徠,下流粗定,分命副將吳學禮援袁州。江西巡撫郭都賢惡其淫掠,檄歸之,而自募土人為戍守。會賊陷長沙、吉州,復陷袁州、岳州,良玉遣馬進忠援袁州,馬士秀援岳州。士秀率水師敗賊岳州城下,二城遂並復。時帝命兵部侍郎呂大器代侯恂為總督,恂解任,中道逮下獄。良玉知其為己故,心鞅鞅,與大器齟齬。賊連陷建昌諸府,大器無兵不能救,良玉亦不援。進忠與賊戰嘉魚,再失利,良玉軍遂不振。會獻忠從荊河入蜀,良玉遣兵追之,距荊州七十里。荊、襄諸賊因自成入關,盡懈。良玉偵知,乃遣副將盧光祖上隨、棗、承德,而惠登相自均、房,劉洪起自南陽,掎賊後,收其空虛地以自為功。

十七年三月,詔封良玉為寧南伯,畀其子夢庚平賊將軍印,功成世守武昌。命給事中左懋第便道督戰,良玉乃條日月進兵狀以聞。疏入,未奉旨,聞京師被陷,諸將洶洶,以江南自立君,請引兵東下。良玉慟哭,誓不許。副將士秀奮曰:「有不奉公令復言東下者,吾擊之!」以巨艦置炮斷江,眾乃定。

福王立,晉良玉為侯,廕一子錦衣衛正千戶,且並封黃得功、高傑、劉澤清、劉良佐為諸鎮,俱蔭子世襲,而以上流之事專委良玉,尋加太子太傅。時李自成敗於關門,良玉得以其間稍復楚西境之荊州、德安、承天。而湖廣巡撫何騰蛟及總督袁繼咸居江西,皆與良玉善,南都倚為屏蔽。

良玉兵八十萬,號百萬,前五營為親軍,後五營為降軍。每春秋肄兵武昌諸山,一山幟一色,山谷為滿。軍法用兩人夾馬馳,曰:「過對」。馬足動地殷如雷,聲聞數里。諸鎮兵惟高傑最強,不及良玉遠甚。然良玉自朱仙鎮之敗,精銳略盡,其後歸者多烏合,軍容雖壯,法令不復相懾。良玉家殲於許州,其在武昌,諸營優娼歌舞達旦,良玉塊然獨處,無姬侍。嘗夜宴僚佐,召營妓十餘人行酒,履濆交錯,少焉左顧而欬,以次引出。賓客肅然,左右莫敢仰視。其統馭有體,為下所服多此類。而是時,良玉已老且病,無中原意矣。

良玉之起由侯恂。恂,故東林也。馬士英、阮大鋮用事,慮東林倚良玉為難,謾語修好,而陰忌之,築板磯城為西防。良玉嘆曰:「今西何所防,殆防我耳。」會朝事日非,監軍御史黃澍挾良玉勢,面觸馬、阮。既返,遣緹騎逮澍,良玉留澍不遣。澍與諸將日以清君側為請,良玉躊躇弗應。亡何,有北來太子事,澍借此激眾以報己怨,召三十六營大將與之盟。良玉反意乃決,傳檄討馬士英,自漢口達蘄州,列舟二百餘里。良玉疾已劇,至九江,邀總督袁繼咸入舟中,袖中出密諭,云自皇太子,劫諸將盟,繼咸正辭拒之。部將郝效忠陰入城,縱火殘其城而去。良玉望城中火光,曰:「予負袁公。」嘔血數升,是夜死。時順治二年四月也。諸將秘不發喪,共推其子夢庚為留後。七日,軍東下,朝命黃得功渡江防剿。

初,夢庚自立,佯語繼咸至池州侯旨。抵池,繼咸密以疏聞,道梗不得達。惠登相者,初為賊,既降,為良玉副將。諸軍自彭澤下,連陷建德、東流,殘安慶城,獨池州不破,貽書登相曰:「留此以待後軍。」登相大詬曰:「若此,則我反不如前為流賊時矣,如先帥末命何!」檄其軍返。夢庚見黑旗船西上,索輕舸追及之,登相與相見大慟。以夢庚不足事,引兵絕江而去,諸將乃議旋師。時大清兵已下泗州,逼儀真矣。夢庚遂偕澍以眾降於九江。

鄧,四川人。天啟初,從軍,積功得守備。安邦彥反,追賊織金,勇冠諸將。已,敗織績河濱。魯欽敗歿,賊犯威清。夜斫營走賊,進都司僉書。討敗苗酋李阿二。自貴州用兵,裨將楊明楷、劉志敏、張云鵬並驍勇,不得為大將,惟以功名聞。

崇禎初,屢遷四川副總兵,與侯良柱共斬安邦彥。京師有警,率六千人勤王,共復遵、永四城。加署都督僉事,世蔭千戶。尋擢總兵官,鎮守遵化。戰喜峰口及洪山,並有功,進秩為真。五年春,叛將亂登、萊,王洪等無功。自請行,命為援剿總兵官,與洪及劉國柱禦賊沙河,戰相當。已而遁走,賊乘之,大敗。尋與諸將金國奇等復登、萊二城,錄功進署都督同知。

戍遵化久,思歸。及登、萊事竣,復以為言。會賊入河北,言者請令剿,怏怏而行。給事中范淑泰劾虐民,帝不問,旋遣近侍監其軍。至濟源,射殺王自用於善陽山,即賊紫金梁也。頃之,賊逼磁州,拒卻之彭城鎮。與左良玉擊賊清池、柳莊,賊走林縣。部將楊遇春邀賊,中伏死。賊用其旗,並誘殺他將,自是輕。俄與良玉逐賊沙河,賊圍湯陰,被困土樵窩,良玉救乃免。已,共破賊官村、沁河、清化、萬善,移師畿南,敗賊白草關。賊犯平山,敗之紅子店、馬種川。賊遁青石嶺敗之紅澗村、醉漢口。賊犯臨城,敗之魚桂嶺。

當是時,賊蔓河溯及畿南,天子特遣倪寵、王樸將京軍,而保定梁甫,河南左良玉,湯九州合軍足殄賊。群帥勢相軋,彼此觀望,託山深道岐以自解,莫利先入,賊遂由澠池南渡。而諸帥各有近侍為中軍,事易掩飾,所報功多不以實也。十一月,賊南遁,追敗之澠池扣子山,至宜陽、盧氏而還。是月以為保定總兵官,代梁甫。

七年正月以賊盡入鄖、襄,命援剿,解南漳圍。尋敗賊胡地沖,斬闖天王、九條龍、草上飛、抓山虎、雙翼虎。剿房縣、竹山、南漳賊,戰獅子崖、石漳山,斬一隻虎、滿天飛。已,擊賊洵陽乜家溝,連戰皆捷,獲首功一千有奇。八月敘五峰山破賊功,進右都督。不善馭軍,軍心亦不附,噪於鄖西,渡河以避之,總督陳奇瑜犒慰乃定。奇瑜集諸將討竹山、竹溪諸賊,頻有功。十一月,賊大入河南,命援剿。

八年春,賊陷新蔡,知縣王信罵賊死,追敗賊羅山。是時,賊陷鳳陽,命自黃州速援安慶。及桐城被圍,竟不至。御史錢守廉劾剿賊羅山,殺良冒功,命總督洪承疇核之。四月,承疇至汝州,令戍樊城,防漢江。是月,部將王允成以剋餉鼓噪,殺其二僕。懼,登樓越牆墮地死。

由小校,大小數百戰,所向克捷。以久戍觖望,恣其下淫掠。大學士王應熊以鄉里庇之,益無所憚。其死也,人以為逸罰云。

賀人龍,米脂人。初以守備隸延綏巡撫洪承疇麾下。崇禎四年,承疇受賊降,命人龍勞以酒,伏兵擊斬三百二十人。其冬,張福臻代承疇,遣人龍剿賊黨雄,斬獲二百有奇。明年夏,從福臻擒賊孫守法。其秋,以所部援剿山西。六年春,與總兵尤世祿復遼州。已,敗賊垣曲、絳縣。進都司僉書。又連破賊水頭鎮、花池塞、湯湖村。會山西賊幾盡,乃還陜西。從巡撫陳奇瑜討平延川賊,浮斬一千有奇。奇瑜擢總督,以人龍自隨。

七年四月擊賊隰州,擒剋天虎,進參將。奇瑜追賊鄖、襄、興、漢,人龍並有功。賊軼車箱峽,陷隴州西去,奇瑜遣人龍救之。甫入隴州,李自成復至,環攻。以人龍同里閈,遣其將高傑移書令反,人龍不報。固守兩月,左光先救至,圍始解。十二月敗賊中莊。明年正月,鳳陽陷,總督洪承疇遣人龍馳救,敗賊睢州。進副總兵。承疇以陜西急,率人龍入關。商、洛賊馬光玉等薄西安,距大軍五十里。承疇命人龍入子午谷,邀賊之南;別將劉成功、王永祥邀賊之北;張全昌從咸陽繞興平東。賊以此不敢南遁,盡走武功、扶風,又渡渭走郿縣。承疇追至王渠鎮,賊方掠南山。人龍、成功等與戰,追奔三十里,至大泥峪,賊棄馬登山走。七月,高迎祥、張獻忠掠秦安、清水,人龍偕全昌破之張家川。已而失利,都司田應龍等死。八月,高傑降,承疇令人龍及遊擊孫守法挾之趨富平,乘夜擊敗賊。人龍尋移守延綏。

九年七月從巡撫孫傳庭大破賊盩啡,擒迎祥。九月,惠登相等屯寶雞,承疇遣人龍等往擊,戰於賈家村。追奔,為賊所截。川將曾榮耀等來援,敗去,人龍坐褫官立功。十年,小紅狼圍漢中,瑞王告急。承疇率人龍兵由兩當趨救,賊解去,詔復人龍官。徽、秦逸賊東趨平、鳳,人龍躡至柳林,不利。賊窺西安,人龍禦之,斬獲多。其冬,自成、登相入四川,承疇率人龍等往援。歲暮至廣元,賊已逼成都,自成別由松潘還陜右。

十一年,承疇督人龍等自階、文窮追,自成走入西羌界,人龍與曹變蛟等大戰二十七日。自成引殘卒入塞,竄山中,謀入四川,為人龍及馬科所追。突漢中,扼於左光先。其黨祁總管降,自成幾滅。詳《變蛟傳》。其冬,京師戒嚴,擢人龍總兵官,帥師入衛。人龍所部多降賊,至山西而噪,尋撫定。抵京,與變蛟等奏捷於太平。明年事定,還陜西。其秋,張獻忠、羅汝才叛,謀入陜。人龍及副將李國奇等扼之興安,乃入川東。楊嗣昌檄陜西總督鄭崇儉率人龍、國奇軍會剿。十二月,人龍擊賊,大敗之。

十三年二月與左良玉大破賊瑪瑙山,人龍得一千三百餘級,降賊將二十五人。六月,汝才、登相犯開縣,總兵鄭嘉棟擊之仙寺嶺,人龍擊之馬弱溪,共斬首一千二百。汝才、登相東西走,追之不能及。時賊盡集於川,監軍萬元吉令川將守巴、巫諸隘,人龍、國奇及楚將張應元、汪雲龍、張奏凱專主追擊。及應元軍入夔,營土地嶺,人龍逗留不至,諸軍遂大敗,人龍竟還陜。已而獻忠、汝才陷劍州,趨廣元,將從間道入漢中。人龍拒之陽平、百丈二關,賊乃退。十二月,嗣昌至重慶,三檄人龍會師,不至。

初,嗣昌惡左良玉,許人龍代為平賊將軍。及戰瑪瑙山,良玉功第一,嗣昌語人龍姑待之。人龍大觖望,效良玉所為,不奉約束,嗣昌亦不能制。賊陷瀘州而北,人龍屯小市廂,隔一水不擊。賊遂越成都走漢州德陽,人龍軍大噪而歸。

十四年三月,嗣昌卒,丁啟睿代,令人龍、國奇出當陽,擊敗自成於靈寶山中。人龍子大明戰歿。九月,總督傅宗龍統人龍、國奇軍出關,次新蔡,遇賊孟家莊。將戰,人龍先走,國奇戰不勝,亦走,宗龍遂歿。十五年正月,總督汪喬年出關擊賊,人龍及鄭嘉棟、牛成虎從。至襄城遇賊,復不戰走,喬年亦歿。帝大怒,欲誅之,慮其為變,姑奪職,戴罪視事。及孫傳庭督師陜西,帝授以意。人龍駐咸陽虞禍,曉夜為備。傳庭以人龍家米脂,其宗族多在賊中,未可輕發,在道佯上疏曰:「人龍臣舊將,願貰其罪,俾從臣自效。」帝亦佯許之。人龍稍自安。傳庭至陜,密與巡撫張爾忠謀,以五月朔召人龍計事,數其罪斬之。其部將周國卿將精卒二百人與同黨魏大亨、賀國賢、高進庫等將逃還涇陽取其孥,與賊為亂。爾忠遣參將孫守法先入涇陽,質其妻子。國卿窮,謀斬大亨等以降。爾忠密聞之大亨,遂斬國卿,函送其首。他部將高傑、高汝利、賀勇、董學禮等十四人俱仍故官,一軍乃定。

高傑,米脂人。與李自成同邑,同起為盜。崇禎七年閏八月,總督陳奇瑜遣參將賀人龍救隴州,被圍大困。自成令傑遺書約人龍反,不報。使者歸,先見傑,後見自成。比圍城兩月不拔,自成心疑傑,遣別部將往代,傑歸守營。自成妻邢氏武多智,掌軍資,每日支糧仗。傑過氏營,分合符驗。氏偉傑貌,與之通,恐自成覺,謀歸降。次年八月遂竊邢氏來歸。洪承疇以付人龍,使其遊擊孫守法挾以破賊,取立效為信,自是傑常隸人龍麾下。十三年,張獻忠敗於瑪瑙山,竄興、歸界上,傑隨人龍及副將李國奇大敗之鹽井。

十五年,人龍以罪誅,命傑為實授遊擊。十月,陜西總督孫傳庭至南陽,自成與羅汝才西行逆之。傳庭以傑與魯某為先鋒,遇於塚頭,大戰敗賊,追奔六十里。汝才見自成敗來救,繞出官軍後。後軍左勷望見賊,怖而先奔,眾軍皆奔,遂大潰,傑所亡失獨少。

十六年進副總兵,與總兵白廣恩為軍鋒,兩人皆降將也。廣恩鷙鰲,素不奉約束,而傑尤凶暴。朝廷以傑為自成所切齒,故命隸傳庭辦賊。九月從傳庭克寶豐,復郟縣。時官軍乘勝深入,乏食。降將李際遇通賊,自成帥精騎大至。傳庭問計於諸將,傑請戰,廣恩不可。傳庭以廣恩為怯,廣恩不懌,引所部遁去。官軍接戰,陷伏中。傑登嶺上望之曰:「不可支矣。」亦麾眾退。軍遂大奔,死者數萬。廣恩走汝州不救,傑乃隨傳庭走河北。已而自山西渡河,轉入潼關,廣恩已先至。十一月,自成攻關,廣恩力戰。而傑怨廣恩以寶豐之敗不救己,亦擁眾不肯救。廣恩戰敗,關遂破,傳庭被殺。自成破西安,據之。傑北走延安,賊將李過追傑。傑東走宜川,河冰適合,遂渡,入蒲津以守。賊至,冰解不得渡,乃免。廣恩既敗,走固原,為賊將追及,遂以城降。十七年進傑總兵。帝令總督李化熙率傑兵馳救山西,而蒲州、平陽已陷久,傑退至澤州,沿途大掠,賊遂薄太原。

京師陷,傑南走,福王封傑興平伯,列於四鎮,領揚州,駐城外。傑固欲入城,揚州民畏傑不納。傑攻城急,日掠廂村婦女,民益惡之。知府馬鳴騄、推官湯來賀堅守月餘。傑知不可攻,意稍怠。閣部史可法議以瓜州予傑,乃止。九月命傑移駐徐州,以左中允衛胤文兼兵科給事中監其軍西討。徐州土賊程繼孔被擒至京師,乘李自成亂逃歸,十二月,傑擒斬之。加太子少傅,廕一子,世襲錦衣僉事。

初,傑伏兵要擊黃得功於土橋,得功幾不免,兩鎮遂相仇怨,事見《得功傳》。傑爭揚州時,可法頗為所窘。至是,傑感可法忠,與謀恢復。議調得功與劉澤清二鎮赴邳、宿防河,傑自提兵直趨歸、開,且瞰宛、洛、荊、襄,以為根本。遂具疏上之,語激切。且云:「得功與臣猶介介前事。臣知報君雪恥而已,安能與同列較短長哉!」然得功終不欲為傑後勁,而澤清尤狡橫難任。可法不得已,調劉良佐赴徐與傑為聲援。

順治二年正月,傑抵歸德。總兵許定國方駐睢州,有言其送子渡河者。傑招定國來會,不應。復邀巡撫越其傑、巡按陳潛夫同往睢州,定國始郊逆。其傑諷傑勿入城,傑心輕定國,不聽,遂入城。十一日,定國置酒享傑。傑飲酣,為定國刻行期,且微及送子事。定國益疑,無離睢意。傑固促之行,定國怒,夜伏兵傳炮大呼。其傑等急遁走,傑醉臥帳中未起,眾擁至定國所殺之。先是,傑以定國將去睢,盡發兵戍開封,所留親卒數十人而已。定國偽恭順,多選妓侍傑,而以二妓偶一卒寢。卒盡醉,及聞炮欲起,為二妓所掣不得脫,皆死。明日,傑部下至,攻城,老弱無孑遺。定國走降大清軍。

傑為人淫毒,揚民聞其死,皆相賀。然是行也,進取意甚銳,故時有惜之者。始朝廷許諸鎮與聞國是,故傑屢條奏救降賊者,及請釋武愫於獄,不允。復疏薦吳甡、鄭三俊、金光辰、姜埰、熊開元、金聲、沈正宗等。大抵其時武臣風尚多類此。傑死,贈太子太保,以其子元爵襲興平伯。

劉澤清,曹縣人。以將材授遼東寧、前衛守備,遷山東都司僉書,加參將。崇禎三年,大清兵攻鐵廠,欲據以絕豐潤糧道。援守三屯總兵楊肇基遣澤清來援,未至鐵廠一十五里,遇大兵,力戰,自辰至午不決。得濟師,轉戰至遵化,夾擊,遂得入城。敘功,加二級至副總兵。五年以侵剋軍糧被劾,詔立功衝要地。六年遷總兵。其冬加左都督,恢復登州有功。八年詔統山東兵防漕。九年,京師戒嚴,統兵入衛,令駐新城為南北控扼,復命留守通州。加左都督、太子太師。

十三年五月,山東大饑,民相聚為寇,曹、濮尤甚。帝命澤清會總兵楊御蕃兵剿捕之。八月降右都督,鎮守山東防海。澤清以生長山東,久鎮東省非宜,請辭任。帝令整旅渡河,合諸鎮星馳援剿。

十六年二月,賊圍開封久,澤清赴援。以朱家寨去汴八里,提五千人南渡,倚河為寨,疏水環之,欲以次結八寨達大堤,築甬道,饋饟城中。壁壘未成,賊來爭。相持三日,互有殺傷。澤清即命拔營去,惶擾奔迸,士爭舟,多溺死者。

澤清為人性恇怯,懷私觀望。嘗妄報大捷邀賞賜,又詭稱墮馬被傷,詔賚藥資四十兩。命赴保定剿賊,不從,日大掠臨清。率兵南下,所至焚劫一空。寇氛日急,給事中韓如愈、馬嘉植皆謀奉使南歸。如愈常劾澤清,過東昌,澤清遣人殺之於道,無敢上聞者。

京師陷,澤清走南都,福王以為諸鎮之一,封東平伯,駐廬州。時武臣各占分地,賦入不以上供,恣其所用,置封疆兵事一切不問。與廷臣互分黨援,干預朝政,排擠異已,奏牘紛如,紀綱盡裂,而澤清所言尤狂悖。王初立,即援靖康故事,請以今歲五月改元,又請宥故輔周延儒助餉臟銀。都御史劉宗周劾諸將跋扈狀,澤清遂兩疏劾宗周,且曰:「上若誅宗周,臣即卸職。」朝廷不得已,溫詔解之。又請禁巡按不得拿訪追臟,請法司嚴緝故總督侯恂及其子方域,朝廷皆曲意從之。

順治二年四月,揚州告急,命澤清等往援,而澤清已潛謀輸款矣。大清惡其反覆,磔誅之。

澤清頗涉文藝,好吟詠。嘗召客飲酒唱和。幕中蓄兩猿,以名呼之即至。一日,宴其故人子,酌酒金甌中,甌可容三升許,呼猿捧酒跪送客。猿猙獰甚,客戰掉,逡巡不敢取。澤清笑曰:「君怖耶?」命取囚撲死階下,剜其腦及心肝,置甌中,和酒,付猿捧之前。飲酹,顏色自若。其兇忍多此類。

祖寬,遼東人。少有勇力。給侍祖大壽家,從軍有功,累官寧遠參將。部卒多塞外降人,所向克捷。

崇禎五年七月,叛將李九成等圍萊州急,詔發關外兵討之。寬與靳國臣、祖大弼、張韜率兵抵昌邑。巡撫朱大黃典獲賊書,約寬等為內應,以示寬等。皆誓滅賊以自明,乃用寬、國臣為前鋒。寬至沙河與賊遇,眾寡不敵,稍卻。會國臣至,拔刀大呼直前,寬、大弼、韜咸殊死戰,大敗賊兵,逐北抵城下,立解萊州圍。是月晦,進兵黃縣。賊傾巢出戰,寬等復大敗之,遂與劉澤清等築長圍以困登州。明年二月,賊始平。語詳《大典傳》。寬以解圍功,進都督僉事。再敘功,世廕外衛副千戶,進副總兵。

八年秋,命為援剿總兵官,督關外兵三千討流賊。十月至河南,巡撫陳必謙、監紀推官湯開遠令與左良玉抵靈寶,至則挫張獻忠於焦村。無何,高迎祥、李自成至,與獻忠合攻閿鄉。寬赴救,賊解而趨靈寶,斷良玉、寬軍不相應,遂東陷陜州,攻洛陽。良玉、寬至,迎祥、自成、獻忠皆走。良玉追迎祥,而寬分擊獻忠,夜督副將祖克勇等趨葛家莊,黎明遇賊,大破之。賊奔嵩縣九皋山,寬伏二軍於山溝誘之。賊趨下,伏發,斬馘九百有奇。尋與副將劉肇基、羅岱遇賊汝州圪料鎮,復大敗賊,伏尸二十餘里,斬馘千六百有奇。獻忠憤,合迎祥、自成兵,與寬戰龍門、白沙,截官軍為二。寬自斷後,士卒殊死鬥,自晨至夜分,復大捷,斬馘一千有奇。迎祥、自成乃走窺光州,寬督副將李輔明躡其後。賊走攻確山,寬等馳救,大破之,斬馘五百八十有奇。自成等遂東走廬州,攻圍七晝夜。明年正月,寬等至,賊奔全椒,遂圍滁州。南京太僕卿李覺斯、知州劉大鞏力禦之。而寬等軍至,奮擊大呼,諸軍無不一當百,自晨至晡,賊大敗。從城東五里追至關山之硃龍橋,橫屍枕藉,水為不流。二月,又從總理盧象昇破賊七頂山,殲自成精卒殆盡。象昇移軍南陽,命寬備鄧州。會賊渡漢江,入鄖、襄,餘眾三萬匿內鄉、淅川山中。象昇命寬與祖大樂等入山搜討。

邊軍強憨,性異他卒,不可以法繩。往時官軍多關中人,與賊鄉里,臨陣相勞苦,拋生口,棄輜重,即縱之去,謂之「打活仗」。邊軍不通言語,逢賊即殺,故多勝。然所過焚廬舍,淫婦女,恃功不戢;又利野戰,憚搜山;且見賊遠竄,非旬朔可定,自以為客將,無持久心。寬卒方過河,噪而逸。象升激勸再三,始聽命。至黨子口,仍按甲不行。而總兵李重鎮素恇怯,冀卸責,眾益思歸。象升乃力陳入山搜剿之難,請令寬、重鎮赴關中討賊。會總督洪承疇亦請之,寬等遂移軍陜西,隸承疇麾下。八月,京師被兵,召入衛。金錄滁州功,進右都督,賚銀幣。事定,命赴寧遠協守。

十一年冬,詔寬率師援畿輔。及山東告急,寬逗遛。明年正月,濟南失守,褫職被逮,坐失陷籓封,竟棄市。

寬敢戰有功,稱驍將。性剛使氣,不為文吏所喜,卒致大闢,莫為論救。

贊曰:左良玉以驍勇之材,頻殲劇寇,遂擁強兵,驕亢自恣,緩則養寇以貽憂,急則棄甲以致潰。當時以不用命罪諸將者屢矣,而良玉偃蹇僨事,未正刑章,姑息釀患,是以卒至稱兵犯闕而不顧也。高傑、祖寬皆剛悍難馴,恃功不戢,而傑尤為兇驁。然傑被戕於銳意進取之時,寬受誅於力戰赴援之後,死非其罪,不能無遺憾焉。


\end{pinyinscope}