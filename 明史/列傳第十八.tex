\article{列傳第十八}

\begin{pinyinscope}
吳良康茂才丁德興耿炳文郭英華雲龍韓政仇成張龍吳復周武胡海張赫華高張銓何真

吳良,定遠人。初名國興,賜名良。雄偉剛直。與弟禎俱以勇略聞。從太祖起濠梁,並為帳前先鋒。良能沒水偵探,禎每易服為間諜。禎別有傳。良從取滁、和,戰采石,克太平,下溧水、溧陽,定集慶,功多。又從徐達克鎮江,下常州,進鎮撫,守丹陽。與趙繼祖等取江陰。張士誠兵據秦望山,良攻奪之,遂克江陰。即命為指揮使守之。

時士誠全據吳,跨淮東、浙西,兵食足。江陰當其要衝,枕大江,扼南北襟喉,士誠數以金帛啖將士,窺釁。太祖諭良曰:「江陰,我東南屏蔽,汝約束士卒,毋外交,毋納逋逃,毋貪小利,毋與爭鋒,惟保境安民而已。」良奉命惟謹,備禦修飭。以敗敵功,進樞密院判官。士誠大舉兵來寇,艨艟蔽江,其將蘇同僉駐君山,指畫進兵。良遣弟禎出北門與戰,而潛遣元帥王子明帥壯士馳出南門。合擊,大敗之,俘斬甚眾。敵宵遁。尋復寇常州,良遣兵從間道殲其援兵於無錫。當是時,太祖數自將爭江、楚上流,與陳友諒角,大軍屢出,金陵空虛。士誠不敢北出侵尺寸地,以良在江陰為屏蔽也。

良仁恕儉約,聲色貨利無所好。夜宿城樓,枕戈達旦。訓將練兵,常如寇至。暇則延儒生講論經史,新學宮,立社學。大開屯田,均徭省賦。在境十年,封疆宴然。太祖常召良勞曰:「吳院判保障一方,我無東顧憂,功甚大,車馬珠玉不足旌其勞。」命學士宋濂等為詩文美之,仍遣還鎮。尋大發兵取淮東,克泰州。士誠兵復出馬馱沙,侵鎮江。巨艦數百,溯江而上。良戒嚴以待。太祖親督大軍禦之。士誠兵遁,追至浮子門。良出兵夾擊,獲卒二千。太祖詣江陰勞軍,周巡壁壘,歎曰:「良,今之吳起也!」吳平,加昭勇大將軍、蘇州衛指揮使,移鎮蘇州。武備益修,軍民輯睦。進都督僉事,移守全州。洪武三年進都督同知,封江陰侯,食祿千五百石,予世券。

四年討靖州、綏寧諸蠻。五年,廣西蠻叛,副征南將軍鄧愈帥平章李伯升出靖州討之。數月,盡平左右兩江及五溪之地,移兵入銅鼓、五開,收潭溪,開太平,殲清洞、崖山之眾於銅關鐵寨。諸蠻皆震懾,內附,粵西遂平。八年督田鳳陽。十二年,齊王封青州。王妃,良女也,遂命良往建王府。十四年卒於青,年五十八。贈江國公,謚襄烈。

子高嗣侯,屢出山西、北平、河南練兵,從北征,帥蕃軍討百夷。二十八年,有罪調廣西,從征趙宗壽。燕師起,高守遼東,與楊文數出師攻永平。燕王謀去高,曰:「高雖怯,差密,文勇而無謀,去高,文無能為也。」乃遺二人書,盛譽高,極詆文,故易其函授之。二人得書,並以聞。建文帝果疑高,削爵徙廣西,獨文守遼東,竟敗。永樂初,復召高鎮守大同,上言備邊方略。八年,帝北征班師,高稱疾不朝,被劾,廢為庶人,奪券。洪熙元年,帝見高名,曰:「高往年多行無禮,其謫戍海南。」高已死,徙其家,會赦得釋。宣德十年,子昇乞嗣,不許。

康茂才,字壽卿,蘄人。通經史大義。事母孝。元末寇亂陷蘄,結義兵保鄉里。立功,自長官累遷淮西宣慰司、都元帥。

太祖既渡江,將士家屬留和州。時茂才移戍采石,扼江渡。太祖遣兵數攻之,茂才力守。常遇春設伏殲其精銳。茂才復立寨天寧洲,又破之。奔集慶,太祖克集慶,乃帥所部兵降。太祖釋之,命統所部從征。明年授秦淮翼水軍元帥,守龍灣。取江陰馬馱沙,敗張士誠兵,獲其樓船。從廖永安攻池州,取樅陽。太祖以軍興,民失農業,命茂才為都水營田使,仍兼帳前總制親兵左副指揮使。

陳友諒既陷太平,謀約張士誠合攻應天。太祖欲其速來,破之。知茂才與友諒有舊,命遣僕持書,紿為內應。友諒大喜,問:「康公安在?」曰:「守江東木橋。」使歸,太祖易橋以石。友諒至,見橋,愕然,連呼「老康」,莫應。退至龍灣,伏兵四起。茂才合諸將奮擊,大破之。太祖嘉茂才功,賜賚甚厚。明年,太祖親征友諒,茂才以舟師從克安慶,破江州,友諒西遁。遂下蘄州、興國、漢陽。沿流克黃梅寨,取瑞昌,敗友諒八指揮,降士卒二萬人。遷帳前親兵副都指揮使。攻左君弼廬州,未下。從援南昌,戰彭蠡,友諒敗死。從征武昌,皆有功。進金吾侍衛親軍都護。從大將軍徐達再攻廬州,克之,取江陵及湖南諸路。改神武衛指揮使,進大都督府副使。士誠攻江陰,太祖自將擊之。比至鎮江,士誠已焚瓜洲遁。茂才追北至浮子門。吳軍遮海口,乘潮來薄。茂才力戰,大敗之。搗淮安馬騾港,拔其水寨,淮安平。尋拔湖州,進逼平江。士誠遣銳卒迎斗,大戰尹山橋。茂才持大戟督戰,盡覆敵眾。與諸將合圍其城,軍齊門。平江下,還取無錫。遷同知大都督府事兼太子右率府使。

洪武元年,從大將軍經略中原,取汴、洛,留守陜州。規運饋餉,造浮橋渡師。招來絳、解諸州,扼潼關,秦兵不敢東向。茂才善撫綏,民立石頌德焉。三年復從大將軍征定西,取興元。還軍道卒。追封蘄國公,謚武康。

子鐸,年十歲,入侍皇太子讀書大本堂。以父功封蘄春侯,食祿一千五百石,予世券。督民墾田鳳陽。帥兵征辰州蠻,平施、疊諸州。從大將軍達北征。又從征南將軍傅友德征雲南,克普定,破華楚山諸寨。卒於軍,年二十三。追封蘄國公,謚忠愍。

子淵幼未襲,授散騎舍人。已,坐事革冠服,勒居山西,遂不得嗣。弘治末,錄茂才後為世襲千戶。

丁德興,定遠人。歸太祖於濠。偉其狀貌,以「黑丁」呼之。從取洪山寨,以百騎破賊數千,盡降其眾。從克滁、和,敗青山盜。從渡江,拔采石,取太平,分兵取溧水、溧陽,皆先登。從破蠻子海牙水寨,搗方山營,擒陳兆先,下集慶,取鎮江。以功進管軍總管。下金壇、廣德、寧國。從平常州。擢左翼元帥。寧國復叛,從胡大海復之。分兵下江陰,取徽州、石埭、池州、樅陽,攻江州,移兵擊安慶。所向皆捷。復援江陰,略江西傍近州縣,攻雙刀趙,挫其鋒。時徐達、邵榮攻宜興,久不下,太祖遣使謂曰:「宜興城西通太湖口,士誠餉道所由,斷其餉則必破。」達乃遣德興絕太湖口,而並力急攻,城遂拔。論功授鳳翔衛指揮使。

陳友諒犯龍江,德興軍於石灰山,力戰,擊敗之。遂從征友諒,搗安慶,克九江,援安豐,敗呂珍,走左君弼。從戰鄱陽,平武昌,克廬州,略定湖南衡州諸郡。又從大將軍收淮東,徵浙西,敗士誠兵於舊館。下湖州,圍平江。卒於軍。贈都指揮使。洪武元年追封濟國公,列祀功臣廟。子忠,龍江衛指揮使,予世襲。

耿炳文,濠人。父君用,從太祖渡江,積功為管軍總管。援宜興,與張士誠兵爭柵,力戰死。炳文襲職,領其軍。取廣德,進攻長興,敗士誠將趙打虎,獲戰船三百餘艘,擒其守將李福安等,遂克長興。長興據太湖口,陸通廣德,與宣、歙接壤,為江、浙門戶。太祖既得其地,大喜,改為長安州,立永興翼元帥府,以炳文為總兵都元帥,守之。溫祥卿者,多智數。避亂來歸,炳文引入幕府,畫守禦計甚悉。張士誠左丞潘元明、元帥嚴再興帥師來爭。炳文奮擊,大敗去。久之,士誠復遣司徒李伯昇帥眾十萬,水陸進攻。城中兵七千,太祖患之,命陳德、華高、費聚往援。伯昇夜劫營,諸將皆潰。炳文嬰城固守,攻甚急,隨方禦之,不解甲者月餘。常遇春復帥援兵至,伯升棄營遁,追斬五千餘人。其明年,改永興翼元帥府為永興衛親軍指揮使司,以炳文為使。已而士誠大發兵,遣其弟士信復來爭。炳文又敗之,獲其元帥宋興祖。士信憤甚,益兵圍城。炳文與費聚出戰,又大敗之。長興為士誠必爭地,炳文拒守凡十年,以寡禦眾,大小數十戰,戰無不勝,士誠迄不得逞。大軍伐士誠,炳文將所部克湖州,圍平江。吳平,進大都督府僉事。

從征中原,克山東沂、嶧諸州。下汴梁,徇河南,扈駕北巡。已,又從常遇春取大同,克晉、冀。從大將軍徐達征陜西,走李思齊、張思道,即鎮其地。浚涇陽洪渠十萬餘丈,民賴其利。尋拜秦王左相都督僉事。

洪武三年,封長興侯,食祿千五百石,予世券。十四年,從大將軍出塞,破元平章乃兒不花於北黃河。十九年從潁國公傅龍德征雲南,討平曲靖蠻。二十一年從永昌侯藍玉北征,至捕魚兒海。二十五年帥兵平陜西徽州妖人之亂。三十年以征西將軍擒蜀寇高福興,俘三千人。

始,炳文守長興,功最高,太祖榜列功臣,以炳文附大將軍達為一等。及洪武末年,諸公、侯且盡,存者惟炳文及武定侯郭英二人;而炳文以元功宿將,為朝廷所倚重。

建文元年,燕王兵起。帝命炳文為大將軍,帥副將軍李堅、寧忠北伐,時年六十有五矣。兵號三十萬,至者惟十三萬。八月次真定,分營滹沱河南北。都督徐凱軍河間,潘忠、楊松駐鄚州,先鋒九千人駐雄縣。值中秋,不設備,為燕王所襲,九千人皆死。忠等來援,過月漾橋,伏發水中。忠、松俱被執,不屈死。鄚州陷。而炳文部將張保者降燕,備告南軍虛實。燕王縱保歸,使張雄、鄚敗狀,謂:「北軍且至。」於是炳文移軍盡渡河,並力當敵。軍甫移,燕兵驟至,循城蹴擊。炳文軍不得成列,敗入城。爭門,門塞,蹈藉死者不可數計。燕兵遂圍城。炳文眾尚十萬,堅守不出。燕王知炳文老將,未易下,越三日,解圍還。而帝驟聞炳文敗,憂甚。太常卿黃子澄遂薦李景隆為大將軍,乘傳代炳文。比至軍,燕師已先一日去。炳文歸,景隆代將,竟至於敗。

燕王稱帝之明年,刑部尚書鄭賜、都御史陳瑛劾炳文衣服器皿有龍鳳飾,玉帶用紅鞓,僭妄不道。炳文懼,自殺。

子璇,前軍都督僉事。尚懿文太子長女江都公主。炳文北伐,璇嘗勸直搗北平。炳文受代歸,不復用,璇憤甚。永東初,杜門稱疾,坐罪死。

璇弟瓛,後軍都督僉事。與江陰侯吳高、都指揮楊文帥遼東兵圍永平,不克,退保山海關。高被間,徙廣西。文守遼東,瓛數請攻永平以動北平,文不聽。後與弟尚寶司卿瑄,皆坐罪死。

郭英,鞏昌侯興弟也。年十八,與興同事太祖。親信,令值宿帳中,呼為「郭四」。從克滁、和、采石、太平,征陳友諒,戰鄱陽湖,皆與有功。從征武昌,陳氏驍將陳同僉持槊突入,太祖呼英殺之,衣以戰袍。攻岳州,敗其援兵,還克廬州、襄陽。授驍騎衛千戶。克淮安、濠州、安豐,進指揮僉事。從徐達定中原,又從常遇春攻太原,走擴廓,下興州、大同。至沙凈州渡河。取西安、鳳翔、鞏昌、慶陽,追敗駕宗哲於亂山,遷本衛指揮副使。進克定西,討察罕腦兒。克登寧州,斬首二千級,進河南都指揮使。時英女弟為寧妃,英將赴鎮,命妃餞英於第,賜白金二十罌,廄馬二十匹。在鎮綏輯流亡,申明約束,境內大治。九年移鎮北平。十三年召還,進前軍都督府僉事。

十四年,從潁川侯傅友德征雲南,與陳桓、胡海分道進攻赤水河路。久雨,河水暴漲。英斬木為筏,乘夜濟。比曉,抵賊營,賊大驚潰。擒烏撒並阿容等。攻克曲靖、陸涼、越州、關索嶺、椅子寨。降大理、金齒、廣南,平諸山寨。十六年復從友德平蒙化、鄧川,濟金沙,取北勝、麗江。前後斬首一萬三千餘級,生擒二千餘人,收精甲數萬,船千餘艘。十七年論平雲南功,封武定侯,食祿二千五百石,予世券。

十八年,加靖海將軍,鎮守遼東。二十年從大將軍馮勝出金山,納哈出降,進征虜右副將軍。從藍玉至捕魚兒海。師還,賞賚甚厚,遣還鄉。明年召入京,命典禁兵。三十年副征西將軍耿炳文備邊陜西,平沔縣賊高福興。及還,御史裴承祖劾英私養家奴百五十餘人,又擅殺男女五人。帝弗問,僉都御史張春等執奏不已,乃命諸戚里大臣議其罪。議上,竟宥之。建文時,從耿炳文、李景隆伐燕,無功。靖難後,罷歸第。永樂元年卒,年六十七。贈營國公,謚威襄。

英孝友,通書史,行師有紀律,以忠謹見親於太祖。又以寧妃故,恩寵尤渥,諸功臣莫敢望焉。

子十二人。鎮,尚永嘉公主。銘,遼府典寶。鏞,中軍右都督。女九人,二為遼郢王妃。女孫為仁宗貴妃,銘出也,以故銘子玹得嗣侯。宣德中,玹署宗人府事,奪河間民田廬,又奪天津屯田千畝,罪其奴而宥玹。英宗初,永嘉公主乞以其子珍嗣侯。珍,英嫡孫也,授錦衣衛指揮僉事。玹卒,子聰與珍爭嗣,遂並停襲,亦授聰如珍官。天順元年,珍子昌以詔恩得襲,聰爭之不得。昌卒,子良當嗣,聰又言良非昌子,復停嗣,授指揮僉事。以屢乞嗣,下獄,尋釋復官。既而郭宗人共乞擇英孫一人嗣英爵。廷臣皆言良本英嫡孫,宜嗣侯。詔可。正德初卒。子勛嗣。

勛桀黠有智數,頗涉書史。正德中,鎮兩廣,入掌三千營。世宗初,掌團營。大禮議起,勛知上意,首右張璁,世宗大愛幸之。勛怙寵,頗驕恣。大學士楊一清惡之,因其賕請事覺,罷營務,奪保傅官階。一清罷,仍總五軍營,董四郊興造。明年督團營。十八年兼領後府。從幸承天,請以五世祖英侑享太廟。廷臣持不可,侍郎唐胄爭尤力。帝不聽,英竟得侑享。其明年,獻皇稱宗,入太廟,進勛翊國公,加太師。

先是,妖人李福達自言能化藥物為金銀。勛與相暱。福達敗,力持其獄,廷臣多得罪者。至是復進方士段朝用,云以其所化金銀為飲食器,可不死。帝益以為忠。給事中戚賢劾勛擅作威福,網利虐民諸事。李鳳來等復以為言。下有司勘,勛京師店舍多至千餘區。副都御史胡守中又劾勛以族叔郭憲理刑東廠,肆虐無辜。帝置勿治。會帝用言官言,給勛敕,與兵部尚書王廷相、遂安伯陳譓同清軍役。敕具,勛不領。言官劾其作威植黨。勛疏辯,有「何必更勞賜敕」語。帝乃大怒,責其「強悖無人臣禮」。於是給事中高時盡發勛奸利事,且言交通張延齡。帝益怒,下勛錦衣獄。二十年九月也。尋諭鎮撫司勿加刑訊。奏上,當勛死罪。帝令法司覆勘。而給事中劉大直復勘勛亂政十二罪,請併治。法司乃盡實諸疏中罪狀,當勛罪絞。帝令詳議。法司更當勛不軌罪斬,沒入妻孥田宅。奏上,留中不下。帝意欲寬勛,屢示意指。而廷臣惡勛甚,謬為不喻者,更坐勛重辟。明年考察言官,特旨貶高時二級,以風廷臣,廷臣終莫為勛請。其冬,勛死獄中。帝憐之,責法司淹繫。褫刑部尚書吳山職,侍郎都御史以下鐫降有差,而免勛籍沒,僅奪誥券而已。

自明興以來,勳臣不與政事。惟勛以挾恩寵、擅朝權、恣為姦慝致敗。勛死數年,其子守乾嗣侯,傳至曾孫培民。崇禎末,死於賊。

華雲龍,定遠人。聚眾居韭山。太祖起兵,來歸。從克滁、和,為千夫長。從渡江,破采石水寨及方山營。下集慶路,生擒元將,得兵萬人,克鎮江,遷總管。攻拔廣德,戰舊館,擒湯元帥,進右副元帥。龍江之役,雲龍伏石灰山,接戰,殺傷相當。雲龍躍馬大呼,搗其中堅,遂大敗友諒兵,乘勝復太平。從下九江、南昌,分兵攻下瑞州、臨江、吉安。從援安豐,戰彭蠡,平武昌。累功至豹韜衛指揮使。從徐達帥兵取高郵,進克淮安,遂命守之,改淮安衛指揮使。尋攻嘉興,降吳將宋興。圍平江,軍於胥門。

從大軍北征,徇下山東郡縣,與徐達會帥通州,進克元都。擢大都督府僉事,總六衛兵留守兼北平行省參知政事。踰年,攻下雲州,獲平章火兒忽答、右丞哈海。進都督同知,兼燕王左相。洪武三年冬,論功封淮安侯,祿一千五百石,予世券。雲龍上言:「北平邊塞,東自永平、薊州,西至灰嶺下,隘口一百二十一,相去可二千二百里。其王平口至官坐嶺,隘口九,相去五百餘里。俱衝要,宜設兵。紫荊關及蘆花山嶺尤要害,宜設千戶守禦所。」又言:「前大兵克永平,留故元八翼軍士千六百人屯田,人月支糧五斗,所得不償費。宜入燕山諸衛,補伍操練。」俱從之。行邊至雲州,襲元平章僧家奴營於牙頭,突入其帳擒之,盡俘其眾。至上都大石崖,攻克劉學士諸寨,驢兒國公奔漠北。自是無內犯者,威名大著。建燕邸,增築北平城,皆其經畫。洪武七年,有言雲龍據元相脫脫第宅,僭用故元宮中物。召還,命何文輝往代。未至京,道卒。

子中襲。李文忠之卒也,中侍疾進藥,坐貶死。二十三年追論中胡黨,爵除。

韓政,睢人。嘗為義兵元帥,帥眾歸太祖,授江淮行省平章政事。李濟據濠州,名為張士誠守,實觀望。太祖使右相國李善長以書招之,不報。太祖歎曰:「濠,吾家也,濟如此,我有國無家可乎!」乃命政帥指揮顧時以雲梯炮石四面攻濠。濟度不能支,始出降。政歸濟於應天。太祖大悅,以時守濠州。

政從徐達攻安豐,扼其四門,潛穴城東龍尾壩,入其城二十餘丈。城壞,遂破之。元將忻都、竹貞、左君弼皆走。追奔四十餘里,擒都。俄而貞引兵來援,與戰城南門,再破走之。淮東、西悉平。已,從大軍平吳。又從北伐,降梁城守將盧斌。分兵扼黃河,斷山東援軍,遂取益都、濟寧、濟南,皆有功。克東平,功尤多,改山東行省平章政事。以師會大將軍於臨清,檄政守東昌。既下大都,命政分兵守廣平。政遂諭降白土諸寨。移守彰德,下蟻尖寨。蟻尖者,在林慮西北二十里,為元右丞吳庸、王居義、小鎖兒所據。大將軍之北伐也,遣將士收復諸山寨,降者相繼,蟻尖獨恃險不下。至是兵逼之,庸誘殺居義及小鎖兒以降,得士卒萬餘人。尋調徵陜西,還兵守禦河北。洪武三年封東平侯,祿千五百石,予世券。移鎮山東。未幾,復移河北。招撫流民,復業甚眾。從左副將軍李文忠搗應昌,至臚朐河。文忠深入,令政守輜重。還,命巡河南、陜西。再從信國公湯和練兵於臨清。十一年二月卒,帝親臨其喪。追封鄆國公。

子勳襲。二十六年坐藍黨誅,爵除。

仇成,含山人。初從軍充萬戶,屢遷至秦淮翼副元帥。太祖攻安慶,敵固守不戰。廖永忠、張志雄破其水寨,成以陸兵乘之,遂克安慶。初,元左丞余闕守安慶,陳友諒將趙普勝陷之。友諒既殺普勝,元帥餘某者襲取之。張定邊復來犯,餘帥走死。至是以成為橫海指揮同知,守其地。時左君弼據廬州,羅友賢以池州叛,無為知州董曾陷死,四面皆賊境。成撫集軍民,守禦嚴密,漢兵不敢東下。從征鄱陽,殲敵涇江口,功最。征平江,敗張士誠兵於城西南。洪武三年,僉大都督府事,鎮遼東。久之,以屯戍無功,降永平衛指揮使。尋復官。十二年論藍玉等征西功,當封。帝念成舊勳,先封為安慶侯,歲祿二千石。二十年充征南副將軍,討平容美諸峒。復從大軍征雲南,功多,予世券,加祿五百石。二十一年七月,有疾。賜內昷,手詔存問。卒,贈皖國公,謚莊襄。子正襲爵。

張龍,濠人。從渡江,定常州、寧國、婺州,皆有功。從征江州,為都先鋒。平武昌,授花槍所千戶。從平淮東,守禦海安。與張士誠將戰於海口,擒彭元帥,俘其卒數百。進攻通州,擊斬賊將。擢威武衛指揮僉事。從平山東、河南。大兵克潼關,以龍為副留守。洪武三年調守鳳翔,改鳳翔衛指揮。賀宗哲悉眾圍城,龍固守。宗哲攻北門,龍出兵搏戰,矢傷右脅,不為動。遂大敗之。進克鳳州,擒李參政等二十餘人。大將軍達入沔州,遣龍別將一軍,由鳳翔入連雲棧,攻興元,降其守將劉思忠。蜀將吳友仁來犯,龍擊卻之。友仁復悉兵薄城,大治攻具。龍從北門突出,繞友仁軍後,敵盡棄甲仗走,自是不復窺興元。召僉大都督府事。十一年副李文忠征西番洮州。論功,封鳳翔侯,祿二千石,世指揮使。復從傅友德征雲南,鎮七星關,破大理、鶴慶,平諸洞蠻。加祿五百石,予世券三十年。二十年從馮勝出金山,降納哈出。明年,勝調降軍征雲南,次常德,叛去。龍追至重慶,收捕之。二十三年春同延安侯唐勝宗督屯田於平越、鎮遠、貴州,議置龍里衛。都勻亂,佐藍玉討平之。以老疾請告。三十年卒。

子麟尚福清公主,授駙馬都尉。孫傑侍公主京師。永樂初,失侯。傑子嗣,宣德十年,援詔恩乞嗣。吏部言:龍侯不嗣者四十年,不許。

吳復,字伯起,合肥人。少負勇略。元末,集眾保鄉里。歸太祖於濠,從克泗、滁、和、采石、太平,累官萬戶。從破蠻子海牙水寨,定集慶。從徐達攻鎮江,斬元平章定定。下丹陽、金壇,克常州,進統軍元帥。徇江陰、無錫,還守常州。張士誠兵奄至,力戰,敗之。追奔至長興,連敗之於高橋、太湖及忠節門,士誠奪氣。從援安豐,平武昌。從徐達克廬州,下漢、沔、荊諸郡縣。授鎮武衛指揮同知,守沔陽。從常遇春下襄陽,別將破安陸,擒元同僉任亮,遂守之。克汝州、魯山。

洪武元年,授懷遠將軍、安陸衛指揮使。悉平鄖、均、房、竹諸山寨之不附者。三年從大將軍征陜西,敗擴廓,擒其將。又敗擴郭於秦州。征吐番,克河州。援漢中,拔南鄭。明年從傅友德平蜀。又明年從鄧愈平九溪、辰州諸蠻,克四十八洞,還守安陸。七年進大都督府僉事。巡北平還,授世襲指揮使。十一年從沐英再征西番,擒三副使,得納鄰哈七站之地。明年,師還,論功封安陸侯,食祿二千石。

十四年,從傅友德征雲南,克普定,城水西。充總兵官,剿捕諸蠻。遂由關索嶺開箐道,取廣西。十六年克墨定苗,至吉剌堡,築安莊、新城,平七百房諸寨,斬獲萬計,轉餉盤江。是年十月,金瘡發,卒於普定。追封黔國公,謚威毅,加祿五百石,予世券。

復臨陣奮發,衝犯矢石,體無完膚。平居恂恂,口不言征伐事。在普定買妾楊氏,年十七。復死,視殮畢,沐浴更衣,自經死。封貞烈淑人。

子傑嗣。屢出山、陜、河南、北平,練兵從征。二十八年,有罪,從征龍州,建功自贖。建文中,帥師援真定,戰白溝河,失律,謫南寧衛指揮使。永樂元年,子璟乞嗣。正統間,再三乞,皆不許。弘治六年,璟孫鐸援詔乞嗣,亦不許。十八年錄復子孫世職千戶。

初,與復以征西番功侯者,又有周武。武,開州人,從定江東,滅漢,收淮東,平吳,積功為指揮僉事。從定中原,進都督僉事。洪武十一年以參將從沐英討西番朵甘,功多。師還,封雄武侯,祿二千石,世指揮使。出理河南軍務,巡撫北邊。二十三年卒,贈汝國公,謚勇襄。

胡海,字海洋,定遠人。嘗入土豪赤塘王總管營,自拔來歸,授百戶。從敗元將賈魯兵,克泗、滁,進萬戶。從渡江,拔蠻子海牙水寨,破陳埜先兵,從取集慶、鎮江。敗元將謝國璽於寧國,選充先鋒。從大軍圍湖州,墮其東南門月城。從攻宜興,下婺州,鏖戰紹興,生得賊四百餘人,進都先鋒。又從戰龍江,克安慶,與漢人相持,八戰,皆大捷,遂入江州。從徐達攻廬州,皆有功。

海驍勇,屢戰屢傷,手足胸腹間金痍皆遍,而鬥益力。士卒從之者無不激勵自效。太祖壯之,授花槍上千戶。復從大軍克荊、澧、衡、潭,擢寶慶衛指揮僉事,遷指揮使,命鎮益陽。從平章楊璟征湖南、廣西未下郡縣。由祁陽進圍永州,與守兵戰於東鄉橋,生得千、萬戶四人,以夜半先登拔之。抵靖江,戰南門,生得萬戶二人。夜四鼓,自北門八角亭先登,功最,命為左副總兵。剿平左江上思蠻。調征蜀,克龍伏隘、天門山及溫湯關,予世襲指揮使,仍鎮益陽。武岡、靖州、五開諸苗蠻先後作亂,悉捕誅首亂而撫其餘眾,遷都督僉事。十四年從征雲南,由永寧趨烏撒,進克可渡河。與副將軍沐英會師攻大理,敵悉眾扼上、下關。定遠侯王弼自洱水東趨上關,英帥大軍趨下關,而遣海以夜四鼓取石門。間道渡河,繞點蒼山後,攀大樹緣崖而上,立旂幟。英士卒望見,皆踴躍大呼,敵眾驚擾。英遂斬關入。海亦麾山上軍馳下,前後夾攻,敵悉潰走。

十七年,論功封東川侯,祿二千五百石,予世券。踰三年,以左參將從征金山。又二年,以征南將軍討平澧州九溪諸蠻寇。師還,乞歸鄉里,厚賚金帛以行。二十四年七月,病疽卒,年六十三。

長子斌,龍虎衛指揮使,從征雲南。過曲靖,猝遇寇,中飛矢卒。贈都督同知。次玉,坐藍黨死。次觀,尚南康公主,為駙馬都尉,未嗣卒。宣德中,公主乞以子忠嗣。詔授孝陵衛指揮僉事,予世襲。

張赫,臨淮人。江淮大亂,團義兵以捍鄉里。嘉山繆把頭招之,不往。聞太祖起,帥眾來附。授千戶,以功進萬戶。從渡江,所至攻伐皆預,以功擢常春翼元帥,守禦常州。尋從戰鄱陽,攻武昌。已,又從大將軍伐張士誠,進圍平江。諸將分門而軍,赫軍閶門。士誠屢出兵突戰,屢挫其鋒。又從大軍克慶元,並下溫、台。洪武元年,擢福州衛都指揮副使,進本衛同知。復命署都指揮使司事。是時,倭寇出沒海島中,乘間輒傅岸剽掠,沿海居民患苦之。帝數遣使齎詔書諭日本國王,又數絕日本貢使,然竟不得倭人要領。赫在海上久,所捕倭不可勝計。最後追寇至琉球大洋,與戰,擒其魁十八人,斬首數十級,獲倭船十餘艘,收弓刀器械無算。帝偉赫功,命掌都指揮印。尋調興化衛。召還,擢大都督府僉事。會遼東漕運艱,軍食後期,帝深以為慮。以赫習海道,命督海運事。久之,封航海侯,予世券。前後往來遼東十二年,凡督十運,勞勩備至,軍中賴以無乏。病卒,追封恩國公,謚莊簡。

子榮,從征雲南有功,為水軍右衛指揮使。孫金盬,福建都指揮使。永樂中,留鎮交阯。

華高,和州人。與俞通海等以巢湖水師來附。從克太平,授總管。從破采石、方山兵。下集慶、鎮江,遷秦淮翼元帥。與鄧愈徇廣德。守將嚴兵城下,高以數騎挑戰,元兵堅壁不動。高衝擊大破之,遂取其城,得兵萬人,糧數千斛。從平常州,進僉行樞密院事。副俞通海擊破趙普勝柵江營。再敗陳友諒。援長興,克武昌。授湖廣行省左丞。帥舟師從克淮東,收浙西。進行省平章政事。洪武三年封廣德侯,歲祿六百石。

高性怯,且無子,請得宿衛。有所征討,輒稱疾不行。令練水師,復以不習辭。帝以故舊優容之。時諸勛臣多出行邊,惟高不遣。最後繕廣東邊海城堡,高請行。帝曰:「卿復自力,甚善。」四年四月事竣。至瓊州卒。初,有言高殖利者,故歲祿獨薄。至是貧不能葬。帝憐之,命補支祿三百石。以無子,納誥券墓中。贈巢國公,謚武莊。授從子岳指揮僉事。

張銓,定遠人。從取太平,定集慶、鎮江、常州、婺州。搗江州,戰鄱陽湖,取鄂渚。收淮東,平吳。累功為指揮僉事。從取中原、燕、晉、秦、蜀,進都督僉事。使建齊王府,事竣,副江夏侯周德興征五溪蠻。已而水盡源、通塔平、散毛諸洞酋作亂,復副德興討平之。從征雲南,由永寧克烏撒。久之,復從傅友德平烏撒及曲靖、普定、龍海、孟定諸蠻。洪武二十三年封永定侯,食祿千五百石,世指揮使。

何真,字邦佐,東莞人。少英偉,好書劍。元至正初,為河源縣務副使,轉淡水場管勾,棄官歸。元末盜起,真聚眾保鄉里。十四年,縣人王成、陳仲玉作亂,真赴告元帥府。帥受賂,反捕真。逃居坭岡,舉兵攻成,不克。久之,惠州人王仲剛與叛將黃常據惠。真擊走常,殺仲剛。以功授惠陽路同知、廣東都元帥,守惠州。海寇邵宗愚陷廣州。真以兵破走之,復其城。擢廣東分省參政,尋擢右丞。贛州熊天瑞引舟師數萬,欲圖真,真迎之胥江。天大雷雨,折天瑞舟檣,擊走之。廣人賴以完。先是真再攻成,誅仲玉而成卒固守。二十六年復圍成,募擒成者,予鈔十千。成奴縛成以出。真予之鈔,命具湯鑊,趨烹奴,號於眾曰:「奴叛主者視此!」緣海叛者皆降。時中原大亂,嶺表隔絕,有勸真效尉佗故事者,不聽。屢遣使由海道貢方物於朝。累進資德大夫、行省左丞。

洪武元年,太祖命廖永忠為征南將軍,帥舟師取廣東。永忠至福州,以書諭真,遂航海趨潮州。師既至,真遣都事劉克佐詣軍門上印章,籍所部郡縣戶口兵糧,奉表以降。永忠聞於朝,賜詔褒真曰:「朕惟古之豪傑,保境安民,以待有德。若竇融、李勣之屬,擁兵據險,角立群雄間,非真主不屈。此漢、唐名臣,於今未見。爾真連數郡之眾,乃不煩一兵,保境來歸,雖竇、李奚讓焉。」永忠抵東莞,真帥官屬迎勞,遂奉詔入朝。擢江西行省參知政事,且諭之曰:「天下分爭,所謂豪傑有三:易亂為治者,上也;保民達變,知所歸者,次也;負固偷安,身死不悔,斯其下矣。卿輸誠納土,不逆顏行,可謂識時務者。」真頓首謝。在官頗著聲望,尤喜儒術,讀書綴文。已,轉山東參政。四年命還廣東,收集舊卒。事竣,仍蒞山東。九年致仕。

大軍征雲南,命真偕其子兵馬指揮貴往。規畫軍餉,置郵驛。遷山西右布政使。再與貴勾軍廣東,擢貴鎮南衛指揮僉事。尋命真為浙江布政使,改湖廣。二十年復致仕,封東莞伯,祿一千五百石,予世券。卒。

子榮嗣。與弟貴及尚寶司丞宏皆坐藍黨死。真弟迪疑禍及己,遂作亂。擊殺南海官軍三百餘人,遁入海島。廣東都司發兵討擒之,伏誅。

贊曰:陳友諒之克太平也,其鋒甚銳,微茂才則金陵之安危未可知矣。吳良守江陰,耿炳文守長興,而吳人不得肆其志。締造之基,其力為多。至若華雲龍、張赫、吳復、胡海之屬,或威著邊疆,或功存海運,搴旗陷陣,所向皆摧。揆之前代功臣,何多讓焉。而又皆能保守祿位,以恩禮令終,斯其尤足嘉美者歟!


\end{pinyinscope}