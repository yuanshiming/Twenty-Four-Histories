\article{太祖本紀}

\begin{pinyinscope}
太祖開天行道肇紀立極大聖至神仁文義武俊德成功高皇帝,諱元璋,字國瑞,姓朱氏。先世家沛,徙句容,再徙泗州。父世珍,始徙濠州之鐘離。生四子,太祖其季也。母陳氏,方娠,夢神授藥一丸,置掌中有光,吞之,寤,口餘香氣。及產,紅光滿室。自是夜數有光起,鄰里望見,驚以為火,輒奔救,至則無有。比長,姿貌雄傑,奇骨貫頂。志意廓然,人莫能測。

至正四年,旱蝗,大饑疫。太祖時年十七,父母兄相繼歿,貧不克葬。里人劉繼祖與之地,乃克葬,即鳳陽陵也。太祖孤無所依,乃入皇覺寺為僧。逾月,遊食合肥。道病,二紫衣人與俱,護視甚至。病已,失所在。凡歷光、固、汝、潁諸州三年,復還寺。當是時,元政不綱,盜賊四起。劉福通奉韓山童假宋後起潁,徐壽輝僭帝號起蘄,李二、彭大、趙均用起徐,眾各數萬,並置將帥,殺吏,侵略郡縣,而方國珍已先起海上。他盜擁兵據地,寇掠甚眾。天下大亂。

十二年春二月,定遠人郭子興與其黨孫德崖等起兵濠州。元將徹里不花憚不敢攻,而日俘良民以邀賞。太祖時年二十四,謀避兵,卜於神,去留皆不吉。乃曰:「得毋當舉大事乎?」卜之吉,大喜,遂以閏三月甲戌朔入濠見子興。子興奇其狀貌,留為親兵。戰輒勝,遂妻以所撫馬公女,即高皇后也。子興與德崖齟齬,太祖屢調護之。秋九月,元兵復徐州,李二走死,彭大、趙均用奔濠,德崖等納之。子興禮大而易均用,均用怨之。德崖遂與謀,伺子興出,執而械諸孫氏,將殺之。太祖方在淮北,聞難馳至,訴於彭大。大怒,呼兵以行,太祖亦甲而擁盾,發屋出子興,破械,使人負以歸,遂免。是冬,元將賈魯圍濠。太祖與子興力拒之。

十三年春,賈魯死,圍解。太祖收里中兵,得七百人。子興喜,署為鎮撫。時彭、趙所部暴橫,子興弱,太祖度無足與共事,乃以兵屬他將,獨與徐達、湯和、費聚等南略定遠。計降驢牌寨民兵三千,與俱東。夜襲元將張知院於橫澗山,收其卒二萬。道遇定遠人李善長,與語,大悅,遂與俱攻滁州,下之。是年,張士誠據高郵,自稱誠王。

十四年冬十月,元丞相脫脫大敗士誠於高郵,分兵圍六合。太祖曰:「六合破,滁且不免。」與耿再成軍瓦梁壘,救之。力戰,衛老弱還滁。元兵尋大至,攻滁,太祖設伏誘敗之。然度元兵勢盛且再至,乃還所獲馬,遣父老具牛酒謝元將曰:「守城備他盜耳,奈何舍巨寇戮良民?」元兵引去,城賴以完。脫脫既破士誠,軍聲大振,會中讒,遽解兵柄,江淮亂益熾。

十五年春正月,子興用太祖計,遣張天祐等拔和州,檄太祖總其軍。太祖慮諸將不相下,秘其檄,期旦日會廳事。時席尚右,諸將先入,皆踞右。太祖故後至,就左。比視事,剖決如流,眾瞠目不能發一語,始稍稍屈。議分工甓城,期三日。太祖工竣,諸將皆後。於是始出檄,南面坐曰:「奉命總諸公兵,今甓城皆後期,如軍法何?」諸將皆惶恐謝。乃搜軍中所掠婦女縱還家,民大悅。元兵十萬攻和,拒守三月,食且盡,而太子禿堅、樞密副使絆住馬、民兵元帥陳野先分屯新塘、高望、雞籠山以絕餉道。太祖率眾破之,元兵皆走渡江。三月,郭子興卒。時劉福通迎立韓山童子林兒於亳,國號宋,建元龍鳳。檄子興子天敘為都元帥,張天祐、太祖為左右副元帥。太祖慨然曰:「大丈夫寧能受制於人耶?」遂不受。然念林兒勢盛,可倚藉,乃用其年號以令軍中。

夏四月,常遇春來歸。五月,太祖謀渡江,無舟。會巢湖帥廖永安、俞通海以水軍千艘來附,太祖大喜,往撫其眾。而元中丞蠻子海牙扼銅城閘、馬場河諸隘,巢湖舟師不得出。忽大雨,太祖喜曰:「天助我也!」遂乘水漲,從小港縱舟還。因擊海牙於峪溪口,大敗之,遂定計渡江。諸將請直趨集慶。太祖曰:「取集慶必自采石始。采石重鎮,守必固,牛渚前臨大江,彼難為備,可必克也。」六月乙卯,乘風引帆,直達牛渚。常遇春先登,拔之。采石兵亦潰。緣江諸壘悉附。諸將以和州饑,爭取資糧謀歸。太祖謂徐達曰:「渡江幸捷,若舍而歸,江東非吾有也。」乃悉斷舟纜,放急流中,謂諸將曰:「太平甚近,當與公等取之。」遂乘勝拔太平,執萬戶納哈出。總管靳義赴水死,太祖曰:「義士也」,禮葬之。揭榜禁剽掠。有卒違令,斬以徇,軍中肅然。改路曰府。置太平興國翼元帥府,自領元帥事,召陶安參幕府事,李習為知府。時太平四面皆元兵。右丞阿魯灰、中丞蠻子海牙等嚴師截姑孰口,陳野先水軍帥康茂才以數萬眾攻城。太祖遣徐達、鄧愈、湯和逆戰,別將潛出其後,夾擊之,擒野先,並降其眾,阿魯灰等引去。秋九月,郭天敘、張天祐攻集慶,野先叛,二人皆戰死,於是子興部將盡歸太祖矣。野先尋為民兵所殺,從子兆先收其眾,屯方山,與海牙掎角以窺太平。冬十二月壬子,釋納哈出北歸。

十六年春二月丙子,大破海牙於采石。三月癸未,進攻集慶,擒兆先,降其眾三萬六千人,皆疑懼不自保。太祖擇驍健者五百人入衛,解甲酣寢達旦,眾心始安。庚寅,再敗元兵於蔣山。元御史大夫福壽,力戰死之,蠻子海牙遁歸張士誠,康茂才降。太祖入城,悉召官吏父老諭之曰:「元政水賣擾,干戈蜂起,我來為民除亂耳,其各安堵如故。賢士吾禮用之,舊政不便者除之,吏毋貪暴殃吾民。」民乃大喜過望。改集慶路為應天府,辟夏煜、孫炎、楊憲等十餘人,葬御史大夫福壽,以旌其忠。

當是時,元將定定扼鎮江,別不華、楊仲英屯寧國,青衣軍張明鑑據揚州,八思爾不花駐徽州,石抹宜孫守處州,其弟厚孫守婺州,宋伯顏不花守衢州,而池州已為徐壽輝將所據,張士誠自淮東陷平江,轉掠浙西。太祖既定集慶,慮士誠、壽輝強,江左、浙右諸郡為所并,於是遣徐達攻鎮江,拔之,定定戰死。夏六月,鄧愈克廣德。

秋七月己卯,諸將奉太祖為吳國公。置江南行中書省,自總省事,置僚佐。貽書張士誠,士誠不報,引兵攻鎮江。徐達敗之,進圍常州,不下。九月戊寅,如鎮江,謁孔子廟。遣儒士告諭父老,勸農桑,尋還應天。

十七年春二月,耿炳文克長興。三月,徐達克常州。夏四月丁卯,自將攻寧國,取之,別不華降。五月,上元、寧國、句容獻瑞麥。六月,趙繼祖克江陰。秋七月,徐達克常熟。胡大海克徽州,八思爾不花遁。冬十月,常遇春克池州,繆大亨克揚州,張明鑑降。十二月己丑,釋囚。是年,徐壽輝將明玉珍據重慶路。

十八年春二月乙亥,以康茂才為營田使。三月己酉,錄囚。鄧愈克建德路。夏四月,徐壽輝將陳友諒遣趙普勝陷池州。是月,友諒據龍興路。五月,劉福通破汴梁,迎韓林兒都之。初,福通遣將分道四出,破山東,寇秦晉,掠幽薊,中原大亂,太祖故得次第略定江表。所過不殺,收召才雋,由是人心日附。冬十二月,胡大海攻婺州,久不下,太祖自將往擊之。石抹宜孫遣將率車師由松溪來援,太祖曰:「道狹,車戰適取敗耳。」命胡德濟迎戰於梅花門,大破之,婺州降,執厚孫。先一日,城中人望見城西五色雲如車蓋,以為異,及是乃知為太祖駐兵地。入城,發粟振貧民,改州為寧越府。辟范祖乾、葉儀、許元等十三人分直講經史。戊子,遣使招諭方國珍。

十九年春正月乙巳,太祖謀取浙東未下諸路。戒諸將曰:「克城以武,戡亂以仁。吾比入集慶,秋毫無犯,故一舉而定。每聞諸將得一城不妄殺,輒喜不自勝。夫師行如火,不戢將燎原。為將能以不殺為武,豈惟國家之利,子孫實受其福。」庚申,胡大海克諸暨。是月,命寧越知府王宗顯立郡學。三月甲午,赦大逆以下。丁巳,方國珍以溫、台、慶元來獻,遣其子關為質,不受。夏四月,俞通海等復池州。時耿炳文守長興,吳良守江陰,湯和守常州,皆數敗士誠兵。太祖以故久留寧越,徇浙東。六月壬戌,還應天。秋八月,元察罕帖木兒復汴梁,福通以林兒退保安豐。九月,常遇春克衢州,擒宋伯顏不花。冬十月,遣夏煜授方國珍行省平章,國珍以疾辭。十一月壬寅,胡大海克處州,石抹宜孫遁。時元守兵單弱,且聞中原亂,人心離散,以故江左、浙右諸郡,兵至皆下,遂西與友諒鄰。

二十年春二月,元福建行省參政袁天祿以福寧降。三月戊子,征劉基、宋濂、章溢、葉琛至。夏五月,徐達、常遇春敗陳友諒於池州。閏月丙辰,友諒陷太平,守將朱文遜,院判花雲、王鼎,知府許瑗死之。未幾,友諒弒其主徐壽輝,自稱皇帝,國號漢,盡有江西、湖廣地,約士誠合攻應天,應天大震。諸將議先復太平以牽之,太祖曰:「不可。彼居上游,舟師十倍於我,猝難復也。」或請自將迎擊,太祖曰:「不可。彼以偏師綴我,而全軍趨金陵,順流半日可達,吾步騎急難引還,百里趨戰,兵法所忌,非策也。」乃馳諭胡大海搗信州牽其後,而令康茂才以書紿友諒,令速來。友諒果引兵東。於是常遇春伏石灰山,徐達陣南門外,楊璟屯大勝港,張德勝等以舟師出龍江關,太祖親督軍盧龍山。乙丑,友諒至龍灣,眾欲戰,太祖曰:「天且雨,趣食,乘雨擊之。」須臾,果大雨,士卒競奮,雨止合戰,水陸夾擊,大破之,友諒乘別舸走。遂復太平,下安慶,而大海亦克信州。初,太祖令茂才紿友諒,李善長以為疑。太祖曰:「二寇合,吾首尾受敵,惟速其來而先破之,則士誠膽落矣。」已而士誠兵竟不出。丁卯,置儒學提舉司,以宋濂為提舉,遣子標受經學。六月,耿再成敗石抹宜孫於慶元,宜孫戰死,遣使祭之。秋九月,徐壽輝舊將歐普祥以袁州降。冬十二月,復遣夏煜以書諭國珍。

二十一年春二月甲申,立鹽茶課。己亥,置寶源局。三月丁丑,改樞密院為大都督府。元將薛顯以泗州降。戊寅,國珍遣使來謝,飾金玉馬鞍以獻。卻之曰:「今有事四方,所需者人材,所用者粟帛,寶玩非所好也。」秋七月,友諒將張定邊陷安慶。八月,遣使於元平章察罕帖木兒。時察罕平山東,降田豐,軍聲大振,故太祖與通好。會察罕方攻益都未下,太祖乃自將舟師征陳友諒。戊戌,克安慶,友諒將丁普郎、傅友德迎降。壬寅,次湖口,追敗友諒於江州,克其城,友諒奔武昌。分徇南康、建昌、饒、蘄、黃、廣濟,皆下。冬十一月己未,克撫州。

二十二年春正月,友諒江西行省丞相胡廷瑞以龍興降。乙卯,如龍興,改為洪都府。謁孔子廟。告諭父老,除陳氏苛政,罷諸軍需,存恤貧無告者,民大悅。袁、瑞、臨江、吉安相繼下。二月,還應天。鄧愈留守洪都。癸未,降人蔣英殺金華守將胡大海,郎中王愷死之,英叛降張士誠。處州降人李祐之聞變,亦殺行樞密院判耿再成反,都事孫炎、知府王道同、元帥朱文剛死之。三月癸亥,降人祝宗、康泰反,陷洪都,鄧愈走應天,知府葉琛、都事萬思誠死之。是月,明玉珍稱帝於重慶,國號夏。夏四月己卯,邵榮復處州。甲午,徐達復洪都。五月丙午,朱文正、趙德勝、鄧愈鎮洪都。六月戊寅,察罕以書來報,留我使人不遣。察罕尋為田豐所殺。秋七月丙辰,平章邵榮、參政趙繼祖謀逆,伏誅。冬十二月,元遣尚書張昶航海至慶元,授太祖江西行省平章政事,不受。察罕子擴廓帖木兒致書歸使者。

二十三年春正月丙寅,遣汪河報之。二月壬申,命將士屯田積穀。是月,友諒將張定邊陷饒州。士誠將呂珍破安豐,殺劉福通。三月辛丑,太祖自將救安豐,珍敗走,以韓林兒歸滁州,乃還應天。夏四月壬戌,友諒大舉兵圍洪都。乙丑,諸全守將謝再興叛,附於士誠。五月,築禮賢館。友諒分兵陷吉安,參政劉齊、知府朱叔華死之。陷臨江,同知趙天麟死之。陷無為州,知州董會死之。秋七月癸酉,太祖自將救洪都。癸未,次湖口,先伏兵涇江口及南湖觜,遏友諒歸路,檄信州兵守武陽渡。友諒聞太祖至,解圍,逆戰於鄱陽湖。友諒兵號六十萬,聯巨舟為陣,樓櫓高十餘丈,綿亙數十里,旌旂戈盾,望之如山。丁亥,遇於康郎山,太祖分軍十一隊以禦之。戊子,合戰,徐達擊其前鋒,俞通海以火炮焚其舟數十,殺傷略相當。友諒驍將張定邊直犯太祖舟,舟膠於沙,不得退,危甚,常遇春從旁射中定邊,通海復來援,舟驟進,水湧太祖舟,乃得脫。己丑,友諒悉巨艦出戰,諸將舟小,仰攻不利,有怖色。太祖親麾之,不前,斬退縮者十餘人,人皆殊死戰。會日晡,大風起東北,乃命敢死士操七舟,實火藥蘆葦中,縱火焚友諒舟。風烈火熾,煙焰漲天,湖水盡赤。友諒兵大亂,諸將鼓噪乘之,斬首二千餘級,焚溺死者無算,友諒氣奪。辛卯,復戰,友諒復大敗。於是斂舟自守,不敢更戰。壬辰,太祖移軍扼左蠡,友諒亦退保渚磯。相持三日,其左、右二金吾將軍皆降。友諒勢益蹙,忿甚,盡殺所獲將士。而太祖則悉還所俘,傷者傅以善藥,且祭其親戚諸將陣亡者。八月壬戌,友諒食盡,趨南湖觜,為南湖軍所遏,遂突湖口。太祖邀之,順流搏戰,及於涇江。涇江軍復遮擊之,友諒中流矢死。張定邊以其子理奔武昌。九月,還應天,論功行賞。先是,太祖救安豐,劉基諫不聽。至是謂基曰:「我不當有安豐之行。使友諒乘虛直搗應天,大事去矣。乃頓兵南昌,不亡何待。友諒亡,天下不難定也。」壬午,自將征陳理。是月,張士誠自稱吳王。冬十月壬寅,圍武昌,分徇湖北諸路,皆下。十二月丙申,還應天,常遇春留督諸軍。

二十四年春正月丙寅朔,李善長等率群臣勸進,不允。固請,乃即吳王位。建百官。以善長為右相國,徐達為左相國,常遇春、俞通海為平章政事,諭之曰:「立國之初,當先正紀綱。元氏闇弱,威福下移,馴至於亂,今宜鑒之。」立子標為世子。二月乙未,復自將征武昌,陳理降,漢、沔、荊、岳皆下。三月乙丑,還應天。丁卯,置起居注。庚午,罷諸翼元帥府,置十七衛親軍指揮使司,命中書省辟文武人材。夏四月,建祠,祀死事丁普郎等於康郎山,趙德勝等於南昌。秋七月丁丑,徐達克盧州。戊寅,常遇春徇江西。八月戊戌,復吉安,遂圍贛州。達徇荊、湘諸路。九月甲申,下江陵,夷陵、潭、歸皆降。冬十二月庚寅,達克辰州,遣別將下衡州。

二十五年春正月己巳,徐達下寶慶,湖湘平。常遇春克贛州,熊天瑞降。遂趨南安,招諭嶺南諸路,下韶州、南雄。甲申,如南昌,執大都督朱文正以歸,數其罪,安置桐城。二月己丑,福建行省平章陳友定侵處州,參軍胡深擊敗之,遂下浦城。丙午,士誠將李伯昇攻諸全之新城,李文忠大敗之。夏四月庚寅,常遇春徇襄、漢諸路。五月乙亥,克安陸。己卯,下襄陽。六月壬子,朱亮祖、胡深攻建寧,戰於城下,深被執,死之。秋七月,令從渡江士卒被創廢疾者養之,死者贍其妻子。九月丙辰,建國子學。冬十月戊戌,下令討張士誠。是時,士誠所據,南至紹興,北有通、泰、高郵、淮安、濠、泗,又北至於濟寧。乃命徐達、常遇春等先規取淮東。閏月,圍泰州,克之。十一月,張士誠寇宜興,徐達擊敗之,遂自宜興還攻高郵。

二十六年春正月癸未,士誠窺江陰,太祖自將救之,士誠遁,康茂才追敗之於浮子門。太祖還應天。二月,明玉珍死,子升自立。三月丙申,令中書嚴選舉。徐達克高郵。夏四月乙卯,襲破士誠將徐義水軍於淮安,義遁,梅思祖以城降。濠、徐、宿三州相繼下,淮東平。甲子,如濠州省墓,置守塚二十家,賜故人汪文、劉英粟帛。置酒召父老飲,極歡,曰:「吾去鄉十有餘年,艱難百戰,乃得歸省墳墓,與父老子弟復相見。今苦不得久留歡聚為樂。父老幸教子弟孝弟力田,毋遠賈,濱淮郡縣尚苦寇掠,父老善自愛。」令有司除租賦,皆頓首謝。辛未,徐達克安豐,分兵敗擴廓於徐州。夏五月壬午,至自濠。庚寅,求遺書。秋八月庚戌,改築應天城,作新宮鐘山之陽。辛亥,命徐達為大將軍,常遇春為副將軍,帥師二十萬討張士誠。御戟門誓師曰:「城下之日,毋殺掠,毋毀廬舍,毋發丘壟。士誠母葬平江城外,毋侵毀。」既而召問達、遇春,用兵當何先。遇春欲直搗平江。太祖曰:「湖州張天騏、杭州潘原明為士誠臂指,平江窮蹙,兩人悉力赴援,難以取勝。不若先攻湖州,使疲於奔命。羽翼既披,平江勢孤,立破矣。」甲戌,敗張天騏於湖州,士誠親率兵來援,復敗之於皁林。九月乙未,李文忠攻杭州。冬十月壬子,遇春敗士誠兵於烏鎮。十一月甲申,張天騏降。辛卯,李文忠下餘杭,潘原明降,旁郡悉下。癸卯,圍平江。十二月,韓林兒卒。以明年為吳元年,建廟社宮室,祭告山川。所司進宮殿圖,命去雕琢奇麗者。是歲,元擴廓帖木兒與李思齊、張良弼構怨,屢相攻擊,朝命不行,中原民益困。

二十七年春正月戊戌,諭中書省曰:「東南久罹兵革,民生凋敝,吾甚憫之。且太平、應天諸郡,吾渡江開創地,供億煩勞久矣。今比戶空虛,有司急催科,重困吾民,將何以堪。其賜太平田租二年,應天、鎮江、寧國、廣德各一年。」二月丁未,傅友德敗擴廓將李二於徐州,執之。三月丁丑,始設文武科取士。夏四月,方國珍陰遣人通擴廓及陳友定,移書責之。五月己亥,初置翰林院。是月,以旱減膳素食,復徐、宿、濠、泗、壽、邳、東海、安東、襄陽、安陸及新附地田租三年。六月戊辰,大雨,群臣請復膳。太祖曰:「雖雨,傷禾已多,其賜民今年田租。」癸酉,命朝賀罷女樂。秋七月丙子,給府州縣官之任費,賜綺帛,及其父母妻長子有差,著為令。己丑,雷震宮門獸吻,赦罪囚。庚寅,遣使責方國珍貢糧。八月癸丑,圜丘、方丘、社稷壇成。九月甲戌,太廟成。朱亮祖帥師討國珍。戊寅,詔曰:「先王之政,罪不及孥。自今除大逆不道,毋連坐。」辛巳,徐達克平江,執士誠,吳地平。戊戌,遣使致書於元主,送其宗室神保大王等北還。辛丑,論平吳功,封李善長宣國公,徐達信國公,常遇春鄂國公,將士賜賚有差。朱亮祖克台州。癸卯,新宮成。

冬十月甲辰,遣起居注吳琳、魏觀以幣求遺賢於四方。丙午,令百官禮儀尚左。改李善長左相國,徐達右相國。辛亥,祀元臣餘闕於安慶,李黼於江州。壬子,置御史臺。癸丑,湯和為征南將軍,吳禎副之,討國珍。甲寅,定律令。戊午,正郊社、太廟雅樂。

庚申,召諸將議北征。太祖曰:「山東則王宣反側,河南則擴廓跋扈,關隴則李思齊、張思道梟張猜忌,元祚將亡,中原塗炭。今將北伐,拯生民於水火,何以決勝?」遇春對曰:「以我百戰之師,敵彼久逸之卒,直搗元都,破竹之勢也。」太祖曰:「元建國百年,守備必固,懸軍深入,饋餉不前,援兵四集,危道也。吾欲先取山東,撤彼屏蔽,移兵兩河,破其籓籬,拔潼關而守之,扼其戶檻。天下形勝入我掌握,然後進兵,元都勢孤援絕,不戰自克。鼓行而西,雲中、九原、關隴可席卷也。」諸將皆曰善。

甲子,徐達為征虜大將軍,常遇春為副將軍,帥師二十五萬,由淮入河,北取中原。胡廷瑞為征南將軍,何文輝為副將軍,取福建。湖廣行省平章楊璟、左丞周德興、參政張彬取廣西。己巳,朱亮祖克溫州。十一月辛巳,湯和克慶元,方國珍遁入海。壬午,徐達克沂州,斬王宣。己丑,廖永忠為征南副將軍,自海道會和討國珍。乙未,頒《大統曆》。辛丑,徐達克益都。十二月甲辰,頒律令。丁未,方國珍降,浙東平。張興祖下東平,兗東州縣相繼降。己酉,徐達下濟南。胡廷瑞下邵武。癸丑,李善長帥百官勸進,表三上,乃許。甲子,告於上帝。庚午,湯和、廖永忠由海道克福州。

洪武元年春正月乙亥,祀天地於南郊,即皇帝位。定有天下之號曰明,建元洪武。追尊高祖考曰玄皇帝,廟號德祖,曾祖考曰恒皇帝,廟號懿祖;祖考曰裕皇帝,廟號熙祖,皇考曰淳皇帝,廟號仁祖,妣皆皇后。立妃馬氏為皇后,世子標為皇太子。以李善長、徐達為左、右丞相,諸功臣進爵有差。丙子,頒即位詔於天下。追封皇伯考以下皆為王。辛巳,李善長、徐達等兼東宮官。甲申,遣使核浙西田賦。壬辰,胡廷瑞克建寧。庚子,鄧愈為征戍將軍,略南陽以北州郡。湯和克延平,執元平章陳友定,福建平。是月,天下府州縣官來朝。諭曰:「天下始定,民財力俱困,要在休養安息,惟廉者能約己而利人,勉之。」二月壬寅,定郊社宗廟禮,歲必親祀,以為常。癸卯,湯和提督海運。廖永忠為征南將軍,朱亮祖副之,由海道取廣東。丁未,以太牢祀先師孔子於國學。戊申,祀社稷。壬子,詔衣冠如唐制。癸丑,常遇春克東昌,山東平。甲寅,楊璟克寶慶。三月辛未,詔儒臣修女誡,戒后妃毋預政。壬申,周德興克全州。丁酉,鄧愈克南陽。己亥,徐達徇汴梁,左君弼降。夏四月辛丑,蘄州進竹簟,卻之,命四方毋妄獻。廖永忠師至廣州,元守臣何真降,廣東平。丁未,祫享太廟。戊申,徐達、常遇春大破元兵於洛水北,遂圍河南。梁王阿魯溫降,河南平。丁巳,楊璟克永州。甲子,幸汴梁。丙寅,馮勝克潼關,李思齊、張思道遁。五月己卯,廖永忠下梧州,潯、貴、容、鬱林諸州皆降。辛卯,改汴梁路為開封府。六月庚子,徐達朝行在。甲辰,海南、海北諸道降。壬戌,楊璟、朱亮祖克靖江。秋七月戊子,廖永忠下象州,廣西平。庚寅,振恤中原貧民。辛卯,將還應天,諭達等曰:「中原之民,久為群雄所苦,流離相望,故命將北征,拯民水火。元祖宗功德在人,其子孫罔恤民隱,天厭棄之。君則有罪,民復何辜。前代革命之際,肆行屠戮,違天虐民,朕實不忍。諸將克城,毋肆焚掠妄殺人,元之宗戚,咸俾保全。庶幾上答天心,下慰人望,以副朕伐罪安民之意。不恭命者,罰無赦。」丙申,命馮勝留守開封。閏月丁未,至自開封。己酉,徐達會諸將兵於臨清。壬子,常遇春克德州。丙寅,克通州,元帝趨上都。是月,徵天下賢才為守令。免吳江、慶德、太平、寧國、滁、和被災田租。八月己巳,以應天為南京,開封為北京。庚午,徐達入元都,封府庫圖籍,守宮門,禁士卒侵暴,遣將巡古北口諸隘。壬申,以京師火,四方水旱,詔中書省集議便民事。丁丑,定六部官制。御史中丞劉基致仕。己卯,赦殊死以下。將士從征者恤其家,逋逃許自首。新克州郡毋妄殺。輸賦道遠者,官為轉運,災荒以實聞。免鎮江租稅。避亂民復業者,聽墾荒地,復三年。衍聖公襲封及授曲阜知縣,並如前代制。有司以禮聘致賢士,學校毋事虛文。平刑,毋非時決囚。除書籍田器稅,民間逋負免征。蒙古、色目人有才能者,許擢用。鰥寡孤獨廢疾者,存恤之。民年七十以上,一子復。他利害當興革不在詔內者,有司具以聞。壬午,幸北京。改大都路曰北平府。徵元故臣。癸未,詔徐達、常遇春取山西。甲午,放元宮人。九月癸亥,詔曰:「天下之治,天下之賢共理之。今賢士多隱巖穴,豈有司失於敦勸歟,朝廷疏於禮待歟,抑朕寡昧不足致賢,將在位者壅蔽使不上達歟?不然,賢士大夫,幼學壯行,豈甘沒世而已哉。天下甫定,朕願與諸儒講明治道。有能輔朕濟民者,有司禮遣。」乙丑,常遇春下保定,遂下真定。冬十月庚午,馮勝、湯和下懷慶,澤、潞相繼下。丁丑,至自北京。戊寅,以元都平,詔天下。十一月己亥,遣使分行天下,訪求賢才。庚子,始祀上帝於圜丘。癸亥,詔劉基還。十二月丁卯,徐達克太原,擴廓帖木兒走甘肅,山西平。己巳,置登聞鼓。壬辰,以書諭明升。

二年春正月乙巳,立功臣廟於雞籠山。丁未,享太廟。庚戌,詔曰:「朕淮右布衣,因天下亂,率眾渡江,保民圖治,今十有五年。荷天眷祐,悉皆戡定。用是命將北征,齊魯之民饋糧給軍,不憚千里。朕軫厥勞,已免元年田租。遭旱民未蘇,其更賜一年。頃者大軍平燕都,下晉、冀,民被兵燹,困徵斂,北平、燕南、河東、山西今年田租亦與蠲免。河南諸郡歸附,久欲惠之,西北未平,師過其地,是以未逞。今晉、冀平矣,西抵潼關,北界大河,南至唐、鄧、光、息,今年稅糧悉除之。」又詔曰:「應天、太平、鎮江、宣城、廣德供億浩穰。去歲蠲租,遇旱惠不及下。其再免諸郡及無為州今年租稅。」庚申,常遇春取大同。是月,倭寇山東濱海郡縣。二月丙寅朔,詔修元史。壬午,耕耤田。三月庚子,徐達至奉元,張思道遁。振陜西饑,戶米三石。丙午,常遇春至鳳翔,李思齊奔臨洮。夏四月丙寅,遇春還師北平。己巳,諸王子受經於博士孔克仁。令功臣子弟入學。乙亥,編《祖訓錄》,定封建諸王之制。徐達下鞏昌。丙子,賜秦、隴新附州縣稅糧。丁丑,馮勝至臨洮,李思齊降。乙酉,徐達襲破元豫王於西寧。五月甲午朔,日有食之。丁酉,徐達下平涼、延安。張良臣以慶陽降,尋叛。癸卯,始祀地於方丘。六月己卯,常遇春克開平,元帝北走。壬午,封陳日煃為安南國王。秋七月己亥,鄂國公常遇春卒於軍,詔李文忠領其眾。辛亥,擴廓帖木兒遣將破原州、涇州。辛酉,馮勝擊走之。丙辰,明升遣使來。八月丙寅,元兵攻大同,李文忠擊敗之。己巳,定內侍官制。諭吏部曰:「內臣但備使令,毋多人,古來若輩擅權,可為鑒戒。馭之之道,當使之畏法,勿令有功,有功則驕恣矣。」癸酉,《元史》成。丙子,封王顓為高麗國王。癸未,徐達克慶陽,斬張良臣,陜西平。是月,命儒臣纂禮書。九月辛丑,召徐達、湯和還,馮勝留總軍事。癸卯,以臨濠為中都。戊午,征南師還。冬十月壬戌,遣楊璟諭明昇。甲戌,甘露降於鐘山,群臣請告廟,不許。辛卯,詔天下郡縣立學。是月,遣使貽元帝書。十一月乙巳,祀上帝於圜丘,以仁祖配。十二月甲戌,封阿答阿者為占城國王。甲申,振西安諸府饑,戶米二石。己丑,大賚平定中原及征南將士。庚寅,擴廓帖木兒攻蘭州,指揮于光死之。是年,占城、安南、高麗入貢。

三年春正月癸巳,徐達為征虜大將軍,李文忠、馮勝、鄧愈、湯和副之,分道北征。二月癸未,追封郭子興滁陽王。戊子,詔求賢才可任六部者。是月,李文忠下興和,進兵察罕腦兒,執元平章竹貞。三月庚寅,免南畿、河南、山東、北平、浙東、江西廣信、饒州今年田租。夏四月乙丑,封皇子樉為秦王,㭎晉王,棣燕王,橚吳王,楨楚王,榑齊王,梓潭王,巳趙王,檀魯王,從孫守謙靖江王。徐達大破擴廓帖木兒於沈兒峪,盡降其眾,擴廓走和林。丙戌,元帝崩於應昌,子愛猷識理達臘嗣。是月,慈利土官覃垕作亂。五月己丑,徐達取興元。分遣鄧愈招諭吐蕃。丁酉,詔守令舉學識篤行之士。己亥,設科取士。甲辰,李文忠克應昌。元嗣君北走,獲其子買的里八剌,降五萬餘人,窮追至北慶州,不及而還。丁未,詔行大射禮。戊申,祀地於方丘,以仁祖配。辛亥,徐達下興元。鄧愈克河州。丁巳,詔開國時將帥無嗣者祿其家。是月旱,齋戒,后妃親執爨,皇太子諸王饋於齋所。六月戊午朔,素服草屨,步禱山川壇,露宿凡三日,還齋於西廡。辛酉,賚將士,省獄囚,命有司訪求通經術明治道者。壬戌,大雨。壬申,李文忠捷奏至,命仕元者勿賀。謚元主曰順帝。癸酉,買的里八剌至京師,群臣請獻俘。帝曰:「武王伐殷用之乎?」省臣以唐太宗嘗行之對。帝曰:「太宗是待王世充耳。若遇隋之子孫,恐不爾也。」遂不許。又以捷奏多侈辭,謂宰相曰:「元主中國百年,朕與卿等父母皆賴其生養,奈何為此浮薄之言?亟改之。」乙亥,封買的里八剌為崇禮侯。丙子,告捷於南郊。丁丑,告太廟,詔示天下。辛巳,徙蘇州、松江、嘉興、湖州、杭州民無業者田臨濠,給資糧牛種,復三年。是月,倭寇山東、浙江、福建濱海州縣。秋七月丙辰,明昇將吳友仁寇漢中,參政傅友德擊卻之。中書左丞楊憲有罪誅。八月乙酉,遣使瘞中原遺骸。冬十月丙辰,詔儒士更直午門,為武臣講經史。癸亥,周德興為征南將軍,討覃垕,垕遁。辛巳,貽元嗣君書。十一月壬辰,北征師還。甲午,告武成於郊廟。丙申,大封功臣。進李善長韓國公,徐達魏國公,封李文忠曹國公,馮勝宋國公,鄧愈衛國公,常遇春子茂鄭國公,湯和等侯者二十八人。己亥,設壇親祭戰沒將士。庚戌,有事於圜丘。辛亥,詔戶部置戶籍、戶帖,歲計登耗以聞,著為令。乙卯,封中書右丞汪廣洋忠勤伯,御史中丞劉基誠意伯。十二月癸亥,復貽元嗣君書,並諭和林諸部。甲子,建奉先殿。庚午,遣使祭歷代帝王陵寢,並加修葺。己卯,賜勳臣田。壬午,以正月至是月,日中屢有黑子,詔廷臣言得失。是年,占城、爪哇、西洋入貢。

四年春正月丙戌,李善長罷,汪廣洋為右丞相。丁亥,中山侯湯和為征西將軍,江夏侯周德興、德慶侯廖永忠副之,率舟師由瞿塘,潁川侯傅友德為征虜前將軍,濟寧侯顧時副之,率步騎由秦隴伐蜀。魏國公徐達練兵北平。戊子,衛國公鄧愈督餉給征蜀軍。庚寅,建郊廟於中都。丁未,詔設科取士,連舉三年,嗣後三年一舉。戊申,免山西旱災田租。二月甲戌,幸中都。壬午,至自中都。元平章劉益以遼東降。是月,蠲太平、鎮江、寧國田租。三月乙酉朔,始策試天下貢士,賜吳伯宗等進士及第、出身有差。乙巳,徙山後民萬七千戶屯北平。丁未,誠意伯劉基致仕。夏四月丙戌,傅友德克階州,文、隆、綿三州相繼下。五月,免江西、浙江秋糧。六月壬午,傅友德克漢州。辛卯,廖永忠克夔州。戊戌,明昇將丁世貞破文州,守將朱顯忠死之。癸卯,湯和至重慶,明昇降。戊申,倭寇膠州。是月,徙山後民三萬五千戶於內地,又徙沙漠遺民三萬二千戶屯田北平。秋七月辛亥,徐達練兵山西。辛酉,傅友德下成都,四川平。乙丑,明昇至京師,封歸義侯。八月甲午,免中都、淮、揚及泰、滁、無為田租。己酉,振陜西饑。是月,高州海寇亂,通判王名善死之。九月庚戌朔,日有食之。冬十月丙申,征蜀師還。十一月丙辰,有事於圜丘。庚申,命官吏犯贓者罪勿貸。是月,免陜西、河南被災田租。十二月,徐達還。是年,安南、浡泥、高麗、三佛齊、暹羅、日本、真臘入貢。

五年春正月癸丑,待制王禕使雲南,詔諭元梁王把匝剌瓦爾密。禕至,不屈死。乙丑,徙陳理、明升於高麗。甲戌,魏國公徐達為征虜大將軍,出雁門,趨和林,曹國公李文忠為左副將軍,出應昌,宋國公馮勝為征西將軍,取甘肅,徵擴廓帖木兒。靖海侯吳禎督海運,餉遼東。衛國公鄧愈為征南將軍,江夏侯周德興、江陰侯吳良副之,分道討湖南、廣西洞蠻。二月丙戌,安南陳叔明弒其主日熞自立,遣使入貢,卻之。三月丁卯,都督僉事藍玉敗擴廓於土剌河。夏四月己卯,振濟南、萊州饑。戊戌,始行鄉飲酒禮。庚子,鄧愈平散毛諸洞蠻。五月壬子,徐達及元兵戰於嶺北,敗績。是月,詔曰:「天下大定,禮儀風俗不可不正。諸遭亂為人奴隸者復為民。凍餒者里中富室假貸之,孤寡殘疾者官養之,毋失所。鄉黨論齒,相見揖拜,毋違禮。婚姻毋論財。喪事稱家有無,毋惑陰陽拘忌,停柩暴露。流民復業者各就丁力耕種,毋以舊田為限。僧道齋醮雜男女,恣飲食,有司嚴治之。閩、粵豪家毋閹人子為火者,犯者抵罪。」六月丙子,定宦官禁令。丁丑,定宮官女職之制。戊寅,馮勝克甘肅,追敗元兵於瓜、沙州。癸巳,定六部職掌及歲終考績法。壬寅,吳良平靖州蠻。甲辰,李文忠敗元兵於阿魯渾河,宣寧侯曹良臣戰沒。乙巳,作鐵榜誡功臣。是月,振山東饑,免被災郡縣田租。秋七月丙辰,湯和及元兵戰於斷頭山,敗績。八月丙申,吳良平五開、古州諸蠻。甲辰,元兵犯雲內,同知黃理死之。九月戊午,周德興平婪鳳、安田諸蠻。冬十月丁酉,馮勝師還。是月,免應天、太平、鎮江、寧國、廣德田租。十一月辛酉,有事於圜丘。甲子,征南師還。壬申,納哈出犯遼東。是月,召徐達、李文忠還。十二月甲戌,詔以農桑學校課有司。辛巳,命百官奏事啟皇太子。庚子,鄧愈為征西將軍,征吐番。壬寅,貽元嗣君書。是年,瑣里、占城、高麗、琉球、烏斯藏入貢。高麗貢使再至,諭自後三年一貢。

六年春正月甲寅,謫汪廣洋為廣東參政。二月乙未,諭暫罷科舉,察舉賢才。壬寅,命御史及按察使考察有司。三月癸卯朔,日有食之。頒《昭鑒錄》,訓誡諸王。戊申,太閱。壬子,徐達為征虜大將軍,李文忠、馮勝、鄧愈、湯和副之,備邊山西、北平。甲子,指揮使於顯為總兵官,備倭。夏四月己丑,令有司上山川險易圖。六月壬午,盱眙獻瑞麥,薦宗廟。壬辰,擴廓帖木兒遣兵攻雁門,指揮吳均擊卻之。是月,免北平、河間、河南、開封、延安、汾州被災田租。秋七月壬寅,命戶部稽渡江以來各省水旱災傷分數,優恤之。壬子,胡惟庸為右丞相,八月乙亥,詔祀三皇及歷代帝王。冬十月辛巳,召徐達、馮勝還。十一月壬子,擴廓帖木兒犯大同,徐達遣將擊敗之,達仍留鎮。甲子,遣兵部尚書劉仁振真定饑。丙寅,冬至,帝不豫,改卜郊。閏月乙亥,錄故功臣子孫未嗣者二百九人。壬午,有事於圜丘。庚寅,頒定《大明律》。是年,暹羅、高麗、占城、真臘、三佛齊入貢。命安南陳叔明權知國事。

七年春正月甲戌,都督僉事王簡、王誠、平章李伯昇屯田河南、山東、北平。靖海侯吳禎為總兵官,都督於顯副之,巡海捕倭。二月丁酉朔,日有食之。戊午,修曲阜孔子廟,設孔、顏、孟三氏學。是月,平陽、太原、汾州、歷城、汲縣旱蝗,並免租稅。夏四月己亥,都督藍玉敗元兵於白酒泉,遂拔興和。壬寅,金吾指揮陸齡討永、道諸州蠻,平之。五月丙子,免真定等四十二禕府州縣被災田租。辛巳,振蘇州饑民三十萬戶。癸巳,減蘇、松、嘉、湖極重田租之半。六月,陜西平涼、延安、靖寧、鄜州雨雹,山西、山東、北平、河南蝗,並蠲田租。秋七月甲子,李文忠破元兵於大寧、高州。壬申,倭寇登、萊。八月甲午朔,祀歷代帝王廟。辛丑,詔軍士陣歿父母妻子不能自存者,官為存養。百姓避兵離散或客死,遺老幼,並資遣還。遠宦卒官,妻子不能歸者,有司給舟車資送。庚申,振河間、廣平、順德、真定饑,蠲租稅。九月丁丑,遣崇禮侯買的里八剌歸,遺元嗣君書。冬十一月壬戌,納哈出犯遼陽,千戶吳壽擊走之。辛未,有事於圜丘。十二月戊戌,召鄧愈、湯和還。是年,阿難功德國、暹羅、琉球、三佛齊、烏斯藏、撒里、畏兀兒入貢。

八年春正月辛未,增祀雞籠山功臣廟一百八人。癸酉,命有司察窮民無告者,給屋舍衣食。辛巳,鄧愈、湯和等十三人屯戍北平、陜西、河南。丁亥,詔天下立社學。是月,河決開封,發民夫塞之。二月甲午,宥雜犯死罪以下及官犯私罪者,謫鳳陽輸作屯種贖罪。癸丑,耕耤田。召徐達、李文忠、馮勝還,傅友德等留鎮北平。三月辛酉,立鈔法。辛巳,罷寶源局鑄錢。

夏四月辛卯,幸中都。丁巳,至自中都。免彰德、大名、臨洮、平涼、河州被災田租。罷營中都。致仕誠意怕劉基卒。五月己巳,永嘉候朱亮祖偕傅友德鎮北平。六月壬寅,指揮同知胡汝平貴州蠻。

秋七月己未朔,日有食之。辛酉,改作太廟。壬戌,召傅友德、朱亮祖還,李文忠、顧時鎮山西、北平。戊辰,詔百官奔父母喪不俟報。京師地震。丁丑,免應天、太平、寧國鎮江及蘄、黃諸府被災田租。八月己酉,元擴廓帖木兒卒。

冬十月丁亥,詔舉富民素行端潔達時務者。壬子,命皇太子諸王講武中都。十一月丁丑,有事於圓丘。十二月戊子,京師地震。甲寅,遣使振蘇州、湖州、嘉興、松江、常州、太平、寧國、杭州水災。是月,納哈出犯遼東,指揮馬雲、葉旺大敗之。

是年,撒里、高麗、占城、暹羅、日本、爪哇、三佛齊入貢。

九年春正月,中山侯湯和,潁川侯傅友德,都督僉事藍玉、王弼,中書右丞丁玉,備邊延安。三月己卯,詔曰:「比年西征燉煌,北伐沙漠,軍需甲仗,皆資山、陜,又以秦、晉二府宮殿之役,重困吾民。平定以來,閭閻未息。國都始建,土木屢興。畿輔既極煩勞,外郡疲於轉運。今蓄儲有餘,其淮、揚、安、徽、池五府及山西、陜西、河南、福建、江西、浙江、北平、湖廣今年租賦,悉免之。」

夏四月庚戌,京師自去年八月不雨,是日始雨。五月癸酉,自庚戌雨,至是日始霽。六月甲午,改行中書省為承宣布政使司。辛丑,李文忠還。

秋七月癸丑朔,日有食之。是月,蠲蘇、松、嘉、湖水災田租,振永平旱災。元將伯顏帖木兒犯延安,傅友德敗降之。八月己酉,遣官省歷代帝王陵寢,禁芻牧,置守陵戶。忠臣烈士祠,有司以時葺治。分遣國子生修嶽鎮海瀆祠。西番朵兒只巴寇罕東,河州指揮甯正擊走之。閏九月庚寅,以災異詔求直言。

冬十月己未,太廟成,自是行合享禮。丙子,命秦、晉、燕、吳、楚、齊諸王治兵鳳陽。十一月壬午,有事於圓丘。戊子,徙山西及真定民無產者田鳳陽。十二月甲寅,振畿內、浙江、湖北水災。己卯,遣都督同知沐英乘傳詣陜西問民疾苦。

是年,覽邦、琉球、安南、日本、烏斯藏、高麗入貢。

十年春正月辛卯,以羽林等衛軍益秦、晉、燕三府護衛。是春,振蘇、松、嘉、湖水災。

夏四月己酉,鄧愈為征西將軍,沐英為副將軍,率師討吐番,大破之。是月,振太平、寧國及宜興、錢塘諸縣水災。五月庚子,韓國公李善長、曹國公李文忠總中書省、大都督府、御史臺,議軍國重事。癸卯,振湖廣水災。丙午,戶部主事趙乾振荊、蘄遲緩,伏誅。六月丁巳,詔臣民言事者,實封達御前。丙寅,命政事啟皇太子裁決奏聞。

秋七月甲申,置通政司。是月,始遣御史巡按州縣。八月庚戌,改建大祀殿於南郊。癸丑,選武臣子弟讀書國子監。九月丙申,振紹興、金華、衢州水災。辛丑,胡惟庸為左丞相,汪廣洋為右丞相。

冬十月戊午,封沐英四平侯。辛酉,賜百官公田。十一月癸未,衛國公鄧愈卒。丁亥,合祀天地於奉天殿。是月,免河南、陜西、廣東、湖廣田租。威茂蠻叛,御史大夫丁玉為平羌將軍,討平之。十二月乙巳朔,日有食之。丁未,錄故功臣子孫百百餘人,授官有差。

是年,占城、三佛齊、暹羅、爪哇、真臘入貢。高麗使五至,以嗣王未立,卻之。

十一年春正月甲戌,封皇子椿為蜀王,柏湘王,桂豫王,模漢王,植衛王。改封吳王橚為周王。己卯,進封湯和信國公。是月,徵天下布政使及知府來朝。二月,指揮胡淵平茂州蠻。三月壬午,命奏事毋關白中書省。是月,第來朝官為三等。

夏四月,元嗣君愛猷識理達臘殂,子脫古思帖木兒嗣。五月丁酉,存問蘇、松、嘉、湖被水災民,戶賜米一石,蠲逋賦六十五萬有奇。六月壬子,遣使祭故元嗣君。己巳,五開蠻叛,殺靖州指揮過興,以辰州指揮楊仲名為總兵官,討之。

秋七月丁丑,振平陽饑。是月,蘇、松、揚、台海溢,遣官存恤。八月,免應天、太平、鎮江、寧國、廣德諸府州秋糧。九月丙申,追封劉繼祖為義惠侯。

冬十月甲子,大祀殿成。十一月庚午,征西將軍西平侯沐英率都督藍玉、玉弼討西番。是月,五開蠻平。

是年,暹羅、闍婆、高麗、琉球、占城、三佛齊、朵甘、烏斯藏、彭亨、百花入貢。

十二年春正月己卯,始合禮天地於南郊。甲申,洮州十八族番叛,命沐英移兵討之。丙申,丁玉平松州蠻。二月戊戌,李文忠督理河、岷、臨、鞏軍事。乙巳,詔曰:「今春雨雪經旬。天下貧民困於饑寒者多有,其令有司給以鈔。」丙寅,信國公湯和率列候練兵臨清。

夏五月癸未,蠲北平田租。六月丁卯,都督馬雲征大寧。秋七月丙辰,丁玉回師討眉縣賊,平之。己未,李文忠還掌大都督府事。八月辛巳,詔凡致仕官復其家,終身無所與。九月己亥,沐英大破西番,擒其部長三副使。

冬十一月甲午,沐英班師,封仇成、藍玉等十二人為侯。庚申,大寧平。十二月,汪廣洋貶廣南,賜死。徵天下博學老成之士至京師。

是年,占城、爪哇、暹羅、日本、安南、高麗入貢。高麗貢黃金百斤、白金萬兩,以不如約,卻之。

十三年春正月戊戌,左丞相胡惟庸謀反,及其黨御史大夫陳寧、中丞塗節等伏誅。癸卯,大祀天地於南郊。罷中書省,廢丞相等官,更定六部官秩,改大都督府為中、左、右、前、後五軍都督府。二月壬戌朔,詔舉聰明正直、孝弟力田、賢良方正、文學術數之士。發丹符,驗天下金穀之數。戊辰,文武官年六十以上者聽致仕,給以誥敕。三月壬辰,減蘇、松、嘉、湖重賦十之二。壬寅,燕王隸之國北平。壬子,沐英襲元將脫火赤於亦集乃,擒之,盡降其眾。

夏四月己丑,命群臣各舉所知。五月甲午,雷震謹身殿。乙未,大赦。丙申,釋在京及臨濠屯田輸作者。己亥,免天下田租。吏以過誤罷者還其職。壬寅,都督濮英進兵赤斤站,獲故元豳王亦憐真及其部曲而還。是月,罷御史臺。命從征士卒老疾者許以子代,老而無子及寡婦,有司資遣還。六月丙寅,雷震奉天門,避正殿省愆。丁卯,罷王府工役。丁丑,置諫院官。

秋八月,命天下學校師生,日給廩膳。九月辛卯,景川侯曹震、營陽侯楊璟、永城侯恭顯屯田北平。乙巳,天壽節,始受群臣朝賀,賜宴於謹身殿,後以為常。丙午,置四輔官,告於太廟。以儒士王本、估佑、襲斅、杜斅、趙民望、吳源為春、夏官。是月,詔陜西衛軍以三分之二屯田。安置翰林學士承旨宋濂於茂州,道卒。

冬十一月乙未,徐達還。丙午,元平章完者不花、乃兒不花犯永平,指揮劉廣戰沒,千戶王輅擊敗之,擒完者不花。十二月,天下府州縣所舉士至者八百六十餘人,授官有差。南雄侯趙庸鎮廣東,討陽春蠻。

是年,琉球、日本、安南、占城、真臘、爪哇入貢,日本以無表卻之。

十四年春正月戊子,徐達為征虜大將軍,湯和、傅友德為左、右副將軍,帥師討乃兒不花。命新授官者各舉所知。乙未,大祀天地於南郊。壬子,罷天下歲造兵器。癸丑,命公候子弟入國學。丙辰,詔求隱逸。二月庚辰,核天下官田。三月丙戌,大赦。辛丑,頒《五經》、《四書》於北方學校。

夏四月庚午,徐達率諸將出塞,至北黃河,擊破元兵,獲全寧四部以歸。五月,五溪蠻叛,江夏侯周德興討平之。

秋八月丙子,詔求明經老成之士,有司禮送京師。庚辰,河決原武、祥符、中牟。辛巳,徐達還。九月壬午朔,傅友德為征南將軍,藍玉、沐英為左、右副將軍,帥征支南。徐達鎮北平。丙午,周德興移師討施州蠻,平之。

冬十月壬子朔,日有食之。癸丑,命法司錄囚,會翰林院給事中及春坊官會議平允以聞。甲寅,免應天、太平、應德、鎮江、寧國田租。癸亥,分遣御史錄囚。己卯,延安侯唐勝宗帥師討浙東山寇,平之。十一月壬午,吉安侯陸仲亨鎮成都。庚戌,趙庸討廣州海寇,大破之。十二月丁巳,命翰林春坊官考駁諸司章奏。戊辰,傅友德大敗元兵於白石江,遂下曲靖。壬申,元梁王把匝剌瓦爾密走普寧自殺。

是年,暹羅、安南、爪哇、朵甘、烏斯藏入貢。以安南寇思明,不納。

十五年春正月辛巳,宴群臣於謹身殿,始用九奏樂。景川侯曹震、定遠侯王弼下威楚路。壬午,元曲靖宣慰司及中慶、澄江、武定諸路俱降,雲南平。己丑,減大辟囚。乙未,大祀天地於南郊。庚戌,命天下朝覲官各舉所知一人。二月壬子。河決河南,命駙馬都尉李祺振之。甲寅,以雲南平,詔天下。閏月癸卯,藍玉、沐英克大理,分兵徇鶴慶、麗江、金齒,俱下。三月庚午,河決朝邑。

夏四月甲申,遷元梁王把匝剌瓦兒密及威順王子伯伯等家屬於耽羅。丙戌,詔天下通祀孔子。壬辰,免畿內、浙江、江西、河南、山東稅糧。五月乙丑,太學成,釋奠於先師孔子。丙子,廣平府吏王允道請開磁州鐵冶。帝曰:「朕聞王者使天下無遺賢,不聞無遺利。今軍器不乏,而民業已定,無益於國,且重擾民。」杖之,流嶺南。丁丑,遣行人訪經明行修之士。

秋七月乙卯,河決滎澤、陽武。辛酉,罷四輔官。乙亥,傅友德、沐英擊烏撒蠻,大敗之。八月丁丑,復設科取士,三年一行,為定制。丙戌,皇后崩。己丑,延安侯唐勝宗、長興侯耿炳文屯田陜西。丁酉,擢秀才曾泰為戶部尚書。辛丑,命徵至秀才分六科試用。九月己酉,吏部以經明行修之士鄭韜等三千七百餘人入見,令舉所知,復遣使徵之。賜韜等鈔,尋各授布政使、參政等官有差。庚午,葬孝慈皇后於孝陵。

冬十月丙子,置都察院。丙申,錄囚。甲辰,徐達還。是月,廣東群盜平,詔趙庸班師。十一月戊午,置殿閣大學士,以邵質、吳伯宗、宋訥、吳沉為之。十二月辛卯,振北平被災屯田士卒。乙亥,永城侯薛顯理山西軍務。

是年,爪哇、琉球、烏斯藏、占城入貢。

十六年春正月乙卯,大祀天地於南郊。戊午,徐達鎮北平。二月丙申,初命天下學校歲貢士於京師。三月甲辰,召征南師還,沐英留鎮雲南。丙寅,復鳳陽、臨淮二縣民徭賦,世世無所與。

夏五月庚申,免畿內各府田租。六月辛卯,免畿內十二州縣養馬戶田租一年,滁州免二年。

秋七月,分遣御史錄囚。八月壬申朔,日有食之。九月癸亥,申國公鄧鎮為征南將軍,討龍泉山寇,平之。

冬十月丁丑,召徐達等還。十二月甲午,刑部尚書開濟有罪誅。

是年,琉球、占城、西番、打箭爐、暹羅、須文達那入貢。

十七年春正月丁未,太祀天地於南郊。戊申,徐達鎮北平。壬戌,湯和巡視沿海諸城防倭。三月戊戌朔,頒科舉取士式。曹國公李文忠卒。甲子,大赦天下。

夏四月壬午,論平雲南功,進封傅友德潁國公,陳桓恆等侯者四人,大賚將士。庚寅,收陣亡遺骸。增築國子學舍。五月丙寅,涼州指揮宋晟討西番於亦集乃,敗之。

秋七月戊戌,禁內官預外事,敕諸司毋通內官監文移。癸丑,詔百官迎養父母者,官給舟車。丁巳,免畿內今年田租之半。庚申,錄囚。壬戌,盱貽人獻天書,斬之。八月丙寅,河決開封。壬申,決杞縣,遣官塞之。己丑,蠲河南諸省逋賦。

冬十月丙子,河南、北平大水,分遣駙馬都尉李祺等振之。閏月癸丑,詔天下罪囚,刑部、都察院詳議,大理寺覆讞後奏決。是月,召徐達還。十二月壬子,蠲雲南逋賦。

是年,琉球、暹羅、安南、占城入貢。

十八年春正月辛未,大祀天地於南郊。癸酉,朝覲官分五等考績,黜陟有差。二月甲辰,以久陰雨雷雹,詔臣民極言得失。己未,魏國公徐達卒。三月壬戌,賜丁顯等進士及第、出身有差。詔中外官父母歿任所者,有司給舟車歸其喪,著為令。乙亥,免畿內今年田租。命天下郡縣瘞暴骨。丙子,初選進士為翰林院、承敕監、六科庶吉士。己丑,戶部侍郎郭桓坐盜官糧誅。

夏四月丁酉,吏部尚書餘熂以罪誅。丙辰,思州蠻叛,湯和為征虜將軍,周德興為副將軍,帥師從楚王楨討之。六月戊申,定外官三年一朝,著為令。

秋七月甲戌,封王禑為高麗國王。庚辰,五開蠻叛。八月庚戌,馮勝、傅友德、藍玉備邊北平。是月,振河南水災。

冬十月己丑,頒《大誥》於天下。癸卯,召馮勝還。甲辰,詔曰:「孟子傳道,有功名教。歷年既久,子孫甚微。近有以罪輸作者,豈禮先賢之意哉。其加意詢訪,凡聖賢後裔輸作者,皆免之。」是月,楚王楨、信國公湯和討平五開蠻。十一月乙亥,蠲河南、山東、北平田租。十二月丙午,詔有司舉孝廉。癸丑,麓川平緬宣慰使思倫發反,都督馮誠敗績,千戶王升死之。

是年,高麗、琉球、安南、暹羅入貢。

十九年春正月辛酉,振大名及江浦水災。甲子,大祀天地於南郊。是月,征蠻師還。二月丙申,耕耤田,癸丑,振河南饑。

夏四月甲辰,詔贖河南饑民所鬻子女。六月甲辰,詔有司存問高年。貧民年八十以上,月給米五斗,酒三斗,肉五斤;九十以上,歲加帛一匹,絮一斤;有田產者罷給米。應天、鳳陽富民年八十以上賜爵社士,九十以上鄉士;天下富民八十以上里士,九十以上社士。皆與縣官均祀,復其家。鰥寡孤獨不能自存者,歲給米六石。士卒戰傷除其籍,賜復三年。將校陣亡,其子世襲加一秩。巖穴之士,以禮聘遣。丁未,振青州及鄭州饑。

秋七月癸未,詔舉經明行修練達時務之士。年六十以上者,置翰林備顧問,六十以下,於六部、布按二司用之。八月甲辰,命皇太子修泗州盱眙祖陵,葬德祖以下帝后冕服。九月庚申,屯田雲南。

冬十月,命官軍已亡子女幼或父母老者皆給全俸,著為令。十二月癸未朔,日有食之。是月,命宋國公馮勝分兵防邊。發北平、山東、山西、河南民運糧於大寧。

是年,高麗、琉球、暹羅、占城、安南入貢。

二十年春正月癸丑,馮勝為征虜大將軍,傅友德、藍玉副之,率師徵納哈出。焚錦衣衛刑具,以繫囚付刑部。甲子,大祀天地於南郊。禮成,天氣清明。侍臣進曰:「此陛下敬天之誠所致。」帝曰:「所謂敬天者,不獨嚴而有禮,當有其實。天以子民之任付於君,為君者欲求事天,必先恤民。恤民者,事天之實也。即如國家命人任守令之事,若不能福民,則是棄君之命,不敬孰大焉。」又曰:「為人君者,父天母地子民,皆職分之所當盡,祀天地,非祈福於己,實為天下蒼生也。」二月壬午,閱武。乙未,耕耤田。三月辛亥,馮勝率師出松亭關,城大寧、寬河、會州、富峪。

夏四月戊子,江夏侯周德興築福建瀕海城,練兵防倭。六月庚子,臨江侯陳鏞從征失道,戰沒。癸卯,馮勝兵踰金山。丁未,納哈出降。閏月庚申,師還次金山,都督濮英殿軍遇伏,死之。

秋八月癸酉,收馮勝將軍印,召還,藍玉攝軍事。景川侯曹震屯田雲南品甸。九月戊寅,封納哈出海西侯。癸未,置大寧都指揮使司。丁酉,安置鄭國公常茂於龍州。丁未,藍玉為征虜大將軍,延安侯唐勝宗、武定侯郭英副之,北征沙漠。是月,城西寧。

冬十月戊申,封朱壽為舳艫侯,張赫為航海侯。是月,馮勝罷歸鳳陽,奉朝請。十一月壬午,普定侯陳桓、靖寧侯葉昇屯田定邊、姚安、畢節諸衛。己丑,湯和還,凡築寧海、臨山等五十九城。十二月,振登、萊饑。

是年,琉球、安南、高麗、占城、真臘、朵甘、烏斯藏入貢。

二十一年春正月辛巳,麓川蠻思倫發入寇馬龍他郎甸,都督甯正擊敗之。辛卯,大祀天地於南郊。甲午,振青州饑,逮治有司匿不以聞者。三月乙亥,賜任亨泰等進士及第、出身有差。丙戌,振東昌饑。甲辰,沐英討思倫發敗之。

夏四月丙辰,藍玉襲破元嗣君於捕魚兒海,獲其次子地保奴及妃主王公以下數萬人而還。五月甲戌朔,日有食之。六月甲辰,信國公湯和歸鳳陽。甲子,傅友德為征南將軍,沐英、陳桓為左、右副將軍,帥師討東川叛蠻。

秋七月戊寅,安置地保奴於琉球。八月癸丑,徙澤、潞民無業者墾河南、北田,賜鈔備農具,復三年。丁卯,藍玉師還,大賚北征將士。戊辰,封孫恪為全寧侯。是月,御製八諭飭武臣。九月丙戌,秦、晉、燕、周、楚、齊、湘、魯、潭九王來朝。癸巳,越州蠻阿資叛,沐英會傅友德討之。

冬十月丁未,東川蠻平。十二月壬戌,進封藍玉涼國公。

是年,高麗、古城、琉球、暹羅、真臘、撒馬兒罕、安南入貢。詔安南三歲一朝,象犀之屬毋獻。安南黎季犛弒其主煒。

二十二年春正月丙戌,改大宗正院曰宗人府,以秦王樉為宗人令,晉王㭎、燕王棣為左、右宗正,周王橚、楚王楨為左、右宗人。丁亥,大祀天地於南郊。乙未,傅友德破阿資於普安。二月己未,藍玉練兵四川。壬戌,禁武臣預民事。癸亥,湖廣千戶夏得忠結九溪蠻作亂,靖寧侯葉昇討平之,得忠伏誅。是月,阿資降。三月庚午,傅友德帥諸將分屯四川,湖廣,防西南蠻。

夏四月己亥,徙江南民田淮南,賜鈔備農具,復三年。癸丑,魏國公徐允恭、開國公常昇等練兵湖廣。甲寅,徙元降王於眈羅。是月,遣御史按山東官匿災不奏者。五月辛卯,置泰寧、朵顏、福餘三衛於兀良哈。

秋七月,傅友德等還。八月乙卯,詔天下舉高年有德識時務者。是月,更定《大明律》。九月丙寅朔,日有食之。

冬十一月丙寅,宣德侯金鎮等練兵湖廣。己卯,思倫發入貢謝罪,麓川平。十二月甲辰,周王橚有罪,遷雲南,尋罷徙,留居京師。定遠侯王弼等練兵山西、河南、陜西。

是年,高麗、安南、占城、暹羅、真臘入貢。元也速迭兒弒其主脫古思帖木兒而立坤帖木兒。高麗廢其主禑,又廢其主昌。安南黎季犛復弒其主日焜。

二十三年春正月丁卯,晉王㭎、燕王棣帥師征元丞相咬住、太尉乃兒不花,征虜前將軍潁國公傅友德等皆聽節制。己卯,大祀天地於南郊。庚辰,貴州蠻叛,延安侯唐勝宗討平之。乙酉,齊王榑帥師從燕王棣北征。贛州賊為亂,東川侯胡海充總兵官,普定侯陳桓、靖寧侯葉昇為副將,討平之。唐勝宗督貴州各衛屯田。二月戊申,藍玉討平西番叛蠻。丙辰,耕耤田。癸亥,河決歸德,發諸軍民塞之。三月癸巳,燕王棣師次迤都,咬住等降。

夏四月,吉安侯陸仲亨等坐胡惟庸黨下獄。丙申,潭王梓自焚死。閏月丙子,藍玉平施南、忠建叛蠻。五月甲午,遣諸公侯還里,賜金幣有差。乙卯,賜太師韓國公李善長死,陸仲亨等皆坐誅。作《昭示姦黨錄》,布告天下。六月乙丑,藍玉遣鳳翔侯張龍平都勻、散毛諸蠻。庚寅,授耆民有才德知典故者官。

秋七月壬辰,河決開封,振之。癸巳,崇明、海門風雨海溢,遣官振之,發民二十五萬築隄。八月壬申,詔毋以吏卒充選舉。藍玉還。是月,振河南、北平、山東水災。九月庚寅朔,日有食之。

冬十月己卯,振湖廣饑。十一月癸丑,免山東被災田租。十二月癸亥,令殊死以下囚輸粟北邊自贖。壬申,罷天下歲織文綺。

是年,墨刺、哈梅里、高麗、占城、真臘、琉球、暹羅入貢。

二十四年春正月癸卯,大祀天地於南郊。戊申,潁國公傅友德為征虜將軍,定遠侯王弼、武定侯郭英副之,備北平邊。丁巳,免山東田租。二月壬申,耕耤田。三月戊子朔,日有食之。魏國公徐輝祖、曹國公李景隆、涼國公藍玉等備邊陜西。乙未,靖寧侯葉昇練兵甘肅。丁酉,賜許觀等進士及第、出身有差。

夏四月辛未,封皇子旃為慶王,權寧王,楩岷王,橞谷王,松韓王,模王,楹安王,桱唐王,棟郢王,彞伊王。癸未,燕王棣督傅友德諸將出塞,敗敵而還。五月戊戌,漢、衛、谷、慶、寧、岷六王練兵臨清。六月己未,詔廷臣參考歷代禮制,更定冠服、居室、器用制度。甲子,久旱錄囚。

秋七月庚子,徙富民實京師。辛丑,免畿內官田租之半。八月乙卯,秦王樉有罪,召還京師。乙丑,皇太子巡撫陜西。乙亥,都督僉事劉真、宋晟討哈梅里,敗之。九月乙酉,遣使諭西域。是月,倭寇雷州,百戶李玉、鎮撫陶鼎戰死。

冬十月丁巳,免北平、河間被水田租。十一月甲午,五開蠻叛,都督僉事茅鼎討平之。庚戌,皇太子還京師,晉王㭎來朝。辛亥,振河南水災。十二月庚午,周王橚復國。辛巳,阿資復叛,都督僉事何福討降之。

是年,天下郡縣賦役黃冊成,計戶千六十八萬四千四百三十五,丁五千六百七十七萬四千五百六十一。琉球、暹羅、別失八里、撒馬兒罕入貢。以占城有篡逆事,卻之。

二十五年春正月戊子,周王橚來朝,庚寅,河決陽武,發軍民塞之,免被水田租。乙未,大祀天地於南郊。何福討都勻、畢節諸蠻,平之。辛丑,令死困輸粟塞十。壬寅,晉王㭎、燕王棣、楚王楨、湘王柏來朝。二月戊午,召曹國公李景隆等還京師。靖寧侯葉升等練兵於河南及臨、鞏、甘、涼、延慶。都督茅鼎等平五開蠻。丙寅,耕耤田。庚辰,詔天下衛所軍以十之七屯田。三月癸未,馮勝等十四人分理陜西、山西、河南諸衛軍務。庚寅,改封豫王桂為代王,漢王楧為肅王,衛王植為遼王。

夏四月壬子,涼國公藍玉征罕東。癸丑,建昌衛指揮月魯帖木兒叛,指揮魯毅敗之。丙子,皇太子標薨。戊寅,都督聶緯、徐司馬、瞿能討月魯帖木兒,俟藍玉還,並聽節制。五月辛巳,藍玉至罕東,寇遁,遂趨建昌。己丑,振陳州原武水災。六月丁卯,西平候沐英卒於雲南。

秋七月庚辰,秦王樉復國。癸未,指揮瞿能敗月魯帖木兒於雙狼寨。八月己未,江夏侯周德興坐事誅。丁卯,馮勝、傅友德帥開國公常昇等分行山西,籍民為軍,屯田於大同、東勝,立十六衛。甲戌,給公侯歲祿,歸賜田於官。丙子,靖寧侯葉升坐胡惟庸黨誅。九月庚寅,立皇孫允炆為皇太孫,高麗李成桂幽其主瑤而自立,以國人表來請命,詔聽之,更其國號曰朝鮮。

冬十月乙亥,沐春襲封西平侯,鎮雲南。十一月甲午,藍玉擒月魯帖木兒,誅之,召玉還。十二月甲戌,宋國公馮勝、潁國公傅友德等兼東宮師保官。閏月戊戌,馮勝為總兵官,傅友德副之,練兵山西、河南、兼領屯衛。

是年,琉球中山、山南、高麗,哈梅里入貢。

二十六年春正月戊申,免天下耆民來朝。辛酉,大祀天地於南郊。二月丁丑,晉王㭎統山西、河南軍出塞,召馮勝、傅友德、常昇、王弼等還。乙酉,蜀王椿來朝。涼國公藍玉以謀反,並鶴慶侯張翼、普定侯陳桓、景川侯曹震、舳艫侯朱壽、東莞伯何榮、吏部尚書詹徽等皆坐誅。己丑,頒《逆臣錄》於天下。庚寅,耕耤田。三月辛亥,代王桂率護衛兵出塞,聽晉王節制。長興侯耿炳文練兵陜西。丙辰,馮勝、傅友德備邊山西、北平,其屬衛將校悉聽晉王、燕王節制。庚申,詔二王軍務大者始以聞。壬戌,會寧侯張溫坐藍玉黨誅。

夏四月乙亥,孝感饑,遣使乘傳發倉貸之。詔自今遇歲饑,先貸後聞,著為令。戊子,周王橚來朝。庚寅,旱,詔群臣直言得失,省獄囚。丙申,以安南擅廢立,絕其朝貢。

秋七月甲辰朔,日有食之。戊申,選秀才張宗浚等隨詹事府官分直文華殿,侍皇太孫。八月,秦、晉、燕、周、齊五王來朝。九月癸丑,代、肅、遼、慶、寧五王來朝。赦胡惟庸、藍玉餘黨。

冬十月丙申,擢國子監生六十四人為布政使等官。十二月,頒《永鑑錄》於諸王。

是年,琉球、爪哇、暹羅入貢。

二十七年春正月乙卯,大祀天地於南郊。辛酉,李景隆為平羌將軍,鎮甘肅。發天下倉穀貸貧民。三月庚子,賜張信等進士及第、出身有差。辛丑,魏國公徐輝祖、安陸侯吳傑備倭浙江。庚戌,課民樹桑棗木棉。甲子,以四方底平,收藏甲兵,示不復用。

秋八月甲戌,吳傑及永定侯張銓率致仕武臣,備倭廣東。乙亥,遣國子監生分行天下。督吏民修水利。丙戌,階、文軍亂,都督甯正為平羌將軍討之。九月,徐輝祖節制陜西沿邊諸軍。

冬十一月乙丑,潁國公傅友德坐事誅。阿資復叛,西平侯沐春擊敗之。十二月乙亥,定遠侯王弼坐事誅。

是年,烏斯藏、琉球、緬、朵甘、爪哇、撒馬兒罕、朝鮮入貢。安南來貢,卻之。

二十八年春正月丙午,階、文寇平,甯正以兵從秦王樉征洮州叛番。丁未,大祀天地於南郊。甲子,西平侯沐春擒斬阿資,越州平。是月,周王橚、晉王、㭎率河南、山西諸衛軍出塞,築城屯田。燕王棣帥總兵官周興出遼東塞。二月丁卯,宋國公馮勝坐事誅。己丑,諭戶部編民百戶為里。婚姻死喪疾病患難,里中富者助財,貧者助力。春秋耕獲,通力合作,以教民睦。

夏六月壬申,詔諸土司皆立儒學。辛巳,周興等自開原追敵至甫答迷城,不及而還。己丑,御奉天門,諭群臣曰:「朕起兵至今四十餘年,灼見情偽,懲創奸頑或法外用刑,本非常典。後嗣止頒《律》與《大誥》,不許用黥剌、剕、劓、閹割之刑。臣下敢以請者,置重典。」又曰:「朕罷丞相,設府、部、都察院分理庶政,事權歸於朝廷。嗣君不許復立丞相。臣下敢以請者置重典。皇親惟謀逆不赦。餘罪,宗親會議取上裁。法司只許舉奏,毋得擅逮。勒諸典章,永為遵守。」

秋八月丁卯,都督楊文為征南將軍,指揮韓觀、都督僉事宋晟副之,討龍州士官趙宗壽。戊辰,信國公湯和卒。辛巳,趙宗壽伏罪來朝,楊文移兵討奉議、南丹叛蠻。九月丁酉,免畿內、山東秋糧。庚戌,頒《皇明祖訓條章》於中外,「後世有言更祖制者,以奸臣論」。十一月乙亥,奉議、南丹蠻悉平。十二月壬辰,詔河南、山東桑棗及二十七年後新墾田,毋徵稅。

是年,朝鮮、琉球、暹羅入貢。

二十九年春正月壬申,大祀天地於南郊。二月癸卯,征虜前將軍胡冕討郴、桂蠻,平之。辛亥,燕王棣帥師巡大寧,周世子有燉帥師巡北平關隘。三月辛酉,楚王楨、湘王柏來朝。甲子,燕王敗敵於徹徹兒山,又追敗之於兀良哈禿城而還。

秋八月丁未,免應天、太平五府田租。九月乙亥,召致仕武臣二千五百餘人入朝,大賚之,各進秩一級。

是年,琉球、安南、朝鮮、烏斯茂入貢。

三十年春正月丙辰,耿炳文為征西將軍,郭英副之,巡西北邊。丙寅,大祀天地於南郊。丁卯,置行太僕寺於山西、北平、陜西、甘肅、遼東,掌馬政。己巳,左都督楊文屯田遼東。是月,沔縣盜起,詔耿炳文討之。二月庚寅,水西蠻叛,都督僉事顧成為征南將軍,討平之。三月癸丑,賜陳安阜等進士及第、出身有差。庚辰,古州蠻叛,龍里千戶吳得、鎮撫井孚戰死。

夏四月己亥,都指揮齊讓為平羌將軍,討之。壬寅,水西蠻平。五月壬子朔,日有食之。乙卯,楚王楨、湘王柏帥師討古州蠻。六月辛巳,賜禮部覆試貢士韓克忠等進士及第、出身有差。己酉,駙馬都尉歐陽倫有罪賜死。

秋八月丁亥,河決開封。甲午,李景隆為征虜大將軍,練兵河南。九月庚戌,漢、沔寇平。戊辰,麓川平緬土酋刀幹孟逐其宣慰使思倫發以叛。乙亥,都督楊文為征虜將軍,代齊讓。

冬十月戊子,停遼東海運。辛卯,耿炳文練兵陜西。乙未,重建國子監先師廟成。十一月癸酉,沐春為征虜前將軍,都督何福等副之,討刀幹孟。

是年,琉球、占城、朝鮮、暹羅、烏斯藏、泥八剌入貢。

三十一年春正月壬戌,大祀天地於南郊。乙丑,遣使之山東、河南課耕。二月乙酉,倭寇寧海,指揮陶鐸擊敗之。辛丑,古州蠻平,召楊文還。甲辰,都督僉事徐凱討平麼些蠻。

夏四月庚辰,廷臣以朝鮮屢生釁隙請討,不許。五月丁未,沐春擊刀幹孟,大敗之。甲寅,帝不豫。戊午,都督楊文從燕王棣,武定侯郭英從遼王植,備禦開平,俱聽燕王節制。

閏月癸未,帝疾大漸。乙酉,崩於西宮,年七十有一。遺詔曰:「朕膺天命三十有一年,憂危積心,日勤不怠,務有益於民。奈起自寒微,無古人之博知,好善惡惡,不及遠矣。今得萬物自然之理,其奚哀念之有。皇太孫允炆仁明孝友,天下歸心,宜登大位。內外文武臣僚同心輔政,以安吾民。喪祭儀物,毋用金玉。孝陵山川因其故,毋改作。天下臣民,哭臨三日,皆釋服,毋妨嫁娶。諸王臨國中,毋至京師。諸不在令中者,推此令從事。」辛卯,葬孝陵。謚曰高皇帝,廟號太祖。永樂元年,謚聖神文武欽明啟運俊德成功統天大孝高皇帝。嘉靖十七年,增謚開天行道肇紀立極大聖至神仁文義武俊德成功高皇帝。

帝天授智勇,統一方夏,緯武經文,為漢、唐、宋諸君所未及。當其肇造之初,能沉幾觀變,次第經略,綽有成算。嘗與諸臣論取天下之略,曰:「朕遭時喪亂,初起鄉土,本圖自全。及渡江以來,觀群雄所為,徒為生民之患,而張士誠、陳友諒尤為巨蠹。士誠恃富,友諒恃強,朕獨無所恃。惟不嗜殺人,布信義,行節儉,與卿等同心共濟。初與二寇相持,士誠尤逼近。或謂宜先擊之。朕以友諒志驕,士誠器小,志驕則好生事,器小則無遠圓,故先攻友諒。鄱陽之役,士誠卒不能出姑蘇一步以為之援。向使先攻士誠,浙西負固堅守,友諒必空國而來,吾腹背受敵矣。二寇既除,北定中原,所以先山東、次河洛,止潼關之兵不遽取秦、隴者,蓋擴廓帖木兒、李思齊、張思道皆百戰之餘,未肯遽下,急之則併力一隅,猝未易定,故出其不意,反旆而北。燕都既舉,然後西征。張、李望絕勢窮,不戰而克,然擴廓猶力抗不屈。向令未下燕都,驟與角力,勝負未可知也。」帝之雄才大略,料敵制勝,率類此。故能戡定禍亂,以有天下。語云「天道後起者勝」,豈偶然哉。

贊曰:太祖以聰明神武之資,抱濟世安民之志,乘時應運,豪傑景從,戡亂摧強,十五載而成帝業。崛起布衣,奄奠海宇,西漢以後所未有也。懲元政廢弛,治尚嚴峻。而能禮致耆儒,考禮定樂,昭揭經義,尊崇正學,加恩勝國,澄清吏治,修人紀,崇鳳都,正後宮名義,內治肅清,禁宦豎不得干政,五府六部官職相維,置衛屯田,兵食俱足。武定禍亂,文致太平,太祖實身兼之。至於雅尚志節,聽蔡子英北歸。晚歲憂民益切,嘗以一歲開支河暨塘堰數萬以利農桑、備旱潦。用此子孫承業二百餘年,士重名義,閭閻充實。至今苗裔蒙澤,尚如東樓、白馬,世承先祀,有以哉。


\end{pinyinscope}