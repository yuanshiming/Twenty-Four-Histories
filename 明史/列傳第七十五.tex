\article{列傳第七十五}

\begin{pinyinscope}
○何鑑馬中錫陸完洪鐘陳鎬蔣昇陳金俞諫周南孫祿馬昊

何鑑,字世光,浙江新昌人。成化五年進士。授宜興知縣。徵拜御史,巡宣府、大同。劾巡撫鄭寧以下數十人不職,按裨將孟璽等罪。還巡太倉。總督太監卒犯法,逮治之,為所構,下錦衣獄。得釋,再按江北。鳳陽皇陵所在,近境取寸木,法皆死,陵軍多倚禁虐民。鑑請以山麓為限,他樵採勿禁,遂著為令。出為河南知府。振累歲饑,條行荒政十事。歷四川左、右布政使。

弘治六年以右副都御史巡撫江南,兼理杭、嘉、湖三府稅糧。蘇、松水災,用便宜發漕米十五萬石振之。與侍郎徐貫疏吳淞、白茆諸渠,泄水入海,水患以除。復巡撫山東,遷刑部侍郎。母憂去。

十八年還朝。時承平久,生齒日繁。孝宗覽天下戶籍數乃視國初反減,咎所司溺職,欲釐正之。敕鑑以故官兼左僉都御史往河南、湖廣、陜西閱實戶口。得戶二十三萬五千有奇,口七十三萬九千有奇,因疏善後十事及軍民利病以聞。會孝宗已崩,武宗悉採納之。

正德二年拜南京兵部尚書參贊機務。鑑前撫江南,嘗按千戶張文冕罪,文冕亡去。至是構於劉瑾,而瑾亦嗛鑒不與通,遂坐以事連罰米。貧不能償,奏愬獲免。

六年正月,召為刑部尚書。時大盜並起,劉寵、劉宸、楊虎、劉惠、齊彥名、朱諒等亂畿輔;方四、曹甫、藍廷瑞、鄢本恕等躪四川;汪澄二、羅光權、王浩八、王鈺五等擾江西,皆稱王。四方告急無虛日。兵部尚書王敞不能辦賊。帝既命洪鐘、陳金、馬中錫督師分討。其年五月,罷敞,以鑑代之。鑑乃選將練兵,錄民間材武士,令鄉聚悉樹柵浚溝,團結相救。河南、山西兵守黃河,斷太行。京操班軍,留守所在城邑。每漕艘運卒一人屯河濱,護運道,通行旅。文武大吏軼賊,請敕峻責之,而褒縣令能擊賊者。以中錫玩寇,奏遣陸完代還,調邊將從完討賊。賊連為邊軍所破,奔迸四出。會中官谷大用、伏羌伯毛銳率師駐臨清,賊遂謀以十二月朔伺帝省牲南郊,乘間犯駕,先一日趨霸州。鑒立奏聞,夜設備。厥明,帝召問鑑。鑒請早出安人心,遂成禮而還。賊知有備,西掠保定諸州縣以去。河南巡撫鄧璋請濟師,鑒言:「山東賊不及萬,官軍奚啻十倍。緣勢要私人營充頭目,撓律攘功,失將士心。請盡遣若屬還。都指揮以下失事,即軍前行戮。益調邊軍助璋。」帝悉從之。尋以捷書屢聞,加鑑太子少保。

明年正月,賊突霸州,京師戒嚴。鑒令邊兵亟邀賊,賊遁去。賊渠楊虎、朱諒死,其黨分擾山東、河南。鑑以山東賊劉寵、劉宸、齊彥名等,責邊將許泰、郤永、劉暉、李鋐;以河南賊劉惠、趙鐩、刑老虎等,責邊將馮禎、時源、神周、金輔。未幾,毛銳敗績,與大用俱召還。鑑乃請用彭澤,與仇鉞同辦河南賊,而以山東賊專委陸完。五月,河南賊平。七月,山東餘賊亦平。陳金、洪鐘亦以次平江西、四川諸賊。帝喜,加鑑太子太保,蔭子錦衣世百戶。鑑乃上言:「群盜蕩平,民罹兵久,乞量免田租,多方振贍。黜貪殘長吏,停不急工役。還民故業,貸以牛種,復其家三年。有訐舊事及怙惡者,並置於理。」帝悉報可。

先是,七月中,鑑以群盜未盡,請留邊將劉暉戍山東,時源戍河南,郤永戍畿輔,李鋐戍淮、揚,各假總兵之職,俟事寧始罷。仇鉞言,邊軍久勞,風土不習,人馬俱病。今賊已漸平,請留三之一討賊,餘悉遣還。廷議,二人議俱是,請四將各千人鎮壓,他將許泰、神周、金輔、溫恭輩俱統所部還邊鎮。帝許之,命延綏軍徑還,遼東、宣府,大同軍過闕勞賜。

帝時好弄兵。群小寵幸者言邊軍憨健過京軍遠甚,宜留之京營。帝以為然。至十一月,三鎮軍畢至,遂命留之,以京軍往代。鑒力陳不可,廷臣集議,復極言其害,帝竟不從。自是,邊軍於大內團操,號為「外四家軍」,而江彬進用矣。

八年,宣府送迤北降人脫脫太等至京,命充御馬監勇士。鑑等上言:「漢、魏徙氐、羌於關中,郭欽、江統皆勸晉武早絕亂階。苻堅處鮮卑於漢南,蔡融亦慮其窺測虛實。今使降人出入禁中,假寵踰分,且生慢侮。萬一北寇聞之,潛使黠賊偽降,以為間諜,寧不為將來患哉?」帝不聽。

寧王宸濠謀復護衛,鑑力遏之。都督白玉以失事罷,厚賄豹房諸倖臣求復,鑑執不從。諸倖臣嗾詗事者發鑒家僮取將校金錢,言官遂交章劾鑑,致仕去。閱九年卒,年八十。

馬中錫,字天祿,故城人。父偉,為唐府長史,以直諫忤王,械送京師,而盡縲其家人。中錫以幼免,乃奔訴巡按御史。御史言於王,釋其家。復奉母走京師訴冤,父竟得白,終處州知府。

中錫舉成化十年鄉試第一,明年成進士,授刑科給事中。萬貴妃弟通驕橫,再疏斥之,再被杖。公主侵畿內田,勘還之民。又嘗劾汪直違恣罪。歷陜西督學副使。

弘治五年,召為大理右少卿。南京守備太監蔣琮與兵部郎中婁性、指揮石文通相訐,連數百人,遣官按,不服。中錫偕司禮太監趙忠等往,一訊得實。性除名,琮下獄抵罪。擢右副都御史,巡撫宣府。劾罷貪耄總兵官馬儀,革鎮守以下私役軍士,使隸尺籍。寇嘗犯邊,督軍敗之。引疾歸,中外交薦。

武宗即位,起撫遼東。還屯田於軍,而劾鎮守太監朱秀置官店、擅馬市諸罪。正德元年入歷兵部左右侍郎。劉瑾初得志,其黨朱瀛冒邊功至數百人。尚書閻仲宇許之,中錫持不可。瑾大恚,中旨改南京工部。明年勒致仕。其冬,逮繫詔獄,械送遼東,責償所收腐粟。踰年事竣,斥為民。瑾誅,起撫大同。中錫居官廉,所至革弊任怨,以故有名。

六年三月,賊劉六等起,吏部尚書楊一清建議遣大臣節制諸道兵。乃薦中錫為右都御史提督軍務,與惠安伯張偉統禁兵南征。

劉六名寵,其弟七名宸,文安人也,並驍悍善騎射。先是,有司患盜,召寵、宸及其黨楊虎、齊彥名等協捕,頻有功。會劉瑾家人梁洪征賄於寵等不得,誣為盜。遣寧杲、柳尚義繪形捕之,破其家。寵等乃投大盜張茂。茂家高樓重屋,復壁深窖,素招亡命為逋逃主。宦官張忠與鄰,茂結為兄,夤緣馬永成、谷大用、于經輩得出入豹房,侍帝蹴鞠,而乘間為盜如故。後數為河間參將袁彪所敗。茂窘,求救於忠。忠置酒私第,招茂、彪東西坐。酒酣,舉觴屬彪字茂曰:「彥實吾弟也,自今毋相厄。」又舉觴屬茂曰:「袁公善爾,爾慎毋犯河間。」彪畏忠,唯唯而已。已,茂為寧杲所擒,寵等相率詣京謀自首。忠與永成為請於帝,且曰:「必獻萬金乃赦。」寵、宸不能辦,逃去。既而瑾誅,有詔許自首。寵等乃出詣官。兵部奏赦之,令捕他盜自效。寵等憚要束,未幾復叛。黨日眾,所至,陷城殺將吏。

中錫等受命出師,敗賊於彰德,既又敗之河間,進左都御史。然賊方熾,諸將率畏懦,莫敢當其鋒,或反與之結。參將桑玉嘗遇賊文安村中。寵、宸窘蹙,跳民家樓上,欲自剄。而玉素受賊賂,故緩之。有頃,彥名持大刀至,殺傷數十人,大呼抵樓下。寵、宸知救至,出,射殺數人。玉大敗。參將宋振禦賊棗強,不發一矢,城遂陷,死者七千人。

當是時,寵、宸等自畿輔犯山東、河南,南下湖廣,抵江西。復自南而北,直窺霸州。楊虎等由河北入山西,復東抵文安,與寵等合,破邑百數,縱橫數千里,所過若無人。中錫雖有時望,不習兵。偉亦紈褲子,見賊強,諸將怯,度不能破賊,乃議招撫。謂盜本良民,由酷吏寧杲與中官貪黷所激,若推誠待之,可毋戰降也。遂下令:賊所在勿捕,過勿邀擊,饑渴則食飲之,降者待以不死。賊聞,欲就撫,相戒毋焚掠。猶豫未定。而朝廷以京軍弱,議發邊兵。中錫欲戰,則兵未集,欲撫,則賊時向背,終不得要領。既建議主撫,不能變。會寵等聞邊兵且至,退屯德州桑園。中錫肩輿入其營,與酒食,開誠慰諭之。眾拜且泣,送馬為壽。寵慷慨請降,宸乃仰天咨嗟曰:「騎虎不得下。今奄臣柄國,人所知也。馬都堂能自主乎?」遂罷會。而是時方詔懸賞格購賊。寵等偵知之,益疑懼,徑去,焚掠如故。獨至故城,戒毋犯馬都堂家。由是,中錫謗大起,謂其以家故縱賊。言官交劾之,下詔切責。中錫猶堅持其說以請。兵部尚書何鑑謂「賊誠解甲則貰死,即不然,毋為所誑」。既而寵等終不降,乃遣侍郎陸完督師,而召中錫、偉還。

初,中錫受命討賊,大學士楊廷和謂楊一清曰:「彼文士耳,不足任也。」竟無功,與偉同下獄論死。中錫死獄中,偉革爵。十一年,巡按御史盧雍追訟中錫冤,謂:「賊實聽撫,僉事許承芳忌之,潛請益兵,疑賊心。及賊再受約,方至軍門,而檻車已就道矣。」朝廷乃復中錫官,賜祭,予蔭。

陸完,字全卿,長洲人。為諸生。中官王敬至蘇,以事庭曳諸生。諸生競起擊之,完不與。惡完者中之,敬遂首列完名上聞。巡撫王恕極論敬罪,完乃得免。舉成化二十三年進士。謁選,恕方為吏部,曰:「是嘗擊奄人者,當為御史。」入臺,果有聲。

正德初,歷江西按察使。寧王宸濠雅重之,時召預曲宴,以金罍為贈。三年冬,擢右僉都御史,巡撫宣府。劉瑾惡完赴闕後期,命以試職視事。明年夏,復改南院,督江防軍。完以都御史試職非故事,懼甚,賄瑾,召為左僉都御史。五年春,拜兵部侍郎。瑾敗,言者劾其黨附,帝不問。

明年,霸州賊劉六、劉七等起,奉楊虎為首。惠安伯張偉、右都御史馬中錫師出無功,逮繫論死。八月,詔完兼右僉都御史提督軍務,統京營、宣府、延綏軍討之。行及涿州,忽傳賊且逼京師,命還軍入衛。會副總兵許泰、游擊郤永等敗楊虎等於霸州,賊南走,京師始解嚴。指揮賀勇等再敗賊信安,副總兵馮楨復大敗之阜城,分兵追擊。賊東圍滄州。會劉六、七中流矢,乃解而南,陷山東縣二十。楊虎兵亦北殘威縣、新河。於是完頻請濟師。益發遼東、山西諸鎮兵逐賊。賊益南,圍濟寧,焚運舟,轉寇曹州。楨、泰、永擊斬二千餘人,獲其魁朱諒。錄功,進完右都御史,諸將皆增秩。中官谷大用、張忠意賊旦暮平,乃自請督師。詔以大用總督軍務,伏羌伯毛銳充總兵官,忠監神鎗,統京軍五千人,會完討賊。

時劉六等縱橫沂、莒間,而楊虎陷宿遷,執淮安知府劉祥、靈璧知縣陳伯安,連陷虹、永城、虞城、夏邑及歸德州。邊兵追及,賊退至小黃河渡口。百戶夏時設伏蹙之,虎溺死。餘賊奔河南,推劉惠為首,大敗副總兵白玉軍,攻陷沈丘,殺都指揮王保,執都指揮潘翀,北陷鹿邑。有陳翰者,與寧龍謀奉惠為奉天征討大元帥,趙鐩副之。翰自為侍謀軍國重務元帥府長史,與龍立東西二廠治事。分其軍為二十八營,以應列宿,營各置都督,聚眾至十三萬。欲牽制官軍,於是惠、鐩擾河南,劉六及齊彥名等擾山東,黨分為二。已而六復轉而北,永敗之濰縣。還趨霸州,帝將出郊省牲,聞之懼,急召完赴援,完擊破之文安。賊南至湯陰,完又督諸將追敗之,先後俘斬千人。

當是時,六等眾號數萬,然多脅從,精銳不過千餘人。自兵部下首功令,官軍追賊,賊輒驅良民前行,急則棄所掠逸去。官軍所殺皆良民,以故捷書屢奏,而賊勢不衰。

明年正月,六等復突霸州,京師戒嚴。詔完及大用、銳還禦近畿,賊乃西掠博野,攻蠡縣、臨城。大用、銳與遇於長垣,大敗。廷議召二人還,別命都御史彭澤同咸寧伯仇鉞辦河南賊,以畿輔、山東賊委完。完遣永追敗劉六於宋家莊。賊南犯滕縣,副總兵劉暉大敗之,賊遂奔登、萊海套。完師次平度,檄永、玉與游擊溫恭三道進攻,命副總兵張俊、李鋐及泰、暉分軍邀其奔逸。賊走,連戰皆大敗之,賊乃變服易馬而遁,先後擒斬二千六百餘人。賊止三百人北走,沿途招聚,勢復張。剽香河、寶坻、玉田,轉攻武清。游擊王杲敗沒,巡撫寧杲兵亦敗,畿輔復震動。而賊轉南至冠縣,暉襲敗之,指揮張勛又敗之平原。賊南奔邳州,渡河抵固始。會河南賊已平,劉六等勢益衰,遂走湖廣。奪舟到夏口,遇都御史馬炳然,殺之。復登陸,焚漢口,為指揮滿弼等追及,劉六中流矢,與子仲淮赴水死。

劉七、齊彥名率五百人舟行,自黃州順流抵鎮江。南京告急,完疾趨而南。帝命彭澤、仇鉞會完軍進剿。大兵盡集江南、北,賊猶乘潮上下肆掠。操江武靖伯趙弘澤、都御史陳世良遇之,敗績,死者無算。七月,賊治舟孟瀆。完等至鎮江,留鉞防守,令恭以騎駐江北,暉、永以舟趨江陰,完率都指揮孫文、傅鎧趨福山港。賊懼,抵通州。颶風大作,棄舟走保狼山。完命同知羅瑋夜導軍登山南蹙之。彥名中槍死,七中矢亦赴水死,餘賊盡平。還朝,進完太子少保左都御史,廕子錦衣世百戶。明年代何鑒為兵部尚書。

完有才智,急功名,善交權勢。劉暉、許泰、江彬皆其部將,後並寵倖用事,完遂行其力。

時宸濠已萌異志。聞完為兵部,致書盛陳舊好,欲復護衛及屯田。完答書,令以祖制為詞。宸濠遂遣人輦金帛巨萬,寓所善教坊臧賢家,遍遺用事貴人,屬錢寧為內主。比奏下,完遂為復請,而以屯田屬戶部,請付廷議。內閣擬旨上,並予之。舉朝嘩然。六科給事中高淓、十三道御史汪賜等力爭,章並下部,久不覆。南京給事中徐文溥繼言之,完乃請納諫官言,帝竟不許。十年改吏部尚書。

宸濠反,就執。中官張永至南昌,搜其籍,得完平日交通事,上之。帝大怒。還至通州,執完。收其母妻子女,封識其家。比還京,反縛之竿,揭姓名於首,雜俘囚中,列凱旋前部以入,將置極刑。值武宗崩,世宗立,法司復奏完交外籓而遺金不卻,處護衛而執奏不堅,當斬。完復乞哀,下廷臣覆讞。以平賊功,在八議之列,遂得減死,戍福建靖海衛。母年九十餘,竟死獄中。

初,完嘗夢至一山曰「大武」。及抵戍所,有山如其名,歎曰:「吾戍已久定,何所逃乎!」竟卒於戍所。

洪鐘,字宣之,錢塘人。成化十一年進士。為刑部主事,遷郎中,奉命安輯江西、福建流民。還言福建武平、上杭、清流、永定,江西安遠、龍南,廣東程鄉皆流移錯雜,習鬥爭,易亂,宜及平時令有司立鄉社學,教之《詩》《書》禮讓。

弘治初,再遷四川按察使。馬湖土知府安鰲恣淫虐,土人怨之刺骨,有司利其金置不問,遷延二十年。僉事曲銳請巡按御史張鸞按治,鐘贊決,捕鰲送京師,置極刑。安氏自唐以來世有馬湖,至是改流官,一方始靖。歷江西、福建左、右布政使。

十一年擢右副都御史,巡撫順天。整飭薊州邊備,建議增築塞垣。自山海關西北至密雲古北口、黃花鎮直抵居庸,延亙千餘里,繕復城堡二百七十所,悉城緣邊諸縣,因奏減防秋兵六千人,歲省挽輸犒賚費數萬計。所部潮河川去京師二百里,居兩山間,廣百餘丈,水漲成巨浸,水退則坦然平陸,寇得長驅直入。鐘言:「關以東三里許,其山外高內庳,約餘二丈,可鑿為兩渠,分殺水勢,而於口外斜築石堰以束水。置關堰內,守以百人,使寇不得馳突,可免京師北顧憂,且得屯種河堧地。」兵部尚書馬文升等請從之。比興工,鑿山,山石崩,壓死者數百人。御史弋福、給事中馬予聰等劾鐘。巡撫張亙等請罷役,不聽。未幾,工成,侍郎張達偕司禮中官往視。還言石洞僅洩小水,地近邊垣多沙石,不利耕種。給事中屈伸等劾鐘欺妄三罪,諸言官及兵部皆請逮鐘。帝以鐘為國繕邊,不當罪,停俸三月。

正德元年,由巡撫貴州召督漕運兼巡撫江北。明年就進右都御史。蘇、松、浙江運舟由下港口及孟瀆河溯大江以達瓜洲,遠涉二百八十餘里,往往遭風濤。鐘言:「孟瀆對江有夾河,可抵白塔河口。舊置四閘,徑四十里。至宜陵鎮再折而北,即抵揚州運河。開濬為便。」從之。改掌南京都察院,就遷刑部尚書。四年冬,加太子少保兼左都御史,掌院事。

五年春,湖廣歲饑盜起。命鐘以本官總制軍務,陜西、河南、四川亦隸焉。沔陽賊楊清、丘仁等僭稱天王、將軍,出沒洞庭間。圍岳州,陷臨湘,官軍屢失利。鐘及總兵官毛倫檄都指揮潘勛、柴奎,布政使陳鎬,副使蔣昇擊破之於麻穰灘,擒斬七百四十餘人,賊遂平。初,鐘掌院事,劉瑾方熾。及瑾誅,言官劾鐘徇瑾撻御史。朝議以鐘討賊,置不問。

時保寧賊藍廷瑞自稱順天王,鄢本恕自稱「刮地王」,其黨廖惠稱「掃地王」,眾十萬餘,置四十八總管,延蔓陜西、湖廣之境。廷瑞與惠謀據保寧,本恕謀據漢中,取鄖陽,由荊、襄東下。巡撫林俊方議遏通江,而惠已至,攻陷其城,殺參議黃瓚,僉事錢朝鳳等遁去。適官軍自他郡還,賊疑援兵至,亦遁。俊益發羅、回及石硅士兵助朝鳳進剿,參議公勉仁亦會。龍灘河漲,賊半渡,羅、回奮擊之,擒斬八百餘人,墜崖溺水甚眾。俊復遣知府張敏、何珊等追之,獲惠,餘眾奔陜西西鄉。鐘乃下令招撫,歸者萬餘人。既而賊收散亡,陷營山,殺僉事王源,縱掠蓬、劍二州。

鐘赴四川,與俊議多不合,軍機牽制,賊益熾。已,乃檄陜西、湖廣、河南兵分道進,湖廣兵先追及於陜西石泉。廷瑞走漢中,都指揮金冕圍之。陜西巡撫藍章方駐漢中,廷瑞遣其黨何虎詣章,乞還川就撫。章以廷瑞本川賊,恐急之必致死,陜且受患,遂令冕護之出境。廷瑞既入川,求降,鐘等令至東鄉聽撫。賊意在緩師,遷延累月,依山結營,要求營山縣或臨江市屯其眾,遣官為質。鐘令漢中通判羅賢入其營。本恕來謁,約既定,會官軍有殺其樵採者,賊復疑懼,遂殺賢,剽如故。官軍為七壘守之,賊不得逸,其黨漸潰。廷瑞以所掠女子詐為己女,結婚於永順土舍彭世麟,冀得間逸去。世麟密白鐘,鐘授方略使圖之。及期,廷瑞、本恕暨其黨王金珠等二十八人咸來會。伏發悉就擒,惟廖麻子得脫。其眾聞變,驚潰渡河。鐘遣兵追擊,俘斬七百餘人,以功進太子太保。

未幾,廖麻子及其黨曹甫掠營山、蓬州。七年,總兵官楊宏,副使張敏、馬昊、何珊等合擊之。賊勢蹙,鐘乃議招撫。敏以單騎詣甫營,甫聽命,遂赴軍門受約束,歸散其黨。而麻子忿甫背己,殺之,并其眾,轉掠川東。官軍不敢擊,潛躡賊後,馘良民為功,土兵虐尤甚。時有謠曰:「賊如梳,軍如篦,土兵如剃。」巡按御史王綸、紀功御史汪景芳劾鐘縱兵不戢。綸復奏鐘樂飲縱遊,致賊自合州渡江陷州縣。詔召鐘還,以彭澤代,鐘遂乞歸。嘉靖三年卒,謚襄惠。

陳鎬,會稽人。成化二十三年進士。既平賊,就遷右副都御史,巡撫湖廣。蔣昇,祁陽人,鎬同年進士。

陳金,字汝礪,應城人,徙武昌。祖坦,夔州知府。父琳,廣西僉事。金舉成化八年進士,除婺源知縣,擢南京御史。

弘治初,出按浙江,還因災異劾文武大僚十九人,侍郎丁永中、南京大理卿吳道宏、南寧伯毛文等多罷去。尋遷山西副使,歷雲南左布政使,討平竹子箐叛苗。

十三年,就拜右副都御史,巡撫其地。孟養酋思祿與孟密酋思揲構兵積年。金奉詔發緬甸、乾崖、隴川、南甸諸部兵,聚糧十二萬,為征討計,而遣參議郭緒往撫之。思祿懼,遂罷兵修貢,金以功賚銀幣。貴州兵敗賊婦米魯,米魯退攻平夷衛及大河、扼勒諸堡。金發兵連破之,增俸一等,召為南京戶部右侍郎。

正德改元,給事中周璽等劾不職大臣,金與焉。詔不問。金以母老乞歸,不允。尋以右都御史總督兩廣軍務。時內臣韋霦等建議,請輸兩廣各司所貯銀於京師。金疏不可,詔留二十餘萬。馬平、洛容僮猖獗,金偕總兵官毛銳發兵十三萬征之,俘斬七千餘人,進左都御史。斷藤峽苗時出剽。金念苗嗜魚鹽,可以利縻也,乃立約束,令民與苗市,改峽曰永通。苗性貪而黠,初陽受約,既乃不予直,殺掠益甚。潯州人為語曰:「永通不通,來葬江中,誰其作者?噫,陳公!」蓋咎金失計也。

三年十月,遷南京戶部尚書。明年冬,召為左都御史,未聞命,以母喪歸。六年二月,江西盜起。詔起金故官,總制軍務。南畿、浙江、福建、廣東、湖廣文武將吏俱隸焉。許便宜從事,都指揮以下不用命者專刑戮。當是時,撫州則東鄉賊王鈺五、徐仰三、傅傑一、揭端三等;南昌則姚源賊汪澄二、王浩八、殷勇十、洪瑞七等;瑞州則華林賊羅光權、陳福一等;而贛州大帽山賊何積欽等又起。官軍累年不能克。金以屬郡兵不足用,奏調廣西狼土兵。明年二月先進兵東鄉,遣參議徐蕃等分屯要害,而令副總兵張勇,土官岑、岑猛各統官兵、目兵擊賊熟塘。進戰南甗,追敗之赤岸蔭嶺。擒仰三,馘鈺五等,克柵二百六十五,斬首萬一千六百餘級,俘七百五十餘人。五月移師姚源,令參政董朴、吳廷舉等分營餘干、安仁、貴溪、鄱陽、樂平遏賊,而親統大軍搗其巢,勇十重創死。會張勇以目兵至,毒弩射殺瑞七、成七等,俘斬共五千餘人。七月乘勝斬光權。華林賊盡平。又督副使王秩等擊大帽山賊,獲積欽,俘斬千七百餘人。半歲間,剿賊幾盡。遂即東鄉立縣,並立萬年縣,招降人居之。前後每奏捷,輒賜璽書嘉勞,賚銀幣。加太子少保,蔭子錦衣世百戶。

金累破劇賊,然所用目兵貪殘嗜殺,剽掠甚於賊,有巨族數百口闔門罹害者。所獲婦女率指為賊屬,載數千艘去。民間謠曰:「土賊猶可,土兵殺我。」金亦知民患之,方倚其力,不為禁。又不能持廉,軍資頗私入。功雖多,士民皆深怨焉。

東鄉之役,兵縱弩射,趫捷若飛,賊大窘。鎏兵要賞千金,金靳不予,乃縱賊使逸。桀黠者多不死,尚數千人。金急欲成功,遂下令招撫。其破姚源賊也,金喜,謂功在旦夕,與將吏置酒高會。賊覘諸要害無守者,乃悉所有賂目兵,乘暮遁去。時賊絕爨已三日,自分必死,沿途棄稚弱,散婦女。及抵貴溪,始得一飽食,遂轉掠衢、徽間。金知失策,亦下令招降。賊首王浩八等故偽降以緩官兵,攻剽如故,卒不能盡賊。紀功給事中黎奭及兩京言官交章劾金。乃召金還,以俞諫代。金遂請終喪去。

十年再起,督兩廣軍務。府江賊王公珣等為亂,金集諸道兵偕總兵官郭勛等分六路討之,斬公珣,大有所俘獲。加少保太子太保,廕子如初。復以饒平捷,詔子先受廕者進一秩。金承召還朝,道得疾歸,詔強起之。十四年冬入掌都察院事。世宗立,請老,命乘傳還。久之,卒。

俞諫,字良佐,桐廬人。父藎,舉進士,官御史,按江西,治外戚王氏、萬氏宗族恣橫罪。坐事,謫澧州判官。大築陂堰,溉田可萬頃。累遷鄖陽知府。

諫舉弘治三年進士,授長清知縣,擢南京御史。遷河南僉事,擒嵩賊呂梅。歷江西參議,平大帽山賊。遷廣東副使,中道召為大理少卿。

正德六年擢右僉都御史,治水蘇、杭諸府,修治圩塘,民享其利。尋進右副都御史,提督操江。八年春,姚源降賊王浩八叛,詔以諫代陳金督江西、浙江、福建諸軍討之。時浩八眾萬餘,屯浙江開化,為同知伍文定等所敗,遁還江西德興,以所執都指揮白弘、江洪為質,求撫於按察使王秩。秩受之,為傳送姚源。浩八奔據貴溪裴源山,餘眾復集,連營十里。諫令秩與副使胡世寧、參政吳廷舉列屯要害,斷其歸路,而躬與都督李鋐乘夜冒雨潛進。大破之,俘斬數千人,遂擒浩八。其黨潰走玉山。諫與南贛巡撫周南、江西巡撫任漢復擊斬七百餘人。餘賊奔姚源,諫督廷舉等進剿,逼擒之。

諫懲金失,一意用兵,而任漢懦。先為布政使,嘗贊金主撫。雖亟上首功,追賊緩,餘當復起。先是,東鄉賊為金所敗乞降,隸世寧,號新兵,而剽掠如故。既懼罪復叛,遣參將桂勇等討擒之。萬年雖立縣,賊尚眾,吏胥多賊黨,官府動息必知之。副使李情治峻急,眾欲叛,畏鋐在餘干不敢發。會鋐卒,王垂七、胡念二等遂作亂。殺情及饒州通判陳達、秦碧,指揮邢世臣等,焚廨舍。諫發兵擒之,亂乃定。言官劾諫及漢、南。兵部請召漢還,命諫兼領巡撫。明年擊臨川賊,斬其魁,而遣參將李隆擊新淦賊。賊踞萬山中,僭稱王且八年。隆等深入,悉就擒,俘斬千七百餘人。錄功,進諫右都御史,巡撫如故。劇賊徐九齡者,初嘯聚建昌、醴源。已,出沒江、湖間,積三十年。黃州、德安、九江、安慶、池州、太平咸被其害。諫討斬之,群盜悉平。寧王宸濠諷御史張鰲山劾諫,十一年召還,遂乞致仕。

嘉靖改元,用薦起故官,總督漕運。青州礦盜王堂等起顏神鎮,流劫東昌、袞州、濟南。都指揮楊紀及指揮楊浩等擊之,浩死,紀僅免。詔責山東將吏,於是諸臣分道逐賊,賊不復屯聚,流劫金鄉、魚臺間。突曹州,欲渡河不得,復掠考城並河西岸,至東明、長垣。河南及保定守臣咸告急。賊黨王友賢等轉掠祥符、封丘,南抵徐州。廷議以諸道巡撫權位相埒,乃命諫與都督魯綱並提督兩畿、山東、河南軍務,以便宜節制諸道兵討之。賊復流至考城。官軍方欲擊,而河南降賊張進引三百騎馳至。中都留守顏愷與俱前,方戰,進忽三麾其旗先卻。賊乘之,官軍大潰,將士死者八百餘人。諫等連營進,賊始滅。其秋,召掌都察院事。踰年卒官,贈太子太保,謚莊襄。

周南,字文化,縉雲人。成化十四年進士。除六合知縣,擢御史,出按畿輔。弘治初,再按廣東,劾總兵官柳景。歷江西右布政使,擢右副都御史,巡撫大同。

武宗初立,寇入宣府,參將陳雄等邀擊,敗之。錄功,增南俸一秩,母喪歸。正德三年,劉瑾擅政,以大同倉粟有浥爛者,逮南及督糧郎中孫祿下詔獄,械送大同,責倍輸。會赦,大同總兵官葉椿等為請,免其倍數。輸畢,釋為民。瑾誅,以故官撫宣府不就,引病歸。明年起督南、贛軍務。南贛巡撫之設,自南始。

汀州大帽山賊張時旺、黃鏞、劉隆、李四仔等聚眾稱王,攻剽城邑,延及江西、廣東之境,數年不靖,官軍討之輒敗。推官莫仲昭、知縣蔣璣、指揮楊澤等被執,賊勢愈熾。南集諸道兵擊之龍牙,擒時旺。義民林富別擊斬鏞於鐵坑。其他諸寨為指揮孫堂等所破。而副使楊璋、僉事凌相等亦擊隆、四仔,擒之。先後斬獲五千人,仲昭等得逸還。捷聞,賜敕獎勞。南乃移師會總督陳金,共平姚源諸賊,境內遂寧。九年春,進右都御史,總督兩廣軍務。踰年乞歸,卒。贈太子少保。

孫祿,棲霞人。弘治九年進士。由戶部主事歷郎中。瑾敗,起故官,累遷至應天府尹。

馬昊,本姓鄒,字宗大,寧夏人。弘治十二年進士。由行人選御史。正德初,遷山東僉事,坐累謫真定推官。境內數有盜,昊教吏士習射,廣設方略,盜發輒獲。再坐累謫判開州。真定吏民伏闕請留,乃免。

遷四川僉事。昊長身驍捷,善騎射,知兵。巨寇方四、曹甫等方熾,洪鐘討之久無功。昊至,閱所部,笑曰:「將不知兵,其何以戰?」於是擇健卒千人分數隊,隊立長,教之。會甫將襲江津,昊從巡撫林俊剿賊,大敗之,俘斬及焚死者二千餘人。明年,方四陷江津,破綦江,薄重慶。昊夜出百騎,舉火擊賊,賊驚潰。乘之,斬獲多,遂合羅、回土兵博賊。賊陳左而伏兵其右,昊以正兵當左,身率百騎搗其伏。伏潰,趨左,左亦潰,四奔婺川,與甫相攻,眾遂散。四變姓名走,為他將所獲。昊再被獎,進副使,與總兵官楊宏擊敗甫。

甫降,而其黨廖麻子併其眾,連陷銅梁、榮昌。坐奪冠帶。時洪鐘已召還,巡撫高崇熙恇怯,復主撫。麻子等陽受約,崇熙遽罷諸軍,令副使張敏徙開縣臨江市民,空其地處之,許給復三年,為請於朝。昊力爭,謂臨江市蜀襟喉,上達重、敘,下連湖、湘,地土饒衍,奈何棄以資賊,自遺患。崇熙不從,昊乃益治兵觀變。其明年,賊果執敏叛。詔逮崇熙,而擢昊右僉都御史代之。賊圍中江,將趨成都。昊以五千騎與總督彭澤敗之。遊擊閻勳追斬麻子劍州,餘眾走,推其黨喻思俸為主。總兵官陳珣追至富村,賊偽降。因北渡江,襲殺都指揮姚震,轉入巴山故巢。尋出走大安鎮,珣不敢前。而陜西兵與賊戰潰,賊遂越寧羌犯略陽。珣軍鼓噪,賊夜走,度廣元,為官軍所遏,還趨通、巴招餘黨。諸將率稱病不擊賊,詔逮珣,且讓昊。昊乃與彭澤督諸軍獲思俸西鄉山中,復與澤平內江賊駱松祥,群盜悉靖。錄功,進副都御史。

十年,亦不剌寇松籓,番人磨讓六少等乘機亂,為之鄉導,西土大震。昊招土番為間,發兵掩擊之。千戶張倫等夜率熟番攻破賊,獲磨讓六少,亦不剌遁去。昊以松潘地險阻,番人往往邀劫饋運,乃督參將張傑等修築墻柵,自三舍堡至風洞關,凡五十里。賜敕褒之。

烏蒙、芒部二府壤接筠連、珙縣,圍亙千里,山箐深阻。諸蠻僰人子、羿子、仲家子、苗子、惈、佫等雜居其中。有僰人子普法惡者,通漢語,曉符錄。妄言彌勒出世,自稱蠻王,煽諸夷作亂。流民謝文禮、謝文義應之。都指揮杜琮戰敗,文義奪其胄。十二年,昊督指揮曹昱進討,法惡敗,走保青山寨。昊分據水口,絕其汲道,闕南方圍待之。賊乏水渴,突南圍,官軍遮擊。法惡中流矢死,諸蠻大奔。以功,再進右都御史,廕子錦衣世百戶。

昊有才氣,能應變,揮霍自喜,所向輒有功。然官川中久,狎其俗,銳意立功名,卒以是敗。先是,亦不剌既遁,昊移兵攻小東路番寨未下,茂州群蠻懼見侵,遂糾生苗圍城堡。參將芮錫等討之,兵敗,指揮龐昇等皆死。又嘗遣副總兵張傑、副使吳澧擊松潘南北二路番,不利,亡軍士三千餘人,匿不以聞。僰蠻平,不置戍守,遽班師。請改高縣為州,設長吏,增高、珙、筠連田租千八百石,令指揮魏武度田奪降人業給之軍民。而珙縣知縣步梁窺昊意,誘殺降人阿尚。杜琮以亡胄故,怨文義,潛使人購其頭。於是文義乘群蠻怨,嗾之,遂大訌。攻高、慶符二縣,破其城。琮率兵禦之,又敗,死傷七百人。自黎雅以西,天全六番皆相繼亂。南京給事中孫懋暨巡按御史盧雍、黎龍先後劾昊。十四年遂遣官逮昊。行至河南,疏稱疾篤,留於家。世宗即位,始就逮,尋削籍歸。楊一清、胡世寧薦之,為桂萼所駁而止。久之,卒。

贊曰:何鑒綰中樞,能任諸將滅賊,蓋其時楊廷和在政府,閣部同心,故克奏效云爾。馬中錫雅負時望,而軍旅非其所長,適用取敗。然觀劉宸阻降之言,亦可以觀朝事矣。陸完交結之罪浮於首功,得從八議,有佚罰焉。洪鐘、陳金威略甚著,而土兵之謠,聞之心惻,斯又統戎旃者所當留意也。


\end{pinyinscope}