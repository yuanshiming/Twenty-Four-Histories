\article{列傳第二百十一 外國四}

\begin{pinyinscope}
○琉球呂宋合貓里美洛居沙瑤吶嗶嘽雞籠婆羅麻葉甕古麻刺朗馮嘉施蘭文郎馬神

琉球居東南大海中,自古不通中國。元世祖遣官招諭之,不能達。洪武初,其國有三王,曰中山,曰山南,曰山北,皆以尚為姓,而中山最強。五年正月命行人楊載以即位建元詔告其國,其中山王察度遣弟泰期等隨載入朝,貢方物。帝喜,賜《大統曆》及文綺、紗羅有差。七年冬,泰期復來貢,并上皇太子箋。命刑部侍郎李浩齎賜文綺、陶鐵器,且以陶器七萬、鐵器千,就其國市馬。九年夏,泰期隨浩入貢,得馬四十匹。浩言其國不貴紈綺,惟貴磁器、鐵釜,自是賞賚多用諸物。明年遣使賀正旦,貢馬十六匹、硫黃千斤。又明年復貢。山南王承察度亦遣使朝貢,禮賜如中山。十五年春,中山來貢,遣內官送其使還國。明年與山南王並來貢,詔賜二王鍍金銀印。時二王與山北王爭雄,互相攻伐。命內史監丞梁民賜之敕,令罷兵息民,三王並奉命。山北王怕尼芝即遣使偕二王使朝貢。十八年又貢,賜山北王鍍金銀印如二王,而賜二王海舟各一。自是,三王屢遣使奉貢,中山王尤數。二十三年,中山來貢,其通事私攜乳香十斤、胡椒三百斤入都,為門者所獲,當入官。詔還之,仍賜以鈔。

二十五年夏,中山貢使以其王從子及寨官子偕來,請肄業國學。從之,賜衣巾靴襪並夏衣一襲。其冬,山南王亦遣從子及寨官子入國學,賜賚如之。自是,歲賜冬夏衣以為常。明年,中山兩入貢,又遣寨官子肄業國學。是時,國法嚴,中山生與山南生有非議詔書者。帝聞,置之死,而待其國如故。山北王怕尼芝已卒,其嗣王攀安知二十九年春遣使來貢。令山南生肄國學者歸省,其冬復來。中山亦遣寨官子二人及女官生姑、魯妹二人,先後來肄業,其感慕華風如此。中山又遣使請賜冠帶,命禮部繪圖,令自製。其王固以請,乃賜之,并賜其臣下冠服。又嘉其修職勤,賜閩中舟工三十六戶,以便貢使往來。及惠帝嗣位,遣官以登極詔諭其國,三王亦奉貢不絕。

成祖承大統,詔諭如前。永樂元年春,三王並來貢。山北王請賜冠帶,詔給賜如中山。命行人邊信、劉亢齎敕使三國,賜以絨錦、文綺、紗羅。明年二月,中山王世子武寧遣使告父喪,命禮部遣官諭祭,賻以布帛,遂命武寧襲位。四月,山南王從弟汪應祖亦遣使告承察度之喪,謂前王無子,傳位應祖,乞加朝命,且賜冠帶。帝並從之,遂遣官冊封。時山南使臣私齎白金詣處州市磁器,事發,當諭罪。帝曰:「遠方之人,知求利而已,安知禁令。」悉貰之。三年,山南遣寨官子入國學。明年,中山亦遣寨官子六人入國學,並獻奄豎數人。帝曰:「彼亦人子,無罪刑之,何忍?」命禮部還之。部臣言:「還之,慮阻歸化之心,請但賜敕,止其再進。」帝曰:「諭以空言,不若示以實事。今不遣還,彼欲獻媚,必將繼進。天地以生物為心,帝王乃可絕人類乎?」竟還之。五年四月,中山王世子思紹遣使告父喪,諭祭,賜賻冊封如前儀。

八年,山南遣官生三人入國學,賜巾服靴絳、衾褥帷帳,已復頻有所賜。一日,帝與群臣語及之。禮部尚書呂震曰:「昔唐太宗興庠序,新羅、百濟並遣子來學。爾時僅給廩餼,未若今日賜予之周也。」帝曰:「蠻夷子弟慕義而來,必衣食常充,然後嚮學。此我太祖美意,朕安得違之。」明年,中山遣國相子及寨官子入國學,因言:「右長史王茂輔翼有年,請擢為國相。左長史朱復,本江西饒州人,輔臣祖察度四十餘年不懈。今年踰八十,請令致仕還鄉。」從之,乃命復、茂並為國相,復兼左長史致仕,茂兼右長史任其國事。十一年,中山遣寨官子十三人入國學。時山南王應祖為其兄達勃期所弒,諸寨官討誅之,推應祖子他魯每為主,以十三年三月請封。命行人陳季若等封為山南王,賜誥命冠服及寶鈔萬五千錠。

琉球之分三王也,惟山北最弱,故其朝貢亦最稀。自永樂三年入貢後,至是年四月始入貢。其後,竟為二王所併,而中山益強,以其國富,一歲常再貢三貢。天朝雖厭其繁,不能卻也。其冬,貢使還,至福建,擅奪海舶,殺官軍,且毆傷中官,掠其衣物。事聞,戮其為首者,餘六十七人付其主自治。明年遣使謝罪,帝待之如初,其修貢益謹。二十二年春,中山王世子尚巴志來告父喪,諭祭賜賻如常儀。

仁宗嗣位,命行人方彝詔告其國。洪熙元年命中官齎敕封巴志為中山王。宣德元年,其王以冠服未給,遣使來請,命製皮弁服賜之。三年八月,帝以中山王朝貢彌謹,遣官齎敕往勞,賜羅錦諸物。

山南自四年兩貢,終帝世不復至,亦為中山所併矣。自是,惟中山一國朝貢不絕。

正統元年,其使者言:「初入閩時,止具貢物報聞。下人所齎海、螺殼,失於開報,悉為官司所沒入,致來往乏資,乞賜垂憫。」命給直如例。明年,貢使至浙江,典市舶者復請籍其所齎,帝曰:「番人以貿易為利,此二物取之何用,其悉還之,著為令。」使者奏:「本國陪臣冠服,皆國初所賜,歲久敝壞,乞再給。」又言:「小邦遵奉正朔,海道險遠,受曆之使,或半歲一歲始返,常懼後時。」帝曰:「冠服令本邦自製。《大統歷》,福建布政司給予之。」七年正月,中山世子尚忠來告父喪,命給事中餘忭、行人劉遜封忠為中山王。敕使之用給事中,自茲始也。忭等還,受其黃金、沉香、倭扇之贈,為偵事者所覺,並下吏,杖而釋之。十二年二月,世子尚思達來告父喪,命給事中陳傅、行人萬祥往封。

景泰二年,思達卒,無子,其叔父金福攝國事,遣使告喪。命給事中喬毅、行人童守宏封金福為王。五年二月,金福弟泰久奏:「長兄金福殂,次兄布里與兄子志魯爭立,兩傷俱殞,所賜印亦毀壞。國中臣民推臣權攝國事,乞再賜印鎮撫遠籓。」從之。明年四月命給事中嚴誠、行人劉儉封泰久為王。天順六年三月,世子尚德來告父喪,命給事中潘榮、行人蔡哲封為王。

成化五年,其貢使蔡璟言:「祖父本福建南安人,為琉球通事,傳至璟,擢長史。乞如制賜誥贈封其父母。」章下禮官,以無例而止。明年,福建按察司言:「貢使程鵬至福州,與指揮劉玉私通貨賄,並宜究治。」命治玉而宥鵬。七年三月,世子尚圓來告父喪,命給事中丘弘、行人韓文封為王。弘至山東病卒,命給事中官榮代之。十年,貢使至福建,殺懷安民夫婦二人,焚屋劫財,捕之不獲。明年復貢,禮官因請定令二年一貢,毋過百人,不得附攜私物,騷擾道途。帝從之,賜敕戒王。其使者請如祖制,比年一貢,不許。又明年,貢使至,會冊立東宮,請如朝鮮、安南,賜詔齎回。禮官議:琉球與日本、占城並居海外,例不頒詔,乃降敕以文錦、彩幣賜其王及妃。十三年,使臣來,復請比年一貢,不許。明年四月,王卒,世子尚真來告喪,乞嗣爵,復請比年一貢。禮官言,其國連章奏請,不過欲圖市易。近年所遣之使,多係閩中逋逃罪人,殺人縱火,奸狡百端,專貿中國之貨,以擅外蕃之利,所請不可許。乃命給事中董旻、行人張祥往封,而不從其請。十六年,使來,復引《祖訓》條章請比年一貢,帝賜敕戒約之。十八年,使者至,復以為言,賜敕如初。使者攜陪臣子五人來受學,命隸南京國子監。二十二年,貢使來,其王移咨禮部,請遣五人歸省,從之。

弘治元年七月,其貢使自浙江來。禮官言貢道向由福建,今既非正道,又非貢期,宜卻之,詔可。其使臣復以國王移禮部文來,上言舊歲知東宮冊妃,故遣使來賀,非敢違制。禮官乃請納之,而稍減傔從賜賚,以示裁抑之意。三年,使者至,言近歲貢使止許二十五人入都,物多人少,慮致疏虞。詔許增五人,其傔從在閩者,並增給二十人廩食,為一百七十人。時貢使所攜土物,與閩人互市者,為奸商抑勒,有司又從而侵削之。使者訴於朝,有詔禁止。十七年遣使補貢,謂小邦貢物常市之滿剌加,因遭風致失期,命宴賚如制。正德二年,使者來,請比年一貢。禮官言不可許,是時劉瑾亂政,特許之。五年遣官生蔡進等五人入南京國學。

嘉靖二年從禮官議,敕琉球二年一貢如舊制,不得過百五十人。五年,尚真卒,其世子尚清以六年來貢,因報訃,使者還至海,溺死。九年遣他使來貢,並請封。命福建守臣勘報。十一年,世子以國中臣民狀來上,乃命給事中陳侃、行人高澄持節往封。及還,卻其贈。十四年,貢使至,仍以所贈黃金四十兩進於朝,乃敕侃等受之。二十九年來貢,攜陪臣子五人入國學。

三十六年,貢使來,告王尚清之喪。先是,倭寇自浙江敗還,抵琉球境。世子尚元遣兵邀擊,大殲之,獲中國被掠者六人,至是送還。帝嘉其忠順,賜賚有加,即命給事中郭汝霖、行人李際春封尚元為王。至福建,阻風未行。三十九年,其貢使亦至福建,稱受世子命,以海中風濤叵測,倭寇又出沒無時,恐天使有他慮,請如正德中封占城故事,遣人代進表文方物,而身偕本國長史齎回封冊,不煩天使遠臨。巡按御史樊獻科以聞,禮官言:「遣使冊封,祖制也。今使者欲遙受冊命,是委君貺於草莽,不可一。使者本奉表朝貢,乃求遣官代進,是棄世子專遣之命,不可二。昔正德中,占城王為安南所侵,竄居他所,故使者齎回敕命,出一時權宜。今援失國之事,以儗其君,不可三。梯航通道,柔服之常。彼所藉口者倭寇之警、風濤之險爾,不知琛賨之輸納、使臣之往來,果何由而得無患乎?不可四。曩占城雖領封,其王猶懇請遣使。今使者非世子面命,又無印信文移。若輕信其言,倘世子以遣使為至榮,遙拜為非禮,不肯受封,復上書請使,將誰執其咎?不可五。乞命福建守臣仍以前詔從事。至未受封而先謝恩,亦非故事。宜止聽其入貢,其謝恩表文,俟世子受封後遣使上進,庶中國之大體以全。」帝如其言。四十一年夏,遣使入貢謝恩。明年及四十四年並入貢。隆慶中,凡三貢,皆送還中國飄流人口。天子嘉其忠誠,賜敕獎勵,加賚銀幣。

萬曆元年冬,其國世子尚永遣使告父喪,請襲爵。章下禮部,行福建守臣核奏。明年遣使賀登極。三年入貢。四年春,再貢。七月命戶科給事中蕭崇業、行人謝傑齎敕及皮弁冠服、玉珪,封尚永為中山王。明年冬,崇業等未至,世子復遣使入貢,其後,修貢如常儀。八年冬,遣陪臣子三人入南京國學。十九年遣使來貢,而尚永隨卒。禮官以日本方侵噬鄰境,琉球不可無王,乞令世子速請襲封,用資鎮壓。從之。

二十三年,世子尚寧遣人請襲。福建巡撫許孚遠以倭氛未息,據先臣鄭曉領封之議,請遣官一員齎敕至福建,聽其陪臣面領歸國,或遣習海武臣一人,偕陪臣同往。禮官范謙議如其言,且請待世子表至乃許。二十八年,世子以表至,其陪臣請如祖制遣官。禮官餘繼登言:「累朝冊封琉球,伐木造舟,動經數歲。使者蹈風濤之險,小國苦供億之煩。宜一如前議從事。」帝可之,命今後冊封,止遣廉勇武臣一人偕請封陪臣前往,其祭前王,封新王,禮儀一如舊章,仍命俟彼國大臣結狀至乃行。明年秋,貢使以狀至,仍請遣文臣。乃命給事中洪瞻祖、行人王士禎往,且命待海寇息警,乃渡海行事。已而瞻祖以憂去,改命給事中夏子陽,以三十一年二月抵福建。按臣方元彥復以海上多事,警報頻仍,會巡撫徐學聚疏請仍遣武臣。子陽、士禎則以屬國言不可爽,使臣義當有終,乞堅成命慰遠人。章俱未報,禮部侍郎李廷機言:「宜行領封初旨,并武臣不必遣。」於是御史錢桓、給事中蕭近高交章爭其不可,謂:「此事當在欽命未定之前,不當在冊使既遣之後,宜敕所司速成海艘,勿誤今歲渡海之期。俟竣事復命,然後定為畫一之規,先之以文告,令其領封海上,永為遵守。」帝納之。三十三年七月,乃命子陽等速渡海竣事。

當是時,日本方強,有吞滅之意。琉球外禦強鄰,內修貢不絕。四十年,日本果以勁兵三千入其國,擄其王,遷其宗器,大掠而去。浙江總兵官楊宗業以聞,乞嚴飭海上兵備,從之。已而其王釋歸,復遣使修貢,然其國殘破已甚,禮官乃定十年一貢之例。明年修貢如故。又明年再貢,福建守臣遵朝命卻還之,其使者怏怏而去。四十四年,日本有取雞籠山之謀,其地名臺灣,密邇福建,尚寧遣使以聞,詔海上警備。

天啟三年,尚寧已卒,其世子尚豐遣使請貢請封。禮官言:「舊制,琉球二年一貢,後為倭寇所破,改期十年。今其國休養未久,暫擬五年一貢,俟新王冊封更議。」從之。五年遣使入貢請封。六年再貢。是時中國多事,而科臣應使者亦憚行,故封典久稽。

崇禎二年,貢使又至請封,命遣官如故事。禮官何如寵復以履險糜費,請令陪臣領封。帝不從,乃命戶科給事中杜三策、行人楊掄往,成禮而還。四年秋,遣使賀東宮冊立。自是,迄崇禎末,並修貢如儀。後兩京繼沒,唐王立於福建,猶遣使奉貢。其虔事天朝,為外籓最云。

呂宋居南海中,去漳州甚近。洪武五年正月遣使偕瑣里諸國來貢。永樂三年十月遣官齎詔,撫諭其國。八年與馮嘉施蘭入貢,自後久不至。萬曆四年,官軍追海寇林道乾至其國,國人助討有功,復朝貢。時佛郎機強,與呂宋互市,久之見其國弱可取,乃奉厚賄遺王,乞地如牛皮大,建屋以居。王不虞其詐而許之,其人乃裂牛皮,聯屬至數千丈,圍呂宋地,乞如約。王大駭,然業已許諾,無可奈何,遂聽之,而稍徵其稅如國法。其人既得地,即營室築城,列火器,設守禦具,為窺伺計。已,竟乘其無備,襲殺其王,逐其人民,而據其國,名仍呂宋,實佛郎機也。先是,閩人以其地近且饒富,商販者至數萬人,往往久居不返,至長子孫。佛郎機既奪其國,其王遣一酋來鎮,慮華人為變,多逐之歸,留者悉被其侵辱。

二十一年八月,酋郎雷敝裏系朥侵美洛居,役華人二百五十助戰。有潘和五者為其哨官。蠻人日酣臥,而令華人操舟,稍怠,輒鞭撻,有至死者。和五曰:「叛死,箠死,等死耳,否亦且戰死,曷若刺殺此酋以救死。勝則揚帆歸,不勝而見縛,死未晚也。」眾然之,乃夜刺殺其酋,持酋首大呼。諸蠻驚起,不知所為,悉被刃,或落水死。和五等盡收其金寶、甲仗,駕舟以歸。失路之安南,為其國人所掠,惟郭惟太等三十二人附他舟獲返。時酋子郎雷貓吝駐朔霧,聞之,率眾馳至,遣僧陳父冤,乞還其戰艦、金寶,戮仇人以償父命。巡撫許孚遠聞於朝,檄兩廣督撫以禮遣僧,置惟太於理,和五竟留安南不敢返。

初,酋之被戮也,其部下居呂宋者,盡逐華人於城外,毀其廬。及貓吝歸,令城外築室以居。會有傳日本來寇者,貓吝懼交通為患,復議驅逐。而孚遠適遣人招還,蠻乃給行糧遣之。然華商嗜利,趨死不顧,久之復成聚。

其時礦稅使者四出,奸宄蜂起言利,有閻應龍、張嶷者,言呂宋機易山素產金銀,採之,歲可得金十萬兩、銀三十萬兩,以三十年七月詣闕奏聞,帝即納之。命下,舉朝駭異。都御史溫純疏言:「近中外諸臣爭言礦稅之害,天聽彌高。今廣東李鳳至汙辱婦女六十六人,私運財賄至三十巨舟、三百大扛,勢必見戮於積怒之眾。何如及今撤之,猶不失威福操縱之柄。緬酋以寶井故,提兵十萬將犯內地,西南之蠻,岌岌可憂。而閩中奸徒又以機易山事見告。此其妄言,真如戲劇,不意皇上之聰明而誤聽之。臣等驚魂搖曳,寢食不寧。異時變興禍起,費國家之財不知幾百萬,倘或剪滅不早,其患又不止費財矣。

臣聞海澄市舶高寀已歲徵三萬金,決不遺餘力而讓利。即機易越在海外,亦決無遍地金銀,任人採取之理,安所得金十萬、銀三十萬,以實其言。不過假借朝命,闌出禁物,勾引諸番,以逞不軌之謀,豈止煩擾公私,貽害海澄一邑而已哉。

昔年倭患,正緣奸民下海,私通大姓,設計勒價,致倭賊憤恨,稱兵犯順。今以朝命行之,害當彌大。及乎兵連禍結,諸奸且效汪直、曾一本輩故智,負海稱王,擁兵列寨,近可以規重利,遠不失為尉佗。於諸亡命之計得矣,如國家大患何!乞急置於理,用消禍本。」

言官金忠士、曹於汴、朱吾弼等亦連章力爭,皆不聽。

事下福建守臣,持不欲行,而迫於朝命,乃遣海澄丞王時和、百戶乾一成偕嶷往勘。呂宋人聞之大駭。華人流寓者謂之曰:「天朝無他意,特是奸徒橫生事端。今遣使者按驗,俾奸徒自窮,便於還報耳。」其酋意稍解,命諸僧散花道旁,若敬朝使,而盛陳兵衛迓之。時和等入,酋為置宴,問曰:「天朝欲遣人開山。山各有主,安得開?譬中華有山,可容我國開耶?」且言:「樹生金豆,是何樹所生?」時和不能對,數視嶷,嶷曰:「此地皆金,何必問豆所自?」上下皆大笑,留嶷,欲殺之。諸華人共解,乃獲釋歸。時和還任,即病悸死。守臣以聞,請治嶷妄言罪。事已止矣,而呂宋人終自疑,謂天朝將襲取其國,諸流寓者為內應,潛謀殺之。

明年,聲言發兵侵旁國,厚價市鐵器。華人貪利盡鬻之,於是家無寸鐵。酋乃下令錄華人姓名,分三百人為一院,入即殲之。事稍露,華人群走菜園。酋發兵攻,眾無兵仗,死無算,奔大崙山。蠻人復來攻,眾殊死鬥,蠻兵少挫。酋旋悔,遣使議和。眾疑其偽,撲殺之。酋大怒,斂眾入城,設伏城旁。眾飢甚,悉下山攻城。伏發,眾大敗,先後死者二萬五千人。酋尋出令,諸所掠華人貲,悉封識貯庫。移書閩中守臣,言華人將謀亂,不得已先之,請令死者家屬往取其孥與帑。巡撫徐學聚等亟告變於朝,帝驚悼,下法司議奸徒罪。三十二年十二月議上,帝曰:「嶷等欺誑朝廷,生釁海外,致二萬商民盡膏鋒刃,損威辱國,死有餘辜,即梟首傳示海上。呂宋酋擅殺商民,撫按官議罪以聞。」學聚等乃移檄呂宋,數以擅殺罪,令送死者妻子歸,竟不能討也。其後,華人復稍稍往,而蠻人利中國互市,亦不拒,久之復成聚。

時佛郎機已併滿剌加,益以呂宋,勢愈強,橫行海外,遂據廣東香山澳,築城以居,與民互市,而患復中於粵矣。

合貓里,海中小國也。土瘠多山,山外大海,饒魚蟲,人知耕稼。永樂三年九月遣使附爪哇使臣朝貢。其國又名貓里務,近呂宋,商舶往來,漸成富壤。華人入其國,不敢欺陵,市法最平,故華人為之語曰:「若要富,須往貓里務。」有網巾礁老者,最兇悍,海上行劫,舟若飄風,遇之無免者。然特惡商舶不至其地,偶有至者,待之甚善。貓里務後遭寇掠,人多死傷,地亦貧困。商人慮為礁老所劫,鮮有赴者。

美洛居,俗訛為米六合,居東海中,頗稱饒富。酋出,威儀甚備,所部合掌伏道旁。男子削髮,,女椎結。地有香山,雨後香墮,沿流滿地,居民拾取不竭。其酋委積充棟,以待商舶之售。東洋不產丁香,獨此地有之,可以辟邪,故華人多市易。

萬歷時,佛郎機來攻,其酋戰敗請降,乃宥令復位,歲以丁香充貢,不設戍兵而去。已,紅毛番橫海上,知佛郎機兵已退,乘虛直抵城下,執其酋,語之曰:「若善事我,我為若主,殊勝佛郎機也。」酋不得已聽命,復位如故。佛郎機酋聞之大怒,率兵來攻,道為華人所殺,語具《呂宋傳》。

時紅毛番雖據美洛居,率一二歲率眾返國,既返復來。佛郎機酋子既襲位,欲竟父志,大舉兵來襲,值紅毛番已去,遂破美洛居,殺其酋,立己所親信主之。無何,紅毛番至,又破其城,逐佛郎機所立酋,而立美洛居故王之子。自是,歲構兵,人不堪命。華人流寓者,遊說兩國,令各罷兵,分國中萬老高山為界,山以北屬紅毛番,南屬佛郎機,始稍休息,而美洛居竟為兩國所分。

沙瑤,與吶嗶嘽連壞。吶嗶嘽在海畔,沙瑤稍紆入山隈,皆與呂宋近。男女蓄髮椎結,男子用履,婦女跣足。以板為城,豎木覆茅為室。崇釋教,多建禮拜寺。男女之禁甚嚴,夫行在前,其婦與人嘲笑,夫即刃其婦,所嘲笑之人不敢逃,任其刺割。盜不問大小,輒論死。孕婦將產,以水灌之,且以水滌其子,置水中,生而與水習矣。物產甚薄,華人商其地,所攜僅磁器、鍋釜之類,重者至布而止。後佛郎機據呂宋,多侵奪鄰境,惟二國號令不能及。

雞籠山在彭湖嶼東北,故名北港,又名東番,去泉州甚邇。地多深山大澤,聚落星散。無君長,有十五社,社多者千人,少或五六百人。無徭賦,以子女多者為雄,聽其號令。雖居海中,酷畏海,不善操舟,老死不與領國往來。

永樂時,鄭和遍歷東西洋,靡不獻琛恐後,獨東番遠避不至。和惡之,家貽一銅鈴,俾挂諸項,蓋擬之狗國也。其後,人反寶之,富者至掇數枚,曰:「此祖宗所遺。」俗尚勇,暇即習走,日可數百里,不讓奔馬。足皮厚數分,履荊棘如平地。男女椎結,裸逐無所避。女或結草裙蔽體,遇長老則背身而立,俟過乃行。男子穿耳。女子年十五,斷脣旁齒以為飾,手足皆刺文,眾社畢賀,費不貲。貧者不任受賀,則不敢刺。四序,以草青為歲首。土宜五穀,而不善水田。穀種落地,則止殺,謂行好事,助天公,乞飯食。既收獲,即標竹竿於道,謂之插青,此時逢外人便殺矣。村落相仇,刻期而後戰,勇者數人前跳,被殺則立散。其勝者,眾賀之,曰:「壯士能殺人也。」其負者,家眾亦賀之,曰:「壯士不畏死也。次日,即和好如初。地多竹,大至數拱,長十丈,以竹構屋,覆之以茅,廣且長,聚族而居。無曆日、文字,有大事集眾議之。善用鏢鎗,竹柄鐵鏃,銛甚,試鹿鹿斃,試虎虎亦斃。性既畏海,捕魚則於溪澗。冬月聚眾捕鹿,鏢發輒中,積如丘山。獨不食雞雉,但取其毛以為飾。中多大溪,流入海,水澹,故其外名淡水洋。

嘉靖末,倭寇擾閩,大將戚繼光敗之。倭遁居於此,其黨林道乾從之。已,道乾懼為倭所併,又懼官軍追擊,揚帆直抵浡泥,攘其邊地以居,號道乾港。而雞籠遭倭焚掠,國遂殘破。初悉居海濱,既遭倭難,稍稍避居山後。忽中國漁舟從魍港飄至,遂往來通販,以為常。至萬曆末,紅毛番泊舟於此,因事耕鑿,設闤闠,稱臺灣焉。

崇禎八年,給事中何楷陳靖海之策,言:「自袁進、李忠、楊祿、楊策、鄭芝龍、李魁奇、鐘斌、劉香相繼為亂,海上歲無寧息。今欲靖寇氛,非墟其窟不可。其窟維何?臺灣是也。臺灣在彭湖島外,距漳、泉止兩日夜程,地廣而腴。初,貧民時至其地,規魚鹽之利,後見兵威不及,往往聚而為盜。近則紅毛築城其中,與奸民互市,屹然一大部落。墟之之計,非可干戈從事,必嚴通海之禁,俾紅毛無從謀利,奸民無從得食,出兵四犯,我師乘其虛而擊之,可大得志。紅毛舍此而去,然後海氛可靖也。」時不能用。

其地,北自雞籠,南至浪嶠,可一千餘里。東自多羅滿,西至王城,可九百餘里。水道,順風,自雞籠淡水至福州港口。五更可達。自臺灣港至彭湖嶼,四更可達。自彭湖至金門,七更可達。東北至日本,七十更可達。南至呂宋,六十更可達。蓋海道不可以里計,舟人分一晝夜為十更,故以更計道里云。

婆羅,又名文萊,東洋盡處,西洋所自起也。唐時有婆羅國,高宗時常入貢。永樂三年十月遣使者齎璽書、綵幣撫諭其王。四年十二月,其國東、西二王並遣使奉表朝貢。明年又貢。

其地負山面海,崇釋教,惡殺喜施。禁食豕肉,犯者罪死。王薙髮,裹金繡巾,佩雙劍,出入徒步,從者二百餘人。有禮拜寺,每祭用犧。厥貢玳瑁、瑪瑙、硨磲、珠、白焦布、花焦布、降真香、黃蠟、黑小廝。

萬歷時,為王者閩人也。或言鄭和使婆羅,有閩人從之,因留居其地,其後人竟據其國而王之。邸旁有中國碑。王有金印一,篆文,上作獸形,言永樂朝所賜。民間嫁娶,必請此印印背上,以為榮。後佛郎機橫,舉兵來擊。王率國人走入山谷中,放藥水,流出,毒殺其人無算,王得返國。佛郎機遂犯呂宋。

麻葉甕,在西南海中。永樂三年十月遣使齎璽書賜物,招諭其國,迄不朝貢。自占城靈山放舟,順風十晝夜至交欄山,其西南即麻葉甕。山峻地平,田膏腴,收獲倍他國。煮海為鹽,釀蔗為酒。男女椎結,衣長衫,圍之以布。俗尚節義,婦喪夫,剺面剃髮,絕粒七日,與屍同寢,多死。七日不死,則親戚勸以飲食,終身不再嫁。或於焚屍日,亦赴火自焚。產玳瑁、木棉、黃蠟、檳榔、花布之屬。

交欄山甚高廣,饒竹木。元史弼、高興伐爪哇,遭風至此山下,舟多壞,乃登山伐木重造,遂破爪哇。其病卒百餘,留養不歸,後益蕃衍,故其地多華人。

又有葛卜及速兒米囊二國,亦永樂三年遣使持璽書賜物招諭,竟不至。

古麻剌朗,東南海中小國也。永樂十五年九月遣中官張謙齎敕撫諭其王乾剌義亦奔敦,賜之絨錦、糸寧絲、紗羅。十八年八月,王率妻子、陪臣隨謙來朝,貢方物,禮之如蘇祿國王。王言:「臣愚無知,雖為國人所推,然未受朝命,幸賜封誥,仍其國號。」從之,乃賜以印誥、冠帶、儀仗、鞍馬及文綺、金織襲衣,妃以下並有賜。明年正月辭還,復賜金銀錢、文綺、紗羅、彩帛、金織襲衣、麒麟衣,妃以下賜有差。王還至福建,遘疾卒。遣禮部主事楊善諭祭,謚曰康靖,有司治墳,葬以王禮。命其子剌苾嗣為王,率眾歸,賜鈔幣。

馮嘉施蘭,亦東洋中小國。永樂四年八月,其酋嘉馬銀等來朝,貢方物,賜鈔幣有差。六年四月,其酋玳瑁、里欲二人,各率其屬朝貢,賜二人鈔各百錠、文綺六表裏,其從者亦有賜。八年復來貢。

文郎馬神,以木為城,其半倚山。酋蓄繡女數百人。出乘象,則繡女執衣履、刀劍及檳榔盤以從。或泛舟,則酋趺坐床上,繡女列坐其下,與相向,或用以刺舟,威儀甚都。民多縛木水上,築室以居,如三佛齊。男女用五色布纏頭,腹背多袒,或著小袖衣,蒙頭而入,下體圍以幔。初用蕉葉為食器,後與華人市,漸用磁器。尤好磁甕,畫龍其外,死則貯甕中以葬。其俗惡淫,奸者論死。華人與女通,輒削其髮,以女配之,永不聽歸。女苦髮短,問華人何以致長,紿之曰:「我用華水沐之,故長耳。」其女信之,競市船中水以沐。華人故靳之,以為笑端。女或悅華人,持香蕉、甘蔗、茉莉相贈遺,多與之調笑。然憚其法嚴,無敢私通者。

其深山中有村名烏籠里憚,其人盡生尾,見人輒掩面走避。然地饒沙金,商人持貨往市者,擊小銅鼓為號,置貨地上,即引退丈許。其人乃前視,當意者,置金於旁。主者遙語欲售,則持貨去,否則懷金以歸,不交言也。所產有犀牛、孔雀、鸚鵡、沙金、鶴頂、降香、蠟、藤席、惸藤、蓽撥、血竭、肉荳蔻、麞皮諸物。

鄰境有買哇柔者,性兇狠,每夜半盜斬人頭以去,裝之以金。故商人畏之,夜必嚴更以待。

始,文郎馬神酋有賢德,待商人以恩信。子三十一人,恐擾商舶,不令外出。其妻乃買哇柔酋長之妹,生子襲父位,聽其母族之言,務為欺詐,多負商人價直,自是赴者亦稀。


\end{pinyinscope}