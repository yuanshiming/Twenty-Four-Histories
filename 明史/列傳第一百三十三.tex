\article{列傳第一百三十三}

\begin{pinyinscope}
周起元繆昌期周順昌子茂蘭硃祖文顏佩韋等周宗建蔣英黃尊素李應昇萬燝丁乾學等

周起元,字仲先,海澄人。萬歷二十八年鄉試第一,明年成進士。歷知浮梁、南昌,以廉惠稱。

行取入都,注湖廣道御史。方候命,值京察。御史劉國縉疑鄭繼芳假書出起元及李邦華、李炳恭、徐縉芳、徐良彥手,遂目為「五鬼」,繼芳且入之疏中。起元憤,上章自明。居二年,御史命始下。

會太僕少卿徐兆魁以攻東林為御史錢春所劾,起元亦疏劾之。奸人劉世學者,誠意伯劉藎臣從祖也,疏詆顧憲成,起元憤,力斥其謬。藎臣遂訐起元,益詆憲成。起元再疏極論,其同官翟鳳翀、餘懋衡、徐良彥、魏雲中、李邦華、王時熙、潘之祥亦交章論列。且下令捕世學,世學遂遁去。吏部侍郎方從哲由中旨起官,起元力言不可,并刺給事中亓詩教、周永春,吏部侍郎李養正、郭士望等。吏部尚書趙煥出雲中、時熙於外,起元劾其背旨擅權,坐停俸。煥去,鄭繼之代,又出之祥及張鍵。起元亦抗疏糾駁,因言張光房等五人不當擯之部曹。與黨人牴牾,忌者益眾。

尋巡按陜西,風采甚著。卒以東林故,出為廣西參議,分守右江道。柳州大饑,群盜蜂起,起元單騎招劇賊,而振恤饑民甚至。移四川副使,未上。會遼陽破,廷議通州重地,宜設監司,乃命起元以參政蒞之。

天啟三年入為太僕少卿。旋擢右僉都御史,巡撫蘇、松十府。公廉愛民,絲粟無所取。遇大水,百方拯恤,民忘其困。織造中官李實素貪橫,妄增定額,恣誅求。蘇州同知楊姜署府事,實惡其不屈,摭他事劾之。起元至,即為姜辨冤,且上去蠹七事,語多侵實。實欲姜行屬吏禮,再疏誣逮之。起元再疏雪姜,更切直。魏忠賢庇實,取嚴旨責起元,令速上姜貪劣狀。起元益頌姜廉謹,詆實誣毀,因引罪乞罷。忠賢大怒,矯旨斥姜為民。起元復劾實貪恣不法數事,而為姜求寬。實以此斂威,而忠賢遂銜起元不置。分守參政朱童蒙者,先為兵科都給事中,以攻鄒元標講學外遷,失志狂暴,每行道輒鞭撲數十人,血肉狼籍。起元欲糾之,童蒙遂稱病去,起元乃列其貪虐狀以聞。忠賢遂矯旨削起元籍,擢童蒙京卿。

六年二月,忠賢欲殺高攀龍、周順昌、繆昌期、黃尊素、李應昇、周宗建六人,取實空印疏,令其黨李永貞、李朝欽誣起元為巡撫時乾沒帑金十餘萬,日與攀龍輩往來講學,因行居間。矯旨逮起元,至則順昌等已斃獄中。許顯純酷榜掠,竟如實疏,懸贓十萬。罄貲不足,親故多破其家。九月斃之獄中,吳士民及其鄉人無不垂涕者。

莊烈帝嗣位,贈兵部右侍郎,官一子。福王時,追謚忠惠。

繆昌期,字當時,江陰人。為諸生有盛名,舉萬曆四十一年進士,改庶吉士,年五十有二矣。有同年生忌之,揚言為于玉立所薦,自是有東林之目。

張差梃擊事,劉廷元倡言瘋癲,劉光復和之,疏詆發訐者,謂不當詫之為奇貨,居之為元功。昌期憤,語朝士曰:「奸徒狙擊青宮,此何等事,乃以『瘋癲』二字庇天下亂臣賊子,以『奇貨元功』四字沒天下忠臣義士哉!」廷元輩聞其語,深疾之。給事中劉文炳劾大學士吳道南,遂陰詆昌期。時方授檢討,文炳再疏顯攻,昌期即移疾去。既而京察,廷元輩復思中之,學士劉一燝力持乃免。

天啟元年還朝。一燝以次輔當國。其冬,首輔葉向高至。小人間一燝於向高,謂欲沮其來,向高不悅。會給事中孫傑承魏忠賢指,劾一燝及周嘉謨,忠賢遽傳旨允放。昌期急詣向高,力言二人顧命重臣,不可輕逐,內傳不可奉。向高怫然曰:「上所傳,何敢不奉?」昌期曰:「公,三朝老臣。始至之日,以去就力爭,必可得也。若一傳而放兩大臣,異日天子手滑,不復可止矣。」向高默然。昌期因備言一燝質直無他腸,向高意少解。會顧大章亦為向高言之,一燝乃得善去。兩人故向高門下士也。

昌期尋遷左贊善,進諭德。楊漣劾忠賢疏上,昌期適過向高。向高曰:「楊君此疏太率易。其人於上前時有匡正。鳥飛入宮,上乘梯手攫之,其人挽衣不得上。有小璫賜緋者,叱曰:『此非汝分,雖賜不得衣也。』其強直如此。是疏行,安得此小心謹慎之人在上左右?」昌期愕然曰:「誰為此言以誤公?可斬也。」向高色變,昌期徐起去。語聞於漣,漣怒。向高亦內慚,密具揭,請帝允忠賢辭,忠賢大慍。會有言漣疏乃昌期代草者,忠賢遂深怒不可解。及向高去,韓爌秉政,忠賢逐趙南星、高攀龍、魏大中及漣、光斗,爌皆具揭懇留。忠賢及其黨謂昌期實左右之。而昌期於諸人去國,率送之郊外,執手太息,由是忠賢益恨。昌期知勢不可留,具疏乞假,遂落職閒住。

五年春,以汪文言獄詞連及,削職提問。忠賢恨不置。明年二月復於他疏責昌期已削籍猶冠蓋延賓,令緹騎逮問。踰月,復入之李實疏中,下詔獄。昌期慷慨對簿,詞氣不撓,竟坐贓三千,五毒備至。四月晦,斃於獄。

莊烈帝即位,贈詹事兼侍讀學士,錄其一子,詔并予謚。而是時,姚希孟以詞臣持物論,雅不善左光斗、周宗建,力尼之,遂并昌期及周起元、李應升、黃尊素、周朝瑞、袁化中、顧大章,皆不獲謚。福王時,始謚文貞。

周順昌,字景文,吳縣人。萬曆四十一年進士。授福州推官。捕治稅監高寀爪牙,不少貸。寀激民變,劫辱巡撫袁一驥,質其二子,并質副使呂純如。或議以順昌代,順昌不可,純如以此銜順昌。擢吏部稽勛主事。天啟中,歷文選員外郎,署選事。力杜請寄,抑僥倖,清操皭然。乞假歸。

順昌為人剛方貞介,疾惡如仇。巡撫周起元忤魏忠賢削籍,順昌為文送之,指斥無所諱。魏大中被逮,道吳門,順昌出餞,與同臥起者三日,許以女聘大中孫。旂尉屢趣行,順昌瞋目曰:「若不知世間有不畏死男子耶?歸語忠賢,我故吏部郎周順昌也。」因戟手呼忠賢名,罵不絕口。旂尉歸,以告忠賢。御史倪文煥者,忠賢義子也,誣劾同官夏之令,致之死。順昌嘗語人,他日倪御史當償夏御史命。文煥大恚,遂承忠賢指,劾順昌與罪人婚,且誣以贓賄,忠賢即矯旨削奪。先所忤副使呂純如,順昌同郡人,以京卿家居,挾前恨,數譖於織造中官李實及巡撫毛一鷺。已,實追論周起元,遂誣順昌請囑,有所乾沒,與起元等並逮。

順昌好為德於鄉,有冤抑及郡中大利害,輒為所司陳說,以故士民德順昌甚。及聞逮者至,眾咸憤怒,號冤者塞道。至開讀日,不期而集者數萬人,咸執香為周吏部乞命。諸生文震亨、楊廷樞、王節、劉羽翰等前謁一鷺及巡按御史徐吉,請以民情上聞。旂尉厲聲罵曰:「東廠逮人,鼠輩敢爾!」大呼:「囚安在?」手擲鋃鐺於地,聲瑯然。眾益憤,曰:「始吾以為天子命,乃東廠耶!」蜂擁大呼,勢如山崩。旂尉東西竄,眾縱橫毆擊,斃一人,餘負重傷,踰垣走。一鷺、吉不能語。知府寇慎、知縣陳文瑞素得民,曲為解諭,眾始散。順昌乃自詣吏。又三日北行,一鷺飛章告變,東廠刺事者言吳人盡反,謀斷水道,劫漕舟,忠賢大懼。已而一鷺言縛得倡亂者顏佩韋、馬傑、沈揚、楊念如、周文元等,亂已定,忠賢乃安。然自是緹騎不出國門矣。

順昌至京師,下詔獄。許顯純鍛煉,坐贓三千,五日一酷掠,每掠治,必大罵忠賢。顯純椎落其齒,自起問曰:「復能罵魏上公否?」順昌噀血唾其面,罵益厲。遂於夜中潛斃之。時六年六月十有七日也。

明年,莊烈帝即位,文煥伏誅,實下吏,一鷺、吉坐建忠賢祠,純如坐頌璫,並麗逆案。順昌贈太常卿,官其一子。給事中瞿式耜訟諸臣冤,稱順昌及楊漣、魏大中清忠尤著,詔謚忠介。

長子茂蘭,字子佩,刺血書疏,詣闕愬冤,詔以所贈官推及其祖父。茂蘭更上疏,請給三世誥命,建祠賜額。帝悉報可,且命先後慘死諸臣,咸視此例。茂蘭好學砥行,不就蔭敘。國變後,隱居不出,以壽終。

諸生朱祖文者,都督先之孫。當順昌被逮,間行詣都,為納饘粥、湯藥。及徵贓令急,奔走稱貸諸公間。順昌櫬歸,祖文哀慟發病死。

佩韋等皆市人,文元則順昌輿隸也,論大辟。臨刑,五人延頸就刃,語寇慎曰:「公好官,知我等好義,非亂也。」監司張孝流涕而斬之。吳人感其義,合葬之虎丘傍,題曰:「五人之墓」。其地即一鷺所建忠賢普惠祠址也。

周宗建,字季侯,吳江人,尚書用曾孫也。萬曆四十一年進士。除武康知縣,調繁仁和,有異政,入為御史。

天啟元年,為顧存仁、王世貞、陶望齡、顧憲成請謚,追論萬曆朝小人,歷數錢夢皋、康丕揚、亓詩教、趙興邦亂政罪,并詆李三才、王圖。時遼事方棘,上疏責備輔臣。無何,沈陽破,宗建責當事大臣益急,因請破格用人,召還熊廷弼。已,論兵部尚書崔景榮不當信奸人劉保,輔臣劉一燝不當抑言路,因刺右通政林材、光祿卿李本固。材、本固移疾去。魏大中劾王德完庇楊鎬、李如楨,宗建為德完力攻大中,其持論數與東林左。會是歲冬,奉聖夫人客氏既出宮復入,宗建首抗疏極諫,中言:「天子成言,有同兒戲。法宮禁地,僅類民家。聖朝舉動有乖,內外防閑盡廢。此輩一叨隆恩,便思踰分,狎溺無紀,漸成驕恣,釁孽日萌,後患難杜。王聖、朱娥、陸令萱之覆轍,可為殷鑒。」忤旨,詰責。清議由此重之。

明年,廣寧失。廷臣多庇王化貞,欲甚熊廷弼罪。宗建不平,為剖兩人罪案,頗右廷弼,諸庇化貞者乃深疾宗建。京師久旱,五月雨雹。宗建謂陰盛陽衰之徵,歷陳四事:一專譏大學士沈紘;一請寬建言廢黜諸臣;一言廷弼已有定案,不當因此羅織朝士,陰刺兵部尚書張鶴鳴、給事中郭鞏;一則專攻魏進忠,略言:「近日政事,外廷嘖嘖,咸謂奧窔之中,莫可測識,諭旨之下,有物憑焉。如魏進忠者,目不識一丁,而陛下假之嚬笑,日與相親。一切用人行政,墮於其說,東西易向而不知,邪正顛倒而不覺。況內廷之借端,與外廷之投合,互相扶同。離間之漸將起於蠅營,讒構之釁必生於長舌。其為隱禍,可勝言哉!」進忠者,魏忠賢故名也。時方結客氏為對食,廷臣多陰附之,其勢漸熾,見宗建疏,銜次骨,未發也。鄒元標建首善書院,宗建實司其事。元標罷,宗建乞與俱罷,不從。巡視光祿,與給事中羅尚忠力剔奸弊,節省為多。尋請核上供器物,中官怒,取旨詰責。宗建等再疏力持,中人滋不悅。

給事中郭鞏者,先以劾廷弼被謫。廷弼敗,復官,遂深結進忠。知進忠最惡宗建,乃疏詆廷弼,因詆朝廷之薦廷弼者,而宗建與焉。其鋒銳甚,南京御史塗世業和之,詆宗建誤廷弼,且誤封疆。宗建憤,疏駁世業,語侵鞏,抉其結納忠賢事。鞏亦憤,上疏數千言,詆宗建益力,并及劉一燝、鄒元標、周嘉謨、楊漣、周朝瑞、毛士龍、方震孺、江秉謙、熊德陽輩數十人,悉指為廷弼逆黨。宗建益憤,抗疏力駁其謬,且曰:「李維翰、楊鎬、袁應泰、王化貞,皆壞封疆之人也;亓詩教力主催戰,趙興邦賄賣邊臣,皆誤封疆之人也;其他薦維翰、薦鎬、薦應泰、化貞者,亦誤封疆之人也。鞏胡不一擊之,而獨苛求廷弼,且詆薦廷弼者為逆黨哉?」當是時,忠賢勢益盛。宗建慮內外合謀,其禍將大,三年二月遂抗疏直攻忠賢,略言:

臣於去歲指名劾奏,進忠無一日忘臣。於是乘私人郭鞏入都,嗾以傾臣,并傾諸異己者。鞏乃創為「新幽大幽」之說,把持察典,編廷臣數十人姓名為一冊,思一網中之。又為匿名書,羅織五十餘人,投之道左,給事中則劉弘化為首,次及周朝瑞、熊德陽輩若而人,御史則方震孺為首,次及江秉謙輩若而人,而臣亦其中一人也。既欲羅諸臣,以快報復之私,更欲獨中臣,以釋進忠之恨。是察典不出於朝廷,乃鞏及進忠之察典也。幸直道在人,鞏說不行,始別借廷弼,欲一阱陷之。

鞏又因臣論及王安,笑臣有何瓜葛。陛下亦知安之所以死乎?身首異處,肉飽烏鳶,骨投黃犬,古今未有之慘也。鞏即心暱進忠,何至背公滅理,且牽連劉一燝、周嘉謨、楊漣、毛士龍輩,謂盡安黨。請陛下窮究安死果出何人傾害,則此事即進忠一大罪案。鞏之媚進忠,即此可為證據矣。

先朝汪直、劉瑾,雖皆梟獍,幸言路清明,臣僚隔絕,故非久即敗。今權璫報復,反借言官以伸;言官聲勢,反借權璫以重。數月以來,熊德陽、江秉謙、侯震暘、王紀、滿朝薦斥矣,鄒元標、馮從吾罷矣,文震孟、鄭鄤逐矣,近且扼孫慎行、盛以弘,而絕其揆路。摘瓜抱蔓,正人重足。舉朝各愛一死,無敢明犯其鋒者。臣若尚顧微軀,不為入告,將內有進忠為之指揮,旁有客氏為之羽翼,外有劉朝輩為典兵示威,而又有鞏輩蟻附蠅集,內外交通,驅除善類,天下事尚忍言哉!疏入,進忠益怒。率劉朝等環泣帝前,乞自髡以激帝怒。乃令宗建陳交通實狀,將加重譴,宗建回奏益侃直。進忠議廷杖之,閣臣力爭,乃止,奪俸。

會給事中劉弘化、御史方大任等交章助宗建攻進忠、鞏,鞏復力詆諸人。詔下諸疏平議,廷臣為兩解之。乃嚴旨切責,奪鞏、宗建俸三月。是時,劉朝典內操,遂謀行邊。廷臣微聞之,莫敢言。宗建曰:「鞏自謂未嘗通內,今誠能出片紙遏朝,吾請為洗交結之名。」鞏噤不敢發。宗建乃抗疏極諫,歷陳三不可、九害。會朝與進忠有隙,事亦中寢。其冬出按湖廣,以憂歸。

五年三月,大學士馮銓銜御史張慎言嘗論己,屬其門生曹欽程誣劾,而以宗建為首,并及李應昇、黃尊素。忠賢遂矯詔削籍,下撫按追贓。明年以所司具獄緩,遣緹騎逮治。俄入之李實疏中,下詔獄毒訊。許顯純厲聲罵曰:「復能詈魏上公一丁不識乎!」竟坐納廷弼賄萬三千,斃之獄。

宗建既死,徵贓益急。其所親副使蔣英代之輸,亦坐削籍。忠賢敗,詔贈宗建太僕寺卿,官其一子。福王時,追謚忠毅。

蔣英,嘉善人。舉進士,歷知松溪、漳浦、宜興。天啟時,由南京驗封郎中,出為福建副使,遂遭璫禍。忠賢敗,以故官分巡蘇、松,坐事貶秩。未行而宜興民變,上官以英先治宜興,得民心,檄之撫治。宜興非英所轄,辭不得,則單騎往諭,懲豪家僮客數人,令亂民自獻其首惡,亂遂定。宜興故多豪家,修撰陳於泰、編修陳於鼎兄弟尤橫,遂激民變,群執兵鼓噪,勢洶洶。賴英,事旋定。而周延儒方枋國,與陳氏有連,銜英,再貶兩秩,遂歸。

鞏,遷安人。以附忠賢,驟遷至兵部侍郎。莊烈帝定逆案,削籍論配。我大清拔遷安,鞏遁去,後詣闕自言拒聘,上所撰《卻聘書》。兵部尚書梁廷棟論之,下獄坐死。巡撫楊嗣昌為訟冤,得遣戍。

黃尊素,字真長,餘姚人。萬歷四十四年進士。除寧國推官,精敏彊執。

天啟二年,擢御史,謁假歸。明年冬還朝,疏請召還餘懋衡、曹於汴、劉宗周、周洪謨、王紀、鄒元標、馮從吾,而劾尚書趙秉忠、侍郎牛應元、通政丁啟睿頑鈍。秉忠、應元俱引去。山東妖賊既平,餘黨復煽,巡撫王惟儉不能撫馭,尊素疏論之,因言:「巡撫本內外兼用,今盡用京卿,不若揚歷外服者之練習。」又數陳邊事,力詆大將馬世龍,忤樞輔孫承宗意。時帝在位數年,未嘗一召見大臣。尊素請復便殿召對故事,面決大政,否則講筵之暇,令大臣面商可否。帝不能用。

四年二月,大風揚沙,晝晦,天鼓鳴,如是者十日。三月朔,京師地震三,乾清宮尤甚。適帝體違和,人情惶懼。尊素力陳時政十失,末言:「陛下厭薄言官,人懷忌諱,遂有剽竊皮毛,莫犯中扃者。今阿保重於趙嬈,禁旅近於唐末,蕭牆之憂慘於敵國。廷無謀幄,邊無折衝,當國者昧安危之機,誤國者護恥敗之局。不於此進賢退不肖,而疾剛方正直之士如仇仇,陛下獨不為社稷計乎?」疏入,魏忠賢大怒,謀廷杖之。韓爌力救,乃奪俸一年。

既而楊漣劾忠賢,被旨譙讓。尊素憤,抗疏繼之,略言:「天下有政歸近倖,威福旁移,而世界清明者乎?天下有中外洶洶,無不欲食其肉,而可置之左右者乎?陛下必以為曲謹可用,不知不小曲謹,不大無忌;必以為惟吾駕馭,不知不可駕馭,則不可收拾矣。陛下登極以來,公卿臺諫纍累罷歸,致在位者無固志。不於此稱孤立,乃以去一近侍為孤立耶?今忠賢不法狀,廷臣已發露無餘,陛下若不早斷,彼形見勢窮,復何顧忌。忠賢必不肯收其已縱之韁,而凈滌其腸胃;忠賢之私人,必不肯回其已往之棹,而默消其冰山。始猶與士大夫為仇,繼將以至尊為注。柴柵既固,毒螫誰何?不惟臺諫折之不足,即干戈取之亦難矣。」忠賢得疏愈恨。

萬燝既廷杖,又欲杖御史林汝翥,諸言官詣閣爭之。小璫數百人擁入閣中,攘臂肆罵,諸閣臣俯首不敢語。尊素厲聲曰:「內閣絲綸地,即司禮非奉詔不敢至,若輩無禮至此!」乃稍稍散去。無何,燝以創重卒。尊素上言:「律例,非叛逆十惡無死法。今以披肝瀝膽之忠臣,竟殞於磨牙礪齒之兇豎。此輩必欣欣相告,吾儕借天子威柄,可鞭笞百僚。後世有秉董狐筆,繼朱子《綱目》者,書曰『某月某日,郎中萬燝以言事廷杖死』,豈不上累聖德哉!進廷杖之說者,必曰祖制,不知二正之世,王振、劉瑾為之;世祖、神宗之朝,張璁、嚴嵩、張居正為之。奸人欲有所逞,憚忠臣義士掣其肘,必借廷杖以快其私,使人主蒙拒諫之名,己受乘權之實,而仁賢且有抱蔓之形。於是乎為所欲為,莫有顧忌,而禍即移之國家。燝今已矣,辱士殺士,漸不可開。乞復故官,破格賜恤,俾遺孤得扶櫬還鄉,燝死且不朽。」疏入,益忤忠賢意。

八月,河南進玉璽。忠賢欲侈其事,命由大明門進,行受璽禮,百僚表賀。尊素上言:「昔宋哲宗得璽,蔡確等競言祥瑞,改年元符,宋祚卒不競。本朝弘治時,陜西獻玉璽,止令取進,給賞五金。此祖宗故事,宜從。」事獲中止。五年春,遣視陜西茶馬。甫出都,逆黨曹欽程劾其專擊善類,助高攀龍、魏大中虐焰,遂削籍。

尊素謇諤敢言,尤有深識遠慮。初入臺,鄒元標實援之,即進規曰:「都門非講學地,徐文貞已叢議於前矣。」元標不能用。楊漣將擊忠賢,魏大中以告,尊素曰:「除君側者,必有內援。楊公有之乎?一不中,吾儕無噍類矣。」萬景死,尊素諷漣去,漣不從,卒及於禍。大中將劾魏廣微,尊素曰:「廣微,小人之包羞者也,攻之急,則挺而走險矣。」大中不從,廣微益合於忠賢,以興大難。

是時,東林盈朝,自以鄉里分朋黨。江西章允儒、陳良訓與大中有隙,而大中欲駁尚書南師仲恤典,秦人亦多不悅。尊素急言於大中,止之。最後,山西尹同皋、潘雲翼欲用其座主郭尚友為山西巡撫,大中以尚友數問遺朝貴,執不可。尊素引杜征南數遺洛中貴要為言,大中卒不可,議用謝應祥,難端遂作。

汪文言初下獄,忠賢即欲羅織諸人。已,知為尊素所解,恨甚。其黨亦以尊素多智慮,欲殺之。會吳中訛言尊素欲效楊一清誅劉瑾,用李實為張永,授以秘計。忠賢大懼,遣刺事者至吳中凡四輩。侍郎烏程沈演家居,奏記忠賢曰:「事有迹矣。」於是日遣使譙訶實,取其空印白疏,入尊素等七人姓名,遂被逮。使者至蘇州,適城中擊殺逮周順昌旂尉,其城外人并擊逮尊素者。逮者失駕帖,不敢至。尊素聞,即囚服詣吏,自投詔獄。許顯純、崔應元搒掠備至,勒贓二千八百,五日一追比。已,知獄卒將害己,叩首謝君父,賦詩一章,遂死,時六年閏六月朔日也,年四十三。崇禎初,贈太僕卿,任一子。福王時,追謚忠端。

李應昇,字仲達,江陰人。萬歷四十四年進士。授南康推官。出無辜十九人於死,置大猾數人重辟。士民服其公廉,為之謠曰:「前林後李,清和無比。」林謂晉江林學曾,卒官南京戶部侍郎,以清慎著稱者也。九江、南康間有柯、陳二大族,相傳陳友諒苗裔,負固強梗,嘗拒捕,有司議兵之。應昇單騎往諭,皆叩頭聽命,出所匿罪人,一方以定。

天啟二年,徵授御史,謁假歸。明年秋,還朝。時天子暗弱,庶政怠弛。應升上疏曰:「方今遼土淪沒,黔、蜀用兵,紅夷之焰未息,西部之賞日增;逃兵肆掠於畿輔,窮民待盡於催科。逗遛習慣,大將畏敵而不敢前;法紀陵夷,驕兵鼓噪而弗能問。在在增官,日日會議;覆疏衍為故套,嚴旨等若空言。陛下不先振竦精神,發皇志氣,群臣孰肯任怨以破情面之世界者?祖宗有早午晚三朝,猶時御便殿咨訪時政。願俯納臣言,奮然力行,天下事尚可為也。」報聞。

頃之,復陳時政,略曰:「今天下敝壞極矣,在君臣奮興而力圖之。陛下振紀綱,則片紙若霆;大臣捐私曲,則千里運掌;臺諫任糾彈,則百司飲冰。今動議增官,為人營窟,紛紜遷徙,名實乖張。自登、萊增巡撫,而侵冒百餘萬;增招練監軍,而侵冒又十餘萬。邊關內地,將領如蟻,剝軍侵饟,又不知幾十萬。增置總督,何補塞垣;增置京堂,何裨政事。樞貳添注矣,孰慷慨以行邊;司空添注矣,孰拮據以儲備;大將添注矣,只工媒孽而縱逋逃;禮、兵司屬添注二三十人矣,誰儲邊才而精典禮。濫開邊俸,捷徑燃灰,則吏治日壞;白衣攘臂,邪人入幕,則奸弁充斥。臣請斷自聖心,一切報罷。」又言:「今事下部曹,十九寢閣,宜重申國典,明正將領之罪。錦衣旂尉,半歸權要,宜遣官巡視,如京營之制。衛官襲職,比試不嚴,宜申明舊章,無使倖進將校蠶食。逃軍不招,私募乞兒,半分其饟,宜力為創懲。窮民敲撲,號哭滿庭,奸吏侵漁,福堂安坐,宜嚴其法制。」時不能用。俄劾南京都御史王永光庇部郎范得志,顛倒公論,永光尋自引去。

四年正月,疏陳外番、內盜及小人三患,譏切近習,魏忠賢惡之。已,復疏陳民隱,言有十害宜急除,五反宜急去,帝為戒飭所司。京師一日地三震,疏請保護聖躬,速停內操。忠賢領東廠,好用立枷,有重三百斤者,不數日即死,先後死者六七十人。應昇極言宜罷,忠賢大恨。應昇知忠賢必禍國,密草疏列其十六罪,將上,為兄所知,攘其疏毀之,怏怏而止。

楊漣劾忠賢,得嚴旨,應昇憤,即抗疏繼之。中言:「從來奄人之禍,其始莫不有小忠小信以固結主心,根株既深,毒手乃肆。今陛下明知其罪,曲賜包容。彼緩則圖自全之計,急則作走險之謀。蕭牆之間,能無隱禍?故忠賢一日不去,則陛下一日不安。臣為陛下計,莫如聽忠賢引退,以全其命;為忠賢計,亦莫若早自引決,以乞帷蓋之恩。不然惡稔貫盈,他日欲保首領,不可得矣。」又曰:「君側不清,安用彼相。一時寵利有盡,千秋青史難欺。不欲為劉健、謝遷者,并不能為東陽。倘畫策投歡,不幾與焦芳同傳耶?」

時魏廣微方深結忠賢,為之謀主,知應昇譏己,大恨。萬燝之死也,應昇極言廷杖不可再,士氣不可折,譏切忠賢輩甚至。已,代高攀龍草疏劾崔呈秀。呈秀窘,昏夜款門,長跪乞哀,應昇正色固拒,含怒而去。十月朔,帝廟享頒歷,廣微後至,為魏大中等所糾。廣微恚,辨疏詆言者。應升復抗疏論之,且曰:「廣微父允貞為言官,得罪輔臣以去,聲施至今。廣微奈何比言官路馬,斥為此輩?夫不與此輩為伍者,必別與一輩為緣。乞陛下戒諭廣微,退讀父書,保其家聲,毋倚三窟,與言官為難,他日庶可見乃父地下。」廣微益怒,謀之忠賢,將鐫秩。首輔韓爌力救,乃奪祿一年。其月,趙南星等悉被逐,朝事大變。

明年三月,工部主事曹欽程劾應昇護法東林,遂削籍。忠賢恨未已。六年三月,假李實劾周起元疏,入應昇名。遂逮下詔獄,酷掠,坐贓三千。尋於閏六月二日斃之,年甫三十四。崇禎初,贈太僕卿,錄一子。福王時,追謚忠毅。

萬燝,字暗夫,南昌人,兵部侍郎恭孫也。少好學,砥礪名行。舉萬曆四十四年進士,授刑部主事。嘗疏論刑獄干和。

天啟初元,兵事棘,工部需才,調燝工部營繕主事。督治九門垣墉,市銅江南,皆勤於其職。遷虞衡員外郎,司鼓鑄。時慶陵大工未竣,費不貲。燝知內府廢銅山積,可發以助鑄,移牒內官監言之。魏忠賢怒,不發,燝遂具疏以請。忠賢益怒,假中旨詰責。燝旋進屯田郎中,督陵務。

其時,忠賢益肆,廷臣楊漣等交擊,率被嚴旨。燝憤,抗章極論,略言:「人主有政權,有利權,不可委臣下,況刑餘寺人哉?忠賢性狡而貪,膽粗而大,口銜天憲,手握王爵,所好生羽毛,所惡成瘡痏。廕子弟,則一世再世;賚廝養,則千金萬金。毒痡士庶,斃百餘人;威加搢紳,空十數署。一切生殺予奪之權盡為忠賢所竊,陛下猶不覺悟乎?且忠賢固供事先帝者也,陛下之寵忠賢,亦以忠賢曾供事先帝也。乃於先帝陵工,略不厝念。臣嘗屢請銅,靳不肯予。間過香山碧雲寺,見忠賢自營墳墓,其規制弘敞,擬於陵寢。前列生祠,又前建佛宇,璇題耀日,珠網懸星,費金錢幾百萬。為己墳墓則如此,為先帝陵寢則如彼,可勝誅哉!今忠賢已盡竊陛下權,致內廷外朝止知有忠賢,不知有陛下,尚可一日留左右耶?」疏入,忠賢大怒,矯旨廷杖一百,斥為民。執政言官論救,皆不聽。

當是時,忠賢惡廷臣交章劾己,無所發忿,思借燝立威。乃命群奄至燝邸,摔而毆之,比至闕下,氣息纔屬。杖已,絕而復蘇。群奄更肆蹴踏,越四日即卒,時四年七月七日也。

忠賢恨猶不置,羅織其罪,誣以贓賄三百。燝廉吏,破產乃竣。崇禎初,贈光祿卿,官其一子。福王時,謚忠貞。

燝杖死未幾,巡城御史福清林汝翥嘗笞內侍曹進、傅國興,忠賢矯旨杖汝翥如燝。汝翥懼,逃之遵化,自歸於巡撫鄧渼。渼以聞,卒杖之。汝翥起家鄉舉,知沛縣,徐鴻儒攻沛甚急,堅守不下,由此擢御史。崇禎時,仕至浙江副使。汝翥雖受杖,幸不死。而是時,丁乾學、夏之令、吳裕中、劉鐸、吳懷賢、蘇繼歐、張汶諸人,皆忤忠賢致死。

乾學,浙江山陰人,寄籍京師,官檢討。天啟四年,偕給事中郝土膏典試江西,發策刺忠賢。忠賢怒,矯旨鐫三秩,復除其名。已,使人詐為校尉往逮,挫辱之,竟憤鬱而卒。崇禎初,贈侍讀學士。

之令,光山人。知攸、歙二縣,徵授御史。嘗疏論邊事,力詆毛文龍不足恃。忠賢庇文龍,傳旨削之令籍,閣臣救免。及巡皇城,內使馮忠等犯法,劾治之,益為忠賢所銜,崔呈秀亦以事銜之。遂屬御史卓邁劾之令黨比熊廷弼,有詔削奪。頃之,御史倪文煥復劾之令計陷文龍,幾誤疆事。遂逮下詔獄,坐贓拷死。

裕中,江夏人。為順德知縣,徵授御史。大學士丁紹軾陷熊廷弼死,裕中有疏詆紹軾。忠賢傳旨詰裕中為廷弼姻戚,代之報仇,廷杖一百,創重卒。崇禎初,賜贈蔭。

鐸,廬陵人。由刑部郎中為揚州知府。憤忠賢亂政,作詩書僧扇,有「陰霾國事非」句,偵者得之,聞於忠賢。倪文煥者,揚州人也,素銜鐸,遂嗾忠賢逮治之。鐸雅善忠賢子良卿,事獲解,許還故官。良卿從容問鐸:「曩錦衣往逮,索金幾何?」曰:「三千金耳。」良卿令錦衣還之。其人怒,日夜伺鐸隙,言鐸繫獄時,與囚方震孺同謀居間,遂再下獄。會鐸家人有夜醮者,參將張體乾誣鐸咒詛忠賢,刑部尚書薛貞坐以大辟。忠賢誅,貞、體乾並抵罪,鐸贈太僕少卿。

懷賢,休寧人。由國子監生授內閣中書舍人。同官傅應昇者,忠賢甥也,懷賢遇之無加禮,應升恨之。楊漣劾忠賢疏出,懷賢書其上曰:「宜如韓魏公治任守忠故事,即時遣戍。」又與工部主事吳昌期書,有「事極必反,反正不遠」語。忠賢偵知之,大怒曰:「何物小吏,亦敢謗我!」遂矯旨下詔獄,坐以結納汪文言,為左光斗、魏大中鷹犬,拷掠死。崇禎初,贈工部主事。

繼歐,許州人。歷知元氏、真定、柏鄉,入為吏部稽勛主事,累遷考功郎中。將調文選,中旨謂為楊漣私黨,削籍歸。時緹騎四出,同里副使孫織錦素附忠賢,遣人怵繼歐曰:「逮者至矣。」繼歐自經死。崇禎初,贈太常寺卿。

汶,邯鄲人。尚書國彥曾孫也。由廕敘為後軍都督府經歷。嘗被酒詆忠賢,下獄拷掠死。亦獲贈恤。

贊曰:自古閹宦之甘心善類者,莫甚於漢、唐之季,然皆倉卒一時,為自救計耳。魏忠賢之殺諸人也,揚毒焰以快其私,肆無忌憚。蓋主荒政粃之餘,公道淪亡,人心敗壞,兇氣參會,群邪翕謀,故搢紳之禍烈於前古。諸人之受禍也,酷矣哉!


\end{pinyinscope}