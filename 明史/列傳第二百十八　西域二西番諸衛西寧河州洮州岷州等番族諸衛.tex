\article{列傳第二百十八 西域二西番諸衛西寧河州洮州岷州等番族諸衛}

\begin{pinyinscope}
○安定衛阿端衛曲先衛赤斤蒙古衛沙州衛罕東衛罕東左衛哈梅里

西番,即西羌,族種最多,自陜西歷四川、雲南西徼外皆是。其散處河、湟、洮、岷間者,為中國患尤劇。漢趙充國、張奐、段熲,唐哥舒翰,宋王韶之所經營,皆此地也。元封駙馬章古為寧濮郡王,鎮西寧,於河州設吐番宣慰司,以洮、岷、黎、雅諸州隸之,統治番眾。

洪武二年,太祖定陜西,即遣官齎詔招諭,其酋長皆觀望。復遣員外郎許允德招之,乃多聽命。明年五月,吐蕃宣慰使何鎖南普等以元所授金銀牌印宣敕來上,會鄧愈克河州,遂詣軍前降。其鎮西武靖王卜納剌亦以吐蕃諸部來納款。冬,何鎖南普等入朝貢馬及方物。帝喜,賜襲衣。四年正月設河州衛,命為指揮同知,予世襲,知院朵兒只、汪家奴並為指揮僉事。設千戶所八,百戶所七,皆命其酋長為之。卜納剌等亦至京師,為靖南衛指揮同知,其儕桑加朵兒只為高昌衛指揮同知,皆帶刀侍衛。自是,番酋日至。尋以降人馬梅、汪瓦兒並為河州衛指揮僉事。又遣西寧州同知李喃哥等招撫其酋長,至者亦悉授官。乃改西寧州為衛,以喃哥為指揮。

帝以西番產馬,與之互市,馬至漸多。而其所用之貨與中國異,自更鈔法後,馬至者少,患之。八年五月命中官趙成齎羅綺、綾絹並巴茶往河州市之,馬稍集,率厚其值以償。成又宣諭德意,番人感悅,相率詣闕謝恩。山後歸德等州西番諸部落皆以馬來市。

十二年,洮州十八族番酋三副使等叛,據納麟七站之地。命征西將軍沐英等討之,又命李文忠往籌軍事。英等至洮州舊城,寇遁去,追斬其魁數人,盡獲畜產。遂於東籠山南川度地築城置戍,遣使來奏。帝報曰:「洮州,西番門戶,築城戍守,扼其咽喉。」遂置洮州衛,以指揮聶緯、陳暉等六人守之。已,文忠等言官軍守洮州,餉艱民勞。帝降敕諭之曰:「洮州西控番戎,東蔽湟、隴,漢、唐以來備邊要地。今番寇既斥,棄之不守,數年後番人將復為患。慮小費而忘大虞,豈良策哉。所獲牛羊,分給將士,亦足棄兩年軍食。其如敕行之。」文忠等乃不敢違。

秋,何鎖南普及鎮撫劉溫各攜家屬來朝。諭中書省臣曰:「何鎖南普自歸附以來,信義甚堅。前遣使烏斯藏,遠涉萬里,及歸,所言皆稱朕意。今以家屬來朝,宜加禮待。」乃賜米、麥各三十石,劉溫三之一。

英等進擊番寇,大破之,盡擒其魁,俘斬數萬人,獲馬牛羊數十萬。自是,群番震懾,不敢為寇。

十六年,青海酋長史剌巴等七人來歸,賜文綺、寶鈔。時岷州亦設衛,番人歲以馬易茶,馬日蕃息。二十五年又命中官而聶至河州,召必里諸番族,以敕諭之。爭出馬以獻,得萬三百餘匹,給茶三十餘萬觔。命以馬畀河南、山東、陜西騎士。帝以諸衛將士有擅索番人馬者,遣官齎金、銅信符敕諭,往賜涼州、甘州、肅州、永昌、山丹、臨洮、鞏昌、西寧、洮州、河州、岷州諸番族。諭之曰:「往者朝廷有所需,必酬以茶貨,未許私徵。近聞邊將無狀,多假朝命擾害,俾爾等不獲寧居。今特製金、銅信符頒給,遇有徵發,必比對相符始行,否則偽,械至京,罪之。」自是,需求遂絕。

初,西寧番僧三剌為書招降罕東諸部,又建佛剎於碾白南川,以居其眾,至是來朝貢馬,請敕護持,賜寺額。帝從所請,賜額曰瞿曇寺。立西寧僧綱司,以三剌為都綱司。又立河州番、漢二僧綱司,並以番僧為之,紀以符契。自是其徒爭建寺,帝輒錫以嘉名,且賜敕護持。番僧來者日眾。

永樂時,諸衛僧戒行精勤者,多授剌麻、禪師、灌頂國師之號,有加至大國師、西天佛子者,悉給以印誥,許之世襲,且令歲一朝貢,由是諸僧及諸衛士官輻輳京師。其他族種,如西寧十三族、岷州十八族、洮州十八族之屬,大者數千人,少者數百,亦許歲一奉貢,優以宴賚。西番之勢益分,其力益弱,西陲之患亦益寡。

宣德元年,以協討安定、曲先功,加國師吒思巴領占等五人為大國師,給誥命、銀印,秩正四品,加剌麻著星等六人為禪師,給敕命、銀印,秩正六品。

正統五年敕陜西鎮守都督鄭銘、都御史陳鎰曰:「得奏,言河州番民領占等先因避罪,逃居結河里,招集徒黨,占耕土田,不注籍納賦,又藏匿逃亡,剽劫行旅,欲發兵討之。朕念番性頑梗,且所犯在赦前,若遽加師旅,恐累及無辜。宜使人撫諭,令散遣徒黨,還所掠牛羊,兵即勿進,否則加兵未晚。爾等其審之。」番人果輸服。七年再敕銘及都御史王翱等曰:「得鎮守河州都指揮劉永奏:往歲阿爾官等六族三千餘人,列營歸德城下,聲言交易,後乃鈔掠屯軍,大肆焚戮;而著亦匝族番人屢於煖泉亭諸處,潛為寇盜。指揮張瑀擒獲二人,止責償所盜馬,縱之使去。論法,瑀及永皆當究治,今姑令戴罪。爾等即遣官偕三司堂上親詣其寨,曉以利害,令還歸所掠,許其自新,不悛,則進討。蓋馭戎之道,撫綏為先,撫之不從,然後用兵。爾等宜體此意。」番人亦輸服。

成化三年,陜西副使鄭安言:「進貢番僧,自烏斯藏來者不過三之一,餘皆洮、岷寺僧詭名冒貢。進一羸馬,輒獲厚直,得所賜幣帛,製為戰袍,以拒官軍。本以羈縻之,而益致寇掠,是虛國帑而齎盜糧也。」章下禮部,會廷臣議,請行陜西文武諸臣,計定貢期、人數及存留、起送之額以聞,報可。已而奏上,諸自烏斯藏來者皆由四川入,不得徑赴洮、岷,遂著為例。明年冬,洮州番寇擁眾掠鐵城、後川二寨,指揮張翰等率兵禦之,敗去,獲所掠人口以歸。

五年,巡按江孟綸言:「岷州番寇縱橫,村堡為虛。頃令指揮后泰與其弟通反覆開示,生番忍藏、占藏等三十餘族酋長百六十餘人,熟番栗林等二十四族酋長九十一人,轉相告語,悔過來歸,且還被掠人畜,願供徭賦。殺牛告天,誓不再犯。已令副使李從宜賞勞,宣示朝廷恩威,皆歡躍而去。惟熟番綠園一族怙惡不服。」兵部言:「番性無常,朝撫夕叛,未可弛備。請諭邊臣,向化者加意撫綏,犯順者克期剿滅。」帝納其言。

八年,禮官言:「洮、岷諸衛送各族番人赴京,多至四千二百餘人,應賞彩幣人二表裏,帛如之,鈔二十九萬八千有奇,馬直尚在其外。考正統、天順間,各番貢使不過三五百人。成化初,因洮、岷諸處濫以熟番作生番冒送,已定例,生番三年一貢,大族四五人,小族一二人赴京,餘悉遣還。成化六年,副使鄧本瑞妄自招徠,又復冒送,臣部已重申約束。今副使吳等不能嚴飭武備,專事通番,以紓近患。乞降敕切責,務遵前令。」帝亦如其言。

西寧即古湟中,其西四百里有青海,又曰西海,不草豐美。番人環居之,專務畜牧,日益繁滋,素號樂土。正德四年,蒙古部酋亦不刺、阿爾禿廝獲罪其主,擁眾西奔。瞰知青海饒富,襲而據之,大肆焚掠。番人失其地,多遠徙。其留者不能自存,反為所役屬。自是甘肅、西寧始有海寇之患。九年,總制彭澤集諸道軍,將搗其巢。寇詗知之,由河州渡黃河,奔四川,出松潘、茂州境,直走烏斯藏。及大軍引還,則仍返海上,惟阿爾禿廝遁去。

嘉靖二年,尚書金獻民西征,議遣官招撫,許為籓臣,如先朝設安定、曲先諸衛故事。兵部行總制楊一清計度,一清意在征討,言寇精騎不過二三千,餘皆脅從番人,然怨之入骨,時欲報仇,可用為間諜,大舉剿絕。議末定,王憲、王瓊相繼來代,皆以兵寡餉詘,議竟不行。

八年,洮、岷諸番數犯臨洮、鞏昌,內地騷動。樞臣李承勳言:「番為海寇所侵,日益內徙。倘二寇交通,何以善後。昔趙充國不戰而服羌,段穎殺羌百萬而內地虛耗,兩者相去遠矣。乞廣先帝之明,專充國之任,制置方略,悉聽瓊便宜從事。」瓊乃集眾議,且剿且撫。先遣總兵官劉文、遊擊彭椷分布士馬。明年二月自固原進至洮、岷,遣人開示禍福。洮州東路木舍等三十一族,西路答祿失等十三族,岷州西寧溝等十五族,皆聽撫,給白旂犒賜遣歸。惟岷州東路若籠族、西路板爾等十五族及岷州剌即等五族,恃險不服。乃分兵先攻若籠、板爾二族,覆其巢,剌即諸族震慴乞降。凡斬首三百六十餘級,撫定七十餘族,乃班師。自是,洮、岷獲寧,而西寧仍苦寇患。

十一年,甘肅巡撫趙載等言:「亦不剌據海上已二十餘年,其黨卜兒孩獨傾心向化,求帖木哥等屬番來納款。宜因而撫之,或俾之納馬,或令其遣質,或授官給印,建立衛所,為我籓籬,於計為便。」疏甫上,會河套酋吉囊引眾西掠,大破亦不剌營,收其部落大半而去,惟卜兒孩一枝斂眾自保。由是西寧亦獲休息,而納款之議竟寢。及唐龍為總制,寇南掠松潘。龍慮其回巢與諸番及他部勾結為患,奏行甘肅守臣,繕兵積粟,為殄滅計。及龍去,事亦不行。

二十年正月,卜兒孩獻金牌、良馬求款。兵部言:「寇果輸誠通貢,誠西陲大利。乃止獻馬及金牌,未有如往歲遣子入侍、酋長入朝之請,未可遽許。宜令督撫臣偵察情實,並條制馭之策以聞。」報可。會寇勢漸衰,番人亦漸復業,其議復寢。

二十四年設岷州,隸鞏昌府。岷西臨極邊,番漢雜處。洪武時,改土番十六族為十六里,設衛治之,俾稍供徭役。自設州之後,徵發繁重,人日困敝。且番人戀世官,而流官又不樂居,遙寄治他所。越十餘年,督撫合疏言不便,乃設衛如故。

時北部俺答猖獗,歲掠宣、大諸鎮。又羨青海富饒,三十八年攜子賓兔、丙兔等數萬眾,襲據其地。卜兒孩竄走,遂縱掠諸番。已,引去,留賓兔據松山,丙兔據青海,西寧亦被其患。隆慶中,俺答受封順義王,修貢惟謹,二子亦斂戢。

時烏斯藏僧有稱活佛者,諸部多奉其教。丙兔乃以焚修為名,請建寺青海及嘉峪關外,為久居計。廷臣多言不可許,禮官言:「彼已採木興工,而令改建於他所,勢所不能,莫若因而許之,以鼓其善心,而杜其關外之請。況中國之禦戎,惟在邊關之有備。戎之順逆,亦不在一寺之遠近。」帝許之。丙兔既得請,又近脅番人,使通道松潘以迎活佛。四川守臣懼逼,乞令俺答約束其子,毋擾鄰境。俺答言,丙兔止因甘肅不許開市,寧夏又道遠艱難,雖有禁令,不能盡制。宣大總督方逢時亦言開市為便。帝以責陜西督撫,督撫不敢違。萬曆二年冬,許丙兔市於甘肅,賓兔市於莊浪,歲一次。既而寺成,賜額仰華。

先是,亦不剌之據青海,邊臣猶以外寇視之。至是以俺答故,竟視若屬番。諸酋亦以父受王封,不敢大為邊患,而洮州之變乃起。初,洮州番人以河州奸民負其物貨,入掠內地,他族亦乘機為亂。奸民以告河州參將陳堂,堂曰:「此洮州番也,何與我事。」洮州參將劉世英曰:「彼犯河州,非我失事。」由是二將有隙。總督石茂華聞之,令二人及蘭州參將徐勳、岷州守備硃憲、舊洮州守備史經各引兵壓其境,曉以利害。番人懼,即還所掠人畜。世英謂首惡未擒,不可遽已,遂剿破之,殺傷及焚死者無算。軍律,吹銅角乃退兵。堂挾前憾,不待角聲而去,諸部亦多引去。憲、經方深入搜捕,鄰番見其勢孤,圍而殺之。事聞,帝震怒,褫堂、世英職,切責茂華等。茂華乃集諸軍分道進討,斬首百四十餘級,焚死者九百餘人,獲孳畜數十群。諸番震恐遠徙,來降者七十一族,斬送首惡四人,生縛以獻者二人,輸馬牛羊二百六十。稽首謝罪,誓不再犯,師乃還。

自丙兔據青海,有切盡台吉者,河套酋吉能從子,俺答從孫也,從之而西。屢掠番人不得志,邀俺答往助。俺答雅欲侵瓦剌,乃假迎活佛名,擁眾西行。疏請授丙兔都督,賜金印,且開茶市。部議不許,但稍給以茶。俺答既抵瓦剌,戰敗而還。乃移書甘肅守臣,乞假道赴烏斯藏。守臣不能拒,遂越甘肅而南,會諸酋於海上。番人益遭蹂躪,多竄徙。八年春,始以活佛言東還,而切盡弟火落赤及俺答庶兄子永邵卜遂留居青海不去。八月,丙兔率眾掠番並內地人畜,詔絕其市賞。俺答聞之,馳書切責。乃盡還所掠,執獻為惡者六人,自罰牛羊七百。帝嘉其父恭順,賚之銀幣,即以牛羊賜其部人,為惡者付之自治,仍許貢市,俺答益感德。而火落赤侵掠番族不休,守臣檄切盡台吉約束之,亦引罪輸服。及俺答卒,傳至孫扯力克,勢輕,不能制諸酋。

十六年九月,永邵卜部眾有闌入西寧者,副總兵李奎方被酒,躍馬而前。部眾控鞍欲愬,奎拔刀斫之,眾遂射奎死。部卒馳救之,亦多死。守臣不能討,遣使詰責,但獻首惡,還人畜而止。以故無所憚,愈肆侵盜。時丙兔及切盡台吉亦皆死,丙兔子真相移駐莽剌川,火落赤移駐捏工川,逼近西寧,日蠶食番族。番不能支,則折而為寇用。扯力克又西行助之,勢益熾。十八年六月入舊洮州,副總兵李聯芳率三千人御之,盡覆。七月復深入,大掠河州、臨洮、渭源。總兵官劉承嗣與遊擊孟孝臣各將一軍禦之,皆敗績,遊擊李芳等死焉,西陲大震。事聞,命尚書鄭洛出經略。洛前督宣大軍,撫順義王及忠順夫人有恩。遣使趣扯力克東歸,而大布招番之令,來者率善遇之,自是歸附者不絕。火、真二酋自知罪重,又聞套酋卜失兔來助,大敗於水泉口,扯力克復將還巢,始懼。徙帳去,留其黨可卜兔等於莽剌川。明年,總兵官尤繼先破走之。洛更進兵青海,焚仰華寺,逐其餘眾而還。番人復業者至八萬餘人,西陲暫獲休息。已,復聚於青海。

二十三年增設臨洮總兵官,以劉綎任之。未幾,永邵卜諸部犯南川,參將達雲大破之。已,連火、真二酋犯西川,雲又擊破之。明年,諸酋復掠番族,將窺內地。綎部將周國柱禦之莽剌川,又大破之。二十七年糾叛苗犯洮、岷,總兵官蕭如薰等敗之,斬番人二百五十餘級,寇八十二級,撫降番族五千餘人。三十四年復入鎮番黑古城,為總兵官柴國柱所敗。自是屢入鈔掠,不能大得志。

時為陜西患者,有三大寇:一河套,一松山,一青海。青海土最沃,且有番人屏蔽,故患猶不甚劇。崇禎十一年,李自成屢為官軍擊敗,自洮州軼出番地。諸將窮追,復奔入塞內,番族亦遭蹂躪。十五年,西寧番族作亂,總抹官馬爌督諸將五道進剿,斬首七百有奇,撫降三十八族而還。明年冬,李自成遣將陷甘州,獨西寧不下。賊將辛恩忠攻破之,遂進掠青海。諸酋多降附,而明室亦亡。

番有生熟二種。生番獷悍難制。熟番納馬中茶,頗柔服,後浸通生番為內地患。自青海為寇所據,番不堪剽奪,私饋皮幣曰手信,歲時加饋曰添巴,或反為嚮導,交通無忌。而中國市馬亦鮮至,蓋已失捍外衛內之初意矣。

原夫太祖甫定關中,即法漢武創河西四郡隔絕羌、胡之意,建重鎮於甘肅,以北拒蒙古,南捍諸番,俾不得相合。又遣西寧等西衛土官與漢官參治,令之世守。且多置茶課司,番人得以馬易茶。而部族之長,亦許其歲時朝貢,自通名號於天子。彼勢既分,又動於利,不敢為惡。即小有蠢動,邊將以偏師制之,靡不應時底定。自邊臣失防,北寇得越境闌入,與番族交通,西陲遂多事。然究其時之所患,終在寇而不在番,故議者以太祖制馭為善。

安定衛,距甘州西南一千五百里。漢為婼羌,唐為吐蕃地,元封宗室卜煙帖木兒為寧王鎮之。其地本名撒里畏兀兒,廣袤千里,東近罕東,北邇沙州,南接西番。居無城郭,以氈帳為廬舍。產多駝馬牛羊。

洪武三年遣使持詔招諭。七年六月,卜煙帖木兒使其府尉麻答兒等來朝,貢鎧甲刀劍諸物。太祖喜,宴賚其使者,遣官厚賚其王,而分其地為阿端、阿真、苦先、貼里四部,各錫以印。明年正月,其王遣傅卜顏不花來貢,上元所授金、銀字牌,請置安定、阿端二衛,從之。乃封卜煙帖木兒為安定王,以其部人沙刺等為指揮。

九年命前廣東參政鄭九成等使其地,賚王及其部人衣幣。明年,王為沙剌所弒,王子板咱失里復仇,誅沙剌。沙剌部將復殺王子,部內大亂。番將朵兒只巴叛走沙漠,經安定,大肆殺掠,奪其印去,其眾益衰。二十五年,藍玉西征,徇阿真川。土酋司徒哈昝等懼,逃匿山谷不敢出。及肅王之國甘州,遣僧謁王,乞授官以安部眾。王為奏請,帝許之。二十九年命行人陳誠至其地,復立安定衛。其酋長哈孩虎都魯等五十八人悉授指揮、千百戶等官。誠還,酋長隨之入朝,貢馬謝恩。帝厚賚之,復命中官齎銀幣往賜。

永樂元年遣官齎敕撫諭撒里諸部。明年,安定頭目多來朝,擢千戶三即等三人為指揮僉事,餘授官有差,并賜本衛指揮同知哈三等銀幣。未幾,指揮朵兒只束來朝,願納差發馬五百匹,命河州衛指揮康壽往受之。壽言:「罕東、必里諸衛納馬,其直皆河州軍民運茶與之。今安定遼遠,運茶甚難,乞給以布帛。」帝曰:「諸番市馬用茶,已著為例。今姑從所請,後仍給茶。」於是定制,上馬給布帛各二匹,以下遞減。三年,哈三等遣使來貢,奏舉頭目撤力加藏卜等為指揮等官,且請歲納孳畜什一,並從之。四年徙駐苦兒丁之地。

初,安定王之被殺也,其子撒兒只失加為其兄所殺,部眾潰散,子亦攀丹流寓靈藏。十一年五月率眾入朝,自陳家難,乞授職。帝念其祖率先歸附,令襲封安定王,賜印誥。自是朝貢不輟。

二十二年,中官喬來喜、鄧誠使烏斯藏,次畢力術江黃羊川。安定指揮哈三孫散哥及曲先指揮散即思等率眾邀劫之,殺朝使,盡奪駝馬幣物而去。仁宗大怒,敕都指揮李英偕康壽等討之。英等率西寧諸衛軍及隆奔國師賈失兒監藏等十二番族之眾,深入追賊,賊遠遁。英等踰崑崙山西行數百里,抵雅令闊之地,遇安定賊,擊敗之,斬首四百八十餘級,生擒七十餘人,獲駝馬牛十四萬有奇。曲先聞風遠竄,追之不及而還。英以此封會寧伯,壽等皆進秩。大軍既旋,指揮哈三等懼罪,不敢還故地。

宣德元年,帝遣官招諭之,復業者七百餘人。帝並賜彩幣表裏,以安其反側。三年春,賜安定及曲先衛指揮等官五十三人誥命。

初,大軍之討賊也,安定指揮桑哥與罕東衛軍同奉調從征。罕東違令不至,其所轄板納族瞰桑哥軍遠出,盡掠其部內廬帳畜產。事聞,降敕切責,令速歸所掠,違命則發兵進討。已,進桑哥都指揮僉事。

正統元年遣官齎敕諭安定王及桑哥曰:「我祖宗時,爾等順天命,尊朝廷,輸誠效力,始終不替,朝廷恩賚亦久而弗渝。肆朕嗣位,爾等復遵朝命,約束部下,良用爾嘉。茲特遣官往諭朕意,賜以幣帛。宜益順天心,篤忠誠,保境睦鄰,永享太平之福。」三年,桑哥卒,其子那南奔嗣職。九年,那南奔率眾掠曲先人畜。朝廷遣官諭還之,不奉命,反劫其行李。帝怒,敕責安定王追理。王既奉命,又陳詞乞憐。帝乃宥之,諭以保國睦鄰之義。十一年冬,亦攀丹卒,子領占幹些兒襲。時王年幼,叔父指揮同知輟思泰巴佐理國事,其同儕多不相下。王遣之入朝,奏請量加一秩,乃擢都指揮僉事。歷景泰、天順、成化三朝,頻入貢。

弘治三年,領占幹些兒卒,子千奔襲。賜齋糧、麻布,諭祭其父。先是,哈密忠順王卒,無子。廷議安定王與之同祖,遣官擇一人為其後,安定王不許。至是,訪求陜巴於安定,冊為忠順王,命千奔遣送其家屬。千奔怒曰:「陜巴不應嗣王爵,爵應歸綽爾加。」綽爾加者,千奔弟也。且邀厚賞。兵部言:「陜巴實忠順王之孫,素為國人所服。前哈密無主,遣使取應立者,綽爾加自知力弱不肯往。今事定之後,乃爾反覆,所言不可從。」陜巴迄得立。然千奔以立非己意,後哈密數被寇,竟不應援。十七年率眾侵沙州,大掠而去。正德時,蒙古大酋亦不剌、阿爾禿廝侵據青海,縱掠鄰境。安定遂殘破,部眾散亡。

阿端衛,在撒里畏兀兒之地,洪武八年置。後為朵兒只巴殘破,其衛遂廢。永樂四年冬,酋長小薛忽魯札等來朝,貢方物,請復置衛設官,從之,即授小薛等為指揮僉事。

洪熙時,曲先酋散即思邀劫朝使,脅阿端指揮鎖魯丹偕行。已,大軍出征,鎖魯丹懼,率部眾遠竄,失其印。宣德初遣使招撫,鎖魯丹猶不敢歸,依曲先雜處。六年春,西寧都督史昭言:「曲先衛真只罕等本別一部,因其父助散即思為逆,竄處畢力術江。其地當烏斯藏孔道,恐復為亂,宜討之。」帝敕昭曰:「殘寇窮迫,無地自容,宜遣人宥其罪,命復故業。」於是真只罕率所部還居帖兒谷舊地。明年正月入朝,天子喜,授指揮同知,令掌衛事,以指揮僉事卜答兀副之。真只罕因言:「阿端故城在回回境,去帖兒谷尚一月程,朝貢艱,乞移本土為便。」天子從其請,仍給以印,賜璽書撫慰之。迄正統朝,數入貢,後不知所終。

其時西域地亦有名阿端者,貢道從哈密入,與此為兩地云。

曲先衛,東接安定,在肅州西南。古西戎,漢西羌,唐吐蕃,元設曲先答林元帥府。

洪武時,酋長入貢。命設曲先衛,官其人為指揮。後遭朵兒只巴之亂,部眾竄亡,併入安定衛,居阿真之地。永樂四年,安定指揮哈三、散即思、三即等奏:「安定、曲先本二衛,後合為一。比遭吐番把禿侵擾,不獲寧居。乞仍分為二,復先朝舊制。」從之。即令三即為指揮使,掌衛事,散即思副之。又從其請,徙治藥王淮之地。自是屢入貢。

洪熙時,散即思偕安定部酋劫殺朝使。已,大軍往討,散即思率眾遠遁,不敢還故土。宣德初,天子赦其罪,遣都指揮陳通等往招撫,復業者四萬二千餘帳。乃遣指揮失刺罕等入朝謝罪,貢駝馬,待之如初。尋擢散即思都指揮同知,其僚屬悉進官,給以誥命。

五年六月,朝使自西域還,言散即思數率部眾邀劫往來貢使,梗塞道途。天子怒,命都督史昭為大將,率左右參將趙安、王彧及中官王安、王瑾,督西寧諸衛軍及安定、罕東之眾往征之。昭等兵至其地,散即思先遁,其黨脫脫不花等迎敵。諸將縱兵擊之,殺傷甚眾,生擒脫脫不花及男婦三百四十餘人,獲駝馬牛羊三十四萬有奇。自是西番震慴。散即思素狡悍,天子宥其罪,仍怙惡不悛。至是人畜多損失,乃悔懼。明年四月遣其弟副千戶堅都等四人貢馬請罪。復待之如初,令還居故地並歸其俘。

七年,其指揮那那罕言:「往者安定之兵從討曲先,臣二女、四弟及指揮桑哥等家屬被掠者五百人。今散即思已蒙赦宥,而臣等親屬猶未還,望聖明垂憐。」天子得奏惻然,語大臣曰:「朕常以用兵為戒,正恐濫及無辜。彼不自言,何由知之。」即敕安定王亦攀丹等悉歸所掠。其年,散即思卒,命其子都立嗣職,賜敕勉之。十年擢那那罕都指揮僉事,其僚屬進職者八十九人。正統七年遣使貢玉石。成化時,土魯番強,被其侵掠。

弘治中,安定王子陜巴居曲先。廷議哈密無主,迎為忠順王。正德七年,蒙古酋阿爾禿廝亦不剌竄居青海,曲先為所蹂躪,部族竄徙,其衛遂亡。

明初設安定、阿端、曲先、罕東、赤斤、沙州諸衛,給之金牌,令歲以馬易茶,謂之差發。沙州、赤斤隸肅州,餘悉隸西寧。時甘州西南盡皆番族,受邊臣羈絡,惟北面防寇。後諸衛盡亡,亦不剌據青海,土魯番復據哈密,逼處關外。諸衛遷徙之眾又環列甘肅肘腋,獷悍難馴。於是河西外防大寇,內防諸番,兵事日亟。

赤斤蒙古衛。出嘉峪關西行二十里曰大草灘,又三十里曰黑山兒,又七十里曰回回墓,墓西四十里曰騸馬城,並設墩臺,置尞卒。城西八十里即赤斤蒙古。漢燉煌郡地,晉屬晉昌郡,唐屬瓜州,元如之,屬沙州路。

洪武十三年,都督濮英西討,次白城,獲蒙古平章忽都帖木兒。進至赤斤站,獲豳王亦憐真及其部曲千四百人,金印一。師還,復為蒙古部人所據。

永樂二年九月,有塔力尼者,自稱丞相苦術子。率所部男婦五百餘人,自哈剌脫之地來歸。詔設赤斤蒙古所,以塔力尼為千戶,賜誥印、彩幣、襲衣。八年,回回哈剌馬牙叛於肅州,約塔力尼為援。拒不應,而率部下擒賊六人以獻。天子聞之喜,詔改千戶所為衛,擢塔力尼指揮僉事,其部下授官者三人。明年遣使貢馬。又明年以匿叛賊老的罕,將討之。用侍講楊榮言,止兵勿進,而賜敕詰責,塔力尼即擒老的罕來獻。天子嘉之,進秩指揮同知,賜賚甚厚。久之卒,子且旺失加襲,修貢如制,進指揮使。宣德二年再進都指揮同知,其僚屬亦多進秩。

正統元年,其部下指揮可兒即掠西域阿端貢物,殺使臣二十一人。賜敕切責,令還所掠。尋與蒙古脫歡帖木兒、猛哥不花戰,勝之,使來獻捷,進都指揮使。五年,朝使往來哈密者,且旺失加具餱糧、騾馬護送,擢都督僉事。明年,天子聞其部下時往沙州寇掠,或冒沙州名,邀劫西域貢使,遣敕切責。

時瓦剌兵強,數侵掠鄰境。且旺失加懼,欲徙居肅州。天子聞而諭止之,令有警馳報邊將。八年,瓦剌酋也先遣使送馬及酒,欲娶且旺失加女為子婦,娶沙州困即來女為弟婦。二人不欲,並奏遵奉朝命,不敢擅婚。天子以瓦剌方強,其禮意不可卻,諭令各從其願,並以此意諭也先,而二人終不欲。明年,且旺失加稱老不治事。詔授其子阿速都督僉事,代之。也先復遣使求婚,且請親人往受其幣物。阿速虞其詐,拒不從,而遣人乞徙善地。天子諭以土地不可棄,令獎率頭目圖自強。又以其饑困,令邊臣給之粟,所以撫恤者甚至。

先是,苦術娶西番女,生塔力尼;又娶蒙古女,生都指揮瑣合者、革古者二人。各分所部為三,凡西番人居左帳,屬塔力尼,蒙古人居右帳,屬瑣合者,而自領中帳。後苦術卒,諸子來歸,並授官。至是阿速勢盛,欲兼並右帳,屢相仇殺。瑣合者不能支,醖於邊將,欲以所部內屬。邊將任禮遣赴京,請發兵收其部落。帝慮其部人不願內徙,仍遣瑣合者還甘肅,而令禮往取其孥。十三年,邊將護哈密使臣至苦峪。赤斤都指揮總兒加陸等率眾圍其城,聲言報怨。官軍出擊之,獲總兒加陸,已而逃去。事聞,敕責阿速,令縛獻犯者。

景泰二年,也先復遣使持書求婚。會阿速他往,其僚屬以其書來上。兵部尚書于謙言:「赤斤諸衛久為我籓籬,也先無故招降結親,意在撤我屏蔽。宜令邊臣整兵慎防,並敕阿速悉力捍禦,有警馳報,發兵應援。」從之。五年,也先益圖兼并,遣使齎印授阿速,脅令臣服。阿速不從,報之邊臣。會也先被殺,獲已。

天順元年,都指揮馬雲使西域,命賜阿速綵幣,俾護送往還。尋進秩左都督。成化二年卒,子瓦撒塔兒請襲,即以父官授之。其部下指揮敢班數侵盜邊境,邊將誘致之,送京師。天子數其罪,賜賚遣還。六年,其部人以瓦撒塔兒幼弱,其叔父乞巴等二人為部族信服,乞命為都督,理衛事。瓦撒塔兒亦上書,乞予一職,協守邊方。帝從其請,並授指揮僉事。明年,瓦撒塔兒卒,子賞卜塔兒嗣為左都督。

九年,土魯番陷哈密,遣使三人,以書招都督僉事昆藏同叛。昆藏不從,殺其使,以其書來獻。天子嘉之,遣使賜賚,且令發兵攻討。昆藏以力不足,請發官軍數千為助。朝議委都督李文等計度。已,文等進徵,昆藏果以兵來會。會文等頓軍不進,其兵亦還。

十年,賞卜塔兒以千騎入肅州境,將與阿年族番人仇殺。邊臣既諭卻之,兵部請遣人責以大義,有仇則赴愬邊吏,不得擅相侵掠,從之。十四年,其部人言賞卜塔兒幼不更事,指揮僉事加定得眾心,乞遷一秩,俾總衛事。賞卜塔兒亦署名推讓。而罕東酋長復合詞奏舉,且云兩衛番人,待此以靖。帝納其言,擢加定都指揮僉事,暫掌印務。時土魯番猶據哈密。哈密都督罕慎結赤斤為援,復其城,有詔褒賞。

十九年,鄰番野乜克力來侵,大肆殺掠,赤斤遂殘破。其酋長訴於邊臣,給之慄。又命繕治其城,令流移者復業,赤斤自是不振。然弘治中,阿木郎破哈密,猶用其兵。後許進西征,亦以兵來助。正德八年,土魯番遣將據哈密,遂大掠赤斤,奪其印而去。及彭澤經略,始以印來歸。已,番賊犯肅州與中國為難。赤斤當其衝,益遭蹂躪。部眾不能自存,盡內徙肅州之南山,其城遂空。

嘉靖七年,總督王瓊撫安諸郡,核赤斤之眾僅千餘人。乃授賞卜塔兒子鎖南束為都督,統其部帳。

沙州衛。自赤斤蒙古西行二百里曰苦峪,自苦峪南折而西百九十里曰瓜州,自瓜州而西四百四十里始達沙州。漢燉煌郡西域之境,玉門、陽關並相距不遠。後魏始置沙州,唐因之,後沒於吐蕃。宣宗時,張義潮以州內附,置歸義軍,授節度使。宋入於西夏,元為沙州路。

洪武二十四年,蒙古王子阿魯哥失里遣國公抹台阿巴赤、司徒苦兒蘭等來朝,貢馬及璞玉。永樂二年,酋長困即來、買住率眾來歸。命置沙州衛,授二人指揮使,賜印誥、冠帶、襲衣。已而其部下赤納來附,授都指揮僉事。五年夏,敕甘肅總兵官宋晟曰:「聞赤納本買住部曲,今官居其上,高下失倫,已擢買住為都指揮同知。自今宜詳為審定,毋或失序。」八年擢困即來都指揮僉事,其僚屬進秩者二十人。買住卒,困即來掌衛事,朝貢不絕。二十二年,瓦剌賢義王太平部下來貢,中道為賊所梗,困即來遣人衛送至京。帝嘉之,賚以彩幣,尋進秩都督僉事。

洪熙元年,亦力把里及撒馬兒罕先後入貢,道經哈密地,並為沙州賊邀劫。宣宗怒,命肅州守將費瓛剿之。宣德元年,困即來以歲荒人困,遣使貸穀種百石,秋成還官。帝曰:「番人即吾人,何貸為?」命即予之。尋遣中官張福使其地,賚綵幣。七年又奏旱災,敕於肅州授糧五百石。已而哈烈貢使言道經沙州,為赤斤指揮革古者等剽掠。部議赤斤之人遠至沙州為盜,罪不可貸。帝令困即來察之,敕曰:「彼既為盜,不可復容,宜驅還本土,再犯不宥。」九年遣使奏罕東及西番數肆侵侮,掠取人畜,不獲安居,乞徙察罕舊城耕牧。帝遣敕止之曰:「爾居沙州三十餘年,戶口滋息,畜牧富饒,皆朝廷之力。往年哈密嘗奏爾侵擾,今外侮亦自取。但當循分守職,保境睦鄰,自無外患。何必東遷西徙,徒取勞瘁。」又敕罕東、西番,果侵奪人畜,速還之。明年又為哈密所侵,且懼瓦剌見逼,不能自立。乃率部眾二百餘人走附塞下,陳饑窘狀。詔邊臣發粟濟之,且令議所處置。邊臣請移之苦峪,從之。自是不復還沙州,但遙領其眾而已。

正統元年,西域阿端遣使來貢,為罕東頭目可兒即及西番野人剽奪。困即來奉命往追還其貢物,帝嘉之,擢都督同知。四年,其部下都指揮阿赤不花等一百三十餘家亡入哈密。困即來奉詔索之,不予。朝命忠順王還之,又不予。會遣使冊封其新王,即令使人索還所逃之戶。而哈密僅還都指揮桑哥失力等八十四家,餘仍不遣。是罕東都指揮班麻思結久駐牧沙州不去,赤斤都指揮革古者亦納其叛亡。困即來屢訴於朝,朝廷亦數遣敕詰責,諸部多不奉命。四年八月令人偵瓦剌、哈密事,具得其實以聞。帝喜,降敕獎勵,厚賜之。明年遣使入貢,又報迤北邊事,進其使臣二人官。初,困即來之去沙州也,朝廷命邊將繕治苦峪城,率戍卒助之。六年冬,城成,入朝謝恩,貢駝馬,宴賜遣還。七年率眾侵哈密,獲其人畜以歸。

九年,困即來卒,長子喃哥率其弟克俄羅領占來朝。授喃哥都督僉事,其弟都指揮使,賜敕戒諭。既還,其兄弟乖爭,部眾攜貳。甘肅鎮將任禮等欲乘其窘乏,遷之塞內。而喃哥亦來言,欲居肅州之小缽和寺。禮等遂以十一年秋令都指揮毛哈剌等偕喃哥先赴沙州,撫諭其眾,而親率兵隨其後。比至,喃哥意中變,陰持兩端,其部下多欲奔瓦剌。禮等進兵迫之,遂收其全部入塞,居之甘州,凡二百餘戶,千二百三十餘人,沙州遂空。帝以其迫之而來,情不可測,令禮熟計其便。然自是安居內地,迄無後患。而沙州為罕東酋班麻思結所有。獨喃哥弟鎖南奔不從徙,竄入瓦剌,也先封之為祁王。禮偵知其在罕東,掩襲獲之。廷臣請正法,帝念其父兄恭順,免死,徙東昌。

先是,太宗置哈密、沙州、赤斤、罕東四衛於嘉峪關外,屏蔽西陲。至是,沙州先廢,而諸衛亦漸不能自立,肅州遂多事。

罕東衛,在赤斤蒙古南,嘉峪關西南,漢燉煌郡地也。洪武二十五年,涼國公藍玉追逃寇祁者孫至罕東地,其部眾多竄徙。西寧三剌為書招之,遂相繼來歸。三十年,酋鎖南吉剌思遣使入貢,詔置罕東衛,授指揮僉事。

永樂元年偕其兄答力襲入朝,進指揮使。授答力襲指揮同知,並賜冠帶、鈔幣。自是數入貢。十年,安定衛奏罕東數為盜,掠去民戶三百,復糾西番阻截關隘。帝降敕切責,令還所掠。十六年命中官鄧誠使其地。

洪熙元年遣使以即位諭其指揮同知綽兒加,賜白金、文綺。時官軍征曲先賊,罕東指揮使卻里加從征有功,擢都指揮僉事,賜誥世襲。其指揮那那奏所屬番民千五百,例納差發馬二百五十匹,其人多逃居赤斤,乞招撫復業。帝即命招之,并免所負之馬。宣德元年論從征曲先功,擢綽兒加都指揮同知。初,大軍之討曲先也,安定部內及罕東密羅族人悉驚竄。事定,詔指揮陳通等往招。於是罕東復業者二千四百餘帳,男婦萬七千三百餘人,安定部人亦還衛。

正統四年,罕東、安定合眾侵西番申藏族,掠其馬牛雜畜以萬計。其僧訴於邊將,言畜產一空,歲辦差發馬無從出。帝切責二衛,數其殘忍暴橫、違國法、毒鄰境之罪,令悉歸所掠。又諭僧不限舊制,隨所有入貢。明年冬,綽兒加偕班麻思結共侵哈密,獲老稚百人、馬百匹,牛羊無算。忠順王遣使索之,不予。帝聞,復賜敕戒諭。然番人以剽掠為性,天子即有言,亦不能盡從也。六年夏,綽兒加來貢馬,宴賚還。九年卒,子賞卜兒加嗣職,奏乞齋糧、茶布,命悉予之。十一年進都指揮使。

成化九年,土魯番陷哈密。都督李文西征,罕東以兵來助。後都督罕慎復哈密,亦藉其兵,賜敕獎賚。十八年,其部下掠番族,有侵入河清堡者。都指揮梅琛勒兵追之,奪還男婦五十餘人,馬牛雜畜四千五百有奇。邊臣請討其罪,部臣難之。帝曰:「罕東方聽調協取哈密,未有攜貳之形,奈何因小故遽加以兵。宜諭令悔過,不服,則耀兵威之。」二十二年,邊臣言:「比遣官往哈密,與土魯番使臣家屬四百人偕行。道經罕東,為都督把麻奔等掠去,朝使僅免,乞討之。」帝命遣人往諭,如番人例議和,還所掠物,不從則進兵。

弘治中,土魯番復據哈密。兵部馬文升議直搗其城,召指揮楊翥計之。翥言罕東有間道,不旬日可達哈密,宜出賊不意,從此進兵。文升曰:「如若言,發罕東兵三千前行,我師三千後繼,各持數日乾糧,兼程襲之,若何?」翥稱善。文升以屬巡撫許進,進遣人諭罕東如前策。會罕東失期不至,官軍仍由大路進,賊得遁去。十二年,其部人侵西寧隆奔族,掠去印誥及人畜。兵部請敕都督,宣諭其下,毋匿所掠物,盡歸其主,違命則都督自討,從之。

時土魯番日強,數侵掠鄰境,諸部皆不能支。正德中,蒙古大酋入青海,罕東亦遭蹂躪,其眾益衰。後土魯番復陷哈密,直犯肅州。罕東復殘破,相率求內徙,其城遂棄不守。嘉靖時,總督王瓊安輯諸部,移罕東都指揮枝丹部落於甘州。

罕東左衛,在沙州衛故城,憲宗時始建。初,罕東部人奄章與種族不相能,數仇殺,乃率其眾逃居沙州境。朝廷即許其耕牧,歲納馬於肅州。後部落日蕃,益不受罕東統屬。至其子班麻思結,洪熙時從討曲先有功,賞未之及。宣德七年自陳於朝,即命為罕東衛指揮使,賜敕獎賚。然猶居沙州,不還本衛。十年進都指揮使僉事。

正統四年,沙州衛都督困即來以班麻思結侵居其地,乞遣還。天子如其言,賜敕宣諭,班麻思結不奉命。時赤斤衛指揮鎖合者因殺人遁入沙州地,班麻思結納之。鎖合者又令其子往烏斯藏取毒藥,將還攻赤斤。赤斤都督且旺失加以為言,天子即敕諭班麻思結睦鄰保境,無啟釁端。久之,沙州全部悉內徙,思結遂盡有其地。十四年,甘肅鎮臣任禮等奏,班麻思結潛與瓦剌也先通好,近又與哈密手冓兵,宜令還居本衛。天子再賜敕宣諭,亦不奉命。尋進秩都指揮使。歷景泰、天順朝,朝貢不廢。

成化中,班麻思結卒,孫只克嗣職,部眾益盛。其時,土魯番強,侵據哈密。只克與之接境,患其逼己,欲自為一衛。十五年九月奏請如罕東、赤斤例,立衛賜印,捍禦西陲。兵部言:「近土魯番吞噬哈密,罕東諸衛各不自保,西鄙為之不寧。而赤斤、罕東、苦峪又各懷嫌隙,不相救援。倘沙州更無人統理,勢必為強敵所並,邊方愈多事。宜如所請,即於沙州故城置罕東左衛,令只克仍以都指揮使統治。」從之。二十一年,甘肅守臣言:「北寇屢犯沙州,殺掠人畜。又值歲飢,人思流竄。已發粟五百石,令布種,仍乞人給月糧振之。其酋只克有斬級功,亦乞並敘。」乃擢只克都督僉事,餘報可。

弘治七年,指揮王永言:「先朝建哈密衛,當西域要衝。諸番入貢至此,必令少憩以館穀之,或遭他寇剽掠,則人馬可以接護,柔遠之道可謂至矣。今土魯番竊據其地,久而不退。聞罕東左衛居哈密之南,僅三日程,野乜克力居哈密東北,僅二日程,是皆脣齒之地,利害共之。去歲秋,土魯番遣人至只克所,脅令歸附,只克不從。又殺野乜克力頭目,其部人咸思報怨。宜旌勞二部,令并力合攻,永除厥患,亦以寇攻寇一策也。」章下兵部,不能用。十七年,瓦剌及安定部人大掠沙州人畜。只克不能自存,叩嘉峪關求濟。天子既振給之,復諭二部解仇息爭,不得構兵召釁。

正德四年,只克部內番族有劫掠鄰境者,守臣將剿之。兵部言:「西戎強悍,漢、唐以來不能制。我朝建哈密、赤斤、罕東諸衛,授官賜敕,犬牙相制,不惟斷匈奴右臂,亦以壯西土籓籬。今番人相攻,於我何預,而遽欲兵之。宜敕都督只克,曉諭諸族,悔過息兵。」報可。

只克卒,子乞台嗣。十一年,土魯番復據哈密,以兵脅乞台降附,遂犯肅州。左衛不克自立,相率徙肅州塞內。守臣不能拒,因撫納之。

乞台卒,子日羔嗣。十六年秋入朝,乞賞賚。禮官劾其越例,且投疏不由通政司,請治館伴者罪,從之。

乞台既內徙,其部下帖木哥、土巴二人仍居沙州,服屬土魯番,歲輸婦女、牛馬。會番酋徵求苛急,二人怨。嘉靖七年夏,率部族五千四百人來歸,沙州遂為土魯番所有。

哈梅里,地近甘肅,元諸王兀納失里居之。洪武十三年,都督濮英練兵西涼,請出師略地,開哈梅里之路以通商旅。太祖賜璽書曰:「略地之請,聽爾便宜。然將以謀為本,爾慎毋忽。」英遂進兵。兀納失里懼,遣使納款。明年五月遣回回阿老丁來朝貢馬。詔賜文綺,遣往畏吾兒之地,招諭諸番。二十三年,帝聞兀納失里與別部仇殺,諭甘肅都督宋晟等嚴兵備之。明年遣使請於延安、綏德、平涼、寧夏以馬互市。帝曰:「番人黠而多詐。互市之求,安知非覘我。中國利其馬而不虞其害,所喪必多。宜勿聽。自今至者,悉送京師。」時西域回紇來貢者,多為哈梅里所遏。有從他道來者,又遣兵邀殺之。帝聞之怒。八月命都督僉事劉真偕宋晟督兵討之。真等由涼州西出,乘夜直抵城下,四面圍之。其知院岳山夜縋城降。黎明,兀納失里驅馬三百餘匹,突圍而出。官軍爭取其馬,兀納失里率家屬隨馬後遁去。真等攻破其城,斬豳王別兒怯帖木兒、國公省阿朵爾只等一千四百人,獲王子別列怯部屬千七百三十人,金銀印各一,馬六百三十匹。二十五年遣使貢馬騾請罪。帝納之,賜白金、文綺。


\end{pinyinscope}