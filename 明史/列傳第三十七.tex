\article{列傳第三十七}

\begin{pinyinscope}
○蹇義夏原吉俞士吉李文郁鄒師顏

蹇義,字宜之,巴人,初名瑢。洪武十八年進士。授中書舍人,奏事稱旨。帝問:「汝蹇叔後乎?」瑢頓首不敢對。帝嘉其誠篤,為更名義,手書賜之。滿三載當遷,特命滿九載,曰:「朕且用義。」由是朝夕侍左右,小心敬慎,未嘗忤色。惠帝既即位,推太祖意,超擢吏部右侍郎。是時齊泰、黃子澄當國,外興大師,內改制度,義無所建明。國子博士王紳遺書責之,義不能答。

燕師入,迎附,遷左侍郎。數月,進尚書。時方務反建文之政,所更易者悉罷之。義從容言曰:「損益貴適時宜。前改者固不當,今必欲盡復者,亦未悉當也。」因舉數事陳說本末。帝稱善,從其言。

永樂二年兼太子詹事。帝有所傳諭太子,輒遣義,能委曲導意。帝與太子俱愛重之。七年,帝巡北京,命輔皇太子監國。義熟典故,達治體,軍國事皆倚辦。時舊臣見親用者,戶部尚書夏原吉與義齊名,中外稱曰「蹇、夏」。滿三考,帝親宴二人便殿,褒揚甚至。數奉命兼理他部事,職務填委,處之裕如。十七年以父喪歸,帝及太子皆遣官賜祭。詔起復。十九年,三殿災,敕廷臣二十六人巡行天下。義及給事中馬俊分巡應天諸府,問軍民疾苦,黜文武長吏擾民者數人,條興革數十事奏行之。還治部事。明年,帝北征還,以太子曲宥呂震婿主事張鶴朝參失儀,罪義不匡正,逮義繫錦衣衛獄。又明年春得釋。

仁宗即位,義、原吉皆以元老為中外所信。帝又念義監國時舊勞,尤厚倚之。首進義少保,賜冠服、象笏、玉帶,兼食二祿。歷進少師,賜銀章一,文曰「繩愆糾繆」。已,復賜璽書曰:「曩朕監國,卿以先朝舊臣,日侍左右。兩京肇建,政務方殷,卿勞心焦思,不恤身家,二十餘年,夷險一節。朕承大統,贊襄治理,不懈益恭。朕篤念不忘,茲以已意,創製『蹇忠貞印』賜卿。俾藏於家,傳之後世,知朕君臣共濟艱難,相與有成也。」時惟楊士奇亦得賜「貞一」印及敕。尋命與英國公輔及原吉同監修《太宗實錄》。義視原吉尤重厚,然過於周慎。士奇嘗於帝前謂義曰:「何過慮?」義曰:「恐鹵莽為後憂耳。」帝兩是之。楊榮嘗毀義。帝不直榮。義頓首言:「榮無他。即左右有讒榮者,願陛下慎察。」帝笑曰:「吾固弗信也。」宣宗即位,委寄益重。時方修獻陵,帝欲遵遺詔從儉約,以問義、原吉。二人力贊曰:「聖見高遠,出於至孝,萬世之利也。」帝親為規畫,三月而陵成,宏麗不及長陵,其後諸帝因以為制。迨世宗營永陵,始益崇侈云。

帝征樂安,義、原吉及諸學士皆從,預軍中機務,賜鞍馬甲胄弓劍。及還,賚予甚厚。三年從巡邊還。帝以義、原吉、士奇、榮四人者皆已老,賜璽書曰:「卿等皆祖宗遺老,畀輔朕躬。今黃髮危齒,不宜復典冗劇,傷朝廷優老待賢之禮。可輟所務,朝夕在朕左右討論至理,共寧邦家。官祿悉如舊。」明年,郭璡代為尚書。尋以胡濙言,命義等四人議天下官吏軍民建言章奏。復賜義銀章,文曰「忠厚寬宏」。七年詔有司為義營新第於文明門內。

英宗即位,齋宿得疾。遣醫往視,問所欲言。對曰:「陛下初嗣大寶,望敬守祖宗成憲,始終不渝耳。」遂卒,年七十三。贈太師,謚忠定。

義為人質直孝友,善處僚友間,未嘗一語傷物。士奇常言:「張詠之不飾玩好,傅堯俞之遇人以誠,范景仁之不設城府,義兼有之。」子英,有詩名,以廕為尚寶司丞,歷官太常少卿。

夏原吉,字維喆,其先德興人。父時敏,官湘陰教諭,遂家焉。原吉早孤,力學養母。以鄉薦入太學,選入禁中書制誥。諸生或喧笑,原吉危坐儼然。太祖詗而異之。擢戶部主事。曹務叢脞,處之悉有條理,尚書郁新甚重之。有劉郎中者,忌其能。會新劾諸司怠事者。帝欲宥之,新持不可。帝怒,問:「誰教若?」新頓首曰:「堂後書算生。」帝乃下書算生於獄。劉郎中遂言:「教尚書者,原吉也。」帝曰:「原吉能佐尚書理部事,汝欲陷之耶!」劉郎中與書算生皆棄市。建文初,擢戶部右侍郎。明年充採訪使。巡福建,所過郡邑,核吏治,咨民隱。人皆悅服。久之,移駐蘄州。成祖即位,或執原吉以獻。帝釋之,轉左侍郎。或言原吉建文時用事,不可信。帝不聽,與蹇義同進尚書。偕義等詳定賦役諸制。建白三十餘事,皆簡便易遵守。曰:「行之而難繼者,且重困民,吾不忍也。」浙西大水,有司治不效。永樂元年,命原吉治之。尋命侍郎李文郁為之副,復使僉都御史俞士吉齎水利書賜之。原吉請循禹三江入海故蹟,浚吳淞下流,上接太湖,而度地為閘,以時蓄洩。從之。役十餘萬人。原吉布衣徒步,日夜經畫。盛暑不張蓋,曰:「民勞,吾何忍獨適。」事竣,還京師,言水雖由故道入海,而支流未盡疏洩,非經久計。明年正月,原吉復行,浚白茆塘、劉家河、大黃浦。大理少卿袁復為之副。已,復命陜西參政宋性佐之。九月工畢,水洩,蘇、松農田大利。三年還。其夏,浙西大饑。命原吉率俞士吉、袁復及左通政趙居任往振,發粟三十萬石,給牛種。有請召民佃水退淤田益賦者,原吉馳疏止之。姚廣孝還自浙西,稱原吉曰:「古之遺愛也。」亡何,郁新卒,召還,理部事。首請裁冗食,平賦役;嚴鹽法、錢鈔之禁;清倉場,廣屯種,以給邊蘇民,且便商賈。皆報可。凡中外戶口、府庫、田賦贏縮之數,各以小簡書置懷中,時檢閱之。一日,帝問:「天下錢、穀幾何?」對甚悉,以是益重之。當是時,兵革初定,論「靖難」功臣封賞,分封諸籓,增設武衛百司。已,又發卒八十萬問罪安南、中官造巨艦通海外諸國、大起北都宮闕。供億轉輸以鉅萬萬計,皆取給戶曹。原吉悉心計應之,國用不絀。

六年命督軍民輸材北都,詔以錦衣官校從,治怠事者。原吉慮犯者眾,告戒而後行,人皆感悅。

七年,帝北巡,命兼攝行在禮部、兵部、都察院事。有二指揮冒月廩,帝欲斬之。原吉曰:「非律也,假實為盜,將何以加?」乃止。

八年,帝北征,輔太孫留守北京,總行在九卿事。時諸司草創,每旦,原吉入佐太孫參決庶務。朝退,諸曹郎御史環請事。原吉口答手書,不動聲色。北達行在,南啟監國,京師肅然。帝還,賜鈔幣、鞍馬、牢醴,慰勞有加。尋從還南京,命侍太孫周行鄉落,觀民間疾苦。原吉取齏黍以進,曰:「願殿下食此,知民艱。」九載滿,與蹇義皆宴便殿,帝指二人謂群臣曰:「高皇帝養賢以貽朕。欲觀古名臣,此其人矣。」自是屢侍太孫,往來兩京,在道隨事納忠,多所裨益。

十八年,北京宮室成,使原吉南召太子、太孫。既還,原吉言:「連歲營建,今告成。宜撫流亡,蠲逋負以寬民力。」明年,三殿災,原吉復申前請,亟命所司行之。初以殿災詔求直言,群臣多言都北京非便。帝怒,殺主事蕭儀,曰:「方遷都時,與大臣密議,久而後定,非輕舉也。」言者因劾大臣。帝命跪午門外質辨。大臣爭詈言者,原吉獨奏曰:「彼應詔無罪。臣等備員大臣,不能協贊大計,罪在臣等。」帝意解,兩宥之。或尤原吉背初議。曰:「吾輩歷事久,言雖失,幸上憐之。若言官得罪,所損不細矣。」眾始歎服。

原吉雖居戶部,國家大事輒令詳議。帝每御便殿闕門,召語移時,左右莫得聞。退則恂恂若無預者。交阯平,帝問:「遷官與賞孰便?」對曰:「賞費於一時,有限;遷官為後日費,無窮也。」從之。西域法王來朝,帝欲郊勞,原吉不可。及法王入,原吉見,不拜。帝笑曰:「卿欲效韓愈耶?」山東唐賽兒反,事平,俘脅從者三千餘人至。原吉請於帝,悉原之。谷王㭎叛,帝疑長沙有通謀者。原吉以百口保之,乃得寢。

十九年冬,帝將大舉征沙漠。命原吉與禮部尚書呂震、兵部尚書方賓、工部尚書吳中等議,皆言兵不當出。未奏,會帝召賓,賓力言軍興費乏,帝不懌。召原吉問邊儲多寡,對曰:「比年師出無功,軍馬儲蓄十喪八九,災眚迭作,內外俱疲。況聖躬少安,尚須調護,乞遣將往征,勿勞車駕。」帝怒,立命原吉出理開平糧儲。而吳中入對如賓言,帝益怒。召原吉繫之內官監,并繫大理丞鄒師顏,以嘗署戶部也。賓懼自殺。遂并籍原吉家,自賜鈔外,惟布衣瓦器。明年北征,以糧盡引還。已,復連歲出塞,皆不見敵。還至榆木川,帝不豫,顧左右曰:「夏原吉愛我。」崩聞至之三日,太子走繫所,呼原吉,哭而告之。原吉伏地哭,不能起。太子令出獄,與議喪禮,復問赦詔所宜。對以振饑、省賦役、罷西洋取寶船及雲南、交阯採辦諸道金銀課。悉從之。

仁宗即位,復其官。方原吉在獄,有母喪,至是乞歸終制。帝曰:「卿老臣,當與朕共濟艱難。卿有喪,朕獨無喪乎?」厚賜之,令家人護喪,馳傳歸葬,有司治喪事。原吉不敢復言。尋加太子少傅。呂震以太子少師班原吉上,帝命鴻臚引震列其下。進少保,兼太子少傅、尚書如故,食三祿。原吉固辭,乃聽辭太子少傅祿。賜「繩愆糾繆」銀章,建第於兩京。

已而仁宗崩,太子至自南京。原吉奉遺詔迎於盧溝橋。宣宗即位,以舊輔益親重。明年,漢王高煦反,亦以「靖難」為辭,移檄罪狀諸大臣,以原吉為首。帝夜召諸臣議。楊榮首勸帝親征。帝難之。原吉曰:「獨不見李景隆已事耶?臣昨見所遣將,命下即色變,臨事可知矣。且兵貴神速,卷甲趨之,所謂先人有奪人之心也。榮策善。」帝意遂決。師還,賚予加等,賜閽者三人。原吉以無功辭。不聽。

三年,從北巡。帝取原吉橐糗嘗之,笑曰:「何惡也?」對曰;「軍中猶有餒者。」帝命賜以大官之饌,且犒將士。從閱武兔兒山,帝怒諸將慢,褫其衣。原吉曰:「將帥,國爪牙,奈何凍而斃之?」反覆力諫。帝曰:「為卿釋之。」再與蹇義同賜銀印,文曰:「含弘貞靖。」帝雅善繪事,嘗親畫《壽星圖》以賜。其他圖畫、服食、器用、銀幣、玩好之賜,無虛日。五年正月,兩朝實錄成,復賜金幣、鞍馬。旦入謝,歸而卒,年六十五。贈太師,謚忠靖。敕戶部復其家,世世無所與。

原吉有雅景,人莫能測其際。同列有善,即採納之。或有小過,必為之掩覆。吏污所服金織賜衣。原吉曰:「勿怖,污可浣也。」又有污精微文書者,吏叩頭請死。原吉不問,自入朝引咎,帝命易之。呂震嘗傾原吉。震為子乞官,原吉以震在「靖難」時有守城功,為之請。平江伯陳瑄初亦惡原吉,原吉顧時時稱瑄才。或問原吉:「量可學乎?」曰:「吾幼時,有犯未嘗不怒。始忍於色,中忍於心,久則無可忍矣。」嘗夜閱爰書,撫案而歎,筆欲下輒止。妻問之。曰:「此歲終大辟奏也。」與同列飲他所,夜歸值雪,過禁門,有欲不下者,原吉曰:「君子不以冥冥墮行。」其慎如此。

原吉與義皆起家太祖時。義秉銓政,原吉筦度支,皆二十七年,名位先於三楊。仁、宣之世,外兼臺省,內參館閣,與三楊同心輔政。義善謀,榮善斷,而原吉與士奇尤持大體,有古大臣風烈。

子瑄,以廕為尚寶司丞。喜談兵。景泰時,數上章言兵事,有沮者,不獲用。終南京太常少卿。

俞士吉,字用貞,象山人。建文中,為袞州訓導。上書言時政,擢御史。出按鳳陽、徽州及湖廣,能辨釋冤獄。成祖即位,進僉都御史。奉詔以水利書賜原吉,因留督浙西農政。湖州逋糧至六十萬石,同事者欲減其數以聞。士吉曰:「欺君病民,吾不為也。」具以實奏,悉得免。尋為都御史陳瑛所劾,與大理少卿袁復同繫獄。復死獄中,士吉謫為事官,治水蘇、松。既而復職,還上《聖孝瑞應頌》。帝曰:「爾為大臣,不言民間利病,乃獻諛耶!」擲還之。宣德初,仕至南京刑部侍郎,致仕。

李文郁,襄陽人。永樂初,以戶部侍郎副原吉治水有勞。後坐事謫遼東二十年。仁宗即位,召還,為南京通政參議,致仕。

鄒師顏,宣都人。永樂初,為江西參政,坐事免。尋以薦擢御史,有直聲。遷大理丞,署戶部。與原吉同下獄。仁宗立,釋為禮部侍郎。省墓歸,還至通州,卒,貧不能歸葬。尚書呂震聞於朝,宣宗命驛舟送之。詔京官卒者,皆給驛,著為令。

贊曰:《書》曰「敷求哲人,俾輔於爾後嗣」。蹇義、夏原吉自筮仕之初,即以誠篤幹濟受知太祖,至成祖,益任以繁劇。而二人實能通達政體,諳練章程,稱股肱之任。仁、宣繼體,委寄優隆,同德協心,匡翼令主。用使吏治修明,民風和樂,成績懋著,蔚為宗臣。樹人之效,遠矣哉。


\end{pinyinscope}