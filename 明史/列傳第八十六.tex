\article{列傳第八十六}

\begin{pinyinscope}
楊一清王瓊彭澤毛伯溫汪文盛鮑象賢翁萬達

楊一清,字應寧,其先雲南安寧人。父景,以化州同知致仕,攜之居巴陵。少能文,以奇童薦為翰林秀才。憲宗命內閣擇師教之。年十四舉鄉試,登成化八年進士。父喪,葬丹徒,遂家焉。服除,授中書舍人。久之,遷山西按察僉事,以副使督學陜西。一清貌寢而性警敏,好談經濟大略。在陜八年,以其暇究邊事甚悉。入為太常寺少卿,進南京太常寺卿。

弘治十五年用劉大夏薦,擢都察院左副都御史,督理陜西馬政。西番故饒馬,而仰給中國茶飲以去疾。太祖著令,以蜀茶易番馬,資軍中用。久而浸弛,奸人多挾私茶闌出為利,番馬不時至。一清嚴為禁,盡籠茶利於官,以服致諸番,番馬大集。會寇大入花馬池,帝命一清巡撫陜西,仍督馬政。甫受事,寇已退。乃選卒練兵,創平虜、紅古二城以援固原;築垣瀕河以捍靖虜;劾罷貪庸總兵武安侯鄭宏;裁鎮守中官冗費,軍紀肅然。武宗初立,寇數萬騎抵固原,總兵曹雄軍隔絕不相聞。一清帥輕騎自平涼晝夜行,抵雄軍,為之節度,多張疑兵脅寇,寇移犯隆德。一清夜發火炮,響應山谷間。寇疑大兵至,遁出塞。一清以延綏、寧夏、甘肅有警不相援,患無所統攝,請遣大臣兼領之。大夏請即命一清總制三鎮軍務。

尋進右都御史。一清遂建議修邊,其略曰:

陜西各邊,延綏據險,寧夏、甘肅扼河山,惟花馬池至靈州地寬延,城堡復疏。寇毀牆入,則固原、慶陽、平涼、鞏昌皆受患。成化初,寧夏巡撫徐廷璋築邊牆綿亙二百餘里。在延綏者,餘子俊修之甚固。由是,寇不入套二十餘年。後邊備疏,牆塹日夷。弘治末至今,寇連歲侵略。都御史史琳請於花馬池、韋州設營衛,總制尚書秦紘僅修四五小堡及靖虜至環慶治塹七百里,謂可無患。不一二年,寇復深入。是紘所修不足捍敵。臣久官陜西,頗諳形勢。寇動稱數萬,往來倏忽。未至,徵兵多擾費;既至,召援輒後時。欲戰,則彼不來;持久,則我師坐老。臣以為防邊之策,大要有四:修浚牆塹,以固邊防;增設衛所,以壯邊兵;經理靈、夏,以安內附;整飭韋州,以遏外侵。

今河套即周朔方,漢定襄,赫連勃勃統萬城也。唐張仁愿築三受降城,置烽堠千八百所,突厥不敢踰山牧馬。古之舉大事者,未嘗不勞於先,逸於後。夫受降據三面險,當千里之蔽。國初舍受降而衛東勝,已失一面之險。其後又輟東勝以就延綏,則以一面而遮千餘里之衝,遂使河套沃壤為寇巢穴。深山大河,勢乃在彼,而寧夏外險反南備河。此邊患所以相尋而不可解也。誠宜復守東勝,因河為固,東接大同,西屬寧夏,使河套方千里之地,歸我耕牧,屯田數百萬畝,省內地轉輸,策之上也。如或不能,及今增築防邊,敵來有以待之,猶愈無策。

因條具便宜:延綏安邊營石澇池至橫城三百里,宜設墩臺九百座,暖譙九百間,守軍四千五百人;石澇池至定邊營百六十三里,平衍宜牆者百三十一里,險崖峻阜可鏟削者三十二里,宜為墩臺,連接寧夏東路;花馬池無險,敵至仰客兵,宜置衛;興武營守禦所兵不足,宜召募;自環慶以西至寧州,宜增兵備一人;橫城以北,黃河南岸有墩三十六,宜修復。帝可其議。大發帑金數十萬,使一清築牆。而劉瑾憾一清不附己,一清遂引疾歸。其成者,在要害間僅四十里。瑾誣一清冒破邊費,逮下錦衣獄。大學士李東陽、王鏊力救得解。仍致仕歸,先後罰米六百石。

安化王寘鐇反。詔起一清總制軍務,與總兵官神英西討,中官張永監其軍。未至,一清故部將仇鉞已捕執之。一清馳至鎮,宣布德意。張永旋亦至,一清與結納,相得甚歡。知永與瑾有隙,乘間扼腕言曰:「賴公力定反側。然此易除也,如國家內患何。」永曰:「何謂也?」一清遂促席畫掌作「瑾」字。永難之曰:「是家晨夕上前,枝附根據,耳目廣矣。」一清慷慨曰:「公亦上信臣,討賊不付他人而付公,意可知。今功成奏捷,請間論軍事,因發瑾奸,極陳海內愁怨,懼變起心腹。上英武,必聽公誅瑾。瑾誅,公益柄用,悉矯前弊,收天下心。呂強、張承業暨公,千載三人耳。」永曰:「脫不濟,奈何?」一清曰:「言出於公必濟。萬一不信,公頓首據地泣,請死上前,剖心以明不妄,上必為公動。茍得請,即行事,毋須臾緩。」於是永勃然起曰:「嗟乎,老奴何惜餘年不以報主哉!」竟如一清策誅瑾。永以是德一清,左右之,得召還,拜戶部尚書。論功,加太子少保,賜金幣。尋改吏部。

一清於時政最通練,而性闊大。愛樂賢士大夫,與共功名。凡為瑾所構陷者,率見甄錄。朝有所知,夕即登薦,門生遍天下。嘗再帥關中,起偏裨至大將封侯者,累累然不絕。饋謝有所入,緣手即散之。大盜躪中原,一清疏請命將調兵。前後凡數上,皆報可。盜平,加少保、太子太保,廕錦衣百戶。再推內閣,不用。用尚書靳貴,而進一清少傅、太子太傅。給事中王昂論選法幣,指一清植私黨,帝為謫昂。一清更申救,優旨報聞。乾清宮災,詔求直言。一清上書言視朝太遲,享祀太慢,西內創梵宇,禁中宿邊兵,畿內皇店之害,江南織造之擾。因引疾乞歸,帝慰留之。大學士楊廷和憂去,命一清兼武英殿大學士入參機務。

張永尋得罪罷,而義子錢寧用事。寧故善一清,有構之者因蓄怨。會災異,一清自劾,極陳時政,中有「狂言惑聖聽,匹夫搖國是,禁廷雜介胄之夫,京師無籓籬之託」語,譏切近倖,帝弗省。寧與江彬輩聞之,大怒。使優人於帝前為蜚語,刺譏一清。時有考察罷官者,嗾武學生朱大周訐一清陰事,而以寧為內主。給事御史周金、陳軾等交章劾大周妄言,請究主使,帝不聽。一清乃力請骸骨歸,賜敕褒諭,給夫廩如制。帝南征,幸一清第,樂飲兩晝夜,賦詩賡和以十數。一清從容諷止,帝遂不為江浙行。

世宗為世子時,獻王嘗言楚有三傑:劉大夏、李東陽及一清也,心識之。及即位,廷臣交薦一清,乃遣官賜金幣存問,諭以宣召期,趣使有言。一清陳謝,特予一子官中書舍人。嘉靖三年十二月戊午詔一清以少傅、太子太傅改兵部尚書、左都御史,總制陜西三邊軍務。故相行邊,自一清始。溫詔褒美,比之郭子儀。一清至是三為總制,部曲皆踴躍喜。亦不剌竄西海,為西寧洮河害,金獻民言撫便,獨一清請剿。土魯番求貢,陳九疇欲絕之,一清則請撫。時帥諸將肄習行陣,嘗曰:「無事時當如有事隄防,有事時當如無事鎮靜。」

會張璁等力排費宏,御史吉棠因請還一清內閣。給事中章僑、御史侯秩等爭之。帝謫秩官,召一清為吏部尚書、武英殿大學士。既入見,加少師,仍兼太子太傅,非故事也。亡何,《獻皇帝實錄》成,加太子太師、謹身殿大學士。一清以不預纂修辭,不許。王憲奏捷,推功一清,加特進左柱國、華蓋殿大學士。費宏已去,一清遂為首輔。帝賜銀章二,曰「耆德忠正」,曰「繩愆糾違」,令密封言事。與張璁論張永前功,起為提督團營。給事中陸粲請增築邊牆,推明一清曩時議,一清因力從臾之。帝為發帑金,命侍郎王廷相往,然久之亦竟止。《明倫大典》成,加正一品俸。

初,「大禮」議起,一清方家居,見張璁疏,寓書門人喬宇曰:「張生此議,聖人復起,不能易也。」又勸席書早赴召,以定大議。璁等既驟顯,頗引一清。帝亦以一清老臣,恩禮加渥。免常朝日講侍班,朔望朝參,令晨初始入閣視事。御書、和章及金幣、牢醴之賜甚渥。所言邊事、國計,大小無不傾聽。

璁與桂萼既攻去費宏,意一清必援己。一清顧請召謝遷,心怨之。遷未至,璁已入內閣,多所更建。一清引故事稍裁抑,其黨積不平。錦衣聶能遷訐璁,璁欲置之死,一清不可。璁怒,上疏陰詆一清,又嗾黃綰排之甚力。一清疏辨,言璁以能遷故排己,且傍及璁他語。因乞骸骨。帝為兩解之。一清又因災變請戒飭百官和衷,復乞宥議禮諸臣罪,璁益憾。柱萼入內閣,亦不相能。一清屢求去,且言:「今持論者尚紛更,臣獨主安靜;尚刻核,臣獨主寬平。用是多齟齬,願避賢者路。」帝復溫旨褒之。而給事中王準、陸粲發璁、萼招權納賄狀,帝立罷璁、萼,且暴其罪。其黨霍韜攘臂曰:「張、桂行,勢且及我。」遂上疏力攻一清,言其受張永、蕭敬賄。一清再疏辨,乞罷。帝雖慰留之,而璁復召還,韜攻益急,且言法司承一清風指,構成萼罪。帝果怒,令法司會廷臣雜議。出刑部尚書周倫於南京,以侍郎許贊代。讚乃實韜言,請削一清籍。帝令一清自陳。璁乃三上密疏,引一清贊禮功,乞賜寬假,實以堅帝意俾之去。帝果允致仕,馳驛歸,仍賜金幣。明年,璁等構朱繼宗獄,坐一清受張永弟容金錢,為永誌墓,又與容世錦衣指揮,遂落職閒住。一清大恨曰:「老矣,乃為孺子所賣!」疽發背死。遺疏言身被污蔑,死且不瞑,帝令釋贓罪不問。後數年復故官。久之,贈太保,謚文襄。

一清生而隱宮,貌寺人,無子。博學善權變,尤曉暢邊事。羽書旁午,一夕占十疏,悉中機宜。人或訾己,反薦揚之。惟晚與璁、萼異,為所軋,不獲以恩禮終。然其才一時無兩,或比之姚崇云。

王瓊,字德華,太原人。成化二十年進士。授工部主事,進郎中。出治漕河三年,臚其事為志。繼者按稽之,不爽毫髮,由是以敏練稱。改戶部,歷河南右布政使。正德元年,擢右副都御史督漕運。明年入為戶部右侍郎。衡府有賜地,蕪不可耕,勒民出租以為常,王反誣民趙賢等侵據。瓊往按,奪旁近民地予之,賢等戍邊,民多怨者。三年春,廷推吏部侍郎,前後六人,皆不允。最後以瓊上,許之。坐任戶部時邊臣借太倉銀未償,所司奏遲,尚書顧佐奪俸,而瓊改南京。已,復改戶部。八年進尚書。

瓊為人有心計,善鉤校。為郎時悉錄故牘條例,盡得其斂散盈縮狀。及為尚書,益明習國計。邊帥請芻糗,則屈指計某倉、某場CQ糧草幾何;諸郡歲輸、邊卒歲採秋青幾何,曰:「足矣。重索妄也。」人益以瓊為才。

十年代陸完為兵部尚書。時四方盜起,將士以首功進秩。瓊言:「此嬴秦弊政。行之邊方猶可,未有內地而論首功者。今江西、四川妄殺平民千萬,縱賊貽禍,皆此議所致。自今內地征討,惟以蕩平為功,不計首級。」從之。帝時遠遊塞外,經歲不還,近畿盜竊發。瓊請於河間設總兵一人,大名、武定各設兵備副使一人,責以平賊,而檄順天、保定兩巡撫,嚴要害為外防,集遼東、延綏士馬於行在,以護軍駕。中外恃以無恐。孝豐賊湯麻九反,有司請發兵剿。瓊請密敕勘糧都御史許廷光,出不意擒之,無一脫者。四方捷奏上,多推功瓊,數受廕賚,累加至少師兼太子太師,子錦衣世千戶。及營建乾清宮,又廕錦衣千戶者二,寵遇冠諸尚書。十四年,寧王宸濠反。瓊請敕南和伯方壽祥督操江兵防南都,南贛巡撫王守仁、湖廣巡撫秦金各率所部趨南昌,應天巡撫李充嗣鎮京口,淮揚巡撫叢蘭扼儀真。奏上,帝意欲親征,持三日不下。大學士楊廷和趣之,竟下親征詔,命瓊與廷和等居守。先是,瓊用王守仁撫南、贛,假便宜提督軍務。比宸濠反,書聞,舉朝惴惴。瓊曰:「諸君勿憂,吾用王伯安贛州,正為今日,賊旦夕擒耳。」未幾,果如其言。

瓊才高,善結納。厚事錢寧、江彬等,因得自展,所奏請輒行。其能為功於兵部者,亦彬等力也。陸完敗,代為吏部尚書。瓊忌彭澤平流賊,聲望出己上,構於錢寧,中澤危法。又陷雲南巡撫范鏞、甘肅巡撫李昆、副使陳九疇於獄,中外多畏瓊。而大學士廷和亦以瓊所誅賞,多取中旨,不關內閣,弗能堪。明年,世宗入繼,言官交劾瓊,繫都察院獄。瓊力訐廷和,帝愈不直瓊,下廷臣雜議。坐交結近侍律論死,命戍莊浪。瓊復訴年老,改戍綏德。

張璁、桂萼、霍韜用事,以瓊與廷和仇,首薦之,不納。至嘉靖六年有邊警,萼力請用瓊,不果。帝亦憫瓊老病,令還籍為民。御史胡松因劾萼謫外任,其同官周在請宥松,並下詔獄。萼復言瓊前攻廷和,故廷臣群起排之。帝乃命復瓊尚書待用。明年遂以兵部尚書兼右都御史代王憲督陜西三邊軍務。土魯番據哈密,廷議閉關絕其貢,四年矣。至是,其將牙木蘭為酋速檀滿速兒所疑,率從二千求內屬。沙州番人帖木哥、土巴等,素為土魯番役屬者,苦其徵求,亦率五千餘人入附。番人來寇,連為參將雲昌等所敗。其引瓦剌寇肅州者,遊擊彭濬擊退之。賊既失援,又數失利,乃獻還哈密。求通貢,乞歸羈留使臣,而語多謾。瓊奏乞撫納,帝從兵部尚書王時中議,如瓊請。霍韜難之,瓊再疏請詔還番使,通貢如故。自是西域復定,而北寇常為邊患。初入犯莊浪,瓊部諸將遮擊之,斬數十級。俄由紅城子入,殺部餉主簿張文明。明年以數萬騎寇寧夏。已又犯靈州,瓊督遊擊梁震等邀斬七十餘人。其秋,集諸道精卒三萬,按行塞下。寇聞,徙帳遠遁。諸軍分道出,縱野燒,耀兵而還。

先是,南京給事中邱九仞劾瓊,帝慰留之。及璁、萼罷政,諸劾璁、萼黨者咸首瓊,乃令致仕。俄寢前詔,遣慰諭。會番大掠臨洮,瓊集兵討若籠、板爾諸族,焚其巢,斬首三百六十,撫降七十餘族。錄功,加太子太保。瓊在邊,戎備甚飭。寇嘗入山西得利,踰歲復獵境上,陽欲東,瓊令備其西。寇果入,大敗之。諸番蕩平,西陲益靖。甘肅軍民素苦土魯番侵暴,恐瓊去,相率乞守臣奏留。於是巡撫唐澤、巡按胡明善具陳其功,乞如軍民請。優詔獎之。

初,帝惡楊廷和,疑廷臣悉其黨,故連用桂萼、方獻夫為吏部。及獻夫去,帝不欲授他人,久不補。至十年冬,遣行人齎敕召瓊為吏部尚書。南京御史馬等十人力詆瓊先朝遺奸。帝大怒,盡逮等下詔獄,慰諭瓊。未凡,等亦還職。花馬池有警,兵部尚書王憲請發兵。瓊言花馬池備嚴,寇不能入,大軍至,且先退,徒耗中國。憲竟發六千人,比至彰德,寇果遁。明年秋卒官。贈太師,謚恭襄。是年,彭澤已先卒矣。

當正、嘉間,澤、瓊並有才略,相中傷不已,亦迭為進退。而瓊險忮,公論尤不予。然在本兵時功多。而其督三邊也,人以比楊一清云。

彭澤,字濟物,蘭州人。幼學於外祖段堅,有志節。會試二場畢,聞母病,徑歸,母病亦已。登弘治三年進士,授工部主事,歷刑部郎中。勢豪殺人,澤置之辟。中貴為祈免,執不聽。出為徽州知府。澤將遣女,治漆器數十,使吏送其家。澤父大怒,趣焚之,徒步詣徽。澤驚出迓,自吏負其裝。父怒曰:「吾負此數千里,汝不能負數步耶?」入,杖澤堂下。杖已,持裝徑去。澤益痛砥礪。政最,人以方前守孫遇。遇見《循吏傳》中。父喪歸。

正德初,起知真定。閹人數撓禁,澤治一棺於廳事,以死怵之,其人不敢逞。遷浙江副使,歷河南按察使,所至以威猛稱。擢右僉都御史,巡撫遼東。進右副都御史,改保定。未赴,而劉惠、趙鐩等亂河南,命澤與咸寧伯仇鉞提督軍務討之。陳便宜十一事,厚賞峻罰,以激勸將吏。澤體幹修偉,腰帶十二圍,大音聲,與人語若叱吒。始至,大陳軍容,引見諸將校,責以畏縮當死。諸將校股栗伏罪,良久乃釋。遂下令鼓行薄賊,大小數十戰,連破之。甫四月,賊盡平,語詳《鉞傳》。錄功,進右都御史、太子少保,廕子錦衣世百戶。尋代洪鐘總督川、陜諸軍,討四川賊。時鄢本恕、藍廷瑞、廖惠、曹甫已平,惟廖麻子、喻思俸猖獗如故。澤偕總兵官時源數敗賊,部將閻勛追擒麻子於劍州。思俸竄通、巴間,勢復振。澤督諸軍圍之,卒就擒。澤遂移漢中,請班師。未報,而內江、榮昌賊復熾,澤又移師討平之。且平成都亂卒之執知州、指揮者。請班師益力,詔暫留保寧鎮撫。進左都御史、太子太保,廕子如初。澤復請還者再,乃召還。未行,會土魯番據哈密,執忠順王速檀拜牙郎,以其印去,投謾書甘肅,要索金幣。總制鄧璋、甘肅巡撫趙鑑以聞,請遣大臣經略。大學士楊廷和等共薦澤。澤久在兵間,厭之。以鄉土為辭,且引疾,推璋及咸寧侯鉞可任。帝優詔慰勉,乃行。

澤材武知兵,然性疏闊負氣。經略哈密事頗不當,錢寧、王瓊等交齮齕之,遂因此得罪。澤至甘州,土魯番方寇赤斤、苦峪諸衛,遣使索金幣,請還哈密。澤以番人可利啖也,與鑑謀,遣哈密都督寫亦虎仙以幣二千、銀酒槍一賂之,令還哈密城印。未得報,輒奏事平,乞骸骨。召還理院事。巡按御史馮時雍言城未歸,澤不宜遽召。不納。

初,兵部缺尚書,廷臣共推澤,而王瓊得之,且陰阻澤。言官多劾瓊者,由是有隙。澤又使酒常凌瓊,瓊愈欲傾之。澤時時罵錢寧,瓊以語寧,寧未信。瓊乃邀澤飲,匿寧所親屏間,挑澤醉罵使聞之,寧果大怒。會寇大入宣府,廷議以許泰將兵,澤總制東西兩邊軍務。及詔下,罷泰不遣,又不命澤總制,獨令提督兩遊擊兵六千人以行,意以困澤。澤言:「臣文臣,摧鋒陷陣非臣所能獨任。」瓊乃奏遣成國公朱輔。會寇遁,澤還理院事。

寫亦虎仙者,素桀黠。雖居肅州,陰通土魯番酋速檀滿速兒,為之耳目,據城奪印皆其謀。澤初不知而遣之。滿速兒以城印來歸,留速檀拜牙郎如故。虎仙復啖使入寇,曰:「肅州可得也。」滿速兒悅,使其婿馬黑木隨入貢,以覘虛實,且徵賄。澤已還,鑒亦遷去,李昆代巡撫,慮他變,質其使於甘州,而驅虎仙出關。虎仙懼弗去。滿速兒聞之怒,復取哈密,分兵據沙州,自率萬騎寇嘉峪關。遊擊芮寧與參將蔣存禮禦之。寧以七百人先遇寇沙子壩。寇圍寧,而分兵綴存禮軍。寧軍盡沒,遂墮城堡,縱殺掠。詔澤提督三邊軍務往禦。會副使陳九疇繫其使失拜煙答及虎仙等,內應絕,乃復求和。澤兵遂罷。尋乞骸骨歸,馳驛給夫廩如制。

澤既去,瓊追論嘉峪之敗,請窮詰增幣者主名。錢寧從中下其事,大學士梁儲等持之,乃已。會失拜煙答子訟父冤,下法司議,釋寫亦虎仙等。瓊因請遣給事御史勘失事狀,還報無所引。瓊遂劾澤妄增金幣,遺書議和,失信啟釁,辱國喪師;昆、九疇俱宜罪。詔斥澤為民,昆、九疇逮訊。昆謫官,九疇除名。

世宗入繼,錢寧敗,瓊亦得罪。御史楊秉中請召澤,遂即家起兵部尚書、太子太保。昆、九疇亦復官。部事積壞久,澤核功罪,杜干請,兵政一新。初,正德時,廷臣建白戎務奉俞旨者,多廢格。澤請臚列成書,次第修舉。又請敕九邊守臣,策防禦方略,毋畫境自保。鎮、巡居中調度,毋相牽制。諸邊各以農隙築牆浚濠,修墩臺,飭屯堡,為經久計。內地盜甫息,敕守臣練卒伍,立保甲,懲匿盜不舉者。且撫西南諸苗蠻,申海禁,汰京軍老弱。帝咸嘉納。詔遣中官楊金、鄭斌、安川更代鎮守,復令張弼、劉瑤守涼州、居庸。澤持不可,罷弗遣。四川巡撫胡世寧劾分守中官趙欽,澤因請盡罷諸鎮守。時雖不從,其後鎮守竟罷。

嘉靖元年,澤言天下軍官,部皆有帖黃籍,用以黜陟,錦衣獨無,於是置籍如諸衛。錦衣千戶劉瓚等,詔書黜汰,復求還官,司禮中官蕭敬請補監局工匠千五百人,澤皆持不可,帝並從之。帝將授外戚蔣泰等五人為錦衣,澤爭,不納。

在部多所執持。會御史史道以訐楊廷和下獄,澤復劾道。帝因諭言官,惟大奸及機密事專疏奏,餘只具公疏,毋挾私中傷善類。詔下,給事御史交章劾澤阻言路,壞祖宗法。帝乃從吏部言,停前諭。澤不自安,累疏乞休。言者復交劾之,乃加少保,賜敕乘傳歸。錦衣百戶王邦奇憾澤嘗抑己,上書言哈密失國,由澤賂番求和所致,語侵楊廷和、陳九疇等。張璁、桂萼方疾廷和,遂逮九疇廷訊,戍邊。澤復奪官為民,家居鬱鬱以卒。

總制尚書唐龍言:「澤孝友廉直,先後討平群盜,功在盟府。陛下起之田間,俾掌邦政。澤孜孜奉國,復為讒言構罷。今歿已五年,所遺二妾,衣食不給。請核澤往勞,復官加恤,以作忠臣之氣。」不從。隆慶初,復官,謚襄毅。

毛伯溫,字汝厲,吉水人。祖超,廣西知府。伯溫登正德三年進士,授紹興府推官。擢御史,巡按福建、河南。世宗即位,中官張銳、張忠等論死,其黨蕭敬、韋霦陰緩之。伯溫請並誅敬、霦,中官為屏氣。嘉靖初,遷大理寺丞。擢右僉都御史,巡撫寧夏。李福達獄起,坐為大理時失入,褫職歸。用薦起故官,撫山西,移順天,皆未赴。改理院事,進左副都御史。為趙府宗人祐椋所訐,解官候勘。已,復褫職。

十五年冬,皇嗣生,將頒詔外國。禮部尚書夏言以安南久失朝貢,不當遣使,請討之。遂起伯溫右都御史,與咸寧侯仇鸞治兵待命。以父喪辭,不許。明年五月至京,上方略六事。會安南世孫黎寧遣陪臣鄭惟僚等愬莫登庸弒逆,請興師復仇。帝疑其不實,命暫緩師,敕兩廣、雲南守臣勘報,而命伯溫協理院事。御史何維柏請聽伯溫終制,不許。伯溫引疾不出,至禫除始起視事。其冬遷工部尚書。十七年春,黔國公沐朝輔等以登庸降表至,請宥罪許貢。先是,雲南巡撫汪文盛奏登庸聞發兵進討,遣使潛覘。帝已敕遵前詔進兵,文盛又納安南降人武文淵策,具言登庸可破狀,復傳檄安南令奉表獻地。及是,下朝輔奏付廷議,僉言不可許。乃改伯溫兵部尚書兼右都御史,剋期啟行。帝以用兵事重,無必討意,特欲威服之。而兵部尚書張瓚無所畫,視帝意為可否。朝論多主不當興師,顧不敢顯諫。制下數月,兩廣總督侍郎張經以用兵方略上,且言須兵三十萬,餉百六十萬石。欽州知州林希元則極言登庸易取,請即日出師。瓚不敢決,復請廷議。議上無成策,帝不懌,讓瓚,師復止。命伯溫仍協理院事。

明年二月,帝幸承天。詔伯溫總督宣、大、山西軍務。俄選宮僚,加兼太子賓客。大同所轄鎮邊、鎮川、弘賜、鎮河、鎮虜五堡,相距二百餘里,極邊近賊帳。自巡撫張文錦以築堡致亂,後無敢議修者。伯溫曰:「變所由生,以任用匪人,非建議謬也。」卒營之。募軍三千防守,給以閒田,永除其賦。邊防賴焉。錄功,加太子少保。

是時登庸懼討,數上表乞降。帝亦欲因撫之,遣侍郎黃綰招諭。綰多所要求,帝怒,罷綰。再下廷議,咸言當討,帝從之。閏七月命伯溫、鸞南征。文武三品以下不用命者,許軍令從事。伯溫等至廣西,會總督張經,總兵官安遠侯柳珣,參政翁萬達、張岳等議,征兩廣、福建、湖廣狼土官兵凡十二萬五千餘人,分三哨,自憑祥、龍峒、思陵州入,而以奇兵二為聲援。檄雲南巡撫汪文盛帥兵駐蓮花灘,亦分三道進。部署已定,會鸞有罪召還,即以珣代。十九年秋,伯溫等進駐南寧。檄安南臣民,諭以天朝興滅繼絕之義,罪止登庸父子,舉郡縣降者以其地授之。懸重購購登庸父子,而宣諭登庸籍土地、人民納款,即如詔書宥罪。登庸大懼,遣使詣萬達乞降,詞甚哀。萬達送之伯溫所。伯溫承制許之,宣天子恩威,納其圖籍,並所還欽州四峒地。權令還國聽命。馳疏以聞,帝大悅。詔改安南國為安南都統使司,以登庸為都統使,世襲,置十三宣撫司,令自署置。伯溫受命歲餘,不發一矢,而安南定,由帝本不欲用兵故也。論功,加太子太保。

二十一年正月還朝,復理院事。邊關數有警,伯溫請築京師外城。帝已報可,給事中劉養直言,廟工方興,物力難繼,乃命暫止。其年十月,張瓚卒,伯溫代為兵部。瓚貪黷,在部八年,戎備盡墮。伯溫會廷臣議上防邊二十四事,軍令一新。言官建議,請核實新軍、京軍及內府力士、匠役,以裕國儲。伯溫因上冗濫當革者二十餘條,凡錦衣、騰驤諸衛,御馬、內官、尚膳諸監,素為中貴盤踞者,盡在革中。帝稱善,立命清汰。宿弊頗釐,而左右近習多不悅。

二十三年秋,順天巡撫朱方以防秋畢請撤客兵。未幾,寇大入,直逼畿輔。帝震怒,並械總督翟鵬遣戍,斃方杖下。御史舒汀言,方止議撤薊兵,而並撤宣、大,則伯溫與職方郎韓最也。帝遂削伯溫籍,杖最八十,戍極邊。伯溫歸,疽發背卒。穆宗立,復官,賜恤。天啟初,追謚襄懋。

伯溫氣宇沉毅,飲啖兼十人。臨事決機,不動聲色。安南之役,萬達、岳策為多。伯溫力薦於朝,二人遂得任用。

汪文盛,字希周,崇陽人。正德六年進士。授饒州推官。有顧嵩者,挾刃入淮王祐棨府,被執,誣文盛使刺王。下獄訊治,久之得白,還官。事詳《淮王傳》。入為兵部主事,偕同官諫武宗南巡,杖闕下。嘉靖初,歷福州知府,遷浙江、陜西副使,皆督學校。擢雲南按察使。

十五年冬,廷議將討安南。以文盛才,就拜右僉都御史,巡撫其地。黔國公沐朝輔幼,兵事一決於文盛。副使鮑象賢言剿不如撫,文盛然之。會聞莫登庸已篡位,安南舊臣不服,多據地構兵。有武文淵者,據宣光,以所部萬人降。獻進兵地圖,且言舊臣阮仁蓮、黎景瑂等皆分據一方與登庸抗,天兵至,號召國中義士,諸方並起,登庸可擒也。文盛以聞。授文淵四品章服,子弟給冠帶。文盛又招安南旁近諸國助討,皆聽命。乃奏言:「老撾地廣兵眾,可使當一面。八百、車里、孟艮多兵象,可備徵調。酋長俱未襲職,乞免其保勘,先授以官,彼必鼓勇為用。」帝悉從之。文盛乃檄安南所部以土地歸者,仍故職,並諭登庸歸命。攻破鎮守營,方瀛救之失利。登庸部眾多來附,文盛列營樹柵蓮花灘處之。蓮花灘者,蒙自縣地,當交、廣水陸衝,為安南腹裏。登庸益懼,請降,願修貢,因言黎寧阮氏子,所持印亦偽。文盛以聞,朝議不許。既而毛伯溫至南寧,受登庸降如文盛議,安南遂定。是役也,功成於伯溫,然伐謀制勝,文盛功為多。及論功,伯溫及兩廣鎮巡官俱進秩,而文盛止賚銀幣。奸人唐弼請開大理銀礦,帝許之。文盛斥其妄,下之吏。召為大理卿。九廟災,道病,自陳疏少緩,令致仕。卒,賜恤如制。

從子宗伊,字子衡,為文盛後。嘉靖十七年進士。除浮梁知縣,累官兵部郎中。楊繼盛劾嚴嵩及其孫鵠冒功事,宗伊議不撓。忤嵩,自免歸。隆慶初,起南京吏部郎中,歷應天府尹。裁諸司供億,歲省民財萬計。萬曆初,進南京大理卿。三遷戶部尚書總督倉場,致仕,卒。天啟初,追謚恭惠。

鮑象賢,歙人。由進士授御史,歷雲南副使。毛伯溫檄文盛會師,以象賢領中哨。屢遷右副都御史,巡撫陜西,代石簡撫雲南。初,元江土舍那鑑殺知府那憲以叛,布政使徐越往招降被殺。簡攻之未克,坐越事罷,而象賢代之。乃集士、漢兵七萬以討,鑒懼,仰藥死,擇那氏後立之。遷兵部右侍郎,總督兩廣軍務。賊魁徐銓等糾倭橫海上,檄副使汪柏等擊斬之。廣西賊黃父將等擾慶遠,搗其巢,大獲。予象賢一子官。入佐南京兵部。被劾,回籍聽勘。家居十年,起太僕卿。復以右副都御史巡撫山東。召拜兵部左侍郎。年老引去。隆慶初卒。

翁萬達,字仁夫,揭陽人。嘉靖五年進士。授戶部主事。再遷郎中,出為梧州知府。咸寧侯仇鸞鎮兩廣,縱部卒為虐。萬達縛其尤橫者,杖之。閱四年,聲績大著。會朝議將討安南,擢萬達廣西副使,專辦安南事。萬達請於總督張經曰:「莫登庸大言『中國不能正土官弒逆罪,安能問我』。今憑祥州土舍李寰弒其土官珍,思恩府土目盧回煽九司亂,龍州土舍趙楷殺從子燧、爰,又結田州人韋應殺燧弟寶,斷藤峽瑤侯公丁負固。此曹同惡共濟,一旦約為內應,我且不自保。先擒此數人問罪,安南易下耳。」經曰:「然,惟君之所為。」於是誅寰、應,擒回,招還九司,誘殺楷,佯繫訟公丁者紿公丁,執諸坐。以兩軍破平其巢。又議割四峒屬南寧,降峒豪黃賢相。登庸始懼。遷浙江右參政。經以征安南非萬達不可,奏留之,乃命以參政蒞廣西。已而毛伯溫集兵進剿,萬達上書伯溫,言:「揖讓而告成功,上策也。懾之以不敢不從,中策也。芟夷絕滅,終為下策。」伯溫然之。會獲安南諜者丁南傑,萬達解其縛,厚遇,遣之去,怵以天朝兵威。登庸大懼,乃詣伯溫乞降。是役也,萬達功最,賞不逾常格。然帝知其能,遷四川按察使。歷陜西左、右布政使。

二十三年,擢右副都御史,巡撫陜西。尋進兵部右侍郎兼右僉都御史,代翟鵬總督宣、大、山西、保定軍務。劾罷宣府總兵官郤永、副總兵姜奭,薦何卿、趙卿、沈希儀。趙卿遂代永。萬達謹偵候,明賞罰。每當防秋,發卒乘障,陰遣卒傾硃於油,察離次者硃其處。卒歸輒縛,毋敢復離次者。嚴殺降禁,違輒抵死。得降人,撫之如所親,以是益知敵情。寇數萬騎犯大同中路,入鐵裹門,故總兵官張達力戰卻之。又犯鵓鴿谷,參將張鳳、諸生王邦直等戰死。萬達與總兵官周尚文備陽和,而遣騎四出邀擊,頗有斬獲。寇登山,見官兵大集,乃引去。事聞,賜敕獎賚。屢疏請修築邊牆,議自大同東路陽和口至宣府西陽河,須帑銀二十九萬。帝已許之,兵部撓其議,以大同舊有二邊,不當復於邊內築牆。帝不聽。乃自大同東路天城、陽和、開山口諸處為牆百二十八里,堡七,墩臺百五十四;宣府西路西陽河、洗馬林、張家口諸處為牆六十四里,敵臺十。斬崖削坡五十里。工五十餘日成。進右都御史。發代府宗室充灼等叛謀,進左都御史。

已,會宣、大、山西鎮巡官議上邊防修守事宜,其略曰:

山西起保德州黃河岸,歷偏頭,抵老營二百五十四里。大同西路起丫角山,歷中北二路,東抵東陽河鎮口臺六百四十七里。宣府起西陽河,歷中北二路,東抵永寧四海冶千二十三里。凡千九百二十四里,皆逼巨寇,險在外,所謂極邊也。山西老營堡轉南而東,歷寧武、雁門,至平刑關八百里。又轉南而東,歷龍泉、倒馬、紫荊之吳王口、插箭嶺、浮圖峪,至沿河口千七十餘里。又東北,歷高崖、白羊,至居庸關一百八十餘里。凡二千五十餘里,皆峻山層岡,險在內,所謂次邊也。外邊,大同最難守,次宣府,次山西之偏、老。大同最難守者,北路。宣府最難守者,西路。山西偏關以西百五十里,恃河為險;偏關以東百有四里,略與大同西路等。內邊,紫荊、寧武、雁門為要,次則居庸、倒馬、龍泉、平刑。邇年寇犯山西,必自大同;犯紫荊,必自宣府。

先年山西防秋,止守外邊偏、老一帶,歲發班軍六千人備禦,大同仍置兵,寧、雁為聲援。比棄極衝,守次邊,非守要之意。宣府亦專備西、中二路,而北路空虛。且連年三鎮防秋,徵調遼、陜兵馬,糜糧賞不訾,恐難持久。併守之議,實為善經。外邊四時皆防,城堡兵各有分地,冬春徂夏,不必參錯徵發。若泥往事臨時調遣,近者數十里,遠者百餘里,首尾不相應。萬一如往年潰牆而入,越關而南,京師震駭,方始徵調,何益事機?擺邊之兵,未可遽罷。

《易》曰「王公設險以守其國」。「設」之云者,築垣乘障、資人力之謂也。山川之險,險與彼共。垣塹之險,險為我專。百人之堡,非千人不能攻,以有垣塹可憑也。修邊之役,必當再舉。夫定規畫,度工費,二者修邊之事;慎防秋,併兵力,重責成,量徵調,實邊堡,明出塞,計供億,節財用,八者守邊之事。

因條十事上之,帝悉報許。乃請帑銀六十萬兩,修大同西路、宣府東路邊牆,凡八百里。工成,予一子官。

萬達精心計,善鉤校,牆堞近遠,濠塹深廣,曲盡其宜。寇乃不敢輕犯。牆內戍者得以暇耕牧,邊費亦日省。初,客兵防秋,歲帑金一百五十餘萬,添發且數十萬,其後減省幾半。又議掣山西兵並力守大同,巡撫孫繼魯沮之。帝為逮繼魯,悉納萬達言。

萬達更事久,帝深倚之,所請無不從,獨言俺答貢事與帝意左。先是,二十一年,俺答阿不孩使石天爵等款鎮遠堡求貢。言小王子等九部牧青山,艷中國縑帛,入掠止人畜,所得寡,且不能無亡失,故令天爵輸誠。朝議不納。天爵等復至,巡撫龍大有執之。大有進一官,將吏悉遷擢,磔天爵於市。寇怒,大入,屠村堡,信使絕五年。會玉林衛百戶楊威為所掠,威詭能定貢市,遂釋還。俺答阿不孩復遣使款大同左衛塞,邊帥家丁董寶等狃天爵前事,復殺之,以首功報。萬達言:「北敵,弘治前歲入貢,疆場稍寧。自虞臺嶺之戰覆我師,漸輕中國,侵犯四十餘年。石天爵之事,臣嘗痛邊臣失計。今復通款,即不許,當善相諭遣。誘而殺之,此何理也?請亟誅寶等,榜塞上,明告以朝廷德意,解其蓄怨構兵之謀。」帝不聽。未幾,俺答阿不孩復奉印信番文,欲詣邊陳款。萬達為奏曰:「今屆秋,彼可一逞。乃屢被殺戮,猶請貢不已者,緣入犯則利在部落,獲貢則利歸其長。處之克當,邊患可弭。若臣等封疆臣,貢亦備,不貢亦備,不緣此懈也。」兵部尚書陳經等言敵難信,請敕邊臣詰實,責萬達十日內回奏。萬達還其使,與約。至期,使者不至。萬達慮帝督過,以使者去無可究為辭。已而使狎至,牢拒之,好言慰答而已。俺答以通好,散處其眾,不設備,亦不殺哨卒。頃之,復至,詞益恭。萬達又為奏曰:「敵懇懇求貢,去而復來。今宣、大興版築,正當羈縻,使無擾。請限以地、以人、以時。悉聽,即許之貢;不聽,則曲在彼,即拒絕之。」帝責其瀆奏,卒不許。蓋是時曾銑有復套之議,夏言主之,故力絀貢議,且以復套事行諸邊臣議之。萬達議曰:

河套本中國故壤。成祖三犁王庭,殘其部落,舍黃河,衛東勝。後又撤東勝以就延綏,套地遂淪失。然正統、弘治間,我未守,彼亦未取。乃因循畫地守,捐天險,失沃野之利。弘治前,我猶歲搜套,後乃任彼出入,盤據其中,畜牧生養。譬之為家,成業久矣,欲一舉復之,毋乃不易乎!提軍深入,山川之險易,途徑之迂直,水草之有無,皆未熟知。我馬出塞三日已疲,彼騎一呼可集。我軍數萬眾,緩行持重則備益固,疾行趨利則輜重在後。即得小利,歸師尚艱。倘失嚮導,全軍殆矣。彼遷徙遠近靡常。一戰之後,彼或保聚,或佯遁,笳角時動,壁壘相持,已離復合,終不渡河。我軍於此,戰耶,退耶,兩相守耶?數萬眾出塞,亦必數萬眾援之,又以驍將通糧道,是皆至難而不可任者也。

夫馳擊者彼所長,守險者我所便。弓矢利馳擊,火器利守險。舍火器守險,與之馳擊於黃沙白草間,大非計。議者欲整六萬眾,為三歲期。春夏馬瘦,彼弱,我利於征;秋冬馬肥,彼強,我利於守。春搜套,秋守邊,三舉彼必遠遁,我乃拒河守。夫馬肥瘦,我與敵共之。即彼弱,然坐以待,懼其擾擊我,及彼強,又懼其報復我。且六萬之眾,千里襲人,一舉失利,議論蜂起,烏能待三?即三舉三勝,彼敗而守,終不渡河,版築亦無日。

議者見近時搗巢,恒獲首功,昔年城大同五堡,寇不深競,以為套易復。然搗巢,因其近塞,乘不備,勝則倏歸,舉足南向即家門。復套,則深入其地,後援不繼,事勢異也。往城諸邊,近我土,彼原不以為利。套,自其四時駐牧地,肯晏然已乎?事體異也。曰伺彼出套,據河守,先亟築渡口垣牆,以次移置邊堡。彼控弦十餘萬,豈有空套出。築垣二千餘里,豈不日可成?堡非百數十不相聯絡,堡兵非千人不可居,而遊徼尞望者不與,當三十萬眾不止也。況循邊距河,動輒千里,一歲食糜億萬。自內輸邊,自邊輸河,飛挽之艱不可不深慮。若令彼有其隙,我乘其敝,從而圖之,未嘗不可。今塞下喘息未定,邊卒瘡痍未起,橫挑強寇以事非常,愚所不解也。

議上,不省。

其後,俺答與小王子隙。小王子欲寇遼東,俺答以其謀告,請與中國夾攻以立信。萬達不敢聞。使者再至,為言於朝,帝不許。二十七年三月,萬達又言諸部求貢不遂,慚且憤,聲言大舉犯邊,乞令邊臣得便宜從事。帝怒,切責之,通貢議乃絕。其年八月,俺答犯大同不克,退攻五堡,官軍戰彌陀山卻之。趨山西,復敗還。踰月,犯宣府,大掠永寧、隆慶、懷來,軍民死者數萬。萬達坐停俸二級。俄錄彌陀山功,還其俸。俺答將復寇宣府,總兵官趙卿怯,萬達奏以周尚文代。未至,寇犯滴水崖,指揮董抃、江瀚、唐臣、張淮等戰死,遂南下駐隆慶石河營,分遊騎東掠。遊擊王鑰、大同遊擊袁正卻之,寇移而南。會尚文萬騎至,參將田琦騎千餘與合,連戰曹家莊、斬四首,搴其旗,寇據險不退。萬達督參將姜應熊等馳赴,順風鼓噪,揚沙蔽天。寇驚曰:「翁太師至矣!」是夜東去。諸將追擊,連敗之。帝偵萬達督戰狀,大喜,立進兵部尚書兼右副都御史。尋召理部事。以父憂歸。

明年秋,大同失事,督撫郭宗皋、陳耀被逮,詔起萬達代宗皋。萬達方病疽,廬墓間,疏請終制。未達,而俺答犯都城。兵部尚書丁汝夔得罪,遂即以萬達代之。萬達家嶺南,距京師八千里,倍道行四十日抵近京。時寇氛熾,帝日夕彳奚萬達至。遲之,以問嚴嵩。嵩故不悅萬達,言寇患在肘腋,諸臣觀望,非君召不俟駕之義。帝遂用王邦瑞於兵部。不數日,萬達至,具疏自明。帝責其欺慢,念守制,姑奪職,聽別用。仇鸞時為大將軍,寵方盛,銜宿怨,讒言構於帝。萬達遂失眷,降兵部右侍郎兼右僉都御史,經略紫荊諸關。三十年二月,京察,自陳乞終制。帝疑其避事,免歸。瀕行疏謝,復摘訛字為不敬,斥為民。明年十月,兵部尚書趙錦以附仇鸞戍邊,復起萬達代之。未聞命卒,年五十五。

萬達事親孝。父歿,負土成墳。好談性命之學,與歐陽德、羅洪先、唐順之、王畿、魏良政善。通古今,操筆頃刻萬言。為人剛介坦直,勇於任事,履艱危,意氣彌厲。臨陣嘗身先士卒,尤善御將士,得其死力。嘉靖中,邊臣行事適機宜、建言中肯窾者,萬達稱首。隆慶中,追謚襄毅。

贊曰:楊一清、王瓊俱負才略,著績邊陲,有人倫鑒,鋤奸定難因以成功。亦俱任智數。然瓊,其權譎之尤歟!彭澤望甚偉,顧處置哈密,抑何舛也。毛伯溫能任翁萬達、張岳,以成安南之功,不失為持重將。萬達飭邊備,整軍實,其爭復套,知彼知己,尤深識遠慮云。


\end{pinyinscope}