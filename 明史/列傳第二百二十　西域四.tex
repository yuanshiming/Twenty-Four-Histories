\article{列傳第二百二十 西域四}

\begin{pinyinscope}
○撒馬兒罕沙鹿海牙達失干賽藍養夷渴石迭里迷卜花兒別失八里哈烈俺都淮八答黑商於闐失剌思俺的乾哈實哈兒亦思弗罕火剌札乞力麻兒白松虎兒答兒密納失者罕敏真日落

米昔兒黑婁討來思阿速沙哈魯天方默德那坤城哈三等二十九部附魯迷

撒馬兒罕,即漢罽賓地,隋曰漕國,唐復名罽賓,皆通中國。元太祖蕩平西域,盡以諸王、駙馬為之君長,易前代國名以蒙古語,始有撒馬兒罕之名。去嘉峪關九千六百里。元末為之王者,駙馬帖木兒也。

洪武中,太祖欲通西域,屢遣使招諭,而遐方君長未有至者。二十年九月,帖木兒首遣回回滿剌哈非思等來朝,貢馬十五,駝二。詔宴其使,賜白金十有八錠。自是頻歲貢馬駝。二十五年兼貢絨六匹,青梭幅九匹,紅綠撒哈剌各二匹及鑌鐵刀劍、甲胄諸物。而其國中回回又自驅馬抵涼州互市。帝不許,令赴京鬻之。元時回回遍天下,及是居甘肅者尚多,詔守臣悉遣之,於是歸撒馬兒罕者千二百餘人。

二十七年八月,帖木兒貢馬二百。其表曰:「恭惟大明大皇帝受天明命,統一四海,仁德洪布,恩養庶類,萬國欣仰。咸知上天欲平治天下,特命皇帝出膺運數,為億兆之主。光明廣大,昭若天鏡,無有遠近,咸照臨之。臣帖木兒僻在萬里之外,恭聞聖德寬大,超越萬古。自古所無之福,皇帝皆有之。所未服之國,皇帝皆服之。遠方絕域,昏昧之地,皆清明之。老者無不安樂,少者無不長遂,善者無不蒙福,惡者無不知懼。今又特蒙施恩遠國,凡商賈之來中國者,使觀覽都邑、城池,富貴雄壯,如出昏暗之中,忽睹天日,何幸如之!又承敕書恩撫勞問,使站驛相通,道路無壅,遠國之人咸得其濟。欽仰聖心,如照世之杯,使臣心中豁然光明。臣國中部落,聞茲德音,歡舞感戴。臣無以報恩,惟仰天祝頌聖壽福祿,如天地永永無極。」照世杯者,其國舊傳有杯光明洞徹,照之可知世事,故云。帝得表,嘉其有文。明年命給事中傅安等齎璽書、幣帛報之。其貢馬,一歲再至,以千計,並賜賓鈔償之。

成祖踐阼,遣使敕諭其國。永樂三年,傅安等尚未還,而朝廷聞帖木兒假道別失八里率兵東,敕甘肅總兵官宋晟儆備。五年六月,安等還。初,安至其國被留,朝貢亦絕。尋令人導安遍歷諸國數萬里,以誇其國廣大。至是帖木兒死,其孫哈里嗣,乃遣使臣虎歹達等送安還,貢方物。帝厚賚其使,遣指揮白阿兒忻台等往祭故王,而賜新王及部落銀幣。其頭目沙里奴兒丁等亦貢駝馬。命安等賜其王綵幣,與貢使偕行。七年,安等還,王遣使隨入貢。自後,或比年,或間一歲,或三歲,輒入貢。十三年遣使隨李達、陳誠等入貢。暨辭歸,命誠及中官魯安偕往,賜其頭目兀魯伯等白銀、彩幣。其國復遣使隨誠等入貢。十八年復命誠及中官郭敬齎敕及彩幣報之。宣德五年秋、冬,頭目兀魯伯米兒咱等遣使再入貢。七年遣中官李貴等齎文綺、羅錦賜其國。

正統四年貢良馬,色玄,蹄額皆白。帝愛之,命圖其像,賜名瑞DO,賞賚有加。十年十月書諭其王兀魯伯曲烈乾曰:「王遠處西陲,恪修職貢,良足嘉尚。使回,特賜王及王妻子彩幣表裏,示朕優待之意。」別敕賜金玉器、龍首杖、細馬鞍及諸色織金文綺,官其使臣為指揮僉事。

景泰七年貢馬駝、玉石。禮官言:「舊制給賞太重。今正、副使應給一等、二等賞物者,如舊時。三等人給綵緞四表裏,絹三匹,織金糸寧絲衣一襲。其隨行鎮撫、舍人以下,遞減有差。所進阿魯骨馬每匹彩緞四表裏、絹八匹,駝三表裏、絹十匹,達達馬不分等第,每匹糸寧絲一匹、絹八匹、折鈔絹一匹,中等馬如之,下等者亦遞減有差。」制可。又言:「所貢玉石,堪用者止二十四塊,六十八斤,餘五千九百餘斤不適於用,宜令自鬻。而彼堅欲進獻,請每五斤賜絹一匹。」亦可之。已而使臣還,賜王卜撒因文綺、器物。天順元年命都指揮馬雲等使西域,敕獎其鎖魯檀毋撒,賜彩幣,令護朝使往還。鎖魯檀者,君長之稱,猶蒙古可汗也。七年復命指揮詹升等使其國。

成化中,其鎖魯檀阿黑麻三入貢。十九年偕亦思罕酋長貢二獅,至肅州,其使者奏請大臣往迎。職方郎中陸容言:「此無用之物,在郊廟不可為犧牲,在乘輿不可被驂服,宜勿受。」禮官周洪謨等亦言往迎非禮,帝卒遣中使迎之。獅日啖生羊二,醋、酐、蜜酪各二瓶。養獅者,光祿日給酒饌。帝既厚加賜賚,而其使者怕六灣以為輕,援永樂間例為請。禮官議從正統四年例,加彩幣五表裏。使者復以為輕,乃加正、副使各二表裏,從者半之,命中官韋洛、鴻臚署丞海濱送之還。其使者不由故道赴廣東,又多買良家女為妻妾,洛等不為禁止。久之,洛上疏委罪於濱,濱坐下吏。其使者請泛海至滿剌加市狻猊以獻,市舶中官韋眷主之,布政使陳選力陳不可,乃已。

弘治二年,其使由滿剌加至廣東,貢獅子、鸚鵡諸物,守臣以聞。禮官耿裕等言:「南海非西域貢道,請卻之。」禮科給事中韓鼎等亦言:「猙獰之獸,狎玩非宜,且騷擾道路,供費不貲,不可受。」帝曰:「珍禽奇獸,朕不受獻,況來非正道,其即卻還。守臣違制宜罪,姑貸之。」禮官又言:「海道固不可開,然不宜絕之已甚,請薄犒其使,量以綺帛賜其王。」制可。明年又偕土魯番貢獅子及哈剌、虎剌諸獸,由甘肅入。鎮守中官傅德、總兵官周玉等先圖形奏聞,即遣人馳驛起送。獨巡按御史陳瑤論其糜費煩擾,請勿納。禮官議如其言,量給犒賞,且言:「聖明在御,屢卻貢獻,德等不能奉行德意,請罪之。」帝曰:「貢使既至,不必卻回,可但遣一二人詣京。獅子諸物,每獸日給一羊,不得妄費。德等貸勿治。」後至十二年始來貢。明年復至。而正德中猶數至。

嘉靖二年,貢使又至。禮官言:「諸國使臣在途者遷延隔歲,在京者伺候同賞,光祿、郵傳供費不貲,宜示以期約。」因列上禁制數事,從之。十二年偕天方、土魯番入貢,稱王者至百餘人。禮官夏言等論其非,請敕閣臣議所答。張孚敬等言:「西域諸王,疑出本國封授,或部落自相尊稱。先年亦有至三四十人者,即據所稱答之。若驟議裁革,恐人情觖望,乞更敕禮、兵二部詳議。」於是言及樞臣王憲等謂:「西域稱王者,止土魯番、天方、撒馬兒罕。如日落諸國,稱名雖多,朝貢絕少。弘、正間,土魯番十三入貢,正德間,天方四入貢,稱王者率一人,多不過三人,餘但稱頭目而已。至嘉靖二年、八年,天方多至六七人,土魯番至十一二人,撒馬兒罕至二十七人。孚敬等言三四十人者,並數三國爾。今土魯番十五王,天方二十七王,撒馬兒罕五十三王,實前此所未有。弘治時回賜敕書,止稱一王。若循撒馬兒罕往歲故事,類答王號,人與一敕,非所以尊中國制外蕃也。蓋帝王之馭外蕃,固不拒其來,亦必限以制,其或名號僭差,言詞侮慢,則必正以大義,責其無禮。今謂本國所封,何以不見故牘?謂部落自號,何以達之天朝?我概給以敕,而彼即據敕恣意往來,恐益擾郵傳,費供億,殫府庫以實谿壑,非計之得也。」帝納其言,國止給一敕,且加詰讓,示以國無二王之義。然諸蕃迄不從,十五年入貢復如故。甘肅巡撫趙載奏:「諸國稱王者至一百五十餘人,皆非本朝封爵,宜令改正,且定貢使名數。通事宜用漢人,毋專用色目人,致交通生釁。」部議從之。二十六年入貢,甘肅巡撫楊博請重定朝貢事宜,禮官復列數事行之。後入貢,迄萬曆中不絕。蓋番人善賈,貪中華互市,既入境,則一切飲食、道途之資,皆取之有司,雖定五年一貢,迄不肯遵,天朝亦莫能難也。

其國東西三千餘里,地寬平,土壤膏腴。王所居城,廣十餘里,民居稠密。西南諸蕃之貨皆聚於此,號為富饒。城東北有土屋,為拜天之所,規制精巧,柱皆青石,雕為花文,中設講經之堂。用泥金書經,裹以羊皮。俗禁酒。人物秀美,工巧過於哈烈,而風俗、土產多與之同。其旁近東有沙鹿海牙、達失干、賽藍、養夷,西有渴石、迭里迷諸部落,皆役屬焉。

沙鹿海牙,西去撒馬兒罕五百餘里。城居小岡上,西北臨河。河名火站,水勢衝急,架浮梁以渡,亦有小舟。南近山,人多依崖谷而居。園林廣茂。西有大沙洲,可二百里。無水,間有之,咸不可飲。牛馬誤飲之,輒死。地生臭草,高尺餘,葉如蓋,煮其液成膏,即阿魏。又有小草,高一二尺,叢生,秋深露凝,食之如蜜,煮為糖,番名達郎古賓。

永樂間,李達、陳誠使其地,其酋即遣使奉貢。宣德七年命中官李貴齎敕諭其酋,賜金織文綺、彩幣。

達失干,西去撒馬兒罕七百餘里。城居平原,周二里。外多園林,饒果木。土宜五穀。民居稠密。李達、陳誠、李貴之使,與沙鹿海牙同。

賽藍,在達失干之東,西去撒馬兒罕千餘里。有城郭,周二三里。四面平曠,居人繁庶。五穀茂殖,亦饒果木。夏秋間,草中生黑小蜘蛛。人被螫,遍體痛不可耐,必以薄荷枝掃痛處,又用羊肝擦之,誦經一晝夜,痛方止,體膚盡蛻。六畜被傷者多死。凡止宿,必擇近水地避之。元太祖時,都元帥薛塔剌海從征賽藍諸國,以炮立功,即此地也。陳誠、李貴之使,與諸國同。

養夷,在賽藍東三百六十里。城居亂山間。東北有大溪,西流入巨川。行百里,多荒城。蓋其地介別失八里、蒙古部落之間,數被侵擾。以故人民散亡,止戍卒數百人居孤城,破廬頺垣,蕭然榛莽。永樂時,陳誠至其地。

渴石,在撒馬兒罕西南三百六十里。城居大村,周十餘里。宮室壯麗,堂以玉石為柱,牆壁窗牖盡飾金碧,綴琉璃。其先,撒馬兒罕酋長駙馬帖木兒居之。城外皆水田。東南近山,多園林。西行十餘里,饒奇木。又西三百里,大山屹立,中有石峽,兩崖如斧劈。行二三里出峽口,有石門,色似鐵,路通東西,番人號為鐵門關,設兵守之。或言元太祖至東印度鐵門關,遇一角獸,能人言,即此地也。

迭里迷,在撒馬兒罕西南,去哈烈二千餘里。有新舊二城,相去十餘里,其酋長居新城。城內外居民僅數百家,畜牧蕃息。城在阿術河東,多魚。河東地隸撒馬兒罕,西多蘆林,產獅子。陳誠、李達嘗使其地。

卜花兒,在撒馬兒罕西北七百餘里。城居平川,周十餘里,戶萬計。市里繁華,號為富庶。地卑下,節序嘗溫,宜五穀桑麻,多絲綿布帛,六畜亦饒。

永樂十三年,陳誠自西域還,所經哈烈、撒馬兒罕、別失八里、俺都淮、八答黑商、迭里迷、沙鹿海牙、賽藍、渴石、養夷、火州、柳城、土魯番、鹽澤、哈密、達失干、卜花兒凡十七國,悉詳其山川、人物、風俗,為《使西域記》以獻,以故中國得考焉。宣德七年命李達撫諭西域,卜花兒亦與焉。

別失八里,西域大國也。南接于闐,北連瓦剌,西抵撒馬兒罕,東抵火州,東南距嘉峪關三千七百里。或曰焉耆,或曰龜茲。元世祖時設宣慰司,尋改為元帥府,其後以諸王鎮之。

洪武中,藍玉征沙漠,至捕魚兒海,獲撒馬兒罕商人數百。太祖遣官送之還,道經別失八里。其王黑的兒火者,即遣千戶哈馬力丁等來朝,貢馬及海青,以二十四年七月達京師。帝喜,賜王彩幣十表裏,其使者皆有賜。九月命主事寬徹、御史韓敬、評事唐鉦使西域。以書諭黑的兒火者曰:「朕觀普天之下,后土之上,有國者莫知其幾。雖限山隔海,風殊俗異,然好惡之情,血氣之類,未嘗異也。皇天眷佑,惟一視之。故受天命為天下主者,上奉天道,一視同仁,俾巨細諸國,殊方異類之君民,咸躋乎仁壽。而友邦遠國,順天事大,以保國安民,皇天監之,亦克昌焉。曩者我中國宋君,奢縱怠荒,奸臣亂政。天監否德,於是命元世祖肇基朔漠,入統中華,生民賴以安靖七十餘年。至於後嗣,不修國政,任用非人,致幻綱盡弛,強陵弱,眾暴寡,民生嗟怨,上達於天。天用是革其命,屬之於朕。朕躬握乾符,以主黔黎。凡諸亂雄擅聲教違朕命者兵偃之,順朕命者德撫之。是以三十年間,諸夏奠安,外蕃賓服。惟元臣蠻子哈剌章等尚率殘眾,生釁寇邊,興師致討,勢不容已。兵至捕魚兒海,故元諸王、駙馬率其部屬來降。有撒馬兒罕數百人以貿易來者,朕命官護歸已三年矣。使者還,王即遣使來貢,朕甚嘉之。王其益堅事大之誠,通好往來,使命不絕,豈不保封國於悠久乎?特遣官勞嘉,其悉朕意。」徹等既至,王以其無厚賜,拘留之。敬、鉦二人得還。

三十年正月復遣官齎書諭之曰:「朕即位以來,西方諸商來我中國互市者,邊將未嘗阻絕。朕復敕吏民善遇之,由是商人獲利,疆埸無擾,是我中華大有惠於爾國也。前遣寬徹等往爾諸國通好,何故至今不返?吾於諸國,未嘗拘留一人,而爾顧拘留吾使,豈理也哉!是以近年回回入境者,亦令於中國互市,待徹歸放還。後諸人言有父母妻子,吾念其至情,悉縱遣之。今復使使諭爾,俾知朝廷恩意,毋梗塞道路,致啟兵端。《書》曰:『怨不在大,亦不在小。惠不惠,懋不懋。』爾其惠且懋哉。」徹乃得還。

成祖即位之冬,遣官齎璽書綵幣使其國。未幾,黑的兒火者卒,子沙迷查乾嗣。永樂二年遣使貢玉璞、名馬,宴賚有加。時哈密忠順王安克帖木兒為可汗鬼力赤毒死,沙迷查干率師討之。帝嘉其義,遣使賚以綵幣,令與嗣忠順王脫脫敦睦。四年夏來貢,命鴻臚寺丞劉帖木兒齎敕幣勞賜,與其使者偕行。秋、冬暨明年夏,三入貢,因言撒馬兒罕本其先世故地,請以兵復之。命中官把太、李達及劉帖木兒齎敕戒以審度而行,毋輕舉,因賜之綵幣。六年,太等還,言沙迷查乾已卒,弟馬哈麻嗣。帝即命太等往祭,並賜其新王。

八年以朝使往撒馬兒罕者,馬哈麻待之厚,遣使齎彩幣賜之。明年貢名馬、文豹,命給事中傅安送其使還,賚金織文綺。時瓦剌使者言馬哈麻將襲其部落,因諭以順天保境之義。十一年,貢使將至甘肅,命所司宴勞,且敕總兵官李彬善遇之。明年冬,有自西域還者,言馬哈麻母及弟相繼卒。帝愍之,命安齎敕慰問,賚以彩幣。已而馬哈麻亦卒,無子,從子納黑失只罕嗣。十四年春,使來告喪。命安及中官李達弔祭,即封其嗣子為王,賚文綺、弓刀、甲胄,其母亦有賜。明年遣使來貢,言將嫁女撒馬兒罕,請以馬市妝奩。命中官李信等以綺、帛各五百匹助之。十六年,貢使速哥言其王為從弟歪思所弒,而自立,徙其部落西去,更國號曰亦力把裡。帝以番俗不足治,授速哥為都督僉事,而遣中官楊忠等賜歪思弓刀、甲胄及文綺、彩幣,其頭目忽歹達等七十餘人並有賜,自是奉貢不絕。

宣德元年,帝嘉其尊事朝廷,遣使賜之鈔幣。明年入貢,授其正、副使為指揮千戶,賜誥命、冠帶,自後使臣多授官。三年貢駝馬,命指揮昌英等齎璽書、綵幣報之。時歪思連歲貢,而其母鎖魯檀哈敦亦連三歲來貢。歪思卒,子也先不花嗣。正統元年遣使來朝,貢方物,後亦頻入貢。故王歪思之婿卜賽因亦遣使來貢。十年,也先不花卒,也密力虎者嗣。明年貢馬駝方物,命以綵幣賜王及王母。景泰三年貢玉石三千八百斤,禮官言其不堪用,詔悉收之,每二斤賜帛一匹。天順元年命千戶於志敬等以復辟諭其王,且賜綵幣。成化元年,禮官姚夔等定西域朝貢期,令亦力把里三歲、五歲一貢,使者不得過十人,自是朝貢遂稀。

其國無城郭宮室,隨水草畜牧。人性獷悍,君臣上下無體統。飲食衣服多與瓦剌同。地極寒,深山窮谷,六月亦飛雪。

哈烈,一名黑魯,在撒馬兒罕西南三千里,去嘉峪關萬二千餘里,西域大國也。元駙馬帖木兒既君撒馬兒罕,又遣其子沙哈魯據哈烈。

洪武時,撒馬兒罕及別失八里咸朝貢,哈烈道遠不至。二十五年遣官詔諭其王,賜文綺、綵幣,猶不至。二十八年遣給事中傅安、郭驥等攜士卒千五百人往,為撒馬兒罕所留,不得達。三十年又遣北平按察使陳德文等往,亦久不還。

成祖踐阼,遣官齎璽書綵幣賜其王,猶不報命。永樂五年,安等還。德文遍歷諸國,說其酋長入貢,皆以道遠無至者,亦於是年始還。德文,保昌人,采諸方風俗作為歌詩以獻。帝嘉之,擢僉都御史。明年復遣安齎書幣往哈烈,其酋沙哈魯把都兒遣使隨安朝貢。七年達京師,復命齎賜物偕其使往報。明年,其酋遣使朝貢。

撒馬兒罕酋哈里者,哈烈酋兄子也,二人不相能,數構兵。帝因其使臣還,命都指揮白阿兒忻台齎敕諭之曰:「天生民而立之君,俾各遂其生。朕統御天下,一視同仁,無間遐邇,屢嘗遣使諭爾。爾能虔修職貢,撫輯人民,安於西徼,朕甚嘉之。比聞爾與從子哈里構兵相仇,朕為惻然。一家之親,恩愛相厚,足制外侮。親者尚爾乖戾,疏者安得協和。自今宜休兵息民,保全骨肉,共享太平之福。」因賜綵幣表裏,并敕諭哈里罷兵,亦賜綵幣。

白阿兒忻台既奉使,遍詣撒馬兒罕、失剌思、俺的乾、俺都淮、土魯番、火州、柳城、哈實哈兒諸國,賜之幣帛,諭令入朝。諸酋長咸喜,各遣使偕哈烈使臣貢獅子、西馬、文豹諸物。十一年達京師。帝喜,御殿受之,犒賜有加。自是諸國使並至,皆序哈烈於首。及歸,命中官李達、吏部員外郎陳誠、戶部主事李暹、指揮金哈藍伯等送之,就齎璽書、文綺、紗羅、布帛諸物分賜其酋。十三年,達等還,哈烈諸國復遣使偕來,貢文豹、西馬及他方物。明年再貢,及還,命陳誠齎書幣報之,所過州縣皆宴餞。十五年遣使隨誠等來貢。明年復貢,命李達等報如初。十八年偕于闐、八答黑商來貢。二十年復偕于闐來貢。

宣德二年,其頭目打剌罕亦不剌來朝,貢馬。自仁宗不勤遠略,宣宗承之,久不遣使絕域,故其貢使亦稀至。七年復命中官李貴通西域,敕諭哈烈酋沙哈魯曰:「昔朕皇祖太宗文皇帝臨御之日,爾等尊事朝廷,遣使貢獻,始終如一。今朕恭膺天命,即皇帝位,主宰萬方,紀元宣德。小大政務,悉體皇祖奉天恤民,一視同仁之心。前遣使臣齎書幣往賜,道阻而回。今已開通,特命內臣往諭朕意。其益順天心,永篤誠好,相與還往,同為一家,俾商旅通行,各遂所願,不亦美乎?」因賜以文綺、羅錦。貴等未至,其貢使法虎兒丁已抵京師,卒於館。命官致祭,有司營葬。尋復遣使隨貴貢駝馬、玉石。明年春,使者歸。復命貴護送,賜其王及頭目彩幣。是年秋及正統三年並來貢。

英宗幼沖,大臣務休息,不欲疲中國以事外蕃,故遠方通貢者甚少。至天順元年,復議通西域。大臣莫敢言,獨忠義衛吏張昭抗疏切諫,事乃止。七年,帝以中夏乂安,而遠蕃朝貢不至,分遣武臣齎璽書、綵幣往諭。於是都指揮海榮、指揮馬全往哈烈。然自是來者頗稀,即哈烈亦不以時貢。

嘉靖二十六年,甘肅巡撫楊博言:「西域入貢人多,宜為限制。」禮官言:「祖宗故事,惟哈密每年一貢,貢三百人,送十一赴京,餘留關內,有司供給。他若哈烈、哈三、土魯番、天方、撒馬兒罕諸國,道經哈密者,或三年、五年一貢,止送三五十人,其存留賞賚如哈密例。頃來濫放入京,宜敕邊臣恪遵此例,濫放者罪之。」制可。然是時哈烈已久不至,嗣後朝貢遂絕。

其國在西域最強大。王所居城,方十餘里。壘石為屋,平方若高臺,不用梁柱瓦甓,中敞,虛空數十間。囪牖門扉,悉雕刻花文,繪以金碧。地鋪氈罽,無君臣、上下、男女,相聚皆席地趺坐。國人稱其王曰鎖魯檀,猶言君長也。男髡首纏以白布,婦女亦白布蒙首,僅露雙目。上下相呼皆以名。相見止稍屈身,初見則屈一足三跪,男女皆然。食無匕箸,有瓷器。以葡萄釀酒。交易用錢,大小三等,不禁私鑄。惟輸稅於酋長,用印記,無印者禁不用。市易皆徵稅十二。不知斗斛,止設權衡。無官府,但有管事者,名曰刀完。亦無刑法,即殺人亦止罰錢。以姊妹為妻妾。居喪止百日,不用棺,以布裹屍而葬。常於墓間設祭,不祭祖宗,亦不祭鬼神,惟重拜天之禮。無干支朔望,每七日為一轉,周而復始。歲以二月、十月為把齋月,晝不飲食,至夜乃食,周月始茹葷。城中築大土室,中置一銅器,周圍數丈,上刻文字如古鼎狀。游學者皆聚此,若中國太學然。有善走者,日可三百里,有急使,傳箭走報。俗尚侈靡,用度無節。

土沃饒,節候多暖少雨。土產白鹽、銅鐵、金銀、琉璃、珊瑚、琥珀、珠翠之屬。多育蠶,善為紈綺。木有桑、榆、柳、槐、松、檜,果有桃、杏、李、梨、葡萄、石榴,穀有粟、麥、麻、菽,獸有獅、豹、馬、駝、牛、羊、雞、犬。獅生於阿術河蘆林中,初生目閉,七日始開。土人於目閉時取之,調習其性,稍長則不可馴矣。其旁近俺都淮、八答黑商,並隸其國。

俺都淮,在哈烈西北千三百里,東南去撒馬兒罕亦如之。城居大村,周十餘里。地平衍無險,田土膏腴,民物繁庶,稱樂土。自永樂八年至十四年偕哈烈通貢,後不復至。

八答黑商,在俺都淮東北。城周十餘里。地廣無險阻,山川明秀,人物樸茂。浮屠數區,壯麗如王居。西洋、西域諸賈多販鬻其地,故民俗富饒。初為哈烈酋沙哈魯之子所據。永樂六年命內官把太、李達賜其酋敕書彩幣,並及哈實哈兒、葛忒郎諸部,諭以往來通商之意,皆即奉命。自是,東西萬里行旅無滯。十二年,陳誠使其國。十八年遣使來貢,命誠及內官郭敬齎書幣往報。天順五年,其王馬哈麻遣使來貢。明年復貢。命使臣阿卜都剌襲父職,為指揮同知。

于闐,古國名,自漢迄宋皆通中國。永樂四年遣使來朝,貢方物。使臣辭歸,命指揮神忠母撒等齎璽書偕行,賜其酋織金文綺。其酋打魯哇亦不刺金遣使者貢玉璞,命指揮尚衡等齎書幣往勞。十八年偕哈烈、八答黑商諸國貢馬,命參政陳誠、中官郭敬等報以綵幣。二十年貢美玉,賜賚有加。二十二年貢馬及方物。時仁宗初踐阼,即宴賚遣還。

先是,永樂時,成祖欲遠方萬國無不臣服,故西域之使歲歲不絕。諸蕃貪中國財帛,且利市易,絡繹道途。商人率偽稱貢使,多攜馬、駝、玉石,聲言進獻。既入關,則一切舟車水陸、晨昏飲饌之費,悉取之有司。郵傳困供億,軍民疲轉輸。比西歸,輒緣道遲留,多市貨物。東西數千里間,騷然繁費,公私上下罔不怨咨。廷臣莫為言,天子亦莫之恤也。至是,給事中黃驥極陳其害。仁宗感其言,召禮官呂震責讓之。自是不復使西域,貢使亦漸稀。

于闐自古為大國,隋、唐間侵并戎盧、扞彌、渠勒、皮山諸國,其地益大。南距艸嶺二百餘里,東北去嘉峪關六千三百里。大略艸嶺以南,撒馬兒罕最大;以北,于闐最大。元末時,其主暗弱,鄰國交侵。人民僅萬計,悉避居山谷,生理蕭條。永樂中,西域憚天子威靈,咸修職貢,不敢擅相攻,于闐始獲休息。漸行賈諸蕃,復致富庶。桑麻黍禾,宛然中土。其國東有白玉河,西有綠玉河,又西有黑玉河,源皆出崑崙山。土人夜視月光盛處,入水採之,必得美玉。其鄰國亦多竊取來獻。迄萬曆朝,於闐亦間入貢。

失刺思,近撒馬兒罕。永樂十一年遣使偕哈烈、俺的乾、哈實哈兒等八國,隨白阿兒忻臺入貢方物,命李達、陳誠等齎敕偕其使往勞。十三年冬,其酋亦不剌金遣使隨達等朝貢,天子方北巡。至明年夏始辭還,復命誠偕中官魯安齎敕及白金、綵緞、紗羅、布帛賜其酋。十七年遣使偕亦思弗罕諸部貢獅子、文豹、名馬,辭還。復命安等送之,賜其酋絨錦、文綺、紗羅、玉繫腰、磁器諸物。時車駕頻歲北征,乏馬,遣官多齎彩幣、磁器,市之失剌思及撒馬兒罕諸國。其酋即遣使貢馬,以二十一年八月謁帝於宣府之行宮。厚賜之,遣還京師,其人遂久留內地不去。仁宗嗣位,趣之還,乃辭去。

宣德二年貢駝馬方物,授其使臣阿力為都指揮僉事,賜誥命、冠帶。嗣後久不貢。成化十九年與黑婁、撒馬兒罕、把丹沙諸國共貢獅子,詔加優賚。弘治五年,哈密忠順王陜巴襲封歸國,與鄰境野乜克力酋結婚。失剌思酋念其貧,偕旁國亦不剌因之酋,率其平章鎖和卜台、知院滿可,各遣人請頒賜財物,助之成婚。朝議義之,厚賜陜巴,並賜二國及其平章、知院綵幣。嘉靖三年與旁近三十二部並遣使貢馬及方物。其使者各乞蟒衣、膝襴、磁器、布帛。天子不能卻,量予之,自是貢使亦不至。

俺的乾,西域小部落。元太祖盡平西域,封子弟為王鎮之,其小者則設官置戍,同於內地。元亡,各自割據,不相統屬。洪武、永樂間,數遣人招諭,稍稍來貢。地大者稱國,小者止稱地面。迄宣德朝,效臣職、奉表箋、稽首闕下者,多至七八十部。而俺的乾,則永樂十一年與哈烈並貢者也。迨十四年,魯安等使哈烈、失剌思諸思,復便道賜其酋長文綺。然地小不能常貢,後竟不至。

哈實哈兒,亦西域小部落。永樂六年,把太、李達等齎敕往賜,即奉命。十一年遣使隨白阿兒忻台入朝,貢方物。宣德時亦來朝貢。天順七年命指揮劉福、普賢使其地。其貢使亦不能常至。

亦思弗罕,地近俺的乾。永樂十四年使俺都淮、撒馬兒罕者道經其地,賜其酋文綺諸物。十七年偕鄰國失剌思共貢獅、豹、西馬,賚白金、鈔幣。使臣辭還,命魯安等送之。有馬哈木者,願留京師。從其請。成化十九年與撒馬兒罕共貢獅子、名馬、番刀、兜羅、鎖幅諸物,賜賚有加。

先是,宣德六年,有亦思把罕遣使臣迷兒阿力朝貢,或云即亦思弗罕。

火剌札,國微弱。四圍皆山,鮮草木。水流曲折,亦無魚蝦。城僅里許,悉土屋,酋所居亦卑陋。俗敬僧。永樂十四年遣使朝貢,命所經地皆禮待。弘治五年,其地回回怕魯灣等由海道貢玻璃、瑪瑙諸物。孝宗不納,賜道里費遣還。

乞力麻兒,永樂中遣使來貢,惟獸皮、鳥羽、罽褐。其俗喜射獵,不事耕農。西南傍海,東北林莽深密,多猛獸、毒蟲。有逵巷,無市肆,交易用鐵錢。

白松虎兒,舊名速麻里兒。嘗有白虎出松林中,不傷人,亦不食他獸,旬日後不復見。國人異之,稱為神虎,曰此西方白虎所降精也,因改國名。其地無大山,亦不生樹木,無毒蟲、猛獸之害,然物產甚薄。永樂中嘗入貢。

答兒密,服屬撒馬兒罕。居海中,地不百里,人不滿千家。無城郭,上下皆居板屋。知耕植,有毛褐、布縷、馬駝、牛羊。刑止箠朴。交易兼用銀錢。永樂中遣使朝貢,賜《大統曆》及文綺、藥、茶諸物。

納失者罕,東去失剌思數日程,皆舟行。城東平原,饒水草,宜畜牧。馬有數種,最小者高不過三尺。俗重僧,所至必供飲食。然好氣健鬥,鬥不勝者,眾嗤之。永樂中遣使朝貢。使臣還,歷河北,轉關中,抵甘肅,有司皆置宴。

敏真城,永樂中來貢。其國地廣,多高山。日中為市,諸貨駢集,貴中國磁、漆器。產異香、駝、馬。

日落國,永樂中來貢。弘治元年,其王亦思罕答兒魯密帖里牙復貢。使臣奏求糸寧、絲、夏布、磁器,詔皆予之。

米昔兒,一名密思兒。永樂中遣使朝貢。既宴賚,命五日一給酒饌、果餌,所經地皆置宴。正統六年,王鎖魯檀阿失剌福復來貢。禮官言:「其地極遠,未有賜例。昔撒馬兒罕初貢時,賜予過優,今宜稍損。賜王彩幣十表裏,紗、羅各三匹,白BF絲布、白將樂布各五匹,洗白布二十匹,王妻及使臣遞減。」從之。自後不復至。

黑婁,近撒馬兒罕,世為婚姻。其地山川、草木、禽獸皆黑,男女亦然。宣德七年遣使來朝,貢方物。正統二年,其王沙哈魯鎖魯檀遣指揮哈只馬黑麻奉貢。命齎敕及金織糸寧絲、綵絹歸賜其王。六年復來貢。景泰四年偕鄰境三十一部男婦百餘人,貢馬二百四十有七,騾十二,驢十,駝七,及玉石、風砂、鑌鐵刀諸物。天順七年,王母塞亦遣指揮僉事馬黑麻捨兒班等奉貢。賜綵幣表裏、糸寧、絲襲衣,擢其使臣為指揮同知,從者七人俱為所鎮撫。成化十九年與失剌思、撒馬兒罕、把丹沙共貢獅子。把丹沙之長亦稱鎖魯檀馬黑麻,景泰七年嘗入貢,至是復偕至。弘治三年又與天方諸國貢駝、馬、玉石。

討來思,地小,周徑不百里。城近山。山下有水,赤色,望之如火。俗佞佛。婦人主家柄。產牛羊馬駝,有布縷毛褐。土宜穄麥,無稻。交易用錢。宣德六年入貢。明年命中官李貴齎璽書獎勞,賜文綺、綵帛。以地小不能常貢。

阿速,近天方、撒馬兒罕,幅員甚廣。城倚山面川。川南流入海,有魚鹽之利。土宜耕牧。敬佛畏神,好施惡鬥。物產富,寒暄適節,人無饑寒,夜鮮寇盜,雅稱樂土。永樂十七年,其酋牙忽沙遣使貢馬及方物,宴賚如制。以地遠不能常貢。天順七年命都指揮白全等使其國,竟不復再貢。

沙哈魯,在阿速西海島中。永樂中遣七十七人來貢,日給酒饌、果餌,異於他國。其地,山川環抱,饒畜產,人性樸直,恥斗好佛。王及臣僚處城中,庶人悉處城處。海產奇物,西域賈人以輕直市之,其國人不能識。

天方,古筠沖地,一名天堂,又曰默伽。不道自忽魯謨斯四十日始至,自古里西南行,三月始至。其貢使多從陸道入嘉峪關。

宣德五年,鄭和使西洋,分遣其儕詣古里。聞古里遣人往天方,因使人齎貨物附其舟偕行。往返經歲,市奇珍異寶及麒麟、獅子、駝雞以歸。其國王亦遣陪臣隨朝使來貢。宣宗喜,賜賚有加。正統元年始命附爪哇貢舟還,賜幣及敕獎其王。六年,王遣子賽亦得阿力與使臣賽亦得哈三以珍寶來貢。陸行至哈剌,遇賊,殺使臣,傷其子右手,盡劫貢物以去,命守臣察治之。

成化二十三年,其國中回回阿力以兄納的游中土四十餘載,欲往雲南訪求。乃攜寶物鉅萬,至滿剌加,附行人左輔舟,將入京進貢。抵廣東,為市舶中官韋眷侵剋。阿力怨,赴京自訴。禮官請估其貢物,酬其直,許訪兄於雲南。時眷懼罪,先已夤緣於內。帝乃責阿力為間諜,假貢行奸,令廣東守臣逐還,阿力乃號泣而去。弘治三年,其王速檀阿黑麻遣使偕撒馬兒罕、土魯番貢馬、駝、玉石。

正德初,帝從御馬太監谷大用言,令甘肅守臣訪求諸番騍馬、騸馬,番使云善馬出天方。守臣因請諭諸番貢使,傳達其王,俾以入貢。兵部尚書劉宇希中官指,議令守臣善擇使者與通事,親詣諸番曉諭,從之。十三年,王寫亦把剌克遣使貢馬、駝、梭幅、珊瑚、寶石、魚牙刀諸物,詔賜蟒龍金織衣及麝香、金銀器。

嘉靖四年,其王亦麻都兒等遣使貢馬、駝、方物。禮官言:「西人來貢,陜西行都司稽留半年以上始為具奏。所進玉石悉粗惡,而使臣所私貨皆良。乞下按臣廉問,自今毋得多攜玉石,煩擾道途。其貢物不堪者,治都司官罪。」從之。明年,其國額麻都抗等八王各遣使貢玉石,主客郎中陳九川簡退其粗惡者,使臣怨。通事胡士紳亦憾九川因詐為使臣奏,詞誣九川,盜玉,坐下詔獄拷訊。尚書席書、給事中解一貫等論救,不聽,竟戍邊。

十一年遣使偕土魯番、撒馬兒罕、惟密諸國來貢,稱王者至三十七人。禮官言:「舊制,恰哈密與朵顏三衛比歲一貢,貢不過三百人。三衛地近,盡許入都。哈密則十遣其二,餘留待於邊。若西域則越在萬里,素非屬國,難視三衛貢期,而所遣使人倍踰恒數。番文至二百餘通,皆以索取叛人牙木蘭為詞。竊恐託詞窺伺,以覘朝廷處分。邊臣不遵明例,概行起送,有乖法體。乞下督撫諸臣,遇諸番人入貢,分別存留起送,不得概遣入京。且嚴飭邊吏,毋避禍目前,貽患異日,貪納款之虛名,忘禦邊之實策。」帝可其奏。

故事,諸番貢物至,邊臣驗上其籍,禮官為按籍給賜。籍所不載,許自行貿易。貢使既竣,即有餘貨,責令攜歸。願入官者,禮官奏聞,給鈔。正德末,黠番猾胥交關罔利,始有貿易餘貨令市儈評直、官給絹鈔之例。至是,天方及土魯番使臣,其籍餘玉石、銼刀諸貨,固求準貢物給賞。禮官不得已,以正德間例為請,許之。

番使多賈人,來輒挾重貲與中國市。邊吏嗜賄,侵剋多端,類取償於公家。或不當其直,則咆哮不止。是歲,貢使皆黠悍,既習知中國情,且憾邊吏之侵剋也,屢訴之,禮官卻不問。鎮守甘肅中官陳浩者,當番使入貢時,令家奴王洪多索名馬、玉石諸物,使臣憾之。一日,遇洪於衢,即執詣官以證實其事。禮官言事關國體,須大有處分,以服遠人之心。乃命三法司、錦衣衛及給事中各遣官一員赴甘肅按治,洪迄獲罪。

十七年復貢,其使臣請游覽中土。禮官疑有狡心,以非故事格之。二十二年偕撒馬兒罕、土魯番、哈密、魯迷諸國貢馬及方物。後五六年一貢,迄萬曆中不絕。

天方於西域為大國,四時常似夏,無雨雹霜雪,惟露最濃,草木皆資之長養。土沃,饒慄、麥、黑黍。人皆頎碩。男子削髮,以布纏之。婦女則編髮蓋頭,不露其面。相傳回回設教之祖曰馬哈麻者,首於此地行教,死即葬焉。墓頂常有光,日夜不熄。後人遵其教,久而不衰,故人皆向善。國無苛擾,亦無刑罰,上下安和,寇賊不作,西土稱為樂國。俗禁酒。有禮拜寺,月初生,其王及臣民咸拜天,號呼稱揚以為禮。寺分四方,每方九十間,共三百六十間,皆白玉為柱,黃甘玉為地。其堂以五色石砌成,四方平頂。內用沉香大木為梁凡五,又以黃金為閣。堂中垣墉,悉以薔薇露、龍涎香和土為之。守門以二黑獅。堂左有司馬儀墓,其國稱為聖人塚。土悉寶石,圍牆則黃甘玉。兩旁有諸祖師傳法之堂,亦以石築成,俱極其壯麗。其崇奉回回教如此。

瓜果、諸畜,咸如中國。西瓜、甘瓜有一人不能舉者,桃有重四五斤者,雞、鴨有重十餘斤者,皆諸番所無也。馬哈麻墓後有一井,水清而甘。泛海者必汲以行,遇颶風,取水灑之即息。當鄭和使西洋時,傳其風物如此。其後稱王者至二三十人,其俗亦漸不如初矣。

默德那,回回祖國也,地近天方。宣德時,其酋長遣使偕天方使臣來貢,後不復至。相傳,其初國王謨罕驀德生而神靈,盡臣服西域諸國,諸國尊為別諳拔爾,猶言天使也。國中有經三十本,凡三千六百餘段。其書旁行,兼篆、草、楷三體,西洋諸國皆用之。其教以事天為主,而無像設。每日西向虔拜。每歲齋戒一月,沐浴更衣,居必易常處。隋開皇中,其國撒哈八撒阿的乾葛思始傳其教入中國。迄元世,其人遍於四方,皆守教不替。

國中城池、宮室、市肆、田園,大類中土。有陰陽、星曆、醫藥、音樂諸技。其織文、製器尤巧。寒暑應候,民殷物繁,五穀六畜咸備。俗重殺,不食豬肉。嘗以白布蒙頭,雖適他邦,亦不易其俗。

坤城,西域回回種。宣德五年,其使臣者馬力丁等來朝,貢駝馬。時有開中之令,使者即輸米一萬六千七百石於京倉中鹽。及辭還,願以所納米獻官。帝曰:「回人善營利,雖名朝貢,實圖貿易,可酬以直。」於是予帛四十匹、布倍之。其後亦嘗貢。

自成祖以武定天下,欲威制萬方,遣使四出招徠。由是西域大小諸國莫不稽顙稱臣,獻琛恐後。又北窮沙漠,南極溟海,東西抵日出沒之處,凡舟車可至者,無所不屆。自是,殊方異域鳥言侏人離之使,輻輳闕廷。歲時頒賜,庫藏為虛。而四方奇珍異寶、名禽殊獸進獻上方者,亦日增月益。蓋兼漢、唐之盛而有之,百王所莫並也。餘威及於後嗣,宣德、正統朝猶多重譯而至。然仁宗不務遠略,踐阼之初,即撤西洋取寶之船,停松花江造舟之役,召西域使臣還京,敕之歸國,不欲疲中土以奉遠人。宣德繼之,雖間一遣使,尋亦停止,以故邊隅獲休息焉。

今采故牘嘗奉貢通名天朝者,曰哈三,曰哈烈兒,曰沙的蠻,曰哈的蘭,曰掃蘭,曰乜克力,曰把力黑,曰俺力麻,曰脫忽麻,曰察力失,曰幹失,曰卜哈剌,曰怕剌,曰你沙兀兒,曰克失迷兒,曰帖必力思,曰火壇,曰火占,曰苦先,曰牙昔,曰牙兒乾,曰戎,曰白,曰兀倫,曰阿端,曰邪思城,曰捨黑,曰擺音,曰克,計二十九部。以疆域褊小,止稱地面。與哈烈、哈實哈兒、賽藍、亦力把力、失剌思、沙鹿海牙、阿速、把丹皆由哈密入嘉峪關,或三年、五年一貢,入京者不得過三十五人。其不由哈密者,更有乞兒、麻米兒、哈蘭可脫、蠟燭、也的乾、剌竹、亦不剌、因格失、迷乞兒、吉思羽奴、思哈辛十一地面,亦嘗通貢。

魯迷,去中國絕遠。嘉靖三年遣使貢獅子、西牛。給事中鄭一鵬言:「魯迷非嘗貢之邦,獅子非可育之獸,請卻之,以光聖德。」禮官席書等言:「魯迷不列《王會》,其真偽不可知。近土魯番數侵甘肅,而邊吏於魯迷冊內,察有土魯番之人。其狡詐明甚,請遣之出關,治所獲間諜罪。」帝竟納之,而令邊臣察治。

五年冬,復以二物來貢。既頒賜,其使臣言,長途跋涉,費至二萬二千餘金,請加賜。御史張祿言:「華夷異方,人物異性,留人養畜,不惟違物,抑且拂人。況養獅日用二羊,養西牛日用果餌。獸相食與食人食,聖賢皆惡之。又調御人役,日需供億。以光祿有限之財,充人獸無益之費,殊為拂經。乞返其人,卻其物,薄其賞,明中國聖人不貴異物之意。」不納。乃從禮官言,如弘治撒馬兒罕例益之。二十二年偕天方諸國貢馬及方物,明年還至甘州。會迤北賊入寇,總兵官楊信令貢使九十餘人往御,死者九人。帝聞,褫信職,命有司棺斂歸其喪。二十七年、三十三年並入貢。其貢物有珊瑚、琥珀、金剛鉆、花瓷器、鎖服、撒哈剌帳、羚羊角、西狗皮、舍列猻皮、鐵角皮之屬。


\end{pinyinscope}