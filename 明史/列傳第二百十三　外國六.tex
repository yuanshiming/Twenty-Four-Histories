\article{列傳第二百十三 外國六}

\begin{pinyinscope}
○浡泥滿剌加蘇門答剌須文達那蘇祿西洋瑣里瑣里覽邦淡巴百花彭亨那孤兒黎伐南渤利阿魯柔佛丁機宜巴喇西佛郎機和蘭

浡泥,宋太宗時始通中國。洪武三年八月命御史張敬之、福建行省都事沈秩往使。自泉州航海,閱半年抵闍婆,又踰月至其國。王馬合謨沙傲慢不為禮,秩責之,始下座拜受詔。時其國為蘇祿所侵,頗衰耗,王辭以貧,請三年後入貢。秩曉以大義,王既許諾,其國素屬闍婆,闍婆人間之,王意中沮。秩折之曰:「闍婆久稱臣奉貢,爾畏闍婆,反不畏天朝邪?」乃遣使奉表箋,貢鶴頂、生玳瑁、孔雀、梅花大片龍腦、米龍腦、西洋布、降真諸香。八月從敬之等入朝。表用金,箋用銀,字近回鶻,皆鏤之以進。帝喜,宴賚甚厚。八年命其國山川附祀福建山川之次。

永樂三年冬,其王麻那惹加那遣使入貢,乃遣官封為國王,賜印誥、敕符、勘合、錦綺、彩幣。王大悅,率妃及弟妹子女陪臣泛海來朝。次福建,守臣以聞。遣中官往宴賚,所過州縣皆宴。六年八月入都朝見,帝獎勞之。王跪致詞曰:「陛下膺天寶命,統一萬方。臣遠在海島,荷蒙天恩,賜以封爵。自是國中雨暘時順,歲屢豐登,民無災厲,山川之間,珍奇畢露,草木鳥獸,亦悉蕃育。國中耆老咸謂此聖天子覆冒所致。臣願睹天日之表,少輸悃誠,不憚險遠,躬率家屬陪臣,詣闕獻謝。」帝慰勞再三,命王妃所進中宮箋及方物,陳之文華殿。王詣殿進獻畢,自王及妃以下悉賜冠帶、襲衣。帝乃饗王於奉天門,妃以下饗於他所,禮訖送歸會同館。禮官請王見親王儀,帝令準公侯禮。尋賜王儀仗、交倚、銀器、傘扇、銷金鞍馬、金織文綺、紗羅、綾絹衣十襲,余賜賚有差。十月,王卒於館。帝哀悼,輟朝三日,遣官致祭,賻以繒帛。東宮親王皆遣祭,有司具棺郭、明器,葬之安德門外石子岡,樹碑神道。又建祠墓側,有司春秋祀以少牢,謚曰恭順。賜敕慰其子遐旺,命襲封國王。

遐旺與其叔父上言:「臣歲供爪哇片腦四十斤,乞敕爪哇罷歲供,歲進天朝。臣今歸國,乞命護送,就留鎮一年,慰國人之望。並乞定朝貢期及傔從人數。」帝悉從之,命三年一貢,傔從惟王所遣,遂敕爪哇國免其歲供。王辭歸,賜玉帶一、金百兩、銀三千兩及錢鈔、錦綺、紗羅、衾褥、帳幔、器物,餘皆有賜。以中官張謙、行人周航護行。

初,故王言:「臣蒙恩賜爵,臣境土悉屬職方,乞封國之後山為一方鎮。」新王復以為言,乃封為長寧鎮國之山。御製碑文,令謙等勒碑其上。其文曰:

上天佑啟我國家萬世無疆之基,誕命我太祖高皇帝全撫天下,休養生息,以治以教,仁聲義問,薄極照臨,四方萬國,奔走臣服,充湊於廷。神化感動之機,其妙如此。朕嗣守鴻圖,率由典式。嚴恭祗畏,協和所統。無間內外,均視一體。遐邇綏寧,亦克承予意。

乃者浡泥國王,誠敬之至,知所尊崇,慕尚聲教,益謹益虔,率其眷屬、陪臣,不遠數萬里,浮海來朝,達其志,通其欲,稽顙陳辭曰:「遠方臣妾,丕冒天子之恩,以養以息,既庶且安。思見日月之光,故不憚險遠,輒敢造廷。」又曰:「覆我者天,載我者地。使我有土地人民之奉,田疇邑井之聚,宮室之居,妻妾之樂,和味宜服,利用備器,以資其生,強罔敢侵,眾罔敢暴,實惟天子之賜。是天子功德所加,與天地並。然天仰剛見,地蹐則履,惟天子遠而難見,誠有所不通。是以遠方臣妾,不敢自外,踰歷山海,躬詣闕延,以伸其悃。」朕曰:「惟天,惟皇考,付予以天下,子養庶民。天與皇考,視民同仁,予其承天與皇考之德,惟恐弗堪,弗若汝言。」乃又拜手稽首曰:「自天子建元之載,臣國時和歲豐,山川之藏,珍寶流溢,草木之無葩者皆華而實,異禽和鳴,走獸蹌舞。國之黃叟咸曰,中國聖人德化漸暨,斯多嘉應。臣土雖遠,實天子之氓,故奮然而來覲也。」朕觀其言文貌恭,動不踰則,悅喜禮教,脫略夷習,非超然卓異者不能。稽之載籍,自古逷遠之國,奉若天道,仰服聲教,身致帝廷者有之。至於舉妻子、兄弟、親戚、陪臣頓首稱臣妾於階陛之下者,惟浡泥國王一人;西南諸蕃國長,未有如王賢者。王之至誠貫於金石,達於神明,而令名傳於悠久,可謂有光顯矣。

茲特錫封王國中之山為張寧鎮國之山,賜文刻石,以著王休,於昭萬年,其永無斁。系之詩曰:「炎海之墟,浡泥所處。煦仁漸義,有順無迕。撦撦賢王,惟化之慕。導以象胥,遹來奔赴。同其婦子、兄弟、陪臣,稽顙闕下,有言以陳。謂君猶天,遺以休樂,一視同仁,匪偏厚薄。顧茲鮮德,弗稱所云。浪舶風檣,實勞懇勤。稽古遠臣,順來怒DM。以躬或難,矧曰家室。王心亶誠,金石其堅。西南蕃長,疇與王賢。矗矗高山,以鎮王國。鑱文於石,懋昭王德。王德克昭,王國攸寧。於萬斯年,仰我大明。」

八年九月遣使從謙等入貢謝恩。明年復命謙賜其王錦綺、紗羅、彩絹凡百二十匹,其下皆有賜。十年九月,遐旺偕其母來朝。命禮官宴之會同館,光祿寺旦暮給酒饌。明日,帝饗之奉天門,王母亦有宴。越二日,再宴,賜王冠帶、襲衣,王母、王叔父以下,分賜有差。明年二月辭歸。賜金百,銀五百,鈔三千錠,錢千五百緡,錦四,綺帛紗羅八十,金織文繡、文綺衣各一,衾褥、幃幔、器物咸具。自十三年至洪熙元年四入貢,後貢使漸稀。

嘉靖九年,給事中王希文言:「暹羅、占城、琉球、爪哇、浡泥五國來貢,並道東莞。後因私攜賈客,多絕其貢。正德間,佛郎機闌入流毒,概行屏絕。曾未幾年,遽爾議復,損威已甚。」章下都察院,請悉遵舊制,毋許混冒。

萬曆中,其王卒,無嗣,族人爭立。國中殺戮幾盡,乃立其女為王。漳州人張姓者,初為其國那督,華言尊官也,因亂出奔。女主立,迎還之。其女出入王宮,得心疾,妄言父有反謀。女主懼,遣人按問其家,那督自殺。國人為訟冤,女主悔,絞殺其女,授其子官。後雖不復朝貢,而商人往來不絕。

國統十四洲,在舊港之西,自占城四十日可至。初屬爪哇,後屬暹羅,改名大泥。華人多流寓其地。嘉靖末,閩、粵海寇遺孽逋逃至此,積二千餘人。萬歷時,紅毛番強商其境,築土庫以居。其入彭湖互市者,所攜乃大泥國文也。諸風俗、物產,具詳《宋史》。

滿剌加,在占城南。順風八日至龍牙門,又西行二日即至。或云即古頓遜,唐哥羅富沙。

永樂元年十月遣中官尹慶使其地,賜以織金文綺、銷金帳幔諸物。其地無王,亦不稱國,服屬暹羅,歲輸金四十兩為賦。慶至,宣示威德及招徠之意。其酋拜里迷蘇剌大喜,遣使隨慶入朝貢方物,三年九月至京師。帝嘉之,封為滿剌加國王,賜誥印、綵幣、襲衣、黃蓋,復命慶往。其使者言:「王慕義,願同中國列郡,歲效職貢,請封其山為一國之鎮。」帝從之。製碑文,勒山上,末綴以詩曰:「西南巨海中國通,輸天灌地億載同。洗日浴月光景融,雨崖露石草木濃。金花寶鈿生青紅,有國於此民俗雍。王好善義思朝宗,願比內郡依華風。出入導從張蓋重,儀文裼襲禮虔恭。大書貞石表爾忠,爾國西山永鎮封。山君海伯翕扈從,皇考陟降在彼穹。後天監視久彌隆,爾眾子孫萬福崇。」慶等再至,其王益喜,禮待有加。

五年九月遣使入貢。明年,鄭和使其國,旋入貢。九年,其王率妻子陪臣五百四十餘人來朝。抵近郊,命中官海壽、禮部郎中黃裳等宴勞,有司供張會同館。入朝奉天殿,帝親宴之,妃以下宴他所。光祿日致牲牢上尊,賜王金繡龍衣二襲、麒麟衣一襲,金銀器、帷幔衾衣周悉具,妃以下皆有賜。將歸,賜王玉帶、儀仗、鞍馬,賜妃冠服。瀕行,賜宴奉天門,再賜玉帶、儀仗、鞍馬、黃金百、白金五百、鈔四十萬貫、錢二千六百貫、錦綺紗羅三百匹、帛千匹、渾金文綺二、金織通袖膝襴二;妃及子姪陪臣以下,宴賜有差。禮官餞於龍江驛,復賜宴龍潭驛。十年夏,其姪入謝。及辭歸,命中官甘泉偕往,旋又入貢。

十二年,王子母乾撒於的兒沙來朝,告其父訃。即命襲封,賜金幣。嗣後,或連歲,或間歲入貢以為常。

十七年,王率妻子陪臣來朝謝恩。及辭歸,訴暹羅見侵狀。帝為賜敕諭暹羅,暹羅乃奉詔。二十二年,西里麻哈剌以父沒嗣位,率妻子陪臣來朝。

宣德六年遣使者來言:「暹羅謀侵本國,王欲入朝,懼為所阻,欲奏聞,無能書者,令臣三人附蘇門答剌貢舟入訴。」帝命附鄭和舟歸國,因令和齎敕諭暹羅,責以輯睦鄰封,毋違朝命。初,三人至,無貢物,禮官言例不當賞。帝曰:「遠人越數萬里來醖不平,豈可無賜。」遂賜襲衣、彩幣,如貢使例。

八年,王率妻子陪臣來朝。抵南京,天已寒,命俟春和北上,別遣人齎敕勞賜王及妃。洎入朝,宴賚如禮。及還,有司為治舟。王復遣其弟貢駝馬方物。時英宗已嗣位,而王猶在廣東。賜敕獎王,命守臣送還國。因遣古里、真臘等十一國使臣,附載偕還。

正統十年,其使者請賜王息力八密息瓦兒丟八沙護國敕書及蟒服、傘蓋,以鎮服國人。又言:「王欲親詣闕下,從人多,乞賜一巨舟,以便遠涉。」帝悉從之。

景泰六年,速魯檀無答佛哪沙貢馬及方物,請封為王。詔給事中王暉往。已,復入貢,言所賜冠帶毀於火。命製皮弁服、紅羅常服及犀帶紗帽予之。

天順三年,王子蘇丹芒速沙遣使入貢,命給事中陳嘉猷等往封之。越二年,禮官言:「嘉猷等浮海二日,至烏豬洋,遇颶風,舟壞,飄六日至清瀾守禦所獲救。敕書無失,諸賜物悉沾水。乞重給,令使臣復往。」從之。

成化十年,給事中陳峻冊封占城王,遇安南兵據占城不得入,以所齎物至滿剌加,諭其王入貢。其使者至,帝喜,賜敕嘉獎。十七年九月,貢使言:「成化五年,貢使還,飄抵安南境,多被殺,餘黥為奴,幼者加宮刑。今已據占城地,又欲吞本國。本國以皆為王臣,未敢與戰。」適安南貢使亦至,滿剌加使臣請與廷辨。兵部言事屬既往,不足深較。帝乃因安南使還,敕責其王,並諭滿剌加,安南復侵陵,即整兵待戰。尋遣給事中林榮、行人黃乾亨冊封王子馬哈木沙為王。二人溺死,贈官賜祭,予廕,恤其家,餘敕有司海濱招魂祭,亦恤其家。復遣給事中張晟、行人左輔往。晟卒於廣東,命守臣擇一官為輔副,以終封事。

正德三年,使臣端亞智等入貢。其通事亞劉,本江西萬安人蕭明舉,負罪逃入其國,賂大通事王永、序班張字,謀往浡泥索寶。而禮部吏侯永等亦受賂,偽為符印,擾郵傳。還至廣東,明舉與端亞智輩爭言,遂與同事彭萬春等劫殺之,盡取其財物。事覺,逮入京。明舉凌遲,萬春等斬,王永減死罰米三百石,與張字、侯永並戍邊,尚書白鉞以下皆議罰。劉瑾因此罪江西人,減其解額五十名,仕者不得任京職。

後佛郎機強,舉兵侵奪其地,王蘇端媽末出奔,遣使告難。時世宗嗣位,敕責佛郎機,令還其故土。諭暹羅諸國王以救災恤鄰之義,迄無應者,滿剌加竟為所滅。時佛郎機亦遣使朝貢請封,抵廣東,守臣以其國素不列《王會》,羈其使以聞。詔予方物之直遣歸,後改名麻六甲云。

滿剌加所貢物有瑪瑙、珍珠、玳瑁、珊瑚樹、鶴頂、金母鶴頂、瑣服、白苾布、西洋布、撒哈剌、犀角、象牙、黑熊、黑猿、白麂、火雞、鸚鵡、片腦、薔薇露、蘇合油、梔子花、烏爹泥、沉香、速香、金銀香、阿魏之屬。

有山出泉流為溪,土人淘沙取錫煎成塊曰斗錫。田瘠少收,民皆淘沙捕魚為業。氣候朝熱暮寒。男女椎髻,身體黝黑,間有白者,唐人種也。俗淳厚,市道頗平。自為佛郎機所破,其風頓殊。商舶稀至,多直詣蘇門答剌。然必取道其國,率被邀劫,海路幾斷。其自販於中國者,則直達廣東香山澳,接跡不絕云。

蘇門答剌,在滿剌加之西。順風九晝夜可至。或言即漢條枝,唐波斯、大食二國地,西洋要會也。

成祖初,遣使以即位詔諭其國。永樂二年遣副使聞良輔、行人甯善賜其酋織金文綺、絨錦、紗羅招徠之。中官尹慶使爪哇,便道復使其國。三年,鄭和下西洋,復有賜。和未至,其酋宰奴里阿必丁已遣使隨慶入朝,貢方物。詔封為蘇門答剌國王,賜印誥、綵幣、襲衣。遂比年入貢,終成祖世不絕。鄭和凡三使其國。

先是,其王之父與鄰國花面王戰,中矢死。王子年幼,王妻號於眾曰:「孰能為我報仇者,我以為夫,與共國事。」有漁翁聞之,率國人往擊,馘其王而還。王妻遂與之合,稱為老王。既而王子年長,潛與部領謀,殺老王而襲其位。老王弟蘇幹剌逃山中,連年率眾侵擾。十三年,和復至其國,蘇幹剌以頒賜不及己,怒,統數萬人邀擊。和勒部卒及國人禦之,大破賊眾,追至南渤利國,俘以歸。其王遣使入謝。

宣德元年遣使入賀。五年,帝以外蕃貢使多不至,遣和及王景弘遍歷諸國,頒詔曰:「朕恭膺天命,祗承太祖高皇帝、太宗文皇帝、仁宗昭皇帝大統,君臨萬邦,體祖宗之至仁,普輯寧於庶類。已大赦天下,紀元宣德。爾諸蕃國,遠在海外,未有聞知。茲遣太監鄭和、王景弘等齎詔往諭,其各敬天道,撫人民,共享太平之福。」凡歷二十餘國,蘇門答剌與焉。明年遣使入貢者再。八年貢麒麟。

九年,王弟哈利之漢來朝,卒於京。帝憫之,贈鴻臚少卿,賜誥,有司治喪葬,置守塚戶。時景弘再使其國,王遣弟哈尼者罕隨入朝。明年至,言王老不能治事,請傳位於子。乃封其子阿卜賽亦的為國王,自是貢使漸稀。

成化二十二年,其使者至廣東,有司驗無印信勘合,乃藏其表於庫,卻還其使。別遣番人輸貢物京師,稍有給賜。自後貢使不至。

迨萬曆間,國兩易姓。其時為王者,人奴也。奴之主為國大臣,握兵柄。奴桀黠,主使牧象,象肥。俾監魚稅,日以大魚奉其主。主大喜,俾給事左右。一日隨主入朝,見王尊嚴若神,主鞠躬惟謹,出謂主曰:「主何恭之甚?」主曰:「彼王也,焉敢抗。」曰:「主第不欲王爾,欲之,主即王矣。」主詫,叱退之。他日又進曰:「王左右侍衛少,主擁重兵出鎮,必入辭,請以奴從。主言有機事,乞屏左右,王必不疑。奴乘間剌殺之,奉主為王,猶反掌耳。」主從之,奴果殺王,大呼曰:「王不道,吾殺之。吾主即王矣。敢異議者,齒此刃!」眾懾服不敢動,其主遂篡位,任奴為心腹,委以兵柄。未幾,奴復殺主而代之。乃大為防衛,拓其宮,建六門,不得闌入,雖勛貴不得帶刀上殿。出乘象,象駕亭而帷其外,如是者百餘,俾人莫測王所在。

其國俗頗淳,出言柔媚,惟王好殺。歲殺十餘人,取其血浴身,謂可除疾。貢物有寶石、瑪絜、水晶、石青、回回青、善馬、犀牛、龍涎香、沉香、速香、木香、丁香、降真香、刀、弓、錫、鎖服、胡椒、蘇木、硫黃之屬。貨舶至,貿易稱平。地本瘠,無麥有禾,禾一歲二稔。四方商賈輻輳。華人往者,以地遠價高,獲利倍他國。其氣候朝如夏,墓各秋,夏有瘴氣。婦人裸體,惟腰圍一布。其他風俗類滿剌加。篡弒後,易國名曰啞齊。

須文達那,洪武十六年,國王殊旦麻勒兀達朌遣使俺八兒來朝,貢馬二匹,幼苾布十五匹,隔著布、入的力布各二匹,花滿直地二,番綿紬直地二,兜羅綿二斤,撒剌八二個,幼賴革著一個,撒哈剌一個,及薔薇水、沉香、降香、速香諸物。命賜王《大統曆》、綺羅、寶鈔,使臣襲衣。或言須文達那即蘇門答剌,洪武時所更,然其貢物與王之名皆不同,無可考。

蘇祿,地近浡泥、闍婆。洪武初,發兵侵浡泥,大獲,以闍婆援兵至,乃還。

永樂十五年,其國東王巴都葛叭哈剌、西王麻哈剌叱葛剌麻丁、峒王妻叭都葛巴剌卜並率其家屬頭目凡三百四十餘人,浮海朝貢,進金鏤表文,獻珍珠、寶石、玳瑁諸物。禮之若滿剌加,尋並封為國王。賜印誥、襲衣、冠帶及鞍馬、儀仗器物,其從者亦賜冠帶有差。居二十七日,三王辭歸。各賜玉帶一,黃金百,白金二千,羅錦文綺二百,帛三百,鈔萬錠,錢二千緡,金繡蟒龍、麒麟衣各一。東王次德州,卒於館。帝遣官賜祭,命有司營葬,勒碑墓道,謚曰恭定,留妻妾傔從十人守墓,俟畢三年喪遣歸。乃遣使齎敕諭其長子都馬含曰:「爾父知尊中國,躬率家屬陪臣,遠涉海道,萬里來朝。朕眷其誠悃,已錫王封,優加賜賚,遣官護歸。舟次德州,遭疾殞歿。朕聞之,深為哀悼,已葬祭如禮。爾以嫡長,為國人所屬,宜即繼承,用綏籓服。今特封爾為蘇祿國東王。爾尚益篤忠貞,敬承天道,以副眷懷,以繼爾父之志。欽哉。」

十八年,西王遣使入貢。十九年,東王母遣王叔叭都加蘇里來朝,貢大珠一,其重七兩有奇。二十一年,東王妃還國,厚賜遣之。明年入貢,自後不復至。萬曆時,佛郎機屢攻之,城據山險,迄不能下。

其國,於古無所考。地瘠寡粟麥,民率食魚蝦,煮海為鹽,釀蔗為酒,織竹為布。氣候常熱。有珠池,夜望之,光浮水面。土人以珠與華人市易,大者利數十倍。商舶將返,輒留數人為質,冀其再來。其旁近國名高藥,出玳瑁。

西洋瑣里,洪武二年命使臣劉叔勉以即位詔諭其國。三年平定沙漠,復遣使臣頒詔。其王別里提遣使奉金葉表,從叔勉獻方物。賜文綺、紗羅諸物甚厚,並賜《大統曆》。

成祖頒即位詔於海外諸國,西洋亦與焉。永樂元年命副使聞良輔、行人甯善使其國,賜絨錦、文綺、紗羅。已,復命中官馬彬往使,賜如前。其王即遣使來貢,附載胡椒與民市。有司請徵稅,命勿徵。二十一年偕古里、阿丹等十五國來貢。

瑣里,近西洋瑣里而差小。洪武三年,命使臣塔海帖木兒齎詔撫諭其國。五年,王卜納的遣使奉表朝貢,並獻其國土地山川圖。帝顧中書省臣曰:「西洋諸國素稱遠蕃,涉海而來,難計歲月。其朝貢無論疏數,厚往薄來可也。」乃賜《大統曆》及金織文綺、紗羅各四匹,使者亦賜幣帛有差。

覽邦,在西南海中。洪武九年,王昔里馬哈剌札的剌札遣使奉表來貢。詔賜其王織金文綺、紗羅,使者宴賜如制。永樂、宣德中,嘗附鄰國朝貢。其地多沙礫,麻麥之外無他種。商賈鮮至。山坦迤無峰巒,水亦淺濁。俗好佛,勤賽祀。厥貢,孔雀、馬、檀香、降香、胡椒、蘇木。交易用錢。

淡巴,亦西南海中國。洪武十年,其王佛喝思羅遣使上表,貢方物,賜賚有差。其國,石城瓦屋。王乘輿,官跨馬,有中國威儀。土衍水清,草木暢茂,畜產甚伙。男女勤於耕織,市有貿易,野無寇盜,稱樂土焉。厥貢,苾布、兜羅綿被、沉香、速香、檀香、胡椒。

百花,居西南海中。洪武十一年,其王剌丁剌者望沙遣使奉金葉表,貢白鹿、紅猴、龜筒、玳瑁、孔雀、鸚鵡、哇哇倒掛鳥及胡椒、香、蠟諸物。詔賜王及使者綺、幣、襲衣有差。國中氣候恒燠,無霜雪,多奇花異卉,故名百花。民富饒,尚釋教。

彭亨,在暹羅之西。洪武十一年,其王麻哈剌惹答饒遣使齎金葉表,貢番奴六人及方物,宴賚如禮。永樂九年,王巴剌密瑣剌達羅息泥遣使入貢。十年,鄭和使其國。十二年,復入貢。十四年,與古里、爪哇諸國偕貢,復令鄭和報之。

其國,土田沃,氣候常溫,米粟饒足,煮海為鹽,釀椰漿為酒。上下親狎,無寇賊。然惑於鬼神,刻香木為像,殺人祭賽,以禳災祈福。所貢有象牙、片腦、乳香、速香、檀香、胡椒、蘇木之屬。

至萬曆時,有柔佛國副王子娶彭亨王女,將婚,副王送子至彭亨,彭亨王置酒,親戚畢會。婆羅國王子為彭亨王妹婿,舉觴獻副王,而手指有巨珠甚美,副王欲之,許以重賄。王子靳不予,副王怒,即歸國發兵來攻。彭亨人出不意,不戰自潰。王與婆羅王子奔金山。浡泥國王,王妃兄也,聞之,率眾來援。副王乃大肆焚掠而去。當是時,國中鬼哭三日,人民半死。浡泥王迎其妹歸,彭亨王隨之,而命其長子攝國。已,王復位,次子素凶悍,遂毒殺其父,弒其兄自立。

那孤兒,以蘇門答剌之西,壤相接。地狹,止千餘家。男子皆以墨剌面為花獸之狀,故又名花面國。猱頭裸體,男女止單布圍腰。然俗淳,田足稻禾,強不侵弱,富不驕貧,悉自耕而食,無寇盜。永樂中,鄭和使其國。其酋長常入貢方物。

黎伐,在那孤兒之西。南大山,北大海,西接南渤利。居民三千家,推一人為主。隸蘇門答剌,聲音風俗多與之同。永樂中,嘗隨其使臣入貢。

南渤利,在蘇門答剌之西。順風三日夜可至。王及居民皆回回人,僅千餘家。俗朴實,地少穀,人多食魚蝦。西北海中有山甚高大,曰帽山,其西復大海,名那沒黎洋,西來洋船俱望此山為準。近山淺水內,生珊瑚樹,高者三尺許。永樂十年,其王馬哈麻沙遣使附蘇門答剌使入貢。賜其使襲衣,賜王印誥、錦綺、羅紗、綵幣。遣鄭和撫諭其國。終成祖時,比年入貢,其王子沙者罕亦遣使入貢。宣德五年,鄭和遍賜諸國,南渤利亦與焉。

阿魯,一名啞魯,近滿剌加。順風三日夜可達。風俗、氣候大類蘇門答剌。田瘠少收,盛藝芭蕉、椰子為食。男女皆裸體,以布圍腰。永樂九年,王速魯唐忽先遣使附古里諸國入貢。賜其使冠帶、彩幣、寶鈔,其王亦有賜。十年,鄭和使其國。十七年,王子段阿剌沙遣使入貢。十九年、二十一年,再入貢。宣德五年,鄭和使諸蕃,亦有賜。其後貢使不至。

柔佛,近彭亨,一名烏丁礁林。永樂中,鄭和遍歷西洋,無柔佛名。或言和曾經東西竺山,今此山正在其地,疑即東西竺。萬曆間,其酋好構兵,鄰國丁機宜、彭亨屢被其患。華人販他國者多就之貿易,時或邀至其國。

國中覆茅為屋,列木為城,環以池。無事通商於外,有事則召募為兵,稱強國焉。地不產穀,常易米於鄰壤。男子薙髮徒跣佩刀,女子蓄髮椎結,其酋則佩雙刀。字用茭曌葉,以刀刺之。婚姻亦論門閥。王用金銀為食器,群下則用磁。無匕箸。俗好持齋,見星方食。節序以四月為歲首。居喪,婦人薙髮,男子則重薙,死者皆火葬。所產有犀、象、玳瑁、片腦、沒藥、血竭、錫、蠟、嘉文簟、木棉花、檳榔、海菜、窩燕、西國米、跂吉柿之屬。

始其國吉寧仁為大庫,忠於王,為王所倚信。王弟以兄疏己,潛殺之。後出行墮馬死,左右咸見吉寧仁為祟,自是家家祀之。

丁機宜,爪哇屬國也,幅員甚狹,僅千餘家。柔佛黠而雄,丁機宜與接壤,時被其患。後以厚幣求婚,稍獲寧處。其國以木為城。酋所居,旁列鐘鼓樓,出入乘象。以十月為歲首。性好潔,酋所食啖,皆躬自割烹。民俗類爪哇,物產悉如柔佛。酒禁甚嚴,有常稅。然大家皆不飲,維細民無籍者飲之,其曹偶咸非笑。婚者,男往女家持其門戶,故生女勝男。喪用火葬。華人往商,交易甚平。自為柔佛所破,往者亦鮮。

巴剌西,去中國絕遠。正德六年遣使臣沙地白入貢,言其國在南海,始奉王命來朝,舟行四年半,遭風飄至西瀾海,舟壞,止存一小艇,又飄流八日,至得吉零國,居一年。至秘得,居八月。乃遵陸行,閱二十六日抵暹羅,以情告王,獲賜日給,且賜婦女四人,居四年。迄今年五月始附番舶入廣東,得達闕下。進金葉表,貢祖母綠一,珊瑚樹、琉璃瓶、玻璃盞各四,及瑪瑙珠、胡黑丹諸物。帝嘉其遠來,賜賚有加。

佛郎機,近滿剌加。正德中,據滿剌加地,逐其王。十三年遣使臣加必丹末等貢方物,請封,始知其名。詔給方物之直,遣還。其人久留不去,剽劫行旅,至掠小兒為食。已而夤緣鎮守中貴,許入京。武宗南巡,其使火者亞三因江彬侍帝左右。帝時學其語以為戲。其留懷遠驛者,益掠買良民,築室立寨,為久居計。

十五年,御史丘道隆言:「滿剌加乃敕封之國,而佛郎機敢併之,且啖我以利,邀求封貢,決不可許。宜卻其使臣,明示順逆,令還滿剌加疆土,方許朝貢。倘執迷不悛,必檄告諸蕃,聲罪致討。」御史何鰲言:「佛郎機最凶狡,兵械較諸蕃獨精。前歲駕大舶突入廣東會城,礮聲殷地。留驛者違制交通,入都者桀驁爭長。今聽其往來貿易,勢必爭鬥殺傷,南方之禍殆無紀極。祖宗朝貢有定期,防有常制,故來者不多。近因布政吳廷舉謂缺上供香物,不問何年,來即取貨。致番舶不絕於海澨,蠻人雜遝於州城。禁防既疏,水道益熟。此佛郎機所以乘機突至也。乞悉驅在澳番舶及番人潛居者,禁私通,嚴守備,庶一方獲安。」疏下禮部,言:「道隆先宰順德,鰲即順德人,故深晰利害。宜俟滿剌加使臣至,廷詰佛郎機侵奪鄰邦、擾亂內地之罪,奏請處置。其他悉如御史言。」報可。

亞三侍帝驕甚。從駕入都,居會同館。見提督主事梁焯,不屈膝。焯怒,撻之。彬大詬曰:「彼嘗與天子嬉戲,肯跪汝小官邪?」明年,武宗崩,亞三下吏。自言本華人,為番人所使,乃伏法,絕其朝貢。其年七月,又以接濟朝使為詞,攜土物求市。守臣請抽分如故事,詔復拒之。其將別都盧既以巨礮利兵肆掠滿剌加諸國,橫行海上,復率其屬疏世利等駕五舟,擊破巴西國。

嘉靖二年遂寇新會之西草灣,指揮柯榮、百戶王應恩禦之。轉戰至稍州,向化人潘丁茍先登,眾齊進,生擒別都盧、疏世利等四十二人,斬首三十五級,獲其二舟。餘賊復率三舟接戰。應恩陣亡,賊亦敗遁。官軍得其礮,即名為佛郎機,副使汪鋐進之朝。九年秋,鋐累官右都御史,上言:「今塞上墩臺城堡未嘗不設,乃冠來輒遭蹂躪者,蓋墩臺止尞望,城堡又無制遠之具,故往往受困。當用臣所進佛郎機,其小止二十斤以下,遠可六百步者,則用之墩臺。每墩用其一,以三人守之。其大至七十斤以上,遠可五六里者,則用之城堡。每堡用其三,以十人守之。五里一墩,十里一堡,大小相依,遠近相應,寇將無所容足,可坐收不戰之功。」帝悅,即從之。火礮之有佛郎機自此始。然將士不善用,迄莫能制寇也。

初,廣東文武官月俸多以番貨代,至是貨至者寡,有議復許佛郎機通市者。給事中王希文力爭,乃定令,諸番貢不以時及勘合差失者,悉行禁止,由是番舶幾絕。巡撫林富上言:「粵中公私諸費多資商稅,番舶不至,則公私皆窘。今許佛郎機互市有四利。祖宗時諸番常貢外,原有抽分之法,稍取其餘,足供御用,利一。兩粵比歲用兵,庫藏耗竭,籍以充軍餉,備不虞,利二。粵西素仰給粵東,小有徵發,即措辦不前,若番舶流通,則上下交濟,利三。小民以懋遷為生,持一錢之貨,即得展轉販易,衣食其中,利四。助國裕民,兩有所賴,此因民之利而利之,非開利孔為民梯禍也。」從之。自是佛郎機得入香山澳為市,而其徒又越境商於福建,往來不絕。

至二十六年,朱紈為巡撫,嚴禁通番。其人無所獲利,則整眾犯漳州之月港、浯嶼。副使柯喬等禦卻之。二十八年又犯詔安。官軍迎擊於走馬溪,生擒賊首李光頭等九十六人,餘遁去。紈用便宜斬之,怨紈者御史陳九德遂劾其專擅。帝遣給事中杜汝禎往驗,言此滿剌加商人,歲招海濱無賴之徒,往來鬻販,無僭號流劫事,紈擅自行誅,誠如御史所劾。紈遂被逮,自殺。蓋不知滿剌加即佛郎機也。

自紈死。海禁復弛,佛郎機遂縱橫海上無所忌。而其市香山澳、壕鏡者,至築室建城,雄踞海畔,若一國然,將吏不肖者反視為外府矣。壕鏡在香山縣南虎跳門外。先是,暹羅、占城、爪哇、琉球、浡泥諸國互市,俱在廣州,設市舶司領之。正德時,移於高州之電白縣。嘉靖十四年,指揮黃慶納賄,請於上官,移之壕鏡,歲輸課二萬金,佛郎機遂得混入。高棟飛甍,櫛比相望,閩、粵商人趨之若鶩。久之,其來益眾。諸國人畏而避之,遂專為所據。四十四年偽稱滿刺加入貢。已,改稱蒲都麗家。守臣以聞,下部議,言必佛郎機假託,乃卻之。

萬曆中,破滅呂宋,盡擅閩、粵海上之利,勢益熾。至三十四年,又於隔水青州建寺,高六七丈,閎敞奇閟,非中國所有。知縣張大猷請毀其高墉,不果。明年,番禺舉人盧廷龍會試入都,請盡逐澳中諸番,出居浪白外海,還我壕鏡故地,當事不能用。番人既築城,聚海外雜番,廣通貿易,至萬餘人。吏其土者,皆畏懼莫敢詰,甚有利其寶貨,佯禁而陰許之者。總督戴耀在事十三年,養成其患。番人又潛匿倭賊,敵殺官軍。四十二年,總督張鳴岡檄番人驅倭出海,因上言:「粵之有澳夷,猶疽之在背也。澳之有倭賊,猶虎之傅翼也。今一旦驅斥,不費一矢,此聖天子威德所致。惟是倭去而番尚存,有謂宜剿除者,有謂宜移之浪白外洋就船貿易者,顧兵難輕動。而壕鏡在香山內地,官軍環海而守,彼日食所需,咸仰於我,一懷異志,我即制其死命。若移之外洋,則巨海茫茫,奸宄安詰?制禦安施?似不如申明約束,內不許一奸闌出,外不許一倭闌入,無啟釁,無弛防,相安無患之為愈也。」部議從之。居三年,設參將於中路雍陌營,調千人戍之,防禦漸密。天啟元年,守臣慮其終為患,遣監司馮從龍等毀其所築青州城,番亦不敢拒。

其時,大西洋人來中國,亦居此澳。蓋番人本求市易,初無不軌謀,中朝疑之過甚,迄不許其朝貢,又無力以制之,故議者紛然。然終明之世,此番固未嘗為變也。其人長身高鼻,貓晴鷹嘴,拳髮赤鬚,好經商,恃強陵轢諸國,無所不往。後又稱乾系臘國。所產多犀象珠貝。衣服華潔,貴者冠,賤者笠,見尊長輒去之。初奉佛教,後奉天主教。市易但伸指示數,雖累千金不立約契,有事指天為誓,不相負。自滅滿剌加、巴西、呂宋三國,海外諸蕃無敢與抗者。

和蘭,又名紅毛番,地近佛郎機。永樂、宣德時,鄭和七下西洋,歷諸番數十國,無所謂和蘭者。其人深目長鼻,髮眉鬚皆赤,足長尺二寸,頎偉倍常。

萬曆中,福建商人歲給引往販大泥、呂宋及咬留吧者,和蘭人就諸國轉販,未敢窺中國也。自佛郎機市香山,據呂宋,和蘭聞而慕之。二十九年駕大艦,攜巨礮,直薄呂宋。呂宋人力拒之,則轉薄香山澳。澳中人數詰問,言欲通貢市,不敢為寇。當事難之。稅使李道即召其酋入城,遊處一月,不敢聞於朝,乃遣還。澳中人慮其登陸,謹防禦,始引去。

海澄人李錦及奸商潘秀、郭震,久居大泥,與和蘭人習。語及中國事,錦曰:「若欲通貢市,無若漳州者。漳南有彭湖嶼,去海遠,誠奪而守之,貢市不難成也。」其酋麻韋郎曰:「守臣不許,奈何?」曰:「稅使高寀嗜金銀甚,若厚賄之,彼特疏上聞,天子必報可,守臣敢抗旨哉。」酋曰:「善。」錦乃代為大泥國王書,一移寀,一移兵備副使,一移守將,俾秀、震齎以來。守將陶拱聖大駭,亟白當事,繫秀於獄,震遂不敢入。初,秀與酋約,入閩有成議,當遣舟相聞,而酋卞急不能待,即駕二大艦,直抵彭湖。時三十二年之七月。汛兵已撤,如入無人之墟,遂伐木築舍為久居計。錦亦潛入漳州偵探,詭言被獲逃還,當事已廉知其狀,並繫獄。已而議遣二人諭其酋還國,許以自贖,且拘震與俱。三人既與酋成約,不欲自彰其失,第云「我國尚依違未定」。而當事所遣將校詹獻忠齎檄往諭者,乃多攜幣帛、食物,覬其厚酬。海濱人又潛載貨物往市,酋益觀望不肯去。當事屢遣使諭之,見酋語輒不競,愈為所慢。而寀己遣心腹周之範詣酋,說以三萬金饋寀,即許貢市,酋喜與之。盟已就矣,會總兵施德政令都司沈有容將兵往諭。有容負膽智,大聲論說,酋心折,乃曰:「我從不聞此言。」其下人露刃相詰,有容無所懾,盛氣與辨,酋乃悔悟,令之範還所贈金,止以哆羅嗹、玻璃器及番刀、番酒饋寀,乞代奏通市。寀不敢應,而撫、按嚴禁奸民下海,犯者必誅,由是接濟路窮,番人無所得食,十月末揚帆去。巡撫徐學聚劾秀、錦等罪,論死、遣戍有差。

然是時佛郎機橫海上,紅毛與爭雄,復泛舟東來,攻破美洛居國,與佛郎機分地而守。後又侵奪臺灣地,築室耕田,久留不去,海上奸民,闌出貨物與市。已,又出據彭湖,築城設守,漸為求市計。守臣懼禍,說以毀城遠徙,即許互市。番人從之,天啟三年果毀其城,移舟去。巡撫商周祚以遵諭遠徙上聞,然其據臺灣自若也。已而互市不成,番人怨,復築城彭湖,掠漁舟六百餘艘,俾華人運土石助築。尋犯廈門,官軍禦之,俘斬數十人,乃詭詞求款。再許毀城遠徙,而修築如故。已,又泊風櫃仔,出沒浯嶼、白坑、東椗、莆頭、古雷、洪嶼、沙洲、甲洲間,要求互市。而海寇李旦復助之,濱海郡邑為戒嚴。

其年,巡撫南居益初至,謀討之。上言:「臣入境以來,聞番船五艘續至,與風櫃仔船合,凡十有一艘,其勢愈熾。有小校陳士瑛者,先遣往咬留吧宣諭其王,至三角嶼遇紅毛船,言咬留吧王已往阿南國,因與士瑛偕至大泥,謁其王。王言咬留吧國主已大集戰艦,議往彭湖求互市,若不見許,必至構兵。蓋阿南即紅毛番國,而咬留吧、大泥與之合謀,必不可以理諭。為今日計,非用兵不可。」因列上調兵足餉方略,部議從之。四年正月遣將先奪鎮海港而城之,且築且戰,番人乃退守風櫃城。居益增兵往助,攻擊數月,寇猶不退,乃大發兵,諸軍齊進。寇勢窘,兩遣使求緩兵,容運米入舟即退去。諸將以窮寇莫追,許之,遂揚帆去。獨渠帥高文律等十二人據高樓自守,諸將破擒之,獻俘於朝。彭湖之警以息,而其據臺灣者猶自若也。

崇禎中,為鄭芝龍所破,不敢窺內地者數年,乃與香山佛郎機通好,私貿外洋。十年駕四舶,由虎跳門薄廣州,聲言求市。其酋招搖市上,奸民視之若金穴,蓋大姓有為之主者。當道鑒壕鏡事,議驅斥,或從中撓之。會總督張鏡心初至,力持不可,乃遁去。已,為奸民李葉榮所誘,交通總兵陳謙為居停出入。事露,葉榮下吏。謙自請調用以避禍,為兵科凌義渠等所劾,坐逮訊。自是,奸民知事終不成,不復敢勾引,而番人猶據臺灣自若。

其本國在西洋者,去中華絕遠,華人未嘗至。其所恃惟巨舟大礮。舟長三十丈,廣六丈,厚二尺餘,樹五桅,後為三層樓。旁設小囪置銅礮。桅下置二丈巨鐵礮,發之可洞裂石城,震數十里,世所稱紅夷礮,即其製也。然以舟大難轉,或遇淺沙,即不能動。而其人又不善戰,故往往挫衄。其所役使名烏鬼。入水不沉,走海面若平地。其柁後置照海鏡,大徑數尺,能照數百里。其人悉奉天主教。所產有金、銀、琥珀、瑪瑙、玻璃、天鵝絨、瑣服、哆囉嗹。國土既富,遇中國貨物當意者,不惜厚資,故華人樂與為市。


\end{pinyinscope}