\article{列傳第四十五}

\begin{pinyinscope}
金純張本郭敦郭璡鄭辰柴車劉中敷孫機張鳳周瑄子紘楊鼎翁世資黃鎬胡拱辰陳俊林鶚潘榮夏時正

金純,字德修,泗州人。洪武中國子監生。以吏部尚書杜澤薦,授吏部文選司郎中。三十一年出為江西布政司右參政。成祖即位,以蹇義薦,召為刑部右侍郎。時將營北京,命採木湖廣。永樂七年從巡北京。八年從北征,遷左侍郎。

九年命與宋禮同治會通河,又同徐亨、蔣廷瓚浚魚王口黃河故道。初,太祖用兵梁、晉間,使大將軍徐達開塌場口,通河於泗。又開濟寧西耐牢坡引曹、鄆河水,以通中原之運。其後故道浸塞,至是純疏治之。自開封北引水達鄆城,入塌場,出穀亭北十里為永通、廣運二閘。十四年改禮部左侍郎。越二月,進尚書。十五年從巡北京。十九年同給事中葛紹祖巡撫四川。仁宗即位,改工部。居數月,又改刑部。明年兼太子賓客。

宣德三年,純有疾,帝命醫視療。稍間,免其朝參,俾護疾視事。會暑,敕法司理滯囚。純數從朝貴飲,為言官所劾。帝怒曰:「純以疾不朝而燕於私,可乎?」命繫錦衣獄。既念純老臣,釋之,落太子賓客。八月予致仕去。

純在刑部,仁宗嘗諭純:「法司近尚羅織,言者輒以誹謗得罪,甚無謂。自今告誹謗者勿論。」純亦務寬大,每誡屬吏不得妄椎擊人。故當純時,獄無瘐死者。正統五年卒。贈山陽伯。

張本,字致中,東阿人。洪武中,自國子生授江都知縣。燕兵至揚州,御史王彬據城抗,為守將所縛。本率父老迎降。成祖以滁、泰二知州房吉、田慶成率先歸附,命與本並為揚州知府,偕見任知府譚友德同涖府事。尋擢本江西布政司右參政。

永樂四年召為工部左侍郎。坐事免官,冠帶辦事。明年五月復官。尋以奏牘書銜誤左為右,為給事中所劾。帝命改授本部右侍郎而宥其罪。

七年,皇太子監國,奏為刑部右侍郎。善摘奸。命督北河運。躬自相視,立程度,舟行得無滯。會疾作,太子賜之狐裘冠鈔,遣醫馳視。十九年將北征,命本及王彰分往兩直隸、山東、山西、河南,督有司造車挽運。明年即命本督北征餉。

仁宗即位,拜南京兵部尚書兼掌都察院事。召見,言時政得失,且請嚴飭武備。帝嘉納之,遂留行在兵部。

宣德初,工部侍郎蔡信乞征軍匠家口隸錦衣衛。本言:「軍匠二萬六千人,屬二百四十五衛所,為匠者暫役其一丁。若盡取以來,家以三四丁計之,數近十萬。軍伍既缺,人情驚駭,不可。」帝善本言。

征漢庶人,從調兵食。庶人就擒,命撫輯其眾,而錄其餘黨。還以軍政久敝,奸人用貸脫籍,而援平民實伍,言於帝。擇廷臣四出釐正之。時馬大孳息,畿內軍民為畜牧所困。本請分牧於山東、河南及大名諸府。山東、河南養馬自此始。晉王濟熿坐不軌奪爵,本奉命散其護衛軍於邊鎮。

四年命兼太子賓客。戶部以官田租減,度支不給,請減外官俸及生員軍士月給。帝以軍士艱,不聽減。餘下廷議,本等持不可,乃止。陽武侯薛祿城獨石諸戍成,本往計守禦之宜。還奏稱旨,命兼掌戶部。本慮邊食不足,而諸邊比歲稔,請出絲麻布帛輸邊易穀,多者三四十萬石,少者亦十萬石,儲偫頓充。六年病卒,賜賻三萬緡,葬祭甚厚。

本廉介有執持,尚刻少恕。錄高煦黨,脅從者多不免。成祖宴近臣,銀器各一案,因以賜之。獨本案設陶器,諭曰:「卿號『窮張』,銀器無所用。」本頓首謝,其為上知如此。

郭敦,字仲厚,堂邑人。洪武中,以鄉舉入太學,授戶部主事。遷衢州知府,多惠政。衢俗,貧者死不葬,輒焚其屍。敦為厲禁,且立義阡,俗遂革。禁民聚淫祠。敦疾,民勸弛其禁,弗聽,疾亦瘳。在衢七年。永樂初,坐累徵,耆老數百人伏闕乞留,不得。後廷臣言敦廉正,召補監察御史。遷河南左參政,調陜西。十六年春,胡濙言敦有大臣體,擢禮部右侍郎兼太僕寺卿,偕給事中陶衎巡撫順天。二十年督北征餉。

仁宗即位,以大行喪不齋宿,降太僕卿。旋進戶部左侍郎,兼詹事府少詹事。宣德二年進尚書。陜西旱,命與隆平侯張信整飭庶務當行者,同三司官計議奏行。敦乃請蠲逋賦,振貧乏,考黜貪吏,罷不急之務,凡十數事。悉從之。歲餘,召還。在部多所興革,罷王田之奪民業者,令民開荒不起科。建漕運議,民運至瓜洲、儀真,資衛卒運至京。民甚便之。

敦事親孝,持身廉。同官有為不義者,輒厲色待之,其人悔謝乃已。性好學,公退,手不釋卷。六年,卒官,年六十二。

郭璡,字時用,初名進,新安人。永樂初,以太學生擢戶部主事。歷官吏部左、右侍郎。仁宗即位,命兼詹事府少詹事,更名璡。

宣宗初,掌行在詹事府。吏部尚書蹇義老,輟部務,帝欲以璡代。璡厚重勤敏,然寡學術。楊士奇言恐璡不足當之,宜妙擇大臣通經術知今古者,帝乃止。踰年,卒為尚書。諭以呂蒙正夾袋,虞允文材館錄故事。璡由是留意人才。識進士李賢輔相器,授吏部主事,後果為名相。時外官九年考滿,部民走闕下乞留,輒增秩復任。璡慮有妄者,請覆實。從之。

璡雖長六卿,然望輕。又政歸內閣,自布政使至知府闕,聽京官三品以上薦舉;既又命御史、知縣,皆聽京官五品以上薦舉。要職選擢,皆不關吏部。正統初,左通政陳恭言:「古者擇任庶官,悉由選部,職任專而事體一。今令朝臣各舉所知,恐開私謁之門,長奔競之風,乞杜絕,令歸一。」下吏部議。璡遜謝不敢當,事遂寢。

正統六年,御史曹恭以災異請罷大臣不職者。帝命科道官參議。璡及尚書吳中、侍郎李庸等被劾者二十人。璡等自陳,帝切責而宥之。璡子亮受賂為人求官。事覺,御史孫毓等劾璡。乃令璡致仕,而以王直代。

鄭辰,字文樞,浙江西安人。永樂四年進士,授監察御史。江西安福民告謀逆事,命辰往廉之,具得誣狀。福建番客殺人,復命辰往。止坐首惡,釋其餘。南京敕建報恩寺,役囚萬人。蜚語言役夫謗訕,恐有變,命辰往驗。無實,無一得罪者。谷庶人謀不軌,復命辰察之,盡得其蹤跡。帝語方賓曰:「是真國家耳目臣矣。」十六年超遷山西按察使,糾治貪濁不少貸。潞州盜起,有司以叛聞,詔發兵討捕。辰方以事朝京師,奏曰:「民苦徭役而已,請無發兵。」帝然之。還則屏騶從,親入山谷撫諭。盜皆感泣,復為良民。禮部侍郎蔚綬轉粟給山海軍,辰統山西民輦任。民勞,多逋耗,綬令即山海貸償之。辰曰:「山西民貧而悍,急之恐生變。不如緩之,使自通有無。」用其言,卒無逋者。丁內艱歸,軍民詣御史乞留。御史以聞,服闋還舊任。

宣德三年召為南京工部右侍郎。初,兩京六部堂官缺,帝命廷臣推方面官堪內任者。蹇義等薦九人。獨辰及邵、傅啟讓,帝素知其名,即真授,餘試職而已。

英宗即位,分遣大臣考察天下方面官。辰往四川、貴州、雲南,悉奏罷其不職者。雲南布政使周璟居妻喪,繼娶。辰劾其有傷風教,璟坐免。正統二年,奉命振南畿、河南饑。時河堤決,即命辰伺便修塞。或議自大名開渠,引諸水通衛河,利灌輸。辰言勞民不便,事遂寢。遷兵部左侍郎,與豐城侯李彬轉餉宣府、大同。鎮守都督譚廣撓令,劾之,事以辦。八年得風疾,告歸。明年卒。

辰為人重義輕財。初登進士,產悉讓兄弟。在山西與同僚杜僉事有違言。杜卒,為治喪,資遣其妻子。

柴車,字叔輿,錢塘人。永樂二年,以舉人授兵部武選司主事,歷員外郎。八年,帝北征,從尚書方賓扈行。還遷江西右參議。坐事,左遷兵部郎中,出知岳州府,復入為郎中。

宣德五年擢兵部侍郎。明年,山西巡按御史張勖言,大同屯田多為豪右占據,命車往按。得田幾二千頃,還之軍。

英宗初,西鄙不靖。以車廉幹,命協贊甘肅軍務。調軍給餉,悉得事宜。初,朵兒只伯寇涼州,副總兵劉廣喪師。不以實聞,顧飾功要賞。車劾其罪,械廣至京。賜車金幣,旌其直。岷州土官后能冒功得陞賞,車奏請加罪。能復請,命宥之。車反覆論其不可,曰:「詐冒如能者,實繁有徒,臣方次第按核。今宥能,何以戢眾?若無功得官,則捐軀死敵者,何以待之?」朝廷雖從能請,然嘉車賢,遣使勞賜之。

正統三年,以破朵兒只伯功,增俸一級。在邊,章數十上,悉中時病。同事多不悅,車持益堅。嘗建言:「漠北降人,朝廷留之京師,雖厚爵賞,其心終異。如長脫脫木兒者,昔隨其長來歸,未幾叛去。今乃復來,安知他日不再叛,宜徙江南,離其黨類。」事下兵部,請處之河間、德州。帝報可。後降者悉以此令從事。稽核屯田豪占者,悉清出之,得六百餘頃。四年進兵部尚書,參贊如故。尋命兼理陜西屯田。明年召還,命與僉都御史曹翼歲更代出鎮。及期病甚。詔遣大理寺少卿程富代翼,而命車歸治疾。未及行,六年六月卒。

車在江西時,以採木入閩,經廣信。廣信守,故人也,饋蜜一罌。發視之,乃白金。笑曰:「公不知故人矣」,卻不受。同事邊塞者多以宴樂為豪舉。車惡之,遂斷酒肉。其介特多此類。

劉中敷,大興人,初名中孚。燕王舉兵,以諸生守城功,授陳留丞。擢工部員外郎。仁宗監國,命署部事,賜今名。遷江西右參議。宣德三年遷山東右參政,進左布政使。質直廉靜,吏民畏懷。歲大侵,言於巡撫,減賦三之二。

正統改元,父憂奪情,俄召拜戶部尚書。帝沖年踐阼,慮群下欺己,治尚嚴。而中官王振假以立威,屢摭大臣小過,導帝用重典,大臣下吏無虛歲。三年諷給事御史劾中敷與左侍郎吳璽等,下獄,釋還職。

六年,言官劾中敷專擅。詔法司於內廷雜治。當流,許輸贖。帝特宥之。其冬,中敷、璽及右侍郎陳瑺請以供御牛馬分牧民間。言官劾其變亂成法,並下獄論斬。詔荷校長安門外,凡十六日而釋。瓦剌入貢,詔問馬駝芻菽數,不能對,復與璽、瑺論斬繫獄。中敷以母病,特許歸省。明年冬,當決囚,法司以請。命璽、瑺戍邊,中敷俟母終其奏。已,釋為民。

景帝立,起戶部左侍郎兼太子賓客。時方用兵,論功行賞無虛日。中敷言府庫財有限,宜撙節以備緩急。帝嘉納。景泰四年卒。贈尚書。

中敷性淡泊,食不重味,仕宦五十年,家無餘資。

子璉,正統十年進士。授刑科給事中,累官太僕寺卿。恥華靡,居官剛果。左遷遼東苑馬寺卿,卒。

子機,幼有孝行。成化十四年進士。改庶吉士。正德中,代張彩為吏部尚書,以人言乞歸。起南京兵部尚書,參贊機務。流賊犯江上,眾議擇將。適都督李昴自貴州罷官至,機即召任之。昂以無朝命辭。機曰:「機奉敕有云,『敕所不載,聽便宜』。此即朝命也。」眾服其膽識。致仕歸,卒。

張鳳,字子儀,安平人。父益,官給事中。永樂八年從征漠北,歿於陣。鳳登宣德二年進士。授刑部主事。讞江西叛獄,平反數百人。

正統三年十二月,法司坐事盡繫獄,遂擢鳳本部右侍郎。以主事擢侍郎,前時未有也。明年命提督京倉。六年改戶部,尋調南京。適尚書久闕,鳳遂掌部事。貴州奏軍衛乏糧,乞運龍江倉及兩淮鹽於鎮遠府易米。鳳以龍江鹽雜泥沙,不堪易米給軍,盡以淮鹽予之,然後以聞。帝嘉賞。又言留都重地,宜歲儲二百萬石,為根本計。從之,遂為令。南京糧儲,舊督以都御史,十二年冬命鳳兼理。廉謹善執法,號「板張」。

景泰二年進尚書。四年改兵部,參贊軍務。戶部尚書金濂卒,召鳳代之。時四方兵息,而災傷特甚。帝屢詔寬恤。鳳偕廷臣議上十事,明年復先後議上八事,咸報可。鳳以災傷蠲賦多,國用益詘,乃奏言:「國初天下田八百四十九萬餘頃,今數既減半,加以水旱停征,國用何以取給。京畿及河南、山東無額田,甲方墾闢,乙即訐其漏賦。請準輕則徵租,不惟永絕爭端,亦且少助軍國。」報可。給事中成章等劾鳳擅更祖制,楊穟等復爭之。帝曰:「國初都江南,轉輸易。今居極北,可守常制耶?」四方報凶荒者,鳳請令御史勘實。議者非之。

英宗復辟,調南京戶部,仍兼督糧儲。五年二月卒。

鳳有孝行。性淳樸。故人死,聘其女為子婦,教其子而養其母終身。同學友蘇洪好面斥鳳過,及為鳳屬官猶然。鳳待之如初,聞其貧,即賙給之。

周瑄,字廷玉,陽曲人。由鄉舉入國學。正統中,除刑部主事,善治獄。十三年遷員外郎。明年,帝北征。郎中當扈從者多託疾,瑄請行。六師覆沒,瑄被創歸,擢署郎中。校尉受賕縱盜,以仇人代。瑄辨雪之,抵校尉罪。外郡送囚,一日至八百人。瑄慮其觸熱,三日決遣之殆盡。

景泰元年,以尚書王直薦,超拜刑部右侍郎。久之,出振順天、河間饑。未竣,而英宗復位。有司請召還。不聽。復賜敕,令便宜處置。瑄遍歷所部,大舉荒政,先後振饑民二十六萬五千,給牛種各萬餘,奏行利民八事。事竣還,明年轉左。帝方任門達、逯杲,數興大獄。瑄委曲開諭,多所救正,復飭諸郎毋避禍。以故移部定罪者,不至冤濫。官刑部久,屬吏不敢欺。意主寬恕,不為深文。同佐部者安化孔文英,為御史時按黃巖妖言獄,當坐者三千人,皆白其誣,獨械首從一人論罪。及是居部,與瑄並稱長者。七年命瑄署掌工部事。

瑄恬靜淡榮利。成化改元,為侍郎十六年矣,始遷右都御史。督理南京糧儲,捕懲作奸者數輩,宿弊為清。鳳陽、淮、徐饑,以瑄言發廩四十萬以振。久之,遷南京刑部尚書。令諸司事不須勘者,毋出五日。獄無滯囚。暑疫,悉遣輕繫者,曰:「召汝則至。」囚歡呼去,無失期者。

為尚書九載,屢疏乞休。久之乃得請。家無田園,卜居南京。卒,贈太子少保,謚莊懿。

長子經,尚書,自有傳。次子紘,進士,為南京吏科給事中。兩以災異言事。帝並嘉納。未幾,與御史張昺閱軍,為中官蔣琮誣奏,貶南京光祿署丞。仕終山東參議。

楊鼎,字宗器,陜西咸寧人。家貧力學,舉鄉會試第一。正統四年,殿試第二。授編修。久之,與侍講杜寧等十人,簡入東閣肄業。鼎居侍從,雅欲以功名見。嘗建言修飭戎備、通漕三邊二事。同輩誚其迂,鼎益自信。也先將寇京師,詔行監察御史事,募兵兗州。

景泰三年進侍講兼中允。五年超擢戶部右侍郎。天順初轉左。陳汝言譖之。帝不聽。三年冬以陪祀陵寢不謹下獄,贖杖還職。帝嘗命中官牛玉諭旨,欲取江南折糧銀實內帑,而以他稅物充武臣俸。鼎不可。馬牛芻乏,議征什二,又以民艱力沮。皆報罷。七年,尚書年富有疾,詔鼎掌部事。

成化四年,代馬昂為戶部尚書,而以翁世資為侍郎。六年,鼎疏言:「陜西外患四寇,內患流民。然寇害止邊塞,流民則疾在腹心。漢中僻居萬山,襟喉川蜀,四方流民數萬,急之生變,置之有後憂。請暫設監司一人,專領其事。其願附籍者聽之,不願者資遣。兼與守臣練士馬,修城池,庶可弭他日患。」詔從之。湖廣頻歲饑,發廩已盡。及是有秋,用鼎言,發庫貯銀布,易米備災。淮、徐、臨、德四倉,舊積糧百餘萬石,後餉乏民饑,輒請移用,粟且匱。鼎議上贖罪、中鹽、折鈔、征逋六事行之。由是諸倉有儲蓄。尋加太子少保。

鼎居戶部,持廉,然性頗拘滯。十五年秋,給事御史劾鼎非經國才。鼎再疏求去。賜敕馳驛歸,命有司月給米二石,歲給役四人,終其身。大臣致仕有給賜,自鼎始也。卒,贈太子太保,謚莊敏。

子時畼,進士,累官侍講學士。多識典故,有用世才。時敷,舉人,廬墓被旌,官兵部司務。

翁世資者,莆田人。正統七年進士。除戶部主事,歷郎中。天順元年拜工部右侍郎。四年命中官往蘇、松、杭、嘉、湖增織彩幣七千匹。世資以東南水潦,民艱食,議減其半。尚書趙榮、左侍郎霍瑄難之,世資請身任其咎,乃連署以諫。帝果怒,詰主議者。榮等委之世資,遂下詔獄,謫衡州知府。成化初,擢江西左布政使。坐事下吏,尋得白。大軍征兩廣,轉江西餉,需十萬人。世資議齎直就易嶺南米。民得不擾。以右副都御史巡撫山東。歲饑,發倉儲五十餘萬石以振,撫流亡百六十二萬人。召為戶部右侍郎,佐鼎。久之,代薛遠總督倉場,進尚書。十七年還理部事。閱二年,致仕。

黃鎬,字叔高,侯官人。正統十二年以進士試事都察院。未半歲,以明習法律授御史。

十四年按貴州。群苗盡叛,道梗塞。靖遠伯王驥等自麓川還,軍無紀律,苗襲其後,官軍大敗。鎬赴平越,遇賊幾死。夜跳入城,賊圍之。議者欲棄城走,鎬曰:「平越,貴州咽喉,無平越是無貴州也。」乃偕諸將固守。置密疏竹筒中,募土人間行乞援於朝,且劾驥等覆師狀。景帝命保定伯梁珤等合川、湖軍救之,圍始解。城被困已九月,掘草根煮弩鎧而食之,死者相枕籍。城卒全,鎬功為多。復留按一年。久之,遷廣東僉事,改浙江。

成化初,以大臣會薦,擢廣東左參政。高、雷、廉負海多盜,鎬討平之。再遷廣西左布政使。以右副都御史總督南京糧儲,歷吏部左、右侍郎。十六年拜南京戶部尚書。

鎬有才識,敏吏事,理鹽政,多所釐剔,時論稱之。十九年致仕,道卒。贈太子少保,謚襄敏。

胡拱辰,字共之,淳安人。正統四年進士。為黟縣知縣,有惠政,擢御史。疏陳時弊八事。父艱歸。

景帝即位,詔科道官憂居者悉起復。拱辰至,屢疏以選將、保邦、修德、弭災為言,出為貴州左參政。白水堡仡佬頭目沈時保素梗化,拱辰言於總兵官方瑛遣將擒之。一方遂寧。至畢節,平宣慰使隴富亂,威行邊徼。母憂去,御史追劾其受賕事,下浙江按臣執訊。事白,調廣東。歷廣西、四川左、右布政使,皆有平寇功。

成化八年拜南京右副都御史,提督操江。十一年就遷兵部右侍郎。儲位虛久,與尚書崔恭等請冊立,言甚切。其年復就改左副都御史總理糧儲,就進工部尚書。節財省事,人皆便之。以年至乞歸。

弘治中,巡按御史陳銓言:「拱辰退休十餘年,生平清操如一日,乞加禮異以勵臣節。」詔有司月給廩二石,歲隸四人。正德元年,年九十。遣行人齎敕存問,賚羊酒,加賜廩、隸。三年正月卒。贈太子少傅,謚莊懿。

陳俊,字時英,莆田人。舉鄉試第一。正統十三年進士。除戶部主事。督天津諸衛軍採草,奏減新增額三十五萬束。豪猾侵蘇、松改折銀七十餘萬兩,俊往督,不數月畢輸。尚書金濂以為能,俾典諸曹章奏。歷郎中。

天順五年,兩廣用兵,俊督餉。時州縣殘破,帑藏殫虛,弛鹽商越境令,引加米二斗,軍興賴以無乏。母喪,不聽歸,蠻平始還。初,俊為主事,奔父喪,賻者皆卻之。至是文武將吏醵金賻,亦不納。

成化初,擢南京太常少卿。四年召拜戶部右侍郎。俊練習錢穀。四方災傷,邊鎮急芻餉。奏請遝至,裁決咸當,尚書楊鼎深倚之。京師大饑,先後發太倉粟八十萬石平糴。石值六錢,豪猾乘時射利。俊請糴以升斗為率,過一石勿與。饑民獲濟。尋議用兵河套,敕俊赴河南、山、陜,會巡撫諸臣畫芻餉,發帑金二十萬助之。俊以邊庾空竭,歲又不登,而榆林道險遠,轉輸難,乃發金於內地市易。修西安、韓城、同官徑道,以利飛輓。還朝,進俸一級,歷吏部左、右侍郎。

九載滿,拜南京戶部尚書。尋改兵部,參贊機務。先是,參贊之任,不專屬兵部,自薛遠後,繼以俊,遂為定制。久之,就改吏部。二十一年,星變,率九卿陳時弊二十事,皆極痛切。帝多采納。而權倖所不便者,終格不行。明年乞致仕。詔加太子少保,賜敕馳傳還。卒,謚康懿。

林鶚,字一鶚,浙江太平人。景泰二年進士。授御史,監京畿鄉試。陳循等訐考官,鶚邑子林挺預薦,疑鶚有私,逮挺考訊。挺實無他,得白。

英宗復辟,仿先朝故事,出廷臣為知府。鶚得鎮江。召見,賜膳及道里費,諭所以擢用意。鶚感激,革弊舉廢,治甚有聲。漕故經孟瀆,險甚。巡撫崔恭議鑿河,自七里港引金山上流通丹陽避之。鶚言:「道里遠,多石,且壞民廬墓。請按京口閘、甘露壩故迹,浚之令通舟。春夏啟閘,秋冬度壩,功力省便。」恭從其議,遂為永利。居五年,以才任治劇,調蘇州。

成化初,超遷江西按察使。有犯大辟賄達官求生者,鶚執愈堅。廣東寇剽贛州急。調兵御之,遁去。廣信妖賊妄稱天神惑眾,捕戮其魁,立解散。歷左、右布政使。歲饑,奏減民租十五萬石。

成化六年,擢南京刑部右侍郎。母憂服除,召為刑部右侍郎。執法不撓。十二年得疾卒。

鶚事母孝謹,對妻子無惰容。不妄交與,公餘輒危坐讀書。歿不能具棺斂,友人為經紀其喪。鶚在蘇州,先聖像剝落。鶚曰:「塑像,非古也,昔太祖於國學用木主。」命改從之。嘉靖中,御史趙大佑上其節行,贈刑部尚書,謚恭肅。

潘榮,字尊用,龍溪人。正統十三年進士。犒師廣東,還,除吏科給事中。

景泰初,疏論停起復、抑奔競數事。帝納之。尋進右給事中。四年九月上言:「致治之要,莫切於納諫。比以言者忤聖意,諭禮部,凡遇建言,務加審察。或假以報復,具奏罪之。此令一下,廷臣喪氣,以言為諱。國家有利害,生民有得失,大臣有奸慝,何由而知?況今巨寇陸梁,塞上多事,奈何反塞言者路。望明詔臺諫,知無不言,緘默者罪。並敕閣部大臣,勿搜求參駁,虧傷治體。」疏入,報聞。

天順六年使琉球,還,遷都給事中。成化六年三月偕同官上言:「近雨雪愆期,災異迭見。陛下降詔自責,躬行祈禱,詔大臣盡言,宜上天感格。而今乃風霾晝晦,沴氣赤而復黑,豈非應天之道有未盡歟?夫人君敬天,不在齋戒祈禱而已。政令乖宜,下民失所;崇尚珍玩,費用不經;後宮無序,恩澤不均;爵濫施於賤工,賞妄及於非分,皆非敬天之道。願陛下日御便殿,召大臣極陳缺失而釐革之,庶災變可弭。」時萬妃專寵,群小夤緣進寶玩,官賞冗濫,故榮等懇言之。帝不能用。是年遷南京太常少卿。

又七年,就擢戶部右侍郎。尋改右副都御史、總督南京糧儲。積奇羨數萬石以備荒。十七年召為戶部左侍郎,尋署部事。英國公張懋等四十三人自陳先世以大功錫爵,子孫承繼,所司輒減歲祿,非祖宗報功意。榮等言:「懋等於無事時妄請增祿,若有功何以勸賞?況頻年水旱,國用未充,所請不可許。」事乃寢。中官趙陽等乞兩淮鹽十萬引,帝已許之。榮等言:「近禁勢家中鹽,詔旨甫頒,而陽等輒違犯,宜正其罪。」帝為切責陽等。

南京戶部尚書黃鎬罷,以榮代之。孝宗嗣位,謝政歸。賜月廩、歲夫如制。九年卒,年七十有八。贈太子太保。

夏時正,字季爵,仁和人。正統十年進士。除刑部主事。景泰六年以郎中錄囚福建,出死罪六十餘人。中有減死、詔充所在濱海衛軍者,時正慮其入海島為變,轉發之山東,然後以聞。因言:「凡福建減死囚,俱宜戍之北方。」法司是其言,而請治違詔罪。帝特宥之。時正又言:「通番及劫盜諸獄,以待會讞。淹引時月,囚多瘐死。請令所司斷決。」詔從之,且推行之天下。

天順初,擢大理寺丞。久之,以便養,遷南京大理少卿。成化五年遷本寺卿。明年春命巡視江西災傷。除無名稅十餘萬石,汰諸司冗役數萬,奏罷不職吏二百餘人,增築南昌濱江堤及豐城諸縣陂岸,民賴其利。嘗上奏,不具齎奏人姓名,吏科論其簡恣。帝宥其罪,錄彈章示之。遂乞休歸。僦居民舍,布政使張瓚為築西湖書院居之。家食三十年,年近九十而卒。

時正雅好學。閑居久,多所著述,於稽古禮文事尤詳。

贊曰:金純等黽勉奉公,當官稱職。加之禔躬清白,操行無虧,固列卿之良也。鄭辰之廉事,周瑄之治獄,皆有仁人之用心,君子哉。


\end{pinyinscope}