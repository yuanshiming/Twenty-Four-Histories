\article{列傳第六十}

\begin{pinyinscope}
○羅亨信侯璡楊寧王來孫原貞孫需張憲朱鑒楊信民張驥竺淵耿定王晟鄧顒馬謹程信白圭子鉞張瓚謝士元孔鏞李時敏鄧廷瓚王軾劉丙

羅亨信,字用實,東莞人。永樂二年進士。改庶吉士,授工科給事中。出視浙江水災,奏蠲三縣租。進吏科右給事中,坐累謫交阯為吏。居九年,仁宗嗣位,始召入為御史。核通州倉儲,巡按畿內,清軍山西,皆有聲。宣德中,有薦其堪方面者。命食按察僉事俸,待遷。

英宗即位之三月,擢右僉都御史,練兵平涼、西寧。正統二年,蔣貴討阿台、朵兒只伯,亨信參其軍務。至魚兒海,貴等以芻餉不繼,留十日引還。亨信讓之曰:「公等受國厚恩,敢臨敵退縮耶?死法孰與死敵?」貴不從。亨信上章言貴逗遛狀。帝以其章示監督尚書王驥等。明年進兵,大破之。亨信以參贊功,進秩一等。

父喪歸葬。還朝,改命巡撫宣府、大同。參將石亨請簡大同民三之一為軍,亨信奏止之。十年進右副都御史,巡撫如故。時遣官度二鎮軍田,一軍八十畝外,悉徵稅五升。亨信言:「文皇帝時,詔邊軍盡力墾田,毋征稅,陛下復申命之。今奈何忽為此舉?塞上諸軍,防邊勞苦,無他生業,惟事田作。每歲自冬徂春,迎送瓦剌使臣,三月始得就田,七月又復刈草,八月以後,修治關塞,計一歲中曾無休暇。況邊地磽瘠,霜早收薄,若更徵稅,則民不復畊,必致竄逸。計臣但務積粟,不知人心不固,雖有粟,將誰與守?」帝納其言而止。

初,亨信嘗奏言:「也先專候釁端,以圖入寇。宜預於直北要害,增置城衛為備。不然,恐貽大患。」兵部議,寢不行。及土木之變,人情洶懼。有議棄宣府城者,官吏軍民紛然爭出。亨信仗劍坐城下,令曰:「出城者斬!」又誓諸將為朝廷死守,人心始定。也先挾上皇至城南,傳命啟門。亨信登城語曰:「奉命守城,不敢擅啟。」也先逡巡引去。赤城、雕鶚、懷來、永寧、保安諸守將棄城遁,並按其罪。

當是時,車駕既北,寇騎日薄城下,關門左右皆戰場。亨信與總兵楊洪以孤城當其衝,外禦強寇,內屏京師。洪既入衛,又與朱謙共守,勞績甚著。著兜鍪處,顛髮盡禿。景帝即位,進左副都御史。明年,年七十有四矣,乞致仕。許之。歸八年,卒於家。

侯璡。字廷玉。澤州人。少慷慨有志節。登宣德二年進士,授行人。

烏撒、烏蒙土官以爭地相仇殺,詔遣璡及同官章聰諭解之,正其疆理而還。副侍郎章敞使交阯,關門卑,前驅傴而入,璡叱曰:「此狗竇耳,奈何辱天使!」交人為毀關,乃入。及歸,饋遺無所受。遷兵部主事。

正統初,從尚書柴車等出鐵門關禦阿台有功,進郎中。從王驥征麓川,至金齒。驥自統大軍擊思任發,而遣璡援大侯州。賊眾三萬至,督都指揮馬讓、盧鉞擊走之。遂由高黎貢山兼程夜行,會大軍,壓其巢。麓川平,拜禮部右侍郎,參贊雲南軍務,詔與楊寧二年更代。驥再徵麓川。璡以功遷左。九年代還。母憂,起復,尋調兵部。十一年復代寧鎮雲南。思機發竄孟養,驥復南征。璡與都督張軏分兵進抵金沙江,破之鬼哭山。璽書褒賚。

景泰初,貴州苗韋同烈叛,圍新添、平越、清平、興隆諸衛。命璡總督貴州軍務討之。時副總兵田禮巳解新添、平越圍,璡遂遣兵攻敗都盧、水西諸賊,貴州道始通。又調雲南兵,由烏撒會師,開畢節諸路,檄普安土兵援安南衛,而自率師攻紫塘、彌勒等十餘寨。會賊復圍平越,回師擊退之。遂分哨七盤坡、羊腸河、楊老堡,解清平圍,東至重安江,與驥兵會。興隆抵鎮遠道皆通。捷聞,進兵部尚書。進克賞改苗,擒其渠王阿同等三十四人。別賊阿趙偽稱趙王,率眾掠清平,璡復討擒之。水西苗阿忽等六族皆自乞歸化,詔璡隨方處置。

景泰元年八月以勞瘁卒於普定,年五十三。賜祭葬,廕其子錦衣衛世襲千戶。

楊寧,字彥謐,歙人。宣德五年進士。授刑部主事。機警多才能,負時譽。

正統初,從尚書魏源巡視宣、大。四年與都督吳亮征麓川。賊款軍門約降,寧曰:「兵未加而先降,誘我也,宜嚴兵待之。」不聽,令寧督運金齒。已而賊果大至,官兵敗績。諸將獲罪,寧擢郎中。復從王驥至騰衝破賊,寧與太僕少卿李蕡督戰,並有功。師還,寧超拜刑部右侍郎。遭母憂,奪情。

九年代侯璡參贊雲南軍務。時麓川甫平,寧以騰衝地要害,與都督沐昂築城置衛,設戍兵控諸蠻。邊方遂定。居二年,召還。

閩、浙盜起,命寧鎮江西。賊至,輒擊敗之。暇則詢民疾苦,境內嚮服。

景泰初,召拜禮部尚書,偕胡濙理部事。迤北可汗遣使入貢,寧言:「宜留使數日,宴勞賜予,視也先使倍厚。彼性多猜,二人必內構,邊患可緩。」帝務誠信,不許。其冬,以足疾調南刑部。七年為御史莊昇所劾,遣核未報。寧力詆言官,都察院再劾寧脅制言路。詔免其罪,錄狀示之。英宗復辟,命致仕。踰年卒。

寧有才而善交權貴。嘗自敘前後戰功,乞世廕。子堣方一歲,遂得新安衛副千戶。

王來,字原之,慈谿人。宣德二年以會試乙榜授新建教諭。寧王府以諸生充樂舞,來請易以道士。諸王府設樂舞生始此。

六年,以薦擢御史,出按蘇、松、常、鎮四府。命偕巡撫周忱考察屬吏,敕有「請自上裁」語。來言:「賊民吏,去之惟恐不速,必請而後行,民困多矣。」帝為改敕賜之。中官陳武以太后命使江南,橫甚,來數抑之。武還,愬於帝。帝問都御史顧佐:「巡按誰也?」佐以來對。帝歎息稱其賢,曰「識之」。及報命,獎諭甚至。

英宗即位,以楊士奇薦,擢山西左參政。言:「流民所在成家。及招還故土,每以失產復逃去。乞請隨在附籍便。」又言:「郡縣官不以農業為務,致民多游惰,催征輒致已命。朝廷憫其失業,下詔蠲除,而田日荒閒,租稅無出,累及良民。宜擇守長賢者,以課農為職。其荒田,令附近之家通力合作,供租之外,聽其均分,原主復業則還之。蠶桑可裨本業者,聽其規畫。仍令提學風憲官督之,庶人知務本。」從之。

來居官廉,練達政事。侍郎于謙撫山西,亟稱其才,可置近侍。而來執法嚴,疾惡尤甚,以公事杖死縣令不職者十人。逮下獄,當徒。遇赦,以原官調補廣東。來自此始折節為和平,而政亦修舉。正統十三年遷河南左布政使。明年改左副都御史,巡撫河南及湖廣襄陽諸府。也先逼京師,來督兵勤王。渡河,聞寇退,乃引還。

景泰元年,貴州苗叛。總督湖廣、貴州軍務侯璡卒於軍,進來右都御史代之。與保定伯梁珤,都督毛勝、方瑛會兵進討。至靖州,賊掠長沙、寶慶、武岡。來等分道邀擊,俘斬三千餘人,賊遁去。已,復出掠,官軍連戰皆捷。賊魁韋同烈據興隆,劫平越、清平諸衛,來與方瑛擊敗之。賊退保香爐山,山陡絕。勝、瑛與都督陳友三道進,來與珤大軍繼之。先後破三百餘寨,會師香爐山下。發炮轟崖石。聲動地。賊懼,縛同烈並賊將五十八人降。餘悉解散。遂移軍清平,且檄四川兵共剿都勻、草塘諸賊。賊望風具牛酒迎降。

賊平,班師。詔留來、珤鎮撫。尋命來兼巡撫貴州。奏言:「近因黔、楚用兵,暫行鬻爵之例。今寇賊稍寧,惟平越、都勻等四衛乏餉。宜召商中鹽,罷納米例。」從之。

三年十月召還,加兼大理寺卿。珤以來功大,乞加旌異。都給事中蘇霖駁之,乃止。來還在道,以貴州苗復反,敕回師進討。明年,事平。召為南京工部尚書。英宗復辟,六尚書悉罷。來歸。成化六年卒於家。

孫原貞,名瑀,以字行,德興人。永樂十三年進士。授禮部主事,歷郎中。英宗初,用薦擢河南右參政。居官清慎,有吏才。

正統八年,大臣會薦,遷浙江左布政使。久之,盜大起閩、浙間,赦而再叛。景帝即位,發兵討之。原貞嘗策賊必叛,上方略,請為備。至是即命原貞參議軍事,深入擒其魁。而溫州餘賊猶未滅,命都指揮李信為都督僉事,調軍討之。遂拜原貞兵部左侍郎,參信軍務,鎮守浙江。丁母憂,當去,副都御史軒輗請留之。報可。

景泰元年,原貞進兵搗賊巢。俘斬賊首陶得二等,招撫三千六百餘人,追還被掠男女。捷聞,璽書獎勵。請奔喪。踰月,還鎮。復分兵剿平餘寇。奏析瑞安地增置泰順,析麗水、青田二縣地置雲和、宣平、景寧四邑,建官置戍,盜患遂息。論功,進秩一等。浙官田賦重,右布政使楊瓚請均於民田輕額者。詔原貞督之,田賦以平。三年請褒贈禦賊死事武臣。指揮同知脫綱、王瑛,都指揮僉事沈轔、崔源,皆得贈恤。六月進兵部尚書,鎮守如故。未幾,命考察福建庶官,因留鎮焉。福州、建寧二府,舊有銀冶,因寇亂罷。朝議復開,原貞執不可,乃寢。

五年冬,疏言:

四方屯軍,率以營繕、轉輸諸役妨耕作。宜簡精銳實伍,餘悉歸之農。茍增萬人屯,即歲省支倉糧十二萬石,且積餘糧六萬石,兵食豈有不足哉。

今歲漕數百萬石,道路費不貲。如浙江糧軍兌運米,石加耗米七斗。民自運米,石加八斗。其餘計水程遠近加耗。是田不加多,而賦斂實倍,欲民無困,不可得也。況今太倉無十數年之積,脫遇水旱,其何以濟!宜量入為出,汰冗食浮費。俟倉儲既裕,漸減歲漕數,而民困可蘇也。

臣昔官河南,稽諸逃民籍凡二十餘萬戶,悉轉徙南陽、唐、鄧、襄、樊間。群聚謀生,安保其不為盜?宜及今年豐,遣近臣循行,督有司籍為編戶,給田業,課農桑,立社學、鄉約、義倉,使敦本務業。生計既定,徐議賦役,庶無他日患。時不能盡用。後劉千斤之亂,果如原貞所料。

已,復鎮浙江。英宗復位,罷歸。成化十年卒,年八十七。

原貞所至有勞績,在浙江尤著名。

孫需,字孚吉,成化八年進士。為常州府推官,疑獄立剖,擢南京御史。劾僧繼曉,忤旨,予杖,出為四川副使。弘治中,累官右副都御史,巡撫河南。歲凶,募民築汴河堤,堤成而饑者亦濟。鎮守中官劉郎貪橫。奸民赴郎訟者,需以法論之遣戍。郎為跪請,執不聽,郎恨次骨。大臣子橫于鄉,需抑之。郎與謀,改需撫陜西。尋改撫鄖陽,安輯流民,占籍者九萬餘戶。正德元年召為南京兵部右侍郎。四年就拜禮部尚書。未兩月,劉瑾惡之,追論撫河南時事,罰米輸邊。廷推需刑部尚書,中旨令致仕。瑾誅,起南京工部尚書,就改刑部,再改吏部。十三年乞休去。嘉靖初卒,謚清簡。

張憲,字廷式,與需同里,同舉進士,相代為尚書。嘗為浙江右布政使,後以工部右侍郎督易州山廠,公帑無毫髮私。歷南京禮部尚書。劉瑾勒致仕。瑾誅,起工部。卒。

朱鑑,字用明,晉江人。童時刲股療父疾。舉鄉試,授蒲圻教諭。

宣德二年,與廬陵知縣孔文英等四十三人以顧佐薦,召於各道觀政三月,遂擢御史。巡按湖廣,諭降梅花峒賊蕭啟寧等。請復舊制,同副使、僉事按行所部,問民疾苦。湖湘俗,男女婚嫁多踰三十。鑑申明禮制,其俗遂變。三載代歸。

正統五年復按廣東。奏設欽州守備都指揮。奉命錄囚,多所平反,招撫逋叛甚眾。還朝,請天下按察司增僉事一人,專理屯田,遂為定制。

七年,用薦擢山西左參政。奏減平陽採薪供邊夫役。景帝監國,進布政使。尋擢右副都御史,巡撫其地。上言:「也先奸詭百端,殺掠無已。復假和親,遣使覘伺。以送駕為名,覬得開關延接。稍示抗拒,彼即有辭。其謀既深,我慮宜遠。宜暫罷中貴監軍之制,假總兵以生殺權,使志無所撓,計有所施。整散兵,募勇士,重懸賞格。鼓勸義旅,徵勤王兵,數道並進,戮力復仇。庶大駕可還,敵兵自退。曩者江南寇發,皆以誅王振為名。夫事歸朝廷則治,歸宦官則亂。昔高皇帝與群臣議事,必屏去左右,恐泄事機。乞杜權倖之門,凡軍國重事,屬任大臣,必當有濟。」景帝嘉納之。

時瓦剌窺塞下,鑒日夜為守禦計。景泰元年,敵數萬騎攻雁門,都指揮李端擊卻之。尋犯河曲及義井堡,殺二指揮,圍忻、代諸州,石亨等不能禦。長驅抵太原城北,山西大震。命鑒移鎮雁門,而別遣都督僉事王良鎮太原。援兵漸集,敵亦饜,乃引去。時山西仍遘兵荒,鑑外飭戎備,內撫災民,勞瘁備至。

二年十月,鎮守山西都御史羅通召還。命鑒兼領其事。明年詔遣大臣行天下,黜陟有司。禮部侍郎鄒幹至山西,多所論劾。鑒請召乾還,乾因極論鑒徇護,帝是乾言。其年十月召鑑佐院事。至京,致仕去。

初,景帝易儲,鑒貽大學士陳循書,言不可。且曰:「陛下於上皇,當避位以全大義。」循大駭。英宗復位,鑒詣闕上表賀。帝曰:「鑑老疾,何妄來?其速令還。」家居二十餘年卒。

楊信民,名誠,以字行,浙江新昌人。鄉舉入國學。宣德時,除工科給事中。母憂歸。營葬土石必躬舁數百步,曰:「吾葬吾母而專役他人,吾不安也。」服闋,改刑科。

正統中,清軍江西,還奏民隱五事,多議行。尋以王直薦,擢廣東左參議。清操絕俗,嘗行田野,訪利弊為更置。性剛負氣,按察使郭智不法,信民劾之下獄。黃翰代智,信民復發其奸。已,又劾僉事韋廣,廣遂訐信民,因與翰俱被逮。軍民嘩然,詣闕下乞留信民。詔復信民官,而翰、廣鞫實,除名。

景帝監國,于謙薦之,命守備白羊口。會廣東賊黃蕭養圍廣州急,嶺南人乞信民,乃以為右僉都御史巡撫其地。士民聞而相慶曰:「楊公來矣。」時廣州被圍久,將士戰輒敗,禁民出入,樵采絕。而鄉民避賊來者拒不納,多為賊所害,民益愁苦歸賊。信民至,開城門,發倉廩,刻木鍥給民,得出入。賊見木鍥曰:「此楊公所給也」,不敢傷。避賊者悉收保,民若更生。信民益厲甲兵,多方招撫,降者日至。乃使使持檄入賊營,諭以恩信。蕭養曰:「得楊公一言,死不恨。」克日請見。信民單車詣之,隔濠與語。賊黨望見,讙曰:「果楊公也!」爭羅拜,有泣下者。賊以大魚獻,信民受之不疑。

蕭養且降,而都督董興大軍至。賊忽中變。夜有大星隕城外,七日而信民暴疾卒。時景泰元年三月乙卯也。軍民聚哭,城中皆縞素。賊聞之,亦泣曰:「楊公死,吾屬無歸路矣。」未幾,興平賊,所過村聚多殺掠。民仰天號曰:「楊公在,豈使吾曹至是!」訃聞,賜葬祭,錄其子玖為國子生。廣東民赴京請建祠,許之。成化中,賜謚恭惠。久之,從選人盧從愿請,命有司歲以其忌日祭焉。

張驥,字仲德,安化人。永樂中舉於鄉,入國學。宣德初授御史。出按江西,慮囚福建,有仁廉聲。

正統八年,吏部尚書王直等應詔,博舉廷臣公廉有學行者,驥與焉。遷大理右寺丞,巡撫山東。先是,濟南設撫民官,專撫流民。後反為民擾,驥奏罷之。俗遇旱,輒伐新葬塚墓,殘其肢體,以為旱所由致,名曰「打旱骨樁」,以驥言禁絕。還朝,進右少卿。已,命巡視濟寧至淮、揚饑民。驥立法捕蝗,停不急務,蠲逋發廩,民賴以濟。

十三年冬,巡撫浙江。初,慶元人葉宗留與麗水陳鑑胡聚眾盜福建寶豐諸銀礦,已而群盜自相殺,遂為亂。九年七月,福建參議竺淵往捕,被執死。宗留僭稱王。時福建鄧茂七亦聚眾反,勢甚張。宗留、鑑胡附之,流剽浙江、江西、福建境上。參議耿定,僉事王晟及都督僉事陳榮,指揮劉真,都指揮吳剛、龔禮,永豐知縣鄧顒,前後敗歿。遂昌賊蘇牙、俞伯通剽蘭溪,又與相應,遠近震動。驥至,遣金華知府石瑁擊斬牙等,撫定其餘黨。而鑒胡方以爭忿殺宗留,專其眾,自稱大王,國號太平,建元泰定。偽署將帥,圍處州,分掠武義、松陽、龍泉、永康、義烏、東陽、浦江諸縣。未幾,茂七死,鑑胡勢孤。驥命麗水丞丁寧率老人王世昌等齎榜入賊巢招之,鑑胡遂偕其黨出降。惟陶得二不就撫,殺使者,入山為亂如故。時十四年四月也。驥既招降鑒胡,而別賊蘇記養等掠金華,亦為官軍所獲,賊勢乃益衰。

其秋,景帝嗣位,召驥還,卒於道。驥所至,咸有建樹,山東、兩浙民久而思之。鑑胡至京,帝宥不誅。更遇赦,釋充留守衛軍。也先入犯,鑑胡乘間亡,被獲,伏誅。

竺淵,奉化人。耿定,和州人。王晟鄆城人。鄧顒,樂昌人。俱進士。顒兵潰被執,不屈死。詔為營葬。淵等贈官,錄一子。

馬謹,字守禮,新樂人。宣德二年進士。事父母孝,遭喪,親負土以葬。

正統中,以御史按浙江。時修備倭海船,徵材于嚴、衢諸郡。謹恐軍士藉勢肆斬伐,請禁飭之,報可。所至,貪猾屏跡。疏振台、處、寧、紹四府饑。吏部驗封郎中缺人久,帝令推擇。會謹九載滿,尚書郭璡薦謹廉直,遂用之。十年薦擢湖廣右布政使。

正統末,湖南叛苗掠靖州。命謹同御史侯爵撫諭,參將張善率兵繼之。謹等至,招數千人復業,其出掠者擊敗之。尋與善破淇溪諸寨。景泰初,復與善大破臘婆諸洞。已,同參將李震擊破青龍渡、馬楊山諸賊,追奔至雞心嶺,先後斬首千四百有奇。師還,靖州賊復出掠,搗其巢,斬獲如前。武岡、城溪諸賊結廣西蠻,據青肺山,復與震攻破之。獲賊楊光拳等五百六十人,斬首倍之。扶城諸寨,聞風款附。

謹出入行間三歲,衝冒鋒鏑,與諸將同,而運籌轉餉功尤多。轉左布政使。錄功,進秩一等。六年五月,遷右副都御史,仍支二品俸。巡撫河南,撫流民三萬一千餘戶。天順初,廢巡撫官,謹亦罷歸,久之卒。

謹性廉介,楊士奇嘗稱為「冰霜鐵石」。

程信,字彥實,其先休寧人。洪武中戍河間,因家焉。信舉正統七年進士,授吏科給事中。

景帝即位,薦起薛瑄等三人。也先犯京師,信督軍守西城,上言五事。都督孫鏜擊也先失利,欲入城,信不納,督軍從城上發箭炮助之。鏜戰益力,也先遂卻。

景泰元年請振畿輔饑民,復河間學官、生徒因用兵罷遣者,皆報可。進左給事中。以天變上中興固本十事。其言敬天,則請帝敦孝友之實以答天心。帝嘉納之。

明年二月出為山東右參政,督餉遼東。巡撫寇深奏盜糧一石以上者死,又置新斛視舊加大,屬信鉤考。信立碎之,曰:「奈何納人於死!」深由是不悅信。尋以憂去,服闋,起四川參政。理松潘餉,偕侍郎羅綺破黑虎諸寨。

天順元年,信入賀。時方錄景泰間進言者,特擢信太僕卿。京衛馬舊多耗,信定期徵之。三營大將石亨、孫鏜、曹欽並以「奪門」功有寵,庇諸武臣,為言太僕苛急,請改隸兵部。信言:「高皇帝令太僕馬數,勿使人知。若隸兵部,馬登耗,太僕不得聞。脫有警,馬不給,誰任其咎?」帝是之,乃隸太僕如故。

明年,改左僉都御史,巡撫遼東。都指揮夏霖恣不法,僉事胡鼎發其四十罪,信以聞,下霖錦衣獄。門達言信不當代奏,帝責令陳狀。時寇深方掌都察院,修前郤,劾信。徵下詔獄,降南京太僕少卿。五年召為刑部右侍郎。母憂歸。

成化元年起兵部,尋轉左。四川戎縣山都掌蠻數叛,陷合江等九縣。廷議發大軍討之。以襄城伯李瑾充總兵官,太監劉恒為監督,進信尚書,提督軍務。至永寧,分道進。都督芮成由戎縣;巡撫貴州都御史陳宜、參將吳經由芒部;都指揮崔旻由普市冰腦;南寧伯毛榮由李子關;巡撫四川都御史汪浩、參將宰用由渡船鋪;左右遊擊將軍羅秉忠、穆義由金鵝池;而信與瑾居中節制。轉戰六日,破龍背、豹尾諸寨七百五十餘。明年至大壩,焚寨千四百五十。前後斬首四千五百有奇,俘獲無算。按諸九姓不奉化者遷瀘州衛,於渡船鋪增置關堡。改大壩為太平川長官司,分山都掌地,設官建治控制之。帝降璽書嘉勞。錄功,進兼大理寺卿,與白圭同蒞兵部。言官劾信上首功不實。信四疏乞休,不許。信欲有為,而阻於圭,不自得,數稱疾。

六年春旱,應詔言兵事宜更張者四,兵弊宜申理者五。大略言:延綏、兩廣歲遭劫掠,宜擇大臣總制;四方流民多聚荊、襄,宜早區畫;京軍操練無法,功次升賞未當。語多侵圭。圭奏寢之。改南京兵部,參贊機務。明年致仕,踰年卒。贈太子少保,謚襄毅。

信有才力,識大體。征南蠻時,制許便宜從事。迄班師,未嘗擅賞、戮一人。曰:「刑賞,人主大柄也,不得已而假之人。幸而事集,輒自專,非人臣所宜。」在南京,守備臣欲預錢穀訟獄事,信曰:「守備重臣,所以謹非常也。若此,乃有司職耳。」論者韙之。子敏政,見《文苑傳》。

白圭,字宗玉,南宮人。正統七年進士。除御史,監朱勇軍,討兀良哈有功。巡按山西,辨疑獄百餘。從車駕北征,陷土木。脫還,景帝命往澤州募兵。尋遷陜西按察副使,擢浙江右布政使。福建賊鄭懷冒流剽處州,協諸將平之。

天順二年,貴州東苗干把豬等僭號,攻劫都勻諸處。詔進右副都御史,贊南和侯方瑛軍往討。圭以谷種諸夷為東苗羽翼,先剿破百四十七寨。遂會兵青崖,復破四百七十餘寨,乘勝攻六美山。乾把豬就擒,諸苗震讋。湖廣災,就命圭巡撫。

四年召為兵部右侍郎。明年,孛來寇莊浪。圭與都御史王竑贊都督馮宗軍務,分兵巡邊。圭敗之固原州。七年進工部尚書。

成化元年,荊、襄賊劉千斤等作亂。敕撫寧伯朱永為總兵官,都督喜信、鮑政為左右參將,中官唐慎、林貴奉監之,而以圭提督軍務,發京軍及諸道兵會討。

千斤,名通,河南西華人。縣門石狻猊重千斤,通隻手舉之,因以為號。正統中,流民聚荊、襄間,通竄入為妖言,潛謀倡亂。石龍者,號石和尚,聚眾剽掠。通與共起兵,偽稱漢王,建元德勝,流民從者四萬人。圭等至南漳,賊迎戰,敗之,乘勝逼其巢。通奔壽陽,謀走陜西。圭遣兵扼其道,通乃退保大市,與苗龍合。官軍又破之雁坪,斬通子聰及其黨苗虎等。賊退保後巖山,據險下木石如雨。諸軍四面攻,圭往來督戰,士皆蟻附登。賊大敗。擒通及其眾三千五百餘人,獲賊子女萬一千有奇,焚其廬舍,夷險阻而還。石龍與其黨劉長子等逸去,轉掠四川,連陷巫山、大昌。圭等分兵蹙之,長子縛龍以降,餘寇悉平。錄功,加圭太子少保,增俸一級。遭父憂,葬畢,視事。

三年改兵部尚書,兼督十二團營。六年,阿羅出等駐牧河套,陜西數被寇。圭言鎮巡官偷肆宜治。延綏巡撫王銳、鎮守太監秦剛、總兵官房能俱獲罪去。圭乃議大舉搜河套,發京兵及他鎮兵十萬屯延綏。而以輸餉責河南、山西、陜西民,不給,則預徵明年賦。於是內地騷然。而前後所遣三大將朱永、趙輔、劉聚,皆畏怯不任戰,卒以無功。十年卒官,年五十六。贈少傅,謚恭敏。

圭性簡重,公退即閉閣臥,請謁皆不得通。在貴州時,有憤中官虐而欲刺之者,誤入圭所。圭擁衾問之,其人驚曰:「乃吾公耶?」即自刎,不殊,仆於地。圭呼燭起視,傅以善藥,遣之。人服其量。

次子鉞,字秉德。進士及第,授編修。累官太子少保,禮部尚書。習典故,以詞翰稱。卒,贈太子太保,謚文裕。

張瓚,字宗器,孝感人。正統十三年進士。授工部主事,遷郎中,歷知太原、寧波二府,有善政。

成化初,市舶中官福住貪恣,瓚禁戢其下。住誣瓚於朝,瓚遂列住罪。住被責,其黨多抵法。大臣會薦,遷廣東參政,轉浙江左布政使。

十年冬,以右副都御史巡撫四川。播州致仕宣慰楊輝言,所屬夭壩干、灣溪諸寨及重安長官司為生苗竊據,請王師進討。詔瓚諭還侵地,不服則徵之。瓚率兵討定,請設安寧宣撫司,即授輝子友為宣撫以鎮。詔可,賜敕獎勞。以母老乞歸,母已卒。

會松、茂番寇邊,詔起復視事。先是,僉事林璧言:「松茂曩為大鎮。都御史寇深、侍郎羅綺嘗假便宜,專制其地,故有功。今惟設兩參將,以副使居中調度。事權輕,臨敵稟令制府,千里請戰,謀洩機緩,未有能獲利者。宜別置重臣彈壓,或即命瓚兼領,專其責成。」十二年七月命瓚兼督松茂、安綿、建昌軍務。瓚至軍,審度形勢,改大壩舊設副使於安綿,而令副總兵堯彧軍松潘,參將孫暠軍威、疊,為夾攻計。乘間修河西舊路,作浮梁,治月城。避偏橋棧道,軍獲安行,轉餉無阻。十四年六月攻白草壩、西坡、禪定數大寨,斬獲亡算。徇茂州、疊溪,所過降附。抵曲山三寨,攻破之,再討平白草壩餘寇。先後破滅五十二寨,賊魁撒哈等皆殲。他一百五寨悉獻馬納款,諸番盡平。留兵戍要害,增置墩堡,乃班師。帝嘉其功,徵拜戶部左侍郎,辭歸終制。

十五年起左副都御史,總督漕運,兼巡撫江北諸府。十八年,歲大祲,疏請振濟。發銀五萬兩,復敕瓚移淮安倉糧分振,而瓚已卒。

瓚功名著西蜀。其後撫蜀者如謝士元輩,雖有名,不及瓚。惟夭壩乾之役,或言楊輝溺愛庶長子友,欲官之,詐言生苗為亂,瓚信而興師,其功不無矯飾云。

謝士元,字仲仁,長樂人。景泰五年進士。授戶部主事。督通州倉,陳四弊,屢與監倉宦官忤。天順七年擢建昌知府。地多盜,為軍將所庇。士元以他事持軍將,奸發輒得。民懷券訟田宅,士元叱曰:「偽也,券今式,而所訟乃二十年事。」民驚服,訟為衰止。考滿,進從三品俸,治府事如故,以憂去。

服闋,起知廣信。永豐有銀礦,處州民盜發之,聚數千人。將士憚其驍彍,不敢剿。士元勒兵趨之,賊遮刺士元,傷左股。裹創力戰,獲其魁,塞礦穴而還。入覲,改永平。遭喪不赴。

服闋,擢四川右參政,進右布政使。弘治元年就擢右副都御史,巡撫其地。土番大小姓者,將煽亂,士元托行邊,馳詣其地。賊恐,羅拜道左,徐慰遣之。歲大祲,流民趨就食。士元振恤有方,全活者數萬。明年,坐事下獄。事白,遂致仕。

孔鏞,字韶文,長洲人。景泰五年進士。知都昌縣,分戶九等以定役,設倉水次,便收斂,民甚賴之。以弟銘尚寧府郡主,改知連山。瑤、僮出沒鄰境,縣民悉竄。鏞往招之,民驚走。鏞炊飯民舍,留錢償其直以去。民乃漸知親鏞,相率還。鏞慰勞振恤,俾復故業,教以戰守。道路漸通,縣治遂復。都御史葉盛徵廣西,以鏞從。諸將妄殺者,鏞輒力爭,所全活甚眾。

成化元年,用葉盛等薦,擢高州試知府。前知府劉海以瑤警,閉城門自護。鄉民避瑤至者輒不納,還為瑤所戕。又疑民陰附賊,輒戮之。賊緣是激眾怒,為內應,城遂陷。鏞至,開門納來者,流亡日歸。城不能容,別築城東北居之。附郭多暴骸,民以疫死,復為義塚瘞焉。

時賊屯境內者凡十餘部,而其魁馮曉屯化州,鄧公長屯茅峒,屢招不就。鏞一日單騎從二人直抵茅峒。峒去城十里許,道遇賊徒,令還告曰:「我新太守也。」公長驟聞新守至,亟呼其黨擐甲迎。及見鏞坦易無騶從,氣大沮。鏞徐下馬,入坐庭中,公長率其徒馳甲羅拜。鏞諭曰:「汝曹故良民,迫凍餒耳。前守欲兵汝,吾今奉命為汝父母。汝,我子也。信我,則送我歸,賚汝粟帛。不信,則殺我,即大軍至,無遺種矣。」公長猶豫,其黨皆感悟泣下。鏞曰:「餒矣,當食我。」公長為跪上酒饌。既食,曰:「日且暮,當止宿。」夜解衣酣寢。賊相顧駭服。再宿而返。見道旁裸而懸樹上者纍纍,詢之,皆諸生也,命盡釋之。公長遣數十騎擁還,城中人望見,皆大驚,謂知府被執,來紿降也,盡登陴。鏞止騎城外,獨與羸卒入,取穀帛,使載歸。公長益感激,遂焚其巢,率黨數千人來降。

公長既降,諸賊次第納款,惟曉恃險不服。鏞選壯士二百人,乘夜抵化州。曉倉皇走匿,獲其妻子以歸,撫恤甚厚,曉亦以五百人降。已,與僉事陶魯敗賊廖婆保。他賊先後來犯,多敗去。境內大定。上官交薦,擢按察副使,分巡高、雷二府。益招劇賊染定、侯大六、鄧辛酉等,給田產,分處內地為官,備他盜。廣西賊犯信宜、岑溪,皆擊敗之。治績聞,賜誥命旌異。遭喪,服除,改廣西。瑤、僮聞鏞至,悉遠循。

十四年,兵部上其功,賚銀幣,尋進按察使。荔浦賊來寇,總督朱英以兵屬鏞,擊平之,進食二品祿。

已,遷左布政使。旋以右副都御史巡撫貴州。清平部苗阿溪者,桀驁多智。其養子阿賴尤有力,橫行諸部中,守臣皆納溪賂,驕不可制。鏞行部至清平,詢得溪所暱者二人。遂以計擒溪,磔之,并討平雞背苗,郡蠻震懾。

鏞居官廉。歷仕三十餘年,皆在邊陲,觸瘴成疾。乞骸骨,不許。弘治二年召為工部右侍郎,道卒,年六十三。

平樂李時敏者,為信宜知縣。嘗與鏞共平瑤亂,有功,遷知化州。粵人以孔李並稱。

鄧廷瓚,字宗器,巴陵人。景泰五年進士。知淳安縣,有惠政。丁母憂,服除,遷太僕寺丞。貴州新設程番府,地在萬山中,蠻僚雜居,吏部難其人,特擢廷瓚為知府。至則悉心規畫,城郭、衢巷、學校、壇廟、廨舍,以次興建。榜諭諸僚受約束。政平令和。巡撫陳儼上其治行。帝令久任。九載秩滿,始遷山東左參政,尋進右布政使。

弘治二年以右副都御史巡撫貴州。廷瓚自令至守,淹常調者踰三十年。至是去知府止三歲,遂得開府。以生母憂歸。服闋,還原任。都勻苗乜富架、長腳等作亂,敕廷瓚提督軍務,同湖廣總兵官顧溥、貴州總兵官王通等討之。副使吳倬遣熟苗詐降富架,誘令入寇,伏兵擒其父子。官軍乘勝連破百餘寨,生繫長腳以歸,群蠻震懾。廷瓚言:「都勻;清平舊設二衛、九長官司,其人皆世祿,自用其法,恣虐,激變苗民,亂四十餘年。今元兇就除,非大更張不可。請改為府縣,設流官與土官兼治,庶可久安。」因上善後十一事,帝悉從之。遂設府一,曰都勻,州二,曰獨山、麻哈;縣一,曰清平。苗患自此漸戢。論功,進右都御史。

八年召掌南京都察院事。甫數月,命提督兩廣軍務兼巡撫。越二年,進左。廷瓚治尚簡易,於吏事但總大綱,結群蠻以恩信,不輕用兵,而兵出必成功。鬱林、雪CL、大桂諸蠻及四會饑民作亂,以次討平,兩廣遂無事。十三年復召掌南院。未行,卒。贈太子少保,謚襄敏。

廷瓚有雅量,待人不疑,時多稱其長者。至所設施,動中機宜。其在貴州平苗功為尤偉云。

王軾,字用敬,公安人。天順八年進士。授大理右評事,遷右寺正。錄囚四川,平反百餘人,擢四川副使。歲凶,請官銀十萬兩為糴費。以按嘉定同知盛崇仁贓罪,被訐下吏。事白,還職,改陜西。

弘治初,擢四川按察使。三年遷南京右僉都御史,提督操江。八年進右副都御史,總理南京糧儲,旋命巡撫貴州。明年入為大理卿,詔與刑部裁定條例頒天下。

十三年拜南京戶部尚書。尋命兼左副都御史,督貴州軍務,討普安賊婦米魯。時鎮守中官楊友、總兵官曹愷、巡撫錢鉞共發兵討魯,大敗於阿馬坡。都指揮吳遠被執,普安幾陷。友等請濟師,乃以命軾。軾未至,而友等遣人招賊。賊揚言欲降,益擁眾攻圍普安、安南衛城,斷盤江道,勢愈熾。又乘間劫執友。右布政使閭鉦,按察使劉福,都指揮李宗武、郭仁、史韜、李雄、吳達等死焉。

軾至,以便宜調廣西、湖廣、雲南、四川官軍、土兵八萬人,合貴州兵,分八道進,使致仕都督王通將一軍。十五年正月,參將趙晟破六墜寨。賊遁,過盤江。都指揮張泰等渡江追擊,指揮劉懷等遂進解安南衛圍,而愷、通及都指揮李政亦各破賊寨。賊還攻平夷衛及大河、扼勒諸堡,都御史陳金以雲南兵禦之。賊遁歸馬尾籠寨。官軍聚攻益急,土官鳳英等格殺米魯,餘黨遂平。用兵凡五月,破賊寨千餘,斬首四千八百有奇,俘獲一千二百。捷聞,帝大喜,嘉勞。召還京,賜賚有加,錄功,加太子少保。已,改南京兵部,參贊機務。連乞致仕,不允。武宗立,遇疾復請。詔加太子太保。賜敕乘傳歸。卒,贈太保,謚襄簡。

劉丙,字文煥,南雄知府實孫也。成化末,登進士。選庶吉士,改御史,巡按雲南。雲南諸司吏,舊不得給由,父滿子代,丙請如例考入官。流戍僉發,必經兵部,多淹延致死。丙請屬之撫、按。土官無後者,請錄其弟姪,勿令妻妾冒冠服。俱著為例。後督兩淮鹽課,中官請引二萬為織造費,部議許之,丙執不可,得減四之三。歷福建、四川副使,俱督學校,三遷四川左布政使。

正德六年以右副都御史巡撫湖廣。所部鎮溪千戶所、筸子坪長官司與貴州銅仁,四川酉陽、梅桐諸土司,犬牙相錯。弘治中,錯溪苗龍麻陽與銅仁苗龍童保聚眾攻剽,土官李樁等實縱之,而筸子百夫長龍真與通謀。後遂四出劫掠,遠近騷然,先後守臣莫能制。丙將討之,賊入連山深箐,為拒守計。丙率師破其數寨。賊走據天生崖及六龍山。貴州巡撫沈林兵繼至,連攻破之。前後擒童保等二百人,斬首八百九十餘級。都指揮潘勛又破鎮、筸諸寨,擒麻陽等百六十人,斬首級如前,餘賊遠遁。璽書獎勵。

丙操履清介,敢任事。所至嚴明,法令修舉。遷工部右侍郎,採木入山。越二載,犯風痺得疾,卒。詔贈尚書,謚恭襄。

贊曰:英、景間,瓦剌逼西陲,邊圉孔棘;而黃蕭養、葉宗留之徒劫掠嶺南、浙、閩境上。其後荊、襄流民嘯聚,則以劉通、石龍為之魁。他若都勻、松、茂、黔、楚諸苗、瑤叛者數起。羅亨信、侯璡諸人,保固封圻,誅虓禁亂,討則有功,撫則信著,宣力封疆,無忝厥任矣。孔鏞以知府服叛瑤,其才力有過人者。韓愈言柳中丞行事適機宜,風採可畏愛。不如是,惡能以有為哉。


\end{pinyinscope}