\article{列傳第三十九}

\begin{pinyinscope}
△茹瑺嚴震直張紞毛泰亨王鈍鄭賜郭資呂震李至剛方賓吳中劉觀

茹瑺,衡山人。洪武中,由監生除承敕郎,歷通政使。勤於職,太祖賢之。二十三年拜右副都御史,又試兵部尚書,尋實授,加太子少保。及惠帝即位,改吏部,與黃子澄不相能,刑部尚書暴昭發其贓罪,出掌河南布政司事。尋復召為兵部尚書。

燕兵至龍潭,帝遣瑺及曹國公李景隆、都督同知王佐詣燕軍議和。瑺等見成祖,伏地流汗,不能發一言。成祖曰:「公等言即言耳,何懼至是。」久之,乃言奉詔割地講和。成祖笑曰:「吾無罪而削為庶人,今救死,何以地為!且皇考封諸子,已各有分地矣。其縛姦臣來,吾即解甲謁孝陵歸籓。」瑺等唯唯頓首還。

成祖入京師,召瑺。瑺首勸進。成祖既即位,下詔言景隆、瑺、佐及陳瑄事太祖忠,功甚重。封瑺忠誠伯,食祿一千石,終其身。仍兵部尚書、太子少保。選其子鑑為秦府長安郡主儀賓。即命瑺出營郡主府第。

還朝,坐不送趙王,遣歸里。既而為家人所訟,逮至京。釋還。過長沙不謁谷王,王以為言。時方重籓王禮,谷王又開金川門有功,帝意向之。陳瑛遂劾瑺違祖制,逮下錦衣獄。瑺知不免,命子銓市毒藥,服之死。時永樂七年二月也。法司劾銓毒其父,請以謀殺父母論。後以銓實承父命,減死,與兄弟家屬二十七人謫戍廣西河池。仁宗立,釋還。宣宗與所沒田廬。

瑺居官謹慎,謙和有容。其死也,人頗惜之。

嚴震直,字子敏,烏程人。洪武時以富民擇糧長,歲部糧萬石至京師,無後期,帝才之。二十三年特授通政司參議,再遷為工部侍郎。二十六年六月進尚書。時朝廷事營建,集天下工匠於京師,凡二十餘萬戶。震直請戶役一人,書其姓名、所業於官,有役則按籍更番召之,役者稱便。鄉民訴其弟姪不法,帝付震直訊。具獄上,帝以為不欺,赦其弟姪。已,坐事降御史,數雪冤獄。

二十八年討龍州,使震直偕尚書任亨泰諭安南。還,條奏利病,稱旨。尋命修廣西興安縣靈渠。審度地勢,導湘、漓二江,浚渠五千餘丈,築渼潭及龍母祠土堤百五十餘丈,又增高中江石堤,建陡閘三十有六,鑿去灘石之礙舟者,漕運悉通。歸奏,帝稱善。

三十年二月疏言:「廣東舊運鹽八十五萬餘引於廣西,召商中買。今終年所運,纔十之一。請分三十萬八千餘引貯廣東,別募商入粟廣西之糧衛所,支鹽廣東,鬻之江西南安、贛州、吉安、臨江四府便。」帝從之。廣鹽行於江西自此始。

其年四月擢右都御史,尋復為工部尚書。建文中,嘗督餉山東,已而致仕。成祖即位,召見,命以故官巡視山西。至澤州,病卒。

張紞,字昭季,富平人。洪武中,舉明經。為東宮侍書,累遷試左通政。十五年,雲南平,出為左參政。陛辭,帝賦詩二章賜之。歷左布政使。二十年春入覲,治行為天下第一,特令吏部勿考。賜璽書曰:「曩者討平西南,命官撫守,爾紞實先往,於今五年。諸蠻聽服,誠信相孚,克恭乃職,不待考而朕知其功出天下十二牧上。故嘉爾績,命爾仍治滇南。往,欽哉。」紞在滇凡十七年,土地貢賦、法令條格皆所裁定。民間喪祭冠婚咸有定制,務變其俗。滇人遵用之。朝士董倫、王景輩謫其地,皆接以禮意。

惠帝即位,召為吏部尚書。詔徵遺逸士集闕下。紞所選用,皆當其才。會修《太祖實錄》,命試翰林編纂官,紞奏楊士奇第一。士奇由是知名。

成祖入京師,錄中朝姦臣二十九人,紞與焉。以茹瑺言,宥仍故職。無何,帝臨朝而歎,咎建文時之改官制者。乃令紞及戶部尚書王鈍解職務,月給半俸,居京師。紞懼,自經於吏部後堂,妻子相率投池中死。

紞在吏部,值變官制,小吏張祖言曰:「高皇帝立法創制,規模甚遠。今更之,未必勝,徒滋人口,願公力持之。」紞不能用,然心賢祖,奏為京衛知事。後紞死,屬吏無敢視者,唯祖經紀其喪。世傳燕師入京,紞即自經死;嚴震直奉使至雲南,遇建文君悲愴吞金死。考諸國史,非其實也。

時有毛泰亨者,建文時為吏部侍郎,與紞同事。紞死,泰亨亦死。

王鈍,字士魯,太康人。元末猗氏縣尹。洪武中,徵授禮部主事,歷官福建參政,以廉慎聞。遣諭麓川,卻其贈。或曰:「不受恐遠人疑貳。」鈍乃受之。還至雲南,輸之官庫。二十三年遷浙江左布政使。在浙十年,名與張紞埒。帝嘗稱於朝,以勸庶僚。

建文初,拜戶部尚書。成祖入,踰城走,為邏卒所執。詔仍故官。未幾,與紞俱罷。尋命同工部尚書嚴震直等分巡山西、河南、陜西、山東,又同新昌伯唐雲經理北平屯種。承制再上疏言事,皆允行。永樂二年四月賜敕以布政使致仕。既歸,鬱鬱死。

子淪,永樂四年進士。仁宗時遷鄭王府左長史,數以禮諫王。嘗擬荀卿《成相篇》,撰十二章以獻。語切,與王不合。召改戶部郎中。英宗即位,擢戶部右侍郎,巡撫浙江,有惠政。母喪起復,入覲,留攝部事。尋以老乞歸,卒。

鄭賜,字彥嘉,建寧人。洪武十八年進士。授監察御史。時天下郡邑吏多坐罪謫戍,賜嘗奉命於龍江編次行伍。方暑,諸囚憊甚。賜脫其械,俾僦舍止息,周其飲食,病者與醫藥,多所全活。秩滿當遷,湖廣布政司參議闕,命賜與檢討吳文為之。二人協心劃弊,民以寧輯,苗獠畏懷。母喪,去。服除,改北平參議,事成祖甚謹。復坐累謫戍安東屯。及惠帝即位,成祖及楚王楨皆舉賜為長史。不許,召為工部尚書。燕兵起,督河南軍扼燕。成祖入京師,李景隆訐賜罪亞齊、黃。逮至,帝曰:「吾於汝何如,乃相背耶?」賜曰:「盡臣職耳。」帝笑釋之,授刑部尚書。

永樂元年,劾都督孫岳擅毀太祖所建寺,詔安置海南。岳,建文時守鳳陽,嘗毀寺材,修戰艦以禦燕軍,燕知其有備,取他道南下,故賜劾之。二年劾李景隆陰養亡命,謀不軌。又與陳瑛同劾耿炳文僭侈,炳文自經死。皆揣帝意所惡者。祁陽教諭康孔高朝京師還,枉道省母。會母疾,留侍九閱月不行。賜請逮問孔高,罪當杖。帝曰:「母子暌數年,一旦相見難遽舍,況有疾,可矜也。」命復其官。

三年秋,代李至剛為禮部尚書。四年正月,西域貢佛舍利,賜因請釋囚。帝曰:「梁武、元順溺佛教,有罪者不刑,紀綱大壞,此豈可效!」是年六月朔,日當食,陰雲不見,賜請賀。不許。賜言「宋盛時嘗行之。」帝曰:「天下大矣,京師不見,如天下見之何?」卒不許。

賜為人頗和厚,然不識大體,帝意輕之。為同官趙羾所間,六年六月憂悸卒。帝疑其自盡。楊士奇曰:「賜有疾數日,惶懼不敢求退。昨立右順門,力不支仆地,口鼻有噓無吸。」語未竟,帝曰:「微汝言,幾誤疑賜。賜固善人,才短耳。」命予葬祭。洪熙元年贈太子少保,謚文安。

郭資,武安人。洪武十八年進士。累官北平左布政使,陰附於成祖。及兵起,張昺等死,資與左參政孫瑜、按察司副使墨麟、僉事呂震率先降,呼萬歲。成祖悅,命輔世子居守。

成祖轉戰三年,資主給軍餉。及即位,以資為戶部尚書,掌北平布政司。北京建,改行部尚書,統六曹事。定都,仍改戶部。時營城郭宮殿,置官吏及出塞北征,工役繁興,資舉職無廢事。仁宗立,以舊勞兼太子賓客。尋以老病,加太子太師,賜敕致仕。宣德四年,復起戶部尚書,奉職益勤。八年十二月卒,年七十三。贈湯陰伯,謚忠襄。官其子佑戶部主事。

資治錢穀有能稱,仁宗嘗以問楊士奇。對曰:「資性強毅,人不能干以私。然蠲租詔數下不奉行,使陛下恩澤不流者,資也。」

呂震,字克聲,臨潼人。洪武十九年以鄉舉入太學。時命太學生出稽郡邑壤地,以均貢賦。震承檄之兩浙,還奏稱旨,擢山東按察司試僉事。入為戶部主事,遷北平按察司僉事。燕兵起,震降於成祖,命侍世子居守。永樂初,遷真定知府,入為大理寺少卿。三年遷刑部尚書。六年改禮部。皇太子監國,震婿主事張鶴朝參失儀,太子以震故宥之。帝聞之怒,下震及蹇義於錦衣衛獄。已,復職。仁宗即位,命兼太子少師,尋進太子太保兼禮部尚書。宣德元年四月卒。

震嘗三奉命省親,兩值關中饑,令所司出粟振之,還始以聞。然無學術,為禮官,不知大體。成祖崩,遺詔二十七日釋縗服。及期,震建議群臣皆易烏紗帽,黑角帶。近臣言:「仁孝皇后崩,既釋縗服,太宗易素冠布腰絰。」震勃然變色,詆其異己。仁宗黜震議,易素冠布腰絰。洪熙元年,分遣群臣祀嶽鎮海瀆及先代帝王陵,震乞祀周文、武、成、康。便道省母,私以妻喪柩與香帛同載。祀太廟致齋,飲酒西番僧舍,大醉歸,一夕卒。

震為人佞諛傾險。永樂時,曹縣獻騶虞,榜葛剌國、麻林國進麒麟,震請賀。帝曰:「天下治安,無麒麒何害?」貴州布政使蔣廷瓚言:「帝北征班師,詔至思南大巖山,有呼萬歲者三。」震言:「此山川效靈。」帝曰:「山谷之聲,空虛相應,理或有之。震為國大臣,不能辯其非,又欲因之進媚,豈君子事君之道?」郎中周訥請封禪,震力贊之,帝責其謬。震雖累受面斥,然終不能改。金水河、太液池冰,具樓閣龍鳳花卉狀。帝召群臣觀之。震因請賀。不許。而隆平侯張信奏太和山五色雲見,侍郎胡濙圖上瑞光榔梅靈芝,震率群臣先後表賀云。

成祖初巡北京,命定太子留守事宜。震請常事聽太子處分,章奏分貯南京六科,回鑾日通奏。報可。十一年、十四年,震再請如前制。十七年,帝在北京,因事索章奏,侍臣言留南京。帝忘震前請,曰:「章奏宜達行在,豈禮部別有議耶?」問震。震懼罪,曰:「無之,奏章當達行在。」三問,對如前。遂以擅留奏章,殺右給事中李能。眾知能冤,畏震莫敢言。尹昌隆之禍,由震構之。事具《昌隆傳》。夏原吉、方賓以言北征餉絀得罪,以震兼領戶、兵部事。震亦自危。帝令官校十人隨之,曰:「若震自盡,爾十人皆死。」

震有精力,能彊記,才足以濟其為人。凡奏事,他尚書皆執副本,又與左右侍郎更進迭奏。震既兼三部,奏牘益多,皆自占奏,侍郎不與也。情狀委曲,千緒萬端,背誦如流,未嘗有誤。嘗扈北狩,帝見碑立沙磧中,率從臣讀其文。後一年,與諸文學臣語及碑,詔禮部遣官往錄之。震言不須遣使,請筆札帝前疏之。帝密使人拓其本校之,無一字脫誤者。

子熊。宣宗初立,震數於帝前乞官,至流涕。帝不得已,授兵科給事中。

李至剛,名鋼,以字行,松江華亭人。洪武二十一年舉明經。選侍懿文太子,授禮部郎中。坐累謫戍邊,尋召為工部郎中,遷河南右參議。河決汴堤,至剛議借王府積木,作筏濟之。建文中,調湖廣左參議,坐事繫獄。

成祖即位,左右稱其才,遂以為右通政。與修《太祖實錄》,朝夕在上左右,稱說洪武中事,甚見親信。尋進禮部尚書。永樂二年冊立皇太子,至剛兼左春坊大學士,直東宮講筵,與解縉後先進講。已,復坐事下獄,久之得釋,降禮部郎中。恨解縉,中傷之。縉下獄,詞連至剛,亦坐繫十餘年。仁宗即位,得釋,復以為左通政。給事中梁盛等劾至剛輩十餘人,當大行晏駕,不宿公署,飲酒食肉,恬無戚容。帝念至剛先朝舊人,出為興化知府,時年已七十。再歲,歿於官。

至剛為人敏給,能治繁劇,善傅會。首發建都北平議,請禁言事者挾私,成祖從之。既得上心,務為佞諛。嘗言太祖忌辰,宜傚宋制,令僧道誦經。山東野蠶成繭,至剛請賀。陜西進瑞麥,至剛率百官賀。帝皆不聽。中官使真臘,從者逃三人,國王以國中三人補之。帝令遣還,至剛言:「中國三人,安知非彼私匿?」帝曰:「朕以至誠待內外,何用逆詐。」所建白多不用。

妻父麗重法,至剛為乞免。帝曰:「獄輕重,外人何以知之?」至剛曰:「都御史黃信為臣言。」帝怒,誅信。初,至剛與解縉交甚厚。帝書大臣姓名十人,命縉疏其人品,言至剛不端。縉謫廣西,至剛遂奏其怨望,改謫交阯。

方賓,錢塘人。洪武時由太學生試兵部郎中。建文中,署應天府事。坐罪戍廣東。以茹瑺薦,召復官。成祖入京師,賓與侍郎劉俊等迎附,特見委用,進兵部侍郎。四年,俊以尚書出征黎利,賓理部事,有幹才,應務不滯。性警敏,能揣上意,見知於帝,頗恃寵貪恣。七年進尚書,扈從北京,兼掌行在吏部事。明年從北征,與學士胡廣、金幼孜、楊榮,侍郎金純並與機密。自後帝北巡,賓輒扈從。

十九年,議親征。尚書夏原吉、吳中、呂震與賓共議,宜且休兵養民。未奏,會帝召賓,賓言糧餉不足,召原吉,亦以不給對。帝怒,遣原吉視糧開平,旋召還下獄。賓方提調靈濟宮。中使進香至,語賓以帝怒。賓懼,自縊死。帝實無意殺賓,聞賓死,乃益怒,戮其屍。

吳中,字思正,武城人。洪武末,為營州後屯衛經歷。成祖取大寧,迎降。以轉餉捍禦功,累遷至右都御史。永樂五年,改工部尚書。從北征,艱歸。起復,改刑部。十九年,與夏原吉、方賓等同以言北征餉絀,忤旨繫獄。仁宗即位,出之,復其官,兼詹事,加太子少保。宣德元年從征樂安。三年坐以官木石遺中官楊慶作宅,下獄,落宮保,奪祿一年。正統六年,殿工成,進少師。明年卒,年七十。追封茌平伯,謚榮襄。

中勤敏多計算。先後在工部二十餘年,北京宮殿,長、獻、景三陵,皆中所營造。職務填委,規畫井然。然不恤工匠,又湛於聲色,時論鄙之。

劉觀,雄縣人。洪武十八年進士。授太谷縣丞,以薦擢監察御史。三十年遷署左僉都御史。坐事下獄,尋釋。出為嘉興知府,丁父憂去。

永樂元年,擢雲南按察使,未行,拜戶部右侍郎。二年調左副都御史。時左都御史陳瑛殘刻,右都御史吳中寬和,觀委蛇二人間,務為容悅。四年,北京營造宮室,觀奉命採木浙江,未幾還。明年冬,帝以山西旱,命觀馳傳往,散遣採木軍民。六年,鄭賜卒,擢禮部尚書。十二月與刑部尚書呂震易官。坐事為皇太子譴責。帝在北京聞之,以大臣有小過,不宜遽折辱,特賜書諭太子。八年,都督僉事費瓛討涼州叛羌,命觀贊軍事。還,坐事,謫本部吏。十三年還職,改左都御史。十五年督浚河漕。十九年命巡撫陜西,考察官吏。

仁宗嗣位,兼太子賓客,旋加太子少保,給二俸。時大理少卿弋謙數言事,帝厭其繁瑣。尚書呂震、大理卿虞謙希旨劾奏,觀復令十四道御史論其誣妄,以是為輿論所鄙。

時未有官妓之禁。宣德初,臣僚宴樂,以奢相尚,歌妓滿前。觀私納賄賂,而諸御史亦貪縱無忌。三年六月朝罷,帝召大學士楊士奇、楊榮至文華門,諭曰:「祖宗時,朝臣謹飭。年來貪濁成風,何也?」士奇對曰:「永樂末已有之,今為甚耳。」榮曰:「永樂時,無踰方賓。」帝問:「今日誰最甚者?」榮對曰:「劉觀。」又問:「誰可代者?」士奇、榮薦通政使顧佐。帝乃出觀視河道,以佐為右都御史。於是御史張循理等交章劾觀,並其子輻諸贓污不法事。帝怒,逮觀父子,以彈章示之。觀疏辨。帝益怒,出廷臣先後密奏,中有枉法受賕至千金者。觀引伏,遂下錦衣衛獄。明年將置重典。士奇、榮乞貸其死。乃謫輻戍遼東,而命觀隨往,觀竟客死。七年,士奇請命風憲官考察奏罷有司之貪污者,帝曰:「然。向使不罷劉觀,風憲安得肅。」

贊曰:成祖封茹瑺,以事太祖有功。然考之,未有所表見,意史軼之歟?嚴震直之於廣西,張紞之於雲南,治效卓然。王鈍、鄭賜為方伯、監司,聲績頗著。至其晚節,皆不克自振,惜夫。郭資、呂震之徒,有幹濟才,而操行無取。李至剛之險,吳中、劉觀之墨,又不足道矣。


\end{pinyinscope}