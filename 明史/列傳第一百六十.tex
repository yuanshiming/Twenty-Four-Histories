\article{列傳第一百六十}

\begin{pinyinscope}
金國鳳楊振楊國柱曹變蛟朱文德李輔明劉肇基乙邦才馬應魁莊子固

金國鳳,宣府人。崇禎中,以副總兵守松山。十二年二月,大清以重兵來攻,環城發炮,臺堞俱摧。城中人負扉以行。國鳳間出兵突擊,輒敗還,乃以木石甃補城壞處。大清兵屢登屢卻,遂分兵攻塔山、連山,令銳卒分道穴城。國鳳多方拒守,終不下,閱四旬圍解。帝大喜,立擢署都督僉事,為寧遠團練總兵官。再論功,署都督同知,蔭錦衣衛千戶。是年十月,大清兵復攻寧遠。國鳳憤將士心匡怯,率親丁數十人出據北山岡鏖戰。移時矢盡力竭,與二子俱死。帝聞痛悼,贈特進榮祿大夫,左都督,賜祭葬,有司建祠,增世職三級。總督洪承疇上言:「國鳳素懷忠勇。前守松山,兵不滿三千,乃能力抗強敵,卒保孤城。非其才力優也,以事權專,號令一,而人心肅也。迨擢任大將,兵近萬人,反致隕命。非其才力短也,由營伍紛紜,號令難施,而人心不一也。乞自今設連營節制之法,凡遇警守城,及統兵出戰,惟總兵官令是聽。庶軍心齊肅,戰守有資,所係於封疆甚大。」帝即允行之。及國鳳父子柩歸,帝念其忠,命所過有司給以舟車,且加二祭。其妻張氏援劉綎例,乞加宮保。部議格不行,而請於世職增級外,再廕本衛試百戶世襲,以勸忠臣。帝可之。

當松山被圍,巡撫方一藻議遣兵救援,諸將莫敢應。獨副將楊振請行,至呂洪山遇伏,一軍盡覆。振被執,令往松山說降。未至里許,踞地南向坐,語從官李祿曰:「為我告城中人堅守,援軍即日至矣。」祿詣城下致振語,城中守益堅。振、祿皆被殺。事聞,命優恤。

振,義州衛人。世為本衛指揮使。天啟二年,河東失守,歸路梗,其母自縊。振隨父及弟夜行晝伏,渡鴨綠江入皮島。毛文龍知其父子才,並署軍職。文龍死,振歸袁崇煥,為寧遠千總。崇禎二年從入衛。救開平有功,進都司僉書。郵馬山之戰,以遊擊進參將。久之,擢副總兵。監視中官高起潛招致之,不往。中以他事,落職。用一藻薦,復官,及是死難。

振從父國柱,崇禎九年為宣府總兵官。十一年冬,入衛畿輔,從總督盧象昇戰賈莊。象升敗歿,國柱當坐罪。大學士劉宇亮、侍郎孫傳庭皆言其身入重圍,非臨敵退卻者比。乃充為事官,戴罪圖功。十四年,祖大壽被困錦州,總督洪承疇率八大將往救。國柱先至松山,陷伏中。大清兵四面呼降,國柱太息,語其下曰:「此吾兄子昔年殉難處也,吾獨為降將軍乎!」突圍,中矢墮馬卒。事聞,贈恤如制。

國柱二子俱殀。妻何氏以所遺甲胄弓矢及戰馬五十三匹獻諸朝。帝深嘉歎,命授一品夫人,有司月給米石,餼之終身。

曹變蛟,文詔從子也,幼從文詔積軍功至遊擊。崇禎四年從復河曲。明年連破賊紅軍友等於張麻村、隴安、水落城、唐毛山,又破劉道江等於銅川橋,勇冠諸軍。以御史吳甡薦,進參將。文詔移山西,變蛟從戰輒勝。及文詔改鎮大同,山西巡撫許鼎臣言:「晉賊紫金梁雖死,老回回、過天星、大天王、蠍子塊、闖塌天諸渠未滅。變蛟驍勇絕人,麾下健兒千百,才乃文詔亞,乞留之晉中。」從之。

七年,群賊入湖廣,命變蛟南征。文詔困於大同,又命北援。七月遇大清兵廣武,有戰功。其冬,文詔失事論戍,變蛟亦以疾歸。明年,文詔起討陜西賊,變蛟以故官從。大捷金嶺川,鏖真寧之湫頭鎮,皆為軍鋒。文詔既戰歿,變蛟收潰卒,復成一軍。總督洪承疇薦為副總兵,置麾下,與高傑破賊關山鎮,逐北三十餘里。又與副將尤翟文、遊擊孫守法追闖王高迎祥,與戰鳳翔官亭,斬首七百餘級。又與總兵左光先敗迎祥乾州。迎祥中箭走,斬首三百五十餘級。已而迎祥自華陰南原絕大嶺,夜出朱陽關。光先戰不利,賴變蛟陷陣,乃獲全。九年破闖將澄城。偕光先等追至靖虜衛,轉戰安定、會寧,抵靜寧、固寧,賊屢挫。其秋追混天星等,敗之蒲城。賊西走平涼、鞏昌,復擊破之。

十年二月,巡撫孫傳庭部卒許忠叛,勾賊混十萬謀犯西安。變蛟方西追過天星,聞亂急還,賊遂遁。傳庭已誅迎祥,其黨闖將混天星、過天星踞洮、岷、階、文深谷間。承疇遣變蛟、光先及祖大弼、孫顯祖合擊。四月望,入山,遇賊郭家壩,大雨。諸將力戰,賊死傷無算,食盡引還。九月,階州陷,與光先並停俸。俄擢都督僉事,為臨洮總兵官。當是時,承疇、傳庭共矢滅賊。傳庭戰於東,承疇戰於西,東賊幾盡。賊在西者,復由階、成出西和、禮縣。光先、顯祖皆無功,獨變蛟降小紅狼。餘賊竄走徽州、兩當、成、鳳間,不敢大逞。十月,賊瞷蜀中虛,陷寧羌州,分三道,連陷三十餘州縣。承疇率變蛟等由沔縣歷寧羌,過七盤、朝天二關。山高道狹,士馬饑疲,歲暮抵廣元,賊已走還秦。變蛟等回軍邀擊,斬首五百餘級。時兵部尚書楊嗣昌創「四正六隅」之說,限三月平賊。十一年四月以滅賊踰期,普議降罰,變蛟、光先並鐫五級,戴罪辦賊。

賊之再入秦也,其渠魁號六隊者,與大天王、混天王、爭管王四部連營東犯,混天星、過天星二部仍伏階、文,獨闖將李自成以三月自洮州出番地。承疇令變蛟偕賀人龍追之,連戰斬首六千七百有奇。番地乏食,賊多死亡。變蛟轉戰千里,身不解甲者二十七晝夜。餘賊潰入塞。大弼駐洮州,扼戰不力。乃走入岷州及西和、禮縣山中。變蛟還剿,賊伏匿不敢出,惟六隊勢猶張。六月,光先自固原進兵,賊已奔隴州、清水。光先追至秦州,六隊及爭管王復走成縣、階州,為變蛟所扼。其別部號三隊及仁義王、混天王降於光先,而自成、六隊及其黨祁總管避秦兵,復謀犯蜀,副將馬科、賀人龍拒之。將還走階、文及西鄉,憚變蛟,乃走漢中,又為光先所扼。六隊、祁總管皆降,惟自成東遁。承疇令變蛟窮追,而設三伏於潼關之南原。變蛟追及,大呼斫賊。伏盡起,賊屍相枕藉。村民用大棒擊逃者。自成妻女俱失,從七騎遁去。餘皆降。是時,曹兵最強,各鎮依之以為固,錄關中平賊功,進變蛟左都督。

十一月,京師戒嚴,召承疇入衛,變蛟及光先從之。明年二月,抵近畿,帝遣使迎勞,將士各有賜。未幾,戰渾河,無功。再戰太平砦北,小有斬獲。及解嚴,留屯遵化。麾下皆秦卒,思歸,多逃亡者,追斬之乃定。時張獻忠、羅汝才既降復叛,陜西再用兵。總督鄭崇儉乞令變蛟兵西還,帝不許,尋用為東協總兵官。

十三年五月,錦州告急。從總督承疇出關,駐寧遠。七月與援剿總兵左光先、山海總兵馬科、寧遠總兵吳三桂、遼東總兵劉肇基,遇大清兵於黃土臺及松山、杏山,互有殺傷。大清兵退屯義州。承疇議遣變蛟、光先、科之兵入關養銳,留三桂、肇基於松、杏間,佯示進兵狀。又請解肇基任,代以王廷臣;遣光先西歸,代以白廣恩。部議咸從之,而請調旁近邊軍,合關內外見卒十五萬人備戰守。用承疇言,師行糧從,必芻糧足支一歲,然後可議益兵。帝然之,敕所司速措給。

徵宣府總兵楊國柱、大同總兵王樸、密雲總兵唐通各揀精兵赴援。以十四年三月偕變蛟、科、廣恩先後出關,合三桂、廷臣凡八大將,兵十三萬,馬四萬,並駐寧遠。

承疇主持重,而朝議以兵多餉艱,職方郎張若麒趣戰。承疇念祖大壽被圍久,乃議急救錦州。七月二十八日,諸軍次松山,營西北岡。數戰,圍不解。八月,國柱戰歿,以山西總兵李輔明代之。承疇命變蛟營松山之北,乳峰山之西,兩山間列七營,環以長壕。俄聞我太宗文皇帝親臨督陣,諸將大懼。及出戰,連敗,餉道又絕。樸先夜遁。通、科、三桂、廣恩、輔明相繼走。自杏山迤南沿海,東至塔山,為大清兵邀擊,溺海死者無算。變蛟、廷臣聞敗,馳至松山,與承疇固守。三桂、樸奔據杏山。越數日,欲走還寧遠。至高橋遇伏,大敗,僅以身免。先後喪士卒凡五萬三千七百餘人。自是錦州圍益急,而松山亦被圍,應援俱絕矣。九月,承疇、變蛟等盡出城中馬步兵,欲突圍出,敗還。守半年,至明年二月,副將夏成德為內應,松山遂破。承疇、變蛟、廷臣及巡撫丘民仰,故總兵祖大樂,兵備道張斗、姚恭、王之楨,副將江翥、饒勳、朱文德,參將以下百餘人皆被執見殺,獨承疇與大樂獲免。

文德,義州衛人,後家錦州。崇禎時,積功至松山副將。忤監視中官高起潛,為所中,斥罷。十一年起故官。及城被圍,領前鋒拒守甚力,城破竟死。

三月,大壽遂以錦州降。杏山、塔山連失,京師大震。詔賜諸臣祭葬,有司建祠。變蛟妻高氏以贈廕請,乃贈榮祿大夫、太子少保,世蔭錦衣指揮僉事。

法司會鞫王樸罪。御史郝晉言:「六鎮罪同,皆宜死。三桂實遼左主將,不戰而逃,奈何反加提督?」兵部尚書陳新甲覆議,請獨斬樸,勒科軍令狀,再失機即斬決。三桂失地應斬,念守寧遠功,與輔明、廣恩、通皆貶秩,充為事官。

輔明,遼東人,累官副總兵。崇禎八年從祖寬擊賊,連蹙之嵩縣、汝州、確山。明年追破賊於滁州。敘功,加都督僉事。十二年擢山西總兵官,被劾罷。明年從承疇出關,使代國柱,竟敗。十六年為援剿總兵。是冬,大清兵薄寧遠,輔明馳援,軍敗猶力戰,歿於陣。事聞,贈特進榮祿大夫、左都督,世廕錦衣副千戶,賜祭葬,列壇前屯祀之。

樸,榆林衛人。父威,官左都督,九佩將印,為提鎮者五十年。兄世欽,里居殉難,見《尤世威傳》中。樸由父廕屢遷京營副將。崇禎六年,賊躪畿南,命樸與倪寵為總兵官,將京軍六千,監以中官楊應朝、盧九德,屢有斬獲功,進右都督。明年代曹文詔鎮大同,進左都督。九年秋,都城被兵,詔樸入衛,賚蟒衣彩幣,竟無功。十一年加太子太保。是冬,從總督盧象昇入衛,方戰欒城、束鹿間。或言大同有警即引兵歸。及是救錦州,以首逃下詔獄。十五年五月伏誅。

科,起偏裨至大帥,戰功亞變蛟,與三桂同守寧遠有功。十六年春,督兵入衛,賜宴武英殿,命從大學士吳甡南征,不果行。明年三月從李建泰西征。李自成兵至,科遂降,封懷仁伯。

廣恩,初從混天猴為盜。既降,屢立戰功。松山敗還,代馬科鎮山海關。是年十一月,京師戒嚴,廣恩入衛,賚銀幣羊酒。俄戰龍王口,稍有斬獲,以捷聞。帝始惡廣恩觀望,降旨譙責,而冀其後效,特命敘功。明年四月合八鎮兵戰螺山,悉潰敗。總督趙光抃請帝召之入,用為武經略。廣恩以帝頻戮大將,己又多過,懼不敢至,假索餉名,頓真定。大學士吳甡將南征,密請帝嚴旨逮治,而己力救,率之剿寇。廣恩感甚。無何,帝遣中官齎二萬金犒其軍,且諭以溫旨。廣恩遂驕,不為甡用,大掠臨洺關,徑歸陜西。帝不得已,命隸督師孫傳庭辦賊。十月,郟縣師覆,加廣恩盪寇將軍,俾緣道收潰卒以保潼關。未幾,潼關亦破,廣恩西奔固原。賊將追躡及之,即開門降。自成大喜,握手共飲,封桃源伯。

通,口辯無勇略。既敗歸,仍鎮密雲。其年冬,奉詔入衛,命守禦三河、平谷。大清兵下山東,通尾之而南,抵青州,迄不敢一戰。明年復尾而北,戰螺山,敗績。已,命從甡南征。甡未行而斥,乃令通轄薊鎮西協。五月汰密雲總兵官,命兼轄中協四路。尋用孔希貴於西協,而命通專轄中協。十月,關外有警,命率師赴援,以銀牌二百為賞功用。事定,復移鎮西協。帝顧通厚,有蟒衣玉帶之賜,召見稱卿而不名,錫之宴,獎勞備至。明年,賊逼宣府,命移守居庸,封定西伯。無何,賊犯關,即偕中官杜之秩迎降,京師遂陷。

光先,梟將也,與賊角陜西,功最多。自遼左遣還,廢不用。後聞廣恩從賊,亦詣賊降。

又有陳永福者,守開封,射李自成中目。及自成陷山西,令廣恩諭之降。永福懼誅,意猶豫。自成折箭以示信,乃降,封為文水伯。後自成敗還山西,永福為守太原,殺晉府宗室殆盡。

劉肇基,字鼎維,遼東人。嗣世職指揮僉事,遷都司僉書,隸山海總兵官尤世威麾下。崇禎七年從世威援宣府,又從剿中原賊。進遊擊,戍雒南蘭草川。明年遇賊,戰敗傷臂。未幾,世威罷,肇基及遊擊羅岱分將其兵,與祖寬大破賊汝州,斬首千六百有奇。後從寬數有功,而其部下皆邊軍,久戍思歸,與寬軍噪而走。總理盧象昇乃遣之入秦。其秋,畿輔有警,始還山海,竟坐前罪解職,令從征自效。俄以固守永平功復職,屢遷遼東副總兵。

十二年冬,薊遼總督洪承疇請用為署總兵官,分練寧遠諸營卒。兵部尚書傅宗龍稍持之,帝怒,下宗龍獄,擢肇基都督僉事任之。明年三月,錦州有警。承疇命吳三桂偕肇基赴松山為聲援。三桂困松、杏間,肇基救出之,喪士卒千人。七月與曹變蛟等戰黃土臺及松山、杏山。九月,復戰杏山,肇基軍稍卻。承疇甄別諸將,解肇基職,代以王廷臣。十七年春,加都督同知,提督南京大教場,及福王立,史可法督師淮、揚,肇基請從征自效。屢加左都督、太子太保。可法議分布諸將,奏薦李成棟、賀大成、王之綱、李本身、胡茂楨為總兵官。成棟鎮徐州,大成揚州,之綱開封。本身、茂楨隸高傑麾下,為前鋒。而令肇基駐高家集,李樓鳳駐睢寧,以防河。棲鳳本甘肅總兵,以地失留淮、揚間也。閣標前鋒,則用張天祿駐瓜洲。十一月,肇基、棲鳳以可法命謀取宿遷。初八日渡河,復其城。越數日,大清兵圍邳州,軍城北,肇基軍城南,相持半月,大清兵引去。

順治二年三月,大清兵抵揚州,可法邀諸將赴援。獨肇基自白洋河趨赴,過高郵不見妻子。既入城,請乘大清兵未集,背城一戰。可法持重,肇基乃分守北門,發炮傷圍者。已而城破,率所部四百人巷戰,格殺數百人。後騎來益眾,力不支,一軍皆沒。副將乙邦才、馬應魁、莊子固等皆同死。

乙邦才,青州人。崇禎中,以隊長擊賊於河南、江北間。大將黃得功與賊戰霍山,單騎逐賊,陷淖中。賊圍而射之,馬斃,得功徒步斗。天將暮,僅餘二矢。邦才大呼衝賊走,得功乃得出。邦才授以己馬,分矢與之,且走且射,殪追騎十餘人,始得及其軍。得功自是知邦才。

時有張衡者,亦以驍敢名。賊圍六安急,總督馬士英救之。甫至,斥其左右副將,而號於軍中曰:「孰為乙邦才、張衡者?」兩人入謁,即牒補副將,以其兵授之,曰:「為我入六安,取知州狀來報。」兩人出,即簡精騎二百,夜衝賊陣而入,繞城大呼,曰:「大軍至矣,固守勿懈!」城中人喜,守益堅。兩人促知州署狀,復奪圍出,不損一騎。

時潁、壽、六安、霍山諸州縣數被寇,邦才大小十餘戰,咸有功。及可法鎮揚州,攜之行。至是戰敗,自刎死。

馬應魁,字守卿,貴池入。初為小將,率家丁五十人巡村落間。猝遇賊,眾懼慾奔。應魁大聲曰:「勿怖死!死,命也。」連發二矢殪二賊,賊即退。可法因拔為副總兵,俾領旗鼓。每戰披白甲,大書「盡忠報國」四字於背,至是巷戰死。

莊子固,字憲伯,遼東人,年十三,殺人亡命。後從軍有功,積官至參將。嘗從山西總兵許定國救開封,軍半道噪歸,定國獲罪。子固輯餘眾,得免議。後可法出鎮,用為副總兵,俾興屯於徐州、歸德間。子固募壯士七百人,以赤心報國為號。聞揚州被圍,率眾馳救,三日而至。城將破,欲擁可法出城,遇大清兵,格鬥死。

他若副將樓挺、江雲龍、李豫,參將陶國祚、許謹、馮國用、陳光玉、李隆、徐純仁,游擊李大忠、孫開忠,都司姚懷龍、解學曾等十餘人,皆以巷戰死。

贊曰:金國鳳之善守,曹變蛟之力戰,均無愧良將材。然而運移事易,難於建功,而易於挫敗,遂至謀勇兼絀,以身殉之。蓋天命有歸,莫之為而為者矣。


\end{pinyinscope}