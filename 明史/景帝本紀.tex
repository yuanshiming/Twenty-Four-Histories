\article{景帝本紀}

\begin{pinyinscope}
恭仁康定景皇帝,諱祁鈺,宣宗次子也。母賢妃吳氏。英宗即位,封郕王。

正統十四年秋八月,英宗北狩,皇太后命王監國。丙寅,移通州糧入京師。徵兩畿、山東、河南備倭運糧諸軍入衛,召寧陽侯陳懋帥師還。戊辰,兵部侍郎于謙為本部尚書。令群臣直言時事,舉人材。己巳,皇太后詔立皇子見深為皇太子。恤陣亡將士。庚午,籍王振家。辛未,右都御史陳鎰撫安畿內軍民。壬申,都督石亨總京營兵。乙亥,諭邊將,瓦剌秦駕至,不得輕出。輸南京軍器於京師。修撰商輅、彭時入閣預機務。是月,廣東賊黃蕭養作亂。九月癸未,王即皇帝位,遙尊皇帝為太上皇帝,以明年為景泰元年,大赦天下,免景泰二年田租十之三。甲申,夷王振族。庚寅,處州賊平。癸巳,指揮僉事季鐸奉皇太后命,達於上皇。甲午,祭宣府、土木陣亡將士,瘞遺骸。乙未,總兵官安鄉伯張安討廣州賊,敗死。指揮僉事王清被執,死之。辛丑,給事中孫祥、郎中羅通為右副都御史,守紫荊居庸關。甲辰,遣御史十五人募兵畿內、山東、山西、河南。都督同知陳友帥師討湖廣、貴州叛苗。乙巳,遣使奉書上皇。丙午,苗圍平越衛,調雲南、四川兵會王驥討之。參議楊信民為右僉都御史,討廣東賊。

冬十月戊申,也先擁上皇至大同。壬子,詔諸王勤王。乙卯,于謙提督諸營,石享及諸將分守九門。丙辰,也先陷紫荊關,孫祥死之,京師戒嚴。丁巳,詔宣府、遼東總兵官,山東、河南、山西、陜西巡撫及募兵御史將兵入援。戊午,也先薄都城,都督高禮、毛福壽敗之於彰義門。己未,右通政王復、太常少卿趙榮使也先營,朝上皇於土城。庚申,徵兵於朝鮮,調河州諸衛士軍入援。于謙、石亨等連敗也先眾於城下。壬戌,寇退。甲子,出紫荊關,丁卯,詔止諸王兵。瓦剌可汗脫脫不花使來。辛未,昌平伯楊洪充總兵官,都督孫鏜、范廣副之,剿畿內餘寇。十一月癸未,修沿邊關隘。辛卯,毛福壽為副總兵,討辰州叛苗。壬辰,上皇至瓦剌。乙未,侍郎耿九疇撫安南畿流民,賜復三年。十二月庚戌,尊皇太后為上聖皇太后。辛亥,王驥為平蠻將軍,充總兵官,討貴州叛苗。都督同知董興為左副總兵,討廣東賊,戶部侍郎孟鑒參贊軍務。癸丑,尊母賢妃為皇太后。甲寅,立妃汪氏為皇后。丙辰,大赦。己未,石亨、楊洪、柳溥分練京營兵。戊辰,祭陣亡官軍於西直門外。

是年,琉球中山、占城、烏斯藏、撒馬兒罕入貢。

景泰元年春正月丁丑朔,罷朝賀。辛巳,城昌平。壬午,享太廟。丙戌,大祀天地於南郊。閏月甲寅,瓦刺寇寧夏。癸亥,詔會試取士毋拘額。庚午,大同總兵官郭登敗瓦剌於沙窩,又追敗之於栲栳山,封登定襄伯。是月,免大名、真定、開封、衛輝被災稅糧。二月戊寅,耕耤田。癸未,懸賞格招陷敵軍民。丙戌,石亨為鎮朔大將軍,帥師巡大同。都指揮同知楊能充遊擊將軍,巡宣府。壬辰,太監喜寧伏誅。三月己酉,瓦剌寇朔州。辛亥,錄土木死事諸臣後。癸丑,瓦剌寇寧夏、慶陽。乙卯,寇朔州。癸亥,免畿內逋賦及夏稅。

夏四月丙子,廣東都指揮李升、何貴帥兵捕海賊,戰死。辛巳,瓦剌寇大同,官軍擊卻之。丁亥,保定伯梁珤代王驥討貴州叛苗。戊子,大理寺丞李茂錄囚南京,考黜百司,訪軍民利病。丙申,瓦剌寇雁門。己亥,都督同知劉安充總兵官,練兵於保定、真定及涿、易、通三州,僉都御史曹泰參贊軍務。庚子,振山東饑。辛丑,振畿內被寇州縣。癸卯,瓦剌寇大同,郭登擊卻之。五月乙巳,免山西被災稅糧。瓦剌掠河曲、代州,遂南犯,詔劉安督涿、易諸軍禦之。戊申,瓦剌寇雁門,益黃花鎮戍兵衛陵寢。癸丑,董興擊破廣東賊,黃蕭養伏誅。壬戌,振大同被寇軍民。丙寅,侍郎侯璡、副總兵田禮大破貴州苗。辛未,瓦剌遣使請和。六月壬午,瓦剌寇大同,郭登擊卻之。丙戌,也先復擁上皇至大同。丁亥,左都御史陳鎰、王文以鞫太監金英家人不實下獄,尋釋之。戊子,瓦剌寇宣府,都督朱謙、參將紀廣禦卻之。戊戌,免山東被災州縣稅糧。乙亥,給事中李實、大理寺丞羅綺使瓦剌。

秋七月庚戌,尚書侯璡、參將方瑛破貴州苗,擒其酋獻京師。庚申,右都御史楊善、工部侍郎趙榮使瓦剌。停山西民運糧大同。癸亥,李實、羅綺還。己巳,楊善至瓦剌,也先許上皇歸。八月癸酉,上皇發瓦剌。戊寅,祀社稷。甲申,遣侍讀商輅迎上皇於居庸關。丙戌,上皇還京師。帝迎於東安門,入居南宮。帝帥百官朝謁。庚寅,赦天下。辛卯,刑部右侍郎江淵兼翰林學士,直文淵閣,預機務。九月癸丑,巡撫河南副都御史王來總督湖廣、貴州軍務,討叛苗。

冬十月辛卯,錄囚。癸巳,免畿內逋賦。十一月辛亥,禮部尚書胡濙請令百官賀上皇萬壽節。十二月丙申,復請明年正旦百官朝上皇於延安門。皆不許。

是年,朝鮮貢馬者三。

二年春正月庚戌,大禮天地於南郊。壬子,詔天下朝覲官當黜者運糧口外。二月辛未,釋奠於先師孔子。辛卯,以星變修省,詔廷臣條議寬恤諸政。癸巳,詔畿內及山東巡撫官舉廉能吏專司勸農,授民荒田,貸牛種。三月壬寅,賜柯潛等進士及第、出身有差。

夏四月乙酉,梁珤、王來等破平越苗,獻俘京師。甲午,瓦剌寇宣府馬營,敕遊擊將軍石彪等巡邊。乙未,命石亨選京營兵操練,尚書石璞總督軍務。五月乙巳,城固原。六月戊辰朔,日當食不見。己卯,詔貴州各衛修舉屯田。

秋七月戊申,普定、永寧、畢節諸苗復叛,梁珤等留軍討之。八月壬申,南京地震。辛巳,復午朝。九月乙卯,禁諸司起復。

冬十月己丑,免山西被災稅糧。十二月庚寅,禮部左侍郎王一寧、祭酒蕭兼翰林學士,直文淵閣,預機務。是月,也先弒其主脫脫不花。

是年,安南、琉球中山、瓦剌、哈密入貢。

三年春正月丙午,大祀天地於南郊。二月乙酉,副都御史劉廣衡錄南京囚。戊子,戶部尚書金濂以違詔下獄,尋釋之。三月戊午,毛福壽討湖廣巴馬苗,克之。

夏五月甲午,廢皇太子見深為沂王,立皇子見濟為皇太子。廢皇后汪氏,立太子母杭氏為皇后。封上皇子見清榮王,見淳許王。大赦天下。丙申,築沙灣堤成。辛丑,河南流民復業者,計口給食五年。乙巳,官顏、孟二氏子孫各一人。六月乙亥,罷各省巡撫官入京議事。是月,大兩,河決沙灣。

秋七月乙未,左都御史王翱總督兩廣軍務。壬寅,王一寧卒。八月乙丑,振徐、兗水災。戊辰,都御史洪英,尚書孫原貞、薛希璉等分行天下,考察官吏。丁丑,振兩畿水災州縣,免稅糧。乙酉,振南畿、河南、山東流民。九月庚寅,江淵起復。辛卯,以南京地震,兩淮大水,河決,命都御史王文巡視安輯。乙未,振兩畿、山東、山西、福建、廣西、江西、遼東被災州縣。閏月癸未,開處州銀場。是月,福建盜起。

冬十月戊戌,左都御史王文兼翰林學士,直文淵閣,預機務。丙辰,都督孫鏜、僉事石彪協守大同,都督同知衛潁,僉事楊能、張欽協守宣府,備也先。十一月己未朔,日有食之。戊辰,都督方瑛平白石崖諸苗。甲戌,安輯畿內、山東、山西逃民,復賦役五年。是月,免山東及淮、徐水災稅糧。十二月癸巳,始立團營,太監阮讓、都督楊俊等分統之,聽于謙、石亨、太監劉永誠、曹吉祥節制。是月,免河南及永平被災秋糧。

是年,瓦剌、琉球中山、爪哇、暹羅、安南、哈密、烏斯藏入貢。

四年春正月辛未,大祀天地於南郊。二月戊子,五開、清浪諸苗復叛,梁珤、王來討之。

庚戌,免江西去年被災秋糧。三月戊寅,開建寧銀場。

夏四月戊子,築沙灣決口。運南京倉粟振徐州。五月丁巳,發徐、淮倉振饑民。己巳,王文起復。甲戌,徐州復大水,民益饑。發支運及鹽課糧振之。丁丑,發淮安倉振鳳陽。乙酉,沙灣河復決。六月壬辰,吏部尚書何文淵以給事中林聰言下獄,尋令致仕。辛亥,瘞土木、大同、紫荊關暴骸。

秋七月庚辰,停諸不急工役。八月己丑,振河南饑。甲午,也先自立為可汗。

冬十月庚寅,詔天下鎮守、巡撫官督課農桑。甲午,諭德徐有貞為左僉都御史,治沙灣決河。戊戌,也先遣使來。十一月辛未,皇太子見濟薨。十二月乙未,免山東被災稅糧。乙巳,賚邊軍。

是年,琉球中山、安南、爪哇、日本、占城、哈密、瓦剌入貢。

五年春正月戊午,黃河清,自龍門至於芮城。甲子,大祀天地於南郊。壬申,罷福州,建寧銀場。甲戌,平江侯陳豫、學士江淵撫輯山東、河南被災軍民。二月乙巳,以雨暘弗時,詔修省,求直言。三月壬子,賜孫賢等進士及第、出身有差。辛酉,學士江淵振淮北饑民。王文撫恤南畿。甲子,總督兩廣副都御史馬昂破瀧水瑤。庚辰,緬甸執獻思機發。

夏四月壬午朔,日有食之。辛卯,方瑛破草塘苗,封瑛南和伯。五月甲子,禮部郎中章綸、御史鐘同以請復沂王為皇太子下錦衣衛獄。六月戊子,錄囚。

秋七月癸酉,振南畿水災。八月丁酉,復命天下巡撫官赴京師議事。九月壬戌,免蘇、松、常、揚、杭、嘉、湖漕糧二百餘萬石。

冬十月庚辰,副都御史劉廣衡巡撫浙江、福建,專司討賊。十一月戊午,罷蘇、松、常、鎮織造採辦。十二月,免南畿、浙江被災稅糧。

是年,安南、琉球中山、爪哇入貢。也先為知院阿剌所殺。

六年春正月戊午,大祀天地於南郊。二月壬午,太監王誠同法司、刑科錄囚。大理少卿李茂等錄南京、浙江囚。

夏四月丙子朔,日有食之。辛巳,敕戶、兵二部及兩畿、山東、河南、浙江、湖廣撫、按、三司官條寬恤事,罷不急諸務。五月己巳,禱雨於南郊。六月乙亥,宋懦朱熹裔孫梃為翰林院世襲《五經》博士。癸未,河決開封。

秋七月乙亥。沙灣決口隄成。庚寅,以南京災異屢見,敕群臣修省。八月庚申,南京大理少卿廖莊又請復沂王為皇太子,杖於闕下,并杖章綸、鐘同於獄,同卒。九月乙亥,振蘇、松饑民米麥一百餘萬石。

冬十月戊午,免陜西被災稅糧。十一月乙亥,南和伯方瑛為平蠻將軍充總兵官,討湖廣苗。十二月己巳,免南畿被災秋糧。

是年,琉球中山、暹羅、哈密、滿剌加入貢。

七年春正月己卯,尚書石璞撫安湖廣軍民。壬午,大祀天地於南郊。二月庚申,皇后崩。甲子,營壽陵。三月戊寅,免雲南被災稅糧。

夏五月戊寅,以水旱災異,敕內外諸臣修省。辛卯,宋儒周敦頤裔孫冕為翰林院世襲《五經》博士。六月庚申,葬肅孝皇后。

冬十月癸卯,振江西饑。十二月己亥,方瑛大破湖廣苗。戊午,振畿內、山東、河南水災。癸亥,帝不豫,罷明年元旦朝賀。是冬,免畿內、山東被災稅糧,並蠲逋賦。

是年,琉球中山、撒馬兒罕、烏斯藏入貢。

八年春正月戊辰,免江西被災稅糧。丁丑,帝輿疾宿南郊齋宮。己卯,群臣請建太子,不聽。壬午,武清侯石亨、副都御史徐有貞等迎上皇復位。二月乙未,廢帝為郕王,遷西內。皇太后吳氏以下悉仍舊號。癸丑,王薨於西宮,年三十。謚曰戾。毀所營壽陵,以親王禮葬西山,給武成中衛軍二百戶守護。

成化十一年十二月戊子,制曰:「朕叔郕王踐阼,戡難保邦,奠安宗社,殆將八載。彌留之際,奸臣貪功,妄興讒構,請削帝號。先帝旋知其枉,每用悔恨,以次抵諸奸於法,不幸上賓,未及舉正。朕敦念親親,用成先志,可仍皇帝之號,其議謚以聞。」遂上尊謚。敕有司繕陵寢,祭饗視諸陵。

贊曰:景帝當倥傯之時,奉命居攝,旋王大位以繫人心,事之權而得其正者也。篤任賢能,勵精政治,強寇深入而宗社乂安,再造之績良云偉矣。而乃汲汲易儲,南內深錮,朝謁不許,恩誼恝然。終於輿疾齋宮,小人乘間竊發,事起倉猝,不克以令名終,惜夫!


\end{pinyinscope}