\article{列傳第一百八十二 忠義六}

\begin{pinyinscope}
○夏統春薛聞禮等陳美郭裕等諶吉臣張國勛等盧學古朱士完等陳萬策李開先許文岐李新等郭以重岳璧郭金城崔文榮朱士鼎徐學顏李毓英等馮雲路熊鷿明睿易道暹傅可知蔡道憲周二南等張鵬翼歐陽顯宇等劉熙祚王孫蘭程良籌程道壽黃世清楊暄朱一統等唐時明薛應玢唐夢鯤段復興靳聖居等簡仁瑞何相劉等司五教張鳳翮都任王家錄等祝萬齡王徵等陳瑸周鳳岐王徵俊宋之俊等丁泰運尚大倫等

夏統春,字元夫,桐城人。為諸生,慷慨有才志。用保舉授黃陂丞,嘗攝縣事,著廉能聲。十五年,賊犯黃陂。統春已遷麻陽知縣,未赴,乃督眾拒守,凡十五晝夜,賊忽解去。統春度賊必再至,而眾已疲甚,休於家。閱五日,賊果突至,城遂陷。統春巷戰,力竭被執,欲屈之。統春指賊魁大罵,賊怒,斷其右手。復以左手指賊罵,賊又斷之。罵不已,乃割其舌,目怒視,眥欲裂,賊又剜其目。猶以頭觸賊,遂支解之。

有薛聞禮者,武進人。由府吏官黃陂典史。歲歉,民逋漕粟。聞禮奉使過漢口,貸於所知得千金,以代民逋。十六年,張獻忠陷黃陂,愛聞禮才,挾與俱去,暮即亡歸。會賊所設偽官為士民殺死,聞禮曰「禍大矣」,令士民遠避,而己獨留以當之。俄賊至,將屠城。聞禮挺身曰:「殺偽官者,我也。」賊欲活之,詈不止,乃見殺。

當是時,賊延蔓中原,覆名城不可勝數。其以小吏死難,有何宗孔、賈儒秀、張達、郝瑞日諸人。宗孔,紫陽典史。十一年五月,流賊再陷其城,死之。儒秀,商南典史,城陷,抗節死。達,興山典史。十四年二月,張獻忠自蜀來攻,都司徐日耀戰歿,達被縛,罵賊不屈死。瑞日,陜西人,為固始巡檢。羅山為賊陷,上官令瑞日攝縣事。單騎攜二童以往,至則止僧寺,將招流移為守禦計。未踰月,賊遣偽官至,土寇萬朝勛與之合。誘執瑞日,說之降,不從,拘於家。一日,朝勳置酒宴群賊,醉臥,瑞日潛入其室,殺之。將奔鳳陽,雨阻,復見縶。賊愛其勇,欲留之,叱曰:「我雖小吏,亦朝廷臣子,肯為賊用耶!」遂被害,二僕亦死。

有朱耀者,固始人。與父允義、兄炳、思成並以勇力聞。八年,賊來犯,耀父子力戰卻之。明年,賊復至。耀出戰,手馘數十人,追之,陷伏中,大罵死。允義曰:「我必報子仇。」炳謂思成曰:「我二人必報弟仇。」三人率眾奮擊,賊解去,城獲全。

陳美,字在中,新建人。崇禎時由鄉舉知宜城縣。兵燹之餘,民生凋瘵。及張獻忠據穀城,人情益懼,美安輯備至。襄陽陷,賊兵來犯。美偕守備劉相國迎擊,賊中伏敗去。巡按御史上其功,獲敘錄。撫治都御史王永祚以六等課所部有司,美居上上。薦於朝,未及擢用。十五年冬,李自成長驅犯襄陽,左良玉先奔,永祚及知府以下俱遁。賊入城,鄉官羅平、知州蔡思繩、福州通判宋大勛殉節。賊分兵寇宜城、棗陽、穀城、光化、均州。美守宜城,固拒八晝夜。城陷,抗罵不已,為賊磔死。訓導陽城田世福亦死之。

棗陽知縣郭裕,清江舉人。甫視事,張獻忠至。左良玉屯近邑,裕單騎邀與共禦,賊卻去。至是,賊將劉福來攻,裕發炮石,擊傷多。賊憤,攻益力,城陷。身被數槊,大罵。賊支解之,闔門遇害。

光化知縣萬敬宗,南昌人,貢生,到官以死自誓。賊薄城,遂自盡。賊義之,引去,城獲全。鄉官韓應龍,舉人,歷長蘆鹽運使,不受偽職,自縊死。穀城知縣周建中亦殉節。均州知州胡承熙被熱不屈,與其子爾英俱死。承熙有能聲,永祚課屬吏,亦列上上,遷刑部員外郎,未行,遇難。賊犯鄖陽,同知劉璇死之。保康陷,知縣萬惟壇與妻李氏俱列之。璇,永年人。惟壇,曹縣人。俱貢生。

諶吉臣,字仲貞,南昌人。父應華,萬歷時,以參將援朝鮮,戰歿。吉臣由舉人為雲夢知縣。崇禎十五年十二月,李自成陷襄陽,其黨賀一龍陷德安。吉臣急遣孥歸,身誓死勿去。明年正月,雲夢陷,被執,不食累日。賊臨以兵,吉臣乞速死。賊壯之,授以官,不屈。驅上馬,曰:「我失守封疆,當死此,更安往。」乃見殺。福王時,贈太僕寺丞。

賊分兵犯旁邑,應城陷,訓導張國勳死之。國勳,黃陂人。城將陷,詣文廟抱先師木主大哭,為賊所執,大罵不屈,支解死。妻子十餘人皆殉節。

袁啟觀者,雲夢諸生也。賊據城,啟觀立寨自守。賊執去,出題試之。啟觀曰:「汝既知文,亦知亂臣賊子,人人得而誅之耶?」賊怒,殺之。

安陸城陷,知縣分水濮有容一門十九人皆死。鄉民結寨自保,賊將白旺連破數十寨,諸生廖應元守益堅。奸人執送旺,旺問:「汝欲何為?」厲聲曰:「欲殺賊耳!」賊怒,射殺之。應山舉人劉申錫養死士百人,城陷,謀恢復。兵敗,為旺所殺,百人皆戰死。沔陽陷,同知馬飆死之。

盧學古,夏縣人。舉人。歷承天府同知,攝荊門州事。崇禎十五年十二月,李自成寇荊門,學古誓死守。學正黃州張郊芳、訓導黃岡程之奇亦盟諸生於大成殿,佐城守。賊環攻四日,無援,城陷。學古罵賊不絕口,剖腹而死。郊芳、之奇亦不屈死。

有朱士完者,潛江舉人。鄉試揭榜夕,夢墨幟墮其墓門,粉書「亂世忠臣」四字。至是,賊破承天,長驅陷潛江。士完被執,械送襄陽,道由泗港,齧指血書己盡節處,遂自經。賊所過焚毀,士完所題壁獨存。

彭大翮者,竟陵之青山人。賊逼承天,大翮出所著《平賊權略》上之當事,不能用。遂自集一旅保鄉曲,邀斬賊過當。賊怒,雨夜襲之。大融太息曰:「吾子孫陣亡已盡,吾何用生為!」赴水死。

賊既陷荊門,遂向荊州。巡撫陳睿謨急渡江入城,奉惠王常潤南奔,監司以下皆奔,士民遂開門迎賊。訓導撖君錫正衣冠端坐明倫堂。賊至,欲屈之,詬罵而死。君錫,字賓王,絳縣人。賊大索縉紳,故相張居正子尚寶丞允修不食死。戶部員外郎李友蘭不屈死。諸生王維籓率妻朱及二女避難,為賊所掠。維籓令妻女赴井死,遂見殺。諸生王圖南被執,抗罵死。

夷陵李雲,由鄉舉知潁川州,州人祠祀之。謝事歸。流賊熾,大書「名義至重,鬼神難欺」二語於牖以自警。及城陷,不屈。執至江陵,絕食死。呂調元者,歸州千戶也。城陷,士民悉歸附,調元獨率部卒格鬥,陷重圍中。招之降,大罵,死亂刀下。

陳萬策,江陵人。天啟中,與同邑李開先先後舉於鄉,並有時名。崇禎十六年正月,李自成據襄陽,設偽官。其吏政府侍郎石首喻上猷,先為御史,降賊,薦兩人賢可用。自成遣使具書幣征之。萬策隱龍灣市,賊使至,歎曰:「我為名誤,既不能奮身滅賊,尚可惜頂踵耶?」夜自經。賊使至開先家,開元瞋目大罵,頭觸牆死。福王時,俱命優恤。

許文岐,字我西,仁和人。祖子良,巡撫貴州右僉都御史。父聯樞,廣西左參政。文岐,崇禎七年進士。歷南京職方郎中。賊大擾江北,佐尚書范景文治戎備,景文甚倚之。遷黃州知府,射殺賊前鋒一隻虎,奪大纛而還。獄有重囚七人,縱歸省,剋期就獄,皆如約至,乃請於上官貸之。十三年遷下江防道副使,駐蘄州。賊魁賀一龍、藺養成等萃蘄、黃間,文岐設備嚴。賊黨張雄飛將南渡,命遊擊楊富焚其舟,賊乃卻。巡撫宋一鶴上其功。副將張一龍善馭兵,文岐重之。嘗共宿帳中,軍中夜呼噪,文岐曰「此奸人乘夜思遁耳」,堅臥不出。質明,叛兵百餘人奪門遁,一龍追獲盡斬之,一軍肅然。楊富既久鎮蘄,一鶴復遣參將毛顯文至,不相得,兵民洶洶。文岐會二將,以杯酒釋之,始無患。十五年,左良玉潰兵南下大掠。文岐立馬江口迎之,兵莫敢犯。時警報日急,人無固志,會擢督糧參政當行,文岐歎曰:「吾為天子守孤城二載矣,分當死封疆,雖危急,奈何棄之。」遣妻奉母歸,檄富、顯文出屯近郊,為固守計。無何,荊王府將校郝承忠潛通張獻忠。明年大舉兵來攻,文岐發炮斃賊甚眾。夜將半,雪盈尺,賊破西門入,文岐巷戰。雪愈甚,炮不得發,遂被執。獻忠聞其名,不殺,擊之後營。時舉人奚鼎鉉等數十人同繫,文岐密謂曰:「觀賊老營多烏合,凡此數萬卒皆被掠良民,若告以大義,同心協力,賊可殲也。」於是陰相結,期四月起事,以柳圈為信。謀洩,獻忠索之,果得柳圈,縛文岐斬之。將死,語人曰:「吾所以不死者,志滅賊耳。今事不成,天也。」含笑而死,時文岐陷賊中已七十餘日矣。事聞,贈太僕卿。

賊既陷蘄州,遂屠其民。鄉官陜西僉事李新舉家被執,賊欲屈之。新叱曰:「我昔官秦中,爾輩方為廝養,今日肯屈膝廝養耶!」賊怒,新抱父屍就刃。其時屬吏死節者,惟麻城教諭定遠蕭頌聖、蘄水訓導施州童天申。

郭以重,黃州人。世為衛指揮。崇禎十六年,城陷,自他所來赴難。其妻欲止之,叱曰:「朝家畀我十三葉金紫,不能易一死哉!吾將先殺汝。」妻乃不敢言。既至,遇賊欲脅之去,堅不從。露刃懾之,乃好謂賊曰:「從汝非難,但抱小兒者,吾妻也,汝為我殺之,吾無累矣。」賊如其言。以重即奪賊刀擊斬一賊,群賊擁至,遂赴水死。

先是,蘄州破,指揮岳璧自屋墮地,不死。賊執至城上,欲降之。厲聲曰:「我世臣也,城亡與亡,豈降賊!」賊刃之,仆地。氣將絕,瞋目曰:「我死為鬼,當滅汝!」時大雪,血流丈餘,目眥不合。

同時,郭金城為羅田守將,賊逼城,率所部五百人戰,斬級百餘,追之英山。賊大集,困三日,突圍不得出,被執。脅降不從,見殺。

崔文榮,海寧衛人。世指揮僉事,舉武會試,授南安守備。崇禎中,臨、藍盜起,逼桂陽,桂王告急。文榮督所部會剿,卻賊四萬人。以功,擢武昌參將。十六年四月,張獻忠犯漢陽,文榮渡江襲斬六百級。已而城陷,武昌震懼。巡撫宋一鶴既死,承天新任巡撫王聚奎未至,武昌素不宿重兵,城空虛。或議撤江上兵以守,文榮曰:「守城不如守江,團風、煤炭、鴨蛋諸洲,淺不及馬腹,縱之飛渡,而坐守孤城,非策也。」當事不從。賊果從團風渡江,陷武昌縣。縣無人,賊出營樊口,文榮軍洪山寺扼之。既,斂兵入城,以他將代守。賊全軍由鴨蛋洲畢渡,抵洪山,守將亦退入城。文榮以武勝門當賊衝,偕故相賀逢聖協守,賊攻之不能下。

監軍參政王揚基時已擢右僉都御史,巡撫承天、德安二郡,未聞命,尚駐武昌。見勢急,與推官傅上瑞詭言有事漢陽,開門遁去,人情益洶洶。先是,楚王出資募兵,應募者率蘄、黃潰卒及賊間諜,至是開文昌、保安二門納賊。文榮方出鬥還,闔城扉不及,躍馬大呼,殺三人。賊攢槊刺之,洞胸死。有朱士鼎者,起家武進士,為巡江都司。城陷被執,賊喜其勇敢,欲大用之。戟手大罵,賊斷其右手,乃以左手染血灑賊,賊又斷之,不死。賊退,令人縛筆於臂,能作楷字。招集舊卒,訓練如常。

徐學顏,字君復,永康人。母疾,禱於天,請以身代。夜夢神人授藥,旦識其形色,廣覓之,得荊瀝,疾遂愈。父為中城兵馬指揮,忤權要人下吏。學顏三疏訟冤,所司格不上,遍叩諸公卿莫為雪,將置重辟。學顏號泣爭於刑部,不能得,至齧臂血濺於庭,乃獲釋歸。推所居大宅讓其弟,尚義疏財,族黨德之。崇禎三年建東宮,詔舉孝友廉潔、博物洽聞可勵俗維風者,有司以學顏應,寢不行。十二年以恩貢生授楚府左長史,引義匡輔,王甚敬之。十五年冬,諸司長官及武昌知府、江夏知縣並以朝覲行,學顏攝江夏事,繕修守具。楚府新募兵,即令學顏將之。明年五月晦,新軍內叛,城陷。學顏格鬥,斷左臂,大罵不屈,為賊支解,一家二十餘人殉之。通判固安李毓英亦舉家自縊。

武昌知縣鄒逢吉被害。同死者,武昌衛經歷汪文熙、巡檢戴良瑄及僧官一人,俱罵賊不屈,腰斬。賊既陷武昌,分兵陷屬邑,於是嘉魚知縣霍山王良鑑、蒲圻知縣臨川曾栻俱抗節死。事聞,學顏贈僉事,毓英等贈恤有差。

馮雲路,字漸卿,黃岡人。好學勵行,年三十,即棄諸生,從賀逢聖講學,遂寓居武昌,著書數百卷。崇禎三年,巡按御史林鳴球薦其賢,並上所著書,不用。及賊將渡江,雲路貽書逢聖曰:「在內,以寧湖為止水。在外,以漢江為汩羅。」寧湖者,雲路談經處也。城既陷,乘桴入寧湖。賊遣使來聘,遙應曰:「我平生只讀忠孝書,未嘗讀降賊書也。」遂投湖死。從游諸生汪延陛亦死焉。

其同邑熊寔,字渭公,亦移居武昌。喜邵子《皇極書》,頗言未來事。十六年元旦,盡以所撰《性理格言》、《圖書懸象》、《大易參》諸書付其季弟,曰:「善藏之。」城破前一日,貽書雲路,言「明日當覓我某樹下。」及期行樹傍,賊追至,躍入荷池以死。

有諸生明睿者,江夏人。城破,賊獨不入其門。睿慨然曰:「安有父母之邦覆,而偷生茍活者!」語家人:「速從我入井,否則速去。」於是妻及二子、二女並諸婢以次投井。睿笑曰:「吾今曠然無累矣。」從容榜諸門,赴井死,時人號為明井。

先是,賊陷黃岡,諸生易道暹者,字曦侯。好學尚氣節,居深山中,積書滿家。賊氛漸逼,道暹惜所積書,又以己所著書多,不忍棄,逡巡未行。及賊至,子為瑚急奉母走青峰巖,道暹攜幼子為璉擔事以行。遇賊,紿曰:「餘書賈也。賊笑曰:「汝易曦侯,何紿我。」道暹曰:「若既知我,當聽我一言,慎毋殺人焚廬舍。」賊曰:「若身不保,尚為他人言耶!」道暹厲色叱賊,賊怒殺之。為璉請代,賊並殺之。未幾,為瑚亦被殺。

時黃陂諸生傅可知亦以叱賊死。可知幼喪父,臥柩下三年。六十喪母,啜粥三年。黃陂陷,被執,可知年已踰八十。賊憫其老不殺,俾養馬,叱曰:「我為士數十年,肯役於賊耶!」延頸就刃,賊殺之。

蔡道憲,字元白,晉江人。崇禎十年進士。為長沙推官。地多盜,察豪民通盜者,把其罪而任之。盜方劫富家分財,收者已至。召富家還所失物,皆愕不知所自。惡少年閉戶謀為盜,啟戶,捕卒已坐其門,驚逸去。吉王府宗人恣為奸,道憲先治而後啟王。王召責之,抗聲曰:「今四海鼎沸,寇盜日滋。王不愛民,一旦鋌而走險,能獨與此曹保富貴乎?」王悟,謝遣之。

十六年五月,張獻忠陷武昌,長沙大震。承天巡撫王揚基率所部千人,自岳州奔長沙。道憲請還駐岳州,曰:「岳與長沙脣齒也,並力守岳則長沙可保,而衡、永亦無虞。」揚基曰:「岳,非我屬也。」道憲曰:「棄北守南,猶不失為楚地。若南北俱棄,所屬地安在?」揚基語塞,乃赴岳州。及賊入蒲圻,即遁去。湖廣巡撫王聚奎遠駐袁州,憚賊不敢進。道憲亦請移岳,聚奎不得已至岳,數日即徙長沙。道憲曰:「賊去岳遠,可繕城以守。彼犯岳,猶憚長沙援。若棄岳,長沙安能獨全。」聚奎不從。賊果以八月陷岳州,直犯長沙。先是,巡按御史劉熙祚令道憲募兵,得壯丁五千訓練之,皆可用。至是親將之,與總兵官尹先民等扼羅塘河。聚奎聞賊逼,大懼,撤兵還城。道憲曰:「去長沙六十里有險,可柵以守,毋使賊踰此。」又不從。

時知府堵胤錫入覲未返,通判周二南攝攸縣事,城中文武無幾。賊薄城,士民盡竄。聚奎詭出戰,遽率所部遁。道憲獨拒守,賊繞城呼曰:「軍中久知蔡推官名,速降,毋自苦。」道憲命守卒射之斃。越三日,先民出戰,敗還。賊奪門入,先民降。道憲被執,賊啖以官,嚼齒大罵。釋其縛,延之上坐,罵如故。賊曰:「汝不降,將盡殺百姓。」道憲大哭曰:「願速殺我,毋害我民。」賊知終不可奪,磔之,其心血直濺賊面。健卒林國俊等九人隨不去,賊亦令說道憲降。國俊曰:「吾主畏死去矣,不至今日。」賊曰:「爾主不降,爾輩亦不得活。」國俊曰:「我輩畏死亦去矣,不至今日。」賊並殺之,四卒奮然曰:「願瘞主屍而死。」賊許之,乃解衣裹道憲骸,瘞之南郊醴陵坡,遂自刎。道憲死時年二十九,贈太僕少卿,謚忠烈。

二南,字汝為,雲南人。由選貢為長沙通判,盡職業,與道憲深相得。擢岳州知府,士民固留,乃以新秩還長沙,後亦死。

邑中舉人馮一第走湘鄉,將乞師他所,賊繫其母與兄招之。一第歸就縛,賊將斬之,一老僧伏地哭請免。賊乃去其兩手置營中,一夕死,母兄獲免。賊陷東安,舉人唐德明仰藥死。犯耒陽,諸生謝如珂拒戰死。

張鵬翼,西充人。崇禎中,由選貢生授衡陽知縣。十六年八月,張獻忠逼衡州,巡撫王聚奎、李乾德及監司以下皆遁,士民盡奔竄。鵬翼獨守空城,賊至即陷。脅使降,戟髯詬詈,賊縛而投諸江,妻子赴水死。

賊之趨岳州也,巴陵教諭桂陽歐陽顯宇時攝縣事,死焉。其趨臨湘也,知縣莆田林不息抗罵不屈,斷其兩手殺之。湘陰陷,知縣大埔楊開率家屬十七人投水死。其丞賴萬耀攝醴陵縣事,城破亦死之。長沙府照磨莫可及,宜興人,攝寧鄉縣事,殉城死。二子若鼎、若鈺號慟奔赴,遇害。衡州既陷,屬縣衡山亦失守,知縣富順董我前、教諭分宜彭允中,皆盡節。府教授永明蔣道亨攝武陵縣事,抱印罵賊,見殺。其他文武將吏,非降則逃。長沙史可鏡,官給事中,丁艱歸,降賊,賊用為湖廣巡撫。及賊棄湖廣入四川,李乾德復還長沙,執可鏡,加榜掠,械送南都伏法。

乾德者,亦鵬翼同邑人。崇禎四年進士。十六年歷右僉都御史撫治鄖陽,未赴,改湖南。時武昌已陷,乾德守岳州。獻忠攻急,乾德棄城走長沙,岳州遂陷。轉徙衡、永,賊至,輒先避,長沙、衡、永皆隨陷。獻忠入四川,乃還長沙,以失地,謫赴督師王應熊軍前自效。永明王立,擢兵部侍郎,巡撫川南。乾德入蜀,其鄉邑已陷,父亦被難,乃說諸將袁韜攻佛圖關,復重慶。韜及武大定久駐重慶,食盡。乾德說嘉定守將楊展與大定結為兄弟,資之食。已而惡展,構韜殺之,據嘉定,蜀人咸不直乾德。會劉文秀自雲南至,擒韜,陷嘉定,乾德乃驅家人及其弟御史升德,俱赴水死。

劉熙祚,字仲緝,武進人。父純仁,泉州推官。熙祚舉天啟四年鄉試。崇禎中,為興寧知縣。奸民啖斷腸草,脅人財物。熙祚令贖罪者必以草,以是致死者勿問,草以漸少,弊亦止,課最,徵授御史。十五年冬巡按湖南。李自成陷荊、襄諸郡,張獻忠又破蘄、黃,臨江欲渡。熙祚以明年二月抵岳州,檄諸將分防江滸,偏沅、鄖陽二撫聯絡形勢。會賊馬守應據澧州,窺常德,土寇甘明揚等助之。熙祚馳至常德,擊斬明揚。五月還長沙。

及武昌、岳州相繼陷,急令總兵尹先民、副將何一德督萬人守羅塘河,扼要害。而巡撫王聚奎乃撤守長沙,賊遂長驅至。聚奎率潰將孔全彬、黃朝宣、張先璧等走湘潭,長沙不能守。惠王避地至長沙,與吉王謀出奔,熙祚奉以奔衡州。衡州,桂王封地也,聚奎兵至,大焚劫,王及吉、惠二王皆登舟避亂。熙祚單騎赴永州為城守計。未幾,聚奎復走祁陽,衡州遂陷。永士民聞之,空城逃。三王至永州,聚奎繼至,越日全彬等亦至,劫庫金去。熙祚乃遣部將護三王走廣西,而己返永州拒守。賊騎追執之,獻忠踞桂王宮,叱令跪,不屈。賊群毆之,自殿墄曳至端禮門,膚盡裂。使降將尹先民說之,終不變,見殺。事聞,贈太常少卿,謚忠毅。弟永祚,字叔遠,由選貢生屢遷興化同知,擒賊曾旺。後以副使知興化府事。大清兵入城,仰藥死。弟綿祚,字季延。崇禎四年進士。為吉安永豐知縣。鄰境九蓮山,界閩、粵,賊窟其中,綿祚請會剿。賊怒,率眾攻。綿祚出擊,三戰三捷。賊益大至,綿祚伏兵黃牛峒,大破之。積勞得疾,請告歸卒。兄弟三人並死王事。

王聚奎既失永州,後伺賊退,潛還武昌,為代者何騰蛟所劾,夤緣免。

王孫蘭,字畹仲,無錫人。崇禎四年進士。累遷成都知府。蜀宗人虐民,民相聚,將焚內江王第。孫蘭撫諭之,乃解。父擾,服闋,起官紹興,修荒政。遷廣東副使,分巡南雄、韶州二府。連州瑤賊為亂,馳剿,三戰皆捷。十六年,張獻忠大亂湖南,湖南之郴州宜章與韶接壤。孫蘭乞援督府,不應,最後以七百人至,一宿復調去。及賊陷衡州,肆屠戮。韶所轄樂昌、乳源、仁化,逋竄一空。連州守將先據城叛,韶士民聞之,空城逃,而賊所設偽官傳檄將至。孫蘭仰天歎曰:「失封疆當死,賊陷城又當死,吾盍先死乎!」遂自縊。既死,賊竟不至,朝廷憫其忠,予贈恤。

程良籌,字持卿,孝感人,工部尚書註子也。天啟五年進士。時註為太常少卿,不附魏忠賢。御史王士英劾其為趙南星、李三才私黨,忠賢遂矯旨並良籌除名,永不敘錄。未出仕而除名,前此未有也。崇禎元年起官,歷文選員外郎,掌選事。麻城李長庚為尚書,以同鄉故,甚倚之。正郎久缺不推補,同列多忌,朝論亦少之。長庚用推舉失當削籍,良籌亦下吏遣戍,久乃釋歸。

十六年,李自成犯承天,孝感亦陷。良籌以白雲山險峻,與同邑參政夏時亨築壘聚守。賊使說降,良籌毀其書。賊怒,設長圍攻之,相持四十餘日,解去。時漢陽、武昌亦為張獻忠所陷,四面皆賊,獨白雲孤處其間,賊頗患之。已,武昌為官軍所復,良籌號召遠近諸寨,掎角進兵。其冬,遂復孝感、雲夢。十二月,進薄德安,兵敗,退保白蓮寨。寨中人素通賊,為內應,良籌遂被執。說降,不屈,羈之密室。明年正月,左良玉遣將攻德安。賊懼,擁良籌令止外兵,不從。賊棄城去,逼良籌偕行,又不從,逐被殺。贈太常少卿。程道壽者,良籌里人也,嘗為來安知縣。賊陷孝感,置掌旅守之。道壽結里中壯士,擊殺掌旅。賊復至,杖之,繫獄,令為書招良籌。道壽曰:「我不能助白雲滅汝,肯助汝耶?」遂見殺。

黃世清,字澄海,滕縣人。父中色,吏部員外郎。世清登崇禎七年進士,除戶部主事,榷滸墅關,有清操。歷員外郎,屢遷右參議,分守商、雒,駐商州。城屢遭兵,四野蕭然,民皆入保城中。而客兵所過淫掠,民苦兵甚於賊。世清下令兵不得闌入城。未幾,關中兵經其地,有二卒撾門,榜以徇。督撫發兵,誡毋犯黃參議令。李自成躪荊、襄,遠近震動。世清一子方幼,屬友人養人,誓身殉。十六年十月,自成敗孫傳庭軍,長驅入關,遣右營十萬人從南陽犯商州。世清憑城守,有奸民投賊,至城下說降,世清佯與語,發炮斃之,懸其首城上曰:「懷二心者視此!」士民皆效死,炮矢盡,繼以石,石盡,婦人掘街砌繼之。城陷,世清坐堂上,麾其僕朱化鳳去,化鳳願同死。賊牽世清下,化鳳叱曰:「奴才不得無禮!」賊批其頰,化鳳聲色愈厲。執至賊帥袁宗第營,世清植立,賊欲屈之,化鳳曰:「吾主堂堂憲司,肯拜賊耶!」賊先殺之,授世清以防禦札。罵不受,與一家十三人皆遇害。贈光祿卿。

楊暄,高平人。崇禎十三年進士。授渭南知縣。歲大凶,畢力拯救,民稍獲安。十六年冬,李自成入潼關,兵備僉事楊王休降。教授許嗣復分守上南門,城破,持梃斗,詈賊死,妻女被掠皆自殺。賊遂抵渭南。暄已擢兵部主事,未行,與訓導蔡其城同守。會舉人王命誥開門迎賊,暄被縛,索印不與,詬罵死。其城亦死之。

賊遂陷西安,咸陽知縣趙躋昌被害。屬邑望風降。蒲城知縣朱一統獨謀拒守,曰:「吾家七世衣冠,安可臣賊。」或言他州縣甲榜者皆已納款,一統曰:「此事寧論資格耶。」以體肥,令家人擴井口以待。會衙兵叛,奪印趣迎降。一統瞋目叱曰:「吾一日未死,印不可得!」日暮,左右盡散,從容赴井死。縣丞沁源姚啟崇亦死焉。一統,平定人,起家乙榜。

有朱迥滼者,沈府宗室也,由宗貢生為白水知縣。明習吏事,下不敢欺。賊潛入城,猶手弓射賊,與學官魏歲史、劉進並被難。

唐時明,字爾極,固始人。舉於鄉。崇禎中,為長垣教諭。子路墓祀田為豪家奪,時明復其故。由國子學正屢遷鳳翔知府。十六年十月聞李自成入潼關,亟治戰守備。俄潰兵大掠,西入無固志。及自成據西安,分兵來冠,典史董尚質開門迎賊,時明被執。偽相牛金星曰:「吾主求賢若渴,君至西京,不次擢用。」時明叱曰:「我天朝命吏,肯臣賊耶!」金星令尚質說降,厲聲責之。賊令縛赴西安,時明託妻子於友人,至興平,乘間自縊。鳳翔既陷,屬城叛降。隴州同知薛應玢,武進人。時攝州事,勒兵守城。城陷,詈賊死。寶雞知縣唐夢鯤,番禺舉人。歷知仙居、天台、富川、分水四縣。在富川,有撫瑤功。坐累,謫池州經歷,攝貴池縣事。左良玉擁兵下,鄉民奔入城,守者拒,夢鯤令悉納之。及改寶雞,賊已過潼關,星馳抵任。賊逼縣,知不可守,自經死。

段復興,字仲方,陽穀人。崇禎七年進士。歷右參議,分守慶陽。十六年十月,李自成據西安,傳檄諭降。復興裂其檄,集眾守。踰月,賊薄城,圍數匝,發炮石殺賊滿濠。久之,勢不支。拜辭其母,聚妻妾子女於樓,置薪其上,復乘城督戰。城陷,趨歸火其樓,母亦赴火死。乃持鐵鞭走北門,擊殺數賊,遂自刎。士民葬之西河坪,立祠祀之。同時死難者,慶陽推官靳聖居、安化知縣袁繼登。聖居,字淑孔,長垣人。崇禎元年進士,歷知濟源、萊陽二縣。屢謫復起,蒞慶陽時,已授刑部主事,未行,遇賊,佐復興死守。城破被執,罵不絕口死。繼登,南畿人。起家選貢,蒞任未浹歲即遘變,見賊求速死,賊殺之。

其陷寧州也,知州董琬死之。宗室朱新鍱者,以貢生授中部知縣。自成使人持檄招降,新鍱碎之。歎曰:「城小無兵,空令士民受禍,計惟自靖耳。」令妻妾子女盡縊,乃投繯死。

簡仁瑞,字季麟,榮縣人。由舉人歷安西官同知,遷平涼知府。十六年冬,賊入關,諸王及監司以下官謀遁走。仁瑞謁韓王曰:「長安有重兵,訛言不足信。殿下輕棄三百年宗社,欲何之?縱賊壓境,延、寧、甘、涼諸軍足相援,必不能支,同死社稷,亦不辱二祖列宗。」王不從。是夕,其護衛卒噪,挾王及諸郡王、宗室斬關出奔,脅仁瑞行。仁瑞曰:「吾平涼守也,吾去,誰與守?」眾遂去。仁瑞乃撤四關居民入城,以土石塞門為死守計。未幾,賊檄至,乃召所活死囚數輩,謂之曰:「吾昔嘗生汝,汝亦有以報我乎?」皆對曰:「唯命。」即托以幼子,令衛出。明日,賊抵城下,士民數人草降書,乞僉名署印。仁瑞怒叱責之,正衣冠,自經堂上。平涼既陷,屬城悉降。華亭教諭鄒姓者,援曾子居武城義,欲避去。訓導何相劉止之曰:「吾輩委質為臣,安可以賓師自待?」乃率諸生共守,乃城陷,與教諭皆殉難。

司五教,字敬先,內黃人。篤學有志行。崇禎時,以歲貢為內丘訓導。十一年,邑被兵,佐長吏拒守有功。遷城固知縣,剿山寇滅之。十六年冬,賊據關中,郡縣風靡,五教激士民固守。有諸生謀內應,捕斬之,竿其首城上。無何,偽帥田見秀擁兵至,五教且戰且守。賊悉兵攻四日而城陷,既見執,厲聲罵賊。賊去其冠帶,輒自取冠之,罵益厲,乃被磔。

鄉官張鳳翮,字健沖。天啟五年進士。崇禎中官御史,極論四川巡撫王維章貪劣,而請召還給事中章正宸,不納。出按雲南,還朝,言:「陛下議均輸再徵一年,民力已竭,討賊諸臣泄泄沓沓,徒糜數百萬金錢。」帝納其言,敕兵部飛騎勒熊文燦進兵,而張獻忠已叛矣。十五年遷浙江右參政,未任而罷。賊陷城,脅之仕,不屈死。

都任,字弘若,祥符人。萬歷四十一年進士。授南京兵部主事,進郎中,屢遷四川右參政。天啟五年大計,左遷江西僉事,復屢遷陜西左布政使。崇禎五年又謫山東右參政。再遷山西按察使。任性剛嚴,多忤物,數謫徙,終不變。月朔,同僚朝晉王,任據《會典》爭,不赴。巡按御史張孫振誣劾提學僉事袁繼咸,任數慰問繼咸,贐其行。孫振怒,復中以大計,貶秩歸。後復起,歷右布政使兼副使,飭榆林兵備。

十六年九月,巡撫崔源之罷去,代者張鳳翼未至,總兵官王定從孫傳庭出關,大敗奔還,遠近震恐。李自成遂據西安,遣其將李過以精卒數萬徇三邊,延安、綏德相繼陷。定懼,詭言討河套寇,率所部遁去,榆林益空虛。任急集軍民,慷慨流涕,諭以大義,與督餉員外郎王家錄、副將惠顯等議城守。城中多廢將,任以尤世威知兵,推為主帥,率諸將王世欽等數十人誓死守。賊遣使招降,任斬以徇。賊大眾麕至,十一月望,城被圍,至二十七日,城陷,任猶巷戰,力不支,被執。欲降之,大罵不屈,遂見殺。世威等皆死,詳見世威傳中。

家錄,黃岡人,舉於鄉。時已擢關南兵備僉事,未行,與任協守。圍急,男子皆乘城,家錄令婦人運水灌城,水厚數寸,賊不能攻。攻城陷,家錄自剄死。

一時同死者,里居戶部主事張雲鶚,知州彭卿、柳芳,湖廣監紀趙彬,皆不屈死。指揮崔重觀自焚死,傅佑與妻杜氏自縊死。中軍劉光祐罵賊死。材官李耀,善射,矢盡,自刎死。同營李光裕趣家人死,亦自刎死;張天敘焚其積貯,自縊死。指揮黃廷政與弟千戶廷用、百戶廷弼奮力殺賊,同死。千戶賀世魁偕妻柳氏自縊死。參將馬鳴節聚妻子室中,自焚死。里居戰死則山海副總兵楊明、定邊副總兵張發、孤山副總兵王永祚、西安參將李應孝。在官死事則遊擊傅德、潘國臣、李國奇、晏維新、陳二典、劉芳馨、文侯國,都司郭遇吉,中軍楊正韡、柳永年、馬應舉,旗鼓文經國,守備尤勉、惠漸、賀大雷、楊以偉,指揮李文焜、文燦。而副將常懷、李登龍,遊擊孫貴、尤養鯤,守備白慎衡、李宗敘,亦以守鄉土遭難。諸生則陳義昌、沈浚、沈演、白拱極、白含章罵賊死,張連元、連捷、李可柱、胡一奎、李蔭祥自經死。一城之中,婦女死義者數千人,井中屍滿,賊遂屠其城。

榆林為天下雄鎮,兵最精,將材最多,然其地最瘠,餉又最乏,士常不宿飽。乃慕義殉忠,志不少挫,無一屈身賊庭,其忠烈又為天下最。事聞,天子嗟悼,將大行褒恤,國亡不果。

祝萬齡,咸寧人。父世喬。有至行,以父遠遊不歸,年十五即獨身訪求,瀕死,歷數千里,卒得之。後由選貢通判南康,以清慎著。萬齡師鄉人馮從吾,舉萬歷四十四年進士。累官保定知府。天啟六年,魏忠賢盡毀天下書院,萬齡憤。逆黨李魯生遂劾萬齡倡訛言,謂天變、地震、物怪、人妖,悉由毀書院所致,非聖誣天實甚。萬齡遂落職。崇禎初,用薦起黃州知府,集諸生定惠書院,迪以正學。居三年,遷河南副使,監軍磁州。輝縣之北與山西陵川之南,有村曰水峪,回賊竊據數十年,大為民患。萬齡與山西監司王肇生合兵擊,六戰焚其巢三百餘,賊遂平。錄功,加右參政。流賊自山西入河北,掠新鄉。萬齡邀擊之,賊走陵川。已,復大至,坐失事,削籍歸。湯開遠訟其冤,不納。久之,廷臣交薦,未及用,而西安陷。萬齡深衣大帶,趣至關中書院,哭拜先聖,投繯死。僉事涇陽王徵、太常寺卿耀州宋師襄、懷慶通判咸寧竇光儀、儀封知縣長安徐方敬、芮城知縣咸寧徐芳聲、舉人宗室朱誼巉及席增光皆里居,城破,並抗節死。

陳瑸,漳浦人。天啟五年進士。授慈谿知縣。崇禎十年為袁州推官,拒楚賊有功。屢遷右參議,分守湖南,討平八排賊。十六年,張獻忠陷長沙,圍參政周鳳岐於澧州。瑸督兵往救,軍敗,被執。欲降之,不屈,斷手割肝而死。鳳岐,永康人。萬曆末年進士。歷工部郎中,掌節慎庫,忤奄人,落職歸。崇禎初,起故宮,進四川副使。苗人爭界,為立碑畫疆以定之。改右參政,分守澧州。賊來犯,援軍敗沒,城遂陷。賊帥親解其縛,說以降,怒罵而死。

王徵俊,字夢卜,陽城人。天啟五年進士。授韓城知縣。崇禎初,流賊來犯,禦卻之。坐大計,謫歸德照磨。巡按御史李日宣薦於朝,給事中呂黃鐘請用天下必不可少之人,亦及徵俊,乃量移滕縣知縣。累官右參政,分守寧前,以憂歸。十七年二月,賊陷陽城,被執不屈,繫之獄。士民爭頌其德,賊乃釋之。抵家北面再拜,投繯卒。

其時士大夫居家盡節者,靈石宋之俊、翼城史可觀、陽曲朱慎鏤。之俊舉進士,歷官登萊監軍副使,忤巡按謝三賓,互訐於朝,落職歸。三賓亦貶秩。及遇變,之俊受刑死。妻喬詈賊撞階死。女斂屍畢,拔簪刺喉死。可觀,太常少卿學遷子。官中書舍人,加鴻臚少卿。城陷,自縊死。慎鏤,晉府宗室,攝靈丘郡王府事。賊陷太原,冠帶祀家廟,驅家人入廟中,焚之,己亦投火死。

丁泰運,字孟尚,澤州人。崇禎十三年進士。除武陟知縣,調河內,著廉直聲。十七年二月,賊將劉方亮自蒲阪渡河。巡按御史蘇京托言塞太行道,先遁去,與陜西巡撫李化熙同抵寧郭驛。俄兵變,化熙被傷走。兵執京,披以婦人服,令插花行,稍違,輒抶之以為笑樂。叛將陳永福引賊至,京即迎降。賊遂逼懷慶,監司以下皆竄。泰運獨守南城,力不支,被執。賊擁見方亮,使跪不屈,燒鐵鎖炙之,亦不從,乃遇害。

賊既陷懷慶,尋陷彰德。安陽人尚大倫,字崇雅。由進士歷官刑部郎中。有國學生白夢謙以救黃道周系獄,大倫議寬之,忤尚書意,遂罷歸。城陷,抗節死。參將榆林王榮乃其子師易,皆死之。又有王橓徵,由鄉舉歷官蒲州知州,忤豪宗,謝事歸。為賊所執,傳詣李自成,道憤恨不食死。


\end{pinyinscope}