\article{列傳第六十四}

\begin{pinyinscope}
○李賢呂原子常岳正彭時商輅劉定之

李賢,字原德,鄧人。舉鄉試第一,宣德八年成進士。奉命察蝗災於河津,授驗封主事,少師楊士奇欲一見,賢竟不往。

正統初,言:「塞外降人居京師者盈萬,指揮使月俸三十五石,實支僅一石,降人反實支十七石五斗,是一降人當京官十七員半矣。宜漸出之外,省冗費,且消患未萌。」帝不能用。時詔文武臣誥敕,非九年不給。賢言:「限以九年,或官不能滿秩,或親老不待,不得者十八九,無以勸臣下。請仍三年便。」從之。遷考功郎中,改文選。扈從北征,師覆脫還。

景泰二年二月上正本十策,曰勤聖學,顧箴警,戒嗜欲,絕玩好,慎舉措,崇節儉,畏天變,勉貴近,振士風,結民心。帝善之,命翰林寫置左右,備省覽。尋又陳車戰火器之利,帝頗採納。是冬,擢兵部右侍郎,轉戶部。也先數貢馬,賢謂輦金帛以強寇自弊,非策。因陳邊備廢馳狀,於謙請下其章厲諸將。轉吏部,採古二十二君行事可法者。曰《鑒古錄》,上之。

英宗復位,命兼翰林學士,入直文淵閣,與徐有貞同預機務。未幾,進尚書。賢氣度端凝,奏對皆中機宜,帝深眷之。山東饑,發帑振不足,召有貞及賢議,有貞謂頒振多中飽。賢曰:「慮中飽而不貸,坐視民死,是因噎廢食也。。」遂命增銀。

石亨、曹吉祥與有貞爭權,並忌賢。諸御史論亨、吉祥,亨、吉祥疑出有貞、賢意,訴之帝,下二人獄。會有風雷變,得釋,謫賢福建參政。未行,王翱奏賢可大用,遂留為吏部左侍郎。踰月,復尚書,直內閣如故。亨知帝向賢,怒,然無可如何,乃佯與交歡。賢亦深自匿,非宣召不入,而帝益親賢,顧問無虛日。

孛來近塞獵。亨言傳國璽在彼,可掩而取,帝色動。賢言釁不可啟,璽不足寶,事遂寢。亨益惡賢。時帝亦厭亨、吉祥驕橫,屏人語賢曰:「此輩干政,四方奏事者先至其門,為之奈何?」賢曰:「陛下惟獨斷,則趨附自息。」帝曰:「向嘗不用其言,乃怫然見辭色。」賢曰:「願制之以漸。」當亨、吉祥用事,賢顧忌不敢盡言,然每從容論對,所以裁抑之者甚至。及亨得罪,帝復問賢「奪門」事。賢曰:「『迎駕』則可,『奪門』豈可示後?天位乃陛下固有,『奪』即非順。且爾時幸而成功,萬一事機先露,亨等不足惜,不審置陛下何地!」帝悟曰:「然。」賢曰:「若郕王果不起,群臣表請陛下復位,安用擾攘為?此輩又安所得邀升賞,招權納賄安自起?老成耆舊依然在職,何至有殺戮降黜之事致干天象?《易》曰『開國承家,小人勿用』,正謂此也。」帝曰:「然。」詔自今章奏勿用「奪門」字,併議革冒功者四千餘人。至成化初,諸被革者愬請。復以賢言,并奪太平侯張瑾、興濟伯楊宗爵,時論益大快之。

帝既任賢,所言皆見聽。於謙嘗分遣降人南征,陳汝言希宦官指,盡召之還。賢力言不可。帝曰:「吾亦悔之。今已就道,後當聽其願去者。」帝憂軍官支俸多,歲入不給。賢請汰老弱於外,則費省而人不覺。帝深納焉。時歲有邊警,天下大水,江南北尤甚。賢外籌邊計,內請寬百姓,罷一切徵求。帝用其言,四方得蘇息。七年二月,空中有聲,帝欲禳之,命賢撰青詞。賢言君不恤民,天下怨叛,厥有鼓妖。因請行寬恤之政,又請罷江南織造,清錦衣獄,止邊臣貢獻,停內外採買。帝難之。賢執爭數四,同列皆懼。賢退曰:「大臣當知無不言,可卷舌偷位耶?」終天順之世,賢為首輔,呂原、彭時佐之,然賢委任最專。

初,御史劉濬劾柳溥敗軍罪,觸帝怒。賢言御史耳目官,不宜譴。石亨譖賢曲護。帝浸疏賢,尋悟,待之如初。每獨對,良久方出。遇事必召問可否,或遣中官就問。賢務持大體,尤以惜人才、開言路為急。所薦引年富、軒輗、耿九疇、王竑、李秉、程信、姚夔、崔恭、李紹等,皆為名臣。時勸帝延見大臣,有所薦,必先與吏、兵二部論定之。及入對,帝訪文臣,請問王翱;武臣,請問馬昂。兩人相左右,故言無不行,而人不病其專,惟群小與為難。

曹欽之反也,擊賢東朝房,執將殺之,逼草奏釋己罪。賴王翱救,乃免。賢密疏請擒賊黨。時方擾攘,不知賢所在。得疏,帝大喜。裹傷入見,慰勞之,特加太子太保。賢因言,賊既誅,急宜詔天下停不急務,而求直言以通閉塞。帝從之。

門達方用事,錦衣官校恣橫為劇患。賢累請禁止,帝召達誡諭之。達怙寵益驕,賢乘間復具陳達罪,帝復召戒達。達銜次骨,因袁彬獄陷賢,賢幾不免,語載達傳。

帝不豫,臥文華殿。會有間東宮於帝者,帝頗惑之,密告賢。賢頓首伏地曰:「此大事,願陛下三思。」帝曰:「然則必傳位太子乎?」賢又頓首曰:「宗社幸甚。」帝起,立召太子至。賢扶太子令謝。太子謝,抱帝足泣,帝亦泣,讒竟不行。

憲宗即位,進少保、華蓋殿大學士,知經筵事。是年春,日黯無光,賢偕同官上言:「日,君象。君德明,則日光盛。惟陛下敬以修身,正以御下,剛以斷事,明以察微,持之不怠,則天變自弭,和氣自至。」翌日又言:「天時未和,由陰氣太盛。自宣德至天順間,選宮人太多,浣衣局沒官婦女愁怨尤甚,宜放還其家。」帝從之,中外欣悅。五月大雨雹,大風飄瓦,拔郊壇樹。賢言:「天威可畏,陛下當凜然加省,無狎左右近幸。崇信老成,共圖國是。」有司請造鹵簿。賢言:「內庫尚有未經御者,今恩詔甫頒,方節財用,奈何復為此。」帝即日寢之。每遇災變,必與同官極陳無隱,而於帝初政,申誡尤切。

門達既竄,其黨多投匿名書構賢。賢乞罷,有詔慰留。吳后廢,言官請誅牛玉,語侵賢,又有造蜚語搆賢者。帝命衛士宿賢家,護出入。成化二年三月遭父喪,詔起復。三辭不許,遣中官護行營葬。還至京,又辭。遣使宣意,遂視事。其年冬卒,年五十九。帝震悼,贈太師,謚文達。

賢自以受知人主,所言無不盡。景帝崩,將以汪后殉葬,用賢言而止。惠帝少子幽禁已六十年,英宗憐欲赦之,以問賢。賢頓首曰:「此堯、舜用心也!天地祖宗實式憑之。」帝意乃決。帝嘗祭山川壇,以夜出未便,欲遣官代祀。賢引祖訓爭之,卒成禮而還。嘗言內帑餘財,不以恤荒濟軍,則人主必生侈心,而移之於土木禱祠聲色之用。前後頻請發帑振貸恤邊,不可勝計。故事,方面官敕三品京官保舉。賢患其營競,令吏部每缺舉二人,請帝簡用。並推之例始此。

自三楊以來,得君無如賢者。然自郎署結知景帝,超擢侍郎,而所著書顧謂景帝為荒淫。其抑葉盛,擠岳正,不救羅倫,尤為世所惜云。

呂原,字逢原,秀水人。父嗣芳,萬泉教諭。兄本,景州訓導。嗣芳老,就養景州,與本相繼卒。貧不能歸葬,厝於景,原時至墓慟哭。久之,奉母南歸,家益貧。知府黃懋奇原文,補諸生,遣入學,舉鄉試第一。

正統七年,進士及第,授編修。十二年,與侍講裴綸等十人同選入東閣肄業,直經筵。景泰初,進侍講,與同官倪謙授小內侍書於文華殿東廡。帝至,命謙講《國風》。原講《堯典》,皆稱旨。問何官,並以中允兼侍講對。帝曰:「品同耳,何相兼為?」進二人侍講學士,兼中允。尋進左春坊大學士。

天順初,改通政司右參議,兼侍講。徐有貞、李賢下獄之明日,命入內閣預機務。石享、曹吉祥用事,貴倨,獨敬原。原朝會衣青袍,亨笑曰:「行為先生易之。」原不答。尋與岳正列亨、吉祥罪狀,疏留中。二人怒,摘敕諭中語,謂閣臣謗訕。帝大怒,坐便殿,召對,厲聲曰:「正大膽敢爾!原素恭謹,阿正何也?」正罷去,原得留。李賢既復官入閣柄政,原佐之。未幾,彭時亦入,三人相得甚歡。賢通達,遇事立斷。原濟以持重,庶政稱理。其年冬,進翰林院學士。

六年,遭母喪,水漿不入口三日。詔葬畢即起視事。原乞終制。不允。乃之景州,啟父兄殯歸葬,舟中寢苫哀毀。體素豐,至是羸瘠。抵家甫襄事而卒,年四十五。贈禮部左侍郎,謚文懿。

原內剛外和,與物無競。性儉約,身無紈綺。歸裝惟賜衣數襲,分祿恤宗姻。

子常,字秉之。以蔭補國子生,供事翰林,遷中書舍人。疏乞應試,所司執故事不許。憲宗特許之,遂舉順天鄉試。舍人得赴試自常始。累遷禮部郎中,好學能文,諳掌故。琉球請歲一入貢,回回貢使乞道廣東歸國,皆以非制格之。以薦進南京太僕寺少卿。故事,太僕馬數,不令他官知。以是文籍磨滅,登耗無稽。常曰:「他官不與聞,是也;當職者,可貿貿耶?」議請三年一校勘,著為例。累遷南京太常卿,輯《典故因革》若干卷。正德初,致仕歸。

岳正,字季方,漷縣人。正統十三年會試第一,賜進士及第,授編修,進左贊善。

天順初,改修撰,教小內侍書。閣臣徐有貞、李賢下獄,帝既用呂原預政,頃之,薛瑄又致仕,帝謀代者。王翱以正薦,遂召見文華殿。正長身美須髯,帝遙見,色喜。既登陛,連稱善。問年幾何,家安在,何年進士,正具以對。復大喜曰:「爾年正強仕,吾北人,又吾所取士,今用爾內閣,其盡力輔朕。」正頓首受命。趨出,石亨、張軏遇之左順門,愕然曰:「何自至此?」比入,帝曰:「朕今日自擇一閣臣。」問為誰,帝曰:「岳正。」兩人陽賀。帝曰:「但官小耳,當與吏部左侍郎兼學士。」兩人曰:「陛下既得人,俟稱職,加秩未晚。」帝默然,遂命以原官入閣。

正素豪邁,負氣敢言。及為帝所拔擢,益感激思自效。掌欽天監侍郎湯序者,亨黨也,嘗奏災異,請盡去奸臣。帝問正,正言:「奸臣無指名。即求之,人人自危。且序術淺,何足信也。」乃止。有僧為妖言,錦衣校邏得之,坐以謀反。中官牛玉請官邏者,正言:「事縱得實,不過坐妖言律,邏者給賞而已,不宜與官。」僧黨數十人皆得免。或為匿名書列曹吉祥罪狀,吉祥怒,請出榜購之。帝使正撰榜格,正與呂原入見曰:「為政有體,盜賊責兵部,姦宄責法司,豈有天子出榜購募者?且事緩之則自露,急之則愈匿,此人情也。」帝是其言,不問。亨從子彪鎮大同,獻捷,下內閣問狀。使者言捕斬無算,不能悉致,皆梟置林木間。正按地圖指詰之,曰:「某地至某地,皆沙漠,汝梟置何所?」其人語塞。

時亨、吉祥恣甚,帝頗厭之。正從容言:「二人權太重,臣請以計間之。」帝許焉。正出見吉祥曰:「忠國公常令杜清來此何為者?」吉祥曰:「辱石公愛,致誠款耳。」正曰:「不然,彼使伺公所為耳。」因勸吉祥辭兵柄。復詣亨,諭令自戢。亨、吉祥揣知正意,怒。吉祥見帝,免冠,泣請死。帝內愧,慰諭之,召正責漏言。

會承天門災,正極言亨將為不軌,且言:「陳汝言,小人。今既為尚書,可用盧彬為侍郎。二人者俱譎悍,若同事必相齮齕,乘其隙可並去之。」徐有貞再下獄,復云:「用有貞則天變可弭。」帝皆不納。及敕諭廷臣,命正視草。正草敕曰:「乃者承天門災,朕心震驚,罔知所措。意敬天事神,有未盡歟?祖宗成憲有不遵歟?善惡不分,用舍乖歟?曲直不辨,刑獄冤歟?徵調多方,軍旅勞歟?賞賚無度,府庫虛歟?請謁不息,官爵濫歟?賄賂公行,政事廢歟?朋奸欺罔,附權勢歟?群吏弄法,擅威福歟?征斂徭役太重,而閭閻靡寧歟?讒諂奔競之徒倖進,而忠言正士不用歟?抑有司闒茸酷暴,貪冒無厭,而致軍民不得其所歟?此皆傷和致災之由,而朕有所未明也。今朕省愆思咎,怵惕是存。爾群臣休戚惟均,其洗心改過,無蹈前非,當行者直言無隱。」敕下,舉朝傳誦。而亨、吉祥構蜚語,謂正賣直謗訕。帝怒,命仍授內侍書。明日,謫欽州同知。道漷,以母老留旬日。陳汝言令巡校言狀,且言正嘗奪公主田。遂逮系詔獄,杖百,戍肅州。行至涿,夜宿傳舍。手拲急,氣奔且死。涿人楊四醉卒酒,脫正拲,刳其中,且厚賂卒,乃得至戍所。亨、吉祥既誅,帝謂李賢曰:「岳正固嘗言之。」賢曰:「正有老母,得放歸田里,幸甚。」乃釋為民。

憲宗立,御史呂洪等請復正與楊瑄官,詔正以原官直經筵,纂修《英宗實錄》。初,正得罪,都督僉事季鐸乞得其宅,至是敕還正。正還朝,自謂當大用,而賢欲用為南京祭酒,正不悅。忌者偽為正劾賢疏草,賢嗛之。

成化元年四月,廷推兵部侍郎清理貼黃,以正與給事中張寧名並上。詔以為私,出正為興化知府,而寧亦補外。正至官,築堤溉田數千頃,節縮浮費,經理預備倉,欲有所興革。鄉士大夫不利其所為,騰謗言。正亦厭吏職,五年入覲,遂致仕。又五年卒,年五十五。無子,大學士李東陽、御史李經,其婿也。

正博學能文章,高自期許,氣屹屹不能下人。在內閣才二十八日,勇事敢言,便殿論奏,至唾濺帝衣。有規以信而後諫者,慨然曰:「上顧我厚,懼無以報稱,子乃以諫官處我耶?」英宗亦悉其忠,其在戍所,嘗念之曰:「岳正倒好,只是大膽。」正聞自為像贊,述帝前語,末言:「臣嘗聞古人之言,蓋將之死而靡憾也。」其自信不回如此。然意廣才疏,欲以縱橫之術離散權黨,反為所噬,人皆迂而惜之。嘉靖中,追贈太常寺卿,謚文肅。

彭時,字純道,安福人。正統十三年進士第一,授修撰。明年,郕王監國,令同商輅入閣預機務。聞繼母憂,力辭,不允,乃拜命。釋褐逾年參大政,前此未有也。尋進侍讀。

景泰元年,以兵事稍息,得請終制。然由此忤旨。服除,命供事翰林院,不復與閣事。易儲,遷左春坊大學士。《寰宇通志》成,遷太常寺少卿。俱兼侍讀。

天順元年,徐有貞既得罪,岳正、許彬相繼罷。帝坐文華殿召見時,曰:「汝非朕所擢狀元乎?」時頓首。明日仍命入閣,兼翰林院學士。閣臣自三楊後,進退禮甚輕。為帝所親擢者,唯時與正二人。而帝方向用李賢,數召賢獨對。賢雅重時,退必咨之。時引義爭可否,或至失色。賢初小忤,久亦服其諒直,曰:「彭公,真君子也。」慈壽皇太后上尊號,詔告天下。時欲推恩,賢謂一年不宜再赦。時曰:「非赦也,宜行優老典。朝臣父母七十與誥敕,百姓八十給冠帶,是『老吾老以及人之老』也。」賢稱善,即奏行之。

帝愛時風度,選庶吉士。命賢盡用北人,南人必若時者方可。賢以語時。俄中官牛玉宣旨,時謂玉曰:「南士出時上者不少,何可抑之?」已,選十五人,南六人與焉。

門達構賢,帝惑之,曰:「去賢,行專用時矣。」或傳其語,時矍然曰:「李公有經濟才,何可去?」因力直之。且曰:「賢去,時不得獨留。」語聞,帝意乃解。

帝大漸,口占遺命,定后妃名分,勿以嬪御殉葬,凡四事,付閣臣潤色。時讀竟,涕下,悲愴不自勝。中官復命,帝亦為隕涕。

憲宗即位,議上兩宮尊號。中官夏時希周貴妃旨,言錢后久病,不當稱太后。而貴妃,帝所生母,宜獨上尊號。賢曰:「遺詔已定,何事多言。」時曰:「李公言是也。朝廷所以服天下,在正綱常。若不爾,損聖德非小。」頃之,中官復傳貴妃旨:「子為皇帝,母當為太后,豈有無子而稱太后者?宣德間有故事。」賢色變,目時。時曰:「今日事與宣德間不同。胡后表讓位,退居別宮,故在正統初不加尊。今名分固在,安得為比?」中官曰:「如是何不草讓表?」時曰:「先帝存日未嘗行,今誰敢草?若人臣阿意順從,是萬世罪人也。」中官厲聲怵以危語。時拱手向天曰:「太祖、太宗神靈在上,孰敢有二心!錢皇后無子,何所規利而為之爭?臣義不忍默者,欲全主上聖德耳。若推大孝之心,則兩宮並尊為宜。」賢亦極言之,議遂定。及將上寶冊,時曰:「兩宮同稱則無別,錢太后宜加兩字,以便稱謂。」乃尊為慈懿皇太后,貴妃為皇太后。越數日,中官覃包至內閣曰:「上意固如是。但迫於太后,不敢自主,非二公力急,幾誤大事。」時閣臣陳文默無語,聞包言,甚愧。禮成,進吏部右侍郎,兼學士,同知經筵。

成化改元,進兵部尚書,兼官如故。明年秋,乞歸省。三年二月詔趣還朝,《英宗實錄》成,加太子少保。兼文淵閣大學士。

四年,慈懿太后崩,詔議山陵。時及商輅、劉定之言:「太后作配先帝,正位中宮,陛下尊為太后,詔示天下。先帝全夫婦之倫,陛下盡母子之愛,於義俱得。今梓宮當合葬裕陵,主當祔廟,此不易之禮。比聞欲別卜葬地,臣等實懷疑懼。竊謂皇上所以遲疑者,必以今皇太后萬壽後,當與先帝同尊,自嫌二后並配,非祖宗制。考之於古,漢文帝尊所生母薄太后,而呂后仍祔長陵。宋仁宗追尊生母李宸妃,而劉後仍祔太廟。今若陵廟之制稍有未合,則有乖前美,貽譏來葉。」於是諸大臣相繼言之。帝猶重違太后意,時偕朝臣伏文華門泣請。帝與太后皆感動,始從時議。

彗見三臺,時等言:「外廷大政固所當先,宮中根本尤為至急。諺云『子出多母』。今嬪嬙眾多,維熊無兆。必陛下愛有所專,而專寵者已過生育之期故也。望均恩愛,為宗社大計。」時帝專寵萬貴妃,妃年已近四十,時故云然。又言:「大臣黜陟,宜斷自宸衷,或集群臣僉議。不可悉委臣下,使大權旁落。」帝雖不能從,而心嘉其忠。

都御史項忠討滿四不利。朝議命撫寧侯朱永將京軍往赴。永故難其行,多所邀請。時惡其張大,且度軍可無行,第令整裝待。會忠馳奏,已圍賊石城。帝遣中官懷恩、黃賜偕兵部尚書白圭、程信等至閣議。時曰:「賊四出攻剽,鋒誠不可當。今入石城自保,我軍圍甚固,此困獸易擒耳。」信曰:「安知忠不退師?」時曰:「彼部分已定,何故自退?且今出師,度何時到?」信曰:「來春。」時曰:「如此,益緩不及事。事成敗,冬月決矣。」信忿,出危言曰:「忠若敗,必斬一二人,然後出師。」眾危之,問時何見。曰:「觀忠疏曲折,知其能。若聞別遣禁軍,將退避不敢任,賊不可知矣。」時惟商輅然其言。至冬,賊果平,人乃大服。改吏部尚書。

五年得疾在告。逾三月,帝趣赴閣視事,免朝參。是冬,無雪。疏言:「光祿寺採辦,各城門抽分,掊克不堪。而獻珍珠寶石者,倍估增直,漁竭帑藏。乞革其弊,以惠小民。」帝優詔褒納。畿輔、山東、河南旱,請免夏稅鹽鈔,及太僕寺賠課馬。京師米貴,請發倉儲五十萬石平糶。並從之。時以舊臣見倚重,遇事爭執無所避。而是時帝怠於政,大臣希得見。萬安同在閣,結中貴戚畹,上下壅隔,時頗懷憂。

七年,疾復作,乞致仕。帝慰留之,不得去。冬,彗復見,時言政本七事:一,毋惑佛事,糜金錢;二,傳旨專委司禮監,毋令他人,以防詐偽;三,延見大臣議政事;四,近幸賜予太多,工匠冒官無紀,而重囚死徙者,法不蔽罪。宜戒淫刑僭賞;五,虛懷受諫,勿惡切直;六,戒廷臣毋依違,凡政令失當,直言論奏;七,清理牧馬草地,減退勢要莊田。皆切中時弊。

寧晉伯劉聚為從父太監永誠請封謚,且乞祠額,禮部執故事卻之。帝特賜額曰「褒功」,命內閣擬封謚。時等言:「即予永誠,將來守邊內臣皆援此陳乞,是變祖宗法自今日始。」或言宋童貫封王,時曰:「貫封王在徽宗末年,豈盛世事耶?」乃寢。

時每因災變上言,或留中,或下所司,多阻隔,悒悒不得志。五年以後,凡七在告,帝輒命醫就視,數遣內臣賜賚。十一年正月,以秩滿進少保。踰月卒,年六十。贈太師,謚文憲。

時立朝三十年,孜孜奉國,持正存大體,公退未嘗以政語子弟。有所論薦,不使其人知。燕居無惰容,服御儉約,無聲樂之奉,非其義不取,有古大臣風。

商輅,字弘載,淳安人。舉鄉試第一。正統十年,會試、殿試皆第一。終明之世,三試第一者,輅一人而已。除修撰,尋與劉儼等十人進學東閣。輅豐姿瑰偉,帝親簡為展書官。郕王監國,以陳循、高穀薦入內閣,參機務。徐珵倡南遷議,輅力沮之。其冬,進侍讀。景泰元年遣迎上皇於居庸,進學士。

三年,錦衣指揮盧忠令校尉上變,告上皇與少監阮浪、內使王瑤圖復位。帝震怒,捕二人下詔獄,窮治之。忠筮於術者同寅,寅以大義折之,且曰:「此大兇兆,死不足贖。」忠懼,佯狂以冀免。輅及中官王誠言於帝曰:「忠病風,無足信,不宜聽妄言,傷大倫。」帝意少解。乃並下忠獄,坐以他罪,降為事官立功。殺瑤,錮浪於獄,事得不竟。

太子既易,進兵部左侍郎,兼左春坊大學士如故,賜第南薰里。塞上腴田率為勢豪侵據,輅請核還之軍。開封、鳳陽諸府饑民流濟寧、臨清間,為有司驅逐。輅憂其為變,請招墾畿內八府閒田,給糧種,民皆有所歸。鐘同、章綸下獄,輅力救得無死。《寰宇通志》成,加兼太常卿。

景帝不豫,群臣請建東宮,不許。將繼奏,輅援筆曰:「陛下宣宗章皇帝之子,當立章皇帝子孫。」聞者感動。以日暮,奏未入,而是夜石亨輩已迎復上皇。明日,王文、于謙等被收,召輅與高穀入便殿,溫旨諭之,命草復位詔。亨密語輅,赦文毋別具條款。輅曰:「舊制也,不敢易。」亨輩不悅,諷言官劾輅朋奸,下之獄。輅上書自愬《復儲疏》在禮部,可覆驗,不省。中官興安稍解之,帝愈怒。安曰:「向者此輩創議南遷,不審置陛下何地。」帝意漸釋,乃斥為民。然帝每獨念「輅,朕所取士,嘗與姚夔侍東宮」,不忍棄之。以忌者,竟不復用。

成化三年二月召至京,命以故官入閣。輅疏辭,帝曰:「先帝已知卿枉,其勿辭。」首陳勤學、納諫、儲將、防邊、省冗官、設社倉、崇先聖號、廣造士法凡八事。帝嘉納之。其言納諫也,請召復元年以後建言被斥者。於是羅倫、孔公恂等悉復官。

明年,彗星見,給事中董旻、御史胡深等劾不職大臣,並及輅。御史林誠詆輅曾與易儲,不宜用,帝不聽。輅因求罷。帝怒,命廷鞫諸言者,欲加重譴。輅曰:「臣嘗請優容言者,今論臣反責之,如公論何?」帝悅,旻等各予杖復職。尋進兵部尚書。久之,進戶部。《宋元通鑒綱目》成,改兼文淵閣大學士。皇太子立,加太子少保,進吏部尚書。十三年進謹身殿大學士。

輅為人,平粹簡重,寬厚有容,至臨大事,決大議,毅然莫能奪。

仁壽太后莊戶與民爭田,帝欲徙民塞外。輅曰:「天子以天下為家,安用皇莊為?」事遂寢。乾清宮門災,工部請採木川、湖。輅言宜少緩,以存警畏,從之。

悼恭太子薨,帝以繼嗣為憂。紀妃生皇子,六歲矣,左右畏萬貴妃,莫敢言。久之,乃聞於帝。帝大喜,欲宣示外廷,遣中官至內閣諭意。輅請敕禮部擬上皇子名,於是廷臣相率稱賀。帝即命皇子出見廷臣。越數日,帝復御文華殿,皇子侍,召見輅及諸閣臣。輅頓首曰:「陛下踐祚十年,儲副未立,天下引領望久矣。當即立為皇太子,安中外心。」帝頷之。是冬,遂立皇子為皇太子。

初,帝召見皇子留宮中,而紀妃仍居西內。輅恐有他患,難顯言,偕同官上疏曰:「皇子聰明岐嶷,國本攸繫。重以貴妃保護,恩逾己出。但外議謂皇子母因病別居,久不得見。宜移就近所,俾母子朝夕相接,而皇子仍藉撫育於貴妃,宗社幸甚。」由是紀妃遷永壽宮。逾月,妃病篤。輅請曰:「如有不諱,禮宜從厚。」且請命司禮監奉皇子,過妃宮問視,及制衰服行禮。帝皆是之。

帝將復郕王位號,下廷議。輅極言王有社稷功,位號當復,帝意遂決。帝建玉皇閣於宮北,命內臣執事,禮與郊祀等,輅等爭罷之。黑眚見,疏弭災八事,曰:番僧國師法王,毋濫賜印章;四方常貢外,勿受玩好;許諸臣直言;分遣部使慮囚,省冤獄;停不急營造,實三邊軍儲;守沿邊關隘;設雲南巡撫。帝優詔褒納。

中官汪直之督西廠也,數興大獄。輅率同官條直十一罪,言:「陛下委聽斷於直,直又寄耳目於群小如韋瑛輩。皆自言承密旨,得顓刑殺,擅作威福,賊虐善良。陛下若謂擿奸禁亂,法不得已,則前此數年,何以帖然無事?且曹欽之變,由逯杲刺事激成,可為懲鑒。自直用事,士大夫不安其職,商賈不安於途,庶民不安於業,若不亟去,天下安危未可知也。」帝慍曰:「用一內豎,何遽危天下,誰主此奏者?」命太監懷恩傳旨,詰責厲甚。輅正色曰:「朝臣無大小,有罪皆請旨逮問,直擅抄沒三品以上京官。大同、宣府,邊城要害,守備俄頃不可缺。直一日械數人。南京,祖宗根本地,留守大臣,直擅收捕。諸近侍在帝左右,直輒易置。直不去,天下安得無危?」萬安、劉珝、劉吉亦俱對,引義慷慨,恩等屈服。輅顧同列謝曰:「諸公皆為國如此,輅復何憂。」會九卿項忠等亦劾直,是日遂罷西廠。直雖不視廠事,寵幸如故。譖輅嘗納指揮楊曄賄,欲脫其罪。輅不自安,而御史戴縉復頌直功,請復西廠,輅遂力求去。詔加少保,賜敕馳傳歸。輅既去,士大夫益俯首事直,無敢與抗者矣。

錢溥嘗以不遷官,作《禿婦傳》以譏輅。高瑤請復景帝位號,黎淳疏駁,極詆輅。輅皆不為較,待之如平時。萬貴妃重輅名,出父像,屬為贊,遺金帛甚厚。輅力辭,使者告以妃意。輅曰:「非上命,不敢承也。」貴妃不悅,輅終不顧。其和而有執如此。

及謝政,劉吉過之,見其子孫林立,歎曰:「吉與公同事歷年,未嘗見公筆下妄殺一人,宜天之報公厚。」輅曰:「正不敢使朝廷妄殺一人耳。」居十年卒,年七十三。贈太傅,謚文毅。

子良臣,成化初進士,官翰林侍講。

劉定之,字主靜,永新人。幼有異稟。父授之書,日誦數千言。不令作文,一日偶見所為《祀灶文》。大異之。舉正統元年會試第一,殿試及第,授編修。

京城大水,應詔陳十事,言:「號令宜出大公,裁以至正,不可茍且數易。公卿侍從,當數召見,察其才能心術而進退之。降人散處京畿者,宜漸移之南方。郡縣職以京朝官補,使迭相出入,內外無畸重。薦舉之法,不當拘五品以上。可仿唐制,朝臣遷秩,舉一人自代,吏部籍其名而簡用之。武臣子孫,教以韜略。守令牧養為先務,毋徒取乾辦。群臣遭喪,乞永罷起復以教孝。僧尼蠹國當嚴絕。富民輸粟授官者,有犯宜追奪。」疏入留中。十三年,弟寅之與鄉人相訐,辭連定之。下獄,得白。秩滿,進侍講。

景帝即位,復上言十事,曰:

自古如晉懷、愍、宋徽、欽,皆因邊塞外破,籓鎮內潰,救援不集,馴致播遷。未有若今日以天下之大,數十萬之師,奉上皇於漠北,委以與寇者也。晉、宋遭禍亂,棄故土,偏安一隅,尚能奮於既衰,以御方張之敵。未有若今日也先乘勝直抵都城。以師武臣之眾,既不能奮武以破賊,又不能約和以迎駕。聽其自來而自去者也。國勢之弱,雖非旦夕所能強,豈可不思自強之術而力行之。臣愚敢略陳所見。

近日京軍之戰,但知堅壁持重,而不能用奇制勝。至前敗而後不救,左出而右不隨。謂宜仿宋吳玠、吳璘三疊陣法,互相倚恃,迭為救護。至鐵騎沖突,必資刀斧以制之。郭子儀破安祿山八萬騎,用千人執長刀如牆而進。韓世忠破兀術拐子馬,用五百人執長斧,上揕人胸,下斫馬足。是刀斧揮霍便捷,優於火槍也。

紫荊、居庸二關,名為關塞,實則坦途。今宜增兵士,繕亭障,塞蹊隧。陸則縱橫掘塹,名曰「地網」。水則水豬泉令深,名曰「水櫃」。或多植榆柳,以制奔突,或多招鄉勇,以助官軍。此皆古所嘗為,已有明效。

往者奉使之臣,充以驛人駔夫,招釁啟戎,職此之故。今宜擇內蘊忠悃,外工專對,若陸賈、富弼其人者,使備正介之選,庶不失辭辱國。

臣於上皇朝,乞徙漠北降人,知謀短淺,未蒙採納。比乘國釁,奔歸故土,寇掠畿甸者屢見告矣。宜乘大兵聚集時,遷之南方。使與中國兵民相錯雜,以牽制而變化之。且可省俸給,減漕輓,其事甚便。

天下農出粟,女出布,以養兵也。兵受粟於倉,受布於庫,以衛國也。向者兵士受粟布於公門,納月錢於私室。於是手不習擊刺之法,足不習進退之宜。第轉貨為商,執技為工,而以工商所得,補納月錢。民之膏血,兵之氣力,皆變為金銀以惠奸宄。一旦率以臨敵,如驅羊拒狼,幾何其不敗也!今宜痛革其弊,一新簡練之政,將帥踵舊習者誅毋赦。如是而兵威不振者,未之有也。

守令朘民,猶將帥之剝兵也。宜嚴糾考,慎黜陟。犯贓者舉主與其罰,然後貪墨者寡,薦舉者慎,民安而邦本固矣。

古販繒屠狗之夫,俱足助成帝業。今于謙、楊善亦非出自將門。然將能知將,宜令各舉所知,不限門閥。公卿侍從,亦令舉勇力知謀之士,以備將材。庶搜羅既廣,禦侮有人。

昔者漢圖恢復,所恃者諸葛亮。南宋御金,所恃者張浚。彼皆忠義夙著,功業久立。及街亭一敗,亮辭丞相。符離未捷,浚解都督,何則?賞罰明則將士奮也。昨德勝門下之戰,未聞摧陷強寇,但迭為勝負,互殺傷而已。雖不足罰,亦不足賞。乃石亨則自伯進侯,于謙則自二品遷一品。天下未聞其功,但見其賞,豈不怠忠臣義士之心乎?可令仍循舊秩,勿躐新階,他日勛名著而爵賞加,正未為晚。夫既與不忍奪者,姑息之政;既進不肯退者,患失之心。上不行姑息之政,下不懷患失之心,則治平可計日而望也。

向者御史建白,欲令大臣入內議政,疏寢不行。夫人主當總攬威權,親決機務。政事早朝未決者,日御便殿,使大臣敷奏。言官察其邪正而糾劾之,史官直書簡冊,以示懲勸。此前代故事,祖宗成法也,願陛下遵而行之。若僅封章入奏,中旨外傳,恐偏聽獨任,致生奸亂,欲治化之成難矣。

人主之德,欲其明如日月以察直枉,仁如天地以覆群生,勇如雷霆以收威柄。故司馬光之告君,以仁明武為言,即《中庸》所謂知仁勇也。知仁勇非學而能之哉?夫經莫要於《尚書》、《春秋》,史莫正於《通鑒綱目》。陛下留心垂覽。其於君也,既知禹、湯、文、武之所以興,又知桀、紂、幽、厲之所以替,而趨避審矣。於馭內臣也,既知有呂強、張承業之忠,又知有仇士良、陳弘志之惡;於馭廷臣也,既知有蕭、曹、房、杜之良,又知有李林甫、楊國忠之奸,而用舍當矣。如是則於知仁勇之德,豈不大有助哉。茍徒如嚮者儒臣進講,誦述其善,諱避其惡,是猶恐道路之有陷阱,閉目而過之,其不至於冥行顛僕者幾何。

今天下雖遭大創,尚如金甌之未缺。誠能本聖學以見之政治,臣見國勢可強,仇恥可雪,兄弟之恩可全,祖宗之制可復,亦何憚而不為此。

書奏,帝優詔答之。

三年遷洗馬。也先使者乞遣報使,帝堅不許。定之疏引故事以請,帝下廷議,竟不果遣。久之,遷右庶子。天順改元,調通政司左參議,仍兼侍講。尋進翰林學士。憲宗立,進太常少卿,兼侍讀學士,直經筵。

成化二年十二月,以本官入直文淵閣,進工部右侍郎,兼翰林學士。江西、湖廣災,有司方徵民賦。定之言國儲充積,倉庾至不能容。而此張口待哺之氓,乃責其租課,非聖主恤下意。帝感其言,即命停征。四年進禮部左侍郎。萬貴妃專寵,皇后希得見,儲嗣未兆。郕王女及笄未下嫁。定之因久旱,並論及之。且請經筵兼講太祖御制諸書,斥異端邪教,勿令害政耗財。帝留其疏不下。五年卒官。贈禮部尚書,謚文安。

定之謙恭質直,以文學名一時。嘗有中旨命制元宵詩,內使卻立以俟。據案伸紙,立成七言絕句百首。又嘗一日草九制,筆不停書。有質宋人名字者,就列其世次,若譜系然,人服其敏博。

贊曰:英宗之復闢也,當師旅饑饉之餘,民氣未復,權奸內訌,柱石傾移,朝野多故,時事亦孔棘矣。李賢以一身搘拄其間,沛然若有餘。獎厲人材,振飭綱紀。迨憲、孝之世,名臣相望,猶多賢所識拔。偉哉宰相才也。彭時、商輅侃侃守義,盡忠獻納,粹然一出於正。其於慈懿典禮,非所謂善成君德者歟?輅科名與宋王曾、宋庠埒,德望亦無愧焉。呂原、岳正、劉定之雖相業未優,而原之行誼,正之氣概,定之之建白,咸有可稱,故以時次,並列於篇。


\end{pinyinscope}