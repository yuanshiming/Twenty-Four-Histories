\article{武宗本紀}

\begin{pinyinscope}
武宗承天達道英肅睿哲昭德顯功弘文思孝毅皇帝,諱厚照,孝宗長子也。母孝康敬皇后。弘治五年,立為皇太子。性聰穎,好騎射。

十八年五月,孝宗崩。千寅,即皇帝位。以明年為正德元年,大赦天下,除弘治十六年以前逋賦。戊申,小王子犯宣府,總兵官張俊敗績。庚戌,太監苗逵監督軍務,保國公朱暉為征虜將軍,充總兵官,右都御史史琳提督軍務,禦之。秋八月甲寅,尊皇太后為太皇太后,皇后為皇太后。丙子,召朱暉等還。九月甲午,南京地震。丁酉,振陜西饑。冬十月丙辰,小王子犯甘肅。庚午,葬敬皇帝於泰陵。十一月甲申,御文華殿日講。是年,占城、安南入貢。

正德元年春正月乙酉,享太廟。己丑,大祀天地於南郊。二月壬子,御經筵。乙丑,耕耤田。三月甲申,釋奠於先師孔子。夏五月丙申,減蘇、杭織造歲幣。六月辛酉,禁吏民奢靡。免陜西被災稅糧。是日,大風雨壞郊壇獸瓦。庚午,諭群臣修省。秋八月乙卯,復遣內官南京織造。戊午,立皇后夏氏。冬十月丁巳,戶部尚書韓文帥廷臣請誅亂政內臣馬永成等八人,大學士劉健、李東陽、謝遷主之。戊午,韓文等再請,不聽。以劉瑾掌司禮監,丘聚、谷大用提督東、西廠,張永督十二團營兼神機營,魏彬督三千營,各據要地。劉健、李東陽、謝遷乞去,健、遷是日致仕。己未,東陽復乞去,不允。壬戌,吏部尚書焦芳兼文淵閣大學士,吏部侍郎王鏊兼翰林學士,入閣預機務。戊辰,停日講。十一月甲辰,罷韓文。十二月丁巳,命錦衣衛官點閱給事中。癸酉,除曲阜孔氏田賦。是年,哈密、烏斯藏入貢。

二年春正月乙亥朔,日有食之。乙酉,大祀天地於南郊。閏月庚戌,杖給事中艾洪、呂翀、劉蒨及南京給事中戴銑、御史薄彥徽等二十一人於闕下。二月戊戌,杖御史王良臣於午門,御史王時中荷校於都察院。三月辛未,以大學士劉健、謝遷,尚書韓文、楊守隨、張敷華、林瀚五十三人黨比,宣戒群臣。是月,敕各鎮守太監預刑名政事。夏五月戊午,度僧道四萬人。己巳,復寧王宸濠護衛。六月甲戌,孝宗神主祔太廟。戊寅,罷修邊垣,輸其費於京師。秋八月丙戌,作豹房。冬十月甲申,逮各邊巡撫都御史及管糧郎中下獄。丙戌,南京戶部尚書楊廷和為文淵閣大學士,預機務。十二月壬辰,開浙江、福建、四川銀礦。是年,琉球入貢。

三年春正月丁未,大祀天地於南郊。辛亥,大計外吏,中旨罷翰林學士吳儼、御史楊南金。二月己巳,令京官告假違限及病滿一年者皆致仕。三月乙卯,賜呂柟等進士及第、出身有差。夏四月乙亥,軍民納銀,得授都指揮僉事以下官。六月壬辰,得匿名文書於御道,跪群臣奉天門外詰之。下三百餘人於錦衣衛獄,尋釋之。秋七月壬子,命天下選樂工送京師。八月辛巳,立內廠,劉瑾領之。庚寅,下韓文錦衣衛獄,罰輸米千石於大同。是月,山東盜起。九月癸卯,削致仕尚書雍泰、馬文升、許進、劉大夏籍。辛酉,逮劉大夏下獄,戍肅州。癸亥,振南京饑。冬十月辛未,南京工部侍郎畢亨振湖廣、河南饑。十一月乙未,振鳳陽諸府饑。是年,安南、哈密、撒馬兒罕、烏斯藏入貢。

四年春正月丙午,大祀天地於南郊。二月丙戌,削劉健、謝遷籍。三月甲辰,振浙江饑。己酉,吏部侍郎張彩請不時考察京官,從之。夏四月乙亥,王鏊致仕。六月戊子,吏部尚書劉宇兼文淵閣大學士,預機務。秋八月辛酉,遣使核各邊屯田。是月,義州軍變。閏九月,小王子犯延綏,圍總兵官吳江於隴州城。冬十一月甲子,犯花馬池,總制尚書才寬戰死。十二月庚戌,奪劉健、謝遷等六百七十五人誥敕。是年,兩廣、江西、湖廣、陜西、四川並盜起。琉球、安南、哈密、土魯番、撒馬兒罕入貢。

五年春正月丁卯,大祀天地於南郊。庚辰,籍故尚書秦紘家。二月癸巳,兵部尚書曹元為吏部尚書兼文淵閣大學士,預機務。三月辛未,禱雨,釋獄囚,免正德三年逋賦。乙酉,江西賊熾,右都御史王哲巡視南、贛,刑部尚書洪鐘總制川、陜、河南、鄖陽軍務兼振恤湖廣。夏四月庚寅,安化王寘鐇反,殺巡撫都御史安惟學、總兵官姜漢。丙午,起右都御史楊一清總制寧夏、延綏、甘、涼軍務,涇陽伯神英充總兵官,討寘鐇。辛亥,詔赦天下。太監張永總督寧夏軍務。是日,遊擊將軍仇鉞襲執寘鐇,寧夏平。五月癸未,焦芳致仕。六月庚子,帝自號大慶法王,所司鑄印以進。丙午,劉宇罷。秋七月壬申,洪鐘討沔陽賊,平之。八月甲午,劉瑾以謀反下獄。詔自正德二年後所更政令悉如舊。戊戌,治劉瑾黨,吏部尚書張綵下獄。己亥,曹元罷。丁未,革寧王護衛。戊申,劉瑾伏誅。己酉,釋謫戍諸臣。九月丙辰,論平寘鐇功,封仇鉞咸寧伯。戊午,吏部尚書劉忠、梁儲並兼文淵閣大學士,預機務。己未,以平寘鐇、劉瑾功,封太監張永兄富、弟容皆為伯。癸酉,封義子指揮同知朱德、太監谷大用兄大寬、馬永成兄山、魏彬弟英皆為伯。冬十月己亥,戮張彩尸於市。十二月己丑,賊陷江津,僉事吳景死之。是年,日本、占城、哈密、撒馬兒罕、土魯番、烏斯藏入貢。

六年春正月甲子,大祀天地於南郊。癸酉,賊陷營山,殺僉事王源。二月丙申,寘鐇伏誅。己酉,起左都御史陳金總制江西軍務討賊。三月戊辰,賜楊慎等進士及第、出身有差。庚午,惠安伯張偉充總兵官,右都御史馬中錫提督軍務,討直隸、河南、山東賊。丙子,免被寇州縣稅糧一年。是月,小王子入河套,犯沿邊諸堡。夏四月癸未,劉忠乞省墓歸。是月,淮安盜起。六月,山西盜起。秋七月壬申,賊犯文安,京師戒嚴。癸酉,調宣府、延綏兵入援。八月己卯,兵部侍郎陸完將邊軍討賊。四川巡撫都御史林俊擒斬賊首藍廷瑞、鄢本恕。甲申,賊劉六犯固安。丙戌,召張偉、馬中錫還。九月丙寅,再調宣府及遼東兵益陸完軍。冬十月癸未,賊陷長山,典史李暹戰死。甲申,賊焚糧艘於濟寧州。丁酉,甘州副總兵白琮敗小王子於柴溝。十一月庚戌,太監谷大用、張忠、伏羌伯毛銳帥京軍會陸完討賊。丙辰,戶部侍郎叢蘭、王瓊振兩畿、河南、山東。戊午,京師地震。辛酉,敕修省。乙亥,瘞暴骨。十二月癸巳,禮部尚書費宏兼文淵閣大學士,預機務。甲午,清河口至柳鋪,黃河清三日。辛丑,賊掠蒼溪,兵備副使馮傑敗死。是年,自畿輔迄江、淮、楚、蜀,盜賊殺官吏,山東尤甚,至破九十餘城,道路梗絕。琉球、哈密入貢。

七年春正月甲寅,賊犯霸州,京師戒嚴。丁巳,陷大城,知縣張汝舟、主簿李銓戰死。己未,大祀天地於南郊。二月丁丑,副都御史彭澤、咸寧伯仇鉞提督軍務,太監陸訚監軍,討河南賊。己卯,賊犯萊州,指揮僉事蔡顯等力戰死。三月辛未,副總兵時源敗績於河南,都督僉事馮禎力戰死。夏五月丙午,陸完敗賊於萊州,山東賊平。甲寅,左都御史陳金討平撫州賊。丙寅,賊殺副都御史馬炳然於武昌江中。閏月壬辰,仇鉞敗賊於光山,河南賊平。秋七月癸巳,江西賊殺副使周憲於華林。丁酉,振四川饑。八月癸亥,陸完追殲劉七等賊於狼山。九月乙酉,陳金討平華林賊。戊子,召洪鐘還。都御史彭澤總制四川軍務。丙申,賜義子一百二十七人國姓。冬十月,免河南、江西、浙江被災寇者稅糧。十一月壬申,時源為平賊將軍,會彭澤討四川賊。丁亥,留大同、宣府、遼東兵於京營,李東陽諫,不聽。十二月丁卯,李東陽致仕。是月,免兩畿、山東、山西、陜西被災寇者稅糧。是年,安南、日本、哈密入貢。

八年春正月癸酉,右副都御史俞諫代陳金討江西賊。壬午,大祀天地於南郊。乙酉,以邊將江彬、許泰分領京營,賜國姓。尋設兩官廳軍,命彬、泰分領之。癸巳,戶部侍郎叢蘭、僉都御史陳玉巡邊。二月丙午,以平賊功,封太監谷大用弟大亮、陸訚姪永皆為伯。三月戊子,置鎮國府處宣府官軍。甲午,以旱敕群臣修省。夏四月乙丑,彭澤破賊於劍州。五月辛巳,仇鉞充總兵官,帥京營兵禦敵於大同。六月戊戌,河決黃陵岡。乙卯,俞諫破賊於貴溪。秋八月,免南畿水災稅糧。土魯番襲據哈密。冬十月丁未,俞諫連破賊於東鄉,江西賊平。十二月,南京刑部侍郎鄧璋振江西饑。是年,哈密入貢。

九年春正月丁丑,大祀天地於南郊。庚辰,乾清宮災。二月庚子,帝始微行。丙午,禮部尚書靳貴兼文淵閣大學士,預機務。癸丑,彭澤、時源討平四川賊。三月辛巳,賜唐皋等進士及第、出身有差。夏四月丁酉,復寧王護衛,予屯田。五月乙丑,費寵致仕。己丑,彭澤總督甘肅軍務,經理哈密。六月乙卯,開雲南銀礦。秋七月乙丑,小王子犯宣府、大同。太監張永提督軍務,都督白玉充總兵官,帥京營兵禦之。八月辛卯朔,日有食之。辛丑,小王子犯白羊口。乙巳,京師地震。己未,小王子入寧武關,掠忻州、定襄、寧化。九月壬戌,犯宣府、蔚州。庚午,帝狎虎被傷,不視朝,編修王思以諫謫饒平驛丞。冬十月己酉,遣使採木於川、湖。十一月辛酉,廢歸善王當沍為庶人,自殺。十二月甲寅,建乾清宮,加天下賦一百萬。是年,安南、哈密、烏斯藏入貢。十年春正月癸亥,薄暮,享太廟。戊辰,薄暮,祀天地於南郊。三月壬申,楊廷和以憂去。夏閏四月辛酉,吏部尚書楊一清兼武英殿大學士,預機務。戊寅,召彭澤還。秋八月丙寅,小王子犯固原。冬十二月癸丑朔,日有食之。己卯,免南畿旱災秋糧。是年,琉球、安南、哈密、撒馬兒罕入貢。

十一年春正月乙未,大祀天地於南郊。夏四月,振河南饑。五月庚寅,土魯番以哈密來歸。甲辰,錄自宮男子三千四百餘人充海戶。是月,振陜西饑。秋七月乙未,小王子犯薊州白羊口,太監張忠監督軍務,左都督劉暉充總兵官,帥東西官廳軍禦之。丙午,工部侍郎趙璜、俞琳飭畿內武備。八月丁巳,左都御史彭澤、成國公朱輔帥京營兵防邊。庚申,賜宛平縣被寇者人米二石。甲子,楊一清致仕。丁丑,禮部尚書蔣冕兼文淵閣大學士,預機務。九月,土魯番復據哈密,侵肅州,殺遊擊芮寧。冬十月己酉朔,享太廟,遣使代行禮。十一月甲申,免湖廣被災稅糧。是年,琉球、天方入貢。

十二年春正月己丑,大祀天地於南郊。遂獵於南海子,夜中還,御奉天殿受朝賀。三月癸巳,賜舒芬等進士及第、出身有差。戊戌,以兩淮、浙江、四川、河東鹽課充陜西織造。夏四月壬子,靳貴致仕。丙辰,副總兵鄭廉敗土魯番於瓜州。五月丙子,禮部尚書毛紀兼東閣大學士,預機務。六月乙巳朔,日有食之。秋八月甲辰,微服如昌平。乙巳,梁儲、蔣冕、毛紀追及於沙河,請回蹕,不聽。己酉,至居庸關,巡關御史張欽閉關拒命,乃還。丙辰,至自昌平。戊午,夜視朝。癸亥,副都御史吳廷舉振湖廣饑。丙寅,夜微服出德勝門,如居庸關。辛未,出關,幸宣府,命谷大用守關,毋出京朝官。九月辛卯,河決城武。壬辰,如陽和,自稱總督軍務威武大將軍總兵官。庚子,輸帑銀一百萬兩於宣府。冬十月癸卯,駐蹕順聖川。甲辰,小王子犯陽和,掠應州。丁未,親督諸軍禦之,戰五日。辛亥,寇引去,駐蹕大同。十一月丁亥,召楊廷和復入閣。戊子,還至宣府。十二月癸亥,群臣赴行在請還宮,不得出關而還。閏月丁亥,迎春於宣府。是年,琉球、烏斯藏入貢。

十三年春正月辛丑朔,帝在宣府。丙午,至自宣府,命群臣具彩帳、羊酒郊迎,御帳殿受賀。丁未,罷南郊致齋。庚戌,大祀天地於南郊,遂獵於南海子。辛亥,還宮。辛酉,復如宣府。是月,振兩畿、山東水災。給京師流民米,人三斗。瘞死者。二月己卯,太皇太后崩。壬午,至自宣府。三月戊辰,如昌平。夏四月己巳朔,謁六陵,遂幸密雲。五月己亥朔,日有食之。駐蹕喜峰口。戊申,至自喜峰口。六月庚辰,太皇太后梓宮發京師,帝戎服從。甲申,葬孝貞純皇后。乙酉,至自昌平。秋七月己亥。錄應州功,敘廕升賞者五萬餘人。丙午,復如宣府。八月乙酉,如大同。九月庚子,次偏頭關。癸丑,敕曰:「總督軍務威武大將軍總兵官朱壽親統六師,肅清邊境,特加封鎮國公,歲支祿米五千石。吏部如敕奉行。」甲寅,封朱彬為平虜伯,朱泰為安邊伯。冬十月戊辰,渡河。己卯,次榆林。十一月庚子,調西官廳及四衛營兵赴宣、大。壬子,次綏德,幸總兵官戴欽第。十二月戊寅,渡河,幸石州。戊子,次太原。是年,琉球、天方、瓦剌入貢。

十四年春正月丙申朔,帝在太原。甲辰,改卜郊。壬子,還宣府。二月壬申,至自宣府。丁丑,大祀天地於南郊,遂獵於南海子。是日,京師地震。己丑,帝自加太師,諭禮部曰:「總督軍務威武大將軍總兵官太師鎮國公朱壽將巡兩畿、山東,祀神祈福,其具儀以聞。」三月癸丑,以諫巡幸,下兵部郎中黃鞏六人於錦衣衛獄,跪修撰舒芬百有七人於午門五日。金吾衛都指揮僉事張英自刃以諫,衛士奪刃,得不死,鞫治,杖殺之。乙卯,下寺正周敘、行人司副餘廷瓚、主事林大輅三十三人於錦衣衛獄。戊午,杖舒芬等百有七人於闕下。是日,風霾晝晦。夏四月甲子,免南畿被災稅糧。戊寅,杖黃鞏等三十九人於闕下,先後死者十一人。五月己亥,詔山東、山西、陜西、河南、湖廣流民歸業者,官給廩食、廬舍、牛種,復五年。六月丙子,寧王宸濠反,巡撫江西右副都御史孫燧、南昌兵備副使許逵死之。戊寅,陷南康。己卯,陷九江。秋七月甲辰,帝自將討宸濠,安邊伯朱泰為威武副將軍。帥師為先鋒。丙午,宸濠犯安慶,都指揮楊銳、知府張文錦禦卻之。辛亥,提督南贛汀漳軍務副都御史王守仁帥兵復南昌。丁巳,守仁敗宸濠於樵舍,擒之。八月癸未,車駕發京師。丁亥,次涿州,王守仁捷奏至,秘不發。冬十一月乙巳,漁於清江浦。壬子,冬至,受賀於太監張陽第。十二月辛酉,次揚州。乙酉,渡江。丙戌,至南京。是歲,淮、揚饑,人相食。撒馬兒罕入貢。

十五年春正月庚寅朔,帝在南京。癸巳,改卜郊。夏四月己未,振淮、揚諸府饑。六月丁巳。次牛首山,諸軍夜驚。秋七月,小王子犯大同、宣府。八月癸未,免江西稅糧。閏月癸巳,受江西俘。丁酉,發南京。癸卯,次鎮江,幸大學士楊一清第,臨故大學士靳貴喪。九月己巳,漁於積水池,舟覆,救免,遂不豫。冬十月庚戌,次通州。十一月庚申,治交通宸濠者罪,執吏部尚書陸完赴行在。十二月己丑,宸濠伏誅。甲午,還京師,告捷於郊廟社稷。丁酉,大祀天地於南郊。初獻疾作,不克成禮。是年,琉球、占城、佛郎機、土魯番入貢。

十六年春正月癸亥,改卜郊。二月己亥,巡撫雲南副都御史何孟春討平彌勒州苗。三月癸丑朔,日有食之。庚申,改西宮廳為威武團營。乙丑,大漸,諭司禮監曰:「朕疾不可為矣。其以朕意達皇太后,天下事重,與閣臣審處之。前事皆由朕誤,非汝曹所能預也。」丙寅,崩於豹房,年三十有一。遺詔召興獻王長子嗣位。罷威武團營,遣還各邊軍,革京城內外皇店,放豹房番僧及教坊司樂人。戊辰,頒遺詔於天下,釋繫囚,還四方所獻婦女,停不急工役,收宣府行宮金寶還內庫。庚午,執江彬等下獄。世宗入立。五月己未,上尊謚,廟號武宗,葬康陵。

贊曰:明自正統以來,國勢浸弱。毅皇手除逆瑾,躬御邊寇,奮然欲以武功自雄。然耽樂嬉游,暱近群小,至自署官號,冠履之分蕩然矣。猶幸用人之柄躬自操持,而秉鈞諸臣補苴匡救,是以朝綱紊亂,而不底於危亡。假使承孝宗之遺澤,制節謹度,有中主之操,則國泰而名完,豈至重後人之訾議哉!


\end{pinyinscope}