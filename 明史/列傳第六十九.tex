\article{列傳第六十九}

\begin{pinyinscope}
○徐溥邱濬劉健謝遷李東陽王鏊劉忠

徐溥,字時用,宜興人。祖鑒,瓊州知府,有惠政。溥,景泰五年進士及第。授編修。憲宗初,擢左庶子,再遷太常卿兼學士。成化十五年拜禮部右侍郎,尋轉左,久之改吏部。孝宗嗣位,兼文淵閣大學士,參預機務。旋進禮部尚書。

弘治五年,劉吉罷,溥為首輔,屢加少傅、太子太傅。溥承劉吉恣睢之後,鎮以安靜,務守成法。與同列劉健、李東陽、謝遷等協心輔治,事有不可,輒共爭之。欽天監革職監正李華為昌國公張巒擇葬地,中旨復官。溥等言:「即位以來,未嘗有內降。倖門一開,未流安底。臣等不敢奉詔。」八年,太皇太后召崇王來朝,溥等與尚書倪岳諫,帝為請乃已。占城奏安南侵擾,帝欲遣大臣往解。溥等言:「外國相侵,有司檄諭之足矣,無勞遣使。萬一抗令,則虧損國體,問罪興師,後患滋大。」於是罷不遣。

是年十二月,詔撰三清樂章。溥等言:「天至尊無對。漢祀五帝,儒者猶非之,況三清乃道家妄說耳。一天之上,安得有三大帝?且以周柱下史李耳當其一,以人鬼列天神,矯誣甚矣。郊祀樂章皆太祖所親製,今使製為時俗詞曲以享神明,褻瀆尤甚。臣等誦讀儒書,邪說俚曲素所不習,不敢以非道事陛下。國家設文淵閣,命學士居之,誠欲其謨謀政事,講論經史,培養本原,匡弼闕失,非欲其阿諛順旨,惟言莫違也。今經筵早休,日講久曠,異端乘間而入。此皆臣等無狀,不足以啟聖心,保初政。憂愧之至,無以自容。數月以來,奉中旨處分未當者封還,執奏至再至三。願陛下曲賜聽從,俾臣等竭駑鈍,少有裨益,非但樂章一事而已。」奏入,帝嘉納之。

帝自八年後,視朝漸晏,溥等屢以為言。中官李廣以燒煉齋醮寵。十年二月,溥等上疏極論曰:「舊制,內殿日再進奏,事重者不時上聞,又常面召儒臣,咨訪政事。今奏事日止一次,朝參之外,不得一望天顏。章奏批答不時斷決,或稽留數月,或竟不施行。事多壅滯,有妨政體。經筵進講,每歲不過數日,正士疏遠,邪說得行。近聞有以齋醮修煉之說進者。宋徽宗崇道教,科儀符籙最盛,卒至乘輿播遷。金石之藥,性多酷烈。唐憲宗信柳泌以殞身,其禍可鑒。今龍虎山上清宮、神樂觀、祖師殿及內府番經廠皆焚毀無餘,彼如有靈,何不自保?天厭其穢,亦已明甚。陛下若親近儒臣,明正道,行仁政,福祥善慶,不召自至,何假妖妄之說哉!自古奸人蠱惑君心者,必以太平無事為言。唐臣李絳有云:『憂先於事,可以無憂。事至而憂,無益於事。』今承平日久,溺於晏安。目前視之,雖若無事,然工役繁興,科斂百出,士馬罷敝,閭閻困窮,愁歎之聲上干和氣,致熒惑失度,太陽無光,天鳴地震,草木興妖,四方奏報殆無虛月,將來之患灼然可憂。陛下高居九重,言官皆畏罪緘默。臣等若復不言,誰肯為陛下言者。」帝感其言。

三月甲子,御文華殿,召見溥及劉健、李東陽、謝遷,授以諸司題奏曰:「與先生輩議。」溥等擬旨上,帝應手改定。事端多者,健請出外詳閱。帝曰:「盍就此面議。」既畢,賜茶而退。自成化間,憲宗召對彭時、商輅後,至此始再見,舉朝詡為盛事。然終溥在位,亦止此一召而已。

尋以災異求言,廷臣所上封事,經月不報,而言官論救何鼎忤旨待罪者久,溥等皆以為言。於是悉下諸章,而罷諸言官弗問。溥時年七十,引年求退,不許。詔風雨寒暑免朝參。

十一年,皇太子出閤,加少師兼太子太師,進華蓋殿大學士。以目疾乞歸。帝眷留,久之乃許,恩賚有加。踰年卒,贈太師,謚文靖。

溥性凝重有度,在內閣十二年,從容輔導。人有過誤,輒為掩覆,曰:「天生才甚難,不忍以微瑕棄也。」屢遇大獄及逮繫言官,委曲調劑。孝宗仁厚,多納溥等所言,天下陰受其福。嘗曰:「祖宗法度所以惠元元者備矣,患不能守耳。」卒無所更置。性至孝,嘗再廬墓。自奉甚薄,好施予。置義田八百畝贍宗族,請籍記於官,以垂永久,帝為復其徭役。

邱濬,字仲深,瓊山人。幼孤,母李氏教之讀書,過目成誦。家貧無書,嘗走數百里借書,必得乃已。舉鄉試第一,景泰五年成進士。改庶吉士,授編修。濬既官翰林,見聞益廣,尤熟國家典故,以經濟自負。

成化元年,兩廣用兵,濬奏記大學士李賢,指陳形勢,纚纚數千言。賢善其計,聞之帝,命錄示總兵官趙輔、巡撫都御史韓雍。雍等破賊,雖不盡用其策,而濬以此名重公卿間。秩滿,進侍講。與修《英宗實錄》,進侍講學士。《續通鑒綱目》成,擢學士,遷國子祭酒。時經生文尚險怪,濬主南畿鄉試,分考會試皆痛抑之。及是,課國學生尤諄切告誡,返文體於正。尋進禮部右侍郎,掌祭酒事。

濬以真德秀《大學衍義》於治國平天下條目未具,乃博採群書補之。孝宗嗣位,表上其書,帝稱善,賚金幣,命所司刊行。特進禮部尚書,掌詹事府事。修《憲宗實錄》,充副總裁。弘治四年,書成,加太子太保,尋命兼文淵閣大學士參預機務。尚書入內閣者自濬始,時年七十一矣。濬以《衍義補》所載皆可見之行事,請摘其要者奏聞,下內閣議行之。帝報可。

明年,濬上言:「臣見成化時彗星三見,遍掃三垣,地五六百震。邇者彗星見天津,地震天鳴無虛日,異鳥三鳴於禁中。《春秋》二百四十年,書彗孛者三,地震者五,飛禽者二。今乃屢見於二十年之間,甚可畏也。願陛下體上天之仁愛,念祖宗之艱難,正身清心以立本而應務。謹好尚不惑於異端,節財用不至於耗國,公任使不失於偏聽。禁私謁,明義理,慎儉德,勤政務,則承風希寵、左道亂政之徒自不敢肆其奸,而天災弭矣。」因列時弊二十二事。帝納之。六年以目疾免朝參。

濬在位,嘗以寬大啟上心,忠厚變士習。顧性褊隘,嘗與劉健議事不合,至投冠於地。言官建白不當意,輒面折之。與王恕不相得,至不交一言。六年大計群吏,恕所奏罷二千人。濬請未及三載者復任,非貪暴有顯跡者勿斥,留九十人。恕爭之不得,求去。太醫院判劉文泰嘗往來濬家,以失職訐恕,恕疑文泰受濬指,而言者嘩然,言疏稿出濬手。恕竟坐罷,人以是大不直濬。給事中毛珵,御史宋惪、周津等交章劾濬不可居相位,帝不問。踰年,加少保。八年卒,年七十六。贈太傅,謚文莊。

濬廉介,所居邸第極湫隘,四十年不易。性嗜學,既老,右目失明,猶披覽不輟。議論好矯激,聞者駭愕。至修《英宗實錄》,有言於謙之死當以不軌書者。濬曰:「己巳之變,微于公社稷危矣。事久論定,誣不可不白。」其持正又如此。正德中,以巡按御史言賜祠於鄉。曰「景賢」。

劉健,字希賢,洛陽人。父亮,三原教諭,有學行。健少端重,與同邑閻禹錫、白良輔遊,得河東薛瑄之傳。舉天順四年進士,改庶吉士,授編修。謝交遊,鍵戶讀書,人以木強目之。然練習典故,有經濟志。

成化初,修《英宗實錄》,起之憂中,固辭,不許。書成,進修撰,三遷至少詹事,充東宮講官,受知於孝宗。既即位,進禮部右侍郎兼翰林學士,入內閣參預機務。弘治四年進尚書兼文淵閣大學士,累加太子太保,改武英殿。十一年春,進少傅兼太子太傅,代徐溥為首輔。

健學問深粹,正色敢言,以身任天下之重。清寧宮災,太監李廣有罪自殺。健與同列李東陽、謝遷疏言:「古帝王未有不遇災而懼者。向來奸佞熒惑聖聽,賄賂公行,賞罰失當,災異之積,正此之由。今幸元惡殄喪,聖心開悟,而餘慝未除,宿弊未革。伏願奮發勵精,進賢黜姦,明示賞罰。凡所當行,斷在不疑,毋更因循,以貽後悔。」帝方嘉納其言,而廣黨蔡昭等旋取旨予廣祭葬、祠額。健等力諫,僅寢祠額。南北言官指陳時政,頻有所論劾,一切皆不問。國子生江瑢劾健、東陽杜抑言路。帝慰留健、東陽,而下瑢於獄,二人力救得釋。

十三年四月,大同告警,京師戒嚴。兵部請甄別京營諸將,帝召健及東陽、遷至平臺面議去留。乃去遂安伯陳韶等三人,而召鎮遠侯顧溥督團營。時帝視朝頗晏,健等以為言,頷之而已。

十四年秋,帝以軍興缺餉,屢下廷議。健等言:「天下之財,其生有限。今光祿歲供增數十倍,諸方織作務為新巧,齋醮日費鉅萬。太倉所儲不足餉戰士,而內府取入動四五十萬。宗籓、貴戚之求土田奪鹽利者,亦數千萬計。土木日興,科斂不已。傳奉冗官之俸薪,內府工匠之餼廩,歲增月積,無有窮期,財安得不匱?今陜西、遼東邊患方殷,湖廣、貴州軍旅繼動,不知何以應之。望陛下絕無益之費,躬行節儉,為中外倡,而令群臣得畢獻其誠,講求革弊之策,天下幸甚。」明年四月,以災異陳勤朝講、節財用、罷齋醮、公賞罰數事。及冬,南京、鳳陽大水,廷臣多上言時務,久之不下。健等因極陳怠政之失,請勤聽斷以振紀綱,帝皆嘉納。《大明會典》成,加少師兼太子太師、吏部尚書、華蓋殿大學士。與東陽、遷同賜蟒衣。閣臣賜蟒自健等始。

帝孝事兩宮太后甚謹,而兩宮皆好佛、老。先是,清寧宮成,命灌頂國師設壇慶贊,又遣中官齎真武像,建醮武當山,使使詣泰山進神袍,或白晝散燈市上。帝重違太后意,曲從之,而健等諫甚力。十五年六月詔擬《釋迦啞塔像贊》,十七年二月詔建延壽塔朝陽門外,除道士杜永祺等五人為真人,皆以健等力諫得寢。

是年夏,小王子謀犯大同,帝召見閣臣。健請簡京營大帥,因言京軍怯不任戰,請自今罷其役作以養銳氣。帝然之。退復條上防邊事宜,悉報允。未幾,邊警狎至,帝惑中官苗逵言,銳欲出師。健與東陽、遷委曲阻之,帝意猶未回。兵部尚書劉大夏亦言京軍不可動,乃止。

帝自十三年召對健等後,閣臣希得進見。及是在位久,益明習政事,數召見大臣,欲以次革煩苛,除宿弊。嘗論及理財,東陽極言鹽政弊壞,由陳乞者眾,因而私販數倍。健進曰:「太祖時茶法始行,駙馬歐陽倫以私販坐死,高皇后不能救。如倫事,孰敢為陛下言者?」帝曰:「非不敢言,不肯言耳。」遂詔戶部核利弊,具議以聞。

當是時,健等三人同心輔政,竭情盡慮,知無不言。初或有從有不從,既乃益見信,所奏請無不納,呼為「先生」而不名。每進見,帝輒屏左右。左右間從屏間竊聽,但聞帝數數稱善。諸進退文武大臣,釐飭屯田、鹽、馬諸政,健翊贊為多。

未幾,帝疾大漸,召健等入乾清宮。帝力疾起坐,自敘即位始末甚詳,令近侍書之。已,執健手曰:「先生輩輔導良苦。東宮聰明,但年尚幼,好逸樂。先生輩常勸之讀書,輔為賢主。」健等欷歔,頓首受命而出。翌日帝崩。

武宗嗣位,健等釐諸弊政,凡孝宗所欲興罷者,悉以遺詔行之。劉瑾者,東宮舊豎也,與馬永成、谷大用、魏彬、張永、邱聚、高鳳、羅祥等八人俱用事,時謂之「八黨」。日導帝遊戲,詔條率沮格不舉。京師淫雨自六月至八月。健等乃上言:「陛下登極詔出,中外歡呼,想望太平。今兩月矣,未聞汰冗員幾何,省冗費幾何。詔書所載,徒為空文。此陰陽所以失調,雨暘所以不若也。如監局、倉庫、城門及四方守備內臣增置數倍,朝廷養軍匠費鉅萬計,僅足供其役使,寧可不汰?文武臣曠職僨事、虛糜廩祿者,寧可不黜?畫史、工匠濫授官職者多至數百人,寧可不罷?內承運庫累歲支銀數百餘萬,初無文簿,司鑰庫貯錢數百萬,未知有無,寧可不勾校?至如縱內苑珍禽奇獸,放遣先朝宮人,皆新政所當先,而陛下悉牽制不行,何以尉四海之望?」帝雖溫詔答之,而左右宦豎日恣,增益且日眾。享祀郊廟,帶刀被甲擁駕後。內府諸監局僉書多者至百數十人,光祿日供驟益數倍。健等極陳其弊,請勤政、講學,報聞而已。

正德元年二月,帝從尚書韓文言,畿甸皇莊令有司徵課,而每莊仍留宦官一人、校尉十人。健等言「皇莊既以進奉兩宮,自宜悉委有司,不當仍主以私人,反失朝廷尊親之意」,因備言內臣管莊擾民。不省。

吏、戶、兵三部及都察院各有疏爭職掌為近習所撓。健等擬旨,上不從,令再擬。健等力諫,謂:「奸商譚景清之沮壞鹽政,北征將士之無功授官,武臣神英之負罪玩法,御用監書篆之濫收考較,皆以一二人私恩,壞百年定制。況今政令維新,而地震天鳴,白虹貫日,恒星晝見,太陽無光。內賊縱橫,外寇猖獗。財匱民窮,怨謗交作。而中外臣僕方且乘機作奸,排忠直猶仇讎,保奸回如骨肉。日復一日,愈甚於前,禍變之來恐當不遠。臣等受知先帝,叨任腹心。邇者旨從中下,略不與聞。有所擬議,竟從改易。似此之類,不可悉舉。若復顧惜身家,共為阿順,則罔上誤國,死有餘辜。所擬四疏,不敢更易,謹以原擬封進。」不報。

居數日,又言:「臣等遭逢先帝,臨終顧命,心卷心卷以陛下為託,痛心刻骨,誓以死報。即位詔書,天下延頸,而朝令夕改,迄無寧日。百官庶府,仿傚成風,非惟廢格不行,抑且變易殆盡。建言者以為多言,幹事者以為生事,累章執奏謂之瀆擾,釐剔弊政謂之紛更。憂在於民生國計,則若罔聞知,事涉於近幸貴戚,則牢不可破。臣等心知不可,義當盡言。比為鹽法、賞功諸事,極陳利害,拱俟數日,未蒙批答。若以臣等言是,宜賜施行,所言如非,即當斥責。乃留中不報,視之若無。政出多門,咎歸臣等。宋儒朱子有言『一日立乎其位,則一日業乎其官;一日不得乎其官,則不敢一日立乎其位。』若冒顧命之名而不盡輔導之實,既負先帝,又負陛下,天下後世其謂臣何?伏乞聖明矜察,特賜退休。」帝優旨慰留之,疏仍不下。

越五日,健等復上疏,歷數政令十失,指斥貴戚、近倖尤切。因再申前請。帝不得已,始下前疏,命所司詳議。健知志終不行,首上章乞骸骨,李東陽、謝遷繼之,帝皆不許。既而所司議上,一如健等指。帝勉從之,由是諸失利者咸切齒。

六月庚午復上言:「近日以來,免朝太多,奏事漸晚,游戲漸廣,經筵日講直命停止。臣等愚昧,不知陛下宮中復有何事急於此者。夫濫賞妄費非所以崇儉德,彈射釣獵非所以養仁心,鷹犬狐兔田野之物不可育於朝廷,弓矢甲胄戰鬥之象不可施於宮禁。今聖學久曠,正人不親,直言不聞,下情不達,而此數者雜交於前,臣不勝憂懼。」帝曰:「朕聞帝王不能無過,貴改過。卿等言是,朕當行之。」健等乃錄廷臣所陳時政切要者,請置坐隅朝夕省覽:曰無單騎馳驅,輕出宮禁;曰無頻幸監局,泛舟海子;曰無事鷹犬彈射;曰無納內侍進獻飲膳。疏入,報聞。

先是,孝宗山陵畢,健等即請開經筵。常初勉應之,後數以朝謁兩宮停講,或云擇日乘馬。健等陳諫甚切至。八月,帝既大婚,健等又請開講。命俟九月,至期又命停午講。健等以先帝故事,日再進講,力爭不得。

當是時,健等懇切疏諫者屢矣,而帝以狎近群小,終不能改。既而遣中官崔杲等督織造,乞鹽萬二千引。所司執奏,給事中陶諧、徐昂,御史杜旻、邵清、楊儀等先後諫,健等亦言不可。帝召健等至煖閣面議,頗有所詰問,健等皆以正對。帝不能難,最後正色曰:「天下事豈皆內官所壞?朝臣壞事者十常六七,先生輩亦自知之。」因命鹽引悉如杲請。健等退,再上章言不可。帝自愧失言,乃俞健等所奏。於是中外咸悅,以帝庶幾改過。

健等遂謀去「八黨」,連章請誅之。言官亦交論群閹罪狀,健及遷、東陽持其章甚力。帝遣司禮詣閣曰:「朕且改矣,其為朕曲赦若曹。」健等言:「此皆得罪祖宗,非陛下所得赦。」復上言曰:「人君之於小人,不知而誤用,天下尚望其知而去之。知而不去則小人愈肆。君子愈危,不至於亂亡不已。且邪正不並立,今舉朝欲決去此數人,陛下又知其罪而故留之左右,非特朝臣疑懼,此數人亦不自安。上下相猜,中外不協,禍亂之機始此矣。」不聽。健等以去就爭。瑾等八人窘甚,相對涕泣。而尚書韓文等疏復入,於是帝命司禮王岳等詣閣議,一日三至,欲安置瑾等南京。遷欲遂誅之,健推案哭曰:「先帝臨崩,執老臣手,付以大事。今陵土未乾,使若輩敗壞至此,臣死何面目見先帝!」聲色俱厲。岳素剛正疾邪,慨然曰:「閣議是。」其儕范亨、徐智等亦以為然。是夜,八人益急,環泣帝前。帝怒,立收岳等下詔獄,而健等不知,方倚岳內應。明日,韓文倡九卿伏闕固爭,健逆謂曰:「事且濟,公等第堅持。」頃之,事大變,八人皆宥不問,而瑾掌司禮。健、遷遂乞致仕,賜敕給驛歸,月廩、歲夫如故事。

健去,瑾憾不已。明年三月辛未詔列五十三人為奸黨,榜示朝堂,以健為首。又二年削籍為民,追奪誥命。瑾誅,復官,致仕。後聞帝數巡遊,輒歎息不食曰:「吾負先帝。」世宗立,命行人齎敕存問,以司馬光、文彥博為比,賜賚有加。及年躋九十,詔撫臣就第致束帛、餼羊、上尊,官其孫成學中書舍人。嘉靖五年卒,年九十四。遺表數千言,勸帝正身勤學,親賢遠佞。帝震悼,賜恤甚厚,贈太師,謚文靖。

健器局嚴整,正己率下。朝退,僚寀私謁,不交一言。許進輩七人欲推焦芳入吏部,健曰:「老夫不久歸田,此座即焦有,恐諸公俱受其害耳。」後七人果為芳所擠。

東陽以詩文引後進,海內士皆抵掌談文學,健若不聞,獨教人治經窮理。其事業光明俊偉,明世輔臣鮮有比者。

孫望之,進士。

謝遷,字於喬,餘姚人。成化十年鄉試第一。明年舉進士,復第一。授修撰,累遷左庶子。

弘治元年春,中官郭鏞請豫選妃嬪備六宮。遷上言:「山陵未畢,禮當有待。祥禫之期,歲亦不遠。陛下富於春秋,請俟諒陰既終,徐議未晚。」尚書周洪謨等如遷議,從之。帝居東宮時,遷已為講官,及是,與日講,務積誠開帝意。前夕必正衣冠習誦,及進講,敷詞詳切,帝數稱善。進少詹事兼侍講學士。

八年詔同李東陽入內閣參預機務。遷時居憂,力辭,服除始拜命。進詹事兼官如故,皇太子出閣,加太子少保、兵部尚書兼東閣大學士。上疏勸太子親賢遠佞,勤學問,戒逸豫,帝嘉之。尚書馬文升以大同邊警,餉饋不足,請加南方兩稅折銀。遷曰:「先朝以南方賦重,故折銀以寬之。若復議加,恐民不堪命。且足國在節用,用度無節,雖加賦奚益。」尚書倪岳亦爭之,議遂寢。

孝宗晚年慨然欲釐弊政。而內府諸庫及倉場、馬坊中官作奸骫法,不可究詰。御馬監、騰驤四衛勇士自以禁軍不隸兵部,率空名支餉,其弊尤甚。遷乘間言之,帝令擬旨禁約。遷曰:「虛言設禁無益,宜令曹司搜剔弊端,明白奏聞。然後嚴立條約,有犯必誅,庶積蠹可去。」帝俞允之。

遷儀觀俊偉,秉節直亮。與劉健、李東陽同輔政,而遷見事明敏,善持論。時人為之語曰:「李公謀,劉公斷,謝公尤侃侃。」天下稱賢相。

武宗嗣位,屢加少傅兼太子太傅。數諫,帝弗聽。因天變求去甚力,帝輒慰留。及請誅劉瑾不克,遂與健同致仕歸,禮數俱如健。而瑾怨遷未已。焦芳既附瑾入內閣,亦憾遷嘗舉王鏊、吳寬自代,不及己,乃取中旨勒罷其弟兵部主事迪,斥其子編修丕為民。

四年二月,以浙江應詔所舉懷才抱德士餘姚周禮、徐子元、許龍,上虞徐文彪,皆遷同鄉,而草詔由健,欲因此為二人罪。矯旨謂「餘姚隱士何多,此必徇私援引」,下禮等詔獄,詞連健、遷。瑾欲逮健、遷,籍其家,東陽力解。芳從旁厲聲曰:「縱輕貸,亦當除名!」旨下,如芳言,禮等咸戍邊。尚書劉宇復劾兩司以上訪舉失實,坐罰米,有削籍者。且詔自今餘姚人毋選京官,著為令。其年十二月,言官希瑾指,請奪健、遷及尚書馬文升、劉大夏、韓文、許進等誥命,詔並追還所賜玉帶服物。同時奪誥命者六百七十五人。當是時,人皆為遷危,而遷與客圍棋、賦詩自若。瑾誅,復職,致仕。

世宗即位,遣使存問,起迪參議,丕復官翰林。遷乃遣子正入謝。勸帝勤學、法祖、納諫,優旨答之。嘉靖二年復詔有司存問。六年,大學士費宏舉遷自代,楊一清欲阻張璁,亦力舉遷。帝乃遣行人齎手敕即家起之,命撫、按官敦促上道。遷年七十九矣,不得已拜命,比至,而璁已入閣,一清以官尊於遷無相下意。遷居位數月,力求去。帝待遷愈厚,以天寒免朝參,除夕賜御製詩。及以病告,則遣醫賜藥餌,光祿致酒餼,使者相望於道。遷竟以次年三月辭歸。十年卒於家,年八十有三。贈太傅,謚文正。

迪仕至廣東布政使。丕鄉試第一,弘治末進士及第。歷官吏部左侍郎,贈禮部尚書。

李東陽,字賓之,茶陵人,以戍籍居京師。四歲能作徑尺書,景帝召試之,甚喜,抱置膝上,賜果鈔。後兩召講《尚書》大義,稱旨,命入京學。天順八年,年十八,成進士,選庶吉士,授編修。累遷侍講學士,充東宮講官。

弘治四年,《憲宗實錄》成,由左庶子兼侍講學士,進太常少卿,兼官如故。五年,旱災求言。東陽條摘《孟子》七篇大義,附以時政得失,累數千言,上之。帝稱善。閣臣徐溥等以詔敕繁,請如先朝王直故事,設官專領。乃擢東陽禮部右侍郎兼侍讀學士,入內閣專典誥敕。八年以本官直文淵閣參預機務,與謝遷同日登用。久之,進太子少保、禮部尚書兼文淵閣大學士。

十七年,重建闕里廟成,奉命往祭。還,上疏言:

臣奉使遄行,適遇亢旱。天津一路,夏麥已枯,秋禾未種,挽舟者無完衣,荷鋤者有菜色。盜賊縱橫,青州尤甚。南來人言:江南、浙東流亡載道,戶口消耗,軍伍空虛,庫無旬日之儲,官缺累歲之俸。東南,財賦所出,一歲之饑已至於此;北地呰窳,素無積聚,今秋再歉,何以堪之。事變之生,恐不可測。臣自非經過其地,則雖久處官曹,日理章疏,猶不得其詳,況陛下高居九重之上耶?

臣訪之道路,皆言冗食太眾,國用無經。差役頻煩,科派重疊。京城土木繁興,供役軍士財力交殫,每遇班操,寧死不赴。勢家巨族,田連郡縣,猶請乞不已。親王之籓,供億至二三十萬。游手之徒,託名皇親僕從,每於關津都會大張市肆,網羅商稅。國家建都於北,仰給東南,商賈驚散,大非細故。更有織造內官,縱群小掊擊,閘河官吏莫不奔駭,鬻販窮民所在騷然,此又臣所目擊者。

夫閭閻之情,郡縣不得而知也;郡縣之情,廟堂不得而知也;廟堂之情,九重亦不得而知也;始於容隱,成於蒙蔽。容隱之端甚小,蒙蔽之禍甚深。臣在山東,伏聞陛下以災異屢見,敕群臣盡言無諱。然詔旨頻降,章疏畢陳,而事關內廷、貴戚者,動為制肘,累歲經時,俱見遏罷。誠恐今日所言,又為虛文。乞取從前內外條奏,詳加採擇,斷在必行。

帝嘉歎,悉付所司。

是時,帝數召閣臣面議政事。東陽與首輔劉健等竭心獻納,時政闕失必盡言極諫。東陽工古文,閣中疏草多屬之。疏出,天下傳誦。明年,與劉健、謝遷同受顧命。

武宗立,屢加少傅兼太子太傅。劉瑾入司禮,東陽與健、遷即日辭位。中旨去健、遷,而東陽獨留。恥之,再疏懇請,不許。初,健、遷持議欲誅瑾,詞甚厲,惟東陽少緩,故獨留。健、遷瀕行,東陽祖餞泣下。健正色曰:「何泣為?使當日力爭,與我輩同去矣。」東陽默然。

瑾既得志,務摧抑縉紳。而焦芳入閣助之虐,老臣、忠直士放逐殆盡。東陽悒悒不得志,亦委蛇避禍。而焦芳嫉其位己上,日夕構之瑾。先是,東陽奉命編《通鑑纂要》。既成,瑾令人摘筆畫小疵,除謄錄官數人名,欲因以及東陽。東陽大窘,屬芳與張彩為解,乃已。

瑾兇暴日甚,無所不訕侮,於東陽猶陽禮敬。凡瑾所為亂政,東陽彌縫其間,亦多所補救。尚寶卿崔璿、副使姚祥、郎中張瑋以違制乘肩輿,從者妄索驛馬,給事中安奎、御史張彧以核邊餉失瑾意,皆荷重校幾死。東陽力救,璿等謫戍,奎、彧釋為民。

三年六月壬辰,朝退,有遺匿名書於御道數瑾罪者,詔百官悉跪奉天門外。頃之,執庶僚三百餘人下詔獄。次日,東陽等力救,會瑾亦廉知其同類所為,眾獲宥。後數日,東陽疏言寬恤數事,章下所司。既而戶部覆奏,言糧草虧折,自有專司,巡撫官總領大綱,宜從輕減。瑾大怒,矯旨詰責數百言,中外駭歎。瑾患盜賊日滋,欲戍其家屬並鄰里及為之囊橐者。或自陳獲盜七十人,所司欲以新例從事。東陽言,如是則百年之案皆可追論也,乃免。劉健、謝遷、劉大夏、楊一清及平江伯陳熊輩幾得危禍,皆賴東陽而解。其潛移默奪,保全善類,天下陰受其庇。而氣節之士多非之。侍郎羅上書勸其早退,至請削門生籍。東陽得書,俯首長歎而已。

焦芳既與中人為一,王鏊雖持正,不能與瑾抗,東陽乃援楊廷和共事,差倚以自強。已而鏊辭位,代者劉宇、曹元皆瑾黨,東陽勢益孤。東陽前已加少師兼太子太師,後瑾欲加芳官,詔東陽食正一品祿。四年五月,《孝宗實錄》成,編纂諸臣當序遷,所司援《會典》故事。詔以劉健等前纂修《會典》多糜費,皆奪升職,東陽亦坐降俸。居數日,乃以《實錄》功復之。

五年春,久旱,下詔恤刑。東陽等因上詔書所未及者數條,帝悉從之。而法司畏瑾,減死者止二人。其秋,瑾誅,東陽乃上疏自列曰:「臣備員禁近,與瑾職掌相關。凡調旨撰敕,或被駁再三,或徑自改竄,或持回私室,假手他人,或遞出謄黃,逼令落橐,真假混淆,無從別白。臣雖委曲匡持,期於少濟,而因循隱忍,所損亦多。理宜黜罷。」帝慰留之。

寘鐇平,加特進左柱國,廕一子尚寶司丞,為御史張芹所劾。帝怒,奪芹俸。東陽亦乞休辭蔭,不許。時焦芳、曹元已罷,而劉忠、梁儲入,政事一新。然張永、魏彬、馬永成、谷大用等猶用事,帝嬉遊如故。皇子未生,多居宿於外。又議大興豹房之役,建寺觀禁中。東陽等憂之,前後上章切諫,不報。七年,東陽等以京師及山西、陜西、雲南、福建相繼地震,而帝講筵不舉,視朝久曠,宗社祭享不親,禁門出入無度,谷大用仍開西廠,屢上疏極諫,帝亦終不聽。

九載秩滿,兼支大學士俸。河南賊平,廕子世錦衣千戶。再疏力辭,改蔭六品文官。其冬,帝欲調宣府軍三千入衛,而以京軍更番戍邊。東陽等力持不可,大臣、臺諫皆以為言。中官旁午索草敕,帝坐乾清宮門趣之,東陽等終不奉詔。明日竟出內降行之,江彬等遂以邊兵入豹房矣。東陽以老疾乞休,前後章數上,至是始許。賜敕、給廩隸如故事。又四年卒,年七十。贈太師,謚文正。

東陽事父淳有孝行。初官翰林時,常飲酒至夜深,父不就寢,忍寒待其歸,自此終身不夜飲於外。為文典雅流麗,朝廷大著作多出其手。工篆隸書,碑版篇翰流播四裔。獎成後進,推挽才彥,學士大夫出其門者,悉粲然有所成就。自明興以來,宰臣以文章領袖縉紳者,楊士奇後,東陽而已。立朝五十年,清節不渝。既罷政居家,請詩文書篆者填塞戶限,頗資以給朝夕。一日,夫人方進紙墨,東陽有倦色。夫人笑曰:「今日設客,可使案無魚菜耶?」乃欣然命筆,移時而罷,其風操如此。

王鏊,字濟之,吳人。父琬,光化知縣。鏊年十六,隨父讀書,國子監諸生爭傳誦其文。侍郎葉盛、提學御史陳選奇之,稱為天下士。成化十年鄉試,明年會試,俱第一。廷試第三,授編修。杜門讀書,避遠權勢。

弘治初,遷侍講學士,充講官。中貴李廣導帝遊西苑,鏊講文王不敢盤於遊田,反復規切,帝為動容。講罷,謂廣曰:「講官指若曹耳。」壽寧侯張巒故與鏊有連,及巒貴,鏊絕不與通。東宮出閣,大臣請選正人為宮僚,鏊以本官兼諭德。尋轉少詹事,擢吏部右侍郎。

嘗奏陳邊計,略言:「昨火篩入寇大同,陛下宵旰不寧,而緣邊諸將皆嬰城守,無一人敢當其鋒者,此臣所不解也。臣竊謂今日火篩、小王子不足畏,而嬖倖亂政,功罪不明,委任不專,法令不行,邊圉空虛,深可畏也。比年邊將失律,往往令戴罪殺賊。副總兵姚信擁兵不進,亦得逃罪。此人心所以日懈,士氣所以不振也。望陛下大奮乾剛,時召大臣,咨詢邊將勇怯。有罪必罰,有功必賞,專主將之權。起致仕尚書秦紘為總制,節制諸邊,提督右都御史史琳坐鎮京營,遙為聲援。厚恤沿邊死事之家,召募邊方驍勇之士,用間以攜其部曲。分兵掩擊,出奇制勝,寇必不敢長驅深入。」從之。又言:「宜仿前代制科,如博學宏詞之類,以收異材。六年一舉,尤異者授以清要之職,有官者加秩。數年之後,士類濯磨,必以通經學古為高,脫去謏聞之陋。」時不能用。尋以父憂歸。

正德元年四月起左侍郎,與韓文諸大臣請誅劉瑾等「八黨」。俄瑾入司禮,大學士劉健、謝遷相繼去,內閣止李東陽一人。瑾欲引焦芳,廷議獨推鏊。瑾迫公論,命以本官兼學士與芳同入內閣。踰月,進戶部尚書文淵閣大學士。明年加少傅兼太子太傅。

景帝汪后薨,疑其禮。鏊曰:「妃廢不以罪,宜復故號,葬以妃,祭以后。」乃命輟朝,致祭如制。憲宗廢后吳氏之喪,瑾議欲焚之以滅迹,曰「不可以成服」。鏊曰:「服可以不成,葬不可薄也。」從之。尚寶卿崔璿等三人荷校幾死。鏊謂瑾曰:「士可殺,不可辱。今辱且殺之,吾尚何顏居此。」李東陽亦力救,璿等得遣戍。瑾銜尚書韓文,必欲殺之,又欲以他事中健、遷,鏊前後力救得免。或惡楊一清於瑾,謂築邊牆糜費。鏊爭曰:「一清為國修邊,安得以功為罪。」瑾怒劉大夏,逮至京,欲坐以激變罪死。鏊爭曰:「岑猛但遷延不行耳,未叛何名激變?」時中外大權悉歸瑾,鏊初開誠與言,間聽納。而芳專媕阿,瑾橫彌甚,禍流縉神。鏊不能救,力求去。四年,疏三上,許之。賜璽書、乘傳、有司給廩隸,咸如故事。家居十四年,廷臣交薦不起。

世宗即位,遣行人存問。鏊疏謝,因上講學、親政二篇。帝優詔報聞,官一子中書舍人。嘉靖三年復詔有司存問。未幾卒,年七十五。贈太傅,謚文恪。

鏊博學有識鑒,文章爾雅,議論明暢。晚著《性善論》一篇,王守仁見之曰:「王公深造,世未能盡也。」少善制舉義,後數典鄉試,程文魁一代。取士尚經術,險詭者一切屏去。弘、正間,文體為一變。

劉忠,字司直,陳留人。成化十四年進士。改庶吉士,授編修。弘治四年,《憲宗實錄》成,遷侍講,直經筵,尋兼侍東宮講讀。又九年進侍讀學士。

武宗即位,以宮寮擢學士,掌翰林院,仍直經筵。正德二年,劉瑾用事,日導帝遊戲,亂祖宗舊章。忠上言戒逸遊、崇正學數事。已,因進講與楊廷和傅經義,規帝闕失,而指斥近倖尤切。帝謂瑾曰:「經筵,講書耳,浮詞何為?」瑾素惡兩人,因諷吏部尚書許進出之南京。南京諸部惟右侍郎一人,進特請用為禮部左侍郎。命下,外議籍籍,進患之,甫兩月,即擢忠本部尚書。其冬,就改吏部。時留都一御史,素驕橫;一郎中,張彩所暱也,秩滿,皆署下考。疾吏胥詭名寄籍,督諸曹核汰千人。大計京官,所黜多於前。又疏請不時糾劾,以示勸懲,無待六年考黜。詔可之。忠在南京正直有風采。然是時,瑾方以嚴苛折辱士大夫,而忠操繩墨待下,糾劾過峻。時論遂謂忠附會瑾意,頗歸怨焉。

五年二月改吏部尚書兼翰林學士,專典制詔。兩疏乞休,不報。瑾誅,以本官兼文淵閣大學士,入閣預機務。甫數日,以平寧夏功,加少傅兼太子太傅。故事,閣臣加官無遽至三孤者。忠無功驟得,不自安,連疏固辭,不許。瑾雖誅,張永、魏彬輩擅政,大臣復爭與交歡,忠獨無所顧。永嘗遣廖鵬謁忠,忠僕隸遇之,又卻其餽,由是與永輩左。前後乞休疏七八上,皆慰留。明年命典會試。甫畢,帝以試錄文義多舛,召李東陽示之。忠知為中官所掎,乞省墓。詔乘傳還。抵家,再上章乞致仕,報許。給月廩、歲隸終其身。

世宗即位,屢薦不起。遣行人存問,忠奏謝,因有所獻納,帝褒其忠愛。嘉靖二年卒,年七十二。贈太保,謚文肅。

贊曰:徐溥以寬厚著,邱濬以博綜聞。觀其指事陳言,懇懇焉為憂盛危明之計,可謂勤矣。劉健、謝遷正色直道,蹇蹇匪躬。閹豎亂政,秉義固諍。志雖不就,而剛嚴之節始終不渝。有明賢宰輔,自三楊外,前有彭、商,後稱劉、謝,庶乎以道事君者歟。李東陽以依違蒙詬,然善類賴以扶持,所全不少。大臣同國休戚,非可以決去為高,遠蹈為潔,顧其志何如耳。王鏊、劉忠持正不阿,奉身早退。此誠明去就之節,烏能委蛇俯仰以為容悅哉。


\end{pinyinscope}