\article{列傳第六十五}

\begin{pinyinscope}
○王翱年富王竑李秉姚夔王復林聰葉盛

王翱,字九皋,鹽山人。永樂十三年,初會試貢士於行在。帝時欲定都北京,思得北士用之。翱兩試皆上第,大喜,特召賜食。改庶吉士,授大理寺左寺正,左遷行人。

宣德元年,以楊士奇薦,擢御史,時官吏有罪,不問重輕,許運磚還職。翱請犯贓吏但許贖罪,不得復官,以懲貪黷。帝從之。五年巡按四川。松潘蠻竊發,都督陳懷駐成都,相去八百餘里,不能制。翱上便宜五事:請移懷松籓;而松茂軍糧於農隙齊力起運,護以官軍,毋專累百姓,致被劫掠。吏不給由為民蠹,令自首毋隱;州縣土司遍設社學,會川銀場歲運米八千餘石給軍,往返勞費,請令有罪者納粟自贖。詔所司議詳運糧事,而遷蠹吏北京,餘悉允行。

英宗即位,廷議遣文武大臣出鎮守。擢翱右僉都御史,偕都督武興鎮江西,懲貪抑奸,吏民畏愛。正統二年召還院。四年,處州賊流劫廣信,命翱往捕,盡俘以還。是年冬,松潘都指揮趙諒誘執國師商巴,掠其財,與同官趙得誣以叛。其弟小商巴怒,聚眾剽掠。命翱及都督李安軍二萬征之。而巡按御史白其枉,詔審機進止。翱至,出商巴於獄,遣人招其弟,撫定餘黨,而劾誅諒,戍得,復商巴國師。松潘遂平。六年代陳鎰鎮陜西,軍民之借糧不能償者,核免之。

七年冬,提督遼東軍務。翱以軍令久馳,寇至,將士不力戰,因諸將庭謁,責以失律罪,命左右曳出斬之。皆惶恐叩頭,願效死贖。翱乃躬行邊,起山海關抵開原,繕城垣,浚溝塹。五里為堡,十里為屯,使烽燧相接。練將士,室鰥寡。軍民大悅。又以邊塞孤遠,軍餉匱,緣俗立法,令有罪得收贖。十餘年間,得穀及牛羊數十萬,邊用以饒。

八年以九載滿,進右副都御史。指揮孫璟鞭殺戍卒,其妻女哭之亦死。他卒訴璟殺一家三人。翱曰:「卒死法,妻死夫,女死父,非殺也。」命璟償其家葬薶費,璟感激。後參將遼東,追敵三百里,事李秉為名將。

十二年與總兵曹義等出塞,擊兀良哈,擒斬百餘人,獲畜產四千六百,進右都御史。十四年,諸將破敵廣平山,進左。脫脫不花大舉犯廣寧,翱方閱兵,寇猝至,眾潰。翱入城自保。或謂城不可守,翱手劍曰:「敢言棄城者斬。」寇退,坐停俸半載。

景泰三年,召還掌院事。易儲,加太子太保。潯、梧瑤亂,總兵董興、武毅推委不任事,于謙請以翁信、陳旺易之,而特遣一大臣督軍務,乃以命翱。兩廣有總督自翱始。翱至鎮,將吏讋服,推誠撫諭,瑤人向化,部內無事。明年召入為吏部尚書。初,何文淵協王直掌銓,多私,為言官攻去。翱代,一循成憲。

天順改元,直致仕,翱始專部事。石亨欲去翱,翱乞休。已得請,李賢力爭乃留。及賢為亨所逐,亦以翱言留,兩人相得歡甚。帝每用人必咨賢,賢以推翱,以是翱得行其志。

帝眷翱厚,時召對便殿,稱「先生」不名。而翱年幾八十,多忘,嘗令郎談倫隨入。帝問故,翱頓首曰:「臣老矣,所聆聖諭,恐遺誤,令此郎代識之,其人誠謹可信也。」帝喜,吏部主事曹恂已遷江西參議,遇疾還。翱以聞,命以主事回籍。恂怒,伺翱入朝,捽翱胸,摑其面,大聲詬詈。事聞,下詔獄。翱具言恂實病,得斥歸,時服其量。

五年加太子少保。成化元年進太子太保,雨雪免朝參。屢疏乞歸,輒慰留,數遣醫視疾。三年,疾甚,乃許致仕。未出都卒,年八十有四。贈太保,謚忠肅。

翱在銓部,謝絕請謁,公餘恆宿直廬,非歲時朔望謁先祠,未嘗歸私第。每引選,或值召對,侍郎代選。歸雖暮,必至署閱所選,惟恐有不當也。論薦不使人知,曰:「吏部豈快恩怨地耶。」自奉儉素。景帝知其貧,為治第鹽山。孫以廕入太學,不使應舉,曰:「勿妨寒士路。」婿賈傑官近畿,翱夫人數迎女,傑恚曰:「若翁典銓,移我官京師,反手爾。何往來不憚煩也!」夫人聞之,乘間請翱。翱怒,推案,擊夫人傷面。傑卒不得調。其自遼東還朝也,中官同事者重翱,贐明珠數顆,翱固辭。其人曰:「此先朝賜也,公得毋以贓卻我乎。」不得已,納而藏焉。中官死,召其從子還之。為都御史時,夫人為娶一妾,逾半歲語翱。翱怒曰:「汝何破我家法!」即日具金幣返之。妾終不嫁,曰:「豈有大臣妾嫁他人者?」翱卒,妾往奔喪,其子養之終身。李賢嘗語人曰:「皋陶言九德,王公有其五:亂而敬,擾而毅,簡而廉,剛而塞,強而義也。」然性頗執。嘗有詔舉賢良方正、經明行修及山林隱逸士。至者率下部試,翱黜落,百不取一二。性不喜南士。英宗嘗言:「北人文雅不及南人,顧質直雄偉,緩急當得力。」翱由是益多引北人。晚年徇中官郭聰囑,為都御史李秉所劾,翱自引伏,蓋不無小損云。子孫世官錦衣千戶。

年富,字大有,懷遠人。本姓嚴,訛為年。以會試副榜授德平訓導。年甫踰冠,嚴重如老儒。宣德三年課最,擢吏科給事中。糾正違失,務存大體。帝以六科任重,命科擇二人掌其事,乃以富與賈銓並掌刑科。都御史顧佐等失入死罪十七人,富劾之。帝詰責佐等。

英宗嗣位,上言:「永樂中,招納降人,縻以官爵,坐耗國帑,養亂招危,宜遣還故土。府軍前衛幼軍,本選民間子弟,隨侍青宮。今死亡殘疾,僉補為擾。請於二十五所內,以一所補調,勿更累民。軍民之家,規免稅徭,冒僧道者累萬,宜悉遣未度者復業。」議多施行。

遷陜西左參政,尋命總理糧儲。陜西歲織綾絹毼九百餘匹。永樂中,加織駝毼五十匹,富請罷之。官吏諸生衛卒祿廩,率以邊餉減削,富請復其舊。諸邊將校占墾腴田有至三四十頃者,富奏每頃輸賦十二石。都督王禎以為過重,疏爭之。廷議減三之二,遂為定額。又會計歲用,以籌軍餉,言:「臣所部歲收二稅百八十九萬石,屯糧七十餘萬石。其間水旱流移,蠲逋負,大率三分減一,而歲用乃至百八十餘萬,入少出多。今鎮守諸臣不量國計,競請益兵,餉何由給?請減冗卒,汰駑馬,杜侵耗之弊。」帝可其奏。三邊士馬,供億浩繁,軍民疲遠輸,豪猾因緣為奸利。富量遠近,定徵科,出入慎鉤考,宿弊以革,民困大蘇。富遇事,果敢有為,權勢莫能撓,聲震關中。然執法過嚴,僥倖者多不悅,以是屢遭誣謗。陜西文武將吏恐失富,咸上章陳其勞,乃得停俸留任。

九載滿,遷河南右布政使。復有言富苛虐者,帝命核舉主,將坐之。既知舉富者,少師楊溥也,意乃解。富至河南,歲饑,流民二十餘萬,公剽掠。巡撫于謙委富輯之,皆定。土木敗後,邊境道阻,部檄富轉饟,無後期者,進左。

景泰二年春,以右副都御史巡撫大同,提督軍務。時經喪敗,法弛,弊尤甚。富一意拊循,奏免秋賦,罷諸州縣稅課局,停太原民轉餉大同。武清侯石亨、武安侯鄭宏、武進伯朱瑛,令家人領官庫銀帛,糴米實邊,多所乾沒。富首請按治。詔宥亨等,抵家人罪。亨所遣卒越關抵大同,富復劾亨專擅。亨輸罪。已,削襄垣王府菜戶,又杖其廚役之署教授事者。又劾分守中官韋力轉、參將石彪及山西參政林厚罪。是時,富威名重天下,而諸豪家愈側目,相與摭富罪。于謙方當事,力保持之。帝亦知富深,故得行其志。林厚力詆富,帝曰:「厚怨富、誣富耳。朕方付富邊事。豈輕聽人言加辱耶。」削厚官。

六年,母憂,起復。七年,富上言:「諸邊鎮守監槍內官增於前,如陽和、天城,一城二人,擾民殊甚,請減汰。」事格不行。又言:「高皇帝定制,軍官私罪收贖,惟笞則然。杖即降授,徒流俱充軍,律明甚。近犯贓者,輕皆復職,重惟立功。刑不足懲,更無顧憚。此皆法官過也。」下廷議,流徒輸贖如故,惟於本衛差操,不得領軍。英國公張懋及鄭宏各置田莊於邊境,歲役軍耕種,富劾之,還軍於伍。

天順元年革巡撫官,富亦罷歸。頃之,石彪以前憾劾富,逮下詔獄。帝問李賢,賢稱富能祛弊。帝曰:「此必彪為富抑,不得逞其私耳。」賢曰:「誠如聖諭,宜早雪之。」諭門達從公問事。果無驗,乃令致仕。

明年,以廷臣薦,起南京兵部右侍郎,未上,改戶部,巡撫山東。道聞屬邑蝗,馳疏以聞。改左副都御史,巡撫如故。官吏習富威名,望之讋服,豪猾屏跡。

四年春,戶部缺尚書,李賢舉富。左右巧阻之。帝語賢曰:「戶部非富不可,人多不喜富,此富所以為賢也。」特召任之。富酌贏縮,謹出納,躬親會計,吏不能欺。事關利害者,僚屬或不敢任,富曰:「第行之,吾當其責,諸君毋署名可也。」由是部事大理。丁父憂,奪哀如初。

憲宗立,富以陜西頻用兵,而治餉者非人,請黜左布政孫毓,用右布政楊璿、參政婁良、西安知府餘子俊。吏部尚書王翱論富侵官,請下於理。富力辯曰:「薦賢為國,非有所私也。」因乞骸骨。帝慰留之,為黜毓。頃之,病疽卒。賜謚恭定。

富廉正強直,始終不渝,與王翱同稱名臣。初,英宗嘗諭李賢曰:「戶部如年富不易得。」賢對曰:「若他日繼翱為吏部,非富不可。」然性好疑,尤惡干請。屬吏黠者,故反其意嘗之。欲事行,故言不可,即不行,故言可。富輒為所賣。

王竑,字公度,其先江夏人。祖俊卿,坐事戍河州,遂著籍。竑登正統四年進士。十一年授戶科給事中,豪邁負氣節,正色敢言。

英宗北狩,郕王攝朝午門,群臣劾王振誤國罪。讀彈文未起,王使出待命。眾皆伏地哭,請族振。錦衣指揮馬順者,振黨也,厲聲叱言者去。竑憤怒,奮臂起,捽順發呼曰:「若曹奸黨,罪當誅,今尚敢爾!」且罵且嚙其面,眾共擊之,立斃。朝班大亂。王恐,遽起入,竑率群臣隨王後。王使中官金英問所欲言,曰:「內官毛貴、王長隨亦振黨,請置諸法。」王命出二人。眾又捶殺之,血漬廷陛。當是時,竑名震天下,王亦以是深重竑。且召諸言官,慰諭甚至。

王即帝位,也先犯京師,命竑與王通、楊善守禦京城,擢右僉都御史,督毛福壽、高禮軍。寇退,詔偕都指揮夏忠等鎮守居庸。竑至,簡士馬,繕阨塞,劾將帥不職者,壁壘一新。

景泰元年四月,浙江鎮守中官李德上言:「馬順等有罪,當請命行誅。諸臣乃敢擅殺。非有內官擁護,危矣。是皆犯闕賊臣。不宜用。」章下廷議。于謙等奏曰:「上皇蒙塵,禍由賊振。順等實振腹心。陛下監國,群臣共請行戮,而順猶敢呵叱。是以在廷文武及宿衛軍士忠憤激發,不暇顧忌,捶死三人。此正《春秋》誅亂賊之大義也。向使乘輿播遷,奸黨猶在,國之安危殆未可知。臣等以為不足問。」帝曰:「誅亂臣,所以安眾志。廷臣忠義,朕已知之,卿等勿以德言介意。」八月,竑以疾還朝。尋命同都督僉事徐恭督漕運,治通州至徐州運河。明年,尚寶司檢順牙牌不得,順子請責之竑,帝許焉。諸諫官言:「順黨奸罪重,廷臣共除之,遑問牙牌。且非竑一人事,若責之竑,忠臣懼矣。」乃寢前旨。是年冬,耿九疇召還,敕竑兼巡撫淮、揚、廬三府,徐、和二州,又命兼理兩淮鹽課。

四年正月,以災傷疊見,方春盛寒,上言:「請敕責諸臣痛自修省,省刑薄斂,罷無益之工,嚴無功之賞,散財以收民心,愛民以植邦本。陛下益近親儒臣,講道論德,進君子,退小人,以回天意。」且引罪乞罷。帝納其言,遂下詔修省,求直言。

先是,鳳陽、淮安、徐州大水,道殣相望。竑上疏奏,不待報,開倉振之。至是山東、河南饑民就食者坌至,廩不能給。惟徐州廣運倉有餘積,竑欲盡發之,典守中官不可。竑往告曰:「民旦夕且為盜。若不吾從,脫有變,當先斬若,然後自請死耳。」中官憚竑威名,不得已從之。竑乃自劾專擅罪,因言「廣運所儲僅支三月,請令死罪以下,得於被災所入粟自贖。」帝復命侍郎鄒乾齎帑金馳赴,聽便宜。竑乃躬自巡行散振,不足,則令沿淮上下商舟,量大小出米。全活百八十五萬餘人。勸富民出米二十五萬餘石,給饑民五十五萬七千家。賦牛種七萬四千餘,復業者五千五百家,他境流移安輯者萬六百餘家。病者給藥,死者具槥,所鬻子女贖還之,歸者予道里費。人忘其饑,頌聲大作。初,帝聞淮、鳳饑,憂甚。及得竑發廣運倉自劾疏,喜曰:「賢哉都御史!活我民矣。」尚書金濂、大學士陳循等皆稱竑功。是年十月,就進左副都御史。時濟寧亦饑,帝遣尚書沈翼齎帑金三萬兩往振。翼散給僅五千兩,餘以歸京庫。竑劾翼奉使無狀,請仍易米備振,從之。

明年二月上言:「比年饑饉薦臻,人民重困。頃冬春之交,雪深數尺,淮河抵海冰凍四十餘里,人畜僵死萬餘,弱者鬻妻子,強者肆劫奪,衣食路絕,流離載途。陛下端居九重,大臣安處廊廟,無由得見。使目擊其狀,未有不為之流涕者也。陛下嗣位以來,非不敬天愛民,而天變民窮特甚者,臣竊恐聖德雖修而未至,大倫雖正而未篤,賢才雖用而未收其效,邪佞雖屏而未盡其類,仁愛施而實惠未溥,財用省而上供未節,刑罰寬而冤獄未伸,工役停而匠力未息,法制頒而奉行或有更張,賦稅免而有司或仍牽制。有一於此,皆足以干和召變。伏望陛下修厥德以新厥治。欽天命,法祖宗,正倫理,篤恩義,戒逸樂,絕異端,斯修德有其誠矣。進忠良,遠邪佞,公賞罰,寬賦役,節財用,戒聚斂,卻貢獻,罷工役,斯圖治有其實矣。如是而災變不息,未之有也。」帝褒納之,敕內外臣工同加修省。

六年,霍山民趙玉山自稱宋裔,以妖術惑眾為亂,竑捕獲之。先後劾治貪濁吏,革糧長之蠹民者,民大稱便。

英宗復辟,革巡撫官,改竑浙江參政。數日,石亨、張軏追論竑擊馬順事,除名,編管江夏。居半歲,帝於宮中得竑疏,見「正倫理,篤恩義」語,感悟。命遣官送歸田里,敕有司善視之。

天順五年,孛來寇莊浪,都督馮宗等出討。用李賢薦,起竑故官,與兵部侍郎白圭參贊軍務。明年正月,竑與宗擊退孛來於紅崖子川。圭等還,竑仍留鎮。至冬,乃召還。明年春,復令督漕撫淮、揚。淮人聞竑再至,歡呼迎拜,數百里不絕。

憲宗即位,給事中蕭斌、御史呂洪等,共薦竑及宣府巡撫李秉堪大用。下廷議,尚書王翱、大學士李賢請從其言。帝曰:「古人君夢卜求賢,今獨不能從輿論所與乎?」即召竑為兵部尚書,秉為左都御史。命下,朝野相慶。

時將用兵兩廣,竑舉韓雍為總督。雍新得罪,眾難之。竑曰:「天子方棄瑕錄用,雍有罪不當用,竑非罪廢者耶?」卒用雍。竑條上進剿事宜,且言將帥征討,毋得奏攜私人,妄冒首功。又請復京營舊額,禁勢家豪帥擅役禁軍。於是命竑同給事中、御史六人簡閱十二營軍士。竑以擇兵不若擇將,共奏罷營職八十餘人,而慎簡材武補之。

兵部清理貼黃缺官,竑偕諸大臣舉修撰岳正、都給事中張寧,為李賢所沮,竟出二人於外,並罷會舉例。竑憤然曰:「吾尚可居此耶?」即引疾求退。帝方向用竑,優詔慰留,日遣醫視疾。竑請益切。九月命致仕去。竑為尚書一年,謝病者四月,人以未竟其用為惜。既去,中外薦章百十上,並報寢。

初,竑號其室曰「戇庵。」既歸,改曰「休庵。」杜門謝客,鄉人希得見。時李秉亦罷歸,日出入里閈,與故舊談笑游燕。竑聞之曰:「大臣何可不養重自愛?」秉聞之,亦笑曰:「所謂大臣,豈以立異鄉曲、尚矯激為賢哉。」時兩稱之。竑居家二十年,弘治元年十二月卒,年七十五。正德間,贈太子少保,益莊毅。淮人立祠祀之。

李秉,字執中,曹縣人。少孤力學,舉正統元年進士,授延平推官。沙縣豪誣良民為盜而淫其室,秉捕治豪。豪誣秉,坐下獄。副使侯軏直之,論豪如法,由是知名。徵入都察院理刑,將授御史,都御史王文薦為本院經歷,尋改戶部主事。宣府屯田為豪占,秉往視,歸田於民,而請罷科索,邊人賴之。兩淮鹽課弊覺,逮數百人。秉往核,搜得偽印,逮者以白。

景帝立,進郎中。景泰二年命佐侍郎劉璉督餉宣府,發璉侵牟狀。即擢右僉都御史代璉,兼參贊軍務。宣府軍民數遭寇,牛具悉被掠。朝廷遣官市牛萬五千給屯卒。人予直,市穀種。璉盡以畀京軍之出守者,一不及屯卒,更停其月餉,而徵屯糧甚急。秉盡反璉政,厚恤之。軍卒自城守外,悉得屯作。凡使者往來及宦官鎮守供億科斂者,皆奏罷,以官錢給費。尋上邊備六事,言:「軍以有妻者為有家,月餉一石,無者減其四。即有父母兄弟而無妻,概以無家論,非義。當一體增給。」從之。時宣府億萬庫頗充裕,秉益召商中鹽納糧,料飭戎裝,市耕牛給軍,軍愈感悅。

三年冬命兼理巡撫事。頃之,又命提督軍務。秉盡心邊計,不恤嫌怨。劾都指揮楊文、楊鑒,都督江福貪縱,罪之。論守獨石內官弓勝田獵擾民,請徵還。又劾總兵官紀廣等罪,廣訐秉自解。帝召秉還,以言官交請,乃命御史練綱、給事中嚴誠往勘,卒留秉。時邊民多流移,秉廣行招徠,復業者奏給月廩。瘞土木、鷂兒嶺暴骸,乞推行諸塞。軍家為寇所殺掠無依者,官為養贍,或資遣還鄉。釐諸弊政,所條奏百十章,多允行。諜報寇牧近邊,廷議遣楊俊會宣府兵出剿。秉曰:「塞外原諸部牧地,非犯邊也。掩殺倖功,非臣所敢聞。」乃止。諸部質所掠男婦求易米,朝議成丁者予一石,幼者半之。諸部概乞一石,鎮將不可。秉曰:「是輕人重粟也。」如其言予之。自請專擅罪,帝以為識體。

天順初,罷巡撫官,改督江南糧儲。初,江南蘇、松賦額不均。陳泰為巡撫,令民田五升者倍徵,官田重者無增耗,賦均而額不虧。秉至,一守其法。尋坐舉知府違例被逮,帝以秉過微,宥之。復任,請滸墅關稅悉徵米備荒。又發內官金保監淮安倉科索罪。

御史李周等左遷,秉疏救。帝怒,將罪之。會廷議復設巡撫,大臣薦秉才,遂命巡撫大同。都指揮孫英先以罪貶職還衛,總兵李文妄引詔書,令復職。秉至,即斥之。裨將徐旺領騎卒操練,秉以旺不勝任,解其官。未幾,天城守備中官陳例久病,秉請易以羅付。帝責秉專擅,徵下詔獄。指揮門達並以前舉知府、救御史及斥孫英等為秉罪。法司希旨,斥為民。居三年,用閣臣薦,起故官,蒞南京都察院。憲宗立,進右副都御史,復撫宣府。數月,召拜左都御史。

成化改元,掌大計,黜罷貪殘,倍於其舊。明年秋,命整飭遼東抵大同邊備。至即劾鎮守中官李良、總兵武安侯鄭宏失律罪,出都指揮裴顯於獄,舉指揮崔勝、傅海等,擊敵鳳皇山。捷聞,璽書嘉勞。秉乃往巡視宣府、大同,更將帥,申軍令而還。未幾,命為總督,與武清伯趙輔分五道出塞,大捷。帝勞以羊酒,賜麒麟服,加太子少保。

三年冬,吏部尚書王翱致仕,廷推代者,帝特擢秉任之。秉銳意澄仕路。監生需次八千餘人,請分別考核。黜庸劣者數百人,於是怨謗紛起。左侍郎崔恭以久次當得尚書,而秉得之,頗不平。右侍郎尹旻嘗學於秉,秉初用其言,既而疏之。侍讀彭華附中貴,數以私干秉,秉不聽。胥怨秉。御史戴用請兩京堂上官及方面正佐,如正統間例,會廷臣保舉;又吏部司屬與各部均升調,不得久擅要地,且驟遷。語侵吏部,吏部持之。帝令兩京官四品以上,吏部具缺,取上裁。而御史劉璧、吳遠、馮徽爭請仍歸吏部。帝怒,詰責言者。會朝覲考察,秉斥退者眾,又多大臣鄉故,眾怨交集。而大理卿王概亦欲去秉代其位,乃與華謀,嗾同鄉給事中蕭彥莊劾秉十二罪,且言其陰結年深御史附己以攬權。帝怒,下廷議。恭、旻輒言「吾兩人諫之不聽」,刑部尚書陸瑜等附會二人意為奏。帝以秉徇私變法,負任使,落秉太子少保致仁。所連鮑克寬、李沖調外任;丘陵、張穆、陳民弼、孫遇、李齡、柳春皆罷。命彥莊指秉所結御史,不能對。久之,以璧等三人名上,遂俱下詔獄,出之外。陵等實良吏,有名,以讒黜,眾議不平。陵尤不服,連章訐彥莊。廷訊,陵詞直。帝惡彥莊誣罔。謫大寧驛丞。

方秉之被劾也,勢洶洶,且逮秉。秉謂人曰:「為我謝彭先生,秉罪惟上所命。第毋令入獄,入則秉必不出,恐傷國體。」因具疏引咎,略不自辨。時天下舉子方會試集都下,奮罵曰:「李公天下正人,為奸邪所誣。若罪李公,願罷我輩試以贖。」及帝薄責秉,乃已。秉行,官屬餞送,皆欷歔,有泣下者。秉慷慨揖諸人,登車而去。秉去,恭遂為尚書。

秉誠心直道。夷險一節,與王竑並負重望。家居二十年,中外薦疏十餘上,竟不起。弘治二年卒。贈太子太保。後謚襄敏。

子聰、明、智,孫邦直,皆舉鄉試。聰,南宮知縣,以彥莊劾罷歸。明,建寧府同知。智,南陽府知府。邦直,寧波府同知,彥莊謫後,署大寧縣,以科斂為盜所殺。

姚夔,字大章,桐廬人。孝子伯華孫也。正統七年進士,鄉、會試皆第一。明年授吏科給事中,陳時政八事。又言:「預備倉本振貧民。而里甲慮貧者不能償,輒隱不報。致稱貸富室,倍稱還之。收獲甫畢,遽至乏絕。是貧民遇凶年饑,豐年亦饑也。乞敕天下有司。歲再發廩,必躬勘察,先給其最貧者。」帝立命行之。

景帝監國,諸大臣議勸即位,未決。以問諸言官,夔曰:「朝廷任大臣,正為社稷計,何紛紛為?」議遂定。也先薄京城,請急徵宣府、遼東兵入衛。景泰元年,超擢南京刑部右侍郎。四年就改禮部,奉敕考察雲南官吏。還朝,留任禮部。

景帝不豫,尚書胡濙在告,夔強起之,偕群臣疏請復太子。不允。明日,夔欲率百官伏闕請,而石亨輩已奉上皇復位,出夔南京禮部。英宗雅知夔,及聞復儲議,驛召還,進左侍郎。天順二年改吏部。知府某以貪敗,賄石亨求復,夔執不可,遂止。七年代石瑁為禮部尚書。

成化二年,帝從尚書李賓言,令南畿及浙江、江西、福建諸生,納米濟荒得入監。夔奏罷之。四年以災異屢見,疏請「均愛六宮,以廣繼嗣。乞罷西山新建塔院,斥遠阿叱哩之徒。勸視經筵,裁決庶政。親君子,遠小人,節用度,愛名器。服食言動,悉遵祖宗成憲,以回天意。」且言「今日能守成化初政足矣。」帝優旨答之。他所請十事,皆立報可。

慈懿太后崩,中旨議別葬,閣臣持不可,下廷議。夔言:「太后配先帝二十餘年,合葬升祔,典禮具在。一有不慎,違先帝心,損母后之德。他日有據禮議改者,如陛下孝德何?」疏三上,又率群臣伏文華門哭諫。帝為固請周太后,竟得如禮。後孝宗見夔及彭時疏,謂劉健曰:「先朝大臣忠厚為國乃如此!」彗星見,言官連劾夔,夔求去,不允。帝信番僧,有封法王、佛子者,服用僭擬無度。奸人慕之,競為其徒。夔力諫,勢稍減。

五年代崔恭為吏部尚書。雨雪失時,陳時弊二十事。七年加太子少保。彗星見,復偕群臣陳二十八事,大要以絕求請,禁採辦,恤軍匠,減力役,撫流民,節冗費為急。帝多採納。明年九月,南畿、浙江大水。夔請命廷臣共求安民弭患之術。每遇災異,輒請帝振恤,憂形於色。明年卒,贈少保,謚文敏。

夔才器宏遠,表裏洞達。朝議未定者,夔一言立決。其在吏部,留意人才,不避親故。初,王翱為吏部,專抑南人,北人喜之。至夔,頗右南人,論薦率能稱職。

子璧,由進士歷官兵部郎中。項忠劾汪直,璧預其謀。直構忠,連璧下獄,謫廣西思明同知,謝病歸。

夔從弟龍,與夔同舉進士,除刑部主事,累官福建左布政使。右布政使劉讓同年不相能。讓粗暴,龍亦乏清操。成化初入覲,王翱兩罷之。

王復,字初陽,固安人。正統七年進士。授刑科給事中。聲容宏偉,善敷奏。擢通政參議。

也先犯京師,邀大臣出迎上皇。眾憚行,復請往。乃遷右通政,假禮部侍郎,與中書舍人趙榮偕。敵露刃夾之,復等不為懾。還仍蒞通政事,再遷通政使。天順中,歷兵部左右侍郎。

成化元年,延綏總兵官房能奏追襲河套部眾,有旨獎勞。復以七百里趨戰非宜,且恐以僥倖啟釁,請敕戒諭,帝是之。進尚書。錦衣千戶陳玨者,本畫工。及卒,從子錫請襲百戶。復言:「襲雖先帝命,然非軍功,宜勿許。」遂止。

毛里孩擾邊,命復出視陜西邊備。自延綏抵甘肅,相度形勢,上言:「延綏東起黃河岸,西至定邊營,接寧夏花馬池,索紆二千餘里。險隘俱在內地,而境外乃無屏障,止憑墩堡以守。軍反居內,民顧居外。敵一入境,官軍未行,民遭掠已盡矣。又西南抵慶陽,相去五百餘里,烽火不接。寇至,民猶不知。其迤北墩堠,率皆曠遠,非御邊長策。請移府谷、響水等十九堡,置近邊要地。而自安邊營接慶陽,自定邊營接環州,每二十里築墩臺一,計凡三十有四。隨形勢為溝牆,庶息響相聞,易於守御。」其經略寧夏,則言:「中路靈州以南,本無亭燧。東西二路,營堡遼絕,聲聞不屬,致敵每深入。亦請建置墩臺如延綏,計為臺五十有八。」

其經略甘肅,則言:「永昌、西寧、鎮番、莊浪俱有險可守。惟涼州四際平曠,敵最易入。又水草便利,輒經年宿留。遠調援軍,兵疲銳挫,急何能濟。請於甘州五衛內,各分一千戶所,置涼州中衛,給之印信。其五所軍伍,則於五衛內餘丁選補。且耕且練,斯戰守有資,兵威自振。」又言:「洪武間建東勝衛,其西路直達寧夏,皆列烽堠。自永樂初,北寇遠遁,因移軍延綏,棄河不守。誠使兵強糧足,仍準祖制,據守黃河,萬全計也。今河套未靖,豈能遽復?然亦宜因時損益。延綏將校視他鎮為少,調遣不足,請增置參將二人,統軍九千,使駐要地,互相援接,實今日急務。」奏上,皆從之。

復在邊建置,多合機宜。及還朝,言者謂治兵非復所長。特命白圭代之,改復工部。謹守法度,聲名逾兵部。時中官請修皇城西北回廊,復議緩其役。給事中高斐亦言災沴頻仍,不宜役萬人作無益。帝皆不許。中官領騰驤四衛軍者,請給胖襖鞋褲。復執不可,曰:「朝廷制此,本給征行之士,使得刻日戒途,無勞縫紉。京軍則歲給冬衣布棉,此成憲也,奈何渝之?」大應法王札實巴死,中官請造寺建塔。復言:「大慈法王但建塔,未嘗造寺。今不宜創此制。」乃止命建塔,猶發軍四千人供役云,十四年加太子少保。

復好古嗜學,守廉約,與人無城府,當官識大體。居工部十二年,會災異,言官言其衰老,乞休。不許。居二月,汪直諷言官更劾復及鄒幹、薛遠。乃傳旨,並令致仕歸。久之,卒。贈太子太保,謚莊簡。

林聰,字季聰,寧德人。正統四年進士。授吏科給事中。景泰元年進都給事中。時方多故,聰慷慨論事,無所諱。中官金英家人犯法,都御史陳鎰、王文治之,不罪英。聰率同列劾鎰、文畏勢從奸,並及御史宋瑮,謝琚,皆下獄。已而復職。聰又言瑮、琚不任風紀,二人竟調外。中官單增督京營有寵,朝士稍忤者輒遭辱;家奴白晝殺人,奪民產,侵商稅。聰發其奸,下詔獄。獲宥。增自是不敢肆。

三年春,疏言:「臣職在糾察刑獄。妖僧趙才興之疏族百口,律不當坐,而抄提至京。叛人王英,兄不知情,家口律不當逮,而俱配流所。雖終見原,然其始受害已不堪矣。湖廣巡撫蔡錫以劾副使邢端,為所訐,繫獄經年,而端居職如故。侍郎劉璉督餉侵隱,不為無罪。較沈固、周忱乾沒萬計,孰為輕重?璉下獄追徵,而固、忱不問。犯人徐南與子中書舍人頤,俱坐王振黨當斬,乃論南大闢,頤止除名。皆刑罰之失平者。」帝是之。端下獄,璉得釋,南亦減死,除名。

東宮改建,聰有異論,遷春坊司直郎。四年春,學士商輅言聰敢言,不宜置之散地,乃復為吏科都給事中。上言奪情非令典,請永除其令。帝納之。初,正統中,福建銀場額重,民不堪。聰恐生變,請輕之。時弗能用,已果大亂。及是復極言其害,竟得減免。

五年三月,以災異偕同官條上八事,雜引五行諸書,累數千言。大略以絕玩好,謹嗜欲,為崇德之本。而修人事,在進賢退奸。武清侯石亨、指揮鄭倫身享厚祿,而多奏求田地;百戶唐興多至一千二百餘頃,宜為限制。餘如罷齋醮、汰僧道,慎刑獄,禁私役軍士,省輪班工匠,皆深中時弊。帝多採納。

先是,吏部尚書何文淵以聰言下獄,致仕去。及是,吏部除副使羅箎為按察使,參政李輅、僉事陳永為布政使。聰疏爭之,並言山西布政使王瑛老,宜罷。箎等遂還故官,瑛致仕。御史白仲賢以久次,擢廣東按察使。聰言仲賢奔競,不當超擢,乃遷鎮江知府。兵部主事吳誠夤緣得吏部,聰劾之,遂改工部。諸司憚聰風裁,聰所言,無敢不奉行者,吏部尤甚。內閣及諸御史亦並以聰好論建,弗善也。

其年冬,聰甥陳和為教官,欲得近地便養。聰為言於吏部。御史黃溥等遂劾聰挾制吏部;並前劾仲賢為私其鄉人參政方員,欲奪仲賢官予之;與吳誠有怨,輒劾誠;福建參政許仕達囑聰求進,聰舉仕達堪巡撫。並劾尚書王直阿聰。章下廷訊,坐專擅選法,論斬。高谷、胡濙力救。帝亦自知聰,止貶國子學正。

英宗復闢,超拜左僉都御史,出振山東饑,活饑民百四十五萬。還進右副都御史,捕江、淮鹽盜。以便宜,擒戮渠魁數人,餘悉解散,而奏籍指揮之受盜賂者。母憂起復,再辭。不許。

天順四年,曹欽反。將士妄殺,至割乞兒首報功,市人不敢出戶。聰署院事,急令獲賊者必生致,濫殺為止。錦衣官校惡欽殺指揮逯杲,悉捕欽姻識。千戶龔遂榮及外舅賀三亦在繫中。人知其冤,莫敢直,聰辨出之。其他湔雪者甚眾。七年冬,以刑部囚自縊,諸給事中劾紀綱廢弛,與都御史李賓俱下獄。尋釋。

成化二年,淮南、北饑,聰出巡視。奏貸漕糧及江南餘糧以振,民德之如山東。明年偕戶部尚書馬昂清理京軍,進右都御史。七年代王越巡撫大同。歲餘,遇疾致仕。再歲,以故官起掌南院。前掌院多不樂御史言事,聰獨獎勵之。或咎聰,聰曰:「己既不言,又禁他人言,可乎?」

十三年秋,召拜刑部尚書,尋加太子少保。聰以舊德召用,持大體,秉公論,不嚴而肅,時望益峻。十五年,偕中官汪直、定西侯蔣琬按遼東失事狀。直庇巡撫陳鉞,聰不能爭,論者惜焉。十八年乞歸不得,卒於位,年六十八。贈少保,謚莊敏。

聰為諫官,嚴重不可犯。實恂恂和易,不為嶄絕之行。以故不肖者畏之,而賢者多樂就焉。景泰時,士大夫激昂論事,朝多直臣,率聰與葉盛為之倡。

葉盛,字與中,昆山人。正統十年進士,授兵科給事中。師覆土木,諸將多遁還,盛率同列請先正扈從失律者罪,且選將練兵,為復仇計。郕王即位,例有賞賚,盛以君父蒙塵辭。不許。

也先迫都城,請罷內府軍匠備徵操。又請令有司儲糧科給戰士,遣散卒取軍器於天津,以張外援。三日間,章七八上,多中機宜。寇退,進都給事中。言:「勸懲之道,在明賞罰。敢戰如孫鏜,死事如謝澤、韓青,當賞。其他守御不嚴,赴難不力者,皆當罰。」大臣陳循等議召還鎮守居庸都御史羅通,並留宣府都督楊洪掌京營。盛言:「今日之事,邊關為急。往者獨石、馬營不棄,駕何以陷土木?紫荊、白羊不破,寇何以薄都城?今紫荊、倒馬諸關,寇退幾及一月,尚未設守御。宣府為大同應援,居庸切近京師,守之尤不可非人。洪等既留,必求如洪者代之,然後可以副重寄而集大功。」帝是之。尋命出安集陳州流民。

景泰元年還朝,言:「流民雜五方,其情不一。雖幸成編戶,而鬥爭仇殺時時有之,宜專官綏撫。」又言:「畿輔旱蝗相仍,請加寬恤。」帝多採納。京衛武臣及其子弟多驕惰不習兵。盛請簡拔精壯,備操守京城。勳戚所置市廛,月征稅。盛以國用不足,請籍其稅佐軍餉。皆從之。明年,上弭災防患八事。帝以兵革稍息,頗事宴游,盛請復午朝故事,立報可。當是時,帝虛懷納諫,凡六科聯署建請,多盛與林聰為首。廷臣議事,盛每先發言,往復論難。與議大臣或不悅曰:「彼豈少保耶?」因呼為「葉少保」。然物論皆推盛才。

擢右參政,督餉宣府。尋以李秉薦,協贊都督僉事孫安軍務。初,安嘗領獨石、馬營、龍門衛、所四城備御,英宗即北狩,安以四城遠在塞外,勢孤,奏棄之內徙。至是廷議命安修復。盛與闢草萊,葺廬舍,庀戰具,招流移,為行旅置爰鋪,請帑金買牛千頭以賦屯卒,立社學,置義冢,療疾扶傷。兩歲間,四城及赤城、雕鶚諸堡次第皆完,安由是進副總兵。而守備中官弓勝害安,奏安疾宜代。帝以問盛,言:「安為勝所持,故病。今諸將無踰安者。」乃留安,且遣醫視疾。已又劾勝,卒調之他鎮。

英宗復位,盛遭父憂,奔喪。天順二年召為右僉都御史,巡撫兩廣。乞終制,不許。瀧水瑤鳳弟吉肆掠,督諸將生擒之。時兩廣盜蜂起,所至破城殺將。諸將怯不敢戰,殺平民冒功,民相率從賊。盛以蠻出沒不常,請自今攻劫城池者始以聞,餘止類奏。疏至兵部,駁不行。盛與總兵官顏彪破賊寨七百餘所。彪頗濫殺,謗者遂以咎盛。六年命吳禎撫廣西,而盛專撫廣東。

憲宗立,議事入都,給事中張寧等欲薦之入閣。以御史呂洪言遂止,而以韓雍代撫廣東。初,編修邱濬與盛不相能。大學士李賢入濬言,及是草雍敕曰:「無若葉盛之殺降也。」盛不置辨。稍遷左僉都御史,代李秉巡撫宣府。請量減中鹽米價,以勸商裕邊。復舉官牛官田之法,墾田四千餘頃。以其餘積市戰馬千八百匹,修堡七百餘所,邊塞益寧。

成化三年秋,入為禮部右侍郎,偕給事毛弘按事南京。還改吏部。出振真定、保定饑,議清莊田,分養民間種馬,置倉涿州、天津,積粟備荒,皆切時計。

滿都魯諸部久駐河套,兵部尚書白圭議以十萬眾大舉逐之,沿河築城抵東勝,徙民耕守。帝壯其議。八年春,敕盛往會總督王越,巡撫馬文升、餘子俊、徐廷璋詳議。初,盛為諫官,喜言兵,多所論建。既往來三邊,知時無良將,邊備久虛,轉運勞費,搜河套復東勝未可輕議。乃會諸臣上疏,言「守為長策。如必決戰,亦宜堅壁清野,伺其惰歸擊之,令一大創,庶可遏再來。又或乘彼入掠,遣精卒進搗其巢,令彼反顧,內外夾擊,足以有功。然必守固,而後戰可議也。」帝善其言,而圭主復套。師出,竟無功。人以是服盛之先見。

八年轉左侍郎。十年卒,年五十五。謚文莊。

盛清修積學,尚名檢,薄嗜好,家居出入常徒步。生平慕範仲淹,堂寢皆設其像。志在君民,不為身計,有古大臣風。

贊曰:天順、成化間,六部最稱得人。王翱等正直剛方,皆所謂名德老成人也。觀翱與李秉、年富之任封疆,王竑之擊奸黨、活饑民,王復之籌邊備,姚夔之典秩宗,林聰、葉盛之居言路,所表見,皆自卓卓。其聲實茂著,系朝野重望,有以哉。


\end{pinyinscope}