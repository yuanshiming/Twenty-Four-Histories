\article{列傳第六十二}

\begin{pinyinscope}
○史昭劉昭李達巫凱曹義施聚許貴子寧周賢子玉歐信王璽魯鑑子麟孫經劉寧周璽莊鑒彭清姜漢子奭孫應熊安國杭雄

史昭,合肥人。永樂初,積功至都指揮僉事。八年充總兵官,鎮涼州。土軍老的罕先與千戶虎保作亂,虎保敗,老的罕就撫。昭上書言其必叛狀。未至,而老的罕果叛。昭與都指揮滿都等擊平之。移鎮西寧。

仁宗立,進都督僉事。上言西寧風俗鄙悍,請設學校如中土。報可。宣德初,昭以衛軍守禦,不暇屯種,其家屬願力田者七百七十餘人,請俾耕藝,收其賦以足軍食。從之。五年,曲先衛都指揮使散即思邀劫西域使臣,昭率參將趙安偕中官王安、王瑾討之。長驅至曲先,散即思望風遁,擒其黨答答不花等,獲男女三百四十人,馬駝牛羊三十餘萬,威震塞外。捷聞,璽書慰勞,賞賚加等。

七年春,以征西將軍鎮寧夏。孛的達里麻犯邊,遣兵擊之。至闊台察罕,俘獲甚眾。進都督同知。

正統初,昭以寧夏孤懸河外,東抵綏德二千里,曠遠難守,請於花馬池築哨馬營,增設烽堠,直接哈剌兀速之境。邊備大固。尋進右都督。時阿台、朵兒只伯數寇邊。詔昭與甘肅守將蔣貴、趙安進剿。並無功,被昭切責,貶都督僉事。三年復右都督,八年以老召還。明年卒。

昭居寧夏十二年,老成持重,兵政修舉,亦會敵勢衰弱,邊境得無事。兵部尚書王驥、寧夏參將王榮嘗舉其過。朝議,以昭守邊久,習兵事,不易也。而與昭並為邊將最久、有勳績可稱者,都督同知劉昭鎮西寧二十年;都指揮李達鎮洮州至四十年。並為蕃漢所畏服。

劉昭,全椒人。永樂五年以都指揮同知使朵甘、烏思藏,建驛站。還至靈藏,番賊邀劫,昭敗之。進都指揮使,鎮河州。宣德二年,副陳懷討平松潘寇。累進都督同知,移西寧。復鎮河州,兼轄西寧。罕東酋札兒加邀殺中官使西域者,奪璽書金幣去。命昭副甘肅總兵官劉廣討之。札兒加請還所掠書幣,貢馬贖罪。帝以窮寇不足深治,命昭等還。

李達,定遠人。累官都督僉事。正統中,致仕。

巫凱,句容人。由廬州衛百戶積功至都指揮同知。永樂六年以從英國公張輔平交阯功,遷遼東都指揮使。十一年召帥所部會北京。明年從征沙漠,命先還。凱言諸衛兵宜以三之二守禦,而以其一屯糧,開原市馬悉給本衛乘操。從之。

宣宗立,以都督僉事佩征虜前將軍印,代朱榮鎮遼東。時中國人自塞外脫歸者,令悉送京師,俟親屬赴領。凱言遠道往來,恐致失所,阻遠人慕歸心。乃更令有馬及少壯者送京師,餘得自便。敵掠西山,凱擊敗之,盡得所掠者,降敕褒勉。

帝嘗遣使造舟松花江招諸部。地遠,軍民轉輸大困,多逃亡。會有警,凱力請罷其役,而逃軍入海西諸部者已五百餘人。既而造舟役復興,中官阮堯民、都指揮劉清等董之。多不法,致激變。凱劾堯民等,下之吏。

英宗登極,進都督同知,上言邊情八事。請厚恤死事者家,益官吏折俸鈔,歲給軍士冬衣布棉,軍中口糧芻粟如舊制,且召商實邊。俱允行。未幾,為兵部尚書王驥所劾。朝廷知凱賢,令凱自陳。并諭廷臣,文武官有罪得實始奏,誣者罪不貸。凱由是得行其志。正統三年十二月有疾,命醫馳視,未至而卒。

凱性剛毅,饒智略,馭眾嚴而有恩。在遼東三十餘年,威惠並行,邊務修飭。前後守東陲者,曹義外皆莫及。

義,字敬方,儀真人。以燕山左衛指揮僉事累功至都督僉事,副凱守遼東。凱卒,代為總兵官。凱,名將,義承其後,廉介有守,遼人安之。兀良哈犯廣寧前屯,詔切責,命王翱往飭軍務,劾義死罪。頃之,義獲犯邊孛台等,詔戮於市。自是義數與兀良哈戰。正統九年,會朱勇軍夾擊。斬獲多,進都督同知,累官左都督。義在邊二十年,無赫赫功,然能謹守邊陲。其麾下施聚、焦禮等皆至大將。英宗復辟,特封義豐潤伯,聚亦封懷柔伯。居四年,義卒,贈侯,謚莊武。繼室李氏殉,詔旌之。

施聚,其先沙漠人,居順天通州。父忠為金吾右衛指揮使,從北征,陣歿,聚嗣職。宣德中,備禦遼東,累擢都指揮同知。以義薦,進都指揮使。義與兀良哈戰,聚皆從。也先逼京師,景帝詔聚與焦禮俱入衛。聚慟哭,即日引兵西。部下進牛酒,聚揮之曰:「天子安在?吾屬何心饗此!」比至,寇已退,乃還。聚以勇敢稱,官至左都督。值英宗推恩,得封伯。後義二年卒,贈侯,謚威靖。義三傳至棟,聚四傳至瑾,吏部皆言不當復襲,世宗特許之。傳爵至明亡。

許貴,字用和,江都人,永新伯成子也。襲職為羽林左衛指揮使。安鄉伯張安舉貴將才,試騎射及策皆最,擢署都指揮同知。尋以武進伯朱冕薦擢山西行都司,督操大同諸衛士馬。

正統末,守備大同西路。也先入寇,從石亨戰陽和後口,敗績,貴力戰得還。英宗北狩,邊城悉殘破,大同當敵衝,人心尤恟懼。貴以忠義激戰士。敵來,擊敗之。進都指揮使。

景泰元年春,充右參將。敵寇威遠,追敗之浦州營,奪還所掠人畜。敵萬騎逼城下,禦卻之。再遷都督同知。大同乏馬,命求民間,得八百餘匹。所司不給直,貴為請,乃予之。嘗募死士入賊壘,劫馬百餘,悉畀戰士,士皆樂為用。分守中官韋力轉淫虐,眾莫敢言,貴劾奏之。三年,疾還京。英宗復辟,命理左府事,尋調南京。

松潘地雜番苗,密邇董卜韓胡,舊設參將一人。天順五年,守臣告警,廷議設副總兵,以貴鎮守。未抵鎮而山都掌蠻叛,詔便道先翦之。貴分兩哨直抵其巢,連破四十餘寨。斬首千一百級,生擒八百餘人,餘賊遠遁。貴亦感嵐氣,未至松潘卒。帝為輟朝一日,賜賻及祭葬如制。

子寧,字志道。正統末,自以舍人從軍有功,為錦衣千戶。貴歿,嗣指揮使。用薦擢署都指揮僉事,守禦柴溝堡。

成化初,充大同遊擊將軍。寇入犯,與同官秦傑等禦之小龍州澗,擒其右丞把禿等十一人。改督宣府操練,移延綏。地逼河套,寇數入掠孤山堡。寧提孤軍奮擊之,三戰皆捷,寇渡河走。明年復以三千騎入沙河墩,與總兵官房能禦之。寇退,復掠康家岔。寧出塞百五十里,追與戰,獲馬牛羊千餘而還。

時能守延綏,無將略,巡撫王銳請濟師。詔大同巡撫王越帥眾赴,越遣寧出西路。破敵黎家澗,進都指揮同知。復遣寧與都指揮陳輝追寇,獲馬騾六百。朝廷以阿羅出復入河套,頻擾邊,命越與朱永禦,而以寧才,擢都督僉事,佩靖虜副將軍印,代能充總兵官。寧起世胄,不十年至大將,同列推讓不及,父友多隸部下,亦不以為驟。踰月,寇大入,永遣寧及遊擊孫鉞禦之。至波羅堡,相持三日夜,寇乃解去。亡失多,寧以力戰得出,卒被賞。至冬,賊入安邊,寧追擊有功。

七年又與諸將孫鉞、祝雄等敗寇於滉忽都河,璽書褒獎。迤北開元王把哈孛羅屢欲降,內懼朝廷見罪,外畏河羅出仇之,彳旁徨不決。寧請撫慰以固其心,卒降之。明年,參將錢亮敗績於師婆澗。士卒死者十三四,寧與越等俱被劾。帝不罪。時滿都魯等屢犯延綏,寧帥鎮兵力戰。寇不得志,乃出西路,直犯環慶、固原。寧將輕騎夜襲之鴨子湖,奪馬畜而還。又明年,寇入榆咻澗,與巡撫餘子俊敗之。滿都魯等大入西路,留其家紅鹽池。越乘間與寧及宣府將周玉襲破其巢。進署都督同知。與子俊築邊牆,增營堡,寇患少衰。

十八年,寇分數道入,寧蹙之邊牆,獲級百二十。予實授。時越方鎮大同,命寧與易鎮。至則與鎮守太監汪直不協。巡撫郭鏜以聞,調直南京。小王子大入。寧知敵勢盛,欲持重俟隙,乃斂兵守,而別遣將劉寧、董升與周璽相犄角。寇大掠,焚代王別堡。王趣戰,使眾哭於轅門。寧憤,與鏜等營城外。寇以十餘人為誘,太監蔡新部騎馳擊。寧將士爭赴之,遇伏大敗,死者千餘人。寧奔夏米莊,鏜、新馳入城。會璽等來援,寇乃退。寧還,陣亡家婦子號呼詬詈,擲以瓦礫,寧大喪氣。已而寇復入,劉寧、宋澄、莊鑒等禦之。十戰,少利,寇退。寧等掩其敗,更以捷聞。巡按程春震發之,與鏜、新俱下獄。鏜降六官,新以初任降三官,寧降指揮同知閒住。

弘治中,用薦署都指揮使,分領操練。十一年十二月卒。贈都督僉事。

寧束髮從軍,大小百十餘戰,身被二十七創。性沉毅,守官廉,待士有恩,不屑干進。劉寧、神英、李杲皆出麾下。子泰,自有傳。

周賢,滁人,襲宣府前衛千戶。景泰初,累功至都指揮僉事,守備西貓兒峪,助副總兵孫安守石八城。尋充右參將,代安鎮守。兀良哈入寇,總兵官過興令宣府副將楊信及賢合擊。賢不俟信,徑擊敗之。信被劾,都御史李秉言信緩師,賢亦棄約。帝兩宥之。

天順初,總兵官楊能奏賢擢都督僉事。寇駐塞下,能檄賢與大軍會,失期,徵下獄。以故官赴寧夏,隸定遠伯石彪。寇二萬騎入安邊營。彪率賢等擊之,連戰皆捷,追至野馬澗、半坡墩,寇大敗。而賢追不已,中流矢卒。詔贈都督同知。賢初下吏,自以不復用,及得釋,感激誓死報,竟如其志。

子玉,字廷璧,當嗣指揮使。以父死事,超二官為萬全都司都指揮同知,督理屯田。進都指揮使,充宣府遊擊將軍。

成化九年,會昌侯孫繼宗等奉詔舉將才,玉為首。詔率所部援延綏,從王越襲紅鹽池。進署都督僉事,還守宣府。寇入馬營、赤城,擊敗之。兵部言宣府諸大帥無功,玉所部三千人能追敵出境,請加一秩酬其勞,乃予實授。尋充宣府副總兵。

十三年佩征朔將軍印,鎮宣府。破敵紅崖,追奔至水磨灣。進署都督同知。十七年五月,寇復入犯,參將吳儼、少監崖榮追出塞,至赤把都,為所遮,兵分為三,皆被圍。儼、榮走據北山,因甚。守備張澄率兵進,力戰,解二圍。抵北山下,儼、榮已夜遁。澄拔其眾而還,死者過半。澄所部七百人,亦多戰死。詔錄澄功,治儼等罪。玉先以葛谷堡、赤城頻受掠,凡三被論,至是復以節制不嚴見劾。帝皆置不問。

十九年,小王子犯大同。敗總兵官許寧。入順聖川大掠,以六千騎寇宣府。玉將二千人前行,巡撫秦紘兵繼進,至白腰山擊敗之。指揮曹洪邀擊至西陽河,都指揮孫成亦敗寇七馬房。時寇乘勝,氣甚銳,竟為玉等所挫,一時稱其功。未幾,寇復入,玉伏兵敗之。朱永至大同,復會玉軍擊敗之鵓鴿峪。進署右都督。

餘子俊築邊牆,玉不為力,且與紘不相能。子俊惡之,奏與寧夏神英易鎮。久之,復移鎮甘肅。孝宗嗣位,實授右都督。

玉督邊牆工峻急,部卒張伏興等以瓦石投之。兵部言,悍卒漸不可長,遂戮伏興,戍其黨。

土魯番貢獅子,願獻還哈密成及金印,贖所留使者。玉為之奏,帝命與巡撫王繼經畫。既果來歸,玉等皆受賚。七年,病歸,尋卒。謚武僖。

玉初為偏裨,及鎮宣府,甚有名。後蒞甘肅,部下屢失事,又侵屯田。死後事發,子襲職,降二等。

歐信,嗣世職金吾右衛指揮使。景泰三年以廣東破賊功,擢都指揮同知。已,命守備白羊口,遷大寧都指揮使。

天順初,以都督僉事充參將,守備廣東雷、廉諸府。巡撫葉盛薦其廉勇。進都督同知,代副總兵翁信。兩廣瑤僮陷開建,殺官吏,帝趣進兵。信破賊化州之馬里村,再破之石城,擊斬海南衛反者邵瑄。

時所在盜群起,將吏不能定。廣西參將范信守潯、梧,瑤盡在境內,陰納瑤賂,縱使越境流劫,約毋犯己。於是雷、廉、高、肇悉被寇。帝命廣西總兵官陳涇及信合剿。時有斬馘,而賊勢不衰,朝廷猶倚范信。會涇以罪徵,乃擢范信都督僉事充副總兵,鎮廣東,而命信佩征蠻將軍印,代涇鎮廣西。

成化元年,賊掠英德諸縣,信討斬五百餘人,奪還人口。韓雍督師,令信等分五哨,攻破大藤峽。已而餘賊復入潯州,信被劾獲宥,召還,理前府事。

七年春,充總兵官,鎮守遼東,累敗福餘三衛。言者謂信已老,請召還。巡撫彭誼奏:「官軍耆老五千餘人,皆言信忠謹有謀勇,累立戰功,威鎮邊陲。年六旬,騎射勝壯士,不宜召回。」乃留鎮如故。久之,陳鉞代誼。鉞貪功,信不能違,十四年為巡按王崇之所劾。其冬,乃召歸。尋遣中官汪直等往按,直右鉞,歸罪信等。下獄,鐫一官閑住,飲恨而卒。

范信既徙廣東,賊勢愈盛,劫掠不止,乃語人曰:「今賊仍犯廣東,亦我遣之耶。」而是時都督顏彪佩征夷將軍印,討賊久無功,濫殺良民報捷。嶺南人咸疾之。

王璽,太原左衛指揮同知也。成化初,擢署都指揮僉事,守禦黃河七墅。巡撫李侃薦於朝。阿魯出寇延綏,命充遊擊將軍赴援,戰孤山堡,敗之。寇再入,戰漫天嶺、劉宗塢及漫塔、水磨川,皆有功。進都指揮同知,充副總兵,鎮守寧夏。九年以將才與周玉同薦。十二年擢署都督僉事,充總兵官,鎮守甘肅。

黃河以西,自莊浪抵肅州南山,其外番人阿吉等二十九族所居也。洪武間,立石畫界,約樵牧毋越疆,歲久湮廢,諸番往往闌入,而中國無賴人又潛與交通為邊患。璽請「復畫疆域,召集諸番,諭以界石廢,恐官軍欺凌諸部,今復立之,聽界外駐牧,互市則入關。如此,番人必聽命,可潛消他日憂。」帝稱善,從之。

十七年進署都督同知。時璽以都督僉事為總兵官,而魯鑑以署都督同知為參將,璽恐難於節制,乞解兵柄,故有是命。

初,哈密為土魯番所擾,使其將牙蘭守之。都督罕慎寄居苦峪口,近赤斤、罕東,數相攻,罕慎勢窮無援。朝議敕璽築城苦峪,別立哈密衛以居之。璽遣諜者間牙蘭。牙蘭不聽,得其所羈掠九十餘人以歸,具悉虛實。十七年召集赤斤、罕東將士,犒以牛酒,令助罕慎。罕慎合二衛兵,夜襲哈密及剌木等八城,遂復其地,仍令罕慎居之。事聞,獎勞,賚金幣。已,罕東入寇,璽禦卻之,請興師以討。帝念其常助罕慎,第遣使責諭。明年,北寇殺哨卒,璽率參將李俊及赤斤兵擊之於狼心山、黑河西,多所斬獲。

二十年移鎮大同。璽有復哈密功,官不進,陳於朝,乃實授都督同知。

璽習韜略,諳文事,勇而有謀。廷臣多稱之。在邊二十餘年,為番人所憚。弘治元年卒。賜祭葬,贈恤有加。

魯鑒,其先西大通人。祖阿失都鞏卜失加,明初率部落歸附,太祖授為百夫長,俾統所部居莊浪。傳子失加,累官莊浪衛指揮同知。正統末,鑒嗣父職。久之,擢署都指揮僉事。

成化四年,固原滿四反,鑑以土兵千人從征。諸軍圍石城,日挑戰,鑒出則先驅,入則殿後,最為賊所憚。賊平,進署都督同知。尋充左參將,分守莊浪。命其子麟為百戶,統治土軍。十七年坐寇入境,戴罪立功。尋充左副總兵,協守甘肅。寇犯永昌。被劾。鑒疏辨,第停其俸兩月。俄命充總兵官,鎮守延綏。自陳往功,予實授。

孝宗立,得疾,致仕。弘治初,命麟襲指揮使,加都指揮僉事。已,進同知,充甘肅遊擊將軍。

魯氏世守西陲,有捍禦功,至鑑官益顯,其世業益大,而所部土軍生齒又日盛。麟既移甘肅,帝以土軍非鑑不能治,特起治之,且命有司建坊旌其世績。鑑乃條上邊務四事,多議行。鑑有材勇,遇敵輒冒矢石,數被傷不為沮,故能積功至大將。十五年以舊創疾發,卒。贈右都督,賜恤如制。

時麟已由甘肅參將擢左副總兵,豪健如其父,而恭順不如。先為遊擊時,寇入永昌,失律,委罪副將陶禎。下御史按,當戍邊,但貶一秩,遊擊如故。暨為副將,調韋州禦寇。寇大入不能擊,遣都指揮楊琳邀之孔壩溝。琳大敗,不救,連被劾。麟自醖,止停俸二月。時已授麟子經官,令約束土軍。而麟奏經幼,土人不受要束,乞歸治之。不俟報,徑歸。帝用劉大夏言,從其請。武宗立,甘肅巡撫畢亨薦經及麟謀勇,令率所部協戰守。正德二年,經既襲指揮使,自陳嘗隨父有功,乃以為都指揮僉事。未幾,麟卒,贈都督僉事。賜祭葬。故事,都指揮無恤典,以經乞,破例予之。

經積戰功,再遷都指揮使充左參將,分守莊浪。復自陳功閥,兵部執不可。帝特命為署都督僉事。世宗立,乞休。巡撫許鳳翔言經力戰被創,疾行愈,且世將敢戰,知名異域,今邊患棘,不宜聽其去。帝乃諭留,且勞以銀幣。尋充副總兵分守如故。嘉靖六年冬,以都督同知充總兵官,鎮守延綏。大學士楊一清言:「經守莊浪二十餘載,屢立戰功,其部下土軍非他人所能及。雖其子瞻已為指揮僉事,奉命統轄,然年尚少。今陜西總兵官張鳳乃延綏世將,若調鳳延綏,而改經陜西,自可彈壓莊浪,無西顧患。」帝立從之。居二年,竟以疾致仕。

久之,命瞻以本官守備山丹。經奏言:「自臣高祖後,世守茲土。今臣家居,瞻又移他鎮,土軍皇皇,不欲別附,若因此生他患,是隳先業而負世恩也,乞令守故業。」可。

二十二年,宣、大有警,詔經簡壯士五千赴援。至而邊患已息,乃遣還。以經力疾趨召,厚賚之。明年瞻卒。經以次子及孫皆幼,請得自轄土軍。詔許之。

經驍勇,奉職寡過,繼祖父為大帥,保功名,稱良將。三十五年卒。賜恤如制。

劉寧,字世安,其先山陽人。襲世職,為永寧衛指揮使。勇敢善戰。自以冗散無所見,會延綏用兵,疏請效死。尚書白圭許之。屢以功遷都指揮使,充宣府遊擊將軍。

周璽,字廷玉,遷安人。嗣職為開平衛指揮使。負氣習兵書,善騎射。以征北功,擢署都指揮僉事充右參將,分守陽和,敕部兵三千訓肄聽調。成化十六年,從王越征威寧海子,累進都指揮使。

時邊寇無虛歲。十八年分道入掠,璽與遊擊董昇戰黑石崖,寧戰塔兒山,皆有功。璽進署都督僉事,遷大同副總兵。寧進都督僉事,改左參將,分守陽和。

十九年秋,亦思馬因大入。大同總兵官許寧分遣璽守懷仁,寧與董昇營西山,自將中軍,擊之夏米莊,敗績。寧、昇被圍數重,幾陷。亟發巨炮擊之,敵多死,圍乃解。璽聞中軍失利,亟還兵援。夜遇敵,乘勝前,銳甚。璽厲將士曰:「今日有進無退!」大呼陷陣,敵少卻。久之,短兵接。臂中流矢,拔鏃戰益急,與子鵬及麾下壯士擊殺數十人。會寧兵至,中軍潰卒亦稍集,敵乃退,許寧等亦還。無何復入掠。寧將兵三千,遇之聚落站西,連戰敗之。復敗之白登、柳林,又追敗之小鵓鴿谷。而大同西路參將莊鑒亦邀其歸路,戰於牛心山,敵遂遁。時諸將多失利,許寧以下獲罪,而璽以功予實授,寧超遷都督同知,莊鑑以所部無失亡,亦賚銀幣。

鑑,遼東人。天順中,襲定遼右衛指揮使。驍猛有膽決。遇賊輒奮,數有功,累官都督僉事,掌左府。弘治十一年佩鎮朔將軍印,鎮宣府。以才與大同總兵官張俊易鎮。兵部侍郎熊繡奏其經畫功,進都督同知。

璽尋以右副總兵分守代州,兼督偏頭諸關,而改寧左副總兵,協守大同。二人並著功北邊,稱名將。璽以偏頭去太原遠,請改分守為鎮守,又以鎮守不當聽節制,乞易總兵銜。憲宗皆曲從之。弘治初,移鎮陜西,討平扶風諸縣附籍回回。三年佩征西將軍印,鎮守寧夏,甫一歲卒。且死,召諸子曰:「吾佩印分閫,分已足,獨未嘗大破敵,抱恨入地矣。」連呼「殺賊」而瞑。子鵬,累官錦衣衛指揮僉事。

璽歿後三年,而寧佩平羌將軍印,鎮甘肅。其冬,寇犯涼州,寧與戰抹山墩,擒斬五十餘,相持至暮,收輜重南行。寇復來襲,擒其長一人。明日,參將顏玉來援,副將陶禎兵亦至,寇乃遁。俘其稚弱,獲馬駝牛羊二千,進右都督。明年,與巡撫許進襲破土魯番於哈密,進左都督,增俸百石,以疾還京。十三年,大同告警,命寧為副總兵,從平江伯陳銳禦之。銳無將略,與寧不協,止毋戰,寇遂得志去,坐停半俸閒住。尋以參將贊畫朱暉軍務,亦無功。寧自陳哈密功,乞封伯,詔還全俸。

寧有膽智,為大同副將時,入貢者數萬人懷異志。寧率二十騎直抵其營,眾駭愕。有部長勒馬引弓出。寧前下馬,與諸部長坐,舉策指畫,宣天子威德。一人語不遜,寧摑其面,奮臂起,其長叱之退。寧復坐與語,呼酒歡飲,皆感悟,卒如約。嘗仿古番上法,以五十八人為隊,隊伍重為陣,建五色幟。又各建五巨幟於中軍,中幟起,五陣各視其色應之,循環無端,每戰用是取勝。晚再赴大同,已老病,帥又怯懦,故無成功,然孝宗朝良將稱寧。十七年卒,贈廣昌伯。

彭清,字源潔,榆林人。初襲綏德衛指揮使,以功擢都指揮僉事。弘治初,充右參將,分守肅州。寇入犯,率兵躡之,獲馬駝器仗及所掠人畜而還。尋與巡撫王繼恢復哈密有功。

清雖位偏校,而好謀有勇略,名聞中朝,尤為尚書馬文升所器。嘗引疾乞休,文升力言於朝,慰留之。八年,甘肅有警,以文升薦,擢左副總兵,仍守甘肅。未幾,巡撫許進乞移清涼州。而是時哈密復為土魯番所據,文升方密圖恢復,倚清成功,言「肅州多故,而清名著西域,不可易」,乃寢。

文升既得楊翥策,銳欲搗哈密襲牙蘭,乃發罕東、赤斤暨哈密兵,令清統之為前鋒,從許進潛往。行半月,抵其城下,攻克之。牙蘭已先遁,乃撫安哈密遺種,全師而還。是役也,文升授方略,擬從間道往,而進仍由故道,牙蘭遂逸去,斬獲無幾。然番人素輕中國,謂不能涉其地,至是始知畏。清功居多,稍遷都指揮使。

十年,總兵官劉寧罷,擢清都督僉事代之。其冬,土魯番歸哈密忠順王陜巴,且乞通貢,西域復定。屢辭疾,請解兵柄,不允。十五年卒。

清御士有恩,久鎮西陲,威名甚著,番夷憚之。性廉潔,在鎮遭母及妻妹四喪,貧不能歸葬。卒之日,將士及庶民婦豎皆流涕。遺命其子不得受賻贈,故其喪亦不能歸。帝聞之,命撫臣發帑錢,資送歸里,賜祭葬如制。

姜漢,榆林衛人。弘治中,嗣世職,為本衛指揮使。御史胡希顏薦其材勇,進都指揮僉事,充延綏遊擊將軍。十八年春,寇犯寧夏興武營。漢帥所部馳援,遇於中沙墩,擊敗之。賜敕獎勞。武宗嗣位,寇大舉犯宣、大,漢偕副總兵曹雄、參將王戟分道援,有功。尋代雄為副總兵,協守延綏。正德三年移守涼州。明年冬,擢署都督僉事,充總兵官,鎮寧夏。

漢馭軍嚴整,得將士心。甫數月而安化王寘鐇謀逆,置酒召漢及巡撫安惟學等宴。酒半,其黨何錦等率眾入,即座上執漢。漢奮起,怒罵不屈,遂殺之。子奭逃免。賊平,訟於朝。詔賜祭葬。有司為立祠,春秋祭之。嘉靖時,復從巡撫張珩請,賜額「憫忠」。

奭當嗣職,帝以漢死事,特進一官,為都指揮僉事。十一年,回賊魏景陽作亂,華陰諸縣悉被害,巡撫蕭翀檄奭討之,獲景陽。進署都指揮同知,充右參將守肅州。嘉靖二年,擢右副總兵,分守涼州,進署都督僉事,充總兵官,鎮甘肅。

回賊犯甘州,奭與戰張欽堡,敗走之。未幾,西海賊八千騎犯涼州。奭率遊擊周倫等襲擊於苦水墩,大敗之,斬首百餘級,殲其長,還所掠人口千二百、畜產二千。都指揮張錦亦戰死。錄功,進署都督同知。吉囊他部寇莊浪,奭與遇分水嶺,再勝之。遂至平嶺。敵騎大集,奭伏兵誘之,復斬其長一人,獲首功七十,予實授。十六年春,寇大入甘州,不能禦,貶二秩戴罪。尋以永昌破敵功,復署都督僉事。其冬,坐前罪罷。久之,以薦擢副總兵,協守大同,為總督翁萬達劾罷,卒。

子應熊,嗣指揮使,擢宣府西路參將。二十七年春,俺答寇大同,總兵官周尚文戰曹家莊,應熊從萬達自懷來鼓噪揚塵而西。寇不測眾寡,遂遁。累進都督僉事,充總兵官,鎮守寧夏。三十二年,套寇數萬騎屯賀蘭山,遣精騎掠紅井。應熊戒將士固守以綴敵,而潛師攻敵營,斬首百四十級,進都督同知。越二年,套寇數萬踏冰西渡,由寧夏山後直抵莊涼。應熊等掩擊,獲首功百餘,進右都督。御史崔揀劾其縱寇,褫職逮問,充為事官,赴塞上立功。四十年秋,寇六萬餘騎犯居庸岔道口,應熊被圍於南溝,中五鎗墮馬,參將胡鎮殺數人奪之歸。其冬,復為右都督,充總兵官,鎮守大同。以招徠塞外人口,增俸一級。

四十二年,寇大舉犯畿輔,詔應熊等入援,諸鎮兵盡集,見敵勢盛,不敢擊。給事中李瑜遂劾應熊及宣大總督江東、保定總兵官祝福坐視胡鎮被圍,一卒不發。帝怒,降敕嚴責。會寇將遁,應熊禦之密雲,頗有斬獲。寇退,帝令江東第諸將功,以應熊為首,詔增其祖職二級。已,錄防秋勞,進左都督。總督趙炳然劾其縱寇互市,殘害朔州,坐戍邊。穆宗立,赦還。

子顯祚襲職,累官署都督僉事,總兵官,歷鎮山西、宣府。子弼,亦至都督僉事,為援遼總兵官。姜氏為大將,著邊功,凡五世。

安國,字良臣,綏德衛人。初為諸生,通《春秋》子史,知名里中。後襲世職,為指揮僉事。正德三年中武會舉第一,進署指揮使,赴陜西三邊立功。劉瑾要賄,國同舉六十人咸無貲,瑾乃編之行伍,有警聽調,禁其擅歸。六十人者悉大窘,儕於戍卒,不聊生。而邊臣憚瑾,竟無有收恤之者。寘鐇反,肆赦,始放還。通政叢蘭請收用,瑾怒,諷給事中張瓚等劾諸人皆庸才,悉停其加官。瑾誅,始以故官分守寧夏西路。尋進署都指揮僉事,充右參將,擢右副總兵,協守大同,徙延綏。

十一年冬,寇二萬騎分掠偏頭關諸處,國偕遊擊杭雄馳敗之岢嵐州,斬首八十餘級,獲馬千餘匹。寇遂遁。初,寇大入白羊口,帝遣中官張忠、都督劉暉、侍郎丁鳳統京軍討之,比至,已飽掠去。忠、暉恥無功,紀功御史劉澄甫攘國等功歸之,大行遷賞,忠等悉增祿,予世廕。尚書王瓊亦加少保,廕子錦衣。國時以署都督僉事為寧夏總兵官,僅予實授,意不平,不敢自列,乃具疏力辭,為部卒重傷者乞敘錄。瓊請再敘國功,始進都督同知。

當是時,佞倖擅朝,債帥風大熾,獨國以材武致大將。端謹練戎務,所至思盡職,推將材者必歸焉。在鎮四年卒。特謚武敏。

杭雄,字世威,世為綏德衛總旗。雄承廕,數先登,積首功,六遷至指揮使。

正德七年進署都指揮僉事,剿賊四川,尋守備西寧。用尚書楊一清薦,擢延綏遊擊將軍。從都御史彭澤經略哈密,偕副將安國破敵岢嵐,進都督僉事。改參將,擢都督同知,統邊兵操於西內。武宗幸宣府、大同,雄扈從,即拜大同總兵官。

嘉靖初,汰傳奉官,雄當貶,以方守邊,命署都督僉事,鎮守如故。小王子萬餘騎入沙河堡,雄戰卻之。未幾,復大入,不能禦,求罷不許。移延綏,召僉書後軍都督府。

三年秋,土魯番侵甘肅,詔尚書金獻民視師,以雄佩平虜大將軍印,充總兵官,提督陜西、延綏、寧夏、甘肅四鎮軍務。列侯出征,始佩大將軍印,無授都督者,至是特以命雄。甫至,寇已破走,而雄亦得廕錦衣千戶。既班師,復出鎮寧夏。吉囊大入,總督王憲檄雄等破之,進都督同知。寇八千騎乘冰犯寧夏。雄及副總兵趙鎮禦之,前鋒陷伏中,雄等皆敗。總督王瓊劾之,奪官閒住。明年卒。

雄敢戰。嘗以數騎行邊,敵麕至。乃下馬積鞍為壘,跪而射之。敵退,解衣,腋凝血,乃知中飛矢。武宗在大同,見雄氈帷敝甚,曰:「老杭窮乃爾。」寇至,帝將親擊。雄叩馬諫曰:「主人畜犬,不使吠盜,奚用犬為?願聽臣等效力。」帝笑而止。少役延綏巡撫行臺,既貴,每至臺議事,不敢正席坐,曰:「此當年役所也。」正德、嘉靖間,西北名將,馬永而下稱雄云。

贊曰:時平則將略無由見。或綰符出鎮,守疆禦侮,著有勞效,以功名終,亦足尚矣。許貴、周賢、魯鑒、姜漢家世為將,勛閥相承,而賢與漢死事尤烈。彭清、杭雄之清節,斯又其最優者歟。


\end{pinyinscope}