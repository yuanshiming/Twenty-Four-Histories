\article{列傳第一百六十八 何騰蛟章曠 傅作霖 瞿式耜汪皞等}

\begin{pinyinscope}
何騰蛟,字雲從,貴州黎平衛人。天啟元年舉於鄉。崇禎中授南陽知縣。地四達,賊出沒其間,數被挫去。已,從巡撫陳必謙破賊安皋山,斬首四百餘級,又討平土寇,益知名。遷兵部主事,進員外郎,出為懷來兵備僉事,調口北道。才住精敏,所在見稱。遭母憂,巡撫劉永祚薦其賢,乞奪情任事。騰蛟不可,固辭歸。服除,起淮徐兵備僉事。討平土寇,部內宴然。

十六年冬,拜右僉都御史,代王聚奎巡撫湖廣。時湖北地盡失,止存武昌,屯左良玉大軍,軍橫甚。騰蛟與良玉交歡,得相安。明年春,遣將惠登相、毛憲文復德安、隨州。

五月,福王立。詔至,良玉駐漢陽,其部下有異議,不欲開讀。騰蛟曰:「社稷安危,繫此一舉。倘不奉詔,吾以死殉之。」抵良玉所,而良玉已聽正紀盧鼎言,開讀如禮。正紀者,良玉所置官名也。八月,福王命加騰蛟兵部右侍郎,兼撫湖南,代李乾德。尋以故官總督湖廣、四川、雲南、貴州、廣西軍務,召總督楊鶚還。明年三月,南京有北來太子事,中外以為真,朝臣皆曰偽。騰蛟力言不可殺,與當國者大忤。

無何,良玉舉兵反,邀騰蛟偕行,不可,則盡殺城中人以劫之。士民爭匿其署中,騰蛟坐大門縱之入。良玉破垣舉火,避難者悉焚死。騰蛟急解印付家人,令速走,將自剄,為良玉部將擁去。良玉欲與同舟,不從,乃置之別舟,以副將四人守之。舟次漢陽門,乘間跌入江水。四人懼誅,亦赴水。騰蛟漂十餘里,漁舟救之起,則漢前將軍關壯繆侯廟前也。家人懷印者亦至,相視大驚。覓漁舟,忽不見。遠近謂騰蛟忠誠得神佑,益歸心焉。

騰蛟乃從寧州轉瀏陽,抵長沙。集諸屬吏堵胤錫、傅上瑞、嚴起恒、章曠、周大啟、吳晉錫等,痛哭盟誓。分士馬舟艦糗糧,各任其一。令胤錫攝湖北巡撫,上瑞攝湖南巡撫,曠為總督監軍,大啟提督學政。起恒故衡永道,即督二郡軍食,晉錫以長沙推官攝郴桂道事。即遣曠調副將黃朝宣、張先璧、劉承胤兵。朝宣自燕子窩,先璧自漵浦,承胤自武岡,先後至,兵勢稍振。而是時良玉已死。

順治二年五月,大兵下南都。唐王聿鍵自立於福州。王居南陽時,素知騰蛟賢,委任益至。李自成斃於九宮山,其將劉體仁、郝搖旂等以眾無主,議歸騰蛟。率四五萬人驟入湘陰,距長沙百餘里。城中人不知其求歸也,懼甚。朝宣即引兵還燕子窩。上瑞請騰蛟出避,騰蛟曰:「死於左,死於賊,一也,何避焉。」長沙知府周二南請往偵之,以千人護行。賊謂其迎敵也,射殺之,從行者盡死。城中益懼,士女悉竄。騰蛟與曠謀,遣部將萬人鵬等二人往撫。賊見止二騎,迎入演武場,飲之酒。二人不交一言,與痛飲。飲畢,賊問來意,答言督師以湘陰褊小,不足容大軍,請即移長沙。因致騰蛟手書召之曰:「公等歸朝,誓永保富貴。」搖旂等大喜,與大鵬至長沙。騰蛟開誠撫慰,宴飲盡歡,犒從官牛酒。命先璧以卒三萬馳射,旌旗蔽天。搖旂等大悅,招其黨袁宗第、藺養成、王進才、牛有勇皆來歸,驟增兵十餘萬,聲威大震。

未幾,自成將李錦、高必正擁眾數十萬逼常德。騰蛟令胤錫撫降之,置之荊州。錦,自成從子,後賜名赤心。必正則自成妻高氏弟也。高氏語錦曰:「汝願為無賴,抑願為大將邪?」錦曰:「何謂也?」曰:「為賊無論,既以身許國,當愛民,受主將節制,有死無二,吾所願也。」錦曰:「諾。」騰蛟慮錦跋扈,他日過其營,請見高氏,再拜,執禮恭。高氏悅,戒其子毋忘何公,錦自是無異志。

自成亂天下二十年,陷帝都,覆廟社,其眾數十萬悉歸騰蛟。而騰蛟上疏,但言元兇已除,稍洩神人憤,宜告謝郊廟,卒不言己功。唐王大喜,立拜東閣大學士兼兵部尚書,封定興伯,仍督師。而疑自成死未實。騰蛟言自成定死,身首已糜爛。不敢居功,因固辭封爵。不允,令規取江西及南都。

當是時,降卒既眾,騰蛟欲以舊軍參之,乃題授朝宣、先璧為總兵官,與承胤、赤心、郝永忠、宗第、進才及董英、馬進忠、馬士秀、曹志建、王允成、盧鼎並開鎮湖南、北,時所謂十三鎮者也。永忠即搖旂,英,騰蛟中軍,志建則故巡按劉熙祚中軍,餘皆良玉舊將也。

騰蛟銳意東下,拜表出師。明年正月與監軍御史李膺品先赴湘陰,期大會岳州。先璧逗遛,諸營亦觀望,獨赤心自湖北至,為大兵所敗而還,諸鎮兵遂罷,騰蛟威望由此損。時諸將皆驕且貪殘,朝宣尤甚,劫人而剝其皮。永忠效之,殺民無虛日。騰蛟不能制。故總督楊鶚者,剋餉失軍心,至是復夤緣為偏沅總督。騰蛟以為言,乃召鶚還。

王數議出關,為鄭氏所阻。騰蛟屢請幸贛,協力取江西。王遣使徵兵,騰蛟發永忠精騎五千往。永忠不肯前,五月始抵郴州。會大兵破汀州,聿鍵被執死,贛州亦失。騰蛟聞王死,大慟,厲兵保境如平時。已,聞永明王立,乃稍自安。王尋以騰蛟為武英殿大學士,加太子太保。王進才故守益陽,聞大兵漸逼,還長沙。

四年春,進才揚言乏餉,大掠,並及湘陰。適大兵至長沙,進才走湖北。騰蛟不能守,單騎走衡州,長沙、湘陰並失。盧鼎時守衡州,而先璧兵突至,大掠。鼎不能抗,走永州。先璧遂挾騰蛟走祁陽,又間道走辰州。騰蛟脫還,走永州。甫至,鼎部將復大掠。鼎走道州,騰蛟與侍郎嚴起恒走白牙市,大兵遂下衡、永。初,騰蛟建十三鎮以衛長沙,至是皆自為盜賊。大兵入衡州,守將黃朝宣降。數其罪,支解之,遠近大快。大清以一知府守永州,副將周金湯瞷城虛,夜鼓噪而登,知府出走,金湯遂入永。

六月,騰蛟在白牙。王密遣中使告以劉承胤罪,令入武岡除之。騰蛟乃走謁王,王及太后皆召見。承胤由小校,以騰蛟薦至大將,已漸倨。騰蛟在長沙徵其兵,承胤大怒,言:「先調朝宣、先璧軍,皆章曠親行,今乃折箠使我。」遂馳至黎平,執騰蛟子,索餉數萬。子走訴騰蛟,騰蛟遣曠行,承胤乃以眾至。騰蛟為請於王,得封定蠻伯,且與為姻,承胤益驕。至是爵安國公,勛上柱國,賜尚方劍,益坐大。忌騰蛟出己上,欲奪其權,請用為戶部尚書,專領餉務,王不許。王召騰蛟圖承胤,騰蛟無兵,命以雲南援將趙印選、胡一青兵隸之。及辭朝,賜銀幣,命廷臣郊餞。承胤伏千騎襲騰蛟,印選卒力戰,盡殲之,騰蛟乃還駐白牙。

八月,大兵破武岡,承胤降。王走靖州,又走柳州。時常德、寶慶已失,永亦再失。王將返桂林,而城中止焦璉軍,騰蛟率印選、一青入為助。而南安侯郝永忠忽擁眾萬餘至,與璉兵欲鬥,會宜章伯盧鼎兵亦至,騰蛟為調劑,桂林以安。乃遣璉、永忠、鼎、印選、一青分扼興安、靈川、永寧、義寧諸州縣。十一月,大兵逼全州,騰蛟督五將合禦。

五年正月,王居桂林,加騰蛟太師,進爵為侯,子孫世襲。二月,大兵破全州,至興安。永忠兵大潰,奔桂林,逼王西,縱兵大掠。騰蛟自永福至。大兵知桂林有變,直抵北門。騰蛟督璉、一青等分三門拒守,大兵乃還全州。會金聲桓、李成棟叛大清,以兵附。大兵在湖南者姑退,騰蛟遂取全州。復遣保昌侯曹志建、宜章侯盧鼎、新興侯焦璉、新寧侯趙印選攻永州,圍城三月,大小三十六戰,十一月朔克之。未幾,監軍御史餘鯤起、職方主事李甲春取寶慶,諸將亦取衡州,馬進忠取常德,所失地多復。

騰蛟議進兵長沙。會督師堵胤錫惡進忠,招忠貞營李赤心軍自夔州至,令進忠讓常德與之。進忠大怒,盡驅居民出城,焚廬舍,走武岡。寶慶守將王進才亦棄城走,他守將皆潰。赤心等所至皆空城,旋棄走,東趨長沙。騰蛟時駐衡州,大駭。六年正月檄進忠由益陽出長沙,期諸將畢會,而親詣忠貞營,邀赤心入衡。部下卒六千人,懼忠貞營掩襲,不護行,止攜吏卒三十人往。將至,聞其軍已東,即尾之至湘潭。湘潭空城也,赤心不守而去,騰蛟乃入居之。大兵知騰蛟入空城,遣將徐勇引軍入。勇,騰蛟舊部將也,率其卒羅拜,勸騰蛟降。騰蛟大叱,勇遂擁之去。絕食七日,乃殺之。永明王聞之哀悼,賜祭者九,贈中湘王,謚文烈,官其子文瑞僉都御史。

章曠,字於野,松江華亭人。崇禎十年進士。授沔陽知州。十六年三月,賊將郝搖旂陷其城,同知馬飆死之。曠走免,謁總督袁繼咸於九江,署為監紀。從諸將方國安、毛憲文、馬進忠、王允成等復漢陽。武昌巡按御史黃澍令署漢陽推官兼攝府事,承德巡撫王揚基令署分巡道事。明年四月,憲文偕惠登相復德安,揚基檄曠往守。城空無人,衛官十數人齎印送賊將白旺。曠收斬之,日夕為警備。居三月,代者李藻至,巡撫何騰蛟檄曠署荊西道事。曠去,藻失將士心,城復陷。給事中熊汝霖、御史游有倫劾曠沔陽失城罪,侯訊黃州。用騰蛟薦,令戴罪立功。

福王立南京,左良玉將犯闕。騰蛟至長沙,以曠為監軍。副將黃朝宣者,故巡撫宋一鶴部將,駐燕子窩,騰蛟令曠召之來。副將張先璧屯精騎三千於漵浦,復屬曠召之,留為親軍,而以朝宣戍茶陵。又令曠調劉承胤兵於武岡。會李自成死,其下劉體仁、郝搖旂、袁宗第、藺養成、王進才、牛有勇六大部各擁數萬兵至。騰蛟與曠計,盡撫其眾,軍容大壯。左良玉死,其將馬進忠、王允成無所歸,突至岳州。偏沅巡撫傅上瑞大懼,曠曰:「此無主之兵,可撫也。」入其營,與進忠握手,指白水為誓,進忠等皆從之。進忠即賊中渠魁混十萬也。時南京已破,大兵逼湖南,諸將皆畏怯,曠獨悉力禦。唐王擢為右僉都御史,提督軍務,恢剿湖北。

曠有智略,行軍不避鋒鏑。身扼湘陰、平江之衝,湖南恃以無恐。嘗戰岳州,以後軍不繼而還。已,又大戰大荊驛。永明王加兵部右侍郎。長沙守將王進才與狼兵將覃遇春哄,大掠而去。騰蛟奔衡州,曠亦走寶慶,長沙遂失。騰蛟駐祁陽,曠來會。騰蛟以兵事屬曠,而謁王武岡。曠移駐永州,見諸大將擁兵,聞警輒走,抑鬱而卒。

傅作霖,武陵人。由鄉舉仕唐王,大學士蘇觀生奏為職方主事,監紀其軍。觀生歿,倚何騰蛟長沙,改監軍御史。永明王在全州,超拜兵部左侍郎,掌部事,尋進尚書,從至武岡。時劉承胤擅政,作霖與相善,故驟遷。及大兵逼武岡,承胤議迎降,作霖勃然責之。承胤遣使納款,大兵入城,作霖冠帶坐堂上。承胤力勸之降,不從,遂被殺。妾鄭有殊色,被執,驅之過橋,躍入水中死。

有蕭曠者,武昌諸生,為承胤坐營參將。騰蛟題為總兵官,管黎平參將事。及承胤降,令降將陳友龍招曠,曠不從。已而城破,死之。

傅上瑞,初為武昌推官,賊圍城,遁走。久之,騰蛟薦為長沙僉事,又令攝偏沅巡撫事。勸騰蛟設十三鎮,卒為湖南大害。唐王時,用騰蛟薦,擢右僉都御史,實授偏沅巡撫。性反覆,棄騰蛟如遺。武岡破,大兵逼沅州,上瑞出降。踰年,與劉承胤並誅死。

瞿式耜,字起田,常熟人。禮部侍郎景淳孫,湖廣參議汝說子也。舉萬歷四十四年進士。授吉安永豐知縣,有惠政。天啟元年調江陵。永豐民乞留,命再任。以憂歸。崇禎元年,擢戶科給事中,疏言李國普宜留內閣,王永光宜典銓,曹于汴宜秉憲,鄭三俊、畢懋良宜總版曹,李邦華宜主戎政。帝多采其言。俄陳朝政不平,為王之寀請恤,孫慎行訟冤,速楊鎬、王化貞之誅,白楊漣、左光斗結毒之謗,追論故相魏廣微、顧秉謙、馮銓、黃立極之罪。因言奪情建祠之朱童蒙不可寬,積愆久廢之湯賓尹不可用。帝亦納之。又極論來宗道、楊景辰附逆不可居政府,二人旋罷去。御史袁弘勛劾大學士劉鴻訓,逆黨徐大化實主之。川貴總督張鶴鳴先已被廢,其復用由魏忠賢。式耜並疏論。已,頌楊漣、魏大中、周順昌為清中之清,忠中之忠,三人遂賜謚。未幾,陳時務七事,言:「起廢不可不核,升遷不可不漸,會推不可不慎。謚典宜嚴,刑章宜飭,論人宜審,附璫者宜區分。」又極論館選奔競之弊,請臨軒親試。末言:「古有左右史,記天子言動。今召對時勤,宜令史官入侍紀錄,昭示朝野。」事多議行。時將定逆案,請盡發紅本,定其情罪輕重。又言宣府巡撫徐良彥不附逆奄,為崔呈秀誣劾遣戍,亟當登用。良彥遂獲起。

式耜矯矯立名,所建白多當帝意,然搏擊權豪,大臣多畏其口。十月詔會推閣臣,禮部侍郎錢謙益以同官周延儒方言事蒙眷,慮並推則己絀,謀沮之。式耜,謙益門人也,言於當事者,擯延儒弗推,而列謙益第二。溫體仁遂發難,延儒助之。謙益奪官閒住,式耜坐貶謫。式耜嘗頌貴寧參政胡平表殺賊功,請優擢。其後平表為貴州布政使,坐不謹罷。式耜再貶二秩,遂廢於家。久之,常熟奸民張漢儒希體仁指,訐謙益、式耜貪肆不法。體仁主之,下法司逮治。巡撫張國維、巡按路振飛交章白其冤,不聽。比兩人就獄,則體仁已去位,獄稍解。謙益坐削籍,式耜贖徒。言官疏薦,不納。

十七年,福王立於南京。八月起式耜應天府丞。已,擢右僉都御史,代方震孺巡撫廣西。明年夏,甫抵梧州,聞南京破。靖江王亨嘉謀僭號,召式耜。拒不往,而檄思恩參將陳邦傳助防。止狼兵,勿應亨嘉調。亨嘉至梧,劫式耜,幽之桂林,遣入取其敕印。初,式耜議立桂端王子安仁王。及唐王監國,式耜以為倫序不當立,不奉表勸進。至是為亨嘉所幽,乃遣使賀王,因乞援。王喜,而亨嘉為丁魁楚所攻,勢窘,乃釋式耜。式耜與中軍官焦璉召邦傳共執亨嘉,亂遂定。唐王擢式耜兵部右侍郎,協理戎政,以晏日曙來代。式耜不入朝,退居廣東。

順治三年九月,大兵破汀州。式耜與魁楚等議立永明王由榔,乃迎王梧州,以十月十日監國肇慶。進式耜吏部右侍郎、東閣大學士,兼掌吏部事。未幾,贛州敗報至,司禮王坤迫王赴梧州。式耜力爭,不得。十一月朔,蘇觀生立唐王聿於廣州。式耜乃與魁楚等定議迎王還肇慶,遣總督林佳鼎禦觀生兵,敗歿。式耜視師峽口。十二月望,大兵破廣州。王坤趣王西走。式耜趨赴王,王已越梧而西。

四年正月,大兵破肇慶,逼梧州,巡撫曹曄迎降。王欲走依何騰蛟於湖廣,丁魁楚、呂大器、王化澄皆棄王去,止式耜及吳炳、吳貞毓等從,乃由平樂抵桂林。二月,大兵襲平樂,分兵趨桂林。王將走全州,式耜極陳桂林形勢,請留,不許。自請留守,許之。進文淵閣大學士,兼兵部尚書,賜劍,便宜從事。平樂、潯州相繼破,桂林危甚。總督侍郎朱盛濃走靈川,巡按御史辜延泰走融縣,布政使朱盛水調、副使楊垂雲、桂林知府王惠卿以下皆遁,惟式耜與通判鄭國籓,縣丞李世榮及都司林應昌、李當瑞、沈煌在焉。王令兵部右侍郎丁元曄代盛濃,御史魯可藻代延泰。未赴而大兵已於三月薄桂林,以騎數十突入文昌門,登城樓瞰式耜公署。式耜急令援將焦璉拒戰。

初,永明王為賊執,璉率眾攀城上,破械出之。王病不能行,璉負王以行。王以此德璉,用破靖江王功,命為參將。及是戰守三月,璉功最多,元曄、可藻亦盡力。式耜身立矢石中,與士卒同甘苦。積雨城壞,吏士無人色,式耜督城守自如,故人無叛志。援兵索餉而嘩,式耜括庫不足,妻邵捐簪珥佐之。既而璉兵主客不和,噪而去,城幾破者數矣。會陳邦彥等攻廣州,大兵引而東,桂林獲全。璉亦復陽朔及平樂,陳邦傳亦由潯復梧州。王聞捷,封式耜臨桂伯,璉新興伯,元曄等進秩有差。

式耜初請王返全州,不聽。已,請還桂林。王已許之,會武岡破,王由靖州走柳州,式耜復請還桂林。十一月,大兵自湖南逼全州,式耜偕騰蛟拒卻。已,梧州復破,王方在象州,欲走南寧。以大臣力爭,乃以十二月還桂林。

五年二月,南安侯郝永忠駐桂林,惡城外團練兵,盡破水東十八村,殺戮無算,與式耜構難。式耜力調劑,永忠乃駐興安。大兵前驅至靈川,永忠戰敗,奔入桂林,請王即夕西走。式耜力爭,不聽。左右皆請速駕,式耜又爭。王曰:「卿不過欲予死社稷爾。」式耜為泣下沾衣。王甫行,永忠即大掠,捶殺太常卿黃太元。式耜家亦被掠,家人矯騰蛟令箭,乃出城。日中,趙印選諸營自靈川至,亦大掠,城內外如洗。永忠走柳州,印選等走永寧。明日,式耜息城中餘燼,安撫遠近。焦璉及諸鎮周金、湯兆佐、胡一青等各率所部至,騰蛟軍亦至。三月,大兵知桂林有變,來襲,抵北門。騰蛟督諸將拒戰,城獲全。時王駐南寧,式耜遣使慰三宮起居。王始知式耜無恙,為泣下。

閏三月,廣東李成棟、江西金聲桓皆叛大清,據地歸,式耜請王還桂林。王從成棟請,將赴廣州。式耜慮成棟挾王自專,如劉承胤事,力爭之,乃駐肇慶。十一月,永州、寶慶、衡州並復。式耜以機會可乘,請王還桂林,圖出楚之計,不納。慶國公陳邦傳守潯州,自稱世守廣西,欲如黔國公例。式耜特疏劾之,會中外多爭者,邦傳乃止。廣西巡撫魯可藻自署銜巡撫兩廣,式耜亦疏駮之。式耜身在外,政有闕,必疏諫。嘗曰:「臣與主上患難相隨,休戚與共,不同他臣。一切大政,自得與聞。」王為褒納。而是時成棟子元胤專朝政,知敬式耜,袁彭年、丁時魁、金堡等遂爭相倚附。六年正月,時魁等逐朱天麟,不欲何吾騶為首輔。召式耜入直,以文淵印畀之,式耜終不入也。未幾,騰蛟、聲桓、成棟相繼敗歿,國勢大危。朝士方植黨相角,式耜不能禁。

七年正月,南雄破。王懼,走梧州。諸大臣訐時魁等下獄,式耜七疏論救。胡執恭之擅封孫可望也,式耜疏請斬之。皆不納。九月,全州破。開國公趙印選居桂林,衛國公胡一青守榕江,與寧遠伯王永祚皆懼不出兵,大兵遂入嚴關。十月,一青、永祚入桂林分餉,榕江無戍兵,大兵益深入。十一月五日,式耜檄印選出,不肯行,再趣之,則盡室逃。一青及武陵侯楊國棟、綏寧伯蒲纓、寧武伯馬養麟亦逃去。永祚迎降,城中無一兵。式耜端坐府中,家人亦散。部將戚良勛請式耜上馬速走,式耜堅不聽,叱退之。俄總督張同敞至,誓偕死,乃相對飲酒,一老兵侍。召中軍徐高付以敕印,屬馳送王。是夕,兩人秉燭危坐。黎明,數騎至。式耜曰:「吾兩人待死久矣」,遂與偕行,至則踞坐於地。諭之降,不聽,幽於民舍。兩人日賦詩倡和,得百餘首。至閏十一月十有七日,將就刑,天大雷電,空中震擊者三,遠近稱異,遂與同敞俱死。同敞,大學士居正曾孫,事見《居正傳》。

時桂林殉難者光祿少卿汪皞投水死。其破平樂也,守將鎮西將軍朱旻如自剄。

有周震者,官中書舍人,居全州,慷慨尚氣節,武岡失,全州危,震邀文武將吏盟於神,誓死拒守。條城守事宜,上之留守瞿式耜。式耜即題為御史,監全州軍。無何,郝永忠、盧鼎自全州撤兵還桂林。守全諸將議舉城降,震力爭不可,眾怒殺之,全州遂失。

贊曰:何騰蛟、瞿式耜崎嶇危難之中,介然以艱貞自守。雖其設施經畫,未能一睹厥效,要亦時勢使然。其於鞠躬盡瘁之操,無少虧損,固未可以是為訾議也。夫節義必窮而後見,如二人之竭力致死,靡有二心,所謂百折不回者矣。明代二百七十餘年養士之報,其在斯乎!其在斯乎!


\end{pinyinscope}