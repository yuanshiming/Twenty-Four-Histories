\article{列傳第四十四}

\begin{pinyinscope}
吳允誠子克忠孫瑾薛斌子綬弟貴李賢吳成滕定金順金忠蔣信李英從子文毛勝焦禮毛忠孫銳和勇羅秉忠

吳允誠,蒙古人。名把都帖木兒,居甘肅塞外塔溝地,官至平章。永樂三年,與其黨倫都兒灰率妻子及部落五千、馬駝萬六千,因宋晟來歸。帝以蒙古人多同名,當賜姓別之。尚書劉俊請如洪武故事,編為勘合。允誠得賜姓名,授右軍都督僉事。倫都兒灰亦賜姓名柴秉誠,授後軍都督僉事。余授官賜冠帶,給畜產鈔幣有差。使領所部居涼州耕牧。晟以招徠功,封西寧侯。自是降附者益眾,邊境日安,由允誠始。

七年往亦集乃覘敵,擒哈剌等二十餘人,進都督同知。明年從出塞,敗本雅失里,進右都督。尋進左都督。與中官王安追闊脫赤,至把力河獲之。封恭順伯,食祿千二百石,予世券。允誠三子:答蘭、管者、克勤。允誠與二子從軍,留其妻及管者居涼州。番人虎保等誘脅允誠眾,欲叛去。允誠妻與管者謀,召部將都指揮保住、卜顏不花等擒其黨,誅之。帝喜,降敕獎之,賜縑鈔羊米甚厚,授管者指揮僉事。保住賜姓名楊效誠,授指揮僉事。韃靼可汗鬼力赤遇弒,其下多潰。答蘭與別立哥請出塞自效,有功。別立哥者,秉誠子也。

帝征瓦剌,允誠父子皆從。師還,命仍居涼州備邊。允誠卒,贈國公,謚忠壯。

命答蘭更名克忠,襲其爵。再徵阿魯台,從行。三征阿魯台,復從。兄弟皆有功。洪熙元年以戚里恩,克忠進侯。時管者已積功至都指揮同知,亦封廣義伯。克忠嘗充副總兵巡邊。正統九年統兵出喜峰口,徵兀良哈,有功,加太子太保。

土木之變,克忠與其弟都督克勤子瑾為後拒。寇突至,驟戰不勝。敵兵據山上,飛矢石如雨,官軍死傷略盡。克忠下馬射,矢竭,猶殺數人,與克勤俱歿於陣。贈邠國公,謚忠勇。克勤贈遵化伯,謚僖敏。

瑾被執,逃歸,嗣侯。英宗嘗欲使瑾守甘肅,辭曰:「臣,外人,若用臣守邊,恐外裔輕中國。」帝善其言,乃止。曹欽反,瑾與從弟琮聞變,椎長安門上告。門閉,欽攻不得入,遂縱火。瑾將五六騎與欽力戰死。贈涼國公,謚忠壯,予世券。

三傳至曾孫繼爵。嘗守備南京。傳子汝胤孫惟英,與繼爵皆總督京營戎政。崇禎末,都城陷,汝胤弟勛衛汝征偕妻女投繯死。

管者卒,子嗣。管者妻早奴亦有智略,嘗親入朝獻良馬。朝廷多其忠。卒,管者弟克勤子琮嗣,鎮守寧夏。成化四年,滿四反。琮坐激變,且臨陣先退,下獄論死。謫戍邊,爵除。

薛斌,蒙古人,本名脫歡。父薛台,洪武中歸附,賜姓薛,累官燕山右護衛指揮僉事。斌嗣職,從起兵,累遷都督僉事。從北征有功,進都督同知。永樂十八年封永順伯,祿九百石,世指揮使。

斌卒,子壽童方五歲。從父貴引見仁宗,立命嗣伯,賜名綬。長,驍勇善戰。正統十四年秋,與成國公朱勇等遇敵於鷂兒嶺。軍敗,弦斷矢盡,猶持空弓擊敵。敵怒,支解之。既而知其本蒙古人也,曰:「此吾同類,宜勇健若此。」相與哭之。謚武毅。子輔,孫勛,並得嗣伯。勳子璽乃嗣指揮使,如券文。

貴,本名脫火赤,斌之弟。以舍人從燕王起兵,屢脫王於險。積官都指揮使。再從北征,進都督僉事。永樂二十年封安順伯,祿九百石。宣德元年進侯,加祿三百石,予世券。卒,贈濱國公,謚忠勇。無子,從子山嗣為指揮使。天順改元,以復辟恩,命山子忠嗣伯。卒,子瑤嗣。弘治中卒,子昂降襲指揮使。

李賢,初名丑驢,韃靼人。元工部尚書。洪武二十一年來歸,通譯書。太祖賜姓名,授燕府紀善。侍燕世子最恭謹。「靖難」師起,有勞績,累遷都指揮同知。凡塞外表奏及朝廷所降詔敕,皆命賢譯。賢亦屢陳所見,成祖皆採納之。仁宗即位,念舊勞,進後軍都督僉事,再進右都督,賜賚甚渥。尋召見,憫其病,封忠勤伯。食祿千一百石。尋卒。

吳成,遼陽人,初名買驢。父通伯,元遼陽行省右丞。太祖時,觀童來降,通伯父子與俱。買驢更今姓名,充總旗,數從大軍出塞。建文元年授永平衛百戶。降燕,從戰皆有功,三遷都指揮僉事,始知名。南軍聞吳買驢名,多於陣上指目之。設伏淝河,進兵小河,合戰齊眉山,攻敗靈璧軍,皆殊死鬥,功多。

成祖即位,授都指揮使。從征本雅失里。疾戰,本雅失里以七騎遁。從征阿魯台,合朱榮兵為前鋒,追至闊灣海。召還,進都督僉事。又三從出塞。洪熙元年進左都督。從陽武侯薛祿徵大松嶺,為前鋒,有功,增祿米。宣宗初,以成嘗宿衛東宮,錄舊勞,封清平伯,祿千一百石,予世券。從征樂安,復與薛祿為前鋒。事定,出守備興和。成好畋獵而不修武備。寇伺其出獵,卒入城,掠其妻孥以去。帝聞之,置不罪。已而阿魯台入貢,還其家口。三年,帝北征,從敗賊於寬河,進侯,祿如故。八年卒。贈渠國公,謚壯勇。

子忠前死,忠子英嗣伯。卒,子璽嗣。坐貪淫奪爵。久乃復之。卒,無子,從弟琮嗣。四傳至玄孫遵周。崇禎末,京師陷,被殺。

滕定,父瓚住,元樞密知院。洪武中,來降。授會州衛指揮僉事,賜姓滕。從燕起兵,進燕山右衛指揮使。卒,定嗣官,屢從出塞,有功,進至都督僉事。宣德四年封奉化伯,祿八百石。正統初卒。子福嗣,為指揮使。

金順,本名阿魯哥失里。永樂中來降,授大寧都指揮僉事。從敗本雅失里,又敗阿魯台,累進都督僉事。宣德三年從巡北邊,有斬捕功。明年封順義伯,祿八百石。卒,子忠嗣,為指揮僉事。

金忠者,蒙古王子也先土乾也。素桀黠,為阿魯台所忌。永樂二十一年,成祖親征漠北,至上莊堡,率妻子部屬來降。時六師深入,寇已遠遁。帝方恥無功,見其來歸,大喜。賜姓名,封忠勇王,賜冠帶織金襲衣,命坐列侯下。輟御前珍羞賜之,復賜金銀寶器。忠大喜過望。班師在道,忠騎從,數問寇中事,眷寵日隆。明年,忠請為前鋒,討阿魯台自效。帝初不許。會大同、開平警報至,諸將請從忠言。帝復出塞,忠與陳懋為前鋒。而阿魯台聞王師復出,倉皇渡答蘭納木兒河遁去。忠、懋至河不見寇,抵白邙山,卒無所遇,乃班師。仁宗嗣位,加太子太保,並支二俸。

宣德三年親征兀良哈,敗寇於寬河。忠與把台請自效,帝許之。或言不可遣,帝曰:「去留任所欲耳。朕有天下,獨少此二人邪。」二人獲數十人、馬牛數百來獻。帝喜,命中官酌以金卮,遂賜之。明年加太保。六年秋卒。命有司治喪葬。

把台者,忠之甥,從忠來降,授都督僉事。宣德初,賜姓名蔣信。正統中,封忠勇伯。從駕陷土木,也先使隸賽罕王帳下。信雖居朔漠,志常在中國。每詣上皇所慟哭,擁衛頗至。已,竟從駕還,詔復給其祿。景泰五年卒。贈侯,謚僖順。子也兒索忽襲爵。天順初,更名善。弘治中卒。無子,爵絕。

李英,西番人。父南哥,洪武中率眾歸附,授西寧州同知,累功進西寧衛指揮僉事。英嗣官。

永樂十年,番酋老的罕叛,英擊之。討來川,俘斬三百六十人。夜雪,賊遁,追盡獲之,進都指揮僉事。番僧張答里麻者,通譯書。成祖授以左覺義。居西寧,恣甚。以計取西番貢使資,納逋逃,交通外域,肆惡十餘年。英發其事,磔死,籍其家。西陲快之。

末年,中官喬來喜、鄧誠等使西域,道安定、曲先,遇賊見殺,掠所齎金幣。仁宗璽書諭赤斤、罕東及安定、曲先,詰賊主名。而敕英與土官指揮康壽等進討。英詗知安定指揮哈三孫散哥、曲先指揮散即思實殺使者,遂率兵西入。賊驚走。追擊,踰崑崙山,深入數百里。至雅令闊,與安定賊遇,大敗之,俘斬千一百餘人,獲馬牛雜畜十四萬。曲先賊聞風遠遁,安定王桑爾加失夾等懼,詣闕謝罪。宣宗嘉英功,遣使褒諭,宴勞之,令馳驛入朝。既至,擢右府左都督,賜賚加等。宣德二年封會寧伯,祿千一百石,并封南哥如子爵。

英恃功而驕,所為多不法。寧夏總兵官史昭奏英父子有異志。南哥上章辯。賜敕慰諭之。英家西寧,招逋逃七百餘戶,置莊墾田,豪奪人產,復為兵部及言官所劾。帝宥英,追逃者入官。七年,西寧指揮祁震子成當襲父職。庶兄監藏,英甥也,欲奪之。成從祖太平攜成赴京辯。英遣人篡取太平及其義兒杖之,義兒竟死。言官交劾,並及前罪,遂下英詔獄,奪爵論死。正統二年始釋。後稍給其祿。尋卒。英宗復辟,官其子昶錦衣指揮同知。尋進都指揮使,用薦擢左軍都督僉事,屢分典營務,以嚴慎稱。

英從子文,宣德間為陜西行都司都指揮僉事。西番思俄可嘗盜他部善馬,都指揮穆肅求不得。會思俄可以畜產鬻於邊,肅誣以盜,收掠致死,番人惶駭思亂。文劾之,逮肅下吏,西陲以寧。累官都指揮使。

天順元年冒迎駕功,進都督僉事。未幾,以右都督出鎮大同。寇二千餘騎犯威遠,文率師敗之,封高陽伯。石亨敗,革奪門冒功者官。文自首,帝以守邊不問。

四年秋,孛來大舉入寇,文按兵不戰。遂入鴈門,大掠忻、代諸州。京師震恐。寇退,徵文下詔獄,論斬。帝宥文死,降都督僉事,立功延綏。既而進都督同知。成化中,哈密為土魯番所併,求救於朝。詔文與右通政劉文往甘肅經略之,無功而還。弘治初卒。正德初贈高陽伯。

毛勝,字用欽,初名福壽,元右丞相伯卜花之孫。伯父那海,洪武中歸附,以「靖難」功至都指揮同知,無子。勝父安太嗣為羽林指揮使,傳子濟,無子,勝嗣。論濟征北功,進都指揮使。嘗逃歸塞外,尋復自還。

正統七年以征麓川功,擢都督僉事。靖遠伯王驥請選在京番將舍人,捕苗雲南。乃命勝與都督冉保統六百人往。已,再徵麓川,即命二人充左右參將。賊平,進都督同知。

十四年夏,也先謀入寇,勝偕平鄉伯陳懷等率京軍三萬鎮大同。懷遇寇戰歿,勝脫還。以武清伯石亨薦,景帝進勝左都督,督三千營操練。

貴州苗大擾,詔勝往討。未行而也先逼京師。勝禦之彰義門北,擊退之。越二日,引兵西直門外,解都督孫鏜圍。明日,都督武興戰歿於彰義門,寇乘勝進。勝與都御史王竑急援之,寇遂引卻。勝追襲至紫荊關,頗有斬獲。事定,乃命以左副總兵統河間、東昌降夷赴貴州。賊首韋同烈據香壚山作亂,勝與總兵梁珤、右副總兵方瑛等從總督王來分道夾擊。勝進自重安江,大破之。會師山下,環四面攻之。賊窘,縛同烈降。

還討湖廣巴馬諸處反賊,克二十餘寨,擒賊首吳奉先等百四十人,斬首千餘級,封南寧伯,予世券。疏請更名,從之。移鎮騰衝。金齒芒市長官刀放革潛結麓川遺孽思卜發為變,勝設策擒之。

巡按御史牟俸劾其貪暴不法數十事,且言勝本降人,狡猾難制,今又數通外夷,恐貽邊患。詔巡撫覆實,卒置不問。天順二年卒。贈侯,謚莊毅。

子榮嗣。坐石亨黨,發廣西立功。成化初,鎮貴州,尋移兩廣。卒,子文嗣。弘治初協守南京,傳爵至明亡乃絕。

焦禮,字尚節,蒙古人。父把思台,洪武中歸附,為通州衛指揮僉事。子勝嗣,傳至義榮,無子,以勝弟謙嗣,累功至都指揮同知。卒,子管失奴幼,謙弟禮借襲其職,備禦遼東。

宣德初,禮當還職。宣宗念禮守邊勞,命居職如故,別授管失奴指揮使。禮尋以年勞,累進都指揮同知。正統中,積功至右都督。英宗北狩,景帝命充左副總兵,守寧遠。未幾,也先逼京城,詔禮率師入衛。寇退還鎮。景泰四年,賊二千餘騎犯興水堡,禮擊走之。璽書獎勵,進左都督。

英宗復辟,以禮守邊有功,召入覲。封東寧伯,世襲,賜賚甚厚。遣還鎮。兵部以禮年垂八十,不可獨任,奏遣都指揮鄧鐸協同守備。居無何,禮奏鐸欺侮,請更調。命都指揮張俊代鐸。天順七年卒於鎮。贈侯,謚襄毅。

禮有膽略,精騎射,善以少擊眾。守寧遠三十餘年,士卒樂為用,邊陲寧謐。

孫壽嗣爵。卒,無子,弟俊嗣。成化末,歷鎮甘肅、寧夏。弘治中,掌南京前府,兼督操江。出鎮貴州、湖廣。俊少事商販,既貴,能下士,而折衝非所長。卒,子淇嗣。嘗分典京營。正德中,賄劉瑾,出鎮兩廣。踰年卒,弟洵嗣。洵雖嗣爵,先業盡為淇妻所有。生母卒,無以葬,哀憤得疾卒。無子,以再從子棟嗣。嘉靖中,提督五軍營,兼掌中府。踰十年,改總兵湖廣。卒,贈太子太保,謚莊僖。傳爵至明亡乃絕。

毛忠,字允誠,初名哈喇,西陲人。曾祖哈喇歹,洪武初歸附,起行伍為千戶,戰歿。祖拜都從征哈密,亦戰歿。父寶以驍勇充總旗,至永昌百戶。

忠襲職時,年二十。膂力絕人,善騎射。常從太宗北征。宣德五年征曲先叛寇,有功。八年征亦不剌山,擒偽少師知院。九年出脫歡山,十年征黑山寇,皆擒其酋。各進一官,歷指揮同知。

正統三年,從都督蔣貴徵朵兒只伯,先登陷陣,大獲,擢都指揮僉事。十年以守邊勞,進同知,始賜姓。明年,從總兵官任禮收捕沙洲衛都督喃哥部落,徙之塞內,進都指揮使。十三年率師至罕東,生縶喃哥弟偽祁王鎖南奔並其部眾,擢都督僉事,始賜名忠。尋充右參將,協守甘肅。

景泰初,侍郎李實使漠北,還言忠數遣使通瓦剌。詔執赴京。既至,兵部論其罪,請置大辟。景帝不許。請貶官,發福建立功。乃遣之福建,而官秩如故。令甘肅守臣徙其家屬京師。初,忠之徵沙漠也,獲番僧加失領真以獻。英宗赦不誅。後逃之瓦剌,為也先用。憾忠,欲陷之。遂宣言忠與也先交通,而朝廷不察也。英宗在塞外獨知之,比復辟,即召還。而忠在福建亦屢有斬馘功,乃擢都督同知,充左副總兵,鎮守甘肅。陛見,慰諭甚至,賜玉帶、織金蟒衣。

天順二年,寇大入甘肅,巡撫芮釗劾奏諸將失事罪。部議忠功足贖罪,置不問。三年以鎮番破賊功,進左都督。五年,孛來以數萬騎分掠西寧、莊浪、甘肅諸道,入涼州。忠鏖戰一日夜,矢盡力疲。賊來益眾,軍中皆失色。忠意氣彌厲,拊循將士,復殊死鬥。賊見終不可勝,而援軍亦至,遂解去。忠竟全師還。七年,永昌、涼州、莊浪塞外諸番屢為邊患。忠與總兵官衛穎分討之。忠先破巴哇諸大族。其昝咂、馬吉思諸族,他將不能下者,忠復擊破之。論功,忠止增祿百石,而穎乃得世券。忠以為言,遂封伏羌伯。

成化四年,固原賊滿四據石城反。詔忠移師討之,與總督項忠等夾攻賊巢。忠由木頭溝直抵炮架山下,多所斬獲,賊稍卻。冒矢石連奪山北、山西兩峰,而項忠等軍亦克山之東峰。及石城東、西二門,賊大窘,相對哭。忽昏霧起,他哨舉煙掣軍,賊遂併力攻忠。忠力戰不已,為流矢所中,卒,年七十五。從子海、孫鎧前救忠,亦死。

忠為將嚴紀律,善撫士。其卒也,西陲人弔哭者相望於道。事聞,贈侯,謚武勇,予世券。弘治中,從有司言,建忠義坊於蘭州,以表其里。又從巡撫許進言,建武勇祠於甘州城東,春秋致祭。

孫銳,襲伯爵。成化中,協守南京。弘治初,出鎮湖廣,改兩廣。平蠻賊,累有功,咸璽書獎勵。九年以廣西破賊,增歲祿二百石。言官劾銳廣置邸舍,私造大舶以通番商。置不問。思恩土官岑浚反,與總督潘蕃討平之。既又討平賀縣僮賊。加官至太子太傅。正德三年,劉瑾欲殺尚書劉大夏,坐以處置田州事失宜,並逮銳下詔獄。獄具,革其加官並歲祿五百石。已而賄瑾,起督漕運。踰年,瑾誅,被劾罷。六年,盜劉宸等擾畿甸,命銳與中官谷大用討之。所統京軍皆驕惰不習戰。明年正月,遇賊於長垣,與戰大敗,身被傷,亡將印。會許泰援軍至,僅免。言官交劾,乃召還。以與大用同事,竟不罪。世宗即位,復起鎮湖廣。居三年卒。贈太傅,謚威襄。

傳子江及漢。漢,嘉靖中掌南京左府,提督操江,改總督漕運。未上,給事中楊上林劾其所至貪墨,詔褫職逮問。卒,無子,從子桓嗣。卒,子登嗣。萬曆中,掌中軍府事垂二十年。又再傳而明亡。

和勇,初名脫脫孛羅,和寧王阿魯台孫也。阿魯台既為瓦剌脫歡所殺,子阿卜只俺窮蹙,款塞來歸。宣宗授以左都督,賜第京師。卒,勇襲指揮使,帶俸錦衣衛,積功至都督僉事。天順元年詔加同知,賜姓名。久之以兩廣多寇,命充遊擊將軍,統降夷千人往討。時總兵顏彪無將略,賊勢愈熾。廣西巡撫吳禎殺降冒功,得優賞。彪效之,亦殺平民報捷。朝廷進彪官,勇亦進右都督。既而師久無功,言官劾文武將吏之失事者。詔停勇俸,充為事官。

成化初,趙輔、韓雍征大藤峽賊,詔勇以所部從征。其冬,賊大破,進左都督,增祿百石。三年召督效勇營訓練。尋上言:「大藤峽之役,臣與趙輔同功。輔還京,餘賊復叛。臣親搗賊巢,縶其魁,誅其黨,還被掠男女四千人。今輔已封伯,而臣止進秩,惟陛下憐察。」憲宗以勇再著戰功,特封靖安伯。十年卒。謚武敏,世襲指揮使。

勇性廉謹。在兩廣時,諸將多營私漁利,勇獨無所取。時論稱之。

羅秉忠,初名克羅俄領占,沙州衛都督僉事困即來子也。兄喃哥既襲父職,英宗復命秉忠為指揮使,協理衛事。既而喃哥率千二百人內徙,詔居之東昌、平山二衛,給田廬什器,所以撫恤甚厚。喃哥卒,秉忠為都指揮使,代領其眾。

英宗北狩,塞上多警。朝議恐降人乘機為變,欲徙之南方。會貴州苗亂,都督毛福壽南征,即擢秉忠都督僉事,率所部援剿。積戰功至左都督。天順初,始賜姓名。曹欽之反,番官多從之者。秉忠亦坐下獄,籍其家。久之,上章自辯,乃得釋。成化初,尚書程信討山都掌蠻,秉忠以遊擊將軍從。既抵永寧,分兵六道。秉忠由金鵝江進,大破之。論功,封順義伯。十六年卒。謚榮壯,子孫世指揮使。

贊曰:明興,諸番部懷太祖功德,多樂內附,賜姓名授官職者不可勝紀。繼以成祖銳意遠圖,震耀威武,於是吳允誠、金忠之徒,率眾來屬,遂得列爵授任,比肩勛舊。或以戰功自奮,錫券受封,傳世不絕。視夫陸梁倔強者,順逆殊異,不其昭歟!土木以還,勢以不競,邊政日弛,火篩、俺答諸部騷動無寧歲。盛衰之故概可攷焉。


\end{pinyinscope}