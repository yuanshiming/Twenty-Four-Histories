\article{列傳第一百六十五}

\begin{pinyinscope}
袁繼咸張亮金聲江天一丘祖德溫璜吳應箕尹民興等沈猶龍李待問章簡陳子龍夏允彞徐孚遠侯峒曾{{閻應元等

硃集璜等楊文驄孫臨等陳潛夫陸培沈廷揚林汝翥林惣鄭為虹黃大鵬王士和胡上琛熊緯

袁繼咸,字季通,宜春人。天啟五年進士。授行人。崇禎三年冬,擢御史,監臨會試,坐縱懷挾舉子,謫南京行人司副,遷主客員外郎。七年春,擢山西提學僉事。未行,總理戶、工二部中官張彞憲有朝覲官齎冊之奏。繼咸疏論之,謂:「此令行,上自籓臬,下至守令,莫不次第參謁,屏息低眉,跪拜於中官之座,率天下為無恥事,大不便。」彞憲大恚,與繼咸互訐奏。帝不聽,乃孑身赴任。久之,巡撫吳甡薦其廉能。而巡按御史張孫振以請屬不應,疏誣繼咸臟私事。帝怒,逮繼咸,責甡回奏。甡賢繼咸,斥孫振。諸生隨至都,伏闕訴冤,繼咸亦列上孫振請屬狀及其贓賄數事。詔逮孫振,坐謫戍;繼咸得復官。十年,除湖廣參議,分守武昌。以兵搗江賊巢興國、大冶山中,擒賊首呂瘦子,降其黨十餘人。詔兼僉事,分巡武昌、黃州。擊退賊老回回、革裏眼等七大部黃陂、黃安,築黃岡城六千餘丈。

十二年,移淮陽,忤中官楊顯名,奏鐫二秩調用。督師楊嗣昌以其知兵,引參軍事。明年四月擢右僉都御史,撫治鄖陽。未一年,襄陽陷,被逮,戍貴州。十五年,廷臣交薦,起故官,總理河北屯政。未赴,賊逼江西。廷議設重臣總督江西、湖廣、應天、安慶軍務,駐九江。擢繼咸兵部右侍郎兼右僉都御史以行。賊已陷武昌,左良玉擁兵東下。繼咸遇良玉於蕪湖,激以忠義。良玉即還,恢復武昌。廷議呂大器來代,繼咸仍督屯政。大器、良玉不協,長沙、袁州俱陷,仍推繼咸代之。甫抵鎮而京師陷。

福王立南都,頒詔武昌,良玉不拜詔。繼咸致書言倫序正,良玉乃拜受詔。繼咸入朝,高傑新封興平伯。繼咸曰:「封爵以勸有功。無功而封,有功者不勸。跋扈而封,跋扈愈多。」王曰:「事已行,奈何?」繼咸曰:「馬士英引傑渡江,宜令往輯。」王曰:「彼不欲往,輔臣史可法願往。」繼咸曰:「陛下嗣位,固以恩澤收人心,尤宜以紀綱肅眾志。乞振精神,申法紀。冬春間,淮上未必無事。臣雖駑,願奉六龍為澶淵之舉。」王有難色。因詣榻前密奏曰:「左良玉雖無異圖,然所部多降將,非孝子順孫。陛下初登大寶,人心危疑,意外不可不慮,臣當星馳回鎮。」許之。因赴閣責可法不當封傑,士英嗛之。俄陳致治守邦大計,引宋高宗用黃潛善、汪伯彥事,語復侵士英。會湖廣巡按御史黃澍劾奏士英十大罪,士英擬旨逮治。澍與良玉謀,陰諷將士大嘩,欲下南京索餉,保救澍。繼咸為留江漕十萬石、餉十三萬金給之,且代澍申理,以良玉依仗澍為言。士英不得已,免逮澍。繼咸既與士英隙,所奏悉停寢。

明年正月,繼咸言:「元朔者,人臣拜手稱觴之日,陛下嘗膽臥薪之時。念大恥未雪,宜以周宣之未央問夜為可法,以晚近長夜之飲、角牴之戲為可戒。省土木之功,節浮淫之費。戒諭臣工,後私鬥而急公仇。臣每歎三十年來,徒以三案葛藤血戰不已。若《要典》一書,已經先帝焚毀,何必復理其說。書茍未進,宜寢之;即已進,宜毀之。至王者代興,從古亦多異同。平、勃迎立漢文,不聞窮治朱虛之過;房、杜決策秦邸,不聞力究魏徵之非。固其君豁達大度,亦其大臣公忠善謀,翊贊其美。請再下寬大之詔,解圜扉疑入之囚,斷草野株連之案。」王降旨俞其言。

群小皆不喜繼咸,汰其軍餉六萬,軍中有怨言,繼咸疏爭不得。又以江上兵寡,鄭鴻逵戰艦不還,議更造,檄九江僉事葉士彥於江流截買材木。士彥家蕪湖,與諸商暱,封還其檄。繼咸以令不行,疏劾士彥。士彥同年御史黃耳鼎亦劾繼咸,言繼咸有心腹將校勸左良玉立他宗,良玉不從云。良玉嘗不拜監國詔,聞之益疑懼,上疏明與繼咸無隙,耳鼎受指使而言,《要典》宜再焚。江東人乃由是交口言繼咸、良玉倡和,脅制朝廷矣。會都下又有偽太子之事,良玉爭不得,遂與士英輩有隙。繼咸疏言:「太子真偽,非臣所能懸揣。真則望行良玉言,偽則不妨從容審處,多召東宮舊臣辨識,以解中外之疑。」疏未達,良玉已反。

初,繼咸聞李自成兵敗南下,命部將郝效忠、陳麟、鄧林奇守九江,自統副將汪碩畫、李士元等援袁州,防賊由岳州、長沙入江西境。既已登舟,聞良玉反,復還九江。良玉舟在北岸,貽書繼咸,願握手一別,為皇太子死。九江士民泣請繼咸往,紓一方難。繼咸會良玉於舟中,良玉語及太子下獄事,大哭。次日,舟移南岸,良玉袖出皇太子密諭,劫諸將盟。繼咸正色曰:「密諭何從來?先帝舊德不可忘,今上新恩亦不可負,密諭何從來?」良玉色變,良久乃曰:「吾約不破城,改檄為疏,駐軍侯旨。」繼咸歸,集諸將於城樓而灑泣曰:「兵諫非正。晉陽之甲,《春秋》惡之,可同亂乎?」遂約與俱拒守。而效忠及部將張世勳等則已出與良玉合兵,入城殺掠。繼咸聞之,欲自盡。黃澍入署拜泣曰:「寧南無異圖。公以死激成之,大事去矣。」副將李士春亦密白繼咸隱忍,至前途,王文成之事可圖也。繼咸以為然,遂出責良玉。良玉已疾篤,夜望見城中火起,大哭曰:「予負臨侯!」臨侯,繼咸別號也。嘔血數升,遂死。其子夢庚秘不發喪,諸將推為帥,移舟東。中朝皆疑繼咸、良玉同反。而南都時已破,諸鎮多納款。繼咸勸夢庚旋師,不聽。遣人語林奇、碩畫、士元毋為不忠事,林奇、碩畫、士元避皖湖中,遣人陰逆繼咸。繼咸已為效忠紿赴其軍。將及湖口,而夢庚、效忠降於我大清,遂執繼咸北去,館內院。至明年三月,終不屈,乃殺之。

有張亮者,四川人。舉於鄉。崇禎時,歷榆林兵備參議,用薦改安廬兵備,監禁軍討賊,頻有功。十七年擢右僉都御史,巡撫其地。福王既立,亮聞李自成兵敗西奔,奏言賊勢可乘,請解職視賊所向,督兵進討,從之。尋召入京議事,復遣還任。明年四月,左夢庚陷安慶,亮被執。夢庚北行,挾亮與俱,乘間赴河死。

金聲字正希,休寧人。好學,工舉子業,名傾一時。崇禎元年成進士,授庶吉士。明年十一月,大清兵逼都城,聲慷慨乞面陳急務,帝即召對平臺。退具疏言:「臣書生素矢忠義,遭遇聖明,日夜為陛下憂念天下事。今兵逼京畿,不得不急為君父用。夫通州、昌平,都城左右翼,宜戍以重兵。而天津漕艘所聚,尤宜亟防。今天下草澤之雄,欲效用國家者不少,在破格用之耳。臣所知申甫有將才。臣願仗聖天子威靈,與練敢戰士,為國家捍強敵,惟陛下立賜裁許。」

申甫者,僧也,好談兵,方私製戰車火器。帝納聲言,取其車入覽,授都司僉書。即日召見,奏對稱旨,超擢副總兵,敕募新軍,便宜從事。改聲御史,參其軍。甫倉猝募數千人,皆市井游手,所需軍裝戎器又不時給。而是時大清兵在郊圻久,勢當速戰,急出營柳林。總理滿桂節制諸軍,甫不肯為下。桂卒掠民間,甫軍捕之,桂輒索去。聲以兩軍不和聞,帝即命聲調護。亡何,桂歿,甫連敗於柳林、大井,乃結車營盧溝橋。大清兵繞出其後,御車者惶懼不能轉,殲戮殆盡,甫亦陣亡。聲痛傷之,言甫受事日淺,直前衝鋒,遺骸矢刃殆遍,非喋血力戰不至此。帝亦傷之,命予恤典。

聲恥無功,請率參將董大勝兵七百人,甫遺將古壁兵百人,及豪傑義從數百人,練成一旅,為劉之綸奇兵,收桑榆之效,不許。俄以清核軍需告竣,奏繳關防,請按律定罪,再疏請罷斥,皆不許。東江自毛文龍被殺,兵力弱,勢孤。聲因東宮冊立,自請頒詔朝鮮,俾聯絡東江,張海外形勢。帝雖嘉其意,亦不果用。

尋上疏言:「陛下曉夜焦勞,日親天下之事,實未嘗日習天下之人。必使天下才不才,及才長短,一一程量不爽,方可斟酌位置。往者,陛下數召對群臣,問無所得,鮮當聖心,遂厭薄之。臣愚妄謂陛下泰交尚未殷,顧問尚未數,不得謂召對無益也。願自今間日御文華,令京卿、翰林、臺諫及中行、評博等官,輪番入直,博咨廣詢。而內外有職業者,亦得不時進見。政事得失,軍民利病,廟堂舉錯,邊塞情形,皆與臣工考究於燕閑之間。歲月既久,品量畢呈。諸臣才不才,及才長短,豈得逃聖鑒。」帝未及報,聲再疏懇言之,終不用,遂屢疏乞歸。

後大學士徐光啟薦聲同修歷書,辭不就。以御史召,亦不赴。八年春,起山東僉事,復兩疏力辭。鄉郡多盜,聲團練義勇,為捍禦。十六年,風陽總督馬士英遣使者李章玉徵貴州兵討賊,迂道掠江西,為樂平吏民所拒擊。比抵徽州境,吏民以為賊,率眾破走之。章玉諱激變,謂聲及徽州推官吳翔風主使。士英以聞,聲兩疏陳辨。帝察其無罪,不問。其年冬,廷臣交薦,即命召用,促入都陛見,未赴而京師陷。

福王立於南京,超擢聲左僉都御史,聲堅不起。大清兵破南京,列郡望風迎降。聲糾集士民保績溪、黃山,分兵扼六嶺。寧國丘祖德、徽州溫璜、貴池吳應箕等多應之。乃遣使通表唐王,授聲右都御史兼兵部右侍郎,總督諸道軍。拔旌德、寧國諸縣。九月下旬,徽故御史黃澍降於大清,王師間道襲破之。

聲被執至江寧,語門入江天一日:「子有老母,不可死。」對曰:「天一同公起兵,可不同公殉義乎!」遂偕死。唐王贈聲禮部尚書,謚文毅。天一,歙諸生。

丘祖德,字念修,成都人。崇禎十年進士。授寧國推官,以才調濟南。用薦超擢僉事,分巡東昌。山東土寇猖獗,帝因給事中張元始言,令祖德及東兗道李恪專任招撫,寇多解散。十五年調官沂州。其冬用兵部尚書張國維薦,擢右僉都御史,巡撫保定。十六年罣察典,解職侯勘。事白,以故官代王永吉撫山東。京師覆,賊遣使招降。祖德斬之,謀發兵拒守。會中軍梅應元叛,率部卒索印,祖德乃南奔。

福王時,御史沈宸荃劾祖德及河南總督黃希憲輕棄封疆,詔削籍提訊,久之獲釋。而成都亦陷,無家可歸,流寓寧國。金聲起兵績溪,祖德與寧國舉人錢文龍,諸生麻三衡、沈壽蕘等各舉兵應之。時郡城已失,祖德駐華陽,三衡駐稽亭,他蜂起者又十餘部,約共攻郡城。不克,壽蕘陣歿,祖德退還山中。大清兵攻拔其寨,被獲,磔死,其子亦死。越四日,三衡軍敗,亦死。壽蕘,都督有容子。三衡,布政使溶孫也。三衡兵既起,旁近吳太平、阮恒、阮善長、劉鼎甲、胡天球、馮百家與俱起,號七家軍,皆諸生也。三衡既敗,太平等亦死。

溫璜,初名以介,字于石,烏程人。大學士體仁再從弟也。母陸守節被旌。璜久為諸生,有學行。崇禎十六年秋舉進士。授徽州推官。甫蒞任,聞京師陷,亟練民兵,為保障計。明年,南京亦覆。知府秦祖襄及諸僚屬皆遁,璜乃盡攝其印,召士民慰諭之。金聲舉兵績溪,璜與掎角,且轉餉給其軍,而徙家屬於村民舍。未幾,聲敗,璜嚴兵自守。郡中故御史黃澍以城獻,璜趨歸村舍,刃其妻茅氏及長女,遂自剄死。

吳應箕,字次尾,貴池人。善今古文,意氣橫厲一世。阮大鋮以附璫削籍,僑居南京,聯絡南北附璫失職諸人,劫持當道。應箕與無錫顧杲、桐城左國材、蕪湖沈士柱、餘姚黃宗義、長洲楊廷樞等為《留都防亂公揭》討之,列名者百四十餘人,皆復社諸生也。後大鋮得志,謀殺周鑣,應箕獨入獄護視。大鋮聞,急遣騎捕之,應箕夜亡去。南都不守,起兵應金聲,敗走山中,被獲,慷慨就死。其同時舉兵者有尹民興、吳漢超、龐昌胤、謝球、司石磐、王湛、魯之璵。

民興,字宣子,崇禎初舉進士。歷知寧國、涇二縣,除奸釐蠹,有神明之稱。行取入都,為陳啟新所訐,謫福建按察司檢校。十五年春,疏陳時務十四事,帝喜,召為職方主事。數召對,言多當帝意,即擢本司郎中。周延儒出督師,命從軍贊畫。延儒被譴,下民興吏,除名,久之始釋。福王立,起故官,尋謝病歸,流寓涇縣。南京失,與諸生趙初浣等據城拒守,大清兵攻破城,初浣死之,民興走免。唐王以為御史,事敗歸,卒於家。

漢超,宣城諸生。崇禎十七年聞都城變,謀募兵赴難,會福王立,乃已。明年,南都覆,棄家走涇縣,從尹民興起兵。兵敗,匿華陽山中。先是,丘祖德、麻三衡諸軍潰,保華陽,有徐淮者部署之。漢超與合,連取句容、溧水、高淳、溧陽、涇、太平諸縣。明年正月襲寧國,夜緣南城登。兵潰,城中按首事者。漢超已出城,念母在,且恐累族人,入見曰:「首事者我也。」剖其腹,膽長三寸。妻戚自擲樓下死。

昌胤,西充人。崇禎十年進士。授青陽知縣。南京覆,走匿九華山,謀舉兵。事洩被執,夜死旅店中。

球,溧陽諸生,僉事鼎新子也。毀家募兵。兵散,被執而死。

石磐,鹽城諸生,與都司酆某同舉兵,兵敗被執。酆言:「此儒生,吾劫之為書記耳。」石磐曰:「吾首事,奈何諱之!」繫獄六十餘日,與酆偕死。

湛,太倉諸生。城已下,與兄淳復集里人數百圍城。城中兵出擊,淳赴水死,湛被斫死。

之璵,歷官副總兵,駐福山。蘇州既降,諸生陸世鑰聚眾焚城樓。之璵率千人入城,與大清兵戰,潰走,之璵戰死。

其時以諸生死者,有六合馬純仁、邳州王台輔。南京既下,六合即歸附,純仁題銘橋柱,抱石投水死。台輔,當崇禎末,聞宦官復出鎮,將草疏極諫。甫入都,都城陷,乃還。福王時,東平伯劉澤清、御史王燮張樂大宴於睢寧。台輔衰糸至直入,責之曰:「國破君亡,此公等臥薪嘗膽、食不下咽時,顧置酒大會耶!」左右欲鞭之,燮曰:「狂生也。」命引去。及南京覆,台輔視其廩曰:「此吾所樹,盡此死。」明年,粟盡,北面再拜,自縊死。

沈猶龍,字雲升,松江華亭人。萬曆四十四年進士。除鄞縣知縣。天啟初,徵授御史,出為河南副使。崇禎元年,召復故官,進太僕少卿,拜右僉都御史,巡撫福建。江西妖賊張普薇等作亂,猶龍遣遊擊黃斌卿協剿,大破之。增秩賜金,以憂歸。服闋,起兵部右侍郎兼右僉都御史,總督兩廣軍務,兼廣東巡撫。

十七年冬,福王召理部事,不就,乞葬親歸。明年,南京失守,列城望風下。閏六月,吳淞總兵官吳志葵自海入江,結水寨於泖湖。會總兵官黃蜚擁千艘自無錫至,與合。猶龍乃偕里人李待問、章簡等,募壯士數千人守城,與二將相掎角,而參將侯承祖守金山。八月,大清兵至,二將敗於春申浦,城遂被圍。未幾破,猶龍出走,中矢死。待問守東門,簡守南門,城破,俱被殺。華亭教諭眭明永題詩明倫堂,投繯死。諸生戴泓赴池死。嘉定舉人傅凝之參志葵軍事,兵敗,赴水死。大清兵遂攻金山,承祖與子世祿猶固守。城既破,巷戰逾時,世祿中四十矢,被獲,死之。承祖亦被獲,說之降,不從,遂被殺。志葵、蜚既敗,執至江陰城下,令說城中人降。志葵說之,蜚不語,城迄不下,後皆被殺。

待問,字存我,崇禎末進士。授中書舍人。工文章,兼精書法。簡,字坤能。舉於鄉,官羅源知縣。

陳子龍,字臥子,松江華亭人。生有異才,工舉子業,兼治詩賦古文,取法魏、晉,駢體尤精妙。崇禎十年進士。選紹興推官。

東陽諸生許都者,副使達道孫也。家富,任俠好施,陰以兵法部勒賓客子弟,思得一當。子龍嘗薦諸上官,不用,東陽令以私憾之。適義烏奸人假中貴名招兵事發,都葬母山中,會者萬人。或告監司王雄曰:「都反矣。」雄遽遣使收捕,都遂反。旬日間聚眾數萬,連陷東陽、義烏、浦江,遂逼郡城,既而引去。巡撫董象恆坐事逮,代者未至,巡按御史左光先以撫標兵,命子龍為監軍討之,稍有俘獲。而遊擊蔣若來破其犯郡之兵,都乃率餘卒三千保南砦。雄欲撫賊,語子龍曰:「賊聚糧據險,官軍不能仰攻,非曠日不克。我兵萬人,止五日糧,奈何?」子龍曰:「都,舊識也,請往察之。」乃單騎入都營,責數其罪,諭令歸降,待以不死。遂挾都見雄。復挾都走山中,散遣其眾,而以二百人降。光先與東陽令善,竟斬都等六十餘人於江滸。子龍爭,不能得。

以定亂功,擢兵科給事中。命甫下而京師陷,乃事福王於南京。其年六月,言防江之策莫過水師,海舟議不可緩,請專委兵部主事何剛訓練,從之。太僕少卿馬紹愉奉使陛見,語及陳新甲主款事。王曰:「如此,新甲當恤。」廷臣無應者,獨少詹事陳盟曰可。因命予恤,且追罪嘗劾新甲者。廷臣懲劉孔昭殿上相爭事,不敢言。子龍與同官李清交章力諫,事獲已。

未幾,列上防守要策,請召還故尚書鄭三俊,都御史易應昌、房可壯、孫晉,並可之。又言:「中使四出搜巷。凡有女之家,黃紙貼額,持之而去,閭井騷然。明旨未經有司,中使私自搜採,甚非法紀。」乃命禁訛傳誑惑者。子龍又言:「中興之主,莫不身先士卒,故能光復舊物。今入國門再旬矣,人情泄沓,無異昇平。清歌漏舟之中,痛飲焚屋之內,臣不知其所終。其始皆起於姑息一二武臣,以至凡百政令皆因循遵養,臣甚為之寒心也。」亦不聽。明年二月乞終養去。

子龍與同邑夏允彞皆負重名,允彞死,子龍念祖母年九十,不忍割,遁為僧。尋以受魯王部院職銜,結太湖兵,欲舉事。事露被獲,乘間投水死。

夏允彞,字彞仲。弱冠舉於鄉,好古博學,工屬文。是時東林講席盛,蘇州高才生張溥、楊廷樞等慕之,結文會名復社。允彞與同邑陳子龍、徐孚遠、王光承等亦結幾社相應和。崇禎十年,與子龍同成進士,授長樂知縣,善決疑獄。他郡邑不能決者,上官多下長樂。居五年,邑大治。吏部尚書鄭三俊舉天下廉能知縣七人,以允彞為首。帝召見,大臣方岳貢等力稱其賢,將特擢。會丁母憂,未及用。

北都變聞,允彞走謁尚書史可法,與謀興復。聞福王立,乃還。其年五月擢吏部考功司主事。疏請終制,不赴。御史徐復陽希要人旨,劾允彞及其同官文德翼居喪授職為非制,以兩人皆東林也。兩人實未嘗赴官,無可罪。吏部尚書張捷遽議貶秩調用。

未幾,南都失,徬徨山澤間,欲有所為。聞友人侯峒曾、黃淳耀、徐水幵等皆死,乃以八月中賦絕命詞,自投深淵以死。允彞死後二年,子完淳、兄之旭並以陳子龍獄詞連及,亦死。而同社徐孚遠,舉於鄉,因松江破,遁入海,死於島中。

侯峒曾,字豫瞻,嘉定縣人。給事中震暘子也。天啟五年成進士,授南京武選司主事,丁父憂。崇禎七年入都。兵部尚書張鳳翼薦為職方郎中,峒曾力辭,乃改南京文選司主事。由稽勛郎中遷江西提學參議。給事中耿始然督賦至,他監司以屬禮見,峒曾獨與抗禮。益王勢方熾,歲試黜兩宗生,王怒,使人誚讓,峒曾不為動。遷廣東副使,不赴。起浙江右參政,分守嘉、湖。漕卒擊傷秀水知縣李向中,峒曾請於撫按,捕戮首惡,部內肅然。吏部尚書鄭三俊舉天下賢能監司五人,峒曾與焉。召為順天府丞,未赴而京師陷。

福王時,用為左通政,辭不就。及南京覆,州縣多起兵自保。嘉定士民推峒曾為倡,偕里人黃淳耀、張錫眉、董用圓、馬元調、唐全昌、夏雲蛟等誓死固守。大清兵來攻,峒曾乞師於吳淞總兵官吳志葵。志葵遣游擊蔡祥以七百人來赴,一戰失利,束甲遁,外援遂絕,城中矢石俱盡。七月三日大雨,城隅崩,架巨木支之。明日雨益甚,城大崩,大清兵入。峒曾拜家廟,挈二子元演、元潔並沈於池。錫眉、用圓、元調、全昌、雲蛟皆死之。錫眉、用圓皆舉人。用圓官秀水教諭。元調、全昌、雲蛟並諸生。

其時聚眾城守而死者有江陰閻應元、崑山朱集璜之屬。

應元,字麗亨,順天通州人。崇禎中,為江陰典史。十七年,海賊顧三麻入黃田港,應元往禦,手射殺三人。賊退,以功遷英德主簿,道阻不赴,寓居江陰。

明年五月,南京亡,列城皆下。閏六月朔,諸生許用倡言守城,遠近應者數萬人。典史陳明遇主兵,用徽人邵康公為將。而前都司周瑞龍泊江口,相掎角。戰失利,大清兵逼城下。徽入程璧盡散家貲充餉,而身乞師於吳淞總兵官吳志葵。志葵至,璧遂不返。康公戰不勝,瑞龍水軍亦敗去,明遇乃請應元入城,屬以兵事。

大清兵力攻城,應元守甚固。東平伯劉良佐用牛皮帳攻城東北,城中用炮石力擊。良佐乃移營十方庵,令僧陳利害。良佐旋策馬至,應元誓以大義,屹不動。及松江破,大清兵來益眾,四圍發大炮,城中死傷無算,猶固守。八月二十一日,大清兵從祥符寺後城入,眾猶巷戰,男婦投池井皆滿。明遇、用皆舉家自焚。應元赴水,被曳出,死之。

訓導馮厚惇冠帶縊於明倫堂,娣及妻王結示任投井死。里居中書舍人戚勳令妻及子女、子婦先縊,乃舉火自焚,從死者二十人。舉人夏維新,諸生王華、呂九韶自刎死。

貢生黃毓祺者,好學,有盛名,精釋氏學。與門人徐趨舉兵行塘,以應城內兵。及城陷,兩人逸去。明年冬,趨偵江陰無備,率壯士十四人襲之。不克,皆死。毓祺既逸去,避江北。其子大湛、大洪被收,兄弟方爭死。而毓祺以敕印事發,逮繫江寧獄,將刑,其門人告之期,命取襲衣自斂,趺坐而逝。

朱集璜,字以發,崑山貢生。學行為鄉里所推,教授弟子數百人。南京既亡,崑山議拒守,而縣丞閻茂才已遣使迎降。縣人共執殺茂才,以六月望,推舊將王佐才為帥,集璜及周室瑜、陶琰、陳大任等共舉兵。參將陳宏勛、前知縣楊永言率壯士百人為助。佐才亦邑人,嘗官狼山副總兵,年老矣。大清兵至,宏勛率舟師迎戰,敗還,遊擊孫志尹戰歿。城陷,永言遁去。佐才縱民出走,而己冠帶坐帥府,被殺。集璜投東禪寺後河死。門人孫道民、張謙同日死。室瑜、琰、大任亦死之。室瑜子朝礦、大任子思翰皆同死。室瑜舉於鄉,官儀封知縣。琰、大任皆諸生。

時以守禦死者,蘇達道、莊萬程、陸世鏜、陸雲將、歸之甲、周復培、陸彥沖。代父死者,沈徵憲、朱國軾。救母死者,徐洺。自盡者,徐溵、王在中、吳行貞。

楊文驄,字龍友,貴陽人。浙江參政師孔子。萬曆末,舉於鄉。崇禎時,官江寧知縣。御史詹兆恒劾其貪污,奪官侯訊。事未竟,福王立於南京,文驄戚馬士英當國,起兵部主事,歷員外郎、郎中,皆監軍京口。以金山踞大江中,控制南北,請築城以資守禦,從之。文驄善書,有文藻,好交遊,干士英者多緣以進。其為人豪俠自喜,頗推獎名士,士亦以此附之。

明年遷兵備副使,分巡常、鎮二府,監大將鄭鴻逵、鄭彩軍。及大清兵臨江,文驄駐金山,扼大江而守。五月朔,擢右僉都御史,巡撫其地,兼督沿海諸軍。文驄乃還駐京口,合鴻逵等兵南岸,與大清兵隔江相持。大清兵編大筏,置燈火,夜放之中流,南岸軍發炮石,以為克敵也,日奏捷。初九日,大清兵乘霧潛濟,迫岸。諸軍始知,倉皇列陣甘露寺。鐵騎衝之,悉潰。文驄走蘇州。十三日,大清兵破南京,百官盡降。命鴻臚丞黃家鼒往蘇州安撫,文驄襲殺之,遂走處州。時唐王已自立於福州矣。

初,唐王在鎮江時,與文驄交好。至是,文驄遣使奉表稱賀。鴻逵又數薦,乃拜兵部右侍郎兼右僉都御史,提督軍務,令圖南京。加其子鼎卿左都督、太子太保。鼎卿,士英甥也。士英遣迎福王,遇王於淮安。王貧CI甚,鼎卿賙給之,王與定布衣交,以故寵鼎卿甚。及鼎卿上謁,王以故人子遇之,獎其父子,擬以漢朝大、小耿。然其父子以士英故,多為人詆諆。

明年,衢州告急。誠意侯劉孔昭亦駐處州,王令文驄與共援衢。七月,大清兵至,文驄不能禦,退至浦城,為追騎所獲,與監紀孫臨俱不降被戮。

臨,字武公,桐城人,兵部侍郎晉之弟。文驄招入幕,奏為職方主事,竟與同死。

其時起兵旁掠郡縣者有吳易,字日生,吳江人。生有膂力,跅弛不羈。崇禎末,成進士。福王時,謁史可法於揚州。可法異其才,題授職方主事,為己監軍。明年,奉檄徵餉江南,未還而揚州失,已而吳江亦失。易走太湖,與同邑舉人孫兆奎,諸生沈自駉、自炳,武進吳福之等謀舉兵。旬日得千餘人,屯於長白蕩,出沒旁近諸縣,道路為梗。唐王聞之,授兵部右侍郎兼右僉都御史,總督江南諸軍。文驄奏易斬獲多,進為兵部尚書。魯王亦授易兵部侍郎,封長興伯。八月,大清兵至,易遂敗走。父承緒、妻沈及女皆投水死,自駉、自炳、福之亦死焉,兆奎被獲,一軍盡殲。明年,易鄉人周瑞復聚眾長白蕩,迎易入其營。八月,事洩被獲,死之。福之,鐘巒子也。兆奎兵敗時,慮易妻女被辱,視其死而後行,故被獲。械至江寧,死之。

陳潛夫,字元倩,錢塘人。家貧落魄,好大言以駴俗。崇禎九年舉於鄉,益廣交遊,為豪舉,好臧否人,里中人惡之。友人陸培兄弟為文逐潛夫,潛夫乃避居華亭。十六年冬,授開封推官。大河南五郡盡為賊據,開封被河灌,城虛無人,長吏皆寄居封丘。有勸潛夫弗往者,不聽,馳之封丘。會叛將陳永福率賊兵出山西,其子德為巡撫秦所式部將,縛巡按御史蘇京去。潛夫募民兵千,請於所式及總兵卜從善、許定國,令共剿,皆不肯行。潛夫乃以十七年正月奉周王渡河居杞縣,檄召旁近長吏,設高皇帝位,歃血誓固守。賊所設偽巡撫梁啟隆居開封,他偽官散布郡邑間甚眾,而開封東西諸土寨剽掠公行,相攻殺無已。潛夫轉側杞、陳留間,朝夕不自保。聞西平寨副將劉洪起勇而好義,屢殺賊有功,躬往說之。五月五日方誓師,而都城失守。報至,乃慟哭,令其下縞素。洪起兵萬,號五萬,潛夫兵三千,俘杞偽官,啟隆聞風遁去。遂渡河而北,大破賊將陳德於柳園。時李自成已敗走山西,而南陽賊乘間犯西平,洪起引還,潛夫亦隨而南。

福王立南京,潛夫傳露布至,朝中大喜,即擢監軍御史,巡按河南。潛夫乃入朝言:「中興在進取,王業不偏安。山東、河南地,尺寸不可棄。豪傑結寨自固者,引領待官軍。誠分命籓鎮,以一軍出潁、壽,一軍出淮、徐,則眾心競奮,爭為我用。更頒爵賞鼓舞,計遠近,畫城堡俾自守,而我督撫將帥屯銳師要害以策應之。寬則耕屯為食,急則披甲乘墉,一方有警,前後救援,長河不足守也。汴梁一路,臣聯絡素定,旬日可集十餘萬人。誠稍給糗糧,容臣自將,臣當荷戈先驅,諸籓鎮為後勁,河南五郡可盡復。五郡既復,畫河為固,南連荊楚,西控秦關,北臨趙、衛,上之則恢復可望,下之則江淮永安,此今日至計也。兩淮之上,何事多兵,督撫紛紜,並為虛設。若不思外拒,專事退守,舉土地甲兵之眾致之他人,臣恐江淮亦未可保也。」

當是時,開封、汝寧間列寨百數,洪起最大;南陽列寨數十,蕭應訓最大,洛陽列寨亦數十,李際遇最大。諸帥中獨洪起欲效忠,潛夫請予挂印為將軍。馬士英不聽,而用其姻婭越其傑巡撫河南。潛夫自九月入覲,便道省親,甫五日即馳赴河上。所建白皆不用,諸鎮兵無至者。其傑老憊不知兵。兵部尚書張縉彥總督河南、山東軍務,止提空名,不能馭諸將。其冬,應訓復南陽及泌陽、舞陽、桐柏,遣子三傑獻捷。潛夫授告身,飲之灑,鼓吹旌旗前導出。三傑喜過望,往謁其傑。其傑故為尊嚴,厲辭詰責,詆為賊。三傑泣而出,萌異心。潛夫過諸寨,皆鐃吹送迎;其傑間過之,諸寨皆閉門不出。其傑恚,譖潛夫於士英。士英怒,冬盡,召潛夫還,以凌駉代。潛夫亦遭外艱歸。

明年三月,給事中林有本疏劾御史彭遇颽,並及潛夫。士英以遇颽己私人,置不問,獨令議潛夫罪。先是,有童氏者,自言福王繼妃,廣昌伯劉良佐具禮送之。潛夫至壽州,見車馬騶從傳呼皇后來,亦稱臣朝謁。及童氏入都,王以為假冒,下之獄。遂責潛夫私謁妖婦,逮下獄治之。

未幾,南都不守,潛夫得脫歸。聞魯王監國紹興,渡江往謁,命復故官,加太僕少卿,監軍,乃自募三百人列營江上。尋進大理寺少卿,兼御史如故。順治三年五月晦,江上師盡潰,潛夫走至山陰化龍橋,偕妻妾二孟氏同赴水死,年三十七。

始為文逐潛夫者陸培,字鯤庭,舉進士,為行人,奉使事竣歸省。南京既覆,聞潞王又降,以繩授二僕,從容就縊而死,年二十九。培少負俊才,有文名,行誼修謹,客華亭,嘗卻奔女於室云。

沈廷揚,字季明,崇明人。好談經濟。崇禎中,由國子生為內閣中書舍人。十二年冬,帝以山東多警,運道時梗,議復海運。廷揚生海濱,習水道,上疏極言其便,且輯海運書五卷以呈。帝喜,即命造海舟試之。廷揚乘二舟由淮安出海,抵天津,僅半月。帝大喜,即加戶部郎中,往登州與巡撫徐人龍計海運事。初,寧遠軍餉率用天津船,自登州侯東南風,轉粟至天津;又侯西南風轉至寧遠。廷揚請從登州直達寧遠,帝用其議,省費多。十五年命再赴淮安督海運,事竣,加光祿少卿,仍領其事。

及京師陷,福王命廷揚以海舟防江。尋命兼理餉務,饋江北諸軍。南京失守,走還鄉里。後航海至舟山,依黃斌卿。唐王在福建,授兵部右侍郎,總督水師。魯王授官亦如之。魯王航海之明年,廷揚督舟師北上,抵福山,次鹿苑。夜分颶風大作,舟膠於沙,為大清兵所執。諭之降,不從,乃就戮。

林汝翥,字大葳,福清人。舉於鄉,授沛縣知縣。天啟二年,戰卻徐鴻儒兵,緝妖人王普光黨有功,特擢御史。四年六月,巡視京城。民曹大妻與人奴角口,服毒死。火者曹進、傅國興率眾大掠奴主家,用大錐錐其主,刑官不敢問。汝翥捕得進,進懼劾,請受杖,遂杖之五十。國興邀於道,罵不已,汝翥收繫之,亦請受杖,復杖之。魏忠賢大怒,立傳旨廷杖汝翥。先數日,群奄毆殺萬璟。汝翥大懼,逸至遵化。巡撫鄧水美為代題,都御史孫瑋、御史潘雲翼等交章論救。不聽,卒杖之,削籍歸。崇禎初,起官右參議,分守溫處道,不赴。久之,起瓊州道,坐奸民煽亂,貶秩歸。福王時,起雲南僉事,已而解職。魯王次長垣,召為兵部右侍郎,與員外郎林惣攻福寧,戰敗被執,諭降不從,繫之,吞金屑而死。

惣,字子野,汝翥同邑人。崇禎十六年進士。授海寧知縣。邑有妖人以劍術惑眾,聚千人,惣捕殺之。南都覆,杭州亦不守,卒乘機乞餉,環署大噪。惣罪為首者,而如其請。以城孤不能存,引去。唐王以為御史,改文選員外郎,募兵福寧。聞王被殺,大慟,走匿山中。及魯王航海至長垣,福清鄉兵請惣為主,與汝翥共攻城,歿於陣。

鄭為虹,字天玉,江都人。崇禎十六年進士。除浦城知縣。唐王道浦城,知其廉,及自立,召為御史。部民相率乞留,有十不可去之疏。乃令以御史巡視仙霞關,駐浦城。尋令巡撫上遊四府,兼領關務。鄭芝龍部將奪民舟,為虹叱責之。芝龍訴於王,王為諭解。然是時芝龍已懷異志,盡撤守關將,仙霞嶺二百里間無一人。順治三年八月,大清兵長驅直入,為虹亟還浦城,縱士民出走,自守空城。無何,被執,與給事中黃大鵬並死之,年二十有五。

大鵬,字文若,建陽人。崇禎十三年進士。為義烏知縣,有能聲。唐王召為兵科給事中,從至建寧,令與為虹共守仙霞嶺,竟同死。時王在延平,聞仙霞關失守,倉猝走汀州。守延平者為王士和,從走汀州者有胡上琛、熊緯,皆以死事著。

士和,字萬育,金谿人。崇禎中,舉於鄉。南京既覆,江西亦被兵,士和避入閩,授吏部司務。疏陳時政闕失,凡數千言,唐王刊賜文武諸臣,且召士和入對,嘉獎備至,擢兵部主事。未一月擢延平知府。八月,王走汀州,留兵部侍郎曹覆泰偕士和居守。俄警報疊至,士和召父老曰:「吾雖一月郡守,當與城存亡。若等可速出,毋使數萬生靈盡膏斧鑕。」眾泣,士和亦泣。退入內署,謂友人曰:「吾一介書生,數月而忝二千石,安敢偷生。」其友勸止之,正色曰:「君子愛人以德,姑息何為。」從容正衣冠,閉戶投繯死。

上琛,字席公。世襲福州右衛指揮使。好讀書,能詩。既襲職,復舉武鄉試。唐王時,官錦衣衛指揮,遷署都督僉事,充御營總兵官,從至汀州。王被執,上琛奔還福州,謂家人曰:「吾世臣,不可茍活,為我採毒草來。」妾劉年二十,願同死。上琛喜曰:「汝幼婦亦能死耶!」遂整冠帶與妾共飲藥酒而卒。

緯,字文江,南昌人。崇禎十六年進士。授行人。兩京既覆,每飲酒,輒涕泗交橫下。友人語之曰:「昔狼瞫有言『吾未獲死所』,子既有志,曷求所乎?」乃赴延平謁唐王,擢給事中。尋扈行至汀州,遘變,從官皆散,緯仍奔赴。遇大清兵,死之。

贊曰:廢興之故,豈非天道哉。金聲等以烏合之師,張皇奮呼,欲挽明祚於已廢之後,心離勢渙,敗不旋踵,何尺寸之能補。然卒能致命遂志,視死如歸,事雖無成,亦存其志而已矣。


\end{pinyinscope}