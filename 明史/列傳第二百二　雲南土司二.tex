\article{列傳第二百二 雲南土司二}

\begin{pinyinscope}
○姚安鶴慶武定尋甸麗江元江永昌新化威遠北勝灣甸鎮康大侯瀾滄衛麓川

姚安,本漢弄棟、蜻蛉二縣地。唐置姚州都督府,以民多姚姓也。天寶間,南詔蒙氏改為弄棟府。宋時,段氏改姚州。元立統矢千戶所,天曆間,陞姚安路。

洪武十五年定雲南,改為府。十六年,姚安土官自久作亂。官兵往討,師次九十九莊,自久遁去。明年復寇品甸。西平侯沐英奏以土官高保為姚安府同知、高惠為姚安州同知。保、惠從英擊自久,平之。二十年命普定侯陳桓、靖寧侯葉升往雲南總制諸軍,於定邊、姚安等處立營屯種。二十六年,保以襲職,遣其弟貢馬謝恩。

宣德九年,姚安土知府高賢遣使貢馬。弘治中,土官高棟與普安叛賊戰,死於板橋驛。嘉靖三十年,土官高鵠當元江之變布政司徐樾遇害,奮身赴救,死之。萬曆中,同知高金以征緬功,賜四品服。所屬大姚縣,有鐵索箐者,本惈種。依山險,以剽掠為業,旁郡皆受其害。弘治間,稍有歸命者,分隸於姚安、姚州。嘉靖中,乃專屬姚安。其渠羅思者,有幻術,造偽印稱亂。萬曆元年,巡撫鄒應龍與總兵官沐昌祚討平之,諸郡乃安。

鶴慶,唐時名鶴川,南詔置謀統郡。元初,置鶴州。至元中,陞鶴慶府,尋改為路。

洪武中,大軍平雲南,分兵拔三營、萬戶砦,獲偽參政寶山帖木兒等六十七人。置鶴慶府,以土官高隆署府事。十七年以董賜為知府、高仲為同知、賜子節為安寧知州、楊權為劍川知州。賜率其屬來朝,貢馬及方物,詔賜冠帶並織金文綺、布帛、鈔錠。十八年以賜為雲南前衛世襲指揮僉事。賜,安寧州人,世為酋長。大軍入滇,率眾來降,復從軍討賊有功,故與子節並有世襲知府、知州之命。及賜來朝,以父子俱受顯榮,無以仰報,子幼沖,不達政治,乞還父子所授官,而自為安寧知州。帝曰:「爾能綏靖邊鄙,授爾官以酬爾勳。今辭尊居卑,奈何?」命潁國公傅友德及諸大臣議之。皆以賜既有功,不可聽其辭,而節之官則可免。乃改賜明威將軍雲南前衛世襲指揮僉事,諭曰:「雲南前衛密邇安寧,特命爾是職。爾其綏輯遠人,以安邊鄙,其毋再辭。」

二十年,劍川土官楊奴叛。大理衛指揮鄭祥討之,斬八十餘人,楊奴遁。未幾,還劍川,復聚蠻為亂,祥復以兵擊斬之。二十四年置鶴慶衛。三十年改鶴慶府為軍民府。永樂十五年,順州知州王義言:「沾被聖化三十餘年,聲教所郕,言語漸通,子弟亦有俊秀,請建學教育。」從之。

正統二年,副使徐訓奏鶴慶土知府高倫與弟純屢逞兇惡,屠戮士庶,與母楊氏並叔宣互相賊害。敕黔國公沐昂諭使輸款,如恃強不服,即調軍擒捕。五年復敕昂等曰:「比聞土知府高倫妻劉氏同倫弟高昌等,糾集羅羅、麼些人眾,肆行兇暴。事發,不從逮訊。敕至,即委官至彼勘實,量調官軍擒捕首惡,並逮千戶王蕙及高宣等至京質問。」八年,鶴慶民楊仕潔妻阿夜珠告倫謀殺其子,復命法司移文勘驗。已而大理衛千戶奏報,倫擅率軍馬欲謀害親母,又稱其母告倫不孝及私斂民財,多造兵器,殺戮軍民,支解梟令等罪。遂敕黔國公沐晟等勘覆。及奏至,言倫所犯皆實,罪應死。倫復屢訴,因與叔宣爭襲,又與千戶王蕙爭娶妾,以致挾仇誣陷。所勘殺死,皆病死及強盜拒捕之人。倫母楊亦訴倫無不孝,實由宣等陷害。復敕晟及御史嚴恭確訪。既而奏當倫等皆伏誅。高氏族人無可繼者,帝命於流官中擇人,以綏遠蠻。乃擢瀘州知府林遒節為知府。鶴慶之改流官自此始。

武定,南詔三十七部之一。宋淳熙間,大理段氏以阿歷為羅武部長。三傳至矣格,當元世祖時,為北部土官總管。至元七年改武定路,置南甸縣。

洪武十四年,雲南下,武定女土官商勝首先歸附。十五年改為武定軍民府,以勝署府事。十六年,勝遣人來朝,貢馬。詔賜勝誥命、朝服及錦幣、鈔錠。十七年以和曲土官豆派為知州。二十一年發內帑,令於武定、德昌、會川諸處,市馬三千匹。宣德元年,元謀縣故土知縣吾忠子政來朝。

正德二年四月,武定雨雹,溪水漲,決堤壞田,隕霜露殺麥。七月廢武定所屬之南甸縣改隸和曲州,石舊縣改隸祿勸州。三年,土知府鳳英以從征功,進秩右參政,仍知府事,請賜金帶,部議不可。帝以英有軍功,給之。明年,英貢馬謝恩,賜如例。

嘉靖七年,土舍鳳朝文作亂。殺同知以下官吏,劫州印,舉兵與尋甸賊安銓合犯雲南府,撫臣以聞。時安銓未平,朝文復起,滇中大擾。詔以右都御史伍文定為兵部尚書,提督雲、貴、川、湖軍務,調四鎮土漢官軍討賊。五月,黔國公沐紹勛疏言:「臣奉命會同巡撫等調發官軍,分道剿撫。諸賊抗逆,執留所遣官軍二人,所調集各土舍,又重自疑畏。臣謹以便宜榜示,先給冠帶,待後奏請承襲,眾始感奮。於二月進兵,擊斬強賊十餘人,賊奔回武定。乞敕部授臣方略,俾獲便宜行事,並宥各土舍往罪,凡有功者,俱許承襲,作其敵愾之氣。」帝納之,賜敕獎勵。賊既敗歸,其黨稍散。初,朝文紿其眾,謂武定知府鳳詔母子已戮,朝廷且盡剿武定蠻眾。至是,鳳詔同其母率眾自會城往,蠻民相顧錯愕,咸投鳳詔降。朝文計無所出,絕普渡而走,官兵追及,復敗之。朝文率家奴數人,取道霑益州,奔至東川之湯郎箐,為追兵所及,磔死。銓眾猶盛,遁據尋甸故巢,列寨數十。官兵分哨夾攻之,諸寨先後破,乃併力攻拔其必古老巢。銓奔東川,入芒部,為土舍祿慶所執,賊平。是役也,生擒渠賊千餘人,斬首二千九百餘級,俘獲男婦千二百餘,撫散蠻黨二萬有奇,奪器械牛馬無算。捷聞,銓、朝文皆梟示,籍其產,家屬戍邊。

十六年命土知府瞿氏掌印。初,府印自洪武以來俱掌於土官,正德間有司議以畀流官同知,土知府職專巡捕、徵糧而已。及鳳詔死,瞿氏以母襲子官,所轄四十七馬頭阿台等,數請以印屬瞿氏。吏部覆言,係舊例,宜如其請,從之。

四十二年,瞿氏老,舉鳳詔妻索林自代。比索林襲,遂失事姑禮。瞿氏大恚,乃收異姓兒繼祖入鳳氏宗,挾其甥婿貴州水西土舍安國亨、四川建昌土官鳳氏兵力,欲廢索林,以繼祖嗣。不克,乃具疏自稱為索林囚禁,令繼祖詣闕告之。繼祖歸,詐稱受朝命襲職,驅目兵逼奪府印。索林抱印奔會城,撫按官諭解之。索林歸武定,視事如故,而復聽繼祖留瞿氏所,於是婦姑嫌隙益甚。索林謀誅繼祖,事洩,繼祖遂大發兵圍府,行劫和曲、祿勸等州縣,殺傷調至土官王心一等兵。索林復抱印走雲南,巡撫曹忭下令收印,逮其左右鄭竤繫獄,令瞿氏暫理府事;貸繼祖,責其自新。

四十四年添設府通判一員。四十五年築武定新城成,巡撫呂光洵遣鄭竤回府復業。鄭竤者,前為索林謀殺繼祖者也。繼祖執而殺之,糾眾攻新城。臨安通判胡文顯督百戶李鰲、土舍王德隆往援,至雞溪子隘,遇伏,鰲及德隆俱死。僉事張澤督尋甸兵二千餘馳救,亦敗,澤及千戶劉裕被執。鎮巡官促諸道兵並進,逼繼祖東山寨,圍之。繼祖懼,攜澤及索林走照姑。已,復殺澤。官軍追之急,由直勒渡過江,趨四川,依東川婦家阿科等。巡按劉思問以狀聞,敕雲南、四川會兵討賊。

初,繼祖之走東川也,土官鳳氏與之通。已而見滇、蜀官軍與土舍祿紹先等兵皆會,乃背繼祖,發卒七千人來援,繼祖益窮。賊帥者色赴紹先營降,斬繼祖以獻。姚縣土官高繼先復擒其餘黨,姚安府同知高欽及第鈞,謀主趙士傑等皆伏誅。守臣議改設流官,猶不欲絕鳳氏,授索林支屬鳳曆子思堯經歷,給莊百餘。鳳曆以不得知府怨望,陰結四川七州及水西宣慰安國亨謀作亂。流官知府劉宗寅遣諭之,不聽,遂聚眾稱思堯知府,夜襲府城。城中嚴備不能入,退屯魯墟。宗寅夜出兵,砍其營,賊潰,追至馬刺山,擒鳳曆,伏誅。

萬曆三十五年,繼祖姪阿克久徙金沙江外,賊黨鄭舉等誘阿克作亂,陰結江外會川諸蠻,直陷武定,大肆劫掠。連破元謀、羅次諸城,索府印。會流官知府攜印會城,不能得。賊以無印難號召,劫推官,請冠帶、印信。鎮撫以兵未集,懼,差人以府印授之。賊退入武定,立阿克為知府。鎮撫調集土兵,分五路進剿,克復武定、元謀、羅次、祿豐、嵩明等州縣,擒阿克及其黨至京師,礫於市。武定平,遂悉置流官。

尋甸,古滇國地,DI刺蠻居之,號仲扎溢源部,後為烏蠻裔斯丁所奪,號斯丁部。蒙氏為尋甸,至段氏,改仁德部。元初,置仁德萬戶,後改府。洪武十五年定雲南,仁德土官阿孔等貢馬及方物,改為尋甸軍民府。十六年,土官安陽來朝,貢馬及虎皮、氈衫等物,詔賜衣服、錦綺、鈔錠。十七年以尋甸土官沙琛為知府。二十三年置木密關守禦千戶所於尋甸之甸頭易龍驛,又置屯田所於甸頭里果馬里,聯絡耕種,以為邊備。是後,土官皆按期入貢。

成化十二年,兵部奏,土官舍人安宣聚眾殺掠,命鎮守官相機撫捕。十四年,土知府安晟死,兄弟爭襲,遂改置流官。嘉靖六年,安銓作亂,乃土舍之失職者也,侵掠嵩明、木密、楊林等處。巡撫傅習檄守巡官討之,大敗,賊遂陷尋甸、嵩明,殺指揮王升、唐功等,知府馬性魯棄城走。時武定鳳朝文叛,銓與之合,久之伏誅,事詳前傳。

麗江,南詔蒙氏置麗水節度。宋時麼些蠻蒙醋據之。元初,置茶罕章宣慰司。至元中,改置麗江路軍民總管府,後改宣撫司。洪武十五年置麗江府。十六年,蠻長木德來朝貢馬,以木德為知府,羅克為蘭州知州。十八年,巨津土酋阿奴聰叛,劫石門關,千戶浦泉戰死。吉安侯陸仲亨率指揮李榮、鄭祥討之,賊戰敗,遁入山谷,捕獲誅之。時木德從征,又從西平侯沐英征景東、定邊,皆有功,予世襲。二十四年,木德死,子初當襲。初守巨津州石門關,與西番接境。既襲職,英請以初弟虧為千夫長,代守石門,從之。二十六年十月,西平侯沐春奏,麗江土民每歲輸白金七百六十兩,皆麼些洞所產,民以馬易金,不諳真偽,請令以馬代輸,從之。三十年改為麗江軍民府,從春請也。永樂十六年,檢校龐文郁言,本府及寶山、巨津、通安、蘭州四州歸化日久,請建學校,從之。

宣德五年,麗江府奏浪滄江寨蠻者保等聚眾劫掠。黔國公沐晟委官撫諭,不服,部議再行招撫。已,蘭州土官羅牙等奏,者保拒命,請發兵討之。帝命黔國公及雲南三司相機行,勿緣細故激變蠻民。正統五年,賜知府木森誥命,加授大中大夫資治少尹,以征麓川功也。成化十一年,知府木嶔奏,鶴慶千夫長趙賢屢糾群賊越境殺掠,乞調旁衛官軍擒剿,命移知守臣計畫。嘉靖三十九年,知府木高進助殿工銀二千八百兩,詔加文職三品服色,給誥命。四十年又進木植銀二千八百兩,詔進一級,授亞中大夫,給誥命。

萬曆三十一年,巡按御史宋興祖奏:「稅使內監楊榮欲責麗江土官退地,聽採。竊以麗江自太祖令木氏世官,守石門以絕西域,守鐵橋以斷吐蕃,滇南藉為屏籓。今使退地聽採,必失遠蠻之心。即令聽諭,已使國家歲歲有吐籓之防;倘或不聽,豈獨有傷國體。」疏上,事得寢。

三十八年,知府木增以征蠻軍興,助餉銀二萬餘兩,乞比北勝土舍高光裕例,加級。部覆賜三品服色,巡按御史劾其違越,請奪新恩,從之。四十七年,增復輸銀一萬助遼餉。泰昌元年,錄增功,賞白金表裏,其子懿及舍目各賞銀幣有差。天啟二年,增以病告,加授左參政致仕。五年,特給增誥命,以旌其忠。雲南諸土官,知詩書好禮守義,以麗江木氏為首云。

元江,古西南夷極邊境,曰惠籠甸,又名因遠部。南詔蒙氏以屬銀生節度,徙白蠻蘇、張、周、段等十姓戍之。又開威遠等處,置威遠夾。後和泥侵據其地。宋時,儂智高之黨竄居於此,和泥又開羅槃甸居之,後為麼些、徒蠻、阿僰諸部所據。元時內附。至元中,置元江萬戶府。後於威遠更置元江路,領羅槃、馬籠等十二部,屬臨安、廣西、元江等處宣慰司。

洪武十五年改元江府。十七年,土官那直來朝貢象,以那直為元江知府,賜襲衣冠帶。十八年置因遠羅必甸長官司隸之,以土酋白文玉為副長官。二十年遣經歷楊大用往元江等府練兵,時百夷屢為邊患,帝欲發兵平之故也。二十六年置元江府儒學。二十七年,知府那榮及白文玉等來朝貢。

永樂三年,榮復入朝貢。帝厚加賜予,遂改為元江軍民府,給之印信。榮請躬率兵及饋運,往攻八百,帝嘉勞之。元江府又奏,石屏州洛夾橋,每歲江水衝壞,止令本府修理,民不堪,乞命石屏州協治,從之。九年,那榮率頭目人等來朝,貢馬及金銀器,賜予如例。十二年,故土知府那直子那邦入貢方物。

宣德五年,黔國公沐晟奏,元江土知府那忠,被賊刀正、刀龍等焚其廨宇及經歷印信。今獲刀龍、刀洽赴京,乞如永樂故事,發遼東安置,以警邊夷,從之。命禮部鑄印給之。正統元年,因遠羅必甸長官司遣人來朝貢馬。正德二年以那端襲土知府。

嘉靖二十五年,土舍那鑑殺其姪土知府那憲,奪其印,並收因遠驛印記。巡撫應大猷以聞,命鎮巡官發兵剿之。二十九年,那鑑懼,密約交蠻武文淵謀亂。撫按官胡奎、林應箕,總兵官沐朝弼以聞,請以副使李維、參政胡堯時督兵剿之,制可。那鑑益縱兵攻掠村寨。沐朝弼與巡撫石簡調武定、北勝、亦佐等土、漢兵,分五哨。調兵既集,朝弼與簡駐臨安,分部進兵。破木龍寨,降甘莊,賊勢漸蹙。那鑑遣經歷張維及生儒數人詣南羨監督王養浩所乞降。時左布政徐樾以督餉至南羨,樾迂闇,聞維言,謂鑑誠計窮,乃約翼日今鑑面縛出城來降。左右咸謂夷詐不可信,樾不聽,如期親率百人往城下受降。鑒縱象馬夷兵突出沖之,樾及左右皆死。巡按趙炳然以聞,並參朝弼、簡及養浩等失事罪。帝降敕切責,褫簡職,養浩等各住俸,剋期捕賊贖罪。朝弼與簡乃督集五哨兵,環元江而壁。令南羨哨督兵渡江攻城,選路通哨、甘莊哨各精卒二千佐之。那鑒知二哨精卒悉歸南羨,潛遣兵象乘虛衝路通哨。官兵不意賊至,倉猝燒營走。監督郝維嶽奔入甘莊哨,甘莊亦大潰,督哨李維亦遁,惟餘南羨逼城而軍。武定女土官瞿氏、寧州土舍祿紹先、廣南儂兵頭目陸友仁咸恨那鑑戕主奪嫡,誓死不退。督哨王養浩因激獎之,翼日鼓噪攻城,賊大敗,閉門不出。官兵圍之,鑑乞降。官兵懲徐樾之敗,不應。城中析屋而爨,斗米銀三四錢。時瘴毒起,大兵乃復撤,期秋末征之,朝弼以事聞。帝定二哨失事諸臣罪,行撫臣厚賞瞿氏、祿紹先、陸友仁等,敕朝弼會同新撫臣鮑象賢鳩兵討賊。

三十二年,象賢至鎮,調集土、漢兵七萬人,廣集糧運,剋期分哨進剿元江,為必取計。那鑑懼,伏藥死。象賢檄百戶汪輔入城,撫諭其眾,擒其賊首,及戕土官那憲之阿捉,殺布政徐樾之光龍、光色等,皆斬首以獻。鑑子恕輸所占那旂、封鑾等村寨,並出所掠鎮沅府印,納象十二隻,輸屢歲逋賦。象賢命官民推那氏當立者,眾舉前土官那端從孫從仁。象賢疏言其狀,請廢恕,貸其死,命從仁暫統其眾,加汪輔以千戶職,從之。萬曆十三年以元江土舍那恕招降車里功,許襲祖職,賞銀幣。領長官司一,曰因遠羅必甸。

永昌,古哀牢固。漢武帝時,置不韋縣。東漢置瀾滄郡,尋改永昌郡。唐屬姚州,後為南詔蒙氏所據,歷段氏、高氏皆為永昌府。元初,於永昌立三千戶所,隸大理萬戶府。至元間置永昌州,尋為府,隸大理路,置金齒等處宣撫司治。洪武十五年定雲南,立金齒衛。以元雲南右丞觀音保為金齒指揮使,賜姓名李觀。十六年,永昌州土官申保來朝,詔賜錦二匹、織金文綺二匹、衣一襲及鈒花銀帶、鞾襪。十七年以申保為永昌府同知。四月,金齒土官段惠遣把事及其子弟來貢,賜綺帛鈔有差。置施甸長官司,以土酋阿干為副長官,賜冠帶。

十八年置金齒衛指揮使司。二十年,遣使諭金齒衛指揮儲傑、嚴武、李觀曰:「金齒遠在邊徼,土民不遵禮法。爾指揮李觀處事寬厚,名播蠻中,為諸蠻所愛。然其下多恃功放恣,有乖軍律,故特命傑、武輔之。觀之寬,可以綏遠;傑、武之嚴,可以馭下。敕至,其整練諸軍,以觀外變。」

二十三年罷永昌府,改金齒衛為軍民指揮使司。時西平侯沐英言,永昌居民鮮少,宜以府衛合為軍民使司,從之。置鳳谿長官司,以永昌府通判阿鳳為長官。二十四年置永平衛。永樂元年,賜金齒土官百戶汪用鈔一百錠、彩幣四表裏,以西平侯沐晟遣用招安罕的法,故賞之。洪熙元年,金齒軍民指揮司及騰衝守禦千戶所等土官貢馬,賜鈔幣。

宣德五年設金齒軍民指揮使司騰衝州,置土知州一員。時騰衝守禦所土官副千戶張銘言,其地遠在極邊,麓川宣慰思任發不時侵擾,乞設州治。帝從之,即以銘為騰衝知州。八年置騰衝州庫扛關、庫刀關、庫勒關、古湧二關。先是,騰衝州奏,本州路通麓川、緬甸諸處,人民逃徙者多,有誤差發貢獻。舊四百夫長隸騰沖千戶所,其庫扛關等五處,皆軍民兼守。今四百夫已隸本州,止州民守之。乞於五處置巡檢司,以土軍尹黑、張保、李輔、郭節等為巡檢。正統二年以非額革之。嘉靖元年復設永昌軍民府。領州一、縣二。其長官司二,曰施甸,曰鳳谿。

新化,本馬龍、他郎二甸,阿僰諸部蠻據之。元憲宗時內附,立為二千戶所,隸寧州萬戶府。至元間,以馬龍等甸管民官併於他郎甸,置司,隸元江路。洪武初,改名馬龍他郎甸長官司,直隸雲南布政司。後陞為新化州。十七年以普賜為馬龍他郎甸副長官。宣德八年,故長官普賜弟土舍普寧等來朝,貢馬,賜鈔幣。八月,黔國公沐晟奏,摩沙勒寨萬夫長刀甕及弟刀眷糾蠻兵侵占馬龍他郎甸長官司衙門,殺掠人民,請遣都督同知沐昂討之。帝命遣人撫諭,但得刀甕,毋擾平民。正統二年,晟等奏甕不服招撫,請調附近官土兵,令都督昂剿捕。帝以蠻眾仇殺乃其本性,可仍撫諭之,事遂不竟。其地有馬龍諸山,居摩沙勒江右。兩岸束隘如峽,地勢極險,故改州以鎮之。

威遠,唐南詔銀生府地,舊為濮落雜蠻所居。大理時,為百夷所據。元至元中,置威遠州。洪武十五年平雲南後,改威遠蠻棚府為威遠州。三十五年,以土官刀算黨為威遠知州。永樂二年,算黨為車里所擄,奪其地,命西平侯諭之,乃還算黨並侵地。三年,算黨進象馬方物謝,頒降敕諭金字紅牌,賜之金帶、織金文綺、襲衣及銀鈔、錦幣。二十二年,土官刀慶罕等來朝,貢馬及方物,賜慶罕鈔八十錠,糸寧絲、羅紗,及頭目以下,皆有加。

宣德三年,刀慶罕遣頭目招剛、刀著中等來貢,賜予如例,就令齎敕及織金糸寧絲、紗羅賜之,仍給信符、勘合底簿。八年,威遠州奏其地與車里接境,累被各土官劫掠,播孟實當要衝,乞置巡檢司,以把事劉禧為巡檢,從之。

正統二年,土知州刀蓋罕遣人貢馬及銀器,賜彩幣等物,並以新信符給之。正統六年給威遠土知州刀蓋罕金牌,命合兵剿麓川叛寇,以捷聞。敕曰:「叛寇思任發侵爾境土,脅爾從逆。爾母招曩猛能秉大義,效忠朝廷,悉出金貲,分賚頭目。爾母子躬擐甲胄,賈勇殺賊,斬其頭目派罕,追逐餘賊過江,溺死數千,斬首數百,得其戰艦戰象,仍留兵守賊所據江口地。忠義卓然,深足嘉尚。今特升爾正五品,授奉政大夫、修正庶尹,封爾母為太宜人,俱錫誥命、銀帶及彩幣表裏,酬爾母子勛勞。陶孟、刀孟經等亦賜賚有差。爾宜益勉忠義,以副朕懷。」

時西南諸部多相仇殺,所給金牌、信符,燒毀不存。景泰六年,刀蓋罕、隨乃吾等來朝貢,因命其管屬本州人民,復給與金牌、信符、織金文綺,賜敕諭遣之。成化元年,威遠州土舍刀朔罕遣頭目刀昔思貢象馬並金銀器,賜予如例。其俗勇健,男女走險如飛。境內有河,汲水練炭上即成鹽。無秤斗,以簍計多寡量之。

北勝,唐貞元中,南詔異牟尋始開其地,名北方夾,徙瀰河白蠻及羅落、麼些諸蠻,以實其地,號成偈夾,又改名善巨郡。宋時,大理段氏改為成紀鎮。元初,內附。至元中,置施州,尋改北勝州。後為府,隸麗江路軍民宣撫司。洪武十五年改為州,隸鶴慶府,後屬瀾滄衛。永樂五年,土官百夫長楊克即牙舊來貢馬,賜鈔幣。宣德四年,土判官高琳子瑛來貢方物,請襲父職。十年,土知府高瑛來朝貢,賜鈔幣。正統七年,以北勝州直隸雲南布政司,設流官吏目一員,以州蠻苦於瀾滄衛官軍侵漁也。

萬曆四十八年,北勝州土同知高世懋死,異母弟世昌襲。其族姪蘭妄稱世昌奸生,訟之官,不聽。世昌懼逼,走麗江避之。尋還至瀾滄,宿客舍,蘭圍而縱火,殺其家七十餘人,發其祖父墓,自稱欽授把總,大掠。麗江知府木增請討之,謂法紀弁髦,尾大不掉,不治將有隱憂。上官嘉其義,調增率其部進剿,獲蘭梟之。

灣甸,蠻名細夾。元中統初,內附,屬鎮康路。洪武十七年置灣甸縣。永樂元年三月設灣甸長官司,以西平侯沐晟奏地近麓川,地廣人稠故也。尋仍改為灣甸州,以土官刀景發為知州,給印章、金牌並置流官吏目一員。四年,帝以灣甸道里險遠,每歲朝貢,令自今三年一貢,著為令。如慶賀、謝恩之類,不拘此例。六年,刀景發遣人來朝,貢馬及方物,賜鈔幣。七年,刀景發子景懸等來朝,貢馬,賜予如例。宣德八年以土官刀景項弟景辦法繼兄職。州有流官吏目一員。州鄰木邦、順寧,日以侵削。成化五年,灣甸州土官舍人景拙法遣使刀胡猛等來朝,貢象馬並金銀器,賜宴並衣服彩幣有差。

萬曆十一年,土官景宗真率弟宗材導木邦叛賊罕虔入寇姚關,宗真死於陣,擒宗材斬之。景真子幼,貸死,降為州判官。後從討猛廷瑞有功,復舊職。灣甸地多瘴。有黑泉,漲時,飛鳥過之輒墮。

鎮康,蠻名石夾,本黑僰所居。元中統初,內附。至元十三年立鎮康路軍民總管府,領三甸。洪武十五年,改為鎮康府,十七年改為州。永樂二年遣官頒信符及金字紅牌於鎮康州。七年以灣甸同知曩光為知州。初,鎮康地隸灣甸,曩光請增設署所,故有是命。九年以中官徐亮使西南蠻,曩光阻道,詔責之,至是,遣人來朝謝罪。十四年,鎮康州長官司遣人貢馬,賜鈔幣。二十一年,知府刀孟廣來朝,貢馬。宣德三年賜鎮康州土目刀門淵等鈔幣有差。成化五年,知州刀門戛遣使貢馬及金銀器,賜予如例,及妻。

鎮康後亦為木邦、順寧所侵削。隆慶間,知州悶坎者,罕虔妻以女,因附虔歸緬。坎敗死,其弟悶恩歸義。恩死,子悶枳襲,木邦思禮誘之歸緬,不從。天啟二年,木邦兵據喳哩江,枳奔姚關,守備遣官撫之,乃退。

大侯,蠻名孟祐,百夷所居。元中統初內附,屬麓川路。洪武二十四年置大侯長官司。永樂二年頒給信符、金字紅牌。三年,大侯長官司長官刀奉偶遣子刀奉董貢馬及銀器,賜鈔幣。六年,長官刀奉偶遣弟不納狂來貢,賜予如例。

宣德四年陞大侯長官司為大侯州,以土官刀奉漢為知州。時刀奉漢奏:「大侯蠻民復業者多,歲納差發銀二百五十兩。灣甸、鎮康二長官民少,歲納差發銀各百兩,永樂中俱升為州,乞援二州例。」帝諭吏部曰:「大侯民多復業,亦其長官善撫綏也,宜增秩旌之。」故有是命。八年,大侯州入貢,遣內官雲仙往撫之,并賜錦綺有差。

正統三年,土官刀奉漢子刀奉送來貢,命齎敕并織金文綺絨錦諸物,賜刀奉漢并及其妻。初,奉漢令把事傅永瑤來朝,貢馬,奏欲與木邦宣慰罕門法共起土兵十萬,協同征剿麓川,乞賜金牌、信符,以安民心。特賜之,復降敕嘉獎。七年,敕刀奉漢子刀奉送襲大侯知州,賜冠帶、印章、彩段表裏,以奉送能率土兵助討麓川也。十一年,大侯知州奉外法等貢銀器、象馬,賜彩幣、衣服有差。十二年敕賜大侯州奉敬法、刀奉送等并其妻彩幣,命來使齎與之。

萬曆中,土目奉學婿於順寧知府猛廷瑞,後巡撫陳用賓誣奏廷瑞與學反狀,廷瑞斬奉學首以獻,學兄赦守大侯如故。子奉先與其族舍猛麻、奉恭爭殺抗命,次年討平之,改為雲州,設流官。

瀾滄,元為北勝州地。洪武中,屬鶴慶府。二十八年置瀾滄衛。二十九年於州南築城,置今衛司。領北勝、浪渠、永寧三州。永樂四年以永寧州升為府。正統七年以北勝州直隸布政司,今衛只領州一。弘治十一年,福建布政李韶以前任雲南參議,知土俗事宜,上疏言四事。一謂瀾滄衛與北勝州同一城,地域廣遠,與四川建昌西番野番相通。邇年西番土舍章輗等倚恃山險,招服野番千餘家為莊戶,遂致各番生拗,動輒殺人,州官無兵不能禁止。衛官大廢軍政,恬不加意。又姚安府、大羅衛、賓川州地方有賊穴六七,軍民受害。請添設兵備副使於瀾滄衛城,以姚安、大羅、賓川、鶴麗、大理、洱海、景東諸府州衛所,皆令屬之。於野番則用撫流民法,於賊巢則用立保甲法,朝夕經理,則內外寇患皆可弭矣。因從其議,設兵備副使一員於瀾滄城。

麓川、平緬,元時皆屬緬甸。緬甸,古朱波地也。宋寧宗時,緬甸、波斯等國進白象,緬甸之名自此始。緬在雲南之西南,最窮遠。與八百國、占城接境。有城郭室屋,人皆樓居,地產象馬。元時最強盛。元嘗遣使招之,始入貢。

洪武六年遣使田儼、程斗南、張禕、錢允恭齎詔往諭。至安南,留二年,以道阻不通。有詔召之,惟儼還,餘皆道卒。十五年,大兵下雲南,進以大理,下金齒。平緬與金齒壤地相接,土蠻思倫發聞之懼,遂降。因置平緬宣慰使司,以綸發為宣慰使。十七年八月,倫發遣刀令孟獻方物,並上元所授宣慰使司印。詔改平緬宣慰使為平緬軍民宣慰使司,並賜倫發朝服、冠帶及織金文綺、鈔錠。尋改平緬軍民宣慰使司為麓川平緬軍民宣慰使司。麓川與平緬連境,元時分置兩路以統其所部,至是以倫發遣使貢,命兼統麓川之地。

十八年,倫發反,率眾寇景東。都督馮誠率兵擊之,值天大霧,猝遇寇,失利,千戶王升戰死。

二十年,敕諭西平侯沐英等曰:「近御史李原名歸自平緬,知蠻情詭譎,必為邊患。符到,可即於金齒、楚雄、品甸及瀾滄江中道,葺壘深池,以固營柵,多置火銃為守備。寇來。勿輕與戰。又以往歲人至百夷,多貪其財貨,不顧事理,貽笑諸蠻。繼今不許一人往平緬,即文移亦慎答之,毋忽。」明年,倫發誘群蠻入寇馬龍他郎甸之摩沙勒寨。英遣都督寧正擊破之,斬首千五百餘級。倫發悉舉其眾,號三十萬,象百餘,寇定邊,欲報摩沙勒之役,新附諸蠻皆為盡力。英選師三萬亟趨至,賊列象陣搏戰。英列弩注射,突陣大呼,象多傷,其蠻亦多中矢斃,蠻氣稍縮。次日,英率將士,益置火鎗、神機箭,更番射,象奔,賊大敗。搗其寨,斬首三萬餘級,降卒萬餘人。象死者半,生獲三十有七。倫發遁,以捷聞。帝遣使諭英移師逼景東屯田,固壘以待大軍集,勿輕受其降。

二十二年,倫發遣把事招綱等來言:「往者逆謀,皆由把事刀廝郎、刀廝養所為。乞貸死,願輸貢賦。」雲南守臣以聞。乃遣通政司經歷楊大用齎敕往諭思倫發修臣禮,悉償前日兵費,庶免問罪之師。倫發聽命,遂以象、馬、白金、方物入貢謝罪,大用並令獻叛首刀廝郎等一百三十七人,平緬遂平。自是,三年每來朝貢。二十七年,倫發來朝,貢馬、象、方物。已,遣京衛千戶郭均英往賜思倫發公服、襆頭、金帶、象笏。

二十八年,緬國王使來言,百夷屢以兵侵奪其境。明年,緬使復來訴。帝遣行人李思聰等使緬國及百夷。思倫發聞詔,俯伏謝罪,願罷兵。適其部長刀幹孟叛,思聰聽朝廷威德諭其部眾,叛者稍退。思倫發欲倚使者服其下,強留之,以象、馬、金寶為賂,思聰諭卻之。歸述其山川、人物、風俗、道路之詳,為《百夷傳紀》以進,帝褒之。初,平緬俗不好佛。有僧至自雲南,善為因果報應之說,倫發信之。又有金齒戍卒逃入其境,能為火硫、火砲之具,倫發喜其技能,俾繫金帶,與僧位諸部長上。刀幹孟等不服,遂與其屬叛,攻騰衝。倫發率其家走雲南,西平侯沐春遣送至京師。帝憫之,命春為征南將軍,何福、徐凱為副將軍,率雲南、四川諸衛兵往討刀幹孟。並遣倫發歸,駐潞江上,招諭其部眾。賜倫發黃金百兩、白金百五十兩、鈔五百錠。又敕春曰:「思倫發窮而歸我,當以兵送還。若至雲南,先遣人往諭乾孟毋怙終不臣,必歸而主。倘不從,則聲罪討之。」

時幹孟既逐倫發,亦懼朝廷加兵,乃遣人詣西平侯請入貢,春以聞。三十一年奏:「乾孟欲假朝廷威以拒忽都,其言入貢,未可信。」帝遣人諭春曰:「遠蠻詭詐誠有之,姑從所請,審度其宜,毋失事機。」春以兵送倫發於金齒,使人諭刀干孟,乾孟不從。遣左軍都督何福、瞿能等,將兵五千討之。踰高良公山,直搗南甸,大破之,殺刀名孟,斬獲甚眾。回兵擊景罕寨。寨憑高據險,堅守不下,官軍糧械俱盡,賊勢益張。福使告急於春,春率五百騎往救,乘夜至潞江,詰旦渡。率騎馳躪,揚塵蔽天。賊不意大軍至,驚懼,遂破之。乘勝擊崆峒寨,賊夜潰。乾孟遣人乞降,事聞,朝廷以其狡詐,命春俟變討之。春尋病卒,乾孟竟不降。又命都督何福往討,未幾,擒乾孟歸,倫發始還平緬,踰年卒。

永樂元年,思倫發子散朋來朝,貢馬。賜絨錦、織金文綺、紗羅並傔從鈔有差。二年遣內官張勤等頒賜麓川。麓川、平緬、木邦、孟養俱遣人來貢,各賜之鈔幣。時麓川平緬宣慰使思行發所遣頭目刀門賴訴孟養、木邦數侵其地。禮部請以孟養、木邦朝貢使付法司,正其罪。帝謂蠻眾攻奪常事,執一二人罪之,不足以革其俗,且曲直未明,遽罪其使,失遠人心。命西平侯諭之,遣員外郎左緝使八百國,并使賜麓川平緬宣慰冠帶、襲衣。

五年,麓川平緬所隸孟外頭目刀發孟來朝,貢象及金器,散朋亦貢馬,各賜鈔幣。六年,思行發貢馬、方物謝,賜金牌、信符。黔國公沐晟言:「麓川、平緬所隸孟外、陶孟,土官刀發孟之地,為頭目刀薛孟侵據,請命思行發諭刀薛孟歸侵地。」從之。七年,行發來貢,遣中官雲仙等齎敕,賜金織文綺、紗羅。至麓川,行發失郊迎禮,仙責之。行發惶懼,九年遣刀門奈來貢謝罪。帝貸之,仍命宴勞其使,并遣賜行發文錦、金織糸寧絲紗羅。

十一年,行發請以其弟思任發代職,從之。任發遣頭目刀弄發貢象六、馬百匹及金銀器皿等物謝恩。二十年,任發遣使奉表來貢,並謝侵南甸州罪,遣中官雲仙齎賜並敕戒之。洪熙元年遣內官段忠、徐亮以即位詔諭麓川。宣德元年遣使諭西南夷,賜麓川錦綺有差,以其勤修職貢也。時麓川、木邦爭界,各訴於朝,就令使者諭解之,俾安分毋侵越。黔國公沐晟奏,麓川所屬思陀甸火頭曲比為亂,請發兵討,帝命姑撫之。置麓川平緬宣慰司所轄大店地驛丞一員,以土人刀捧怯為之,從宣慰刀暗發奏也。

三年,雲南三司奏,麓川宣慰使思任發奪南甸州地,請發兵問罪。帝命晟同三司、巡撫詳計以聞。敕任發保境安民,不得侵鄰疆,陷惡逆,以滋罪咎。晟以任發侵奪南甸、騰衝之罪不可宥,請發官軍五萬及諸土兵討之。帝以交址、四川方用兵,民勞未息,宜再行招諭。不得已,其調雲南土官軍及木邦宣慰諸蠻兵剿之。八年遣內官雲仙齎敕至麓川,賜思任發幣物,諭其勿與木邦爭地抗殺。

正統元年,免麓川平緬軍民宣慰司所欠差發銀二千五百兩。以任發奏其地為木邦所侵,百姓希少,無從辦納。部執不可,帝特蠲之。初,洪武間,克平雲南,惟百夷部長思倫發未服,後為頭目刀幹孟所逐,赴京陳訴。命為宣慰,回居麓川。分其地,設孟養、木邦、孟定三府,隸雲南;設潞江、乾崖、大侯、灣甸四長官司,隸金齒。永樂元年升孟養、木邦為宣慰司。孟養宣慰刀木旦與鄰境仇殺而死,緬甸乘機并其地。未幾,緬甸宣慰新加斯又為木邦宣慰所殺。時倫發已死,子行發襲,亦死。次子任發襲為麓川宣慰,狡獪愈於父兄,差發金銀,不以時納,朝廷稍優容之。會緬甸之危,任發侵有其地,遂欲盡復其故地,稱兵擾邊,侵孟定府及灣甸等州,殺掠人民。而南甸知州刀貢罕亦奏麓川奪其所轄羅卜思莊等二百七十八村。於是晟奏:「思任發連年累侵孟定、南甸、乾崖、騰衝、潞江、金齒等處,自立頭目刀珍罕、土官早亨等相助為暴,叛形已著。近又侵及金齒,勢甚猖獗。已遣諸衛馬步官軍至金齒守禦,乞調大兵進討。」朝命選將,廷臣舉右都督方政、都督僉事張榮往雲南,協同鎮守右都督昂率兵討之。任發方修貢冀緩師,而晟遽信其降,無渡江意。任發乃遣眾萬餘奪潞江,沿江造船三百艘,欲取雲龍,又殺死甸順、江東等處軍餘殆盡。帝以賊勢日甚,責晟等玩寇養患。政亦至軍,欲出戰,晟不可。政造舟欲濟師,晟又不許。政不勝憤,乃獨率麾下與賊將緬簡戰,破賊舊大寨。賊奔景罕,指揮唐清復擊破之。又追之高黎共山下,共斬三千餘級。乘勝深入,逼任發上江。上江,賊重地也。政遠攻疲甚,求援於晟,晟怒其違節制渡江,不遣。久之,以少兵往,至夾象石,又不進。政追至空泥,知晟不救,賊出象陣衝擊,軍殲,政死焉。晟聞敗,乃請益軍。帝遣使者責狀,仍調湖廣官軍三萬一千五百人、貴州一萬人、四川八千五百人,令吳亮、馬翔統之,至雲南,聽晟節制,仍敕晟豫籌糧Я。而晟懼罪,暴卒。

時任發兵愈橫,犯景東,剽孟定,殺大侯知州刀奉漢等千餘人,破孟賴諸寨,孟璉長官司諸處皆降之。任發仍遣人以象馬金銀來修貢,復致番書於雲南總兵官,謂:「始因潞江安撫司線舊法相邀報仇,其後線舊法乃誣己為入寇,致大軍壓境,惶恐無地。今欲遣使謝罪,乞為導奏。」帝降敕許赦其罪。時刑部侍郎何文淵疏請罷麓川師,命下廷臣議。於是行在兵部尚書王驥及英國公張輔等,皆以為「麓川負恩怙惡,在所必誅,須更選將練兵,以昭天討。如思任發早自悔禍,縛詣軍門,生全之恩,取自上裁。」帝然之。已而侍講劉球復以息兵請如文淵議。部覆以麓川之徵,已有成命,報聞。

六年以定西伯蔣貴為平蠻將軍,都督李安、劉聚副之,以兵部尚書王驥總督雲南軍務,大會諸道兵十五萬討之。時任發遣賊將刀令道等十二人,率眾三萬餘,象八十隻,抵大侯州,欲奪景東、威遠。而驥將抵金齒,任發遣人乞降,驥受之,密令諸將分道入。右參將冉保從東路攻細甸、灣甸水寨,入鎮康,趨孟定。驥與貴由中路至上江,會騰衝。左參將宮聚自下江據夾象石。至期,合攻之。賊拒守嚴,銃弩飛石,交下如雨。次日,乘風焚其柵,火竟夜不息。官軍力戰,拔上江寨,斬刀放戛父子,擒刀孟項,前後斬馘五萬餘,以捷聞。

七年,驥率兵渡下江,通高黎貢山道。至騰衝,留都督李安領兵提備。驥由南甸至羅卜思莊,前軍抵杉木籠。時任發率眾二萬餘據高山,立硬寨,連環七營,首尾相應。驥遣宮聚、劉聚分左右翼緣嶺上,驥將中軍橫擊之,賊遁。軍進馬鞍山,搗賊寨。寨兩面拒江壁立,周迴三十里皆立柵開塹,軍不可進,而賊從間道潛師出馬鞍山後。驥戒中軍毋動,命指揮方瑛率精騎六千突入賊寨,斬首數百級,復誘敗其象陣。而從東路者,合木邦人馬,招降孟通諸寨。元江同知杜凱等亦率車里及大侯蠻兵五萬,招降孟璉長官司並攻破烏木弄、戛邦等寨,斬首二千三百餘級。齊集麓川,守西峨渡,就通木邦信息。百道環攻,復縱火焚其營,賊死不可勝算。任發父子三人並挈其妻孥數人,從間道渡江,奔孟養。搜獲原給虎符、金牌、信符、宣慰司印及所掠騰沖千戶等印三十二。麓川平。捷聞,命還師。時任發敗走孟蒙,復為木邦宣慰所擊,追過金沙江,走孟廣。緬甸宣慰卜剌當亦起兵攻之。帝命木邦、緬甸能效命擒任發獻者,即以麓川地與之。未幾,任發為緬人擒,緬人挾之求地。其子思機發窮困,乞來朝謝罪,先遣其弟招賽入貢,帝命遣還雲南安置。機發窺大兵歸,圖恢復,據麓川出兵侵擾。於是復命王驥、蔣貴等統大軍再徵麓川。驥率師至金齒,機發遣頭目刀籠肘偕其子詣軍門求降。驥遣人至緬甸索任發,緬佯諾不遣。驥至騰衝,與蔣貴、沐昂分五營進,緬人亦聚眾待。驥欲乘大師攻之,見其眾盛,未易拔,又恐多一麓川敵,乃宣言犒師,而命貴潛焚其舟數百艘,進師薄之。緬甸堅執前詔,必予地乃出任發,復詭以機發致仇為解。驥乃趨者藍,搗機發巢,破之。機發脫走,俘其妻子部眾,立隴川宣慰司而歸。時思機發竊據孟養,負固不服,自如也。

十一年,緬甸始以任發及其妻孥三十二人獻至雲南。任發於道中不食,垂死。千戶王政斬之,函首京師。其子機發屢乞降,遣頭目刀孟永等修朝貢,獻金銀。言蒙朝廷調兵征討,無地逃死,乞貸餘生,詞甚哀。帝命受其貢,因敕總兵官沐斌及參贊軍務侍郎楊寧等,以朝廷既貸思機發以不死,經畫善後長策以聞,並賜敕諭思機發。十二年,總兵官黔國公沐斌奏:「臣遣千戶明庸齎敕招諭思機發,以所遣弟招賽未歸,疑懼不敢出。近緬甸以機發掠其牛馬、金銀,欲進兵攻取。臣等議遣人分諭木邦、緬甸諸宣慰司,令集蠻兵,剋期過江,分道討機發。臣等率官軍萬人駐騰衝,以助其勢。賊四面受敵,必成擒矣。」從之。已,命授機發弟招賽為頭目,給冠帶、月糧、房屋,隸錦衣衛,其從人俱令於馴象所供役。先是,招賽安置雲南,其黨有欲稱亂者,乃命招賽來京,且冀以招徠機發也。帝既命雲南出兵剿機發,及沐斌等至騰衝,督諸軍追捕,機發終不出,潛匿孟養,遣其徒來貢。許以恩貸,復不至。斌以春瘴作,江漲不可渡,糧亦乏,引兵還。帝以斌師出無功,復命兵部尚書靖遠伯王驥總督軍務,都督同知宮聚佩平蠻將軍印,率南京、雲南、湖廣、四川、貴州官軍、土軍十三萬人往討之。至是,驥凡三征麓川矣。帝密諭驥曰:「萬一思機發遠遁,則先擒刀變蠻,平其巢穴。或遁入緬地,緬人黨蔽,亦相機擒之。庶蠻眾知懼,大軍不為徒出。」又敕諭斌,軍事悉與驥會議而行。又敕諭木邦、緬甸、南甸、乾崖、隴川等宣慰司罕蓋發等,各整兵備船,積糧以俟調度。

十四年,驥率諸將自騰衝會師,由干崖造舟,至南牙山舍舟陸行,抵沙壩,復造舟至金沙江。機發於西岸埋柵拒守。大軍順流下至管屯,適木邦、緬甸兩宣慰兵十餘萬亦列於沿江兩岸,緬甸備舟二百餘為浮梁濟師,併力攻破其柵寨,得積穀四十萬餘石。軍飽,銳氣增倍。賊領眾至鬼哭山,築大寨於兩峰上,築二寨為兩翼,又築七小寨,綿亙百餘里。官軍分道並進,皆攻拔之,斬獲無算,而思機發、思卜發復奔遁。

時王師踰孟養至孟那。孟養在金沙江西,去麓川千餘里,諸部皆震讋曰:「自古,漢人無渡金沙江者,今王師至此,真天威也。」驥還兵,其部眾復擁任發少子思祿據孟養地為亂。驥等慮師老,度賊不可滅,乃與思祿約,許土目得部勒諸蠻,居孟養如故,立石金沙江為界,誓曰「石爛江枯,爾乃得渡」。思祿亦懼,聽命,乃班師。捷聞,帝為告廟云。

景泰元年,雲南總兵官沐璘奏:「緬甸宣慰已擒獲思機發,又將思卜發放歸孟養,恐緬人復挾為奇貨,不若緩之,聽其自獻便。」從之。五年,緬人索舊地,左參將胡誌等諭以戛等處地方與之,乃送思機發及其妻孥六人至金沙江村,志等檻送京師。南寧伯毛福壽以聞,乃誅思機發於京師。七年,任發子思卜發奏:「臣父兄犯法,時臣幼無知。今不敢如父兄所為,甚畏朝廷法,謹備差發銀五百兩、象三、馬六及方物等,遣使人入貢,惟天皇帝主哀憐。」因賜敕戒諭,並賚思卜發與妻錦幣及其使鈔幣有差。

成化元年,總兵官沐瓚等以思任發之孫思命發至京師,乃逆賊遺孽,不可留,請發沿海登州衛安置,月給米二石,從之。麓川亡。先是,麓川之初平也,分其地立隴川宣撫使司,因以恭項為宣撫使。恭項者,故麓川部長,首先歸順效力有功,因命於麓川故地開設宣撫。已,頭目曩渙等復來歸,願捕賊自效。帝命還守本土,有功,即加敘。諸凡來歸者視此例。遂以刀歪孟為本司同知,刀落曩為副使,隴帚為僉事,俱賜冠帶,從宣撫恭項請也。恭項子恭立來貢,給賜如例,並授恭立為長史。未幾,隴川宣撫失印,請再給。帝責恭項以不能宣揚國威,反失印,罪應不宥,姑從寬頒。時板蹇據者藍寨,侵擾隴川,百夫長刀門線、刀木立進兵圍之,斬板蹇等二十三人。命賜有功者皆為冠帶把事,並賚織金文綺。

正統十一年,木邦宣慰罕蓋發來求麓川故地。有司以已設隴川宣撫司,建官分管,以孟止地予之,報可。十二年敕諭恭項,言:「比者,總兵奏爾與百夫長刀木立相仇殺,人民懷怨,欲謀害爾父子。今遷爾於雲南,俾不失所,且遣官護爾家屬完聚,其體憫恤,無懷疑懼。」既而總兵官言:「隴川致亂,皆由恭項暴殺無辜,刻虐蠻人。同知刀歪孟為蠻眾信服,乞安置項於別衛,以刀歪孟代。」帝以恭項來歸,屈法宥之,命於曲靖安置,並遣敕往諭。

景泰七年,隴川宣撫多外悶遣人貢象、馬及金銀器皿、方物,賜彩幣、襲衣如例。仍命齎敕賜之,以多外悶初修朝貢故也。成化十九年,以隴川宣撫司多歪孟子亨法代職。初,隴川與木邦相鄰,爭地仇殺,構兵不息。嘉靖中,土舍多鯨刃兄自襲,下鎮巡官按問,伏辜,還職兄子多參。詔貰其罪,並戒木邦罕孟毋得復黨鯨爭職。

萬曆初,緬甸莽瑞體叛,來招隴川宣撫多士寧,士寧不從。其記室岳鳳者,江西撫州人,黠而多智,商於隴川,士寧信任之,妻以妹。鳳曲媚士寧,陰奪其權,與三宣六慰各土舍罕拔等歃血盟,誘士寧往擺古,歸附緬酋。陰使其子曩烏鴆士寧並殺其妻女,奪印投緬,受緬偽命,代士寧為宣撫。及瑞體死,子應裏嗣,鳳父子臣服之。誘敗官軍,獻士寧母胡氏及親族六百餘人於應裏,盡殺之,多氏之宗幾盡。初,鳳之附於緬也,為瑞體招諸部,拒中國,傷官軍,逆勢浸成,緬深倚之。久之,以緬不足恃。而鄧川土知州何鈺,鳳友婿也,初使人招鳳,鳳執使獻緬。及是,鈺復開示百方,與之盟誓。時官軍亦大集,諸將劉綎、鄧子龍各率勁師至,環壁四面。鳳懼,乃令妻子及部曲來降。綎責令獻金牌、符印及蠻莫、猛密地。乃以送鳳妻子還隴川為名,分兵趨沙木籠山,先據其險,而自領大兵馳入隴川。鳳度無可脫,遂詣軍門降。綎復率兵進緬,緬將先遁,留少兵隴川,綎攻之,鳳子曩烏亦降,綎乃攜鳳父子往攻蠻莫,蠻莫賊知鳳降,馳報應裏,發兵圖隴川。綎乘機掩殺,賊窘,乞降,縛緬人及象馬來獻。遂招撫孟養賊,賊將乘象走,追獲之。復移師圍孟璉,生擒其魁,隴川平。獻俘於朝,帝為告謝郊廟,時萬歷十二年九月也。踰年復鑄隴川宣撫司及孟定府印,升孟密安撫為宣撫司。添設安撫司二,曰蠻莫,曰耿馬;長官司二,曰孟璉,曰孟養;千戶所二,一居姚關,一居孟淋砦,皆名之曰鎮安;並鑄印記,建大將行署於蠻莫。從雲南巡撫劉世曾之議也。於是,多士寧之子思順襲隴川宣撫使。

二十九年,莽應裏分道入犯,一入遮放、芒市,一入臘撒蠻顙,一入杉木籠,並出隴川。多思順不敵,奔猛卯。緬初以猛卯同知多俺為向導,寇東路。至是大軍遣木邦罕欽擒多俺殺之。未幾,思順死,蠻莫思正乘喪襲隴川,據其妻罕氏。三十五年,思順子安民以守將索賂,叛入緬。已而緬聽撫,遣安民歸。安民久據蠻灣,桀驁甚,署永騰參將周會遣二指揮襲之,敗績。王師亟討,其族人挾其弟多安靖誅之以獻。時安靖尚幼,勢孤,詔俟其長給之印。安民弟安邦治亦附緬,後寄居蠻莫。其地有馬安、摩黎、羅木等山,極險峻,麓川之所恃為巢穴者也。


\end{pinyinscope}