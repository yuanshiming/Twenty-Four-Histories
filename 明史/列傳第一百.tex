\article{列傳第一百}

\begin{pinyinscope}
俞大猷盧鏜湯克寬戚繼光弟繼美硃先劉顯郭成李錫黃應甲尹鳳張元勛

俞大猷,字志輔,晉江人。少好讀書。受《易》於王宣、林福,得蔡清之傳。又聞趙本學以《易》推衍兵家奇正虛實之權,復從受其業。嘗謂兵法之數起五,猶一人之身有五體,雖將百萬,可使合為一人也。已,又從李良欽學劍。家貧屢空,意嘗豁如。父歿,棄諸生,嗣世職百戶。

舉嘉靖十四年武會試。除千戶,守禦金門。軍民囂訟難治,大猷導以禮讓,訟為衰止。海寇頻發,上書監司論其事。監司怒曰:「小校安得上書?」杖之,奪其職。尚書毛伯溫征安南,復上書陳方略,請從軍。伯溫奇之。會兵罷,不果用。

二十一年,俺答大入山西,詔天下舉武勇士。大猷詣巡按御史自薦,御史上其名兵部。會伯溫為尚書,送之宣大總督翟鵬所。召見論兵事,大猷屢折鵬。鵬謝曰:「吾不當以武人待子。」下堂禮之,驚一軍,然亦不能用。大猷辭歸,伯溫用為汀漳守備。蒞武平,作讀易軒,與諸生為文會,而日教武士擊劍。連破海賊康老,俘斬三百餘人。擢署都指揮僉事,僉書廣東都司。新興恩平峒賊譚元清等屢叛,總督歐陽必進以屬大猷。乃令良民自為守,而親率數人遍詣賊峒,曉以禍福,且教之擊劍,賊駭服。有蘇青蛇者,力格猛虎,大猷紿斬之,賊益驚。乃詣何老貓峒,令歸民侵田,而招降渠魁數輩。二邑以寧。

二十八年,朱紈巡視福建,薦為備倭都指揮。會安南入寇,必進奏留之。先是,安南都統使莫福海卒,子宏瀷幼。其大臣阮敬謀立其婿莫敬典,范子儀謀立其黨莫正中,互讎殺。正中敗,挈百餘人來歸。子儀收殘卒遁海東。至是妄言宏瀷死,迎正中歸立。剽掠欽、廉等州,嶺海騷動。必進檄大猷討之。馳至廉州,賊攻城方急。大猷以舟師未集,遣數騎諭降,且聲言大兵至。賊不測,果解去。無何,舟師至,設伏冠頭嶺。賊犯欽州,大猷遮奪其舟。追戰數日,生擒子儀弟子流,斬首千二百級。窮追至海東雲屯,檄宏瀷殺子儀函首來獻。事平,嚴嵩抑其功不敘,但賚銀五十兩而已。

是年,瓊州五指山黎那燕構感恩、昌化諸黎共反,必進復檄大猷討。而朝議設參將於崖州,即以大猷任之。乃會廣西副將沈希儀諸軍,擒斬賊五千三百有奇,招降者三千七百。大猷言於必進曰:「黎亦人也,率數年一反一征,豈上天生人意?宜建城設市,用漢法雜治之。」必進納其言。大猷乃單騎入峒,與黎定要約,海南遂安。

三十一年,倭賊大擾浙東。詔移大猷寧、台諸郡參將。會賊破寧波昌國衛,大猷擊卻之。復攻陷紹興臨山衛,轉掠至松陽。知縣羅拱辰力禦賊,而大猷邀諸海,斬獲多,竟坐失事停俸。未幾,逐賊海中,焚其船五十餘,予俸如故。越二年,賊據寧波普陀。大猷率將士攻之,半登,賊突出,殺武舉火斌等三百人,坐戴罪辦賊。俄敗賊吳淞所,詔除前罪,仍賚銀幣。賊自健跳所入掠,大猷運戰破之。旋代湯克寬為蘇松副總兵。所將卒不三百人,徵諸道兵未集,賊犯金山,大猷戰失利。時倭屯松江枯林者盈二萬,總督張經趣之戰,大猷固不可。及永順、保靖兵稍至,乃從經大破賊於王江涇,功為趙文華、胡宗憲所攘,不敘。坐金山失律,謫充為事官。

柘林倭雖敗,而新倭三十餘艘突青村所,與南沙、小烏口、浪港諸賊合,犯蘇州陸涇壩,直抵婁門,敗南京都督周于德兵。賊復分為二,北掠滸墅,南掠橫塘,延蔓常熟、江陰、無錫之境,出入太湖。大猷偕副使任環大敗賊陸涇壩,焚舟三十餘。又遮擊其自三丈浦出海者,沉七艘,賊乃退泊三板沙。頃之,他倭犯吳江。大猷及環又邀破之鶯脰湖,賊走嘉興。

三板沙賊掠民舟將遁,大猷追擊於馬蹟山,擒其魁。金涇、許浦、白茅港賊俱出海,大猷追擊於茶山,焚五舟。賊走保馬蹟山、三板沙,將士復追及,壞其三舟。江陰蔡港倭亦去,官兵分擊於馬蹟、馬圖、寶山。值颶風作,賊舟多覆。柘林倭亦為官兵所擊沉二十餘舟,餘賊退登陸。已,復泛舟出海。大猷及僉事董邦政分擊,獲九舟。而賊又遭風壞三舟,餘三百人登岸,走據華亭陶宅鎮,屢敗趙文華等大軍。夜屯周浦永定寺,官兵四集進圍之。而柘林失風賊九舟巢於川沙窪,糾合至四十餘艘,勢猶未已。巡撫曹邦輔劾大猷縱賊,帝怒,奪其世廕,責取死罪招,立功自贖。時周浦賊圍急,乘夜東北奔,為遊擊曹克新所邀,斬首百三十,遂與川沙窪賊合。諸軍日夜擊海。大猷偕副使王崇古入洋追之,及於老鸛觜,焚巨艦八,斬獲無算。餘賊奔上海浦東。

初,以倭患急,特命都督劉遠為浙江總兵官,兼轄蘇、松諸郡,數月無所為。廷臣爭言大猷才。三十五年三月遂罷遠,以大猷代。賊犯西庵、沈莊及清水窪。大猷偕邦政擊敗之,賊走陶山,詔還世廕。賊自黃浦遁出海,大猷追敗之。其年冬,以與平徐海功,加都督僉事。海既平,浙西倭悉靖。獨寧波舟山倭負險,官兵環守不能克。是時土兵狼兵悉已遣歸,而川、貴所調麻寮、大剌、鎮溪、桑植兵六千始至。大猷乘大雪,四面攻之。賊死戰,殺土官一人。諸軍益競,進焚其柵,賊多死,其逸出者復殪,賊盡平。加大猷署都督同知。

明年,胡宗憲方圖汪直,用盧鏜言將與通市,大猷力爭不可。及直誘入下吏,其黨毛海峰等遂據舟山,阻岑港自守。大猷環攻之,時小勝。然苦仰攻,將士先登多死,新倭又大至。朝廷趣宗憲甚急,宗憲謾為大言以對。廷臣競詆宗憲,并劾大猷。乃奪大猷及參將戚繼光職,期一月內平賊。大猷等懼,攻益力,賊益死守。三十七年七月乃自岑港移柯梅,造舟成,泛海去。大猷等橫擊之,沈其一舟,餘賊遂揚帆而南,流劫閩、廣。大猷先後殺倭四五千,賊幾平。而官軍圍賊已一年,宗憲亦利其去,陰縱之,不督諸將邀擊。比為御史李瑚所劾,則委罪大猷縱賊以自解。帝怒,逮繫詔獄,再奪世廕。

陸炳與大猷善,密以己資投嚴世蕃解其獄,令立功塞上。大同巡撫李文進習其才,與籌軍事。乃造獨輪車拒敵馬。嘗以車百輛,步騎三千,大挫敵安銀堡。文進上其制於朝,遂置兵車營。京營有兵車,自此始也。文進將襲板升,謀之大猷,果大獲,詔還世廕。寇掠廣武,大猷拒卻之。先論平汪直功,許除罪錄用。及是,鎮篁有警,川湖總督黃光昇薦大猷,即用為鎮篁參將。

廣東饒平賊張璉數攻陷城邑,積年不能平。四十年七月詔移大猷南贛,合閩、廣兵討之。時宗憲兼制江西,知璉遠出,檄大猷急擊。大猷謂:「宜以潛師搗其巢,攻其必救,奈何以數萬眾從一夫浪走哉?」乃疾引萬五千人登柏嵩嶺,俯瞰賊巢。璉果還救,大猷連破之,斬首千二百餘級。賊懼,不出。用間誘璉出戰,從陣後執之,并執賊魁蕭雪峰。廣人攘其功,大猷不與較。散餘黨二萬,不戮一人。擢副總兵,協守南、贛、汀、漳、惠、湖諸郡。遂乘勝徵程鄉盜,走梁寧,擒徐東洲。林朝曦者,獨約黃積山大舉。官軍攻斬積山,朝曦遁,後亦為徐甫宰所滅。大猷尋擢福建總兵官,與戚繼光復興化城,共破海倭。詳《繼光傳》。繼光先登,受上賞,大猷但賚銀幣。

四十二年十月徙鎮南贛。明年改廣東。潮州倭二萬與大盜吳平相掎角,而諸峒藍松三、伍端、溫七、葉丹樓輩日掠惠、潮間。閩則程紹錄亂延平,梁道輝擾汀州。大猷以威名懾群盜,單騎入紹祿營,督使歸峒,因令驅道輝歸,兩人卒為他將所滅。惠州參將謝敕與伍端、溫七戰,失利。以「俞家軍」至,恐之,端乃驅諸酋以歸。無何,大猷果至,七被擒。端自縛,乞殺倭自效。大猷使先驅,官軍繼之,圍倭鄒塘,一日夜克三巢,焚斬四百有奇,又大破之海豐。倭悉奔崎沙、甲子諸澳,奪漁舟入海。舟多沒於風,脫者二千餘人,還保海豐金錫都。大猷圍之兩月,賊食盡,欲走。副將湯克寬設伏邀之,手斬其梟將三人。參將王詔等繼至,賊遂大潰。乃移師潮州,以次降藍松三、葉丹樓。遂使招降吳平,居之梅嶺。平未幾復叛,造戰艦數百,聚眾萬餘,築三城守之,行劫濱海諸郡縣。福建總兵官戚繼光襲平,平遁保南澳。四十四年秋入犯福建,把總朱璣等戰沒於海中。大猷將水兵,繼光將陸兵,夾擊平南澳,大破之。平僅以身免,奔據饒平鳳凰山。繼光留南澳。大猷部將湯克寬、李超等躡賊後,連戰不利,平遂掠民舟出海。閩廣巡按御史交章論之,大猷坐奪職。平卒為克寬所追擊,遠遁以免,不敢入犯矣。

河源、翁源賊李亞元等猖獗。總督吳桂芳留大猷討之,徵兵十萬,分五哨進。大猷使間攜賊黨而親搗其巢,生擒亞元,俘斬一萬四百,奪還男婦八萬餘人。乃還大猷職,以為廣西總兵官。故事:以勳臣總兩廣兵,與總督同鎮梧州。帝用給事中歐陽一敬議,兩廣各置大帥,罷勳臣,乃召恭順侯吳繼爵還京,以大猷代,予平蠻將軍印。而以劉顯鎮廣東。兩廣並置帥,自大猷及顯始也。伍端死,其黨王世橋復叛,劫執同知郭文通。大猷連敗之,其部下執以獻。進署都督同知。

海賊曾一本者,吳平黨也。既降復叛,執澄海知縣,敗官軍,守備李茂才中炮死。詔大猷暫督廣東兵協討。隆慶二年,一本犯廣州,尋犯福建。大猷合郭成、李錫軍擒滅之。錄功,進右都督。

廣西古田僮黃朝猛、韋銀豹等,嘉靖末嘗再劫會城庫,殺參政黎民表。巡撫殷正茂徵兵十四萬,屬大猷討之。分七道進,連破數十巢。賊保潮水,巢極巔,攻十餘日未下。大猷佯分兵擊馬浪賊,而密令參將王世科乘雨夜登山設伏。黎明炮發,賊大驚。諸軍攀援上,賊盡死。馬浪諸巢相繼下。斬獲八千四百有奇,擒朝猛、銀豹,百年積寇盡除。進世廕為指揮僉事。

大猷為將廉,馭下有恩。數建大功,威名震南服。而巡按李良臣劾其奸貪,兵部力持之,詔還籍候調。起南京右府僉書。未任,以都督僉事為福建總兵官。萬曆元年秋,海寇突閭峽澳,坐失利奪職。復以署都督僉事起後府僉書,領車營訓練。三疏乞歸。卒,贈左都督,謚武襄。

大猷負奇節,以古賢豪自期。其用兵,先計後戰,不貪近功。忠誠許國,老而彌篤,所在有大勛。武平、崖州、饒平旨為祠祀。譚綸嘗與書曰:「節制精明,公不如綸。信賞必罰,公不如戚。精悍馳騁,公不如劉。然此皆小知,而公則甚大受。」戚謂威繼光,劉謂劉顯也。

子咨皋,福建總兵官。

盧鏜,汝寧衛人。嘉靖時由世廕歷福建都指揮僉事,為都御史朱紈所任。紈自殺,鏜亦論死。尋赦免,以故官備倭福建。遷都指揮。擊賊嘉興,敗,責戴罪。尋擢參將,分守浙東濱海諸郡,與副將大猷大破賊王江涇。旋督保靖土兵及蜀將陳正元兵擊賊張莊,焚其壘。追擊之後港,為賊所敗。賊出沒台州外海,都指揮王沛敗之大陳山。賊登山,官軍焚其舟。鏜會剿,擒其酋林碧川等,餘倭盡滅。別賊掠諸縣,指揮閔溶等敗死,鏜奪職,戴罪。

旋以薦擢協守江浙副總兵。賊陷仙居,趨台州,鏜破之彭溪。乃與胡宗憲共謀滅徐海。宗憲招汪直,鏜亦說日本使善妙令擒直。直與日本貳,卒伏誅。倭犯江北,鏜馳援破之,又敗北洋倭二十餘舨。賊斂舟三沙,復流劫江北。巡撫李遂劾鏜縱賊,鏜已擢都僉事,為江南、浙江總兵官,奪職視事。以通政唐順之薦復職如初。尋以誅汪直功,進都督同知。倭復犯浙東。水陸十餘戰,斬首千四百有奇。總督宗憲以蕩平聞,鏜復增俸賚金。鏜擢用由宗憲,宗憲敗,給事中丘橓劾鏜八罪。逮治,免歸。

鏜有將略。倭難初興,諸將悉望風潰敗,獨鏜與湯克寬敢戰,名亞俞、戚云。

克寬,邳州衛人。父慶,嘉靖中江防總兵官。克寬承世廕,歷官都指揮僉事,充浙江參將。倭犯溫州,克寬擊敗之。別賊寇嘉興屬邑,克寬至海鹽被圍。偕參政潘恩等拒守,賊不能克,乃焚掠而去。無何,陷乍浦城,轉掠奉化、寧海。克寬追圍於獨山民家,火焚之。賊半死,餘奪圍遁。

時濱海多被倭患,而將士無紀律,賊至輒奔,議設大將統制江、淮。乃命克寬為副總兵,駐金山衛,提督海防諸軍。倭三百人泊崇明南沙。克寬偕僉事任環攻之,敗績。賊移舟寶山,克寬追敗之南家觜。賊乃轉寇嘉定、上海間,被劾奪官從軍。倭二千餘分掠蘇、松。克寬逆戰採淘港,斬首八百餘級。都御史王忬薦為浙西參將。遇賊嘉、湖,復失利,詔以白衣辦賊。總督張經議搗賊柘林,令克寬將廣西土兵屯乍浦,與副將大猷等相掎角。大戰王江涇,斬級二千。會趙文華劾經惑克寬言縱倭飽揚,遂併逮問,論死。久之,赦免。

廣東用兵,命赴軍前自效。從大猷大破倭海豐,還世廕。俄以為惠、潮參將,復從大猷破吳平。平未幾復振,克寬已擢狼山副總兵,命留討賊。俄敗之陽江烏豬洋。平窘,奔安南。都御史吳桂芳檄安南協討,遣克寬以舟師會,夾擊平萬橋山下。焚其舟,擒斬四百人,平遠竄。乃進克寬署都督僉事,為廣東總兵官。曾一本突海豐、惠來間,克寬倡議撫之,令居潮陽下澮地。未幾,激民變,一本亦反,詔逮克寬訊治。尋赦免,赴蘇鎮立功。萬曆四年,炒蠻入掠古北口。克寬偕參將苑宗儒追出塞,遇伏,戰死。

戚繼光,字元敬,世登州衛指揮僉事。父景通,歷官都指揮,署大寧都司,入為神機坐營,有操行。繼光幼倜儻負奇氣。家貧,好讀書,通經史大義。嘉靖中嗣職,用薦擢署都指揮僉事,備倭山東。改僉浙江都司,充參將,分部寧、紹、台三郡。

三十六年,倭犯樂清、瑞安、臨海,繼光援不及,以道阻不罪。尋會俞大猷兵,圍汪直餘黨於岑港。久不克,坐免官,戴罪辦賊。已而倭遁,他倭復焚掠台州。給事中羅嘉賓等劾繼光無功,且通番。方按問,旋以平汪直功復官,改守台、金、嚴三郡。

繼光至浙時,見衛所軍不習戰,而金華、義烏俗稱慓悍,請召募三千人,教以擊刺法,長短兵迭用,由是繼光一軍特精。又以南方多藪澤,不利馳逐,乃因地形制陣法,審步伐便利,一切戰艦、火器、兵械精求而更置之。「戚家軍」名聞天下。

四十年,倭大掠桃渚、圻頭。繼光急趨寧海,扼桃渚,敗之龍山,追至鴈門嶺。賊遁去,乘虛襲台州。繼光手殲其魁,蹙餘賊瓜陵江盡死。而圻頭倭復趨台州,繼光邀擊之仙居,道無脫者。先後九戰皆捷,俘馘一千有奇,焚溺死者無算。總兵官盧鏜、參將牛天錫又破賊寧波、溫州。浙東平,繼光進秩三等。閩、廣賊流入江西。總督胡宗憲檄繼光援。擊破之上坊巢,賊奔建寧。繼光還浙江。

明年,倭大舉犯福建。自溫州來者,合福寧、連江諸倭攻陷壽寧、政和、寧德。自廣東南澳來者,合福清、長樂諸倭攻陷玄鐘所,延及龍嚴、松溪、大田、古田、莆田。是時寧德已屢陷。距城十里有橫嶼,四面皆水路險隘,賊結大營其中。官軍不敢擊,相守踰年。其新至者營牛田,而酋長營興化,東南互為聲援。閩中連告急,宗憲復檄繼光剿之。先擊橫嶼賊。人持草一束,填壕進。大破其巢,斬首二千六百。乘勝至福清,搗敗牛田賊,覆其巢,餘賊走興化。急追之,夜四鼓抵賊柵。連克六十營,斬首千數百級。平明入城,興化人始知,牛酒勞不絕。繼光乃旋師。抵福清,遇倭自東營澳登陸,擊斬二百人。而劉顯亦屢破賊。閩宿寇幾盡。於是繼光至福州飲至,勒石平遠臺。

及繼光還浙後,新倭至者日益眾,圍興化城匝月。會顯遣卒八人齎書城中,衣刺「天兵」二字。賊殺而衣其衣,紿守將得人,夜斬關延賊。副使翁時器、參將畢高走免,通判奚世亮攝府事,遇害,焚掠一空。留兩月,破平海衛,據之。初,興化告急,時帝已命俞大猷為福建總兵官,繼光副之。及城陷,劉顯軍少,壁城下不敢擊。大猷亦不欲攻,需大軍合以困之。四十二年四月,繼光將浙兵至。於是巡撫譚綸令將中軍,顯左,大猷右,合攻賊於平海。繼光先登,左右軍繼之,斬級二千二百,還被掠者三千人。綸上功,繼光首,顯、大猷次之。帝為告謝郊廟,大行敘賚。繼光先以橫嶼功,進署都督僉事,及是進都督同知,世廕千戶,遂代大猷為總兵官。

明年二月,倭餘黨復糾新倭萬餘,圍仙遊三日。繼光擊敗之城下,又追敗之王倉坪,斬首數百級,餘多墜崖谷死,存者數千奔據漳浦蔡丕嶺。繼光分五哨,身持短兵緣崖上,俘斬數百人,餘賊遂掠漁舟出海去。久之,倭自浙犯福寧,繼光督參將李超等擊敗之。乘勝追永寧賊,斬馘三百有奇。尋與大猷擊走吳平於南澳,遂擊平餘孽之未下者。

繼光為將號令嚴,賞罰信,士無敢不用命。與大奠均為名將。操行不如,而果毅過之。大猷老將務持重,繼光則飆發電舉,屢摧大寇,名更出大猷上。

隆慶初,給事中吳時來以薊門多警,請召大猷、繼光專訓邊卒。部議獨用繼光,乃召為神機營副將。會譚綸督師遼、薊,乃集步兵三萬,徵浙兵三千,請專屬繼光訓練。帝可之。二年五月命以都督同知總理薊州、昌平、保定三鎮練兵事,總兵官以下悉受節制。至鎮,上疏言:

薊門之兵,雖多亦少。其原有七營軍不習戎事,而好末技,壯者役將門,老弱僅充伍,一也。邊塞逶迄,絕鮮郵置,使客絡釋,日事將迎,參游為驛使,營壘皆傳舍,二也。寇至,則調遣無法,遠道赴期,卒斃馬僵,三也。守塞之卒約束不明,行伍不整,四也。臨陣馬軍不用馬,而反用步,五也。家丁盛而軍心離,六也。乘障卒不擇衝緩,備多力分,七也。七害不除,邊備曷修?

而又有士卒不練之失六,雖練無益之弊四。何謂不練?夫邊所藉惟兵,兵所藉惟將;今恩威號令不足服其心,分數形名不足齊其力,緩急難使,一也。有火器不能用,二也。棄土著不練,三也。諸鎮入衛之兵,嫌非統屬,漫無紀律,四也。班軍民兵數盈四萬,人各一心,五也。練兵之要在先練將。今注意武科,多方保舉似矣,但此選將之事,非練將之道,六也。何謂雖練無益?今一營之卒,為炮手者常十也。不知兵法五兵迭用,當長以衛短,短以救長,一也。三軍之士各專其藝,金鼓旗幟,何所不蓄?今皆置不用,二也。弓矢之力不強於寇,而欲藉以制勝,三也。教練之法,自有正門。美觀則不實用,實用則不美觀,而今悉無其實,四也。

臣又聞兵形象水,水因地而制流,兵因地而制勝。薊之地有三。平原廣陌,內地百里以南之形也。半險半易,近邊之形也。山谷仄隘,林薄蓊翳,邊外之形也。寇入平原,利車戰。在近邊,利馬戰。在邊外,利步戰。三者迭用,乃可制勝。今邊兵惟習馬耳,未嫻山戰、林戰、谷戰之道也,惟浙兵能之。願更予臣浙東殺手、炮手各三千,再募西北壯士,足馬軍五枝,步軍十枝,專聽臣訓練,軍中所需,隨宜取給,臣不勝至願。

又言:「臣官為創設,諸將視為綴疣,臣安從展布?」

章下兵部,言薊鎮既有總兵,又設總理,事權分,諸將多觀望,宜召還總兵郭琥,專任繼光。乃命繼光為總兵官,鎮守薊州、永平、山海諸處,而浙兵止弗調。錄破吳平功,進右都督。寇入青山口,拒卻之。

自嘉靖以來,邊牆雖修,墩臺未建。繼光巡行塞上,議建敵臺。略言:「薊鎮邊垣,延袤二千里,一瑕則百堅皆瑕。比來歲修歲圮,徒費無益。請跨牆為臺,睥睨四達。臺高五丈,虛中為三層,臺宿百人,鎧仗糗糧具備。令戍卒畫地受工,先建千二百座。然邊卒木彊,律以軍法將不堪,請募浙人為一軍,用倡勇敢。」督撫上其議,許之。浙兵三千至,陳郊外。天大雨,自朝至日昃,植立不動。邊軍大駭,自是始知軍令。五年秋,臺功成。精堅雄壯,二千里聲勢聯接。詔予世蔭,賚銀幣。

繼光乃議立車營。車一輛用四人推挽,戰則結方陣,而馬步軍處其中。又製拒馬器,體輕便利,遏寇騎衝突。寇至,火器先發,稍近則步軍持拒馬器排列而前,間以長鎗、筤筅。寇奔,則騎軍逐北。又置輜重營隨其後,而以南兵為選鋒,入衛兵主策應,本鎮兵專戍守。節制精明,器械犀利,薊門軍容遂為諸邊冠。

當是時,俺答已通貢,宣、大以西,烽火寂然。獨小王子後土蠻徙居插漢地,控弦十餘萬,常為薊門憂。而朵顏董狐狸及其兄子長昂交通土蠻,時叛時服。萬曆元年春,二寇謀入犯。馳喜峰口,索賞不得,則肆殺掠,獵傍塞,以誘官軍。繼光掩擊,幾獲狐狸。其夏,復犯桃林,不得志去。長昂亦犯界嶺。官軍斬獲多,邊吏諷之降,狐狸乃款關請貢。廷議給以歲賞。明年春,長昂復窺諸口不得入,則與狐狸共逼長禿令入寇。繼光逐得之以歸。長禿者,狐狸之弟,長昂叔父也。於是二寇率部長親族三百人,叩關請死罪,狐狸服素衣叩頭乞赦長禿。繼光及總督劉應節等議,遣副將史宸、羅端詣喜峰口受其降。皆羅拜,獻還所掠邊人,攢刀設誓。乃釋長禿,許通貢如故。終繼光在鎮,二寇不敢犯薊門。

尋以守邊勞,進左都督。已,增建敵臺,分所部十二區為三協,協置副將一人,分練士馬。炒蠻入犯,湯克寬戰死,繼光被劾,不罪。久之,炒蠻偕妻大嬖只襲掠邊卒,官軍追破之。土蠻犯遼東,繼光急赴,偕遼東軍拒退之。繼光已加太子太保,錄功加少保。

自順義受封,朝廷以八事課邊臣:曰積錢穀、修險隘、練兵馬、整器械、開屯田、理鹽法、收塞馬、散叛黨。三歲則遣大臣閱視,而殿最之。繼光用是頻廕賚。南北名將馬芳、俞大猷前卒,獨繼光與遼東李成梁在。然薊門守甚固,敵無由入,盡轉而之遼,故成梁擅戰功。

自嘉靖庚戌俺答犯京師,邊防獨重薊。增兵益餉,騷動天下。復置昌平鎮,設大將,與薊相脣齒。猶時躪內地,總督王忬、楊選並坐失律誅。十七年間,易大將十人,率以罪去。繼光在鎮十六年,邊備修飭,薊門宴然。繼之者,踵其成法,數十年得無事。亦賴當國大臣徐階、高拱、張居正先後倚任之。居正尤事與商確,欲為繼光難者,輒徙之去。諸督撫大臣如譚綸、劉應節、梁夢龍輩咸與善,動無掣肘,故繼光益發舒。

居正歿半歲,給事中張鼎思言繼光不宜於北,當國者遽改之廣東。繼光悒悒不得志,強一赴,踰年即謝病。給事中張希皋等復劾之,竟罷歸。居三年,御史傅光宅疏薦,反奪俸。繼光亦遂卒。

繼光更歷南北,並著聲。在南方戰功特盛,北則專主守。所著《紀效新書》、《練兵紀實》,談兵者遵用焉。

弟繼美,亦為貴州總兵官。

有硃先者,嘉興人。當繼光時,為薊鎮南兵營參將,遷副總兵。後數為廣東、福建總兵官。

初起家武舉,募海濱鹽徒為一軍。自胡宗憲為御史至總督,皆倚任。先大小數十戰,皆先登,殺倭甚眾。以功授都司。

宗憲被逮,先解官護行。宗憲釋還,先乃歸。御史按福建,巡撫王詢侵軍費,檄先證之。先曰:「先,王公部將也,不敢誣府主。」御史怒,坐先萬金,論死繫獄,閱八年始白。萬曆初,用薦起圜山把總。歷登閫帥,以年老謝事歸。復起,辭不赴。

先為將有膽智,砥節首公。其處宗憲、詢二事,時論以為有國士風。

劉顯,南昌人。生而膂力絕倫,稍通文義。家貧落魄,之叢祠欲自經,神護之不死。間行入蜀,為童子師。已,冒籍為武生。嘉靖三十四年,宜賓苗亂,巡撫張臬討之。顯從軍陷陣,手格殺五十餘人,擒首惡三人。諸軍繼進,賊盡平。顯由是知名。官副千戶,輸貲為指揮僉事。

南京振武營初設,用兵部尚書張鏊薦,召令訓練。擢署都指揮僉事,僉書浙江都司。遷參將,分守蘇、松。倭犯江北,逼泗州,鏊檄顯防浦口。顯測賊將遁,追擊至安東。方暑,披單衣,率四騎誘賊,伏精甲岡下。賊出,斬一人。所乘馬中矢,下拔其鏃,射殺追者。誘至岡下,大敗之去。賊出所俘女子蠱將士。顯悉送有司。明日伺賊出,潛毀其舟。賊敗走舟,舟已焚,死者無算。顯進秩三等。尋遷副總兵,協守江、浙。

三沙倭復劫江北,被圍於劉家莊。顯以銳卒數千至,巡撫李遂令盡護江北軍。顯率所部直入,諸營繼之,自辰迄酉,賊巢破,逐北至白駒場、茅花墩,斬首六百有奇,賊盡殄。而遂謂賊由三沙來,實盧鏜及顯罪。顯坐停俸。已,應天巡撫翁大立薦顯驍勇,請久任,帝可之。振武營兵變後,諸將務姑息,兵益驕。給事中魏元吉薦顯署都督僉事,節制其軍。顯挈蜀卒五百人往,一軍貼然。閩賊流入江西,大掠石城、臨州、東鄉、金谿,殺吏民萬計。詔顯赴剿,擊敗之陽湖,賊乃遁。

四十一年五月,廣東賊大起。詔顯充總兵官鎮守。會福建倭患棘,顯赴援。與參將戚繼光連破賊,賊略盡。而新倭大至,攻陷興化城。顯以兵少,逼城未敢戰,被劾,戴罪。賊以間攻據平海衛。他倭劫福清,謀與平海倭合。顯及俞大猷合於遮浪,盡殲之。平海倭欲遁,為把總許朝光所邀敗。乃盡焚其舟,退還舊屯。戚繼光亦至,顯與大猷共助擊之,遂復興化。錄功,進先所廕世職二秩。江北倭未平,廷議設總兵官於狼山,統制大江南北,改顯任之。顯行部通州,以敕書許節制知府以下,而同知王汝言不為禮,劾奏,鐫其秩。已,移鎮浙江。

顯有將略,居官不守法度。巡按御史劾之,革任候勘。用巡撫劉幾薦,命充為事官,鎮守如故。隆慶改元,以軍政拾遺被劾,貶秩視事。用巡撫谷中虛薦,還故官,移鎮貴州。廣西儂賊者念父子僭稱王,攻剽安順。巡撫阮文中檄顯剿,俘斬五百餘人。四川巡撫會省吾議征都掌蠻,令顯移鎮其地。復被劾罷,省吾奏留之。

都掌蠻者,居敘州戎縣,介高、珙、筠連、長寧、江安、納溪六縣間,古滬戎也。成化初為亂,程信討平之。正德中,普法惡復為亂,馬昊討平之。至是,其酋阿大、阿二、方三等據九絲山,剽遠近。其山修廣,而四隅峭仄。東北則雞冠嶺、都都寨、凌霄峰三岡,峻壁數千仞。有阿茍者,居凌霄峰,為賊耳目,威儀出入如王者。省吾議討之,屬顯軍事。起故將郭成、安大朝為佐,調諸土兵,合官軍凡十四萬人。萬厲改元三月,畢集敘州,誘執阿茍,攻拔凌霄,進逼都都寨。三酋遣其黨阿墨固守。官軍頓匝月,鑿灘以通漕,擊斬阿墨,拔其寨。阿大自守雞冠。顯令人誘以官,而分五哨盡壁九絲城下。乘無備,夜半腰糸亙上,斬關入。遲明,諸將畢至。阿二、方三走保牡豬寨。郭成破雞冠,獲阿大。諸軍攻牡豬,擒方三。阿二走,追獲於貴州大盤山。克寨六十餘,獲賊魁三十六,俘斬四千六百,拓地四百餘里,得諸葛銅鼓九十三,銅鐵鍋各一。阿大泣曰:「鼓聲宏者為上,可易千牛,次者七八百。得鼓二三,便可僭號稱王。鼓山顛,群蠻畢集,今已矣。」鍋狀如鼎,大可函牛,刻畫有文彩。相傳諸葛亮以鼓鎮蠻。鼓失,則蠻運終矣。錄功,進顯都督同知。已而剿餘孽,復俘斬千一百有奇。

都掌蠻既滅,顯引疾求去,而以有司阻撓為言。詔聽顯節制,顯益行其志。擊西川番沒舌、丟骨、人荒諸砦,斬其首惡,撫餘眾而還。建昌傀廈、洗馬諸番,咸獻首惡。西陲以寧。九年冬卒官。子綎,自有傳。

郭成,四川敘南衛人。由世職歷官蘇松參將,進副總兵。倭犯通州,為守將李錫所敗,轉掠崇明三沙。成擊沈其舟,斬首百三十餘級。隆慶元年冬,擢署都督僉事,為廣東總兵官。渡海追曾一本,大獲,進署都督同知。叛將周雲翔等殺參將耿宗元,亡入賊中。屯平山大安峒,將寇海豐。成偕南贛軍夾擊之,斬首千三百餘級,獲被掠通判潘槐而下六百餘人,生縶雲翔。潮州諸屬邑,賊巢以百數。郭明據林樟,胡一化據北山洋,陳一義據馬湖,剽劫二十載。成督諸軍擊殺明等,俘斬千三百有奇。四川都掌蠻為亂,詔成移鎮。尋被劾,罷歸。

萬曆改元,命劉顯大征,詔成充為事官,為之副。先登九絲山,生縶阿大。初,成父為蠻殺,乃以所斬首級及生擒諸蠻置父墓前,剖心致祭,鄉人壯之。尋僉書南京後府,出為貴州總兵官,鎮守銅仁。成有膽智。每苗出掠,潛遺壯士入其砦,斬馘而出。嘗挺身入林箐察賊。苗一日數驚,曰:「郭將軍至矣。」相戒莫敢犯。復被劾,罷歸。

起四川總兵官。永寧宣撫奢效忠卒,其妻奢世統無子,妾奢世續子崇周幼。前總兵劉顯因命世續署宣撫印。世統怒,攻奪其落紅寨。世續奔永寧。成遣義兒郭天心偕指揮禹嘉績按問。天心遂據世續永寧私第,罄取其資,而成亦入落紅,盡掠奢氏九世之積。效忠弟沙卜遂拒殺裨將三人,執天心等。撫、按交章劾成,下吏,遣戍雲南。會有松茂之役,薦從軍。成乃將七千人,直抵黃沙。屢破賊,與總兵官李應祥盡平河東西諸巢,以功授參將。復偕應詳大破膩乃諸賊,增世職二級。膩乃黨楊九乍復出為亂,成討平之。火落赤擾西寧,四川巡撫李尚思以地近松潘,檄成軍松林,游擊萬鏊軍漳臘。寇不敢逼,西陲獲安。楊應龍叛,成進討,無功,戴罪辦理。尋卒於官。

李錫,歙人。世新安衛千戶。倭警,數有功,為通州守備。屢擢揚州參將,江北副總兵。隆慶元年冬,以署都督僉事為福建總兵官。

海寇曾一本橫行閩、廣間,俞大猷將赴廣西,總督劉燾令會閩師夾擊。一本至閩,錫出海稟之,與大猷遇賊柘林澳,三戰皆捷。賊遁馬耳澳復戰。會廣東總兵官郭成率參將王詔等以師會,次菜蕪澳,分三哨進。一本駕大舟力戰,諸將連破之,毀其舟。詔生擒一本及其妻,斬首七百餘,死水火者萬計。時廣寇惟一本最強,錫、大猷、成共平之,而錫功最鉅。其後一本餘黨梁本豪復亂,為黃應甲所擒,然視錫時力較易。錫論功,加署都督同知。倭入寇,擊卻之。

六年春,以征蠻將軍代大猷鎮廣西平樂。府江者,桂林抵梧州驛道也。南北亙五百里,兩岸崇山深箐,賊巢盤互。自嘉靖間張岳破平後,至是復猖獗。嘗執知州邀重賄。道路梗塞,城門晝閉。大猷議討之,會罷官去。巡撫郭應聘與錫計,徵兵六萬,令參將錢鳳翔、王世科,都指揮王承恩、董龍各將一軍,以副使鄭茂、金柱,僉事夏道南監之,錫居中節制。破賊巢數十,斬馘五千有奇,僮酋楊錢甫等悉授首。錄功,進世職二等。

柳州懷遠,瑤、僮、伶、侗環居之,瑤尤獷悍。侵據縣治久,吏民率寓郡城。隆慶時大征古田,諸瑤懼而聽命。知縣馬希武之官,繕城塹,程役過嚴,諸瑤殺希武及經歷等五人,復反。巡撫應聘與總督殷正茂議征之。萬曆元年正月,錫進次長安鎮。會連雨雪,乃退師。益征浙東鳥銃手、湖廣永順鉤刀手及狼兵十萬人,令參將鳳翔、世科,都指揮楊照、戚繼美,故參將亦孔昭、魯國賢,六道並進,監以副使沈子木。錫自統水師,次羅江,獨當其衝。時賊屯板江大洲,累石樹柵,潛以舟來襲。錫伏舟敗之,水陸並進。會鳳翔等亦至,賊悉舟西遁。追擊,連破數巢。賊據楓木大山,前阻堤澗,鼓噪出。諸軍奮擊,而別以奇兵繞其後。賊大奔,保天鵝嶺。錫以水軍截潯江,督諸將攻斬渠魁二人。乘勝復破數巢,直抵清州界。賊奔大巢,亙數里,崖壁峭絕,為重柵拒官軍,鏢弩矢石雨下。婦人裸體揚箕,擲牛羊犬首為厭勝。諸軍大呼直上,四面舉火,賊盡殲。先後破巢一百四十,獻馘三千五百有奇,俘獲撫降者無算。

永福、永寧、柳城并以賊告,洛容僮又殺典史。錫令王瑞討永寧,楊照討柳城,參將門崇文討永福,亦孔昭討洛容,己帥舟師屯理定江,節制諸軍。甫二旬,四道並捷。斬首四千五百有奇,洛容賊首陶浪金等俱伏誅。錫以功進秩二等。巡按御史唐鍊言錫一年內破賊二百一十四巢,獲首功一萬二千餘級,宜久其任。帝可之。尋從凌雲翼大破羅旁賊,授世廕百戶。六年,卒官。

黃應甲者,不知何許人。隆慶中,以潯梧左參將從俞大猷討平韋銀豹,進秩二等。萬曆五年屢遷浙江總兵官。改鎮廣東。龍川鮑時秀者,妻杜氏,有妖術。乃據義都緱嶺,立二十四方大總,自號無敵峒王,既降復反。應甲討平之。醿戶蘇觀陛、周才雄招亡命數千人,縱掠雷、廉間,殺斷州千戶田治。應甲率五軍並進,生擒觀陛、才雄,斬首四百餘級,其黨縛酋長陳泉以降。

未幾,梁本豪亂。本豪,故曾一本黨,亦醿戶也。一本誅,竄海中,習水戰,遠通西洋。且結倭兵為助,殺千戶,掠通判以去。十年六月,總督陳瑞與應甲謀,分水軍二,南駐老萬山備倭,東駐虎門備醿,別以兩軍備外海,兩軍扼要害。水軍沈醿舟二十,生禽本豪。諸軍競進,大破之石茅洲。賊復奔潭洲沙灣,聚舟二百,及倭舟十,相掎角。諸將合追,先後俘斬千六百有奇,沈其舟二百餘,撫降者二千五百。帝為告郊廟,大行敘賚,應甲等進秩有差。他倭寇瓊崖,應甲斬首二百餘,奪其舟。再賜金。旋入僉左軍府。罷歸,卒。

尹鳳,字德輝,南京人。錫總兵福建時部將也,世府軍後衛指揮同知。鳳早孤。讀書,嫻騎射。嘉靖中舉武科,鄉、會試皆第一。擢署都指揮僉事,備倭福建。徙僉浙江都司,進福建參將。倭陷福清、南安,連宗出海。鳳邀擊,沈其七舟。追至外洋,連戰滸嶼、東洛、七礁,擒斬二百人。擊倭梅花洋,走之,追至橫山,擒斬二百六十。大小凡十數戰,內地賴以稍寧。改掌浙江都司,謝病歸。隆慶初,以故官蒞福建,從錫平曾一本。萬曆初,累官署都督僉事,提督京城巡捕。未幾,謝事歸。

張元勳,字世臣,浙江太平人。嗣世職為海門衛新河所百戶。沈毅有謀。值倭警,隸戚繼光麾下。有功,進千戶。從破橫嶼諸賊,屢進署都指揮僉事,充福建游擊將軍。隆慶初,破倭福安,改南路參將。從李錫破曾一本,進副總兵。

五年春,擢署都督僉事,代郭成為總兵官,鎮守廣東。惠州河源賊唐亞六、廣州從化賊萬尚欽、韶州英德賊張廷光劫掠郡縣,莫能制。明年,元勳進剿。斬馘六百有奇,亞六等授首,餘黨悉平。肇慶恩平十三村賊陳金鶯等,與鄰邑苔村三巢賊羅紹清、林翠蘭、譚權伯,藤峒、九逕十寨賊黃飛鶯、丘勝富、黃高暉、諸可行、黃朝富等,相煽為亂。故事:兩粵惟大征得敘功,雕剿不敘,故諸將不樂雕剿。總督殷正茂與元勳計,令雕剿得論功,諸軍爭奮。正茂又密遣副將梁守愚、游擊王瑞等屯恩平,若常戍者,掩不備,斬翠蘭等,生擒紹清、權伯以獻。其諸路雕剿者,效首功二千四百有奇,還被掠子女千三百餘人,生得金鶯,惟高暉等亡去。元勛逐北至藤峒,又生獲勝富、可行、朝富等八十人。部將鄧子龍等亦獲高暉、飛鶯。三巢、十寨、十三村諸賊盡平,餘悉就撫。

惠、潮地相接,山險木深。賊首藍一清、賴元爵與其黨馬祖冒、黃民太、曾廷鳳、黃鳴時、曾萬璋、李仲山、卓子望、葉景清、曾仕龍等各據險結砦,連地八百餘里,黨數萬人。正茂議大征。會金鶯等已滅,諸賊頗懼。廷鳳、萬璋並遣子入學,祖昌、景清亦佯乞降。正茂知其詐,徵兵四萬,令參將李誠立、沈思學、王詔,游擊王瑞等分將之,元勳居中節制,監司陳奎、唐九德、顧養謙、吳一介監其軍,數道並進。賊敗,乃憑險自守。官軍遍搜深箐邃谷間。而元勳偕九德,追亡至南嶺。一日夜馳至養謙所,擊破李坑,生得子望等。明年破烏禽嶂。仕龍阻高山,元勳佯飲酒高會,忽進兵擊擒之。先後獲大賊首六十一人,次賊首六百餘人,破大小巢七百餘所,擒斬一萬二千有奇。帝為宣捷,告郊廟,進元勳署都督同知,世廕百戶。元勳復討斬餘賊千三百有奇,撫定降者。巨寇皆靖。

潮州賊林道乾之黨諸良寶既撫復叛,襲殺官軍,掠六百人入海。再犯陽江,敗走。乃據潮故巢,居高山巔,不出戰。官軍營淤泥中。副將李誠立挑戰,墜馬傷足,死者二百人。賊出掠而敗,走巢固守。元勳積草土與賊壘平,用火攻之,斬首千一百餘級。時萬曆二年三月也。捷聞,進世廕一級。遺孽魏朝義等四巢亦降。尋與胡宗仁共平良寶黨林鳳。惠、潮遂無賊。其冬,倭陷銅鼓石、雙魚城。元勳大破之儒峒,俘斬八百餘級。進秩為真。五年,從總督凌雲翼大徵羅旁賊,斬首萬六千餘級。進都督,改廕錦衣。尋以疾致仕,卒於家。

元勳起小校。大小百十戰,威名震嶺南。與廣西李錫並稱良將。

贊曰:世宗朝,老成宿將以俞大猷為稱首,而數奇屢躓。以內外諸臣攘敓,而掩遏其功者眾也。戚繼光用兵,威名震寰宇。然當張居正、譚綸任國事則成,厥後張鼎思、張希皋等居言路則廢。任將之道,亦可知矣。劉顯平蠻引疾,而以有司阻撓為辭,有以夫!李錫、張元勛首功甚盛,而不蒙殊賞,武功所由不競也。


\end{pinyinscope}