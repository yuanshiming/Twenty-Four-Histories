\article{列傳第一百五十一}

\begin{pinyinscope}
宋一鶴沈壽崇蕭漢馮師孔黃絅等林日瑞郭天吉等蔡懋德趙建極等衛景瑗朱家仕等硃之馮朱敏泰等陳士奇{{陳纁等龍文光劉佳引劉之勃劉鎮籓

宋一鶴,宛平人。為諸生,見天下大亂,即究心兵事。崇禎三年舉於鄉。授教諭,以薦遷丘縣知縣,復以薦加東昌同知,仍知縣事。

巡按御史禹好善以一鶴知兵,薦之。授兵部員外郎,尋擢天津兵備僉事,改飭汝南兵備,駐信陽。

時熊文燦總理南畿、河南、山西、陜西、湖廣、四川軍務,主撫議。一鶴降其盜魁黃三耀,又降其死賊順天王之黨劉喜才。一鶴先後剿劇賊,斬首七百有奇。從副將龍在田破賊固始,一鶴毒殺其賊千人。左良玉降其賊李萬慶,一鶴撫而定之數萬。文燦屢上其功,薦之,進副使,調鄖陽。

文燦誅,楊嗣昌代,以一鶴能,薦之,擢右僉都御史,代方孔炤巡撫湖廣。時湖廣賊為諸將所逼,多竄入四川。一鶴以雲南軍移鎮當陽,中官劉元斌以京軍移鎮荊門,相掎角。左良玉等大破賊於瑪瑙山,一鶴敘功增俸。遣副將王允成、孫應元等大破賊汝才五大營於豐邑坪,斬首三千餘級。嗣昌署一鶴荊楚第一功。獻忠陷襄陽,與革裏眼、左金王等東萃黃州、汝寧間。一鶴移駐蘄州,焚舟,遏賊渡。賊移而北,一鶴又斷橫江,賊不敢渡。

嗣昌卒,丁啟睿代。啟睿破獻忠於麻城,會一鶴及鳳陽總督朱大典、安慶巡撫鄭二陽蹙賊左金王、老回回等於潛山、懷寧山中。一鶴又督參將王嘉謨等追破左金王、爭世王、治世王於燈草坪,斬首千八百級。十五年,遣部將陳治等合江北兵,破賊於桐城、舒城。

一鶴起鄉舉,不十年秉節鉞,廷臣不能無忮。御史衛周允上疏醜詆一鶴。一鶴屢建功,然亦往往蒙時詬。嗣昌父名鶴,一鶴投揭,自署其名曰「一鳥」,楚人傳笑之。一鶴亦連疏引疾,帝疑其偽,下所司嚴核。先以襄陽陷,奪職戴罪,至是許解官候代。

趨救汝寧,汝寧城已陷。十二月,襄陽、德安、荊州連告陷,一鶴趨承天護獻陵。陵軍柵木為城,賊積薪燒之,煙窨純德山。城穿,一鼓而登,犯獻陵,毀禋殿。守陵巡按御史李振聲、總兵官錢中選皆降,遂攻承天,歲除,明年正月二日,有以城下賊者。城陷,一鶴自經,故留守沈壽崇、鐘祥知縣蕭漢俱死,分巡副使張鳳翥走入山中。先是左良玉軍擾襄、樊,一鶴疏糾之。既,良玉自襄走承天,軍饑而掠,乞餉於一鶴,不許。良玉銜之。至是,一鶴謀留良玉兵,良玉走武昌,故及於難。

壽崇,宣城人,都督有容子。崇禎初武進士。忤巡按,被劾罷,未行而賊至,遂及於難。贈都督僉事,廕子錦衣百戶。

漢,字雲濤,南豐人。崇禎十年進士。秩滿將行,賊薄城,即辭家廟,授帨於妾媵曰:「男忠女烈,努力自盡。」遂出登陴,拒守五晝夜。元旦,突圍出,趨獻陵。賊騎環之,漢大呼:「鐘祥令在,誰敢驚陵寢者!」賊挾之去,不殺,說降,不聽。明日,城陷,送漢吉祥寺,謹視之,求死不得。越三日,從僧榻得剃刀,藏之,取敝紙書楊繼盛絕命詞,紙盡,投筆起,復拾土塊畫「鐘祥縣令蕭漢願死此寺」十字於壁,即對壁自剄,血正濺字上,死矣。賊嘉其義,用錦衣斂而瘞之。賊退,其門人改斂之以時服,曰:「嗚呼,大白其無黷乎!吾師肯服賊服乎!」悉易之。詔贈漢大理寺丞。

振聲,米脂人。與自成同縣而同姓,自成呼之為兄,後復殺之。將發獻陵,大聲起山谷,若雷震虎嗥,懼,乃止。

馮師孔,字景魯,原武人。萬曆四十四年進士。授刑部主事,歷員外郎、郎中。恤刑陜西,釋疑獄百八十人。天啟初,出為真定知府,遷井陘兵備副使,憂歸。

崇禎二年,起臨鞏兵備,改固原,再以憂歸。服闋,起懷來兵備副使,移密雲。忤鎮守中官鄧希詔。希詔摭他事劾之,下吏,削籍歸。

十五年詔舉邊才,用薦起故官,監通州軍。勤王兵集都下,剽劫公行,割婦人首報功。師孔大怒,以其卒抵死。明年,舉天下賢能方面官,鄭三俊薦師孔。六月,擢右僉都御史,代蔡官治巡撫陜西,調兵食,趣總督孫傳庭出關。

當是之時,賊十三家七十二營降,師殆盡,惟李自成、張獻忠存。自成尤強,據襄陽。以河洛、荊襄四戰之地,關中其故鄉,士馬甲天下,據之可以霸,決策西向。憚潼關天險,將自淅川龍車寨間道入陜西。傳庭聞之,令師孔率四川、甘肅兵駐商、雒為掎角,而師孔趣戰。無何,我師敗績於南陽,賊遂乘勝破潼關,大隊長驅,勢如破竹。師孔整眾守西安,人或咎師孔趣師致敗也。賊至,守將王根子開門入之。十月十一日,城陷,師孔投井死。同死者,按察使黃絅,長安知縣吳從義,秦府長史章尚絅,指揮崔爾達。

炯,字季侯,光州人。天啟二年進士。崇禎中,以淮海兵備副使憂歸。流賊陷州城,絅方廬墓山中,子彞如死於賊,其妹亦被難。服除,起臨鞏兵備副使,調番兵,大破李自成潼關原。尋以右參政分守洮岷,擢陜西按察使。自成勸之降,叱曰:「潼關之役,汝,我戮餘也,今日肯降汝耶?」妻王赴井,絅得間亦赴井,皆死。贈太常卿,謚忠烈。

尚絅,會稽人。聞城陷,投印井中,冠服趨王府端禮門雉經。贈按察司副使。

從義,山陰人。兒時夢一人拊其背曰:「歲寒松柏,其在斯乎。」崇禎十三年成進士,之官,兵荒,從義練丁壯三百人殺賊。賊破秦,從義曰:「嗟乎,豈非天哉!吾唯昔夢是踐矣。」遂投井死。贈按察司僉事。

爾達,不知何許人,亦投井死之。自是長安多義井。

賊遂執秦王存樞,處其宮署,置百官,稱王西安。坐王府中,日執士大夫拷掠,索金錢,分兵四出攻抄。有小吏邱從周者,長不及三尺,乘醉罵自成曰:「若一小民無賴,妄踞王府,將僭偽號,而所為暴虐若此,何能久!」賊怒,斫殺之。而布政使平湖陸之祺及里居吏部郎乾州宋企郊、提學僉事真寧鞏焴皆降賊,得寵用。

先是,戶部尚書倪元璐奏曰:「天下諸籓,孰與秦、晉?秦晉山險,用武國也。請諭二王,以剿賊保秦責秦王,以遏賊不入責晉王。王能殺賊,假王以大將軍權;不能殺賊,悉輸王所有餉軍,與其齎盜。賊平,益封王各一子如親王,亦足以明報矣。二王獨不鑒十一宗之禍乎?賢王忠而熟於計,必知所處矣。」書上,不報。至是,賊果破秦,悉為賊有焉。

林日瑞,字浴元,詔安人。萬曆四十四年進士。崇禎初,以江西右參政憂歸。服闋,起故官,分守湖廣。屬縣鉛山界閩,妖人聚山中謀不軌,圍鉛山。日瑞擊敗之,搗其巢。屢遷陜西左、右布政使。

十五年夏,遷右僉都御史,代呂大器巡撫甘肅。明年十一月,李自成屠慶陽。其別將賀錦犯蘭州,蘭州人開城迎賊。賊遂渡河。涼州、莊浪二衛降,即進逼甘州。日瑞聞賊急,結西羌,嚴兵以待,而自率副將郭天吉等扼諸河干。十二月,賊踏冰過,直抵甘州城下。日瑞入城,戰且守。大雪深丈許,樹盡介,角幹折,手足皸瘃,守者咸怨。賊乘夜坎雪而登,城陷,執日瑞。誘以官,不從,磔於市。

初,日瑞撫甘肅,廷議以其不任也,遣楊汝經代之。未至,日瑞遂及於難。

天吉及總兵官馬爌,撫標中軍哈維新、姚世儒,監紀同知藍臺,里居總兵官羅俊傑、趙宦,並死之。賊殺居民四萬七千餘人。三邊既陷,列城望風降,惟西寧衛固守不下。賊無後顧,乃長驅而東。福王時,贈日瑞兵部尚書,臺太僕寺少卿,皆賜祭葬。

蔡懋德,字維立,崑山人。少慕王守仁為人,著《管見》,宗良知之說。舉萬曆四十七年進士,授杭州推官。天啟間,行取入都。同鄉顧秉謙柄國,懋德不與通,秉謙怒,以故不得顯擢。授禮部儀制主事,進祠祭員外郎。尚書率諸司往謁魏忠賢祠,懋德託疾不赴。

崇禎初,出為江西提學副使,好以守仁《拔本塞源論》教諸生,大抵釋氏之緒論。遷浙江右參政,分守嘉興、湖州。劇盜屠阿丑有眾千餘,出沒太湖。懋德曰:「此可計擒也。」悉召瀕湖豪家,把其罪,簡壯士與同發,遂擒阿丑。皆曰:「懋德知兵。」內艱,服除,起井陘兵備。旱,懋德禱,即雨。他鄉爭迎以禱,又輒雨。調寧遠,以守松山及修臺堡功,數敘賚。會災異求言,懋德上《省過》、《治平》二疏,規切君相,一時咸笑為迂。

懋德好釋氏,律身如苦行頭陀。楊嗣昌謂其清修弱質,不宜處邊地,改濟南道。濟南新殘破,大吏多缺人,懋德攝兩司及三道印。遷山東按察使、河南右布政使。田荒穀貴,民苦催科,賊復以先服不輸租相煽誘。懋德亟檄州縣停征,上疏自劾,詔鐫七級視事。十四年冬,擢右僉都御史,巡撫山西。召對,賜酒饌、銀幣。明年春,抵任,討平大盜王冕。十月,統兵入衛京師,詔扼守龍泉、固關二關。李自成已陷河南,懋德禦之河上。

十六年冬,自成破潼關,據西安,盡有三秦。十二月,懋德師次平陽,遣副將陳尚智扼守河津。山西、京師右背、蒲州北抵保德,悉鄰賊,依黃河為險。然窮冬冰合,賊騎得長驅。懋德連章告急,請禁旅及保定、宣府、大同兵疾赴河干合拒。中朝益積憂山西,言防河者甚眾,然無兵可援。懋德以疲卒三千,當百萬狂寇。時太原洶洶,晉王手教趣懋德還省。十八日,懋德去平陽。二十日,賊抵河津,自船窩東渡,尚智走還平陽。二十二日,賊攻平陽,拔之。尚智奔入泥源山中。二十八日,懋德還太原。

明年正月,自成稱王於西安。賊既渡河,轉掠河東,列城皆陷。於是山西巡按御史汪宗友上言曰:「晉河二千里,平陽居其半。撫臣懋德不待春融冰泮,遽爾平陽返旆,賊即於明日報渡矣。隨行馬步千人,即時倍道西向,召集陳尚智叛卒,移檄各路防兵援剿,乃不發一兵。歲終至省,臣言宜提一旅,星馳而前,張疑聲討,尚冀桑榆之收,無如不聽何。賊日遣偽官,匝月,餘郡皆失,是誰之過歟!」有詔奪官候勘,以郭景昌代之。

二十三日,尚智叛降於賊。於是懋德誓師於太原,布政使趙建極,監司毛文炳、藺剛中、畢拱辰,太原知府孫康周,署陽曲縣事長史范志泰等官吏軍民咸在。懋德哭,眾皆哭。罷官命適至,或請出城候代。懋德不可,曰:「吾已辦一死矣,景昌即至,吾亦與俱死。」調陽和兵三千協守東門。剛中慮其內應,移之南關之外。遣部將張雄分守新南門,召中軍副總兵應時盛入參謀議。懋德等登城。

二月五日,賊至城下。遣部將牛勇、朱孔訓、王永魁出戰,死之。明日,自成具鹵簿,督眾攻城,陽和兵叛降賊。又明日,晝晦,懋德草遺表。須臾大風起,拔木揚砂。調張雄守大南門,雄已縋城出降,語其黨曰:「城東南角樓,火器火藥皆在,我下即焚樓。」夜中火起,風轉烈,守者皆散。賊登城,懋德北面再拜,出遺表付友人賈士璋間道達京師,語人曰:「吾學道有年,已勘了死生,今日吾致命時也。」即自剄,麾下持之。時盛請下城巷戰,顧懋德曰:「上馬。」懋德上馬,時盛持矛突殺賊數十人。至炭市口,賊騎充斥,時盛呼曰:「出西門。」懋德遽下馬曰:「我當死封疆,諸君自去。」眾復擁懋德上馬,至水西門。懋德叱曰:「諸君欲陷我不忠耶!」復下馬,據地坐。時盛已出城,殺妻子,還顧不見,復斫門入,語懋德曰:「請與公俱死。」遂偕至三立祠。懋德就縊未絕,時盛釋甲加其肩,乃絕。時盛取弓弦自經。建極危坐公堂,賊擁之見自成,不屈,將斬之。下階呼萬歲者再,曰:「臣失守封疆,死有餘罪。」自成以為呼己也,曳還。建極瞋目曰:「我呼大明皇帝,寧呼賊耶!」立射殺之。時自成執晉王,據王宮云。

文炳被殺,妻趙、妾李亦投井死,子兆夢甫數歲,賊掠去。士民以其忠臣子也,贖而歸之。欲降剛中,不從,殺之。首即墮,復躍起丈餘,賊皆辟易。賊適得新刀,拱辰睨之。問:「何睨!」曰:「欲得此斫頭耳。」遂取斬之。康周巷戰死,志泰不食死。自懋德而下,太原死事凡四十有六人,賊皆尸之城上。自成恨懋德之不降也,驗其尸,以刃斷頸而去。福王時,以懋德不守河為失策,乃謚忠襄,賜祭葬而不予贈廕,余賜恤有差。間考四十六人,行事多缺,姓名不傳,莫得而次云。

建極,河南永寧人。賊掠永寧時,建極五子皆死,後生三子又夭,至是趙氏一門竟絕。

文炳,字夢石,鄭州人。以吏科給事中出為山西兵備副使。為給事時,楊嗣昌督師,議調民兵討賊。文炳言:「民兵可守不可調,不若官軍乘馬便殺賊。」又言:「當大計,主計者喜奔競,抑廉靜,宜令官得互糾不公者。」帝皆納其言。

剛中,字坦生,陵縣人。為南京給事中,奏保護留都六事,又陳漕事救弊之要。山東饑,疏言:「民死而丁存,田荒而賦在,安得不為盜!宜清戶口並里甲。」皆切時病。遷山西副使。

拱辰,字星伯,掖縣人。知朝邑、鹽城二縣,數遷數貶。歷淮徐兵備僉事,督漕侍郎史可法謂其不任,移之冀寧。

建極、文炳、剛中、拱辰由進士。康周,字晉侯,安丘人,由鄉舉。時盛,遼陽諸生。為懋德所知,拔隸幕下,至都督僉事。志泰,虞城人。餘莫考。

太原既破,賊移檄遠近,所至郡縣望風結寨以拒官兵。而其仗義死難,陷胸斷脰而甘心者,則有若安邑知縣房之屏,宛平人,起家鄉舉。城陷,北向拜天子,入署拜其母,命妻子各自盡,遂投井,賊曳出斬之。忻州知州楊家龍,字惕若,曲陽人。為寧鄉知縣,凡七年,流亡復其業。遷忻,賊即至,曰:「此城必不守,我出,爾民可全也。」出城罵賊而死。州人祠祀之。代州參將閻夢夔,鹿邑人,汾州知州侯君昭,皆城亡與亡。汾陽知縣劉必達袖出罵賊文,賊誦而殺之。其義勇范奇芳,刺殺一偽都尉而自剄。寧武兵備副使王孕懋,字有懷,由太原知府遷。自成既陷太原,遣使說降,孕懋斬之,與總兵官周遇吉共守,城陷自殺,妻楊投井殉之。孕懋,霸州人,進士。遇吉自有傳。寧武失,賊破三關,犯大同。

衛景瑗,字仲玉,韓城人。天啟五年進士,授河南推官。

崇禎四年征授御史,劾首輔周延儒納賄行私數事,復劾吏部侍郎曾楚卿憸邪。帝不納。出按真定諸府。父喪,不俟命竟歸。服闋,起故官。疏救給事中傅朝佑、李汝璨以論溫體仁下吏,故帝不懌,左遷行人司正。歷尚寶、大理丞,進少卿。十五年春,擢右僉都御史,巡撫大同。歲饑疫,疏乞振濟。搜軍實,練火器,戢豪宗,聲績甚著。

十七年正月,李自成將犯山西,宣大總督王繼謨檄大同總兵官姜瓖扼之河上,瓖潛使納款而還。景瑗不知其變也,及山西陷,景瑗邀瓖歃血守。瓖出告人曰:「衛巡撫,秦人也,將應賊矣。」代王疑之,不見景瑗,永慶王射殺景瑗僕。會景瑗有足疾,不時出,兵事,瓖主之。瓖兄瑄,故昌平總兵也,勸瓖降賊。瓖慮其下不從,人犒之銀,言勵守城將士,代王信之。諸郡王分門守,瓖每門遣卒二百人助守。

至三月朔,賊抵城下。瓖即射殺永慶王,開門迎賊入。紿景瑗計事,景瑗乘馬出,始知其變也,自墜馬下。賊執之見自成,自成欲官之。景瑗據地坐,大呼皇帝而哭,賊義之,曰「忠臣也」,不殺。景瑗猝起,以頭觸階石,血淋漓。賊引出,顧見瓖,罵曰:「反賊,與我盟而叛,神其赦汝耶!」賊使景瑗母勸之降。景瑗曰:「母年八十餘矣,當自為計。兒,國大臣,不可以不死。」母出,景瑗謂人曰:「我不罵賊者,以全母也。」初六日自縊於僧寺。賊歎曰:「忠臣!」移其妻子空舍,戒毋犯。殺代王傳齊及其宗室殆盡。

分巡副使朱家仕,盡驅妻妾子女入井,而己從之,死者十有六人。督儲郎中徐有聲、山陰知縣李倬亦死之。諸生李若蔡自題其壁曰:「一門完節」,一家九人自經。家仕,河州人。

福王時,贈景瑗兵部尚書,謚忠毅。

賊既陷大同,以兵徇陽和,長驅向宣府。

硃之馮,字樂三,大興人。天啟五年進士。授戶部主事,榷稅河西務。課贏,貯公帑無所私。以外艱去。

崇禎二年起故官,進員外郎。坐罣誤,謫浙江布政司理問。稍遷行人司副,歷刑部郎中,浙江驛傳僉事,青州參議。盜劫沂水民,株連甚眾。之馮捕得真盜,大獄盡解。擒治樂安土豪李中行,權貴為請,不聽。進副使,齎表入都,寄家屬濟南。濟南破,妻馮匿姑及子於他所,而自沉於井。姑李聞之,為絕粒而死。柩還,之馮廬墓側三年。起河東副使。河東大猾朱全宇潛通秦賊,之馮至則執殺之,部內以寧。之馮自妻死不再娶,亦不置妾媵,一室蕭然。

十六年正月,擢右僉都御史,巡撫宣府。司餉主事張碩抱以剋餉激變,群縛碩抱。之馮出撫諭,貸商民貲給散,而密捕誅首惡七人,劾碩抱下吏。軍情帖然。

明年三月,李自成陷大同。之馮集將吏於城樓,設高皇帝位,歃血誓死守,懸賞格勵將士。而人心已散,監視中官杜勛且與總兵王承允爭先納款矣,見之馮叩頭,請以城下賊。之馮大罵曰:「勛,爾帝所倚信,特遣爾,以封疆屬爾,爾至即通賊,何面目見帝!」勛不答,笑而去。俄賊且至,勛蟒袍鳴騶,郊迎三十里之外,將士皆散。之馮登城太息,見大炮,語左右:「為我發之!」默無應者。自起爇火,則炮孔丁塞,或從後掣其肘。之馮撫膺嘆曰:「不意人心至此!」仰天大哭。賊至城下,承允開門入之,訛言賊不殺人,且免徭賦,則舉城嘩然皆喜,結彩焚香以迎。左右欲擁之馮出走,之馮叱之,乃南向叩頭,草遺表,勸帝收人心,厲士節,自縊而死。賊棄屍濠中,濠旁犬日食人屍,獨之馮無損也。

同日死者,督糧通判硃敏泰、諸生姚時中、副將寧龍及繫獄總兵官董用文、副將劉九卿及里居知縣申以孝,其他婦女死義者又十餘人。福王時,贈之馮兵部尚書,謚忠壯。

勛既降賊,從攻京師,射書於城中。城中初聞勛死宣府,帝為予贈廕立祠,至是以為鬼。守城監王承恩倚女牆而與語,縋勛入見帝,盛稱自成,「上可自為計」。復縋之出,笑語諸守監曰:「吾輩富貴自在也。」

陳士奇,字平人,漳浦人也。好學,有文名,不知兵。舉天啟五年進士,授中書舍人。崇禎四年考選,授禮部主事,擢廣西提學僉事。父憂歸。服闋,起重慶兵備,尋改貴州,復督學政。母憂闋,起贛州兵備參議,進副使,督四川學政。廷臣交章薦士奇知兵。

十五年秋,擢右僉都御史,代廖大奇巡撫四川。松潘兵變,眾數萬,士奇諭以禍福,咸就撫。搖、黃賊十三家,縱橫川東北十餘年,殺掠軍民無算;執少壯,文其面為軍,至數十萬。士奇檄副使陳其赤、葛征奇,參將趙榮貴等進討,屢告捷。而賊狡,迄不能制。士奇本文人,再督學政,好與諸生談兵,朝士以士奇知兵。及秉節鉞,反以文墨為事,軍政廢弛。石砫女將秦良玉嘗圖全蜀形勢,請益兵分守十三隘,扼賊奔突。置不問,蜀以是擾。

明年十二月,朝議以其不任,命龍文光代之。士奇方候代,而陽平將趙光遠擁兵二萬,護瑞王常浩自漢中來奔,士民避難者又數萬,至保寧,蜀人震駭。士奇馳責光遠曰:「若退守陽平關,為吾捍衛,不惜二萬金犒軍。如頓此,需厚餉,吾頭可斷,餉不可得也。」光遠退屯陽平,王以三千騎奔重慶。明年四月,文光受代,士奇將行,京師告變。士奇自以知兵也,曰:「必報國仇。」遂留駐重慶,遣水師參將曾英擊賊於忠州,焚其舟;遣趙榮貴禦賊於梁山。獻忠由葫蘆壩左步右騎,翼舟而上,二將敗奔,遂奪佛圖關,陷涪州。士奇征石砫援兵不至。或勸:「公已謝事,宜去。」士奇不可。賊抵城下,擊以滾炮,賊死無數。二十日夜,黑雲四布,賊穴地轟城。城陷,王、士奇及副使陳糸熏、知府王行儉、知縣王錫俱被執。士奇大罵,賊縛於教場,將殺之,忽雷雨晦冥,咫尺不見。獻忠仰而詬曰:「我殺人,何與天事!」用大炮向天叢擊。俄晴霽,遂肆僇。士奇罵不絕口而死,王亦遇害,賊集軍民三萬七千餘人,斫其臂。遂犯成都。

糸熏,本關南兵備副使,護瑞王入蜀,及於難。行儉,字質行,宜興人。崇禎十年進士,守重慶,善撫馭,為賊臠死。錫,新建人,崇禎十三年進士,除巴縣知縣。嘗從士奇殲土寇彭長庚之黨,又斬搖、黃賊魁馬超。至是,賊蒙巨板穴城,錫灌以熱油,多死。及被執,大罵,抉其齒,罵不已。捶膝使跪,益仡立。舁至教場,縛樹上射之,又臠而烙之。既死,復毀其骨。

指揮顧景聞城陷,入瑞王府,以己所乘馬乘王,鞭而走,遇賊呼曰:「賊寧殺我,無犯帝子。」賊刺殺王,景遂死之。

龍文光,馬平人,天啟二年進士。崇禎十七年,以川北參政擢右僉都御史,代陳士奇巡撫四川。聞命,與總兵官劉佳引率兵三千,由順慶馳赴之。部署未定,數日而城陷。賊盡驅文武將吏及軍民男婦於東門之外,將戮之,忽有龍尾下垂,賊以為瑞,遂停刑。文光、佳引卒不屈,賊殺文光於濯錦橋,佳引自投於浣花溪。

劉之勃,字安侯,鳳翔人。崇禎七年進士。授行人,擢御史。上節財六議,言:「先朝馬萬計,草場止五六所,今馬漸少,場反增二倍,可節省者一。水衡工役費,歲幾百萬,近奉明旨,朝廷不事興作,而節慎庫額數襲為常,可節省者二。諸鎮兵馬時敗潰而餉額不減,虛伍必多,可節省者三。光祿宴享賜賚,大抵從簡,而監局廚役多冗濫,可節省者四。三吳織造,澤、潞機杼,以及香蠟、藥材、陶器,無歲不貢,積之內為廢物,輸之下皆金錢,可節省者五。軍前監紀、監軍、贊畫之官,不可勝紀,平時則以一人而縻千百人之餉,臨敵又以千百人而衛一人之身,耗食兼耗兵,可節省者六。」又疏陳東廠三弊,言:「東廠司緝訪,而內五城,外巡按,以及刑部、大理皆不能舉其職,此不便於官守。奸民千里首告,假捏姓名,一紙株連,萬金立罄,此不便於民生。子弟訐父兄,奴僕訐家主,部民訐官長,東廠皆樂聞,此不便於國體。」帝皆納其言。

十五年出按四川。十六年秋,類報災異,請緩賦省刑,亦弭災一術,時不能用。明年正月,張獻忠大破川中郡邑。四月,聞都城失守,人心益恟懼。舉人楊鏘、劉道貞等謀擁蜀王至澍監國,之勃不可,躍入池中,議乃寢。八月,賊逼成都,之勃與巡撫龍文光、建昌兵備副使劉士斗等分陴拒守,總兵官劉鎮籓出戰而敗。賊穴城,實以火藥;又刳大木長數丈者合之,纏以帛,貯藥,向城樓。之勃厲眾奮擊,賊卻二三里,皆喜,以為將去也。初九日黎明,火發,北樓陷,木石飛蔽天,守陴者皆散,賊遂入城。蜀王率妃妾自沉於菊井。鎮籓突圍出,赴浣花溪死之。之勃等被執,賊以之勃同鄉,欲用之,之勃勸以不殺百姓,輔立蜀世子。不從,遂大罵,賊攢箭射殺之。時福王立於南京,擢之勃右僉都御史,巡撫四川,已不及聞矣。

贊曰:潼關既破,李自成乘勝遂有三秦,渡河而東,勢若燎原。宣、大繼覆,明亡遂決。一時封疆諸臣後先爭死,可不謂烈哉!然平陽之旆甫東,船窩之警旋告。死非難,所以處死為難,君子不能無憾於懋德焉已。若夫一鶴之死顯陵,士奇之死夔州,劉之勃、龍文光之死成都,不亦得死所者歟!


\end{pinyinscope}