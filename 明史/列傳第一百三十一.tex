\article{列傳第一百三十一}

\begin{pinyinscope}
趙南星鄒元標孫慎行盛以弘高攀龍馮從吾

趙南星,字夢白,高邑人。萬曆二年進士。除汝寧推官。治行廉平,稍遷戶部主事。張居正寢疾,朝士群禱,南星與顧憲成、姜士昌戒弗往。居正歿,調吏部考功。引疾歸。

起歷文選員外郎。疏陳天下四大害,言:「楊巍乞休,左都御史吳時來謀代之,忌戶部尚書宋糸熏聲望,連疏排擠。副都御史詹仰庇力謀吏、兵二部侍郎。大臣如此,何以責小臣,是謂幹進之害。禮部尚書沈鯉、侍郎張位、諭德吳中行、南京太僕卿沈思孝相繼自免,獨南京禮部侍郎趙用賢在,詞臣黃洪憲輩每陰讒之,言官唐堯欽、孫愈賢、蔡系周復顯為詆誣。眾正不容,宵人得志,是謂傾危之害。州縣長吏選授太輕,部寺之官計日而取郡守,不問才行。而撫按論人贓私有據,不曰未甚,則曰任淺,概止降調,其意以為惜才,不知此乃惜不才也。吏治日汙,民生日瘁,是謂州縣之害。鄉官之權大於守令,橫行無忌,莫敢誰何。如渭南知縣張棟,治行無雙,裁抑鄉官,被讒不獲行取,是謂鄉官之害。四害不除,天下不可得治。」疏出,朝論韙之。而中所抨擊悉時相所庇,於是給事中李春開起而駁之。其疏先下,南星幾獲譴。給事中王繼光、史孟麟、萬自約,部曹姜士昌、吳正志並助南星詆春開,且發時來、仰庇、洪憲讒諂狀。春開氣沮,然南星卒以病歸。再起,歷考功郎中。

二十一年大計京官,與尚書孫鑨秉公澄汰。首黜所親都給事中王三餘及金龍甥文選員外郎呂廕昌,他附麗政府及大學士趙志皋弟皆不免,政府大不堪。給事中劉道隆因劾吏部議留拾遺庶僚非法。得旨,南星等專權植黨,貶三官。俄因李世達等疏救,斥南星為民。後論救者悉被譴,鑨亦去位,一時善類幾空。事具鑨傳。

南星裏居,名益高,與鄒元標、顧憲成,海內擬之「三君」。中外論薦者百十疏,卒不起。

光宗立,起太常少卿。俄改右通政,進太常卿,至則擢工部右侍郎。居數月,拜左都御史,慨然以整齊天下為任。天啟三年大計京官,以故給事中亓詩教、趙興邦、官應震、吳亮嗣先朝結黨亂政,議黜之,吏科都給事中魏應嘉力持不可。南星著《四凶論》,卒與考功郎程正己置四人不謹。他所澄汰,一如為考功時。浙江巡按張素養薦部內人材,及姚宗文、邵輔忠、劉廷元,南星劾其謬,素養坐奪俸。先是,巡方者有提薦之例,南星已奏止之;而陜西高弘圖、山西徐揚先、宣大李思啟、河東劉大受,復踵行如故,南星並劾奏之,巡方者始知畏法。

尋代張問達為吏部尚書。當是時,人務奔競,苞苴恣行,言路橫尤甚。每文選郎出,輒邀之半道,為人求官,不得則加以惡聲,或逐之去。選郎即公正無如何,尚書亦太息而已。南星素疾其弊,銳意澄清,獨行己志,政府及中貴亦不得有所干請,諸人憚其剛嚴不敢犯。有給事為貲郎求鹽運司,即注貲郎王府,而出給事於外。知縣石三畏素貪,夤緣將行取,南星亦置之王府。時進士無為王官者,南星不恤也。

魏忠賢雅重之,嘗於帝前稱其任事。一日,遣娣子傅應星介一中書贄見,南星麾之去。嘗並坐弘政門,選通政司參議,正色語忠賢曰:「主上沖齡,我輩內外臣子宜各努力為善。」忠賢默然,怒形於色。大學士魏廣微,南星友允貞子也,素以通家子畜之。廣微入內閣,嘗三至南星門,拒勿見。又嘗嘆曰:「見泉無子。」見泉,允貞別號也。廣微恨刺骨,與忠賢比而齕南星。

東林勢盛,眾正盈朝。南星益搜舉遺佚,布之庶位。高攀龍、楊漣、左光斗秉憲;李騰芳、陳于廷佐銓;魏大中、袁化中長科道;鄭三俊、李邦華、孫居相、饒伸、王之寀輩悉置卿貳。而四司之屬,鄒維璉、夏嘉遇、張光前、程國祥、劉廷諫亦皆民譽。中外忻忻望治,而小人側目,滋欲去南星。給事中傅櫆以維璉改吏部己不與聞,首假汪文言發難,劾南星紊舊制,植私人。維璉引去,南星奏留之,小人愈恨。會漣劾忠賢疏上,宮府益水火。南星遂杜門乞休,不許。

攀龍之劾崔呈秀也,南星議戍之。呈秀窘,夜走忠賢邸,叩頭乞哀,言:「不去南星及攀龍、漣等,我兩人未知死所。」忠賢大以為然,遂與定謀。會山西缺巡撫,河南布政使郭尚友求之。南星以太常卿謝應祥有清望,首列以請。既得旨,而御史陳九疇受廣微指,言應祥嘗知嘉善,大中出其門,大中以師故,謀於文選郎嘉遇而用之,徇私當斥。大中、嘉遇疏辯,語侵九疇,九疇再疏力詆,並下部議。南星、攀龍極言應祥以人望推舉,大中、嘉遇無私,九疇妄言不可聽。忠賢大怒,矯旨黜大中、嘉遇,并黜九疇,而責南星等朋謀結黨。南星遽引罪求去,忠賢復矯旨切責,放歸。明日,攀龍亦引去。給事中沈惟炳論救,亦出之外。俄以會推忤忠賢意,並斥于廷、漣、光斗、化中,引南星所擯徐兆魁、喬應甲、王紹徽等置要地。小人競進,天下大柄盡歸忠賢矣。

忠賢及其黨惡南星甚,每矯敕諭,必目為元凶。於是御史張訥劾南星十大罪,并劾維璉、國祥、嘉遇及王允成。得旨,並削籍。令再奏南星私黨,訥復列上邦華及孫鼎相等十四人,並貶黜。自是為南星擯棄者,無不拔擢,其素所推獎者,率遭奇禍。諸乾進速化之徒,一擊南星,輒遂所欲。而石三畏亦起為御史,疏攻南星及李三才、顧憲成、孫丕揚、王圖等十五人。死者皆削奪,縉紳禍益烈。尋以汪文言獄詞連及南星,下撫按提問。適郭尚友巡撫保定,而巡按馬逢皋亦憾南星,乃相與庭辱之。笞其子清衡及外孫王鐘龐,繫之獄,坐南星贓萬五千。南星家素貧,親故捐助,始獲竣。卒戍南星代州,清衡莊浪,鐘龐永昌。嫡母馮氏、生母李氏,並哀慟而卒。子生七齡,驚怖死。南星抵戍所,處之怡然。

莊烈帝登極,有詔赦還。巡撫牟志夔,忠賢黨也,故遲遣之,竟卒於戍所。崇禎初,贈太子太保,謚忠毅。櫆、呈秀、廣微、九疇、兆魁、應甲、紹徽、訥、三畏、尚友、志夔,俱名麗逆案,為世大僇焉。

鄒元標,字爾瞻,吉水人。九歲通《五經》。泰和胡直,嘉靖中進士,官至福建按察使,師歐陽德、羅洪先,得王守仁之傳。元標弱冠從直遊,即有志為學。舉萬曆五年進士。觀政刑部。

張居正奪情,元標抗疏切諫。且曰:「陛下以居正有利社稷耶?居正才雖可為,學術則偏;志雖欲為,自用太甚。其設施乖張者,如州縣入學,限以十五六人,有司希指,更損其數。是進賢未廣也。諸道決囚,亦有定額,所司懼罰,數必取盈。是斷刑太濫也。大臣持祿茍容,小臣畏罪緘默,有今日陳言而明日獲譴者。是言路未通也。黃河泛濫為災,民有駕蒿為巢、啜水為餐者,而有司不以聞。是民隱未周也。其他用刻深之吏,沮豪傑之材,又不可枚數矣。伏讀敕諭『朕學尚未成,志尚未定,先生既去,前功盡隳』,陛下言及此,宗社無疆之福也。雖然,弼成聖學,輔翼聖志者,未可謂在廷無人也。且幸而居正丁艱,猶可挽留;脫不幸遂捐館舍,陛下之學將終不成,志將終不定耶?臣觀居正疏言『世有非常之人,然後辦非常之事』,若以奔喪為常事而不屑為者,不知人惟盡此五常之道,然後謂之人。今有人於此,親生而不顧,親死而不奔,猶自號於世曰我非常人也,世不以為喪心,則以為禽彘,可謂之非常人哉?」疏就,懷之入朝,適廷杖吳中行等。元標俟杖畢,取疏授中官,紿曰:「此乞假疏也。」及入,居正大怒,亦廷杖八十,謫戍都勻衛。衛在萬山中,夷獠與居,元標處之怡然。益究心理學,學以大進。巡按御中承居正指,將害元標。行次鎮遠,一夕,御史暴死。

元標謫居六年,居正歿,召拜吏科給事中。首陳培聖德、親臣工、肅憲紀、崇儒行、飭撫臣五事。尋劾罷禮部尚書徐學謨、南京戶部尚書張士佩。

徐學謨者,嘉定縣人。嘉靖中,為荊州知府。景恭王之籓德安,欲奪荊州城北沙市地。學謨力抗不予,為王所劾,下撫按逮問,改官。荊州人德之,稱沙市為「徐市」。居正素與厚。萬曆中,累遷右副都御史,撫治鄖陽。居正歸葬父,學謨事之謹,召為刑部侍郎。越二年,擢禮部尚書。自弘治後,禮部長非翰林不授,惟席書以言「大禮」故,由他曹遷;萬士和不由翰林,然先歷其部侍郎。學謨徑拜尚書,廷臣以居正故,莫敢言。居正卒,學謨急締姻於大學士申時行以自固。及奉命擇壽宮,通政參議梁子琦劾其始結居正,繼附時行,詔為奪子琦俸。元標復劾之,遂令致仕歸。

慈寧宮災,元標復上時政六事,中言:「臣曩進無欲之訓,陛下試自省,果無欲耶?寡欲耶?語云:『欲人勿聞,莫若勿為。』陛下誠宜翻然自省,加意培養。」當是時,帝方壯齡,留意聲色游宴,謂元標刺己,怒甚,降旨譙責。首輔時行以元標己門生,而劾罷其姻學謨,亦心憾,遂謫南京刑部照磨。就遷兵部主事。召改吏部,進員外郎,以病免。起補驗封。陳吏治十事,民瘼八事,疏幾萬言。文選缺員外郎,尚書宋糸熏請用元標,久不獲命,糸熏連疏趣之。給事中楊文煥、御史何選亦以為言。帝怒,詰責糸熏,謫文煥、選於外,而調元標南京。刑部尚書石星論救,亦被譙讓。元標居南京三年,移疾歸。久之,起本部郎中,不赴。旋遭母憂,里居講學,從游者日眾,名高天下。中外疏薦遺佚,凡數十百上,莫不以元標為首。卒不用。家食垂三十年。

光宗立,召拜大理卿。未至,進刑部右侍郎。天啟元年四月還朝,首進和衷之說,言:「今日國事,皆二十年諸臣醞釀所成。往者不以進賢讓能為事,日錮賢逐能,而言事者又不降心平氣,專務分門立戶。臣謂今日急務,惟朝臣和衷而已。朝臣和,天地之和自應。向之論人論事者,各懷偏見,偏生迷,迷生執,執而為我,不復知有人,禍且移於國。今與諸臣約,論一人當惟公惟平,毋輕搖筆端,論一事當懲前慮後,毋輕試耳食,以天下萬世之心,衡天下萬世之人與事,則議論公,而國家自享安靜和平之福。」因薦塗宗浚、李邦華等十八人。帝優詔褒納。居二日,復陳拔茅闡幽、理財振武數事,及保泰四規。且請召用葉茂才、趙南星、高攀龍、劉宗周、丁元薦,而恤錄羅大紘、雒于仁等十五人。帝亦褒納。

初,元標立朝,以方嚴見憚,晚節務為和易。或議其遜初仕時,元標笑曰:「大臣與言官異。風裁踔絕,言官事也。大臣非大利害,即當護持國體,可如少年悻動耶?」時朋黨方盛,元標心惡之,思矯其弊,故其所薦引不專一途。嘗欲舉用李三才,因言路不與,元標即中止。王德完譏其首鼠,元標亦不較。南京御史王允成等以兩人不和,請帝諭解。元標言:「臣與德完初無纖芥,此必有人交構其間。臣嘗語朝士曰:『方今上在沖歲,敵在門庭,只有同心共濟。倘復黨同伐異,在國則不忠,在家則不孝。世自有無偏無黨之路,奈何從室內起戈矛耶?』」帝嗣位已久,而先朝廢死諸臣猶未贈恤,元標再陳闡幽之典,言益懇切。

其年十二月改吏部左侍郎。未到官,拜左都御史。明年,典外察,去留惟公。御史潘汝楨、過庭訓雅有物議,及庭訓秩滿,汝楨注考溢美。元標疏論之,兩人並引疾去。已,言丁已京察不公,專禁錮異己,請收錄章家禎、丁元薦、史記事、沈正宗等二十二人。由是諸臣多獲昭雪。又言:「明詔收召遺佚,而諸老臣所處猶是三十年前應得之官,宜添注三品崇秩,昭陛下褒尊耆舊至意。」帝納其言。於是兩京太常、太僕、光祿三卿各增二員。

孫慎行之論「紅丸」也,元標亦上疏曰:「乾坤所以不毀者,惟此綱常;綱常所以植立者,恃此信史。臣去年舟過南中,南中士大夫爭言先帝猝然而崩,大事未明,難以傳信。臣初不謂然。及既入都,為人言先帝盛德,宜速登信史。諸臣曰:『言及先帝彌留大事,令人閣筆,誰敢領此?』臣始有疑於前日之言。元輔方從哲不伸討賊之義,反行賞奸之典,即謂無其心,何以自解於世?且從哲秉政七年,未聞建樹何事,但聞馬上一日三趣戰,喪我十萬師徒。訊問誰秉國成,而使先帝震驚,奸人闖宮,豺狼當路,憸邪亂政?從哲何詞以對?從來懲戒亂賊,全在信史。失今不成,安所底止。」時刑部尚書黃克纘希內廷意,群小和之,而從哲世居京師,黨附者眾,崔文昇黨復彌縫於內,格慎行與眾議,皆不得伸。未幾,慎行及王紀偕逐,元標疏救,不聽。

元標自還朝以來,不為危言激論,與物無猜。然小人以其東林也,猶忌之。給事中朱童蒙、郭允厚、郭興治慮明年京察不利己,潛謀驅逐。會元標與馮從吾建首善書院,集同志講學,童蒙首請禁之。元標疏辨求去,帝已慰留,允厚復疏劾,語尤妄誕。而魏忠賢方竊柄,傳旨謂宋室之亡由於講學,將加嚴譴。葉向高力辨,且乞同去,乃得溫旨。興治及允厚復交章力攻,興治至比之山東妖賊。元標連疏請益力,詔加太子少保,乘傳歸。陛辭,上《老臣去國情深疏》,歷陳軍國大計,而以寡欲進規,人為傳誦。四年,卒於家。明年,御史張訥請毀天下講壇,力詆元標,忠賢遂矯旨削奪。崇禎初,贈太子太保、吏部尚書,謚忠介。

童蒙等既劾元標,遂得罪清議,尋以年例外遷。及忠賢得志,三人並召還。歲餘,允厚至戶部尚書、太子太保。童蒙至右副都御史,巡撫延綏,母死不持服,為忠賢建生祠。興治亦加至太僕卿。忠賢敗,三人並麗逆案云。

孫慎行,字聞斯,武進人。幼習聞外祖唐順之緒論,即嗜學。萬曆二十三年舉進士第三人,授編修,累官左庶子。數請假里居,鍵戶息交,覃精理學。當事請見,率不納。有以政事詢者,不答。

四十一年五月,由少詹事擢禮部右侍郎,署部事。當是時,郊廟大享諸禮,帝二十餘年不躬親,東宮輟講至八年,皇長孫九齡未就外傅,瑞王二十三未婚,楚宗人久錮未釋,代王廢長立幼,久不更正,臣僚章奏一切留中,福府莊田取盈四萬頃,慎行並切諫。已,念東宮開講,皇孫出閣,係宗社安危,疏至七八上。代王廢長子鼎渭,立愛子鼎莎,李廷機為侍郎時主之,其後,群臣爭者百餘疏,帝皆不省。慎行屢疏爭,乃獲更置。楚宗人擊殺巡撫趙可懷,為首六人論死,復錮英憔等二十三人於高牆,禁蘊鈁等二十三人於遠地。慎行力白其非叛,諸人由此獲釋。皇太子儲位雖定,福王尚留京師,須莊田四萬頃乃行,宵小多窺伺。廷臣請之國者愈眾,帝愈遲之。慎行疏十餘上,不見省。最後,貴妃復請帝留王慶太后七旬壽節,群議益籍籍。慎行乃合文武諸臣伏闕力請,大學士葉向高亦爭之強。帝不得已,許明年季春之國,群情始安。韓敬科場之議,慎行擬黜敬。而家居時素講學東林,敬黨尤忌之。會吏部缺侍郎,廷議改右侍郎李鋕於左,而以慎行為右,命俱未下。御史過廷訓因言鋕未履任,何復推慎行,給事中亓詩教和之。慎行遂四疏乞歸,出城候命,帝乃許之。已而京察,御史韓浚等以趣福王之國,謂慎行邀功,列之拾遺疏中。帝察其無罪,獲免。

熹宗立,召拜禮部尚書。初,光宗大漸,鴻臚寺丞李可灼以紅鉛丸藥進。俄帝崩,廷臣交章劾之。大學士方從哲擬旨令引疾歸,賚以金幣。天啟元年四月,慎行還朝,上疏曰:

先帝驟崩,雖云夙疾,實緣醫人用藥不審。閱邸報,知李可灼紅丸乃首輔方從哲所進。夫可灼官非太醫,紅丸不知何藥,乃敢突然以進。昔許悼公飲世子藥而卒,世子即自殺,《春秋》猶書之為弒。然則從哲宜何居?速引劍自裁以謝先帝,義之上也;合門席稿以待司寇,義之次也;乃悍然不顧,至舉朝共攻可灼,僅令回籍調理,豈不以己實薦之,恐與同罪與?臣以為從哲縱無弒之心,卻有弒之事;欲辭弒之名,難免弒之實。實錄中即欲為君父諱,不敢不直書方從哲連進藥二丸,須臾帝崩,恐百口無能為天下後世解也。

然從哲之罪實不止此。先是則有皇貴妃欲為皇后事,古未有天子既崩而立后者。倘非禮官執奏,言路力持,幾何不遺禍宗社哉!繼此則有謚皇祖為恭皇帝事。歷考晉、隋、周、宋,其末世亡國之君率謚曰「恭」,而以加之我皇祖,豈真不學無術?實乃咒詛君國,等於亡王,其設心謂何?後此則有選侍垂簾聽政事。劉遜、李進忠麼麼小豎,何遂膽大揚言。說者謂二豎早以金寶輸從哲家,若非九卿、臺諫力請移宮,選侍一日得志,陛下幾無駐足所。聞爾時從哲濡遲不進,科臣趣之,則云遲數日無害。任婦寺之縱橫,忍君父之杌隉,為大臣者宜爾乎?臣在禮言禮,其罪惡逆天,萬無可生之路。若其他督戰誤國,罔上行私,縱情蔑法,干犯天下之名義,釀成國家之禍患者,臣不能悉數也。陛下宜急討此賊,雪不共之仇!毋詢近習,近習皆從哲所攀援也;毋拘忌諱,忌諱即從哲所布置也。并急誅李可灼,以洩神人之憤。

時朝野方惡從哲,慎行論雖過刻,然爭韙其言。顧近習多為從哲地,帝乃報曰:「舊輔素忠慎,可灼進藥本先帝意。卿言雖忠愛,事屬傳聞。并進封移宮事,當日九卿、臺諫官親見者,當據實會奏,用釋群疑。」於是從哲疏辨。刑部尚書黃克纘右從哲,亦曲為辨。慎行復疏折之,曰:「由前則過信可灼,有輕進藥之罪,由後則曲庇可灼,有不討賊之罪,兩者均無辭乎弒也。從哲謂移宮有揭,但諸臣之請在初二,從哲之請在初五。爾時章疏入乾清不入慈慶者已三日,國政幾於中斷,非他輔臣訪知,與群臣力請,其害可勝言哉!伏讀聖諭『輔臣義在體國,為朕分憂。今似此景象,何不代朕傳諭一言,屏息紛擾,君臣大義安在?』又云『朕凌虐不堪,晝夜涕泣六七日。』夫從哲為顧命元臣,使少肯義形於色,何至令至尊憂危如此!惟阿婦寺之意多,戴聖明之意少,故敢於凌皇祖,悖皇考,而欺陛下也。」末復力言克纘之謬。章並下廷議。既而議上,惟可灼下吏戍邊,從哲置不問。

山東巡撫奏,五月中,日中月星並見。慎行以為大異,疏請修省,語極危切。秦王誼漶由旁枝進封,其四子法不當封郡王,厚賄近倖,遂得溫旨。慎行堅不奉詔,三疏力爭,不得。七月謝病去。

其冬,廷推閣臣,以慎行為首,吏部侍郎盛以弘次之。魏忠賢抑不用,用顧秉謙、朱國禎、朱延禧、魏廣微,朝論大駭。葉向高連疏請用兩人,竟不得命。已,忠賢大熾,議修《三朝要典》,「紅丸」之案以慎行為罪魁。其黨張訥遂上疏力詆,有詔削奪。未幾,劉志選復兩疏追劾,詔撫按提問,遣戍寧夏。未行,莊烈帝嗣位,以赦免。

崇禎元年,命以故官協理詹事府,力辭不就。慎行操行峻潔,為一時搢紳冠。朝士數推轂入閣,吏部尚書王永光力排之,迄不獲用。八年廷推閣臣,屢不稱旨,最後以慎行及劉宗周、林釬名上,帝即召之。慎行已得疾,甫入都,卒。贈太子太保,謚文介。

盛以弘,字子寬,潼關衛人。父訥,字敏叔。訥父德,世職指揮也,討洛南盜戰死。訥號泣請於當事,水漿不入口者數日,為發兵討斬之。久之,舉隆慶五年進士。由庶吉士累官吏部右侍郎。與尚書陳有年、左侍郎趙參魯共釐銓政。母憂歸,以篤孝聞。卒,贈禮部尚書。天啟初,謚文定。

以弘,萬曆二十六年進士。由庶吉士累官禮部尚書。天啟三年謝病歸。魏忠賢亂政,落其職。崇禎初,起故官,協理詹事府,卒官。明世,衛所世職用儒業顯者,訥父子而已。

高攀龍,字存之,無錫人。少讀書,輒有志程朱之學。舉萬曆十七年進士,授行人。四川僉事張世則進所著《大學初義》,詆程、朱章句,請頒天下。攀龍抗疏力駮其謬,其書遂不行。

侍郎趙用賢、都御史李世達被訐去位,朝論多咎大學士王錫爵。攀龍上疏曰:

近見朝宁之上,善類擯斥一空。大臣則孫鑨、李世達、趙用賢去矣,小臣則趙南星、陳泰來、顧允成、薛敷教、張納陛、于孔兼、賈巖斥矣。邇者李禎、曾乾亨復不安其位而乞去矣,選郎孟化鯉又以推用言官張棟,空署而逐矣。

夫天地生才甚難,國家需才甚亟,廢斥如此,後將焉繼。致使正人扼腕,曲士彈冠,世道人心何可勝慨!且今陛下朝講久輟,廷臣不獲望見顏色。天言傳布,雖曰聖裁,隱伏之中,莫測所以。故中外群言,不曰:「輔臣欲除不附己」,則曰「近侍不利用正人」。陛下深居九重,亦曾有以諸臣賢否陳於左右;而陛下於諸臣,亦嘗一思其得罪之故乎?果以為皆由聖怒,則諸臣自孟化鯉而外,未聞忤旨,何以皆罷斥?即使批鱗逆耳,如董基等,陛下已嘗收錄,何獨於諸臣不然?臣恐陛下有祛邪之果斷,而左右反借以行媢嫉之私;陛下有容言之盛心,而臣工反遺以拒諫諍之誚。傳之四海,垂諸史冊,為聖德累不小。

輔臣王錫爵等,跡其自待,若愈於張居正、申時行,察其用心,何以異於五十步笑百步?即如諸臣罷斥,果以為當然,則是非邪正,恒人能辨,何忍坐視至尊之過舉,得毋內洩其私憤,而利於斥逐之盡乎?末力詆鄭材、楊應宿讒諂宜黜。應宿亦疏訐攀龍,語極妄誕。疏並下部院,議請薄罰兩臣,稍示懲創。帝不許,鐫應宿二秩,謫攀龍揭陽添注典史。御史吳弘濟等論救,并獲譴。攀龍之官七月,以事歸。尋遭親喪,遂不出,家居垂三十年。言者屢薦,帝悉不省。

熹宗立,起光祿丞。天啟元年進少卿。明年四月,疏劾戚畹鄭養性,言:「張差梃擊實養性父國泰主謀。今人言籍籍,咸疑養性交關奸宄,別懷異謀,積疑不解,當思善全之術。至劉保謀逆,中官盧受主之,劉于簡獄詞具在。受本鄭氏私人,而李如楨一家交關鄭氏,計陷名將,失地喪師。于簡原供,明言李永芳約如楨內應。若崔文昇素為鄭氏腹心,知先帝癥虛,故用泄藥,罪在不赦。陛下僅行斥逐,而文昇猶潛住都城。宜勒養性還故里,急正如楨、文升典刑,用章國法。」疏入,責攀龍多言,然卒遣養性還籍。

孫慎行以「紅丸」事攻舊輔方從哲,下廷議。攀龍引《春秋》首惡之誅,歸獄從哲。給事中王志道為從哲解,攀龍遺書切責之。尋改太常少卿,疏陳務學之要,因言:「從哲之罪非止紅丸,其最大者在交結鄭國泰。國泰父子所以謀危先帝者不一,始以張差之梃,繼以美姝之進,終以文昇之藥,而從哲實左右之。力扶其為鄭氏者,力鋤其不為鄭氏者;一時人心若狂,但知鄭氏,不知東宮。此賊臣也,討賊則為陛下之孝。而說者乃曰『為先帝隱諱則為孝』,此大亂之道也。陛下念聖母則宣選侍之罪,念皇考則隆選侍之恩,仁之至義之盡也,而說者乃曰『為聖母隱諱則為孝』。明如聖諭,目為假託;忠如楊漣,謗為居功。人臣避居功,甘居罪,君父有急,袖手旁觀,此大亂之道也。惑於其說,孝也不知其為孝,不孝也以為大孝;忠也不知其為忠,不忠也以為大忠。忠孝皆可變亂,何事不可妄為?故從哲、養性不容不討,奈何猶令居輦轂下!」時從哲輩奧援甚固,摘疏中「不孝」語激帝怒,將加嚴譴。葉向高力救,乃奪祿一年。旋改大理少卿。鄒元標建書院,攀龍與焉。元標被攻,攀龍請與同罷,詔留之。進太僕卿,擢刑部右侍郎。

四年八月,拜左都御史。楊漣等群擊魏忠賢,勢已不兩立。及向高去國,魏廣微日導忠賢為惡,而攀龍為趙南星門生,並居要地。御史崔呈秀按淮、揚還,攀龍發其穢狀,南星議戍之。呈秀窘,急走忠賢所,乞為義兒,遂摭謝應祥事,謂攀龍黨南星。嚴旨詰責,攀龍遽引罪去。頃之,南京御史游鳳翔出為知府,訐攀龍挾私排擠。詔復鳳翔故官,削攀龍籍。呈秀憾不已,必欲殺之,竄名李實劾周起元疏中,遣緹騎往逮。攀龍晨謁宋儒楊龜山祠,以文告之。歸與二門生一弟飲後園池上,聞周順昌已就逮,笑曰:「吾視死如歸,今果然矣。」入與夫人語,如平時。出,書二紙告二孫曰:「明日以付官校。」因遣之出,扃戶。移時諸子排戶入,一燈熒然,則已衣冠自沈於池矣。發所封紙,乃遺表也,云:「臣雖削奪,舊為大臣,大臣受辱則辱國。謹北向叩頭,從屈平之遺則。」復別門人華允誠書云:「一生學問,至此亦少得力。」時年六十五。遠近聞其死,莫不傷之。

呈秀憾猶未釋,矯詔下其子世儒吏。刑部坐世儒不能防閑其父,謫為徒。崇禎初,贈太子少保,兵部尚書,謚忠憲,授世儒官。

初,海內學者率宗王守仁,攀龍心非之。與顧憲成同講學東林書院,以靜為主。操履篤實,粹然一出於正,為一時儒者之宗。海內士大夫,識與不識,稱高、顧無異詞。攀龍削官之秋,詔毀東林書院。莊烈帝嗣位,學者更修復之。

馮從吾,字仲好,長安人。萬歷十七年進士。改庶吉士,授御史。巡視中城,閹人修刺謁,拒卻之。禮科都給事中胡汝寧傾邪狡猾,累劾不去。從吾發其奸,遂調外。時當大計,從吾嚴邏偵,苞苴絕跡。

二十年正月,抗章言:「陛下郊廟不親,朝講不御,章奏留中不發。試觀戊子以前,四裔效順,海不揚波;己丑以後,南倭告警,北寇渝盟,天變人妖,疊出累告。勵精之效如彼,怠斁之患如此。近頌敕諭,謂聖體違和,欲借此自掩,不知鼓鐘於宮,聲聞於外。陛下每夕必飲,每飲必醉,每醉必怒。左右一言稍違,輒斃杖下,外庭無不知者。天下後世,其可欺乎!願陛下勿以天變為不足畏,勿以人言為不足恤,勿以目前晏安為可恃,勿以將來危亂為可忽,宗社幸甚。」帝大怒,欲廷杖之。會仁聖太后壽辰,閣臣力解得免。尋告歸,起巡長蘆鹽政。潔己惠商,奸宄斂迹。既還朝,適帝以軍政大黜兩京言官。從吾亦削籍,猶以前疏故也。

從吾生而純愨,長志濂、洛之學,受業許孚遠。罷官歸,杜門謝客,取先正格言,體驗身心,造詣益邃。家居二十五年。光宗踐阼,起尚寶卿,進太僕少卿,並以兄喪未赴。俄改大理。

天啟二年擢左僉都御史。甫兩月,進左副都御史。廷議「三安」,從吾言:「李可灼以至尊嘗試,而許其引疾,當國何心!至梃擊之獄,與發奸諸臣為難者,即奸人也。」由是群小惡之。

已,與鄒元標共建首善書院,集同志講學其中,給事中朱童蒙遂疏詆之。從吾言:「宋之不競,以禁講學故,非以講學故也。我二祖表章《六經》,天子經筵,皇太子出閣,皆講學也。臣子以此望君,而己則不為,可乎?先臣守仁,當兵事倥傯,不廢講學,卒成大功。此臣等所以不恤毀譽,而為此也。」因再稱疾求罷,帝溫詔慰留。而給事中郭允厚、郭興治復相繼詆元標甚力。從吾又上言:「臣壯歲登朝,即與楊起元、孟化鯉、陶望齡輩立講學會,自臣告歸乃廢。京師講學,昔已有之,何至今日遂為詬厲?」因再疏引歸。

四年春,起南京右都御史,累辭未上,召拜工部尚書。會趙南星、高攀龍相繼去國,連疏力辭,予致仕。明年秋,魏忠賢黨張訥疏詆從吾,削籍。鄉人王紹徽素銜從吾,及為吏部,使喬應甲撫陜,捃摭百方,無所得。乃毀書院,曳先聖像,擲之城隅。從吾不勝憤悒,得疾卒。崇禎初,復官,贈太子太保,謚恭定。

贊曰:趙南星諸人,持名檢,勵風節,嚴氣正性,侃侃立朝,天下望之如泰山喬嶽。《詩》有之,「邦之司直」,其斯人謂歟?權枉盈廷,譴謫相繼,「人之云亡,邦國殄瘁」,悲夫!


\end{pinyinscope}