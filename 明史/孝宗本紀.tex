\article{孝宗本紀}

\begin{pinyinscope}
孝宗達天明道純誠中正聖文神武至仁大德敬皇帝,諱祐堂,憲宗第三子也。母淑妃紀氏,成化六年七月生帝於西宮。時萬貴妃專寵,宮中莫敢言。悼恭太子薨後,憲宗始知之,育周太后宮中。十一年,敕禮部命名,大學士商輅等因以建儲請。是年六月,淑妃暴薨,帝年六歲,哀慕如成人。十一月,立為皇太子。

二十三年八月,憲宗崩。九月壬寅,即皇帝位。大赦天下,以明年為弘治元年。丁未,斥諸佞倖侍郎李孜省、太監梁芳、外戚萬喜及其黨,謫戍有差。冬十月丁卯,汰傳奉宮,罷右通政任傑、侍郎蒯鋼等千餘人,論罪戍斥。革法王、佛子、國師、真人封號。乙亥,尊皇太后周氏為太皇太后,皇后王氏為皇太后。丙子,立妃張氏為皇后。丁亥,萬安罷。壬辰,追謚母淑妃為孝穆皇太后。癸巳,吏部左侍郎兼翰林學士徐溥入閣預機務。十一月癸丑,尹直罷。乙卯,詹事劉健為禮部侍郎兼翰林學士,入閣預機務。戊午,下梁芳、李孜省於獄。十二月壬午,葬純皇帝於茂陵。是月,免江西、湖廣被災稅糧。是年,安南、暹羅、哈密、土魯番、烏斯藏、琉球入貢。封占城王子古來為王,諭安南黎灝還占城侵地。

弘治元年春正月己亥,享太廟。丙午,大祀天地於南郊。己未,始考察鎮守武臣。二月戊戌,祭社稷。丁未,耕耤田。封哈密衛左都督罕慎為忠順王。丙辰,禁廷臣請託公事。三月乙丑,疏文武大臣及中外四品以上官姓名,揭文華殿壁。癸酉,釋奠於先師孔子。乙亥,小王子寇蘭州,都指揮廖斌擊敗之。丙子,御經筵。丁丑,命儒臣日講。夏四月甲寅,以天暑錄囚。嗣後歲以為常。六月癸巳朔,日有食之。秋七月戊辰,減浙江銀課,汰管理銀場官。八月乙巳,小王子犯山丹、永昌。辛亥,犯獨石、馬營。冬十月乙卯,振湖廣、四川饑。十一月甲申,妖僧繼曉伏誅。乙酉,免河南被災秋糧。是年,土魯番殺忠順王罕慎,復據哈密。琉球、占城、撒馬兒罕、烏斯藏入貢。

二年春正月丁卯,收已故內臣賜田,給百姓。辛未,大祀天地於南郊。二月癸巳,振四川饑。三月己未,免陜西被災秋糧三分之二。戊寅,閉會川衛銀礦。

夏五月庚申,河決開封,入沁河,役五萬人治之。秋七月癸亥,以京師霪雨、南京大風雷修省,求直言。戊寅,振畿內水災。免稅糧,給貧民麥種。八月丁酉,復四川流民復業者雜役三年。己酉,憲宗神主祔太廟。十一月戊午,順天饑,發粟平糶。十二月甲申朔,日有食之。辛卯,賜於謙謚,立祠曰「旌功」。是年,土魯番入貢。撒馬兒罕貢獅子、鸚鵡,卻之。三年春正月甲子,大祀天地於南郊。二月壬辰,免河南被災秋糧。甲午,戶部請免南畿、湖廣稅糧。上曰:「凶歲義當損上益下。必欲取盈,如病民何。」悉從之。三月丙辰,命天下預備倉積粟,以里數多寡為差,不及額者罪之。庚午,賜錢福等進士及第、出身有差。甲戌,侍郎張海、通政使元守直閱邊。秋九月庚戌,禁內府加派供御物料。閏月癸巳,禁宗室、勛戚奏請田土及受人投獻。冬十一月甲辰,停工役,罷內官燒造瓷器。十二月辛亥,以彗星見,敕群臣修省,陳軍民利病。己未,京師地震。壬戌,減供御品物,罷明年上元燈火。是年,琉球、安南、哈密、撒馬兒罕、天方、土魯番入貢。

四年春正月癸未,以修省罷上元節假。己丑,大祀天地於南郊,停慶成宴。二月己巳,敕法司曰:「曩因天道示異,敕天下諸司審錄重囚,發遣數十百人。朕以為與其寬之於終,孰若謹之於始。嗣後兩京三法司及天下問刑官,務存心仁恕,持法公平,詳審其情罪所當,庶不背於古聖人欽恤之訓。」六月辛亥,京師地震。

秋八月庚戌,蘇、松、浙江水,停本年織造。乙卯,南京地震。己未,封皇弟祐榰為壽王,祐梈汝王,祐橓涇王,祐樞榮王,祐楷申王。冬十月丙辰,以皇長子生,詔天下。戊午,河溢,振河南被災者。乙丑,禮部尚書丘濬兼文淵閣大學士,預機務。十一月庚辰,振南畿災。十二月甲子,土魯番以哈密地及金印來歸。是年,暹羅入貢。

五年春正月壬午,大祀天地於南郊。二月丙寅,命陜巴襲封忠順王。庚午,減陜西織造絨毼之半。三月戊寅,立皇子厚照為皇太子,大赦。錄太祖廟配享功臣絕封者後。辛卯,廣西副總兵馬俊、參議馬鉉、千戶王珊等討古田叛僮,遇伏死。夏六月丁未,免南畿去年被災稅糧。秋七月甲午,振南京、浙江、山東饑。八月癸卯,劉吉致仕。乙丑,停蘇、松、浙江額外織造,召督造官還。冬十月壬戌,湖廣總兵官鎮遠侯顧溥、貴州巡撫都御史鄧廷瓚、太監江惪會師討貴州黑苗。十一月丙申,閉溫、處銀坑。十二月丁巳,荊王見潚有罪,廢為庶人。是年,琉球、烏斯藏、土魯番入貢。火剌札國貢方物,不受,給廩食遣還。

六年春正月己卯,大祀天地於南郊。二月甲寅,錄常遇春、李文忠、鄧愈、湯和後裔,世襲指揮使。丁巳,擢布政使劉大夏右副都御史,治張秋決河。三月癸未,賜毛澄等進士及第、出身有差。夏四月己亥,土魯番速檀阿黑麻襲執陜巴,據哈密。己酉,侍郎張海、都督同知緱謙經略哈密。辛酉,久旱,敕修省,求直言。五月丙寅,小王子犯寧夏,殺指揮趙璽。閏月乙未,免南京被災秋糧。六月庚午,捕蝗。壬申,都御史閔珪擊破古田叛僮。秋八月甲戌,免順天被災夏稅。九月丁酉,免陜西被災夏稅。冬十月丙寅,以災傷罷明年上元燈火。庚辰,停甘肅織造絨毼。十一月庚申,振京師流民。十二月己卯,敕天下鎮巡官修省。

是年,安南、烏斯藏、土魯番、暹羅入貢。

七年春正月丁酉,大祀天地於南郊。二月甲子,以去年冬孝陵風雷之變,遣使祭告,修省,求直言,命內外慎刑獄,決輕繫。三月癸巳,貴州黑苗平。戊申,兩畿捕蝗。夏五月甲辰,太監李興、平江伯陳銳同劉大夏治張秋決河。秋七月乙巳,京師地震。丙午,工部侍郎徐貫、巡撫副都御史何鑑經理南畿水利。九月丁亥,以水災停蘇、松諸府所辦物料,留關鈔、戶鹽備振。冬十一月壬子,京師地震。十二月甲戌,張秋河工成。乙卯,振甘、涼被兵軍民,給牛種。是年,免北京、河南、湖廣、陜西、山西被災稅糧。琉球入貢。以土魯番據哈密,卻其貢使。

八年春正月乙未,大祀天地於南郊。以太皇太后不豫,免慶成宴。壬子,甘肅總兵官劉寧敗小王子於涼州。二月乙卯朔,日有食之。戊午,丘濬卒。乙丑,禮部侍郎李東陽、少詹事謝遷入閣預機務。己卯,黃陵岡河口工成。三月壬辰,免湖廣被災稅糧。己亥,寧夏地震。夏四月甲寅,蘇、松各府治水工成。壬戌,諭吏部、都察院,人材進退,考察務得實跡,不可偏聽枉人。五月己丑,免南畿被災秋糧。

秋七月丁亥,封宋儒楊時將樂伯,從祀孔子廟庭。戊子,廣西副總兵歐磐擊破平樂叛瑤。八月癸亥,以四方災異數見,敕群臣修省。冬十一月己酉,免直隸被災秋糧。十二月辛酉,巡撫甘肅僉都御史許進、總兵官劉寧入哈密,土魯番遁,遂班師。是年,爪哇、占城、烏斯藏入貢。乜克力諸部款肅州塞求入貢,卻之。

九年春正月壬辰,大祀天地於南郊。二月庚午,免河南被災稅糧。辛未,右通政張璞、大理少卿馬中錫閱邊。三月丙申,賜朱希周等進士及第、出身有差。夏四月戊子,以岷王膺鉟奏,逮武岡知州劉遜。給事中、御史龐泮、劉紳等諫,下錦衣衛獄,尋釋之。六月庚子,免江西被災稅糧。秋八月壬寅,免湖廣被災秋糧。九月己酉,禁勢家侵奪民利。是年,日本、琉球、烏斯藏入貢。

十年春正月庚戌,大祀天地於南郊。三月辛亥,以旱霾修省,求直言。甲子,召大學士劉健、李東陽、謝遷於文華殿議庶政,後以為常。夏五月戊辰,小王子犯潮河川。己巳,犯大同。六月己卯,侍郎劉大夏、李介理宣府、大同軍餉。秋七月癸丑,都督楊玉帥京營軍,備永平。冬十一月庚子,土魯番歸陜巴,乞通貢。是年,免南畿、山西、陜西被災稅糧,振山東、四川水災。安南、暹羅、烏斯藏入貢。

十一年春正月丁未,大祀天地於南郊。二月己巳,小王子遣使求貢。夏五月戊申,甘肅參將楊翥敗小王子於黑山。秋七月己酉,總制三邊都御史王越襲小王子於賀蘭山後,敗之。癸亥。徐溥致仕。八月癸未,振祥符民被河患者。冬十月丙寅,命工作不得役團營軍士。甲戌,清寧宮災。丁亥,敕群臣修省,求直言,罷明年上元燈火。十一月壬子,免陜西織造羊絨。閏月壬戌朔,日有食之。乙酉,罷福建織造綵布。十二月庚子。禁中外奢靡踰制。壬子,以清寧宮災詔赦天下。是年,免山西、陜西、兩畿、廣西、廣東被災稅糧。土魯番、烏斯藏入貢。

十二年春正月辛未,大祀天地於南郊,免慶成宴。二月壬辰,免山東被災夏稅。戊申,嚴左道惑眾之禁。三月丁丑,賜倫文敘等進士及第、出身有差。夏四月癸巳,敕宣、大、延綏備邊。是月,免湖廣、江西被災稅糧。五月戊寅,免南畿被災秋糧。六月甲辰,闕里先師廟災,遣使慰祭。秋八月,免河南、南畿被災夏稅。九月壬午,普安賊婦米魯作亂。甲申,重建清寧宮成。是年,占城、烏斯藏、土魯番、爪哇、撒馬兒罕入貢。

十三年春正月乙丑,大祀天地於南郊。二月戊子,免山西被災稅糧,庚寅,定問刑條例。乙未,嚴旌舉連坐之法。夏四月,火篩寇大同,游擊將軍王杲敗績於威遠衛。乙巳,平江伯陳銳為靖虜將軍,充總兵官,太監金輔監軍,戶部左侍郎許進提督軍務,禦之。五月甲寅朔,日有食之。丙辰,召大學士劉健、李東陽、謝遷於平臺,議京營將領。癸亥,火篩大舉入寇大同左衛,遊擊將軍張俊禦卻之。六月甲申,免江西被災秋糧,停山、陜採辦物料。庚子,召陳銳、金輔還,保國公朱暉、太監扶安往代,益兵禦寇。秋七月己巳,京師地震。八月辛卯,振江西水災。冬十月戊申,兩京地震。是月,小王子諸部寇大同。十二月辛丑,火篩寇大同,南掠百餘里。是年,小王子部入居河套,犯延綏神木堡。琉球、土魯番、烏斯藏入貢。

十四年春正月庚戌朔,陜西地大震。乙未,大祀天地於南郊。二月己亥,罷陜西織造中官。夏四月庚辰,工部侍郎李鐩總督延綏邊餉。戊子,保國公朱暉、提督軍務都御史史琳、監軍太監苗逵分道進師延綏。戊戌,免陜西、山西物料。是月,火篩諸部寇固原。五月庚戌,振大同被兵軍民,免稅糧。辛酉,免陜西被災稅糧。戊辰,修闕里先師廟。命各布政使司上地里圖。秋七月丁未,泰寧衛賊犯遼東,掠長勝諸屯堡。癸亥。南京戶部尚書王軾兼左副都御史提督軍務,討貴州賊婦米魯。丁卯,朱暉、史琳襲小王子於河套。庚午,分遣給事中、御史清理屯田。閏月乙酉,都指揮王泰禦小王子於鹽池,戰死。戊戌,振兩畿、江西、山東、河南水災。八月己酉,免河南被災稅糧。是月,火篩諸部犯固原,大掠韋州、環縣、萌城、靈州。己巳,減光祿寺供應,如元年制。火篩諸部犯寧夏東路。九月丙子朔,日有食之。丁亥,遣使募兵於延綏、寧夏、甘、涼。甲辰,召史琳還,起秦紘為戶部尚書兼副都御史,代之。冬十一月癸巳,分遣侍郎何鑑、大理寺丞吳一貫振恤兩畿、山東、河南饑民。十二月戊辰,遼東大饑,振之。是月,寇出河套。是年,免湖廣、江西、山西、山東、陜西、河南、畿內被災稅糧。安南、琉球入貢。

十五年春正月丙子,朱暉帥師還。丙戌,大祀天地於南郊。二月癸丑,免河南被災稅糧。三月癸未,罷饒州督造瓷器中官。庚寅,賜康海等進士及第、出身有差。夏四月壬寅,振京師貧民。五月庚子,免湖廣被災秋糧。秋七月己卯,錄劉基後裔世襲指揮使。己丑,王軾破斬米魯,貴州賊平。辛卯,命各邊衛設養濟院、漏澤園。八月庚戌,以南京、鳳陽霪雨大風,江溢為災。遣使祭告,敕兩京群臣修省。九月庚午朔,日有食之。戊子,放減內府所畜鳥獸。冬十月癸卯,罷明年上元燈火。十一月壬申,瓊州黎賊作亂。甲午,罷廣東採珠。十二月己酉,《大明會典》成。辛亥,以疾不視朝。是月,免南畿被災秋糧。是年,琉球、安南入貢。

十六年春正月癸酉,遣官代享太廟。二月辛丑,視朝。戊申,大祀天地於南郊。三月癸巳,免山西被災稅糧。夏四月辛亥,敕宣、大嚴邊備。五月戊子,以雲南災變敕群臣修省。刑部侍郎樊瑩巡視雲、貴,察官吏,問民疾苦。

秋七月,廣東官軍討黎賊,敗之。九月丁丑,振兩畿、浙江、山東、河南、湖廣被災軍民。冬十一月甲戌,罷營造器物及明年上元煙火。是月,免南畿被災秋糧。十二月丙午,免淮、揚、浙江物料。是年,安南、暹羅、哈密、土魯番、撒馬兒罕入貢。

十七年春正月辛未,南京工部侍郎高銓振應天饑。甲戌,大祀天地於南郊。壬午,嚴誣告之禁。二月甲寅,減供用物料。己未,嚴讖緯妖書之禁。庚申,免浙江被災稅糧。三月壬戌,太皇太后崩。癸未,定太廟各室一帝一后之制。夏四月己酉,葬孝肅皇太后。閏月辛酉,闕里先師廟成,遣大學士李東陽祭告。庚午,免山東被災稅糧。乙亥,以四方災荒敕群臣修省。庚辰,命諸司詳議害民弊政。五月壬辰,罷南京、蘇、杭織造中宮。六月乙亥,始命兩京五品以下官六年一考察。辛巳,召劉健、李東陽於暖閣,議邊務。癸未,火篩入大同,指揮鄭瑀力戰死。秋七月癸巳,工部侍郎李鐩、大理少卿吳一貫、通政司參議便叢蘭分道經略邊塞。甲午,左副都御史閻仲宇、通政司參議熊偉分理邊餉。八月戊辰,命天下撫、按、三司官奏軍民利病,土民建言可採者,所司以聞。甲申,免南畿被災夏稅。丁亥,召馬文升、戴珊於暖閣,諭以明年考察,務訪實跡,以求至當。九月庚寅,諭法司不得任情偏執,致淹獄囚。甲寅,太常少卿孫交經略宣、大邊務。丁巳,御暖閣,諭劉健、李東陽、謝遷:「諸邊首功,巡按御史察勘,動淹歲年,非所以示勸。自今奏報,以遠近立限。違者詰治。」諭講官進講直言毋諱。冬十一月戊子,罷雲南銀場。十二月庚午,申閉糴之禁。甲申,免湖廣被災秋糧。是年,琉球、撒馬兒罕、哈密、烏斯藏入貢。

十八年春正月己丑,小王子諸部圍靈州,入花馬池,遂掠韋州、環縣。戶部侍郎顧佐理陜西軍餉。乙未,大祀天地於南郊。甲辰,小王子陷寧夏清水營。二月戊辰,御奉天門,諭戶、兵、工三部曰:「方今生齒漸繁,而戶口、軍伍日就耗損,此皆官司撫恤無方、因仍茍且所致。其悉議弊政以聞。」三月癸卯,賜顧鼎臣等進士及第、出身有差。夏四月戊寅,刑部侍郎何鑑撫輯荊、襄流民。甲申,帝不豫。五月庚寅,大漸,召大學士劉健、李東陽、謝遷受顧命。辛卯,崩於乾清宮,年三十有六。六月庚申,上尊謚,廟號孝宗,葬泰陵。

贊曰:明有天下,傳世十六,太祖、成祖而外,可稱者仁宗、宣宗、孝宗而已。仁、宣之際,國勢初張,綱紀修立,淳樸未漓。至成化以來,號為太平無事,而晏安則易耽怠玩,富盛則漸啟驕奢。孝宗獨能恭儉有制,勤政愛民,兢兢於保泰持盈之道,用使朝序清寧,民物康阜。《易》曰:「無平不陂,無往不復,艱貞無咎。」知此道者,其惟孝宗乎!


\end{pinyinscope}