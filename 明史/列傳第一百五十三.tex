\article{列傳第一百五十三}

\begin{pinyinscope}
范景文倪元璐李邦華王家彥孟兆祥子章明施邦曜凌義渠

崇禎十有七年三月,流賊李自成犯京師。十九日丁未,莊烈帝殉社稷。文臣死國者,東閣大學士范景文而下,凡二十有一人。福王立南京,並予贈謚。皇清順治九年,世祖章皇帝表章前代忠臣,所司以范景文、倪元璐、李邦華、王家彥、孟兆祥、子章明、施邦曜、凌義渠、吳麟徵、周鳳翔、馬世奇、劉理順、汪偉、吳甘來、王章、陳良謨、申佳允、許直、成德、金鉉二十人名上。命所在有司各給地七十畝,建祠致祭,且予美謚焉。

范景文,字夢章,吳橋人。父永年,南寧知府。景文幼負器識,登萬曆四十一年進士,授東昌推官。以名節自勵,苞苴無敢及其門。歲大饑,盡心振救,闔郡賴之。用治行高等,擢吏部稽勳主事,歷文選員外郎,署選事。泰昌時,群賢登進,景文力為多,尋乞假去。

天啟五年二月,起文選郎中,魏忠賢暨魏廣微中外用事,景文同鄉,不一詣其門,亦不附東林,孤立行意而已。嘗言:「天地人才,當為天地惜之。朝廷名器,當為朝廷守之。天下萬世是非公論,當與天下萬世共之。」時以為名言。視事未彌月,謝病去。

崇禎初,用薦召為太常少卿。二年七月,擢右僉都御史,巡撫河南。京師戒嚴,率所部八千人勤王,餉皆自齎。抵涿州,四方援兵多剽掠,獨河南軍無所犯。移駐都門,再移昌平,遠近恃以無恐。明年三月,擢兵部添注左侍郎,練兵通州。通鎮初設,兵皆召募,景文綜理有法,軍特精。嘗請有司實行一條鞭法,徭役歸之官,民稍助其費,供應平買,不立官價名。帝令永著為例。居二年,以父喪去官。

七年冬,起南京右都御史。未幾,就拜兵部尚書,參贊機務。屢遣兵戍池河、浦口,援廬州,扼滁陽,有警輒發,節制精明。嘗與南京戶部尚書錢春以軍食相訐奏,坐鐫秩視事。已,敘援剿功,復故秩。十一年冬,京師戒嚴,遣兵入衛。楊嗣昌奪情輔政,廷臣力爭多被謫,景文倡同列合詞論救。帝不悅,詰首謀,則自引罪,且以眾論僉同為言。帝益怒,削籍為民。

十五年秋,用薦召拜刑部尚書,未上,改工部。入對,帝迎勞曰:「不見卿久,何臒也!」景文謝。十七年二月,命以本官兼東閣大學士,入參機務。未幾,李自成破宣府,烽火逼京師。有請帝南幸者,命集議閣中。景文曰:「固結人心,堅守待援而已,此外非臣所知。」及都城陷,趨至宮門,宮人曰:「駕出矣。」復趨朝房,賊已塞道。從者請易服還邸,景文曰:「駕出安歸?」就道旁廟草遺疏,復大書曰:「身為大臣,不能滅賊雪恥,死有餘恨。」遂至演象所拜辭闕墓,赴雙塔寺旁古井死。景文死時,猶謂帝南幸也。贈太傅,謚文貞。本朝賜謚文忠。

倪元璐,字玉汝,上虞人。父凍,歷知撫州、淮安、荊州、瓊州四府,有當官稱。

天啟二年,元璐成進士,改庶吉士,授編修。冊封德府,移疾歸。還朝,出典江西鄉試。暨復命,則莊烈帝踐阼,魏忠賢已伏誅矣。楊維垣者,逆奄遺孽也,至是上疏並詆東林、崔、魏。元璐不能平,崇禎元年正月上疏曰:

臣頃閱章奏,見攻崔、魏者必與東林並稱邪黨。夫以東林為邪黨,將以何者名崔、魏?崔、魏既邪黨矣,擊忠賢、呈秀者又邪黨乎哉!東林,天下才藪也,而或樹高明之幟,繩人過刻,持論太深,謂之非中行則可,謂之非狂狷不可。且天下議論,寧假借,必不可失名義;士人行己,寧矯激,必不可忘廉隅。自以假借矯激為大咎,於是彪虎之徒公然背畔名義,決裂廉隅。頌德不已,必將勸進;建祠不已,必且呼嵩。而人猶且寬之曰:「無可奈何,不得不然耳。」充此無可奈何、不得不然之心,又將何所不至哉!乃議者以忠厚之心曲原此輩,而獨持已甚之論苛責吾徒,所謂舛也。今大獄之後,湯火僅存,屢奉明綸,俾之酌用,而當事者猶以道學封疆,持為鐵案,毋亦深防其報復乎?然臣以為過矣。年來借東林媚崔、魏者,其人自敗,何待東林報復?若不附崔、魏,又能攻去之,其人已喬嶽矣,雖百東林烏能報復哉?臣又伏讀聖旨,有「韓爌清忠有執,朕所鑒知」之諭。而近聞廷臣之議,殊有異同,可為大怪。爌相業光偉,他不具論,即如紅丸議起,舉國沸然,爌獨侃侃條揭,明其不然。夫孫慎行,君子也,爌且不附,況他人乎!而今推轂不及,點灼橫加,則徒以其票擬熊廷弼一事耳。廷弼固當誅,爌不為無說,封疆失事,纍纍有徒,乃欲獨殺一廷弼,豈平論哉?此爌所以閣筆也。然廷弼究不死於封疆而死於局面,不死於法吏而死於奸璫,則又不可謂後之人能殺廷弼,而爌獨不能殺之也。又如詞臣文震孟正學勁骨,有古大臣之品,三月居官,昌言獲罪,人以方之羅倫、舒芬。而今起用之旨再下,謬悠之譚不已,將毋門戶二字不可重提耶?用更端以相遮抑耶?書院、生祠,相勝負者也,生祠毀,書院豈不當修復!

時柄國者悉忠賢遺黨,疏入,以論奏不當責之。於是維垣復疏駁元璐。元璐再疏曰:

臣前疏原為維垣發也。陛下明旨曰:「分別門戶,已非治征」,曰「化異為同」,曰「天下為公」,而維垣則倡為孫黨、趙黨、熊黨、鄒黨之說。是陛下於方隅無不化,而維垣實未化;陛下於正氣無不伸,而維垣不肯伸。

維垣怪臣盛稱東林,以東林嘗推李三才而護熊廷弼也。抑知東林有力擊魏忠賢之楊漣,首劾崔呈秀之高攀龍乎!忠賢窮凶極惡,維垣猶尊稱之曰「廠臣公」、「廠臣不愛錢」、「廠臣知為國為民」,而何責乎三才?五彪五虎之罪,刑官僅擬削奪,維垣不駁正,又何誅乎廷弼?維垣又怪臣盛稱韓爌。夫舍爌昭然忤璫之大節,而加以罔利莫須有之事,已為失平。至廷弼行賄之說,乃忠賢借以誣陷清流,為楊、左諸人追贓地耳,天下誰不知,維垣猶守是說乎?維垣又怪臣盛稱文震孟。夫震孟忤璫削奪,其破帽策蹇傲蟒玉馳驛語,何可非?維垣試觀數年來破帽策蹇之輩,較超階躐級之儔,孰為榮辱。自此義不明,畏破帽策蹇者,相率而頌德建祠,希蟒玉馳驛者呼父、呼九千歲而不怍,可勝歎哉!維垣又怪臣盛稱鄒元標。夫謂都門聚講為非則可,謂元標講學有他腸則不可。當日忠賢驅逐諸人,毀廢書院者,正欲箝學士大夫之口,恣行不義耳。自元標以偽學見驅,而逆璫遂以真儒自命,學宮之內,儼然揖先聖為平交。使元標諸人在,豈遂至此!維垣又駁臣假借矯激。夫當崔、魏之世,人皆任真率性,頌德建祠。使有一人假借矯激,而不頌不建,豈不猶賴是人哉!維垣以為真小人,待其貫滿可攻去之,臣以為非計也。必待其貫滿,其敗壞天下事已不可勝言,雖攻去之,不已晚乎!即如崔、魏,貫滿久矣,不遇聖明,誰攻去之?維垣終以無可奈何為頌德建祠者解,臣以為非訓也。假令呈秀一人舞蹈稱臣於逆璫,諸臣亦以為無可奈何而從之乎?又令逆璫以兵劫諸臣使從叛逆,諸臣亦靡然從之,以為無可奈何而然乎?維垣又言「今日之忠直,不當以崔、魏為對案」,臣謂正當以崔、魏為對案也。夫人品試之崔、魏而定矣,故有東林之人,為崔、魏所恨其牴觸、畏其才望而必欲殺之逐之者,此正人也。有攻東林之人,雖為崔、魏所借,而勁節不阿,或遠或逐者,亦正人也。以崔、魏定邪正,猶以明鏡別妍媸。維垣不取證於此,而安取證哉!

總之東林之取憎於逆璫獨深,其得禍獨酷。在今日當曲原其被抑之苦,不當毛舉其尺寸之瑕。乃歸逆璫以首功,代逆璫而分謗,斯亦不善立論者矣。

疏入,柄國者以互相詆訾兩解之。當是時,元兇雖殛,其徒黨猶盛,無敢頌言東林者。自元璐疏出,清議漸明,而善類亦稍登進矣。

元璐尋進侍講。其年四月,請毀《三朝要典》,言:「梃擊、紅丸、移宮三議,哄於清流,而《三朝要典》一書,成於逆豎。其議可兼行,其書必當速毀。蓋當事起議興,盈廷互訟。主梃擊者力護東宮,爭梃擊者計安神祖。主紅丸者仗義之言,爭紅丸者原情之論。主移宮者弭變於幾先,爭移宮者持平於事後。數者各有其是,不可偏非。總在逆璫未用之先,雖甚水火,不害塤篪,此一局也。既而楊漣二十四罪之疏發,魏廣微此輩門戶之說興,於是逆璫殺人則借三案,群小求富貴則借三案。經此二借,而三案全非矣。故凡推慈歸孝於先皇,正其頌德稱功於義父,又一局也。網已密而猶疑有遺鱗,勢已重而或憂其翻局。崔、魏諸奸始創立私編,標題《要典》,以之批根今日,則眾正之黨碑;以之免死他年,即上公之鐵券。又一局也。由此而觀,三案者,天下之公議;《要典》者,魏氏之私書。三案自三案,《要典》自《要典》也。今為金石不刊之論者,誠未深思。臣謂翻即紛囂,改亦多事,惟有毀之而已。」帝命禮部會詞臣詳議。議上,遂焚其板。侍講孫之獬,忠賢黨也,聞之,詣閣大哭,天下笑之。

元璐歷遷南京司業、右中允。四年,進右諭德,充日講官,進右庶子。上制實八策:曰間插部,曰繕京邑,曰優守兵,曰靖降人,曰益寇餉,曰儲邊才,曰奠輦轂,曰嚴教育。又上制虛八策:曰端政本,曰伸公議,曰宣義問,曰一條教,曰慮久遠,曰昭激勸,曰勵名節,曰假體貌。其端政本,悉規切溫體仁;其伸公議,則詆張捷薦呂純如謀翻逆案事。捷大怒,上疏力攻,元璐疏辨,帝俱不問。八年,遷國子祭酒。

元璐雅負時望,位漸通顯。帝意嚮之,深為體仁所忌。一日,帝手書其名下閣,令以履歷進,體仁益恐。會誠意伯劉孔昭謀掌戎政,體仁餌孔昭使攻元璐,言其妻陳尚存,而妾王冒繼配復封,敗禮亂法。詔下吏部核奏,其同里尚書姜逢元,侍郎王業浩、劉宗周及其從兄御史元珙,咸言陳氏以過被出,繼娶王非妾,體仁意沮。會部議行撫按勘奏,即擬旨云:「登科錄二氏並列,罪跡顯然,何待行勘。」遂落職閒住。孔昭京營不可得,遂以南京操江償之。

十五年九月,詔起兵部右侍郎兼侍讀學士。明年春抵都,陳制敵機宜,帝喜。五月,超拜戶部尚書兼翰林院學士,仍充日講官。祖制,浙人不得官戶部。元璐辭,不許。帝眷元璐甚,五日三賜對。因奏:「陛下誠用臣,臣請得參兵部謀。」帝曰:「已諭樞臣,令與卿協計。」當是時,馮元飆為兵部,與元璐同志,鉤考兵食,中外想望治平。惟帝亦以用兩人晚,而時事益不可為,左支右詘,既已無可奈何。故事,諸邊餉司悉中差,元璐請改為大差,兼兵部銜,令清核軍伍,不稱職者即遣人代之。先是,屢遣科臣出督四方租賦,元璐以為擾民無益,罷之,而專責撫按。戶部侍郎莊祖誨督剿寇餉,憂為盜劫,遠避之長沙、衡州。元璐請令督撫自催,毋煩朝使。自軍興以來,正供之外,有邊餉,有新餉,有練餉,款目多,黠吏易為奸,元璐請合為一。帝皆報可。時國用益詘,而災傷蠲免又多。元璐計無所出,請開贖罪例,且令到官滿歲者,得輸貲給封誥。帝亦從之。

先是,有崇明人沈廷揚者,獻海運策,元璐奏聞。命試行,乃以廟灣船六艘聽運進。月餘,廷揚見元璐,元璐驚曰:「我已奏聞上,謂公去矣,何在此?」廷揚曰:「已去復來矣,運已至。」元璐又驚喜聞上。上亦喜,命酌議。乃議歲糧艘,漕與海各相半行焉。十月,命兼攝吏部事。陳演忌元璐,風魏藻德言於帝曰:「元璐書生,不習錢穀。」元璐亦數請解職。

十七年二月,命以原官專直日講。踰月,李自成陷京師,元璐整衣冠拜闕,大書幾上曰:「南都尚可為。死,吾分也,勿以衣衾斂。暴我屍,聊志吾痛。」遂南向坐,取帛自縊而死。贈少保,吏部尚書,謚文正。本朝賜謚文正。

李邦華,字孟闇,吉水人。受業同里鄒元標,與父廷諫同舉萬曆三十一年鄉試。父子自相鏃礪,布衣徒步赴公車。明年,邦華成進士,授涇縣知縣,有異政。行取,擬授御史。值黨論初起,朝士多詆顧憲成,邦華與相拄,遂指目邦華東林。以是,越二年而後拜命,陳法祖用人十事:曰內閣不當專用詞臣,曰詞臣不當專守館局,曰詞臣不當教習內書堂,曰六科都給事中不當內外間阻,曰御史升遷不當概論考滿,曰吏部乞假不當積至正郎,曰關倉諸差不當專用舉貢任子,曰調簡推知不當驟遷京秩,曰進士改教不當概從內轉,曰邊方州縣不當盡用鄉貢。疏上,不報。

四十一年,福王之籓已有期,忽傳旨莊田務足四萬頃。廷臣相顧愕眙,計田數必不足,則期將復更,然無敢抗言爭之者。邦華首疏諫,廷臣乃相繼爭,期得毋易。巡視銀庫,上祛弊十事,中貴不便,格不行。巡按浙江,織造中官劉成死,命歸其事於有司,別遣中官呂貴錄成遺貲。貴族奸民紀光詭稱機戶,詣闕保留貴代成督造。邦華極論二人交關作奸罪。光疏不由通政,不下內閣,以中旨行之。邦華三疏爭,皆不報。是時神宗好貨,中官有所進奉,名為孝順。疏中刺及之,並劾左右大奄之黨貴者,於是期滿久不得代。

四十四年引疾歸。時群小力排東林,指鄒元標為黨魁。邦華與元標同里,相師友,又性好別黑白。或勸其委蛇,邦華曰:「寧為偏枯之學問,不作反覆之小人。」聞者益嫉之。明年以年例出為山東參議。其父廷諫時為南京刑部郎中,亦罷歸。邦華乃辭疾不赴。天啟元年起故官,飭易州兵備。明年遷光祿少卿,即還家省父。四月,擢右僉都御史,代畢自嚴巡撫天津。軍府新立,庶務草創,邦華至,極力振飭,津門軍遂為諸鎮冠。進兵部右侍郎,復還家省父。四年夏抵京,奄黨大嘩,謂樞輔孫承宗以萬壽節入覲,將清君側之惡,邦華實召之。乃立勒承宗還鎮,邦華引疾去。明年秋,奄黨劾削其官。

崇禎元年四月,起工部右侍郎,總督河道。尋改兵部,協理戎政。還朝,召見,旋知武會試,事竣入營。故事,冬至郊,列隊扈蹕,用軍八萬五千人。至是,增至十萬有奇。時方郊,總督勳臣缺,邦華兼攝其事。所設雲輦、龍旌、寶纛、金鼓、旗幟、甲胄、劍戟,煥然一新,帝悅。明年春,幸學,亦如之。命加兵部尚書。時戎政大壞,邦華先陳更操法、慎揀選、改戰車、精火藥、專器械、責典守、節金錢、酌兌馬、練大炮九事。

京營故有占役、虛冒之弊。占役者,其人為諸將所役,一小營至四五百人,且有賣閒、包操諸弊。虛冒者,無其人,諸將及勛戚、奄寺、豪強以蒼頭冒選鋒壯丁,月支厚餉。邦華核還占役萬,清虛冒千。三大營軍十餘萬,半老弱。故事,軍缺聽告補,率由賄得。邦華必親校,非年壯力強者不錄,自是軍鮮冒濫。三營選鋒萬,壯丁七千,餉倍他軍,而疲弱不異。邦華下令,每把總兵五百,月自簡五人,年必二十五以下,力必二百五十斤以上,技必兼弓矢火炮,月一解送,補選鋒壯丁之缺,自是人人思奮。三大營領六副將,又分三十六營,官以三百六十七人計,所用掾史皆積猾。邦華按罪十餘人,又行一歲二考察之令,自是諸奸為戢。

營馬額二萬六千,至是止萬五千。他官公事得借騎,總督、協理及巡視科道,例有坐班馬,不肖且折橐入錢,營馬大耗。邦華首減己班馬三之一,他官借馬,非公事不得騎,自是濫借為希。

京營歲領太僕銀萬六千兩,屯田籽粒銀千六十兩,犒軍製器胥徒工食取給焉。各官取之無度,歲用不敷。邦華建議,先協理歲取千四百,總督巡視遞節減,自是營帑遂裕。

營將三百六十,聽用者稱是。一官缺,請託紛至。邦華悉杜絕,行計日省成法。每小營各置簿,月上事狀於協理,以定殿最。舊制,三大營外復設三備兵營,營三千人,餉視正軍,而不習技擊,益為豪家隱冒。邦華核去四千餘人,又汰老弱千,疏請歸並三大營不另設,由是戎政大釐。

倉場總督南居益言:「京營歲支米百六十六萬四千餘石,視萬曆四十六年增五萬七千餘石,宜減省。」邦華因上議軍以十二萬為額,餉以百四十四萬石為額,歲省二十二萬有奇。帝亦報可,著為令。帝知邦華忠,奏無不從,邦華亦感帝知,不顧後患。諸失利者銜次骨,而怨謗紛然矣。

其年十月,畿輔被兵,簡精卒三千守通州,二千援薊州,自督諸軍營城外,軍容甚壯。俄有命邦華軍撤還守陴,於是偵者不敢遠出,聲息遂斷,則請防寇賊,緝間諜,散奸宄,禁訛言。邦華自聞警,衣不解帶,捐貲造炮車及諸火器,又以外城單薄,自請出守。而諸不逞之徒,乃構蜚語入大內。襄城伯李守錡督京營,亦銜邦華扼己,乘間詆諆。邦華自危,上疏陳情,歸命於帝。會滿桂兵拒大清兵德勝門外,城上發大炮助桂,誤傷桂兵多。都察院都事張道澤遂劾邦華,言官交章論列,遂罷邦華閒住。自是代者以為戒,率因循姑息,戎政不可問矣。邦華前後罷免家居二十年。父廷諫無恙。

十二年四月,起南京兵部尚書,定營制,汰不急之將,並分設之營。謂守江南不若守江北,防下流不若防上流。乃由浦口歷滁、全椒、和,相形勢,繪圖以獻。於浦口置沿江敵臺,於滁設戍卒,於池河建城垣,於滁、椒咽喉則築堡於藕塘。和遭屠戮,請以隸之太平。又請開府采石之山,置哨太平之港,大墾當塗閒田數萬頃資軍儲。徐州,南北要害,水陸交會,請宿重兵,設總督,片檄徵調,奠陵京萬全之勢。皆下所司,未及行,以父憂去。

十五年冬,起故官,掌南京都察院事,俄代劉宗周為左都御史。都城被兵,即日請督東南援兵入衛,力疾上道。明年三月抵九江。左良玉潰兵數十萬,聲言餉乏,欲寄帑於南京,艨艟蔽江東下。留都士民一夕數徙,文武大吏相顧愕眙。邦華嘆曰:「中原安靜土,東南一角耳。身為大臣,忍坐視決裂,袖手局外而去乎!」乃停舟草檄告良玉,責以大義。良玉氣沮,答書語頗恭。邦華用便宜發九江庫銀十五萬餉之,而身入其軍,開誠慰勞。良玉及其下皆感激,誓殺賊報國,一軍遂安。帝聞之,大喜,陛見嘉勞。邦華跪奏移時,數詔起立,溫語如家人,中官屏息遠伏。其後召對百官,帝輒目注邦華云。舊例,御史出巡,回道考核。邦華謂回道而後黜,害政已多。論罷巡按、巡鹽御史各一人。奉命考試御史,黜冒濫者一人,追黜御史無顯過而先任推官著貪聲者一人。臺中始畏法。

十七年二月,李自成陷山西。邦華密疏請帝固守京師,仿永樂朝故事,太子監國南都。居數日未得命,又請定、永二王分封太平、寧國二府,拱護兩京。帝得疏意動,繞殿行,且讀且歎,將行其言。會帝召對群臣,中允李明睿疏言南遷便,給事中光時亨以倡言洩密糾之。帝曰:「國君死社稷,正也,朕志定矣。」遂罷邦華策不議。未幾,賊逼都城,亟詣內閣言事。魏藻德漫應曰:「姑待之。」邦華太息而出。已,率諸御史登城,群奄拒之不得上。十八日,外城陷,走宿文信國祠。明日,內城亦陷,乃三揖信國曰:「邦華死國難,請從先生於九京矣。」為詩曰:「堂堂丈夫兮聖賢為徒,忠孝大節兮誓死靡渝,臨危授命兮吾無愧吾。」遂投繯而絕。贈太保、吏部尚書,謚忠文。本朝賜謚忠肅。

王家彥,字開美,莆田人。天啟二年進士。授開化知縣,調蘭谿。擢刑科給事中,彈擊權貴無所避。

崇禎四年,請釋大學士錢龍錫於獄,龍錫得減死。請推行按月奏報例於四方,獄囚得無久淹。閩海盜劉香擾郡邑,撫鎮追剿多失利,朝議召募,將大舉。家彥言:「舊制,衛所軍餼於官,無別兵亦無別將,統於各衛之指揮。寨設號船,聊絡呼應,又添設遊擊等官,雖支洋窮港,戈船相望。臣愚以今日策防海,莫若復舊制,勤訓練。練則衛所軍皆勁卒,不練雖添設召募兵,猶驅市人而戰之,糜餉擾民無益,賊終不能盡。」時以為名言。奉命巡青,所條奏多議行。

先是,隆慶間太僕種馬額存十二萬五千,邊馬至二十六萬。言者以民間最苦養馬,所納馬又不足用,議馬徵銀十兩,加草料銀二兩,歲可得銀百四十四萬兩。中樞楊博持不可,詔折其半,而馬政始變。萬曆九年議盡行改折,南寺歲徵銀二十二萬,北寺五十一萬,銀入冏寺而馬政日弛。家彥極陳其弊,請改國初種馬及西番茶馬之制。又班軍舊額十六萬,後減至七萬,至是止二萬有奇,更有建議盡征行糧、月糧,免其番上者。家彥時巡京營,力陳不可,且請免其工役,盡歸行伍。帝皆褒納其言。遵化鐵冶久廢,奸民請開之,家彥言有害無利。復有請開開化雲霧山以興屯者,亦以家彥言而止。

屢遷戶科都給事中。軍興餉詘,總督盧象昇有因糧加餉之議,戶部尚書侯恂請於未被寇之地,士大夫家賦銀兩者,加二錢;民間五兩以上者,兩加一錢。家彥言:「民賦五兩上者,率百十家成一戶,非富民,不可以朘削。」軍食不足,畿輔、山東、河南、江北召買米豆輸天津,至九十餘萬石,吏胥侵耗率數十萬。家彥請嚴治,帝並採納焉。憂歸。

十二年起吏科都給事中。流寇日熾,緣墨吏朘民,民益走為盜。盜日多,民生日蹙。家彥上疏曰:「臣見秦、晉之間,饑民相煽,千百為群。其始率自一鄉一邑,守令早為之所,取《周官荒政十二》而行之,民何至接踵為盜,盜何至潰裂以極?論者謂功令使然,催科急者書上考,督責嚴者號循良,不肖而墨者以束濕濟其饕餮,一二賢明吏束於文法,展布莫由。惟稍寬文網,壹令撫綏,盜之聚者可散,散者可不復聚。又舊制捕蝗令,吏部歲九月頒勘合於有司,請實意舉行。」帝皆納之。擢大理丞,進本寺少卿。

十五年遷太僕卿。家彥向言馬政,帝下兵部檄陜西督撫,未能行。至是,四疏言馬耗之故,請行官牧及金牌差發遺制。且言:「課馬改折,舊增至二十四萬兩,已重困。楊嗣昌不恤民,復增三十七萬,致舊額反逋,不可不釐正。」帝手其疏,語執政曰:「家彥奏皆善。」敕議行。然軍興方亟,不能盡舉也。

頃之,擢戶部右侍郎。都城被兵,命協理戎政。即日登陴,閱視內外城十六門。雪夜,攜一燈,步巡城堞,人無知者。翊日校勤惰,將士皆服,爭自勵。初,分守阜成門,後移安定門,寢處城樓者半歲。解嚴,賜宴午門,增秩一等。

十七年二月,廷推戶部尚書。帝曰:「戎政非家彥不可。」特留任。賊逼京師,襄城伯李國禎督京營,又命中官王德化盡督內外軍。國禎發三大營軍城外,守陴益少。諸軍既出城,見賊輒降,降卒反攻城,城上人皆其儕,益無固志。廷臣分門守,家彥守安定門。號令進止由中官,沮諸臣毋得登城,又縋叛監杜勳上,與密約而去。帝手敕兵部尚書張縉彥登城察視,家彥從,中官猶固拒,示之手敕,問勳安在,曰:「去矣。」秦、晉二王欲上城,家彥曰:「二王降賊,即賊也。賊安得上!」頓足哭。偕縉彥詣宮門請見,不得入。黎明,城陷,家彥投城下,不死,自縊於民舍,遭賊焚,殘其一臂,僕收其餘體焉。贈太子太保、兵部尚書,謚忠端。本朝賜謚忠毅。

孟兆祥,字允吉,山西澤州人也。世籍交河,舉於鄉,九赴會試。天啟二年始擢第,除大理左評事。

崇禎初,遷吏部稽勛主事,歷文選員外郎。門生謁選請善地,兆祥正色拒之,其人悚然退。進稽勳郎中,歷考功。忤權要,貶行人司副,稍遷光祿丞,進少卿。歷左通政、太僕卿,旋進通政使,拜刑部右侍郎。

賊薄都城,兆祥分守正陽門。襄城伯李國禎統京營軍,稽月餉不予,士無固志。城陷,兆祥曰:「社稷已覆,吾將安之!」自經於門下。

長子章明,字綱宜,甫成進士,兆祥揮之曰:「我死,汝可去。」對曰:「君父大節也,君亡父死,我何生為!」乃投繯於父之側。兆祥妻呂,章明妻王相向哭,既而曰:「彼父子死忠矣,我二人獨不能死乎!」皆自縊。兆祥贈刑部尚書,謚忠貞,章明河南道御史,謚節愍。本朝賜兆祥謚忠靖,章明貞孝。

施邦曜,字爾韜,餘姚人。萬歷四十一年進士。不樂為吏,改順天武學教授,歷國子博士、工部營繕主事,進員外郎。魏忠賢興三殿工,諸曹郎奔走其門,邦曜不往。忠賢欲困之,使拆北堂,期五日,適大風拔屋,免譙責。又使作獸吻,仿嘉靖間製,莫考。夢神告之,發地得吻,嘉靖舊物也,忠賢不能難。

遷屯田郎中,稍遷漳州知府,盡知屬縣奸盜主名,每發輒得,闔郡驚為神。盜劉香、李魁奇橫海上,邦曜縶香母誘之,香就擒。魁奇援鄭芝龍事請撫,邦曜言於巡撫鄒維璉討平之。遷福建副使、左參政、四川按察使、福建左布政使,並有聲。

或餽之朱墨竹者,姊子在旁請受之。曰:「不可。我受之,即彼得以乘間而嘗我,我則示之以可欲之門矣。」性好山水。或勸之遊峨嵋,曰:「上官遊覽,動煩屬吏支應,傷小民幾許物力矣。」其潔己愛民如此。

歷兩京光祿寺卿,改通政使。黃道周既謫官,復逮下詔獄。國子生塗仲吉上書訟之,邦曜不為封進,而大署其副封曰:「書不必上,論不可不存。」仲吉劾邦曜,邦曜以副封上。帝見其署語,怒,下仲吉獄,而奪邦曜官。踰年起南京通政使。入都陛見,陳學術、吏治、用兵、財賦四事,帝改容納焉。出都三日,命中使召還,曰:「南京無事,留此為朕效力。」吏部推刑部右侍郎。帝曰:「邦曜清執,可左副都御史。」時崇禎十六年十二月也。

明年,賊薄近郊。邦曜語兵部尚書張縉彥檄天下兵勤王,縉彥慢弗省,邦曜太息而去。城陷,趨長安門,聞帝崩,慟哭曰:「君殉社稷矣,臣子可偷生哉!」即解帶自經。僕救之蘇,恨曰:「是兒誤我!」賊滿衢巷,不得還邸舍,望門求縊,輒為居民所麾。乃命家人市信石雜澆酒,即途中服之,血迸裂而卒。

邦曜少好王守仁之學,以理學、文章、經濟三分其書而讀之,慕義無窮。魯時生者,里同年生也,官庶吉士,歿京師。邦曜手治含斂,以女妻其子。嘗買一婢,命灑掃,至東隅,捧篲凝視而泣。怪問之,曰:「此先人御史宅也。時墮環茲地,不覺悽愴耳。」邦曜即分嫁女資,擇士人歸之。其篤於內行如此。贈太子少保、左都御史,謚忠介。本朝賜謚忠愍。

凌義渠,字駿甫,烏程人。天啟五年進士。除行人。崇禎三年授禮科給事中,知無不言。三河知縣劉夢煒失餉銀三千,責償急,自縊死,有司責其家。義渠言:「以金錢殞命吏,恐天下議朝廷重金,意不在盜也。」帝特原之。宜興、溧陽及遂安、壽昌民亂,焚掠巨室。義渠言:「魏羽林軍焚領軍張彞第,高歡以為天下事可知,日者告密漸啟,籓國悍宗入京越奏,里閭小故叫閽聲冤,僕豎侮家長,下吏箝上官,市儈持縉紳,此《春秋》所謂六逆也。天下所以治,恃上下之分。防維決裂,即九重安所藉以提挈萬靈哉!」義渠與溫體仁同里,無所附麗。給事中劉含輝劾體仁擬旨失當,被貶二秩。義渠言:「諫官不得規執政失,而委申飭權於部院,反得制言路。大臣以攬權為奉旨,小臣以結舌為盡職,將貽國家無窮憂。」兵部尚書張鳳翼敘廢將陳狀猷功,為給事中劉昌所駁,昌反被斥。義渠言:「今上下盡相蒙,疆埸欺蔽為甚。官方盡濫徇,武弁倖功為甚。中樞不職,捨其大,摘其細,已足為言者羞。辨疏一入,調用隨之。自今奸弊叢生,功罪倒置,言者將杜口。」不納。

三遷兵科都給事中。東江自毛文龍後,叛者接踵。義渠言:「東島孤懸海外,轉餉艱,向仰給朝鮮。今路阻絕不得食,內潰可慮。」居無何,眾果潰,挾帥求撫。義渠言:「請陽撫陰剿,同惡必相戕。」及命新帥出海,義渠言:「殲渠散黨宜速,速則可圖功,遲則更生他釁。」後其語皆驗。

義渠居諫垣九年,建白多。吏科給事中劉安行惡之,以年例出義渠福建參政。尋遷按察使,轉山東右布政使,所至有清操。召拜南京光祿寺卿,署應天尹事。

十六年,入為大理卿。明年三月,賊犯都城,有旨召對。趨赴長安門,旦不啟扉。俄傳城陷,還。已,得帝崩問。負牆哀號,首觸柱,血被面。門生勸無死,義渠厲聲曰:「爾當以道義相勖,何姑息為!」揮使去。據几端坐,取生平所好書籍盡焚之,曰:「無使賊手污也。」旦日具緋衣拜闕,作書辭父。已,自繫,奮身絕吭而死,年五十二。贈刑部尚書,謚忠清。本朝賜謚忠介。

贊曰:范景文、倪元璐等皆莊烈帝腹心大臣,所共圖社稷者,國亡與亡,正也。當時壎顏屈節,僥倖以偷生者,多被刑掠以死,身名俱裂,貽詬無窮。而景文等樹義烈於千秋,荷褒揚於興代,名與日月爭光。以彼潔此,其相去得失何如也。


\end{pinyinscope}