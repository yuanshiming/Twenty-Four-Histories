\article{列傳第一百四十一}

\begin{pinyinscope}
王應熊何吾騶張至發孔貞運黃士俊劉宇亮薛國觀袁愷程國祥蔡國用范復粹方逢年張四知等陳演魏藻德李建泰

王應熊,字非熊,巴縣人。萬歷四十一年進士。天啟中,歷官詹事,以憂歸。

崇禎三年,召拜禮部右侍郎。明年冬,帝遣宦官出守邊鎮,應熊上言:「陛下焦勞求治,何一不倚信群臣,乃群臣不肯任勞任怨,致陛下萬不獲已,權遣近侍監理。書之青史,謂有聖明不世出之主,而群工不克仰承,直當愧死。且自神宗以來,士習人心不知職掌何事,有舉《會典》律例告之者,反訝為申、韓刑名。近日諸臣之病,非臨事不擔當之故,乃平時未講求之過也;亦非因循於夙習之故,實愆忘於舊章之過也。」語皆迎帝意,遂蒙眷注。嘗酗酒,詬尚書黃汝良,為給事中馮元飆所劾,汝良為之隱,乃解。五年,進左侍郎,元飆發其貪汙狀,帝不省。

應熊博學多才,熟諳典故,而性谿刻強很,人多畏之。周延儒、溫體仁援以自助,咸與親善。及延儒罷,體仁援益力。六年冬,廷推閣臣,應熊望輕不與,特旨擢禮部尚書兼東閣大學士,與何吾騶並入參機務。命下,朝野胥駭。給事中章正宸劾之曰:「應熊強愎自張,縱橫為習,小才足覆短,小辨足濟貪,今大用,必且芟除異己,報復恩仇,混淆毀譽。況狼籍封靡,淪於市行。願收還成命,別選忠良。且訛言謂左右先容,由他途以進,使天下薰心捷足之徒馳騁而起,為聖德累不小。」帝大怒,下正宸詔獄,削籍歸。有勸應熊為文彥博者,應熊咈然,佯具疏引退,語多憤激。屢為給事中范淑泰、御史吳履中所攻,帝皆不問。

八年正月,流賊陷鳳陽,毀皇陵。巡撫楊一鵬,應熊座主;巡按吳振纓,體仁姻也。二人恐帝震怒,留一鵬、振纓疏未上,俟恢復報同奏之,遂擬旨令撫按戴罪。主事鄭爾說、胡江交章詆應熊、體仁朋比誤國,帝怒謫二人,而給事中何楷、許譽卿、范淑泰,御史張纘曾、吳履中、張肯堂言之不已。淑泰言:「一鵬《恢復疏》以正月二十一日,《核察失事情形疏》以正月二十八日,天下有未失事先恢復者哉?應熊改填月日,欺誑之罪難辭。」且劾其他受賄事。帝顧應熊厚,皆不聽,而鐫楷、纘曾秩,慰諭應熊。應熊亦屢疏辯,謂:「座主門生,誼不容薄,敢辭比之名。票擬實臣起草,敢辭誤之罪。」楷益憤,屢疏糾之,最後復疏言:「故事,奏章非發抄,外人無由聞;非奉旨,邸報不許抄傳。臣疏六月初十日上,十四日始奉明旨,應熊乃於十三日奏辯,旨尚未下,應熊何由知?臣不解者一。且旨下必由六科抄發。臣疏十四日下,而百戶趙光修先送錦衣堂上官,則疏可不由科抄矣。臣不解者二。」應熊始懼,具疏引罪。帝下其家人及直日中書七人於獄。獄具,家人戍邊,中書貶二秩。應熊乃屢疏乞休去,乘傳賜道里費,行人護行。帝亦知應熊不協人望,特己所拔擢,不欲以人言去也。

十二年遣官存問。其弟應熙橫於鄉,鄉人詣闕擊登聞鼓,列狀至四百八十餘條,贓一百七十餘萬,詞連應熊。詔下撫按勘究。會應熊復召,事得解。

時延儒再相,患言者攻己,獨念應熊剛很,可藉以制之,力言於帝。十五年冬,遣行人召應熊。明年六月,應熊未至,延儒已罷歸。給事中龔鼎孳密疏言:「陛下召應熊,必因其秉國之日,眾口交攻,以為孤立無黨;孰知其同年密契,肺腑深聯,恃延儒在也。臣去年入都,聞應熊賄延儒為再召計。延儒對眾大言,至尊欲起巴縣。巴縣者,應熊也。未幾,召命果下。以政本重地,私相援引,是延儒雖去猶未去,天下事何堪再誤!」帝得疏心動,留未下。已而延儒被逮,不即赴,俟應熊至,始尾之行。一日,帝顧中官曰:「延儒何久不至?」對曰:「需王應熊先入耳。」帝益疑之。九月,應熊至,宿朝房。請入對,不許;請歸田,許之,乃慚沮而返。

十七年三月,京師陷。五月,福王立於南京。八月,張獻忠陷四川。乃改應熊兵部尚書兼文淵閣大學士,總督川、湖、雲、貴軍務,專辦川寇。時川中諸郡,惟遵義未下,應熊入守之,縞素誓師,開幕府,傳檄討賊。明年奏上方略,請敕川陜、湖貴兩總督,鄖陽、湖廣、貴州、雲南四巡撫出師合討,并劾四川巡撫馬體乾縱兵淫掠,革職提問。命未達而南都亡,體乾居職如故。已而獻忠死,諸將楊展等各據州縣自雄,應熊不能制。其部將曾英最有功,復重慶,屢破賊兵。王祥亦出師綦江相犄角。祥才武不及英,而應熊委任過之。又明年十月,獻忠餘黨孫可望、李定國等南走重慶,英戰歿。可望襲破遵義,應熊遁入永寧山中,旋卒於畢節衛。一子陽禧,死於兵,竟無後。

何吾騶,香山人。萬曆四十七年進士。由庶吉士歷官少詹事。崇禎五年,擢禮部右侍郎。六年十一月,加尚書,同王應熊入閣。溫體仁久柄政,欲斥給事中許譽卿。已擬旨,文震孟爭之,吾騶亦助為言。體仁訐奏,帝奪震孟官,兼罷吾騶。詳見《震孟傳》。

居久之,唐王自立於福州,召為首輔,與鄭芝龍議事輒相牴牾。閩疆既失,踉蹌回廣州。永明王以原官召之,為給事中金堡、大理寺少卿趙昱等所攻。引疾辭去,卒於家。

張至發,淄川人。萬歷二十九年進士。歷知玉田、遵化。行取,授禮部主事,改御史。時齊、楚、浙三黨方熾,至發,齊黨也,上疏陳內降之弊。因言:「陛下惡結黨,而秉揆者先不能超然門戶外。頃讀科臣疏云:『日來慰諭輔臣溫旨,輔臣與司禮自相參定,方聽御批。』果若人言,天下事尚可問耶?」語皆刺葉向高,帝不報。時言官爭排東林,戶部郎中李朴不平,抗疏爭。至發遂劾朴背公死黨,誑語欺君,帝亦不報。

尋出按河南。福王之籓洛陽,中使相望於道。至發以禮裁之,無敢橫。宗祿不給,為置義田,以贍貧者。四十三年,豫省饑,請留餉備振,又請改折漕糧,皆報聞。還朝,引病歸。

天啟元年,進大理寺丞。三年請終養。魏忠賢黨薦之,矯旨令吏部擢用,至發方養親不出。

崇禎五年,起順天府丞,進光祿卿。精核積弊,多所釐正,遂受帝知。八年春,遷刑部右侍郎。六月,帝將增置閣臣,以翰林不習世務,思用他官參之,召廷臣數十人,各授一疏,令擬旨。遂擢至發禮部左侍郎兼東閣大學士,與文震孟同入直。自世宗朝許贊後,外僚入閣,自至發始。

時溫體仁為首輔,錢士升、王應熊、何吾騶次之。越二年,體仁輩盡去,至發遂為首輔。萬曆中,申時行、王錫爵先後柄政,大旨相紹述,謂之「傳衣缽」。至發代體仁,一切守其所為,而才智機變遜之,以位次居首,非帝之所注也。嘗簡東宮講官,擯黃道周,為給事中馮元飆所刺。至發怒,兩疏詆道周,而極頌體仁孤執不欺,復為編修吳偉業所劾。講官項煜論至發把持考選,庇兒女姻任濬而抑成勇。至發上章辯,帝遂逐煜去。

內閣中書黃應恩悍戾,體仁、至發輩倚任之,恃勢恣橫。及為正字,不當復為東宮侍書,恐帝與太子開講同日也。至發不諳故事,令兼之。應恩不能兼,講官撰講義送應恩繕錄,拒不納。檢討楊士聰論之,至發揭寢其疏,士聰復上書閣中,極論其事,至發終庇之。會復故總督楊鶴官,許給誥命,應恩當撰文。因其子嗣昌得君,力為洗雪。忤旨,將加罪,至發擬公揭救。同官孔貞運、傅冠曰:「曩許士柔事,吾輩未嘗救,獨救應恩何也?」至發咈然曰:「公等不救,我自救之。」連上三揭。帝不聽,特降諭削應恩籍,嗣昌疏救,亦不聽。無何,大理寺副曹荃發應恩賕請事,詞連至發。至發憤,連疏請勘。帝雖優旨褒答,卒下應恩獄。至發乃具疏,自謂當去者三,而未嘗引疾,忽得旨回籍調理,時人傳笑,以為遵旨患病云。

至發頗清彊。起自外吏,諸翰林多不服,又始終惡異己,不能虛公延攬。帝亦惡其洩漏機密,聽之去。且不遣行人護行,但令乘傳,賜道里費六十金、彩幣二表裏,視首輔去國彞典,僅得半焉。既歸,捐貲改建淄城,賜敕優獎。俄以徽號禮成,遣官存問。十四年夏,帝思用舊臣,特敕召周延儒、賀逢聖及至發,獨至發四疏辭。明年七月病歿。先屢加太子太傅、禮部尚書、文淵閣大學士。及卒,贈少保,祭葬,廕子如制。

代至發為首輔者,孔貞運。代貞運者,劉宇亮。貞運,句容人,至聖十三代孫也。萬曆四十七年以殿試第二人授編修。天啟中,充經筵展書官,纂修《兩朝實錄》。莊烈帝嗣位,貞運進講《皇明寶訓》,稱述祖宗勤政講學事,帝嘉納之。

崇禎元年,擢國子監祭酒,尋進少詹,仍管監事。二年正月,帝臨雍,貞運進講《書經》。唐貞觀時,祭酒孔穎達講《孝經》,有釋奠頌。孔氏子孫以國師進講,至貞運乃再見。帝以聖裔故,從優賜一品服。冬十月,畿輔被兵,條上御敵城守應援數策。尋以艱歸。六年服闋,起南京禮部侍郎。越二年,遷吏部左侍郎。

九年六月,與賀逢聖、黃士俊並入內閣。時體仁當國,欲重治復社,值其在告,貞運從寬結之。體仁怒語人曰:「句容亦聽人提索矣。」自是不敢有所建白。及至發去位,貞運代之,乃揭救鄭三俊、錢謙益,俱從寬擬。帝親定考選諸臣,下輔臣再閱,貞運及薛國觀有所更。迨命下,閣擬悉不從,而帝以所擇十八卷下部議行。適新御史郭景昌等謁貞運於朝房,貞運言所下諸卷,說多難行。景昌與辯,退即上疏劾之。帝雖奪景昌俸,貞運卒引歸。十七年五月,莊烈帝哀詔至,貞運哭臨,慟絕不能起。舁歸,得疾遽卒。

黃士俊,順德人。萬曆三十五年殿試第一。授修撰,歷官禮部尚書。崇禎九年入閣,累加少傅,予告歸。父母俱在堂,錦衣侍養,人以為榮。唐王以原官召,未赴。後相永明王,耄不能決事,數為臺省論列。辭歸而卒。

劉宇亮,綿竹人。萬曆四十七年進士。屢遷吏部右侍郎。崇禎十年八月,擢禮部尚書,與傅冠、薛國觀同入閣。宇亮短小精悍,善擊劍。居翰林,常與家僮角逐為樂。性不嗜書,館中纂修、直講、典試諸事,皆不得與。座主錢士升為之援,又力排同鄉王應熊,張己聲譽,竟獲大用。明年六月,貞運罷歸,遂代為首輔。其冬,都城戒嚴,命閱視三大營及勇衛營軍士,兩日而畢。又閱視內城九門,外城七門,皆茍且卒事。

時大清兵深入,帝憂甚,宇亮自請督察軍情。帝喜,即革總督盧象升任,命宇亮往代。字亮請督察,而帝忽改為總督,大懼,與國觀及楊嗣昌謀,且具疏自言。乃留象昇,而宇亮仍往督察,各鎮勤王兵皆屬焉。甫抵保定,聞象升戰歿,過安平,偵者報大清兵將至,相顧無人色,急趨晉州避之,知州陳弘緒閉門不納,士民亦歃血誓不延一兵。宇亮大怒,傳令箭:亟納師,否則軍法從事。弘緒亦傳語曰:「督師之來以禦敵也,今敵且至,奈何避之?芻糧不繼,責有司,欲入城,不敢聞命。」宇亮乃馳疏劾之,有旨逮治。州民詣闕訟冤,願以身代者千計,弘緒得鐫級調用。帝自是疑宇亮不任事,徒擾民矣。

明年正月至天津。憤諸將退避,疏論之,因及總兵劉光祚逗遛狀。國觀方冀為首輔,與嗣昌謀傾宇亮,遽擬旨軍前斬光祚。比旨下,光祚適有武清之捷,宇亮乃繫光祚於獄,而具疏乞宥,繼上武清捷音。國觀乃擬嚴旨,責以前後矛盾,下九卿科道議。僉謂宇亮玩弄國憲,大不敬。宇亮疏辯,部議落職閒住,給事中陳啟新、沈迅復重劾之,改擬削籍。帝令戴罪圖功,事平再議。宇亮竟以此去位,而國觀代為首輔矣。已而定失事者五案,宇亮終免議。久之,卒於家。

薛國觀,韓城人。萬曆四十七年進士。授萊州推官。天啟四年,擢戶部給事中,數有建白。魏忠賢擅權,朝士爭擊東林。國觀所劾御史游士任、操江都御史熊明遇、保定巡撫張鳳翔、兵部侍郎蕭近高、刑部尚書喬允升,皆東林也。尋遷兵科右給事中,於疆事亦多所論奏。忠賢遣內臣出鎮,偕同官疏爭。七年,再遷刑科都給事中。

崇禎改元,忠賢遺黨有欲用王化貞,寬高,出胡嘉棟者,國觀力持不可。奉命祭北鎮醫無閭,還言關內外營伍虛耗、將吏侵克之弊,因薦大將滿桂才。帝褒以忠讜,令指將吏侵剋者名,列上副將王應暉等六人,詔俱屬之吏。陜西盜起,偕鄉人仕於朝者,請設防速剿,并追論故巡撫喬應甲納賄縱盜罪。削應甲籍,籍其贓。國觀先附忠賢,至是大治忠賢黨,為南京御史袁耀然所劾。國觀懼,且虞掛察典,思所以撓之,乃劾吏科都給事中沈惟炳、兵科給事中許譽卿,言:「兩人主盟東林,與瞿式耜掌握枚卜。文華召對,陛下惡章允儒妄言,嚴旨處分。譽卿乃持一疏授惟炳,使同官劉斯珣邀臣列名,臣拒不應,遂使耀然劾臣。臣自立有品,不入東林,遂罹其害。今朝局惟論東林異同向背,借崔、魏為題,報仇傾陷。今又把持京察,而式耜以被斥之人,久居郭外,遙制察典,舉朝無敢言。」末詆耀然賄劉鴻訓得御史。帝雖以撓察典責之,國觀卒免察。然清議不容,旋以終養去。

三年秋,用御史陳其猷薦,起兵科都給事中。遭母憂,服闋,起禮科都給事中,遷太常少卿。九年,擢左僉都御史。明年八月,拜禮部左侍郎兼東閣大學士,入參機務。國觀為人陰鷙谿刻,不學少文。溫體仁因其素仇東林,密薦於帝,遂超擢大用之。

十一年六月,進禮部尚書。其冬,首輔劉宇亮出督師,國觀與楊嗣昌比,手冓罷宇亮。明年二月代其位。敘剿寇功,加太子太保、戶部尚書,進文淵閣;敘城守功,加少保、吏部尚書,進武英殿。

先為首輔者,體仁最當帝意,居位久。及張至發、孔貞運、劉宇亮繼之,皆非帝意所屬,故旋罷去。國觀得志,一踵體仁所為,導帝以深刻,而才智彌不及,操守亦弗如。帝初頗信響之,久而覺其奸,遂及於禍。

始帝燕見國觀,語及朝士貪婪。國觀對曰:「使廠衛得人,安敢如是。」東廠太監王德化在側,汗流沾背,於是專察其陰事。國觀任中書王陛彥,而惡中書周國興、楊餘洪,以漏詔旨、招權利劾之,並下詔獄。兩人老矣,斃廷杖下,其家人密緝國觀通賄事,報東廠。而國觀前匿史褷所寄銀,周、楊兩家又誘褷蒼頭首告。由是諸事悉上聞,帝意漸移。

史褷者,清苑人。為御史無行,善結納中官,為王永光死黨。巡按淮、揚,括庫中贓罰銀十餘萬入己橐。攝巡鹽,又掩取前官張錫命貯庫銀二十餘萬。及以少卿家居,檢討楊士聰劾吏部尚書田唯嘉納周汝弼金八千推延綏巡撫,褷居間,并發褷盜鹽課事。褷得旨自陳,遂訐士聰,而鹽課則請敕淮、揚監督中官楊顯名核奏。俄而錫命子沆訐褷,給事中劉焜芳復劾褷侵盜有據。又嘗勒富人於承祖萬金,事發,則遣家人齎重貲謀於黠吏,圖改舊籍。帝乃怒,褫褷職,褷急攜數萬金入都,主國觀邸。謀既定,出疏攻焜芳及其弟炳芳、煒芳。閣臣多徇褷,擬嚴旨,帝不聽,止奪焜芳官候訊。及顯名核疏上,力為褷解,而不能諱者六萬金;褷下獄。會有兵事,獄久不結,瘐死。都人籍籍謂褷所攜貲盡為國觀有,家人證之,事大著。國觀猶力辨褷贓為黨人構陷,帝不聽。

帝初憂國用不足,國觀請借助,言:「在外群僚,臣等任之;在內戚畹,非獨斷不可。」因以武清侯李國瑞為言。國瑞者,孝定太后兄孫,帝曾祖母家也。國瑞薄庶兄國臣,國臣憤,詭言「父貲四十萬,臣當得其半,今請助國為軍貲」。帝初未允,因國觀言,欲盡借所言四十萬者,不應則勒期嚴追。或教國瑞匿貲勿獻,拆毀居第,陳什器通衢鬻之,示無所有。嘉定伯周奎與有連,代為請。帝怒,奪國瑞爵,國瑞悸死。有司追不已,戚畹皆自危。因皇五子病,交通宦官宮妾,倡言孝定太后已為九蓮菩薩,空中責帝薄外家,諸皇子盡當殀,降神於皇五子。俄皇子卒,帝大恐,急封國瑞七歲兒存善為侯,盡還所納金銀,而追恨國觀,待隙而發。

國觀素惡行人吳昌時。及考選,昌時虞國觀抑己,因其門人以求見。國觀偽與交歡,擬第一,當得吏科。迨命下,乃得禮部主事。昌時大恨,以為賣己,與所善東廠理刑吳道正謀,發丁憂侍郎蔡奕琛行賄國觀事。帝聞之,益疑。

十三年六月,楊嗣昌出督師,有所陳奏。帝令擬諭,國觀乃擬旨以進。帝遂發怒,下五府九卿科道議奏。掌都督府魏國公徐允禎、吏部尚書傅永淳等不測帝意,議頗輕,請令致仕或閒住。帝度科道必言之,獨給事中袁愷會議不署名,且疏論永淳徇私狀,而微詆國觀藐肆妒嫉。帝不懌,抵疏於地曰:「成何糾疏!」遂奪國觀職,放之歸,怒猶未已。

國觀出都,重車纍纍,偵事者復以聞。而東廠所遣伺國觀邸者,值陛彥至,執之,得其招搖通賄狀。詞所連及,永淳、奕琛暨通政使李夢辰、刑部主事硃永佑等十一人。命下陛彥詔獄窮治。頃之,愷再疏,盡發國觀納賄諸事,永淳、奕琛與焉。國觀連疏力辨,詆愷受昌時指使,帝不納。

至十月,陛彥獄未成,帝以行賄有據,即命棄市,而遣使逮國觀。國觀遷延久不赴,明年七月入都。令待命外邸,不以屬吏,國觀自謂必不死。八月初八日夕,監刑者至門,猶鼾睡。及聞詔使皆緋衣,蹶然曰:「吾死矣!」倉皇覓小帽不得,取蒼頭帽覆之。宣詔畢,頓首不能出聲,但言「吳昌時殺我」,乃就縊。明日,使者還奏。又明日許收斂,懸梁者兩日矣。輔臣戮死,自世廟夏言後,此再見云。法司坐其贓九千,沒入田六百畝,故宅一區。

國觀險忮,然罪不至死,帝徒以私憤殺之,贓又懸坐,人頗有冤之者。

袁愷,聊城人。既劾罪國觀,後為給事中宋之普所傾,罷去。福王時,起故官,道卒。

程國祥,字仲若,上元人。舉萬歷三十二年進士。歷知確山、光山二縣,有清名。遷南京吏部主事,乞養歸。服闋,起禮部主事。天啟四年,吏部尚書趙南星知其可任,調為己屬,更歷四司。發御史楊玉珂請屬,玉珂被謫,國祥亦引疾歸。其冬,魏忠賢既逐南星,御史張訥劾國祥為南星邪黨,遂除名。

崇禎二年,起稽勳員外郎。遷考功郎中,主外計,時稱公慎。御史龔守忠詆國祥通賄,國祥疏辯。帝褒以清執,下都察院核奏,事得白,守忠坐褫官。尋遷大理右寺丞。歷太常卿、南京通政使,就遷工部侍郎,復調戶部。

九年冬,召拜戶部尚書。楊嗣昌議增餉,國祥不敢違。而是時度支益匱,四方奏報災傷者相繼。國祥多方區畫,亦時有所蠲減,最後建議,借都城賃舍一季租,可得五十萬,帝遂行之。勛戚奄豎悉隱匿不奏,所得僅十三萬,而怨聲載途。然帝由是眷國祥。

十一年六月,帝將增置閣臣,出御中極殿,召廷臣七十餘人親試之。發策言:「年來天災頻仍,今夏旱益甚,金星晝見五旬,四月山西大雪。朝廷腹心耳目臣,務避嫌怨。有司舉劾,情賄關其心。剋期平賊無功,而剿兵難撤。外敵生心,邊餉日絀。民貧既甚,正供猶艱。有司侵削百方,如火益熱。若何處置得宜,禁戢有法,卿等悉心以對。」會天大雨,諸臣面對後,漏已深,終考者止三十七人。顧帝意已前定,特假是為名耳。居數日,改國祥禮部尚書,與楊嗣昌、方逢年、蔡國用、范復粹俱兼東閣大學士,入參機務。時劉宇亮為首輔,傅冠、薛國觀次之,又驟增國祥等五人。國觀、嗣昌最用事,國祥委蛇其間,自守而已。明年四月召對,無一言。帝傳諭責國祥緘默,大負委任,國祥遂乞休去。

國祥始受業於焦竑,歷任卿相,布衣蔬食,不改儒素。與其子上俱撰有詩集。國祥歿後,家貧,不能舉火。上營葬畢,感疾卒,無嗣。

蔡國用,金谿人。萬曆三十八年進士。由中書舍人擢御史。天啟五年陳時政六事,詆葉向高、趙南星,而薦亓詩教、趙興邦、邵輔忠、姚宗文等七人,魏忠賢喜,矯旨褒納。尋忤璫意,勒令閒住。

崇禎元年起故官,屢遷工部右侍郎。督修都城,需石甚急,不克辦。國用建議取牙石用之。牙石者,舊列崇文、宣武兩街,備駕出除道者也。帝閱城,嘉其功,遂欲大用。十一年六月,廷推閣臣,國用望輕,不獲與,特旨擢禮部尚書,入閣辦事。累加少保,改吏部尚書、武英殿。十三年六月卒於官,贈太保,謚文恪。國用居位清謹,與同列張四知皆庸才,碌碌無所見。

范復粹,黃縣人。萬曆四十七年進士。除開封府推官。崇禎元年為御史。廷議移毛文龍內地,復粹言:「海外億萬生靈誰非赤子,倘棲身無所,必各據一島為盜,後患方深。」又言:「袁崇煥功在全遼,而尚寶卿董懋中詆為逆黨所庇,持論狂謬。」懋中遂落職,文龍亦不果移。

巡按江西,請禁有司害民六事。時大釐郵傳積弊,減削過甚,反累民,復粹極陳不便。丁艱歸。服闋,還朝,出按陜西。陳治標治本之策:以任將、設防、留餉為治標;廣屯、蠲賦、招撫為治本。帝褒納之。廷議有司督賦缺額,兼罪撫按,復粹力言不可。

由大理右寺丞進左少卿。居無何,超拜禮部左侍郎兼東閣大學士。時同命者五人,翰林惟方逢年,餘皆外僚,而復粹由少卿,尤屬異數。蓋帝欲閣臣通知六部事,故每部簡一人:首輔劉宇亮由吏部,國祥以戶,逢年以禮,嗣昌以兵,國用以工,刑部無人,復粹以大理代之。累加少保,進吏部尚書、武英殿。

十三年六月,國觀罷,復粹為首輔。給事中黃雲師言「宰相須才識度三者」,復粹恚,因自陳三者無一,請罷,溫旨慰留。御史魏景琦劾復粹及張四知學淺才疏,伴食中書,遺譏海內。帝以妄詆下之吏。明年,加少傅兼太子太傅,改建極殿。賊陷洛陽,復粹等引罪乞罷,不允。帝御乾清宮左室,召對廷臣,語及福王被害,泣下。復粹曰:「此乃天數。」帝曰:「雖氣數,亦賴人事挽回。」復粹等不能對。帝疾初愈,大赦天下,命復粹錄囚,自尚書傅宗龍以下,多所減免。是年五月致仕。國變後,卒於家。

方逢年,遂安人。萬曆四十四年進士。天啟四年,以編修典湖廣試,發策有「巨璫大蠹」語,且云「宇內豈無人焉?有薄士大夫而覓皋、夔、稷、契於黃衣閹尹之流者」。魏忠賢見之,怒,貶三秩調外。御史徐復陽希指劾之,削籍為民。

崇禎初,起原官,累遷禮部侍郎。十一年詔廷臣舉邊才,逢年以汪喬年應。未幾,擢禮部尚書,入閣輔政。其冬,刑科奏摘參未完疏,逢年以犯贓私者,人亡產絕,親戚坐累,幾同瓜蔓,遂輕擬以上。而帝意欲罪刑部尚書劉之鳳,責逢年疏忽。逢年引罪,即罷歸。

福王時,復原官,不召。魯王三召之,用其議,定稱魯監國。紹興破,王航海,逢年追不及,與方國安等降於我大清。已而以蠟丸書通閩,事洩被誅。

張四知者,費縣人。天啟二年進士。由庶吉士授檢討。崇禎中,歷官禮部右侍郎。貌寢甚,嘗患惡瘍。十一年六月,廷推閣臣忽及之。給事中張淳劾其為祭酒時貪污狀,四知憤,帝前力辨,言己孤立,為廷臣所嫉。帝意頗動,薛國觀因力援之。明年五月與姚明恭、魏照乘俱拜禮部尚書兼東閣大學士。

明恭,蘄水人。出趙興邦門,公論素不予。崇禎十一年,由詹事遷禮部侍郎,教習庶吉士。給事中耿始然劾其與副都御史袁鯨比而為奸利,帝不聽。明年遂柄用。

照乘,滑人。天啟時,為吏部都給事中。崇禎十一年,歷官兵部侍郎。明年,國觀引入閣。

三人者,皆庸劣充位而已。四知加太子太保,進吏部尚書、武英殿。明恭加太子太保,進戶部尚書、文淵閣。照乘加太子少傅,進戶部尚書、文淵閣。帝自即位,務抑言官,不欲以其言斥免大臣。彈章愈多,位愈固。四知秉政四載,為給事中馬嘉植,御史鄭崑貞、曹溶等所劾,帝皆不納,十五年六月始致仕。照乘亦四載,御史楊仁願、徐殿臣、劉之渤相繼論劾,引疾去。明恭甫一載,鄉人詣闕訟之,請告歸。後四知降於我大清。

陳演,井研人。祖效,萬曆間以御史監征倭軍,卒於朝鮮,贈光祿卿。演登天啟二年進士,改庶吉士,授編修。崇禎時,歷官少詹事,掌翰林院,直講筵。十三年正月,擢禮部右侍郎,協理詹事府。

演庸才寡學,工結納。初入館,即與內侍通。莊烈帝簡用閣臣,每親發策,以所條對覘能否。其年四月,中官探得帝所欲問數事,密授演,條對獨稱旨,即拜禮部左侍郎兼東閣大學士,與謝升同入閣。明年,進禮部尚書,改文淵閣。十五年,以山東平盜功加太子少保,改戶部尚書、武英殿。被劾乞罷,優旨慰留。明年五月,周延儒去位,遂為首輔。尋以城守功,加太子太保。十七年正月考滿,加少保,改吏部尚書、建極殿。踰月罷政。再踰月,都城陷,遂及於難。

演為人既庸且刻。惡副都御史房可壯、河南道張煊不受屬,因會推閣臣讒於帝,可壯等六人俱下吏。王應熊召至,旋放還,演有力焉。

自延儒罷後,帝最倚信演。臺省附延儒者,盡趨演門。當是時,國勢累卵,中外舉知其不支。演無所籌畫,顧以賄聞。及李自成陷陜西,逼山西,廷議撤寧遠吳三桂兵入守山海關,策應京師。帝意亦然之,演持不可。後帝決計行之,三桂始用海船渡遼民入關,往返者再,而賊已陷宣、大矣。演懼不自安,引疾求罷。詔許之,賜道里費五十金,彩幣四表裏,乘傳行。

演既謝事,薊遼總督王永吉上疏力詆其罪,請置之典刑,給事中汪惟效、孫承澤亦極論之。演入辭,謂佐理無狀,罪當死。帝怒曰:「汝一死不足蔽辜!」叱之去。演貲多,不能遽行。賊陷京師,與魏藻德等俱被執,繫賊將劉宗敏營中。其日獻銀四萬,賊喜,不加刑。四月八日,已得釋。十二日,自成將東禦三桂,慮諸大臣為後患,盡殺之。演亦遇害。

魏藻德,順天通州人。崇禎十三年舉進士。既殿試,帝思得異才,復召四十八人於文華殿,問:「今日內外交訌,何以報仇雪恥?」藻德即以「知恥」對,又自敘十一年守通州功。帝善之,擢置第一,授修撰。

十五年,都城戒嚴,疏陳兵事。明年三月,召對稱旨。藻德有口才。帝以己所親擢,且意其有抱負,五月,驟擢禮部右侍郎兼東閣大學士,入閣輔政。藻德力辭部銜,乃改少詹事。正統末年,兵事孔棘,彭時以殿試第一人,踰年即入閣,然仍故官修撰,未有超拜大學士者。陳演見帝遇之厚,曲相比附。八月,補行會試引為副總裁,越蔣德璟、黃景昉而用之。藻德居位,一無建白,但倡議令百官捐助而已。十七年二月,詔加兵部尚書兼工部尚書、文淵閣大學士,總督河道、屯田、練兵諸事,駐天津,而命方岳貢駐濟寧,蓋欲出太子南京,俾先清道路也。有言百官不可令出,出即潛遁者,遂止不行。

及演罷,藻德遂為首輔。同事者李建泰、方岳貢、范景文、邱瑜,皆新入政府,莫能補救。至三月,都城陷,景文死之,藻德、岳貢、瑜並被執,幽劉宗敏所。賊下令勒內閣十萬金,京卿、錦衣七萬,或五三萬,給事、御史、吏部、翰林五萬至一萬有差,部曹數千,勛戚無定數。藻德輸萬金,賊以為少,酷刑五日夜,腦裂而死。復逮其子追徵,訴言:「家已罄盡。父在,猶可丐諸門生故舊。今已死,復何所貸?」賊揮刃斬之。

李建泰,曲沃人。天啟五年進士。歷官國子祭酒,頗著聲望。崇禎十六年五月,擢吏部右侍郎。十一月,以本官兼東閣大學士,與方岳貢並命。疏陳時政切要十事,帝皆允行。

明年正月,李自成逼山西。建泰慮鄉邦被禍,而家富於貲,可藉以佐軍,毅然有滅賊志,常與同官言之。會平陽陷,帝臨朝歎曰:「朕非亡國之君,事事皆亡國之象。祖宗櫛風沐雨之天下,一朝失之,何面目見於地下!朕願督師親決一戰,身死沙場無所恨,但死不瞑目耳!」語畢痛哭。陳演、蔣德璟諸輔臣請代,俱不許。建泰頓首曰:「臣家曲沃,願出私財餉軍,不煩官帑,請提師以西。」帝大喜,慰勞再三,曰:「卿若行,朕仿古推轂禮。」建泰退,即請復故御史衛楨固官;授進士凌駉職方主事,並監軍;參將郭中傑為副總兵,領中軍事;薦進士石釭聯絡延、寧、甘、固義士,討賊立功。帝俱從之。加建泰兵部尚書,賜尚方劍,便宜從事。

二十六日,行遣將禮。駙馬都尉萬煒以特牲告太廟。日將午,帝御正陽門樓,衛士東西列,自午門抵城外,旌旗甲仗甚設。內閣五府六部都察院掌印官及京營文武大臣侍立,鴻臚贊禮,御史糾儀。建泰前致辭,帝獎勞有加,賜之宴。御席居中,諸臣陪侍,酒七行,帝手金卮親酌建泰者三,即以賜之,乃出手敕曰「代朕親征」。宴畢,內臣為披紅簪花,用鼓樂導尚方劍而出。建泰頓首謝,且辭行,帝目送之。行數里,所乘肩輿忽折,眾以為不祥。

建泰以宰輔督師,兵食並絀,所攜止五百人。甫出都,聞曲沃已破,家貲盡沒,驚怛而病。日行三十里,士卒多道亡。至定興,城門閉不納。留三日,攻破之,笞其長吏。抵保定,賊鋒已逼,不敢前,入屯城中。已而城陷,知府何復、鄉官張羅彥等並死之。建泰自刎不殊,為賊將劉方亮所執,送賊所。

賊既敗,大清召為內院大學士。未幾,罷歸。姜瓖反大同,建泰遙應之。兵敗被擒,伏誅。

贊曰:天下治亂,系於宰輔。自溫體仁導帝以刻深,治尚操切,由是接踵一跡。應熊剛很,至發險忮。國觀陰鷙,一效體仁之所為,而國家之元氣已索然殆盡矣。至於演、藻德之徒,機智弗如,而庸庸益甚,禍中於國,旋及其身,悲夫!


\end{pinyinscope}