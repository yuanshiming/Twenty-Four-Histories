\article{選舉志}

選舉之法,大略有四:曰學校,曰科目,曰薦舉,曰銓選。學校以教育之,科目以登進之,薦舉以旁招之,銓選以布列之,天下人才盡於是矣。明制,科目為盛,卿相皆由此出,學校則儲才以應科目者也。其徑由學校通籍者,亦科目之亞也,外此則雜流矣。然進士、舉貢、雜流三途並用,雖有畸重,無偏廢也。薦舉盛於國初,後因專用科目而罷。銓選則入官之始,舍此蔑由焉。是四者釐然具載其本末,而二百七十年間取士得失之故可睹已。

科舉必由學校,而學校起家,可不由科舉。學校有二:曰國學,曰府、州、縣學。府、州、縣學諸生入國學者,乃可得官,不入者不能得也。入國學者,通謂之監生。舉人曰舉監,生員曰貢監,品官子弟曰廕監,捐貲曰例監。同一貢監也,有歲貢,有選貢,有恩貢,有納貢。同一廕監也,有官生,有恩生。

國子學之設自明初乙巳始。洪武元年令品官子弟及民俊秀通文義者,並充學生。選國琦、王璞等十餘人,侍太子讀書禁中。入對謹身殿,姿狀明秀,應對詳雅。太祖喜,因厚賜之。天下既定,詔擇府、州、縣學諸生入國子學。又擇年少舉人趙惟一等及貢生董昶等入學讀書,賜以衣帳,命於諸司先習吏事,謂之歷事監生。取其中尤英敏者李擴等入文華、武英堂說書,謂之小秀才。其才學優贍、聰明俊偉之士,使之博極羣書,講明道德經濟之學,以期大用,謂之老秀才。初,改應天府學為國子學,後改建於雞鳴山下。既而改學為監,設祭酒、司業及監丞、博士、助教、學正、學錄、典籍、掌饌、典簿等官。分六堂以館諸生,曰率性、修道、誠心、正義、崇志、廣業。學旁以宿諸生,謂之號房。厚給稟餼,歲時賜布帛文綺、襲衣巾靴。正旦元宵諸令節,俱賞節錢。孝慈皇后積糧監中,置紅倉二十餘舍,養諸生之妻子。歷事生未娶者,賜錢婚聘,及女衣二襲,月米二石。諸生在京師歲久,父母存,或父母亡而大父母、伯叔父母存,皆遣歸省,人賜衣一襲,鈔五錠,為道里費。其優恤之如此。而其教之之法,每旦,祭酒、司業坐堂上,屬官自監丞以下,首領則典簿,以次序立。諸生揖畢,質問經史,拱立聽命。惟朔望給假,餘日升堂會饌,乃會講、復講、背書,輪課以為常。所習自《四子》本經外,兼及劉向說苑及律令、書、數、《御製大誥》。每月試經、書義各一道,詔、誥、表、策論、判、內科二道。每日習書二百餘字,以二王、智永、歐、虞、顏、柳諸帖為法。每班選一人充齋長,督諸生工課。衣冠、步履、飲食,必嚴飭中節。夜必宿監,有故而出必告本班教官,令齋長帥之以白祭酒。監丞置集衍簿,有不遵者書之,再三犯者決責,四犯者至發遣安置。其學規條目,屢次更定,寬嚴得其中。堂宇宿舍,飲饌澡浴,俱有禁例。省親、畢姻回籍,限期以道里遠近為差。違限者謫選遠方典史,有罰充吏者。司教之官,必選耆宿。宋訥、吳顒等由儒士擢祭酒,訥尤推名師。歷科進士多出太學,而戊辰任亨泰廷對第一,太祖召訥褒賞,撰題名記,立石監門。辛未許觀亦如之。進士題名碑由此相繼不絕。每歲天下按察司選生員年二十以上、厚重端秀者,送監考留。會試下第舉人,入監卒業。又因諫官關賢奏,設為定例。府、州、縣學歲貢生員各一人,翰林考試經、書義各一道,判語一條,中式者一等入國子監,二等達中都,不中者遣還,提調教官罰停廩祿。於是直省諸士子雲集輦下。雲南、四川皆有士官生,日本、琉球、暹羅諸國亦皆有官生入監讀書,輒加厚賜,並給其從人。永、宣間,先後絡繹。至成化、正德時,琉球生猶有至者。中都之置國學也,自洪武八年。至二十六年乃革,以其師生併入京師。永樂元年始設北京國子監。十八年遷都,乃以京師國子監為南京國子監,而太學生有南北監之分矣。

太祖慮武臣子弟但習武事,鮮知問學,命大都督府選入國學,其在鳳陽者即肄業於中都。命韓國公李善長等考定教官、生員高下,分列班次,曹國公李文忠領監事以繩核之。嗣後勳臣子弟多入監讀書。嘉靖元年令公、侯、伯未經任事、年三十以下者,送監讀書,尋令已任者亦送監,而年少勳戚爭以入學為榮矣。

六堂諸生,有積分之法,司業二員分為左右,各提調三堂。凡通《四書》未通經者,居正義、崇志、廣業。一年半以上,文理條暢者,升修道、誠心。又一年半,經史兼通、文理俱優者,乃升率性。升至率性,乃積分。其法,孟月試本經義一道,仲月試論一道,詔、誥、表、內科一道,季月試經史第一道,判語二條。每試,文理俱優者與一分,理優文劣者與半分,紕繆者無分。歲內積八分者為及格,與出身。不及者仍坐堂肄業。如有才學超異者,奏請上裁。

洪武二十六年,盡擢監生劉政、龍鐔等六十四人為行省布政、按察兩使,及參政、參議、副使、僉事等官。其一旦而重用之,至於如此。其為四方大吏者,蓋無算也。李擴等自文華、武英擢御史,擴尋改給事中兼齊相府錄事,蓋臺諫之選亦出於太學。其常調者乃為府、州、縣六品以下官。

初,以北方喪亂之餘,人鮮知學,遣國子生林伯雲等三百六十六人分教各郡。後乃推及他省,擇其壯歲能文者為教諭等官。太祖雖間行科舉,而監生與薦舉人才參用者居多,故其時布列中外者,太學生最盛。一再傳之後,進士日益重,薦舉遂廢,而舉貢日益輕。雖積分歷事不改初法,南北祭酒陳敬宗、李時勉等加意振飭,已漸不如其始。眾情所趨向,專在甲科。宦途升沉,定於謁選之日。監生不獲上第,即奮自鏃礪,不能有成,積重之勢然也。迨開納粟之例,則流品漸淆,且庶民亦得援生員之例以入監,謂之民生,亦謂之俊秀,而監生益輕。於是同處太學,而舉、貢得為府佐貳及州縣正官,官、恩生得選部、院、府、衛、司、寺小京職,尚為正途。而援例監生,僅得選州縣佐貳及府首領官;其授京職者,乃光祿寺、上林苑之屬;其願就遠方者,則以雲、貴、廣西及各邊省軍衛有司首領,及衛學、王府教授之缺用,而終身為異途矣。

舉人入監,始於永樂中。會試下第,輒令翰林院錄其優者,俾入學以俟後科,給以教諭之俸。是時,會試有副榜,大抵署教官,故令入監者亦食其祿也。宣德八年嘗命禮部尚書胡濙與大學士楊士奇、楊榮選副榜舉人龍文等二十四人,送監進學。翰林院三月一考其文,與庶起士同,頗示優異。後不復另試,則取副榜年二十五以上者授教職,年未及者,或依親,或入監讀書。既而不拘年齒,依親、入監者皆聽。依親者,回籍讀書,依親肄業也。又有丁憂、成婚、省親、送幼子,皆仿依親例,限年復班。正統中,天下教官多缺,而舉人厭其卑冷,多不願就。十三年,御史萬節請敕禮部多取副榜,以就教職。部臣以舉人願依親入監者十之七,願就教職者僅十之三,但宜各隨所欲,卻其請不行。至成化十三年,御史胡璘言:「天下教官率多歲貢,言行文章不足為人師範,請多取舉人選用,而罷貢生勿選。」部議歲貢如其舊,而舉人教官仍許會試。自後就教者亦漸多矣。嘉靖中,南北國學皆空虛,議盡發下第舉人入監,且立限期以趣之。然舉人不願入監者,卒不可力強。於是生員歲貢之外,不得不頻舉選貢以充國學矣。

貢生入監,初由生員選擇,既命各學歲貢一人,故謂之歲貢。其例亦屢更。洪武二十一年,定府、州、縣學以一、二、三年為差。二十五年,定府學歲二人,州學二歲三人,縣學歲一人。永樂八年,定州縣戶不及五里者,州歲一人,縣間歲一人。十九年,令歲貢照洪武二十一年例。宣德七年,復照洪武二十五年例。正統六年,更定府學歲一人,州學三歲二人,縣學間歲一人。弘治、嘉靖間,仍定府學歲二人,州學二歲三人,縣學歲一人,遂為永制。後孔、顏、孟三氏,及京學、衛學、都司、土官,川、雲、貴諸遠省,其按年充貢之法,亦間有增減云。歲貢之始,必考學行端莊、文理優長者以充之。其後但取食廩年深者。弘治中,南京祭酒章懋言:「洪、永間,國子生以數千計,今在監科貢共止六百餘人,歲貢挨次而升,衰遲不振者十常八九。舉人坐監,又每後時。差撥不敷,教養罕效。近年有增貢之舉,而所拔亦挨次之人,資格所拘,英才多滯。乞於常貢外令提學行選貢之法,不分廩膳、增廣生員,通行考選,務求學行兼優、年富力強、累試優等者,乃以充貢。通計天下之廣,約取五六百人。以後三、五年一行,則人才可漸及往年矣。」乃下部議行之。此選貢所由始也。選貢多英才,入監課試輒居上等,撥歷諸司亦有幹局。歲貢頹老,其勢日絀,則惟願就教而不願入監。嘉靖二十七年,祭酒程文德請將廷試歲貢惟留即選者於部,而其餘盡使入監。報可。歲貢諸生合疏言,家貧親老,不願入監。禮部復請從其所願,而盡使舉人入監。又從之。舉人入監不能如期,南京祭酒潘晟至請設重罰以趣其必赴。於是舉人、選貢、歲貢三者迭為盛衰,而國學之盈虛亦靡有定也。萬曆中,工科郭如心言:「選貢非祖制,其始欲補歲貢之乏,其後遂妨歲貢之途,請停其選。」神宗以為然。至崇禎時,又嘗行之。恩貢者,國家有慶典或登極詔書,以當貢者充之。而其次即為歲貢。納貢視例監稍優,其實相仿也。

廕子入監,明初因前代任子之制,文官一品至七品,皆得蔭一子以世其祿。後乃漸為限制,在京三品以上方得請廕,謂之官生。出自特恩者,不限官品,謂之恩生。或即與職事,或送監讀書。官生必三品京官,成化三年從助教李伸言也。時給事中李森不可。帝諭,責其刻薄;第令非歷任年久政績顯著者,毋得濫敘而已。既得廕敘,由提學官考送部試,如貢生例,送入監中。時內閣呂原子翾由廕監補中書舍人,七年辛卯乞應順天鄉試。部請從之。給事中芮畿不可。帝允翾所請,不為例。然其後以廕授舍人者,俱得應舉矣。嘉、隆以後,宰相之子有初授即為尚寶司丞,徑轉本司少卿,由光祿、太常以躋九列者,又有以軍功廕錦衣者,往往不由太學。其他並入監。恩生之始,建文元年錄吳雲子黼為國子生,以雲死節雲南也。正德十六年定例,凡文武官死於忠諫者,一子入監。其後守土官死節亦皆得廕子矣。又弘治十八年定例,東宮侍從官,講讀年久輔導有功者,歿後,子孫乞恩,禮部奏請上裁。正德元年復定,其祖父年勞已及三年者,一子即授試中書舍人習字;未及三年者,一子送監讀書。八年復定,東宮侍班官三年者,一子入監。又萬曆十二年定例,三品日講官,雖未考滿,一子入監。

例監始於景泰元年,以邊事孔棘,令天下納粟納馬者入監讀書,限千人止。行四年而罷。成化二年,南京大饑,守臣建議,欲令官員軍民子孫納粟送監。禮部尚書姚夔言:「太學乃育才之地,近者直省起送四十歲生員,及納草納馬者動以萬計,不勝其濫。且使天下以貨為賢,士風日陋。」帝以為然,為卻守臣之議。然其後或遇歲荒,或因邊警,或大興工作,率援往例行之,訖不能止。此舉、貢、蔭、例諸色監生,前後始末之大凡也。

監生歷事,始於洪武五年。建文時,定考覈法上、中、下三等。上等選用,中、下等仍歷一年再考。上等者依上等用,中等者不拘品級,隨才任用,下等者回監讀書。永樂五年,選監生三十八人隸翰林院,習四夷譯書。九年辛卯,鐘英等五人成進士,俱改庶起士。壬辰、乙未以後,譯書中會試者甚多,皆改庶起士,以為常。歷事生成名,其蒙恩遇如此。仁宗初政,中軍都督府奏監生七人吏事勤慎,請注選授官。帝不許,仍令入學,由科舉以進。他歷事者,多不願還監。於是通政司引奏,六科辦事監生二十人滿日,例應還監,仍願就科辦事。帝復召二十人者,諭令進學。蓋是時,六科給事中多缺,諸生覬得之。帝察知其意,故不授官也。宣宗以教官多缺,選用監生三百八十人,而程富等以都御史顧佐之薦,使於各道歷政三月,選擇任之,所謂試御史也。監生撥歷,初以入監年月為先後,丁憂、省祭,有在家延留七八年者,比至入監,即得取撥。陳敬宗、李時勉先後題請,一以坐監年月為淺深。其後又以存省、京儲、依親、就學、在家年月,亦作坐堂之數。其患病及他事故,始以虛曠論。諸生互爭年月資次,各援科條。成化五年,祭酒陳鑒以兩詞具聞,乞敕禮部酌中定制,為禮科所駁。鑒復奏,互爭之。乃下部覆議,請一一精核,仍計地理遠近、水程日月以為準。然文稱往來,紛錯繁揉,上下伸縮,弊端甚多,卒不能畫一也。初令監生由廣業升率性,始得積分出身。天順以前,在監十餘年,然後撥歷諸司,歷事三月,仍留一年,送吏部銓選。其兵部清黃及隨御史出巡者,則以三年為率。其後,以監生積滯者多,頻減撥歷歲月以疏通之。每歲揀選,優者輒與撥歷,有未及一年者。弘治八年,監生在監者少,而吏部聽選至萬餘人,有十餘年不得官者。祭酒林瀚以坐班人少,不敷撥歷,請開科貢。禮部尚書倪岳覆奏,科舉已有定額,不可再增,惟請增歲貢人數,而定諸司歷事,必須日月滿後,方與更替,使諸生坐監稍久,選人亦無壅滯。及至嘉靖十年,監生在監者不及四百人,諸司歷事歲額以千計。禮部尚書李時引岳前議言:「岳權宜二法,一增歲額以足坐班生徒,一議差歷以久坐班歲月。於是府、州、縣學以一歲二貢、二歲三貢、一歲一貢為差,行之四歲而止。其諸司歷事,三月考勤之後,仍歷一年,其餘寫本一年,清黃、寫誥、清軍、清匠三年,以至出巡等項,俱如舊例日月。今國學缺人,視弘治間更甚,請將前件事例,參酌舉行。」並從之,獨不增貢額。未幾,復以祭酒許誥、提學御史胡時善之請,詔增貢額,如岳、時前議。隆、萬以後,學校積馳,一切循故事而已。崇禎二年,從司業倪嘉善言,復行積分法。八年,從祭酒倪元璐言,以貢選為正流,援納為閏流。貢選不限撥期,以積分歲滿為率,援納則依原定撥歷為率。而歷事不分正雜,惟以考定等第為歷期多寡。諸司教之政事,勿與猥雜差遣。滿日,校其勤惰,開報吏部。不率者,回監教習。時監規頹廢已久,不能振作也。凡監生歷事,吏部四十一名,戶部五十三名,禮部十三名,大理寺二十八名,通政司五名,行人司四名,五軍都督府五十名,謂之正歷。三月上選,滿日增減不定。又有諸司寫本,戶部十名,禮部十八名,兵部二十名,刑部十四名,工部八名,都察院十四名,大理寺、通政司俱四名,隨御史出巡四十二名,謂之雜歷。一年滿日上選。又有諸色辦事,清黃一百名,寫誥四十名,續黃五十名,清軍四十名,天財庫十名,初以三年謂之長差,後改一年上選;承運庫十五名,司禮監十六名,尚寶司六名,六科四十名,初作短差,後亦定一年上選。又有隨御史刷卷一百七十八名,工部清匠六十名,俱事完日上選。又有禮部寫民情條例七十二名,光祿寺刷卷四名,修齋八名,參表二十名,報訃二十名,齎俸十二名,錦衣衛四名,兵部查馬冊三十名,工部大木廠二十名,後府磨算十名,御馬監四名,天財庫四名,正陽門四名,崇文、宣武、朝陽、東直俱三名,阜城、西直、安定、德勝俱二名,以半年滿日回監。

郡縣之學,與太學相維,創立自唐始。宋置諸路州學官,元頗因之,其法皆未具。迄期,天下府、州、縣、衛所,皆建儒學,教官四千二百餘員,弟子無算,教養之法備矣。洪武二年,太祖初建國學,諭中書省臣曰:「學校之教,至元其弊極矣。上下之間,波頹風靡,學校雖設,名存實亡。兵變以來,人習戰爭,惟知干戈,莫識俎豆。朕惟治國以教化為先,教化以學校為本。京師雖有太學,而天下學校未興。宜令郡縣皆立學校,延師儒,授生徒,講論聖道,使人日漸月化,以復先王之舊。」於是大建學校,府設教授,州設學正,縣設教諭,各一。俱設訓導,府四,州三,縣二。生員之數,府學四十人,州、縣以次減十。師生月廩食米,人六斗,有司給以魚肉。學官月俸有差。生員專治一經,以禮、樂、射、御、書、數設科分教,務求實才,頑不率者黜之。十五年,頒學規於國子監,又頒禁例十二條於天下,鐫立臥碑,置明倫堂之左。其不遵者,以違制論。蓋無地而不設之學,無人而不納之教。庠聲序音,重規疊矩,無間於下邑荒徼,山陬海涯。此明代學校之盛,唐、宋以來所不及也。生員雖定數於國初,未幾即命增廣,不拘額數。宣德中,定增廣之額:在京府學六十人,在外府學四十人,州、縣以次減十。成化中,定衛學之例:四衛以上軍生八十人,三衛以上軍生六十人,二衛、一衛軍生四十人,有司儒學軍生二十人;土官子弟,許入附近儒學,無定額。增廣既多,於是初設食廩者謂之廩膳生員,增廣者謂之增廣生員。及其既久,人才愈多,又於額外增取,附於諸生之末,謂之附學生員。凡初入學者,止謂之附學,而廩膳、增廣,以歲科兩試等第高者補充之。非廩生久次者,不得充歲貢也。士子未入學者,通謂之童生。當大比之年,間收一二異敏,三場並通者,俾與諸生一體入場,謂之充場儒士。中式即為舉人,不中式仍候提學官歲試,合格乃准入學。提學官在任三歲,兩試諸生。先以六等試諸生優劣,謂之歲考。一等前列者,視廩膳生有缺,依次充補,其次補增廣生。一二等皆給賞,三等如常,四等撻責,五等則廩、增遞降一等,附生降為青衣,六等黜革。繼取一二等為科舉生員,俾應鄉試,謂之科考。其充補廩、增給賞,悉如歲試。其等第仍分為六,而大抵多置三等。三等不得應鄉試,撻黜者僅百一,亦可絕無也。生儒應試,每舉人一名,以科舉三十名為率。舉人屢廣額,科舉之數亦日增。及求舉者益眾,又往往於定額之外加取,以收士心。凡督學者類然。嘉靖十年,嘗下沙汰生員之令,御史楊宜爭之而止。萬曆時,張居正當國,遂核減天下生員。督學官奉行太過,童生入學,有一州縣僅錄一人者,其科舉減殺可推而知也。生員入學,初由巡按御史,布、按兩司及府州縣官。正統元年始特置提學官,專使提督學政,南、北直隸俱御史,各省參用副使、僉事。景泰元年罷提學官。天順六年復設,各賜敕諭十八條,俾奉行之。直省既設提學,有所轄太廣,及地最僻遠,歲巡所不能及者,乃酌其宜。口外及各都司、衛所、土官以屬分巡道員,直隸廬、鳳、淮、揚、滁、徐、和以屬江北巡按,湖廣衡、永、郴以屬湖南道,辰、靖以屬辰沅道,廣東瓊州以屬海南道,某肅衛所以屬巡按御史,亦皆專敕行事。萬曆四十一年,南直隸分上下江,湖廣分南北,始各增提學一員。提學之職,專督學校,不理刑名。所受詞訟,重者送按察司,輕者發有司,直隸則轉送巡按御史。督、撫、巡按及布、按二司,亦不許侵提學職事也。明初,優禮師儒,教官擢給事、御史,諸生歲貢者易得美官。然鉗束亦甚謹。太祖時,教官考滿,兼核其歲貢生員之數。後以歲貢為學校常例。二十六年,定學官考課法,專以科舉為殿最。九年任滿,核其中式舉人,府九人、州六人、縣三人者為最。其教官又考通經,即與升遷。舉人少者為平等,即考通經亦不遷。舉人至少及全無者為殿,又考不通經,則黜降。其待教官之嚴如此。生員入學十年,學無所成者,及有大過者,俱送部充吏,追奪廩糧。至正統十四年申明其制而稍更之。受贓、姦盜、冒籍、宿娼、居喪娶妻妾所犯事理重者,直隸發充國子監膳夫,各省發充附近儒學膳夫、齋夫,滿日為民,俱追廩米。犯輕充吏者,不追廩米。其待諸生之嚴又如此。然其後教官之黜降,生員之充發,皆廢格不行,即臥碑亦具文矣。諸生上者中式,次者廩生,年久充貢,或選拔為貢生。其累試不第、年踰五十、願告退閒者,給與冠帶,仍復其身。其後有納粟馬捐監之例,則諸生又有援例而出學者矣。提學官歲試校文之外,令教官舉諸生行優劣者一二人,賞黜之以為勸懲。此其大較也。諸生應試之文,通謂之舉業。《四書》義一道,二百字以上。經義一道,三百字以上。取書旨明晰而已,不尚華採也。其後標新領異,益漓厥初。萬曆十五年,禮部言:「唐文初尚靡麗而士趨浮薄,宋文初尚鉤棘而人習險譎。國初舉業有用六經語者,其後引《左傳》、《國語》矣,又引《史記》、《漢書》矣。《史記》窮而用六子,六子窮而用百家,甚至佛經、《道藏》摘而用之,流弊安窮。弘治、正德、嘉靖初年,中式文字純正典雅。宜選其尤者,刊布學宮,俾知趨向。」因取中式文字一百十餘篇,奏請刊布,以為準則。時方崇尚新奇,厭薄先民矩矱,以士子所好為趨,不遵上指也。啟、禎之間,文體益變,以出入經史百氏為高,而恣軼者亦多矣。雖數申詭異險僻之禁,勢重難返,卒不能從。論者以明舉業文字比唐人之詩,國初比初唐,成、弘、正、嘉比盛唐,隆、萬比中唐,啟、禎比晚唐云。

自儒學外,又有宗學、社學、武學。宗學之設,世子、長子、眾子、將軍、中尉年未弱冠者俱與焉。其師,於王府長史、紀善、伴讀、教授等官擇學行優長者除授。萬曆中,定宗室子十歲以上,俱入宗學。若宗子眾多,分置數師,或於宗室中推舉一人為宗正,領其事。令學生誦習《皇明祖訓》、《孝順事實》、《為善陰騭》諸書,而《四書》、《五經》、《通鑑》、性理亦相兼誦讀。尋復增宗副二人。子弟入學者,每歲就提學官考試,衣冠一如生員。已復令一體鄉試,許得中式。其後宗學浸多,頗有致身兩榜、起家翰林者。

社學,自洪武八年,延師以教民間子弟,兼讀《御製大誥》及本朝律令。正統時,許補儒學生員。弘治十七年,令各府、州、縣建立社學,選擇明師,民間幼童十五以下者送入讀書,講習冠、婚、喪、祭之禮。然其法久廢,浸不舉行。

武學之設,自洪武時置大寧等衛儒學,教武官子弟。正統中,成國公朱勇奏選驍勇都指揮等官五十一員,熟嫻騎射幼官一百員,始命兩京建武學以訓誨之。尋命都司、衛所應襲子弟年十歲以上者,提學官選送武學讀書,無武學者送衛學或附近儒學。成化中,敕所司歲終考試入學武生。十年以上學無可取者,追廩還官,送營操練。弘治中,從兵部尚書馬文升言,刑《武經七書》分散兩京武學及應襲舍人。嘉靖中,移京城東武學於皇城西隅廢寺,俾大小武官子弟及勳爵新襲者,肄業其中,用文武重臣教習。萬曆中,兵部言,武庫司專設主事一員管理武學,近者裁去,請復專設。教官升堂,都指揮執弟子禮,請遵《會典》例,立為程式。詔皆如議。崇禎十年,令天下府、州、縣學皆設武學生員,提學官一體考取。已又申《會典》事例,簿記功能,有不次擢用、黜退、送操、獎罰、激厲之法。時事方棘,無所益也。

科目者,沿唐、宋之舊,而稍變其試士之法,專取四子書及《易》、《書》、《詩》、《春秋》、《禮記》五經命題試士。蓋太祖與劉基所定。其文略仿宋經義,然代古人語氣為之,體用排偶,謂之八股,通謂之制義。三年大比,以諸生試之直省,曰鄉試。中式者為舉人。次年,以舉人試之京師,曰會試。中式者,天子親策於廷,曰廷試,亦曰殿試。分一、二、三甲以為名第之次。一甲止三人,曰狀元、榜眼、探花,賜進士及第。二甲若干人,賜進士出身。三甲若干人,賜同進士出身。狀元、榜眼、探花之名,制所定也。而士大夫又通以鄉試第一為解元,會試第一為會元,二、三甲第一為傳臚云。子、午、卯、酉年鄉試,辰、戌、丑、未年會試。鄉試以八月,會試以二月,皆初九日為第一場,又三日為第二場,又三日為第三場。初設科舉時,初場試經義二道,《四書》義一道;二場論一道;三場策一道。中式後十日,復以騎、射、書、算、律五事試之。後頒科舉定式,初場試《四書》義三道,經義四道。《四書》主朱子《集註》,《易》主程《傳》、朱子《本義》,《書》主蔡氏傳及古註疏,《詩》主朱子《集傳》,《春秋》主左氏、公羊、穀梁三傳及胡安國、張洽傳,《禮記》主古註疏。永樂間,頒《四書五經大全》,廢註疏不用。其後,《春秋》亦不用張洽傳,禮記止用陳澔《集說》。二場試論一道,判五道,詔、誥、表、內科一道。三場試經史時務策五道。

廷試,以三月朔。鄉試,直隸於京府,各省於布政司。會試,於禮部。主考,鄉、會試俱二人。同考,鄉試四人,會試八人。提調一人,在內京官,在外布政司官。會試,禮部官監試二人,在內御史,在外按察司官。會試,御史供給收掌試卷;彌封、謄錄、對讀、受卷及巡綽監門,搜檢懷挾,俱有定員,各執其事。舉子,則國子生及府、州、縣學生員之學成者,儒士之未仕者,官之未入流者,皆由有司申舉性資敦厚、文行可稱者應之。其學校訓導專教生徒,及罷閑官吏,倡優之家,與居父母喪者,俱不許入試。試卷之首,書三代姓名及其籍貫年甲,所習本經,所司印記。試日入場,講問、代冒者有禁。晚未納卷,給燭三枝。文字中迴避御名、廟號,及不許自序門第。彌封編號作三合字。考試者用墨,謂之墨卷。謄錄用硃,謂之硃卷。試士之所,謂之貢院。諸生席舍,謂之號房。人一軍守之,謂之號軍。試官入院,輒封鑰內外門戶。在外提調、監試等謂之外簾官,在內主考、同考謂之內簾官。廷試用翰林及朝臣文學之優者,為讀卷官。共閱對策,擬定名次,候臨軒。或如所擬,或有所更定,傳制唱第。狀元授修撰,榜眼、探花授編修,二、三甲考選庶起士者,皆為翰林官。其他或授給事、御史、主事、中書、行人、評事、太常、國子博士,或授府推官、知州、知縣等官。舉人、貢生不第、入監而選者,或授小京職,或授府佐及州縣正官,或授教職。此明一代取士之大略也。終明之世,右文左武。然亦嘗設武科以收之,可得而附列也。

初,太祖起事,首羅賢才。吳元年設文武二科取士之令,使有司勸諭民間秀士及智勇之人,以時勉學,俟開舉之歲,充貢京師。洪武三年,詔曰:「漢、唐及宋,取士各有定制,然但貴文學而不求德藝之全。前元待士甚優,而權豪勢要,每納奔競之人,夤緣阿附,輒竊仕祿。其懷材抱道者,恥與並進,甘隱山林而不出。風俗之弊,一至於此。自今年八月始,特設科舉,務取經明行修、博通古今、名實相稱者。朕將親策於廷,第其高下而任之以官。使中外文臣皆由科舉而進,非科舉者毋得與官。」於是京師行省各舉鄉試:直隸貢額百人,河南、山東、山西、陝西、北平、福建、江西、浙江、湖廣皆四十人,廣西、廣東皆二十五人,才多或不及者,不拘額數。高麗、安南、占城,詔許其國士子於本國鄉試,貢赴京師。明年會試,取中一百二十名。帝親製策問,試於奉天殿,擢吳伯宗第一。午門外張掛黃榜,奉天殿宣諭,賜宴中書省。授伯宗為禮部員外郎,餘以次授官有差。時以天下初定,令各行省連試三年,且以官多缺員,舉人俱免會試,赴京聽選。又擢其年少俊異者張唯、王輝等為翰林院編修,蕭韶為秘書監直長,令入禁中文華堂肄業,太子贊善大夫宋濂等為之師。帝聽政之暇,輒幸堂中,評其文字優劣,日給光祿酒饌。每食,皇太子、親王迭為之主,賜白金、弓矢、鞍馬及冬夏衣,寵遇之甚厚。既而謂所取多後生少年,能以所學措諸行事者寡,乃但令有司察舉賢才,而罷科舉不用。至十五年,復設。十七年始定科舉之式,命禮部頒行各省,後遂以為永制,而薦舉漸輕,久且廢不用矣。十八年廷試,擢一甲進士丁顯等為翰林院修撰,二甲馬京等為編修,吳文為檢討。進士之入翰林,自此始也。使進士觀政於諸司,其在翰林、承敕監等衙門者,曰庶起士。進士之為庶起士,亦自此始也。其在六部、都察院、通政司、大理寺等衙門者仍稱進士,觀政進士之名亦自此始也。其後試額有增減,條例有變更,考官有內外輕重,闈事有是非得失。其細者勿論,其有關於國是者不可無述也。

鄉試之額,洪武十七年詔不拘額數,從實充貢。洪熙元年始有定額。其後漸增。至正統間,南北直隸定以百名,江西六十五名,他省又自五而殺,至雲南二十名為最少。嘉靖間,增至四十,而貴州亦二十名。慶、曆、啟、禎間,兩直隸益增至一百三十餘名,他省漸增,無出百名者。交阯初開以十名為額,迨棄其地乃止。會試之額,國初無定,少至三十二人,其多者,若洪武乙丑、永樂丙戌,至四百七十二人。其後或百名,或二百名,或二百五十名,或三百五十名,增損不一,皆臨期奏請定奪。至成化乙未而後,率取三百名,有因題請及恩詔而廣五十名或百名者,非恒制也。

初制,禮闈取士,不分南北。自洪武丁丑,考官劉三吾、白信蹈所取宋琮等五十二人,皆南士。三月,廷試,擢陳安阝為第一。帝怒所取之偏,命侍讀張信等十二人覆閱,安阝亦與焉。帝猶怒不已,悉誅信蹈及信、安阝等,戍三吾於邊,親自閱卷,取任伯安等六十一人。六月復廷試,以韓克忠為第一。皆北士也。然訖永樂間,未嘗分地而取。洪熙元年,仁宗命楊士奇等定取士之額,南人十六,北人十四。宣德、正統間,分為南、北、中卷,以百人為率,則南取五十五名,北取三十五名,中取十名。景泰初,詔書遵永樂間例。二年辛未,禮部方奉行,而給事中李侃爭之,言:「部臣欲專以文詞,多取南人。」刑部侍郎羅綺亦助侃言。事下禮部,覆奏:「臣等奉詔書,非私請也。」景帝命遵詔書,不從侃議。未幾,給事中徐廷章復請依正統間例。五年甲戌,會試,禮部奏請裁定,於是復從廷章言,分南、北、中卷:南卷,應天及蘇、松諸府,浙江、江西、福建、湖廣、廣東;北卷,順天、山東、山西、河南、陝西;中卷,四川、廣西、雲南、貴州及鳳陽、廬州二府,滁、徐、和三州也。成化二十二年,萬安當國,周洪謨為禮部尚書,皆四川人,乃因布政使潘稹之請,南北各減二名,以益於中。弘治二年,復從舊制。嗣後相沿不改。惟正德三年,給事中趙鐸承劉瑾指,請廣河南、陝西、山東、西鄉試之額。乃增陝西為百,河南為九十五,山東、西俱九十。而以會試分南、北、中卷為不均,乃增四川額十名,并入南卷,其餘併入北卷,南北均取一百五十名。蓋瑾陝西人,而閣臣焦芳河南人,票旨相附和,各徇其私。瑾、芳敗,旋復其舊。

初制,兩京鄉試,主考皆用翰林。而各省考官,先期於儒官、儒士內聘明經公正者為之,故有不在朝列累秉文衡者。景泰三年,令布、按二司同巡按御史,推舉見任教官年五十以下、三十以上、文學廉謹者,聘充考官。於是教官主試,遂為定例。其後有司徇私,聘取或非其人,監臨官又往往侵奪其職掌。成化十五年,御史許進請各省俱視兩京例,特命翰林主考。帝諭禮部嚴飭私弊,而不從其請。屢戒外簾官毋奪主考權,考官不當,則舉主連坐。又令提學考定教官等第,以備聘取。然相沿既久,積習難移。弘治十四年,掌國子監謝鐸言:「考官皆御史方面所辟召,職分即卑,聽其指使,以外簾官預定去取,名為防閑,實則關節,而科舉之法壞矣。乞敕兩京大臣,各舉部屬等官素有文望者,每省差二員主考,庶幾前弊可革。」時未能從。嘉靖七年,用兵部侍郎張璁言,各省主試皆遣京官或進士,每省二人馳往。初,兩京房考亦皆取教職,至是命各加科部官一員,閱兩科、兩京房考,復罷科部勿遣,而各省主考亦不遣京官。至萬曆十一年,詔定科場事宜。部議復舉張璁之說,言:「彼時因主考與監臨官禮節小嫌,故行止二科而罷,今宜仍遣廷臣。」由是浙江、江西、福建、湖廣皆用編修、檢討,他省用科部官,而同考亦多用甲科,教職僅取一二而已。蓋自嘉靖二十五年從給事中萬虞愷言,各省鄉試精聘教官,不足則聘外省推官、知縣以益之。四十三年,又從南京御史奏,兩京同考用京官進士,《易》、《詩》、《書》各二人,《春秋》、《禮記》各一人,其餘乃參用教官。萬曆四年,復議兩京同考、教官衰老者遣回,北京取足於觀政進士、候補甲科,南京於附近知縣、推官取用。至是教官益絀。

初制,會試同考八人,三人用翰林,五人用教職。景泰五年,從禮部尚書胡濙請,俱用翰林、部曹。其後房考漸增。至正德六年,命用十七人,翰林十一人,科部各三人。分《詩經》房五,《易經》、《書經》各四,《春秋》、《禮記》各二。嘉靖十一年,禮部尚書夏言論科場三事,其一言會試同考,例用講讀十一人,今講讀止十一人,當盡入場,方足供事。乞於部科再簡三四人,以補翰林不足之數。世宗命如所請。然偶一行之,輒如其舊。萬曆十一年,以《易》卷多,減《書》之一以增於《易》。十四年,《書》卷復多,乃增翰林一人,以補《書》之缺。至四十四年,用給事中餘懋孳奏,《詩》、《易》各增一房,共為二十房,翰林十二人,科部各四人,至明末不變。

洪武初,賜諸進士宴於中書省。宣德五年,賜宴於中軍都督府。八年,賜宴於禮部,自是遂著為令。

庶起士之選,自洪武乙丑擇進士為之,不專屬於翰林也。永樂二年,既授一甲三人曾棨、周述、周孟簡等官,復命於第二甲擇文學優等楊相等五十人,及善書者湯流等十人,俱為翰林院庶起士,庶起士遂專屬翰林矣。復命學士解縉等選才資英敏者,就學文淵閣。縉等選修撰棨,編修述、孟簡,庶起士相等共二十八人,以應二十八宿之數。庶起士周忱自陳少年願學。帝喜而俞之,增忱為二十九人。司禮監月給筆墨紙,光祿給朝暮饌,禮部月給膏燭鈔,人三錠,工部擇近第宅居之。帝時至館召試。五日一休沐,必使內臣隨行,且給校尉騶從。是年所選王英、王直、段民、周忱、陳敬宗、李時勉等,名傳後世者,不下十餘人。其後每科所選,多寡無定額。永樂十三年乙未選六十二人,而宣德二年丁未止邢恭一人,以其在翰林院習四夷譯書久,他人俱不得與也。弘治四年,給事中塗旦以累科不選庶起士,請循祖制行之。大學士徐溥言:「自永樂二年以來,或間科一選,或連科屢選,或數科不選,或合三科同選,初無定限。或內閣自選,或禮部選送,或會禮部同選,或限年歲,或拘地方,或採譽望,或就廷試卷中查取,或別出題考試,亦無定制。自古帝王儲才館閣以教養之。本朝所以儲養之者,自及第進士之外,止有庶起士一途,而或選或否。且有才者未必皆選,所選者未必皆才,若更拘地方、年歲,則是已成之才又多棄而不用也。請自今以後,立為定制,一次開科,一次選用。令新進士錄平日所作論、策、詩、賦、序、記等文字,限十五篇以上,呈之禮部,送翰林考訂。少年有新作五篇,亦許投試翰林院。擇其詞藻文理可取者,按號行取。禮部以糊名試卷,偕閣臣出題考試於東閣,試卷與所投之文相稱,即收預選。每科所選不過二十人,每選所留不過三五輩,將來成就必有足賴者。」孝宗從其請,命內閣同吏、禮二部考選以為常。自嘉靖癸未至萬曆庚辰,中間有九科不選。神宗常命間科一選。禮部侍郎吳道南持不可。崇禎甲戌、丁丑,復不選,餘悉遵例。其與選者,謂之館選。以翰、詹官高資深者一人課之,謂之教習。三年學成,優者留翰林為編修、檢討,次者出為給事、御史,謂之散館。與常調官待選者,體格殊異。

成祖初年,內閣七人,非翰林者居其半。翰林纂修,亦諸色參用。自天順二年,李賢奏定纂修專選進士。由是非進士不入翰林,非翰林不入內閣,南、北禮部尚書、侍郎及吏部右侍郎,非翰林不任。而庶起士始進之時,已羣目為儲相。通計明一代宰輔一百七十餘人,由翰林者十九。蓋科舉視前代為盛,翰林之盛,則前代所絕無也。

輔臣子弟,國初少登第者。景泰七年,陳循、王文以其子北闈下第,力攻主考劉儼,臺省譁然論其失。帝勉徇二人意,命其子一體會試,而心薄之。正德三年,焦芳子黃中會試中式,芳引嫌不讀卷。而黃中居二甲之首,芳意猶不慊,至降調諸翰林以泄其忿。六年,楊廷和子慎廷試第一,廷和時亦引嫌不讀卷。慎以高才及第,人無訾之者。嘉靖二十三年廷試,翟鑾子汝儉、汝孝俱在試中。世宗疑二人濫首甲,抑第一為第三,以第三置三甲。及拆卷,而所擬第三者,果汝孝也,帝大疑之。給事中王交、王堯日因劾會試考官少詹事江汝璧及諸房考朋私通賄,且追論順天鄉試考官秦鳴夏、浦應麒阿附鑾罪,乃下汝璧等鎮撫司獄。獄具,詔杖汝璧、鳴夏、應麒,並革職閑住,而勒鑾父子為民。神宗初,張居正當國。二年甲戌,其子禮闈下第,居正不悅,遂不選庶起士。至五年,其子嗣修遂以一甲第二人及第。至八年,其子懋修以一甲第一人及第。而次輔呂調陽、張四維、申時行之子,亦皆先後成進士。御史魏允貞疏陳時弊,言輔臣子不宜中式。帝為謫允貞。十六年,右庶子黃洪憲主順天試,王錫爵子衡為榜首。禮部郎中高桂論劾舉人李鴻等,并及衡,言:「自故相子一時並進,而大臣之子遂無見信於天下者。今輔臣錫爵子衡,素號多才,青雲不難自致,而人猶疑信相半,宜一體覆試,以明大臣之心跡。」錫爵怒甚,具奏申辨,語過激。刑部主事饒伸復抗疏論之。帝為謫桂於外,下伸獄,削其官。覆試所劾舉人,仍以衡第一,且無一人黜者。二十年會試,李鴻中式。鴻,大學士申時行婿也。榜將發,房考給事中某持之,以為宰相之婿不當中。主考官張位使十八房考公閱,皆言文字可取,而給事猶持不可。位怒曰:「考試不憑文字,將何取衷?我請職其咎。」鴻乃獲收。王衡既被論,當錫爵在位,不復試禮闈。二十九年乃以一甲第二人及第。自後輔臣當國,其子亦無登第者矣。

科場弊竇既多,議論頻數。自太祖重罪劉三吾等,永、宣間大抵帖服。陳循、王文之齮劉儼也,高穀持之,儼亦無恙。弘治十二年會試,大學士李東陽、少詹事程敏政為考官。給事中華昶劾敏政鬻題與舉人唐寅、徐泰,乃命東陽獨閱文字。給事中林廷玉復攻敏政可疑者六事。敏政謫官,寅泰皆斥譴。寅,江左才士,戊午南闈第一,論者多惜之。嘉靖十六年,禮部尚書嚴嵩連摘應天、廣東試錄語,激世宗怒。應天主考及廣東巡按御史俱逮問。二十二年,帝手批山東試錄譏訕,逮御史葉經杖死闕下,布政以下皆遠謫,亦嵩所中傷也。四十年,應天主考中允無錫吳情取同邑十三人,被劾,與副考胡傑俱謫外。南畿翰林遂不得典應天試矣。萬曆四年,順天主考高汝愚中張居正子嗣修、懋修,及居正黨吏部侍郎王篆子之衡、之鼎。居正既死,御史丁此呂追論其弊,且言:「汝愚以『舜亦以命禹』為試題,殆以禪受阿居正。」當國者惡此呂,謫於外,而議者多不直汝愚。三十八年會試,庶子湯賓尹為同考官,與各房互換闈卷,共十八人。明年,御史孫居相劾賓尹私韓敬,其互換皆以敬故。時吏部方考察,尚書孫丕揚因置賓尹、敬於察典。敬頗有文名,眾亦惜敬,而以其宣黨,謂其宜斥也。四十四年會試,吳江沈同和第一,同里趙鳴陽第六。同和素不能文,文多出鳴陽手,事發覺,兩人並謫戍。天啟四年,山東、江西、湖廣、福建考官,皆以策問譏刺,降諭切責。初命貶調,既而褫革,江西主考丁乾學至下獄擬罪,蓋觸魏忠賢怒也。先是二年辛酉,中允錢謙益典試浙江,所取舉人錢千秋卷七篇大結,跡涉關節。榜後為人所訐,謙益自檢舉,千秋謫戍。未幾,赦還。崇禎二年會推閣臣,謙益以禮部侍郎與焉,而尚書溫體仁不與。體仁摘千秋事,出疏攻謙益。謙益由此罷,遂終明世不復起。其他指摘科場事者,前後非一,往往北闈為甚,他省次之。其賄買鑽營、懷挾倩代、割卷傳遞、頂名冒籍,弊端百出,不可窮究,而關節為甚。事屬曖昧,或快恩仇報復,蓋亦有之。其他小小得失,無足道也。

歷科事跡稍異者:永樂初,兵革倉猝,元年癸未,始令各省鄉試。二年甲申會試,以事變不循午未之舊。七年己丑會試,中陳燧等九十五人。成祖方北征,皇太子令送國子監進學,俟車駕還京廷試。九年辛卯,始擢蕭時中第一。宣德五年庚戌,帝臨軒發策畢,退御武英殿,謂翰林儒臣曰:「取士不尚虛文,有若劉蕡、蘇轍輩直言抗論,朕當顯庸之。」乃賦《策士歌》以示讀卷官,顧所擢第一人林震,亦無所表見也。八年癸丑,廷試第一人曹鼐,由江西泰和典史會試中式。正統七年壬戌,刑部吏南昱、公陵驛丞鄭溫亦皆中式。十年乙丑,會試、廷試第一皆商輅。輅,淳安人,宣宗末年乙卯,浙榜第一人。三試皆第一,士子豔稱為三元,明代惟輅一人而已。廷試讀卷盡用甲科,而是年兵部尚書徐晞、十三年戶部侍郎餘亨乃吏員,天順元年丁丑讀卷左都御史楊善乃譯字生,時猶未甚拘流品也。迨後無雜流會試及為讀卷官者矣。七年癸未試日,場屋火,死者九十餘人,俱贈進士出身,改期八月會試。明年甲申三月,始廷試。時英宗已崩,憲宗以大喪未踰歲,御西角門策之。正德三年戊辰,太監劉瑾錄五十人姓名以示主司,因廣五十名之額。十五年庚辰,武宗南巡,未及廷試。次年,世宗即位,五月御西角門策之,擢楊維聰第一。而張璁即是榜進士也,六七年間,當國用事,權侔人主矣。嘉靖八年己丑,帝親閱廷試卷,手批一甲羅洪先、楊名、歐陽德,二甲唐順之、陳束、任瀚六人對策,各加評獎。大學士楊一清等遂選順之、束、瀚及胡經等共二十人為庶起士,疏其名上,請命官教習。忽降諭云:起士之選,祖宗舊制誠善。邇來大臣徇私選取,市恩立黨,於國無益,自今不必選留。唐順之等一切除授,吏、禮二部及翰林院會議以聞。」尚書方獻夫等遂阿旨謂順之等不必留,并限翰林之額,侍讀、侍講、修撰各三員,編修、檢討各六員。著為令。蓋順之等出張璁、霍韜門,而心以大禮之議為非,不肯趨附,璁心惡之。璁又方欲中一清,故以立黨之說進,而故事由此廢。迨十一年壬辰,已罷館選,至九月復舉行之。十四年乙未,帝親製策問,手自批閱,擢韓應龍第一。降諭論一甲三人及二甲第一名次前後之由。禮部因以聖諭列登科錄之首,而十二人對策,俱以次刊刻。二十年辛丑,考選庶起士題,文曰《原政》,詩曰《讀大明律》,皆欽降也。四十四年乙丑廷試,帝始不御殿。神宗時,御殿益稀矣。天啟二年壬戌會試,命大學士何宗彥、朱國祚為主考。故事,閣臣典試,翰、詹一人副之。時已推禮部尚書顧秉謙,特旨命國祚。國祚疏辭,帝曰:「今歲,朕首科,特用二輔臣以光重典,卿不必辭。」嗣後二輔臣典試以為常。是年開宗科,朱慎鷸成進士,從宗彥、國祚請,即授中書舍人。崇禎四年,朱統飾成進士,初選庶起士。吏部以統飾宗室,不宜官禁近,請改中書舍人。統飾疏爭,命仍授庶起士。七年甲戌,知貢舉禮部侍郎林釺言,舉人顏茂猷文兼《五經》,作二十三義。帝念其該洽,許送內簾。茂猷中副榜,特賜進士,以其名另為一行,刻於試錄第一名之前。《五經》中式者,自此接跡矣。

武科,自吳元年定。洪武二十年俞禮部請,立武學,用武舉。武臣子弟於各直省應試。天順八年,令天下文武官舉通曉兵法、謀勇出眾者,各省撫、按、三司,直隸巡按御史考試。中式者,兵部同總兵官於帥府試策略,教場試弓馬。答策二道,騎中四矢、步中二矢以上者為中式。騎、步所中半焉者次之。成化十四年,從太監汪直請,設武科鄉、會試,悉視文科例。弘治六年,定武舉六歲一行,先策略,後弓馬。策不中者不許騎射。十七年,改定三年一試,出榜賜宴。正德十四年定,初場試馬上箭,以三十五步為則;二場試步下箭,以八十步為則;三場試策一道。子、午、卯、酉年鄉試。嘉靖初,定制,各省應武舉者,巡按御史於十月考試,兩京武學於兵部選取,俱送兵部。次年四月會試,翰林二員為考試官,給事中、部曹四員為同考。鄉、會場期俱於月之初九、十二、十五。起送考驗監試張榜,大率仿文闈而減殺之。其後倏罷倏復。又仿文闈南北卷例,分邊方、腹裏。每十名,邊六腹四以為常。萬曆三十八年,定會試之額,取中進士以百名為率。其後有奉詔增三十名者,非常制也。穆、神二宗時,議者嘗言武科當以技勇為重。萬曆之末,科臣又請特設將材武科,初場試馬步箭及槍、刀、劍、戟、拳搏、擊刺等法,二場試營陣、地雷、火藥、戰車等項,三場各就其兵法、天文、地理所熟知者言之。報可而未行也。崇禎四年,武會試榜發,論者大嘩。帝命中允方逢年、倪元璐再試,取翁英等百二十人。逢年、元璐以時方需才,奏請殿試傳臚,悉如文例。乃賜王來聘等及第、出身有差。武舉殿試自此始也。十四年,諭各部臣特開奇謀異勇科。詔下,無應者。

太祖下金陵,辟儒士范祖幹、葉儀。克婺州,召儒士許元、胡翰等,日講經史治道。克處州,徵耆儒宋濂、劉基、章溢、葉琛至建康,創禮賢館處之。以濂為江南等處儒學提舉,溢、琛為營田僉事,基留帷幄預謀議。甲辰三月,敕中書省曰:「今土宇日廣,文武並用。卓犖奇偉之才,世豈無之。或隱於山林,或藏於士伍,非在上者開導引拔之,無以自見。自今有能上書陳言、敷宣治道、武略出眾者,參軍及都督府具以名聞。或不能文章而識見可取,許詣闕面陳其事。郡縣官年五十以上者,雖練達政事,而精力既衰,宜令有司選民間俊秀年二十五以上、資性明敏、有學識才幹者辟赴中書,與年老者參用之。十年以後,老者休致,而少者已熟於事。如此則人才不乏,而官使得人。其下有司,宣布此意。」於是州縣歲舉賢才及武勇謀略、通曉天文之士,間及兼通書律者。既而嚴選舉之禁,有濫舉者逮治之。吳元年,遣起居注吳林、魏觀等以幣帛求遺賢於四方。洪武元年,徵天下賢才至京,授以守令。其年冬,又遣文原吉、詹同、魏觀、吳輔、趙壽等分行天下,訪求賢才,各賜白金而遣之。三年,諭廷臣曰:「六部總領天下之務,非學問博洽、才德兼美之士,不足以居之。慮有隱居山林,或屈在下僚者,其令有司悉心推訪。」六年,復下詔曰:「賢才國之寶也。古聖王勞於求賢,若高宗之於傅說,文王之於呂尚。彼二君者,豈其智不足哉?顧皇皇於版築鼓刀之徒者。蓋賢才不備,不足以為治。鴻鵠之能遠舉者,為其有羽翼也;蛟龍之能騰躍者,為其有鱗鬣也;人君之能致治者,為其有賢人而為之輔也。山林之士德行文藝可稱者,有司采舉,備禮遣送至京,朕將任用之,以圖至治。」是年,遂罷科舉,別令有司察舉賢才,以德行為本,而文藝次之。其目,曰聰明正直,曰賢良方正,曰孝弟力田,曰儒士,曰孝廉,曰秀才,曰人才,曰耆民。皆禮送京師,不次擢用。而各省貢生亦由太學以進。於是罷科舉者十年,至十七年始復行科舉,而薦舉之法並行不廢。時中外大小臣工皆得推舉,下至倉、庫、司、局諸雜流,亦令舉文學才幹之士。其被薦而至者,又令轉薦。以故山林巖穴、草茅窮居,無不獲自達於上,由布衣而登大僚者不可勝數。耆儒鮑恂、餘詮、全思誠、張長年輩,年九十餘,徵至京,即命為文華殿大學士。儒士王本、杜斅、趙民望、吳源,特置為四輔官兼太子賓客。賢良郭有道,秀才范敏、曾泰,稅戶人才鄭沂,儒士趙翥,起家為尚書。儒士張子源、張宗德為侍郎。耆儒劉堉、關賢為副都御史。明經張文通、阮仲志為僉都御史。人才赫從道為大理少卿。孝廉李德為府尹。儒士吳顒為祭酒。賢良欒世英、徐景昇、李延中,儒士張璲、王廉為布政使。孝弟李好誠、聶士舉,賢良蔣安素、薛正言、張端,文學宋亮為參政。儒士鄭孔麟、王德常、黃桐生,賢良餘應舉、馬衛、許安、范孟宗、何德忠、孫仲賢、王福、王清,聰明張大亨、金思存為參議,凡其顯擢者如此。其以漸而躋貴仕者,又無算也。嘗諭禮部:「經明行修練達時務之士,徵至京師。年六十以上七十以下者,置翰林以備顧問。四十以上六十以下者,於六部及布、按兩司用之。」蓋是時,仕進無他途,故往往多驟貴者。而吏部奏薦舉當除官者,多至三千七百餘人,其少者亦至一千九百餘人。又俾富戶耆民皆得進見,奏對稱旨,輒予美官。而會稽僧郭傳,由宋濂薦擢為翰林應奉,此皆可得而考者也。洎科舉復設,兩途並用,亦未嘗畸重輕。建文、永樂間,薦舉起家猶有內授翰林、外授籓司者。而楊士奇以處士,陳濟以布衣,遽命為《太祖實錄》總裁官,其不拘資格又如此。自後科舉日重,薦舉日益輕,能文之士率由場屋進以為榮;有司雖數奉求賢之詔,而人才既衰,第應故事而已。

宣宗嘗出御製《猗蘭操》及《招隱詩》賜諸大臣,以示風勵。實應者寡,人情亦共厭薄。正統元年,行在吏部言:「宣德間,嘗詔天下布、按二司及府、州、縣官舉賢良方正各一人,迄今尚舉未已,宜止之。」帝以朝廷求賢不可止,自今來者,六部、都察院、翰林院堂上官考試,中者錄用,不中者黜之。薦舉者益稀矣。天順元年詔:「處士中,有學貫天人、才堪經濟、高蹈不求聞達者,所司具實奏聞。」御史陳迹奏崇仁儒士吳與弼學行,命江西巡撫韓雍禮聘赴京。至則召見,命為左諭德。與弼辭疾不受。帝又命李賢引見文華殿,從容顧問曰:「重卿學行,特授宮僚,煩輔太子。」與弼固辭。賜宴文華殿,命賢侍宴,降敕褒賚,遣行人送歸,蓋殊典也。至成化十九年,廣東舉人陳獻章被薦,授翰林院檢討,而聽其歸,典禮大減矣。其後弘治中浙江儒士潘辰,嘉靖中南直隸生員文徵明、永嘉儒士葉幼學,皆以薦授翰林院待詔。萬曆中,湖廣舉人瞿九思亦授待詔,江西舉人劉元卿授國子監博士,江西處士章潢僅遙授順天府訓導。而直隸處士陳繼儒、四川舉人楊思心等雖皆被薦,下之禮部而已。崇禎九年,吏部復議舉孝廉,言:「祖宗朝皆偶一行之,未有定制。今宜通行直省,加意物色,果有孝廉、懷才抱德、經明行修之士,由司道以達巡按,覆核疏聞,驗試錄用。」於時薦舉紛紛遍天下,然皆授以殘破郡縣,卒無大效。至十七年,令豫、楚被陷州縣員缺悉聽撫、按官辟選更置,不拘科目、雜流、生員人等。此則皇遽求賢,非承平時舉士之典。至若正德四年,浙江大吏薦餘姚周禮、徐子元、許龍,上虞徐文彪。劉瑾以四人皆謝遷同鄉,而草詔出於劉健,矯旨下禮等鎮撫司,謫戍邊衛,勒布政使林符、邵寶、李贊及參政、參議、府縣官十九人罰米二百石,并削健、遷官,且著令,餘姚人不得選京官。此則因薦舉而得禍者,又其變也。

任官之事,文歸吏部,武歸兵部,而吏部職掌尤重。吏部凡四司,而文選掌銓選,考功掌考察,其職尤重。選人自進士、舉人、貢生外,有官生、恩生、功生、監生、儒士,又有吏員、承差、知印、書算、篆書、譯字、通事諸雜流。進士為一途,舉貢等為一途,吏員等為一途,所謂三途並用也。京官六部主事、中書、行人、評事、博士,外官知州、推官、知縣,由進士選。外官推官、知縣及學官,由舉人、貢生選。京官五府、六部首領官,通政司、太常、光祿寺、詹事府屬官,由官廕生選。州、縣佐貳,都、布、按三司首領官,由監生選。外府、外衛、鹽運司首領官,中外雜職、入流未入流官,由吏員、承差等選。此其大凡也。其參差互異者,可推而知也。初授者曰聽選,升任者曰升遷。選人之法,每年吏部六考、六選。凡引選六,類選六,遠方選二。聽選及考定升降者,雙月大選,其序定於單月。改授、改降、丁憂、候補者,單月急選。其揀選,三歲舉行。舉人乞恩,歲貢就教,無定期。凡升遷,必滿考。若員缺應補不待滿者,曰推陞。內閣大學士、吏部尚書,由廷推或奉特旨。侍郎以下及祭酒,吏部會同三品以上廷推。太常卿以下,部推。通、參以下吏部於弘政門會選。詹事由內閣,各衙門由各掌印。在外官,惟督、撫廷推,九卿共之,吏部主之。布、按員缺,三品以上官會舉。監、司則序遷。其防邊兵備等,率由選擇保舉,付以敕書,邊府及佐貳亦付敕。薊遼之昌平、薊州等,山西之大同、河曲、代州等,陝西之固原、靜寧等六十有一處,俱為邊缺,尤慎選除。有功者越次擢,誤封疆者罪無赦。內地監司率序遷,其後亦多超遷不拘次,有一歲中四五遷、由僉事至參政者。監、司多額外添設,守巡之外往往別立數銜,不能畫一也。在外府、州、縣正佐,在內大小九卿之屬員,皆常選官,選授遷除,一切由吏部。其初用拈鬮法,至萬曆間變為掣簽。二十九年,文選員外郎倪斯蕙條上銓政十八事,其一曰議掣簽。尚書李戴擬行報可,孫丕揚踵而行之。後雖有譏其失者,終明世不復更也。洪武間,定南北更調之制,南人官北,北人官南。其後官制漸定,自學官外,不得官本省,亦不限南北也。初,太祖嘗御奉天門選官,且諭毋拘資格。選人有即授侍郎者,而監、司最多,進士、監生及薦舉者,參錯互用。給事、御史,亦初授升遷各半。永、宣以後,漸循資格,而臺省尚多初授。至弘、正後,資格始拘,舉、貢雖與進士並稱正途,而軒輊低昂,不啻天壤。隆慶中,大學士高拱言:「國初,舉人躋八座為名臣者甚眾。後乃進士偏重,而舉人甚輕,至於今極矣。請自授官以後,惟考政績,不問其出身。」然勢已積重,不能復返。崇禎間,言者數申「三途並用。」之說。間推一二舉人如陳新甲、孫元化者,置之要地,卒以傾覆。用武舉陳啟新為給事,亦聲名潰裂。於是朝端又以為不若循資格。而甲榜之誤國者亦正不少也。

給事中、御史謂之科道。科五十員,道百二十員。明初至天順、成化間,進士、舉貢、監生皆重選補。其遷擢者,推官、知縣而外,或由學官。其後監生及新科進士皆不得與。或庶起士改授,或取內外科目出身三年考滿者考選,內則兩京五部主事、中、行、評、博,國子監博士、助教等,外則推官、知縣。自推、知入者,謂之行取。其有特薦,則俸雖未滿,亦得與焉。考選視科道缺若干,多寡無定額。其授職,吏部、都察院協同注擬,給事皆實補,御史必試職一年始實授,惟庶起士否。嘉靖、萬曆間,常令部曹不許改科道,後亦間行之。舉貢、推、知,例得與進士同考選,大抵僅四之一。嘉靖間,嘗令監生與選。已罷不行。萬曆中,百度廢馳。二十五年,臺省新舊人數不足當額設之半。三十六年,科止數人,道止二人。南科以一人攝九篆者二歲,南道亦止一人。內臺既空,外差亦缺,淮、揚、蘇、松、江西、陝西、廣東西、宣大、甘肅、遼東巡按及陝西之茶馬,河東之鹽課,缺差至數年。給事中陳治則請急考選,不報。三十九年,考選疏上,復留中不下。推、知擬擢臺省,候命闕下,去留不得自如。四十六年,掌河南道御史王象恒復言:「十三道御史在班行者止八人,六科給事中止五人,而冊封典試諸差,及內外巡方報滿告病求代者踵至,當亟議變通之法。」大學士方從哲亦言:「考選諸臣,守候六載,艱苦備嘗。吏部議諮禮部、都察院按次題差,蓋權宜之術。不若特允部推,令諸臣受命供職,足存政體。」卒皆不報。至光宗初,前後考選之疏俱下,而臺省一旦森列矣。考選之例,優者授給事中,次者御史,又次者以部曹用。雖臨時考試,而先期有訪單,出於九卿、臺省諸臣之手,往往據以為高下。崇禎三年,吏部考選畢,奏應擢給事、御史若干人,而以中書二人訪單可否互異,具疏題請。帝責其推諉,令更確議,而不責訪單之非體也。京官非進士不得考選,推、知則舉貢皆行取。然天下守令,進士十三,舉貢十七;推、知行取,則進士十九,舉貢纔十一。舉貢所得,又大率有臺無省,多南少北。御史王道純以為言。帝謂用人當論才,本不合拘資格,下所司酌行之。初制,急缺風憲,不時行取。神宗時,定為三年,至是每年一舉。帝從吏部尚書閔洪學請,仍以三年為期。此選擇言路之大凡也。

保舉者,所以佐銓法之不及,而分吏部之權。自洪武十七年命天下朝覲官舉廉能屬吏始。永樂元年,命京官文職七品以上,外官至縣令,各舉所知一人,量才擢用。後以貪污聞者,舉主連坐,蓋亦嘗間行其法。然洪、永時,選官並由部請。至仁宗初,一新庶政,洪熙元年,特申保舉之令。京官五品以上及給事、御史,外官布、按兩司正佐及府、州、縣正官,各舉所知。惟見任府、州、縣正佐官及曾犯贓罪者,不許薦舉,其他官及屈在下僚,或軍民中有廉潔公正才堪撫字者,悉以名聞。是時,京官勢未重,臺省考滿,由吏部奏陞方面郡守。既而定制,凡布按二司、知府有缺,令三品以上京官保舉。宣德三年,況鐘、趙豫等以薦擢守蘇、松諸府,賜敕行事。十年用郭濟、姚文等為知府,亦如之。其所奏保者,郎中、員外、御史及司務、行人、寺副皆與,不依常調也。後多有政績。部曹及御史,由堂上官薦引,類能其官。而長吏部者,蹇義、郭璡亦屢奉敕諭。帝又慮諸臣畏連坐而不舉,則語大學士楊溥以全才之難,謂:「一言之薦,豈能保其終身,欲得賢才,尤當厚教養之法。」故其時吏治蒸蒸,稱極盛焉。沿及英宗,一遵厥舊。然行之既久,不能無弊,所舉或鄉里親舊、僚屬門下,素相私比者。方面大吏方正、謝莊等由保舉而得罪。而無官保舉者,在內御史,在外知府,往往九年不遷。正統七年,罷薦舉縣令之制。十一年,御史黃裳言:「給事、御史,國初奏遷方面郡守。近年方面郡守率由廷臣保陞,給事、御史以糾參為職,豈能無忤於一人。乞敕吏部仍按例奏請除授。」帝是其言,命部議行。明年,給事中餘忭復指正、莊等事敗,謂宜坐舉主。且言方面郡守有缺,吏部當奏請上裁。尚書王直、英國公張輔等言,方面郡守,保舉陞用,稱職者多,未可擅更易。英宗仍從輔、直言,而採忭疏,許言官指劾。十三年,御史塗謙復陳,舉薦得方面郡守,輒改前操之弊。請仍遵洪武舊制,於內外九年考滿官內揀擇陞授,或親擇朝臣才望者任之。詔可。大臣舉官之例遂罷。景泰中,復行保舉。給事中林聰陳推舉驟遷之弊,言:「今缺參政等官三十餘員,請暫令三品以上官保舉。自後惟布、按兩司三品以上官連名共舉,其餘悉付吏部。」詔並從之。成化五年,科道官復請保舉方面,吏部因并及郡守。帝從言官請,而命知府員缺仍聽吏部推舉。踰年,以會舉多未當,并方面官第令吏部推兩員以聞,罷保舉之令。既而都御史李賓請令在京五品以上管事官及給事、御史,各舉所知以任州縣。從之。弘治十二年,復詔部院大臣各舉方面郡守。吏部因請依往年御史馬文升遷按察使、屠滽遷僉都御史之例,超擢一二,以示激勸,而未經大臣薦舉者亦兼採之。並從其議。當是時,孝宗銳意求治,命吏、兵二部,每季開兩京府部堂上及文武方面官履歷,具揭帖奏覽.第兼保舉法行之,不專恃以為治也。正德以後,具帖之制漸廢。嘉靖八年,給事中夏言復請循弘治故事,且及舉劾賢否略節,每季孟月,部臣送科以達御前,命著為令。而保舉方面郡守之法,終明世不復行矣。

至若坐事斥免、因急才而薦擢者,謂之起廢。家居被召、因需缺而預補者,謂之添註。此又銓法之所未詳,而中葉以後間嘗一行者也。

考滿、考察,二者相輔而行。考滿,論一身所歷之俸,其目有三:曰稱職,曰平常,曰不稱職,為上、中、下三等。考察,通天下內外官計之,其目有八:曰貪,曰酷,曰浮躁,曰不及,曰老,曰病,曰罷,曰不謹。考滿之法,三年給由,曰初考,六年曰再考,九年曰通考。依《職掌》事例考覈升降。諸部寺所屬,初止署職,必考滿始實授。外官率遞考以待覈。雜考或一二年,或三年、九年。郡縣之繁簡或不相當,則互換其官,謂之調繁、調簡。

洪武十一年,命吏部課朝覲官殿最。稱職而無過者為上,賜坐而宴。有過而稱職者為中,宴而不坐。有過而不稱職者為下,不預宴,序立於門,宴者出,然後退。此朝覲考覈之始也。十四年,其法稍定。在京六部五品以下,聽本衙門正官察其行能,驗其勤怠。其四品以上,及一切近侍官與御史為耳目風紀之司,及太醫院、欽天監、王府官不在常選者,任滿黜陟,取自上裁。直隸有司首領官及屬官,從本司正官考覈,任滿從監察御史覆考。各布政使司首領官,俱從按察司考覈。其茶馬、鹽馬、鹽運、鹽課提舉司、軍職首領官,俱從布政司考覈,仍送按察司覆考。其布政司四品以上,按察司、鹽運司五品以上,任滿黜陟,取自上裁。內外入流并雜職官,九年任滿,給由赴吏部考覈,依例黜陟。果有殊勳異能、超邁等倫者,取自上裁。

又以事之繁簡,與歷官之殿最,相參互核,為等第之升降。其繁簡之例,在外府以田糧十五萬石以上,州以七萬石以上,縣以三萬石以上,或親臨王府都、布政、按察三司,并有軍馬守禦,路當驛道,邊方衝要供給處,俱為事繁。府糧不及十五萬石,州不及七萬石,縣不及三萬石,及僻靜處,俱為事簡。在京諸司,俱從繁例。

十六年,京官考覈之制稍有裁酌,俱由其長開具送部核考。十八年,吏部言天下布、按、府、州、縣朝覲官,凡四千一百一十七人,稱職者十之一,平常者十之七,不稱職者十之一,而貪污闒茸者亦共得十之一。帝令稱職者陞,平常者復職,不稱職者降,貪污者付法司罪之,闒茸者免為民。永、宣間,中外官舊未有例者,稍增入之。又從部議,初考稱職、次考未經考覈、今考稱職者,若初考平常、次考未經考覈、今考稱職者,俱依稱職例陞用。自時厥後,大率遵舊制行之。中間利弊不可枚舉,而其法無大變更也。

考察之法,京官六年,以巳、亥之歲,四品以上自陳以取上裁,五品以下分別致仕、降調、閒住為民者有差,具冊奏請,謂之京察。自弘治時,定外官三年一朝覲,以辰、戌、丑、未歲,察典隨之,謂之外察。州縣以月計,上之府,府上下其考,以歲計,上之布政司。至三歲,撫、按通核其屬事狀,造冊具報,麗以八法。而處分察例有四,與京官同。明初行之,相沿不廢,謂之大計。計處者,不復敘用,定為永制。洪武四年命工部尚書朱守仁廉察山東萊州諸郡官吏。六年,令御史臺御史及各道按察司察舉有司官有無過犯,奏報黜陟,此考察之始也。洪熙時,命御史考察在外官,以奉命者不能無私,諭吏部尚書蹇義嚴加戒飭,務矢至公。景泰二年,吏部、都察院考察當黜退者七百三十餘人。帝慮其未當,仍集諸大臣更考,存留者三之一。成化五年,南京吏部右侍郎章綸、都察院右僉都御史高明考察庶官。帝以各衙門掌印官不同僉名,疑有未當,令侍郎葉盛、都給事中毛弘從公體勘,亦有所更定。弘治六年考察,當罷者共一千四百員,又雜職一千一百三十五員。帝諭:「方面知府必指實跡,毋虛文泛言,以致枉人。府州以下任未三年者,亦通核具奏。」尚書王恕等具陳以請,而以府、州、縣官貪鄙殃民者,雖年淺不可不黜。帝終謂人才難得,降諭諄諄,多所原宥。當黜而留者九十餘員。給事、御史又交章請黜遺漏及宜退而留者,復命吏部指實跡,恕疏各官考語及本部訪察者以聞。帝終以考語為未實,諭令復核。恕以言不用,且疑有中傷者,遂力求去。至十四年,南京吏部尚書林瀚言,在外司府以下官,俱三年一次考察,兩京及在外武職官,亦五年一考選,惟兩京五品以下官,十年始一考察,法大闊略。旨下,吏部覆請如瀚言,而京官六年一察之例定矣。京察之歲,大臣自陳。去留既定,而居官有遺行者,給事、御史糾劾,謂之拾遺。拾遺所攻擊,無獲免者。弘、正、嘉、隆間,士大夫廉恥自重,以掛察典為終身之玷。至萬曆時,閣臣有所徇庇,間留一二以撓察典,而羣臣水火之爭,莫甚於辛亥、丁巳,事具各傳中。黨局既成,互相報復,至國亡乃已。

兵部凡四司,而武選掌除授,職方掌軍政,其職尤要。凡武職,內則五府、留守司,外則各都司、各衛所及三宣、六慰。流官八等,都督及同知、僉事,都指揮使、同知、僉事,正副留守。世官九等:指揮使及同知、僉事,衛、所鎮撫,正、副千戶,百戶,試百戶。直省都指揮使二十一,留守司二,衛九十一,守禦、屯田、羣牧千戶所二百十有一。此外則苗蠻土司,皆聽部選。自永樂初增立三大營,各設管操官,各哨有分管、坐營官、坐司官。景泰中,設團營十,已復增二,各有坐營官,俱特命親信大臣提督之,非兵部所銓擇也。凡大選,曰色目,曰狀貌,曰才行,曰封贈,曰襲蔭。其途有四,曰世職,曰武舉,曰行伍,曰納級。初,武職率以勳舊。太祖慮其不率,以《武士訓戒錄》、《大誥武臣錄》頒之。後乃參用將材,三歲武舉,六歲會舉,每歲薦舉,皆隸部除授。久之,法紀隳壞,選用紛雜。正德間,冒功陞授者三千有奇。嘉靖中,詹事霍韜言:「成化中,增太祖時軍職四倍,今又增幾倍矣。錦衣初額官二百五員,今至千七百員,殆增八倍。洪武初,軍功襲職子弟年二十者比試,初試不中,襲職署事,食半俸。二年再試,中者食全俸,仍不中者充軍。其法至嚴,故職不冗而俸易給。自永樂後,新官免試,舊官即比試,賄賂無不中,此軍職所以日濫也。永樂平交阯,賞而不升。邇者不但獲馘者陞,而奏帶及緝妖言捕盜者亦無不升,此軍職所以益冗也。宜命大臣循清黃例,內外武職一切差次功勞,考其祖宗相承,叔侄兄弟繼及。或洪、永年間功,或宣德以後功,或內監弟侄恩廕,或勳戚駙馬子孫,或武舉取中,各分數等,默寓汰省之法。或許世襲,或許終身,或許繼,或不許繼,各具冊籍,昭示明白,以為激勸。」於是命給事中夏言等查覈冒濫。言等指陳其弊,言:「鎮守官奏帶舊止五名,今至三四百名,蓋一人而奏帶數處者有之,一時而數處獲功者有之。他復巧立名色,紀驗不加審核,銓選又無駁勘,其改正重升、併功加授之類,弊端百出,宜盡革以昭神斷。」部核如議。恩倖冗濫者,裁汰以數千計,宿蠹為清。萬曆十五年,復詔嚴加察核。且嘗命提、鎮、科道會同兵部,品年資,課技藝,序薦剡,分為三等,名曰公選。然徒飾虛名,終鮮實效也。

武官爵止六品,其職死者襲,老疾者替,世久而絕,以旁支繼。年六十者子替。明初定例,嫡子襲替,長幼次及之。絕者,嫡子庶子孫次及之;又絕者,以弟繼。永樂後,取官舍旗軍餘丁曾歷戰功者,令原帶俸及管事襲替,悉因之。其降級子孫仍替見降職事。弘治時,令旁支減級承襲。正德中,令旁支入總旗。嘉靖間,旁支無功者,不得保送。凡升職官舍,如父職。其陣亡保襲者,流官一等。凡襲替官舍,以騎射試之。大抵世職難核,故例特詳,而長弊叢奸,亦復不少。

官之大者,必會推。五軍都督府掌印缺,於見任公、侯、伯取一人。僉書缺,於帶俸公、侯、伯及在京都指揮,在外正副總兵官,推二人。錦衣衛堂上官及前衛掌印缺,視五府例推二人。都指揮、留守以下,上一人。正德十六年,令五府及錦衣衛必由都指揮屢著勳猷者升授。諸衛官不世,獨錦衣以世。

武之軍政,猶文之考察也。成化二年,令五年一行,以見任掌印、帶俸、差操及初襲官一體考覈。十三年令兩京通考以為常。五府大臣及錦衣衛堂上官自陳候旨,直省總兵官如之。在內五府所屬並直省衛所官,悉由巡視官及部官注送;在外都司、衛所官,由撫、按造冊繳部。副參以下,千戶以上,由都、布、按三司察注送撫,諮部考舉題奏。錦衣衛管戎務者倍加嚴考,南、北鎮撫次之。各衛所及地方守禦並各都司隸巡撫者,例同。惟管漕運者不與考。
