\article{列傳第一百五十九}

\begin{pinyinscope}
賀世賢尤世功童仲揆陳策周敦吉等張神武等羅一貫劉渠祁秉忠滿桂孫祖壽趙率教朱國彥官惟賢張奇化何可綱黃龍李惟鸞金日觀楚繼功等

賀世賢,榆林衛人。少為廝養,後從軍,積功至沈陽遊擊,遷義州參將。萬曆四十六年七月,清河被圍,副將鄒儲賢固守。城破,率親丁鏖戰城南,與參將張俱死。部將二十人、兵民萬餘殲焉。世賢駐靉陽,聞變,疾馳出塞,得首功百五十有四級,進副總兵。

明年,楊鎬四路出師。世賢副李如柏出清河。劉綎深入中伏,勸如柏往救,不從,綎遂覆歿。尋擢都督僉事,充總兵官,駐虎皮驛。鐵嶺被圍,世賢馳援,城已破,邀獲首功百餘級。泰昌元年九月連戰灰山、撫安堡,獲首功二百有奇。當是時,四方宿將鱗集遼左,率縮朒不敢戰,獨世賢數角斗有功,同列多忌之。移鎮沈陽。經略袁應泰下納降令。廣寧總兵李光榮疑世賢所納多,以狀聞。巡撫薛國用亦奏三可慮,兵部尚書崔景榮請拒勿納,而置己納於他所。然世賢所納卒不可散,同列遂謗其有異志。

天啟元年三月,我大清以重兵薄沈陽。世賢及總兵尤世功掘塹濬壕,樹大木為柵,列楯車火器木石,環城設兵,守城法甚具。大清先以數十騎來偵,世功兵躡之,殺四人。世賢勇而輕,嗜酒。旦日飲酒,率親丁千,出城逆擊,期盡敵而反。大清兵佯敗,世賢乘銳進。倏精騎四合,世賢戰且卻,抵西門,身被十四矢。城中聞世賢敗,各鳥獸竄,而降丁復叛,斷城外弔橋。或勸世賢走遼陽,曰:「吾為大將,不能存城,何面目見袁經略乎!」揮鐵鞭馳突圍中,擊殺數人,中矢墜馬而死。世功引兵援,亦戰死。

世功亦榆林人。萬歷中,舉武鄉試,歷沈陽遊擊。張承廕之敗也,世功脫歸,坐褫職。經略楊鎬言其身負重傷,才堪策勵,乃補武精營遊擊。鎬四路出師,世功隸李如柏麾下,得全。尋以副總兵守沈陽。熊廷弼代鎬,愛其才,與副將朱萬良並倚任。廷弼罷,袁應泰代,議三路出師,用為總兵官。未行,而沈陽被兵,死於戰。贈少保、左都督,增世廕三級,再廕指揮僉事,世襲,賜祭葬,建祠曰「愍忠」。

世賢既歿,或疑其叛降,恤典故不及。四川副使車樸為訟冤,格眾議不果。

童仲揆,南京人。舉武會試,歷都指揮,掌四川都司。萬曆末,擢副總兵,督川兵援遼,與同官陳策並充援剿總兵官。熹宗初立,經略袁應泰招蒙古諸部,處之遼、沈二城。仲揆力諫,不聽。

明年,天啟改元,應泰欲城清河、撫順。議三路出師,用大將十人,各將兵萬餘,仲揆、策當其二。未行,而大清兵已逼沈陽。兩人馳救,次渾河。遊擊周敦吉曰:「事急矣,請直抵沈陽,與城中兵夾擊,可以成功。」已,聞沈陽陷,諸將皆憤曰:「我輩不能救沈,在此三年何為!」敦吉固請與石砫都司秦邦屏先渡河,營橋北,仲揆、策及副將戚金、參將張名世統浙兵三千營橋南。邦屏結陣未就,大清兵來攻,卻復前者三,諸軍遂敗。敦吉、邦屏及參將吳文傑、守備雷安民等皆死。他將走入浙兵營,被圍數匝。副將朱萬良、姜弼不救,及圍急始前,一戰即敗走。大清兵盡銳攻浙營。營中用火器,多殺傷。火藥盡,短兵接,遂大潰。策先戰死,仲揆將奔,金止之,乃還兵斗。力盡矢竭,揮刀殺十七人。大清兵萬矢齊發,仲揆與金、名世及都司袁見龍、鄧起龍等並死焉。萬良既遁,經略將斬之,乞勩罪自效。及遼陽被攻,果陷陣死。

自遼左用兵,將士率望風奔潰,獨此以萬餘人當數萬眾。雖力絀而覆,時咸壯之。事聞,贈策少保、左都督,增世廕三級,再廕本衛指揮僉事,世襲,賜祭葬,建祠曰:「愍忠」。仲揆贈都督同知,增世廕三級,祀祀。金、起龍贈都督僉事,增世廕三級,附祀。名世先有罪系獄,尚書薛三才薦其善火器,命從征立功。文傑亦先褫職。及死,並得復官,贈三級,增世廕二級。見龍等皆予贈廕,他副將至把總戰死者百二十餘人,贈廕有差。

敦吉,先為四川永寧參將。永寧宣撫奢效忠卒,子崇明幼,其妻奢世統與妾奢世續爭印,相攻者十餘年。後崇明襲職,世續猶匿印不予。都司張神武與敦吉謀,盡掠其積聚子女,擒世續以歸。其部目閻宗傳怒,以求主母為名,大掠永寧、赤水、普市、麾尼,數百里成兵墟。事聞,敦吉、神武並論死。遼東告警,命敦吉從軍自效,及是鏖戰死,贈恤如制。

神武,新建人。萬曆中舉武會試第一。授四川都司僉書。既論死,遼左兵興,用經略袁應泰薦,詔諭從征立功。神武率親丁二百四十餘,疾馳至廣寧。會遼陽已失,巡撫薛國用固留之,不可,曰:「奉命守遼陽,非守廣寧也。」曰:「遼陽歿矣,若何之?」曰:「將以殲敵。」曰:「二百人能殲敵乎?」曰:「不能,則死之。」前至遼河,遇逃卒十餘萬。神武以忠義激其帥,欲與還戰,帥不從。乃獨率所部渡河,抵首山,去遼陽十七里而軍。將士不食已一日,遇大清兵,疾呼奮擊,孤軍無援,盡歿於陣。監軍御史方震孺繪神武像,率將士羅拜,為文祭之。詔贈都督僉事,世廕千戶,立祠祀之。

又有楊宗業、梁仲善者,皆援遼總兵官。宗業歷鎮浙江、山西。楊鎬四路敗後,命提兵赴援,至是父子並戰死。仲善亦戰死遼陽城下。宗業贈都督同知,世廕千戶;仲善贈都督僉事,增世廕三級。並從祠附祀。

羅一貫,甘州衛人。以參將守西平堡。遼陽陷,西平地最衝,一貫悉力捍禦。巡撫王化貞言於朝,加副總兵。時化貞駐廣寧,經略熊廷弼駐右屯,總兵劉渠以二萬人守鎮武,祁秉忠以萬人守閭陽,而一貫帥三千人守西平。已,定議,各繕隍堅壘,急則互相援,違者必誅。明年正月,大清兵西渡河,經撫戒勿輕戰。兵漸近,參將黑雲鶴出擊。一貫止之,不從。明日,雲鶴戰敗,奔還城,追兵殲焉。一貫憑城固拒,用炮擊傷者無算。大清樹旗招降,且遣使來說,一貫不從。又明日,騎益眾,環城力攻。一貫流矢中目,不能戰。火藥矢石盡,乃北面再拜,曰:「臣力竭矣。」遂自剄。都司陳尚仁、王崇信亦死之。化貞知城未下,信遊擊孫得功語,盡發廣寧兵。以得功及中軍遊擊祖大壽為前鋒,令會秉忠赴援,廷弼亦遣使督渠進戰,遇大清兵於平陽。得功懷異志,欲引去。乃分兵為左右翼,稍卻,推渠、秉忠前。渠等力戰,頗有殺傷。得功及副將鮑承先走,後軍見之亦奔,遂大潰。渠戰死。秉忠被二刀三矢,家眾扶上馬,奪圍出,創重,卒於途。副將劉征擊殺十餘人,乃死。大壽走覺華島。得功遂降。越二日,廣寧即破。事聞,贈一貫都督同知,世廕副千戶;渠、秉忠少保,左都督,增世廕三級,再廕指揮僉事。皆賜祭葬,建祠並祀。

一貫子俊傑承廕,崇禎中仕至宣府總兵官,免歸。李自成犯甘州,城陷,死之。

渠,京城巡捕營副將也,以御史楊鶴薦,擢總兵官,援剿遼東。遼陽被圍,廣寧總兵李光榮不能救,反斷河橋截軍民歸路,總督文球劾罷之,即以渠代。西平告急,帥鎮武兵往援,遂戰歿。

秉忠,陜西人。萬曆四十四年為永昌參將。銀定、歹青以二千餘騎人塞,秉忠提兵三百拒之,轉戰兩晝夜。援軍至,始遁。秉忠追還所掠人畜,邊人頌之。擢涼州副總兵。經略袁應泰薦其智勇,令率私卒守蒲河。至則遼陽已破,命為援剿總兵官,駐防閭陽,援西平,竟死。

自遼左軍興,總兵官陣亡者凡十有四人:撫順則張承廕,四路出師則杜松、劉綎、王宣、趙夢麟,開原則馬林,沈陽則賀世賢、尤世功,渾河則童仲揆、陳策,遼陽則楊宗業、梁仲善。是役,渠與秉忠繼之。朝端恤典,俱極優崇。而僨軍之將,若李如柏、麻承恩輩,竟有未膺顯戮者。

滿桂,蒙古人,幼入中國,家宣府。稍長,便騎射。每從征,多斬馘。軍令,獲敵首一,予一官,否則賚白金五十。桂屢得金,不受職。年及壯,始為總旗。又十餘年為百戶。後屢遷潮河川守備。楊鎬四路師敗,薦小將知兵者數人,首及桂。移守黃土嶺。為總督王象乾所知,進石塘路遊擊、喜峰口參將。

天啟二年,大學士孫承宗行邊,桂入謁。壯其貌,與談兵事,大奇之。及出鎮山海,即擢副總兵,領中軍事。承宗幕下,文武輻輳,獨用桂。桂椎魯甚,然忠勇絕倫,不好聲色,與士卒同甘苦。

明年,承宗議出關修復寧遠。問誰可守者。馬世龍薦孫諫及李承先,承宗皆不許。袁崇煥、茅元儀進曰:「滿桂可。但為公中軍,不敢請耳。」承宗曰;「既可,安問中軍。」呼桂語之,慨然請行。世龍猶疑其不可,承宗不聽。即日置酒,親為之餞。桂至寧遠,與崇煥協心城築,屹然成重鎮。語具《崇煥傳》中。

時蒙古部落駐牧寧遠東鄙,遼民來歸者悉遭劫掠,承宗患之。四年二月,遣桂及總兵尤世祿襲之大凌河。諸部號泣西竄,東鄙以寧。拱兔、炒花、宰賽諸部陽受款而陰懷反側。桂善操縱,諸部咸服,歲省撫賞銀不貲。初,城中郭外,一望丘墟。至是軍民五萬餘家,屯種遠至五十里。承宗上其功。詔擢都督僉事,加銜總兵。承宗乃令典後部,與前部趙率教相掎角。督餉郎中楊呈秀侵剋軍糧,副將徐漣激之變,圍崇煥署。憚桂家卒勇猛,不敢犯,結隊東走。桂與崇煥追斬首惡,撫餘眾而還。

六年正月,我大清以數萬騎來攻,遠邇大震,桂與崇煥死守。始攻西南城隅,發西洋紅夷炮,傷攻者甚眾。明日轉攻南城,用火器拒卻之,圍解。帝大喜,擢都督同知,實授總兵官。再論功,加右都督,廕副千戶,世襲。桂疏謝,並自敘前後功。優詔褒答,再進左都督。

桂初與率教深相得。是役也,怒其不親救,相責望。帝聞之,下敕戒勉。而崇煥復與桂不和,言其意氣驕矜,謾罵僚屬,恐壞封疆大計,乞移之別鎮,以關外事權歸率教。舉朝皆知桂可用,慮同城或僨事,遂召還。督師王之臣力言桂不可去,而召命已下。又請用之關門。崇煥皆不納。閏六月乃命以故秩僉書中軍府事。未幾,崇煥亦自悔,請仍用之臣言,帝可之,命桂掛印移鎮關門,兼統關外四路及燕河、建昌諸軍,賜尚方劍以重事權。

七年五月,大清兵圍錦州,分兵略寧遠。桂遣兵救,被圍笊籬山。桂與總兵尤世祿赴之,大戰相當。遂入寧遠城,與崇煥為守禦計。俄大清兵進薄城下,桂率副將尤世威等出城迎,頗有殺傷,桂亦身被重創。捷聞,加太子太師,世蔭錦衣僉事。及崇煥休去,之臣再督師,盛推桂才,請仍鎮寧遠。會蒙古炒花諸部離散,桂與之臣多收置之麾下。

莊烈帝已嗣位,詔之臣毋蹈袁應泰、王化貞故轍,並責桂阿之臣意。桂遂請病乞休,不允。崇禎元年七月,言官交劾之臣,因及桂。之臣罷,桂亦召還府。適大同總兵渠家楨失事,命桂代之。大同久恃款弛備,插部西侵,順義王遂入境大掠。家楨及巡撫張翼明論死,插部遂挾賞不去。桂至,遍閱八路七十二城堡,邊備大修,軍民恃以無恐。

明年冬十月,大清兵入近畿。十一月詔諭勤王。桂率五千騎入衛,次順義,與宣府總兵侯世祿俱戰敗,遂趨都城。帝遣官慰勞,犒萬金,令與世祿俱屯德勝門。無何,合戰,世祿兵潰,桂獨前鬥。城上發大炮佐之,誤傷桂軍,桂亦負傷,令入休甕城。旋與袁崇煥、祖大壽並召見,桂解衣示創,帝深嘉歎。十二月朔復召見,下崇煥獄,賜桂酒饌,令總理關、寧將卒,營安定門外。

桂驍勇敢戰。所部降丁間擾民,桂不能問。副將申甫所統多市人,桂軍凌之。夜發矢,驚其營,有死者。御史金聲以聞,帝亦不問。及大壽軍東潰,乃拜桂武經略,盡統入衛諸軍,賜尚方劍,趣出師。桂曰:「敵勁援寡,未可輕戰。」中使趣之急,不得已,督黑雲龍、麻登雲、孫祖壽諸大將,以十五日移營永定門外二里許,列柵以待。大清兵自良鄉回,明日昧爽,以精騎四面蹙之。諸將不能支,大敗,桂及祖壽戰死,雲龍、登雲被執。帝聞,震悼,遣禮部侍郎徐光啟致祭,贈少師,世廕錦衣僉事,襲陞三級,賜祭葬,有司建祠。

孫祖壽,字必之,昌平人。萬曆中舉武鄉試,授固關把總。天啟二年歷官署都督僉事,為薊鎮總兵官。

孫承宗行邊,議於薊鎮三協十二路分設三大將。以祖壽領西協,轄石匣、古北、曹家、牆子四路,駐遵化。而江應詔領東協,駐關門,轄山海關、一片石、燕河、建昌四路。馬世龍領中協,駐三屯營,轄馬蘭、松棚、喜峰、太平四路。經略王在晉、總督王象乾僉謂:「永平設鎮,本以衛山海。今移之三屯,則去山海四百里,於應援為疏。遵化去三屯止六十里,今並列兩鎮,於建牙為贅。請令世龍仍鎮永平,以東協四路分隸世龍、應詔,而以中、西二協專隸之祖壽,仍鎮三屯。」章下兵部,署事侍郎張經世議如其言,承宗堅執如初。乃命祖壽移鎮遵化。七年,錦州告警,祖壽赴援,不敢戰,被劾罷歸。及是都城被兵,散家財,招回部曲,從滿桂赴斗,竟死,贈恤如制。

祖壽初守固關,遘危疾,妻張氏割臂以療,絕飲食者七日。祖壽生,而張氏旋死,遂終身不近婦人。為大帥,部將以五百金遺其子於家,卻不受。他日來省,賜之卮酒曰:「卻金一事,善體吾心,否則法不汝宥也。」其秉義執節如此。

趙率教,陜西人。萬曆中,歷官延綏參將,屢著戰功。已,劾罷。遼事急,詔廢將蓄家丁者赴軍前立功。率教受知於經略袁應泰,擢副總兵,典中軍事。

天啟元年,遼陽破,率教潛逃,罪當死,倖免。明年,王化貞棄廣寧,關外諸城盡空。率教請於經略王在晉,願收復前屯衛城,率家丁三十八人以往。蒙古據其地,不敢進,抵中前所而止。其年,游擊魯之甲以樞輔孫承宗令,救難民六千口,至前屯,盡驅蒙古於郊外。率教乃得入,編次難民為兵,繕雉堞,謹斥堠,軍府由是粗立。既而承宗令裨將陣練以川、湖土兵來助,前屯守始固。而率教所招流亡至五六萬。擇其壯者從軍,悉加訓練。餘給牛種,大興屯田,身自督課,至手足胼胝。承宗出關閱視,大喜,以己所乘輿贈之。

蒙古虎墩兔素為總督王象乾所撫。其部下抽扣兒者,善為盜,率教捕斬四人。招撫僉事萬有孚與率教有隙,遂以故敗款事訴之象乾。象乾告兵部尚書董漢儒,將斬之,賴承宗貽書漢儒,得不死。

時承宗分關內外為五部。以馬世龍、王世欽、尤世祿領中、左、右部,而令率教與副將孫諫領前、後部,部各萬五千人。率教仍駐前屯。四年九月,承宗暴其功於朝。擢署都督僉事,加銜總兵。五年冬,承宗去,高第來代,諸將多所更置。率教善事第,第亦委信之。

六年二月,蒙古以寧遠被圍,乘間入犯平川、三山堡。率教禦之,斬首百餘級,奪馬二百匹,追至高臺堡乃還。捷聞,帝大喜,立擢都督同知,實授總兵官,代楊麒鎮山海關。尋論功,再進右都督,世廕本衛副千戶。時滿桂守寧遠,亦有盛名,與率教深相得。及寧遠被圍,率教遣一都司、四守備東援。桂惡其稽緩,拒不納,以袁崇煥言,乃令入。既解圍,率教欲分功。桂不許,且責其不親援,兩人遂有隙。中朝聞之,下敕戒諭。而桂又與崇煥不和。乃召還桂,令率教盡統關內外兵,移鎮寧遠。

七年正月,大清兵南征朝鮮。率教督兵抵三岔河為牽制,卒無功。三月,崇煥議修築錦州、大凌河、中左所三城,漸圖恢復。率教移鎮錦州護工,再加左都督。五月,大清兵圍錦州,率教與中官紀用、副將左輔、朱梅等嬰城固守。發大炮,頗多擊傷。相持二十四日,圍始解。時桂亦著功寧遠,因稱「寧、錦大捷」。魏忠賢等蒙重賞。率教加太子少傅,廕錦衣千戶,世襲。

崇禎元年八月移鎮永平,兼轄薊鎮八路。踰月,掛平遼將軍印,再移至關門。明年,大清兵由大安口南下。率教馳援,三晝夜抵三屯營。總兵朱國彥不令入,遂策馬而西。十一月四日戰於遵化,中流矢陣亡,一軍盡歿。帝聞痛悼,賜恤典,立祠奉祀。

率教為將廉勇,待士有恩,勤身奉公,勞而不懈,與滿桂並稱良將。二人既歿,益無能辦東事者。

國彥以崇禎二年四月為薊鎮中協總兵官,駐三屯營。十一月六日,大清兵臨城,副將朱來同等挈家潛遁。國彥憤,榜諸人性名於通衢。以所積俸銀五百餘、衣服器具盡給部卒。具冠帶西向稽首,偕妻張氏投繯死。

官惟賢,萬曆未,為甘肅裴家營守備。天啟二年以都司僉書署鎮番參將事,歷宣府遊擊、延綏西路參將,仍移鎮番。五年春,河套、松山諸部入犯,惟賢偕參將丁孟科大敗之,斬首二百四十餘級。明年春,班記刺麻台吉復糾松山銀定、歹成及矮木素、三兒台吉,以三千騎來犯。惟賢再敗之,獲首功二百有奇。三兒台吉被創死。進惟賢副總兵。其冬,銀定等以三兒之死挾憤圖報,益糾河套土巴台吉等分道入掠。惟賢及鎮將徐永壽等亦分道拒之,共獲首功百有六十。七年春,銀定、賓兔、矮木素、班記刺麻合土賣火力赤等由黑水河入。惟賢及西路副將陳洪範大破之,斬首百八十餘級。當是時,西部頻寇邊,惟賢屢挫其鋒。其秋,王之臣督師遼東,乞惟賢赴關門。

明年,崇禎改元,惟賢至,用為山海北路副總兵。二年冬,京師有警。惟賢入衛,總理馬世龍令急援寶坻、漷縣。明年正月九日,大清兵自撫寧向山海。翼日,至鳳凰店,離關三十里列三營。惟賢與參將陳維翰等設兩營以待,合戰,互有殺傷。已,大清兵返撫寧,世龍令惟賢率維翰及遊擊張奇化、李居正、王世選、王成等往襲遵化。至城西波羅灣,城中兵出擊,前鋒殊死戰。大清兵收入城,後隊乘勢進攻,城上矢石如雨。尋復遣兵出戰,惟賢陷陣,中箭死,士卒殺傷者三百餘人,奇化亦戰歿。

何可綱,遼東人。天啟中,以守備典袁崇煥寧遠道中軍,廉勇善撫士卒。六年,寧遠被圍,佐崇煥捍禦有功,進都司僉書。明年再被兵,復堅守。遷參將,署寧遠副將事。崇禎元年,巡撫畢自肅令典中軍。及崇煥再出鎮,復以副將領中軍事,靖十三營之變。崇煥欲更置大將,上言:「臣昔為巡撫,定議關外止設一總兵。其時魏忠賢竊柄,崔呈秀欲用其私黨,增設三四人,以致權勢相衡,臂指不運。乃止留寧遠及前鋒二人,而臂指之不運猶故也。臣以為寧遠一路,斷宜併歸前鋒。總兵駐關內者,掛平遼將軍印,轄山、石二路,而以前屯隸之。駐關外者,掛征遼前鋒將軍印,轄寧遠一衛,而以錦州隸之。薊遼總兵趙率教久習遼事,宜與山海麻登雲相易,掛平遼將軍印。關外總兵舊有朱梅、祖大壽。梅已解任,宜併歸大壽,駐錦州,而以臣中軍何可綱專防寧遠。可綱仁而有勇,廉而能勤,事至善謀,其才不在臣下。臣向所建豎,實可綱力,請加都督僉事,仍典臣中軍。則一鎮之費雖裁,一鎮之用仍在。臣妄謂五年奏凱者,仗此三人之力,用而不效,請治臣罪。」帝悉從之。可綱佐崇煥更定軍制,歲省餉百二十萬有奇。以春秋二防功,進職右都督。

二年冬,京師被兵,與大壽從崇煥入衛,數有功。崇煥下吏,乃隨大壽東潰,復與歸朝。明年正月,永平、灤州失守,可綱戰古冶鄉及雙望,頗有斬獲。四月,樞輔孫承宗令可綱督諸將營雙望諸山,以綴永平之師。令大壽諸軍直趨灤州。灤州既復,大清兵棄永平去,可綱遂入其城。論功,加太子太保、左都督。已而錦州被圍,可綱督諸將赴救,立功郵馬山,復進秩。四年築城大凌河,命可綱偕大壽護版築。八月甫竣工,大清以十萬眾來攻,可綱等堅守不下。久之,糧盡援絕。大壽及諸將皆欲降,獨可綱不從,令二人掖出城外殺之,可綱顏色不變,亦不發一言,含笑而死。

黃龍,遼東人。初以小校從復錦州,積功至參將。崇禎三年從大軍復灤州,功第一,遷副總兵。尋論功進秩三等,為都督僉事,世廕副千戶。登萊巡撫孫元化以劉興治亂東江,請龍往鎮。兵部尚書梁廷棟亦薦龍為總兵官,與元化恢復四衛,從之。

先是,毛文龍死,袁崇煥分其兵二萬八千為四協,命副將陳繼盛,參將劉興治、毛承祚、徐敷奏主之。後改為兩協,繼盛領東協,興治攝西協。語詳《崇煥傳》。興治兇狡好亂,與繼盛不相能。其兄參將興祚陣亡,繼盛誤聽諜報,謂未死。興治憤,擇日為興祚治喪,諸將咸弔。繼盛至,伏兵執之,並執理餉經歷劉應鶴等十一人。袖出一書,宣於眾,詭言此繼盛誣興祚詐死,及以謀叛誣陷己者,遂殺繼盛及應鶴等。又偽為島中商民奏一通,請優恤興祚,而令興治鎮東江。舉朝大駭,以海外未遑詰也。興冶與諸弟兄放舟長山島,大肆殺掠。島去登州四十里。時登萊總兵官張可大赴援永平,帝用廷棟言,趣可大還登州,授副將周文郁大將印,令撫定興冶。會永平已復,興治稍戢,返東江。龍蒞皮島受事,興治猶桀驁如故。四年三月復作亂,杖其弟興基,殺參將沈世魁家眾。世魁率其黨夜襲殺興治,亂乃定。

遊擊耿仲明之黨李梅者,通洋事覺,龍繫之獄。仲明弟都司仲裕在龍軍,謀作亂。十月率部卒假索餉名圍龍署,擁至演武場,折股去耳鼻,將殺之。諸將為救免。未幾,龍捕斬仲裕,疏請正仲明罪。會元化劾龍剋餉致兵嘩,帝命充為事官,而核仲明主使狀。仲明遂偕孔有德反,以五年正月陷登州,招島中諸將。旅順副將陳有時、廣鹿島副將毛承祿皆往從之。龍急遣尚可喜、金聲桓等撫定諸島,而躬巡其地,慰商民,誅叛黨,縱火焚其舟。賊黨高成友者據旅順,斷關寧、天津援師。龍令遊擊李維鸞偕可喜等擊走之,即移駐其地,援始通。其冬,有德等欲棄登州走入海,龍遣副將龔正祥率舟師四千邀之廟島。颶風破舟,正祥陷賊中。後居登州,謀為內應,事露被殺。初,龍駐旅順大治兵。賊拘龍母妻及子以脅之,龍不顧。

六年二月,有德、仲明屢為巡撫朱大典所敗,航海遁去。龍度有德等必遁,遁必經旅順,邀擊之。有德幾獲而逸。斬賊魁李九成子應元,生擒毛承祿、蘇有功、陳光福及其黨高志祥等十六人,獲首級一千有奇,奪還婦女無算,獻俘於朝。帝大喜,磔承祿等,傳首九邊,復龍官。承祿,文龍族家子也。

有德等大憤,欲報龍。會賊舟泊鴨綠江,龍盡發水師剿之。七月,有德等偵知旅順空虛,遂引大清兵來襲。龍數戰皆敗,火藥矢石俱盡,語部將譚應華曰:「敵眾我寡,今夕城必破。若速持吾印送登州,不能赴,即投諸海可也。」應華出,龍率惟鸞等力戰。圍急,知不能脫,自剄死。惟鸞及諸將項祚臨、樊化龍、張大祿、尚可義俱死之。事聞,贈龍左都督,賜祭葬,予世廕,建祠曰:「顯忠」。惟鸞等附祀。以副總兵沈世魁代龍為總兵官。

世魁本市儈,其女有殊色,為毛文龍小妻。世魁倚勢橫行島中,至是為大帥。七年二月,廣鹿島副將尚可喜降於我大清,島中勢益孤。十年,朝鮮告急,世魁移師皮島為聲援。有德等來襲,世魁戰敗,率舟師走石城,副將金日觀陣歿。登萊總兵陳洪範來援,不戰而走。世魁亦陣亡,士卒死傷者萬餘。從子副將志科集潰卒至長城島,欲得世魁敕印。監軍副使黃孫茂不予,志科怒殺之,並殺理餉通判邵啟。副將白登庸遂率所部降大清。諸島雖有殘卒,不能成軍,朝廷亦不置大帥,以登萊總兵遙領之而已。明年夏,楊嗣昌決策盡徙其兵民寧、錦,而諸島一空。

金日觀,不知何許人。天啟五年以將才授守備,效力關門。擢鎮標中軍遊擊,加參將行薊鎮東路遊擊事,專領南兵。崇禎初,加副總兵,守馬蘭峪。三年正月,大清兵破京東列城。兵部侍郎劉之綸遣部將吳應龍等結營毛山,規取羅文谷關。師敗,日觀遣二將馳援,亦敗歿。大清兵乘勝據府君、玉皇二山,進攻馬蘭城甚急。日觀堅守,親然大炮。炮炸,焚頭目手足,意氣不衰。乞援於總理馬世龍。令參將王世選等赴救,兵乃退。尋復以二千餘騎來攻,日觀偕世選等死守不下。朝廷獎其功,驟加都督同知。四月,與副將謝尚政、曹文詔等攻復大安城,遂偕諸軍復遵化。錄功,進左都督。時總兵鄧轄馬蘭、松棚二路,日觀應受節制。以銜都督同知,不屑為之下。總督曹文衡劾日觀器小易盈,恃功驕縱,帝特戒飭而已。久之,移萊州副總兵。十年春,大清兵攻朝鮮,命從登萊總兵陳洪範往救,駐師皮島。大清遣孔有德、耿仲明、尚可喜等先攻鐵山。四月分兵攻皮島,水陸夾攻。副將白登庸先遁,洪範亦避走石城。登庸尋帥所部降。日觀偕諸將楚繼功等相持七晝夜,力不支,陣歿,島城隨破。贈特進光祿大夫、太子太師,世廕錦衣副千戶,建祠。繼功等贈恤有差。

贊曰:古人有言,彼且為我死,故我得與之俱生。故死封疆之臣,君子重之。觀遼左諸帥,委身許國,見危不避,可謂得死所者與!於時優恤之典非不甚渥,然而無救於危亡者,廟算不定,僨事者不誅,文墨議論之徒從而撓之,徒激勸忠義無益也。


\end{pinyinscope}