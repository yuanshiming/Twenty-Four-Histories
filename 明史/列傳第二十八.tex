\article{列傳第二十八}

\begin{pinyinscope}
魏觀陶垕仲王佑劉仕貆王溥徐均王宗顯王興宗呂文燧王興福蘇恭讓趙庭蘭王觀楊卓羅性道同歐陽銘盧熙兄熊王士弘倪孟賢郎敏青文勝

魏觀,字巳山,蒲圻人。元季隱居蒲山。太祖下武昌,聘授國子助教,再遷浙江按察司僉事。吳元年,遷兩淮都轉運使。入為起居注。奉命偕吳琳以幣帛求遺賢於四方。洪武元年,建大本堂,命侍太子說書及授諸王經。未幾,又命偕文原吉、詹同、吳輔、趙壽等分行天下,訪求遺才,所舉多擢用。三年,轉太常卿,攷訂諸祀典。稱旨,改侍讀學士,尋遷祭酒。明年坐考祀孔子禮不以時奏,謫知龍南縣,旋召為禮部主事。五年,廷臣薦觀才,出知蘇州府。前守陳寧苛刻,人呼「陳烙鐵」。觀盡改寧所為,以明教化、正風俗為治。建黌舍。聘周南老、王行、徐用誠,與教授貢潁之定學儀;王彞、高啟、張羽訂經史;耆民周壽誼、楊茂、林文友行鄉飲酒禮。政化大行,課績為天下最。明年擢四川行省參知政事。未行,以部民乞留,命還任。

初,張士誠以蘇州舊治為宮,遷府治於都水行司。觀以其地湫隘,還治舊基。又浚錦帆涇,興水利。或譖觀興既滅之基。帝使御史張度廉其事,遂被誅。帝亦尋悔,命歸葬。

陶垕仲,名鑄,以字行,鄞人。洪武十六年,以國子生擢監察御史。糾彈不避權貴。劾刑部尚書開濟至死,直聲動天下。未幾,擢福建按察使。誅贓吏數十人,興學勸士,撫恤軍民。帝下詔褒異。布政使薛大方貪暴,垕仲劾奏之。大方辭相連,並逮至京。訊實,坐大方罪,詔垕仲還官。垕仲言:「臣父昔為方氏部曲,以故官例徙鳳陽。臣幼弱,依兄撫養,至於有成。今兄亦為鳳陽軍吏。臣叨聖恩,備位司憲。欲推祿養報生育恩,使父母兄弟得復聚處,實戴聖天子孝治天下至意。」帝特許迎養,去徙籍。垕仲清介自持,祿入悉以贍賓客。未幾,卒官。

時廣西僉事王佑,泰和人。按察使尋適嘗咨以政體。佑曰:「蠻方之人瀆倫傷化,不及此時明禮法、示勸懲,後難治」適從之,廣西稱治。蜀平,徙佑知重慶州。招徠撫輯,甚得民和。坐事免官,卒。

劉仕貆,字伯貞,安福人。父閈,元末隱居不仕。仕貆少受父學。紅巾賊亂,掠其鄉,母張氏率群婦女沉茨潭死。賊械仕貆,久之得釋。洪武初,以供役為安福丞張禧所辱,仕貆憤,益力學。十五年應「賢良」舉,對策稱旨,授廣東按察司僉事,分司瓊州。瓊俗善蠱。上官至,輒致所產珍貨為贄。受則喜,不受則懼按治,蠱殺之。仕瓊者多為所汙。仕貆廉且惠,輕徭理枉,大得民和。雖卻其贄,夷人不忍害也。辱仕貆者張禧,適調丞瓊山,以屬吏謁,大慚怖。仁貆待之與他吏等。未幾,朝議省僉事官,例降東莞河泊使。渡河遇風,歿於水。同僚張仕祥葬之鴉磯。

後有王溥者,桂林人。洪武末為廣東參政,亦以廉名。其弟自家來省,屬吏與同舟,贈以布袍。溥命還之,曰:「一衣雖微,不可不慎,此汙行辱身之漸也。」糧運由海道多漂沒,溥至庾嶺,相度形勢,命有司鑿石填塹,修治橋梁,易以車運。民甚便之。居官數年,笥無重衣,庖無兼饌。以誣逮下詔獄,僚屬饋贐皆不受,曰:「吾豈以患難易其心哉!」事白得歸,卒。

時有徐均者,陽春主簿也。地僻,土豪得盤踞為姦。邑長至,輒餌以厚賂。從而把持之。均至,吏白:「應往視莫大老。」莫大老者,洞主也。均曰:「此非王民邪?不來且誅!」出雙劍示之。大老恐,入謁。均廉得其不法事,繫之獄。詰朝,以兩瓜及安石榴數枚為饋,皆黃金美珠也。均不視,械送府。府官受賕縱之歸,復致前饋。均怒,欲捕治之,而府檄調均攝陽江,陽江大治。以憂去官。

王宗顯,和州人,僑居嚴州。胡大海克嚴,禮致幕中。太祖征婺州,大海以宗顯見。太祖曰:「我鄉里也。」命至婺覘敵。宗顯潛得城中虛實及諸將短長,還白太祖。太祖喜曰:「我得婺,以爾為知府。」既而元樞密同僉寧安慶與守將帖木烈思貳,遣都事縋城請降,開東門納兵,與宗顯所刺事合。改婺州為寧越府,以宗顯知府事。宗顯故儒者,博涉經史。開郡學,聘葉儀、宋濂為《五經》師;戴良為學正;吳沉、徐源等為訓導。自兵興,學校久廢,至是始聞絃誦聲。未幾,卒官。

太祖之下婺也,又以王興宗為金華知縣。興宗,故隸人也,李善長、李文忠皆以為不可。太祖曰:「興宗從我久,勤廉能斷,儒生法吏莫先也。」居三年,果以治行聞。遷判南昌,改知嵩州。時方籍民為軍,興宗奏曰:「元末聚民為兵,散則仍為民。今軍民分矣,若籍為軍,則無民,何所征賦?」帝曰:「善。」遷懷慶知府。上計至京,帝以事詰諸郡守,至興宗,獨曰:「是守公勤不貪,不須問。」再遷蘇州,擢河南布政使。陛辭,帝曰:「久不見爾,老矣,我鬚亦白。」宴而遣之,益勤其職。後坐累得白,卒於官。

同時有呂文燧,字用明,永康人。元末盜起,文燧散家財,募壯士得三千人,與盜連戰,破走之。三授以官,皆不受。太祖定婺,置永康翼,以文燧為左副元帥兼知縣事。尋召為營田司經歷,擢知廬州府。浙西平,徙知嘉興。松江民作亂,寇嘉興,文燧柵內署,帥壯士拒守。李文忠援至,賊就擒,諸將因欲屠城。文燧曰:「作亂者賊也,民何罪?」力止之。滿三載,入朝。奉詔持節諭闍婆國,次興化,疾卒。明年,嘉興佐貳以下坐鹽法死者數十人,有司以文燧嘗署名公牘,請籍其家。帝曰:「文燧誠信,必不為姦利,且沒於使事,可念也,勿籍。」

一時郡守以治行稱者,又有王興福、蘇恭讓二人。

興福,隨人。初守徵州,有善政,遷杭州。杭初附,人心未安。興福善撫輯,民甚德之。秩滿當遷,郡人遮道攀留。興福諭遣之曰:「非余能惠父老,父老善守法耳。」太祖嘉之,擢吏部尚書。坐事左遷西安知府,卒官。

恭讓,玉田人。舉「聰明正直」。任漢陽知府,為治嚴明而不苛。有重役,輒詣上官反復陳說,多得減省。

而知漢陽縣者趙庭蘭,徐人。亦能愛民任事。朝廷嘗遣使征陳氏散卒,他縣多以民丁應,庭蘭獨言縣無有。漢陽人言郡守則稱「恭讓」,言縣令則稱「庭蘭」云。

王觀,字尚賓,祥符人。性耿介,儀度英偉,善談論。由鄉薦入太學,擢知蘇州府。公廉有威。黠吏錢英屢陷長官,觀捶殺之。事聞,太祖遣行人齎敕褒之,勞以御酒。歲大山昆,民多逋賦,部使者督甚急。觀置酒,延諸富人,勸貸貧民償,辭指誠懇,富人皆感動,逋賦以完。朝廷嘉其能,榜以勵天下。守蘇者前有季亨、魏觀,後有姚善、況鐘,皆賢,稱「姑蘇五太守」,並祀學宮。

楊卓,字自立,泰和人。洪武四年進士,授吏部主事。踰年,遷廣東行省員外郎。田家婦獨行山中,遇伐木卒,欲亂之。婦不從,被殺。官拷同役卒二十人,皆引服。卓曰::「卒人眾,必善惡異也,可盡抵罪乎?」列二十人庭下,熟視久之,指兩卒曰:「殺人者,汝也!」兩卒大驚,服罪。坐事謫田鳳陽,復起為杭州通判。有兄弟爭田者,累歲不決,卓至,垂涕開諭,遂罷爭。卓精吏事,吏不能欺。而治平恕,民悅服焉。病免,卒。

卓同邑羅性,字子理。洪武初舉於鄉,授德安同知。有大盜久不獲,株連繫獄者數百人。性至郡,悉出所繫。約十日得賊即盡貸。眾叩頭願盡力,七日果得。嘗治蔬圃,得窖鐵萬餘斤。會方賦鐵造軍器,民爭求售。性曰:「此天所以濟民也,吾何預焉。」悉以充賦。秩滿赴京,坐用棗木染軍衣,謫戍西安。性博學。時四方老師宿儒在西安者數十人,吳人鄒奕曰:「合吾輩所讀書,庶幾羅先生之半。」年七十卒。

道同,河間人。其先蒙古族也。事母以孝聞。洪武初,薦授太常司贊禮郎,出為番禺知縣。番禺故號「煩劇」,而軍衛尤橫,數鞭辱縣中佐吏,前令率不能堪。同執法嚴,非理者一切抗弗從,民賴以少安。

未幾,永嘉侯朱亮祖至,數以威福撼同,同不為動。土豪數十輩抑買市中珍貨,稍不快意,輒巧詆以罪。同械其魁通衢。諸豪家爭賄亮祖求免。亮祖置酒召同,從容言之。同厲聲曰:「公大臣,奈何受小人役使!」亮祖不能屈也。他日,亮祖破械脫之,借他事笞同。富民羅氏者,納女於亮祖,其兄弟因怙勢為奸。同復按治,亮祖又奪之去。同積不平,條其事奏之。未至,亮祖先劾同訕傲無禮狀。帝不知其由,遂使使誅同。會同奏亦至。帝悟,以為同職甚卑,而敢斥言大臣不法事,其人骨鯁可用。復使使宥之。兩使者同日抵番禺,後使者甫到,則同已死矣。縣民悼惜之,或刻木為主祀於家,卜之輒驗,遂傳同為神云。

當同未死時,布政使徐本雅重同。同方笞一醫未竟,而本急欲得醫,遣卒語同釋之。同岸然曰:「徐公乃亦效永嘉侯耶?」笞竟,始遣。自是上官益嚴憚,然同竟用此取禍。

先是有歐陽銘者,亦嘗以事抗將軍常遇春。

銘,字日新,泰和人。以薦除江都縣丞。兵燹後,民死徙者十七八。銘招徠拊循,漸次復業。有繼母告子不孝者,呼至案前,委曲開譬,母子泣謝去,卒以慈孝稱。嘗治廨後隙地,得白金百兩,會部符征漆,即市之以輸。遷知臨淄。遇春師過其境,卒入民家取酒,相毆擊,一市盡嘩。銘笞而遣之。卒訴令罵將軍,遇春詰之。曰;「卒,王師,民亦王民也。民毆且死,卒不當笞耶?銘雖愚,何至詈將軍?將軍大賢,奈何私一卒,撓國法?」遇春意解,為責軍士以謝。後大將軍徐達至,軍士相戒曰:「是健吏,曾抗常將軍者,毋犯也。」銘為治廉靜平恕,暇輒進諸生講文藝,或單騎行田間,課耕獲。邑大治。秩滿入覲,卒。

盧熙,字公暨,崑山人。兄熊,字公武,為兗州知府。時兵革甫定,會營魯王府。又浚河,大役並興。熊竭心調度,民以不擾。後坐累死。熙以薦授睢州同知。有惠愛,命行知府事。適御史奉命搜舊軍,睢民濫入伍者千人,檄熙追送。熙令民自實,得嘗隸尺籍者數人畀之。御史怒,繫曹吏,必盡得,不則以格詔論。同官皆懼。熙曰:「吾民牧也。民散,安用牧?」乃自詣御史曰:「州軍籍盡此矣。迫之,民且散,獨有同知在耳,請以充役。」御史怒斥去,堅立不動。已,知不能奪,乃罷去。後卒於官。貧不能喪,官為具殮。喪歸,吏民挽哭者塞道,大雨,無一人卻者。

又王士弘者,知寧海縣。靖海侯吳禎奉命收方氏故卒。無賴子誣引平民,台、溫騷然。士弘上封事,辭極懇切。詔罷之,民賴以安。

倪孟賢,南昌人。知麗水縣。民有賣卜者,干富室不應,遂詣京告大姓陳公望等五十七人謀亂。命錦衣衛千戶周原往捕之。孟賢廉得實,謂僚屬曰:「朝廷命孟賢令是邑,忍坐視善良者橫被茶毒耶?」即具疏聞。復令耆老四十人赴闕訴。下法司鞫實,論告密者如律。

又樂平奸民亦詣闕訴大姓五十餘家謀逆,饒州知州郎敏力為奏辨。詔誅奸民,而被誣者得盡釋。

青文勝,字質夫,夔州人。仕為龍陽典史。龍陽瀕洞庭,歲罹水患,逋賦數十萬,敲撲死者相踵。文勝慨然詣闕上疏,為民請命。再上,皆不報。歎曰:「何面目歸見父老!」復具疏,擊登聞鼓以進,遂自經於鼓下。帝聞大驚,憫其為民殺身,詔寬龍陽租二萬四千餘石,定為額。邑人建祠祀之。妻子貧不能歸,養以公田百畝。萬曆十四年,詔有司春秋致祭,名其祠曰「惠烈」。

贊曰:太祖起閭右,稔墨吏為民害,嘗以極刑處之。然每旌舉賢能,以示勸勉,不專任法也。嘗遣行人齎敕併鈔三十錠、內酒一尊,賜平陽知縣張礎。又建陽知縣郭伯泰、丞陸鎰,為政不避權勢,遣使勞以酒醴,遷其官。丹徒知縣胡夢通、丞郭伯高,金壇丞李思進,坐事當逮,民詣闕,言多善政。帝並賜內尊,降敕褒勞。永州守餘彥誠、齊東令鄭敏等十人坐事下獄,部民列政績以請,皆復官。宜春令沈昌等四人更擢郡守。其自下僚不次擢用者,寧遠尉王尚賢為廣西參政,祥符丞鄒俊為大理卿,靜寧州判元善為僉都御史,芝陽令李行素為刑部侍郎。至如懷寧丞陳希文、宜興簿王復春,先以善政擢,已知其貪肆,旋置重典。所以風厲激勸者甚至,以故其時吏治多可紀述云。


\end{pinyinscope}