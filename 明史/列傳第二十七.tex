\article{列傳第二十七}

\begin{pinyinscope}
錢唐(程徐韓宜可(周觀政歐陽韶蕭岐(門克新))馮堅茹太素曾秉正李仕魯陳汶輝葉伯巨鄭士利方徵周敬心王朴

錢唐,字惟明,象山人。博學敦行。洪武元年,舉明經。對策稱旨,特授刑部尚書。二年詔孔廟春秋釋奠,止行於曲阜,天下不必通祀。唐伏闕上疏言:「孔子垂教萬世,天下共尊其教,故天下得通祀孔子,報本之禮不可廢。」侍郎程徐亦疏言:「古今祀典,獨社稷、三皇與孔子通祀。天下民非社稷、三皇則無以生,非孔子之道則無以立。堯、舜、禹、湯、文、武、周公,皆聖人也。然發揮三綱五常之道,載之於經,儀範百王,師表萬世,使世愈降而人極不墜者,孔子力也。孔子以道設教,天下祀之,非祀其人,祀其教也,祀其道也。今使天下之人,讀其書,由其教,行其道,而不得舉其祀,非所以維人心、扶世教也。」皆不聽。久之,乃用其言。帝嘗覽《孟子》,至「草芥」「寇仇」語,謂:「非臣子所宜言」,議罷其配享。詔:「有諫者以大不敬論。」唐抗疏入諫曰:「臣為孟軻死,死有餘榮。」時廷臣無不為唐危。帝鑒其誠懇,不之罪。孟子配享亦旋復。然卒命儒臣修《孟子節文》云。

唐為人彊直。嘗詔講《虞書》,唐陛立而講。或糾唐草野不知君臣禮,唐正色曰:「以古聖帝之道陳於陛下,不跪不為倨。」又嘗諫宮中不宜揭武后圖。忤旨,待罪午門外竟日。帝意解,賜之食,即命撤圖。未幾,謫壽州,卒。

程徐,字仲能,鄞人。元名儒端學子也。至正中,以明《春秋》知名。歷官兵部尚書,致仕。明兵入元都,妻金抱二歲兒與女瓊赴井死。洪武二年,偕危素等自北平至京。授刑部侍郎,進尚書,卒。徐精勤通敏,工詩文,有集傳於世。

韓宜可,字伯時,浙江山陰人。元至正中,行御史臺辟為掾,不就。洪武初,薦授山陰教諭,轉楚府錄事。尋擢監察御史,彈劾不避權貴。時丞相胡惟庸、御史大夫陳寧、中丞塗節方有寵於帝,嘗侍坐,從容燕語。宜可直前,出懷中彈文,劾三人險惡似忠,奸佞似直,恃功怙寵,內懷反側,擢置臺端,擅作威福,乞斬其首以謝天下。帝怒曰:「快口御史,敢排陷大臣耶!」命下錦衣衛獄,尋釋之。

九年出為陜西按察司僉事。時官吏有罪者,笞以上悉謫屯鳳陽,至萬數。宜可疏,爭之曰:「刑以禁淫慝,一民軌,宜論其情之輕重,事之公私,罪之大小。今悉令謫屯,此小人之幸,君子殆矣。乞分別,以協眾心。」帝可之。已,入朝京師。會賜諸司沒官男女,宜可獨不受。且極論:「罪人不孥,古之制也。有事隨坐,法之濫也。況男女,人之大倫,婚姻踰時,尚傷和氣。合門連坐,豈聖朝所宜!」帝是其言。後坐事將刑,御謹身殿親鞫之,獲免。復疏,陳二十餘事,皆報可。未幾,罷歸。已,復徵至。命撰祀鐘山、大江文;諭日本、征烏蠻詔,皆稱旨,特授山西右布政使。尋以事安置雲南。惠帝即位,用檢討陳性善薦,起雲南參政,入拜左副都御史,卒於官。是夜大星隕,櫪馬皆驚嘶,人謂:「宜可當之」云。

帝之建御史臺也,諸御史以敢言著者,自宜可外,則稱周觀政。

觀政亦山陰人。以薦授九江教授,擢監察御史。嘗監奉天門。有中使將女樂入,觀政止之。中使曰:「有命」,觀政執不聽。中使慍而入,頃之出報曰:「御史且休,女樂已罷不用。」觀政又拒曰:「必面奉詔。」已而帝親出宮,謂之曰:「宮中音樂廢缺,欲使內家肄習耳。朕已悔之,御史言是也。」左右無不驚異者。觀政累官江西按察使。

前觀政者,有歐陽韶,字子韶,永新人。薦授監察御史。有詔:日命兩御史侍班。韶嘗侍直,帝乘怒將戮人。他御史不敢言,韶趨跪殿廷下,倉卒不能措詞,急捧手加額,呼曰:「陛下不可!」帝察韶朴誠,從之。未幾,致仕,卒於家。

蕭岐,字尚仁,泰和人。五歲而孤,事祖父母以孝聞。有司屢舉不赴。洪武十七年,詔征賢良,強起之。上十便書,大意謂:帝刑罰過中,訐告風熾。請禁止實封以杜誣罔;依律科獄以信詔令。凡萬餘言。召見,授潭王府長史。力辭,忤旨,謫雲南楚雄訓導。岐即日行,遣騎追還。歲餘,改授陜西平涼。再歲致仕。復召與錢宰等考定《書》傳。賜幣鈔,給驛歸。嘗輯《五經要義》;又取《刑統八韻賦》,引律令為之解,合為一集。嘗曰:「天下之理本一,出乎道必入乎刑。吾合二書,使觀者有所省也。」學者稱「正固先生」。

當是時,太祖治尚剛嚴,中外凜凜,奉法救過不給。而岐所上書過切直,帝不為忤。厥後以言被超擢者,有門克新。

克新,鞏昌人。泰州教諭也。二十六年,秩滿來朝。召問經史及政治得失。克新直言無隱。授贊善。時紹興王俊華以善文辭,亦授是職。上諭吏部曰:「左克新,右俊華,重直言也。」初,教官給由至京,帝詢民疾苦。岢嵐吳從權、山陰張桓皆言:「臣職在訓士,民事無所與。」帝怒曰:「宋胡瑗為蘇、湖教授,其教兼經義治事;漢賈誼、董仲舒皆起田里,敷陳時務;唐馬周不得親見太宗,且教武臣言事。今既集朝堂,朕親詢問,俱無以對,志聖賢之道者固如是乎!」命竄之邊方。且榜諭天下學校,使為鑒戒。至是克新以亮直見重。不數年,擢禮部尚書。尋引疾,命太醫給藥物,不輟其奉。及卒,命有司護喪歸葬。

馮堅,不知何許人,為南豐典史。洪武二十四年上書言九事:「一曰養聖躬。請清心省事,不與細務,以為民社之福。二曰擇老成。諸王年方壯盛,左右輔導。願擇取老成之臣出為王官,使得直言正色,以圖匡救。三曰攘要荒。請務農講武,屯戍邊圉,以備不虞。四曰勵有司。請得廉正有守之士,任以方面。旌別屬吏,具實以聞而黜陟之。使人勇於自治。五曰褒祀典。請敕有司採歷代忠烈諸臣,追加封謚,俾末俗有所興勸。六曰省宦寺。晨夕密邇,其言易入,養成禍患而不自知。請裁去冗員,可杜異日陵替之弊。七曰易邊將。假以兵柄,久在邊圉,多致縱佚。請時遷歲調,不使久居其任。不惟保全勳臣,實可防將驕卒惰、內輕外重之漸。八曰訪吏治。廉幹之才,或為上官所忌,僚吏所嫉。上不加察,非激勸之道。請廣布耳目,訪察廉貪,以明黜陟。九曰增關防。諸司以帖委胥吏,俾督所部,輒加箠楚,害及於民。請增置勘合以付諸司,聽其填寫差遣,事訖繳報,庶所司不輕發以病民,而庶務亦不致曠廢。」書奏,帝嘉之,稱其知時務,達事變。又謂侍臣曰:「堅言惟調易邊將則未然。邊將數易,則兵力勇怯。敵情出沒,出川形勝,無以備知。倘得趙充國、班超者,又何取數易為哉!」乃命吏部擢堅左僉都御史,在院頗持大體。其明年,卒於任。

茹太素,澤州人。洪武三年,鄉舉,上書稱旨,授監察御史。六年擢四川按察使,以平允稱。七年五月召為刑部侍郎,上言:「自中書省內外百司,聽御史、按察使檢舉。而御史臺未有定考,宜令守院御史一體察核。磨勘司官吏數少,難以檢核天下錢糧,請增置若干員,各分為科。在外省衛,凡會議軍民事,各不相合,致稽延。請用按察司一員糾正。」帝皆從之。明年,坐累降刑部主事。陳時務累萬言,太祖令中書郎王敏誦而聽之。中言:「才能之士,數年來幸存者百無一二,今所任率迂儒俗吏。」言多忤觸。帝怒,召太素面詰,杖於朝。次夕,復於宮中令人誦之,得其可行者四事。慨然曰:「為君難,為臣不易。朕所以求直言,欲其切於情事。文詞太多,便至熒聽。太素所陳,五百餘言可盡耳。」因令中書定奏對式,俾陳得失者無繁文。摘太素疏中可行者下所司,帝自序其首,頒示中外。

十年,與同官曾秉正先後同出為參政,而太素往浙江。尋以侍親賜還里。十六年召為刑部試郎中。居一月,遷都察院僉都御史。復降翰林院檢討。十八年九月擢戶部尚書。

太素抗直不屈,屢瀕於罪,帝時宥之。一日,宴便殿,賜之酒曰:「金盃同汝飲,白刃不相饒。」太素叩首,即續韻對曰:「丹誠圖報國,不避聖心焦。」帝為惻然。未幾,謫御史,復坐排陷詹徽,與同官十二人俱鐐足治事。後竟坐法死。

曾秉正,南昌人。洪武初,薦授海州學正。九年,以天變詔群臣言事。秉正上疏數千言,大略曰:「古之聖君不以天無災異為喜,惟以祗懼天譴為心。陛下聖文神武,統一天下,天之付與,可謂盛矣。兵動二十餘年,始得休息。天之有心於太平亦已久矣;民之思治亦切矣。創業與守成之政,大抵不同。開創之初,則行富國強兵之術,用趨事赴功之人。大統既立,邦勢已固。則普天之下,水土所生,人力所成,皆邦家倉庫之積;乳哺之童,垂白之叟,皆邦家休養之人。不患不富庶,惟保成業於永久為難耳。於此之時,當盡革向之所為,何者足應天心,何者足慰民望,感應之理,其效甚速。」又言天既有警,則變不虛生。極論《大易》、《春秋》之旨。帝嘉之,召為思文監丞。未幾,改刑部主事。十年擢陜西參政。會初置通政司,即以秉正為使。在位數言事,帝頗優容之。尋竟以忤旨罷。貧不能歸,鬻其四歲女。帝聞大怒,置腐刑,不知所終。

李仕魯,字宗孔,濮人。少穎敏篤學,足不窺戶外者三年。聞鄱陽朱公遷得宋朱熹之傳,往從之遊,盡受其學。太祖故知仕魯名,洪武中,詔求能為朱氏學者,有司舉仕魯。入見,太祖喜曰:「吾求子久,何相見晚也!」除黃州同知。曰:「朕姑以民事試子,行召子矣。」期年,治行聞。十四年,命為大理寺卿。

帝自踐阼後,頗好釋氏教。詔征東南戒德僧,數建法會於蔣山。應對稱旨者輒賜金示闌袈裟衣,召入禁中,賜坐與講論。吳印、華克勤之屬,皆拔擢至大官,時時寄以耳目。由是其徒橫甚,讒毀大臣。舉朝莫敢言,惟仕魯與給事中陳汶輝相繼爭之。汶輝疏言:「古帝王以來,未聞縉紳緇流,雜居同事,可以相濟者也。今勳舊耆德咸思辭祿去位,而緇流憸夫乃益以讒間。如劉基、徐達之見猜,李善長、周德興之被謗,視蕭何、韓信,其危疑相去幾何哉?伏望陛下於股肱心膂,悉取德行文章之彥,則太平可立致矣。」帝不聽。諸僧怙寵者,遂請為釋氏創立職官。於是以先所置善世院為僧錄司。設左、右善世、左、右闡教、左、右講經覺義等官,皆高其品秩。道教亦然。度僧尼道士至踰數萬。仕魯疏言:「陛下方創業,凡意指所向,即示子孫萬世法程,奈何捨聖學而崇異端乎!」章數十上,亦不聽。

仕魯性剛介,由儒術起,方欲推明朱氏學,以闢佛自任。及言不見用,遽請於帝前,曰:「陛下深溺其教,無惑乎臣言之不入也!還陛下笏,乞賜骸骨歸田里。」遂置笏於地。帝大怒,命武士捽搏之,立死階下。

陳汶輝,字耿光,詔安人。以薦授禮科給事中,累官至大理寺少卿。數言得失,皆切直。最後忤旨,懼罪,投金水橋下死。

仕魯與汶輝死數歲,帝漸知諸僧所為多不法,有詔清理釋道二教云。

葉伯巨,字居升,寧海人。通經術。以國子生授平遙訓導。洪武九年星變,詔求直言。伯巨上書,略曰:

臣觀當今之事,太過者三:分封太侈也,用刑太繁也,求治太速也。

先王之制,大都不過三國之一,上下等差,各有定分,所以強幹弱枝,遏亂源而崇治本耳。今裂土分封,使諸王各有分地,蓋懲宋、元孤立,宗室不競之弊。而秦、晉、燕、齊、梁、楚、吳、蜀諸國,無不連邑數十。城郭宮室亞於天子之都,優之以甲兵衛士之盛。臣恐數世之後,尾大不掉,然後削其地而奪之權,則必生觖望。甚者緣間而起,防之無及矣。議者曰:『諸王皆天子骨肉,分地雖廣,立法雖侈,豈有抗衡之理?』臣竊以為不然。何不觀於漢、晉之事乎?孝景,高帝之孫也;七國諸王,皆景帝之同祖父兄弟子孫也。一削其地,則遽構兵西向。晉之諸王,皆武帝親子孫也,易世之後,迭相攻伐,遂成劉、石之患。由此言之,分封踰制,禍患立生。援古證今,昭昭然矣。此臣所以為太過者也。

昔賈誼勸漢文帝,盡分諸國之地,空置之以待諸王子孫。向使文帝早從誼言,則必無七國之禍。願及諸王未之國之先,節其都邑之制,減其衛兵,限其疆理,亦以待封諸王之子孫。此制一定,然後諸王有賢且才者入為輔相,其餘世為籓屏,與國同休。割一時之恩,制萬世之利,消天變而安社稷,莫先於此。

臣又觀歷代開國之君,未有不以任德結民心,以任刑失民心者。國祚長短,悉由於此。古者之斷死刑也,天子撤樂減膳,誠以天生斯民,立之司牲,固欲其並生,非欲其即死。不幸有不率教者入於其中,則不得已而授之以刑耳。議者曰:宋、元中葉,專事姑息,賞罰無章,以致亡滅。主上痛懲其弊,故制不宥之刑,權神變之法,使人知懼而莫測其端也。臣又以為不然。開基之主垂範百世,一動一靜,必使子孫有所持守。況刑者,民之司命,可不慎歟!夫笞、杖、徒、流、死,今之五刑也。用此五刑,既無假貸,一出乎大公至正可也。而用刑之際,多裁自聖衷,遂使治獄之吏務趨求意旨。深刻者多功,平反者得罪。欲求治獄之平,豈易得哉!近者特旨,雜犯死罪,免死充軍。又刪定舊律諸則,減宥有差矣。然未聞有戒敕治獄者務從平恕之條。是以法司猶循故例。雖聞寬宥之名,未見寬宥之實。所謂實者,誠在主上,不在臣下也。故必有罪疑惟輕之意,而後好生之德洽於民心,此非可以淺淺期也。

何以明其然也?古之為士者,以登仕為榮,以罷職為辱。今之為士者,以溷跡無聞為福,以受玷不錄為幸,以屯田工役為必獲之罪,以鞭笞捶楚為尋常之辱。其始也,朝廷取天下之士,網羅捃摭,務無餘逸。有司敦迫上道,如捕重囚。比到京師,而除官多以貌選。所學或非其所用,所用或非其所學。洎乎居官,一有差跌,茍免誅戮,則必在屯田工役之科。率是為常,不少顧惜,此豈陛下所樂為哉?誠欲人之懼而不敢犯也。竊見數年以來,誅殺亦可謂不少矣,而犯者相踵。良由激勸不明,善惡無別。議賢議能之法既廢,人不自勵,而為善者怠也。有人於此,廉如夷、齊,智如良、平,少戾於法。上將錄長棄短而用之乎?將舍其所長、苛其所短而置之法乎?茍取其長而舍其短,則中庸之材爭自奮於廉智。倘苛其短而棄其長,則為善之人皆曰:某廉若是,某智若是,朝廷不少貸之,吾屬何所容其身乎!致使朝不謀夕,棄其廉恥,或事掊克,以備屯田工役之資者,率皆是也。若是非用刑之煩者乎?

漢嘗徙大族於山陵矣,未聞實之以罪人也。今鳳陽皇陵所在,龍興之地,而率以罪人居之,怨嗟愁苦之聲充斥園邑,殆非所以恭承宗廟意也。且夫強敵在前,則揚精鼓銳,攻之必克,擒之必獲,可也。今賊突竄山谷,以計求之,庶或可得。顧勞重兵,彼方驚散,入不可蹤跡之地。捕之數年,既無其方,而乃歸咎於新附戶籍之細民,而遷徙之。騷動數千里之地,室家不得休居,雞犬不得寧息。況新附之眾,向者流移他所,朝廷許其復業。今附籍矣,而又復遷徙,是法不信於民也。夫戶口盛而後田野闢,賦稅增。今責守令年增戶口,正為是也。近者已納稅糧之家,雖承旨分釋還家,而其心猶不自安。已起戶口,雖蒙憐恤,而猶見留開封祗候。訛言驚動,不知所出。況太原諸郡,外界邊境,民心如此,甚非安邊之計也。臣願自今朝廷宜存大體,赦小過。明詔天下,修舉「八議」之法,嚴禁深刻之吏。斷獄平允者超遷之,殘酷裒斂者罷黜之。鳳陽屯田之制,見在居屯者,聽其耕種起科。已起戶口、見留開封者,悉放復業。如此則足以隆好生之德,樹國祚長久之福。而兆民自安,天變自消矣。

昔者周自文、武至於成、康,而教化大行;漢自高帝至於文、景,而始稱富庶。蓋天下之治亂,氣化之轉移,人心之趨向,非一朝一夕故也。今國家紀元,九年於茲,偃兵息民,天下大定。紀綱大正,法令修明,可謂治矣。而陛下切切以民俗澆漓,人不知懼,法出而奸生,令下而詐起。故或朝信而幕猜者有之;昨日所進,今日被戮者有之。乃至令下而尋改,已赦而復收。天下臣民莫之適從。臣愚謂天下之趨於治,猶堅冰之泮也。冰之泮,非太陽所能驟致。陽氣發生,土脈微動,然後得以融釋。聖人之治天下,亦猶是也。刑以威之,禮以導之,漸民以仁,摩民以義,而後其化熙熙。孔子曰:「如有王者,必世而後仁。」此非空言也。

求治之道,莫先於正風俗;正風俗之道,莫先於守令知所務;使守令知所務,莫先於風憲知所重;使風憲知所重,莫先於朝廷知所尚。古郡守、縣令,以正率下,以善導民,使化成俗美。征賦、期會、獄訟、簿書,固其末也。今之守令以戶口、錢糧、獄論為急務;至於農桑、學校,王政之本,乃視為虛文而置之,將何。以教養斯民哉?以農桑言之:方春州縣下一白帖,里甲回申文狀而已,守令未嘗親視種藝次第、旱澇戒備之道也。以學校言之:廩膳諸生,國家資之以取人才之地也。今四方師生,缺員甚多。縱使具員,守令亦鮮有以禮讓之實作其成器者。朝廷切切於社學,屢行取勘師生姓名、所習課業。乃今社鎮城郭,或但置立門牌,遠村僻處則又徒存其名,守令不過具文案、備照刷而已。上官分部按臨,亦但循習故常,依紙上照刷,未嘗巡行點視也。興廢之實,上下視為虛文。小民不知孝弟忠信為何物,而禮義廉恥掃地矣。風紀之司,所以代朝廷宣導德化,訪察善惡。聽訟讞獄,其一事耳。今專以獄訟為要。忠臣、孝子、義夫、節婦,視為末節而不暇舉,所謂宣導風化者安在哉?其始但知以去一贓吏、決一獄訟為治,而不知勸民成俗,使民遷善遠罪,乃治之大者。此守令風憲未審輕重之失也。

《王制》論鄉秀士升於司徒曰「選士」,司徒論其秀士而升於太學曰「俊士」,大樂正又論造士之秀升之司馬曰「進士」,司馬辨論官材,論定,然後官之;任官,然後爵之。其考之之詳若此,故成周得人為盛。今使天下諸生考於禮部,升於太學,歷練眾職,任之以事,可以洗歷代舉選之陋,上法成周。然而升於太學者,或未數月,遽選入官,間或委以民社。臣恐其人未諳時務,未熟朝廷禮法,不能宣導德化,上乖國政,而下困黎民也。開國以來,選舉秀才不為不多,所任名位不為不重,自今數之,在者有幾?臣恐後之視今,亦猶今之視昔。昔年所舉之人,豈不深可痛惜乎!凡此皆臣所為求治太速之過也。

昔者宋有天下蓋三百餘年。其始,以禮義教其民,當其盛時,閭閻里巷皆有忠厚之風,至於恥言人之過失。洎乎末年,忠臣義士視死如歸,婦人女子羞被污辱,此皆教化之效也。元之有國,其本不立,犯禮義之分,壞廉恥之防。不數十年,棄城降敵者不可勝數,雖老儒碩臣甘心屈辱。此禮義廉恥不振之弊。遺風流俗至今未革,深可怪也。臣謂:莫若敦仁義,尚廉恥。守令則責其以農桑、學校為急,風憲則責其先教化、審法律,以平獄緩刑為急。如此,則德澤下流,求治之道庶幾得矣。郡邑諸生升於太學者,須令在學肄業,或三年,或五年,精通一經,兼習一藝,然後入選。或宿衛,或辦事,以觀公卿大夫之能,而後任之以政,則其學識兼懋,庶無敗事。且使知祿位皆天之祿位,而可以塞凱覦之心也。治道既得,陛下端拱穆清,待以歲月,則陰陽調而風雨時,諸福吉祥莫不畢至。尚何天變之不消哉?

書上,帝大怒曰:「小子間吾骨肉,速逮來,吾手射之!」既至,丞相乘帝喜以奏,下刑部獄。死獄中。

先是,伯巨將上書,語其友曰:「今天下惟三事可患耳,其二事易見而患遲,其一事難見而患速。縱無明詔,吾猶將言之,況求言乎。」其意蓋謂分封也。然是時諸王止建籓號,未曾裂土,不盡如伯巨所言。迨洪武末年,燕王屢奉命出塞,勢始強。後因削奪稱兵,遂有天下,人乃以伯巨為先見云。

鄭士利,字好義,寧海人。兄士元,剛直有才學,由進士歷官湖廣按察使僉事。荊、襄卒乘亂掠婦女,吏不敢問,士元立言於將領,還所掠。安陸有冤獄,御史臺已讞上,士元奏其冤,得白。會考校錢穀冊書,空印事覺。凡主印者論死,佐貳以下榜一百,戍遠方。士元亦坐是繫獄。時帝方盛怒,以為欺罔,丞相御史莫敢諫。士利歎曰:「上不知,以空印為大罪。誠得人言之,上聖明,寧有不悟?」會星變求言。士利曰:「可矣。」既而讀詔:「有假公言私者,罪。」士利曰:「吾所欲言,為天子殺無罪者耳。吾兄非主印者,固當出。需吾兄杖出乃言,即死不恨。」

士元出,士利乃為書數千言,言數事,而於空印事尤詳。曰:「陛下欲深罪空印者,恐奸吏得挾空印紙,為文移以虐民耳。夫文移必完印乃可。今考較書策,乃合兩縫印,非一印一紙比。縱得之,亦不能行,況不可得乎?錢穀之數,府必合省,省必合部,數難懸決,至部乃定。省府去部遠者六七千里,近亦三四千里,冊成而後用印,往返非期年不可。以故先印而後書。此權宜之務,所從來久,何足深罪?且國家立法,必先明示天下而後罪犯法者,以其故犯也。自立國至今,未嘗有空印之律。有司相承,不知其罪。今一旦誅之,何以使受誅者無詞?朝廷求賢士,置庶位,得之甚難。位至郡守,皆數十年所成就。通達廉明之士,非如草菅然,可刈而復生也。陛下奈何以不足罪之罪,而壞足用之材乎?臣竊為陛下惜之。」書成,閉門逆旅泣數日。兄子問曰:「叔何所苦?」士利曰:「吾有書欲上,觸天子怒,必受禍。然殺我,生數百人,我何所恨!」遂入奏。帝覽書,大怒,下丞相御史雜問,究使者。士利笑曰;「顧吾書足用否耳。吾業為國家言事,自分必死,誰為我謀?」獄具,與士元皆輸作江浦,而空印者竟多不免。

方徵,字可久,莆田人。以鄉舉授給事中。嘗侍遊後苑,與聯詩句。太祖知其有母在,賜白金,馳驛歸省。還改監察御史,出為懷慶知府。徵志節甚偉,遇事敢直言。居郡時,因星變求言,疏言:「風憲官以激濁揚清為職。今不聞旌廉拔能,專務羅織人罪,多征贓罰,此大患也。朝廷賞罰明信,乃能勸懲。去年各行省官吏以用空印罹重罪,而河南參政安然、山東參政朱芾俱有空印,反遷布政使,何以示勸懲?」帝問羅織及多征贓罰者為誰,徵指河南僉事彭京以對。貶沁陽驛丞。十三年,以事逮至京,卒。

周敬心,山東人,太學生也。洪武二十五年,詔求曉曆數者,敬心上疏極諫,且及時政數事。略曰:

臣聞國祚長短,在德厚薄,不在曆數。三代尚矣,三代而下,最久莫如漢、唐、宋,最短莫如秦、隋、五代。其久也以有道,其短也以無道。陛下膺天眷命,救亂誅暴。然神武威斷則有餘,寬大忠厚則不足。陛下若效兩漢之寬大,唐、宋之忠厚,講三代所以有道之長,則帝王之祚可傳萬世,何必問諸小道之人耶?

臣又聞陛下連年遠征,北出沙漠,為恥不得傳國璽耳。昔楚平王時,琢卞和之玉,至秦始名為「璽」,歷代遞嬗,以訖後唐。治亂興廢,皆不在此。石敬瑭亂,潞王攜以自焚,則秦璽固已毀矣。敬瑭入洛,更以玉製。晉亡入遼,遼亡遺於桑乾河。元世祖時,札剌爾者漁而得之。今元人所挾,石氏璽耳。昔者三代不知有璽,仁為之璽,故曰「聖人大寶曰位,何以守位曰仁。」陛下奈何忽天下之大璽,而求漢、唐、宋之小璽也?

方今力役過煩,賦斂過厚。教化溥而民不悅;法度嚴而民不從。昔汲黯言於武帝曰:「陛下內多欲而外施仁義,奈何欲效唐、虞之治乎?」方今國則願富,兵則願強,城池則願高深,宮室則願壯麗,土地則願廣,人民則願眾。於是多取軍卒,廣籍資財,征伐不休,營造無極,如之何其可治也?臣又見洪武四年錄天下官吏,十三年連坐胡黨,十九年逮官吏積年為民害者,二十三年罪妄言者。大戮官民,不分臧否。其中豈無忠臣、烈士、善人、君子?於茲見陛下之薄德而任刑矣。水旱連年,夫豈無故哉!

言皆激切。報聞。

王朴,同州人。洪武十八年進士。本名權,帝為改焉。除吏科給事中,以直諫忤旨罷。旋起御史。陳時事千餘言。性鯁直,數與帝辨是非,不肯屈。一日,遇事爭之彊。帝怒,命戮之。及市,召還,諭之曰:「汝其改乎?」朴對曰:「陛下不以臣為不肖,擢官御史,奈何摧辱至此!使臣無罪,安得戮之?有罪,又安用生之?臣今日願速死耳。」帝大怒,趣命行刑。過史館,大呼曰:「學士劉三吾志之:某年月日,皇帝殺無罪御史朴也!」竟戮死。帝撰《大誥》,謂朴誹謗,猶列其名。

有張衡者,萬安人,朴同年進士。授禮科給事中。奏疏剴切。擢禮部侍郎。以清慎見褒,載於《大誥》。後亦以言事坐死。

贊曰:太祖英武威斷,廷臣奏對,往往失辭。而錢唐、韓宜可、李仕魯輩,抱其朴誠,力諍於堂陛間,可謂古之遺直矣。伯巨、敬心以縫掖諸生,言天下至計,雖違於信而後諫之義,然原厥本心,由於忠愛。以視末季沽名賣直之流,有不可同日而語者也。


\end{pinyinscope}