\article{列傳第一百七十三 文苑一}

\begin{pinyinscope}
明初,文學之士承元季虞、柳、黃、吳之後,師友講貫,學有本原。宋濂、王禕、方孝孺以文雄,高、楊、張、徐、劉基、袁凱以詩著。其他勝代遺逸,風流標映,不可指數,蓋蔚然稱盛已。永、宣以還,作者遞興,皆沖融演迤,不事鉤棘,而氣體漸弱。弘、正之間,李東陽出入宋、元,溯流唐代,擅聲館閣。而李夢陽、何景明倡言復古,文自西京、詩自中唐而下,一切吐棄,操觚談藝之士翕然宗之。明之詩文,於斯一變。迨嘉靖時,王慎中、唐順之輩,文宗歐、曾,詩仿初唐。李攀龍、王世貞輩,文主秦、漢,詩規盛唐。王、李之持論,大率與夢陽、景明相倡和也。歸有光頗後出,以司馬、歐陽自命,力排李、何、王、李,而徐渭、湯顯祖、袁宏道、鐘惺之屬,亦各爭鳴一時,於是宗李、何、王、李者稍衰。至啟、禎時,錢謙益、艾南英準北宋之矩矱,張溥、陳子龍擷東漢之芳華,又一變矣。有明一代,文士卓卓表見者,其源流大抵如此。今博考諸家之集,參以眾論,錄其著者,作《文苑傳》。

○楊維楨陸居仁錢惟善胡翰蘇伯衡王冕郭奎劉炳戴良王逢丁鶴年危素張以寧石光霽秦裕伯趙壎宋僖等徐一夔趙捴謙樂良等陶宗儀顧德輝等袁凱高啟楊基等王行唐肅宋克等孫蕡王佐等王蒙郭傳

楊維楨,字廉夫,山陰人。母李,夢月中金錢墜懷,而生維楨。少時,日記書數千言。父宏,築樓鐵崖山中,繞樓植梅百株,聚書數萬卷,去其梯,俾誦讀樓上者五年,因自號鐵崖。元泰定四年成進士,署天台尹,改錢清場鹽司令。狷直忤物,十年不調。會修遼、金、宋三史成,維楨著《正統辯》千餘言,總裁官歐陽元功讀且嘆曰:「百年後,公論定於此矣。」將薦之而不果,轉建德路總管府推官。擢江西儒學提舉,未上,會兵亂,避地富春山,徙錢塘。張士誠累招之,不赴,遣其弟士信咨訪之,因撰五論,具書復士誠,反覆告以順逆成敗之說,士誠不能用也。又忤達識丞相,徙居松江之上,海內薦紳大夫與東南才俊之士,造門納履無虛日。酒酣以往,筆墨橫飛。或戴華陽巾,披羽衣坐船屋上,吹鐵笛,作《梅花弄》。或呼侍兒歌《白雪》之辭,自倚鳳琶和之。賓客皆蹁躚起舞,以為神仙中人。

洪武二年,太祖召諸儒纂禮樂書,以維楨前朝老文學,遣翰林詹同奉幣詣門,維楨謝曰:「豈有老婦將就木,而再理嫁者邪?」明年,復遣有司敦促,賦《老客婦謠》一章進御,曰:「皇帝竭吾之能,不強吾所不能則可,否則有蹈海死耳。」帝許之,賜安車詣闕廷,留百有一十日,所纂敘便例定,即乞骸骨。帝成其志,仍給安車還山。史館胄監之士祖帳西門外,宋濂贈之詩曰:「不受君王五色詔,白衣宣至白衣還」,蓋高之也。抵家卒,年七十五。

維楨詩名擅一時,號鐵崖體,與永嘉李孝光、茅山張羽、錫山倪瓚、崑山顧瑛為詩文友,碧桃叟釋臻、知歸叟釋現、清容叟釋信為方外友。張雨稱其古樂府出入少陵、二李間,有曠世金石聲。宋濂稱其論撰,如睹商敦、周彞,雲雷成文,而寒芒橫免。詩震蕩陵厲,鬼設神施,尤號名家云。維楨徙松江時,與華亭陸居仁及僑居錢惟善相倡和。惟善,字思復,錢塘人。至正元年,省試《羅剎江賦》,時鎖院三千人,獨惟善據枚乘《七發》辨錢塘江為曲江,由是得名,號曲江居士。官副提舉。張士誠據吳,遂不仕。居仁,字宅之,中泰定三年鄉試,隱居教授,自號雲松野衲。兩人既歿,與維楨同葬乾山,人目為三高士墓。

胡翰,字仲申,金華人。幼聰穎異常兒。七歲時,道拾遺金,坐守侍其人還之。長從蘭谿吳師道、浦江吳萊學古文,復登同邑許謙之門。同郡黃溍、柳貫以文章名天下,見翰文,稱之不容口。游元都,公卿交譽之。與武威余闕、宣城貢師泰尤善。或勸之仕,不應。既歸,遭天下大亂,避地南華山,著書自適。文章與宋濂、王禕相上下。太祖下金華,召見,命與許元等會食中書省。後侍臣復有薦翰者,召至金陵。時方籍金華民為兵,翰從容進曰:「金華人多業儒,鮮習兵,籍之,徒糜餉耳。」太祖即罷之。授衢州教授。洪武初,聘修《元史》,書成,受賚歸。愛北山泉石,卜築其下,徜徉十數年而終,年七十有五。所著有《春秋集義》,文曰《胡仲子集》,詩曰《長山先生集》。

蘇伯衡,字平仲,金華人,宋門下侍郎轍之裔也。父友龍,受業許謙之門,官蕭山令,行省都事。明師下浙東,坐長子仕閩,謫徙滁州。李善長奏官之,力辭歸。伯衡警敏絕倫,博洽群籍,為古文有聲。元末貢於鄉。太祖置禮賢館,伯衡與焉。歲丙午用為國子學錄,遷學正。被薦,召見,擢翰林編修。力辭,乞省覲歸。洪武十年,學士宋濂致仕,太祖問誰可代者,濂對曰:「伯衡,臣鄉人,學博行修,文詞蔚贍有法。」太祖即徵之,入見,復以疾辭,賜衣鈔而還。二十一年聘主會試,事竣復辭還。尋為處州教授,坐表箋誤,下吏死。二子恬、怡,救父,并被刑。

王冕,字元章,諸暨人。幼貧,父使牧牛,竊入學舍,聽諸生誦書,墓乃返,亡其牛,父怒撻之,已而復然。母曰:「兒癡如此,曷不聽其所為。」冕因去依僧寺,夜坐佛膝上,映長明燈讀書。會稽韓性聞而異之,錄為弟子,遂稱通儒。性卒,門人事冕如事性。屢應舉不中,棄去,北游燕都,客秘書卿泰不花家,擬以館職薦,力辭不就。既歸,每大言天下將亂,攜妻孥隱九里山,樹梅千株,桃杏半之,自號梅花屋主,善畫梅,求者踵至,以幅長短為得米之差。嘗仿《周官》著書一卷,曰:「持此遇明主,伊、呂事業不難致也。」太祖下婺州,物色得之,置幕府,授諮議參軍,一夕病卒。

同時郭奎、劉炳皆早參戎幕,以詩名。奎,字子章,巢縣人。從余闕學,治經,闕亟稱之。太祖為吳國公,來歸,從事幕府。朱文正開大都督府於南昌,命奎參其軍事,文正得罪,奎坐誅。炳,字彥昺,鄱陽人。至正中,從軍於浙。太祖起淮南,獻書言事,用為中書典簽。洪武初,從事大都督府,出為知縣。閱兩考,以病告歸,久之卒。

戴良,字叔能,浦江人。通經、史百家暨醫、卜、釋、老之說。學古文於黃溍、柳貫、吳萊。貫卒,經紀其家。太祖初定金華,命與胡翰等十二人會食省中,日二人更番講經、史,陳治道。明年,用良為學正,與宋濂、葉儀輩訓諸生。太祖既旋師,良忽棄官逸去。辛丑,元順帝用薦者言,授良江北行省儒學提舉。良見時事不可為,避地吳中,依張士誠。久之,見士誠將敗,挈家泛海,抵登、萊,欲間行歸擴廓軍,道梗,寓昌樂數年。洪武六年始南還,變姓名,隱四明山。太祖物色得之。十五年召至京師,試以文,命居會同館,日給大官膳,欲官之,以老疾固辭,忤旨。明年四月暴卒,蓋自裁也。元亡後,惟良與王逢不忘故主,每形於歌詩,故卒不獲其死云。良世居金華九靈山下,自號九靈山人。

逢,字原吉,江陰人。至正中,作《河清頌》,臺臣薦之,稱疾辭。張士誠據吳,其弟士德用逢策,北降於元以拒明。太祖滅士誠,欲辟用之,堅臥不起,隱上海之烏涇,歌詠自適。洪武十五年以文學徵,有司敦迫上道。時子掖為通事司令,以父年高,叩頭泣請,乃命吏部符止之。又六年卒,年七十,有《梧溪詩集》七卷。逢自稱席帽山人。

時又有丁鶴年者,回回人。曾祖阿老丁與弟烏馬兒皆世商。元世祖征西域,軍乏饟,老丁杖策軍門,盡以貲獻。論功,賜田宅京師,奉朝請。烏馬兒累官甘肅行省左丞。父職馬祿丁,以世廕為武昌縣達魯花赤,有惠政,解官,留葬其地。至正壬辰,武昌被兵,鶴年年十八,奉母走鎮江。母歿,鹽酪不入口者五年。避地四明。方國珍據浙東,最忌色目人,鶴年轉徙逃匿,為童子師,或寄僧舍,賣漿自給。及海內大定,牒請還武昌,而生母已道阻前死,瘞東村廢宅中,鶴年慟哭行求,母告以夢,乃嚙血沁骨,斂而葬焉。烏斯道為作《丁孝子傳》。鶴年自以家世仕元,不忘故國,順帝北遁後,飲泣賦詩,情詞悽惻。晚學浮屠法,廬居父墓,以永樂中卒。鶴年好學洽聞,精詩律,楚昭、莊二王咸禮敬之。正統中,憲王刻其遺文行世。

危素,字太僕,金谿人,唐撫州刺史全諷之後。少通《五經》,遊吳澄、范梈門。至正元年用大臣薦授經筵檢討。修宋、遼、金三史及注《爾雅》成,賜金及宮人,不受。由國子助教遷翰林編修。纂后妃等傳,事逸無據,素買餳餅饋宦寺,叩之得實,乃筆諸書,卒為全史。遷太常博士、兵部員外郎、監察御史、工部侍郎,轉大司農丞、禮部尚書。

時亂將亟,素每抗論得失。十八年參中書省事,請專任平章定住總西方兵,毋迎帝師悮軍事,用普顏不花為參政,經略江南,立兵農宣撫使司以安畿內,任賢守令以撫流竄之民。且曰:「今日之事,宜臥薪嘗膽,力圖中興。」尋進御史臺治書侍御史。二十年拜參知政事,俄除翰林學士承旨,出為嶺北行省左丞。言事不報,棄官居房山。素為人侃直,數有建白,敢任事。上都宮殿火,敕重建大安、睿思二閣,素諫止之。請親祀南郊,築北郊,以斥合祭之失。因進講陳民間疾苦,詔為發錢粟振河南、永平民。淮南兵亂,素往廉問,假便宜發楮幣,振維揚、京口饑。居房山者四年。明師將抵燕,淮王帖木兒不花監國,起為承旨如故。素甫至而師入,乃趨所居報恩寺,入井。寺僧大梓力挽起之,曰:「國史非公莫知。公死,是死國史也。」素遂止。兵迫史庫,往告鎮撫吳勉輩出之,《元實錄》得無失。

洪武二年授翰林侍講學士,數訪以元興亡之故,且詔撰《皇陵碑》文,皆稱旨。頃之,坐失朝,被劾罷。居一歲,復故官,兼弘文館學士,賜小庫,免朝謁。嘗偕諸學士賜宴,屢遣內官勸之酒,御製詩一章,以示恩寵,命各以詩進,素詩最後成,帝獨覽而善之曰:「素老成,有先憂之意。」時素已七十餘矣。御史王著等論素亡國之臣,不宜列侍從,詔謫居和州,守余闕廟,歲餘卒。

先是,至元間,西僧嗣古妙高欲毀宋會稽諸陵。夏人楊輦真珈為江南總攝,悉掘徽宗以下諸陵,攫取金寶,裒帝后遺骨,瘞於杭之故宮,築浮屠其上,名曰鎮南,以示厭勝,又截理宗顱骨為飲器。真珈敗,其資皆籍於官,顱骨亦入宣政院,以賜所謂帝師者。素在翰林時,宴見,備言始末。帝歎息良久,命北平守將購得顱骨於西僧汝納所,諭有司厝於高坐寺西北。其明年,紹興以永穆陵圖來獻,遂敕葬故陵,實自素發之云。

張以寧,字志道,古田人。父一清,元福建、江西行省參知政事。以寧年八歲,或訟其伯父於縣繫獄,以寧詣縣伸理,尹異之,命賦《琴堂詩》,立就,伯父得釋,以寧用是知名。泰定中,以《春秋》舉進士,由黃巖判官進六合尹,坐事免官,滯留江、淮者十年。順帝徵為國子助教,累至翰林侍讀學士,知制誥。在朝宿儒虞集、歐陽元、揭傒斯、黃溍之屬相繼物故,以寧有俊才,博學強記,擅名於時,人呼小張學士。

明師取元都,與危素等皆赴京,奏對稱旨,復授侍講學士,特被寵遇。帝嘗登鐘山,以寧與朱升、秦裕伯等扈從擁翠亭,給筆札賦詩。洪武二年秋,奉使安南,封其主陳日煃為國王,御製詩一章遣之。甫抵境,而日煃卒,國人乞以印詔授其世子,以寧不聽,留居洱江上,諭世子告哀於朝,且請襲爵。既得令,俟後使者林唐臣至,然後入境將事。事竣,教世子服三年喪,令其國人效中國行頓首稽首禮。天子聞而嘉之,賜璽書,比諸陸賈、馬援,再賜御製詩八章。及還,道卒,詔有司歸其柩,所在致祭。

以寧為人潔清,不營財產,奉使往還,補被外無他物。本以《春秋》致高第,故所學尤專《春秋》,多所自得,撰《胡傳辨疑》最辨博,惟《春王正月考》未就,寓安南踰半歲,始卒業。元故官來京者,素及以寧名尤重。素長於史,以寧長於經。素宋、元史槁俱失傳,而以寧《春秋》學遂行。

門人石光霽,字仲濂,泰州人。讀書五行俱下。洪武十三年以明經舉,授國子學正,進博士,作《春秋鉤玄》,能傳以寧之學。

裕伯,字景容,大名人。仕元,累官至福建行省郎中。遭世亂,棄官,客揚州。久之,復避地上海。居母喪盡禮。張士誠據姑蘇,遣人招之,拒不納。吳元年,太祖命中書省檄起之。裕伯對使者曰:「食元祿二十餘年而背之,不忠也。母喪未終,忘哀而出,不孝也。」乃上中書省固辭。洪武元年復徵,稱病不出。帝乃手書諭之曰:「海濱民好鬥,裕伯智謀之士而居此地,堅守不起,恐有後悔。」裕伯拜書,涕泗橫流,不得已,偕使者入朝。授侍讀學士,固辭,不允。與張以寧等扈從,登鐘山擁翠亭,給筆札賦詩,甚見寵待。二年改待制,旋為治書侍御史。三年始詔設科取士,以裕伯與御史中丞劉基為京畿主考官。裕伯博辨善論說,占奏悉當帝意,帝數稱之。出知隴州,卒於官。

趙壎,字伯友,新喻人,好學,工屬文。元至正中舉於鄉,為上猶教諭。洪武二年,太祖詔修《元史》,命左丞相李善長為監修官,前起居注宋濂、漳州府通判王禕為總裁官,征山林遺逸之士汪克寬、胡翰、宋僖、陶凱、陳基、曾魯、高啟、趙汸、張文海、徐尊生、黃篪、傅恕、王錡、傅著、謝徽為纂修官,而壎與焉。以是年二月,開局天界寺,取元《經世大典》諸書,用資參考。至八月成,諸儒並賜齎遣歸。而順帝一朝史猶未備,乃命儒士歐陽佑等往北平采遺事。明年二月還朝,重開史局,仍以宋濂、王禕為總裁,征四方文學士朱右、貝瓊、朱廉、王彞、張孟兼、高遜志、李懋、李汶、張宣、張簡、杜寅、殷弼、俞寅及壎為纂修官。先後纂修三十人,兩局並與者,壎一人而已。閱六月,書成,諸儒多授官,惟壎及朱右、朱廉不受歸。

尋召修日曆,授翰林編修。高麗遣使朝貢,賜宴,樂作,使者以國喪辭。熏進曰:「小國之喪,不廢大國之禮。」太祖甚悅,命與宋濂同職史館,濂兄事之。嘗奉詔撰《甘露頌》,太祖稱善。出為靖江王府長史,卒。

始與壎同纂修者汪克寬、陶凱、曾魯、高啟、趙汸、貝瓊、高遜志並有傳,今自宋僖以下可考者,附著於篇。

宋僖,字無逸,餘姚人。元繁昌教諭,遭亂歸。史事竣,命典福建鄉試。

陳基,字敬初,臨海人。少與兄聚受業於義烏黃溍,從溍游京師,授經筵檢討。嘗為人草諫章,力陳順帝並后之失,順帝欲罪之,引避歸里。已,奉母入吳,參太尉張士誠軍事。士誠稱王,基獨諫止,欲殺之,不果。吳平,召修《元史》,賜金而還。洪武三年冬卒。初,士誠與太祖相持,基在其幕府,書檄多指斥,及吳亡,吳臣多見誅,基獨免。世所傳《夷白集》,其指斥之文猶備列云。

張文海,鄞人,與同里傅恕並入史館。

徐尊生,字大年,淳安人。《元史》成,受賜歸,復同修日曆。後以宋濂薦授翰林應奉,文字草制,悉稱旨。尋以老疾辭還。

傅恕,字如心,鄞人。學通經史,與同郡烏斯道、鄭真皆有文名。洪武二年詣闕陳治道十二策,曰:正朝廷、重守令、馭外蕃、增祿秩、均民田、更法役、黜異端、易服制、興學校、慎選舉、罷榷鹽、停榷茶。太祖嘉納之,遂命修《元史》。事竣,授博野知縣,後坐累死。

斯道,字繼善,慈谿人,與兄本良俱有學行。洪武中,斯道被薦授石龍知縣,調永新,坐事謫役定遠,放還,卒。斯道工古文,兼精書法。子緝,亦善詩文。洪武四年舉鄉試第一,授臨淮教諭。入見,賜之宴,賦詩稱旨,除廣信教授,自號榮陽外史。

傅著,字則明,長洲人。史成,歸為常熟教諭。魏觀行鄉飲酒禮,長洲教諭周敏侍其父南老,著侍其父玉,皆降而北面立,觀禮者以為盛事焉。歷官知府,卒。

謝徽,字元懿,長洲人。史成,授翰林國史院編修。尋擢吏部郎中,力辭不拜,歸。復起國子助教,卒。徽博學工詩文,與同邑高啟齊名。弟恭,字元功,亦能詩。

朱右,字伯賢,臨海人。史成,辭歸。已,徵修日曆、寶訓,授翰林編修。遷晉府右長史。九年卒官。

朱廉,字伯清,義烏人。幼力學,從黃溍學古文。知府王宗顯辟教郡學。李文忠鎮嚴州,延為釣臺書院山長。洪武初,《元史》成,不受官歸。尋徵修日曆,除翰林編修。八年扈駕中都,進詩十章,太祖稱善,為和六章賜之。已而授楚王經,遷楚府右長史。久之,辭疾歸。廉好程、朱之學,嘗取《朱子語類》,摘其精義,名曰《理學纂言》。

王彞,字常宗,其先蜀人,父為崑山教授,遂卜居嘉定。少孤貧,讀書天台山中,師事王貞文,得蘭谿金履祥之傳,學有端緒。嘗著論力詆楊維楨,目為文妖。《元史》成,賜銀幣還。又以薦入翰林,母老乞歸。坐知府魏觀事,與高啟俱被殺。

張孟兼,浦江人,名丁,以字行。史成,授國子學錄,歷禮部主事、太常司丞。劉基嘗為太祖言:「今天下文章,宋濂第一,其次即臣基,又次即孟兼。」太祖頷之。孟兼性傲,嘗坐累謫輸作。已,復官,太祖顧孟兼謂濂曰:「卿門人邪?」濂對:「非門人,乃邑子也。其為文有才,臣劉基嘗稱之。」太祖熟視孟兼曰:「生骨相薄,仕宦,徐徐乃可耳。」未幾,用為山西僉事。廉勁疾惡,糾摘奸猾,令相牽引,每事輒株連數十人。吏民聞張僉事行部,凜然墮膽。聲聞於朝,擢山東副使。布政使吳印者,僧也,太祖驟貴之,寵眷甚,孟兼易之。印謁孟兼,由中門入,孟兼杖守門卒。已,又以他事與相拄。太祖先入印言,逮笞孟兼。孟兼憤,捕為印書奏者,欲論以罪。印復上書言狀,太祖大怒曰:「豎儒與我抗邪!」械至闕下,命棄市。

李汶,字宗茂,當塗人。博學多才,史成,除巴東知縣,移南和。晚年歸里,以經學訓後進。

張宣,字藻重,江陰人。洪武初,以考禮徵。尋預修《元史》,太祖親書其名,召對殿廷,即日授翰林編修,呼為小秀才。奉詔歸娶,年已三十矣。六年坐事謫徙濠梁,道卒。

張簡,字仲簡,吳縣人。初師張雨為道士,隱居鴻山。元季兵亂,以母老歸養,遂返儒服。洪武三年,薦修《元史》。當元季,浙東、西士大夫以文墨相尚,每歲必聯詩社,聘一二文章鉅公主之,四方名士畢至,宴賞窮日夜,詩勝者輒有厚贈。臨川饒介為元淮南行省參政,豪於詩,自號醉樵,嘗大集諸名士賦《醉樵歌》。簡詩第一,贈黃金一餅;高啟次之,得白金三斤;楊基又次之,猶贈一鎰。

杜寅,字彥正,吳縣人。史成,官岐寧衛知事。洪武八年,番賊既降復叛,寅與經歷熊鼎俱被害。

徐一夔,字大章,天台人。工文,與義烏王禕善。洪武二年八月詔纂修禮書,一夔及儒士梁寅、劉于、曾魯、周子諒、胡行簡、劉宗弼、董彞、蔡深、滕公琰並與焉。明年書成,將續修《元史》,禕方為總裁官,以一夔薦。一夔遺書曰:

邇者縣令傳命,言朝廷以續修《元史》見征,且云執事謂僕善敘事,薦之當路,私心竊怪執事何心卷心卷於不材多病之人也。僕素謂執事知我,今自審終不能副執事之望,何也?

近世論史者,莫過於日曆,日曆者,史之根柢也。自唐長壽中,史官姚璹奏請撰時政記,元和中,韋執誼又奏撰日曆。日曆以事繫日,以日繫月,以月繫時,以時繫年,猶有《春秋》遺意。至於起居注之說,亦專以甲子起例,蓋紀事之法無踰此也。

往宋極重史事,日曆之修,諸司必關白。如詔誥則三省必書,兵機邊務則樞司必報,百官之進退,刑賞之予奪,臺諫之論列,給舍之繳駁,經筵之論答,臣僚之轉對,侍從之直前啟事,中外之囊封匭奏,下至錢穀、甲兵、獄訟、造作,凡有關政體者,無不隨日以錄。猶患其出於吏牘,或有訛失。故歐陽修奏請宰相監修者,於歲終檢點修撰官日所錄事,有失職者罰之。如此,則日曆不至訛失,他時會要之修取於此,實錄之修取於此,百年之後紀、志、列傳取於此,此宋氏之史所以為精確也。

元朝則不然,不置日曆,不置起居注,獨中書置時政科,遣一文學掾掌之,以事付史館。及一帝崩,則國史院據所付修實錄而已。其於史事,固甚疏略。幸而天曆間虞集仿六典法,纂《經世大典》,一代典章文物粗備。

是以前局之史,既有十三朝實錄,又有此書可以參稽,而一時纂修諸公,如胡仲申、陶中立、趙伯友、趙子常、徐大年輩皆有史才史學,廠堇而成書。至若順帝三十六年之事,既無實錄可據,又無參稽之書,惟憑采訪以足成之,竊恐事未必核也,言未必馴也,首尾未必穿貫也。而向之數公,或受官,或還山,復各散去。乃欲以不材多病如僕者承之於後,僕雖欲仰副執事之望,曷以哉!謹奉狀左右,乞賜矜察。

一夔遂不至。未幾,用薦署杭州教授。召修《大明日歷》,書成,將授翰林院官,以足疾辭,賜文綺遣還。

趙捴謙,名古則,更名謙,餘姚人。幼孤貧,寄食山寺,與朱右、謝肅、徐一夔輩定文字交。天台鄭四表善《易》,則從之受《易》。定海樂良、鄞鄭真明《春秋》,山陰趙俶長於說《詩》,迮雨善樂府,廣陵張昱工歌詩,無為吳志淳、華亭朱芾工草書篆隸,捴謙悉與為友。博究《六經》、百氏之學,尤精六書,作《六書本義》,復作《聲音文字通》,時目為考古先生。洪武十二年命詞臣修《正韻》,捴謙年二十有八,應聘入京師,授中都國子監典簿。久之,以薦召為瓊山縣學教諭。二十八年,卒於番禺。其後,門人柴欽,字廣敬,以庶吉士與修《永樂大典》,進言其師所撰《聲音文字通》當採錄,遂奉命馳傳,即其家取之。

樂良,字季本。迮雨,字士霖。趙俶,字本初。洪武中,官國子監博士。以年老乞歸,加翰林待制。

張昱,字光弼,廬陵人。仕元,為江浙行省左、右司員外郎,行樞密院判官。留居西湖壽安坊,貧無以葺廬,酒間為瞿佑誦所作詩,笑曰:「我死埋骨湖上,題曰詩人張員外墓足矣。」太祖征至京,憫其老,曰「可閑矣」,厚賜遣還,乃自號可閑老人。年八十三卒。

吳志淳,字主一,元末知靖安、都昌二縣。奏除待制翰林,為權倖所阻,避兵於鄞。

朱芾,字孟辨,洪武初,官編修,改中書舍人。

陶宗儀,字九成,黃巖人。父煜,元福建、江西行樞密院都事。宗儀少試有司,一不中即棄去,務古學,無所不窺。出游浙東、西,師事張翥、李孝光、杜本。為詩文,咸有程度,尤刻志字學,習舅氏趙雍篆法。浙帥泰不華、南臺御史丑驢舉為行人,又辟為教官,皆不就。張士誠據吳,署為軍諮,亦不赴。洪武四年詔徵天下儒士,六年命有司舉人才,皆及宗儀,引疾不赴。晚歲,有司聘為教官,非其志也。二十九年率諸生赴禮部試,讀《大誥》,賜鈔歸,久之卒。所著有《輟耕錄》三十卷,又葺《說郛》、《書史會要》、《四書備遺》,並傳於世。

顧德輝,字仲瑛,崑山人。家世素封,輕財結客,豪宕自喜。年三十,始折節讀書,購古書、名畫、彞鼎、秘玩,築別業於茜涇西,曰玉山佳處,晨夕與客置酒賦詩其中。四方文學士河東張翥、會稽楊維楨、天台柯九思、永嘉李孝光,方外士張雨、於彥、成琦、元璞輩,咸主其家。園池亭榭之盛,圖史之富暨餼館聲伎,並冠絕一時。而德輝才情妙麗,與諸名士亦略相當。嘗舉茂才,授會稽教諭,辟行省屬官,皆不就。張士誠據吳,欲強以官,去隱於嘉興之合溪。尋以子元臣為元水軍副都萬戶,封德輝武略將軍、飛騎尉、錢塘縣男。母喪歸綽溪,士誠再辟之,遂斷髮廬墓,自號金粟道人。及吳平,父子並徙濠梁。洪武二年卒。士誠之據吳也,頗收召知名士,東南士避兵於吳者依焉。

孫作,字大雅,江陰人。為文醇正典雅,動有據依。嘗著書十二篇,號《東家子》,宋濂為作《東家子傳》。元季,挈家避兵於吳,盡棄他物,獨載書兩簏。士誠廩祿之,旋以母病謝去,客松江,眾為買田築室居焉。洪武六年聘修《大明日曆》,授翰林編修,乞改太平府教授。召為國子助教,尋分教中都,踰年還國學,抉授司業,歸卒於家。

元末文人最盛,其以詞學知名者,又有張憲、周砥、高明、藍仁之屬。

張憲,字思廉,山陰人。學詩於楊維楨,最為所許。負才不羈,嘗走京師,恣言天下事,眾駭其狂。還入富春山,混緇流以自放。一日,升高呼所親,語曰:「禍至矣,亟去!」三日而寇至,死者五百家。後仕張士誠,為樞密院都事。吳平,變姓名,寄食杭州報國寺以歿。

周砥,字履道,吳人,僑無錫。博學工文詞,與宜興馬治善,遭亂客治家,治為具舟車,盡窮陽羨山溪之勝。其鄉多富人,與治善者咸置酒招砥。砥心厭之,一日貽書別治,夜半遁去,游會稽,歿於兵。治,字孝常,亦能詩。洪武時為內丘知縣,終建昌知府。

高明,字則誠,永嘉人。至正五年進士,授處州錄事,闢行省掾。方國珍叛,省臣以明諳海濱事,擇以自從,與論事不合。及國珍就撫,欲留置幕下,即日解官,旅寓鄞之櫟社。太祖聞其名,召之,以老疾辭,還卒於家。

藍仁,字靜之。弟智,字明之,崇安人。元時,清江杜本隱武夷,崇尚古學,仁兄弟俱往師之,授以四明任士林詩法,遂謝科舉,一意為詩。後辟武夷書院山長,遷邵武尉,不赴。內附後,例徙濠梁,數月放歸,卒。智,洪武十年被薦,起家廣西僉事,著廉聲。

袁凱,字景文,松江華亭人。元末為府吏,博學有才辨,議論飆發,往往屈座人。洪武三年薦授御史。武臣恃功驕恣,得罪者漸眾,凱上言:「諸將習兵事,恐未悉君臣禮。請於都督府延通經學古之士,令諸武臣赴都堂聽講,庶得保族全身之道。」帝敕臺省延名士直午門,為諸將說書。後帝慮囚畢,命凱送皇太子覆訊,多所矜減。凱還報,帝問「朕與太子孰是?」凱頓首言:「陛下法之正,東宮心之慈。」帝以凱老猾持兩端,惡之。凱懼,佯狂免,告歸,久之以壽終。凱工詩,有盛名。性詼諧,自號海叟。背戴烏巾,倒騎黑牛,游行九峰間,好事者至繪為圖。初,在楊維楨座,客出所賦《白燕詩》,凱微笑,別作一篇以獻。維楨大驚賞,遍示座客,人遂呼袁白燕云。

高啟,字季迪,長洲人。博學工詩。張士誠據吳,啟依外家,居吳淞江之青丘。洪武初,被薦,偕同縣謝徽召修《元史》,授翰林院國史編修官,復命教授諸王。三年秋,帝御闕樓,啟、徽俱入對,擢啟戶部右侍郎,徽吏部郎中。啟自陳年少不敢當重任,徽亦固辭,乃見許。已,並賜白金放還。啟嘗賦詩,有所諷刺,帝嗛之未發也。及歸,居青丘,授書自給。知府魏觀為移其家郡中,旦夕延見,甚歡。觀以改修府治,獲譴。帝見啟所作上梁文,因發怒,腰斬於市,年三十有九。明初,吳下多詩人,啟與楊基、張羽、徐賁稱四傑,以配唐王、楊、盧、駱云。

基,字孟載,其先蜀嘉州人,祖宦吳中,生基,遂家焉。九歲背誦《六經》,及長著書十萬餘言,名曰《論鑒》。遭亂,隱吳之赤山。張士誠辟為丞相府記室,未幾辭去,客饒介所。明師下平江,基以饒氏客安置臨濠,旋徙河南。洪武二年放歸。尋起為滎陽知縣,謫居鐘離。被薦為江西行省幕官,以省臣得罪,落職。六年起官,奉使湖廣。召還,授兵部員外郎,遷山西副使。進按察使,被讒奪官,謫輸作,竟卒於工所。初,會稽楊維楨客吳中,以詩自豪。基於座上賦《鐵笛歌》,維楨驚喜,與俱東,語從游者曰:「吾在吳,又得一鐵矣。若曹就之學。優於老鐵學也。」

張羽,字來儀,後以字行,本潯陽人。從父宦江浙,兵阻不獲歸,與友徐賁約,卜居吳興。領鄉薦,為安定書院山長,再徙於吳。洪武四年征至京師,應對不稱旨,放還。再徵授太常司丞。太祖重其文,十六年自述滁陽王事,命羽撰廟碑。尋坐事竄嶺南,未半道,召還。羽自知不免,投龍江以死。羽文章精潔有法,尤長於詩,作畫師小米。

徐賁,字幼文,其先蜀人,徙常州,再徙平江。工詩,善畫山水。張士誠辟為屬,已謝去。吳平,謫徙臨濠。洪武七年被薦至京。九年春,奉使晉、冀,有所廉訪。暨還,檢其橐,惟紀行詩數首,太祖悅,授給事中。改御史,巡按廣東。又改刑部主事,遷廣西參議。以政績卓異,擢河南左布政使。大軍征洮、岷,道其境,坐犒勞不時,下獄瘐死。

王行,字止仲,吳縣人。幼隨父依賣藥徐翁家,徐媼好聽稗官小說,行日記數本,為媼誦之。媼喜,言於翁,授以《論語》,明日悉成誦。翁大異之,俾盡讀家所有書,遂淹貫經史百家言。未弱冠,謝去,授徒齊門,名士咸與交。富人沈萬三延之家塾,每文成,酬白金鎰計,行輒麾去曰:「使富而可守,則然臍之慘不及矣。」洪武初,有司延為學校師。已,謝去,隱於石湖。其二子役於京,行往視之,涼國公藍玉館於家,數薦之太祖,得召見。後玉誅,行父子亦坐死。

始吳中用兵,所在多列炮石自固,行私語所知曰:「兵法柔能制剛,若植大竹於地,繫布其端,炮石至,布隨之低昂,則人不能害,而炮石無所用矣。」後常遇春取平江,果如其法。行亦自負知兵,以及於禍云。

初,高啟家北郭,與行比鄰,徐賁、高遜志、唐肅、宋克、餘堯臣、張羽、呂敏、陳則皆卜居相近,號北郭十友,又稱十才子。啟、賁、遜志、羽自有傳。

唐肅,字處敬,越州山陰人。通經史,兼習陰陽、醫卜、書數。少與上虞謝肅齊名,稱會稽二肅。至正壬寅舉鄉試。張士誠時,為杭州黃岡書院山長,遷嘉興路儒學正。士誠敗,例赴京。尋以父喪還。洪武三年用薦召修禮樂書,擢應奉翰林文字。其秋,科舉行,為分考官,免歸。六年謫佃濠梁,卒。子之淳,字愚士,宋濂亟稱之。建文二年,用方孝孺薦,擢翰林侍讀,與孝孺共領修書事,卒於官。

謝肅,官至福建僉事,坐事死。

宋克,字仲溫,長洲人。偉軀幹,博涉書史。少任俠,好學劍走馬,家素饒,結客飲博。迨壯,謝酒徒,學兵法,周流無所遇,益以氣自豪。張士誠欲羅致之,不就。性抗直,與人議論期必勝,援古切今,人莫能難也。杜門染翰,日費十紙,遂以善書名天下。時有宋廣,字昌裔,亦善草書,稱二宋。洪武初,克任鳳翔同知,卒。

堯臣,字唐卿,永嘉人。入吳,為士誠客。城破,例徙濠梁。洪武二年放還,授新鄭丞。

呂敏,字志學,無錫人。元時為道士,洪武初,官無錫教諭。十三年舉人才,不知其官所終。

陳則,字文度,崑山人。洪武六年舉秀才,授應天府治中。俄擢戶部侍郎,以閱實戶口,出為大同府同知,進知府。

孫蕡,字仲衍,廣東順德人。性警敏,書無所不窺。詩文援筆立就,詞採爛然。負節概,不妄交游。何真據嶺南,開府辟士,與王佐、趙介、李德、黃哲並受禮遇,稱五先生。廖永忠南征,蕡為真草降表,永忠辟典教事。洪武三年始行科舉,蕡與其選,授工部織染局使,遷虹縣主簿。兵燹後,蕡勞徠安輯,民多復業。居一年,召為翰林典籍,與修《洪武正韻》。九年遣監祀四川。居久之,出為平原主簿。坐累逮繫,俾築京師望都門城垣。蕡謳唫為粵聲,主者以奏。召見,命誦所歌詩,語皆忠愛,乃釋之。十五年起為蘇州經歷,復坐累戍遼東。已,大治藍玉黨,蕡嘗為玉題畫,遂論死。臨刑,作詩長謳而逝。時門生黎貞亦戍遼東,蕡屍乃得收斂。貞,字彥晦,新會人。工詩文,嘗為本邑訓導,以事被誣,戍遼陽十八年,從游者甚眾。放還卒。蕡所著,有《通鑑前編綱目》、《孝經集善》、《理學訓蒙》及《西庵集》、《和陶集》,多佚不傳。番禺趙純稱其究極天人性命之理,為一時儒宗云。

王佐,字彥舉,先河東人,元末侍父官南雄,經亂不能歸,遂占籍南海。與蕡結詩社。構辭敏捷,佐不如蕡,句意沉著,蕡亦不如佐。何真使佐掌書記,參謀議。真歸朝,佐亦還里。洪武六年被薦,徵為給事中。太祖賜宋濂黃馬,復為歌,命侍臣屬和,佐立成。性不樂樞要,將告歸。時告者多獲重譴,或尼之曰:「君少忍,獨不虞性命邪?」佐乃遲徊二年,卒乞骸歸。

趙介,字伯貞,番禺人。博通六籍及釋、老書。氣豪邁,無仕進意。行以囊自隨,遇景,賦詩投其中,日往來西樵泉石間。有司累薦,皆辭免。洪武二十二年坐累逮赴京,卒於南昌舟次。四子,潔、絢、繹、純,皆善詩文,工篆隸。絢,隱居不出,有父風。純,仕御史。

李德,字仲修,番禺人。洪武三年以明經薦授洛陽典史,歷南陽、西安二府幕官,並能其職。以年衰乞改漢陽教諭,秩滿,調義寧。義寧在粵西,荒陋甚,德為振舉,文教漸興,解官歸卒。德初好為詩,晚究洛、閩之學,謂誠意為古聖喆心要,故嶺南人稱理學,必曰李仲修云。黃哲,亦番禺人。歷仕州郡,以治行稱。

王蒙,字叔明,湖州人,趙孟頫之甥也。敏於文,不尚矩度。工畫山水,兼善人物。少時賦宮詞,仁和俞友仁見之,曰「此唐人佳句也」,遂以妹妻焉。元末官理問,遇亂,隱居黃鶴山,自稱黃鶴山樵。洪武初,知泰安州事。蒙嘗謁胡惟庸於私第,與會稽郭傳、僧知聰觀畫。惟庸伏法,蒙坐事被逮,瘐死獄中。

郭傳,一名正傳,字文遠。洪武七年,帝御武樓,賜學士宋濂坐,謂曰:「天下既定,朕方垂意宿學之士,卿知其人乎?」對曰:「會稽有郭傳者,學有淵源,其文雄贍新麗,其議論根據《六經》,異才也。」既而濂持其文以進,帝召見於謹身殿,授翰林應奉,直起居注。遷兵部主事,再遷考功監丞,進監令,出署湖廣布政司參政。


\end{pinyinscope}