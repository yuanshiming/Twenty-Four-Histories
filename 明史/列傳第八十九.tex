\article{列傳第八十九}

\begin{pinyinscope}
陶琰子滋王縝李充嗣吳廷舉弟廷弼方良永弟良節子重傑王爌王軏徐問張邦奇族父時徹韓邦奇弟邦靖周金吳嶽譚大初

陶琰,字廷信,絳州人。父銓,進士,陜西右參議。琰舉成化七年鄉試第一,十七年成進士,授刑部主事。

弘治初,進員外郎。歷固原兵備副使。練士卒,廣芻粟。歷九年,部內晏如。遷福建按察使,浙江左布政使。正德初,以右副都御史巡撫河南,遷刑部右侍郎。陜西遊擊徐謙訐御史李高。謙故劉瑾黨,行厚賂,欲中高危法。琰往按,直高。瑾怒,假他事下琰詔獄,褫其職,又罰米四百石輸邊。瑾誅,起左副都御史,總督漕運兼巡撫準、揚諸府。

六年,轉南京刑部侍郎。明年,賊劉七等將犯江南,王浩八又入衢州。進琰右都御史,巡視浙江。至則七等已滅,浩八聽撫。會寧、紹瀕海地颶風大作,居民漂沒萬數。琰出帑金振救,而大築蕭山至會稽堤五萬餘丈。奏設兵備道守要害,防浩八黨出沒,遣將擊斬其渠魁。遂城開化、常山、遂安、蘭谿,境內以靖。復命總督漕運,七疏乞歸。世宗嗣位,起故官。凡三督漕,軍民習其政,不嚴而肅。

琰性清儉,飯惟一疏。每到官及罷去,行李止三竹笥。尋加戶部尚書。嘉靖元年召拜工部尚書。其冬,改南京兵部,加太子少保。未浹歲,屢引年乞體。加太子太保,乘傳歸,有司歲時存問。又九年卒,年八十有四。贈少保,謚恭介。

子滋,以進士授行人。諫武宗南巡,杖闕下,謫國子學正。嘉靖初,歷兵部郎中。率同官伏闕爭「大禮」,再受杖,謫戍榆林。兵部尚書王時中等言,琰老病呻吟,冀父子一相見,乞改調近衛。不許。十五年赦還,卒。

王縝,字文哲,東莞人。父恪,寶慶知府。縝登弘治六年進士,選庶吉士,授兵科給事中。劾三邊總制王越附汪直、李廣,不可復玷節鉞。出理南畿屯田。有司徵松江白絺六千匹,縝言絺非正供。且請停上清宮役。詔皆罷之。累遷工科都給事中。武宗初立,內府工匠以營造加恩。縝率同官言:「陛下初登大寶,工匠末技已有以微勞進者,誠不可示後世。宜散遣先朝諸畫士,革工匠所授官。」帝不能用。中官張永請改築通州新城,縝言泰陵工作方興,不當復興無益之役。帝乃止。正德元年出為山西右參政。歷福建布政使,遷右副都御史,巡撫蘇、松諸府。協平江西賊王浩八。乾清宮災,疏請養宗室子宮中,定根本;去南京新增內官,召還建言被黜諸臣。不報。已,調鄖陽巡撫,遷南京刑部右侍郎。世宗即位,陳正本十事。嘉靖二年就擢戶部尚書。卒官。

李充嗣,字士修,內江人。給事中蕃孫也。登成化二十三年進士,改庶吉士。弘治初,授戶部主事。以從父臨安為郎中,改刑部。坐累,謫岳州通判。久之,移隨州知州,擢陜西僉事,歷雲南按察使。正德九年,舉治行卓異,累遷右副都御史,巡撫河南。歲大祲。請發帑金移粟振之,不足,則勸貸富室。時流民多聚開封,煮糜哺之。踰月,資遣還鄉。初,鎮守中官廖堂黨於劉瑾,假進貢名,要求百端,繼者以為常。充嗣言:「近中官進貢,有古銅器、窯變盆、黃鷹、角鷹、錦雞、走狗諸物,皆借名科斂。外又有拜見銀、須知銀及侵扣驛傳快手月錢、河夫歇役之屬,無慮十餘事,苛派動數十萬。其左右用事者,又私於境內抑買雜物,擅榷商賈貨利。乞嚴行禁絕。」詔但禁下人科取而已。

十二年移撫應天諸府。寧王宸濠反,充嗣謂尚書喬宇曰:「都城守禦屬於公,畿輔則充嗣任之。」乃自將精兵萬人,西屯采石。遣使入安慶城中,令指揮楊銳等堅守。傳檄部內,聲言京邊兵十萬旦夕至,趣供餉,以紿賊。賊果疑懼。事定,兵部及巡按御史胡潔言其功。時已就進戶部右侍郎,乃賜敕嘉勞。有建議修蘇、松水利者,進充嗣工部尚書兼領水利事。未幾,世宗嗣位,遣工部郎林文霈、顏如翽佐之。開白茅港,疏吳淞江,六閱月而訖工。語詳《河渠志》。

嘉靖元年論平宸濠功,加太子少保。蘇、松白糧輸內府。正德時驟增內使五千人,糧亦加十三萬石。帝用充嗣言,減從故額。又請常賦外盡蠲歲辦之浮額者,內府徵收,監以科道官,毋縱內臣苛索。帝俱從之。尋改南京兵部尚書。七年致仕,卒。久之,詔贈太子太保,謚康和。

吳廷舉,字獻臣,其先嘉魚人,祖戍梧州,遂家焉。成化二十三年登進士,除順德知縣。上官屬修中貴人先祠,廷舉不可。市舶中官市葛,以二葛與之,曰:「非產也。」中官大怒。御史汪宗器亦惡廷舉,曰:「彼專抗上官,市名耳。」會廷舉毀淫祠二百五十所,撤其材作堤,葺學宮、書院。宗器謂有所侵盜,執下獄。按之不得間,慚而止。為縣十年,稍遷成都同知。憂歸,補松江。

用尚書馬文升、劉大夏薦,擢廣東僉事。從總督潘蕃討平南海、清遠諸盜。正德初,歷副使。發總鎮中官潘忠二十罪。忠亦訐廷舉他事,逮繫詔獄。劉瑾矯詔,枷之十餘日,幾死。戍雁門,旋赦免。楊一清薦其才,擢江西右參政。敗華林賊於連河。從陳金大破姚源賊。其黨走裴源,復從俞諫破之。賊首胡浩三既撫復叛,廷舉往諭,為所執。居三月,盡得其要領,誘使攜。及得還,浩三果殺其兄浩二,內亂。官兵乘之,遂擒浩三。與副使李夢陽不協,奏夢陽侵官,因乞休。不俟命竟去,坐停一歲俸。起廣東右布政使,復佐陳金平府江賊。擢右副都御史,振湖廣饑。已,復出湖南定諸夷疆地。寧王宸濠有逆謀,疏陳江西軍政六事,為豫防計。

世宗立,召為工部右侍郎,旋改兵部。上疏詆陸完、王瓊、梁儲及少傅蔣冕,而自以為己昔居憲職無一言,乞罷黜以儆幸位。時完早得罪,瓊及儲已罷去,廷舉借以傾冕。冕遂求罷。帝頗不直廷舉,調南京工部,而慰諭冕。冕固請留之,不聽。

嘉靖元年,廷舉乞休。尋以災異復自劾求罷,勸帝修德應天,因奏行其部興革十二事。尋就改戶部,遷右都御史,巡撫應天諸府。長洲知縣郭波以事挫織造中官張志聰。志聰伺波出,倒曳之車後。典史蕭景腆操兵教場,急率兵救。百姓登屋,飛瓦擊志聰。志聰奏逮波、景腆,廷舉具白志聰貪黷狀。帝乃降波五級,調景腆遠方,志聰亦召還。

三年,以「大禮」議未定,請如洪武中修《孝慈錄》故事,令兩京部、寺、臺、省及天下督、撫各條所見,並詢家居老臣,采而行之,匯為一書,以詔後世。時已定稱本生考,廷舉窺帝意不慊,故為此奏。給事中張原、劉祺交劾之,不報。尋改南京工部尚書,辭不拜,稱疾乞休。帝慰留。已,復辭,且引白居易、張詠詩,語多詼諧,中復用嗚呼字。帝怒,以廷舉怨望無人臣禮,勒致仕。

廷舉面如削瓜。衣敝帶穿,不事藻飾。言行必自信,人莫能奪。其在太學時,兄事羅。病痢,僕死,自煮藥飲之。負以如廁,一晝夜數十反。嘗語人曰:「獻臣生我。」廷舉好薛瑄、胡居仁學,尊事陳獻章。居湫隘,亡郭外田,有書萬卷。及卒,總督姚鏌庀其喪。隆慶中,追謚清惠。

弟廷弼,舉於鄉。廷舉枷吏部前,廷弼臥其械下。刑部主事宿進為奏記張糸採,乃得釋。

方良永,字壽卿,莆田人。弘治三年進士。督逋兩廣,峻卻饋遺,為布政使劉大夏所器。還授刑部主事。進員外郎,擢廣東僉事。瓊州賊符南蛇為亂,大夏時為總督,檄攝海南兵備,會師討平之。御史坐良永失利。大夏已入為本兵,為白於朝,賚銀幣。

正德初,父喪除,待銓闕下。外官朝見畢,必謁劉瑾。鴻臚導良永詣左順門叩頭畢,令東向揖瑾,良永竟出。或勸詣瑾家,良永不可。及吏部除良永河南撫民僉事,中旨勒致仕。既去,瑾怒未已,欲假海南殺人事中之。刑部郎中周敏力持,乃不坐。瑾誅,起湖廣副使。尋擢廣西按察使。發巡按御史朱志榮罪至謫戍。遷山東右布政使。旋調浙江,改左。

錢寧以鈔二萬鬻於浙,良永上疏曰:「四方盜甫息,瘡痍未瘳,浙東西雨雹。寧廝養賤流,假義子名,躋公侯之列。賜予無算,納賄不貲,乃敢攫民財,戕邦本。有司奉行急於詔旨,胥吏緣為奸,椎膚剝髓,民不堪命。鎮守太監王堂、劉璟畏寧威,受役使。臣何敢愛一死,不以聞。乞陛下下寧詔獄,明正典刑,並治其黨,以謝百姓。」寧懼,留疏不下。謀遣校尉捕假勢鬻鈔者,以自飾於帝,而請以鈔直還之民,陰召還前所遣使。寧初欲散鈔遍天下,先行之浙江、山東,山東為巡撫趙璜所格,而良永白發其奸,寧自是不敢鬻鈔矣。寧方得志,公卿、臺諫無敢出一語。良永以外僚訟言誅之,聞者震悚。良永念母老,恐中禍,三疏乞休去。

世宗即位,中外交薦。拜右副都御史,撫治鄖陽。以母老,再疏乞終養。都御史姚鏌請破格褒寵。尚書喬宇、孫交言,良永家無贏資,宜用侍郎潘禮、御史陳茂烈故事,賜廩米。詔月給三石。久之,母卒,詔賜祭葬。皆異數也。服除,以故官巡撫應天,即家賜敕。至衢州疾作,連疏乞致仕,未報遽歸,卒。卒後有南京刑部尚書之命。暨訃聞,賜恤如制,謚簡肅。

良永侍父疾,衣不解帶者三月。母病,良永年六十餘矣,手進湯藥無少怠。居倚廬哀毀,稱純孝焉。素善王守仁,而論學與之異。嘗語人曰:「近世專言心學,自謂超悟獨到,推其說以自附於象山,而上達於孔子。目賢聖教人次第為小子無用之學,程、朱而下無不受擯,而不知其入於妄。」

弟良節,官廣東左布政使,亦有治行。

子重傑,舉於鄉,以孝聞。

王爌,字存納,黃巖人。弘治十五年進士。除太常博士。正德時,屢遷刑科都給事中。武定侯郭勛鎮兩廣,行事乖謬。詔自陳,勛強辨,爌等駁之。都察院覆奏,不錄爌言,爌並劾都御史彭澤。帝責澤,置勛不問。御史林有年直言下獄,浙江僉事韓邦奇忤中官被逮,爌皆救之。帝幸大同久不反,爌力請回鸞。又與工科石天柱救彭澤,忤王瓊。中旨調兩人於外,爌得惠州推官。世宗立,召復都給事中。旋擢太僕少卿,改太常。嘉靖三年遷應天府尹。歲大祲,奏免其賦。居四年,遷南京刑部右侍郎,以母老歸養。家居十年,起故官。尋擢南京右都御史。守備中官進表,率以兩御史監禮。爌曰:「中官安得役御史?」止之。舉賀入朝,謁內閣夏言。言倨甚,大臣多隅坐,爌獨引坐正之。言不悅,爌遂謝病歸。

爌與御史潘壯不相能。壯坐大獄,詔爌提問。爌力白壯罪,至忤旨。人以此稱爌長者。卒,贈工部尚書。

王軏,開平衛人,弘治十二年進士。正德初,歷工部員外郎,屢遷山東左布政使。嘉靖初,入為順天府尹。房山地震,軏言召災有由,語多指斥。忤旨切責。尋遷副都御史,巡撫四川。芒部土官知府隴慰死,庶子政與嫡子壽爭立,朝議立壽。政倚烏撒,數構兵,使人誘殺壽,奪其印。軏請討之。乃會貴州兵分道進,擒政於水西,招降四十九寨。璽書獎勞。

時將營仁壽宮,就拜軏工部右侍郎,督採大木。工罷,召還,改戶部。核九門苜蓿地,以餘地歸之民。勘御馬監草場,釐地二萬餘頃,募民以佃。房山民以牧馬地獻中官韋恒,軏釐歸之官。奸人馮賢等復獻中官李秀,秀為請於帝,軏抗疏劾之。帝雖宥秀,竟治賢等如律。出核勛戚莊田,請如周制,計品秩,別親疏,以定多寡,非詔賜而隱占者俱追斷。戶部尚書梁材採其言,兼并者悉歸官。稍進左侍郎。

初,軏之平隴政也,以隴氏無後,請改設流官,兵部尚書李鉞等然之。遂改芒部為鎮雄府,分置四長官司,授隴氏疏屬阿濟等為長官,而擢重慶通判程洸為試知府。隴氏舊部沙保等攻執洸,奪其印,欲復立隴氏後。巡撫王廷相等破保,洸得還。保子普奴復連烏撒、水西苗攻剽畢節諸衛。帝命伍文定圖之。以朝議不合,召還。御史戴金因言:「芒部改流之議,諸司咸執不可。軏徇洸邪說,違眾獨行,致疆場不靖。」遂罷軏官。

以兵部尚書李承勛薦,起故官,總督倉場。再遷南京戶部尚書。御史龔湜劾軏老悖;吏部言軏居官儉素,搢紳儀表。帝乃責湜妄言。久之,就改兵部,參贊機務。詔舉將材,薦鄭卿、沈希儀等二十一人,皆擢用。居四年,以老乞罷。疏中言享年若干,帝以為非告君體,勒為民。久之卒。

徐問,字用中,武進人。弘治十五年進士。授廣平推官。遷刑部主事,歷兵部,出為登州知府。地濱海多盜,問盡捕之。調臨江。修築壞堤七十二。轉長蘆鹽運使。運司故利藪,自好者不樂居。問曰:「吾欲清是官也。」終任不取一錢。累遷廣東左布政使。

嘉靖十一年以治行卓異,拜右副都御史,巡撫貴州。獨山州賊蒙鉞弒父為亂。問聞南丹、泗城欲助逆,檄廣西撫按伐其謀。又檄鉞弟釗復父仇,事平得承襲。鉞援絕。問督大兵分道入,誅之。捷聞,賜金綺,召為兵部右侍郎。疏陳武備八事。又言:「兩廣、雲、貴半土司,深山密菁,瑤、僮、羅、僰所窟穴。邊將喜功召釁,好為掃穴之舉。王師每入,巨憝潛蹤,所誅戮率無辜赤子。興大兵,費厚餉,以易無辜命,非陛下好生意。宜敕邊臣布威信,嚴阨塞,謹哨探,使各安邊境,以絕禍萌。」帝深納其言。尋引疾歸。二十一年,召為南京禮部侍郎。久之,就遷戶部尚書。復引疾去,卒於家。

問清節自勵。居官四十年,敝廬蕭然,田不滿百畝。好學不倦,粹然深造,為士類所宗。隆慶初,謚莊裕。

張邦奇,字常甫,鄞人。年十五作《易解》及《釋國語》。登弘治末年進士,改庶吉士,授檢討。出為湖廣提學副使。下教曰:「學不孔、顏,行不曾、閔,雖文如雄、褒,吾且斥之。」在任三四年,諸生競勸。時世宗方為興世子,獻皇遣就試。乃特設兩案,己居北而使世子居南。文成,送入學。世宗由此知邦奇。嘉靖初,提學四川,以親孝乞歸。久之,桂萼掌銓,去留天下提學官,起邦奇福建。未幾,選外僚入坊局,改右庶子,遷南京祭酒。以身為教,學規整肅。就遷吏部侍郎。丁外艱歸。

帝嘗奉太后謁天壽諸陵,語及擇相。太后曰:「先皇嘗言提學張邦奇器識,他日可為宰相,其人安在?」帝憬然曰:「尚未用也。」服闋,即召為吏部右侍郎,掌部事。推轂善類,人不可干以私。銓部升除,多受教政府,邦奇獨否,大學士李時銜之。郭勛家人犯法,舁重賄請寬,邦奇不從。帝欲即授邦奇尚書,為兩人沮止。尋改掌翰林院事,充日講官,加太子賓客,改掌詹事府。九載考績,晉禮部尚書。以母老欲便養,乃改南京吏部。復改兵部,參贊機務。帝猶念邦奇,時與嚴嵩語及之。嵩曰:「邦奇性至孝,母老,不樂北來。」帝信其言,遂不召。二十三年卒,年六十一。贈太子太保,謚文定。

邦奇之學以程、朱為宗。宗王守仁友善,而語每不合。躬修力踐,跬步必謹。晝之所為,夕必書於冊。性篤孝,以養親故,屢起輒退。其母後邦奇卒,壽至百歲。邦奇事寡嫂如事母。所著《學庸傳》、《五經說》及文集,粹然一出於正。

族父時徹,少邦奇二十歲,受業於邦奇。仕至南京兵部尚書。有文名。

韓邦奇,字汝節,朝邑人。父紹宗,福建副使。邦奇登正德三年進士,除吏部主事,進員外郎。六年冬,京師地震,上疏陳時政闕失。忤旨,不報。會給事中孫禎等劾臣僚不職者,並及邦奇。吏部已議留,帝竟以前疏故,黜為平陽通判。遷浙江僉事,轄杭、嚴二府。宸濠令內豎假飯僧,聚千人於杭州天竺寺,邦奇立散遣之。其儀賓託進貢假道衢州,邦奇詰之曰:「入貢當沿江下,奚自假道?歸語王,韓僉事不可誑也。」

時中官在浙者凡四人,王堂為鎮守,晁進督織造,崔泬主市舶,張玉管營造。爪牙四出,民不聊生。邦奇疏請禁止,又數裁抑堂。邦奇閔中官採富陽茶魚為民害,作歌哀之。堂遂奏邦奇沮格上供,作歌怨謗。帝怒,逮至京,下詔獄。廷臣論救,皆不聽,斥為民。

嘉靖初,起山東參議。乞休去。尋用薦,以故官蒞山西。再乞休去。起四川提學副使,入為春坊右庶子。七年偕同官方鵬主應天鄉試,坐試錄謬誤,謫南京太僕丞。復乞歸。起山東副使,遷大理丞,進少卿,以右僉都御史巡撫宣府。入佐院事,進右副都御史,巡撫遼東。時遼陽兵變,侍郎黃宗明言邦奇素有威望,請假以便宜,速往定亂。帝方事姑息,不從,命與山西巡撫任洛換官。至山西,為政嚴肅,有司供具悉不納,間日出俸米易肉一斤。居四年,引疾歸。

中外交薦,以故官起督河道。遷刑部右侍郎,改吏部。拜南京右都御史,進兵部尚書,參贊機務。致仕歸。三十四年,陜西地大震,邦奇隕焉。贈太子少保,謚恭簡。

邦奇性嗜學。自諸經、子、史及天文、地理、樂律、術數、兵法之書,無不通究。著述甚富。所撰《志樂》,尤為世所稱。

弟邦靖,字汝度。年十四舉於鄉。與邦奇同登進士,授工部主事。榷木浙江,額不充,被劾,以守官廉得免。進員外郎。乾清宮災,指斥時政甚切。武宗大怒,下之詔獄。給事中李鐸等以為言,乃奪職為民。世宗即位,起山西左參議,分守大同。歲饑,人相食,奏請發帑,不許。復抗疏千餘言,不報。乞歸,不待命輒行。軍民遮道泣留。抵家病卒,年三十六。未幾,邦奇亦以參議蒞大同。父老因邦靖故,前迎,皆泣下。邦奇亦泣。

邦奇嘗廬居,病歲餘不能起。邦靖藥必分嘗,食飲皆手進。後邦靖病亟,邦奇日夜持弟泣,不解衣者三月。及歿,衰絰蔬食,終喪弗懈。鄉人為立「孝弟碑」。

周金,字子庚,武進人。正德三年進士。授工科給事中。累遷戶科都給事中。疏言:「京糧歲入三百五萬,而食者乃四百三萬,當痛為澄汰。中官迎佛及監織造者濫乞引鹽,暴橫道路,當罷。都督馬昂納有妊女弟,當誅昂而還其女。」朝議用兵土魯番,復哈密。金言西邊虛憊,而土魯番險遠,且青海之寇窺伺西寧,不宜計哈密。已,卒從金議。

嘉靖元年由太僕寺少卿遷都察院右僉都御史,巡撫延綏。邊人貧甚。金為招商聚粟,廣屯積芻,以時給其食。改撫宣府,進右副都御史。大同叛卒殺張文錦,邊鎮兵皆驕。宣府總督侍郎馮清苛刻。諸軍請糧不從,且慾鞭之,眾轟然圍清府署。金方病,出坐院門,召諸軍官數之曰:「是若輩剝削之過!」欲痛鞭之。軍士氣稍平,擁而前請曰:「總制不恤我耳!」金從容諭以利害,眾乃散解去,得無變。改撫保定。巡按御史李新芳疑廣平知縣謀己,欲抶之。知府為之解,並欲執知府,發兵二千捕之。知府及佐貳皆走,一城盡空。金發其罪狀,而都御史王廷相庇新芳,與相爭。帝卒下新芳刑部,黜官。金遷兵部右侍郎。未幾,進右都御史,總督漕運,巡撫鳳陽諸府。久之,擢南京刑部尚書,就轉戶部。二十四年致仕歸,歲餘卒。贈太子太保,謚襄敏。

吳嶽,字汝喬,汶上人。嘉靖十一年進士。授戶部主事,歷郎中。督餉宣府,吏進羨金數千,拒之。出知廬州府。稅課歲萬金,例輸府,嶽以代郵傳費。西山薪故供官爨,嶽弛以利民。以憂去。服除,改保定,治如廬州。歷山西副使、浙江參政、湖廣按察使、山西右布政使,並以清靜得民。遷右僉都御史,巡撫保定六府。奏裁徵發冗費十六七,民力遂寬。甫浹歲,引疾去。久之,以貴州巡撫征。尋進左副都御史,協理院事。隆慶元年,歷吏部左、右侍郎。京察竣,給事中胡應嘉有所申救。嶽詣內閣抗聲曰:「科臣敢留考察罷黜官,有故事乎?」應嘉遂得譴。遷南京禮部尚書,就改吏部。抑浮薄,杜僥倖,南都縉紳憚之。上疏陳六事,帝頗納其言。尋改兵部,參贊機務。未上,給由過家,病卒。詔贈太子太保,謚介肅。

嶽清望冠一時,禔躬嚴整。尚書馬森言平生見廉節士二人,嶽與譚大初耳。岳知廬州時,王廷守蘇州,以公事遇京口。嶽召為金山遊,攜酒一缶,肉一斤,菜數束。廷笑曰:「止是乎?」嶽亦笑曰:「足供我兩人食矣。」歡竟日而還。去廬日,假一蓋禦雨,至即命還之。

譚大初,字宗元,始興人。嘉靖十七年進士。授工部主事。憂歸。起補戶部,改戶科給事中。數論事。歷兵科左給事中,出為江西副使。清軍,多所釋。御史孫慎以失額為疑,大初曰:「失額罪小,殃民罪大。」嚴嵩親黨奪民田,治之不少貸。遷廣西右參政,投劾歸。久之,起故官河南。未上,擢南京右通政。俄遷應天府尹。將赴南都,而穆宗即位,乞以參政致仕,不許。隆慶元年召拜工部右侍郎,尋遷戶部左侍郎,督倉場。海瑞為僉都御史,大初力薦瑞。已而屢疏乞休,不允。拜南京戶部尚書,引疾去。家居,田不及百畝。卒年七十五。謚莊懿。

贊曰:當正、嘉之際,士大夫刓方為圓,貶其素履,羔羊素絲之節浸以微矣。陶琰諸人清操峻特,卓然可風。南都列卿,後先相望,不其賢乎。琰之督漕,充嗣之守御,良永之遏錢寧,周金之弭亂卒,所豎立甚偉。至琰子之直節,廷弼、邦靖之篤行,尤無忝其父兄云。


\end{pinyinscope}