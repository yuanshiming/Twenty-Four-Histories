\article{列傳第一百十九}

\begin{pinyinscope}
顧憲成歐陽東鳳吳炯顧允成張納陛賈巖諸壽賢彭遵古錢一本子春於孔兼陳泰來史孟麟薛敷教安希范吳弘濟譚一召孫繼有劉元珍龐時雍葉茂才

顧憲成,字叔時,無錫人。萬曆四年,舉鄉試第一。八年成進士,授戶部主事。大學士張居正病,朝士群為之禱,憲成不可。同官代之署名,憲成手削去之。居正卒,改吏部主事。請告歸三年,補驗封主事。

十五年,大計京朝官,都御史辛自修掌計事。工部尚書何起鳴在拾遺中,自修坐是失執政意。給事中陳與郊承風旨並論起鳴、自修,實以攻自修而庇起鳴。於是二人並罷,并責御史糾起鳴者四人。憲成不平,上疏語侵執政,被旨切責,謫桂陽州判官。稍遷處州推官。丁母憂,服除,補泉州推官。舉公廉第一。擢吏部考功主事,歷員外郎。會有詔三皇子並封王。憲成偕同官上疏曰:

皇上因《祖訓》立嫡之條,欲暫令三皇子並封王,以待有嫡立嫡,無嫡立長。臣等伏而思之,「待」之一言,有大不可者。太子,天下本。豫定太子,所以固本。是故有嫡立嫡,無嫡立長,就見在論是也,待將來則非也。我朝建儲家法,東宮不待嫡,元子不並封。廷臣言甚詳,皇上概弗省,豈皇上創見有加列聖之上乎?有天下者稱天子,天子之元子稱太子。天子繫乎天,君與天一體也;太子繫乎父,父子一體也。主鬯承祧,於是乎在,不可得而爵。今欲並封三王,元子之封何所係乎?無所係,則難乎其為名;有所係,則難乎其為實。

皇上以為權宜云耳。夫權宜者,不得已而行之也。元子為太子,諸子為籓王,於理順,於分稱,於情安,有何不得已而然乎?耦尊鈞大,逼所由生。皇上以《祖訓》為法,子孫以皇上為法。皇上不難創其所無,後世詎難襲其所有?自是而往,幸皆有嫡可也,不然,是無東宮也。又幸而如皇上之英明可也,不然,凡皇子皆東宮也,無乃啟萬世之大患乎?皇后與皇上共承宗祧,期於宗祧得人而已。皇上之元子諸子,即皇后之元子諸子。恭妃、皇貴妃不得而私之,統於尊也。豈必如輔臣王錫爵之請,須拜皇后為母,而後稱子哉?

況始者奉旨,少待二三年而已,俄改二十年,又改於二十一年,然猶可以歲月期也。今曰「待嫡」,是未可以歲月期也。命方布而忽更,意屢遷而愈緩。自並封命下,叩閽上封事者不可勝數,至里巷小民亦聚族而竊議,是孰使之然哉?人心之公也。而皇上猶責輔臣以擔當。錫爵夙夜趣召,乃排群議而順上旨,豈所謂擔當?必積誠感悟納皇上於無過之地,乃真擔當耳。不然,皇上且不能如天下何,而況錫爵哉!

皇上神明天縱,非溺寵狎暱之比。而不諒者,見影而疑形,聞響而疑聲,即臣等亦有不能為皇上解者。皇上盛德大業,比隆三五。而乃來此意外之紛紛,不亦惜乎!伏乞令皇元子早正儲位,皇第三子、皇第五子各就王爵。父父子子,君君臣臣,兄兄弟弟。宗廟之福,社稷之慶,悉在是矣。

憲成又遺書錫爵,反覆辨論。其後並封議遂寢。

二十一年京察。吏部尚書孫籥、考功郎中趙南星盡黜執政私人,憲成實左右之。及南星被斥,憲成疏請同罷,不報。尋遷文選郎中。所推舉率與執政牴牾。先是,吏部缺尚書,錫爵欲用羅萬化,憲成不可,乃用陳有年。後廷推閣臣,萬化復不與。錫爵等皆恚,萬化乃獲推,會帝報罷而止。及是,錫爵將謝政,廷推代者。憲成舉故大學士王家屏,忤帝意,削籍歸。事具有年傳。

憲成既廢,名益高,中外推薦無慮百十疏,帝悉不報。至三十六年,始起南京光祿少卿,力辭不就。四十年,卒於家。天啟初,贈太常卿。魏忠賢亂政,其黨石三畏追論之,遂削奪。崇禎初,贈吏部右侍郎,謚端文。

憲成姿性絕人,幼即有志聖學。暨削籍里居,益覃精研究,力闢王守仁「無善無惡心之體」之說。邑故有東林書院,宋楊時講道處也,憲成與弟允成倡修之,常州知府歐陽東鳳與無錫知縣林宰為之營構。落成,偕同志高攀龍、錢一本、薛敷教、史孟麟、于孔兼輩講學其中,學者稱涇陽先生。當是時,士大夫抱道忤時者,率退處林野,聞風響附,學舍至不能容。憲成嘗曰:「官輦轂,志不在君父,官封疆,志不在民生,居水邊林下,志不在世道,君子無取焉。」故其講習之餘,往往諷議朝政,裁量人物。朝士慕其風者,多遙相應和。由是東林名大著,而忌者亦多。

既而淮撫李三才被論,憲成貽書葉向高、孫丕揚為延譽。御史吳亮刻之邸抄中,攻三才者大嘩。而其時于玉立、黃正賓輩附麗其間,頗有輕浮好事名。徐兆魁之徒遂以東林為口實。兆魁騰疏攻憲成,恣意誣詆。謂滸墅有小河,東林專其稅為書院費;關使至,東林輒以書招之,即不赴,亦必致厚餽;講學所至,僕從如雲,縣令館穀供億,非二百金不辦;會時必談時政,郡邑行事偶相左,必令改圖;及受黃正賓賄。其言絕無左驗。光祿丞吳炯上言為一致辨,因言:「憲成貽書救三才,誠為出位,臣嘗咎之,憲成亦自悔。今憲成被誣,天下將以講學為戒,絕口不談孔、孟之道,國家正氣從此而損,非細事也。」疏入,不報。嗣後攻擊者不絕,比憲成歿,攻者猶未止。凡救三才者,爭辛亥京察者,衛國本者,發韓敬科場弊者,請行勘熊廷弼者,抗論張差梃擊者,最後爭移宮、紅丸者,忤魏忠賢者,率指目為東林,抨擊無虛日。借魏忠賢毒焰,一網盡去之。殺戮禁錮,善類為一空。崇禎立,始漸收用。而朋黨勢已成,小人卒大熾,禍中於國,迄明亡而後已。

歐陽東鳳,字千仞,潛江人。年十四喪父,哀毀骨立。母病嘔血,跪而食之。舉於鄉,縣令憫其貧,遺以田二百畝,謝不受。萬曆十七年成進士,除興化知縣。大水壞堤,請振於上官不應,遂自疏於朝。坐越奏停俸,然竟如所請。屢遷南京刑部郎中,擢平樂知府。撫諭生瑤,皆相親如子弟。因白督學監司,擇其俊秀者入學,瑤漸知禮讓。稅使橫行,東鳳力抗之。以才調常州。布帷瓦器,胥吏不能牟一錢,擒奸人劇盜且盡。憲成輩講學,為建東林書院。居四年,謝事歸。起山西副使,擢南京太僕少卿,並辭不就。卒於家。

吳炯,字晉明,松江華亭人。萬曆十七年成進士,授杭州推官。入為兵部主事,乞假歸。恬靜端介,不騖榮利。家居十二年,始起故官。久之,進光祿丞。天啟中,累遷南京太僕卿。魏忠賢私人石三畏追論炯黨庇憲成,落職閒住。崇禎初,復官。炯家世素封,無子,置義田以贍族人。郡中貧士及諸生赴舉者,多所資給。嘗輸萬金助邊,被詔旌獎。

顧允成,字季時,憲成弟。性耿介,厲名節,舉萬曆十一年會試,十四年始赴殿試。對策中有曰:「陛下以鄭妃勤於奉侍,冊為皇貴妃,廷臣不勝私憂過計。請立東宮,進封王恭妃,非報罷則峻逐。或不幸貴妃弄威福,其戚屬左右竊而張之,內外害可勝言!頃張居正罔上行私,陛下以為不足信,而付之二三匪人。恐居正之專,尚與陛下二。此屬之專,遂與陛下一。二則易間,一難圖也。」執政駭且恚,置末第。

會南畿督學御史德清人房寰連疏詆都御史海瑞,允成不勝憤。偕同年生彭遵古、諸壽賢抗疏劾之。略言:「寰妒賢醜正,不復知人間羞耽事。臣等自幼讀書,即知慕瑞,以為當代偉人。寰大肆貪汙,聞瑞之風,宜愧且死,反敢造言逞誣,臣等所為痛心。」因劾其欺罔七罪。始寰疏出,朝野多切齒。而政府庇之,但擬旨譙讓。及得允成等疏,謂寰已切讓,不當出位妄奏,奪三人冠帶,還家省愆,且令九卿約束辦事進士,毋妄言時政。南京太僕卿沈思孝上言:「二三年來,今日以建言防人,明日以越職加人罪,且移牒諸司約禁,而進士觀政者,復令堂官鉗束之。夫禁其作奸犯科可也,而反禁其讜言直諫;教其砥行立節可也,而反教以緘默取容。此風一開,流弊何極。諫官避禍希寵不言矣,庶官又不當言;大臣持祿養交不言矣,小臣又不許言。萬一權奸擅朝,傾危宗社,陛下安從聞之?臣歷稽先朝故事,練綱、鄒智、孫磐、張璁並以書生建言,未聞以為罪,獨奈何錮允成等耶?」疏入,忤旨被責,三人遂廢。寰復詆瑞及思孝,其言絕狂誕,自是獲罪清議,出為江西副使。給事中張鼎思劾其奸貪,寰亦訐鼎思請寄事。諸給事中不平,連章攻寰,寰與鼎思並謫,遂不復振。

久之,南京御史陳邦科請錄用允成等,不許。巡按御史復言之,詔許以教授用。允成歷任南康、保定。入為國子監博士,遷禮部主事。三王並封制下,偕同官張納陛、工部主事岳元聲合疏諫曰:「冊立大典,年來無敢再瀆者,以奉二十一年舉行之明詔。茲既屆期,群臣莫不引領。而元輔王錫爵星駕趣朝,一見禮部尚書羅萬化、儀制郎于孔兼,即戒之弗言,慨然獨任,臣等實喜且慰。不意陛下出禁中密札,竟付錫爵私邸,而三王並封之議遂成,即次輔趙志皋、張位亦不預聞。夫天下事非一家私議。元子封王,祖宗以來未有此禮,錫爵安得專之,而陛下安得創之!」當是時,光祿丞朱維京、給事中王如堅疏先入。帝震怒,戍極邊。維京同官塗傑、王學曾繼之,斥為民。及是諫者益眾,帝知不可盡斥,但報「遵旨行」。已而竟寢。

未幾,吏部尚書孫鑨等以拾遺事被責。允成謂閣臣張位實為之,上疏力詆位,因及錫爵。納陛亦抗章極論,并侵附執政者。帝怒,謫允成光州判官,納陛鄧州判官。皆乞假歸,不復出。

納陛,字以登,宜興人。年十六,從王畿講學。舉萬曆十七年進士。由刑部主事改禮部。生平尚風節。鄉邑有利害,輒為請於有司而後已。東林書院之會,納陛為焉。又與同邑史孟麟、吳正志為麗澤大會,東南人士爭赴之。

時與允成等同以部曹爭三王並封,又爭拾遺事者,戶部主事滁人賈巖,亦貶曹州判官。投劾歸,卒。天啟中,贈允成、納陛光祿少卿,巖尚寶丞。

諸壽賢,字延之,崑山人。既釋褐,上疏願放歸田,力學十年,然後從政。章下所司,寢不奏。既斥歸。久之,起南陽教授。入為國子助教,擢禮部主事。戚里中貴幹請,輒拒之。遘疾,請告歸,授徒自給。久之卒。

彭遵古,麻城人,終光祿少卿。

錢一本,字國瑞,武進人。萬曆十一年進士。除廬陵知縣,徵授御史。入臺即發原任江西巡按祝大舟貪墨狀,大舟至遣戍。已,論請從祀曹端、陳真晟、羅倫、羅洪先於文廟。出按廣西。

帝以張有德請備大禮儀物,復更冊立東宮期,而申時行柄國,不能匡救。一本上論相、建儲二疏。其論相曰:

昨俞旨下輔臣,令輔臣總政。夫朝廷之政,輔臣安得總之?內閣代言擬旨,本顧問之遺,遇有章奏,閣臣宜各擬一旨。今一出時行專斷。皇上斷者十一,時行斷者十九。皇上斷謂之聖旨,時行斷亦謂之聖旨。惟嫌怨所在,則以出自聖斷為言,罪何可勝誅。所當論者一。

評事雒于仁進四藥之箴,陛下欲見之施行,輔臣力勸留中。既有言及輔臣之章,亦盡留中不下。道吾君以遂非文過如此,復安望其盡忠補過耶?所當論者二。

科場弊竇,汙人齒頰,而敢擬原無私弊之旨,以欺吾君。臣請執政子弟有中式而被人指摘者,除名改蔭。又與見從仕籍者,暫還里居,俟父致政,乃議進止。毋令犬馬報主之心,不勝其牛馬子孫之計。所當論者三。

大臣以身殉國,安復有家。乃以遠臣為近臣府庫,又合遠近之臣為內閣府庫。開門受賂自執政始,而歲歲申餽遺之禁何為哉?所當論者四。

墨敕斜封,前代所患;密啟言事,先臣弗為。今閣臣或有救援之舉,或有密勿之謀,類具揭帖以進,雖格言正論,讜議忠謀,已類斜封密啟之為,非有公聽並觀之正。況所言公,當與天下公言之;所言私,忠臣不私。奈何援中書之故事,啟留中之弊端,昭恩怨之所由,示威福之自己。所當論者五。

我國家仿古為治,部院即分職之六卿,內閣即論道之三公。未聞三公可盡攬六卿之權,歸一人掌握,而六卿又頫首屏氣,唯唯聽命於三公,必為請教而後行也。所當論者六。

三公職在論道。師,道之教訓。今講幄經年不御,是何師也?傅,傅之德義。今外帑匱乏,私藏充盈,不能一為救正,是何傅也?保,保其身體。今聖躬常年靜攝,尚以多疾為辭,是何保也?其兼銜必曰太子之師、傅、保,而冊立皇元子之儀,至今又復改遲,臣不知其所兼者何職矣。所當論者七。

翰林一途,謂之儲相。累貲躡級,循列卿位,以覬必得。遂使國家命相之大任,僅為閣臣援引之私物。庸者習軟熟結納之態,黠者恣憑陵侵奪之謀。外推內引,璫閣表裏。始進不正,安望其終?故自來內閣之臣一據其位,遠者二十年,近者十年,不敗不止。嵩之鑒不遠,而居正蹈之;居正之鑒不遠,而時行又蹈之。繼其後者庸碌罷駑,或甚於時行;褊隘執拗,又復為居正。若非大破常格,公天下以選舉,相道終未可言。所當論者八。

先民詢芻蕘之言,明王設誹謗之木。今大臣懼人攻己,而欲鉗天下之口,不目之為奸、為邪、為浮薄,必詈之為讒、為謗、為小人。目前之耳目可塗,身後之是非難罔。所當論者九。

君臣之分,等於天地。今上名之曰總政,己亦居之曰總政。以其身居於寵利之極,耐彈忍辱,必老死於位而後已。古所謂元老大臣,乃如是其不知進退存亡者耶?大臣既無難進易退之節,天下安有頑廉懦立之風!舉一世之人心風俗,糜爛於乞祼登壟之坑,滔滔而莫之止。是故陛下之治,前數年不勝其操切慘刻,而勢焰爍人;後數年不勝其姑息委靡,而賢愚共貫。前之政自居正總,今之政自時行總,而皆不自朝廷總故也。所當論者十。

然君道莫先論相,而取人亦在君身,願陛下勿以國本為兒戲。昔孔子以九經告君,而先之修身、勸賢。大抵讒夫女謁貨利之交,一有惑溺,則內之心志決不清明,外之身體決不強固。矧以艷處之褒姒,而為善譖之驪姬,狐媚既以蠱其心,鹿臺又復移其志。陛下之方寸,臣知其不能自持者多矣,抑何以貴德尊士,而修身取人哉!

其論國本曰:

陛下所以遲遲建儲者,謂欲效皇祖世宗之為耳。然皇祖中年嘗立莊敬為太子,封皇考為裕王,非終不立太子也。矧今日事體又迥然不同。皇貴妃寵過皇后。其處心積慮,無一日而不萌奪嫡之心,無一日而不思為援立其子之計。此世宗時所無也。凡子必依於母,皇元子之母壓於皇貴妃之下。陛下曰「長幼有序」,皇貴妃曰「貴賤有等」。倘一日遂其奪嫡之心,不審陛下何以處此?此世宗時所無也。景王就封,止皇考一人在京。今則章服不別,名分不正。弟既憑母之寵而朝夕近倖,母又覬子之立而日夜樹功。此世宗時所無也。傳聞陛下先曾失言於皇貴妃,皇貴妃執此為信。及今不斷,蠱惑日深,剛斷日餒,事體日難。此世宗時所無也。

前者有旨,不許諸司激擾,愈致遲延,非陛下預設機阱,以禦天下言者乎!使屆期無一人言及,則佯為不知,以冀其遲延。有一人言及,則禦之曰「此來激擾我也」,改遲一年。明年又一人言及,則又曰「此又來激擾我也」,又改二三年。必使天下無一人敢言而後已,庶幾依違遷就,以全其衽席暱愛之私,而曾不顧國本從此動搖,天下從此危亂。臣以為陛下之禦人至巧,而為謀則甚拙也。此等機智,不可以罔匹夫匹婦,顧欲以欺天下萬世耶!

疏入,留中。時廷臣相繼爭國本,惟一本言最戇直。帝銜之。無何,杖給事中孟養浩。中旨以養浩所逞之詞根托一本,造言誣君,搖亂大典,遂斥為民。屢薦,卒不用。一本既罷歸,潛心《六經》濂、洛諸書,尤研精《易》學。與顧憲成輩分主東林講席,學者稱啟新先生。里居二十五年,預剋卒日,賦詩志之,如期而逝。天啟初,贈太僕寺少卿。

子春,字若木,萬曆三十二年進士。歷知高陽、獻二縣,徵授御史。太僕少卿徐兆魁攻李三才,因痛詆顧憲成。春三疏首發其憸邪。出按湖廣,請予禮部侍郎郭正域及光祿少卿顧憲成恤典。楚宗人以訐偽王事,錮高牆者甚眾,春為訟冤,尋復請釋回故宗家屬,語甚切至。咸寧知縣滿朝薦久繫,奏請釋之,因請并釋王邦才、卞孔時。又再疏劾守備中官杜茂,且備陳採榷之害,言:「臣不忍皇上聽小人之謀,名出漢桓、唐德下,為我明基禍之主。」帝以湖廣地為福王莊田。春三疏力爭,帝降旨切責。葉向高致政去,方從哲為首輔。春抗疏言:「今天下人材則朝虛野實,貨財則野虛朝實。從哲不能救正,而第於福王則無事不曲從。臣嘗歎皇上有為堯、舜之資,而輔佐無人。僅得王家屏、沈鯉,又俱不信用。其餘大抵庸惡陋劣,奸回媢嫉之徒,不意至從哲而風益下。臣聞從哲每向人言,輒云內相之意,是甘為萬安、焦芳,曾趙志皋,沈一貫之不若也。」從哲疏辨乞去。帝慰留,而責春妄言瀆奏,出為福建右參議。尋丁父艱。天啟初,起故官。召為尚寶少卿,歷遷光祿卿。五年,魏忠賢黨門克新劾春倚恃東林,父作子述,削籍歸。

崇禎九年,召拜通政使。遷戶部右侍郎,歷尚書。總督倉場,條行釐弊十事。以勞瘁予告。未幾,起南京戶部尚書。疏請皇太子出閣,從之。累疏引疾,不允。九年,條上戰守之策,并論賊三可擊狀。帝如議敕行。十一年,黃道周、劉同升等諫楊嗣昌奪情,被貶謫。范景文等疏救,春名與焉。明年正月,削景文籍,置春不問。春為御史,甚有聲。及居大僚,循職無咎。會上疏請改折白糧,忤旨,罷歸。是年卒。

于孔兼,字元時,金壇人。萬曆八年進士。授九江推官。入為禮部主事,再遷儀制郎中。疏論都御史吳時來晚節不終,不當謚忠恪,因請謚楊爵、陳瓚、孟秋。乃奪時來謚,而謚爵忠介。大學士王家屏以爭冊立求去。孔兼上言:「陛下徇內嬖之情,而搖主鬯之器。不納輔臣之言,反重諫官之罰。且移怒吏部,削籍三人。夫萬國欽獲罪申時行,饒伸獲罪王錫爵,非獲罪於陛下也。輔臣於數千里外,能遙制朝權若此,毋乃陛下以此示恩,欲其復來共成他圖耶!自陛下有近日之舉,而善類寒心,邪臣鼓掌。將來逢君必巧,豫教無期,申生、楊廣再見於今,此宗廟之不利,非直臣等憂也。」帝得疏,怒甚。已,竟留中。

明年正月,有詔並封三王。孔兼與員外郎陳泰來合疏爭曰:「立嫡之訓,自古有之。然歷考祖宗以來,未有虛東宮之位以候嫡子者。昔陛下正位東宮,年甫六歲,仁聖皇太后方在盛年,先皇帝曾不少待,陛下豈不省記乎?地逼則嫌生,禮殊則分定。願收還新諭,建儲、封王一時並舉,宗社幸甚。」未報。孔兼又言:「陛下堅持待嫡之說,既疑群臣謗訕,又謂朝綱倒持,遂欲坐諫者以無禮於君之罪。夫謂元子當立不容緩者,君子也。此有禮於君者,王如堅諸人是也。謂並封可行逢上意者,小人也。此無禮於君者,許夢熊一人是也。今欲以無禮之罪,而加之有禮於其君者,何以服人心,昭國法?臣又惟巫蠱之謗啟於堯母;承乾之誅成於偏愛。自古亂臣,未有不窺人君之隙而逢迎以遂其奸者。始錫爵之兩諭並擬,其負國誤君大矣。既不能轉移君心決計於初,乃以杜門求去為計。夫前無失策,一去可以成名。失而後爭,爭而不得,雖去不足塞責矣。人謂錫爵言無不盡,特苦陛下聽斷之不行。臣則云陛下悔心已萌,特憂錫爵感孚之未至。若姑云徐徐,坐視君父之過舉,錫爵縱不為宗社計,獨不為身名計乎?」會廷臣多諫者,其事竟寢。

亡何,考功郎中趙南星坐京察削籍。孔兼、泰來各疏救。帝積前恨,謫孔兼安吉判官,泰來饒平典史。孔兼投牒歸。家居二十年,杜門讀書,矩矱整肅,鄉人稱之無間言。

泰來,字伯符,平湖人。年十九,舉萬曆五年進士,授順天教授,進國子博士。見執政與言路相水火,上書規之,坐是五年不調。南京禮部郎中馬應圖,泰來同邑,又同年生也,十三年,上疏譏切執政,又力詆給事中齊世臣,御史龔懋賢、蔡系周、孫愈賢、吳定,而盛稱吳中行、趙用賢、沈思孝、李植諸人。忤旨,謫大同典史。給事中王致祥、御史柴祥等希執政意,復連章劾應圖,且言泰來為點定奏章。帝以應圖既貶不問。泰來引疾歸。久之,起禮部主事,進員外郎。疏請建儲,不報。踰年遂卒,年三十六。天啟中,孔兼、泰來俱贈光祿少卿。

于氏為金壇望族。孔兼祖湛,戶部侍郎。兄文熙,大名兵備副使。再從弟仕廉,南京戶部侍郎,有清望。史孟麟,字際明,宜興人。萬曆十一年進士。授庶吉士,改吏科給事中。疏劾少詹事黃洪憲典試作奸,左都御史吳時來沮抑言路。執政庇之,格不行。員外郎趙南星、主事姜士昌相繼劾兩人,并及副都御史詹仰庇。執政滋不說。吏科都給事中陳與郊素附執政,屬同官李春開三疏訐南星、士昌妄言。帝止下春開疏,而留南星、士昌奏不發。給事中王繼光、萬自約不平,復抗章論時來等,詞甚峻切。孟麟亦上疏力攻春開,語并侵執政,因求罷,不許。孟麟竟自引歸。春開亦謝病去,後以考察罷。孟麟尋召為兵科右給事中。

二十年,大學士趙志皋、張位言:「凡會議會推,並令廷臣類奏,取自上裁,用杜專權。」孟麟疏爭曰:「自臣通籍以來,竊見閣臣侵部院之權,言路希閣臣之指,官失其守,言失其責久矣。陛下更置輔臣,與天下更始,政事歸六部,公論付言官,天下方欣欣望治,奈何忽有此令?曩太祖罷中書省,分設六部,恐其專也;而官各有職,不相侵越,則又惟恐其不專。蓋以一事任一官,則專不為害;即使敗事,亦罪有所歸。此祖宗建官之意也。今令諸臣各書所見,類奏以聽上裁,則始以一部之事,分而散之於諸司;究以諸司之權,合而收之於禁密。事雖上裁,旨由閣擬。脫有私意奸其間,內託上旨,外諉廷言,誰執其咎?又脫有馮保、張居正者,夤緣為奸,授意外廷,小人趨承,扶同罔上,朝廷不得察其非,當官不能爭其是,又誰執其咎?臣竊謂政權分之六部,不可以為專。惟六部不專,則必有專之者。是乃收攬威權之漸,必不可從也。」忤旨,不納。

再遷吏科都給事中。三王並封議起,孟麟、于孔兼等詣王錫爵邸爭之。又進《或問》一篇,別白尤力。尚書孫鑨、考功郎中趙南星掌癸巳京察,孟麟實佐之。南星以讒言斥,孟麟亦引疾歸。召拜太僕少卿,復以疾去。

孟麟素砥名節,復與東林講會,時望益重。家居十五年,召起故官,督四夷館。會睹梃擊事,疏請冊立皇太孫,絕群小覬覦之望。且救御史劉光復。帝怒,謫兩浙鹽運判官。熹宗立,稍遷南京禮部主事。累擢太僕卿,卒。

薛敷教,字以身,武進人。祖應旂,字仲常。嘉靖十四年進士。由慈谿知縣屢遷南京考功郎中,主京察。大學士嚴嵩嘗為給事中王曄所劾,囑尚寶丞諸傑貽書應旂,令黜曄。應旂反黜傑,嵩大怒。應旂又黜常州知府符驗,嵩令御史桂榮劾應旂挾私黜郡守,謫建昌通判。歷浙江提學副使。應旂雅工場屋文字,與王鏊、唐順之、瞿景淳齊名。其閱文所品題,百不失一。以大計罷歸,顧憲成兄弟方少,從之學,敷教遂與善,用風節相期許。及舉萬歷十七年進士,與高攀龍同出趙南星門,益以名教自任。

會南京御史王籓臣疏劾巡撫周繼,不具揭都察院,為其長耿定向所劾。左都御史吳時來因請申飭憲規,籓臣坐停俸。敷教上言:「時來壅遏言路,代人狼噬。而二三輔臣,曲學險詖,又故繩庶寀,以崇九列,塞主上聰明。宜嚴黨邪之禁,更易兩都臺長,以清風憲。」疏上,大學士申時行等疏言:「故事,御史建白,北京即日投揭臺長,南京則以三日。籓臣廢故事,薄罰未為過。必如敷教言,將盡抑大臣而後可耶?」副都御史詹仰庇劾敷教煽惑人心,淆亂國是。詔敷教歸,省過三年,以教職用。大學士許國以敷教其門生,而疏語侵己,尤憤,自請罷斥。因言:「邇來建言成風,可要名,可躐秩,又可掩過,故人競趨之為捷徑,此風既成,莫可救止。方今京師訛言東南赤旱,臣未為憂,而獨憂此區區者,彼止一時之災,此則世道之慮也。」時來亦乞休,力詆敷教及主事饒伸。帝慰留國、時來。都給事中陳與郊復上疏極詆建言諸臣,帝亦不問。

二十年夏,起敷教鳳翔教授,旋遷國子助教。明年,力爭三王並封,又上書王錫爵。尋以救南星,謫光州學正。省母歸,遂不復出。敷教禔身嚴苦,垢衣糲食,終身未嘗受人饋。家居二十年,力持清議,大吏有舉動,多用敷教言而止。後與憲成兄弟及攀龍輩講學。卒,贈尚寶司丞。

安希范,字小范,無錫人。萬曆十四年進士。授行人。遷禮部主事,乞便養母,改南京吏部。二十一年,行人高攀龍以趙用賢去國,疏爭之,與鄭材、楊應宿相訐。攀龍謫揭陽典史。御史吳弘濟復爭,亦被黜。希范上疏曰:「近年以來,正直之臣不安於位。趙南星、孟化鯉為選郎,秉公持正,乃次第屏黜。趙用賢節概震天下,止以吳鎮豎子一疏而歸,使應宿、材得窺意指,交章攻擊。至如孫金龍之清修公正,李世達之練達剛明,李禎之孤介廉方,並朝廷儀表。鑨、世達先後去國,禎亦堅懷去志,天下共惜諸臣不用,而疑閣臣媢嫉,不使竟其用也。高攀龍一疏,正直和平,此陛下忠臣,亦輔臣諍友。至如應宿辨疏,塗面喪心,無復人理。明旨下部科勘議,未嘗不是攀龍非應宿。及奉處分之詔,則應宿僅從薄謫,攀龍又竄炎荒。輔臣誤國不忠,無甚於此。乃動輒自文,諉之宸斷。坐視君父過舉,弼違補袞之謂何!茍俟降斥之後,陽為申救,以愚天下耳目,而天下早已知其肺腑矣。吳弘濟辨別君子小人,較若蒼素,乃與攀龍相繼得罪。臣之所惜,不為二臣,正恐君子皆退,小人皆進,誰為受其禍者。乞陛下立斥應宿、材,為小人媚灶之戒;復攀龍、弘濟官,以獎忠良;并嚴諭閣臣王錫爵,無挾私植黨,仇視正人。則相業光而聖德亦光矣。」時南京刑部郎中譚一召、主事孫繼有方以劾錫爵被譴。希范疏入,帝怒,斥為民。希范恬靜簡易,與東林講學之會。熹宗嗣位,將起官,先卒。贈光祿少卿。

吳弘濟,字春陽,秀水人。希范同年進士。由蒲圻知縣擢御史。連劾福建巡撫司汝濟、大理卿吳定、戎政侍郎郝傑、薊遼總督顧養謙,不納。三王並封詔下,偕同官抗疏爭。既而以論應宿、攀龍事,貶二秩調外。王錫爵等疏救,給事、御史、執政疏每上,輒重其罰,竟斥為民。未幾卒。熹宗時,贈官如希范。

譚一召,大庾人。孫繼有,餘姚人。一召疏曰:「輔臣錫爵再輔政以來,斥逐言者無虛月。攀龍、弘濟之黜,一何甚也。自趙南星秉公考察,錫爵含怒積憤。故南星一掛彈章而斥,于孔兼、薛敷教、張納陛等以申救而斥,孟化鯉等以推張棟而斥,李世達、孫鑨又相繼罷去矣。怒心橫生,觸事輒發,又安知是非公論耶!」繼有疏曰:「吳弘濟救攀龍則黜,黃紀賢、吳文梓救弘濟則罰,鄭材傾陷善類,而黜罰不加,何其舛也。今所指為攀龍罪者,以攀龍謂陛下不親一事,批答盡出輔臣。然疏內初無此語,何以服攀龍心?然此猶小者耳。本兵、經略,安危所係,乃以匪人石星、宋應昌任之,豈不誤國家大計哉!」與一召疏並上。帝怒曰:「近罪攀龍,出朕獨斷。小臣無狀,詆誣閣臣,朋奸黨惡,不可不罪。其除一召名,謫繼有極邊雜職。」給事中葉繼美疏救二人及希范。帝益怒,并除繼有名,遣官逮希范、一召,奪繼美俸一年。錫爵力救,詔免逮。諸人遂廢於家。繼有終知府。

劉元珍,字伯先,無錫人。萬曆二十三年進士。初授南京禮部主事,進郎中,親老歸養。起南京職方,釐汰老弱營軍,歲省銀二萬有奇。

三十三年京察,吏部侍郎楊時喬、都御史溫純,盡黜政府私人錢夢皋等。大學士沈一貫密為地,詔給事、御史被黜者皆留,且不下察疏。元珍方服闋需次,抗疏言:「一貫自秉政以來,比暱憸人,叢集奸慝,假至尊之權以售私,竊朝廷之恩以市德,罔上不忠,孰大於是!近見夢皋有疏,每以黨加人。從古小人未有不以朋黨之說先空善類者。所關治亂安危之機,非細故也。」疏奏,留中。一貫亟自辨,乞明示獨斷之意,以釋群疑。夢皋亦詆元珍為溫純鷹犬。疏皆不報。未幾,敕諭廷臣以留用言官之故,貶元珍一秩,調邊方。一貫佯救,給事、御史侯慶遠、葉永盛等亦爭之,不從。時員外郎賀燦然、南京御史朱吾弼相繼論察典。而主事龐時雍則直攻一貫欺罔者十,誤國者十,且曰:「一貫之富貴日崇,陛下之社稷日壞。頃南郊雷震,正當一貫奏請頒行敕諭之時。意者天厭其奸,以警悟陛下,俾早除讒慝乎!」帝得疏怒,命并元珍、燦然貶三秩,調極邊。頃之,慶遠及御史李柟等申救。帝益怒,奪其俸,謫元珍等極邊雜職。俄御史周家棟指陳時政,語過激。帝遷怒元珍等,皆除其名。然察疏亦下,諸被留者皆自免去。

光宗即位,起元珍光祿少卿。時遼、沈既沒,故贊畫主事劉國縉入南四衛,以招撫軍民為名,投牒督餉侍郎,令發舟南濟。議者欲推為東路巡撫,元珍上疏言:「國縉乃李成梁義兒,成梁棄封疆,國縉為營免,遂基禍本。楊鎬、李如柏喪師,國縉甫為贊畫,即奏保二人,欲坐杜松以違制。創議用遼人,冒官帑二十萬金募土兵三萬,曾不得一卒之用。被劾解官,乃忽擁數萬眾,欲問道登、萊,竄處內地。萬一敵中間諜闌入其間,何以備之?」疏下兵部巡撫議,遂寢。

未幾,元珍卒官。初,元珍罷歸,以講學為事。表節義,恤鰥寡,行義重於時。

時雍,汶上人。萬曆二十年進士。知丹徒縣,歷戶、兵二部主事。既除名,未及起用而卒。

葉茂才,字參之,無錫人。萬曆十七年進士。除刑部主事,以便養改南京工部。榷稅蕪湖,課登,輒縱民舟去。既而課羨,請以餉邊卒,不取一錢。就改吏部,進郎中,三遷南京大理丞。復引疾。四十年,起南京太僕少卿。時朝士方植黨爭權。祭酒湯賓尹、修撰韓敬既敗,其黨猶力庇之。御史湯世濟者,敬邑人也,疏陳時政,陰詆發敬奸弊者。茂才馳疏駁之。其黨給事中官應震輩遂連疏力爭。茂才更具揭發其隱,因移疾乞休。世濟益恚,偕同年金汝諧、牟志夔攻之不已。茂才再疏折之,竟自引去。當是時,黨人悉踞言路,凡他曹有言,必合力逐之。茂才既去,黨人益專,無復操異議者。天啟初,召為太僕少卿,改太常,皆不赴。四年,擢南京工部右侍郎。明年抵官。甫三月,以時政日非,謝病歸。友人高攀龍被逮,赴水死,使者將逮其子,茂才力救免之。未幾卒。

茂才恬淡寡嗜好。通籍四十年,家食強半。始同邑顧憲成、允成、安希范、劉元珍及攀龍並建言去國,直聲震一時,茂才只以醇德稱。及官太僕,清流盡斥,邪議益棼,遂奮身與抗,人由是服其勇。時稱「東林八君子」,憲成、允成、攀龍、希范、元珍、武進錢一本、薛敷教及茂才也。

贊曰:成、弘以上,學術醇而士習正,其時講學未盛也。正、嘉之際,王守仁聚徒於軍旅之中,徐階講學於端揆之日,流風所被,傾動朝野。於是搢紳之士,遺佚之老,聯講會,立書院,相望於遠近。而名高速謗,氣盛招尤,物議橫生,黨禍繼作,乃至眾射之的,咸指東林。甘陵之部,洛、蜀之爭,不烈於是矣。憲成諸人,清節姱修,為士林標準。雖未嘗激揚標榜,列「君宗」、「顧」、「俊」之目,而負物望者引以為重,獵時譽者資以梯榮,附麗游揚,薰蕕猥雜,豈講學初心實然哉?語曰「為善無近名」,士君子亦可以知所處矣。


\end{pinyinscope}