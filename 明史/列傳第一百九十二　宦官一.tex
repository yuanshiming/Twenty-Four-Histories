\article{列傳第一百九十二 宦官一}

\begin{pinyinscope}
明太祖既定江左,鑒前代之失,置宦者不及百人。迨末年頒《祖訓》,乃定為十有二監及各司局,稍稱備員矣。然定制不得兼外臣文武銜,不得御外臣冠服,官無過四品,月米一石,衣食於內庭。嘗鐫鐵牌置宮門曰:「內臣不得干預政事,預者斬。」敕諸司不得與文移往來。有老閹供事久,一日從容語及政事,帝大怒,即日斥還鄉。嘗用杜安道為御用監。安道,外臣也,以鑷工侍帝數十年,帷幄計議皆與知,性縝密不泄,過諸大臣前一揖不啟口而退。太祖愛之,然亡他寵異,後遷出為光祿寺卿。有趙成者,洪武八年以內侍使河州市馬。其後以市馬出者,又有司禮監慶童等,然皆不敢有所干竊。建文帝嗣位,御內臣益嚴,詔出外稍不法,許有司械聞。及燕師逼江北,內臣多逃入其軍,漏朝廷虛實。文皇以為忠於己,而狗兒輩復以軍功得幸,即位後遂多所委任。永樂元年,內官監李興奉敕往勞暹羅國王。三年,遣太監鄭和帥舟師下西洋。八年,都督譚青營有內官王安等。又命馬靖鎮甘肅,馬騏鎮交阯。十八年置東廠,令刺事。蓋明世宦官出使、專征、監軍、分鎮、刺臣民隱事諸大權,皆自永樂間始。

初,太祖制,內臣不許讀書識字。後宣宗設內書堂,選小內侍,令大學士陳山教習之,遂為定制。用是多通文墨,曉古今,逞其智巧,逢君作奸。數傳之後,勢成積重,始於王振,卒於魏忠賢。考其禍敗,其去漢、唐何遠哉。雖間有賢者,如懷恩、李芳、陳矩輩,然利一而害百也。今摭其有關成敗者,作《宦官傳》。

○鄭和侯顯金英興安範弘等王振曹吉祥劉永誠懷恩覃吉汪直梁芳錢能等何鼎鄧原等李廣蔣琮劉瑾張永谷大用魏彬等

鄭和,雲南人,世所謂三保太監者也。初事燕王於籓邸,從起兵有功。累擢太監。成祖疑惠帝亡海外,欲蹤跡之,且欲耀兵異域,示中國富強。永樂三年六月,命和及其儕王景弘等通使西洋。將士卒二萬七千八百餘人,多齎金幣。造大舶,修四十四丈、廣十八丈者六十二。自蘇州劉家河泛海至福建,復自福建五虎門揚帆,首達占城,以次遍歷諸番國,宣天子詔,因給賜其君長,不服則以武懾之。五年九月,和等還,諸國使者隨和朝見。和獻所俘舊港酋長。帝大悅,爵賞有差。舊港者,故三佛齊國也,其酋陳祖義,剽掠商旅。和使使招諭,祖義詐降,而潛謀邀劫。和大敗其眾,擒祖義,獻俘,戮於都市。

六年九月,再往錫蘭山。國王亞烈苦柰兒誘和至國中,索金幣,發兵劫和舟。和覘賊大眾既出,國內虛,率所統二千餘人,出不意攻破其城,生擒亞烈苦柰兒及其妻子官屬。劫和舟者聞之,還自救,官軍復大破之。九年六月獻俘於朝。帝赦不誅,釋歸國。是時,交阯已破滅,郡縣其地,諸邦益震讋,來者日多。

十年十一月,復命和等往使,至蘇門答剌。其前偽王子蘇幹剌者,方謀弒主自立,怒和賜不及己,率兵邀擊官軍。和力戰,追擒之喃渤利,並俘其妻子,以十三年七月還朝。帝大喜,賚諸將士有差。

十四年冬,滿剌加、古里等十九國咸遣使朝貢,辭還。復命和等偕往,賜其君長。十七年七月還。十九年春復往,明年八月還。二十二年正月,舊港酋長施濟孫請襲宣慰使職,和齎敕印往賜之。比還,而成祖已晏駕。洪熙元年二月,仁宗命和以下番諸軍守備南京。南京設守備,自和始也。宣德五年六月,帝以踐阼歲久,而諸番國遠者猶未朝貢,於是和、景弘復奉命歷忽魯謨斯等十七國而還。

和經事三朝,先後七奉使,所歷占城、爪哇、真臘、舊港、暹羅、古里、滿剌加、渤泥、蘇門答剌、阿魯、柯枝、大葛蘭、小葛蘭、西洋瑣里、瑣里、加異勒、阿撥把丹、南巫里、甘把里、錫蘭山、喃渤利、彭亨、急蘭丹、忽魯謨斯、比剌、溜山、孫剌、木骨都束、麻林、剌撒、祖法兒、沙里灣泥、竹步、榜葛剌、天方、黎伐、那孤兒,凡三十餘國。所取無名寶物,不可勝計,而中國耗廢亦不貲。自宣德以還,遠方時有至者,要不如永樂時,而和亦老且死。自和後,凡將命海表者,莫不盛稱和以夸外番,故俗傳三保太監下西洋,為明初盛事云。

當成祖時,銳意通四夷,奉使多用中貴。西洋則和、景弘,西域則李達,迤北則海童,而西番則率使侯顯。

侯顯者,司禮少監。帝聞烏思藏僧尚師哈立麻有道術,善幻化,欲致一見,因通迤西諸番。乃命顯齎書幣往迓,選壯士健馬護行。元年四月奉使,陸行數萬里,至四年十二月始與其僧偕來,詔駙馬都尉沐昕迎之。帝延見奉天殿,寵賚優渥,儀仗鞍馬什器多以金銀為之,道路烜赫。五年二月建普度大齋於靈谷寺,為高帝、高后薦福。或言卿雲、天花、甘露、甘雨、青鳥、青獅、白象、白鶴及舍利祥光,連日畢見,又聞梵唄天樂自空而下。帝益大喜,廷臣表賀,學士胡廣等咸獻《聖孝瑞應歌》詩。乃封哈立麻萬行具足十方最勝圓覺妙智慧善普應祐國演教如來大寶法王西天大善自在佛,領天下釋教,給印誥制如諸王,其徒三人亦封灌頂大國師,再宴奉天殿。顯以奉使勞,擢太監。

十一年春復奉命,賜西番尼八剌、地湧塔二國。尼八剌王沙的新葛遣使隨顯入朝,表貢方物。詔封國王,賜誥印。十三年七月,帝欲通榜葛剌諸國,復命顯率舟師以行,其國即東印度之地,去中國絕遠。其王賽佛丁遣使貢麒麟及諸方物。帝大悅,錫予有加。榜葛剌之西,有國曰沼納樸兒者,地居五印度中,古佛國也,侵榜葛剌。賽佛丁告於朝。十八年九月命顯往宣諭,賜金幣,遂罷兵。宣德二年二月復使顯賜諸番,遍歷烏斯藏、必力工瓦、靈藏、思達藏諸國而還。途遇寇劫,督將士力戰,多所斬獲。還朝,錄功升賞者四百六十餘人。

顯有才辨,強力敢任,五使絕域,勞績與鄭和亞。

金英者,宣宗朝司禮太監也,親信用事。宣德七年賜英及范弘免死詔,辭極褒美。英宗立,與興安並貴幸。及王振擅權,英不敢與抗。正統十四年夏旱,命英理刑部、都察院獄囚,築壇大理寺。英張黃蓋中坐,尚書以下左右列坐。自是六年一審錄,制皆如此。其秋,英宗北狩,中外大震。郕王使英、安等召廷臣問計。侍讀徐珵倡議南遷,安叱之,令扶珵出,大言曰:「敢言遷者斬!」遂入告太后,勸郕王任於謙治戰守。或曰叱珵者,英也。

也先入寇,至德勝門,景帝敕安與李永昌同于謙、石亨總理軍務。永昌,亦司禮近侍也。景泰元年十一月,英犯贓罪,下獄論死。帝令禁錮之,終景帝世廢不用,獨任安。也先遣使議和,請迎上皇,廷議報使。帝不懌,令安出,呼群臣曰:「公等欲報使,孰可者?孰為文天祥、富弼!」詞色俱厲。尚書王直面折之,安語塞。及遣都給事中李寔往,敕書不及迎上皇。寔驚,走白內閣,遇安。安復詬曰:「若奉黃紙詔行耳,他何預!」及易儲諸,人遂疑安預謀矣。

安有廉操,且知于謙賢,力護之。或言帝任謙太過,安曰:「為國分憂如于公者,寧有二人!」

英宗復辟,蓋磔景帝所用太監王誠、舒良、張永、王勤等,謂其與黃厷手冓邪議,易太子,且與於謙、王文謀立外籓。於是給事、御史皆言安與誠、良等為黨,宜同罪。帝宥之,但奪職。是時,中官坐誅者甚眾,安僅獲免云。安佞佛,臨歿,遺命舂骨為灰,以供浮屠。

范弘,交阯人,初名安。永樂中,英國公張輔以交童之美秀者還,選為奄,弘及王瑾、阮安、院浪等與焉。占對嫻雅,成祖愛之,教令讀書,涉經史,善筆札,侍仁宗東宮。宣德初,為更名,累遷司禮太監,偕英受免死詔,又偕英及御用太監王瑾同賜銀記。正統時,英宗眷弘,嘗目之曰蓬來吉士。十四年從征,歿於土木,喪歸,葬香山水安寺,弘建也。而王瑾至景泰時始卒。

瑾,初名陳蕪。宣宗為皇太孫時,朝夕給事。及即位,賜姓名。從征漢王高煦還,參預四方兵事,賞賚累巨萬,數賜銀記曰「忠肝義膽」,曰「金貂貴客」,曰「忠誠自勵」,曰「心跡雙清。」又賜以兩宮人,官其養子王椿。其受寵眷,英、弘莫逮也。

阮安有巧思,奉成祖命營北京城池宮殿及百司府廨,目量意營,悉中規制,工部奉行而已。正統時,重建三殿,治楊村河,並有功。景泰中,治張秋河,道卒,囊無十金。

阮浪至景帝時,為御用監少監。英宗居南宮,浪入侍,賜鍍金繡袋及鍍金刀。浪以贈門下皇城使王瑤。錦衣衛指揮盧忠者,險人也,見瑤袋刀異常製,醉瑤而竊之,以告尚衣監高平。平令校尉李善上變,言浪傳上皇命,以袋刀結瑤謀復位。景帝下浪、瑤詔獄,忠證之,浪、瑤皆磔死,詞終不及上皇。英宗復辟,磔忠及平,而贈浪太監。

王振,蔚州人。少選入內書堂。侍英宗東宮,為局郎。初,太祖禁中官預政。自永樂後,漸加委寄,然犯法輒置極典。宣宗時,袁琦令阮巨隊等出外採辦。事覺,琦磔死,巨隊等皆斬。又裴可烈等不法,立誅之。諸中官以是不敢肆。及英宗立,年少。振狡黠得帝懽,遂越金英等數人掌司禮監,導帝用重典御下,防大臣欺蔽。於是大臣下獄者不絕,而振得因以市權。然是時,太皇太后賢,方委政內閣。閣臣楊士奇、楊榮、楊溥,皆累朝元老,振心憚之未敢逞。至正統七年,太皇太后崩,榮已先卒,士奇以子稷論死不出,溥老病,新閣臣馬愉、曹鼐勢輕,振遂跋扈不可制。作大第皇城東,建智化寺,窮極土木。興麓川之師,西南騷動。侍講劉球因雷震上言陳得失,語刺振。振下球獄,使指揮馬順支解之。大理少卿薛瑄、祭酒李時勉素不禮振。振摭他事陷瑄幾死,時勉至荷校國子監門。御史李鐸遇振不跪,謫戍鐵嶺衛。駙馬都尉石璟詈其家閹,振惡賤己同類,下璟獄。怒霸州知州張需禁飭牧馬校卒,逮之,並坐需舉主王鐸。又械戶部尚書劉中敷,侍郎吳璽、陳瑺於長安門。所忤恨,輒加罪謫。內侍張環、顧忠、錦衣衛卒王永心不平,以匿名書暴振罪狀。事發,磔於市,不覆奏。

帝方傾心嚮振,嘗以先生呼之。賜振敕,極褒美。振權日益積重,公侯勛戚呼曰翁父。畏禍者爭附振免死,賕賂輳集。工部郎中王祐以善諂擢本部侍郎,兵部尚書徐晞等多至屈膝。其從子山、林至廕都督指揮。私黨馬順、郭敬、陳官、唐童等並肆行無忌。久之,構釁瓦剌,振遂敗。瓦剌者,元裔也。十四年,其太師也先貢馬,振減其直,使者恚而去。秋七月,也先大舉入寇,振挾帝親征。廷臣交諫,弗聽。至宣府,大風雨,復有諫者,振益虓怒。成國公朱勇等白事,咸膝行進。尚書鄺埜、王佐忤振意,罰跪草中。其黨欽天監正彭德清以天象諫,振終弗從。八月己酉,帝駐大同,振益欲北。鎮守太監郭敬以敵勢告,振始懼。班師,至雙寨,雨甚。振初議道紫荊關,由蔚州邀帝幸其第,既恐蹂鄉稼,復改道宣府。軍士紆回奔走,壬戌始次土木。瓦剌兵追至,師大潰。帝蒙塵,振乃為亂兵所殺。敗報聞,百官慟哭,都御史陳鎰等廷奏振罪,給事中王竑等立擊殺馬順及毛、王二中官。郕王命臠王山於市,並振黨誅之,振族無少長皆斬。振擅權七年,籍其家,得金銀六十餘庫,玉盤百,珊瑚高六七尺者二十餘株,他珍玩無算。先是,郭敬鎮大同,幾造箭鏃數十甕,以振命遺瓦剌,瓦剌輒報以良馬。及帝親征,西寧侯宋瑛、駙馬都尉井源為前鋒,遇敵陽和,敬又撓使敗。至是逃歸,亦坐誅。

英宗復辟,顧念振不置。用太監劉恆言,賜振祭,招魂以葬,祀之智化寺,賜祠曰精忠。而振門下曹吉祥復以奪門功,有寵顓政。

曹吉祥,灤州人。素依王振。正統初,徵麓川,為監軍。征兀良哈,與成國公朱勇、太監劉永誠分道。又與寧陽侯陳懋等征鄧茂七於福建,吉祥每出,輒選達官、跳盪卒隸帳下,師還畜於家,故家多藏甲。

景泰中,分掌京營。後與石亨結,帥兵迎英宗復位。遷司禮太監,總督三大營。嗣子欽,從子鉉、金睿等皆官都督,欽進封昭武伯,門下廝養冒官者多至千百人,朝士亦有依附希進者,權勢與石亨埒,時並稱曹、石。二人惡言官有言,共譖於帝,命吏部尚書王翱察核年三十五以上者留,不及者調用。於是給事何等十三人改州判官,御史吳禎等二十三人改知縣。會有風雷雨雹之變,帝乃悟,悉還其職。未幾,二人爭寵有隙,御史楊瑄、張鵬劾之,吉祥乃復與亨合,乘間醖帝。帝為下瑄等詔獄,而逮治閣臣徐有貞、李賢等。事具賢傳。承天門災,帝命閣臣岳正草罪己詔,詔語激切。吉祥、亨復醖正謗訕,帝又謫正。燄益張,朝野仄目。

久之,帝覺其奸,意稍稍疑。及李賢力言奪門非是,始大悟,疏吉祥。無何,石亨敗,吉祥不自安,漸蓄異謀,日犒諸達官,金錢、穀帛恣所取。諸達官恐吉祥敗而己隨黜退也,皆願盡力效死。欽問客馮益曰:「自古有宦官子弟為天子者乎?」益曰:「君家魏武,其人也。」欽大喜。天順五年七月,欽私掠家人曹福來,為言官所劾。帝令錦衣指揮逮杲按之,降敕遍諭群臣。欽驚曰:「前降敕,遂捕石將軍。今復爾,殆矣。」謀遂決。是時甘、涼告警,帝命懷寧侯孫鏜西征,未發。吉祥使其黨掌欽天監太常少卿湯序擇是月庚子昧爽,欽擁兵入,而已以禁軍應之。謀定,欽召諸達官夜飲。是夜,鏜及恭順侯吳瑾俱宿朝房。達官馬亮恐事敗,逸出,走告瑾。瑾趣鏜由長安右門隙投疏入。帝急縶吉祥於內,而敕皇城及京城九門閉弗啟。欽知亮逸,中夜馳往逮杲家,殺杲,斫傷李賢於東朝房。以杲頭示賢曰:「杲激我也。」又殺都御史寇深於西朝房。攻東、西長安門不得入,縱火。守衛者拆河需磚石塞諸門。賊往來叫呼門外。鏜遣二子急召西征軍擊欽於東長安門。欽走攻東安門,道殺瑾。復縱火,門毀。門內聚薪益之,火熾,賊不得入。天漸曙,欽黨稍稍散去。鏜勒兵逐欽,斬鉉、金睿,鏜子軏斫欽中膊。欽走突安定諸門,門盡閉。奔歸家,拒戰。會大雨如注,鏜督諸軍大呼入,欽投井死。遂殺鐸,盡屠其家。越三日,磔吉祥於市。湯序、馮益及吉祥姻黨皆伏誅。馬亮以告反者,授都督。

英宗始任王振,繼任吉祥,凡兩致禍亂。其他宦者若跛兒乾、亦失哈、喜寧、韋力轉、牛玉之屬,率兇狡。土木之敗,跛兒乾、喜寧皆降敵。跛兒幹助敵反攻,射內使黎定。既又為敵使至京,有所需索,景帝執而誅之。喜寧數為也先畫策,索賞賜,導入邊寇掠。上皇患之,言於也先;使寧還京索禮物,而命校尉袁彬以密書報邊臣。至獨石,參將楊俊擒寧送京師,景泰元年二月磔於市。亦失哈鎮遼東。敵犯廣寧,亦失哈禁官軍勿出擊。百戶施帶兒降敵,為脫脫不花通於亦失哈。正統十四年冬,帶兒逃歸,巡按御史劉孜並劾亦失哈及他不法事。景帝命誅帶兒,而置亦失哈不問。韋力轉者,性淫毒,鎮守大同,多過惡。銜軍妻不與宿,杖死其軍。又與養子妻淫戲,射殺養子。天順元年,工部侍郎霍瑄發力轉僭用金器若王者,及強娶所部女為妾諸不法事。帝怒,執之下錦衣衛獄,既而宥之。牛玉事,詳《吳廢后傳》。

其與吉祥分道征兀良哈者劉永誠,永樂時,嘗為偏將,累從北征。宣德、正統中,再擊兀良哈。後監鎮甘、涼,戰沙漠,有功。景泰末,掌團營。英宗復辟,勒兵從,官其嗣子聚。成化中,永誠始卒。

懷恩,高密人,兵部侍郎戴綸族弟也。宣宗殺綸,並籍恩父太僕卿希文家。恩方幼,被宮為小黃門,賜名懷恩。憲宗朝,掌司禮監。時汪直理西廠,梁芳、韋興等用事。恩班在前,性忠鯁無所撓,諸閹咸敬憚之。員外郎林俊論芳及僧繼曉下獄,帝欲誅之,恩固爭。帝怒,投以硯曰:「若助俊訕我。」恩免冠伏地號哭。帝叱之出。恩遣人告鎮撫司曰:「汝曹諂芳傾俊。俊死,汝曹何以生!」徑歸,稱疾不起。帝怒解,遣醫視恩,卒釋俊。會星變,罷諸傳奉官。御馬監王敏請留馬房傳奉者,帝許之。敏謁恩,恩大罵曰:「星變,專為我曹壞國政故。今甫欲正之,又為汝壞,天雷擊汝矣!」敏愧恨,遂死。進寶石者章瑾求為錦衣衛鎮撫,恩不可,曰:「鎮撫掌詔獄,奈何以賄進。」當是時,尚書王恕以直諫名,恩每嘆曰:「天下忠義,斯人而已。」憲宗末,惑萬貴妃言,欲易太子,恩固爭。帝不懌,斥居鳳陽。孝宗立,召歸,仍掌司禮監,力勸帝逐萬安,用王恕。一時正人匯進,恩之力也。卒,賜祠額曰顯忠。

同時有覃吉者,不知所由進,以老閹侍太子。太子年九歲,吉口授《四書》章句及古今政典。憲宗賜太子莊田,吉勸毋受,曰:「天下皆太子有也。」太子偶從內侍讀佛經,吉入,太子驚曰:「老伴來矣。」亟手《孝經》。吉跪曰:「太子誦佛書乎?」曰:「無有。《孝經》耳。」吉頓首曰:「甚善。佛書誕,不可信也。」弘治之世,政治醇美,君德清明,端本正始,吉有力焉。

汪直者,大藤峽瑤種也。初給事萬貴妃於昭德宮,遷御馬監太監。成化十二年,黑眚見宮中,妖人李子龍以符術結太監韋舍私入大內,事發,伏誅。帝心惡之,銳欲知外事。直為人便黠,帝因令易服,將校尉一二人密出伺察,人莫知也,獨都御史王越與結歡。明年設西廠,以直領之,列官校刺事。南京鎮監覃力朋進貢還,以百艘載私鹽,騷擾州縣。武城縣典史詰之,力朋擊典史,折其齒,射殺一人。直廉得以聞,逮治論斬。力朋後得倖免,而帝以此謂直能摘姦,益幸直。直乃任錦衣百戶韋瑛為心腹,屢興大獄。

建寧衛指揮楊曄,故少師榮曾孫也,與父泰為仇家所告,逃入京,匿姊夫董璵所。璵為請瑛,瑛陽諾而馳報直。直即捕曄、璵考訊,三琶之。琶者,錦衣酷刑也。骨節皆寸解,絕而復蘇。曄不勝苦,妄言寄金於其叔父兵部主事士偉所。直不復奏請,捕士偉下獄,並掠其妻孥。獄具,曄死獄中,泰論斬,士偉等皆謫官,郎中武清、樂章,行人張廷綱,參政劉福等皆無故被收案。自諸王府邊鎮及南北河道,所在校尉羅列,民間鬥詈雞狗瑣事,輒置重法,人情大擾。直每出,隨從甚眾,公卿皆避道。兵部尚書項忠不避,迫辱之,權焰出東廠上。

五月,大學士商輅與萬安、劉珝、劉吉奏其狀。帝震怒,命司禮太監懷恩、覃吉、黃高至閣下,厲色傳旨,言:「疏出誰意?」輅口數直罪甚悉,因言:「臣等同心一意,為國除害,無有先後。」珝慷慨泣下。恩遂據實以奏。頃之,傳旨慰勞。翼日,尚書忠及諸大臣疏亦入。帝不得已,罷西廠,使懷恩數直罪而宥之,令歸御馬監,調韋瑛邊衛,散諸旗校還錦衣。中外大悅。然帝眷直不衰。直因言閣疏出司禮監黃賜、陳祖生意,為楊曄報復。帝即斥賜、祖生於南京。御史戴縉者,佞人也,九年秩滿不得遷。窺帝旨,盛稱直功。詔復開西廠,以千戶吳綬為鎮撫,直焰愈熾。未幾,令東廠官校誣奏項忠,且諷言官郭鏜、馮貫等論忠違法事。帝命三法司、錦衣衛會問。眾知出直意,無敢違,竟勒忠為民。而左都御史李賓亦失直旨褫職,大學士輅亦罷去。一時九卿劾罷者,尚書董方、薛遠及侍郎滕昭、程萬里等數十人。以所善王越為兵部尚書兼左都御史,陳鉞為右副都御史巡撫遼東。

十五年秋,詔直巡邊,率飛騎日馳數百里,御史、主事等官迎拜馬首,箠撻守令。各邊都御史畏直,服櫜鞬迎謁,供張百里外。至遼東,陳鉞郊迎蒲伏,廚傳尤盛,左右皆有賄。直大悅。惟河南巡撫秦紘與直抗禮,而密奏直巡邊擾民。帝弗省。兵部侍郎馬文升方撫諭遼東,直至不為禮,又輕鉞,被陷坐戍,由是直威勢傾天下。

直年少喜兵。陳鉞諷直徵伏當加,立邊功自固。直聽之,用撫寧侯朱永總兵,而自監其軍。師還,永封保國公,鉞晉右都御史,直加祿米。又用王越言,詐稱亦思馬因犯邊。詔永同越西討,直為監軍。越封威寧伯,直再加祿米。已,伏當加寇遼東,亦思馬因寇大同,殺掠甚眾。遼東巡按強珍發鉞奸狀,直右鉞謫珍。於是惡直者,指王越、陳鉞為二鉞。小中官阿丑工俳優,一日於帝前為醉者謾罵狀。人言駕至,謾如故。言汪太監至,則避走。曰:「今人但知汪太監也。」又為直狀,操兩鉞趨帝前。旁人問之,曰:「吾將兵,仗此兩鉞耳。」問何鉞,曰:「王越、陳鉞也。」帝聽然而笑,稍稍悟,然廷臣猶未敢攻直也。會東廠尚銘獲賊得厚賞,直忌,且怒銘不告。銘懼,乃廉得其所洩禁中秘語奏之,盡發王越交通不法事,帝始疏直。

十七年秋,命直偕越往宣府禦敵。敵退,直請班師。不許,徙鎮大同,而盡召將吏還,獨留直、越。直既久鎮不得還,寵日衰。給事御史交章奏其苛擾,請仍罷西廠。閣臣萬安亦力言之。而大同巡撫郭鏜復言直與總兵許寧不和,恐誤邊事。帝乃調直南京御馬監,罷西廠不復設。中外欣然。尋又以言官言,降直奉御,而褫逐其黨王越、戴縉、吳綬等。陳鉞已致仕,不問。韋瑛後坐他事誅,人皆快之,然直竟良死。縉由御史不數年至南京工部尚書。越、鉞頗以材進。縉無他能,工側媚而已。

西廠廢,尚銘遂專東廠事。聞京師有富室,輒以事羅織,得重賄乃已。賣官鬻爵,無所不至。帝尋覺之,謫充南京凈軍,籍其家,輦送內府,數日不盡。而陳準代為東廠。準素善懷恩,既代銘,誡諸校尉曰:「有大逆,告我。非是,若勿預也。」都人安之。

梁芳者,憲宗朝內侍也。貪黷諛佞,與韋興比。而諂萬貴妃,日進美珠珍寶悅妃意。其黨錢能、韋眷、王敬等,爭假採辦名,出監大鎮。帝以妃故,不問也。妖人李孜省、僧繼曉皆由芳進,共為姦利。取中旨授官,累數千人,名傳奉官,有白衣躐至太常卿者。陜西巡撫鄭時論芳被黜,陜民哭送之。帝聞頗悔,斥傳奉官十人,繫六人獄,詔自後傳旨授官者俱覆奏,然不罪芳也。刑部員外郎林俊以劾芳及繼曉下獄。久之,帝視內帑,見累朝金七窖俱盡,謂芳及韋興曰:「糜費帑藏,實由汝二人。」興不敢對。芳曰:「建顯靈宮及諸祠廟,為陛下祈萬年福耳。」帝不懌曰:「吾不汝瑕,後之人將與汝計矣」。芳大懼,遂說貴妃勸帝廢太子,而立興王。會泰山累震,占者言應在東朝。帝懼,乃止。孝宗立,謫芳居南京,尋下獄,興亦斥退。正德初,群閹復薦興司香太和山,兼分守湖廣行都司地方。尚書劉大夏、給事中周璽、御史曹來旬諫,不聽。興遂復用,而芳卒廢以死。

錢能,芳黨也。憲宗時,鄭忠鎮貴州,韋朗鎮遼東,能鎮雲南,並恣縱,而能尤橫。貴州巡撫陳宣劾忠,因請盡撤諸鎮監,帝不允。而雲南巡按御史郭陽顧上疏譽能,請留之雲南。舊制,安南貢道出廣西,後請改由雲南,弗許也。能詐言安南捕盜兵入境,請遣指揮使郭景往諭其王,詔從之。能遂令景以玉帶、彩繒、犬馬遺王,紿其貢使改道雲南。邊吏格之不得入,乃去。復遣景與指揮盧安等索寶貨於乾崖、孟密諸土司,至逼淫曩罕弄女孫,許為奏授宣撫。踰三年,事發。詔巡撫都御史王恕廉之,捕景,景赴井死。再遣刑部郎中鍾蕃往按,事皆實。帝宥能,而致其黨九人於法。指揮姜和、李祥不就逮,能復上疏為二人求宥,帝曲從之。巡按御史甄希賢復劾能杖守礦千戶一人死,亦不罪。召歸,安置南京。復夤緣得南京守備。時恕為南京參贊尚書,能心憚恕不敢肆。久之卒。

韋眷、王敬亦芳黨。眷為廣東市舶太監,縱賈人通諸番,聚珍寶甚富。請以廣南均徭戶六十隸市舶。布政使彭韶爭之,詔給其半。眷又誣奏布政使陳選,被逮道卒,自是,人莫敢逆眷者。弘治初,眷因結蔡用妄舉李父貴冒紀太后族,降左少監,撤回京。事詳《紀太后傳》。

王敬好左道,信妖人王臣。使南方,挾臣同行。偽為詔,括書畫、古玩,聚白金十萬餘兩。至蘇州,召諸生使錄妖書,且辱之。諸生大嘩。巡撫王恕以聞。東廠尚銘亦發其事。詔斬臣,而黜敬充孝陵衛凈軍。

何鼎,餘杭人,一名文鼎,性忠直。弘治初,為長隨,上疏請革傳奉官,為儕輩所忌。壽寧侯張鶴齡兄弟出入宮禁,嘗侍內庭宴。帝如廁,鶴齡倚酒戴帝冠,鼎心怒。他日鶴齡復窺御帷,鼎持大瓜欲擊之,奏言:「二張大不敬,無人臣禮。」皇后激帝怒,下鼎錦衣獄。問主使,鼎曰:「有。」問為誰,曰:「孔子、孟子也。」給事中龐泮、御史吳山及尚書周經、主事李昆、進士吳宗周先後論救,帝以后故,俱不納。后竟使太監李廣杖殺鼎。帝追思之,賜祭勒其文於碑。是時,中官多守法,奉詔出鎮者,福建鄧原、浙江麥秀、河南藍忠、宣府劉清,皆謙潔愛民。兵部上其事,賜敕旌勵。又有司禮太監蕭敬者,歷事英宗、憲宗,諳習典故,善鼓琴。帝嘗語劉大夏曰:「蕭敬朕所顧問,然未嘗假以權也。」獨李廣、蔣琮得帝寵任,後二人俱敗,而敬至世宗朝,年九十餘始卒。

李廣,孝宗時太監也。以符籙禱祀蠱帝,因為奸弊,矯旨授傳奉官,如成化間故事,四方爭納賄賂。又擅奪畿內民田,專鹽利巨萬。起大第,引玉泉山水,前後繞之。給事葉紳、御史張縉等交章論劾,帝不問。十一年,廣勸帝建毓秀亭於萬歲山。亭成,幼公主殤,未幾,清寧宮災。日者言廣建亭犯歲忌,太皇太后恚曰:「今日李廣,明日李廣,果然禍及矣。」廣懼自殺。帝疑廣有異書,使使即其家索之,得賂籍以進,多文武大臣名,饋黃白米各千百石。帝驚曰:「廣食幾何,乃受米如許。」左右曰:「隱語耳,黃者金,白者銀也。」帝怒,下法司究治。諸交結廣者,走壽寧侯張鶴齡求解,乃寢勿治。廣初死時,司設監太監為請祠額葬祭,及是以大學士劉健等言,罷給祠額,猶賜祭。

蔣琮,大興人。孝宗時,守備南京。沿江蘆場,舊隸三廠。成化初,江浦縣田多沉於江,而瀕江生沙洲六,民請耕之,以補沉江田額。洲與蘆場近,又瓦屑壩廢地及石城門外湖地,故不隸三廠。太監黃賜為守備時,受奸民獻,俱指為蘆場,盡收其利。民已失業,而歲額租課仍責償之民。孝宗立,縣民相率醖於朝,下南京御史姜綰等覆按。弘治二年,綰等劾琮與民爭利,且用揭帖抗詔旨。琮條辨綰疏,而泛及御史劉愷、方岳等及南京諸司違法事。給事中韓重因星變請斥琮及太監郭鏞等,以弭天怒,未報。而太監陳祖生復奏戶部主事盧錦、給事中方向私種南京後湖田事。後湖者,洪武時置黃冊庫其中,令主事、給事中各一人守之,百司不得至。歲久湖塞,錦、向於湖灘稍種蔬伐葦,給公用,故為祖生所奏。事下南京法司。適郭鏞奉使兩廣,道南京,往觀焉。御史紘等因劾鏞擅遊禁地。鏞怒,歸醖於帝,言府尹楊守隨勘錦、向失出,御史不劾奏,獨繩內臣。帝乃遣太監何穆、大理寺少卿楊謐再勘後湖田,並覆綰、琮訐奏事。

明年,奏上,褫錦職,謫守隨、向以下官有差。又勘琮不當受獻地,私囑勘官,所訐事皆誣,綰等劾琮亦多不實,並宜逮治。詔逮綰等。御史伊宏、給事中陳璚等皆言不宜以一內臣而置御史十人於獄,不聽。綰等鐫級調外,而宥琮不問。時劉吉竊柄,素惡南京御史劾己,故興此獄。尚書王恕、李敏,給事中趙竑,御史張賓先後言琮、綰同罪異罰,失平,亦不納。琮由是益無忌。久之,廣洋衛指揮石文通奏琮僭侈殺人,掘聚寶山傷皇陵氣,及毆殺商人諸罪。琮竟免死,充孝陵凈軍。

劉瑾,興平人。本談氏子,依中官劉姓者以進,冒其姓。孝宗時,坐法當死,得免。已,得侍武宗東宮。武宗即位,掌鐘鼓司,與馬永成、高鳳、羅祥、魏彬、丘聚、谷大用、張永並以舊恩得幸,人號「八虎」,而瑾尤狡狠。嘗慕王振之為人,日進鷹犬、歌舞、角牴之戲,導帝微行。帝大歡樂之,漸信用瑾,進內官監,總督團營。孝宗遺詔罷中官監槍及各城門監局,瑾皆格不行,而勸帝令內臣鎮守者各貢萬金。又奏置皇莊,漸增至三百餘所,畿內大擾。

外廷知八人誘帝游宴,大學士劉健、謝遷、李東陽驟諫,不聽。尚書張昇,給事中陶諧、胡煜、楊一瑛、張襘,御史王渙、趙佑,南京給事御史李光翰、陸崑等,交章論諫,亦不聽。五官監候楊源以星變陳言,帝意頗動。健、遷等復連疏請誅瑾,戶部尚書韓文率諸大臣繼之。帝不得已,使司禮太監陳寬、李榮、王岳至閣,議遣瑾等居南京。三反,健等執不可。尚書許進曰:「過激將有變。」健不從。王岳者,素謇直,與太監范亨、徐智心嫉八人,具以健等語告帝,且言閣臣議是。健等方約文及諸九卿詰朝伏闕面爭,而吏部尚書焦芳馳白瑾。瑾大懼,夜率永成等伏帝前環泣。帝心動,瑾因曰:「害奴等者王岳。岳結閣臣欲制上出入,故先去所忌耳。且鷹犬何損萬幾。若司禮監得人,左班官安敢如是。」帝大怒,立命瑾掌司禮監,永成掌東廠,大用掌西廠,而夜收岳及亨、智充南京凈軍。旦日諸臣入朝,將伏闕,知事已變,於是健、東陽皆求去。帝獨留東陽,而令焦芳入閣,追殺岳、亨於途,箠智折臂。時正德元年十月也。

瑾既得志,遂以事革韓文職,而杖責請留健、遷者給事中呂翀、劉郤及南京給事中戴銑等六人,御史薄彥徽等十五人。守備南京武靖伯趙承慶、府尹陸珩、尚書林瀚,皆以傳翀、郤疏得罪,珩、瀚勒致仕,削承慶半祿。南京副都御史陳壽,御史陳琳、王良臣,主事王守仁,復以救銑等謫杖有差。瑾勢日益張,毛舉官僚細過,散布校尉,遠近偵伺,使人救過不贍。因顓擅威福,悉遣黨閹分鎮各邊。敘大同功,遷擢官校至一千五百六十餘人,又傳旨授錦衣官數百員。《通鑑纂要》成,瑾誣諸翰林纂修官謄寫不謹,皆被譴,而命文華殿書辦官張駿等改謄,超拜官秩。駿由光祿卿擢禮部尚書,他授京卿者數人,裝潢匠役悉授官。創用枷法,給事中吉時,御史王時中,郎中劉繹、張瑋,尚寶卿顧璿,副使姚祥,參議吳廷舉等,並摭小過,枷瀕死,始釋而戍之。其餘枷死者無數。錦衣獄徽纆相屬。惡錦衣僉事牟斌善視獄囚,杖而錮之。府丞周璽、五官監候楊源杖至死。源初以皇變陳言,罪瑾者也。瑾每奏事,必偵帝為戲弄時。帝厭之。亟麾去曰:「吾用若何事,乃溷我!」自此遂專決,不復白。

二年三月,瑾召群臣跪金水橋南,宣示奸黨,大臣則大學士劉健、謝遷,尚書則韓文、楊守隨、張敷華、林瀚,部曹則郎中李夢陽,主事王守仁、王綸、孫磐、黃昭,詞臣則檢討劉瑞,言路則給事中湯禮敬、陳霆、徐昂、陶諧、劉郤、艾洪、呂翀、任惠、李光翰、戴銑、徐蕃、牧相、徐暹、張良弼、葛嵩、趙士賢,御史陳琳、貢安甫、史良佐、曹閔、王弘、任諾、李熙、王蕃、葛浩、陸崑、張鳴鳳、蕭乾元、姚學禮、黃昭道、蔣欽、薄彥徽、潘鏜、王良臣、趙佑、何天衢、徐玨、楊璋、熊卓、朱廷聲、劉玉等,皆海內號忠直者也。又令六科寅入酉出,使不得息,以困苦之。令文臣毋輒予封誥,痛繩文吏。寧王宸濠圖不軌,賂瑾求復護衛,瑾予之,濠反謀遂成。瑾不學,每批答章奏,皆持歸私第,與妹婿禮部司務孫聰、華亭大猾張文冕相參決,辭率鄙冗,焦芳為潤色之,東陽頫首而已。

當是時,瑾權擅天下,威福任情。有罪人溺水死,乃坐御史匡翼之罪。嘗求學士吳儼賄,不得,又聽都御史劉宇讒,怒御史楊南金,乃以大計外吏奏中,落二人職。授播州土司楊斌為四川按察使。令奴婿閭潔督山東學政。公侯勳戚以下,莫敢鈞禮,每私謁,相率跪拜。章奏先具紅揭投瑾,號紅本,然後上通政司,號白本,皆稱劉太監而不名。都察院奏讞誤名瑾,瑾怒詈之,都御史屠滽率屬跪謝乃已。遣使察核邊倉,都御史周南、張鼐、馬中錫、湯全、劉憲,布政以下官孫祿、冒政、方矩、華福、金獻民、劉遜、郭緒、張翼,郎中劉繹、王藎等,並以赦前罪,下獄追補邊粟,憲至瘐死。又察鹽課,杖巡鹽御史王潤,逮前運使甯舉、楊奇等。察內甲字庫,謫尚書王佐以下百七十三人。復創罰米法,嘗忤瑾者,皆擿發輸邊。故尚書雍泰、馬文升、劉大夏、韓文、許進,都御史楊一清、李進、王忠,侍郎張縉,給事中趙士賢,任良弼,御史張津,陳順、喬恕、聶賢、曹來旬等數十人悉破家,死者繫其妻孥。

其年夏,御道有匿名書詆瑾所行事,瑾矯旨召百官跪奉天門下。瑾立門左詰責,日暮收五品以下官盡下獄。明日,大學士李東陽申救,瑾亦微聞此書乃內臣所為,始釋諸臣。而主事何釴、順天推官周臣、進士陸伸已暍死。是日酷暑,太監李榮以冰瓜啖群臣,瑾惡之。太監黃偉憤甚,謂諸臣曰:「書所言皆為國為民事,挺身自承,雖死不失為好男子,奈何枉累他人。」瑾怒,即日勒榮閒住,而逐偉南京。時東廠、西廠緝事人四出,道路惶懼。瑾復立內行廠,尤酷烈,中人以微法,無得全者。又悉逐京師客傭,令寡婦盡嫁,喪不葬者焚之,輦下洶洶幾致亂。都給事中許天錫欲劾瑾,懼弗克,懷疏自縊。

瑾故急賄,凡入覲、出使官皆有厚獻。給事中周鑰勘事歸,以無金自殺。其黨張綵曰:「今天下所餽遺公者,非必皆私財,往往貸京師,而歸則以庫金償。公奈何斂怨貽患。」瑾然之。會御史歐陽雲等十餘人以故事入賂,瑾皆舉發致罪。乃遣給事、御史十四人分道盤察,有司爭厚斂以補帑。所遣人率阿瑾意,專務搏擊,劾尚書顧佐、侶鐘、韓文以下數十人。浙江鹽運使楊奇逋課死,至鬻其女孫。而給事中安奎、潘希曾,御史趙時中、阮吉、張彧、劉子厲,以無重劾下獄。奎、彧枷且死,李東陽疏救,始釋為民。希曾等亦皆杖斥,忤意者謫斥有差。又矯旨籍故都御史錢鉞、禮部侍郎黃景、尚書秦紘家。凡瑾所逮捕,一家犯,鄰里皆坐,或瞰河居者,以河外居民坐之。屢起大獄,冤號遍道路。《孝宗實錄》成,翰林預纂修者當遷秩,瑾惡翰林官素不下己,調侍講吳一鵬等十六人南京六部。

是時,內閣焦芳、劉宇,吏部尚書張彩,兵部尚書曹元,錦衣衛指揮楊玉、石文義,皆為瑾腹心。變更舊制,令天下巡撫入京受敕,輸瑾賂。延綏巡撫劉宇不至,逮下獄。宣府巡撫陸完後至,幾得罪,既賂,乃令試職視事。都指揮以下求遷者,瑾第書片紙曰「某授某官」,兵部即奉行,不敢復奏。邊將失律,賂入,即不問,有反陞擢者。又遣其黨丈邊塞屯地,誅求苛刻。邊軍不堪,焚公廨,守臣諭之始定。給事中高淓丈滄州,所劾治六十一人,至劾其父高銓以媚瑾。又以謝遷故,令餘姚入毋授京官。以占城國使人亞劉謀逆獄,裁江西鄉試額五十名,仍禁授京秩如餘姚,以焦芳惡彭華故也。瑾又自增陜西鄉試額至百名,亦為芳增河南額至九十五名,以優其鄉士。其年,帝大赦,瑾峻刑自如。刑部尚書劉璟無所彈劾,瑾詬之。璟懼,劾其屬王尚賓等三人,乃喜。給事中郗夔核榆林功,懼失瑾意,自縊死。給事中屈銓、祭酒王雲鳳請編瑾行事,著為律令。

五年四月,安化王寘鐇反,檄數瑾罪。瑾始懼,匿其檄,而起都御史楊一清、太監張永為總督,討之。初,與瑾同為八虎者,當瑾專政時,有所請多不應,永成、大用等皆怨瑾。又欲逐永,永以譎免。及永出師還,欲因誅瑾,一清為畫策,永意遂決。瑾好招致術士,有俞日明者,妄言瑾從孫二漢當大貴。兵仗局太監孫和數遺以甲仗,兩廣鎮監潘午、蔡昭又為造弓弩,瑾皆藏於家。永捷疏至,將以八月十五日獻俘,瑾使緩其期。永慮有變,遂先期入,獻俘畢,帝置酒勞永,瑾等皆侍。及夜,瑾退,永出寘鐇檄,因奏瑾不法十七事。帝已被酒,俯首曰:「瑾負我。」永曰:「此不可緩。」永成等亦助之。遂執瑾,繫於菜廠,分遣官校封其內外私第。次日晏朝後,帝出永奏示內閣,降瑾奉御,謫居鳳陽。帝親籍其家,得偽璽一,穿宮牌五百及衣甲、弓弩、哀衣、玉帶諸違禁物。又所常持扇,內藏利匕首二。始大怒曰:「奴果反。」趣付獄。獄具,詔磔於市,梟其首,榜獄詞處決圖示天下。族人、逆黨皆伏誅。張綵獄斃,磔其屍。閣臣焦芳、劉宇、曹元而下,尚書畢亨、朱恩等,共六十餘人,皆降謫。已,廷臣奏瑾所變法,吏部二十四事,戶部三十餘事,兵部十八事,工部十三事,詔悉釐正如舊制。

張永,保定新城人。正德初,總神機營,與瑾為黨。已而惡其所為,瑾亦覺其不附己也,言於帝,將黜之南京。永知之,直趨帝前,訴瑾陷己。帝召瑾與質,方爭辯,永輒奮拳毆瑾。帝令谷大用等置酒為解,由是二人益不合。及寘鐇反,命永及右都御史楊一清往討。帝戎服送之東華門,賜關防、金瓜、鋼斧以行,寵遇甚盛。瑾亦忌之,而帝方永,不能間也。師出,寘鐇已擒,永遂率五百騎撫定餘黨。還次靈州,與一清言,欲奏瑾不法事。一清曰:「彼在上左右,公言能必入乎?不如以計誅之。」因為永畫策,永大喜,語詳一清傳。是時,瑾兄都督同知景祥死,京師籍籍謂瑾將以八月十五日俟百官送葬,因作亂。適永捷疏至,將以是日獻俘,瑾使緩其期,欲俟事成並擒永。或以告永,永先期入獻俘,是夜遂奏誅瑾。

於是英國公張懋、兵部尚書王敞等,奏永輯寧中外,兩建奇勛,遂封永兄富為泰安伯、弟容為安定伯。涿州男子王豸嘗刺龍形及「人王」字於足,永以為妖人,擒之。兵部尚書何鑑乞加永封,下廷臣議。永欲身自封侯,引劉永誠、鄭和故事風廷臣,內閣以非制格之。永意沮,乃辭免恩澤。吏部尚書楊一清言宜聽永讓,以成其賢,事竟已。久之,坐庫官盜庫銀事,閑住。九年,北邊有警,命永督宣府、大同、延綏軍禦之,寇退乃還。

寧王宸濠反,帝南征,永率邊兵二千先行。時王守仁已擒宸濠,檻車北上。永以帝意遮守仁,欲縱宸濠於鄱陽湖,俟帝至與戰。守仁不可,至杭州詣永。永拒不見,守仁叱門者徑入,大呼曰:「我王守仁也,來與公議國家事,何拒我!」永為氣懾。守仁因言江西荼毒已極,王師至,亂將不測。永大悟,乃曰:「群小在側,永來,欲保護聖躬耳,非欲攘功也。」因指江上檻車曰:「此宜歸我。」守仁曰:「我何用此。」即付永,而與永偕還江西。時太監張忠等已從大江至南昌,方窮治逆黨,見永至,大沮。永留數旬,促忠同歸,江西賴以安。忠等屢讒守仁,亦賴永營解獲免。武宗崩,永督九門防變。世宗立,御史蕭淮奏谷大用、丘聚輩蠱惑先帝,黨惡為奸,並及永。詔永閒住。已而淮復劾永在江西不法事,再降永奉御,司香孝陵,然永在江西,實非有不法也。嘉靖八年,大學士楊一清等言,永功大,不可泯,乃起永掌御用監,提督團營。未幾卒。

谷大用者,瑾掌司禮監時提督西廠,分遣官校遠出偵事。江西南康民吳登顯等,五月五日為競渡,誣以擅造龍舟,籍其家,天下皆重足屏息。建鷹房草場於安州,奪民田無數。瑾誅,大用辭西廠。未幾,帝復欲用之,大學士李東陽力諫乃止。六年,劉六、劉七反,命大用總督軍務,偕伏羌伯毛銳、兵部侍郎陸完討之。大用駐臨清,召邊將許泰、郤永、江彬、劉暉等入內地,聽調遣。久之無功,會賊過鎮江狼山,遇颶風舟覆,陸完兵至殲之,遂封大用弟大亮為永清伯。而先是平寘鐇時,其兄大寬已封高平伯矣,義子冒升賞者,不可勝紀。世宗立,以迎立功賜金幣。給事中閻閎極論之,尋降奉御,居南京。已,召守康陵。嘉靖十年籍其家。

魏彬,當瑾時,總三千營。瑾誅,代掌司禮監。其年,敘寧夏功,封弟英鎮安伯,馬永成兄山亦封平涼伯。世宗立,彬不自安,為英辭伯爵。詔改都督同知,世襲錦衣指揮使。給事中楊秉義、徐景嵩、吳嚴皆言彬附和逆瑾,結姻江彬,宜置極典。帝宥不問。已而御史復論之,始令閒住。

張忠,霸州人。正德時御馬太監,與司禮張雄、東廠張銳並侍豹房用事,時號三張,性皆兇悖。忠利大盜張茂財,結為弟,引入豹房,侍帝蹴鞠。而雄至怨其父不愛己致自宮,拒不見。同儕勸之,乃垂簾杖其父,然後相抱泣,其無人理如此。銳以捕妖言功,加祿至一百二十石。每緝事,先令邏卒誘人為奸,乃捕之,得賄則釋,往往以危法中人。三人並交通宸濠,受臧賢、錢寧等賄,以助成其叛。寧王反,忠勸帝親征。其遮王守仁捷,欲縱宸濠鄱陽,待帝自戰,皆忠之謀也。

是時,又有吳經者,尤親暱。帝南征,經先至揚州。嘗夜半燃炬通衢,遍入寡婦、處女家,掠以出,號哭震遠近,許以金贖,貧者多自經。先是,又有劉允者,以正德十年奉敕往迎烏斯藏僧,所齎金寶以百餘萬計。廷臣交章諫,不聽。允至成都,治裝幾餘,費又數十萬,公私匱竭。既至,為番人所襲。允走免,將士死者數百人,盡亡其所齎。及歸,武宗已崩,世宗用御史王鈞等言,張忠、吳經發孝陵衛充軍,張雄、張銳下都察院鞫治、允亦得罪。

世宗習見正德時宦侍之禍,即位後御近侍甚嚴,有罪撻之至死,或陳尸示戒。張佐、鮑忠,麥福、黃錦輩,雖由興邸舊人掌司禮監,督東廠,然皆謹飭不敢大肆。帝又盡撤天下鎮守內臣及典京營倉場者,終四十餘年不復設,故內臣之勢,惟嘉靖朝少殺云。


\end{pinyinscope}