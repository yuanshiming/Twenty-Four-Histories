\article{列傳第一百五十}

\begin{pinyinscope}
傅宗龍汪喬年張國欽等楊文岳傅汝為等孫傳庭

傅宗龍,字仲綸,昆明人。萬曆三十八年進士。除銅梁知縣,調巴縣,行取,入為戶部主事。久之,授御史。

天啟元年,遼陽破,帝下募兵之令,宗龍請行。一月餘,得精卒五千。明年,安邦彥反,圍貴陽,土寇蜂起。請發帑金濟滇將士,開建昌,通由蜀入滇之路,別設偏沅巡撫,罷湖廣退怯總兵薛來允。帝多採納之。又上疏自請討賊,言:「為武定、尋甸患者,東川土酋祿千鐘。為沾益、羅平患者,賊婦設科及其黨李賢輩。攻圍普安,為滇、黔門戶患者,龍文治妻及其黨尹二。困安南,據關索嶺者,沙國珍及羅應魁輩。困烏撒者,安效良。臣皆悉其生平,非臣敵。臣願以四川巡按兼貴州監軍,滅此群醜。」帝大喜,下所司議。會宗龍以疾歸,不果行。

四年正月,貴州巡撫王三善為降賊陳其愚所紿,敗歿。其夏即家起宗龍巡按其地,兼監軍。初,部檄滇撫閔洪學援黔,以不能過盤江而止。宗龍既被命,洪學令參政謝存仁、參將袁善及土官普名聲、沙如玉等以兵五千送之。宗龍直渡盤江,戰且行,寇悉破。乃謝遣存仁、善,以名聲等土兵七百人入貴陽,擒斬其愚,軍民大快。宗龍盡知黔中要害及土酋逆順,將士勇怯。巡撫蔡復一倚信之,請敕宗龍專理軍務,設中軍旗鼓,裨將以下聽賞罰,可之。宗龍乃條上方略,又備陳黔中艱苦,請大發餉金,亦報可。初,三善令監軍道臣節制諸將,文武不和,進退牽制。宗龍反其所為,令監軍給芻糧,核功罪,不得專進止。由是諸將用命,連破賊汪家沖、蔣義寨,直抵織金。

五年正月,總理魯欽敗績於陸廣河。宗龍上言:「不合滇、蜀,則黔不能平賊;不專總督任,則不能合滇、蜀兵。請召還朱燮元,以復一兼督四川,開府遵義,而移蜀撫駐永寧,滇撫駐霑益,黔撫駐陸廣,沅撫駐偏橋,四面並進,發餉二百萬金給之。更設黔、蜀巡撫。」帝以復一新敗,令解官,即以燮元代,而命尹同皋撫蜀,王瑊撫黔,沅撫閔夢得移鎮,一如宗龍議。

陸廣敗後,諸苗復蠢動。復一、宗龍謀,討破烏粟、螺螄、長田諸叛苗,大破平越賊,毀其砦百七十,賊黨漸孤。宗龍乃條上屯守策,言:

蜀以屯為守,黔則當以守為屯。蓋安酋土地半在水外,仡佬、龍仲、蔡苗諸雜種,緩急與相助。賊有外籓,我無邊蔽,黔兵所以分力愈詘。臣謂以守為屯者,先發兵據河,奪賊所恃。然後撫剿諸種,隨渡口大小,置大小寨,深溝高壘,置烽墩炮臺。小渡則塞以木石,使一粟不入水內,一賊不出水外,賊無如我何。又令沿河兵習水戰,當賊耕耨時,頻出奇兵,渡河擾之。賊不敢附河而居,而後我可以議屯也。

屯之策有二:一曰清衛所原田,一曰割逆賊故壤,而以衛所之法行之。蓋黔不患無田,患無人。客兵聚散無常,不能久駐,莫若仿祖制,盡舉屯田以授有功,因功大小,為官高下,自指揮至總、小旗,畀以應得田為世業,而禁其私賣買。不待招徠,戶口自實。臣所謂以守為屯者如此。然兵當用四萬八千人,餉當歲八十餘萬,時當閱三年,如此而後賊可盡滅也。

部議從之。

復一卒,王瑊代,事悉倚辦。宗龍乃漸剪水外逆黨,將大興屯田。邦彥懼,謀沮之,六年三月,大舉渡河入寇。宗龍擊破邦彥趙官屯,斬老蟲添,威名大著。當是時,大帥新亡,全黔震動,燮元遠在蜀,瑊擁虛位,非宗龍,黔幾殆。詔加太僕少卿。憂歸。

崇禎三年起故官。用孫承宗薦,擢右僉都御史,巡撫順天。未幾,拜兵部右侍郎兼僉都御史,總督薊、遼、保定軍務。

用小故奪官矣。居久之,十年十月流寇大入蜀,陷蜀三十餘州縣,帝拊髀而思宗龍曰:「使宗龍撫蜀,賊安至是哉!」趣即家起宗龍。宗龍至蜀,代王維章與總兵羅尚文禦卻賊。十二年五月,以楊嗣昌薦,召為兵部尚書,去蜀。宗龍自定黔亂後,凡十有四年,輒起用,用不久輒遷去。八月至京,入見帝。宗龍為人伉直任氣,不能從諛承意。帝憤中樞失職,嗣昌以權詭得主知。宗龍樸忠,初入見,即言民窮財盡。帝頗然之,顧豤言不已,遂怫然曰:「卿當整理兵事爾。」既退,語嗣昌曰:「何哉?宗龍善策黔,而所言卑卑,皆他人唾餘,何也?」自是所奏請,多中格。

熊文粲既罷,宗龍乃言:「向者賊流突東西,嗣昌故建分剿之策。今則流突者各止其所,臣請收勢險節短之效。總理止轄楚、豫,秦督兼轄四川,鳳督兼轄安慶,各率所轄撫鎮,期十二月成功。」因薦湖廣巡撫方孔炤堪代文燦。帝不用,用嗣昌督師。

嗣昌既督師,上章請兵食,不悉應,劾中樞不任。宗龍亦劾嗣昌徒耗敝國家,不能報效,以氣凌廷臣。會薊遼總督洪承疇請用劉肇基為團練總兵官,中官高起潛又揭肇基恇怯,宗龍不即覆。帝遂發怒,責以抗旨,令對狀。奏上,復以戲視封疆下吏。法司擬戍邊,不許,欲置之死。在獄二年矣,十四年春,嗣昌死,尚書陳新甲薦其才,帝未有以應也,良久曰:「樸忠,吾以夙負用之,宜盡死力。」遂釋之出獄,以兵部右侍郎兼右僉都御史代丁啟睿,總督陜西三邊軍務。

當是之時,李自成有眾五十萬,自陷河、洛,犯開封,羅汝才復自南陽趨鄧、淅,與合兵。帝命宗龍專辦自成。議盡括關中兵餉以出,然屬郡旱蝗,已不能應。

九月四日,以川、陜兵二萬出關,次新蔡,與保督楊文岳兵會。賀人龍、李國奇將秦兵,虎大威將保兵,共結浮橋,東渡汝,合兵趨項城。五日,兩軍畢渡,走龍口。自成、汝才亦結浮橋於上流,將趨汝寧。覘兩督兵至,盡伏精銳於林中,陽驅諸賊自浮橋西渡。人龍使後騎覘賊,還報曰:「賊向汝矣,結浮橋將渡矣。」宗龍、文岳夜會諸將於龍口,詰朝將戰。

六日,兩軍並進,中道一騎馳而告曰:「賊畢渡矣。」復進,一騎馳而告曰:「賊半渡矣,三分渡其二矣。」宗龍、文岳曰:「驅之。」走三十里,至於孟家莊,日卓午。人龍、大威曰:「馬力乏矣,詰朝而戰,止兵為營。」諸軍弛馬甲,植戈錞,散行墟落求芻牧。賊覘之,塵起於林中,伏甲並出搏我兵。人龍有馬千騎不戰,國奇以麾下兵迎擊之,不勝。秦兵、保兵俱潰,人龍、大威奔沈丘,國奇從之,三帥師潰。宗龍、文岳合兵屯火燒店,賊以步兵攻其營。諸軍鳴大炮,震死賊百餘。日暮,賊引去。宗龍軍西北,文岳軍東南,畫塹而守。保兵宵潰,保督副將挾文岳騎而馳,夜奔於項城。宗龍復分秦兵立營於東南,諸將分壁當賊壘。

九日,檄人龍、國奇還兵救,二帥不應。宗龍曰:「彼避死,宜不來,吾豈避死哉!」語其麾下曰:「宗龍老矣,今日陷賊中,當與諸軍決一死戰,不能效他人捲甲走也。」召裨校李本實,即文岳壁穿塹築壘以拒賊。賊亦穿壕二重以圍之。

十一日,秦師食盡,宗龍殺馬騾以享軍。明日,營中馬騾盡,殺賊取其屍分啖之。十八日,營中火藥、鉛子、矢並盡。宗龍簡士卒,夷傷死喪之餘,有眾六千。夜半,潛勒諸軍突賊營,殺千餘人,潰圍出。諸軍星散,宗龍徒步率諸軍且戰且走。十九日,日卓午,未至項城八里,賊追及之,執宗龍,呼於門曰:「秦督圍隨官丁也,請啟門納秦督。」宗龍大呼曰:「我秦督也,不幸墮賊手,左右皆賊耳。」賊唾宗龍。宗龍罵賊曰:「我大臣也,殺則殺耳,豈能為賊賺城以緩死哉!」賊抽刀擊宗龍,中其腦而仆,CU其耳鼻死城下。事聞,帝曰:「若此,可謂樸忠矣。」復官兵部尚書,加太子少保,謚忠壯,廕子錦衣世百戶,予祭葬。

人龍、國奇兵潰歸陜,賊遂屠項城。分兵屠商水、扶溝,遂攻葉縣。

汪喬年,字歲星,遂安人。天啟二年進士。授刑部主事,歷郎中。母憂歸。

崇禎二年起工部,遷青州知府。以治行卓異,遷登萊兵備副使,乞終養歸。父喪除,起官平陽,遷陜西右參政,提督學校。再以卓異,就遷按察使。喬年清若自勵,惡衣菲食,之官,攜二僕,不以家自隨。為青州,行廊置土銼十餘,訟者自炊候鞫,吏無敢索一錢。自負才武,休沐輒馳騎,習弓刀擊刺,寢處風露中。

十四年,擢右僉都御史,巡撫陜西。時李自成已破河南,聲言入關。喬年疾驅至商、洛,不見賊。賊圍開封,而三邊總督傅宗龍亦至陜,議抽兵括餉,則關中兵食已盡,無以應。宗龍、喬年握手欷歔而別。未幾,宗龍敗歿於項城,喬年流涕歎曰:「傅公死,討賊無人矣。」已,又聞詔擢喬年兵部右侍郎,總督三邊軍務,代宗龍。部檄踵至,趣出關。是時,關中精銳盡沒於項城。喬年曰:「兵疲餉乏,當方張之寇。我出,如以肉喂虎耳。然不可不一出,以持中原心。」乃收散亡,調邊卒,得馬步三萬人。

十五年正月,率總兵賀人龍、鄭嘉棟、牛成虎出潼關。先是,臨潁為賊守,左良玉破而屠之,盡獲賊所擄。自成聞之怒,舍開封而攻良玉,良玉退保郾城,賊圍之急。喬年諸將議曰:「郾城危在旦夕。吾趨郾,賊方銳,難與爭鋒。吾聞襄城距郾四舍,賊老砦咸在。吾舍郾而以精銳攻其必應,賊必還兵救,則郾城解矣。郾城解,我擊其前,良玉乘其背,賊可大破也。」諸將皆曰:「善。」乃留步兵火器於洛陽,簡精騎萬人兼程進。次郟縣,襄城人張永祺等迎喬年。

二月二日,喬年入襄城,分人龍、嘉棟、成虎軍三路,駐城東四十里,逼郾城而軍,而自勒兵駐城外。賊果解郾城而救襄城。賊至,三帥奔,良玉救不至,軍大潰。喬年歎曰:「此吾死所也。」率步卒千餘入城守。賊穴地實火藥攻城,喬年亦穿阱,視所鑿,長矛刺之。賊炮擊喬年坐纛,雉堞盡碎,左右環泣請避之,喬年怒,以足蹴其首曰:「汝畏死,我不畏死也。」十七日,城陷,巷戰,殺三賊,自剄不殊,為賊所執,大罵。賊割其舌,磔殺之。襄城人建祠而祀之。

時張國欽、張一貫、黨威、李萬慶及監紀西安同知孫兆祿、材官李可從、襄城知縣曹思正從喬年,皆死之。萬慶者,降將射塌天也。又有馬帥某者,逸其名。兆祿,鹽山人。可從,盩厔人。黨威,神木人。餘莫考。黨威則嘗擊賊於西雒峪,擒賊首竇阿婆者也。

自成購永祺不得,屠其族,劓刖諸生劉漢臣等百九十人。自成數月之間再敗秦師,獲馬二萬,降秦兵又數萬,威震河雒。

初,喬年之撫陜西也,奉詔發自成先塚。米脂令邊大受,河間靜海舉人,健令也,詗得其族人為縣吏者,掠之。言:「去縣二百里曰李氏村,亂山中,十六塚環而葬,中其始祖也。相傳,穴,仙人所定,壙中鐵燈檠,鐵燈不滅,李氏興。」如其言發之螻蟻數石,火光熒熒然。CU棺,骨青黑,被體黃毛,腦後穴大如錢,赤蛇盤,三四寸,角而飛,高丈許,咋咋吞日光者六七,反而伏。喬年函其顱骨、臘蛇以聞,焚其餘,雜以穢,棄之。自成聞之,嚙齒大恨曰:「吾必致死於喬年。」既殺喬年,由西華攻陳州。

楊文岳,字斗望,南充人。萬曆四十七年進士。授行人。天啟五年,擢兵科給事中,屢遷禮科都給事中。

崇禎二年,出為江西右參政,歷湖廣、廣西按察使,雲南、山西左右布政使,以右副都御史巡撫登、萊。十二年擢兵部右侍郎,總督保定、山東、河北軍務,代孫傳庭。

十四年正月,李自成陷洛陽,犯開封,文岳率總兵虎大威以眾二萬赴救。渡河,賊先遁,追擊於鳴皋。還,駐兵開封。疫作,乃頓兵於汝寧,出屯西平、新蔡間。七月,自成走內鄉、淅川,與羅汝才合。文岳趨鄧州,自成還攻之。文岳戰三捷,斬其魁一條龍、一隻龍,賊遁去。

九月,會陜西總督傅宗龍於新蔡,與賊遇,大潰於孟家莊,再潰於火燒店。部將挾文岳夜入於項城。明日奔陳州,宗龍遂覆沒。事聞,文岳革職,充為事官,戴罪自贖。乃收集散亡,率所部就巡撫高名衡防巳。賊遂破葉縣,拔泌陽,乘勝陷南陽,殺唐王,下鄧州等十四城,再圍開封。

明年正月,文岳馳救開封,論功復官。臨潁為賊守,左良玉破而屠之,退保郾城。自成圍郾城。二月,督師丁啟睿及文岳、大威救郾城。賊潰,距官軍數里而營。文岳、啟睿相掎角,持十一晝夜。總督汪喬年出關,賊引去,再攻開封。六月,詔起侯恂兵部右侍郎,總督保定、山東、河南、湖北軍務,代文岳。命所司察文岳罪狀。七月朔,文岳、啟睿合良玉、大威及楊德政、方國安四總兵之師,次朱仙鎮。諸軍盡潰,啟睿、文岳奔汝寧。賊渡河,追奔四百里,官軍失亡數萬。詔褫官候勘。

九月,文岳在汝寧,夜襲賊營有功。賊既灌開封,旋敗孫傳庭兵,以閏十一月悉眾薄汝寧,老回回、革裏眼、左金王等畢會。文岳遣都司康世德以輕騎偵賊,世德走還汝,將其步騎五百,夜縱火噪而奔。十三日,群賊並至,壓汝寧五里而軍。監軍僉事孔貞會以川兵屯城東,文岳以保兵屯城西。賊兵進攻,相持一晝夜。川兵潰,殺傷數百。賊奪其馬騾,悉眾攻保兵,漸不支。僉事王世琮、知府傅汝為、通判朱國寶縋將士入城,副將賈悌、參將馮名聖亦掖文岳、貞會登城。

明日,賊四面環攻,戴扉以陣,矢石雲梯堵牆而立。城頭矢炮擂石雨集,賊死傷山積,而攻不休。一鼓百道並登,執文岳及世琮、國寶、悌、名聖於城頭,殺汝陽知縣文師頤於城上。汝為聞變,赴水死。賊擁文岳等見自成,大罵,賊怒,縛之城南三里鋪,以大炮擊之,洞胸糜骨而死。士民屠戮數萬,焚公私廨舍殆盡。貞會執去,不知所終。自成以文岳死忠,備禮斂之。遂拔營走確山、信陽、泌陽,嚮襄陽,虜崇王由樻、崇世子、諸王妃及河南懷安諸王以行。

汝為,字於宣,江陵人。崇禎七年進士。世琮,字仲發,達州人。國寶,成都人。師頤,全州人。皆舉人。世琮嘗為汝寧推官,討土寇,流矢貫耳不為動,時號王鐵耳者也。師頤蒞任甫三日。

孫傳庭,字百雅,代州振武衛人。自父以上,四世舉於鄉。傳庭儀表頎碩,沈毅多籌略。萬曆四十七年成進士,授永城知縣,以才調商丘。天啟初,擢吏部驗封主事,屢遷稽勳郎中,請告歸。家居久不出。

崇禎八年秋,始遷驗封郎中,超遷順天府丞。陜西巡撫甘學闊不能討賊,秦之士大夫嘩於朝,乃推邊才用傳庭,以九年三月受代。傳庭蒞秦,嚴徵發期會,一從軍興法。秦人愛之不如總督洪承疇,然其才自足辦賊。賊首整齊王據商、雒,諸將不敢攻,檄副將羅尚文擊斬之。

當是時,賊亂關中,有名字者以十數,高迎祥最強,拓養坤黨最眾,所謂闖王、蠍子塊者也。傳庭設方略,親擊迎祥於盩厔之黑水峪,擒之,及其偽領哨黃龍、總管劉哲,獻俘闕下。錄功,增秩一等。而賊黨自是乃共推李自成為闖王矣。明年,養坤及其黨張耀文來降。已而養坤叛去,諭其下追斬之。擊賊惠登相於涇陽、三原,登相西走。河南賊馬進忠、劉國能等十七部入渭南,追之出關,復合河南兵夾擊之,先後斬首千餘級。進忠等復擾商、雒、藍田,叛卒與之合,將犯西安。遣左光先、曹變蛟追走之渭南,降其渠一條龍,招還脅從。募健兒擊餘賊,斬聖世王、瓦背、一翅飛,降鎮天王、上山虎,又殲白捍賊渠魁數人。關南稍靖。遣副將盛略等敗賊大天王於寶雞,賊走入山谷,傳庭追之鳳翔。他賊出棧道,謀越關犯河南,還軍擊,賊走伏斜谷,復大敗之,降其餘眾。西安四衛,舊有屯軍二萬四千,田二萬餘頃,其後田歸豪右,軍盡虛籍。傳庭釐得軍萬一千有奇,歲收屯課銀十四萬五千餘兩,米麥萬三千五百餘石。帝大喜,增秩,賚銀幣。

會楊嗣昌入為本兵,條上方略。洪承疇以秦督兼剿務,而用廣撫熊文燦為總理。分四正六隅,馬三步七,計兵十二萬,加派至二百八十萬,期百日平賊。傳庭移書爭之,曰:「無益,且非特此也。部卒屢經潰蹶,民力竭矣,恐不堪命。必欲行之,賊不必盡,而害中於國家。」累數千言,嗣昌大忤。部議,秦撫當一正面,募土著萬人,給餉銀二十三萬,以商、雒等處為汛守。傳庭知其不可用也,乃核帑藏,蠲贖鍰,得銀四萬八千,市馬募兵,自辦滅賊具,不用部議。會諸撫報募兵及額,傳庭疏獨不至。嗣昌言軍法不行於秦,自請白衣領職,以激帝怒。傳庭奏曰:「使臣如他撫,籍郡縣民兵上之,遂謂及額,則臣先所報屯兵已及額矣。況更有募練馬步軍,數且踰萬,何嘗不遵部議。至百日之期,商、雒之汛守,臣皆不敢委。然使賊入商、雒,而臣不能禦,則治臣罪。若臣扼商、雒,而踰期不能滅賊,誤剿事者必非臣。」嗣昌無以難,然銜之彌甚。傳庭兩奉詔進秩,當加部銜,嗣昌抑弗奏。十一年春,賊破漢陰、石泉,則坐傳庭失援,削其所加秩。

傳庭出扼商、雒。大天王等犯慶陽、寶雞,還軍戰合水,破走之,獲其二子,追擊之延安。過天星、混天星等從徽、秦趨鳳翔,逼澄城。傳庭分兵五道擊之楊家嶺、黃龍山,大破之,斬首二千餘級。大天王知二子不殺,遂降。賊引而北,犯延安。傳庭策鄜州西、合水東三四百里,荒山邃谷,賊入當自斃,乃率標兵中部遏其東,檄變蛟、慶陽拒其西,伏兵三水、淳化間。賊饑,出掠食,則大張旗幟,鳴鼓角以邀之,一日夜馳二百五十里。賊大驚,西奔,至職田莊,遇伏而敗;復走寶雞,取棧道,再中伏大敗;折而走隴州關山道,又為伏兵所挫。三敗,賊死者無算,過天星、混天星並降。又逐賊邠、寧間,陷陣,獲其渠。河南賊馬進忠、馬光玉驅宛、洛之眾,箕張而西。傳庭擊之,賊還走。又設伏於潼關原,變蛟逐賊入伏。而闖王李自成者,為洪承疇所逐,盡亡其卒,以十八騎潰圍遁。關中群盜悉平,是為崇禎之十一年春也。捷聞,大喜,先敘澄城之捷,命加傳庭部銜。嗣昌仍格不奏。

當是時,總理熊文燦主撫。湖廣賊張獻忠已降,惟河南賊如故。羅汝才、馬進忠、賀一龍、左金王等十三部西窺潼關,聯營數十里。傳庭計曰:「天下大寇盡在此矣。我出擊其西,總理擊其東,賊不降則滅。此賊平,天下無賊矣。獻忠即狙伏,無能為也。」乃遂引兵東,大敗賊閿鄉、靈寶山間,貫其營而東,復自東以西。賊窘甚,以文燦招降手諭上,言旦夕且降。傳庭曰:「爾曹日就熊公言撫,而日攻堡屠寨不已,是偽也。降即解甲來,有說即非真降,吾明日進兵矣。」明日擐甲而出,得文燦檄於途中曰:「毋妒吾撫功。」又進,得本兵嗣昌手書,亦云。傳庭怏怏撤兵還。然賊迄不就撫,移瞰商、雒。文燦悔,期傳庭夾擊。屬吏王文清等三戰三敗之,賊奔內鄉、淅川而去。傳庭既屢建大功,其將校數奉旨優敘,嗣昌務抑之不為奏。傳庭懇請上其籍於部,嗣昌曰:「需之。」

十月,京師戒嚴,召傳庭及承疇入衛,擢兵部右侍郎兼右僉都御史,代總督盧象昇督諸鎮援軍,賜劍。當是時,傳庭提兵抵近郊,與嗣昌不協,又與中官高起潛忤,降旨切責,不得朝京師。承疇至,郊勞,且命陛見,傳庭不能無觖望。無何,嗣昌用承疇以為薊督,欲盡留秦兵之入援者守薊、遼。傳庭曰:「秦軍不可留也。留則賊勢張,無益於邊,是代賊撤兵也。秦軍妻子俱在秦,兵日殺賊以為利,久留於邊,非嘩則逃,不復為吾用,必為賊用,是驅民使從賊也。安危之機,不可不察也。」嗣昌不聽。傳庭爭之不能得,不勝鬱鬱,耳遂聾。

傳庭初受命,疏言:「年來疆事決裂,由計畫差謬。事竣,當面請決大計。」明年,帝移傳庭總督保定、山東、河南軍務。既解嚴,疏請陛見。嗣昌大驚,謂傳庭將傾之,斥來役齎疏還之傳庭。傳庭慍,引疾乞休。嗣昌又劾其託疾,非真聾,帝遂發怒,斥為民,下巡撫楊一俊核真偽,一俊奏言:「真聾,非託疾。」並下一俊獄。傳庭長系待決,舉朝知其冤,莫為言。在獄三年,文燦、嗣昌相繼敗。而是時,闖王李自成者,已攻破河南矣,犯開封,執宗龍,殺唐王,兵散而賊益橫。帝思傳庭言,朝士薦者益眾。

十五年正月,起傳庭兵部右侍郎,親御文華殿問剿賊安民之策,傳庭侃侃言。帝嗟歎久之,燕勞賞賚甚渥,命將禁旅援開封。開封圍已解,賊殺陜督汪喬年,帝即命傳庭往代。大集諸將於關中,縛援剿總兵賀人龍,坐之麾下,數而斬之。謂其開縣噪歸,猛帥以孤軍失利而獻、曹出柙也;又謂其遇敵先潰,新蔡、襄城連喪二督也。諸將莫不灑然動色者。

傳庭既已誅殺人龍,威讋三邊,日夜治軍為平賊計,而賊遂已再圍開封。詔御史蘇京監延、寧、甘、固軍,趣傳庭出關。傳庭上言:「兵新募,不堪用。」帝不聽。傳庭不得已出師,以九月抵潼關。大雨連旬,自成決馬家口河灌開封。開封已陷,傳庭趨南陽,自成西行逆秦師。傳庭設三覆以待賊:牛成虎將前軍,左勷將左,鄭嘉棟將右,高傑將中軍。成虎陽北以誘賊,賊奔入伏中,成虎還兵而鬥,高傑、董學禮突起翼之,左勷、鄭嘉棟左右橫擊之。賊潰東走,斬首千餘。追三十里,及之郟縣之塚頭,賊棄甲仗軍資於道,秦兵趨利。賊覘我軍囂,反兵乘之,左勷、蕭慎鼎之師潰,諸軍皆潰。副將孫枝秀躍馬以追賊,擊殺數十騎,賊兵圍之,馳突不得出,馬蹶被執,植立不撓。以刃臨之,瞠目不答。一人曰:「此孫副將也。」遂殺之。參將黑尚仁亦被執不屈而見殺,覆軍數千,材官小將之歿者,張渼奎、李棲鳳、任光裕、戴友仁以下七十有八人。賊倍獲其所喪馬。傳庭走鞏,由孟入關,執斬慎鼎;罰勷馬以二千,以勷父光先故,貸勷。是役也,天大雨,糧不至,士卒採青柿以食,凍且餒,故大敗。豫人所謂「柿園之役」也。

傳庭既已敗歸陜西,計守潼關,扼京師上游。且我軍新集,不利速戰,乃益募勇士,開屯田,繕器,積粟,三家出壯丁一。火車載火炮甲仗者三萬輛,戰則驅之拒馬,止則環以自衛。督工苛急,夜以繼日,秦民不能堪。而關中頻歲饑,駐大軍餉乏,士大夫厭苦傳庭所為,用法嚴,不樂其在秦。相與嘩於朝曰:「秦督玩寇矣。」又相與危語恫脅之曰:「秦督不出關,收者至矣。」明年五月,命兼督河南、四川軍務,尋進兵部尚書,改稱督師,加督山西、湖廣、貴州及江南、北軍務,賜劍。趣戰益急。傳庭頓足歎曰:「奈何乎!吾固知往而不返也。然大丈夫豈能再對獄吏乎!」頃之,不得已遂再議出師。總兵牛成虎將前鋒,高傑將中軍,王定、官撫民將延、寧兵為後勁,白廣恩統火車營,檄左良玉赴汝寧夾擊。當是時,自成已據有河南、湖北十餘郡,自號新順王,設官置戍,營襄陽而居之。將由內、淅窺商、雒,盡發荊、襄兵會於氾水、滎澤,伐竹結筏,人佩三葫蘆,將謀渡河。傳庭分兵防禦。八月十日,傳庭出師潼關,次於閿鄉。二十一日,師次陜州,檄河南諸軍渡河進剿。九月八日,師次汝州,偽都尉四天王李養純降。養純言賊虛實:諸賊老營在唐縣,偽將吏屯寶豐,自成精銳盡聚於襄城。遂破賊寶豐,斬偽州牧陳可新等。遂搗唐縣,破之,殺家口殆盡,賊滿營哭。轉戰至郟縣,遂擒偽果毅將軍謝君友,斫賊坐纛,尾自成幾獲。賊奔襄城,大軍遂進逼襄城。賊懼謀降,自成曰:「無畏!我殺王焚陵,罪大矣,姑決一死戰。不勝,則殺我而降未晚也。」而大軍時皆露宿與賊持,久雨道濘,糧車不能前。士饑,攻郟破之,獲馬騾啖之立盡。雨七日夜不止,後軍嘩於汝州。賊大至,流言四起。不得已還軍迎糧,留陳永福為後拒。前軍既移,後軍亂,永福斬之不能止。賊追及之南陽,官軍還戰。賊陣五重,饑民處外,次步卒,次馬軍,又次驍騎,老營家口處內。戰破其三重。賊驍騎殊死鬥,我師陣稍動,廣恩軍將火車者呼曰:「師敗矣!」脫挽輅而奔,車傾塞道,馬掛於衡不得前,賊之鐵騎凌而騰之,步賊手白棓遮擊,中者首兜鍪俱碎。自成空壁躡我,一日夜,官兵狂奔四百里,至於孟津,死者四萬餘,失亡兵器輜重數十萬。傳庭單騎渡垣曲,由閿鄉濟。賊獲督師坐纛,乘勝破潼關,大敗官軍。傳庭與監軍副使喬遷高躍馬大呼而歿於陣,廣恩降賊。傳庭屍竟不可得。傳庭死,關以內無堅城矣。

初,傳庭之出師也,自分必死,顧語繼妻張夫人曰:「爾若何?」夫人曰:「丈夫報國耳,毋憂我。」及西安破,張率二女三妾沉於井,揮其八歲兒世寧亟避賊去之。兒逾牆墮民舍中,一老翁收養之。長子世瑞聞之,重趼入秦,得夫人尸井中,面如生。翁歸其弟世寧,相扶攜還。道路見者,知與不知皆泣下。傳庭死時,年五十有一矣。傳庭再出師皆以雨敗也。或言傳庭未死者,帝疑之,故不予贈廕。傳庭死而明亡矣。

贊曰:流賊蔓延中原,所恃以禦賊者獨秦兵耳。傅宗龍、孫傳庭遠近相望,倚以辦賊。汪喬年、楊文嶽奮力以當賊鋒,而終於潰僨。此殆有天焉,非其才之不任也。傳庭敗死,賊遂入關,勢以愈熾。存亡之際,所系豈不重哉!


\end{pinyinscope}