\article{恭閔帝本紀}

\begin{pinyinscope}
恭閔惠皇帝諱允炆。太祖孫,懿文太子第二子也。母妃呂氏。帝生潁慧好學,性至孝。年十四,待懿文太子疾,晝夜不暫離。更二年,太子薨,居喪毀瘠。太祖撫之曰:「而誠純孝,顧不念我乎。」洪武二十五年九月,立為皇太孫。二十九年,重定諸王見東宮儀制,朝見後於內殿行家人禮,以諸王皆尊屬也。初,太祖命太子省決章奏,太子性仁厚,於刑獄多所減省。至是以命太孫,太孫亦復佐以寬大。嘗請於太祖,遍考禮經,參之歷朝刑法,改定洪武《律》畸重者七十三條,天下莫不頌德焉。

三十一年閏五月,太祖崩。辛卯,即皇帝位。太赦天下,以明年為建文元年。是日,葬高皇帝於孝陵。詔行三年喪。群臣請以日易月。帝曰:「朕非效古人亮陰不言也。朝則麻冕裳,退則齊衰杖絰,食則饘粥,郊社宗廟如常禮。」遂命定儀以進。丙申,詔文臣五品以上及州縣官各舉所知,非其人者坐之。六月,省并州縣,革冗員。兵部侍郎齊泰為本部尚書,翰林院修撰黃子澄為太常卿,同參軍國事。

秋七月,召漢中府教授方孝孺為翰林院侍講。詔行寬政,赦有罪,蠲逋賦。八月,周王橚有罪,廢為庶人,徙雲南。詔興州、營州、開平諸衛軍全家在伍者,免一人。天下衛所軍單丁者,放為民。九月,雲南總兵官西平侯沐春卒於軍,左副將何福代領其眾。

冬十一月,工部侍郎張昺為北平布政使,謝貴、張信掌北平都指揮使司,察燕陰事。詔求直言,舉山林才德之士。十二月癸卯,何福破斬刀幹孟,麓川平。是月,賜天下明年田租之半,釋黥軍及囚徒還鄉里。

是年,暹羅、占城入貢。

建文元年春正月癸酉,受朝,不舉樂。庚辰,大祀天地於南郊,奉太祖配。修《太祖實錄》。二月,追尊皇考曰孝康皇帝,廟號興宗,妣常氏曰孝康皇后。尊母妃呂氏曰皇太后,冊妃馬氏為皇后。封弟允熥為吳王,允熞衡王,允熙徐王。立皇長子文奎為皇太子。詔告天下,舉遺賢。賜民高年米肉絮帛,鰥寡孤獨廢疾者官為牧養。重農桑,興學校,考察官吏,振罹災貧民,旌節孝,瘞暴骨,蠲荒田租。衛所軍戶絕都除勿勾。詔諸王毋得節制文武吏士,更定內外大小官制。三月,釋奠於先師孔子。罷天下諸司不急務。都督宋忠、徐凱、耿王瓛帥兵屯開平、臨清、山海關。調北平、永清二衛軍於彰德、順德。侍郎暴昭、夏原吉等二十四人充採訪使,分巡天下。甲午,京師地震,求直言。

夏四月,湘王柏自焚死。齊王榑、代王桂有罪,廢為庶人。遣燕王世子高熾及其弟高煦、高燧還北平。六月,岷王楩有罪,廢為庶人,徙漳州。己酉,燕山護衛百主戶倪諒上變,燕旗校於諒等伏誅。詔讓燕王棣,逮王府官僚。北平都指揮張信叛附於燕。

秋七月癸酉,燕王棣舉兵反,殺布政使張昺、都司謝貴。長史葛誠、指揮盧振、教授余逢辰死之。參政郭資、副使墨麟、僉事呂震等降於燕。指揮馬宣走薊州,僉瑱走居庸。宋忠趨北平,聞變退保懷來。通州、遵化、密雲相繼降燕。丙子,燕兵陷薊州,馬宣戰死。己卯,燕兵陷居庸關。甲申,陷懷來,宋忠、俞瑱被執死,都指揮彭聚、孫泰力戰死,永平指揮使郭亮等叛降燕。壬辰,谷王橞自宣府奔京師。長興侯耿炳文為征虜大將軍,駙馬都尉李堅、都督甯忠為左、右副將軍,帥師討燕。祭告天地宗廟社稷,削燕屬籍。詔曰:「邦家不造,骨肉周親屢謀僭逆。去年,周庶人橚僭為不軌,辭連燕、齊、湘三王。朕以親親故,止正橚罪。今年齊王榑謀逆,又與棣、柏同謀,柏伏罪自焚死,榑已廢為庶人。朕以棣於親最近,未忍窮治其事。今乃稱兵構亂,圖危宗社,獲罪天地祖宗,義不容赦。是用簡發大兵,往致厥罰。咨爾中外臣民軍士,各懷忠守義,與國同心,掃茲逆氛,永安至治。」尋命安陸侯吳傑,江陰侯吳高,都督耿瓛,都指揮盛庸、潘忠、楊松、顧成、徐凱、李友、陳暉、平安,分道並進。置平燕布政使司於真定,尚書暴昭掌司事。

八月己酉,耿炳文兵次真定,徐凱屯河間,潘忠、楊松屯鄚州。壬子,燕兵陷雄縣,潘忠、楊松戰於月漾橋,被執。鄚州陷。壬戌,耿炳文及燕兵戰於滹沱河北,敗績,李堅、甯忠、顧成被執,炳文退保真定。燕兵攻之不克,引去。召遼王植、寧王權歸京師,權不至,詔削護衛。丁卯,曹國公李景隆為征虜大將軍,代耿炳文。九月戊辰,吳高、耿瓛、楊文帥遼東兵,圍永平。戊寅,景隆兵次河間,燕兵援永平,吳高退保山海關。

冬十月,燕兵自劉家口間道襲陷大寧,守將朱鑑死之。總兵官劉真、都督陳亨援大寧,亨叛降燕。燕以寧王權及朵顏三衛卒歸北平。辛亥,李景隆重圍北平,燕兵還救。十一月辛未,李景隆及燕兵戰於鄭村壩,敗績,奔德州,諸軍盡潰。燕王棣再上書於朝。帝為罷齊泰、黃子澄官,仍留京師。

二年春正月丙寅朔,詔天下來朝官勿賀。丁卯,釋奠於先師孔子。二月,燕兵陷蔚州,進攻大同。李景隆自德州赴援,燕兵還北平。保定知府雒僉叛降燕。甲子,復以都察院為御史府。均江、浙田賦。詔曰:「國家有惟正之供,江、浙賦獨重,而蘇、松官田悉準私稅,用懲一時,豈可為定則。今悉與減免,畝毋踰一斗。蘇、松人仍得官戶部。」三月丙寅朔,日有食之。賜胡廣等進士及第、出身有差。

夏四月己未,李景隆及燕兵戰於白溝河,敗之。明日復戰,敗績,都督瞿能、越巂侯俞淵、指揮滕聚等皆戰死,景隆奔德州。五月辛未,奔濟南。燕兵陷德州,遂攻濟南。庚辰,景隆敗績於城下,南走。參政鐵鉉、都督盛庸悉力禦之。六月己酉,遣尚寶丞李得成諭燕罷兵。

秋八月癸巳,承天門災,詔求直言。戊申,盛庸、鐵鉉擊敗燕兵,濟南圍解,復德州。九月,詔錄洪武中功臣罪廢者後。辛未,封盛庸歷城侯,擢鐵鉉山東布政使,參贊軍務,尋進兵部尚書。以庸為平燕將軍,都督陳暉、平安副之。庸屯德州,平安及吳傑屯定州,徐凱屯滄州。

冬十月,召李景隆還,赦不誅。庚申,燕兵襲滄州,徐凱被執。十二月甲午,燕兵犯濟寧,薄東昌。乙卯,盛庸擊敗之。斬其將張玉。丙辰,復戰,又敗之,燕兵走館陶。庸軍勢大振,檄諸屯軍合擊燕,絕其歸路。

三年春正月辛酉朔,凝命神寶成,告天地宗廟,御奉天殿受朝賀。乙丑,吳傑、平安邀擊燕兵於深州,不利。辛未,大祀天地於南郊。丁丑,享太廟,告東昌捷。復齊泰、黃子澄官。三月辛巳,盛庸敗燕兵於夾河,斬其將譚淵。再戰不利,都指揮莊得、楚智等力戰死。壬午,復戰,敗績,庸走德州。丁亥,都督何福援德州。癸巳,貶齊泰、黃子澄、諭燕罷兵。閏月己亥,吳傑、平安及燕戰於槁城,敗績,還保真定。燕兵掠真定、順德廣平、大名。棣上書讀召還諸將息兵,遣大理少卿薛巖報之。是月,《禮制》成,頒行天下。

夏五月甲寅,盛庸以兵扼燕餉道,不克。棣復遣使上書,下其使於獄。六月壬申,燕將李遠寇沛縣,焚糧艘。壬午,都督袁宇邀擊之,敗績。

秋七月己丑,燕兵掠彰德。丁酉,平安自真定攻北平。壬寅,大同守將房昭帥兵由紫荊關趨保定,駐易州西水寨。九月甲辰,平安及燕將劉江戰於北平,敗績,還保真定。

冬十月丁巳,真定諸將遣兵援房昭,及燕王戰於齊眉山,敗績。十一月壬辰,遼東總兵官楊文攻永平,及劉江戰於昌黎,敗績。己亥,平安敗燕將李彬於楊村。十二月癸亥,燕兵焚真定軍儲。詔中官奉使侵暴吏民者,所在有司繫治。是月,駙馬都尉梅殷鎮淮安。《太祖實錄》成。

四年春正月甲申,召故周王橚於蒙化,居之京師。燕兵連陷東阿、東平、汶上、兗州、濟陽,東平吏目鄭華,濟陽教諭王省皆死之。甲申,魏國公徐輝祖帥師援山東。燕兵陷沛縣,知縣顏伯瑋、主簿唐子清、典史黃謙死之。癸丑,薄徐州。二月甲寅,都督何福及陳暉、平安軍濟寧,盛庸軍淮上。己卯,更定品官勳階。三月,燕兵攻宿州,平安追及於淝河,斬其將王真,遇伏敗績,宿州陷。

夏四月丁卯,何福、平安敗燕兵於小河,斬其將陳文。甲戌,徐輝祖等敗燕兵於齊眉山,斬其將李斌,燕兵懼,謀北歸。會帝聞訛言,謂燕兵已北,召輝祖還,何福軍亦孤。庚辰,諸將及燕兵大戰於靈璧,敗績,陳暉、平安、禮部侍郎陳性善、大理寺卿彭與明皆被執。五月癸未,楊文帥遼東兵赴濟南,潰於直沾。己丑,盛庸軍潰於淮上,燕兵渡淮,趨揚州。指揮王禮等叛降燕,御史王彬、指揮崇剛死之。辛丑,燕兵至六合,諸軍迎戰,敗績。壬寅,詔天下勤王,遣御史大夫練子寧、侍郎黃觀、修撰王叔英分道徵兵。召齊泰、黃子澄還。蘇州知府姚善、寧波知府王璡、徽州知府陳彥回、樂平知縣張彥方各起兵入衛。甲辰,遣慶成郡主如燕師,議割地罷兵。

六月癸丑,盛庸帥舟師敗燕兵於浦子口,復戰不利。都督僉事陳瑄以舟師叛附於燕。乙卯,燕兵渡江,盛庸戰於高資港,敗績。戊午,鎮江守將童俊叛降燕。庚申,燕兵至龍潭。辛酉,命諸王分守都城,遣李景隆及兵部尚書茹瑺、都督王佐如燕軍,申前約。壬戌,復遣谷王橞、安王楹往。皆不聽。甲子,遣使齊蠟書四出,促勤王兵。乙丑,燕兵犯金川門,左都督徐增壽謀內應,伏誅。谷王橞及李景隆叛,納燕兵,都城陷。宮中火起,帝不知所終。燕王遣中使出帝后屍於火中,越八日壬申葬之。

或云帝由地道出亡。正統五年。有僧自雲南至廣西,詭稱建文皇帝。恩恩知府岑瑛聞於朝。按問,乃鈞州人楊行祥,年已九十餘,下獄,閱四月死。同謀僧十二人,皆戍遼東。自後滇、黔、巴、蜀間,相傳有帝為僧時往來跡。正德、萬曆、崇禎間,諸臣請續封帝後,及加廟謚,皆下部議,不果行。大清乾隆元年,詔廷臣集議,追謚曰恭閔惠皇帝。

贊曰:惠帝天資仁厚。踐阼之初,親賢好學,召用方孝孺等。典章制度,銳意復古。嘗因病晏朝,尹昌隆進諫,即深自引咎,宣其疏於中外。又除軍衛單丁,減蘇、松重賦,皆惠民之大者。乃革命而後,紀年復稱洪武,嗣是子孫臣庶以紀載為嫌,草野傳疑,不無訛謬。更越聖朝,得經論定,尊名壹惠,君德用彰,懿哉。


\end{pinyinscope}