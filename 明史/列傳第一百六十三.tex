\article{列傳第一百六十三}

\begin{pinyinscope}
張慎言子履旋徐石麒解學龍高倬黃端伯等左懋第祁彪佳

張慎言,字金銘,陽城人。祖昇,河南參政。慎言舉萬歷三十八年進士。除壽張知縣,有能聲。調繁曹縣,出庫銀糴粟備振,連值荒歲,民賴以濟。泰昌時,擢御史。踰月,熹宗即位。時方會議三案,慎言言:「皇祖召諭百工,不究張差黨與,所以全父子之情;然必摘發奸謀,所以明君臣之義。至先皇踐阼,蠱惑之計方行,藥餌之奸旋發。崔文昇投涼劑於積憊之餘,李可灼進紅丸於大漸之際,法當駢首,恩反賜金。誰秉國成,一至此極!若夫鼎湖再泣,宗廟之鼎鬯為重,則先帝之簪履為輕。雖神廟鄭妃且先徙以為望,選侍不即移宮,計將安待。」無何,賈繼春以請安選侍被譴,慎言抗疏救之。帝怒,奪俸二年。

天啟初,出督畿輔屯田,言:「天津、靜海、興濟間,沃野萬頃,可墾為田。近同知盧觀象墾田三千餘畝,其溝洫廬舍之制,種植疏浚之方,犁然具備,可仿而行。」因列上官種、佃種、民種、軍種、屯種五法。又言:「廣寧失守,遼人轉徙入關者不下百萬。宜招集津門,以無家之眾,墾不耕之田便。」詔從之。嘗疏薦趙南星,劾馮銓,銓大恨。五年三月,慎言假歸,銓屬曹欽程論劾,誣盜曹縣庫銀三千,遂下撫按徵臟,編戍肅州。

莊烈帝即位,赦免。崇禎元年起故官。會當京察,請先治媚璫者附逆之罪,其他始付考功,報可。旋擢太僕少卿,歷太常卿、刑部右侍郎。讞耿如杞獄,不稱旨,并尚書韓繼思下吏,尋落職歸。久之,召為工部右侍郎。國用不支,廷議開採、鼓鑄、屯田、鹽法諸事。慎言屢疏陳奏,悉根本計。大學士楊嗣昌議改府州縣佐為練備、練總,慎言以更制事大,歷陳八議,其後卒不能行。由左侍郎遷南京戶部尚書,七疏引疾,不允。就改吏部尚書,掌右都御史事。

十七年三月,京師陷。五月,福王即位南京,命慎言理部事。上中興十議:曰節鎮,曰親籓,曰開屯,曰叛逆,曰偽命,曰褒恤,曰功賞,曰起廢,曰懲貪,曰漕稅。皆嘉納。時大起廢籍,慎言薦吳甡、鄭三俊。命甡陛見,三俊不許,大學士高弘圖所擬也。勳臣劉孔昭,趙之龍等一日朝罷,群詬於廷,指慎言及甡為奸邪,叱吒徹殿陛。給事中羅萬象言:「慎言平生具在,甡素有清望,安得指為奸邪?」孔昭等伏地痛哭,謂慎言舉用文臣,不及武臣,囂爭不已。又疏劾慎言,極詆三俊。且謂::「慎言當迎立時,阻難懷二心。乞寢牲陛見命,且議慎言欺蔽罪。」慎言疏辨,因乞休。萬象又言:「首膺封爵者,四鎮也。新改京營,又加二鎮銜,何嘗不用武。年來封疆之法,先帝多寬武臣,武臣報先帝者安在?祖制以票擬歸閣臣,參駁歸言官,不聞委勳臣以糾劾也。使勛臣得兼糾劾,文臣可勝逐哉!」史可法奏:「慎言疏薦無不當。諸臣痛哭喧呼,滅絕法紀,恐驕弁悍卒益輕朝廷。」御史王孫蕃言:「用人,吏部職掌。奈何廷辱塚宰。」弘圖等亦以不能戢和文武,各疏乞休,不允。

甡既不出,慎言乞休得請,加太子太保,廕一子。山西盡陷於賊,慎言無家可歸,流寓蕪湖、宣城間。國亡後,疽發於背,戒勿藥,卒,年六十九。

慎言少喪二親,鞠於祖母。及為御史,訃聞,引義乞歸,執喪三年以報。

子履旋,舉崇禎十五年鄉試。賊陷陽城,投崖死。事聞,贈御史。

徐石麒,字寶摩,嘉興人。天啟二年進士。授工部營繕主事,筦節慎庫。魏忠賢兼領惜薪司,所需悉從庫發,石麒輒持故事格之。其黨噪於庭,不為動。御史黃尊素坐忤忠賢下詔獄,石麒為盡力。忠賢怒,執新城侯王升子下獄,令誣賄石麒,捕繫其家人,勒完臟而削其籍。

崇禎三年,起南京禮部主事,就遷考功郎中。八年佐尚書鄭三俊京察,澄汰至公。歷尚寶卿、應天府丞。十一年春入賀。三俊時為刑部尚書,議侯恂獄不中,得罪。石麒疏救,釋之。石麒官南京十餘年,至是始入為左通政,累遷光祿卿、通政使。十五年擢刑部右侍郎,讞吏部尚書李日宣等獄。帝曰:「枚卜大典,日宣稱詡徇私。」石麒予輕比,貶二秩。先是,會推閣臣,日宣一再推,因及副都御史房可壯、工部右侍郎宋玫、大理寺卿張三謨,石麒與焉。召對便殿,石麒獨不赴。及是帝怒,戍日宣及吏科都給事中章正宸、河南道御史張煊,奪可壯、玫、三謨及讞獄左侍郎惠世揚官。石麒代世揚掌部事,旋進左。

當是時,帝以威刑馭下,法官引律,大抵深文附會,予重比。石麒奉命清獄,推明律意,校正今斷獄之不合於律者十餘章,先以白同官。以次審理十三司囚,多寬減。然廉公,一時大法赫然,無敢倖免者。兵部尚書陳新甲下獄,朝士多營救。石麒持之曰:「人臣無境外交。未有身在朝廷,不告君父而專擅便宜者。新甲私款辱國,當失陷城寨律,斬。」帝曰:「未中,可覆擬。」乃論新甲陷邊城四,陷腹城七十二,陷親籓七,從來未有之奇禍。當臨敵缺乏,不依期進兵策應,因而失誤軍機者斬。奏上,新甲棄市,新甲黨皆大恨。

石麒尋擢本部尚書。中官王裕民坐劉元斌黨,元斌縱軍淫掠,伏誅,裕民以欺隱不舉下獄。帝欲殺之,初令三法司同鞫,後專付刑部,石麒議戍煙瘴。奏成,署院寺名以進。帝怒其失出,召詰都御史劉宗周,對曰:「此獄非臣讞。」徐曰:「臣雖不與聞,然閱讞同,已曲盡情事。刑官所執者法耳。法如是止,石麒非私裕民也。」帝曰:「此奴欺罔實甚,卿等焉知?」令石麒改讞詞,棄之市。無何,宗周以救姜埰,熊開元獲嚴譴,僉都御史金光辰救之,奪職。石麒再疏留,不納。、開元既下詔獄,移刑部定罪。石麒據原詞擬開元贖徒,埰謫戍,不復鞫訊。帝責對狀,石麒援故事對。帝大怒,除司官三人名,石麒落職閒住。

福王監國,召拜右都御史,未任,改吏部尚書。奏陳省庶官、慎破格、行久任、重名器、嚴起廢、明保舉、交堂廉七事。時方考選,與都御史劉宗周矢公甄別,以年例出御史黃耳鼎、給事中陸朗於外。朗賄奄入得留用,石麒發其罪。朗恚,詆石麒,石麒稱疾乞休。耳鼎亦兩疏劾石麒,并言其枉殺陳新甲。石麒疏辯,求去益力。馬士英擬嚴旨,福王不許,命馳驛歸。

石麒剛方清介,扼於權奸,悒悒不得志。士英挾定策功,將圖封,石麒議格之。中官田成輩納賄請囑,石麒悉拒不應。由是中外皆怨,構之去。去後以登極恩,加太子太保。

明年,南都亡。石麒時居郡城外,城將破,石麒曰:「吾大臣也,城亡與亡!」復入居城中,以閏月二十六日朝服自縊死,年六十有八。

解學龍,字石帆,揚州興化人。萬歷四十一年進士。歷金華、東昌二府推官。天啟二年,擢刑科給事中。遼東難民多渡海聚登州,招練副使劉國縉請帑金十萬振之,多所乾沒。學龍三疏發其弊,國縉遂獲譴。王紀忤魏忠賢削籍,學龍言:「紀亮節弘猷,召置廊廟,必能表正百僚,裁決大務。」失忠賢意,不報。已,劾川、貴舊總督張我續貪淫漏網,新總督楊述中縮朒卸責,帝不罪。學龍通曉政務。上言:

遼左額兵舊九萬四千有奇,幾餉四十餘萬。今關上兵止十餘萬,月餉乃二十二萬。遼兵盡潰,關門宜募新兵。薊鎮舊有額兵,乃亦給厚糈召募。舊兵以其餉厚,悉竄入新營,而舊額又如故,漏卮可勝言。國初,文職五千四百有奇,武職二萬八千有奇。神祖時,文增至一萬六千餘,武增至八萬二千餘矣。今不知又增幾倍。誠度冗者汰之,歲可得餉數十萬。裁冗吏,核曠卒,俾衛所應襲子弟襲職而不給俸,又可得數十萬。

京邊米一石,民輸則非一石也。以民之費與國之收衷之,國之一,民之三。關餉一斛銀四錢,以易錢則好米值錢百,惡米止三四十錢,又其下腐臭不可食。以國之費與兵之食衷之,兵之一,國之三。總計之,民費其六,而兵食其一。況小民作奸欺漕卒,漕卒欺官司,官司欺天子,展轉相欺,米已化為糠粃沙土;兼濕熱蒸變,食不可咽,是又化有用之六,為無用之一矣。臣以為莫如修屯政,屯政修則地闢而民有樂土,粟積而人有固志。昔吳璘守天水,縱橫鑿渠,綿亙不絕,名曰「地網」,敵騎不能逞。今仿其制,溝涂之界,各樹土所宜木,小可獲薪果之饒,大可得抗扼之利,敵雖強,何施乎。

帝亟下所司,而議竟中格。稍進右給事中。五年九月,御史智鋌劾學龍及編修侯恪為東林鷹犬,遂削籍。

崇禎元年起歷戶科都給事中。以民貧盜起,請大清吏治。尋劾薊撫王應豸剋餉激變,又上足餉十六事。帝皆採納。遷太常少卿、太僕卿。五年改右僉都御史,巡撫江西。疏言:「臣所部州縣七十八,而坐逋賦降罰者至九十人。由數歲之逋責於一歲,數人之逋責於一人,故終無及額之日也。請別新舊,酌多寡,立帶征之法。」可之。四方盜賊蜂起,江西獨無重兵,學龍以為言,詔增置千人。討平都昌、萍鄉諸盜,合閩兵擊破封山妖賊張普薇等,賊遂殄滅。

十二年冬,擢南京兵部右侍郎。明年春,將解任,遵例薦舉屬吏,并及遷謫官黃道周。帝怒,徵下獄,責其黨庇行私,廷杖八十,削其籍,移入詔獄,竟坐遣戍。十五年秋,道周召還,半道請釋學龍,不聽。

十七年五月,福王立於南京,召拜兵部左侍郎。十月擢刑部尚書。時方治從賊之獄,仿唐制六等定罪。學龍議定,以十二月上之:

其一等應磔者:吏部員外郎宋企郊,舉人牛金星,平陽知府張嶙然,太僕少卿曹欽程,御史李振聲、喻上猷,山西提學參議黎志升,陜西左布政使陸之祺,兵科給事中高翔漢,潼關道僉事楊王休,翰林院檢討劉世芳十一人也。

二等應斬秋決者:刑科給事中光時亨,河南提學僉事鞏焴,庶吉士周鍾,兵部主事方允昌四人也。

三等應絞擬贖者:翰林修撰兼戶、兵二科都給事中陳名復,戶科給事中楊枝起、廖國遴,襄陽知府王承曾,天津兵備副使原毓宗,庶吉士何胤光,少詹事項煜七人也。

四等應戍擬贖者:禮部主事王孫蕙,翰林院檢討梁兆陽,大理寺正錢位坤,總督侍郎侯恂,山西副使王秉鑑,御史陳羽白、裴希度、張懋爵,禮部郎中劉大鞏,吏部員外郎郭萬象,給事中申芝芳、金汝礪,舉人吳達,修撰揚廷鑑及黃繼祖十五人也。

五等應徒擬贖者:通政司參議宋學顯,諭德方拱乾,工部主事繆沅,給事中呂兆龍、傅振鐸,進士吳剛思,檢討方以智、傅鼎銓,庶吉士張家玉及沈元龍十人也。

六等應杖擬贖者:工部員外郎潘同春,禮部員外郎吳泰來,主事張琦,行人王于曜,行取知縣周壽明,進士徐家麒及向列星、李㭎八人也。

其留北俟後定奪者:少詹事何瑞徵、楊觀光,太僕少卿張若麒,副使方大猷,戶部侍郎黨崇雅,吏部侍郎熊文舉,太僕卿葉初春,給事中龔鼎孳、戴明說、孫承澤、劉昌,御史塗必泓、張鳴駿,司業薛所蘊,通政參議趙京仕,編修高爾儼,戶部郎中衛周祚及黃紀、孫襄十九人也。

其另存再議者:給事中翁元益、郭充、庶吉士魯慄、吳爾壎、史可程、王自超、白胤謙、梁清標、楊棲鶚、張元琳、呂崇烈、李化麟、朱積、趙熲、劉廷琮,吏部郎中侯佐,員外郎左懋泰,禮部郎中吳之琦,兵部員外郎鄒明魁,行人許作梅,進士胡顯,太常博士龔懋熙及王之牧、王皋、梅鶚、姬琨、朱國壽、吳嵩胤二十八人也。

其已奉旨錄用者:兵部尚書張縉彥,給事中時敏,諭德衛胤文、韓四維,御史蘇京,行取知縣黃國琦、施鳳儀,兵部郎中張正聲,內閣中書舍人顧大成及姜荃林等十人也。

得旨:「周鍾等不當緩決,陳名夏等未蔽厥辜,侯恂、宋學顯、吳剛思、方以智、潘同春等擬罪未合。新榜進士盡污偽命,不當復玷班聯。」令再議。惟方拱乾結納馬、阮,特旨免其罪。

明年正月,學龍奉詔擬周鍾、光時亨等各加一等,潘同春諸臣皆侯補小臣,受偽無據,仍執前律。當是時,馬、阮必欲殺周鍾。學龍欲緩其死,謀之次輔王鐸,乘士英注籍上之,且請停刑。鐸即擬俞旨,褒以詳慎平允。士英聞之大怒,然事已無及。大鋮暨其黨張捷、楊維垣聲言欲劾學龍,學龍引疾。命未下,保國公朱國弼、御史張孫振等詆其曲庇行私,遂削籍。

大鋮既殺鍾、時亨,即傳旨二等罪斬者謫允充南金齒軍,三等罪絞者充廣西邊衛軍,四等以下俱為民,永不敘用。然學龍所定案亦多漏網,而所擬一等諸犯,皆隨賊西行,實未嘗正刑辟也。黃繼祖、沈元龍、向列星、李㭎、黃紀、孫襄、王之牧、王皋、梅鶚、姬琨、朱國壽、吳嵩胤、姜荃林,皆未詳其官。

學龍歸,南都旋失。久之卒於家。

高倬,字枝樓,忠州人。天啟五年進士。除德清知縣,調金華。崇禎四年,徵授御史。薊遼總督曹文衡與總監鄧希詔相訐奏。詔殫力幹濟,以副委任。倬乃上疏言:「文衡亢臟成性,必不能仰鼻息於中官;希詔睚眥未忘,何能化戈矛為同氣。封疆事重,宜撤希詔安文衡心。若文衡不足用,宜更置,勿使中官參之。諸邊鎮臣如希詔不少,使人效希詔,督撫之展布益難。即諸邊督撫如文衡亦不少,使人效文衡,將邊事之廢壞愈甚。」疏入,貶一秩視事。巡視草場,坐失火下吏。廷臣申救,不納。逾年熱審,給事中吳甘來以為言,始釋歸。起上林署丞,稍遷大理右寺副。

十一年五月,火星逆行,詔修省。倬以近者刑獄滋繁,法官務停閣,請敕諸司剋期奏報,大者旬,小者五日。其奉旨覆讞者,或五日三日,務俾積案盡疏,囹圄衰減。帝為採納。屢遷南京太僕卿。太僕故駐滁州,滁為南都西北門戶。請募州人為兵,保障鄉土,從之。十六年二月擢右僉都御史,提督操江。其秋,操江改任武臣劉孔昭,召倬別用,未赴而京師陷。

福王立南京,拜倬工部右侍郎。御用監內官請給工料銀,置龍鳳几榻諸器物及宮殿陳設金玉諸寶,計貲數十萬,倬請裁省。光祿寺辦御用器至萬五千七百有奇,倬又以為言。皆不納。明年二月,由左侍郎拜刑部尚書。國破,倬投繯死。

是時,大臣殉難者:倬與張捷、楊維垣、庶僚則有黃端伯、劉成治、吳嘉胤、龔廷祥。

端伯,字元公,建昌新城人。崇禎元年進士。歷寧波、杭州二府推官。行取赴都,母憂歸。服闋入都,疏陳益王居建昌不法狀。王亦劾端伯離間親籓,及出妻酗酒諸事。有詔侯勘,避居廬山。福王立,大學士姜曰廣薦起之。明年三月授儀制主事。五月,南都破,百官皆迎降。端伯不出,捕繫之。閱四月,諭之降,不從,卒就戮。

成治,字廣如,漢陽人。崇禎七年進士。福王時,歷官戶部郎中。國破,忻城伯趙之龍將出降,入戶部封府庫。成治憤,手搏之,之龍跳而免。成治自經。

嘉胤,字繩如,松江華亭人。由鄉舉歷官戶部主事。奉使出都,聞變,還謁方孝孺祠,投繯死。

廷祥,字伯興,無錫人。馬世奇門人也。崇禎十六年進士。為中書舍人。城破,衣冠步至武定橋投水死。

時又有欽天監博士陳于階、國子生吳可箕、武舉黃金璽、布衣陳士達,並死焉。

左懋第,字蘿石,萊陽人。崇禎四年進士。授韓城知縣,有異政。遭父喪,三年不入內寢,事母盡孝。十二年,擢戶科給事中。疏陳四弊,謂民困、兵弱、臣工委頓、國計虛耗也。又陳貴粟之策,令天下贖罪者盡輸粟,鹽筴復開中之舊,令輸粟邊塞充軍食。彗星見,詔停刑,懋第請馬上速傳。又請嚴禁將士剽掠,有司朘削。請散米錢,振輦下饑民,收養嬰孩。明年正月,剿餉罷征,亦請馬上速行,恐遠方吏不知,先已征,民不沾實惠。帝並採納。

三月,大風霾。帝布袍齋居,禱之不止。懋第言:「去秋星變,朝停刑而夕即滅。今者不然,豈陛下有其文未修其實乎?臣敢以實進。練餉之加,原非得已。乃明旨減兵以省餉,天下共知之,而餉猶未省,何也?請自今因兵征餉,預使天下知應加之數,官吏無所逞其奸,以信陛下之明詔。而刑獄則以睿慮之疑信,定諸囚之死生,諸疑於心與疑信半者,悉從輕典。豈停刑可止彗,解網不可以返風乎?且陛下屢沛大恩,四方死者猶枕藉,盜賊未見衰止,何也?由蠲停者止一二。存留之賦,有司迫考成,催征未敢緩,是以莫救於凶荒。請於極荒州縣,下詔速停,有司息訟,專以救荒為務。」帝曰:「然。」於是上災七十五州縣新、舊、練三餉並停。中災六十八州縣止征練餉,下災二十八州縣秋成督征。

十四年督催漕運,道中馳疏言:「臣自靜海抵臨清,見人民饑死者三,疫死者三,為盜者四。米石銀二十四兩,人死取以食,惟聖明垂念。」又言:「臣自魚臺至南陽,流寇殺戮,村市為墟。其他饑疫死者,屍積水涯,河為不流,振手求安可不速。」已又陳安民息盜之策,請核荒田,察逋戶,予以有生之樂,鼓其耕種之心。又言:「臣有事河乾一載,每進父老問疾苦,皆言練餉之害。三年來,農怨於野,商嘆於途。如此重派,所練何兵?兵在何所?剿賊禦邊,效安在?奈何使眾心瓦解,一至此極乎!」又言:「臣去冬抵宿遷,見督漕臣史可法,言山東米石二十兩,而河南乃至百五十兩,漕儲多逋。朝議不收折色,需本色。今淮、鳳間麥大熟,如收兩地折色,易麥轉輸,豈不大利。昔劉晏有轉易之法。今歲河北大稔,山東東、兗二郡亦有收。誠出內帑二三十萬,分發所司,及時收糴,於國計便。」帝即命議行。屢遷刑科左給事中。

十六年秋,出察江防。明年五月,福王立,進兵科都給事中,旋擢右僉都御史,巡撫應天、徽州諸府。時大清兵連破李自成,朝議遣使通好,而難其人。懋第母陳歿於燕,懋第欲因是返柩葬,請行。乃拜懋第兵部右侍郎兼右僉都御史,與左都督陳弘範、太僕少卿馬紹愉偕,而令懋第經理河北,聯絡關東諸軍。馬紹愉者,故兵部郎官也,嘗為陳新甲通款事至義州而還。新甲既誅,紹愉以督戰致衄,為懋第劾罷。及是紹愉已起官郎中,乃進為少卿,副懋第。懋第言:「臣此行致祭先帝后梓宮,訪東宮二王蹤跡。臣既充使臣,勢不能兼理封疆。且紹愉臣所劾罷,不當復與臣共事。必用臣經理,則乞命弘範同紹愉出使,而假臣一旅,偕山東撫臣收拾山東以待,不敢復言北行。如用臣與弘範北行,則去臣經理,但銜命而往,而罷紹愉勿遣。」閣部議止紹愉,改命原任薊督王永吉。王令仍遵前諭。

懋第瀕行言:「臣此行,生死未卜。請以辭闕之身,效一言。願陛下以先帝仇恥為心,瞻高皇之弓劍,則思成祖列聖之陵寢何存;撫江上之殘黎,則念河北、山東之赤子誰恤。更望時時整頓士馬,必能渡河而戰,始能扼河而守;必能扼河而守,始能畫江而安。」眾韙其言。王令齎白金十萬兩、幣帛數萬匹,以兵三千人護行。八月,舟渡淮。十月朔,次張家灣,本朝傳令止許百人從行。

懋第衰糸至入都門,至則館之鴻臚寺。請祭告諸陵及改葬先帝,不可,則陳太牢於旅所,哭而奠之。即以是月二十有八日遣還出都。弘範乃請身赴江南招諸將劉澤清等降附,而留懋第等勿遣。於是自滄州追還懋第,改館太醫院。順治二年六月,聞南京失守,慟哭。其從弟懋泰先為吏部員外郎,降賊,後歸本朝授官矣,來謁懋第。懋第曰:「此非吾弟也。」叱出之。至閏月十二日,與從行兵部司務陳用極,游擊王一斌,都司張良佐、劉統、王廷佐俱以不降誅,而紹愉獲免。

祁彪佳,字弘吉,浙江山陰人。祖父世清白吏。彪佳生而英特,豐姿絕人。弱冠,第天啟二年進士,授興化府推官。始至,吏民易其年少。及治事,剖決精明,皆大畏服。外艱歸。崇禎四年,起御史。疏陳賞罰之要,言:「黔功因一級疑,稽三年之敘,且恩及督撫總帥帷幄大臣,而陷敵衝鋒之士不預,何以勵行間。山東之變,六誠連陷,未嘗議及一官,欺蒙之習不可不破。」帝即命議行。又言:「九列之長,詰責時聞,四朝遺老或蒙重譴。諸臣怵嚴威,競迎合以保名位。臣所慮於大臣者此也。方伯或一二考,臺員或十餘載,竟不得遷除,監司守令多貶秩停俸。臣子精神才具無餘地,展布曷由。急功赴名之民不勝其掩罪匿瑕。臣所慮於小臣者此也。國家聞鼙鼓思將帥,茍得其人,推轂築壇,禮亦宜之。若必依序循資,冒濫之竇雖可清,獎拔之術或未盡。臣所慮於武臣者此也。撫按則使中官監視會同,隙開水火,其忠顯;潛通交結,其患深。臣所慮於內臣者此也。」忤旨譙責。

尋上《合籌天下全局疏》,以策關、寧,制登海為二大要。分析中州、秦、晉之流賊,江右、楚、粵之山賊,浙、閩、東粵之海賊,滇、黔、楚、蜀之土賊為四大勢。極控制駕馭之宜,而歸其要於戢行伍以節餉,實衛所以銷兵。復陳民間十四大苦:曰里甲,曰虛糧,曰行戶,曰搜贓,曰欽提,曰隔提,曰訐訟,曰窩訪,曰私稅,曰私鑄,曰解運,曰馬戶,曰鹽丁,曰難民。帝善其言,下之所司。出按蘇、松諸府,廉積猾四人杖殺之。宜興民發首輔周延儒祖墓,又焚翰林陳于鼎、于泰廬,亦發其祖墓。彪佳捕治如法,而於延儒無所徇,延儒憾之。回道考核,降俸,尋以侍養歸。家居九年,母服終,召掌河南道事。十六年佐大計,問遺莫敢及門。刷卷南畿,乞休,不允,便道還家。

北都變聞,謁福王於南京。王監國,或請登極。彪佳請發喪,服滿議其儀,從之。高傑兵擾揚州,民奔避江南,奸民乘機剽敚,命彪佳往宣諭,斬倡亂者數人,一方遂安。遷大理寺丞,旋擢右僉都御史,巡撫江南。蘇州諸生檄討其鄉官從賊者,奸民和之。少詹事項煜及大理寺正錢位坤、通政司參議宋學顯、禮部員外郎湯有慶之家皆被焚劫。常熟又焚給事中時敏家,毀其三代四棺。彪佳請議從逆諸臣罪,而治焚掠之徒以加等,從之。

詔設廠衛緝事官。彪佳上言:「洪武初,官民有犯,或收繫錦衣衛,高皇帝見非法凌虐,焚其刑具,送囚刑部。是祖制原無詔獄也。後乃以羅織為事,雖曰朝廷爪牙,實為權奸鷹狗。舉朝盡知其枉,而法司無敢雪。慘酷等來、周,平反無徐、杜。此詔獄之弊也。洪武十五年改儀鑾司為錦衣衛,耑掌直駕侍衛等事,未嘗令緝事也。永樂間設立東廠,始開告密門。兇人投為廝役,赤手鉅萬。飛誣及於善良,招承出於私拷,怨憤滿乎京畿。欲絕苞苴,而苞苴彌盛;欲清奸宄,而奸宄益多。此緝事之弊也。古者刑不上大夫。逆瑾用事,始去衣受杖。本無可殺之罪,乃蒙必死之刑。朝廷受愎諫之名,天下反歸忠直之譽。此廷杖之弊也。」疏奏,乃命五城御史體訪,而緝事官不設。

督輔部將劉肇基、陳可立、張應夢、於永綬駐京口,浙江入衛都司黃之奎亦部水陸兵三四千戍其地。之奎御軍嚴。四將兵恣橫,刃傷民,浙兵縛而投之江,遂有隙。已而守備李大開統浙兵斫鎮兵馬,鎮兵與相擊,射殺大開。亂兵大焚掠,死者四百人。彪佳至,永綬等遁去。彪佳劾治四將罪,賙恤被難家,民大悅。

高傑駐瓜洲,跋扈甚,彪佳剋期往會。至期,風大作,傑意彪佳必無來。彪佳攜數卒衝風渡,傑大駭異,盡撤兵衛,會彪佳於大觀樓。彪佳披肝膈,勉以忠義,共獎王室。傑感歎曰:「傑閱人多矣,如公,傑甘為死!公一日在吳,傑一日遵公約矣。」共飯而別。

群小疾彪佳,競詆諆,以沮登極、立潞王為言,彪佳竟移疾去。明年五月,南都失守。六月,杭州繼失,彪佳即絕粒。至閏月四日,紿家人先寢,端坐池中而死,年四十有四。唐王贈少保、兵部尚書,謚忠敏。

贊曰:張慎言、徐石麒等皆北都舊臣,剛方練達,所建白悉有裨時政。令其受事熙朝,從容展布,庶幾乎列卿之良也。而遭時不造,內外交訌,動輒齟齬,雖老成何能設施乾濟哉!左懋第仗節全貞,蹈死不悔,於奉使之義,亦無愧焉。


\end{pinyinscope}