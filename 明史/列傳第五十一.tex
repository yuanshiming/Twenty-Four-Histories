\article{列傳第五十一}

\begin{pinyinscope}
○李時勉陳敬宗劉鉉薩琦邢讓李紹林瀚子庭昂庭機孫燫烴謝鐸魯鐸趙永

李時勉,名懋,以字行,安福人。成童時,冬寒,以衾裹足納桶中,誦讀不已。中永樂二年進士。選庶吉士,進學文淵閣,與修《太祖實錄》。授刑部主事,復與重修《實錄》。書成,改翰林侍讀。

性剛鯁,慨然以天下為己任。十九年,三殿災,詔求直言。條上時務十五事。成祖決計都北京,時方招徠遠人。而時勉言營建之非,及遠國入貢人不宜使群居輦下。忤帝意。已,觀其他說,多中時病,抵之地;復取視者再,卒多施行。尋被讒下獄。歲餘得釋,楊榮薦復職。

洪熙元年復上疏言事。仁宗怒甚,召至便殿,對不屈。命武士撲以金瓜,脅折者三,曳出幾死。明日,改交阯道御史,命日慮一囚,言一事。章三上,乃下錦衣衛獄。時勉於錦衣千戶某有恩,千戶適蒞獄,密召醫,療以海外血竭,得不死。仁宗大漸,謂夏原吉曰:「時勉廷辱我。」言已,勃然怒,原吉慰解之。其夕,帝崩。

宣帝即位已踰年,或言時勉得罪先帝狀。帝震怒,命使者:「縛以來,朕親鞫,必殺之。」已,又令王指揮即縛斬西市,毋入見。王指揮出端西旁門,而前使者已縛時勉從端東旁門入,不相值。帝遙見罵曰:「爾小臣敢觸先帝!疏何語?趣言之。」時勉叩頭曰:「臣言諒暗中不宜近妃嬪,皇太子不宜遠左右。」帝聞言,色稍霽。徐數至六事止。帝令盡陳之。對曰:「臣惶懼不能悉記。」帝意益解,曰:「是第難言耳,草安在?」對曰:「焚之矣。」帝乃太息,稱時勉忠,立赦之,復官侍讀。比王指揮詣獄還,則時勉已襲冠帶立階前矣。

宣德五年修《成祖實錄》成,遷侍讀學士。帝幸史館,撒金錢賜諸學士。皆俯取,時勉獨正立。帝乃出餘錢賜之。正統三年以《宣宗實錄》成,進學士,掌院事兼經筵官。六年代貝泰為祭酒。八年乞致仕,不允。

初,時勉請改建國學。帝命王振往視,時勉待振無加禮。振銜之,廉其短,無所得。時勉嘗芟彞倫堂樹旁枝,振遂言時勉擅伐官樹入家。取中旨,與司業趙琬、掌饌金鑑並枷國子監前。官校至,時勉方坐東堂閱課士卷,徐呼諸生品第高下,顧僚屬定甲乙,揭榜乃行。方盛署,枷三日不解。監生李貴等千餘人詣闕乞貸。有石大用者,上章願以身代。諸生圜集朝門,呼聲徹殿庭。振聞諸生不平,恐激變。及通政司奏大用章,振內慚。助教李繼請解於會昌侯孫忠。忠,皇太后父也。忠生日,太后使人賜忠家。忠附奏太后,太后為言之帝。帝初不知也,立釋之。繼不拘檢柙,時勉嘗規切之。繼不能盡用,然心感時勉言,至是竟得其助。

大用,豐潤人。樸魯,初不為六館所知,及是名動京師。明年中鄉試,官至戶部主事。

九年,帝視學。時勉進講《尚書》,辭旨清朗。帝悅,賜予有加。連疏乞致仕,不允。十二年春乃得請。朝臣及國子生餞都門外者幾三千人,或遠送至登舟,候舟發乃去。

英宗北狩,時勉日夜悲慟。遣其孫驥詣闕上書,請選將練兵,親君子,遠小人,褒表忠節,迎還車駕,復仇雪恥。景泰元年得旨褒答,而時勉卒矣,年七十七。謚文毅。成化五年,以其孫顒請,改謚忠文,贈禮部侍郎。

時勉為祭酒六年,列格、致、誠、正四號,訓勵甚切。崇廉恥,抑奔競,別賢否,示勸懲。諸生貧不能婚葬者,節省餐錢為贍給。督令讀書,燈火達旦,吟誦聲不絕,人才盛於昔時。

始,太祖以宋訥為祭酒,最有名。其後寧化張顯宗申明學規,人比之訥。而胡儼當成祖之世,尤稱人師。然以直節重望為士類所依歸者,莫如時勉。英國公張輔暨諸侯伯奏,願偕詣國子監聽講。帝命以三月三日往。時勉升師席,諸生以次立,講《五經》各一章。畢事,設酒饌,諸侯伯讓曰:「受教之地,當就諸生列坐。」惟輔與抗禮。諸生歌《鹿鳴》之詩,賓主雍雍,盡暮散去,人稱為太平盛事。

陳敬宗,字光世,慈谿人。永樂二年進士。選庶吉士,進學文淵閣,與修《永樂大典》。書成,授刑部主事。又與修《五經四書大全》,再修《太祖實錄》,授翰林侍講。內艱歸。

宣德元年起修兩朝實錄。明年轉南京國子監司業。帝諭之曰:「侍講,清華之選;司業,師儒之席。位雖不崇,任則重矣。」九年,秩滿,選祭酒,正統三年上書言:「舊制,諸生以在監久近,送諸司歷事。比來,有因事予告者,遷延累歲,至撥送之期始赴,實長奸惰,請以肄業多寡為次第。又近有願就雜職之例,士風卑陋,誠非細故,請加禁止。」從之。

敬宗美鬚髯,容儀端整,步履有定則,力以師道自任。立教條,革陋習。六館士千餘人,每升堂聽講,設饌會食,整肅如朝廷。稍失容,即令待罪堂下。僚屬憚其嚴,誣以他事,訟之法司。周忱與敬宗善,曰:「盍具疏自理。」為屬草,辭稍遷就。敬宗驚曰:「得無誑君耶?」不果上,事亦竟白。

滿考,入京師,王振欲見之,令忱道意。敬宗曰:「吾為諸生師表,而私謁中貴,何以對諸生?」振知不可屈,乃貽之文錦羊酒,求書程子《四箴》,冀其來謝。敬宗書訖,署名而已。返其幣,終不往見。王直為吏部尚書,從容謂曰:「先生官司成久,將薦公為司寇。」敬宗曰:「公知我者,今與天下英才終日論議,顧不樂耶?」性善飲酒,至數斗不亂。襄城伯李隆守備南京,每留飲,聲伎滿左右。竟日舉杯,未嘗一盼。其嚴重如此。

十二年冬乞休,不允。景泰元年九月與尚書魏驥同引年致仕。家居不輕出。有被其容接者,莫不興起。天順三年五月卒,年八十三。後贈禮部侍郎,謚文定。

初,敬宗與李時勉同在翰林,袁忠徹嘗相之。曳二人並列曰:「二公他日功名相埒。」敬宗儀觀魁梧,時勉貌稍寢,後二人同時為兩京祭酒。時勉平恕得士,敬宗方嚴。終明世稱賢祭酒者,曰南陳北李。

劉鉉,字宗器,長洲人。生彌月而孤。及長,刲股療母疾。母卒,哀毀,以孝聞。永樂中,用善書徵入翰林,舉順天鄉試,授中書舍人。宣德時,預修成祖、仁宗《實錄》,遷兵部主事,仍供事內廷。正統中,再修《宣宗實錄》,進侍講。以學士曹鼐等薦,與修撰王振教習庶吉士。

景帝立,進侍講學士,直經筵。三年,以高穀薦,遷國子祭酒。時以國計不足,放遣諸生,不願歸者停其月廩。鉉言:「養才,國家急務。今倉廩尚盈,奈何靳此?」遂得復給。又今甄別六館生,年老貌寢,學藝疏淺者,斥為民。鉉言:「諸生荷教澤久,豈無片長?況離親戚,棄墳墓,艱苦備至,一旦被斥,非朝廷育才意。乞揀年貌衰而有學者,量授之官。」帝可其奏。尋以母喪歸。服闕,赴都,陳詢已為祭酒。帝重鉉,命與詢並任。天順初,改少詹事,侍東宮講讀。明年十月卒。帝及太子皆賜祭,賻贈有加。憲宗立,贈禮部侍郎,謚文恭。

鉉性介特,言行不茍。教庶吉士及課國子生,規條嚴整,讀書至老彌篤。仲子瀚以進士使南方。瀕行,閱其衣篋。比還,篋如故,乃喜曰:「無玷吾門矣。」瀚官終副使,能守父訓。

薩琦,字廷珪。其先西域人,後著籍閩縣。舉宣德五年進士。歷官禮部侍郎兼少詹事。天順元年卒。琦有文德,狷潔不茍合。名行與鉉相頡頏云。

邢讓,字遜之。襄陵人。年十八,舉於鄉,入國子監。為李時勉所器,與劉珝齊名。登正統十三年進士。改庶吉士,授檢討。

景泰元年,李實自瓦剌還,請再遣使迎上皇。景帝不許。讓疏曰:「上皇於陛下有君之義,有兄之恩,安得而不迎?且令寇假大義以問我,其何辭以應?若從群臣請,仍命實齎敕以往,且述迎復之指。雖上皇還否未可必,而陛下恩義之篤昭然於天下。萬一迎而不許,則我得責直於彼,以興問罪之師,不亦善乎。」疏入,帝委曲諭解之。天順末,父憂歸。未終喪,起修《英宗實錄》,進修撰。

成化二年超遷國子祭酒。慈懿太后崩,議祔廟禮,讓率僚屬疏諫。兩京國學教官,例不得遷擢,讓等以為言,由科目者,滿考得銓敘。讓在太學,亦力以師道自任,修《辟雍通志》,督諸生誦小學及諸經。痛懲謁告之弊,時以此見稱,而謗者亦眾。為人負才狹中。意所輕重,輒形於詞色,名位相軋者多忌之。

五年擢禮部右侍郎。越二年,以在國子監用會饌錢事,與後祭酒陳鑒、司業張業、典籍王允等,俱得罪坐死。諸生訴闕下,請代。復詔廷臣雜治,卒坐死,贖為民。

鑑既得罪,吏部尚書姚夔請起致仕禮部侍郎李紹為祭酒。馳召之,而紹已卒。

紹字克述,安福人。宣德八年進士。改庶吉士,授檢討。大學士楊士奇臥病,英宗遣使詢人才,士奇舉紹等五人以對。土木之敗,京師戒嚴,朝士多遣家南徙。紹曰:「主辱臣死,奚以家為?」卒不遣。累遷翰林學士。以李賢、王翱薦,擢禮部侍郎。成化二年以疾求解職。紹好學問,居官剛正有器局,能獎掖後進。其卒也,帝深惜之。

林瀚,字亨大,閩人。父元美,永樂末進士,撫州知府。瀚舉成化二年進士。改庶吉士,授編修。再遷諭德,請急歸。

弘治初,召修《憲宗實錄》。充經筵講官。稍遷國子監祭酒,進禮部右侍郎,掌監事如故。典國學垂十年,饌銀歲以百數計,悉貯之官,以次營立署舍。師儒免僦居,由瀚始。歷吏部左、右侍郎。

十三年拜南京吏部尚書。以災異率群僚陳十二事。御史王獻臣自遼東逮下詔獄,儒士孫伯堅等夤緣為中書舍人。瀚疏爭,忤旨。乞罷,不許。已,奏請重根本:曰保固南京,曰佑啟皇儲,曰撫綏百姓,曰增進賢才。

正德元年四月,吏部尚書馬文升去位,言官丘俊、石介等薦瀚。帝用侍郎焦芳,乃改瀚南京兵部,參贊機務。命未至,瀚引疾乞休,因陳養正心、崇正道、務正學、親正人四事。優詔慰留。時災異數見,瀚及南京諸臣條時政十二事。語涉近倖,多格不行。

瀚素剛方,與守備中官不合,他內臣進貢道其地者,瀚每裁抑之,遂交譖於劉瑾。會劉健、謝遷罷政,瀚聞太息。言官戴銑等以留健、遷被徵,瀚獨贐送,瑾聞益恨。明年二月假銑等獄詞,謫瀚浙江參政。致仕。旋指為奸黨。瑾誅,復官,致仕。予月廩歲隸如故事。尋命有司歲時存問。瀚為人謙厚,而自守介然。卒年八十六。贈太子太保,謚文安。子九人,庭昂、庭機最顯。

庭昂,字利瞻。瀚次子也。弘治十二年進士。授兵部主事。歷職方郎中。吏部尚書張彩欲改為御史,固謝之,乃以為蘇州知府。頻歲大水,疏請停織造,罷繁征,割關課備振。再上,始報可。遷雲南左參政。正德九年,以父老乞侍養。時子炫已成進士,官禮部主事,亦謁假歸。三世一堂,鄉人稱盛事。

嘉靖初,父憂,服闋,起官江西,歷湖廣左、右布政使。舉治行卓異,擢右副都御史,巡撫保定諸府。歷工部右侍郎。應詔言郊壇大工,南城、西苑相繼興作,請以儉約先天下。又因災傷,乞撤還採木、燒造諸使。進左,拜尚書,加太子太保。時帝方大興土木功,庭昂所規畫多稱意。會詔建沙河行宮,庭昂議加天下田賦,為御史桑喬、給事中管見所劾。乞罷,歸卒,贈少保,謚康懿。炫終通政司參議。

庭機,字利仁,瀚季子也。嘉靖十四年進士。改庶吉士,授檢討,遷司業,擢南京祭酒,累遷至工部尚書。穆宗立,調禮部,俱官陪京。時子燫已為祭酒,遂致仕歸。萬曆九年卒,年七十有六。贈太子太保,謚文僖。子燫、烴。

燫,字貞恒,庭機長子。嘉靖二十六進士。改庶吉士,授檢討。景恭王就邸,命燫侍講讀。三遷國子祭酒。自燫祖瀚,父庭機,三世為祭酒,前此未有也。,隆度改元,擢禮部右侍郎,充日講官。寇犯邊,條上備邊七事。改吏部,調南京吏部,署禮部事。魏國公徐鵬舉廢長立幼,燫持不可。萬曆元年進工部尚書,改禮部。仍居南京。名位一與父庭機等。母喪去官。服闋,以庭機篤老侍養,家居七年,先父庭機卒。贈太子少保,謚文恪。明代三世為尚書,並得謚文。林氏一家而已。子世勤,性篤孝。芝生者三,枯篁復青。御史上其事,被旌。

烴字貞耀,庭機次子也。嘉靖四十一年進士。授戶部主事,歷廣西副使。兄燫卒,請急歸養。久之,歷太僕少卿。因災異極陳礦稅之害,請釋逮繫諸臣。不報。終南京工部尚書致仕。林氏三世五尚書,皆內行修潔,為時所稱。

謝鐸,字鳴治,浙江太平人。天順末進士。改庶吉士,授編修,預修《英宗實錄》。性介特,力學慕古,講求經世務。

成化九年校勘《通鑑綱目》,上言:「《綱目》一書,帝王龜鑒。陛下命重加考定,必將進講經筵,為致治資也。今天下有太平之形,無太平之實,因仍積習,廢實徇名。曰振綱紀,而小人無畏忌;曰勵風俗,而縉紳棄廉恥。飭官司,而污暴益甚;恤軍民,而罷敝益極。減省有制,而興作每疲於奔命;蠲免有詔,而徵斂每困於追呼。考察非不舉,而倖門日開;簡練非不行,而私撓日眾。賞竭府庫之財,而有功者不勸;罰窮讞覆之案,而有罪者不懲。以至修省祈禱之命屢頒,水旱災傷之來不絕。禁垣被震,城門示災,不思疏動旋轉,以大答天人之望,是則誠可憂也。願陛下以古證今,兢兢業業,然後可長治久安,而載籍不為無用矣。」帝不能從。

時塞上有警,條上備邊事宜,請養兵積粟,收復東勝、河套故疆。又言:「今之邊將,無異晚唐債帥。敗則士卒受其殃,捷則權豪蒙其賞。且剋侵軍餉,辦納月錢,三軍方怨憤填膺,孰肯為國效命者?」語皆切時弊。秩滿,進侍講,直經筵。遭兩喪,服除,以親不逮養,遂不起。

弘治初,言者交薦,以原官召修《憲宗實錄》。三年擢南京國子祭酒。上言六事,曰擇師儒,慎科貢,正祀典,廣載籍,復會饌,均撥歷。其正祀典,請進宋儒楊時而罷吳澄。禮部尚書傅瀚持之,乃進時而澄祀如故。

明年謝病去。家居將十年,薦者益眾。會國子缺祭酒,部議起之。帝素重鐸,擢禮部右侍郎,管祭酒事。屢辭,不許。時章懋為南祭酒,兩人皆人師,諸生交相慶。居五年,引疾歸。

鐸經術湛深,為文章有體要。兩為國子師,嚴課程,杜請謁,增號舍,修堂室,擴廟門。置公廨三十餘居其屬。諸生貧者周恤之,死者請官定制為之殮。家居好周恤族黨,自奉則布衣蔬食。正德五年卒。贈禮部尚書,謚文肅。

魯鐸,字振之,景陵人。弘治十五年會試第一。歷編修。閉門自守,不妄交人。武宗立,使安南,卻其饋。

正德二年遷國子監司業。累擢南祭酒,尋改北。鐸屢典成均,教士切實為學,不專章句。士有假歸廢學者,訓飭之,悔過乃已。久之,謝病歸。

嘉靖初,以刑部尚書林俊薦,用孝宗朝謝鐸故事,起南祭酒。踰年,復請致仕。累徵不起,卒。謚文恪。

鐸以德望重於時。居鄉,有盜掠牛馬,或紿云:「魯祭酒物也」,舍之去。大學士李東陽生日,鐸為司業,與祭酒趙永皆其門生也,相約以二帕為壽。比檢笥,亡有,徐曰:「鄉有饋乾魚者,盍以此往?」詢諸庖,食過半矣,以其餘詣東陽。東陽喜,為烹魚置酒,留二人飲,極歡乃去。

永,字爾錫,臨淮人。與鐸同年進士,亦官編修。復與鐸相繼為祭酒。尋遷南京禮部侍郎。大學士楊一清重其才,欲引以自助,乃為他語挑之。永正色曰:「可以纓冠污吾道乎?」遂請致仕去。人服其廉介。

贊曰:明太祖時,國學師儒,體貌優重。魏觀、宋訥為祭酒,造就人才,克舉其職。諸生銜命奉使,往往擢為大官,不專以科目進也。中葉以還,流品稍雜,撥歷亦為具文,成均師席,不過為儒臣序遷之地而已。李時勉、陳敬宗諸人,方廉清鯁,表範卓然,類而傳之,庶觀者有所法焉。


\end{pinyinscope}