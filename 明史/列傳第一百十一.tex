\article{列傳第一百十一}

\begin{pinyinscope}
盛應期朱衡翁大立潘志伊潘季馴萬恭吳桂芳傅希摯王宗沐子士崧士琦士昌從子士性劉東星胡瓚徐貞明{{伍袁萃

盛應期,字思徵,吳江人。弘治六年進士。授都水主事,出轄濟寧諸閘。太監李廣家人市私鹽至濟,畏應期,投水中去。會南京進貢內官誣應期阻薦新船,廣從中構逮應期及主事范璋下詔獄。璋筦衛河,亦忤中貴者也。獄具,謫雲南驛丞。稍遷祿豐知縣。正德初,歷雲南僉事。武定知府鳳應死,其妻攝府事,子朝鳴為寇。應期單車入其境,母子惶怖,歸所侵。策鳳氏終亂,奏降其秩,設官制之。寢不行,後卒叛。與御史張璞、副使晁必登抑鎮守太監梁裕。裕劾三人,俱逮下詔獄,璞竟拷死。

會乾清宮災,應期得復職,四遷至陜西右布政使。擢右副都御史,巡撫四川。討平天全六番招討使高文林。會泉江僰蠻普法惡作亂,富順奸民謝文禮、文義附之。法惡死,指揮何卿等先後討誅文禮、文義。應期賚銀幣,以憂歸。嘉靖二年,起故官,巡撫江西。宸濠亂後,瘡痍未復,奏免雜調緡錢數十萬,請留轉輸南京米四十七萬,銀二十萬,以食饑民。又令諸府積穀備荒至百餘萬。尋進兵部右侍郎,總督兩廣軍務。將行,籍上積穀數。帝以陳洪謨代,而獎賚應期。後洪謨積益多,亦被賚。應期至廣,偕撫寧侯朱麒督參將李璋等,討平思恩土目劉召,復賚銀幣。朝議大征岑猛。應期條上方略七事,言廣兵疲弱不可用。麒等恚。會御史許中劾應期暴虐,麒等因相與為流言。御史鄭洛書復劾應期賄結權貴。應期已遷工部侍郎,引疾歸。

六年,黃河水溢入漕渠,沛北廟道口淤數十里,糧艘為阻,侍郎章拯不能治。尚書胡世寧、詹事霍韜、僉事江良材請於昭陽湖東別開漕渠,為經久計。議未定,以御史吳仲言召拯還,即家拜應期右都御史以往。應期乃議於昭陽湖東,北進江家口,南出留城口,開浚百四十餘里,較疏舊河力省而利永。夫六萬五千,銀二十萬兩,克期六月。工未成,會旱災修省,言者多謂開河非計,帝遽令罷役。應期請展一月竟其功,不聽。初,應期請令郎中柯維熊分浚支河,維熊力贊新河之議,至是亦言不便。應期上章自理,帝怒,詔與維熊俱奪職。世寧言:「新河之議倡自臣。應期克期六月,今四月,功已八九。緣程工促急,怨讟煩興。維熊反覆變詐,傾大臣,誤國事。自古國家僨大事,必責首議,臣請與同罷。」帝不許。後更赦,復官致仕,卒。應期罷後三十年,朱衡循新河遺跡成之,運道蒙利焉。

朱衡,字士南,萬安人。嘉靖十一年進士。歷知尤溪、婺源,有治聲。遷刑部主事,歷郎中。出為福建提學副使,累官山東布政使。三十九年,進右副都御史巡撫其地。奏言:「比遼左告饑,暫弛登、萊商禁,轉粟濟之。猾商遂竊載他貨,往來販易,並開青州以西路。海島亡命,陰相構結,禁之便。」從之。召為工部右侍郎。

四十四年,進南京刑部尚書。其秋,河決沛縣飛雲橋,東注昭陽湖,運道淤塞百餘里。改衡工部尚書兼右副都御史,總理河漕。衡馳至決口,舊渠已成陸。而故都御史盛應期所開新河,自南陽以南東至夏村,又東南至留城,故址尚在。其地高,河決至昭陽湖止,不能復東,可以通運,乃定議開新河,築堤呂孟湖以防潰決。河道都御史潘季馴以為浚舊渠便,議與衡不合。衡持益堅,引占魚、薛沙諸水入新渠,築馬家橋堤以遏飛雲橋決口,身自督工。劾罷曹濮副使柴淶,重繩吏卒不用命者,浮議遂起。明年,給事中鄭欽劾衡虐民倖功,詔遣給事中何起鳴往勘,工垂竣矣。及秋,河決馬家橋,議者紛然謂功不可成。起鳴初主衡議,亦變其說,與給事中王元春、御史黃襄交章請罷衡。會新河已成,乃止。河長一百九十四里。漕艘由境山入,通行至南陽。未幾,季馴以憂去,詔衡兼理其事。

隆慶元年,加太子少保。山水驟溢,決新河,壞漕艘數百。給事中吳時來言:「新河受東、兗以南費、嶧、鄒、滕之水。以一堤捍群流,豈能不潰?宜分之以殺其勢。」衡乃開支河四,洩其水入赤山湖。明年秋,召還部。又明年,衡上疏曰:「先臣宋禮浚治舊渠,測量水平,計濟寧平地與徐州境山巔相準,北高南下,懸流三十丈。故魯橋閘以南稍啟立涸,舟行半月始達。東、兗之民增閘挑淺,苦力役者百六十年。屬者改鑿新渠,遠避黃流,舍卑就高,地形平衍,諸閘不煩起閉,舟行日可百餘里,夫役漫無事事。近河道都御史翁大立奏請裁革,宜可聽。」於是汰閘官五,夫役六千餘,以其僦直為修渠費。四年秋,河決睢寧,起季馴總理。明年冬,閱視河道給事中雒遵劾罷季馴,言廷臣可使,無出衡右者。六年正月,詔兼左副都御史,經理河道。

穆宗崩,大學士高拱以山陵工請召衡。會邳州工亦竣,衡遂還朝。衡先後在部,禁止工作,裁抑浮費,所節省甚眾。穆宗時,內府監局加征工料,濫用不訾,衡隨時執奏。未幾,詔南京織造太監李佑趨辦袍緞千八百餘匹,衡因言官孫枝、姚繼可、嚴用和、駱問禮先後諫,再疏請,從之。帝切責太監崔敏,傳令南京加造緞十餘萬匹。衡議停新造,但責歲額,得減新造三之二。命造鰲山燈,計費三萬餘兩,又命建光泰殿、瑞祥閣於長信門,衡皆奏止之。及神宗即位,首命停織造,而內臣不即奉詔,且請增織染所顏料。衡奏爭,皆得請。皇太后傳諭發帑金修涿州碧霞元君廟。衡復爭,報聞。

衡性強直,遇事不撓,不為張居正所喜。萬歷二年,給事中林景暘劾衡剛心復。衡再疏乞休。詔加太子太保,馳驛歸。其年夏,大雨壞昭陵迍恩殿,追論督工罪,奪宮保。卒年七十三。子維京,自有傳。

翁大立,餘姚人。嘉靖十七年進士。累官山東左布政使。三十八年,以右副都御史巡撫應天、蘇州諸府。蘇州以倭警募壯士,後兵罷無所歸,群聚剽奪。大立得其主名,捕甚急。惡少懼,夜劫縣衛獄,縱囚自隨,攻都御史行署,大立率妻子遁。知府王道行督兵力拒之,乃斬葑門,奔入太湖為盜。命大立戴罪捕賊,尋被劾罷。久之,起故官,巡撫山東。遭喪不赴。

隆慶二年,命督河道。朱衡既開新河,漕渠便利。大立因頌新河之利有五,而請浚回回墓以達鴻溝,引昭陽之水沿鴻溝出留城,以溉湖下腴田千頃。未幾,又請鑿邵家嶺,令水由地濱溝出境山,入漕河。帝皆從之。三年七月,河大決沛縣,漕艘阻不進。帝從大立請,大行振貸。大立又請漕艘後至者貯粟徐州倉,平價出糶。詔許以三萬石賚民。大立以下民昏墊、閭閻愁困狀帝莫能周知,乃繪圖十二以獻。且言:「時事可憂,更不止此。東南財賦區,而江海泛溢,粒米不登,京儲可慮一也。邊關千里,悉遭洪水,墩堡傾頹,何恃以守?可慮二也。畿輔、山東、河南,霪雨既久,城郭不完,寇盜無備,可慮三也。江海間颶風鼓浪,舟艦戰卒,悉入波流,海防可慮四也。淮、浙鹽場鹹泥盡沒,灶戶流移,商賈至,國課可慮五也。望陛下以五患十二圖付公卿博議,速求拯濟之策。」帝留圖備覽,下其奏於所司。

當是時,黃河既決,淮水復漲。自清河縣至通濟閘抵淮安城西淤三十餘里,決方、信二壩出海,平地水深丈餘,寶應湖堤往往崩壞。山東沂、莒、郯城水溢,從沂河、直河出邳州,人民多溺死。大立奔走經營,至四年六月,鴻溝、境山諸工,及淮流疏濬,次第告成。帝喜,錫賚有差。時大立已升工部右侍郎,旋改兵部,為左。會代者陳大賓未至,而山東沙、薛、汶、泗諸水驟漲,決仲家淺諸處,黃河又暴至,茶城復淤。已而淮自泰山廟至七里溝亦淤十餘里。其明年,遂為給事中宋良佐劾罷。萬歷二年,起南京刑部右侍郎,就改吏部。明年入為刑部右侍郎,再遷南京兵部尚書。六年,致仕歸。

先是,隆慶末,有錦衣指揮周世臣者,外戚慶雲侯裔也。家貧無妻,獨與婢荷花兒居。盜入其室,殺世臣去。把總張國維入捕盜,惟荷花兒及僕王奎在,遂謂二人姦弒其主。獄成,刑部郎中潘志伊疑之,久不決。及大立以侍郎署部事,憤荷花兒弒主,趣志伊速決。志伊終疑之,乃委郎中王三錫、徐一忠同讞。竟無所平反,置極刑。踰數年,獲真盜。都人競稱荷花兒冤,流聞禁中。帝大怒,欲重譴大立等。會給事中周良寅、蕭彥復劾之,乃追奪大立職,調一忠、三錫於外。志伊時已知九江府,亦謫知陳州。

志伊,吳江人。進士,終廣西右參政。歷官有聲。

潘季馴,字時良,烏程人。嘉靖二十九年進士。授九江推官。擢御史,巡撫廣東。行均平里甲法,廣人大便。臨代去,疏請飭後至者守其法,帝從之。進大理丞。四十四年,由左少卿進右僉都御史,總理河道。與朱衡共開新河,加右副都御史。隆慶四年,河決邳州、睢寧。起故官,再理河道,塞決口。明年,工竣,坐驅運船入新溜漂沒多,為勘河給事中雒遵劾罷。

萬曆四年夏,再起官,巡撫江西。明年冬,召為刑部右侍郎。是時,河決崔鎮,黃水北流,清河口淤澱,全淮南徙,高堰湖堤大壞,淮、揚、高郵、寶應間皆為巨浸。大學士張居正深以為憂。河漕尚書吳桂芳議復老黃河故道,而總河都御史傅希摯欲塞決口,束水歸漕,兩人議不合。會桂芳卒,六年夏,命季馴以右都御史兼工部左侍郎代之。季馴以故道久湮,雖濬復,其深廣必不能如今河,議築崔鎮以塞決口,築遙堤以防潰決。又:「淮清河濁,淮弱河強,河水一斗,沙居其六,伏秋則居其八,非極湍急,必至停滯。當藉淮之清以刷河之濁,築高堰束淮入清口,以敵河之強,使二水並流,則海口自濬。即桂芳所開草灣,亦可不復修治。」遂條上六事,詔如議。

明年冬,兩河工成。又明年春,加太子太保,進工部尚書兼左副都御史。季馴初至河上,歷虞城、夏邑、商丘,相度地勢。舊黃河上流,自新集經趙家圈、蕭縣,出徐州小浮橋,極深廣。自嘉靖中北徙,河身既淺,遷徙不常,曹、單、豐、沛常苦昏墊。上疏請復故河。給事中王道成以方築崔鎮高堰,役難並舉。河南撫按亦陳三難,乃止。遷南京兵部尚書。十一年正月,召改刑部。

季馴之再起也,以張居正援。居正歿,家屬盡幽繫,子敬修自縊死。季馴言:「居正母逾八旬,旦暮莫必其命,乞降特恩宥釋。」又以治居正獄太急,宣言居正家屬斃獄者已數十人。先是,御史李植、江東之輩與大臣申時行、楊巍相訐。季馴力右時行、巍,痛詆言者,言者交怒。植遂劾季馴黨庇居正,落職為民。十三年,御史李棟上疏訟曰:「隆慶間,河決崔鎮,為運道梗。數年以來,民居既奠,河水安流,咸曰:『此潘尚書功也。』昔先臣宋禮治會通河,至於今是賴,陛下允督臣萬恭之請,予之謚廕。今季馴功不在禮下,乃當身存之日,使與編戶齒,寧不隳諸臣任事之心,失朝廷報功之典哉!」御史董子行亦言季馴罪輕責重。詔俱奪其俸。其後論薦者不已。

十六年,給事中梅國樓復薦,遂起季馴右都御史,總督河道。自吳桂芳後,河漕皆總理,至是復設專官。明年,黃水暴漲,衝入夏鎮,壞田廬,居民多溺死。季馴復築塞之。十九年冬,加太子太保、工部尚書兼右都御史。

季馴凡四奉治河命,前後二十七年,習知地形險易。增築設防,置官建閘,下及木石樁埽,綜理纖悉,積勞成病。三疏乞休,不允。二十年,泗州大水,城中水三尺,患及祖陵。議者或欲開傅寧湖至六合入江,或欲浚周家橋入高、寶諸湖,或欲開壽州瓦埠河以分淮水上流,或欲弛張福堤以洩淮口。季馴謂祖陵王氣不宜輕洩,而巡撫周寀、陳于陛、巡按高舉謂周家橋在祖陵後百里,可疏濬,議不合。都給事中楊其休請允季馴去。歸三年卒,年七十五。

萬恭,字肅卿,南昌人。嘉靖二十三年進士。授南京文選主事,歷考功郎中。壽王喪過南京,中貴欲令朝王妃,恭厲聲曰:「禮不朝后,況妃乎!」遂止。就遷光祿少卿,入改大理。

四十二年,寇逼通州,帝方急兵事。以兵部右侍郎蔡汝楠、協理戎政侍郎喻時不勝任,調之南京,欲代以鄭曉、楊順、葛縉,手詔問徐階。階以曉文士,順、縉匪人,請命吏部推擇。帝乃諭尚書嚴訥越格求之,遂以湖廣參政李燧代時,而命恭代汝楠。恭列上選兵、議將、練兵車、火器諸事,皆報可。明年,燧罷,眾將推恭,恭引疾。及用趙炳然,恭起視事。於是給事中胡應嘉劾恭奸欺。恭奏辯,部議調恭。詔勿問。恭不自安,力請劇邊自效。乃命兼僉都御史,巡撫山西。甫至,寇犯龍鬚墩,恭伏兵擊卻之。未幾,寇五萬騎至朔州川,恭與戰老高墓。列車為陣,發火器,寇少卻。忽風起,火反焚車,寇復大至。諸將殊死戰,寇乃去。事聞,賚銀幣。巡撫故無旗牌,恭請得之。濱河州縣患套寇東掠,歲鑿冰以防,恭為築牆四十里。教人以耕及用水車法,民大利之。浹歲,以內艱歸。

隆慶初,給事中岑用賓等拾遺及恭。吏部尚書楊博議,仍用之邊方。暨服闋,恭遂不出。六年春,給事中劉伯燮薦恭異才。會河決邳州,運道大阻,已遣尚書朱衡經理,復命恭以故官總理河道。恭與衡築長堤,北自磨臍溝迄邳州直河,南自離林迄宿遷小河口,各延三百七十里。費帑金三萬,六十日而成。高、寶諸河,夏秋泛濫,歲議增堤,而水益漲。恭緣堤建平水閘二十餘,以時洩蓄,專令浚湖,不復增堤,河遂無患。

恭強毅敏達,一時稱才臣。治水三年,言者劾其不職,竟罷歸。家居垂二十年卒。孫燝,自有傳。

吳桂芳,字子實,新建人。嘉靖二十三年進士。授刑部主事。有崔鑑者,年十三,忿父妾凌母,手刃之。桂芳為著論擬赦。尚書聞淵曰:「此董仲舒《春秋》斷獄,柳子厚《復讎議》也。鑒遂得宥。及淵入吏部,欲任以言職。會聞繼母病,遽請歸,留之不可。起補禮部,歷遷揚州知府。禦倭有功,遷俸一級。又建議增築外城。揚有二城,自桂芳始。歷浙江左布政使,進右僉都御史,巡撫福建。父喪歸。起故官,撫治鄖陽。尋進右副都御史總理河道,未任。兩廣總督張臬以非軍旅才被劾罷,部議罷總督,改桂芳兵部右侍郎兼右僉都御史,提督兩廣軍務兼理巡撫。

兩廣群盜河源李亞元、程鄉葉丹樓連歲為患,潮州舊倭屯據鄒塘。桂芳先討倭。以降賊伍端為前驅,官軍繼進,一日夜克三巢,焚斬四百餘人。帝深嘉之,令與南贛提督吳百朋乘勝滅賊。而新倭寇福建省為戚繼光所敗,流入境。桂芳、百朋會調土、漢兵,乘其初至,急擊之。倭懼,悉奔甲子崎沙,奪漁舟入海。暴風起,皆覆溺死。脫者還海豐,副總兵湯克寬擒斬殆盡。因建議海道副使轄東莞以西至瓊州,領番夷市舶,更設海防僉事,巡東莞以東至惠潮,專禦倭寇。又進討亞元、丹樓,平之。

降賊王西橋、吳平已撫復叛。西橋掠東莞,敗都指揮劉世恩兵,執肇慶同知郭文通以求撫。桂芳擒斬之,進討平。平初據南澳,為戚繼光、俞大猷所敗,奔饒平鳳凰山,掠民舟出海,自陽江奔安南。桂芳檄安南萬寧宣撫司進剿,遣克寬以舟師會之,夾擊平萬橋山下。乘風縱火,平軍死無算,擒斬三百九十餘人。參將傅應嘉言平已擒,後復云溺死。福建巡撫汪道昆奏聞,桂芳不肯,曰:「風火交熾時,何以知其必死也?」平黨林道乾復窺南澳,時議設參將戍守。桂芳言:「澳中地險而腴。元時曾設兵戍守,戍兵即據以叛,此禦盜生盜也,不如戍柘林便。」從之。召為南京兵部右侍郎,尋改北部。隆慶初,轉左,以疾乞歸。言官數論薦。

萬曆三年冬,即家起故官,總督漕運兼巡撫鳳陽。明年春,桂芳以淮、揚洪潦奔流,惟雲梯關一徑入海,致海湧橫沙,河流汎溢,而興、鹽、高、寶諸州縣所在受災,請益開草灣及老黃河故道以廣入海之路,修築高郵東西二堤以蓄湖水。皆下所司議行。未幾,草灣河工告成。是年秋,河決曹縣、徐州、桃源,給事中劉鉉疏議漕河,語侵桂芳。桂芳疏辯曰:「草灣之開,以高、寶水患衝齧,疏以拯之,非能使上游亦不復漲也。今山陽以南諸州縣,水落布種,斗米四分,則臣斯舉亦既得策矣。若徐、邳以上,非臣所屬,臣何與焉。」因請罷。御史邵陛言:「諸臣以河漲歸咎草灣,阻任事氣,乞策勵桂芳,益底厥績,而詰責河臣傅希摯曠職。」從之。

其明年,希摯議塞崔鎮決口,束水歸漕,而桂芳欲衝刷成河以為老黃河入海之道。廷議以二人意見不合,改希摯撫陜西,以李世達代。未幾,又改世達他任,命桂芳兼理河漕。六年正月,詔進工部尚書兼右副都御史,居職如故。未踰月,卒。尋以高郵湖堤成,贈太子少保。

傅希摯,衡水人。累官右僉都御史,巡撫山東。隆慶末,戶部以餉乏議裁山東、河南民兵,希摯爭之而止。改總理河道。以茶城淤塞,開梁山以下寧洋山,出右洪口。萬曆五年,進右副都御史,巡撫陜西。已遷戶部右侍郎,坐隴右礦賊未靖,論罷。起總督漕運,歷南京戶、兵二部尚書。召理戎政,以老被劾。加太子少保致仕。

王宗沐,字新甫,臨海人。嘉靖二十三年進士。授刑部主事。與同官李攀龍、王世貞輩,以詩文相友善。宗沐尤習吏治。歷江西提學副使。修白鹿洞書院,引諸生講習其中。

三遷山西右布政使。所部歲祲,宗沐因入覲上疏曰:「山西列郡俱荒,太原尤甚。三年於茲,百餘里不聞雞聲。父子夫婦互易一飽,命曰『人市』。宗祿八十五萬,累歲缺支,饑疫死者幾二百人。夫山西京師右掖,自故關出真定,自忻、代出紫荊,皆不過三日。宣、大之糧雖派各郡,而運本色者皆在太原。饑民一聚,蹂踐劫奪,歲供宣、大兩鎮六十七萬餉,誰為之辦?此可深念者一也。四方奏水旱者以十分上,部議常裁而為三,所免不過存留者而已。今山西所謂存留者,二鎮三關之輸也。存留乃反急於起運,是山西終不蒙分毫之寬。此可深念者二也。開疆萬山之中,巖阻巉絕,太原民不得至澤、潞,安望就食他所?獨真定米稍可通。然背負車運,率二斗而致一斗,甫至壽陽,則價已三倍矣。是可深念者三也。饑民相聚為盜,招之不可,勢必撲殺。小則支庫金,大則請內帑。與其發帑以賞殺盜之人,孰若發帑使不為盜?此可深念者四也。近丘富往來誘惑,邊民妄傳募人耕田不取租稅。愚民何知,急不暇擇,長邊八百餘里,誰要之者?彼誘而眾,我逃而虛。此可深念者五也。」因請緩征逋賦,留河東新增鹽課以給宗祿。尋改廣西左布政使,再補山東。

隆慶五年,給事中李貴和請開膠萊河。宗沐以其功難成,不足濟運,遺書中朝止之。拜右副都御史,總督漕運兼巡撫鳳陽,極陳運軍之苦,請亟優恤。又以河決無常,運者終梗,欲復海運,上疏曰:「自會通河開濬以來,海運不講已久。臣近官山東,嘗條斯議。巡撫都御史梁夢龍毅然試之,底績無壅,而慮者輒苦風波。夫東南之海,天下眾水之委也,茫渺無山,趨避靡所,近南水暖,蛟龍窟宅。故元人海運多驚,以其起自太倉、嘉定而北也。若自淮安而東,引登、萊以泊天津,是謂北海,中多島嶼,可以避風。又其地高而多石,蛟龍有往來而無窟宅。故登州有海市,以石氣與水氣相搏,映石而成,石氣能達於水面,以石去水近故也。北海之淺,是其明驗。可以佐運河之窮,計無便於此者。」因條上便宜七事。明年三月遂運米十二萬石自淮入海,五月抵天津。敘功,與夢龍俱進秩,賜金幣。而南京給事中張煥言:「比聞八舟漂沒,失米三千二百石。宗沐預計有此,私令人糴補。夫米可補,人命可補乎?宗沐掩飾視聽,非大臣誼。」宗沐疏辯求勘。詔行前議,習海道以備緩急。未幾,海運至即墨,揚颶風大作,覆七舟,都給事中賈三近、御史鮑希顏及山東巡撫傅希摯俱言不便,遂寢。時萬曆元年也。

宗沐以徐、邳俗獷悍,多姦猾,濱海鹽徒出沒,六安、霍山礦賊竊發,奏設守將。又召豪俠巨室三百餘人充義勇,責令捕盜,後多以功給冠帶。遷南京刑部右侍郎,召改工部。尋進刑部左侍郎,奉敕閱視宣、大、山西諸鎮邊務。母喪歸。九年,以京察拾遺罷,不敘。居家十餘年卒。贈刑部尚書。天啟初,追謚襄裕。

子士崧、士琦、士昌,從子士性,皆進士。士崧官刑部主事。士琦歷重慶知府。播州宣慰使楊應龍叛,承總督邢玠檄至松坎撫定之。遂進兵備副使,治其地。尋以山東參政監軍朝鮮有功,超擢河南右布政使。坐應龍復叛,降湖廣右參政。歷山東右布政使,佐餘宗濬封順義王,進秩賜金。擢右副都御史巡撫大同,被劾擬調。未幾卒。

士昌由龍谿知縣擢兵科給事中。寇犯固原、甘肅,方議諸將罪,而延綏兩以捷聞。兵部請告廟宣捷,士昌奏止之。改禮科。礦稅興,疏言:「近日御題黃纛,遍布關津;聖旨朱牌,委褻蔀屋。遂使三家之村,雞犬悉盡;五都之市,絲粟皆空。且稅以店名,無異北齊之市肆;官從內遣,何啻西苑之斜封!」不報。二十九年,帝將冊立東宮,而故緩其期。士昌偕同官楊天民極諫,謫貴州鎮遠典史。屢遷大理右丞署事,與張問達共定張差獄。旋進右少卿,擢右僉都御史,巡撫福建。歸卒。

士性,字恒叔,由確山知縣徵授禮科給事中。首陳天下大計,言朝廷要務二,曰親章奏,節財用;官司要務三,曰有司文網,督學科條,王官考核;兵戎要務四,曰中州武備,晉地要害,北寇機宜,遼左戰功。疏凡數千言,深切時弊,多議行。詔製鰲山燈,未幾,慈寧宮火,士性請停前詔,帝納之。楊巍議黜丁此呂,士性劾巍阿輔臣申時行,時行納巍邪媚,皆失大臣誼。寢不行。時行,士性座主也。久之,疏言:「朝廷用人,不宜專取容身緘默,緩急不足恃者。請召還沈思孝、吳中行、艾穆、鄒元標、黃道瞻、蔡時鼎、聞道立、顧憲成、孫如法、姜應麟、馬應圖、王德新、盧洪春、彭遵古、諸壽賢、顧允成等。忤旨,不報。遷吏科給事中,出為四川參議,歷太僕少卿。河南缺巡撫,廷推首王國,士性次之。帝特用士性。士性疏辭,言資望不及國。帝疑其矯,且謂國實使之,遂出國於外,調士性南京。久之,就遷鴻臚卿,卒。

劉東星,字子明,沁水人。隆慶二年進士。改庶吉士,授兵科給事中。大學士高拱攝吏部,以非時考察,謫蒲城縣丞。徙盧氏知縣,累遷湖廣左布政使。萬曆二十年,擢右僉都御史,巡撫保定。時朝鮮以倭難告。王師調集,悉會天津,而天津、靜海、滄州、河間皆被災。東星請漕米十萬石平糶,民乃濟。召為左副都御史。進吏部右侍郎,以父老請侍養歸,瀕行而父卒。

二十六年,河決單之黃堌,運道堙阻,起工部左侍郎兼右僉都御史,總理河漕。初,尚書潘季馴議開黃河上流,循商、虞而下,歷丁家道口出徐州小浮橋,即元賈魯所浚故道也,朝廷以費巨未果。東星即其地開濬。起曲里鋪至三仙臺,抵小浮橋。又濬漕渠自徐、邳至宿。計五閱月工竣,費僅十萬。詔嘉其績,進工部尚書兼右副都御史。明年,渠邵伯、界首二湖。又明年,奉開泇河。泇界滕、嶧間,南通淮、海,引漕甚便。前總督翁大立首議開濬,後尚書朱衡、都御史傅希摯復言之。朝廷數遣官行視,乞無成畫。河臣舒應龍嘗鑿韓莊,工亦中輟。東星力任其役。初議費百二十萬,及工起,費止七萬,而渠已成十之三。會有疾,求去。屢旨慰留。卒官。後李化龍循其遺跡,與李三才共成之,漕永便焉。

東星性儉約。歷官三十年,敝衣蔬食如一日。天啟初,謚莊靖。

胡瓚,字伯玉,桐城人。萬曆二十三年進士。授都水主事。分司南旺司兼督泉閘,駐濟寧。泗水所注,瓚修金口壩遏之。造舟汶上,為橋於寧陽,民不病涉。河決黃堌,瓚憂之。會劉東星來總河漕,瓚與往復論難。謂黃堌不杜,勢且易黃而漕;漕南北七百里,以涓涓之泉,安能運萬千有奇之艘,使及期飛渡?贊東星濬賈魯河故道,益治汶、泗間泉數百。尋源竟委,著《泉河史》上之。瓚治泉,一夫濬一泉,各有分地,省其勤惰而賞罰之。冬則養其餘力,不征於官。以疏浚運道有功,增秩一等。二十七年督修琉璃河橋。三年橋成,省費七萬有奇。累官江西左參政。予告歸,久之卒。

徐貞明,字孺東,貴溪人。父九思,見《循吏傳》。貞明舉隆慶五年進士。知浙江山陰縣,敏而有惠。萬曆三年,徵為工科給事中。會御史傅應禎獲罪,貞明入獄調護,坐貶太平府知事。十三年,累遷尚寶司丞。初,貞明為給事中,上水利、軍班二議,謂:

神京雄據上游,兵食宜取之畿甸,今皆仰給東南。豈西北古稱富強地,不足以實廩而練卒乎?夫賦稅所出,括民脂膏,而軍船夫役之費,常以數石致一石,東南之力竭矣。又河流多變,運道多梗,竊有隱憂。聞陜西、河南故渠廢堰,在在有之;山東諸泉,引之率可成田;而畿輔諸郡,或支河所經,或澗泉自出,皆足以資灌溉。北人未習水利,惟苦水害,不知水害未除,正由水利未興也。蓋水聚之則為害,散之則為利。今順天、真定、河間諸郡,桑麻之區,半為沮洳,由上流十五河之水惟泄於貓兒一灣,欲其不汎濫而壅塞,勢不能也。今誠於上流疏渠濬溝,引之灌田,以殺水勢,下流多開支河,以泄橫流,其澱之最下者,留以瀦水,稍高者,皆如南人築圩之制,則水利興,水患亦除矣。至於永平、灤州抵滄州、慶雲,地皆萑葦,土實膏腴。元虞集欲於京東濱海地築塘捍水以成稻田。若仿集意,招徠南人,俾之耕藝,北起遼海,南濱青齊,皆良田也。宜特簡憲臣,假以事權,毋阻浮議,需以歲月,不取近功。或撫窮民而給其牛種,或任富室而緩其徵科,或選擇健卒分建屯營,或招徠南人許其占籍。俟有成績,次及河南、山東、陜西。庶東南轉漕可減,西北儲蓄常充,國計永無絀矣。

其議軍班則言:

東南民素柔脆,莫任遠戍。今數千里勾軍,離其骨肉。而軍壯出於戶丁,幫解出於里甲,每軍不下百金。而軍非土著,志不久安,輒賂衛官求歸。衛官利其賂,且可以冒餉也,因而縱之。是困東南之民,而實無補於軍政也。宜仿匠班例,軍戶應出軍者,歲徵其錢,而召募土著以足之便。

事皆下所司。兵部尚書譚綸言勾軍之制不可廢。工部尚書郭朝賓則以水田勞民,請俟異日。事遂寢。及貞明被謫,至潞河,終以前議可行,乃著《潞水客談》以畢其說。其略曰:

西北之地,旱則赤地千里,潦則洪流萬頃,惟雨暘時若,庶樂歲無饑,此可常恃哉?惟水利興而後旱潦有備,利一。中人治生,必有常稔之田,以國家之全盛,獨待哺於東南,豈計之得哉?水利興則餘糧棲畝皆倉庾之積,利二。東南轉輸,其費數倍。若西北有一石之入,則東南省數石之輸,久則蠲租之詔可下,東南民力庶幾稍蘇,利三。西北無溝洫,故河水橫流,而民居多沒。修復水田,則可分河流,殺水患,利四。西北地平曠,寇騎得以長驅。若溝洫盡舉,則田野皆金湯,利五。游民輕去鄉土,易於為亂。水利興則業農者依田里,而游民有所歸,利六。招南人以耕西北之田,則民均而田亦均,利七。東南多漏役之民,西北罹重徭之苦,以南賦繁而役減,北賦省而徭重也。使田墾而民聚,則賦增而北徭可減,利八。沿邊諸鎮有積貯,轉輸不煩,利九。天下浮戶依富家為佃客者何限,募之為農而簡之為兵,屯政無不舉矣,利十。塞上之卒,土著者少。屯政舉則兵自足,可以省遠募之費,蘇班戍之勞,停攝勾之苦,利十一。宗祿浩繁,勢將難繼。今自中尉以下,量祿之田,使自食其土,為長子孫計,則宗祿可減,利十二。修復水利,則仿古井田,可限民名田。而自昔養民之政漸可舉行,利十三。民與地均,可仿古比閭族黨之制,而教化漸興,風俗自美,利十四也。

譚綸見而美之曰:「我歷塞上久,知其必可行也。」已而順天巡撫張國彥、副使顧養謙行之薊州、永平、豐潤、玉田,皆有效。及是貞明還朝,御史蘇瓚、徐待力言其說可行,而給事中王敬民又特疏論薦,帝乃進貞明少卿,賜之敕,令往會撫按諸臣勘議。

時瓚方奉命巡關,復獻議曰:「治水與墾田相濟,未有水不治而田可墾者。畿輔為患之水莫如盧溝、滹沱二河。盧溝發源於桑乾,滹沱發源於泰戲,源遠流長。又合深、易、濡、泡、沙、滋諸水,散入各澱,而泉渠溪港悉注其中。以故高橋、白洋諸澱,大者廣圍一二百里,小亦四五十里。每當夏秋淫潦,膏腴變為瀉鹵,菽麥化為萑葦,甚可惜也。今治水之策有三:浚河以決水之壅,疏渠以殺澱之勢,撤曲防以均民之利而已。」帝並下貞明。

貞明乃躬歷京東州縣,相原隰,度土宜,周覽水泉分合,條列事宜以上。戶部尚書畢鏘等力贊之,因採貞明疏,議為六事:請郡縣有司以墾田勤惰為殿最,聽貞明舉劾;地宜稻者以漸勸率,宜黍宜粟者如故,不遽責其成;召募南人,給衣食農具,俾以一教十;能墾田百畝以上,即為世業,子弟得寄籍入學,其卓有明效者,仿古孝弟力田科,量授鄉遂都鄙之長;墾荒無力者,貸以穀,秋成還官,旱潦則免;郡縣民壯,役止三月,使疏河芟草,而墾田則募專工。帝悉從之。其年九月,遂命貞明兼監察御史領墾田使,有司撓者劾治。

貞明先詣永平,募南人為倡。至明年二月,已墾至三萬九千餘畝。又遍歷諸河,窮源竟委,將大行疏濬。而奄人、勛戚之占閑田為業者,恐水田興而己失其利也,爭言不便,為蜚語聞於帝。帝惑之。三月,閣臣申時行等以風霾陳時政,力言其利。帝意終不釋。御史王之棟,畿輔人也,遂言水田必不可行,且陳開滹沱不便者十二。帝乃召見時行等,諭令停役。時行等請罷開河,專事墾田。已,工部議之棟疏,亦如閣臣言。帝卒罷之,而欲追罪建議者,用閣臣言而止。貞明乃還故官。尋乞假歸。十八年卒。

貞明識敏才練,慨然有經世志。京東水田實百世利,事初興而即為浮議所撓,論者惜之。初議時,吳人伍袁萃謂貞明曰:「民可使由,不可使知。君所言,得無太盡耶?」貞明問故。袁萃曰:「北人懼東南漕儲派於西北,煩言必起矣。」貞明默然。已而之棟竟劾奏如袁萃言。

袁萃,字聖起,吳縣人。舉萬曆五年會試。又三年釋褐,授貴溪知縣。擢兵部主事,進員外郎,署職方事。李成梁子如楨求為錦衣大帥,袁萃力爭,寢之。出為浙江提學僉事。巡撫牒數十人寄學,立卻還之。歷廣東海北道副使。中官李敬轄珠池,其參隨擅殺人,袁萃捕論如法。請告歸。所撰《林居漫錄》、《彈園雜志》多貶斥當世公卿大夫,而於李三才、於玉立尤甚云。

贊曰:事功之難立也,始則群疑朋興,繼而忌口交鑠,此勞臣任事者所為腐心也。盛應期諸人治漕營田,所規畫為軍國久遠大計,其奏效或在數十年後。而當其時浮議滋起,或以輟役,或以罷官,久之乃食其利而思其功。故曰:「可與樂成,難與慮始。」信夫!


\end{pinyinscope}