\article{列傳第一百六十二}

\begin{pinyinscope}
史可法任民育等何剛等高弘圖姜曰廣周鑣雷縯祚

史可法,字憲之,大興籍,祥符人。世錦衣百戶。祖應元舉於鄉,官黃平知州,有惠政。語其子從質曰:「我家必昌。」從質妻尹氏有身,夢文天祥入其舍,生可法。以孝聞。舉崇禎元年進士,授西安府推官,稍遷戶部主事,歷員外郎、郎中。

八年,遷右參議,分守池州、太平。其秋,總理侍郎盧象升大舉討賊。改可法副使,分巡安慶、池州,監江北諸軍。黃梅賊掠宿松、潛山、太湖,將犯安慶,可法追擊之潛山天堂寨。明年,祖寬破賊滁州,賊走河南。十二月,賊馬守應合羅汝才、李萬慶自鄖陽東下。可法馳駐太湖,扼其衝。十年正月,賊從間道突安慶石牌,尋移桐城。參將潘可大擊走賊,賊復為廬、風軍所扼,回桐城,掠四境。知縣陳爾銘嬰城守,可法與可大剿捕。賊走廬江,犯潛山,可法與左良玉敗之楓香驛,賊乃竄潛山、太湖山中。三月,可大及副將程龍敗歿於宿松。賊分其黨搖天動別為一營,而合八營二十餘萬眾,分屯桐城之練潭、石井、陶沖。總兵官牟文綬、劉良佐擊敗之掛車河。

當是時,陜寇聚漳、寧,分犯岷、洮、秦、楚、應、皖,群盜遍野。總理盧象升既改督宣、大,代以王家禎,祖寬關外兵亦北歸。未幾,上復以熊文燦代家禎,專撫賊。賊益狂逞,盤牙江北,南都震驚。七月擢可法右僉都御史,巡撫安慶、廬州、太平、池州四府,及河南之光州、光山、固始、羅田,湖廣之蘄州、廣濟、黃梅,江西之德化、湖口諸縣,提督軍務,設額兵萬人。賊已東陷和州、含山、定遠、六合,犯天長、盱眙,趨河南。可法奏免被災田租。冬,部將汪雲鳳敗賊潛山,京軍復連破老回回舒城、廬江,賊遁入山。時監軍僉事湯開遠善擊賊,可法東西馳禦,賊稍稍避其鋒。十一年夏,以平賊逾期,戴罪立功。

可法短小精悍,面黑,目爍爍有光。廉信,與下均勞苦。軍行,士不飽不先食,未授衣不先禦,以故得士死力。連敗賊英山、六合,順天王乞降。十二年夏,丁外艱去。服闋,起戶部右侍郎兼右僉都御史。代朱大典總督漕運,巡撫鳳陽、淮安、揚州,劾罷督糧道三人,增設漕儲道一人,大浚南河,漕政大釐。拜南京兵部尚書,參贊機務。因武備久弛,奏行更新八事。

十七年四月朔,聞賊犯闕,誓師勤王。渡江抵浦口,聞北都既陷,縞衣發喪。會南都議立君,張慎言、呂大器、姜曰廣等曰:「福王由崧,神宗孫也,倫序當立,而有七不可:貪、淫、酗酒、不孝、虐下、不讀書、干預有司也。潞王常言芳,神宗姪也,賢明當立。」移牒可法,可法亦以為然。鳳陽總督馬士英潛與阮大鋮計議,主立福王,咨可法,可法以七不可告之。而士英已與黃得功、劉良佐、劉澤清、高傑發兵送福王至儀真,於是可法等迎王。五月朔,王謁孝陵、奉先殿,出居內守備府。群臣入朝,王色赧欲避。可法曰:「王毋避,宜正受。」既朝,議戰守。可法曰:「王宜素服郊次,發師北征,示天下以必報仇之義。」王唯唯。明日再朝,出議監國事。張慎言曰:「國虛無人,可遂即大位。」可法曰:「太子存亡未卜,倘南來若何?」誠意伯劉孔昭曰:「今日既定,誰敢復更?」可法曰:「徐之。」乃退。又明日,王監國,廷推閣臣,眾舉可法、高弘圖、姜曰廣。孔昭攘臂欲並列,眾以本朝無勛臣入閣例,遏之。孔昭勃然曰:「即我不可,馬士英何不可?」乃并推士英。又議起廢,推鄭三俊、劉宗周、徐石麒。孔昭舉大鋮,可法曰:「先帝欽定逆案,毋復言。」越二日,拜可法禮部尚書兼東閣大學士,與士英、弘圖並命。可法仍掌兵部事,士英仍督師鳳陽。乃定京營制,如北都故事,侍衛及錦衣衛諸軍,悉入伍操練。錦衣東西兩司房,及南北兩鎮撫司官,不備設,以杜告密,安人心。

當是時,士英旦夕冀入相。及命下,大怒,以可法七不可書奏之王。而擁兵入覲,拜表即行。可法遂請督師,出鎮淮、揚。十五日,王即位。明日,可法陛辭,加太子太保,改兵部尚書、武英殿大學士。士英即以是日入直,議分江北為四鎮。東平伯劉澤清轄淮、海,駐淮北,經理山東一路。總兵官高傑轄徐、泗,駐泗水,經理開、歸一路。總兵官劉良佐轄鳳、壽,駐臨淮,經理陳、杞一路。靖南伯黃得功轄滁、和,駐廬州,經理光、固一路。可法啟行,即遣使訪大行帝后梓宮及太子二王所在,奉命祭告鳳、泗二陵。

可法去,士英、孔昭輩益無所憚。孔昭以慎言舉吳甡,嘩殿上,拔刀逐慎言。可法馳疏解,孔昭卒扼甡不用。可法祭二陵畢,上疏曰:「陛下踐阼初,祗謁孝陵,哭泣盡哀,道路感動。若躬謁二陵,親見泗、鳳蒿萊滿目,雞犬無聲,當益悲憤。願慎終如始,處深宮廣廈,則思東北諸陵魂魄之未安;享玉食大庖,則思東北諸陵麥飯之無展;膺圖受籙,則念先帝之集木馭朽,何以忽遘危亡;早朝晏罷,則念先帝之克儉克勤,何以卒隳大業。戰兢惕厲,無時怠荒,二祖列宗將默佑中興。若晏處東南,不思遠略,賢奸無辨,威斷不靈,老成投簪,豪傑裹足,祖宗怨恫,天命潛移,東南一隅未可保也。」王嘉答之。

得功、澤清、傑爭欲駐揚州。傑先至,大殺掠,屍橫野。城中恟懼,登陴守,傑攻之浹月。澤清亦大掠淮上。臨淮不納良佐軍,亦被攻。朝命可法往解,得功、良佐、澤清皆聽命。乃詣傑。傑素憚可法,可法來,傑夜掘坎十百,埋暴骸。旦日朝可法帳中,辭色俱變,汗浹背。可法坦懷待之,接偏裨以溫語,傑大喜過望。然傑亦自是易可法,用己甲士防衛,文檄必取視而後行。可法夷然為具疏,屯其眾於瓜洲,傑又大喜。傑去,揚州以安,可法乃開府揚州。

六月,大清兵擊敗賊李自成,自成棄京師西走。青州諸郡縣爭殺偽官,據城自保。可法請頒監國、登極二詔,慰山東、河北軍民心。開禮賢館,招四方才智,以監紀推官應廷吉領其事。八月出巡淮安,閱澤清士馬。返揚州,請餉為進取資。士英靳不發,可法疏趣之。因言:「邇者人才日耗,仕途日淆,由名心勝而實意不修,議論多而成功少。今事勢更非昔比,必專主討賊復仇。舍籌兵籌餉無議論,舍治兵治餉無人才。有摭拾浮談、巧營華要者,罰無赦!」王優詔答之。

初,可法虞傑跋扈,駐得功儀真防之。九月朔,得功、傑構兵,曲在傑。賴可法調劑,事得解。北都降賊諸臣南還,可法言:「諸臣原籍北土者,宜令赴吏、兵二部錄用,否則恐絕其南歸之心。」又言:「北都之變,凡屬臣子皆有罪。在北者應從死,豈在南者非人臣?即臣可法謬典南樞,臣士英叨任鳳督,未能悉東南甲疾趨北援,鎮臣澤清、傑以兵力不支,折而南走。是首應重論者,臣等罪也。乃因聖明繼統,金未鉞未加,恩榮疊被。而獨於在北諸臣毛舉而概繩之,豈散秩閒曹,責反重於南樞、鳳督哉。宜摘罪狀顯著者,重懲示儆。若偽命未污,身被刑辱,可置勿問。其逃避北方、俳徊而後至者,許戴罪討賊,赴臣軍前酌用。」廷議並從之。

傑居揚州,桀驁甚。可法開誠布公,導以君臣大義。傑大感悟,奉約束。十月,傑帥師北征。可法赴清江浦,遣官屯田開封,為經略中原計。諸鎮分汛地,自王家營而北至宿遷,最衝要,可法自任之,築壘緣河南岸。十一月四日,舟次鶴鎮,諜報我大清兵入宿遷。可法進至白洋河,令總兵官劉肇基往援。大清兵還攻邳州,肇基復援之,相持半月而解。

時自成既走陜西,猶未滅,可法請頒討賊詔書,言:

自三月以來,大仇在目,一矢未加。昔晉之東也,其君臣日圖中原,而僅保江左;宋之南也,其君臣盡力楚、蜀,而僅保臨安。蓋偏安者,恢復之退步,未有志在偏安,而遽能自立者也。大變之初,黔黎灑泣,紳士悲哀,猶有朝氣。今則兵驕餉絀,文恬武嬉,頓成暮氣矣。河上之防,百未經理,人心不肅,威令不行。復仇之師不聞及關、陜,討賊之詔不聞達燕、齊。君父之仇,置諸膜外。夫我即卑宮菲食,嘗膽臥薪,聚才智精神,枕戈待旦,合方州物力,破釜沉舟,尚虞無救。以臣觀廟堂謀畫,百執事經營,殊未盡然。夫將所以能克敵者,氣也;君所以能禦將者,志也。廟堂志不奮,則行間氣不鼓。夏少康不忘出竇之辱,漢光武不忘爇薪之時。臣願陛下為少康、光武,不願左右在位,僅以晉元、宋高之說進也。

先皇帝死於賊,恭皇帝亦死於賊,此千古未有之痛也。在北諸臣,死節者無多;在南諸臣,討賊者復少。此千古未有之恥也。庶民之家,父兄被殺,尚思穴胸斷豆,得而甘心,況在朝廷,顧可漠置。臣願陛下速發討賊之詔,責臣與諸鎮悉簡精銳,直指秦關,懸上爵以待有功,假便宜而責成效,絲綸之布,痛切淋漓,庶海內忠臣義士,聞而感憤也。

國家遘此大變,陛下嗣登大寶,與先朝不同。諸臣但有罪之當誅,曾無功之足錄。今恩外加恩未已,武臣腰玉,名器濫觴。自後宜慎重,務以爵祿待有功,庶猛將武夫有所激厲。兵行最苦無糧,搜括既不可行,勸輸亦難為繼。請將不急之工程,可已之繁費,朝夕之燕衎,左右之進獻,一切報罷。即事關典禮,亦宜概從節省。蓋賊一日未滅,即有深宮曲房,錦衣玉食,豈能安享!必刻刻在復仇雪恥,振舉朝之精神,萃萬方之物力,盡并於送將練兵一事,庶人心可鼓,天意可回。

可法每繕疏,循環諷誦,聲淚俱下,聞者無不感泣。

比大清兵已下邳、宿,可法飛章報。士英謂人曰:「渠欲敘防河將士功耳。」慢弗省。而諸鎮逡巡無進師意,且數相攻。明年,是為大清順治之二年,正月,餉缺,諸軍皆饑。頃之,河上告警。詔良佐、得功率師扼潁、壽,傑進兵歸、徐。傑至睢州,為許定國所殺。部下兵大亂,屠睢旁近二百里殆盡。變聞,可法流涕頓足嘆曰:「中原不可為矣。」遂如徐州,以總兵李本身為提督,統傑兵。本身者,傑甥也。以胡茂順為督師中軍,李成棟為徐州總兵,諸將各分地,又立傑子元爵為世子,請恤於朝。軍乃定。傑軍既還,於是大梁以南皆不守。土英忌可法威名,加故中允衛胤文兵部右侍郎,總督興平軍,以奪可法權。胤文,傑同鄉也,陷賊南還,傑請為己監軍。傑死,胤文承士英旨,疏誚可法。士英喜,故有是命,駐揚州。二月,可法還揚州。未至,得功來襲興平軍,城中大懼。可法遣官講解。乃引去。

時大兵已取山東、河南北,逼淮南。四月朔,可法移軍駐泗州,護祖陵。將行,左良玉稱兵犯闕,召可法入援。渡江抵燕子磯,得功已敗良玉軍。可法乃趨天長,檄諸將救盱眙。俄報盱眙已降大清,泗州援將侯方巖全軍沒。可法一日夜奔還揚州。訛傳定國兵將至,殲高氏部曲。城中人悉斬關出,舟楫一空。可法檄各鎮兵,無一至者。二十日,大清兵大至,屯班竹園。明日,總兵李棲鳳、監軍副使高岐鳳拔營出降,城中勢益單。諸文武分陴拒守。舊城西門險要,可法自守之。作書寄母妻,且曰:「死葬我高皇帝陵側。」越二日,大清兵薄城下,炮擊城西北隅,城遂破。可法自刎不殊,一參將擁可法出小東門,遂被執。可法大呼曰:「我史督師也。」遂殺之。揚州知府任民育,同知曲從直、王纘爵,江都知縣周志畏、羅伏龍,兩淮鹽運使楊振熙,監餉知縣吳道正,江都縣丞王志端,賞功副將汪思誠,幕客盧渭等皆死。

可法初以定策功加少保兼太子太保,以太后至加少傅兼太子太傅,敘江北戰功加少師兼太子太師,擒劇盜程繼孔功加太傅,皆力辭,不允。後以宮殿成,加太師,力辭,乃允。可法為督師,行不張蓋,食不重味,夏不箑,冬不裘,寢不解衣。年四十餘,無子,其妻欲置妾。太息曰:「王事方殷,敢為兒女計乎!」歲除遣文牒,至夜半,倦索酒。庖人報殽肉已分給將士,無可佐者,乃取鹽鼓下之。可法素善飲,數斗不亂,在軍中絕飲。是夕,進數十觥,思先帝,泫然淚下,憑几臥。比明,將士集轅門外,門不啟,左右遙語其故。知府民育曰:「相公此夕臥,不易得也。」命鼓人仍擊四鼓,戒左右毋驚相公。須臾,可法寤,聞鼓聲,大怒曰:「誰犯吾令!」將士述民育意,乃獲免。嘗孑處鈴閣或舟中,有言宜警備者,曰:「命在天。」可法死,覓其遺骸。天暑,眾屍蒸變,不可辯識。踰年,家人舉袍笏招魂,葬於揚州郭外之梅花領。其後四方弄兵者,多假其名號以行,故時謂可法不死云。

可法無子,遺命以副將史德威為之後。有弟可程,崇禎十六年進士。擢庶吉士。京師陷,降賊。賊敗,南歸,可法請置之理。王以可法故,令養母。可程遂居南京,後流寓宜興,閱四十年而卒。

任民育,字時澤,濟寧人。天啟中鄉舉,善騎射。真定巡撫徐標請於朝,用為贊畫,理屯事。真定失,南還。福王時,授亳州知州。以才擢揚州知府,可法倚之。城破,緋衣端坐堂上,遂見殺,闔家男婦盡赴井死。

從直,遼東人,與其子死東門。纘爵,鄞人,工部尚書佐孫。志畏,亦鄞人,進士,年少好氣,數遭傑將士窘辱,求解職。會伏龍至,可法命代之。伏龍,新喻人。故梓潼知縣,受代甫三日。振熙,臨海人。道正,餘姚人。志端,孝豐人。思誠,字純一,貴池人。

渭,字渭生,長洲諸生,可法出鎮淮、揚,謂等伏闕上書,言:「秦檜在內,李綱居外,宋終北轅。」不納。居禮賢館久,可法才渭。渭方歲貢,當得官,不受職,而擬授崑山歸昭等二十餘人為通判、推官、知縣。甫二旬,城陷,渭監守鈔關,投於河。昭死西門,從死者十七人。

時同守城死者,又有遵義知府何剛、庶吉士吳爾壎。而揚州諸生殉養者,有高孝纘、王士琇、王纘、王績、王續等。又有武生戴之籓、醫者陳天拔、畫士陸愉、義兵張有德、市民馮應昌、舟子徐某,並自盡。他婦女死節者,不可勝紀。

何剛,字愨人,上海人。崇禎三年舉於鄉。見海內大亂,慨然有濟世之志。交天下豪俊,與東陽許都善,語之曰:「子所居天下精兵處,盍練一旅以待用。」都諾而去。

十七年正月,入都上書,言:「國家設制科,立資格,以約束天下豪傑。此所以弭亂,非所以戡亂也。今日救生民,匡君國,莫急於治兵。陛下誠簡強壯英敏之士,命知兵大臣教習之,講韜鈐,練筋骨,拓膽智,時召而試之。學成優其秩,寄以兵柄,必能建奇功。臣讀戚繼光書,繼光數言義烏、東陽兵可用。誠得召募數千,加之訓練,準繼光遺法,分布河南郡縣,大寇可平。」因薦都及錢塘進士姚奇胤、桐城諸生周岐、陜西諸生劉湘客、絳州舉人韓霖。帝壯其言,即擢剛職方主事,募兵金華。而都作亂已前死,霖亦為賊用,剛不知,故並薦之。

剛出都,都城陷,馳還南京。先是,賊逼京師,剛友陳子龍、夏允彞將聯海舟達天津,備緩急,募卒二千人,至是令剛統之。子龍入為兵科,言防江莫如水師,更乞廣行召募,委剛訓練,從之。剛乃上疏言:「臣請陛下三年之內,宮室不必修,百官禮樂不必備。惟日救天下才,智者決策,廉者理財,勇者禦敵。爵賞無出此三者,則國富兵強,大敵可服。若以驕悍之將馭無制之兵,空言恢復,是卻行而求前也。優游歲月,潤色偏安,錮豪傑於草間,迫梟雄為盜賊,是株守以待盡也。惟廟堂不以浮文取士,而以實績課人,則真才皆為國用,而議論亦省矣。分遣使者羅草澤英豪,得才多者受上賞,則梟傑皆畢命封疆,而盜魁亦少矣。東南人滿,徙之江北,或賜爵,或贖罪,則豪右皆盡力南畝,而軍餉亦充矣。」時不能用。

尋進本司員外郎,以其兵隸史可法。可法大喜得剛,剛亦自喜遇可法知已。士英惡之,出剛遵義知府。可法垂涕曰:「子去,吾誰仗?」剛亦泣,願死生無相背。踰月,揚州被圍,佐可法拒守。城破,投井死。

吳爾壎,崇德人。崇禎十六年進士,授庶吉士。賊敗南還,謁可法,請從軍贖罪,可法遂留參軍事。其父子屏方督學福建,爾壎斷一指畀故人祝淵曰:「君歸語我父母,悉出私財畀我餉軍。我他日不歸,以指葬可也。」從高傑北征至睢州,傑被難,爾壎流寓祥符。遇一婦人,自言福王妃。爾壎因守臣附疏以進,詔斥其妄言,逮之,可法為救免。後守揚州新城,投井死。

高弘圖,字研文,膠州人。萬曆三十八年進士。授中書舍人,擢御史。柧棱自持,不依麗人。天啟初,陳時政八患,請用鄒元標、趙南星。巡按陜西,題薦屬吏,趙南星糾之,弘圖不能無望,代還,移疾去。魏忠賢亟攻東林,其黨以弘圖嘗與南星有隙,召起弘圖故官。入都,則楊漣、左光斗、魏大中等已下詔獄,鍛煉嚴酷。弘圖果疏論南星,然言「國是已明,雷霆不宜頻擊」,「詔獄諸臣,生殺宜聽司敗法」,則頗謂忠賢過當者。疏中又引漢元帝乘船事,忠賢方導帝遊幸,不悅,矯旨切責之。後諫帝毋出蹕東郊,又極論前陜西巡撫喬應甲罪,又嘗語刺崔呈秀。呈秀、應甲皆忠賢黨,由是忠賢大怒,擬順天巡按不用。弘圖乞歸,遂令閒住。

莊烈帝即位,起故官。劾罪田詔、劉志選、梁夢環。擢太僕少卿,復移疾去。三年春,召拜左僉都御史,進左副都御史。五年遷工部右侍郎。方入署,總理戶、工二部中官張彞憲來會,弘圖恥之,不與共坐,七疏乞休。帝怒,遂削籍歸,家居十年不起。

十六年,召拜南京兵部右侍郎,就遷戶部尚書。明年三月,京師陷,福王立,改弘圖禮部尚書兼東閣大學士。疏陳新政八事。一,宣義問。請聲逆賊之罪,鼓發忠義。一,勤聖學。請不俟釋服,日御講筵。一,設記注。請召詞臣入侍,日記言動。一,睦親籓。請如先朝踐極故事,遣官齎璽書慰問。一,議廟祀。請權附列聖神主於奉先殿,仍於孝陵側望祀列聖山陵。一,嚴章奏。請禁奸宄小人借端妄言,脫罪僥倖。一,收人心。請蠲江北、河南、山東田租,毋使賊徒藉口。一,擇詔使。請遣官招諭朝鮮,示牽制之勢。並褒納焉。

當是時,朝廷大議多出弘圖手。馬士英疏薦阮大鋮,弘圖不可。士英曰:「我自任之。」乃命大鋮假冠帶陛見。大鋮入見,歷陳冤狀,以弘圖不附東林引為證。弘圖則力言逆案不可翻,大鋮、士英並怒。一日,閣中語及故庶吉士張溥,士英曰:「我故人也,死,酹而哭之。」姜曰廣笑曰:「公哭東林者,亦東林耶?」士英曰:「我非畔東林者,東林拒我耳。」弘圖因縱臾之,士英意解。而劉宗周劾疏自外至,大鋮宣言曰廣實使之,於是士英怒不可止。而薦張捷、謝升之疏出,朝端益水火矣。內札用戶部侍郎張有譽為尚書,弘圖封還,具奏力諫,卒以廷推簡用。中官議設東廠,弘圖爭不得。遂乞休,不許,加太子少師,改戶部尚書,文淵閣。尋以太后至,進太子太保。

其年十月,弘圖四疏乞休,乃許之。弘圖既謝政,無家可歸,流寓會稽。國破,逃野寺中,絕粒而卒。

姜曰廣,字居之,新建人。萬歷末,舉進士,授庶吉士,進編修。天啟六年奉使朝鮮,不攜中國一物往,不取朝鮮一錢歸,朝鮮人為立懷潔之碑。明年夏,魏忠賢黨以曰廣東林,削其藉。崇禎初,起右中充。九年,積官至吏部右侍郎。坐事左遷南京太常卿,遂引疾去。十五年,起詹事,掌南京翰林院。莊烈帝嘗言:「曰廣在講筵,言詞激切,朕知其人。」每優容之。

北都變聞,諸大臣議所立。曰廣、呂大器用周鑣、雷縯祚言,主立潞王,而諸帥奉福籓至江上。於是文武官並集內官宅,韓贊周令各署名籍。曰廣曰:「無遽,請祭告奉先殿而後行。」明日至奉先殿,諸勛臣語侵史可法,曰廣呵之,於是群小咸目攝曰廣。廷推閣臣,以曰廣異議不用,用史可法、高弘圖、馬士英。及再推詞臣,以王鐸、陳子壯、黃道周名上,而首曰廣。乃改曰廣禮部尚書兼東閣大學士,與鐸並命。鐸未至,可法督師揚州,曰廣與弘圖協心輔政。而士英挾擁戴功,內結勛臣硃國弼、劉孔昭、趙之龍,外連諸鎮劉澤清、劉良佐等,謀擅朝權,深忌曰廣。

未幾,士英特薦起阮大鋮。曰廣力爭不得,遂乞休,言:

前見文武交競,既慚無術調和;近睹逆案忽翻,又愧不能寢弭。遂棄先帝十七年之定力,反陛下數日前之明詔。臣請以前事言之。臣觀先帝之善政雖多,而以堅持逆案為尤美;先帝之害政間有,而以頻出口宣為亂階。用閣臣內傳矣,用部臣勳臣內傳矣,用大將用言官內傳矣。而所得閣臣,則淫貪巧猾之周延儒也,逢君朘民奸險刻毒之溫體仁、楊嗣昌也,偷生從賊之魏藻德也;所得部臣,則陰邪貪狡之王永光、陳新甲;所得勛臣,則力阻南遷盡撤守禦狂稚之李國禎;所得大將,則紈褲支離之王樸、。倪寵;所得言官,則貪橫無賴之史褷、陳啟新也。凡此皆力排眾議,簡目中旨,後效可睹。

今又不然。不必僉同,但求面對,立談取官。陰奪會推之柄,陽避中旨之名,決廉恥之大防,長便佞之惡習。此豈可訓哉!

臣待罪綸扉,茍好盡言,終蹈不測之禍。聊取充位,又來鮮恥之譏。願乞骸骨還鄉里。

得旨慰留,士英、大鋮等滋不悅。國弼、孔昭遂以誹謗先帝,誣蔑忠臣李國禎為言,交章攻之。

劉澤清故附東林,擁立議起,亦主潞王。至是入朝,則力詆東林以自解免。且曰:「中興所恃在政府。今用輔臣,宜令大帥僉議。」曰廣愕然。越數日,澤清疏劾呂大器、雷縯祚,而薦張捷、鄒之麟、張孫振、劉光斗等。已,又請免故輔周延儒臟。曰廣曰:「是欲漸干朝政也。」乃下部議,竟不許。

曰廣嘗與士英交詆王前。宗室硃統金類者,素無行,士英啖以官,使擊曰廣。澤清又假諸鎮疏攻劉宗周及曰廣,以三案舊事及迎立異議為言,請執下法司,正謀危君父之罪。頃之,統金類復劾曰廣五大罪,請並劉士楨、王重、楊廷麟、劉宗周、陳必謙、周鑣、雷縯祚置之理,必謙、鑣以是逮。曰廣既連遭誣蔑,屢疏乞休,其年九月始得請。入辭,諸大臣在列。曰廣曰:「微臣觸忤權奸,自分萬死,上恩寬大,猶許歸田。臣歸後,願陛下以國事為重。」士英熟視曰廣,詈曰:「我權奸,汝且老而賊也。」既出,復於朝堂相詬詈而罷。

曰廣骨鯁,扼於憸邪,不竟其用,遂歸。其後左良玉部將金聲桓者,已降於我大清,既而反江西,迎曰廣以資號召。聲桓敗,曰廣投偰家池死。

周鑣,字仲馭,金壇人。父秦峙,雲南布政使。鑣舉鄉試第一,崇禎元年成進士,授南京戶部主事,榷稅蕪湖。憂歸,服闋,授南京禮部主事。極論內臣言官二事,言:「張彞憲用而高弘圖、金鉉罷,王坤用而魏呈潤、趙東曦斥,鄧希詔用而曹文衡罷閑,王弘祖、李曰輔、熊開元罹罪。每讀邸報,半屬內侍溫綸。自今鍛煉臣子,委褻天言,祗徇中貴之心,臣不知何所極也。言官言出禍隨,黃道周諸臣薦賢不效,而惠世揚、劉宗周勿獲進;華允誠諸臣驅奸無濟,而陳于廷、姚希孟、鄭三俊皆蒙譴。每奉嚴諭,率皆直臣封章。自今播棄忠良,獎成宵小,祗快奸人之計,臣益不知何所極矣。」帝怒斥為民,鑣由是名聞天下。

初,鑣世父尚書應秋、叔父御史維持,以附魏忠賢並麗逆案,鑣恥之。通籍後,即交東林,矯矯樹名節。及被放,與宣城沈壽民讀書茅山,廷臣多論薦之。十五年起禮部主事,進郎中,為吏部尚書鄭三俊所倚。然為人好名,頗飾偽,給事中韓如愈疏論之,罷歸。

福王立於南京。馬士英既逐呂大器,以鑣及雷縯祚曾立潞王議,令朱統金類劾曰廣,因言鑣、縯祚等皆曰廣私黨,請悉置於理,遂令逮治。而士英劾鑣從弟鐘從逆,並及鑣。鐘亦逮治。阮大鋮居金陵時,諸生顧杲等出《留都防亂公揭》討之,主之者鑣也,大鋮以故恨鑣。鑣獄急,屬御史陳丹衷求解於士英,為緝事者所獲,丹衷出為長沙知州。於是察處御史羅萬爵希大鋮指,上疏痛詆鑣。而光祿卿祁逢吉,鑣同邑人,見人輒詈鑣,遂得為戶部侍郎。亡何,左良玉稱兵檄討士英罪,言引用大鋮,構陷鑣、縯祚,鍛煉周內。士英、大鋮益怒。大鋮謂鑣實召良玉兵,王乃賜鑣、縯祚自盡,鐘棄市。

雷縯祚,太湖人。崇禎三年舉於鄉。十三年夏,帝思破格用人,而考選止及進士,特命舉人貢生就試教職者,悉用為部寺司屬推官知縣,凡二百六十三人,號為庚辰特用。而縯祚得刑部主事。明年三月劾楊嗣昌六大罪可斬,鳳陽總督朱大典、安慶巡撫鄭二陽、河南巡撫高名衡、山東巡撫王公弼宜急易,帝不悅。十五年擢武德道兵備僉事。山東被兵,縯祚守德州,有詔獎勵。乃疏劾督師范志完縱兵淫掠,折除軍餉,構結大黨。帝心善其言,以淫掠事責兵部,而令糸寅祚再陳。志完者,首輔周延儒門生也,縯祚意有所忌,久不奏。

明年五月,延儒下廷議,寅祚乃奏言:「志完兩載僉事,驟陟督師,非有大黨,何以至是。大僚則尚書范景文等,詞林則諭德方拱乾等,言路則給事中朱徽、沈胤培、袁彭年等,皆其黨也。方德州被攻,不克去,掠臨清,又五日,志完始至。聞後部破景州,則大懼,欲避入德州城。漏三下,邀臣議。臣不聽,志完乃偕流寓詞臣拱乾見臣南城古廟。臣告以督師非入城官,薊州失事,由降丁內潰,志完不懌而去。若夫座主當朝,罔利曲庇,隻手有燎原之勢,片語操生死之權,稱功頌德,遍於班聯。臣不忍見陛下以周、召待大臣,而大臣以嚴嵩、傅國觀自待也。臣外籓小吏,乙榜孤蹤,不言不敢,盡言不敢,感陛下虛懷俯納,故不避首輔延儒與舉國媚附時局,略進一言。至中樞主計請餉必餽常例,天下共知,他乾沒更無算。」

疏入,帝益心動。命議舊計臣李待問、傅淑訓,樞臣張國維及戶科荊永祚,兵科沈迅、張嘉言罪,而召縯祚陛見。越數日,抵京。又數日入封,召志完、拱乾質前疏中語,拱乾為志完辨,帝頷之。問縯祚稱功頌德者誰,對曰:「延儒招權納賄,如起廢、清獄、蠲租,皆自居為功。考選臺諫,盡收門下。凡求總兵巡撫者,必先賄幕客董廷獻。」帝怒,逮廷獻,誅志完,而令縯祚還任。糸寅祚尋以憂去。

福王時,統金類劾曰廣,因及之,遂逮治。明年四月與鑣同賜自盡。故事,小臣無賜自盡者。因良玉兵東下,故大鋮輩急殺之。

贊曰:史可法憫國步多艱,忠義奮發,提兵江滸,以當南北之沖,四鎮棋布,聯絡聲援,力圖興復。然而天方降割,權臣掣肘於內,悍將跋扈於外,遂致兵頓餉竭,疆圉曰蹙,孤城不保,志決身殲,亦可悲矣!高弘圖、姜曰廣皆蘊忠謀,協心戮力,而扼於權奸,不安其位。蓋明祚傾移,固非區區一二人之所能挽也。


\end{pinyinscope}