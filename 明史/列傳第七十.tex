\article{列傳第七十}

\begin{pinyinscope}
○王恕子承裕馬文升劉大夏

王恕,字宗貫,三原人。正統十三年進士。由庶吉士授大理左評事,進左寺副。嘗條刑罰不中者六事,皆議行之。遷揚州知府,發粟振饑不待報,作資政書院以課士。天順四年以治行最,超遷江西右布政使,平贛州寇。憲宗嗣位,詔大臣嚴核天下方面官,乃黜河南左布政使侯臣等十三人,而以恕代臣。

成化元年,南陽、荊、襄流民嘯聚為亂,擢恕右副都御史撫治之。會丁母憂,詔奔喪兩月即起視事。恕辭,不許。與尚書白圭共平大盜劉通,復討破其黨石龍。嚴束所部毋濫殺,流民復業。移撫河南。論功,進左副都御史,稍遷南京刑部右侍郎。父憂,服除,以原官總督河道。浚高郵、邵伯諸湖,修雷公、上下句城、陳公四塘水閘。因災變,請講求弭災策。帝為賜山東租一年,畿輔亦多減免。旋改南京戶部左侍郎。

十二年,大學士商輅等以雲南遠在萬里,西控諸夷,南接交阯,而鎮守中官錢能貪恣甚,議遣大臣有威望者為巡撫鎮壓之,乃改恕左副都御史以行,就進右都御史。初,能遣指揮郭景奏事京師,言安南捕盜兵闌入雲南境,帝即命景齎敕戒約之。舊制,使安南必由廣西,而景直自雲南往。能因景遺安南王黎灝玉帶、寶絳、蟒衣、珍奇諸物。灝遣將率兵送景還,欲遂通雲南道。景懼後禍,紿先行白守關者。因脫歸,揚言安南寇至,關吏戒嚴。黔國公沐琮遣人諭其帥,始返。而諸臣畏能,匿不奏。能又頻遣景及指揮盧安、蘇本等交通乾崖、孟密諸土官,納其金寶無算。恕皆廉得之。遣騎執景,景懼自殺,因劾能私通外國,罪當死。詔遣刑部郎中潘蕃往按之。能又以其間,驛進黃鸚鵡。恕請禁絕,且盡發能貪暴狀,言:「昔交阯以鎮守非人,致一方陷沒。今日之事殆又甚焉。陛下何惜一能,不以安邊徼。」能大懼,急屬貴近請召恕還。而是時商輅、項忠諸正人方以忤汪直罷,遂改恕掌南京都察院,參贊守備機務。能事立解,籓勘上得實,置不問。

恕居雲南九月,威行徼外,黔國以下咸惕息奉令。疏凡二十上,直聲動天下。當是時,安南納江西叛人王姓者為謀主,潛遣諜入臨安,又於蒙自市銅鑄兵器,將伺間襲雲南。恕請增設副使二員,以飭邊備,謀遂沮。

還南京數月,遷兵部尚書,參贊如故。考選官屬,嚴拒請託,同事者咸不悅。而錢能歸,屢譖恕於帝。帝亦銜恕數直言,遂命兼右副都御史巡撫南畿。舊制,應天、鎮江、太平、寧國、廣德官田征半租,民田全免。其後,民田率歸豪右,而官田累貧民。恕乃量減官田耗,稍增之民田。常州時有羨米,乃奏以六萬石補夏稅,又補他府戶口鹽鈔六百萬貫,公私便焉。所部水災,奏免秋糧六十餘萬石。周行振貸,全活二百餘萬口。江南歲輸白糧,民多至破產,而光祿概以給庖人、賤工。又中官暴橫,四方輸上供物,監收者率要羨入。織造繒採及採花卉禽鳥者,絡繹道路。恕先後論列,皆不納。

中官王敬挾妖人千戶王臣南行採藥物、珍玩,所至騷然,長吏多被辱。至蘇州,召諸生寫妖書,諸生大嘩。敬奏諸生抗命。恕亟疏言:「當此凶歲,宜遣使振濟,顧乃橫索玩好。昔唐太宗諷梁州獻名鷹,明皇令益州織半臂褙子,進琵琶桿撥鏤牙合子諸物,李大亮、蘇頲不奉詔。臣雖無似,有慕斯人。」因盡列敬等罪狀。敬亦誣奏恕并及常州知府孫仁,仁被逮。仁,新淦人,由進士歷知府,為人方峻,敬至不為禮,以是見忤。恕抗章救,三疏劾敬。會中官尚銘亦發敬奸狀,乃下敬等獄,戍其黨十九人,而棄臣市,傳首南京。仁亦得釋歸,後積官至巡撫寧夏右副都御史。

二十年復改恕南京兵部尚書。時錢能亦守備南京,語人曰:「王公,天人也,吾敬事而已。」恕坦懷待之,能卒斂戢。林俊之下獄也,恕言:「天地止一壇,祖宗止一廟,而佛至千餘寺。一寺立,而移民居且數百家,費內帑且數十萬,此舛也。俊言當,不宜罪。」帝得疏不懌。恕侃侃論列無少避。先後應詔陳言者二十一,建白者三十九,皆力阻權倖。天下傾心慕之,遇朝事有不可,必曰:「王公胡不言也?」則又曰:「公疏且至矣」,已,恕疏果至。時為謠曰:「兩京十二部,獨有一王恕。」於是貴近皆側目,帝亦頗厭苦之。

二十二年起用傳奉官,恕諫尤切,帝愈不悅。恕先加太子少保,會南京兵部侍郎馬顯乞罷,忽附批落恕宮保致仕,朝野大駭。恕數為巡撫,歷侍郎至尚書,皆在留都。以好直言,終不得立朝。既歸,名益高,臺省推薦無虛月。工部主事仙居王純比恕汲黯,至予杖,謫思南推官。

孝宗即位,始用廷臣薦,召入為吏部尚書,尋加太子太保。先是,中外劾大學士劉吉者,必薦恕,吉以是大恚。凡恕所推舉,必陰撓之。弘治元年閏正月,言官劾兩廣總督宋旻、漕運總督邱鼐等三十七人宜降黜,中多素有時望者。吉竟取中旨允之,章不下吏部。恕以不得其職,拜疏乞去,不許。陜西缺巡撫,恕推河南布政使蕭禎。詔別推,恕執奏曰:「陛下不以臣不肖,任臣銓部。倘所舉不效,臣罪也。今陛下安知禎不才而拒之?是必左右近臣意有所屬。臣不能承望風指,以固祿位。且陛下既以禎為不可用,是臣不可用也,願乞骸骨。」帝乃卒用禎。

時言官多稱恕賢且老,不當任劇職,宜置內閣參大政。最後,南京御史吳泰等復言之。帝曰:「朕用蹇義、王直故事,官恕吏部,有謀議未嘗不聽,何必內閣也。」恕嘗侍經筵,見帝困熱暑,請依故事大寒暑暫停,仍進講義於宮中。進士董傑、御史湯鼐、給事中韓重等遂交章論駁,恕待罪請解職,優詔不許。恕上言:「臣蒙國厚恩,日夕思報。人見陛下任臣過重,遂望臣太深,欲臣盡取朝政更張之,如宋司馬光故事。無論臣才遠不及光,即今亦豈元祐時。且六卿分職,各有攸司,臣豈敢越而謀之。但傑等責臣良是,臣無所逃罪,惟乞放還。」。帝復優詔勉留之。恕感激眷遇,益以身任國事。方以疾在告,聞帝頗擢用宦官,至有賜蟒衣給莊田者,具疏切諫。中官黃順請起復匠官潘俊供役,恕言不可以小臣壞重典。再執奏,竟報許。

劉吉既憾恕,吉所陷壽州知州劉概及言官周紘、張昺、湯鼐、姜綰等,恕又抗章力救,吉以是益恨,乃合私人魏璋等共排之。恕先後推用羅明、熊懷、強珍、陳壽、邱鼐、白思明等,咸諷璋等糾駁。恕知志不得行,連章求去。帝輒慰留,且以其老特免午朝,遇大風雨雪,早朝亦免。

徽王見沛乞歸德州田,已得旨。恕言王國懿親,不當爭尺寸地,使小民失業,帝婉辭報焉。盧溝橋成,中官李興乞進文思院副使潘俊等官。恕言:「營造常職,安得錄功。成化季始有此事,陛下初政幸已革汰,奈何復行?且山陵大工未聞陞職,援例奏乞,將何詞拒之?」帝納其言。已,修京城河橋,帝復從興請授四人官,許五人冠帶。恕執奏,不從,再疏爭曰:「臣職掌銓選,義當盡言,而再疏莫回天聽,以為業已許之不可易。夫事求其當,設未當,雖十易何害。不然,流弊有不可救者。」報聞。先後以災異條七事,以星變陳二十事,咸切時弊。壽寧伯張巒請勳號、誥券。恕言:「錢、王兩太后正位中宮數十年,錢承宗、王源始邀封爵。今皇后立甫三年,巒已封伯。遽有此請,累聖德,不可許。」通政經歷高祿,巒妹婿也,超遷本司參議。恕言:「天下之官以待天下之士,勿私貴戚,妨公議。」中旨以次等御醫徐生超補院判,恕請選上考者,不納。文華殿中書舍人杜昌等夤緣遷秩,御醫王玉自陳乞進官,恕皆力爭寢之。

是時劉吉已罷,而邱濬入閣,亦與恕不相能。初,濬以禮部尚書掌詹事,與恕同為太子太保。恕長六卿,位濬上。及濬入閣,恕以吏部弗讓也,濬由是不悅。恕考察天下庶官,已黜而濬調旨留之者九十餘人。恕屢爭不能得,因力求罷,不許。太醫院判劉文泰者,故往來濬家以救遷官,為恕所沮,銜恕甚。恕里居日,嘗屬人作傳,鏤板以行。濬謂其沽直謗君,上聞罪且不小。文泰心動,乃自為奏草,示除名都御史吳禎潤色之。訐恕變亂選法。且傳中自比伊、周,於奏疏留中者,概云不報。以彰先帝拒諫,無人臣禮。欲中以奇禍。恕以奏出濬指,抗言:「臣傳作於成化二十年,致仕在二十二年,非有望於先帝也。且傳中所載,皆足昭先帝納諫之美,何名彰過。文泰無賴小人,此必有老於文學多陰謀者主之。」帝下文泰錦衣獄,鞫之得實,因請逮濬、恕及禎對簿。帝心不悅恕,乃貶文泰御醫。責恕沽名,焚所鏤版。置濬不問。恕再疏請辨理,不從,遂力求去。聽馳驛歸,不賜敕,月廩、歲隸亦頗減。廷論以是不直濬。及濬卒,文泰往弔,濬妻叱之出曰:「以若故,使相公齮王公,負不義名,何弔為!」恕揚歷中外四十餘年,剛正清嚴,始終一致。所引薦耿裕、彭韶、何喬新、周經、李敏、張悅、倪岳、劉大夏、戴珊、章懋等,皆一時名臣。他賢才久廢草澤者,拔擢之恐後。弘治二十年間,眾正盈朝,職業修理,號為極盛者,恕力也。武宗嗣位,遣行人齎敕存問,賚羊酒,益廩隸,且諭以讜論無隱。恕陳國家大政數事,帝優詔報之。正德三年四月卒,年九十三。平居食啖兼人,卒之日小減。閉戶獨坐,忽有聲若雷,白氣瀰漫,瞰之瞑矣。訃聞,輟朝,贈特進左柱國太師,謚端毅。五子、十三孫,多賢且顯。

少子承裕,字天宇。七歲能詩,弱冠著《太極動靜圖說》。恕官吏部,令日接賓客,以是周知天下賢才,選用無不當。舉弘治六年進士。恕致政,承裕即告歸侍養。起授兵科給事中,出理山東、河南屯田。減登、萊糧額,三畝徵一斗,還青州、彰德軍田先賜王府者三百六十餘頃。武宗立,屢遷吏科都給事中。以言事忤劉瑾,罰米輸塞上。再遷太僕卿。嘉靖六年累官南京戶部尚書。清逋稅一百七十萬石,積羨銀四萬八千餘兩。帝手書「清平正直」褒之。在部三年,致仕,卒。贈太子少保,謚康僖。

馬文升,字負圖,鈞州人。貌瑰奇多力。登景泰二年進士,授御史。歷按山西、湖廣,風裁甚著。還領諸道章奏。母喪除,超遷福建按察使。成化初,召為南京大理卿,以父喪歸。

滿四之亂,陜西巡撫陳價下吏,即家起文升右副都御史代價。馳至軍,與總督項忠討平之。事具忠傳。錄功進左副都御史,巡撫如故。文升數條奏便宜,務選將練兵,修安邊營至鐵鞭城烽堠,剪除劇賊。西固番族不即命者悉滅之。修茶政,易番馬八千有奇,以給士卒。振鞏昌、臨洮饑民,撫安流移。績甚著。是時,孛羅忽、滿都魯、加思蘭比歲犯邊。文升請駐兵韋州,而設伏諸堡待之。遂敗寇黑水口,擒其平章迭烈孫,又敗之湯羊嶺,斬首二百,名其嶺曰:「得勝坡」,勒石紀之而還。文升軍功甚盛,奏捷不為誇張,中亦無主之者,以是賞薄。至九年冬,總制王越以大捷奏,文升亦遣子琇報功。廷臣勘奏不實,坐停俸三月。

十一年春,代越總制三邊軍務,尋入為兵部右侍郎。明年八月,整飭遼東軍務。巡撫陳鉞貪而狡,將士小過輒罰馬,馬價騰踴。文升上邊計十五事,因請禁之,鉞由是嗛文升。文升還部轉左。十四年春,鉞以掩殺冒功激變,中官汪直欲自往定之。帝令司禮太監懷恩等七人詣內閣會兵部議。恩欲遣大臣往撫,以沮直行。文升疾應曰:「善。」恩入白,帝即命文升往。直不悅,欲令其私人王英與俱,文升謝絕之。疾馳至鎮,宣璽書撫慰,無不聽撫者。又請前以也先亂失授官璽書者十餘人,得襲官。事定,直欲攘其功,請於帝,挾王英馳至開原,再下令招撫。文升乃推功與直,然直內慚。文升又與直抗禮,奴視其左右,直益不喜。而陳鉞益諂事直,得直懽。日夜譖文升,思中之未有以發也。文升還,賜牢醴。明年春,以遼東屢失事,遣直偕定西侯蔣琬、尚書林聰等按之。會餘子俊劾鉞,鉞疑出文升意,傾之益急。直因奏文升行事乖方,禁邊人市農器,致怨叛。乃下文升詔獄,謫戍重慶衛。直既傾文升,則與鉞大發兵激功,鉞以是驟遷至尚書。

十九年,直敗,文升復官。明年起為左副都御史巡撫遼東。文升凡三至遼,軍民聞其來皆鼓舞。益禁抑中官、總兵,使不得朘削,眾益大喜。

二十一年進右都御史,總督漕運。淮、徐、和饑,移江南糧十萬石、鹽價銀五萬兩振之。是年冬,召為兵部尚書。明年,以李孜省譖,調南京。

孝宗即位,召拜左都御史。弘治元年上言:「憲宗朝,岳鎮海瀆諸廟,用方士言置石函,周以符篆,貯金書道經、金銀錢、寶石及五穀為厭勝具,宜毀。」從之。又上言十五事,悉議行。帝耕藉田,教坊以雜戲進。文升正色曰:「新天子當使知稼穡艱難,此何為者?」即斥去。御史徐瑁、賀霖失承旨下獄。文升言初政不宜輒罪言官,遂得釋。尋命提督十二團營。

明年,代餘子俊為兵部尚書,督團營如故。承平既久,兵政廢弛,西北部落時伺塞下。文升嚴核諸將校,黜貪懦者三十餘人。奸人大怨,夜持弓矢伺其門,或作謗書射入東長安門內。帝聞,詔錦衣緝捕,給騎士十二,衛文升出入。文升乞休,優詔不許。

小王子以數萬騎牧大同塞下,勢洶洶。文升以疾在告,帝使中官挾醫視,因就問計。文升謂「彼方敗於他部,無能為。請密為備,而揚聲逼之,必徙去。」已而果然。遭繼母憂,詔起復,再疏辭,不許。西北別部野乜克力,其長曰亦剌思王,曰滿哥王,曰亦剌因王,各遣使款肅州塞,乞貢且互市。巡撫許進、總兵官劉寧為請,文升言互市可許,入貢不可許,乃卻之。

土魯番既襲執陜巴,而令牙蘭據守哈密,僭稱可汗,侵沙州,迫罕東諸部附己。文升議,此寇桀驁,不大創終不知畏,宜用漢陳湯故事襲斬之。察指揮楊翥熟番情,召詢以方略。翥備陳罕東至哈密道路,請調罕東兵三千為前鋒,漢兵三千繼之,持數日糧,間道兼程進,可得志。文升喜,遂請於帝,敕發罕東、赤斤、哈密兵,令副總兵彭清將之,隸巡撫許進往討,果克之,語詳《進傳》。

團營軍不足,請於錦衣及騰驤四衛中選補。已得請矣,中官寧瑾阻之。文升及兵科蔚春等言詔旨宜信,不納。陜西地大震。文升言:「此外寇侵凌之兆。今火篩方跳梁,而海內民困財竭,將懦兵弱。宜行仁政以養民,講武備以固圉。節財用,停齋醮,止傳奉冗員,禁奏乞閒地。日視二朝,以勤庶政。且撤還陜西織造內臣,振恤被災者家。」帝納其言,內臣立召還。

文升為兵部十三年,盡心戎務,於屯田、馬政、邊備、守禦,數條上便宜。國家事當言者,即非職守,亦言無不盡。嘗以太子年及四齡,當早諭教。請擇醇謹老成知書史如衛聖楊夫人者,保抱扶持,凡言語動止悉導之以正。若內庭曲宴,鐘鼓司承應,元宵鰲山,端午競渡諸戲,皆勿令見。至於佛、老之教,尤宜屏絕,恐惑眩心志。山東久旱,浙江及南畿水災,文升請命所司振恤,練士卒以備不虞。帝皆深納之。民困賦役,文升極陳其害,謂:「今民田十稅四五,其輸邊塞者糧一石費銀一兩以上,豐年用糧八九石方易一兩。若絲綿布帛之輸京師者,交納之費過於所輸,南方轉漕通州至有三四石致一石者。中州歲役五六萬人治河,山東、河南修塞決口夫不下二十萬,蘇、松治水亦如之。湖廣建吉、興、岐、雍四王府,江西益、壽二府,山東衡府,通計役夫不下百萬。諸王之國役夫供應亦四十萬。加以湖廣征蠻,山、陜防邊,供饋餉給軍旅者又不知凡幾。賦重役繁,未有甚於此時者也。宜嚴敕內外諸司,省煩費,寬力役,毋擅有科率,王府之工宜速竣。庶令困敝少蘇。更乞崇正學,抑邪術,以清聖心;節財用,省工作,以培邦本。」詔下所司詳議。他所論奏者甚眾。在班列中最為耆碩,帝亦推心任之。自太子太保屢加至少保兼太子太傅,歲時賜賚,諸大臣莫敢望也。

吏部尚書屠滽罷,廷推文升。御史魏英等言兵部非文升不可,帝亦以為然。乃命倪岳代滽,而加文升少傅以慰之。岳卒,以文升代。南京、鳳陽大風雨壞屋拔木,文升請帝減膳撤樂,修德省愆,御經筵,絕遊宴;停不急務,止額外織造;振饑民,捕盜賊。已,又上吏部職掌十事。帝悉褒納。一品九載滿,加少師兼太子太師。帝以將考察,特召文升及都御史戴珊、史琳至煖閣,諭以秉公黜陟。又以文升年高重聽,再呼告之,命左右掖之下階。始文升為都御史,王恕在吏部,兩人皆以正直任天下事。疏出,天下傳誦。恕去,人望皆歸文升。迨為吏部,年已八十。修髯長眉,遇事侃侃不少衰。

孝宗崩,文升承遺詔請汰傳奉官七百六十三人,命留太僕卿李綸等十七人,餘盡汰之。正德元年,御用監中官王瑞復請用新汰者七人,文升不奉詔。給事中安奎刺得瑞納賄狀,劾之。瑞恚,誣文升抗旨,更下廷議,皆是文升,帝終不聽。文升因乞歸,不許。

是時,朝政已移於中官,文升老,日懷去志。會兩廣缺總督,文升推兵部侍郎熊繡。繡怏怏不欲出,其鄉人御史何天衢遂劾文升徇私欺罔。文升連疏求去,許之。賜璽書、乘傳,月廩歲隸有加。家居,非事未嘗入州城。語及時事,輒顰蹙不答。居三年,劉瑾亂政,坐文升前用雍泰為朋黨,除其名。五年六月卒,年八十五。瑾誅,復官,贈特進光祿大夫、太傅,謚端肅。

文升有文武才,長於應變,朝端大議往往待之決。功在邊鎮,外國皆聞其名。尤重氣節,厲廉隅,直道而行。雖遭讒詬,屢起屢仆,迄不少貶。子璁,以鄉貢士待選吏部,文升使請外,曰:「必大臣子而京秩,誰當外者?」卒後踰年,大盜趙鐩等剽河南,至鈞州,以文升家在,捨之去。攻泌陽,毀焦芳家,束草若芳像裂之。嘉靖初,加贈文升左柱國、太師。

劉大夏,字時雍,華容人。父仁宅,由鄉舉知瑞昌縣。流民千餘家匿山中,邏者索賂不得,誣民反。眾議加兵。仁宅單騎招之,民爭出訴,遂罷兵,擢廣西副使。

大夏年二十舉鄉試第一。登天順八年進士,改庶吉士。成化初,館試當留,自請試吏。乃除職方主事,再遷郎中。明習兵事,曹中宿弊盡革。所奏覆多當上意,尚書倚之若左右手。汪直好邊功,以安南黎灝敗於老撾,欲乘間取之。言於帝,索永樂間討安南故牘。大夏匿弗予,密告尚書餘子俊曰:「兵釁一開,西南立糜爛矣。」子俊悟,事得寢。朝鮮貢道故由鴉鶻關,至是請改由鴨綠江。尚書將許之,大夏曰:「鴨綠道徑,祖宗朝豈不知,顧紆迴數大鎮,此殆有微意。不可許。」乃止。中官阿九者,其兄任京衛經歷,以罪為大夏所笞。憲宗入其譖,捕繫詔獄,令東廠偵之無所得。會懷恩力救,乃杖二十而釋之。十九年,遷福建右參政,以政績聞。聞父訃,一宿即行。

弘治二年服闋,遷廣東右布政使。田州、泗城不靖,大夏往諭,遂順命。後山賊起,承檄討之。令獲賊必生致,驗實乃坐,得生者過半。改左,移浙江。

六年春,河決張秋,詔博選才臣往治。吏部尚書王恕等以大夏薦,擢右副都御史以行。乃自黃陵岡浚賈魯河,復浚孫家渡、四府營上流,以分水勢。而築長隄,起胙城歷東明、長垣抵徐州,亙三百六十里。水大治,更名張秋鎮曰「安平鎮」。孝宗嘉之,賜璽書褒美,語詳《河渠志》。召為左副都御史,歷戶部左侍郎。

十年命兼左僉都御史,往理宣府兵餉。尚書周經謂曰:「塞上勢家子以市糴為私利,公毋以剛賈禍。」大夏曰:「處天下事,以理不以勢,俟至彼圖之。」初,塞上糴買必粟千石、芻萬束乃得告納,以故,中官、武臣家得操利權。大夏令有芻粟者,自百束十石以上皆許,勢家欲牟利無所得。不兩月儲積棄羨,邊人蒙其利。明年秋,三疏移疾歸,築草堂東山下,讀書其中。越二年,廷臣交薦,起右都御史,總制兩廣軍務。敕使及門,攜二僮行。廣人故思大夏,鼓舞稱慶。大夏為清吏治,捐供億,禁內外鎮守官私役軍士,盜賊為之衰止。

十五年拜兵部尚書,屢辭乃拜命。既召見,帝曰:「朕數用卿,數引疾何也?」大夏頓首言:「臣老且病,竊見天下民窮財盡,脫有不虞,責在兵部,自度力不辦,故辭耳。」帝默然。南京、鳳陽大風拔木,河南、湖廣大水,京師苦雨沈陰。大夏請凡事非祖宗舊而害軍民者,悉條上釐革。十七年二月又言之。帝命事當興革者,所司具實以聞,乃會廷臣條上十六事,皆權倖所不便者,相與力尼之。帝不能決,下再議。大夏等言:「事屬外廷,悉蒙允行。稍涉權貴,復令察核。臣等至愚,莫知所以。」久之,乃得旨:「傳奉官疏名以請;幼匠、廚役減月米三斗;增設中官,司禮監核奏;四衛勇士,御馬監具數以聞。餘悉如議。」織造、齋醮皆停罷,光祿省浮費巨萬計,而勇士虛冒之弊亦大減。制下,舉朝歡悅。先是,外戚、近倖多干恩澤,帝深知其害政,奮然欲振之。因時多災異,復宣諭群臣,令各陳缺失。大夏乃復上數事。

其年六月再陳兵政十害,且乞歸。帝不許,令弊端宜革者更祥具以聞。於是,大夏舉南北軍轉漕番上之苦,及邊軍困敝、邊將侵剋之狀極言之。帝乃召見大夏於便殿,問曰:「卿前言天下民窮財盡。祖宗以來征斂有常,何今日至此?」對曰:「正謂不盡有常耳。如廣西歲取鐸木,廣東取香藥,費固以萬計,他可知矣。」又問軍,對曰:「窮與民等。」帝曰:「居有月糧,出有行糧,何故窮?」對曰:「其帥侵剋過半,安得不窮。」帝太息曰:「朕臨御久,乃不知天下軍民困,何以為人主!」遂下詔嚴禁。當是時,帝方銳意太平,而劉健為首輔,馬文升以師臣長六卿,一時正人充布列位。帝察知大夏方嚴,且練事,尤親信。數召見決事,大夏亦隨事納忠。

大同小警,帝用中官苗逵言,將出師。內閣劉健等力諫,帝猶疑之,召問大夏曰:「卿在廣,知苗逵延綏搗巢功乎?」對曰:「臣聞之,所俘婦稚十數耳。賴朝廷威德,全師以歸。不然,未可知也。」帝默然良久,問曰:「太宗頻出塞,今何不可?」對曰:「陛下神武固不後太宗,而將領士馬遠不逮。且淇國公小違節制,舉數十萬眾委沙漠,奈何易言之。度今上策惟守耳。」都御史戴珊亦從旁贊決,帝遽曰:「微卿曹,朕幾誤。」由是,師不果出。

莊浪土帥魯麟為甘肅副將,求大將不得,恃其部眾強,徑歸莊浪。廷臣懼生變,欲授以大帥印,又欲召還京,處之散地。大夏請獎其先世忠順,而聽麟就閒。麟素貪虐失眾心,兵柄已去無能為,竟怏怏病死。

帝欲宿兵近地為左右輔。大夏言:「保定設都司統五衛,祖宗意當亦如此。請遣還操軍萬人為西衛,納京東兵密雲、薊州為東衛。」帝報可。中官監京營者恚失兵,揭飛語宮門。帝以示大夏曰:「宮門豈外人能至?必此曹不利失兵耳。」由是,間不得行。

帝嘗諭大夏曰:「臨事輒思召卿,慮越職而止。後有當行罷者,具揭帖以進。」大夏頓首曰:「事之可否,外付府部,內咨閣臣可矣。揭帖滋弊,不可為後世法。」帝稱善。又嘗問:「天下何時太平?」對曰:「求治亦難太急。但用人行政悉與大臣面議,當而後行,久之天下自治。」嘗乘間言四方鎮守中官之害。帝問狀,對曰:「臣在兩廣見諸文武大吏供億不能敵一鎮守,其煩費可知。」帝曰:「然祖宗來設此久,安能遽革?第自今必廉如鄧原、麥秀者而後用,不然則已之。」大夏頓首稱善。大夏每被召,跪御榻前。帝左右顧,近侍輒引避。嘗對久,憊不能興,呼司禮太監李榮掖之出。一日早朝,大夏固在班,帝偶未見,明日諭曰:「卿昨失朝耶?恐御史糾,不果召卿。」其受眷深如此。特賜玉帶、麒麟服,所賚金幣、上尊,歲時不絕。

未幾,孝宗崩,武宗嗣位,承詔請撤四方鎮守中官非額設者。帝止撤均州齊元。大夏復議上應撤者二十四人,又奏減皇城、京城守視中官,皆不納。頃之,列上傳奉武臣當汰者六百八十三人,報可。大漢將軍薛福敬等四十八人亦當奪官,福敬等故不入侍以激帝怒。帝遽命復之,而責兵部對狀,欲加罪。中官寧瑾頓首曰:「此先帝遺命,陛下列之登極詔書,不宜罪。」帝意乃解。中官韋興者,成化末得罪久廢,至是夤緣守均州。言官交諫,大夏等再三爭,皆不聽。正德元年春,又言:「鎮守中官,如江西董讓、薊州劉瑯、陜西劉雲、山東朱雲貪殘尤甚,乞按治。」帝不悅。大夏自知言不見用,數上章乞骸骨。其年五月,詔加太子太保,賜敕馳驛歸,給廩隸如制。給事中王翊、張襘請留之,吏部亦請如翊、衣會言,不報。

大夏忠誠懇篤,遇知孝宗,忘身徇國,於權倖多所裁抑。嘗請嚴核勇士,為劉瑾所惡。劉宇亦憾大夏,遂與焦芳譖於瑾曰:「籍大夏家,可當邊費十二。」三年九月,假田州岑猛事,逮繫詔獄。瑾欲坐以激變律死,都御史屠滽持不可,瑾謾罵曰:「即不死,可無戍耶?」李東陽為婉解,且瑾詗大夏家實貧,乃坐戍極邊。初擬廣西,芳曰:「是送若歸也」,遂改肅州。大夏年已七十三,布衣徒步過大明門下,叩首而去。觀者歎息泣下,父老攜筐送食,所至為罷市、焚香祝劉尚書生還。比至戍所,諸司憚瑾,絕餽問,儒學生徒傳食之。遇團操,輒荷戈就伍。所司固辭,大夏曰:「軍,固當役也。」所攜止一僕。或問何不挈子姓,曰:「吾宦時,不為子孫乞恩澤。今垂老得罪,忍令同死戍所耶?」大夏既遣戍,瑾猶摭他事罰米輸塞上者再。

五年夏,赦歸。瑾誅,復官,致仕。清軍御史王相請復廩隸,錄其子孫。中官用事者終嗛之,不許。大夏歸,教子孫力田謀食。稍贏,散之故舊宗族。預自為壙志,曰:「無使人飾美,俾懷愧地下也。」十一年五月卒,年八十一。贈太保,謚忠宣。

大夏嘗言:「居官以正己為先。不獨當戒利,亦當遠名。」又言:「人生蓋棺論定,一日未死,即一日憂責未已。」其被逮也,方鋤菜園中,入室攜數百錢,跨小驢就道。赦歸,有門下生為巡撫者,枉百里謁之。道遇扶犁者,問孰為尚書家,引之登堂,即大夏也。朝鮮使者在鴻臚寺館遇大夏邑子張生,因問起居曰:「吾國聞劉東山名久矣。」安南使者入貢曰:「聞劉尚書戍邊,今安否?」其為外國所重如此。

贊曰:王恕砥礪風節,馬文升練達政體,劉大夏篤棐自將,皆具經國之遠猷,蘊畜君之正志。綢繆庶務,數進讜言,迹其居心行己,磊落光明,剛方鯁亮,有古大臣節概。歷事累朝,享有眉壽,朝野屬望,名重遠方。《詩》頌老成,《書》稱黃髮,三臣者近之矣。恕昧遠名之戒,以作傳見疏。而文升,大夏被遇孝宗之朝,明良相契,荃宰一心。迨至宦豎乘權,耆舊擯斥,進退之際所係詎不重哉!


\end{pinyinscope}