\article{列傳第一百五十二}

\begin{pinyinscope}
賀逢聖(傅冠尹如翁}}南居益族父企仲(族弟居業周士樸呂維祺(弟維祮王家禎焦源溥兄源清李夢辰宋師襄麻僖王道純

田時震朱崇德崇德子國棟

賀逢聖,字克繇,江夏人。與熊廷弼少同里閈,而不相能。為諸生,同受知於督學熊尚文。尚文並奇二生,曰:「熊生,干將、莫邪也;賀生,夏瑚、商璉也。」舉於鄉。家貧,就應城教諭。萬曆四十四年,殿試第二人,授翰林編修。

天啟間,為洗馬。當是時,廷弼已再起經略遼東矣。廣寧之敗,同鄉官將揭白廷弼之冤,意逢聖且沮之。逢聖作色曰:「此乃國家大事,吾安敢小嫌介介,不以明!」即具草上之。湖廣建魏忠賢生祠,忠賢聞上梁文出逢聖手,大喜,即日詣逢聖。逢聖曰:「誤,借銜陋習耳。」忠賢咈然去。翌日削逢聖籍。

莊烈帝即位,復官,連進秩。九年六月,以禮部尚書兼東閣大學士,入閣輔政,加太子太保,改文淵閣。十一年致政。十四年再入閣。明年再致政。

逢聖為人廉靜,束修砥行。帝頗事操切,逢聖終無所匡言。其再與周延儒同召,帝待之不如延儒。及予告,宴餞便殿,賜金,賜坐蟒。感激大哭,伏地不能起,帝亦汍瀾動容焉。

是時,湖廣賊大擾。明年春,張獻忠連陷蘄、黃,逼江夏。有大冶人尹如翁,逢聖門生,走三百里,持一僧帽、一袈裟來貽逢聖。逢聖反其衣曰:「子第去,毋憂我。」如翁去。五月,壬戌晦,賊陷武昌,執逢聖,叱曰:「我朝廷大臣,若曹敢無禮!」賊麾使去,遂投墩子湖死也。賊來自夏,去以秋雲。大吏望衍而祭,有神夢於湖之人,「我守賀相殊苦,汝受而視之,有黑子在其左手,其徵是。」覺而異之,俟於湖,赫然而尸出,驗之果是,蓋沉之百有七十日,面如生。以冬十一月壬子殮,大吏揮淚而葬之。

初,城之陷也,逢聖載家人以其句鹿出墩子,鑿其氐艡,皆溺。賀氏死者,妻危氏,子覲明,子婦曾氏、陳氏,孫三人,次子光明自他所來,凡二十餘人。福王時,贈少傅,謚文忠,祭葬蔭子如制。

如翁去,歸大冶。大冶城破,其慷慨而死者,如翁也。

其後有傅冠。冠,字元甫,進賢人。祖炯,南京刑部尚書。天啟二年,冠舉進士第二,授翰林編修。崇禎十年秋,由禮部右侍郎拜尚書兼東閣大學士。性簡易,有章奏發自御前,冠以為揭帖,援筆判其上。既知誤,惶恐引罪,帝即放歸。唐王時,命以原官督師江西。嗜酒,或劾之,乃致仕。大清下江西,冠走匿門人泰寧汪亨龍家。亨龍執而獻之有司,殺之汀州,血漬地,久而猶鮮。

南居益,字思受,渭南人,尚書企仲族子、師仲從子也。曾祖從吉與曾伯祖大吉皆進士。兩人子姓,科第相繼。

企仲,大吉孫,萬曆八年進士。以祖母年高,請終養。祖母既歿,授刑部主事。客寓貲其家,夫婦並歿,企仲呼其子還之。吏部尚書孫丕揚以為賢,調為己屬。歷文選郎,擢太僕少卿,進太僕卿。三十年,帝以疾詔免礦稅,釋繫囚,錄建言貶斥諸臣。既而悔之,命礦稅如故,餘所司議行。吏、刑二部尚書李戴、蕭大亨遲數日未奏,企仲請亟罷二人,而敕二部亟如詔奉行。帝大恚,傳諭亟停二事,落企仲一官。給事中蕭近高,御史李培、餘懋衡亦請信明詔,帝益怒,並奪其俸,且命益重前貶謫官鄒元標等罰,欲以鉗言者。諸閣臣力爭,乃止。而給事中張鳳翔迎帝意,劾企仲他事,遂削籍。天啟初,起太常卿,累遷南京吏部尚書,以老致仕。師仲父軒,吏部郎中,嘗著《通鑑綱目前編》。師仲至南京禮部尚書。

居益少厲操行,舉萬曆二十九年進士,授刑部主事。三遷廣平知府,擢山西提學副使,鴈門參政,歷按察使、左右布政使,並在山西。

天啟二年,入為太僕卿。明年擢右副都御史,巡撫福建。紅毛夷者,海外雜種,紺眼,赤鬚髮,所謂和蘭國也,自昔不通中土,由大泥、咬留吧二國通閩商。萬曆中,奸民潘秀引其人據彭湖求市,巡撫徐學聚令轉販之二國。二國險遠,商舍而之呂宋。夷人疑呂宋邀商舶,攻之,又寇廣東香山澳,皆敗,不敢歸國,復入彭湖求市,且築城焉。巡撫商周祚拒之,不能靖。會居益代周祚,賊方犯漳、泉,招日本、大泥、咬留吧及海寇李旦等為助。居益使人招旦,說攜大泥、咬留吧。賊帥高文律懼,遣使求款,斬之,築城鎮海港,逼賊風櫃。賊窮蹙,泛舟去,遂擒文律,海患乃息。五年遷工部右侍郎,總督河道。魏忠賢銜居益敘功不及己,格其賞。給事中黃承昊復論居益倚傍門戶,躐躋通顯,遂削籍去。閩人詣闕訟之,不聽。乃立祠以祀,勒碑於彭湖及平遠臺。

崇禎元年,起戶部右侍郎,總督倉場。陜西鎮缺餉至三十餘月,居益請以陜賦當輸關門者留三十萬,紓其急,報可。畿輔戒嚴,居益在通州,為城守計甚備。會工部尚書張鳳翔坐軍械不具下吏,四司郎中瘐死者三,遂詔居益代鳳翔。未幾,試炮而炸,兵部尚書梁廷棟劾郎中王守履失職。守履懼,詆兵部郎中王建侯誣己。廷議不如守履言,遂下獄。居益疏捄,帝以為徇私,削籍歸。廷杖守履六十,斥為民。尋敘城守功,復居益冠帶。

十六年,李自成陷渭南,責南氏餉百六十萬。企仲年八十三矣,遇害。誘降居益及企仲子禮部主事居業,皆不從。明年正月,賊遣兵擁之去,加炮烙。二人終不屈,絕食七日而死。

周士樸,字丹其,商丘人。萬曆四十一年進士。除曲沃知縣。泰昌元年征授禮科給事中。中官王添爵選凈身男子,索賄激變。守陵劉尚忠鼓陵軍挾賞。劉朝等假齎送軍器名,出行山海外,勢洶洶。織造李實訐周起元。群璫索冬衣,辱尚書鐘羽正。士樸皆疏爭。士樸性剛果,不能委蛇隨俗,尤好與中官相搘柱,深為魏忠賢所惡。會當擢京卿,忠賢持不下,士樸遂謝病歸。

崇禎元年,起太常少卿,歷戶部左、右侍郎,拜工部尚書。帝命中官張彝憲監戶、工二部出納,士樸恥之,數與齟齬。彞憲訐於帝,士樸疏對辭直,帝無以難。未幾,駙馬都尉齊贊元以遂平長公主塋價,士樸不引瑞安大長公主例,而壽寧大長公主薨則引瑞安例,上疏醜詆之,遂削其籍。

十五年,廷臣交薦,不召。其年八月,李自成陷商丘,與妻曹、妾張、子舉人業熙、子婦沈同日縊死。

呂維祺,字介孺,新安人。祖母牛氏以守節被旌。父孔學,事母孝,捐粟千二百石振饑,兩旌孝義。維祺舉萬曆四十一年進士,授兗州推官,擢吏部主事,更歷四司。光宗崩,皇長子未踐阼,內侍導幸小南城。維祺謁見慈慶宮,言梓宮在殯,乘輿不得輕動,乃止。天啟初,歷考功、文選員外郎,進驗封郎中,告歸。開封建魏忠賢生祠,遺書士大夫戒勿預。忠賢毀天下書院,維祺立芝泉講會,祀伊洛七賢。

崇禎元年,起尚寶卿,遷太常少卿,督四夷館。明年四月,廷議軍餉,維祺陳奏十五事。其冬,奏防微八事,言:「陛下初勤批答,今或留中,留中多則疑慮起,當防一。初虛懷商榷,及擬旨一不當,改擬徑行,豈無當執奏,當防二。初無疑厭,疑厭諸臣自取,今且共、夔並進,當防三。初日御講筵,今始傳免,當防四。初寡嗜慾,慎宴游,今或偶涉,當防五。初慎刑獄,今有下詔獄者,且登聞頻擊,恐長嚚訟風,當防六。初重廷推,今間用陪,非常典,當防七。初樂讜言,今或譴訶時及,當防八。」帝優旨報之。

三年,擢南京戶部右侍郎,總督糧儲。設會計簿,鉤考隱沒侵欺,及積逋不輸,各數十百萬,大者彈奏,小者捕治。立法嚴督屯課,倉庾漸充。條上六議,曰稽出入以杜侵漁,增比較以完積案,設本科以重題覆,時會計以核支收,定差序以杜營私,禁差假以修職業。帝稱善,即行之。

六年,拜南京兵部尚書,參贊機務。清冒伍八千餘名。請申飭江防,鳳陵單外為憂,弗省。八年正月,賊犯江北,遣參將薛邦臣防全椒,趙世臣戍浦口。世臣潰走,南京震動,鳳陽亦旋告陷。大計拾遺,言官復劾他事,遂除名。時維祺父孔學避賊洛陽,維祺乃歸留洛,立伊洛會,及門二百餘人。著《孝經本義》成,上之。

十二年,洛陽大饑。維祺勸福王常洵散財餉士,以振人心,王不省。乃盡出私廩,設局振濟。事聞,復官。然饑民多從賊者,河南賊復大熾。無何,李自成大舉來攻,維祺分守洛陽北城。夜半,總兵王紹禹之軍有騎而馳者,周呼於城上,城外亦呼而應之,於是城陷。賊有識維祺者曰:「子非振饑呂尚書乎?我能活爾,爾可以間去。」維祺弗應,賊擁維祺去。時福王常洵匿民舍中,賊跡而執之,遇維祺於道。維祺反接,望見王,呼曰:「王,綱常至重。等死耳,毋屈膝於賊!」王瞠不語。見賊渠於周公廟,按其項使跪,不屈,延頸就刃而死。時十四年之正月某日也。維祺年五十有五,贈太子少保,祭葬,蔭子如制。而維祺之家在新安者,十六年城陷,家亦破。

弟維祮,字泰孺,由選貢生為樂平知縣者也。至是解職歸,亦抗節死。贈按察僉事。福王立南京,加贈維祺太傅,謚忠節。

王家禎,長垣人。萬曆三十五年進士。天啟間,歷官左僉都御史,巡撫甘肅。松山部長銀定、歹成擾西鄙二十餘年,家禎至,三犯三卻之,先後斬首五百四十。擢戶部右侍郎,轉左。崇禎元年攝部事,邊餉不以時發。秋,遼東兵鼓噪,巡撫畢自肅自縊死。帝大怒,削家禎籍。已,敘甘肅功,復其冠帶。

九年七月,京師被兵,起兵部左侍郎,尋以本官兼右僉都御史,總理河南、湖廣、山西、陜西、四川、江北軍務,代盧象升討賊。會河南巡撫陳必謙罷,即命兼之。督將士會剿賊馬進忠等於南陽,復遣兵救襄陽,大戰牌樓閣。其冬,家丁鼓噪,燒開封西門。家禎夜自外歸,慰諭犒賞,詰旦,發往南陽討土寇楊四以去。楊四者,舞陽劇盜也。初,四與其黨郭三海、侯馭民等降於必謙,至是復叛,故家禎有是遣。其後南陽同知萬年策與監紀推官湯開遠,諸將左良玉、牟文綬等連破四,四焚死,其黨亦為諸將所擒誅云。

當是時,流賊盡趨江北,留都震驚。言者謂家禎奉命討安慶賊,未嘗一出中州。帝亦以家丁之變心輕之。明年四月乃以總理授熊文燦,令家禎專撫河南。文燦未至,詔遣左良玉援安慶,家禎不遣。秋,劉國能犯開封,裨將李春貴等戰歿。議罪,家禎落職閒住。久之,李自成陷京師,遣兵據長垣,設偽官。家禎與其子元炌並自經死。

焦源溥,字涵一,三原人。萬曆四十一年進士。歷知沙河、浚二縣,考最,召為御史。

熹宗嗣位,移宮議起,刑部尚書黃克纘請寬盜寶諸奄。源溥折之曰:「光宗,神宗元子也;為元子者為忠,則為福籓者非忠。孝端、孝靖,神宗后也;為二後者為忠,則為鄭貴妃者非忠。孝元、孝和,光宗后也;為二后者為忠,則為李選侍者非忠。貴妃三十年心事,人誰不知?張差持梃,危在呼吸,尚忍言哉!況當先帝御極之初,忽傳皇祖封后之命,請封不得,冶容進矣。張差之梃不中,則投以女優之惑;崔文昇之藥不速,則促以李可灼之丸。痛哉!先帝欲諱言進御之事,遂甘蒙不白之冤。今即厚待貴妃,始終恩禮,而鄭養性之都督不可不奪也,崔文昇不可不磔也。若竟置弗問,不幾於忘父乎!李選侍一宮人,更非貴妃比,如聖諭阻陛下於煖閣,挾陛下以垂簾,及凌虐聖母狀,有臣子所不忍言者。今即為選侍乞憐,第可求曲宥前辜,量從優典,而移宮始末不可得而抹摋也,盜寶諸奄不可得而寬宥也。若竟置諸奄弗問,不幾於忘母乎!」疏上,舉朝寒懼。

天啟二年憂歸。服闋還朝,出按真定諸府,例轉鳳陽兵備副使。時崔文昇出鎮兩淮,欲甘心源溥,遂移疾歸。

崇禎二年起故官,分巡河東道,遷寧武參政,有平寇功,就遷山西按察使。七年擢右僉都御史,巡撫大同。時邊事日棘,兵缺伍,餉又久乏,歲洊饑,民淘馬糞以食。源溥請蠲振增餉,當事不能應。踰年,自劾求去,遂罷歸。十六年冬,李自成陷關中,與從兄源清同被執,勒令輸金。源溥瞋目大罵,賊拔其舌,支解之。

源清,字湛一,由進士歷官宣府巡撫。七年秋,坐萬全左衛失守,奪官謫戍。久之釋還,年七十。至是抗節,不食七日死。

李夢辰,字元居,睢州人。崇禎元年進士。授庶吉士,改兵科給事中,時盜起陜西,山東曹、濮間之盜,道梗三百餘里,河北有回賊。夢辰歷陳其狀,請敕將吏急防。五年,上疏言:「中外交訌,秦、晉、齊、魯多亂,兩河居中尤要地。鉛硝久市直未償,漕米歲輸累無已,宗祿併徵,南陽加派,河決歲歉,郵傳催科之患百出,民室如懸罄,生計日不支,急難誰肯用命?兩河標兵、磁兵,新舊不滿七千,一有警,防禦何資?今日之務急防河,繕城,備器,練鄉兵,治甲胄,尤以收拾人心為本。」帝命所司嚴飭。六年冬,鉅盜盡萃河北。夢辰慮其南犯,請敕河南諸道監司急防渡口,而巡撫移駐衛輝,與山西、保定二撫臣掎角急擊。帝方下兵部議,賊已從澠池潛渡。自是中州郡縣無日不告警矣。

累遷本科左給事中。復言:「將驕軍悍,鄧、張外嘉之兵弒主而叛,曹文詔、艾萬年之兵望賊而奔,尤世威、徐來朝之兵離汛而遁,今者,張全昌、趙光遠之兵且倒戈為亂矣。滎澤劫庫殺人,偃師列營對壘。且全昌等會剿豫賊,隨處逗遛,及中途兵變,全昌竟東行,光遠始西向。驕抗如此,安可不重治。」帝頗採其言。進吏科都給事中。都御史唐世濟薦霍維華,福建巡按應喜臣薦周維京,冀並翻逆案。夢辰疏駁之,世濟、喜臣皆下吏謫戍。

尋擢太常少卿,累遷至通政使。坐代人削章奏,貶秩調任。未幾,有持金囑中書舍人某賄大學士,求為副都御史者。邏卒廉得之,詞連夢辰。帝令夢辰自奏,事得白。然夢辰竟坐是削籍。

十五年春,賊攻開封,不克,遂去,陷西華,屠陳州,逼睢州。時州缺正官,夢辰歸,即乘城主守。無何,賊從他門入,擁夢辰見羅汝才。汝才問所欲,曰:「我大臣,但欲死爾。」汝才去,遣其客說降,且進之酒。夢辰覆杯於地,太息起,扼吭而卒。妻王氏,方病,聞之,不食死。

宋師襄,耀州人。萬曆四十四年進士。歷官御史。

天啟三年五月請罷內操,言:「自劉朝營脫死,與沈紘謀為固寵計。紘以募兵為朝外護,朝以內操為紘內援。宮府內外,知有朝而不知天子。天牖聖聰,一旦發露,屏之南京。然朝雖去,而三千虎旅安歸?世未有蓄怨藏怒之人潛布左右而不為患者,今惟有散之而已。夫平日卵翼朝者,黃克纘也,亡何以戎政內宣;抄參朝者,毛士龍也,未幾以構陷削籍。豈非握兵據要,轉相恐喝,以至是乎?」帝以內操祖宗故事,不納。又陳足財之策,請減上供,汰冗官,核營造,省賚賞。皆宦官所不便,格不行。奉聖夫人客氏子及中官王體乾、宋晉、魏進忠等十二人俱世襲錦衣。進忠者,魏忠賢也。師襄力諫。又言左都御史熊尚文、工部侍郎周應秋、登萊巡撫袁可立當去不去,光祿卿須之彥、太常卿呂純如不當來而來。帝皆不聽。

四年,巡按河南。陛辭,言:「今之言者,皆曰治平要務,乃終日籌邊事、商國計、飭吏治、計民生、弭盜賊,而漫無實效。所以然者,臺諫以進言為責,條奏一入,即雲盡職,言之行否,置弗問矣。六曹以題覆為責,題覆一上,即云畢事,事之行否,置弗問矣。內閣以票擬為責,票擬一定,即為明綸,旨之行否,置弗問矣。上謾下欺,釀成大患。今人怨已極,天怒已甚,災害並至,民不聊生,相聚思亂,十而八九。臣恐今日之患,不在遼左、黔、蜀,而在數百年休養之赤子也。」明年復命薦部內人才,首及尚書盛以弘。魏忠賢責以徇私,貶一秩調任,師襄遂歸。

崇禎元年召復官,擢太僕少卿,累遷至太常卿,致仕。奸人宋夢郊假師襄手書營兵部。事覺,師襄被逮,繫獄者二年。至徐石麒為刑部,始得雪。十六年冬,賊陷耀州,師襄死之。

麻僖,慶陽人。父永吉,由庶吉士為御史,終湖廣按察使,以清操聞。僖舉萬曆三十五年進士,授庶吉士,改兵科給事中。代王長子鼎渭訐父廢長立幼,僖劾代王無君鼎渭無父。四十年,疏陳納諫諍、舉枚卜、補大僚、登遺佚、速考選數事,不報。已,復請重武科、復比試、清納級、汰家丁、恤班操、急邊餉,時亦不能用。遼東巡撫楊鎬請用舊將李如梅,以僖言,改用張承蔭。承蔭未至而鎮遠堡、曹莊相繼失事,鎬皆不以實聞。僖兩疏劾之,鎬旋引去。已,與同官孫振基等劾熊廷弼殺人媚人。又言湯賓尹取韓敬,關節顯然,語具《振基傳》。尋乞假歸。四十五年京察,賓尹黨用事,以僖倚附東林,謫山西按察知事。

天啟二年,起兵部主事,歷尚寶丞、少卿,改太常。五年六月,魏忠賢黨御史陳世颭劾之,遂落職。崇禎初,復官,致仕家居。十六年冬,李自成陷慶陽,僖死之。

王道純,字懷鞠,蒲城人。天啟五年進士。授中書舍人。崇禎三年擢御史。疏陳破資格之說,言銓除、舉劾、考選,甲乙科太低昂,宜變通,則賢才日廣。帝命所司即行,而甲科勢重,卒不能返。流賊躪關中,道純請急振饑民,毋使從賊,報可。已,劾罷光祿卿蘇晉、參政張爾基。四年,劾吏部尚書王永光當去者三,不可留者四,不納。

巡按山東。其時李九成、孔有德叛於吳橋,南下。道純移書巡撫余大成,令討捕,大成不信。再促之,遂托疾請告,與登萊巡撫孫元化遣使招撫。道純以為非,請敕二撫速剿。及賊陷登州,元化被縶,大成猶主招撫。道純憤,抗疏力爭,帝即命道純監軍。及徐從治代大成,謝璉代元化,並入萊州,為賊困。在外調度,止道純一人。賊遣人偽乞撫,道純焚書斬使,馳疏言:「賊日以撫愚我,一撫而六城陷,再撫而登州亡,三撫而黃縣失,今四撫而萊州被圍。我軍屢挫,安能復戰?乞速發大軍,拯此危土。」時周延儒、熊明遇主撫議,道純反被責讓。明遇遣職方主事張國臣贊畫軍事,國臣入賊中招諭。賊佯許之,攻圍如故。及總督劉宇烈至,進兵沙河,道純與之俱。宇烈中情怯,頓兵不進,日議撫,尋棄軍奔。道純復請速討,不納。迨巡撫謝璉被執,帝震怒,逮宇烈,召道純還京,而明遇亦罷去。宇烈下吏,引道純分過。道純疏駁其所奏十餘事,命所司並按。又劾明遇、國臣交通誤國十罪,語侵延儒。疏未下,延儒洩之國臣,國臣亦劾道純十罪,道純遂並劾延儒。帝皆不問。已而賊平,道純竟坐監軍溺職,斥為民。

十五年以廷臣薦,將起用,未果。及李自成陷蒲城,道純抗節死。福王時,贈恤如制。

田時震,富平人。天啟二年進士。歷知光山、靈寶。崇禎二年入為御史,疏劾南京戶部尚書范濟世、順天巡撫單明詡、御史卓邁黨逆罪,而請免故御史夏之令誣坐贓,並從之。劾劉鴻訓納田仰金,囑吏部尚書王永光用為四川巡撫,仰迄罷去。時震以發鴻訓私,進秩一等。未幾,又劾永光及溫體仁,忤旨切責。御史袁弘勛者,永光心腹也,被劾罷職,永光力援之。時震言:「弘勛因閣臣劉鴻訓賄敗,輒肆瀆辯。不知鴻訓之差快人意者,正以能別白徐大化、霍維華諸人之奸而斥去之,安得借此為翻案之端耶?弘勛計行,大化、維華輩將乘間抵隙,害不可勝言。」因薦故光祿少卿史記事,蕭然四壁,講學著書,亟宜召用,帝不納。

時震既屢忤永光,遂以年例出為江西右參議,調山西,就遷左參政,罷歸。十六年冬,流賊陷富平,授以偽職,不屈死。

同邑朱崇德,字淳庵,侍郎國棟父也。國棟中天啟二年進士,歷戶科給事中。吏部侍郎張捷薦逆案呂純如,國棟上疏力詆。已,又劾兩廣總督熊文燦,招撫海盜劉香,奏詞掩飾欺罔五罪,帝切責文燦。而國棟累遷巡撫山東右僉都御史,督治昌平。十五年卒。

國棟卒之明年,富平陷於賊。賊驅崇德往長安,中道稱病。賊見其老,以為果病也,聽之歸。崇德曰:「始吾所以隱忍者,為九族計也,今得死所矣。」乃北面再拜,自縊死。是時關中諸死節者甫議恤,而國變至。福王立,始贈崇德右副都御史。

贊曰:流賊荼毒中原,所至糜爛。士大夫遘難者,不死則辱。然當其時,徘徊隱忍、蒙垢而終以自戕者,亦不少矣。賀逢聖諸人從容就義,臨患難而不易其節,一死顧不重哉!逢聖與南居益、周士樸公方清正,呂維祺邃學純修,固中朝賢士大夫。宋師襄所謂「上謾下欺,釀成大患」,末季之習,痛哉其言之也。


\end{pinyinscope}