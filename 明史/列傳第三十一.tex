\article{列傳第三十一}

\begin{pinyinscope}
王艮(高遜志)廖昇魏冕鄒瑾龔泰周是修程本立黃觀王叔英林英黃鉞曾鳳韶王良陳思賢龍溪六生臺溫二樵程通黃希範葉惠仲黃彥清蔡運石允常高巍韓郁高賢寧王璡周縉牛景先程濟等。

王艮,字敬止,吉水人。建文二年進士。對策第一。貌寢,易以胡靖,即胡廣也。艮次之,又次李貫。三人皆同里,並授修撰,如洪武中故事,設文史館居之。預修《太祖實錄》及《類要》、《時政記》諸書。一時大著作皆綜理之。數上書言時務。

燕兵薄京城,艮與妻子訣曰:「食人之祿者,死人之事。吾不可復生矣。」解縉、吳溥與艮、靖比舍居。城陷前一夕,皆集溥舍。縉陳說大義,靖亦奮激慷慨,艮獨流涕不言。三人去,溥子與弼尚幼,歎曰:「胡叔能死,是大佳事。」溥曰:「不然,獨王叔死耳。」語未畢,隔墻聞靖呼:「外喧甚,謹視豚。」溥顧與弼曰:「一豚尚不能舍,肯舍生乎?」須臾艮舍哭,飲鴆死矣。縉馳謁,成祖甚喜。明日薦靖,召至,叩頭謝。貫亦迎附。後成祖出建文時群臣封事千餘通,令縉等編閱。事涉兵農、錢穀者留之,諸言語干犯及他,一切皆焚毀。因從容問貫、縉等曰:「爾等宜皆有之。」眾未對,貫獨頓首曰:「臣實未嘗有也。」成祖曰:「爾以無為美耶?食其祿,任其事,當國家危急,官近侍獨無一言可乎?朕特惡夫誘建文壞祖法亂政者耳。」後貫遷中允,坐累,死獄中。臨卒歎曰:「吾愧王敬止矣。」

有高遜志者,艮座主也,蕭縣人,寓嘉興。幼嗜學,師貢師泰、周伯琦等。文章典雅,成一家言。征修《元史》,入翰林,累遷試吏部侍郎。以事謫朐山。建文初,召為太常少卿,與董倫同主會試。得士自艮外,胡靖、吳溥、楊榮、金幼孜、楊溥、胡濙、顧佐等皆為名臣。燕師入,存歿無可考。

廖昇,襄陽人。不知其所以進,學行最知名,與方孝孺、王紳相友善。洪武末,由左府斷事擢太常少卿。建文初,修《太祖實錄》,董倫、王景為總裁官;昇與高遜志為副總裁官;李貫、王紳、胡子昭、楊士奇、羅恢、程本立為纂修官。皆一時選。燕師渡江,朝廷遣使請割地。不許。昇聞而慟哭,與家人訣,自縊死。殉難諸臣,升死最先。其後陳瑛奏諸臣逆天命,效死建文君,請行追戮,亦首及昇云。

時為瑛追論者,有魏冕等。冕官御史。燕兵犯闕,都督徐增壽徘徊殿廷,有異志。冕率同官毆之,與大理丞鄒瑾大呼,請速加誅。明日,宮中火起。有勸冕降者,厲聲叱之。遂自殺,瑾亦死。瑾、冕皆永豐人。其同里鄒朴,官秦府長史。聞瑾死,憤甚,不食卒。或曰即瑾子也。

又都給事中龔泰,義烏人。由鄉薦起家。燕王入金川門,泰被縛,以非奸黨釋,不殺。自投城下死。泰嘗遊學宮,狂人擠之,溺池中幾死,弗校。人服其量。

周是修,名德,以字行,泰和人。洪武末,舉明經,為霍邱訓導。太祖問家居何為。對曰:「教人子弟,教弟力田。」太祖喜,擢周府奉祀正。逾年,從王北征至黑山,還遷紀善。建文元年,有告王不法者,官屬皆下吏。是修以嘗諫王得免,改衡府紀善。衡王,惠帝母弟,未之籓。是修留京師,預翰林纂修,好薦士,陳說國家大計。燕兵渡淮,與蕭用道上書指斥用事者。用事者怒,共挫折之,是修屹不為動。京城失守,留書別友人江仲隆、解縉、胡靖、蕭用道、楊士奇,付以後事。具衣冠,為贊繫衣帶間。入應天府學,拜先師畢,自經於尊經閣,年四十九。燕王即帝位,陳瑛言是修不順天命,請追戮。帝曰:「彼食其祿,自盡其心,勿問。」

是修外和內剛,志操卓犖。非其義,一介不茍得也。嘗曰:「忠臣不計得失,故言無不直;烈女不慮死生,故行無不果。」嘗輯古今忠節事為《觀感錄》。其學自經史百家,陰陽醫卜,靡不通究。為文援筆立就而雅贍條達。初與士奇、縉、靖及金幼孜、黃淮、胡儼約同死。臨難,惟是修竟行其志云。

程本立,字原道,崇德人。先儒頤之後。父德剛,負才氣不仕。元將路成兵過皁林,暴掠。德剛為陳利害。成悅,戢其部眾。俗奏,官之,辭去。本立少有大志,讀書不事章句。洪武中,旌孝子,太祖嘗謂之曰:「學者爭務科舉,以窮經為名而無實學。子質近厚,當志聖賢之學。」本立益自力。聞金華朱克修得硃熹之傳於許謙,往從之遊。舉明經、秀才。除秦府引禮舍人,賜楮幣、鞍馬。母憂去官,服除,補周府禮官,從王之開封。二十年春進長史。從王入覲。坐累,謫雲南馬龍他郎甸長官司吏目。留家大梁,攜一僕之任。土酋施可伐煽百夷為亂,本立單騎入其巢,諭以禍福,諸酋咸附。未幾,復變。西平侯沐英、布政使張紞知本立賢,屬行縣典兵事,且撫且禦。自楚雄、姚安抵大理、永昌鶴慶、麗江。山行野宿,往來綏輯凡九年,民夷安業。三十一年奏計京師。學士董倫、府尹向寶交薦之。徵入翰林,預修《太祖實錄》,遷右僉都御史。俸入外,不通饋遺。建文三年坐失陪祀,貶官,仍留纂修。《實錄》成,出為江西副使。未行,燕兵入,自縊死。

黃觀,字伯瀾,一字尚賓,貴池人。父贅許,從許姓。受學於元待制黃冔。冔死節,觀益自勵。洪武中,貢入太學。繪父母墓為圖,贍拜輒淚下。二十四年,會試、廷試皆第一。累官禮部右侍郎,乃奏復姓。建文初,更官制,左、右侍中次尚書。改觀右侍中,與方孝孺等並親用。燕王舉兵,觀草制,諷其散軍歸籓,敕身謝罪,辭極詆斥。四年奉詔募兵上遊,且督諸郡兵赴援。至安慶,燕王已渡江入京師,下令暴左班文職奸臣罪狀,觀名在第六。既而索國寶,不知所在,或言:「已付觀出收兵矣!」命有司追捕,收其妻翁氏並二女給象奴。奴索釵釧市酒肴,翁氏悉與之持去,急攜二女及家屬十人,投淮清橋下死。觀聞金川門不守,歎曰:「吾妻有志節,必死。」招魂,葬之江上。命舟至羅剎磯,朝服東向拜,投湍急處死。

觀弟覯,先匿其幼子,逃他處。或云覯妻畢氏孀居母家,遺腹生子,故黃氏有後於貴池。

初,觀妻投水時,嘔血石上,成小影,陰雨則見,相傳為大士像。僧舁至庵中。翁氏見夢曰;「我黃狀元妻也。」比明,沃以水,影愈明,有愁慘狀。後移至觀祠,名翁夫人血影石。今尚存。

王叔英,字原採,黃巖人。洪武中,與楊大中、葉見泰、方孝孺、林右並徵至。叔英固辭歸。二十年以薦為仙居訓導,改德安教授。遷漢陽知縣,多惠政。歲旱,絕食以禱,立應。建文時,召為翰林修撰。上《資治八策》,曰:「務問學、謹好惡、辨邪正、納諫諍、審才否、慎刑罰、明利害、定法制」。皆援證古今,可見之行事。又曰;「太祖除奸剔穢,抑強鋤梗,如醫去病,如農去草。去病急或傷體膚,去草嚴或傷禾稼。病去則宜調燮其血氣,草去則宜培養其根苗。」帝嘉納之。

燕兵至淮,奉詔募兵。行至廣德,京城不守。會齊泰來奔,叔英謂泰貳心,欲執之。泰告以故,乃相持慟哭,共圖後舉。已,知事不可為,沐浴更衣冠,書絕命詞,藏衣裾間,自經於元妙觀銀杏樹下。天台道士盛希年葬之城西五里。其詞曰:「人生穹壤間,忠孝貴克全。嗟予事君父,自省多過愆。有志未及竟,奇疾忽見纏。肥甘空在案,對之不下咽。意者造化神,有命歸九泉。嘗念夷與齊,餓死首陽巔。周粟豈不佳,所見良獨偏。高蹤渺難繼,偶爾無足傳。千秋史官筆,慎勿稱希賢。」又題其案曰:「生既已矣,未有補於當時。死亦徒然,庶無慚於後世。」燕王稱帝,陳瑛簿錄其家。妻金氏自經死,二女下錦衣獄,赴井死。

叔英與孝孺友善,以道義相切劘。建文初,孝孺欲行井田。叔英貽書曰:「凡人有才固難,能用其才尤難。子房於漢高,能用其才者也;賈誼於漢文,不能用其才者也。子房察高帝可行而言,故高帝用之,一時受其利。雖親如樊、酈,信如平、勃,任如蕭、曹,莫得間焉。賈生不察而易言,且言之太過,故絳、灌之屬得以短之。方今明良相值,千載一時。但事有行於古,亦可行於今者,夏時周冕之類是也。有行於古,不可行於今者,井田封建之類是也。可行者行,則人之從之也易,而民樂其利。難行而行,則從之也難,而民受其患。」時井田雖不行,然孝孺卒用《周官》更易制度,無濟實事,為燕王藉口。論者服叔英之識,而惜孝孺不能用其言也。

時御史古田林英亦在廣德募兵,知事無濟,再拜自經。妻宋氏下獄,亦自經死。

黃鉞,字叔揚,常熱人。少好學。家有田在葛澤陂,鉞父令督耕其中。鉞從友人家借書,竊讀不廢。縣舉賢良,授宜章典史。建文元年,舉湖廣鄉試。明年賜進士,授刑科給事中。三年丁父憂。方孝孺弔之,屏人問曰:「燕兵日南,蘇、常、鎮江,京師左輔也。君吳人,朝廷近臣,今雖去,宜有以教我。」鉞曰:「三府唯鎮江最要害。守非其人,是撤垣而納盜也。指揮童俊狡不可任,奏事上前,視遠而言浮,心不可測也。蘇州知府姚善,忠義激烈,有國士風。然仁有餘而禦下寬,恐不足定亂。且國家大勢,當守上游,兵至江南,禦之無及也。」孝孺乃因鉞附書於善。善得書,與鉞相對哭,誓死國。鉞至家,依父殯以居。

燕兵至江上,善受詔統兵勤王,以書招鉞。鉞知事不濟,辭以營葬畢乃赴。既而童俊果以鎮江降燕。鉞聞國變,杜門不出。明年以戶科左給事中召,半途自投於水。以溺死聞,故其家得不坐。

曾鳳韶,廬陵人。洪武末年進士。建文初,嘗為監察御史。燕王稱帝,以原官召,不赴。又以侍郎召,知不可免,乃刺血書衣襟曰:「予生廬陵忠節之邦,素負剛鯁之腸。讀書登進士第,仕宦至繡衣郎。慨一死之得宜,可以含笑於地下,而不愧吾文天祥。」囑妻李氏、子公望:「勿易我衣,即以此殮。」遂自殺,年二十九。李亦守節死。

王良,字天性,祥符人。洪武末,累官僉都御史,坐緩其僚友獄,貶刑部郎中。建文中,歷遷刑部左侍郎。議減燕府人罪,不稱旨,出為浙江按察使。燕王即位,頗德之,遣使召良。良執使者將斬之,眾劫之去。良集諸司印於私第,將自殺,未即決。妻問故。曰:「吾分應死,未知所以處汝耳。」妻曰:「君男子,乃為婦人謀乎?」饋良食。食已,抱其子入後園,置子池旁,投水死。良殮妻畢,以子付友人家,遂積薪自焚,印俱毀。成祖曰:「死固良分,朝廷印不可毀。毀印,良不得無罪。」徙其家於邊。

陳思賢,茂名人。洪武末,為漳州教授,以忠孝大義勖諸生。每部使者涖漳,參謁時必請曰:「聖躬安否?」燕王登極詔至,慟哭曰:「明倫之義,正在今日。」堅臥不迎詔。率其徒吳性原、陳應宗、林玨、鄒君默、曾廷瑞、呂賢六人,即明倫堂為舊君位,哭臨如禮。有司執之送京師,思賢及六生皆死。六生皆龍溪人。嘉靖中,提學副使邵銳立祠祀思賢,以六生侑食。

又台州有樵夫,日負薪入市,口不貳價。聞燕王即帝位,慟哭投東湖死。而溫州樂清亦有樵夫,聞京師陷,其鄉人卓侍郎敬死,號慟投於水。二樵皆逸其名。

程通,績溪人。嘗上書太祖,乞除其祖戍籍。詞甚哀,竟獲請。已,授遼府紀善。燕師起,從王泛海歸京師,上封事數千言,陳禦備策,進左長史。永樂初,從王徙荊州。有言其前上封事多指斥者。械至,死於獄。家屬戍邊。并捕其友人徽州知府黃希范,論死,籍其家。

葉惠仲,臨海人。與兄夷仲並有文名,以知縣徵修《太祖實錄》,遷知南昌府。永樂元年,坐直書《靖難》事,族誅。

黃彥清,歙人。官國子博士,以名節自勵。坐在梅殷軍中私謚建文帝,誅死。

蔡運,南康人。歷官四川參政。勁直不諧於俗,罷歸。復起知賓州,有惠政。永樂初,亦追論奸黨死。

石允常,寧海人。洪武二十七年進士。官河南僉事,廉介有聲。坐事謫常州同知。建文末,帥兵防江。軍潰,棄官去。後追錄廢周籓事,繫獄二年。免死戍邊。

高巍,遼州人,尚氣節,能文章。母蕭氏有痼疾,巍左右侍奉,至老無少懈。母死,蔬食廬墓三年。洪武中,旌孝行,由太學生試前軍都督府左斷事。疏墾河南、山東、北平荒田。又條上抑末技、慎選舉、惜名器數事。太祖嘉納之。尋以決事不稱旨,當罪,減死戍貴州關索嶺。特許弟姪代役,曰:「旌孝子也。」

及惠帝即位,上疏乞歸田里。未幾,遼州知州王欽應詔辟巍。巍因赴吏部上書論時政。用事者方義削諸王,獨巍與御史韓郁先後請加恩。略曰:「高皇帝分封諸王,此之古制。既皆過當,諸王又率多驕逸不法,違犯朝制。不削,朝廷綱紀不立;削之,則傷親親之恩。賈誼曰:『欲天下治安,莫如眾建諸侯而少其力。』今盍師其意,勿行晁錯削奪之謀,而效主父偃推恩之策。在北諸王,子弟分封於南;在南,子弟分封於北。如此則籓王之權,不削而自削矣。臣又願益隆親親之禮,歲時伏臘使人饋問。賢者下詔褒賞之。驕逸不法者,初犯容之,再犯赦之,三犯不改,則告太廟廢處之。豈有不順服者哉!」書奏,帝頷之。

已而燕兵起,命從李景隆出師參贊軍務。巍復上書,言:「臣願使燕。披忠膽,陳義禮,曉以禍福,感以親親之誼,令休兵歸籓。」帝壯其言,許之。巍至燕,自稱:

國朝處士高巍再拜上書燕王殿下:太祖上賓,天子嗣位,布維新之政,天下愛戴,皆曰「內有聖明,外有籓翰,成、康之治,再見於今矣。」不謂大王顯與朝廷絕,張三軍,抗六師,臣不知大王何意也。今在朝諸臣,文者智輳,武者勇奮,執言仗義,以順討逆。勝敗之機明於指掌。皆云大王「藉口誅左班文臣,實則吳王濞故智,其心路人所共知。」巍竊恐奸雄無賴,乘隙奮擊,萬一有失,大王得罪先帝矣。今大王據北平,取密雲,下永平,襲雄縣,掩真定。雖易若建瓴,然自兵興以來,業經數月,尚不能出蕞爾一隅地。且大王所統將士,計不過三十萬。以一國有限之眾應天下之師,亦易罷矣。大王與天子義則君臣,親則骨肉,尚生離間。況三十萬異姓之士能保其同心協力,效死於殿下乎?巍每念至此,未始不為大王水麗泣流涕也。

願大王信巍言:上表謝罪,再修親好。朝廷鑒大王無他,必蒙寬宥。太祖在天之靈亦安矣。倘執迷不悟,舍千乘之尊,捐一國之富,恃小勝,忘大義,以寡抗眾,為僥倖不可成之悖事,巍不知大王所稅駕也。況大喪未終,毒興師旅,其與泰伯、夷、齊求仁讓國之義不大逕庭乎?雖大王有肅清朝廷之心,天下不無篡奪嫡統之議。即幸而不敗,謂大王何如人?

巍白髮書生,蜉蝣微命,性不畏死。洪武十七年蒙太祖高皇帝旌臣孝行。巍竊自負:既為孝子,當為忠臣。死忠死孝,巍至願也。如蒙賜死,獲見太祖在天之靈,巍亦可以無愧矣。

書數上,皆不報。

已而景隆兵敗,巍自拔南歸。至臨邑,遇參政鐵鉉,相持痛哭。奔濟南,誓死拒守,屢敗燕兵。及京城破,巍自經死驛舍。

郁疏略曰:

諸王親則太祖遺體,貴則孝康皇帝手足,尊則陛下叔父。使二帝在天之靈,子孫為天子,而弟與子遭殘戮,其心安乎?臣每念至此,未嘗不流涕也。此皆豎儒偏見,病籓封太重,疑慮太深,乃至此。夫脣亡齒寒,人人自危。周王既廢,湘王自焚,代府被摧,而齊臣又告王反矣。為計者必曰:「兵不舉則禍必加」。是朝廷執政激之使然。

燕舉兵兩月矣,前後調兵不下五十餘萬,而一矢無獲。謂之國有謀臣可乎?經營既久,軍興輒乏,將不效謀,士不效力。徒使中原無辜赤子困於轉輸,民不聊生,日甚一日。九重之憂方深,而出入帷幄與國事者,方且揚揚自得。彼其勸陛下削籓國者,果何心哉?諺曰:「親者割之不斷,疏者續之不堅。」殊有理也。陛下不察,不待十年,悔無及矣。

臣至愚,感恩至厚,不敢不言。幸少垂洞鑒,興滅繼絕,釋代王之囚,封湘王之墓,還周王於京師,迎楚、蜀為周公。俾各命世子持書勸燕,罷兵守籓,以慰宗廟之靈。明詔天下,撥亂反正,篤厚親親,宗社幸甚。

不聽。燕師渡江,郁棄官遁去,不知所終。

高賢寧,濟陽儒學生。嘗受學于教諭王省,以節義相砥礪。建文中,貢入太學。燕兵破德州,圍濟南。賢寧適在圍中,不及赴。是時燕兵勢甚張,黃子澄等謀遣使議和以怠之。尚寶司丞李得成者,慷慨請行,見燕王城下。王不聽,圍益急。參政鐵鉉等百計禦之。王射書城中諭降。賢寧作《周公輔成王論》,射城外。王悅其言,為緩攻。相持兩月,卒潰去。燕王即位後,賢寧被執入見。成祖曰:「此作論秀才耶?秀才好人,予一官。」賢寧固辭。錦衣衛指揮紀綱,故劣行被黜生也,素與賢寧善,勸就職。賢寧曰:「君為學校所棄,故應爾。我食廩有年,義不可,且嘗辱王先生之教矣。」綱為言於帝,竟得歸,年九十七卒。

王璡,字器之,日照人。博通經史,尤長於《春秋》。初為教授,坐事謫遠方。洪武末,以賢能薦,授寧波知府。夜四鼓即秉燭讀書,聲徹署外。間詣學課諸生,諸生率四鼓起,誦習無敢懈。毀境內淫祠,三皇祠亦在毀中,或以為疑。璡曰:「不當祠而祠曰『淫』,不得祠而祠曰『瀆』。惟天子得祭三皇,於士庶人無預,毀之何疑。」自奉儉約,一日饌用魚羹,璡謂其妻曰:「若不憶吾啖草根時耶?」命撤而埋之,人號「埋羹太守。」燕師臨江,璡造舟艦謀勤王,為衛卒縛至京。成祖問:「造舟何為?」對曰:「欲泛海趨瓜洲,阻師南渡耳。」帝亦不罪,放還里,以壽終。

周縉,字伯紳,武昌人。以貢入太學,授永清典史,攝令事。成祖舉兵,守令相率迎降。永清地尤近,縉獨為守禦計。已,度不可為,懷印南奔。道聞母卒,歸終喪。燕兵已迫,糾義旅勤王,聞京師不守,乃走匿。吏部言:「前北平所屬州縣官朱寧等二百九十人,當皇上『靖難』,俱棄職逃亡。宜置諸法。」詔令入粟贖罪,遣戍興州。有司遂捕縉,械送戍所。居數歲,子代還,年八十而沒。朱寧等皆無考。

牛景先,不知何許人。官御史。金川門開,易服宵遁,卒於杭州僧寺。已而窮治齊、黃黨,籍其家。

燕兵之入,一夕朝臣縋城去者四十餘人。其姓名爵里莫可得而考。然世相傳,有程濟及河西傭、補鍋匠之屬。

程濟,朝邑人。有道術。洪武末官岳池教諭。惠帝即位,濟上書言:「某月日北方兵起。」帝謂非所宜言,逮至,將殺之。濟大呼曰:「陛下幸囚臣。臣言不驗,死未晚。」乃下之獄。已而燕兵起,釋之,改官編修。參北征軍淮上,敗,召還。或曰,徐州之捷,諸將樹碑紀功,濟一夜往祭,人莫測。後燕王過徐,見碑大怒,趣左右椎之。再椎,遽曰:「止,為我錄文來。」已,按碑行誅,無得免者。而濟名適在椎脫處。然考其實,徐州未嘗有捷也。金川門啟,濟亡去。或曰帝亦為僧出亡,濟從之。莫知所終。

河西傭,不知何許人。建文四年冬,披葛衣行乞金城市中。已,至河西為傭於莊浪魯氏。取直買羊裘,而以故葛衣覆其上,破縷縷不肯棄。力作倦,輒自吟哦,或夜聞其哭聲。久之,有京朝官至,識人庸,欲與語,走南山避之。或問京朝官:「傭何人?」官亦不答。在莊浪數年,病且死,呼主人屬曰:「我死勿殮。西北風起,火我,勿埋我骨。」魯家從其言。

補鍋匠者,常往來夔州、重慶間。業補鍋,凡數年,川中人多識之。一日,於夔州市遇一人,相顧愕然。已,相持哭,共入山巖中,坐語竟日。復相持哭,別去。其人即馮翁也。翁在夔以章句授童子,給衣食,能為古詩。詩後題「馬二子」,或「馬公」,或「塞馬先生」。後二人皆不知所終。

又會稽有二隱者:一雲門僧,一若耶溪樵。僧每泛舟賦詩,歸即焚之。樵每於溪沙上以荻畫字,已,輒亂其沙。人有疑之者,從後抱持觀之,則皆孤臣去國之詞也。

時又有玉山樵者,居金華之東山,麻衣戴笠,終身不易。嘗為王姓者題詩曰「宗人」,故疑其王姓云。雪庵和尚,人疑其為葉希賢,見《練子寧傳》。

其後數十年,松陽王詔游治平寺,於轉輪藏上得書一卷,載建文亡臣二十餘人事跡。楮墨斷爛,可識者僅九人。梁田玉、梁良玉、梁良用、梁中節皆定海人,同族,同仕於朝。田玉,官郎中,京師破,去為僧。良玉,官中書舍人,變姓名,走海南,鬻書以老。良用為舟師,死於水。中節好《老子》、《太玄經》,為道士。何申、宋和、郭節,俱不知何許人,同官中書。申使蜀,至峽口聞變,嘔血,疽發背死。和及節挾卜筮書走異域,客死。何洲,海州人。不知何官,亦去為卜者,客死。郭良,官籍俱無考,與梁中節相約棄官為道士。餘十一人並失其姓名。縉雲鄭僖紀其事,為《忠賢奇秘錄》,傳於世。

及萬歷時,江南又有《致身錄》,云得之茅山道書中。建文時,侍書吳江史仲彬所述,紀帝出亡後事甚具。仲彬、程濟、葉希賢、牛景先皆從亡之臣。又有廖平、金焦諸姓名,而雪庵和尚、補鍋匠等,具有姓名、官爵。一時士大夫皆信之。給事中歐陽調律上其書於朝,欲為請謚立祠。然考仲彬實未嘗為侍書,《錄》蓋晚出,附會不足信。

贊曰:靖難之役,朝臣多捐軀殉國。若王艮以下諸人之從容就節,非大義素明者不能也。高巍一介布衣,慷慨上書,請歸籓服。其持論甚偉,又能超然遠引,晦跡自全,可稱奇士。若夫行遯諸賢,雖其姓字雜出於諸家傳紀,未足徵信,而忠義奇節,人多樂道之者。《傳》曰:「與其過而去之,寧過而存之。」亦足以扶植綱常,使懦夫有立志也。


\end{pinyinscope}