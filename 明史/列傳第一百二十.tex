\article{列傳第一百二十}

\begin{pinyinscope}
魏允貞弟允中劉廷蘭王國餘懋衡李三才

魏允貞,字懋忠,南樂人。萬曆五年進士。授荊州推官。大學士張居正歸葬,群吏趨事恐後,允貞獨不赴,且抶其奴。治行最,徵授御史。吏部尚書梁夢龍罷。允貞言:「銓衡任重。往者會推之前,所司率受指執政或司禮中官,以故用非其人。」帝納其言,特用嚴清,中外翕服。俄劾兵部尚書吳兌,兌引去。已,陳時弊四事,言:「自居正竊柄,吏、兵二部遷除必先關白,故所用悉其私人。陛下宜與輔臣精察二部之長,而以其職事歸之。使輔臣不侵部臣之權以行其私,部臣亦不乘輔臣之間以自行其私,則官方自肅。自居正三子連登制科,流弊迄今未已。請自今輔臣子弟中式,俟致政之後始許廷對,庶倖門稍杜。自居正惡聞讜言,每遇科道員缺,率擇才性便給、工諂媚、善逢迎者授之,致昌言不聞,佞臣得志。自今考選時,陛下宜嚴敕所司,毋循故轍。俺答自通市以來,邊備懈弛。三軍月餉,既剋其半以充市賞,復剋其半以奉要人,士無宿飽,何能禦寇?至遼左戰功,尤可駭異。軍聲則日振於前,生齒則日減於舊。奏報失真,遷敘逾格,賞罰無章,何以能國哉!」疏入,下都察院。

先是,居正既私其子,他輔臣呂調陽子興周,張四維子泰徵、甲徵,申時行子用懋,皆相繼得舉。甲徵、用懋將廷對,而允貞疏適上。四維大慍,言:「臣待罪政府,無所不當聞。今因前人行私,而欲臣不預聞吏、兵二部事,非制也。」因為子白誣,且乞骸骨。時行亦疏辨。帝並慰留,而責允貞言過當。戶部員外郎李三才奏允貞言是,並貶秩調外。允貞得許州判官。給事中御史周邦傑、趙卿等論救,不納。允貞雖謫,然自是輔臣居位,其子無復登第者。久之,累遷右通政。

二十一年,以右僉都御史巡撫山西。允貞素剛果,清操絕俗。以所部地疹民貧,力裁幕府歲供及州縣冗費,以其銀數萬繕亭障,建烽堠,置器市馬易粟。又奏免平陽歲額站銀八萬,以所省郵傳羨補之。雁門、平定軍以逋屯糧竄徙,允貞奏除其租,招令復業。岢嵐互市,省撫賞銀六萬。汾州有兩郡王,宗人與軍民雜處,知州秩卑不能制,奏改為府。自款市成,邊政廢。允貞視要害,築邊牆萬有餘丈。政聲大著。帝亦數嘉其能。會詔中官張忠礦山西,允貞抗疏極諫,不報。已,西河王知燧請開解州、安邑、絳縣礦,以儀賓督之。指揮王守信請開平定、稷山諸礦。帝並報允。允貞恐民愈擾,請令忠兼領,亦不納。

三殿災,詔求直言。允貞言咎在輔臣,歷數趙志皋、張位罪。且曰:「前二臣以二月加恩,踰月兩宮災。今年又加恩,而三殿復災。天意昭然。」位等力辨,求罷。帝慰留,責允貞邊臣不當言朝事,因屢推不用,遂肆狂言,奪俸五月。頃之,允貞疏舉遺賢,請召還王家屏、陳有年、沈鯉、李世達、王汝訓及小臣史孟麟、張棟、萬國欽、馬經綸、顧憲成、趙南星、鄒元標等,疏留中。以久次,進右副都御史。

二十八年春,疏陳時政缺失,言:「行取諸臣,幾經論薦,陛下猶不輕予一官。彼魯坤、馬堂、高淮、陳朝輩,試之何事,舉之何人,乃令其銜命橫行,生殺予奪,恣出其口。廷臣所陳率國家大計,一皆寢閣,甚者嚴譴隨之。彼報稅之徒,悉無賴奸人,鄉黨不齒,顧乃朝奏夕報,如響應聲。臣不解也。胥徒入鄉,民間猶擾,況緹騎四出,如虎若狼,家室立破。如吳寶秀、華鈺諸人,禍至慘矣,而陛下曾不一念及。錢穀出入,上下相稽,猶多奸弊。敕使手握利權,動逾數萬。有司不敢問,撫按不敢聞,豈無吮膏血以自肥者,而陛下曾不一察及。金取於滇,不足不止;珠取於海,不罄不止;錦綺取於吳越,不極奇巧不止。乃元老聽其投閒,直臣幾於永錮,是陛下之愛賢士,曾不如愛珠玉錦綺也。」疏奏,亦不省。

先是,張忠以開礦至,後孫朝復至榷稅,誅求百方,允貞每事裁抑。會忠杖死太平典史武三傑,朝使者逼殺建雄縣丞李逢春,允貞疏暴其罪。朝怒,劾允貞抗命沮撓。帝留允貞疏不下,而下朝疏於部院。吏部尚書李戴、都御史溫純等力稱允貞賢,請下允貞疏平議。帝並留中。山西軍民數千恐允貞去,相率詣闕愬冤,兩京言官亦連章論救。帝乃兩置不問。明年,忠以夏縣知縣彭應春抗禮,劾貶之。允貞請留應春,不報。

允貞父已九十餘,允貞歲歲乞侍養,章二十上。廷議以敕使害民,非允貞不能制,固留之。其年五月,請益力,始聽歸。士民為立祠。已,閱視者奏允貞守邊勞,即家進兵部右侍郎。尋卒。天啟初,追謚介肅。

弟允中、允孚。允中為諸生,副使王世貞大器之。歲鄉試,世貞戒門吏曰:「非魏允中第一,無伐鼓以傳也。」已而果然。時無錫顧憲成、漳浦劉廷蘭並為舉首,負俊才,時人稱「三解元」。尋與廷蘭舉萬曆八年進士。張居正專政,災異見,而中外方競頌功德。允中、廷蘭各上書座主申時行,勸之補救。時行不能用。允中尋授太常博士,擢吏部稽勳主事,調考功,未幾卒。允孚官刑部郎中,亦有名。

廷蘭與兄廷蕙、廷芥亦皆舉進士,有名。世所稱「南樂三魏」、「漳浦三劉」者也。

王國,字子楨,耀州人。萬曆五年進士。選庶吉士,改御史。出視畿輔屯田,清成國公朱允禎等所侵地九千六百餘頃。張居正疾篤,疏薦其座主潘晟入內閣,帝從之。國與同官魏允貞、雷士楨及給事中王繼光、孫煒、牛惟炳、張鼎思抗言不可,寢其命。已,極論中官馮保罪。且言:「居正死,保令徐爵索其家名琴七、夜光珠九、珠簾五、黃金三萬、白金十萬。居正子簡修躬齎至保邸,而保揚言陛下取之,誣汙聖德。」因發曾省吾、王篆表裏結納狀。國疏自外至,與李植疏先後上。帝已納植言罪保,植遂受知,而國亦由此顯名。還朝,薦王錫爵、陸樹聲、胡執禮、耿定向、海瑞、胡直、顏鯨、魏允貞。尋出督南畿學政,以疾歸。

起掌河南道。首輔申時行欲置所不悅者十九人察典,吏部尚書楊巍等依違其間,國力持不可。時行以御史馬允登資在國前,乃起允登掌察,而國佐之。諸御史咸集,允登書十九人姓名,曰:「諸人可謂公論不容者矣。」國熟視,叱曰:「諸人獨忤執政耳。天日監臨,何出此語!」允登意不回。國怒,奮前欲毆允登。允登走,國環柱逐之,同列救解。事聞,兩人並調外,國得四川副使。移疾歸。而十九人賴國以免。

久之,起故官,蒞山西。改督河南學政,遷山東參政。所在以公廉稱。召為太僕少卿。復出為山西副使,歷南京通政使。三十七年,以兵部右侍郎兼右僉都御史巡撫保定。歲凶,屢上寬恤事宜。大盜劉應第、董世耀聚眾稱王,剽劫遠近,督兵討滅之。進右都御史,巡撫如故。國剛介。與弟吏部侍郎圖並負時望,為黨人所忌。乞休歸,卒。

餘懋衡,字持國,婺源人。萬曆二十年進士。除永新知縣。征授御史。時以殿工,礦稅使四出,驕橫。懋衡上疏言:「與其騷擾里巷,榷及雞豚,曷若明告天下,稍增田賦,共襄殿工。今避加賦之名,而為竭澤之計,其害十倍於加賦。」忤旨,停俸一年。

巡按陜西。稅監梁永輦私物於畿輔,役人馬甚眾。懋衡奏之。永大恨,使其黨樂綱賄膳夫毒懋衡。再中毒,不死。拷膳夫,獲所予賄及餘蠱。遂上疏極論永罪,言官亦爭論永,帝皆不省。永慮軍民為難,召亡命擐甲自衛。御史王基洪聲言永必反,具陳永斬關及殺掠吏民狀。巡撫顧其志頗為永諱,永乃藉口辨。帝疑御史言不實。而咸寧、長安二知縣持永益急。永黨王九功輩多私裝,恐為有司所跡,託言永遣,乘馬結陣馳去。縣棣追及之華陰,相格鬥,已皆被繫,懋衡遂以反逆聞。永窘甚,爪牙盡亡,獨綱在,乃教永誣劾咸寧知縣滿朝薦,朝薦被逮。永不久亦撤還,關中始靖。懋衡尋以憂歸。起掌河南道事。擢大理右寺丞,引疾去。

天啟元年,起歷大理左少卿,進右僉都御史,與尚書張世經共理京營戎政。進右副都御史,改兵部右侍郎,俱理戎政。三年八月,廷推南京吏部尚書,以懋衡副李三才;推吏部左侍郎,以曹于汴副馮從吾。帝皆用副者。大學士葉向高等力言不可,弗聽。懋衡、於汴亦以資後三才等,力辭新命,引疾歸。明年十月,再授前職。懋衡以璫勢方張,堅臥不起。既而奸黨張訥醜詆講學諸臣,以懋衡、從吾及孫慎行為首,遂削奪。崇禎初,復其官。

李三才,字道甫,順天通州人。萬曆二年進士。授戶部主事,歷郎中。與南樂魏允貞、長垣李化龍以經濟相期許。及允貞言事忤執政,抗疏直之,坐謫東昌推官。再遷南京禮部郎中。會允貞、化龍及鄒元標並官南曹,益相與講求經世務,名籍甚。遷山東僉事。所部多大猾積盜,廣設方略,悉擒滅之。遷河南參議,進副使。兩督山東、山西學政,擢南京通政參議,召為大理少卿。

二十七年,以右僉都御史總督漕運,巡撫鳳陽諸府。時礦稅使四出。三才所部,榷稅則徐州陳增、儀真暨祿,鹽課則揚州魯保,蘆政則沿江邢隆,棋布千里間。延引奸徒,偽鍥印符,所至若捕叛亡,公行攘奪。而增尤甚,數窘辱長吏。獨三才以氣凌之,裁抑其爪牙肆惡者,且密令死囚引為黨,輒捕殺之,增為奪氣。然奸民以礦稅故,多起為盜。浙人趙一平用妖術倡亂。事覺,竄徐州,易號古元,妄稱宋後。與其黨孟化鯨、馬登儒輩聚亡命,署偽官,期明年二月諸方並起。謀洩,皆就捕。一平亡之寶坻,見獲。三才再疏陳礦稅之害,言:「陛下愛珠玉,民亦慕溫飽;陛下愛子孫,民亦戀妻孥。奈何陛下欲崇聚財賄,而不使小民享升斗之需;欲綿祚萬年,而不使小民適朝夕之樂。自古未有朝廷之政令、天下之情形一至於斯,而可幸無亂者。今闕政猥多,而陛下病源則在溺志貨財。臣請渙發德音,罷除天下礦稅。欲心既去,然後政事可理。」踰月未報,三才又上言:「臣為民請命,月餘未得請。聞近日章奏,凡及礦稅,悉置不省。此宗社存亡所關,一旦眾畔土崩,小民皆為敵國,風馳塵騖,亂眾麻起,陛下塊然獨處,即黃金盈箱,明珠填屋,誰為守之?」亦不報。三十年,帝有疾,詔罷礦稅,俄止之。三才極陳國勢將危,請亟下前詔,不聽。

清口水涸阻漕,三才議浚渠建閘,費二十萬,請留漕粟濟之。督儲侍郎趙世卿力爭,三才遂引疾求去。帝惡其委避,許之。淮揚巡按御史崔邦亮、巡漕御史李思孝、給事中曹於汴、御史史學遷、袁九皋交章乞留。而學遷言:「陛下以陳增故,欲去三才,託詞解其官。年來中使四出,海內如沸。李盛春之去以王虎,魏允貞之去以孫朝,前漕臣李誌之去亦以礦稅事。他監司守令去者,不可勝數,今三才復繼之。淮上軍民以三才罷,欲甘心於增,增避不敢出。三才不當去可知。」疏仍不答。三才遂去淮之徐州。連疏請代,未得命。會侍郎謝傑代世卿督儲,復請留。乃命三才供事俟代者,帝亦竟不遣代也。

明年九月,復疏言:「乃者迅雷擊陵,大風拔木,洪水滔天,天變極矣。趙古元方磔於徐,李大榮旋梟於亳,而睢州巨盜又復見告,人離極矣。陛下每有徵求,必曰『內府匱乏』。夫使內府果乏,是社稷之福也,所謂貌病而天下肥也。而其實不然。陛下所謂匱乏者,黃金未遍地,珠玉未際天耳。小民饔飧不飽,重以徵求,箠楚無時,桁楊滿路,官惟丐罷,民惟請死,陛下寧不惕然警悟邪!陛下毋謂臣禍亂之言為未必然也;若既已然矣,將置陛下何地哉!」亦不報。既而睢盜就獲,三才因奏行數事,部內晏然。

翕人程守訓以貲官中書,為陳增參隨。縱橫自恣,所至鼓吹,盛儀衛,許人告密,刑拷及婦孺。畏三才,不敢至淮。三才劾治之,得贓數十萬。增懼為己累,并搜獲其奇珍異寶及僭用龍文服器。守訓及其黨俱下吏伏法,遠近大快。

三十四年,皇孫生。詔併礦稅,釋逮繫,起廢滯,補言官,既而不盡行。三才疑首輔沈一貫尼之,上疏陰詆一貫甚力。繼又言:「恩詔已頒,旋復中格,道路言前日新政不過乘一時喜心,故旋開旋蔽。」又謂:「一貫慮沈鯉、朱賡逼己。既忌其有所執爭,形己之短,又恥其事不由己,欲壞其成。行賄左右,多方蠱惑,致新政阻格。」帝得疏,震怒。嚴旨切責,奪俸五月。其明年,暨祿卒。三才因請盡撤天下稅使,帝不從,命魯保兼之。

是時顧憲成里居,講學東林,好臧否人物。三才與深相結,憲成亦深信之。三才嘗請補大僚,選科道,錄遺佚。因言:「諸臣只以議論意見一觸當途,遂永棄不收,要之於陛下無忤。今乃假天子威以錮諸臣,復假忤主之名以文己過。負國負君,罪莫大此。」意為憲成諸人發。已,復極陳朝政廢壞,請帝奮然有為,與天下更始。且力言遼左阽危,必難永保狀。帝皆置不省。

三才揮霍有大略,在淮久,以折稅監得民心。及淮、徐歲侵,又請振恤,蠲馬價。淮人深德之。屢加至戶部尚書。會內閣缺人,建議者謂不當專用詞臣,宜與外僚參用,意在三才。及都御史缺,需次內召。由是忌者日眾,謗議紛然。工部郎中邵輔忠遂劾三才大奸似忠,大詐似直,列具貪偽險橫四大罪,御史徐兆魁繼之。三才四疏力辨,且乞休。給事中馬從龍、御史董兆舒、彭端吾、南京給事中金士衡相繼為三才辨。大學士葉向高言三才已杜門待罪,宜速定去留,為漕政計。皆不報。已而南京兵部郎中錢策,南京給事中劉時俊,御史劉國縉、喬應甲,給事中王紹徵、徐紹吉、周永春、姚宗文、硃一桂、李瑾,南京御史張邦俊、王萬祚,復連章劾三才。而給事中胡忻、曹于汴,南京給事中段然,御史史學遷、史記事、馬孟禎、王基洪,又交章論救。朝端聚訟,迄數月未已。憲成乃貽書向高,力稱三才廉直,又貽書孫丕揚力辨之。御史吳亮素善三才,即以兩書附傳邸報中,由是議者益嘩。應甲復兩疏力訐,至列其十貪五奸。帝皆不省。三才亦力請罷,疏至十五上。久不得命,遂自引去。帝亦不罪也。

三才既家居,忌者慮其復用。四十二年,御史劉光復劾其盜皇木營建私第至二十二萬有奇。且言三才與於玉立遙執相權,意所欲用,銓部輒為推舉。三才疏辨,請遣中官按問。給事中劉文炳、御史李徵儀、工部郎中聶心湯、大理丞王士昌,助光復力攻三才。徵儀、心湯,三才嘗舉吏也。三才憤甚,自請籍其家。工部侍郎林如楚言宜遣使覆勘。光復再疏,并言其侵奪官廠為園囿。御史劉廷元遂率同列繼之,而潘汝禎又特疏論劾。既而巡按御史顏思忠亦上疏如光復指。三才益憤,請諸臣會勘,又請帝親鞫。乃詔徵儀偕給事中吳亮嗣往。

其明年,光復坐事下獄。三才陽請釋之,而復力為東林辨白,曰:「自沈一貫假撰妖書,擅僇楚宗,舉朝正人攻之以去。繼湯賓尹、韓敬科場作奸,孽由自取,於人何尤。而今之黨人動與正人為仇,士昌、光復尤為戎首。挺身主盟,力為一貫、敬報怨。騰說百端,攻擊千狀。以大臣之賢者言之,則葉向高去矣,王象乾、孫瑋、王圖、許弘綱去矣,曹于汴、胡忻、朱吾弼、葉茂才、南企仲、朱國禎等去矣,近又攻陳薦、汪應蛟去矣。以小臣之賢者言之,梅之煥、孫振基、段然、吳亮、馬孟禎、湯兆京、周起元、史學遷、錢春等去矣,李朴、鮑應鰲、丁元薦、龐時雍、吳正志、劉宗周等去矣。合於己則留,不合則逐。陛下第知諸臣之去,豈知諸黨人驅之乎?今奸黨仇正之言,一曰東林,一曰淮撫。所謂東林者,顧憲成讀書講學之所也。從之游者如高攀龍、姜士昌、錢一本、劉元珍、安希范、岳元聲、薛敷教,並束身厲名行,何負國家哉?偶曰東林,便成陷井。如鄒元標、趙南星等被以此名,即力阻其進。所朝上而夕下者,惟史繼偕諸人耳。人才邪正,實國祚攸關,惟陛下察焉。」疏入,眾益恨之。亮嗣等既往勘,久之無所得。第如光復言還報,遂落職為民。

天啟元年,遼陽失。御史房可壯連疏請用三才。有詔廷臣集議。通政參議吳殿邦力言不可用,至目之為盜臣。御史劉廷宣復薦三才,言:「國家既惜其才,則用之耳,又何議,然廣寧已有王化貞,不若用之山海。」帝是其言,即欲用三才,而廷議相持未決。詹事公鼐力言宜用,刑部侍郎鄒元標、僉都御史王德完並主之。已,德完迫眾議,忽變前說。及署議,元標亦不敢主。議竟不決,事遂寢。三年,起南京戶部尚書,未上卒。後魏忠賢亂政,其黨御史石三畏追劾之。詔削籍,奪封誥。崇禎初復官。

三才才大而好用機權,善籠絡朝士。撫淮十三年,結交遍天下。性不能持廉,以故為眾所毀。其後擊三才者,若邵輔忠、徐兆魁輩,咸以附魏忠賢名麗逆案。而推轂三才,若顧憲成、鄒元標、趙南星、劉宗周,皆表表為時名臣。故世以三才為賢。

贊曰:朋黨之成也,始於矜名,而成於惡異。名盛則附之者眾。附者眾,則不必皆賢而胥引之,樂其與己同也。名高則毀之者亦眾。毀者不必不賢而怒而斥之,惡其與己異也。同異之見岐於中,而附者毀者爭勝而不已,則黨日眾,而為禍熾矣。魏允貞、王國、餘懋衡皆以卓犖閎偉之概,為眾望所歸。李三才英遇豪俊,傾動士大夫,皆負重名。當世黨論之盛,數人者實為之魁,則好同惡異之心勝也。《易》曰:「渙其群,元吉。」知此者,其惟聖人乎!


\end{pinyinscope}