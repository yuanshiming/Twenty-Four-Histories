\article{列傳第一百九}

\begin{pinyinscope}
袁洪愈子一鶚譚希思王廷瞻郭應聘吳文華耿定向弟定理定力王樵子肯堂魏時亮陳瓚郝傑胡克儉趙參魯張孟男衛承芳李禎丁賓

袁洪愈,字抑之,吳縣人。舉嘉靖二十五年鄉試第一。明年成進士,授中書舍人。擢禮科給事中。劾檢討梁紹儒阿附權要,文選郎中白璧招權鬻官,尚書萬鏜、侍郎葛守禮不檢下。詔切責鏜、守禮,下璧詔獄,斥紹儒於外。紹儒,大學士嚴嵩私人也。已,陳邊務數事,詔俱從之。嵩屬吏部尚書吳鵬,出為福建僉事。歷河南參議、山東提學副使、湖廣參政,所在以清節著。嵩敗,召為南京太僕少卿,就遷太常。隆慶五年,以疾歸。

萬曆中,起故官,遷南京工部右侍郎,進右都御史,掌南院事,就改禮部尚書。南京御史譚希思疏論中官、外戚,且請循舊制,內閣設絲綸簿,宮門置鐵牌。詔下南京都察院勘訊,將坐以誣罔。洪愈已改官,代者未至,乃具言希思所陳,載王可大《國憲家猷》、薛應旂《憲章錄》二書。帝以所據非頒行制書,謫希思雜職。洪愈尋上疏請禁干謁,又極諫屯田廢壞之害,乞令商人中鹽,免內地飛挽。皆議行。

萬曆十五年,就改吏部。其冬引年乞休。帝重其清德,加太子少保致仕。洪愈通籍四十餘年,所居不增一椽,出入徒步。卒,年七十四。巡撫周孔教捐金葬之。贈太子太保,謚安節。

子一鶚,以廕,官治中。饘粥不繼以死。

希思,茶陵人。歷右副都御史,巡撫四川。

王廷瞻,字稚表,黃岡人。父濟,參政。廷瞻舉嘉靖三十八年進士,授淮安推官。入為御史,督畿輔屯政。穆宗在裕邸,欲易莊田,廷瞻不可。隆慶元年,所部久雨。請自三宮以下及裕府莊田改入乾清宮者,悉蠲其租。詔減十之五。已,言勛戚莊田太濫,請於初給時裁量田數,限其世次,爵絕歸官。制可。高拱再輔政,廷瞻常論拱,遂引疾歸。神宗立,起故官。歷太僕卿。萬曆五年,以右僉都御史巡撫四川。番屢犯松潘。廷瞻令副使楊一桂、總兵官劉顯剿之,殲其魁,群蠻納款。風村、白草諸番,久居二十八砦,率男婦八千餘人來降。復命總兵顯討建昌、傀廈、洗馬、姑宰、鐵口諸叛番,皆獻首惡出降。增俸一級,進右副都御史,撫南、贛。

入為南京大理卿。歷兩京戶部左、右侍郎,以右都御史出督漕運兼巡撫鳳陽諸府。寶應氾光湖堤蓄水濟運,平江伯陳瑄所築也。下流無所洩,決為八淺,匯成巨潭,諸鹽場皆沒。淮流復奔入,勢益水匈水勇前巡撫李世達等議開越河避其險,廷瞻承之。鑿渠千七百七十六丈,為石閘三,減水閘二,石堤三千三十六丈,子堤五千三百九十丈,費公帑二十餘萬,八月竣事。詔旨褒嘉,賜河名弘濟。進廷瞻戶部尚書,巡撫如故。尋改南京刑部尚書。未上,乞歸。久之卒。贈太子少保。兄廷陳,見《文苑傳》。

郭應聘,字君賓,莆田人。嘉靖二十九年進士。授戶部主事。歷郎中,出為南寧知府。遷威茂兵備副使,轉廣東參政。從提督吳桂芳平李亞元,別擊賊首張韶南、黃仕良等。遷廣西按察使,歷左、右布政使。隆慶四年大破古田賊,斬獲七千有奇。已,從巡撫殷正茂平古田,再進秩。

正茂遷總督,遂擢應聘右副都御史代之。府江瑤反。府江上起陽朔,下達昭平,亙三百餘里。諸瑤夾江居,怙險剽劫。成化、正德間,都御史韓雍、陳金討平之,至是攻圍荔浦、永安,劫知州楊惟執、指揮胡翰。事聞,大學士張居正奏假便宜,寓書應聘曰:「炎荒瘴癘區,役數萬眾,不宜淹留,速破其巢,則餘賊破膽。」應聘集土、漢兵六萬,令總兵官李錫進討。未行,而懷遠瑤亦殺知縣馬希武反。應聘與正茂議先征府江,三閱月悉定,乃檄錫討懷遠。天大雨雪,無功而還。懷遠,古牂牁,地界湖、貴靖、黎諸州,環郭皆瑤,編氓處其外。嘉靖中,征之不克,知縣寄居府城,遙示羈縻而已。古田既復,瑤懾兵威,願服屬,希武始入其地。議築城,董作過峻,瑤遂亂,希武見殺。及是,師出無功。應聘益調諸路兵,鎮撫白杲、黃土、大梅、青淇侗、僮,以孤賊勢,而錫與諸將連破賊,斬其魁,懷遠乃下。事皆具錫傳。初議行師,錫以陽朔金寶嶺賊近,欲先滅之。應聘曰:「君第往,吾自有處。」錫行數日,應聘與按察使吳一介出不意襲殺其魁。比懷遠克復,陽朔亦定,乃分遣諸將門崇文、楊照、亦孔昭等討洛容、上油、邊山。五叛瑤悉平。神宗大悅,進兵部右侍郎兼右副都御史,巡撫如故。

萬曆二年,召為戶部右侍郎,尋以憂歸。八年起,改兵部,兼右僉都御史,仍撫廣西。時十寨初下,應聘與總督劉堯誨奏設三鎮,隸賓州,以土巡檢守之,而統於思恩參將,十寨遂安。進右都御史兼兵部右侍郎,總督兩廣軍務。前總督多受將吏金,應聘悉謝絕。踰年,召掌南京都察院,以吳文華代。頃之,就拜兵部尚書,參贊機務。久之,引疾歸。應聘在廣西,奏復陳獻章、王守仁祠。劉臺謫戍潯州,為僦居供廩,歿復賻斂歸其喪,像祀之。官南京,與海瑞敦儉素,士大夫不敢侈汰。歸七月卒。贈太子少保,謚襄靖。

吳文華,字子彬,連江人。父世澤,府江兵備副使,有威名。文華舉嘉靖三十五年進士,授南京兵部主事。歷四川右參政,與平土官鳳繼祖。四遷河南左布政使。萬歷三年,以右副都御史巡撫廣西。討平南鄉、陸平、周塘、板寨瑤及昭平黎福莊父子。偕總督凌雲翼征河池、咘咳、北三瑤。三瑤未為逆,雲翼喜事,殺戮甚慘,得蔭襲,文華亦受賞。遷戶部右侍郎,請終養歸。起兵部右侍郎兼右僉都御史,仍撫廣西。遷總督兩廣軍務,巡撫廣東。進右都御史。會巡撫吳善、總兵呼良朋討平嚴秀珠。岑崗賊李珍、江月照拒命久,文華購擒月照,平珍。尋入為南京工部尚書,就改兵部。引疾去。仍起南京工部,力辭,虛位三年以待。卒,年七十八。贈太子少保,謚襄惠。

耿定向,字在倫,黃安人。嘉靖三十五年進士。除行人,擢御史。嚴嵩父子竊政,吏部尚書吳鵬附之。定向疏鵬六罪,因言鵬婿學士董份總裁會試,私鵬子紹,宜併斥。嵩為營護,事竟寢。出按甘肅,舉劾無所私。去任,行笥一肩。有以石經餽者,留境上而去。還督南京學政。隆慶初,擢大理右寺丞。高拱執政,定向嘗譏其褊淺無大臣度,拱嗛之。及拱掌吏部,以考察謫定向橫州判官。拱罷,量移衡州推官。萬曆中,累官右副都御史。吏部侍郎陸光祖為御史周之翰所劾,光祖已留,定向復頌光祖賢,詆之翰。給事中李以謙言定向擠言官,定向求去,帝不問。歷刑部左、右侍郎,擢南京右都御史。御史王籓臣劾應天巡撫周繼,疏發逾月不以白定向。定向怒,守故事力爭,自劾求罷,且詆籓臣論劾失當。因言故江西巡撫陳有年、四川巡撫徐元泰皆賢,為御史方萬山、王麟趾劾罷,今宜召用,而量罰籓臣。籓臣坐停俸二月。於是給事中許弘綱、觀政進士薛敷教、南京御史黃仁榮及麟趾連章劾定向。麟趾言:「南臺去京師遠,章疏先傳,人得為計。如御史孫鳴治論魏國公徐邦瑞,陳揚善論主事劉以煥,皆因奏辭豫聞,一則夤緣倖免,一則捃摭被誣。故邇來投揭有遲浹月者,事理宜然,非自籓臣始。」語並侵大學士許國、左都御史吳時來、副都御史詹仰庇。執政方惡言者,勒敷教還籍省過,麟趾、仁榮亦停俸。時已除定向戶部尚書督倉場,定向因力辭求退。章屢上,乃許。卒,年七十三。贈太子少保,謚恭簡。

定向初立朝有時望。後歷徐階、張居正、申時行、王錫爵四輔,皆能無齟齬。至居正奪情,寓書友人譽為伊尹而貶言者,時議訾之。其學本王守仁。嘗招晉江李贄於黃安,後漸惡之,贄亦屢短定向。士大夫好禪者往往從贄遊。贄小有才,機辨,定向不能勝也。贄為姚安知府,一旦自去其髮,冠服坐堂皇,上官勒令解任。居黃安,日引士人講學,雜以婦女,專崇釋氏,卑侮孔、孟。後北遊通州,為給事中張問達所劾,逮死獄中。

定向弟定理、定力。定理終諸生。與定向俱講學,專主禪機。定力,隆慶中進士,除工部主事。萬曆中,累官右僉都御史,督操江,疏陳礦使之患。再遷南京兵部右侍郎。卒,贈尚書。

王樵,字明遠,金壇人。父臬,兵部主事。諫武宗南巡,被杖。終山東副使。樵舉嘉靖二十六年進士,授行人。歷刑部員外郎。著《讀律私箋》,甚精核。胡宗憲計降汪直,欲赦直以示信。樵言此叛民與他納降異,直遂誅。遷山東僉事,移疾歸。萬曆初,張居正柄國,雅知樵,起補浙江僉事,擢尚寶卿。劉臺劾居正,居正乞歸。諸曹奏留之,樵獨請全諫臣以安大臣,略言:「自古明主欲開言路,言不當,猶優容之;大臣欲廣上德,人攻己,猶薦拔之。如宋文彥博於唐介是也。今居正留而臺得罪,無乃非仁宗待唐介意乎!」居正大恚,出為南京鴻臚卿。旋因星變自陳,罷之。家居十餘年,起南京太僕少卿,時年七十餘矣。歲中再遷大理卿,尋拜南京刑部右侍郎。誠意伯劉世延主使殺人,樵當世延革任。尋就擢右都御史。給事中盧大中劾其衰老,帝令致仕。

樵恬澹誠愨,溫然長者。邃經學,《易》、《書》、《春秋》皆有纂述。卒,贈太子少保,謚恭簡。

子肯堂,字宇泰。舉萬曆十七年進士,選庶吉士,授檢討。倭寇朝鮮,疏陳十議,願假御史銜練兵海上。疏留中,因引疾歸。京察,降調。家居久之,吏部侍郎楊時喬薦補南京行人司副。終福建參政。肯堂好讀書,尤精於醫,所著《證治準繩》該博精粹,世競傳之。

魏時亮,字工甫,南昌人。嘉靖三十八年進士。授中書舍人,擢兵科給事中。隆慶元年正月七日,有詔免朝,越三日,復傳免。時亮以新政不宜遽怠,上疏切諫。尋以左給事中副檢討許國使朝鮮。故事,王北面聽詔,使者西面。時亮爭之,乃南面宣詔。還,進戶科給事中,因列上遼東事宜。已,請慎起居,罷游宴,日御便殿省章奏,召大臣裁決。報聞。興都莊地八千三百頃,中官奪民田,復增八百頃,立三十六莊。帝從撫按奏,屬有司徵租,還兼併者於民。中官張堯為請,又許之。時亮極諫,不納。

帝臨朝,拱默未嘗發一言。及石州陷,有請帝詰問大臣者。越二日,講罷,帝果問石州破狀。中官王本輒從旁詬諸臣欺蔽。帝慍,目懾之,本猶刺刺語。帝不悅而罷。時亮劾本無人臣禮,大不敬,且數其不法數事。疏雖不行,士論壯之。十月初,詔停日講。時亮率同列言未冱寒,不宜遽輟。俄請以薛瑄、陳獻章、王守仁從祀文廟,章下所司。又言方春東作,宜敕有司釋輕繫,停訟獄,詔可。

明年六月言:「今天下大患三:籓祿不給也,邊餉不支也,公私交困也。宗籓有一時之計,有百世之計。亟立宗學,教之禮讓,祿萬石者歲捐五之一,二千石者十之一,千石者二十之一,以贍貧宗,立為定制。此一時計也。各宗聚居一城,貧日益甚,宜令就近散處,給閒田使耕以代祿;奸生之孽,重行黜削。此百世計也。邊餉莫要於屯鹽,近遴大臣龐尚鵬、鄒應龍、凌儒經理,事權雖重,顧往河東者兼理四川,往江北者兼理山東、河南,往江南者兼理浙、湖、雲、貴,重內地而輕塞下,非初旨也。且一人領數道,曠遠難周。請在內地者專責巡撫,令尚鵬等三人分任塞下屯事,久任責成,有功待以不次,則利興而邊儲自裕。今天下府庫殫虛,百姓困瘁,而建議者欲罄天下庫藏輸內府,以濟旦夕之用。脫州郡有變,何以待之?夫守令以養民為職,要在勸農桑、清徭賦、重鄉約、嚴保甲,而簿書獄訟、催科巧拙不與焉。」疏上,多議行。其冬,復疏言:「天下可憂在民窮,能為民紓憂者,知府而已,宜慎重其選。治行卓越,即擢京卿若巡撫,則人自激勸。督學者,天下名教所繫,當擇學行兼懋者,毋限以時。教行望峻,則召為祭酒或入翰林,以示風勵。」下部議,卒不行。

三年,擢太僕少卿。初,徐階、高拱相構,時亮與朝臣攻去拱。已而拱復入,考察言官,排異己者;時亮及陳瓚、張檟已擢京卿,皆被斥。時亮坐不謹,落職。萬曆十二年,用丘橓、餘懋學等薦,起南京大理丞。累遷右副都御史,攝京營戎政,陳安攘要務十四事。尋請以水利、義倉、生養、賦役、清獄、弭盜、善俗七條課守令,歲終報部院及科,計吏時以修廢定殿最。又請皇長子出閣講學。歷刑部左、右侍郎,拜南京刑部尚書。踰年卒官。

時亮初好交遊,負意氣。嘗劾罷左都御史張永明,為時論所非。時亮亦悔之。中遭挫抑,潛心性理之學。天啟中,謚莊靖。

陳瓚,字廷稞,常熟人。嘉靖三十五年進士。授江西永豐知縣。治最,擢刑科給事中。劾罷嚴嵩黨祭酒王才、諭德唐汝楫。遷左給事中。劾文選郎南軒,請錄建言廢斥者。帝震怒,杖六十除名。隆慶元年,起官吏科,請恤楊最、楊爵、羅洪先、楊繼盛,而誅奸黨之殺沈煉者。帝可之,楊順、路楷皆逮治。其冬,擢太常少卿。高拱惡瓚為徐階所引,瓚已移疾歸,竟坐浮躁謫洛川丞,不赴。萬曆中,累官刑部左侍郎。初,瓚為拱所惡,被斥,及張居正柄政,亦惡之,不召。居正死,始以薦起會稽縣丞。其後官侍郎。稽勳郎顧憲成疏論時弊謫官,瓚責大學士王錫爵曰:「憲成疏最公,何以得譴?」錫爵曰:「彼執書生之言,徇道旁之口,安知吾輩苦心。」瓚曰:「恐書生之言當信,道旁之口當察,憲成苦心亦不可不知也。」錫爵默然。瓚前後忤執政如此。卒官,贈右都御史,謚莊靖。賈見《鄒應龍傳》。

郝傑,字彥輔,蔚州人。父銘,御史。傑舉嘉靖三十五年進士,授行人,擢御史。隆慶元年,巡撫畿輔。冬,寇大入永平,疏請蠲被掠地徭賦,且言:「比年罰行於文臣而於武弁,及於主帥而略於偏裨,請飭法以振國威。」俱報可。已,劾薊督劉燾、巡撫耿隨卿觀望,寇退則斷死者報首功,又奪遼東將士棒槌崖戰績,並論副使沈應乾,遊擊李信、周冕罪。帝為黜應乾,下信、冕獄,敕燾、隨卿還籍聽勘。詔遣中官李祐督蘇、杭織造,工部執奏,不從。傑言:「登極詔書罷織造甫一年,敕使復遣,非畫一之政。且內臣專恣,有司剝下奉之,損聖德非小。」帝終不聽。駕幸南海子,命京營諸軍盡從。徐階、楊博等諫,不聽,傑復爭之,卒不從。刑部侍郎洪朝選以拾遺罷,上疏自辨,傑等劾其違制,遂削職。以嘗論高拱非宰輔器,為所嫉。及拱再召,傑遂請急去。拱罷,起故官。旋以私議張居正逐拱非是,出為陜西使。再遷山東左布政使。被劾,降遼東苑馬寺卿兼海道兵備,加山東按察使。

十七年,擢右僉都御史,巡撫遼東。以督諸將擊敵,錄一子官。時李成梁為總兵官,威望甚著,然上功不無抵欺。寇入塞,或斂兵避,既退,始尾襲老弱,或乘虛搗零部,誘殺附塞者充首功,習以為常。督撫諸臣庇之,傑獨不與比。十九年春,成梁用參將郭夢徵策,使副將李寧襲板升於鎮夷堡,獲老弱二百八十餘級。師旋,為別部所遮,寧先走,將士數千人失亡大半,成梁飾功邀敘。傑具奏草,直言其故,要總督蹇達共奏。達匿其草,自為奏論功。巡按御史胡克儉馳疏劾寧,詞連成梁,亦詆傑。兵部置寧罪不議。克儉大憤,盡發成梁、達隱蔽狀。先是,十八年冬,海州被掠十三日,副將孫守廉不戰,成梁亦弗救。克儉既劾守謙,申時行、許國庇之,止令聽勘。克儉乃言:「臣初劾守廉,時行以書沮臣;及劾寧,又與國諭臣寬其罪。徇私背公,將壞邊事。」並歷詆一鶚、達及兵科給事中張應登朋奸欺罔,達置傑會稿功罪疏不奏,遂追數成梁前數年冒功狀。帝謂成梁前功皆由巡按勘報,克儉懸度妄議。卒置成梁等不問,而心以傑為不欺。

旋就進右副都御史。日本陷朝鮮,達遣裨將祖承訓以三千人往,皆沒。事聞,傑亦被劾,帝特免之。朝鮮王避難將入遼,傑請擇境外善地處之,且周給其從官、衛士,報可。尋遷兵部右侍郎,總督薊、遼、保定軍務。召理戎政,進右都御史。日本封貢議起,傑曰:「平秀吉罪不勝誅,顧加以爵命,荒外聞之,謂中朝無人。」議不合,徙南京戶部尚書。移疾歸。起南京工部尚書。就改兵部,參贊機務。卒官。贈太子少保。

胡克儉,字共之,光山人。萬曆十四年進士。由庶吉士改御史,巡按山東。遼東其所轄也,奏禁買功、竊級諸弊。既劾成梁,為要人所忌。會克儉劾左都御史李世達曲庇罪囚,至詆為賊,執政遂言克儉妄排執法大臣,不可居言路,謫蘄水丞。上官以事遣歸,里居三十年。光宗立,起光祿少卿。天啟中,歷刑部右侍郎。五年冬,逆黨李恒茂論其衰朽,落職歸。崇禎初,復官。卒贈尚書。克儉本姓扶,冒胡姓,久之始復故。

趙參魯,字宗傳,鄞人。隆慶五年進士。選庶吉士,改戶科給事中。萬曆二年,慈聖太后立廟涿州,祀碧霞元君。部科臣執奏,不從。參魯斥其不經,且言:「南北被寇,流害生民,興役濬河,鬻及妻子。陛下發帑治橋建廟,已五萬有奇。茍移振貧民,植福當更大。」亦不聽。南京中官張進醉辱給事中王頤,給事中鄭岳、楊節交章論,未報,參魯復上言:「進乃守備中官申信黨,不併治信,無以厭人心。」時信方結馮保,朝議遂奪岳等俸,謫參魯高安典史。遷饒州推官,擢福建提學僉事,請急歸。遭喪,服除,仍督學福建。歷南京太常卿。十七年,以右副都御史巡撫福建。申嚴海禁,戮姦商通倭者。遷大理卿。召為刑部左侍郎,改兵部,旋改吏部。日本封貢議起,參魯持不可。總督顧養謙不懌,爭於朝,且言參魯熟倭情,宜任。章下廷臣,參魯復持前說,因著《東封三議》,辨利害甚悉。其後封事卒不成。拜南京刑部尚書。誠意伯劉世延妄指星象,欲起兵勤王,被劾下吏,參魯當以死。南京工部主事趙學仕以侵牟為侍郎周思敬所劾,擬戍。學仕移罪家僮,法司予輕比。御史朱吾弼復劾之,並及參魯,言學仕乃大學士志皋族父,故參魯庇之。參魯乞休。吏部尚書孫丕揚等言參魯履行素高,不當聽其去,詔留之。累加太子太保。致仕,卒,謚端簡。

張孟男,字元嗣,中牟人。嘉靖四十四年進士。授廣平推官。稍遷漢中同知。入為順天治中,累進尚寶丞。高拱以內閣兼吏部,其妻,孟男姑也,自公事外無私語。拱憾之,四歲不遷。及拱被逐,親知皆引匿,孟男獨留拱邸,為治裝送之郊。張居正用事,擢孟男太僕少卿。孟男復不附,失居正意,不調。久之,居正敗,始累遷南京工部右侍郎。尋召入,以本官掌通政司事。

萬曆十七年,帝不視朝者八月,孟男疏諫,且言:「嶺南人訟故都御史李材功,蔡人訟故令曹世卿枉,章並留中,其人繫兵馬司,橐饘不繼,莫必其生,虧損聖德。」帝心動,乃間一御門。其冬,改戶部,進左侍郎。尋拜南京工部尚書,就改戶部。時留都儲峙耗竭,孟男受事,粟僅支二年,不再歲遂有七年之蓄。水衡修倉,發公羨二千金助之。或謂奈何耘人田,孟男曰:「公家事,乃畫區畔耶?」南京御史陳所聞劾孟男貪鄙,吏部尚書孫鑨言孟男忠誠謹恪,臺臣所論,事由郎官,帝乃留之。孟男求去,不允。再疏請,乃聽歸。久之,召拜故官。

三十年春,有詔罷礦稅。已,弗果行。孟男率同列諫,不報。加太子少保。五上章乞歸,不許。時礦稅患日劇,孟男草遺疏數千言,極陳其害,言:「臣備員地官,所徵天下租稅,皆鬻男市女、朘骨割肉之餘也。臣以催科為職,臣得其職,而民病矣。聚財以病民,虐民以搖國,有臣如此,安所用之?臣不勝哀鳴,為陛下杞人憂耳。」屬其子上之,明日遂卒。南京尚書趙參魯等奏其清忠,贈太子太保。

衛承芳,字君大,達州人。隆慶二年進士。萬曆中,累官溫州知府。公廉善撫字。進浙江副使,謝病歸。薦起山東參政,歷南京鴻臚卿。吏部推太常少卿朱敬循為右通政,以承芳貳之。敬循,大學士賡子也。賡言:「承芳臣同年進士,恬淡之操,世罕能及,臣子不當先。」帝許焉。尋遷南京光祿卿,擢右副都御史,巡撫江西。嚴絕饋遺,屬吏爭自飭。入為南京兵部右侍郎,就拜戶部尚書。福王乞蘆洲,自江都抵太平南北千餘里,自遣內官徵課。承芳抗疏爭,卒不從。萬歷間,南京戶部尚書有清名者,前有張孟男,後則稱承芳。尋就改吏部。卒官。贈太子太保,謚清敏。

李禎,字維卿,安化人。隆慶五年進士。除高平知縣,徵授御史。萬歷初,傅應禎以直言下詔獄,禎與同官喬嚴、給事中徐貞明擁入護視之,坐謫長蘆鹽運司知事。遷歸德推官、禮部主事,三遷順天府丞。十八年,洮、河有警,極言貢市非策,因歷詆邊吏四失。帝以納款二十年,不當咎始事,遂寢其議。以右僉都御史巡撫湖廣,言:「知縣梁道凝循吏,反注下考,宜懲挾私者以勵其餘。薦舉屬吏,不應專及高秩,下僚如趙蛟、楊果者,亦當顯旌之。」蛟、果,萬歷初以吏員超擢者也。詔皆報可。召為左僉都御史,再遷戶部右侍郎。趙用賢以絕婚事被訐,戶部郎中鄭材復詆之。禎駁材疏,語侵其父洛。材憤,疏詆禎,禎遂乞休,不允。御史宋興祖請改材他部避禎,全大臣體,乃出材南京。禎尋調兵部,進左侍郎。

二十四年,日本封貢事僨,首輔趙志皋、尚書石星俱被劾。廷臣議戰守,章悉下兵部。禎等言:「今所議惟戰守封三事。封則李宗城雖征,楊方亨尚在。若遽議罷,無論中國數百人淪於異域,而我兵食未集,勢難遠征。宜令方亨靜俟關白來迎則封,不迎則止。我以戰守為實務,而相機應之。且朝鮮素守禮,王師所屯,宜嚴禁擾掠。」得旨如議。而疏內言志皋、星當去。詔詰禎,止令議戰守事,何擅及大臣去留,姑勿問。志皋自是不悅。明年,星得罪,命禎攝部事。禎以平壤、王京、釜山皆朝鮮要地,請修建大城,興屯開鎮,且列上戰守十五策,俱允行。後又數上方略。

四川被寇,禎言:「川、陜接界,而松潘向無寇患者,以諸番為屏蔽也。自俺答西牧,隴右騷然。其後隴右備嚴,寇不得逞,而禍乃移之川矣。今諸番彊半折入於西部。臣閱地圖,從北界迤西間道達蜀地,多不隔三舍。幸層巖疊嶂,屹然天險,如鎮虜堡為漳臘門戶,虹橋關為松城咽喉。關堡之外,或嶺或崖,皆可據守。守阿玉嶺,則不能越咂際而窺堡。守黃勝場,則不敢踰塞墩而寇關。他如橫山、寡石崖尤為要害,皆當亟議防禦,令撫鎮臣計畫以聞。」報可。

禎質直方剛,署事規畫頗當。有欲即用為尚書者,志皋以故憾,陰沮之。而張位、沈一貫雅與經略邢玠、經理楊鎬通,亦不便禎所為,言禎非將材,惟蕭大亨堪任。帝不聽。其後玠、鎬益無功,志皋等又請罷禎,御史況上進劾禎庸鄙。帝皆不聽。甘肅缺巡撫,禎以劉敏寬名上。給事楊應文言敏寬方坐事勘,不當推舉。帝以詰禎,禎言:「前奉詔,敏寬須巡撫缺用,臣故舉之。」帝怒禎不引罪,調之南京。後考察,南京言官拾遺及禎,遂命致仕。

久之,起南京刑部尚書。踰年復引疾,不俟報徑歸,帝怒。大學士葉向高言:「禎實病,不可深責。十餘年來,大臣乞休得請者,百無一二。李廷機、趙世卿皆羈留數載,疏至百餘上。今尚書孫丕揚、李化龍又以考察軍政疏不下,相率求去。若復踵禎轍,實傷國體。諸臣求去,約有數端。疾病當去,被言當去,不得其職當去。宜曲體其情,可留留之,不可留則聽之。」帝竟奪禎職閒住。未幾卒。

丁賓,字禮原,嘉善人。隆慶五年進士。授句容知縣。征授御史。大學士張居正,賓座主也,誣劉臺以贓,屬賓往遼東按之。賓力辭,忤居正意,去官。萬曆十九年,用薦起故官,復以憂去。起南京大理丞。累遷南京右僉都御史兼督操江。江防多懈,賓率將校乘一舟往來周視,增守兵戍要害,部內宴然。南衛世職率赴京師請襲,留滯不得官,賓請就南勘襲。妖民劉天緒左道事覺,兵部尚書孫鑛欲窮治之,詔下法司訊鞫。賓兼攝刑部大理事,力平反,論七人死,餘皆獲釋。召拜工部左侍郎,尋擢南京工部尚書。自上元至丹陽道路,盡易以石,行旅頌之。數引年乞罷,光宗立,始予致仕。

賓官南都三十年,每遇旱潦,輒請振貸,時出家財佐之。初以御史家居,及丁憂歸,連三年大饑,咸捐資以振。至天啟五年,復捐粟三千石振貧民,以資三千金代下戶之不能輸賦者。撫按錄上其先後事,時已加太子少保,詔進太子太保,旌其門。以年高,三被存問。崇禎六年卒,年九十一。謚清惠。

贊曰:南京卿長,體貌尊而官守無責,故為養望之地,資地深而譽聞重者處焉。或彊直無所附麗,不為執政所喜,則以此遠之。袁洪愈諸人類以清彊居優閒之地,不竟其用,亦以自全。乾時冒進之徒,可以風矣。


\end{pinyinscope}