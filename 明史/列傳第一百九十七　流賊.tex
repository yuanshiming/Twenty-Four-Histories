\article{列傳第一百九十七 流賊}

\begin{pinyinscope}
盜賊之禍,歷代恒有,至明末李自成、張獻忠極矣。史冊所載,未有若斯之酷者也。永樂中,唐賽兒倡亂山東。厥後乘瑕弄兵,頻見竊發,然皆旋就撲滅。惟武宗之世,流寇蔓延,幾危宗社,而卒以掃除。莊烈帝勵精有為,視武宗何啻霄壤,而顧失天下,何也?明興百年,朝廷之綱紀既肅,天下之風俗未澆。孝宗選舉賢能,布列中外,與斯民休養生息者十餘年,仁澤深而人心固,元氣盛而國脈安。雖以武之童昏,亟行稗政,中官倖夫,濁亂左右,而本根尚未盡拔,宰輔亦多老成。迨盜賊四起,王瓊獨典中樞,陸完、彭澤分任閫帥,委寄既專,旁撓絕少,以故危而不亡。莊烈帝承神、熹之後,神宗怠荒棄政,熹宗暱近閹人,元氣盡澌,國脈垂絕。向使熹宗御宇復延數載,則天下之亡不再傳矣。

莊烈之繼統也,臣僚之黨局已成,草野之物力已耗,國家之法令已壞,邊疆之搶攘已甚。莊烈雖銳意更始,治核名實,而人才之賢否,議論之是非,政事之得失,軍機之成敗,未能灼見於中,不搖於外也。且性多疑而任察,好剛而尚氣。任察則苛刻寡恩,尚氣則急遽失措。當夫群盜滿山,四方鼎沸,而委政柄者非庸即佞,剿撫兩端,茫無成算。內外大臣救過不給,人懷規利自全之心。言語戇直,切中事弊者,率皆摧折以去。其所任為閫帥者,事權中制,功過莫償。敗一方即戮一將,隳一城即殺一吏,賞罰太明而至於不能罰,制馭過嚴而至於不能制。加以天災流行,饑饉洊臻,政繁賦重,外訌內叛。譬一人之身,元氣羸然,疽毒並發,厥癥固已甚危,而醫則良否錯進,劑則寒熱互投,病入膏肓,而無可救,不亡何待哉?是故明之亡,亡於流賊,而其致亡之本,不在於流賊也。嗚呼!莊烈非亡國之君,而當亡國之運,又乏救亡之術,徒見其焦勞瞀亂,孑立於上十有七年。而帷幄不聞良、平之謀,行間未睹李、郭之將,卒致宗社顛覆,徒以身殉,悲夫!

自唐賽兒以下,本末易竟,事具剿賊諸臣傳中。獨志其亡天下者,立李自成、張獻忠傳。

○李自成張獻忠

李自成,米脂人,世居懷遠堡李繼遷寨。父守忠,無子,禱於華山,夢神告曰:「以破軍星為若子。」已,生自成。幼牧羊於邑大姓艾氏,及長,充銀川驛卒。善騎射,鬥很無賴,數犯法。知縣晏子賓捕之,將置諸死,脫去為屠。天啟末,魏忠賢黨喬應甲為陜西巡撫,硃童蒙為延綏巡撫,貪黷不詰盜,盜由是始。

崇禎元年,陜西大饑,延綏缺餉,固原兵劫州庫。白水賊王二,府谷賊王嘉胤,宜川賊王左掛、飛山虎、大紅狼等,一時並起。有安塞馬賊高迎祥者,自成舅也,與饑民王大梁聚眾應之。迎祥自稱闖王,大梁自稱大梁王。二年春,詔以楊鶴為三邊總督,捕之。參政劉應遇擊斬王二、王大梁,參政洪承疇擊破王左掛,賊稍稍懼。會京師戒嚴,山西巡撫耿如巳勤王兵嘩而西,延綏總兵吳自勉、甘肅巡撫梅之煥勤王兵亦潰,與群盜合。延綏巡撫張夢鯨恚死,承疇代之,召故總兵杜文煥督延綏、固原兵,便宜剿賊。

三年,王左掛、王子順、苗美等戰屢敗,乞降。而王嘉胤掠延安、慶陽間,楊鶴撫之,不聽,從神木渡河犯山西。是時,秦地所徵曰新餉,曰均輸,曰間架,其目日增,吏因緣為姦,民大困。以給事中劉懋議,裁驛站,山、陜游民仰驛糈者,無所得食,俱從賊,賊轉盛。兵部郎中李繼貞奏曰:「延民饑,將盡為盜,請以帑金十萬振之。」帝不聽。而嘉胤已襲破黃甫川、清水、木瓜三堡,陷府谷、河曲。又有神一元、不沾泥、可天飛、郝臨庵、紅軍友、點燈子、李老柴、混天猴、獨行狼諸賊,所在蜂起,或掠秦,或東入晉,屠陷城堡。官兵東西奔擊,賊或降或死,旋滅旋熾。延安賊張獻忠亦聚眾據十八寨,稱八大王。

四年,孤山副將曹文詔破賊河曲,王嘉胤遁去。已,復自岳陽突犯澤、潞,為左右所殺,其黨共推王自用號紫金梁者為魁。自用結群賊老回回、曹操、八金剛、掃地王、射塌天、閻正虎、滿天星、破甲錐、刑紅狼、上天老、蠍子塊、過天星、混世王等及迎祥、獻忠共三十六營,眾二十餘萬,聚山西。自成乃與兄子過往從迎祥,與獻忠等合,號闖將,未有名。楊鶴撫賊不效被逮,洪承疇代鶴,張福臻代承疇,督諸將曹文詔、楊嘉謨剿賊,所向克捷,陜地略定。而山西賊大盛,剽掠寧鄉、石樓、稷山、聞喜、河津間。

五年,賊分道四出,連陷大寧、隰州、澤州、壽陽諸州縣,全晉震動。乃罷巡撫宋統殷,以許鼎臣代之,與宣大總督張宗衡分督諸將。宗衡督虎在威、駕人龍、左良玉等兵八千人,駐平陽,責以平陽、澤、潞四十一州縣。鼎臣督張應昌、頗希牧、艾萬年兵七千人,駐汾州,責以汾、太、沁遼三十八州縣。賊亦轉入磨盤山,分眾為三:「閻正虎據交城、文水,窺太原;邢紅狼、上天龍據吳城,窺汾州;自用、獻忠突沁州、武鄉,陷遼州。

六年春,官兵共進力擊。自用懼,乞降於故錦衣僉事張道浚。約未定,陽和兵襲之。賊怒,敗約去。會總兵官曹文詔率陜西兵至,偕諸將猛如虎、虎大威、頗希牧、艾萬年、張應昌等合剿,屢戰皆大克,前後殺混世王、滿天星、姬關鎖、翻山動,掌世王、顯道神等,破自用、獻忠、老回回、蠍子塊、掃地王諸賊。其後,自用又為川將鄧射殺之。山西三大盜俱敗。初,賊之破澤州也,分其眾,南踰太行,掠濟源、清化、修武,圍懷慶。官軍擊之,賊遁走。別賊復闌入西山,大掠順德、真定間。大名道盧象昇力戰劫賊。賊自邢臺摩天嶺西下,抵武安,敗總兵左良玉,河北三府焚劫殆遍。潞王上疏告急,兼請衛鳳、泗陵寢。詔特遣總兵倪寵、王樸率京營兵六千人,與諸將並進。賊聞之,欲從河內走太行。文詔邀擊之,不敢進。賊之敗於山西者,亦奔河北合營,迎祥、自成、獻忠、曹操、老回回等俱至。京兵蹙其後,左良玉,湯九州等扼其前,連戰於青店、石岡、石坡、牛尾、柳泉、猛虎村,屢敗之。賊欲逸,阻於河,大困。賊素畏文詔、道浚,道浚先坐事遣戍,文詔轉戰秦、晉、河北,遇賊輒大克,御史復劾其驕倨,調大同總兵去。賊遂詭辭乞降,監軍太監楊進朝信之,為入奏。會天寒河冰合,賊突從毛家寨策馬徑渡。河南諸軍無扼河者,賊遂連陷澠池、伊陽、盧氏三縣。河南巡撫玄默率諸將盛兵待之,賊竄入盧氏山中,由間道直走內鄉,掠鄖陽,又分掠南陽、汝寧,入棗陽、當陽,偪湖廣。巡撫唐暉斂兵守境。犯歸、巴、夷陵等處,破夔州,攻廣元,逼四川,所在告急。

七年春,特設山、陜、河南、湖廣、四川總督,專辦賊,以延綏巡撫陳奇瑜為之,以盧象昇撫治鄖陽,為奇瑜破賊延水關有威名,而象升歷戰陣知兵也。於是奇瑜自均州入,與象昇並進,師次烏林關,斬賊數千級。賊走漢南,奇瑜以湖廣不足憂,引兵西擊。始,賊自澠池渡河,高迎祥最強,自成屬焉。及入河南,自成與兄子過結李牟、俞彬、白廣恩、李雙喜、顧君恩、高傑等自為一軍。過、傑善戰,君恩善謀。及奇瑜兵至,獻忠等奔商、雒,自成等陷於興安之車箱峽。會大雨兩月,馬乏芻多死,弓矢皆脫,自成用君恩計,賄奇瑜左右,詐降。奇瑜意輕賊,許之,檄諸將按兵毋殺,所過州縣為具糗傳送。賊甫渡棧,即大噪,盡屠所過七州縣。而略陽賊數萬亦來會,賊勢愈張。奇瑜坐削籍,而自成名始著矣。已,洪承疇代奇瑜,李喬巡撫陜西,吳甡巡撫山西。大學士溫體仁謂甡曰:「流賊癬疥疾,勿憂也。」未幾,西寧兵變,承疇甫受命而東,聞變遽返。迎祥、自成遂入鞏昌、平涼、臨洮、鳳翔諸府數十州縣。敗賀人龍、張天禮軍,殺固原道陸夢龍。圍隴州四十餘日,承疇檄總兵左光先與人龍合擊,大破之。會朝廷亦命豫、楚、晉、蜀兵四道入陜,迎祥、自成遂竄入終南山。已而東出,陷陳州、靈寶、汜水、滎陽。聞左良玉將至,移壁梅山、溱水間。部賊拔上蔡,燒汝寧郛。乃命承疇出關追賊,與山東巡撫朱大典并力擊,賊偵知之。

八年正月,大會於滎陽。老回回、曹操、革裏眼、左金王、改世王、射塌天、橫天王、混十萬、過天星、九條龍、順天王及迎祥、獻忠共十三家七十二營,議拒敵,未決。自成進曰:「一夫猶奮,況十萬眾乎!官兵無能為也。宜分兵定所向,利鈍聽之天。」皆曰:「善。」乃議革裏眼、左金王當川、湖兵,橫天王、混十萬當陜兵,曹操、過天星扼河上,迎祥、獻忠及自成等略東方,老回回、九條龍往來策應。陜兵銳,益以謝塌天、改世王。所破城邑,子女玉帛惟均。眾如自成言。先是,南京兵部尚書呂維祺懼賊南犯,請加防鳳陽陵寢,不報。及迎祥、獻忠東下,江北兵單。固始、霍丘俱失守。賊燔壽州,陷潁州,知州尹夢鰲、州判趙士寬戰死,殺故尚書張鶴鳴。乘勝陷鳳陽,焚皇陵,留守署正朱國相等皆戰死。事聞,帝素服哭,遣官告廟。逮漕運都御史楊一鵬棄市,以朱大典代之,大徵兵討賊。賊乃大書幟曰古元真龍皇帝,合樂大飲。自成從獻忠求皇陵監小閹善鼓吹者,獻忠不與。自成怒,偕迎祥西趨歸德,與曹操、過天星合,復入陜西。獻忠獨東下廬州。

承疇方馳至汝州,命諸將左良玉、湯九州、尤世威、徐來朝、陳永福、鄧、張應昌分扼湖廣、河南、鄖陽諸關隘,召曹文詔為中軍。文詔未至,以兵亂死。迎祥、自成從終南山出,大掠富平、寧州。老回回、獻忠、曹操、蠍子塊、過天星諸賊,聞承疇出關,先後皆走陜西,焚掠西安、平涼、鳳翔諸郡。承疇亟還救,分遣諸將擊老回回等,令副總兵劉成功、艾萬年擊迎祥、自成於寧州。萬年中伏戰死,文詔怒,復擊之,亦中伏戰死。群賊乘勝掠地,火照西安城中。承疇力禦之涇陽、三原間,決死戰,賊不得過。獻忠、老回回等由他道轉突朱陽關,守關將徐來臣軍潰死,尤世威中箭遁。於是群賊皆出關,分十三營東犯,而迎祥、自成獨留陜西。

時盧象昇已改湖廣巡撫,總理直隸、河南、山東、四川、湖廣諸軍務。詔承疇督關中,象昇督關外。賊亦分兵,迎祥略武功、扶風以西,自成略富平、固州以東。承疇遣將追自成,小捷,至醴泉。賊將高傑通於自成妻邢氏,懼誅,挾之來降。承疇身追自成,大戰渭南、臨潼,自成大敗東走。迎祥亦屢敗,東踰華陰南原,絕嶺,偕自成出朱陽關,與獻忠合。冬十一月,群賊薄閿鄉,左良玉、祖寬禦之不克,遂陷陜州,進攻雒陽。河南巡撫陳必謙督良玉、寬援雒陽,獻忠走嵩、汝。迎祥、自成走偃師、鞏縣,略魯山、葉縣,陷光州,象昇擊敗之確山。

九年春,迎祥、自成攻廬州,不拔。陷含山、和州,殺知州黎弘業及在籍御史馬如蛟等。又攻滁州,知州劉大鞏、太僕卿李覺斯堅守不下。象昇親督祖寬、羅岱、楊世恩等來援,戰於朱龍橋,賊大敗,屍咽水不流。北攻壽州,故御史方震孺堅守。折而西,入歸德,邊將祖大樂破之。走密、登封,故總兵湯九州戰死。分道犯南陽、裕州,必謙援南陽,象升援裕,令大樂等擊賊,殺迎祥、自成精銳幾盡。賊復分兵再入陜,迎祥由鄖、襄趨興安、漢中,自成由南山踰商、雒,走延綏,犯鞏昌北境。諸將左光先、曹變蛟破之,自成走環縣。未幾,官軍敗於羅家山,盡亡士馬器仗,總兵官俞沖霄被執。自成執復振,進圍綏德,欲東渡河,山西兵遏之。復西掠米脂,呼知縣邊大綬,曰:「此吾故鄉也,勿虐我父老。」遺之金,令修文廟。將襲榆林,河水驟長,賊淹死甚眾,乃改道,從韓城而西。時象升及大樂、寬等皆入援京師。孫傳庭新除陜西巡撫,銳意滅賊。秋七月,擒迎祥於盩啡,獻俘闕下,磔死。於是賊黨乃共推自成為闖王矣。是月,犯階、徽。未幾,出、隴,犯鳳翔,渡渭河。

十年,犯涇陽、三原。蠍於塊、過天星俱來會。傳庭督變蛟連戰七日,皆克,蠍子塊降。自成與過天星奔秦州。入蜀,陷寧羌,破七盤關,陷廣元,總兵官侯良柱戰死,遂連陷昭化、劍州、梓潼、江油、黎雅、青川等州縣。劍州知州徐尚卿、吏目李英俊、昭化知縣王時化、郫縣主簿張應奇、金堂典史潘夢科皆死。進攻成都,七日不克,巡撫王維章坐避賊徵。

十一年春,官軍敗賊梓潼,自成奔白水,食盡。承疇、傳庭合擊於潼關原,大破之。自成盡亡其卒,獨與劉宗敏、田見秀等十八騎潰圍,竄伏商、洛山中。其年,獻忠降,自成勢益衰。承疇改薊遼總督,傳庭改保定總督。傳庭以疾辭,逮下獄。二人去,自成稍得安。總理熊文燦方主撫,諜者或報自成死,益寬之。

十二年夏,獻忠反穀城。自成大喜,出收眾,眾復大集。陜西總督鄭崇儉發兵圍之,令曰「圍師必缺。」自成乃由缺走,突武關,往依獻忠。獻忠欲圖之,覺,遁去。楊嗣昌督師夷陵,檄令降,自成出謾語。官軍圍自成於巴西、魚復諸山中,自成大困,欲自經,養子雙喜勸而止。賊將多出降。劉宗敏者,藍田鍛工也,最驍勇,亦欲降。自成與步入叢祠,顧而嘆曰:「人言我當為天子,盍卜之,不吉,斷我頭以降。」宗敏諾,三卜三吉。宗敏還,殺其兩妻,謂自成曰:「吾死從君矣。」軍中壯士聞之,亦多殺妻子願從者。自成乃盡焚輜重,輕騎由鄖、均走河南。河南大旱,斛穀萬錢,饑民從自成者數萬。遂自南陽出,攻宜陽,殺知縣唐啟泰。攻永寧,殺知縣武大烈,戕萬安王采金輕。攻偃師,知縣徐日泰罵賊死。時十三年十二月也。

自成為人高顴深,鴟目曷鼻,聲如豺。性猜忍,日殺人斮足剖心為戲。所過,民皆保塢堡不下。杞縣舉人李信者,逆案中尚書李精白子也,嘗出粟振饑民,民德之曰:「李公子活我。」會繩伎紅娘子反,擄信,強委身焉。信逃歸,官以為賊,囚獄中。紅娘子來救,饑民應之,共出信。盧氏舉人牛金星磨勘被斥,私入自成軍為主謀,潛歸,事洩坐斬,已,得末減。二人皆往投自成,自成大喜,改信名曰巖。金星又薦卜者宋獻策,長三尺餘,上讖記云:「十八子,主神器。」自成大悅。巖因說曰:「取天下以人心為本,請勿殺人,收天下心。」自成從之,屠戮為減。又散所掠財物振饑民,民受餉者,不辨巖、自成也,雜呼曰:「李公子活我。」巖復造謠詞曰:「迎闖王,不納糧。」使兒童歌以相煽,從自成者日眾。

十四年正月攻河南,有營卒勾賊,城遂陷,福王常洵遇害。自成兵汋王血,雜鹿醢嘗之,名「福祿酒。」王世子由崧裸而逃。自成發王邸金振饑民,遂移攻開封。時張獻忠亦陷襄陽,戕襄王翊銘。王開封者周王恭枵,聞賊至,急發庫金募死士,與巡撫都御史高名衡等固守。自成攻七晝夜,解去,屠密縣。賊魁羅汝才、土寇袁時中皆歸自成。時中眾二十萬,號小袁營。汝才即曹操,與獻忠同降復叛去者也。

自成初為迎祥裨將,至是勢大盛。帝以故尚書傅宗龍為陜西總督,使專辦自成,別敕保定總督楊文岳會師。宗龍馳入關,與巡撫汪喬年調兵,兵已發盡,乃檄河南大將李國奇、賀人龍兵隸部下,亟出關。文岳率虎大威軍俱至新蔡,與自成遇。人龍卒先奔,國奇、大威繼之,宗龍、文岳以親軍築壘自固。夜,文岳兵潰奔陳州,宗龍與賊持數日,食盡,突圍走,被執死。自成陷葉縣,殺副將劉國能,遂圍左良玉於郾城。喬年代宗龍總督,出關,次襄城,自成盡銳攻之,喬年與副將李萬慶皆死。自成劓刖諸生百九十人。遂乘勝陷南陽、鄧州十四城,再圍開封。巡撫名衡、總兵陳永福力拒之,射中自成目,炮殪上天龍等,自成益怒。

自成每攻城,不用古梯衝法,專取瓴甋,得一磚即歸營臥,後者必斬。取磚已,即穿穴穴城。初僅容一人,漸至百十,次第傅土以出。過三五步,留一土柱,繫以巨糸亙。穿畢,萬人曳糸亙一呼,而柱折城崩矣。名衡於城上鑿橫道,聽其下有聲,用毒穢灌之,多死。賊乃即城壞處用火攻法,實藥甕中,火燃藥發,當者輒糜碎,名曰放迸。

十五年正月,城半圮,賊用放迸法攻之,鐵騎數千馳噪,伺城頹即擁入城。城故宋汴都,金人所重築也。厚數丈,土堅,火外擊,賊騎多殲,自成駭而去。南陷西華,尋屠陳州,副使關永傑、知州侯君擢皆罵賊死。歸德、睢州、寧陵、太康數十郡縣,悉殘毀。商丘知縣梁以樟創死復蘇,全家殲焉。已,復攻開封,築長圍為持久計。詔起孫傳庭為總督,釋故尚書侯恂命督師,召左良玉援開封。良玉至朱仙鎮,大敗,奔襄陽。諸軍皆屯河北,不敢進。開封食盡。山東總兵劉澤清亦奉詔至。傳庭知開封急,大會諸將西安,亟出關來救。未至,名衡等議決朱家寨口河灌賊,賊亦決馬家口河欲灌城。秋九月癸未,天大雨,二口並決,聲如雷,潰北門入,穿東南門出,注渦水。城中百萬戶皆沒,得脫者惟周王、妃、世子及撫按以下不及二萬人。賊亦漂沒萬餘,乃拔營西南去。

先是,有馬守應稱老回回、賀一龍稱革裏眼、賀錦稱左金王、劉希堯稱爭世王、藺養成稱亂世王者,皆附自成,時號「革左五營。」自成乃西迎傳庭兵,遇於南陽,傳庭軍潰走,豫人所謂柿園之敗也。是時大清兵南侵,京師方告急,朝廷不暇復討賊。自成乃收群賊,連營五百餘里,再屠南陽,進攻汝寧。總兵虎大威中炮死,楊文岳被殺。自成乃脅崇王由樻使從軍,遂由確山、信陽、泌陽向襄陽。左良玉望風南走,自成入襄陽。分徇屬城及德安諸州縣,皆下,再破夷陵、荊門州。自成自攻荊州,湘陰王儼金尹遇害,燒獻陵木城,穿毀宮殿。

十六年春陷承天。將發獻陵,有聲震山谷,懼而止。帝掠潛山、京山、雲夢、黃陂、孝感等州縣,皆下。先驅偪漢陽,良玉走九江。攻鄖陽,撫治都御史徐起元及王光恩力守不下。光恩,賊反正者也。

自成自號奉天倡義大元帥,號羅汝才代天撫民威德大將軍。分其眾,曰標營,領兵百隊;曰先、後、左、右營,各領兵三十餘隊。標營白幟黑纛,自成獨白鬃大纛銀浮屠;左營幟白,右緋,前黑,後黃,纛隨其色。五營以序直晝夜,次第休息,巡徼嚴密。逃者謂之落草,磔之。收男子十五以上、四十以下者為兵。精兵一人,主芻、掌械、執爨者十人。軍令不得藏白金,過城邑不得室處,妻子外不得攜他婦人。寢興悉用單布幕。綿甲厚百層,矢炮不能入。一兵倅馬三四匹,冬則以茵褥籍其蹄。剖人腹為馬槽以飼馬,馬見人,輒鋸牙思噬若虎豹。軍止,即出較騎射,曰站隊。夜四鼓,蓐食以聽令。所過崇岡峻阪,騰馬直上。水惟憚黃河,若淮、泗、涇、渭,則萬眾翹足馬背,或抱鬣緣尾,呼風而渡,馬蹄所壅閼,水為不流。臨陣,列馬三萬,名三堵墻。前者返顧,後者殺之。戰久不勝,馬兵佯則誘官兵,步卒長鎗三萬,擊刺如飛,馬兵回擊,無不大勝。攻城,迎降者不殺,守一日殺十之三,二日殺十之七,三日屠之。凡殺人,束屍為燎,謂之打亮。城將陷,步兵萬人環堞下,馬兵巡徼,無一人得免。獻忠雖至殘忍,不逮也。諸營較所獲,馬騾者上賞,弓夭鉛銃者次之,幣帛又次之,珠玉為下。

自成不好酒色,脫粟粗糲,與其下共甘苦。汝才妻妾數十,被服紈綺,帳下女樂數部,厚自奉養,自成嘗嗤鄙之。汝才眾數十萬,用山西舉人吉珪為謀主。自成善攻,汝才善戰,兩人相須若左右手。自成下宛、葉,克梁、宋,兵強士附,有專制心,顧獨忌汝才。乃召汝才所善賀一龍宴,縛之,晨以二十騎斬汝才於帳中,悉兼其眾。

自成在中州,所略城輒焚毀之。及渡漢江,謀以荊、襄為根本,改襄陽曰襄京,修襄王宮殿居之。改禹州曰均平府,承天府曰揚武州,他府縣多所更易。

牛金星教以創官爵名號,大行署置。自成無子,兄子過及妻弟高一功,迭居左右,親信用事。田見秀、劉宗敏為權將軍,李巖、賀錦、劉希堯等為制將軍,張鼐、黨守素等為威武將軍,谷可成、任維榮等為果毅將軍,凡五營二十二將。又置上相、左輔、右弼、六政府侍郎、郎中、從事等官。要地設防禦使,府曰尹,州曰牧,縣曰令。封崇王由樻襄陽伯、邵陵王在城棗陽伯、保寧王紹圮宣城伯、肅寧王術受順義伯。以張國紳為上相,牛金星為左輔,來儀為右弼。國紳,安定人,嘗官參政。既降,獻文翔鳳妻鄧氏以媚自成。自成惡其傷同類,殺之,而歸鄧氏於其家。六政府侍郎則石首喻上猷、江陵蕭應坤、招遠楊永裕、米脂李振聲、江陵鄧巖忠、西安姚錫胤,尋以宣城丘之陶代振聲為兵政府侍郎。其餘受偽職者甚眾,不具載。

使高一功、馮雄守襄陽,任繼光守荊州,藺養成、牛萬才守夷陵,王文曜守澧州,白旺守安陸,蕭雲林守荊門,謝應龍守漢川,周鳳梧守萬禹州。於是河南、湖廣、江北諸賊莫不聽命。自成既殺汝才、一龍,又襲殺養成,奪守應兵,擊殺袁時中於杞縣。獻忠方據武昌,自成遣使賀,且脅之曰:「老回回已降,曹操輩誅死,行及汝矣。」獻忠大懼,南入長沙。當是時,十三家七十二營諸大賊,降死殆盡,惟自成、獻忠存,而自成獨勁,遂自稱曰新順王。集牛金星等議兵所向。金星請先取河北,直走京師。楊永裕請下金陵,斷燕都糧道。從事顧君恩曰:「金陵居下流,事雖濟,失之緩。直走京師,不勝,退安所歸,失之急。關中,大王桑梓邦也,百二山河,得天下三分之二,宜先取之,建立基業。然後旁略三邊,資其兵力,攻取山西,後向京師,庶幾進戰退守,萬全無失。」自成從之。

傳庭之敗於柿園而歸陜也,大治兵,制火車二萬輛,募壯士,使白廣恩、高傑將,欲俟賊饑而擊之。朝議日督戰,不得已出關。以牛成虎、盧光祖為前鋒,由靈寶入洛。高傑為是中軍,檄廣恩從新安來會。河南將陳永福守新灘,四川將秦翼明出商、洛,為掎角。前鋒敗賊澠池,至寶豐,再拔其城。次郟。自成率萬騎還戰,復大敗,幾被擒。會天大雨,道濘,糧車不進。自成遣輕騎出汝州,要截糧道。傳庭乃分軍三,令廣恩從大道,令高傑親隨從間道,迎糧,令永福守營。傳庭既行,永福兵亦爭發,不可禁,遂為賊所躡。至南陽,傳庭還戰,賊陣五重,官軍克其三。已而稍卻,火車奔,騎兵亦大奔。賊縱鐵騎踐之,傳庭大敗。自成空壁追,一日夜踰四百里,官軍死者四萬餘人,失兵器輜重數十萬。傳庭奔河北,轉趨潼關,氣敗沮不復振。

冬十月,自成陷潼關,傳庭死,遂連破華陰、渭南、華、商、臨潼。進攻西安,守將王根子開東門納賊。自成執秦王存樞以為權將軍,永壽王誼曈為制將軍。巡撫馮師孔以下死者十餘人,布政使陸之祺等俱降。自成大掠三日,下令禁止。改西安曰長安,稱西京。賜顧君恩女樂一部,賞入關策也。大發民,修長安城,開馳道。自成每三日親赴教場校射,百姓望見黃龍纛,咸伏地呼萬歲。諸將白廣恩、高汝利、左光先、梁甫先行後皆降。陳永福以先射中自成目,保山巔不敢下,自成折箭為誓,招之,亦降。惟高傑以竊自成妻走延安,為李過所追,折而東,渡宜川,絕蒲津以守。

自成兵所至風靡,乃詣米脂祭墓。向為軍所發,焚棄遺骴,築土封之。求其宗人,贈金封爵以去。改延安府曰天保府,米脂曰天保縣,清澗曰天波府。鳳翔不下,屠之。始,自成入陜西,自謂故鄉,毋有侵暴,未一月抄掠如故。又以士大夫必不附己,悉索諸薦紳,搒掠徵其金,死者瘞一穴。榆林故死守,李過等不能克,自成大發兵攻陷之。副使都任,總兵王世國、尤世威等,俱不屈死。乘勝取寧夏,屠慶陽,執韓王亶脊。移攻蘭州,甘肅巡撫林日端等亦死。進陷西寧,於是肅州、山丹、永昌、鎮番、莊浪皆降,陜西地悉歸自成。又遣賊渡河,陷平陽,殺宗室三百餘人。高傑奔澤州。詔以餘應桂總督三邊,收邊兵剿賊,然全陜已沒,應桂不能進。

十七年正月庚寅朔,自成稱王於西安,僭國號曰大順,改元永昌,改名自晟。追尊其曾祖以下,加謚號,以李繼遷為太祖。設天佑殿大學士,以牛金星為之。增置六政府尚書,設弘文館、文諭院、諫議、直指使、從政、統會、尚契司、驗馬寺、知政使、書寫房等官。以乾州宋企郊為吏政尚書、平湖陸之祺為戶政尚書、真寧鞏焴為禮政尚書、歸安張嶙然為兵政尚書。復五等爵,大封功臣,侯劉宗敏以下九人,伯劉體純以下七十二人,子三十人,男五十五人。定軍制。有一馬儳行列者斬之,馬騰入田苗者斬之。籍步兵四十萬、馬兵六十萬。兵政侍郎楊王休為都肄,出橫門,至渭橋,金鼓動地。令弘文館學士李化鱗等草檄馳諭遠近,指斥乘輿。是日,大風霾,黃霧四塞。事聞,帝大驚,召廷臣議。大學士李建泰請督師,帝許之。

時山西自平陽陷,河津、稷山、滎河皆陷,他府縣多望風送款。二月,自成渡河,破汾州,徇河曲、靜樂,攻太原,執晉王求桂,巡撫蔡懋德死之。北徇忻、代,寧武總兵周遇吉戰死。自成先遣游兵入故關,掠大名、真定而北。身率眾賊並邊東犯,陷大同,巡撫衛景瑗、總兵朱三樂死。自成殺代王傳齊,代籓宗室殆盡。犯宣府,總兵姜環迎降,巡撫朱之馮死。遂犯陽和,由柳溝逼居庸,總兵官唐通、太監杜之秩迎降。

三月十三日,焚昌平,總兵官李守鑅死。始,賊欲偵京師虛實,往往陰遣人輦重貨,賈販都市,又令充部院諸掾吏,探刺機密。朝廷有謀議,數千里立馳報。及抵昌平,兵部發騎探賊,賊輒勾之降,無一還者。賊游騎至平則門,京師猶不知也。十七日,帝召問群臣,莫對,有泣者。俄頃賊環攻九門,門外先設三大營,悉降賊。京師久乏餉,乘陴者少,益以內侍。內侍專守城事,百司不敢問。

十八日,賊攻益急,自成駐彰義門外,遣降賊太監杜勳縋入見帝,求禪位。帝怒,叱之下,詔親征。日暝,太監曹化淳啟彰義門,賊盡入。帝出宮,登煤山,望烽火徹天,歎息曰:「苦我民耳。」徘徊久之,歸乾清宮,令送太子及永王、定王於戚臣周奎、田弘遇第,劍擊長公主,趣皇后自盡。十九日丁未,天未明,皇城不守,嗚鐘集百官,無至者。乃復登煤山,書衣襟為遺詔,以帛自縊於山亭,帝遂崩。太監王承恩縊於側。

自成毰笠縹衣,乘烏駁馬,入承天門。偽丞相牛金星,尚書宋企郊、喻上猷,侍郎黎志升、張嶙然等騎而從。登皇極殿,據御座,下令大索帝后,期百官三日朝見。文臣自范景文、勳戚自劉文炳以下,殉節者四十餘人。宮女魏氏投河,從者二百餘人。象房象皆哀吼流淚。太子投周奎家,不得入,二王亦不能匿,先後擁至,皆不屈,自成羈之宮中。長公主絕而復蘇,舁至,令賊劉宗敏療治。

已,乃知帝后崩,自成命以宮扉載出,盛柳棺,置東華門外,百姓過者皆掩泣。越三日己酉,味爽,成國公朱純臣、大學士魏藻德率文武百官入賀,皆素服坐殿前。自成不出,群賊爭戲侮,為椎背、脫帽,或舉足加頸,相笑樂,百官懾伏不敢動。太監王德化叱諸臣曰:「國亡君喪,若曹不思殯先帝,乃在此耶!」因哭,內侍數十人皆哭,藻德等亦哭。顧君恩以告自成,改殮帝后,用兗冕禕翟,加葦廠云。大學士陳演勸進,不許。封太子為宋王。放刑部、錦衣衛繫囚。

自成自居西安,建置官吏,至是益盡改官制。六部曰六政府,司官曰從事,六科曰諫議,十三道曰直指使,翰林院曰弘文館,太僕寺曰驗馬寺,巡撫曰節度使,兵備曰防禦使,知府州縣曰尹、曰牧、曰令。召見朝官,自成南響坐,金星、宗敏、企郊等左右雜坐,以次呼名,分三等授職。自四品以下少詹事梁紹陽、楊觀光等無不污偽命,三品以上獨用故侍郎侯恂。其餘勛戚、文武諸臣奎、純臣、演、藻德等共八百餘人,送宗敏等營中,拷掠責賕賂,至灼肉折脛,備諸慘毒。藻德遇馬世奇家人,泣曰:「吾不能為若主,今求死不得。」賊又編排甲,令五家養一賊,大縱淫掠,民不勝毒,縊死相望。徵諸勳戚大臣金,金足輒殺之。焚太廟神主,遷太祖主於帝王廟。

時賊黨已陷保定,李建泰降,畿內府縣悉附。山東、河南遍設官吏,所至無違者。及淮,巡撫路振飛發兵拒之,乃去。自成謂真得天命,金星率賊眾三表勸進,乃從之,令撰登極儀,諏吉日。及自成升御座,忽見白衣人長數丈,手劍怒視,座下龍爪鬣俱動,自成恐,亟下。鑄金璽及永昌錢,皆不就。聞山海關總兵吳三桂兵起,乃謀歸陜西。

初,三桂奉詔入援,至山海關,京師陷,猶豫不進。自成劫其父襄,作書招之,三桂欲降。至灤州,聞愛姬陳沅被劉宗敏掠去,憤甚,疾歸山海,襲破賊將。自成怒,親部賊十餘萬,執吳襄於軍,東攻山海關,以別將從一片石越關外。三桂懼,乞降於我大清。四月二十二日,自成兵二十萬,陣於關內,自北山亙海。我兵對賊置陣,三桂居右翼末,悉銳卒搏戰,殺賊數千人,賊亦力鬥,圍開復合。戰良久,我兵從三桂陣右突出,衝賊中堅,萬馬奔躍,飛矢雨墮,天大風,沙石飛走,擊賊如雹。自成方挾太子登高岡觀戰,知為我兵,急策馬下岡走。我兵追奔四十里,賊眾大潰,自相踐踏死者無算,僵屍遍野,溝水盡赤。自成奔永平,我兵逐之。三桂先驅至永平,自成殺吳襄,奔還京師。

時牛金星居守,諸降人往謁,執門生禮甚恭。金星曰:「訛言方起,諸君宜簡出。」由是降者始懼,多竄伏矣。自成至,悉鎔所拷索金及宮中帑藏、器皿,鑄為餅,每餅千金,約數萬餅,騾車載歸西安。二十九日丙戌僭帝號於武英殿,追尊七代皆為帝后,立妻高氏為皇后。自成被冠冕,列仗受朝。金星代行郊天禮。是夕焚宮殿及九門城樓。詰旦,挾太子、二王西走,而使偽將軍左光先、谷可成殿。

五月二日,我大清兵入京師,下令安輯百姓,為帝后發喪,議謚號,遣將偕三桂追自成。時福王已監國南京,大學士史可法督師討賊。自成至定州,我兵追之,與戰,斬谷可成,左光先傷足,賊負而逃。自成西走真定,益發眾來攻,我兵復擊之。自成中流矢創甚,西踰故關,入山西。會我兵東返,自成乃鳩合潰散,走平陽。

李巖者,故勸自成以不殺收人心者也。及陷京師,保護懿安皇后令自盡。又獨於士大夫無所拷掠,金星等大忌之。定州之敗,河南州縣多反正,自成召諸將議,巖請率兵往。金星陰告自成曰:「巖雄武有大略,非能久下人者。河南,巖故鄉,假以大兵,必不可制。十八子之讖,得非巖乎?」因譖其欲反。自成令金星與巖飲,殺之,賊眾俱解體。

自成歸西安,復遣賊陷漢中,降總兵趙光遠,進略保寧。時獻忠以兵拒之,乃還。八月建祖禰廟成,將往祀,忽寒慄不能就禮。自成始以巖言,謬為仁義,及巖死,又屢敗,復強很自用,偽尚書張第元、耿始然皆以小忤死。制銅鏌,官吏坐賕,即鏌斬。民盜一雞者死。西人大懼。

順治二年二月,我兵攻潼關,偽伯馬世耀以六十萬眾迎戰,敗死。潼關破,自成遂棄西安,由龍駒寨走武岡,入襄陽,復走武昌。我兵兩道追躡,連蹙之鄧州、承天、德安、武昌,窮追至賊老營,大破之者八。當是時,左良玉東下,武昌虛無人。自成屯五十餘日,賊眾尚五十餘萬,改江夏曰瑞符縣。尋為我兵所迫,部眾多降,或逃散。自成走咸寧、蒲圻,至通城,竄於九宮山。秋九月,自成留李過守寨,自率二十騎略食山中,為村民所困、不能脫,遂縊死。或曰村民方築堡,見賊少,爭前擊之,人馬俱陷泥淖中,自成腦中鉏死。剝其衣,得龍衣金印,眇一目,村民乃大驚,謂為自成也。時我兵遣識自成者驗其尸,朽莫辨。獲自成兩從父偽趙侯、偽襄南侯及自成妻妾二人,金印一。又獲偽汝侯劉宗敏、偽總兵左光先、偽軍師宋獻策。於是斬自成從父及宗敏於軍。牛金星、宋企郊等皆遁亡。

自成兄子過改名錦,偕諸賊帥奉高氏降於總督何騰蛟。時唐王立於閩,賜錦名赤心,封高氏忠義夫人,號其軍曰忠貞營,隸騰蛟麾下。永明王時,赤心封興國侯,尋死。

張獻忠者,延安衛柳樹澗人也,與李自成同歲生。長隸延綏鎮為軍,犯法當斬,主將陳洪範奇其狀貌,為請於總兵官王威釋之,乃逃去。

崇禎三年,陜西賊大起,王嘉胤據府谷,陷河曲。獻忠以米脂十八寨應之,自稱八大王。明年,嘉胤死,其黨王自用復聚眾三十六營,獻忠及高迎祥、羅汝才、馬守應等皆為之渠。其冬,洪承疇為總督,獻忠及汝才皆就撫。已而叛入山西,偕群賊焚掠。尋擾河北,又偕渡河。自是,陜西、河南、湖廣、四川,江北數千里地,皆被蹂躪。當此之時,賊渠率眾無專主,遇官軍,人自為斗,勝則爭進,敗則竄山谷不相顧。官軍遇賊追殺,亦不知所逐何賊也。賊或分或合,東西奔突,勢日強盛。

八年,十三家會滎陽,議敵官軍。守應欲北渡,獻忠嗤之,守應怒,李自成為解,乃定議。獻忠始與高迎祥並起作賊,自成乃迎祥偏裨,不敢與獻忠並。及是遂相頡頏,與俱東掠,連破河南、江北諸縣,焚皇陵。已而迎祥、自成西去。獻忠獨東,圍廬州、舒城,俱不下。攻桐城,陷廬江,屠巢、無為、潛山、太湖、宿松諸城,應天巡撫張國維禦之。獻忠從英、霍遁,道麻城,合守應等入關,會迎祥於鳳翔。已,復出商、洛,屯靈寶,以待迎祥。迎祥至,則合兵復東。總兵官左良玉、祖寬擊之,獻忠與迎祥分道走。寬追獻忠,戰於嵩縣及九皋山,三戰皆克,俘斬甚眾。獻忠恚,再合迎祥眾還戰,復大敗。迎祥尋與自成入陜西,而守應、汝才諸賊,各盤踞鄖陽、商、洛山中,不能救,獻忠亦遁山中。

明年秋,總督盧象昇去,苗胙土巡撫湖廣,不習兵。於是獻忠自均州,守應自新野,蠍子塊自唐縣,並犯襄陽,眾二餘萬。總兵秦翼明兵寡不能禦,湖廣震動。獻忠糾汝才、守應及闖塌天諸賊,順流東下,與江北賊賀一龍、賀錦等合,烽火達淮、揚。南京兵部尚書范景文、操江都御史黃道直、總兵官楊御蕃分汛固守,安池道副使史可法親率兵當賊衝。賊從間道犯安慶,連營百里,巡撫國維告警。詔左良玉、馬爌、劉良佐合兵援之,遂大破賊。賊走潛山之天王古寨,國維檄良玉搜山,良玉不應,尋北去。賊乃復出太湖,連蘄、黃,敗官軍於酆家店,殺參將程龍、陳於王等四十餘人。會總兵官牟文綬偕良佐來援,復破賊。賊皆遁,獻忠入湖廣。是時,河南、湖廣賊十五家,惟獻忠最狡黠驍勍,次則汝才。獻忠嘗偽為官兵,欲給宛城,良玉適至,獻忠倉皇走,前鋒羅岱射之中額,良玉馬追及,刃拂獻忠面,馬馳以免。會熊文燦為總理,刊檄撫賊。闖塌天者,本名劉國能,與獻忠有郤,詣文燦降。獻忠創甚,不能戰,大恐。

十一年春,偵知陳洪範隸文燦麾下為總兵,大喜,因遣間齎重幣獻洪範曰:「獻忠蒙公大恩,得不死,公豈忘之邪?願率所部降以自效。」洪範亦喜,為告文燦,受其降。巡按御史林銘球、分巡道王瑞栴與良玉謀,俟獻忠至執之,文燦不可。獻忠遂據穀城,請十萬人餉,文燦不敢決。時群賊皆聚南陽,屠掠旁州縣。文燦赴裕州,益大發檄撫賊。汝才以戰敗乞降於太和山監軍太監李繼改。明年,射塌天、混十萬、過天星、關索、王光恩等十三家渠帥,先後俱降。陜西總督洪承疇、巡撫孫傳庭復大破李自成,自成竄崤、函山中,朝廷皆謂賊撲剪殆盡。

獻忠在穀城,訓卒治甲仗,言者頗疑其欲反。帝方信兵部尚書楊嗣昌言,謂文燦能辦賊,不復憂也。夏五月,獻忠叛,殺知縣阮之鈿,隳穀城,陷房縣,合汝才兵,殺知縣郝景春。十三家降賊一時並叛,惟王光恩不從。獻忠去房縣,左良玉追擊之,羅岱為前鋒,至羅犬英山,岱中伏死,良玉大敗。

嗣昌已拜大學士,乃自請督師,帝大悅。十月朔,嗣昌至襄陽,集諸將議進兵。時群賊大掠,賀一龍、賀錦犯隨、應、麻、黃,與官軍相持。汝才及過天星竄伏漳、房、興、遠,獻忠踞湖廣、四川界,將西犯。嗣昌視東略稍緩,乃宿輜重襄陽,浚濠築城甚固,令良玉專力剿獻忠。

十三年閏正月,良玉擊賊枸坪關,獻忠遁,追至瑪瑙山。賊據山拒敵,良玉先登,賀人龍、李國奇夾擊,大敗之,斬首千三百餘級,擒獻忠妻妾。湖廣將張應元、汪之鳳追敗之水右壩。川將張令、方國安又邀擊於岔溪。獻忠奔柯家坪,張令逐北深入,被圍,應元、之鳳援之,復破賊。獻忠率千餘騎竄興、歸山中,勢大蹙。

初,良玉之進兵也,與嗣昌議不合。獻忠遣間說良玉,良玉乃圍而弗攻。獻忠因得與山民市鹽芻米酷,收潰散,掩旗息鼓,益西走白羊山。時汝才及過天星從寧昌窺大昌、巫山,欲渡江,為官兵所扼。獻忠至,遂與之合。獻忠雖累敗,氣益盛,立馬江岸,有不前赴者,輒戮之。賊爭死鬥,官軍退走。賊畢渡,屯萬頃山,歸、巫大震。已而汝才、過天星犯開縣不利,汝才東走,過天星復軼開縣而西。諸將往復追逐,獻忠乃悉眾攻楚兵於土地嶺,副將汪之鳳戰死。遂陷大昌,進屯開縣,張令戰死,石砫女土司秦良玉亦敗。汝才復自東至,與獻忠轉趨達州。川撫邵捷春退扼涪江。賊北陷劍州,將入漢中。總兵官趙光遠、賀人龍守陽平、百丈險。賊不得過,乃復走巴西。涪江師潰,捷春論死。獻忠屠綿州,越成都,陷滬州,北渡隱永川,走漢川、德陽,入巴州。又自巴走達州,復至開縣。

先是,嗣昌聞賊入川,進駐重慶。監軍萬元吉曰:「賊或東突,不可無備,宜分中軍間道出梓潼,扼歸路。」嗣昌不聽,擬令諸將盡赴滬州追賊。

十四年正月,總兵猛如虎、參將劉士傑追之開縣之黃陵城,賊還戰,官軍大敗,士傑及遊擊郭開等皆死。獻忠果東出,令汝才拒鄖撫袁繼咸兵,自率輕騎,一日夜馳三百里,殺督師使者於道,取軍符,紿陷襄陽城。獻忠縛襄王翊銘置堂下,屬之酒曰:「我欲借王頭,使楊嗣昌以陷籓誅,王其努力盡此酒。」遂殺之,并殺鄖襄道張克儉、推官鄺曰廣,復得其所失妻妾。又去,陷樊城、當陽、郟。合汝才入光州,殘商城、羅山、息縣、信陽、固始。分軍犯茶山、應城,陷隨州。偽張良玉幟,入泌陽。再攻應山,不克,去。攻鄖陽,守將王光恩力戰,始解。又拔鄖西,群盜附者萬計,遂東略地。獻忠自瑪瑙山之敗,心畏良玉,及屢勝,有驕色。秋八月,良玉追擊之信陽,大破之,降賊眾數萬。獻忠傷股,乘夜東奔,良玉急追之。會大雨,江溢道絕,官軍不能進,獻忠走免。已,復出商城,將向英山,又為副將王允成所破,眾道散且盡,從騎止數十。時汝才已先與自成合,獻忠遂投自成。自成以部曲遇之,不從。自成欲殺之,汝才諫曰:「留之使擾漢南,分官軍兵力。」乃陰與獻忠五百騎,使遁去。道糾土賊一斗穀、瓦罐子等,眾復盛,然猶佯推自成。先是,賊營革、左二賀陷含、巢、潛諸縣,欲西合獻忠,以湖廣官兵沮不得達。及汴圍急,督師丁啟睿及左良玉皆往援汴,獻忠乘間陷亳州,入英、霍山中,與革、左、二賀相見,皆大喜。

明年合攻,陷舒城、六安,掠民益軍。陷廬州,知府鄭履祥死。陷無為、廬江,習水師於巢湖。太監盧九德以總兵官黃得功、劉良佐之兵戰於夾山,敗績,江南大震。鳳陽總督高斗光、安慶巡撫鄭二陽逮治,詔起馬士英代斗光。是秋,得功、良佐大破賊於潛山,獻忠腹心婦豎盡走蘄水,革、左二賀北投自成。已,獻忠復襲陷太湖。會良玉避自成東下,盡撤湖廣兵自從。獻忠聞之,又襲陷黃梅。

十六年春,連陷廣濟、蘄州、蘄水。入黃州,黃民盡逃,乃驅婦女鏟城,尋殺之以填塹。麻城人湯志者,大姓奴也,殺諸生六十人,以城降賊。獻中改麻城為州。又西陷漢陽,全軍從鴨蛋洲渡,陷武昌,執楚王華奎,籠而沈諸江,盡殺楚宗室。錄男子二十以下、十五以上為兵,餘皆殺之。由鸚鵡洲至道士洑,浮胔蔽江,踰月人脂厚累寸,魚鱉不可食。獻忠遂僭號,改武昌曰天授府,江夏曰上江縣。據楚王第,鑄西王之寶,偽設尚書、都督、巡撫等官,開科取士。以興國州柯、陳兩姓土官悍勇,招降之。題詩黃鶴樓。下令發楚邸金振饑民。蘄、黃等二十一州縣悉附。

時李自成在襄陽,聞之忌且怒,貽書譙責。左良玉兵復西上,偽官吏多被擒殺。獻忠懼,乃悉眾趨岳州、長沙。於是監軍道王質、沔陽知州章曠、武昌生員程天一、白雲寨長易道三皆起兵討賊,蘄、黃、漢陽三府皆反正。獻忠遂陷咸寧、蒲圻,偪岳州。沅撫李乾德、總兵孔希貴等據城陵磯拒戰,三戰三克,殲其前部。獻忠怒,百道並進,乾德等不支,皆走,岳州陷。獻忠欲渡洞庭湖,卜於神,不吉,投珓而訽。將渡,風大作,獻忠怒,連巨舟千艘,載婦女焚之,水光夜如晝。騎而逼長沙,巡按劉熙祚奉吉王、惠王走衡州,總兵尹先民降,長沙陷。尋破衡州,吉王、惠王、桂王俱走永州。乃拆桂府材,載至長沙,造偽殿,而自追三王於永。熙祚命中軍護三王入廣西,身入永死守,城陷見殺。又陷寶慶、常德,發故督師楊嗣昌祖墓,斬其屍見血。攻道州,守備沈至緒戰歿,其女再戰,奪父屍還,城獲全。遂東犯江西,陷吉安、袁州、建昌、撫州、永新、安福、萬載、南豐諸府縣。廣東大震,南、韶屬城官民盡逃。賊有獻計取吳、越者,獻忠憚良玉在,不聽,決策入川中。

十七年春陷夔州,至萬縣,水漲,留屯三月。已,破涪州,敗守道劉麟長、總兵曾英兵。進陷佛圖關。破重慶,瑞王常浩遇害。是日,天無雲而雷,賊有震者。獻忠怒,發巨砲與天角。遂進陷成都,蜀王至澍率妃、夫人以下投於井,巡撫龍文光被殺。是時我大清兵已定京師,李自成遁歸西安。南京諸臣尊立福王,命故大學士王應熊督川、湖軍事,兵力弱,不能討賊。獻忠遂僭號大西國王,改元大順,冬十一月庚寅,即偽位,以蜀王府為宮,名成都曰西京。用汪兆麟為左丞相,嚴錫命為右丞相。設六部五軍都督府等官,王國麟、江鼎鎮、龔完敬等為尚書。養子孫可望、艾能奇、劉文秀、李定國等皆為將軍,賜姓張氏,分徇諸府州縣,悉陷之。保寧、順慶先已降自成,置官吏,獻忠悉逐去。自成發兵攻,不克,遂據有全蜀。惟遵義一郡及黎州土司馬金堅不下。

獻忠黃面長身虎頷,人號黃虎。性狡譎,嗜殺,一日不殺人,輒悒悒不樂。詭開科取士,集於青羊宮,盡殺之,筆墨成丘塚。坑成都民於中園。殺各衛籍軍九十八萬。又遣四將軍分屠各府縣,名草殺。偽官朝會拜伏,呼獒數十下殿,獒所嗅者,引出斬之,名天殺。又創生剝皮法,皮未去而先絕者,刑者抵死。將卒以殺人多少敘功次,共殺男女六萬萬有奇。賊將有不忍至縊死者。偽都督張君用、王明等數十人,皆坐殺人少,剝皮死,並屠其家。脅川中士大夫使受偽職,敘州布政使尹伸、廣元給事中吳宇英不屈死。諸受職者,後尋亦皆見殺。其慘虐無人理,不可勝紀。又用法移錦江,涸而闕之,深數丈,埋金寶億萬計,然後決堤放流,名水藏,曰:「無為後人有也。」當是時,曾英、李占春、于大海、王祥、楊展、曹勛等議兵並起,故獻忠誅殺益毒。川中民盡,乃謀窺西安。

順治三年,獻忠盡焚成都宮殿廬舍,夷其城,率眾出川北,又欲盡殺川兵。偽將劉進忠故統川兵,聞之,率一軍逃。會我大清兵至漢中,進忠來奔,乞為鄉導。至鹽亭界,大霧。獻忠曉行,猝遇我兵於鳳凰坡,中矢墜馬,蒲伏積薪下。於是我兵擒獻忠出,斬之。

川中自遭獻忠亂,列城內雜樹成拱,狗食人肉若猛獸虎豹,嚙人死輒棄去,不盡食也。民逃深山中,草衣不食久,遍體皆生毛。獻忠既誅,賊黨可望、能奇、文秀、定國等潰入川南,殺曾英、李乾德等,後皆降於永明王。


\end{pinyinscope}