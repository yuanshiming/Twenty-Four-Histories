\article{列傳第七十四}

\begin{pinyinscope}
○韓文顧佐陳仁張敷華楊守隨弟守隅許進子誥贊論雍泰張津陳壽樊瑩熊繡潘蕃胡富張泰吳文度張鼐冒政王璟李欽

韓文,字貫道,洪洞人,宋宰相琦後也。生時,父夢紫衣人抱送文彥博至其家,故名之曰文。成化二年舉進士,除工科給事中。核韋州軍功,劾寧晉伯劉聚,都御史王越、馬文升等濫殺妄報。尋劾越薦李秉、王竑。語頗涉兩宮,帝怒,撻之文華殿庭。已,進右給事中,出為湖廣右參議。中貴督太和山,乾沒公費。文力遏之,以其羨易粟萬石,備振貸。九谿土酋與鄰境爭地相攻,文往諭,皆服。閱七年,轉左。

弘治改元,王恕以文久淹,用為山東左參政。居二年,用倪岳薦,擢雲南左布政使。以右副都御史巡撫湖廣,移撫河南,召為戶部右侍郎。母喪除,起改吏部,進左。十六年拜南京兵部尚書。歲侵,米價翔踴。文請預發軍餉三月,戶部難之。文曰:「救荒如救焚,有罪,吾自當之。」乃發廩十六萬石,米價為平。明年召拜戶部尚書。

文凝厚雍粹,居常抑抑。至臨大事,剛斷無所撓。武宗即位,賞賚及山陵、大婚諸費,需銀百八十萬兩有奇,部帑不給。文請先發承運庫,詔不許。文言:「帑藏虛,賞賚自京邊軍士外,請分別給銀鈔,稍益以內庫及內府錢,并暫借勛戚賜莊田稅,而敕承運庫內官核所積金銀,著之籍。且盡罷諸不急費。」帝不欲發內帑,命文以漸經畫。文持大體,務為國惜財。真人陳應衣盾、大國師那卜堅參等落職,文請沒其資實國帑。舊制,監局、倉庫內官不過二三人,後漸添注,或一倉十餘人,上林苑、林衡署至三十二人,文力請裁汰。淳安公主賜田三百頃,復欲奪任丘民業,文力爭乃止。

孝宗時,外戚慶雲、壽寧侯家人及商人譚景清等奏請買補殘鹽至百八十萬引。文條鹽政夙弊七事,論殘鹽尤切。孝宗嘉納,未及行而崩,即入武宗登極詔中,罷之。侯家復奏乞,下部更議,文等再三執奏,弗從,竟如侯請。正德元年,內閣及言官復論之,詔下廷議。文言:「鹽法之設,專以備邊。今山、陜饑,寇方大入,度支匱絀,飛挽甚難。奈何壞祖宗法,忽邊防之重。」景清復陳乞如故。文等劾其桀悍,請執付法官。帝不得已,始寢前令。

榮王乞霸州莊田,崇王請自徵莊田租,勿令有司與,文皆持卻之。保定巡撫王璟請革皇莊,廷議從之,帝命再議。文請命巡撫官召民佃,畝征銀三分輸內庫,而盡撤中官管莊者,大學士劉健等亦力言內臣管莊擾民。乃命留中官各一人、校尉十人,餘如文議。中旨索寶石、西珠,文請屏絕珍奇,以養儉德。報可。帝將大婚,取戶部銀四十萬兩,文連疏請,得免四之一。

文司國計二年,力遏權倖,權倖深疾之。而是時青宮舊奄劉瑾等八人號「八虎」,日導帝狗馬、鷹兔、歌舞、角牴,不親萬幾。文每退朝,對僚屬語及,輒泣下。郎中李夢陽進曰:「公大臣,義共國休戚,徒泣何為。諫官疏劾諸奄,執政持甚力。公誠及此時率大臣固爭,去『八虎』易易耳。」文捋鬚昂肩,毅然改容曰:「善。縱事勿濟,吾年足死矣,不死不足報國。」即偕諸大臣伏闕上疏,略曰:「人主辨奸為明,人臣犯顏為忠。況群小作朋,逼近君側,安危治亂胥此焉關。臣等伏睹近歲朝政日非,號令失當。自入秋來,視朝漸晚。仰窺聖容,日漸清削。皆言太監馬永成、谷大用、張永、羅祥、魏彬、丘聚、劉瑾、高鳳等造作巧偽,淫蕩上心。擊球走馬,放鷹逐犬,俳優雜劇,錯陳於前。至導萬乘與外人交易,狎暱媟褻,無復禮體。日遊不足,夜以繼之,勞耗精神,虧損志德。遂使天道失序,地氣靡寧。雷異星變,桃李秋華。考厥占候,咸非吉徵。此輩細人,惟知蠱惑君上以便己私,而不思赫赫天命。皇皇帝業,在陛下一身。今大婚雖畢,儲嗣未建。萬一遊宴損神,起居失節,雖齏粉若輩,何補於事。高皇帝艱難百戰,取有四海。列聖繼承,以至陛下。先帝臨崩顧命之語,陛下所聞也。奈何姑息群小,置之左右,以累聖德?竊觀前古奄宦誤國,為禍尤烈,漢十常侍、唐甘露之變,其明驗也。今永成等罪惡既著,若縱不治,將來益無忌憚,必患在社稷。伏望陛下奮乾剛,割私愛,上告兩宮,下諭百僚,明正典刑,以回天地之變,泄神人之憤,潛削禍亂之階,永保靈長之業。」疏入,帝驚泣不食。瑾等大懼。

時內閣劉健、謝遷等方持言官章不肯下,文疏復入。帝遣司禮太監李榮、王岳等詣閣議。一日三至,健等持益堅。岳素剛直,獨曰:「閣議是。」是夜,八人者環泣帝前。帝怒,立收岳下詔獄,而外廷固未之知也。明日,文倡九卿科道再詣闕固爭。俄有旨,宥八人不問。健、遷倉皇致仕去。八人各分據要地,瑾掌司禮,時事遂大變。

瑾恨文甚,日令人伺文過。踰月,有以偽銀輸內庫者,遂以為文罪。詔降一級致仕,郎中陳仁謫鈞州同知。給事中徐昂乞留文原官。中旨謂顯有囑託,落文職,以顧佐代,並除昂名。二年三月榜奸黨姓名,自劉健、謝遷外,尚書則文為首,餘若張敷華、楊守隨、林瀚等凡五十三人,列於朝堂。文子高唐知州士聰,刑部主事士奇,皆削籍。文出都門,乘一藍輿,行李一車而已。瑾恨未已,坐以遺失部籍,逮文及侍郎張縉下詔獄。數月始釋,罰米千石輸大同。尋復罰米者再,家業蕩然。

瑾誅,復官,致仕。世宗即位,遣行人齎璽書存問,賚羊酒。令有司月給廩四石,歲給役夫六人終其身。復加太子太保,廕一孫光祿寺署丞。嘉靖五年卒,年八十有六。贈太傅,謚忠定。

士聰,舉人。罷官後,不復仕。士奇進士,終湖廣參政。少子士賢,亦由舉人為開封同知。孫廷瑋,進士,行太僕卿。

顧佐,字良弼,臨淮人。成化五年進士。授刑部主事,歷郎中。按錦衣指揮牛循,中官顧雄、鐘欽罪,無所撓。出為河間知府。弘治中,再遷大理少卿,擢右僉都御史巡撫山西。宗室第宅,官為繕,費不貲,佐請悉令自營治。正統末,權發太原、平陽民戍邊,後久不代,佐奏令更代。入為左副都御史,勘罷遼東總兵官李杲、太監任良、巡撫張玉。,歷戶部左、右侍郎,出理陜西軍食。善區畫,儲蓄餘三年。正德改元。代韓文為尚書。劉瑾憾文,捃摭萬端。部有故冊逸,欲以為文罪,逼佐上其事。佐不可,坐事奪俸三月。佐乃再疏乞歸,從之。瑾憾不置,三罰米輸塞上,至千餘石。家貧,稱貸以償。卒,贈太子太保。

陳仁,字子居,莆田人。成化末進士。弘治中,官戶部郎中。闕里先聖廟災,疏請修省。陜西進古璽,仁抗疏斥其偽。詔召番僧領占竹於四川,仁疏諫。又請復建文忠臣方孝孺等官。多格不行。正德初,瑾以贗銀事坐尚書文罪,仁並謫。後瑾誅,累擢至浙江右布政使。

張敷華,字公實,安福人。父洪,御史,死土木難。敷華少負氣節。年七歲,里社樹為祟,麾群兒盡伐之。景泰初,錄死事後,入國學。舉天順八年進士,選庶吉士。成化元年,與劉大夏願就部曹。除兵部主事,歷郎中。廉重不撓,名等於大夏。

十一年,出為浙江參議。景寧礦盜起,至數千人。敷華諭散之,執其魁十二人。居浙十餘年,歷布政使。弘治初,遷湖廣。歲饑,令府縣大修學宮,以人庸直資餓者。擢右副都御史,巡撫山西。中道奔喪,服闋還故官。部內賦輸大同,困於折價。敷華請太原以北可通車者仍輸米,民便之。改撫陜西,製婚娶、喪葬之式,納民於禮。妖僧據終南山為逆,廷議用兵,尚書馬文升曰:「張都御史能辦此。」敷華果以計縛僧歸。遷南京兵部右侍郎。

十二年改右都御史,總督漕運兼巡撫淮、揚諸府。高郵湖堤圮,浚深溝以殺水勢。又築寶應堤。民利賴焉。改掌南京都察院。與吏部尚書林瀚、僉都御史林俊、祭酒章懋,稱「南都四君子」,就遷刑部尚書。

正德元年召為左都御史。其冬,大臣與言官請去劉瑾等,內閣力主之。帝猶豫,敷華乃上言:「陛下宴樂逸遊,日狎憸壬,政令與詔旨相背,行事與成憲交乖,致天變上干,人心下拂。今給事中劉蒨,御史朱廷聲、徐鈺等連章論列,但付所司。英國公懋與臣等列名上請,但云『朕自處置」。臣竊歎惑,請略言時政之弊。如四十萬庫藏已竭,而取用不已。六七歲童子何知,而招為勇士。織造已停,傳奉已革,尋復如故。鹽法、莊田方遣官清核,而奏乞之疏隨聞。中官監督京營、鎮守四方者,一時屢有更易。政令紛拏,弊端滋蔓。夫國家大事,百人爭之不足,數人壞之有餘。願陛下審察。」疏入,不報。

既而朝事大變,宦官勢益張。至除夕朝罷,忽傳旨與楊守隨俱致仕。敷華即日就道。至徐州洪,坐小艇,觸石幾溺死。瑾恨未已,欲借湖廣倉儲浥爛,坐以贓罪。修撰康海過瑾曰:「吾秦人愛張公如父母,公忍相薄耶?」瑾意稍解,猶坐敷華奸黨,與守隨等榜名朝堂。明年六月病且革,衣冠揖家廟,就榻而卒。瑾誅後二年,贈太子少保,謚簡肅。

敷華性剛介。弘治時,劉大夏常薦之,帝曰:「敷華誠佳,但為人太峻耳。」為部郎奉使,盜探其囊,得七金而已。

孫鰲山,官御史。

楊守隨,字維貞,鄞人,侍郎守陳從弟也。舉成化二年進士,授御史。巡視漕運,核大同軍餉,巡按江西,所至以風採見憚。

六年,疏陳六事,言:「郕王受命艱危時,削平禍亂,功甚大。歿乃謚以『戾』,公論不平。此非先帝意,權奸逞私憾者為之也。亟宜改易,彰陛下親親之仁。尚書李秉效忠守法,一時良臣,為蕭彥莊誣劾致仕,乞即召還。律令犯公罪者不罷,近御史朱賢、婁芳等並除名,乞復其官,且戒所司毋法外加罪,一以律令從事。西征之役,以數萬甲兵討出沒不常之寇,千里轉輸,曠日持久。恐外患未平,內地先敝。乞速班師,戒邊臣慎封守。近例,軍官犯罪未結正者,遇赦即原,致此曹遷延,以希倖免。自今眾證明白者,即據律定案,毋使逃罪。雖遇赦免,亦不得管軍。在外官俸、兵餉,有踰年不給者,由郡縣蓄積少也。請於起運外,量加存留,以濟乏匱。」疏奏,時不能從。太常少卿孫廣安母喪起復,守隨與給事中李和等連章論之,乃令守制。

八年冬以災異陳時政九事。廷議四方災傷,停遣刷卷御史。會昌侯孫繼宗請并停在京者,守隨言:「繼宗等任情作奸,恐罪及,假此祈免。」帝置繼宗不問,而刷卷如故。山東饑,廷議吏納銀免考,授冠帶。守隨極言不可,帝即罷之。擢應天府丞,未上,母憂歸。服除無缺,添註視事。初,李孜省授太常寺丞,因守隨言改上林監副,憾之。至是譖於帝,中旨責守隨不當添註,調南寧知府。

弘治初,召為應天府尹,勘南京守備中官蔣琮罪。琮嗾其黨郭鏞劾守隨按給事方向獄不公,謫廣西右參政。久之,進按察使。八年召為南京右僉都御史,提督操江。歷兩京大理卿。九載滿,進工部尚書,仍掌大理寺。刑部獄送寺覆讞者多加刑,主事朱論其非。守隨言:「自永樂間,寺已設刑具。部囚多未得實,安得不更訊。」帝乃寢奏。孝宗崩,中官張瑜等以誤用御藥下獄,守隨會訊杖之。

正德元年四月,守隨奏:「每歲熱審,行於京師而不行於南京,五歲一審錄,詳於在京而略於在外,皆非是。請更定其制。」報可。中官李興擅伐陵木論死,令家人以銀四十萬兩求變其獄。守隨持之堅,獄不得解。廷臣之爭餘鹽也,中旨詰「是何大事?」守隨語韓文曰:「事誠有大於是者。」文遂偕九卿伏闕論「八黨」。文等既逐,守隨憤,獨上章極論之曰:

陛下嗣位以來,左右迫臣,不能只承德意,盡取先朝良法而更張之,盡誣先朝碩輔而刬汰之。天下嗷嗷,莫措手足,致古今罕見之災,交集數月以內。陛下獨不思其故乎?內臣劉瑾等八人,奸險佞巧,誣罔恣肆,人目為「八虎」,而瑾尤甚,日以荒縱導陛下。或在西海擎鷹搏兔,或於南城躡峻登高,禁內鼓鉦震於遠邇,宮中火炮聲徹晝夜。淆雜尊卑,陵夷貴賤。引車騎而供執鞭之役,列市肆而親商賈之為。致陛下日高未朝,漏盡不寢。此數人者,方且竊攬威權,詐傳詔旨。放逐大臣,刑誅臺諫。邀阻封章,廣納貨賂。傳奉冗員,多至千百。招募武勇,收及孩童。紫綬金貂盡予爪牙之士,蟒衣玉帶濫授心腹之人。附己者進官,忤意者褫職。內外臣僚。但知畏瑾,不知畏陛下。向也二三大臣受遺夾輔,今則有潛交默附、漏泄事機者矣。向也南北群僚,矢心痛疾,今則有畫策主文,依附時勢者矣。而且數易邊境將帥之臣,大更四方鎮守之職,志欲何為?夫太阿之柄不可授人。今陛下於兵刑財賦之區,機務根本之地,悉以委之。或掌團營,或主兩廠,或典司禮,或督倉場,大權在手,彼復何憚?於是大行殺戮,廣肆誅求。府藏竭於上,財力匱於下,武勇疲於邊。上下胥讒,神人共憤。陛下猶不覺悟,方且謂委任得人,何其舛也!伏望大奮乾綱,立置此曹重典,遠鑒延熹之失,毋使臣蹈蕃、武已覆之轍。

疏入,帝不省。瑾輩深銜之,傳旨致仕。守隨去,李興遂以中旨免死矣。

瑾憾未釋。三年四月坐覆讞失出,逮赴京繫獄,罰米千石輸塞上。踰年,復坐庇鄉人重獄,除名,追毀誥命,再罰米二百石。守隨家立破。瑾誅,復官。又十年卒,年八十五。贈太子少保,謚康簡。

從弟守隅,由進士歷官江西參政,有政績。寧府祿米,石征銀一兩,後漸增十之五。守隅入請於王,裁減如舊。瑾惡守隨,並罷守隅官。瑾死後,起官四川,終廣西布政使。

許進,字季升,靈寶人。成化二年進士。除御史。歷按甘肅、山東,皆有聲。陳鉞激變遼東,為御史強珍所劾,進亦率同官論之。汪直怒,構珍下獄,摘進他疏偽字,廷杖之幾殆。滿三考,遷山東副使。辨疑獄,人稱神明。分巡遼東,坐累,徵下詔獄。孝宗嗣位,釋還。

弘治元年擢右僉都御史,巡撫大同。小王子久不通貢,遣使千五百餘人款關,進以便宜納之。請於朝,詔許五百人至京師。已而屢盜邊,進被劾,不問。三年復窺邊,進等整軍待之。新寧伯譚祐以京軍援,乃遁去。又乞通貢,進再為請,帝許之。當是時,大同士馬盛強,邊防修整。貢使每至關,率下馬脫弓矢入館,俯首聽命,無敢譁者。會進與分守中官石巖相訐,巖徵還,進亦謫袞州知府。

七年遷陜西按察使。土魯番阿黑麻攻陷哈密,執忠順王陜巴去,使其將牙蘭守之。尚書馬文升謂復哈密非進不可,乃薦為右僉都御史,巡撫甘肅。明年蒞鎮,告諸將曰:「小醜陸梁,謂我不敢深入耳。堂堂天朝不能發一鏃塞外,何以慰遠人。」諸將難之。乃獨與總兵官劉寧謀,厚結小列禿,使以四千騎往,殺數百人,小列禿中流矢卒。小列禿故與土魯番世相仇,及死,其子卜六阿歹益憤。進復厚結之,使斷賊道,無令東援牙蘭,而重犒赤斤、罕東及哈密遺種之居苦峪者,令出兵助討。十一月,副將彭清以精騎千五百出嘉峪關前行,寧與中官陸訚統二千五百騎繼之。越八日,諸軍俱會,羽集乜川。薄暮大風揚沙,軍士寒栗僵臥。進出帳外勞軍,有異烏悲鳴,將士多雨泣。進慷慨曰:「男兒報國,死沙場幸耳,何泣為!」將士皆感奮。夜半風止,大雨雪。時番兵俱集,惟罕東兵未至,眾欲待之。進曰:「潛師遠襲,利在捷速,兵已足用,不須待也。」及明,冒雪倍道進。又六日奄至哈密城下。牙蘭已先遁去,餘賊拒守。官軍四面並進,拔其城,獲陜巴妻女。賊退保土剌。土剌,華言大臺也。守者八百人,諸軍再戰不下。問其俘,則皆哈密人為牙蘭所劫者,進乃令勿攻。或欲盡殲之,進不可,遣使撫諭即下。於是探牙蘭所嚮,分守要害。而疏請懷輯罕東諸衛為援,散土魯番黨與孤其勢,遂班師。錄功,加右副都御史。明年移撫陜西,歷戶部右侍郎,進左。十三年,火篩大舉犯大同,邊將屢敗。敕進與太監金輔、平江伯陳銳率京軍禦之,無功。言官劾輔等玩寇,並論進,致仕去。

武宗即位,乃起為兵部左侍郎,提督團營。正德元年代劉大夏為尚書。七月應詔陳時政八事,極言內監役京軍,守皇城內侍橫索月錢諸弊,多格不行。又以帝狎比群小,請崇聖學,以古荒淫主為戒,不納。中官王岳奏官校王縉等緝事捕盜功,各進一秩。進言:「邊將出萬死馘一賊,始獲晉級。此輩乃冒濫得之,孰不解體?」又言:「團營軍非為營造設,宜悉令歸伍。」居兵部半歲,改吏部,明年加太子少保。

進以才見用,能任人,性通敏。劉瑾弄權,亦多委蛇徇其意,而瑾終不悅。方進督團營時,與瑾同事。每閱操,談笑指揮,意度閒雅,瑾及諸將咸服。一日操畢,忽呼三校前,各杖數十。瑾請其故,進出權貴請託書示之。瑾陽稱善,心不喜。至是,欲去進用劉宇代。焦芳以干請不得,亦因擠進。三年八月,南京刑部郎中闕,適無實授員外郎,進循故事以署事主事二人上。瑾以為非制,令對狀。進不引咎,三降嚴旨譙責。不得已請罪,乃令致仕。未幾,坐用雍泰削其籍。二子誥、贊在翰林,俱輸贖調外任。尋與劉健等六百七十五人,並追奪誥命。瑾又摘進在大同時籍軍出雇役錢,失勾校,欲籍其家。會瑾誅得解,復官致仕。未聞命卒,年七十四。嘉靖五年謚襄毅。

子誥、贊、詩、詞、論。詩,工部郎中。詞,知府。

誥,字廷綸,進次子也。弘治十二年進士。授戶科給事中。出視延綏軍儲,論丁糧、丁草之害,帝褒納之。尋劾監督中官苗逵貪肆罪,進刑科右給事中。正德元年,父進為兵部尚書。故事,大臣子不得居言職,遂改翰林檢討。及進忤劉瑾削籍,並謫誥全州判官。父喪歸。久之,薦起尚寶丞。復引疾歸,家居授徒講學。嘉靖初,起南京通政參議,改侍講學士,直經筵,遷太常卿掌國子監。請於太學中建敬一亭,勒御製《敬一箴註》、程子《四箴》、范浚《心箴》於石。帝悅從之。帝將正文廟祀典,誥請用木主。文華殿東室舊有釋像,帝命撤去。誥所撰《道統書》言宜崇祀五帝、三王,以周公、孔子配。帝即採用其言。十一年擢吏部右侍郎。其冬,拜南京戶部尚書,弟讚亦長戶部。兄弟並司兩京邦計,縉紳以為榮。卒官,贈太子太保,謚莊敏。

誥官祭酒時,諸生旅櫬不能歸者三十餘,皆為葬之,衣食不繼者並周恤。然頗善傅會。時有白鵲之瑞,誥獻論,司業陳寰獻頌,並宣付史館。給事中張裕、謝存儒,御史馮恩皆劾誥,裕至比之祝欽明。帝怒,下裕獄,謫福建布政司照磨,存儒亦調邊方。恩詆誥學術迂邪,誥求罷。帝曰:「恩所詆乃指前日去土偶用木主事也。爾以是介意邪?」其為帝眷寵如此。

言贊,字廷美,進第三子也。弘治九年進士。授大名推官。亦以辨疑獄知名,召拜御史。正德元年改編修。劉瑾逐進,言讚亦出為臨淄知縣。累遷浙江左布政使。

嘉靖六年入為光祿卿,歷刑部左、右侍郎。知州金輅謫戍,賂武定侯郭勛。勛遣人篡取之,指揮王臣不與。縛臣以歸,掠取其賄。事覺,言讚等請論如律。帝憐勛,諭法司毋刑輅等,輅等遂不承。尚書高友璣在告,坐畏縮,被劾去。言贊請如常訊,具得勛納賄狀,乃再奪其祿。

八年,進尚書。詔許六部歷事監生發廷臣奸弊。有詹摐者,訐吏部侍郎徐縉,下都御史汪鋐訊。摐語塞,已論罪,摐復訐縉及通政陳經等。再下鋐訊,鋐力斥其妄。會太常卿彭澤欲傾縉代之,偽為縉書抵張孚敬求解,復惎孚敬劾縉賄己。縉疏辨,詔法司會錦衣衛訊。言讚等卒論摐誣罔,而縉行賄事莫能白,坐除名。帝方嘉摐能奉詔言事,竟宥摐罪。於是無賴子率持朝士陰事,索資財,妄構事端入奏,諸司為惕息。軍人童源訐中官張永造塋,犯天壽山龍脈,復嗾永弟容僕王謙等發容違法事。奸人張雄又為謙草奏,詆言贊與兄誥及汪鋐、廖道南、史道,內臣黃錦輩數十人受容重賂,源亦上疏助之。鞫得實,源等並戍極邊,告訐始少衰。

十年,改言讚戶部尚書。馳驛歸省母。母先卒。服未闋,詔以為吏部尚書,服除始入朝。帝以言贊醇謹,虛位待。及至,論列不當意。詔選宮僚,閣臣多引私黨,言官劾罷十餘人,帝以屬吏部。言贊乃舉霍韜、毛伯溫、顧璘、呂柟、鄒守益、徐階、任瀚、薛蕙、周鈇、趙時春等,詔璘、柟、蕙仍故官,餘俱用之。屢加少保兼太子太保。九廟災,自陳免。居半歲,帝難其代,復起言讚任之。請發內帑,借百官俸,括富民財,開鬻爵之令,以濟邊需。時議內地築墩堡,言贊謂非計。帝以借俸、括財非盛世事,已之。墩堡議亦寢。翟鑾、嚴嵩柄政,多所請托。郎中王與齡勸言贊發之。嵩辨之強,帝眷嵩,反切責言讚,除與齡籍。言贊自是懾嵩不敢抗,亦頗以賄聞矣。鑾罷,帝謀代者。嵩以言讚柔和易制,引之。詔以本官兼文淵閣大學士參預機務。政事一決於嵩,言讚無所可否。久之加少傅。以年踰七十,數乞休。帝責其忘君愛身,落職閒住。歸三年卒。後復官,贈少師,謚文簡。

論,字廷議,進少子也。嘉靖五年進士。授順德推官,入為兵部主事,改禮部。好談兵,幼從父歷邊境,盡知阨塞險易,因著《九邊圖論》上之。帝喜,頒邊臣議行,自是以知兵聞。累遷南京大理寺丞。會廷推順天巡撫,論名列第二。帝曰:「是上《九邊圖論》者」,即拜右僉都御史,任之。白通事以千餘騎犯黃崖口,論督將士敗之。再犯大木谷,復為官軍所卻。錄功,進右副都御史。歲餘,以病免。俺答薄都城,起故官撫山西。錄防秋功,進兵部右侍郎,召理京營戎政。以築京師外城轉左。

三十三年出督宣、大、山西軍務。奸人呂鶴初與邱富以左道惑眾。富叛降俺答,為之謀主。鶴遣其黨闌出塞外,引寇入犯,為偵卒所獲。論遣兵捕鶴,並誅其黨。以功進右都御史,再以功進兵部尚書,蔭子錦衣世千戶。翁萬達為總督,築大同邊牆六百里,里建一墩臺於牆內。後以兵少墻不能守,盡撤而守臺。論言:「兵既守臺,則寇攻牆不得用其力。及寇入牆,率震駭逃散。請改築於牆外,每三百步建一臺,俾矢石相及。去墻不得越三十步,高廣方四丈五尺,其顛損三之一,上置女牆、營舍,守以壯士十人。下築月城,穴門通出入。度工費不過九萬金,數月而足。」詔立從之。寇萬騎犯山西,論督軍遮破之朔州川。其犯宣府、龍門者,亦為將士所敗,先後俘斬五百三十有奇。加太子太保,蔭子如初。

三十五年,兵部尚書楊博以父喪去,召論代之。當是時,嚴嵩父子用事,將帥率以賄進。南北用兵,帝責中樞甚急。丁汝夔、王邦瑞、趙錦、聶豹,咸不得善去。論時已老,重自顧念。一切將帥黜陟,兵機進止,悉聽世蕃指揮,望由此損。俺答子辛愛憤總督楊順納其逃妾,擁眾圍大同右衛城數重,城中析屋而爨。帝聞,深以為憂,密問嵩。嵩意欲棄之而難於發言,則請降諭問本兵。論請復右衛軍馬,歲辦五十萬金,故為難詞,冀以動帝。帝顧亟措餉發兵,易置文武將吏,右衛圍亦尋解。給事中吳時來劾楊順,因言論雷同附和,日昏酣,置邊警度外。帝遂削論籍。嵩微為之解,亦不能救也。

三十八年復起故官,督薊、遼、保定軍務。把都兒犯薊西,論厚集精銳以待。至則為遊擊胡鎮所破。分掠沙兒嶺、燕子窩,又卻,乃遁去。事聞,厚齎銀幣。尋又奏密雲、昌平二鎮防秋,須餉銀三十餘萬。給事中鄭茂言論奏請過多,請察其侵冒弊,詔論回籍聽勘。給事中鄧棟往核,具得虛冒狀,奪官閒住。未幾卒,年七十二。隆慶初,復官,謚恭襄。

曾孫浩然,由世廕歷官太子太保,左都督。浩然子達胤,錦衣指揮。李自成陷京師,不屈死。其從兄佳胤,弘農衛指揮。崇禎十四年賊破靈寶,持刀赴斗,死焉。

雍泰,字世隆,咸寧人。成化五年進士。除吳縣知縣。太湖漲,沒田千頃,泰作堤為民利,稱「雍公堤」。民妾亡去,妾父訟其夫密殺女匿屍湖石下。泰詰曰:「彼密殺汝女,汝何以知匿所。且此非兩月尸,必汝殺他人女,冀得賂耳。」一考而服。

召為御史,巡鹽兩淮。灶丁無妻者,泰為婚匹。出知鳳陽府。父憂去,服闋起知南陽。餘子俊督師,薦為大同兵備副使,擢山西按察使。泰剛廉,所至好搏擊豪強。太原知府尹珍塗遇弗及避,泰召至,跽而數之。珍不服,泰竟笞珍。珍訴於朝,且告泰非罪杖人死,逮下詔獄。王恕請寬泰罪,會事經赦,乃降湖廣參議。弘治四年轉浙江右布政使,復以母憂去。

十二年起右副都御史,巡撫宣府。官馬死,軍士不能償,泰言於朝,以官帑市。邊軍貧,有妻者輒鬻,泰請官為資給。尚書周經因令貧者給聘財,典賣者收贖,軍盡歡。參將王傑有罪,泰劾之,下泰逮問。泰又請按千戶八人,帝以泰屢抑武臣,方詔都察院行勘。而參將李稽坐事畏泰重劾,乞受杖,泰取大杖決之。稽乃奏泰凌虐,帝遣給事中徐仁偕錦衣千戶往按。傑復使人走登聞鼓下,訟泰妄逮將校至八十六人,並及其婿納賂事。法司核上,褫為民。

武宗立,給事中潘鐸等薦泰有敢死之節,克亂之才。吏部尚書馬文升遂起泰南京右副都御史,提督操江,固辭不赴。正德三年春,許進為吏部,復起前官。七月擢南京戶部尚書。劉瑾,泰鄉人也,怒泰不與通,甫四日即令致仕。謂進私泰,遂削二人籍,而追斥馬文升及前薦泰者尚書劉大夏、給事中趙士賢、御史張津等為民,其他罰米輸邊者又五十餘人。泰歸,居韋曲別墅,不入城市。瑾誅,復官,致仕。年八十卒。卒時榻下有聲若霆者。

泰奉身儉素。貴賓至,不過二肉。為尚書,無緋衣。及卒,家人始製以斂。天啟中,追謚端惠。

張津,字廣漢,博羅人。成化末進士,除建陽知縣。築城郭,遏礦盜,建朱熹、蔡元定諸賢祠,置祭田畀其子孫。憂歸,補大治,徵授御史。弘治十四年冬,吏部缺尚書,廷臣推馬文升、閔珪,而津偕同官文森、曾大有請用致仕尚書周經、兩廣總督劉大夏。忤旨下詔獄。給事御史論救,得釋。已,言:「陛下延訪大臣,而庶官不預,非所以明目達聰也。乞命卿佐侍從及考滿朝覲諸外僚,咸得以時進見,通達下情。」武宗初,巡按廣西,劾總鎮中官韋經擅移官帑。預平富賀賊,被賚,出為泉州知府。坐嘗舉泰,勒為民。劉瑾敗,起寧波知府,遷山東左參政,擢右僉都御史,提督操江。進右副都御史,巡撫應天諸府。所部水旱,請停織造。車駕北巡,疏諫,不報。浙孝豐奸民據深山拒捕,積二十年莫能制。津托別事赴浙,悉縛之。加戶部右侍郎,巡撫如故。帝自宣府還,復欲北幸,津疏切諫,不報。卒,贈南京戶部尚書。

陳壽,字本仁,其先新淦人。祖志弘,洪武間代兄戍遼東,遂籍寧遠衛。壽少貧甚。,得遺金,坐守至夜分,還其主。從鄉人賀欽學,登成化八年進士,授戶科給事中。視宣、大邊防,劾去鎮守中官不檢者。又嘗劾萬貴妃兄弟及中官梁芳、僧繼曉,繫詔獄。得釋,屢遷都給事中。

弘治元年,王恕為吏部,擢壽大理丞。劉吉憾恕,諷御史劾壽不習刑名,冀以罪恕。竟調壽南京光祿少卿,就轉鴻臚卿。

十三年冬,以右僉都御史巡撫延綏。火篩數盜邊,前鎮巡官俱得罪去。壽至,蒐軍實,廣間諜,分布士馬為十道,使互相應援,軍勢始振。明年,諸部大入,先以百餘騎來誘。諸將請擊之,壽不可。自出帳,擁數十騎,據胡床指麾飲食。寇望見,疑之,引去。諸道襲擊,斬獲甚多。朝廷方遣苗逵等重兵至,而壽已奏捷。孝宗嘉之,加錄一等。逵欲乘勝搗巢。駐延綏久,戰馬三萬匹日費芻菽不貲。壽請出牧近塞,就水草,眾有難色。壽跨馬先行,眾皆從之,省費數十萬。當戰捷時,或勸注子弟名籍,壽曰:「吾子弟不知弓槊,寧當與血戰士同受賞哉?」竟不許。

十六年以右副都御史掌南院。正德初,劉瑾矯詔逮南京科道戴銑、薄彥徽等,壽抗章論救。瑾怒,令致仕。尋坐延綏倉儲虧損,罰米二千三百石、布千五百匹。貧不能償,上章自訴。瑾廉知壽貧,特免之。中官廖堂鎮陜西貪暴,楊一清以壽剛果,九年正月起撫其地。堂初奉詔製氈幄百六十間,贏金數萬,將遺權倖。壽檄所司留備振,復戒諭堂勿假貢獻名有所科取。堂怒,將傾之。壽四疏乞休,不得。堂爪牙數十輩散府縣漁利,壽命捕之,皆逃歸,氣益沮。其秋,拜南京兵部侍郎,陜人號呼擁輿,移日不得行。踰年,乞駭骨,就進刑部尚書,致仕。

壽為給事中,言時政無隱,獨不喜劾人,曰:「吾父戒吾勿作刑官,易枉人。言官枉人尤甚,吾不敢妄言也。」嘉靖改元,詔進一品階,遣有司存問,時年八十有三。壽廉,歷官四十年,無家可歸。寓南京,所居不蔽風雨。其卒也,尚書李充嗣、府尹寇天敘為之斂。又數年,親舊賻助,始得歸葬新淦。

樊瑩,字廷璧,常山人。天順末,舉進士,引疾歸養。久之,授行人,使蜀不受餽,土官作卻金亭識之。

成化八年,擢御史。山東盜起,奉命捕獲其魁。清軍江北,所條奏多著為例。改按雲南,交阯誘邊氓為寇,馳檄寢其謀。出知松江府。運夫苦耗折,瑩革民夫,令糧長專運,而寬其綱,用以優之。賦役循周忱舊法,稍為變通,民困大蘇。憂歸,起知平陽。

弘治初,詔大臣舉方面官。侍郎黃孔昭以瑩應,尚書王恕亦器之,擢河南按察使。黃河為患,民多流移。瑩巡振,全活甚眾。河南田賦多積弊,巡撫都御史徐恪欲考本末,眾難之。瑩曰:「視萬猶千,視千猶百耳,何難。」恪以屬瑩部吏鉤考,旬日間,宿蠹一清。四年遷應天府尹。守備中官蔣琮與言官訐奏,所蔓引多至罪黜。瑩承命推鞫,初若不為異者,琮大喜。後奏其傷孝陵山脈事,琮遂下獄,充凈軍。

七年遷南京工部右侍郎,尋改右副都御史巡撫湖廣。錦田賊結兩廣瑤、僮為寇,瑩諭散餘黨,戮首惡十八人。歲餘,以疾乞休。家居七年,中外交薦,起故官撫治鄖陽,旋改南京刑部右侍郎。

十六年,雲南景東衛晝晦七日,宜良地震如雷,曲靖大火數發,貴州亦多災異,命瑩巡視。至則劾鎮巡官罪,黜文武不職者千七百人。廉知景東之變,乃指揮吳勇侵官帑,圖脫罪,因云霧晦冥虛張其事,劾罪之。還進本部尚書。

武宗踐阼,致仕歸。劉瑾以會勘隆平侯爭襲事,連及瑩,削籍。明年又坐減松江官布,罰米五百石輸邊。瑩素貧,至是益窘。三年十一月卒,年七十五。瑾敗,復官,贈太子少保,謚清簡。

瑩性誠愨,農月坐籃輿戴笠,子孫舁行田間,曰:「非徒視稼,欲子孫習勞也。」其後人率教,多愿朴力學者。

熊繡,字汝明,道州人,其先以戍籍自豐城徙焉。繡舉成化二年進士,授行人。奉使楚府,巡茶四川,力拒饋遺。擢御史,巡按陜西。左布政於璠以官帑銀饋苑馬卿邵進,繡發其罪。璠遁赴京訐繡,帝並下繡吏,謫知清豐,璠、進亦除名。久之,鳳翔闕知府,擢繡任之。

弘治初,遷山東左參政,進右布政使。七年以右副都御史巡撫延綏。榆林初僅小堡,屯兵備冬。景泰中,始移巡撫、總兵官居之,遂為西北巨鎮,城隘弗能容,繡因請增築千二百餘丈。涖鎮數年,練兵積粟,邊政修舉。歷兵部左、右侍郎,尚書劉大夏深倚信之。勝騰四衛勇士額三四萬人,率虛籍。歲糜錢穀數十萬,多入奄人家。廷臣屢請稽核,輒被撓。十八年命繡清釐,未竟而孝宗崩。朝政漸變,繡力持不顧,得詭冒者萬四千人。御馬太監寧瑾等疏請復舊,給事御史交章劾瑾,大夏亦力爭。武宗不得已從之,而宥瑾等不問。

正德元年擢右都御史,總督兩廣軍務兼巡撫事。既抵鎮,盡裁幕府供億,秋毫無所取。二年與總兵官伏羌伯毛銳討平賀縣僮。劉瑾以前汰勇士事深疾繡,伺察無所得。召掌南京都察院事,尋以中旨罷之。已,復摭延綏倉儲浥爛為繡罪,罰米五百石,責繡躬輸於邊。繡家遂破。

十年閏四月卒,無子。巡撫秦金頌其清節於朝,贈刑部尚書。太僕少卿何孟春以繡承繼孫幼且貧,無以為養,請如主事張鳳翔孔琦例,賜月廩,且乞予謚。遂謚莊簡,給其孫米月一石。

潘蕃,字廷芳,崇德人。初冒鐘姓,既顯始復。成化二年舉進士,授刑部主事。歷郎中。雲南鎮守中官錢能為巡撫王恕所劾,詔蕃按,盡得其實。出為安慶知府,改鄖陽。時府治初設,陜、洛流民畢聚。蕃悉心撫循,皆成土著。累遷山東、湖廣左右布政使。

弘治九年,以右副都御史巡撫四川,兼提督松潘軍務。宣布威信,蠻人畏服,單車行松、茂莫敢犯。遷南京兵部右侍郎,就改刑部。

十四年進右都御史,總督兩廣。帳下士舊不下萬人,蕃汰之,纔給使令而已。黎寇符南蛇亂海南,聚眾數萬。蕃令副使胡富調狼土兵討斬之,平賊巢千二百餘所。論功,進左都御史。已,又平歸善劇賊古三仔、唐大鬢等。思恩知府岑濬與田州知府岑猛相仇殺,攻陷田州,猛窮乞援。蕃諭濬罷兵,不從,乃與鎮守太監韋經、總兵官伏羌伯毛銳集兵十餘萬,分六哨討之。濬死,傳首軍門,斬級四千七百,盡平其地。迴軍討平南海縣豐湖賊褟元祖。捷聞,璽書嘉勞。蕃奏,思恩宜設流官,猛構兵失地,宜降同知,俾還守舊土。兵部尚書劉大夏議,猛世濟凶惡,不宜歸舊治,請兩府皆設流官,而降猛為千戶,徙之福建。帝從之。正德改元之正月召為南京刑部尚書。踰年,致仕。

初,蕃去兩廣,岑猛據田州不肯徙,知府謝湖畏猛悍,亦逗遛。事聞,逮湖詔獄。湖委罪蕃及韋經、毛銳,經復委罪於尚書大夏。劉瑾方惡大夏,遂並逮四人。大夏以不從蕃言為罪,而蕃亦坐不能撫猛,俱謫戍肅州,三年九月也。既而瑾從戶部郎中莊言,遣太監韋霦核廣東庫藏,奏應解贓罰諸物多朽敝,梧州貯鹽利軍賞銀六十餘萬兩不以時解。逮問蕃及前總督大夏、前左布政使仁和沈銳等八百九十九人,罰米輸邊。銳廉介,已遷南京刑部右侍郎,乞休歸,至是奪職。瑾誅,蕃以原官致仕。踰六年,卒。銳至嘉靖初,始復職致仕。

方蕃解官歸,無屋,稅他人宅居之。與鄉人飲,露坐花下,醉則任所之。其風致如此。

胡富,字永年,績溪人。成化十四年進士。授南京大理評事。弘治初,歷福建僉事。福寧繫囚二百餘人,富一訊皆定,囹圄頓空。以憂去,起補山東,遷廣東副使。四會瑤亂,剿擒五百餘人。瀧水瑤出沒無時,富度其所經地,得荒田三千餘頃,招僮戶耕牧其中。瑤畏僮不敢出擾,居民得田作。符南蛇圍儋州,富與參議劉信往覘。賊突至,殺信,富手斬劇賊一人,賊乃退。還益兵討平之。歷陜西左、右布政使。

正德初,入為順天府尹。三年進南京大理寺卿,就遷戶部右侍郎。五年正月坐大理時勘事遲緩,勒致仕。亦瑾意也。瑾敗,起故官。七年拜本部尚書。南都倉儲僅支一年,富在部三載,有六年積。上十餘事,率權貴所不便,格不行,遂引年歸。嘉靖元年卒。贈太子少保,謚康惠。

張泰,字叔亨,廣東順德人。成化二年進士。除知沙縣。時經鄧茂七之亂,泰撫綏招集,流亡盡復。入為御史,偕同官諫萬貴妃干政,廷杖幾斃。出督京畿學校,以憂去,家居十餘年。

弘治五年起故官,按雲南。孟密土舍思揲構亂,以兵遏木邦宣慰使罕挖法於孟乃寨。守臣撫諭,拒不聽。泰與巡撫張誥集兵示必討,思揲懼,始罷兵。滇池溢,為民災,泰築堤以弭其患。還朝,乞罷織造內臣,減皇莊及貴戚莊田被災稅賦,給畿省災民牛種。詔止給牛種,餘不行。寇入永昌,甘肅遊擊魯麟委罪副總兵陶禎,而總兵官劉寧疏言守臣不和,詔泰往勘。泰奏鎮守太監傅德、故總兵官周玉侵據屯田。巡撫馮續減削軍餉,寇數入莫肯為禦,失士卒六百餘、馬駝牛羊二萬皆不以聞。帝怒,下之吏。德降內使,錮南京,續編氓口外。泰又言甘州膏腴地悉為中官、武臣所據,仍責軍稅;城北草湖資戍卒牧馬,今亦被占。請悉歸之軍,且推行於延、寧二鎮,詔皆從之。遷太僕少卿,改大理。

初,薊州民田多為牧馬草場所侵,又侵御馬監及神機營草場、皇莊,貧民失業,草場亦虧故額。孝宗屢遣給事中周旋,侍郎顧佐、熊翀等往勘,皆不能決。至是命泰偕錦衣官會巡撫周季麟復勘。泰密求得永樂間舊籍,參互稽考,田當歸民者九百三十餘頃,而京營及御馬監牧地咸不失故額。奏入,駁議者再,尚書韓文力持之,留中未下。及武宗嗣位,文再請,始出泰奏,流亡者咸得復業。

尋遷右副都御史督儲南京。奏釐革十二事,多報可。正德二年,召為工部右侍郎,踰年遷南京右都御史。泰清謹。劉瑾專權,朝貴爭賂遺。泰奏表至京,惟饋土葛。瑾憾之,其年十月令以南京戶部尚書致仕。明年七月卒,摭他事罰米數百石。瑾誅,予葬祭如制。

吳文度,字憲之,晉江人,從父客江寧,遂家焉。登成化八年進士,除龍泉知縣,徵授南京御史。偕同官孫需等論妖僧繼曉,被廷杖。尋遷汀州知府。瑤弗靖,設方略綏撫,瑤承賦如居民。弘治中歷江西左參政,山西、河南左、右布政使。正德元年遷右副都御史,巡撫雲南。師宗州賊阿本等作亂,諭不從,乃遣參議陳一經等督軍二萬攻之,別遣兵截盤江,據賊巢背,先後俘斬千人。入歷戶部侍郎。三年冬進南京右都御史。方文度召自雲南,劉瑾以地產金寶,屢責賄。文度無以應,瑾深銜之。會工部尚書李鐩致仕,廷推文度及南京戶部侍郎王珩,遂改文度南京戶部尚書,與珩俱致仕。命下,舉朝駭異。既歸,所居屋僅數椽。瑾誅,未及用而卒。珩,趙人。起家進士,亦以清操聞。

張鼐,字用和,歷城人。成化十一年進士。授襄陵知縣,入為御史。憲宗末年數笞言官,鼐力諫。又嘗劾妖僧繼曉、方士鄧常恩等。帝心惡之。出按江西。盜賊多強宗佃僕,鼐與巡撫閔珪交奏其事。尹直等構之,乃貶珪而坐鼐尹旻黨,謫郴州判官。

弘治初,擢河南僉事,進參議,以協治黃陵岡遷副使。十五年進按察使。鼐官河南久,屢遭河患,督治有方,民為立祠。是年秋,擢右僉都御史巡撫遼東。時軍政久馳,又許餘丁納資助驛遞,給冠帶,復其身。邊人競援例避役。鼐言不可,因條上定馬制、核屯糧、清隱占、稽客戶、減軍伴數事,悉允行。尋劾分守中官劉恭貪虐罪,築邊牆自山海關迄開原靉陽堡凡千餘里。遼撫自徐貫後,歷張岫、張玉、陳瑤、韓重四人,多得罪去,至鼐稱能。

武宗立,移撫宣府。正德改元,召還,尋進右副都御史署院事。有知縣犯贓當褫職,卒殺人當抵死。劉瑾納重賄,欲寬之,鼐執不可,出為南京右都御史。焦芳子黃中欲強市其居,畀通政魏訥,鼐不從,芳父子亦怨之。會瑾遣給事中王翊等核遼東軍餉,還奏芻粟多浥爛,遂以為守臣罪,逮鼐及繼任巡撫馬中錫、鄧章,前參政冒政,參議方矩,郎中王藎、劉繹下詔獄,令其家人輸米遼東。鼐坐輸二千石,以力不辦,繫遼東。久之,總兵官毛倫等具奏諸人苦狀,請得折價,瑾勉從之。閱三年事始竟,皆斥為民。瑾誅,復官。鼐前卒,世宗初予恤。

冒政,泰州人。鼐同年進士,歷官右副都御史,巡撫寧夏。守官廉,劉瑾覬賄不得,遂假遼東事逮之,罰米至三千石。瑾誅,復職致仕。久之,卒。

王璟,字廷采,沂人。成化八年進士。為登封知縣。歷兩京御史。

弘治十四年,以南京鴻臚卿拜右僉都御史,理兩浙鹽政。振荒浙江,奏行荒政十事,多所全活。十七年冬巡撫保定。武宗立,太監夏綬乞於真定諸府歲加葦場稅,少監傅琢請履畝核靜海、永清、隆平諸縣田,太監張峻欲稅寧晉小河往來客貨,詔皆許之。又以莊田故,遣緹騎逮民魯堂等二百餘人,畿南騷動。璟抗疏切諫。尚書韓文等力持之,管莊內臣稍得召還。

正德元年四月引疾致仕,命馳傳歸。三年坐累奪官閒住。六年起撫山西。製火槍萬餘,槍藏箭六,皆傅毒藥,用以禦寇,寇不敢西。累遷右都御史。已,遷左,以張綸為右都御史代之。後陳金以太子太保左都御史入院,位璟上,人號璟「中都御史」焉。時群小用事,大臣靡然附之,璟獨守故操。再進太子太保。世宗立,致仕,卒。贈少保,謚恭靖。

初,璟自保定巡撫歸,其後兵科給事中高淓勘滄州鹽山牧地,劾六十一人,及璟與前巡撫都御史高銓。銓即淓父也。詔去職者勿問,璟、銓並獲免。

銓,江都人,累官南京戶部尚書。正德二年廷推左都御史,瑾勒令致仕。尋坐事逮下獄,復坐隆平侯家襲爵事除名,罰米五百石。後瑾益事操切,每遣使勘核,多務苛急承瑾意,淓遂並銓在劾中。淓後官至光祿少卿,以劾父不齒於人。瑾誅,銓復官致仕,卒。贈太子少保。

朱欽,字懋恭,邵武人。師吳與弼,以學行稱。舉成化八年進士,授寧波推官。治最,徵授御史。出督漕運,按河南,清軍廣西,並著風節。

弘治中,遷山東副使,歷浙江按察使。十五年入覲。吏部舉天下治行卓異者六人,欽與焉。僉都御史林俊又舉欽自代,乃稍遷湖廣左布政使。

武宗立,以右副都御史巡撫山東。中官王岳被謫,道死。欽上言:「岳謫守祖陵,罪狀未暴,賜死道路,不厭人心。臣知岳為劉瑾輩所惡,必瑾譖毀以至此。望陛下察岳非辜,懲瑾讒賊。」疏至,瑾屏不奏,銜之。欽以山東俗淫酗,嚴禁市酤,令濟南推官張元魁察之,犯者罪及鄰。比有懼而自縊者,其母欲奏訴,元魁與知府趙璜賄之乃已。瑾使偵事校尉發之,俱逮下詔獄,勒欽致仕,璜除名,元魁謫戍。瑾憾欽未已,摭前湖廣時小故,下巡按御史逮問。俄坐山東勘地事,斥為民。又坐修曲阜先聖廟會計數多,罰輸米六百石塞下。又坐撫山東時,以民夫給事尚書秦紘家,再下巡按御史逮問。瑾誅,乃復官。十五年卒,年七十七。與弼之門以宦學顯者,欽為稱首。

贊曰:武宗初,劉、謝受遺輔政,韓文、張敷華等為列卿長,當路多正人,國事有賴。「八虎」潛伏左右,雖未敢顯與朝士為難,固腹心之蠹也。夫以外攻內,勢所甚難。況相權之輕,遠異前代,雖抱韓琦之忠,初無書敕之柄。區區爭勝於筆舌間,此難必之剛明之主,而以望之武宗,庸有濟乎?一擊不勝,反噬必毒,消長之機,間不容發。宦豎之貽禍烈也,籲可畏哉!


\end{pinyinscope}