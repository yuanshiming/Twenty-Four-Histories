\article{列傳第八十四 張璁胡鐸 桂萼 方獻夫 夏言}

\begin{pinyinscope}
張璁,字秉用,永嘉人。舉於鄉,七試不第。將謁選,御史蕭鳴鳳善星術象的變化過程和以之推測人事吉兇的雙重含義。《國語·周語,語之曰:「從此三載成進士,又三載當驟貴。」璁乃歸。正德十六年登第,年四十七矣。

世宗初踐阼,議追崇所生父興獻王。廷臣持之,議三上三卻。璁時在部觀政,以是年七月朔上疏曰:「孝子之至,莫大乎尊親。尊親之至,莫大乎以天下養。陛下嗣登大寶,即議追尊聖考以正其號,奉迎聖母以致其養,誠大孝也。廷議執漢定陶、宋濮王故事,謂為人後者為之子,不得顧私親。夫天下豈有無父母之國哉?《記》曰:『禮非天降,非地出,人情而已。』漢哀帝、宋英宗固定陶、濮王子,然成帝、仁宗皆預立為嗣,養之宮中,其為人後之義甚明。故師丹、司馬光之論行於彼一時則可。今武宗無嗣,大臣遵祖訓,以陛下倫序當立而迎立之。遺詔直曰『興獻王長子』,未嘗著為人後之義。則陛下之興,實所以承祖宗之統,與預立為嗣養之宮中者較然不同。議者謂孝廟德澤在人,不可無後。假令聖考尚存,嗣位今日,恐弟亦無後兄之義。且迎養聖母,以母之親也。稱皇叔母,則當以君臣禮見,恐子無臣母之義。《禮》『長子不得為人後』,聖考止生陛下一人,利天下而為人後,恐子無自絕其父母之義。故在陛下謂入繼祖後,而得不廢其尊親則可;謂為人後,以自絕其親則不可。夫統與嗣不同,非必父死子立也。漢文承惠帝後,則以弟繼;宣帝承昭帝後,則以兄孫繼。若必奪此父子之親,建彼父子之號,然後謂之繼統,則古有稱高伯祖、皇伯考者,皆不得謂之統乎?臣竊謂今日之禮,宜別立聖考廟於京師,使得隆尊親之孝,且使母以子貴,尊與父同,則聖考不失其為父,聖母不失其為母矣。」帝方扼廷議,得璁疏大喜,曰:「此論出,吾父子獲全矣。」亟下廷臣議。廷臣大怪駭,交起擊之。禮官毛澄等執如初。會獻王妃至通州,聞尊稱禮未定,止不肯入。帝聞而泣,欲避位歸籓。璁乃著《大禮或問》上之,帝於是連駁禮官疏。廷臣不得已,合議尊孝宗曰「皇考」,興獻王曰「本生父興獻帝」,璁亦除南京刑部主事以去,追崇議且寢。

至嘉靖三年正月,帝得桂萼疏心動,復下廷議。汪俊代毛澄為禮部,執如澄。璁乃復上疏曰:「陛下遵兄終弟及之訓,倫序當立。禮官不思陛下實入繼大統之君程朱之學。與門人李塨共倡「實學」,強調「習行」、「習動」、,而強比與為人後之例,絕獻帝天性之恩,蔑武宗相傳之統,致陛下父子、伯姪、兄弟名實俱紊。寧負天子,不敢忤權臣,此何心也?伏睹聖諭云:『興獻王獨生朕一人,既不得承緒,又不得徽稱,罔極之恩何由得報?』執政窺測上心,有見於推尊之重,故今日爭一帝字,明日爭一皇字。而陛下之心,亦日以不帝不皇為歉。既而加稱為帝,謂陛下心既慰矣,故留一皇字以覘陛下將來未盡之心,遂敢稱孝宗為皇考,稱興獻帝為本生父。父子之名既更,推崇之義安在?乃遽詔告天下,乘陛下不覺,陷以不孝。《禮》曰:『君子不奪人之親,亦不可奪親也。』陛下尊為萬乘,父子之親,人可得而奪之,又可容人之奪之乎?故今日之禮不在皇與不皇,惟在考與不考。若徒爭一皇字,則執政必姑以是塞今日之議,陛下亦姑以是滿今日之心,臣恐天下知禮者,必將非笑無已也。」與桂萼第二疏同上。帝益大喜,立召兩人赴京。命未達,兩人及黃宗明、黃綰復合疏力爭。及獻帝改稱「本生皇考」,閣臣以尊稱既定,請停召命,帝不得已從之。二人已在道,復馳疏曰:「禮官懼臣等面質,故先為此術,求遂其私。若不亟去本生之稱,天下後世終以陛下為孝宗之子,墮禮官欺蔽中矣。」帝益心動,趣召二人。

五月抵都,復條上七事。眾洶洶,欲撲殺之。萼懼,不敢出。璁閱數日始朝。給事御史張翀、鄭本公等連章力攻,帝益不悅,特授二人翰林學士。二人力辭,且請面折廷臣之非。給事御史李學曾、吉棠等言:「璁、萼曲學阿世,聖世所必誅。以傳奉為學士,累聖德不少。」御史段續、陳相又特疏論,并及席書。帝責學曾等對狀,下續、相詔獄。刑部尚書趙鑑亦請置璁、萼於理,語人曰:「得俞旨,便捶殺之。」帝責以朋奸,亦令對狀。璁、萼乃復列欺罔十三事,力折廷臣。及廷臣伏闕哭爭,盡繫詔獄予杖。死杖下者十餘人,貶竄相繼。由是璁等勢大張。其年九月卒用其議定尊稱。帝益眷倚璁、萼,璁、萼益恃寵仇廷臣,舉朝士大夫咸切齒此數人矣。

四年冬,《大禮集議》成,進詹事兼翰林學士。後議世廟神道、廟樂、武舞及太后謁廟,帝率倚璁言而決。璁緣飾經文,委曲當帝意他一切科學的知識體系只有象幾何學那樣,運用演繹法,從,帝益器之。璁急圖柄用,為大學士費宏所抑,遂與萼連章攻宏。帝亦知其情,留宏不即放。五年七月,璁以省墓請。既辭朝,帝復用為兵部右侍郎,兼官如故。給事中杜桐、楊言、趙廷瑞交章力詆,并劾吏部尚書廖紀引用邪人。帝怒,切責之。兩京給事御史解一貫、張錄、方紀達、戴繼先等復交章論不已,皆不聽。尋進璁左侍郎,復與萼攻費宏。明年二月興王邦奇獄,構陷楊廷和等,宏及石珤同日罷。

吏部郎中彭澤以浮躁被斥,璁言:「昔議禮時,澤勸臣進《大禮或問》,致招眾忌。今諸臣去之,將以次去臣等。」澤乃得留。居三日,復言:「臣與舉朝抗四五年,舉朝攻臣至百十疏。今修《大禮全書》,元惡寒心,群奸側目。故要略方進,讒謗繁興。使《全書》告成,將誣陷益甚。」因引疾求退以要帝,帝優詔慰留。吏部闕尚書,推前尚書喬宇、楊旦;禮部尚書亦缺,推侍郎劉龍、溫仁和。仁和以俸深爭。璁言宇、旦乃楊廷和黨,而仁和亦不宜自薦。帝命大臣休致者,非奉詔不得推舉,宇等遂廢。

璁積怒廷臣,日謀報復。會山西巡按馬金錄治反賊李福達獄,詞連武定侯郭勛,法司讞如金錄擬。璁讒於帝,謂廷臣以議禮故陷勛。帝果疑諸臣朋比養生論三國魏嵇康著。討論養生及形神關系問題。否定,乃命璁署都察院,桂萼署刑部,方獻夫署大理,覆讞,盡反其獄,傾諸異己者。大臣顏頤壽、聶賢以下咸被搒掠,金錄等坐罪遠竄。帝益以為能,獎勞之便殿,賚二品服,三代封誥。京察及言官互糾,已黜御史十三人,璁掌憲,復請考察斥十二人。又奏行憲綱七條,鉗束巡按御史。其年冬,遂拜禮部尚書兼文淵閣大學士入參機務,去釋褐六年耳。

楊一清為首輔,翟鑾亦在閣,帝侍之不如璁。嘗諭璁:「朕有密諭毋泄,朕與卿帖悉親書。」璁因引仁宗賜楊士奇等銀章事,帝賜璁二章,文曰「忠良貞一」,曰「繩愆弼違」,因并及一清等。璁初拜學士,諸翰林恥之,不與並列。璁深恨。及侍讀汪佃講《洪範》不稱旨,帝令補外。璁乃請自講讀以下量才外補,改官及罷黜者二十二人,諸庶吉士皆除部屬及知縣,由是翰苑為空。七年正月,帝視朝,見璁、萼班兵部尚書李承勛下,意嗛之。一清因請加散官,乃手敕加二人太子太保。璁辭以未建青宮,官不當設,乃更加少保兼太子太保。《明倫大典》成,復進少傅兼太子太傅、吏部尚書、謹身殿大學士。一清再相,頗由璁、萼力,傾心下二人。而璁終以壓於一清,不獲盡如意,遂相齟齬。指揮聶能遷劾璁,璁欲置之死。一清擬旨稍輕,璁益恨,斥一清為奸人鄙夫。一清再疏引退,且刺璁隱情。帝手敕慰留,因極言璁自伐其能,恃寵不讓,良可歎息。璁見帝忽暴其短,頗愧沮。

八年秋,給事中孫應奎劾一清、萼並及璁,其同官王準復劾璁私參將陳璠,宜斥。璁乞休者再,詞多陰詆一清漢學又稱「樸學」。指漢儒考據訓詁之學。漢人治學,好,帝乃褒諭璁。而給事中陸粲復劾其擅作威福,報復恩怨。帝大感悟,立罷璁。頃之,其黨霍韜力攻一清,微為璁白。璁行抵天津,帝命行人齎手敕召還。一清遂罷去,璁為首輔。

帝自排廷議定「大禮」,遂以制作禮樂自任。而夏言始用事,乃議皇后親蠶,議勾龍、棄配社稷,議分祭天地,議罷太宗配祀,議朝日、夕月別建東、西二郊,議祀高禖,議文廟設主更從祀諸儒,議祧德祖正太祖南向,議祈穀,議大禘,議帝社帝稷,奏必下璁議。顧帝取獨斷,璁言亦不盡入。其諫罷太宗配天,三四往復,卒弗能止也。

十年二月,璁以名嫌御諱請更。乃賜名孚敬,字茂恭,御書四大字賜焉。夏言恃帝眷,數以事訐孚敬。孕敬銜之,未有以發。納彭澤言構陷行人司正薛侃,因侃以害言。廷鞫事露,旨斥其忮罔。御史譚纘、端廷赦、唐愈賢交章劾之。帝諭法司令致仕,孚敬乃大慚去。未幾,遣行人齎敕召之。明年三月還朝,言已擢禮部尚書,益用事。李時、翟鑾在閣,方獻夫繼入,孚敬亦不能專恣如曩時矣。八月,彗星見東井,帝心疑大臣擅政,孚敬因求罷。都給事中魏良弼詆孚敬奸,孚敬言:「良弼以濫舉京營官奪俸,由臣擬旨,挾私報復。」給事中秦鰲劾孚敬強辨飾奸,言官論列輒文致其罪,擬旨不密,引以自歸,明示中外,若天子權在其掌握。帝是鰲言,令孚敬自陳狀,許之致仕。李時請給廩隸、敕書,不許。再請,乃得馳傳歸。十二年正月,帝復思之,遣鴻臚齎敕召。四月還朝。六月,彗星復見畢昴間,乞避位,不許。明年進少師兼太子太師、華蓋殿大學士。

初,潞州陳卿亂,孚敬主用兵,賊竟滅。大同再亂,亦主用兵,薦劉源清為總督,師久無功。其後亂定,代王請大臣安輯。夏言遂力詆用兵之謬,請如王言,語多侵孚敬。孚敬怒,持王疏不行。帝諭令與言交好,而遣黃綰之大同,相機行事。孚敬以議不用,稱疾乞休,疏三上。已而子死,請益力。帝報曰:「卿無疾,疑朕耳。」孚敬復上奏,不引咎,且歷詆同議禮之萼、獻夫、韜、綰等。帝詰責之,乃復起視事。帝於文華殿後建九五齋、恭默室為齋居所,命輔臣賦詩。孚敬及時各為四首以上。已,數召見便殿,從容議政。

十四年春得疾,帝遣中官賜尊牢,而與時言,頗及其執拗,且不惜人才以叢怨狀。又遣中官賜藥餌,手敕言:「古有剪鬚療大臣疾者,朕今以己所服者賜卿。」孚敬幸得溫諭,遂屢疏乞骸骨。命行人御醫護歸,有司給廩隸如制。明年五月,帝復遣錦衣官齎手敕視疾,趣其還。行至金華,疾大作,乃歸。十八年二月卒。帝在承天,聞之傷悼不已。

孚敬剛明果敢,不避嫌怨。既遇主,亦時進讜言。帝欲坐張延齡反,族其家。孚敬諍曰:「延齡,守財虜耳,何能反?」數詰問,對如初。及秋盡當論,孚敬上疏謂:「昭聖皇太后春秋高,卒聞延齡死,萬一不食,有他故,何以慰敬皇帝在天之靈?」帝恚,責孚敬:「自古強臣令主非一,若今愛死囚令主矣。當悔不從廷和事敬皇帝耶?」帝故為重語心妻止孚敬,而孚敬意不已。以故終昭聖皇太后世,延齡得長繫。他若清勛戚莊田,罷天下鎮守內臣,先後殆盡,皆其力也。持身特廉,痛惡贓吏,一時苞苴路絕。而性狠愎,報復相尋,不護善類。欲力破人臣私黨,而己先為黨魁。「大禮」大獄,叢詬沒世。顧帝始終眷禮,廷臣卒莫與二,嘗稱少師羅山而不名。其卒也,禮官請謚。帝取危身奉上之義,特謚文忠,贈太師。

時有胡鐸者,字時振,餘姚人。弘治末進士。正德中,官福建提學副使。嘉靖初,遷湖廣參政要矛盾。它和次要矛盾在一定條件下可以相互轉化。如中國,累官南京太僕卿。鐸與璁同舉於鄉。「大禮」議起,鐸意亦主考獻王,與璁合。璁要之同署,鐸曰:「主上天性固不可違,天下人情亦不可拂。考獻王不已則宗,宗不已則入廟,入廟則當有祧。以籓封虛號之帝,而奪君臨治世之宗,義固不可也。入廟則有位,將位於武宗上乎,武宗下乎?生為之臣,死不得躋於君。然魯嘗躋僖公矣,恐異日不乏夏父之徒也。」璁議遂上。旋被召。鐸方服闋赴京,璁又要同疏,鐸復書謝之,且與辨繼統之義。「大禮」既定,鐸又貽書勸召還議禮諸人,養和平之福,璁不能從。鐸與王守仁同鄉,不宗其學;與璁同以考獻王為是,不與同進。然其辨繼統,謂國統絕而立君寓立賢之意,蓋大謬云。

桂萼,字子實,安仁人。正德六年進士。除丹徒知縣。性剛使氣,屢忤上官,調青田不赴。用薦起知武康,復忤上官下吏。

嘉靖初,由成安知縣遷南京刑部主事。世宗欲尊崇所生,廷臣力持,已稱興獻王為帝,妃為興國太后,頒詔天下二歲矣,萼與張璁同官,乃以二年十一月上疏曰:「臣聞帝王事父孝,故事天明;事母孝,故事地察。未聞廢父子之倫,而能事天地主百神者也。今禮官失考典章,遏絕陛下純孝之心,納陛下於與為人後之非,而滅武宗之統,奪獻帝之宗,且使興國太后壓於慈壽太后,禮莫之盡,三綱頓廢,非常之變也。乃自張璁、霍韜獻議,論者指為干進,逆箝人口,致達禮者不敢駁議。切念陛下侍興國太后,慨興獻帝弗祀,已三年矣,拊心出涕,不知其幾。願速發明詔,稱孝宗曰『皇伯考』,興獻帝『皇考』,別立廟大內,正興國太后之禮,定稱聖母,庶協事天事地之道。至朝臣所執不過宋《濮議》耳。按宋范純仁告英宗曰『陛下昨受仁宗詔,親許為之子,至於封爵,悉用皇子故事,與入繼之主不同』,則宋臣之論,亦自有別。今陛下奉祖訓入繼大統,未嘗受孝宗詔為之子也,則陛下非為人後,而為入繼之主也明甚。考興獻帝,母興國太后,又何疑?臣聞非天子不議禮;天下有道,禮樂自天子出。臣久欲以請,乃者復得席書、方獻夫二疏。伏望奮然裁斷,將臣與二臣疏並付禮官,令臣等面質。」帝大喜,明年正月手批議行。

三月,萼復上疏曰:「自古帝王相傳,統為重,嗣為輕。故高皇帝法前王,著兄終弟及之訓。陛下承祖宗大統,正遵高皇帝制。執政乃無故任己私,背祖訓,其為不道,尚可言哉。臣聞道路人言,執政窺伺陛下至情不已,則加一皇字而已。夫陛下之孝其親,不在於皇不皇,惟在於考不考。使考獻帝之心可奪,雖加千百字徽稱,何益於孝?陛下遂終其身為無父人矣。逆倫悖義如此,猶可使與斯議哉!」與璁疏並上。帝益大喜,召赴京。

初,議禮諸臣無力詆執政者,至萼遂斥為不道,且欲不使議。其言恣肆無忌,朝士尤疾之。召命下,眾益駭愕,群起排擊,帝不為動。萼復偕璁論列不已,遂召為翰林學士,卒用其言。萼自是受知特深。

四年春,給事中柯維熊言:「陛下親君子而君子不容,如林俊、孫交、彭澤之去是也。遠小人而小人尚在,如張璁、桂萼之用是也。且今伏闕諸臣多死徙,而御史王懋、郭楠又謫譴,竊以為罰過重矣。」萼、璁遂求去,優詔慰留。尋進詹事兼翰林學士。議世廟神道及太后謁廟禮,復排廷議,希合帝指。帝益以為賢,兩人氣益盛。而閣臣抑之,不令與諸翰林等。兩人乃連章攻費宏並石珤,齮之去。給事中陳洸犯重辟,萼與尚書趙鑑攘臂爭,為南京給事中所劾,不問。嘗陳時政,請預蠲六年田租,更登極初宿弊,寬登聞鼓禁約,復塞上開中制,懲奸徒阻絕養濟院,聽窮民耕城垣陾地,停外吏赴部考滿,申聖敬,廣聖孝,凡數事。多議行。

六年三月,進禮部右侍郎,兼官如故。時方京察,南京言官拾遺及萼。萼上言:「故輔楊廷和廣植私黨,蔽聖聰者六年,今次第斥逐,然遺奸在言路。昔憲宗初年,命科道拾遺後,互相糾劾,言路遂清,請舉行如制。」章下吏部,侍郎孟春等言:「憲宗無此詔。萼被論報復,無以厭眾心。」萼言:「詔出憲宗文集。春欲媚言官,宜並按問。」章下部再議,春等言成化中科道有超擢巡撫不稱者,憲宗命互劾,去者七人,非考察拾遺比。帝終然萼言,趣令速舉。給事御史爭之,並奪俸。春等乃以御史儲良才等四人名上。帝獨黜良才,而特旨斥給事中鄭自璧、孟奇。且令部院再核,復黜給事中餘經等四人、南京給事中顧溱等數人,乃已。

其年九月改吏部左侍郎。是月拜禮部尚書,兼翰林學士。故事,尚書無兼學士者,自萼始。甫踰月,遷吏部尚書,賜銀章二,曰「忠誠靜慎」,曰「繩愆匡違」,令密封言事與輔臣埒。七年正月,手敕加太子太保。《明倫大典》成,加少保兼太子太傅。

萼既得志,日以報怨為事。陳九疇、李福達、陳洸之獄,先後株連彭澤、馬錄、葉應驄等甚眾,或被陷至謫戍。廷臣莫不畏其兇威。獨疏薦建言獄罪鄧繼曾、季本等,因事貶謫黃國用、劉秉鑑等,諸人得量移。世亦稍以此賢萼。然王守仁之起也,萼實薦之。已,銜其不附己,力齮齕。及守仁卒,極言醜詆,奪其世封,諸恤典皆不予。八年二月命以本官兼武英殿大學士入參機務。初,萼、璁赴召,廷臣欲仿先朝馬順故事,於左順門捶殺之,走武定侯郭勛家以免。勛遂與深相結,亦蒙帝眷典禁兵。久之,勛奸狀大露,璁、霍韜力庇勛。萼知帝已惡之,獨疏其兇暴貪狡數事,勛遂獲罪。楊一清為首輔持重,萼、璁好紛更,且惡其壓己,遂不相能。

給事中孫應奎請鑒別三臣賢否,詆萼最力。帝已疑萼,令滌宿愆,全君臣終始之義。萼乃大懼,疏辨,且稱疾乞休。帝報曰:「卿行事須勉徇公議,庶不負前日忠。」萼益懼。給事中王準因劾萼舉私人李夢鶴為御醫。詔下吏部,言夢鶴由考選,無私。帝終以為疑,命太醫院更考。言官知帝意已移,給事中陸粲極論其罪,並言夢鶴與萼家人吳從周、序班桂林居間行賄事。奏入,帝大悟,立奪萼官,以尚書致仕。璁亦罷政。帝復列二人罪狀詔廷臣,略言:「其自用自恣,負君負國,所為事端昭然眾見,而萼尤甚。法當置刑典,特寬貸之。」遂下夢鶴等法司,皆首服。無何,霍韜兩疏訟萼,言一清與法司構成萼贓罪。一清遂去位,刑部尚書周倫調南京,郎中、員外皆奪職,命法司會錦衣鎮撫官再讞。乃言夢鶴等假託行私,與萼無與。詔削夢鶴、林籍,從周論罪,萼復散官。是時璁已召還。史館儒士蔡圻知帝必復萼,疏頌萼功,請召之。帝乃賜敕,令撫按官趣上道。萼未至,國子生錢潮等復請趣萼。帝怒曰:「大臣進退,麼麼敢與聞耶?」並圻下吏。明年四月還朝,盡復所奪官,仍參機務。

萼初銳意功名,勇任事,不恤物議,驟被摧抑,氣為之懾,不敢復放恣。居位數月,屢引疾,帝輒優旨慰留。十年正月得請歸,卒於家。贈太傅,謚文襄。

萼所論奏,《帝王心學論》、《皇極論》、《易·復卦》、《禮·月令》及進《禹貢圖》、《輿地圖說》,皆有裨君德時政。性猜狠,好排異己,以故不為物論所容。始與璁相得歡甚,比同居政府,遂至相失。

方獻夫,字叔賢,南海人。生而孤。弱冠舉弘治十八年進士,改庶吉士。乞歸養母,遂丁母憂。正德中,授禮部主事,調吏部,進員外郎。與主事王守仁論學,悅之,遂請為弟子。尋謝病歸,讀書西樵山中者十年。

嘉靖改元,夏還朝,道聞「大禮」議未定,草疏曰:

先王制禮,本緣人情。君子論事,當究名實。竊見近日禮官所議,有未合乎人情,未當乎名實者,一則守《禮經》之言,一則循宋儒之說也。臣獨以為不然。按《禮經·喪服》傳曰「何如而可以為人後?支子可也」。又曰「為人後者孰後?後大宗也」。「大宗者,尊之統也」。「不可以絕,故族人以支子後大宗也。適子不得後大宗」。為是禮者,蓋謂有支子而後可以為人後,未有絕人之後以為人後者也。今興獻帝止生陛下一人,別無支庶,乃使絕其後而後孝宗,豈人情哉!且為人後者,父嘗立之為子,子嘗事之為父,故卒而服其服。今孝宗嘗有武宗矣,未嘗以陛下為子。陛下於孝宗未嘗服三年之服,是實未嘗後孝宗也,而強稱之為考,豈名實哉!為是議者,未見其合於《禮經》之言也。

又按程頤《濮議》謂「英宗既以仁宗為父,不當以濮王為親」。此非宋儒之說不善,實今日之事不同。蓋仁宗嘗育英宗於宮中,是實為父子。孝宗未嘗育陛下於宮中,其不同者一。孝宗有武宗為子矣,仁宗未嘗有子也,其不同者二。濮王別有子可以不絕,興獻帝無別子也,其不同者三。豈得以濮王之事比今日之事哉?為是議者,未見其善述宋儒之說也。

若謂孝宗不可無後,故必欲陛下為子,此尤不達於大道者也。推孝宗之心,所以必欲有後者,在不絕祖宗之祀,不失天下社稷之重而已,豈必拘拘父子之稱,而後為有後哉。孝宗有武宗,武宗有陛下,是不絕祖宗之祀,不失天下社稷之重矣,是實為有後也。且武宗君天下十有六年。不忍孝宗之無後,獨忍武宗之無後乎?此尤不通之說也。夫興獻帝當父也,而不得父。孝宗不當父也,而強稱為父。武宗當繼也,而不得繼。是一舉而三失焉,臣未見其可也。

且天下未嘗有無父之國也。瞽瞍殺人,舜竊負而逃。今使陛下舍其父而有天下,陛下何以為心哉!臣知陛下純孝之心,寧不有天下,決不忍不父其父也。說者又謂興獻帝不當稱帝,此尤不達於大道者也。孟子曰「孝子之至,莫大乎尊親」。周公追王太王王季,子思以為達孝。豈有子為天子,父不得稱帝者乎?今日之事,臣嘗為之說曰:陛下之繼二宗,當繼統而不繼嗣。興獻之異群廟,在稱帝而不稱宗。夫帝王之體,與士庶不同。繼統者,天下之公,三王之道也。繼嗣者,一人之私,後世之事也。興獻之得稱帝者,以陛下為天子也。不得稱宗者,以實未嘗在位也。伏乞宣示朝臣,復稱孝宗曰『皇伯』,興獻帝曰『皇考』,別立廟祀之。夫然後合於人情,當乎名實,非唯得先王制禮之意,抑亦遂陛下純孝之心矣。

疏具,見廷臣方抵排異議,懼不敢上,為桂萼所見,與席書疏並表上之。帝大喜,立下廷議。廷臣遂目獻夫為奸邪,至不與往還。獻夫乃杜門乞假,既不得請,則進《大禮》上下二論,其說益詳。時已召張璁、桂萼於南京,至即用為翰林學士,而用獻夫為侍講學士。攻者四起,獻夫亦力辭。帝卒用諸人議定「大禮」,由是荷帝眷與璁、萼埒。四年冬進少詹事。獻夫終不自安,謝病歸。

六年召修《明倫大典》。獻夫與霍韜同里,以議禮相親善,又同赴召,乃合疏言:「自古力主為後之議者,宋莫甚於司馬光,漢莫甚於王莽。主《濮議》者,光為首,呂誨、范純仁、呂大防附之,而光之說惑人最甚。主哀帝議者,莽為首,師丹、甄邯、劉歆附之,而莽之說流毒最深。宋儒祖述王莽之說以惑萬世,誤後學。臣等謹按《漢書》、《魏志》、《宋史》,略采王莽、師丹、甄邯之奏,與其事始末,及魏明帝之詔,濮園之議,論正以附其後。乞付纂修官,參互考訂,俾天下臣子知為後之議實起於莽,宋儒之論實出於莽,下洗群疑,上彰聖孝。」詔語下其書於史館。還朝未幾,命署大理寺事,與璁、萼覆讞李福達獄。萼等議馬錄重辟,獻夫力爭得減死。其年九月拜禮部右侍郎,仍兼學士,直經筵日講。尋代萼為吏部左侍郎,復代為禮部尚書。《明倫大典》成,加太子太保。

獻夫視璁、萼性寬平,遇事亦間有執持,不盡與附會。萼反陳洸獄,請盡逮問官葉應驄等,以獻夫言多免逮。思恩、田州比歲亂,獻夫請專任王守仁,而罷鎮守中官鄭潤、總兵官朱騏,帝乃召潤、騏還。思、田既平,守仁議築城建邑,萼痛詆之。獻夫歷陳其功狀,築城得毋止。璁、萼與楊一清構,獻夫因災異進和衷之說,且請收召謫戍削籍餘寬、馬明衡輩,而倍取進士之數。帝優詔答之,寬等卒不用。獻夫以尼僧、道姑傷風化,請勒令改嫁,帝從之。又因霍韜言,盡汰僧道無牒、毀寺觀私創者。帝欲殺陳后喪,獻夫引禮固爭。尋復代萼為吏部尚書。萼、璁罷政,詔吏部核兩人私黨。獻夫言:「陸粲等所劾百十人,誣者不少。昔攻璁、萼者,以為黨而去之。今附璁、萼者,又以為黨而去之。縉紳之禍何時已。」乃奏留黃綰等二十三人,而黜儲良才等十二人。良才者,初為御史,以考察黜。上疏詆楊廷和,指吏部侍郎孟春等為奸黨,萼因請復其職。至是斥去,時論快之。安昌伯錢維圻卒,庶兄維垣請嗣爵。獻夫言外戚之封不當世及,歷引漢、唐、宋事為證。帝善其言,下廷議,外戚遂永絕世封。

璁、萼既召還,羽林指揮劉永昌劾都督桂勇,語侵萼及兵部尚書李承勛。又劾御史廖自顯,自顯坐逮。已,又訐兵部郎中盧襄等。獻夫請按治永昌,毋令奸人以蜚語中善類。帝不從。獻夫遂求退,帝亦不允。給事中孫應奎劾獻夫私其親故大理少卿洗光、太常卿彭澤。帝不聽。都給事中夏言亦劾獻夫壞選法,徙張璁所惡浙江參政黃卿於陜西,而用璁所愛黨以平代,邪回之彭澤踰等躐遷太常,及他所私暱,皆有迹,疑獻夫交通賄賂。疏入,帝令卿等還故官。獻夫及璁疏辨,因引退。帝重違二人意,復令卿等如前擬。

頃之,給事中薛甲言:「劉永昌以武夫劾塚宰,張瀾以軍餘劾勛臣,下凌上替,不知所止,願存廉遠堂高之義,俾小人不得肆攻訐。」章下吏部。獻夫等請從甲言,敕都察院嚴禁吏民,毋得言壽張亂政,並飭兩京給事御史及天下撫按官論事,先大體毋責小疵。當是時,帝方欲廣耳目,周知百僚情偽,得獻夫議不懌,報罷。於是給事中饒秀劾甲阿附:「自劉永昌後,言官未聞議大臣,獨夏言、孫應奎、趙漢議及璁、獻夫耳。漢已蒙詰譴,言、應奎所奏皆用人行政之失,甲乃指為毛舉細故,而頌大臣不已。貪縱如郭勛,亦不欲人言。必使大臣橫行,群臣緘口。萬一有逆人廁其間,奈何!」奏入,帝心善其言。下吏部再議。甲具疏自明,帝惡其不俟部奏,命削二官出之外。部謂甲已處分,不復更議。帝責令置對,停獻夫俸一月,郎官倍之。獻夫不自得,兩疏引疾。帝即報允,然猶虛位以俟。

十年秋有詔召還。獻夫疏辭,舉梁材、汪鋐、王廷相自代。帝手詔褒答,遣行人蔡靉趣之。靉及門,獻夫潛入西樵,以疾辭。既而使命再至,云將別用,乃就道。明年五月至京,命以故官兼武英殿大學士入閣輔政。初,賜獻夫銀章曰「忠誠直諒」,令有事密封奏聞。獻夫歸,上之朝,至是復賜如故。吏部尚書王瓊卒,命獻夫掌之。獻夫家居,引體自尊,監司謁見,輒稱疾不報。家人姻黨橫於郡中,鄉人屢訐告,僉事龔大稔聽之。獻夫還朝,囑大稔。會大稔坐事落職,疑獻夫為之,遂上疏列其不法數事,詞連霍韜。獻夫疏辨,帝方眷獻夫,大稔遂被逮削籍。十月彗見東井。御史馮恩詆獻夫兇奸肆巧辨,播弄威福,將不利於國家,故獻夫掌吏部而彗見。帝怒,下之獄。獻夫亦引疾乞休,優詔不允。

獻夫飾恬退名,連被劾,中恧。雖執大政,氣厭厭不振。獨帝欲殺張延齡,常力爭。而其時桂萼已前卒。張璁最寵,罷相者屢矣。霍韜、黃宗明言事一不當,輒下之吏。獻夫見帝恩威不測,居職二歲,三疏引疾。帝優詔許之,令乘傳,予道里費。家居十年卒。先已加柱國、少保,乃贈太保,謚文襄。

獻夫緣議禮驟貴。與璁、萼共事,持論頗平恕,故人不甚惡之。

夏言,字公謹,貴溪人。父鼎,臨清知州。言舉正德十二年進士,授行人,擢兵科給事中。性警敏,善屬文。及居言路,謇諤自負。世宗嗣位,疏言:「正德以來,壅蔽已極。今陛下維新庶政,請日視朝後,御文華殿閱章疏,召閣臣面決。或事關大利害,則下廷臣集議。不宜謀及褻近,徑發中旨。聖意所予奪,亦必下內閣議而後行,絕壅蔽矯詐之弊。」帝嘉納之。奉詔偕御史鄭本公、主事汪文盛核親軍及京衛冗員,汰三千二百人,復條九事以上。輦下為肅清。

嘉靖初,偕御史樊繼祖等出按莊田,悉奪還民產。劾中官趙霦、建昌侯張延齡,疏凡七上。請改後宮負郭莊田為親蠶廠、公桑園,一切禁戚里求請及河南、山東奸人獻民田王府者。救被逮永平知府郭九皋。莊奉夫人弟邢福海、肅奉夫人弟顧福,傳旨授錦衣世千戶,言力爭不可。諸疏率諤諤,為人傳誦。屢遷兵科都給事中。勘青羊山平賊功罪,論奉悉當。副使牛鸞獲賊中交通名籍,言請毀之以安眾心。孝宗朝,令吏、兵二部每季具兩京大臣及在外文武方面官履歷進御,正德後漸廢,以言請復之。

七年,調吏科。當是時,帝銳意禮文事。以天地合祀非禮,欲分建二郊,并日月而四。大學士張孚敬不敢決,帝卜之太祖亦不吉,議且寢。會言上疏請帝親耕南郊,后親蠶北郊,為天下倡。帝以南北郊之說,與分建二郊合,令孚敬諭旨,言乃請分祀天地。廷臣持不可,孚敬亦難之,詹事霍韜詆尤力。帝大怒,下韜獄。降璽書獎言,賜四品服俸,卒從其請。又贊成二郊配饗議,語詳《禮志》。言自是大蒙帝眷。郊壇工興,即命言監之。延綏饑,言薦僉都御史李如圭為巡撫。吏部推代如圭者,帝不用,再推及言。御史熊爵謂言出如圭為己地,至比之張糸採。帝切責爵,令言毋辨。而言不平,訐爵且辭新命,帝乃止。

孚敬頤指百僚,無敢與抗者。言自以受帝知,獨不為下。孚敬乃大害言寵,言亦怨孚敬驟用彭澤為太常卿不右己,兩人遂有隙。言抗疏劾孚敬及吏部尚書方獻夫。孚敬、獻夫皆疏辨求去。帝顧諸人厚,為兩解之。言既顯,與孚敬、獻夫、韜為難,益以強直厚自結。帝欲輯郊禮為成書,擢言侍讀學士,充纂修官,直經筵日講,仍兼吏科都給事中。言又贊帝更定文廟祀典及大禘禮,帝益喜。十年三月遂擢少詹事,兼翰林學士,掌院事,直講如故。言眉目疏朗,美鬚髯,音吐弘暢,不操鄉音。每進講,帝必目屬,欲大用之。孚敬忌彌甚,遂與彭澤構薛侃獄,下言法司。已,帝覺孚敬曲,乃罷孚敬而釋言。八月,四郊工成,進言禮部左侍郎,仍掌院事。踰月,代李時為本部尚書。去諫官未浹歲拜六卿,前此未有也。

時士大夫猶惡孚敬,恃言抗之。言既以開敏結帝知,又折節下士。御史喻希禮、石金請宥「大禮」大獄得罪諸臣。帝大怒,令言劾。言謂希禮、金無他腸,請帝寬恕。帝責言對狀,逮二人詔獄,遠竄之,言引罪乃已。以是大得公卿間聲。帝制作禮樂,多言為尚書時所議,閣臣李時、翟鑾取充位。帝每作詩,輒賜言,悉酬和勒石以進,帝益喜。奏對應制,倚待立辦。數召見,諮政事,善窺帝旨,有所傅會。賜銀章一,俾密封言事,文曰「學博才優」。先後賜繡蟒飛魚麒麟服、玉帶、兼金、上尊、珍饌、時物無虛月。孚敬、獻夫復相繼入輔。知帝眷言厚,亦不敢與較。已而皆謝事。議禮諸人獨霍韜在,仇言不置。十五年以順天府尹劉淑相事,韜、言相攻訐。韜卒不勝,事詳《韜傳》中。言由是氣遂驕。郎中張元孝、李遂與小忤,即奏謫之。皇子生,帝賜言甚渥。初加太子太保,進少傅兼太子太傅。閏十二月遂兼武英殿大學士入參機務。扈蹕謁陵,還至沙河,言庖中火,延郭勛、李時帳,帝付言疏六亦焚。言當獨引罪,與勛等合謝,被譙責焉。時李時為首輔,政多自言出。顧鼎臣入,恃先達且年長,頗欲有所可否。言意不悅,鼎臣遂不敢與爭。其冬,時卒,言為首輔。十八年,以祗薦皇天上帝冊表,加少師、特進光祿大夫、上柱國。明世人臣無加上柱國者,言所自擬也。

武定侯郭勛得幸,害言寵。而禮部尚書嚴嵩亦心妒言。言與嵩扈蹕承天,帝謁顯陵畢,嵩再請表賀,言乞俟還京。帝報罷,意大不懌。嵩知帝指,固以請,帝乃曰:「禮樂自天子出可也。」令表賀,帝自是不悅言。帝幸大峪山,言進居守敕稍遲,帝責讓。言懼請罪。帝大怒曰:「言自卑官,因孚敬議郊禮進,乃怠慢不恭,進密疏不用賜章,其悉還累所降手敕。」言益懼,疏謝。請免追銀章、手敕,為子孫百世榮,詞甚哀。帝怒不解,疑言毀損,令禮部追取。削少師勳階,以少保尚書大學士致仕。言乃以手敕四百餘,并銀章上之。居數日,怒解,命止行。復以少傅、太子太傅入直,言疏謝。帝悅,諭令勵初忠,秉公持正,免眾怨。言心知所云眾怨者,郭勛輩也,再疏謝。謂自處不敢後他人,一志孤立,為眾所忌。帝復不悅,詰責之。惶恐謝,乃已。未幾,雷震奉天殿。召言及鼎臣不時至。帝復詰讓,令禮部劾之。言等請罪,帝復讓言傲慢,並責鼎臣。已,乃還所追銀章、御書。陜西奏捷,復少師、太子太師,進吏部尚書,華蓋殿。江淮賊平,璽書獎勵,賜金幣,兼支大學士俸。

鼎臣已歿,翟鑾再入,恂恂若屬吏然,不敢少齟齬。而霍韜入掌詹事府數修怨。以郭勛與言有隙,結令助己,三人日相構。既而韜死,言、勛交惡自若。九廟災,言方以疾在告,乞罷,不允。昭聖太后崩,詔問太子服制,言報疏有訛字。帝切責言,言謝罪且乞還家治疾。帝益怒,令以少保、尚書、大學士致仕。言始聞帝怒己,上御邊十四策,冀以解。帝曰:「言既蘊忠謀,何堅自愛,負朕眷倚,姑不問。」初,言撰青詞及他文,最當帝意。言罷,獨翟鑾在,非帝所急也。及將出都,詣西苑齋宮叩首謝。帝聞而憐之,特賜酒饌,俾還私第治疾,俟後命。會郭勛以言官重劾,亦引疾在告。京山侯崔元新有寵,直內苑,忌勛。帝從容問元:「言、勛皆朕股肱,相妒何也?」元不對。帝問言歸何時,曰:「俟聖誕後,始敢請。」又問勛何疾,曰:「勛無疾,言歸即出耳。」帝頷之。言官知帝眷言惡勛,因共劾勛。勛辨語悖謾,帝怒,削勛同事王廷相籍。給事中高時者,言所厚也,盡發勛貪縱不法十數事。遂下勛獄,復言少傅、太子太師、禮部尚書、武英殿大學士,疾愈入直。言雖在告,閣事多取裁。治勛獄,悉其指授。二十一年春,一品九年滿,遣中使賜銀幣、寶鈔、羊酒、內饌。盡復其官階,璽書獎美,賜宴禮部。尚書、侍郎、都御史陪侍。當是時,帝雖優禮言,然恩眷不及初矣。

慈慶、慈寧兩宮宴駕,勛嘗請改其一居太子。言不可,合帝意。至是帝猝問太子當何居,言忘前語,念興作費煩,對如勛指。帝不悅。又疑言官劾勛出言意。及建大享殿,命中官高忠監視,言不進敕稿。入直西苑諸臣,帝皆令乘馬,又賜香葉束髮巾,用皮帛為履。言謂非人臣法服,不受,又獨乘腰輿。帝積數憾欲去言,而嚴嵩因得間之。嵩與言同鄉,稱先達,事言甚謹。言入閣援嵩自代,以門客畜之,嵩心恨甚。言既失帝意,嵩日以柔佞寵。言懼斥,呼嵩與謀。嵩則已潛造陶仲文第,謀齮言代其位。言知甚慍,諷言官屢劾嵩。帝方憐嵩不聽也,兩人遂大卻。六月,嵩燕見,頓首雨泣,愬言見凌狀。帝使悉陳言罪,嵩因振暴其短。帝大怒,手敕禮部,歷數言罪,且曰:「郭勛已下獄,猶千羅百織。言官為朝廷耳目,專聽言主使。朕不早朝,言亦不入閣。軍國重事,取裁私家。王言要密,視等戲玩。言官不一言,徒欺謗君上,致神鬼怒,雨甚傷禾。」言大懼,請罪。居十餘日,獻帝諱辰,猶召入拜,候直西苑。言因謝恩乞骸骨,語極哀。疏留八日,會七月朔日食既,下手詔曰:「日食過分,正坐下慢上之咎,其落言職閒住。」帝又自引三失,布告天下。御史喬佑、給事中沈良才等皆具疏論言,且請罪。帝大怒,貶黜十三人。高時以劾勛故,獨謫遠邊。於是嚴嵩遂代言入閣。

言久貴用事,家富厚,服用豪侈,多通問遺。久之不召,監司府縣吏亦稍慢易之,悒悒不樂。遇元旦、聖壽必上表賀,稱「草土臣」。帝亦漸憐之,復尚書、大學士。至二十四年,帝微覺嵩貪恣,復思言,遣官齎敕召還,盡復少師諸官階,亦加嵩少師,若與言並者。言至,直陵嵩出其上。凡所批答,略不顧嵩,嵩噤不敢吐一語。所引用私人,言斥逐之,亦不敢救,銜次骨。海內士大夫方怨嵩貪忮,謂言能壓嵩制其命,深以為快。而言以廢棄久,務張權。文選郎高簡之戍,唐龍、許成名、崔桐、王用賓、黃佐之罷,王杲、王、孫繼魯之獄,皆言主之。貴州巡撫王學益、山東巡撫何鰲為言官論劾,輒擬旨逮訊。龍故與嵩善,事牽世蕃,其他所譴逐不盡當,朝士仄目。最後御史陳其學以鹽法事劾崔元及錦衣都督陸炳,言擬旨令陳狀,皆造言請死,炳長跪乃得解。二人與嵩比而構言,言未之悟也。帝數使小內豎詣言所,言負氣岸,奴視之;嵩必延坐,親納金錢袖中。以故日譽嵩而短言。言進青詞往往失帝旨,嵩聞益精治其事。

未幾,河套議起。言故慷慨以經濟自許,思建立不世功。因陜西總督曾銑請復河套,贊決之。嵩與元、炳媒孽其間,竟以此敗。江都人蘇綱者,言繼妻父也,雅與銑善。銑方請復河套,綱亟稱於言。言倚銑可辦,密疏薦之,謂群臣無如銑忠者。帝令言擬旨,優獎之者再。銑喜,益銳意出師。帝忽降旨詰責,語甚厲。嵩揣知帝意,遂力言河套不可復,語侵言。言始大懼,謝罪,且言「嵩未嘗異議,今乃盡諉於臣」。帝責言「強君脅眾」,嵩復騰疏攻言,言亦力辨。而帝已入嵩譖,怒不可解。二十七年正月盡奪言官階,以尚書致仕,猶無意殺之也。會有蜚語聞禁中,謂言去時怨謗。嵩復代仇鸞草奏訐言納銑金,交關為奸利,事連蘇綱,遂下銑、綱詔獄。嵩與元、炳謀,坐銑交結近侍律斬,綱戍邊,遣官校逮言。言抵通州,聞銑所坐,大驚墮車曰:「噫!吾死矣。」再疏訟冤,言:「鸞方就逮,上降諭不兩日,鸞何以知上語,又何知嵩疏而附麗若此?蓋嵩與崔元輩詐為之以傾臣。嵩靜言庸違似共工,謙恭下士似王莽,奸巧弄權、父子專政似司馬懿。在內諸臣受其牢籠,知有嵩不知有陛下。在外諸臣受其箝制,亦知有嵩不知有陛下。臣生死係嵩掌握,惟歸命聖慈,曲賜保全。」帝不省。獄成,刑部尚書喻茂堅、左都御史屠僑等當言死,援議貴議能條以上。帝不從,切責茂堅等,奪其俸,猶及言前不戴香冠事。其年十月竟棄言市。妻蘇流廣西,從子主事克承、從孫尚寶丞朝慶,削籍為民。言死時年六十有七。

言豪邁有俊才,縱橫辨博,人莫能屈。既受特眷,揣帝意不欲臣下黨比,遂日與諸議禮貴人抗。帝以為不黨,遇益厚,然卒為嚴嵩所擠。言死,嵩禍及天下,久乃多惜言者。而言所推轂徐階,後卒能去嵩為名相。隆慶初,其家上書白冤狀,詔復其官,賜祭葬,謚文愍。言始無子。妾有身,妻忌而嫁之,生一子。言死,妻逆之歸,貌甚類言。且得官矣,忽病死。言竟無後。

贊曰:璁、萼、獻夫議尊興獻帝,本人子至情,故其說易入。原其初議未嘗不準情禮之中,乃至遭時得君,動引議禮自固,務快恩仇。於是知其建議之心,非有惓惓忠愛之實,欲引其君於當道也。言所奏定典禮,亦多可採。而志驕氣溢,卒為嵩所擠。究觀諸人立身本末與所言是非,固兩不相掩雲。


\end{pinyinscope}