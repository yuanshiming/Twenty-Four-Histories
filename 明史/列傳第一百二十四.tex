\article{列傳第一百二十四}

\begin{pinyinscope}
李植羊可立江東之湯兆京金士衡王元翰孫振基子必顯丁元薦于玉立李樸夏嘉遇

李植,字汝培。父承式,自大同徙居江都,官福建布政使。植舉萬曆五年進士,選庶吉士,授御史。十年冬,張居正卒,馮保猶用事。其黨錦衣指揮同知徐爵居禁中,為閱章奏,擬詔旨如故。居正黨率倚爵以自結於保,爵勢益張。而帝雅銜居正、保,未有以發。御史江東之首暴爵奸,並言兵部尚書梁夢龍與爵交歡,以得吏部,宜斥。帝下爵獄,論死,夢龍罷去。植遂發保十二大罪。帝震怒,罪保。植、東之由是受知於帝。

明年,植巡按畿輔,請寬居正所定百官乘驛之禁,從之。帝用禮部尚書徐學謨言,將卜壽宮於大峪山。植扈行閱視,謂其地未善。欲偕東之疏爭,不果。明年,植還朝。時御史羊可立亦以追論居正受帝知。三人更相結,亦頗引吳中行、趙用賢、沈思孝為重。執政方忌中行、用賢,且心害植三人寵。會爭御史丁此呂事及論學謨卜壽宮之非,與申時行等相拄,卒被斥去。

初,兵部員外郎嵇應科、山西提學副使陸檄、河南參政戴光啟為鄉會試考官,私居正子嗣修、懋修、敬修。居正敗,此呂發其事。又言:「禮部侍郎何雒文代嗣修、懋修撰殿試策,而侍郎高啟愚主南京試,至以『舜亦以命禹』為題,顯為勸進。」大學士申時行、餘有丁、許國皆嗣修等座主也,言考官止據文藝,安知姓名,不宜以此為罪,請敕吏部核官評,以定去留。尚書楊巍議黜雒文,改調應科、檄,留啟愚、光啟,而言此呂不顧經旨,陷啟愚大逆。此呂坐謫。植、東之及同官楊四知、給事中王士性等不平,交章劾巍,語侵時行。東之疏言:「時行以二子皆登科,不樂此呂言科場事。巍雖庇居正,實媚時行。」時行、巍並求去。帝欲慰留時行,召還此呂,以兩解之。有丁、國言不謫此呂,無以安時行、巍心。國反覆詆言者生事,指中行、用賢為黨。中行、用賢疏辨求去,語皆侵國,用賢語尤峻。國避位不出。於是左都御史趙錦,副都御史石星,尚書王遴、潘季馴、楊兆,侍郎沈鯉、陸光祖、舒化、何起鳴、褚鈇,大理卿溫純,及都給事中齊世臣、御史劉懷恕等,極論時行、國、巍不宜去。主事張正鵠、南京郎中汪應蛟、御史李廷彥、蔡時鼎、黃師顏等又力攻請留三臣者之失。中行亦疏言:「律禁上言大臣德政。邇者襲請留居正遺風,輔臣辭位,群起奏留,贊德稱功,聯章累牘。此諂諛之極,甚可恥也。祖宗二百餘年以來,無諫官論事為吏部劾罷者,則又壅蔽之漸,不可長也。」帝竟留三臣,責言者如錦等指。其後,啟愚卒為南京給事中劉一相劾去,時行亦不能救也。

帝追仇居正甚,以大臣陰相庇,獨植、東之、可立能發其奸,欲驟貴之,風示廷臣。一相又劾錦衣都督劉守有匿居正家資。帝乃諭內閣黜守有,超擢居正所抑丘橓、餘懋學、趙世卿及植、東之凡五人。時行等力為守有解,言橓等不宜驟遷。帝重違大臣意,議雖寢,心猶欲用植等。頃之,植劾刑部尚書潘季馴朋黨奸逆,誣上欺君,季馴坐削籍。帝遂手詔吏部擢植太僕少卿,東之光祿少卿,可立尚寶少卿,並添註。廷臣益忌植等。

十三年四月旱,御史蔡系周言:「古者,朝有權臣,獄有冤囚,則旱。植數為人言:『至尊呼我為兒,每觀沒入寶玩則喜我。』其無忌憚如此。陛下欲雪枉,而刑部尚書之枉,先不得雪。今日之旱,實由於植。」又曰:「植迫欲得中行柄國,以善其後;中行迫欲得植秉銓,而騁其私。倘其計得行,勢必盡毒善類,今日旱災猶其小者。」其他語絕狂誕。所稱尚書,謂季馴也。疏上,未報,御史龔懋賢、孫愈賢繼之。東之發憤上疏曰:「思孝、中行、用賢及張岳、鄒元標數臣,忠義天植,之死不移,臣實安為之黨,樂從之遊。今指植與交歡為黨,則植猶未若臣之密,願先罷臣官。」不允。可立亦抗言:「奸黨懷馮、張私惠,造不根之辭,以傾建言諸臣,勢不盡去臣等不止。」乞罷職。章下內閣,時行等請詰可立奸黨主名。帝仍欲兩為之解,寢閣臣奏,而敕都察院:「自今諫官言事,當顧國家大體,毋以私滅公,犯者必罪。」植、東之求去,不許。給事御史齊世臣、吳定等交章劾可立不當代植辨。報曰:「朕方憂旱,諸臣何紛爭?」乃已。七月,御史龔仲慶又劾植、中行、思孝為邪臣,帝惡其排擠,出之外。世臣及御史顧鈐等連章論救,不聽。

是時,竟用學謨言,作壽宮於大峪山。八月,役既興矣,大學士王錫爵,植館師,東之、可立又嘗特薦之於朝,錫爵故以面折張居正,為時所重。三人念時行去,錫爵必為首輔,而壽宮地有石,時行以學謨故主之,可用是罪也,乃合疏上言:「地果吉則不宜有石,有石則宜奏請改圖。乃學謨以私意主其議,時行以親故贊其成。今鑿石以安壽宮者,與曩所立表,其地不一。朦朧易徙,若弈棋然,非大臣謀國之忠也。」時行奏辨,言:「車駕初閱時,植、東之見臣直廬,力言形龍山不如大峪。今已二年,忽創此議。其借事傾臣明甚。」帝責三人不宜以葬師術責輔臣,奪俸半歲。三人以明習葬法薦侍郎張岳、太常何源。兩人方疏辭,錫爵忽奏言恥為植三人所引,義不可留,因具奏不平者八事。大略言:「張、馮之獄,上志先定,言者適投其會,而輒自附於用賢等攖鱗折檻之黨。且謂舍建言別無人品;建言之中,舍採摭張、馮舊事,別無同志。以中人之資,乘一言之會,超越朝右,日尋戈矛。大臣如國、巍、化輩,曩嘗舉為正人。一言相左,日謀事刂刃,皆不平之大者。」御史韓國楨,給事中陳與郊、王敬民等因迭攻植等,帝下敬民疏,貶植戶部員外郎,東之兵部員外郎,可立大理評事。張岳以諸臣紛爭,具疏評其賢否,頗為植、東之、可立地,請令各宣力一方,以全終始。於時行、國、錫爵、巍、化、光祖、世臣、定、愈賢皆褒中寓刺,而力詆季馴、懋賢、系周、仲慶,惟中行、用賢、思孝無所譏貶。帝責岳頌美大臣,且支蔓,不足定國是,岳坐免。帝猶以植言壽宮有石數十丈,如屏風,其下皆石,恐寶座將置於石上。閏月,復躬往視之,終謂大峪吉,遂調三人於外。御史柯梃因自言習葬法,力稱大峪之美,獲督南畿學政。而植同年生給事中盧逵亦承風請正三人罪,士論哂之。

植、東之、可立自以言事見知,未及三歲而貶。植得綏德知州,旋引疾歸。居十年,起沅州知州。屢官右僉都御史,巡撫遼東。時二十六年也。植墾土積粟,得田四萬畝,歲獲糧萬石。戶部推其法九邊。以倭寇退,請因師旋,選主、客銳卒,驅除宿寇,恢復舊遼陽。詔下總督諸臣詳議,不果行。奏稅監高淮貪暴,請召還,不報。後淮激變,委阻撓罪於植。植疏辨乞休,帝慰留之。明年,錦、義失事,巡按御史王業弘劾植及諸將失律。植以卻敵聞,且詆業弘。業弘再疏劾植欺蔽,詔解官聽勘。勘已,命家居聽用,竟不召。卒,贈兵部右侍郎。

可立,汝陽人。由安邑知縣為御史,與植等並擢。已,由評事調大名推官。終山東僉事。

江東之,字長信,歙人。萬歷五年進士。由行人擢御史。首發馮保、徐爵奸,受知於帝。僉都御史王宗載嘗承張居正指,與於應昌共陷劉臺,東之疏劾之。故事,御史上封事,必以副封白長官。東之持入署,宗載迎謂曰:「江御史何言?」曰:「為死御史鳴冤。」問為誰?曰:「劉臺也。」宗載失氣反走,遂與應昌俱得罪。東之出視畿輔屯政,奏駙馬都尉侯拱宸從父豪奪民田,置於理。先是,皇子生,免天下田租三之一,獨不及皇莊及勛戚莊田。東之為言,減免如制。還朝,擢光祿少卿,改太僕。坐爭壽宮事,與李植、羊可立皆貶。東之得霍州知州,以病免。久之,起鄧州,進湖廣僉事。三遷大理寺右少卿。二十四年,以右僉都御史巡撫貴州。擊高砦叛苗,斬首百餘級。京察,被劾免官。復以遣指揮楊國柱討楊應龍敗績事,黜為民。憤恨抵家卒。

東之官行人時,刑部郎舒邦儒闔門病疫死,遺孤一歲,人莫敢過其門。東之經紀其喪,提其孤歸,乳之。舒氏卒有後。

湯兆京,字伯閎,宜興人。萬曆二十年進士。除豐城知縣。治最,徵授御史。連劾禮部侍郎朱國祚、薊遼總督萬世德,帝不問。巡視西城,貴妃宮閹豎塗辱禮部侍郎敖文禎,兆京彈劾,杖配南京。時礦稅繁興,奸人競言利。有謂開海外機易山,歲可獲金四十萬者,有請徵徽、寧諸府契稅,鬻高淳諸縣草場者,帝意俱向之。兆京偕同官金忠士、史學遷、溫如璋交章力諫,不報。出按宣府、大同,請罷稅使張曄、礦使王虎、王忠,亦不納。掌河南道。佐孫丕揚典京察,所譴黜皆當,而被黜者之黨爭相攻擊。兆京亦十餘疏應之。其詞直,卒無以奪也。詳具丕揚傳中。尋出按順天諸府。守陵中官李浚誣軍民盜陵木,逮繫無虛日。兆京按宣府時奏之,浚亦誣訐兆京。帝遣使按驗,事已白,而諸被繫者猶未釋,兆京悉縱遣之。東廠太監盧受縱其下橫都市,兆京論如法。

還復掌河南道。福王久不之國,兆京倡給事御史伏闕固請,卒不得命。南京缺提學御史,吏部尚書趙煥調浙江巡按呂圖南補之,尋以年例出三御史於外,皆不咨都察院。兆京引故事爭。圖南之調,為給事中周永春所劾,棄官歸。兆京及御史王時熙、汪有功為圖南申雪,語侵永春,并及煥,二人連章辨,兆京亦爭之強。帝欲安煥,為稍奪兆京俸。兆京以不得其職,拜疏徑歸。御史李邦華、周起元、孫居相遂助兆京攻煥。帝亦奪其俸,然煥亦引去。

兆京居官廉正,遇事慷慨。其時黨勢已成,正人多見齮齕。兆京力維持其間,清議倚以為重。屢遭排擊,卒無能一言污之者。天啟中,贈太僕少卿。

金士衡,字秉中,長洲人。父應徵,雲南參政,以廉能稱。士衡舉萬曆二十年進士,授永豐知縣,擢南京工科給事中。疏陳礦稅之害,言:「曩者採於山,榷於市,今則不山而采,不市而榷矣。刑餘小醜,市井無藉,安知遠謀,假以利柄,貪饕無厭。楊榮啟釁於麗江,高淮肆毒於遼左,孫朝造患於石嶺,其尤著者也。今天下水旱盜賊,所在而有。蕭、碭、豐、沛間河流決隄,居人為魚鱉,乃復橫徵巧取以蹙之。獸窮則攫,鳥窮則啄,禍將有不可言者。」甘肅地震,復上疏曰:「往者湖廣冰雹,順天晝晦,豐潤地陷,四川星變,遼東天鼓震,山東、山西則牛妖,人妖、今甘肅天鳴地裂,山崩川竭矣。陛下明知亂徵,而泄泄從事,是以天下戲也。」因極言邊糈告匱,宜急出內帑濟餉,罷撤稅使,毋事掊克,引鹿臺、西園為戒。帝皆不聽。南京督儲尚書王基、雲南巡撫陳用賓拾遺被劾,給事中錢夢皋、御史張以渠等考察被黜,為沈一貫所庇,帝皆留之。士衡疏爭。侍郎周應賓、黃汝良、李廷機當預推內閣。士衡以不協人望,抗章論。姜士昌、宋燾言事得罪,並申救之。給事中王元翰言軍國機密不宜抄傳,詔併禁章奏未下者。由是中朝政事,四方寂然不得聞。士衡力陳其非便。疏多不行。帝召王錫爵為首輔,以被劾奏辨,語過憤激,士衡馳疏劾之。尋擢南京通政參議。時元翰及李三才先後為言者所攻,士衡並為申雪。三十九年,大計京官。掌南察者,南京吏部侍郎史繼偕,齊、楚、浙人之黨也,與孫丕揚北察相反,凡助三才、元翰者悉斥之。士衡亦謫兩浙鹽運副使,不赴。天啟初,起兵部員外郎。累遷太僕少卿。引疾去,卒於家。

先是,楊應龍伏誅,貴州宣慰使安疆臣邀據故所侵地。總督王象乾不許。士衡遂劾象乾起釁。後象乾弟象恒巡撫蘇、松,以兄故頗銜士衡。廉知其清介狀,稱說不置云。

王元翰,字伯舉,雲南寧州人。萬曆二十九年進士。選庶吉士。三十四年,改吏科給事中。意氣陵厲,以諫諍自任。時廷臣習偷惰,法度盡弛。會推之柄散在九列科道。率推京卿,每署數倍舊額。而建言諸臣,一斥不復。大臣被彈,率連章詆訐。元翰悉疏論其非。

尋進工科右給事中,巡視廠庫,極陳惜薪司官多之害。其秋上疏,極言時事敗壞,請帝味爽視朝,廷見大臣,言官得隨其後,日陳四方利病。尋復陳時事,言:「輔臣,心膂也。硃賡輔政三載,猶未一覯天顏,可痛哭者一。九卿強半虛懸,甚者闔署無一人。監司、郡守,亦曠年無官,或一人綰數符。事不切身,政自茍且,可痛哭者二。兩都臺省寥寥幾人。行取入都者,累年不被命。庶常散館亦越常期。御史巡方事竣,遣代無人。威令不行,上下胥玩,可痛哭者三。被廢諸臣,久淪山谷。近雖奉詔敘錄,未見連茹匯征。茍更閱數年,日漸銷鑠。人之云亡,邦國殄瘁,可痛哭者四。九邊歲餉,缺至八十餘萬,平居凍餒,脫巾可虞;有事怨憤,死綏無望。塞北之患,未可知也。京師十餘萬兵,歲靡餉二百餘萬,大都市井負販游手而已。一旦有急,能驅使赴敵哉?可痛哭者五。天子高拱深居,所恃以通下情者,只章疏耳,今一切高閣。慷慨建白者莫不曰『吾知無濟,第存此議論耳』。言路惟空存議論,世道何如哉!可痛哭者六。榷稅使者滿天下,致小民怨聲徹天,降災召異。方且指殿工以為名,借停止以愚眾。是天以回祿警陛下,陛下反以回祿剝萬民也。眾心離叛,而猶不知變,可痛哭者七。郊廟不親,則天地祖宗不相屬;朝講不御,則伏機隱禍不上聞。古今未有如此而天下無事者。且青宮輟講,亦已經年,親宦官宮妾,而疏正人端士,獨奈何不為宗社計也!可痛哭者八。」帝皆不省。

武定賊阿克作亂。元翰上言:「克本小醜,亂易平也。至雲南大害,莫甚貢金、榷稅二事。民不堪命,至殺稅使,而征榷如故。貢金請減,反增益之。眾心憤怒,使亂賊假以為名。賊首縱撲滅,虐政不除,滇之為滇,猶未可保也。」俄言:「礦稅之設,本為大工。若捐內帑數百萬金,工可立竣,毋徒苦四方萬姓。」疏皆不報。尋兩疏劾貴州巡撫郭子章等凡四人,言:「子章曲庇安疆臣,堅意割地,貽西南大憂。且嘗著《婦寺論》,言人主當隔絕廷臣,專與宦官宮妾處,乃相安無患。子章罪當斬。」不納。

先是,廷推閣臣。元翰言李廷機非宰相器。已而黃汝良推吏部侍郎,全天敘推南京禮部侍郎。汝良,廷機邑人;天敘,朱賡同鄉也。元翰極論會推之弊,譏切政府,二人遂不用。至是,將推兩京兵部尚書蕭大亨、孫幰為吏部尚書。元翰亦疏論二人,并言職方郎申用懋為大亨謀主,太常少卿唐鶴徵為幰謀主,亦當斥。尋因災異,乞亟罷賡、大亨及副都御史詹沂。且言:「近更有二大變。大小臣工志期得官,不顧嗤笑,此一變也。陛下不恤人言,甚至天地譴告亦悍然弗顧,此又一變也。有君心之變,然後臣工之變因之。在今日,挽天地洪水寇賊之變易,挽君心與臣工之變難。」又言:「陛下三十年培養之人才,半掃除於申時行、王錫爵,半禁錮於沈一貫、朱賡。」因薦鄒元標、顧憲成等十餘人。無何,又劾給事中喻安性、御史管橘敗群叢穢,皆不報。掌廠內官王道不法,疏暴其罪,亦弗聽。

元翰居諫垣四年,力持清議。摩主闕,拄貴近,世服其敢言。然銳意搏擊,毛舉鷹鷙,舉朝咸畏其口。吏科都給事中陳治則與元翰不相能,御史鄭繼芳,其門人也,遂劾元翰盜庫金,剋商人貲,奸贓數十萬。元翰憤甚,辨疏詆繼芳北鄙小賊,語過激。於是繼芳黨劉文炳、王紹徽、劉國縉等十餘疏並攻之,而史記事、胡忻、史學遷、張國儒、馬孟禎、陳于廷、吳亮、金士衡、劉節、劉蘭輩則連章論救。帝悉不省。元翰乃盡出其筐篋,舁置國門,縱吏士簡括,慟哭辭朝而去。吏部坐擅離職守,謫刑部檢校。後孫丕揚主京察,斥治則、國縉等,亦以浮躁坐元翰,再貶湖廣按察知事。方繼芳之發疏也,即潛遣人圍守元翰家。及元翰去,所劾贓無有,則謂寄之記事家。兩黨分爭久不息。而是時劾李三才者亦指其貪,諸左右元翰者又往往左右三才,由是臣僚益相水火,而朋黨之勢成矣。

天啟初,累遷刑部主事。魏忠賢亂政,其黨石三畏劾之,削籍。莊烈帝即位,復官。將召用,為尚書王永光所尼。元翰乃流寓南都,十年不歸。卒,遂葬焉。

孫振基,字肖岡,潼關衛人。萬曆二十九年進士。除莘縣知縣,調繁安丘。三十六年四月,以治行征,與李成名等十七人當授給事中,先除禮部主事。四十年十月命始下,振基得戶科。時吏部推舉大僚,每患乏才,振基力請起廢。

韓敬者,歸安人也,受業宣城湯賓尹。賓尹分校會試,敬卷為他考官所棄。賓尹搜得之,強總裁侍郎蕭雲舉、王圖錄為第一。榜發,士論大譁。知貢舉侍郎吳道南欲奏之,以雲舉、圖資深,嫌擠排前輩,隱不發。及廷對,賓尹為敬夤緣得第一人。後賓尹以考察褫官,敬亦稱病去,事三年矣。會進士鄒之麟分校順天鄉試,所取童學賢有私,於是御史孫居相并賓尹事發之。下禮官,會吏部都察院議,顧不及賓尹事。振基乃抗疏請並議,未得命。禮部侍郎翁正春等議黜學賢,謫之麟,亦不及賓尹等。振基謂議者庇之,再疏論劾。帝乃下廷臣更議。御史王時熙、劉策、馬孟禎亦疏論其事,而南京給事中張篤敬證尤力。方賓尹之分校也,越房取中五人,他考官效之,競相搜取,凡十七人。時賓尹雖廢,中朝多其黨,欲藉是寬敬。正春乃會九卿趙煥及都給事中翁憲祥、御史餘懋衡等六十三人議坐敬不謹,落職閒住。御史劉廷元、董元儒、過庭訓,敬同鄉也,謂敬關節果真,罪非止不謹,執不署名,意欲遷延為敬地。正春等不從,持初議上。廷元遂疏劾之,公議益憤。振基、居相、篤敬及御史魏雲中等連章論列。給事中商周祚亦敬同鄉,議并罪道南。孟禎以道南發奸,不當罪,再疏糾駁。帝竟如廷元等言,敕部更核。廷元黨亓詩教遂劾正春首鼠兩端,正春尋引去。

會熊廷弼之議亦起。初,賓尹家居,嘗奪生員施天德妻為妾,不從,投繯死。諸生馮應祥、芮永縉輩訟於官,為建祠,賓尹恥之。後永縉又發諸生梅振祚、宣祚朋淫狀。督學御史熊廷弼素交歡賓尹,判牒言此施、湯故智,欲藉雪賓尹前恥。又以所司報永縉及應祥行劣,杖殺永縉。巡按御史荊養喬遂劾廷弼殺人媚人,疏上,徑自引歸。廷弼亦疏辨。都御史孫瑋議鐫養喬秩,令廷弼解職候勘。時南北臺諫議論方囂,各自所左右。振基、孟禎、雲中策及給事李成名、麻僖、陳伯友,御史李邦華、崔爾進、李若星、潘之祥、翟鳳翀、徐良彥等持勘議甚力。而篤敬及給事中官應震、姜性、吳亮嗣、梅之煥、亓詩教、趙興邦,御史黃彥士,南京御史周遠等駁之,疏凡數十上。振基及諸給事御史復極言廷弼當勘,斥應震等黨庇,自是黨廷弼者頗屈。帝竟納瑋言,令廷弼解職。其黨大恨。吏部尚書趙煥者,惟詩教言是聽,乃以年例出振基及雲中、時熙於外。振基得山東僉事,瑋亦引去。

振基勁直敢言。居諫垣僅半歲,數有建白。既去,科場議猶未定,策復上疏極論。而賓尹黨必欲十七人並罪,以寬敬。孫慎行代正春,復集廷臣議。仍坐敬關節,而為十七人昭雪。疏竟留中。賓尹、敬有奧援,外廷又多助之,故議久不決。篤敬復上疏論敬,陰詆諸黨人。諸黨人旋出之外,并逐慎行。既而居相、策引去,之祥外遷。孟禎不平,疏言:「廷弼聽勘一事,業逐去一總憲,外轉兩言官矣,獨介介於之祥。敬科場一案,亦去兩侍郎、兩言官矣,復斷斷於篤敬,毋乃已甚乎!」孟禎遂亦調外。凡與敬為難者,朝無一人。敬由是得寬典,僅謫行人司副。蓋七年而事始竣云。振基到官,尋以憂去,卒於家。

子必顯,字克孝。萬曆四十四年進士。官文選員外郎,為尚書趙南星所重。天啟五年冬,魏忠賢羅織清流,御史陳睿謨劾其世投門戶,遂削籍。崇禎二年,起驗封郎中,移考功。明年,移文選。尚書王永光雅不喜東林,給事中常自裕因劾其推舉不當數事,且詆以貪污。御史吳履中又劾其紊亂選法。必顯兩疏辨,帝不聽,謫山西按察司經歷,量移南京禮部主事。道出柘城、歸德,適流賊來犯,皆為設守,完其城。一時推知兵。歷尚寶司丞、大理左寺丞。十一年冬,都城被兵,兵部兩侍郎皆缺,尚書楊嗣昌請不拘常格,博推才望者遷補,遂擢必顯右侍郎。甫一月,無疾而卒。

丁元薦,字長孺,長興人。父應詔,江西僉事。元薦舉萬曆十四年進士。請告歸。家居八年,始謁選為中書舍人。甫期月,上封事萬言,極陳時弊。言今日事勢可寒心者三:饑民思亂也,武備積弛也,日本封貢也。可浩歎者七:征斂苛急也,賞罰不明也,忠賢廢錮也,輔臣妒嫉也,議論滋多也,士習敗壞也,褒功恤忠未備也。坐視而不可救藥者二,則紀綱、人心也。其所言輔臣,專斥首輔王錫爵,元薦座主也。

二十七年京察。元薦家居,坐浮躁論調。閱十有二年,起廣東按察司經歷,移禮部主事。甫抵官,值京察事竣,尚書孫丕揚力清邪黨,反為其黨所攻。副都御史許弘綱故共掌察,見群小橫甚,畏之,累疏請竣察典,語頗示異。群小藉以攻丕揚。察疏猶未下,人情杌隉,慮事中變,然無敢言者。元薦乃上言弘綱持議不宜前卻,并盡發諸人隱狀。黨人惡之,交章論劾無虛日。元薦復再疏辨晰,竟不安其身而去。其後邪黨愈熾,正人屏斥殆盡,至有以「《六經》亂天下」語入鄉試策問者。元薦家居不勝憤,復馳疏闕下,極詆亂政之叛高皇、邪說之叛孔子者。疏雖不報,黨人益惡之。四十五年京察,遂復以不謹削籍。天啟初,大起遺佚。元薦格於例,獨不召。至四年,廷臣交訟其冤,起刑部檢校,歷尚寶少卿。明年,朝事大變,復削其籍。

元薦初學於許孚遠,已,從顧憲成遊。慷慨負氣,遇事奮前,屢躓無少挫。通籍四十年,前後服官不滿一載。同郡沈淮召入閣,邀一見,謝不往。嘗過高攀龍,請與交歡,辭曰:「吾老矣,不能涉嫌要津。」遽別去。當東林、浙黨之分,浙黨所彈射東林者,李三才之次,則元薦與於玉立。

玉立,字中甫,金壇人。萬曆十一年進士。除刑部主事,進員外郎。二十年七月,疏陳時政闕失,言:「陛下寵幸貴妃,宴逸無度。恣行威怒,鞭笞群下,宮人奄豎無辜死者千人。夫人懷必死之心,而使處肘腋房闥間,倘因利乘便,甘心一逞,可不寒心!田義本一奸豎,陛下寵信不疑。邇者奏牘或下或留,推舉或用或否,道路籍籍,咸謂義簸弄其間。蓋義以陛下為城社,而外廷之憸邪又以義為城社。黨合謀連,其禍難量。且陛下一惑於嬖倖,而數年以來,問安視膳,郊廟朝講,一切不行。至邊烽四起,禍亂成形,猶不足以動憂危之情,奪晏安之習。是君身之不修,未有甚於今日者矣。夫宮庭震驚,而陛下若罔聞,何以解兩宮之憂?深拱禁中,開夤緣之隙,致邪孽侵權,而陛下未察其奸,何以杜旁落之漸?萬國欽輩未嘗忤主,而終於禁錮,何以勵骨鯁之臣?上下隔越,國議、軍機無由參斷,而陛下稱旨下令,終不出閨闥之間,何以盡大臣之謀?忠良多擯,邪佞得名,何以作群臣之氣?遠近之民,皆疑至尊日求般樂,不顧百姓塗炭,何以系天下之心?」因力言李如松、麻貴不可為大將,鄭洛不當再起,石星不堪為本兵。疏入,不報。

尋進郎中,謝病歸。久之,起故官。康丕揚輩欲以妖書陷郭正域,玉立獨左右之。會有言醫人沈令譽實為妖書者,搜其篋,得玉立與吏部郎中王士騏書,中及其起官事。帝方下吏部按問,而玉立遽疏辨。帝怒,褫其官。

玉立倜儻好事。海內建言廢錮諸臣,咸以東林為歸。玉立與通聲氣,東林名益盛。而攻東林者,率謂玉立遙制朝權,以是詬病東林。玉立居家久之,數被推薦。三十七年,稍起光祿丞,辭不赴。言者猶齮齕不已,御史馬孟禎抗章直之,帝皆不省。又三年,以光祿少卿召,終不出。天啟初,錄先朝罪譴諸臣,玉立已前卒,贈尚寶卿。

李朴,字繼白,朝邑人。萬歷二十九年進士。由彰德推官入為戶部主事。四十年夏,朴以朝多朋黨,清流廢錮,疏請破奸黨,錄遺賢,因為顧憲成、于玉立、李三才、孫丕揚辨謗,而薦呂坤、姜士昌、鄒元標、趙南星。帝不聽。明年,再遷郎中。齊、楚、浙三黨勢盛,稍持議論者,群噪逐之。主事沈正宗、賀烺皆與相拄,坐貶官。朴性戇,積憤不平。其年十二月,上疏曰:

朝廷設言官,假之權勢,本責以糾正諸司,舉刺非法,非欲其結黨逞威,挾制百僚,排斥端人正士也。今乃深結戚畹近坐,威制大僚;日事請寄,廣納賂遺;褻衣小車,遨遊市肆,狎比娼優;或就飲商賈之家,流連山人之室。身則鬼蜮,反誣他人。此蓋明欺至尊不覽章奏,大臣柔弱無為,故猖狂恣肆,至於此極。臣謂此輩皆可斬也。

孫瑋、湯兆京、李邦華、孫居相、周起元各爭職掌,則群攻之。今或去或罰,惟存一居相,猶謂之黨。夫居相一人耳,何能為?彼浙江則姚宗文、劉廷元輩,湖廣則官應震、吳亮嗣、黃彥士輩,山東則亓詩教、周永春輩,四川則田一甲輩,百人合為一心,以擠排善類,而趙興邦輩附麗之。陛下試思居相一人敵宗文輩百人,孰為有黨耶?乃攻東林者,今日指為亂政,明日目為擅權,不知東林居何官?操何柄?在朝列言路者,反謂無權,而林下投閒杜門樂道者,反謂有權,此不可欺三尺豎子,而乃以欺陛下哉!至若黃克纘贓私鉅萬,已敗猶見留;顧憲成清風百代,已死猶被論;而封疆坐死如陳用賓,科場作奸如韓敬,趨時鬻爵如趙煥,殺人媚人如熊廷弼,猶為之營護,為之稱冤。國典安在哉!

望俯察臣言,立賜威斷,先斬臣以謝諸奸,然後斬諸奸以謝天下,宗社幸甚。

疏奏,臺諫皆大恨。宗文等及其黨力詆,并侵居相,而一甲且羅織其贓私。帝雅不喜言官,得朴疏,心善之。會大學士葉向高、方從哲亦謂朴言過當,乃下部院議罰。而朴再疏發亮嗣、應震、彥士、一甲贓私,及宗文、廷元庇韓敬、興邦媚趙煥狀,且言:「詩教為群兇盟主,實社稷巨蠹,陛下尤不可不察。」帝為下詔切責言官,略如朴指。黨人益怒,排擊無虛日。侍郎李汝華亦以屬吏出位妄言劾朴。部院議鐫朴三級,調外任,帝持不下。至明年四月,吏部奉詔起廢,朴名預焉。於是黨人益譁,再起攻朴,并及文選郎郭存謙。存謙引罪,攻者猶未已。朴益憤,復陳浙人空國之由,追咎沈一貫,詆宗文及毛一鷺甚力,以兩人皆浙產也。頃之,又再疏劾宗文、一鷺及其黨董定策等。帝皆置不問。其年六月,始用閣臣言,下部院疏,謫朴州同知。自後黨人益用事,遂以京察落其職。

天啟初,起用,歷官參議。卒,贈太僕少卿。魏忠賢竊柄,御史安伸追論,詔奪其贈。崇禎初,復焉。

夏嘉遇,字正甫,松江華亭人。萬曆三十八年進士。授保定推官。

四十五年,用治行征。當擢諫職,先注禮部主事。帝久倦勤,方從哲獨柄國。碌碌充位,中外章奏悉留中。惟言路一攻,則其人自去,不待詔旨。臺諫之勢,積重不返,有齊、楚、浙三方鼎峙之名。齊則給事中亓詩教、周永春,御史韓浚。楚則給事中官應震、吳亮嗣。浙則給事中姚宗文、御史劉廷元。而湯賓尹輩陰為之主。其黨給事中趙興邦、張延登、徐紹吉、商周祚,御史駱駸曾、過庭訓、房壯麗、牟志夔、唐世濟、金汝諧、彭宗孟、田生金、李徵儀、董元儒、李嵩輩,與相倡和,務以攻東林排異己為事。其時考選久稽,屢趣不下,言路無幾人,盤踞益堅。後進當入為臺諫者,必鉤致門下,以為羽翼,當事大臣莫敢攖其鋒。

詩教者,從哲門生,而吏部尚書趙煥鄉人也。煥耄昏,兩人一聽詩教。詩教把持朝局,為諸黨人魁。武進鄒之麟者,浙人黨也。先坐事謫上林典簿,至是為工部主事,附詩教、浚。求吏部不得,大恨,反攻之,并詆從哲。詩教怒,煥為黜之麟。時嘉遇及工部主事鐘惺、中書舍人尹嘉賓、行人魏光國皆以才名,當列言職。詩教輩以與之麟善,抑之,俾不與考選。以故嘉遇不能無怨。

四十七年三月,遼東敗書聞,嘉遇遂抗疏言:「遼左三路喪師,雖緣楊鎬失策,揆厥所由,則以縱貸李維翰故。夫維翰喪師辱國,罪不容誅,乃僅令回籍聽勘。誰司票擬?則閣臣方從哲也;誰司糾駁?則兵科趙興邦也。參貂白鏹,賂遺繹絡,國典邊防,因之大壞。惟陛下立斷。」疏入,未報。從哲力辨,嘉遇再疏劾之,并及詩教。於是詩教、興邦及亮嗣、延登、壯麗輩交章力攻。詩教謂嘉遇不得考選,故挾私狂逞。嘉遇言:「詩教於從哲,一心擁戴,相倚為奸。凡枚卜、考選諸大政,百方撓阻,專務壅蔽,遏絕主聰。遂致綱紀不張,戎馬馳突,臣竊痛之。今內治盡壞,縱日議兵食、談戰守,究何益於事?故臣為國擊奸,冀除禍本,雖死不避,尚區區計升沉得喪哉!」

時興邦以右給事中掌兵科。先有旨,俟遼東底寧,從優敘錄。至是以嘉遇連劾,吏部遂立擢為太常少卿。嘉遇益憤,疏言:「四路奏功,興邦必將預其賞。則今日事敗,興邦安得逃其罰?且不罰已矣,反從而超陟之。是臣彈章適為薦剡,國家有如是法紀哉!」疏奏,諸御史復合詞攻嘉遇。嘉遇復疏言:「古人有云,見無禮於君者逐之,如鷹鸇之逐鳥雀也。詩教、興邦謂臣不得臺諫而怒。夫爵位名秩,操之天子,人臣何敢干?必如所言,是考選予奪,二臣實專之。此無禮於君者一。事寧優敘,非明旨乎?乃竟蔑而棄之。此無禮於君者二。魏光國疏論詩教,為通政沮格。夫要截實封者斬。自來奸臣不敢為,而詩教為之。此無禮於君者三。二奸每事請託,一日以七事屬職方郎楊成喬。成喬不聽,遂逐之去。詩教以舊憾欲去其鄉知府,考功郎陳顯道不從,亦逼之去。夫吏、兵二部,天子所以馭天下也,而二奸敢侵越之。此無禮於君者四。有臣如此,臣義豈與俱生哉!」

先是,三黨諸魁交甚密,後齊與浙漸相貳。布衣汪文言者,素游黃正賓、於玉立之門,習知黨人本末。後玉立遣之入都,益悉諸黨人所為,策之曰:「浙人者,主兵也,齊、楚則應兵。成功之後,主欲逐客矣,然柄素在客,未易逐,此可構也。」遂多方設奇間之,諸人果相疑。而鄒之麟既見惡齊黨,亦交鬥其間。揚言齊人張鳳翔為文選,必以年例斥宗文、廷元。於是齊、浙之黨大離。及是嘉遇五疏力攻,詩教輩亦窘。而浙人唐世濟、董元儒遂助嘉遇排擊。自是亓、趙之勢頓衰,興邦竟不果遷,自引去。時論快焉。

光宗立,嘉遇乞改南部,就遷吏部員外郎。天啟中,趙南星秉銓,召為考功員外郎,改文選署選事。時左光斗、魏大中以嘉遇與之麟、韓敬同年相善,頗疑之。已,見嘉遇公廉,亦皆親善。及陳九疇劾謝應祥,語連嘉遇,鐫三級,調外,語具南星傳。未幾,黨人張訥誣劾南星,并及嘉遇,遂除名。尋鍛煉光斗、大中獄,誣嘉遇嘗行賄。逮訊論徒,憤恨發病卒。崇禎初,贈太常少卿。

贊曰:李植、江東之諸人,風節自許,矯首抗俗,意氣橫厲,抵排群枉,迹不違乎正。而質之矜而不爭、群而不黨之義,不能無疚心焉。「古之矜也廉,今之矜也忿戾」,聖人所為致慨於末世之益衰也。


\end{pinyinscope}