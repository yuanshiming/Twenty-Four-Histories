\article{列傳第二百十二 外國五}

\begin{pinyinscope}
○占城賓童龍真臘暹羅爪哇闍婆蘇吉丹碟里日羅夏治三佛齊

占城居南海中,自瓊州航海順風一晝夜可至,自福州西南行十晝夜可至,即周越裳地。秦為林邑,漢為象林縣。後漢末,區連據其地,始稱林邑王。自晉至隋仍之。唐時,或稱占不勞,或稱占婆,其王所居曰占城。至德後,改國號曰環。迄周、宋,遂以占城為號,朝貢不替。元世祖惡其阻命,大舉兵擊破之,亦不能定。

洪武二年,太祖遣官以即位詔諭其國。其王阿荅阿者先已遣使奉表來朝,貢象虎方物。帝喜,即遣官齎璽書、《大統曆》、文綺、紗羅,偕其使者往賜,其王復遣使來貢。自後或比歲貢,或間歲,或一歲再貢。未幾,命中書省管勾甘桓、會同館副使路景賢齎詔,封阿荅阿者為占城國王,賜彩幣四十、《大統曆》三千。三年遣使往祀其山川,尋頒科舉詔於其國。

初,安南與占城搆兵,天子遣使諭解,而安南復相侵。四年,其王奉金葉表來朝,長尺餘,廣五寸,刻本國字。館人譯之,其意曰:「大明皇帝登大寶位,撫有四海,如天地覆載,日月照臨。阿荅阿者譬一草木爾,欽蒙遣使,以金印封為國王,感戴忻悅,倍萬恒情。惟是安南用兵,侵擾疆域,殺掠吏民。伏願皇帝垂慈,賜以兵器及樂器、樂人,俾安南知我占城乃聲教所被,輸貢之地,庶不敢欺陵。」帝命禮部諭之曰:「占城、安南並事朝廷,同奉正朔,乃擅自構兵,毒害生靈,既失事君之禮,又乖交鄰之道。已咨安南國王,令即日罷兵。本國亦宜講信修睦,各保疆土。所請兵器,於王何吝,但兩國互構而賜占城,是助爾相攻,甚非撫安之義。樂器、樂人,語音殊異,難以遣發。爾國有曉華言者,其選擇以來,當令肄習。」因命福建省臣勿徵其稅,示懷柔之意。

六年,貢使言:「海寇張汝厚、林福等自稱元帥,剽劫海上。國主擊破之,賊魁溺死,獲其舟二十艘、蘇木七萬斤,謹奉獻。」帝嘉之,命給賜加等。冬,遣使獻安南之捷。帝謂省臣曰:「去冬,安南言占城犯境;今年,占城謂安南擾邊,未審曲直。可遣人往諭,各罷兵息民,毋相侵擾。」十年與安南王陳耑大戰,耑敗死。十二年,貢使至都,中書不以時奏。帝切責丞相胡惟庸、汪廣洋,二人遂獲罪。遣官賜王《大統曆》及衣幣,令與安南修好罷兵。

十三年遣使賀萬壽節。帝聞其與安南水戰不利,賜敕諭曰:「曩者安南兵出,敗於占城。占城乘勝入安南,安南之辱已甚。王能保境息民,則福可長享;如必驅兵苦戰,勝負不可知,而鷸蚌相持,漁人得利,他日悔之,不亦晚乎?」十六年貢象牙二百枝及方物。遣官賜以勘合、文冊及織金文綺三十二、磁器萬九千。十九年遣子寶部領詩那日忽來朝,賀萬壽節,獻象五十四,皇太子亦有獻。帝嘉其誠,賜賚優渥,命中官送還。明年復貢象五十一及伽南、犀角諸物,帝加宴賚。還至廣東,復命中官宴餞,給道里費。

真臘貢象,占城王奪其四之一,其他失德事甚多。帝聞之,怒。二十一年夏,命行人董紹敕責之。紹未至,而其貢使抵京。尋復遣使謝罪,乃命宴賜如制。

時阿荅阿者失道,大臣閣勝懷不軌謀,二十三年弒王自立。明年遣太師奉表來貢,帝惡其悖逆,卻之。三十年後,復連入貢。

成祖即位,詔諭其國。永樂元年,其王占巴的賴奉金葉表朝貢,且告安南侵掠,請降敕戒諭。帝可之,遣行人蔣賓興、王樞使其國,賜以絨、錦、織金文綺、紗羅。明年,以安南王胡牴奏,詔戢兵,遣官諭占城王。而王遣使奏:「安南不遵詔旨,以舟師來侵,朝貢人回,賜物悉遭奪掠。又畀臣冠服、印章,俾為臣屬。且已據臣沙離牙諸地,更侵掠未已,臣恐不能自存。乞隸版圖,遣官往治。」帝怒,敕責胡牴,而賜占城王鈔幣。

四年貢白象方物,復告安南之難。帝大發兵往討,敕占城嚴兵境上,遏其越逸,獲者即送京師。五年攻取安南所侵地,獲賊黨胡烈、潘麻休等獻俘闕下,貢方物謝恩。帝嘉其助兵討逆,遣中官王貴通齎敕及銀幣賜之。

六年,鄭和使其國。王遣其孫舍楊該貢象及方物謝恩。十年,其貢使乞冠帶,予之。復命鄭和使其國。

十三年,王師方征陳季擴,命占城助兵。尚書陳洽言:「其王陰懷二心,愆期不進,反以金帛、戰象資季擴,季擴以黎蒼女遺之。復約季擴舅陳翁挺侵升華府所轄四州十一縣地。厥罪維均,宜遣兵致討。」帝以交址初平,不欲勞師,但賜敕切責,俾還侵地,王即遣使謝罪。十六年,遣其孫舍那挫來朝。命中官林貴、行人倪俊送歸,有賜。

宣德元年,行人黃原昌往頒正朔,繩其王不恪,卻所酬金幣以歸,擢戶部員外郎。

正統元年,瓊州知府程瑩言:「占城比年一貢,勞費實多。乞如暹羅諸國例,三年一貢。」帝是之,敕其使如瑩言,賜王及妃綵幣。然番人利中國市易,雖有此令,迄不遵。

六年,王占巴的賴卒,其孫摩訶賁該以遺命遣王孫述提昆來朝貢,且乞嗣位。乃遣給事中管曈、行人吳惠齎詔,封為王,新王及妃並有賜。七年春,述提昆卒於途,帝憫之,遣官賜祭。八年遣從子且揚樂催貢舞牌旗黑象。

十一年,敕諭摩訶賁該曰:「邇者,安南王黎濬遣使奏王欺其孤幼,曩已侵升、華、思、義四州,今又屢攻化州,掠其人畜財物。二國俱受朝命,各有分疆,豈可興兵構怨,乖睦鄰保境之義。王宜祗循禮分,嚴飭邊臣,毋恣肆侵軼,貽禍生靈。」並諭安南嚴行備禦,毋挾私報復。先是,定三年一貢之例,其國不遵。及詰其使者,則云:「先王已逝,前敕無存,故不知此令。」是歲,貢使復至,再敕王遵制,賜王及妃彩幣。冬復遣使來貢。

十二年,王與安南戰,大敗被執。故王占巴的賴姪摩訶貴來遣使奏:「先王抱疾,曾以臣為世子,欲令嗣位。臣時年幼,遜位於舅氏摩訶賁該。後屢興兵伐安南,致敵兵入舊州古壘等處,殺掠人畜殆盡,王亦被擒。國人以臣先王之姪,且有遺命,請臣代位。辭之再三,不得已始於府前治事。臣不敢自專,伏候朝命。」乃遣給事中陳誼、行人薛乾封為王,諭以保國交鄰,並諭國中臣民共相輔翼。十三年敕安南送摩訶賁該還國,不奉命。

景泰三年遣使來貢,且告王訃。命給事中潘本愚、行人邊永封其弟摩訶貴由為王。

天順元年入貢,賜其正副使鈒花金帶。二年,王摩訶槃羅悅新立,遣使奉表朝貢。四年復貢,自正使以下賜紗帽及金銀角帶有差。使者訴安南見侵,因敕諭安南王。九月,使來,告王喪。命給事中黃汝霖、行人劉恕封王弟槃羅茶全為王。

八年入貢。憲宗嗣位,應頒賜蕃國錦幣,禮官請付使臣齎回,從之。使者復訴安南見侵,求索白象。乞如永樂時,遣官安撫,建立界牌石,以杜侵陵。兵部以兩國方爭,不便遣使,乞令使臣歸諭國王,務循禮法,固封疆,捍外侮,毋輕構禍,從之。

成化五年入貢。時安南索占城犀象、寶貨,令以事天朝之禮事之。占城不從,大舉往伐。七年破其國,執王槃羅茶全及家屬五十餘人,劫印符,大肆焚掠,遂據其地。王弟槃羅茶悅逃山中,遣使告難。兵部言:「安南吞並與國,若不為處分,非惟失占城歸附之心,抑恐啟安南跋扈之志。宜遣官齎敕宣諭,還其國王及眷屬。」帝慮安逆命,令俟貢使至日,賜敕責之。

八年,以槃羅茶悅請封,命給事中陳峻、行人李珊持節往。峻等至新州港,守者拒之,知其國已為安南所據,改為交南州,乃不敢入。十年冬還朝。

安南既破占城,復遣兵執槃羅茶悅,立前王孫齋亞麻弗菴為王,以國南邊地予之。十四年,遣使朝貢請封,命給事中馮義、行人張瑾往封之。義等多攜私物,既至廣東,聞齋亞麻弗菴已死,其弟古來遣使乞封。義等慮空還失利,亟至占城。占城人言,王孫請封之後,即為古來所殺,安南以偽敕立其國人提婆苔為王。義等不俟奏報,輒以印幣授提婆苔封之,得所賂黃金百餘兩,又往滿剌加國盡貨其私物以歸。義至海洋病死。瑾具其事,并上偽敕於朝。

十七年,古來遣使朝貢,言:「安南破臣國時,故王弟槃羅茶悅逃居佛靈山。比天使齎封誥至,已為賊人執去,臣與兄齋亞麻弗菴潛竄山谷。後賊人畏懼天威,遣人訪覓臣兄,還以故地。然自邦都郎至占臘止五處,臣兄權國未幾,遽爾隕歿。臣當嗣立,不敢自專,仰望天恩,賜之冊印。臣國所有土地本二十七處,四府、一州、二十二縣。東至海,南至占臘,西至黎人山,北至阿本喇補,凡三千五百餘里。乞特諭交人,盡還本國。」章下廷議,英國公張懋等請特遣近臣有威望者二人往使。時安南貢使方歸,即賜敕詰責黎灝,令速還地,毋抗朝命。禮官乃劾瑾擅封,執下詔獄,具得其情,論死。時古來所遣使臣在館,召問之,云:「古來實王弟,其王病死,非弒。提婆苔不知何人。」乃命使臣暫歸廣東,俟提婆苔使至,審誠偽處之。使臣候命經年,提婆苔使者不至,乃令還國。

二十,年敕古來撫諭提婆苔,使納原降國王印,宥其受偽封之罪,仍為頭目。提婆苔不受命,乃遣給事中李孟暘、行人葉應冊封古來為國王。孟暘等言:「占城險遠,安南構兵未已,提婆苔又竊據其地,稍或不慎,反損國威。宜令來使傳諭古來,詣廣東受封,並敕安南悔禍。」從之。古來乃自老撾挈家赴崖州,孟暘竣封事而返。古來又欲躬詣闕廷,奏安南之罪。二十三年,總督宋旻以聞。廷議遣大臣一人往勞,檄安南存亡繼絕,迎古來返占城。帝報可,命南京右都御史屠滽往。至廣東,即傳檄安南,宣示禍福。募健卒二千人,駕海舟二十艘,護古來還國。安南以滽大臣奉特遣,不敢抗,古來乃得入。

明年,弘治改元,遣使入貢。二年遣弟卜古良赴廣東,言:「安南仍肆侵陵,乞如永樂時遣將督兵守護。」總督秦紘等以聞。兵部言:「安南、占城皆《祖訓》所載不征之國。永樂間命將出師,乃正黎賊弒逆之罪,非以鄰境交惡之故。今黎灝修貢惟謹,古來膚受之醖,容有過情,不可信其單詞,勞師不征之國。宜令守臣回咨,言近交人殺害王子古蘇麻,王即率眾敗之,仇恥已雪。王宜自強修政,撫颻國人,保固疆圉,仍與安南敦睦修好。其餘嫌細故,悉宜捐除。倘不能自強,專藉朝廷發兵渡海,代王守國,古無是理。」帝如其言。三年遣使謝恩。其國自殘破後,民物蕭條,貢使漸稀。

十二,年遣使奏:「本國新州港之地,仍為安南侵奪,患方未息。臣年已老,請及臣未死,命長子沙古卜洛襲封,庶他日可保國土。」廷議:「安南為占城患,已非一日。朝廷嘗因占城之醖,累降璽書,曲垂誨諭。安南前後奏報,皆言祗承朝命,土地人民,悉已退還。然安南辨釋之語方至,而占城控訴之詞又聞,恐真有不獲已之情。宜仍令守臣切諭安南,毋貪人土地,自貽禍殃,否則議遣偏師往問其罪。至占城王長子,無父在襲封之理。請令先立為世子攝國事,俟他日當襲位時,如例請封。」帝報允。尋遣王孫沙不登古魯來貢。

十八年,古來卒。子沙古卜洛遣使來貢,不告父喪,但乞命大臣往其國,仍以新州港諸地封之。別有占奪方輿之奏,微及父卒事。給事中任良弼等言:「占城前因國土削弱,假貢乞封,仰仗天威,讋伏鄰國。其實國王之立不立,不係朝廷之封不封也。今稱古來已歿,虛實難知。萬一我使至彼,古來尚存,將遂封其子乎?抑義不可而已乎?迫脅之間,事極難處。如往時科臣林霄之使滿剌加,不肯北面屈膝,幽餓而死,迄不能問其罪。君命國威,不可不慎。大都海外諸蕃,無事則廢朝貢而自立,有事則假朝貢而請封。今者貢使之來,豈急於求封,不過欲復安南之侵地,還粵東之逃人耳。夫安南侵地,璽書屢諭歸還,占據如故。今若再諭,彼將玩視之,天威褻矣。倘我使往封占城,羈留不遣,求為處分,朝廷將何以應?又或拘我使者,令索逃人,是以天朝之貴臣,質於海外之蠻邦。宜如往年古來就封廣東事,令其領敕歸國,於計為便。」禮部亦以古來存亡未明,請令廣東守臣移文占城勘報,從之,既而封事久不行。

正德五年,沙古卜洛遣叔父沙係把麻入貢,因請封。命給事中李貫、行人劉廷瑞往。貫抵廣東憚行,請如往年古來故事,令其使臣領封。廷議:「遣官已二年,今若中止,非興滅繼絕義。倘其使不願領封,或領歸而受非其人,重起事端,益傷國體,宜令貫等亟往。」貫終憚行,以乏通事、火長為詞。廷議令廣東守臣采訪其人,如終不得,則如舊例行。貫復設詞言:「臣奉命五載,似憚風波之險,殊不知占城自古來被逐後,竄居赤坎邦都郎,國非舊疆,勢不可往。況古來乃前王齋亞麻弗菴之頭目,殺王而奪其位。王有三子,其一尚存,義又不可。律以《春秋》之法,雖不興問罪之師,亦必絕朝貢之使。奈何又為采訪之議,徒延歲月,於事無益。」廣東巡按丁楷亦附會具奏,廷議從之。十年令其使臣齎敕往,自是遂為故事,其國貢使亦不常至。

嘉靖二十二年遣王叔沙不登古魯來貢,訴數為安南侵擾,道阻難歸。乞遣官護送還國,報可。

其國無霜雪,四時皆似夏,草木常青。民以漁為業,無二麥,力穡者少,故收獲薄。國人皆食檳榔,終日不離口。不解朔望,但以月生為初,月晦為盡,不置閏。分晝夜為十更,非日中不起,非夜分不臥,見月則飲酒、歌舞為樂。無紙筆,用羊皮槌薄熏黑,削細竹蘸白灰為字,狀若蚯蚓。有城郭甲兵,人性狠而狡,貿易多不平。戶皆北向,民居悉覆茅簷,高不得過三尺。部領分差等,門高卑亦有限。飲食穢污,魚非腐爛不食,釀不生蛆不為美。人體黑,男蓬頭,女椎結,俱跣足。

王,瑣里人,崇釋教。歲時采生人膽入酒中,與家人同飲,且以浴身,曰「通身是膽」。其國人采以獻王,又以洗象目。每伺人於道,出不意急殺之,取膽以去。若其人驚覺,則膽已先裂,不足用矣。置眾膽於器,華人膽輒居上,故尤貴之。五六月間,商人出,必戒備。王在位三十年,則避位入深山,以兄弟子姪代,而己持齋受戒,告於天曰:「我為君無道,願狼虎食我,或病死。」居一年無恙,則復位如初。國中呼為「昔嚟馬哈剌」,乃至尊至聖之稱也。

國不甚富,惟犀象最多。烏木、降香,樵以為薪。棋柟香獨產其地一山,酋長遣人守之,民不得采,犯者至斷手。

有鱷魚潭,獄疑不決者,令兩造騎牛過其旁,曲者,魚輒躍而食之,直者,即數往返,不食也。有尸頭蠻者,一名屍致魚,本婦人,惟無瞳神為異。夜中與人同寢,忽飛頭食人穢物,來即復活。若人知而封其頸,或移之他所,其婦即死。國設厲禁,有而不告者,罪及一家。

賓童龍國,與占城接壤。或言如來入舍衛國乞食,即其地。氣候、草木、人物、風土,大類占城,惟遭喪能持服,葬以僻地,設齋禮佛,婚姻偶合。酋出入乘象或馬,從者百餘人,前後贊唱。民編茅覆屋。貨用金、銀、花布。

有崑崙山,節然大海中,與占城及東、西竺鼎峙相望。其山方廣而高,其海即曰崑崙洋。諸往西洋者,必待順風,七晝夜始得過,故舟人為之諺曰:「上怕七州,下怕崑崙,針迷舵失,人船莫存。」此山無異產。

人皆穴居巢處,食果實魚蝦,無室廬井灶。

真臘,在占城南,順風三晝夜可至。隋、唐及宋皆朝貢。宋慶元中,滅占城而并其地,因改國名曰占臘。元時仍稱真臘。

洪武三年,遣使臣郭徵等齎詔撫諭其國。四年,其國巴山王忽爾那遣使進表,貢方物,賀明年正旦。詔賜《大統曆》及彩幣,使者亦給賜有差。六年進貢。十二年,王參答甘武者持達志遣使來貢,宴賜如前。十三年復貢。十六年遣使齎勘合文冊賜其王。凡國中使至,勘合不符者,即屬矯偽,許縶縛以聞。復遣使賜織金文綺三十二、磁器萬九千。其王遣使來貢。十九年遣行人劉敏、唐敬偕中官齎磁器往賜。明年,敬等還,王遣使貢象五十九、香六萬斤。尋遣使賜其王鍍金銀印,王及妃皆有賜。其王參烈實田比邪甘菩者遣使貢象及方物。明年復貢象二十八、象奴三十四人、番奴四十五人,謝賜印之恩。二十二年三貢。明年復貢。

永樂元年,遣行人蔣賓興、王樞以即位詔諭其國。明年,王參烈婆田比牙遣使來朝,貢方物。初,中官使真臘,有部卒三人潛遁,索之不得,王以其國三人代之,至是引見。帝曰:「華人自逃,於彼何預而責償?且語言不通,風土不習,吾焉用之?」命賜衣服及道里費,遣還。三年遣使來貢,告故王之喪。命鴻臚序班王孜致祭,給事中畢進、中官王琮齎詔封其嗣子參烈昭平牙為王。進等還,嗣王遣使偕來謝恩。六年、十二年再入貢。使者以其國數被占城侵擾,久留不去。帝遣中官送之還,並敕占城王罷兵修好。十五年、十七年並入貢。宣德、景泰中,亦遣使入貢。自後不常至。

其國城隍周七十餘里,幅員廣數千里。國中有金塔、金橋、殿宇三十餘所。王歲時一會,羅列玉猿、孔雀、白象、犀牛於前,名曰百塔洲。盛食以金盤、金椀,故有「富貴真臘」之諺。民俗富饒。天時常熱,不識霜雪,禾一歲數稔。男女椎結,穿短衫,圍梢布。刑有劓、刖、刺配,盜則去手足。番人殺唐人罪死;唐人殺番人則罰金,無金則鬻身贖罪。唐人者,諸番呼華人之稱也,凡海外諸國盡然。婚嫁,兩家俱八日不出門,晝夜燃燈。人死置於野,任烏鳶食,俄頃食盡者,謂為福報。居喪,但髡其髮,女子則額上剪髮如錢大,曰用此報親。文字以麂鹿雜皮染黑,用粉為小條畫於上,永不脫落。以十月為歲首,閏悉用九月。夜分四更。亦有曉天文者,能算日月薄蝕。其地謂儒為班詰,僧為苧姑,道為八思。班詰不知讀何書,由此入仕者為華貫。先時項掛一白線以自別,既貴曳白如故。俗尚釋教,僧皆食魚、肉,或以供佛,惟不飲酒。其國自稱甘孛智,後訛為甘破蔗,萬曆後又改為柬埔寨。

暹羅,在占城西南,順風十晝夜可至,即隋、唐赤土國。後分為羅斛、暹二國。暹土瘠不宜稼,羅斛地平衍,種多獲,暹仰給焉。元時,暹常入貢。其後,羅斛強,併有暹地,遂稱暹羅斛國。

洪武三年,命使臣呂宗俊等齎詔諭其國。四年,其王參烈昭田比牙遣使奉表,與宗俊等偕來,貢馴象、六足龜及方物,詔賜其王錦綺及使者幣帛有差。已,復遣使賀明年正旦,詔賜《大統曆》及綵幣。五年貢黑熊、白猿及方物。明年復來貢。其王之姊參烈思寧別遣使進金葉表,貢方物於中宮,卻之。已而其姊復遣使來貢,帝仍卻之,而宴賚其使。時其王懦而不武,國人推其伯父參烈寶田比邪思里哆囉祿主國事,遣使來告,貢方物,宴賚如制。已而新王遣使來貢、謝恩,其使者亦有獻,帝不納。已,遣使賀明年正旦,貢方物,且獻本國地圖。

七年,使臣沙里拔來貢。言去年舟次烏豬洋,遭風壞舟,飄至海南,賴官司救護,尚存飄餘兜羅綿、降香、蘇木諸物進獻,廣東省臣以聞。帝怪其無表,既言舟覆,而方物乃有存者,疑其為番商,命卻之。諭中書及禮部臣曰:「古諸侯於天子,比年一小聘,三年一大聘。九州之外,則每世一朝,所貢方物,表誠敬而已。惟高麗頗知禮樂,故令三年一貢。他遠國,如占城、安南、西洋瑣里、爪哇、浡泥、三佛齊、暹羅斛、真臘諸國,入貢既頻,勞費太甚。今不必復爾,其移牒諸國俾知之。」然而來者不止。其世子蘇門邦王昭祿群膺亦遣使上箋於皇太子,貢方物。命引其使朝東宮,宴賚遣之。八年再入貢。其舊明臺王世子昭孛羅局亦遣使奉表朝貢,宴賚如王使。

十年,昭祿群膺承其父命來朝。帝喜,命禮部員外郎王恒等齎詔及印賜之,文曰「暹羅國王之印」,並賜世子衣幣及道里費。自是,其國遵朝命,始稱暹羅;比年一貢,或一年兩貢。至正統後,或數年一貢云。

十六年,賜勘合文冊及文綺、磁器,與真臘等。二十年貢胡椒一萬斤、蘇木一萬斤。帝遣官厚報之。時溫州民有市其沉香諸物者,所司坐以通番,當棄市。帝曰:「溫州乃暹羅必經之地,因其往來而市之,非通番也。」乃獲宥。二十一,年貢象三十、番奴六十。二十二年,世子昭祿群膺遣使來貢。二十三,年貢蘇木、胡椒、降香十七萬斤。

二十八年,昭祿群膺遣使朝貢,且告父喪。命中官趙達等往祭,敕世子嗣王位,賜賚有加。諭曰:「朕自即位以來,命使出疆,周於四維,足履其境者三十六,聲聞於耳者三十一,風殊俗異。大國十有八,小國百四十九,較之於今,暹羅最近。邇者使至,知爾先王已逝。王紹先三之緒,有道於邦家,臣民懽懌。茲特遣人錫命,王其罔失法度,罔淫於樂,以光前烈。欽哉。」成祖即位,詔諭其國。永樂元年賜其王昭祿群膺哆囉諦剌駝紐鍍金銀印,其王即遣使謝恩。六月,以上高皇帝尊謚,遣官頒詔,有賜。八月復命給事中王哲、行人成務賜其王錦綺。九月命中官李興等齎敕,勞賜其王,其文武諸臣並有賜。

二年有番船飄至福建海岸,詰之,乃暹羅與琉球通好者。所司籍其貨以聞,帝曰:「二國修好,乃甚美事,不幸遭風,正宜憐惜,豈可因以為利。所司其治舟給粟,俟風便遣赴琉球。」是月,其王以帝降璽書勞賜,遣使來謝,貢方物。賜齎有加,並賜《列女傳》百冊。使者請頒量衡為國永式,從之。

先是,占城貢使返,風飄其舟至彭亨,暹羅索取其使,羈留不遣。蘇門答剌及滿剌加又訴暹羅恃強發兵奪天朝所賜印誥。帝降敕責之曰:「占城、蘇門答剌、滿剌加與爾俱受朝命,安得逞威拘其貢使,奪其誥印。天有顯道,福善禍淫,安南黎賊可為鑒戒。其即返占城使者,還蘇門答剌、滿剌加印誥。自今奉法循理,保境睦鄰,庶永享太平之福。」時暹羅所遣貢使,失風飄至安南,盡為黎賊所殺,止餘孛黑一人。後官軍征安南,獲之以歸。帝憫之,六年八月命中官張原送還國,賜王幣帛,令厚恤被殺者之家。九月,中官鄭和使其國,其王遣使貢方物,謝前罪。

七年,使來祭仁孝皇后,命中官告之几筵。時奸民何八觀等逃入暹羅,帝命使者還告其主,毋納逋逃。其王即奉命遣使貢馬及方物,并送八觀等還,命張原齎敕幣獎之。十年命中官洪保等往賜幣。

十四年,王子三賴波羅摩剌答刂的賴遣使告父之喪。命中官郭文往祭,別遣官齎詔封其子為王,賜以素錦、素羅,隨遣使謝恩。十七年命中官楊敏等護歸。以暹羅侵滿剌加,遣使責令輯睦,王復遣使謝罪。宣德八年,王悉里麻哈賴遣使朝貢。

初,其國陪臣柰三鐸等貢舟次占城新州港,盡為其國人所掠。正統元年,柰三鐸潛附小舟來京,訴占城劫掠狀。帝命召占城使者與相質。使者無以對,乃敕占城王,令盡還所掠人物。已,占城移咨禮部言:「本國前歲遣使往須文達那,亦為暹羅賊人掠去,必暹羅先還所掠,本國不敢不還。」三年,暹羅貢使又至,賜敕曉以此意,令亟還占城人物。十一年,王思利波羅麻那惹智剌遣使入貢。

景泰四年,命給事中劉洙、行人劉泰祭其故王波羅摩剌答刂的賴,封其嗣子把羅蘭米孫剌為王。天順元年賜其貢使鈒花金帶。六年,王孛剌藍羅者直波智遣使朝貢。

成化九年,貢使言天順元年所頒勘合,為蟲所蝕,乞改給,從之。十七年,貢使還,至中途竊買子女,且多載私鹽,命遣官戒諭諸番。先是,汀州人謝文彬,以販鹽下海,飄入其國,仕至坤岳,猶天朝學士也。後充使來朝,貿易禁物,事覺下吏。

十八年遣使朝貢,且告父喪,命給事中林霄、行人姚隆往封其子國隆勃剌略坤息剌尤地為王。弘治十年入貢。時四夷館無暹羅譯字官,閣臣徐溥等請移牒廣東,訪取能通彼國言語文字者,赴京備用,從之。正德四年,暹羅船有飄至廣東者,市舶中官熊宣與守臣議,稅其物供軍需。事聞,詔斥宣妄攬事柄,撤還南京。十年進金葉表朝貢,館中無識其字者。閣臣梁儲等請選留其使一二人入館肄習,報可。嘉靖元年,暹羅、占城貨船至廣東。市舶中官牛榮縱家人私市,論死如律。三十二年遣使貢白象及方物,象死於途,使者以珠寶飾其牙,盛以金盤,并尾來獻。帝嘉其意,厚遣之。

隆慶中,其鄰國東蠻牛求婚不得,慚怒,大發兵攻破其國。王自經,擄其世子及天朝所賜印以歸。次子嗣位,奉表請印,予之。自是為東蠻牛所制,嗣王勵志復仇。萬歷間,敵兵復至,王整兵奮擊,大破之,殺其子,餘眾宵遁,暹羅由是雄海上。移兵攻破真臘,降其王。從此歲歲用兵,遂霸諸國。

六年遣使入貢。二十年,日本破朝鮮,暹羅請潛師直搗日本,牽其後。中樞石星議從之,兩廣督臣蕭彥持不可,乃已。其後,奉貢不替。崇禎十六年猶入貢。

其國,周千里,風俗勁悍,習於水戰。大將用聖鐵裹身,刀矢不能入。聖鐵者,人腦骨也。王,瑣里人。官分十等。自王至庶民,有事皆決於其婦。其婦人志量,實出男子上。婦私華人,則夫置酒同飲,恬不為怪,曰:「我婦美,而為華人所悅也。」崇信釋教,男女多為僧尼,亦居菴寺,持齋受戒。衣服頗類中國。富貴者,尤敬佛,百金之產,即以其半施之。氣候不正,或寒或熱,地卑濕,人皆樓居。男女椎結,以白布裹首。富貴者死,用水銀灌其口而葬之。貧者則移置海濱,即有群鴉飛啄,俄頃而盡,家人拾其骨號泣而棄之於海,謂之鳥葬。亦延僧設齋禮佛。交易用海。是年不用,則國必大疫。其貢物,有象、象牙、犀角、孔雀尾、翠羽、龜筒、六足龜、寶石、珊瑚、片腦、米腦、糠腦、腦油、腦柴、薔薇水、碗石、丁皮、阿魏、紫梗、藤竭、藤黃、硫黃、沒藥、烏爹泥、安息香、羅斛香、速香、檀香、黃熟香、降真香、乳香、樹香、木香、丁香、烏香、胡椒、蘇木、肉豆蔻、白豆蔻、蓽茇、烏木、大楓子及撒哈剌、西洋諸布。其國有三寶廟,祀中官鄭和。

爪哇在占城西南。元世祖時,遣使臣孟琪往,黥其面。世祖大舉兵伐之,破其國而還。

洪武二年,太祖遣使以即位詔諭其國。其使臣先奉貢於元,還至福建而元亡,因入居京師。太祖復遣使送之還,且賜以《大統歷》。三年以平定沙漠頒詔曰:「自古為天下主者,視天地所覆載,日月所照臨,若遠若近,生人之類,莫不欲其安土而樂生。然必中國安,而後四方萬國順附。邇元君妥懽帖木兒,荒淫昏弱,志不在民。天下英雄,分裂疆宇。朕憫生民之塗炭,興舉義兵,攘除亂略。天下軍民共尊朕居帝位,國號大明,建元洪武。前年克取元都,四方底定。占城、安南、高麗諸國,俱來朝貢。今年遣將北征,始知元君已沒,獲其孫買的里八刺,封為崇禮侯。朕仿前代帝王,治理天下,惟欲中外人民,各安其所。又慮諸蕃僻在遠方,未悉朕意,故遣使者往諭,咸使聞知。」九月,其王昔里八達剌蒲遣使奉金葉表來朝,貢方物,宴賚如禮。

五年又遣使隨朝使常克敬來貢,上元所授宣敕三道。八年又貢。十年,王八達那巴那務遣使朝貢。其國又有東、西二王,東蕃王勿院勞網結,西蕃王勿勞波務,各遣使朝貢。天子以其禮意不誠,詔留其使,已而釋還之。十二年,王八達那巴那務遣使朝貢。明年又貢。時遣使賜三佛齊王印綬,爪哇誘而殺之。天子怒,留其使月餘,將加罪,已,遣還,賜敕責之。十四年遣使貢黑奴三百人及他方物。明年又貢黑奴男女百人、大珠八顆、胡椒七萬五千斤。二十六年再貢。明年又貢。

成祖即位,詔諭其國。永樂元年又遣副使聞良輔、行人甯善,賜其王絨、綿、織金文綺、紗羅。使者既行,其西王都馬板遣使入賀,復命中官馬彬等賜以鍍金銀印。西王遣使謝賜印,貢方物。而東王孛令達哈亦遣使朝貢,請印,命遣官賜之。自後,二王並貢。

三年遣中官鄭和使其國。明年,西王與東王構兵,東王戰敗,國被滅。適朝使經東王地,部卒入市,西王國人殺之,凡百七十人。西王懼,遣使謝罪。帝賜敕切責之,命輸黃金六萬兩以贖。六年再遣鄭和使其國。西王獻黃金萬兩,禮官以輸數不足,請下其使於獄。帝曰:「朕於遠人,欲其畏罪而已,寧利其金耶?」悉捐之。自後,比年一貢,或間歲一貢,或一歲數貢。中官吳賓、鄭和先後使其國。時舊港地有為爪哇侵據者,滿剌加國王矯朝命索之。帝乃賜敕曰:「前中官尹慶還,言王恭待敕使,有加無替。比聞滿剌加國索舊港之地,王甚疑懼。朕推誠待人,若果許之,必有敕諭,王何疑焉。小人浮詞,慎勿輕聽。」十三年,其王改名揚惟西沙,遣使謝恩,貢方物。時朝使所攜卒有遭風飄至班卒兒國者,爪哇人珍班聞之,用金贖還,歸之王所。十六年,王遣使朝貢,因送還諸卒。帝嘉之,賜敕獎王,並優賜珍班。自是,朝貢使臣大率每歲一至。

正統元年,使臣馬用良言:「先任八諦來朝,蒙恩賜銀帶。今為亞烈,秩四品,乞賜金帶。」從之。閏六月遣古里、蘇門答剌、錫蘭山、柯枝、天方、加異勒、阿丹、忽魯謨斯、祖法兒、甘巴里、真臘使臣偕爪哇使臣郭信等同往。賜爪哇敕曰:「王自我先朝,修職勿怠。朕今嗣服,復遣使來朝,意誠具悉。宣德時,有古里等十一國來貢,今因王使者歸,令諸使同往。王其加意撫颻,分遣還國,副朕懷遠之忱。」五年,使臣回,遭風溺死五十六人,存者八十三人,仍返廣東。命所司廩給,俟便舟附歸。

八年,廣東參政張琰言:「爪哇朝貢頻數,供億費煩,敝中國以事遠人,非計。」帝納之。其使還,賜敕曰:「海外諸邦,並三年一貢。王亦宜體恤軍民,一遵此制。」十一年復三貢,後乃漸稀。

景泰三年,王巴剌武遣使朝貢。天順四年,王都馬班遣使入貢。使者還至安慶,酗酒,與入貢番僧斗,僧死者六人。禮官請治伴送行人罪,使者敕國王自治,從之。成化元年入貢。弘治十二年,貢使遭風舟壞,止通事一舟達廣東。禮官請敕所司,量予賜賚遣還,其貢物仍進京師,制可。自是貢使鮮有至者。

其國近占城,二十晝夜可至。元師西征,以至元二十九年十二月發泉州,明年正月即抵其國,相去止月餘。宣德七年入貢,表書「一千三百七十六年」,蓋漢宣帝元康元年,乃其建國之始也。地廣人稠。性兇悍,男子無少長貴賤皆佩刀,稍忤輒相賊,故其甲兵為諸蕃之最。字類瑣里,無紙筆,刻於茭曌葉。氣候常似夏,稻歲二稔。無几榻匕箸。人有三種:華人流寓者,服食鮮華;他國賈人居久者,亦尚雅潔;其本國人最污穢,好啖蛇蟻蟲蚓,與犬同寢食,狀黝黑,猱頭赤腳。崇信鬼道。殺人者避之三日即免罪。父母死,舁至野,縱犬食之;不盡,則大戚,燔其餘。妻妾多燔以殉。

其國一名莆家龍,又曰下港,曰順塔。萬曆時,紅毛番築土庫於大澗東,佛郎機築於大澗西,歲歲互市。中國商旅亦往來不絕。其國有新村,最號饒富。中華及諸番商舶,輻輳其地,寶貨填溢。其村主即廣東人,永樂九年自遣使表貢方物。

闍婆,古曰闍婆達。宋元嘉時,始朝中國。唐曰訶陵,又曰社婆,其王居闍婆城,宋曰闍婆,皆入貢。洪武十一年,其王摩那駝喃遣使奉表,貢方物,其後不復至。或曰爪哇即闍婆。然《元史爪哇傳》不言,且曰:「其風俗、物產無所考。」太祖時,兩國並時入貢,其王之名不同。或本為二國,其後為爪哇所滅,然不可考。

蘇吉丹,爪哇屬國,後訛為思吉港。國在山中,止數聚落。酋居吉力石。其水潏,舟不可泊。商船但往饒洞,其地平衍,國人皆就此貿易。其與國有思魯瓦及豬蠻。豬蠻多盜,華人鮮至。

碟里,近爪哇。永樂三年遣使附其使臣來貢。其地尚釋教,俗淳少訟,物產甚薄。

日羅夏治,近爪哇。永樂三年遣使附其使臣入貢。國小,知種藝,無盜賊。亦尚釋教,所產止蘇木、胡椒。

三佛齊,古名乾陀利。劉宋孝武帝時,常遣使奉貢。梁武帝時數至。宋名三佛齊,修貢不絕。

洪武三年,太祖遣行人趙述詔諭其國。明年,其王馬哈剌札八剌卜遣使奉金葉表,隨入貢黑熊、火雞、孔雀、五色鸚鵡、諸香、苾布、兜羅被諸物。詔賜《大統曆》及錦綺有差。戶部言其貨舶至泉州,宜徵稅,命勿徵。

六年,王怛麻沙那阿者遣使朝貢,又一表賀明年正旦。時其國有三王。七年,王麻那哈寶林邦遣使來貢。八年正月復貢。九月,王僧伽烈宇蘭遣使,隨招諭拂菻國朝使入貢。

九年,怛麻沙那阿者卒,子麻那者巫里嗣。明年遣使貢犀牛、黑熊、火雞、白猴、紅緣鸚鵡、龜筒及丁香、米腦諸物。使者言:「嗣子不敢擅立,請命於朝。」天子嘉其義,命使臣齎印,敕封為三佛齊國王。時爪哇強,已威服三佛齊而役屬之,聞天朝封為國王與己埒,則大怒,遣人誘朝使邀殺之。天子亦不能問罪,其國益衰,貢使遂絕。

三十年,禮官以諸蕃久缺貢,奏聞。帝曰:「洪武初,諸蕃貢使不絕。邇者安南、占城、真臘、暹羅、爪哇、大琉球、三佛齊、浡泥、彭亨、百花、蘇門答剌、西洋等三十國,以胡惟庸作亂,三佛齊乃生間諜,紿我使臣至彼。爪哇王聞知,遣人戒飭,禮送還朝。由是商旅阻遏,諸國之意不通。惟安南、占城、真臘、暹羅、大琉球朝貢如故,大琉球且遣子弟入學。凡諸蕃國使臣來者,皆以禮待之。我視諸國不薄,未知諸國心若何。今欲遣使爪哇,恐三佛齊中途沮之。聞三佛齊本爪哇屬國,可述朕意,移咨暹羅,俾轉達爪哇。」於是部臣移牒曰:「自有天地以來,即有君臣上下之分,中國四裔之防。我朝混一之初,海外諸蕃,莫不來享。豈意胡惟庸謀亂,三佛齊遂生異心,紿我信使,肆行巧詐。我聖天子一以仁義待諸蕃,何諸蕃敢背大恩,失君臣之禮。倘天子震怒,遣一偏將將十萬之師,恭行天罰,易如覆手,爾諸蕃何不思之甚。我聖天子嘗曰:『安南、占城、真臘、暹羅、大琉球皆修臣職,惟三佛齊梗我聲教。彼以蕞爾之國,敢倔強不服,自取滅亡。』爾暹羅恪守臣節,天朝眷禮有加,可轉達爪哇,令以大義告諭三佛齊,誠能省愆從善,則禮待如初。」時爪哇已破三佛齊,據其國,改其名曰舊港,三佛齊遂亡。國中大亂,爪哇亦不能盡有其地,華人流寓者往往起而據之。有梁道明者,廣州南海縣人,久居其國,『閩、粵軍民泛海從之者數千家,推道明為首,雄視一方。會指揮孫鉉使海外,遇其子,挾與俱來。

永樂三年,成祖以行人譚勝受與道明同邑,命偕千戶楊信等齎敕招之。道明及其黨鄭伯可隨入朝,貢方物,受賜而還。

四年,舊港頭目陳祖義遣子士良,道明遣從子觀政並來朝。祖義,亦廣東人,雖朝貢,而為盜海上,貢使往來者苦之。五年,鄭和自西洋還,遣人招諭之。祖義詐降,潛謀邀劫。有施進卿者,告於和。祖義來襲被擒,獻於朝,伏誅。時進卿適遣婿丘彥誠朝貢,命設舊港宣慰司,以進卿為使,錫誥印及冠帶。自是,屢入貢。然進卿雖受朝命,猶服屬爪哇,其地狹小,非故時三佛齊比也。二十二年,進卿子濟孫告父訃,乞嗣職,許之。洪熙元年遣使入貢,訴舊印為火毀,帝命重給。其後,朝貢漸稀。

嘉靖末,廣東大盜張璉作亂,官軍已報克獲。萬曆五年商人詣舊港者,見璉列肆為蕃舶長,漳、泉人多附之,猶中國市舶官云。

其地為諸蕃要會,在爪哇之西,順風八晝夜可至。轄十五洲,土沃宜稼。語云:「一年種穀,三年生金。」言收獲盛而貿金多也。俗富好淫。習於水戰,鄰國畏之。地多水,惟部領陸居,庶民皆水居。編筏築室,系之於樁。水漲則筏浮,無沉溺患。欲徙則拔樁去之,不費財力。下稱其上曰詹卑,猶國君也。後大酋所居,即號詹卑國,改故都為舊港。初本富饒,自爪哇破滅,後漸致蕭索,商舶鮮至。其他風俗、物產,具詳《宋史》。


\end{pinyinscope}