\article{列傳第五十二}

\begin{pinyinscope}
○鄒緝鄭維桓柯暹弋謙黃驥黃澤孔友諒范濟聊讓郭佑胡仲倫華敏賈斌左鼎練綱曹凱許仕達劉煒尚褫單宇姚顯楊浩張昭賀煬

高瑤虎臣

鄒緝,字仲熙,吉水人。洪武中舉明經,授星子教諭。建文時入為國子助教。成祖即位,擢翰林侍講。立東宮,兼左中允,屢署國子監事。

永樂十九年,三殿災,詔求直言,緝上疏曰:

陛下肇建北京,焦勞聖慮,幾二十年。工大費繁,調度甚廣,冗官蠶食,耗費國儲。工作之夫,動以百萬,終歲供役,不得躬親田畝以事力作。猶且徵求無藝,至伐桑棗以供薪,剝桑皮以為楮。加之官吏橫徵,日甚一日。如前歲買辦顏料,本非土產,動科千百。民相率斂鈔,購之他所。大青一斤,價至萬六千貫。及進納,又多留難,往復展轉,當須二萬貫鈔,而不足供一柱之用。其後既遣官採之產所,而買辦猶未止。蓋緣工匠多派牟利,而不顧民艱至此。

夫京師天下根本。人民安則京師安,京師安則國本固而天下安。自營建以來,工匠小人假託威勢,驅迫移徙,號令方施,廬舍已壞。孤兒寡婦哭泣叫號,倉皇暴露,莫知所適。遷移甫定,又復驅令他徒,至有三四徙不得息者。及其既去,而所空之地,經月逾時,工猶未及。此陛下所不知,而人民疾怨者也。

貪官污吏,遍布內外,剝削及於骨髓。朝廷每遣一人,即是其人養活之計。虐取苛求,初無限量。有司承奉,惟恐不及。間有廉彊自守、不事干媚者,輒肆讒毀,動得罪譴,無以自明。是以使者所至,有司公行貨賂,剝下媚上,有同交易。夫小民所積幾何,而內外上下誅求如此。

今山東、河南、山西、陜西水旱相仍,民至剝樹皮掘草根以食。老幼流移,顛踣道路,賣妻鬻子以求茍活。而京師聚集僧道萬餘人,日耗廩米百餘石,此奪民食以養無用也。

至報效軍士,朝廷厚與糧賜。及使就役,乃驕傲橫恣,閑遊往來。此皆姦詭之人,懼還原伍,假此規避,非真有報效之心也。

朝廷歲令天下織錦、鑄錢,遣內官買馬外蕃,所出常數千萬,而所取曾不能一二。馬至雖多,類皆駑下。責民牧養,騷擾殊甚。及至死傷,輒令賠補。馬戶貧困,更鬻妻子。此尤害之大者。

漠北降人,賜居室,盛供帳,意欲招其同類也。不知來者皆懷窺覘,非真遠慕王化,甘去鄉士。宜求來朝之後,遣歸本國,不必留為後日子孫患。

至宮觀禱祠之事,有國者所當深戒。古人有言,淫祀無福。況事無益以害有益,蠹財妄費者乎!凡此數事,皆下失民心,上違天意。怨讟之興,實由於此。

夫奉天殿者,所以朝群臣,發號令,古所謂明堂也。而災首及焉,非常之變也。非省躬責己,大布恩澤,改革政化,疏滌天下窮困之人,不能回上天譴怒。前有監生生員,以單丁告乞侍親,因而獲罪遣戍者,此實有虧治體。近者大赦,法司執滯常條,當赦者尚復拘繫。並乞重加湔洗,蠲除租賦,一切勿征。有司百官全其廩祿,拔簡賢才,申行薦舉,官吏貪贓蠹政者,核其罪而罷黜之。則人心歡悅,和氣可臻,所以保安宗社,為國家千萬年無窮之基,莫有大於此者矣。

且國家所恃以久長者,惟天命人心,而天命常視人心為去留。今天意如此,不宜勞民。當還都南京,奉謁陵廟,告以災變之故。保養聖躬休息於無為。毋聽小人之言,復有所興作,以誤陛下於後也。

書奏,不省。

時三殿初成,帝方以定都詔天下,忽罹火災,頗懼,下詔求直言。及言者多斥時政,帝不懌,而大臣復希旨詆言者。帝於是發怒,謂言事者謗訕,下詔嚴禁之,犯者不赦。侍讀李時勉、侍講羅汝敬俱下獄;御史鄭維桓、何忠、羅通、徐瑢,給事中柯暹俱左官交阯。惟緝與主事高公望、庶吉士楊復得無罪。是年冬,緝進右庶子兼侍講。明年九月卒於官。

緝博極群書,居官勤慎,清操如寒士。子循,宣德中為翰林待詔,請贈父母。帝諭吏部曰:「曩皇祖征沙漠,朕守北京,緝在左右,陳說皆正道,良臣也,其予之。」

鄭維桓,慈谿人。永樂十三年進士。出知交阯南清州,卒。柯暹,池州建德人。由鄉舉出知交阯酹州。累官浙江、雲南按察使。

弋謙,代州人。永樂九年進士。除監察御史。出按江西,言事忤旨,貶峽山知縣。復坐事免歸。

仁宗在東宮,素知謙骨鯁。及嗣位,召為大理少卿。直陳時政,言官吏貪殘,政事多非洪武之舊,及有司誅求無藝。帝多採納。既復言五事,詞太激,帝乃不懌。尚書呂震、吳中,侍郎吳廷用,大理卿虞謙等因劾謙誣罔,都御史劉觀令眾御史合糾謙。帝召楊士奇等言之,士奇對曰:「謙不諳大體,然心感超擢恩,欲圖報耳。主聖則臣直,惟陛下優容之。」帝乃不罪謙。然每見謙,詞色甚厲。士奇從容言:「陛下詔求直言,謙言不當,觸怒。外廷悚惕,以言為戒。今四方朝覲之臣皆集闕下,見謙如此,將謂陛下不能容直言。」帝惕然曰:「此固朕不能容,亦呂震輩迎合以益朕過,自今當置之。」遂免謙朝參,令專視司事。

未幾,帝以言事者益少,復召士奇曰:「朕怒謙矯激過實耳,朝臣遂月餘無言。爾語諸臣,白朕心。」士奇曰:「臣空言不足信,乞親降璽書。」遂令就榻前書敕引過曰:「朕自即位以來,臣民上章以數百計,未嘗不欣然聽納。茍有不當,不加譴訶,群臣所共知也。間者,大理少卿弋謙所言,多非實事,群臣迎合朕意,交章奏其賣直,請置諸法。朕皆拒而不聽,但免謙朝參。而自是以來,言者益少。今自去冬無雪,春亦少雨,陰陽愆和,必有其咎,豈無可言?而為臣者,懷自全之計,退而默默,何以為忠?朕於謙一時不能含容,未嘗不自愧咎。爾群臣勿以前事為戒,於國家利弊、政令未當者,直言勿諱。謙朝參如故。」時中官採木四川,貪橫。帝以謙清直,命往治之。擢謙副都御史,賜鈔以行,遂罷採木之役。

宣德初,交阯右布政戚遜以貪淫黜,命謙往代。王通棄交阯,謙亦論死。正統初,釋為民。土木之變,謙布衣走闕下,薦通及甯懋、阮遷等十三人,皆奇才可用。眾議以通副石亨,謙請專任通,事遂寢。廷臣以謙負重名,奏留之,亦不報。景泰二年復至京,疏薦通等,不納。罷歸,未幾卒。仁宗性寬大,容直言,謙以故得無罪,反責呂震等。而黃驥言西域事,帝亦誚震而行其言。

驥,全州人。洪武中,中鄉舉。為沙縣教諭。永樂時擢禮科給事中,常三使西域。仁宗初,上疏言:「西域貢使多商人假托,無賴小人投為從者,乘傳役人,運貢物至京師,賞賚優厚。番人慕利,貢無虛月,致民失業妨農。比其使還,多齎貨物,車運至百餘輛。丁男不足,役及婦女。所至辱驛官,鞭夫隸,無敢與較者。乞敕陜西行都司,惟哈密諸國王遣使入貢者,許令來京,止正副使得乘驛馬,陜人庶少蘇。至西域所產,惟馬切邊需,應就給甘肅軍士。其岡砂、梧桐、鹼之類,皆無益國用,請一切勿受,則來者自稀,浮費益省。」帝以示尚書呂震,且讓之曰:「驥嘗奉使,悉西事。卿西人,顧不悉邪?驥言是,其即議行。」後遷右通政,與李琦、羅汝敬撫諭交阯,不辱命。使還,尋卒。

黃澤,閩縣人。永樂十年進士。擢河南左參政。南陽多流民,拊循使復業。嘗率丁役至北京,周恤備至。久之,調湖廣。仁宗即位,入覲,言時政,多見采。

宣宗立,下詔求言。澤上疏言正心、恤民、敬天、納諫、練兵、重農、止貢獻、明賞罰、遠嬖倖、汰冗官十事。其言遠嬖倖曰:「刑餘之人,其情幽陰,其慮險譎。大姦似忠,大詐似信,大巧似愚。一與之親,如飲醇酒,不知其醉;如噬甘臘,不知其毒。寵之甚易,遠之甚難。是以古者宦寺不使典兵干政,所以防患於未萌也。涓涓弗塞,將為江河。此輩宜一切疏遠,勿使用事。漢、唐已事,彰彰可監。」當成祖時,宦官稍稍用事,宣宗浸以親幸。澤於十事中此為尤切。帝雖嘉歎,不能用也。其後設內書堂,而中人多通書曉文義。宦寺之盛,自宣宗始。

宣德三年擢浙江布政使。復上言平陽、麗水等七縣銀冶宜罷,并請盡罷諸坑冶,語甚切。帝歎息曰:「民困若此,朕何由知?遣官驗視,酌議以聞。」澤在官有政績,然多暴怒,鹽運使丁鎡不避道,撻之,為所奏。巡按御史馬謹亦劾澤九載秩滿,自出行縣,斂白金三千兩償官物,且越境過家。遂逮下獄。正統六年黜為民。初,澤奏金華、台州戶口較洪武時耗減,而歲造弓箭如舊,乞減免。下部議得允,而澤已罷官踰月矣。

孔友諒,長洲人。永樂十六年進士。改庶吉士,出知雙流縣。宣宗初,上言六事:

一曰,守令親民之官,古者不拘資格,必得其人;不限歲月,使盡其力。今居職者多不知撫字之方,而廉幹得民心者,又遷調不常,差遣不一。或因小事連累,朝夕營治,往來道路,日不暇給。乞敕吏部,擇才望素優及久歷京官者任之。諭戒上司,毋擅差遣,假以歲月,責成治效。至遠缺佐貳,多經裁減,獨員居職。或遇事赴京,多委雜職署事,因循茍且,政令無常,民不知畏。今後路遠之缺,常留一正員任事,不得擅離,庶法有常守。

二曰,科舉所以求賢,必名實相副,非徒誇多而已。今秋闈取士動一二百人。弊既多端,僥倖過半。會試下第,十常八九,其登第者,實行或乖。請於開科之歲,詳核諸生行履。孝弟忠信、學業優贍者,乃許入試。庶浮薄不致濫收,而國家得真才之用。

三曰,祿以養廉,祿入過薄,則生事不給。國朝制祿之典,視前代為薄。今京官及方面官稍增俸祿,其餘大小官自折鈔外,月不過米二石,不足食數人。仰事俯育,與道路往來,費安所取資?貪者放利行私,廉者終窶莫訴。請敕戶部勘實天下糧儲,以歲支之餘,量增官俸。仍令內外風憲官,採訪廉潔之吏,重加旌賞。則廉者知勸,貪者知戒。

四曰,古者賦役量土宜,驗丁口,不責所無,不盡所有。今自常賦外,復有和買、採辦諸事。自朝廷視之,不過令有司支官錢平買。而無賴之輩,關通吏胥,壟斷貨物,巧立辨驗、折耗之名,科取數倍,姦弊百端。乞盡停採買,減諸不急務,則國賦有常,民無科擾。

其二事言汰冗員,任風憲,言者多及之,不具載。

宣德八年,命吏部擇外官有文學者六十八人試之,得友諒及進士胡端禎等七人,悉令辦事六科。居二年,皆授給事中,惟友諒未授官而卒。

范濟,元進士。洪武中,以文學舉為廣信知府,坐累謫戍興州。宣宗即位,濟年八十餘矣,詣闕言八事:

其一曰楮幣之法,昉於漢、唐。元造元統交鈔,後又造中統鈔。久而物重鈔輕,公私俱敝,乃造至元鈔與中統鈔兼行。子母相權,新陳通用。又令民間以昏鈔赴平準庫,中統鈔五貫得換至元鈔一貫。又其法日造萬錠,共計官吏俸稍、內府供用若干,天下正稅雜課若干,斂發有方,周流不滯,以故久而通行。太祖皇帝造大明寶鈔。以鈔一貫當白金一兩,民歡趨之。迄今五十餘年,其法稍弊,亦由物重鈔輕所致。願陛下因時變通,重造寶鈔,一準洪武初制,使新舊兼行。取元時所造之數而增損之,審國家度支之數而權衡之。俾鈔少而物多,鈔重而物輕。嚴偽造之條,開倒換之法,推陳出新,無耗無阻,則鈔法流通,永永無弊。

其二曰備邊之道,守險為要。若朔州、大同、開平、宣府、大寧,乃京師之籓垣,邊徼之門戶。士可耕,城可守。宜盛兵防禦,廣開屯田,修治城堡,謹烽火,明斥堠。毋貪小利,毋輕遠求,堅壁清野,使無所得。俟其憊而擊之,得利則止,毋窮追深入。此守邊大要也。

其三曰兵不在多,在於堪戰。比者多發為事官吏人民充軍塞上,非白面書生,則老弱病廢。遇有征行,有力者得免,貧弱者備數。器械不完,糗糧不具。望風股栗,安能效死?今宜選其壯勇,勤加訓練,餘但令乘城擊柝,趨走牙門,庶幾各得其用。

其四曰民病莫甚於勾軍。衛所差官至六七員,百戶差軍旗亦二三人,皆有力交結及畏避徵調之徒,重賄得遣。既至州縣,擅作威福,迫脅里甲,恣為姦私。無丁之家,誅求不已;有丁之戶,詐稱死亡。託故留滯,久而不還。及還,則以所得財物,遍賄官吏,朦朧具覆。究其所取之丁,十不得一,欲軍無缺伍難矣。自今軍士有故,令各衛報都督府及兵部,府、部諜布政、按察司。令府州縣準籍貫姓名,勾取送衛,則差人騷擾之弊自絕。

其五曰洪武中令軍士七分屯田,三分守城,最為善策。比者調度日繁,興造日廣,虛有屯種之名,田多荒蕪。兼養馬、採草、伐薪、燒炭,雜役旁午,兵力焉得不疲、農業焉得不廢?願敕邊將課卒墾荒,限以頃畝,官給牛種,稽其勤惰,明賞罰以示勸懲。則塞下田可盡墾,轉餉益紓,諸邊富實,計無便於此者。

其六曰,學校者,風化之源,人材所自出,貴明體適用,非徒較文藝而已也。洪武中妙選師儒,教養甚備,人材彬彬可觀。邇來士習委靡,立志不弘,執節不固。平居無剛方正大之氣,安望其立朝為名公卿哉!宜選良士為郡縣學官,擇民間子弟性行端謹者為生徒,訓以經史,勉以節行。俟其有成,貢於國學。磨礱砥礪,使其氣充志定,卓然成材,然後舉而用之,以任天下國家事無難矣。

其七曰兵者凶器,聖人不得已而用之。漢高祖解平城之圍,未聞蕭、曹勸以復仇;唐太宗禦突厥於便橋,未聞房、杜勸以報怨。古英君良相不欲疲民力以誇武功,計慮遠矣。洪武初年嘗赫然命將,欲清沙漠。既以饋運不繼,旋即頒師。遂撤東勝衛於大同,塞山西陽武谷口,選將練兵,扼險以待。內修政教,外嚴邊備,廣屯田,興學校,罪貪吏,徙頑民。不數年間,朵兒只巴獻女,伯顏帖木兒、乃兒不花等相繼擒獲,納哈出亦降。此專務內治,不勤遠略之明效也。伏望遠鑒漢、唐,近法太祖,毋以窮兵黷武為快,毋以犁庭掃穴為功。棄捐不毛之地,休養冠帶之民,俾竭力於田桑,盡心於庠序。邊塞絕傷痍之苦,閭里絕呻吟之聲。將無倖功,士無夭閼,遠人自服,荒外自歸。國祚靈長於萬年矣。

其八曰官不在眾,在乎得人。國家承大亂後,因時損益,以府為州,以州為縣。繼又裁併小縣之糧不及俸者,量民數以設官。民多者縣設丞薄,少者知縣、典史而已。其時官無廢事,民不愁勞。今籓、臬二司及府、州、縣官,視洪武中再倍,政愈不理,民愈不寧,姦弊叢生,詐偽滋起。甚有官不能聽斷,吏不諳文移,乃容留書寫之人,在官影射,賄賂公行,獄訟淹滯,皆官冗吏濫所致也。望斷自宸衷,凡內外官吏,並依洪武中員額,冗濫者悉汰,則天工無曠,庶績咸熙,而天下大治矣。

奏上,命廷臣議之。尚書呂震以為文辭冗長,且事多已行,不足采。帝曰:「所言甚有學識,多契朕心,當察其素履以聞。」震乃言:「濟故元進士,曾守郡,坐事戍邊。」帝曰:「惜哉斯人!令久淹行伍,今猶足用。」震曰:「年老矣。」帝曰:「國家用人,正須老成,但不宜任以繁劇。」乃以濟為儒學訓導。

聊讓,蘭州人。肅府儀衛司餘丁也。好學有志尚,明習時務。景帝嗣位,懲王振蒙蔽,大闢言路,吏民皆得上書言事。景泰元年六月,讓詣闕陳數事,其略曰:

邇歲土木繁興,異端盛起,番僧絡驛,污吏縱橫,相臣不正其非,御史不劾其罪,上下蒙蔽,民生凋瘵。狡寇犯邊,上皇播越。陛下枕戈嘗膽之秋,可不拔賢舉能,一新政治乎?昔宗、岳為將,敵國不敢呼名;韓、范鎮邊,西賊聞之破膽。司馬光居相位,強鄰戒勿犯邊。今文武大臣之有威名德望者,宜使典樞要,且延訪智術才能之士,布滿朝廷,則也先必畏服,而上皇可指日還矣。

大臣,陽也;宦寺,陰也。君子,陽也;小人,陰也。近日食地震,陰盛陽微,謫見天地。望陛下總攬乾綱,抑宦寺使不得預政,遏小人俾不得居位,則陰陽順而天變弭矣。天下治亂,在君心邪正。田獵是娛,宮室是侈,宦寺是狎,三者有一,足蠱君心。願陛下涵養克治,多接賢士大夫,少親宦官宮妾,自能革奢靡,戒遊佚,而心無不正矣。

堯立謗木,恐人不言,所以聖;秦除謚法,恐人議己,所以亡。陛下廣從諫之量,旌直言之臣,則國家利弊,閭閻休戚,臣下無所顧忌,而言無不盡矣。蘇子曰:「平居無犯顏敢諫之臣,則臨難必無仗節死義之士。」願陛下恒念是言而審察之。

書奏,帝頗嘉納之。後四年,讓登進士。官知縣卒。

景泰二年,監生郭佑亦上書言兵事,略曰:

逆寇犯順,上皇蒙塵,此千古非常之變,百世必報之仇也。今使臣之來,動以數千,務驕蹇責望於我,而我乃隱忍姑息,致賊勢日張,我氣日索,求和與和,求戰與戰,是和戰之權,不在我而在賊也。願陛下結人心,親賢良,以固國本,廣儲蓄,練將士,以壯國氣。正分定名,裁之以義。如桀驁侵軼,則提兵問罪。使大漠之南,不敢有匹馬闌入,乃可保百年無虞。不然西北力罷,東南財竭,不能一日安枕矣。昨以國用耗乏,謀國大臣欲紓一時之急,令民納粟者賜冠帶。今軍旅稍寧,行之如故。農工商販之徒,不較賢愚,惟財是授。驕親戚,誇鄉里,長非分之邪心。贓污吏罷退為民,欲掩閭黨之恥,納粟納草,冠帶而歸。前以冒貨去職,今以輸貨得官,何以禁貪殘、重名爵?況天下統一,藏富在民,未至大不得已,而舉措如此,是以空乏啟寇心也。章下廷議,格不行。

又有胡仲倫者,雲南鹽課提舉司吏目也。緣事入都,會上皇北狩,也先欲妻以妹,上皇因遣廣寧伯劉安入言於帝。仲倫上疏爭之,言:「今日事不可屈者有七。降萬乘之尊,與諧婚媾,一也。敵假和議,使我無備,二也。必欲為姻,驕尊自大,三也。索金帛,使我坐困,四也。以送駕為名,乘機入犯,五也。逼上皇手詔,誘取邊城,六也。欲求山後之地,七也。稍從其一,大事去矣。曩上皇在位,王振專權。忠諫者死,鯁直者戍;君子見斥,小人驟遷。章奏多決中旨,黑白混淆,邪正倒置。閩、浙之寇方殷,瓦剌之釁大作。陛下宜親賢遠姦,信賞必罰,通上情,達下志。賣國之姦無所投隙,倉卒之變末由發機,朝廷自此尊,天下自此安矣。」帝嘉納焉。

又有華敏者,南京錦衣軍餘也。意氣慷慨,讀書通大義,憤王振亂國,與儕輩言輒裂眥怒詈。景泰三年九月上書曰:

近年以來,內官袁琦、唐受、喜寧、王振專權害政,致國事傾危。望陛下防微杜漸,總攬權綱,為子孫萬世法。不然恐禍稔蕭牆,曹節、侯覽之害,復見於今日。臣雖賤陋,不勝痛哭流涕。謹以虐軍害民十事,為陛下痛切言之。內官家積金銀珠玉,累室兼籝,從何而至?非內盜府藏,則外朘民膏。害一也。怙勢矜寵,占公侯邸舍,興作工役,勞擾軍民。害二也。家人外親,皆市井無籍之子,縱橫豪悍,任意作奸,納粟補官,貴賤淆雜。害三也。建造佛寺,耗費不貲,營一己之私,破萬家之產。害四也。廣置田莊,不入賦稅,寄戶郡縣,不受征徭,阡陌連亙,而民無立錐。害五也。家人中鹽,虛占引數,轉而售人,倍支鉅萬,壞國家法,豪奪商利。害六也。奏求塌房,邀接商旅,倚勢賒買,恃強不償,行賈坐敝,莫敢誰何。害七也。賣放軍匠,名為伴當,俾辦月錢,致內府監局營作乏人,工役煩重,并力不足。害八也。家人貿置物料,所司畏懼,以一科十,虧官損民。害九也。監作所至,非法酷刑,軍匠塗炭,不勝怨酷。害十也。章下禮部,寢不行。

又有賈斌者,商河人,山西都司令史也。亦疏言宦官之害,引漢桓帝、唐文宗、宋徽欽為戒。且獻所輯《忠義集》四卷,採史傳所記直諫盡忠守節之士,而宦官恃寵蠹政,可為鑒戒者附焉,乞命工刊布。禮部以其言當,乞垂鑒納,不必刊行。帝報聞。

左鼎,字周器,永新人。正統七年進士。明年,都御史王文以御史多闕,請會吏部於進士選補。帝從之。尚書王直考鼎及白圭等十餘人,曉諳刑名,皆授御史。而鼎得南京。尋改北,巡按山西。

時英宗北狩,兵荒洊臻。請蠲太原諸府稅糧,停大同轉餉夫,以蘇其困。也先請和,抗言不可。尋以山東、河南饑,遣鼎巡視,民賴以安。律,官吏故勘平人致死者抵罪,時以給事中于泰言,悉得寬貰。鼎言:「小民無知,情貸可也。官吏深文巧詆,與故殺何異?法者,天下之公,不可意為輕重。」自是論如律。

景泰四年疏言:

瓦剌變作,將士無用,由軍政不立。謂必痛懲前弊,乃今又五年矣。貂蟬盈座,悉屬公侯;鞍馬塞途,莫非將帥。民財歲耗,國帑日虛。以天下之大,土地兵甲之眾,曾不能振揚威武,則軍政仍未立也。昔太祖定律令,至太宗,暫許有罪者贖,蓋權宜也。乃法吏拘牽,沿為成例,官吏受枉法財,悉得減贖。骫骳如此,復何顧憚哉。國初建官有常,近始因事增設。主事每司二人,今有增至十人者矣。御史六十人,今則百餘人矣。甚至一部有兩尚書,侍郎亦倍常額,都御史以數十計,此京官之冗也。外則增設撫民、管屯官。如河南參議,益二而為四,僉事益三而為七,此外官之冗也。天下布、按二司各十餘人,乃歲遣御史巡視,復遣大臣巡撫鎮守。夫今之巡撫鎮守,即曩之方面御史也。為方面御史,則合眾人之長而不足;為巡撫鎮守,則任一人之智而有餘。有是理邪?至御史遷轉太驟,當以六年為率。令其通達政事,然後可以治人。巡按所係尤重,毋使初任之員,漫然嘗試。其餘百執事,皆當慎擇而久任之。帝頗嘉納。

未幾,復言:國家承平數十年,公私之積未充。一遇軍興,抑配橫徵,鬻官市爵,率行衰世茍且之政,此司邦計者過也。臣請痛抑末技,嚴禁遊惰,斥異端使歸南畝,裁冗員以省虛糜。開屯田而實邊,料士伍而紓饟。寺觀營造,供佛飯僧,以及不急之工,無益之費,悉行停罷。專以務農重粟為本,而躬行節儉以先之,然後可阜民而裕國也。倘忍不加務,任掊克聚斂之臣行朝三暮四之術,民力已盡而徵發無已,民財已竭而賦斂日增。茍紓目前之急,不恤意外之虞,臣竊懼焉。章下戶部。尚書金濂請解職,帝不許。鼎言亦不盡行。

踰月,以災異,偕同官陳救弊恤民七事。末言:「大臣不乏奸回,宜黜罷其尤,用清政本。」帝善其言,下詔甄別,而大臣辭職並慰留。給事中林聰請明諭鼎等指實劾奏,鼎、聰等乃共論吏部尚書何文淵、刑部尚書俞士悅、工部侍郎張敏、通政使李錫不職狀。錫罷,文淵致仕。

鼎居官清勤,卓有聲譽。御史練綱以敢言名,而鼎尤善為章奏。京師語曰:「左鼎手,練綱口。」自公卿以下咸憚之。

鼎出為廣東右參政。會英宗復位,以郭登言,召為左僉都御史。踰年卒。

練綱,字從道,長洲人。祖則成,洪武時御史。綱舉鄉試,入國子監。歷事都察院。郕王監國,上中興八策。也先將入犯,復言:「和議不可就,南遷不可從,有持此議者,宜立誅。安危所倚,惟於謙、石亨當主中軍,而分遣大臣守九門,擇親王忠孝著聞者,令同守臣勤王。檄陜西守將調番兵入衛。」帝悉從之。

綱有才辨,急功名。都御史陳鎰、尚書俞士悅皆綱同里,念綱數陳是政有聲,且畏其口,遂薦之,授御史。

景泰改元,上時政五事。巡視兩淮鹽政。駙馬都尉趙輝侵利,劾奏之。三年冬,偕同官應詔陳八事,並允行。亡何,復偕同官上言:「吏部推選不公,任情高下,請置尚書何文淵、右侍郎項文曜於理。尚書王直、左侍郎俞山素行本端,為文曜等所罔,均宜按問。」帝雖不罪,終以綱等為直。明年命出贊延綏軍務,自陳名輕責重,乞授僉都御史。帝曰:「遷官可自求耶?」遂寢其命。

初,京師戒嚴,募四方民壯分營訓練,歲久多逃,或赴操不如期,廷議編之尺籍。綱等言:「召募之初,激以忠義,許事定罷遣。今展轉輪操,已孤所望,況其逃亡,實迫寒餒,豈可遽著軍籍。邊方多故,倘更召募,誰復應之?」詔即除前令。

五年巡按福建,與按察使楊玨互訐,俱下吏。謫玨黃州知府,綱邠州判官。久之卒。

曹凱,字宗元,益都人。正統十年進士。授刑科給事中。磊落多壯節。

英宗北征,諫甚力,且曰:「今日之勢,大異澶淵。彼文武忠勇,士馬勁悍。今中貴竊權,人心玩愒。此輩不惟以陛下為孤注,即懷、愍、徽、欽亦何暇恤?」帝不從,乘輿果陷。凱痛哭竟日,聲徹禁庭,與王竑共擊馬順至死。

景泰中,遷左。給事中林聰劾何文淵、周旋,詔宥之。凱上殿力諍,二人遂下吏。時令輸豆得補官,凱爭曰:「近例,輸豆四千石以上,授指揮。彼受祿十餘年,費已償矣,乃令之世襲,是以生民膏血養無功子孫,而彼取息長無窮也。有功者必相謂曰:吾以捐軀獲此,彼以輸豆亦獲此,是朝廷以我軀命等於荏菽,其誰不解體!乞自今惟令帶俸,不得任事傳襲,文職則止原籍帶俸。」帝以為然,命已授者如故,未授者悉如凱議。

福建巡按許仕達與侍郎薛希璉相訐,命凱往勘。用薦,擢浙江右參政。時諸衛武職役軍辦納月錢,至四千五百餘人,以凱言禁止。鎮守都督李信擅募民為軍,糜餉萬餘石,凱劾奏之。信雖獲宥,諸助信募軍者咸獲罪,在浙數年,聲甚著。

初,凱為給事,常劾武清侯石亨。亨得志,修前憾,謫凱衛經歷,卒。

許仕達,歙人。正統十年進士。擢御史。景泰元年四月上疏言災沴數見,請帝痛自修省。帝深納之。未幾,復請於經筵之餘,日延儒臣講論經史。帝亦優詔褒答。巡按福建,劾鎮守中官廖秀,下之獄。秀訐仕達,下鎮守侍郎薛希璉等廉問。會仕達亦劾希璉貪縱,乃命凱及御史王豪往勘。還奏,兩人互有虛實。而耆老數千人乞留仕達,給事中林聰,閩人也,亦為仕達言。乃命留任,且敕希璉勿構郤。仕達厲風紀,執漳州知府馬嗣宗送京師。大理寺劾其擅執,帝以執贓吏不問。期滿當代,耆老詣闕請留,不許。未幾,即以為福建左參政。天順中,歷山東、貴州左、右布政使。

劉煒,字有融,慈谿人。正統四年進士。授南京刑科給事中。副都御史周銓以私憾撻御史。諸御史范霖、楊永與尚褫等十人共劾銓,煒與同官盧祥等復劾之。銓下詔獄,亦訐霖、永及煒、祥等。王振素惡言官,盡逮下詔獄。霖、永坐絞,後減死。他御史或戍或謫。煒、祥事白留任,而銓已先瘐死。煒累進都給事中。

景泰四年,戶部以邊儲不足,奏令罷退官非贓罪者,輸米二十石,給之誥敕。煒等言:「考退之官,多有罷軟酷虐、荒溺酒色、廉恥不立者,非止贓罪已也。賜之誥敕,以何為辭?若但褒其納米,則是朝廷誥敕止直米二十石,何以示天下後世?此由尚書金濂不識大體,有此謬舉。」帝立為已之。山東歲歉,戶部以尚書沈翼習其地民瘼,請令往振。及往,初無方略。煒因劾翼,且言:「其地已有尚書薛希璉、少卿張固鎮撫,又有侍郎鄒幹、都御史王竑振濟,而復益之以冀,所謂『十羊九牧』。乞還冀南京戶部,而專以命希璉等。」從之。平江侯陳豫鎮臨清,事多違制。煒劾之,豫被責讓。

明年,都督黃竑以易儲議得帝眷,奏求霸州、武清縣地。煒等抗章言:「竑本蠻僚,遽蒙重任。怙寵妄干,乞地六七十里,豈盡無主者?請正其罪。」帝宥竑,遣戶部主事黃岡、謝昶往勘。還奏,果民產。戶部再請罪竑,帝卒宥焉。昶官至貴州巡撫,以清慎稱。

煒天順初出為雲南參政,改廣東,分守惠、潮二府。潮有巨寇,招之不服,會兵進剿,誅其魁。改蒞南韶。會大軍征兩廣,以勞瘁卒官。

尚褫,字景福,羅山人。正統四年進士。除行人。上書請毋囚繫大臣。擢南御史。以劾周銓下獄,與他御史皆謫驛丞,得雲南虛仁驛。景泰五年冬因災異上書陳數事,中言:「忠直之士,冒死陳言。執政者格以條例,輕則報罷,重則中傷,是言路雖開猶未開也。釋教盛行,誘煽聾俗,由掌邦禮者畏王振勢,度僧多至此,宜盡勒歸農。」章下禮部,尚書胡濙惡其刺己,悉格不行。量移豐城知縣,為邑豪誣構繫獄,尋得釋。

成化初,大臣會薦,擢湖廣僉事。初有詔,荊、襄流民,許所在附籍。都御史項忠復遣還鄉,督其急,多道死。褫憫之,陳牒巡撫吳琛請進止。琛以報忠,忠怒劾褫。中朝知其意在恤民,卒申令流民聽附籍,不願,乃遣還鄉。褫為僉事十年,所司上其治行,賜誥旌異。致仕卒。

單宇。字時泰,臨川人。正統四年進士。除嵊縣知縣。馭吏嚴。吏欲誣奏宇,宇以聞。坐不并上吏奏,逮下獄。事白,調諸暨。

遭喪服除,待銓京師。適英宗北狩,宇憤中官監軍,諸將不得專進止,致喪師。疏請盡罷之,以重將權。景帝不納。

初,王振佞佛,請帝歲一度僧。其所修大興隆寺,日役萬人,糜帑數十萬,閎麗冠京都。英宗為賜號「第一叢林」,命僧大作佛事,躬自臨幸,以故釋教益熾。至是宇上書言:「前代人君尊奉佛氏,卒致禍亂。近男女出家累百千萬,不耕不織,蠶食民間。營構寺宇,遍滿京邑,所費不可勝紀。請撤木石以建軍營,銷銅鐵以鑄兵仗,罷遣僧尼,歸之民俗,庶皇風清穆,異教不行。」疏入,為廷議所格。復知侯官。

而咸陽姚顯以鄉舉入國學,亦上言:「曩者修治大興隆寺,窮極壯麗,又奉僧楊某為上師,儀從侔王者。食膏梁,被組繡,藐萬乘若弟子。今上皇被留賊庭,乞令前赴瓦剌,化諭也先。誠能奉駕南還,庶見護國之力。不然,佛不足信彰彰矣。」當景泰時,廷臣諫事佛者甚眾,帝卒不能從。而中官興安最用事,佞佛甚於振。請帝建大隆福寺,嚴壯與興隆並。四年三月,寺成,帝剋期臨幸。河東鹽運判官濟寧楊浩切諫,乃止。

宇好學有文名,三為縣,咸以慈惠聞。居侯官,久之卒。

顯後為齊東知縣,移武城,公廉剛正。用巡撫翁世資薦,擢太僕丞。浩初以鄉舉入國學,除官未行,遂抗疏,聲譽籍甚。累官右副都御史,巡撫延綏。

張昭,不知何許人。天順初,為忠義前衛吏。英宗復辟甫數月,欲遣都指揮馬雲等使西洋,廷臣莫敢諫。昭聞之,上疏曰:「安內救民,國家之急務,慕外勤遠,朝廷之末策。漢光武閉關謝西域,唐太宗不受康國內附,皆深知本計者也。今畿輔、山東仍歲災歉,小民絕食逃竄,妻子衣不蔽體,被薦裹席,鬻子女無售者。家室不相完,轉死溝壑,未及埋瘞,已成市臠,此可為痛哭者也。望陛下用和番之費,益以府庫之財,急遣使振恤,庶饑民可救。」奏下公卿博議,言雲等已輕遣。宜籍記所市物俟命。帝命姑已之。

天順三年秋,建安老人賀煬亦上書論時事,言:「今銓授縣令,多年老監生。逮滿九載,年幾七十,茍且貪污。宜擇年富有才能者,其下僚及山林抱德士,亦當推舉。景泰朝,錄先賢顏、孟、程、朱子孫,授以翰林博士,俾之奉祀。然有官無祿,宜班給以昭崇儒之意。黃乾、劉龠、蔡沈、真德秀配祠朱子,亦景泰間從僉事呂昌之請,然未入祝辭,宜增補。預備義倉,本以振貧民,乃豪猾多冒支不償,致廩庾空虛。乞令出粟義民,各疏里內饑民,同有司散放。」未幾,又言:「朝廷建學立師,將以陶熔士類。而師儒鮮積學,草野小夫夤緣津要,初解免園之冊,已廁鶚薦之群。及受職泮林,猥瑣貪饕,要求百故;而授業解惑,莫措一詞。生徒亦往往玩愒歲月,佻達城闕,待次循資,濫升太學。,侵尋老耋,倖博一官。但廠堇身家之謀,無復功名之念。及今不嚴甄選,人材日陋。士習日非矣。」帝善其言,下所司行之。

高瑤,字庭堅,閩縣人。由鄉舉為荊門州學訓導。成化三年五月抗疏陳十事。其一言:「正統己巳之變,先帝北狩,陛下方在東宮,宗社危如一髮。使非郕王繼統,國有長君,則禍亂何由平、鑒輿何由返?六七年間,海宇寧謐,元元樂業,厥功不細。迨先帝復辟,貪天功者遂加厚誣,使不得正其終,節惠隮祀,未稱典禮。望特敕禮官集議,追加廟號,盡親親之恩。」章下,廷議久不決。至十二月始奏:「追崇廟號,非臣下敢擅議,惟陛下裁決。」而左庶子黎淳力爭,謂不當復,且言:「瑤此言有死罪二:一誣先帝為不明,一陷陛下於不孝。臣以謂瑤此舉,非欲尊郕王,特為群邪進用階,必有小人主之者。」帝曰:「景泰往過,朕未嘗介意,豈臣子所當言?淳為此奏,欲獻諂希恩耶?」議遂寢。然帝終感瑤言。久之,竟復郕王帝號。

瑤後知番禺縣,多異政。發中官韋眷通番事,沒其貲鉅萬於官。眷憾甚,誣奏於朝。瑤及布政使陳選俱被逮,士民泣送者塞道。瑤竟謫戍永州。釋還,卒。

黎淳,華容人。天順元年進士第一。官至南京禮部尚書,頗有名譽。其與瑤爭郕王廟號也,專欲阿憲宗意,至以昌邑、更始比景帝,為士論所薄。當成化時,言路大阻,給事、御史多獲譴。惟瑤以卑官建危議,卒無罪。時皆稱帝盛德云。

又有虎臣者,麟遊人。成化中貢入太學。上言天下士大夫過先聖廟,宜下輿馬。從之。省親歸,會陜西大饑,巡撫鄭時將請振,臣齎奏行,陳饑歉狀,詞激切,大獲振貸。已,上言:「臣鄉比歲災傷,人相食,由長吏貪殘,賦役失均。請敕有司審民戶,編三等以定科徭。」從之。孝宗踐阼,將建棕棚萬歲山,備登眺。臣抗疏切諫。祭酒費訚懼禍及,鋃鐺縶臣堂樹下。俄官校宣臣至左順門,傳旨慰諭曰:「若言是,棕棚已毀矣。」訚大慚,臣名遂聞都下。頃之,命授七品官,乃以為雲南柷嘉知縣,卒官。

贊曰:明自太祖開基,廣闢言路。中外臣寮,建言不拘所職。草野微賤,奏章咸得上聞。沿及宣、英,流風未替。雖升平日久,堂陛深嚴,而逢掖布衣。刀筆掾史,抱關之冗吏,荷戈之戍卒,朝陳封事,夕達帝閽。採納者榮顯其身,報罷者亦不之罪。若仁宗之復弋謙朝參,引咎自責,即懸鞀設鐸,復何以加。以此為招,宜乎慷慨發憤之徒扼腕而談世務也。英、景之際,《實錄》所載,不可勝書。今掇其著者列於篇。迨憲宗季年,閹尹擅朝,事勢屢變,別自為卷,得有考焉。


\end{pinyinscope}