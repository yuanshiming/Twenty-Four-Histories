\article{列傳第七十三}

\begin{pinyinscope}
○李敏葉淇賈俊劉璋黃紱張悅張鎣鷿鐘曾鑒梁璟王詔徐恪李介子昆黃珂王鴻儒叢蘭吳世忠

李敏,字公勉,襄城人。景泰五年進士。授御史。天順初,奉敕撫定貴州蠻。還,巡按畿內。以薊州餉道經海口,多覆溺,建議別開三河達薊州,以避其險,軍民利之。

成化初,用薦超遷浙江按察使。再任湖廣。歷山西、四川左、右布政使。十三年擢右副都御史,巡撫大同。敵騎出沒塞下,掩殺守墩軍,敏伏壯士突擒之。修治垣塹,敵不敢犯。十五年召為兵部右侍郎。踰四年,病歸。河南大饑,條上救荒數事。詔以左副都御史巡撫保定諸府。二十一年改督漕運,尋召拜戶部尚書。

先是,敏在大同,見山東、河南轉餉至者,道遠耗費,乃會計歲支外,悉令輸銀。民輕齎易達,而將士得以其贏治軍裝,交便之。至是,并請畿輔、山西、陜西州縣歲輸糧各邊者,每糧一石征銀一兩,以十九輸邊,依時值折軍餉,有餘則召糴以備軍興。帝從之。自是北方二稅皆折銀,由敏始也。崇文門宣課司稅,多為勢要所侵漁。敏因馬文升言請增設御史主事監視。御史陳瑤斥敏聚斂,敏再疏求去。帝慰留之。貴戚請隙地及鷹房、牧馬場千頃,敏執不可,事得寢。

當憲宗末,中官、佞倖多賜莊田。既得罪,率辭而歸之官,罪重者奪之。然不以賦民。敏請召佃,畝科銀三分,帝從之,然他莊田如故也。會京師大水,敏乃極陳其害,言:「今畿輔皇莊五,為地萬二千八百餘頃;勛戚、中官莊三百三十有二,為地三萬三千一百餘頃。官校招無賴為莊頭,豪奪畜產,戕殺人,污婦女,民心痛傷。災異所由生。皇莊始正統間,諸王未封,相閑地立莊。王之籓,地仍歸官,其後乃沿襲。普天之下,莫非王土,何必皇莊。請盡革莊戶,賦民耕。畝概征銀三分,充各宮用度。無皇莊之名,而有足用之效。至權要莊田,亦請擇佃戶領之,有司收其課,聽諸家領取。悅民心,感和氣,無切於此。」時不能用。

南京御史與守備太監蔣琮相訐,御史咸逮謫,而琮居職如故。敏再疏力爭,皆不聽。弘治四年,得疾乞休,帝為遣醫視療。已,復力請,乃以葉淇代,詔敏乘傳歸。未抵家卒。贈太子少保,謚恭靖。

敏生平篤行誼,所得祿賜悉以分昆弟、故人。里居時,築室紫雲山麓,聚書數千卷,與學者講習。及巡撫大同,疏籍之於官,詔賜名紫雲書院。大同孔廟無雅樂,以敏奏得頒給如制云。

葉淇,字本清,山陽人。景泰五年進士。授御史。天順初,石亨譖之下吏,考訊無驗,出為武陟知縣。成化中累官大同巡撫。孝宗立,召為戶部侍郎。弘治四年代李敏為尚書,尋加太子少保。哈密為土魯番所陷,守臣請給其遺民廩食,處之內地,淇曰:「是自貽禍也。」寢其奏。奸民獻大名地為皇莊,淇議歸之有司。內官龍綬請開銀礦,淇不可。帝從之。已,綬請長蘆鹽二萬引,鬻於兩淮以供織造費。淇力爭,竟不納。

淇居戶部六年,直亮有執,能為國家惜財用。每廷議用兵,輒持不可。惟變開中之制,令淮商以銀代粟,鹽課驟增至百萬,悉輸之運司,邊儲由此蕭然矣。九年四月乞休,歸卒。贈太子太保。

從子贄,進士,歷官刑部右侍郎,以清操聞。

賈俊,字廷傑,束鹿人。以鄉舉入國學。天順中,選授御史。歷巡浙江、山西、陜西、河南、南畿,所至有聲。

成化十三年,自山東副使超拜右僉都御史,巡撫寧夏。在鎮七年,軍民樂業,召為工部右侍郎。二十一年奉敕振饑河南。尋轉左,數月拜尚書。時專重進士,舉人無至六卿者,俊獨以重望得之。及孝宗踐阼,尚書王恕、李敏、周洪謨、餘子俊、何喬新,都御史馬文升,皆一時民譽,俊參其間,亦稱職。

諸王府第、塋墓悉官予直,而儀仗時繕修。內官監欲頻興大工,俊言王府既有祿米、莊田,請給半直;儀仗非甚敝,不得煩有司;公家所宜營,惟倉庫、城池,餘皆停罷。帝報可。弘治四年,中官奏修沙河橋,請發京軍二萬五千及長陵五衛軍助役。內府寶鈔司乞增工匠。浙江及蘇、松諸府方罹水災而織造錦綺至數萬匹。俊皆執奏,並得寢。

工部政務與內府監局相表裏,而內官監專董工役,職尤相關。俊不為所撓,工役大省。太廟後殿成,加太子少保。足疾,致仕。詔許乘傳歸,給夫廩如制。踰年卒。

俊廉慎,居工部八年,望孚朝野。

代之者劉璋,字廷信,延平人。天順初進士。歷官中外有聲。居工部,亦數有爭執,名亞於俊。

黃紱,字用章,其先封丘人。曾祖徙平越,遂家焉。紱登正統十三年進士,除行人,歷南京刑部郎中。剛廉,人目之曰「硬黃」。大猾譚千戶者,占民蘆場,莫敢問,紱奪還之民。

成化九年,遷四川左參議。久之,進左參政。按部崇慶,旋風起輿前,不得行。紱曰:「此必有冤,吾當為理。」風遂散。至州,禱城隍神,夢若有言州西寺者。寺去州四十里,倚山為巢,後臨巨塘。僧夜殺人沉之塘下,分其資。且多藏婦女於窟中。紱發吏兵圍之,窮詰,得其狀,誅僧毀其寺。倉吏倚皇親乾沒官糧巨萬,紱追論如法,威行部中。歷四川、湖廣左、右布政使。奏閉建昌銀礦。兩京工興,湖廣當輸銀二萬,例征之民,紱以庫羨充之。荊王奏徙先壟,紱恐為民擾,執不可。

二十二年,擢右副都御史,巡撫延綏。劾參將郭鏞,都指揮鄭印、李鐸、王琮等抵罪,計捕奸豪張綱。申軍令,增置墩堡,邊政一新。出見士卒妻衣不蔽體,嘆曰:「健兒家貧至是,何面目臨其上。」亟豫給三月餉,親為拊循。會有詔毀庵寺,紱因盡汰諸尼,以給壯士無妻者。及紱去,多攜子女拜送於道。

弘治三年,拜南京戶部尚書。言官以紱進頗驟,頻有言。帝不聽,就改左都御史,焚差歷簿於庭曰:「事貴得人耳,資勞久近,豈立官意哉。」紱歷官四十餘年,性卞急,不能容物。然操履潔白,所至有建樹。六年乞休,未行卒。

張悅,字時敏,松江華亭人。舉天順四年進士,授刑部主事,進員外郎。

成化中出為江西僉事,改督浙江學校。力拒請託,校士不糊名,曰:「我取自信而已。」遷四川副使,進按察使。遭喪,服闋補湖廣。王府承奉張通縱恣,悅繩以法。及入覲,中官尚銘督東廠,眾競趨其門,悅獨不往。銘銜甚,伺察無所得。銘敗,召拜左僉都御史。

孝宗立,遷工部右侍郎,轉吏部左侍郎。王恕為尚書,悅左右之,嘗兩攝選事。弘治六年夏,大旱,求言。陳遵舊章、恤小民、崇儉素、裁冗食、禁濫罰數事。又上修德、圖治二疏。並嘉納。俄遷南京右都御史,就改吏部尚書。九年復改兵部,參贊機務。以年至,累疏乞休。詔加太子少保,馳傳歸。卒贈太子太保,謚莊簡。

時與悅同里而先為南京兵部尚書者張鎣,字廷器,正統十三年進士。景泰初,擢御史。歷江西副使按察使、陜西左布政使。成化三年以右副都御史巡撫寧夏。寧夏城,土築,鎣始甃以磚。道河流,溉靈州屯田七百餘頃。以父喪去。服除,起撫河間諸府,改大同,歷刑部左、右侍郎。十八年擢本部尚書。明年加太子少保。又明年,再以憂歸。弘治元年起南京兵部尚書,卒官,贈太子太保,謚莊懿。

侶鐘,字大器,鄆城人。成化二年進士。授御史,巡鹽兩淮。按浙江還,掌諸道章奏。汪直諷鐘劾馬文升,鐘不可,被譖杖闕下。以都御史王越薦,擢大理寺丞,再遷右少卿。寇入大同,廷議遣大臣巡視保定諸府,乃以命鐘。居數月,即擢右副都御史巡撫其地。河間瀕海民地為勢家所據,鐘奪還之。召為刑部右侍郎。丁內艱,僦運艘載母柩南還。督漕總兵官王信奏之,逮下吏。會當路方逐尹旻黨,而鐘與旻為同鄉,乃貶二秩為曲靖知府,改徽州,復入為大理寺左少卿。

弘治三年,以右副都御史巡撫蘇、松諸府,盡心荒政。召為戶部侍郎總督倉場,尋改吏部。十一年遷右都御史。居二年,進戶部尚書。

十五年,上天下會計之數,言:「常入之賦,以蠲免漸減,常出之費,以請乞漸增,入不足當出。正統以前軍國費省,小民輸正賦而已。自景泰至今,用度日廣,額外科率。河南、山東邊餉,浙江、雲南、廣東雜辦,皆昔所無。民已重困,無可復增。往時四方豐登,邊境無調發,州縣無流移。今太倉無儲,內府殫絀,而冗食冗費日加於前。願陛下惕然省憂,力加損節。且敕廷臣共求所以足用之術。」帝乃下廷臣議。議上十二事,其罷傳奉冗官,汰內府濫收軍匠,清騰驤四衛勇士,停寺觀齋醮,省內侍、畫工、番僧供應,禁王府及織造濫乞鹽引,令有司徵莊田租,皆權倖所不便者。疏留數月不下,鐘乃復言之。他皆報可,而事關權幸者終格不行。

奸商投外戚張鶴齡,乞以長蘆舊引十七萬免追鹽課,每引納銀五分,別用價買各場餘鹽如其數,聽鬻販,帝許之。後奸民援例乞兩淮舊引至百六十萬,鐘等力持,皆不聽。自此鹽法大壞,奸人橫行江湖,官司無如何矣。

東廠偵事者發鐘子瑞受金事,鐘屢疏乞休,命馳驛歸。正德時,劉瑾摭鐘在部時事,至罰米者三。又數年卒。

曾鑒,字克明,其先桂陽人,以戍籍居京師。天順八年進士。授刑部主事。通州民十餘人坐為盜,獄已具,鑒辨其誣。已,果獲真盜。成化末,歷右通政,累遷工部左侍郎。弘治十三年進尚書。

孝宗在位久,海內樂業,內府供奉漸廣,司設監請改造龍毯、素毯一百有奇。鑒等言:「毯雖一物,然征毛毳於山、陜,採綿紗諸料於河南,召工匠於蘇、松,經累歲,勞費百端。祈賜停止。」不聽。內府針工局乞收幼匠千人,鑒等言:「往年尚衣監收匠千人,而兵仗局效之,收至二千人。軍器局、司設監又效之,各收千人。弊源一開,其流無已。」於是命減其半。太監李興請辦元夕煙火,有詔裁省,因鑒奏盡罷之。十六年,帝納諸大臣言召還織造中官,中官鄧瑢以請,帝又許之。鑑等極言,乃命減三之一。其冬,言諸省方用兵,且水旱多盜賊,乞罷諸營繕及明年煙火、龍虎山上清宮工作。帝皆報從。

正德元年,雷震南京報恩寺塔,守備中官傅容請修之。鑑言天心示儆,不宜重興土木以勞民力,乃止。御馬監太監陳貴奏遷馬房,欽天監官倪謙覆視,請從之。給事中陶諧等劾貴假公營私,并劾謙阿附,不聽。鑒執奏,謂馬房皆由欽天監相視營造,其後任意增置者,宜令拆毀改正,葺以己資,庶牧養無妨而民不勞。報可。內織染局請開蘇、杭諸府織造,上供錦綺為數二萬四千有奇。鑑力請停罷,得減三分之半。太監許鏞等各齎敕於浙江諸處抽運木植,亦以鑑言得寢。

孝宗末,閣部大臣皆極一時選,鑑亦持正。及與韓文等請誅宦官不勝,諸大臣留者率巽順避禍,鑒獨守故操。有詔賜皇親夏儒第,帝嫌其隘,欲拓之。鑑力爭,不從。明年春,中官黃準守備鳳陽,從其請,賜旗牌。鑑等言大將出征及諸邊守將,乃有旗牌,內地守備無故事,乃寢。其年閏正月致仕。旋卒。贈太子太保。

梁璟,字廷美,崞縣人。天順八年進士。授兵科給事中。

成化時,屢遷都給事中。項忠征荊、襄,驅流民復業。璟劾其縱兵逼迫,較賊更慘,語具《忠傳》。延綏用兵,令山西預征芻粟,民相率逃亡。璟疏陳其困,得寬減。畿輔八府舊止設巡撫一人,駐薊州以禦邊,不能兼顧。璟請順天、永平二府分設一巡撫,以薊州邊務屬之,令巡撫陳濂專撫保定六府兼督紫荊諸關。朝議從之,遂為定制。已,與同官韓文、王詔等奏請起致仕尚書王竑、李秉,而斥都御史王越,并及宮闈隱事,被撻文華殿。武靖伯趙輔西征不敢戰,稱病求還,復謀典營府事。璟等極論其罪,乃令養疾歸。

九載秩滿,擢陜西左參政,分守洮、岷。西番入寇,督兵斬其魁。內艱服闋,還原任,歷左、右布政使。先後在陜十五年,多政績。

孝宗嗣位,遷右副都御史,巡撫湖廣。弘治二年,民饑,請免征兩京漕糧八十九萬餘石,從之。帝登極詔書已罷四方額外貢獻,而提督武當山中官復貢黃精、梅筍、茶芽諸物。武當道士先止四百,至是倍之,所度道童更倍,咸衣食於官。月給油蠟、香楮,灑掃夫役以千計。中官陳喜又攜道士三十餘人,各領護持敕,所至張威虐。璟皆奏請停免,多見採納。外艱服除,再撫四川。七年召拜南京吏部右侍郎。久之,就進戶部尚書。致仕歸,卒。

王詔,字文振,趙人。生有異姿,學士曹鼐奇之,妻以女。天順末,登進士,授工科給事中。睿皇后崩,值秋享太廟,時議謂不當以卑廢尊。詔言《禮》有喪不祭,無已,則移日俟釋服。議雖不行,識者是焉。勘牧馬草場,劾會昌侯孫繼宗、撫寧侯朱永侵占罪。時方面官我,令京卿三品保舉。詔言恐長奔競風,不聽。累遷都給事中。八年七月敕修隆善寺工竣,授工匠三十人官尚寶少卿,任道遜等以書碑皆進秩。詔上疏力諫,不省。已,偕梁璟等論及宮闈事,帝大怒,召至文華殿面詰之,詔仰呼曰:「臣等言雖不當,然區區犬馬之誠,知為國而已。」乃杖而釋之。出為湖廣右參政。原傑經略荊、襄,詔襄理功為多。以父憂去。服除再任,遷右布政使。

弘治元年,轉貴州左布政使。其冬,以右副都御史巡撫雲南。土官好爭襲,所司入其賄,變亂曲直,生邊患。詔不通苞苴,一斷以法,且去弊政之不便者。諸夷歸命,邊徼寧戢。有故官不能歸者,妻子多鬻為奴。詔為資遣,得歸者甚眾。洪武中,尚書吳雲繼王禕死事,後禕謚忠文,歲祀之,而不及雲。詔以為請,乃謚雲忠節,與禕並祀。四年召拜南京兵部右侍郎,未上,卒。

徐恪,字公肅,常熟人。成化二年進士。授工科給事中。中官欲出領抽分廠,恪等疏爭。中官怒,請即遣恪等,將摭其罪,無所得乃已。出為湖廣左參議,遷河南右參政。陜西饑,當轉粟數萬石。恪以道遠請輸直,上下稱便。

弘治初,歷遷左、右布政使。徽王府承奉司違制置吏,恪革之。王奏恪侵侮,帝賜書誡王。河徙逼開封,有議遷籓府三司於許州者,恪言非便,遂寢。四年拜右副都御史,巡撫其地。奏言:「秦項梁、唐龐勛、元方谷珍輩往往起東南。今東南民力已竭,加水旱洊臻,去冬彗掃天津,直吳、越地。乞召還織造內臣,敕撫按諸臣加意拊循,以弭異變。」帝不從。故事,王府有大喪,遣中官致祭,所過擾民。成化末,始就遣王府承奉。及帝即位,又復之。恪請如先帝制,并條上汰冗官、清賦稅、禁科擾、定贖例、革抽分數事,多議行。戶部督逋急,恪以災變請緩其事。御史李興請於鄖陽別設三司,割南陽、荊州、襄陽、漢中、保寧、夔州隸之。恪陳五不可,乃止。

恪素剛正。所至,抑豪右,祛奸弊。及為巡撫,以所部多王府,持法尤嚴,宗人多不悅。平樂、義寧二王遂訐恪減祿米、改校尉諸事。勘無驗,坐恪入王府誤行端禮門,欲以平二王忿。帝知恪無他,而以二王幼,降敕切責,命湖廣巡撫韓文與恪易任。吏民罷市,泣送數十里不絕。屬吏以羨金贐,揮之去。至則值岐王之國,中使攜鹽數百艘,抑賣於民,為恪所持阻不行。其黨密構於帝。居一歲,中旨改南京工部右侍郎。恪上疏曰:「大臣進用,宜出廷推,未聞有傳奉得者。臣生平不敢由他途進,請賜罷黜。」帝慰留,乃拜命。勢要家濫索工匠者,悉執不予。十一年考績入都,得疾,遂致仕,卒。

李介,字守貞,高密人。成化五年進士。選庶吉士,改御史,巡鹽兩浙,還掌河南道事。以四方災傷,陳時政數事,帝多採用之。介敢言,遇事不可,輒率同列論奏。忤帝意,兩撻於庭。九載滿,擢大理丞,進少卿。

弘治改元,遷右僉都御史,巡撫宣府。尋召佐院事。歷兵部左、右侍郎。十年夏,北寇謀犯大同,命介兼左僉都御史,往督軍餉,且經略之。比至,寇已退,乃大修戎備。察核官田牛具錢還之軍,以其資償軍所逋馬價,邊人感悅。先後條上便宜二十事。卒,贈尚書。

子昆,字承裕。弘治初進士。歷禮部主事。中官何鼎建言下獄,臺諫救之,咸被責。昆復論救,弗聽。父憂歸,起改兵部主事。帝將建延壽塔於城外,昆復疏諫。正德初,群小用事。請黜邪枉,進忠直,杜宦戚請乞,節中外侈費,皆不報。進員外郎,忤尚書劉宇,貶知解州。屢遷陜西左布政使。十年以右副都御史巡撫甘肅。與總督彭澤經略哈密,兵部尚書王瓊劾澤處置失宜,語連昆,下吏。法司言昆設謀遏強寇,功不可掩。瓊不從,謫浙江副使。世宗立,瓊得罪。復官,巡撫順天。尋召為兵部右侍郎,嘉靖初,改左。大同軍亂,殺巡撫張文錦。昆奉命往撫,承制曲赦之,還請收恤文錦。帝方惡其激變,不從。遇疾歸,久之卒。

黃珂,字鳴玉,遂寧人。成化二十年進士。授龍陽知縣。治行聞,擢御史,出按貴州。金達長官何輪謀不軌,計擒之,改設流官。賊婦米魯亂,奏劾巡撫錢鉞、總兵官焦俊等,皆得罪。改按畿輔,歷山西按察使。

正德四年,擢右僉都御史巡撫延綏。安化王寘鐇反,傳檄四方,用討劉瑾為名。他鎮畏瑾,不敢以聞。珂封上其檄,因陳便宜八事,而急令副總兵侯勛、參將時源分兵扼河東,賊遂不敢出。亦不剌寇邊,珂偕總兵官馬昂督軍戰,敗之木瓜山。六年復寇邊,珂檄副總兵王勛等七將分據要害夾擊,復敗之。屢賜璽書,銀幣。

是年秋,入為戶部右侍郎,總督倉場。河南用兵,出理軍餉。主客兵十餘萬,追奔轉戰,遷止無常。珂隨方轉輸,軍興無乏,錄功增俸一級。改刑部,進左侍郎,已改佐兵部。寧王宸濠謀復護衛,珂執議獨堅。九年擢南京右都御史,尋就拜工部尚書。以年至乞休歸,卒。贈太子少保,謚簡肅。

王鴻儒,字懋學,南陽人。少工書,家貧為府書佐。知府段堅愛其書,留署中,親教之。遣入學校為諸生,遂舉鄉試第一。成化末,登進士,授南京戶部主事。累遷郎中,擢山西僉事,進副使,俱督學政。居九年,士風甚盛。孝宗嘗語劉大夏曰:「籓臬中若王鴻儒,他日可大用也。」正德改元,謝病歸。劉瑾擅政,收召名流。四年夏,起為國子祭酒,以父喪去。再起南京戶部侍郎,歷吏部右侍郎,尋轉左。十四年遷南京戶部尚書。甫履任,宸濠反,命督軍餉,疽發於背,遂卒,謚文莊。

鴻儒為學,務窮理致用,為世所推。左吏部,清正自持,門無私謁。

弟鴻漸,鄉試亦第一。以進士累官山東右布政使,以廉靜稱。

叢蘭,字廷秀,文登人。弘治三年進士。為戶科給事中。中官梁芳、陳喜、汪直、韋興,先以罪擯斥,復夤緣還京。蘭因清寧宮災,疏陳六事,極論芳等罪,諸人遂廢。尋言:「吏部遵詔書,請擢用建言詿誤諸臣,而明旨不盡從,非所以示信。失儀被糾,請免送詔獄。畿內征徭繁重,富民規免,他戶代之,宜釐正。」章下所司。進兵科右給事中。都督僉事吳安以傳奉得官,蘭請罷之。時命撥團營軍八千人修九門城濠,蘭言:「臣頃簡營軍,詔許專事訓練,無復差撥,命下未幾,旋復役之,如前詔何。」遂罷遣。遷通政參議。小王子犯大同,命經略紫荊、倒馬諸關塞蹊隧可通敵騎者百十所。

正德三年進左通政。明年冬出理延綏屯田。安化王寘鐇反,蘭奏陳十事,中言:「文武官罰米者,鬻產不能償。朝臣謫戍,刑官妄引新例鍛煉成獄,沒其家資。校尉遍行邊塞,勢焰薰灼,人不自保。」劉瑾大惡之,矯旨嚴責。給事中張瓚、御史汪賜等遂希旨劾蘭。瑾方憂邊事,置不問。數月,瑾誅,進通政使。俄擢戶部右侍郎,督理三邊軍餉。

六年,陜西巡撫都御史藍章以四月寇亂移駐漢中。會河套有警,乃命蘭兼管固、靖等處軍務。蘭上言:「陜西起運糧草,數為大戶侵牟,請委官押送。每鎮請發內帑銀數萬,預賣糧草。御史張彧清出田畝,請蠲免子粒,如弘治十八年以前科則。靈州鹽課,請照例開中,召商糴糧。軍士折色,主者多剋減,乞選委鄰近有司散給。」從之。

是年冬,南畿及河南歲侵,命蘭往振。未赴而河北賊自宿遷渡河,將逼鳳陽。乃命蘭以本官巡視廬、鳳、滁、和,兼理振濟。河南白蓮賊趙景隆自稱宋王,掠歸德,蘭遣指揮石堅、知州張思齊等擊斬之。九月,賊平。論功賚金幣,增俸一級,召還理部事。部無侍郎缺,乃命添註。明年,大同有警,命巡視居庸、龍泉諸關。尋兼督宣、大軍餉,進右都御史,總制宣、大、山東軍務。令內地皆築堡,寇至收保如塞下。寇五萬騎自萬全右衛趨蔚州大掠,又三萬騎入平虜南城,以失事停半歲俸。

十年夏,改督漕運,尋兼巡撫江北。中官劉允取佛烏思藏,道蘭境,入謁,辭不見。允需舟五百餘艘、役夫萬餘人,蘭馳疏極陳其害。不報。居四年,以事忤兵部尚書王瓊,解漕務,專任巡撫。寧王宸濠反,蘭移鎮瓜州。十五年,遷南京工部尚書。

世宗即位,御史陳克宅劾蘭附江彬。帝以蘭素清謹,釋勿問。蘭遂乞休去。卒,贈太子少保。

吳世忠,字懋貞,金谿人。弘治三年進士。授兵科給事中。兩畿及山東、河南、浙江民饑,有詔振恤,所司俟勘覆。世忠極言其弊,因條上興水利、復常平二事,多施行。已,請恤建文朝殉難諸臣,乞賜爵謚,崇廟食,且錄其子孫,復其族屬,為忠義勸。章下禮官,寢不行。尚書王恕被訐求去,上疏請留之。壽寧侯張鶴齡求勘河間賜地,其母金夫人復求不已。帝命遣使,世忠言:「侯家仰托肺腑,豈宜與小民爭尺寸?命部勘未已,內臣繼之。內臣未已,大臣又繼之。剝民斂怨,非國家福,龍非外戚之福。」不聽。

大同總兵官神英、副總兵趙昶等,因馬市令家人以違禁彩繒易馬,番人因闌入私易鐵器。既出塞,復潛兵掠蔚州,陷馬營,轉剽中東二路。英等擁兵不救,巡撫劉瓛、鎮守中官孫振又不以實聞。十一年,事發,世忠往勘。上疏備陳大同邊備廢馳、士卒困苦之狀。因極言英、瓛等貪利畏敵,蕩無法度。英落職,瓛、振召還,昶及遊擊劉淮、參將李嶼等俱逮問。已而瓛改大理少卿,昶以大理丞吳一貫覆言獻僅鐫級。世忠復極論瓛罪,且詆一貫,帝皆不問。闕里文廟災,陳八事,不能盡用。

寇犯延綏、大同,世忠言:「國初設七十二衛,軍士不下百萬。近軍政日壞,精卒不能得一二萬人。此兵足憂也。太倉之儲,本以備軍。近支費日廣,移用日多。倘興師十萬,犒賜無所取給。此食足憂也。正統己巳之變尚有石亨、楊洪,邇所用李杲、阮興、趙昶、劉淮之屬,先後皆敗。今王璽、馬昇又以失事告。此將帥足憂也。國家多事,大臣有以鎮之。邇者忠正多斥,貪庸獲存。既鮮匡濟之才,又昧去就之節,安能懾強敵壯國勢乎?此任人足憂也。政多舛乖,民日咨怨。京軍敝力役,京民苦催科,畿甸覬恩尤切。顧使不樂其生至此,臨難誰與死守?此民心足憂也。天變屢征,火患頻發。雲南地震壓萬餘家,大同馬災踣二千匹。此天意足憂也。願順好惡以收人心,肅念慮以回天意,遣文武重臣經略宣、大,以飭邊防。策免諸臣不肖者,而起素有才望,如何喬新、劉大夏、倪岳、戴珊、張敷華、林俊諸人,以任國事。則賊將望風遠循,而邊境可無憂矣。」帝以言多詆毀,切責之。尋乞大同增置臺堡,以閒田給軍耕墾,不徵其稅。江西歲饑盜起,請簡巡撫,黜有司貪殘者。又請築京師外城。所司多從其議。再遷吏科左給事中,擢湖廣參議,坐事降山東僉事。

正德四年閏九月召為光祿少卿,旋改尚寶司卿。其年冬,與通政叢蘭等出理邊屯,世忠往薊州。明年奏言:「占種盜賣,積弊已久。若一一究問,恐人情不安,請量為處分。」從之。劉瑾敗,言官劾其嘗請清核屯田,助瑾為虐。世忠故方鯁,朝議寬之,得免。再遷大理少卿。八年擢右僉都御史巡撫延綏。寇在河套,逐之失利,乃引疾歸。

贊曰:明至英宗以後,幸門日開。傳奉請乞,官冗役繁,用度奓汰,盛極孽衰,國計坐絀。李敏諸人斤斤為國惜財,抵抗近幸,以求紓民。然涓滴之助,無補漏卮。國家當承平殷阜之世,侈心易萌。近習乘之,糜費日廣。《易》曰:「節以制度,不傷財,不害民」,又曰「不節若,則嗟若」,此恭儉之主所為凜凜也。


\end{pinyinscope}