\article{列傳第九十四}

\begin{pinyinscope}
馬錄顏頤壽聶賢湯沐劉琦盧瓊沈漢王科程啟充張逵鄭一鵬唐樞杜鸞葉應聰藍田黃綰解一貫鄭洛書張錄陸粲劉希簡王準邵經邦劉世揚趙漢魏良弼秦鰲張寅葉洪

馬錄,字君卿,信陽人。正德三年進士。授固安知縣。居官廉明,徵為御史,按江南諸府。世宗即位,疏言:「江南之民最苦糧長。白糧輸內府一石,率費四五石。他如酒醋局、供應庫以至軍器、胖襖、顏料之屬輸內府者,費皆然。戶部侍郎秦金等請從錄言,命石加耗一斗,毋得苛求。中官黃錦誣劾高唐判官金坡,詔逮之,連五百餘人。錄言:祖宗內設法司,外設撫、按,百餘年刑清政平。先帝時,劉瑾、錢寧輩蠱惑聖聰,動遣錦衣官校,致天下洶洶。陛下方勤新政,不虞復有高唐之命。」給事中許復禮等亦以為言,獄得少解。嘉靖二年大計天下庶官,被黜者多訐撫、按,以錄言禁止。

五年出按山西,而妖賊李福達獄起。福達者,崞人。初坐妖賊王良、李鉞黨,戍山丹衛。逃還,更名午,為清軍御史所勾,再戍山海衛。復逃居洛川,以彌勒教誘愚民邵進祿等為亂。事覺,進祿伏誅,福達先還家,得免。更姓名曰張寅,往來徐溝間,輸粟得太原衛指揮使。子大仁、大義、大禮皆冒京師匠籍。用黃白術干武定侯郭勛,勛大信幸。其仇薛良訟於錄,按問得實。檄洛川父老雜辨之,益信。勛為遺書錄祈免,錄不從,偕巡撫江潮具獄以聞,且劾勛庇奸亂法。章下都察院,都御史聶賢等覆如錄奏,力言勛黨逆罪。詔福達父子論死,妻女為奴,沒其產,責勛對狀。勛懼,乞恩,因為福達代辨,帝置不問。會給事中王科、鄭一鵬、程輅、常泰、劉琦、鄭自璧、趙廷瑞、沈漢、秦祐、張逵、陳皋謨,御史程啟充、盧瓊、邵豳、高世魁、任淳,南京御史姚鳴鳳、潘壯、戚雄、王獻,評事杜鸞,刑部郎中劉仕,主事唐樞,交章劾勛,謂罪當連坐。勛亦累自訴,且以議禮觸眾怒為言,帝心動。勛復乞張璁、桂萼為援。璁、萼素惡廷臣攻己,亦欲借是舒宿憤,乃謂諸臣內外交結,借端陷勛,將漸及諸議禮者。帝深入其言,而外廷不知,攻勛益急。帝益疑,命取福達等至京下三法司訊,既又命會文武大臣更訊之,皆無異詞。帝怒,將親訊,以楊一清之言而止,仍下廷鞫。尚書顏頤壽等不敢自堅,改擬妖言律斬。帝猶怒,命法司俱戴罪辦事,遣官往械錄、潮及前問官布政使李璋、按察使李玨、僉事章綸、都指揮馬豕等。時璋、玨已遷都御史,璋巡撫寧夏,玨巡撫甘肅,皆下獄廷訊。乃反前獄,抵良誣告罪。

帝以罪不及錄,怒甚。命璁、萼、方獻夫分署三法司事,盡下尚書頤壽,侍郎劉玉、王啟,左都御史賢,副都御史劉文莊,僉都御史張潤,大理卿湯沐,少卿徐文華、顧佖,寺丞汪淵獄,嚴刑推問遂搜錄篋,得大學士賈詠、都御史張仲賢、工部侍郎閔楷、御史張英及寺丞淵私書。詠引罪致仕去,仲賢等亦下獄。萼等上言:「給事中琦、泰,郎中仕,聲勢相倚,挾私彈事,佐錄殺人。給事中科、一鵬、祐、漢、輅,評事鸞,御史鳴鳳、壯、雄,扶同妄奏,助成奸惡。給事中逵,御史世魁,方幸寅就死,得誣勛謀逆,率同連名,同聲駕禍。郎中司馬相妄引事例,故意增減,誣上行私。邇者言官締黨求勝,內則奴隸公卿,外則草芥司屬,任情恣橫,殆非一日,請大奮乾斷,彰國法。」帝納其言,並下諸人獄,收繫南京刑部。先是,廷臣會訊,太僕卿汪元錫、光祿少卿餘才偶語曰:「此獄已得情,何再鞫?」偵者告萼,以聞,亦逮問。

萼等遂肆搒掠。錄不勝刑,自誣故入人罪。萼等乃定爰書,言寅非福達,錄等恨勛,構成冤獄,因列諸臣罪名。帝悉從其言。謫戍極邊,遇赦不宥者五人:璋、玨、綸、豕、前山西副使遷大理少卿文華。謫戍邊衛者七人:琦、逵、泰、瓊、啟充、仕及知州胡偉。為民者十一人:賢、科、一鵬、祐、漢、輅、世魁、淳、鳴鳳、相、鸞。革職閒住者十七人:頤壽、玉、啟、潮、文莊、沐、佖、淵、元錫、才、楷、仲賢、潤、英、壯、雄、前大理丞遷僉都御史毛伯溫。其他下巡按逮問革職者,副使周宣等復五人。良抵死,眾證皆戍,寅還職。錄以故入人死未決,當徒。帝以為輕,欲坐以奸黨律斬。萼等謂張寅未死,而錄代之死,恐天下不服,宜永戍煙瘴地,令緣及子孫。乃戍廣西南丹衛,遇赦不宥。帝意猶未慊,語楊一清等曰:「與其佼及後世,不若誅止其身,從《舜典》『罰弗及嗣』之意。」一清曰:「祖宗制律具有成法,錄罪不中死律。若法外用刑,吏將緣作奸,人無所措手足矣。」帝不得已,從之。以萼等平反有功,勞諭之文華殿,賜二品服俸、金帶、銀幣,給三代誥命。遂編《欽明大獄錄》頒示天下。時嘉靖六年九月壬午也。至十六年,皇子生,肆赦。諸謫戍者俱釋還,惟錄不赦,竟卒於戍所。

顏頤壽,巴陵人,居官有清望。

聶賢,長壽人。為御史清廉。奪官五年,用薦起工部尚書,改刑部尚書。致仕,卒。謚榮襄。

湯沐,字新之,江陰人。弘治九年進士。除崇德知縣,徵授御史。正德初,嘗劾中官苗逵、保國公朱暉等罪,出為湖廣僉事。劉瑾以沐不附己,用牙儈同寅訐學士張芮事波及沐,謫武義知縣。瑾誅,復為廣東僉事。累遷右副都御史,巡撫貴州。請立土官世系籍,絕其爭襲之弊,而令其子弟入學,報可。嘉靖二年改撫四川,入為大理卿。既坐福達獄罷歸,家居六年,薦章數十上,不召,卒。沐居官三十載,屏絕餽遺,以廉潔稱。

劉琦,字廷珍,洛川人。正德九年進士。嘉靖初,由行人授兵科給事中。時給京軍冬衣布棉恒過期,以琦請,即命琦立給。李福達逃洛川,琦知之甚悉。事覺,琦疏陳顛末,因劾郭勛黨逆,又與御史張問行劾勛侵盜草場租銀。既而馬錄獄具,坐琦佐使殺人,下獄,謫戍沈陽。閱十年赦歸,卒。

盧瓊,字獻卿,浮梁人。正德六年進士。由固始知縣入為御史。嘉靖改元,上言:「景皇帝有撥亂大功,而實錄猶稱郕戾王。敬皇帝深仁厚澤,而實錄成於焦芳手,是非顛倒。乞詔儒臣改撰。」帝惟命史官正《孝宗實錄》之不當者,然亦未有所正也。出按畿輔。桂萼疾臺諫排己,考察京官既竣,令科道互糾劾。吏科都給事中王俊民等爭之,瓊與同官劉隅等亦言交相批抵報復,非盛世事。帝切責俊民、隅,奪其俸五月,瓊等皆三月,而命部院考之。瓊竟以劾勛謫戍邊。赦還,卒。

沈漢,字宗海,吳江人。正德十六年進士。授刑科給事中。中官馬俊、王堂久廢,忽自南京召至,漢論止之。改元詔書蠲四方逋稅,漢以民間已納者多飽吏橐,請已征未解者,作來年正課。又言近籍沒奸黨貲數千萬,請悉發以補歲入不足之數。皆報可。嘉靖二年,以災異指斥時政。尚書林俊去位,復抗章爭之。戶部郎中牟泰坐吏盜官帑,下詔獄貶官。漢言:「吏為奸利,在泰未任前。事敗,泰發之。泰無罪。」因極言刑獄宜付法司,毋委鎮撫。不納。大獄起,法司皆下吏。漢言:「祖宗之法不可壞,權倖之漸不可長,大臣不可辱,妖賊不可赦。」遂並漢收繫,除其名。家居二十年,卒。曾孫璟,萬曆中為吏部員外郎。請王恭妃封號,忤旨,降行人司正。天啟初,贈少卿。

王科,字進卿,涉縣人。正德十二年進士。授藍田知縣。城隘,且無水,科導西山水入城,拓而廣之,遂為望邑。毀境內淫祠,以其材葺學宮。嘉靖四年徵為工科給事中。嘗劾兵部尚書金獻民無功,總兵官趙文、種勛失事,及陜西織造內官擾民,郭勛任奸人郭彪、鄭鸞,剝軍害民狀。又言:「三司首領、州縣佐貳以秩卑為上官所輕棄,率貪冒不自惜,宜拔擢其廉能者。而諸邊財計之職,不宜處下才。鹽運官廉,當遷敘。」大獄起,劾勛,遂下獄削籍。

方諸臣之被罪也,舉朝皆知其冤,莫敢白。踰月,南京御史吳彥獨抗章請寬之。上怒,斥於外。已而御史張祿亦以為言。忤旨,切讓。自是無敢言者。十一年,桂萼已死,張璁亦免相,聶賢、毛伯溫始起用。張潤、汪元錫、李玨、閔楷亦相繼收錄。唯臺諫、曹郎竟無一人召復者。隆慶初,諸人皆復職贈官。錄首贈太僕少卿,琦、瓊俱光祿少卿,漢、科俱太常少卿。

當萼等反福達之獄,舉朝不直萼等。而以寅、福達姓名錯互,亦或疑之。至四十五年正月,四川大盜蔡伯貫就擒。自言學妖術於山西李同。所司檄山西,捕同下獄。同供為李午之孫,大禮之子,世習白蓮教,假稱唐裔,惑眾倡亂,與大獄錄姓名無異,同竟伏誅。暨穆宗即位,御史龐尚鵬言:「據李同之獄,福達罪益彰,而當時流毒縉紳至四十餘人。衣冠之禍,可謂慘烈。郭勛世受國恩,乃黨巨盜,陷朝紳。職樞要者承其頤指,鍛煉周內。萬一陰蓄異謀,人人聽命,禍可勝言哉!乞追奪勛等官爵,優恤馬錄諸人,以作忠良之氣。」由是,福達獄始明。

程啟充,字以道,嘉定州人。正德三年進士。除三原知縣,入為御史。嬖倖子弟家人濫冒軍功,有至都督賜蟒玉者。啟充言:「定制,軍職授官,悉準首功。今倖門大啟,有買功、冒功、寄名、竄名、併功之弊。權要家賄軍士金帛,以易所獲之級,是謂買功。衝鋒斬馘者,甲也,而乙取之,甚者殺平民以為賊,是謂冒功。身不出門閭,而名隸行伍,是謂寄名。賄求掾吏,洗補文冊,是謂竄名。至有一人之身,一日之間,不出京師,而東西南朔四處報功者,按名累級,驟至高階,是謂併功。此皆壞祖宗法,解將士體,乞嚴為察革。」帝不能用。

十一年正旦,群臣待漏入賀,日晡禮始成。及散朝,已昏夜。眾奔趨而出,顛仆相踐踏。將軍趙朗者,死於禁門。啟充具奏其狀,請帝昧爽視朝,以圖明作之治。都督馬昂進妊身女弟,啟充等力爭。既又極陳冗官、冗兵、冗費之弊,乞通行革罷。帝皆不省。騰驤四衛軍改編各衛者,奉詔撤回,而各衛遺籍仍支糧,糜倉儲八十七萬餘石。啟充力言之,冒支弊絕。以憂歸。

世宗即位,起故官,即爭興獻帝皇號。嘉靖元年正月郊祀方畢,清寧宮小房火。啟充言:「災及內寢,良由徇情之禮有戾天常,僭逼之名深乖典則。輔臣執議,禮臣建明,不能敵經生之邪說,佞倖之諛辭,動假母后以箝天下之口。臣謂不正大禮,不黜邪說,所謂修省皆具文也。況邇者旨由中出而內閣不知,奸黨獄成而曲為庇護。諫臣斥逐,耳目有壅蔽之虞;大臣疏遠,股肱有痿痺之患。司禮之權重於宰相,樞機之地委之宦官。邇臣貪濁,頻有遷除;邊帥僨師,不聞譴斥。莊田之賞賚過多,潛邸之乞恩未已。伏望陛下仰畏天明,俯察眾聽,親大臣,肅庶政,以回災變。」報聞。

尋出按江西。得宸濠通蕭敬、張銳、陸完等私書,欲亟去孫燧,云:「代者湯沐、梁宸可,其次王守仁亦可。」因論敬、銳等罪,並言守仁黨逆,宜追奪。給事中汪應軫訟守仁功,言:「逆濠私書,有詔焚毀。啟充輕信被黜知縣章立梅捃摭之辭,復有此奏,非所以勸有功。」主事陸澄亦為守仁奏辨。御史向信因劾應軫與澄。帝曰:「守仁一聞宸濠變,仗義興兵,戡定大難,特加封爵,以酬大功,不必更議。」帝從太監梁棟請,遣中官督南京織造。啟充偕同官及科臣張嵩等極諫,不納。

啟充素蹇諤,張璁、桂萼惡之。會郭勛庇李福達獄,為啟充所劾,璁、萼因指啟充挾私,謫戍邊衛。十六年赦還。言者交薦,不復用,卒。隆慶初,贈光祿少卿。

張逵,字懋登,餘姚人。正德十六年進士。改庶吉士。嘉靖元年,授刑科給事中。疏言:「陛下臨御之初,國是大定。今舉動漸乖,弊端旋復。齋醮繁興,爵賞無紀。政事不關於宰執者非一,刑罰不行於貴近者甚多。臺諫會奏而斥為瀆擾,大臣執法而責以回奏。至如崔元封侯,蔣輪市寵,陳萬言乞賜第,先朝貴戚未有若是恩倖也。廖鵬緩死,劉暉得官,李隆復遣官勘問,先朝罪人未有若是淹縱也。願陛下一反目前之所為。」報聞。給事中劉最、鄧繼曾謫官,逵疏救,不聽。尋伏闕爭「大禮」,下獄廷杖。

四年十一月上疏曰:「近廷臣所上封事,陛下批答必曰『已有旨處置』,是已行者不可言也。曰『尚議處未定』,是未行者不可言也。二者不言,則是終無可言也。且今日言者,已非陛下初政時比矣。初年,事之大者,既會疏公言之,又各疏獨言之。一不得行,則相聚環視,以不得其言為愧。近者不然,會疏則刪削忌諱以避禍,獨疏則毛舉纖微以塞責。一不蒙譴,則交相慶賀,以茍免為幸。消讜直之氣,長循默之風,甚非朝廷福也。」章下所司。

尋進右給事中。王科、陳察劾郭勛,帝慰留之。逵與同官鄭自璧、趙廷瑞言:「勛倚奸成橫,用酷濟貪,籠絡貨資,漁獵營伍,為妖賊李福達請屬,為逆黨陸完雪冤。溫旨諭留,是旌使縱也。」既復言:「福達誑惑愚民,稱兵犯順。勛黨叛逆,罪不容誅。」不聽。

尋以言事忤旨,黜為吳江縣丞。復坐福達獄逮問,謫戍遼東邊衛。居十年,母死不得歸,哀痛而卒。隆慶初,贈光祿少卿。

鄭一鵬,字九萬,莆田人。正德十六年進士。改庶吉士。嘉靖初,官至戶科左給事中。一鵬性伉直,居諫垣中最敢言。御史曹嘉論大學士楊廷和,因言內閣柄太重。一鵬駁之曰:「太宗始立內閣,簡解縉等商政事,至漏下數十刻始退。自陛下即位,大臣宣召有幾?張銳、魏彬之獄,獻帝追崇之議,未嘗召廷和等面論。所擬旨,內多更定,未可謂專也。」

帝用中官崔文言,建醮乾清、坤寧諸宮,西天、西番、漢經諸廠,五花宮兩暖閣、東次閣,莫不有之。一鵬言:「禱祀繁興,必魏彬、張銳餘黨。先帝已誤,陛下豈容再誤?臣巡視光祿,見一齋醮蔬食之費,為錢萬有八千。陛下忍斂民怨,而不忍傷佞倖之心。況今天災頻降,京師道殣相望;邊境戍卒,日夜荷戈,不得飽食,而為僧道靡費至此,此臣所未解。」報聞。

東廠理刑千戶陶淳曲殺人,論謫戍。詔覆案,改擬帶俸。一鵬與御史李東等執奏,并劾刑部侍郎孟鳳,帝不聽。給事中鄧繼曾、修撰呂柟、編修鄒守益以言獲罪,一鵬皆疏救。

宮中用度日侈,數倍天順時。一鵬言:「今歲災用詘,往往借支太倉,而清寧、仁壽、未央諸宮,每有贏積,率饋遺戚里。曷若留供光祿,彰母后德?」帝命乾清、坤寧二宮暫減十之一。魯迷貢獅子、西牛、西狗、西馬及珠玉諸物。一鵬引漢閉玉門關謝西域故事,請敕邊臣量行賞賚,遣還國,勿使入京,彰朝廷不寶遠物之盛德。不聽。尋伏闕爭「大禮」,杖於廷。

侍郎胡瓚、都督魯綱督師討大同叛卒,列上功狀,請遍頒文武大臣、臺諫、部曹及各邊撫、按、鎮、監賞。一鵬言:「桂勇誅郭監等,在瓚未至之先。徐氈兒等之誅,事由朱振,於瓚無與。瓚欲邀功冒賞,懼眾口非議,乃請並敘以媚之。夫自大同構難,大臣臺諫誰為陛下畫一策者?孤城窮寇尚多逋逃,各邊鎮、撫相去數千里,安在其能犄角也?」請治瓚等欺罔罪,賞乃不行。

時諸臣進言多獲譴,而一鵬間得俞旨,益發舒言事。論楊宏不宜推寧夏總兵官;席書不宜訐費宏,留其弟春為修撰;王憲夤緣貴近,鄧璋敗事甘肅,不宜舉三邊總督;服闋尚書羅欽順、請告祭酒魯鐸、被謫修撰呂柟宜召置經筵;廷臣乞省親養疾,不宜概不許。諸疏皆侃侃。會武定侯郭勛欲得虎賁左衛以廣其第,使指揮王琬等言,衛湫隘不足居吏士,而民郭順者願以宅易之。順,勛家奴也,其宅更湫隘。一鵬與同官張嵩劾勛:「以敝宅易公署,驕縱罔上。昔竇憲改沁水園,卒以逆誅。勛謀奪朝廷武衛,其惡豈止憲比?部臣附勢曲從,宜坐罪。」尚書趙璜等因自劾。詔還所易,勛甚銜之。而一鵬復以李福達獄劾勛,桂萼、張璁因坐以妄奏,拷掠除名。九廟災,言官會薦遺賢及一鵬,竟不復召。久之,卒。隆慶初復官,贈光祿少卿。

唐樞,字惟中,歸安人。嘉靖五年進士。授刑部主事。言官以李福達獄交劾郭勛,然不得獄辭要領。樞上疏言:

李福達之獄,陛下駁勘再三,誠古帝王欽恤盛心。而諸臣負陛下,欺蔽者肆其讒,謅諛者溷其說,畏威者變其辭,訪緝者淆其真。是以陛下惑滋甚,而是非卒不能明。臣竊惟陛下之疑有六。謂謀反罪重,不宜輕加於所疑,一也。謂天下人貌有相似,二也。謂薛良言弗可聽,三也。謂李玨初牒明,四也。謂臣下立黨傾郭勛,五也。謂崞、洛證佐皆仇人,六也。臣請一一辨之。

福達之出也,始而王良、李鉞從之,其意何為?繼而惠慶、邵進祿等師之,其傳何事?李鐵漢十月下旬之約,其行何求?「我有天分」數語,其情何謀?「太上玄天,垂文秘書」,其辭何指?劫庫攻城,張旗拜爵,雖成於進祿等,其原何自?鉞伏誅於前,進祿敗露於後,反狀甚明。故陜西之人曰可殺,山西之人曰可殺,京畿中無一人不曰可殺,惟左右之人曰不可,則臣不得而知也。此不必疑一也。

且福達之形最易辨識,或取驗於頭禿,或證辨於鄉音,如李二、李俊、李三是其族,識之矣。發於戚廣之妻之口,是其孫識之矣。始認於杜文柱,是其姻識之矣。質證於韓良相、李景全,是其友識之矣。一言於高尚節、王宗美,是鄜州主人識之矣。再言於邵繼美、宗自成,是洛川主人識之矣。三言於石文舉等,是山、陜道路之人皆識之矣。此不必疑二也。

薛良怙惡,誠非善人。至所言張寅之即福達,即李午,實有明據,不得以人廢言。況福達蹤跡譎密,黠慧過人,人咸墮其術中,非良狡猾亦不能發彼陰私。從來發摘告訐之事,原不必出之敦良朴厚之人。此不當疑三也。

李玨因見薛良非善人,又見李福達無龍虎形、硃砂字,又見五臺縣張子真戶內實有張寅父子,又見崞縣左廂都無李福達、李午名,遂茍且定案,輕縱元兇。殊不知五臺自嘉靖元年黃冊始收,寅父子忽從何來?納粟拜官,其為素封必非一日之積,前此何以隱漏?崞縣在城坊既有李伏答,乃於左廂都追察,又以李午為真名,求其貫址,何可得也?則軍籍之無考,何足據也?況福達既有妖術,則龍虎形、硃砂字,安知非前此假之以惑眾,後此去之以避罪?亦不可盡謂薛良之誣矣。此不當疑四也。

京師自四方來者不止一福達,既改名張寅,又衣冠形貌似之,郭勛從而信之,亦理之所有。其為妖賊餘黨,亦意料所不能及。在勛自有可居之過,在陛下既宏議貴之恩,諸臣縱有傾勛之心,亦安能加之罪乎?此不用疑五也。

鞫獄者曰誣,必言所誣何因。曰讎,必言所讎何事。若曰薛良,讎也,則一切證佐非讎也。曰韓良相、戚廣,讎也,則高尚節、屈孔、石文舉,非讎也。曰魏泰、劉永振,讎也,則今布按府縣官非讎也。曰山、陜人,讎也,則京師道路之人非讎也。此不用疑六也。

望陛下六疑盡釋,明正福達之罪。庶群奸屏跡,宗社幸甚。

疏入,帝大怒,斥為民。其後《欽明大獄錄》刪樞疏不載。

樞少學於湛若水,深造實踐。又留心經世略,九邊及越、蜀、滇、黔險阻阨塞,無不親歷。躡屩茹草,至老不衰。隆慶初,復官。以年老,加秩致仕。會高拱憾徐階,謂階恤錄先朝建言諸臣,乃彰先帝之過,請悉停之,樞竟不錄。

杜鸞,字羽文,陜西咸寧人。正德末進士。授大理評事。嘉靖初,伏闕爭《大禮》,杖午門外。長沙盜李鑑與父華劫村聚,華誅,鑑得脫。後復行劫,捕獲之。席書時撫湖廣,劾知府宋卿故入鑑。帝遣大臣按之,言鑑盜有狀,帝命逮鑑至京。書上言:「臣以議禮忤朝臣,問官故與臣左。乞敕法司會官覆。」於是鸞會御史蘇恩再訊,無異詞,疏言:「書以惡卿故為鑑奏辨,且以議禮為言。夫大禮之議,發於聖孝。書偶一言當意,動援此以挾陛下,壓群僚。壞亂政體,莫此為甚。」帝重違書意,竟免鑑死,戍遼東。

已,復有張寅之獄。鸞與刑部郎中司馬相、御史高世魁司其牘。鸞上言:「往者李鑑之獄,陛下徇席書言,誤恩廢法,權倖遂以鬻獄為常,請託無忌。今勛謀又成矣。書曰『以議禮招怨』,勛亦曰『以議禮招怨』。書曰『欲殺鑑以仇臣』,勛亦曰:『欲殺寅以仇臣』。簧鼓聖聰,如出一口。以陛下尊親之盛典,為奸邪掩覆之深謀,將使賄賂公行,亂賊接踵,非聖朝福也。」已而桂萼等力反前獄,鸞坐除名。

初,書之欲寬李鑒也,給事中管律言:「比言事者,每借議禮為詞。或乞休,或引罪,或為人辨愬,於議禮本不相涉,而動必援引牽附,何哉?蓋小人欲中傷人,以非此不足激陛下怒;而欲自固其寵,又非此不足得陛下歡也。乞誡自今言事者,據事直陳,毋假借,以累聖德。」帝是其言,命都察院曉示百官。越二日,御史李儼以世廟成,請恤錄議禮獲罪諸臣,且請詳察是非:「議禮是而行事非者,不以是掩非。議禮非而行事是者,不以非掩是。使黨與全消,時靡有爭,則大公之治也。」未幾,給事中陳皋謨亦言:「獻皇帝追崇之禮,實出陛下至情。書輩乃貪為己功,互相黨援,恣情喜怒,作福作威。若李鑑父子,成案昭然。書曲為申救,謂『眾以議禮憾臣,因陷鑒死』。夫議禮者,朝廷之公典,合與不合,何至深讎?縱使讎書,鑒非書子弟親戚交游也,何故讎之?至郭勛黨庇奸人,請屬事露,則又代奸人妄訴,亦以議禮激眾怒為言,不至於濫恩廢法不已,豈不大可異哉!乞亟斥書、勛而置鑑重典,窮按勛請託事,使人心曉然,知權奸不足恃,國法不可干,然後逆節潛消,悻門永塞。」帝弗聽。

葉應驄,字肅卿,鄞人。正德十二年進士。授刑部主事。偕同官諫南巡,杖三十。嘉靖初,歷郎中。伏闕爭「大禮」,再下獄廷杖。

給事中潮陽陳洸素無賴。家居與知縣宋元翰不相能,令其子柱訐元翰謫戍。元翰摭洸罪及帷薄事刊布之,名《辨冤錄》。洸由是不齒於清議,尚書喬宇出之為湖廣僉事。洸初嘗言獻帝不可稱皇。而是時張璁、桂萼輩以議禮驟顯,洸乃上疏言璁等議是,宜急去本生之稱,因詆宇及文選郎夏良勝,而稱引其黨前給事中于桂、閻閎、史道,前御史曹嘉。帝即還洸等職,謫良勝於外。洸遂劾大學士費宏、尚書金獻民、趙鑑、侍郎吳一鵬、朱希周、汪偉、郎中餘才、劉天民、員外郎薛蕙、給事中鄭一鵬悉邪黨,而薦廖紀等十五人。俄又劾吏部尚書楊旦等。帝益大喜。立罷旦,擢紀代之。璁、萼輩遂引以擊異己。給事中趙漢、御史朱衣等交章劾洸,而御史張日韜、戴金、藍田又特疏論之。田並劾席書,且封上元翰《辨冤錄》。都御史王時中請罷洸聽勘。洸奏:「群奸恨臣抗議大禮,將令撫按殺臣,請遣一錦衣往」。洸意,錦衣可利誘也。得旨遣應驄及錦衣千戶李經。應驄與焚香誓天,會御史熊蘭、塗相等雜治,具上洸罪狀至百七十二條。除赦前及暖昧者勿論,當論者十三條。罪惡極,宜斬,妻離異,子柱絞。洸懼,亡詣闕申訴。帝持應驄奏不下。尚書趙鑒、副都御史張潤、給事中解一貫、御史鄭本公等連章執奏。帝不得已,始命覆核。郎中黃綰力持應驄議。書、萼為居間不能得,要璁共奏,謂洸議禮臣,為法官所中。帝入其言,命免罪為民。大理卿湯沐及鑑、一貫更爭之,不聽。未幾,「大禮」書成,並原洸妻子。應驄尋遷吉安知府,母喪歸。

六年,驄、萼益用事。而萼方掌刑部,廷臣馬錄等以劾郭勛下獄。洸謂乘此故案可反也,上書訐應驄等。萼因訟洸冤。遂逮洸、應驄、元翰、綰,而令按察使張祐等還籍候命,詞連四百人。九卿及錦衣衛廷訊,應驄對曰:「某所持者王章耳,必欲直洸,惟諸公命。」刑部尚書胡世寧等心知洸罪重,而懲前大獄,不敢執。會是日黃霧四塞,獄弗竟。次日,又大風拔木。有詔修省,不用刑。乃當應驄按事不實律,為民,元翰、綰及田等貶斥有差,洸授冠帶。霍韜再疏為洸訟不能得,洸益憾應驄。逾數年,更令人奏應驄勘獄時,酷殺無辜二十六人,下巡按李美覆勘。美言死者皆有狀,非故殺。刑部尚書許贊白應驄無罪。帝特謫應驄戍遼東。是獄也,始終八載。凡攻洸與治洸獄者無不得罪,逮捕至百數十人。天下惡萼輩奸橫,益羞言議禮臣矣。

應驄赴戍所,道經蘇州。知府治具候之,立解維去,致餽不受。十六年赦歸。明堂大享禮成,復寇帶。應驄敦行誼,好著書,數更患難氣不挫。

黃綰,息人。為刑部主事,諫南巡被杖。歷郎中,出為紹興知府,以寬大為治。被徵時,士民哭震野,爭致贐,綰止取二錢。至京,下詔獄,瘐死。隆慶初,贈太常少卿。

藍田,即墨人。爭「大禮」被杖。張璁掌都察院,考察其屬,落職歸。

解一貫,字曾唯,交城人。正德十六年進士。除工科給事中。陳講學、修德、親賢、孝親、任相、遠奸、用諫、謹令、戒欲、恤民十事。世宗嘉納之。嘉靖元年偕御史出核牧馬草場。太監閻洪等奏遣中官一人與俱,一貫言不可,乃已。還朝,劾太監谷大用、李璽奪產殃民罪,帝宥之。而內臣、勛戚所據莊田,率歸之民。帝為后父陳萬言營第,極壯麗。一貫力請裁節,復助楊廷和爭織造,皆不納。歷刑科左右給事中。雲南巡按郭楠以建言,廣東按察使張祐、副使孫懋以辱官校,皆逮治;御史方啟顏以杖死宦官家人落職;元城知縣張好古以拘責戚畹家族鐫級,一貫皆論救。忤旨,停俸。

尋進吏科都給事中。教授王價、錄事錢予勛以考察罷,假議禮希復用。一貫等言:「如此,將壞祖宗百年制。」事竟寢。張璁、桂萼日擊費宏不已,一貫偕同官言:「宏立朝行事,律以古大臣固不能無議。但入仕至今,未聞有大過。至璁、萼平生奸險,特以議禮一事偶合聖心。超擢以來,憑恃寵靈,凌轢朝士。與宏積怨已久,欲奪其位而居之。陛下以累疏俱付所司,而於其終乃曰『爾等宜各修乃職』,蓋所以陰折其奸謀者至矣。二三臣不體至意,或專攻宏,或兼攻璁、萼,不知能去宏,不能去璁、萼也。君子難進易退,小人則不然。宏恤人言,顧廉恥,猶可望以君子。璁、萼則小人之尤,何所忌憚?茍其計得行,則奸邪氣勢愈增,善類中傷無已,天下事將大有可慮者。」時鄭洛書、張錄皆論三人事,而一貫言尤切。詔下之所司。璁、萼等銜不已,竟謫開州判官以卒。

鄭洛書,字啟範,莆田人。弱冠登進士,授上海知縣,有善政。嘉靖四年召拜御史。張璁、桂萼以陳九川事訐費宏,洛書與同官鄭氣言:「九川事,人謂璁、萼與謀,固已得罪公論,而宏取與之際亦未明。夫朝廷有紀綱,大臣重進退,宏、璁、萼皆不可不去。宏不去,則有持祿保位之誚,璁、萼不去,亦冒蹊田奪牛之嫌。」詔責洛書妄言。

帝賜尚書趙鑑、席書詩翰,洛書言:「陛下眷禮大臣,此虞廷賡歌之風也。願推此心以念舊。如致仕大臣劉健、謝遷、林俊、孫交等,特降宸章,咨訪時政,則聖德益宏。又推此心以赦過。如遷謫豐熙、劉濟、餘寬、王元正等,特垂仁恩,量與牽復,則聖度益廣。」報聞。李福達獄起,帝將親鞫之,洛書曰:「陛下操獨斷之威,使法官盡得罪,雖有張釋之、于定國,不獲抗辨於人主之前,何以使刑罰中!」帝怒,將罪之,楊一清力解而止。尋出視南畿學政,道聞喪歸。

十二年京察事竣,更命科道官互糾,洛書被劾落職。給事中饒秀為御史所劾,無所泄憤,復劾洛書及王重賢等九人貪污闒茸。重賢等皆降黜。時論駭之。洛書家居再踰歲卒,年三十九。子開,往依上海。上海人治田百畝資之。歲一至,收其入以歸。

張錄,字宗制,城武人。正德六年進士。授太常博士,擢御史。嘉靖初,伏闕爭「大禮」,下獄廷杖。出按畿輔,劾宣府諸將失事,皆伏辜。西域魯迷貢獅子、西牛方物,言所貢玉石計費二萬三千餘金,往來且七年,邀中國重賞。錄言:「明王不貴異物。今二獅日各飼一羊,是歲用七百餘羊也。牛食芻菽,今乃食果餌,則食人之食矣。願返其獻,歸其人,薄其賞,以阻希望心。」帝不能用。

張璁擢兵部侍郎,錄與諸御史爭之,不聽。璁與桂萼屢攻費宏,錄言:「今水旱相仍,變異迭出,正臣工修省時。諸人為國股肱,相傾排若此,欲弭災變,不亦難乎?乞並黜三人,以回天譴。」帝為戒諭璁、萼。後璁以侍郎總臺事,修前憾。言錄不諳憲體,遂罷歸。家居二十年,卒。

陸粲,字子餘,長州人。少謁同里王鏊,鏊異之曰:「此子必以文名天下。」嘉靖五年成進士,選庶吉士。七試皆第一。張璁、桂萼盡出庶吉士為部曹、縣令,粲以才獨得工科給事中。勁挺敢言。疏言:「我朝太祖至宣宗,大臣造膝陳謀,不啻家人父子。自英宗幼沖,大臣為權宜計,常朝奏事,先日擬旨,其餘政事具疏封進,沿襲至今。今陛下銳意圖治,願每日朝罷,退御便殿,延見大臣;侍從臺諫輪日奏對;撫按籓臬廷辭入謝,召訪便宜;復妙選博聞有道之士,更番入直,講論經史,如仁宗弘文閣故事。則上下情通,而天下事畢陳於前矣。」帝不能用。既言資格獨重進士,致貢舉無上進階,州縣教職過輕,王官終身禁錮,皆宜變通。因陳久任使、慎考察、汰冗官諸事,而終之以復制科,仿唐、宋法,數歲一舉,以待異才:「高者儲之禁近,其次分置諸曹,先有官者遞進,庶人才畢出,野無遺賢。」

尋偕御史郗元洪清核馬房錢穀。抗疏折御馬太監閻洪,宿弊為清。與同官劉希簡爭張福達獄。帝怒,俱下詔獄。杖三十,釋還職。事具《熊浹傳》。

張璁、桂萼並居政府,專擅朝事。給事中孫應奎、王準發其私,帝猶溫旨慰諭。粲不勝憤,上疏曰:

璁、萼,兇險之資,乖僻之學。曩自小臣贊大禮,拔置近侍,不三四年位至宰弼。恩隆寵異,振古未聞。乃敢罔上逞私,專權招賄,擅作威福,報復恩仇。璁狠愎自用,執拗多私。萼外若寬迂,中實深刻。忮忍之毒一發於心,如蝮蛇猛獸,犯者必死。臣請姑舉數端言之。

萼受尚書王瓊賂遺鉅萬,連章力薦,璁從中主之,遂得起用。昌化伯邵傑,本邵氏養子,萼納重賄,竟使奴隸小人濫襲伯爵。萼所厚醫官李夢鶴假托進書,夤緣受職,居室相鄰,中開便戶往來,常與萼家人吳從周等居間。又引鄉人周時望為選郎,交通鬻爵。時望既去,胡森代之。森與主事楊麟、王激又輔臣鄉里親戚也。

銓司要地,盡布私人。典選僅踰年,引用鄉故,不可悉數。如致仕尚書劉麟,其中表親也。侍郎嚴嵩,其子之師也。僉都御史李如圭,由按察使一轉徑入內臺,南京太僕少卿夏尚朴,由知府期月遂得清卿,禮部員外張敔假歷律而結知,御史戴金承風搏擊,甘心鷹犬,皆萼姻黨,相與朋比為奸者也。禮部尚書李時柔和善逢,猾狡多智,南京禮部尚書黃綰曲學阿世,虛談眩人,諭德彭澤夤緣改秩,躐玷清華,皆陰厚於璁而陽附於萼者也。

璁等威權既盛,黨與復多,天下畏惡,莫敢訟言。不亟去之,兇人之性不移,將來必為社稷患。

帝大感悟,立下詔暴璁、萼罪狀,罷其相;而以粲不早發,下之吏。

既而詹事霍韜力詆粲,謂楊一清嗾之。希簡言:「璁、萼去位由聖斷。且使犬謂之嗾,韜以言官比之犬,侮朝廷。」而帝竟納韜言,召璁還,奪一清官,下希簡詔獄,釋還職,謫粲貴州都鎮驛丞。

稍遷永新知縣。前後獲盜數百人,姦猾屏跡。久之,以念母乞歸。論薦者三十餘疏,皆報罷。霍韜亦薦粲,粲曰:「天下事大壞憸人手,尚欲以餘波污我耶?」母歿,毀甚,未終喪而卒。

劉希簡,字以順,漢州人。進士。除行人。為工科給事中,甫五月,兩以直言得罪,聲大振。久之,謫縣丞。終鞏昌知府。

王準。字子推,世籍秦府儀衛司。準以進士授知縣。為禮科給事中,巡視京營,劾郭勛專恣罪。明年,劾璁、萼引私人。璁、萼罷,準亦下吏,謫富民典史。稍遷知縣。都御史汪鋐萘希璁指,以考察罷之。

邵經邦,字仲德,仁和人。正德十六年進士。授工部主事。榷荊州稅,甫三月,稅額滿,遂啟關任商舟往來。進員外郎。

嘉靖八年冬十月,日有食之。經邦時官刑部,上疏曰:

茲者正陽之月,有日食之異。質諸《小雅十月》之篇,變象懸符。說《詩》者謂陰壯之甚,由不用善人,而其咎專歸皇父。然則今之調和變理者,得無有皇父其人乎?邇陛下納陸粲言,命張璁、桂萼致仕。尋以璁議禮有功,復召輔政。人言籍籍,陛下莫之恤也。乃天變若此,安可勿畏?

夫議禮與臨政不同。議禮貴當,臨政貴公。正皇考之徽稱,以明父子之倫,禮之當也。雖排眾論,任獨見,而不以為偏。若夫用人行政,則當辨別忠邪,審量才力,與天下之人共用之,乃為公耳。今陛下以璁議禮有功,不察其人,不揆其才,而加之大任,似私議禮之臣也。私議禮之臣,是不以所議者為公禮也。夫禮唯至公,乃可萬世不易。設近於私,則固可守也,亦可變也。陛下果以尊親之典為至當,而欲子孫世世守之乎?則莫若於諸臣之進退,一付諸至公,優其賚予,全其終始,以答其議禮之功,而博求海內碩德重望之賢,以弼成正大光明之業,則人心定,天道順,俾萬年之後,廟號世宗,子孫百世不遷,顧不偉歟?如徒加以非分之任,使之履盈蹈滿,犯天人之怒,亦非璁等福也。

帝大怒,立下鎮撫司拷訊。獄上,請送法司擬罪。帝曰:「此非常犯,不必下法司。」遂謫戍福建鎮海衛。十六年,皇子生,大赦。惟經邦與豐熙等八人不在赦例。

經邦之戍所,閉戶讀書。與熙及同戍陳九川,時相討論。居鎮海三十七年卒。閩人立寓賢祠祀三人。隆慶初復官。

劉世揚,字實甫,閩人。正德十二年進士。改庶吉士,除刑科給事中。世宗即位,議加興獻帝皇號,世揚疏諫。都察院牒司禮監,攝中官吳善良。帝手批原牒付刑科,以善良付司禮。世揚言:「祖宗制,凡降詔旨必書於題奏疏揭,或登聞鼓狀,乃發六科,宣於諸曹。或國有大事,上命先發,諸曹必補牘,於次日早朝進之,無竟批文牘者。今旨從中出,褻天語,更舊制,不可。」帝不聽。已,列先朝直臣舒芬、馬汝驥、王思、王應軫、張原等二十人,請加恩以旌忠直,諸臣各進秩一等。嘗因災異,世揚請仿古人幾杖箴銘之義,取聖賢格言書殿廡,帝納之。

歷吏科左給事中,進都給事中。與同官李仁劾詹事顧鼎臣汙佞,且言今日詹事即他日輔臣。帝怒,詰詹事進輔臣,出何典例?世揚等引罪。帝怒不解,予杖,下詔獄,既乃得釋。帝以久旱躬禱,世揚言在獄繫囚及建言謫戍諸臣怨咨之氣,上干天和,請悉疏釋。帝不能用。張璁、桂萼被劾罷,帝責諫官不言。世揚等乃盡劾璁、萼黨尚書王瓊而下數十人,章下吏部。而尚書方獻夫亦璁、萼黨也,但去編修金璐、御史敖鉞、太僕丞姚奎、郎中劉汝輗、員外郎張敔、郭憲、待詔葉幼學、儲良才八人而已。未幾,復偕同官趙漢等陳修省八事。中言:「大學士石瑤貞介,歿未易名。尚書李鐩,國之盜臣,身後遺金得謚。給事中鄭一鵬坐論楊一清再杖削職,一清敗,一鵬宜復官。」

世揚發璁、萼黨,見憾於璁,一鵬又嘗忤璁、萼。會璁已再相,而瑤實前賜謚,璁因激帝怒,謂給事言皆妄。乃謫世揚江西布政司照磨,停漢等俸,然鐩謚亦由此奪。世揚屢遷河南提學僉事。告歸,卒。

趙漢,字鴻逵,平湖人。正德六年進士。授建昌推官。擢南京戶科給事中,改兵科。嘉靖初,尚書林俊以執奏獄囚李鳳陽,被旨詰責。漢因言:「太監崔文亂政,巧逞奸欺,不特庇一李鳳陽而已。工部尚書趙璜發文家人罪。文輒捕其諜者,痛杖幾死,曰『此杖寄與趙尚書』,其無狀至此。望急譴逐,毋為新政累。」不聽。已,哭爭「大禮」,繫詔獄廷杖。

歷吏科左給事中。以疾去。起故官,遷工科都給事中。疏言:「內閣桂萼、翟鑾稱病三月,未嘗以曠職懇辭。張璁久專政權,亦未聞引賢共濟。乞諭鑾、萼亟去,簡用兩京大臣及家居耆舊,以分璁任。」上摘其訛字詰之,諭璁毋避,趣赴閣。璁因言漢忠謀,宜令備列堪內閣者。帝即令漢舉所欲用,漢惶恐言:「臣欲璁引賢,無私主。」帝怒,責漢對不以實,趣以名上。漢益懼,言:「輔臣簡命,出自朝廷,非小臣所敢預。」帝乃宥之,奪俸一月。尋出為陜西右參政,告歸。久之,以故官起山西。不數月復致仕。

子伊,廣西副使。年四十,即以養父歸。屢征不起。

魏良弼,字師說,新建人。嘉靖二年進士。授松陽知縣,召拜刑科給事中。採木侍郎黃衷事竣歸家,乞致仕,未許。緝事者奏衷潛入京師。帝怒,奪衷職。良弼言衷大臣,入都豈能隱,乞正言者欺罔罪,不報。

張璁、桂萼初罷相,詔察其黨。給事中劉世揚等議及良弼。以吏部言,得留。尋命巡視京營。劾罷提督五軍營保定伯梁永福、太僕卿曾直,罪武定侯郭勛家奴,論團營兵政之弊,又請發銀米振京師饑,直聲大著。會南京御史馬等以劾吏部尚書王瓊被逮,良弼請釋之。帝怒,並下詔獄。論贖還職,仍奪俸一年。三遷至禮科都給事中。

十一年八月,彗星見東井,芒長丈餘。良弼引占書言:「彗星晨見東方,君臣爭明。彗孛出井,奸臣在側。大學士張孚敬專橫竊威福,致奸星示異,亟宜罷黜。」孚敬奏良弼挾私。帝已疑孚敬,兩疏皆報聞。給事中秦鰲疏再入,孚敬竟罷去。踰月,良弼復偕同官劾吏部尚書汪鋐。帝方向鋐,奪良弼俸。鋐、孚敬俱恨良弼。

明年元日,副都御史王應鵬坐事下詔獄。良弼言履端之始,不宜以微過繫大臣。帝怒,再下詔獄。獄卒訝曰:「公又來耶!」為垂涕。尋復職,奪俸。時孚敬復起柄政,與鋐修前郤,以考察後命科道官互糾,又奏上十一人,又不及良弼。孚敬益怒,擬旨切責,令吏部再考。鋐乃別糾二十六人,而良弼及秦鰲、葉洪皆前劾孚敬、鋐者,中外大駭。良弼竟坐不謹削籍。隆慶初,詔起廢籍。以年老即家拜太常少卿,致仕,卒。天啟初,追謚忠簡。

葉洪,字子源,德州人。嘉靖八年進士。授戶科給事中。十一年肇舉祈穀禮於圜丘,帝不親祀。洪疏諫,帝責洪妄言。尋巡視京營,進工科右給事中。汪金宏遷吏部尚書,洪極論其奸,忤旨奪俸。明年考察,鋐修怨,遂坐洪浮躁,貶寧國縣丞。居二年,復以大計奪其職。言者屢訟冤,不復用。

秦鰲,字子元,崑山人。嘉靖五年進士。授行人。擢兵科給事中。劾魏國公徐鵬舉、中官賴義不法狀,義罷還。彗星見,劾張孚敬妒賢病國,擬議詔旨,輒引以自歸。帝遂罷孚敬。已,孚敬再相。汪鋐承風指以考察謫鰲東陽縣丞。屢遷福建右參議。卒官。

又有張寅者,太倉人。嘉靖初進士。歷南京御史。嘗劾禮部侍郎黃綰十罪。比張孚敬罷政,寅言其憸邪蠹政,不可悉數,請追所賜封誥、銀章之屬,明正其辟。並劾左都御史汪鋐陰賊邪媚。帝怒,謫高唐判官。屢遷南京文選郎中。會簡宮僚,改春坊右司直兼翰林院檢討。未幾,被劾罷。

贊曰:《書》曰:「非佞折獄,惟良折獄,罔非在中。」又曰:「明啟刑書,胥占咸庶中。」正言折獄之不可不得其中也。張寅、李金盬,罪狀昭然。中於郭勛、席書之說,廷臣獲罪,而寅還職,金盬宥死。陳洸罪至百七十二條,竟得免死,而猶上書訟冤,凡攻洸之惡與治洸之獄者,逮捕至百數十人。皆由議禮觸眾怒,一言有以深入帝隱。甚矣,佞人之可畏也。夫反成案似於明,出死罪似於仁,而不知其借端報復,刑罰失中。佞良之辨,可弗審歟!


\end{pinyinscope}