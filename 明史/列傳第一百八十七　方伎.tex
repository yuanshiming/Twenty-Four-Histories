\article{列傳第一百八十七 方伎}

\begin{pinyinscope}
左氏載醫和、緩、梓慎、裨灶、史蘇之屬,甚詳且核。下逮巫祝,亦往往張其事以神之。論者謂之浮誇,似矣。而《史記》傳扁鵲、倉公,日者,龜策,至黃石、赤松、倉海君之流,近於神仙荒忽,亦備錄不遺。范蔚宗乃以方術名傳。夫藝人術士,匪能登乎道德之途。然前民利用,亦先聖之緒餘,其精者至通神明,參造化,詎曰小道可觀已乎!

明初,周顛、張三豐之屬,蹤跡秘幻,莫可測識,而震動天子《左傳·昭公十年》有「非知之實難,將在行之」的觀點,老,要非妄誕取寵者所可幾。張中、袁珙,占驗奇中。夫事有非常理所能拘者,淺見鮮聞不足道也。醫與天文皆世業專官,亦本《周官》遺意。攻其術者,要必博極於古人之書,而會通其理,沈思獨詣,參以考驗,不為私智自用,乃足以名當世而為後學宗。今錄其最異者,作《方伎傳》。真人張氏,道家者流,而世蒙恩澤,其事蹟關當代典故,撮其大略附於篇。

○滑壽葛乾孫呂復倪維德周漢卿王履周顛張中張三豐袁珙子忠徹戴思恭盛寅皇甫仲和仝寅吳傑許紳王綸凌雲李玉李時珍繆希雍周述學張正常劉淵然等

滑壽,字伯仁,先世襄城人,徙儀真,後又徙餘姚。幼警敏好學,能詩。京口王居中,名醫也。壽從之學,授《素問》、《難經》。既卒業,請於師曰:「《素問》詳矣,多錯簡。愚將分藏象、經度等為十類,類抄而讀之。《難經》又本《素問》、《靈樞》,其間榮衛藏府與夫經絡腧穴,辨之博矣,而缺誤亦多。愚將本其義旨,注而讀之可乎?」居中躍然稱善。自是壽學日進。壽又參會張仲景、劉守真、李明之三家而會通之,所治疾無不中。既學鍼法於東平高洞陽,嘗言:「人身六脈雖皆有係屬,惟督任二經,則苞乎腹背,有專穴。諸經滿而溢者,此則受之,宜與十二經並論。」乃取《內經骨空》諸論及《靈樞篇》所述經脈,著《十四經發揮》三卷,通考隧穴六百四十有七。他如《讀傷寒論抄》、《診家樞要》、《痔瘺篇》又採諸書《本草》為《醫韻》,皆有功於世。晚自號攖寧生。江、浙間無不知攖寧生者。年七十餘,容色如童孺,行步蹻捷,飲酒無算。天台朱右摭其治疾神效者數十事,為作傳,故其著述益有稱於世。

葛乾孫,字可久,長洲人。父應雷,以醫名。時北方劉守真、張潔古之學未行於南。有李姓者,中州名醫,官吳下,與應雷談論,大駭歎,因授以張、劉書。自是江南有二家學。乾孫體貌魁碩,好擊刺戰陣法。後折節讀書,兼通陰陽、律曆、星命之術。屢試不偶,乃傳父業。然不肯為人治疾,或施之,輒著奇效,名與金華朱丹溪埒。富家女病四支痿痺,目瞪不能食,眾醫治罔效。乾孫命悉去房中香奩、流蘇之屬,掘地坎,置女其中。久之,女手足動,能出聲。投藥一丸,明日女自坎中出矣。蓋此女嗜香,脾為香氣所蝕,故得是癥。其療病奇中如此。

呂復,字元膺,鄞人。少孤貧,從師受經。後以母病求醫,遇名醫衢人鄭禮之,遂謹事之,因得其古先禁方及色脈藥論諸書,試輒有驗。乃盡購古今醫書,曉夜研究,自是出而行世,取效若神。其於《內經》、《素問》、《靈樞》、《本草》、《難經》、《傷寒論》、《脈經》、《脈訣》、《病原論》、《太始天元玉冊公誥》、《六微旨》、《五常政》、《玄珠密語》、《中藏經》、《聖濟經》等書,皆有辨論。前代名醫如扁鵲、倉公、華佗、張仲景至張子和、李東垣諸家,皆有評騭。所著有《內經或問》、《靈樞經脈箋》、《五色診奇眩》、《切脈樞要》、《運氣圖說》、《養生雜言》諸書甚眾。浦江戴良採其治效最著者數十事,為醫案。歷舉仙居、臨海教諭,台州教授,皆不就。

倪維德,字仲賢,吳縣人。祖、父皆以醫顯。維德幼嗜學,已乃業醫,以《內經》為宗。病大觀以來,醫者率用裴宗元、陳師文《和劑局方》,故方新病多不合。乃求金人劉完素、張從正、李杲三家書讀之,出而治疾,無不立效。周萬戶子,八歲昏眊,不識饑飽寒暑,以土炭自塞其口。診之曰:「此慢脾風也。脾藏智,脾慢則智短。」以疏風助脾劑投之,即愈。顧顯卿右耳下生癭,大與首同,痛不可忍。診之曰:「此手足少陽經受邪也。」飲之藥,踰月愈。劉子正妻病氣厥,或哭或笑,人以為崇。診之曰:「兩手脈俱沉,胃脘必有所積,積則痛。」問之果然,以生熟水導之,吐痰涎數升愈。盛架閣妻左右肩臂奇癢,延及頭面,不可禁,灼之以艾,則暫止。診之曰:「左脈沉,右脈浮且盛,此滋味過盛所致也。」投以劑,旋愈。林仲實以勞得熱疾,熱隨日出入為進退,暄盛則增劇,夜涼及雨則否,如是者二年。診之曰:「此七情內傷,陽氣不升,陰火漸熾。故溫則進,涼則退。」投以東垣內傷之劑,亦立愈。他所療治,多類此。常言:「劉、張二氏多主攻,李氏惟調護中氣主補,蓋隨時推移,不得不然。」故其主方不執一說。常患眼科雜出方論,無全書,著《元機啟微》,又校訂《東垣試效方》,並刊行於世。洪武十年卒,年七十五。

周漢卿,松陽人。醫兼內外科,鍼尤神。鄉人蔣仲良,左目為馬所踶,睛突出如桃。他醫謂係絡已損不可治。漢卿封以神膏,越三日復故。華州陳明遠瞽十年。漢卿視之,曰:「可鍼也。」為翻睛刮翳,焱然辨五色。武城人病胃痛,奮擲乞死。漢卿納藥於鼻,俄噴赤蟲寸許,口眼悉具,痛旋止。馬氏婦有娠,十四月不產,尪且黑。漢卿曰:「此中蠱,非娠也。」下之,有物如金魚,病良已。永康人腹疾,佝僂行。漢卿解衣視之,氣衝起腹間者二,其大如臂。刺其一,砉然鳴,又刺其一亦如之,加以按摩,疾遂愈。長山徐嫗癇疾,手足顫掉,裸而走,或歌或笑。漢卿刺其十指端,出血而痊。錢塘王氏女生瘰癆,環頭及腑,凡十九竅。竅破白沈出,將死矣。漢卿為剔竅母深二寸,其餘烙以火,數日結痂愈。山陰楊翁項有疣如瓜大,醉仆階下,潰血不能止。疣潰者必死。漢卿以藥糝其穴,血即止。義烏陳氏子腹有塊,捫之如罌。漢卿曰:「此腸癰也。」用大鍼灼而刺之,入三寸許,膿隨鍼迸出有聲,愈。諸暨黃生背曲,須杖行。他醫皆以風治之,漢卿曰:「血澀也。」刺兩足崑崙穴,頃之投杖去。其捷效如此。

王履,字安道,崑山人。學醫於金華朱彥修,盡得其術。嘗謂張仲景《傷寒論》為諸家祖,後人不能出其範圍。且《素問》云「傷寒為病熱」,言常不言變,至仲景始分寒熱,然義猶未盡。乃備常與變,作《傷寒立法考》。又謂《陽明篇》無目痛,《少陰篇》言胸背滿不言痛,《太陰篇》無嗌乾,《厥陰篇》無囊縮,必有脫簡。乃取三百九十七法,去其重者二百三十八條,復增益之,仍為三百九十七法。極論內外傷經旨異同,併《中風》、《中暑辨》,名曰《溯洄集》,凡二十一篇。又著《百病鉤玄》二十卷,《醫韻統》一百卷,醫家宗之。履工詩文,兼善繪事。嘗遊華山絕頂,作圖四十幅,記四篇,詩一百五十首,為時所稱。

自滑壽以下五人,皆生於元,至明初始卒。

周顛,建昌人,無名字。年十四,得狂疾,走南昌市中乞食,語言無恒,皆呼之曰顛。及長,有異狀,數謁長官,曰「告太平」。時天下寧謐,人莫測也。後南昌為陳友諒所據,顛避去。太祖克南昌,顛謁道左。洎還金陵,顛亦隨至。一日,駕出,顛來謁。問「何為」,曰「告太平」。自是屢以告。太祖厭之,命覆以巨缸,積薪煆之。薪盡啟視,則無恙,頂上出微汗而已。太祖異之,命寄食蔣山僧寺。已而僧來訴,顛與沙彌爭飯,怒而不食且半月。太祖往視顛,顛無饑色。乃賜盛饌,食已閉空室中,絕其粒一月,比往視,如故。諸將士爭進酒饌,茹而吐之,太祖與共食則不吐。太祖將征友諒,問曰:「此行可乎?」對曰:「可。」曰:「彼已稱帝,克之不亦難乎?」顛仰首視天,正容曰:「天上無他座。」太祖攜之行,舟次安慶,無風,遣使問之,曰:「行則有風。」遂命牽舟進,須臾風大作,直抵小孤。太祖慮其妄言惑軍心,使人守之。至馬當,見江豚戲水,歎曰:「水怪見,損人多。」守者以告。太祖惡之,投諸江。師次湖口,顛復來,且乞食。太祖與之食,食已,即整衣作遠行狀,遂辭去。友諒既平,太祖遣使往廬山求之,不得,疑其仙去。洪武中,帝親撰《周顛仙傳》,紀其事。

張中,字景華,臨川人。少應進士舉不第,遂放情山水。遇異人,授數學,談禍福,多奇中。太祖下南昌,以鄧愈薦召至,賜坐。問曰:「予下豫章,兵不血刃,此邦之人其少息乎?」對曰:「未也。旦夕此地當流血,廬舍毀且盡,鐵柱觀亦僅存一殿耳。」未幾,指揮康泰反,如其言。尋又言國中大臣有變,宜豫防。至秋,平章邵榮、參政趙繼祖伏甲北門為亂,事覺伏誅。陳友諒圍南昌三月,太祖伐之,召問之。曰:「五十日當大勝,亥子之日獲其渠帥。」帝命從行,舟次孤山,無風不能進。乃以洞玄法祭之,風大作,遂達鄱陽。大戰湖中,常遇春孤舟深入,敵舟圍之數重,眾憂之。曰:「無憂,亥時當自出。」已而果然。連戰大勝,友諒中流矢死,降其眾五萬。自啟行至受降,適五十日。始南昌被圍,帝問「何日當解」,曰「七月丙戌」。報至,乃乙酉,蓋術官算曆,是月差一日,實在丙戍也。其占驗奇中,多若此。為人狷介寡合。與之言,稍涉倫理,輒亂以他語,類佯狂玩世者。嘗好戴鐵冠,人稱為鐵冠子云。

張三豐,遼東懿州人,名全一,一名君寶,三豐其號也。以其不飾邊幅,又號張邋遢。頎而偉,龜形鶴背,大耳圓目,鬚髯如戟。寒暑惟一衲一蓑,所啖,升斗輒盡,或數日一食,或數月不食。盡經目不忘,游處無恒,或云能一日千里。善嬉諧,旁若無人。嘗游武當諸巖壑,語人曰:「此山異日必大興。」時五龍、南巖、紫霄俱毀於兵,三豐與其徒去荊榛,辟瓦礫,創草廬居之,已而舍去。

太祖故聞其名,洪武二十四年遣使覓之,不得。後居寶雞之金臺觀。一日自言當死,留頌而逝,縣人共棺殮之。及葬,聞棺內有聲,啟視則復活。乃遊四川,見蜀獻王。復入武當,歷襄、漢,蹤跡益奇幻。永樂中,成祖遣給事中胡濙偕內侍朱祥齎璽書香幣往訪,遍歷荒徼,積數年不遇。乃命工部侍郎郭璡、隆平侯張信等,督丁夫三十餘萬人,大營武當宮觀,費以百萬計。既成,賜名太和太岳山,設官鑄印以守,竟符三豐言。

或言三豐金時人,元初與劉秉忠同師,後學道於鹿邑之太清宮,然皆不可考。天順三年,英宗賜誥,贈為通微顯化真人,終莫測其存亡也。

袁珙,字廷玉,鄞人。高祖鏞,宋季舉進士。元兵至,不屈,舉家十七人皆死。父士元,翰林檢閱官。珙生有異稟,好學能詩。嘗遊海外洛伽山,遇異僧別古崖,授以相人術。先仰視皎日,目盡眩,布赤黑豆暗室中,辨之,又懸五色縷窗外,映月別其色,皆無訛,然後相人。其法以夜中燃兩炬視人形狀氣色,而參以所生年月,百無一謬。

珙在元時已有名,所相士大夫數十百,其於死生禍福,遲速大小,並刻時日,無不奇中。南臺大夫普化帖木兒,由閩海道見珙。珙曰:「公神氣嚴肅,舉動風生,大貴驗也。但印堂司空有赤氣,到官一百十四日當奪印。然守正秉忠,名垂後世,願自勉。」普署臺事於越,果為張士誠逼取印綬,抗節死。見江西憲副程徐曰:「君帝座上黃紫再見,千日內有二美除。但冷笑無情,非忠節相也。」徐於一年後拜兵部侍郎,擢尚書。又二年降於明,為吏部侍郎。嘗相陶凱曰:「君五岳朝揖而氣色未開,五星分明而光澤未見,宜藏器待時。不十年以文進,為異代臣,官二品,其在荊、揚間乎!」凱後為禮部尚書、湖廣行省參政。其精類如此。洪武中,遇姚廣孝於嵩山寺,謂之曰:「公,劉秉忠之儔也,幸自愛。」後廣孝薦於燕王,召至北平。王雜衛士類己者九人,操弓矢,飲肆中。珙一見即前跪曰:「殿下何輕身至此。」九人者笑其謬,珙言益切。王乃起去,召珙宮中,諦視曰:「龍行虎步,日角插天,太平天子也。年四十,鬚過臍,即登大寶矣。」已見籓邸諸校卒,皆許以公侯將帥。王慮語洩,遣之還。及即位,召拜太常寺丞,賜冠服、鞍馬、文綺、寶鈔及居第。帝將建東宮,而意有所屬,故久不決。珙相仁宗曰:「天子也。」相宣宗曰:「萬歲天子。」儲位乃定。

珙相人即知其心術善惡。人不畏義,而畏禍患,往往因其不善導之於善,從而改行者甚多。為人孝友端厚,待族黨有恩。所居鄞城西,繞舍種柳,自號柳莊居士,有《柳莊集》。永樂八年卒,年七十有六。賜祭葬,贈太常少卿。

子忠徹,字靜思。幼傳父術。從父謁燕王,王宴北平諸文武,使忠徹相之。謂都督宋忠面方耳大,身短氣浮,布政使張昺面方五小,行步如蛇,都指揮謝貴擁腫蚤肥而氣短,都督耿瓛顴骨插鬢,色如飛火,僉都御史景清身短聲雄,於法皆當刑死。王大喜,起兵意益決。及為帝,即召授鴻臚寺序班,賜齎甚厚。遷尚寶寺丞,已,改中書舍人,扈駕北巡。駕旋,仁宗監國,為讒言所中,帝怒,榜午門,凡東宮所處分事,悉不行。太子憂懼成疾,帝命蹇義、金忠偕忠徹視之。還奏,東宮面色青藍,驚憂象也,收午門榜可愈。帝從之,太子疾果已。帝嘗屏左右,密問武臣朱福、朱能、張輔、李遠、柳升、陳懋、薛祿,文臣姚廣孝、夏原吉、蹇義及金忠、呂震、方賓、吳中、李慶等禍福,後皆驗。九載秩滿,復為尚寶司丞,進少卿。

禮部郎周訥自福建還,言閩人祀南唐徐知諤、知誨,其神最靈。帝命往迎其像及廟祝以來,遂建靈濟宮於都城,祀之。帝每遘疾,輒遣使問神。廟祝詭為仙方以進,藥性多熱,服之輒痰壅氣逆,多暴怒,至失音,中外不敢諫。忠徹一日入侍,進諫曰:「此痰火虛逆之癥,實靈濟宮符藥所致。」帝怒曰:「仙藥不服,服凡藥耶?」忠徹叩首哭,內侍二人亦哭。帝益怒,命曳二內侍杖之,且曰:「忠徹哭我,我遂死耶?」忠徹惶懼,趨伏階下,良久始解。帝識忠徹於籓邸,故待之異於外臣。忠徹亦以帝遇己厚,敢進讜言,嘗諫外國取寶之非,武臣宜許行服,衍聖公誥宜改賜玉軸,聞之韙之。

宣德初,睹帝容色曰:「七日內,宗室當有謀叛者。」漢王果反。嘗坐事下吏罰贖。正統中,復坐事下吏休致。二十餘年卒,年八十有三。

忠徹相術不殊其父,世所傳軼事甚多,不具載。其相王文,謂「面無人色,法曰瀝血頭」。相于謙,謂「目常上視,法曰望刀眼」。後果如其言。然性陰險,不如其父,與群臣有隙,即緣相法於上前齮齕之。頗好讀書,所著有《人相大成》及《鳳池吟稿》、《符臺外集》,載元順帝為瀛國公子云。

戴思恭,字原禮,浦江人,以字行。受學於義烏朱震亨。震亨師金華許謙,得朱子之傳,又學醫於宋內侍錢塘羅知悌。知悌得之荊山浮屠,浮屠則河間劉守真門人也。震亨醫學大行,時稱為丹溪先生。愛思恭才敏,盡以醫術授之。洪武中,徵為御醫,所療治立效,太祖愛重之。燕王患瘕,太祖遣思恭往治,見他醫所用藥良是,念何以不效,乃問王何嗜。曰:「嗜生芹。」思恭曰:「得之矣。」投一劑,夜暴下,皆細蝗也。晉王疾,思恭療之愈。已,復發,即卒。太祖怒,逮治王府諸醫。思恭從容進曰:「臣前奉命視王疾,啟王曰:『今即愈,但毒在膏肓,恐復作不可療也。』今果然矣。」諸醫由是免死。思恭時已老,風雨輒免朝。太祖不豫,少間,出御右順門,治諸醫侍疾無狀者,獨慰思恭曰:「汝仁義人也,毋恐。」已而太祖崩,太孫嗣位,罪諸醫,獨擢思恭太醫院使。永樂初,以年老乞歸。三年夏,復徵入,免其拜,特召乃進見。其年冬,復乞骸骨,遣官護送,齎金幣,踰月而卒,年八十有二,遣行人致祭。所著有《證治要訣》、《證治類元》、《類證用藥》諸書,皆DA括丹溪之旨。又訂正丹溪《金匱鉤玄》三卷,附以己意。人謂無愧其師云。

盛寅,字啟東,吳江人。受業於郡人王賓。初,賓與金華戴原禮游,冀得其醫術。原禮笑曰:「吾固無所吝,君獨不能少屈乎?」賓謝曰:「吾老矣,不能復居弟子列。」他日伺原禮出,竊發其書以去,遂得其傳。將死,無子,以授寅。寅既得原禮之學,復討究《內經》以下諸方書,醫大有名。永樂初,為醫學正科。坐累,輸作天壽山。列侯監工者,見而奇之,令主書算。先是有中使督花鳥於江南,主寅舍,病脹,寅愈之。適遇諸途,驚曰:「盛先生固無恙耶!予所事太監,正苦脹,盍與我視之。」既視,投以藥立愈。會成祖較射西苑,太監往侍。成祖遙望見,愕然曰:「謂汝死矣,安得生?」太監具以告,因盛稱寅,即召入便殿,令診脈。寅奏,上脈有風濕病,帝大然之,進藥果效,遂授御醫。一日,雪霽,召見。帝語白溝河戰勝狀,氣以甚厲。寅曰:「是殆有天命耳。」帝不懌,起而視雪。寅復吟唐人詩「長安有貧者,宜瑞不宜多」句,聞者咋舌。他日,與同官對弈御藥房。帝猝至,兩人斂枰伏地,謝死罪。帝命終之,且坐以觀,寅三勝。帝喜,命賦詩,立就。帝益喜,賜象牙棋枰並詞一闋。帝晚年猶欲出塞,寅以帝春秋高,勸毋行。不納,果有榆木川之變。

仁宗在東宮時,妃張氏經期不至者十月,眾醫以妊身賀。寅獨謂不然,出言病狀。妃遙聞之曰:「醫言甚當,有此人何不令早視我。」及疏方,乃破血劑。東宮怒,不用。數日病益甚,命寅再視,疏方如前。妃令進藥,而東宮慮墮胎,械寅以待。已而血大下,病旋愈。當寅之被繫也,闔門惶怖曰:「是殆磔死。」既三日,紅仗前導還邸舍,賞賜甚厚。

寅與袁忠徹素為東宮所惡,既愈妃疾,而怒猶未解,懼甚。忠徹曉相術,知仁宗壽不永,密告寅,寅猶畏禍。及仁宗嗣位,求出為南京太醫院。宣宗立,召還。正統六年卒。兩京太醫院皆祀寅。寅弟宏亦精藥論,子孫傳其業。

初,寅晨直御醫房,忽昏眩欲死,募人療寅,莫能應。一草澤醫人應之,一服而愈。帝問狀,其人曰:「寅空心入藥房,猝中藥毒。能和解諸藥者,甘草也。」帝問寅,果空腹入,乃厚賜草澤醫人。

皇甫仲和,睢州人。精天文推步學。永樂中,成祖北征,仲和與袁忠徹扈從。師至漠北,不見寇,將引還,命仲和占之,言:「今日未申間,寇當從東南來。王師始卻,終必勝。」忠徹對如之。比日中不至,復問,二人對如初。帝命械二人,不驗,將誅死。頃之,中官奔告曰:「寇大至矣。」時初得安南神炮,寇一騎直前,即以炮擊之,一騎復前,再擊之,寇不動。帝登高望之曰:「東南不少卻乎?」亟麾大將譚廣等進擊,諸將奮斫馬足,寇少退。俄疾風揚沙,兩軍不相見,寇始引去。帝欲即夜班師,二人曰:「明日寇必降,請待之。」至期果降,帝始神其術,授仲和欽天監正。

英宗將北征,仲和時已老,學士曹鼐問曰:「駕可止乎?胡、王兩尚書已率百官諫矣。」曰:「不能也,紫微垣諸星已動矣。」曰:「然則奈何?」曰:「盍先治內。」曰:「命親王監國矣。」曰:「不如立儲君。」曰:「皇子幼,未易立也。」曰:「恐終不免立。」及車駕北狩,景帝遂即位。寇之薄都城也,城中人皆哭。仲和曰:「勿憂,雲向南,大將氣至,寇退矣。」明日,楊洪等入援,寇果退。一日出朝,有衛士請占。仲和辭,衛士怒。仲和笑曰:「汝室中妻妾正相鬥,可速返。」返則方鬥不解。或問:「何由知?」曰:「彼問時,適見兩鵲鬥屋上,是以知之。」其占事率類此。

仝寅,字景明,安邑人。年十二歲而瞽,乃從師學京房術,占禍福多奇中。父清游大同,攜之行塞上。石亨為參將,頗信之,每事咨焉。英宗北狩,遣使問還期。筮得《乾》之初,曰:「大吉。四為初之應,初潛四躍,明年歲在午,其幹庚。午,躍候也。庚良,更新也。龍歲一躍,秋潛秋躍,明年仲秋駕必復。但繇勿用,應在淵,還而復,必失位。然象龍也,數九也。四近五,躍近飛。龍在丑,丑曰赤奮若,復在午。午色赤,午奮於丑,若,順也,天順之也。其於丁,象大明也。位於南方,火也。寅其生,午其王,壬其合也。至歲丁丑,月寅,日午,合於壬,帝其復辟乎?」已而悉驗。

石亨入督京營,挾自隨。及也先逼都城,城中人恟懼,或請筮之,寅曰:「彼驕我盛,戰必勝。」寇果敗去。明年,也先請遣使迎上皇,廷臣疑其詐。寅言於亨曰:「彼順天仗義,我中國反失奉迎禮,寧不貽笑外蕃。」亨乃與於謙決計,上皇果還。景泰三年,指揮盧忠告變,事連南宮。帝殺中官阮浪,猶窮治不已,外議洶洶。忠一日屏人請筮,寅叱之曰:「是兆大凶,死不足贖。」忠懼而徉狂,事得不竟。已而忠果伏誅。英宗復辟,將官寅,寅固辭。命賜金錢金卮諸物。其父官指揮僉事,將赴徐州。英宗慮寅偕行,乃授錦衣百戶,留京師。寅見石亨勢盛,每因筮戒之,亨不能用,卒及於禍。寅以筮游公卿貴人間,莫不信重之,然無一語及私。年幾九十乃卒。

吳傑,武進人。弘治中,以善醫徵至京師,試禮部高等。故事,高等入御藥房,次入太醫院,下者遣還。傑言於尚書曰:「諸醫被徵,待次都下十餘載,一旦遣還,誠流落可憫。傑願辭御藥房,與諸人同入院。」尚書義而許之。正德中,武宗得疾,傑一藥而愈,即擢御醫。一日,帝射獵還,憊甚,感血疾。服傑藥愈,進一官。自是,每愈帝一疾,輒進一官,積至太醫院使,前後賜彪虎衣、繡春刀及銀幣甚厚。帝每行幸,必以傑扈行。帝欲南巡,傑諫曰:「聖躬未安,不宜遠涉。」帝怒,叱左右掖出。及駕還,漁於清江浦,溺而得疾。至臨清,急遣使召傑,比至,疾已深,遂扈歸通州。時江彬握兵居左右,慮帝晏駕己得禍,力請幸宣府。傑憂之,語近侍曰:「疾亟矣,僅可還大內。倘至宣府有不諱,吾輩寧有死所乎!」近侍懼,百方勸帝,始還京師。甫還而帝崩,彬伏誅,中外晏然,傑有力焉。未幾致仕。子希周,進士,戶科給事中;希曾,舉人。

又有許紳者,京師人。嘉靖初,供事御藥房,受知於世宗,累遷太醫院使,歷加工部尚書,領院事。二十年,宮婢楊金英等謀逆,以帛縊帝,氣已絕。紳急調峻藥下之,辰時下藥,未時忽作聲,去紫血數升,遂能言,又數劑而愈。帝德紳,加太子太保、禮部尚書,賜齎甚厚。未幾,紳得疾,曰:「吾不起矣。曩者宮變,吾自分不效必殺身,因此驚悸,非藥石所能療也。」已而果卒,賜謚恭僖,官其一子,恤典有加。明世,醫者官最顯,止紳一人。

其士大夫以醫名者,有王綸、王肯堂。綸,字汝言,慈谿人,舉進士。正德中,以右副都御史巡撫湖廣,精於醫,所在治疾,無不立效。有《本草集要》、《名醫雜著》行於世。肯堂所著《證治準繩》,為醫家所宗,行履詳父《樵傳》。

凌雲,字漢章,歸安人。為諸生,棄去。北遊泰山,古廟前遇病人,氣垂絕,雲嗟歎久之。一道人忽曰:「汝欲生之乎?」曰:「然。」道人鍼其左股,立蘇,曰:「此人毒氣內侵,非死也,毒散自生耳。」因授雲鍼術,治疾無不效。

里人病嗽,絕食五日,眾投以補劑,益甚。雲曰:「此寒濕積也,穴在頂,鍼之必暈絕,逾時始蘇。」命四人分牽其髮,使勿傾側,乃鍼,果暈絕。家人皆哭,雲言笑自如。頃之,氣漸蘇,復加補,始出鍼,嘔積痰斗許,病即除。有男子病後舌吐。雲兄亦知醫,謂雲曰:「此病後近女色太蚤也。舌者心之苗,腎水竭,不能制心火,病在陰虛。其穴在右股太陽,是當以陽攻陰。」雲曰:「然。」如其穴針之,舌吐如故。雲曰:「此知瀉而不知補也。」補數劑,舌漸復故。

淮陽王病風三載,請於朝,召四方名醫,治不效。雲投以鍼,不三日,行步如故。金華富家歸,少寡,得狂疾,至裸形野立。雲視曰:「是謂喪心。吾鍼其心,心正必知恥。蔽之帳中,慰以好言釋其愧,可不發。」乃令二人堅持,用涼水噴面,鍼之果愈。吳江婦臨產,胎不下者三日,呼號求死。雲鍼刺其心,鍼出,兒應手下。主人喜,問故。曰:「此抱心生也。手鍼痛則舒。」取兒掌視之,有鍼痕。

孝宗聞雲名,召至京,命太醫官出銅人,蔽以衣而試之,所刺無不中,乃授御醫。年七十七,卒於家。子孫傳其術,海內稱鍼法者,曰歸安凌氏。

有李玉者,官六安衛千戶,善針灸。或病頭痛不可忍,雖震雷不聞。玉診之曰:「此蟲啖腦也。」合殺蟲諸藥為末,吹鼻中,蟲悉從眼耳口鼻出,即愈。有跛人扶雙杖至,玉針之,立去其仗。兩京號「神鍼李玉」。兼善方劑。或病痿,玉察諸醫之方,與治法合而不效,疑之。忽悟曰:「藥有新陳,則效有遲速。此病在表而深,非小劑能愈。」乃熬藥二鍋傾缸內,稍冷,令病者坐其中,以藥澆之,踰時汗大出,立愈。

李時珍,字東璧,蘄州人。好讀醫書,醫家《本草》,自神農所傳止三百六十五種,梁陶弘景所增亦如之,唐蘇恭增一百一十四種,宋劉翰又增一百二十種,至掌禹錫、唐慎微輩,先後增補合一千五百五十八種,時稱大備。然品類既煩,名稱多雜,或一物而析為二三,或二物而混為一品,時珍病之。乃窮搜博採,芟煩補闕,歷三十年,閱書八百餘家,稿三易而成書,曰《本草綱目》。增藥三百七十四種,釐為一十六部,合成五十二卷。首標正名為綱,餘各附釋為目,次以集解詳其出產、形色,又次以氣味、主治附方。書成,將上之朝,時珍遽卒。未幾,神宗詔修國史,購四方書籍。其子建元以父遺表及是書來獻,天子嘉之,命刊行天下,自是士大夫家有其書。時珍官楚王府奉祠正,子建中,四川蓬溪知縣。

又吳縣張頤、祁門汪機、杞縣李可大、常熟繆希雍皆精通醫術,治病多奇中。而希雍常謂《本草》出於神農,朱氏譬之《五經》,其後又復增補別錄,譬之註疏,惜硃墨錯互。乃沈研剖析,以本經為經,別錄為緯,著《本草單方》一書,行於世。

周述學,字繼志,山陰人。讀書好深湛之思,尤邃於曆學,撰《中經》。用中國之算,測西域之占。又推究五緯細行,為《星道五圖》,於是七曜皆有道可求。與武進唐順之論曆,取歷代史志之議,正其訛舛,刪其繁蕪。又撰《大統萬年二曆通議》,以補歷代之所未及。自歷以外,圖書、皇極、律呂、山經、水志、分野、輿地、算法、太乙、壬遁、演禽、風角、鳥占、兵符、陣法、卦影、祿命、建除、葬術、五運六氣、海道鍼經,莫不各有成書,凡一千餘卷,統名曰《神道大編》。嘉靖中,錦衣陸炳訪士於經歷沈煉,煉舉述學。炳禮聘至京,服其英偉,薦之兵部尚書趙錦。錦就訪邊事,述學曰:「今歲主有邊兵,應在乾艮。艮為遼東,乾則宣、大二鎮,京師可無虞也。」已而果然。錦將薦諸朝,會仇鸞聞其名欲致之,述學識其必敗,乃還里。總督胡宗憲征倭,招至幕中,亦不能薦,以布衣終。

張正常,字仲紀,漢張道陵四十二世孫也。世居貴溪龍虎山。元時賜號天師。太祖克南昌,正常遣使上謁,已而兩入朝。洪武元年入賀即位。太祖曰:「天有師乎?」乃改授正一嗣教真人,賜銀印,秩視二品。設寮佐,曰贊教,曰掌書。定為制。

長子宇初嗣。建文時,坐不法,奪印誥。成祖即位,復之。宇初嘗受道法於長春真人劉淵然,後與淵然不協,相詆訐。永樂八年卒,弟宇清嗣。宣德初,淵然進號大真人,宇清入朝懇禮部尚書胡濙為之請,亦加號崇謙守靜。

再傳至曾孫元吉,年幼,敕其祖母護持,而贈其父留綱為真人,封母高氏為元君。景泰五年入朝,乞給道童四百二十人度牒。濙復為請,許之。尋欲得大真人號,濙為請,又許之。天順七年再乞給道童三百五十人度牒,禮部尚書姚夔持不可,詔許度百五十人。

憲宗立,元吉復乞加母封,改太元君為太夫人,以吏部言不許,乃止。初,元吉已賜號沖虛守素昭祖崇法安恬樂靜玄同大真人,母慈惠靜淑太元君,至是加元吉號體玄悟法淵默靜虛闡道弘法妙應大真人,母慈和端惠貞淑太真君。然元吉素兇頑,至僭用乘輿器服,擅易制書。奪良家子女,逼取人財物。家置獄,前後殺四十餘人,有一家三人者。事聞,憲宗怒,械元吉至京,會百官廷訊,論死。於是刑部尚書陸瑜等請停襲,去真人號,不許。命仍舊制,擇其族人授之,有妄稱天師,印行符籙者,罪不貸。時成化五年四月也。元吉坐繫二年,竟以夤緣免死,杖百,發肅州軍,尋釋為庶人。

族人元慶嗣,弘治中卒。子彥嗣,嘉靖二年進號大真人。彥知天子好神仙,遣其徒十餘人乘傳詣雲南、四川採取遺經、古器進上方,且以蟒衣玉帶遺鎮守中貴,為雲南巡撫歐陽重所劾,不問。十六年禱雪內庭有驗,賜金冠玉帶、蟒衣銀幣,易金印,敕稱卿不名。彥入朝所經,郵傳供應或後期,常山知縣吳襄等至下按臣治。

傳子永緒,嘉靖末卒,無子。吏部主事郭諫臣乘穆宗初政,上章請奪其世封。下江西守臣議,巡撫任士憑等力言宜革,乃去真人號,改授上清觀提點,秩五品,給銅印,以其宗人國祥為之。萬曆五年,馮保用事,復國祥故封,仍予金印。國祥傳至應京。崇禎十四年,帝以天下多故,召應京有所祈禱。既至,命賜宴。禮臣言:「天順中制,真人不與宴,但賜筵席。今應京奉有優旨,請仿宴法王佛子例,宴於靈濟宮,以內官主席。」從之。明年三月,應京請加三官神封號,中外一體尊奉。禮官力駁其謬,事得寢。張氏自正常以來,無他神異,專恃符籙,祈雨驅鬼,間有小驗。顧代相傳襲,閱世既久,卒莫廢去云。

劉淵然者,贛縣人。幼為祥符宮道士,頗能呼召風雷。洪武二十六年,太社聞其名,召至,賜號高道,館朝天宮。永樂中,從至北京。仁宗立,賜號長春真人,給二品印誥,與正一真人等。宣德初,進大真人。七年乞歸朝天宮,御製山水圖歌賜之。卒年八十二,閱七日入殮,端坐如生。淵然有道術,為人清靜自守,故為累朝所禮。其徒有邵以正者,雲南人,早得法於淵然。淵然請老,薦之,召為道籙司左玄義。正統中,遷左正一,領京師道教事。景泰時,賜號悟玄養素凝神沖默闡微振法通妙真人。天順三年,將行慶成宴。故事,真人列二品班末,至是,帝曰:「殿上宴文武官,真人安得與。」其送筵席與之,遂為制。

又有沈道寧者,亦有道術。仁宗初,命為混元純一沖虛湛寂清靜無為承宣布澤助國佐民廣大至道高士,階正三品,賜以法服。

時有浮屠智光者,亦賜號圓融妙慧凈覺弘濟輔國光範衍教灌頂廣善大國師,賜以金印。智光,武定人。洪武時,奉命兩使烏斯藏諸國。永樂時,又使烏斯藏,迎尚師哈立麻,遂通番國諸經,多所譯解。歷事六朝,寵錫冠群僧,與淵然輩淡泊自甘,不失戒行。迨成化、正德、嘉靖朝,邪妄雜進,恩寵濫加,所由與先朝異矣。


\end{pinyinscope}