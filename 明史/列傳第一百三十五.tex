\article{列傳第一百三十五}

\begin{pinyinscope}
劉綎喬一琦李應祥童元鎮陳璘吳廣鄧子龍馬孔英

劉綎,字省吾,都督顯子。勇敢有父風,用廕為指揮使。

萬歷初,從顯討九絲蠻。先登,擒其酋阿大。以功遷雲南迤東守備,改南京小教場坐營。

十年冬,緬甸犯永昌、騰越,巡撫劉世曾請濟師。明年春,擢綎遊擊將軍,署騰衝守備事。緬甸去雲南遠,自其酋莽瑞體以兵服諸番,勢遂強,數擾邊境。江西人岳鳳者,商隴川,驍桀多智,為宣撫多士寧記室,士寧妻以妹。鳳誘士寧往見瑞體,潛與子曩烏鴆殺之,并殺其妻子,奪金牌印符,受瑞體偽命,代士寧為宣撫。瑞體死,子應裏嗣。鳳結耿馬賊罕虔、南甸土舍刀落參、芒市土舍放正堂,與應裏從父猛別、弟阿瓦等,各率象兵數十萬攻雷弄、盞達、乾崖、南甸、木邦、老姚、思甸諸處,殺掠無算。窺騰越、永昌、大理、蒙化、景東、鎮沅、元江。已,陷順寧,破盞達,又令曩烏引緬兵突猛淋。指揮吳繼勛等戰死。鄧川土官知州何鈺,鳳僚婿也,使使招之,鳳縶獻應裏。

當是時,車里、八百、孟養、木邦、孟艮、孟密、蠻莫皆以兵助賊,賊勢益盛。黔國公沐昌祚聞警,移駐洱海,巡撫劉世曾亦移楚雄。大征漢土軍數萬,令參政趙睿壁蒙化,副使胡心得壁騰衝,陸通霄壁趙州,僉事楊際熙壁永昌,與監軍副使傅寵、江忻督參將胡大賓等分道進擊。大小十餘戰,積級千六百有奇,猛別、落參皆殪。參將鄧子龍擊斬罕虔於姚關。應裏趣鳳東寇姚關,北據灣甸、芒市。會綎至軍,軍大振。鳳懼,乃令妻子及部曲來降,綎責令獻金牌印符及蠻莫、孟密地。乃以送鳳妻子還隴川為名,分兵趨沙木籠山,據其險,而己馳入隴川境。鳳度四面皆兵,遂詣軍門降。綎復率兵進緬,緬將先遁,留少兵隴川。綎攻之,鳳子曩烏亦降。綎乃攜鳳父子往攻蠻莫,乘勝掩擊。賊窘,縛緬人及象馬來獻,蠻莫平。遂招撫孟養賊,賊將乘象走,追獲之。復移師圍孟璉,生擒其魁。

雲南平,獻俘於朝。帝為告謝郊廟,受百官賀。大學士申時行以下,悉進官蔭子。綎亦進副總兵,予世廕。乃改孟密安撫司為宣撫,增設安撫二,曰蠻莫,曰耿馬,長官司二,曰孟璉,曰孟養;千戶所二,一居姚關,一居猛淋。皆名之曰「鎮安」。命綎以副總兵署臨元參將,移鎮蠻莫。初,鳳降本以計誘,而巡撫世曾稱陣擒,遂行獻俘禮,敘功及閣部。

未幾,緬人復大舉寇孟密。孟密兵戰敗,賊遂圍五章。把總高國春率五百人援,破賊數萬,連摧六營,為西南戰功第一,進官,世廕副千戶。綎亦優敘。蠻莫設安撫,以土官思順有功,特授之。綎納其重賄,又縱部將謝世祿等淫虐,思順大怨。

綎,將家子。父顯部曲多健兒,綎擁以自雄。征緬之役,勒兵金沙江,築將臺於王驥故址,威名甚盛。然性貪,御下無法。兵還至騰衝,甲而噪,焚民居。綎在蠻莫,聞之馳至,犒以金錢,始定。思順恐禍及,叛歸莽酋。詔革糸廷任,以遊擊候調。

無何,羅雄變起。羅雄者,曲靖屬州也,者氏世為知州。嘉靖時,者濬嗣職,殺營長而奪其妻,生子繼榮。濬年老無他子,繼榮得襲職,遂弒浚。妖僧王道、張道以繼榮有異相,奉為主。用符術練丁甲,煽聚徒黨,獨外弟隆有義不從。十三年冬,繼榮分黨四剽,廣西師宗、陸涼諸府州咸被患。巡撫劉世曾檄調漢土軍,屬監司程正誼、鄭璧等分禦之。會綎解官至霑益,世曾喜,令與裨將劉紹桂、萬鏊分道討。綎直搗繼榮寨,拔之,獲其妻妾數人,繼榮逸去。綎連克三砦,斬王道、張道,追亡至阿拜江。隆有義部卒斬繼榮首以獻,賊盡平。時首功止五十餘級,而撫降者萬餘人,論者稱其不妄殺。初,綎破繼榮,有論其私財物者,功不錄。世曾為辨誣,乃賜白金。尋用為廣西參將,移四川。

二十年召授五軍三營參將。會朝鮮用師,綎請率川兵五千赴援,詔以副總兵從征。至則倭已棄王京遁,綎趨尚州烏嶺。嶺亙七十里,峭壁通一線,倭拒險。別將查大受、祖承訓等間道踰槐山,出烏嶺後。倭大驚,遂移駐釜山浦。糸廷及承訓等進屯大邱、忠州,以金羅水兵布釜山海口,朝鮮略定。未幾,倭遣小西飛納款,遂犯咸安、晉州,逼全羅。提督李如松急遣李平胡、查大受屯南原,祖承訓、李寧屯咸陽,綎屯陜川,扼之。倭果分犯,請將並有斬獲。倭乃從釜山移西生浦,送王子歸朝鮮。帝命撤如松大軍還,止留綎及遊擊吳惟忠合七千六百人,分扼要口。總督顧養謙力主盡撤,綎、惟忠亦先後還。

屬播酋楊應龍作亂,擢綎四川總兵官。綎戍朝鮮二年,勞甚,覬勘功優敘,乃賄御史宋興祖。興祖以聞,法當褫。部議綎功多,請盡革雲南所加功級,以副總兵鎮四川。尋以應龍輸款,而青海寇數擾邊,特設臨兆總兵官,移綎任之。

二十四年三月,火落赤、真相、昆都魯、歹成、他卜囊等掠番窺內地。綎部將周國柱等擊之莽剌川腦,斬首百三十有奇,獲馬牛雜畜二萬計。帝為告郊廟宣捷。綎等進秩予蔭有差。

明年五月,朝鮮再用師。詔綎充禦倭總兵官,提督漢土兵赴討。又明年二月抵朝鮮,則楊鎬、李如梅已敗。經略邢玠乃分軍為三,中董一元,東麻貴,西則綎,而陳璘專將水兵。綎營水源。倭亦分三路,西行長據順天,壕砦深固。綎欲誘執之,遣使請與期會。使者三反,綎皆單騎俟道中。行長覘知之,乃信,期以八月朔定約。至期,綎部卒洩其謀,行長大驚,逸去。綎進攻失利。監軍參政王士琦怒,縛其中軍。綎懼,力戰破之,賊退不敢出。諸將三道進,綎挑戰破之,驅賊入大城。已,賊聞平秀吉死,將遁。綎夜半攻奪栗林、曳橋,斬獲多。石曼子引舟師救,陳璘邀擊之海中。行長遂棄順天,乘小艘遁。

班師,進綎都督同知,世廕千戶。遂移師征楊應龍。會四川總兵官萬鏊罷,即以綎代之。時兵分八道,川居其四。川東又分為二,以綦江道最要,令綎當之。應龍熟綎才,頗懼,益兵守要害。二十八年正月,諸將克丁山、銅鼓、嚴村,遂直搗楠木、山羊、簡臺三峒。峒絕險,賊將穆照等眾數萬連營,諸將憚之。綎分兵攻其三面,大戰於李漢壩,生擒其魁,餘賊奔入峒。乘勢克三關,直搗峒前,焚之,賊多死。盡克三峒,擒穆照及賊魁吳尚華。是日,綎督戰,左持金,右挺劍,大呼曰:「用命者賞,不用命者齒劍!」鬥死者四十人,遂大捷。應龍乃遣子朝棟、惟棟及其黨楊珠統銳卒數萬,由松坎、魚渡、羅古池三道並進。綎伏萬人羅古,待松坎賊;以萬人伏營外,待魚渡賊;而別以一軍策應。賊果至,伏盡起。綎率部下轉戰,斬首數百,追奔五十里。賊聚守石虎關,糸廷亦掘塹守。

初,綎聞征播命,逗遛,多設難以要朝廷。言官交劾,議調南京右府僉書。綎至是聞之,即辭任。總督李化龍以平播非綎不可,固留之,力薦於朝。糸廷乃復受事,踰夜郎舊城,攻克賊滴淚、三坡、瓦窯坪、石虎諸隘,直抵婁山關。婁山萬峰插天,叢箐中一徑纔數尺,賊設木關十三座,排柵置深坑,百險俱備。綎分奇兵為左右路,間道趨關後,而自督大軍仰攻,奪其關,追至永安莊,兩路軍亦會。綎老將持重,慮賊衝突,聯諸營:一據婁山關為老營,一據白石口為腰營,一據永安莊為前營。都指揮王芬者,勇而寡謀,每戰輒請為前鋒,連勝有輕敵心,獨營松門埡之沖,距大營數里。賊方有烏江之勝,謀再奪婁山。適穆照遣使洩芬孤軍狀,賊乃襲殺芬,守備陳大剛、天全招討楊愈亦死,失亡士卒二千人。綎聞,親率騎卒往救,部將周以德、周敦吉分兩翼夾攻,賊始大奔,追至養馬城而還。是日,應龍幾被獲,乃不敢窺婁山。綎懲前失,答刂近關堅壁,且請濟師。踰十餘日,克後水囤,營於冠子山。尋會馬孔英、吳廣諸軍,逼海龍囤下,與諸將共平賊,綎功為多。

初,李化龍薦綎,言官謂綎嘗納應龍賄,宜奪官從軍。部議謫為事官,戴罪辦賊。綎德化龍,使使齎玉帶一、黃金百、白金千投化龍家,為化龍父所叱。投巡按御史崔景榮家,亦如之。化龍、景榮並奏其事,詔革綎任,永不收錄,沒其物於官。已,錄平播功,進左都督,世廕指揮使。

三十六年,雲南阿克反,起綎討賊總兵官。未至,賊已平,寢前命。四十年,四川建昌惈亂,命綎為總兵官討之。偕參政王之機分八道督諸將攻,而己居中節制,克桐槽、沈渣、阿都、廈卜、越北諸砦,大小五十六戰,斬馘三千三百有奇,諸惈巢穴一空。

綎為將,數被黜抑,性驕恣如故。嘗拳毆馬湖知府詹淑,淑改調,綎奪祿半年。久之,以軍政拾遺罷歸。

四十六年,帝念遼警,召為左府僉書。明年二月,經略楊鎬令綎及杜松、李如柏、馬林四路出師。綎兵四萬,由寬佃,副使康應乾監之,遊擊喬一琦別監朝鮮軍並進。綎鎮蜀久,好用蜀兵。久待未至,遂行。而所分道獨險遠,重岡疊嶺,馬不成列。次深河,連克牛毛、馬家二砦。大清兵五百守董鄂路,聞糸廷軍至,逆戰。綎縱兵圍數重,大清兵眾寡不敵,失二裨將,傷五十人,餘潰圍出。綎已深入三百里,杜松軍覆猶不知。復整眾進,遇大清兵,綎引軍登阿布達里岡,將布陣,大清兵亦登岡,出其上,而別以一軍趨綎西。岡上軍自高馳下,奮擊綎軍,綎殊死戰。趨綎西者復從旁夾擊,綎軍不能支。大清兵乘勢追擊,遇綎後二營軍。未及陳,復為大清兵所乘,大潰,綎戰死。養子劉招孫者,最驍勇,突圍,手格殺數人,亦死。士卒脫者無幾。

時應乾及朝鮮軍營富察之野,大清遂移師邀之。應乾兵及朝鮮兵列械將戰,狂風驟起,揚沙石。應乾發火器,反擊己營,大亂。大清兵趨擊,大破之,掩殺幾盡,應乾以數百騎免。一琦亦為大清兵所破,走入朝鮮營。朝鮮都元帥姜弘立、副元帥金景瑞懼,率眾降,一琦投崖死。楊鎬聞杜松、馬林師敗,馳召綎及李如柏還。騎未至,綎已覆,獨如柏全。事聞,帝遣中使祭陣亡將士,恤綎家。

綎於諸將中最驍勇。平緬寇,平羅雄,平朝鮮倭,平播酋,平惈,大小數百戰,威名震海內。綎死,舉朝大悚,邊事日難為矣。綎所用鑌鐵刀百二十斤,馬上輪轉如飛,天下稱「劉大刀」。天啟初,贈少保,世廕指揮僉事,立祠曰「表忠」。一琦,字伯珪,上海人。

李應祥,湖廣九溪衛人。以武生從軍,積功至廣西思恩參將。

萬歷七年,巡撫張任大征十寨,應祥與有功。即其地設三鎮,築城列戍。應祥方職營建,會擢松潘副總兵,當事者奏留之,以新秩蒞舊任。從總兵王尚文大破馬平賊韋王明,尋以署都督僉事,入為五軍營副將。

十三年,改南京左府僉事,出為四川總兵官。松、茂諸番列砦四十八,歲為吏民患。王廷瞻撫蜀時,嘗遣副將吳子忠擊破丟骨、人荒、沒舌三砦,諸酋乃降。故事,諸番歲有賞賚,番恃強要索無已。其來堡也,有下馬、上馬、解渴、過堡酒及熱衣氣力偏手錢;戍軍更番,亦奉以錢,曰新班、架梁、放狗、屣草、掛彩。廷瞻一切除之,西陲稍靖。僅六七年,勢復猖獗。是年夏,楊柳番出攻普安堡,犯歸水崖、石門坎,遂入金瓶堡,殺守將。巡撫雒遵屬應祥討之。提卒三千入茂州,克一巖。番恃險,剽如故。

無何,遵罷,徐元泰代。檄諭之,使三反,番不應。窺蒲江關,斷歸水崖、黃土坎道,築牆五哨溝,絕東南聲援。見官軍少,相顧笑曰:「如此磨子兵,奈我何?」磨子者,謂屢旋轉而數不增也。其冬突平夷堡,掠良民,刳其腸,繞二牛角,牛奔,腸寸裂。明年正月,遂圍蒲江關,炮毀雉堞。守將朱文達出,斬數十人。賊稍解,東南路始通。

元泰決計大征。諸路兵悉集,乃命遊擊周于德將播州兵為前鋒,遊擊邊之垣將酉陽兵為後拒,故總兵郭成將敘、馬兵扼其吭,參將朱文達將平茶兵擊其脅,而應祥居中節制,參議王鳳監之。應祥令軍中各樹赤、白幟一,良民陷賊者徒手立赤幟下,熟番不附賊者徒手立白幟下,即免罪。番雖多,遇急不相救。國師喇嘛者,狡猾,聯姻青海酋丙兔與灣仲、占柯等,刻木連大小諸姓,歃血詛盟。至是,邀灣仲、占柯先犯歸化以嘗官軍。于德誘擒喇嘛、灣仲,守備曹希彬復擊斬占柯。丟骨、人荒、沒舌三砦最強,于德皆攻克,復連破卜洞王諸砦。文達、成、之垣亦各拔數砦,與于德軍合,遂攻破蜈蚣、茹兒諸巢。嘉靖初,之垣祖輪以指揮討茹兒賊,被殺,漆其頭為飲器。及是六十年,之垣乃得之,以還葬焉。

賊屢北,窘,悉棄輜重餌官軍。官軍不顧,斬關入,賊多死,河東平。尋渡河而西,連破西坡、西革、歪地、乾溝、樹底諸巢。有小粟谷者,首亂,覘大軍西,不設備。郭成夜襲之,大獲。牛尾砦尤險惡,將士三路夾攻,火其柵,斬酋合兒結父子。河西亦平。諸軍得所積稞粟,留十日,盡焚其砦,以六月班師。其逃窮谷者,求偏頭結賽乞降,應祥令埋奴設誓,然後許之。埋奴者,番人反接其奴,獻軍前,呼天而誓,即牽至要路,掘坎埋之,露其首,凡埋二十三人。偏頭結賽雅善天竺僧,僧言歲在雞犬,番有阨。偏頭信之,預匿山谷中,逸賊以為神,跡而拜求之,故偏頭為之請。是役也,焚碉房千六百有奇,生擒賊魁三十餘人,俘馘以千餘計。自是群番震驚,不敢為患,邊人樹碑記績焉。

建昌、越巂諸衛,番惈雜居。建昌逆酋曰安守,曰五咱,曰王大咱,與越巂邛部黑骨夷並起為亂。巡撫徐元泰議討,徵兵萬八千。仍以文達、之垣分將,應祥統之,副使周光鎬監其軍。十一月,光鎬先渡瀘,黑骨與大咱已據相嶺,焚三峽橋;五咱等亦寇禮州、德昌二所。時徵兵未集,光鎬先設疑,以嘗相嶺賊,賊果退據桐槽。桐槽者,大咱巢穴也。已而諸道兵盡抵越巂,應祥令文達攻五咱,之垣攻大咱,姑置黑骨夷弗問。夜半走三百里抵禮州,賊半渡,文達擊敗之,遂渡河搗其巢。之垣亦屢破桐槽,大咱亡入山峪中。

無何,五咱據磨旗山挑戰。官軍夾擊,賊退保毛牛山。山延袤六七百里,連大小西番界,文達兵大破之。五咱西遁,與安守合,結砦西谿。會所征鹽井剌馬兵三千至,猙獰跳躍,類非人形,諸番所深畏。應祥偵賊將劫營,乃潛移己營,而令剌馬兵屯其處。夜分賊來襲,剌馬起擊之,伏屍狼籍。諸將遂進攻西谿,逐北至磨砦七板番,連兵圖五咱,而令裨將田中科營麥達,逼安守。會諜者報守謀襲中科,應祥夜飲材官高逢勝三巨觥,令率敢死士三百疾趨七十里,抵麥達而伏。守夜至,遇伏被擒。守為群寇魁,守殪,西南邛笮、苴蘭、靡莫諸酋皆震怖。商山四堡番乞降於之垣,大小七板番乞降於文達,各埋奴道左,呼號頓首,誓世世不敢叛。五咱勢窮,走昌州,亦為裨將王言所獲。

土木安四兒者,居連昌城中,潛剽掠於外。至是知禍及,率黨數百人走據虛郎溝。諸軍既滅五咱,應祥遣之北,示將討黑骨者,四兒遂弛備。將士忽還軍襲之,獲四兒。

復討大咱。初,大咱敗,匿所親普雄酋姑咱所。大軍至,姑咱懼,密告裨將王之翰,之翰搜得大咱;而黑夷酋阿弓等七人在大孤山,亦先為之翰所擒。於是建昌、越巂諸番盡平。上首功二千有奇,撫降者三千餘人。時萬歷十五年七月也。

邛部屬夷膩乃者,地近馬湖。其酋撒假與外兄安興、木瓜夷白祿、雷坡賊楊九乍等,數侵掠內地。巡撫曾省吾議討之。會有都蠻之役,不果。乃建六堡,益戍兵千二百人,而諸蠻鴟張如故。及建、越興師,又藏納叛人。元泰乃令都指揮李獻忠等分剿。賊詐降,誘執獻忠等三將,殺士卒數千人,勢益猖獗。應祥等師旋,元泰益征播州、酉陽諸土兵,合五萬人,令應祥督文達、之垣及周于德諸將三道入,故總兵郭成亦從征。十一月,於德首敗白祿兵,追至馬蝗山,懸索以登,賊潰。乘勢攻木瓜夷,射殺白祿。追至利濟山,雪深數尺,於德先登,復大敗賊,毀其巢。初,撒假與九乍率萬人據山,播州兵擊走之。至是,文達復破之大田壩,合于德兵追逐,所向皆捷。遊擊萬鏊躡擊撒假於鼠囤,獲其妻子。郭成復至三寶山大戰,生擒撒假。安興據巢守,文達、鏊分道入,獲其母妻。安興擲金於途,以緩追者,遂得脫。已,諸軍深入,竟獲之。他夷惈畏威降者二千餘人,悉獻還土田,願修職貢,兵乃罷。凡斬首一千六百九十餘,俘獲七百三十有奇,以其地置屏山縣。論功,應祥屢加都督同知,元泰亦至兵部尚書。

當是時,蜀中劇寇盡平,應祥威名甚著。御史傅霈按部,詰應祥冒饟。應祥賄以千金,為所奏,罷職。兵部舉應祥僉書南京右府,給事中薛三才持不可。

二十八年大征播州。貴州總兵官童元鎮逗遛,總督李化龍劾之,薦應祥代。時分兵八道,貴州分烏江、興隆二道。詔元鎮充為事官,由烏江入,應祥由興隆入,諸道剋二月望進兵。應祥未受事,副將陳寅等已連克數囤,拒賊四牌高囤下,別遣兵從間道直搗龍水囤。他將蔡兆吉又自乾坪抵箐岡,過四牌。賊首謝朝俸營其地,四面峭壁深箐,前二關。賊從高鼓噪,官軍殊死戰,俘朝俸妻子,乘勢抵河畔。會烏江敗書聞,斂兵不進者旬日。及應祥受任,益趣諸將急渡。寅等乃取他道渡河,而潛為浮橋以濟師。諸軍渡,賊失險,乞降者相繼,應祥悉受之。賊所恃止黃灘一關,壁立,眾死守。會賊徒石勝俸等率萬餘人降,告曰:「去黃灘三十里有三關,入播門戶也,先襲破之,則黃灘孤難守。」應祥然其計,令偕陳寅率精卒四千夜抵關下。勝俸以數十騎誘開門,殲其戍卒。黃灘賊懼。寅督諸將渡河攻關前,勝俸由墳林暗渡襲關後,賊乃大敗。應祥直抵海龍囤,合諸道兵共滅楊應龍。

播既平,還鎮銅仁。明年,改鎮四川。播遺賊吳洪、盧文秀等惡有司法嚴,而遵義知縣蕭鳴世失眾心。洪等遂稱應龍有子,聚眾為亂。應祥偕副使傅光宅捕之,盡獲。應祥尋卒於官。以平播功,贈左都督,世廕千戶。

應祥為將,謀勇兼資,所至奏績。平蜀三大寇,功最多。

童元鎮,桂林右衛人。萬歷中為指揮,從討平樂賊莫天龍有功,屢遷遊擊將軍。高江瑤反,從呼良朋破平之。歷永寧、潯、梧參將,進副總兵。擢署都督僉事,為廣西總兵官。未幾,改廣東。

二十三年,總督陳大科以元鎮熟蠻事,仍移廣西。岑溪西北為上、下七山,介蒼藤間,有平田、黎峒、白板、九密等三十七巢。東南為六十三山,有孔亮、陀田、桑園、古欖、魚修等百餘巢,與廣東羅旁接。山險箐深,環數百里無日色,賊首潘積善等據之,久為民患。及羅旁平,積善懼,乞降。為設參將於大峒,兵千餘戍之。其後,將領多掊克,士卒又疲弱,賊復生心,時出剽。會歲饑,粵東亡命浪賊數百人潛入七山,誘諸瑤為亂。元鎮先以參將戍岑溪,得諸瑤心。至是,積善及其黨韋月咸願招撫自效,六十三山諸瑤多受約束。有訛言將剿北科瑤者,諸瑤謂紿己,大恨,遂與孔亮山賊攻月,殺之,火大峒參將署。督撫陳大科、戴耀屬元鎮討之。時副將陳璘、參將吳廣罷官里居,大科起令將兵,與元鎮並進。賊伐大木塞道,環布箛簽,元鎮佯督軍開道,而潛從小徑上。孔亮山賊憑高,弩矢雨下。諸軍用火器攻,大破之。俘馘千五百有奇,餘招撫復業。時府江韋扶仲等亦據險亂,元鎮與參政陸長庚謀,募瑤為間,乘夜獲其妻子,誘出劫,伏兵擒之。餘黨悉平。元鎮以功增秩賜金。

會日本破朝鮮。廷議由浙、閩泛海搗其巢,牽制之,乃改元鎮浙江。既而事寢,移鎮貴州。

二十八年,李化龍大征楊應龍,令元鎮督永順、鎮雄、泗城諸土軍,由烏江進。元鎮憚應龍,久駐銅仁不進,屢趣乃行。時劉綎、吳廣諸軍已進,群賊議分兵守,其黨孫時泰曰:「兵分則力薄。乘官軍未集,先破其弱者,餘自退矣。」應龍善之。聞元鎮發烏江,應龍喜曰:「此易與耳。」謀縱之渡江,密以計取。監軍按察使楊寅秋言烏江去播不遠,宜俟諸道深入,與俱進,元鎮不從。於是永順兵先奪烏江,賊遣千餘人沿江叫罵以誘之。諸軍既濟,復奪老君關。前哨參將謝崇爵乘勢督泗城及水西兵再拔河渡關。三月望,賊以步騎數千先衝水西軍,軍中驅象出戰,賊多傷。俄駕象者斃,象反走,擲火器者又誤擊己營,陣亂。泗城兵先走,崇爵亦走,爭浮橋,橋斷,殺溺死者數千人。

河渡既敗,烏江相去六十里,猶未知。明日,參將楊顯發永順兵三百出哨,道遇賊數萬,咸為水西裝。永順兵不之疑,賊掩殺三百人,亦襲其裝,直趨烏江。烏江軍信為水西、永順軍,不設備,遂為賊所破,爭先渡江。賊先斷浮橋,士卒多溺死,顯及二子與焉。元鎮所部三萬人,不存什一,將校止崇爵等三人,江水為不流。

貴陽聞警,居民盡避入城,遠近震動。化龍用上方劍斬崇爵,益徵兵,檄鎮雄土官隴澄邀賊歸路。隴澄者,即安堯臣,水西安疆臣弟也。軍不與元鎮合,獨全,當事頗疑其通賊。寅秋以鎮雄去播止二日,令搗巢立效,澄許之。河渡未敗時,澄已遣部將劉岳、王嘉猷攻拔苦竹關及半壩嶺。暨敗,二將移新站。賊伏兵大水田,別以五千人來襲,敗還。嘉猷乃揚聲搗大水田,而潛以一軍拔大夫關,直抵馬坎,斷賊歸路,與疆臣合,賊遂遁。會都指揮徐成將兵至,合泗城土官岑紹勳兵,再克河渡關。賊將張守欽、袁五受據長箐、萬丈林,永順兵擊破之,生擒守欽。攻清潭洞,復擒五受。會朝議責元鎮敗狀,令李應祥並將其軍,遂合水西、鎮雄諸部,直抵海龍囤,竟滅賊。

兵初興,元鎮坐逗遛,謫為事官。及是,逮至京,下吏,罪當死。法司援前岑溪功,謫戍煙瘴。遇赦,廣西巡撫戴耀為請,部議不許,竟卒於戍所。

陳璘,字朝爵,廣東翁源人。嘉靖末,為指揮僉事。從討英德賊有功,進廣東守備。與平大盜賴元爵及嶺東殘寇。萬曆初,討平高要賊鄧勝龍,又平揭陽賊及山賊鐘月泉,屢進署都指揮僉事,僉書廣東都司。

官軍攻諸良寶,副將李成立戰敗。總督殷正茂請假璘參將,自將一軍。賊平,授肇慶遊擊將軍,徙高州參將。總督凌雲翼將大徵羅旁,先下令雕剿。璘所破凡九十巢。已,分十道大征。璘從信宜入,會諸軍,覆滅之,以其地置羅定州及東安、西寧二縣。即遷璘副總兵,署東安參將事。未幾,餘孽殺吏民,責璘戴罪辦賊。璘會他將朱文達攻破石牛、青水諸巢,斬捕三百六十餘人,授俸如故。

時東安初定,璘大興土木,營寺廟,役部卒,且勒其出貲。卒咸怒,因事倡亂,掠州縣,為巡按御史羅應鶴所劾,詔奪璘官。既而獲賊,乃除罪,改狼山副總兵。

璘有謀略,善將兵,然所至貪黷,復被劾褫官。廢久之,朝士多惜其才,不敢薦。二十年,朝鮮用兵,以璘熟倭情,命添註神機七營參將,至則改神樞右副將。無何,擢署都督僉事,充副總兵官,協守薊鎮。明年正月詔以本官統薊、遼、保定、山東軍,禦倭防海。會有封貢之議,暫休兵,改璘協守漳、潮。坐賄石星,為所奏,復罷歸。

二十五年,封事敗,起璘故官,統廣東兵五千援朝鮮。明年二月,擢禦倭總兵官,與麻貴、劉綎並將。部卒次山海關鼓噪,璘被責。尋令提督水軍,與貴、綎及董一元分道進,副將陳蠶、鄧子龍,遊擊馬文煥、季金、張良相等皆屬焉,兵萬三千餘人,戰艦數百,分布忠清、全羅、慶尚諸海口。初,賊泛海出沒,官軍乏舟,故得志。及見璘舟師,懼不敢往來海中。會平秀吉死,賊將遁,璘急遣子龍偕朝鮮將李舜臣邀之。子龍戰沒,蠶、金等軍至,邀擊之,倭無鬥志,官軍焚其舟。賊大敗,脫登岸者又為陸兵所殲,焚溺死者萬計。時綎方攻行長,驅入順天大城。璘以舟師夾擊,復焚其舟百餘。石曼子西援行長,璘邀之半洋,擊殺之,殲其徒三百餘。賊退保錦山,官軍挑之不出。已,渡匿乙山。崖深道險,將士不敢進。璘夜潛入,圍其巖洞。比明,砲發,倭大驚,奔後山,憑高以拒。將士殊死攻,賊遁走。璘分道追擊,賊無脫者。論功,璘為首,綎次之,貴又次之。進璘都督同知,世蔭指揮僉事。

師甫旋,會有徵播之役。命璘為湖廣總兵官,由偏橋進,副將陳良玭由龍泉,受璘節制。二十八年二月,軍次白泥,楊應龍子朝棟率眾二萬渡烏江迎戰。璘前禦之,而分兩翼躡其後。賊少挫,追奔至龍溪山,賊合四牌賊共拒。四牌在江外,與江內七牌皆五司遺種、九股惡苗,素助賊。璘廣招撫,乃進軍龍溪。偵知賊有伏,令遊擊陳策用火器擊之,賊據險,矢石雨下。璘先登,斬小校退者以徇,把總吳應龍等陷陣,賊大潰,退四牌保兒囤。璘二裨將逼之,中伏。璘募死士從應龍等奮擊,賊復潰,奔據囤巔,夜由山後遁。黎明追及於袁家渡,復敗之。四牌之賊遂盡。

三月望,諸軍為浮橋渡江。知賊將張佑、謝朝俸、石勝俸等營七牌野豬山,璘即夜發抵苦練坪。前鋒與戰,後軍至,夾擊之,賊敗逃深箐,官軍遂入苦菜關。會童元鎮烏江師敗,璘懼,請退師,總督李化龍不可。璘乃進營楠木橋,次湄潭。賊悉聚青蛇、長坎、瑪瑙、保子四囤,地皆絕險,而青蛇尤甚。璘議,同日攻則兵力弱,止攻一囤,則三囤必相助,乃先攻三囤,次及青蛇。良玭師亦來會,令伏囤後,別以一軍守板角關,防賊逸。璘督諸將力攻三日,賊死傷無算,三囤遂下。青蛇四面陡絕,璘圍其三面,購死士從瑪瑙後附葛至山背舉炮,賊惶駭,諸軍進攻,焚其茅屋。賊退入囤內,木石交下。將士冒死上,毀大柵二重,前後擊之,賊大敗,斬首一千九百有奇,七牌之賊亦盡。

乃分兵六道,攻克大小三渡關,乘勝抵海龍囤下。諸將俱攻囤前,獨水西安疆臣攻其後,相持四十餘日。其下受賊重賄,多與通,且潛以火藥遺賊,故賊不備。其後璘知之,與監軍者謀,令疆臣退一舍,璘移其處,置鐵牌百餘,距囤丈許,賊強弩無所施,又為箛板於柵前,賊每夜出劫,為釘傷,不敢復出。應龍勢窮,相聚哭。化龍初有令,諸將分日攻。六月六日,璘與吳廣當進兵。璘夜四更銜枚上,賊鼾睡,斬其守關者,樹白幟,鳴炮,賊大驚潰散,應龍自焚。廣軍亦至,賊盡平。

遂移師討皮林。皮林在湖、貴交,與九股苗相接。有吳國佐者,洪州司特峒寨苗也,桀黠無賴,其從父大榮以叛誅,國佐收其妾。黎平府持之急,遂反,自稱「天皇上將」,其黨石纂太稱「太保」,合攻上黃堡,誘敗參將黃沖霄,追至永從縣,殺守備張世忠,炙而啖之,掠屯堡七十餘,焚五開南城,陷永從,圍中潮所。時方征播州,未暇討。既平播,偏沅巡撫江鐸命璘與良玭合兵討之,良玭失利。明年,鐸移駐靖州,命璘率副將李遇文等七道進,璘擒苗酋銀貢等。遊擊宋大斌攻破特峒,焚之。國佐逃天浦四十八寨,復入古州毛洞,追獲之。石纂太逃廣西上巖山,指揮徐時達誘縛之。賊黨楊永祿率眾萬餘屯白沖。遊擊沈弘猷等夾攻,生擒永祿。諸苗悉平。

征播時,璘投賄李化龍家。會劉綎使為化龍父所麾,璘使走。化龍疏於朝,綎獲罪,璘獨免。後兵部尚書田樂推璘鎮貴州,給事中洪瞻祖遂劾璘營求。帝以璘東西積戰功,卒如樂議。貴東西二路苗:曰仲家苗,盤踞貴龍、平新間,為諸苗巨魁;在水硍山介銅仁、思石者曰山苗,紅苗之羽翼也。自平播後,貴州物力大屈,苗益生心,剽掠無虛日。三十三年冬,巡撫郭子章請於朝。明年四月,令璘軍萬人攻水硍,遊擊劉岳督宣慰安疆臣兵萬人攻西路,並克之。乃令璘移新添,獨攻東路,復克之。生獲酋十二人,斬首三千餘級,招降者萬三千餘人,部內遂靖。改鎮廣東,卒官。先敘平播功,加左都督,世蔭指揮使。既卒,以平苗功,贈太子太保,再蔭百戶。

吳廣,廣東人。以武生從軍,累著戰功,歷福建南路參將,坐事罷歸。會岑溪瑤反,總督陳大科檄廣從總兵童元鎮討之。將士少卻,廣手斬一卒以徇,遂大破之。論功,復故官。

萬歷二十五年,以副總兵從劉綎禦倭朝鮮,領水軍與陳璘相犄角,俘斬甚眾。甫班師,大征播州,擢廣總兵官,以一軍出合江。副將曹希彬以一軍出永寧,受廣節制。廣屯二郎壩,大行招徠。賊驍將郭通緒迎戰,將士襲走之。陶洪、安村、羅村三砦土官各出降,他部來歸者數萬,廣擇其壯者從軍。通緒扼穿崖囤,廣督土漢軍擊破之。劉綎、馬孔英已入播,廣猶頓二郎,總督李化龍趣之。乃議分四哨進攻崖門,別遣永寧女土官奢世續等督夷兵二千,扼桑木埡諸要害,以防饟道。諸將連破數囤,進營母豬塘。楊應龍懼,令通緒盡發關外兵拒敵。廣伏砲手五百於磨搶埡外南岡下,而遣裨將趙應科挑戰。埡夾兩山中,甚隘,通緒橫槊衝應科,應科佯北。通緒追出埡,遇伏,急旋馬,中炮墜,方躍上他馬,伏兵攢刺之殪,餘賊大奔。官軍逐北,賊盡降,遂薄崖門。徑小止容一騎,賊眾萬餘出關拒戰。希彬懸賞千金,士攀崖競進,追至第四關,關上男婦盡哭。賊黨自殺其魁羅進恩,率萬餘人出降。其第一關猶拒不下,廣乘夜疾進,奪其關,關內民爭獻牛酒。劉綎、馬孔英已入關,李應祥、陳璘猶在關外。廣合希彬軍連戰紅碗、水土崖、分水關皆捷,遂進營水牛塘,應龍大懼。知廣軍孤深入,謀欲襲之,乃遣人詐降。廣測其詐,堅壁以待,應龍擁眾三萬直衝大營,諸將殊死戰。會他將來援,賊乃退。廣遂與諸道軍逼海龍囤。賊詐令婦人乞降,哭囤上,又詐報應龍仰藥死,廣信之。已,知其詐,急燒第二關,奪三山,絕賊樵汲,賊益窘。旋與陳璘從囤後登,應龍急自焚死。獲其子朝棟,出應龍屍烈焰中。廣中毒矢,失聲,絕而復蘇,遂以本官鎮四川。踰年卒。

初,廣之頓二郎也,有言其受賄養寇者,詔謫充為事官。後論功贈都督同知,世廕千戶。

鄧子龍,豐城人。貌魁梧,驍捷絕倫。嘉靖中,江西賊起,掠樟樹鎮。子龍應有司募,破平之。累功授廣東把總。

萬歷初,從大帥張元勛討平巨盜賴元爵。已,從平陳金鶯、羅紹清。賊魁黃高暉逸,子龍入山生獲之。遷銅鼓石守備。尋擢署都指揮僉事,掌浙江都司。被論當奪職,帝以子龍犯輕,會麻陽苗金道侶等作亂,擢參將討之。大破賊,解散其黨。五開衛卒胡若盧等火監司行署,撻逐守備及黎平守。靖州、銅鼓、龍里諸苗咸響應為亂。子龍火其東門以致賊,而潛兵入北門,賊遂滅。

十一年閏二月,緬甸犯雲南。詔移子龍永昌。木邦部耿馬奸人罕虔與岳鳳同為逆,說緬酋莽應裏內侵,虔從掠千崖、南甸。已,引渡查理江,直犯姚關,灣甸土知州景宗真及弟宗材助之。子龍急戰攀枝樹下,陣斬宗真、虔,生獲宗材。虔子招罕、招色奔三尖山,令叔罕老率蒲人藥弩手五百阻要害,子龍餌蒲人以金,盡知賊間道。乃命裨將鄧勇等提北勝、蒗渠諸番兵,直搗賊巢,而預伏兵山後夾擊。夜半上,生擒招罕、招色、罕老及其黨百三十餘人,斬首五百餘級,尖山巢空,乃撫流移數千人。會劉綎亦俘岳鳳以獻。帝悅,進子龍副總兵,予世廕。無何,緬人復寇猛密,把總高國春大破之。子龍以犄角功,亦優敘。自是,蠻人先附緬者,多來附。

永昌、騰衝夙號樂土,自岳、罕猖亂,始議募兵,所募多亡命,乃立騰衝、姚安兩營。劉綎將騰軍,子龍將姚軍,不相能,兩軍斗。帝以兩將皆有功,置不問。既而綎罷,劉天俸代;天俸逮,遂以子龍兼統之。子龍抑騰兵,每工作,輒虐用之,而右姚兵。及用師隴川,子龍故為低昂,椎牛饗士,姚兵倍騰兵,騰兵大不堪,欲散去。副使姜忻令他將轄之,乃定。而姚兵久驕,因索餉作亂,由永昌、大理抵會城,所過剽掠。諸兵夾擊之,斬八十四級,俘四百餘人,亂始靖。子龍坐褫官下吏。

十八年,孟養賊思箇叛。子龍方對簿,巡撫吳定請令立功自贖,帝許之。命未至,定已與黔國公沐昌祚遣將卻之。無何,丁改十寨賊普應春、霸生等作亂,勢張甚。定大征漢土軍,令子龍軍其右,遊擊楊威軍其左,大破之,斬首一千二百級,招降六千六百人。帝為告謝郊廟,宣捷受賀,復子龍副總兵,署金山參將事。先是,猛廣土官思仁烝其嫂甘線姑,欲妻之,弗克。偕其黨丙測叛歸緬,數導入寇。二十年攻孟養,犯蠻莫,土同知思紀奔等練山。子龍擊敗之,乃去。子龍尋被劾罷歸。

二十六年,朝鮮用師。詔以故官領水軍,從陳璘東征。倭將渡海遁,璘遣子龍偕朝鮮統制使李舜臣督水軍千人,駕三巨艦為前鋒,邀之釜山南海。子龍素慷慨,年踰七十,意氣彌厲,欲得首功,急攜壯士二百人躍上朝鮮舟,直前奮擊,賊死傷無算。他舟誤擲火器入子龍舟,舟中火,賊乘之,子龍戰死。舜臣赴救,亦死。事聞,贈都督僉事,世蔭一子,廟祀朝鮮。

馬孔英者,宣府塞外降丁也,積戰功為寧夏參將。

萬曆二十年,哱拜反,引套寇入掠,孔英屢擊敗之。卜失兔入下馬關,從麻貴邀擊,大獲。進本鎮副總兵。二十四年九月,著力兔、宰僧犯平虜、橫城。孔英偕參將鄧鳳力戰,斬首二百七十有奇,賜金幣。令推大將缺,乃擢署都督僉事,以總兵官蒞舊任,尋進秩為真。二十七年,著力兔、宰僧復犯平虜、興武,孔英與杜桐等分道襲敗之。再入,又敗之。

會大征播州楊應龍。詔發陜西四鎮兵,令孔英將以往。兵分八道,孔英道南川,獨險遠,去應龍海龍囤六七百里。未至,重慶推官高折枝監紀軍事,請獨當一面。乃與參將周國柱先以石砫宣撫馬千乘兵破賊金築,復督酉陽宣撫冉御龍敗賊於官壩。孔英至軍,平茶、邑梅兵亦集,軍容甚壯。先師期一日入真州,用土官鄭葵、路麟為鄉道,別遣邊兵千扼明月關。諸軍鼓行前,連破四寨,次赤崖,抵清水坪、封寧關,破賊營十數,逼桑木關,關內民降者日千計。折枝結三大砦處之,禁殺掠,降者日眾,賊益孤。關為賊要害,山險箐深,賊憑高拒。乃令千乘、御龍出關左右,國柱搗其中。賊用標槍藥矢,銳甚。官軍殊死戰,奪其關,逐北至風坎關,賊復大敗。連破九杵、黑水諸關,苦竹、羊崖、銅鼓諸寨。國柱攻金子壩,無一人,疑有伏,焚空砦十九,嚴兵以待,賊果突出,擊敗之。孔英乃留王之翰兵守白玉臺,衛饟道,平茶、邑梅兵守桑木關,而親帥大軍進營金子壩。

應龍聞桑木關破,大懼,遣弟世龍及楊珠以銳卒劫之翰營。之翰走,殺饟卒無算。平茶兵來援,賊始退,孔英還擊世龍,復卻。裨將劉勝奮擊,賊乃奔。官軍進朗山口,由郎山進蒙子橋,深箐蓊翳,賊處處設伏,悉剿平之。應龍益懼,遣其黨詐降,謀為內應,折枝盡斬之,伏以待。珠果夜劫營,伏發,賊驚潰,追奔至高坪。已,奪賊養馬城,直抵海龍第二關下,賊守兵益多。孔英軍已深入,而諸道未有至者。酉陽、延綏兵皆退,賊躡殺官軍六十人。居數日,劉綎兵至,乃合兵連克海崖、海門諸關。賊走保囤上,竟覆滅。

初,總督李化龍剋師期,諸將莫利先入。孔英所將邊卒及諸土兵,皆獷悍,監紀折枝勇而有謀,故師獨先。八道圍海龍,諸將以囤後易攻,爭走其後,孔英獨壁關前。錄功,進都督同知,世蔭千戶。

久之,以總兵官鎮貴州。平金築、定番叛苗,生擒首惡阿包、阿牙等。已而欲襲黃柏山苗。苗知之,先發,敗官兵,匿不報。又誘執苗酋石阿四,稱陣擒冒功。為巡撫胡桂芳所劾,罷歸卒。

贊曰:播州之役,諸將用命,合八道師,歷時五月,僅乃克之,可謂勞矣。劉綎勇略冠諸將,勞最多,其後死事亦最烈。鄧子龍始事姚安,名與綎幹,垂老致命,廟祀海隅。昔人謂「武官不惜死」,兩人者蓋無愧於斯言也夫。


\end{pinyinscope}