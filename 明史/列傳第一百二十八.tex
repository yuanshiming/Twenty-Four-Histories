\article{列傳第一百二十八}

\begin{pinyinscope}
葉向高劉一燝兄一焜一煜韓爌朱國祚朱國禎何宗彥孫如游孫嘉績

葉向高,字進卿,福清人。父朝榮,養利知州。向高甫妊,母避倭難,生道旁敗廁中。數瀕死,輒有神相之。舉萬曆十一年進士,授庶吉士,進編修。遷南京國子司業,改左中允,仍視司業事。二十六年,召為左庶子,充皇長子侍班官。礦稅橫行,向高上疏,引東漢西邸聚錢事為鑒,不報。尋擢南京禮部右侍郎。久之,改吏部。再陳礦稅之害,又請罷遼東稅監高淮,語皆切至。妖書獄興,移書沈一貫力諫。一貫不悅,以故滯南京九年。後一貫罷,沈鯉亦去,朱賡獨當國。帝命增閣臣。三十五年五月,擢向高禮部尚書兼東閣大學士,與王錫爵、于慎行、李廷機並命。十一月,向高入朝,慎行已先卒,錫爵堅辭不出。明年,首輔賡亦卒,次輔廷機以人言久杜門,向高遂獨相。

當是時,帝在位日久,倦勤,朝事多廢弛,大僚或空署,士大夫推擇遷轉之命往往不下,上下乖隔甚。廷臣部黨勢漸成,而中官榷稅、開礦,大為民害。帝又寵鄭貴妃,福王不肯之國。向高用宿望居相位,憂國奉公,每事執爭效忠藎。帝心重向高,體貌優厚,然其言大抵格不用,所救正十二三而已。東宮輟講者五年,廷臣屢請不得命。三十七年二月,向高擇吉以請,亦不報。自是歲春秋必懇請,帝皆不納。貴妃王氏,太子生母也,薨四日不發喪。向高以為言,乃發喪。而禮官上其儀注,稽五日不行。向高復爭之,疏乃下。福王府第成,工部以之國請,向高擬旨上。帝不發,改明春。及期迫,向高請先飭儀衛舟車,帝不納。四十一年春,廷臣交章請,復諭改明春。已,忽傳旨,莊田非四萬頃不行,廷臣大駭。向高因進曰:「田四萬頃,必不能足,之國且無日,明旨又不信於天下矣。且王疏引祖制,而祖制無有是事。曩惟世宗時景王有之。景王久不之國,皇考在裕邸,危疑不安,此何可效也?」帝報曰:「莊田自有成例,且今大分已定,何猜?」向高因疏謝,言:「皇考時,名位雖未正,然講讀不輟,情意通。今東宮輟講八年,且不奉天顏久,而福王一日兩見,以故不能無疑。惟堅守明春期,而無以莊田藉口,天下疑自釋。」帝報福王無一日兩見事。

向高有裁斷,善處大事。錦衣百戶王曰乾者,京師奸人也,與孔學、趙宗舜、趙思聖等相訐告。刑官讞未竟,曰乾乃入皇城放炮上疏。刑官大驚,將擬曰乾死罪。曰乾遂訐奏鄭妃內侍姜嚴山與學等及妖人王三詔用厭勝術詛咒皇太后、皇太子死,擁立福王。帝震怒,繞殿行半日,曰:「此大變事,宰相何無言?」內侍即跪上向高奏。奏言:「此事大類往年妖書,然妖書匿名難詰,今兩造具在,一訊即情得。陛下當靜處之,稍張皇,則中外大擾。至其詞牽引貴妃、福王,尤可痛恨。臣與九卿所見皆同,敢以聞。」帝讀竟太息曰:「吾父子兄弟全矣。」明日,向高又言:「曰乾疏不宜發。發則上驚聖母,下驚東宮,貴妃、福王皆不安。宜留中,而別諭法司治諸奸人罪,且速定明春之國期,以息群喙,則天下帖然無事。」帝盡用其言,太子、福王得相安。貴妃終不欲福王之國,言明年冬太后七十壽,王宜留慶賀。帝令內閣宣諭。向高留上諭弗宣,請今冬預行慶壽禮,如期之國。帝遣中使至向高私邸,必欲下前諭。向高言:「外廷喧傳陛下欲假賀壽名留福王,約千人伏闕請。今果有此諭,人情益疑駴,將信王曰乾妖言,朝端必不靜。聖母聞之,亦必不樂。且潞王聖母愛子,亦居外籓,何心卷心卷福王為?」因封還手諭。帝不得已從之,福王乃之國。

向高嘗上疏言:「今天下必亂必危之道,蓋有數端,而災傷寇盜物怪人妖不與焉。廊廟空虛,一也。上下否隔,二也。士大夫好勝喜爭,三也。多藏厚積,必有悖出之釁,四也。風聲氣習日趨日下,莫可挽回,五也。非陛下奮然振作,簡任老成,布列朝署,取積年廢弛政事一舉新之,恐宗社之憂,不在敵國外患,而即在廟堂之上也。」其言絕痛切。帝知其忠愛,不能行。

初,向高入閣。未幾,陳用人理財策,力請補缺官,罷礦稅。見帝不能從,乃陳上下乖離之病。兩疏乞罷,帝不允。向高自獨相,即請增閣臣,帝不聽。及吏部尚書孫丕揚以薦賢不用求去,向高特疏請留,亦不報,遂引疾。屢諭,乃出視事。已,又言:「臣屢求去,輒蒙恩諭留。顧臣不在一身去留,而在國家治亂。今天下所在災傷死亡,畿輔、中州、齊魯流移載道,加中外空虛,人才俱盡。罪不在他人,臣何可不去。且陛下用臣,則當行其言。今章奏不發,大僚不補,起廢不行,臣微誠不能上達,留何益?誠用臣言,不徒縻臣身,臣溘先朝露,有餘幸矣。」帝不省。京師大水,四方多奏水旱。向高又言:「自閣臣至九卿臺省,曹署皆空,南都九卿亦止存其二。天下方面大吏,去秋至今,未嘗用一人。陛下萬事不理,以為天下長如此,臣恐禍端一發,不可收也。」帝亦不省。四十年春,向高以歷代帝王享國四十年以上者,自三代迄今止十君,勸帝力行新政。因復以用人行政請,亦不報。向高志不行,無月不求去,帝輒優旨勉留。向高復言:「臣進退可置不問,而百僚必不可盡空,臺諫必不可盡廢,諸方巡按必不可不代。中外離心,輦轂肘腋間,怨聲憤盈,禍機不測,而陛下務與臣下隔絕。帷幄不得關其忠,六曹不得舉其職,舉天下無一可信之人,而自以為神明之妙用,臣恐自古聖帝明王無此法也。」

先是,向高疾,閣中無人,章奏就其家擬旨者一月。及是,向高堅臥益久,即家擬旨如前,論者以為非體,向高亦自言其非,堅乞去。帝卒不命他相,遣鴻臚官慰留。至帝萬壽節,始起視事。其後,向高主癸丑會試,章奏皆送闈中,尤異事云。帝考選科道七十餘人,命久不下。向高懇請數十疏,越二年乃下。言官既多,攻擊紛起。帝心厭之,章悉留中。向高請盡付所司,定其去留。因言:「大臣者,小臣之綱。今六卿止趙煥一人,而都御史十年不補,彈壓無人,人心何由戢?」帝但責言官妄言,而大僚迄不補。向高請增置閣臣,章至百餘上,帝始用方從哲、吳道南。向高疏謝,因引退,優詔不允。

四十二年二月,皇太后崩。三月,福王之國。向高乞歸益數,章十餘上。至八月,允其去。向高以三載考績,進太子太保、文淵閣大學士;敘延綏戰功,加少保兼太子太保,改戶部尚書、武英殿;一品三載滿,加少傅兼太子太傅,改吏部尚書、建極殿。至是,命加少師兼太子太師,賜白金百,彩幣四,表裏大紅坐蟒一襲,遣行人護歸。

向高在相位,務調劑群情,輯和異同。然其時黨論已大起,御史鄭繼芳力攻給事中王元翰,左右兩人者相角。向高請盡下諸疏,敕部院評曲直,罪其論議顛倒者一二人,以警其餘,帝不報。諸臣既無所見得失,益樹黨相攻。未幾,又爭李三才之事,黨勢乃成。無錫顧憲成家居,講學東林書院,朝士爭慕與游。三才被攻,憲成貽書向高暨尚書孫丕揚,訟其賢。會辛亥京察,攻三才者劉國縉以他過掛察典,喬應甲亦用年例出外,其黨大嘩。向高以大體持之,察典得無撓,而兩黨之爭遂不可解。及後,齊、楚、浙黨人攻東林殆盡。浸尋至天啟時,王紹徽等撰所謂《東林點將錄》,令魏忠賢按氏名逐朝士。以向高嘗右東林,指目為黨魁云。

向高歸六年,光宗立,特詔召還。未幾,熹宗立,復賜敕趣之。屢辭,不得命。天啟元年十月還朝,復為首輔。言:「臣事皇祖八年,章奏必發臣擬。即上意所欲行,亦遣中使傳諭。事有不可,臣力爭,皇祖多曲聽,不欲中出一旨。陛下虛懷恭己,信任輔臣,然間有宣傳滋疑議。宜慎重綸音,凡事令臣等擬上。」帝優旨報聞。旋納向高請,發帑金二百萬,為東西用兵之需。

熹宗初政,群賢滿朝,天下欣欣望治。然帝本沖年,不能辨忠佞。魏忠賢、客氏漸竊威福,構殺太監王安,以次逐吏部尚書周嘉謨及言官倪思輝等。大學士劉一燝亦力求去。向高言:「客氏出復入,而一燝顧命大臣不得比保姆,致使人揣摩於奧穾不可知之地,其漸當防。」忠賢見向高疏刺己,恨甚。既而刑部尚書王紀削籍,禮部尚書孫慎行、都御史鄒元標先後被攻致仕去。向高爭不得,因請與元標同罷。帝不聽,而忠賢益恨向高。

向高為人光明忠厚,有德量,好扶植善類。再入相,事沖主,不能謇直如神宗時,然猶數有匡救。給事中章允儒請減上供袍服。奄人激帝怒,命廷杖。向高論救者再,乃奪俸一年。御史帥眾指斥宮禁,奄人請帝出之外,以向高救免。給事中傅櫆救王紀,將貶謫,亦以向高言僅奪俸。紀既罷去,御史吳甡、王祚昌薦之,部議以故官召。忠賢怒,將重譴文選郎,向高亦救免。給事中陳良訓疏譏權奄,忠賢摘其疏中「國運將終」語,命下詔獄,窮治主使。向高以去就爭,乃奪俸而止。熊廷弼、王化貞論死,言官勸帝速決。向高請俟法司覆奏,帝從之。有請括天下布政司、府、州、縣庫藏盡輸京師者,向高言:「郡邑藏已竭,籓庫稍餘。倘盡括之,猝有如山東白蓮教之亂,何以應之?」帝皆不納。

忠賢既默恨向高,而其時朝士與忠賢抗者率倚向高。忠賢乃時毛舉細故,責向高以困之。向高數求去。四年四月,給事中傅櫆劾左光斗、魏大中交通汪文言,招權納賄,命下文言詔獄。向高言:「文言內閣辦事,實臣具題。光斗等交文言事暖昧,臣用文言顯然。乞陛下止罪臣,而稍寬其他,以消縉紳之禍。」因力求速罷。當是時,忠賢欲大逞,憚眾正盈朝,伺隙動。得櫆疏喜甚,欲藉是羅織東林,終憚向高舊臣,并光斗等不罪,止罪文言。然東林禍自此起。

至六月,楊漣上疏劾忠賢二十四大罪。向高謂事且決裂,深以為非。廷臣相繼抗章至數十上,或勸向高下其事,可決勝也。向高念忠賢未易除,閣臣從中挽回,猶冀無大禍。乃具奏稱忠賢勤勞。朝廷寵待厚,盛滿難居,宜解事權,聽歸私第,保全終始。忠賢不悅,矯帝旨敘己功勤,累百餘言。向高駭曰:「此非奄人所能,必有代為草者。」探之,則徐大化也。忠賢雖憤,猶以外廷勢盛,未敢加害。其黨有導以興大獄者,忠賢意遂決。於是工部郎中萬燝以劾忠賢廷杖,向高力救,不從,死杖下。無何,御史林汝翥亦以忤奄命廷杖。汝翥懼,投遵化巡撫所。或言汝翥向高甥也,群奄圍其邸大噪。向高以時事不可為,乞歸已二十餘疏,至是請益力。乃命加太傅,遣行人護歸,所給賜視彞典有加。尋聽辭太傅,有司月給米五石,輿夫八。

向高既罷去,韓爌、朱國禎相繼為首輔,未久皆罷。居政府者皆小人,清流無所依倚。忠賢首誣殺漣,光斗等次第戮辱,貶削朝士之異己者,善類為一空云。熹宗崩,向高亦以是月卒,年六十有九。崇禎初,贈太師,謚文忠。

劉一燝,字季晦,南昌人。父曰材,嘉靖中進士,陜西左布政使。萬曆十六年,一燝與兄一焜、一煜並舉於鄉。越七年,又與一煜並舉進士。改庶吉士,授檢討。

一焜為考功郎,掌京察。大學士沈一貫欲庇其私人錢夢皋、鐘兆斗等,屬一燝為請。一燝謝不可,夢皋等竟以中旨留,由是忤一貫意。尋歷祭酒,詹事,掌翰林院事。四十五年春,京察,黨人用事,謀逐孫承宗、繆昌期等,一燝力保持得免。故事,掌院無滿歲不遷者,一景居四年,始遷禮部右侍郎,教習庶吉士。光宗即位,擢禮部尚書兼東閣大學士,參預機務,偕何宗彥、韓爌並命。時內閣止方從哲一人。

萬曆末年,神宗欲用史繼偕、沈紘。兩人方在籍,帝命召之。未及至,帝復命宗彥、一燝、爌。明日,復命朱國祚及舊輔葉向高。而宗彥、國祚、向高亦皆在籍,惟一燝、爌入直。甫拜命,帝已得疾,一燝偕諸臣召見乾清宮。明日九月朔,帝崩。諸臣入臨畢,一燝詰群奄:「皇長子當柩前即位,今不在,何也?」群奄東西走,不對。東宮伴讀王安前曰:「為李選侍所匿耳。」一燝大聲言:「誰敢匿新天子者?」安曰:「徐之,公等慎勿退。」遂趨入白選侍。選侍頷之,復中悔,挽皇長子裾。安直前擁抱,疾趨出。一燝見之,急趨前呼萬歲,捧皇長子左手,英國公張惟賢捧右手,掖升輦。及門,宮中厲聲呼:「哥兒卻還!」使使追躡者三輩。一燝傍輦疾行,翼升文華殿,先即東宮位,群臣叩頭呼萬歲。

事稍定,選侍猶趨還乾清。時選侍居乾清。一燝曰:「乾清不可居,殿下宜暫居慈慶。」皇長子心憚選侍,然之。一燝語安曰:「主上沖年,無母后。外庭有事,吾受過;宮中起居,公等不得辭責。」明日,周嘉謨及左光斗疏請移宮。時首輔從哲徘徊其間,已,又欲緩移宮。一燝曰:「本朝故事,仁聖,嫡母也,移慈慶;慈聖,生母也,移慈寧。今何日,可姑緩耶?」初五日,偕同官請即日降旨,踔立宮門以俟。選侍不得已,移噦鸞宮,天子復還乾清,事始大定。帝既踐阼,從哲被劾在告,一燝遂當國,與爌相得甚歡。念內廷惟王安力衛新天子,乃引與共事。安亦傾心嚮之。所奏請,無不從。發內帑,抑近侍,搜遺逸,舊德宿齒布滿九列,中外欣欣望治焉。

明年,天啟改元,沈陽失。廷臣多請復用熊廷弼。一燝亦言:「廷弼守遼一載,殘疆宴然,不知何故翦除。及下廷議,又皆畏懼,不敢異同。嗣後軍國大事,陛下當毅然主持,賴諸臣洗心滌慮,悉破雷同附和,其憂國奉公。」帝優旨褒答。尋有詔盡謫前排廷弼者姚宗文等官。言路多怨一燝。一燝嘗言:「任天下事者,惟六官。言路張,則六官無實政。善治天下者,俾六官任事,言路得繩其愆,言官陳事,政府得裁其是,則天下治。」於是一切條奏悉下部議,有不經者,詔格之。

初,選侍將移宮,其內豎李進忠、劉朝、田詔等盜內府秘藏,過乾清門仆,金寶墮地。帝怒,悉下法司,案治甚急。群奄懼,構蜚語,言帝薄先朝妃嬪,致選侍移宮日,跣足投井,以搖惑外廷。御史賈繼春遂上安選侍書。刑部尚書黃克纘、給事中李春曄、御史王業浩輩張大其辭,欲脫盜奄罪。帝惡繼春妄言,且疑其有黨,將嚴譴之。一燝謂天子新即位,輒疑臣下朋黨,異時奸人乘間,士大夫必受其禍。乃具疏開帝意,為繼春解,而反覆言朋黨無實。繼春得削籍去。御史張慎言、高弘圖疏救繼春,帝欲并罪,亦以一燝言而止。帝憾選侍甚,必欲誅盜奄。王安為司禮,亦惡之。諸奄百方救,卒不得。久而帝漸忘前事,安亦為魏忠賢排死,諸奄乃厚賄忠賢為地,而上疏辨冤。帝果免朝、詔死,下其疏法司。一燝執奏,詔等議誅久,無可雪,疏直下部,前無此制。帝不得已,下其疏於閣。一燝復言:「此疏外不由通政司,內不由會極門,例不當擬旨,謹封還原疏。」由是忠賢輩大恨,朝等亦竟免死,益任用。

定陵工成,忠賢欲以為功。一燝援故事,內臣非司禮掌印及提督陵工不得濫廕,止擬加恩三等。諸言官論客氏被謫者,一燝皆疏救,又請出客氏於外。及言官交章論沈紘,紘疑一燝主之,與忠賢、客氏等比,而齮一燝。一燝持大體,不徇言路意。言路頗怨。又密窺魏、客等漸用事,一燝勢孤,是年四月,候補御史劉重慶遂力詆一燝不可用。帝怒謫重慶。一燝再論救,不聽。而職方郎中餘大成、御史安伸、給事中韋蕃、霍維華交章劾一燝。帝不問。既而維華外轉,其同官孫傑疑一燝屬嘉謨為之,上疏力攻一燝。一燝疏辨求罷。帝已慰留,給事中侯震暘、御史陳九疇復劾之,并刺其結納王安。於是一燝四疏乞歸,忠賢從中主之,傳旨允其去。

先是,從哲去,帝數稱一燝為首輔,一燝不敢當,虛位俟葉向高。及向高至,入讒言,謂一燝尼己。至是,知其無他,力稱一燝有翼衛功,不可去。帝復慰留,一燝堅臥不起。二年三月,疏十二上,乃令乘傳歸。既歸,兵部尚書張鶴鳴興奸細杜茂、劉一獻獄,欲指一獻為一燝族,株連之。刑部尚書王紀不可,遂被斥去,而一燝得白。鶴鳴,一燝向所推轂者也。已而忠賢大熾,矯旨責一燝誤用廷弼,削官,追奪誥命,勒令養馬。崇禎改元,詔復官,遣官存問。一燝在位,累加少傅、太子太傅、吏部尚書、中極殿大學士。八年卒,贈少師。福王時,追謚文端。

一焜,字元丙。萬歷二十年進士。授行人。歷考功郎中,佐侍郎楊時喬典京察,盡斥執政私人。已,改文選,遷太常少卿,以憂去。久之,由故官擢右僉都御史,巡撫浙江。帝遣中官曹奉建鎮海寺於普陀山。一焜偕巡按李邦華爭不可,不聽。織造中官劉成卒,一焜屢疏請勿遣代。已得請,會命中官呂貴護成遺裝,奸人遂請留貴督織造,疏直達禁中。一焜與邦華極論其罪,帝卒命貴代之。一焜復疏爭,不報。貴既任,條行十事,多侵擾。一焜疏駁,且禁治其爪牙,貴為斂威。一焜以暇築龕山海塘千二百丈,浚復餘杭南湖,民賴其利。御史沈珣誣訐其贓私,一焜自引去。卒,贈工部右侍郎。

一煜,兵部郎中。

韓爌,字象雲,蒲州人。萬曆二十年進士。選庶吉士。進編修,歷少詹事,充東宮講官。四十五年,擢禮部右侍郎,協理詹事府。久之,命教習庶吉士。

泰昌元年八月,光宗嗣位,拜禮部尚書兼東閣大學士,入參機務。未幾,光宗疾大漸,與方從哲、劉一燝同受顧命。時宮府危疑,爌竭誠翼衛,中外倚以為重。大帥李如柏、如楨兄弟有罪,當逮治,中旨寬之。爌與一燝執奏,逮如律。以登極恩,加太子太保、戶部尚書、文淵閣大學士。從哲去,一燝當國,爌協心佐理。

天啟元年正月,兩人以帝為皇孫時,未嘗出閣讀書,請於十二日即開經筵,自後日講不輟,從之。遼陽失,都城震驚。爌、一燝以人情偷玩,擬御札戒勵百官,共圖實效,帝納之。廷臣以兵餉大絀,合詞請發帑,爌、一燝亦以為言,詔發百萬兩。大婚禮成,加少保、吏部尚書、武英殿大學士,廕一子尚寶司丞。未幾,以貴州平苗功,加少傅、太子太傅、建極殿大學士。帝封乳母客氏為奉聖夫人,大婚成,當出外,仍留之宮中。御史畢佐周切諫,六科、十三道復連署爭,皆不納。爌、一燝引祖制為言,乃命俟梓宮發引,擇日出宮。

二年四月,禮部尚書孫慎行劾方從哲用李可灼紅丸藥,罪同弒逆,廷議紛然。一燝已去位,爌特疏白其事,曰:

先帝以去年八月朔踐阼。臣及一燝以二十四日入閣。適鴻臚寺官李可灼云有仙丹欲進。從哲愕然,出所具問安揭,有「進藥十分宜慎」語。臣等深以為然,即諭之去。二十七日召見群臣,先帝自言不用藥已二十餘日。至二十九日遇兩內臣,言帝疾已大漸,有鴻臚寺官李可灼來思善門進藥。從哲及臣等皆言彼稱仙丹,便不敢信。是日仍召見。諸臣問安畢,先帝即顧皇上,命臣等輔佐為堯、舜。又語及壽宮,臣等以先帝山陵封,則云:「是朕壽宮。」因問有鴻臚官進藥。從哲奏云:「李可灼自謂仙丹,臣等未敢信。」先帝即命傳宣。臣等出,移時可灼至,同入診視,言病源及治法甚合。先帝喜,命速進。臣等復出,令與諸醫商榷。一燝語臣,其鄉兩人用此,損益參半。諸臣相視,實未敢明言宜否。須臾,先帝趣和藥,臣等復同入。可灼調以進,先帝喜曰:「忠臣,忠臣。」臣等出,少頃,中使傳聖體服藥後暖潤舒暢,思進飲膳,諸臣歡躍而退。比申末,可灼出云:「聖上恐藥力不繼,欲再進一丸。」諸醫言不宜驟。乃傳趣益急,因再進訖。臣等問再服復何狀,答言平善如初。此本日情事也。次日,臣等趨朝,而先帝已於卯刻上賓矣,痛哉!

方先帝召見群臣時,被袞憑几,儼然顧命。皇上焦顏侍側,臣等環跪徬徨,操藥而前,籲天以禱。臣子際此,憾不身代。凡今所謂宜慎宜止者,豈不慮於心,實未出於口,抑且不以萌諸心。念先帝臨御雖止旬月,恩膏實被九垓。為臣子者宜何如頌揚,何如紀述。乃禮臣忠憤之激談,與遠邇驚疑之紛議,不知謂當時若何情景,而進藥始末實止如此。若不據實詳剖,直舉非命之凶稱,加諸考終之令主,恐先帝在天之靈不無恫怨,皇上終天之念何以為懷。乞渙發綸音,布告中外,俾議法者勿以小疑成大疑,編摹者勿以信史為謗史。

文震孟建言獲譴,論救甚力。三年,以山東平妖賊功,加少師、太子太師。時葉向高當國,爌次之。及楊漣劾魏忠賢二十四大罪,忠賢頗懼,求援於爌。爌不應,忠賢深銜之。既向高罷,爌為首輔,每事持正,為善類所倚。然向高有智術,籠絡群奄,爌惟廉直自持,勢不能敵。而同官魏廣微又深結忠賢,

遍引邪黨。其冬,忠賢假會推事逐趙南星、高攀龍,爌急率朱國禎等上言:「陛下一日去兩大臣,臣民失望。且中旨徑宣,不復到閣,而攀龍一疏,經臣等擬上者,又復更易,大駭聽聞,有傷國體。」忠賢益不悅,傳旨切責。未幾,又逐楊漣、左光斗、陳于廷,朝政大變,忠賢勢益張。

故事,閣中秉筆止首輔一人。廣微欲分其柄,囑忠賢傳旨,諭爌同寅協恭,而責次輔毋伴食。爌惶懼,即抗疏乞休。略言:「臣備位綸扉,咎愆日積。如詰戎宜先營衛,而觀兵禁掖,無能紓宵旰憂。忠直尚稽召還,而榜掠朝堂,無能回震霆怒。後先諸臣之罷斥,諭旨中出之紛更,不能先時深念,有調劑之方,又不能臨事執持,為封還之戇。皆臣罪之大者。皇上釋此不問,責臣以協恭,責同官以協贊。同官奉詔以從事,臣欲補過無由矣。乞亟褫臣官,為佐理溺職之戒。」得旨:「卿親承顧命,當竭忠盡職。乃歸非於上,退有後言。今復悻悻求去,可馳驛還籍。」諸輔臣請如故事,加以體貌,不報。爌疏謝,有「左右前後務近端良,重綸綍以重仕途,肅紀綱以肅朝寧」語。忠賢及其黨益恨。爌去,朱國禎為首輔。李蕃攻去之,顧秉謙代其位。公卿庶僚,皆忠賢私人矣。

五年七月,逆黨李魯生劾爌,削籍除名。又假他事坐贓二千,斃其家人於獄。爌鬻田宅,貸親故以償,乃棲止先墓上。

莊烈帝登極,復故官。崇禎元年,言者爭請召用,為逆黨楊維垣等所扼,但賜敕存問,官其一子。至五月,始遣行人召之。十二月還朝,復為首輔。帝御文華後殿閱章奏,召爌等,諭以擬旨務消異同,開誠和衷,期於至當。爌等頓首謝,退言:「上所諭甚善,而密勿政機,諸臣參互擬議,不必顯言分合。至臣等晨夕入直,勢不能報謝賓客。商政事者,宜相見於朝房,而一切禁私邸交際。」帝即諭百僚遵行。

二年正月,大學士劉鴻訓以張慶臻敕書事被重譴,爌疏救,不聽。溫體仁訐錢謙益,御史任贊化亦疏訐體仁。帝召見廷臣,體仁力詆贊化及御史毛羽健為謙益死黨。帝怒,切責贊化。爌請寬贊化以安體仁。帝因謂:「進言者不憂國而植黨,自名東林,於朝事何補?」爌退,具揭言:「人臣不可以黨事君,人君亦不可以黨疑臣。但當論其才品臧否,職業修廢,而黜陟之。若戈矛妄起於朝堂,畛域橫分於宮府,非國之福也。」又率同官力救贊化,不納。皇長子生,請盡蠲天下積逋,報可。

時大治忠賢黨,爌與李標、錢龍錫主之。列上二百六十二人,罪分六等,名曰「欽定逆案」,頒行天下。言者爭擊吏部尚書王永光,南京禮部主事王永吉言之尤力。帝怒,將罪之。爌等言永吉不宥,永光必不安,乃止奪俸一年。工部尚書張鳳翔奏廠、庫積弊。帝怒,召對廷臣詰責。巡視科道王都、高賚明二人力辨,帝命錦衣官執之,爌、標、龍錫並救解。而是日永光以羽健疏劾,請帝究主使者。爌退,申救都等,因言永光不宜請究言官。帝不納,然羽健卒獲免。

初,熊廷弼既死,傳首九邊,屍不得歸葬。至是,其子詣闕疏請。爌等因言:「廷弼之死,由逆奄欲殺楊漣、魏大中,誣以行賄,因盡殺漣等,復懸坐廷弼贓銀十七萬,刑及妻孥,冤之甚者。」帝乃許收葬。

時遼事急,朝議汰各鎮兵。又以兵科給事中劉懋疏,議裁驛卒。帝以問爌,爌言:「汰兵止當清占冒及增設冗兵爾。衝地額兵不可汰也。驛傳疲累,當責按臣核減,以蘇民困,其所節省,仍還之民。」帝然之。御史高捷、史褷以罪免,永光力引之。都御史曹於汴持不可,永光再疏爭。爌言,故事當聽都察院咨用。帝方眷永光,不從。九月,以將行慶典,請停秋決,亦不從。

時逆案雖定,永光及袁弘勛、捷、褷輩日為翻案計。至十月,大清兵入畿甸,都城戒嚴。初,袁崇煥入朝,嘗與錢龍錫語邊事。龍錫,東林黨魁也,永光等謀因崇煥興大獄,可盡傾東林。倡言大清兵之入,由崇煥殺毛文龍所致。捷遂首攻龍錫,逐之。明年正月,中書舍人加尚寶卿原抱奇故由輸貲進,亦劾爌主款誤國,招寇欺君,郡邑殘破,宗社阽危,不能設一策,拔一人,坐視成敗,以人國僥倖,宜與龍錫並斥。其言主款者,以爌,崇煥座主也。帝重去爌,貶抱奇秩。無何,左庶子丁進以遷擢愆期怨爌,亦劾之,而工部主事李逢申劾疏繼上。爌即三疏引疾。詔賜白金彩幣,馳驛遣行人護歸,悉如彝典。進、逢申並爌會試所舉士也。爌先後作相,老成慎重。引正人,抑邪黨,天下稱其賢,獨嘗庇王永光云。十七年春,李自成陷蒲州,迫爌出見,不從。賊執其孫以脅。廣止一孫,乃出見,賊釋其孫。爌歸,憤鬱而卒,年八十矣。

硃國祚,字兆隆,秀水人。萬曆十一年進士第一。授修撰。進洗馬,為皇長子侍班官,尋進諭德。日本陷朝鮮,石星惑沈惟敬言,力主封貢。國祚面詰星:「此我鄉曲無賴,因緣為奸利耳,公獨不計辱國乎?」星不能用。二十六年,超擢禮部右侍郎。湖廣稅監陳奉橫甚。國祚貽書巡按御史曹楷,令發其狀。帝怒,幾逮楷,奉亦因此撤去。尚書餘繼登卒,國祚攝部事。

時皇長子儲位未定,冠婚踰期,國祚屢疏諫。戚臣鄭國泰請先冠婚,後冊立。國祚抗疏言:「本朝外戚不得與政事。冊立大典,非國泰所宜言。況先冊立,後冠婚,其儀仗、冠服之制,祝醮、敕戒之辭,升降、坐立之位,朝賀拜舞之節,因名制分,因分制禮,甚嚴且辨。一失其序,名分大乖。違累朝祖制,背皇上明綸,犯天下清議,皆此言也。」又言:「冊立之事,理不可緩。初謂小臣激聒,故遲之。後群臣勿言,則曰待嫡。及中官久無所出,則曰皇長子體弱,須其強。今又待兩宮落成矣。自三殿災,朝廷大政令率御文華殿。三禮之行,在殿不在宮。頃歲趣辦珠寶,戶部所進,視陛下大婚數倍之。遠近疑陛下借珠寶之未備,以遲典禮。且詔旨採辦珠寶,額二千四百萬,而天下賦稅之額乃止四百萬。即不充國用,不給邊需,猶當六年乃足。必待取盈而後舉大禮,幾無時矣。」已,又言:「太祖、成祖、仁宗,即位初,即建儲貳。宣宗、英宗冊為皇太子時,止二歲,憲宗、孝宗止六歲,陛下亦以六歲。未聞年十九而不冊立者。」國祚攝尚書近二年,爭國本至數十疏,儲位卒定。

陜西狄道山崩,其南湧小山五,國祚請修省。社稷壇枯樹生煙,復陳安人心、收人望、通下情、清濫獄四事。雲南巡撫陳用賓進土物,國祚劾之。尋轉左侍郎,改吏部。御史湯兆京劾其縱酒踰檢,帝不問,國祚遂引疾歸。

光宗即位,以國祚嘗侍潛邸,特旨拜禮部尚書兼東閣大學士,入閣參機務。天啟元年六月還朝。尋加太子太保,進文淵閣。國祚素行清慎,事持大體,稱長者。明年會試,故事,總裁止用內閣一人,是科用何宗彥及國祚,有譏其中旨特用者。國祚既竣事,即求罷,優詔不允。都御史鄒元標侍經筵而躓,帝遣中使問狀。國祚進曰:「元標在先朝直言受杖,故步履猶艱。」帝為之改容。刑部尚書王紀為魏忠賢所逐,國祚合疏救,復具私揭爭之。紀為禮部侍郎時,嘗以事忤國祚者也。

三年,進少保、太子太保、戶部尚書,改武英殿。十三疏乞休,詔加少傅兼太子太傅,乘傳歸。明年卒。贈太傅,謚文恪。從子大啟,文選郎中,終刑部左侍郎。

同時朱國禎,字文寧,烏程人。萬曆十七年進士。累官祭酒,謝病歸,久不出。天啟元年,擢禮部右侍郎,未上。三年正月,拜禮部尚書兼東閣大學士,與顧秉謙、朱延禧、魏廣微並命。閣中已有葉向高、韓爌、何宗彥、朱國祚、史繼偕,又驟增四人,直房幾不容坐。六月,國禎還朝,秉謙、延禧以列名在後,謙居其次。改文淵閣大學士,累加少保兼太子太保。魏忠賢竊國柄,國禎佐向高,多所調護。四年夏,楊漣劾忠賢,廷臣多勸向高出疏,至有詬者。向高慍甚,國禎請容之。及向高密奏忤忠賢,決計去,謂國禎曰:「我去,蒲州更非其敵,公亦當早歸。」蒲州謂爌也。向高罷,爌為首輔,爌罷,國禎為首輔。廣微與忠賢表裏為奸,視國禎蔑如。其冬為逆黨李蕃所劾,三疏引疾。忠賢謂其黨曰:「此老亦邪人,但不作惡,可令善去。」乃加少傅,賜銀幣,蔭子中書舍人,遣行人送歸,月廩、輿夫皆如制。崇禎五年卒。贈太傅,謚文肅。

何宗彥,字君美。其父由金谿客隨州,遂家焉。宗彥舉萬曆二十三年進士。累官詹事。四十二年遷禮部右侍郎,署部事。福王之國河南,請求無已。宗彥上疏,言可慮者有六,帝不聽。又屢疏請東宮講學,皇孫就傅,及瑞、惠、桂三王婚禮。太子生母王貴妃薨,不置守墳內官,又不置墳戶贍地,宗彥力爭之。梃擊事起,宗彥因言:「天下疑陛下薄太子久。太子處積輕之勢,致慈慶宮門止守以耄年二內侍,中門則寂無一人。乞亟下張差廷訊,凡青宮諸典禮,悉允臣部施行,宗社幸甚。」不報。尋轉左侍郎,署部如故。四十四年冬,隆德殿災,宗彥請通下情,修廢政,補曠官。明年,皇長孫年十三,未就傅,宗彥再疏力言。自是頻歲懇請,帝終不納。四十六年六月,京師地震。上修省三事。時帝不視朝已三十載,朝政積弛,庶官盡曠。明年秋,遼事益棘。宗彥率僚屬上言:「自三路喪師,開原、鐵嶺相繼沒,沈陽孤危。請陛下臨朝,與臣等面籌兵食大計。」帝亦不報。

宗彥清修有執。攝尚書事六年,遇事侃侃敷奏,時望甚隆。其年十二月,會推閣臣,廷臣多首宗彥,獨吏科給事中張延登不署名,遂不獲與。宗彥旋乞假去。御史薛敷政、蕭毅中、左光斗、李徵儀、倪應春、彭際遇、張新詔等交章惜之,而延登同官亓詩教、薛鳳翔又屢疏糾駁。其時齊黨勢盛,非同類率排去之。宗彥無所附麗,故終不安其位。明年,神宗崩,光宗立,即家拜禮部尚書兼東閣大學士。天啟元年夏還朝。屢加少師兼太子太師、吏部尚書、建極殿大學士。四年正月卒官,贈太傅,謚文毅。

弟宗聖,由鄉舉歷官工部主事。以附魏忠賢,驟加本部右侍郎。崇禎初,削籍,論配,名麗逆案。

孫如游,字景文,餘姚人,都御史燧曾孫也。萬曆二十三年進士。累官禮部右侍郎。四十七年冬,左侍郎何宗彥去位,署印無人,大學士方從哲屢以如游請。明年三月始得命。部事叢積,如游決遣無滯。時白蓮、無為諸邪教橫行,宗彥嘗疏請嚴禁,如游復申其說。帝從之。七月,帝疾大漸,偕諸大臣受顧命。

帝崩,鄭貴妃懼禍,深結李選侍,為請封后。選侍喜,亦為請封太后以悅之。楊漣語如游曰:「皇長子非選侍所愛。選侍后,嫡矣,他日將若何?亟白執政,用遺詔舉冊立。登極三日,公即援詔以請。」如游然之。八月朔,光宗即位。三日,如游請建東宮,帝納之。俄遵遺旨諭閣臣,封貴妃為皇太后。如游奏曰:「考累朝典禮,以配而后者,乃敵體之經;以妃而后者,則從子之義。祖宗以來,豈無抱衾之愛,而終引去席之嫌,此禮所不載也。先帝念貴妃勞,不在無名之位號;陛下體先帝志,亦不在非分之尊崇。若義所不可,則遵命非孝,遵禮為孝。臣不敢曲徇,自蹈不忠之罪。」疏入,未報。

如游尋進本部尚書。帝既命建東宮,又言皇長子體質清弱,稍緩冊立期。如游力持不可。二十三日,命封選侍為皇貴妃。期已定矣,越三日,帝又趣之。如游奏曰:「先奉諭上孝端皇后、孝靖皇太后尊謚,又封郭元妃、王才人為皇后,禮皆未竣,貴妃之封宜在後。既聖諭諄切,且有保護聖儲功,即如先所定期,亦無不可。」帝許之。選侍以貴妃為未足,必欲得皇后。二十九日,再召廷臣,選侍迫皇長子言之。如游曰:「上欲封選侍為皇貴妃,當即具儀進。」帝漫應曰:「諾。」選侍聞,大不悅。明日,帝崩,朝事大變。如游請改冊封期,報可。熹宗為皇孫時,未就傅。即位七日,如游即請開講筵,亦報可。

十月,命以東閣大學士入參機務。言者詆其不由廷推,交章論列。如游亦屢乞去,帝輒勉留。天啟元年二月,上疏言:「祖宗任用閣臣,多由特簡。遠者無論,在世廟,則有張璁、桂萼、方獻夫、夏言、徐階、袁煒、嚴訥、李春芳;在穆廟,則有陳以勤、張居正、趙貞吉;在神廟,則有許國、趙志皋、張位。即皇考之用朱國祚,亦特簡也。今陛下沖齡,臣才品又非諸臣比,有累至尊知人之明。乞速賜骸骨,還田里。」帝仍留之。如游十四疏乞去,乃加太子太保、文淵閣大學士,遣官護送,廕子給賜悉如彞典。家居四年卒。贈少保,謚文恭。

孫嘉績,字碩膚。崇禎十年進士。授南京工部主事,召改兵部。大清兵薄都城,按營不動,眾莫測。嘉績曰:「此待後至者,即舉眾南下爾。」越三日,蒙古兵數萬果從青山口入,即日南下。於是尚書楊嗣昌以嘉績知兵,調為職方員外郎。進郎中。督師中官高起潛譖之,會有發其納賄事,遂下獄。已,黃道周亦下獄。嘉績躬親飲食湯藥,力調護之,因從受《易》。會諸生塗仲吉疏救道周,帝益怒,移獄錦衣嚴訊。諸生與道周往來者多詭詞自脫,獨嘉績無所隱。擬雜犯死罪,繼擬煙瘴充軍,皆不允。保定總督張福臻陛見,薦嘉績才,請用為參謀,不聽。徐石麒為刑部尚書,具爰書奏,乃釋之。福王時,起九江兵備僉事,未赴。魯王監國紹興,擢右僉都御史,累進東閣大學士。王航海,嘉績從至舟山。其年遘疾卒。

贊曰:熹宗初,葉向高以宿望召起,海內正人倚以為重,卒不能有所匡救。蓋政柄內移,非一日之積,勢固無如何也。劉一燝、韓爌諸人,雖居端揆之地,而宵小比肩,權璫掣肘,紛撓杌隉,幾不自全。硃國祚、何宗彥絀於黨人,孫如游又皆以中旨特用,為外廷所詬。於是而知明良相遭,誠千載之一遇也夫!


\end{pinyinscope}