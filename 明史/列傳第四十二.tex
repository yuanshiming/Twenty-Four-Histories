\article{列傳第四十二}

\begin{pinyinscope}
○張輔高士文徐政黃福劉俊呂毅劉昱陳洽侯保馮貴伍云陳忠李任等李彬柳升崔聚史安陳鏞李宗昉潘禋梁銘王通陶季容陳汀

張輔,字文弼,河間王玉長子也。燕師起,從父力戰,為指揮同知。玉歿東昌,輔嗣職。從戰夾河、槁城、彰德、靈璧,皆有功。從入京師,封信安伯,祿千石,予世券。妹為帝妃。邱福、硃能言輔父子功俱高,不可以私親故薄其賞。永樂三年進封新城侯,加祿三百石。

是時安南黎季犛弒其主,自稱太上皇,立子蒼為帝。其故王之孫陳天平自老撾來奔,季犛佯請歸國。帝遣都督黃中以兵五千送之,前大理卿薛為輔。季犛伏兵芹站,殺天平,亦死。帝大怒,命成國公朱能為征夷將軍,輔為右副將軍,帥豐城侯李彬等十八將軍,兵八十萬,會左副將軍西平侯沐晟,分道進討。兵部尚書劉俊贊軍事,行部尚書黃福、大理寺卿陳洽給饋餉。

四年十月,能卒於軍,輔代領其眾。自憑祥進師,度坡壘關,望祭安南境內山川,檄季犛二十罪。進破隘留、雞陵二關,道芹站,走其伏兵,抵新福。晟軍亦自雲南至,營於白鶴。安南有東、西二都,依宣、洮、沲、富良四江為險,賊緣江南北岸立柵,聚舟其中,築城於多邦隘,城柵橋艦相連九百餘里,兵眾七百萬,欲據險以老輔師。輔自新福移軍三帶州,造船圖進取。會帝聞朱能卒,敕拜輔為將軍,制詞以李文忠代開平王常遇春為比,且言乘冬月瘴癘未興,宜及時滅賊。十二月,輔軍次富良江北,遣驃騎將軍朱榮破賊嘉林江,遂與晟合軍進攻多邦城。佯欲他攻以懈賊,令都督黃中等將死士,人持炬火銅角,夜四鼓,越重濠,雲梯傅其城。都指揮蔡福先登,士蟻附而上,角鳴,萬炬齊舉,城下兵鼓噪繼進,遂入城。賊驅象迎戰。輔以畫獅蒙馬衝之,翼以神機火器。象皆反走,賊大潰。斬其帥二人,追至傘圓山,盡焚緣江木柵,俘斬無算。進克東都,輯吏民,撫降附,來歸者日以萬計。遣別將李彬、陳旭取西都,又分軍破賊援兵。季犛焚宮室倉庫逃入海,三江州縣皆望風降。

明年春,輔遣清遠伯王友等濟自注江,悉破籌江、困枚、萬劫、普賴諸寨,斬首三萬七千餘級。賊將胡杜聚舟盤灘江。輔使降將陳封襲走之,盡得其舟。遂定東潮、諒江諸府州。尋擊破季犛舟師於木丸江,斬首萬級,擒其將校百餘人,溺死者無算。追至膠水縣悶海口,還軍。築城鹹子關,令都督柳升守之。已,賊由富良江入。輔與晟夾岸迎戰。升等以舟師橫擊,大破之,馘斬數萬,江水為赤,乘勝窮追。時天旱水淺,賊棄舟陸走。官軍至,忽大雨水漲,遂畢渡。五月至奇羅海口,獲季犛及其子蒼,並偽太子諸王將相大臣等,檻送京師。安南平。得府州四十八,縣一百八十,戶三百十二萬。求陳氏後不得,遂設交阯布政司,以其地內屬。自唐之亡,交阯淪於蠻服者四百餘年,至是復入版圖。帝為詔告天下,諸王百官奉表稱賀。

六年夏,輔振旅還京師。再賜宴奉天殿,帝為賦《平安南歌》,進封英國公,歲祿三千石,予世券。其年冬,陳氏故臣簡定復叛。命沐晟討之,敗績於生厥江。明年春,復命輔佩征虜將軍印,帥師往討。時簡定已僭稱越上皇,別立陳季擴為皇,勢張甚。輔就叱覽山伐木造舟,招諒江北諸避寇者復業。遂進至慈廉州,破喝門江,克廣威州孔目柵。遇賊鹹子關。賊舟六百餘,保江東南岸。輔帥陳旭等以劃船戰,乘風縱火,擒賊帥二百餘人,盡得其舟。追至太平海口。賊將阮景異以三百艘迎敵,復大破之。於是季擴自言陳氏後,遣使求紹封。輔曰:「向者遍索陳王後不應,今詐也。吾奉命討賊,不知其他。」遂遣朱榮、蔡福等以步騎先進,輔帥舟師繼之。自黃江至神投海,會師清化,分道入磊江,獲簡定於美良山中,及其黨送京帥。八年正月進擊賊餘黨,斬數千人,築京觀,惟季擴未獲。帝留沐晟討之,召輔班師。謁帝於興和,命練兵宣府、萬全,督運北征。

時陳季擴雖請降,實無悛心。乘輔歸,攻剽如故,晟不能制。交人苦中國約束,又數為吏卒侵擾,往往起附賊,乍服乍叛,將帥益玩寇。九年正月,仍命輔與沐晟協力進討。輔至,申軍令。都督黃中素驕,違節度。詰之不遜,斬以徇。將士惕息,無敢不用命者。其年七月破賊帥阮景異於月常江,獲船百餘,生擒偽元帥鄧宗稷等,又捕斬別部賊首數人。以瘴癘息兵。明年八月擊賊於神投海。賊舟四百餘,分三隊,銳甚。輔衝其中堅,賊卻,左右隊迭進,官軍與相鉤連,殊死戰。自卯至巳,大破賊,擒渠帥七十五人。進軍乂安府,賊將降者相繼。

十一年冬,與晟會順州,戰愛子江。賊驅象前行。輔戒士卒,一矢落象奴,二矢射象鼻。象奔還,自蹂其眾。裨將楊鴻、韓廣、薛聚等乘勢繼進,矢落如雨,賊大敗。擒其帥五十六人。追至愛母江,盡降其眾。明年正月進至政平州。聞賊屯暹蠻、昆蒲諸柵,遂引兵往。懸崖側徑,騎不得進。輔與將校徒步行山箐中。夜四鼓抵其巢,悉擒阮景異、鄧容等。季擴走老撾,遣指揮師祐以兵索之,破其三關。遂縛季擴及其孥送京師。賊平。承制,以賊所取占城地,設升、華、思、義四州,增置衛所,官其降人,留軍守之而還。十三年春至京。旋命為交阯總兵官往鎮。而餘寇陳月湖等復作亂,輔悉討平之。十四年冬召還。

輔凡四至交阯,前後建置郡邑及增設驛傳遞運,規畫甚備。交人所畏惟輔。輔還一年而黎利反,累遣將討之,無功。至宣德時,柳升敗沒,王通與賊盟,倉卒引還。廷議棄交阯,輔爭之不能得也。

仁宗即位,掌中軍都督府事,進太師,並支二俸。尋命輔所受太師俸於北京倉支給。時百官俸米皆給於南京,此蓋特恩云。成祖喪滿二十七日,帝素冠麻衣以朝。而群臣皆已從吉,惟輔與學士楊士奇服如帝。帝嘆曰:「輔,武臣也,而知禮過六卿。」益見親重。尋命知經筵事,監修《實錄》。

宣德元年,漢王高煦謀反,誘諸功臣為內應,潛遣人夜至輔所。輔執之以聞,盡得其反狀,因請將兵擊之。帝決策親征,命輔扈行。事平,加祿三百石。輔威名益盛,而久握兵。四年,都御史顧佐請保全功臣。詔輔解府務,朝夕侍左右,謀畫軍國重事,進階光祿大夫左柱國,朝朔望。英宗即位,加號翊連佐理,知經筵、監修《實錄》如故。

輔雄毅方嚴,治軍整肅,屹如山岳。三定交南,威名聞海外。歷事四朝,連姻帝室,而小心敬慎,與蹇、夏、三楊,同心輔政。二十餘年,海內宴然,輔有力焉。王振擅權,文武大臣望塵頓首,惟輔與抗禮。也先入犯,振導英宗親征,輔從行,不使預軍政。輔老矣,默默不敢言。至土木,死於難,年七十五。追封定興王,謚忠烈。

子懋,九歲嗣公。憲宗閱騎射西苑。懋三發連中,賜金帶。歷掌營府,累加至太師。嘗上言防邊事宜,諫止發京營兵作圓通寺。弘治中,御史李興、彭程下獄,懋論救。復請罷作真武觀,免織造,召還中官董織者。武宗即位,與群小狎遊,懋率文武大臣諫,其言皆切直。然性豪侈,又頗朘削軍士,屢為言者所糾。嗣公凡六十六年,握兵柄者四十年,尊寵為勛臣冠。正德十年卒,年亦七十五。贈寧陽王,謚恭靖。萬歷十一年與朱希忠並削王號。孫崙嗣。傳爵至世澤,流寇陷京師,遇害。

初,輔之定交阯也,先後百餘戰。其從徵死事最著者,有高士文、徐政。

士文,咸陽人。洪武中,以小校從征雲南及金山有功,為燕山左護衛百戶。質直剛果,善騎謝。從成祖起兵,累官都督僉事。從張輔征交阯。黎季BX既擒,餘黨竄山谷中,出沒為寇。五年八月,士文帥所部敗之廣源,進圍其寨。晝夜急攻,垂破,賊突走。士文追與戰,中飛石死。所部復追賊,賊失巢潰散,遂為指揮程瑒所滅。朝廷念士文功,追封建平伯,令其子福嗣,祿千三百石,予世券。三傳至孫牛,無子,以義子為嗣。事覺,爵除。

徐政,儀真人。建文時,為揚州衛副千戶,以城降成祖,累遷都指揮同知。從征交阯,奪船於三帶江以濟大軍。拔西都,戰鹹子關,皆有功。陳季擴反,盤灘地最要衝,張輔遣政守之。七年八月,賊黨阮景異來攻,與戰,飛鎗貫脅,猶督兵力戰,竟敗賊。賊退,腹潰而死。

黃福,字如錫,昌邑人。洪武中,由太學生歷金吾前衛經歷。上書論國家大計。太祖奇之,超拜工部右侍郎。建文時,深見倚任。成祖列姦黨二十九人,福與焉。成祖入京師,福迎附。李景隆指福姦黨,福曰:「臣固應死,但目為姦黨,則臣心未服。」帝置不問,復其官。未幾,拜工部尚書。永樂三年,陳瑛劾福不恤工匠,改北京行部尚書。明年坐事,逮下詔獄,謫充為事官。已,復職,督安南軍餉。

安南既平,郡縣其地,命福以尚書掌布政、按察二司事。時遠方初定,軍旅未息,庶務繁劇。福隨事制宜,咸有條理。上疏言:「交阯賦稅輕重不一,請酌定,務從輕省。」又請:「循瀘江北岸至欽州,設衛所,置驛站,以便往來。開中積鹽,使商賈輸粟,以廣軍儲。官吏俸廩,倉粟不足則給以公田。」又言:「廣西民餽運,陸路艱險,宜令廣東海運二十萬石以給。」皆報可。於是編氓籍,定賦稅,興學校,置官師。數召父老宣諭德意,戒屬吏毋苛擾。一切鎮之以靜,上下帖然。時群臣以細故謫交阯者眾,福咸加拯恤,甄其賢者與共事,由是至者如歸。鎮守中官馬騏怙寵虐民,福數裁抑之。騏誣福有異志。帝察其妄,不問。仁宗即位,召還,命兼詹事,輔太子。福在交阯凡十九年。及還,交人扶攜走送,號泣不忍別。福還,交阯賊遂劇,訖不能靖。仁宗崩,督獻陵工。

宣德元年,馬騏激交阯復叛。時陳洽以兵部尚書代福,累奏乞福還撫交阯。會福奉使南京,召赴闕,敕曰:「卿惠愛交人久,交人思卿,其為朕再行。」仍以工部尚書兼詹事,領二司事。比至,柳升敗死,福走還。至雞陵關,為賊所執,欲自殺。賊羅拜下泣曰:「公,交民父母也,公不去,我曹不至此。」力持之。黎利聞之曰:「中國遣官吏治交阯,使人人如黃尚書,我豈得反哉!」遣人馳往守護,饋白金、餱糧,肩輿送出境。至龍州,盡取所遺歸之官。還,為行在工部尚書。

四年與平江伯董漕事,議令江西、湖廣、浙江及江南北諸郡民,量地遠近,轉粟於淮、徐、臨清,而令軍士接運至北京,民大稱便。五年陳足兵食省役之要。其言足食,謂:「永樂間雖營建北京,南討交阯,北征沙漠,資用未嘗乏。比國無大費,而歲用僅給。即不幸有水旱,徵調將何以濟?請役操備營繕軍士十萬人,於濟寧以北,衛輝、真定以東,緣河屯種。初年自食,次年人收五石,三年收倍之。既省京倉口糧六十萬石,又省本衛月糧百二十萬石,歲可得二百八十萬石。」帝善之,下行在戶、兵二部議。郭資、張本言:「緣河屯田實便,請先以五萬頃為率,發附近居民五萬人墾之。但山東近年旱饑,流徙初復,衛卒多力役,宜先遣官行視田以俟開墾。」帝從之。命吏部郎中趙新等經理屯田,福總其事。既而有言軍民各有常業,若復分田,役益勞擾,事竟不行。改戶部尚書。

七年,帝於宮中覽福《漕事便宜疏》,出以示楊士奇曰:「福言智慮深遠,六卿中誰倫比者?」對曰:「福受知太祖,正直明果,一志國家。永樂初,建北京行部,綏輯凋瘵,及使交阯,總籓憲,具有成績,誠六卿所不及。福年七十矣,諸後進少年高坐公堂理政事,福四朝舊人,乃朝暮奔走勞悴,殊非國家優老敬賢之道。」帝曰:「非汝不聞此言。」士奇又曰:「南京根本重地,先帝以儲宮監國。福老成忠直,緩急可倚。」帝曰:「然。」明日改福官南京。明年兼掌南京兵部。英宗即位,加少保,參贊南京守備襄城伯李隆機務。留都文臣參機務,自福始。隆用福言,政肅民安。正統五年正月卒,年七十八。

福豐儀修整,不妄言笑。歷事六朝,多所建白。公正廉恕,素孚於人。當官不為赫赫名,事微細無不謹。憂國忘家,老而彌篤。自奉甚約,妻子僅給衣食,所得俸祿,惟待賓客周匱乏而已。初,成祖手疏大臣十人,命解縉評之,惟於福曰:「秉心易直,確乎有守。」無少貶。福參贊南京時,嘗坐李隆側。士奇寄聲曰:「豈有孤卿而旁坐者?」福曰:「焉有少保而贊守備者邪?」卒不變。然隆待福甚恭。公退,即推福上坐,福亦不辭。士奇之省墓也,道南京,聞福疾,往候之。福驚曰:「公輔幼主,一日不可去左右,奈何遠出?」士奇深服其言。兵部侍郎徐琦使安南回,福與相見石城門外。或指福問安南來者曰:「汝識此大人否?」對曰:「南交草木,亦知公名,安得不識?」福卒,贈謚不及,士論頗不平。成化初,始贈太保,謚忠宣。

劉俊,字子士,江陵人。洪武十八年進士。除兵部主事,歷郎中。遇事善剖決,為帝所器。二十八年擢右侍郎。建文時,為侍中。成祖即位,進尚書。

永樂四年大征安南,以俊參贊軍務。俊為人縝密勤敏,在軍佐畫籌策有功,還受厚賚。未幾,簡定復叛,俊再出參贊沐晟軍務。六年冬,晟與簡定戰生厥江,敗績。俊行至大安海口,颶風作,揚沙晝晦,且戰且行,為賊所圍,自經死。洪熙元年三月,帝以俊陷賊不屈,有司不言,未加褒恤,敕責禮官。乃賜祭,贈太子少傅,謚節愍。官其子奎給事中。

與俊同死者呂毅、劉昱。

毅,項城人。以濟南衛百戶從成祖渡江,積功至都督僉事。與同官黃中充左右副將軍,佐征南將軍韓觀鎮廣西。尋與中將兵送故安南王孫陳天平歸國,至芹站,天平被劫去,坐奪官。帝薄毅罪,起為鷹揚將軍,從張輔討季BX有功,掌交阯都司事。至是與賊戰,深入陷陣死。

昱,武城人。自吏科給事中遷左通政,出為河南參政,改交阯。嚴肅有治材,吏民畏憚。軍敗,亦死之。

陳洽,字叔遠,武進人。好古力學,與兄濟、弟浚並有名。洪武中,以善書薦授兵科給事中。嘗奉命閱軍,一過輒識之。有再至者,輒叱去。帝嘉其能,賜金織衣。父戍五開歿,洽奔喪。會蠻叛道梗,冒險間行,負父骨以歸。建文中以茹瑺薦,起文選郎中。

成祖即位,擢吏部右侍郎,改大理卿。安南兵起,命洽赴廣西,與韓觀選士卒從征。及大軍出,遂命贊軍務,主饋餉。安南平,轉吏部左侍郎。是時黃福掌布、按二司事,專務寬大,拊循其民。洽甄拔才能,振以風紀。核將士功罪,建置土官,經理兵食,剖決如流。還朝,命兼署禮部、工部事。七年復參張輔軍討簡定,平之。還,從帝北征,與輔練兵塞外。九年復與輔往交阯,討陳季擴。居五年,進兵部尚書,復留贊李彬軍事。

仁宗召黃福還,以洽掌布、按二司,仍參軍務。中官馬騏貪暴,洽不能制,反者四起,黎利尤桀黠。而榮昌伯陳智、都督方政不相能,寇勢日張。洽上疏言:「賊雖乞降,內懷詭詐,黨羽漸盛,將不可制。乞諭諸將速滅賊,毋為所餌。」宣宗降敕切責智等,令進兵,復敗於茶籠州,帝乃削智、政官爵。命成山侯王通佩征夷將軍印往討,洽仍贊其軍。宣德元年九月,通至交阯。十一月進師應平,次寧橋。洽與諸將言地險惡,恐有伏,宜駐師覘賊。通不聽,麾兵徑渡,陷泥淖中。伏發,官軍大敗。洽躍馬入賊陣,創甚墜馬。左右欲扶還,洽張目叱曰:「吾為國大臣,食祿四十年,報國在今日,義不茍生。」揮刀殺賊數人,自剄死。事聞,帝歎曰:「大臣以身殉國,一代幾人!」贈少保,謚節愍。官其子樞刑科給事中。

自黎利反,用兵三四年,將吏先後死者甚眾。

侯保,贊皇人。由國子生歷知襄城、贛榆、博興三縣,有善政。交阯初設府縣,擇人撫綏,以保知交州府,遷右參政。永樂十八年,黎利反,保以黃江要害,築堡守之。賊至,力拒數月,出戰,不勝死。

馮貴,武陵人。舉進士,為兵科給事中。從張輔征交阯,督兵餉。累遷左參政。涖事明敏,善撫流亡。士兵二千人,驍果善戰,貴撫以恩意,數擊賊有功,中官馬騏盡奪之。黎利反,貴以羸卒數百,禦賊於瑰縣,力屈而死。仁宗時,尚書黃福言狀,贈貴左布政使,保右布政使。然貴嘗言交阯產金,遂命以參議提督金場,時論非之。

伍雲,定遠人。以荊州護衛指揮同知從征交阯,破坡壘、隘留、多邦城,拔東、西二都,皆有功。賊平,調昌江衛。仁宗初,隨方政討黎利於茶籠,深入陷陣死。

陳忠,臨淮人。初為寬河副千戶。以「靖難」功,積官指揮同知。坐事戍廣西。從征交阯,自箇招市舁小舟入江,劫黎季BX水寨,破之。攻多邦城,先登。論功,還故官,調交州左衛。屢與賊戰有功,進都指揮同知。黎利寇清化,忠戰死。仁宗憫之,與雲皆優恤如制。

李任,永康人。以燕山衛指揮僉事從成祖起兵,累功為都指揮同知。宣德元年從征交阯,守昌江。黎利以昌江為官軍往來要路,悉力攻之。時都督蔡福為賊所獲,逼令招任降。任於城上罵福曰:「汝為大將,不能殺賊,反為賊用,狗彘不食汝餘。」發炮擊之。賊擁福去,大集兵象飛車衝梯,薄城環攻。任與指揮顧福帥精騎出城掩擊,燒其攻具。賊又築土山,臨射城中,鑿地道潛入城。任、福隨方禦之。死守九月餘,前後三十戰。賊聞征夷將軍柳升兵將至,益兵來攻。二年四月城陷,任、福猶帥死士三戰,三敗賊。賊驅象大至,不能支,皆自剄死。內官馮智、指揮劉順俱自經。城中軍民婦女不屈死者數千人。

劉子輔,廬陵人。由國子生擢監察御史,巡按浙江。性廉平,浙人德之。按察使周新不茍許與,獨稱子輔賢。遷廣東按察使。坐累,左遷諒江知府,善撫循其民。黎利反,子輔與守將集兵民死守亦九閱月,與昌江先後同陷。子輔曰:「吾義不污賊刃。」即自縊死。一子一妾皆死。

何忠,字廷臣,江陵人。由進士為監察御史。廉慎,人莫敢干以私。永樂中,三殿災,言事忤旨,出為政平知州。民安其政。寧橋之敗,王通詭與賊和,而請濟師於朝,為賊所遮不得達。賊遣使奉表入謝。通乃遣忠及副千戶桂勝與偕行,以奏還土地為辭,陰令請兵。至昌江,內官徐訓泄其謀。賊遂拘忠、勝,臨以白刃。二人瞋目怒罵不屈,並忠子皆被害。

徐麒,桂林中衛指揮使,與南寧千戶蔡顒守邱溫。時賊勢已熾,將吏多棄城遁。邱溫被圍,麒與顒猶帥疲卒固守,城陷皆死,無一降者。

易先,湘陰人。以國子生授諒山知府,有善政。歲滿還朝,郡人乞留。詔進秩三品還任。賊破諒山,先自縊死。

周安以指揮僉事守備乂安。黎利勢張,都督蔡福以芻糧將盡,退就東關。既行,千戶包宣以其眾詣賊降。安等至富良江為賊所蹙,俱陷賊。賊逼蔡福詣諸城說降。安憤甚,潛與眾謀,俟官軍至為內應。包宣之,以告利。利收安,將殺之,安曰:「吾天朝臣子,豈死賊手!」與指揮陳麟躍起奪賊刀,殺數人,皆自刎死。所部九千餘人,悉被殺。

交阯布政使弋謙以任等十二人死事聞。宣宗歎息,贈任都督同知,福、順、麒都指揮同知,安指揮同知,顒指揮僉事,勝正千戶,並令子孫承襲。子輔、先布政司參政,忠府同知,智太監,並予誥賜祭。惟麟嘗與朱廣開門納賊,故贈恤不及。已而黎利稱臣,歸蔡福、朱廣等六人,盡棄市,籍其家。

李彬,字質文,鳳陽人。父信,從太祖渡江,積功為濟川衛指揮僉事。彬嗣職。從潁國公傅友德出塞,斬獲多。還,與築諸邊城。成祖起兵,彬歸附,為前鋒,轉戰有功。累遷右軍都督僉事。永樂元年四月,以邱福議,封豐城侯,祿千石,予世券。明年,襄城伯李濬討永新叛寇,命彬帥師策應。未至,寇平,命以所統鎮廣東。四年召還,捕南陽皁君山寇。其年七月,以左參將齎征夷副將軍印授沐晟,進討安南。十二月,彬及雲陽伯陳旭破安南西都,又大敗賊於木丸江。安南平,論功,與旭皆以臨敵稽緩,不益封,加祿五百石。尋充總兵官,備倭海上。移兵討擒長沙賊李法良,又帥浙、閩兵捕海寇。

十年命往甘肅與西寧侯宋琥經略降酋。彬與柳升嚴兵境上,而令土官李英防野馬川。涼州酋老的罕叛,都指揮何銘戰死,英追躡,盡俘其眾。老的罕走赤斤蒙古。帝欲發兵,彬言道遠餉難繼,宜緩圖之。明年代琥鎮甘肅,赤斤蒙古縛老的罕以獻。帝嘉彬功,賜賚甚厚。十二年從北征,領右哨,破敵於忽失溫,追奔至土剌河。師還,受上賞,移鎮陜西。

十五年二月命佩征夷將軍印,鎮交阯。至則破擒陸那縣賊阮貞,遣都督朱廣等平順州及北晝諸寨。明年,清化府土巡檢黎利反,彬遣廣討破之。利遁去。十七年遣都督同知方政襲利於可藍柵,獲其將軍阮個立等。利走老撾。師還,復出為寇。都指揮黃誠擊走之,以暑雨旋師。

當是時,交人反者四起,彬遣諸將分道往討:方政討車綿子等於嘉興,鄭公證於南策,丁宗老於大灣;朱廣討譚興邦等於別部;都指揮徐謜討范軟於俄樂;指揮陳原瑰討陳直誠於惡江;都指揮王忠討楊恭於峽山。皆先後報捷。而賊勢尤劇者,彬輒自將往擊。潘僚者,乂安土知府也。為中官馬騏所虐,反衙儀。彬擊敗之,追至玉麻州,擒其酋,進焚賊柵。僚竄老撾,彬遣都指揮師祐帥師往。僚以老撾兵迎戰,破之農巴林,悉降其眾。范玉者,塗山寺僧也,反東潮州。彬往討,敗之江中。玉脫走,追獲之東潮。而鄭公證之黨黎姪復起,都指揮陳忠等累敗之於小黃江。彬自將追捕,至鎮蠻,盡縛其眾。於是諸賊略平,惟黎利數出沒,聚眾磊江,屢為徐謜、方政所敗,復遁去。

十九年,彬以餽運不繼,請令官軍與土軍參錯屯田,並酌屯守征行多寡之數以聞。帝從之。將發兵入老撾索黎利。老撾懼,請自捕以獻,會彬疾作而罷。明年正月卒。繼之者孟瑛、陳智、李安、方政,皆不能討。王通代鎮,賊勢益盛,交阯遂不可守。

彬卒,贈茂國公,謚剛毅。

子賢嗣,宣德三年從出塞,還修永寧、隆慶諸城。正統初,鎮大同,尋守備南京。景泰二年卒,贈豐國公,謚忠憲。

子勇嗣,再傳至孫旻。正德中鎮貴州,擒思南、石阡流賊,平武定諸蠻有功,加太子太傅。嘉靖初,鎮湖廣,有威惠,楚人安之。徙兩廣。武定侯郭勛典京營,以罪罷。世宗以旻遠鎮無內黨,召代之,尋坐事罷。卒謚武襄,無子。

從子熙嗣,出鎮湖廣。楚世子獄,株連甚眾,熙言於御史,平反二百餘人。討平沅州、麻陽叛蠻。卒,無子。從子儒嗣,傳至孫承祚,天啟時附魏忠賢,請設海外督理內臣,又請予忠賢九錫。崇禎初,奪爵戍邊。子開先嗣為伯,都城陷。遇害。

柳升,懷寧人。襲父職為燕山護衛百戶。大小二十餘戰,累遷左軍都督僉事。永樂初,從張輔征交阯,破賊魯江,斬其帥阮子仁等。守鹹子關。賊入富良江,舟亙十餘里,截江立寨,陸兵亦數萬人。輔將步騎,升將水軍,夾攻,大敗之,獲偽尚書阮希周等。又敗賊於奇羅海口,得舟三百。部卒得季BX及其子澄。升齎露布獻俘,被賞賚。師還,封安遠伯,祿千石,予世券。

七年同陳瑄帥舟師巡海,至青州海中,大破倭,追至金州白山島而還。明年從北征,至回曲津,將神機火器為前鋒,大敗阿魯台。進封侯,加祿五百石,仍世伯爵。出鎮寧夏,討斬叛將馮答蘭帖木兒等。召還,總京營兵。十二年復從北征,將大營兵戰忽蘭、忽失溫,以火器破敵。

十八年,蒲臺妖婦唐賽兒反。命升與都指揮劉忠將京軍往剿,圍其寨。升自以大將,意輕賊。賊乞降,信之。夜為所襲,忠中流矢死,賽兒遁去。及明始覺,追獲其黨百餘人。都指揮衛青力戰解安邱圍。升忌其功,摧辱之。徵下獄,已,得釋。

二十年復從北征,將中軍破兀良哈於屈裂兒河,予世侯。帝五出塞,升皆從,數有功,寵待在列侯右。仁宗即位,命掌右府,加太子太傅。

宣德元年冬,成山侯王通征黎利,敗聞。命升為征虜副將軍,充總兵官,保定伯梁銘為左副總兵,都督崔聚為參將,尚書李慶贊軍務,帥步騎七萬,會黔國公沐晟往討。時賊勢已盛,道路梗絕,朝廷久不得交阯奏報。二年六月,有軍丁李茂先者三人,間道走京師,言昌江被圍急。帝授三人百戶。敕升急進援,而昌江已於四月陷。九月,升始入隘留關。利偽為國人上書,請立陳氏後,升不啟封以聞。賊緣途據險列柵,官軍連破之,抵鎮夷關。升以賊屢敗,易之。時李慶、梁銘皆病甚。郎中史安、主事陳鏞言於慶曰:「柳將軍辭色皆驕。驕者,兵家所忌。賊或示弱以誘我,未可知也。防賊設伏,璽書告誡甚切,公宜力言之。」慶強起告升,升不為意。至倒馬坡,與百餘騎先馳度橋,橋遽壞,後隊不得進。賊伏四起,升陷泥淖中,中鏢死。其日,銘病卒。明日,慶亦卒。又明日,崔聚帥軍至昌江。賊來益眾,官軍殊死鬥。賊驅象大戰。陣亂,賊大呼:「降者不死。」官軍或死或走,無降者,全軍盡沒。史安、陳鏞及李宗昉、潘禋皆死之。

崔聚,懷遠人。從成祖起兵。八年從北征,敗敵於廣漠戍。洪熙元年累遷左軍都督僉事。至是力戰被執,賊百計降之,終不屈死。

升質直寬和,善撫士卒,勇而寡謀,遂及於敗。升敗,沐晟師不得進,亦引還。王通孤軍援絕,遂棄交阯。朝議以升喪師,不令子溥襲爵,久之乃許。正統十二年贈升融國公,謚襄愍。

溥,初掌中府,出鎮廣西。廉慎,然無將略,承山雲後,不能守成法,過於寬弛。瑤、僮相煽為亂,溥先後討斬大藤峽賊渠,破柳州、思恩諸蠻寨,而賊滋蔓如故。景泰初,兵事亟,召掌右府,總神機營。事定,復出鎮。天順初召還,防宣府、大同,累進太傅。陜西有警,命佩平虜大將軍印往禦。敵再入涼州,溥閉壁不出,敵飽掠去。躡取數十級報捷,被劾,落太傅閑住。尋復起掌神機營。卒,謚武肅。

孫景嗣,景子文,文子珣,凡三世皆鎮兩廣,有平蠻功。嘉靖十九年命珣佩征夷副將軍印,徵安南莫登庸。登庸乞降,加太子太傅。又以討瓊州黎賊功,加少保。卒贈太保,謚武襄。傳至明亡,爵絕。史安,字志靜,豐城人。廉重好學,由進士歷官儀制司郎中。

陳鏞,字叔振,錢塘人。由庶吉士授祠祭司主事。楊士奇稱其清介端確,表裏一出於正。

李宗昉,不知何許人,亦以主事從。

潘禋,鄞人。以後軍都事從。嘗勸升持重,廣偵探,引芹站、寧橋事為戒,升不聽。軍敗,格鬥死。

梁銘,汝陽人。以燕山前衛百戶從仁宗守北平。李景隆圍城,戰甚力。積功至後軍都督僉事,侍仁宗監國。永樂八年坐事下獄。十九年赦復職,副都督胡原捕倭廣東。仁宗即位,進都督同知。以參將佩征西將軍印,同都督同知陳懷鎮寧夏。追論守城功,封保定伯,祿千石,予世券。宣德初,御史石璞劾其貪黷,下獄,當奪爵,宥之。副柳升征交阯。升敗,銘病卒。銘勇敢善戰,能得士卒心。既死,崔聚獨以眾入,全軍遂覆。

子珤嗣。正統末,充副總兵,討福建盜鄧茂七,擊斬餘賊於九龍山。班師,而賊黨復作,謫充為事官。從石亨立功,復爵。景泰元年拜平蠻將軍,代王驥討貴州苗。其冬,分四道進攻,大敗之,斬首七千有奇,破寨五百。明年自沅州進兵,與都督方瑛破賊於興澤,又大破之香爐山,俘偽王韋同烈等,擒斬數千人。分兵攻都勻、草塘諸苗,悉震恐降。師還,苗復叛,珤復與瑛討平之。論功,進侯,益祿五百石。四年討平湖廣清浪叛苗。天順元年出鎮陜西,破敵涼州,又破敵靖虜堡。召還,理左府事。成化初卒。贈蠡國公,謚襄靖。

珤天資平恕,數總兵柄,未嘗妄殺一人。子弟從征,以功授官,輒辭不受,人以為賢。傳爵至世勳,崇禎初提督京營。京師陷,遇害。

王通,咸寧人,金鄉侯真子也。嗣父官為都指揮使,將父兵,轉戰有功,累進都督僉事。復以父死事故,封武義伯,祿千石,予世券。永樂七年董營長陵。十一年進封成山侯,加祿二百石。明年從北征,領左掖。二十年從出塞,以大軍殿,連出塞,並領右掖。仁宗即位,命掌後府,加太子太保。

時交阯總兵官豐城侯李彬已前卒,榮昌伯陳智、都督方政以參將代鎮,不協。黎利益張,數破郡邑,殺將吏。智出兵數敗,宣宗削智爵,而命通佩征夷將軍印,帥師往討。黎利弟善攻交州城,都督陳濬等擊卻之。會通至,分道出擊。參將馬瑛破賊於石室縣。通引軍與瑛合,至應平之寧橋中伏,軍大潰,死者二三萬人,尚書陳洽與焉。通中傷還交州,利在乂安聞之,自將精卒圍東關。通氣沮,陰遣人許為利乞封,而檄清化迤南地歸利。按察使楊時習執不可,通厲聲叱之。清化守羅通亦不肯棄城,與指揮打忠堅守。朝廷遣柳升等助通,未至。

二年二月,利攻城。通以勁兵五千出不意搗賊營,破之,斬其司空丁禮以下萬餘級。利惶懼欲走,諸將請乘勝急擊。通猶豫三日不出,賊勢復振。樹柵掘濠塹,四出攻掠,分兵陷昌江、諒江,而圍交益急。通斂兵不出。利乞和,通以聞。會柳升戰歿,沐晟師至水尾縣不得進。通益懼,更啗利和,為利馳上謝罪表。

其年十月,大集官吏軍民出城,立壇與利盟,約退師。因宴利,遺利錦綺,利亦以重寶為謝。十二月,通令太監山壽與陳智等由水路還欽州,而自帥步騎還廣西。至南寧,始以聞。會廷議厭兵,遂棄交阯。交阯內屬者二十餘年,前後用兵數十萬,饋餉至百餘萬,轉輸之費不與焉,至是棄去。官吏軍民還者八萬六千餘人,其陷於賊及為賊所戮者不可勝計。而土官嚮義者陶季容、陳汀之屬,乃往往自拔來歸。

明年,通還京,群臣交劾,論死繫獄,奪券,籍其家。正統四年特釋為民。景帝立,起都督僉事,守京城。禦也先有功,進同知,守天壽山,還其家產。景泰三年卒。天順元年詔通子琮嗣成山伯。琮子鏞,成化時賜原券。傳爵至明亡。

陶季容者,世為水尾土官。交阯平,以為土知縣。歷歸化知州,遷宣化府同知,守北閑堡。宣德元年遣所部阮執先等追賊,至清波縣為所獲。既而遣執先還,招季容,脅以兵,不為動。宣宗聞之,擢宣化知府,降敕獎勞。賊復遣人誘季容,季容執以送沐晟,而導官軍敗賊於水尾。王通棄交阯,季容率官屬入朝。

陳汀,古雷縣千夫長,數從方政擊賊有功,政信倚之。王通棄地,汀北行,為賊所得,授以官,令守交州東關。汀挈其家九十餘人從間道走。賊追之,家屬盡陷,汀獨身入欽州。帝嘉其義,以為指揮,厚賚之。

他若土官阮世寧、阮公庭,皆不願從利,率所部來歸,乞居龍州、陳州之地。帝命加意撫恤,資糧器用官給之。

贊曰:成祖因季犛纂立,興師問罪以彰天討,求陳氏後不得,從而郡縣其地,得取亂侮亡之道矣。蠻疆險遠,易動難馴,數年之間叛者數起,柳升以輕敵喪師,王通以畏怯棄地。雖黃福惠愛在交,叛人心折,而大勢已去,再至無功。宣宗用老成謀國之言,廓然置之度外,良以其得不為益,失不為損,事勢所不必爭,非獨憚於勞民而絀於籌餉也。嘗考黃福與張輔書言:「惡本未盡除,守兵不足用。馭之有道,可以漸安。守之無法,不免再變。」權交事之始終,蓋惜張輔之不得為滇南之沐氏也。


\end{pinyinscope}