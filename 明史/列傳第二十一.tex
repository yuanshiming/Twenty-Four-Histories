\article{列傳第二十一}

\begin{pinyinscope}
廖永安俞通海(弟通源淵胡大海(養子德濟))欒鳳耿再成張德勝汪興祖趙德勝南昌康郎山兩廟忠臣附桑世傑(劉成茅成楊國興胡深孫興祖曹良臣周顯常榮張耀濮英于光等

廖永安,字彥敬。德慶侯永忠兄也。太祖初起,永安兄弟偕俞通海等以舟師自巢湖來歸。太祖親往收其軍,遂以舟師攻元中丞蠻子海牙於馬場河。元人駕樓船,不利進退,而永安輩操舟若飛,再戰,再破元兵,始定渡江策。頃之,發江口。永安舉帆,請所向,命直指牛渚。西北風方驟,頃刻達岸。太祖急揮甲士鼓勇以登,采石鎮兵皆潰,遂乘勝取太平。授管軍總管。以舟師破海牙水柵,擒陳兆先,入集慶。擢建康翼統軍元帥。以舟師從取鎮江,克常州,擢同僉江南行樞密院事。又以舟師同常遇春自銅陵趨池州。合攻,破其北門,執徐壽輝守將,遂克池州。偕俞通海拔江陰之石牌戍,降張士誠守將欒瑞。擢同知樞密院事。又以舟師破士誠兵於常熟之福山港。再破之通州之狼山,獲其戰艦以歸。遂從徐達復宜興,乘勝深入太湖。遇吳將呂珍,與戰。後軍不繼,舟膠淺,被執。

永安長水戰,所至輒有功。士誠愛其才勇,欲降之,不可,為所囚。太祖壯永安不屈,遙授行省平章政事,封楚國公。永安被囚凡八年,竟死於吳。吳平,喪還,太祖迎祭於郊。

洪武六年,帝念天下大定,諸功臣如永安及俞通海、張德勝、耿再成、胡大海、趙德勝、桑世傑皆已前沒,猶未有謚號,乃下禮部定議。議曰:「有元失馭,四海糜沸。英傑之士,或起義旅,或保一方,泯泯棼棼,莫知所屬。真人奮興,不期自至,龍行而雲,虎嘯而鳳。若楚國公臣永安等,皆熊羆之士、膂力之才,非陷堅沒陣,即罹變捐軀,義與忠俱,名耀天壤。陛下混一天下,追維舊勞,爵祿及子孫,烝嘗著祀典。易名定謚,於禮為宜。臣謹按謚法:以赴敵逢難,謚臣永安武閔;殺身克戎,謚臣通海忠烈;奉上致果,謚臣張德勝忠毅;勝敵致強,謚臣大海武莊;闢土斥境,武而不遂,謚臣再成武壯;折衝禦侮,壯而有力,謚臣趙德勝武桓。臣世傑,業封永義侯,與漢世祖封寇恂、景丹相類,當即以為謚。」詔曰:「可。」九年皆加贈開國輔運推誠宣力武臣、光祿大夫、柱國。已,又改封永安鄖國公。無子,授其從子升為指揮僉事。

俞通海,字碧泉。其先濠人也,父廷玉徙巢。子三人,通海、通源、淵。元末,盜起汝、潁。廷玉父子與趙普勝、廖永安等結寨巢湖,有水軍千艘。數為廬州左君弼所窘,遣通海間道歸太祖。太祖方駐師和陽,謀渡江,無舟楫。通海至,大喜曰:「天贊我也!」親往撫其軍。而趙普勝叛去。元兵以樓船扼馬場河等口。瀕湖惟一港可通,亦久涸。會天大雨,水深丈餘,乃引舟出江,至和陽。通海為人沉毅,治軍嚴而有恩,士樂為用。巢湖諸將皆長於水戰,而通海為最。從破海牙諸水寨,授萬戶。從渡江,克采石,取太平,徇下諸屬縣。海牙復以戰艦截採石,而陳兆先合淮兵二十萬屯方山,相犄角。通海與廖永安等擊之,大敗其眾,海牙遁。進破兆先,取集慶路。從湯和拔鎮江,遷秦淮翼元帥。偕諸將取丹陽、金壇、常州。遷行樞密院判官。從克寧國,下水陽,因以舟師略太湖,降張士誠守將於馬跡山,艤舟胥口。呂珍兵暴至,諸將欲退。通海曰:「不可,彼眾我寡,退則情見。不如擊之。」乃身先疾鬥,矢下如雨,中右目,不能戰。命帳下士被己甲督戰。敵以為通海也,不敢逼,徐解去。由是一目遂眇。已,偕永安等克石牌戍,奪馬馱沙而還。

普勝既叛歸友諒,陷池州,遣別將守,而自據樅陽水寨。太祖方征浙東,以樅陽為憂。通海往攻,大破之。普勝陸走,盡獲其舟,遂復池州。遷僉樞密院事。陳友諒犯龍灣,偕諸將擊走之,追焚其舟於慈湖,擒七帥,逐北至采石。功最,進樞密院同知。從攻友諒,下銅陵,克九江,掠蘄、黃。從徐達擊叛將祝宗、康泰,復南昌。從援安豐,敗士誠兵。還攻廬州。

友諒大舉圍南昌。從太祖擊之。遇於康郎山,舟小不能仰攻,力戰幾不支。通海乘風縱火焚其舟二十餘,敵少挫。太祖舟膠,友諒驍將張定邊直前,犯太祖舟。常遇春射中定邊,通海飛舸來援,舟驟進水湧,太祖舟得脫。而通海舟復為敵巨艦所壓,兵皆以頭抵艦,兜鍪盡裂,僅免。明日復戰,偕廖永忠等以七舟置火藥,焚敵舟數百。踰二日,復以六舟深入。敵連大艦力拒。太祖登舵樓望,久之無所見,意已沒。有頃,六舟繞敵艦出,飄颻若游龍。軍士讙噪,勇氣百倍,戰益力。友諒兵大敗。師次左蠡,通海進曰:「湖有淺,舟難回旋。莫若入江,據敵上流。彼舟入,即成擒矣。」遂移師出湖,水陸結柵。友諒不敢出。居湖中一月,食盡,引兵突走,竟敗死。是役也,通海功最多。師還,賜良田金帛。

明年從平武昌。拜中書省平章政事。總兵略劉家港,進逼通州,敗士誠兵,擒其將朱瓊、陳勝。進攝江淮行中書省事,鎮廬州。從徐達平安豐。又從克湖州,略太倉。秋毫不犯,民大悅。圍平江,戰滅渡橋。搗桃花塢,中流矢,創甚,歸金陵。太祖幸其第,問曰:「平章知予來問疾乎?」通海不能語。太祖揮涕而出。翼日卒,年三十八。太祖臨哭甚哀,從官衛士皆感涕。追封豫國公,侑享太廟,肖像功臣廟。洪武三年,改封虢國公,謚忠烈。

通海父廷玉官僉樞密院事,先卒,追封河間郡公。通海無子,弟通源嗣其官。

通源,字百川。從大將軍征中原,偕副將軍馮勝等會兵太原,定河中。渡河,克鹿臺,取鳳翔、鞏昌、涇州,守開城。會張良臣據慶陽再叛,大將軍命諸將分兵蹙之。通源自臨洮疾趨至涇,略其西,顧時略其北,傅友德略其東,陳德略其南。大將軍逼城下,良臣援絕糧盡,敗死。遂克慶陽。征定西,克興元,皆先登。洪武三年封南安侯,歲祿千五百石,予世券。四年從廖永忠伐蜀,又從徐達出塞,撫甘肅,有功。徙江南豪民十四萬田鳳陽。又攻雲南,徵廣南蠻,俘斬數萬。二十二年詔還鄉,賜鈔五萬,置第於巢。未行,卒。子祖,病不能嗣。逾年,追論胡黨,以通源死,不問,爵除。

淵以父兄故,充參侍舍人。從征,積功授都督僉事。通源既坐黨,太祖念廷玉、通海功,二十五年封淵越巂侯,歲祿二千五百石,予世券。帥師討建昌叛賊,城越巂。明年坐累失侯,遣還里。建文元年召復爵。隨大軍征燕,戰沒於白溝河。次子靖嗣官。

胡大海,字通甫,虹人。長身、鐵面,智力過人。太祖初起,大海走謁滁陽,命為前鋒。從渡江,與諸將略地,以功授右翼統軍元帥,宿衛帳下。從破寧國,副院判鄧愈戍之。遂拔徽州,略定其境內。元將楊完者,以十萬眾來攻,大海戰城下,大破走之。遂與鄧愈、李文忠自昱嶺關攻建德。敗元師於淳安,遂克建德。再敗楊完者,降溪洞兵三萬人。進樞密院判官。克蘭溪,從取婺州,遷僉樞密院事。下諸暨,守將宵遁。萬戶沈勝既降復叛,大海擊敗之,生擒四千餘人。改諸暨為諸全州。移兵攻紹興,再破張士誠兵。太祖以寧、越重地,召大海使守之。士誠將呂珍圍諸全,大海救之。珍堰水灌城,大海奪堰,反灌珍營。珍勢蹙,於馬上折矢誓。請各解兵,許之。郎中王愷曰:「珍猾賊,不可信,不如因擊之。」大海曰:「言出而背之,不信;既縱而擊之,不武。」師還,人皆服其威信。尋攻處州,走元將石抹宜孫,遂定處州七邑。

陳友諒寇龍江,命分軍搗信州,以牽制敵。大海用王愷言,親引兵往,遂克信州,以為廣信府。信方絕糧,或勸還師。大海曰:「此閩、楚襟喉地也,可棄之乎?」築城浚隍以守之。先是,軍糧少,所得郡縣,將士皆徵糧於民,名曰寨糧。民甚病之。大海以為言,始命罷去。進江南行省參知政事,鎮金華。

初,嚴州既下,苗將蔣英、劉震、李福皆自桐廬來歸。大海喜其驍勇,留置麾下。至是,三人者謀作亂,晨入分省署,請大海觀弩於八詠樓。大海出,英遣其黨跪馬前,詐訴英過。大海未及答,反顧英。英出袖中槌擊大海,中腦仆地。並其子關住、郎中王愷皆遇害。英等大掠城中,奔於吳。其後,李文忠攻杭州,杭人執英以降。太祖命誅英,刺其血以祭大海。

大海善用兵,每自誦曰:「吾武人,不知書,惟知三事而已:不殺人,不掠婦女,不焚毀廬舍。」以是軍行遠近爭附。及死,聞者無不流涕。又好士,所至輒訪求豪雋。劉基、宋濂、葉琛、章溢之見聘也,大海實薦之。追封越國公,謚武莊,肖像功臣廟,配享太廟。

初,太祖克婺州,禁釀酒。大海子首犯之。太祖怒,欲行法。時大海方征越,都事王愷請勿誅,以安大海心。太祖曰:「寧可使大海叛我,不可使我法不行。」竟手刃之。及關住復被殺,大海遂無後。

養子德濟,字世美,不知何許人。大海帥以歸太祖。從攻婺州,為誘兵,大破元兵於梅花門外,擒其將季彌章,由是知名。既下信州,太祖以德濟為行樞密院同僉,使守之。陳友諒將李明道來寇,德濟與力戰。大海來援,夾擊之,擒明道及其宣慰王漢二。及大海為蔣英所害,處州降將李祐之亦殺院判耿再成以叛。張士誠聞浙東亂,遣其弟士信寇諸全。德濟自信州往救,乘懈得入城,與知州欒鳳、院判謝再興分門守。夜半,出敵不意,砍士信營,破走之。擢浙江行省參知政事,移守新城。士誠將李伯昇帥步騎大入寇。德濟固守,乞師於李文忠。文忠馳救,德濟出兵夾擊,大破之,詳文忠傳。

時德濟所部有潛移家入新城者,文忠疑德濟使然。誅其都事羅彥敬,欲微戒德濟。將士皆怒,走告德濟。德濟怡然曰:「右丞殺彥敬,自為廣信作戰衣有弊耳,再言者斬!」於是太祖召德濟褒諭之,而責文忠失將士心。且曰:「胡德濟之量,汝不及也。」擢浙江行省右丞,賜駿馬。未幾,改左丞,移鎮杭州。從大將軍徐達出定西。德濟軍失利,達斬其部將數人,械至京師。帝念舊功,釋之。復以為都指揮使,鎮陜西,卒。

欒鳳,高郵人。知諸全,有能聲。方士信來攻,與謝再興力守,數出奇兵挫敵。再興使部校鬻貨於杭,太祖慮其輸我軍虛實,召再興還,而以參軍李夢庚總制諸全軍馬。既而念再興功,為兄子文正娶其長女,命徐達娶其幼女。復遣守諸全。再興忿夢庚出己上,鳳復以細故繩之,遂叛,殺鳳。鳳妻王氏以身蔽鳳,並殺之。執夢庚,降於士誠,夢庚亦死之。太祖以再興數有功,叛非其志,故鳳與夢庚皆不得恤云。

耿再成,字德甫,五河人。從太祖於濠,克泗、滁州。元兵圍六合,太祖救之,與再成軍瓦梁壘。力戰,度不敵,引還。元兵尾至,太祖設伏澗側,令再成誘敵,大敗之。以鎮撫從渡江,下集慶。以元帥守鎮江,以行樞密院判官守長興,再守揚州。從取金華,為前鋒,屯縉雲之黃龍山以遏敵衝。與胡大海破石抹宜孫於處州,克其城,守之。宜孫來攻,又敗之慶元。

再成持軍嚴,士卒出入民間,蔬果無所捐。金華苗帥蔣英等叛,殺胡大海。處州苗帥李祐之等聞之,亦作亂。再成方對客飯,聞變,上馬,收戰卒不滿二十人,迎賊罵曰:「賊奴!國家何負汝,乃反。」賊攢槊刺再成。再成揮劍連斷數槊,中傷墜馬,大罵不絕口死。胡深等收其屍,槁葬之。後改葬金陵聚寶山。追封高陽郡公,侑享太廟,肖像功臣廟。洪武十年加贈泗國公,謚武壯。

子天璧,聞父難,糾部曲殺賊。比至,李文忠已破賊斬之。遂以天璧守處州。拒方國珍、張士誠皆有功,擢指揮副使。克浦城,搗建寧,走陳友定。征襄陽,進至西安。招諭河州、臨洮,皆下。改杭州指揮同知。七年出海捕倭,深入外洋,溺死。

張德勝,字仁輔,合肥人。才略雄邁。與俞通海等以舟師自巢來歸。從渡江,克采石、太平。陳埜先來攻,與湯和等破擒之。授太平興國翼總管。破蠻子海牙水寨,擒陳兆先。下集慶,克鎮江,授秦淮翼元帥。取常州,擢樞密院判。克寧國,收長槍兵。下太湖,略馬蹟山。攻宜興,取馬馱沙及石牌寨。進僉樞密院事。趙普勝陷池州,德勝往援,弗及,還,從徐達拔宜興。普勝復掠青陽、石埭。德勝與戰柵江口,破走之。已,復同通海擊敗其眾,遂復池州。引兵自無為趨浮山,走普勝將胡總管,追,敗之青山,逐北至潛山。陳友諒將郭泰逆戰沙河,破斬之,遂克潛山。友諒犯龍江,德勝總舟師迎戰,殺傷相當。德勝大呼,麾諸將奮擊。友諒軍披靡,遂大敗。與諸將追及之慈湖,縱火焚其舟。至採石,大戰,沒於陣。追封蔡國公,謚忠毅,肖像功臣廟,侑享太廟。子宣幼。養子興祖嗣職。

興祖,巢人。本汪姓。既嗣職,從破安慶,克江州,拔蘄、黃,取南昌。從援安豐,大敗張士誠兵。鄱陽之戰,與廖永忠等以六舟深入。又邀擊友諒於涇江口。功最,擢湖廣行省參政。從平武昌,遂克廬州,略地至通州而還。進大都督府僉事。從徐達取淮東,下浙西。進同知大都督府事。大軍北征,別將衛軍由徐州克沂、青、東平,乘勝至東阿,降元參政陳璧及所部五萬餘人。孔子五十六世孫衍聖公希學帥曲阜知縣希舉、鄒縣主簿孟思諒等迎謁於軍門,興祖禮之。兗東州縣聞風皆下,遂取濟寧、濟南。

洪武元年,以都督兼右率府使,從攻樂安,克汴梁、河、洛,還守濟寧。與大將軍會師德州,帥舟師並河進,遂克元都。徇下永平,西取大同,將三衛卒守之。再敗元兵,斬獲無算。時德勝子宣已長,命為宣武衛指揮同知。而興祖復姓為汪。三年進克武、朔二州,獲元知院馬廣等。帥兵至大同北口,大敗元兵,獲擴廓弟金剛奴等四百餘人。未幾,命為晉王武傅,兼山西行都督府僉事。四年從前將軍傅友德合兵伐蜀,克階、文,乘勝至五里關,中飛石死。蜀平,詔都督興祖歿於王事,優賞其子,追封東勝侯,予世券。

興祖子幼,命與宣同居。以疾卒,爵除。

趙德勝,濠人。為元義兵長,善馬槊,每戰先登。隸王忙哥麾下,察其必敗。太祖取滁陽,德勝母在軍中,乃棄其妻來從。太祖喜,賜之名,為帳前先鋒。從取鐵佛岡,攻三汊河,破張家寨,克全椒、後河諸寨。援六合,中流矢,幾殆。擊雞籠山,搗烏江,下和州、含山。夜襲陳埜先營,拔板門、鐵長官二寨,遂取儀真。授總管府先鋒。從渡江,下太平,克蕪湖、句容、溧水、溧陽,皆有功。從常遇春敗蠻子海牙於采石,破陳兆先營於方山,下集慶,功最。從徐達取鎮江,破苗軍水寨。下丹陽、金壇,平寧國。轉領軍先鋒。取廣德,破張士誠水寨。復從遇春攻常州,解牛塘圍,復廣德、寧國。取江陰,攻常熟,擒張士德。從攻湖州。宜興叛,還兵定之。擢中翼左副元帥。陳友諒犯龍江。龍江第一關曰虎口城,太祖以屬德勝。友諒至,力戰。伏兵起,友諒大敗。遂復太平。下銅陵臨山寨,略黃山橋及馬馱沙,征高郵。有功,進後翼統軍元帥。

從太祖西征,破安慶水寨,乘風溯小孤山。距九江五里,友諒始知,倉皇遁去。遂克九江,徇黃梅、廣濟,克瑞昌、臨江、吉安,還下安慶。進克撫州,取新淦。討南昌叛將,復其城,炮傷肩。授僉江南行樞密院事。與朱文正、鄧愈共守南昌。平羅友賢於池州,破友諒將於西山。復臨江、吉安、撫州。未幾,友諒大舉兵圍南昌。德勝帥所部數千,背城逆戰,射殺其將,敵大沮。明日復合,環城數匝。友諒親督戰,晝夜攻,城且壞。德勝帥諸將死戰,且戰且築,城壞復完。暮坐城門樓,指揮士卒。弩中腰膂,鏃入六寸,拔出之,歎曰:「吾自壯歲從軍,傷矢石屢矣,無重此者。丈夫死不恨,恨不能掃清中原耳。」言畢而絕,年三十九。追封梁國公,謚武桓,列祀功臣廟,配享太廟。

德勝剛直沉鷙,馭下嚴肅。未嘗讀書,臨機應變,動合古法。平居篤孝友如修士。

友諒圍南昌八十五日,先後戰死者凡十四人。

張子明者,領兵千戶也。洪都圍久,內外隔絕,朱文正遣子明告急於應天。以東湖小漁舟從水關潛出,夜行晝止,半月始得達。太祖問友諒兵勢。對曰:「兵雖盛,戰鬥死者不少。今江水日涸,賊巨艦將不利。援至可破也。」太祖謂子明:「歸語而帥:堅守一月,吾自取之。」還至湖口,為友諒所獲。令誘城中降,子明佯諾。至城下,大呼:「我張大舍。已見主上,令諸公堅守,救且至!」賊怒,攢槊殺之。追封忠節侯。

友諒攻撫州,樞密院判李繼先乘城戰死;左翼元帥牛海龍突圍死;左副元帥趙國旺引兵燒戰艦,敵追至,投橋下死;百戶徐明躍馬出射賊,賊知明名,併力攻,被執死;軍士張德山夜半潛出城,焚賊舟,賊覺,死;夏茂成守城樓,中飛炮死;右翼元帥同知朱潛、統軍元帥許珪俱戰死。蔣必勝陷吉安,參政劉齊、知府朱文華被執,不屈死。趙天麟守臨江,友諒攻之,城陷,不屈死。祝宗、康泰叛,陷洪都,知府葉琛與行省都事萬思誠迎戰,皆死。事平,皆贈爵侯伯以下有差,立忠臣廟於豫章,並祠十四人,以德勝為首。而康郎山戰死者三十五人,首丁普郎。

普郎初為陳友諒將,守小孤山。偕傅友德來降,授行樞密院同知,數有功。及援南昌,大戰鄱陽湖。自辰至午,普郎身被十餘創,首脫猶直立,執兵作斗狀,敵驚為神。時七月己丑也。追贈濟陽郡公。

張志雄亦友諒將,素驍勇,號長張。從趙普勝守安慶。友諒殺普勝,志雄怨,來降,為樞密院判。至是舟檣折,敵攢刺之,知不能脫,遂自刎。

元帥餘昶、右元帥陳弼、徐公輔皆以其日戰沒。

先一日,左副指揮韓成,元帥宋貴、陳兆先戰沒。兆先者,埜先從子,既被擒,太祖以其兵備宿衛。感帝大度,效死力,至是戰死。韓成子觀至都督,別有傳。

越四日,辛卯,復大戰,副元帥昌文貴、左元帥李信、王勝、劉義死。

八月壬戌,扼敵涇江口,同知元帥李志高、副使王咬住亦戰死。

其他偏裨死事者,千戶姜潤、王鳳顯、石明、王德、朱鼎、王清、常德勝、袁華、陳沖、王喜仙、汪澤、丁宇、史德勝、裴軫、王理、王仁,鎮撫常惟德、鄭興、逯德山、羅世榮、曹信。凡贈公一人、侯十二人、伯二人、子十五人、男六人,肖像康郎山忠臣廟,有司歲致祭。

又程國勝者,徽人。以義兵元帥來歸,敗楊完者。累功至萬戶。守南昌。與牛海龍夜劫友諒營。海龍中流矢死,國勝泅水得脫,抵金陵。從太祖戰鄱陽。張定邊直前犯太祖舟,國勝與韓成、陳兆先駕舸左右奮擊,太祖舟脫。國勝等繞出敵艦後,援絕,力戰死。而南昌城中謂國勝已前死,故豫章、康山兩廟俱得預祀云。

桑世傑,無為人。亦自巢湖來歸。趙普勝有異志,世傑發其謀,普勝逸去。從渡江,以舟師破元水軍。授秦淮翼元帥。下鎮江,徇金壇、丹陽,攻寧國長槍諸軍,克水陽,平常州。判行樞密院事。略地江陰、宜興。

初,石牌民朱定,販鹽無賴,與富民趙氏有隙,遂告變,滅趙氏,授江陰判官。尋復為盜,元遣兵捕之。定聞張士誠據高郵,乃導士誠由通州渡江,遂陷平江。以定為參政,而遣元帥欒瑞戍石牌。及大兵既取江陰,瑞尚據石牌,導舟師往來。太祖命永安及世傑擊之,世傑力戰死,瑞亦降。張氏窺江路絕。太祖念其功,贈安遠大將軍、輕車都尉、永義侯,侑享太廟。

子敬以父死事,累官都督府僉事。洪武二十三年,封徽先伯,歲祿千七百石,予世券。明年同徐輝祖等防邊,尋令屯軍平陽,坐藍玉黨死。

又劉成者,靈璧人。以統兵總管從耿炳文定長興,為永興翼左副元帥,數佐炳文敗士誠兵。李伯昇以十萬眾來攻,城中兵僅七千。太祖遣兵援之,未至,炳文嬰城守。成引數士騎出西門,擊敗伯昇兵,擒其將宋元帥。轉至東門,敵悉兵圍之,遂戰死。贈懷遠將軍,立廟長興。

茅成,定遠人。自和州從軍,隸常遇春麾下。克太平,始授萬戶。從定常州、寧國,進總管。克衢州,授副元帥。守金華,改太平興國翼元帥。從克安慶,援安豐,戰鄱陽,克武昌,授武德衛千戶。尋進指揮副使。取贛州、安陸、襄陽、泰州,皆有功。從徐達攻平江,焚張士誠戰船,築長圍困之。達攻婁門,士誠出兵戰,成擊敗之。突至外郛,中叉死。贈東海郡公,祀功臣廟。

同時死者,有楊國興,亦定遠人。以右翼元帥守宜興。初,常州人陳保二聚眾,號「黃包軍」。即降復叛,誘執詹、李二將。國興執斬之。授神武衛指揮使。至是攻閶門戰死,以其子益襲指揮使。

胡深,字仲淵,處州龍泉人。穎異有智略,通經史百家之學。元末兵亂,嘆曰:「浙東地氣盡白,禍將及矣。」乃集里中子弟自保。石抹宜孫以萬戶鎮處州,辟參軍事,募兵數千,收捕諸山寇。溫州韓虎等殺主將叛。深往諭之,軍民感泣,殺虎以城降。已,偕章溢討龍泉之亂,搜旁縣盜,以次平之。宜孫時已進行省參政,承制命深為元帥。戊戌十二月,太祖親征婺州。深帥兵車數百輛往援,至松溪不能救,敗去,婺遂下。明年,耿再成侵處州,宜孫分遣元帥葉琛、參謀林彬祖、鎮撫陳中真及深帥兵拒戰。會胡大海兵至,與再成合,大破之,進抵城下。宜孫戰敗,與葉琛、章溢走建寧,處州遂下。深以龍泉、慶元、松陽、遂昌四縣降。

太祖素知深名,召見,授左司員外郎,遣還處州。招集部曲,從征江西。既定,命以親軍指揮守吉安。處州苗軍叛,殺守將耿再成,深從平章邵榮討誅之。會改中書分省為浙東行中書省,遂以深為行省左右司郎中,總制處州軍民事。時山寇竊發,人情未固,深募兵萬餘人,捕誅渠帥。沿海軍素驍,誅其尤橫者數人,患遂息。癸卯九月,諸全叛將謝再興以張士誠兵犯東陽。左丞李文忠令深引兵為前鋒,再興敗走。深建議以諸全為浙東籓屏,乃度地去諸全五十里並五指山築新城,分兵戍守。太祖初聞再興叛,急馳使詣文忠,別為城守計。至則工已竣。後士誠將李伯昇大舉來侵,頓新城下,不能拔,敗去。太祖嘉深功,賜以名馬。

太祖稱吳王,以深為王府參軍,仍守處州。溫州豪周宗道聚眾據平陽。數為方國珍從子明善所偪,以城來歸。明善怒,攻之。深遣兵擊走明善,遂下瑞安,進兵溫州。方氏懼,請歲輸銀三萬充軍實。乃命深班師,復還鎮。陳友定兵至,破之,追至浦城,又敗其守兵,城遂下。進拔松溪,獲其守將張子玉。因請發廣信、撫州、建昌三路兵,規取八閩。太祖喜曰:「子玉驍將,擒之則友定破膽。乘勢攻之,理無不克。」因命廣信指揮朱亮祖由鉛山、建昌,左丞王溥由杉關,會深齊進。已,亮祖等克崇安,進攻建寧。友定將阮德柔固守。深視氛祲不利,欲緩之。亮祖曰:「師已至此,庸可緩乎?且天道幽遠,山澤之氣變態無常,何足徵也。」時德柔兵屯錦江,逼深陣後。亮祖督戰益急。深引兵還擊,破其二柵。德柔軍力戰,友定自以銳師夾擊。日已暮,深突圍走,馬蹶被執,遂遇害,年五十二。追封縉雲郡伯。

太祖嘗問宋濂曰:「胡深何如人?」對曰:「文武才也。」太祖曰:「誠然。浙東一障,吾方賴之。」而深以久任鄉郡,志圖平閩以報效,竟以死徇。深馭眾寬厚,用兵十餘年,未嘗妄戮一人。守處州,興學造士。縉雲田稅重,以新沒入田租償其數。鹽稅什一,請半取之,以通商賈。軍民皆懷其惠云。

孫興祖,濠人。從太祖渡江,積功為都先鋒。戰龍江,遷統軍元帥。破瑞昌八陣營,擢天策衛指揮使。興祖沉毅有謀,大將軍徐達雅重之。克泰州,以達請,命守海陵。海陵,士誠兵入淮要地也。興祖整軍令,練士伍,防禦甚嚴。吳兵自海口來侵,擊敗之,擒彭元帥。平江既下,命興祖取通州,士誠守將已詣徐達降。進大都督府副使,移鎮彭城。達既定關陜,旋師北向,檄興祖會東昌。從克元都。置燕山六衛,留兵三萬人,命興祖守之,領大都督分府事。大兵西征,擴廓由居庸窺北平。達謂諸將:「北平有孫都督,不足慮。」遂直搗太原。語詳《達傳》。洪武三年,帥六衛卒從達出塞,次三不剌川,遇敵,力戰死,年三十五。太祖悼惜之,追封燕山侯,謚忠愍,配享通州常遇春祠。

未幾,中書省以都督同知汪興祖兼俸事入奏。帝聞奏興祖名,歎息,命以月俸給故燕山侯興祖家。以其長子恪襲武德衛指揮使。久之,歷都督僉事。二十一年,以右參將從藍玉北征,至捕魚兒海。論功封全寧侯,歲祿二千石,予世券。恪謹敏,有儒將風。從征楚、蜀,還駐沔陽,簡閱各衛所軍士備邊。二十五年,進兼太子太保。未幾,籍兵山西,從宋國公勝練兵。旋召還,賜第中都。後坐藍玉黨死。

曹良臣,安豐人。潁寇起,聚鄉里築堡自固。歸太祖於應天,為江淮行省參政。從取淮東,收浙西,進行省左丞。從大軍取元都,略地至澤、潞。進山西行省平章,還守通州。時大兵出山西,通州守備單弱,所部不滿千人。元丞相也速將萬騎營白河。良臣曰:「吾兵少,不可與戰。彼眾雖多,亡國之餘,敗氣不振,當以計走之。」乃密遣指揮仵勇等於瀕河舟中多立赤幟,亙三十餘里,鉦鼓聲相聞。也速大駭,遁去。良臣出精騎逐交百餘里,元兵自是不敢窺北平。復從大將軍達擊擴廓帖木兒於定西,敗之。洪武三年封宣寧侯,歲祿九百石,予世券。

明年從伐蜀,克歸州山寨,取容美諸土司。會周德興拔茅岡覃垕寨,自白鹽山伐木開道,出紙坊溪以趨夔州,進克重慶。明年從副將軍文忠北征,至臚朐河,收其部籓。文忠帥良臣持二十日糧,兼程進至土剌河。哈剌章渡河拒戰,少卻。追至阿魯渾河,敵騎大集。將士皆殊死戰,敵大敗走,而良臣與指揮周顯、常榮、張耀皆戰死。事聞,贈良臣安國公,謚忠壯,列祀功臣廟。子泰襲侯,坐藍玉黨死,爵除。

顯,合肥人。從渡江,累功至指揮同知。洪武三年,以收應昌紅羅山寨,遷指揮使。榮,開平王遇春再從弟,歷指揮僉事。遇春卒於軍,榮護喪還。從朱亮祖平蜀,累官至振武衛指揮同知。耀,壽州人,初從陳埜先。建康下,始歸附。累功為守禦福建指揮使,守興化。至是俱戰沒,帝厚恤諸臣家,命有司各表其墓。

濮英,廬州人。初以勇力為百夫長,積功至西安衛指揮。坐軍政不修,召還詰責,遣葉昇代之。昇更言其賢,令還衛。洪武十九年,太祖命耿炳文選陜西都司衛所卒備邊,惟英所練稱勁旅,加都督僉事。明年命師所部隨大將軍馮勝北征。抵金山,降納哈出,遂班師,而以英將奇兵三千人為殿。納哈出餘眾竄匿者尚數十萬,聞帥旋,設伏於途,謀俟大軍過竄取之。未發。英後至,猝為所乘,衝突不能出,馬踣,遂見執。敵既得英,思挾為質。英絕食不言,乘間引佩刀剖腹死。事聞,贈金山侯,謚忠襄。明年進贈樂浪公。封其子璵為西涼侯,祿二千五百石,予世券。二十三年,命練兵東昌,又令駐臨清,訓練士卒。二十五年,召還,同宋國公勝等簡閱山西士馬。璵能修父職,帝甚嘉之。復令籍山西民兵,所籍州縣最多,事集而不擾。明年坐藍玉黨,戍五開死。

洪武中指揮使死事者,又有于光、嚴德、孫虎。

光,都昌人。初事徐壽輝,鎮浮梁。陳友諒弒壽輝,光以浮梁來降,授樞密院判。積功為鷹揚衛指揮,鎮鞏昌。擴廓圍蘭州,光赴援至馬蘭灘,戰敗被執。以徇城下。光大呼曰:「公等但堅守,徐將軍將大軍旦夕至矣!」賊怒,批其頰,遂被殺,祀功臣廟。

嚴德,太平人。從起兵,積功為海寧衛指揮。從朱亮祖討方國珍,戰歿於台州。追封天水郡公。

孫虎,不知何許人。從援池州,下於潛、昌化,定建德、諸全,皆有功。授千戶。克新城、桐、廬,進海寧衛指揮使。平嘉興盜。從副將軍李文忠北征,由東道入應昌,至落馬河與元兵戰死。追封康安郡伯。

又指揮僉事劉廣,戍永平,禦寇戰死。涼州衛百戶劉林戍涼州,也先帖木兒叛,戰死。邊人壯之,名其所居竇融臺為劉林臺。錢塘衛千戶袁興,全椒人。從征雲南,自請為前鋒,陷陣死。並褒贈有差。

贊曰:明祖之興,自決策渡江,始力爭於東南數千里之內,摧友諒,滅士誠,然後北定中原,南圖閩、粵,則廖永安胡大海以下諸人,厥功豈細哉!計不旋踵,效命疆場,雖勛業未竟,然褒崇廟祀,竹帛爛然。以視功成命爵、終罹黨籍者,其猶幸也夫。


\end{pinyinscope}