\article{李善長、汪廣洋列傳}


李善長,字百室,定遠人。少讀書有智計,習法家言,策事多中。太祖略地滁陽,善長迎謁。知其為里中長者,禮之,留掌書記。嘗從容問曰:「四方戰鬥,何時定乎?」對曰:「秦亂,漢高起布衣,豁達大度,知人善任,不嗜殺人,五載成帝業。今元綱既紊,天下土崩瓦解。公濠產,距沛不遠。山川王氣,公當受之。法其所為,天下不足定也。」太祖稱善。從下滁州,為參謀,預機畫,主饋餉,甚見親信。太祖威名日盛,諸將來歸者,善長察其材,言之太祖。復為太祖布款誠,使皆得自安。有以事力相齟齬者,委曲為調護。郭子興中流言,疑太祖,稍奪其兵柄。又欲奪善長自輔,善長固謝弗往。太祖深倚之。太祖軍和陽,自將擊雞籠山寨,少留兵佐善長居守。元將諜知來襲,設伏敗之,太祖以為能。

太祖得巢湖水師,善長力贊渡江。既拔采石,趨太平,善長預書榜禁戢士卒。城下,即揭之通衢,肅然無敢犯者。太祖為太平興國翼大元帥,以為帥府都事。從克集慶。將取鎮江,太祖慮諸將不戢下,乃佯怒欲置諸法,善長力救得解。鎮江下,民不知有兵。太祖為江南行中書省平章,以為參議。時宋思顏、李夢庚、郭景祥等俱為省僚,而軍機進退,賞罰章程,多決於善長。改樞密院為大都督府,命兼領府司馬,進行省參知政事。

太祖為吳王,拜右相國。善長明習故事,裁決如流,又嫻於辭命。太祖有所招納,輒令為書。前後自將征討,皆命居守,將吏帖服,居民安堵,轉調兵餉無乏。嘗請榷兩淮鹽,立茶法,皆斟酌元制,去其弊政。既復制錢法,開鐵冶,定魚稅,國用益饒,而民不困。吳元年九月論平吳功,封善長宣國公。改官制,尚左,以為左相國。太祖初渡江,頗用重典,一日,謂善長:「法有連坐三條,不已甚乎?」善長因請自大逆而外皆除之,遂命與中丞劉基等裁定律令,頒示中外。

太祖即帝位,追帝祖考及冊立后妃太子諸王,皆以善長充大禮使。置東宮官屬,以善長兼太子少師,授銀青榮祿大夫、上柱國,錄軍國重事,餘如故。已,帥禮官定郊社宗廟禮。帝幸汴梁,善長留守,一切聽便宜行事。尋奏定六部官制,議官民喪服及朝賀東宮儀。奉命監修《元史》,編《祖訓錄》、《大明集禮》諸書。定天下嶽瀆神祗封號,封建諸王,爵賞功臣,事無巨細,悉委善長與諸儒臣謀議行之。

洪武三年大封功臣。帝謂:「善長雖無汗馬勞,然事朕久,給軍食,功甚大,宜進封大國。」乃授開國輔運推誠守正文臣、特進光祿大夫、左柱國、太師、中書左丞相,封韓國公,歲祿四千石,子孫世襲。予鐵券,免二死,子免一死。時封公者,徐達、常遇春子茂、李文忠、馮勝、鄧愈及善長六人。而善長位第一,制詞比之蕭何,褒稱甚至。

善長外寬和,內多忮刻。參議李飲冰、楊希聖,稍侵善長權,即按其罪奏黜之。與中丞劉基爭法而訽。基不自安,請告歸。太祖所任張昶、楊憲、汪廣洋、胡惟庸皆獲罪,善長事寄如故。貴富極,意稍驕,帝始微厭之。四年以疾致仕,賜臨濠地若干頃,置守塚戶百五十,給佃戶千五百家,儀仗士二十家。踰年,病愈,命董建臨濠宮殿。徙江南富民十四萬田濠州,以善長經理之,留濠者數年。七年擢善長弟存義為太僕丞,存義子伸、佑皆為群牧所官。九年以臨安公主歸其子祺,拜駙馬都尉。初定婚禮,公主修婦道甚肅。光寵赫奕,時人艷之。祺尚主後一月,御史大夫汪廣洋、陳寧疏言:「善長狎寵自恣,陛下病不視朝幾及旬,不問候。駙馬都尉祺六日不朝,宣至殿前,又不引罪,大不敬。」坐削歲祿千八百石。尋命與曹國公李文忠總中書省大都督府御史臺,同議軍國大事,督圜丘工。

丞相胡惟庸初為寧國知縣,以善長薦,擢太常少卿,後為丞相,因相往來。而善長弟存義子佑,惟庸從女婿也。十三年,惟庸謀反伏誅,坐黨死者甚眾,善長如故。御史臺缺中丞,以善長理臺事,數有所建白。十八年,有人告存義父子實惟庸黨者,詔免死,安置崇明。善長不謝,帝銜之。又五年,善長年已七十有七,耄不檢下。嘗欲營第,從信國公湯和假衛卒三百人,和密以聞。四月,京民坐罪應徙邊者,善長數請免其私親丁斌等。帝怒按斌,斌故給事惟庸家,因言存義等往時交通惟庸狀。命逮存義父子鞫之,詞連善長,云:「惟庸有反謀,使存義陰說善長。善長驚叱曰:『爾言何為者!審爾,九族皆滅!』已,又使善長故人楊文裕說之云:『事成當以淮西地封為王。』善長驚不許,然頗心動。惟庸乃自往說善長,猶不許。居久之,惟庸復遣存義進說,善長歎曰:『吾老矣。吾死,汝等自為之!』」或又告善長云:「將軍藍玉出塞,至捕魚兒海,獲惟庸通沙漠使者封績,善長匿不以聞。」於是御史交章劾善長。而善長奴盧仲謙等,亦告善長與惟庸通賂遺,交私語。獄具,謂善長元勛國戚,知逆謀不發舉,狐疑觀望懷兩端,大逆不道。會有言星變,其占當移大臣。遂并其妻女弟姪家口七十餘人誅之。而吉安侯陸仲亨、延安侯唐勝宗、平涼侯費聚、南雄侯趙庸、滎陽侯鄭遇春、宜春侯黃彬、河南侯陸聚等,皆同時坐惟庸黨死,而已故營陽侯楊璟、濟寧侯顧時等追坐者又若干人。帝手詔條列其罪,傅著獄辭,為《昭示姦黨三錄》,布告天下。善長子祺與主徙江浦,久之卒。祺子芳、茂,以公主恩得不坐。芳為留守中衛指揮,茂為旗手衛鎮撫,罷世襲。

善長死之明年,虞部郎中王國用上言:「善長與陛下同心,出萬死以取天下,勳臣第一,生封公,死封王,男尚公主,親戚拜官,人臣之分極矣。藉令欲自圖不軌,尚未可知,而今謂其欲佐胡惟庸者,則大謬不然。人情愛其子,必甚於兄弟之子,安享萬全之富貴者,必不僥倖萬一之富貴。善長與惟庸,猶子之親耳,於陛下則親子女也。使善長佐惟庸成,不過勛臣第一而已矣,太師國公封王而已矣,尚主納妃而已矣,寧復有加於今日?且善長豈不知天下之不可倖取。當元之季,欲為此者何限,莫不身為齏粉,覆宗絕祀,能保首領者幾何人哉?善長胡乃身見之,而以衰倦之年身蹈之也。凡為此者,必有深仇激變,大不得已,父子之間或至相挾以求脫禍。今善長之子祺備陛下骨肉親,無纖芥嫌,何苦而忽為此。若謂天象告變,大臣當災,殺之以應天象,則尤不可。臣恐天下聞之,謂功如善長且如此,四方因之解體也。今善長已死,言之無益,所願陛下作戒將來耳。」太祖得書,竟亦不罪也。

汪廣洋,字朝宗,高郵人,流寓太平。太祖渡江,召為元帥府令史,江南行省提控。置正軍都諫司,擢諫官,遷行省都事,累進中書右司郎中。尋知驍騎衛事,參常遇春軍務。下贛州,遂居守,拜江西參政。

洪武元年,山東平,以廣洋廉明持重,命理行省,撫納新附,民甚安之。是年召入為中書省參政。明年出參政陜西。三年,李善長病,中書無官,召廣洋為左丞。時右丞楊憲專決事。廣洋依違之,猶為所忌,嗾御史劾廣洋奉母無狀。帝切責,放還鄉。憲再奏,徙海南。憲誅,召還。其冬,封忠勤伯,食祿三百六十石。誥詞稱其專刂繁治劇,屢獻忠謀,比之子房、孔明。及善長以病去位,遂以廣洋為右丞相,參政胡惟庸為左丞。廣洋無所建白,久之,左遷廣東行省參政,而帝心終善廣洋,復召為左御史大夫。十年復拜右丞相。廣洋頗耽酒,與惟庸同相,浮沉守位而已。帝數誡諭之。

十二年十二月,中丞塗節言劉基為惟庸毒死,廣洋宜知狀。帝問之,對曰:「無有。」帝怒,責廣洋朋欺,貶廣南。舟次太平,帝追怒其在江西曲庇文正,在中書不發楊憲姦,賜敕誅之。

廣洋少師餘闕,淹通經史,善篆隸,工為歌詩。為人寬和自守,與姦人同位而不能去,故及於禍。贊曰:明初設中書省,置左右丞相,管領樞要,率以勛臣領其事。然徐達、李文忠等數受命征討,未嘗專理省事。其從容丞弼之任者,李善長、汪廣洋、胡惟庸三人而已。惟庸敗後,丞相之官遂廢不設。故終明之世,惟善長、廣洋得稱丞相。獨惜善長以布衣徒步,能擇主於草昧之初,委身戮力,贊成鴻業,遂得剖符開國,列爵上公,乃至富極貴溢,於衰暮之年自取覆滅。廣洋謹厚自守,亦不能發奸遠禍。俱致重譴,不亦大負爰立之初心,而有愧置諸左右之職業也夫?


