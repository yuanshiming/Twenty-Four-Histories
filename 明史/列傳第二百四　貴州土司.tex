\article{列傳第二百四 貴州土司}

\begin{pinyinscope}
貴州,古羅施鬼國。漢西南夷牂牁、武陵諸傍郡地。元置八番、順元諸軍民宣慰使司,以羈縻之。明太祖既克陳友諒,兵威遠振,思南宣慰、思州宣撫率先歸附,即令以故官世守之,時至正二十五年也。及洪武五年,貴州宣慰靄翠與宋蒙古歹及普定府女總管適爾等先後來歸,皆予以原官世襲。帝方北伐中原,未遑經理南荒。又田仁智等歲修職貢,最恭順,乃以衛指揮僉事顧成築城以守,賦稅聽自輸納,未置郡縣。

永樂十一年,思南、思州相仇殺,始命成以兵五萬執之,送京師。乃分其地為八府四州,設貴州布政使司,而以長官司七十五分隸焉,屬戶部。置貴州都指揮使,領十八衛,而以長官司七隸焉,屬兵部。府以下參用土官。其土官之朝貢符信屬禮部,承襲屬吏部,領土兵者屬兵部。其後府并為六,州並為四,長官司或分或合,釐革不一。其地西接滇、蜀,東連荊、粵。太祖於《平滇詔書》言:「靄翠輩不盡服之,雖有雲南不能守也」,則志已在黔,至成祖遂成之。然貴州地皆崇山深菁,鳥道蠶叢,諸蠻種類,嗜淫好殺,畔服不常。靄翠歸附之初,請討其隴居部落。帝曰:「中國之兵,豈外夷報怨之具。」及仁智入朝,帝諭之曰:「天下守土之臣,皆朝廷命吏,人民皆朝廷赤子,汝歸善撫之,使各安其生,則汝可長享富貴。夫禮莫大於敬上,德莫盛於愛下,能敬能愛,人臣之道也。」二十一年,部臣以貴州逋賦請,帝曰:「蠻方僻遠,來納租賦,是能遵聲教矣。逋負之故,必由水旱之災,宜行蠲免。自今定其數以為常,從寬減焉。」二十九年,清水江之亂既平,守臣以賊首匿宣慰家,宜並罪。帝曰:「蠻人鴟張鼠伏,自其常態,勿復問。」明初御蠻之道,其後世之龜鑑也夫。

○貴陽思南思州附鎮遠銅仁黎平安順都勻平越石阡新添金築安撫司附

貴陽府,舊為程番長官司。洪武初,置貴州宣慰司,隸四川。永樂十一年改隸貴州。成化十二年置程番府。隆慶三年移程番府為貴陽府,與宣慰司同城,府轄城北,司轄城南。萬曆時,改為貴陽軍民府。領安撫司一,曰金築;領長官司十八,曰貴竹,曰麻嚮,曰本瓜,曰大華,曰程番,曰韋番,曰方番,曰洪番,曰臥龍番,曰金石番,曰小龍番,曰羅番,曰大龍番,曰小程番,曰上馬橋,曰盧番,曰盧山,曰平伐。其貴州宣慰司所領長官司九,曰水東,曰中曹,曰青山,曰札佐,曰龍里,曰白納,曰底寨,曰乖西,曰養龍坑。

自蜀漢時,濟火從諸葛亮南征有功,封羅甸國王。後五十六代為宋普貴,傳至元阿畫,世有土於水西宣慰司。靄翠,其裔也,後為安氏。洪武初,同宣慰宋蒙古歹來歸,賜名欽,俱令領原職世襲。及設布政使司,而宣慰司如故。安氏領水西,宋氏領水東。八番降者,皆令世其職。六年詔靄翠位各宣慰之上。靄翠每年貢方物與馬,帝賜錦綺鈔幣有加。十四年,宋欽死,妻劉淑貞隨其子誠入朝,賜米三十石、鈔三百錠、衣三襲。時靄翠亦死,妻奢香代襲。都督馬曄欲盡滅諸羅,代以流官,故以事撻香,激為兵端。諸羅果怒,欲反。劉淑貞聞止之,為走醖京師。帝既召問,命淑貞歸,招香,賜以綺鈔。十七年,奢香率所屬來朝,並訴曄激變狀,且願效力開西鄙,世世保境。帝悅,賜香錦綺、珠翠、如竟冠、金環、襲衣,而召曄還,罪之。香遂開偏橋、水東,以達烏蒙、烏撒及容山、草塘諸境,立龍場九驛。二十年,香進馬二十三匹,每歲定輸賦三萬石。子安的襲,貢馬謝恩。帝曰:「安的居水西,最為誠恪。」命禮部厚賞其使。二十五年,的來朝,賜三品服並襲衣金帶、白金三百兩、鈔五十錠。香復遣其子婦奢助及其部長來貢馬六十六匹,詔賜香銀四百兩,錦綺鈔幣有差。自是每歲貢獻不絕,報施之隆,亦非他土司所敢望也。二十九年,香死,朝廷遣使祭之,的貢馬謝恩。

正統七年,水西宣慰隴富自陳:「祖父以來,累朝皆賜金帶。臣蒙恩受職,乞如例。」從之。是時,宋誠之子斌年老,以子昂代,昂死,然代。十四年賜敕隴富母子,嘉其調兵保境之功。隴富頗驕。天順三年,東苗之亂,富不時出兵,聞朝廷有意督之,乃進馬謝罪,賜敕警之。富死,姪觀襲。觀老,子貴榮襲。巡撫陳儀以西堡獅子孔之平,由觀與子貴榮統部眾二萬攻白石崖,四旬而克,家自餽餉,口不言功,特給觀正三品昭勇將軍誥。初,安氏世居水西,管苗民四十八族,宋氏世居貴州城側,管水東、貴竹等十長官司,皆設治所於城內,銜列左右。而安氏掌印,非有公事不得擅還水西。至是總兵官為之請,許其以時巡歷所部,趣辦貢賦,聽暫還水西,以印授宣慰宋然代理。貴榮老,請以子佐襲,命賜貴榮父子錦糸寧。

先是,宋然貪淫,所管陳湖等十二馬頭科害苗民,致激變。而貴榮欲並然地,誘其眾作亂。於是阿朵等聚眾二萬餘,署立名號,攻陷寨堡,襲據然所居大羊腸,然僅以身免。貴榮遽以狀上,冀令己按治之。會阿朵黨洩其情,官軍進討。貴榮懼,乃自率所部為助。及賊平,貴榮已死,坐追奪,然坐斬。然奏世受爵土,負國厚恩。但變起於榮,而身陷重辟,乞分釋。因從末減,依土俗納粟贖罪。都御史請以貴築、平伐七長官司地設立府縣,皆以流官撫理。巡撫覆奏以蠻民不願,遂寢。宋氏亦遂衰,子孫守世官,衣租食稅,聽徵調而已。

時安萬鐘應襲,驕縱不法。漢民張純、土目烏掛等導之游獵,酒酣,輒射人為戲。又嘗撻其左右,為所殺。無子,其從弟萬鎰宜襲,鎰以賊未獲辭。烏掛等遂以疏族幼子普者冒萬鐘弟曰萬鈞告襲,承勘官入其賄,遂暫委鐘妻奢播攝事。萬鎰悔不立,而恨烏掛之主其謀也,遂以兵襲烏掛,烏掛亦發兵相仇殺,皆以萬鐘之死為辭。巡按御史上其狀,以萬鎰宜襲,但與烏掛相誣訐,宜各宥輸贖。而梟殺鐘者,并戍純等,受其賄者亦罰治,詔如之。未幾,鎰死,子阿寫幼,命以萬銓借襲。萬銓有助平阿向功,提督尚書伍文定為之請。萬銓亦自陳其功,乞加參政銜,賜蟒衣,帝命賜以應得之服。後阿寫長,襲職,改名仁。未幾死,子國亨襲。淫虐,乃以事殺萬銓之子信。信兄智與其母別居於安順州,聞之,因告國亨反。巡撫王諍遽請發兵誅國亨,智遂為總兵安大朝畫策,且約輸兵糧數萬。及師至陸廣河,智糧不至。諍乃令人諭國亨,而止大朝毋進。兵已渡河,為國亨所敗。國亨懼大誅,遣使哀辭乞降,朝廷未之許。巡撫阮文中至,檄捕諸反者,密使語國亨,亟出諸奸徒,割地以處安智母子,還所費兵糧,朝廷當待汝以不死。於是國亨悉聽命,帝果赦不誅,而命國亨子民襲。國亨事起於隆慶四年,至成曆五年乃已。國亨既革任,日遣人至京納賂,為起復地。十三年,播州宣慰楊應龍以獻大木得賜飛魚服,國亨亦請以大木進,乞還給冠帶誥封如播例。既而木竟不至,乃諉罪於木商。上怒,命奪所賚。國亨請補貢以明不欺,上仍如所請。

萬曆二十六年,國亨子疆臣襲職。會播州楊應龍反,疆臣亦以戕殺安定事為有司所案。科臣有言其逆節漸萌者,詔不問,許殺賊圖功。疆臣奏稱:「播警方殷,臣心未白。」上復優詔報之。巡撫郭子章許疆臣以應龍平後還播所侵水西烏江地六百里以酬功,於是疆臣兵從沙溪入。有蜚語水西佐賊者,總督李化龍檄詰之,疆臣遂執賊二十餘人,率所部奪落濛關,至大水田,焚桃溪莊。應龍伏誅。初,應龍之祖以內難走水西,客死。宣慰萬銓挾之,索水煙、天旺地,聽還葬,其地遂為水西所據。及播州平,分其地為遵義、平越二府,分隸蜀、黔,以渭河中心為界。總督王象乾代化龍,命疆臣歸所侵播州地。子章奏言:「侵地始於萬銓,而非疆臣。安氏迫取於楊相喪亂之時,非擅取於應龍蕩平之日。且臣曾許其裂土,今反奪其故地,臣無面目以謝疆臣,願罷去。」象乾疏言:「疆臣征番,殲應龍子惟楝不實,首功可知。至佯敗棄陣,送藥往來,欺君助逆,迹已昭然。令還侵地,不咎既往,已屬國家寬大。若因其挾而予之,彼不為恩,我且示弱。疆臣既無功,不與之地,正所以全撫臣之信。宜留撫臣罷臣,以為重臣無能與蕞爾苗噂沓者之戒。」於是清疆之議,累年不決。兵部責令兩省巡按御史勘報,而南北言官交章詆象乾貪功起釁。科臣呂邦耀復劾子章納賄縱奸,子章求去益力。象乾執疆臣所遣入京行賄之人與金,以聞於朝。然議者多右疆臣,尚書蕭大亨遂主巡按李時華疏,謂:「征播之役,水西不惟假道,且又助兵。矧失之土司,得之土司,播固輸糧,水亦納賦,不宜以土地之故傷字小之仁,地宜歸疆臣。」於是疆臣增官進秩,其母得賜祭,水西尾大之患,亦於是乎不可制矣。

三十六年,疆臣死,弟堯臣襲。四十一年,烏撒土舍謀逐安效良,堯臣以追印為名,領兵數萬長驅入滇,直薄霑益州,所過焚掠,備極慘毒。朝廷方以越境擅兵欲加堯臣罪,而堯臣死。子位幼,命其妻奢社輝攝事。社輝,永寧宣撫奢崇明女弟。崇明子寅獷悍,與社輝爭地,相仇恨。而安邦彥者,位之叔父也,素懷異志,陰與崇明合。及崇明反,調兵水西,邦彥遂挾位叛以應之,位幼弱不能制。邦彥更招故宣慰土舍宋萬化為助,率兵趨畢節,陷之,分兵破安順、平壩、霑益。而萬化亦率苗仲九股陷龍里,遂圍貴陽,自稱羅甸王,時天啟二年二月也。巡撫李枟方受代,聞變,與巡按御史史永安悉力拒守。賊攻不能克,則沿巖制柵,斷城中出入。鎮將張彥芳將兵二萬赴援,隔龍里不得進。貴州總兵楊愈懋、推官郭象儀與賊戰於江門而死。外援既絕,攻益急,城中糧盡,人相食,而拒守不遺餘力。中朝方急遼,不之省。已,以王三善為巡撫,倉卒調兵食,大會將士,分兵二道進。三日抵龍頭營,屢敗賊兵,遂奪龍里。邦彥聞新撫自將大兵數十萬,懼甚,遂退屯龍洞。前鋒楊明楷率烏羅兵擊死安邦俊,遂乘勝抵貴陽城下,先以五騎傳呼曰:「新撫至矣。」舉城懽呼更生。貴陽被圍十餘月,城中軍民男婦四十萬,至是餓死幾盡,僅餘二百人。詳《李枟》及《三善傳》中。

貴陽圍既解,邦彥遠遁陸廣河外。三善遣使諭社輝母子縛邦彥以降。大軍至者日益眾,三善欲因糧於敵。又諸軍視賊過易,楊明楷營於三十里外。邦彥復糾諸苗來攻,師敗,明楷為所執。邦彥勢復張,合眾欲再圍貴陽。三善遣兵三路禦之,破生苗寨二百餘,擒萬化等,焚其積聚數萬。龍里、定番四路並通,諸苗畔者相繼降。邦彥氣奪不敢出,於鴨池、陸廣諸要地掘塹屯兵,為自守計。時奢崇明為蜀兵所敗,計窮投水西,與邦彥合。

三年,三善督兵攻大方賊巢,擒土司何中尉等,進營紅崖。連破天台、水腳等七囤,奪其天險。別將亦破賊於羊耳,追至鴨池河,奪其戰象。遂深入至紅鳥岡,諸苗奔潰。三善率兵直入大方,奢社輝、安位焚其巢,竄火灼堡,邦彥奔織金。位遂遣人赴鎮遠,乞降於總督楊述中。許之,令擒崇明父子自贖,一意主撫。而三善責並獻邦彥,當並用剿,議不合。往返間已逾數月,邦彥得益兵為備。三善糧不繼,焚大方,還貴州,道遇賊,三善為所害。邦彥率數萬眾來追,總理魯欽力禦之,大戰數日,大軍無糧,乘夜皆潰,欽自剄。賊燒劫諸堡,苗兵復助逆,貴陽三十里外樵蘇不行,城中復大震。初,大方東倚播,北倚藺,相為掎角。後播、藺既平,賊惟恃烏撒為援,而畢節為四夷交通處。當三善由貴陽陸廣深入大方百七十里,皆羅鬼巢窟,以失地利而陷。天啟間,朱燮元為蜀督,建議滇兵出霑益,遏安效良應援,分兵於天生橋、尋甸等處,以絕其走;蜀兵臨畢節,扼其交通之路,而別出龍場巖後,以奪其險;黔兵由普定渡思臘河,徑趨邦彥巢,由陸廣、鴨池搗其虛;粵西兵出泗城,分道策應;然後大軍由遵義鼓行而前。尋以憂去,未及用。總督閔夢得繼之,亦以貴州抵大方路險,賊惟恃畢節一路外通。我兵宜從永寧始,自永寧而普市,而摩泥,而赤水,百五十里皆坦途。赤水有城郭可憑而守,宜結營進逼。四十里為白巖,六十里為層臺,又六十里為畢節。畢節至大方不及六十里,賊必併力來禦,須重兵扼之,斷其四走之路,然後遵義、貴陽剋期而進,亦不果用。及是黔事棘,詔起燮元總督貴、雲、川、廣。於是燮元再蒞黔,時崇禎元年也。

奢崇明自號大梁王,安邦彥自號四裔大長老,其部眾悉號元帥。悉力趨永寧,先犯赤水。燮元授意守將佯北,誘深入,度賊已抵永寧,分遣別將林兆鼎從三岔入,王國禎從陸廣入,劉養鯤從遵義入。邦彥分兵四應,力不支。羅乾象復以奇兵繞其背,急擊之,賊大驚潰,崇明、邦彥皆授首。邦彥亂七年而誅。燮元乃移檄安位,赦其罪,許歸附。位豎子不能決,其下謀合潰兵來拒。燮元扼其要害,四面迭攻,斬首萬餘級。復得向導,輒發窖粟就食,賊益饑。復遣人至大方燒其室廬,位大恐,遂率四十八目出降。燮元奏請許之,報可。而前助邦彥故宣慰宋萬化之子嗣殷亦至是始剿滅。乃以宋氏洪邊十二馬頭地置開州,建城設官。燮元復遣兵平擺金五洞諸叛苗,水西勢益孤。十年,安位死,無嗣,族屬爭立。朝議欲乘其敝郡縣之。燮元奏未可驟,乃傳檄土目,諭以威德,諸苗爭納土獻印。貴陽甫定,而明亦旋亡矣。

思南,即唐思州。宋宣和中,番部田祐恭內附,世有其地。元改宣慰司。明洪武初,析為二宣慰,屬湖廣。永樂十一年置思南府,領長官司四:曰水德江,曰蠻夷,曰沿河祐溪,曰朗溪。思州領長官司四:曰都坪峨異溪,曰都素,曰施溪,曰黃道溪。

初,太祖起兵平偽漢,略地湖南。思南宣慰使田仁智遣都事楊琛來歸附,並納元所授宣慰誥。帝以率先來歸,俾仍為思南道宣慰使,以三品銀印給之,並授琛為宣撫使。思州宣撫使田仁厚亦遣都事林憲、萬戶張思溫來獻鎮遠、古州軍民二府,婺川、功水、常寧等十縣,龍泉、瑞溪、沿河等三十四州。於是命改思州宣撫為思南鎮西等處宣慰使司,以仁厚為使,俱歲朝貢不絕。

二年,仁厚死,子弘正襲。帝以思南土官世居荒服,未嘗詣闕,詔令率其部長入朝。九年,仁智入覲,加賜織金文綺,並諭以敬上愛下保守爵祿之道。仁智辭歸,至九江龍城驛病卒。有司以聞,遣官致祭,並敕送柩歸思南。時思州田弘正與其弟弘道等來朝,帝命禮部皆優賜。十一年,仁智子大雅襲,奉表謝恩。命思南收集各洞弩手二千人,備徵調。十四年,大雅入朝。十八年,思州諸洞蠻作亂,命信國公湯和等討之。時寇出沒不常,聞師至,輒竄山谷間,退則復出剽掠。和等師抵其地,恐蠻人驚潰,乃令軍士於諸洞分屯立柵,與蠻人雜耕,使不復疑。久之,以計擒其魁,餘黨悉定,留兵鎮之。二十年移思南宣慰於鎮遠。大雅來謝恩。思州宣慰弘正死,子琛襲。三十年,大雅母楊氏來朝。

永樂八年,大雅死,子宗鼎襲。初,宗鼎兇暴,與其副使黃禧構怨,奏訐累年。朝廷以田氏世守其土,又先歸誠,曲與保全,改禧為辰州知府。未幾,思州宣慰田琛與宗鼎爭沙坑地有怨。禧遂與琛結,圖宗鼎,構兵。琛自稱天主,禧為大將,率兵攻思南。宗鼎挈家走,琛殺其弟,發其墳墓,並戮其母屍。宗鼎訴於朝,屢敕琛、禧赴闕自辨,皆拒命不至,潛使奸人入教坊司,伺隙為變。事覺,遣行人蔣廷瓚召之,命鎮遠侯顧成以兵壓其境,執琛、禧械送京師,皆引服。琛妻冉氏尤強悍,遣人招誘臺羅等寨苗普亮為亂,冀朝廷遣琛還招撫,以免死。帝聞而錮之。

以宗鼎窮蹙來歸,得未減,令復職,還思南。而宗鼎必得報怨,以絕禍根。帝以宗鼎幸免禍,不自懲,乃更逞忿,亦留之。宗鼎出誹言,因發祖母陰事,謂與禧奸,實造禍本。祖母亦發宗鼎縊殺親母瀆亂人倫事。帝命刑部正其罪,諭戶部尚書夏原吉曰:「琛、宗鼎分治思州、思南,皆為民害。琛不道,已正其辜。宗鼎滅倫,罪不可宥。其思州、思南三十九長官地,可更郡縣,設貴州布政使司總轄之。」命顧成剿臺羅諸寨。成斬苗賊普亮,思州乃平。十二年遂分其地為八府四州,貴州為內地,自是始。兩宣慰廢,田氏遂亡。

正統初,蠻夷長官司奏土官衙門婚姻,皆從土俗,乞頒恩命。帝以土司循襲舊俗,因親結婚者,既累經赦宥不論,繼今悉依朝廷禮法,違者罪之。景泰間,思南府奏府四面皆山,關隘五處,無城可守,乞發附近土軍修築。命巡撫王來經畫之。

鎮遠,故為豎眼大田溪洞。元初,置鎮遠沿邊溪洞招討使,後改為鎮遠府。洪武五年改為州,隸湖廣。永樂十一年仍改府,屬貴州。領長官司二:曰遍橋,曰邛水十五洞。領縣二:曰鎮遠,即金容金達、楊溪公俄二長官司地;曰施秉,即施秉長官司地也。洪武二十年,土官趙士能來朝,貢馬。三十年,鎮遠鬼長菁等處苗民作亂,指揮萬繼、百戶吳彬戰死。都指揮許能率兵會偏橋衛軍擊敗之,眾散走。永樂初,鎮遠長官何惠言:「每歲修治清浪、焦溪、鎮遠三橋,工費浩大。所部臨溪部民,皆佯、儣、人苗、佬,力不勝役,乞令軍民參助。」從之。

宣德初,鎮遠邛水奧洞蠻苗章奴劫掠清浪道中,為思州都坪峨異溪長官司所獲。其父苗銀總劫取之,聚兵欲攻思州。因令赤溪洞長官楊通諒往撫,銀總伏兵殺諒,又掠埂洞。命總兵官蕭授調辰、沅諸衛兵萬四千人剿之,會於清浪衛,指揮張名討銀總,克奧洞,盡殺其黨,銀總遁。正統三年革鎮遠州,以鎮遠、施秉二長官司隸鎮遠府。十二年,巡按御史虞禎奏:「貴州蠻賊出沒,撫之不從,捕之不得,若非設策,難以控制。臣觀清水江等處,峭壁層崖,僅通一徑出入,彼得恃險為惡。若將江外山口盡行閉塞,江內山口并津渡俱設關堡,屯兵守禦,又擇寨長有才幹者為辦事官,庶毋疏虞。」從之。十四年命振偏橋衛,以被苗寇殺掠,不能自存,有司以請,從之。

天順七年,鎮守湖廣太監郭閔奏:「貴州洪江賊苗蟲蝦等糾合二千餘人,偽稱王侯,攻劫鎮遠屯寨。撫諭不服,請合兵進討。」命總兵官李震、李安等分道入,賊退守平坤寨,官兵追至清水江,獲蟲蝦,並斬賊首飛天侯、苗老底、額頭等六百四十餘人,并復黎平之赤溪湳洞,賊平。弘治十年改鎮遠金容金達長官司為鎮遠州,設流官。時土官碖父子罪死,土人思得流官,守臣以聞,報可。

萬曆末,邛水長官司楊光春貪暴,土目彭必信濟之箕斂。苗不堪,將上訴改設流官,光春與必信遂謀反,言官兵欲剿諸苗,當斂金贖,得金五百餘。都御史何起鳴詗知之,捕光春下獄,瘐死。於是每四戶擇壯兵一人,立四哨,不為兵者佐糗糧魚鹽,簡土吏何文奎等掌之。必信復醵諸苗金,訴於朝,言巴也、梁止諸寨為亂,指揮使陶效忠不問,反索土官楊光春金而殺之。改舊例用新法,不便。書上,意自得,歸謁知府王一麟。一麟縛之下獄,檄諸苗,言:「若等十五洞所苦者,以兵餉月米三斗過甚耳。然歲給白蟲鋪米,每洞月八斗,他於平溪驛剩餘徵銀兩,皆可足餉。我為若通之,毋為必信所誣。」苗皆悅服,乃坐必信罪。時有土舍楊載清者應襲推官,嘗中貴州鄉試,命於本衛加俸級優異之。

天啟五年,巡撫傅宗龍奏:「苗寇披猖,地方受害,乞敕偏沅撫臣移鎮偏橋,勿復回沅,凡思、石、偏、鎮等處俾練兵萬餘人,平時以之剿苗,大徵即統為督臣後勁,庶苗患寧而西賊之氣亦漸奪矣。」報可。

銅仁,元為銅人大小江等處軍民長官司。洪武初,改為銅仁長官司。永樂十一年置銅仁府。萬曆二十六年始改銅仁長官司為縣治。領長官司五:曰省溪,曰提溪,曰大萬山,曰烏羅,曰平頭著可。烏羅者,本永樂時分置貴州八府之一也,所屬有朗溪長官司、答意長官司、治古長官司,而平頭著可長官司亦隸焉。

宣德五年,烏羅知府嚴律己言:「所屬治古、答意二長官石各野等聚眾出沒銅仁、平頭、甕橋諸處,誘脅蠻賊石雞娘並筸子坪長官吳畢郎等共為亂,招撫不從。緣其地與鎮溪、酉陽諸蠻接境,恐相煽為亂。請調官土軍分據要地,絕其糧道,且捕且撫。事平之後,宜置衛所巡司以守之。」事聞,命總兵官蕭授及鎮巡諸司議。於是授築二十四堡,環其地守之。兵力分,卒難扞禦。賊四出劫掠,殺清浪衛鎮撫葉受,勢益獗。七年,巡按御史以聞,且言生苗之地不過三百餘里,乞別遣良將督諸軍殄滅。授言:「殘苗吳不爾等遁入筸子坪,結生苗龍不登等攻劫湖廣五寨及白崖諸寨,為患滋甚。宜令川、湖、貴州接境諸官軍、土兵分路併力攻剿,庶除邊患。」從之。既降敕諭授,言:「暴師久,恐蹉跌為蠻羞,或撫或剿,朕觀成功,不從中制。

八年,授奏言:「臣受命統率諸軍進攻賊巢,破新郎等寨,前後生擒賊首吳不跳等二百一十二人,斬吳不爾、王老虎、龍安軸等五百九十餘級,皆梟以徇,餘黨悉平。還所掠軍民男婦九十八口,悉給所親。獲賊婦女幼弱一千六百餘口,以給從征將士。」并械吳不跳等獻京師。帝顧謂侍臣曰:「蠻苗好亂,自取滅亡,然於朕心,不能無惻然也。」授威服南荒,前後凡二十餘年。

正統三年革烏羅府,所屬治古、答意二長官司,亂後殘民無幾,亦並革之,以烏羅、平頭著可隸銅仁,以朗溪隸思南,從巡按御史請也。景泰七年,平頭著可長官司奏其地多為蠻賊侵害,乞立土城固守,從之。成化十一年,總兵官李震奏:「烏羅苗人石全州,妄稱元末明氏子孫,僭稱明王,糾眾於執銀等處作亂,鄰洞多應之。因調官軍往剿,石全州已就擒,而諸苗攻劫未已。」命鎮巡官設策撫捕,未幾平。嘉靖二十二年,平頭苗賊龍桑科作亂,流劫湖廣桂陽間,甚獗。帝以諸苗再叛,責激亂者,而起都御史萬鏜往討之。明年,鏜奏叛苗以次殄滅,惟龍母叟雖降,然其罪大,宜置重典。命安置遼東。未幾,龍子賢復叛。二十六年,湖貴巡按御史奏官軍討賊不力,降旨切責。三十九年,總兵官石邦憲剿之,擒首惡龍老羅等,遂平。

黎平,元潭溪地也。洪武初,仍各長官司。永樂十一年改置黎平、新化二府。宣德十年並新化入黎平。領長官司十三:曰潭溪,曰八舟,曰洪舟泊里,曰曹滴洞,曰古州,曰西山陽洞,曰湖耳,曰亮寨,曰歐陽,曰新化,曰中林驗洞,曰赤溪湳洞,曰龍里。

初,洪武三年,辰州衛指揮劉宣武率兵招降湖耳、潭溪、新化、萬平江、歐陽諸洞,於是諸洞長官皆來朝,納元所授印敕。帝命皆仍其原官,以轄洞民,隸辰州衛。既改龍里長官司為龍里衛,又增立五開衛以鎮之,隸思州。二十九年,清水江蠻金牌黃作亂,都司發兵捕之,金牌黃遁去。捕獲其黨五百餘人,械至京,以其脅從,宥死,戍遠衛。既有言金牌黃匿宣慰家者,詔勿問。三十年,古州洞蠻林寬者,自號小師,聚眾作亂,攻龍里。千戶吳得、鎮撫井孚力戰死之。寬遂犯新化,突至平茶,千戶紀達率壯士擊之。達突陣殺數人,以鎗橫挑一人擲之,流矢中臂,達拔矢復戰。賊驚曰:「是平茶紀蒙邪?」遁去。蠻稱官為蒙云。已,復熾,命湖廣都指揮使齊讓為平羌將軍,統兵五萬征之。既以讓逗遛,命楊文代之。又命楚王楨、湘王柏各率護衛兵進討,城銅鼓衛。未幾,讓擒寬等,械入京,誅之。三十一年復平其餘黨,並俘獲三十岡等處洞蠻二千九百人以歸,遂班師。

永樂五年,寨長韋萬木來朝,自陳所統四十七寨,乞設官。因設西山陽洞長官司,以萬木為屯長。宣德六年改永從蠻夷長官司為永從縣,置流官,以土官李瑛絕故也。又割思州新溪等十一寨隸黎平赤溪湳洞長官司。正統四年,計砂苗賊苗金蟲等糾合洪江生苗,偽立統千侯、統萬侯名號,劫掠四出,命都督蕭授調兵剿之。賊首苗總牌等為都督吳亮所戮,洪江生苗遂詣軍門降。授諭遣之,命千戶尹勝誘執苗金蟲,斬以徇。

景泰五年,巡撫王永壽以苗賊蒙能攻圍龍里、新化、銅鼓諸城,乞調兵剿之。時賊欲取龍里為巢穴,攻破亮寨、銅鼓、羅圍堡諸城,都指揮汪迪為賊所殺。朝議以南和伯方瑛為平蠻將軍,統湖廣諸軍討之。蒙能糾賊眾三萬出攻平溪衛,瑛遣指揮鄭泰等以火鎗攻,斃賊三千人,能亦死。而能黨李珍等尚煽惑苗眾,官軍計擒之,克復銅鼓、藕洞,連破鬼板等一百六十餘寨,覃洞、上隆諸苗悉降。

天順元年,鎮守太監阮讓言:「東苗為貴州諸苗之首,負固據險,僭號稱王,逼脅他種,東苗平則諸苗服。臣會同方瑛計議,並請師期。」於是頒諭四川、湖廣諸宣慰、宣撫會師討賊。三年,督理軍務都御史白圭以谷種山箐,乃東苗羽翼,宜先剿。因同瑛進青崖,令總兵李貴進牛皮箐,參將劉玉進谷種,參將李震進鬼山。所向皆捷,克水車壩等一百十七寨。諸將復合兵青崖,攻石門山,克擺傷等三十九寨。仍分兵四路,進攻董農、竹蓋、甲底等四百三十七寨。賊首乾把豬退守六美山。合兵大進,斬五千餘級,生擒乾把豬,送京師伏誅。先是,麻城人李添保以逋賦逃入苗中,詭稱唐後,聚眾萬餘,僭稱王,建元武烈。署故賊首蒙能子聰為總兵官,遺之銀印敕書,縱兵剽掠,震動遠近。至是為李震所敗,餘賊大潰。添保僅以身免,潛入鬼池及絞洞諸寨,復煽諸苗劫攻中林、龍里,亦為震擒,伏誅。

萬歷二十八年,皮林逆苗吳國佐、石纂太等作亂。國佐本洪州司特洞寨苗,頗知書,嘗入永從學為生員,素桀黠,皮林諸苗推服之。因娶叛人吳大榮妾,為黎平府所持,遂反。自稱天皇上將,陽聽撫而陰與播賊通。纂太亦自稱太保,殺百戶黃鐘等百餘人,與國佐合兵圍上黃堡。參將黃沖霄討之,敗績。殺守備張世忠,焚五開,破永從縣,圍中潮所。總兵陳良玭、陳璘合湖、貴兵進討,亦失利,國佐益橫。二十九年命巡撫江鐸會兵分七路進剿,苗據險不出。陳璘潛師奪隘,縱火焚其巢。國佐逃,擒之,纂太亦為他將誘縛,皆伏誅。

安順,普里部蠻所居。元世祖置普定府,成宗時改普定路,又為普安路,並屬雲南。洪武初為普定府,十六年改為安順州,隸四川。正統三年改屬貴州。萬曆中改安順軍民府,以普安等州屬焉。普安,故軍民府也,初隸雲南,尋廢為衛。永樂間改為州,始隸貴州,領長官司二:曰寧谷,曰西堡。

洪武五年,普定府女總管適爾及其弟阿甕來朝,遂命適爾為知府,許世襲。六年設普定府流官二員。十四年城普定。十五年,普定軍民知府者額來朝,賜米及衣鈔,命諭其部眾,有子弟皆令入國學。十六年,者額遣弟阿昌及八十一砦長阿窩等來朝。二十年詔徵普定、安順等州六長官赴京,命以銀二十萬備糴,遣普定侯陳桓等率諸軍駐普安屯田,明年,越州叛苗阿資率眾寇普安,燒府治,大肆剽掠。征南將軍傅友德擊走之,旦詣軍門降,遂改軍民府為指揮使司。二十三年,西平侯沐英奏普安百夫長密即叛,殺屯田官軍及驛丞試百戶。調指揮張泰討之於盤江木窄關,官軍失利。更調指揮蔣文統烏撒、畢節、永寧三衛軍剿之,乃遁。二十六年,普定西堡長官司阿德及諸寨長作亂,命貴州都指揮顧成討平之。二十八年,成討平西堡土官阿傍。三十一年,西堡滄浪寨長必莫者聚眾亂,阿革傍等亦糾三千餘人助惡。成皆擊斬之,其地悉平。

永樂元年,故普安安撫者昌之子慈長言:「建文時父任是職,宜襲,吏部罷之。本境地闊民稠,輸糧三千餘石,乞仍前職報效。」命仍予安撫。十三年改普安安撫司為普安州。十四年,慈長謀占營長地,且強娶民人妻為妾,殺其夫,閹其子。事聞,命布政司孟驥按狀。慈長糾兵萬餘圍驥,驥以計擒之,逮至京,死於獄。

天順四年,西堡蠻賊聚眾焚劫,鎮守貴州內官鄭忠、右副總兵李貴請調川雲都司官兵二萬,並貴州宣慰安隴富兵二萬進剿。至阿果,擒賊首楚得隆等,斬首二百餘級。餘賊奔白石崖,復斬級七百餘,焚其巢而還。十年,安順土知州張承祖與所屬寧谷寨長官顧鐘爭地仇殺。下巡撫究治,命各貢馬贖罪。

成化十四年,貴州總兵吳經奏,西堡獅子孔洞等苗作亂,先調雲南軍八千助防守。聞雲南有警,乞改調沅州、清浪諸軍應援。十五年,經奏已擒斬賊首阿屯、堅婁等,以捷聞。

弘治十一年,普安州土判官隆暢妻米魯反。米魯者,霑益州土知州安民女也,適暢被出,居其父家。暢老,前妻子禮襲,父子不相能。米魯與營長阿保通,因令阿保諷禮迎己,禮與阿保同烝之。暢聞怒,立殺禮,毀阿保寨。阿保挾魯與其子阿鮓等攻暢,暢走雲南。時東寧伯焦俊為總兵官,與巡撫錢鉞和解之。魯於道中毒暢死,遂與保據寨反。暢妾曰適烏,生二子,別居南安。米魯欲並殺之,築寨圍其城。又別築三寨於普安,而令阿鮓等防守。名所居寨曰承天,自號無敵天王,出入建黃纛,官兵不能制,鎮巡以聞。發十衛及諸土兵萬三千人分道進,責安民殺賊自贖。民乃攻斬阿保父子於查剌寨,米魯亡走。焦俊等責安民獻魯,民陰資魯兵五百襲殺適烏及其二子,據別寨殺掠,又自請襲為女土官。鎮巡官皆受魯賂,請宥魯。嚴旨切責,必得魯乃已。貴州副使劉福陰索賂於魯,故緩師。賊益熾,官兵敗於阿馬坡,都指揮吳遠被擄,普安幾陷。帝命南京戶部尚書王軾、巡撫陳金、都指揮李政進剿,破二十餘寨。魯竄馬尾籠,官兵圍之,就擒,伏誅。安民自辨,得赦。正德元年,暢族婦適擦襲土判官,赴京朝貢,帝嘉之。或曰適擦亦暢妾云。

西堡阿得、獅子孔阿江二種,皆革僚也。初據滄浪六寨,不供常賦。土官溫愷懼罪自縊,其子廷玉請免賦,不允。往征,為其寨長乜呂等所殺。六年,廷玉弟廷瑞訴於守臣,會乜呂死,指揮楊仁撫其眾。巡撫蕭翀請令其輸賦,免用兵,從之。

都勻,元曰都雲。洪武十九年置都勻安撫司。二十九年改為軍民指揮使司,屬四川。永樂十一年改隸貴州。弘治七年置府,領州二,曰麻哈,曰獨山,即合江洲陳蒙爛土長官司地。領縣一,曰清平,即清平長官司地也。領長官司八:屬府者曰都勻,曰平浪,曰邦水,曰平州六洞;屬獨山者曰九名九姓,曰豐寧;屬麻哈者曰樂平,曰平定。洪武二十二年,都督何福奏討都勻叛苗,斬四千七百餘級,擒獲六千三百九十餘人,收降寨洞一百五十二處。二十三年城都勻衛,命指揮同知董庸守之。二十五年,九名九姓蠻亂,命何福平之。二十八年,豐寧三藍等寨亂,命顧成平之。二十九年,平浪蠻殺土官王應名,都指揮程暹平之。應名妻吳攜九歲子阿童來訴,詔予襲。永樂四年,鎮遠侯顧成招諭合江州十五寨來歸。

宣德元年,平浪賊紀那、阿魯等占副長官地,殺掠葉果諸寨,招諭不聽。詔蕭授平之。七年,陳蒙爛土副長官張勉奏,所司去衛遠,地連古州生苗,與廣西僚洞近,化從寨長韋翁同等煽亂,乞立堡,並請調泗城州土兵一千鎮守,從之。九年,翁同糾下高太刀蠻合廣西賊韋萬良等恣殺掠。指揮陳原討擒萬良等三人,翁同遂聽撫,而落昌、蔡郎等四十寨仍聚眾拒敵。總兵蕭授遣指揮顧勇進討,平之。

成化十四年,陳蒙爛土長官司張鏞奏:「夭壩乾賊首齎果侵掠,請於所侵大陳、大步等寨設一司,隸安寧宣撫。」而豐寧長官司楊泰亦奏峰峒陸光翁等聚爛土為亂。先是,宣慰楊輝平夭壩乾後,即灣溪立安寧宣撫司。爛土諸苗惡其逼己,至是果等既攻陷夭漂,遂圍豐寧。時輝已致仕,子愛承襲,力弗支,求援於川、貴二鎮。各奏聞,命仍起輝,會兵討之。十六年,鏞復奏齎果糾合九姓、豐寧並荔波賊萬人,攻剽愈亟。帝責諸守臣玩寇。於是巡撫謝杲言:「自天順四年以來,諸苗攻劫舟溪等處,不靖至今。」乃命鎮守太監張成、總兵吳經相機剿撫。二十年,爛土苗賊龍洛道潛號稱王,聲言犯都勻、清平諸衛。豐寧長官楊泰與土目楊和有隙,誘廣西泗城州農民九千,於銕坑等一百餘寨殺掠,於是苗患愈盛。弘治二年,苗賊七千人攻圍楊安堡,都指揮劉英統兵覘之,為所困。命鎮巡官往援,乃得出。五年命鎮遠侯顧溥率官兵八萬人,巡撫鄧廷瓚提督軍務,太監江德監諸軍,往征之。七年,諸軍分道進剿,令熟苗詐降於賊,誘令入寇,伏兵擒之,直搗其巢,凡破一百十餘寨,以捷聞。於是開置都勻府及獨山、麻哈二州。

正德三年,都勻長官司吳欽與其族吳敏爭襲仇殺,鎮巡以聞,言:「欽之祖賴洪武間立功為長官,陣亡。子琮幼,弟貴署之。及琮長,仍襲,傳至欽三世。敏不得以貴故妄爭。」詔可之。

嘉靖十五年,平浪叛苗王聰攻奪凱口屯,執參將李佑等。初,王阿向先世為土官,為王仲武先人所奪,至阿向,與仲武爭印煽亂。總兵楊仁、巡撫陳克宅平之,斬阿向等,盡逐其黨,以地屬都勻府,改名滅苗鎮。仲武因諸苗失業,陰為招復,旋科索之。諸苗不勝怨,遂推阿向餘孽王聰、王佑為主。巡按楊春芳遣李佑等撫諭之,賊質佑等,乞還土田官印,乃釋之三月不克,復調宣慰安萬銓兵合剿。萬銓力戰破賊,聰等皆伏誅,前後斬首二百六十餘級,降苗寨一百五十餘,男婦二萬餘口。捷聞,敘功賞賚有差。又有黑苗曰夭漂者,在湖、貴、川、廣界,與者亞鼎足居。萬歷六年,夭漂請內附。都御史遣指揮郭懷恩及長官金篆往問狀,而阻於者亞,乃遠從丹彰間道通夭漂。會苗坪、黨銀等亦以格於者亞不得通,都御史王緝遣使責者亞部長阿斗。斗願歸附平定,緝謂斗故養善牌部,何故欲屬平定,必有他謀。下吏按驗,果得實,蓋欲往平定借諸蒙兵襲養善,皆內地奸人夭金貴等導之。遂治金貴罪,以者亞仍屬養善,路遂通。於是苗坪、夭漂皆請奉貢賦,比編氓,名其地曰歸化,隸都勻府。凡使命往來,自生齒以上,悉跪拜迎送,夾騶從行,前吹蘆笙,唱蠻歌,呼導而馳。事聞,帝嘉之。七年,者亞、阿斗以反誅,乃罷樂平吏目,增設麻哈州州判一員,令居樂平司,以養鵝、者亞、羊腸諸苗屬之。

初,者亞、阿斗反,答干寨阿其應之。斗誅,阿其屢犯順。十四年,土舍吳楠、王國聘慮阿其叵測禍及己,請以答干、雞賈、甲多諸寨屬蒙詔,立宣威營,歲輸賦。獨阿其不服,引者亞殘苗圍宣威營大噪,曰:「此我地,誰令爾營此?」蒙詔常征秋稅,阿其度使至,以血釁門,令勿通。居常張傘鼓角,繪龍鳳器,遂與雞賈、甲多、仰枯諸苗擊牛酒為誓,劫歸化,官兵不敢近。獨山土吏蒙天眷願以兵進剿,乃使人佯言,漢已黜蒙詔,令以宣威營地還阿其,旦暮撤兵去矣。阿其乃親馳樂邦牛場詗視,言人人同,遂弛備。天眷驟入,斬阿其,雞賈、甲多皆降。其屬蒙詔者,自答干、雞賈、甲多外,有塘蛙、當井、斗坡等十七寨。小橋熟苗龍木恰視寨事,年老,子俸襲,頒糧者遂不及恰,恰輒奪俸之有以為養。俸訴於官,官逮問恰,非罪之也。恰輒鎖漢使,已而逐之曰:「速去,此我家事,再來我當以烏雞諸寨踐漢邊矣。」官以計擒之,死獄中。無何,龍化龍羊山苗引川苗作亂,曰:「漢無故殺苗,苗請報之。」官軍戰不利。既而都司蔡兆吉招諭令降,待以不死,於是諸苗皆散,俸視事如故。

四十三年,平州長官楊進雄兇惡,土人苦之。雄無子,以兄繼祿子珂為後,既生子治安,而疏珂。珂怨雄,雄乃奪珂財產,并其父逐之。珂頗得民心,遂為亂,據唐宿坉,攻雄。雄敗走,屠其家。各上疏訐奏,詔推問。都御史趙釴以雄不法,逮之獄,檄獨山土酋蒙繼武諭珂歸命,許改土為流以安之。治安計不便,乃陰許以六洞賂繼武借兵。繼武乃發兵攻珂,復平州,珂走廣西之泗城。繼武遂屯耕六洞地,六洞民不服,復助珂,與繼武相攻。珂復據平州。巡撫吳岳招降其父繼祿,六洞乃安。

平越,古黎峨里。元為平月長官司。洪武十四年置衛。十七年改為軍民指揮使司,屬四川。萬曆中,始置府,置貴州。領州一,曰黃平,即黃平安撫司地。領縣四:曰平越,曰湄潭,曰甕安,即甕水、草塘二長官司地;曰餘慶,即白泥、餘慶二長官司地。領長官司一,曰揚義。初,洪武八年,貴州江力、江松、剌回四十餘寨苗把具、播共桶等連結苗、僚二千作亂,平越安撫司乞兵援,命指揮同知胡汝討之。九年,黃平蠻僚都麻堰亂,宣撫司捕之,不克,千戶所以兵討之,亦敗。乃命重慶諸衛合擊,大破之,平其地。十九年,平越衛麻哈苗楊孟等作亂,命傅友德平之。時麻哈長官宋成陣歿,命其子襲。二十二年,察隴、牛場、乾溪苗亂,傅友德平之。二十三年命延安侯唐勝宗往黃平、平越、鎮遠、貴州諸處訓練軍士,提督屯田,相機剿寇。

正統末,鎮遠蠻苗金臺偽稱順天王,與播州苗相煽亂,遂圍平越、新添等衛。半年城中糧盡,官兵逃者九千餘人,貴州東路閉。時王驥征麓川,班師過其地,不之顧。景泰元年命保定伯梁珤佩平蠻將軍印督師進剿,大破之,平八十餘寨,擒賊首王阿同等,平越諸衛圍乃解。二年,都御史王來奏,貴州苗韋同烈聚眾於興隆之截洞,復攻平越、清平等衛。梁珤自沅州發兵由東路,都督方瑛由西路,合兵興隆,擊破之,同烈退保香爐山。瑛由龍場,都督陳友由萬潮山,都督毛福壽由重安江,攻破黎樹、翁滿等三百餘寨,斬三千餘級,招撫袞水等二百餘寨,合兵香爐山下。眾縛同烈降,械至京。五年,副總兵李貴奏,黎從等寨賊首阿拿、王阿傍、苗金虎等偽號苗王,與銅鼓諸賊相應,乞加兵。七年,巡撫蔣琳奏,剿苗賊於平越,斬四百餘級。其阿傍等據車碗寨,仍為亂於清平、平越地方,殺指揮王巳,據香爐山,掠偏橋。

正德十一年命巡撫秦金剿之。初,黔、楚之交,群苗嘯聚,連寨相望。而香爐山周迴四十里,高數百尋,四面徒絕,其上平衍,向為叛苗巢穴。阿傍等據之,糾諸寨苗作亂。巡撫鄒文盛、總兵官李昂等分漢、土兵為五,克其前柵。密遣人援崖先登,殺賊守路者,眾蟻附而上,焚賊巢,擒阿傍,餘賊猶堅守不下。參將洛忠等詭言招撫,自山後擊之,殲焉。遂移師龍頭、黎、蘭等寨,悉破之,賊遂平。

天啟四年,凱里土司楊世慰叛,合安邦彥兵與平茶群苗來修怨,復窺香爐山,搖動四衛,梗塞糧運。總督楊述中檄總兵魯欽馳至清平,相機進剿,調副使顏欲章等為後援。欽督將領攻破巖頭,分遣朗溪司田景祥截平茶賊援。用藥弩及炮殺傷賊眾,賊乘夜遠遁。自是不敢再窺爐山,四衛得安。

石阡,本思州地。永樂十一年置府,隸貴州,領長官司四:曰石阡,曰苗民,曰葛彰葛商,曰龍泉坪。宣德六年,葛彰葛商長官安民奏:「前以官鈔糴糧儲備,令蠻民守視。溪洞險僻,無所支用,恐歲外腐爛,賠納實難,請以充有司祭祀過使廩給之用。」縱之。萬曆中,改龍泉坪為縣。

新添衛,故麥新地也。宋時克麥新地,乃改為新添。元置新添葛蠻安撫司。洪武四年置長官司。二十三年改為衛。二十九年置新添衛軍民指揮使司,領長官司五:曰新添,曰小平伐,曰把平寨,曰丹平,曰丹行。洪武五年春,新添安撫宋亦憐真子仁來朝。其秋,平伐、蘆山、山木等砦長來降。七年,平伐、谷霞、谷浪等苗攻劫的敖諸寨,指揮僉事張岱討之。岱攻谷峽、剌向關破之,追至的敖,大破之,擒的令、的若而還,蠻大讋。

永樂二年置丹行、丹平二長官。宣德元年,新添土舍宋志道糾洞蠻肆掠,蕭授討擒之。九年,丹行土舍羅朝煽誘寨長卜長、逃民羅阿記等侵占臥龍番長官龍保地,又攻猱平寨焚劫。時苗民素憚指揮李政,尚書王驥因奏遣政往撫諭。景泰二年,苗賊有在新添行劫,聚於西廬者,官軍破之以聞。成化九年,以旱災免新添衛糧。

萬歷三十四年,貴州巡撫郭子章討平貴州苗,斬獲苗長吳老喬、阿倫、阿皆等十二人,招降男婦甚眾。先是,東西二路苗名曰仲家者,盤踞貴龍、平新之間,為諸苗渠帥。其在水硍山介於銅仁、思、石者,曰山苗,紅苗之羽翼也,窺黔自平播後財力殫竭,有輕漢心,經年剽掠無虛日。子章奏討之,命相機進兵。子章乃命總兵陳璘、參政洪澄源率官軍五千,益以土兵五千,攻水硍山。監軍布政趙健率宣慰土兵萬人,使遊擊劉岳等督之。及兩路會師,皆九十餘日而克。二寇既平,專命總兵陳璘率漢、土兵五千移營新添,進攻東路苗,不一月復克其六箐,諸苗盡平。

金築安撫司,洪武四年,故元安撫密定來朝貢馬,詔賜文綺三匹,置金築長官司,秩正六品,隸四川行省,以密定為長官,世襲。十四年敕勞密定曰:「西南諸部雖歸附,然暫入貢而已。爾密定首獻馬五百匹,以助征討,其誠可嘉,故遣特使往諭,俟班師之日,重勞爾功。」升金築長官司為安撫司,仍以密定為安撫使,予世襲。十六年,密定遣使貢方物。十八年,密定遣弟保珠來貢。二十九年以金築安撫司隸貴州軍民指揮使司。永樂初年,金築安撫得垛來朝,賜絨錦文綺。洪熙、宣德改元,皆貢馬。十年,直隸貴州布政司。正統五年,安撫金鏞貢馬。成化、弘治、隆慶時歷朝貢。萬曆四十年,吏部覆巡撫胡桂芳奏:「金築安撫土舍金大章乞改土為流,設官建治,欽定州名,鑄給印信,改州判為流官。授大章土知州,予四品服色,不許管事。子孫承襲,隸州於貴陽府。」遂改金築安撫司為廣順州。


\end{pinyinscope}