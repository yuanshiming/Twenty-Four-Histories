\article{列傳第一百三十二}

\begin{pinyinscope}
楊漣左光斗弟光先魏大中子學洢學濂周朝瑞袁化中顧大章弟大韶王之寀

楊漣,字文孺,應山人。為人磊落負奇節。萬曆三十五年成進士,除常熟知縣。舉廉吏第一,擢戶科給事中,轉兵科右給事中。

四十八年,神宗疾,不食且半月,皇太子未得見。漣偕諸給事、御史走謁大學士方從哲,御史左光斗趣從哲問安。從哲曰:「帝諱疾。即問左右,不敢傳。」漣曰:「昔文潞公問宋仁宗疾,內侍不肯言。潞公曰:『天子起居,汝曹不令宰相知,將毋有他志,速下中書行法。』公誠日三問,不必見,亦不必上知,第令宮中知廷臣在,事自濟。公更當宿閣中。」曰:「無故事。」漣曰:「潞公不訶史志聰,此何時,尚問故事耶?」越二日,從哲始率廷臣入問。及帝疾亟,太子尚躊躇宮門外。漣、光斗遣人語東宮伴讀王安:「帝疾甚,不召太子,非帝意。當力請入侍,嘗藥視膳,薄暮始還。」太子深納之。

無何,神宗崩。八月丙午朔,光宗嗣位。越四日,不豫。都人喧言鄭貴妃進美姬八人,又使中官崔文昇投以利劑,帝一晝夜三四十起。而是時,貴妃據乾清宮,與帝所寵李選侍相結,貴妃為選侍請皇后封,選侍亦請封貴妃為皇太后。帝外家王、郭二戚畹,遍謁朝士,泣朔宮禁危狀,謂:「帝疾必不起,文昇藥故也,非誤也。鄭、李交甚固,包藏禍心。」廷臣聞其語,憂甚。而帝果趣禮部封貴妃為皇太后。漣、光斗乃倡言於朝,共詰責鄭養性,令貴妃移宮,貴妃即移慈寧。漣遂劾崔文昇用藥無狀,請推問之。且曰:「外廷流言,謂陛下興居無節,侍御蠱惑。必文升藉口以掩其用藥之奸,文昇之黨煽布以預杜外廷之口。既損聖躬,又虧聖德,罪不容死。至貴妃封號,尤乖典常。尊以嫡母,若大行皇后何?尊以生母,若本生太后何?請亟寢前命。」疏上,越三日丁卯,帝召見大臣,并及漣,且宣錦衣官校。眾謂漣疏忤旨,必廷杖,囑從哲為解。從哲勸漣引罪,漣抗聲曰:「死即死耳,漣何罪?」及入,帝溫言久之,數目漣,語外廷毋信流言。遂逐文昇,停封太后命。再召大臣皆及漣。

漣自以小臣預顧命感激,誓以死報。九月乙亥朔,昧爽,帝崩。廷臣趨入,諸大臣周嘉謨、張問達、李汝華等慮皇長子無嫡母、生母,勢孤孑甚,欲共託之李選侍。漣曰:「天子寧可託婦人?且選侍昨於先帝召對群臣時,強上入,復推之出,是豈可託幼主者?請亟見儲皇,即呼萬歲,擁出乾清,暫居慈慶。」語未畢,大學士方從哲、劉一燝、韓爌至,漣趣諸大臣共趨乾清宮。閽人持梃不容入,漣大罵:「奴才!皇帝召我等。今已晏駕,若曹不聽入,欲何為!」閽人卻,乃入臨。群臣呼萬歲,請於初六日登極,而奉駕至文華殿,受群臣嵩呼。駕甫至中宮,內豎從寢閣出,大呼:「拉少主何往?主年少畏人!」有攬衣欲奪還者。漣格而訶之曰:「殿下群臣之主,四海九州莫非臣子,復畏何人!」乃擁至文華殿。禮畢,奉駕入慈慶宮。當是時,李選侍居乾清。一燝奏曰:「殿下暫居此,俟選侍出宮訖,乃歸乾清宮。」群臣遂退議登極期,語紛紛未定,有請改初三者,有請於即日午時者。漣曰:「今海宇清晏,內無嫡庶之嫌。父死之謂何?含斂未畢,袞冕臨朝,非禮也。」或言登極則人心安,漣曰:「安與不安,不在登極早暮。處之得宜,即朝委裘何害?」議定,出過文華殿。太僕少卿徐養量、御史左光斗至,責漣誤大事,唾其面曰:「事脫不濟,汝死,肉足食乎!」漣為竦然。乃與光斗從周嘉謨於朝房,言選侍無恩德,必不可同居。

明日,嘉謨、光斗各上疏請選侍移宮。初四日得俞旨。而選侍聽李進忠計,必欲皇長子同居,惡光斗疏中「武氏」語,議召皇長子,加光斗重譴。漣遇內豎於麟趾門,內豎備言狀。漣正色曰:「殿下在東宮為太子,今則為皇帝,選侍安得召?且上已十六歲,他日即不奈選侍何,若曹置身何地?」怒目視之,其人退。給事中惠世揚、御史張潑入東宮門,駭相告曰:「選侍欲垂簾處光斗,汝等何得晏然?」漣曰:「無之。」出皇極門,九卿科道議上公疏,未決。

初五日傳聞欲緩移宮期。漣及諸大臣畢集慈慶宮門外,漣語從哲趣之。從哲曰:「遲亦無害。」漣曰:「昨以皇長子就太子宮猶可,明日為天子,乃反居太子宮以避宮人乎?即兩宮聖母如在,夫死亦當從子。選侍何人,敢欺藐如此!」時中官往來如織,或言選侍亦顧命中人。漣斥之曰:「諸臣受顧命於先帝,先帝自欲先顧其子,何嘗先顧其嬖媵?請選侍於九廟前質之,若曹豈食李家祿者?能殺我則已,否則,今日不移死不去。」一燝、嘉謨助之,詞色俱厲,聲徹御前。皇長子使使宣諭,乃退。復抗疏言:「選侍陽托保護之名,陰圖專擅之實,宮必不可不移。臣言之在今日,殿下行之在今日,諸大臣贊決之,亦惟今日。」其日,選侍遂移宮,居仁壽殿。明日庚辰,熹宗即位。自光宗崩,至是凡六日。漣與一燝、嘉謨定宮府危疑,言官惟光斗助之,餘悉聽漣指。漣鬚髮盡白,帝亦數稱忠臣,未幾,遷兵科都給事中。御史馮三元等極詆熊廷弼,漣疏諭其事,獨持平。旋劾兵部尚書黃嘉善八大罪,嘉善罷去。

當選侍之移宮也,漣即言於諸大臣曰:「選侍不移宮,非所以尊天子。既移宮,又當有以安選侍。是在諸公調護,無使中官取快私仇。」既而諸奄果為流言。御史賈繼春遂上書內閣,謂不當於新君御極之初,首勸主上以違忤先帝,逼逐庶母,表裏交構,羅織不休,俾先帝玉體未寒,遂不能保一姬女。蓋是時,選侍宮奴劉遜、劉朝、田詔等以盜寶繫獄,詞連選侍父。諸奄計無所出,則妄言選侍投繯,皇八妹入井,以熒惑朝士。繼春藉其言,首發難。於是光斗上疏述移宮事。而帝降諭言選侍氣毆聖母,及要挾傳封皇后,與即日欲垂簾聽政語,又言:「今奉養李氏於噦鸞宮,尊敬不敢怠。」大學士從哲封還上諭。帝復降諭言選侍過惡,而自白贍養優厚,俾廷臣知。未幾,噦鸞宮災。帝諭內閣,言選侍暨皇八妹無恙。而是時,給事中周朝瑞謂繼春生事,繼春與相詆諆,乃復上書內閣,有:「伶仃之皇八妹,入井誰憐;孀寡之未亡人,雉經莫訴」語。朝瑞與辨駁者再。漣恐繼春說遂滋,亦上《敬述移宮始末疏》,且言:「選侍自裁,皇八妹入井,蜚語何自,臣安敢無言。臣寧使今日忤選侍,無寧使移宮不速,不幸而成女后獨覽文書、稱制垂簾之事。」帝優詔褒漣志安社稷,復降諭備述宮掖情事。繼春及其黨益忌漣,詆漣結王安,圖封拜。漣不勝憤,冬十二月抗章乞去,即出城候命。帝復褒其忠直而許之歸。天啟元年春,繼春按江西還,抵家,見帝諸諭,乃具疏陳上書之實。帝切責,罷其官。漣、繼春先後去,移宮論始息。

天啟二年起漣禮科都給事中,旋擢太常少卿。明年冬,拜左僉都御史。又明年春,進左副都御史。而是時魏忠賢已用事,群小附之,憚眾正盈朝,不敢大肆。漣益與趙南星、左光斗、魏大中輩激揚諷議,務植善類,抑憸邪。忠賢及其黨銜次骨,遂興汪文言獄,將羅織諸人。事雖獲解,然正人勢日危。其年六月,漣遂抗疏劾忠賢,列其二十四大罪,言:

高皇帝定令,內官不許干預外事,只供掖廷灑掃,違者法無赦。聖明在御,乃有肆無忌憚,濁亂朝常,如東廠太監魏忠賢者。敢列其罪狀,為陛下言之。

忠賢本市井無賴,中年凈身,夤入內地,初猶謬為小忠、小信以倖恩,繼乃敢為大奸、大惡以亂政。祖制,以擬旨專責閣臣。自忠賢擅權,多出傳奉,或徑自內批,壞祖宗二百餘年之政體,大罪一。

劉一燝、周嘉謨,顧命大臣也,忠賢令孫傑論去。急於翦己之忌,不容陛下不改父之臣,大罪二。

先帝賓天,實有隱恨,孫慎行、鄒元標以公義發憤,忠賢悉排去之。顧於黨護選侍之沈紘,曲意綢繆,終加蟒玉。親亂賊而仇忠義,大罪三。

王紀、鐘羽正先年功在國本。及紀為司寇,執法如山;羽正為司空,清修如鶴。忠賢構黨斥逐,必不容盛時有正色立朝之直臣,大罪四。

國家最重無如枚卜。忠賢一手握定,力阻首推之孫慎行、盛以弘,更為他辭以錮其出。豈真欲門生宰相乎?大罪五。

爵人於朝,莫重廷推。去歲南太宰、北少宰皆用陪推,致一時名賢不安其位。顛倒銓政,掉弄機權,大罪六。

聖政初新,正資忠直。乃滿朝薦、文震孟、熊德陽、江秉謙、徐大相、毛士龍、侯震暘等,抗論稍忤,立行貶黜,屢經恩典,竟阻賜環。長安謂天子之怒易解,忠賢之怒難調,大罪七。

然猶曰外廷臣子也。去歲南郊之日,傳聞宮中有一貴人,以德性貞靜,荷上寵注。忠賢恐其露己驕橫,托言急病,置之死地。是陛下不能保其貴幸矣,大罪八。

猶曰無名封也。裕妃以有妊傳封,中外方為慶幸。忠賢惡其不附己,矯旨勒令自盡。是陛下不能保其妃嬪矣,大罪九。

猶曰在妃嬪也。中宮有慶,已經成男,乃忽焉告殞,傳聞忠賢與奉聖夫人實有謀焉。是陛下且不能保其子矣,大罪十。

先帝青宮四十年,所與護持孤危者惟王安耳。即陛下倉卒受命,擁衛防維,安亦不可謂無勞。忠賢以私忿,矯旨殺於南苑。是不但仇王安,而實敢仇先帝之老奴,況其他內臣無罪而擅殺擅逐者,又不知幾千百也,大罪十一。

今日獎賞,明日祠額,要挾無窮,王言屢褻。近又於河間毀人居屋,起建牌坊,鏤鳳雕龍,干雲插漢,又不止塋地僭擬陵寢而已,大罪十二。

今日廕中書,明日蔭錦衣。金吾之堂口皆乳臭,誥敕之館目不識丁。如魏良弼、魏良材、魏良卿、魏希孔及其甥傅應星等,濫襲恩蔭,褻越朝常,大罪十三。

用立枷之法,戚畹家人駢首畢命,意欲誣陷國戚,動搖中宮。若非閣臣力持,言官糾正,椒房之戚,又興大獄矣,大罪十四。

良鄉生員章士魁,坐爭煤窯,託言開礦而致之死。假令盜長陵一抔土,何以處之?趙高鹿可為馬,忠賢煤可為礦,大罪十五。

王思敬等牧地細事,責在有司。忠賢乃幽置檻阱,恣意搒掠,視士命如草菅,大罪十六。

給事中周士樸執糾織監。忠賢竟停其升遷,使吏部不得專銓除,言官不敢司封駁,大罪十七。

北鎮撫劉僑不肯殺人媚人,忠賢以不善鍛煉,遂致削籍。示大明之律令可以不守,而忠賢之律令不敢不遵,大罪十八。

給事中魏大中遵旨蒞任,忽傳旨詰責。及大中回奏,臺省交章,又再褻王言。毋論玩言官於股掌,而煌煌天語,朝夕紛更,大罪十九。

東廠之設,原以緝奸。自忠賢受事,日以快私仇、行傾陷為事。縱野子傅應星、陳居恭、傅繼教輩,投匭設阱。片語稍違,駕帖立下,勢必興同文館獄而後已,大罪二十。

邊警未息,內外戒嚴,東廠訪緝何事?前奸細韓宗功潛入長安,實主忠賢司房之邸,事露始去。假令天不悔禍,宗功事成,未知九廟生靈安頓何地,大罪二十一。

祖制,不蓄內兵,原有深意。忠賢與奸相沈紘創立內操,藪匿奸宄,安知無大盜、刺客為敵國窺伺者潛入其中。一旦變生肘腋,可為深慮,大罪二十二。

忠賢進香涿州,警蹕傳呼,清塵墊道,人以為大駕出幸。及其歸也,改駕四馬,羽幢青蓋,夾護環遮,儼然乘輿矣。其間入幕效謀,叩馬獻策者,實繁有徒。忠賢此時自視為何如人哉?大罪二十三。

夫寵極則驕,恩多成怨。聞今春忠賢走馬御前,陛下射殺其馬,貸以不死。忠賢不自伏罪,進有傲色,退有怨言,朝夕隄防,介介不釋。從來亂臣賊子,只爭一念,放肆遂至不可收拾,奈何養虎兕於肘腋間乎!此又寸臠忠賢,不足盡其辜者,大罪二十四。

凡此逆跡,昭然在人耳目。乃內廷畏禍而不敢言,外廷結舌而莫敢奏。間或奸狀敗露,則又有奉聖夫人為之彌縫。甚至無恥之徒,攀附枝葉,依託門牆,更相表裏,迭為呼應。積威所劫,致掖廷之中,但知有忠賢,不知有陛下;都城之內,亦但知有忠賢,不知有陛下。即如前日,忠賢已往涿州,一切政務必星夜馳請,待其既旋,詔旨始下。天顏咫尺,忽慢至此,陛下之威靈尚尊於忠賢否邪?陛下春秋鼎盛,生殺予奪,豈不可以自主?何為受制么紵小醜,令中外大小惴惴莫必其命?伏乞大奮雷霆,集文武勳戚,敕刑部嚴訊,以正國法,并出奉聖夫人於外,用消隱憂,臣死且不朽。

忠賢初聞疏,懼甚。其黨王體乾及客氏力為保持,遂令魏廣微調旨切責漣。先是,漣疏就欲早朝面奏。值次日免朝,恐再宿機洩,遂於會極門上之,忠賢乃得為計。漣愈憤,擬對仗復劾之,忠賢詗知,遏帝不御朝者三日。及帝出,群閹數百人衷甲夾陛立,敕左班官不得奏事,漣乃止。

自是,忠賢日謀殺漣。至十月,吏部尚書趙南星既逐,廷推代者,漣注籍不與。忠賢矯旨責漣大不敬,無人臣禮,偕吏部侍郎陳于廷、僉都御史左光斗並削籍。忠賢恨不已,再興汪文言獄,將羅織殺漣。五年,其黨大理丞徐大化劾漣、光斗黨同伐異,招權納賄,命逮文言下獄鞫之。許顯純嚴鞫文言,使引漣納熊廷弼賄。文言仰天大呼曰:「世豈有貪贓楊大洪哉!」至死不承。大洪者,漣別字也。顯純乃自為獄詞,坐漣贓二萬,遂逮漣。士民數萬人擁道攀號,所歷村市,悉焚香建醮,祈祐漣生還。比下詔獄,顯純酷法拷訊,體無完膚。其年七月遂於夜中斃之,年五十四。

漣素貧,產入官不及千金。母妻止宿譙樓,二子至乞食以養。征贓令急,鄉人競出貲助之,下至賣菜傭亦為輸助。其節義感人如此。崇禎初,贈太子太保、兵部尚書,謚忠烈,官其一子。

左光斗,字遺直,桐城人。萬歷三十五年進士。除中書舍人。選授御史,巡視中城。捕治吏部豪惡吏,獲假印七十餘,假官一百餘人,輦下震悚。

出理屯田,言:「北人不知水利,一年而地荒,二年而民徙,三年而地與民盡矣。今欲使旱不為災,澇不為害,惟有興水利一法。」因條上三因十四議:曰因天之時,因地之利,因人之情;曰議濬川,議疏渠,議引流,議設壩,議建閘,議設陂,議相地,議築塘,議招徠,議擇人,議擇將,議兵屯,議力田設科,議富民拜爵。其法犁然具備,詔悉允行。水利大興,北人始知藝稻。鄒元標嘗曰:「三十年前,都人不知稻草何物,今所在皆稻,種水田利也。」閹人劉朝稱東宮令旨,索戚畹廢莊。光斗不啟封還之,曰:「尺土皆殿下有,今日安敢私受。」閹人憤而去。

光宗崩,李選侍據乾清宮,迫皇長子封皇后。光斗上言:「內廷有乾清宮,猶外廷有皇極殿,惟天子御天得居之,惟皇后配天得共居之。其他妃嬪雖以次進御,不得恒居,非但避嫌,亦以別尊卑也。選侍既非嫡母,又非生母,儼然尊居正宮,而殿下乃退處慈慶,不得守几筵,行大禮,名分謂何?選侍事先皇無脫簪戒旦之德,於殿下無拊摩養育之恩,此其人,豈可以託聖躬者?且殿下春秋十六齡矣,內輔以忠直老成,外輔以公孤卿貳,何慮乏人,尚須乳哺而襁負之哉?況睿哲初開,正宜不見可欲,何必託於婦人女子之手?及今不早斷決,將借撫養之名,行專制之實。武氏之禍再見於今,將來有不忍言者。」時選侍欲專大權,廷臣箋奏,令先進乾清,然後進慈慶。得光斗箋,大怒,將加嚴譴。數遣使宣召光斗,光斗曰:「我天子法官也,非天子召不赴。若輩何為者?」選侍益怒,邀熹宗至乾清議之。熹宗不肯往,使使取其箋視之,心以為善,趣擇日移宮,光斗乃免。當是時,宮府危疑,人情危懼,光斗與楊漣協心建議,排閹奴,扶沖主,宸極獲正,兩人力為多。由是朝野並稱為「楊左」。

未幾,御史賈繼春上書內閣,言帝不當薄待庶母。光斗聞之,即上言:「先帝宴駕,大臣從乾清宮奉皇上出居慈慶宮,臣等以為不宜避選侍。故臣於初二日具《慎守典禮肅清宮禁》一疏,宮中震怒,禍幾不測。賴皇上保全,發臣疏於內閣。初五日,閣臣具揭再催,奉旨移宮。至初六日,皇上登極,駕還乾清。宮禁肅然,內外寧謐。夫皇上既當還宮,則選侍之當移,其理明白易曉。惟是移宮以後,自宜存大體,捐小過。若復株連蔓引,使宮闈不安,即於國體有損。乞立誅盜寶宮奴劉遜等,而盡寬其餘。」帝乃宣諭百官,備述選侍凌虐聖母諸狀。及召見,又言:「朕與選侍有仇。」繼春用是得罪去。

時廷臣議改元。或議削泰昌弗紀,或議去萬曆四十八年,即以今年為泰昌,或議以明年為泰昌,後年為天啟。光斗力排其說,請從今年八月以前為萬歷,以後為泰昌,議遂定。孫如游由中旨入閣,抗疏請斥之。出督畿輔學政,力杜請寄,識鑒如神。

天啟初,廷議起用熊廷弼,罪言官魏應嘉等。光斗獨抗疏爭之,言廷弼才優而量不宏,昔以守遼則有餘,今以復遼則不足。已而廷弼竟敗。三年秋,疏請召還文震孟、滿朝薦、毛士龍、徐大相等,并乞召繼春及范濟世。濟世亦論「移宮」事與光斗異者,疏上不納。其年擢大理丞,進少卿。

明年二月拜左僉都御史。是時,韓爌、趙南星、高攀龍、楊漣、鄭三俊、李邦華、魏大中諸人咸居要地,光斗與相得,務為危言核論,甄別流品,正人咸賴之,而忌者浸不能容。光斗與給事中阮大鋮同里,招之入京,會吏科都給事中缺,當遷者,首周士樸,次大鋮,次大中。大鋮邀中旨,勒士樸不遷,以為己地。趙南星惡之,欲例轉大鋮,大鋮疑光斗發其謀,恨甚。熊明遇、徐良彥皆欲得僉都御史,而南星引光斗為之,兩人亦恨光斗。江西人又以他故銜大中,遂共嗾給事中傅櫆劾光斗、大中與汪文言比而為奸。光斗疏辨,且詆櫆結東廠理刑傅繼教為昆弟。櫆恚,再疏訐光斗。光斗乞罷,事得解。

楊漣劾魏忠賢,光斗與其謀,又與攀龍共發崔呈秀贓私,忠賢暨其黨咸怒。及忠賢逐南星、攀龍、大中,次將及漣、光斗。光斗憤甚,草奏劾忠賢及魏廣微三十二斬罪,擬十一月二日上之,先遣妻子南還。忠賢詗知,先二日假會推事與漣俱削籍。群小恨不已,復構文言獄,入光斗名,遣使往逮。父老子弟擁馬首號哭,聲震原野,緹騎亦為雪涕。至則下詔獄酷訊。許顯純誣以受楊鎬、熊廷弼賄,漣等初不承,已而恐以不承為酷刑所斃,冀下法司,得少緩死為後圖。諸人俱自誣服,光斗坐贓二萬。忠賢乃矯旨,仍令顯純五日一追比,不下法司,諸人始悔失計。容城孫奇逢者,節俠士也,與定興鹿正以光斗有德於畿輔,倡議醵金,諸生爭應之。得金數千,謀代輸,緩其獄,而光斗與漣已同日為獄卒所斃,時五年七月二十有六日也,年五十一。

光鬥既死,贓猶未竟。忠賢令撫按嚴追,繫其群從十四人。長兄光霽坐累死,母以哭子死。都御史周應秋猶以所司承追不力,疏趣之,由是諸人家族盡破。及忠賢定《三朝要典》,「移宮」一案以漣、光斗為罪魁,議開棺僇屍。有解之者,乃免。忠賢既誅,贈光斗右都御史,錄其一子。已,再贈太子少保。福王時,追謚忠毅。

弟光先,由鄉舉官御史,巡按浙江。任滿,既出境,許都反東陽。光先聞變疾返,討平之。福王既立,馬士英薦阮大鋮,光先爭不可。後大鋮得志,逮光先。亂亟道阻,光先間行走徽嶺。緹騎索不得,乃止。

魏大中,字孔時,嘉善人。自為諸生,讀書砥行,從高攀龍受業。家酷貧,意豁如也。舉於鄉,家人易新衣冠,怒而毀之。第萬曆四十四年進士,官行人。數奉使,秋毫無所擾。

天啟元年,擢工科給事中。楊鎬、李如楨既論大辟,以僉都御史王德完言,大學士韓爌遽擬旨減死。大中憤,抗疏力爭,詆德完晚節不振,盡喪典型,語并侵爌。帝為詰責大中,而德完恚甚,言曩不舉李三才為大中所怒。兩人互詆訐,疏屢上,爌亦引咎辭位。御史周宗建、徐揚先、張捷、徐景濂、溫皋謨,給事中朱欽相右德完,交章論大中,久而後定。

明年偕同官周朝瑞等兩疏劾大學士沈紘,語侵魏進忠、客氏。及議「紅丸」事,力請誅方從哲、崔文昇、李可灼,且追論鄭國泰傾害東宮罪。持議峻切,大為邪黨所仄目。太常少卿王紹徽素與東林為難,營求巡撫,大中惡其人,特疏請斥紹徽,紹徽卒自引去。再遷禮科左給事中。是時恤典冒濫,每大臣卒,其子弟夤緣要路以請,無不如志。大中素疾之,一切裁以典制。

四年遷吏科都給事中。大中居官不以家自隨,二蒼頭給爨而已,入朝則鍵其戶,寂無一人。有外吏以苞苴至,舉發之,自是無敢及大中門者。吏部尚書趙南星知其賢,事多咨訪。朝士不能得南星意,率怨大中。而是時牴排東林者多屏廢,方恨南星輩次骨。東林中,又各以地分左右。大中嘗駁蘇松巡撫王象恒恤典,山東人居言路者咸怒。及駁浙江巡撫劉一焜,江西人亦大怒。給事中章允儒,江西人也,性尤忮,嗾其同官傅櫆假汪文言發難。

文言者,歙人。初為縣吏,智巧任術,負俠氣。於玉立遣入京刺事,輸貲為監生,用計破齊、楚、浙三黨。察東宮伴讀王安賢而知書,傾心結納,與談當世流品。光、熹之際,外廷倚劉一燝,而安居中以次行諸善政,文言交關力為多。魏忠賢既殺安,府丞邵輔忠遂劾文言,褫其監生。既出都,復逮下吏,得末減。益游公卿間,輿馬嘗填溢戶外。大學士葉向高用為內閣中書,大中及韓爌、趙南星、楊漣、左光斗與往來,頗有迹。

會給事中阮大鋮與光斗、大中有隙,遂與允儒定計,囑櫆劾文言,并劾大中貌陋心險,色取行違,與光斗等交通文言,肆為奸利。疏入,忠賢大喜,立下文言詔獄。大中時方遷吏科,上疏力辯,詔許履任。御史袁化中、給事中甄淑等相繼為大中、光斗辨。大學士葉向高以舉用文言,亦引罪求罷。獄方急,御史黃尊素語鎮撫劉僑曰:「文言無足惜,不可使搢紳禍由此起。」僑頷之,獄辭無所連。文言廷杖褫職,牽及者獲免。大中乃遵旨履任。明日,鴻臚報名面恩,忠賢忽矯旨責大中互訐未竣,不得赴新任。故事,鴻臚報名狀無批諭旨者,舉朝駭愕。櫆亦言中旨不宜旁出,大中乃復視事。

未幾,楊漣疏劾忠賢,大中亦率同官上言:「從古君側之奸,非遂能禍人國也。有忠臣不惜其身以告之君,而其君不悟,乃至於不可救。今忠賢擅威福,結黨與,首殺王安以樹威於內,繼逐劉一燝、周嘉謨、王紀以樹威於外,近且斃三戚畹家人以樹威於三宮。深結保姆客氏,伺陛下起居;廣布傅應星、陳居恭、傅繼教輩,通朝中聲息。人怨於下,天怒於上,故漣不惜粉身碎首為陛下陳。今忠賢種種罪狀,陛下悉引為親裁,代之任咎。恐忠賢所以得溫旨,即出忠賢手,而漣之疏,陛下且未及省覽也。陛下貴為天子,致三宮列嬪盡寄性命於忠賢、客氏,能不寒心?陛下謂宮禁嚴密,外廷安知,枚乘有言『欲人弗知,莫若弗為』,未有為其事而他人不知者。又謂左右屏而聖躬將孤立。夫陛下一身,大小臣工所擁衛,何藉於忠賢?若忠賢、客氏一日不去,恐禁廷左右悉忠賢、客氏之人,非陛下之人,陛下真孤立於上耳。」忠賢得疏大怒,矯旨切讓,尚未有以罪也。大學士魏廣微結納忠賢,表裏為奸,大中每欲糾之。會孟冬時享,廣微偃蹇後至,大中遂抗疏劾之。廣微慍,益與忠賢合。忠賢勢益張,以廷臣交攻,陽示斂戢,且曲從諸所奏請,而陰伺其隙。迨吏部推謝應祥巡撫山西,廣微遂嗾所親陳九疇劾大中出應祥門,推舉不公,貶三秩,出之外,盡逐諸正人吏部尚書趙南星等,天下大權一歸於忠賢。

明年,逆黨梁夢環復劾文言,再下詔獄。鎮撫許顯純自削牘以上,南星、漣、光斗、大中及李若星、毛士龍、袁化中、繆昌期、鄒維璉、鄧渼、盧化鰲、錢士晉、夏之令、王之寀、徐良彥、熊明遇、周朝瑞、黃龍光、顧大章、李三才、惠世揚、施天德、黃正賓輩,無所不牽引,而以漣、光斗、大中、化中、朝瑞、大章為受楊鎬、熊廷弼賄,大中坐三千,矯旨俱逮下詔獄。鄉人聞大中逮去,號泣送者數千人。比入鎮撫司,顯純酷刑拷訊,血肉狼籍。其年七月,獄卒受指,與漣、光斗同夕斃之,故遲數日始報。大中屍潰敗,至不可識。莊烈帝嗣位,忠賢被誅,廣微、櫆、九疇、夢環並麗逆案。大中贈太常卿,謚忠節,錄其一子。

長子學洢,字子敬。為諸生,好學工文,有至性。大中被逮,學洢號慟欲隨行。大中曰:「父子俱碎,無為也。」乃微服間行,刺探起居。既抵都,邏卒四布,變姓名匿旅舍,晝伏夜出,稱貸以完父贓。贓未竟,而大中斃,學洢慟幾絕。扶櫬歸,晨夕號泣,遂病。家人以漿進,輒麾去,曰:「詔獄中,誰半夜進一漿者?」竟號泣死。崇禎初,有司以狀聞,詔旌為孝子。

次子學濂,有盛名。舉崇禎十六年進士。擢庶吉士。明年,李自成逼京師,與同官吳爾壎慷慨有所論建,大學士范景文以聞。莊烈帝特召見兩人,將任用之。無何,京師陷,不能死,受賊戶部司務職,頹其家聲。既而自慚,賦絕命詞二章,縊死。去帝殉社稷時四十日矣。

文言之再下詔獄也,顯純迫令引漣等。文言備受五毒,不承,顯純乃手作文言供狀。文言垂死,張目大呼曰:「爾莫妄書,異時吾當與面質。」顯純遂即日斃之。漣、大中等逮至,無可質者,贓懸坐而已。諸所誣趙南星、繆昌期輩,亦並令撫按追贓。衣冠之禍,由此遍天下。始熊廷弼論死久,帝以孫承宗請,有詔待以不死。刑部尚書喬允升等遂欲因朝審寬其罪,大中力持不可。及忠賢殺大中,乃坐以納廷弼賄云。

周朝瑞,字思永,臨清人。萬曆三十五年進士。授中書舍人。

光宗嗣位,擢吏科給事中,疏請收錄先朝遺直。俄陳慎初三要,曰信仁賢,廣德澤,遠邪佞。因請留上供金花銀,以佐軍興。詞多斥中貴。中貴皆惡之,激帝怒,貶秩調外,時列諫垣甫四日也。未出都而熹宗立,詔復故官。疏請容納直言,又陳考選諸弊。日講將舉,進君臣交警之規。帝並褒納。賈繼春之請安李選侍也,朝瑞力駁之,與繼春往復者數四。

天啟元年再遷禮科左給事中。時遼事方棘,朝瑞請於閣臣中推通曉兵事者二人專司其事,而以職方郎一人專理機宜,給事中二人專主封駁,帝可之。雄縣知縣王納諫為閹人所誣,中旨鐫秩。給事中毛士龍以糾駁閹人,為府丞邵輔忠所陷,中旨除名。朝瑞並抗疏論列。十二月辛巳,日上有一物覆壓,忽大風揚沙,天盡赤,都人駭愕,所司不以聞。朝瑞請帝修省,而嚴敕內外臣工,毋鬥爭誤國,更詰責所司不奏報之罪,帝納之。時帝踐祚歲餘,未嘗親政,權多旁落,朝瑞請帝躬覽萬機。帝降旨,言政委閣臣,祖宗舊制不可紊,然其時政權故不在閣也。

明年二月,廣寧失,詔停經筵日講。朝瑞等上言:「此果出聖意,輔臣當引義爭。如輔臣阿中涓意,則其過滋大。且主上沖齡,志意未定,獨賴朝講不輟,諸臣得一覲天顏,共白指鹿之奸。今常朝已漸傳免,倘併講筵廢之,九閽既隔,無謁見時,司馬門之報格不入,呂大防之貶不及知,國家大事去矣。」會禮部亦以為言,乃命日講如故。

已,偕諸給事御史惠世揚、左光斗等極論大學士沈紘結中官練兵,為肘腋之賊。紘疏辨。朝瑞等盡發其賄交魏進忠、盧受、劉朝、客氏,而末復侵其私人邵輔忠、徐大化。語過激,奪疏首世揚俸。大化嘗承要人指,力攻熊廷弼,朝瑞惡之。無何,王化貞棄廣寧逃,大化又請立誅廷弼。朝瑞以廷弼才可用,請令帶罪守山海,疏四上,並抑不行。大化遂力詆朝瑞,朝瑞憤,亦醜詆大化,所司為兩解之。朝瑞方擢太僕少卿,而大化為魏忠賢腹心,必欲殺朝瑞,竄其名汪文言獄中,與楊漣等五人並逮下鎮撫獄,坐妄議「移宮」及受廷弼賄萬金。五日再訊,搒掠備至,竟斃之獄。崇禎初,贈大理卿,予一子官。福王時,謚忠毅。

袁化中,字民諧,武定人。萬曆三十五年進士。歷知內黃、涇陽,有善政。

泰昌元年擢御史。時熹宗沖齡踐阼,上無母后,宮府危疑。化中上疏劾輔臣方從哲,報聞。天啟元年二月,疏陳時事可憂者八:曰宮禁漸弛,曰言路漸輕,曰法紀漸替,曰賄賂漸章,曰邊疆漸壞,曰職掌漸失,曰宦官漸盛,曰人心漸離。語皆剴切。出按宣、大,以憂歸。服除,起掌河南道。

楊漣劾魏忠賢,化中亦率同官上疏曰:「忠賢障日蔽月,逞威作福,視大臣如奴隸,斥言官若孤雛,殺內廷外廷如草菅。朝野共危,神人胥憤,特陛下未之知,故忠賢猶有畏心。今漣已侃詞入告矣,陛下念潛邸微勞,或貸忠賢以不死。而忠賢實自懼一死,懼死之念深,將挺而走險,騎虎難下,臣恐其橫逞之毒不在搢紳,而即在陛下。陛下試思,深宮之內,可使多疑多懼之人日侍左右,而不為防制哉?」疏入,忠賢大恨。

錦衣陳居恭者,忠賢爪牙也,為漣所論及,亦攻忠賢自解。化中特疏劾之,落其職。毛文龍獻俘十二人,而稚兒童女居其八。化中力請釋之,因言文龍敘功之濫。忠賢素庇文龍,益不悅。崔呈秀按淮、揚,贓私狼籍,回道考核,化中據實上之,崔呈秀大恨。會謝應祥廷推被訐,化中與其事,呈秀遂嗾忠賢貶化中秩,調之外。已,竄入汪文言獄詞中,逮下詔獄。呈秀令許顯純坐以楊鎬、熊廷弼賄六千,酷刑拷掠,於獄中斃之。崇禎初,贈太僕卿,官其一子。福王時,追謚忠愍。

顧大章,字伯欽,常熟人。父雲程,南京太常卿。大章與弟大韶,孿生子也。大章舉萬曆三十五年進士,授泉州推官,乞改常州教授。父喪除,值朝中朋黨角立,正士日摧。大章慨然曰:「昔賈彪不入『顧』『廚』之目,卒西行以解其難。餘向與東林疏,可以彪自況也。」乃入都,補國子博士。與朝士通往來,陰察其交關肯綮,清流賴之。

稍遷刑部主事。以奉使歸。還朝,天啟已改元,進員外郎。尚書王紀令署山東司事。司轄輦轂,最難任。自遼陽失,五城及京營巡捕日以邏奸細為事,稍有蹤迹,率論死。絕無左驗者二百餘人,所司莫敢讞,多徙官去,囚未死者僅四之一。大章言於紀曰:「以一身易五十人命且甘之,矧一官乎!」即日會讞,繫三人,餘悉移大理釋放。紀大嗟服。佟卜年之獄,紀用大章言擬流卜年,未上而紀斥。侍郎楊東明署事,欲置之大辟,大章力爭,卒擬流。忤旨,詰責,竟論卜年辟,瘐死獄中。

魏忠賢欲借劉一獻株累劉一燝,大章力辨其非,忠賢大恨。卜年、一獻事具《紀》、《一燝傳》中。熊廷弼、王化貞之下吏也,法司諸屬二十八人共讞,多有議寬廷弼者。大章因援「議能」、「議勞」例,言化貞宜誅,廷弼宜論戍。然二人卒坐死。大章亦遷兵部去,無異議也。會王紀劾罷徐大化,又疏刺客氏,其黨疑紀疏出大章手,恨之。大化令所親御史楊維垣訐大章妄倡「八議」,鬻大獄,大章疏辨。維垣四疏力攻,言納廷弼賄四萬,且列其鬻獄數事,反覆詆訐不休。大章危甚,賴座主葉向高保持之,下所司驗問,都御史孫瑋等白其誣。帝以大章瀆辨,稍奪其俸,大章遂引歸。

五年起官。歷禮部郎中,陜西副使。大化已起大理丞,與維垣為忠賢鷹犬,因假汪文言獄連及大章,逮下鎮撫拷掠,坐贓四萬。及楊漣等五人既死,群小聚謀,謂諸人潛斃於獄,無以厭人心,宜付法司定罪,明詔天下。乃移大章刑部獄,由是漣等慘死狀外人始聞。比對簿,大章詞氣不撓。刑部尚書李養正等一如鎮撫原詞,以「移宮」事牽合封疆,坐六人大辟。爰書既上,忠賢大喜,矯詔布告四方,仍移大章鎮撫。大章慨然曰:「吾安可再入此獄!」呼酒與大韶訣,趣和藥飲之,不死,投繯而卒。崇禎初,贈太僕卿,官其一子。福王時,追謚裕愍。

初,大章等被逮,祕獄中忽生黃芝,光彩遠映。及六人畢入,適成六瓣,或以為祥。大章嘆曰:「芝,瑞物也,而辱於此,吾輩其有幸乎?」已而果然。

大韶,字仲恭,老於諸生。通經史百家及內典,於《詩》、《禮》、《儀禮》、《周官》多所發明,他辨駁者復數萬言。嘗以為宋、元以來述者之事備,學者但當誦而不述,將死,始繕所箋《詩》、《禮》、《莊子》,曰《炳燭齋隨筆》云。

王之寀,字心一,朝邑人。萬曆二十九年進士。除清苑知縣,遷刑部主事。

四十三年五月初四日酉刻,有不知姓名男子,持棗木梃入慈慶宮門,擊傷守門內侍李鑑。至前殿簷下,為內侍韓本用等所執,付東華門守衛指揮朱雄等收之。慈慶宮者,皇太子所居宮也。明日,皇太子奏聞,帝命法司按問。巡皇城御史劉廷元鞫奏:「犯名張差,薊州人。止稱吃齋討封,語無倫次。按其迹,若涉瘋癲,稽其貌,實係黠猾。請下法司嚴訊。」時東宮雖久定,帝待之薄。中外疑鄭貴妃與其弟國泰謀危太子,顧未得事端,而方從哲輩亦頗關通戚畹以自固。差被執,舉朝驚駭,廷元以瘋癲奏。刑部山東司郎中胡士相偕員外郎趙會楨、勞永嘉共訊,一如廷元指。言:「差積柴草,為人所燒,氣憤發癲。於四月內訴冤入京,遇不知名男子二人,紿令執梃作冤狀。乃由東華門入,直至慈慶宮門。按律當斬,加等立決。」稿定未上。山東司主治京師事,署印侍郎張問達以屬之。而士相、永嘉與廷元皆浙人,士相又廷元姻也,瘋癲具獄,之寀心疑其非。

是月十一日,之寀值提牢散飯獄中,末至差,私詰其實。初言「告狀」,復言「涼死罷,已無用」。之寀令置飯差前:「吐實與飯,否則餓死。」麾左右出,留二吏扶問之。始言:「小名張五兒。有馬三舅、李外父令隨不知姓名一老公,說事成與汝地幾畝。比至京,入不知街道大宅子。一老公飯我云:『汝先衝一遭,遇人輒打死,死了我們救汝。』畀我棗木棍,導我由後宰門直至宮門上,擊門者墮地。老公多,遂被執。」之寀備揭其語,因問達以聞。且言差不癲不狂,有心有膽。乞縛兇犯於文華殿前朝審,或敕九卿科道三法司會問。疏入未下,大理丞王士昌、行人司正陸大受、戶部主事張庭、給事中姚永濟等連上疏趣之。而大受疏有「奸戚」二字,帝惡之,與之寀疏俱不報。廷元復請速檢諸疏,下法司訊斷。御史過庭訓言禍生肘腋,宜亟翦,亦俱不報。庭訓遂移文薊州蹤跡之。知州戚延齡具言其致癲始末,言:「貴妃遣璫建佛寺,璫置陶造甓,居民多鬻薪獲利者。差賣田貿薪往市於璫,土人忌之,焚其薪。差訟於璫,為所責,不勝憤,持梃欲告御狀。」於是原問諸臣據為口實矣。

二十一日,刑部會十三司司官胡士相、陸夢龍、鄒紹光、曾曰唯、趙會禎、勞永嘉、王之寀、吳養源、曾之可、柯文、羅光鼎、曾道唯、劉繼禮、吳孟登、岳駿聲、唐嗣美、馬德灃、朱瑞鳳等再審。差供:「馬三舅名三道,李外父名守才,不知姓名老公乃修鐵瓦殿之龐保,不知街道宅子乃住朝外大宅之劉成。二人令我打上宮門,打得小爺,吃有,著有。」小爺者,內監所稱皇太子者也。又言:「有姊夫孔道同謀,凡五人。」於是刑部行薊州道,提馬三道等,疏請法司提龐保、劉成對鞫,而給事中何士晉與從哲等亦俱以為言。帝乃諭究主使,會法司擬罪。是日,刑部據薊州回文以上。已,復諭嚴刑鞫審,速正典刑。時中外籍籍,語多侵國泰,國泰出揭自白。士晉復疏攻國泰,語具《士晉傳》。

先是,百戶王曰乾上變,言奸人孔學等為巫蠱,將不利於皇太子,詞已連劉成。成與保皆貴妃宮中內侍也。至是,復涉成。帝心動,諭貴妃善為計。貴妃窘,乞哀皇太子,自明無它;帝亦數慰諭,俾太子白之廷臣。太子亦以事連貴妃,大懼,乃緣帝及貴妃意,期速結。二十八日,帝親御慈寧宮,皇太子侍御座右,三皇孫鴈行立左階下。召大學士方從哲、吳道南暨文武諸臣入,責以離間父子,諭令磔張差、龐保、劉成,無他及。因執太子手曰:「此兒極孝,我極愛惜。」既又手約太子體,諭曰:「自襁褓養成丈夫,使我有別意,何不早更置?且福王已之國,去此數千里,自非宣召,能翼而至乎?」因命內侍引三皇孫至石級上,令諸臣熟視,曰:「朕諸孫俱長成,更何說?」顧問皇太子有何語,與諸臣悉言無隱。皇太子具言:「瘋癲之人宜速決,毋株連。」又責諸臣云:「我父子何等親愛,而外廷議論紛如,爾等為無君之臣,使我為不孝之子。」帝又謂諸臣曰:「爾等聽皇太子語否?」復連聲重申之。諸臣跪聽,叩頭出,遂命法司決差。明日磔於市。又明日,司禮監會廷臣鞫保、成於文華門。時已無左證,保、成展轉不承。會太子傳諭輕擬,廷臣乃散去。越十餘日,刑部議流馬三道、李守才、孔道。帝從之,而斃保、成於內廷。其事遂止。

當是時,帝不見群臣二十有五年矣,以之寀發保、成事,特一出以釋群臣疑,且調劑貴妃、太子。念其事似有跡,故不遽罪之寀也。四十五年京察,給事中徐紹吉、御史韓浚用拾遺劾之寀貪,遂削其籍。

天啟初,廷臣多為之訟冤,召復故官。二年二月上《復仇疏》,曰:

《禮》,君父之仇,不共戴天。齊襄公復九世之仇,《春秋》大之。曩李選侍氣毆聖母,陛下再三播告中外,停其貴妃之封,聖母在天之靈必有心安而目瞑者。此復仇一大義也。

乃先帝一生遭逢多難,彌留之際,飲恨以崩。試問:李可灼之誤用藥,引進者誰?崔文升之故用藥,主使者誰?恐方從哲之罪不在可灼、文升下。此先帝大仇未復者,一也。

張差持梃犯宮,安危止在呼吸。此乾坤何等時,乃劉廷元曲蓋奸謀,以瘋癲具獄矣。胡士相等改注口語,以賣薪成招矣。其後復讞,差供同謀舉事,內外設伏多人。守才、三道亦供結黨連謀,而士相輩悉抹去之。當時有內應,有外援。一夫作難,九廟震驚,何物兇徒,敢肆行不道乃爾!緣外戚鄭國泰私結劉廷元、劉光復、姚宗文輩,珠玉金錢充滿其室。言官結舌,莫敢誰何,遂無復顧憚,睥睨神器耳。國泰雖死,罪不容誅。法當開棺戮屍,夷其族,赭其宮,而至今猶未議及。此先帝大仇未復者,二也。

總之,用藥之術,即梃擊之謀。擊不中而促之藥,是文升之藥慘於張差之梃也。張差之前,從無張差;劉成之後,豈乏劉成?臣見陛下之孤立於上矣。

又言:

郎中胡士相等,主瘋癲者也。堂官張問達,調停瘋癲者也。寺臣王士昌疏忠而心佞,評無隻字,訟多溢詞。堂官張問達語轉而意圓,先允瘋癲,後寬奸宄。勞永嘉、岳駿聲等同惡相濟。張差招有「三十六頭兒」,則胡士相閣筆;招有「東邊一起幹事」,則岳駿聲言波及無辜;招有「紅封票,高真人」,則勞永嘉言不及究紅封教。今高一奎見監薊州,係鎮朔衛人。蓋高一奎,主持紅封教者也;馬三道,管給紅票者也;龐保、劉成,供給紅封教多人撒棍者也。諸奸增減會審公單,大逆不道。

疏入,帝不問,而先主瘋癲者恨次骨。

未幾,之寀遷尚寶少卿。踰年,遷太僕少卿,尋轉本寺卿。廷元及岳駿聲、曾道唯以之寀侵己,先後疏辨。之寀亦連疏力折,并發諸人前議差獄時,分金紅廟中,及居間主名甚悉。事雖不行,諸人益疾之。

四年秋,拜刑部右侍郎。明年二月,魏忠賢勢大張,其黨楊維垣首翻「梃擊」之案,力詆之寀,坐除名。俄入之汪文言獄中,下撫按提問。岳駿聲復訐之,且言其逼取鄭國泰二萬金,有詔追治。及修《三朝要典》,其「梃擊」事以之寀為罪首。府尹劉志選復重劾之,遂逮下詔獄,坐贓八千,之寀竟瘐死。崇禎初,復官,賜恤。

自「梃擊」之議起,而「紅丸」、「移宮」二事繼之。兩黨是非爭勝,禍患相尋,迄明亡而後已。

贊曰:國之將亡也,先自戕其善類,而水旱盜賊乘之。故禍亂之端,士君子恒先被其毒。異哉,明之所稱「三案」者!舉朝士大夫喋喋不去口,而元惡大憝因用以剪除善類,卒致楊、左諸人身填牢戶,與東漢季年若蹈一轍。國安得不亡乎!


\end{pinyinscope}