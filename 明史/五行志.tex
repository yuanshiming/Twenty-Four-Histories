\article{五行志}

\begin{pinyinscope}
史志五行,始自《漢書》,詳錄五行傳說及其占應。後代作史者因之。粵稽《洪範》,首敘五行,以其為天地萬物之所莫能外。而合諸人道,則有五事,稽諸天道,則有庶徵。天人相感,以類而應者,固不得謂理之所無。而傳說則條分縷析,以某異為某事之應,更旁引曲證,以伸其說。故雖父子師弟,不能無所抵牾,則果有當於敘疇之意歟。夫茍知天人之應捷於影響,庶幾一言一動皆有所警惕。以此垂戒,意非不善。然天道遠,人道邇,逐事而比之,必有驗有不驗。至有不驗,則見以為無徵而怠焉。前賢之論此悉矣。孔子作《春秋》,紀異而說不書。彼劉、董諸儒之學,頗近於術數禨祥,本無足述。班氏創立此志,不得不詳其學之本原。而歷代之史,往往取前人數見之說,備列簡端。揆之義法,未知所處。故考次洪武以來,略依舊史五行之例,著其祥異,而事應暨舊說之前見者,並削而不載云。

《洪範》曰「水曰潤下」。水不潤下,則失其性矣。前史多以恆寒、恒陰、雪霜、冰雹、雷震、魚孽、蝗蝻、豕禍、龍蛇之孽、馬異、人痾、疾疫、鼓妖、隕石、水潦、水變、黑眚黑祥皆屬之水,今從之。

▲恒寒

景泰四年冬十一月戊辰至明年孟春,山東、河南、浙江、直隸、淮、徐大雪數尺,淮東之海冰四十餘里,人畜凍死萬計。五年正月,江南諸府大雪連四旬,蘇、常凍餓死者無算。是春,羅山大寒,竹樹魚蚌皆死。衡州雨雪連綿,傷人甚多,牛畜凍死三萬六千蹄。成化十三年四月壬戌,開原大雨雪,畜多凍死。十六年七八月,越巂雨雪交作,寒氣若冬。弘治六年十一月,鄖陽大雪,至十二月壬戌夜,雷電大作,明日復震,後五日雪止,平地三尺餘,人畜多凍死。正德元年四月,雲南武定隕霜殺麥,寒如冬。萬曆五年六月,蘇、松連雨,寒如冬,傷稼。四十六年四月辛亥,陜西大雨雪,橐駝凍死二千蹄。

▲恒陰

洪武十八年二月,久陰。正統五年七月戊午、己未及癸亥,曉刻陰沉,四方濃霧不辨人。八年,邳、海二州陰霧彌月,夏麥多損。景泰六年正月癸酉,陰霧四塞,既而成霜附木,凡五日。八年正月甲子,陰晦大霧,咫尺不辨人物。成化四年三月,昏霧蔽天,不見星日者累晝夜。九年三月甲午,四月丁卯,山東黑暗如夜。二十年五月丙申,番禺天晦,良久乃復。二十三年十二月辛卯,大霧不辨人。弘治十五年十一月,景東晝晦者七日。十六年四月辛亥,甘肅昏霧障天,咫尺不辨人物。十八年秋,廣昌大雨霧凡兩月,民病且死者相繼。正德十年四月,巨野陰霧六日,殺穀。十四年三月戊午,陰晦。嘉靖元年正月丁卯,日午,昏霧四塞。三年,江北昏霧,其氣如藥。天啟六年六月丙戌,霧重如雨。閏六月己未,如之。

▲雨雪隕霜

洪武十四年五月丁未,建德雪。六月己卯,杭州晴日飛雪。二十六年四月丙申,榆社隕霜損麥。景泰四年,鳳陽八衛二三月雨雪不止,傷麥。天順四年三月乙酉,大雪,越月乃止。成化二年四月乙巳,宣府隕霜殺青苗。十九年三月辛酉,陜西隕霜。弘治六年十月,南京雨雪連旬。八年四月庚申,榆社、陵川、襄垣、長子、沁源隕霜殺麥豆桑。辛酉,慶陽諸府縣衛所三十五,隕霜殺麥豆禾苗。九年四月辛巳,榆次隕霜殺禾。是月,武鄉亦隕霜。十七年二月壬寅,鄖陽、均州雨雪雹,雪片大者六寸。六月癸亥,雨雪。正德八年四月乙巳,文登、萊陽隕霜殺稼。丙辰,殺穀。十三年三月壬戌,遼東隕霜,禾苗皆死。嘉靖二年三月甲子,郯城隕霜殺麥。辛未,殺禾。二十二年四月己亥,固原隕霜殺麥。隆慶六年三月丁亥,南宮隕霜殺麥。萬曆二十四年四月己亥,林縣雪。二十六年十一月辛亥,彰德隕霜,不殺草。三十八年四月壬寅,貴州暴雪,形如土磚,民居片瓦無存者。四十四年正月,雨紅黃黑三色雪,屋上多巨人跡。崇禎六年正月辛亥,大雪,深二丈餘。十一年五月戊寅,喜峰口雪三尺。十三年四月,會寧隕霜殺稼。十六年四月,鄢陵隕霜殺麥。

▲冰雹

洪武二年六月庚寅,慶陽大雨雹,傷禾苗。三年五月丙辰,蔚州大雨雹,傷田苗。五年五月癸丑夜,中都皇城萬歲山雨冰雹,大如彈丸。七年八月甲午,平涼,延安綏德、米脂雨雹。九月甲子,鞏昌雨雹。八年四月,臨洮、平涼、河州雹傷麥。十四年七月己酉,臨洮大雨雹,傷稼。十八年二月,雨雹。

永樂七年秋,保定、浙東雨雹。十二年四月,河南一州八縣雨雹,殺麥。

正統三年,西、延、平、慶、臨、鞏六府及秦、河、岷、金四州,自夏逮秋,大雨雹。四年五月壬戌,京師大雨雹。五年四月丁酉,平涼諸府大雨雹,傷人畜田禾。六月壬申至丙子,山西行都司及蔚州連日雨雹,其深尺餘,傷稼。八月庚辰,保定大雨雹,深尺餘,傷稼。

景泰五年六月庚寅,易州大方等社雨雹甚大,傷稼百二十五里,人馬多擊死。六年閏六月乙巳,束鹿雨雹如雞子,擊死鳥雀狐兔無算。

天順元年六月己亥,雨雹大如雞卵,至地經時不化,奉天門東吻牌摧毀。八年五月丁巳,雨雹。

成化元年四月庚寅,雨雹大如卵,損禾稼。五月辛酉,又大雨雹。五年閏二月癸未,瓊山雨雹大如斗。八年七月丙午,隴州雨雹大如鵝卵,或如雞子,中有如牛者五,長七八寸,厚三四寸,六日乃消。九年五月丁巳,雨雹如拳。十三年春,湖廣大雨冰雹,牛死無算。十九年六月乙亥,潞州雨雹,大者如碗。二十年二月丙子,清遠雨雹,大如拳。丙戌,大雷電,復雨雹。二十一年三月己丑夜,番禺、南海風雷大作,飛雹交下,壞民居萬餘,死者千餘人。二十二年三月甲寅,南陽雨雹,大如鵝卵。

弘治元年三月壬申夜,融縣雨雹,壞城樓垣及軍民屋舍,死者四人。二年三月戊寅,賓州雨雹如雞子,擊殺牧豎三人,壞廬舍禾稼。庚辰,貴州安莊衛大雷,雨雪雹,壞麥苗。四月辛卯,洮州衛雨冰雹,水湧三丈。四年三月癸卯,裕、汝二州雨雹,大者如牆杵,積厚二三尺,壞屋宇禾稼。四月己酉,洮州衛雨雹及冰塊。水高三四丈,漫城郭,漂房舍,田苗人畜多淹死。五年四月乙丑,莒、沂二州,安丘、郯城二縣,雨雹大如酒盃,傷人畜禾稼。六年八月己巳,長子雨雹,大者如拳,傷禾稼,人有擊死者。辛未,雨雹,大如彈丸,平地壅積。八年二月壬申,永嘉暴風雨,雨雹,大如雞卵,小如彈丸,積地尺餘,白霧四起,毀屋殺黍,禽鳥多死。三月己亥,桐城雨雹,深五尺,殺二麥。己酉,淮、鳳州縣暴風雨雹,殺麥。四月乙亥,常州、泗、邳雨雹,深五寸,殺麥及菜。丙子,沂州雨雹,大者如盤,小者如碗,人畜多擊死。六月乙卯,雨雹。七月乙酉,洮州衛雨冰雹,殺禾。暴水至,人畜多溺死者。丙戌,甘肅西寧大雨雹,殺禾及畜。九年五月丙辰,雨雹。十年二月己卯,江西新城雨冰雹,民有凍死者。三月丁卯,北通州雨冰,深一尺。十三年八月戊子,雨雹。丙午,又雨雹。九月壬戌,又雨雹。十四年四月丁酉,徐州、清河、桃源、宿遷雨冰雹,平地五寸,夏麥盡爛。五月乙亥,登、萊二府雨雹殺禾。七月辛卯,雨雹。

正德元年六月戊辰,宣府馬營堡大雨雹,深二尺,禾稼盡傷。三年四月辛未,涇州雨雹,大如雞卵,壞廬舍菽麥。四年五月甲午,費縣大雨雹,深一尺,壞麥穀。八年十月戊戌,平陽、太原、沁、汾諸屬邑,大雨雹,平地水深丈餘,衝毀人畜廬舍。十一年六月甲戌,宣府大雨雹,禾稼盡死。九月丙申,貴州大雨雹。十二年五月己亥,安肅大雨雹,平地水深三尺,傷禾,民有擊死者。十三年四月壬午,衡州疾風迅雷,雨雹,大如鵝子,棱利如刀,碎屋,斷樹木如剪。

嘉靖元年四月甲申,雲南左衛各屬雨雹,大如雞子,禾苗房屋被傷者無算。五月己未,蓬溪雨雹,大如鵝子,傷亦如之。二年五月丁丑,大同前衛雨雹。四年四月丁未,大同衛雨雹。五月戊子,固安雨雹。五年五月甲辰,滿城雨雹。六月丁巳,大同縣雨冰雹,俱大如雞子。丁卯,萬全都司及宣府皆雨雹,大者如甌,深尺餘。七月癸未,南豐雨雹,大如碗,形如人面。遂昌雨雹,頃刻二尺,大殺麻豆。六年六月癸丑,鎮番衛大雨雹,殺傷三十餘人。十四年三月辛巳,漢中雨雹隕霜殺麥。四月庚子,開封、彰德雨雹殺麥。十八年五月壬辰,慶都、安肅、河間雨冰雹,大如拳,平地五寸,人有死傷者。二十八年三月庚寅,臨清大冰雹,損房舍禾苗。六月丁卯,延川雨雹如斗,壞廬舍,傷人畜。三十四年五月庚子,鳳陽大冰雹,壞民田舍。三十六年三月癸未,沂州雨雹,大如盂,小如雞卵,平地尺餘,徑八十里,人畜傷損無算。四十三年閏二月甲申,雨雹。四月庚寅,又雨雹。

隆慶元年七月辛巳,紫荊關雨雹,殺稼七十里。三年三月辛未,平溪衛雨雹。平地水湧三尺,漂沒廬舍。四月己丑,鄖陽縣雨雹。平地水深二尺。五月癸丑,延綏口北馬營堡雨雹,殺稼七十里。四年四月辛酉,宣府、大同雨雹,厚三尺餘,大如卵,禾苗盡傷。五年四月戊午,大雨雹。六年八月乙丑,祁、定二州大雨雹,傷損禾菽,擊斃三人。

萬曆元年五月辛巳,雨雹。四年四月丙午,博興大雨雹,如拳如卵,明日又如之,擊死男婦五十餘人,牛馬無算,禾麥毀盡。兗州相繼損禾。五月乙巳,定襄雨雹,大者如卵,禾苗盡損。九年八月庚子,遼東等衛雨雹,如雞卵,禾盡傷。十一年閏二月丁卯,泰州、寶應雨雹如雞子,殺飛鳥無算。五月庚子,大雨雹。十三年五月乙酉,宛平大雨雹,傷人畜千計。十五年五月癸巳,喜峰口大雨雹,如棗栗,積尺餘,田禾瓜果盡傷。十九年四月壬子,雨雹。二十一年二月庚寅,貴陽府大雨雹。十月丙戌,武進、江陰大冰雹,傷五穀。二十三年五月乙酉,臨邑雨雹,盡作男女鳥獸形。二十五年八月壬戌,風雹。二十八年六月,山東大風雹,擊死人畜,傷禾苗。河南亦雨冰雹,傷禾麥。三十年四月己未,大雨雹。三十一年五月戊寅,鳳陽皇陵雨雹。七月丁丑,大雨雹。三十四年七月丙戌,又大雨雹。平地水深三尺。三十六年五月戊子,雨雹。四十一年七月丁卯,宣府大雨雹,殺禾稼。四十六年三月庚辰,長泰、同安大雨雹,如斗如拳,擊傷城郭廬舍,壓死者二百二十餘人。十月壬午,雲南雨雹。

天啟二年四月壬辰,大雨雹。

崇禎三年九月辛丑,大雨雹。四年五月,襄垣雨雹,大如伏牛盈丈,小如拳,斃人畜甚眾。六月丙申,大雨雹。七年四月壬戌,常州、鎮江雨雹,傷麥。八年七月己酉,臨縣大冰雹三日,積二尺餘,大如鵝卵,傷稼。十年四月乙亥,大雨雹。閏四月癸丑,武鄉、沁源大雨雹,最大者如象,次如牛。十一年六月甲寅,宣府乾石河山場雨雹,擊殺馬四十八匹。九月,順天雨雹。十二年八月,白水、同官、雒南、隴西諸邑,千里雨雹,半日乃止,損傷田禾。十六年六月丁丑,乾州雨雹,大如牛,小如斗,毀傷牆屋,擊斃人畜。

▲雷震

洪武六年十一月戊申,雷電交作。十三年五月甲午,雷震謹身殿。六月丙寅,雷震奉天門。十月甲戌,雷電。十二月己巳,廣州大風雨雷電。十八年二月甲午,雷電雨雪。二十一年五月辛丑,雷震玄武門獸吻。六月癸卯,暴風,雷震洪武門獸吻。

宣德九年六月甲子,雷震大祀壇外西門獸吻。

正統八年五月戊寅,雷震奉天殿鴟吻。七月辛未,雷震南京西角門樓獸吻。是日,大同巡警軍至沙溝,風雷驟至,裂膚斷指者二百餘人。九年正月辛亥朔,雷電大雨。閏七月壬寅,雷震奉先殿鴟吻。十一年十二月壬寅,大雨雷電,翼日乃止。十四年六月丙辰,南京風雨雷電,謹身殿災。

景泰三年六月庚寅,雷擊宮庭中門,傷人。

天順二年六月己卯,雷震大祀殿鴟吻。四年六月癸丑,雷毀薊州倉廒四。

成化三年六月戊申,雷震南京午門正樓。五年二月乙卯,又震山川壇具服殿之獸吻。八年四月辛未,始雷。十二年十一月癸亥,南京大雷雨。十三年二月甲戌,安慶大雪,既而雷電交作。十一月辛未冬至,杭州大雷雨。戊寅,荊門州大雷電雨雪。十七年七月己亥,雷震郊壇承天門脊獸。十一月丁酉,江南大雷雨雪。

弘治元年五月丙子,辰刻,南京震雷壞洪武門獸吻。巳刻,壞孝陵御道樹。六月己酉,又壞鷹揚衛倉樓,聚寶門旗桿。二年四月庚子,又毀神樂觀祖師殿。三年七月壬子,又壞午門西城牆。六年閏五月丁未,薊州大風雷,拔木偃禾,牛馬有震死者。十二月壬戌,南京雷雨,拔孝陵樹。七年六月癸酉,如之。七月丙辰,福州雷毀城樓。八年十二月丙子,長沙大雷電雨雪。丁丑,南昌、彭水俱大雷電,雨雪雹,大木折。十年四月,雷震宣府西橫嶺之南山,傾三十餘丈。七月乙卯,雷擊吉王府端禮門獸吻。十二年四月丙午,雷震楚府承運殿。十四年閏七月庚辰,福州大風雷,擊壞教場旗桿、城樓、大樹。

正德元年五月壬辰,雷震青州衣甲庫獸吻,有火起庫中。六月辛酉,雷擊西中門柱脊,暴風折郊壇松柏,大祀殿及齋宮獸瓦多墮落者。丙子,南京暴風雨,雷震孝陵白土岡樹。十二月己巳朔,南通州雷再震。四年十二月壬寅,杭州大雨雷電,越二日復作。五年六月丙申,雷震萬全衛柴溝堡,斃墩軍四人。七年五月戊辰,雷震餘干萬春寨旗桿,狀如刀劈。閏五月丁亥,雷震成都衛門及教場旗桿。十年閏四月甲申,薊州賺狗崖、東墩及新開嶺關雷火,震傷三十餘人。十二年八月癸亥,南京祭歷代帝王,雷雨大作,震死齋房吏。十二月庚辰,瑞州大雷電。十六年八月,雷擊奉天門。

嘉靖二年五月丁丑,雷擊觀象臺。四年七月己丑,雷擊南京長安左門獸吻。五年四月戊寅,雷擊阜城門城樓南角獸吻及北九鋪旗桿。十年六月丁巳,雷擊德勝門,破民屋柱,斃者四人。癸亥,雷擊午門角樓及西華門城樓柱。十五年六月甲申,雷擊南京西上門獸吻,震死男婦十餘人。十六年五月戊戌,雷震謹身殿鴟吻。二十八年六月丁酉朔,雷震奉先殿左吻及東室門槅。三十三年四月乙亥,始雷。三十八年六月丙寅,雷擊奉先殿門外南西二牆。

隆慶元年八月,大暑雷震。次日,大寒,如嚴冬。是夕,雷震達旦。四年六月辛酉,雷擊圜丘廣利門鴟吻。

萬曆三年六月己卯,雷擊建極殿鴟吻。壬辰,雷擊端門鴟尾。六年七月壬子,雷擊南京承天門左簷。十三年七月戊子,雷震郊壇廣利門,震傷榜題「利」字及齋宮北門獸吻。十六年八月壬午,雷震南京舊西安門鐘鼓樓獸頭。十九年五月甲戌,太平路、喜峰路並雷擊,墩臺折,傷官軍。二十一年四月戊戌,雷震孝陵大木。二十二年六月己酉,雷雨,西華門災。七月壬辰,雷擊祈穀壇東天門左吻。二十四年二月己酉夜,酃縣大雷雨,火光遍十餘里。二十五年七月庚寅朔,雷毀黃花鎮臺垣及火器。三十二年五月癸酉,雷毀長陵樓,又毀薊鎮松棚路墩臺。三十三年五月庚子,大雷電,擊毀南郊望燈高桿。三十七年八月甲寅,雷劈西城上旗桿。

泰昌元年十月己未,雷毀淮安城樓。

崇禎六年十二月丁亥,大風雪,雷電。九年正月甲戌,雷毀孝陵樹。十年四月乙亥,薊州雷火焚東山二十餘里。十二年七月,雷擊破密雲城鋪樓,所貯砲木皆碎。十月乙未立冬,雷電大作。十四年四月癸丑,雷火起薊州西北,焚及趙家谷,延二十餘里。六月丙午,雷震宣府西門城樓。十五年四月癸卯,雷震南京孝陵樹,火從樹出。十六年五月癸巳朔,雷震通夕不止。次日,見太廟神主橫倒,諸銅器為火所鑠,熔而成灰。六月丙戌,雷震奉先殿鴟吻,槅扇皆裂,銅鐶盡毀。

▲魚孽

嘉靖四十一年二月乙亥,德州九龍廟雨魚,大者數寸。崇禎十年三月,錢塘江木鏚化為魚,有首尾未變者。

▲蝗蝻

洪武五年六月,濟南屬縣及青、萊二府蝗。七月,徐州、大同蝗。六年七月,北平、河南、山西、山東蝗。七年二月,平陽、太原、汾州、歷城、汲縣蝗。六月,懷慶、真定、保定、河間、順德、山東、山西蝗。八年夏,北平、真定、大名、彰德諸府屬縣蝗。建文四年夏,京師飛蝗蔽天,旬餘不息。永樂元年夏,山東、山西、河南蝗。三年五月,延安、濟南蝗。十四年七月,畿內、河南、山東蝗。宣德四年六月,順天州縣蝗。九年七月,兩畿、山西、山東、河南蝗蝻覆地尺許,傷稼。十年四月,兩京、山東、河南蝗蝻傷稼。正統二年四月,北畿、山東、河南蝗。五年夏,順天、河間、真定、順德、廣平、應天、鳳陽、淮安、開封、彰德、兗州蝗。六年夏,順天、保定、真定、河間、順德、廣平、大名、淮安、鳳陽蝗。秋,彰德、衛輝、開封、南陽、懷慶、太原、濟南、東昌、青、萊、兗、登諸府及遼東廣寧前、中屯二衛蝗。七年五月,順天、廣平、大名、河間、鳳陽、開封、懷慶、河南蝗。八年夏,兩畿蝗。十二年夏,保定、淮安、濟南、開封、河南、彰德蝗。秋,永平、鳳陽蝗。十三年七月,飛蝗蔽天。十四年夏,順天、永平、濟南、青州蝗。景泰五年六月,寧國、安慶、池州蝗。七年五月,畿內蝗蝻延蔓。六月,淮安、揚州、鳳陽大旱蝗。九月,應天及太平七府蝗。天順元年七月,濟南、杭州、嘉興蝗。二年四月,濟南、兗州、青州蝗。成化三年七月,開封、彰德、衛輝蝗。九年六月,河間蝗。七月,真定蝗。八月,山東旱蝗。十九年五月,河南蝗。二十二年三月,平陽蝗。四月,河南蝗。七月,順天蝗。弘治三年,北畿蝗。四年夏,淮安、揚州蝗。六年六月,飛蝗自東南向西北,日為掩者三日。七年三月,兩畿蝗。嘉靖三年六月,順天、保定、河間、徐州蝗。隆慶三閏六月,山東旱蝗。萬曆十五年七月,江北蝗。十九年夏,順德、廣平、大名蝗。三十七年九月,北畿、徐州、山東蝗。四十三年七月,山東旱蝗。四十四年四月,復蝗。七月,常州、鎮江、淮安、揚州、河南蝗。九月,江寧、廣德蝗蝻大起,禾黍竹樹俱盡。四十五年,北畿旱蝗。四十六年,畿南四府又蝗。四十七年八月,濟南、東昌、登州蝗。天啟元年七月,順天蝗。五年六月,濟南飛蝗蔽天,田禾俱盡。六年十月,開封旱蝗。崇禎八年七月,河南蝗。十年六月,山東、河南蝗。十一年六月,兩京、山東、河南大旱蝗。十三年五月,兩京、山東、河南、山西、陜西大旱蝗。十四年六月,兩京、山東、河南、浙江大旱蝗。

▲豕禍

嘉靖七年,杭州民家有豕,肉膜間生字。萬曆二十三年春,三河民家生八豕,一類人形,手足俱備,額上一目。三十八年四月,燕河路營生豕,一身二頭,六蹄二尾。六月,大同後衛生豕,兩頭四眼四耳。四十七年六月,黃縣生豕,雙頭四耳,一身八足。七月,寧遠生豕,身白無毛,長鼻象嘴。天啟三年七月,辰州玩平溪生豕,豬身人足,一目。四年三月,神木生豕,額多一鼻逆生,目深藏皮肉,合則不見。四月,榆林生豕,一首二身,二尾八足。六月,霍州生豕,二身二眼,象鼻,四耳四乳。崇禎元年三月,石泉生豕類象,鼻下一目甚大,身無毛,皮肉皆白。六年二月,建昌生豕,二身一首,八蹄二尾。十五年七月,聊城生豕,一首二尾七蹄。

▲龍蛇之孽

成化五年六月,河決杏花營,有卵浮於河,大如人首,下銳上圓,質青白,蓋龍卵也。弘治九年六月庚辰,宣府鎮南口墩驟雨火發,龍起刀鞘內。十八年五月辛卯,日午,旋風大起,雲翳三殿,若有人騎龍入雲者。正德七年六月丁卯夜,招遠有赤龍懸空,光如火,盤旋而上,天鼓隨鳴。十二年六月癸亥,山陽見黑龍,一龍吸水,聲聞數里,攝舟及舟女至空而墜。十三年五月癸丑,常熟俞野村迅雷震電,有白龍一、黑龍二乘雲並下,口中吐火,目睛若炬,撤去民居三百餘家,吸二十餘舟於空中。舟人墜地,多怖死者。是夜紅雨如注,五日乃息。十四年四月,鄱陽湖蛟龍鬥。嘉靖四十年五月癸酉,青浦佘山九蛟並起,湧水成河。萬曆十四年七月戊申,舒城大雷雨,起蛟百五十八,跡如斧劈,山崩田陷,民溺死無算。是歲,建昌民樵於山,逢巨蛇,一角,六足如雞距,不噬不驚,或言此肥遺也。十八年七月,猗氏大水,二龍鬥於村,得遺卵,尋失。十九年六月己未,公安大水,有巨蛇如牛,首赤身黑,修二丈餘,所至隄潰。三十一年五月戊戌,歷城大雨,二龍鬥水中,山石皆飛,平地水高十丈。四十五年八月,安丘青河村青白二龍鬥。

▲馬異

永樂十八年九月,諸城進龍馬。民有牝馬牧於海濱,一日雲霧晦冥,有物蜿蜒與馬接。產駒,具龍文,其色青蒼,謂之龍馬云。宣德七年五月,忻州民武煥家馬生一駒,鹿耳牛尾,玉面瓊蹄,肉文被體如鱗。七月,滄州畜官馬,一產二駒,州以為祥,獻於朝。宣宗曰:「物理之常,何足異也。」

成化十七年六月,興濟馬生二駒。弘治元年二月,景寧屏風山有異物成群,大如羊,狀如白馬,數以萬計。首尾相銜,迤邐騰空而去。嘉靖四十二年四月,海鹽有海馬萬數,岸行二十餘里。其一最巨,高如樓。

▲人痾

前史多志一產三男事,然近歲多有,不可勝詳也,其稍異者志之。洪武二十四年八月,河南龍門婦司牡丹死三年,借袁馬頭之尸復生。宣德元年十一月,行在錦衣衛校尉綦榮妻皮氏一產四子。天順四年四月,揚州民婦一產五男。成化十三年二月,南京鷹揚衛軍陳僧兒妻朱氏一產三男、一女。十七年六月,宿州民張珍妻王氏臍下右側裂,生一子。二十年十二月,徐州婦人肋下生瘤,久之漸大,兒從瘤出。二十一年,嘉善民鄒亮妻初乳生三子,再乳生四子,三乳生六子。弘治十一年六月,騰驤左衛百戶黃盛妻宜氏一產三男一女。十六年五月,應山民張本華妻崔氏生鬚長三寸。是時,鄭陽商婦生鬚三繚,約百餘莖。嘉靖二年六月,曲靖衛舍人胡晟妻生一男,兩頭四手三足。四年,橫涇農孔方協下產肉塊,剖視之,一兒宛然。五年,江南民婦生妖,六目四面,有角,手足各一節,獨爪,鬼聲。十一年,當塗民婦一產三男一女。十二年,貴州安衛軍李華妻生男,兩頭四手四足。二十七年七月,大同右衛參將馬繼舍人馬錄女,年十七化為男子。隆慶二年十二月,靜樂男子李良雨化為婦人。五年二月,唐山民婦生兒從左脅出。萬曆十年,淅川人化為狼。十八年,南宿州民婦一產七子,膚髮紅白黑青各色。三十七年六月,繁峙民李宜妻牛氏一產二女,頭面相連,手足各分。四十六年,廣寧衛民婦產一猴,二角四齒。是時,大同民婦一產四男。崇禎八年夏,鎮江民婦產一子,頂載兩首,臀贅一首,與母俱斃。十五年十一月,曹縣民婦產兒,兩頭,頂上有眼,手過膝。

▲疾疫

永樂六年正月,江西建昌、撫州,福建建寧、邵武自去年至是月,疫死者七萬八千四百餘人。八年,登州寧海諸州縣自正月至六月,疫死者六千餘人。邵武比歲大疫,至是年冬,死絕者萬二千戶。九年七月,河南、陜西疫。十一年六月,湖州三縣疫。七月,寧波五縣疫。正統九年冬,紹興、寧波、台州瘟疫大作,及明年,死者三萬餘人。景泰四年冬,建昌、武昌、漢陽疫。六年四月,西安、平涼疫。七年五月,桂林疫死者二萬餘人。天順五年四月,陜西疫。成化十一年八月,福建大疫,延及江西,死者無算。正德元年六月,湖廣平溪、清涼、鎮遠、偏橋四衛大疫,死者甚眾。靖州諸處自七月至十二月大疫,建寧、邵武自八月始亦大疫。十二年十月,泉州大疫。嘉靖元年二月,陜西大疫。二年七月,南京大疫,軍民死者甚眾。四年九月,山東疫死者四千一百二十八人。三十三年四月,都城內外大疫。四十四年正月,京師饑且疫。萬曆十年四月,京師疫。十五年五月,又疫。十六年五月,山東、陜西、山西、浙江俱大旱疫。崇禎十六年,京師大疫,自二月至九月止。明年春,北畿、山東疫。

▲鼓妖

洪武五年八月己酉,徐溝西北空中有聲如雷。十一年,瑞昌有大聲如鐘,自天而下,無形。天順六年九月乙巳夜,天無雲,西北方有聲如雷。七年二月晦夜,空中有聲。大學士李賢奏,無形有聲謂之鼓妖,上不恤民則有此異。成化十三年正月甲子,代州無雲而雷。十四年八月戊戌,早朝,東班官若聞有甲兵聲者,辟易不成列,久之始定。弘治六年六月丁卯,石州吳城驛無雲而震者再。十七年六月甲申,江西廬山鳴如雷。嘉靖二十九年二月甲子,隆慶州張山營堡山鳴。萬曆十二年十二月己未,蕭縣山鳴如驚濤澎湃,竟夜不止。二十八年八月戊戌,西北方有聲如雷。天啟七年八月丁巳,莊烈即位,朝時,空中有聲如天鼓,發於殿西。崇禎十二年十二月乙未,蕭縣山鳴。是月,西山大鳴如雷,如風濤。十三年二月壬子,浙江省城門夜鳴。十六年冬,建極殿鴟吻中有聲似鵓鳩,曰「苦苦」,其聲漸大,復作犬吠聲,三日夜不止。明年三月辛丑,孝陵夜有哭聲,亦鼓妖也。

▲隕石

成化六年六月壬申,陽信雷聲如嘯,隕石一,碎為三,外黑內青。十四年六月辛亥,臨晉天鳴,隕石縣東南三十里,入地三尺,大如升,色黑。二十三年五月壬寅,束鹿空中響如雷,青氣墜地。掘之得黑石二,一如碗,一如雞卵。弘治三年三月,慶陽雨石無數,大小不一,大者如鵝卵,小者如芡實。四年十月丁巳,光山有紅光如電,自西南往東北,聲如鼓,久之入地,化為石,大如斗。十年二月丙申,修武黑氣入地,化為石,狀如羊首。十二年五月戊寅,朔州有聲,如迅雷,白氣騰上,隕大石三。正德元年八月壬戌,夜有火光落即墨,化為綠石,圓高尺餘。九年五月己卯,濱州有聲隕石。十三年正月己未,鄰水隕石一。嘉靖十二年五月丁未,祁縣有聲如鼓,火流墜地為石。四十二年三月癸卯,懷慶隕石。隆慶二年三月己未,保定新城隕黑石二。萬曆三年五月癸亥,有二流星晝隕景州城北,化為黑石。十七年九月戊午,萬載黑煙騰起,隕石演武廳畔。十九年四月辛酉,遵化隕石二。四十四年正月丁丑,易州及紫荊關有光化石崩裂。崇禎九年九月丁未,太康隕石。

▲水潦

洪武元年六月戊辰,江西永新州大風雨,蛟出,江水入城,高八尺,人多溺死。事聞,使賑之。三年六月,溧水縣江溢,漂民居。四年七月,南寧府江溢,壞城垣。衢州府龍游縣大雨,水漂民廬,男女溺死。五年八月,嵊縣、義烏、餘杭山谷水湧,人民溺死者眾。六年二月,崇明縣為潮所沒。七月,嘉定府龍游縣洋、雅二江漲,翼日南溪縣江漲,俱漂公廨民居。七年八月,高密縣膠河溢,傷禾。八年七月,淮安、北平、河南、山東大水。十二月,直隸蘇州、湖州、嘉興、松江、常州、太平、寧國,浙江杭州俱水。九年,江南、湖北大水。七月,湖廣、山東大水。十年六月,永平灤、漆二水沒民廬舍。七月,北平八府大水,壞城垣。十一年七月,蘇、松、揚、台四府海溢,人多溺死。十月丙辰,河決蘭陽。十二年五月,青田山水沒縣治。十三年十一月,崇明潮決沙岸,人畜多溺死。十四年八月庚辰,河決原武。十五年二月壬子,河南河決。三月庚午,河決朝邑。七月,河溢滎澤、陽武。是歲,北平大水。十七年八月丙寅,河決開封,橫流數十里。是歲,河南、北平俱水。十八年八月,河南又水。是年,江浦、大名水。二十三年正月庚寅,河決歸德。七月癸巳,河決開封,漂沒民居。又海門縣風潮壞官民廬舍,漂溺者眾。是歲,襄陽、沔陽、安陽水。二十四年十月,北平、河間二府水。二十五年正月,河決陽武,開封州縣十一俱水。二十六年十一月,青、兗、濟寧三府水。二十七年三月,寧陽汶河決。二十八年八月,德州大水,壞城垣。三十年八月丁亥,河決開封,三面皆水,犯倉庫。

永樂元年五月,章丘漯河決岸、傷稼。南海、番禺潮溢。八月,安丘縣紅河決。二年六月,蘇、松、嘉、湖四府俱水。七月,湖廣、江西水。九月,河決開封,壞城。三年三月,溫縣水決隄四十餘丈。濟、澇二水溢。八月,杭州屬縣多水,淹男婦四百餘人。七年五月,安陸州江溢,決渲馬灘圩岸千六百餘丈。六月,壽州水決城。是歲,泰興江岸淪於江者三千九百餘丈。渾河決固安。八年五月,平度州濰水及浮糠河決,浸百十三所。七月,平陽縣潮溢,漂廬舍。八月庚申,河溢開封。十二月戊戌,河決汴梁,壞城。九年正月,高郵甓社等九湖及天長諸水暴漲。六月,揚州屬州縣五江潮漲四日,漂人畜甚眾。七月,海寧潮溢,漂溺甚眾。八月,漳、衛二水決隄淹田。九月,雷州颶風暴雨,淹遂溪、海康,壞田禾八百餘頃,溺死千六百餘人。是歲,湖廣、河南水。十年七月,廬溝水漲,壞橋及隄岸,溺死人畜。保定縣決河岸五十四處。十一月,吳橋、東光、興濟、交河、天津決隄傷稼。十二月,安州水決直亭等河口八十九處。十二年十月,臨晉涑河逆流,決姚暹渠堰,流入硝池,淹沒民田,將及鹽池。崇明潮暴至,漂廬舍五千八百餘家。十三年六月,北畿、河南、山東水溢,壞廬舍,沒田禾,臨清尤甚。滏、漳二水漂磁州民舍。十四年夏,南昌諸府江漲,壞民廬舍。七月,開封州縣十四河決隄岸。永平灤、漆二河溢,壞民田禾。福寧、延平、邵武、廣信、饒州、衢州、金華七府,俱溪水暴漲,壞城垣房舍,溺死人畜甚眾。遼東遼河、代子河水溢,浸沒城垣屯堡。十八年夏秋,仁和、海寧潮湧,堤淪入海者千五百餘丈。二十年五月,廣東諸府潮溢,漂廬舍,壞倉糧,溺死三百六十餘人。夏秋,湖廣沔陽江漲,河南北及鳳陽河溢。二十一年五月,峨眉溪水漲,溺死百三十人。八月,瓊州府潮溢,漂溺甚眾。二十二年七月,黃巖潮溢,溺死八百人。九月庚辰,河溢開封。

洪熙元年六月,驟雨,白河溢,衝決河西務、白浮、宋家等口堤岸。臨漳漳、滏二河決堤岸二十四。真定滹沱河大溢,沒三州五縣田。七月,容城白溝河漲,傷禾稼。渾河決廬溝橋東狼窩口,順天、河間、保定、灤州俱水。

宣德元年六七月,江水大漲,襄陽、穀城、均州、鄖縣,緣江民居漂沒者半。黃、汝二水溢,淹開封十州縣及南陽汝州、河南嵩縣。三年五月,邵陽、武岡、湘鄉暴風雨七晝夜,山水驟長,平地高六尺。永寧衛大水,壞城四百丈。六月,渾河水溢,決廬溝河堤百餘丈。七月,北畿七府俱水。五年七月,南陽山水泛漲,衝決堤岸,漂流人畜廬舍。六年六月,渾河溢,決徐家等口,順天、保定、真定、河間州縣二十九俱水。河決開封,沒八縣。七年六月,太原河、汾並溢,傷稼。八年六月,江西瀕江八府江漲,漂沒民田,溺死男婦無算。九年正月,沁鄉沁水漲,決馬曲灣,經獲嘉、新鄉,平地成河。五月,寧海縣潮決,徙地百七十餘頃。六月,渾河決東岸,自狼河口至小屯廠,順天、順德、河間俱水。七月,遼東大水。

正統元年閏六月,順天、真定、保定、濟南、開封、彰德六府俱大水。二年,鳳陽、淮安、揚州諸府,徐、和、滁諸州,河南開封,四五月河、淮泛漲,漂居民禾稼。九月,河決陽武、原武、滎澤。湖廣沿江六縣大水決江堤。三年,陽武河決,武陟沁決,廣平、順德漳決,通州白河溢。四年五月,京師大水,壞官舍民居三千三百九十區。順天、真定、保定三府州縣及開封、衛輝、彰德三府俱大水。七月,滹沱、沁、漳三水俱決,壞饒陽、獻縣、衛輝、彰德堤岸。八月,白溝、渾河二水溢,決保定安州堤。蘇、常、鎮三府俱決,款饒陽、獻縣、衛輝、彰德堤岸。九月,滹沱復決深州,淹百餘里。五年五月至七月,江西江溢,河南河溢。八月,潮決蕭山海塘。六年五月,泗州水溢丈餘,漂廬舍。七月,白河決武清、淳阜縣堤二十二處。八月,寧夏久雨,水泛,壞屯堡墩臺甚眾。八年六月,渾河決固安。八月,台州、松門、海門海潮泛溢,壞城郭、官亭、民舍、軍器。九年七月,揚子江沙洲潮水溢漲,高丈五六尺,溺男女千餘人。閏七月,北畿七府及應天、濟南、岳州、嘉興、湖州、台州俱大水。河南山水灌衛河,沒衛輝、開封、懷慶、彰德民舍,壞衛所城。十年三月,洪洞汾水堤決,移置普潤驛以遠其害。夏,福建大水,壞延平府衛城,沒三縣田禾民舍,人畜漂流無算。河南州縣多大水。七月,延安衛大水,壞護城河堤。九月,廣東衛所多大水。十月,河決山東金龍口陽穀堤。十一年六月,渾河溢固安。兩畿、浙江、河南俱連月大雨水。是歲,太原、兗州、武昌亦俱大水。十二年春,贛州、臨江大水。五月,吉安江漲淹田。十三年六月,大名河決,淹三百餘里,壞廬舍二萬區,死者千餘人。河南、濟南、青、兗、東昌亦俱河決。七月,寧夏大水。河決漢、唐二壩。河南八樹口決,漫曹、濮二州,抵東昌,壞沙灣等堤。十四年四月,吉安、南昌臨江俱水,壞壇廟廨舍。

景泰元年七月,應天大水,沒民廬。三年六月,河決沙灣白馬頭七十餘丈。八月,徐州、濟寧間,平地水高一丈,民居盡圮。南畿、河南、山東、陜西、吉安、袁州俱大水。四年春夏,河連決沙灣。五年六月,揚州潮決高郵、寶應堤岸。七月,蘇、松、淮、揚、廬、鳳六府大水。八月,東、兗、濟三府大水,河漲淹田。六年六月,開封、保定俱大水。閏六月,順天大水,灤河泛溢,壞城垣民舍,河間、永平水患尤甚。武昌諸府江溢傷稼。七年六月,河決開封,河南、彰德田廬淹沒。是歲,畿內、山東俱水。

天順元年夏,淮安、徐州、懷慶、衛輝俱大水,河決。三年六月,穀城、景陵襄水湧泛傷稼。四年夏,湖北江漲,淹沒麥禾。北畿及開封、汝寧大水。七月,淮水決,沒軍民田廬。五年七月,河決開封土城,築磚城禦之。越三日,磚城亦潰,水深丈餘。周王後宮及官民乘筏以避,城中死者無算。襄城水決城門,溺死甚眾。崇明、嘉定、崑山、上海海潮衝決,溺死萬二千五百餘人。浙江亦大水。六年七月,淮安大水,潮溢,溺死鹽丁千三百餘人。七年七月,密雲山水驟漲,軍器、文卷、房屋俱沒。

成化三年六月,江夏水決江口堤岸,迄漢陽,長八百五十丈有奇。五年,湖廣大水。山西汾水傷稼。六年六月,北畿大水。七年閏九月,山東及浙江杭、嘉、湖、紹四府俱海溢,淹田宅人畜無算。九年六月,畿南五府及懷慶俱大水。八月,山東大水。十一年五月,湖廣水。十二年八月,浙江風潮大水。淮、鳳、揚、徐亦俱大水。十三年二月甲戌,安慶大雪。次日大雨,江水暴漲。閏二月,河南大水。九月,淮水溢,壞淮安州縣官舍民屋,淹沒人畜甚眾。十四年四月,襄陽江溢,壞城郭。五月,陜州大水,人多淹死。七月,北畿、山東水。九月,河決開封護城堤五十丈。十八年七月,昌平大水,決居庸關水門四十九,城垣、鋪樓、墩臺一百二。八月,衛、漳、滹沱並溢,自清平抵天津。

弘治二年五月,河決開封黃沙岡抵紅船灣,凡六處,入沁河。所經州縣多災,省城尤甚。七月,順、永、河、保四府州縣大水。八月,盧溝河堤壞。四年八月,蘇、松、浙江水。五年夏秋,南畿、浙江、山東水。七年七月,蘇、常、鎮三府潮溢,平地水五尺,沿江者一丈,民多溺死。九年六月,山陰、蕭山山崩水湧,溺死三百餘人。十四年五月,貴池水漲,蛟出,淹死二百六十餘人,旁邑十二皆大水。七月,廉州及靈山海漲,淹死百五十餘人。閏七月,瓊山颶風潮溢,平地水高七尺。八月,安、寧、池、太四府大水,蛟出,漂流房屋。十五年七月,南京江水泛溢,湖水入城五尺餘。十七年六月,廬山平地水丈餘,溺死星子、德安民,及漂沒廬舍甚眾。

正德元年六月,陜西徽州河溢,漂沒居民孳畜。二年六月,固原河漲,平地水高四尺,人畜溺死。三年九月,延綏、慶陽大水。五年九月,安、寧、太三府大水,溺死二萬三千餘人。十一月,蘇、松、常三府水。六年六月,汜水暴漲,溺死百七十六人,毀城垣百七十餘堵。十二年,順天、河間、保定、真定大水。鳳陽、淮安、蘇、松、常、鎮、嘉、湖諸府皆大水。荊、襄江水大漲。十五年五月,江西大水。十六年七月,遼陽湯跕堡大水決城。

嘉靖元年七月,南京暴風雨,江水湧溢,郊社、陵寢、宮闕、城垣吻脊欄楯皆壞。拔樹萬餘株,江船漂沒甚眾。廬、鳳、淮、揚四府同日大風雨雹,河水泛漲,溺死人畜無算。二年七月,揚、徐復大水。夏、秋間,山東州縣俱大水。八月,蘇、松、常、鎮四府大水,開封亦如之。五年六月,陜西五郎壩大水三丈餘,衝決官舍。徐、沛河溢,壞豐縣城。六年秋,湖廣水。十六年秋,兩畿、山東、河南、陜西、浙江各被水災,湖廣尤甚。二十六年七月丙辰,曹縣河決,城池漂沒,溺死者甚眾。二十七年正月,水幵陽大水沒城。

隆慶元年夏,京師大水。六月,新河占魚口沉運船數百艘。是歲,襄陽、鄖陽水。二年七月,台州颶風,海潮大漲,挾天台山諸水入城,三日溺死三萬餘人,沒田十五萬畝,壞廬舍五萬區。三年閏六月,真定、保定、淮安、濟南、浙江、江南俱大水。七月壬午,河決沛縣,自考城、虞城、曹、單、豐、沛至徐州,壞田廬無算。九月,淮水溢,自清河至通濟閘及淮安城西,淤三十里,決二壩入海。莒、沂、郯城之水又溢出邳州,溺人民甚眾。四年七月,沙、薛、汶、泗諸水驟溢,決仲家淺等漕堤。八月,陜西大水,河決邳州。五年四月,又決邳州,自曲頭集至王家口新堤多壞。是歲,山東、河南大水。

萬曆元年七月,荊州、承天大水。二年六月,福建永定大水,溺七百餘人。是歲,海鹽海大溢,死者數千人。八月庚午,淮安、揚州、徐州河溢傷稼。三年四月,淮、徐大水。五月,淮水大決。六月,杭、嘉、寧、紹四府海湧數丈,沒戰船、廬舍、人畜不計其數。八月,淮、揚、鳳、徐四府州大水,河決高郵、碭山及邵家口、曹家莊。九月,蘇、松、常、鎮四府俱水。四年正月,高郵清水堤決。九月,河決豐、沛、曹、單。十一月,淮、黃交溢。五年閏八月,徐州河淤,淮河南徙,決高郵、寶應諸湖堤。六年六月,清河水溢。七年五月,蘇、松、鳳陽、徐州大水。八月,又水。是歲,浙江大水。九年五月,從化、增城、龍門溪壑泛漲,田禾盡沒,淹死男婦無算。七月,福安洪水踰城,漂沒廬舍殆盡。八月,泰興、海門、如皋大水,塘圩坡埂盡決,溺死者甚眾。十年正月,淮、揚海漲,浸豐利等鹽場三十,淹死二千六百餘人。七月,蘇、松六州縣潮溢,壞田禾十萬頃,溺死者二萬人。十一年四月,承天江水暴漲,漂沒民廬人畜無算。金州河溢沒城。十四年夏,江南、浙江、江西、湖廣、廣東、福建、雲南、遼東大水。十五年五月,浙江大水。七月,開封及陜州、靈寶河決。是歲,杭、嘉、湖、應天、太平五府江湖泛溢,平地水深丈餘。七月終,颶風大作,環數百里,一望成湖。十六年八月,河決東光魏家口。十七年六月,浙江海沸,杭、嘉、寧、紹、台屬縣廨宇多圮,碎官民船及戰舸,壓溺者三百餘人。十九年六月,蘇、松大水,溺人數萬。七月,寧、紹、蘇、松、常五府濱海潮溢,傷稼淹人。九月,泗州大水,州治浸三尺。淮水高於城,祖陵被浸。十月,揚州湖淮漲溢,決邵伯堤五十餘丈,高郵南北閘俱衝。二十年夏秋,真、順、廣、大四府水。二十一年五月,邳州、高郵、寶應大水決湖堤。二十二年七月,鳳陽、廬州大水。二十三年四月,泗水浸祖陵。二十四年秋,杭、嘉、湖三府大水。二十九年八月,沔陽大水入城。三十年六月,京師大水。三十一年五月,成安、永年、肥鄉、安州、深澤,漳、滏、沙、燕河並溢,決堤橫流。祁州、靜海圮城垣、廬舍殆盡。六月,泰安大水,淹八百餘人。八月,泉州諸府海水暴漲,溺死萬餘人。三十二年六月,昌平大水,壞各陵橋道。七月,永平、真、保三府俱水,淹男婦無算。八月,河決蘇家莊,淹豐、沛,黃水逆流灌濟寧、魚臺、單縣。三十五年六月,黃州蛟起,武昌、承天、鄖陽、岳州、常德大水,漂沒廬舍。徽州、寧國、太平、嚴州四府山水大湧,漂人口甚眾。閏六月,京師大水,長安街水深五尺。三十七年九月,福建、江西大水。四十一年六月,通惠河決。七月,京師大水。南畿、江西、河南俱大水。八月,山東、廣西、湖廣俱大水。九月,遼東大水。四十二年,浙江、江西、兩廣俱水。四十四年七月,江西、廣東水。四十六年八月,潮州六縣海颶大作,溺萬二千三百餘人,壞民居三萬間。

天啟三年,睢寧河決。六年秋,河決匙頭灣,倒入駱馬湖,自新安鎮抵邳、宿,民居盡沒。是歲,順天、永平二府大水,邊垣多圮。

崇禎元年七月壬午,杭、嘉、紹三府海嘯,壞民居數萬間,溺數萬人,海寧、蕭山尤甚。三年,山東大水。四年六月,又大水。五年六月壬申,河決孟津口,橫浸數百里。七年五月,邛、眉諸州縣大水,壞城垣、田舍、人畜無算。十年八月,敘州大水,民登州堂及高阜者得免,餘盡沒。十三年五月,浙江大水。十四年七月,福州風潮泛溢,漂溺甚眾。十五年六月,汴水決。九月壬午,河決開封朱家寨。癸未,城圮,溺死士民數十萬。

▲水變

洪武五年,河南黃河竭,行人可涉。天順二年十二月癸未,武強苦井變為甘。弘治十四年八月丙辰,融縣河水紅濁如黃河。十月丙辰,馬湖底渦江水白可鑒,翌日濁如泔漿,凝兩岸沙石上者如土粉,十七日乃澄。丁巳,敘州東南二河白如雪、濃如漿者三日。十五年九月丙戌,濮州井溢,沙土隨水而出。正德十年七月,文安水忽僵立,是日大寒,結為冰柱,高圍俱五丈,中空旁穴。數日而賊至,民避穴中,生全者甚眾。隆慶六年五月,南畿龍目井化為酒。萬歷二十二年四月,南京正陽門水赤三日。二十五年八月甲申,蒲州池塘無風湧波,溢三四尺。臨淄濠水忽漲,南北相向而鬥。又夏莊大灣潮忽起,聚散不恒,聚則丈餘,開則見底。樂安小清河逆流。臨清磚板二閘,無風大浪。三十年閏二月戊午,河州蓮花寨黃河涸。四十六年四月,宣武、正陽門外水赤三里,如血,一月乃止。四十七年四月,宣武門響閘至東御河,水復赤。崇禎十年,寧遠衛井鳴沸,三日乃止。河南汝水變色,深黑而味惡,飲者多病。十三年,華陰渭水赤。十四年,山西潞水北流七晝夜,勢如潮湧。十五年,達州井鳴,濠水變血。十六年,松江自五月至七月不雨,河水盡涸,而泖水忽增數尺。

▲黑眚黑祥

洪武十年正月丁酉,金華、處州雨水如墨汁。十四年正月,黑氣亙天。十一月壬午,黑氣亙天者再。二十一年二月乙卯,黑氣亙天。宣德元年二月戊子,北方黑氣東西亙天。八月辛巳,樂安城中有黑氣如死灰。正統元年九月辛亥,未刻,黑氣亙天,自西南屬東北。二年八月甲申,北方黑氣東西亙天。十四年十一月己丑,晡時,西方有黑氣從地而生。景泰元年二月壬寅,黑氣南北亙天。十月辛未,西南黑氣如煙火,南北亙天。二年四月庚辰,有黑氣如煙,摩地而上。天順五年七月己亥朔,東方有黑氣,須臾蔽天。成化七年四月丙辰,雨黑沙如漆。八年三月庚子,黑氣起西北,臨清、德州晝晦。十二年七月庚戌,京師黑眚見。民間男女露宿,有物金睛修尾,狀如犬貍,負黑氣入牖,直抵密室,至則人昏迷。遍城驚擾,操刃張燈,鳴金鼓逐之,不可得。帝常朝,奉天門侍衛見之而嘩。帝欲起,懷恩持帝衣,頃之乃定。弘治五年二月己巳,北方黑氣東西亙天。六年八月壬申,南京有黑氣,東西百餘丈。十四年四月辛未,應州黑風大作。十六年二月庚子,宜良黑氣迷空,咫尺莫辨人形。正德七年六月壬戌,黑眚見順德、河間及涿,大者如犬,小者如貓,夜出傷人,有至死者。尋見於京師,形赤黑,風行有聲,居民夜持刁斗相警達旦,逾月乃息。後又見於封丘。十二年閏十二月丁丑夜,瑞州有紅氣變白,形如曲尺,中外二黑氣,相鬥者久之。八年十月癸巳,杭州雨黑水。三十七年三月,衡州黑眚見。隆慶二年四月,天雨黑豆。六年四月,杭州黑霧,有物蜿蜒如車輪,目光如電,冰雹隨之。萬歷二十四年十二月辛卯,同安生黑毛。二十五年二月癸亥,湖州黑雨雜以黃沙。崇禎十年,山東雨黑水,新鄉亦如之。十一年,京師有黑眚,狀如貍,入民家為祟,半歲乃止。十三年正月丁卯,黑氣彌空者三日。

《洪範》曰:「火曰炎上。」火不炎上,則失其性矣。前史多以恒燠、草異、火、木、羽蟲之孽、羊禍、火災、火異、赤眚赤祥皆屬之火,今從之。

▲恒燠

洪熙元年正月癸未,以京師一冬不雪,詔諭修省。正統九年冬,畿內外無雪。十二年冬,陜西無雪。景泰六年冬,無雪。天順元年冬,宮中祈雪。是年,直隸、山西、河南、山東皆無雪。二年冬,命百官祈雪。六年冬,直隸、山東、河南皆無雪。成化元年冬,無雪。五年冬,燠如夏。六年二月壬申,以自冬徂春,雨雪不降,敕諭群臣親詣山川壇請禱。十年二月,南京、山東奏,冬春恒燠,無冰雪。十一年冬,以無雪祈禱。十五年冬,直隸、山東、河南、山西無雪。十九年冬,京師、直隸無雪。弘治九年冬,無雪。十五年冬,無雪。十八年冬,溫如春,無雪。正德元年冬,無雪。永嘉自冬至春,麥穗桃李實。三年冬,無雪。六年至九年,連歲無雪。十一年冬,無雪。嘉靖十四年,冬深無雪,遣官遍祭諸神。十九年冬,無雪。二十年十二月癸卯,禱雪於神祇壇。二十四年十二月甲午,命諸臣分告宮廟祈雪。三十二年冬,無雪。三十三年十二月壬申,以災異屢見,即禱雪日為始,百官青衣辦事。三十六年冬,無雪。三十九年冬,無雪。明年,又無雪。帝將躬禱,會大風,命亟禱雪兼禳風變。四十一年至四十五年冬,祈雪無虛歲。隆慶元年冬,無雪。四年冬,無雪。萬曆四年十二月己丑,命禮部祈雪。十六年、十七年、二十九年、三十七年、四十七年,亦如之。崇禎五年十二月癸酉,命順天府祈雪。六年、七年冬,無雪。

▲草異

永樂十六年正月乙丑,同州、澄城、觔陽、朝邑雨穀及蕎麥。正統八年十一月,殿上生荊棘,高二尺。十四年,廣州獄竹床踰年忽青生葉。成化六年二月戊寅,湖廣應山雨粟。弘治八年二月,枯竹開花,實如麥米。苦賣開蓮花。六月甲子,黟縣雨豆,味不可食。九年,黃州民家瓜大如斗,瓤皆赤血。萬曆四十三年四月戊寅,石首雨豆,大小不一,色雜紅黑。崇禎四年、五年,河南草生人馬形,如被甲持矛馳驅戰鬥者然。十三年,徐州田中白豆,多作人面,眉目宛然。

▲羽蟲之孽

萬曆二十五年二月壬午,岳州民家有鴨,含絮裹火,飛上屋,入竹椽茅茨中。火四起,延燒數百家。四十三年四月壬午,雙鶴銜火,飛集掖縣海神廟殿。明日,廟火。崇禎六年,汝寧有鳥,鳩身猴足。鳳陽惡鳥數萬,兔頭、雞身、鼠足,供饌甚肥,犯其骨立死。

▲羊禍

萬曆三十八年四月,崞縣民家羊產羔,一首、二眼、四耳、二尾、八足。三十九年四月,降夷部產羊羔,人面羊身。

▲火災

洪武元年七月丁酉,京師火,延燒永濟倉。三年二月己巳,大河衛火,燔及廣積庫。七月乙未,寶源局火。甲子,鳳臺門軍營火,延燒武德衛軍器局。四年十一月癸亥,京師大軍倉災。五年二月癸未,臨濠府火。壬辰至甲午,京師火,毀龍驤等六衛軍民廬舍。七月丁卯,永清衛軍器庫火。十二月丙戌,京師定遠等衛火,焚及軍器局兵仗。十七年十二月己未,潮州火,官廨民居及倉廩、兵仗、圖籍焚蕩無遺。二十一年二月戊辰,歷代帝王廟火,上元縣治亦災。甲戌,天界、能仁二寺災。二十九年二月辛丑,通州火,燔屋千九百餘。三十年四月甲午,廣南衛火,延燒城樓及衛治倉庫。

建文二年八月癸巳,承天門災。

永樂四年十二月辛亥,甌寧王邸第火,王薨。十三年正月壬子,北京午門災。十九年四月庚子,奉天、謹身、華蓋三殿災。二十年閏十二月戊寅,乾清宮災。

宣德三年三月己亥,東嶽泰山廟火。六年八月,武昌火,延燒楚王宮,譜系敕符俱燼。甲辰,天津右衛北城外火,飛焰入城,燒倉廒。九年二月庚午,京城東南樓火。

正統二年二月,西鎮吳山廟火。三年八月辛酉,順天貢院火,席舍多焚,改期再試。十二月乙亥,韓府承運殿災。四年三月戊午,代府寢殿火。七年正月,廣昌木廠火,焚松木八千八百餘株。戊午,南京內府火,燔廊房六十餘間,圖籍、器用、守衛衣甲皆空。三月辛未,趙城媧皇寢廟火。十年正月庚寅,忠義前後二衛災。是時太倉屢火,遣官禱祭火龍及太歲以禳之。五月甲申,忠義後衛倉復火。癸巳,通州右衛倉火。十一月丁酉,御花房火。十一年秋,武昌火。死者數百人。十二月乙未,周府災。十二年六月,南京山川壇災。十三年二月癸酉,忠義前衛倉火。十四年六月丙辰夜,南京謹身、奉天、華蓋三殿災。

景泰二年六月丙子,青州廢齊府火。三年八月戊寅,秦府火。五年春,南京火,延燒數千家。七年九月壬申,寧府火,延燒八百餘家。

天順元年七月丙寅夜,承天門災。二年五月戊子,器皿廠火。三年九月庚寅,肅州城中火,延燒五千四百餘家,死者六十餘人。四年八月己巳,光祿寺大烹內門火。是歲,楚府頻火,宮殿家廟悉毀。五年三月丁卯,南京朝天宮災。六年六月癸未,楚府火。七年正月丁酉,南京西安門木廠火,延燒皇牆。二月戊辰,會試天下舉人,火作於貢院,御史焦顯扃其門,燒殺舉子九十餘人。

成化二年九月癸未,南京御用監火。六年十一月己亥,江浦火,延燒二百六十餘家。九年七月庚戌,東直門災。十一年四月壬辰夜,乾清宮門災。十三年十一月壬辰,太倉米麥,歲久蒸浥,自焚百餘石。十八年八月丙午,合州火,延燒千五百餘家。乙卯,楚府火凡三發。十一月戊午,南京國子監火。十二月乙卯,器皿廠火。壬辰,寧河王府火。先有妖夜見,或為神,或為王侯,時舉火作欲焚狀,是夜燔府第無遺,冠服器用皆燼。二十年正月戊戌,欽天監火。二十二年六月,臨海縣災,延燒千七百餘家。

弘治元年三月庚寅,南京內花園火。十一月丁丑夜,南京甲字庫災。二年四月乙未,南京神樂觀火。四年二月戊午,禮部官舍火。六年四月甲寅,刑部官舍火。辛酉夜,南京舊內災。八年三月戊子,鎮東等堡躍火星如斗,毀公館倉廒,人馬多斃。十一年,自春徂夏,貴州大火。毀官民房舍千八百餘所,死傷者六千餘人。十月甲戌夜,清寧宮災。十二年六月甲辰夜,闕里聖廟災。十二月,建陽縣書坊火,古今書板皆燼。十三年二月乙酉,禮部官舍火。七月甲寅,南城縣空中有火,乍分乍合,流光下墜十餘丈,隱隱有聲,毀軍民廬舍。庚申,永寧衛雁尾山至居庸關之石縱山,東西四十餘里,南北七十餘里,延燒七晝夜。閏七月辛巳,福州城樓毀。八月己未,沈府火。十一月庚辰,寧河府火。十六年三月庚午,遼東鐵嶺衛墜火如斗。丙子,火起,燒房屋二千五百餘間,死者百餘人。四月戊午,寬河衛倉災,毀米豆四萬餘石。九月戊寅,廣寧衛城火,燔三百餘家。十七年四月丁巳,淮安火焚五百餘家。五月癸巳,正陽門內西廊火,燔武功坊。

正德元年二月庚寅,鄖陽火,毀譙樓官舍,延百餘家。是歲,寧夏左屯衛紅氣亙天,既而火作,城樓臺堡俱燼。六月庚寅,大同平虜城災,燔槁百萬餘。十一月己亥,臨海縣治火,延燒數千家。七年三月己未,嶧縣有火如斗,自空而隕,大風隨之,毀官民房千餘間。火逸城外,延及丘木。庚申,成山衛秦皇廟火,屋宇悉毀,像設如故。是月,文登大桑樹火,樹燔而枝葉無損。五月癸酉至閏五月丙子,遼東懿路城火三作,焚官民廬舍之半。九月壬午,玉山火,燔學舍及民居三百餘家。八年六月辛酉,豐城縣西南連隕火星,如盆如斗。既而火作,至七月初始熄,燔二萬餘家。七月戊子,火隕龍泉縣,焚四千餘家。十月壬寅,饒州及永豐、浮梁火,各燔五百餘家。浮梁學舍災。庚申,臨江火,燔官舍,延八百餘家。九年正月庚辰,乾清宮火。十一年八月丁丑,黔陽火,毀城樓官廨,延七百餘家。十二年正月甲辰,清寧宮小房火。四月,裕陵神宮監火。八月丁卯,南昌火,燔三百家。九月壬午,建安火,燔二百五十餘家。十三年二月己卯,夷陵火,燔七百餘家。八月庚辰,獻陵明樓災。丁酉,延平火,燔五百餘家。十四年四月乙巳,淮安新城火。七月丙辰,泰寧火,燔五千餘家。十五年五月辛卯,靜樂火,燔八百餘家。

嘉靖元年正月己未,清寧宮後三小宮災。楊廷和言廢禮之應,不報。二月己丑,南京針線廠火。己亥,通州城樓火。二年五月丙子,榮府火。九月戊辰,秦府宮殿火。四年三月壬午夜,仁壽宮災,玉德、安喜、景福諸殿俱燼。五年三月乙酉,趙府家廟火。六年三月丁亥,西庫火。八年十月癸未,大內所房災。十年正月辛亥,大內東偏火。四月庚辰,兵、工二部公廨災,毀文籍。十三年六月甲子,南京太廟火,毀前後殿、東西廡、神廚庫。十五年四月癸卯,山西平虜衛火,盡毀神機官庫軍器。十八年二月乙丑,趙州及臨洺鎮行宮俱火。丁卯,駕幸衛輝,行宮四更火,陸炳負帝出,後宮及內侍有殞於火者。六月丁酉,皇城北鼓樓災。二十年四月辛酉夜,宗廟災。毀成、仁二廟主。二十五年五月壬申,盔甲廠火。二十六年十一月壬午,宮中火,釋楊爵於獄。三十一年八月乙丑,南京試院火。三十五年九月戊辰,杭州大火,延燒數千家。三十六年四月丙申,奉天、華蓋、謹身三殿,文武二樓,午門、奉天門俱災。三十七年正月,光祿寺災。三十八年正月癸未,前軍都督府火。四十年十一月辛亥夜,萬壽宮災。四十四年三月己亥夜,大明門內西千步廊火。

隆慶二年正月,浙江省城外災。毀室廬舟艦以千計。三月乙亥,乾清、坤寧兩宮,一時俱燼。五年二月壬子,南京廣、惠二倉火。

萬曆元年十一月己亥,慈寧宮後舍火。三年四月甲戌,工部後廠火。五年十月丙申,禁中火。十一月癸未,宗人府災。十一年十二月庚午夜,慈寧宮災。十二年二月己酉,無逸殿災。十二月癸卯朔,又災。十五年五月甲子,司設監火。十八年三月辛酉,遼東寨山兒堡火,毀城堡器械,傷九十餘人。十九年十二月甲辰,萬法寶殿災。二十一年六月望,太倉公署後樓有炮聲,火藥器械俱燼。二十二年五月壬寅,天火燔鐵嶺衛千餘家。二十四年二月甲寅,潞府門火。三月乙亥,火發坤寧宮,延及乾清宮,俱燼。二十五年二月壬午,杭州火,燒官民房千三百餘間。丙戌,馬湖屏山災,延燔八百餘家,斃二十四人。三月癸卯,泗州大火。燒民房四千餘。盱眙火,燔民房百六十餘間。撥漕糧二萬石以振。六月戊寅,三殿災。火起歸極門,延皇極等殿,文昭、武成二閣,周遭廊房,一時俱燼。十二月甲寅,吏部文選司署火。二十七年十一月壬申,內府火,延燒尚寶司印綬監、工部廊,至銀作局山牆而止。二十八年三月,南陽火,延燒唐府。二十九年正月己巳,鐵嶺衛火,車輛火藥俱燼。八月己卯,大光明東配殿災。三十年二月乙酉,魏國公賜第火。十月丙申,孝陵災。十二月庚子,南海普陀山寺災。三十一年九月戊寅,通州漕艘火。三十三年二月乙丑,御馬監火。五月辛巳,洗白廠火。九月甲午,昭和殿火。丙申,官軍於盔甲廠支火藥,藥年久凝如石,用斧劈之,火突發,聲若震霆,刀鎗火箭迸射百步外,軍民死者無數。十一月丁卯,刑部提牢廳火。三十五年二月乙卯,易州神器庫火。四月丁酉,通州西倉火。十月己卯,南京行人司署毀。三十七年正月庚子,慶府火,燔寢宮及帑藏。三月丙戌,武昌火,越二日又火,共燔二百六十餘家。六月,慶府災。十月戊午,朝日壇火。三十八年四月丁丑夜,正陽門箭樓火。三十九年四月戊子,怡神殿災。四十一年五月壬戌,蜀府災,門殿為燼。四十三年四月壬午,黃花鎮柳溝火,延燒數十里。甲午,蜀府殿庭災。遼東長寧堡自二月至五月,火凡五發,毀房屋人畜無算。閏八月辛亥,通州糧艘火。九月丁丑,湖口稅廨毀。四十四年十一月己巳,隆德殿災。丁亥,南城延喜宮災。四十五年正月壬午,東朝房火,延毀公生門。十一月丙戌,宣禧宮災。四十六年閏四月丁丑夜,開原殷家莊堡臺桿八同時燼。甲申,煖閣廠膳房火。九月壬子,茂陵火。四十七年四月癸酉,盔甲廠火。

泰昌元年十月丁卯,噦鸞宮災。

天啟元年閏二月丙戌,昭和殿災。三月甲辰,杭州火,延燒六千餘家。八月戊子,復災,城內外延毀萬餘家。二年五月丙申,旗纛廟正殿災,火藥盡焚,匠役多死者。三年七月辛卯,南京大內左傍宮災。六年五月戊申,王恭廠災,地中霹靂聲不絕,火藥自焚,煙塵障空,白晝晦冥,凡四五里。五月癸亥,朝天宮災。七月庚寅,登州城樓火。七年十月庚子,寧遠前屯火,傷男婦二百餘人。

崇禎元年四月乙卯,左軍都督府災。五月乙亥,鷹坊司火。丁亥,丁字庫火。七月己卯,公安縣火,毀文廟,延五千餘家。二年十一月庚子,火藥局災。三年三月戊戌,又災。八月癸酉,頭道關災,火器轟擊無餘。六年正月癸丑,濟南舜廟災。七年九月庚申,盔甲廠災。十一年四月戊戌,新火藥局災。傷人甚眾。六月癸巳,安民廠災,震毀城垣廨舍,居民死傷無算。八月丁酉,火藥局又災。

▲火異

成化二十一年正月甲申朔,有火光自中天而少西,墜於下,化為白氣,復曲折上騰,聲如雷。

弘治三年三月庚午,儀隴空中有紅白火焰,長三丈餘,自縣治東北流,至正東六十餘里而墜,聲震如雷。八年三月辛卯,廣寧右衛臺桿火,高五寸,桿如故。十年四月辛丑,阜平有火光,長八九尺,大如轆軸,有聲,自東南至西南而墜。

正德元年三月戊申夜,太原有火如斗大,墜寧化王殿前。廣寧墩臺火發旗桿,凡六。七月壬戌夜,火光墜即墨民家,化為綠石,圓高尺餘。七年三月丁卯夜,大風雷電,餘干仙居寨有光如箭,墜旗竿上,俄如燭龍,光照四野。士卒撼其旗,飛上竿首,既而其火四散,鎗首皆有光如星。十二年五月己亥夜,火隕都察院獄,旋轉久之始滅。十五年六月癸未夜,台州火隕三,大如盤,觸草木皆焦。

嘉靖五年七月甲申,有火球三,大五六尺,從北墜於東,其光燭天。二十年七月丙戌,火球如斗,隕左軍都督府中門東,良久乃滅。

隆慶二年三月戊午,延綏保寧堡城角旗桿出火,灼灼有聲。

萬曆十四年,保定府民間牆壁內出火,三日夜乃熄。十五年二月,綏靖邊城各堡,脊獸旗桿俱出火。軍士以杖撲之,杖亦生火,三更乃熄。二十年三月,陜西空中有火,大如盆,後生三尾,隕於西北。二十一年二月庚辰夜分,大毛山樓上各獸吻俱有火,如雞卵,赤色,即時雨雪,火上嗟嗟有聲。二十三年九月癸巳夜,永寧有火光,形如屋大,隕於西北。永昌、鎮番、寧遠所見同。二十四年二月戊申夜,鄠縣雷雨,遍地火光,十有餘里。二十五年二月癸亥,平涼瓦獸口出火,水灌不滅。八月甲申,肅、涼二州火光在天,形如車輪,尾分三股,約長三丈。

天啟六年五月壬寅朔,厚載門火神廟紅球滾出,前門城樓角有數千螢火,並合如車輪。

崇禎元年,西安有火如碾如斗者數十,色青,焰高尺許,嘗入民居,留數日乃去。用羊豕禳之,不為害,自五月至七月而止。十三年六月壬申,鎮安火光如斛,自西墜地,士木皆焦。

▲赤眚赤祥

成化十三年二月甲午,浙江山陰湧泉如血。

正德元年正月乙酉夜,崇明空中有紅光,曳尾如虹,起東北至西南沒,聲如雷。辛丑,鳳陽紅光發,與日同色,聲如雷。二年八月己亥,赤光見寧夏,長五丈。八年七月甲申,龍泉有赤彈二,自空隕於縣治,形如鵝卵,躍入民居,相鬥久之。

嘉靖三十三年四月戊子,慈谿民家湧血高尺餘。三十七年五月戊辰,東陽民張思齊家地裂五六處,出血如線,高尺許。血凝,犬就食之,掘地無所見。三十九年二月己未,竹溪民家出血。

隆慶六年閏二月癸酉,遼東赤風揚塵蔽天。

萬曆六年七月丁丑,松門衛金鐺家湧血三尺,有聲。十三年四月乙丑,虹民王祿投宿姚壘家,見血出於地,驚走至市,市亦流血。鄉人擊器物噪之,乃止。十九年六月庚戌,慈谿茅家浦湧血八處,大如盆,高尺許。血濺船,船即出血,濺人足,足亦出血,數刻乃絕。二十六年九月甲辰,蕭山賈九經家出血,高尺許。

天啟元年六月庚寅,肇慶民王體積中庭噴血,如跑突泉。

崇禎七年二月戊午,海豐雨血。八年八月戊寅,宣城池中出血。

《洪範》曰:「木曰曲直。」木不曲直,則失其性矣。前史多以恒雨、狂人、服妖、雞禍、鼠孽、木冰、木妖、青眚青祥皆屬之木,今從之。

▲恒雨

洪武十三年七月,海康大雨,壞縣治。二十三年十一月,山東二十九州縣久雨,傷麥禾。

建文元年三月乙卯夜,燕王營於蘇家橋,大雨,平地水三尺,及王臥榻。

永樂元年三月,京師霪雨,壞城西南隅五十餘丈。七月,建寧衛霪雨壞城。二年七月,新安衛霪雨壞城。八月,霪雨壞北京城五千餘丈。六年七月,思明霪雨壞城。七年九月,浙江衛所五,颶風驟雨,壞城,漂流房舍。八年七月,金鄉衛颶風驟雨,壞城垣公廨。十二年九月,密雲後衛霪雨壞城。二十年正月,信豐雨水壞城,瞿城衛如之。二十一年二月,六安衛霪雨壞城。是歲,建昌守禦所,淮安、懷來等衛,皆霪雨壞城。二十二年二月,壽州衛雨水壞城。三月,贛州、振武二衛雨水壞城。四月,霪雨壞密雲及薊州城。是歲,南、北畿、山東州縣,霪雨傷麥禾甚眾。

洪熙元年夏,蘇、松、嘉、湖積雨傷稼。閏七月,京師大雨,壞正陽、齊化、順成等門城垣。

九月,久雨壞密雲中衛城。

宣德元年五月,永嘉、樂清颶風急雨,壞公私廨宇及壇廟。

正統元年七月,順天、山東、河南、廣東霪雨傷稼。四年夏,居庸關及定州衛霪雨壞城。五年二月,南京大風雨,壞北上門脊,覆官民舟。七年,濟南、青、萊、淮、鳳、徐州,五月至六月霪雨傷稼。九年閏七月,野狐嶺等處霪雨壞城及濠塹墩臺。十一年春,江西七府十六縣霪雨,田禾淹沒。十二年六月,瑞金霪雨,市水丈餘,漂倉庫,溺死二百餘人。十三年四月,雨水壞順天古北口邊倉。五月至六月,鳳陽、徽州久雨傷稼。九月,寧都大雨壞城郭廬舍,溺死甚眾。

景泰三年,永平、兗州久雨傷禾。大嵩等二十衛所久雨壞城。四年,南畿、河南、山東府十州一,自五月至於八月霪雨傷稼。五年,杭、嘉、湖大雨傷苗,六旬不止。七月,京師久雨,九門城垣多壞。六年,北畿府五、河南府二久雨傷稼,雲南大理諸府如之。七年,兩畿、江西、河南、浙江、山東、山西、湖廣共府三十,恒雨淹田。

天順元年,濟、兗、青三府大雨閱月,禾盡沒。四年,安慶、南陽雨,自五月至七月,淹禾苗。七年五月,淮、鳳、揚、徐大雨,腐二麥。武昌、漢陽、荊州廬舍漂沒,民皆依山露宿。

成化元年六月,畿東大雨,水壞山海關、永平、薊州、遵化城堡。八月,通州大雨,壞城及運倉。二年,定州積雨,壞城垣及墩臺垛口百七十三。八年七月,南京大風雨,壞天、地壇、孝陵廟宇。鳳陽大雨,壞皇陵牆垣。九年三月,南京大風雨,拔太廟、社稷壇樹。十三年七月,京城大雨。十四年八月,鳳陽大雨,沒城內民居以千計。十七年七月乙酉,南京大風雨,社稷壇及太廟殿宇皆摧。十八年,河南、懷慶諸府,夏秋霪雨三月,塌城垣千一百八十餘丈,漂公署、壇廟、民居三十一萬四千間有奇,淹死一萬一千八百餘人。

弘治二年七月,京師霪雨,求直言。三年七月,南京驟雨,壞午門西城壇。七年七月庚寅,南京大風雨,壞殿宇、城樓獸吻,拔太廟、天、地、社稷壇及孝陵樹。自五月至八月,義州等衛連雨害稼。八年五月,南京陰雨踰月,壞朝陽門北城堵。九月,潮州諸府,颶風暴雨壞城垣廬舍。十年七月,安陸霪雨,壞城郭廬舍殆盡。十一年七月,長安嶺暴風雨,壞城及廬舍。十四年六月,義、錦、廣寧霪雨,壞城垣、墩堡、倉庫、橋梁,民多壓死者。十五年六七月,南京大風雨,孝陵神宮監及懿文陵樹木、橋梁、牆垣多摧拔者。十六年五月,榆林大風雨,毀子城垣,移垣洞於其南五十步。十八年三月,雙山堡大雷雨壞城。六月至八月,京畿連雨。

正德元年七月,鳳陽諸府大雨,平地水深丈五尺,沒居民五百餘家。二年七月,武平大風雨,毀城樓。長泰、南靖大風雨三日夜,平地水深二丈,漂民居八百餘家。十二年,蘇、松、常、鎮、嘉、湖大雨,殺麥禾。十三年,應天、蘇、松、常、鎮、揚大雨彌月,漂室廬人畜無算。十六年,京師久雨傷稼。

嘉靖四年六月,登州大雨壞城。十六年,京師雨,自夏及秋不絕,房屋傾倒,軍民多壓死。二十五年八月,京師大雨,壞九門城垣。三十三年六月,京師大雨,平地水數尺。四十五年九月,鄖陽大霪雨,平地水丈餘。壞城垣廬舍,人民溺死無算。

隆慶元年六月,京師霪雨,遼東自五月至七月雨不止,壞垣牆禾黍。

萬曆元年七月,霪雨。十一年四月,承天大雨水。十二年正月,喜峰口大風雨,壞各墩臺。十五年五月至七月,蘇、松諸府霪雨,禾麥俱傷。六月,京師大雨。二十四年,杭、嘉、湖霪雨傷苗。二十八年七月,興化、莆田、連江、福安大雨數日夜,城垣、橋梁、隄岸俱圮。二十九年春夏,蘇、松、嘉、湖霪雨傷麥。三十二年七月,京師霪雨,城崩。三十三年五月丙申,鳳陽大風雨,損皇陵正殿御座。三十九年春,河南大雨。夏,京師、廣東大雨。廣西積雨五閱月。四十二年,浙江霪雨為災。

天啟六年閏六月,大雨連旬,壞天壽山神路,都城橋梁。是歲,遼東霪雨,壞山海關內外城垣,軍民傷者甚眾。七年,山東州縣二十有八積雨傷禾。

崇禎五年六月,大雨。八月,又雨,衝損慶陵。九月,順天二十七縣霪雨害稼。十一年夏,雨浹旬,圮南山邊垣。十二年十二月,浙江霪雨,阡陌成巨浸。十三年四月至七月,寧、池諸郡霪雨,田半為壑。十五年十月,黃、蘄、德安諸郡縣霪雨。十六年二月戊辰,親祀社稷,大風雨,僅成禮而還。

▲狂人

景泰三年五月癸巳朔,以明日立太子,具香亭於奉天門。有一人自外竟入,執紅棍擊香亭曰:「先打東方甲乙木。」嘉靖十八年,駕將南幸,有軍人孫堂從御路中橋至奉天門下,登金臺,坐久,守門官役無知者。升堂大呼,覺而捕之,乃病狂者。

▲服妖

正德元年,婦女多用珠結蓋頭,謂之瓔珞。十三年正月,車駕還京,令朝臣用曳撒大帽鸞帶。給事中朱鳴陽言,曳撒大帽,行役所用,非見君服。皆近服妖也。十五年十二月,帝平宸濠還京,俘從逆者及懸諸逆首於竿,皆標以白幟,數里皆白。時帝已不豫,見者識其不祥。崇禎時,朝臣好以紗縠、竹籜為帶,取其便易。論者謂金銀重而貴,紗籜賤而輕,殆賤將乘貴也。時北方小民製幘,低側其簷,自掩眉目,名曰「不認親」。其後寇亂民散,途遇親戚,有飲泣不敢言,或掉臂去之者。

▲雞禍

弘治十四年,華容民劉福家雞雛三足。十七年六月,崇明民顧孟文家雞生雛,猴頭而人形,身長四寸,有尾,活動無聲。嘉靖四年,長垣民王憲家雞抱卵,內成人形,耳目口鼻四肢皆具。萬曆二十二年六月,靖邊營軍家雌雞化為雄。崇禎九年,淮安民家牝雞啼躍,化為雄。十年,宣武門外民家白雞,喙距純赤,重四十斤。或曰此皦也,所見之處國亡。十四年,太倉衛指揮姜周輔家雞伏子,兩頭四翼八足。

▲鼠妖

萬曆四十四年七月,常、鎮、淮、揚諸郡,土鼠千萬成群,夜銜尾渡江,絡繹不絕,幾一月方止。四十五年五月,南京有鼠萬餘,銜尾渡江,食禾稼。崇禎七年,寧夏鼠十餘萬,銜尾食苗。十二年,黃州鼠食禾,渡江五六日不絕。時內殿奏章房多鼠盜食,與人相觸而不畏,亦鼠妖也。至甲申元旦後,鼠始屏跡。又秦州關山中鼠化鵪鶉者以數千計。十五年二月,群鼠渡江,晝夜不絕。十月,榆林、定邊諸堡鼠生蝦蟆腹中,一生數十,食苗如割。

▲木冰

洪武四年正月戊申,木冰。六年十二月乙丑,雨木冰。十一年正月丁亥,雨木冰。二十二年正月甲戌,雨木冰。正統三年十月丁丑曉,木介。天順七年十月甲辰,雨木冰。八年正月乙丑,雨木冰。成化十六年正月辛卯曉,雨木冰。二十三年十二月戊辰曉,木介。隆慶三年十一月癸巳,木冰。萬曆十四年冬,蘇、松木冰。崇禎元年十一月,陜西木冰,樹枝盡折。其後大河以北,歲有此異。

▲木妖

弘治八年,長沙楓生李實,黃蓮生黃瓜。九年三月,長寧楠生蓮花,李生豆莢。嘉靖三十七年十月戊辰,泗水沙中湧出大杉木,圍丈五尺,長六丈餘。隆慶五年四月,杭州慄生桃。萬曆十八年五月丁卯,祖陵大松樹孔中吐火,竟日方滅。二十三年十二月癸亥,皇陵樹顛火出,延燒草木。天啟六年四月癸巳,白露著樹如垂綿,日中不散。十月辛酉,南京西華門內有煙無火。禮臣往視,乃舊宮材木,瘞土中久,煙自生,土石皆焦。以水沃之,三日始滅。崇禎六年五月癸巳,霍山縣有木甑飛墮,不知所自來。七年二月丁巳,太康門牡自開者三,知縣集邑紳議其事,梁墮而死。

▲青眚青祥

宣德元年八月辛巳,東南天有青氣,狀如人叉手揖拜。

《洪範》曰:「金曰從革。」金不從革,則失其性矣。前史多以恒暘、詩妖、毛蟲之孽、犬禍、金石之妖、白眚白祥皆屬之金,今從之。

▲恒暘

洪武三年,夏旱。六月戊午朔,步禱郊壇。四年,陜西、河南、山西及直隸常州、臨濠、北平、河間、永平旱。五年夏,山東旱。七年夏,北平旱。二十三年,山東旱。二十六年,大旱,詔求直言。

永樂十三年,鳳陽、蘇州、浙江、湖廣旱。十六年,陜西旱。

宣德元年夏,江西旱。湖廣夏秋旱。二年,南畿、湖廣、山東、山西、陜西、河南旱。七年,河南及大名夏秋旱。八年,南、北畿、河南、山東、山西自春徂夏不雨。九年,南畿、湖廣、江西、浙江及真定、濟南、東昌、兗州、平陽、重慶等府旱。十年,畿輔旱。

正統二年,河南春旱。順德、兗州春夏旱。平涼等六府秋旱。三年,南畿、浙江、湖廣、江西九府旱。四年,直隸、陜西、河南及太原、平陽春夏旱。五年,江西夏秋旱。南畿、湖廣、四川府五,州衛各一,自六月不雨至於八月。六年,陜西旱。南畿、浙江、湖廣、江西府州縣十五,春夏並旱。七年,南畿、浙江、湖廣、江西府州縣衛二十餘,大旱。十年夏,湖廣旱。十一年,湖廣及重慶等府夏秋旱。十二年,南畿及山西、湖廣等府七夏旱。十三年,直隸、陜西、湖廣府州七夏秋旱。十四年六月,順天、保定、河間、真定旱。

景泰元年畿輔、山東、河南旱。二年,陜西府四、衛九旱。三年,江西旱。四年,南北畿、河南及湖廣府三,數月不雨。五年,山東、河南旱。六年,南畿及山東、山西、河南、陜西、江西、湖廣府三十三、州衛十五皆旱。七年,湖廣、浙江及南畿、江西、山西府十七旱。

天順元年夏,兩京不雨,杭州、寧波、金華、均州亦旱。三年,南北畿、浙江、湖廣、江西、四川、廣西、貴州旱。四年,濟南、青州、登州、肇慶、桂林、甘肅諸府衛夏旱。五年,南畿府四、州一,及錦衣等衛連月旱,傷稼。七年,北畿旱。濟南、青州、東昌、衛輝,自正月不雨至於四月。

成化三年,湖廣、江西及南京十一衛旱。四年,兩京春夏不雨。湖廣、江西旱。六年,直隸、山東、河南、陜西、四川府縣衛多旱。八年,京畿連月不雨,運河水涸,順德、真定、武昌俱旱。九年,彰德、衛輝、平陽旱。十三年四月,京師旱。是歲,真定、河間、長沙皆旱。十五年,京畿大旱,順德、鳳陽、徐州、濟南、河南、湖廣皆旱。十八年,兩京、湖廣、河南、陜西府十五、州二旱。山西大旱。十九年,復旱。二十年,京畿、山東、湖廣、陜西、河南、山西俱大旱。二十二年六月,陜西旱,蟲鼠食苗稼,凡九十五州縣。八月,北畿及江西三府旱。九月,溫、台大旱,長沙諸府亦旱。

弘治元年,南畿、河南、四川及武昌諸府旱。三年,兩京、陜西、山東、山西、湖廣、貴州及開封旱。四年,浙江府二,廣西府八,及陜西洮州衛旱。六年,北直、山東、河南、山西及襄陽、徐州旱。七年,福建、四川、山西、陜西、遼東旱。八年,京畿、陜西、山西、湖廣、江西大旱。十年,順天、淮安、太原、平陽、西安、延安、慶陽旱。十一年,河南、山東、廣西、江西、山西府十八旱。十二年夏,河南四府旱。秋,山東旱。十三年,慶陽、太原、平陽、汾、潞旱。十四年,遼東鎮春至秋不雨,河溝盡涸。十六年夏,京師大旱,蘇、松、常、鎮夏秋旱。十八年,北京及應天四十二衛旱。

正德元年,陜西三府旱。二年,貴州、山西旱。三年,江南、北旱。四年,旱,自三月至七月,陜西亦旱。七年,鳳陽、蘇、松、常、鎮、平陽、太原、臨、鞏旱。八年,畿輔及開封、大同、浙江六縣旱。九年,順天、河間、保定、廬、鳳、淮、揚旱。十一年,北畿及兗州、西安、大同旱。十五年,淮、揚、鳳陽州縣三十六及臨、鞏、甘州旱。十六年,兩京、山東、河南、山西、陜西自正月不雨至於六月。

嘉靖元年,南畿、江西、浙江、湖廣、四川、遼東旱。二年,兩京、山東、河南、湖廣、江西及嘉興、大同、成都俱旱,赤地千里,殍殣載道。三年,山東旱。五年,江左大旱。六年,北畿四府,河南、山西及鳳陽、淮安俱旱。七年,北畿、湖廣、河南、山東、山西、陜西大旱。八年,山西及臨洮、鞏昌旱。九年,應天、蘇、松旱。十年,陜西、山西大旱。十一年,湖廣、陜西大旱。十七年夏,兩京、山東、陜西、福建、湖廣大旱。十九年,畿內旱。二十年三月,久旱,親禱。二十三年,湖廣、江西旱。二十四年,南、北畿、山東、山西、陜西、浙江、江西、湖廣、河南俱旱。二十五年,南畿、江西旱。二十九年,北畿、山西、陜西旱。三十三年,兗州、東昌、淮安、揚州、徐州、武昌旱。三十四年,陜西五府及太原旱。三十五年夏,山東旱。三十七年,大旱,禾盡槁。三十九年,太原、延安、慶陽、西安旱。四十年,保定等六府旱。四十一年,西安等六府旱。

隆慶二年,浙江、福建、四川、陜西及淮安、鳳陽大旱。四年夏,旱,詔諸司停刑。六年夏,不雨。

萬曆十一年八月庚戌朔,河東鹽臣言,解池旱涸,鹽花不生。十三年四月戊午,因久旱,步禱郊壇。京師自去秋至此不雨,河井並涸。十四年三月乙巳,以久旱,命順天府祈禱。十七年,蘇、松連歲大旱,震澤為平陸。浙江、湖廣、江西大旱。十八年四月,旱。二十四年,杭、嘉、湖三府旱。二十六年四月,旱。二十七年夏,旱。二十九年,畿輔、山東、山西、河南及貴州黔東諸府衛旱。三十年夏,旱。三十四年夏,亢旱。三十七年,楚、蜀、河南、山東、山西、陜西皆旱。三十八年夏,久旱。濟、青、登、萊四府大旱。三十九年夏,京師大旱。四十二年夏,不雨。四十三年三月,不雨,至於六月。山東春夏大旱,千里如焚。四十四年,陜西旱。秋冬,廣東大旱。四十五年夏,畿南亢旱。四十七年,廣西梧州旱,赤地如焚。

泰昌元年,遼東旱。

天啟元年,久旱。五年,真、順、保、河四府,三伏不雨,秋復旱。七年,四川大旱。

崇禎元年夏,畿輔旱,赤地千里。三年三月,旱,擇日親禱。五年,杭、嘉、湖三府自八月至十月七旬不雨。六年,京師及江西旱。十年夏,京師及河東不雨,江西大旱。十一年,兩京及山東、山西、陜西旱。十二年,畿南、山東、河南、山西、浙江旱。十三年,兩京及登、青、萊三府旱。十四年,兩京、山東、河南、湖廣及宣、大邊地旱。十六年五月辛丑,祈禱雨澤,命臣工痛加修省。

▲詩妖

太祖吳元年,張士誠弟偽丞相士信及黃敬夫、葉德新、蔡彥文用事。時有十七字謠曰「丞相做事業,專靠黃、蔡、葉。一朝西風起,乾鱉。」未幾,蘇州平,士信及三人者皆被誅,此其應也。建文初年,有道士歌於途曰:「莫逐燕,逐燕日高飛,高飛上帝畿。」已忽不見,是靖難之讖也。

正統二年,京師旱,街巷小兒為土龍禱雨,拜而歌曰:「雨帝雨帝,城隍土地。雨若再來,還我土地。」說者謂「雨帝」者,與弟也,帝弟同音。「城隍」者,郕王。「再來」、「還土地」者,復辟也。

萬曆末年,有道士歌於市曰:「委鬼當頭坐,茄花遍地生。」北人讀客為楷,茄又轉音,為魏忠賢、客氏之兆。又成都東門外鎮江橋迴瀾塔,萬曆中布政餘一龍所修也。張獻忠破蜀毀之,穿地取磚,得古碑。上有篆書云:「修塔餘一龍,拆塔張獻忠。歲逢甲乙丙,此地血流紅。妖運終川北,毒氣播川東。吹簫不用竹,一箭貫當胸。漢元興元年,丞相諸葛孔明記。」本朝大兵西征,獻忠被射而死,時肅王為將。又有謠曰:「鄴臺復鄴臺,曹操再出來。」賊羅汝才自號曹操,此其兆也。

▲毛蟲之孽

弘治九年八月,有黑熊自都城蓮池緣城上西直門,官軍逐之下,不能獲。嚙死一人,傷一人。十一年六月,有熊自西直門入城,郎中何孟春曰:「當備盜,亦宜慎火。宋紹興間熊抵永嘉城,州守高世則以熊字能火,戒郡中慎火,果延燒廬舍,此其兆也。」是年,城內多火災。嘉靖五年七月,南城縣有虎,具人手足。四十五年六月,太醫院吏目李乾獻兔,體備五色,以為瑞兔。

▲犬禍

嘉靖二十年,民家生一犬,八足四耳四目。萬曆四十七年七月,懷寧民家產一犬,長五寸,高四寸,一頭二身八腳,狀如人。

▲金異

洪武十一年正月元旦甲戌,早朝,殿上金鐘始叩,忽斷為二。六月丁卯夜,寧夏衛風雨,兜鍪旗槊皆有火光。十二年十二月甲子,徐州衛譙樓銅壺自鳴。乙丑,復鳴。是歲,胡惟庸井中生石筍,去之,筍復旁出者三。次年,惟庸伏誅。建文二年四月乙卯,燕王營於蘇家橋,兵端火光如球,上下相擊,金鐵錚錚,弓絃自鳴。成化十三年六月壬子,雨錢於京師。正德四年三月甲寅,蓋州衛城樓鐘自鳴者三。七年,文登秦始皇廟鐘鼓自鳴。成山衛如之。嘉靖六年五月甲午,京師雨錢。隆慶六年七月七日,有物轟轟,飛至直隸華亭海濱墜於地,乃鐘也。鑄時年月具在,識者謂其來自閩云。萬曆二十一年十月甲申,山東督撫令旗及刀鎗頭皆火出,且有聲。二十六年五月庚寅,古浪城樓大鐘自鳴者三。天啟六年五月丁未,京城石獅擲出城外。銀、錢、器皿飄至昌平閱武場中。崇禎六年五月癸巳,有鐵斧飛落霍山縣。八年十二月辛巳,夜四鼓,山東鎮南城樓大炮鳴如鐘,至黎明,大吼一聲乃止。十三年三月丙申,蘄州城隍廟古鐘自鳴。

▲白眚白祥

洪熙元年六月庚戌,中天有白氣,東西竟天。宣德元年六月癸未夜,有蒼白氣,東西竟天。八月庚辰,東南有白氣,狀如群羊驚走。既滅,有黑氣如死蛇,頃之分為二。弘治五年十二月辛亥夜,東方有白氣,南北亙天,去地五丈。正德元年三月戊申夜,太原空中見紅光,如彎弓,長六七尺。旋變黃,又變白,漸長至二十餘丈,光芒亙天。嘉靖七年十二月望,白氣亙天津。

《洪範》曰:「土爰稼穡。」稼穡不成,則土失其性矣。前史多以恒風、風霾、晦冥、花妖、蟲孽、牛禍、地震、山頹、雨毛、地生毛、年饑、黃眚黃祥皆屬之土,今從之。

▲恒風

宣德六年六月,溫州颶風大作,壞公廨、祠廟、倉庫、城垣。正統四年七月,蘇、松、常、鎮四府大風,拔木殺稼。

天順二年二月,暴風拔孝陵松樹,懿文陵殿獸脊、梁柱多摧。三年四月,順天、河間、真定、保定、廣平、濟南連日烈風,麥苗盡敗。成化十四年八月丁未,南京大風,拔太廟樹。十五年八月辛卯,大風拔孝陵木。二十一年五月,南京大風拔太廟樹,摧大祀殿及皇城各門獸吻。弘治三年六月壬午朔,陜西靖虜衛大風,天地昏暗,變為紅光如火,久之乃息。七年三月己亥,廣寧諸衛狂風,沈陽、錦州城仆百餘丈。正德元年六月辛酉,暴風折郊壇松柏,壞大祀殿、齋宮獸瓦。二年閏正月癸亥,盧龍、遷安大風拔樹毀屋。乙丑,大風壞奉天門右吻。三年二月己丑,大同暴風,屋瓦飛動,三日而止。九年二月丁巳,長樂大雨雹,狂風震電,屋瓦皆飛。五月戊辰,曲阜暴風毀宣聖廟獸吻。十二年四月丙辰,來賓大風雨雹,毀官民廬舍,屋瓦皆飛。十一月癸巳,南京大風雪,仆孝陵殿前樹及圍墻內外松柏。十二月己酉,大理衛大風,壞城樓。十三年三月甲寅,慶符大風雹,壞學宮。十六年十二月辛卯,甘肅行都司狂風,壞官民廬舍樹木無算。嘉靖元年七月己巳,南京暴風雨,郊社、陵寢、宮闕、城垣獸吻、脊欄皆壞,拔樹萬餘株。五年,陜西屢發大風,捲掣廟宇、民居百數十家,了無蹤跡。萬曆十八年三月甲辰,大名狂風,天色乍黑乍赤。二十六年十月癸亥,喜峰路臺西北樓內,旋風大作,黑氣沖天,樓內有火光。三十四年七月丙戌,大風拔朝日壇樹。四十一年八月乙未,青州大風拔樹,傾城屋。天啟元年三月辛亥,大風揚塵四塞。四年五月癸亥,乾清宮東丹墀旋風驟作,內官監鐵片大如屋頂者,盤旋空中,隕於西墀,鏗訇若雷。八月戊戌,薊州寒風殺人。崇禎十四年五月,南陽大風拔屋。七月乙亥,福州大風,壞官署、民舍。十五年五月,保定廣平諸縣怪風,麥禾俱傷。十六年正月丁酉,大風,五鳳樓前門閂風斷三截,建極殿榱桷俱折。

▲風霾晦冥

建文元年七月癸酉,燕王起兵,風雲四起,咫尺不辨人。少焉東方露青天尺許,有光燭地,洞徹上下。天順八年二月壬子,風霾晝晦。成化六年二月丁丑,開封晝晦如夜,黃霾蔽天。三月辛巳,雨霾晝晦。九年三月癸未,濟南諸府狂風晝晦,咫尺莫辨。二十一年三月戊子,大名風霾,自辰迄申,紅黃滿空,俄黑如夜。已而雨沙,數日乃止。京師自正月至三月,風霾不雨。弘治二年二月辛亥,開封晝晦如夜。三月,黃塵四塞,風霾蔽天者累日。四年八月乙卯,南京晦冥。七年三月己亥,廣寧諸衛晝晦。正德五年三月甲子,大風霾,天色晦冥者數日。十六年十一月辛酉,甘肅行都司黑風晝晦,翌日方散。嘉靖元年九月己巳,大風霾,晝晦。八年正月戊戌朔,風霾,晦如夕。二十六年七月乙丑,甘州五衛風霾晝晦,色赤復黃。二十八年三月丙申,風霾四塞,日色慘白,凡五日。三十年正月辛卯,大風揚塵蔽天,晝晦。四十年二月己酉,亦如之。四月癸巳,大風雨,黃土晝晦。四十三年三月望,異風作,赤黃霾,至二十一日乃止。隆慶二年正月元旦,大風揚沙走石,白晝晦冥,自北畿抵江、浙皆同。萬曆十七年正月乙丑,蓋州衛風霾晝晦,壞廨宇、廬舍。二十五年二月戊寅,京師風霾。二十九年四月,連日風霾。三十八年四月戊戌,崇陽風霾晝晦,至夜轉烈,損官民屋木無算。四十八年八月以前,雲南諸府時晝晦。天啟元年四月乙亥午,寧夏洪廣堡風霾大作,墜灰片如瓜子,紛紛不絕,踰時而止。日將沈,作紅黃色,外如炊煙,圍罩畝許,日光所射如火焰,夜分乃沒。四年二月辛丑,風霾晝晦,塵沙蔽天,連日不止。崇禎元年正月癸亥,永年縣晝晦,咫尺不辨人物。七年三月戊子,黃州晝晦如夜。十三年閏正月丙申,南京日色晦朦,風霾大作,細灰從空下,五步外不見一物。後四年三月丙申,風霾晝晦。

▲花孽

弘治十六年九月,安陸桃李華。正德元年九月,宛平棗林莊李花盛開。其冬,永嘉花盡放。六年八月,霸州桃李華。

▲蟲孽

景泰五年三月,畿南五府有蟲食桑,春蠶不育。弘治六年八月己巳,臨晉雨蟲如雪。七年三月,廣寧諸衛有黑蟲墮地,大如蠅,久之入於土。

▲牛禍

正德十二年,徐州牛產犢,一頭二舌,兩尾八足。嘉靖五年七月,南陽牛產犢,一首兩身。六年十一月,漳浦有牛產犢,三目三角。十一年二月,銅仁黃牸產犢,滿身有紋,即死。十二年,山東平山衛牛犢有紋,前兩足及尾悉具鱗甲,中皆毳毛。萬曆十三年九月,光山牛產一物,火光滿地,鱗甲森然,一夕斃。三十七年五月,歷城、高苑二縣牛各產犢,雙頭三眼,兩鼻二口。三十八年三月,獲嘉牛產犢,一身兩頭,四眼四耳,兩口兩足,一尾。三十九年二月,汲縣牛產犢,一膊兩頭,兩口四眼,兩耳七蹄。四月降夷部牛產犢,人頭羊耳。四十五年八月,開州牛產犢,兩口三眼。天啟元年十月,會寧牛產異獸,遍體鱗甲,有火光。三年十月,沅陵牸生犢,一身兩頭三尾。七年三月,莒州牛產犢如麟。崇禎十三年,襄陽牛產犢,兩頭二日。

▲地震

洪武四年正月己丑。鞏昌、臨洮、慶陽地震。五年四月戊戌,梧州府蒼梧、賀州、恭城、立山等處地震。六月癸卯,太原府陽曲縣地震。七月辛亥,又震。壬戌,京師風雨地辰。八月癸未,太原府徐溝縣西北中有聲如雷,地震凡三日。戊戌,陽曲縣地又震。九月壬戌,又震者再。十月戊寅、辛卯,復震。是年,陽曲地凡七震。自六年至十四年,復八震。八年七月戊辰,京師地震。十二月戊子,又震。十一年四月乙巳,寧夏地震,壞城垣。十三年二月甲戌,福州府、廣州府、河州地震。十九年六月辛丑,雲南地震。十一月己卯,復震,有聲。二十三年正月庚辰,山東地震。

建文元年三月甲午,京師地震,求直言。

永樂元年十一月甲午,北京地震。山西、寧夏亦震。二年十一月癸丑,京師、濟南、開封並震,有聲。六年五月壬戌、十一年八月甲子,京師復震。十三年九月壬戌、十四年九月癸卯,京師地震。十八年六月丙午,北京地震。二十二年六月壬申,南京地震。

洪熙元年二月戊午,六安衛地震,凡七日。是歲,南京地震,凡四十有二。

宣德元年七月癸巳,京師地震,有聲,自東南迄西北。是歲,南京地震者九。二年春,復震者十。三年,復屢震。四年,兩京地震。五年正月壬子,南京地震。辛酉,又震。

正統三年三月己亥,京師地震。庚子,又震。甲辰,又震者再。四年六月乙未,復震。八月己亥,又震。五年十月庚午朔,蘭州、莊浪地震十日。十月、十一月屢震,壞城堡廬舍,壓死人畜。十年二月丁巳,京師地震。

景泰二年七月癸丑,京師地震。三年七月,永新珠坑村地陷十七所。是年,南京地震。五年十月庚子,京師地震,有聲,起西北迄東南。六年二月甲午,安福大雷雨。白泉陂羊塘地陷二,一深三丈,廣十餘丈,一深六尺,廣一丈有奇。

天順元年十月乙巳,南京地震。

成化元年四月甲申,鈞州地震,二十三日乃止。三年,四川地震,凡三百七十五。五月壬申,宣府、大同地震,有聲,威遠、朔州亦震,壞墩臺牆垣,壓傷人。四年八月癸巳,京師地震,有聲。十二月戊戌,湖廣地震。五年十二月丙辰,汝寧、武昌、漢陽、岳州同日地震。六年正月丁亥,河南地震。是年,湖廣亦震。十年四月壬午,鶴慶地震。九月己巳,自寅至申,復十五震,壞廨舍民居,傷人畜。十月丁酉,靈州大沙井驛地震,有聲如雷。自後晝夜屢震,至十一月甲寅,一日十一震,城堞房屋多圮。十二年正月辛亥,南京地震。十月辛巳,京師地震。十三年正月己巳,鳳陽、臨淮地震,有聲。閏二月癸卯,臨洮、鞏昌地震,城有頹者。四月戊戌,甘肅地裂,又震,有聲。榆林、涼州亦震。寧夏大震,聲如雷。城垣崩壞者八十三處。甘州、鞏昌、榆林、涼州及沂州、郯城、滕、費、嶧等縣,同日俱震。九月甲戌,京師地三震。十四年六月,廣西太平府地震,至八月乙巳,凡七震。七月,四川鹽井衛地連震,廨宇傾覆,人畜多死。十六年八月丁巳,四川越巂衛一日七震,越數日連震。十七年二月甲寅,南京、鳳陽、廬州、淮安、揚州、和州、兗州及河南州縣,同日地震。五月戊戌,直隸薊州遵化縣地震。六月甲辰,又震,日三次。永平府及遼東寧遠衛亦三震。二十年正月庚寅,京師及永平、宣府、遼東皆震。宣府地裂,湧沙出水。天壽山、密雲、古北口、居庸關城垣墩堡多摧,人有壓死者。五月甲寅,代州地七震。九月辛巳,費縣地陷,深二尺,縱橫三丈許。二十一年二月壬申,泰安地震。三月壬午朔,復震,聲如雷,泰山動搖。後四日復微震,癸巳、乙未、庚子連震。閏四月癸未,鞏昌府、固原衛及蘭、河、洮、岷四州,地俱震,有聲。癸巳,薊州遵化縣地震,有聲,越數日復連震,城垣民居有頹仆者。五月壬戌,京師地再震。九月丙辰,廉州、梧州地震,有聲,連震者十六日。十一月丙寅,京師地震。二十二年六月壬辰,漢中府及寧羌衛地裂,或十餘丈,或六七丈。寶雞縣裂三里,闊丈餘。九月辛亥,成都地日七八震,俱有聲。次日,復震。

弘治元年八月壬寅,漢、茂二州地震,仆黃頭等寨碉房三十七戶,人口有壓死者。戊申,宣府葛峪堡地陷深三尺,長百五十步,闊一丈。沙河中湧疄,高一尺,長七十步。十二月辛卯,四川地震,連三日。二年五月庚申,成都地震,連三日,有聲。三年十二月己未,京師地再震。四年六月辛亥,復三震。八月乙卯,南京地震,屋宇皆搖。淮、揚二府同日震。六年三月,寧夏地震,連三年,共二十震。四月甲辰,開封、衛輝、東昌、兗州同日地震,有聲。七年二月丁丑,曲靖地震,壞房屋,壓死軍民。是歲,兩京並六震。八年三月己亥,寧夏地震十二次,聲如雷,傾倒邊牆、墩臺、房屋,壓傷人。九月甲午至辛丑,安南衛地十二震。十月壬戌至甲子,海州九震。是歲,南京地再震。九年,兩京地震者各二次。十年正月戊午,京師、山西地震。六月乙亥,海豐地震,聲如雷,數日乃止。是歲,真定、寧夏、榆林、鎮番、靈州、太原皆震。屯留尤甚,如舟將覆,屋瓦皆落。十一年六月丙子,桂林地有聲若雷,旋陷九處,大者圍十七丈,小者七丈或三丈。十三年七月己巳,京師地震。十月戊申,兩京、鳳陽同時地震。十四年正月庚戌朔,延安、慶陽二府,同、華諸州,咸陽、長安諸縣,潼關諸衛,連日地震,有聲如雷。朝邑尤甚,頻震十七日,城垣、民舍多摧,壓死人畜甚眾。縣東地拆,水溢成河。自夏至冬,復七震。是日,陜州,永寧、盧氏二縣,平陽府安邑、榮河二縣,俱震,有聲。蒲州自是日至戊午連震。丁丑,福、興、泉、漳四府地俱震。二月乙未,蒲州地又震,至三月癸亥,凡二十九震。八月癸丑,四川可渡河巡檢司地裂而陷,湧泉數十派,衝壞橋梁、莊舍,壓死人畜甚眾。癸酉,貴州地三震。十月辛酉,南京地震。十五年九月丙戌,南京、徐州、大名、順德、濟南、東昌、兗州同日地震,壞城垣、民舍。濮州尤甚,地裂湧水,壓死百餘人。是日,開封、彰德、平陽、澤、潞亦震。十月甲子,山西應、朔、代三州,山陰、馬邑、陽曲等縣,地俱震,聲如雷。丁卯,南京地震。十六年二月庚申,南京地震。十八年六月癸亥,寧夏地震,聲如雷,城傾圮。九月癸巳,杭、嘉、紹、寧四府地震,有聲。甲午,南京及蘇、松、常、鎮、淮、揚、寧七府,通、和二州,同日地震。辛丑,蒲、解二州,絳、夏、平陸、榮河、聞喜、芮城、猗氏七縣地俱震,有聲。而安邑、萬全尤甚,民有壓死者。

正德元年二月癸酉至乙亥,觔陽地震者十餘,有聲如雷。四月癸丑,雲南府連日再震。木密關地震如雷凡五,壞城垣、屋舍,壓傷人。八月丁巳,萊州府鰲山衛地震,聲如雷,城垛壞,以後屢震。萊州自九月至十二月,地震四十五,俱有聲如雷。二年九月庚午,雲南府安州、新興州三日連震,搖撼民居,人有死者。四年三月甲寅,廣寧大興堡地陷,長四尺,寬三尺,深四丈餘。五月己亥夜,武昌見碧光如電者六,有聲如雷,已而地震。六年四月乙未,楚雄地三日五震,至明年五月又連震十三日。十月甲辰,大理府鄧川州、劍川州、洱海衛地震。鶴慶、劍川尤甚,壞城垣、房廨,人有壓死者。十一月戊午,京師地震。保定、河間二府及八縣三衛,山東武定州,同日皆震。霸州連三日十九震。七年五月壬子,楚雄府自是日至甲子,地連震,聲如雷。八月己巳,騰衝衛地震兩日,壞城樓、官民廨宇。赤水湧出,田禾盡沒,死傷甚眾。八年十二月戊戌,成都、重慶二府,潼川、邛二州,地俱震。九年六月甲辰,鳳陽府地震有聲。八月乙巳,京師大震。十月壬辰,敘州府,太原府代、平、榆次等十州縣,大同府應州山陰、馬邑二縣,俱地震,有聲。十年五月壬辰,雲南趙州永寧衛地震,踰月不止,有一日二三十震者。黑氣如霧,地裂水湧,壞城垣、官廨、民居不可勝計,死者數千人,傷倍之。八月丁丑,大理府地震,至九月乙未,復大震四日。十一年八月戊辰,南京地震,武昌府亦震。十二月己未,楚雄、大理二府,蒙化、景東二衛俱震。十二年四月甲子,撫州府及餘干、豐城二縣,泉州府,俱地震。浙江金鄉衛自是日至七月己丑,凡十有五震。六月戊辰,雲南新興州及通海、河西、山習峨諸縣地震,壞城樓、房屋,民有壓死者。九月己卯,濟、青、登、萊四府地震。是歲,泉州二月至六月,金華二月至七月,皆數震。十三年六月己巳,大理府及趙、鄧川二州,浪穹縣地震。是日,蒙化府亦震。十月甲午、十一月癸卯,又震。十四年二月丁丑,京師地震。九月丙午,昌平州、宣府、開平等衛亦震。丙辰,福、興、泉三府地震。十五年三月丙申,安寧、姚安、賓州、蒙化、鶴慶俱地震。蒙化震二日,仆城垣、廬舍,民有壓死者。八月辛酉,景東衛地震,聲如雷。搖仆城牆、廨宇,地多拆裂。乙丑,濟南、東昌、開封地震。

嘉靖二年正月,南京、鳳陽、山東、河南、陜西地震。七月壬申,浙江定海諸衛地震,城堞盡毀。三年正月丙寅朔,兩畿、河南、山東、陜西同時地震。二月辛亥,蘇、常、鎮三府地震。是年,南京震者再。四年八月癸卯,徐州、鳳陽一衛三州縣及懷慶、開封二府俱地震,聲如雷。九月壬申,鳳陽、徐州及開封二縣復震。五年四月癸亥,永昌、騰沖、騰越同日地震。貴州安南衛地震,聲如雷。壞城垣。壬申,復震。六年十月戊辰,京師地震。十二年八月丁酉,京師地震。十五年十月庚寅,京師地震。順天、永平、保定、萬全都司各衛所,俱震,聲如雷。十六年九月癸酉,雲南地震。十八年七月庚寅,楚雄、臨安、廣西地震。十九年四月庚午,洮州、甘肅俱震。二十一年九月甲戌,平陽、固原、寧夏、洮州同日地震,有聲。十一月丁巳,鞏昌、固原、西安、鳳翔地震。二十二年三月乙巳,太原地震,有聲,凡十日。明年三月,復如之。四月庚辰,福、興、泉、漳四府地震。二十三年三月朔,太原地震有聲者十日。二十七年七月戊寅,京師地震,順天、保定二府俱震。八月癸丑,京師復震,登州府及廣寧衛亦震。三十年九月乙未,京師地震,有聲。三十一年二月癸亥,鳳陽府地震,有聲。三月丙戌,山西地震,有聲。三十四年十二月壬寅,山西、陜西、河南同時地震,聲如雷。渭南、華州、朝邑、三原、蒲州等處尤甚。或地裂泉湧,中有魚物,或城郭房屋,陷入地中,或平地突成山阜,或一日數震,或累日震不止。河、渭大泛,華嶽、終南山鳴,河清數日。官吏、軍民壓死八十三萬有奇。三十七年正月庚申,陜西地震。三月丁丑,昌平州地震。五月丁卯,蒲州地連震三日,聲如雷。六月甲申,又震。十月丙午,華州地震,聲如雷。至壬子又震,戊午復大震,傾陷廬舍甚多。三十八年七月辛巳,南京地震,有聲。三十九年四月,嘉興、湖州地震,屋廬搖動如帆。河水撞激,魚皆躍起。四十年二月戊戌,甘肅山丹衛地震,有聲,壞城堡廬舍。六月壬申,太原、大同、榆林地震,寧夏、固原尤甚。城垣、墩臺、府屋皆摧,地湧黑黃沙水,壓死軍民無算,壞廣武、紅寺等城。四十一年正月丙申,京師地震。是歲,寧夏地震,圮邊牆。四十五年正月癸巳,福建福、興、泉三府同日地震。

隆慶二年三月甲寅,陜西慶陽、西安、漢中、寧夏,山西蒲州、安邑,湖廣鄖陽及河南十五州縣,同日地震。戊寅,京師地震。是日,山東登州、四川順義等縣同日震。樂亭地裂三丈餘者二,黑沙水湧出。寧遠城崩。四月癸未,懷慶、南陽、汝寧、寧夏同日地震。乙酉,鳳翔、平涼、西安、慶陽地震,壞城傷人。七月辛酉,陵川地裂三十餘步。三年十一月庚辰,京師地震。四年四月戊戌,京師地震。五年二月丙午,廣西靖江王府及宗室所居、布政司官署,俱地陷。六月辛卯朔,京師地震者三。

萬曆元年八月戊申,荊州地震,至丙寅方止。二年二月癸亥,長汀地震,裂成坑,陷沒民居。三年二月甲戌,湖廣、江西地震。五月戊戌朔,襄陽、鄖陽及南陽府屬地震三日。己亥,信陽亦震。六月戊子,福、汀、漳等府及廣東之海陽縣俱地震。九月戊午,京師地震。十月丁卯,又震。己卯,岷州衛地震。己丑至壬午,連百餘震。四年二月庚辰,薊、遼地震。辛巳,又震。五年二月辛巳,騰越地二十餘震,次日復震。山崩水湧,壞廟廡、倉舍千餘間,民居圮者十之七,壓死軍民甚眾。六年二月辛卯,臨桂村田中青煙直上,隨裂地丈餘,鼓聲轟轟,民居及大樹石皆陷。七年七月戊午,京師地震。八年五月壬午,遵化數震,七日乃止。七月甲午,井坪路地大震,摧城垣數百丈。九年四月己酉,蔚州地震,聲如雷。房屋震裂。大同鎮堡各州縣,同時地震,有聲。十一年二月戊子,承天府地震。十二年二月丁卯,京師地震。五月甲午,又震。十三年二月丁未,淮安、揚州、廬州及上元、江寧、江浦、六合俱地震。江濤沸騰。三月戊寅,山西山陰縣地震,旬有五日乃止。八月己酉,京師地震。十四年四月癸酉,又震。十五年三月壬辰,開封府屬地震者三,彰德、衛輝、懷慶同日震。五月,山西地震。十六年六月庚申,京師地再震。十七年七月己未,杭州、溫州、紹興地震。十八年六月丙子,甘肅臨洮地震,壞城郭、廬舍,壓死人畜無算。八月,福建地屢震。十九年閏三月己巳,昌平州地震。十月戊戌,山丹衛地震,壞城垣。二十三年五月丁酉,京師地震。十二月癸亥,陜西地震,聲若雷。二十四年十一月,福建地震。二十五年正月壬辰朔,四川地震三日。八月己卯,遼陽、廣寧諸衛地震,湧水三日。甲申,京師地震,宣府、薊鎮等處俱震。十二月乙酉,京師地震。二十六年正月丁亥朔,寧夏地震。次日,長樂地陷五丈。八月丁丑,京師地震,有聲。二十七年七月辛未,承天、沔陽、岳州地震。二十八年二月戊寅,京師地震,自艮方西南行,如是者再。三十一年四月丙午,承天府鐘祥縣地震,房屋摧裂。五月戊寅,京師地震。三十二年閏九月庚辰,鞏昌及醴泉地一日十餘震,城郭民居並摧。白陽、吳泉界地裂三丈,溢出黑水,搏激丈餘。三十三年五月辛丑,陸川地震,有聲,壞城垣、府屋,壓死男婦無算。六月庚午,靈川社壇有聲,陷地十餘丈,深丈餘。九月丙申,京師地震者再,自東北向西南行。三十四年六月丙辰,陜西地震。三十五年七月乙卯,松潘、茂州、汶川地震數日。三十六年二月戊辰,京師地震。七月丁酉,又震。三十七年六月辛酉,甘肅地震,紅崖、清水諸堡壓死軍民八百四十餘人,圮邊墩八百七十里,裂東關地。四十年二月乙亥,雲南大理、武定、曲靖地大震,次日又震。緬甸亦震。五月戊戌,雲南大理、曲靖復大震,壞房屋。四十二年九月庚午,山西、河南地震。四十三年二月己卯,揚州地震。狼山寺殿壞塔傾。八月乙亥,楚雄地震如雷,人民驚殞。十月辛酉,京師地震。四十五年五月甲戌,鳳陽府地震。乙亥,復震。八月,濟南地裂者二。四十六年六月壬午,京師地震。九月乙卯,京師地再震,畿輔、山西州縣一十有七及紫荊關,馬水、沿河二口,偏頭、神池同日皆震。四十八年二月庚戌,雲南及肇慶、惠州、荊州、襄陽、承天、沔陽、京山皆地震。

天啟元年四月癸丑,延綏孤山城陷三十五丈,入地二丈七尺。二年二月癸酉,濟南、東昌、河南、海寧地震。三月癸卯,濟南、東昌屬縣八,連震三日,壞民居無數。九月甲寅,平涼、隆德諸縣,鎮戎、平虜諸所,馬剛、雙峰諸堡,地震如翻,壞城垣七千九百餘丈,屋宇萬一千八百餘區,壓死男婦萬二千餘口。十一月癸卯,陜西地震。三年四月庚申朔,京師地震。十月乙亥,復震。閏十月乙卯,雲南地震。十二月丁未,南畿六府二州俱地震,揚州府尤甚。是月戊戌,京師地又震。四年二月丁酉,薊州、永平、山海地屢震,壞城郭廬舍。甲寅,樂亭地裂,湧黑水,高尺餘。京師地震,宮殿動搖有聲,銅缸之水,騰波震盪。三月丙辰、戊午,又震。庚申,又震者三。六月丁亥,保定地震,壞城郭,傷人畜。八月己酉,陜西地震。十二月癸卯,南京地震。六年六月丙子,京師地震。濟南、東昌及河南一州六縣同日震。天津三衛、宣府、大同俱數十震,死傷慘甚。山西靈丘晝夜數震,月餘方止。城郭、廬舍並摧,壓死人民無算。七月辛未,河南地震。九月甲戌,福建地震。十二月戊辰,寧夏石空寺堡地大震。礌山石殿傾倒,壓死僧人。是年,南京地亦震。七年,寧夏各衛營屯堡,自正月己巳至二月己亥,凡百餘震,大如雷,小如鼓如風,城垣、房屋、邊牆、墩臺悉圮。十月癸丑,南京地震,自西北迄東南,隆隆有聲。

崇禎元年九月丁卯,京師地震。三年九月戊戌,南京地震。四年六月乙丑,臨洮、鞏昌地震,壞廬舍,損民畜。五年四月丁酉,南京、四川地震。十月丁卯,山西地震。十一月甲寅,雲南地震。六年正月丁巳,鎮江地裂數丈。七月戊戌,陜西地震。八年冬,山西地震。九年三月戊辰,福建地震。七月丁未,清江城陷。十年正月丙午,南京地震。七月壬午,雲南地震。十月乙卯,四川地震。十二月,陜西西安及海剌同時地震,數月不止。十一年九月壬戌,遼東地震。十二年二月癸巳,京師地震。十三年十一月戊子,南京地震。十四年三月戊寅,福建地震。四月丙寅,湖廣地震。五月戊子,甘肅地震。六月丙午,福建地震。九月甲午,四川地震。十五年五月丙戌,兩廣地震。七月甲申,山西地震。十六年九月,鳳陽地屢震。十一月丙申,山東地震。明年正月庚寅朔,鳳陽地震。乙卯,南京地震。三月辛卯,廣東地震。

▲山頹

洪武六年正月壬戌夜,伏羌高山崩。正統八年十一月,浙江紹興山移於平田。是歲,陜西二處山崩。十三年,陜西夏秋霪雨,通渭、平涼、華亭三縣山傾,軍民壓死者八十餘口。天順四年十月,星子山裂。成化八年七月,隴州北山吼三日,裂成溝,長半里,尋復合。十六年四月壬子,巨津州金沙江北岸白石雪山斷裂里許,兩岸山合,山上草木如故。下塞江流,禾黍盡沒。久之其下漸開,水始泄。六月,長樂平地出小阜,人畜踐之輒陷。明年,復湧一高山。十七年十二月辛丑,壽陽縣城南山崩,聲如牛吼。弘治三年六月乙巳,河州山崩地陷。九年六月庚寅,山陰、蕭山二縣同日大雨山崩。十四年閏七月,烏撒軍民府大雨山崩。十五年八月戊申,宣府合河口石山崩。十八年六月丙子,河州沙子溝夜大雷雨,石岸山崩,移七八里,崩處裂為溝,田廬民畜俱陷。正德元年十二月癸亥,即墨三表山石崩。四年三月甲寅,遼東東山大家峪山崩二處,約丈餘。五年六月癸巳,秦州山崩,傷室廬、禾稼甚眾。龍王溝口山亦崩。六年七月丙寅,夔州麞子溪驟雨,山崩。十三年五月癸亥,雲南黑鹽井山崩,井塞。十五年八月丁丑,雲南趙州大雨,山崩。嘉靖四年七月乙酉,清源賈家山崩。五年四月壬申,貴州歹蘇屯山崩。十九年十二月己巳,峨眉宋皇觀山鳴,震裂,湧泉水八日。二十一年六月乙酉,歸州沙子嶺大雷雨,崖石崩裂,塞江流二里許。二十六年七月癸酉,澄城麻陂山界頭嶺,晝夜吼數日。山忽中斷,移走,東西三里,南北五里。隆慶二年五月庚戌,永寧州山崩。是歲,樂亭地裂三處,俱湧黑沙水。四年八月,湖州山崩,成湖。萬曆二十五年六月,泰山崩。二十七年八月甲午,狄道城東山崩,其下衝成一溝。山南耕地湧出大小山五,高二十餘丈。三十三年八月丙午,鎮江西南華山裂二三尺。三十七年六月辛酉,甘肅南山崩。天啟三年閏十月乙卯,仁壽長山聲震如雷,裂七里,寬三尺,深不可測。崇禎九年十二月,鎮江金雞嶺土山崩。後八年,秦州有二山,相距甚遠,民居其間者數百萬家。一日地震,兩山合,居民並入其中。

▲雨毛、地生毛

洪武十九年九月丙子,天雨絮。宣德元年七月甲午,地生毛,長尺餘。正統八年,浙江地生白毛。成化十三年四月,甘肅地裂,生白毛。十五年五月,常州地生白毛。十七年四月,南京地生白毛。弘治元年五月丙寅,瀘州長寧縣雨毛。正德十二年四月,金華地生黑白毛,長尺餘。

▲年饑

洪武二年,湖廣、陜西饑。四年,陜西洊饑。五年,濟南、東昌、萊州大饑,草實樹皮,食為之盡。六年,蘇州、揚州、真定、延安饑。七年,北平所屬州縣三十三饑。十五年,河南饑。十九年春,河南饑。夏,青州饑。二十年,山東三府饑。二十三年,湖廣三府、二州饑。二十四年,山東及太原饑,徐、沛民食草實。二十五年,山東洊饑。

永樂元年,北畿、山東、河南及鳳陽、淮安、徐州、上海饑。二年,蘇、松、嘉、湖四府饑。四年,南畿、浙江、陜西、湖廣府州縣衛十四饑。五年,順天、保定、河間饑。十年,山東饑。十二年,直省州縣二十四饑。十三年,順天、青州、開封三府饑。十四年,平陽、大同二府饑。十八年,青、萊二府大饑。時皇太子赴北京,過鄒縣,命亟發官粟以賑。

洪熙元年,北畿饑。山東、河南、湖廣及南畿州縣三十四饑。

宣德元年,直省州縣二十九饑。二年,直省縣十四饑。三年,直省州縣十五饑。六年直省縣十饑。八年,以水旱告饑者,府州縣七十有六。九年,南畿、山東、浙江、陜西、山西、江西、四川多告饑,湖廣尤甚。十年,揚、徐、滁、南昌大饑。

正統三年春,平涼、鳳翔、西安、鞏昌、漢中、慶陽、兗州七府及南畿三州、二縣,江西、浙江六縣饑。四年,直省州縣衛十八及山西隰州、大同、宣府、偏頭諸關饑。五年,直省十府、一州、二縣饑。陜西大饑。六年,直省州縣二十六饑。八年夏,湖南饑。秋,應天、鎮江、常州三府饑。九年春,蘇州府饑。是歲,雲南、陜西乏食。十年,陜西、山西饑。十二年夏,淮安、岳州、襄陽、荊州、郴州俱洊饑。十三年,寧、紹二府及州縣七饑。

景泰元年,大名、順德、廣平、保定、處州、太原、大同七府饑。二年,大名、廣平又饑。順天、保定、西安、臨洮、太原、大同、解州饑。三年,淮、徐大饑,死者相枕藉。四年,徐州洊饑。河南、山東及鳳陽饑。五年,兩畿十府饑。六年春,兩畿、山東、山西、浙江、江西、湖廣、雲南、貴州饑,蘇、松尤甚。七年,北畿、山東、江西、雲南又饑。河南亦饑。

天順元年,北畿、山東並饑,發塋墓,斫道樹殆盡。父子或相食。二年,長沙、辰州、永州、常德、岳州五府及銅鼓、五開諸衛饑。四年,湖廣及鎮遠府,都勻、平越諸衛饑。六年,陜西饑。

成化元年,兩畿、浙江、河南饑。二年,南畿饑。四年,兩畿、湖廣、山東、河南無麥。鳳陽及陜西、寧夏、甘、涼饑。五年,陜西洊饑。六年,順天、河間、真定、保定四府饑,食草木殆盡。山西、兩廣、雲南並饑。八年,山東饑。九年,山東又大饑,骼無餘胔。十三年,南畿、山東饑。十四年,北畿、湖廣、河南、山東、陜西、山西饑。十五年,江西饑。十六年,北畿、山東、雲南饑。十八年,南畿、遼東饑。十九年,鳳陽、淮安、揚州三府饑。二十年,陜西饑,道殣相望。畿南及山西平陽饑。二十一年,北畿、山東、河南饑。二十三年,陜西大饑。武功民有殺食宿客者。淮北、山東亦饑。

弘治元年,應天及浙江饑。六年,山東饑。七年,保定、真定、河間三府饑。八年,蘇、松、嘉、湖四府饑。十四年,順天、永平、河間、河南四府饑。遼東大饑。十五年,遼東洊饑。兗州饑。十六年,浙江、山東及南畿四府、三州饑。十七年,淮、揚、廬、鳳洊饑,人相食,且發瘞胔以繼之。十八年,延安諸府饑。

正德三年,廬、鳳、淮、揚四府饑。四年,蘇、松、常、鎮四府饑。五年,山東饑。七年,嘉興、金華、溫、台、寧、紹六府乏食。八年,河間、保定饑。九年春,永平諸府饑,民食草樹殆盡,有闔室死者。秋,關、陜亦饑。十一年,順天、河間饑。河南大饑。十二年春,順天、保定、永平饑。十三年,蘇、松、廬、鳳、淮、揚六府饑。十四年冬,遼東饑,南畿、淮、揚諸府尤甚。十六年,遼東饑。

嘉靖二年,應天及滁州大饑。三年,湖廣、河南、大名、臨清饑。南畿諸郡大饑,父子相食,道殣相望,臭彌千里。四年,河間、沈陽、大同三衛饑。五年,順天、保定、河間三府大饑。六年,遼東大饑。八年,真定、廬、鳳、淮、揚五府,徐、滁、和三州及山東、河南、湖廣、山西、陜西、四川饑,襄陽尤甚。九年,畿內、河南、湖廣、山東、山西大饑。十二年,北畿、山東饑。十五年,湖廣大饑。十七年,北畿饑。河南、鄖陽、襄陽三府饑。二十年,保定、遼東饑。二十一年,順天、永平饑。二十四年,又饑。南畿亦饑。二十五年,順天饑,江西亦饑。二十七年,鞏昌、漢中大饑。三十一年,宣、大二鎮大饑,人相食。三十二年,南畿、廬、鳳、淮、揚、山東、河南、陜西並饑。三十三年,順天及榆林饑。三十六年,遼東大饑,人相食。三十九年,順天、永平饑。四十年,兩畿、山西饑。四十三年,北畿、山東大饑。四十四年,順天饑。四十五年,淮、徐饑。

隆慶元年,蘇、松二府大饑。二年,湖廣饑。

萬曆元年,淮、鳳二府饑,民多為盜。十年,延安、慶陽、平涼、臨洮、鞏昌大饑。十三年,湖廣饑。十五年七月,黃河以北,民食草木。富平、蒲城、同官諸縣,有以石為糧者。十六年,河南饑,民相食。蘇、松、湖三府饑。二十二年,河南大饑,給事中楊明繪《饑民圖》以進,巡按陳登雲進饑民所食雁糞,帝覽之動容。二十八年,山東及河間饑。二十九年,兩畿饑。阜平縣饑,有食其稚子者。蘇州饑,民毆殺稅使七人。三十七年,山西饑。四十年,南畿洊饑,鳳陽尤甚。四十三年,浙江饑。四十四年,山東饑甚,人相食。河南及淮、徐亦饑。四十五年,北畿民食草木,逃就食者,相望於道。山東屬邑多饑。四十六年,陜西饑。四十八年,湖廣大饑。

崇禎元年,陜西饑,延、鞏民相聚為盜。二年,山西、陜西饑。五年,淮、揚諸府饑,流殍載道。六年,陜西、山西大饑。淮、揚洊饑,有夫妻雉經於樹及投河者。鹽城教官王明佐至自縊於官署。七年,京師饑,御史龔廷獻繪《饑民圖》以進。太原大饑,人相食。九年,南陽大饑,有母烹其女者。江西亦饑。十年,浙江大饑,父子、兄弟、夫妻相食。十二年,兩畿、山東、山西、陜西、江西饑。河南大饑,人相食,盧氏、嵩、伊陽三縣尤甚。十三年,北畿、山東、河南、陜西、山西、浙江、三吳皆饑。自淮而北至畿南,樹皮食盡,發瘞胔以食。十四年,南畿饑。金壇民於延慶寺近山見人云,此地深入尺餘,其土可食。如言取之,淘磨為粉粥而食,取者日眾。又長山十里亦出土,堪食,其色青白類茯苓。又石子澗土黃赤,狀如豬肝,俗呼「觀音粉」,食之多腹痛隕墜,卒枕藉以死。是歲,畿南、山東洊饑。德州斗米千錢,父子相食,行人斷絕。大盜滋矣。

▲黃眚黃祥

正統十一年二月辛酉,有異氣現華蓋殿金頂及奉天殿鴟吻之上。成化九年四月乙亥,兩京雨土。十三年四月戊戌,陜西、甘肅冰厚五尺,間以雜沙,有青紅黃黑四色。弘治十年三月己酉,雨土。十一年四月辛巳,雨土。十七年二月甲辰,鄖陽、均州雨沙。嘉靖元年正月丁卯,雨黃沙。十三年二月己未,雨微土。二十一年,象山雨黃霧,行人口耳皆塞。隆慶元年三月甲寅,南鄭雨土。萬歷二十五年二月癸亥,湖州雨黃沙。四十六年三月庚午,暮刻,雨土,濛濛如霧如霰,入夜不止。四十七年二月甲戌,從未至酉,塵沙漲天,其色赤黃。四十八年,山東省城及泰安、肥城皆雨土。崇禎十二年二月壬申,浚縣有黑黃雲起,旋分為二。頃之四塞。狂風大作,黃埃漲天,間以青白氣。五步之外,不辨人蹤,至昏始定。十四年正月壬寅,黃埃漲天。



\end{pinyinscope}