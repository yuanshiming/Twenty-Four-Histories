\article{列傳第一百十七}

\begin{pinyinscope}
劉臺馮景隆孫繼先傅應禎王用汲吳中行子亮元從子宗達趙用賢孫士春艾穆喬璧星葉春及沈思孝丁此呂

劉臺,字子畏,安福人。隆慶五年進士。授刑部主事。萬歷初,改御史。巡按遼東,坐誤奏捷,奉旨譙責。四年正月,臺上疏劾輔臣張居正,曰:

臣聞進言者皆望陛下以堯、舜,而不聞責輔臣以皋、夔。何者?陛下有納諫之明,而輔臣無容言之量也。高皇帝鑒前代之失,不設丞相,事歸部院,勢不相攝,而職易稱。文皇帝始置內閣,參預機務。其時官階未峻,無專肆之萌。二百年來,即有擅作威福者,尚惴惴然避宰相之名而不敢居,以祖宗之法在也。乃大學士張居正偃然以相自處,自高拱被逐,擅威福者三四年矣。諫官因事論及,必曰:「吾守祖宗法。」臣請即以祖宗法正之。

祖宗進退大臣以禮。先帝臨崩,居正托疾以逐拱,既又文致之王大臣獄。及正論籍籍,則抵拱書,令勿驚死。既迫逐以示威,又遺書以市德,徒使朝廷無禮於舊臣。祖宗之法若是乎?

祖宗朝,非開國元勛,生不公,死不王。成國公朱希忠,生非有奇功也,居正違祖訓,贈以王爵。給事中陳吾德一言而外遷,郎中陳有年一爭而斥去。臣恐公侯之家,布賄厚施,緣例陳乞,將無底極。祖宗之法若是乎?

祖宗朝,用內閣塚宰,必由廷推。今居正私薦用張四維、張瀚。四維在翰林,被論者數矣。其始去也,不任教習庶吉士也。四維之為人也,居正知之熟矣。知之而顧用之,夫亦以四維善機權,多憑藉,自念親老,旦暮不測,二三年間謀起復,任四維,其身後托乎?瀚生平無善狀。巡撫陜西,贓穢狼籍。及驟躇銓衡,唯諾若簿吏,官缺必請命居正。所指授者,非楚人親戚知識,則親戚所援引也;非宦楚受恩私故,則恩故之黨助也。瀚惟日取四方小吏,權其賄賂,而其他則徒擁虛名。聞居正貽南京都御史趙錦書,臺諫毋議及塚宰,則居正之脅制在朝言官,又可知矣。祖宗之法如是乎?

祖宗朝,詔令不便,部臣猶訾閣擬之不審。今得一嚴旨,居正輒曰「我力調劑故止是」;得一溫旨,居正又曰「我力請而後得之」。由是畏居正者甚於畏陛下,感居正者甚於感陛下。威福自己,目無朝廷。祖宗之法若是乎?

祖宗朝,一切政事,臺省奏陳,部院題覆,撫按奉行,未聞閣臣有舉劾也。居正定令,撫按考成章奏,每具二冊,一送內閣,一送六科。撫按延遲,則部臣糾之。六部隱蔽,則科臣糾之。六科隱蔽,則內閣糾之。夫部院分理國事,科臣封駁奏章,舉劾,其職也。閣臣銜列翰林,止備顧問,從容論思而已。居正創為是說,欲脅制科臣,拱手聽令。祖宗之法若是乎?

至於按臣回道考察,茍非有大敗類者,常不舉行,蓋不欲重挫抑之。近日御史俞一貫以不聽指授,調之南京。由是巡方短氣,莫敢展布,所憚獨科臣耳。居正於科臣既啖之以遷轉之速,又恐之以考成之遲,誰肯舍其便利,甘彼齮齕,而盡死言事哉?往年趙參魯以諫遷,猶曰外任也;餘懋學以諫罷,猶曰禁錮也;今傅應禎則謫戍矣,又以應禎故,而及徐貞明、喬巖、李禎矣。摧折言官,仇視正士。祖宗之法如是乎?

至若為固寵計,則獻白蓮白燕,致詔旨責讓,傳笑四方矣。規利田宅,則誣遼王以重罪,而奪其府地,今武岡王又得罪矣。為子弟謀舉鄉試,則許御史舒鰲以京堂,布政施堯臣以巡撫矣。起大第於江陵,費至十萬,制擬宮禁,遣錦衣官校監治,鄉郡之脂膏盡矣。惡黃州生儒議其子弟倖售,則假縣令他事窮治無遺矣。編修李維楨偶談及其豪富,不旋踵即外斥矣。蓋居正之貪,不在文吏而在武臣,不在內地而在邊鄙。不然,輔政未幾,即富甲全楚,何由致之?宮室輿馬姬妾,奉御同於王者,又何由致之?

在朝臣工,莫不憤歎,而無敢為陛下明言者,積威之劫也。臣舉進士,居正為總裁。臣任部曹,居正薦改御史。臣受居正恩亦厚矣,而今敢訟言攻之者,君臣誼重,則私恩有不得而顧也。願陛下察臣愚悃,抑損相權,毋俾僨事誤國,臣死且不朽。

疏上,居正怒甚,廷辯之,曰:「在令,巡按不得報軍功。去年遼東大捷,臺違制妄奏,法應降謫。臣第請旨戒諭,而臺已不勝憤。後傅應禎下獄,究詰黨與。初不知臺與應禎同邑厚善,實有所主。乃妄自驚疑,遂不復顧藉,發憤於臣。且臺為臣所取士,二百年來無門生劾師長者,計惟一去謝之。」因辭政,伏地泣不肯起。帝為降御座手掖之,慰留再三。居正強諾,猶不出視事,帝遣司禮太監孫隆齎手敕宣諭,乃起。遂捕臺至京師,下詔獄,命廷杖百,遠戍。居正陽具疏救,乃除名為民,而居正恨不已。臺按遼東時,與巡撫張學顏不相得。至是學顏為戶部,誣臺私贖鍰,居正屬御史于應昌巡按遼東覆之,而令王宗載巡撫江西,廉臺里中事。應昌、宗載等希居正意,實其事以聞,遂戍臺廣西。臺父震龍、弟國,俱坐罪。臺至潯州未幾,飲於戍主所,歸而暴卒。是日居正亦卒。

明年,御史江東之訟臺冤,劾宗載、應昌。詔復臺官,罷宗載、應昌,下所司廉問。南京給事中馮景隆因言遼東巡撫周詠與應昌共陷臺,應昌已罷,詠尚為薊遼總督,亦宜罷。南京御史孫繼先亦發學顏陷臺罪。帝方嚮學顏。以景隆疏中並劾李成梁,學顏為成梁訟。繼先又並劾學顏、成梁。乃謫景隆薊州判官,繼先臨清州判官,置學顏不問。已而江西巡撫曹大埜、遼東巡撫李松,勘報宗載、應昌等朋比傾陷皆有狀。刑部以故入論,奏宗載等遣戍、除名、降黜有差。贈臺光祿少卿,廕一子。天啟初,追謚毅思。

馮景隆,浙江山陰人。萬歷五年進士。嘗訟趙世卿冤,且請召張位、習孔教,申救御史魏允貞,至是謫官。後量移南陽推官。

孫繼先,字胤甫,盂人。隆慶五年進士。居正既敗,繼先請召吳中行、趙用賢、艾穆、沈思孝、鄒元標並及餘懋學、趙應元、傅應禎、朱鴻謨、孟一脈、王用汲。又薦魏學曾、宋糸熏、張岳、毛綱、胡執禮、王錫爵、賈三近、溫純、曹科、陳有年、朱光宇、趙參魯等諸人。既坐謫,終南京吏部主事。

傅應禎,字公善,安福人。隆慶五年進士。除零陵知縣。殲洞庭劇寇,論殺祁陽巨猾,民賴以安。調知溧水。萬曆三年,徵授御史。張居正當國,應禎其門生也,有所感憤,疏陳重君德、蘇民困、開言路三事,言:

邇者雷震端門獸吻,京師及四方地震疊告,曾未聞發詔修省,豈真以天變不足畏耶?真定抽分中使,本非舊典,正統間嘗暫行之,先帝納李芳言,已詔罷遣,而陛下顧欲踵行失德之事,豈真以祖宗不足法耶?給事中硃東光奏陳保治,初非折檻解衣者比,乃竟留中不報,豈真以人言不足恤耶?此三不足者,王安石以之誤宋,不可不深戒也。

陛下登極初,自隆慶改元以前逋租,悉賜蠲除,四年以前免三徵七,恩至渥也。乃上軫恤已至,而下延玩自如,曾未有擔負相屬者,何哉?小民一歲之入,僅足給一歲,無遺力以償負也。近乃定輸不及額者,按撫聽糾,郡縣聽調。諸臣畏譴,督趣倍嚴。致流離接踵,怨咨愁歎,上徹於天。是豈太平之象,陛下所樂聞者哉?請下明詔,自非官吏乾沒,並曠然除之。民困既蘇,則災沴自弭。

陛下登極初,召用直臣石星、李已,臣工無不慶幸。近則趙參魯糾中涓而謫為典史,餘懋學陳時政而錮之終身,他如胡執禮、裴應章、侯於趙、趙煥等封事累上,一切置之,如初政何?臣請擢參魯京職,還懋學故官,為人臣進言者勸。

疏奏,居正以疏中王安石語侵己,大怒,調旨切責;以其詞及懋學,執下詔獄,窮治黨與。應禎瀕死無所承,乃謫戍定海。給事中嚴用和、御史劉天衢等疏救,不聽。方應禎下獄,給事中徐貞明偕御史李禎、喬巖入視之。錦衣帥餘廕以聞,三人亦坐謫。

十一年,用御史孫繼先言,召復官。帝將幸昌平閱壽宮,而薊鎮告警,應禎止帝勿行,且陳邊備甚悉。優詔答之。俄擢南京大理寺丞。將行,奏薦海內知名士三十七人。尋移疾歸,三年而卒。贈本寺右少卿。應禎與同邑劉臺同舉進士,為御史,同忤居正得禍,鄉人並祠祀之。

王用汲,字明受,晉江人。為諸生時,郡被倭,客兵橫市中。會御史按部至,用汲言狀。知府曰:「此何與諸生事?」用汲曰:「範希文秀才時,以天下為己任,矧鄉井之禍乃不關諸生耶?」舉隆慶二年進士,授淮安推官。稍遷常德同知,入為戶部員外郎。

萬歷六年,首輔張居正歸葬其親,湖廣諸司畢會。巡按御史趙應元獨不往,居正嗛之。及應元事竣得代,即以病請。僉都御史王篆者,居正客也,素憾應元,且迎合居正意,屬都御史陳炌劾應元規避,遂除名。用汲不勝憤,乃上言:

御史應元以不會葬得罪輔臣,遂為都御史炌所論,坐託疾欺罔削籍,臣竊恨之。夫疾病人所時有,今在廷大小諸臣,曾以病請者何限。御史陸萬鐘、劉光國、陳用賓皆以巡方事訖引疾,與應元不異也,炌何不並劾之?即炌當世宗朝,亦養病十餘年。後夤緣攀附,驟列要津。以退為進,宜莫如炌。己則行之,而反以責人,何以服天下?陛下但見炌論劾應元,以為恣情趨避,罪當罷斥。至其意所從來,陛下何由知之。如昨歲星變考察,將以弭災也,而所挫抑者,半不附宰臣之人。如翰林習孔教,則以鄒元標之故;禮部張程,則以劉臺之故;刑部浮躁獨多於他部,則以艾穆、沈思孝而推戈;考後劣轉趙志皋,又以吳中行、趙用賢而遷怒。蓋能得輔臣之心,則雖屢經論列之潘晟,且得以不次蒙恩;茍失輔臣之心,則雖素負才名之張岳,難免以不及論調。臣不意陛下省災塞咎之舉,僅為宰臣酬恩報怨之私。且凡附宰臣者,亦各藉以酬其私,可不為太息矣哉!

孟子曰:「逢君之惡其罪大。」臣則謂逢相之惡其罪更大也。陛下天縱聖明,從諫勿咈。諸臣熟知其然,爭欲碎首批鱗以自見。陛下欲織錦綺,則撫臣、按臣言之;欲採珍異,則部臣、科臣言之;欲取太倉光祿,則臺臣、科臣又言之。陛下悉見嘉納,或遂停止,或不為例。至若輔臣意之所向,不論是否,無敢一言以正其非,且有先意結其歡,望風張其焰者,是臣所謂逢也。今大臣未有不逢相之惡者,炌特其較著者爾。

以臣觀之,天下無事不私,無人不私,獨陛下一人公耳。陛下又不躬自聽斷,而委政於眾所阿奉之大臣。大臣益得成其私而無所顧忌,小臣益苦行私而無所愬告,是驅天下而使之奔走乎私門矣。陛下何不日取庶政而勤習之,內外章奏躬自省覽,先以意可否焉,然後宣付輔臣,俾之商榷。閱習既久,智慮益弘,幾微隱伏之間,自無逃於天鑒。夫威福者,陛下所當自出;乾綱者,陛下所當獨攬。寄之於人,不謂之旁落,則謂之倒持。政柄一移,積重難返,此又臣所日夜深慮,不獨為應元一事已也。

疏入,居正大怒,欲下獄廷杖。會次輔呂調陽在告,張四維擬削用汲籍,帝從之。居正以罪輕,移怒四維,厲色待之者累日。用汲歸,屏居郭外,布衣講授,足不賤城市。居正死,起補刑部。未上,擢廣東僉事。尋召為尚寶卿,進大理少卿。會法司議胡檟、龍宗武殺吳仕期獄,傅以謫戍。用汲駁奏曰:「按律,刑部及大小官吏,不依法律、聽從上司主使、出入人罪者,罪如之。蓋謂如上文,罪斬、妻子為奴、財產入官之律也。仕期之死,檟非主使者乎?宗武非聽上司主使者乎?今僅謫戍,不知所遵何律也。」上欲用用汲言,閣臣申時行等謂仕期自斃,宜減等,獄遂定。尋遷順天府尹。歷南京刑部尚書,致仕。

用汲為人剛正,遇事敢為。自尹京後,累遷皆在南,以彊直故也。卒,贈太子太保,謚恭質。

吳中行,字子道,武進人。父性,兄可行,皆進士。性,尚寶丞。可行,檢討。中行甫冠,舉鄉試,性誡無躁進,遂不赴會試。隆慶五年成進士,選庶吉士,授編修。大學士張居正,中行座主也。萬曆五年,居正遭父喪,奪情視事。御史曾士楚、吏科都給事中陳三謨倡疏奏留,舉朝和之,中行獨憤。適彗出西南,長竟天,詔百官修省,中行乃首上疏曰:「居正父子異地分暌,音容不接者十有九年。一旦長棄數千里外,陛下不使匍匐星奔,憑棺一慟,必欲其違心抑情,銜哀茹痛於廟堂之上,而責以訏謨遠猷,調元熙載,豈情也哉!居正每自言謹守聖賢義理,祖宗法度。宰我欲短喪,子曰:『予有三年之愛於其父母乎?』王子請數月之喪,孟子曰:『雖加一日愈於已。』聖賢之訓何如也?在律,雖編氓小吏,匿喪有禁;惟武人得墨衰從事,非所以處輔弼也。即雲起復有故事,亦未有一日不出國門,而遽起視事者。祖宗之制何如也?事系萬古綱常,四方視聽,惟今日無過舉,然後後世無遺議。銷變之道,無踰此者。」

疏既上,以副封白居正。居正愕然曰:「疏進耶?」中行曰:「未進不敢白也。」明日,趙用賢疏入。又明日,艾穆、沈思孝疏入。居正怒,謀於馮保,欲廷杖之。翰林院侍講趙志皋、張位、於慎行、張一桂、田一俊、李長春,修撰習孔教、沈懋學俱具疏救,格不入。學士王錫爵乃會詞臣數十人,求解於居正,弗納。遂杖中行等四人。明日,進士鄒元標疏爭,亦廷杖,五人者,直聲震天下。中行、用賢並稱吳、趙。南京御史朱鴻謨疏救五人,亦被斥。中行等受杖畢,校尉以布曳出長安門,舁以板扉,即日驅出都城。中行氣息已絕,中書舍人秦柱挾醫至,投藥一匕,乃蘇。輿疾南歸,刲去腐肉數十臠,大者盈掌,深至寸,一肢遂空。

九年,大計京官,列五人察籍,錮不復敘。居正死,士楚當按蘇、松,憮然曰:「吾何面目見吳、趙二公!」遂引疾去。三謨已擢太常少卿,尋與士楚俱被劾削籍。廷臣交薦中行,召復故官,進右中允,直經筵。大學士許國攻李植、江東之,詆中行、用賢為其黨。中行奏辨,因乞罷,不許。再遷右諭德。御史蔡系周劾植,復侵中行,中行求去,章四上。詔賜白金、文綺,馳傳歸。言者屢薦,執政抑不召。久之,起侍講學士,掌南京翰林院。同里僉事徐常吉嘗訟中行,事已解,給事中王嘉謨復摭舊事劾之,命家居俟召。尋卒。後贈禮部右侍郎。

子亮、元,從子宗達。亮官御史,坐累貶官,終大理少御。元,江西布政使。宗達,少傅、建極殿大學士。亮尚志節,與顧憲成諸人善。而元深疾東林,所輯《吾徵錄》,詆毀不遺力。兄弟異趣如此。

趙用賢,字汝師,常熟人。父承謙,廣東參議。用賢舉隆慶五年進士,選庶吉士。萬曆初,授檢討。張居正父喪奪情,用賢抗疏曰:「臣竊怪居正能以君臣之義效忠於數年,不能以父子之情少盡於一日。臣又竊怪居正之勛望積以數年,而陛下忽敗之一旦。莫若如先朝楊溥、李賢故事,聽其暫還守制,刻期赴闕,庶父子音容乖暌阻絕於十有九年者,得區區稍伸其痛於臨穴憑棺之一慟也。國家設臺諫以司法紀、任糾繩,乃今嘵嘵為輔臣請留,背公議而徇私情,蔑至性而創異論。臣愚竊懼士氣之日靡,國是之日淆也。」疏入,與中行同杖除名。用賢體素肥,肉潰落如掌,其妻臘而藏之。用賢有女許御史吳之彥子鎮。之彥懼及,深結居正,得巡撫福建。過里門,不為用賢禮,且坐鎮於其弟下,曰:「婢子也」,以激用賢。用賢怒,已察知其受居正黨王篆指,遂反幣告絕。之彥大喜。

居正死之明年,用賢復故官,進右贊善。江東之、李植輩爭向之,物望皆屬焉。而用賢性剛,負氣傲物,數訾議大臣得失,申時行、許國等忌之。會植、東之攻時行,國遂力詆植、東之,而陰斥用賢、中行,謂:「昔之專恣在權貴,今乃在下僚;昔顛倒是非在小人,今乃在君子。意氣感激,偶成一二事,遂自負不世之節,號召浮薄喜事之人,黨同伐異,罔上行私,其風不可長。」於是用賢抗辨求去,極言朋黨之說,小人以之去君子、空人國,詞甚激憤。帝不聽其去。黨論之興,遂自此始。

尋充經筵講官。再遷右庶子,改南京祭酒。薦舉人王之士、鄧元錫、劉元卿,清修積學。又請建儲,宥言官李沂罪。居三年,擢南京禮部右侍郎。以吏部郎中趙南星薦,改北部。尋以本官兼教習庶吉士。

二十一年,王錫爵復入內閣。初,用賢徙南,中行、思孝、植、東之已前貶,或罷去,故執政安之。及是,用賢復以爭三王並封語侵錫爵,為所銜。會改吏部左侍郎,與文選郎顧憲成辨論人才,群情益附,錫爵不便也。用賢故所絕婚吳之彥者,錫爵里人,時以僉事論罷,使其子鎮訐用賢論財逐婿,蔑法棄倫。用賢疏辨,乞休。詔禮官平議。尚書羅萬化以之彥其門生,引嫌力辭。錫爵乃上議曰:「用賢輕絕,之彥緩發,均失也。今趙女已嫁,難問初盟;吳男未婚,無容反坐。欲折其衷,宜聽用賢引疾,而曲貸之彥。」詔從之。用賢遂免歸。戶部郎中楊應宿、鄭材復力詆用賢,請據律行法。都御史李世達、侍郎李禎疏直用賢,斥兩人讒諂,遂為所攻。高攀龍、吳弘濟、譚一召、孫繼有、安希範輩皆坐論救褫職。自是朋黨論益熾。中行、用賢、植、東之創於前,元標、南星、憲成、攀龍繼之。言事者益裁量執政,執政日與枝拄,水火薄射,訖於明亡雲。

用賢長身聳肩,議論風發,有經濟大略。蘇、松、嘉、湖諸府,財賦敵天下半,民生坐困。用賢官庶子時,與進士袁黃商榷數十晝夜,條十四事上之。時行、錫爵以為吳人不當言吳事,調旨切責,寢不行。家居四年卒。天啟初,贈太子少保、禮部尚書,謚文毅。

孫士春、士錦,崇禎十年同舉進士。士春,字景之。第三人及第,授編修。明年,兵部尚書楊嗣昌奪情視事,未幾入閣。少詹事黃道周劾之,下獄。士春上疏曰:「嗣昌墨衰視事,既已罔效,陛下簡入綸扉,自應力辭新命。乃閱其奏牘,徒計歲月久近間,絕無哀痛惻怛之念,何奸悖一至此也!陛下破格奪情,曰人才不足故耳。不知人才所以不振,正由愛功名、薄忠孝致之。且無事不講儲材,有事輕言破格,非用人無弊之道也。臣祖用賢,首論故相奪情,幾斃杖下,臘敗肉示子孫。臣敢背家學,負明主,坐視綱常掃地哉?」帝怒,謫廣東布政司照磨。祖孫並以攻執政奪情斥,士論重之。後復故官,終左中允。

艾穆,字和父,平江人。以鄉舉署阜城教諭,鄰郡諸生趙南星、喬璧星皆就學焉。入為國子助教。張居正知穆名,欲用為誥敕房中書舍人,不應。萬曆初,擢刑部主事。進員外郎,錄囚陜西。時居正法嚴,決囚不如額者罪。穆與御史議,止決二人。御史懼不稱,穆曰:「我終不以人命博官也。」還朝,居正盛氣譙讓。穆曰:「主上沖年,小臣體好生德,佐公平允之治,有罪甘之。」揖而退。

及居正遭喪奪情,穆私居歎息,遂與主事沈思孝抗疏諫曰:「自居正奪情,妖星突見,光逼中天。言官曾士楚、陳三謨甘犯清議,率先請留,人心頓死,舉國如狂。今星變未銷,火災繼起。臣敢自愛其死,不灑血一為陛下言之!陛下之留居正也,動曰為社稷故。夫社稷所重,莫如綱常。而元輔大臣者,綱常之表也。綱常不顧,何社稷之能安?且事偶一為之者,例也;而萬世不易者,先王之制也。今棄先王之制,而從近代之例,如之何其可也。居正今以例留,腆顏就列矣。異時國家有大慶賀、大祭祀,為元輔者,欲避則害君臣之義,欲出則傷父子之情。臣不知陛下何以處居正,居正又何以自處也!徐庶以母故辭於昭烈曰:『臣方寸亂矣。』居正獨非人子而方寸不亂耶?位極人臣,反不修匹夫常節,何以對天下後世!臣聞古聖帝明王勸人以孝矣,未聞從而奪之也。為人臣者,移孝以事君矣,未聞為所奪也。以禮義廉恥風天下猶恐不足,顧乃奪之,使天下為人子者,皆忘三年之愛於其父,常紀墜矣。異時即欲以法度整齊之,何可得耶!陛下誠眷居正,當愛之以德,使奔喪終制,以全大節;則綱常植而朝廷正,朝廷正而百官萬民莫不一於正,災變無不可弭矣。」

時吳中行、趙用賢請令居正奔喪,葬畢還朝,而穆、思孝直請令終制,故居正尤怒。中行、用賢杖六十,穆、思孝皆八十加梏堣,置之詔獄。越三日,以門扉舁出城,穆遣戍涼州。創重不省人事,既而復蘇,遂詣戍所。穆,居正鄉人也。居正語人曰:「昔嚴分宜時未有同鄉攻擊者,我不得比分宜矣。」九年,大計,復置穆、思孝察籍。

及居正死,言官交薦,起戶部員外郎。遷西川僉事,屢遷太僕少卿。十九年秋,擢右僉都御史,巡撫四川。故崇陽知縣周應中、賓州知州葉春及行義過人,穆舉以自代,不報。既之官,有告播州宣慰使楊應龍叛者,貴州巡撫葉夢熊請徵之。蜀人多言應龍強,未易輕舉,穆亦不欲加兵,與夢熊異。朝命兩撫臣會勘,應龍不願赴貴州,乃逮至重慶,對簿論斬,輸贖,放之還。穆病歸,未幾卒。後應龍復叛,議者追咎穆,奪其職。

喬璧星,臨城人。官右僉都御史,亦巡撫四川。

葉春及,歸善人。由鄉舉授福清教諭。上書陳時政,纚纚三萬言。終戶部郎中。

沈思孝,字純父,嘉興人。舉隆慶二年進士。又三年,謁選。高拱署吏部,欲留為屬曹,思孝辭焉,乃授番禺知縣。殷正茂總制兩廣,欲聽民與番人互市,且開海口諸山征其稅,思孝持不可。

萬曆初,舉卓異,又為刑部主事。張居正父喪奪情,與艾穆合疏諫。廷杖,戍神電衛。居正死,召復官,進光祿少卿。政府惡李植、江東之及思孝輩。思孝遷太常少卿,御史龔仲慶希指詆之,思孝遂求去,不許。尋遷順天府尹,坐寬縱冒籍舉人,貶三秩視事。思孝御三品服自若,被劾,調南京太僕卿,仍貶三秩。未幾,謝病歸。

吏部尚書陸光祖起為南京光祿卿。尋進右僉都御史,巡撫挾西。寧夏哱拜叛,詔思孝移駐下馬關,為總督魏學曾聲援。思孝以兵少,請募浙江及宣、大騎卒各五千,發內帑供軍,並乞宥故都御史李材罪,令立功。詔思孝近地召募,而罷材勿遣。思孝與學曾議軍事不合,給事中侯慶遠劾思孝舍門戶而守堂奧,設邏卒以衛妻孥,不任封疆事。改撫河南,辭不赴。

頃之,召為大理卿。中官郝金詐傳懿旨下獄,刑部薄其罪,思孝駁誅之。帝悅,進工部左侍郎。陜西織羊絨為民患,以思孝奏,減十之四。進右都御史,協理戎政。初,廷推李禎為首,思孝次之,帝特用思孝。或疑有奧援,給事中楊東明、鄒廷彥相繼疏劾。帝以廷彥受東明指,謫東明,奪廷彥俸。

二十三年,吏部尚書孫丕揚掌外察,黜參政丁此呂。思孝與東之素善此呂。會御史趙文炳劾文選郎蔣時馨受賄,時馨疑思孝嗾之,遂訐思孝先庇此呂,後求吏部不得,以此二事憾已,遂結江東之、劉應秋等,令李三才屬文炳。帝惡時馨,罷其官。思孝等疏辨,且求去。丕揚言時馨無罪,此呂受贓有狀,思孝不當庇。因上此呂訪單,乞歸。訪單者,吏部當察時,咨公論以定賢否,廷臣因得書所聞以投掌察者。事率核實,然間有因以中所惡者。帝降詔慰留丕揚,逮此呂,詰讓思孝。御史俞價、強思、馮從吾,給事中黃運泰、祝世祿,皆為時馨訟冤,語侵思孝、東之。給事中楊天民、馬經綸、馬文卿又各疏劾思孝,大抵言文炳之疏由思孝,藉以搖丕揚也。思孝屢乞罷,因詆丕揚負國。員外郎岳元聲言大臣相攻,宜兩罷,似並論丕揚、思孝,而其指特攻時馨以及丕揚。疏方上,文炳忽變其說,謂:「元聲、東之述思孝意,迫之救此呂、劾時馨,非己意也。帝皆置不問。

思孝素以直節高天下,然尚氣好勝,動輒多忤,以此呂故,頗被物議。然時馨、此呂皆非端人,丕揚、思孝亦各有所左右。其明年,御史林培請辨忠邪,又力詆思孝、東之;且言:「丕揚杜門半載,辭疏十上,意必得請而後已。思孝則杜門未幾,近見從吾、運泰等罷,謂朝廷不難去言官五六人以安我。此人不去,為朝端害。」帝顧思孝厚,謫培官。乾清宮災,思孝請行皇長子冠禮以回天心。又以日本封事大壞,請亟修戰守備,並論趙志皋、石星誤國。其秋,丕揚去位,思孝亦引疾,詔馳傳歸,朝端議論始息。久之,丕揚復起為吏部,御史史記事復詆思孝與顧天颭合謀欲構陷丕揚。顧憲成、高攀龍力辨其誣,而思孝卒矣。天啟中,贈太子少保。

丁此呂,字右武,新建人。萬歷五年進士。由漳州推官征授御史。慈寧宮災,請撤鰲山,停織造、燒造,還建言譴謫諸臣,去張居正餘黨,速誅徐爵、游七。報聞。尋劾禮部侍郎高啟愚命題示禪授意,謫潞安推官。語詳《李植傳》。尋遷太僕丞,歷浙江右參政。考察論黜,復遣官逮之。大學士趙志皋等再疏乞宥,且言此呂有氣節,未必果貪污。丕揚亦言此呂無逮問條,乞免送詔獄。帝皆不從,逮下鎮撫,謫戍邊。

贊曰:劉臺諸人,皆以論張居正得罪。罰最重者,名亦最高。用汲之免也,幸耳。平心論之,居正為相,於國事不為無功;諸人論之,不無過當。然聞謗而不知懼,忿戾怨毒,務快己意。虧盈好還,禍釀身後。傳曰:「惟善人能受盡言。」於戲難哉!


\end{pinyinscope}