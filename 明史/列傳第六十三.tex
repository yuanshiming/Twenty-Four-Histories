\article{列傳第六十三}

\begin{pinyinscope}
○衛青子穎董興何洪劉雄劉玉仇鉞神英子周曹雄子謙馮禎張俊李鋐楊銳崔文

衛青,字明德,松江華亭人。以薊州百戶降成祖,積功至都指揮僉事,蒞中都留守司事,改山東備倭。

永樂十八年二月,浦臺妖婦林三妻唐賽兒作亂。自言得石函中寶書神劍,役鬼神,剪紙作人馬相戰鬥。徒眾數千,據益都卸石柵寨。指揮高鳳敗歿,勢遂熾。其黨董彥昇等攻下莒、即墨,圍安丘。總兵官安遠侯柳升帥都指揮劉忠圍賽兒寨。賽兒夜劫官軍。軍亂,忠戰死,賽兒遁去。比明,升始覺,追不及,獲賊黨劉俊等及男女百餘人。而賊攻安丘益急,知縣張CM、丞馬捴死戰,賊不能下,合莒、即墨眾萬餘人以攻。青方屯海上,聞之,帥千騎晝夜馳至城下。再戰,大敗之,城中亦鼓噪出,殺賊二千,生擒四千餘,悉斬之。時城中旦夕不能支,青救稍遲,城必陷。比賊敗,升始至,青迎謁。升怒其不待己,捽之出。是日,鰲山衛指揮王真亦以兵百五十人殲賊諸城,賊遂平。而賽兒卒不獲。帝賜書勞青,切責升。尚書吳中等劾升,且言升媢青功。於是下升獄,而擢青山東都指揮使,真都指揮同知,CM、捴左右參議,賞賚有差。青還備倭海上。尋坐事繫獄。宣德元年,帝念其功,釋之,俾復職。時京師營繕役繁,調及防海士卒。青以為言,得番代。英宗立,進都督僉事,尋卒。

青有孝行,善撫士卒,居海上十餘年,海濱人思之,請於朝,立祠以祀。

次子穎,正統初,襲濟南衛指揮使。景帝立,奉詔入衛,再遷至都指揮同知。以石亨薦,擢署都督僉事,管五軍營右哨。論黃花鎮、白羊口及西直門御寇功,累進都督同知。景泰三年協鎮宣府。逾年,召還。天順元年,以「奪門」功封宣城伯,予世券,出鎮甘肅。孛來入犯,不能禦,為有司所劾,詔不問。亨敗,穎以守邊故得無奪。憲宗即位,廷議以穎不勝任,乃召還。會盡革「奪門」世爵,穎以天順間征西番馬吉思、冬沙諸族功自醖,詔如故。成化二年為遼東總兵官,尋引疾罷。給事中陳鉞等劾之,下獄,尋宥之。弘治中卒。贈侯,謚壯勇。

傳子至孫錞。嘉靖時,督神機營,屢加太保兼太子太師。四傳至時泰。崇禎時,掌後府。京師陷,懷鐵券,闔門十七人皆赴井死。

董興,長垣人。初為燕山右衛指揮使,累遷署都指揮同知。正統中,新建伯李玉等舉興將才,進署都指揮使,京營管操。復用薦,擢署都督僉事,充右參將,從寧陽侯陳懋討鄧茂七,破餘黨於建寧,進都督同知。

南海賊黃蕭養圍廣州,安鄉伯張安、都指揮王清戰死,賊眾攻城益急。詔拜興左副總兵,調江西、兩廣軍往討,而以侍郎孟鑒贊理軍務。興用天文生馬軾自隨。興果銳,不能戢下,軾戒之。景泰元年二月,師至廣州。賊舟千餘艘,勢甚熾,而徵兵未至,諸將請濟師。軾曰:「廣民延頸久矣,即以狼兵往擊,猶拉朽耳。」興從之。既而兵大集,進至大洲擊賊,殺溺死者萬餘人,餘多就撫。蕭養中流矢死,函首以獻,俘其父及子等,餘黨皆伏誅。論功,進右都督,留鎮廣東。給事中黃士俊劾興寬縱,降其官。明年復職。

久之,召還,分督京營。與曹吉祥結姻,冒「奪門」功,封海寧伯。未幾,充總兵官,鎮遼東,予世券。議革「奪門」者爵,興以守邊得免。吉祥誅,乃奪興爵,仍右都督,發廣西立功。以錦衣李貴薦,復爵,總兵宣府,再予世券。憲宗嗣位,罷還。已,停世襲。家居十餘年卒。

何洪,全椒人。嗣世職,為成都前衛指揮使。正統中,從征麓川。景泰末,從征天柱、銅鼓。皆有功。屢遷都指揮使,掌四川都司事,與平東苗。憲宗即位,論功,擢都督僉事。巡撫汪浩乞留洪四川,許之。

德陽人趙鐸反,自稱趙王,漢州諸賊皆歸之。連番眾,數陷城,殺將吏。遣其黨何文讓及僧悟昇掠安岳諸縣。洪斬悟昇,生擒文讓。鐸將逼成都,官軍分三路討。洪偕都指揮寧用趨彰明,賊引去。追至梓潼朱家河,力戰,賊少卻。洪乘勝陷陣,後軍不繼,為賊所圍,左右跳蕩,殺賊甚眾,力竭而死。

洪勇敢,善撫士,號令嚴,蜀將無及之者。既死,官軍奪氣。而四川都指揮僉事臨淮劉雄亦戰死。雄剛勁,遇敵輒前,嘗捕賊漢州,生擒七十餘人。及鐸亂,追之羅江大水河,手馘數人,賊連敗。千戶周鼎傷,雄前救之,徑奔賊陣,叢刺死。詔贈洪都督同知,予祭葬,子節襲都指揮僉事。雄贈都指揮同知,賜祭,命子襲職,超二官。

洪雖死,綿竹典史蕭讓帥鄉兵擊鐸,破之。官兵頻進擊,其黨稍散去。鐸勢孤,帥餘賊趨彰明。千戶田儀等設伏梓潼,而參將周貴直搗其巢。賊大敗,夜奔石子嶺。儀亟進,斬鐸,賊盡平。成化元年五月也。

劉玉,字仲璽,磁州人。生有膂力,給侍曹吉祥家。從征麓川,授副千戶,積功至都指揮僉事,天順元年以「奪門」功進都督僉事,尋充右參將,守備潯州。慶遠蠻剽博白及廣東之寧川,玉偕左參將範信邀擊,敗之。俄命分守貴州。從方瑛討東苗,殲乾把豬。討西堡苗,縶其魁楚得。先後斬首千級,毀其巢而還。旋改右副總兵,鎮守貴州。吉祥誅,玉下吏當斬。以道遠不與謀,免死,謫海南副千戶。

六年,帝將以穀登為甘肅副總兵。李賢言登不任,玉老成。乃復以為都督僉事、右副總兵,鎮守涼州。咎咂族叛,會兵平之,進都督同知。

成化四年,滿俊亂固原,白圭舉玉為總兵官,統左右參將夏正、劉清討之,兵分為七。玉與總督項忠抵石城,賊已數敗。會毛忠死,玉亦被圍,中流矢,力戰得出。相持兩月,大小百十戰,竟平之。進左都督,掌右府事。自醖前西堡功,命增俸百石,掌耀武營。六年充左副總兵,從朱永出延綏。五月,河套部入犯,玉帥眾御卻之。逾年卒。贈固原伯,謚毅敏。

玉雖起僕隸,勇決過人,善撫士,所至未嘗衄。滿俊之叛,據石城險,屢敗官軍,玉戰最力。及論功,只賜秩一級,時惜其薄。子文,襲指揮使。

仇鉞,字廷威,鎮原人。初以人庸卒給事寧夏總兵府,大見信愛。會都指揮僉事仇理卒,無嗣,遂令鉞襲其世職,為寧夏前衛指揮同知。理,江都人,故鉞自稱江都仇氏。再以破賊功,進都指揮僉事。

正德二年用總制楊一清薦,擢寧夏遊擊將軍。五年,安化王寘鐇及都指揮何錦、周昂,指揮丁廣反。鉞時駐城外玉泉營,聞變欲遁去。顧念妻子在城中,恐為所屠滅,遂引兵入城。解甲覲寘鐇,歸臥家稱病,以所將兵分隸賊營。錦等信之,時時就問計。鉞亦謬輸心腹。而陰結壯士,遣人潛出城,令還報官軍旦夕至。鉞因紿錦、廣,宜急出兵守渡口,遏東岸兵,勿使渡河。錦、廣果傾營出,而昂獨守城。寘鐇以祃牙召鉞,鉞稱病亟。昂來視,鉞方堅臥呻吟。伏卒猝起,捶殺昂。鉞乃被甲橫刀,提其首,躍馬大呼,壯士皆集,徑馳詣寘鐇第,縛之。傳寘鐇令,召錦等還,而密諭其部曲以擒寘鐇狀。眾遂大潰。錦、廣單騎走賀蘭山,為邏卒所獲,舉事凡十八日而敗。

先是,中朝聞變,議以神英為總兵官,而命鉞為副。俄傳鉞降賊,欲追敕還。大學士楊廷和曰:「鉞必不從賊,令知朝廷擢用,志當益堅。不然,棄良將資敵人耳。」乃不追。事果定。而劉瑾暱陜西總兵官曹雄,盡以鉞功歸之,鉞竟無殊擢。巡按御史閻睿訟其功,詔奪俸三月。瑾誅,始進署都督僉事,充寧夏總兵官。尋論功,封咸寧伯,歲祿千石,予世券。明年冬,召掌三千營。

七年二月拜平賊將軍,偕都御史彭澤討河南盜劉惠、趙鐩,以中官陸訚監其軍。未至,而參將馮禎戰死洛南,賊勢益熾。已,聞官軍將至,遂奔汝州。又聞官軍扼要害,乃走寶豐,復由舞陽、遂平轉掠汝州東南,敗奔固始,抵潁州,屯朱皋鎮。永順宣慰彭明輔等擊之,賊倉猝渡河,溺死者二千人。餘眾走光山,鉞追及之。命諸將神周、姚信、時源、金輔左右夾擊,賊大敗,斬首千四百有奇。湖廣軍亦破其別部賈勉兒於羅田。賊沿途潰散。自六安陷舒城,復還光山,至商城。官軍追之急,賊復南攻六安。將陷,時源等涉河進,敗之七里岡。賊趨廬州,至定遠西又敗。還至六安,分其眾為二。劉惠與趙鐩二弟鐇、鎬帥萬餘人,北走商城。而鐩道遇其徒張通及楊虎遺黨數千人,勢復振,掠鳳陽,陷泗、宿、睢寧、定遠。於是澤與鉞計,使神周追鐩,時源、金輔追惠,姚信追勉兒。勉兒、鐩復合,周信連敗之宿州,追奔至應山,其眾略盡。鐩薙發懷度牒,潛至江夏。飯村店,軍士趙成執送京師,伏誅。源、輔追劉惠,連戰皆捷。惠窘走南召,指揮王謹追及於土地嶺,射中惠左目,自縊死。勉兒數為都指揮朱忠、夏廣所敗,獲之項城丁村。餘黨邢本道、劉資及楊寡婦等先後皆被擒。凡出師四月,而河南賊悉平。

趙鐩,一名風子,文安諸生也。劉七等亂起,鐩挈家匿渚中,賊驅之登陸,將污其妻女。鐩素驍健,有膂力,手格殺二賊。賊聚執之,遂入其黨為之魁。賊專事淫掠,鐩稍有智計,定為部伍,勸其黨無妄殺。移檄府縣,約官吏師儒毋走避,迎者安堵。由是橫行中原,勢出劉六等上。嘗攻鈞州五日,以馬文升方家居,舍之去。有司遣人齎招撫榜至,鐩具疏附奏言:「今群奸在朝,舞弄神器,濁亂海內,誅戮諫臣,屏棄元老,舉動若此,未有不亡國者。乞陛下睿謀獨斷,梟群奸之首以謝天下,即梟臣之首以謝群奸。」其桀黠如此。

鉞既平河南賊,移師會陸完,共滅劉七等於江北。論功,進世侯,增祿百石,仍督三千營。

八年,大同有警,命充總兵官,統京軍御之。鉞上五事,中請遣還京操邊軍,停京軍出征,以省公私之擾,尤切時弊。時不能用。鉞既至,值寇犯萬全沙河。擊之,斬首三級,而軍士亡者二十餘人,寇亦引去。奏捷蒙賚,朝論恥之。

帝詔諸邊將入侍豹房。鉞嘗一入,後輒力辭。十年冬,稱疾解營務。詔給軍三十人役其家。世宗立,再起督三千營,掌前府事。未上卒,年五十七。謚武襄。

子昌以病廢,孫鸞嗣侯。世宗時,怙寵通邊,磔死,爵除。

神英,字景賢,壽州人。天順初,襲父職,為延安衛指揮使,守備寧塞營,屢將騎兵,從都督張欽等征討有功。

成化元年,尚書王復行邊,薦英有謀勇,進都指揮僉事。以從征滿四功,遷都指揮使,充延綏右參將。屢敗加思蘭兵,進署都督僉事。巡撫餘子俊築邊牆,命英董役,工成受賚。久之,充總兵官,鎮守寧夏,移延綏,復移宣府。弘治改元,移大同。十一年,馬市開,英違禁貿易,寇掠蔚州又不救,言官連劾,召還閒住。尋起督果勇營。嘗充右參將,從硃暉禦寇延綏。武宗立,寇犯宣府,與李俊並充左參將,帥京軍以援。尋以都督同知僉事左府,剿近畿劇賊,進右都督。

正德五年,給事中段豸劾英老,命致仕。當是時,劉瑾竊政。總兵官曹雄等以附瑾得重權。英素習瑾,厚賄之。因自陳邊功,乞敘錄,特詔予伯爵。吏、兵二部持之,下廷議。而廷臣希謹指,無不言當封者,遂封涇陽伯,祿八百石。未幾,寘鐇反,命充總兵官討之。未至,賊已滅。其秋,瑾敗,為言官所劾。詔奪爵,以右都督致仕。越二年卒。

子周,輸粟為指揮僉事。累官都指揮使,充延綏右參將。正德六年命以所部兵討河南流賊,數有功,再進都督同知。賊平,遂以副總兵鎮山西。九年秋,寇大入寧武關,躪忻、定襄、寧化。周擁兵不戰,軍民死者數千。詔巡撫官執歸京師。周潛結貴近,行至易州,偽稱病,自陳戰功。帝乃宥周罪,盡削其秩,為總旗,而輸粟指揮如故。已,夤緣江彬入豹房,驟復都督,賜國姓,典兵禁中。遂與彬相倚為聲勢,納賄不貲。彬敗,周亦下獄,伏誅。

曹雄,西安左衛人。弘治末,歷官都指揮僉事,為延綏副總兵。武宗即位,用總督楊一清薦,擢署都督僉事,充總兵官,鎮固原。以瑾同鄉,自附於瑾。瑾欲廣樹黨,日相親重。

正德四年,雄上言:「故事,布、按二司及兵備道臣文移達總兵官者,率由都司轉達。今邊務亟,徵調不時,都司遠在會城,往返千里,恐誤軍機。乞如巡撫大同例,徑呈總兵官便。」兵部尚書曹元希瑾意,覆如其言。既復受瑾屬,奏雄鎮守未佩印,宜如各邊例,特賜印以重其權。乃進雄署都督同知,以延綏總兵官吳江所佩征西將軍印佩之,而別鑄靖虜將軍印予江。及總督才寬禦寇沙窩為所殺,雄擁兵不救。佯引罪,乞解兵柄,令子謐齎奏詣京師。瑾異謐貌,妻以從女,而優詔褒雄,令居職如故,糾者反被責。

寘鐇反寧夏,雄聞變,即統兵壓境上。令都指揮黃正以兵三千入靈州,固士卒心,約鄰境刻期討。密焚大、小二壩積草,與守備史鏞等奪河西船,盡泊東岸。賊黨何錦懼,急帥兵出守大壩,以防決河。雄乃令鏞潛通書仇鉞,俾從中舉事,賊遂成擒。是役也,功雖成於鉞,而居外布置,賊不內顧,雄有勞焉。捷聞,瑾以平賊功歸之,進左都督。謐亦官千戶。雄不安,引咎自劾,推功諸將,降旨慰勞。未幾,瑾敗,言官交劾。降指揮僉事,尋徵下獄,以黨逆論死,籍其家。詔宥之,與家屬永戍海南,遇赦不原。

雄長子謙,讀書能文,有機略,好施予。故參政李崙、主事孔琦家貧甚,雄請周恤其妻子,以勸廉吏,謙意也。御史高胤先被逮,無行貲。謙為治裝,並恤其家。受業楊一清,聞一清將起用,貽書止之曰:「近日關中人材,連茹而起,實山川不幸。獨不留三五輩為後日地耶?」時陜人率附瑾以進,故謙云然。雄下獄,謙亦被繫,為怨家箠死。

馮禎、綏德衛人。起家卒伍,累功為本衛指揮僉事。弘治末,擢署都指揮僉事,守備偏頭關。尋充參將,分守寧夏西路,以勇敢聞。寘鐇反,馳奏告變。事平,進署都指揮同知。已,擢副總兵,協守延綏。

正德六年七月,盜起中原。詔以所部千五百人入討。至阜城,遇賊。禎令軍中毋顧首級、貪虜獲,遂大敗賊。逐北數十里,俘斬八百六十有奇。進解曹州圍,執其魁朱諒。錄功,進都督僉事。

明年春,劉惠、趙鐩亂河南,連陷鹿邑、上蔡、西平、遂平、舞陽、葉,縱掠南頓、新蔡、商水、襄城,復還,駐西平。禎偕副總兵時源,參將神周、金輔擊敗之。賊奔入城,官軍塞其門。乘夜焚死千餘人,斬首稱是,餘賊潰而西。巡撫鄧璋等朝崇王於汝寧,宴飲連日。賊招散亡,勢復振,陷鄢陵、滎陽、汜水、鞏。圍河南府三日,諸軍始集。賊屯洛南,覘官軍饑疲,迎戰。右哨金輔不敢渡洛,禎及源、周方陣,而後哨參將姚信所部京軍先馳,失利,遽遁。陣亂,賊乘之。禎下馬殊死鬥,援絕死焉。贈洛南伯,賜祭葬,授其子大金都督僉事。後賊平,論功,復廕一子世百戶。明年是日,禎死所風霾大作,又明年,亦如之。伊王奏聞,敕有司建祠,歲以死日致祭。尋用給事中李鐸言,歲給米二石,帛二疋,贍其家。

張俊,宣府前衛人。嗣世職,為本衛指揮使。累擢大同游擊將軍。弘治十二年以功進都指揮同知。火篩入大同左衛,大掠八日。俊遣兵三百邀其前,復分兵三百為策應,而親御之荊東莊。依河結營,擊卻三萬餘騎。帝大喜,立擢都督僉事。未幾,總兵官王璽失事被徵,即命俊代之。其冬,以寇入戴罪,尋移鎮宣府。中官苗逵督師延綏,檄大同、宣府卒為探騎。俊持不遣,逵遂劾俊。帝宥俊,而命發卒如逵言。

武宗初立,寇乘喪大入,連營二十餘里。俊遣諸將李稽、白玉、張雄、王鎮、穆榮各帥三千人,分扼要害。俄,寇由新開口毀垣入,稽遽前迎敵;玉、雄、鎮、榮各帥所部拒於虞臺嶺。俊急帥三千人赴援,道傷足,以兵屬都指揮曹泰。泰至鹿角山,被圍。俊力疾,益調兵五千人,持三日糧,馳解泰圍,復援出鎮。又分兵救稽、玉,稽、玉亦潰圍出。獨雄、榮阻山澗,援絕死。諸軍已大困,收兵還。寇追之,行且戰,僅得入萬全右衛城,士馬死亡無算。俊及中官劉清、巡撫李進皆徵還。御史郭東山言,俊扶病馳援,勸懲不宜偏廢,乃許贖罪。

正德五年,起署都督同知,典神威營操練。明年六月,賊楊虎等自山西十八盤還,破武安,掠威、曲周、武城、清河、故城、景州,轉入文安,與劉六等合。都指揮桑玉屢敗,僉事許承芳請濟師。乃命俊充副總兵,與參將王琮統京軍千人討之。往來近畿數月,不能創賊。已,朝議調邊軍協守,賊遂連敗。明年三月,劉六、劉七、齊彥名、龐文宣等敗奔登、萊海套。陸完檄俊軍萊州,合諸將李金宏等邀之。賊遂北走,轉掠寶坻、香河、玉田,俊急偕許泰、郤永遏之。帝喜,勞以白金。賊由武清西去。未幾,得疾召還。後賊平,實授都督同知。久之,卒。

俊為邊將,持廉,有謀勇。其歿也,家無贏資。

李鋐,大同右衛人。世指揮同知,累功進都指揮僉事,充參將,協守大同。山東盜起,詔改游擊將軍,尋充副總兵,與俊等邀賊,復與劉暉部將傅鎧、張椿等數立功。賊平,進都指揮同知,充總兵官,鎮鳳陽諸府。尋以江西盜猖獗,擢署都督僉事,與都御史俞諫同提督軍務。賊王浩八據裴源山,憑高發矢石,官軍幾不支。鋐下馬持刀,督將士殊死鬥,賊乃走。追數十里,擒之。復以次討平劉昌三、胡浩三等。移駐餘干,將擊遺賊之未下者,疽發背,卒於軍。詔贈右都督,蔭子都指揮僉事。

楊銳,字進之,蕭縣人。嗣世職,為南京羽林前衛指揮使。正德初,以才擢掌龍江右衛事,督造漕舟於淮安。

寧王宸濠有異謀,王瓊以安慶居要害,宜置戍,乃進銳署都指揮僉事,守備其地。銳與知府張文錦治戰艦,日督士肄水戰。十四年六月丙子,宸濠反。東下,焚彭澤、湖口、望江。己丑,奄至安慶城下,舟五十餘艘。銳、文錦與指揮崔文、同知林有祿、通判何景暘、懷寧知縣王誥等御之江滸。已,收兵入城,被圍。銳、文軍城西,文錦、有祿軍城北,景抃、誥軍東南。城西尤要沖,銳晝夜拒戰,殺傷賊二百餘,斬其間諜,乃稍卻。

七月丁酉,賊悉兵至,號十萬,舳艫相銜六十餘里。宸濠乘黃艦,泊黃石磯,身自督戰。江西僉事潘鵬在賊軍,安慶人也,宸濠令諭降。呼銳及文錦語,眾心頗搖。吏黃洲者,以大義責數之,鵬慚而退。既復持偽檄至,其家僮見,遙呼之,銳腰斬以徇。將射鵬,鵬遁去,眾心乃定。賊怒,圍城數周,攻益急。銳等殊死戰。賊雲樓數十瞰城中,城中亦造飛樓射賊,夜縋人焚賊樓。賊置天梯,廣二丈,高於城,版蔽之,前後有門,伏兵其中,輪轉以薄城。城上束葦沃膏,燃其端,梯稍近即投之,須臾盡焚,賊多死。時軍衛卒不滿百,乘城皆民兵。老弱婦女饋餉,人運石一二,數日積如山。賊攻城,城上或投石,或沸湯沃之,賊輒傷。銳等射書賊營,諭令解散,有亡去者。乃募死士夜劫賊營,賊大驚擾,比曉稍定。宸濠慚憤,謂其下曰:「安慶且不克,安望南都。」會聞伍文定等破南昌,遂解圍去。文出城襲擊,又破之,旬有八日而圍解。

事聞,武宗大喜,擢銳參將,分守安徽池、太、寧國及九江、鐃、黃。銳薦鄭岳、胡世寧,帝即召用。世宗立,論功,擢都督僉事,廕子世千戶。再遷僉書左府,改南京右府。充總兵官,鎮遼東。改督漕運,鎮淮安。嘉靖十年為巡按御史李循義劾罷,逾年卒。

崔文,世為安慶衛指揮使,守城勞亞於銳。世宗錄其功,超三階為都指揮使,廕子世百戶。江、淮多盜,廷議設總兵官,督上下江防,擢文都督僉事任之。改蒞南京前府,專督操江、久之,卒。

贊曰:衛青等當承平時,不逞竊發,列城擾攘,賴其戡定。雖所敵非堅,然勇敢力戰,功多可紀。或遂身膏原野,若何洪、劉雄、馮禎輩,壯節有足惜者。鉞以心計定亂,銳以城守摧逆,干城之寄,克稱廟謨。神英、曹雄亦有勞績,而以附閹損名,且獲罪。為將者其以跅弛為戒哉。


\end{pinyinscope}