\article{後妃列傳}

明太祖鑒前代女禍,立綱陳紀,首嚴內教。洪武元年,命儒臣修女誡,諭翰林學士硃升曰:「治天下者,正家為先。正家之道,始於謹夫婦。後妃雖母儀天下,然不可俾預政事。至於嬪嬙之屬,不過備職事,侍巾櫛;恩寵或過,則驕恣犯分,上下失序。歷代宮闈,政由內出,鮮不為禍。惟明主能察於未然,下此多為所惑。卿等其纂女誡及古賢妃事可為法者,使後世子孫知所持守。」升等乃編錄上之。

五年六月,命禮臣議宮官女職之制。禮臣上言:「周制,後宮設內官以贊內治。漢設內官一十四等,凡數百人。唐設六局二十四司,官凡一百九十人,女史五十餘人,皆選良家女充之。」帝以所設過多,命重加裁定。於是折衷曩制,立六局一司。局曰尚宮、尚儀、尚服、尚食、尚寢、尚功,司曰宮正,秩皆正六品。每局領四司,其屬二十有四,而尚宮總行六局之事。戒令責罰,則宮正掌之。官七十五人,女史十八人,視唐減百四十餘人,凡以服勞宮寢、祗勤典守而已。諸妃位號亦惟取賢、淑、莊、敬、惠、順、康、寧為稱,閨房雍肅,旨寓深遠。又命工部制紅牌,鐫戒諭后妃之詞,懸於宮中。牌用鐵,字飾以金。復著令典,自后妃以下至嬪御女史,巨細衣食之費,金銀幣帛、器用百物之供,皆自尚宮取旨,牒內使監覆奏,移部臣取給焉。若尚宮不及奏,內使監不覆奏,而輒領於部者,論死。或以私書出外,罪亦如之。宮嬪以下有疾,醫者不得入宮,以證取藥。何其慎也!是以終明之代,宮壼肅清,論者謂其家法之善,超軼漢、唐。

爰自孝慈以迄愍后,考厥族里,次其世代,雖所遇不齊,顯晦異致,而凡居正號者並列於篇。其妃嬪有事實者,亦附見焉。

○后妃一

太祖孝慈高皇后孫貴妃李淑妃郭寧妃惠帝馬皇后成祖仁孝徐皇后王貴妃權賢妃仁宗誠孝張皇后宣宗恭讓胡皇后孝恭孫皇后吳賢妃郭嬪英宗孝莊錢皇后孝肅周太后景帝汪廢后肅孝杭皇后憲宗吳廢后孝貞王皇后孝穆紀太后孝惠邵太后萬貴妃

太祖孝慈高皇后馬氏,宿州人。父馬公,母鄭媼,早卒。馬公素善郭子興,遂以后託子興。馬公卒,子興育之如己女。子興奇太祖,以后歸焉。

后仁慈有智鑒,好書史。太祖有答刂記,輒命后掌之,倉卒未嘗忘。子興嘗信讒,疑太祖。后善事其妻,嫌隙得釋。太祖既克太平,后率將士妻妾渡江。及居江寧,吳、漢接境,戰無虛日,親緝甲士衣鞋佐軍。陳友諒寇龍灣,太祖率師禦之,后盡發宮中金帛犒士。嘗語太祖,定天下以不殺人為本。太祖善之。

洪武元年正月,太祖即帝位,冊為皇后。初,后從帝軍中,值歲大歉,帝又為郭氏所疑,嘗乏食。后竊炊餅,懷以進,肉為焦。居常貯糗Я脯修供帝,無所乏絕,而己不宿飽。及貴,帝比之「蕪蔞豆粥」,「滹沱麥飯」,每對群臣述后賢,同於唐長孫皇后。退以語后。后曰:「妾聞夫婦相保易,君臣相保難。陛下不忘妾同貧賤,願無忘群臣同艱難。且妾何敢比長孫皇后也!」

后勤於內治,暇則講求古訓。告六宮,以宋多賢后,命女史錄其家法,朝夕省覽。或言宋過仁厚,后曰:「過仁厚,不愈於刻薄乎?」一日,問女史:「黃老何教也,而竇太后好之?」女史曰:「清凈無為為本。若絕仁棄義,民復教慈,是其教矣。」后曰:「孝慈即仁義也,詎有絕仁義而為孝慈者哉?」后嘗誦《小學》,求帝表章焉。

帝前殿決事,或震怒,后伺帝還宮,輒隨事微諫。雖帝性嚴,然為緩刑戮者數矣。參軍郭景祥守和州,人言其子持槊欲殺父,帝將誅之。后曰:「景祥止一子,人言或不實,殺之恐絕其後。」帝廉之,果枉。李文忠守嚴州,楊憲誣其不法,帝欲召還。后曰:「嚴,敵境也,輕易將不宜。且文忠素賢,憲言詎可信?」帝遂已。文忠後卒有功。學士宋濂坐孫慎罪,逮至,論死,后諫曰:「民家為子弟延師,尚以禮全終始,況天子乎?且濂家居,必不知情。」帝不聽。會后侍帝食,不御酒肉。帝問故。對曰:「妾為宋先生作福事也。」帝惻然,投箸起。明日赦濂,安置茂州。吳興富民沈秀者,助築都城三之一,又請犒軍。帝怒曰:「匹夫犒天子軍,亂民也,宜誅。」后諫曰:「妾聞法者,誅不法也,非以誅不祥。民富敵國,民自不祥。不祥之民,天將災之,陛下何誅焉!」乃釋秀,戍雲南。帝嘗令重囚築城。后曰:「贖罪罰役,國家至恩。但疲囚加役,恐仍不免死亡。」帝乃悉赦之。帝嘗怒責宮人,后亦佯怒,令執付宮正司議罪。帝曰:「何為?」后曰:「帝王不以喜怒加刑賞。當陛下怒時,恐有畸重。付宮正,則酌其平矣。即陛下論人罪亦詔有司耳。」

一日,問帝:「今天下民安乎?」帝曰:「此非爾所宜問也。」后曰:「陛下天下父,妾辱天下母,子之安否,何可不問!」遇歲旱,輒率宮人蔬食,助祈禱;歲凶,則設麥飯野羹。帝或告以振恤。后曰:「振恤不如蓄積之先備也。」奏事官朝散,會食廷中,后命中官取飲食親嘗之。味弗甘,遂啟帝曰:「人主自奉欲薄,養賢宜厚。」帝為飭光祿官。帝幸太學還,後問生徒幾何,帝曰:「數千。」后曰:「人才眾矣。諸生有廩食,妻子將何所仰給?」於是立紅板倉,積糧賜其家。太學生家糧自后始。諸將克元都,俘寶玉至。后曰:「元有是而不能守,意者帝王自有寶歟。」帝曰:「朕知后謂得賢為寶耳。」后拜謝曰:「誠如陛下言。妾與陛下起貧賤,至今日,恒恐驕縱生於奢侈,危亡起於細微,故願得賢人共理天下。」又曰:法屢更必弊,法弊則奸生;民數擾必困,民困則亂生。」帝嘆曰:「至言也。」命女史書之冊。其規正,類如此。

帝每御膳,后皆躬自省視。平居服大練浣濯之衣,雖敝不忍易。聞元世祖后煮故弓弦事,亦命取練織為衾裯,以賜高年煢獨。餘帛纇絲,緝成衣裳,賜諸王妃公主,使知天桑艱難。妃嬪宮人被寵有子者,厚待之。命婦入朝,待之如家人禮。帝欲訪后族人官之,后謝曰:「爵祿私外家,非法。」力辭而止。然言及父母早卒,輒悲哀流涕。帝封馬公徐王,鄭媼為王夫人,修墓置廟焉。

洪武十五年八月寢疾。群臣請禱祀,求良醫。后謂帝曰:「死生,命也,禱祀何益!且醫何能活人!使服藥不效,得毋以妾故而罪諸醫乎?」疾亟,帝問所欲言。曰:「願陛下求賢納諫,慎終如始,子孫皆賢,臣民得所而已。」是月丙戌崩,年五十一。帝慟哭,遂不復立后。是年九月庚午葬孝陵,謚曰孝慈皇后。宮人思之,作歌曰:「我后聖慈,化行家邦。撫我育我,懷德難忘。懷德難忘,於萬斯年。毖彼下泉,悠悠蒼天。」永樂元年上尊謚曰孝慈昭憲至仁文德承天順聖高皇后。嘉靖十七年加上尊謚曰孝慈貞化哲順仁徽成天育聖至德高皇后。

成穆貴妃孫氏,陳州人。元末兵亂,妃父母俱亡,從仲兄蕃避兵揚州。青軍陷城,元帥馬世熊得之,育為義女。年十八,太祖納焉。及即位,冊封貴妃,位眾妃上。洪武七年九月薨,年三十有二。帝以妃無子,命周王橚行慈母服三年,東宮、諸王皆期。敕儒臣作《孝慈錄》。庶子為生母服三年,眾子為庶母期,自妃始。葬褚岡。賜兄瑛田租三百石,歲供禮。後附葬孝陵。

淑妃李氏,壽州人。父傑,洪武初,以廣武衛指揮北征,卒於陣。十七年九月,孝慈皇后服除,冊封淑妃,攝六宮事。未幾,薨。

寧妃郭氏,濠人郭山甫女。山甫善相人。太祖微時過其家,山甫相之,大驚曰:「公相貴不可言。」因謂諸子興、英曰:「吾相汝曹皆可封侯者以此。」亟遣從渡江,並遣妃侍太祖。後封寧妃。李淑妃薨,妃攝六宮事。山甫累贈營國公,興、英皆以功封侯,自有傳。

惠帝皇后馬氏,光祿少卿全女也。洪武二十八年冊為皇太孫妃。建文元年二月冊為皇后。四年六月,城陷,崩於火。

成祖仁孝皇后徐氏,中山王達長女也。幼貞靜,好讀書,稱女諸生。太祖聞后賢淑,召達謂曰:「朕與卿,布衣交也。古君臣相契者,率為婚姻。卿有令女,其以朕子棣配焉。」達頓首謝。

洪武九年,冊為燕王妃。高皇后深愛之。從王之籓,居孝慈高皇后喪三年,蔬食如禮。高皇后遺言可誦者,后一一舉之不遺。

靖難兵起,王襲大寧,李景隆乘間進圍北平。時仁宗以世子居守,凡部分備禦,多稟命於后。景隆攻城急,城中兵少,后激勸將校士民妻,皆授甲登陴拒守,城卒以全。

王即帝位,冊為皇后。言:「南北每年戰鬥,兵民疲敝,宜與休息。」又言:「當世賢才皆高皇帝所遺,陛下不宜以新舊間。」又言:「帝堯施仁自親始。」帝輒嘉納焉。初,后弟增壽常以國情輸之燕,為惠帝所誅,至是欲贈爵,后力言不可。帝不聽,竟封定國公,命其子景昌襲,乃以告后。后曰:「非妾志也。」終弗謝。嘗言漢、趙二王性不順,官僚宜擇廷臣兼署之。一日,問:「陛下誰與圖治者?」帝曰:「六卿理政務,翰林職論思。」后因請悉召見其命婦,賜冠服鈔幣。諭曰:「婦之事夫,奚止饋食衣服而已,必有助焉。朋友之言,有從有違,夫婦之言,婉順易入。吾旦夕侍上,惟以生民為念,汝曹勉之。」嘗採《女憲》、《女誡》作《內訓》二十篇,又類編古人嘉言善行,作《勸善書》,頒行天下。

永樂五年七月,疾革,惟勸帝愛惜百姓,廣求賢才,恩禮宗室,毋驕畜外家。又告皇太子:「曩者北平將校妻為我荷戈城守,恨未獲隨皇帝北巡,一賚恤之也。」是月乙卯崩,年四十有六。帝悲慟,為薦大齋於靈谷、天禧二寺,聽群臣致祭,光祿為具物。十月甲午,謚曰仁孝皇后。七年營壽陵於昌平之天壽山,又四年而陵成,以后葬焉,即長陵也。帝亦不復立后。仁宗即位,上尊謚曰仁孝慈懿誠明莊獻配天齊聖文皇后,祔太廟。

昭獻貴妃王氏,蘇州人。永樂七年封貴妃。妃有賢德,事仁孝皇后恭謹,為帝所重。帝晚年多急怒。妃曲為調護,自太子諸王公主以下皆倚賴焉。十八年七月薨,禮視太祖成穆孫貴妃。

恭獻賢妃權氏,朝鮮人。永樂時,朝鮮貢女充掖庭,妃與焉。姿質穠農粹,善吹玉簫。帝愛憐之。七年封賢妃,命其父永均為光祿卿。明年十月侍帝北征。凱還,薨於臨城,葬嶧縣。

仁宗誠孝皇后張氏,永城人。父麒以女貴,追封彭城伯,具《外戚傳》。洪武二十八年封燕世子妃。永樂二年封皇太子妃。仁宗立,冊為皇后。宣宗即位,尊為皇太后。英宗即位,尊為太皇太后。

后始為太子妃,操婦道至謹,雅得成祖及仁孝皇后歡。太子數為漢、趙二王所間,體肥碩不能騎射。成祖恚,至減太子宮膳,瀕易者屢矣,卒以后故得不廢。及立為后,中外政事,莫不周知。

宣德初,軍國大議多稟聽裁決。是時海內寧泰,帝入奉起居,出奉遊宴,四方貢獻,雖微物必先上皇太后。兩宮慈孝聞天下。三年,太后遊西苑,皇后皇妃侍,帝親掖輿登萬歲山,奉觴上壽,獻詩頒德。又明年謁長、獻二陵,帝親鞬騎導。至河橋,下馬扶輦。畿民夾道拜觀,陵旁老稚皆山呼拜迎。太后顧曰:「百姓戴君,以能安之耳,皇帝宜重念。」及還,過農家,召老婦問生業,賜鈔幣。有獻蔬食酒漿者,取以賜帝,曰:「此田家味也。」從臣英國公張輔,尚書蹇義,大學士楊士奇、楊榮、金幼孜、楊溥請見行殿。太后慰勞之,且曰:「爾等先朝舊人,勉輔嗣君。」他日,帝謂士奇曰:「皇太后謁陵還,道汝輩行事甚習。言輔,武臣也,達大義。義重厚小心,第寡斷。汝克正,言無避忤,先帝或數不樂,然終從汝,以不敗事。又有三事,時悔不從也。」太后遇外家嚴,弟昇至淳謹,然不許預議國事。

宣宗崩,英宗方九歲,宮中訛言將召立襄王矣。太后趣召諸大臣至乾清宮,指太子泣曰:「此新天子也。」君臣呼萬歲,浮言乃息。大臣請太后垂簾聽政,太后曰:「毋壞祖宗法。第悉罷一切不急務。」時時勖帝向學,委任股肱,以故王振雖寵於帝,終太后世不敢專大政。

正統七年十月崩。當大漸,召士奇、溥入,命中官問國家尚有何大事未辦者。士奇舉三事。一謂建庶人雖亡,當修實錄。一謂太宗詔有收方孝孺諸臣遺書者死,宜弛其禁。其三未及奏上,而太后已崩。遺詔勉大臣佐帝惇行仁政,語甚諄篤。上尊謚曰誠孝恭肅明德弘仁順天啟聖昭皇后,合葬獻陵,祔太廟。

宣宗恭讓皇后胡氏,名善祥,濟寧人。永樂十五年選為皇太孫妃。已,為皇太子妃。宣宗即位,立為皇后。時孫貴妃有寵,后未有子,又善病。三年春,帝令后上表辭位,乃退居長安宮,賜號靜慈仙師,而冊貴妃為后。諸大臣張輔、蹇義、夏原吉、楊士奇、楊榮等不能爭。張太后憫后賢,常召居清寧宮。內廷朝宴,命居孫后上。孫后常怏怏。正統七年十月,太皇太后崩,后痛哭不已,踰年亦崩,用嬪御禮葬金山。

后無過被廢,天下聞而憐之。宣宗後亦悔。嘗自解曰:「此朕少年事。」天順六年,孫太后崩,錢皇后為英宗言:「后賢而無罪,廢為仙師。其沒也,人畏太后,殮葬皆不如禮。」因勸復其位號。英宗問大學士李賢。賢對曰:「陛下此心,天地鬼神實臨之。然臣以陵寢、享殿、神主俱宜如奉先殿式,庶稱陛下明孝。」七年閏七月,上尊謚曰恭讓誠順康穆靜慈章皇后,修陵寢,不祔廟。

宣宗孝恭皇后孫氏,鄒平人。幼有美色。父忠,永城縣主簿也。誠孝皇后母彭城伯夫人,故永城人,時時入禁中,言忠有賢女,遂得入宮。方十餘歲,成祖命誠孝后育之。已而宣宗婚,詔選濟寧胡氏為妃,而以孫氏為嬪。宣宗即位,封貴妃。故事:皇后金寶金冊,貴妃以下,有冊無寶。妃有寵,宣德元年五月,帝請於太后,製金寶賜焉。貴妃有寶自此始。

妃亦無子,陰取宮人子為己子,即英宗也,由是眷寵益重。胡后上表遜位,請早定國本。妃偽辭曰:「后病痊自有子,吾子敢先后子耶?」三年三月,胡后廢,遂冊為皇后。英宗立,尊為皇太后。

英宗北狩,太后命郕王監國。景帝即位,尊為上聖皇太后。時英宗在迤北,數寄禦寒衣裘。及還,幽南宮,太后數入省視。石亨等謀奪門,先密白太后。許之。英宗復辭,上徽號曰聖烈慈壽皇太后。明興,宮闈徽號亦自此始。天順六年九月崩,上尊謚曰孝恭懿憲慈仁莊烈齊天配聖章皇后,合葬景陵,祔太廟。而英宗生母,人卒無知之者。

吳太后,景帝母也,丹徒人。宣宗為太子時,選入宮。宣德三年封賢妃。景帝即位,尊為皇太后。英宗復辟,復稱宣廟賢妃。成化中薨。

郭嬪,名愛,字善理,鳳陽人。賢而有文,入宮二旬而卒。自知死期,書楚聲以自哀。詞曰:「修短有數兮,不足較也。生而如夢兮,死則覺也。先吾親而歸兮,慚予之失孝也。心心妻心妻而不能已兮,是則可悼也。」正統元年八月,追贈皇庶母惠妃何氏為貴妃,謚端靜;趙氏為賢妃,謚純靜;吳氏為惠妃,謚貞順;焦氏為淑妃,謚莊靜;曹氏為敬妃,謚莊順;徐氏為順妃,謚貞惠;袁氏為麗妃,謚恭定;諸氏為淑妃,謚貞靜;李氏為充妃,謚恭順;何氏為成妃,謚肅僖。冊文曰:「茲委身而蹈義,隨龍馭以上賓,宜薦徽稱,用彰節行。」蓋宣宗殉葬宮妃也。

初,太祖崩,宮人多從死者。建文、永樂時,相繼優恤。若張鳳、李衡、趙福、張璧、汪賓諸家,皆自錦衣衛所試百戶、散騎帶刀舍人進千百戶,帶俸世襲,人謂之「太祖朝天女戶」。歷成祖,仁、宣二宗亦皆用殉。景帝以郕王薨,猶用其制,蓋當時王府皆然。至英宗遺詔,始罷之。

英宗孝莊皇后錢氏,海州人。正統七年立為后。帝憫后族單微,欲侯之,后輒遜謝。故后家獨無封。英宗北狩,傾中宮貲佐迎駕。夜哀泣籲天,倦即臥地,損一股。以哭泣復損一目。英宗在南宮,不自得,后曲為慰解。后無子,周貴妃有子,立為皇太子。英宗大漸,遺命曰:「錢皇后千秋萬歲後,與朕同葬。」大學士李賢退而書之冊。

憲宗立,上兩宮徽號,下廷臣議。太監夏時希貴妃意,傳諭獨尊貴妃為皇太后。大學士李賢、彭時力爭,乃兩宮並尊,而稱后為慈懿皇太后。及營裕陵,賢、時請營三壙,下廷議。夏時復言不可,事竟寢。

成化四年六月,太后崩,周太后不欲后合葬。帝使夏時、懷恩召大臣議。彭時首對曰:「合葬裕陵,主祔廟,定禮也。」翼日,帝召問,時對如前。帝曰:「朕豈不知,慮他日妨母后耳。」時曰:「皇上孝事兩宮,聖德彰聞。禮之所合,孝之所歸也。」商輅亦言:「不祔葬,損聖德。」劉定之曰:「孝從義,不從命。」帝默然久之,曰:「不從命尚得為孝耶!」時力請合葬裕陵左,而虛右以待周太后。已,復與大臣疏爭,帝再下廷議。吏部尚書李秉、禮部尚書姚夔集廷臣九十九人議,皆請如時言。帝曰:「卿等言是,顧朕屢請太后未得命。乖禮非孝,違親亦非孝。」明日,詹事柯潛、給事中魏元等上疏,又明日,夔等合疏上,皆執議如初。中旨猶諭別擇葬地。於是百官伏哭文華門外。帝命群臣退。眾叩頭,不得旨不敢退。自已至申,乃得允。眾呼萬歲出。事詳時、夔傳中。是年七月上尊謚曰孝莊獻穆弘惠顯仁恭天欽聖睿皇后,祔太廟。九月合葬裕陵,異隧,距英宗玄堂數丈許,中窒之,虛石壙以待周太后,其隧獨通,而奉先殿祭,亦不設后主。

弘治十七年,周太后崩。孝宗御便殿,出裕陵圖,示大學士劉健、謝遷、李東陽曰:「陵有二隧,若者窒,若者可通往來,皆先朝內臣所為,此未合禮。昨見成化間彭時、姚夔等章奏,先朝大臣為國如此,先帝亦不得已耳。欽天監言通隧上乾先帝陵堂,恐動地脈,朕已面折之。窒則天地閉塞,通則風氣流行。」健等因力贊。帝復問祔廟禮,健等言:「祔二后,自唐始也。祔三后,自宋始也,漢以前一帝一后。曩者定議合祔,孝莊太后居左,今大行太皇太后居右,且引唐、宋故事為證,臣等以此不敢復論。」帝曰:「二后已非,況復三后!」遷曰:「宋祔三后,一繼立,一生母也。」帝曰:「事須師古,太皇太后鞠育朕躬,朕豈敢忘?顧私情耳。祖宗來,一帝一后。今並祔,壞禮自朕始。且奉先殿祭皇祖,特座一飯一匙而已。夫孝穆皇太后,朕生母也,別祀之奉慈殿。今仁壽宮前殿稍寬,朕欲奉太皇太后於此,他日奉孝穆皇太后於後,歲時祭享,如太廟。」於是命群臣詳議。議上,將建新廟,欽天監奏年方有礙。廷議請暫祀周太后於奉慈殿,稱孝肅太皇太后。殿在奉先殿西,帝以祀孝穆,至是中奉孝肅而徙孝穆居左焉。帝始欲通隧,亦以陰陽家言,不果行。

孝肅周太后,英宗妃,憲宗生母也,昌平人。天順元年封貴妃。憲宗即位,尊為皇太后。其年十月,太后誕日,帝令僧道建齋祭。禮部尚書姚夔帥群臣詣齋所,為太后祈福。給事中張寧等劾之。帝是其言,令自後僧道齋醮,百官不得行香。二十三年四月上徽號曰聖慈仁壽皇太后。孝宗立,尊為太皇太后。

先是,憲宗在位,事太后至孝,五日一朝,燕享必親。太后意所欲,惟恐不懽。至錢太后合葬裕陵,太后殊難之。憲宗委曲寬譬,乃得請。孝宗生西宮,母妃紀氏薨,太后育之宮中,省視萬方。及孝宗即位,事太后亦至孝。太后病瘍,久之愈,誥諭群臣曰:「自英皇厭代,予正位長樂,憲宗皇帝以天下養,二十四年猶一日。茲予偶患瘍,皇帝夜籲天,為予請命,春郊罷宴,問視惟勤,俾老年疾體,獲底康寧。以昔視今,父子兩世,孝同一揆,予甚嘉焉。」

弘治十一年冬,清寧宮災,太后移居仁壽宮。明年,清寧宮成,乃還居焉。太后弟長寧伯彧家有賜田,有司請釐正之,帝未許也,太后曰:「奈何以我故骫皇帝法!」使歸地於官。

弘治十七年三月崩,謚孝肅貞順康懿光烈輔天承聖睿皇后,合葬裕陵。以大學士劉健、謝遷、李東陽議,別祀於奉慈殿,不祔廟,仍稱太皇太后。嘉靖十五年,與紀、邵二太后並移祀陵殿,題主曰皇后,不繫帝謚,以別嫡庶。其後穆宗母孝恪、神宗母孝定、光宗母孝靖、熹宗母孝和、莊烈帝母孝純,咸遵用其制。

景帝廢后汪氏,順天人。正統十年冊為郕王妃。十四年冬,王即皇帝位,冊為皇后。后有賢德,嘗念京師諸死事及老弱遇害者暴骨原野,令官校掩埋之。生二女,無子。景泰三年,妃杭氏生子見濟,景帝欲立為太子,而廢憲宗,后執不可。以是忤帝意,遂廢后,立杭氏為皇后。七年,杭后崩,謚肅孝。英宗復位,削皇后號,毀所葬陵,而后仍稱郕王妃。景帝崩,英宗以其後宮唐氏等殉,議及后。李賢曰:「妃已幽廢,況兩女幼,尤可憫。」帝乃已。

憲宗復立為太子,雅知后不欲廢立,事之甚恭。因為帝言,遷之外王府,得盡攜宮中所有而出。與周太后相得甚歡,歲時入宮,敘家人禮。然性剛執。一日,英宗問太監劉桓曰:「記有玉玲瓏繫腰,今何在?」桓言當在妃所。英宗命索之。后投諸井,對使者曰:「無之。」已而告人曰:「七年天子,不堪消受此數片玉耶!」已,有言后出所攜鉅萬計,英宗遣使檢取之,遂立盡。正德元年十二月薨,議祭葬禮。大學士王鏊曰:「葬以妃,祭以后。」遂合葬金山。明年上尊謚曰貞惠安和景皇后。

憲宗廢后吳氏,順天人。天順八年七月立為皇后。先是,憲宗居東宮,萬貴妃已擅寵。后既立,摘其過,杖之。帝怒,下詔曰:「先帝為朕簡求賢淑,已定王氏,育於別宮待期。太監牛玉輒以選退吳氏於太后前復選。冊立禮成之後,朕見舉動輕佻,禮度率略,德不稱位,因察其實,始知非預立者。用是不得已,請命太后,廢吳氏別宮。」立甫踰月耳。后父俊,先授都督同知,至是下獄戍邊。謫玉孝陵種菜,玉從子太常少卿綸、甥吏部員外郎楊琮並除名,姻家懷寧侯孫鏜閑住。於是南京給事中王徽、王淵、朱寬、李翱、李鈞等合疏言玉罪重罰輕,因並劾大學士李賢。帝怒,徽等皆貶邊州判官。

後孝宗生於西宮,后保抱惟謹。孝宗即位,念后恩,命服膳皆如母后禮,官其姪錦衣百戶。正德四年薨。劉瑾欲焚之。大學士王鏊持不可,乃以妃禮葬。

孝貞皇后王氏,上元人。憲宗在東宮,英宗為擇配,得十二人,選后及吳氏、柏氏留宮中。吳氏既立而廢,遂冊為皇后,天順八年十月也。萬貴妃寵冠後宮,后處之淡如。孝宗即位,尊為皇太后。武宗即位,尊為太皇太后。正德五年十二月上尊號曰慈聖康壽。十三年二月崩,上尊謚曰孝貞莊懿恭靖仁慈欽天輔聖純皇后,合葬茂陵,祔太廟。

孝穆紀太后,孝宗生母也,賀縣人。本蠻土官女。成化中征蠻,俘入掖庭,授女史,警敏通文字,命守內藏。時萬貴妃專寵而妒,後宮有娠者皆治使墮。柏賢妃生悼恭太子,亦為所害。帝偶行內藏,應對稱旨,悅,幸之,遂有身。萬貴妃知而恚甚,令婢鉤治之。婢謬報曰病痞。乃謫居安樂堂。久之,生孝宗,使門監張敏溺焉。敏驚曰:「上未有子,奈何棄之。」稍哺粉餌飴蜜,藏之他室,貴妃日伺無所得。至五六歲,未敢剪胎髮。時吳后廢居西內,近安樂堂,密知其事,往來哺養,帝不知也。

帝自悼恭太子薨後,久無嗣,中外皆以為憂。成化十一年,帝召張敏櫛發,照鏡嘆曰:「老將至而無子。」敏伏地曰:「死罪,萬歲已有子也。」帝愕然,問安在。對曰:「奴言即死,萬歲當為皇子主。」於是太監懷恩頓首曰:「敏言是。皇子潛養西內,今已六歲矣,匿不敢聞。」帝大喜,即日幸西內,遣使往迎皇子。使至,妃抱皇子泣曰:「兒去,吾不得生。兒見黃袍有鬚者,即兒父也。」衣以小緋袍,乘小輿,擁至階下,髮披地,走投帝懷。帝置之膝,撫視久之,悲喜泣下曰:「我子也,類我。」使懷恩赴內閣具道其故。群臣皆大喜。明日,入賀,頒詔天下。移妃居永壽宮,數召見。萬貴妃日夜怨泣曰:「群小紿我。」其年六月,妃暴薨。或曰貴妃致之死,或曰自縊也。謚恭恪莊僖淑妃。敏懼,亦吞金死。敏,同安人。

孝宗既立為皇太子,時孝肅皇太后居仁壽宮,語帝曰:「以兒付我。」太子遂居仁壽。一日,貴妃召太子食,孝肅謂太子曰:「兒去,無食也。」太子至,貴妃賜食,曰:「已飽。」進羹,曰:「疑有毒。」貴妃大恚曰:「是兒數歲即如是,他日魚肉我矣。」因恚而成疾。孝宗即位,追謚淑妃為孝穆慈慧恭恪莊僖崇天承聖純皇后,遷葬茂陵,別祀奉慈殿。帝悲念太后,特遣太監蔡用求太后家,得紀父貴、紀祖旺兄弟以聞。帝大喜,詔改父貴為貴,授錦衣衛指揮同知;祖旺為旺,授錦衣衛指揮僉事。賜予第宅、金帛、莊田、奴婢,不可勝計。追贈太后父為中軍都督府左都督,母為夫人。其曾祖、祖父亦如之。遣修太后先塋之在賀者,置守墳戶,復其家。

先是,太后在宮中,嘗自言家賀縣,姓紀,幼不能知親族也。太監郭鏞聞而識之。太監陸愷者,亦廣西人,故姓李,蠻中紀、李同音,因妄稱太后兄,令人訪其族人詣京師。愷女兄夫韋父成者出冒之,有司待以戚畹,名所居里曰迎恩里。貴、旺曰:「韋猶冒李,況我實李氏。」因詐為宗系上有司,有司莫辨也。二人既驟貴,父成亦詣闕爭辨。帝命郭鏞按之。鏞逐父成,猶令馳驛歸。及帝使治后先塋,蠻中李姓者數輩,皆稱太后家,自言於使者。使者還,奏貴、旺不實。復遣給事中孫珪、御史滕祐間行連、賀間,微服入瑤、僮中訪之,盡得其狀,歸奏。帝謫罰鏞等有差,戍貴、旺邊海。自此帝數求太后家,竟不得。

弘治三年,禮部尚書耿裕奏曰:「粵西當大征之後,兵燹饑荒,人民奔竄,歲月悠遠,蹤跡難明。昔孝慈高皇后與高皇帝同起艱難,化家為國,徐王親高皇后父,當后之身,尋求家族,尚不克獲,然後立廟宿州,春秋祭祀。今紀太后幼離西粵,入侍先帝,連、賀非徐、宿中原之地,嬪宮無母后正位之年,陛下訪尋雖切,安從得其實哉!臣愚謂可仿徐王故事,定擬太后父母封號,立祠桂林致祭。」帝曰:「孝穆皇太后早棄朕躬,每一思念,惄焉如割。初謂宗親尚可旁求,寧受百欺,冀獲一是。卿等謂歲久無從物色,請加封立廟,以慰聖母之靈。皇祖既有故事,朕心雖不忍,又奚敢違。」於是封后父推誠宣力武臣特進光祿大夫柱國慶元伯,謚端僖,后母伯夫人,立廟桂林府,有司歲時祀。大學士尹直撰哀冊有云:「睹漢家堯母之門,增宋室仁宗之慟。」帝燕閒念誦,輒欷歔流涕也。

孝惠邵太后,憲宗妃,興獻帝母也。父林,昌化人,貧甚,鬻女於杭州鎮守太監,妃由此入宮。知書,有容色。成化十二年封宸妃,尋進封貴妃。興王之籓,妃不得從。世宗入繼大統,妃已老,目眚矣,喜孫為皇帝,摸世宗身,自頂至踵。已,尊為皇太后。嘉靖元年上尊號曰壽安。十一月崩。帝欲明年二月遷葬茂陵,大學士楊廷和等言:「祖陵不當數興工作,驚動神靈。」不從。謚曰孝惠康肅溫仁懿順協天祐聖皇太后,別祀奉慈殿。七年七月改稱太皇太后。十五年遷主陵殿,稱皇后,與孝肅、孝穆等。

恭肅貴妃萬氏,諸城人。四歲選入掖廷,為孫太后宮女。及長,侍憲宗於東宮。憲宗年十六即位,妃已三十有五,機警,善迎帝意,遂讒廢皇后吳氏,六宮希得進御。帝每遊幸,妃戎服前驅。成化二年正月生皇第一子,帝大喜,遣中使祀諸山川,遂封貴妃。皇子未期薨,妃亦自是不復娠矣。

當是時,帝未有子,中外以為憂,言者每請溥恩澤以廣繼嗣。給事中李森、魏元,御史康永韶等先後言尤切。四年秋,彗星屢見。大學士彭時、尚書姚夔亦以為言。帝曰:「內事也,朕自主之。」然不能用。妃益驕。中官用事者,一忤意,立見斥逐。掖廷御幸有身,飲藥傷墜者無數。孝宗之生,頂寸許無髮,或曰藥所中也。紀淑妃之死,實妃為之。佞倖錢能、覃勤、汪直、梁芳、韋興輩皆假貢獻,苛斂民財,傾竭府庫,以結貴妃歡。奇技淫巧,禱祠宮觀,糜費無算。久之,帝後宮生子漸多,芳等懼太子年長,他日立,將治己罪,同導妃勸帝易儲。會泰山震,占者謂應在東宮。帝心懼,事乃寢。

二十三年春,暴疾薨,帝輟朝七日。謚曰恭肅端慎榮靖皇貴妃,葬天壽山。弘治初,御史曹璘請削妃謚號;魚臺縣丞徐頊請逮治診視紀太后諸醫,捕萬氏家屬,究問當時薨狀。孝宗以重違先帝意,已之。

孝宗孝康張皇后武宗孝靜夏皇后世宗孝潔陳皇后張廢后孝烈方皇后孝恪杜太后穆宗孝懿李皇后孝安陳皇后孝定李太后神宗孝端王皇后劉昭妃孝靖王太后鄭貴妃光宗孝元郭皇后孝和王太后孝純劉太后李康妃李莊妃趙選侍熹宗懿安張皇后張裕妃莊烈帝愍周皇后田貴妃

孝宗孝康皇后張氏,興濟人。父巒,以鄉貢入太學。母金氏,夢月入懷而生后。成化二十三年選為太子妃。是年,孝宗即位,冊立為皇后。帝頗優禮外家,追封巒昌國公,封后弟鶴齡壽寧侯,延齡建昌伯,為後立家廟於興濟,工作壯麗,數年始畢。鶴齡、延齡並注籍宮禁,縱家人為奸利,中外諸臣多以為言,帝以后故不問。

武宗即位,尊為皇太后。五年十二月,以寘鐇平,上尊號曰慈壽皇太后。世宗入繼,稱聖母,加上尊號曰昭聖慈壽。嘉靖三年加上昭聖康惠慈壽。已,改稱伯母。十五年復加上昭聖恭安康惠慈壽。二十年八月崩,謚曰孝康靖肅莊慈哲懿翊天贊聖敬皇后,合葬泰陵,祔廟。

武宗之崩也,江彬等懷不軌。賴后與大學士楊廷和定策禁中,迎立世宗,而世宗事后顧日益薄。元年大婚,初傳昭聖懿旨,既復改壽安太后。壽安者,憲宗妃,興獻帝生母也。廷和爭之,乃止。三年,興國太后誕節,敕命婦朝賀,燕賚倍常。及后誕日,敕免賀。修撰舒芬疏諫,奪俸。御史朱淛、馬明衡、陳逅、季本,員外郎林惟聰等先後言,皆得罪。竟罷朝賀。

初,興國太后以籓妃入,太后猶以故事遇之,帝頗不悅。及帝朝,太后待之又倨。會太后弟延齡為人所告,帝坐延齡謀逆論死,太后窘迫無所出。哀沖太子生,請入賀,帝謝不見。使人請,不許。大學士張孚敬亦為延齡請,帝手敕曰:「天下者,高皇帝之天下,孝宗皇帝守高皇帝法。卿慮傷伯母心,豈不慮傷高、孝二廟心耶?」孚敬復奏曰:「陛下嗣位時,用臣言,稱伯母皇太后,朝臣歸過陛下,至今未已。茲者大小臣工默無一言,誠幸太后不得令終,以重陛下過耳。夫謀逆之罪,獄成當坐族誅,昭聖獨非張氏乎?陛下何以處此!」冬月慮囚,帝又欲殺延齡,復以孚敬言而止。亡何,奸人劉東山者告變,并逮鶴齡下詔獄。太后至衣敝襦席槁為請,亦不聽。久之,鶴齡瘐死。及太后崩,帝竟殺延齡,事詳《外戚傳》。

武宗孝靜皇后夏氏,上元人。正德元年冊立為皇后。嘉靖元年上尊稱曰莊肅皇后。十四年正月崩,合葬康陵,祔廟。初,禮臣上喪儀,帝曰:「嫂叔無服,且兩宮在上,朕服青,臣民如母后服。」禮部尚書夏言曰:「皇上以嫂叔絕服,則群臣不敢素服見皇上,請暫罷朝參。」許之。已而議謚,大學士張孚敬曰:「大行皇后,上嫂也,與累朝元后異,宜用二字或四字。」李時曰:「宜用八。」左都御史王廷相、吏部侍郎霍韜等曰:「均帝后也,何殊!」言集眾議,因奏曰:「古人尚質,謚法簡,稱其行,後人增加,臣子情也。生今世,宜行今制。大行皇后宜如列聖元后謚,二四及八,於禮無據。」帝不從,命再議。群臣請如孚敬言。帝曰:「用六,合陰數焉。」於是上謚孝靜莊惠安肅毅皇后。十五年,帝覺孚敬言非是,敕曰:「孝靜皇后謚不備,不稱配武宗。乃改謚孝靜莊惠安肅溫誠順天偕聖毅皇后。

世宗孝潔皇后陳氏,元城人。嘉靖元年冊立為皇后。帝性嚴厲。一日,與后同坐,張、方二妃進茗,帝循視其手。后恚,投杯起。帝大怒。后驚悸,墮娠崩,七年十月也。喪禮從殺。帝素服御西角門十日,即玄冠玄裳御奉天門,謚曰悼靈,葬襖兒峪。葬之日,梓宮出王門,百官一日臨。給事中王汝梅諫。不聽。十五年,禮部尚書夏言議請改謚。時帝意久釋矣,乃改謚曰孝潔。穆宗即位,禮臣議:「孝潔皇后,大行皇帝元配,宜合葬祔廟。若遵遺制祔孝烈,則舍元配也,若同祔,則二后也。大行皇帝升祔時,宜奉孝潔配,遷葬永陵,孝烈主宜別祀。」報可。隆慶元年二月上尊謚曰孝潔恭懿慈睿安莊相天翊聖肅皇后。

廢后張氏,世宗第二后也。初封順妃。七年,陳皇后崩,遂立為后。是時,帝方追古禮,令后率嬪御親蠶北郊,又日率六宮聽講章聖《女訓》於宮中。十三年正月廢居別宮。十五年薨,喪葬儀視宣宗胡廢后。

孝烈皇后方氏,世宗第三后也,江寧人。帝即位且十年,未有子。大學士張孚敬言:「古者天子立后,並建六宮、三夫人、九嬪、二十七世婦、八十一御妻,所以廣嗣也。陛下春秋鼎盛,宜博求淑女,為子嗣計。」從之。十年三月,后與鄭氏、王氏、閻氏、韋氏、沈氏、盧氏、沈氏、杜氏同冊為九嬪,冠九翟冠,大采鞠衣,圭用次玉,穀文,冊黃金塗,視皇后殺五分之一。至期,帝袞冕告太廟,還服皮弁,御華蓋殿,傳制,遣大臣行冊禮。既冊,從皇后朝奉先殿。禮成,帝服皮弁,受百官賀,蓋創禮也。張后廢,遂立為后,而封沈氏為宸妃,閻氏為麗妃。舊制:立后,謁內廟而已。至是,下禮臣議廟見禮。於是群臣以天子立三宮以承宗廟,《禮經》有廟見之文,乃考據《禮經》,參稽《大明集禮》,擬儀注以上。至期,帝率后謁太廟及世廟。越三日,頒詔天下。明日,受命婦朝。

二十一年,宮婢楊金英等謀弒逆,帝賴后救得免,乃進后父泰和伯銳爵為侯。初,曹妃有色,帝愛之,冊為端妃。是夕,帝宿端妃宮。金英等伺帝熟寢,以組縊帝項,誤為死結,得不絕。同事張金蓮知事不就,走告后。后馳至,解組,帝蘇。后命內監張佐等捕宮人雜治,言金英等弒逆,王寧嬪首謀。又曰:曹端妃雖不與,亦知謀。時帝病悸不能言,后傳帝命收端妃、寧嬪及金英等悉礫於市。并誅其族屬十餘人。然妃實不知也。久之,帝始知其冤。

二十六年十一月乙未,后崩。詔曰:「皇后比救朕危,奉天濟難,其以元后禮葬。」預名葬地曰永陵,謚孝烈,親定謚禮,視昔加隆焉。禮成,頒詔天下。及大祥,禮臣請安主奉先殿東夾室,帝曰:「奉先殿夾室,非正也,可即祔太廟。」於是大學士嚴嵩等請設位於太廟東,皇妣睿皇后之次,後寢藏主則設幄於憲廟皇祖妣之右,以從祔於祖姑之義。帝曰:「祔禮至重,豈可權就。后非帝,乃配帝者,自有一定之序,安有享從此而主藏彼之禮!其祧仁宗,祔以新序,即朕位次,勿得亂禮。」嵩曰:「祔新序,非臣下所敢言,且陰不可當陽位。」乃命姑藏主睿皇后側。

二十九年十月,帝終欲祔后太廟,命再議。尚書徐階言不可,給事中楊思忠是階議,餘無言者。帝覘知狀。及議疏入,謂:「后正位中宮,禮宜祔享,但遽及廟次,則臣子之情,不唯不敢,實不忍也。宜設位奉先殿。」帝震怒。階、思忠惶恐言:「周建九廟,三昭三穆。國朝廟制,同堂異室,與《周禮》不同。今太廟九室皆滿,若以聖躬論,仁宗當祧,固不待言,但此乃異日聖子神孫之事。臣聞夏人之廟五,商以七,周以九。禮由義起,五可七,七可九,九之外亦可加也。請於太廟及奉先殿各增二室,以祔孝烈,則仁宗可不必祧,孝烈皇后可速正南面之位,陛下亦無預祧以俟之嫌。」帝曰:「臣子之誼,當祧當祔,力請可也。茍禮得其正,何避豫為!」於是階等復會廷臣上言:「唐、虞、夏五廟,其祀皆止四世。周九廟,三昭三穆,然而兄弟相及,亦不能盡足六世。今仁宗為皇上五世祖,以聖躬論,仁宗於禮當祧,孝烈皇后於禮當祔。請祧仁宗,祔孝烈皇后於太廟第九室。」因上祧祔儀注。

已而請忌日祭,帝猶銜前議,報曰:「孝烈繼后,所奉者又入繼之君,忌不祭亦可。」階等請益力,帝曰:「非天子不議禮。后當祔廟,居朕室次,禮官顧謂今日未宜,徒飾說以惑眾聽。」因諭嚴嵩等曰:「禮官從朕言,勉強耳。即不忍祧仁宗,且置后主別廟,將來由臣下議處。忌日令奠一卮酒,不至傷情。」於是禮臣不敢復言,第請如敕行。乃許之。後二年,楊思忠為賀表觸忌,予杖削籍。隆慶初,與孝潔皇后同日上尊謚曰孝烈端順敏惠恭誠祗天衛聖皇后,移主弘孝殿。

孝恪杜太后,穆宗生母也,大興人。嘉靖十年封康嬪。十五年進封妃。三十三年正月薨。是時,穆宗以裕王居邸,禮部尚書歐陽德奏喪儀,請輟朝五日,裕王主喪事,遵高皇帝《孝慈錄》,斬衰三年。帝謂當避君父之尊。大學士嚴嵩言:「高帝命周王橚為孫貴妃服慈母服,斬衰三年。是年,《孝慈錄》成,遂為定制,自後久無是事。及茲當作則垂訓於後。」帝命比賢妃鄭氏故事:輟朝二日。賜謚榮淑,葬金山。穆宗立,上謚曰孝恪淵純慈懿恭順贊天開聖皇太后,遷葬永陵,祀主神霄殿。追封后父林為慶都伯,命其子繼宗嗣。

穆宗孝懿皇后李氏,昌平人。穆宗為裕王,選為妃,生憲懷太子。嘉靖三十七年四月薨。帝以部疏稱薨非制,命改稱故,葬金山。穆宗即位,謚曰孝懿皇后,封后父銘德平伯。神宗即位,上尊謚曰孝懿貞惠順哲恭仁儷天襄聖莊皇后,合葬昭陵,祔太廟。

孝安皇后陳氏,通州人。嘉靖三十七年九月選為裕王繼妃。隆慶元年冊為皇后。后無子多病,居別宮。神宗即位,上尊號曰仁聖皇太后,六年加上貞懿,十年加康靜。初,神宗在東宮,每晨謁奉先殿、朝帝及生母畢,必之后所問安,后聞履聲輒喜。既嗣位,孝事兩宮無間。二十四年七月崩,謚曰孝安貞懿恭純溫惠佐天弘聖皇后,祀奉先殿別室。

孝定李太后,神宗生母也,漷縣人。侍穆宗於裕邸。隆慶元年三月封貴妃。生神宗。即位,上尊號曰慈聖皇太后。舊制:天子立,尊皇后為皇太后,若有生母稱太后者,則加徽號以別之。是時,太監馮保欲媚貴妃,因以並尊風大學士張居正下廷臣議,尊皇后曰仁聖皇太后,貴妃曰慈聖皇太后,始無別矣。仁聖居慈慶宮,慈聖居慈寧宮。居正請太后視帝起居,乃徙居乾清宮。

太后教帝頗嚴。帝或不讀書,即召使長跪。每御講筵入,嘗令效講臣進講於前。遇朝期,五更至帝寢所,呼曰「帝起」,敕左右掖帝坐,取水為盥面,挈之登輦以出。帝事太后惟謹,而諸內臣奉太后旨者,往往挾持太過。帝嘗在西城曲宴被酒,令內侍歌新聲,辭不能,取劍擊之。左右勸解,乃戲割其髮。翼日,太后聞,傳語居正具疏切諫,令為帝草罪己御札。又召帝長跪,數其過。帝涕泣請改乃已。六年,帝大婚,太后將返慈寧宮,敕居正曰:「吾不能視皇帝朝夕,先生親受先帝付託,其朝夕納誨,終先帝憑几之誼。」三月加尊號曰宣文。十年加明肅。十二年同仁聖太后謁山陵。二十九年加貞壽端獻。三十四年加恭熹。四十二年二月崩,上尊謚曰孝定貞純欽仁端肅弼天祚聖皇太后,合葬昭陵,別祀崇先殿。

后性嚴明。萬曆初政,委任張居正,綜核名實,幾於富強,后之力居多。光宗之未冊立也,給事中姜應麟等疏請被謫,太后聞之弗善。一日,帝入侍,太后問故。帝曰:「彼都人子也。」太后大怒曰:「爾亦都人子!」帝惶恐,伏地不敢起。蓋內廷呼宮人曰「都人」,太后亦由宮人進,故云。光宗由是得立。群臣請福王之籓,行有日矣,鄭貴妃欲遲之明年,以祝太后誕為解。太后曰:「吾潞王亦可來上壽乎!」貴妃乃不敢留福王。御史曹學程以建言論死。太后憐其母老,言於帝,釋之。后父偉封武清伯。家人嘗有過,命中使出數之,而抵其家人於法。顧好佛,京師內外多置梵剎,動費鉅萬,帝亦助施無算。居正在日,嘗以為言,未能用也。

神宗孝端皇后王氏,餘姚人,生京師。萬曆六年冊立為皇后。性端謹,事孝定太后得其懽心。光宗在東宮,危疑者數矣,調護備至。鄭貴妃顓寵,后不較也。正位中宮者四十二年,以慈孝稱。四十八年四月崩,謚孝端。光宗即位,上尊謚曰孝端貞恪莊惠仁明媲天毓聖顯皇后。會帝崩,熹宗立,始上冊寶,合葬定陵,主祔廟。

與后同日冊封者有昭妃劉氏。天啟、崇禎時,嘗居慈寧宮,掌太后璽。性謹厚,撫愛諸王。莊烈帝禮事之如大母。嘗以歲朝朝見,帝就便坐,俄假寐。太后戒勿驚,命尚衣謹護之。頃之,帝覺,攝衣冠起謝曰:「神祖時海內少事;今苦多難,兩夜省文書,未嘗交睫,在太妃前,困不自持如此。」太妃為之泣下。崇禎十五年薨,年八十有六。

孝靖王太后,光宗生母也。初為慈寧宮宮人。年長矣,帝過慈寧,私幸之,有身。故事:宮中承寵,必有賞賚,文書房內侍記年月及所賜以為驗。時帝諱之,故左右無言者。一日,侍慈聖宴,語及之。帝不應。慈聖命取內起居注示帝,且好語曰:「吾老矣,猶未有孫。果男者,宗社福也。母以子貴,寧分差等耶?」十年四月封恭妃。八月,光宗生,是為皇長子。既而鄭貴妃生皇三子,進封皇貴妃,而恭妃不進封。二十九年冊立皇長子為皇太子,仍不封如故。三十四年,元孫生,加慈聖徽號,始進封皇貴妃。三十九年病革,光宗請旨得往省,宮門猶閉,抉鑰而入。妃目眚,手光宗衣而泣曰:「兒長大如此,我死何恨!」遂薨。大學士葉向高言:「皇太子母妃薨,禮宜從厚。」不報。復請,乃得允。謚溫肅端靖純懿皇貴妃,葬天壽山。

光宗即位,下詔曰:「朕嗣承基緒,撫臨萬方,溯厥慶源,則我生母溫肅端靖純懿皇貴妃恩莫大焉。朕昔在青宮,莫親溫凊,今居禁闥,徒痛桮棬,欲伸罔極之深悰,惟有肇稱乎殷禮。其準皇祖穆宗皇帝尊生母榮淑康妃故事,禮部詳議以聞。」會崩,熹宗即位,上尊謚曰孝靖溫懿敬讓貞慈參天胤聖皇太后,遷葬定陵,祀奉慈殿。后父天瑞,封永寧伯。

恭恪貴妃鄭氏,大興人。萬曆初入宮,封貴妃,生皇三子,進皇貴妃。帝寵之。外廷疑妃有立己子謀。群臣爭言立儲事,章奏累數千百,皆指斥宮闈,攻擊執政。帝概置不問。由是門戶之禍大起。萬曆二十九年春,皇長子移迎禧宮,十月立為皇太子,而疑者仍未已。

先是,侍郎呂坤為按察使時,嘗集《閨範圖說》。太監陳矩見之,持以進帝。帝賜妃,妃重刻之,坤無與也。二十六年秋,或撰《閨範圖說跋》,名曰《憂危竑議》,匿其名,盛傳京師,謂坤書首載漢明德馬后由宮人進位中宮,意以指妃,而妃之刊刻,實藉此為立己子之據。其文託「朱東吉」為問答。「東吉」者,東朝也。其名《憂危》,以坤曾有《憂危》一疏,因借其名以諷,蓋言妖也。妃兄國泰、侄承恩以給事中戴士衡嘗糾坤,全椒知縣樊玉衡並糾貴妃,疑出自二人手。帝重謫二人,而置妖言不問。踰五年,《續憂危竑議》復出。是時太子已立,大學士朱賡得是書以聞。書託「鄭福成」為問答。「鄭福成」者,謂鄭之福王當成也。大略言:「帝於東宮不得已而立,他日必易。其特用朱賡內閣者,實寓更易之義。」詞尤詭妄,時皆謂之妖書。帝大怒,敕錦衣衛搜捕甚急。久之,乃得皦生光者,坐極刑,語詳郭正域、沈鯉傳。

四十一年,百戶王曰乾又告變,言奸人孔學等為巫蠱,將不利於聖母及太子,語亦及妃。賴大學士葉向高勸帝以靜處之,而速福王之籓,以息群言。事乃寢。其後「梃擊」事起,主事王之寀疏言張差獄情,詞連貴妃宮內侍龐保、劉成等,朝議水匈水匈。貴妃聞之,對帝泣。帝曰:「外廷語不易解,若須自求太子。」貴妃向太子號訴。貴妃拜,太子亦拜。帝又於慈寧宮太后几筵前召見群臣,令太子降諭禁株連,於是張差獄乃定。神宗崩,遺命封妃皇后。禮部侍郎孫如游爭之,乃止。及光宗崩,有言妃與李選侍同居乾清宮謀垂簾聽政者,久之始息。

崇禎三年七月薨,謚恭恪惠榮和靖皇貴妃,葬銀泉山。

光宗孝元皇后郭氏,順天人。父維城以女貴,封博平伯,進侯。卒,兄振明嗣。后於萬曆二十九年冊為皇太子妃。四十一年十一月薨,謚恭靖。熹宗即位,上尊謚曰孝元昭懿哲惠莊仁合天弼聖貞皇后,遷葬慶陵,祔廟。

孝和王太后,熹宗生母也,順天人。侍光宗東宮,為選侍。萬曆三十二年進才人。四十七年三月薨。熹宗即位,上尊謚曰孝和恭獻溫穆徽慈諧天鞠聖皇太后,遷葬慶陵,祀奉先殿。崇禎十一年三月以加上孝純太后尊謚,於御用監得后及孝靖太后玉冊玉寶,始命有司獻於廟。忠賢黨王體乾坐怠玩,論死。蓋距上謚時十有八年矣。

孝純劉太后,莊烈帝生母也,海州人,後籍宛平。初入宮為淑女。萬曆三十八年十二月生莊烈皇帝。已,失光宗意,被譴,薨。光宗中悔,恐神宗知之,戒掖庭勿言,葬於西山。及莊烈帝長,封信王,追進賢妃。時莊烈帝居勖勤宮,問近侍曰:「西山有申懿王墳乎?」曰:「有。」「傍有劉娘娘墳乎?」曰:「有。」每密付金錢往祭。及即位,上尊謚曰孝純恭懿淑穆莊靜毘天毓聖皇太后,遷葬慶陵。

帝五歲失太后,問左右遺像,莫能得。傅懿妃者,舊與太后同為淑女,比宮居,自稱習太后,言宮人中狀貌相類者,命后母瀛國太夫人指示畫工,可意得也。圖成,由正陽門具法駕迎入。帝跪迎於午門,懸之宮中,呼老宮婢視之,或曰似,或曰否。帝雨泣,六宮皆泣。

故事:生母忌日不設祭,不服青。十五年六月,帝以太后故,欲追前代生繼七后,同建一廟,以展孝思。乃御德政殿,召大學士及禮臣入,問曰:「太廟之制,一帝一后,祧廟亦然,歷朝繼后及生母凡七位皆不得與,即宮中奉先殿亦尚無祭,奈何?」禮部侍郎蔣德璟曰:「奉先殿外尚有奉慈殿,所以奉繼后及生母者,雖廢可舉也。」帝曰:「奉慈殿外,尚有弘孝、神霄、本恩諸殿。」德璟曰:「內廷規制,臣等未悉。孝宗建奉慈殿,嘉靖間廢之,今未知尚有舊基否?」帝曰:「奉慈已撤,惟奉先尚可拓也。」於是別置一殿,祀孝純及七后云。

康妃李氏,光宗選侍也。時宮中有二李選侍,人稱東、西李。康妃者,西李也,最有寵,嘗撫視熹宗及莊烈帝。光宗即位,不豫,召大臣入,帝御暖閣,憑几,命封選侍為皇貴妃。選侍趣熹宗出曰:「欲封后。」帝不應。禮部侍郎孫如游奏曰:「今兩太后及元妃、才人謚號俱未上,俟四大禮舉行後未晚。」既而帝崩,選侍尚居乾清宮,外廷恟懼,疑選侍欲聽政。大學士劉一燝、吏部尚書周嘉謨、兵科都給事中楊漣、御史左光斗等上疏力爭,選侍移居仁壽殿。事詳一燝、漣傳。

熹宗即位,降敕暴選侍凌毆聖母因致崩逝及妄覬垂簾狀。而御史賈繼春進安選侍揭,與周朝瑞爭駁不已。帝復降敕曰:「九月一日,皇考賓天,大臣入宮哭臨畢,因請朝見。選侍阻朕煖閣,司禮監官固請,乃得出。既許復悔,又使李進忠等再三趣回。及朕至乾清丹陛,進忠等猶牽朕衣不釋。甫至前宮門,又數數遣人令朕還,毋御文華殿也。此諸臣所目睹。察選侍行事,明欲要挾朕躬,垂簾聽政。朕蒙皇考令選侍撫視,飲膳衣服皆皇祖、皇考賜也。選侍侮慢凌虐,朕晝夜涕泣。皇考自知其誤,時加勸慰。若避宮不早,則爪牙成列,朕且不知若何矣。選侍因毆崩聖母,自忖有罪,每使宮人竊伺,不令朕與聖母舊侍言,有輒捕去。朕之苦衷,外廷豈能盡悉。乃諸臣不念聖母,惟黨選侍,妄生謗議,輕重失倫,理法焉在!朕今停選侍封號,以慰聖母在天之靈;厚養選侍及皇八妹,以敬遵皇考之意。爾諸臣可以仰體朕心矣。」已,復屢旨詰責繼春,繼春遂削籍去。

是時,熹宗初即位,委任司禮太監王安,故敕諭如此。久之,魏忠賢亂政。四年封選侍為康妃。五年修《三朝要典》,漣、光斗等皆得罪死,復召繼春,與前旨大異矣。久之,始卒。莊妃李氏,即所稱東李者也。仁慈寡言笑,位居西李前,而寵不及。莊烈帝幼失母,育於西李。既而西李生女,光宗改命東李撫視。天啟元年二月封莊妃。魏忠賢、客氏用事,惡妃持正,宮中禮數多被裁損,憤鬱薨。崇禎初,詔賜妃弟成楝田產。

選侍趙氏者,光宗時,未有封號。熹宗即位,忠賢、客氏惡之,矯旨賜自盡。選侍以光宗賜物列案上,西向禮佛,痛哭自經死。

熹宗懿安皇后張氏,祥符人。父國紀,以女貴,封太康伯。天啟元年四月冊為皇后。性嚴正,數於帝前言客氏、魏忠賢過失。嘗召客氏至,欲繩以法。客、魏交恨,遂誣后非國紀女,幾惑帝聽。三年,后有娠,客、魏盡逐宮人異己者,而以其私人承奉,竟損元子。帝嘗至后宮,后方讀書。帝問何書。對曰:「《趙高傳》也。」帝默然。時宮門有匿名書烈忠賢逆狀者,忠賢疑出國紀及被逐諸臣手。其黨邵輔忠、孫傑等,欲因此興大獄,盡殺東林諸臣,而借國紀以搖動中宮,冀事成則立魏良卿女為后。順天府丞劉志選偵知之,首上疏劾國紀,御史梁夢環繼之,會有沮者乃已。及熹宗大漸,折忠賢逆謀、傳位信王者,后力也。莊烈帝上尊號曰懿安皇后。十七年三月,李自成陷都城,后自縊。順治元年,世祖章皇帝命合葬熹宗陵。

裕妃張氏,熹宗妃也。性直烈。客、魏恚其異己,幽於別宮,絕其飲食。天雨,妃匍匐飲簷溜而死。又慧妃范氏者,生悼懷太子不育,復失寵。李成妃侍寢,密為慧妃乞憐。客、魏知之怒,亦幽成妃於別宮。妃預藏食物簷瓦間,閉宮中半月不死,斥為宮人。崇禎初,皆復位號。

莊烈帝愍皇后周氏,其先蘇州人,徙居大興。天啟中,選入信邸。時神宗劉昭妃攝太后寶,宮中之政悉稟成於熹宗張皇后。故事:宮中選大婚,一后以二貴人陪;中選,則皇太后幕以青紗帕,取金玉跳脫繫其臂;不中,即以年月帖子納淑女袖,侑以銀幣遣還。懿安疑后弱,昭妃曰:「今雖弱,後必長大。」因冊為信王妃。帝即位,立為皇后。

后性嚴慎。嘗以寇急,微言曰:「吾南中尚有一家居。」帝問之,遂不語,蓋意在南遷也。至他政事,則未嘗預。田貴妃有寵而驕,后裁之以禮。歲元日,寒甚,田妃來朝,翟車止廡下。後良久方御坐,受其拜,拜已遽下,無他言。而袁貴妃之朝也,相見甚歡,語移時。田妃聞而大恨,向帝泣。帝嘗在交泰殿與后語不合,推后仆地,后憤不食。帝悔,使中使持貂裀賜后,且問起居。妃尋以過斥居啟祥宮,三月不召。一日,后侍帝於永和門看花,請召妃。帝不應。后遽令以車迎之,乃相見如初。帝以寇亂茹蔬。后見帝容體日瘁,具饌將進,而瀛國夫人奏適至,曰:「夜夢孝純太后歸,語帝瘁而泣,且曰:『為我語帝,食毋過苦。』」帝持奏入宮,后適進饌。帝追念孝純,且感后意,因出奏示后,再拜舉匕箸,相向而泣,淚盈盈沾案。

崇禎十七年三月十八日暝,都城陷,帝泣語后曰:「大事去矣。」后頓首曰:「妾事陛下十有八年,卒不聽一語,至有今日。」乃撫太子、二王慟哭,遣之出宮。帝令后自裁。后入室闔戶,宮人出奏,猶云「皇后領旨」。后遂先帝崩。帝又命袁貴妃自縊,繫絕,久之蘇。帝拔劍斫其肩,又斫所御妃嬪數人,袁妃卒不殊。世祖章皇帝定鼎,謚后曰莊烈愍皇后,與帝同葬田貴妃寢園,名曰思陵。下所司給袁妃居宅,贍養終其身。

有宮人魏氏者,當賊入宮,大呼曰:「我輩必遭賊污,有志者早為計。」遂躍入御河死,頃間從死者一二百人。宮人費氏,年十六,自投眢井中。賊鉤出,見其姿容,爭奪之。費氏紿曰:「我長公主也。」群賊不敢逼,擁見李自成。自成命中官審視之,非是,以賞部校羅某者。費氏復紿羅曰:「我實天潢,義難茍合,將軍宜擇吉成禮。」羅喜,置酒極歡。費氏懷利刃,俟羅醉,斷其喉立死。因自詫曰:「我一弱女子,殺一賊帥足矣。」遂自刎死。自成聞大驚,令收葬之。

恭淑貴妃田氏,陜西人,後家揚州。父弘遇以女貴,官左都督,好佚遊,為輕俠。妃生而纖妍,性寡言,多才藝,侍莊烈帝於信邸。崇禎元年封禮妃,進皇貴妃。宮中有夾道,暑月駕行幸,御蓋行日中。妃命作蘧篨覆之,從者皆得休息。又易小黃門之舁輿者以宮婢。帝聞,以為知禮。嘗有過,謫別宮省愆。所生皇五子,薨於別宮,妃遂病。十五年七月薨。謚恭淑端惠靜懷皇貴妃,葬昌平天壽山,即思陵也。

贊曰:高皇后從太祖備歷艱難,贊成大業,母儀天下,慈德昭彰。繼以文皇后仁孝寬和,化行宮壼,後世承其遺範,內治肅雍。論者稱有明家法,遠過漢、唐,信不誣矣。萬、鄭兩貴妃,亦非有陰鷙之謀、干政奪嫡之事,徒以恃寵溺愛,遂滋謗訕。《易》曰:「閑有家,悔亡。」茍越其閑,悔將無及。聖人之垂戒遠矣哉。

○興宗孝康皇帝孝康皇后呂太后睿宗獻皇帝獻皇后

興宗孝康皇帝標,太祖長子也。母高皇后。元至正十五年生於太平陳迪家。太祖為吳王,立為王世子,從宋濂受經。

吳元年,年十三矣,命省臨濠墓,諭曰:「商高宗舊勞於外,周成王早聞《無逸》之訓,皆知小民疾苦,故在位勤儉,為守成令主。兒生長富貴,習於晏安。今出旁近郡縣,遊覽山川,經歷田野,其因道途險易以知鞍馬勤勞,觀閭閻生業以知衣食艱難,察民情好惡以知風俗美惡,即祖宗所居,訪求父老,問吾起兵渡江時事,識之於心,以知吾創業不易。」又命中書省擇官輔行。凡所過郡邑城隍山川之神,皆祭以少牢。過太平訪迪家,賜白金五十兩。至泗、濠告祭諸祖墓。是冬從太祖觀郊壇,令左右導之農家,遍觀服食器具,又指道旁荊楚曰:「古用此為撲刑,以其能去風,雖傷不殺人。古人用心仁厚如此,兒念之。」洪武元年正月,立為皇太子。帶刀舍人周宗上書乞教太子。帝嘉納。中書省都督府請仿元制,以太子為中書令。帝以元制不足法,令詹同考歷代東宮官制,選勛德老成及新進賢者,兼領東宮官。於是左丞相李善長兼太子少師,右丞相徐達兼太子少傅,中書平章錄軍國重事常遇春兼太子少保,右都督馮宗異兼右詹事,中書平章政事胡廷端、廖永忠、李伯昇兼同知詹事院事,中書左、右丞趙庸、王溥兼副詹事,中書參政楊憲兼詹事丞,傅瓛兼詹事,同知大都督康茂才、張興祖兼左右率府使,大都督府副使顧時、孫興祖同知左右率府事,僉大都督府事吳楨、耿炳文兼左右率府副使,御史大夫鄧愈、湯和兼諭德,御史中丞劉基、章溢兼贊善大夫,治書侍御史文原吉、范顯祖兼太子賓客。諭之曰:「朕於東宮不別設府僚,而以卿等兼領者,蓋軍旅未息,朕若有事於外,必太子監國。若設府僚,卿等在內,事當啟聞,太子或聽斷不明,與卿等意見不合,卿等必謂府僚導之,嫌隙易生。又所以特置賓客諭德等官者,欲輔成太子德性,且選名儒為之,職此故也。昔周公教成王克詰戎兵,召公教康王張皇六師,此居安慮危,不忘武備。蓋繼世之君,生長富貴,暱於安逸,不諳軍旅,一有緩急,罔知所措。二公之言,其並識之。」是年,命選國子生國琦、王璞、張傑等十餘人,侍太子讀書禁中。琦等入對謹身殿,儀狀明秀,應對詳雅。帝喜,因謂殿中侍御史郭淵友等曰:「諸生於文藝習矣,然與太子處,當端其心術,不流浮靡,庶儲德亦獲裨助。」因厚賜之。未幾,以梁貞、王儀為太子賓客,秦庸、盧德明、張昌為太子諭德。

先是,建大本堂,取古今圖籍充其中,征四方名儒教太子諸王,分番夜直,選才俊之士充伴讀。帝時時賜宴賦詩,商搉古今,評論文字無虛日。命諸儒作《鐘山龍蟠賦》。置酒歡甚,自作《時雪賦》,賜東宮官。令三師、諭德朝賀東宮,東宮答拜。又命東宮及王府官編緝古人行事可為鑒戒者,訓諭太子諸王。四年春,製《大本堂玉圖記》,賜太子。

十年,令自今政事並啟太子處分,然後奏聞。諭曰:「自古創業之君,歷涉勤勞,達人情,周物理,故處事咸當。守成之君,生長富貴,若非平昔練達,少有不謬者。故吾特命爾日臨群臣,聽斷諸司啟事,以練習國政。惟仁不失於疏暴,惟明不惑於邪佞,惟勤不溺於安逸,惟斷不牽於文法。凡此皆心為權度。吾自有天下以來,未嘗暇逸,於諸事務惟恐毫髮失當,以負上天付託之意。戴星而朝,夜分而寢,爾所親見。爾能體而行之,天下之福也。」時令儒臣為太子講《大學衍義》。二十二年,置詹事院。

二十四年八月,敕太子巡撫陜西。先是,帝以應天、開封為南北京,臨濠為中都。御史胡子祺上書曰:「天下形勝地可都者四。河東地勢高,控制西北,堯嘗都之,然其地苦寒。汴梁襟帶河、淮,宋嘗都之,然其地平曠,無險可憑。洛陽周公卜之,周、漢遷之,然嵩、邙非有殽函、終南之阻,澗、瀍、伊、洛非有涇、渭、水霸、滻之雄。夫據百二河山之勝,可以聳諸侯之望,舉天下莫關中若也。」帝稱善。至是,諭太子曰:「天下山川惟秦地號為險固,汝往以省觀風俗,慰勞秦父老子弟。」於是擇文武諸臣扈太子行。既行,使諭曰:「爾昨渡江,震雷忽起於東南,導爾前行,是威震之兆也。然一旬久陰不雨,占有陰謀,宜慎舉動,嚴宿衛,施仁布惠,以回天意。」仍申諭從行諸臣以宿頓聞。

比還,獻陜西地圖,遂病。病中上言經略建都事。明年四月丙子薨,帝慟哭。禮官議期喪,請以日易。及當除服,帝不忍。禮官請之,始釋服視朝。八月庚申祔葬孝陵東,謚曰懿文。

太子為人友愛。秦、周諸王數有過,輒調護之,得返國。有告晉王異謀者,太子為涕泣請,帝乃感悟。帝初撫兄子文正、姊子李文忠及沐英等為子,高后視如己出。帝或以事督過之,太子輒告高后為慰解,其仁慈天性然也。太子元妃常氏,繼妃呂氏。生子五:長雄英,次建文皇帝,次允熥,次允熞,次允熙。建文元年追尊為孝康皇帝,廟號興宗。燕王即帝位,復稱懿文皇太子。孝康皇后常氏,開平王遇春女。洪武四年四月冊為皇太子妃。十一年十一月薨,謚敬懿。太祖為輟朝三日。建文元年追尊為孝康皇后。永樂元年復稱敬懿皇太子妃。

皇太后呂氏,壽州人。父本,累官太常卿。惠帝即位,尊為皇太后。燕兵至金川門,迓太后至軍中,述不得已起兵之故。太后還,未至,宮中已火。既而隨其子允熙居懿文陵。永樂元年復稱皇嫂懿文太子妃。

初,太祖冊常妃,繼冊呂妃。常氏薨,呂氏始獨居東宮。而其時秦王樉亦納王保保妹為妃,又以鄧愈女為配,皆前代故事所無也。

睿宗興獻皇帝祐杬,憲宗第四子。母邵貴妃。成化二十三年封興王。弘治四年建邸德安。已,改安陸。七年之籓,舟次龍江,有慈烏數萬繞舟,至黃州復然,人以為瑞。謝疏陳五事。孝宗嘉之,賜予異諸弟。

王嗜詩書,絕珍玩,不畜女樂,非公宴不設牲醴。楚俗尚巫覡而輕醫藥,乃選布良方,設藥餌以濟病者。長史張景明獻所著《六益》於王,賜之金帛,曰:「吾以此懸宮門矣。」邸旁有臺曰陽春,數與群臣賓從登臨賦詩。正德十四年薨,謚曰獻。

王薨二年而武宗崩,召王世子入嗣大統,是為世宗。禮臣毛澄等援漢定陶、宋濮王故事,考孝宗,改稱王為「皇叔父興獻大王」,王妃為「皇叔母」。帝命廷臣集議,未決。進士張璁上書請考興獻王,帝大悅。會母妃至自安陸,止通州不入。帝啟張太后,欲避天子位,奉母妃歸籓。群臣惶懼。太后命進王為興獻帝,妃為興獻后。璁更為《大禮或問》以進,而主事霍韜、桂萼,給事中熊浹議與璁合。帝因諭輔臣楊廷和、蔣冕、毛紀,帝、后加稱「皇」。廷和等合廷臣爭之,未決。嘉靖元年,禁中火,廷和及給事中鄧繼曾、朱鳴陽引五行五事,為廢禮之證。乃輟稱「皇」,加稱本生父興獻帝,尊園曰陵,黃屋監衛如制,設祠署安陸,歲時享祀,用十二籩豆,樂用八佾。帝心終未慊。三年加稱為本生皇考恭穆獻皇帝,興國太后為本生聖母章聖皇太后,建廟奉先殿西,曰觀德殿,祭如太廟。七月,諭去本生號。九月,詔稱孝宗皇伯考,稱獻皇帝曰皇考。

璁、萼等既驟貴,干進者爭以言禮希上意。百戶隨全、錄事錢子勳言獻皇帝宜遷葬天壽山。禮部尚書席書議:「高皇帝不遷祖陵,太宗不遷孝陵,蓋其慎也。小臣妄議山陵,宜罪。」工部尚書趙璜亦言不可。乃止。尊陵名曰顯陵。

明年,修《獻皇帝實錄》,建世廟於太廟左。六年,以觀德殿狹隘,改建崇先殿。七年,命璁等集《明倫大典》,成,加上尊謚曰恭睿淵仁寬穆純聖獻皇帝。親製《顯陵碑》,封松林山為純德山,從祀方澤,次五鎮,改安陸州為承天府。

十七年,通州同知豐坊請加尊皇考廟號,稱宗以配上帝。九月,加上尊謚知天守道洪德淵仁寬穆純聖恭儉敬文獻皇帝,廟號睿宗,祔太廟,位次武宗上。明堂大享,奉主配天,罷世廟之祭。四十四年,芝生世廟柱,復作玉芝官祀焉。穆宗立,乃罷明堂配享。

初,楊廷和等議封益王次子崇仁王厚炫為興王,奉獻帝祀。不允。興國封除。獻帝有長子厚熙,生五日而殤。嘉靖四年贈岳王,謚曰懷獻。

皇后蔣氏,世宗母也。父斅,大興人,追封玉田伯。弘治五年冊為興王妃。世宗入承大統,即位三日,遣使詣安陸奉迎,而令廷臣議推尊禮。咸謂宜考孝宗,而稱興王為皇叔父,妃為皇叔母。議三上,未決。會妃將至,禮臣上入宮儀,由崇文門入東安門,皇帝迎於東華門。不許。再議由正陽門入大明、承天、端門,從王門入宮。又不許。王門,諸王所出入門也。敕曰:「聖母至,御太后車服,從御道入,朝太廟。」故事:后妃無謁廟禮。禮臣難之。時妃至通州,聞考孝宗,恚曰:「安得以吾子為他人子!」留不進。帝涕泣願避位。群臣以慈壽太后命,改稱興獻后,乃入。以太后儀謁奉先、奉慈二殿,不廟見。元年改稱興國太后。三年乃上尊號曰本生章聖皇太后。是年秋,用張璁等言,尊為聖母章聖皇太后。五年,獻帝世廟成,奉太后入謁。七年,上尊號曰慈仁。九年,頒太后所製《女訓》於天下。十五年,奉太后謁天壽山陵,命諸臣進賀行殿。是年加上尊號曰康靜貞壽。

十七年十二月崩,諭禮、工二部將改葬獻皇帝於大峪山,以駙馬都尉京山侯崔元為奉迎行禮使,兵部尚書張瓚為禮儀護行使,指揮趙俊為吉凶儀仗官,翊國公郭勛知聖母山陵事。已,帝親幸大峪相視,令議奉太后南詣合葬。而禮部尚書嚴嵩等言:「靈駕北來,慈宮南詣,共一舉耳。大峪可朝發夕至,顯陵遠在承天,恐陛下春秋念之。臣謂如初議便。」帝曰:「成祖豈不思皇祖耶,何以南孝陵?」因止崔元等毋行,而令趙俊往,且啟視幽宮。是年上尊謚曰慈孝貞順仁敬誠一安天誕聖獻皇后。明年,俊歸,謂顯陵不吉,遂議南巡。九卿大臣許讚等諫。不聽。左都御史王廷相又諫。帝曰:「朕豈空行哉,為吾母耳!」已而侍御呂柟、給事中曾烶、御史劉賢、郎中岳倫等復相繼疏諫。不聽。三月,帝至承天,謁顯陵,作新宮焉,曰:「待合葬也。」歸過慶都,御史謝少南言:「慶都有堯母墓,佚於祀典,請祀之。」帝曰:「帝堯父母異陵,可知合葬非古。」即拜少南左春坊左司直兼翰林院檢討,定議葬大峪山。四月,帝謁長陵,諭嚴嵩曰:「大峪不如純德。」仍命崔元護梓宮南祔。閏七月,合葬顯陵,主祔睿宗廟。

贊曰:興宗、睿宗雖未嘗身為天子,而尊號徽稱典禮具備,其實有不容泯者。史者所以記事也,記事必核其名與實。曰宗曰帝者,當時已定之名,名定而實著焉矣。爰據《元史》裕宗、睿宗列傳之例,別為一卷如右,而各以後附焉。

