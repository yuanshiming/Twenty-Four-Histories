\article{列傳第一百二十五}

\begin{pinyinscope}
傅好禮姜志禮包見捷田大益馮應京何棟如王之翰卞孔時吳宗堯吳寶秀華鈺王正志

傅好禮,字伯恭,固安人。萬曆二年進士。知涇縣,治最,入為御史。嘗陳時政,請節游宴,停內操,罷外戚世封,止山陵行幸,又上崇實、杜漸諸疏。語皆剴直。巡按浙江。歲大侵,條上荒政。行部湖州,用便宜發漕折銀萬兩,易粟振饑民。改按山東。泰安州同知張壽朋當貶秩,文選郎謝廷寀用為永平推官,謂州同知六品,而推官七品也。好禮馳疏劾其非制,廷寀坐停俸,壽朋改調。好禮尋謝病歸。召進光祿少卿,改太常。時稅使四出,海內騷然。二十六年冬,奸民張禮等偽為官吏,群小百十人分據近京要地,稅民間雜物,弗予,捶至死。好禮極論其害,因言:「自朝鮮用兵,畿民富者貧,貧者死,思亂已久,奈何又虐征。國家縱貧,亦不當頭會箕斂,括細民續命之脂膏;況奸徒所得千萬,輸朝廷者什一耳,陛下何利為之。」奏入,四日未報,復具疏請。帝大怒,傳旨鐫三級,出之外。大理卿吳定疏救。帝益怒,謫好禮大同廣昌典史,定鐫三級,調邊方。言官復交章論救,斥定為民。既而帝思好禮言,下其疏,命廠衛嚴緝,逮禮等二十八人詔獄,其害乃除。好禮之官,未幾,請急歸。家居十五年卒。天啟中,贈太常卿。

姜志禮,字立之,丹陽人。萬曆十七年進士。歷建昌、衢州推官,入為大理評事。三十三年,以囚多瘐死,疏言:「犴狴之間,一日斃十五人。積日而計,亦何紀極!又況海內小民,罹災寢而轉死溝壑,及為礦稅所羅織、貂璫所攫噬、含冤畢命者,又復何限!乞亟為矜宥,勿久淹繫,且盡除礦稅,毋使宵人竊弄魁柄,賊虐丞黎。」不報。歷刑部員外,出為泉州知府,遷廣東副使,並有聲。進山東右參政,分守登、萊。福王封國河南,詔賜田二百萬畝,跨山東、湖廣境。既之國,遣中貴徐進督山東賦,勢甚張。志禮抗疏曰:「臣所轄二郡,民不聊生,且與倭鄰,不宜有籓府莊田以擾茲土也明甚。且自高皇帝迄今累十餘世,封王子弟多矣,有賜田二萬頃,延連數十郡者乎?繼此而封,尚有瑞、惠、桂三王也。倘比例以請,將予之乎,不予之乎?況國祚靈長,久且未艾。嗣是天家子姓,各援今日故事以請,臣恐方內土田,不足共諸籓分裂也。」帝大怒,貶三秩為廣西僉事。久之,遷江西參議。天啟三年,由浙江副使入為尚寶少卿,尋進卿。河南進玉璽,魏忠賢欲志禮疏獻之。志禮不可。忠賢怒,令私人劾其衰老,遂乞休。詔加太常少卿致仕,已而削奪。崇禎初,復官。志禮性淳樸,所居多政績,亦以行誼稱於鄉。

包見捷,雲南臨安衛人。萬歷十七年進士。改庶吉士,授戶科給事中,屢遷都給事中。奸人李本立請採珠廣東,帝命中官李敬偕往。見捷極言其害,不聽。時小人蜂起言利。千戶李仁請稅湖口商舟,命中官李道往。主簿田應璧請賣兩淮沒官餘鹽,令稅使魯保兼理。見捷等並力爭。頃之,令道、保節制有司。見捷又陳不便者數事。皆不報。益都知縣吳宗堯劾稅使陳增不法,見捷因請盡罷礦稅。無已,先撤增還。未幾,天津稅使王朝死,見捷請勿遣代。忤旨,切責。以馬堂代朝。見捷又劾堂、保及浙江劉忠。帝不納,益遣高寀、暨祿、李鳳榷稅於京口、儀真、廣東,並專敕行事。又以奸人閻大經言,命高淮征稅遼東。見捷等累請停罷,至是言:「遼左神京肩臂,視他鎮尤重。奸徒敢為禍首,陛下不懲以三尺,急罷開採,則遼事必不可為,而國步且隨之矣。」遼東撫按及山海主事吳鐘英相繼爭。皆不納。時中外爭礦稅者無慮百十疏,見捷言尤數,帝心銜之。居數日,又率司官極論,乃謫見捷貴州布政司都事,餘停俸一年。大學士沈一貫、給事中趙完璧等先後論救,完璧等亦坐停俸。見捷尋引疾去。三十四年,起興業知縣。累遷太僕少卿。久之,以右僉都御史巡撫江西。光宗即位,召拜吏部右侍郎。明年卒官。

田大益,字博真,四川定遠人。萬曆十四年進士。授鐘祥知縣。擢兵科給事中,疏論日本封貢可虞。又言:「東征之役,在將士,則當據今日之斬馘以論功;在主帥,則當視後日之成敗以定議。」時韙其言。母喪除,起補戶科。二十八年十月,疏言:「陛下受命日久,驕泰乘之,布列豺狼,殄滅善類,民無所措,靡不蓄怨含憤,覬一旦有事。願陛下惕然警覺,敬天地,嚴祖宗,毋輕臣工,毋戕民命,毋任閹人,毋縱群小,毋務暴刻,毋甘怠荒,急改敗轍,遵治規,用保祖宗無疆之業。」未幾,極陳礦稅六害,言:

內臣務為劫奪,以應上求。礦不必穴,而稅不必商;民間丘隴阡陌,皆礦也,官吏農工,皆入稅之人也。公私騷然,脂膏殫竭。向所謂軍國正供,反致缺損。即令有司威以刀鋸,只足驅民而速之亂耳。此所謂斂巧必蹶也。

陛下嘗以礦稅之役為裕國愛民。然內庫日進不已,未嘗少佐軍國之需。四海之人,方反脣切齒,而冀以計智甘言,掩天下耳目,其可得乎!此所謂名偽必敗也。

財積而不用,祟將隨之。脫巾不已,至於揭竿,適為奸雄睥睨之資。此時雖家給人予,亦且蹴之覆之而不可及矣。此所謂賄聚必散也。

夫眾心不可傷也。今天下上自簪纓,下至耕夫販婦,茹苦含辛、搤諲側目、而無所控訴者,蓋已久矣。一旦土崩勢成,家為仇,人為敵,眾心齊倡,而海內因以大潰。此所謂怨極必亂也。

國家全盛二百三十餘年,已屬陽九,而東征西討以求快意。上之蕩主心,下之耗國脈。二豎固而良醫走,死氣索而大命傾。此所謂禍遲必大也。

陛下矜奮自賢,沈迷不返。以豪璫奸弁為腹心,以金錢珠玉為命脈。藥石之言,褎如充耳。即令逢、干剖心,皋夔進諫,亦安能解其惑哉!此所謂意迷難救也。

此六者,今之大患。臣畏死不言,則負陛下,陛下拒諫不納,則危宗社。願深察而力反之。

皆不報。明年,疏論湖廣稅監陳奉,救僉事馮應京。忤旨,切責。時武昌民以應京被逮,群聚鼓噪,欲殺奉,奉逃匿楚府以免。大益因上言:「陛下驅率狼虎,飛而食人,使天下之人,剝膚而吸髓,重足而累息,以致天災地坼,山崩川竭。釁自上開,憤由怨積,奈何欲塗民耳目,以自解釋,謾曰權宜哉!今楚人以奉故,沈使者不返矣,且欲甘心巡撫大臣矣。中朝使臣不敢入境偵緩急,踰兩月矣。四方觀聽,惟在楚人。臣意陛下必且曠然易慮,立罷礦稅,以靖四方,奈何猶戀戀不能自割也!夫天下至貴,而金玉珠寶至賤也。積金玉珠寶若泰山,不可市天下尺寸地;而失天下,又何用金玉珠寶為哉!今四方萬姓,見陛下遇楚事而無變志,知禍必不解,必且群起為變。此時即盡戮諸璫以謝天下,寧有濟耶?」帝怒,留中。

又明年遷兵科都給事中。時兩京缺尚書三,侍郎十、科道九十四,天下缺巡撫三、布按監司六十六、知府二十五。大益力請簡補,亦不聽。

三十一年,江西稅監潘相請勘合符牒勿經郵傳。巡按御史吳達可駁之,不聽。大益復守故事力爭,竟如相請。內使王朝嘗言,近京採煤歲可獲銀五千,乃率京營兵劫掠西山諸處。煤戶洶洶,朝以沮撓聞。有旨逮治,皆入都城訴失業狀。沈一貫等急請罷朝,且擬敕諭撫按,未得命。大益言:「國家大柄,莫重於兵。朝擅役禁軍,請急誅,為無將之戒。」御史沈正隆、給事中楊應文、白瑜亦疏諫。帝俱不納。俄用中官陳永壽奏,乃召朝還。遼東稅監高淮擁精騎數百至都城。大益言:「祖制,人臣不得弄兵。淮本掃除之役,敢盜兵權,包禍心,罪當誅。」帝亦不問。

明年八月,極陳君德缺失,言:「陛下專志財利,自私藏外,絕不措意。中外群工,因而泄泄。君臣上下,曾無一念及民。空言相蒙,人怨天怒,妖祲變異,罔不畢集。乃至皇陵為發祥之祖而災,孝陵為創業之祖而災,長陵為奠鼎之祖而亦災。天欲蹶我國家,章章明矣。臣觀十餘年來,亂政亟行,不可枚舉,而病源止在貨利一念。今聖諭補缺官矣,釋繫囚矣,然礦稅不撤,而群小猶恣橫,閭閻猶朘削,則百工之展布實難,而罪罟之羅織必眾。缺官雖補,繫囚雖釋,曾何益哉!陛下中歲以來,所以掩聰明之質,而甘蹈貪愚暴亂之行者,止為家計耳。不知家之盈者國必喪。如夏桀隕於瑤臺,商紂焚於寶玉,幽、厲啟戎於榮夷,桓、靈絕統於私鬻,德宗召難於瓊林,道君兆禍於花石。覆轍相仍,昭然可鑒。陛下邇來亂政,不減六代之季。一旦變生,其何以託身於天下哉!」居月餘,復以星變乞固根本,設防禦,罷礦稅。帝皆不省。又明年,以久次添注太常少卿,卒官。

大益性骨鯁,守官無他營。數進危言,卒獲免禍。蓋時帝倦勤,上章者雖千萬言,大率屏置勿閱故也。

馮應京,字可大,盱眙人。萬曆二十年進士。為戶部主事。督薊鎮軍儲,以廉幹聞。尋改兵部,進員外郎。二十八年,擢湖廣僉事,分巡武昌、漢陽、黃州三府。繩貪墨,摧奸豪,風采大著。稅監陳奉恣橫,巡撫支可大以下唯諾惟謹,應京獨以法裁之。奉掊克萬端,至伐塚毀屋,刳孕婦,溺嬰兒。其年十二月,有諸生妻被辱,訴上官。市民從者萬餘,哭聲動地,蜂涌入奉廨,諸司馳救乃免。應京捕治其爪牙,奉怒,陽餉食而置金其中。應京復暴之,益慚恨。明年正月,置酒邀諸司,以甲士千人自衛,遂舉火箭焚民居。民群擁奉門。奉遣人擊之,多死,碎其屍,擲諸途。可大噤不敢出聲,應京獨抗疏列其十大罪。奉亦誣奏應京撓命,陵敕使。帝怒,命貶雜職,調邊方。給事中田大益、御史李以唐等交章劾奉,乞宥應京。帝益怒,除應京名。是時,襄陽通判邸宅、推官何棟如、棗陽縣知縣王之翰亦忤奉被劾。詔宅、之翰為民,棟如遣逮。俄以都給事中楊應文論救,遂並逮應京、宅、之翰三人。頃之,奉又誣劾武昌同知卞孔時抗拒,孔時亦被逮。

緹騎抵武昌,民知應京獲重譴,相率痛哭。奉乃大書應京名,列其罪,榜之通衢。士民益憤,聚數萬人圍奉廨,奉窘,逃匿楚王府,遂執其斥牙六人,投之江,并傷緹騎;詈可大助虐,焚其府門,可大不敢出。奉潛遣參隨三百人,引兵追逐,射殺數人,傷者不可勝計。日已晡,猶紛拏。應京囚服坐檻車,曉以大義,乃稍稍解散。奉匿楚府,逾月不敢出,亟請還京。大學士沈一貫因極言奉罪,請立代還。言官亦爭以為請。帝未許。俄江西稅監李道亦奏奉侵匿狀,乃召還,隸其事於承天守備杜茂。頃之,東廠奏緹騎有死者。帝慍甚,手詔內閣,欲究主謀。一貫言民心宜靜,請亟遣重臣代可大拊循,因以侍郎趙可懷薦。帝乃褫可大官,令可懷馳往。未至,可大已遣兵護奉行。舟車相銜,數里不絕。可懷入境,亦遣使護之。奉得迄邐去。

應京之就逮也,士民擁檻車號哭,車不得行。既去,則家為位祀之。三郡父老相率詣闕訴冤,帝不省。吏科都給事中郭如星、刑科給事中陳維春更連章劾奉。帝怒,謫兩人邊方雜職,繫應京等詔獄,拷訊久之不釋。應京乃於獄中著書,昕夕無倦。三十二年九月,星變修省。廷臣多請釋繫囚,於是應京及宅、棟如獲釋。之翰先瘐死,而孔時繫獄如故。

應京志操卓犖,學求有用,不事空言,為淮西士人之冠。出獄三年卒。天啟初,贈太常少卿,謚恭節。

何棟如,無錫人。居官守正。既為奉所陷,襄陽人赴闕訴冤,不聽。及出獄,削籍歸,家居十七年。天啟初,始起南京兵部主事。會遼陽陷,時議募兵,棟如自請行。遂齎帑金赴浙江,得六千七百人。甫至而廣寧復陷,又自請出關視形勢。乃進太僕少卿,充軍前贊畫。棟如志銳而才疏。初在浙,不能無浮費。所募兵畏出關,多逃亡。及兩疏論熊廷弼、王化貞功罪,給事中蔡思充、朱童蒙,御史陳保泰遂交章劾之。棟如疏辨,因請非時考察京官,用清朋黨。朝貴大恨,遂下詔獄,榜掠備至。五年秋,坐贓戍滁陽。崇禎初,復官。致仕卒。

王之翰,絳州人。官棗陽。力阻開礦,遂被逮拷死。天啟初,贈光祿少卿。

孔時既長繫,廷臣救者數十上。帝皆不省。四十一年,萬壽節,葉向高復以為言,乃削籍放還。熹宗立,起南京刑部員外郎。

吳宗堯,字仁叔,歙縣人。萬曆二十三年進士。授益都知縣。性強項。中官陳增以開礦至,誣奏福山知縣韋國賢阻撓,被逮削籍。守令多屈節如屬吏,宗堯獨具賓主禮。增黨程守訓,宗堯邑子也。宗堯惡其奸,不與通。驛丞金子登說增開孟丘山礦,宗堯叱其欺罔。子登懼,構於增。日征千人鑿山,多捶死;又誣富民盜礦,三日捕繫五百人。二十六年九月,宗堯盡發增不法事。帝得疏意動,持不下。會給事中包見捷極論增罪,請撤還。帝責增,令檢下。見捷同官郝敬復請治增罪,帝乃不悅,責宗堯狂逞要名。已而山東巡撫尹應元劾增背旨虐民二十罪。帝遂發怒,切責應元,削宗堯籍。敬復抗疏諫,帝益怒,奪俸一年,並奪應元俸。增遂劾宗堯阻撓礦務,且令守訓誣訐之。帝既遣逮治,御史劉景辰、給事中侯慶遠爭之,不聽。使者至,民大嘩,欲殺增。宗堯行,民哭聲震地。既至,下詔獄拷訊,繫經年。禮部郎鮑應鰲等言於沈一貫曰:「南康守吳寶秀已得安居牖下,宗堯何獨不然?」一貫揭入,即釋為民,未幾卒。天啟時,贈光祿少卿,賜祭,錄一子。

吳寶秀,字汝珍,平陽人。萬曆十七年進士。授大理評事。歷寺正,出為南康知府。湖口稅監李道橫甚,寶秀不與通。漕舟南還,乘風揚帆入湖口。道欲榷其貨,遣卒急追之,舟覆,有死者。道遣吏捕漕卒,寶秀拒不發。道怒,劾寶秀及星子知縣吳一元、青山巡檢程資阻撓稅務,詔俱逮治。給事中楊應文等請下撫按公勘。大學士沈一貫、吏部尚書李戴、國子祭酒方從哲等交章為言,俱不報。寶秀妻陳氏慟哭,請偕行,寶秀不可。乃括餘貲及簪珥付其妾曰:「夫子行,以為路費。」夜自經死。寶秀至京,下詔獄。大學士趙志皋上言:「頃臣臥病,聞中外人情洶洶,皆為礦稅一事。南康守吳寶秀逮繫時,其妻至投繯自盡,闔郡號呼,幾成變亂。事關民生向背,宗社安危,臣不敢以將去之身,隱默而不言。」星子民陳英者,方廬墓,約儒士熊應鳳等走京師,伏闕訟冤,乞以身代。於是撫按及南北諸臣論救者疏十餘上,帝皆不省。一日,司禮田義匯諸疏進御前,帝怒擲地。義從容拾起,復進之,叩首曰:「閣臣跪候朝門外,不奉處分不敢退。」帝怒稍平,取閱閣臣疏,命移獄刑部。皇太后亦聞陳氏之死,從容為帝言。至九月,與一元等並釋為民。歸家,踰年卒。

初,南康士民建祠,特祀陳氏,後合寶秀祀之。天啟中,贈太僕少卿,賜祭,錄其一子。

華鈺,字德夫,丹徒人。萬曆二十三年進士。授荊州推官。稅監陳奉僕直馳府署中,鈺笞之。奉佯謝,銜之刺骨。奉所受敕止江稅,乃故移之市,又倍蓰征之。稍與辨,輒毆擊破面。商賈怖匿,負擔者不敢出其途。鈺白御史嚴戢,奉益恨。奉欲榷沙市稅,沙市人群起逐之,奉疑鈺所使。已,欲榷黃州團風鎮稅,復為鎮民所逐,奉又疑經歷車任重教之。遂上疏極論鈺、任重阻撓罪,並及巡按御史曹楷、襄陽知府李商耕、黃州知府趙文煥、荊門知州高則巽等數十人。帝切責楷,貶商耕等三人官,鈺、任重皆被逮。時二十七年八月也。既至,下鎮撫獄訊治,俾引御史楷。鈺堅不承,繫獄中。初,吳宗堯、吳寶秀皆不久即釋。帝欲痛折辱以懼之,於是鈺與馮應京、王正志等先後十餘人悉長繫。廷臣論救章數上,皆不報。獄中有鳥,形類鶴而小,怪鳴,則逮者至。一夕,鳥鳴甚哀。鈺起坐俟之,則應京至。居久之,語鈺以主靜窮理之學,日相與研究。三十二年六月,長陵災,肆赦,鈺與任並釋為民。家居四年卒。天啟中,贈尚寶少卿,賜祭,錄一子。

王正志,祥符人。萬曆二十六年進士。除富平知縣。二十八年,稅使梁永、趙欽肆虐,正志捕其黨李英,杖殺之,因極論二人不法罪。欽亦以李英事訐奏,帝怒,命逮之。給事中陳惟春言正志劾欽罪多,宜提訊;欽所劾正志事宜下撫按核實,免其逮繫。御史李時華亦言近日所逮吳應鴻、勞養魁、蔡如川、甘學書及正志等,俱宜敕下撫按勘虛實,不得以一人單詞枉害良善。皆不報。未幾,梁永亦訐正志。帝命諸抗違欺隱者悉指名劾奏,重治之。宦官盆張,長吏皆喪氣。正志繫詔獄四年,三十一年夏,瘐死。天啟時,贈祭,廕子,皆視鈺。

自礦稅興,中使四出,跆藉有司。謗書一聞,駕帖立下。二十四年,則遼東參將梁心;二十五年,則山東福山知縣韋國賢;二十六年,則山東益都知縣吳宗堯;二十七年,則江西南康知府吳寶秀、星子知縣吳一元、山東臨清守備王煬;二十八年,則廣東新會在籍通判吳應鴻,舉人勞養魁、鐘聲朝、梁斗輝,雲南尋甸知府蔡如川,趙州知州甘學書及正志;二十九年,則湖廣按察僉事馮應京、襄陽通判邸宅、推官何棟如、棗陽知縣王之翰、武昌同知卞孔時、江西饒州通判陳奇可;三十年,則鳳陽臨淮知縣林錝;三十四年,則陜西咸陽知縣宋時際;三十五年,則陜西咸寧知縣滿朝薦;三十六年,則遼東海防同知王邦才、參將李獲陽;皆幽系詔獄,久者至十餘年。煬、應鴻、獲陽斃獄中,其他削籍、貶官有差。至士民幽繫死亡者,尤不可勝紀也。

贊曰:神宗二十四年,軍府千戶仲春請開礦助大工,遂命戶部錦衣官各一人同仲春開採。給事中程紹言嘉靖中採礦,費帑金三萬餘,得礦銀二萬八千五百,得不償失,因罷其役。給事中楊應文繼言之。皆不納。由是卑秩冗僚,下至市井黠桀,奮起言利。而璫使四出,毒流海內,民不聊生,至三十三年乃罷。嗣是軍興徵發,加派再三。府庫未充,膏脂已竭,明室之亡,於是決矣。


\end{pinyinscope}