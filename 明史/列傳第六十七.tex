\article{列傳第六十七}

\begin{pinyinscope}
○羅倫塗棐章懋從子拯黃仲昭莊昶鄒智舒芬崔桐馬汝驥

羅倫,字彝正,吉安永豐人。五歲嘗隨母入園,果落,眾競取,倫獨賜而後受。家貧樵牧,挾書誦不輟。及為諸生,志聖賢學,嘗曰:「舉業非能壞人,人自壞之耳。」知府張瑄憫其貧,周之粟,謝不受。居父母喪,踰大祥,始食鹽酪。

成化二年,廷試,對策萬餘言。直斥時弊,名震都下。擢進士第一,授翰林修撰。踰二月,大學士李賢奔喪畢,奉詔還朝。倫詣賢沮之,不聽。乃上疏曰:

臣聞朝廷援楊溥故事,起復大學士李賢。臣竊謂賢大臣,起復大事,綱常風化繫焉,不可不慎。曩陛下制策有曰:「朕夙夜拳拳,欲正大綱,舉萬目,使人倫明於上,風俗厚於下。」竊謂明人倫,厚風俗,莫先於孝。在禮,子有父母之喪,君三年不呼其門。子夏問:「三年之喪,金革無避,禮歟?」孔子曰:「魯公伯禽有為為之也。今以三年之喪從其利者,吾弗知也。」陛下於賢,以為金革之事起復之歟?則未之有也。以大臣起復之歟?則禮所未見也。

夫為人君,當舉先王之禮教其臣;為人臣,當守先王之禮事其君。昔宋仁宗嘗起復富弼矣,弼辭曰:「不敢遵故事以遂前代之非,但當據《禮經》以行今日之是。」仁宗卒從其請。孝宗嘗起復劉珙矣,珙辭曰:「身在草土之中,國無門庭之寇,難冒金革之名,私竊利祿之實。」孝宗不抑其情。此二君者,未嘗以故事強其臣。二臣者。未嘗以故事徇其君。故史冊書之為盛事,士大夫傳之為美談。無他,君能教臣以孝,臣有孝可移於君也。自是而後,無復禮義。王黼、史嵩之、陳宜中、賈似道之徒,皆援故事起復。然天下壞亂,社稷傾危,流禍當時,遺譏後代。無他,君不教臣以孝,臣無孝可移於君也。陛下必欲賢身任天下之事,則賢身不可留,口實可言。宜降溫詔,俾如劉珙得以言事。使賢於天下之事知必言,言必盡。陛下於賢之言聞必行,行必力。賢雖不起復,猶起復也。茍知之而不能盡言,言之而不能力行,賢雖起復無益也。

且陛下無謂廟堂無賢臣,庶官無賢士。君,盂也;臣,水也。水之方圓,盂實主之。臣之直佞,君實召之。陛下誠於退朝之暇,親直諒博洽之臣,講聖學君德之要,詢政事得失,察民生利病,訪人才賢否,考古今盛衰。舍獨信之偏見,納逆耳之苦言。則眾賢群策畢萃於朝,又何待違先王之《禮經》,損大臣之名節,然後天下可治哉。

臣伏見比年以來,朝廷以奪情為常典,縉紳以起復為美名,食稻衣錦之徒,接踵廟堂,不知此人於天下之重何關耶?且婦於舅姑,喪亦三年;孫於祖父母,服則齊衰。奪情於夫,初無預其妻;奪情於父,初無干其子。今或舍館如故,妻孥不還,乃號於天下曰:「本欲終喪,朝命不許」,雖三尺童子,臣知其不信也。為人父者所以望其子之報,豈擬至於此哉。為人子者所以報其親之心,豈忍至於此哉。枉己者不能直人,忘親者不能忠君。陛下何取於若人而起復之也。

今大臣起復,群臣不以為非,且從而贊之;群臣起復,大臣不以為非,且從而成之。上下成俗,混然同流,率天下之人為無父之歸。臣不忍聖明之朝致綱常之壞、風俗之弊一至此極也。願陛下斷自聖衷,許賢歸家持服。其他已起復者,仍令奔喪,未起復者,悉許終制。脫有金革之變,亦從墨衰之權,使任軍事於外,盡心喪於內。將朝廷端則天下一,大臣法則群臣效,人倫由是明,風俗由是厚矣。

疏入,謫福建市舶司副提舉。御史陳選疏救,不報。御史楊瑯復申救,帝切責之。尚書王翱以文彥博救唐介事諷賢,賢曰:「潞公市恩,歸怨朝廷,吾不可以效之。」亡何,賢卒。明年以學士商輅言召復原職,改南京。居二年,引疾歸,遂不復出。

倫為人剛正,嚴於律己。義所在,毅然必為,於富貴名利泊如也。里居倡行鄉約,相率無敢犯。衣食粗惡。或遺之衣,見道殣,解以覆之。晨留客飲,妻子貸粟鄰家,及午方炊,不為意。以金牛山人跡不至,築室著書其中,四方從學者甚眾。十四年卒,年四十八。嘉靖初,從御史唐龍請,追贈左春坊諭德,謚文毅。學者稱一峰先生。

方倫為提舉時,御史豐城涂棐巡按福建。司禮中官黃賜,延平人也,請見,棐不可。泉州知府李宗學以受賕為棐所按,訐棐自解,賜從中主其奏。棐、宗學俱被徵,詞連倫,當并逮。鎮撫司某曰:「羅先生可至此乎?」即日鞫成上之。倫得免,棐亦復官。

塗棐,天順四年進士。成化中嘗言:「祖宗朝,政事必與大臣面議。自先帝幼沖,未能裁決,柄國者慮其缺遺,假簡易之辭,以便宣布。凡視朝奏事,諭旨輒曰:「所司知之」。此一時權宜,非可循為定制。況批答多參以中官,內閣或不與,尤乖祖制。乞復面議,杜蔽壅之弊。」憲宗不能用。終廣東副使。

章懋,字德懋,蘭谿人。成化二年會試第一,成進士,改庶吉王。明年冬,授編修。

憲宗將以元夕張燈,命詞臣撰詩詞進奉。懋與同官黃仲昭、檢討莊昶疏諫曰:「頃諭臣等撰鰲山煙火詩詞,臣等竊議,此必非陛下本懷,或以兩宮聖母在上,欲備極孝養奉其歡心耳。然大孝在乎養志,不可徒陳耳目之玩以為養也。今川東未靖,遼左多虞,江西、湖廣赤地數千里,萬姓嗷嗷,張口待哺,此正陛下宵旰焦勞,兩宮母后同憂天下之日。至翰林官以論思為職,鄙俚之言豈宜進於君上。伏讀宣宗皇帝御製《翰林箴》有曰『啟沃之言,唯義與仁。堯、舜之道,鄒、魯以陳。』張燈豈堯、舜之道,詩詞豈仁義之言?若謂煙火細故不足為聖德累,則舜何必不造漆器,禹何必不嗜旨酒,漢文何必不作露臺?古帝王慎小謹微必矜細行者,正以欲不可縱,漸不可長也。伏乞將煙火停止,移此視聽以明目達聰,省此資財以振饑恤困,則災祲可銷,太平可致。」帝以元夕張燈,祖宗故事,惡懋等妄言,並杖之闕下,左遷其官。修撰羅倫先以言事被黜,時稱「翰林四諫」。

懋既貶臨武知縣,未行,以給事中毛弘等論救,改南京大理左評事。踰三年,遷福建僉事。平泰寧、沙、尤賊,聽福安民採礦以杜盜源,建議番貨互通貿易以裕商民,政績甚著。滿考入都,年止四十一,力求致仕。吏部尚書尹旻固留之,不可。

既歸,屏跡不入城府。奉親之暇,專以讀書講學為事,弟子執經者日益進。貧無供具,惟脫粟菜羹而已。四方學士大夫高其風,稱為「楓山先生」。家居二十餘年,中外交薦,部檄屢起之,以親老堅不赴。

弘治中,孝宗登用群賢。眾議兩京國學當用名儒,起謝鐸於北監。及南監缺祭酒,遂以懋補之。懋方遭父憂不就。時南監缺司業且二十年,詔特以羅欽順為之,而虛位以待懋。十六年,服闋,懋復固辭。不允,始蒞任。六館士人人自以為得師。監生尤樾母病,例不得歸省,晝夜泣。懋遣之歸,曰:「吾寧以違制獲罪。」武宗立,陳勤聖學、隆繼述、謹大婚、重詔令、敬天戒五事。正德元年乞休,五疏不允。復引疾懇辭,明年三月始得請。五年起南京太常卿,明年又起為南京禮部右侍郎,皆力辭不就。言者屢陳懋德望,請加優禮,詔有司歲時存問。世宗嗣位,即家進南京禮部尚書,致仕。其冬,遣行人存問,而懋已卒,年八十六。贈太子少保,謚文懿。

懋為學,恪守先儒訓。或諷為文章,曰:「小技耳,予弗暇。」有勸以著述者,曰:「先儒之言至矣,芟其繁可也。」通籍五十餘年,歷俸僅滿三考。難進易退,世皆高之。

生三子,兼令業農。縣令過之,諸子釋耒跪迎,人不知其貴公子也。子省懋於南監,徒步往,道為巡檢所笞,已知而請罪,懋慰遣之。晚年,三子一孫盡死。年八十二生少子接,後以廕為國子生。

從子拯,字以道。幼從懋學,登弘治十五年進士,為刑部主事。正德初,忤劉瑾,下詔獄,謫梧州府通判。謹誅,擢南京兵部郎中。嘉靖中,累官工部尚書。桂萼欲復海運,延公卿議得失,拯曰:「海運雖有故事,而風濤百倍於河。且天津海口多淤,自古不聞有濬海者。」議遂寢。南北郊議起,拯言不可,失帝意。尋坐郊壇祭器缺供,落職歸。久之復官。致仕,卒。

黃仲昭,名潛,以字行,莆田人。祖壽生,翰林檢討,有學行。父嘉,束鹿知縣,以善政聞。

仲昭性端謹,年十五六即有志正學。登成化二年進士,改庶吉士,授編修。與章懋、莊昶同以直諫被杖,謫湘潭知縣。在道,用諫官言,改南京大理評事。兩京諸司隸卒率放還而取其月錢,為故事,惟仲昭與羅倫不敢。御史縱子弟取賂,刑部曲為地,仲昭駁正之。有群掠民婦轉鬻者,部坐首惡一人,仲昭請皆坐。連遭父母喪,不離苫塊者四年。服除,以親不逮養,遂不出。

弘治改元,御史姜洪疏薦,吏部尚書王恕檄有司敦趣。比至,恕迓之大門外,揖讓升堂,相向再拜,世兩高之。除江西提學僉事,誨士以正學。久之再疏乞休,日事著述。學者稱「未軒先生」。卒年七十四。

仲昭兄深,御史。深子乾亨,行人。使滿剌加,歿於海。乾亨子如金,廣西提學副使,希雍,蘇州同知。仲昭孫懋,南京戶部侍郎。

莊昶,字孔抃,江浦人。自幼豪邁不群,嗜古博學。舉成化二年進士,改庶吉士,授翰林檢討。與編修章懋、黃仲昭疏諫內廷張燈,忤旨廷杖二十,謫桂陽州判官。尋以言官論救,改南京行人司副。居三年,母憂去。繼丁父憂,哀毀,喪除不復出。卜居定山二十餘年,學者稱「定山先生」。巡撫王恕嘗欲葺其廬,辭之。

昶生平不尚著述,有自得,輒見之於詩。薦章十餘上,部檄屢趣,俱不赴。大學士邱濬素惡昶,語人曰:「率天下士背朝廷者,昶也。」弘治七年有薦昶者,奉詔起用。昶念濬當國,不出且得罪,強起入都。大學士徐溥語郎中邵寶曰:「定山故翰林,復之。」濬聞曰:「我不識所謂定山也。」乃復以為行人司副。俄遷南京吏部郎中。得風疾。明年乞身歸,部臣不為奏。又明年京祭,尚書倪岳以老疾罷之。居二年卒,年六十三。天啟初,追謚文節。

鄒智,字汝愚,合州人。年十二能文。家貧,讀書焚木葉繼晷者三年。舉成化二十二年鄉試第一。

時帝益倦於政,而萬安、劉吉、尹直居政府,智憤之。道出三原,謁致仕尚書王恕,慨然曰:「治天下,在進君子退小人。方今小人在位,毒痡四海,而公顧屏棄田里。智此行非為科名,欲上書天子,別白賢奸,拯斯民於塗炭耳。」恕奇其言,笑而不答。明年登進士。改庶吉士。遂上疏曰:

陛下於輔臣,遇事必咨,殊恩異數必及,亦云任矣。然或進退一人,處分一事,往往降中旨,使一二小人陰執其柄,是既任之而又疑之也。陛下豈不欲推誠待物哉?由其進身之初,多出私門,先有以致陛下之厭薄。及與議事,又唯諾惟謹,伈伈伣伣,若有所不敢,反不如一二俗吏足以任事。此陛下所為疑也,臣竊以為過矣。昔宋仁宗知夏竦懷詐則黜之,知呂夷簡能改過則容之;知杜衍、韓琦、范仲淹、富弼可任則不次擢之。故能北拒契丹,西臣元昊。未聞一任一疑,可以成天下事也。願陛下察孰為竦,孰為夷簡,而黜之容之,孰為衍、琦、仲淹、弼而擢之,日與講論治道,不使小人得參其間,則天工亮矣。

臣又聞天下事惟輔臣得議,惟諫官得言。諫官雖卑,與輔臣等。乃今之諫官以軀體魁梧為美,以應對捷給為賢,以簿書刑獄為職業。不畏天變,不恤人窮。或以忠義激之,則曰:「吾非不欲言,言出則禍隨,其誰吾聽?」嗚呼!既不能盡言效職,而復引過以歸於上。有人心者固如是乎?臣願罷黜浮冗,廣求風節之臣。令仗下糾彈,入閣參議。或請對,或輪對,或非時召對,霽色接之,溫言導之,使得畢誠盡蘊,則天聽開矣。

臣又聞汲黯在朝,淮南寢謀,君子之有益人國也大矣。以陛下之聰明,寧不知君子可任而故屈抑之哉?乃小人巧讒間以中傷之耳。今碩德如王恕,忠鯁如強珍,亮直剛方如章懋、林俊、張吉,皆一時人望,不宜貶錮,負上天生才之意。陛下誠召此數人,置要近之地,使各盡其平生,則天心協矣。

臣又聞高皇帝制閽寺,惟給掃除,不及以政。近者舊章日壞,邪徑日開,人主大權盡出其手。內倚之為相,外倚之為將,籓方倚之為鎮撫,伶人賤工倚之以作奇技淫巧,法王佛子倚之以恣出入宮禁,此豈高皇帝所許哉!願陛下以宰相為股肱,以諫官為耳目,以正人君子為腹心,深思極慮,定宗社長久之計,則大綱正矣。

然其本則在陛下明理何如耳。竊聞侍臣進講無反復論辨之功,陛下聽講亦無從容沃心之益。如此而欲明理以應事,臣不信也。願陛下念義理之難窮,惜日月之易邁,考之經史,驗之身心,使終歲無間,則聖學明而萬事畢治,豈特四事之舉措得其當已耶。

疏入,不報。

智既慷慨負奇,其時御史湯鼐、中書舍人吉人、進士李文祥亦並負意氣,智皆與之善。因相與品核公卿,裁量人物。未幾,孝宗嗣位,弊政多所更。智喜,以為其志且得行,乃復因星變上書曰:

伏讀明詔云「天下利弊所當興革,所在官員人等條具以聞」。此殆陛下知前日登極詔書為奸臣所誤,禁言官毋風聞挾私言事,物論囂然,故復下此條自解耳。夫不曰「朕躬有過,朝政有闕」,而曰「利弊當興革」;不曰「許諸人直言無隱」,而曰「官員人等條具以聞」。陛下所以求言者,已不廣矣。今欲興天下之利,革天下之弊,當求利弊之本原而興且革之,不當毛舉細故,以為利弊在是也。

本原何在?閣臣是已。少師安持祿怙寵,少保吉附下罔上,太子少保直挾詐懷奸,世之小人也。陛下留之,則君德必不就,朝政必不修,此弊所當革者也。致仕尚書王恕忠亮可任大事,尚書王竑剛毅可寢大奸,都御史彭韶方正可決大疑,世之君子也。陛上用之,則君德開明,朝政清肅,此利所當興也。

然君子所以不進,小人所以不退,大抵由宦官權重而已。漢元帝嘗任蕭望之、周堪矣,卒制於弘恭、石顯。宋孝宗嘗任劉俊卿、劉珙矣,卒間於陳源、甘昇。李林甫、牛仙客與高力士相附和,而唐政不綱。賈似道、丁大全與董宋臣相表裏,而宋室不振。君子小人進退之機,未嘗不繫此曹之盛衰。願陛下鑒既往,謹將來,攬天綱,張英斷。凡所以待宦官者,一以高皇帝為法,則君子可進,小人可退,而天下之治出於一矣。以陛下聰明冠世,豈不知刑臣不可委信,然而不免誤用者,殆正心之學未講也。心發於天理,則耳目聰明,言動中節,何宦官之能惑。發於人欲,則一身無主,萬事失綱,投間抵隙,蒙蔽得施。雖有神武之資,亦將日改月化而浸失其初。欲進君子退小人,興天下之利,革天下之弊,豈可得哉?

帝得疏,頷之。居無何,安、直相繼罷斥。而吉任寄如故,銜智刺骨。

鼐常朝當侍班,智告之曰:「祖宗盛時,御史侍班,得面陳政務得失,立取進止。自後惟退而具疏,此君臣情意所由隔也。君幸值維新之日,盍仿先朝故事行之。」及恕赴召至京,智往謁曰:「後世人臣不獲時見天子,故事多茍且。願公且勿受官,先請朝見,取時政不善者歷陳之,力請除革,而後拜命,庶其有濟。若先受官,無復見天子之日矣。」鼐與恕亦未能用其言。

會劉概獄起,吉使其黨魏璋入智名,遂下詔獄。智身親三木,僅屬喘息,慷慨對簿曰:「智見經筵以寒暑輟講,午朝以細事塞責,紀綱廢馳,風俗浮薄,生民憔悴,邊備空虛,私竊以為憂。與鼐等往來論議誠有之,不知其他。」讞者承吉意,竟謫廣東石城所吏目,事具《湯鼐傳》。

智至廣東,總督秦紘檄召修書,乃居會城。聞陳獻章講道新會,往受業,自是學益粹。弘治四年十月得疾遽卒,年二十有六。同年生吳廷舉為順德知縣,殮而歸其喪。天啟初,追謚忠介。

舒芬,字國裳,進賢人。年十二,獻《馴鴈賦》於知府祝瀚,遂知名。正德十二年舉進士第一,授修撰。

時武宗數微行,畋遊無度。其明年,孝貞皇后崩甫踰月,欲幸宣府。託言往視山陵,罷沿道兵衛。芬上言:「陛下三年之內當深居不出,雖釋服之後,固儼然煢疚也。且自古萬乘之重,非奔竄逃匿,未有不嚴侍衛者。又等威莫大於車服,以天子之尊下同庶人,舍大輅袞冕而羸車褻服是御,非所以辨上下、定禮儀。」不聽。

孝貞山陵畢,迎主祔廟,自長安門入。芬又言:「孝貞皇后作配茂陵,未聞失德。祖宗之制,既葬迎主,必入正門。昨孝貞之主,顧從陛下駕由旁門入,他日史臣書之曰「六月己丑,車駕至自山陵,迎孝貞純皇后主入長安門」,將使孝貞有不得正終之嫌,其何以解於天下後世?昨祔廟之夕,疾風迅雷甚雨,意者聖祖列宗及孝貞皇后之靈,儆告陛下也。陛下宜即明詔中外,以示改過。」不報。遂乞歸養,不許。

又明年三月,帝議南巡。時寧王宸濠久蓄異謀,與近倖相結,人情惶懼。言官伏闕諫,忤旨被責讓。芬憂之,與吏部員外郎夏良勝、禮部主事萬潮、庶吉士汪應軫要諸曹連章入諫,眾許諾。芬遂偕編修崔桐,庶吉士江暉、王廷陳、馬汝驥、曹嘉及應軫上疏曰:

「古帝王所以巡狩者,協律度,同量衡,訪遺老,問疾苦,黜陟幽明,式序在位,是以諸侯畏焉,百姓安焉。若陛下之出,不過如秦皇、漢武,侈心為樂而已,非能行巡狩之禮者也。博浪、柏谷,其禍亦可鑒矣。近者西北再巡,六師不攝,四民告病。哀痛之聲,上徹蒼昊。傳播四方,人心震動。故一聞南巡詔書,皆鳥驚獸散。而有司方以迎奉為名,徵發嚴急,江、淮之間蕭然煩費。萬一不逞之徒,乘勢倡亂,為禍非細。且陛下以鎮國公自命,茍至親王國境,或據勛臣之禮以待陛下,將北嚮朝之乎,抑南面受其朝乎?假令循名責實,深求悖謬之端,則左右寵倖無死所矣。尚有事堪痛哭不忍言者:宗籓蓄劉水鼻之釁,大臣懷馮道之心。以祿位為故物,以朝署為市廛,以陛下為弈棋,以革除年間為故事。特左右寵倖知術短淺,無能以此言告陛下耳。使陛下得聞此言,雖禁門之外,亦將警蹕而出,尚敢輕騎慢遊哉?」

疏入,陸完迎謂曰:「上聞有諫者輒恚,欲自引決。諸君且休,勿歸過君上,沽直名。」芬等不應而出。有頃,良勝、潮過芬,扼腕恨完。芬因邀博士陳九川至,酌之酒曰:「匹夫不可奪志,君輩可遂已乎?」明日遂偕諸曹連疏入。帝大怒,命跪闕下五日,期滿復杖之三十。芬創甚,幾斃,舁至翰林院中。掌院者懼得罪,命摽出之,芬曰:「吾官此,即死此耳。」竟謫福建市舶副提舉,裹創就道。

世宗即位,召復故官。嘉靖三年春,昭聖太后壽旦,詔免諸命婦朝賀。芬言:「前者興國太后令旦,命婦朝賀如儀。今遇皇太后壽節,忽行傳免,恐失輕重之宜。乞收成命,以彰聖孝。」帝怒,奪俸三月。時帝欲尊崇本生,芬偕其僚連章極諫。及張璁、桂萼、方獻夫驟擢學士,芬及同官楊維聰、編修王思羞與同列,拜疏乞罷。未幾,復偕同官楊慎等伏左順門哭爭。帝怒,下獄廷杖,奪俸如初。旋遭母喪歸,卒於家,年四十四。世稱「忠孝狀元」。

芬豐神玉立,負氣峻厲,端居竟日無倦容,夜則計過自訟。以倡明絕學為己任。其學貫串諸經,兼通天文律曆,而尤精於《周禮》。嘗曰:「《周禮》視《儀禮》、《禮記》,猶蜀之視吳、魏也。賈氏謂《儀禮》為本,《周禮》為末,妄矣。朱子不加是正,何也?」疾革,其子請所言,惟以未及表章《周禮》為恨。學者稱「梓溪先生」。萬曆中,追謚文節。先是,修撰羅倫以諫謫福建提舉,踰六十年而芬繼之。與倫同鄉同官,所謫地與官又同,福建士大夫遂祀芬配倫云。

崔桐,字來鳳,海門人。鄉試第一,與芬同進士及第。授編修。既諫南巡,並跪闕下,受杖奪俸。嘉靖中,以侍讀出為湖廣右參議,累擢國子祭酒,禮部右侍郎。

馬汝驥,字仲房,綏德人。正德十二年進士。改庶吉士。偕芬等諫南巡,罰跪受杖。教習期滿,當授編修,特調澤州知州。懲王府人虐小民。比王有所屬,輒投其書櫝中不視。陵川知縣貪,汝驥欲黜之。巡按御史為曲解,汝驥不聽,竟褫其官。世宗立,召復編修,尋錄直諫功,增秩一等。預修《武宗實錄》,進修撰。歷兩京國子司業,擢南京右通政,就改國子祭酒,召拜禮部右侍郎。尚書嚴嵩愛重汝驥,入閣稱之,帝特加侍讀學士。汝驥行己峭厲,然性故和易,人望歸焉。卒贈尚書,謚文簡。

應軫等自有傳。

贊曰:詞臣以文學侍從為職,非有言責也。激於名義,侃侃廷諍,抵罪謫而不悔,豈非皎然志節之士歟?奪情之典不始李賢,然自羅倫疏傳誦天下,而朝臣不敢以起復為故事,於倫理所裨,豈淺鮮哉。章懋等引宣宗箴,明國家設官意,不為彰君之過。鄒智指列賢奸,矯拂媮末。舒芬危言聳切,有爰盎攬轡之風。況夫清修峻節,行無瑕尤,若諸子者,洵足以矯文士浮夸之習矣。


\end{pinyinscope}