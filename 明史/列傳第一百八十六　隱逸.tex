\article{列傳第一百八十六 隱逸}

\begin{pinyinscope}
韓愈言:「《蹇》之六二曰『王臣蹇蹇』,而《蠱》之上九曰『高尚其事』,由所居之時不一,而所蹈之德不同。」夫聖賢以用世為心,而逸民以肥遁為節中華書局出版《二程集》校勘本。,豈性分實然,亦各行其志而已。明太祖興禮儒士,聘文學,搜求巖穴,側席幽人,後置不為君用之罰,然韜迹自遠者,亦不乏人。迨中葉承平,聲教淪浹,巍科顯爵,頓天網以羅英俊,民之秀者,無不觀國光而賓王廷矣。其抱瑰材,蘊積學,槁形泉石,絕意當世者,靡得而稱焉。由是觀之,世道升降之端,係所遭逢,豈非其時為之哉。凡徵聘所及,文學行誼可稱者,已散見諸傳。茲取貞節超邁者數人,作《隱逸傳》。

○張介福倪瓚徐舫楊恒陳洄楊引吳海劉閔楊黼孫一元沈周陳繼儒

張介福,字子祺,自懷慶徙吳中。少受學於許衡。二親早終,遂無仕進意。家貧,冬不能具夾襦,或遺以糸寧絮,不受,纖介必以禮。張士誠入吳,有卒犯其家,危坐不為起。刀斫面,仆地,醒復取冠戴之,坐自若。卒怪,以為異物,走去。介福恐發其先墓,往廬焉。士誠聞而欲致之,不可。使其弟往問,答曰:「無樂亂,無貪天禍,無忘國家。」饋之,力辭。已,病革,謂其友曰:「吾志希古人,未能也。惟無污於時,庶幾哉。」遂卒。

倪瓚,字元鎮,無錫人也。家雄於貲,工詩,善書畫。四方名士日至其門。所居有閣曰清閟,幽迥絕塵。藏書數千卷,皆手自勘定。古鼎法書,名琴奇畫,陳列左右。四時卉木,縈繞其外,高木修篁,蔚然深秀,故自號雲林居士。時與客觴詠其中。為人有潔癖,盥濯不離手。俗客造廬,比去,必洗滌其處。求縑素者踵至,瓚亦時應之。至正初,海內無事,忽散其貲給親故,人咸怪之。未幾兵興,富家悉被禍,而瓚扁舟箬笠,往來震澤、三泖間,獨不罹患。張士誠累欲鉤致之,逃漁舟以免。其弟士信以幣乞畫,瓚又斥去。士信恚,他日從賓客遊湖上,聞異香出葭葦間,疑為瓚也,物色漁舟中,果得之。抶幾斃,終無一言。及吳平,瓚年老矣,黃冠野服,混迹編氓。洪武七年卒,年七十四。

徐舫,字方舟,桐廬人。幼輕俠,好擊劍、走馬、蹴踘。既而悔之,習科舉業。已,復棄去,學為歌詩。睦故多詩人,唐有方干、徐凝、李頻、施肩吾,宋有高師魯、滕元秀,號睦州詩派,舫悉取步聚之。既乃遊四方,交其名士,詩益工。行省參政蘇天爵將薦之,舫笑曰:「吾詩人耳,可羈以章紱哉。」竟避去。築室江皋,日苦吟於雲煙出沒間,翛然若與世隔,因自號滄江散人。宋濂、劉基、葉琛、章溢之赴召也,舟溯桐江,忽有人黃冠鹿裘立江上,招基而笑,且語侵之。基望見,急延入舟中。琛、溢競讙謔,各取冠服服之,欲載上黟川,其人不可乃止。濂初未相識,以問。基曰:「此徐方舟也。」濂因起共歡笑,酌酒而別。舫詩有《瑤林》、《滄江》二集。年六十八,丙午春,卒於家。

楊恒,字本初,諸暨人。外族方氏建義塾,館四方遊學士,恒幼往受諸經,輒領其旨要。文峻潔,有聲郡邑間。浦江鄭氏延為師,閱十年退居白鹿山,戴棕冠,披羊裘,帶經耕煙雨間,嘯歌自樂,因自號白鹿生。太祖既下浙東,命欒鳳知州事。鳳請為州學師,恒固讓不起。鳳乃命州中子弟即家問道。政有缺失,輒貽書咨訪。後唐鐸知紹興,欲辟起之,復固辭。宋濂之為學士也,擬薦為國子師,聞不受州郡辟命,乃已。恒性醇篤,與人語,出肺肝相示。事稍乖名義,輒峻言指斥。家無儋石,而臨財甚介,鄉人奉為楷法焉。

時有陳洄者,義烏人。幼治經,長通百家言。初欲以功名顯,既而隱居,戴青霞冠,披白鹿裘,不復與塵事接。所居近大溪,多修竹,自號竹溪逸民。常乘小艇,吹短簫,吹已,叩舷而歌,悠然自適。宋濂俱為之傳。

楊引,吉水人。好學能詩文,為宋濂、陶安所稱。駙馬都尉陸賢從受學,入朝,舉止端雅。太祖喜,問誰教者,賢以引對,立召見,賜食。他日,賢以褻服見,引太息曰:「是其心易我,不可久居此矣。」復以纂修征,亦不就。其教學者,先操履而後文藝。嘗揭《論語鄉黨》篇示人曰:「吾教自有養生術,安事偃仰吐納為。」乃節飲食,時動息,迄老視聽不衰。既歿,安福劉球稱其學探道原,文範後世,去就出處,卓然有陶潛、徐穉之風。

吳海,字朝宗,閩縣人。元季以學行稱。值四方盜起,絕意仕進。洪武初,守臣欲薦諸朝,力辭免。既而徵詣史局,復力辭。嘗言:「楊、墨、釋、老,聖道之賊,管、商、申、韓,治道之賊,稗官野乘,正史之賊,支詞艷說,文章之賊。上之人,宜敕通經大臣,會諸儒定其品目,頒之天下,民間非此不得輒藏,坊市不得輒粥。如是數年,學者生長不涉異聞,其於養德育才,豈曰小補。」因著書一編曰《書禍》,以發明之。與永福王翰善。翰嘗仕元,海數勸之死,翰果自裁。海教養其子偁,卒底成立。平居虛懷樂善,有規過者,欣然立改,因顏其齋曰聞過。為文嚴整典雅,一歸諸理,後學咸宗仰之。有《聞過齋集》行世。

劉閔,字子賢,莆田人。生而純愨。早孤,絕意科舉,求古聖賢禔躬訓家之法,率而行之。祖母及父喪未舉,斷酒肉,遠室家。訓鄰邑,朔望歸,則號哭殯所,如是三年。婦失愛於母,出之,獨居奉養,疾不解衣。母或恚怒,則整衣竟夕跪榻前。祭享奠獻,一循古禮,鄉人莫不欽重。副使羅璟立社學,構養親堂,延閔為師。提學僉事周孟中捐俸助養。知府王弼每祭廟社,必延致齋居,曰:「此人在座,私意自消。」置田二十餘畝贍之,並受不辭。及母歿,即送田還官,廬墓三年。弟婦求分產,閔闔戶自撾,婦感悟乃已。

弘治中,僉都御史林俊上言:「伏見皇太子年踰幼學,習處宮中,罕接外傅,豫教之道似為未備。今講讀侍從諸臣固已簡用,然百司眾職,山林隱逸,不謂無人。以臣所知,則禮部侍郎謝鐸、太僕少卿儲巏、光祿少卿楊廉,可備講員。其資序未合,德行可取者二人,則致仕副使曹時中、布衣劉閔是也。閔,臣縣人,恭慎醇粹,孝行高古。日無二粥,身無完衣,處之晏如。監司劉大夏、徐貫等恒敬禮之。臣謂可禮致時中為宮僚,閔以布衣入侍,必能涵育薰陶,裨益睿質。」時不能用。其後,巡按御史宗彞、饒瑭欲援詔例舉閔經明行修,閔力辭。知府陳效請遂其志,榮以學職。正德元年,遙授儒學訓導。

楊黼,雲南太和人也。好學,讀《五經》皆百遍。工篆籀,好釋典。或勸其應舉,笑曰:「不理性命,理外物耶?」庭前有大桂樹,縛板樹上,題曰桂樓。偃仰其中,歌詩自得。躬耕數畝供甘膬,但求親悅,不顧餘也。注《孝經》數萬言,證群書,根性命,字皆小篆。所用硯乾,將下樓取水,硯池忽滿,自是為常,時人咸異之。父母歿,為傭營葬畢,入雞足,棲羅漢壁石窟山十餘年,壽至八十。子遜迎歸,一日沐浴,令子孫拜,曰:「明日吾行矣。」果卒。

孫一元,字太初,不知何許人,問其邑里,曰:「我秦人也。」嘗棲太白之巔,故號太白山人。或曰安化王宗人,王坐不軌誅,故變姓名避難也。一元姿性絕人,善為詩,風儀秀朗,蹤跡奇譎,烏巾白帢,攜鐵笛鶴瓢,遍遊中原,東踰齊、魯,南涉江、淮,歷荊抵吳越,所至賦詩,談神仙,論當世事,往往傾其座人。鉛山費宏罷相,訪之杭州南屏山,值其晝寢,就臥內與語。送之及門,了不酬答。宏出語人曰:「吾一生未嘗見此人。」時劉麟以知府罷歸,龍霓以僉事謝政,並客湖州,與郡人故御史陵昆善,而長興吳珫隱居好客,三人者並主於其家。珫因招一元入社,稱「苕溪五隱」。一元買田溪上,將老焉。舉人施侃雅善一元,妻以妻妹張氏,生一女而卒,年止三十七。珫等葬之道場山。

沈周,字啟南,長洲人。祖澄,永樂間舉人材,不就。所居曰西莊,日置酒款賓,人擬之顧仲瑛。伯父貞吉,父恒吉,並抗隱。構有竹居,兄弟讀書其中。工詩善畫,臧獲亦解文墨。邑人陳孟賢者,陳五經繼之子也。周少從之遊,得其指授。年十一,游南都,作百韻詩,上巡撫侍郎崔恭。面試《鳳凰臺賦》,援筆立就,恭大嗟異。及長,書無所不覽。文摹左氏,詩擬白居易、蘇軾、陸游,字仿黃庭堅,並為世所愛重。尤工於畫,評者謂為明世第一。

郡守欲薦周賢良,周筮《易》,得《遁》之九五,遂決意隱遁。所居有水竹亭館之勝,圖書鼎彞充牣錯列,四方名士過從無虛日,風流文彩,照映一時。奉親至孝。父歿,或勸之仕,對曰:「若不知母氏以我為命耶?奈何離膝下。」居恒厭入城市,於郭外置行窩,有事一造之。晚年,匿跡惟恐不深,先後巡撫王恕、彭禮咸禮敬之,欲留幕下,並以母老辭。

有郡守徵畫工繪屋壁。里人疾周者,入其姓名,遂被攝。或勸周謁貴遊以免,周曰:「往役,義也,謁貴遊,不更辱乎!」卒供役而還。已而守入覲,銓曹問曰:「沈先生無恙乎?」守不知所對,漫應曰:「無恙。」見內閣,李東陽曰:「沈先生有牘乎?」守益愕,復漫應曰:「有而未至。」守出,倉皇謁侍郎吳寬,問「沈先生何人?」寬備言其狀。詢左右,乃畫壁生也。比還,謁周舍,再拜引咎,索飯,飯之而去。周以母故,終身不遠遊。母年九十九而終,周亦八十矣。又三年,以正德四年卒。

陳繼儒,字仲醇,松江華亭人。幼穎異,能文章,同郡徐階特器重之。長為諸生,與董其昌齊名。太倉王錫爵招與子衡讀書支硎山。王世貞亦雅重繼儒,三吳名下士爭欲得為師友。繼儒通明高邁,年甫二十九,取儒衣冠焚棄之。隱居崑山之陽,構廟祀二陸,草堂數椽,焚香晏坐,意豁如也。時錫山顧憲成講學東林,招之,謝弗往。親亡,葬神山麓,遂築室東佘山,杜門著述,有終焉之志。工詩善文,短翰小詞,皆極風致,兼能繪事。又博文強識,經史諸子、術伎稗官與二氏家言,靡不較核。或刺取瑣言僻事,詮次成書,遠近競相購寫。征請詩文者無虛日。性喜獎掖士類,屨常滿戶外,片言酬應,莫不當意去。暇則與黃冠老衲窮峰泖之勝,吟嘯忘返,足跡罕入城市。其昌為築來仲樓招之至。黃道周疏稱「志尚高雅,博學多通,不如繼儒」,其推重如此。侍郎沈演及御史、給事中諸朝貴,先後論薦,謂繼儒道高齒茂,宜如聘吳與弼故事。屢奉詔徵用,皆以疾辭。卒年八十二,自為遺令,纖悉畢具。


\end{pinyinscope}