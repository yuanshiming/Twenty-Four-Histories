\article{列傳第二十九}

\begin{pinyinscope}
齊泰黃子澄方孝孺盧原質鄭公智林嘉猷胡子昭鄭居貞劉政方法樓璉練子寧宋徵葉希賢茅大芳周嵒卓敬郭任盧迥陳迪黃魁巨敬景清連楹胡閏高翔王度戴德彞謝升丁志方甘霖董鏞陳繼之韓永葉福

齊泰,溧水人。初名德。洪武十七年,舉應天鄉試第一。明年成進士。歷禮、兵二部主事。雷震謹身殿,太祖禱郊廟,擇歷官九年無過者陪祀,德與焉,賜名泰。二十八年,以兵部郎中擢左侍郎。太祖嘗問邊將姓名,泰歷數無遺。又問諸圖籍,出袖中手冊以進,簡要詳密,大奇之。皇太孫素重泰。及即位,命與黃子澄同參國政。尋進尚書。時遺詔諸王臨國中,毋奔喪,王國吏民聽朝廷節制。諸王謂泰矯皇考詔,間骨肉,皆不悅。先是,帝為太孫時,諸王多尊屬,擁重兵,患之。至是因密議削籓。

建文元年,周、代、湘、齊、岷五王相繼以罪廢。七月,燕王舉兵反,師名「靖難」。指泰、子澄為奸臣。事聞,泰請削燕屬籍,聲罪致討。或難之,泰曰:「明其為賊,敵乃可克。」遂定議伐燕,布告天下。時太祖功臣存者甚少,乃拜長興侯耿炳文為大將軍,帥師分道北伐,至真定為燕所敗。子澄薦曹國公李景隆代將,泰極言不可。子澄不聽,卒命景隆將。當是時,帝舉五十萬兵畀景隆,謂燕可旦夕滅。燕王顧大喜曰:「昔漢高止能將十萬,景隆何才,其眾適足為吾資也!」是冬,景隆果敗。帝有懼色,會燕王上書極詆泰、子澄。帝乃解二人任以謝燕,而陰留之京師,仍參密議。景隆遺燕王書,言二人已竄,可息兵。燕王不聽。明年,盛庸捷東昌,帝告廟,命二人任職如故。及夾河之敗,復解二人官求罷兵,燕王曰:「此緩我也。」進益急。

始削籓議起,帝入泰、子澄言,謂以天下制一隅甚易。及屢敗,意中悔,是以進退失據。迨燕兵日逼,復召泰還。未至,京師已不守,泰走外郡謀興復。時購泰急。泰墨白馬走,行稍遠,汗出墨脫。或曰:「此齊尚書馬也。」遂被執赴京,同子澄、方孝孺不屈死。泰從兄弟敬宗等皆坐死,叔時永、陽彥等謫戍。子甫六歲,免死給配,仕宗時赦還。

黃子澄,名水是,以字行,分宜人。洪武十八年,會試第一。由編脩進脩撰,伴讀東宮,累遷太常寺卿。惠帝為皇太孫時,嘗坐東角門謂子澄曰:「諸王尊屬擁重兵,多不法,奈何?」對曰:「諸王護衛兵,纔足自守。倘有變,臨以六師,其誰能支?漢七國非不強,卒底亡滅。大小強弱勢不同,而順逆之理異也。」太孫是其言。比即位,命子澄兼翰林學士,與齊泰同參國政。謂曰:「先生憶昔東角門之言乎?」子澄頓首曰:「不敢忘。」退而與泰謀,泰欲先圖燕。子澄曰:「不然,周、齊、湘、代、岷諸王,在先帝時,尚多不法,削之有名。今欲問罪,宜先周。周王,燕之母弟,削周是剪燕手足也。」謀定,明日入白帝。

會有言周王橚不法者,遂命李景隆帥兵襲執之,詞連湘、代諸府。於是廢肅及岷王楩為庶人;幽代王桂於大同;囚齊王榑於京師。湘王柏自焚死。下燕議周王罪。燕王上書申救,帝覽書惻然,謂事宜且止。子澄與泰爭之,未決,出相語曰:「今事勢如此,安可不斷?」明日又入言曰:「今所慮者獨燕王耳,宜因其稱病襲之。」帝猶豫曰;「朕即位未久,連黜諸王,若又削燕,何以自解於天下?」子澄對曰:「先人者制人,毋為人制。」帝曰:「燕王智勇,善用兵。雖病,恐猝難圖。」乃止。於是命都督宋忠調緣邊官軍屯開平,選燕府護衛精壯隸忠麾下,召護衛胡騎指揮關童等入京,以弱燕。復調北平永清左、右衛官軍分駐彰德、順德,都督徐凱練兵臨清,耿瓛練兵山海關,以控制北平。皆泰、子澄謀也。時燕王憂懼,以三子皆在京師,稱病篤,乞三子歸。泰欲遂收之,子澄曰:「不若遣歸,示彼不疑,乃可襲而取也。」竟遣還。未幾,燕師起,王泣誓將吏曰:「陷害諸王,非由天子意,乃奸臣齊泰、黃子澄所為也。」

始帝信任子澄與泰,聚事削籓。兩人本書生,兵事非其所長。當耿炳文之敗也,子澄謂勝敗常事,不足慮。因薦曹國公李景隆可大任,帝遂以景隆代炳文。而景隆益無能為,連敗於鄭村壩、白溝河,喪失軍輜士馬數十萬。已,又敗於濟南城下。帝急召景隆還,赦不誅。子澄慟哭,請正其罪。帝不聽。子澄拊膺曰:「大事去矣,薦景隆誤國,萬死不足贖罪!」

及燕兵漸南,與齊泰同謫外,密令募兵。子澄微服由太湖至蘇州,與知府姚善倡義勤王。善上言:「子澄才足捍難,不宜棄閒遠以快敵人。」帝復召子澄,未至而京城陷。欲與善航海乞兵,善不可。乃就嘉興楊任謀舉事,為人告,俱被執。子澄至,成祖親詰之。抗辨不屈,磔死。族人無少長皆斬,姻黨悉戍邊。一子變姓名為田經,遇赦,家湖廣咸寧。正德中,進士黃表其後云。

楊任,洪武中由人材起家,歷官袁州知府。時致仕,匿子澄於家,亦磔死。二子禮、益俱斬。親屬戍邊。

方孝孺,字希直,一字希古,寧海人。父克勤,洪武中循吏,自有傳。孝孺幼警敏,雙眸炯炯,讀書日盈寸,鄉人目為「小韓子。」長從宋濂學,濂門下知名士皆出其下。先輩胡翰、蘇伯衡亦自謂弗如。孝孺顧末視文藝,恒以明王道、致太平為己任。嘗臥病,絕糧,家人以告,笑曰:「古人三旬九食,貧豈獨我哉!」父克勤坐「空印」事誅,扶喪歸葬,哀動行路。既免喪,復從濂卒業。

洪武十五年,以吳沉、揭樞薦,召見。太祖喜其舉止端整,謂皇太子曰:「此莊士,當老其才。」禮遣還。後為仇家所連,逮至京。太祖見其名,釋之。二十五年,又以薦召至。太祖曰:「今非用孝孺時。」除漢中教授,日與諸生講學不倦。蜀獻王聞其賢,聘為世子師。每見,陳說道德。王尊以殊禮,名其讀書之廬曰「正學。」

及惠帝即位,召為翰林侍講。明年遷侍講學士,國家大政事輒咨之。帝好讀書,每有疑,即召使講解。臨朝奏事,臣僚面議可否,或命孝孺就扆前批答。時脩《太祖實錄》及《類要》諸書,孝孺皆為總裁。更定官制,孝孺改文學博士。燕兵起,廷議討之,詔檄皆出其手。

建文三年,燕兵掠大名。王聞齊、黃已竄,上書請罷盛庸、吳傑、平安兵。孝孺建議曰:「燕兵久頓大名,天暑雨,當不戰自疲。急令遼東諸將入山海關攻永平;真定諸將渡盧溝搗北平,彼必歸救。我以大兵躡其後,可成擒也。今其奏事適至,宜且與報書,往返踰月,使其將士心懈。我謀定勢合,進而蹴之,不難矣。」帝以為然。命孝孺草詔,遣大理寺少卿薛嵓馳報燕,盡赦燕罪,使罷兵歸籓。又為宣諭數千言授嵓,持至燕軍中,密散諸將士。比至,嵓匿宣諭不敢出,燕王亦不奉詔。五月,吳傑、平安、盛庸發兵擾燕餉道。燕王復遣指揮武勝上書,伸前請。帝將許之。孝孺曰:「兵罷,不可復聚,願毋為所惑。」帝乃誅勝以絕燕。未幾,燕兵掠沛縣,燒糧艘。時河北師老無功,而德州又饋餉道絕,孝孺深以為憂。以燕世子仁厚,其弟高煦狡譎,有寵於燕王,嘗欲奪嫡,謀以計間之,使內亂。乃建議白帝:遣錦衣衛千戶張安齎璽書往北平,賜世子。世子得書不啟封,並安送燕軍前。間不得行。

明年五月,燕兵至江北,帝下詔征四方兵。孝孺曰:「事急矣。遣人許以割地,稽延數日,東南募兵漸集。北軍不長舟楫,決戰江上,勝負未可知也。」帝遣慶成郡主往燕軍,陳其說。燕王不聽。帝命諸將集舟師江上。而陳瑄以戰艦降燕,燕兵遂渡江。時六月乙卯也。帝憂懼,或勸帝他幸,圖興復。孝孺力請守京城以待援兵,即事不濟,當死社稷。乙丑,金川門啟,燕兵入,帝自焚。是日,孝孺被執下獄。

先是,成祖發北平,姚廣孝以孝孺為託,曰:「城下之日,彼必不降,幸勿殺之。殺孝孺,天下讀書種子絕矣。」成祖頷之。至是欲使草詔。召至,悲慟聲徹殿陛。成祖降榻,勞曰:「先生毋自苦,予欲法周公輔成王耳。」孝孺曰:「成王安在?」成祖曰:「彼自焚死。」孝孺曰:「何不立成王之子?」成祖曰:「國賴長君。」孝孺曰:「何不立成王之弟?」成祖曰:「此朕家事。」顧左右授筆札,曰:「詔天下,非先生草不可」孝孺投筆於地,且哭且罵曰:「死即死耳,詔不可草。」成祖怒,命磔諸市。孝孺慨然就死,作絕命詞曰:「天降亂離兮孰知其由,奸臣得計兮謀國用猶。忠臣發憤兮血淚交流,以此殉君兮抑又何求?鳴呼哀哉兮庶不我尤!」時年四十有六。其門人德慶侯廖永忠之孫鏞與其弟銘,檢遺骸瘞聚寶門外山上。

孝孺有兄孝聞,力學篤行,先孝孺死。弟孝友與孝孺同就戮,亦賦詩一章而死。妻鄭及二子中憲、中愈先自經死,二女投秦淮河死。

孝孺工文章,醇深雄邁。每一篇出,海內爭相傳誦。永樂中,藏孝孺文者罪至死。門人王稌潛錄為《侯城集》,故後得行於世。

仁宗即位,諭禮部:「建文諸臣,已蒙顯戮。家屬籍在官者,悉宥為民,還其田土。其外親戍邊者,留一人戍所,餘放還。」萬曆十三年三月,釋坐孝孺謫戍者後裔,浙江、江西、福建、四川、廣東凡千三百餘人。而孝孺絕無後,惟克勤弟克家有子曰孝復。洪武二十五年嘗上書闕下,請減信國公湯和所加寧海賦,謫戍慶遠衛,以軍籍獲免。孝復子琬,後亦得釋為民。世宗時,松江人俞斌自稱孝孺後,一時士大夫信之,為纂《歸宗錄》。既而方氏察其偽,言於官,乃已。神宗初,有詔褒錄建文忠臣,建表忠祠於南京,首徐輝祖,次孝孺云。

孝孺之死,宗族親友前後坐誅者數百人。其門下士有以身殉者,盧原質、鄭公智、林嘉猷,皆寧海人。

原質字希魯,孝孺姑子也。由進士授編修,歷官太常少卿。建文時,屢有建白。燕兵至,不屈,與弟原朴等皆被殺。

公智字叔貞;嘉猷名升,以字行。皆師事孝孺。孝孺嘗曰:「匡我者,二子也。」公智以賢良舉,為御史有聲。嘉猷,洪武丙子以儒士校文四川。建文初,入史館為編修。尋遷陜西僉事。嘗以事入燕邸,知高煦謀傾世子狀。孝孺間燕之謀,實嘉猷發之。

胡子昭,字仲常,初名志高。榮縣人。孝孺為漢中教授時往從學,蜀獻王薦為縣訓導。建文初,與修《太祖實錄》,授檢討。累遷至刑部侍郎。

鄭居貞,閩人。與孝孺友善,以明經歷官鞏昌通判、河南參政。所至有善績。孝孺教授漢中,居貞作《鳳雛行》勖之。諸人皆坐黨誅死。

孝孺主應天鄉試,所得士有長洲劉政、桐城方法。

政,字仲理。燕兵起,草《平燕策》,將上之,以病為家人所沮。及聞孝孺死,遂嘔血卒。

法,字伯通。官四川都司斷事。諸司表賀成祖登極,當署名,不肯,投筆出。被逮,次望江,瞻拜鄉里曰:「得望我先人廬舍足矣。」自沉於江。

成祖既殺孝孺,以草詔屬侍讀樓璉。璉,金華人,嘗從宋濂學。承命不敢辭。歸語妻子曰:「我固甘死,正恐累汝輩耳。」其夕,遂自經。或曰草詔乃括蒼王景,或曰無錫王達云。

練子寧,名安,以字行,新淦人。父伯尚,工詩。洪武初,官起居注。以直言謫外任,終鎮安通判。子寧英邁不群,十八年,以貢士廷試對策,力言:「天之生材有限,陛下忍以區區小故,縱無窮之誅,何以為治?」太祖善其意,擢一甲第二,授翰林修撰。丁母艱,力行古禮。服闋,復官,歷遷工部侍郎。建文初,與方孝孺並見信用,改吏部左侍郎。以賢否進退為己任,多所建白。未幾,拜御史大夫。燕師起,李景隆北征屢敗,召還。子寧從朝中執數其罪,請誅之。不聽。憤激叩首大呼曰:「壞陛下事者,此賊也。臣備員執法,不能為朝廷除賣國奸,死有餘罪。即陛下赦景隆,必無赦臣!」因大哭求死,帝為罷朝。宗人府經歷宋徵、御史葉希賢皆抗疏言景隆失律喪師,懷二心,宜誅。並不納。燕師既渡淮,靖江府長史蕭用道、衡府紀善周是修上書論大計,指斥用事者。書下廷臣議,用事者盛氣以詬二人。子寧曰:「國事至此,尚不能容言者耶?」詬者愧而止。

燕王即位,縛子寧至。語不遜,磔死。族其家,姻戚俱戍邊。子寧從子大亨,官嘉定知縣。聞變,同妻沉劉家河死。里人徐子權以進士為刑部主事,聞子寧死,慟哭賦詩自經。

子寧善文章,孝孺稱其多學而文。弘治中,王佐刻其遺文曰《金川玉屑集》。提學副使李夢陽立金川書院祀子寧,名其堂曰「浩然」。

徵,不知何許人。嘗疏請削罪籓屬籍。燕師入,不屈,並妻子俱死。

希賢,松陽人。亦坐奸黨被殺。或曰去為僧,號雪庵和尚云。

茅大芳,名誧,以字行,泰興人。博學能詩文。洪武中,為淮南學官,召對稱旨。擢秦府長史,制詞以董仲舒為言。大芳益奮激,盡心輔導。額其堂曰「希董」,方孝孺為之記。建文元年遷副都御史。燕師起,遺詩淮南守將梅殷,辭意激烈。聞者壯之。

周璿,洪武末,以天策衛知事建言,擢左僉都御史。燕王稱帝,與大芳並見收,不屈死。而大芳子順童、道壽俱論誅,二孫死獄中。

卓敬,字惟恭,瑞安人。穎悟過人,讀書十行俱下。舉洪武二十一年進士。除戶科給事中,鯁直無所避。時制度未備,諸王服乘擬天子。敬乘間言:「京師,天下視效。陛下於諸王不早辨等威,而使服飾與太子埒,嫡庶相亂,尊卑無序,何以令天下?」帝曰:「爾言是,朕慮未及此。」益器重之。他日與同官見,適八十一人,命改官為「元士」。尋以六科為政事本源,又改曰「源士」。已,復稱給事中。歷官戶部侍郎。

建文初,敬密疏言:「燕王智慮絕倫,雄才大略,酷類高帝。北平形勝地,士馬精強,金、元年由興。今宜徙封南昌,萬一有變,亦易控制。夫將萌而未動者,幾也;量時而可為者,勢也。勢非至剛莫能斷,幾非至明莫能察。」奏入,翌日召問。敬叩首曰:「臣所言天下至計,願陛下察之。」事竟寢。

燕王即位,被執,責以建議徙燕,離間骨肉。敬厲聲曰:「惜先帝不用敬言耳!」帝怒,猶憐其才,命繫獄,使人諷以管仲、魏徵事。敬泣曰:「人臣委贄,有死無二。先皇帝曾無過舉,一旦橫行篡奪,恨不即死見故君地下,乃更欲臣我耶?」帝猶不忍殺。姚廣孝故與敬有隙,進曰:「敬言誠見用,上寧有今日。」乃斬之,誅其三族。

敬立朝慷慨,美豐姿,善談論,凡天官、輿地、律曆、兵刑諸家,無不博究。成祖嘗歎曰::「國家養士三十年,惟得一卓敬。」萬曆初,用御史屠叔方言,表墓建祠。

同時戶部侍郎死者,有郭任、盧迥。

任,丹徒人,一曰定遠人。廉慎有能。建文初,佐戶部。飲食起居俱在公署。時方貶削諸籓,任言:「天下事先本後末則易成。今日儲財粟,備軍實,果何為者?乃北討周,南討湘。舍其本而末是圖,非策也。且兵貴神速,茍曠日持久,銳氣既竭,姑息隨之,將坐自困耳。」燕王聞而惡之。兵起,任與同官盧迥主調兵食。京師失守被擒,不屈死之。子經亦論死,少子戍廣西。

迥,仙居人。爽朗不拘細行。喜飲酒,飲後輒高歌,人謂「迥狂」。及仕,折節恭慎。建文三年,拜戶部侍郎。燕兵入,不屈。縛就刑,長謳而死。台人祀之八忠祠。

陳迪,字景道,宣城人。祖宥賢,明初,從征有功,世撫州守禦百戶,因家焉。迪倜儻有志操。辟府學訓導,為郡草《賀萬壽表》。太祖異之。久之,以通經薦,歷官侍講。出為山東左參政,多惠政。丁內艱。起復,除雲南右布政使。普定、曲靖、烏撒、烏蒙諸蠻煽亂,迪率士兵擊破之,賜金幣。

建文初,徵為禮部尚書。時更修制度,沿革損益,迪議為多。會以水旱詔百官集議,迪請清刑獄,招流民,凡二十餘事,皆從之。尋加太子少保。李景隆等數戰敗,迪陳大計。命督運軍儲。已,聞變,趨赴京師。

燕王即帝位,召迪責問,抗聲不屈。命與子鳳山、丹山等六人磔於市。既死,人於衣帶中得詩及《五噫歌》,辭意悲烈。蒼頭侯來保拾其遺骸歸葬。妻管縊死。幼子珠生五月,乳母潛置溝中,得免。八歲,為怨家所訐。成祖宥其死,戍撫寧。尋徙登州,為蓬萊人。洪熙初,赦還鄉,給田產。成化中,寧國知府塗觀建祠祀迪。弘治間,裔孫鼎舉進士,仕至應天府尹,剛鯁有聲。

黃魁,不知何許人。為禮部侍郎,有學行,習典禮。迪及侍郎黃觀皆愛敬人。燕兵入,不屈死。

有巨敬者,平涼人。為御史,改戶部主事,充史官,以清慎稱。與迪同不屈死,夷其族。

景清,本耿姓,訛景,真寧人。倜儻尚大節,讀書一過不忘。洪武中進士,授編修,改御史。三十年春,召見,命署左僉都御史。以奏疏字誤,懷印更改,為給事中所劾,下詔獄。尋宥之。詔巡察川、陜私茶,除金華知府。建文初,為北平參議。燕王與語,言論明晰,大稱賞。再遷御史大夫。燕師入,諸臣死者甚眾。清素預密謀,且約孝孺等同殉國,至是獨詣闕自歸,成祖命仍其官,委蛇班行者久之。一日早朝,清衣緋懷刃入。先是,日者奏異星赤色犯帝座,甚急。成祖故疑清。及朝,清獨著緋。命搜之,得所藏刃。詰責,清奮起曰:「欲為故主報仇耳!」成祖怒,磔死,族之。籍其鄉,轉相攀染,謂之「瓜蔓抄」,村里為墟。

初,金川門之啟,御史連楹叩馬欲刺成祖,被殺,屍植立不仆。楹,襄垣人。

胡閏,字松友,鄱陽人。太祖征陳友諒,過長沙王吳芮祠,見題壁詩,奇之,立召見帳前。洪武四年,郡舉秀才,入見。帝曰:「此書生故題詩鄱陽廟壁者邪?」授都督府都事,遷經歷。建文初,選右補闕,尋進大理寺少卿。燕師起,與齊、黃輩晝夜畫軍事。京師陷,召閏,不屈,與子傳道俱死。幼子傳慶戍邊。四歲女郡奴入功臣家,稍長識大義,日以爨灰污面。洪熙初,赦還鄉。貧甚,誓不嫁。見者競遺以錢穀,曰:「此忠臣女也。」

高翔,朝邑人。洪武中,以明經為監察御史。建文時,戮力兵事。成祖聞其名,與閏同召,欲用之。翔喪服入見,語不遜。族之,發其先冢,親黨悉戍邊。諸給高氏產者皆加稅,曰:「令世世罵翔也。」

王度,字子中,歸善人。少力學,工文辭,用明經薦為山東道監察御史。建文時,燕兵起,度悉心贊畫。及王師屢敗,度奏請募兵。小河之捷,奉命勞軍徐州。還,方孝孺與度書,誓死社稷。燕王稱帝,坐方黨謫戍賀縣,又坐語不遜,族。

度有智計。盛庸之代景隆,度密陳便宜,是以有東昌之捷。景隆徵還,赦不誅,反用事。忌庸等功,讒間之,度亦見疏。論者以其用有未盡,惜之。

戴德彞,奉化人。洪武二十七年進士。累官侍講。太祖諭之曰:「翰林雖職文學,然既列禁近,凡國家政治得失,民生利害,當知無不言。昔唐陸贄、崔群、李絳在翰林,皆能正言讜論,補益當時。汝宜以古人自期。」已,改監察御史。建文時,改左拾遺。燕王入,召見,不屈,死之。德彞死時,兄弟並從京師。嫂項家居,聞變,度禍且族,令闔舍逃去。匿德彞二子山中,毀戴氏族譜,獨身留家。收者至,無所得,械項至京,搒掠終無一言,戴族獲全。

時御史不屈死者,有諸城謝昇、聊城丁志方。而懷寧甘霖從容就戮,子孫相戒不復仕。

又董鏞,不知何許人。諸御史有志節者,時時會鏞所,誓以死報國。諸將校觀望不力戰,鏞輒露章劾之。城破被殺,家戍極邊。

而給事中死者,則有陳繼之、韓永、葉福三人。

繼之,莆田人,建文二年進士。時江南僧道多腴田,繼之請人限五畝,餘以賦民。從之。兵事亟,數條奏機宜。燕兵入,不屈,見殺,父母兄弟悉戍邊。

永,西安人,或曰浮山。貌魁梧,音吐洪亮,每慷慨論兵事。燕王入,欲官之,抗辭,不屈死。

福,侯官人,繼之同年生。燕兵至,守金川門,城陷,死之。

贊曰:帝王成事,蓋由天授。成祖之得天下,非人力所能禦也。齊、黃、方、練之儔,抱謀國之忠,而乏制勝之策。然其忠憤激發,視刀鋸鼎鑊甘之若飴,百世而下,凜凜猶有生氣。是豈泄然不恤國事而以一死自謝者所可同日道哉!由是觀之,固未可以成敗之常見論也。


\end{pinyinscope}