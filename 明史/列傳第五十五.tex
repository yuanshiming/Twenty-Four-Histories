\article{列傳第五十五}

\begin{pinyinscope}
○曹鼐張益鄺埜王佐丁鉉等孫祥謝澤袁彬哈銘袁敏

曹鼐,字萬鐘,寧晉人。少伉爽有大志,事繼母以孝聞。宣德初,由鄉舉授代州訓導,願授別職,改泰和縣典史。七年督工匠至京師,疏乞入試,復中順天鄉試。明年舉進士一甲第一,賜宴禮部。進士宴禮部,自鼐始。入翰林,為修撰。

正統元年,充經筵講官。《宣宗實錄》成,進侍講,錫三品章服。五年,以楊榮、楊士奇薦,入直文淵閣,參預機務。鼐為人內剛外和,通達政體。榮既歿,士奇常病不視事,閣務多決於鼐。帝以為賢,進翰林學士。十年進吏部左侍郎兼學士。

十四年七月,也先入寇,中官王振挾帝親征。朝臣交章諫,不聽。鼐與張益以閣臣扈從。未至大同,士卒已乏糧。宋瑛、朱冕全軍沒。諸臣請班師,振不許,趣諸軍進。大將朱勇膝行聽命,尚書鄺埜、王佐跪草中,至暮不得請。欽天監正彭德清言天象示警,若前,恐危乘輿。振詈曰:「爾何知!若有此,亦天命也。」鼐曰:「臣子固不足惜,主上繫天下安危,豈可輕進?」振終不從。前驅敗報踵至,始懼,欲還。定襄侯郭登言於鼐、益曰:「自此趨紫荊,裁四十餘里,駕宜從紫荊入。」振欲邀帝至蔚州幸其第,不聽,復折而東,趨居庸。

八月辛酉次土木。地高,掘地二丈不及水。瓦剌大至,據南河。明日佯卻,且遣使通和。帝召鼐草詔答之。振遽令移營就水,行亂。寇騎蹂陣入,帝突圍不得出,擁以去。鼐、益等俱及於難。景帝立,贈鼐少傅、吏部尚書、文淵閣大學士,謚文襄,官其子恩大理評事。英宗復位,加贈太傅,改謚文忠,復官其孫榮錦衣百戶。鼐弟鼎進士,歷吏科都給事中。

張益,字士謙,江寧人。永樂十三年進士。由庶吉士授中書舍人,改大理評事。與修《宣宗實錄》成,改修撰。博學強記,詩文操筆立就,三楊雅重之。尋進侍讀學士,正統十四年入文淵閣。未三月,遽蒙難以歿。景帝立,贈學士,謚文僖。曾孫琮進士。嘉靖初歷官南京右都御史。

鄺埜,字孟質,宜章人。永樂九年進士,授監察御史。成祖在北京,或奏南京鈔法為豪民沮壞,帝遣埜廉視。眾謂將起大獄,埜執一二市豪歸。奏曰:「市人聞令震懼,鈔法通矣。」事遂已。倭犯遼東,戍守失律者百餘人,皆應死。命埜按問,具言可矜狀,帝為宥之。營造北京,執役者鉅萬,命埜稽省,病者多不死。

十六年有言秦民群聚謀不軌者,擢埜陜西按察副使,敕以便宜調兵剿捕。埜白其誣,詔誅妄言者。宣德四年振關中饑。在陜久,刑政清簡。父憂服除,擢應天府尹。蠲苛急政,市征田稅皆酌其平。

正統元年進兵部右侍郎。明年,尚書王驥出督軍,埜獨任部事。時邊陲多警,將帥乏人,埜請令中外博舉謀略材武士,以備任使。六年,山東災。埜請寬民間孳牧馬賠償之令,以蘇其力。

十年進尚書。舊例諸衛自百戶以下當代者,必就試京師,道遠無資者,終身不得代。埜請就令各都司試之,人以為便。瓦剌也先勢盛,埜請為備,又與廷臣議上方略,請增大同兵,擇智謀大臣巡視西北邊務。尋又請罷京營兵修城之役,令休息以備緩急。時不能用。

也先入寇,王振主親征,不與外廷議可否。詔下,埜上疏言:「也先入犯,一邊將足制之。陛下為宗廟社稷主,奈何不自重。」不聽。既扈駕出關,力請回鑾。振怒,令與戶部尚書王佐皆隨大營。埜墮馬幾殆,或勸留懷來城就醫。埜曰:「至尊在行,敢託疾自便乎?」車駕次宣府,朱勇敗沒。埜請疾驅入關,嚴兵為殿。不報。又詣行在申請。振怒曰:「腐儒安知兵事,再言者死!」埜曰:「我為社稷生靈言,何懼?」振叱左右扶出。埜與佐對泣帳中。明日,師覆,埜死,年六十五。

埜為人勤廉端謹,性至孝。父子輔為句容教官,教埜甚嚴。埜在陜久,思一見父,乃謀聘父為鄉試考官。父怒曰:「子居憲司,而父為考官,何以防閑?」馳書責之。埜又嘗寄父褐,復貽書責曰:「汝掌刑名,當洗冤釋滯,以無忝任使,何從得此褐,乃以污我。」封還之。埜奉書跪誦,泣受教。景泰初,贈埜少保,官其子儀為主事。成化初,謚忠肅。

王佐,海豐人。永樂中舉於鄉。卒業太學,以學行聞,擢吏科給事中。器宇凝重,奏對詳雅,為宣宗所簡注。

宣德二年,超拜戶部右侍郎。以太倉、臨清、德州、淮、徐諸倉多積弊,敕佐巡視。平江伯陳瑄言,漕卒十二萬人,歲漕艱苦,乞僉南方民如軍數,更番轉運。詔佐就瑄及黃福議之。佐還奏,東南民力已困,議遂寢。受命治通州至直沽河道。已,赴宣府議屯田事宜。

英宗初立,出鎮河南。奏言軍衛收納稅糧,奸弊百出,請變其制。廷議自邊衛外,皆改隸有司。尋召還,命督理甘肅軍餉。正統元年理長蘆鹽課,三年提督京師及通州倉場,所至事無不辦。

六年,尚書劉中敷得罪,召理部事,尋進尚書。十一年承詔訊安鄉伯張安兄弟爭祿事,坐與法司相諉,被劾下吏,獲釋。時軍旅四出,耗費動以鉅萬,府庫空虛。佐從容調劑,節縮有方。在戶部久,不為赫赫名,而寬厚有度,政務糾紛,未嘗廢學,人稱其君子。

土木之變,與鄺埜、丁鉉、王永和、鄧棨同死難。贈少保,官其子道戶部主事。成化初,謚忠簡。

丁鉉,字用濟,豐城人。永樂中進士。授太常博士。歷工、刑、吏三部員外郎,進刑部郎中。正統三年超拜刑部侍郎。九年出理四川茶課,奏減其常數,以俟豐歲。振饑江淮及山東、河南,民咸賴之。平居恂恂若無能,臨事悉治辦。從征歿,贈刑部尚書,官其子琥大理評事。後謚襄愍。

王永和,字以正,崑山人。少至孝。父病伏枕十八年,侍湯藥無少懈。永樂中舉於鄉,歷嚴州、饒州訓導。以蹇義薦,為兵科給事中。嘗劾都督王彧鎮薊州縱寇,及錦衣馬順不法事。持節冊韓世子妃,糾中官蹇傲罪。以勁直聞。正統六年進都給事中。八年擢工部右侍郎。從征歿,贈工部尚書,官其子汝賢大理評事。後謚襄敏。

鄧棨,字孟擴,南城人。永樂末年進士。授監察御史,奉敕巡按蘇、松諸府。期滿將代去,父老赴闕乞留,得請。旋以憂去。宣德十年,陜西闕按察使,詔廷臣舉清慎有威望者。楊士奇薦棨,遂以命之。正統十年入為右副都御史。北征扈從,師出居庸關,疏請回鑾,以兵事專屬大將。至宣府、大同,復再上章。皆不報。及遇變,同行者語曰:「吾輩可自脫去。」棨曰:「鑾輿失所,我尚何歸!主辱臣死,分也。」遂死。贈右都御史,官其子瑺大理評事。後謚襄敏。

英宗之出也,備文武百官以行。六師覆於土木,將相大臣及從官死者不可勝數。英國公張輔及諸侯伯自有傳。其餘姓氏可考者,卿寺則龔全安、黃養正、戴慶祖、王一居、劉容、凌壽;給事、御史則包良佐、姚銑、鮑輝、張洪、黃裳、魏貞、夏誠、申祐、尹竑、童存德、孫慶、林祥鳳;庶寮則齊汪、馮學明、王健、程思溫、程式、逯端、俞鑑、張瑭、鄭瑄、俞拱、潘澄、錢昺、馬預、尹昌、羅如墉、劉信、李恭、石玉。景帝立,既贈恤諸大臣,自給事、御史以下,皆降敕褒美,錄其子為國子生,一時恤典綦備云。

龔全安,蘭谿人。進士,授工科給事中,累遷左通政。歿贈通政使。黃養正,名蒙,以字行,瑞安人。以善書授中書舍人,累官太常少卿。歿贈太常卿。戴慶祖,溧陽人,王一居,上元人。俱樂舞生,累官太常少卿。歿,俱贈太常卿。包良佐,字克忠。慈谿人。進士,授吏科給事中。鮑輝,字淑大,浙江平陽人。進士,授工科給事中,數有建白。張洪,安福人;黃裳,字元吉,曲江人。俱進士,授御史。裳嘗言寧、紹、台三府疫死三萬人,死者宜蠲租,存者宜振恤。巡視兩浙鹽政,請恤水災。報可。魏貞,懷遠人。進士,官御史。申祐。字天錫,貴州婺川人。父為虎齧。祐持梃奮擊之,得免。舉於鄉,入國學,帥諸生救祭酒李時勉。旋登進士,拜四川道御史,以謇諤聞。尹竑,字太和,巴人;童存德,字居敬,蘭谿人。俱進士,官御史。林祥鳳,字鳴皋,莆田人。由鄉舉授訓導,擢御史。齊汪,字源澄,天台人。以進士歷兵部車駕司郎中。程思溫,婺源人;程式,常熟人;逯端,仁和人。俱進士,官員外郎。俞鑒,字元吉,桐廬人。以進士授兵部職方司主事。駕北征,郎中胡寧當從,以病求代,鑑慷慨許諾。或曰:「家遠子幼奈何?」鑑曰:「為國,臣子敢計身家!」尚書鄺埜知其賢,數與計事,鑒曰:「惟力勸班師耳。」時不能用。張瑭,字廷玉,慈谿人。進士,授刑部主事。尹昌,吉永人。進士,官行人司正。羅如墉,字本崇,廬陵人。進士,授行人。從北征,瀕行,訣妻子,誓以死報國,屬翰林劉儼銘其墓。儼驚拒之,如墉笑曰:「行當驗耳。」後數日果死。劉容,太僕少卿。凌壽,尚寶少卿。夏誠、孫慶皆御史。馮學明,郎中。王健,員外郎。俞拱、潘澄、錢昺,皆中書舍人。馬預,大理寺副。劉信,夏官正。李恭,石玉,序班。里居悉無考。

孫祥,大同人。正統十年進士。授兵科給事中。擢右副都御史,守備紫荊關。也先逼關,都指揮韓青戰死,祥堅守四日。也先由間道入,夾攻之,關破。祥督兵巷戰,兵潰被殺,言官誤劾祥棄城遁。寇退,有司修關,得其屍戰地,焚而瘞之,不以聞。祥弟祺詣闕言冤,詔恤其家。成化改元,錄其子紳為大理寺右評事。

又謝澤者,上虞人。永樂十六年進士。由南京刑部主事出為廣西參政。正統末,擢通政使,守備白羊口。王師敗於土木,守邊者無固志,澤與其子儼訣而行。受事未數日,也先兵大入,守將呂鐸遁。澤督兵扼山口,大風揚沙,不辨人馬。或請移他關避敵,澤不可。寇至,眾潰,澤按劍厲聲叱賊,遂被殺。事聞,遣官葬祭,錄儼為大理評事。

袁彬,字文質,江西新昌人。正統末,以錦衣校尉扈帝北征。土木之變,也先擁帝北去,從官悉奔散,獨彬隨侍,不離左右。也先之犯大同、宣府,逼京師,皆奉帝以行。上下山阪,涉溪澗,冒危險,彬擁護不少懈。帝駐蹕土城,欲奉書皇太后貽景帝及諭群臣,以彬知書令代草。帝既入沙漠,所居止毳帳敝幃,旁列一車一馬,以備轉徙而已。彬周旋患難,未嘗違忤。夜則與帝同寢,天寒甚,恒以脅溫帝足。

有哈銘者,蒙古人。幼從其父為通事,至是亦侍帝。帝宣諭也先及其部下,嘗使銘。也先輩有所陳請,亦銘為轉達。帝獨居氈廬,南望悒鬱。二人時進諧語慰帝,帝亦為解顏。

中官喜寧為也先腹心。也先嘗謂帝曰:「中朝若遣使來,皇帝歸矣。」帝曰:「汝自送我則可,欲中朝遣使,徒費往返爾。」寧聞,怒曰:「欲急歸者彬也,必殺之。」寧勸也先西犯寧夏,掠其馬,直趨江表,居帝南京。彬、銘謂帝曰:「天寒道遠,陛下又不能騎,空取凍飢。且至彼而諸將不納,奈何?」帝止寧計。寧又欲殺二人,皆帝力解而止。也先將獻妹於帝,彬請駕旋而後聘,帝竟辭之。也先惡彬、銘二人,欲殺者屢矣。一日縛彬至曠埜,將支解之。帝聞,如失左右手,急趨救,乃免。彬嘗中寒,帝憂甚,以身壓其背,汗浹而愈。帝居漠北期年,視彬猶骨肉也。

及帝還京,景帝僅授彬錦衣試百戶。天順復辟,擢指揮僉事。尋進同知。帝眷彬甚,奏請無不從。內閣商輅既罷,彬乞得其居第。既又以湫隘,乞官為別建,帝亦報從。彬娶妻,命外戚孫顯宗主之,賜予優渥。時召入曲宴,敘患難時事,歡洽如故時。其年十二月進指揮使,與都指揮僉事王喜同掌衛事。二人嘗受中官夏時囑,私遣百戶季福偵事江西。福者,帝乳媼夫也。詔問誰所遣,二人請罪。帝曰:「此必有主使者。」遂下福吏,得二人受囑狀。所司請治時及二人罪。帝宥時,二人贖徒還職,而詔自今受囑遣官者,必殺無赦。已而坐失囚,喜解職,彬遂掌衛事。五年秋,以平曹欽功,進都指揮僉事。

時門達恃帝寵,勢傾朝野。廷臣多下之,彬獨不為屈。達誣以罪,請逮治。帝欲法行,語之曰:「任汝往治,但以活袁彬還我。」達遂鍛煉成獄。賴漆工楊塤訟冤,獄得解。然猶調南京錦衣衛,帶俸閒住。語詳《達傳》。

越二月,英宗崩,達得罪,貶官都勻。召彬復原職,仍掌衛事。未幾,達徵下獄,充軍南丹。彬餞之於郊,餽以贐。成化初,進都指揮同知。久之,進都指揮使。先是,掌錦衣衛者,率張權勢,罔財賄。彬任職久,行事安靜。

十三年擢都督僉事,蒞前軍都督府。卒於官。世襲錦衣僉事。

哈銘從帝還,賜姓名楊銘,歷官錦衣指揮使,數奉使外蕃為通事。孝宗嗣位,汰傳奉官,銘以塞外侍衛功,獨如故。以壽卒於官。

袁敏者,金齒衛知事也。英宗北征,應募從至大同。及駕還,駐萬全左衛。敏見敵騎逼,請留精兵三四萬人扼其衝,而車駕疾驅入關。王振不納,六師遂覆。敏跳還,上書景帝曰:「上皇曩居九重,所服者袞繡,所食者珍羞,所居者瓊宮瑤室。今駕陷沙漠,服有袞繡乎?食有珍羞乎?居有宮室乎?臣聞之,主辱臣死。上皇辱至此,臣子何以為心,臣不惜碎首刳心。乞遣官一人,或就令臣齎書及服御物問安塞外,以盡臣子之義。臣雖萬死,心實甘之。」命禮部議,竟報寢。

贊曰:異哉,土木之敗也。寇非深入之師,國非積弱之勢,徒以宦豎竊柄,狎寇弄兵,逆眾心而驅之死地,遂致六師撓敗,乘輿播遷,大臣百官身膏草野。夫始之不能制其不出,出不能使之早旋,枕藉疆場,無益於敗。然值倉皇奔潰之時,主辱臣死,志異偷生,亦可無譏於傷勇矣。


\end{pinyinscope}