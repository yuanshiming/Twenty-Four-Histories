\article{列傳第一百七十二 儒林三}

\begin{pinyinscope}
○孔希學孔彥繩顏希惠曾質粹孔聞禮孟希文仲于陛周冕程接道程克仁張文運邵繼祖朱梴朱墅

孔希學,字士行,先聖五十六代孫也,世居曲阜。祖思晦,字明道,仕元為教諭,有學行。仁宗時,以思晦襲封衍聖公,卒謚文肅,子克堅襲。

克堅,字璟夫。至正六年,中書言衍聖公階止嘉議大夫,與爵不稱,乃進通奉大夫,予銀印。十五年有薦其明習禮樂者,徵為同知太常禮儀院事,以子希學襲封。克堅累遷國子祭酒。二十二年,克堅謝病還闕里,後起集賢學士、山東廉訪使,皆不赴。洪武元年三月,徐達下濟寧,克堅稱疾,遣希學來見,達送之京師。希學奏父病不能行,太祖敕諭克堅,末言「稱疾則不可」。會克堅亦來朝,遇使者淮安,惶恐兼程進,見於謹身殿問以年,對曰:「臣年五十有三。」曰:「爾年未邁,而病嬰之。今不煩爾以官。爾家,先聖後,子孫不可不學。爾子溫厚,俾進學。」克堅頓首謝。即日賜宅一區,馬一匹,米二十石。明日復召見,命以訓厲族人。因顧侍臣曰:「先聖後,特優禮之,養以祿而不任以事也。」

十一月命希學襲封衍聖公。置官屬,曰掌書,曰典籍,曰司樂,曰知印,曰奏差,曰書寫,各一人。立孔、顏、孟三氏教授司,教授、學錄、學司各一人。立尼山、洙泗二書院,各設山長一人。復孔氏子孫及顏、孟大宗子孫徭役。又命其族人希大為曲阜世襲知縣。而進衍聖公秩二品,階資善大夫。賜之誥曰:「古之聖人,自羲、農至於文、武,法天治民,明並日月,德化之盛莫有加焉。然皆隨時制宜,世有因革。至於孔子,雖不得其位,會前聖之道而通之,以垂教萬世,為帝者師。其孫子思,又能傳述而名言之,以極其盛。有國家者,求其統緒,尊其爵號,蓋所以崇德報功也。歷代以來,膺襲封者或不能繩其祖武,朕甚閔焉。當臨馭之初,訪世襲者得五十六代孫孔希學,大宗是紹,爰行典禮,以致褒崇。爾其領袖世儒,益展聖道之用於當世,以副朕之至望,豈不偉歟!」希大階承事郎,賜之敕。

三年春,克堅以疾告歸,遣中使慰問。疾篤,詔給驛還家,賜白金文綺,舟次邳州卒。

六年八月,希學服闋入朝,命所司致廩餼,從人皆有賜,復勞以敕,賜襲衣冠帶。九月辭歸,命翰林官餞於光祿寺,賚白金文綺。明年二月,希學言:「先聖廟堂廊廡圮壞,祭器、樂器、法服不備,乞命有司修治。先世田,兵後多蕪,而征賦如故,乞減免。」並從之。自是,每歲入朝,班亞丞相,皆加宴賚。

希學好讀書,善隸法,文詞爾雅。每賓客宴集,談笑揮灑,爛然成章。承大亂之後,廟貌服物,畢力修舉,盡還舊觀。十四年卒。命守臣致祭。

子訥,字言伯,十七年正月襲封。命禮官以教坊樂導送至國學,學官率諸生二千餘人迎於成賢街。自後,每歲入覲,給符乘傳。帝既革丞相官,遂令班文臣首。訥性恭謹,處宗黨有恩。建文二年卒。子公鑒襲。

公鑑,字昭文,有孝行,嗣爵二年卒。成祖即位,遣使致祭。

子彥縉,字朝紳,永樂八年襲,甫十歲,命肄業國學,久之遺歸。十五年修闕里文廟成,御製碑文勒石。仁宗踐阼,彥縉來朝。仁宗語侍臣曰:「外蕃貢使皆有公館。衍聖公假館民間,非崇儒重道意。」遂賜宅東安門外。宣德四年,彥縉將遣使福建市書,咨禮部,部臣以聞,命市與之。已,奏闕里雅樂及樂舞生冠服敝壞,詔所司修治。景泰元年,帝幸學。彥縉率三氏子孫觀禮,賜坐彞倫堂聽講。幸學必先期召衍聖公,自此始。彥縉幼孤,能自立,然與族人不睦。景泰六年,族祖克昫等與彥縉相訐,帝置不問。彥縉子承慶,先卒。孫弘緒,字以敬,甫八歲,而彥縉卒。妾江訴弘緒幼弱,為族人所侵。詔遣禮部郎為治喪,而命其族父公恂理家事。驛召弘緒至京襲封,賜玉帶金印,簡教授一人課其學。英宗復辟,入賀。朝見便殿,握其手,置膝上,語良久。弘緒纔十歲,進止有儀,帝甚悅。每歲入賀聖壽。帝聞其賜第湫隘,以大第易之。凡南城賞花、西苑較射,皆與焉。

公恂,字宗文。景泰五年舉會試。聞母疾,不赴廷對。帝問禮部,得其故,遣使召之。日且午,不及備試卷,命翰林院給筆札。登第,即丁母憂歸。天順初,授禮科給事中。大學士李賢言:「公恂,至聖後,贊善司馬恂,宋大賢溫國公光後,宜輔導太子。」帝喜,同日超拜少詹事,侍東宮講讀。入語孝肅皇后曰:「吾今日得聖賢子孫為汝子傅。」孝肅皇后者,憲宗生母,方以皇貴妃有寵,於是具冠服拜謝,宮中傳為盛事云。成化初,以言事謫漢陽知府,未至,丁父憂。服闋,還故秩,蒞南京詹事府。久之卒。

弘緒少貴,又恃婦翁大學士李賢,多過舉。成化五年被劾,按治,奪爵為庶人,令其弟弘泰襲。弘泰歿,爵仍歸弘緒子。

弘泰,字以和。既嗣爵,弘治十一年,山東按臣言弘緒遷善改行,命復冠帶。明年六月,聖殿災,弘泰方在朝,弘緒率子弟奔救,素服哭廟,蔬食百日。弘泰還,亦齋哭如居喪。弘泰生七月而孤,奉母孝,與弘緒友愛,無間言。十六年卒,弘緒子聞韶襲。

聞韶,字知德。明年,新廟建,規制踰舊,遣大學士李東陽祭告,御製碑文勒石。正德三年以尼山、洙泗二書院及鄒縣子思子廟各有禮事,奏請弟聞禮主之。帝授聞禮《五經》博士,主子思子祀事,世以衍聖公弟為之。兩書院各設學錄一人,薦族之賢者充焉。六年,山東盜起,聞韶與巡撫趙璜請城闕里,遷曲阜縣治以衛廟,不果行。嘉靖二十五年,聞韶卒,子貞幹襲。

貞幹,字用濟。三十五年入朝。卒,子尚賢襲。

尚賢,字象之。巡撫丁以忠言:「尚賢沖年,宜如弘泰例,國學肄業。」從之。萬曆九年,庶母郭氏訐尚賢。帝為革供奉女樂二十六戶,令三歲一朝。十七年,尚賢仍請比歲入賀,許之。尚賢博識。天啟元年卒。子蔭椿先卒,無嗣,從弟子蔭植襲。

蔭植,字對寰。祖貞寧,衍聖公貞幹弟也,仕為《五經》博士。父尚坦,國學生,追封衍聖公。蔭植先為博士,尚賢既喪子,遂育為嗣。天啟四年以覃恩加太子太保。崇禎元年加太子太傅。

孔彥繩,字朝武,衢州西安人,先聖五十九代孫也。宋建炎中,衍聖公端友扈蹕南渡,因家衢州。高宗命以州學為家廟,賜田五頃,以奉祭祀。五傳至洙。元至元間,命歸曲阜襲封。洙讓爵曲阜之弟治。弘治十八年,衢州知府沈傑奏言:「衢州聖廟,自孔洙讓爵之後,衣冠禮儀,猥同氓庶。今訪得洙之六世孫彥繩,請授以官,俾主祀事。」又言:「其先世祭田,洪武初,輕則起科,後改征重稅,請仍改輕,以供祀費。」帝可之。正德元年授彥繩翰林院五經博士,子孫世襲,并減其祭田之稅。

彥繩卒,子承美,字永實,十四年襲。卒,子弘章,字以達,嘉靖二十六年襲。卒,子聞音,字知政,萬曆五年襲。卒,子貞運,字用行,四十三年襲。時以在曲阜者為孔氏北宗,在西安者為南宗云。

顏希惠,復聖五十九代孫也。洪武初,以顏子五十七代孫池為宣德府學教授。十五年改三氏學教授,以奉祀事。池,字德裕。子拳,字克膺。拳子希仁,字士元。景泰三年詔以顏、孟子孫長而賢者各一人,至京官之。其年,希仁為巡按御史顧躭所劾。詔黜希仁,召希惠以為翰林院《五經》博士。未幾,以希惠非嫡子,仍以希仁長子議為之。議,字定伯,成化元年賜第京師,入覲,馳驛以為常。議卒,子鋐,字宗器,十八年襲。卒,子重德,字尚本,正德二年襲。卒,子從祖,字守嗣,襲。卒,無子,嘉靖四十一年以從祖從父重禮之長子肇先為嗣。肇先,字啟源。卒,子嗣慎,字用修襲。卒,長子尹宗先卒,次子尹祚,字永錫,萬曆年襲。尹祚為人博學好義,尹宗之子伯貞既長,遂以其職讓之。伯貞,字叔節,二十七年襲。卒,子幼,弟伯廉,字叔清,三十四年襲。卒,子紹緒,崇禎十四年襲。

曾質粹,字好古,吉安永豐人,宗聖五十九代孫也。其先,都鄉侯據避新莽之亂,徙家豫章,子孫散居撫、吉諸郡間。成化初,山東守臣上言:「嘉祥縣南武山西南,元寨山之東麓,有漁者陷入穴中,得懸棺,碣曰曾參之墓。」詔加修築。正德間,山東僉事錢鋐訪得曾子之後一人於嘉祥山中,未幾而沒。嘉靖十二年,以學士顧鼎臣言,詔求嫡嗣。於是江西撫按以質粹名聞,命回嘉祥,以衣巾奉祀。十八年,授翰林院《五經》博士,子孫世襲。三十九年卒。子昊,未襲卒。昊子繼祖,字繩之,少病目,江西族人袞謀奪其職,為給事中劉不息、御史劉光國所糾,於是罷袞官,而繼祖仍主祀事。卒,子承業,字洪福,萬曆五年襲。卒,子弘毅,字泰東,崇禎元年襲。卒,子聞達,字象輿,十四年襲。

孔聞禮,字知節,衍聖公聞韶弟也。正德二年詔授翰林院《五經》博士,以奉述聖祀事。自後,世以衍聖公弟為之。聞禮卒,嘉靖二十五年,貞寧字用致襲。卒,萬曆二十二年,蔭桂襲。卒,天啟二年,蔭隆襲。卒,八年,尚達襲。卒,崇禎十年,蔭相襲。卒,十四年,蔭錫襲。卒,十六年,蔭鈺襲。

孟希文,字士煥,亞聖五十六代孫也。洪武元年詔以孟子五十四代孫思諒奉祀,世復其家。思諒,字友道,子克仁,字信夫。克仁子希文。景泰三年授希文翰林院《五經》博士,子孫世襲。卒,子元,字長伯,弘治二年襲。卒,子公棨幼,嘉靖二年以元弟亨之子公肇襲。公肇,字先文,少好學,事繼母孔氏,以孝聞。卒,十二年,仍以公棨襲。公棨,字橐文。卒,子彥璞,字朝璽,隆慶元年襲。卒,子承光,萬曆二十九年襲。卒,子弘譽,天啟三年襲。卒,子聞玉,崇禎二年襲。

仲于陛,先賢仲子六十二代孫也。萬曆十五年詔以仲子五十九代孫呂為奉祀。呂子銓。銓子則顯。則顯子於陛。崇禎十六年以衍聖公孔蔭植言,詔授于陛翰林院《五經》博士,子孫世襲,賜泗水縣、濟寧州田六十餘頃,廟戶三十一,以奉其祭祀焉。

周冕,先賢元公周子十二代孫也。其先,道州人,熙寧中,周子葬母江州,子孫因家廬山蓮花峰下。景泰七年,授冕翰林院《五經》博士,子孫世襲,還鄉以奉周子祀事。卒,子繡麟襲。卒,子道襲。卒,子聯芳襲。卒,子濟襲。卒,從弟汝忠襲。卒,子蓮應襲。

程接道,先賢正公程子後也。宋淳熙間,純公程子五世孫有居江寧者,嘗主金陵書院祀事。卒,以名幼學者承之。明初失傳。崇禎三年,河南巡按李日宣請以正公之後為之嗣,詔許之,遂以接道為翰林院《五經》博士,子孫世襲。十四年,土賊于大忠作亂,接道力拒,死之。

程克仁,先賢正公程子十七代孫也,世居嵩縣之六渾。景泰六年授翰林院《五經》博士,子孫世襲,以奉程子祀事。卒,子繼祖襲。卒,仲子世宥襲。卒,子心傳襲。心傳莊重寡言,為鄉黨所稱。卒,弟宗益襲。卒,從子佳引襲。卒,從弟佳祚襲。崇禎十四年為土賊于大忠所殺。

張文運,郿人,先賢明公張子十四代孫也。天啟二年授翰林院《五經》博士,子孫世襲,以奉張子祀事。崇禎三年卒,子承引,以父憂未襲。六年卒,子元祥,本朝康熙元年襲。

邵繼祖,洛陽人,先賢康節公邵子二十七代孫也。崇禎三年,河南巡按吳甡請以繼祖為翰林院《五經》博士,子孫世襲,以奉邵子祀事。詔從之。卒,子養醇襲。

朱梴,字孟齡,先賢文公朱子九世孫也,世居福建建安縣之紫霞洲。景泰六年授翰林院《五經》博士,子孫世襲,以奉朱子祀事。梴為人淳謹,言動有則。卒,子燉,字孔暉襲。燉以事入都,中途遇盜。未幾,有遺金道上者,燉守之,以還其人,人稱其廉介。卒,子塋,字元厚襲。卒,子鎏襲。卒,子法,字兆祖襲。法為人孝友。卒,子楗,字士啟襲。卒,子瑩,字惟玉襲。卒,子之俊,字喬之襲。

硃墅,先賢文公硃子十一世孫也。正德間,給事中戴銑、汪元錫,御史王完等相繼言:「硃子,繼孔子者也。孔子之後有曲阜、西安,硃子之後亦有建安、婺源。今建安恩典已隆,在婺源者,請依闕里之例,錄其子孫一人,量授以官,俾掌祠事。」詔許之。嘉靖二年授墅翰林院《五經》博士。三十八年以本學訓導席端言,令其世襲。墅卒,子鎬襲。卒,子德洪襲。卒,子邦相襲。卒,子煜襲。卒,子坤襲。


\end{pinyinscope}