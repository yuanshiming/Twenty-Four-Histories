\article{列傳第一百四十七 楊鎬李維翰周永春袁應泰薛國用熊廷弼王化貞袁崇煥毛文龍趙光抃範志完}

\begin{pinyinscope}
楊鎬李維翰周永春袁應泰薛國用熊廷弼王化貞袁崇煥毛文龍趙光抃範志完

楊鎬,商丘人。萬歷八年進士。歷知南昌、蠡二縣。入為御史,坐事調大理評事。再遷山東參議,分守遼海道。嘗偕大帥董一元雪夜度墨山,襲蒙古炒花帳,大獲。進副使。墾荒田百三十餘頃,歲積粟萬八千餘石。進參政。

二十五年春,偕副將李如梅出塞,失部將十人,士卒百六十餘人。會朝鮮再用兵,命免鎬罪,擢右僉都御史,經略朝鮮軍務。鎬未至,先奏陳十事,請令朝鮮官民輸粟得增秩、授官、贖罪,及鄉吏奴丁免役,大氐皆茍且之事。又以朝鮮君臣隱藏儲蓄不餉軍,劾奏其罪。由是朝鮮多怨。

當是時,倭將行長、清正等已入據南原、全州,引兵犯全羅、慶尚,逼王京,銳甚。賴沈惟敬就擒,鄉導乃絕。而朝鮮兵燹之餘,千里蕭條,賊掠無所得,故但積粟全羅,為久留計,而中國兵亦漸集。九月朔,鎬始抵王京。會副將解生等屢挫賊,朝鮮軍亦數有功,倭乃退屯蔚山。十二月,鎬會總督邢玠、提督麻貴議進兵方略,分四萬人為三協,副將高策將中軍,李如梅將左,李芳春、解生將右,合攻蔚山。先以少兵嘗賊,賊出戰,大敗,悉奔據島山,結三柵城外以自固。鎬官遼東時,與如梅深相得。及是,游擊陳寅連破賊二柵,第三柵垂拔矣,鎬以如梅未至,不欲寅功出其上,遽鳴金收軍。賊乃閉城不出,堅守以待援。官兵四面圍之,地泥淖,且時際窮冬,風雪裂膚,士無固志。賊日夜發砲,用藥煮彈,遇者輒死,官兵攻圍十日不能下。賊知官兵懈,詭乞降以緩之。明年正月二日,行長救兵驟至。鎬大懼,狼狽先奔,諸軍繼之。賊前襲擊,死者無算。副將吳惟忠、游擊茅國器斷後,賊乃還,輜重多喪失。

是役也,謀之經年,傾海內全力,合朝鮮通國之眾,委棄於一旦,舉朝嗟恨。鎬既奔,挈貴奔趨慶州,懼賊乘襲,盡撤兵還王京,與總督玠詭以捷聞。諸營上軍籍,士卒死亡殆二萬,鎬大怒,屏不奏,止稱百餘人。鎬遭父喪,詔奪情視事。御史汪先岸嘗劾其他罪,閣臣庇之,擬旨褒美,旨久不下。贊畫主事丁應泰聞鎬敗,詣鎬咨後計。鎬示以張位、沈一貫手書,並所擬未下旨,揚揚詡功伐。應泰憤,抗疏盡列敗狀,言鎬當罪者二十八、可羞者十,並劾位、一貫扶同作奸。帝震怒,欲行法。首輔趙志皋營救,乃罷鎬,令聽勘,以天津巡撫萬世德代之。已,東征事竣,給事中楊應文敘鎬功,詔許復用。

三十八年,起撫遼東。襲炒花於鎮安,破之,御史田生金劾其開釁。時遼左多事,鎬力薦李如梅,請復用為大將,為給事中麻僖、御史楊州鶴所劾。鎬疏辨乞休,帝不問,鎬竟引去。

四十六年四月,我大清兵起,破撫順,守將王命印死之。遼東巡撫李維翰趣總兵官張承允往援,與副總兵頗廷相等俱戰歿,遠近大震。廷議鎬熟諳遼事,起兵部右侍郎往經略。既至,申明紀律,征四方兵,圖大舉。至七月,大清兵由鴉鶻關克清河,副將鄒儲賢戰死。詔賜鎬尚方劍,得斬總兵以下官,乃斬清河逃將陳大道、高炫徇軍中。其冬,四方援兵大集,遂議進師。時蚩尤旗長竟天,彗見東方,星隕地震,識者以為敗徵。大學士方從哲、兵部尚書黃嘉善、兵科給事中趙興邦等皆以師久餉匱,發紅旗,日趣鎬進兵。

明年正月,鎬乃會總督汪可受、巡撫周永春、巡按陳王庭等定議,以二月十有一日誓師,二十一日出塞。兵分四道:總兵官馬林出開原攻北,杜松出撫順攻西,李如柏從鴉鶻關出趨清河攻南,東南則以劉綎出寬奠,由涼馬佃搗後,而以朝鮮兵助之。號大兵四十七萬,期三月二日會二道關並進。天大雪,兵不前,師期洩。松欲立首功,先期渡渾河,進至二道關,伏發,軍盡覆。林統開原兵從三岔口出,聞松敗,結營自固。大清兵乘高奮擊,林不支,遂大敗,遁去。鎬聞,急檄止如柏、綎兩軍,如柏遂不進。綎已深入三百里,至深河,大清兵擊之而不動。已,乃張松旗幟,被其衣甲,紿綎。既入營,營中大亂,綎力戰死。惟如柏軍獲全。文武將吏前後死者三百一十餘人,軍士四萬五千八百餘人,亡失馬駝甲仗無算。敗書聞,京師大震。御史楊鶴疏劾之,不報。無何,開原、鐵嶺又相繼失。言官交章劾鎬,逮下詔獄,論死。崇禎二年伏法。

李維翰,睢州人。萬歷四十四年,以右副都御史巡撫遼東。遼三面受敵,無歲不用兵,自稅使高淮朘削十餘年,軍民益困。而先后撫臣皆庸才,玩心妻茍歲月。天子又置萬幾不理,邊臣呼籲,漠然不聞,致遼事大壞。及張承允覆沒,維翰猶獲善歸。至天啟初,始下吏論死。

周永春,金鄉人。官禮科都給事中。齊黨方熾,永春與亓詩教為之魁。尋由太常少卿擢右僉都御史,代維翰為巡撫。值喪敗之後,佐經略調度軍食,拮據勞瘁。越二年,罷歸。天啟初,言官追論開原失陷罪,遣戍。

袁應泰,字大來,鳳翔人。萬歷二十三年進士。授臨漳知縣。築長堤四十餘里,捍禦漳水。調繁河內,穿太行山,引沁水,成二十五堰,溉田數萬頃,鄰邑皆享其利。河決硃旺,役夫多死者。應泰設席為廬,飲食作止有度,民歡然趨事,治行冠兩河。

遷工部主事,歷兵部武選郎中。汰遣假冒世職數百人。遷淮徐兵備參議。山東大饑,設粥廠哺流民,繕城浚濠,修先聖廟,饑者盡得食。更搜額外稅及漕折馬價數萬金,先後發振。戶部劾其擅移官廩,時已遷副使,遂移疾歸。

久之,起河南右參政,以按察使治兵永平。遼事方棘,應泰練兵繕甲,修亭障,飭樓櫓,關外所需芻茭、火藥之屬呼吸立應。經略熊廷弼深賴焉。

泰昌元年九月,擢右僉都御史,代周永春巡撫遼東。逾月,擢兵部右侍郎兼前職,代廷弼為經略,而以薛國用為巡撫。應泰受事,即刑白馬祀神,誓以身委遼。疏言:「臣願與遼相終始,更願文武諸臣無懷二心,與臣相終始。有托故謝事者,罪無赦。」熹宗優詔褒答,賜尚方劍。乃戮貪將何光先,汰大將李光榮以下十餘人,遂謀進取撫順。議用兵十八萬,大將十人,上奏陳方略。

應泰歷官精敏強毅,用兵非所長,規畫頗疏。廷弼在邊,持法嚴,部伍整肅,應泰以寬矯之,多所更易。而是時蒙古諸部大饑,多入塞乞食。應泰言:「我不急救,則彼必歸敵,是益之兵也。」乃下令招降。於是歸者日眾,處之遼、沈二城,優其月廩,與民雜居,潛行淫掠,居民苦之。議者言收降過多,或陰為敵用,或敵雜間諜其中為內應,禍且叵測。應泰方自詡得計,將藉以抗大清兵。會三岔兒之戰,降人為前鋒,陣死者二十餘人,應泰遂用以釋群議。

明年,天啟改元,三月十有二日,我大清兵來攻沈陽。總兵官賀世賢、尤世功出城力戰,敗還。明日,降人果內應,城遂破,二將戰死。總兵官陳策、童仲揆等赴援,亦戰死。應泰乃撤奉集、威寧諸軍,並力守遼陽,引水注濠,沿濠列火器,兵環四面守。十有九日,大清兵臨城。應泰身督總兵官侯世祿、李秉誠、梁仲善、姜弼、硃萬良出城五里迎戰,軍敗多死。其夕,應泰宿營中,不入城。明日,大清兵掘城西閘以洩濠水,分兵塞城東水口,擊敗諸將兵,遂渡濠,大呼而進。鏖戰良久,騎來者益眾,諸將兵俱敗,望城奔,殺溺死者無算。應泰乃入城,與巡按御史張銓等分陴固守。諸監司高出、牛維曜、胡嘉棟及督餉郎中傅國並逾城遁,人心離沮。又明日,攻城急,應泰督諸軍列楯大戰,又敗。薄暮,譙樓火,大清兵從小西門入,城中大亂,民家多啟扉張炬以待,婦女示盛飾迎門,或言降人導之也。應泰居城樓,知事不濟,太息謂銓曰:「公無守城責,宜急去,吾死於此。」遂佩劍印自縊死。婦弟姚居秀從之。僕唐世明憑尸大慟,縱火焚樓死。事聞,贈兵部尚書,予祭葬,官其一子。

國用,洛南人。歷官山東右參政,分守遼海道,以右僉都御史代應泰巡撫遼東。應泰死,廷議將起廷弼,道遠未至,乃進國用兵部右侍郎,代應泰為經略。歷官醇謹,久於遼,日夜憂戰守備。會大清兵不至,得安其位。無何請告,竟卒於官。

熊廷弼,字飛百,江夏人。萬歷二十五年舉鄉試第一。明年成進士,授保定推官,擢御史。

三十六年,巡按遼東。巡撫趙楫與總兵官李成梁棄寬奠新疆八百里,徙編民六萬家於內地。已,論功受賞,給事中宋一韓論之。下廷弼覆勘,具得棄地驅民狀,劾兩人罪,及先任按臣何爾健、康丕揚黨庇。疏竟不下。時有詔興屯,廷弼言遼多曠土,歲於額軍八萬中以三分屯種,可得粟百三十萬石。帝優詔褒美,命推行於諸邊。邊將好搗巢,輒生釁端。廷弼言防邊以守為上,繕垣建堡,有十五利,奏行之。歲大旱,廷弼行部金州,禱城隍神,約七日雨,不雨毀其廟。及至廣寧,踰三日,大書白牌,封劍,使使往斬之。未至,風雷大作,雨如注,遼人以為神。在遼數年,杜餽遺,核軍實,按劾將吏,不事姑息,風紀大振。

督學南畿,嚴明有聲。以杖死諸生事,與巡按御史荊養喬相訐奏。養喬投劾去,廷弼亦聽勘歸。

四十七年,楊鎬既喪師,廷議以廷弼熟邊事,起大理寺丞兼河南道御史,宣慰遼東。旋擢兵部右侍郎兼右僉都御史,代鎬經略。未出京,開原失,廷弼上言:「遼左,京師肩背;河東,遼鎮腹心;開原又河東根本。欲保遼東則開原必不可棄。敵未破開原時,北關、朝鮮猶足為腹背患。今已破開原,北關不敢不服,遣一介使,朝鮮不敢不從。既無腹背憂,必合東西之勢以交攻,然則遼、沈何可守也?乞速遣將士,備芻糧,修器械,毋窘臣用,毋緩臣期,毋中格以沮臣氣,毋旁撓以掣臣肘,毋獨遺臣以艱危,以致誤臣、誤遼,兼誤國也。」疏入,悉報允,且賜尚方劍重其權。甫出關,鐵嶺復失,沈陽及諸城堡軍民一時盡竄,遼陽洶洶。廷弼兼程進,遇逃者,諭令歸。斬逃將劉遇節、王捷、王文鼎,以祭死節士。誅貪將陳倫,劾罷總兵官李如楨,以李懷信代。督軍士造戰車,治火器,浚濠繕城,為守禦計。令嚴法行,數月守備大固。乃上方略,請集兵十八萬,分布靉陽、清河、撫順、柴河、三岔兒、鎮江諸要口,首尾相應,小警自為堵禦,大敵互為應援。更挑精悍者為遊徼,乘間掠零騎,擾耕牧,更番迭出,使敵疲於奔命,然後相機進剿。疏入,帝從之。

廷弼之初抵遼也,令僉事韓原善往撫沈陽,憚不肯行。繼命僉事閻鳴泰,至虎皮驛慟哭而返。廷弼乃躬自巡歷,自虎皮驛抵沈陽,復乘雪夜赴撫順。總兵賀世賢以近敵沮之,廷弼曰:「冰雪滿地,敵不料我來。」鼓吹入。時兵燹後,數百里無人跡,廷弼祭諸死事者而哭之。遂耀兵奉集,相度形勢而還,所至招流移,繕守具,分置士馬,由是人心復固。

廷弼身長七尺,有膽知兵,善左右射。自按遼即持守邊議,至是主守禦益堅。然性剛負氣,好謾罵,不為人下,物情以故不甚附。

明年五月,我大清兵略地花嶺。六月,略王大人屯。八月,略蒲河。將士失亡七百餘人,諸將世賢等亦有斬獲功。而給事中姚宗文騰謗於朝,廷弼遂不安其位。宗文者,故戶科給事中,丁憂歸。還朝,欲補官,而吏部題請諸疏率數年不下,宗文患之。假招徠西部名,屬當事薦己。疏屢上,不得命。宗文計窮,致書廷弼,令代請。廷弼不從,宗文由是怨。後夤緣復吏科,閱視遼東士馬,與廷弼議多不合。遼東人劉國縉先為御史,坐大計謫官。遼事起,廷議用遼人,遂以兵部主事贊畫軍務。國縉主募遼人為兵,所募萬七千餘人,逃亡過半。廷弼聞於朝,國縉亦怨。廷弼為御史時,與國縉、宗文同在言路,意氣相得,並以排東林、攻道學為事。國縉輩以故意望廷弼,廷弼不能如前,益相失。宗文故出國縉門下,兩人益相比,而傾廷弼。及宗文歸,疏陳遼土日蹙,詆廷弼廢群策而雄獨智,且曰:「軍馬不訓練,將領不部署,人心不親附,刑威有時窮,工作無時止。」復鼓其同類攻擊,欲必去之。御史顧慥首劾廷弼出關踰年,漫無定畫;蒲河失守,匿不上聞;荷戈之士徒供挑濬,尚方之劍逞志作威。

當是時,光宗崩,熹宗初立,朝端方多事,而封疆議起。御史馮三元劾廷弼無謀者八、欺君者三,謂不罷,遼必不保。詔下廷議。廷弼憤,抗疏極辨,且求罷。而御史張修德復劾其破壞遼陽。廷弼益憤,再疏自明,云「遼已轉危為安,臣且之生致死。」遂繳還尚方劍,力求罷斥。給事中魏應嘉復劾之。朝議允廷弼去,以袁應泰代。廷弼乃上疏求勘,言:「遼師覆沒,臣始驅羸卒數千,踉蹌出關,至杏山,而鐵嶺又失。廷臣咸謂遼必亡,而今且地方安堵,舉朝帖席。此非不操練、不部署者所能致也。若謂擁兵十萬,不能斬將擒王,誠臣之罪。然求此於今日,亦豈易言。令箭催而張帥殞命,馬上催而三路喪師,臣何敢復蹈前軌?」三元、應嘉、修德等復連章極論,廷弼即請三人往勘。帝從之。御史吳應奇、給事中楊漣等力言不可,乃改命兵科給事中朱童蒙往。廷弼復上疏曰:「臣蒙恩回籍聽勘,行矣。但臺省責臣以破壞之遼遺他人,臣不得不一一陳之於上。今朝堂議論,全不知兵。冬春之際,敵以冰雪稍緩,哄然言師老財匱,馬上促戰。及軍敗,始愀然不敢復言,比臣收拾甫定,而愀然者又復哄然責戰矣。自有遼難以來,用武將,用文吏,何非臺省所建白,何嘗有一效。疆場事,當聽疆場吏自為之,何用拾帖括語,徒亂人意,一不從,輒怫然怒哉!」及童蒙還奏,備陳廷弼功狀,末言:「臣入遼時,士民垂泣而道,謂數十萬生靈皆廷弼一人所留,其罪何可輕議?獨是廷弼受知最深,蒲河之役,敵攻沈陽,策馬趨救,何其壯也?及見官兵駑弱,遽爾乞骸以歸,將置君恩何地?廷弼功在存遼,微勞雖有可紀;罪在負君,大義實無所逃。此則罪浮於功者矣。」帝以廷弼力保危城,仍議起用。

天啟元年,沈陽破,應泰死,廷臣復思廷弼。給事中郭鞏力詆之,並及閣臣劉一燝。及遼陽破,河西軍民盡奔,自塔山至閭陽二百餘里,煙火斷絕,京師大震。一燝曰:「使廷弼在遼,當不至此。」御史江秉謙追言廷弼保守危遼功,兼以排擠勞臣為鞏罪。帝乃治前劾廷弼者,貶三元、修德、應嘉、鞏三秩,除宗文名。御史劉廷宣救之,亦被斥。乃復詔起廷弼於家,而擢王化貞為巡撫。

化貞,諸城人。萬歷四十一年進士。由戶部主事歷右參議,分守廣寧。蒙古炒花諸部長乘機窺塞下,化貞撫之,皆不敢動。朱童蒙勘事還,極言化貞得西人心,勿輕調,隳撫事。化貞亦言遼事將壞,惟發帑金百萬,亟款西人,則敵顧忌不敢深入。會遼、沈相繼亡,廷議將起廷弼,御史方震孺請加化貞秩,便宜從事,令與薛國用同守河西。乃進化貞右僉都御史,巡撫廣寧。廣寧城在山隈,登山可俯瞰城內,恃三岔河為阻,而三岔之黃泥窪又水淺可涉。廣寧止孱卒千,化貞招集散亡,復得萬餘人,激厲士民,聯絡西部,人心稍定。遼陽初失,遠近震驚,謂河西必不能保。化貞提弱卒,守孤城,氣不懾,時望赫然。中朝亦謂其才足倚,悉以河西事付之。而化貞又以登萊、天津兵可不設,諸鎮入衛兵可止。當事益信其有才,所奏請輒報可。時金、復諸衛軍民及東山礦徒,多結砦自固,以待官軍,其逃入朝鮮者,亦不下二萬。化貞請鼓舞諸人,優以爵祿,俾自奮於功名,詔諭朝鮮,褒以忠義,勉之同仇。帝亦從之。

至六月,廷弼入朝,首請免言官貶謫,帝不可。乃建三方布置策:廣寧用馬步列壘河上,以形勢格之,綴敵全力;天津、登、萊各置舟師,乘虛入南衛,動搖其人心,敵必內顧,而遼陽可復。於是登、萊議設巡撫如天津,以陶朗先為之;而山海特設經略,節制三方,一事權。遂進廷弼兵部尚書,兼右副都御史,駐山海關,經略遼東軍務。廷弼因請尚方劍,請調兵二十餘萬,以兵馬、芻糗、器械之屬責成戶、兵、工三部。白監軍道臣高出、胡嘉棟,督餉郎中傅國無罪,請復官任事。議用遼人故贊畫主事劉國縉為登萊招練副使,夔州同知佟卜年為登萊監軍僉事,故臨洮推官洪敷教為職方主事,軍前贊畫,用收拾遼人心,並報允。七月,廷弼將啟行,帝特賜麒麟服一,彩幣四,宴之郊外,命文武大臣陪餞,異數也。又以京營選鋒五千護廷弼行。

先是,袁應泰死,薛國用代為經略,病不任事。化貞乃部署諸將,沿河設六營,營置參將一人,守備二人,畫地分守;西平、鎮武、柳河、盤山諸要害,各置戍設防。議即上,廷弼不謂然,疏言:「河窄難恃,堡小難容,今日但宜固守廣寧。若駐兵河上,兵分則力弱,敵輕騎潛渡,直攻一營,力必不支。一營潰,則諸營俱潰,西平諸戍亦不能守。河上止宜置遊徼兵,更番出入,示敵不測,不宜屯聚一處,為敵所乘。自河抵廣寧,止宜多置烽堠;西平諸處止宜稍置戍兵,為傳烽哨探之用。而大兵悉聚廣寧,相度城外形勢,掎角立營,深壘高柵以俟。蓋遼陽去廣寧三百六十里,非敵騎一日能到,有聲息,我必預知。斷不宜分兵防河,先為自弱之計也。」疏上,優旨褒答。會御史方震孺亦言防河六不足恃,議乃寢。而化貞以計不行,慍甚,盡委軍事於廷弼。廷弼乃請申諭化貞,不得藉口節制,坐失事機。先是,四方援遼之師,化貞悉改為「平遼」,遼人多不悅。廷弼言:「遼人未叛,乞改為『平東』或『征東』,以慰其心。」自是化貞與廷弼有隙,而經、撫不和之議起矣。

八月朔,廷弼言:「三方建置,須聯絡朝鮮。請亟發敕使往勞彼國君臣,俾盡發八道之師,連營江上,助我聲勢。又發詔書憫恤遼人之避難彼國者,招集團練,別為一軍,與朝鮮軍合勢。而我使臣即權駐義州,控制聯絡,俾與登、萊聲息相通,於事有濟。更宜發銀六萬兩,分犒朝鮮及遼人,而臣給與空名札付百道,俾承制拜除。其東山礦徒能結聚千人者,即署都司;五百人者,署守備。將一呼立應,而一二萬勁兵可立致也。」因薦監軍副使梁之垣生長海濱,習朝鮮事,可充命使。帝立從之,且命如行人奉使故事,賜一品服以寵其行。之垣乃列上重事權、定職掌八事,帝亦報可。

之垣方與所司議兵餉,而化貞所遣都司毛文龍已襲取鎮江,奏捷。舉朝大喜,亟命登、萊、天津發水師二萬應文龍,化貞督廣寧兵四萬進據河上,合蒙古軍乘機進取,而廷弼居中節制。命既下,經、撫、各鎮互觀望,兵不果進。頃之,化貞備陳東西情形,言:「敵棄遼陽不守,河東失陷將士日夜望官軍至,即執敵將以降。而西部虎墩兔、炒花咸願助兵。敵兵守海州不過二千,河上止遼卒三千,若潛師夜襲,勢在必克。敵南防者聞而北歸,我據險以擊其惰,可盡也。」兵部尚書張鶴鳴以為然,奏言時不可失。御史徐卿伯復趣之,請令廷弼進駐廣寧,薊遼總督王象乾移鎮山海。會化貞復馳奏:「敵因官軍收復鎮江,遂驅掠四衛屯民。屯民據鐵山死守,傷敵三四千人,敵圍之益急。急宜赴救。」於是兵部愈促進師。化貞即以是月渡河。廷弼不得已出關,次右屯,而馳奏海州取易守難,不宜輕舉。化貞卒無功而還。

化貞為人騃而愎,素不習兵,輕視大敵,好謾語。文武將吏進諫悉不入,與廷弼尤牴牾。妄意降敵者李永芳為內應,信西部言,謂虎墩兔助兵四十萬,遂欲以不戰取全勝。一切士馬、甲仗、糗糧、營壘俱置不問,務為大言罔中朝。尚書鶴鳴深信之,所請無不允,以故廷弼不得行其志。廣寧有兵十四萬,而廷弼關上無一卒,徒擁經略虛號而已。延綏入衛兵不堪用,廷弼請罪其帥杜文煥,鶴鳴議寬之;廷弼請用卜年,鶴鳴上駁議;廷弼奏遣之垣,鶴鳴故稽其餉。兩人遂相怨,事事齟齬。而廷弼亦褊淺剛愎,有觸必發,盛氣相加,朝士多厭惡之。

毛文龍鎮江之捷,化貞自謂發蹤奇功。廷弼言:「三方兵力未集,文龍發之太早,致敵恨遼人,屠戮四衛軍民殆盡,灰東山之心,寒朝鮮之膽,奪河西之氣,亂三方並進之謀,誤屬國聯絡之算,目為奇功,乃奇禍耳!」貽書京師,力詆化貞。朝士方以鎮江為奇捷,聞其言,亦多不服。廷弼又顯詆鶴鳴,謂:「臣既任經略,四方援軍宜聽臣調遣,乃鶴鳴徑自發戍,不令臣知。七月中,臣咨部問調軍之數,經今兩月,置不答。臣有經略名,無其實,遼左事惟樞臣與撫臣共為之。」鶴鳴益恨。至九月,化貞猶言虎墩兔兵四十萬且至,請速濟師。廷弼言:「撫臣恃西部,欲以不戰為戰計。西部與我,進不同進,彼入北道,我入南道,相距二百餘里,敵分兵來應,亦須我自撐拒。臣未敢輕視敵人,謂可不戰勝也。臣初議三方布置,必使兵馬、器械、舟車、芻茭無一不備,而後剋期齊舉,進足戰,退亦足以守。今臨事中亂,雖樞臣主謀於中,撫臣決策於外,卜一舉成功,而臣猶有萬一不必然之慮也。」既而西部竟不至,化貞兵亦不敢進。

廷弼既與化貞隙,中朝右化貞者多詆廷弼。給事中楊道寅謂出、嘉棟不宜用。御史徐景濂極譽化貞,刺廷弼,詆之垣逍遙故鄉,不稱任使。御史蘇琰則言廷弼宜駐廣寧,不當遠駐山海,因言登、萊水師無所用。廷弼怒,抗疏力詆三人。帝皆無所問。而帝於講筵忽問:「卜年係叛族,何擢僉事?國縉數經論列,何起用?嘉棟立功贖罪,何在天津?」廷弼知左右譖之,抗疏辨,語頗憤激。

是時,廷弼主守,謂遼人不可用,西部不可恃,永芳不可信,廣寧多間諜可虞。化貞一切反之,絕口不言守,謂我一渡河,河東人必內應,且騰書中朝,言仲秋之月,可高枕而聽捷音。識者知其必僨事,以疆場事重,無敢言其短者。

至十月,冰合,廣寧人謂大清兵必渡河,紛然思竄。化貞乃與震孺計,分兵守鎮武、西平、閭陽、鎮寧諸城堡,而以大軍守廣寧。鶴鳴亦以廣寧可慮,請敕廷弼出關。廷弼上言:「樞臣第知經略一出,足鎮人心;不知徒手之經略一出,其動搖人心更甚。且臣駐廣寧,化貞駐何地?鶴鳴責經、撫協心同力,而樞臣與經臣獨不當協心同力乎?為今日計,惟樞部俯同於臣,臣始得為陛下任東方事也。」其言甚切至,鶴鳴益不悅。廷弼乃復出關,至右屯,議以重兵內護廣寧,外扼鎮武、閭陽,乃令劉渠以二萬人守鎮武,祁秉忠以萬人守閭陽。又令羅一貫以三千人守西平。復申令曰:「敵來,越鎮武一步者,文武將吏誅無赦。敵至廣寧而鎮武、閭陽不夾攻,掠右屯餉道而三路不救援者,亦如之。」部署甫定,化貞又信諜者言,遽發兵襲海州,旋亦引退。廷弼乃上言:「撫臣之進,及今而五矣。八、九月間屢進屢止,猶未有疏請也。若十月二十五日之役,則拜疏輒行者也,臣疾趨出關,而撫臣歸矣。西平之會,相與協心議守,掎角設營,而進兵之書又以晦日至矣。撫臣以十一月二日赴鎮武,臣即以次日赴杜家屯,比至中途,而軍馬又遣還矣。初五日,撫臣又欲以輕兵襲牛莊,奪馬圈守之,為明年進兵門戶。時馬圈無一敵兵,即得牛莊,我不能守,敵何損,我何益?會將吏力持不可,撫臣亦怏怏回矣。兵屢進屢退,敵已窺盡伎倆,而臣之虛名亦以輕出而損。願陛下明諭撫臣,慎重舉止,毋為敵人所笑。」化貞見疏不悅,馳奏辨。且曰:「願請兵六萬,一舉蕩平。臣不敢貪天功,但厚賚從征將士,遼民賜復十年,海內得免加派,臣願足矣。即有不稱,亦必殺傷相當,敵不復振,保不為河西憂。」因請便宜行事。

時葉向高復當國,化貞座主也,頗右之。廷臣惟太僕少卿何喬遠言宜專守廣寧,御史夏之令言蒙古不可信,款賞無益,給事中趙時用言永芳必不可信,與廷弼合。餘多右化貞,令毋受廷弼節制。而給事中李精白欲授化貞尚方劍,得便宜操縱。孫傑劾一燝以用出、嘉棟、卜年為罪,而言廷弼不宜駐關內。廷弼憤,上言:「臣以東西南北所欲殺之人,而適遘事機難處之會。諸臣能為封疆容則容之,不能為門戶容則去之,何必內借閣部,外借撫道以相困?」又言:「經、撫不和,恃有言官;言官交攻,恃有樞部;樞部佐鬥,恃有閣臣。臣今無望矣。」帝以兩臣爭言,遣兵部堂官及給事中各一人往諭,抗違不遵者治罪。命既下,廷臣言遣官不便,乃下廷臣集議。

初,廷弼之出關也,化貞慮奪己兵權,佯以兵事委廷弼。廷弼上言:「臣奉命控扼山海,非廣寧所得私。撫臣不宜卸責於臣。」會震孺奏經、撫不和,中有化貞心慵意懶語,廷弼據以刺化貞,化貞益不悅。及化貞請一舉蕩平,廷弼乃言:「宜如撫臣約,亟罷臣以鼓士氣。」當是時,中外舉知經、撫不和,必誤疆事,章日上。而鶴鳴篤信化貞,遂欲去廷弼。二年正月,員外郎徐大化希指劾廷弼大言罩世,嫉能妒功,不去必壞遼事。疏並下部,鶴鳴乃集廷臣大議。議撤廷弼者數人,餘多請分任責成。鶴鳴獨言化貞一去,毛文龍必不用命,遼人為兵者必潰,西部必解體,宜賜化貞尚方劍,專委以廣寧,而撤廷弼他用。議上,帝不從,責吏、兵二部再奏。會大清兵逼西平,遂罷議,仍兼任二臣,責以功罪一體。

無何,西平圍急。化貞信中軍孫得功計,盡發廣寧兵,畀得功及祖大壽往會秉忠進戰。廷弼亦馳檄渠撤營赴援。二十二日,遇大清兵平陽橋。鋒始交,得功及參將鮑承先等先奔,鎮武、閭陽兵遂大潰,渠、秉忠戰沒沙嶺,大壽走覺華島。西平守將一貫待援不至,與參將黑雲鶴亦戰歿。廷弼已離右屯,次閭陽。參議邢慎言勸急救廣寧,為僉事韓初命所沮,遂退還。時大清兵頓沙嶺不進。化貞素任得功為腹心,而得功潛降於大清,欲生縛化貞以為功,訛言敵已薄城。城中大亂奔走,參政高邦佐禁之不能止。化貞方闔署理軍書,不知也。參將江朝棟排闥入,化貞怒呵之,朝棟大呼曰:「事急矣,請公速走。」化貞莫知所為。朝棟掖之出上馬,二僕人徒步從,遂棄廣寧,踉蹌走,與廷弼遇大凌河。化貞哭,廷弼微笑曰:「六萬眾一舉蕩平,竟何如?」化貞慚,議守寧遠及前屯。廷弼曰:「嘻,已晚,惟護潰民入關可耳。」乃以己所將五千人授化貞為殿,盡焚積聚。二十六日,偕初命護潰民入關。化貞、出、嘉棟先後入,獨邦佐自經死。得功率廣寧叛將迎大清兵入廣寧,化貞逃已兩日矣。大清兵追逐化貞等二百里,不得食,乃還。報至,京師大震,鶴鳴恐,自請視師。

二月逮化貞,罷廷弼聽勘。四月,刑部尚書王紀、左都御史鄒元標、大理寺卿周應秋等奏上獄詞,廷弼、化貞並論死。後當行刑,廷弼令汪文言賄內廷四萬金祈緩,既而背之。魏忠賢大恨,誓速斬廷弼。及楊漣等下獄,誣以受廷弼賄,甚其罪。已,邏者獲市人蔣應暘,謂與廷弼子出入禁獄,陰謀叵測。忠賢愈欲速殺廷弼,其黨門克新、郭興治、石三畏、卓邁等遂希指趣之。會馮銓亦憾廷弼,與顧秉謙等侍講筵,出市刊《遼東傳》譖於帝曰:「此廷弼所作,希脫罪耳。」帝怒,遂以五年八月棄市,傳首九邊。已,御史梁夢環謂廷弼侵盜軍資十七萬。御史劉徽謂廷弼家資百萬,宜籍以佐軍。忠賢即矯旨嚴追,罄貲不足,姻族家俱破。江夏知縣王爾玉責廷弼子貂裘珍玩,不獲,將撻之。其長子兆珪自剄死,兆珪母稱冤。爾玉去其兩婢衣,撻之四十。遠近莫不嗟憤。

崇禎元年,詔免追贓。其秋,工部主事徐爾一訟廷弼冤,曰:

廷弼以失陷封疆,至傳首陳屍,籍產追贓。而臣考當年,第覺其罪無足據,而勞有足矜也。廣寧兵十三萬,糧數百萬,盡屬化貞,廷弼止援遼兵五千人,駐右屯,距廣寧四十里耳。化貞忽同三四百萬遼民一時盡潰,廷弼五千人,不同潰足矣,尚望其屹然堅壁哉!廷弼罪安在?化貞仗西部,廷弼云「必不足仗」;化貞信李永芳內附,廷弼云「必不足信」。無一事不力爭,無一言不奇中。廷弼罪安在?且屢疏爭各鎮節制不行,屢疏爭原派兵馬不與。徒擁虛器,抱空名,廷弼罪安在?唐郭子儀、李光弼與九節度師同潰,自應收潰兵扼河陽橋,無再往河陽坐待思明縛去之理。今計廣寧西,止關上一門限,不趣扼關門何待?史稱慕容垂一軍三萬獨全,亦無再駐淝水與晉人決戰之理。廷弼能令五千人不散,至大凌河付與化貞,事政相類,寧得與化貞同日道乎!所謂勞有足矜者:當三路同時陷沒,開、鐵、北關相繼奔潰,廷弼經理不及一年,俄進築奉集、沈陽,俄進屯虎皮驛,俄迎扼敵兵於橫河上,於遼陽城下鑿河列柵埋炮,屹然樹金湯。令得竟所施,何至舉榆口關外拱手授人!而今俱抹摋不論,乃其所由必死則有故矣。其才既籠蓋一時,其氣又陵厲一世,揭辯紛紛,致攖眾怒,共起殺機,是則所由必殺其軀之道耳。當廷弼被勘被逮之時,天日輒為無光,足明其冤。乞賜昭雪,為勞臣勸。

不從。明年五月,大學士韓爌等言:

廷弼遺骸至今不得歸葬,從來國法所未有。今其子疏請歸葬,臣等擬票許之。蓋國典皇仁,並行不悖,理合如此。若廷弼罪狀始末,亦有可言。皇祖朝,戊申己酉間,廷弼以御史按遼東,早以遼患為慮,請核地界,飭營伍,聯絡南、北關,大聲疾呼,人莫為應。十年而驗若左券,其可言者一。戊午己未,楊鎬三路喪師,撫順、清河陷沒,皇祖用楊鶴言,召起廷弼代鎬。一年餘,修飭守具,邊患稍寧。會皇祖賓天,廷議以廷弼無戰功,攻使去,使袁應泰代,四閱月而遼亡。使廷弼在,未必至此,其可言者二。遼陽既失,先帝思廷弼言,再起之田間,復任經略。化貞主戰,廷弼主守,群議皆是化貞。廷弼屢言玩師必敗,奸細當防,莫有聽者,徘徊躑躅,以五千人駐右屯。化貞兵十三萬駐廣寧。廣寧潰,右屯乃與俱潰,其可言者三。

假令廷弼於此時死守右屯,捐軀殉封疆,豈非節烈奇男子。不然,支撐寧、前、錦、義間,扶傷救敗,收拾殘黎,猶可圖桑榆之效。乃倉皇風鶴,偕化貞並馬入關,其意以我固嘗言之,言而不聽,罪當末減。此則私心短見,殺身以此,殺身而無辭公論,亦以此。傳首邊庭,頭足異處,亦足為臨難鮮忠者之戒矣。然使誅廷弼者,按封疆失陷之條,偕同事諸臣,一體伏法,廷弼九原目瞑。乃先以賄贓拷坐楊漣、魏大中等,作清流陷阱;既而刊書惑眾,借題曲殺。身死尚懸坐贓十七萬,辱及妻孥,長子兆珪迫極自刎。斯則廷弼死未心服,海內忠臣義士亦多憤惋竊歎者。特以「封疆」二字,噤不敢訟陳皇上之前。

臣等平心論之,自有遼事以來,誆官營私者何算,廷弼不取一金錢,不通一饋問,焦脣敝舌,爭言大計。魏忠賢盜竊威福,士大夫靡然從風。廷弼以長繫待決之人,屈曲則生,抗違則死,乃終不改其強直自遂之性,致獨膺顯戮,慷慨赴市,耿耿剛腸猶未盡泯。今縱不敢深言,而傳首已踰三年,收葬原無禁例,聖明必當垂仁。臣所以娓娓及此者,以茲事雖屬封疆,而實陰繫朝中邪正本末。皇上天縱英哲,或不以臣等為大謬也。

詔許其子持首歸葬。五年,化貞始伏誅。

袁崇煥,字元素,東莞人。萬歷四十七年進士。授邵武知縣。為人慷慨負膽略,好談兵。遇老校退卒,輒與論塞上事,曉其阨塞情形,以邊才自許。

天啟二年正月,朝覲在都,御史侯恂請破格用之,遂擢兵部職方主事。無何,廣寧師潰,廷議扼山海關,崇煥即單騎出閱關內外。部中失袁主事,訝之,家人亦莫知所往。已,還朝,具言關上形勢,曰:「予我軍馬錢穀,我一人足守此。」廷臣益稱其才,遂超擢僉事,監關外軍,發帑金二十萬,俾招募。時關外地悉為哈剌慎諸部所據,崇煥乃駐守關內。未幾,諸部受款,經略王在晉令崇煥移駐中前所,監參將周守廉、遊擊左輔軍,經理前屯衛事。尋令赴前屯安置遼人之失業者,崇煥即夜行荊棘虎豹中,以四鼓入城,將士莫不壯其膽。在晉深倚重之,題為寧前兵備僉事,然崇煥薄在晉無遠略,不盡遵其令。及在晉議築重城八里鋪,崇煥以為非策,爭不得,奏記首輔葉向高。

十三山難民十餘萬,久困不能出。大學士孫承宗行邊,崇煥請:「將五千人駐寧遠,以壯十三山勢,別遣驍將救之。寧遠去山二百里,便則進據錦州,否則退守寧遠,奈何委十萬人置度外?」承宗謀於總督王象乾。象乾以關上軍方喪氣,議發插部護關者三千人往,承宗以為然,告在晉。在晉竟不能救,眾遂沒,脫歸者僅六千人而已。及承宗駁重城議,集將吏謀所守。閻鳴泰主覺華,崇煥主寧遠,在晉及張應吾、邢慎言持不可,承宗竟主崇煥議。已,承宗鎮關門,益倚崇煥,崇煥內拊軍民,外飭邊備,勞績大著。崇煥嘗核虛伍,立斬一校。承宗怒曰:「監軍可專殺耶?」崇煥頓首謝,其果於用法類此。

三年九月,承宗決守寧遠。僉事萬有孚、劉詔力阻,不聽,命滿桂偕崇煥往。初,承宗令祖大壽築寧遠城,大壽度中朝不能遠守,築僅十一,且疏薄不中程。崇煥乃定規制:高三丈二尺,雉高六尺,址廣三丈,上二丈四尺。大壽與參將高見、賀謙分督之,明年迄工,遂為關外重鎮。桂,良將,而崇煥勤職,誓與城存亡;又善撫,將士樂為盡力。由是商旅輻輳,流移駢集,遠近望為樂士。遭父憂,奪情視事。四年九月,偕大將馬世龍、王世欽率水陸馬步軍萬二千,東巡廣寧,謁北鎮祠,歷十三山,抵右屯,遂由水道泛三岔河而還。尋以五防敘勞,進兵備副使,再進右參政。

崇煥之東巡也,請即復錦州、右屯諸城,承宗以為時未可,乃止。至五年夏,承宗與崇煥計,遣將分據錦州、松山、杏山、右屯及大、小凌河,繕城郭居之。自是寧遠且為內地,開疆復二百里。十月,承宗罷,高第來代,謂關外必不可守,令盡撤錦、右諸城守具,移其將士於關內。督屯通判金啟倧上書崇煥曰:「錦、右、大凌三城皆前鋒要地。倘收兵退,既安之民庶復播遷,已得之封疆再淪沒,關內外堪幾次退守耶!」崇煥亦力爭不可,言:「兵法有進無退。三城已復,安可輕撤?錦、右動搖,則寧、前震驚,關門亦失保障。今但擇良將守之,必無他慮。」第意堅,且欲並撤寧、前二城。崇煥曰:「我寧前道也,官此當死此,我必不去。」第無以難,乃撤錦州、右屯、大、小凌河及松山、杏山、塔山守具,盡驅屯兵入關,委棄米粟十餘萬,而死亡載途,哭聲震野,民怨而軍益不振。崇煥遂乞終制,不許。十二月進按察使,視事如故。

我大清知經略易與,六年正月舉大軍西渡遼河,二十三日抵寧遠。崇煥聞,即偕大將桂,副將左輔、朱梅,參將大壽,守備何可剛等集將士誓死守。崇煥更刺血為書,激以忠義,為之下拜,將士咸請效死。乃盡焚城外民居,攜守具入城,清野以待。令同知程維楧詰奸,通判啟倧具守卒食,辟道上行人。檄前屯守將趙率教、山海守將楊麒,將士逃至者悉斬,人心始定。明日,大軍進攻,載楯穴城,矢石不能退。崇煥令閩卒羅立,發西洋巨炮,傷城外軍。明日,再攻,復被卻,圍遂解,而啟倧亦以然砲死。

啟倧起小吏,官經歷,主賞功事,勤敏有志介。承宗重之,用為通判,核兵馬錢糧,督城工,理軍民詞訟,大得眾心。死,贈光祿少卿,世廕錦衣試百戶。

初,中朝聞警,兵部尚書王永光大集廷臣議戰守,無善策。經略第、總兵麒並擁兵關上,不救,中外謂寧遠必不守。及崇煥以書聞,舉朝大喜,立擢崇煥右僉都御史,璽書獎勵,桂等進秩有差。

我大清初解圍,分兵數萬略覺華島,殺參將金冠等及軍民數萬。崇煥方完城,力竭不能救也。高第鎮關門,大反承宗政務,折辱諸將,諸將咸解體,遇麒若偏裨,麒至,見侮其卒。至是,坐失援,第、麒並褫官去,而以王之臣代第,趙率教代麒。

我大清舉兵,所向無不摧破,諸將罔敢議戰守。議戰守,自崇煥始。三月,復設遼東巡撫,以崇煥為之。魏忠賢遣其黨劉應坤、紀用等出鎮。崇煥抗疏諫,不納。敘功,加兵部右侍郎,賚銀幣,世蔭錦衣千戶。

崇煥既解圍,志漸驕,與桂不協,請移之他鎮,乃召桂還。崇煥以之臣奏留桂,又與不協。中朝慮僨事,命之臣專督關內,以關外屬崇煥畫關守。崇煥虞廷臣忌己,上言:「陛下以關內外分責二臣,用遼人守遼土,且守且戰,且築且屯。屯種所入,可漸減海運。大要堅壁清野以為體,乘間擊瑕以為用;戰雖不足,守則有餘;守既有餘,戰無不足。顧勇猛圖敵,敵必仇;奮迅立功,眾必忌。任勞則必召怨,蒙罪始可有功;怨不深則勞不著,罪不大則功不成。謗書盈篋,毀言日至,從古已然,惟聖明與廷臣始終之。」帝優旨褒答。

其冬,崇煥偕應坤、用、率教巡歷錦州、大、小凌河,議大興屯田,漸復第所棄舊土。忠賢與應坤等並因是廕錦衣,崇煥進所廕為指揮僉事。崇煥遂言:「遼左之壞,雖人心不固,亦緣失有形之險,無以固人心。兵不利野戰,只有憑堅城、用大炮一策。今山海四城既新,當更修松山諸城,班軍四萬人,缺一不可。」帝報從之。

先是,八月中,我太祖高皇帝晏駕,崇煥遣使弔,且以覘虛實。我太宗文皇帝遣使報之,崇煥欲議和,以書附使者還報。我大清兵將討朝鮮,欲因此阻其兵,得一意南下。七年正月,再遣使答之,遂大興兵渡鴨綠江南討。朝議以崇煥、之臣不相能,召之臣還,罷經略不設,以關內外盡屬崇煥,與鎮守中官應坤、用並便宜從事。崇煥銳意恢復,乃乘大軍之出,遣將繕錦州、中左、大凌三城,而再使使持書議和。會朝鮮及毛文龍同告急,朝命崇煥發兵援,崇煥以水師援文龍,又遣左輔、趙率教、朱梅等九將將精卒九千先後逼三岔河,為牽制之勢,而朝鮮已為大清所服,諸將乃還。

崇煥初議和,中朝不知。及奏報,優旨許之,後以為非計,頻旨戒諭。崇煥欲藉是修故疆,持愈力。而朝鮮及文龍被兵,言官因謂和議所致。四月,崇煥上言:「關外四城雖延袤二百里,北負山,南阻海,廣四十里爾。今屯兵六萬,商民數十萬,地隘人稠,安所得食?錦州、中左、大凌三城,修築必不可已。業移商民,廣開屯種。倘城不完而敵至,勢必撤還,是棄垂成功也。故乘敵有事江東,姑以和之說緩之。敵知,則三城已完,戰守又在關門四百里外,金湯益固矣。」帝優旨報聞。

時率教駐錦州,護版築,朝命尤世祿來代,又以輔為前鋒總兵官,駐大凌河。世祿未至,輔未入大凌,五月十一日大清兵直抵錦州,四面合圍。率教偕中官用嬰城守,而遣使議和,欲緩師以待救,使三返不決,圍益急。崇煥以寧遠兵不可動,選精騎四千,令世祿、大壽將,繞出大軍後決戰;別遣水師東出,相牽制;且請發薊鎮、宣、大兵,東護關門。朝廷已命山海滿桂移前屯,三屯孫祖壽移山海,宣府黑雲龍移一片石,薊遼總督閻鳴泰移關城;又發昌平、天津、保定兵馳赴上關;檄山西、河南、山東守臣整兵聽調。世祿等將行,大清已於二十八日分兵趨寧遠。崇煥與中官應坤、副使畢自肅督將士登陴守,列營濠內,用炮距擊;而桂、世祿、大壽大戰城外,士多死,桂身被數矢,大軍亦旋引去,益兵攻錦州。以溽暑不能克,士卒多損傷,六月五日亦引還,因毀大、小凌河二城。時稱寧、錦大捷,桂、率教功為多。忠賢因使其黨論崇煥不救錦州為暮氣,崇煥遂乞休。中外方爭頌忠賢,崇煥不得已,亦請建祠,終不為所喜。七月,遂允其歸,而以王之臣代為督師兼遼東巡撫,駐寧遠。及敘功,文武增秩賜廕者數百人,忠賢子亦封伯,而崇煥止增一秩。尚書霍維華不平,疏乞讓廕,忠賢亦不許。

未幾,熹宗崩。莊烈帝即位,忠賢伏誅,削諸冒功者。廷臣爭請召崇煥。其年十一月擢右都御史,視兵部添注左侍郎事。崇禎元年四月,命以兵部尚書兼右副都御史,督師薊遼、兼督登萊、天津軍務,所司敦促上道。七月,崇煥入都,先奏陳兵事,帝召見平臺,慰勞甚至,咨以方略。對曰:「方略已具疏中。臣受陛下特眷,願假以便宜,計五年,全遼可復。」帝曰:「復遼,朕不吝封侯賞。卿努力解天下倒懸,卿子孫亦受其福。」崇煥頓首謝。帝退少憩,給事中許譽卿叩以五年之略。崇煥言:「聖心焦勞,聊以是相慰耳。」譽卿曰:「上英明,安可漫對。異日按期責效,奈何?」崇煥憮然自失。頃之,帝出,即奏言:「東事本不易竣。陛下既委臣,臣安敢辭難。但五年內,戶部轉軍餉,工部給器械,吏部用人,兵部調兵選將,須中外事事相應,方克有濟。」帝為飭四部臣,如其言。

崇煥又言:「以臣之力,制全遼有餘,調眾口不足。一出國門,便成萬里,忌能妒功,夫豈無人。即不以權力掣臣肘,亦能以意見亂臣謀。」帝起立傾聽,諭之曰:「卿無疑慮,朕自有主持。」大學士劉鴻訓等請收還之臣、桂尚方劍,以賜崇煥,假之便宜。帝悉從之,賜崇煥酒饌而出。崇煥以前此熊廷弼、孫承宗皆為人排構,不得竟其志,上言:「恢復之計,不外臣昔年以遼人守遼土,以遼土養遼人,守為正著,戰為奇著,和為旁著之說。法在漸不在驟,在實不在虛,此臣與諸邊臣所能為。至用人之人,與為人用之人,皆至尊司其鑰。何以任而勿貳,信而勿疑?蓋馭邊臣與廷臣異,軍中可驚可疑者殊多,但當論成敗之大局,不必摘一言一行之微瑕。事任既重,為怨實多,諸有利於封疆者,皆不利於此身者也。況圖敵之急,敵亦從而間之,是以為邊臣甚難。陛下愛臣知臣,臣何必過疑懼,但中有所危,不敢不告。」帝優詔答之,賜蟒玉、銀幣,疏辭蟒玉不受。

是月,川、湖兵戍寧遠者,以缺餉四月大噪,餘十三營起應之,縛繫巡撫畢自肅、總兵官朱梅、通判張世榮、推官蘇涵淳於譙樓上。自肅傷重,兵備副使郭廣初至,躬翼自肅,括撫賞及朋椿二萬金以散,不厭,貸商民足五萬,乃解。自肅疏引罪,走中左所,自經死。崇煥以八月初抵關,聞變馳與廣密謀,宥首惡張正朝、張思順,令捕十五人戮之市;斬知謀中軍吳國琦,責參將彭簪古,黜都司左良玉等四人。發正朝、思順前鋒立功,世榮、涵淳以貪虐致變,亦斥之。獨都司程大樂一營不從變,特為獎勵。一方乃靖。

關外大將四五人,事多掣肘。後定設二人,以梅鎮寧遠,大壽仍駐錦州。至是,梅將解任,崇煥請合寧、錦為一鎮,大壽仍駐錦州,加中軍副將何可剛都督僉事,代梅駐寧遠,而移薊鎮率教於關門,關內外止設二大將。因極稱三人之才,謂:「臣自期五年,專藉此三人,當與臣相終始。屆期不效,臣手戮三人,而身歸死於司敗。」帝可之,崇煥遂留鎮寧遠。自肅既死,崇煥請停巡撫,及登萊巡撫孫國楨免,崇煥又請罷不設。帝亦報可。哈剌慎三十六家向受撫賞,後為插漢所迫,且歲饑,有叛志。崇煥召至於邊,親撫慰,皆聽命。二年閏四月,敘春秋兩防功,加太子太保,賜蟒衣、銀幣,蔭錦衣千戶。

崇煥始受事,即欲誅毛文龍。文龍者,仁和人。以都司援朝鮮,逗留遼東,遼東失,自海道遁回,乘虛襲殺大清鎮江守將,報巡撫王化貞,而不及經略熊廷弼,兩人隙始開。用事者方主化貞,遂授文龍總兵,累加至左都督,掛將軍印,賜尚方劍,設軍鎮皮島如內地。皮島亦謂之東江,在登、萊大海中,綿亙八十里,不生草木,遠南岸,近北岸,北岸海面八十里即抵大清界,其東北海則朝鮮也。島上兵本河東民,自天啟元年河東失,民多逃島中。文龍籠絡其民為兵,分布哨船,聯接登州,以為掎角計。中朝是之,島事由此起。

四年五月,文龍遣將沿鴨綠江越長白山,侵大清國東偏,為守將擊敗,眾盡殲。八月,遣兵從義州城西渡江,入島中屯田,大清守將覺,潛師襲擊,斬五百餘級,島中糧悉被焚。五年六月,遣兵襲耀州之官屯寨,敗歸。六年五月,遣兵襲鞍山驛,喪其卒千餘。越數日又遣兵襲撤爾河,攻城南,為大清守將所卻。七年正月,大清兵征朝鮮,并規剿文龍。三月,大清兵克義州,分兵夜搗文龍於鐵山。文龍敗,遁歸島中。時大清惡文龍躡後,故致討朝鮮,以其助文龍為兵端。

顧文龍所居東江,形勢雖足牽制,其人本無大略,往輒敗衄,而歲糜餉無算;且惟務廣招商賈,販易禁物,名濟朝鮮,實闌出塞,無事則鬻參販布為業,有事亦罕得其用。工科給事中潘士聞劾文龍糜餉殺降,尚寶卿董茂忠請撤文龍,治兵關、寧。兵部議不可,而崇煥心弗善也,嘗疏請遣部臣理餉。文龍惡文臣監制,抗疏駁之,崇煥不悅。及文龍來謁,接以賓禮,文龍又不讓,崇煥謀益決。

至是,遂以閱兵為名,泛海抵雙島,文龍來會。崇煥與相燕飲,每至夜分,文龍不覺也。崇煥議更營制,設監司,文龍怫然。崇煥以歸鄉動之,文龍曰:「向有此意,但惟我知東事,東事畢,朝鮮衰弱,可襲而有也。」崇煥益不悅。以六月五日邀文龍觀將士射,先設幄山上,令參將謝尚政等伏甲士幄外。文龍至,其部卒不得入。崇煥曰:「予詰朝行,公當海外重寄,受予一拜。」交拜畢,登山。崇煥問從官姓名,多毛姓。文龍曰:「此皆予孫。」崇煥笑,因曰:「爾等積勞海外,月米止一斛,言之痛心,亦受予一拜,為國家盡力。」眾皆頓首謝。

崇煥因詰文龍違令數事,文龍抗辯。崇煥厲色叱之,命去冠帶縶縛,文龍猶倔強。崇煥曰:「爾有十二斬罪,知之乎?祖制,大將在外,必命文臣監。爾專制一方,軍馬錢糧不受核,一當斬。人臣之罪莫大欺君,爾奏報盡欺罔,殺降人難民冒功,二當斬。人臣無將,將則必誅。爾奏有牧馬登州取南京如反掌語,大逆不道,三當斬。每歲餉銀數十萬,不以給兵,月止散米三斗有半,侵盜軍糧,四當斬。擅開馬市於皮島,私通外番,五當斬。部將數千人悉冒己姓,副將以下濫給札付千,走卒、輿夫盡金緋,六當斬。自寧遠還,剽掠商船,自為盜賊,七當斬。強取民間子女,不知紀極,部下效尤,人不安室,八當斬。驅難民遠竊人參,不從則餓死,島上白骨如莽,九當斬。輦金京師,拜魏忠賢為父,塑冕旒像於島中,十當斬。鐵山之敗,喪軍無算,掩敗為功,十一當斬。開鎮八年,不能復寸土,觀望養敵,十二當斬。」數畢,文龍喪魂魄不能言,但叩頭乞免。崇煥召諭其部將曰:「文龍罪狀當斬否?」皆惶怖唯唯。中有稱文龍數年勞苦者,崇煥叱之曰:「文龍一布衣爾,官極品,滿門封廕,足酬勞,何悖逆如是!」乃頓首請旨曰:「臣今誅文龍以肅軍。諸將中有若文龍者,悉誅。臣不能成功,皇上亦以誅文龍者誅臣。」遂取尚方劍斬之帳前。乃出諭其將士曰:「誅止文龍,餘無罪。」當是時,文龍麾下健校悍卒數萬,憚崇煥威,無一敢動者,於是命棺斂文龍。明日,具牲醴拜奠曰:「昨斬爾,朝廷大法;今祭爾,僚友私情。」為下淚。乃分其卒二萬八千為四協,以文龍子承祚、副將陳繼盛、參將徐敷奏、遊擊劉興祚主之。收文龍敕印、尚方劍,令繼盛代掌。犒軍士,檄撫諸島,盡除文龍虐政。還鎮,以其狀上聞,末言:「文龍大將,非臣得擅誅,謹席稿待罪。」時崇禎二年五月也。帝驟聞,意殊駭,念既死,且方倚崇煥,乃優旨褒答。俄傳諭暴文龍罪,以安崇煥心,其爪牙伏京師者,令所司捕。崇煥上言:「文龍一匹夫,不法至此,以海外易為亂也。其眾合老稚四萬七千,妄稱十萬,且民多,兵不能二萬,妄設將領千。今不宜更置帥,即以繼盛攝之,於計便。」帝報可。

崇煥雖誅文龍,慮其部下為變,增餉銀至十八萬。然島弁失主帥,心漸攜,益不可用,其後致有叛去者。崇煥言:「東江一鎮,牽制所必資。今定兩協,馬軍十營,步軍五,歲餉銀四十二萬,米十三萬六千。」帝頗以兵減餉增為疑,以崇煥故,特如其請。

崇煥在遼,與率教、大壽、可剛定兵制,漸及登萊、天津,及定東江兵制,合四鎮兵十五萬三千有奇,馬八萬一千有奇,歲費度支四百八十餘萬,減舊一百二十餘萬。帝嘉獎之。

文龍既死,甫踰三月,我大清兵數十萬分道入龍井關、大安口。崇煥聞,即督大壽、可剛等入衛。以十一月十日抵薊州,所歷撫寧、永平、遷安、豐潤、玉田諸城,皆留兵守。帝聞其至,甚喜,溫旨褒勉,發帑金犒將士,令盡統諸道援軍。俄聞率教戰歿,遵化、三屯營皆破,巡撫王元雅、總兵朱國彥自盡,大請兵越薊州而西。崇煥懼,急引兵入護京師,營廣渠門外。帝立召見,深加慰勞,咨以戰守策,賜御饌及貂裘。崇煥以士馬疲敝,請入休城中,不許。出與大軍鏖戰,互有殺傷。

時所入隘口乃薊遼總理劉策所轄,而崇煥甫聞變即千里赴救,自謂有功無罪。然都人驟遭兵,怨謗紛起,謂崇煥縱敵擁兵。朝士因前通和議,誣其引敵脅和,將為城下之盟。帝頗聞之,不能無惑。會我大清設間,謂崇煥密有成約,令所獲宦官知之,陰縱使去。其人奔告於帝,帝信之不疑。十二月朔再召對,遂縛下詔獄。大壽在旁,戰栗失措,出即擁兵叛歸。大壽嘗有罪,孫承宗欲殺之,愛其才,密令崇煥救解。大壽以故德崇煥,懼並誅遂叛。帝取崇煥獄中手書,往召大壽,乃歸命。

方崇煥在朝,嘗與大學士錢龍錫語,微及欲殺毛文龍狀。及崇煥欲成和議,龍錫嘗移書止之。龍錫故主定逆案,魏忠賢遺黨王永光、高捷、袁弘勛、史褷輩謀興大獄,為逆黨報仇,見崇煥下吏,遂以擅主和議、專戮大帥二事為兩人罪。捷首疏力攻,褷、弘勳繼之,必欲并誅龍錫。法司坐崇煥謀叛,龍錫亦論死。三年八月,遂磔崇煥於市,兄弟妻子流三千里,籍其家。崇煥無子,家亦無餘貲,天下冤之。

崇煥既縛,大壽潰而去。武經略滿桂以趣戰急,與大清兵戰,竟死,去縛崇煥時甫半月。初,崇煥妄殺文龍,至是帝誤殺崇煥。自崇煥死,邊事益無人,明亡徵決矣。

趙光抃,字彥清,九江德化人。父贊化,工部郎中,光抃舉天啟五年進士。鄉人曹欽程父事魏忠賢,驟得太僕少卿。光抃語之曰:「富貴一時,名節千古,君不可不審。」欽程惡之,即日出贊化為南寧知府。南寧惡地,贊化人宅人祭而死,光抃奔喪歸。

崇禎初,服闋,除工部都水主事,歷兵部職方郎中。十年秋,遣閱薊、遼戎務,盡得邊塞形勢,戰守機宜,列十二事以獻。明年冬,大清兵入密雲,總督吳阿衡敗歿,廷議增設巡撫一人,駐密雲,遂擢光抃右僉都御史任之。至即發監視中官鄧希詔奸謀。帝召希詔還,而令分守中官孫茂霖核實。茂霖為希詔解,光抃反得罪,遣戍廣東。

十五年,兵事益棘,廷臣薦光抃復官。光抃家素饒,聞命,持數萬金入都為軍資。既至,召見德政殿。奏對稱旨,拜兵部右侍郎兼右僉都御史,總督薊州、永平、山海、通州、天津諸鎮軍務。而大清已克薊州,分兵四出,命光抃兼督諸路援軍。諸援軍觀望,河間迤南皆失守,光抃不敢救,尾而南。已,聞塞上警,又驅而北。廷臣交章劾光抃,謂列城被攻不救,退回高陽,坐視淪覆。明年,復論光抃及范志完。四月,大清兵北旋,光抃、唐通、白廣恩等八鎮兵邀於螺山,皆敗走。帝聞,大怒。既解嚴,與志完並獲譴。帝召見雷縯祚,縯祚詆志完,而稱光抃。帝曰:「志完、光抃逗遛河間,獨罪志完,渠服其心乎?」遂並逮光抃。光抃嘗薦廣恩,廣恩抗不赴召,帝以是益惡光抃,卒與志完同日斬西市。

光抃才氣豪邁,而於大慮亦疏。在職方,深為尚書楊嗣昌所倚,曰:「吾不及光抃。」先是,毛文龍據東江,海疆賴之。文龍死,陳繼盛、黃龍、沈世魁代其部,往往為亂,中朝又素以糜餉為憂。及世魁死,島中無帥,光抃慫臾嗣昌撤之。二十年積患一朝而除,而於邊計亦左焉。光抃雖文士,有膽決,嘗遇敵,諸將欲奔,光抃坐地不起,久之,乃引歸。其起戍中也,將士不相習,猝遇大敵,先膽落,故所當輒敗。然受事破軍之餘,身先被創,顧與志完同誅,人咸以為冤。福王時,太僕萬元吉奏復其官。

范志完,虞城人。崇禎四年進士。授永平推官,專理插漢撫賞,意不欲行,上疏言權輕,請得特疏奏軍事。當事者惡之,謫湖廣布政司檢校。擢寧國推官,歷官分巡關內僉事。十四年冬,超擢右僉都御史,巡撫山西。其座主周延儒當國,遂拜志完兵部右侍郎兼右僉都御史,總督薊州、永平、山海、通州、天津諸鎮軍務,代楊繩武。

繩武者,雲南彌勒人也。由庶吉士改授御史。十一年冬,用楊嗣昌薦召見,吐言如流,畫地成圖。帝偉之,遂超擢右僉都御史,巡撫順天。洪承疇困松山,遂擢繩武總督,尋以志完代之,而令繩武總督遼東、寧遠諸軍,出關救松、錦,加銜督師。

明年正月,繩武卒官,贈兵部尚書,廕錦衣、世襲百戶。遂進志完左侍郎,督師出關如繩武,而以張福臻督薊鎮,駐關內。自王樸諸軍敗,兵力益單,松、錦相繼失,志完乃築五城寧遠城南,護轉輸,募土著實之。又議修覺華島城,為掎角勢,帝甚倚之。六月易銜欽命督師,總督薊、遼、昌、通等處軍務,節制登、津撫鎮。遼事急則移駐中後、前屯,關內急則星馳入援,三協有警則會同薊、昌二督並力策應。時關內外並建二督,而關外加督師銜,地望尤尊,又於昌平、保定設二督,於是千里之內有四督臣,又有寧遠、永平、順天、密雲、天津、保定六巡撫,寧遠、山海、中協、西協、昌平、通州、天津、保定八總兵。星羅棋置,無地不防,而事權反不一。

十五年,給事中方士亮劾福臻昏庸,因言移督師關內,則薊督可裁,福臻可罷。於是召還福臻,令志完兼制關內,移駐關門。志完辭,不許。求去,不許。上疏言不能兼薊,請仍設薊督。踰月,始以趙光抃任之。而大清兵已入自牆子嶺,克薊州而兵部劾志完疏防,廷臣亦言志完貪懦,帝以敵兵未退,責令戴罪立功。然志完無謀略,恇怯甚,不敢一戰,所在州縣覆沒,惟尾而呵噪,兵所到剽虜。至德州,僉事雷演祚劾之,自是論列者益眾。帝猶責志完後效,志完終不敢戰。

明年,大清兵攻下海州、贛榆、沭陽、豐縣,已而北旋。志完、光抃卒觀望,皆不進。事定,議罪,召縯祚廷質,問志完逗遛淫掠狀,志完辨。問御史吳履中,對如縯祚言。時座主延儒督師亦無功,遂命下志完獄,以十二月斬志完。

先是,十二年封疆之案,伏罪者三十有六人。至是,失事甚於前,誅止志完、光抃及巡撫馬成名、潘永圖,總兵薛敏忠,副將柏永鎮,其他悉置不問。而保定巡撫楊進得善去,山東巡撫王永吉反獲遷擢。帝之用刑,至是窮矣。

贊曰:三路喪師,收降取敗,鎬與應泰同辜。然君子重繩鎬而寬論應泰,豈不以士所重在節哉!惜乎廷弼以蓋世之材,褊性取忌,功名顯於遼,亦隳於遼。假使廷弼效死邊城,義不反顧,豈不毅然節烈丈夫哉!廣寧之失,罪由化貞,乃以門戶曲殺廷弼,化貞稽誅者且數年。崇煥智雖疏,差有膽略,莊烈帝又以讒間誅之。國步將移,刑章顛覆,豈非天哉!


\end{pinyinscope}