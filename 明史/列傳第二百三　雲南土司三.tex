\article{列傳第二百三 雲南土司三}

\begin{pinyinscope}
△緬甸二宣慰司干崖宣撫潞江南甸二宣撫司芒市者樂甸茶山孟璉即猛臉裡麻鈕兀東倘瓦甸促瓦散金木邦孟密安撫司附孟養車里老撾八百二宣慰司

緬甸,古朱波地。宋寧宗時,緬甸、波斯等國進白象,緬甸通中國自此始。地在雲南西南,最窮遠。有城郭廬舍,多樓居。元至元中,屢討之,乃入貢。

明太祖即位,遣使齎詔諭之。至安南,留二年,以道阻不能達而返,使者多道卒。洪武二十六年,八百國使人入貢,言緬近其地,以遠不能自達。帝乃令西平侯沐春遣使至八百國王所,諭意。於是緬始遣其臣板南速剌至,進方物,勞賜之。二十七年置緬中宣慰使司,以土酋卜剌浪為使。二十八年,卜剌浪遣使貢方物,訴百夷思倫發侵奪境土。二十九年復來訴。帝遣行人李思聰、錢古訓諭緬及百夷各罷兵守土,倫發聽命。會有百夷部長刀幹孟之亂,逐倫發,以故事得已。

永樂元年,緬酋那羅塔遣使入貢。因言緬雖遐裔,願臣屬中國,而道經木邦、孟養,多阻遏。乞命以職,賜冠服、印章,庶免欺陵。詔設緬甸宣慰使司,以那羅塔為宣慰使,遣內臣張勤往賜冠帶、印章。於是緬有二宣慰使,皆入貢不絕。五年,那羅塔遣使貢方物,謝罪。先是,孟養宣慰使刀木旦與戛里相攻,那羅塔乘釁襲之,殺刀木旦及其長子,遂據其地。事聞,詔行人張洪等齎敕諭責。那羅塔懼,歸其境土,而遣人詣闕謝罪。帝諭禮部曰:「蠻既服辜,其釋不問。」仍給以信符,令三年一朝貢。初,卜剌浪分其地,使長子那羅塔管大甸,次子馬者速管小甸。卜剌浪死,那羅塔盡收其弟土地人民。已而其弟復入小甸,遣人來朝,且訴其情。敕諭那羅塔兄弟和好如初,毋乾天討。六年,那羅塔復遣人入貢,謝罪,並謝賜金牌、信符,勞賜遣之。七年復遣中官雲仙等齎敕賜緬酋金織文綺。十二年,緬人來言為木邦侵掠。帝以那羅塔素強橫,遣人諭之,使修好鄰封,各守疆界。

洪熙元年遣內官段忠、徐亮以即位詔諭緬甸。宣德元年遣使往諭雲南土官,賜緬甸錦綺。二年以莽得剌為宣慰使。初,緬甸宣慰使新加斯與木邦仇殺而死,子弟潰散。緬共推莽得剌權襲,許之。自是來貢者只署緬甸,而甸中之稱不復見。八年,莽得剌遣人來貢,復遣雲仙齎敕賜之,並諭其勿侵木邦地。

正統六年給緬甸信符、金牌。時麓川思任發叛,將討之,命緬甸調兵待。七年,任發兵敗,過金沙江,走孟廣,緬人攻之。帝諭能擒獻賊首者,予以麓川地。八年,總督尚書王驥奏,緬甸酋馬哈省、以速剌等已擒獲思任發,不解至,唯以麓川地為言,朝命遂有並徵緬甸之命。是時,大師已集騰衝,緬使致書,期以今冬送思任發至貢章交付。驥與剋期,遣指揮李儀等率精騎通南牙山路,抵貢章,受獻,而緬人送思任發者竟不至。九年,驥駐師江上,緬人亦嚴兵為備,遣人往來江中,覘官軍虛實。驥以麓川未平,緬難不可復作,乃令總兵官蔣貴等潛焚其舟數百,緬人潰,驥亦班師。於是總兵官沐昂奏:「緬恃險黨賊,應加兵,但滇中方連年征討,財力困弊,旱澇相仍,糧餉不給,未可輕舉。臣已遣人諭緬禍福,俾獻賊首。緬宜聽從。」十二年,木邦宣慰罕蓋法,緬甸故宣慰子馬哈省、以速剌,遣使偕千戶王政等獻思任發首及諸俘馘至京,並貢方物。帝命馬哈省、以速剌並為宣慰使,賜敕獎勞,給冠帶、印信。未幾,以速剌奏求孟養、戛里地,且請大軍亟滅思任發之子思機發兄弟,而己出兵為助。帝諭以機發可不戰擒,宜即滅賊以求分地,弗為他人得也。

景泰二年賜緬甸陰文金牌、信符。時以速剌久獲思機發不獻,又放思卜發歸孟養。朝廷知其要挾,故緩之。五年,緬人來索地,參將胡志以銀戛等地與之,乃送機發及其妻孥。帝以思卜發既遠遁,不必窮追,仍加賞錦幣,降敕褒獎。

成化七年,鎮守太監錢能言,緬甸宣慰稱貢章、孟養舊為所轄,欲復得之。帝命往勘,貢章係木邦、隴川分治,孟養係思洪發所掌,非緬境,乃令雲南守臣傳飭諸部。而緬甸以所求地乃前朝所許,貢章乃朝貢必由之途,乞與之。又乞以金齒軍餘李讓為冠帶把事,以備任使。兵部尚書餘子俊等以思洪發不聞有過,豈可奪其地,李讓中國人,而與為把事,亦非體,宜勿許。帝命兵部諭其使,孟養、貢章是爾朝貢所由,當飭邊臣往諭思洪發,以通道往來,不得阻遏,餘勿多望。

弘治元年,緬甸來貢,且言安南侵其邊境。二年遣編修劉戩諭安南罷兵。然緬地鄰孟養,而孟養以緬先執思任發,故怨緬。嘉靖初,孟養酋思陸子思倫糾木邦及孟密,擊破緬,殺宣慰莽紀歲并其妻子,分據其地。緬訴於朝,不報。六年始命永昌知府嚴時泰、衛指揮王訓往勘。思倫夜縱兵鼓噪,焚驛舍,殺齎金牌千戶曹義,時泰倉皇遁,乃別立土舍莽卜信守之而去。值安鳳之亂,不暇究其事。

莽紀歲有子瑞體,少奔匿洞吾母家,其酋養為己子。既長,有其地。洞吾之南有古喇,濱海,與佛郎機鄰。古喇酋兄弟爭立,瑞體和解之,因德瑞體,爭割地為獻,受其約束,號瑞體為噠喇。瑞體乃舉眾絕古喇糧道,殺其兄弟,盡奪其地,諸蠻皆畏服之。時滅緬者木邦、孟養,而與緬相抗者孟密也。孟密土舍兄弟爭立,訴於瑞體。瑞體乃納其弟為婿,改名思忠,遣歸孟密,奪其兄印,因假道攻孟養及迤西諸蠻,以復前仇,又使其黨卓吉侵孟養境。後卓吉為思真婿猛乃頭目別混所殺,瑞體怒,自將攻別混父子,擒之。遂招誘隴川、乾崖、南甸諸土官,欲入寇。既覘知有備,又慮他蠻襲其後,乃遁歸。於是鎮巡官沐朝弼等上其事。兵部覆,荒服之外,治以不治。噠喇已畏威遠遁,傳諭諸蠻,不許交通結納。詔可。時嘉靖三十九年也。

木邦土舍罕拔求襲不得,怒投于緬,潞江宣撫糸泉貴聞之,亦入緬。瑞體自以起孤微,有兵眾,威加諸部,中國復禁絕之,遂謀內侵,乃命糸泉貴趣召隴川土官多士寧。士寧言中國廣大,誡勿妄動,瑞體稍稍寢。未幾,士寧為其下岳鳳所殺,乾崖宣撫刀怕舉亦死。罕拔乃請瑞體入乾崖,乾崖舉,則隴川可坐定也。瑞體子應裏桀黠多智,言於瑞體曰:「隴川、乾崖雖無主,遠難猝取。孟養思個近在肘腋,又吾世仇,萬一乘虛順流下,禍不測。」瑞體深然之,因借木邦兵一萬取乾崖,而自率兵侵孟養。既至,屢為思個所敗,思箇亦退保孟倫,相持久之。而隴川書記岳鳳欺其主幼,私齎賂投緬,結為父子。蠻莫土目思哲亦迎附瑞體,調緬兵萬餘,出入於迤西界上,以牽制思個。復徵木邦罕拔兵,會岳鳳於隴川,襲孟密。

萬曆元年,緬兵至隴川,入之。岳鳳遂盡殺士寧妻子族屬,受緬偽命,據隴川為宣撫。乃與罕拔、思哲盟,必下孟密,奉瑞體以拒中國。偽為錦囊象函貝葉緬文,稱西南金樓白象主莽噠喇弄王書報天皇帝,書中嫚辭無狀。罕拔又為緬招乾崖土舍刀怕文,許代其兄職。怕文拒之,與戰。適應里率眾二十萬分戍隴、乾間,以其兵驟臨之,怕文潰奔永昌。遂取乾崖印,付罕拔妹,以女官攝宣撫,召盞達副使刀思管、雷弄經歷廖元相佐之,同守乾崖,以防中國。於是木邦、蠻莫、隴川、乾崖諸蠻,悉附緬,獨孟養未下。

金騰副使許天琦遣指揮侯度持檄撫諭孟養。思個受檄,益拒緬。緬大發兵攻之,思個告急。會天琦卒,署事羅汝芳犒思箇使,令先歸待援,遂調兵至騰越。個聞援兵至,喜,令土目馬祿喇送等領兵萬餘,絕緬糧道,且導大兵伏戛撒誘緬兵深入。個率蠻卒衝其前,而約援兵自隴川尾擊之。緬兵既敗,糧又絕,屠象馬以食,瑞體窘甚。會有陳於巡撫王凝,言生事不便者,凝馳使止援軍。汝芳聞檄退,思個待援不至。岳鳳偵知之,集隴川兵二千兼程進,導瑞體由間道遁去。思箇追擊之,緬兵大敗,當是時幾獲瑞體。

六年,廷議遣使至孟養,俾思個還所俘緬兵象,並賚以金帛,好言慰諭之。瑞體不謝。七年,永昌千戶辛鳳奉使買象於孟密,思忠執鳳送緬,緬遣回。是年,緬復攻孟養,報戛撒之怨。思個以無援敗,將走騰越,中途為其下所執,送瑞體,殺之,盡並孟養地。八年,巡撫饒仁侃遣人招緬,緬不應。

十年,岳鳳導緬兵襲破干崖,奪罕氏印,俘之。俄,瑞體死,子應裏嗣。岳鳳嗾應裏殺罕拔,盡俘其眾。又說應裏起兵象數十萬,分道內侵。十一年焚掠施甸,寇順寧。鳳子曩烏領眾六萬,突至孟淋寨,指揮吳繼勛、千戶祁維垣戰死。又破盞達,副使刀思定求救不得,城破,妻子族屬皆盡。且窺騰衝、永昌、大理、蒙化、景東、鎮沅諸郡。巡撫劉世曾請以南京坐營中軍劉綎為騰越遊擊,移武靖參將鄧子龍為永昌參將,各提兵五千赴剿,並調諸土軍應援。緬亦合兵犯姚關,綎與子龍大破之於攀枝花地,乘勝追擊,自十年十月至十一年四月,斬首萬餘。復率兵出隴川、孟密,直抵阿瓦,緬將猛勺詣綎降。勺,瑞體弟也。緬將之守隴川、孟養、蠻莫者,皆遁去,岳鳳及其子皆伏誅。官軍定隴川,遂歸。應裏乃以其子思斗守阿瓦,復攻孟養、蠻莫,聲言復仇。副使李材備兵騰衝,遣兵援之,戰於遮浪,大破其象陣,生擒五千餘人。

先是,蠻莫酋思化投緬。材遣人招之,思化降。十九年,應裏復率緬兵圍蠻莫,思化告急。會天暑,軍行不前,裨將萬國春夜馳至,多設火炬為疑兵,緬人懼而退,追敗其眾。二十二年,巡撫陳用賓設八關於騰衝,留兵戍守,募人至暹羅約夾攻緬。緬初以猛卯酋多俺為嚮導,寇東路。至是遣木邦罕欽擒多俺殺之,前築堡於猛卯,大興屯田。是年,緬帥思仁寇蠻莫,敗之,斬其渠丙測。

二十三年,應裏屬孟璉、孟艮二土司求朝貢,鎮巡以聞。朝議令原差官黎景桂齎銀幣賜之,至境,不受。詔以景桂首事貪功納侮,下於理。三十一年,阿瓦雍罕、木邦罕拔子罕衣盍俱入貢,緬勢頓衰。暹羅得楞復連歲攻緬,殺緬長子莽機撾,古喇殘破。自此不敢內犯,然近緬諸部附之如初。崇禎末,蠻莫思綿為緬守曩木河。及黔國公沐天波等隨永明王走蠻莫,思綿使告緬。緬遣人迎之,傳語述萬曆時事,並出神宗璽書,索今篆合之,以為偽。天波出己印與先所頒文檄相比無差,始信。蓋自天啟後,緬絕貢職,無可考驗云。

乾崖,奮名干賴夾,僰人居之。東北接南甸,西接隴川,有平川眾岡。境內甚熱,四時皆蠶,以其絲織五色土錦充貢。元中統初,內附。至元中,置鎮西路軍民總管府,領三甸。洪武十五年改鎮西府。永樂元年設乾崖長官司。二年頒給信符、金字紅牌並賜冠服。三年,乾崖長官曩歡遣頭目奉表貢馬及犀、象、金銀器,謝恩,賜鈔幣。五年設古剌驛,隸乾崖。曩歡復遣子刀思曩朝貢,賜賚如例。自是,三年一朝貢不絕。宣德六年改隸雲南都司。時長官刀弄孟奏,其地近雲南都司,而歲納差發銀於金齒衛,路遠,乞改隸,而輸銀於布政司。從之。正統三年命仍隸金齒軍民指揮使司。六年升乾崖副長官刀怕便為長官司,賜綵幣,以歸附後屢立功,從總兵官沐昂請也。九年升乾崖為宣撫司,以刀怕便為宣撫副使,劉英為同知,從總督王驥請也。

弘治三年,乾崖土舍刀愈怕欺其姪刀怕落幼,劫印奪職。蠻眾不服,遂起兵相攻。四年,按察司副使林俊同參將沐詳移文往諭,始釋兵歸印。事聞,帝以鎮巡官不以時奏報,責之。嘉靖三十九年,緬酋莽瑞體叛,招乾崖諸土官入寇。萬曆初,宣撫刀怕舉死,妻罕氏,木邦宣慰罕拔妹也。拔既叛附緬,召怕舉弟怕文襲職以臣緬,且許以妹。怕文不受,與戰。緬兵十萬驟臨,怕文潰奔永昌。罕拔遂取乾崖印付罕氏。十年,隴川岳鳳破干崖,奪罕氏印。十一年,遊擊劉綎破隴川,鳳降,追印竟不得。而干崖部眾自相承代,亦莫得而考云。

潞江,地在永昌、騰越之間,南負高崙山,北臨潞江,為官道咽喉。地多瘴癘,蠻名怒江甸。至元間,隸柔遠路。永樂元年內附,設潞江長官司。其地舊屬麓川平緬,西平侯奏其地廣人稠,宜設長官司治之。二年頒給信符、金字紅牌。九年,潞江長官司曩璧遣子維羅法貢馬、方物,賜鈔幣,尋陞為安撫司。曩璧來朝,貢象、馬、金銀器,謝恩。

宣德元年,曩璧遣人貢馬,請改隸雲南布政司,從之。遣中官雲仙齎敕及綺幣賜曩璧。三年,黔國公沐晟奏,潞江千夫長刀不浪班叛歸麓川,劫潞江,逐曩璧入金齒,據潞江驛,逐驛丞周禮,立寨固守,斷絕道路,請發兵討。帝敕晟與三司計議。五年,晟奏,刀不浪班懼罪,還所據地,歸舊部,輸役如故,乞宥之。報可。是年置雲南廣邑州。時雲仙還言:「金齒廣邑寨,本永昌副千戶阿干所居。乾嘗奉命招生蒲五千戶向化。今干孫阿都魯同蒲酋莽塞等詣京貢方物,乞於廣邑置州,使阿都魯掌州事,以熟蒲並所招生蒲屬之。」帝從之,遂以阿都魯為廣邑州知州,莽塞為同知,鑄印給之。八年改金齒永昌千戶所為潞江州,隸雲南布政司,以千夫長刀珍罕為知州,刀不浪班為同知,置吏目及清水關巡檢各一員。

正統三年從黔國公沐晟奏,改潞江安撫司仍隸金齒,悉還舊制。五年,安撫使糸泉舊法以麓川思任發叛來告,諭整兵以俟。未幾,麓川賊遣部眾奪據潞江,殺傷官軍,潞江遂削弱。

正德十六年,安撫司土官安捧奪其從弟掩莊田三十八所,掩訟於官,不報。捧遂集蠻兵圍掩寨,縱火屠掠,掩母子妻妾及蠻民男婦死者八十餘人,據有其地。官軍誘執之,捧死於獄。帝命戮屍棄市,其子詔及黨與皆斬。天啟間,有糸泉世祿者,繼襲安撫。

南甸宣撫司,舊名南宋,在騰越南半箇山下,其山巔北多霜雪,南則炎瘴如蒸。元置南甸路軍民總管府,領三甸。洪武十五年改南甸府。永樂十一年改為州,隸布政司。宣德三年,南甸為麓川侵奪,有司請討。不許,降敕誡諭麓川,俾還侵地。五年,南甸州奏:「先被麓川宣慰司奪其境土,賴朝廷威力復之,若不置官司以正疆界,恐侵奪未厭,乞置四巡檢司鎮之。」帝命吏部除官。八年又奏:「與麓川接境,舊十二百夫長在騰沖千戶所時,賴邦哈等處軍民兼守。後麓川侵據,不守者十餘年。今蒙敕諭還,竊恐再侵,百姓逃移,乞於賴邦哈、九浪、莽孟洞三處各置巡檢,以土軍楊義等三人為之。」命下三司勘覆,授之。

正統二年,土知州刀貢罕奏:「麓川思任發奪其所轄羅卜思莊二百七十八村,乞遣使齎金牌、信符諭之退還。」帝敕沐晟處置奏聞。麓川之役自是起。九年陞州為宣撫司,以知州刀落硬為宣撫使,通判劉思勉為土同知。六年頒給金牌、信符、勘合,加敕諭之。十年免所欠差發銀兩,令安業後,仍前科辦。

天順二年復置南甸驛丞一人,以土人為之。時宣撫刀落蓋奏南寧伯毛勝遣騰沖千戶藺愈占其招八地,逼民逃竄。敕雲南三司官同巡按御史詣其地體勘,以所占田寨退還,治勝、愈罪。

南甸所轄羅卜思莊與小隴川,皆百夫長之分地。知事謝氏居曩宋,悶氏居盞西,屬部直抵金沙江,地最廣。司東十五里曰蠻干,宣撫世居之。南百里有關,立木為柵,周一里。曰南牙,甚高,山勢延袤一百餘里,官道經之。上有石梯,蠻人據以為險。

芒市,舊曰怒謀,又曰大枯夾、小枯夾,在永昌西南四百里,即唐史所謂茫施蠻也。元中統初內附。至元十三年立茫施路軍民總管府,領二甸。洪武十五年,置茫施府。正統七年,總兵官沐晟奏:「芒市陶孟刀放革遣人來訴,與叛寇思任發有仇。今任發已遁去,思機發兄弟三人來居麓川者藍地方,願擒以獻。」兵部言:「放革先與任發同惡,今勢窮乃言結釁,譎詐難信。宜敕諭放革,如能去逆效順,當密調土兵助剿機發。」從之。八年,機發令其黨涓孟車等來攻芒市,為官軍所敗。放革來降,靖遠伯王驥請設芒市長官司,以陶孟刀放革為長官,隸金齒衛。成化八年,木邦曩罕弄亂,掠隴川。敕芒市等長官司整兵備調。萬曆初,長官放福與隴川岳鳳聯姻,導緬寇松坡營。事覺,伏誅,立舍目放緯領司事,轄於隴川。芒市川原廣邈,田土富饒,而人稍脆弱云。

者樂甸,本馬龍他郎甸猛摩地,名者島。洪武末內附,隸雲南布政司。永樂元年設者樂甸長官司,改隸雲南都司,以沐晟言其地廣人稠也。十八年,長官刀談來朝,貢馬。自是,皆以刀氏世領司事。其地山險多瘴,介於鎮沅、元江、景東間。日事攻戰,鎧械犀利,兵寡而敕,諸部畏憚之。

茶山長官司,永樂二年頒給信符、金字紅牌。八年,長官早張遣人貢馬。宣德五年置滇灘巡檢司。以長官司奏滇灘當茶山瓦高之衝,蠻寇出沒,民不能安,通事段勝頗曉道理,能安人心,乞置司,以勝為巡檢。從之。

孟璉長官司,永樂四年四月設。時孟璉頭目刀派送遣子壞罕來言,孟璉舊屬麓川平緬宣慰司,後隸孟定府。而孟定知府刀名扛亦故平緬頭目,素與等夷,乞改隸。遂設長官司,隸雲南都司,命刀派送為長官,賜冠帶、印章。正統四年,思任發反,以兵破孟璉,遂降於麓川,為木邦宣慰罕蓋法擊敗。七年,總督王驥征麓川,招降孟璉、亦保等寨。敕賜孟璉故長官司刀派罕子派樂等綵幣,以麓川平故也。嘉靖中,孟璉與孟養、孟密諸部仇殺數十年,司廢。至萬歷十三年,隴川平,復設,稱猛臉云。

里麻長官司,永樂六年設,隸雲南都司,以刀思放為長官。時思放為里麻招剛。招剛者,故西南蠻官名。思放籍其地來朝,請授職事,遂有是命,仍賜印章、冠帶。八年遣頭目貢馬。

鈕兀長官司,宣德八年置。鈕兀、五隆諸寨在和泥之地,其酋任者、陀比等朝貢至京,奏地遠蠻多,請授職以總其眾。兵部請設長官司,從之。遂以任者為長官,陀比為副。

東倘長官司,宣德八年置,隸緬甸宣慰。時緬甸宣慰昔得謀殺當蕩頭目新把的,而奪其地。新把的遣子莽只貢象、馬、方物,乞置司,庶免侵殺,從之。置東倘長官司,命新把的為長官。

瓦甸長官司,初隸金齒,永樂九年改隸雲南都司。土官刀怕賴言金齒遠,都司近,故改隸焉。宣德八年置曲石、高松坡、馬緬三巡檢司。初,長官司言其地山高林茂,寇盜出沒,人民不安,乞置巡檢司,以授通事楊資、楊中、范興三人,從之。命資於曲石,中於高松坡,興於馬緬。正統五年,長官早貴為思任發所獲,殺其守者十七人,挈家來歸。帝嘉其忠順,命所司褒賞,以早貴為安撫,賜彩幣、誥命。

促瓦、散金二長官司,皆永樂五年設,隸雲南都司。其地舊屬麓川,平緬。土蠻註甸八等來朝,請別設長官司,從之。命註甸八等為長官,各給印章。

木邦,一名孟邦。元至元二十六年立木邦路軍民總管府,領三甸。洪武十五年平雲南,改木邦府。建文末,土知府罕的法遣人貢馬及金銀器,賜鈔幣。永樂元年遣內官楊瑄齎敕諭木邦諸土官。明年遣人來貢。時麓川訴木邦侵地,命西平侯諭之,因改木邦為軍民宣慰使司,以知府罕的法為使,賜誥印。時官軍徵八百,罕的法發兵助戰,攻江下等十餘寨,斬首五百餘級。詔遣鎮撫張伯恭、經歷唐復往賜白金、錦幣,及其部領有差。明年遣使貢象馬、方物,謝恩。頒賜如例,復加賜其母及妻錦綺。罕的法卒,其子罕賓發來朝,請襲,命賜冠服。七年遣使謝恩。又遣人奏緬甸宣慰使那羅塔數誘賓發叛,賓發不敢從逆,若天兵下臨,誓當效命。帝嘉其忠,遣中官徐亮齎敕勞之,賜白金三千兩、錦綺三百表裏,祖母、母、妻織金文綺、紗羅各五十疋。自是,每三年遣使貢象馬。十一年,賓發遣使獻緬甸俘。時木邦攻破緬甸城寨二十餘,多所殺獲,獻於京師。

宣德三年遣中官徐亮齎敕及文綺賜襲職宣慰罕門法並及祖母、母、妻。八年,木邦與麓川、緬甸各爭地,訴於朝,帝命沐晟並三司巡按公勘。

正統三年征麓川,敕諭木邦以兵會剿。五年,總兵官沐昂遣人間道達木邦,得報,知宣慰祖母美罕板、其孫宣慰罕蓋法與麓川戰於孟定、孟璉地,殺部長二十人,斬首三萬餘級,獲馬象器械甚眾。帝嘉其功,加授罕蓋法懷遠將軍,封美罕板太淑人,賚以金帶、彩幣。七年,總督王驥奏,罕蓋法遣兵攻拔麓川板罕、貢章等寨,追至孟蒙,獲其孥七人,象十二,麓川酋思任發父子遁孟廣。帝命指揮陳儀往勞之,且曰:「木邦能自效,生縶賊首獻,其酬以麓川土地人民。」八年免木邦歲辦金萬四千兩。木邦遣人謝恩,并獻所獲思任發家屬,復賜敕及彩幣獎勞。十一年,緬甸獻任發首,木邦亦遣使與同獻,且修貢職,因求麓川地。兵部以麓川已設隴川宣撫司,請以孟止地給之,並遣官諭祭其母,以表忠勤,免木邦歲辦銀八錠三年,從之。

暴泰元年,罕蓋法奏乞隴川界者闌景線地,未服,蓋法子罕落法輒發兵據之。隴川宣撫刀歪孟訴於總兵官沐璘。璘遣使諭歸之,而與以底麻之地。四年,罕落法襲父職。族人構難,落法避於孟更,遣人赴總兵官求救。璘以聞,詔左參將胡志調兵撫諭之,與其族人部眾設盟而還。然落法猶避居孟都不敢歸。孟都蠻者,地近隴川,歲調蠻兵二百更番護之。

天順元年,鎮守中官羅珪奏:「罕落法與所部交攻,遣人求援。臣等議委南寧伯毛勝、都督胡志量調官軍,相機剿捕。」帝以非犯邊疆,不許。二年,落法奏為思坑、曩罕弄等所攻,乞兵剿除,命總兵官區處。六年,總兵官沐瓚奏罕落法屢侵隴川地,欲以撥守貴州兵八千調回防禦,詔留其半。

成化十年,木邦所轄孟密蠻婦曩罕弄等侵掠隴川,黔國公沐琮以聞。曩罕弄者,故木邦宣慰罕揲法之女,嫁其孟密部長思外法。地有寶井。罕揲法卒,孫落法嗣。曩罕弄以尊屬不樂受節制,嗾族人與爭。景泰中,叛木邦,逐宣慰,據公署,殺掠鄰境隴川、孟養,兵力日盛,自稱天娘子,其子思柄自稱宣慰。黔國公琮奏委三司官往撫,曩罕弄驕蹇不服,且欲外結交址兵,逼脅木邦、八百諸部,琮等復以聞。兵部尚書張鵬主用兵。詔廷臣集議,皆以孟密與木邦仇殺,並未侵犯邊境,止宜撫諭。因命副都御史程宗馳傳與譯者序班蘇銓往。時成化十八年也。踰年,孟密思柄遣人入貢,宴賜如土官例。已,孟密奏為木邦所擾,乞別設安撫司。張鵬以太監覃平、御史程宗撫馭已有成緒,遂命宗巡撫雲南,敕平偕詣金齒勸諭之,其孟密地或仍隸木邦,或別設安撫,區處具奏。初,曩罕弄竊據孟密,貳於木邦。畏鄰境不平,遣人從間道抵雲南,至京,獻寶石、黃金,乞開設治所,直隸布政司。閣臣萬安欲許之,劉珝、劉吉皆以孟養原木邦屬夷,今曩罕弄叛,而請命於朝,若許之,則土官誰不解體。蘇銓私以告於宗。宗復奏曩罕弄與木邦仇殺已久,勢難再合,已喻諸蠻,示以朝廷德意,宥其罪,開設衙門,令還其所侵地,皆踴躍奉命,木邦亦已允服,乞遂行之。部覆,從之。二十年遂設孟密安撫司,以思柄為使。時孟密據寶井之利,資為結納,而木邦為孟密所侵,兵力積弱,不能報,雖屢奏訴,竟不得直云。

弘治二年,雲南守臣奏,孟密曩罕弄先後占奪木邦地二十七處,又誘其頭目放卓孟等叛,其勢必盡吞後已。乞敕八百宣慰司俾與木邦和好,互相救援。亦敕木邦宣慰收復人心,親愛骨肉,勿使孟密得乘間誘叛,自致孤弱。如孟密聽諭,方許曩罕弄孫承襲。報可,並敕雲南守臣親詣金齒曉諭,復降敕詰責前鎮巡官所以受賂召侮啟釁者。三年追論致仕南京工部尚書程宗罪。先是,宗以右副都御史奉命率蘇銓往撫諭,而銓受思柄金,紿宗奏為設孟密安撫司。銓復教思柄偽歸木邦地,而占據如故,思柄益橫。至是,木邦宣慰罕挖法發其事,時宗已致仕,巡按請追罪之。獄具,帝以事在赦前,不問。六年,雲南守臣奏孟密侵奪木邦,兵連禍結,垂四十餘年,屢撫屢叛,勢愈猖肆,請調兵往討。兵部議以孟密安撫,初隸布政司,今改隸木邦,以致爭殺,仍如初隸可息兵,從之。

初,孟密之復叛木邦也,因木邦宣慰罕挖法親迎婦於孟乃寨,孟密土舍思揲乘虛襲之,據木邦,誘降其頭目高答落等,聚兵阻路。罕挖法不得歸,依孟乃寨者三年。於是巡撫張誥等會奏,議遣文武大員詣孟密撫諭,思揲猶不服。誥乃遣官督率隴川、南甸、乾崖三宣撫司,積糧開道,示以必征之勢,又令漢土官舍耀兵以威之。高答落等懼,謀歸罕挖法。思揲欲殺之,罕挖法乞救於鄰部,調土兵合隴川等三宣撫兵至蠻遮,共圍之。思揲懼,乃罷兵。誥等奏其事,且乞賞有功者。兵部議,罕挖法雖還木邦,思揲猶未悔罪,必令歃血同盟,歸地獻叛,永息爭端,乃可論功行賞,報聞。

九年,罕挖法及思揲各遣使來貢,報賜如例。初,思揲圍蠻遮,木邦宣慰妻求救於孟養思陸。孟密素畏思陸之兵,聞其將至,遂解去。木邦與思陸謀共取孟密,於是蠻中之患,又在孟養矣。自萬安、程宗勘處失宜,諸酋長紛紜進退,中國用兵且數十年。

嘉靖初,思陸子思倫與木邦宣慰罕烈同擊殺緬酋莽紀歲,而分其地。後莽瑞體強,將修怨於木邦。隆慶二年,木邦土舍罕拔告襲,有司索賂不為請。拔怒,與弟罕章集兵梗往來道,商旅不前,而己食鹽亦乏絕,乞於緬。緬以五千籝饋之,自是反德緬,攜金寶象馬往謝之。瑞體亦厚報之,歡甚,約為父子。瑞體死,子應裏用岳鳳言誘拔殺之。時萬曆十一年也。

拔子進忠守木邦,應裏遣弟應龍襲之,其孽子罕鳳與耿馬舍人罕虔欲擒進忠獻應龍。進忠攜妻子內奔,虔等追至姚關,焚順寧而去。十二年,官軍破緬於姚關,立其子欽。欽死,其叔罕衣盍約暹羅攻緬,緬恨之。三十四年,緬以三十萬眾圍其城。請救於內地,不至,城陷,罕衣盍被擄。緬偽立孟密思禮領其眾。事聞,黜總兵官陳賓,木邦遂亡。

孟密自思柄授安撫,繼之者曰思揲,曰思真,真年至百十歲。嘉靖中,土舍兄弟爭襲,走訴於緬。緬人為立其弟,改名思忠,忠遂以其地附緬。萬曆十二年,忠齎偽印來歸,命授為宣撫。已而復投緬,乃以其母罕烘代掌司印。緬攻孟密,罕烘率子思禮、從子思仁奔孟廣,而孟密遂失。十八年,緬復攻孟廣,罕烘、思禮奔隴川,思仁奔工回,而孟廣又失。先是,思仁從罕烘奔孟廣時,有甘糸泉姑者,思忠妻也。思忠既投緬,思仁通於線姑,遂欲妻之,而罕烘不許。至是,罕烘攜糸泉姑走隴川,思仁奔雅蓋,率兵象犯隴川,欲擄糸泉姑去。會隴川有備,弗克,思仁亦走歸緬,緬偽署思仁於孟密,食其地。初,孟密寶井,朝廷每以中官出鎮,司採辦。武宗朝錢能最橫,至嘉靖、隆慶時猶然。萬曆二十年,巡撫陳用賓言,緬酋擁眾直犯蠻莫,其執詞以奉開採使命令,殺蠻莫思正以開道路。全滇之禍,皆自開採啟之。時稅使楊榮縱其下,以開採為名,恣暴橫,蠻人苦之。且欲令麗江退地聽採,緬酋因得執詞深入。巡按宋興祖極言其害,請追還榮等,帝皆不納。凡採辦必先輸官,然後與商賈貿易,每往五六百人。其屬有地羊寨,在孟密東,往來道所必經。人工幻術,採辦人有強索其飲食者,多腹痛死;己所乘馬亦斃,剖之,則馬腹皆木石也。思真嘗剿之,殺數千人,不得絕。至是,復議剿,以兵少中止。

孟養,蠻名迤水,有香柏城。元至元中,於孟養置雲遠路軍民總管府。洪武十五年改為雲遠府。其地故屬平緬宣慰司。平緬思倫發為其下所逐,走京師。帝命西平侯沐春以兵納之,還故地。成祖即位,改雲遠府為孟養府,以土官刀木旦為知府。永樂元年,刀木旦遣人貢方物及金銀器,賜賚遣歸。二年改升軍民宣慰使司,以刀木旦為使,賜誥印。四年,孟養與戛里相仇殺,緬甸宣慰那羅塔乘釁劫之,殺刀木旦及子思欒發而據其地。事聞,詔行人張洪等齎敕諭責緬。那羅塔懼,仍歸其境土。會木邦宣慰使罕賓法以那羅塔侵據孟養,請自率兵討,遂破緬甸城寨二十餘,獲其象、馬獻京師。十四年復設孟養宣慰司,命刀木旦次子刀得孟為使,以木旦姪玉賓為同知。自木旦被害,司遂廢,孟養之人從玉賓散居乾崖、金沙江諸處者三千餘人。朝廷嘗命玉賓署宣慰使以撫之,故仍命為本司同知,令其率眾復業。十五年,刀得孟遣使貢馬及方物。

宣德五年,刀玉賓奏:「伯父刀木旦被殺,蒙朝廷遣官訪玉賓,授同知,又阻於緬難,寄居金齒者二十餘年。今孟養地又為麓川宣慰思任發所據,乞遣兵送歸本土。」帝命黔國公沐晟遣還之,然其地仍為任發所有。時為孟養宣慰者名刀孟賓,亦寄居雲南。及任發敗奔緬甸,子機發潛匿孟養,求撫。

正統十三年敕孟養頭目伴送思機發來朝,許以陞賞,機發疑畏竟不至。帝以孟養宣慰頭目刀變蠻等匿機發,敕數其罪,曰:「孟養乃朝廷開設,爾刀變蠻等敢違朝命,一可伐。思機發係賊子,故縱不捕,二可伐。爾孟養被思任發奪地,逐爾宣慰,見在雲南優養,爾等與仇為黨,三可伐。雲南總兵官世世管屬爾地,奉命捕取賊子,爾等不從調度,四可伐。爾等不過以為山川險阻,官軍未易遽到,又以為氣候瘴癘,官軍不可久居。勢強則拒敵,力弱則奔遁。殊不知昔馬援遠標銅柱,險阻無傷,諸葛亮五月渡瀘,炎蒸無害,皆能破滅蠻眾,開拓境土。況今大軍有必勝之機,麓川之師可為前鑒。爾等速宜悔過自圖,令思機發親自前來,仍與一官一地,令享生全。如不肯出,爾等即擒為上策;迹思機發所在,報與官軍捕取為中策;若代彼支吾,令其逃匿,則並爾等剿滅,悔無及矣。」時已三征麓川,內旨必欲生擒機發,已密諭總督王驥,又敕諭以雲南安置孟養舊宣慰刀孟賓為嚮導。及兵出窮征,機發卒遁去,不可得。於是乃以孟養地給緬甸宣慰馬哈省管治,命捕思機發。時正統十四年也。

景泰二年,任發之子思卜發遣使來貢,求管孟養舊地。廷臣議,孟養地已與緬甸,豈可移易。時朝命雖不許,然卜發已潛據之,即緬甸不能奪也。卜發死,子思洪發嗣,自天順、成化,每朝貢輒署孟養地名,儼然自有其地矣。

成化中,孟養金沙江思陸發遣人貢象馬,宴賜皆如例。思陸發者,思任發之遺孽也。太監錢能鎮雲南,思陸發數以珍寶遺能,因得入貢,稱孟養金沙江思陸發,常規立功以襲祖職。適孟密安撫土舍思揲侵據木邦地,爭殺累年,守臣議征之,思陸發乃請自效。時蠻眾相傳孟密畏思陸兵,參政毛科請於總兵鎮巡官,許之。思陸兵未至,思揲解去。巡撫張誥議調思陸兵,令戮力捕思揲,乃遣使促之發兵。思陸遣大陶孟倫索領蠻兵象馬過江,倫索既過江,指鷹謂使者曰:「我曹猶此鷹,奪得土地,即管食之耳。」科聞之憂甚。時思揲令陶孟思英以兵守蠻莫。孟養兵至,思英堅守不出,已而請和。孟養兵聞官軍聽思英約降,頗有怨言。官軍糧絕,遽引退。倫索亦恐思英絕其歸路,取道干崖而還。科念倫索前語,急戒令孟養還兵守疆界,孟養不聽。初,靖遠伯王驥與之約誓,非總兵官符檄不得渡江。自是遂犯約,數興兵過江與孟密戰。

弘治十二年,雲南巡按謝朝宣奏:

孟養思陸本麓川叛種,竄居金沙江外。成化中,嘗據緬甸之聽盞。弘治七年徵調其兵渡江,遂復據騰衝之蠻莫。又糾木邦兵,攻燒孟密安撫司,殺掠蠻民二千餘人,劫象馬金寶,有并吞孟密覬覦故土之志。迤西人恭們、騰衝人段和為之謀主,屢撫不聽。雲南會城去孟養遠,聲勢難接。曩於金騰添設鎮守太監,為撫蠻安民之計。而近時太監吉慶貪暴無狀,雖嘗陽卻思陸之贄,然蠻知其貪,又烏知不因其卻而更進之。臣聞蠻莫等處,乃水陸會通之地,蠻方器用咸自此出,江西、雲南大理逋逃之民多赴之。雲南差官每多齎違禁物往彼餽送,漏我虛實,為彼腹心。鎮夷關一巡檢耳,安能禁制。臣計孟養甲兵不能當中原一大縣,以雲南之勢臨之,易於壓卵。柰何一調即來,屢撫不退,皆鎮巡失之於初,逋逃奸人謀之於中,撫蠻中官壞之於後。伏望垂念邊民困苦,將雲南鎮守太監止存一員,另用指揮一員守備鎮夷關,驅思陸退歸江外,而移騰衝司於蠻莫,並木邦、孟密不得窺伺,乃為萬全之策。設思陸冥頑不聽撫諭,便當決策用兵,使無噍類,以為土官不法之戒。

先是,吉慶已為思陸請朝貢,至是因朝宣疏,並下鎮巡官議剿撫之宜,數年不決。

十六年,巡撫陳金乃遣金騰參將盧和撫諭思陸。和至騰衝,思陸遣陶孟投書,致方物。和諭以禍福,令掣兵過江,歸所占蠻莫等地,且調隴川、乾崖、南甸三宣撫司蠻兵及戰象,隨官軍分道至金沙江。思陸乃遣大陶孟倫索、怕卓等率所部來見,和等再申諭之。思陸聽命,退還前所據蠻莫等地十三處,撤回象馬蠻兵,渡金沙江而歸。又遣陶孟、招剛等貢象六、銀六百兩並金銀器納款。鎮巡官以聞,並奏言:「蠻莫等地原隸木邦,成化間始為孟密所有,近又為思陸所據,連年構禍,今始平定。既不可復與木邦、孟密,又不可割畀隴川、乾崖、南甸三宣撫,宜暫於騰衝歲檄官軍四百分番守之。思陸前有助平思揲功,今悔禍納款,請賜以名目、冠帶,仍降敕獎諭。」部議以蠻莫等處本木邦分地,在大義宜歸之木邦。其名目、冠帶,貢使已言思陸不願受,不宜輕畀,請賜敕厚勞遣歸之。報可。時思陸覬得宣慰司印,部執不予,於是仍數出兵與木邦、孟密仇殺無寧歲。

嘉靖七年,總兵官沐紹勛、巡撫歐陽重遣參政王汝舟等遍歷諸蠻,諭以禍福。孟養思倫等各願貢象牙、土錦、金銀器,退地贖罪。乃以蠻莫等十三處地方寬廣,諸蠻歷年所爭,屬之騰衝司,檄軍輪守,則煙瘴可虞;屬之木邦,則地勢遼遠,蠻心不順。莫若仍屬孟密管領,歲徵差發銀一千兩,而割孟乃等七處仍歸木邦罕烈,則分願均而忿爭息矣。報可。

萬曆五年,雲南巡按陳文燧言,孟養思個與緬世仇,今更歸順於緬。因引弘治朝先臣劉健嘗議孟養事狀,謂思陸有官猶可制,即無官,其僭自若也,不如因而官之以抗緬。報可。十一年,緬為遊擊劉綎所敗,孟養思威亦殺緬使降於糸廷。十三年,隴川平,乃於孟養立長官司。未幾,長官思真復為緬所擄,部長思遠奉思真妻來歸,給以冠帶,令歸守。思遠乘亂自立為宣慰,貢象進方物。然遠暴虐,諸部恨之,引緬兵至,聲言還思真,思遠奔盞西。有思轟者,內附,與蠻莫酋思正共據險抗緬。三十年,緬攻思正,轟率兵倍道馳救,至則正已被殺。三十二年,緬攻入迤西,轟走死,緬以頭目思華守其地。華死,妻怕氏代理。緬人更番戍守,連年徵發,從行甚苦,曰:「孟養不亡,蠻何得至此!」轟之後曰放思祖,有眾千餘,不敢歸,寄食於乾崖云。

舊制,宣慰遣人俱稱頭目,唯木邦及緬甸又有陶孟及招剛等稱,孟養又有招八稱,皆見於奏章,因其俗不改。

車里,即古產里,為倭泥、貂黨諸蠻雜居之地,古不通中國。元世祖命將兀良吉禋伐交阯,經所部,降之,置撒里路軍民總管府,領六甸,後又置耿凍路耿當、孟弄二州。洪武十五年,蠻長刀坎來降,改置車里軍民府,以坎為知府。坎遣姪豐祿貢方物,詔賜刀坎及使人衣服、綺幣甚厚,以初奉貢來朝故也。十七年復遣其子刀思拂來貢,賜坎冠帶、鈔幣,改置軍民宣慰使司,以坎為使。二十四年,子刀暹答嗣,遣人貢象及方物。二十八年以賜誥命謝恩,予賜皆如例。

永樂元年,刀暹答令其下剽掠威遠知州刀算黨及民人以歸。西平侯沐晟請發兵討,帝命晟移文諭之,如不悛,即以兵繼。又以車里已納威遠印,是悔過之心已萌,不必加兵。晟使至,暹答果懼,還刀算黨及威遠之地,遣人貢馬謝罪。帝以其能改過,宥之。自是頻入貢。朝廷遣內官往車里者,道經八百大甸,為宣慰刀招散所阻。三年,刀暹答遣使請舉兵攻八百,帝嘉其忠。八百伏罪,敕車里班師,復加獎勞。四年遣子刀典入國學,實陰自納質。帝知其隱,賜衣幣慰諭遣還,以道里遼遠,命三年一貢,著為令。十一年,暹答卒。長子刀更孟自立,驕狠失民心,未幾亦卒。更孟長子霸羨年幼,眾推刀賽署司事。刀賽者,更孟弟刀怕漢也。怕漢死,妻以前夫子刀弄冒為暹答孫,請襲。十五年命刀弄襲宣慰使,以更孟從弟刀雙孟為本司同知。十九年,雙孟言刀弄屢以兵侵劫蠻民,乞別設治所,以撫其眾。詔分其地,置靖安宣慰使司,升雙孟為宣慰使,命禮部鑄印給之。

宣德三年,雲南布政司奏刀弄、雙孟相仇殺,弄棄地投老撾,請差官招撫。帝命黔國公計議。六年,黔國公奏,謂奉命招撫刀弄,其母具言布政司差官劉亨征差發金,亨已取去,本司復來征,蠻民因而激變逐弄,弄逃入老撾,尋還境內以死。未嘗棄地外投,亦未嘗與雙孟仇殺。帝命法司執劉亨等罪之。七年,車里土舍刀霸羨請襲,許之,遣行人陸塤齎敕賜冠帶、襲衣。九年,靖安宣慰刀霸供言:「靖安原車里地,今析為二,致有爭端,乞仍併為一,歲貢如例。」帝從其請,革靖安宣慰,仍歸車里,命刀霸供、刀霸羨共為宣慰使,俾上所授靖安宣慰司印。

正統五年命貢使齎敕及綺帛歸賜刀霸羨及妻,嘉其勤修職貢也。六年,麓川宣慰思倫發叛,詔給車里信符、金牌,命合兵剿賊。景泰三年以刀霸羨奉調有功,免其積欠差發金。天順元年,總兵官沐璘奏:「刀霸羨自殺,弟板雅忠等已推兄三寶歷代承職。今板雅忠又作亂,糾合八百相仇殺。」帝命璘亟為撫諭,並勘奏應襲者。二年,帝以三寶歷代者,雖刀更孟之子,乃庶孽奪嫡,謀害刀霸羨,致板雅忠借兵攻殺,不當襲。但蠻民推立,姑從眾願,命襲宣慰使。

成化十六年,交阯黎灝叛,頒偽敕於車里,期會兵共攻八百,車里持兩端。雲南守臣以聞,遣使敕車里諸土官互相保障,勿懷二心。二十年復敕車里等部,懼固封疆,防交人入寇,不得輕與文移,啟釁納侮。嘉靖十一年,緬酋莽應裏據擺古,蠶食諸蠻。車里宣慰刀糯猛折而入緬,有大、小車里之稱,以大車里應緬,而以小車里應中國。萬曆十三年命元江土舍那恕往招,糯猛復歸,獻馴象、金屏、象齒諸物,謝罪。詔受之,聽復職。

天啟七年,巡撫閔洪學奏,緬人侵孟艮,孟艮就車里求救,宣慰刀韞猛遣兵象萬餘赴之。緬人以是恨車里,興兵報復,韞猛年已衰,重賂求和。緬聞韞猛子召河璇有女名召烏岡色美,責獻烏岡。河璇別以女紿之。緬知其詐,大憤,攻車里愈急。韞猛父子不能支,遁至思毛地,緬追執之以去。中朝不及問,車里遂亡。

老撾,俗呼為撾家,古不通中國。成祖即位,老撾土官刀線歹貢方物,始置老撾軍民宣慰使司。永樂二年以刀線歹為宣慰使,給之印。五年遣人來貢。既而帝以刀線歹潛通安南季犛,遣使詰責,諭其悔過。六年,刀線歹遣人貢象馬、方物。七年復進金銀器、犀象、方物謝罪。自是連年入貢,皆賚予如例。帝遣中官楊琳往賜文綺。十年來貢,命禮部加賜焉。

宣德六年遣使齎敕獎諭宣慰刀線達。九年,老撾貢使還,恐道中為他部所阻,給信符,敕孟艮、車里諸部遣人護之。景泰元年請賜土官衣服。故事,無加賜衣服者,命加賜錦幣並及其妻。成化元年頒金牌、信符於老撾。七年鑄給老撾軍民宣慰使司印,以皆為賊焚毀也。十六年,貢使至,會安南攻老撾,鎮守內官錢能以聞。因敕其使兼程回,並量給道里費。明年,安南黎灝率兵九萬,開山為三道,進兵破哀牢,入老撾境,殺宣慰刀板雅及其子二人。其季子怕雅賽走八百,宣慰刀攬那遣兵送至景坎。黔國公沐琮以聞,命怕雅賽襲父職,免其貢物一年,賜冠帶、彩幣,以示優恤。既怕雅賽欲報安南之仇,覬中國發兵為助。帝以老撾、交阯皆服屬中國久,恤災解難,中國體也,令琮慎遣人諭之。

弘治十一年,宣慰舍人招攬章應襲職,遣人來貢,因請賜冠帶及金牌、信符。賚賞如制,其金牌、信符,俟鎮巡官勘奏至日給之。十一月,招攬章遣使入貢。吏部言:「招攬章係舍人,未授職,僭稱宣慰使,雲南三司官冒奏違錯,宜治罪。」宥之。

嘉靖九年,招攬章言:「交阯應襲長子光紹,為叔所逐,出亡老撾,欲調象馬送回。」守臣言:「據招攬章之言,懼納亡之罪,且假我為制服之資,留之啟釁,遣之招兵,宜聽光紹自歸,並責其私納罪。」報可。二十四年,雲南巡撫汪文盛言:「老撾土舍怕雅聞征討安南,首先思奮,且地廣兵多,可獨當一面。八百、車里與老撾相近,孟艮在老撾上流,皆多兵象,可備征討。請免其察勘,就令承襲,以備徵調。」從之。四十四年,土舍怕雅蘭章遣人進舞牌牙象二、母象三、犀角十,雲南守臣以聞。禮部以非貢期,且無漢、緬公文,第來路險遠,跋涉踰年,宜受其所貢,給賞遣之,毋令赴京。報可。時緬勢方張,剪除諸部,老撾亦折而入緬,符印俱失。

萬曆二十六年,緬敗,老撾來歸,奉職貢,請頒印。命復鑄老撾軍民宣慰使司印給之。四十年貢方物,言印信毀於火,請復給,撫鎮官以聞。明年再頒老撾印。時宣慰猶貢象及銀器、緬席,賜予如例。自是不復至云。其俗與木邦同,部長不知姓,有三等:一曰招木弄,一曰招木牛,一曰招木化。而為宣慰者,招木弄也,代存一子,絕不嗣。其地東至水尾,南至交阯,西至八百,北至車里,西北六十八程至雲南布政司。

八百,世傳部長有妻八百,各領一寨,因名八百媳婦。元初征之,道路不通而還,後遣使招附。元統初,置八百等處宣慰司。洪武二十一年,八百媳婦國遣人入貢,遂設宣慰司。二十四年,八百土官刀板冕遣使貢象及方物。先是,西平侯沐英遣雲南左衛百戶楊完者往八百招撫,至是來貢。帝諭兵部尚書茹瑋曰:「聞八百與百夷構兵,仇殺無寧日。朕念八百宣慰遠在萬里外,能修職奉貢,深見至誠。今與百夷構兵,當有以處之。可諭意八百,令練兵固守,俟王師進討。」自是及永樂初,頻遣使入貢,賜予如例。

永樂二年設軍民宣慰使司二,以土官刀招你為八百者乃宣慰使,其弟刀招散為八百大甸宣慰使,遣員外郎左洋往賜印誥、冠帶、襲衣。刀招散遣人貢馬及方物謝恩,命五年一朝貢。是歲,遣內官楊瑄齎敕諭孟定、孟養等部,道經八百大甸,為土官刀招散所阻,弗克進。三年遣使諭刀招散曰:「朕特頒金字紅牌,敕諭與諸邊為信,以禁戢邊吏生事擾害,用福爾眾。諸宣慰皆敬恭聽命,無所違禮。惟爾年幼無知,惑於小人孟乃朋、孟允公等,啟釁生禍,使臣至境,拒卻不納。廷臣咸請興師問罪,朕念八百之人豈皆為惡,兵戈所至,必及無辜,有所不忍。茲特遣司賓田茂、推官林楨齎敕往諭,爾能悔過自新,即將奸邪之人擒送至京,庶境土可保。其或昏迷不悛,發兵討罪,孥戮不貸!」並敕西平侯沐晟嚴兵以待。以馬軍六百、步軍一千四百護內官楊安、郁斌前往。又慮老撾乘車里空虛,或發兵掩襲,或與八百為援,可遣其部長率兵一萬五千往備。三年,刀招你等遣使奉金縷表文,貢金結絲帽及方物。帝命受之,仍加賜予。西平侯沐晟奏:「奉命率師及車里諸宣慰兵至八百境內,破其猛利石厓及者答二寨,又至整線寨。木邦兵破其江下等十餘寨。八百恐,遣人詣軍門伏罪。」乃以所陳詞奏聞。因遣使敕諭車里、木邦等曰:「曩者八百不恭朝命,爾等請舉兵誅討。嘉爾忠誠,已從所請。今得西平侯奏,言八百已伏罪納款。夫有罪能悔,宜赦宥之。敕至,其悉止兵勿進。」遂敕晟班師。四年降敕誡諭刀招散,刀招散遣人貢方物謝罪。帝以不誠,卻之。五年貢使復來謝罪,命禮部受之。

洪熙元年遣內官洪仔生齎敕諭刀招散。宣德七年遣人來貢,因奏波勒土酋常糾土雅之兵入境殺掠,乞發兵討之。帝以八百大甸去雲南五千餘里,波勒、土雅皆未嘗歸化,勞中國為遠蠻役,非計,止降敕撫諭而已。

正統五年,八百貢使奏:「遞年進貢方物,土民不識禮法,不通漢語。乞依永樂間例,仍令通事齎捧金牌、信符,催督進貢,驛路令軍卒護送,庶無疏失。」從之。十年,給八百大甸宣慰司金牌、信符各一,以前所給牌符為暹羅國寇兵焚毀也。

成化十七年,安南黎灝已破老撾,頌偽敕於車里,期會兵攻八百。其兵暴死者數千,傳言為雷所震。八百因遣兵扼其歸路,襲殺萬餘,交敗還。土官刀攬那以報。黔國公沐琮奏:「攬那能保障生民,擊敗交賊,救護老撾。交人嘗以偽敕脅誘八百,八百毀敕,以象蹴之,請頒賞以旌忠義。」帝命雲南布政司給銀百兩、綵幣四表裏以獎之。二十年,刀攬那遣人入貢。雲南守臣言:「交兵雖退,宜令八百諸部飭兵為備。」弘治二年,刀攬那孫刀整賴貢方物,求襲祖職。兵部言:「八百遠離雲南,瘴毒之地,宜免勘予襲。」從之,仍給冠帶。其地東至車里,南至波勒,西至大古喇,與緬鄰,北至孟艮,自姚關東南行五十程始至。平川數千里,有南格剌山,下有河,南屬八百,北屬車里。好佛惡殺,寺塔以萬計。有見侵,乃舉兵,得仇即已,俗名慈悲國。嘉靖間,為緬所並,其酋避居景線,名小八百。自是朝貢遂不至。緬酋應里以弟應龍居景邁城,倚為右臂焉。萬歷十五年,八百大甸上書請恢復,不報。初,四譯館通事惟譯外國,而緬甸、八百如之,蓋二司於六慰中加重焉。


\end{pinyinscope}