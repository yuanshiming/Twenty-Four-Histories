\article{列傳第十九}

\begin{pinyinscope}
顧時吳禎薛顯郭興陳德王志梅思祖金朝興唐勝宗陸仲亨費聚陸聚鄭遇春黃彬葉昇

顧時,字時舉,濠人。倜儻好奇略。從太祖渡江,積功由百夫長授元帥。取安慶、南昌、廬州、泰州,擢天策衛指揮同知。李濟據濠州,時從平章韓政討降之。攻張士誠昇山水寨,引小舫繞敵舟,舟中多俯視而笑。時乘其懈,帥壯士數人,大呼躍入舟。眾大亂,餘舟競進。五太子來援,薛顯又敗之,五太子等降。遂從大將軍平吳,旋師取山東。

洪武元年,拜大都督府副使兼同知率府事。從大將軍定河南北,浚閘以通舟師,自臨清至通州。下元都,與諸將分邏古北諸隘口。從大軍取平陽,克崞州,獲逃將王信等四十六人。取蘭州,圍慶陽。張良臣耀兵城下,擊敗之,獲其勁將九人。良臣乃不敢復出。慶陽平。徐達還京,令時將騎兵略靜寧州,走賀宗哲。西邊悉平。

三年進大都督同知,封濟寧侯,祿千五百石,予世券。四年為左副將軍,副傅友德帥河南、陜西步騎伐蜀。自興元進克階、文,敗蜀兵於漢州,遂克成都。明年副李文忠北征,分道入沙漠。迷失道,糧且盡。遇寇,士疲不能戰。時帥麾下數百人,躍馬衝擊。敵眾引去,獲其輜重糧畜以歸,軍聲大振。六年從徐達鎮北平。踰年,召還。八年復出鎮。十二年卒,年四十六。葬鐘山。追封滕國公,謚襄靖,祔祭功臣廟。

時能以少擊眾,沉鷙不伐。帝甚重之。子敬,金吾衛鎮撫,十五年嗣侯,為左副將軍。平龍泉山寇有功。二十三年追論胡惟庸黨,榜列諸臣,以時為首,敬坐死,爵除。

吳禎,江國襄烈公良弟也。初名國寶,賜名禎。與良俱從克滁、和,渡江克採石,從定集慶。下鎮江、廣德、常州、宣城、江陰,皆有功。又從常遇春自銅陵取池州,以舟師毀其北門,入城。敵艦百餘至,復大敗之,遂克池州。積功,由帳前都先鋒累遷為天興翼副元帥。以千人助良守江陰,數敗吳兵,破士誠水寨,擒其驍將硃定。授英武衛親軍指揮使。又大破吳兵於浮子門。從大將軍徐達帥馬步舟師取湖州,勒奇兵出舊館,大捷。湖州平,遂戍之。從圍平江,破葑、胥二門,進僉大都督府事,撫平江。尋副征南將軍湯和討方國珍,乘潮入曹娥江,毀壩通道,出不意直抵軍廄。國珍亡入海。追及之盤嶼,合戰,自申至戌,敗之,盡獲其戰艦士卒輜重,國珍降。復自海道進取福州,圍其西、南、水部三門,一鼓克之。

洪武元年,進兵破延平,擒陳友定。閩海悉平。還次昌國。會海寇劫蘭秀山,剿平之。兼率府副使。尋為吳王左相兼僉大都督府事。二年,大將軍平陜西還,禎與副將軍馮勝駐慶陽。三年討平沂州答山賊。命為靖海將軍,練軍海上。其冬,封靖海侯,食祿千五百石,予世券。與秦、晉二王傅金朝興、汪興祖並專傅王,解都督府事。仇成戍遼陽,命禎總舟師數萬,由登州餉之。海道險遠,經理有方,兵食無乏。完城練卒,盡收遼海未附之地,降平章高家奴等。坐事謫定遼衛指揮使,尋召還。七年,海上有警,復充總兵官,同都督僉事於顯總江陰四衛舟師出捕倭。至琉球大洋,獲其兵船,獻俘京師。自是常往來海道,總理軍務數年,海上無寇。

十一年,奉詔出定遼,得疾,輿還京師。明年卒。追封海國公,謚襄毅,與良俱肖像功臣廟。子忠嗣侯。二十三年追論禎胡惟庸黨,爵除。

薛顯,蕭人。趙均用據徐州,以顯為元帥,守泗州。均用死,以泗州來降,授親軍指揮,從征伐。南昌平,命顯從大都督朱文正守之。陳友諒寇南昌,顯守章江、新城二門。友諒攻甚急。顯隨方禦之,間出銳卒搏戰,斬其平章劉進昭,擒副樞趙祥。固守三月,乃解。武昌既平,鄧仲謙據新淦不下,顯討斬之,因徇下未附諸郡縣。以功擢江西行省參政。從徐達等收淮東,遂伐張士誠。與常遇春攻湖州。別將游軍取德清,攻升山水寨。士誠遣其五太子盛兵來援,遇春與戰,小卻。顯帥舟師奮擊,燒其船。眾大潰,五太子及朱暹、呂珍等以舊館降,得兵六萬人。遇春謂顯曰:「今日之戰,將軍功,遇春弗如也。」五太子等既降,吳人震恐,湖州遂下。進圍平江,與諸將分門而軍。吳平,進行省右丞。

命從大將軍徐達取中原。瀕行,太祖諭諸將,謂「薛顯、傅友德勇略冠軍,可當一面。」進克兗、沂、青、濟,取東昌、棣州,樂安。還收河南,搗關、陜。渡河,取衛輝、彰德、廣平、臨清。帥馬步舟師取德州、長蘆。敗元兵於河西務,又敗之通州,遂克元都。分兵邏古北諸隘口,略大同,獲喬右丞等三十四人。進征山西,次保定,取七垛寨,追敗脫因貼木兒。與友德將鐵騎三千,略平定西。取太原,走擴廓,降豁鼻馬。邀擊賀宗哲於石州,拔白崖、桃花諸山寨。與大將軍達會平陽,以降將杜旺等十一人見,遂從入關中。抵臨洮,別將攻馬鞍山西番寨,大獲其畜產,襲走元豫王,敗擴廓於寧夏。復與達會師取平涼。張良臣偽以慶陽降,顯往納之。良臣蒲伏道迎,夜劫顯營,突圍免。良臣據城叛,達進圍之。擴廓遣韓扎兒攻原州,以撓明師。顯駐兵靈州,遏之。良臣援絕,遂敗。追賀宗哲於六盤山,逐擴廓出塞外,陜西悉平。

洪武三年冬,大封功臣。以顯擅殺胥吏、獸醫、火者、馬軍及千戶吳富,面數其罪。封永城侯,勿予券,謫居海南。分其祿為三:一以贍所殺吳富及馬軍之家;一以給其母妻,令功過無相掩。顯居海南踰年,帝念之,召還。予世券,食祿一千五百石。

復從大將軍征漠北。數奉命巡視河南,屯田北平,練軍山西,從魏國公巡北邊,從宋國公出金山。二十年冬,召還,次山海衛,卒。贈永國公,謚桓襄。無子,弟綱幼。二十三年追坐顯胡惟庸黨,以死不問,爵除。

郭興,一名子興,濠人。滁陽王郭子興據濠,稱元帥,與隸麾下。太祖在甥館,興歸心焉。軍行,嘗備宿衛,累功授管軍總管,進統軍元帥。圍常州,晝夜不解甲者七月。城下,受上賞。從攻寧國、江陰、宜興、婺州、安慶、衢州,皆下之。戰於鄱陽,陳友諒連巨艦以進,我師屢卻,興獻計以火攻之。友諒死。從征武昌,斬獲多,進鷹揚衛指揮使。從徐達取廬州,援安豐,大敗張士誠兵。平襄陽、衡、澧。還克高郵、淮安。轉戰湖州,圍平江,軍於婁門。吳平,擢鎮國將軍、大都督府僉事。

洪武元年,從達取中原,克汴梁,守禦河南。馮勝取陜州,請益兵守潼關。達曰:「無如興者。」遂調守之。潼關,三秦門戶,時哈麻圖據奉元,李思齊、張思道等與為犄角,日窺伺欲東向。興悉力捍禦。王左丞來攻,大敗之。從徐達帥輕騎直搗奉元。大軍繼進,遂克之。移鎮鞏昌,邊境帖然。

三年為秦王武傅,兼陜西行都督府僉事。其冬,封功臣,興以不守紀律,止封鞏昌侯,食祿一千五百石,予世券。四年伐蜀,克漢州、成都。六年從徐達鎮北平,同陳德敗元兵於答刺海口。十一年練兵臨清。十六年巡北邊。召還,踰年卒。贈陜國公,謚宣武。二十三年追坐胡惟庸黨,爵除。

興女弟為寧妃,弟英武定侯。

季弟德成,性通敏,嗜酒。兩兄積功至列侯,而德成止驍騎舍人。太祖以寧妃故,欲貴顯之。德成辭。帝不悅。頓首謝曰:「臣性耽曲糵,庸暗不能事事。位高祿重,必任職司,事不治,上殆殺我。人生貴適意,但多得錢、飲醇酒足矣,餘非所望。」帝稱善,賜酒百罌,金幣稱之,寵遇益厚。嘗侍宴後苑醉,匍匐脫冠謝。帝顧見德成髮種種,笑曰:「醉風漢,髮如此,非酒過耶?」德成仰首曰:「臣猶厭之,盡薙始快。」帝默然。既醒,大懼。佯狂自放,剃髮、衣僧衣,唱佛不已。帝謂寧妃曰:「始以汝兄戲言,今實為之,真風漢也。」後黨事起,坐死者相屬,德成竟得免。

陳德,字至善,濠人。世農家,有勇力。從太祖於定遠,以萬夫長從戰,皆有功,為帳前都先鋒。同諸將取寧、徽、衢、婺諸城,擢元帥。李伯昇寇長興,德往援,擊走之。從援南昌,大戰鄱陽湖,擒水寨姚平章。太祖舟膠淺,德力戰,身被九矢,不退。從平武昌。大敗張士誠兵於舊館,擢天策衛親軍指揮使。吳平,進僉大都督府事。從大將軍北取中原,克元汴梁。立河南行都督府,以德署府事,討平群盜。征山西,破澤州磨盤寨,獲參政喻仁,遂會大軍克平陽、太原、大同。渡河取奉元、鳳翔,至秦州,元守將呂國公遁,追擒之。徐達圍張良臣於慶陽,良臣恃其兄思道為外援,間使往來,德悉擒獲,慶陽遂下。又大破擴廓於古城,降其卒八萬。

洪武三年,封臨江侯,食祿一千五百石,予世券。明年,從潁川侯傅友德伐蜀,分道入綿州,破龍德,大敗吳友仁之眾,乘勝拔漢州。向大亨、戴壽等走成都,追敗之,遂與友德圍成都。蜀平,賜白金彩幣。復還汴。五年為左副將軍,與馮勝徵漠北,破敵於別篤山,俘斬萬計。克甘肅,取亦集乃路,留兵扼關而還。明年復總兵出朔方,敗敵三岔山,擒其副樞失剌罕等七十餘人。其秋,再出戰於答剌海口,斬首六百級,獲其同僉忻都等五十四人。凡三戰三捷。七年練兵北平。十年還鳳陽。十一年卒。追封杞國公,謚定襄。

子鏞襲封。十六年為征南左副將軍,討平龍泉諸山寇。練兵汴梁。十九年與靖海侯吳禎城會州。二十年從馮勝徵納哈出,將至金山,與大軍異道相失,敗沒。二十三年,追坐德胡惟庸黨,詔書言其征西時有過,被鐫責,遂與惟庸通謀。爵除。

王志,臨淮人。以鄉兵從太祖於濠,下滁、和。從渡江,屢騰柵先登,身冒矢石。授右副元帥。從取常州、寧國、江陰。復宜興,攻高郵,搗九江,下黃梅,鏖戰鄱陽。從平武昌,還克廬州,敗張士誠兵,追奔四十里。以親軍衛指揮使改六安衛,守六安。從幸汴梁,渡河,取懷慶、澤、潞,留守平陽。大將軍徐達西伐,會師克興元。洪武三年,進同知都督府事,封六安侯,歲祿九百石,予世券。移守漢中,帥兵出察罕腦兒塞,還鎮平陽。復從大將軍征沙漠。其後用兵西南,皆以偏將軍從,雖無首功,然持重,未嘗敗衄。其攻合肥,敗樓兒張,擒吳副使,為戰功第一。領山西都司衛所軍務,帝稱其處置得宜。十六年督兵往雲南品甸。繕城池,立屯堡,置驛傳,安輯其民。十九年卒。追封許國公,謚襄簡。

子威,二十二年嗣侯。明年,坐事謫安南衛指揮使。卒,無子。弟琙嗣,改清平衛,世襲。志亦追坐胡惟庸黨,以死不問。

梅思祖,夏邑人。初為元義兵元帥,叛從劉福通。擴廓醢其父。尋棄福通,歸張士誠,為中書左丞,守淮安。徐達兵至,迎降,並獻四州。士誠殺其兄弟數人。太祖擢思祖大都督府副使。從大軍伐吳,克升山水寨。下湖州,圍平江,皆有功。吳平,遷浙江行省右丞。從大將軍伐中原,克山東,取汴、洛,破陜州,下潼關。旋師徇河北,至衛輝。元平章龍二棄城走彰德,師從之。龍二復出走,遂降其城,守之。略定北平未下州郡。從大軍平晉、冀,復從平陜西。別將克邠州,獲元參政毛貴等三十人。從大將軍破擴廓於定西。還自秦州,破略陽,入沔州,取興元。洪武三年,論功封汝南侯,食祿九百石,予世券。四年伐蜀。五年征甘肅。還命巡視山、陜、遼東城池。十四年,四川水盡源、通塔平、散毛諸洞長官作亂,命思祖為征南副將雲南軍,與江夏侯周德興帥兵討平之。十五年復與傅友德平雲南,置貴州都司,以思祖署都指揮使。尋署雲南布政司事,與平章潘元明同守雲南。思祖善撫輯,遠人安之。是年卒,賜葬鐘山之陰。

子義,遼東都指揮使。二十三年追坐思祖胡惟庸黨,滅其家。思祖從子殷,為駙馬都尉,別有傳。

金朝興,巢人。淮西亂,聚眾結寨自保。俞通海等既歸太祖,朝興亦帥眾來附。從渡江,征伐皆預,有功。克常州,為都先鋒。復宜興,為左翼副元帥。平武昌,進龍驤衛指揮同知。平吳,改鎮武衛指揮使。克大同,改大同衛指揮使。取東勝州,獲元平章劉麟等十八人。

洪武三年,論功為都督僉事兼秦王左相。未幾,解都督府事,專傅王。四年從大軍伐蜀。七年帥師至黑城,獲元太尉盧伯顏、平章帖兒不花並省院等官二十五人。遂從李文忠分領東道兵,取和林,語具文忠傳。

朝興沉勇有智略,所至以偏師取勝,雖未為大帥,而功出諸將上。十一年從沐英西征,收納鄰七站地。明年論功封宣德侯,祿二千石,世襲指揮使。十五年從傅友德征雲南,駐師臨安,元右丞兀卜台、元帥完者都、土酋楊政等俱降。朝興撫輯有方,軍民咸悅。進次會川卒,追封沂國公,謚武毅。十七年論平雲南功,改錫世侯券,增祿五百石。

長子鎮嗣封。二十三年追坐朝興胡惟庸黨,降鎮平壩衛指揮使。從征有功,進都指揮使。其後世襲衛指揮使。嘉靖元年,命立傅友德、梅思祖及朝興廟於雲南,額曰「報功」。

唐勝宗,濠人。太祖起兵,勝宗年十八,來歸。從渡江,積功為中翼元帥。從徐達克常州,進圍寧國,扼險力戰,敗其援兵。城遂降。從征婺州,克之。從征池州,力戰,敗陳友諒兵,擢龍驤衛指揮僉事。從征友諒,至安慶,敵固守。勝宗為陸兵疑之,出不意,搗克其水寨。從下南昌,略定江西諸郡。援安豐,攻廬州,戰鄱陽,邀擊涇江口,皆有功。擢驃騎衛指揮同知。從定武昌,徇長沙、沅陵、澧陽。從徐達取江陵,還定淮東。穴城克安豐,追獲元將忻都。為安豐衛指揮使守之。從大將軍伐中原,克汴梁、歸德、許州,輒留守。從大軍克延安,進都督府同知。洪武三年冬封延安侯,食祿千五百石,予世券。坐擅馳驛騎,奪爵,降指揮。捕代縣反者。久之,復爵。

十四年,浙東山寇葉丁香等作亂,命總兵討之,擒賊首併其黨三千餘人。分兵平安福賊,至臨安,降元右丞兀卜台等。十五年巡視陜西,督屯田,簡軍士。明年鎮遼東,奉敕勿通高麗。高麗使至,察其奸,表聞。賜敕褒美,比魏田豫卻烏桓賂,稱名臣。在鎮七年,威信大著。召還,帥師討平貴州蠻。練兵黃平。二十三年坐胡惟庸黨誅,爵除。

陸仲亨,濠人。歸太祖,從征滁州,取大柳樹諸寨。克和陽,擊敗元兵。逐青山群盜。從渡江,取太平,定集慶,從徐達下諸郡縣。授左翼統軍元帥。從征陳友諒,功多,進驃騎衛指揮使。從常遇春討贛州,降熊天瑞,為贛州衛指揮使,節制嶺南北新附諸郡。調兵克梅州、會昌、湘鄉,悉平諸山寨。

洪武元年,帥衛軍與廖永忠等征廣東,略定諸郡縣,會永忠於廣州,降元將盧左丞。廣東平。改美東衛指揮使,擢江西行省平章,代鄧愈鎮襄陽,改同知都督府事。三年冬封吉安侯,祿千五百石,予世券。與唐勝宗同坐事降指揮使。捕寇雁門,同復爵。

十二年與周德興、黃彬等從湯和練兵臨清。未幾,即軍中逮三人至京,既而釋之。移鎮成都,平巨津州叛蠻。烏撒諸蠻復叛,從傅友德討平之。

二十三年,治胡惟庸逆黨,家奴封貼木告仲亨與勝宗、費聚、趙庸皆與通謀,下吏訊。獄具,帝曰:「朕每怪其居貴位有憂色。」遂誅仲亨,籍其家。

初,仲亨年十七,為亂兵所掠。父母兄弟俱亡,持一升麥伏草間。帝見之,呼曰「來」,遂從征伐,至封侯。帝嘗曰:「此我初起時腹心股肱也。」竟誅死。

費聚,字子英,五河人。父德興,以材勇為游徼卒。聚少習技擊。太祖遇於濠,偉其貌,深相結納。

定遠張家堡有民兵,無所屬,郭子興欲招之,念無可使者。太祖力疾請行,偕聚騎而往,步卒九人俱。至寶公河,望其營甚整,弓弩皆外向。步卒懼,欲走。太祖曰:「彼以騎蹴我,走將安往?」遂前抵其營。招諭已定,約三日。太祖先歸,留聚俟之。其帥欲他屬,聚還報。太祖復偕聚以三百人往,計縛其帥,收卒三千人。豁鼻山有秦把頭八百餘人,聚復招降之。遂從取靈璧,克泗、滁、和州。授承信校尉。

既定江東,克長興,立永興翼元帥府,以聚副耿炳文為元帥。張士誠入寇,擊敗之。召領宿衛。援安豐,兩定江西,克武昌,皆從。改永興翼元帥府為永興親軍指揮司,仍副炳文為指揮同知。士誠復入寇,獲其帥宋興祖,再敗之。士誠奪氣,不敢復窺長興。隨征淮安、湖州、平江,皆有功,進指揮使。湯和討方國珍,聚以舟師從海道邀擊。浙東平,復由海道取福州,破延平。歸次昌國,剿海寇葉、陳二姓於蘭秀山。至是,聚始獨將。洪武二年會大軍取西安,改西安衛指揮使,進都督府僉事。鎮守平涼。三年封平涼侯,歲祿千五百石,予世券。

時諸將在邊屯田募伍,歲有常課。聚頗耽酒色,無所事事。又以招降無功,召還,切責之。明年從傅友德征雲南,大戰白石江,擒達里麻。雲南平,進取大理。未幾,諸蠻復叛,命副安陸侯吳復為總兵。授以方略,分攻關索嶺及阿咱等寨,悉下之。蠻地始定。置貴州都指揮使司,以聚署司事。十八年命為總兵官,帥指揮丁忠等征廣南,擒火立達,俘其眾萬人。還鎮雲南。二十三年召還。李善長敗,語連聚。帝曰:「聚曩使姑蘇不稱旨,朕嘗詈責,遂欲反耶!」竟坐黨死,爵除。

子超,徵方國珍,沒於陣。璿,以人材舉官江西參政。孫宏,從征雲南,積功為右衛指揮使。坐奏對不實,戍金齒。

陸聚,不知何許人。元樞密院同知。脫脫敗芝蔴李於徐州,彭大等奔濠。聚撫戢流亡,繕城保境,寇不敢犯。徐達經理江、淮,聚以徐、宿二州降。太祖嘗詔諭:「二州,吾桑梓地,未忍加兵。」及歸附,大悅。以聚為江南行省參政,仍守徐州。遣兵略定沛、魚臺、邳、蕭、宿遷、睢寧。擴廓遣李左丞侵徐,駐陵子村。聚遣指揮傅友德擊之,俘其眾,擒李左丞。又敗元兵於宿州,擒僉院邢端等。從定山東,平汴梁。還鎮,改山東行省參政。從平元都,略大同、保定、真定。攻克車子山及鳳山、城山、鐵山諸寨,分守井陘故關,會師陜西,克承天寨。聚所部皆淮北勁卒,雖燕、趙精騎不及也。北征,沂、邳山民乘間作亂,召聚還,討平之。洪武三年,封河南侯,歲祿九百石,予世券。八年,同衛國公愈屯田陜西,置衛戍守。十二年同信國公和練兵臨清。尋理福建軍務。召還,賜第鳳陽。二十三年坐胡惟庸黨死,爵除。

鄭遇春,濠人。與兄遇霖俱以勇力聞。遇霖與里人有郤,欲殺之,遇春力護,得解取。眾皆畏遇霖,而以遇春為賢。太祖下滁州,遇霖為先鋒。取鐵佛岡、三郤河、大柳等寨,遇春亦累功至總管。攻蕪湖,遇霖戰死,遇春領其眾。時諸將所部不過千人,遇春兼兩隊,而所部尤驍果。累戰功多,授左翼元帥。從平陳友諒,身先士卒,未嘗自言功,太祖異之。取六安,為六安衛指揮僉事。從大將軍定山東、河南北,克朔州,改朔州衛指揮副使。

洪武三年,進同知大都督府事,封滎陽侯,歲祿九百石,予世券。明年命駐臨濠,開行大都督府。坐累奪爵。尋復之,復守朔州。從傅友德平雲南,帥楊文等經略城池屯堡。還京,督金吾諸衛,造海船百八十艘,運餉遼東,籍陜西岷州諸衛官馬。二十三年坐胡惟庸黨死。爵除。

黃彬,江夏人。從歐普祥攻陷袁、吉屬縣,徐壽輝以普祥守袁州。及陳友諒殺壽輝,僭偽號,彬言於普祥曰:「公與友諒比肩,奈何下之?友諒驕恣,非江東敵也。保境候東師,當不失富貴。」普祥遂遣使納款。友諒遣弟友仁攻之。彬與普祥敗其眾,獲友仁。友諒懼,約分界不相犯,乃釋友仁。時江、楚諸郡皆為陳氏有,袁扼其要害,潭、岳、贛兵不得出。友諒勢大蹙。太祖兵臨之,遂棄江州,彬力也。太祖至龍興,令普祥仍守袁州,而以彬為江西行省參政。未幾,普祥死,彬領其眾。普祥故殘暴,彬盡反所為,民甚安之。從常遇春征贛州。饒鼎臣據吉安,為熊天瑞聲援。遇春兵至,鼎臣走安福,彬以兵躡之。鼎臣走茶陵,天瑞乃降。永新守將周安叛,彬從湯和執安,鼎臣亦殪。移鎮袁州,招集諸山寨。江西悉定。進江淮行省中書左丞。洪武三年封宜春侯,歲祿九百石,予世券。四年,贛州上猶山寇叛,討平之。五年,古州等洞蠻叛,以鄧愈為征南將軍,三道出師,彬與營陽侯璟出澧州。師還,賜第中都。明年從徐達鎮北平,出練兵沂州、臨清。二十三年坐胡惟庸黨死,爵除。

葉昇,合肥人。左君弼據廬,升自拔來歸。以右翼元帥從征江州,以指揮僉事從取吳,以府軍衛指揮使從定明州。洪武三年論功,僉大都督府事。明年從征西將軍湯和以舟師取蜀。越二年,出為都指揮使,鎮守西安,討平慶陽叛寇。十二年復僉大都督府事。西番叛,與都督王弼征之,降乞失迦,平其部落。復討平延安伯顏帖木兒,擒洮州番酋。論功,封靖寧侯,歲祿二千石,世指揮使。鎮遼東,修海、蓋、復三城。在鎮六年,邊備修舉,外寇不敢犯。發高麗賂遺,帝屢賜敕,與唐勝宗同褒。

二十年,命同普定侯陳桓總制諸軍於雲南定邊、姚安,立營、屯田,經理畢節衛。明年,東川、龍海諸蠻叛,昇以參將從沐英討平之。已而湖廣安福所千戶夏德忠誘九溪洞蠻為寇,升同胡海等討之。潛兵出賊後,掩擊,擒德忠。立永定、九溪二衛,因留屯襄陽。贛州山賊復結湖廣峒蠻為寇。升為副將軍,同胡海等討平之,俘獲萬七千人。升凡三平叛蠻。再出練兵甘肅、河南。二十五年八月,坐交通胡惟庸事覺,誅死。涼國公藍玉,昇姻也,玉敗,復連及昇,以故名隸兩黨云。

贊曰:諸將當草昧之際,上觀天命,委心明主,戰勝攻取,克建殊勳,皆一時之智勇也。及海內寧謐,乃名隸黨籍,或追論,或身坐,鮮有能自全者。圭裳之錫固足酬功,而礪帶之盟不克再世,亦可慨矣夫。


\end{pinyinscope}