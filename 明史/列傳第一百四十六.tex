\article{列傳第一百四十六}

\begin{pinyinscope}
許譽卿華允誠魏呈潤胡良機李曰輔趙東曦毛羽健黃宗昌韓一良吳執御吳彥芳王績燦章正宸黃紹傑李世祺傅朝佑

莊鰲獻李汝璨姜埰弟垓熊開元方士亮詹爾選湯開遠成勇陳龍正

許譽卿,字公實,華亭人。萬曆四十四年進士,授金華推官。

天啟三年,徵拜吏科給事中。疏言錦衣世職,不當濫畀保姆奄尹。織造中官李實誣劾蘇州同知楊姜侵撫按職。中旨謂姜賄譽卿出疏,停譽卿俸半年。楊漣劾魏忠賢,譽卿亦抗疏極論忠賢大逆不道:「視漢之朋結趙嬈,唐之勢傾中外,宋之典兵矯詔、謀間兩宮何異!」忠賢大怒。又言:「內閣政本重地,而票擬大權拱手授之內廷。廠衛一奉打問之旨,五毒備施。邇復用立枷法,士民槁項斃者不知凡幾。又行數十年不行之廷杖,流毒縉紳,豈所以昭君德哉!祖制,宦官不典兵。今禁旅日繁,內操未罷,聚虎狼於蕭牆之內,逞金革於禁闥之中,不為早除,必貽後患。」於是忠賢怒益甚。會趙南星、高攀龍被逐,譽卿偕同列論救,遂鐫秩歸。

莊烈帝即位,誅崔、魏,將大計天下吏。奄黨房壯麗、安伸、楊維垣之徒冀收餘燼,屢詔起廢,輒把持使不得進,引其同類。譽卿時已起兵科給事中,具疏爭。吏部尚書王永光素附璫,仇東林,尤陰鷙。詔定逆案,頌璫者即黨逆。永光嘗頌璫,治逆案,陰護持之。南京給事中陳堯言疏劾永光璫孽,不當正銓席。然帝方眷永光,責堯言。譽卿又抗疏爭,於是都給事中薛國觀以己亦璫孽也,遂訐譽卿及同官沈惟炳東林主盟,結黨亂政。譽卿上疏自白,即日引去。

七年起故官,歷工科都給事中。明年正月,流賊陷潁州,譽卿請急調五千人守鳳陽。疏入而鳳陽已陷,皇陵毀焉。譽卿痛憤,直發本兵張鳳翼固位失事,及大學士溫體仁、王應熊玩寇速禍罪。言:「賊在秦、晉時,早設總督,遏其渡河,禍止西北一隅耳,乃侍郎彭汝楠避不肯行。及賊入楚、豫,人言交攻,然後不得已而議設之。侍郎汪慶百又避不行,乃推極邊之陳奇瑜。鞭長不及,釀成今日之禍,非樞臣之固位失事乎?流寇發難已久,樞臣因東南震鄰,始有淮撫操江移鎮之疏,識者已恨其晚。及奉旨,則曰不必移鎮。臣觀各地方稍有兵力,賊即不敢輕犯。鳳陽何地,使巡撫早移,豈有今日!今樞臣以曾請移鎮藉口,撫臣以不必移鎮為詞,則輔臣慾諱玩寇速禍,其可得哉!」帝以苛求責之。

而是時言官吳履中等復交章劾體仁、應熊交相贊美,「其擬旨慰留曰忠悃,曰藎畫,曰絕私奉公,曰弘濟時艱。不知時事至此,忠藎安在,而奉公濟艱者何事也?」譽卿再疏論,帝仍不問。譽卿曰:「皇上臨馭有年,法無假貸,獨於誤國輔臣不一問。今者巡撫楊一鵬、巡按吳振纓且相繼就逮矣。輔臣顧從容入直,退食委蛇,謂可超然事外乎?」帝終不聽。

譽卿在天啟時,謝升方為文選郎。及是,升長吏部,譽卿猶滯垣中。以資深當擢京卿,升希體仁意,出之南京。大學士文震孟慍語侵升,升亦慍。適山東布政使勞永嘉賄營登萊巡撫,主給事中宋之普家,陞等列之舉首,為給事中張第元所發。帝以詰升,言路因欲攻升及都御史唐世濟。譽卿以世濟恃體仁,惡尤甚,當先去之。御史張纘曾乃獨劾升,升疑出譽卿及震孟意,之普又構之升。先是,福建布政使申紹芳亦欲得登萊巡撫,譽卿曾言之升。升遂疏攻譽卿,謂其營求北缺,不欲南遷,為把持朝政地,并及囑紹芳事。體仁從中主之,譽卿遂削籍,紹芳逮問遣戍。十五年,御史劉逵及給事中楊枝起相繼論薦,竟不果用。福王立,起光祿卿,不赴。國變,薙髮為僧,久之卒。

華允誠,字汝立,無錫人。曾祖舜欽,瑞州知府。祖啟直,四川參政。允誠舉天啟二年進士。從同里高攀龍講學首善書院,先後旋里,遂受業為弟子,傳其主靜之學。四年春,從攀龍入都,授都水司主事。攀龍去官,允誠亦告歸。

崇禎改元,起營繕主事,進員外郎。二年冬,京師戒嚴,分守德勝門,四十餘日不懈,帝微行察知之,賜白金,敘功,加俸一年,改職方員外郎。五年六月,以溫體仁、閔洪學亂政,疏陳三大可惜、四大可憂。略言:

當事借皇上剛嚴,而佐以舞文擊斷之術,倚皇上綜核,而騁其訟逋握算之能,遂使和恆之世競尚刑名,清明之躬浸成叢脞。以聖主圖治之盛心,為諸臣鬥智之捷徑。可惜一。

帥屬大僚,驚魂於回奏認罪;封駁重臣,奔命於接本守科。遂使直指風裁徒徵事件,長吏考課惟問錢糧。以多士靖共之精神,為案牘鉤較之能事。可惜二。

廟堂不以人心為憂,政府不以人才為重。四海漸成土崩瓦解之形,諸臣但有角戶分門之念。意見互觭,議論滋擾。遂使剿撫等於築舍,用舍有若舉棋。以興邦啟聖之歲時,為即聾從昧之舉動。可惜三。

人主所以總一天下者,法令也。喪師誤國之王化貞,與楊鎬異辟;潔己愛民之餘大成,與孫元化並逮。甚至一言一事之偶誤,執訊隨之。遂使刑罰不中,鈇鉞無威。一可憂也。

國家所恃以為元氣者,公論也。直言敢諫之士一鳴輒斥,指佞薦賢之章目為奸黨,不惟不用其言,並錮其人,又加之罪。遂使喑默求容,是非共蔽。二可憂也。

國家所賴以防維者,廉恥也。近者中使一遣,妄自尊大,群僚趨走,惟恐後時。皇上以近臣可倚,而不知倖竇已開;以操縱惟吾,而不知屈辱士大夫已甚。遂使阿諛成風,羞惡盡喪。三可憂也。

國家所藉以進賢退不肖者,銓衡也。我朝罷丞相,以用人之權歸之吏部,閣臣不得侵焉。今次輔體仁與冢臣洪學,同邑朋比,惟異己之驅除。閣臣兼操吏部之權,吏部惟阿閣臣之意,造門請命,夜以為常。黜陟大柄,只供報復之私。甚至庇同鄉,則逆黨公然保舉,而白簡反為罪案;排正類,則講官借題逼逐,而薦剡遂作爰書。欺莫大於此矣,擅莫專於此矣,黨莫固於此矣。遂使威福下移,舉措倒置。四可憂也。

疏入,帝詰其別有指使。允誠乃列上洪學徇私數事,且曰:「體仁生平,糸臂塗顏,廉隅掃地。陛下排眾議而用之,以其悻直寡諧,豈知包藏禍心,陰肆其毒。又有如洪學者,為之羽翼,遍植私人,戕盡善類,無一人敢犯其鋒者,臣復受何人指使?」帝以體仁純忠亮節,而摘疏中「握定機關」語,再令陳狀。允誠復上言:「二人朋比,舉朝共知。溫育仁不識一丁,以家貲而首拔。鄧英以論沈演而謫,羅喻義以『左右非人』一語而逐。此非事之章明較著者乎?」帝亦悟兩人同里有私,乃奪允誠俸半年,而洪學亦旋罷去。

其冬,以省親歸,孝養母。母年八十三而終。後為福王驗封員外郎,十餘日即引疾歸。

允誠踐履篤實,不慕榮達。延儒再召,遣人以京卿啖之,允誠拒不應。入南都,士英先造請,亦不報謝。國變後,屏居墓田,不肯薙髮,與從孫尚濂駢斬於南京。

魏呈潤,字中嚴,龍溪人。崇禎元年進士。由庶吉士改兵科給事中。

三年冬,疏陳兵屯之策:「請敕順天、保定兩巡撫簡所部壯士,大邑五百人,小邑二三百人,分營訓練。而天津翟鳳翀、通州范景文、昌平侯恂並建節鉞,宜令練兵之外兼營屯田。」又陳閩海剿撫機宜六事。並議行。

明年夏,久旱求言。疏言:「驛站所裁,纔六十萬,未足充軍餉十一,而郵傳益疲,勢必再編里甲。是猶剜肉醫瘡,瘡未瘳而肉先潰。關外舊兵十八萬,額餉七百餘萬;今兵止十萬七千,合薊門援卒,非溢原數,加派五百九十萬外,新增又百四十餘萬,猶憂不足,可不為稽核乎!邊報告急,非臣子言功之日,而小捷頻聞,躐加峻秩,門客廝養詭名戎籍,不階而升,悉糜俸料,臣懼其難繼也。江淮旱災,五湖之間,海岸為谷,舊穀不登,新絲未熟,上供織造,宜且暫停。銓法壞於事例,正途日壅,不可不疏通。撫按諸臣捐貲助餉,大抵索之民間,顧奉急公之褒。上蒙而下削,不可不禁飭。」又條陳數策,請大修北方水政。帝皆納其言。

熹宗時,司業朱之俊議建魏忠賢祠國學旁,下教有「功不在禹下」語,置籍,責諸生捐助。及帝即位,委過諸生陸萬齡、曹代何以自解,首輔韓爌以同鄉庇之,漏逆案。及是,之俊已遷侍講。呈潤發其奸,請與萬齡棄西市,之俊由是廢。

宣府監視中官王坤以冊籍委頓,劾巡按御史胡良機。帝奪良機官,即令坤按核。呈潤上言:「我國家設御史巡九邊,秩卑而任鉅。良機在先朝以糾逆璫削籍,今果有罪,則有回道考核之法在,而乃以付坤。且邊事日壞,病在十羊九牧。既有將帥,又有監司;既有督撫,有巡方,又有監視。一官出,增一官擾,中貴之威,又復十倍。御史偶獲戾,且莫自必其命,誰復以國事抗者。異日九邊聲息,監視善惡,奚從而聞之?乞召還良機,毋使仰鼻息於中貴。」帝以呈潤黨比,貶三級,出之外。

良機者,南昌人也,字省之。萬曆四十四年進士。天啟間為御史,嘗糾魏忠賢之惡不減汪直、劉瑾。忠賢憾之,以年例遷廣東參議。良機方按貴州,不候代而去,遂斥為民。崇禎元年起故官,按宣、大二鎮。年滿當代,以其敏練,再巡一年。至是,遂為坤劾罷。

時又有御史李曰輔者,亦以論中官獲譴,廷臣交章論救,不聽。而御史趙東曦又疏劾坤,亦獲譴云。

曰輔,字元卿,亦南昌人也,與胡良機同里閈。萬曆中舉於鄉,為成都推官。與巡撫朱燮元計兵事,偕諸將攻復重慶。崇禎四年,擢南京御史。時中官四出,張彞憲總理戶、工錢糧,唐文征提督京營戎政,王坤監餉宣府,劉文忠監餉大同,劉允中監餉山西。又命王應期監軍關、寧,張國元監軍東協,王之心監軍中協,鄧希詔監軍西協,又命吳直監餉登島,李茂奇監茶馬陜西。曰輔上疏諫曰:「邇者一日遣內臣四,尋又遣用五,非兵機則要地也。廷臣方交章,而登島、陜西又有兩閹之遣。假專擅之權,駭中外之聽,啟水火之隙,開依附之門,灰任事之心,藉委卸之口。臣愚實為寒心。陛下踐阼初,盡撤內臣,中外稱聖。昔何以撤,今何以遣?天下多故,擇將為先。陛下不築黃金臺招頗、牧,乃汲汲內臣是遣,曾何補理亂之數哉!」帝怒,謫曰輔廣東布政司照磨。

東曦,字馭初,上海人。萬曆四十七年進士。崇禎五年,由知縣人為刑科給事中,請興屯塞下,以充軍用,不報。適宣塞有私和事,王坤時監宣餉,且請代。東曦上言:「宣塞失事,陛下赫然震怒,逮巡撫沈棨,罷本兵熊明遇。乃監視王坤方會飲城樓,商榷和議,邊臣倚庇,欺蔽日甚。坤不得辭扶同罪,反侈邊烽已熄為己功,且請代。夫內臣之遣,陛下一用之,非不易之典,今即盡撤之,猶謂不早。坤顧請代,圖彌縫於去後。願陛下正坤罪,撤各使還京。」帝言:「宣鎮擅和,實坤奏發,何謂欺隱?」調東曦外任,謫福建布政司都事。

異時呈潤起官,以光祿署丞終。良機起光祿典簿,終南京吏部主事。東曦稍遷行人司正、禮部郎中,奉使還里。福王時,召東曦為給事中,曰輔為御史,而二人者皆已死矣。

毛羽健,字芝田,公安人。天啟二年進士。崇禎元年,由知縣徵授御史。好言事,首劾楊維垣八大罪及阮大鋮反覆變幻狀,二人遂被斥。

王師討安邦彥久無功。羽健言:「賊巢在大方,黔其前門,蜀遵、永其後戶。由黔進兵,必渡陸廣奇險,七晝夜抵大方,一夫當關,千人自廢,王三善、蔡復一所以屢敗也。遵義距大方三日程,而畢節止百餘里平衍,從此進兵,何患不克?」因畫上足兵措餉方略,並薦舊總督朱燮元、閔夢得等。帝即議行,後果平賊。已,陳驛遞之害:「兵部勘合有發出,無繳入。士紳遞相假,一紙洗補數四。差役之威如虎,小民之命如絲。」帝即飭所司嚴加釐革,積困為蘇。

當是之時,閹黨既敗,東林大盛。而朝端王永光陰陽閃爍,溫體仁猾賊,周延儒回佞。言路新進標直之徒,尤競抨擊以為名高。體仁之訐錢謙益也,以科場舊事,延儒助之惡,且目攻己者為結黨欺君,帝怒而為之罷會推矣。御史黃宗昌疏糾體仁熱中枚卜,欲以「結黨」二字破前此公論之不予,且箝後來言路之多口。羽健亦憤朋黨之說,曰:「彼附逆諸奸既不可用,勢不得不用諸奸擯斥之人。如以今之連袂登進者為相黨而來,抑將以昔之鱗次削奪者為相黨而去乎!陛下不識在朝諸臣與奸黨諸臣之孰正孰邪,不觀天啟七年前與崇禎元年後之天下乎,孰危孰安?今日語太平則不足,語剔弊則有餘,諸臣亦何負國家哉!一夫高張,輒疑舉朝皆黨,則株連蔓引,不且一網盡哉!」帝責羽健疑揣,而以前條陳驛遞原之。

太常少卿謝升求巡撫於永光,永光長吏部,升當推薊鎮,畏而引病以避,後推太僕則不病。羽健劾升、永光朋比,宜並罪。永光召對文華殿,力詆羽健,請究主使之者。大學士韓爌曰:「究言官,非體也。」帝不從,已而宥之。一日,帝御文華殿,獨召延儒語良久,事秘,舉朝疑駭。羽健曰:「召見不以盈廷而以獨侍,清問不以朝參而以燕間;更漏已沉,閣門猶啟。漢臣有言『所言公,公言之;所言私,王者不受私』。」疏入,切責。羽健既積忤權要,其黨思因事去之。及袁崇煥下獄,主事陸澄源以羽健嘗疏譽崇煥,劾之,落職歸。卒。

黃宗昌,字長倩,即墨人。天啟二年進士。崇禎初,為御史,請斥矯旨偽官,言:「先帝賓天在八月二十三日。三殿敘功止先一日,正當帝疾大漸之時,豈能安閒出詔?凡加銜進秩,皆魏氏官也。」得旨:「汰敘功冒濫者。」宗昌爭曰:「臣所糾乃矯旨,非冒濫也。冒濫猶可容,矯偽不可貸。」遂列上黃克纘、范濟世、霍維華、邵輔忠、呂純如等六十一人,乞罷免。帝以列名多,不聽。尋劾罷逆黨尚書張我續、侍郎呂圖南、通政使岳駿聲、給事中潘士聞、御史王珙。又劾周延儒貪穢數事,帝怒,停俸半年。既而劾體仁,不納。

二年冬,巡按湖廣。岷王禋洪為校尉侍聖及善化王長子企鋀等所弒,參政龔承薦等不以實聞,獄不決者久之。宗昌至,群奸始伏辜。帝責問前諸臣失出罪,宗昌糾承薦等。時體仁、延儒皆已入閣,而永光意忌,以為不先劾承薦也。鐫宗昌四級,宗昌遂歸。

十五年,即墨被兵,宗昌率鄉人拒守,城全。仲子基中流矢死,其妻周氏及三妾郭氏、二劉氏殉之,謂之「一門五烈」。

莊烈帝初在位,銳意圖治,數召見群臣論事。然語不合,輒訶譴。而王永光長吏部,尤樂沮之。澄城人韓一良者,元年授戶科給事中,言:「陛下平臺召對,有『文官不愛錢』語,而今何處非用錢之地?何官非愛錢之人?向以錢進,安得不以錢償。以官言之,則縣官為行賄之首,給事為納賄之尤。今言者俱咎守令不廉,然守令亦安得廉?俸薪幾何,上司督取,過客有書儀,考滿、朝覲之費,無慮數千金。此金非從天降,非從地出,而欲守令之廉,得乎?臣兩月來,辭卻書帕五百金,臣寡交猶然,餘可推矣。伏乞陛下大為懲創,逮治其尤者。」帝大喜,召見廷臣,即令一良宣讀。讀已,以疏遍視閣臣曰:「一良忠鯁,可僉都御史。」永光請令指實。一良唯唯,如不欲告訐人者,則令密奏。五日不奏,而舉周應秋、閻鳴泰一二舊事為言,語頗侵永光。帝乃再召見一良、永光及廷臣,手前疏循環頌,音琅然,而曰「此金非從天降,非從地出」,則掩卷而歎。問一良:「五百金誰之饋也?」一良卒無所指。固問,則對如前。帝欲一良指實,將有所懲創,一良卒以風聞謝,大不懌。謂大學士劉鴻訓曰:「都御史可輕授耶!」叱一良前後矛盾,褫其官。

吳執御,字朗公,黃巖人。天啟二年進士。除濟南推官。德州建魏忠賢祠,不赴。

崇禎三年,徵授刑科給事中。明年請除掣簽法,使人地相配,議格不行。請蠲畿輔加派,示四方停免之期,曉然知息肩有日,不至召亂。請罷捐助搜括,毋為貪墨藏奸藪。帝以沽名市德責之。

劾吏部尚書王永光比匪:「用王元雅而封疆誤,聽張道浚賄舉尹同皋而祖制紊。國家立法懲貪,而永光誨貪,官邪何日正,寵賂何日清。」帝以永光清慎,不納其言。請召黃克纘、劉宗周、鄭鄤,忤旨譙讓。又言:「往者邊警,袁崇煥、王元雅擁金錢數百萬,士馬數十萬,狼狽失守,而史應聘、王象雲、張星、左應選以一邑抗強敵。故曰籌邊不在增兵餉,而在擇人。請畿輔東北及秦、晉沿邊州縣,選授精敏甲科,賜璽書,畀本地租賦,撫練軍民自禦寇。邊關文武吏繕修戰守外,責以理財,如先臣王翱、葉盛輩所為。客兵可撤,餉省可數百萬。」帝時未審執御所論畿輔、秦、晉也,而曰:「歲賦留本地,則國用何資?」不聽。

又劾首輔周延儒攬權,其姻親陳于泰及幕客李元功等交關為奸利。初,執御行取入都,延儒遣元功招之,不赴,至是竟劾延儒。又陳內外陰陽之說:「九邊、中原、廟堂之上,無非陰氣;心膂大臣,不皆君子。」帝以所稱「陽剛君子」無主名,令指實。執御乃以前所薦劉宗周三人,及姜曰廣、文震孟、陳仁錫、黃道周、倪元璐、曹于汴、惠世揚、羅喻義、易應昌對。會御史吳彥芳言:「執御所舉固真君子,他若侍郎李瑾、李邦華、畢懋康、倪思輝、程紹皆忠良當用,通政使章光岳邪媚當斥。」帝怒其朋比,執政復從中構之,遂削二人籍,下法司訊。時御史王績燦方以薦李邦華、劉宗周等下獄,而執御、彥芳復繼之,舉朝震駭。言官為申救,卒坐三人贖徒三年。

彥芳,字延祖,歙縣人,為御史。大凌被圍,疏論孫承宗。又駁逆案呂純如辨冤之謬。登州用兵,請設監島中官。至是譴歸。

績燦,宇偉奏,安福人。與給事中鄧英陳奸吏私派之弊,又進賜環、起廢、容諫三說。薦張鳳翔、李邦華、劉宗周、惠世揚,遂獲罪卒。福王時,復官。

彥芳、績燦兩人者,皆以天啟五年舉進士。彥芳授莆田知縣,績燦授興化知縣,又皆以治行高等擢崇禎四年御史,並有聲。其免官也,又皆以薦才不中,與吳執御同論譴云。

章正宸,字羽侯,會稽人。從學同里劉宗周,有學行。崇禎四年進士。由庶吉士改禮科給事中。勸帝法周、孔,黜管、商,崇仁義,賤富強。

禮部侍郎王應熊者,溫體仁私人也,廷推閣臣,望輕不得與。體仁引為助,為營入閣。正宸上言:「應熊強愎自張,何緣特簡。事因多擾,變以刻成,綜核傷察,宜存渾厚。奈何使很傲之人,與贊平明之治哉?」帝大怒,下獄拷訊,竟削籍歸。

九年冬,召為戶科給事中,遷吏科都給事中。周延儒再相,帝尊禮之特重。正宸出其門,與搘拄。歲旦朝會,帝隆師傅禮,進延儒等而揖之曰:「朕以天下聽先生。」正宸曰:「陛下隆禮閣臣,願閣臣積誠以格君心。毋緣中官,毋修恩怨,毋以寵利居成功,毋以爵祿私親暱。」語皆風刺延儒。延儒欲用宣府巡撫江禹緒為宣大總督,正宸持不可,吏部希延儒指,用之。延儒欲起江陵知縣史調元,正宸止之。延儒以罪輔馮銓力得再召,欲假守涿功復銓冠帶,正宸爭之,事遂寢。其不肯阿徇如此。未幾,會推閣臣,救李日宣,謫戍均州。語在《日宣傳》。

福王立,召復正宸故官。正宸痛舉朝無討賊心,上疏曰:「比者河北、山左各結營寨,擒殺偽官,為朝廷效死力。忠義所激,四方響應。宜亟檄江北四鎮,分渡河、淮,聯絡諸路,一心齊力,互為聲援。兩京血脈通,而後塞井陘,絕孟津,據武關以攻隴右。陛下縞素,親率六師,駐蹕淮上,聲靈震動,人切同仇,勇氣將自倍。簡車徒,選將帥,繕城塹,進寸則寸,進尺則尺,據險處要,以規中原。天下大矣,渠無人應運而出哉?」魏國公徐弘基薦逆案張捷,部議並起用鄒之麟、張孫振、劉光斗,安遠侯柳祚昌等薦起阮大鋮,正宸並疏諫,不納。改大理丞,正宸請假歸。魯王監國,署舊官。事敗,棄家為僧。

黃紹傑,萬安人。天啟五年進士。授中書舍人。

崇禎元年,考選給事中。需次,劾罷奄黨南京御史李時馨、徐復陽。補授兵科。五年,薊遼總督曹文衡與監視中官鄧希詔相訐。紹傑言:「文衡烈士,受內臣指摘,何顏立三軍上。希詔內豎,訐邊臣辱國,大不便。宜亟更文衡而罷希詔。」帝不聽。久之,文衡以閒住去,紹傑遷刑科左給事中。

七年五月,因旱求言。紹傑疏論大學士溫體仁曰:「漢世災異,策免三公,宰執亦引罪以求罷。今者久旱,陛下修明政治,納讜言,可謂應天以實矣,而雨澤不降,何哉?天有所甚怒而不解也。次輔溫體仁者,秉政數載,上干天和,無歲不旱,無日不風霾,無處不盜賊,無人不愁怨。秉政既久,窺瞷益工,中外趨承益巧。一人當用,則曰:『體仁意未遽爾也。』一事當行,則曰:『體仁聞恐不樂也。』覆一疏,建一議,又曰:『慮體仁有他屬。』不然,則:『體仁忌諱,毋攖其兇鋒也。』凡此召變之尤。願陛下罷體仁以回天意。體仁罷而甘霖不降,殺臣以正欺君之罪。」帝方眷體仁,貶紹傑一秩。體仁辨,且訐其別有指授。紹傑言:「廷臣言事,指及乘輿,猶荷優容,一字涉體仁,必遭貶黜。誰不自愛,為人指授耶?」因列其罪狀:東南不肯設立總督,庇兵部侍郎彭汝楠,致失機宜;用貪穢胡鐘麟為職方郎,而黜李繼貞;囑尚書閔洪學起私人唐世濟為南京總憲,錮正人瞿式耜等;庇姻婭沈棨為宣撫,私款辱國;庇主考丁進,從寬磨勘。且曰:「臣所仰祝聖明,洞燭體仁奸欺者,其說則有兩端。下惟朋黨一語,可以箝言官之口,挑善類之禍;上惟票擬一語,可以激聖明之怒,蓋僨誤之愆。」體仁猶辨,且以朋黨為言。紹傑遂言:「體仁受銅商王誠金,體仁長子受巡撫棨及兩淮巡鹽高欽順等金,皆萬計。體仁用門幹王治,東南之利皆其轉輸。體仁私邸兩被盜,失黃金寶玉無算,匿不敢言。」帝怒,調為上林苑署丞,遷行人司副。八年,賊犯皇陵,紹傑再劾體仁誤國召寇,再謫應天府檢校。屢遷南京吏部郎中,卒。

先是,七年正月,給事中李世祺論溫體仁及大學士吳宗達,並劾兵部尚書張鳳翼溺職狀。帝怒,謫福建按察司檢校。世祺,字壽生,青浦人。天啟二年進士,授行人。

崇禎三年,擢刑科給事中,陳大計之當定者二:曰兵食之計,民生之計;大弊之當厘者三:曰六曹之弊在吏胥,邊吏之弊在欺隱,貪墨之弊在奢靡。夏旱,禱雨未應,乃進修政之說三:曰恤畿甸,議催科,預儲備。帝並納之。中官出鎮,世祺上言:「祖宗立法,錢穀兵馬,軍民各分事權,防專擅。內閣入奉天顏,出司兵食,內廷意旨既得而陰伺之,外廷事權又得而顯操之。魏忠賢盜弄神器,則賴聖天子躬翦除之,而奈何復躬自蹈之。」不聽。

五年八月,淫雨損山陵,昌平地動。世祺上言:「日者輔理調燮無聞,精神為固寵之用;統軍衡才無術,緩急無可恃之人。中樞決策,掩耳盜鈴;主計持籌,醫瘡剜肉。州縣迫功令,鞭策不前;六曹窘簿書,救過不贍。簪筆執簡之臣,接跡囹圄;考槃絪軸之士,抗聲鴻舉。一人議,疑及眾人;一事訾,疑及眾事。黃衣之使,頡頏卿貳之堂;貂蟬之座,雄踞節鉞之上。低眉則氣折,強項則釁開。各邊監視之遣,已將期月,初雖間有摘發,至竟同歸模棱,效不效可概見。伏願撤回各使,以明陰不乾陽之分。然後採公論以進退大臣,酌事情以衡量小臣,釋疑忌之根,開功名之路,庶天變可回,時艱可濟。」帝以借端瀆奏,切責之。

給事中陳贊化劾周延儒,謂:「延儒嘗語人曰:今上,羲皇上人也。此成何語?臣聞之世祺。」帝詰世祺,則言聞之贊化。帝詰責者再三,世祺執如初,乃已。至是論體仁絕世之奸,大貪之尤,遂貶官。久之,起行人司副,屢遷太僕寺卿。遣祭魯王,事竣旋里。國變,杜門不出,久之卒。

傅朝佑,字右君,臨川人。有孝行。萬曆中舉鄉試第一,師事鄒元標。天啟二年成進士,授中書舍人。

崇禎三年,考選給事中。永平初復,列上善後七事。帝採納之,補授兵科。明年八月,疏劾首輔周延儒:「以機械變詐之心,運刑名督責之術。見佞則加之膝,結袁弘勛、張道濬為腹心;遇賢則墜之淵,擯錢象坤、劉宗周於草莽。傾陷正士,加之極刑,曰『上意不測也』;攘竊明旨,播諸朝右,曰『吾意固然也』。皇上因旱求言,則恐其揚己過,故削言官以立威;皇上慎密兵機,則欲其箝人口,故挫直臣以怵眾。往時糾其罪惡者盡遭斥逐,而親知鄉曲遍列要津。大臣之道固如是乎?」忤旨切責。

屢遷工科左給事中,陳當務十二事:一納諫,二恤民,三擇相,四勿以內批用輔臣,五勿使中官司彈劾,六勿令法外加濫刑,七止緹騎,八停內操,九抑武臣驕玩,十廣起廢,十一敕有司修城積粟,十二講聖諭六條。出封益籓,事竣還里。

九年,即家進刑科都給事中。還朝愆期,為給事中陳啟新所劾,貶秩調外。未行,疏論溫體仁六大罪。略言:

陛下當邊警時,特簡體仁入閣。體仁乃不以道事君,而務刑名。窺陛下意在振作,彼則借以快恩仇;窺陛下治尚精明,彼則託以張威福。此謂得罪於天子。鳳陽、昌平鐘靈之地,體仁曾無未雨綢繆,兩地失守,陵寢震驚。此謂得罪於祖宗。燮理職在三公,體仁為相,日月交蝕,星辰失行,風霾迭見,四方告災,歲比不登,地震河決,城陷井枯,曾莫之懲,則日尋恩怨,圖報睚眥。此謂得罪於天地。強敵內逼,大盜四起,高麗旦暮且陷。體仁冒賞冒蔭,中外解體因之。此謂得罪於封疆。體仁子見屏於復社諸生,募人糾彈,株連不已。且七年又議裁減茂才,國家三百年取士之經,一旦壞於體仁之手。此謂得罪於聖賢。同生天地,誰無本心,體仁自有肺腸,偏欲殘害忠良。只今文武臣僚,幾數百人,駢首囹圄,天良盡喪。此謂得罪於心性。

夫人主之辨奸在明,而人主之去姦在斷。伏願陛下大施明斷,速去體仁。毋以天變為不足畏,毋以人言為不足恤,毋以群小之逢迎為必可任,毋以一己之清明為必可恃。大赦天下,除苛政,庶倒懸可解,太平可致。

帝怒,除名,下吏按治。踰月,體仁亦罷。

中官杜勳雅重朝佑,令其上疏請罪,而己從中主之,可復故官,朝佑不應。十一年冬,國事益棘,獲罪者益眾,獄幾滿。朝佑乃從獄中上書,請寬恤,語過激。會有邊警,未報也。明年春,責以顛倒賢奸,恣意訕侮,廷杖六十,創重而卒。

當時臺省競言事,言不中多獲譴。章正宸、莊鰲獻、李汝璨之徒好直諫,朝佑嘗疏稱之。

鰲獻,字任公,晉江人。崇禎六年,由庶吉士改兵科給事中,上《太平十二策》,極論東廠之害。忤旨,貶浙江布政司照磨。

汝璨,字用章,南昌人。崇禎時為刑科給事中。十年閏月因旱求言,陳回天四要,論財用政事之弊。又言:「八、九年來,乾和召災,始於端揆,積於四海。水旱盜賊,頻見疊出,勢將未已,何怪其然。」帝怒,削籍歸。國變,衰絰北面哀號,作《祈死文》祈死,竟死。

汝璨、朝佑既死,福王時,復官。鰲獻事福王,復官,久之卒。

姜埰,字如農,萊陽人。崇禎四年進士。授密雲知縣,調儀真,遷禮部主事。十五年,擢禮科給事中。

山陽武舉陳啟新者,崇禎九年詣闕上書,言:「天下三大病。士子作文,高談孝悌仁義,及服官,恣行奸慝。此科目之病也。國初典史授都御史,貢士授布政,秀才授尚書,嘉靖時猶三途並用,今惟一途。舉貢不得至顯官,一舉進士,橫行放誕。此資格之病也。舊制,給事、御史,教官得為之,其後途稍隘,而舉人、推官、知縣猶與其列,今惟以進士選。彼受任時,先以給事、御史自待,監司、郡守承奉不暇,剝下虐民,恣其所為。此行取考選之病也。請停科目以絀虛文,舉孝廉以崇實行,罷行取考選以除積橫之習,蠲災傷田賦以蘇民困,專拜大將以節制有司便宜行事。」捧疏跪正陽門三日,中官取以進。帝大喜,立擢吏科給事中,歷兵科左給事中。劉宗周、詹爾選等先後論之。歙人楊光先訐其出身賤役,及徇私納賄狀。帝悉不究。然啟新在事所條奏,率無關大計。御史王聚奎劾其溺職,帝怒,謫聚奎。以僉都御史李先春議聚奎罰輕,並奪其職。久之,御史倫之楷劾其請託受賕,還鄉驕橫,始詔行勘。未上而啟新遭母憂,埰因劾其不忠不孝,大奸大詐。遂削啟新籍,下撫按追贓擬罪。啟新竟逃去,不知所之。國變後,為僧以卒。

時帝以寇氛未息,民罹鋒鏑,建齋南城。埰上疏諫,不報。已,陳蕩寇二策,曰明農業,收勇敢。帝善其言。

初,溫體仁及薛國觀排異己及建言者。周延儒再相,盡反所為,廣引清流,言路亦蜂起論事。忌者乃造二十四氣之說,以指朝士二十四人,直達御前。帝適下詔戒諭百官,責言路尤至。埰疑帝已入其說,乃上言:「陛下視言官重,故責之嚴。如聖諭云『代人規卸,為人出缺』者,臣敢謂無其事。然陛下何所見而云?倘如二十四氣蜚語,此必大奸巨憝,恐言者不利己,而思以中之,激至尊之怒,箝言官之口,人皆喑默,誰與陛下言天下事者?」先是,給事中方士亮論密雲巡撫王繼謨不勝任,保定參政錢天錫因夤緣給事中楊枝起、廖國遴,以屬延儒,及廷推,遂得俞旨。適帝有「為人出缺」諭,蓋舉廷臣積習告戒之,非為天錫發也。埰探之未審,謂帝實指其事,倉卒拜疏。而帝於是時方憂勞天下,默告上帝,戴罪省愆,所頒戒諭,詞旨哀痛,讀者感傷。埰顧反覆詰難,若深疑於帝者,帝遂大怒,曰:「埰敢詰問詔旨,藐玩特甚。」立下詔獄考訊。掌鎮撫梁清宏以獄詞上,帝曰:「埰情罪特重。且二十四氣之說,類匿名文書,見即當毀,何故累騰奏牘?其速按實以聞。」時行人熊開元亦以建言下錦衣衛。帝怒兩人甚,密旨下衛帥駱養性,令潛斃之獄。養性懼,以語同官。同官曰:「不見田爾耕、許顯純事乎?」養性乃不敢奉命,私以語同鄉給事中廖國遴,國遴以語同官曹良直。良直即疏劾養性「歸過於君,而自以為功。陛下無此旨,不宜誣謗;即有之,不宜洩。」請並誅養性、開元。養性大懼,帝亦不欲殺諫臣,疏竟留中。會鎮撫再上埰獄,言掠訊者再,供無異詞。養性亦封還密旨。乃命移刑官定罪,尚書徐石麒等擬埰戍,開元贖徒。帝責以徇情骫法,令對狀。乃奪石麒及郎中劉沂春官,而逮埰、開元至午門,並杖一百。埰已死,埰弟垓口溺灌之,乃復蘇,仍繫刑部獄。明年秋,大疫,命諸囚出外收保。埰、開元出,即謁謝賓客。帝以語刑部尚書張忻,忻懼,復禁之獄。十七年二月始釋埰,戍宣州衛。將赴戍所而都城陷。

福王立,遇赦,起故官。丁父艱,不赴。國變後,流寓蘇州以卒。且死,語其二子曰:「吾奉先帝命戍宣州,死必葬我敬亭之麓。」二子如其言。

垓,字如須,崇禎十三年進士。授行人。埰下獄,垓盡力營護。後聞鄉邑破,父殉難,一門死者二十餘人。垓請代兄系獄,釋埰歸葬,不許。即日奔喪,奉母南走蘇州。初,垓為行人,見署中題名碑,崔呈秀、阮大鋮與魏大中並列,立拜疏請去二人名。及大鋮得志,滋欲殺垓甚。垓乃變姓名,逃之寧波。國亡乃解。

熊開元,字魚山,嘉魚人。天啟五年進士。除崇明知縣,調繁吳江。

崇禎四年,徵授吏科給事中。帝遣中官王應期等監視關、寧軍馬,開元抗疏爭,不納。王化貞久繫不決,奸人張應時等疏頌其功,請以身代死,俾戴罪立功。開元疏駁之,言:「化貞家貲鉅萬,每會朝審,輒買燕市少年,雜立道旁,投熊廷弼瓦礫,嗟嘆化貞不休,以此熒惑上聽。今應時復敢為此請,宜立肆化上貞市朝。」化貞卒正法。

時有令,有司徵賦不及額者不得考選。給事中周瑞豹考選而後完賦,帝怒,貶謫之,命如瑞豹者悉以聞。於是開元及御史鄭友元等三人並貶二秩調外,開元不赴官。久之,起山西按察司照磨,遷光祿寺監事。

十三年,遷行人司副。左降官率驟遷,開元以淹久頗觖望。會光祿丞缺,開元詣首輔周延儒述己困頓狀。延儒適以他事輒命駕出,開元大慍。會帝以畿輔被兵求言,官民陳事者,報名會極門,即日召對。

開元欲論延儒,次日即請見。帝召入文昭閣,開元請密論軍事。帝屏左右,獨輔臣在,開元不敢言,但奏軍事而出。越十餘日,復請見。帝御德政殿,秉燭坐,開元從輔臣入,奏言:「《易》稱『君不密則失臣,臣不密則失身』,請輔臣暫退。」延儒等引退者再,帝不許。開元遂言:「陛下求治十五年,天下日以亂,必有其故。」帝曰:「其故安在?」開元言:「今所謀畫,惟兵食寇賊。不揣其本,而末是圖,雖終日夜不寢食,求天下治無益也。陛下臨御以來,輔臣至數十人,不過陛下曰賢,左右曰賢而已,未必諸大夫國人皆曰賢也。天子心膂股肱,而任之易如此。庸人在高位,相繼為奸,人禍天殃,迄無衰止。迨言官發其罪狀,誅之斥之,已敗壞不可復救矣。」帝與詰問久之,疑開元有所為,曰:「爾意有人欲用乎?」開元辨無有,且奏且頻目延儒。延儒謝,帝曰:「天下不治皆朕過,於卿等何與?」開元言:「陛下令大小臣工不時面奏,而輔臣在左右,誰敢為異同之論以速禍?且昔日輔臣,繁刑厚斂,屏棄忠良,賢人君子攻之。今輔臣奉行德意,釋纍囚,蠲逋賦,起廢籍,賢人君子皆其所引用。偶有不平,私慨歎而已。」帝責開元有私。開元辨,延儒等亦前為解。

開元復請遍召廷臣,問以輔臣賢否。「輔臣心事明,諸臣流品亦別。陛下若不察,將吏狃情面賄賂,失地喪師,皆得無罪,誰復為陛下捐軀報國者?」延儒等奏情面不盡無,賄賂則無有。開元復言:「敵兵入口四十餘日,未聞逮治一督、撫。」帝曰:「督、撫初推,人以為賢,數月後即以為不賢,必欲去之而後快。邊方與內地不同,使人何以展布。」開元言:「四方督、撫,率自監司。明日廷推,今日傳單,其人姓名不列。至期,吏部出諸袖,諸臣唯唯而已。既推後,言官轉相採訪,而其人伎倆亦自露於數月間,故人得而指之。非初以為賢,繼以為不賢也。」帝命之退。延儒等請令補牘,從之。

當是時,開元欲發延儒罪,以其在側不敢言。而延儒慮其補牘,謀沮之。大理卿孫晉、兵部侍郎馮元飆責開元:「首輔多引賢者。首輔退,賢者且盡逐。」開元意動。大理丞吳履中至,亦以開元言為驟。禮部郎中吳昌時者,開元知吳江時所拔士也,復致書言之。開元乃止述奏辭,不更及延儒他事。帝方信延儒,大清兵又未退,焦勞甚。得奏,大怒,令錦衣衛逮治。衛帥駱養性,開元鄉人也,雅怨延儒,次日即以獄上。帝益怒,曰:「開元讒譖輔弼,必使朕孤立於上,乃便彼行私,必有主使者。養性不加刑,溺職甚,其再嚴訊以聞。」十二月朔,嚴刑詰供主謀。開元堅不承,而盡發延儒之隱,養性具以聞。帝乃廷杖開元,繫獄。

始,方士亮劾罷密雲巡撫王繼謨,參政錢天錫得巡撫。御史孫鳳毛發其事,劾給事中楊枝起、廖國遴為天錫夤緣,因言開元面奏,實二人主之,欲令邱瑜秉政,陳演為首輔。御史李陳玉亦言之。帝以開元已下吏,不問,而責令鳳毛陳奏。鳳毛死,其子訴冤,謂國遴、枝起鴆殺之。兩人及天錫並削職下獄。士亮又言恐代繼謨者未能勝繼謨,繼謨得留任。十六年六月,延儒罷,言官多救開元者,不報。刑部擬贖徒,不許。明年正月,遣戍杭州。

未幾,京師陷,福王召起吏科給事中。丁母艱,不赴。唐王立,起工科左給事中。連擢太常卿、左僉都御史,隨征東閣大學士。乞假歸。汀州破,棄家為僧,隱蘇州之靈巖以終。

士亮,歙縣人。崇禎四年進士。歷嘉興、福州推官,擢兵科給事中。與同官朱徽、倪仁禎等謁大學士謝陞於朝房,升言:「人主以不用聰明為高。今上太用聰明,致天下盡壞。」又曰:「款事諸君不必言,皇上祈簽奉先殿,意已決。」諸人退,謂升誹謗君父,洩禁中語。仁禎、國遴等交章論之,斥升大不道,無人臣禮。士亮及他言官繼之,疏數十上。帝大怒,削升籍。已而士亮連劾諸督撫張福臻、徐世廕、朱大典、葉廷貴,及兵部侍郎呂大器、甘肅總兵馬爌,事多施行。又請召舊諫臣姚思孝、何楷、李化龍、張作楫、張焜芳、李模、詹爾選、李右讜、林蘭友、成勇、傅元初,而恤已死者吳執御、魏呈潤、傅朝佑、吳彥芳、王績燦、葛樞,帝頗採納。周延儒出督師,請士亮贊畫軍務。延儒獲譴,士亮亦削職下獄,久之釋歸。福王時,復官。國變後卒。

詹爾選,字思吉,撫安人。崇禎四年進士。授太常博士。八年,擢御史。時詔廷臣舉守令,爾選言:「縣令多而難擇,莫若精擇郡守。郡守賢,縣令無不賢。」因請起用侍郎陳子壯、推官湯開遠,報聞。

明年,疏劾陳啟新:「宜召九卿科道,覿面敷陳,罄其底蘊。果有他長,然後授官。遽爾授官,非所以重名器。吏部尚書謝升、大學士溫體仁不加駁正,尸素可愧。」帝怒。未幾,大學士錢士升以爭武生李璡搜括富戶,忤旨,引罪乞休去。爾選上疏曰:

輔臣引咎求黜,遽奉回籍之諭。夫人臣所以不肯言者,其源在不肯去耳。輔臣肯言肯去,臣實榮之,獨不能不為朝廷惜此一舉也。璡以非法導主上,其端一開,大亂將至。輔臣憂心如焚,忽奉改擬之命,遂爾執奏。皇上方嘉許不暇,顧以為疑君要譽耶?人臣無故疑其君,非忠也;乃謂吾君萬舉萬當者,第容悅之借名,必非忠。人臣沽名,義所不敢出也,乃人主不以名譽鼓天下,使其臣尸位保寵,寡廉鮮恥,亦必非國家利。

況今天下疑皇上者不少矣。將驕卒惰,尚方不靈,億萬民命,徒供武夫貪冒,則或疑過於右武。穿札與操觚並課,非是者弗錄。人見賣牛買馬,絀德齊力,徒使強寇混跡於道途,父兄莫必其子弟,則或疑緩於敷文。免覲之說行,上意在蘇民困也,而或疑朝宗之大義,不敵數萬路用之金錢;駁問之事煩,上意在懲奸頑也,而或疑明啟之刑書,幾禁加等之紛亂。

其君子憂驅策之無當,其小人懼陷累之多門,明知一切茍且之政,或拊心愧恨,或對眾欷歔。輔臣不過偶因一事,代天下發憤耳,而竟鬱鬱以去,恐後之大臣無復有敢言者矣。大臣不敢言,而小臣愈難望其言矣。所日與皇上言者,惟苛細刻薄不識大體之徒,似忠似直,如狂如癡,售則挺身招搖,敗則潛形逋竄,駭心志而龠耳目,毀成法而釀隱憂,天下事尚忍言哉!祈皇上以遠大宅心,以簡靜率憲,責大臣弼違之義,作言官敢諫之風。寧獻可替否,毋藉口聖明獨斷,掩聖主之謙沖;寧進禮退義,毋藉口君恩未酬,飾引身之濡滯。臣愚不勝心卷心卷。

疏入,帝震怒,召見武英殿,詰之曰:「輔臣之去,前旨甚明,汝安得為此言?」對曰:「皇上大開言路,輔臣乃以言去國,恐後來大臣以言為戒,非皇上求言意。」帝曰:「建言乃諫官事,大臣何建言?」對曰:「大臣雖在格心,然非言亦無由格。大臣止言其大者,決無不言之理。大臣不言,誰當言者?」帝曰:「朕如此焦勞,天下尚疑朕乎?即尚方劍何嘗不賜,彼不能用,何言不靈?」對曰:「誠如聖諭。但臣見督理有參疏,未蒙皇上大處分,與未賜何異?」帝曰:「刑官擬罪不合,朕不當駁乎?」對曰:「刑官不職,但當易其人,不當侵其事。」帝曰:「汝言一切茍且之政,何者為茍且?」對曰:「加派。」帝曰:「加派,因賊未平,賊平何難停。汝尚有言乎?」對曰:「搜括抽扣亦是。」帝曰:「此供軍國之用,非輸之內帑。汝更何言?」對曰:「即捐助亦是。」帝曰:「本令願捐者聽,何嘗強人?」時帝聲色俱厲,左右皆震懾,而爾選詞氣不撓。帝又詰發憤諸語,及帖黃簡略,斥為欺罔,命錦衣提下。爾選叩頭曰:「臣死不足惜,皇上幸聽臣,事尚可為。即不聽,亦可留為他日思。」帝愈怒,罪且不測,諸大臣力救,乃命繫於直廬。明日下都察院議罪,議止停俸。帝以語涉誇詡,并罪視草御史張三謨,令吏部同議。請鐫五級,以雜職用。復不許,乃削籍歸。自後言者屢薦,皆不聽。十五年,給事中沈迅、左懋第相繼薦。有詔召還,未及赴而都城陷。

福王立,首起故官。未上,群小用事,憚爾選鯁直,令補外僚,遂不出。國變後,又十二年而終。

湯開遠,字伯開,主事顯祖子也。早負器識,經濟自許。崇禎五年,由舉人為河南府推官。帝惡廷臣玩心妻,持法過嚴。開遠疏諫曰:

陛下臨御以來,明罰敕法。自小臣至大臣,蒙重譴下禁獄者相繼,幾於刑亂國用重典矣。見廷臣薦舉不當,疑為黨徇;惡廷臣執奏不移,疑為藐抗。以策勵望諸臣,於是戴罪者多,而不開以立功之路;以詳慎責諸臣,於是引罪者眾,而不諒其致誤之由。墨吏宜逮,然望稍寬出入,無絀能臣。至三時多害,五方交警,諸臣怵參罰,惟急催科,民窮則易為亂。陛下寬一分在臣子,即寬一分在民生,此可不再計決者。尤望推諸臣以心,待諸臣以禮,諭中外法司以平允。至錦衣禁獄,非寇賊奸宄,不宜輕入。

帝怒,摘其疏中「桁楊慘毒,遍施勞臣」語,責令指實。乃上奏曰:

時事孔棘,諸臣有過可議,亦有勞可準;有罪可程,亦有情可原。究之議過不足懲過,而後事轉因前事以灰心;聲罪不足服罪,而故者更藉誤者以實口。綜核太過則要領失措,懲創太深則本實多缺。往往上以為宜詳宜新之事,而下以為宜略宜仍之事;朝所為縲辱擯棄不少愛之人,又野所為推重愾歎不可少之人。上與下異心,朝與野異議,欲天下治平,不可得也。

蘇州僉事左應選任昌黎縣令,率土著保孤城。事平之日,擢任監司。乃用小過,卒以贓擬。城池失守者既不少貸,捍禦著績者又不獲原,諸臣安所適從哉?事急則鉅萬可捐,事平則錙銖必較。向使昌黎不守,同於遵、永,不知費朝廷幾許金錢,安所得涓滴而問之?臣所惜者此其一。

給事中馬思理、御史高倬,值草場火發,狂奔盡氣,無救燎原,此不過為法受過耳,更欲以他罪論,則甚矣。今歲盛夏雪雹,地震京圻,草場不爇自焚。陛下不寬刑修省,反嚴鞫而長繫之,非所以召天和,稱善事也。臣所惜者此其一。

宣大巡按胡良機,陛下知其諳練,兩任巖疆,尋因過誤褫革,輿論惜之,豈成命終難反汗哉!臣所惜者此其一。

監兌主事吳澧,宵旦河干,經營漕事,運弁稽違,量行責戒,乃褫革之,又欲究治之。夫兵嘩則為兵易將,將嘩則為武抑文,勇於嘩而怯於斗,安用此驕兵驕將為也!臣所惜者此又其一。

末復為都御史陳于廷、易應昌申辨。帝怒,切責之。

河南流賊大熾,開遠監左良玉軍,躬擐甲胄,屢致克捷。帝以天下用兵,意頗重武,督、撫失事多逮繫,而大將率姑息。開遠以為偏,八年十月上疏曰:

比年寇賊縱橫,撫、鎮為要。乃陛下於撫臣則懲創之,於鎮臣則優遇之。試觀近日諸撫臣,有不褫奪、不囚繫者乎?諸帥臣及偏裨,有一禮貌不崇、陞蔭不遂者乎?即觀望敗衄罪狀顯著者,有不寬假優容者乎?夫懲創撫臣,欲其惕而戒也;優遇武臣,欲其感而奮也。然而封疆日破壞、寇賊日蔓延者,分別之法少也。撫臣中清操如沈棨,幹濟如練國事,捍禦兩河、身自為將如元默,拮據兵事、沮賊長驅如吳甡,或麗爰書,或登白簡,其他未可悉數。而武臣桀驁恣睢,無日不上條陳,爭體統。一旦有警,輒逡巡退縮,即嚴旨屢頒,褒如充耳。如王樸、尤世勛、王世恩輩,其罪可勝誅哉!

秦撫甘學闊有《法紀全疏》一疏,請正縱賊諸弁以法,明旨顧切責之。然則自今以後,敗將當不問矣。文臣未必無才能,乃有寧甘斥黜必不肯任不敢任者,以任亦罪,不任亦罪,不任之罪猶輕,而任之罪更重也。誠欲使諸臣踴躍任事,在寬文法,原情實,分別去留,毋以一眚棄賢才。至韎韐之夫,不使怯且欺者倖乎其間,則賞罰以平,文武用命矣。

帝以撫臣不任者,無所指實,責令再陳。乃上言曰:

朝廷賞罰無章,於是諸臣之不肯任不敢任者罪,而肯任敢任者亦罪,且其罪反重。勸懲無當,欲勘定大亂,未之前聞。從來無詘督臣以伸庸帥者,至今而楊嗣昌不得關其說;從來無抑言路以伸劣弁者,至今而王肇坤不得保其秩。王樸心匡怯暴著,聽敵飽去,猶得與吳甡並論,播之天下,不大為口實哉!若撫臣之不肯任不敢任者,如了陜西之胡廷晏,山西之仙克謹、宋統殷、許鼎臣,何以當日處分視後皆輕?練國事、元默承大壞極敝之後,竭力撐持,何以當日處分較前更重?且近日為辦寇而誅督臣者一,逮督臣撫臣者二,褫撫臣者亦二。甚至巡方與撫臣並議,而並逮兩按臣;計典與失事牽合,而並褫南樞臣。若監司、守令之獲重譴者,不可勝紀。試問前後諸帥臣,有一誅且逮者乎?即降而偏裨,有一誅且逮者乎?甚至避寇、縱寇、養寇、助寇者,皆置弗問。即或處分,不過降級戴罪而已。然則諸將之不肯任不敢任者,直謂之無罪可乎?是陛下於文武二途,委任同,責成不同。明旨所謂一體者,終非一體矣。

不特此也。按臣曾周當舊撫艱去,力障寇鋒,初非失事,乃竟從逮配,將來無肯任敢任之按臣矣。道臣祝萬齡拮据兵食,寢餌俱廢,至疽發於背,而遽行削籍,將來無肯任敢任之監司矣。史洪謨作令宜陽,戰守素備,賊渡澠池,不敢薄城,及知六安,復有全城之績,而褫奪驟加,將來無肯任敢任之州縣矣。賊薄永寧,舊蜀撫張論與子給事鼎延傾貲募士,夙夜登陴,及論物故,鼎延請恤,并其子官奪之,將來無肯任敢任之鄉官矣。吏部惟雜職多弊,臣鄉吳羽文竭力厘剔,致刀筆賈豎哄然而起,羽文略不為撓,乃以起廢一事,長繫深求,將來無肯任敢任之部曹矣。

臣讀明旨,謂諸事皆經確核,以議處有銓部,議罪有法司,稽核糾舉有按臣也。不知詔旨一下,銓部即議降議革,有肯執奏曰「此不當處」者乎?一下法司,即擬配擬戍,有肯執奏曰「此不當罪」者乎?至查核失事,按臣不過據事上聞,有原功中之罪、罪中之功,乞貸於朝廷者乎?是非諸臣不肯分別也,知陛下一意重創,言之必不聽,或反以甚其罪也。所以行間失事,無日不議處議罪,而於蕩寇安民毫無少補。則今日所少者,豈非大公之賞罰哉!

帝得奏大怒,命削籍,撫按解京訊治。河南人聞之,若失慈母。左良玉偕將士七十餘人合奏乞留,巡按金光辰亦備列其功狀以告。帝為動容,命釋還戴罪辦賊。

十年正月,討平舞陽大盜楊四。論功當進秩,總理王家禎復薦之。乃擢按察僉事,監安、廬二郡軍。其年冬,太子將出閣。奏言:「陛下言教不如身教。請謹幽獨,恤民窮,優大臣,容直諫,寬拙吏,薄貨財,疏滯獄,俾太子得習見習聞,為他日出治臨民之本。」帝深納之。

是時,賊大擾江北,開遠數有功。巡撫史可法薦其治行卓異,進秩副使,監軍如故。十三年,與總兵官黃得功等大破革裏眼諸賊,賊遂乞降。朝議將用為河南巡撫,竟以勞瘁卒官,軍民咸為泣下。贈太僕少卿。

成勇,字仁有,安樂人。天啟五年進士。授饒州推官。謁鄒元標於吉水,師事之。中使至,知府以下郊迎,勇不往,且捕笞其從人。丁內外艱。歷開封、歸德二府推官。流賊攻歸德,擊走之。

崇禎十年,行取入京。時變考選例,優者得為翰林。公論首勇,而吏部尚書田唯嘉抑之,勇得南京吏部主事以去。明年二月,帝御經筵,問講官保舉考選得失,諭德黃景昉訟勇及朱天麟屈。帝親策諸臣,天麟得翰林,而勇以先赴南京不與。尋用御史塗必泓言,授南京御史。

楊嗣昌奪情入閣,言者咸獲譴。勇憤,其年九月上疏言:「嗣昌秉樞兩年,一籌莫展,邊警屢驚,群寇滿野。清議不畏,名教不畏,萬世公義不畏,臣竊為青史慮。」疏入,帝大怒,削籍提訊,詰主使姓名。勇獄中上書言:「臣十二年外吏,數十日南臺,無權可招,無賄可納,不知有黨。」帝怒,竟戍寧波衛。中外薦者十餘疏,不召。後以御史張瑋言,執政合詞請擢用,帝以勇宥罪方新,不當復職,命以他官用。甫聞命,而京師陷。

福王時,起御史,不赴。披緇為僧,越十五年而終。

陳龍正,字惕龍,嘉善人。父於王,福建按察使。龍正遊高攀龍門。崇禎七年成進士,授中書舍人。時政尚綜核,中外爭為深文以避罪,東廠緝事尤冤濫。

十一年五月,熒惑守心,下詔修省,有「哀懇上帝」語。龍正讀之泣,上《養和》、《好生》二疏。略曰:「回天在好生,好生無過減死。皋陶贊舜曰『罪疑惟輕』,是聖人於折獄不能無失也。蓋獄情至隱,人命至重,故不貴專信,而取兼疑,不務必得,而甘或失。臣居家所見聞,四方罪犯,無甚窮凶奇謀者,及來京師,此等乃無虛月。且罪案一成,立就誅磔,亦宜有所懲戒,何犯者若此纍纍?臣願陛下懷帝舜之疑,寧使聖主有過仁之舉,臣下獲不經之愆。」蓋陰指東廠事也。越數日,果諭提督中官王之心不得輕視人命云。其冬,京師戒嚴,詔廷臣舉堪任督、撫者。御史葉紹顒舉龍正。久之,刑部主事趙奕昌請訪求天下真賢才。帝令奕昌自舉,亦以龍正對。帝皆不用。

龍正居冷曹,好言事。十二年十月,彗星見。是歲冬至,大雷電雨雹。十三年二月,京師大風,天黃日眚,浹旬不解。龍正皆應詔條奏,大指在聽言省刑。

十五年夏,帝復下詔求言,云「拯困蘇殘,不知何道」。龍正上言:「拯困蘇殘,以生財為本,但財非折色之謂。以折色為財,則取於人而易盡,必知本色為財,則生於地而不窮。今持籌之臣曰設處,曰搜括,曰加派,皆損下之事,聚斂之別名也。民日病,國奚由足?臣謂宜專意墾荒,申明累朝永不起科之制,招集南人巨賈,盡墾荒田,使畿輔、河南、山東菽粟日多,則京倉之積,邊軍之餉,皆可隨宜取給。或平糴,或拜爵,或中監,國家命脈不專倚數千里外之轉運,則民間加派自可盡除。」然是時中原多殘破,有田不得耕,龍正執常理而已。翌日復進《用人探本疏》,帝皆優容焉。

給事中黃雲師劾其學非而博,言偽而辯,又以進墾荒議為陵競。帝不問。時議欲用龍正為吏部,御史黃澍以偽學詆之。十七年正月,左遷南京國子監丞。甫抵家而京師陷。

福王立於南京,用為祠祭員外郎,不就。南京不守,龍正已得疾,遂卒。

贊曰:崇禎時,僉壬相繼枋政,天下多故,事之可言者眾矣。許譽卿諸人,抨擊時宰,有直臣之風。然傅朝佑死杖下,姜埰、熊開元得重譴,而詹爾選抗雷霆之威,顧獲放免。言天子易,言大臣難,信哉。湯開遠以疏遠處僚,侃侃論事,憤惋溢於辭表。就其所列國勢,亦重可慨矣夫!


\end{pinyinscope}