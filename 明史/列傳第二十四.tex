\article{列傳第二十四}

\begin{pinyinscope}
陶安錢用壬詹同朱升崔亮(牛諒答祿與權張籌朱夢炎劉仲質陶凱曾魯任昂李原名樂韶鳳

陶安,字主敬,當塗人。少敏悟,博涉經史,尤長於《易》。元至正初,舉江浙鄉試,授明道書院山長,避亂家居。太祖取太平,安與耆儒李習率父老出迎。太祖召與語。安進曰:「海內鼎沸,豪傑並爭,然其意在子女玉帛,非有撥亂、救民、安天下心。明公渡江,神武不殺,人心悅服,應天順人。以行弔伐,天下不足平也。」太祖問曰:「吾欲取金陵,何如?」安曰:「金陵,古帝王都。取而有之,撫形勝以臨四方,何向不克?」太祖曰:「善。」留參幕府,授左司員外郎,以習為太平知府。習字伯羽,年八十餘矣,卒於官。

安從克集慶,進郎中。及聘劉基、宋濂、章溢、葉琛至,太祖問安:「四人者何如?」對曰:「臣謀略不如基,學問不如濂,治民之才不如溢、琛。」太祖多其能讓。黃州初下,思得重臣鎮之,無逾安者,遂命知黃州。寬租省徭,民以樂業。坐事謫知桐城,移知饒州。陳友定兵攻城,安召吏民諭以順逆,嬰城固守。援兵至,敗去。諸將欲盡戮民之從寇者,安不可。太祖賜詩褒美,州民建生祠事之。

吳元年,初置翰林院,首召安為學士。時徵諸儒議禮,命安為總裁官。尋與李善長、劉基、周禎、滕毅、錢用壬等刪定律令。

洪武元年,命知制誥兼修國史。帝嘗御東閣,與安及章溢等論前代興亡本末。安言喪亂之源,由於驕侈。帝曰:「居高位者易驕,處佚樂者易侈。驕則善言不入,而過不聞;侈則善道不立,而行不顧。如此者,未有不亡。卿言甚當。」又論學術。安曰:「道不明,邪說害之也。」帝曰:「邪說害道,猶美味之悅口,美色之眩目。邪說不去,則正道不興,天下何從治?」安頓首曰:「陛下所言,可謂深探其本矣。」安事帝十餘歲,視諸儒最舊。及官侍從,寵愈渥。御製門帖子賜之曰:「國朝謀略無雙士,翰苑文章第一家。」時人榮之。御史或言安隱過。帝詰曰:「安寧有此,且若何從知?」曰:「聞之道路。」帝大怒,立黜之。

洪武元年四月,江西行省參政闕,帝以命安,諭之曰:「朕渡江,卿首謁軍門,敷陳王道。及參幕府,裨益良多。繼入翰林,益聞讜論。江西上游地,撫綏莫如卿。」安辭。帝不許。至任,政績益著。其年九月卒於官。疾劇,草上時務十二事。帝親為文以祭,追封姑孰郡公。

子晟,洪武中為浙江按察使,以貪賄誅。其兄昱亦坐死。發家屬四十餘人為軍,後死亡且盡。所司復至晟家勾補,安繼妻陳詣闕訴,帝念安功,除其籍。

初,安之裁定諸禮也,廣德錢用壬亦多所論建。

用壬,字成夫。元南榜進士第一,授翰林編修。出使張士誠,留之,授以官。大軍下淮、揚,來歸。累官御史臺經歷,預定律令。尋與陶安等博議郊廟、社稷諸儀。其議釋奠、耤田,皆援據經文及漢、魏以來故事以定其制。詔報可,語詳《禮志》。洪武元年分建六部官,拜用壬禮部尚書。凡禮儀、祭祀、宴享、貢舉諸政,皆專屬禮官。又詔與儒臣議定乘輿以下冠服諸式。時儒生多習古義,而用壬考證尤詳確,然其後諸典禮亦多有更定云。其年十二月,請告歸。

詹同,字同文,初名書,婺源人。幼穎異,學士虞集見之曰:「才子也。」以其弟槃女妻之。至正中,舉茂才異等,除郴州學正。遇亂,家黃州,仕陳友諒為翰林學士承旨。太祖下武昌,召為國子博士,賜名同。時功臣子弟教習內府,諸博士治一經,不盡通貫。同學識淹博,講《易》、《春秋》最善。應教為文,才思泉湧,一時莫與並。遷考功郎中,直起居注。會議袷禘禮,同議當,遂用之。

洪武元年,與侍御史文原吉、起居注魏觀等循行天下,訪求賢才。還,進翰林直學士,遷侍讀學士。帝御下峻,御史中丞劉基曰:「古者公卿有罪,盤水加劍,詣請室自裁,所以勵廉恥,存國體也。」同時侍側,遂取《戴記》及賈誼疏以進,復剴切言之。帝嘗與侍臣言:聲色之害甚於鴆毒,創業之君,為子孫所承式,尤不可不謹。同因舉成湯不邇聲色,垂裕後昆以對。其因事納忠如此。

四年進吏部尚書。六年兼學士承旨,與學士樂韶鳳定釋奠先師樂章。又以渡江以來,征討平定之跡,禮樂治道之詳,雖有紀載,尚未成書,請編《日曆》。帝從之,命同與宋濂為總裁官,吳伯宗等為纂修官。七年五月書成,自起兵臨濠至洪武六年,共一百卷。同等又言:《日曆》秘天府,人不得見。請仿唐《貞觀政要》,分輯聖政,宣示天下。帝從之。乃分四十類,凡五卷,名曰《皇明寶訓》。嗣後凡有政跡,史官日記錄之,隨類增入焉。是年賜敕致仕,語極褒美。未行,帝復命與濂議大祀分獻禮。久之,起承旨,卒。

同以文章結主知,應制占對,靡勿敏贍。帝嘗言文章宜明白顯易,通道術,達時務,無取浮薄。同所為多稱旨,而操行尤耿介,故至老眷注不衰。

子徽,字資善,洪武十五年舉秀才。官至太子少保兼吏部尚書。有才智,剛決不可犯。勤於治事,為帝所獎任。然性險刻。李善長之死,徽有力焉。藍玉下獄,語連徽及子尚寶丞紱,並坐誅。

同從孫希原,為中書舍人,善大書。宮殿城門題額,往往皆希原筆也。

朱升,字允升,休寧人。元末舉鄉薦,為池州學正,講授有法。蘄、黃盜起,棄官隱石門。數避兵逋竄,卒未嘗一日廢學。太祖下徽州,以鄧愈薦,召問時務。對曰:「高築牆,廣積糧,緩稱王。」太祖善之。吳元年,授侍講學士,知制誥,同修國史。以年老,特免朝謁。洪武元年進翰林學士,定宗廟時享齋戒之禮。尋命與諸儒修《女誡》,採古賢后妃事可法者編上之。大封功臣,制詞多升撰,時稱典核。踰年,請老歸,卒年七十二。

升自幼力學,至老不倦。尤邃經學。所作諸經旁注,辭約義精。學者稱楓林先生。子同官禮部侍郎,坐事死。

崔亮,字宗明,槁城人。元浙江行省掾。明師至舊館,亮降,授中書省禮曹主事。遷濟南知府。以母憂歸。洪武元年冬,禮部尚書錢用壬請告去,起亮代之。初,亮居禮曹時,即位、大祀諸禮皆其所條畫,丞相善長上之朝,由是知名。及為尚書,一切禮制用壬先所議行者,亮皆援引故實,以定其議。考證詳確,逾於用壬。

二年,議上仁祖陵曰「英陵」,復請行祭告禮。太常博士孫吾與以漢、唐未有行者,駁之。亮曰:「漢光武加先陵曰『昌』,宋太祖亦加高祖陵曰『欽』,曾祖陵曰『康』,祖陵曰『定』,考陵曰『安』,蓋創業之君尊其祖考,則亦尊崇其陵。既尊其陵,自應祭告,禮固緣人情而起者也。」廷議是亮。頃之,亮言:「《禮運》曰『禮行於郊,則百神受職。』今宜增天下神祗壇於圜丘之東,方澤之西。」又言:「《郊特牲》『器用陶匏』,《周禮疏》『外祀用瓦』。今祭祀用瓷,與古意合。而槃盂之屬,與古尚異,宜皆易以瓷,惟籩用竹。」又請大祀前七日,陪祀官詣中書受誓戒,戒辭如唐禮。又依《周禮》定五祀及四時薦新、稞禮、圭瓚、鬱鬯之制。並言旗纛月朔望致祭,煩而瀆,宜止,行於當祭之月。皆允行。帝嘗謂亮:「先賢有言:『見其生不忍見其死,聞其聲不忍食其肉。』今祭祀省牲於神壇甚邇,心殊未安。」亮乃奏考古省牲之儀,遠神壇二百步。帝大喜。

帝慮郊社諸祭,壇而不屋,或驟雨沾服。亮引宋祥符九年南郊遇雨,於太尉廳望祭,及元《經世大典》壇垣內外建屋避風雨故事,奏之。遂詔:建殿於壇南,遇雨則望祭。而靈星諸祠亦皆因亮言建壇屋焉。時仁祖已配南北郊,而郊祀禮成後,復詣太廟恭謝。亮言宜罷,惟先祭三日,詣太廟以配享告。詔可。帝以日中有黑子,疑祭天不順所致,欲增郊壇從祀之神。亮執奏:漢、唐煩瀆,不宜取法。乃止。帝一日問亮曰:「朕郊祀天地,拜位正中,而百官朝參則班列東西,何也?」亮對曰:「天子祭天,升自午陛,北向,答陽之義也;祭社,升自子陛,南向,答陰之義也。若群臣朝參,當避君上之尊,故升降皆由卯陛,朝班分列東西,以避馳道,其義不同。」亮倉卒占對,必傅經義,多此類。

自郊廟祭祀外,朝賀山呼、百司箋奏、上下冠服、殿上坐墩諸儀及大射軍禮,皆亮所酌定。惟言「大祀帝親省牲,中祀、小祀之牲當遣官代」,帝命:「親祭者皆親省」。又請依唐制,令郡國奏祥瑞。帝以災異所係尤重,命有司驛聞,與亮議異焉。三年九月,卒於官。其後牛諒、答祿與權、張籌、牛夢炎、劉仲質之屬,亦各有所論建。

牛諒,字士良,東平人。洪武元年,舉秀才,為典簿。與張以寧使安南還,稱旨,三遷至禮部尚書。更定釋奠及大祀分獻禮,與詹同等議省牲、冠服。御史答祿與權請祀三皇。太祖下其議禮官,併命考歷代帝王有功德者廟祀之。七年正月,諒奏:三皇立廟京師,春秋致祭。漢、唐以下,就陵立廟。帝為更定行之,亦詳《禮志》。是年怠職,降主事。未幾,復官。後仍以不任職罷。諒著述甚多,為世傳誦。

答祿與權,字道夫,蒙古人。仕元為河南北道廉訪司僉事。入明,寓河南永寧。洪武六年,用薦授秦府紀善,改御史。請重刊律令。盱眙民進瑞麥,與權請薦宗廟。帝曰:「以瑞麥為朕德所致,朕不敢當。其必歸之祖宗。御史言是也。」明年出為廣西按察僉事。未行,復為御史。上書請祀三皇。下禮官議,遂并建帝王廟。且遣使者巡視歷代諸陵寢。設守陵戶二人,三年一祭,其制皆由此始。又請行禘禮,議格不行。改翰林修撰,坐事降典籍,尋進應奉。十一年以年老致仕。禘禮至嘉靖中始定。

張籌,字惟中,無錫人。父翼,嘗勸張士誠將莫天佑降,復請於平章胡美勿僇降人,城中人得完。以詹同薦,授翰林應奉,改禮部主事。奉詔與尚書陶凱編集漢、唐以來籓王事跡,為《歸鑑錄》。洪武九年,由員外郎進尚書,與學士宋濂定諸王妃喪服之制。籌記誦淹博,在禮曹久,諳於歷代禮文沿革。然頗善附會。初,陶安等定圜丘、方澤、宗廟、社稷諸儀,行數年矣。洪武九年,籌為尚書,乃更議合社稷為一壇,罷勾龍、棄配位,奉仁祖配饗,以明祖社尊而親之之道,遂以社稷與郊廟祀並列上祀。識者竊非之。已,出為湖廣參政。十年坐事罰輸作。十二年仍起禮部員外郎。後復官,以事免。

朱夢炎,字仲雅,進賢人。元進士,為金谿丞。太祖召居賓館,命與熊鼎集古事,為質直語,教公卿子弟,名曰《公子書》。洪武十一年,自禮部侍郎進尚書。帝方稽古右文,夢炎援古證今,剖析源流,如指諸掌,文章詳雅有根據。帝甚重之。卒於官。

劉仲質,字文質,分宜人。洪武初,以宜春訓導薦入京,擢翰林典籍,奉命校正《春秋本末》。十五年拜禮部尚書,命與儒臣定釋奠禮,頒行天下學校。每歲春秋仲月,通禮孔子如儀。時國子學新成,帝將行釋菜。侍臣有言:孔子雖聖,人臣也,禮宜一奠再拜。帝曰:「昔周太祖如孔子廟,左右謂不宜拜。周太祖曰:『孔子,百世帝王師,何敢不拜!』今朕有天下,敬禮百神,於先師禮宜加崇。」乃命仲質詳議。仲質請帝服皮弁執圭,詣先師位前,再拜,獻爵,又再拜,退易服。乃詣彞倫堂命講,庶典禮隆重。詔曰「可」。又立學規十二條,合欽定九條,頒賜師生。已,復奉命頒劉向《說苑》、《新序》於學校,令生員講讀。是年冬改華蓋殿大學士,帝為親製誥文。坐事貶御史。後以老致仕。仲質為人厚重篤實,博通經史,文體典確,常當帝意焉。

陶凱,字中立,臨海人。領至正鄉薦,除永豐教諭,不就。洪武初,以薦徵入,同修《元史》。書成,授翰林應奉,教習大本堂,授楚王經。三年七月與崔亮並為禮部尚書,各有敷奏。軍禮及品官墳塋之制,凱議也。其年,亮卒。凱獨任,定科舉式。明年會試,以凱充主考官,取吳伯宗等百二十人程文進御,凱序其首簡,遂為定例。帝嘗諭凱曰:「事死如事生,朕養已不逮,宜盡追遠之道。」凱以太廟已有常祀,乃請於乾清宮左別建奉先殿,以奉神御。明奉先殿之制自此始。五年,凱言:「漢、唐、宋時皆有會要,紀載時政。今起居注雖設,其諸司所領諭旨及奏事簿籍,宜依會要,編類為書,庶可以垂法後世。下臺省府者,宜各置銅櫃藏之,以備稽考,俾無遺闕。」從之。明年二月,出為湖廣參政。致仕。八年起為國子祭酒。明年改晉王府左相。

凱博學,工詩文。帝嘗厭前代樂章多諛辭,或未雅馴,命凱與詹同更撰,甚稱旨。長至侍齋宮,言:宜有篇什以紀慶成。遂命凱首唱,諸臣俱和,而宋濂為之序。其後扈行陪祀,有所獻,帝輒稱善。一時詔令、封冊、歌頌、碑志多出其手云。凱嘗自號「耐久道人」。帝聞而惡之。坐在禮部時朝使往高麗主客曹誤用符驗,論死。

曾魯,字得之,新淦人。年七歲,能暗誦《五經》,一字不遺。稍長,博通古今。凡數千年國體、人才,制度沿革,無不能言者。以文學聞於時。元至正中,魯帥里中豪,集少壯保鄉曲。數具牛酒,為開陳順逆。眾皆遵約束,無敢為非義者。人號其里曰「君子鄉」。

洪武初,修《元史》,召魯為總裁官。史成,賜金帛,以魯居首。乞還山,會編類禮書,復留之。時議禮者蜂起。魯眾中揚言曰:「某禮宜據某說則是,從某說則非。」有辨詰者,必歷舉傳記以告。尋授禮部主事。開平王常遇春薨,高麗遣使來祭。魯索其文視之,外襲金龍黃帕,文不署洪武年號。魯讓曰:「龍帕誤耳,納貢稱籓而不奉正朔,於義何居?」使者謝過,即令易去。安南陳叔明篡立,懼討,遣使入貢以覘朝廷意。主客曹已受其表,魯取副封視之,白尚書詰使者曰:「前王日熞,今何驟更名?」使者不敢諱,具言其實。帝曰:「島夷乃狡獪如此耶!」卻其貢。由是器重魯。

五年二月,帝問丞相:「魯何官?對曰:「主事耳。」即日超六階,拜中順大夫、禮部侍郎。魯以「順」字犯其父諱,辭,就朝請下階。吏部持典制,不之許。戍將捕獲倭人,帝命歸之。儒臣草詔,上閱魯稿大悅,曰:「頃陶凱文已起人意,魯復如此,文運其昌乎!」未幾,命主京畿鄉試。甘露降鐘山,群臣以詩賦獻,帝獨褒魯。是年十二月引疾歸,道卒。淳安徐尊生嘗曰:「南京有博學士二人,以筆為舌者宋景濂,以舌為筆者曾得之也。」魯屬文不留槁,其徒間有所輯錄,亦未成書云。

洪武中,禮部侍郎二十餘人,其知名者,自曾魯外,有劉崧、秦約、陳思道、張衡數人。崧自有傳。

約,崇明人,字文仲。博學,工辭章。洪武初,以文學舉。召試《慎獨箴》,約文第一,立擢禮部侍郎。母老乞歸。已,復召入陳三事,皆切直。仍乞歸,卒。

思道,山陰人,字執中。以進士授刑部主事。帝賞其執法,超拜兵部侍郎,益勵風節,人莫敢干以私。改禮部,乞歸。居家,不殖生產。守令造門不得見。久之,卒。

衡事別載。

任昂,字伯顒,河陰人。元末舉進士,除知寧晉縣,不赴。洪武初,薦起為襄垣訓導,擢御史。十五年拜禮部尚書。帝加意太學,罷祭酒李敬、吳顒,命昂增定監規八條。遂以曹國公李文忠、大學士宋訥兼領國子監事。會司諫關賢上言:「邇來郡邑所司非人,師道不立,歲選士多缺;甚至俊秀生員,點充承差,乖朝廷育賢意。」昂乃奏定天下:歲貢士從翰林院考試,以為殿最。明年,命科舉與薦舉並行。昂條上科場成式,視前加詳,取士制始定。廣東都指揮狄崇、王臻以妾為繼室,乞封。下廷議,昂持不可,從之。遂命昂及翰林院定嫡妾封贈例,因詔偕吏部定文官封贈例十一,廕敘例五,頒示中外。

尋請更定冕服之制。及朝參坐次。又奏毀天下淫祠,正祀典稱號:「蜀祀秦守李冰,附以漢守文翁、宋守張詠;密縣祀太傅卓茂;鈞州祀丞相黃霸;彭澤祀丞相狄仁傑,皆遺愛在民。李龍遷祀於隆州,謝夷甫祀於福州,皆為民捍患。吳丞相陸遜以勞定國,宜祀於吳,以子抗、從子凱配。元總管李黼立祀江州,元帥餘闕立廟安慶,皆以死勤事。從闕守皖,全家殉義者,有萬戶李宗可,宜配享闕廟。」皆報可。明年命以鄉飲酒禮頒天下,復令制大成樂器,分頒學宮。是時,以八事考課外吏,及次第雲南功賞,事不隸禮部,帝皆令昂主其議。尋予告歸。

李原名,字資善,安州人。洪武十五年,以通經儒士舉為御史。二十年使平緬歸,言:「思倫發懷詐窺伺,宜嚴邊備。靖江王以大理印行令旨,非法,為遠人所輕。」稱旨,擢禮部尚書。自是遠方之事多咨之。高麗奏遼東文、高、和、定州皆其國舊壤,乞就鐵嶺屯戍。原名言:「數州皆入元版圖,屬於遼,高麗地以鴨綠江為界。今鐵嶺已置衛,不宜。」復有陳請,帝命諭其國守分土,無生釁。安南歲貢方物,帝念其勞民,原名以帝意諭之,令三年一貢,自是為定制。又以帝命行養老之政,申明府州縣歲貢多寡之數,定官民巾服之式,皆著為令。

初,以答祿與權言,建歷代帝王廟。至是原名請以風后、力牧等三十六人侑享。帝去趙普、安章、阿術而增陳平、馮異、潘美、木華黎,餘悉如原名奏。魯王薨,定喪服之制。進士王希曾請喪出母,原名謂非禮,宜禁。凡郊祀、宗廟、社稷、嶽瀆諸制,先後儒臣論定,時有詳略,帝悉令原名更正之。諸禮臣惟原名在任久。二十三年以老致仕。

樂韶鳳,字舜儀,全椒人。博學能文章。謁太祖於和陽,從渡江,參軍事。洪武三年,授起居注,數遷。六年拜兵部尚書,與中書省、御史臺、都督府定教練軍士法。改侍講學士,與承旨詹同正釋奠先師樂章,編集《大明日曆》。七年,帝以祭禮駕還,應用樂舞前導,命韶鳳等撰詞。因撰《神降祥》、《神貺惠》、《酣酒》、《色荒》、《禽荒》諸曲以進,凡三十九章,曰《回鑾樂歌》,皆寓規諫。禮部具《樂舞圖》以上,命太常肄習之。

明年,帝以舊韻出江左,多失正,命與廷臣參考中原雅音正之。書成,名《洪武正韻》。又命孝陵寢朔望祭祀及登壇脫舄諸禮議,皆詳稽故實。俱從之。尋病免。未幾,復起為祭酒。奉詔定皇太子與諸王往復書答刂禮,考據精詳,屢被褒答。十三年致仕歸,以壽終。弟暉、禮、毅,皆知名。

贊曰:明初之議禮也,宋濂方家居,諸儀率多陶安裁定。大祀禮專用安議,其餘參匯諸說,從其所長:祫禘用詹同,時享用朱升,釋奠、耕耤用錢用壬,五祀用崔亮,朝會用劉基,祝祭用魏觀,軍禮用陶凱。皆能援據經義,酌古準今,郁然成一代休明之治。雖折中斷制,裁自上心,諸臣之功亦曷可少哉。


\end{pinyinscope}