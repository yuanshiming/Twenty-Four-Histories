\article{公主列傳}


○公主

明制,皇姑曰大長公主,皇姊妹曰長公主,皇女曰公主,俱授金冊,祿二千石,婿曰駙馬都尉。親王女曰郡主,郡王女曰縣主,孫女曰郡君,曾孫女曰縣君,玄孫女曰鄉君,婿皆儀賓。郡主祿八百石,餘遞減有差。郡主以下,恩禮既殺,無足書者。今依前史例,作《公主傳》,而駙馬都尉附焉。

○仁祖二女太祖十六女福成慶陽二主附興宗四女成祖五女仁宗七女宣宗二女英宗八女景帝一女憲宗五女孝宗三女睿宗二女世宗五女穆宗六女神宗十女光宗九女熹宗二女莊烈帝六女仁祖二女

太原長公主,淳皇后所生,嫁王七一,早卒。洪武三年追冊,并贈七一榮祿大夫駙馬都尉,遣使具衣冠改葬於盱眙。

曹國長公主,太原主母妹,嫁李貞。主性純孝,助貞理家尤勤儉,早卒。貞攜子文忠避兵,依太祖於滁陽。洪武元年二月追冊主為孝親公主,封貞恩親侯駙馬都尉。先是,兵亂,主未葬,命有司具禮葬於李氏先墓。詔曰:「公主祠堂碑亭,其制悉視功臣之贈爵為王者。」三年改冊主隴西長公主。五年,以文忠貴,加冊曹國長公主,并進貞右柱國曹國公。貞性孝友恭謹。初,文忠守嚴州,屢以征伐事出,皆委貞權掌軍務。文忠克桐廬,以所俘卒送嚴。嚴城空虛,俘卒謀叛去。貞饗其眾,醉而縛之,以歸應天。太祖嘉之,累授官如子爵,賜甲第西華門玄津橋之西。帝數臨幸,太子諸王時往起居,親重無與比。晚歲尤折節謙抑,嘗曰:「富貴而忘貧賤,君子不為也。」十二年冬卒。贈隴西王,謚恭獻。文忠自有傳。太祖十六女臨安公主,洪武九年下嫁李祺,韓國公善長子也。是時始定公主婚禮,先期賜駙馬冠誥並朝服,儀從甚盛。主執婦道甚備。祺,功臣子,帝長婿,頗委任之。四方水旱,每命祺往振濟。二十三年,善長坐事死。祺前卒,主至永樂十九年薨。

寧國公主,孝慈皇后生。洪武十一年下嫁梅殷。殷字伯殷,汝南侯思祖從子也,天性恭謹,有謀略,便弓馬。太祖十六女諸駙馬中,尤愛殷。時李文忠以上公典國學,而殷視山東學政,賜敕褒美,謂殷精通經史,堪為儒宗。當世皆榮之。

帝春秋高,諸王強盛。殷嘗受密命輔皇太孫。及燕師日逼,惠帝命殷充總兵官鎮守淮安。悉心防禦,號令嚴明。燕兵破何福軍,執諸將平安等,遣使假道於殷,以進香為名。殷答曰:「進香,皇考有禁,不遵者為不孝。」王大怒,復書言:「今興兵誅君側惡,天命有歸,非人所能阻。」殷割使者耳鼻縱之,曰:「留汝口為殿下言君臣大義。」王為氣沮。而鳳陽守徐安亦拆浮橋,絕舟檝以遏燕。燕兵乃涉泗,出天長,取道揚州。王即帝位,殷尚擁兵淮上,帝迫公主齧血為書投殷。殷得書慟哭,乃還京。既入見,帝迎勞曰:「駙馬勞苦。」殷曰:「勞而無功耳。」帝默然。

永樂二年,都御史陳瑛奏殷畜養亡命,與女秀才劉氏朋邪詛咒。帝曰:「朕自處之。」因諭戶部考定公、侯、駙馬、伯儀仗從人之數,而別命錦衣衛執殷家人送遼東。明年冬十月,殷入朝,前軍都督僉事譚深、錦衣衛指揮趙曦擠殷笪橋下,溺死,以殷自投水聞。都督同知許成發其事。帝怒,命法司治深、曦罪,斬之,籍其家。遣官為殷治喪,謚榮定,而封許成為永新伯。

初,公主聞殷死,謂上果殺殷,牽衣大哭,問駙馬安在。帝曰:「為主跡賊,無自苦。」尋官殷二子,順昌為中府都督同知,景福為旗手衛指揮使,賜公主書曰:「駙馬殷雖有過失,兄以至親不問。比聞溺死,兄甚疑之。都督許成來首,已加爵賞,謀害之人悉置重法,特報妹知之。」瓦剌灰者,降人也,事殷久,謂深、曦實殺殷,請於帝,斷二人手足,剖其腸祭殷,遂自經死。十二月進封公主為寧國長公主。宣德九年八月薨,年七十一。

初,主聞成祖舉兵,貽書責以大義。不答。成祖至淮北,貽主書,命遷居太平門外,勿罹兵禍。主亦不答。然成祖故重主,即位後,歲時賜與無算,諸王莫敢望。殷孫純,成化中舉進士,知定遠縣,忤上官,棄歸。襲武階,為中都副留守。

崇寧公主,洪武十七年下嫁牛城,未幾薨。

安慶公主,寧國主母妹。洪武十四年下嫁歐陽倫。倫頗不法。洪武末,茶禁方嚴,數遣私人販茶出境,所至繹騷,雖大吏不敢問。有家奴周保者尤橫,輒呼有司科民車至數十輛。過河橋巡檢司,擅捶辱司吏。吏不堪,以聞。帝大怒,賜倫死,保等皆伏誅。

汝寧公主,洪武十五年與懷慶、大名二主先後下嫁,而主下嫁陸賢,吉安侯仲亨子也。

懷慶公主,母成穆孫貴妃。下嫁王寧。寧,壽州人,既尚主,掌後軍都督府事。建文中,嘗洩中朝事於燕,籍其家,繫錦衣衛獄。成祖即位,稱寧孝於太祖,忠於國家,正直不阿,橫遭誣構,封永春侯,予世券。寧能詩,頗好佛。嘗侍帝燕語,勸帝誦佛經飯僧,為太祖資福。帝不懌,自是恩禮漸衰。久之,坐事下獄,見原,卒。子貞亮,官羽林前衛僉事,先寧卒。宣德十年,貞亮子彞援詔書言公主嫡孫當嗣侯。不許,命以衛僉事帶俸,奉主祀。寧又有子貞慶,工詩,與劉溥等稱「十才子」。

大名公主,下嫁李堅。堅,武陟人。父英,洪武初為驍騎右衛指揮僉事。從征雲南陣沒,贈指揮使。堅有才勇,既尚主,掌前軍都督府事。建文初,以左副將軍從伐燕。及戰,勝負略相當,封灤城侯,予世券。滹沱河之戰,燕卒薛祿刺堅墮馬被擒,械送北平,道卒。子莊年七歲,嗣侯。成祖即位,莊父姓名在姦黨中,以主故獲宥。主懼禍,遂納侯誥券。宣德元年,主薨。莊在南京師事劉溥,放浪詩酒,以壽終。

福清公主,母鄭安妃。洪武十八年下嫁張麟,鳳翔侯龍子也。麟未嗣侯卒。永樂十五年,主薨。

壽春公主,洪武十九年下嫁傅忠,穎國公友德子也。先是,九年二月定制:公主未受封者,歲給糸寧絲紗絹布線;已封,賜莊田一區,歲征租一千五百石,鈔二千貫。主為太祖所愛,賜吳江縣田一百二十餘頃,皆上腴,歲入八千石,踰他主數倍。二十一年薨,賜明器儀杖以葬。

十公主,早薨。

南康公主,洪武二十一年下嫁胡觀,東川侯海子也。海嘗以罪奪祿田。及觀尚主,詔給田如故。觀初在選中,帝命黃巖,徐宗實教之。既婚,督課益嚴,又為書數千言,引古義相戒勸。觀執弟子禮甚恭。太祖為大喜。建文三年,觀從李景隆北征,為燕兵所執。永樂初,奉使晉府還,科道官劾觀僭乘晉王所賜棕輿。詔姑宥之。已,都御史陳瑛等劾觀強取民間子女,又娶娼為妾,且言:「預知李景隆逆謀,陛下曲加寬宥,絕無悛心,宜正其罪。」遂罷觀朝請,尋自經死。宣德中,主為子忠乞嗣,詔授孝陵衛指揮僉事,進同知。正統三年,主薨。永嘉公主,母郭惠妃。洪武二十二年下嫁郭鎮,武定侯英子也。英卒,鎮不得嗣。宣德十年,主乞以子珍嗣,語在《英傳》。景泰六年,主薨。世宗即位,元孫勛有寵,為主乞追謚,特賜謚貞懿。

十三公主,早薨。

含山公主,母高麗妃韓氏。洪武二十七年下嫁尹清。建文初,清掌後府都督事,先主卒。主至天順六年始薨,年八十有二。

汝陽公主,永嘉主同母妹,與含山主同年下嫁謝達。達父彥,鳳陽人,少育於孫氏,冒其姓。數從征討有功,累官前軍都督僉事,詔復謝姓,遷其子尚主。仁宗即位,主以屬尊,與寧國、懷慶、大名、南康、永嘉、含山、寶慶七主皆進稱大長公主。自後諸帝即位,公主進封長公主、大長公主皆如制。

寶慶公主,太祖最幼女,下嫁趙輝。輝父和以千戶從征安南陣沒,輝襲父官。先是,成祖即位,主甫八歲,命仁孝皇后撫之如女。永樂十一年,輝以千戶守金川門,年二十餘,狀貌偉麗,遂選以尚主。主既為后所撫,裝齎視他主倍渥,婚夕特詔皇太子送入邸。主性純淑,宣德八年薨。輝至成化十二年始卒。凡事六朝,歷掌南京都督及宗人府事。家故豪侈,姬妾至百餘人,享有富貴者六十餘年,壽九十。

福成公主,南昌王女,母王氏。嫁王克恭。克恭嘗為福建行省參政,後改福州衛指揮使。

慶陽公主,蒙城王女,嫁黃琛。琛本名寶,武昌人,以帳前參隨舍人擢兵馬副指揮。太祖愛其謹厚,配以王女。累從征討,積功至龍江翼守禦千戶。洪武元年冊兩王女為公主,授克恭、琛為駙馬都尉,遷琛淮安衛指揮使。四年三月,禮官上言:「皇姪女宜改封郡主,克恭、琛當上駙馬都尉誥。」帝曰:「朕惟姪女二人,不忍遽加降奪,其稱公主駙馬如故。」公主歲給祿米五百石,視他主減三之二,駙馬止食本官俸。擢琛中都留守,卒官。子鉉至都督僉事。主至建文時,改封慶成郡主。燕師南下,主嘗詣軍中議和,蓋成祖從姊。或謂福成、慶陽皆太祖從姊者,誤也。

興宗四女江都公主,洪武二十七年下嫁耿璿,長興侯炳文子也。累官前軍都督僉事。主為懿文太子長女。初稱江都郡主,建文元年進公主,璿為駙馬都尉。炳文之伐燕也,璿嘗勸直搗北平。會炳文罷歸,謀不用。永樂初,稱疾不出,坐罪死。主復降為郡主,憂卒。

宜倫郡主,永樂十五年下嫁於禮。

三女,無考。

南平郡主,未下嫁,永樂十年薨,追冊。

○成祖五女

永安公主,下嫁袁容。容,壽州人,父洪以開國功,官都督。洪武二十八年選容為燕府儀賓,配永安郡主。燕兵起,有戰守功。永樂元年進郡主為公主,容駙馬都尉;再論功,封廣平侯,祿一千五百石,予世券。凡車駕巡幸,皆命容居守。

初,都指揮款台乘馬過容門,容怒其不下,棰之幾死。帝聞之,賜趙王高燧書曰:「自洪武來,往來駙馬門者,未聞令下馬也。昔晉王敦為駙馬,縱恣暴橫,卒以滅亡。汝其以書示容,令械辱款台之人送京師。」容由是斂戢。

十五年,主薨,停容侯祿。宣宗即位,復故。卒,贈沂國公,謚忠穆。子禎嗣,卒,無子。庶弟瑄,正統初乞嗣。帝曰:「容封以公主恩,禎嗣以公主子。瑄庶子,可長陵衛指揮僉事。」天順元年詔復侯爵,卒。弟秀,成化十五年嗣,卒。侄輅乞嗣侯,言官持不可。帝曰:「詔書許子孫嗣。輅,容孫也,輅後毋嗣,仍世衛僉事。」輅卒,子夔,弘治間乞嗣侯。不許。

永平公主,下嫁李讓。讓,舒城人,與袁容同歲選為燕府儀賓。燕兵起,帥府兵執謝貴等,取大寧,戰白溝河有功,署掌北平布政司事,佐仁宗居守。其父申,官留守左衛指揮同知。惠帝欲誘致讓,曰:「讓來,吾宥爾父。」讓不從,力戰破平安兵。帝遂殺申,籍其家,姻族皆坐死或徒邊。永樂元年進讓駙馬都尉,封富陽侯,食祿千石,掌北京行部事。卒,贈景國公,謚恭敏。子茂芳嗣侯。仁宗即位,以茂芳母子在先帝時有逆謀,廢為庶人,追奪其父讓並三代誥券毀之。是年,茂芳死。正統九年,主薨。天順元年詔與茂芳子輿伯爵,卒。成化間,授輿子欽長陵衛指揮僉事。

安成公主,文皇后生。成祖即位,下嫁宋琥,西寧侯晟子也。正統八年,主薨。

咸寧公主,安成主同母妹。永樂九年下嫁宋瑛,琥弟也。襲西寧侯。正統五年,主薨。十四年,瑛與武進伯朱冕禦也先於陽和,戰死。

常寧公主,下嫁沐昕,西平侯英子。主恭慎有禮,通《孝經》、《女則》。永樂六年薨,年二十二。

○仁宗七女

嘉興公主,昭皇后生。宣德三年下嫁井源。正統四年薨。後十年,源死土木之難。

慶都公主,宣德三年下嫁焦敬。正統五年薨。

清河公主,宣德四年下嫁李銘。八年薨。

真定公主,母李賢妃,與清河主同年下嫁王誼。景泰元年薨。

德安公主,早薨。仁宗即位之十月,與蘄王瞻垠同日追封,謚悼簡。冊辭謂第四女,蓋早殤,名次未定也。又五女延平公主,六女德慶公主,俱未下嫁薨。

○宣宗二女

順德公主,正統二年下嫁石璟。璟,昌黎人。天順五年,曹欽反,璟帥眾殺賊,擒其黨脫脫。詔獎勞。成化十四年奉祀南京,踰年卒。

常德公主,章皇后生。正統五年下嫁薛桓。成化六年薨。

○英宗八女

重慶公主,與憲宗同母。天順五年下嫁周景。景字德彰,安陽人,好學能書。英宗愛之,閑燕遊幸多從。憲宗立,命掌宗人府事。居官廉慎,詩書之外無所好。主事舅姑甚孝,衣履多手製,歲時拜謁如家人禮。景每早朝,主必親起視飲食。主之賢,近世未有也。弘治八年,景卒。又四年,主薨,年五十四。子賢歷官都指揮僉事,有聲。

嘉善公主,母王惠妃。成化二年下嫁王增,兵部尚書驥孫也。弘治十二年薨。

淳安公主,成化二年下嫁蔡震。震行醇謹。正德中,劉瑾下獄,詔廷訊。有問者,瑾輒指其人附己,廷臣無敢詰。震歷聲曰:「我皇家至戚,應不附爾!」趣獄卒考掠之,瑾乃服罪,以是知名。嘉靖中卒,贈太保,謚康僖。

崇德公主,母楊安妃。成化二年下嫁楊偉,興濟伯善孫也。弘治二年薨。

廣德公主,母萬宸妃。成化八年下嫁樊凱。二十年八月薨。

宜興公主,母魏德妃。成化九年下嫁馬誠。正德九年薨。

隆慶公主,母高淑妃。成化九年下嫁游泰。十五年薨。

嘉祥公主,母妃劉氏。成化十三年下嫁黃鏞。後六年薨。

○景帝一女

固安公主,英宗復辟,降稱郡主。成化時,年已長,憲宗以閣臣奏,五年十一月下嫁王憲。禮儀視公主,以故尚書蹇義賜第賜之。

○憲宗五女

仁和公主,弘治二年下嫁齊世美。嘉靖二十三年薨。

永康公主,弘治六年下嫁崔元。元,代州人,世宗入繼,以迎立功封京山侯,給誥券。禮部言:「奉迎乃臣子之分,遽膺封爵,無故事。」帝曰:「永樂初年,太宗入繼大統,駙馬都尉王寧以翊戴功封永春侯,何得言無故事。」給事中底蘊、御史高越等連章論其不可。皆不聽。已,坐張延齡事下詔獄,尋釋。元好交文士,播聲譽,寵幸優渥,勛臣戚畹莫敢望焉。嘉靖二十八年卒。贈左柱國太傅兼太子太傅,謚榮恭。駙馬封侯贈官不以軍功自元始。主先元薨。

德清公主,弘治九年下嫁林岳。岳字鎮卿,應天人,少習舉子業,奉母孝,撫弟巒極友愛。主亦有賢行,事姑如齊民禮。岳卒於正德十三年,主孀居三十一年始薨。

長泰公主,成化二十三年薨,追冊。

仙遊公主,弘治五年薨,追冊。

○孝宗三女

太康公主,弘治十一年薨,未下嫁。

永福公主,嘉靖二年下嫁鄔景和。景和,崑山人,嘗奉旨直西苑,撰玄文,以不諳玄理辭。帝不悅。時有事清馥殿,在直諸臣俱行祝釐禮,景和不俟禮成而出。已而賞賚諸臣,景和與焉。疏言:「無功受賞,懼增罪戾。乞容辭免,俾洗心滌慮,以效他日馬革裹尸、啣環結草之報。」帝大怒,謂詛咒失人臣禮,削職歸原籍,時主已薨矣。三十五年入賀聖誕畢,因言:「臣自五世祖寄籍錦衣衛,世居北地。今被罪南徙,不勝犬馬戀主之私。扶服入賀,退而私省公主墳墓,丘封翳然,荊棘不剪。臣切自念,狐死尚正首丘,臣託命貴主,獨與逝者魂魄相弔於數千里外,不得春秋祭掃,拊心傷悔,五內崩裂。臣之罪重,不敢祈恩,惟陛下幸哀故主,使得寄籍原衛,長與相依,死無所恨。」帝憐而許之。隆慶二年復官。卒贈少保,謚榮簡。

永淳公主,下嫁謝詔。

○睿宗二女

長寧公主,早薨。善化公主,早薨。嘉靖四年,二主同日追冊。

○世宗五女

常安公主,未下嫁。嘉靖二十八年薨,追冊。

思柔公主,後常安主二月薨,年十二,追冊。

寧安公主,嘉靖三十四年下嫁李和。

歸善公主,嘉靖二十三年薨,追冊,葬祭視太康主。

嘉善公主,嘉靖三十六年下嫁許從誠。四十三年薨。

○穆宗六女

蓬萊公主,早薨。

太和公主,早薨。隆慶元年與蓬萊主同日追冊。

壽陽公主,萬曆九年下嫁侯拱辰。國本議起,拱辰掌宗人府,亦具疏力爭。卒贈太傅,謚榮康。

永寧公主,下嫁梁邦瑞。萬曆三十五年薨。

瑞安公主,神宗同母妹。萬曆十三年下嫁萬煒。崇禎時,主累加大長公主。所產子及庶子長祚、弘祚皆官都督。煒官至太傅,管宗人府印。嘗以親臣侍經筵,每文華進講,佩刀入直。李建泰西征,命煒以太牢告廟,年七十餘矣。國變,同子長祚死於賊。弘祚投水死,長祚妻李氏亦赴井死。

延慶公主,萬曆十五年下嫁王昺。昺嘗救御史劉光復,觸帝怒,削職。光宗立,復官。

○神宗十女

榮昌公主,萬曆二十四年下嫁楊春元。四十四年,春元卒。久之,主薨。

壽寧公主,二十七年下嫁冉興讓。主為神宗所愛,命五日一來朝,恩澤異他主。崇禎時,洛陽失守,莊烈帝命興讓同太監王裕民、給事中葉高標往慰福世子於河北。都城陷,興讓死於賊。

靜樂、雲和、雲夢、靈丘、仙居、泰順、香山、天台八公主,皆早世,追冊。

○光宗九女

懷淑公主,七歲而薨,追冊。餘五女皆早世,未封。

寧德公主,下嫁劉有福。

遂平公主,天啟七年下嫁齊贊元。崇禎末,贊元奔南京,主前薨。

樂安公主,下嫁鞏永固。永固,字洪圖,宛平人,好讀書,負才氣。崇禎十六年二月,帝召公、侯、伯於德政殿,言:「祖制,勛臣駙馬入監讀書,習武經弓馬。諸臣各有子弟否?」成國公朱純臣、定國公徐允禎等皆以幼對。而永固獨上疏,請肄業太學。帝褒答之。總督趙光抃以邊事繫獄,特疏申救。又請復建文皇帝廟謚。事雖未行,時論韙焉。甲申春,賊破宣、大,李邦華請太子南遷,為異議所格。及事急,帝密召永固及新樂侯劉文炳護行。叩頭言:「親臣不藏甲,臣等難以空手搏賊。」皆相向涕泣。十九日,都城陷。時公主已薨,未葬,永固以黃繩縛子女五人繫柩旁,曰:「此帝甥也,不可汙賊手。」舉劍自刎,闔室自焚死。

熹宗二女。皆早世。

○莊烈帝六女

坤儀公主,周皇后生。追謚。

長平公主,年十六,帝選周顯尚主。將婚,以寇警暫停。城陷,帝入壽寧宮,主牽帝衣哭。帝曰:「汝何故生我家!」以劍揮斫之,斷左臂;又斫昭仁公主於昭仁殿。越五日,長平主復蘇。大清順治二年上書言:「九死臣妾,局蹐高天,願髡緇空王,稍申罔極。」詔不許,命顯復尚故主,土田邸第金錢車馬錫予有加。主涕泣。逾年病卒。賜葬廣寧門外。

餘三女,皆早世,無考。

