\article{擴廓帖木兒、陳友定、把匝剌瓦爾密列傳}




擴廓帖木兒,沈丘人。本王姓,小字保保,元平章察罕帖木兒甥也。察罕養為子,順帝賜名擴廓帖木兒。汝、潁盜起,中原大亂,元師久無功。至正十二年,察罕起義兵,戰河南、北,擊賊關中、河東,復汴梁,走劉福通,平山東,降田豐,滅賊幾盡。既而總大軍圍益都,田豐叛,察罕為王士誠所刺,事具《元史》。察罕既死,順帝即軍中拜擴廓太尉、中書平章政事、知樞密院事,如察罕官。帥兵圍益都,穴地而入,克之。執豐、士誠,剖其心以祭察罕,縛陳猱頭等二十餘人獻闕下。東取莒州,山東地悉定。至正二十二年也。

初,察罕定晉、冀,孛羅帖木兒在大同,以兵爭其地,數相攻,朝廷下詔和解,終不聽。擴廓既平齊地,引軍還,駐太原,與孛羅構難如故。會朝臣老的沙、禿堅獲罪於太子,出奔孛羅,孛羅匿之。詔削孛羅官,解其兵柄。孛羅遂舉兵反,犯京師,殺丞相搠思監,自為左丞相,老的沙為平章,禿堅知樞密院。太子求援於擴廓,擴廓遣其將白鎖住以萬騎入衛,戰不利,奉太子奔太原。踰年,擴廓以太子令舉兵討孛羅,入大同,進薄大都。順帝乃襲殺孛羅於朝。擴廓從太子入覲,以為太傅、左丞相。當是時,微擴廓,太子幾殆。擴廓功雖高,起行間,驟至相位,中朝舊臣多忌之者。而擴廓久典軍,亦不樂在內,居兩月,即請出治兵,南平江、淮。詔許之,封河南王,俾總天下兵,代皇太子出征,分省中官屬之半以自隨。鹵簿甲仗互數十里,軍容甚盛。時太祖已滅陳友諒,盡有江、楚地,張士誠據淮東、浙西。擴廓知南軍強,未可輕進,乃駐軍河南,檄關中四將軍會師大舉。四將軍者,李思齊、張思道、孔興、脫列伯也。

思齊,羅山人,與察罕同起義兵,齒位略相埒。得檄大怒曰:「吾與若父交,若髮未燥,敢檄我耶!」令其下一甲不得出武關。思道等亦皆不聽調。擴廓歎曰:「吾奉詔總天下兵,而鎮將不受節制,何討賊為!」乃遣其弟脫因帖木兒以一軍屯濟南,防遏南軍,而自引兵西入關,攻思齊等。思齊等會兵長安,盟於含元殿舊基,併力拒擴廓。相持經年,數百戰未能決。順帝使使諭令罷兵,專事江、淮。擴廓欲遂定思齊等,然後引軍東。乃遣其驍將貊高趨河中,欲出不意搗鳳翔,覆思齊巢穴。貊高所將多孛羅部曲,行至衛輝,軍變,脅貊高叛擴廓,襲衛輝、彰德據之,罪狀擴廓於朝。

初,太子之奔太原也,欲用唐肅宗靈武故事自立。擴廓不可。及還京師,皇后諭指令以重兵擁太子入城,脅順帝禪位。擴廓未至京三十里,留其軍,以數騎入朝。由是太子銜之,而順帝亦心忌擴廓。廷臣嘩言擴廓受命平江、淮,乃西攻關中,今罷兵不奉詔,跋扈有狀。及貊高奏至,順帝乃削擴廓太傅、中書左丞相,令以河南王就食邑汝南,分其軍隸諸將;而以貊高知樞密院事兼平章,總河北軍,賜其軍號「忠義功臣」。太子開撫軍院於京師,總制天下兵馬,專備擴廓。

擴廓既受詔,退軍澤州,其部將關保亦歸於朝。朝廷知擴廓勢孤,乃詔李思齊等東出關,與貊高合攻擴廓,而令關保以兵戍太原。擴廓憤甚,引軍據太原,盡殺朝廷所置官吏。於是順帝下詔盡削擴廓官爵,令諸軍四面討之。是時明兵已下山東,收大梁。梁王阿魯溫,察罕父也,以河南降。脫因帖木兒敗走,餘皆望風降遁,無一人抗者。既迫潼關,思齊等倉皇解兵西歸,而貊高、關保皆為擴廓所擒殺。順帝大恐,下詔歸罪於太子,罷撫軍院,悉復擴廓官,令與思齊等分道南討。詔下一月,明兵已逼大都,順帝北走。擴廓入援不及,大都遂陷,距察罕死時僅六年云。

明兵已定元都,將軍湯和等自澤州徇山西。擴廓遣將禦之,戰於韓店,明師大敗。會順帝自開平命擴廓復大都,擴廓乃北出雁門,將由保安徑居庸以攻北平。徐達、常遇春乘虛搗太原,擴廓還救。部將豁鼻馬潛約降於明。明兵夜劫營,營中驚潰。擴廓倉卒以十八騎北走,明兵遂西入關。思齊以臨洮降。思道走寧夏,其弟良臣以慶陽降,既而復叛,明兵破誅之。於是元臣皆入於明,唯擴廓擁兵塞上,西北邊苦之。

洪武三年,太祖命大將軍徐達總大兵出西安,搗定西。擴廓方圍蘭州,趨赴之。戰於沈兒峪,大敗,盡亡其眾,獨與妻子數人北走,至黃河,得流木以渡,遂奔和林。時順帝崩,太子嗣立,復任以國事。踰年,太祖復遣大將軍徐達、左副將軍李文忠、征西將軍馮勝將十五萬眾,分道出塞取擴廓。大將軍至嶺北,與擴廓遇,大敗,死者數萬人。劉基嘗言於太祖曰:「擴廓未可輕也。」至是帝思其言,謂晉王曰:「吾用兵未嘗敗北。今諸將自請深入,敗於和林,輕信無謀,致多殺士卒,不可不戒。」明年,擴廓復攻雁門,命諸將嚴為之備,自是明兵希出塞矣。其後,擴廓從其主徙金山,卒於哈剌那海之衙庭,其妻毛氏亦自經死,蓋洪武八年也。

初,察罕破山東,江、淮震動。太祖遣使通好。元遣戶部尚書張昶、郎中馬合謀浮海如江東,授太祖榮祿大夫、江西等處行中書省平章政事,賜以龍衣御酒。甫至而察罕被刺,太祖遂不受,殺馬合謀,以張昶才,留官之。及擴廓視師河南,太祖乃復遣使通好,擴廓輒留使者不遣。凡七致書,皆不答。既出塞,復遣人招諭,亦不應。最後使李思齊往。始至,則待以禮。尋使騎士送歸,至塞下,辭曰:「主帥有命,請公留一物為別。」思齊曰:「吾遠來無所齎。」騎士曰:「願得公一臂。」思齊知不免,遂斷與之。還,未幾死。太祖以是心敬擴廓。一日,大會諸將,問曰:「天下奇男子誰也?」皆對曰:「常遇春將不過萬人,橫行無敵,真奇男子。」太祖笑曰:「遇春雖人傑,吾得而臣之。吾不能臣王保保,其人奇男子也。」竟冊其妹為秦王妃。

張昶仕明,累官中書省參知政事,有才辨,明習故事,裁決如流,甚見信任。自以故元臣,心嘗戀戀。會太祖縱降人北還,昶附私書訪其子存亡。楊憲得書稿以聞,下吏按問。昶大書牘背曰:「身在江南,心思塞北。」太祖乃殺之。而擴廓幕下士不屈節縱出塞者,有蔡子英。

子英,永寧人,元至正中進士。察罕開府河南,辟參軍事,累薦至行省參政。元亡,從擴廓走定西。明兵克定西,擴廓軍敗,子英單騎走關中,亡入南山。太祖聞其名,使人繪形求得之,傳詣京師。至江濱,亡去,變姓名,賃舂。久之,復被獲。械過洛陽,見湯和,長揖不拜。抑之跪,不肯。和怒,爇火焚其鬚,不動。其妻適在洛,請與相見,子英避不肯見。至京,太祖命脫械以禮遇之,授以官,不受。退而上書曰:「陛下乘時應運,削平群雄,薄海內外,莫不賓貢。臣鼎魚漏網,假息南山。曩者見獲,復得脫亡。七年之久,重煩有司追跡。而陛下以萬乘之尊,全匹夫之節,不降天誅,反療其疾,易冠裳,賜酒饌,授以官爵,陛下之量包乎天地矣。臣感恩無極,非不欲自竭犬馬,但名義所存,不敢輒渝初志。自惟身本韋布,智識淺陋,過蒙主將知薦,仕至七命,躍馬食肉十有五年,愧無尺寸以報國士之遇。及國家破亡,又復失節,何面目見天下士。管子曰:『禮義廉恥,國之四維。』今陛下創業垂統,正當挈持大經大法,垂示子孫臣民。奈何欲以無禮義、寡廉恥之俘囚,廁諸維新之朝、賢士大夫之列哉!臣日夜思維,咎往昔之不死,至於今日,分宜自裁。陛下待臣以恩禮,臣固不敢賣死立名,亦不敢偷生茍祿。若察臣之愚,全臣之志,禁錮海南,畢其餘命,則雖死之日,猶生之年。或王蠋閉戶以自縊,李芾闔門以自屠,彼非惡榮利而樂死亡,顧義之所在,雖湯鑊有不得避也。渺焉之軀,上愧古人,死有餘恨,惟陛下裁察。」帝覽其書,益重之,館之儀曹。忽一夜大哭不止。人問其故,曰:「無他,思舊君耳。」帝知不可奪,洪武九年十二月命有司送出塞,令從故主於和林。

陳友定,一名有定,字安國,福清人,徙居汀之清流。世業農。為人沉勇,喜遊俠。鄉里皆畏服。至正中,汀州府判蔡公安至清流募民兵討賊,友定應募。公安與語,奇之,使掌所募兵,署為黃土砦巡檢。以討平諸山寨功,遷清流縣尹。陳友諒遣其將鄧克明等陷汀、邵,略杉關。行省授友定汀州路總管禦之。戰於黃土,大捷,走克明。踰年,克明復取汀州,急攻建寧。守將完者帖木兒檄友定入援,連破賊,悉復所失郡縣。行省上其功第一,進參知政事。已,置分省於延平,以友定為平章,於是友定盡有福建八郡之地。

友定以農家子起傭伍,目不知書。及據八郡,數招致文學知名士,如閩縣鄭定、廬州王翰之屬,留置幕下。麤涉文史,習為五字小詩,皆有意理。然頗任威福,所屬違令者輒承制誅竄不絕。漳州守將羅良不平,以書責之曰:「郡縣者,國家之土地。官司者,人主之臣役。而廥廩者,朝廷之外府也。今足下視郡縣如室家,驅官僚如圉僕,擅廥廩如私藏,名雖報國,實有鷹揚跋扈之心。不知足下欲為郭子儀乎,抑為曹孟德乎?」友定怒,竟以兵誅良。而福清宣慰使陳瑞孫、崇安令孔楷、建陽人詹翰拒友定不從,皆被殺。於是友定威震八閩,然事元未嘗失臣節。是時張士誠據浙西,方國珍據浙東,名為附元,歲漕粟大都輒不至。而友定歲輸粟數十萬石,海道遼遠,至者嘗十三四。順帝嘉之,下詔褒美。

太祖既定婺州,與友定接境。友定侵處州。參政胡深擊走之,遂下浦城,克松溪,獲友定將張子玉,與朱亮祖進攻建寧,破其二柵。友定遣阮德柔以兵四萬屯錦江,繞出深後,斷其歸路,而自帥牙將賴政等以銳師搏戰,德柔自後夾擊。深兵敗,被執死。太祖既平方國珍,即發兵伐友定。將軍胡廷美、何文輝由江西趨杉關,湯和、廖永忠由明州海道取福州,李文忠由浦城取建寧,而別遣使至延平,招諭友定。友定置酒大會諸將及賓客,殺明使者,瀝其血酒甕中,與眾酌飲之。酒酣,誓於眾曰:「吾曹並受元厚恩,有不以死拒者,身磔,妻子戮。」遂往視福州,環城作壘。距壘五十步,輒築一臺,嚴兵為拒守計。已而聞杉關破,急分軍為二,以一軍守福,而自帥一軍守延平,以相掎角。及湯和等舟師抵福之五虎門,平章曲出引兵逆戰敗,明兵緣南臺蟻附登城。守將遁去,參政尹克仁、宣政使朵耳麻不屈死,僉院柏帖木兒積薪樓下,殺妻妾及二女,縱火自焚死。

廷美克建寧,湯和進攻延平。友定欲以持久困之,諸將請出戰,不許。數請不已,友定疑所部將叛,殺蕭院判。軍士多出降者。會軍器局災,城中礮聲震地,明師知有變,急攻城。友定呼其屬訣曰:「大事已去,吾一死報國,諸君努力。」因退入省堂,衣冠北面再拜,仰藥死。所部爭開城門納明師。師入,趨視之,猶未絕也。舁出水東門,適天大雷雨,友定復蘇。械送京師。入見,帝詰之。友定歷聲曰:「國破家亡,死耳,尚何言。」遂併其子海殺之。

海,一名宗海,工騎射,亦喜禮文士。友定既被執,自將樂歸於軍門,至是從死。

元末所在盜起,民間起義兵保障鄉里,稱元帥者不可勝數,元輒因而官之。其後或去為盜,或事元不終,惟友定父子死義,時人稱完節焉。友定既死,興化、泉州皆望風納疑。獨漳州路達魯花赤迭里彌實具公服,北面再拜,引斧斫印章,以佩刀剚喉而死。時云「閩有三忠」,謂友定、柏帖木兒、迭里彌實也。

鄭定,字孟宣。好擊劍,為友定記室。及敗,浮海入交、廣間。久之,還居長樂。洪武末,累官至國子助教。王翰,字用文,仕元為潮州路總管。友定敗,為黃冠,棲永泰山中者十載。太祖聞其賢,強起之,自刎死,有子偁知名。

為友定所辟者,又有伯顏子中。子中,其先西域人,後仕江西,因家焉。子中明《春秋》,五舉有司不第,行省辟授東湖書院山長,遷建昌教授。子中雖儒生,慷慨喜談兵。江西盜起,授分省都事,使守贛州,而陳友諒兵已破贛。子中倉卒募吏民,與鬥城下,不勝,脫身間道走閩。陳友定素知之,辟授行省員外郎。出奇計,以友定兵復建昌,浮海如元都獻捷。累遷吏部侍郎。持節發廣東何真兵救閩,至則真已降於廖永忠。子中跳墜馬,折一足,致軍前。永忠欲脅降之,不屈。永忠義而舍之。乃變姓名,冠黃冠,遊行江湖間。太祖求之不得,簿錄其妻子,子中竟不出。嘗齎鴆自隨,久之事浸解,乃還鄉里。洪武十二年詔郡縣舉元遺民。布政使沈立本密言子中於朝,以幣聘。使者至,子中太息曰:「死晚矣。」為歌七章,哭其祖父師友,飲鴆而死。

當元亡時,守土臣仗節死者甚眾。明兵克太平,總管靳義赴水死。攻集慶,行臺御史大夫福壽戰敗,嬰城固守。城破,猶督兵巷戰,坐伏龜樓指揮。左右或勸之遁,福壽叱而射之,遂死於兵。參政伯家奴、達魯花赤達尼達思等皆戰死。克鎮江,守將段武、平章定定戰死。克寧國,百戶張文貴殺妻妾自刎死。克徽州,萬戶吳訥戰敗自殺。克婺州,浙東廉訪使楊惠、婺州達魯花赤僧住戰死。克衢州,總管馬浩赴水死。石抺宜孫守處州,其母與弟厚孫先為明兵所獲,令為書招之。不聽。比克處,宜孫戰敗,走建寧,收集士卒,欲復處州。攻慶元,為耿再成所敗,還走建寧。半道遇鄉兵,被殺,部將李彥文葬之龍泉。太祖嘉其忠,遣使致祭,復其處州生祠。又祠福壽於應天,餘闕於安慶,李黼於江州。闕、黼事具《元史》。

其後大軍北克益都,平章普顏不花不屈死。克東昌,平章申榮自經死。真定路達魯花赤鈒納錫彰聞王師取元都,朝服登城西崖,北面再拜,投崖死。克奉元,西臺御史桑哥失里與妻子俱投崖死,左丞拜泰古逃入終南山,郎中王可仰藥死,檢校阿失不花自經死。三原縣尹朱春謂其妻曰:「吾當死以報國。」妻曰:「君能盡忠,妾豈不能盡節。」亦俱投繯死。又大軍攻永州,右丞鄧祖勝固守,食盡力窮,仰藥死。克梧州,吏部尚書普顏帖木兒戰死,張翱赴水死。克靖江,都事趙元隆、陳瑜、劉永錫,廉訪使僉事帖木兒不花,元帥元禿蠻,萬戶董丑漢,府判趙世傑皆自殺。至如劉福通、徐壽輝、陳友諒等所破郡縣,守吏將帥多死節者,已見《元史》,不具載,載其見《明實錄》者。

又有劉諶,江西人,為仁壽教官。明玉珍入蜀,棄官隱瀘州。玉珍欲官之,不就。鳳山趙善璞隱深山,明玉珍聘為學士,亦不就。而張士誠破平江時,參軍楊椿挺身戰,刃交於胸,嗔目怒罵死,妻亦自經。士誠又以書幣徵故左司員外郎楊乘於松江,乘具酒醴告祖禰,顧西日晴明,曰:「人生晚節,如是足矣。」夜分自經死。其親籓死事最烈者,有雲南梁王。梁王把匝剌瓦爾密,元世祖第五子雲南王忽哥赤之裔也。封梁王,仍鎮雲南。順帝之世,天下多故,雲南僻遠,王撫治有威惠。至正二十三年,明玉珍僭號於蜀,遣兵三道來攻,王走營金馬山。明年以大理兵迎戰,玉珍兵敗退。久之,順帝北去,大都不守,中國無元尺寸地,而王守雲南自若;歲遣使自塞外達元帝行在,執臣節如故。

未幾,明師平四川,天下大定。太祖以雲南險僻,不欲用兵。明年正月,北平守將以所得王遣往漠北使者蘇成來獻,太祖乃命待制王禕齎詔偕成往招諭。王待禕以禮。會元嗣君遣使脫脫來征餉,脫脫疑王有他意,因脅以危語。王遂殺禕而以禮斂之。踰三年,太祖復遣湖廣參政吳雲偕大軍所獲雲南使臣鐵知院等往。知院以己奉使被執,誘雲改制書紿王。雲不從,被殺。王聞雲死,收其骨,送蜀給孤寺。

太祖知王終不可以諭降,乃命傅友德為征南將軍,藍玉、沐英為副,帥師征之。洪武十四年十二月下普定。王遣司徒平章達里麻率兵駐曲靖。沐英引軍疾趨,乘霧抵白石江。霧解,達里麻望見大驚。友德等率兵進擊,達里麻兵潰被擒。先是,王以女妻大理酋段得功,嘗倚其兵力,後以疑殺之,遂失大理援。至是達里麻敗,失精甲十餘萬。王知事不可為,走普寧州之忽納砦,焚其龍衣,驅妻子赴滇池死。遂與左丞達的、右丞驢兒夜入草舍,俱自經。太祖遷其家屬於耽羅。贊曰:洪武九年,方谷珍死,宋濂奉敕撰墓碑,於一時群雄,皆直書其名,獨至察罕,曰齊國李忠襄王,順逆之理昭然可見矣。擴廓百戰不屈,欲繼先志,而齎恨以死。友定不作何真之偷生,梁王恥為納哈出之背國,要皆元之忠臣也。《詩》曰「其儀一兮,心如結兮」,《易》曰「苦節悔亡」,其伯顏子中、蔡子英之謂歟。嘗謂元歸塞外,一時從臣必有賦《式微》之章於沙漠之表者,惜其姓字湮沒,不得見於人間。然則若子英者,又豈非厚幸哉!

