\article{列傳第五十八}

\begin{pinyinscope}
○于謙子冕吳寧王偉

於謙,字廷益,錢塘人。生七歲,有僧奇之曰:「他日救時宰相也。」舉永樂十九年進士。

宣德初,授御史。奏對,音吐鴻暢,帝為傾聽。顧佐為都御史,待寮屬甚嚴,獨下謙,以為才勝己也。扈蹕樂安,高煦出降,帝命謙口數其罪。謙正詞嶄嶄,聲色震厲。高煦伏地戰慄,稱萬死。帝大悅。師還,賞賚與諸大臣等。

出按江西,雪冤囚數百。疏奏陜西諸處官校為民害,詔遣御史捕之。帝知謙可大任,會增設各部右侍郎為直省巡撫,乃手書謙名授吏部,超遷兵部右侍郎,巡撫河南、山西。謙至官,輕騎遍歷所部,延訪父老,察時事所宜興革,即俱疏言之。一歲凡數上,小有水旱,輒上聞。

正統六年疏言:「今河南、山西積穀各數百萬。請以每歲三月,令府州縣報缺食下戶,隨分支給。先菽秫,次黍麥,次稻。俟秋成償官,而免其老疾及貧不能償者。州縣吏秩滿當遷,預備糧有未足,不聽離任。仍令風憲官以時稽察。」詔行之。河南近河處,時有衝決。謙令厚築隄障,計里置亭,亭有長,責以督率修繕。並令種樹鑿井,榆柳夾路,道無渴者。大同孤懸塞外,按山西者不及至,奏別設御史治之。盡奪鎮將私墾田為官屯,以資邊用。威惠流行,太行伏盜皆避匿。在官九年,遷左侍郎,食二品俸。

初,三楊在政府,雅重謙。謙所奏,朝上夕報可,皆三楊主持。而謙每議事京師,空橐以入,諸權貴人不能無望。及是,三楊已前卒,太監王振方用事。適有御史姓名類謙者,嘗忤振。謙入朝,薦參政王來、孫原貞自代。通政使李錫阿振指,劾謙以久不遷怨望,擅舉人自代。下法司論死,繫獄三月。已而振知其誤,得釋,左遷大理寺少卿。山西、河南吏民伏闕上書,請留謙者以千數,周、晉諸王亦言之,乃復命謙巡撫。時山東、陜西流民就食河南者二十餘萬,謙請發河南、懷慶二府積粟以振。又奏令布政使年富安集其眾,授田給牛種,使里老司察之。前後在任十九年,丁內外艱,皆令歸治喪,旋起復。

十三年以兵部左侍郎召。明年秋,也先大入寇,王振挾帝親征。謙與尚書鄺埜極諫,不聽。埜從治兵,留謙理部事。及駕陷土木,京師大震,眾莫知所為。郕王監國,命群臣議戰守。侍講徐珵言星象有變,當南遷。謙厲聲曰:「言南遷者,可斬也。京師天下根本,一動則大事去矣,獨不見宋南渡事乎!」王是其言,守議乃定。時京師勁甲精騎皆陷沒,所餘疲卒不及十萬,人心震恐,上下無固志。謙請王檄取兩京、河南備操軍,山東及南京沿海備倭軍,江北及北京諸府運糧軍,亟赴京師。以次經畫部署,人心稍安。即遷本部尚書。

郕王方攝朝,廷臣請族誅王振。而振黨馬順者,輒叱言官。於是給事中王竑廷擊順,眾隨之。朝班大亂,衛卒聲洶洶。王懼欲起,謙排眾直前掖王止,且啟王宣諭曰:「順等罪當死,勿論。」眾乃定。謙袍袖為之盡裂。退出左掖門,吏部尚書王直執謙手歎曰「國家正賴公耳。今日雖百王直何能為!」當是時,上下皆倚重謙,謙亦毅然以社稷安危為己任。

初,大臣憂國無主,太子方幼,寇且至,請皇太后立郕王。王驚謝至再。謙揚言曰:「臣等誠憂國家,非為私計。」王乃受命。九月,景帝立,謙入對,慷慨泣奏曰:「寇得志,要留大駕,勢必輕中國,長驅而南。請飭諸邊守臣協力防遏。京營兵械且盡,宜亟分道募民兵,令工部繕器甲。遣都督孫鏜、衛穎、張軏、張儀、雷通分兵守九門要地,列營郭外。都御史楊善、給事中王竑參之。徙附郭居民入城。通州積糧,令官軍自詣關支,以贏米為之直,毋棄以資敵。文臣如軒輗者,宜用為巡撫。武臣如石亨、楊洪、柳溥者,宜用為將帥。至軍旅之事,臣身當之,不效則治臣罪。」帝深納之。

十月敕謙提督各營軍馬。而也先挾上皇破紫荊關直入,窺京師。石亨議斂兵堅壁老之。謙不可,曰:「奈何示弱,使敵益輕我。」亟分遣諸將,率師二十二萬,列陣九門外:都督陶瑾安定門,廣寧伯劉安東直門,武進伯朱瑛朝陽門,都督劉聚西直門,鎮遠侯顧興祖阜成門,都指揮李端正陽門,都督劉得新崇文門,都指揮湯節宣武門,而謙自與石亨率副總兵范廣、武興陳德勝門外,當也先。以部事付侍郎吳寧,悉閉諸城門,身自督戰。下令,臨陣將不顧軍先退者,斬其將。軍不顧將先退者,後隊斬前隊。於是將士知必死,皆用命。副總兵高禮、毛福壽卻敵彰義門北,擒其長一人。帝喜,令謙選精兵屯教場以便調用,復命太監興安、李永昌同謙理軍務。

初,也先深入,視京城可旦夕下。及見官軍嚴陣待,意稍沮。叛閹喜寧嗾使邀大臣迎駕,索金帛以萬萬計,復邀謙及王直、胡濙等出議。帝不許,也先氣益沮。庚申,寇窺德勝門。謙令亨設伏空舍,遣數騎誘敵。敵以萬騎來薄,副總兵范廣發火器,伏起齊擊之。也先弟孛羅、平章卯那孩中炮死。寇轉至西直門,都督孫堂禦之,亨亦分兵至,寇引退。副總兵武興擊寇彰義門,與都督王敬挫其前鋒。寇且卻,而內官數百騎欲爭功,躍馬競前。陣亂,興被流矢死,寇逐至土城。居民升屋,號呼投磚石擊寇,嘩聲動天。王竑及福壽援至,寇乃卻。相持五日,也先邀請既不應,戰又不利,知終弗可得志,又聞勤王師且至,恐斷其歸路,遂擁上皇由良鄉西去。謙調諸將追擊,至關而還。論功,加謙少保,總督軍務。謙曰:「四郊多壘,卿大夫之恥也,敢邀功賞哉!」固辭,不允。乃益兵守真、保、涿、易諸府州,請以大臣鎮山西,防寇南侵。

景泰元年三月,總兵朱謙奏敵二萬攻圍萬全,敕范廣充總兵官御之。已而寇退,謙請即駐兵居庸,寇來則出關剿殺,退則就糧京師。大同參將許貴奏,迤北有三人至鎮,欲朝廷遣使講和。謙曰:「前遣指揮季鐸、岳謙往,而也先隨入寇。繼遣通政王復、少卿趙榮,不見上皇而還。和不足恃,明矣。況我與彼不共戴天,理固不可和。萬一和而彼肆無厭之求,從之則坐敝,不從則生變,勢亦不得和。貴為介胄臣,而恇怯如此,何以敵愾,法當誅。」移檄切責。自是邊將人人主戰守,無敢言講和者。

初,也先多所要挾,皆以喜寧為謀主。謙密令大同鎮將擒寧,戮之。又計授王偉誘誅間者小田兒。且因諜用間,請特釋忠勇伯把台家,許以封爵,使陰圖之。也先始有歸上皇意,遣使通款,京師稍解嚴。謙上言:「南京重地,撫輯須人。中原多流民,設遇歲荒,嘯聚可虞。乞敕內外守備及各巡撫加意整飭。防患未然,召還所遣召募文武官及鎮守中官在內地者。」

於時八月,上皇北狩且一年矣。也先見中國無釁,滋欲乞和,使者頻至,請歸上皇。大臣王直等議遣使奉迎,帝不悅曰:「朕本不欲登大位,當時見推,實出卿等。」謙從容曰:「天位已定,寧復有他,顧理當速奉迎耳。萬一彼果懷詐,我有辭矣。」帝顧而改容曰:「從汝,從汝。」先後遣李實、楊善往。卒奉上皇以歸,謙力也。

上皇既歸,瓦剌復請朝貢。先是,貢使不過百人,正統十三年至三千餘,賞賚不饜,遂入寇。及是又遣使三千來朝,謙請列兵居庸關備不虞。京師盛陳兵,宴之。因言和議難恃,條上安邊三策。請敕大同、宣府、永平、山海、遼東各路總兵官增修備禦。京兵分隸五軍、神機、三千諸營,雖各有總兵,不相統一,請擇精銳十五萬,分十營團操。團營之制自此始。具《兵志》中。瓦剌入貢,每攜故所掠人口至。謙必奏酬其使,前後贖還累數百人。

初,永樂中,降人安置近畿者甚眾。也先入寇,多為內應。謙謀散遣之。因西南用兵,每有征行,輒選其精騎,厚資以往,已更遣其妻子,內患以息。楊洪自獨石入衛,八城悉以委寇。謙使都督孫安以輕騎出龍門關據之,募民屯田,且戰且守,八城遂復。貴州苗未平,何文淵議罷二司,專設都司,以大將鎮之。謙曰:「不設二司,是棄之也。」議乃寢。謙以上皇雖還,國恥未雪,會也先與脫脫不花構,請乘間大發兵,身往討之,以復前仇,除邊患。帝不許。

謙之為兵部也,也先勢方張;而福建鄧茂七、浙江葉宗留、廣東黃蕭養各擁眾僭號;湖廣、貴州、廣西、瑤、僮、苗、僚所至蜂起。前後徵調,皆謙獨運。當軍馬倥傯,變在俄頃,謙目視指屈,口具章奏,悉合機宜。僚吏受成,相顧駭服。號令明審,雖勳臣宿將小不中律,即請旨切責。片紙行萬里外,靡不惕息。其才略開敏,精神周至,一時無與比。至性過人,憂國忘身。上皇雖歸,口不言功。東宮既易,命兼宮僚者支二俸。諸臣皆辭,謙獨辭至再。自奉儉約,所居僅蔽風雨。帝賜第西華門,辭曰:「國家多難,臣子何敢自安。」固辭,不允。乃取前後所賜璽書、袍、錠之屬,悉加封識,歲時一省視而已。

帝知謙深,所論奏無不從者。嘗遣使往真定、河間採野菜,直沽造乾魚,謙一言即止。用一人,必密訪謙。謙具實對,無所隱,不避嫌怨。由是諸不任職者皆怨,而用弗如謙者,亦往往嫉之。比寇初退,都御史羅通即劾謙上功簿不實。御史顧躭言謙太專,請六部大事同內閣奏行。謙據祖制折之,戶部尚書金濂亦疏爭,而言者捃摭不已。諸御史以深文彈劾者屢矣,賴景帝破眾議用之,得以盡所設施。

謙性故剛,遇事有不如意,輒拊膺歎曰:「此一腔熱血,意灑何地!」視諸選耎大臣、勳舊貴戚意頗輕之,憤者益眾。又始終不主和議,雖上皇實以是得還,不快也。徐珵以議南遷,為謙所斥。至是改名有貞,稍稍進用,嘗切齒謙。石亨本以失律削職,謙請宥而用之,總兵十營,畏謙不得逞,亦不樂謙。德勝之捷,亨功不加謙而得世侯,內愧,乃疏薦謙子冕。詔赴京師,辭,不允。謙言:「國家多事,臣子義不得顧私恩。且亨位大將,不聞舉一幽隱,拔一行伍微賤,以裨軍國,而獨薦臣子,於公議得乎?臣於軍功,力杜僥倖,決不敢以子濫功。」亨復大恚。都督張軏以征苗失律,為謙所劾,與內侍曹吉祥等皆素憾謙。

景泰八年正月壬午,亨與吉祥、有貞等既迎上皇復位,宣諭朝臣畢,即執謙與大學士王文下獄。誣謙等與黃竑構邪議,更立東宮;又與太監王誠、舒良、張永、王勤等謀迎立襄王子。亨等主其議,嗾言官上之。都御史蕭惟禎定讞。坐以謀逆,處極刑。文不勝誣,辯之疾,謙笑曰:「亨等意耳,辯何益?」奏上,英宗尚猶豫曰:「於謙實有功。」有貞進曰:「不殺于謙,此舉為無名。」帝意遂決。丙戌改元天順,丁亥棄謙市,籍其家,家戍邊。遂溪教諭吾豫言謙罪當族,謙所薦舉諸文武大臣並應誅。部議持之而止。千戶白琦又請榜其罪,鏤板示天下,一時希旨取寵者,率以謙為口實。

謙自值也先之變,誓不與賊俱生。嘗留宿直廬,不還私第。素病痰,疾作,景帝遣興安、舒良更番往視。聞其服用過薄,詔令上方製賜,至醯菜畢備。又親幸萬歲山,伐竹取瀝以賜。或言寵謙太過,興安等曰:「彼日夜分國憂,不問家產,即彼去,令朝廷何處更得此人?」及籍沒,家無餘資,獨正室鐍鑰甚固。啟視,則上賜蟒衣、劍器也。死之日,陰霾四合,天下冤之。指揮朵兒者,本出曹吉祥部下,以酒酹謙死所,慟哭。吉祥怒,抶之。明日復酹奠如故。都督同知陳逵感謙忠義,收遺骸殯之。踰年,歸葬杭州。逵,六合人。故舉將才,出李時勉門下者也。皇太后初不知謙死,比聞,嗟悼累日。英宗亦悔之。

謙既死,而亨黨陳汝言代為兵部尚書。未一年敗,贓累巨萬。帝召大臣入視,愀然曰:「于謙被遇景泰朝,死無餘資。汝言抑何多也!」亨俯首不能對。俄有邊警,帝憂形於色。恭順侯吳瑾侍,進曰:「使于謙在,當不令寇至此。」帝為默然。是年,有貞為亨所中,戍金齒。又數年,亨亦下獄死,吉祥謀反族誅,謙事白。

成化初,冕赦歸,上疏訟冤,得復官賜祭。誥曰:「當國家之多難,保社稷以無虞,惟公道之獨恃,為權奸所並嫉。在先帝已知其枉,而朕心實憐其忠。」天下傳誦焉。弘治二年,用給事中孫需言,贈特進光祿大夫、柱國、太傅,謚肅愍。賜祠於其墓曰「旌功」,有司歲時致祭。萬曆中,改謚忠肅。杭州、河南、山西皆世奉祀不絕。

冕,字景瞻,廕授副千戶,坐戍龍門。謙冤既雪,并復冕官。自陳不願武職,改兵部員外郎。居官有幹局,累遷至應天府尹。致仕卒。無子,以族子允忠為後,世襲杭州衛副千戶,奉祠。

吳寧,字永清,歙人。宣德五年進士,除兵部主事。正統中,再遷職方郎中。郕王監國,謙薦擢本部右侍郎。謙禦寇城外,寧掌部事,命赴軍中議方略。比還,城門弗啟,寇騎充斥,寧立雨中指揮兵士,移時乃入。寇既退,畿民猶日數驚,相率南徙。或議仍召勤王兵。寧曰:「是益之使驚也,莫若告捷四方,人心自定。」因具奏行之。景泰改元,以疾乞歸,後不復出。家居三十餘年卒。

寧方介有識鑒。嘗為謙擇婿,得千戶朱驥。謙疑之,寧曰:「公他日當得其力。」謙被刑,驥果歸其喪,葬之。驥自有傳。

王偉,字士英,攸人。年十四,隨父謫戍宣府。宣宗巡邊,獻《安邊頌》,命補保安州學生。舉正統元年進士,改庶吉士,授戶部主事。英宗北狩,命行監察御史事,集民壯守廣平。謙引為職方司郎中。軍書填委,處分多中窾會,遂薦擢兵部右侍郎。出視邊,叛人小田兒為敵間,謙屬偉圖之。會田兒隨貢使入,至陽和城,壯士從道旁突出,斷其頭去,使者不敢詰。

偉喜任智數。既為謙所引,恐嫉謙者目己為朋附,嘗密奏謙誤,冀自解。帝以其奏授謙,謙叩頭謝。帝曰:「吾自知卿,何謝為?」謙出,偉問:「上與公何言?」謙笑曰:「我有失,望君面規我,何至爾邪?」出奏示之,偉大慚沮。然竟坐謙黨,罷歸。成化三年復官,請毀白琦所鏤板。踰年,告病歸卒。

贊曰:于謙為巡撫時,聲績表著,卓然負經世之才。及時遘艱虞,繕兵固圉。景帝既推心置腹,謙亦憂國忘家,身繫安危,志存宗社,厥功偉矣。變起奪門,禍機猝發,徐、石之徒出力而擠之死,當時莫不稱冤。然有貞與亨、吉祥相繼得禍,皆不旋踵。而謙忠心義烈,與日月爭光,卒得復官賜恤。公論久而後定,信夫。


\end{pinyinscope}