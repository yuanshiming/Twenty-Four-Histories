\article{列傳第九十九}

\begin{pinyinscope}
馬永梁震祝雄王效劉文周尚文趙國忠馬芳子林孫炯爌飆)何卿沈希儀石邦憲

馬永,字天錫,遷安人。生而魁岸,驍果有謀。習兵法,好《左氏春秋》。嗣世職為金吾左衛指揮使。正德時,從陸完擊賊有功,進都指揮同知。江彬練兵西內,永當隸彬,稱疾避之。守備遵化,寇入馬蘭峪,參將陳乾被劾,擢永代。戰柏崖、白羊峪,皆有功。

十三年進都督僉事,充總兵官,鎮守蘇州。盡汰諸營老弱,聽其農賈,取備直給健卒,由是永所將獨雄於諸鎮。武宗至喜峰口,欲出塞,永叩馬諫。帝注視久之,笑而止。中路擦崖當敵衝,無城堡,耕牧者輒被掠。永令人持一月糧,營崖表,版築其內。城廨如期立,乃遷軍守之。錄功,進署都督同知。

嘉靖元年,金山礦盜作亂。遣指揮康雄討平之,塞其礦。朵顏把兒孫結諸部邀賞不得,盜邊。永迎擊洪山口,而伏兵斷其後,斬獲過當,進右都督。已,復馘其驍將,把兒孫不敢復擾邊。大同兵變,殺巡撫張文錦,命桂勇為總兵官往鎮,而議將撫之。永言:「逆賊干紀,朝廷赦其脅從,恩至渥也,顧猶抗命。今不剿,春和北寇南牧,叛卒勾連,禍滋大。宜亟調鄰鎮兵,剋期攻城,曉譬利害,懸破格之賞,令賊自相斬為功,元凶不難殄也。」乃命永督諸軍與侍郎胡瓚往。會亂平,乃還鎮。

永上書為陸完請恤典,且乞宥議禮獲罪諸臣。帝大怒,奪永官,寄祿南京後府。巡按御史丘養浩言:「永仁以恤軍,廉以律已,固邊防,卻強敵,軍民安堵,資彼長城。聞永去,遮道乞留,且攜子女欲遂逃移。夫陸完久死炎瘴,非有權勢可託。永徒感國士知,欲救區區之報。不負知己,寧負國家?祈曲賜優容,俾還鎮。」順天巡撫劉澤及給事、御史交章救之,俱被譴。永竟廢不用。永杜門讀書,清約如寒士。久之,用薦僉書南京前府。大同軍再亂,廷臣交薦。召至,已就撫,復還南京。

十四年,遼東兵變。罷總兵官劉淮,以永代之。大清堡守將徐顥誘殺泰寧衛九人。部長把當孩怒,寇邊,永擊斬之。其族屬把孫借朵顏兵報讎,復為永所卻。已,復入犯。中官王永戰敗,永坐戴罪。

遼東自軍變後,首惡雖誅,漏綱者眾。悍卒無所憚,結黨叫呼,動懷不逞。廣寧卒佟伏、張鑒等乘旱饑,倡眾為亂,諸營軍憚永無應者。伏等登譙樓,鳴鼓大噪,永率家眾仰攻。千戶張斌被殺,永戰益力,盡殲之。事聞,進左都督。

永畜士百餘人,皆西北健兒,驍勇敢戰。遼東變初定,帝問將於李時。時薦永,且曰:「其家眾足用也。」帝曰:「將須文武兼,寧專恃勇乎?」時曰:「遼土新定,須有威力者鎮之。」至是,竟得其力。都御史王廷相言:「永善用兵,且廉潔,宜仍用之蘇鎮,作京師籓屏。」未及調,卒。遼人為罷市。喪過蘇州,州人亦灑泣。兩鎮並立祠。

永為將,厚撫間諜,得敵人情偽,故戰輒勝。雅知人,所拔卒校,後多至大帥。尚書鄭曉稱永與梁震有古良將風。

梁震,新野人。襲榆林衛指揮使。嘉靖七年進署都指揮僉事,協守寧夏興武營。尋充延綏遊擊將軍。廉勇,好讀兵書,善訓士,力挽彊命中,數先登。擢延綏副總兵。與總兵官王效卻敵鎮遠關,進都督僉事。

吉囊、俺答犯延綏,震敗之黃甫川。尋犯響水、波羅,參將任傑大敗之。吉囊復以十萬騎入寇,震大破之乾溝,獲首功百餘。先後被獎賚。已,增俸一等。乾溝凡三十里,當敵衝。震濬使深廣,築土牆其上,寇不復輕犯。

十四年進都督同知,充陜西總兵官。尋論黃甫川功,進右都督。明年移鎮大同。大同亂兵連殺巡撫張文錦、總兵官李瑾。繼瑾者魯綱,威不振,兵益驕,文武大吏不敢要束。廷議以為憂,移震往。震素畜健兒五百人,至則下令軍中,申約束。鎮兵素憚震,由是帖服。寇入犯,震破之牛心山,斬級百餘。寇慎,駐近邊伺隙。時車駕祀山陵,震伏將士於諸路。寇果入,大破之宣寧灣,又破之紅崖兒,斬獲甚眾。進左都督,蔭一子百戶。震父棟,前陣亡。震辭廕子,乞父祭葬,帝喜而許之。毛伯溫督師,與震修鎮邊諸堡,不數月工成。卒,贈太子太保,賜其家銀幣,加贈太保,謚武壯。

震有機略,號令明審。前後百十戰,未嘗少挫。時率健兒出塞劫敵營,或議其啟釁。震曰:「凡啟釁者,謂寇不擾邊,我橫挑邀功也。今數深入,乃不思一挫之耶?」震歿,健兒無所歸。守臣以聞,編之伍,邊將猶頗得其力。

代震者遼東祝雄,起家世廕。歷都督僉事。自山西副總兵遷鎮大同。被劾解職,起鎮薊州。善撫士,治軍肅。寇入塞,率子弟為士卒先。子少卻,行法不貸。世宗書其名御屏。為將三十年,布袍氈笠,不異卒伍。既歿,遺貲僅供殮具。薊人祠祀之。

王效,延綏人。讀書能文辭,嫻韜略。騎射絕人,中武會試。嘉靖中,累官都指揮僉事,充延綏右參將。出神木塞,搗寇雙乃山,斬獲多。尋擢延綏副總兵。十一年冬,進署都督僉事,充總兵官,代周尚文鎮寧夏。吉囊犯鎮遠關,效與梁震敗之柳門。追北蜂窩山,蹙溺之河,斬首百四十有奇。璽書獎賚。

吉囊十萬騎復窺花馬池,效、震拒之不得入,轉犯干溝。震分兵擊,遂趨固原。總兵官劉文力戰,寇趨青山峴,大掠安定、會寧。效方敗別部於鼠湖,追至沙湖,疾移師往援,破之安定,再破之靈州,先後斬首百五十餘級。總制三邊尚書唐龍以大捷聞,而巡按御史奏諸將失事罪。給事中戚賢往勘,奏:「安、會二縣多殺掠,文當罪。然麾下卒僅八千,倍道蒙險,攖八九萬方張之寇,殊死戰,宜以功贖。震乾溝,效鼠湖、沙湖、安定、靈州之戰,以孤軍八百,當寇萬餘,功俱足錄。龍亦善調度。」詔文奪職,震、效賚銀幣,龍一子入監。是役也功多,執政尼之,故賞薄。御史周鈇以為言,龍、效、震各加一級,效進都督同知。尋以清水營功,進右都督。寇以輕騎犯寧夏,效伏兵打鎧口,俟其半入橫擊,敗之,而防河卒復以戰艘邀斬其奔渡者。捷聞,進左都督。寇憤,設伏誘敗之,貶右都督。十六年移鎮宣府。踰年卒,謚武襄。

效言行謹飭,用兵兼謀勇,威名著西陲。與馬永、梁震、周尚文並為名將。

劉文者,陽和衛人。襲指揮同知。屢遷署都督僉事,涼州右副總兵。嘉靖八年以總兵官鎮陜西。大破洮、岷叛番若籠、板爾諸族,斬首三百六十有奇。十一年,寇西掠還,將犯寧夏河東,文擊破之。積前功,進都督同知。後落職,起鎮延綏,改甘肅。卒,亦謚武襄。

周尚文,字彥章,西安後衛人。幼讀書,粗曉大義。多謀略,精騎射。年十六,襲指揮同知。屢出塞有功,進指揮使。寘鑠反,遏黃河渡口,獲叛賊丁廣等,推掌衛事。關內回賊四起,倚南山,尚文次第平之。御史劉天和劾中貴廖堂繫詔獄,事連尚文。拷掠令引天和,終不承,久之始釋。已,守備階州。計擒叛番,進署都指揮僉事,充甘肅遊擊將軍。嘉靖元年改寧夏參將。尋進都指揮同知,為涼州副總兵。御史按部莊浪,猝遇寇。尚文亟分軍擁御史,而自引麾下射之,寇乃遁。嘗追寇出塞,寇來益眾。尚文軍半至,麾下皆恐。乃從容下馬,解鞍背崖力戰,所殺傷相當。部將丁杲來援,寇始退。尚文被創甚,乃告歸。尋起故官。吉囊數踏冰入。尚文築牆百二十里,澆以水,冰滑不可上。冰泮則令力士持長竿鐵鉤,鉤殺渡者。九年,擢署都督僉事,充寧夏總兵官。王瓊築邊牆,尚文督其役。且浚渠開屯,軍民利之。寇掠西海,過寧夏,巡撫楊志學議發兵邀。尚文不從,劾解職。久之,起山西副總兵。寇由偏頭關趨岢嵐,尚文轉戰三百里,破之,與子君佐俱傷,賚銀幣。尋以總兵官鎮延綏。寇犯紅山墩,力戰敗之,被賚。吉囊復大掠清平堡,坐奪俸。

尚文優將才,負氣桀傲,所至與文吏競。文吏又往往挫折之,以故彌不相得。巡撫賈啟劾尚文老誖,兵部請調之甘肅。帝不從,各奪其俸。巡按張光祖言兩人必不可共處,乃革尚文任,亦貶啟秩。吉囊大入,抵固原。天和時已為總督,激尚文立功。奮擊之黑水苑,殺其子號小十王者,獲首功百三十餘。乃以為都督同知。

二十一年,用薦為東官廳聽徵總兵官兼僉後府事。嚴嵩為禮部尚書,子世蕃官後府都事,驕蹇。尚文面叱,將劾奏之,嵩謝得免。調世蕃治中,以避尚文,銜次骨。其秋以總兵官鎮大同,請增餉及馬。兵部言尚文陳請過當,方被詔切責,而尚文與巡撫趙錦不協,乞休,弗允,日相構。御史王三聘乞移之他鎮。廷議:大同敵衝,尚文假此避,不宜墮其奸謀。乃以錦為甘肅巡撫。吉囊數萬騎犯前衛。尚文與戰黑山,殺其子滿罕歹,追至涼城。斬獲多,進右都督。已,寇由宣府逼畿甸,出大同塞而北。尚文邀之,稍有俘獲。後寇復大舉,犯鵓鴿谷,將南下。尚文備陽和,遣騎四出邀寇。寇遁,賜敕獎勞之。

總督翁萬達議築邊牆,自宣府西陽和至大同開山口,延袤二百餘里,以屬尚文。乃益築陽和以西至山西丫角山,凡四百餘里,敵臺千餘。斥屯田四萬餘頃,益軍萬三千有奇。帝嘉其功,進左都督,加太子太保,永除屯稅。叛人充灼召小王子寇邊,尚文偵得其使者,加太保,蔭子錦衣世千戶。終明之世,總兵官加三公者,尚文一人而已。

初,俺答及吉囊諸子盛強,諸邊歲受其患,大同尤甚。自尚文蒞鎮,與總督萬達、巡撫詹榮規畫戰守備邊,民息肩者數年。尚文益招叛人,孤敵勢,歸者相屬。二十七年八月,俺答伏兵五堡旁,誘指揮顧相等出,圍之彌陀山。尚文急督副總兵林椿、參將呂勇、遊擊李梅及二子君佐、君仁出塞援,圍始解。相及指揮周奉,千戶呂愷、郝經等已陣歿。尚文轉戰,次野口,伏突起。殊死戰,斬其長一人。相持月餘乃引去。尚文設伏,殺其殿卒而還。尚文三子俱罪戍,至是以父功得釋。俺答數萬騎犯宣府,萬達檄尚文大破之曹家床。錄功兼太子太傅,賜賚有加。其年卒,年七十五。

尚文清約愛士,得士死力。善用間,知敵中曲折,故戰輒有功。自二十年後,俺答頻擾邊。宿將王效、馬永、梁震皆前死,惟尚文存,威名最盛。嚴嵩父子謀傾陷。功高,帝方籍以抗強敵,讒不得入。暨卒,格恤典不予,給事中沈束以為言。嵩激帝怒,錮束詔獄。穆宗立,始贈太傅,謚武襄。

趙國忠,字伯進,錦州衛人,嗣指揮職。嘉靖八年舉武會試,進都指揮僉事,守備靉陽。擢錦義右參將。連破敵,增秩,賜金幣,進署都督僉事,為遼東總兵官。禦敵有功,斬級百七十有奇。進都督同知,賜賚踰等。敵以八百騎從鴉鶻關入。都指揮康雲戰歿,裨將三人亦死,詔國忠戴罪立功。已,坐事被劾,命白衣視事。守備張文瀚禦敵死,國忠坐解任。尋起西官廳右參將,授都督僉事,提督東官廳。俺答大舉犯宣府,總兵官趙卿不任戰,命國忠代之。至岔道,寇已為周尚文所敗,東走。國忠命參將孫勇率精卒逆擊於大滹沱,敗之。與尚文分道擊,寇盡走,以功受賚。復坐寇入,降俸二等。俺答薄京師,國忠趨入衛,壁沙河北。已,移護諸陵。寇騎至天壽山,見國忠陣紅門前,不敢入。三十一年,再鎮遼東。小王子打來孫以數萬騎寇錦州,國忠禦卻之。明年入獅子口,督參將李廣等逐出塞,斬擒五十人。寇屢入榆林堡、高臺、蛤利河。先後掩擊,獲首功百五十有奇,進秩一等。尋被論罷。

國忠善戰,射穿札,為將有威嚴。歷兩鎮,繕亭障,練士馬,邊防賴之。

馬芳,字德馨,蔚州人。十歲為北寇所掠,使之牧。芳私以曲木為弓,剡矢射,俺答獵,虎虓其前,芳一發斃之。乃授以良弓矢、善馬,侍左右。芳陽為之用,而潛自間道亡歸。周尚文鎮大同,奇之,署為隊長。數御寇有功,當得官,以父貧,悉受賞以養。

嘉靖二十九年秋,寇犯懷柔、順義。芳馳斬其將,授陽和衛總旗。寇嘗入威遠,伏驍騎鹽場,而以二十騎挑戰。芳知其詐,用百騎薄伏所,三分其軍銳,以次擊之。奮勇跳盪,敵騎辟易十里,斬首凡九十級。已,復禦之新平。寇營野馬川,剋日戰。芳度寇且遁,急乘之,斬級益多。眾方賀,芳遽策馬曰:「賊至矣。」趣守險,而身斷後。頃之,寇果麕至。芳戰益力,寇乃去。亡何,戰泥河,復大破之。累遷指揮僉事。以功,進都指揮僉事,充宣府遊擊將軍。復以功,超遷都督僉事,隸總督為參將。戰鎮山墩不利,奪俸。已,襲寇有功,進二秩,為右都督。尋以功進左,賜蟒袍。偏裨加左都督,自芳始也。

三十六年,遷薊鎮副總兵,分守建昌。土蠻十萬騎薄界嶺口,芳與總兵官歐陽安斬首數十,獲驍騎猛克兔等六人。寇不知芳在,芳免胄示之,驚曰:「馬太師也!」遂卻。捷聞,廕世總旗。未幾,辛愛、把都兒大入,躪遵化、玉田。芳追戰金山寺有功,而州縣破殘多,總督王忬以下俱獲罪,芳亦貶都督僉事。尋移守宣府。寇大入山西,芳一日夜馳五百里及之,七戰皆捷。已,復為左都督,就擢總兵官,以功進二秩。寇薄通州,芳入衛,令專護京師。寇退,再進一秩。尋與故總兵劉漢出北沙灘,搗寇巢。已,坐寇入,令戴罪。

四十五年七月,辛愛以十萬騎入西路,芳迎之馬蓮堡。堡圮,眾請塞之,不可。請登臺,亦不可。開堡四門,偃旗鼓,寂若無人。比暮,野燒燭天,囂呼達旦。芳臥,日中不起,敵騎窺者相屬,莫測所為。明日,芳蹶然起,乘城,指示眾曰:「彼軍多反顧,且走。」勒兵追擊,大破之。隆慶初,或為辛愛謀,以五萬騎犯蔚州,誘芳出,而以五萬騎襲宣府城,可得志。芳豫伐木環城,寇至不可上,遂解去。頃之,率參將劉潭等出獨石塞外二百里,襲其帳於長水海。還至塞,追者及鞍子山。迎戰,又大敗之。子千戶。

芳有膽智,諳敵情,所至先士卒。一歲數出師搗巢,或躬督戰,或遣裨將。家蓄健兒,得其死力。嘗命三十人出塞四百里,多所斬獲,寇大震。芳乃帥師至大松林,頓舊興和衛,登高四望,耀兵而還。

時大同被寇,視宣府尤甚。總督陳其學恐擾畿輔,令總兵官趙岢扼紫荊關。寇乃縱掠懷仁、山陰間,岢坐貶三秩,遂調芳與易鎮。俺答轉犯威遠幾破,會其學率胡鎮等救,而芳軍亦至,相拒十餘日,乃走。芳謂諸將曰:「大同非宣府比,與我間一牆耳。寇不時至,非大創之不可。」乃將兵出右衛,戰威寧海子,破之。其年,俺答就撫,塞上遂無事。

萬歷元年,閱視侍郎吳百朋發芳行賄事,勒閒住。已,起僉書前軍都督府。順義王要賞,聲言渝盟,復用芳鎮宣府。七年以疾乞歸。又二年卒。

芳起行伍,十餘年為大帥。戰膳房堡、朔州、登鷹巢、鴿子堂、龍門、萬全右衛、東嶺、孤山、土木、乾莊、岔道、張家堡、得勝堡、大沙灘,大小百十接,身被數十創,以少擊眾,未嘗不大捷。擒部長數十人,斬馘無算,威名震邊陲,為一時將帥冠。石州城陷,副將田世威、參將劉寶論死,芳乞寢己廕子,贖二將罪,為御史所劾,敕戒諭。後世威復為將,遇芳薄,芳不與校,識者多之。

二子,棟、林。棟官至都督,無所見。林,由父廕累官大同參將。萬曆二十年,順義王撦力克縶獻史、車二部長,林以制敵功,進副總兵。二十七年擢署都督僉事,為遼東總兵官。林雅好文學,能詩,工書,交遊多名士,時譽籍甚,自許亦甚高。嘗陳邊務十策,語多觸文吏,寢不行。稅使高淮橫恣,林力與抗。淮劾奏之,坐奪職。給事中侯先春論救,改林戍煙瘴,先春亦左遷二官。久之,遇赦免。

遼左用兵,詔林以故官從征。楊鎬之四路出師也,令林將一軍由開原出三岔口,而以遊擊竇永澄監北關軍並進。林軍至尚間崖結營浚壕,嚴斥堠自衛。及聞杜松軍敗,方移營,而大清兵已逼。乃還兵,別立營,浚壕三周,列火器壕外,更布騎兵於火器外,他士卒皆下馬,結方陣壕內。又一軍西營飛芬山。杜松軍既覆,大清兵乘銳薄林軍。見林壕內軍已與壕外合而陳,縱精騎直前衝之。林軍不能支,遂大敗。副將麻巖戰死,林僅以數騎免。死者彌山谷,血流尚間崖下,水為之赤。大清遂移兵擊飛芬山。僉事潘宗顏等一軍亦覆。北關兵聞之,遂不敢進。林既喪師,謫充為事官,俾守開原。時蒙古宰賽、爰兔許助林兵,林與結約,恃此不設備。其年六月,大清兵忽臨城。林列眾城外,分少兵登陴。大清兵設盾梯進攻,而別以精騎擊破林軍之營東門外者。軍士爭門入,遂乘勢奪門,攻城兵亦踰城入。林城外軍望見盡奔。大清兵據城邀擊,壕不得渡,悉殲之。林及副將于化龍、參將高貞、遊擊於守志、守備何懋官等,皆死焉。尋贈都督同知,進世廕二秩。林雖更歷邊鎮,然未經強敵,無大將才。當事以虛名用之,故敗。

林五子,燃、熠、炯、爌,飆。燃、熠,戰死尚間崖。炯,天啟中湖廣總兵官。協討貴州叛賊,從王三善至大方,數戰皆捷。已,大敗,三善自殺。炯潰歸。得疾而卒。

爌幼習兵略,天啟中為遼東遊擊。督師閣部孫承宗以其父死王事,獎用之,命代王楹守中右所。及巡撫袁崇煥更營制,以故官掌前鋒左營。數有功,屢遷至副總兵,守徐州。崇禎八年正月,賊陷鳳陽,大掠而去。爌及守備駱舉率兵入,以恢復告,遂留戍其地。八月,賊擾河南。總督朱大典命移駐穎、毫。事定,還徐州。十年,賊犯桐城,爌赴救,破之羅唱河。尋以護陵功,增秩一級。歸德、徐州間有地曰朱家廠,土寇據之,時出掠。爌剿滅之。賊犯固始,大典檄爌及遊擊張士儀等分戍霍兵西南,扼賊東下,賊遂走六安。大典又移爌等駐壽州東,兼護二陵。當是時,長、淮南北,專以陵寢為重。爌馳驅數年,幸無失事。

十二年六月擢總兵官,鎮守天津。久之,移鎮甘肅。十五年督三協副將王世寵、王加春、魯胤昌等討破叛番,斬首七百餘級,撫安三十八族而還。其冬,督師孫傳庭檄召不至,疏劾之。帝令察爌堪辦賊,許戴罪圖功,否即以賜劍從事。及爌至軍,傳庭貸其罪。已,復以逗留淫掠被劾,帝仍令載罪自效。明年秋,傳庭將出關。有傳賊自內鄉窺商、雒者,檄爌移商州扼其北犯。已而傳庭師覆,爌遂還鎮。未幾,賊陷延綏、寧夏,遂陷蘭州,渡河抵甘州還攻之。爌與巡撫林日瑞竭力固守。賊乘雪夜坎而登。士卒寒甚,不能戰,城遂陷。爌、日瑞及中軍哈維新、姚世儒皆死焉。弟飆為沔陽州同知,城陷,亦死之。爌父子兄弟並死國難。

何卿,成都衛人。有志操,習武事。正德中,嗣世職為指揮僉事。以能,擢筠連守備。從巡撫盛應期擊斬叛賊謝文禮、文義。世宗立,論功,進署都指揮僉事,充左參將,協守松潘。

嘉靖初,芒部土舍隴政、土婦支祿等叛。卿討之,斬首二百餘級,降其眾數百人。政奔烏撒,卿檄土官安寧擒以獻。寧佯諾,而匿政不出。巡撫湯沐言狀,帝奪卿冠帶。川、貴兵合討,賊始滅,還冠帶如初。五年春擢卿副總兵,仍鎮松潘。隴氏已絕,改芒部為鎮雄府,設流官。未幾,政遺黨沙保復叛。卿偕參將魏武、參議姚汝皋等並進,斬保等賊首七人,餘盡殄。錄功,武最,卿次之,賜賚有差。黑虎五砦番反,圍長安諸堡,烏都、鵓鴿諸番亦繼叛。卿皆破平之,就進都督僉事。威茂番十餘砦連兵劫軍饟,且攻茂州及長寧諸堡,要撫賞。卿與副使朱紈築茂州外城以困之。旋以計殘其眾,戰屢捷,遂攻深溝,焚其碉砦。諸番窘,請贖罪。卿責獻首惡,番不應。復分剿淺溝、渾水二砦殲之。諸番乃爭獻首惡,插血斷指耳,誓不復叛。卿乃與刻木為約,分處其曹,畫疆守,松潘路復通。巡撫潘鑑等上二人功,詔賚銀幣,進署都督同知,鎮守如故。久之,以疾致仕。

二十三年,塞上多警。召卿,以疾辭。帝怒,奪其都督,命以都指揮使詣部聽調。未幾,寇逼畿輔,命營盧溝橋。松潘副總兵李爵為巡撫丘養浩劾罷,詔以卿代。給事中許天倫言卿賄養浩劾爵,自為地。帝怒,褫卿及養浩官,令巡按冉崇禮核實。時兵事棘,翁萬達復薦卿,還其都督僉事,都東官廳軍馬。已而崇禮具言爵貪污,「卿鎮松潘十七年,為蜀保障,軍民頌德,且貧,安所得賄?」帝意乃解。四川白草番為亂,副總兵高岡鳳被劾。兵部尚書路迎奏卿代之。卿再蒞松潘,將士咸喜。乃會巡撫張時徹討擒渠惡數人,俘斬九百七十有奇,克營砦四十七,毀碉房四千八百,獲馬牛器械儲積各萬計。進署都督同知。卿素有威望,為番人所憚。自威茂迄松潘、龍安夾道築牆數百里,行旌往來,無剽奪患。先後蒞鎮二十四年,軍民戴之若慈母。再以疾歸。

三十三年,倭寇海上。詔卿與沈希儀各率家眾赴蘇、松軍門。明年充副總兵,總理浙江及蘇、松海防。卿,蜀中名將,不諳海道,年已老,兵與將不習,竟不能有所為。為巡按御史周如斗劾罷,卒。

沈希儀,字唐佐,貴縣人。嗣世職為奉議衛指揮使。機警有膽勇,智計過絕於人。正德十二年,調徵永安。以數百人搗陳村砦,馬陷淖中,騰而上,連馘三酋,破其餘眾。進署都指揮僉事。義寧賊寇臨桂,還巢,希儀追之。巢有兩隘,賊伏兵其一,使熟瑤紿官兵入。希儀策其詐,急從別隘直抵賊巢。賊倉卒還救,遂大破之。荔浦賊八千渡江東掠,希儀率五百人駐白面砦,待其歸。砦去蛟龍、滑石兩灘各數里。希儀以滑石灘狹,雖眾可薄,蛟龍灘廣,濟則難圖,欲誘致之滑石。乃樹旗百蛟龍灘,守以羸卒,然柴以疑之。賊果趨滑石。希儀預以小艦載勁卒伏葭葦中。賊渡且半,乘瀧急沖之,兩岸軍噪而前,賊眾多墜水死,收所掠而還。從副總兵張祐連破臨桂、灌陽、古田賊。進署都指揮同知,掌都司事。

嘉靖五年,總督姚鏌將討田州岑猛。用希儀計,間猛婦翁歸順土酋岑璋,使圖猛,而分兵五哨進。希儀將中哨,當工堯。工堯,賊要地,聚眾守之。希儀夜遣軍三百人,緣山上,繞出其背。比明合戰,則所遣軍已立幟山巔,賊大潰敗。猛走歸順,為璋所執,田州平。希儀功最,鏌抑之,止受賚。鏌議設流官,希儀曰:「思恩以流官故,亂至今未已。田州復然,兩賊且合從起。」鏌不從。以希儀為右參將,分守思、田。希儀請還鄉治裝。以參將張經代守。甫一月,田州復叛,鏌罷歸。王守仁代,多用希儀計,思、田復定。

改右江柳慶參將,駐柳州。象州、武宣、融縣瑤反,討破之。謝病歸,頃之還故任。柳在萬山中,城外五里即賊巢,軍民至無地可田,而官軍素罷不任戰。又賊耳目遍官府,閨闥動靜無不知。希儀謂欲大破賊,非狼兵不可,請於制府。調那地狼兵二千來,戍兵稍振。乃求得與瑤通販易者數十人,持其罪而厚撫之,使詗賊。賊動靜,希儀亦無不知。希儀每出兵,雖肘腋親近不得聞。至期鳴號,則諸軍咸集。令一人挾旗引諸軍行,不測所往。及駐軍設伏,賊必至,遇伏輒奔。官軍擊之,無不如志。已,賊寇他所,官軍又先至。遠村僻聚,賊度官軍所不逮者,往寇之,官軍又未嘗不在,賊驚以為神。希儀得賊巢婦女畜產,果鄰巢者悉還之,惟取陰助賊者。諸瑤盡讋伏,無敢嚮賊。

希儀初至,令熟瑤得出入城中,無所禁。因厚賞其黠者,使為諜。後漸令瑤婦入見其妻,賚以酒食繒帛。其夫常以賊情告者,則陰厚之。諸瑤婦利賞,爭勸其夫輸賊情,或自入府言之。以故,賊益無所匿形。希儀每於風雨晦冥夜,偵賊所止宿,分遣人齎銃潛伏舍旁。中夜銃舉,賊大駭曰:「老沈來矣!」咸挈妻子匍匐上山。兒啼女號,或寒凍觸厓石死,爭怨悔作賊非計。至曉下山,則寂無人聲。他巢亦然,眾愈益驚。潛遣人入城偵之,則希儀故居城中不出也。賊膽落,多易面為熟瑤。

韋扶諫者,馬平瑤魁也,累捕不得。有報扶諫逃鄰賊三層巢者,希儀潛率兵剿之,則又與三層賊往劫他所。希儀盡俘三層巢妻子歸,希儀俘賊妻子盡以畀狼兵,至是獨閉之空舍,飲食之。使熟瑤往語其夫曰:「得韋扶諫,還矣。」諸瑤聞,悉來謁希儀。今入室視之,妻子固無恙。乃共誘扶諫出巢,縛以獻,易妻子還。希儀剜扶諫目,支解之,懸諸城門。諸瑤服希儀威信,益不敢為盜。自是,柳城四旁數百里,無敢攘奪者。

希儀嘗上書於朝,言狼兵亦瑤、僮耳。瑤、僮所在為賊,而狼兵死不敢為非,非狼兵順,而瑤、僮逆也。狼兵隸土官,瑤、僮隸流官。土官令嚴足以制狼兵,流官勢輕不能制瑤、僮。若割瑤、僮分隸之旁近土官,土官世世富貴,不敢有他望。以國家之力制土官,以土官之力制瑤、僮,皆為狼兵,兩廣世世無患矣。時不能用。至十六年而有思恩岑金之變。

初,思恩土官岑浚既誅,改設流官,以其酋二人韋貴、徐五為土巡檢,分掌其兵各萬餘。夷民不樂漢法,凡數叛。鎮安有男子名金,自言浚子。鎮安土官乃潛召其舊部酋長,出金而與之盟曰:「若小主也。」諸酋羅拜,擁金歸,聚兵五千,將攻城,復故地,遠近洶洶。浚誅時,其酋楊留者無所歸,率黨千餘人詣賓州,應募為打手。希儀在賓,留入言,欲往見小主人。希儀故患金,及聞留言,益大駭。因好謂留曰:「是岑浚第九子耶?我向征田州固聞之。」因自語:「岑氏其復乎?」欲以深動留,留果喜。已,召留密室,言:「予我重賂,即為金復官。」且出,復呼入曰:「韋貴、徐五今分將思恩兵,必讎金,善防之。」留益大信。金遂從五千人因留以見。門者奔告,請無納。希儀罵曰:「金,土官子,非賊,奈何不納?」引入,厚結之,又引以詣兵備副使,隨以計漸散其五千人。卒縛金,留亦自恨死,思恩復寧。已,從總督張經大破斷藤峽、弩灘賊,受賚歸。

希儀鎮柳、慶久,渠魁宿猾捕誅殆盡。先後搗巢,斬馘積五千餘級,未嘗悉奏功,故多不敘。十九年復謝病,柳人祀之山雲祠。旋起四川左參將,分守敘、滬及貴州迤西諸處。其冬,擢署都督僉事,充總兵官,鎮貴州。復謝病歸。塞上多警,召天下名將至京師,希儀在召中。希儀鎮柳、慶,每戰必先登,身數被創,陰雨輒痛劇,故數謝病。至京,亦以病辭。帝疑其規避,褫都督官,令赴部候用。翁萬達薦其才。會江、淮多盜,議設督捕總兵官,乃復希儀署都督僉事以往。

二十六年以為廣東副總兵。命自今將領至自川、廣、雲、貴者,毋推京營及西北邊,著為令。從總督張岳大破賀縣賊倪仲亮等,予實授,仍賚銀幣。瓊州五指山熟黎素畏法,供徭賦,知州邵浚虐取之。其酋那燕遂結崖州、感恩、昌化諸黎為亂。總督歐陽必進議并萬州、陵水黎討之,分兵五道。希儀適病,最後至,謂必進曰:「萬州、陵水黎未有黨惡之實,奈何并誅,益樹敵?莫若止三道。」必進從之。希儀乃偕參將武鸞、俞大猷等直入五指山下,斬那燕及其黨五千四百有奇,俘獲者五之一,招降三千七百人。捷聞,進都督同知,改貴州總兵官。復從岳平銅仁叛苗龍許保、吳黑苗。又以病歸。倭寇海上,命督川、廣兵赴剿。無功,為周如斗劾罷。

希儀為人坦率,居恒謔笑,洞見肺腑。及臨敵,應變出奇,人莫測。尤善撫士卒。常染危病,卒多自戕以禱於神。最後一人,至以箭穿其喉。其得士心如此。

石邦憲,字希尹,貴州清平衛人。嘉靖七年嗣世職為指揮使。累功,進署都指揮僉事,充銅仁參將。苗龍許保、吳黑苗叛,總督張岳議征之,而賊陷印江、石阡,邦憲坐逮問。岳以銅仁賊巢穴,而邦憲有謀勇,乃奏留之。邦憲遂與川、湖兵進貴州,破苗砦十有五。竄山箐者,搜戮殆盡。上功,邦憲第一。未及敘,而許保等突入思州,執知府李允簡以去。邦憲急邀,奪之歸。坐是停俸戴罪。賊既破思州,復糾餘黨,與湖廣蠟爾山苗合,欲攻石阡。不克,還過省溪。千戶安大朝等邀之,斬獲大半,盡奪其輜重,賊不能軍。邦憲乃使使購老穀、老犬革等執許保送軍門,而黑苗竄如故。復以計購烏朗土官田興邦等斬黑苗,賊盡平。遂進署都督僉事,充總兵官,代沈希儀鎮貴州。

臺黎砦苗關保倡亂,四川容山、廣西洪江諸苗應之。遠近騷然,撫剿莫能定。邦憲與湖廣兵分道討破之,傳檄十八砦,許執首惡贖罪。諸苗聽撫,設盟受約而還。

播州宣慰楊烈殺長官王黼,黼黨李保等治兵相攻且十年,總督馮岳與邦憲討平之。真州苗盧阿項為亂,邦憲以兵七千編筏渡江,直抵磨子崖。策賊必夜襲,先設備。賊至,擊敗之。賊求援於播州吳鯤。諸將懼,邦憲曰:「水西宣慰安萬銓,播州所畏也。吾調水西兵攻烏江,聲楊烈縱鯤助逆罪,烈奚暇救人乎?」已,水西兵至。邦憲進逼其巢,乘風縱火,斬關而登,賊大奔潰,擒賊首父子,斬獲四百七十餘人。進署都督同知。

破地隆阡叛苗四砦,又破答千諸砦,擒其渠魁。地隆阡遺賊龍老三、龍得奎結龍停苗老夭、扳凳苗石章保等縱兵掠,執石耶洞土官妻冉氏以歸,攻梅平砦。官軍要擒老三。得奎走免,復與老夭等攻破平南營囤。邦憲偵冉氏在老夭所,陽議贖,而潛擊殺老夭。官軍遂入龍停砦,并執扳凳砦苗龍老內,令執獻章保。於是諸苗悉降。白洗、養鵝諸苗叛,討擒其魁,降百餘砦。

湖廣漵浦瑤沈亞當等為亂,總督石勇檄邦憲討之,生擒亞當,斬獲二百有奇。漵浦甫平,銅仁、都勻苗相煽叛。邦憲亟馳還,率守備安大朝進剿。先破彪山砦賊,乘勝略定諸砦。獲賊首龍老羅、王三等,餘黨盡平。又與總督黃光昇,修湖北墩臺、烽堠百十所,招降冷水溪諸洞苗二十八砦。

播州容山副長官土舍韓甸與正長官土舍張問相攻,甸屢勝,遂糾生苗剽湖、貴境,垂二十年。問亦糾黨自助。邦憲討之,斬百餘人。問潛出,被獲。官軍乘勝入甸巢。會暮,大雨,迷失道。守備葉勛、百戶魏國相等陷伏中,死焉。邦憲奪圍出,還軍鎮遠。再徵之,賊沿江守。邦憲佯與爭,而別自上流三十里編竹以渡。水陸並進,大破之。斬甸,容山平。進右都督。

尋與巡撫吳維嶽招降平州叛酋楊珂,剿平龍里衛賊阿利等。當是時,水西宣慰安國亨恃眾跋扈,謁上官,辭色不善,輒鼓眾言雚噪而出。邦憲召責之曰:「爾欲反耶?吾視爾釜中魚爾。爾兵孰與雲、貴、川、湖多?爾四十八酋長,吾鑄四十八印畀之。朝下令,夕滅爾矣。」國亨叩頭謝,為斂戢。隆慶元年剿平鎮遠苗。已,又破誅白泥土官楊贇及苗酋龍力水等。部內帖然。

邦憲生長黔土,熟苗情。善用兵,大小數十百戰,無不摧破。前後進秩者四,賚銀幣十有三。所得俸賜,悉以饗士,家無贏資。為總兵官十七年,威鎮蠻中。與四川何卿、廣西沈希儀並稱一時名將。明年卒官。贈左都督。

贊曰:嗚呼,明至中葉,曷嘗無邊材哉!如馬永、梁震、周尚文、沈希儀之徒,出奇制勝,得士卒死力,雖古名將何以加焉?然功高賞薄,起蹶靡常。此無異故,其抗懷奮激,無以結歡在朝柄政重人,宜其齟齬不相入也。馬芳三代為將,父子兄弟先後殉國,偉矣哉!


\end{pinyinscope}