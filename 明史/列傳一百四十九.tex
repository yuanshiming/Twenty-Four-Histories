\article{列傳一百四十九}

\begin{pinyinscope}
盧象昇弟象晉象觀從弟象同劉之綸邱民仰邱禾嘉

盧象昇,字建斗,宜興人。祖立志,儀封知縣。象昇白皙而臞,膊獨骨,負殊力。舉天啟二年進士,授戶部主事。歷員外郎,稍遷大名知府。

崇禎二年,京師戒嚴,募萬人入衛。明年,進右參政兼副使,整飭大名、廣平、順德三府兵備,號「天雄軍」。又明年舉治行卓異,進按察使,治兵如故。象昇雖文士,善射,嫻將略,能治軍。

六年,山西賊流入畿輔,據臨城之西山。象昇擊卻之,與總兵梁甫、參議寇從化連敗賊。賊走還西山,圍遊擊董維坤冷水村。象昇設伏石城南,大破之,又破之青龍岡,又破之武安。連斬賊魁十一人,殲其黨,收還男女二萬。三郡之民,安堵者數歲。象昇每臨陣,身先士卒,與賊格鬥,刃及鞍勿顧,失馬即步戰,逐賊危崖,一賊自巔射中象昇額,又一矢僕夫斃馬下,象昇提刀戰益疾。賊駭走,相戒曰:「盧廉使遇即死,不可犯。」象升以是有能兵名。賊懼,南渡河。

明年,賊入楚,陷鄖陽六縣。命象昇以右僉都御史,代蔣允儀撫治鄖陽。時蜀寇返楚者駐鄖之黃龍灘,象昇與總督陳奇瑜分道夾擊,自烏林關、乜家溝、石泉壩、康寧坪、獅子山、太平河、竹木砭、箐口諸處,連戰皆捷,斬馘五千六百有奇,漢南寇幾盡。因請益鄖主兵,減稅賦,繕城郭,貸鄰郡倉穀,募商採銅鑄錢,鄖得完輯。

八年五月,擢象升右副都御史,代唐暉巡撫湖廣。八月,命總理江北、河南、山東、湖廣、四川軍務,兼湖廣巡撫。總督洪承疇辦西北,象昇辦東南。尋解巡撫任,進兵部侍郎,加督山西、陜西軍務,賜尚方劍,便宜行事。汝、洛告警,象昇倍道馳入汝。賊部眾三十餘萬,連營百里,勢甚盛。象昇督副將李重鎮、雷時聲等擊高迎祥於城西,用強弩射殺賊千餘人。迎祥、李自成走,陷光州,象昇復大破之確山。先是,大帥曹文詔、艾萬年陣亡,尤世威敗衄,諸將率畏賊不敢前,象昇每慷慨灑泣,激以忠義。軍中嘗絕三日餉,象昇亦水漿不入口,以是得將士心,戰輒有功。

九年正月,大會諸將於鳳陽。象昇乃上言曰:「賊橫而後調兵,賊多而後增兵,是為後局;兵至而後議餉,兵集而後請餉,是為危形。況請餉未敷,兵將從賊而為寇,是八年來所請之兵皆賊黨,所用之餉皆盜糧也。」又言:「總督、總理宜有專兵專餉。請調咸寧、甘、固之兵屬總督,薊、遼、關、寧之兵屬總理。」又言:「各直省撫臣,俱有封疆重任,毋得一有賊警即求援求調。不應則吳、越也,分應則何以支。」又言:「臺諫諸臣,不問難易,不顧死生,專以求全責備。雖有長材,從何展布。臣與督臣,有剿法無堵法,有戰法無守法。」言皆切中機宜。

於是迎祥圍廬州,不克,分道陷含山、和州,進圍滁州。象昇率總兵祖寬、遊擊羅岱救滁州,大戰城東五里橋,斬賊首搖天動,奪其駿馬。賊連營俱潰,逐北五十里,朱龍橋至關山,積尸填溝委塹,滁水為不流。賊乃北趨鳳陽,圍壽州,突潁、霍、蕭、碭、靈璧、虹,窺曹、單。總兵劉澤清拒河,乃掠考城、儀封而西。其犯亳者,折入歸德。永寧總兵官祖大樂邀擊之,賊乃北向開封。陳永福敗之朱仙鎮,賊遂走登封,與他賊合,分趨裕州、南陽。象昇合寬、大樂、岱兵大破之七頂山,殲自成精騎殆盡。已,次南陽,令大樂備汝寧,寬備鄧州,而躬率諸軍蹙賊。遣使告湖廣巡撫王夢尹、鄖陽撫治宋祖舜曰:「賊疲矣,東西邀擊,前阻漢江,可一戰殲也。」兩人竟不能禦,賊遂自光化潛渡漢入鄖。象昇遣總兵秦翼明、副將雷時聲由南漳、穀城入山擊賊。寬等騎軍,不利阻隘,副將王進忠軍嘩,羅岱、劉肇基兵多逃,追之則彎弓內向。象昇乃調四川及筸子土兵,搜捕均州賊。是時,楚、豫賊及迎祥等俱在秦、楚、蜀之交萬山中,象昇自南陽趨襄陽進兵。賊多兵少,而河南大饑,餉乏,邊兵益洶洶。承疇、象昇議,關中平曠,利騎兵,以寬、重鎮軍入陜,而襄陽、均、宜、穀、上津、南漳,環山皆賊。七月,象昇渡淅河而南。九月,追賊至鄖西。

京師戒嚴,有詔入衛,再賜尚方劍。既行,賊遂大逞,駸驍乎不可復制矣。既解嚴,詔遷兵部左侍郎,總督宣、大、山西軍務。大興屯政,穀熟,畝一鐘,積粟二十餘萬。天子諭九邊皆式宣、大。

明年春,聞宣警,即夜馳至天城。矢檄旁午,言二百里外乞炭馬蹄闊踏四十里。象昇曰:「此大舉也。」問:「入口乎?」曰:「未。」象昇曰:「殆欲右窺雲、晉,令我兵集宣,則彼乘虛入耳。」因檄雲、晉兵勿動,自率師次右衛,戒邊吏毋輕言戰。持一月,象昇曰:「懈矣,可擊。」哨知三十六營離牆六十里,潛召雲師西來,宣師東來,自督兵直子午,出羊房堡,計日鏖戰。乞炭聞之遂遁。象昇在陽和,乞炭不敢近邊。五月,丁外艱,疏十上,乞奔喪。時楊嗣昌奪情任中樞,亦起陳新甲制中,而令象昇席喪候代。進兵部尚書。新甲在遠,未即至。

九月,大清兵入牆子嶺、青口山,殺總督吳阿衡,毀正關,至營城石匣,駐於牛蘭。召宣、大、山西三總兵楊國柱、王樸、虎大威入衛,三賜象昇尚方劍,督天下援兵。象昇麻衣草履,誓師及郊,馳疏報曰:「臣非軍旅才。愚心任事,誼不避難。但自臣父奄逝,長途慘傷,潰亂五官,非復昔時;兼以草土之身踞三軍上,豈惟觀瞻不聳,尤虞金鼓不靈。」已聞總監中官高起潛亦衰絰臨戎,象昇謂所親曰:「吾三人皆不祥之身也。人臣無親,安有君。樞輔奪情,亦欲予變禮以分諐耶?處心若此,安可與事君。他日必面責之。」當是時,嗣昌、起潛主和議。象昇聞之,頓足歎曰:「予受國恩,恨不得死所,有如萬分一不幸,寧捐軀斷脰耳。」及都,帝召對,問方略。對曰:「臣主戰。」帝色變,良久曰:「撫乃外廷議耳,其出與嗣昌、起潛議。」出與議,不合。明日,帝發萬金犒軍,嗣昌送之,屏左右,戒毋浪戰,遂別去。師次昌平,帝復遣中官齎帑金三萬犒軍。明日,又賜御馬百,太僕馬千,銀鐵鞭五百。象昇曰:「果然外廷議也,帝意銳甚矣。」決策議戰,然事多為嗣昌、起潛撓。疏請分兵,則議宣、大、山西三帥屬象昇,關、寧諸路屬起潛。象升名督天下兵,實不及二萬。次順義。

先是,有瞽而賣卜者周元忠,善遼人,時遣之為媾。會嗣昌至軍,象升責數之曰:「文弱,子不聞城下盟《春秋》恥之,而日為媾。長安口舌如鋒,袁崇煥之禍其能免乎?」嗣昌頰赤,曰:「公直以尚方劍加我矣。」象升曰:「既不奔喪,又不能戰,齒劍者我也,安能加人?」嗣昌辭遁。象昇即言:「元忠講款,往來非一日,事始於薊門督監,受成於本兵,通國聞之,誰可諱也?」嗣昌語塞而去。又數日,會起潛安定門,兩人各持一議。新甲亦至昌平,象升分兵與之。當是時,象昇自將馬步軍列營都城之外,衝鋒陷陣,軍律甚整。

大清兵南下,三路出師:一由淶水攻易,一由新城攻雄,一由定興攻安肅。象昇遂由涿進據保定,命諸將分道出擊,大戰於慶都。編修楊廷麟上疏言:「南仲在內,李綱無功;潛善秉成,宗澤殞恨。國有若人,非封疆福。」嗣昌大怒,改廷麟兵部主事,贊畫行營,奪象昇尚書,侍郎視事。命大學士劉宇亮輔臣督師,巡撫張其平閉闉絕餉。俄又以雲、晉警,趣出關,王樸徑引兵去。

象昇提殘卒,次宿三宮野外。畿南三郡父老聞之,咸叩軍門請曰:「天下洶洶且十年,明公出萬死不顧一生之計為天下先。乃奸臣在內,孤忠見嫉。三軍捧出關之檄,將士懷西歸之心,棲遲絕野,一飽無時。脫巾狂噪,雲帥其見告矣。明公誠從愚計,移軍廣順,召集義師。三郡子弟喜公之來,皆以昔非公死賊,今非公死兵,同心戮力,一呼而裹糧從者可十萬,孰與只臂無援,立而就死哉!」象昇泫然流涕而謂父老曰:「感父老義。雖然,自予與賊角,經數十百戰未嘗衄。今者,分疲卒五千,大敵西衝,援師東隔,事由中制,食盡力窮,旦夕死矣,無徒累爾父老為也。」眾號泣雷動,各攜床頭斗粟餉軍,或貽棗一升,曰:「公煮為糧。」十二月十一日,進師至鉅鹿賈莊。起潛擁關、寧兵在雞澤,距賈莊五十里而近,象昇遣廷麟往乞援,不應。師至蒿水橋,遇大清兵。象昇將中軍,大威帥左,國柱帥右遂戰。夜半,觱篥聲四起。旦日,騎數萬環之三匝。象昇麾兵疾戰,呼聲動天,自辰迄未,炮盡矢窮。奮身鬥,後騎皆進,手擊殺數十人,身中四矢三刃,遂仆。掌牧楊陸凱懼眾之殘其屍而伏其上,背負二十四矢以死。僕顧顯者殉,一軍盡覆。大威、國柱潰圍乃得脫。

起潛聞敗,倉皇遁,不言象昇死狀。嗣昌疑之,有詔驗視。廷麟得其屍戰場,麻衣白網巾。一卒遙見,即號泣曰:「此吾盧公也。」三郡之民聞之,哭失聲。順德知府於潁上狀,嗣昌故靳之,八十日而後殮。明年,象昇妻王請恤。又明年,其弟象晉、象觀又請,不許。久之,嗣昌敗,廷臣多為言者,乃贈太子少師、兵部尚書,賜祭葬,世廕錦衣千戶。福王時,追謚忠烈,建祠奉祀。

象昇少有大志,為學不事章句。居官勤勞倍下吏,夜刻燭,雞鳴盥櫛,得一機要,披衣起,立行之。暇即角射,箭銜花,五十步外,發必中。愛才惜下如不及,三賜劍,未嘗戮一偏裨。

高平知縣侯弘文者,奇士也。僑寓襄陽,散家財,募滇軍隨象昇討賊。象昇移宣、大,弘文率募兵至楚,巡撫王夢尹以擾驛聞。象昇上疏救,不得,弘文卒遣戍。天下由是惜弘文而多象昇。

象昇好畜駿馬,皆有名字。嘗逐賊南漳,敗,追兵至沙河,水闊數丈,一躍而過,即所號五明驥也。

方象昇之戰歿也,嗣昌遣三邏卒察其死狀。其一人俞振龍者,歸言象昇實死。嗣昌怒,鞭之三日夜,且死,張目曰:「天道神明,無枉忠臣。」於是天下聞之,莫不欷歔,益恚嗣昌矣。

其後南都亡,象觀赴水死,象晉為僧,一門先後赴難者百餘人。從弟象同及其部將陳安死尤烈。

象觀,崇禎十五年,鄉薦第一,成進士。官中書。象晉、象同皆諸生。

象昇死時,年三十九。

劉之綸,字元誠,宜賓人。家世務農。之綸少從父兄力田,間艾薪樵,賣之市中。歸而學書,銘其座曰「必為聖人」,里中由是號之綸劉聖人。天啟初,舉鄉試。奢崇明反,以策干監司扼賊歸路,監司不能用。

崇禎元年第進士,改庶吉士。與同館金聲及所客申甫三人者相與為友,造單輪火車、偏廂車、獸車,刳木為西洋大小炮,不費司農錢。

明年冬,京師戒嚴。聲上書得召見,薦之綸及甫。帝立召之綸、甫。之綸言兵,了了口辨。帝大悅,授甫京營副總兵,資之金十七萬召募;改聲御史,監其軍;授之綸兵部右侍郎,副尚書閔夢得協理京營戎政。於是之綸賓賓以新進驟躋卿貳矣。

初,正月元日有黑氣起東北亙西方。甫見之大驚,趨語之綸、聲曰:「天變如此,汝知之乎?今年當喋血京城下,可畏也。」聞者皆笑。及冬十一月三日,大清兵破遵化,十五日至壩上,二十日薄都城,自北而西。都人從城上望之,如雲萬許片馳風,須臾已過。遂克良鄉,還至蘆溝,夜殺甫一軍七千餘人,黎明掩殺大帥滿桂、孫祖壽,生擒黑雲龍、麻登雲以去。之綸曰:「元日之言驗矣。」請行,無兵,則請京營兵,不許;則請關外川兵,不許;則議召募,召募得萬人,遂行。抵通州,時永平已陷,天大雨雪。之綸奏軍機,七上,不報。

明年正月,師次薊。當是時,大清兵蒙古諸部號十餘萬,駐永平;諸勤王軍數萬在薊。之綸乃與總兵馬世龍、吳自勉約,由薊趨永平,牽之無動,而自率兵八路進攻遵化。既由石門至白草頂,距遵化八里娘娘山而營,世龍、自勉不赴約。二十二日,大清兵自永平趨三屯營,驍騎三萬,望見山上軍,縱擊之。之綸發炮,炮炸,軍營自亂。左右請結陣徐退,以為後圖,之綸叱曰:「毋多言!吾受國恩,吾死耳!」嚴鼓再戰,流矢四集。之綸解所佩印付家人,「持此歸報天子」,遂死。一軍皆哭,拔營野戰,皆死之。尸還,矢飲於顱,不可拔,聲以齒嚙之出,以授其家人。

初,講官文震孟入都,之綸、聲往見之,震孟教以持重。之綸既受命視師,驟貴,廷臣抑之。震孟使人諷之,謂宜辭侍郎而易科銜以行,不聽。既行,通州守者不納,雨雪宿古廟中,御史董羽宸劾其行留。之綸曰:「小人意忌,有事則委卸,無事則議論,止從一侍郎起見耳。乞削臣今官,賜骸骨。」不許。及戰死,天子嘉其忠,從優恤,贈兵部尚書。震孟止之曰:「死綏,分也,侍郎非不尊。」遂不予贈,賜一祭半葬,任一子。之綸母老,二子幼,貧不能返柩,請於朝,給驛還。久之,贈尚書。後十五年,聲死難。

邱民仰,字長白,渭南人。萬曆中舉於鄉。以教諭遷順天東安知縣,釐宿弊十二事。河齧,歲旱蝗,為文祭禱。河他徙,蝗亦盡。調繁保定之新城。

崇禎二年,縣被兵,晨夕登陴守。四方勤王軍畢出其地,民仰調度有方,民不知擾。擢御史,號敢言。時四方多盜,鎮撫率怯懦不敢戰,釀成大亂。吳橋兵變,列城多陷,巡撫余大成、孫元化皆主撫。流賊擾山西,巡撫宋統殷下令,殺賊者抵死。民仰先後疏論其非,後皆如民仰言。遭妻喪,告歸。出為河間知府,遷天津副使,調大同監軍汝寧,遷永平右參政,移督寧前兵備。民仰善理劇,以故所移皆要地。

十三年三月,擢右僉都御史,代方一藻巡撫遼東,按行關外八城,駐寧遠。十四年春,錦州被圍,填壕毀塹,聲援斷絕。有傳其帥祖大壽語者:「逼以車營,毋輕戰。」總督洪承疇集兵,民仰轉餉,未發。帝憂之。朝議兩端。命郎中張若麒就行營計議,若麒至,則趣進師。七月,師次乳峰,去錦州五六里而營,旦日,楊國柱之軍潰。踰月,王樸軍亦潰。未幾,馬科等五將皆潰。大清兵掘松山,斷我歸路,遂大敗,蹂躪殺溺無算,退保松山。圍急,外援不至,芻糧竭。至明年二月,且半年矣,城破,承疇降,民仰死,若麒跳從海上蕩漁舟而還,寧遠、關門勁旅盡喪。事聞,帝驚悼甚,設壇都城,承疇十六,民仰六,賜祭盡哀。贈民仰右副都御史,官為營葬,錄其一子。尋命建祠都城外,與承疇並列,帝將親臨祭焉。將祭,聞承疇降,乃止。

邱禾嘉,貴州新添衛人。舉萬曆四十一年鄉試,好談兵。天啟時,安邦彥反,捐資製器,協擒其黨何中蔚。選祁門教諭,以貴州巡撫蔡復一請,遷翰林待詔,參復一軍。

崇禎元年,有薦其知兵者,命條上方略。帝稱善,即授兵部職方主事。三年正月,薊遼總督梁廷棟入主中樞,銜總理馬世龍違節制,命禾嘉監紀其軍。時永平四城失守,樞輔孫承宗在關門,聲息阻絕。薊遼總督張鳳翼未至,而順天巡撫方大任老病不能軍,惟禾嘉議通關門聲援,率軍入開平。二月,大清兵來攻,禾嘉力拒守,乃引去。已,分略古治鄉,禾嘉令副將何可綱、張洪謨、金國奇、劉光祚等迎戰,抵灤州。甫還,而大清兵復攻牛門、水門,又督參將曹文詔等轉戰,抵遵化而返。無何,四城皆復。

寧遠自畢自肅遇害,遂廢巡撫官,以經略兼之,至是議復設。廷棟力推禾嘉才,超拜右僉都御史,巡撫其地,兼轄山海關諸處。禾嘉初蒞鎮,大清兵以二萬騎圍錦州,禾嘉督諸將赴救,城獲全。登萊巡撫孫元化議徹島上兵於關外,規復廣寧及金、海、蓋三衛,禾嘉議用島兵復廣寧、義州、右屯。廷棟慮其難,以咨承宗。承宗上奏曰:「廣寧去海百八十里,去河百六十里,陸運難。義州地偏,去廣寧遠,必先據右屯,聚兵積粟,乃可漸逼廣寧。」又言:「右屯城已隳,修築而後可守。築之,敵必至,必復大、小凌河,以接松、杏、錦州。錦州繞海而居敵,難陸運。而右屯之後即海,據此則糧可給,兵可聚,始得為發軔地。」奏入,廷棟力主之,於是有大凌築城之議。

會禾嘉訐祖大壽,大壽亦發其贓私。承宗不欲以武將去文臣,抑使弗奏,密聞於朝,請改禾嘉他職。四年五月,命調南京太僕卿,以孫穀代。穀未至,部檄促城甚急。大壽以兵四千據其地,發班軍萬四千人築之,護以石硅土兵萬人。禾嘉往視之,條九議以上。工垂成,廷棟罷去。廷議大凌荒遠不當城,撤班軍赴薊,責撫鎮矯舉,令回奏。禾嘉懼,盡撤防兵,留班軍萬人,輸糧萬石濟之。

八月,大清兵抵城下,掘濠築牆,四面合圍,別遣一軍截錦州大道。城外堠臺皆下,城中兵出,悉敗還。禾嘉聞之,馳入錦州,與總兵官吳襄、宋偉合兵赴救。離松山三十餘里,與大清兵遇,大戰長山、小凌河間,互有傷損。九月望,大清兵薄錦州,分五隊直抵城下。襄、偉出戰不勝,乃入城。二十四日,監軍張春會襄、偉兵,過小凌河東五里,築壘列車營,為大凌聲援。大清兵扼長山,不得進。禾嘉遣副將張洪謨、祖大壽、靳國臣、孟道等出戰五里莊,亦不勝。夜趨小凌河,至長山接戰,大敗。春及副將洪謨、楊華徵、薛大湖等三十三人俱被執,副將張吉甫、滿庫、王之敬等戰歿。大壽不敢出,凌城援自此絕。敗書聞,舉朝震駭。孫穀代禾嘉,未至而罷,改命謝璉。璉畏懼,久不至。後兵事亟,召璉駐關外,禾嘉留治中。及是聞敗,移駐松山,圖再舉,言官以推委詆之帝。帝以禾嘉獨守松山,非卸責,戒飭而已。

大凌糧盡食人馬。大清屢移書招之,大壽許諾,獨副將可綱不從。十月二十七日,大壽殺可綱,與副將張存仁等三十九人投誓書約降。是夕出見,以妻子在錦州,請設計誘降錦州守將,而留諸子於大清。禾嘉聞大凌城炮聲,謂大壽得脫,與襄及中官李明臣、高起潛發兵往迎。適大壽偽逃還,遂俱入錦州。大凌城人民商旅三萬有奇,僅存三之一,悉為大清所有,城亦被毀。十一月六日,大清復攻杏山,明日攻中左所。城上用炮擊,乃退。大壽入錦州,未得間,而禾嘉知其納款狀,具疏聞於朝。因初奏大壽突圍出,前後不讎,引罪請死。於是言官交劾,嚴旨飭禾嘉。而帝於大壽欲羈縻之,弗罪也。

新撫璉已至,禾嘉猶在錦州,會廷議山海別設巡撫。詔罷璉,令方一藻撫寧遠,禾嘉仍以僉都御史巡撫山海、永平。尋論築城召釁罪,貶二秩,巡撫如故。禾嘉請為監視中官設標兵。御史宋賢詆其諂附中人,帝怒,貶賢三秩。禾嘉持論每與承宗異,不為所喜,時有詆諆。既遭喪敗,廷論益不容,遂堅以疾請。五年四月,詔許還京,以楊嗣昌代。令其妻代陳病狀。乃命歸田,未出都卒。

明世舉於鄉而仕至巡撫者,隆慶朝止海瑞,萬曆朝張守中、艾穆。莊烈帝破格求才,得十人:邱民仰、宋一鶴、何騰蛟、張亮以忠義著,劉可訓以武功聞,劉應遇、孫元化、徐起元皆以勤勞致位,而陳新甲官最顯。

贊曰:危亂之世,未嘗乏才,顧往往不盡其用。用矣,或掣其肘而驅之必死。若是者,人實為之,要之亦天意也。盧象昇在莊烈帝時,豈非不世之才,乃困抑之以至死,何耶!至忠義激發,危不顧身,若劉之綸、邱民仰之徒,又相與俱盡,則天意可知矣。


\end{pinyinscope}