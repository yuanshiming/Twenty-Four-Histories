\article{列傳第一百四十三}

\begin{pinyinscope}
劉宗周祝淵王毓蓍黃道周葉廷秀

劉宗周,字起東,山陰人。父坡,為諸生。母章氏妊五月而坡亡。既生宗周,家酷貧,攜之育外家。後以宗周大父老疾,歸事之,析薪汲水,持藥糜。然體孱甚,母嘗憂念之不置,遂成疾。又以貧故,忍而不治。萬曆二十九年,宗周成進士,母卒於家。宗周奔喪,為堊室中門外,日哭泣其中。服闋,選行人,請養大父母。遭喪,居七年始赴補。母以節聞於朝。

時有崑黨、宣黨與東林為難。宗周上言:「東林,顧憲成講學處。高攀龍、劉永澄、姜士昌、劉元珍,皆賢人。於玉立、丁元薦,較然不欺其志,有國士風。諸臣摘流品可也,爭意見不可;攻東林可也,黨崑、宣不可。」黨人大嘩,宗周乃請告歸。

天啟元年,起儀制主事。疏言:「魏進忠導皇上馳射戲劇,奉聖夫人出入自由。一舉逐諫臣三人,罰一人,皆出中旨,勢將指鹿為馬,生殺予奪,制國家大命。今東西方用兵,奈何以天下委閹豎乎?」進忠者魏忠賢也,大怒,停宗周俸半年。尋以國法未伸請戮崔文升以正弒君之罪,戮盧受以正交私之罪,戮楊鎬、李如楨、李維翰、鄭之范以正喪師失地之罪,戮高出、胡嘉棟、康應乾、牛維曜、劉國縉、傅國以正棄城逃潰之罪;急起李三才為兵部尚書,錄用清議名賢丁元薦、李朴等,諍臣楊漣、劉重慶等,以作仗節徇義之氣。帝切責之。累遷光祿丞、尚寶、太僕少卿,移疾歸。四年,起右通政,至則忠賢逐東林且盡,宗周復固辭。忠賢責以矯情厭世,削其籍。

崇禎元年冬,召為順天府尹。辭,不許。明年九月入都,上疏曰:

陛下勵精求治,宵旰靡寧。然程效太急,不免見小利而速近功,何以致唐、虞之治?夫今日所汲汲於近功者,非兵事乎?誠以屯守為上策,簡卒節餉,修刑政而威信布之,需以歲月,未有不望風束甲者,而陛下方銳意中興,刻期出塞。當此三空四盡之秋,竭天下之力以奉饑軍而軍愈驕,聚天下之軍以博一戰而戰無日,此計之左也。

今日所規規於小利者,非國計乎?陛下留心民瘼,惻然恫CR,而以司農告匱,一時所講求者皆掊克聚斂之政。正供不足,繼以雜派;科罰不足,加以火耗。水旱災傷,一切不問,敲撲日峻,道路吞聲,小民至賣妻鬻子以應。有司以掊克為循良,而撫字之政絕;上官以催徵為考課,而黜陟之法亡。欲求國家有府庫之財,不可得已。

功利之見動,而廟堂之上日見其煩苛。事事糾之不勝糾,人人摘之不勝摘,於是名實紊而法令滋。頃者,特嚴贓吏之誅,自宰執以下,坐重典者十餘人,而貪風未盡息,所以導之者未善也。賈誼曰:「禮禁未然之先,法施已然之後。」誠導之以禮,將人人有士君子之行,而無狗彘之心,所謂禁之於未然也。今一切詿誤及指稱賄賂者,即業經昭雪,猶從吏議,深文巧詆,絕天下遷改之途,益習為頑鈍無恥,矯飾外貌以欺陛下。士節日隳,官邪日著,陛下亦安能一一察之。

且陛下所以勞心焦思於上者,以未得賢人君子用之也,而所嘉予而委任者,率奔走集事之人:以摘發為精明,以告訐為正直,以便給為才住,又安所得賢者而用之?得其人矣,求之太備,或以短而廢長;責之太苛,或因過而成誤。

且陛下所擘畫,動出諸臣意表,不免有自用之心。臣下救過不給,讒諂者因而間之,猜忌之端遂從此起。夫恃一人之聰明,而使臣下不得盡其忠,則耳目有時壅;憑一人之英斷,而使諸大夫國人不得衷其是,則意見有時移。方且為內降,為留中,何以追喜起之盛乎?數十年來,以門戶殺天下幾許正人,猶蔓延不已。陛下欲折君子以平小人之氣,用小人以成君子之公,前日之覆轍將復見於天下也。

陛下求治之心,操之太急。醞釀而為功利,功利不已,轉為刑名;刑名不已,流為猜忌;猜忌不已,積為壅蔽。正人心之危,所潛滋暗長而不自知者。誠能建中立極,默正此心,使心之所發,悉皆仁義之良,仁以育天下,義以正萬民,自朝廷達於四海,莫非仁義之化,陛下已一旦躋於堯、舜矣。

帝以為迂闊,然歎其忠。

未幾,都城被兵,帝不視朝,章奏多留中不報。傳旨辦布囊八百,中官競獻馬騾,又令百官進馬。宗周曰:「是必有以遷幸動上者。」乃詣午門叩頭諫曰:「國勢強弱,視人心安危。乞陛下出御皇極門,延見百僚,明言宗廟山陵在此,固守外無他計。」俯伏待報,自晨迄暮,中官傳旨乃退。米價騰躍,請罷九門稅,修賈區以處貧民,為粥以養老疾,嚴行保甲之法,人心稍安。

時樞輔諸臣多下獄者,宗周言:「國事至此,諸臣負任使,無所逃罪,陛下亦宜分任咎。禹、湯罪己,興也勃焉。曩皇上以情面疑群臣,群臣盡在疑中,日積月累,結為陰痞,識者憂之。今日當開示誠心,為濟難之本,御便殿以延見士大夫,以票擬歸閣臣,以庶政歸部、院,以獻可替否予言官。不效,從而更置之,無坐錮以成其罪。乃者朝廷縛文吏如孤雛,而視武健士不啻驕子,漸使恩威錯置。文武皆不足信,乃專任一二內臣,閫以外次第委之。自古未有宦官典兵不誤國者。」又劾馬世龍、張鳳翼、吳阿衡等罪,忤帝意。

三年以疾在告,進祈天永命之說,言:

法天之大者,莫過於重民命,則刑罰宜當宜平。陛下以重典繩下,逆黨有誅,封疆失事有誅。一切詿誤,重者杖死,輕者謫去,朝署中半染赭衣。而最傷國體者,無如詔獄。副都御史易應昌以平反下吏,法司必以鍛煉為忠直,蒼鷹乳虎接踵於天下矣。願體上天好生之心,首除詔獄,且寬應昌,則祈天永命之一道也。

法天之大者,莫過於厚民生,則賦斂宜緩宜輕。今者宿逋見征及來歲預征,節節追呼,閭閻困敝,貪吏益大為民厲。貴州巡按蘇琰以行李被訐於監司。巡方黷貨,何問下吏?吸膏吮脂之輩,接迹於天下矣。願體上天好生之心,首除新餉,并嚴飭官方,則祈天永命之又一道也。

然大君者,天之宗子;輔臣者,宗子之家相。陛下置輔,率由特簡。亦願體一人好生之心,毋驅除異己,構朝士以大獄,結國家朋黨之禍;毋寵利居成功,導人主以富強,釀天下土崩之勢。

周延儒、溫體仁見疏不懌。以時方禱雨,而宗周稱疾,指為偃蹇,激帝怒,擬旨詰之。且令陳足兵、足餉之策,宗周條畫以對,延儒、體仁不能難。

為京尹,政令一新,挫豪家尤力。閹人言事輒不應,或相詬誶,宗周治事自如。武清伯蒼頭毆諸生,宗周捶之,枷武清門外。嘗出,見優人籠篋,焚之通衢。周恤單丁下戶尤至。居一載,謝病歸,都人為罷市。

八年七月,內閣缺人,命吏部推在籍者,以孫慎行、林釬及宗周名上。詔所司敦趨,宗周固辭不許。明年正月入都,慎行已卒,與釬入朝。帝問人才、兵食及流寇猖獗狀。宗周言:「陛下求治太急,用法太嚴,布令太煩,進退天下士太輕。諸臣畏罪飾非,不肯盡職業,故有人而無人之用,有餉而無餉之用,有將不能治兵,有兵不能殺賊。流寇本朝廷赤子,撫之有道,則還為民。今急宜以收拾人心為本,收拾人心在先寬有司。參罰重則吏治壞,吏治壞則民生困,盜賊由此日繁。」帝又問兵事。宗周言:「禦外以治內為本。內治修,遠人自服,干羽舞而有苗格。願陛下以堯、舜之心,行堯、舜之政,天下自平。」對畢趨出。帝顧體仁迂其言,命釬輔政,宗周他用。旋授工部左侍郎。踰月,上《痛憤時艱疏》,言:

陛下銳意求治,而二帝三王治天下之道未暇講求,施為次第猶多未得要領者。首屬意於邊功,而罪督遂以五年恢復之說進,是為禍胎。己巳之役,謀國無良,朝廷始有積輕士大夫之心。自此耳目參於近侍,腹心寄於干城,治術尚刑名,政體歸叢脞,天下事日壞而不可救。廠衛司譏察,而告訐之風熾;詔獄及士紳,而堂廉之等夷;人人救過不給,而欺罔之習轉甚;事事仰成獨斷,而諂諛之風日長。三尺法不伸於司寇,而犯者日眾,詔旨雜治五刑,歲躬斷獄以數千,而好生之德意泯。刀筆治絲綸而王言褻,誅求及瑣屑而政體傷。參罰在錢穀而官愈貪,吏愈橫,賦愈逋;敲撲繁而民生瘁,嚴刑重斂交困而盜賊日起。總理任而臣下之功能薄,監視遣而封疆之責任輕。督、撫無權而將日懦,武弁廢法而兵日驕,將懦兵驕而朝廷之威令并窮於督、撫。朝廷勒限平賊,而行間日殺良報功,生靈益塗炭。一旦天牖聖衷,撤總監之任,重守令之選,下弓旌之招,收酷吏之威,布維新之化,方與二三臣工洗心滌慮,以聯泰交,而不意君臣相遇之難也。得一文震孟而以單辭報罷,使大臣失和衷之誼;得一陳子壯而以過戇坐辜,使朝寧無吁咈之風。此關於國體人心非淺鮮者。

陛下必體上天生物之心以敬天,而不徒倚風雷;必念祖宗鑒古之制以率祖,而不輕改作。以簡要出政令,以寬大養人才,以忠厚培國脈。發政施仁,收天下泮渙之人心,而且還內廷掃除之役,正懦帥失律之誅,慎天潢改授之途。遣廷臣齎內帑巡行郡國為招撫使,赦其無罪而流亡者。陳師險隘,堅壁清野,聽其窮而自歸。誅渠之外,猶可不殺一人,而畢此役,奚待於觀兵哉。

疏入,帝怒甚,諭閣臣擬嚴旨再四。每擬上,帝輒手其疏覆閱,起行數周。已而意解,降旨詰問,謂大臣論事宜體國度時,不當效小臣歸過朝廷為名高,且獎其清直焉。

時太僕缺馬價,有詔願捐者聽,體仁及成國公朱純臣以下皆有捐助。又議罷明年朝覲。宗周以輸貲、免覲為大辱國。帝雖不悅,心善其忠,益欲大用。體仁患之,募山陰人許瑚疏論之,謂宗周道學有餘,才住不足。帝以瑚同邑,知之宜真,遂已不用。

其秋,三疏請告去。至天津,聞都城被兵,遂留養疾。十月,事稍定,乃上疏曰:

己巳之變,誤國者袁崇煥一人。小人競修門戶之怨,異己者概坐以崇煥黨,日造蜚語,次第去之。自此小人進而君子退,中官用事而外廷浸疏。文法日繁,欺罔日甚,朝政日隳,邊防日壞。今日之禍,實己巳以來釀成之也。

且以張鳳翼之溺職中樞也,而俾之專征,何以服王洽之死?以丁魁楚等之失事於邊也,而責之戴罪,何以服劉策之死?諸鎮勤王之師,爭先入衛者幾人,不聞以逗留蒙詰責,何以服耿如杞之死?今且以二州八縣之生靈,結一飽颺之局,則廷臣之累累若若可幸無罪者,又何以謝韓爌、張鳳翔、李邦華諸臣之或戍或去?豈昔為異己驅除,今不難以同己相容隱乎?臣於是而知小人之禍人國無已時也。

昔唐德宗謂群臣曰:「人言盧巳奸邪,朕殊不覺。」群臣對曰:「此乃巳之所以為奸邪也。」臣每三覆斯言,為萬世辨奸之要。故曰:「大奸似忠,大佞似信。」頻年以來,陛下惡私交,而臣下多以告訐進;陛下錄清節,而臣下多以曲謹容;陛下崇勵精,而臣下奔走承順以為恭;陛下尚綜核,而臣下瑣屑吹求以示察。凡若此者,正似信似忠之類,究其用心,無往不出於身家利祿。陛下不察而用之,則聚天下之小人立於朝,有所不覺矣。天下即乏才,何至盡出中官下?而陛下每當緩急,必委以大任。三協有遣,通、津、臨、德有遣;又重其體統,等之總督。中官總督,置總督何地?總督無權,置撫、按何地?是以封疆嘗試也。

且小人每比周小人,以相引重,君子獨岸然自異。故自古有用小人之君子,終無黨比小人之君子。陛下誠欲進君子退小人,決理亂消長之機,猶復用中官參制之,此明示以左右袒也。有明治理者起而爭之,陛下即不用其言,何至并逐其人?而御史金光辰竟以此逐,若惟恐傷中官心者,尤非所以示天下也。

至今日刑政之最舛者,成德,傲吏也,而以贓戍,何以肅懲貪之令?申紹芳,十餘年監司也,而以莫須有之鑽刺戍,何以昭抑競之典?鄭鄤之獄,或以誣告坐,何以示敦倫之化?此數事者,皆為故輔文震孟引繩批根,即向驅除異己之故智,而廷臣無敢言。

陛下亦無從知之也。嗚呼,八年之間,誰秉國成,而至於是!臣不能為首揆溫體仁解矣。語曰:「誰生厲階,至今為梗。」體仁之謂也。

疏奏,帝大怒,體仁又上章力詆,遂斥為民。

十四年九月,吏部缺左侍郎,廷推不稱旨。帝臨朝而嘆,謂大臣:「劉宗周清正敢言,可用也。」遂以命之。再辭不得,乃趨朝。道中進三札:一曰明聖學以端治本,二曰躬聖學以建治要,三曰重聖學以需治化,凡數千言。帝優旨報之。明年八月,未至擢左都御史。力辭,有詔敦趨。踰月,入見文華殿。帝問都察院職掌安在,對曰:「在正己以正百僚。必存諸中者,上可對君父,下可質天下士大夫,而後百僚則而象之。大臣法,小臣廉,紀綱振肅,職掌在是,而責成巡方其首務也。巡方得人,則吏治清,民生遂。」帝曰:「卿力行以副朕望。」乃列建道揆、貞法守、崇國體、清伏奸、懲官邪、飭吏治六事以獻,帝褒納焉。俄劾御史喻上猷、嚴雲京而薦袁愷、成勇,帝並從之。其後上猷受李自成顯職,卒為世大詬。

冬十月,京師被兵。請旌死事盧象升,而追戮誤國奸臣楊嗣昌,逮跋扈悍將左良玉;防關以備反攻,防潞以備透渡,防通、津、臨、德以備南下。帝不能盡行。

閏月晦日召見廷臣於中左門。時姜埰、熊開元以言事下詔獄,宗周約九卿共救。入朝,聞密旨置二人死。宗周愕然謂眾曰:「今日當空署爭,必改發刑部始已。」及入對,御史楊若橋薦西洋人湯若望善火器,請召試。宗周曰:「邊臣不講戰守屯戍之法,專恃火器。近來陷城破邑,豈無火器而然?我用之制人,人得之亦可制我,不見河間反為火器所破乎?國家大計,以法紀為主。大帥跋扈,援師逗遛,奈何反姑息,為此紛紛無益之舉耶?」因議督、撫去留,則請先去督師范志完。且曰:「十五年來,陛下處分未當,致有今日敗局。不追禍始,更弦易轍,欲以一切茍且之政,補目前罅漏,非長治之道也。」帝變色曰:「前不可追,善後安在?」宗周曰:「在陛下開誠布公,公天下為好惡,合國人為用舍,進賢才,開言路,次第與天下更始。」帝曰:「目下烽火逼畿甸,且國家敗壞已極,當如何?」宗周曰:「武備必先練兵,練兵必先選將,選將必先擇賢督、撫,擇賢督、撫必先吏、兵二部得人。宋臣曰:『文官不愛錢,武官不惜死,則天下太平。』斯言,今日鍼砭也。論者但論才望,不問操守;未有操守不謹,而遇事敢前,軍士畏威者。若徒以議論捷給,舉動恢張,稱曰才望,取爵位則有餘,責事功則不足,何益成敗哉?」帝曰:「濟變之日,先才後守。」宗周曰:「前人敗壞,皆由貪縱使然;故以濟變言,愈宜先守後才。」帝曰:「大將別有才局,非徒操守可望成功。」宗周曰:「他不具論,如范志完操守不謹,大將偏裨無不由賄進,所以三軍解體。由此觀之,操守為主。」帝色解曰:「朕已知之。」敕宗周起。

於是宗周出奏曰:「陛下方下詔求賢,姜埰、熊開元二臣遽以言得罪。國朝無言官下詔獄者,有之自二臣始。陛下度量卓越,妄如臣宗周,戇直如臣黃道周,尚蒙使過之典,二臣何不幸,不邀法外恩?」帝曰:「道周有學有守,非二臣比。」宗周曰:「二臣誠不及道周,然朝廷待言官有體,言可用用之,不可置之。即有應得之罪,亦當付法司。今遽下詔獄,終於國體有傷。」帝怒甚,曰:「法司錦衣皆刑官,何公何私?且罪一二言官,何遽傷國體?有如貪贓壞法,欺君罔上,皆可不問乎?」宗周曰:「錦衣,膏梁子弟,何知禮義?聽寺人役使。即陛下問貪贓壞法,欺君罔上,亦不可不付法司也。」帝大怒曰:「如此偏黨,豈堪憲職!」有間曰:「開元此疏,必有主使,疑即宗周。」金光辰爭之。帝叱光辰,并命議處。翼日,光辰貶三秩調用,宗周革職,刑部議罪。閣臣持不發,捧原旨御前懇救,乃免,斥為民。

歸二年而京師陷。宗周徒步荷戈,詣杭州,責巡撫黃鳴駿發喪討賊,鳴駿誡以鎮靜,宗周勃然曰:「君父變出非常,公專閫外,不思枕戈泣血,激勵同仇,顧藉口鎮靜,作遜避計耶?」鳴駿唯唯。明日,復趣之。鳴駿曰:「發喪必待哀詔。」宗周曰:「嘻,此何時也,安所得哀詔哉!」鳴駿乃發喪。問師期,則曰:「甲仗未具。」宗周嘆曰:「嗟乎,是烏足與有為哉!」乃與故侍郎朱大典,故給事中章正宸、熊汝霖召募義旅。將發,而福王監國於南京,起宗周故官。宗周以大仇未報,不敢受職,自稱草莽孤臣,疏陳時政,言:

今日大計,舍討賊復仇,無以表陛下渡江之心;非毅然決策親征,無以作天下忠義之氣。

一曰據形勝以規進取。江左非偏安之業,請進圖江北。鳳陽號中都,東扼徐、淮,北控豫州,西顧荊、襄,而南去金陵不遠,請以駐親征之師。大小銓除,暫稱行在,少存臣子負罪引慝之心。從此漸進,秦、晉、燕、齊必有響應而起者。

一曰重籓屏以資彈壓。淮、揚數百里,設兩節鉞,不能禦亂,爭先南下,致江北一塊土,拱手授賊。督漕路振飛坐守淮城,久以家屬浮舟遠地,是倡之逃也;於是鎮臣劉澤清、高傑遂有家屬寄江南之說。軍法臨陣脫逃者斬,臣謂一撫二鎮皆可斬也。

一曰慎爵賞以肅軍情。請分別各帥封賞,孰當孰濫,輕則收侯爵,重則奪伯爵。夫以左帥之恢復而封,高、劉之敗逃亦封,又誰不當封者?武臣既濫,文臣隨之,外臣既濫,中璫隨之,恐天下聞而解體也。

一曰核舊官以立臣紀。燕京既破,有受偽官而叛者,有受偽官而逃者,有在封守而逃者,有奉使命而逃者,法皆不赦。亟宜分別定罪,為戒將來。

至於偽命南下,徘徊順逆之間,實繁有徒;必且倡為曲說,以惑人心,尤宜誅絕。

又言:

當賊入秦流晉,漸過畿南,遠近洶洶,獨大江南北晏然,而二三督撫不聞遣一騎以壯聲援,賊遂得長驅犯闕。坐視君父之危亡而不救,則封疆諸臣之當誅者一。凶問已確,諸臣奮戈而起,決一戰以贖前愆,自當不俟朝食。方且仰聲息於南中,爭言固圉之策,卸兵權於閫外,首圖定策之功,則封疆諸臣之當誅者又一。新朝既立之後,謂宜不俟終日,首遣北伐之師。不然,則亟馳一介,間道北進,檄燕中父老,起塞上名王,哭九廟,厝梓宮,訪諸王。更不然,則起閩帥鄭芝龍,以海師下直沽,九邊督鎮合謀共奮,事或可為。而諸臣計不出此,則舉朝謀國不忠之當誅者又一。罪廢諸臣,量從昭雪,自應援先帝遺詔及之,今乃概用新恩。誅閹定案,前後詔書鶻突,勢必彪虎之類,盡從平反而後已,則舉朝謀國不忠之當誅者又一。臣謂今日問罪,當自中外諸臣不職者始。

詔納其言,宣付史館,中外為悚動。而馬士英、高傑、劉澤清恨甚,滋欲殺宗周矣。

宗周連疏請告不得命,遂抗疏劾士英,言:

陛下龍飛淮甸,天實予之。乃有扈蹕微勞,入內閣,進中樞,宮銜世廕,晏然當之不疑者,非士英乎?於是李沾侈言定策,挑激廷臣矣。劉孔昭以功賞不均,發憤冢臣,朝端嘩然聚訟,而群陰且翩翩起矣。借知兵之名,則逆黨可以然灰,寬反正之路,則逃臣可以汲引,而閣部諸臣且次第言去矣。中朝之黨論方興,何暇圖河北之賊?立國之本紀已疏,何以言匡攘之略?高傑一逃將也,而奉若驕子,浸有尾大之憂。淮、揚失事,不難譴撫臣道臣以謝之,安得不長其桀驁,則亦恃士英卵翼也。劉、黃諸將,各有舊汛地,而置若弈棋,洶洶為連雞之勢,至分剖江北四鎮以慰之,安得不啟其雄心,則皆高傑一人倡之也。京營自祖宗以來,皆勳臣為政,樞貳佐之。陛下立國伊始,而有內臣盧九德之命,則士英有不得辭其責者。

總之,兵戈盜賊,皆從小人氣類感召而生,而小人與奄宦又往往相表裏。自古未有奄宦用事,而將帥能樹功於方域者。惟陛下首辨陰陽消長之機,出士英仍督鳳陽,聯絡諸鎮,決用兵之策。史可法即不還中樞,亦當自淮而北,歷河以南,別開幕府,與士英相掎角。京營提督,獨斷寢之。書之史冊,為弘光第一美政。

王優詔答之,而促其速入。

士英大怒,即日具疏辭位,且揚言於朝曰:「劉公自稱草莽孤臣,不書新命,明示不臣天子也。」其私人朱統金類遂劾宗周疏請移蹕鳳陽:「鳳陽,高牆所在,欲以罪宗處皇上,而與史可法擁立潞王。其兵已伏丹陽,當急備。」而澤清、傑日夜謀所以殺宗周者不得,乃遣客十輩往刺宗周。宗周時在丹陽,終日危坐,未嘗有惰容,客前後至者,不敢加害而去。而黃鳴駿入覲,兵抵京口,與防江兵相擊斗。士英以統金類言為信也,亦震恐。於是澤清疏劾:「宗周陰撓恢復,欲誅臣等,激變士心,召生靈之禍。」劉良佐亦具疏言宗周力持「三案」,為門戶主盟,倡議親征,圖晁錯之自為居守,司馬懿之閉城拒君。疏未下,澤清復草一疏,署傑、良佐及黃得功名上之,言:「宗周勸上親征,謀危君父,欲安置陛下於烽火凶危之地。蓋非宗周一人之謀,姜曰廣、吳甡合謀也。曰廣心雄膽大,翊戴非其本懷,故陰結死黨,翦除諸忠,然後迫劫乘輿,遷之別郡。如甡、宗周入都,臣等即渡江赴闕,面訐諸奸,正《春秋》討賊之義。」疏入,舉朝大駭,傳諭和衷集事。宗周不得已,以七月十八日入朝。初,澤清疏出,遣人錄示傑。傑曰:「我輩武人,乃預朝事耶?」得功疏辨:「臣不預聞。」士英寢不奏。可法不平,遣使遍詰諸鎮,咸云不知,遂據以入告,澤清輩由是氣沮。

士英既嫉宗周,益欲去之,而薦阮大鋮知兵。有詔冠帶陛見。未幾,中旨特授兵部添注右侍郎。宗周曰:「大鋮進退,係江左興亡,老臣不敢不一爭之。不聽,則亦將歸爾。」疏入,不聽,宗周遂告歸,詔許乘傳。將行,疏陳五事:

一曰修聖政,毋以近娛忽遠猷。國家不幸,遭此大變,今紛紛制作,似不復有中原志者。土木崇矣,珍奇集矣,俳優雜劇陳矣;內豎充廷,金吾滿座,戚畹駢闐矣;讒夫昌,言路扼,官常亂矣。所謂狃近娛而忽遠圖也。

一曰振王綱,無以主恩傷臣紀。自陛下即位,中外臣工不曰從龍,則曰佐命。一推恩近侍,則左右因而秉權;再推恩大臣,則閣部可以兼柄;三推恩勳舊,則陳乞至今未已;四推恩武弁,則疆場視同兒戲。表裏呼應,動有藐視朝廷之心;彼此雄長,即為犯上無等之習。禮樂征伐,漸不出自天子,所謂褻主恩而傷臣紀也。

一曰明國是,無以邪鋒危正氣。朋黨之說,小人以加君子,釀國家空虛之禍,先帝末造可鑒也。今更為一二元惡稱冤,至諸君子後先死於黨、死於徇國者,若有餘戮。揆厥所由,止以一人進用,動引三朝故事,排抑舊人。私交重,君父輕,身自樹黨,而坐他人以黨,所謂長邪鋒而危正氣也。

一曰端治術,無以刑名先教化。先帝頗尚刑名,而殺機先動於溫體仁。殺運日開,怨毒滿天下。近如貪吏之誅,不經提問,遽科罪名;未科罪名,先追贓罰。假令有禹好善之巡方,借成德以媚權相,又孰辨之?又職方戎政之奸弊,道路嘖有煩言,雖衛臣有不敢問者,則廠衛之設何為?徒令人主虧至德,傷治體,所謂急刑名而忘教化也。

一曰固邦本,毋以外釁釀內憂。前者淮、揚告變,未幾而高、黃二鎮治兵相攻。四鎮額兵各三萬,不以殺敵而自相屠毒,又日煩朝廷講和,何為者!夫以十二萬不殺敵之兵,索十二萬不殺敵之餉,必窮之術耳。不稍裁抑,惟加派橫徵。蓄一二蒼鷹乳虎之有司,以天下徇之已矣,所謂積外釁而釀內憂也。

優詔報聞。

明年五月,南都亡。六月,潞王降,杭州亦失守。宗周方食,推案慟哭,自是遂不食。移居郭外,有勸以文、謝故事者。宗周曰:「北都之變,可以死,可以無死,以身在田里,尚有望於中興也。南都之變,主上自棄其社稷,尚曰可以死,可以無死,以俟繼起有人也。今吾越又降矣,老臣不死,尚何待乎?若曰身不在位,不當與城為存亡,獨不當與土為存亡乎?此江萬里所以死也。」出辭祖墓,舟過西洋港,躍入水中,水淺不得死,舟人扶出之。絕食二十三日,始猶進茗飲,後勺水不下者十三日,與門人問答如平時。閏六月八日卒,年六十有八。其門人徇義者有祝淵、王毓蓍。

淵,字開美,海寧人。崇禎六年舉於鄉。自以年少學未充,棲峰巔僧舍,讀書三年,山僧罕見其面。十五年冬,會試入都,適宗周廷諍姜埰、熊開元削籍。淵抗疏曰:「宗周戇直性成,忠孝天授,受任以來,蔬食不飽,終宵不寢,圖報國恩。今四方多難,貪墨成風,求一清剛臣以司風紀,孰與宗周?宗周以迂戇斥,繼之者必淟涊;宗周以偏執斥,繼之者必便捷。淟涊便捷之夫進,必且營私納賄,顛倒貞邪。乞收還成命,復其故官,天下幸甚。」帝得疏不懌,停淵會試,下禮官議。淵故不識宗周,既得命往謁。宗周曰:「子為此舉,無所為而為之乎,抑動於名心而為之也?」淵爽然避席曰:「先生名滿天下,誠恥不得列門牆爾,願執贄為弟子。」明年,從宗周山陰。禮官議上,逮下詔獄,詰主使姓名。淵曰:「男兒死即死爾,何聽人指使為!」移刑部,進士共疏出淵。未幾,都城陷,營死難太常少卿吳麟征喪,歸其柩。詣南京刑部,竟前獄,尚書諭止之。上疏請誅奸輔,通政司抑不奏。給事中陳子龍疏薦淵及待詔塗仲吉義士,可為臺諫。仲吉者,漳浦人,以諸生走萬里上書明黃道周冤,得罪杖譴者也。不許。

宗周罷官家居,淵數往問學。嘗有過,入曲室長跪流涕自手過。杭州失守,淵方葬母,趣竣工。既葬,還家設祭,即投繯而卒,年三十五也。踰二日,宗周餓死。

毓蓍,字元趾,會稽人。為諸生,跌宕不羈。已,受業宗周之門,同門生咸非笑之。杭州不守,宗周絕粒未死,毓蓍上書曰:「願先生早自裁,毋為王炎午所弔。」俄一友來視,毓蓍曰:「子若何?」曰:「有陶淵明故事在。」毓蓍曰:「不然。吾輩聲色中人,慮久則難持也。」一日,遍召故交歡飲,伶人奏樂。酒罷,攜燈出門,投柳橋下,先宗周一月死。鄉人私謚正義先生。

宗周始受業於許孚遠。已,入東林書院,與高攀龍輩講習。馮從吾首善書院之會,宗周亦與焉。越中自王守仁後,一傳為王畿,再傳為周汝登、陶望齡,三傳為陶奭齡,皆雜於禪。奭齡講學白馬山,為因果說,去守仁益遠。宗周憂之,築證人書院,集同志講肄。且死,語門人曰:「學之要,誠而已,主敬其功也。敬則誠,誠則天。良知之說,鮮有不流於禪者。」宗周在官之日少,其事君,不以面從為敬。入朝,雖處暗室,不敢南嚮。或訊大獄,會大議,對明旨,必卻坐拱立移時。或謝病,徒步家居,布袍粗飯,樂道安貧。聞召就道,嘗不能具冠裳。學者稱念臺先生。子汋,字伯繩。

黃道周,字幼平,漳浦人。天啟二年進士。改庶吉士,授編修,為經筵展書官。故事,必膝行前,道周獨否,魏忠賢目攝之。未幾,內艱歸。

崇禎二年起故官,進右中允。三疏救故相錢龍錫,降調,龍錫得減死。五年正月方候補,遘疾求去。瀕行,上疏曰:

臣自幼學《易》,以天道為準。上下載籍二千四百年,考其治亂,百不失一。陛下御極之元年,正當《師》之上九,其爻云:「大君有命,開國承家,小人勿用。」陛下思賢才不遽得,懲小人不易絕,蓋陛下有大君之實,而小人懷干命之心。臣入都以來,所見諸大臣皆無遠猷,動尋苛細,治朝寧者以督責為要談,治邊疆者以姑息為上策。序仁義道德,則以為迂昧而不經;奉刀筆簿書,則以為通達而知務。一切磨勘,則葛藤終年;一意不調,而株連四起。陛下欲整頓紀綱,斥攘外患,諸臣用之以滋章法令,摧折縉紳;陛下欲剔弊防奸,懲一警百,諸臣用之以借題修隙,斂怨市權。且外廷諸臣敢誑陛下者,必不在拘攣守文之士,而在權力謬巧之人;內廷諸臣敢誑陛下者,必不在錐刀泉布之微,而在阿柄神叢之大。惟陛下超然省覽,旁稽載籍,自古迄今,決無數米量薪,可成遠大之猷,吹毛數睫,可奏三五之治者。彼小人見事,智每短於事前,言每多於事後。不救凌圍,而謂凌城必不可築;不理島民,而謂島眾必不可用;兵逃於久頓,而謂亂生於無兵;餉糜於漏邑,而謂功銷於無餉。亂視熒聽,浸淫相欺,馴至極壞,不可復挽,臣竊危之。自二年以來,以察去弊,而弊愈多;以威創頑,而威滋殫。是亦反申、商以歸周、孔,捐苛細以崇惇大之時矣。

帝不懌,摘「葛藤」、「株連」數語,令具陳。道周上言曰:

邇年諸臣所目營心計,無一實為朝廷者。其用人行事,不過推求報復而已。自前歲春月以後,盛談邊疆,實非為陛下邊疆,乃為逆璫而翻邊疆也;去歲春月以後,盛言科場,實非為陛下科場,乃為仇隙而翻科場也。此非所謂「葛藤」、「株連」乎?自古外患未弭,則大臣一心以憂外患;小人未退,則大臣一心以憂小人。今獨以遺君父,而大臣自處於催科比較之末。行事而事失,則曰事不可為;用人而人失,則曰人不足用。此臣所謂舛也。三十年來,釀成門戶之禍,今又取縉紳稍有器識者,舉網投阱,即緩急安得一士之用乎!凡絕餌而去者,必非鰌魚;戀棧而來者,必非駿馬。以利祿豢士,則所豢者必嗜利之臣;以箠楚驅人,則就驅者必駑駘之骨。今諸臣之才具心術,陛下其知之矣。知其為小人而又以小人矯之,則小人之焰益張;知其為君子而更以小人參之,則君子之功不立。天下總此人才,不在廊廟則在林藪。臣所知識者有馬如蛟、毛羽健、任贊化,所聞習者有惠世揚、李邦華,在仕籍者有徐良彥、曾櫻、朱大典、陸夢龍、鄒嘉生,皆卓犖駿偉,使當一面,必有可觀。

語皆刺大學士周延儒、溫體仁,帝益不懌,斥為民。

九年用薦召,復故官。明年閏月,久旱修省,道周上言:「近者中外齋宿,為百姓請命,而五日內繫兩尚書,未聞有人申一疏者。安望其戡亂除凶,贊平明之治乎?陛下焦勞於上,小民展轉於下,而諸臣括囊其間,稍有人心,宜不至此。」又上疏曰:「陛下寬仁弘宥,有身任重寄至七八載罔效、擁權自若者。積漸以來,國無是非,朝無枉直,中外臣工率茍且圖事,誠可痛憤。然其視聽一系於上。上急催科則下急賄賂;上樂鍥核,則下樂巉險;上喜告訐,則下喜誣陷。當此南北交訌,奈何與市井細民,申勃谿之談,修睚眥之隙乎。」時體仁方招奸人構東林、復社之獄,故道周及之。

旋進右諭德,掌司經局,疏辭。因言己有三罪、四恥、七不如。三罪、四恥,以自責。七不如者,謂「品行高峻,卓絕倫表,不如劉宗周;至性奇情,無愧純孝,不如倪元璐;湛深大慮,遠見深計,不如魏呈潤;犯言敢諫,清裁絕俗,不如詹爾選、吳執御;志尚高雅,博學多通,不如華亭布衣陳繼儒、龍溪舉人張燮;至圜土累系之臣,朴心純行,不如李汝璨、傅朝佑;文章意氣,坎坷磊落,不如錢謙益、鄭鄤。」鄤方被杖母大詬,帝得疏駭異,責以顛倒是非。道周疏辯,語復營護鄤。帝怒,嚴旨切責。

道周以文章風節高天下,嚴冷方剛,不諧流俗。公卿多畏而忌之,乃藉不如鄤語為口實。其冬,擇東宮講官。體仁已罷,張至發當國,擯道周不與。其同官項煜、楊廷麟不平,上疏推讓道周。至發言:「鄤杖母,明旨煌煌,道周自謂不如,安可為元良輔導。」道周遂移疾乞休,不許。

十一年二月,帝御經筵。刑部尚書鄭三俊方下吏,講官黃景昉救之,帝未許。而帝適追論舊講官姚希孟嘗請漕儲全折以為非。道周聽未審,謂帝將寬三俊念希孟也,因言:「故輔臣文震孟一生蹇直,未蒙帷蓋恩。天下士,生如三俊,歿如震孟、希孟,求其影似,未可多得。」帝以所對失實,責令回奏。再奏再詰,至三奏乃已。凡道周所建白,未嘗得一俞旨,道周顧言不已。

六月,廷推閣臣。道周已充日講官,遷少詹事,得與名。帝不用,用楊嗣昌等五人。道周乃草三疏,一劾嗣昌,一劾陳新甲,一劾遼撫方一藻,同日上之。其劾嗣昌,謂:

天下無無父之子,亦無不臣之子。衛開方不省其親,管仲至比之豭狗;李定不喪繼母,宋世共指為人梟。今遂有不持兩服,坐司馬堂如楊嗣昌者。宣大督臣盧象昇以父殯在途,搥心飲血,請就近推補,乃忽有并推在籍守制之旨。夫守制者可推,則聞喪者可不去;聞喪者可不去,則為子者可不父,為臣者可不子。即使人才甚乏,奈何使不忠不孝者連苞引蘗,種其不祥以穢天下乎?嗣昌在事二年,張網溢地之談,款市樂天之說,才智亦可睹矣,更起一不祥之人,與之表裏。陛下孝治天下,縉紳家庭小小勃谿,猶以法治之,而冒喪斁倫,獨謂無禁,臣竊以為不可也。

其論新甲,言:

其守制不終,走邪徑,託捷足。天下即甚無才,未宜假借及此。古有忠臣孝子無濟於艱難者,決未有不忠不孝而可進乎功名道德之門者也。臣二十躬耕,手足胼胝,以養二人。四十餘削籍,徒步荷擔二千里,不解CS屨。今雖踰五十,非有妻子之奉,婢僕之累。天下即無人,臣願解清華,出管鎖鑰,何必使被棘負塗者,祓不祥以玷王化哉!

其論一藻,則力詆和議之非。帝疑道周以不用怨望,而「縉紳」、「勃谿」語,欲為鄭鄤脫罪,下吏部行譴。嗣昌因上言:「鄤杖母,禽獸不如。今道周又不如鄤,且其意徒欲庇凶徒,飾前言之謬,立心可知。」因自乞罷免,帝優旨慰之。

七月五日,召內閣及諸大臣於平臺,并及道周。帝與諸臣語所司事,久之,問道周曰:「凡無所為而為者,謂之天理;有所為而為者,謂之人欲。爾三疏適當廷推不用時,果無所為乎?」道周對曰:「臣三疏皆為國家綱常,自信無所為。」帝曰:「先時何不言?」對曰:「先時猶可不言,至簡用後不言,更無當言之日。」帝曰:「清固美德,但不可傲物遂非。且惟伯夷為聖之清,若小廉曲謹,是廉,非清也。」時道周所對不合指,帝屢駁,道周復進曰:「惟孝弟之人始能經綸天下,發育萬物。不孝不弟者,根本既無,安有枝葉。」嗣昌出奏曰:「臣不生空桑,豈不知父母?顧念君為臣綱,父為子綱,君臣固在父子前。況古為列國之君臣,可去此適彼;今則一統之君臣,無所逃於天地之間。且仁不遺親,義不後君,難以偏重。臣四疏力辭,意詞臣中有如劉定之、羅倫者,抗疏為臣代請,得遂臣志。及抵都門,聞道周人品學術為人宗師,乃不如鄭鄤。」帝曰:「然,朕正擬問之。」乃問道周曰:「古人心無所為,今則各有所主,故孟子欲正人心,息邪說。古之邪說,別為一教,今則直附於聖賢經傳中,係世道人心更大。且爾言不如鄭鄤,何也?」對曰:「匡章見棄通國,孟子不失禮貌,臣言文章不如鄤。」帝曰:「章子不得於父,豈鄤杖母者比。爾言不如,豈非朋比?」道周曰:「眾惡必察。」帝曰:「陳新甲何以走邪徑,託捷足?且爾言軟美容悅,叩首折枝者誰耶?」道周不能對,但曰:「人心邪則行徑皆邪。」帝曰:「喪固凶禮,豈遭凶者即凶人,盡不祥之人?」道周曰:「古三年喪,君命不過其門。自謂凶與不祥,故軍禮鑿凶門而出。奪情在疆外則可,朝中則不可。」帝曰:「人既可用,何分內外?」道周曰:「我朝自羅倫論奪情,前後五十餘人,多在邊疆。故嗣昌在邊疆則可,在中樞則不可;在中樞猶可,在政府則不可。止嗣昌一人猶可,又呼朋引類,竟成一奪情世界,益不可。」帝又詰問久之。帝曰:「少正卯當時亦稱聞人,心逆而險,行僻而堅,言偽而辨,順非而澤,記醜而博,不免聖人之誅。今人多類此。」道周曰:「少正卯心術不正,臣心正無一毫私。」帝怒。有間,命出候旨。道周曰:「臣今日不盡言,臣負陛下;陛下今日殺臣,陛下負臣。」帝曰:「爾一生學問,止成佞耳!」叱之退。道周叩首起,復跪奏:「臣敢將忠佞二字剖析言之。夫人在君父前,獨立敢言為佞,豈在君父前讒諂面諛為忠耶?忠佞不別,邪正淆矣,何以致治?」帝曰:「固也,非朕漫加爾以佞。但所問在此,所答在彼,非佞而何?」再叱之退。顧嗣昌曰:「甚矣,人心偷薄也。道周恣肆如此,其能無正乎?」乃召文武諸臣,咸聆戒諭而退。

是時,帝憂兵事,謂可屬大事者惟嗣昌,破格用之。道周守經,失帝意,及奏對,又不遜。帝怒甚,欲加以重罪,憚其名高,未敢決。會劉同升、趙士春亦劾嗣昌,將予重譴,而部擬道周譴顧輕。嗣昌懼道周輕,則論己者將無已時也,亟購人劾道周者。有刑部主事張若麒謀改兵部,遂阿嗣昌意上疏曰:「臣聞人主之尊,尊無二上;人臣無將,將而必誅。今黃道周及其徒黨造作語言,虧損聖德。舉古今未有之好語盡出道周,無不可歸過於君父。不頒示前日召對始末,背公死黨之徒,鼓煽以惑四方,私記以疑後世,掩聖天子正人心息邪說至意,大不便。」帝即傳諭廷臣,毋為道周劫持相朋黨,凡數百言。貶道周六秩,為江西按察司照磨,而若麒果得兵部。

久之,江西巡撫解學龍薦所部官,推獎道周備至。故事,但下所司,帝亦不覆閱。而大學士魏照乘惡道周甚,則擬旨責學龍濫薦。帝遂發怒,立削二人籍,逮下刑部獄,責以黨邪亂政,並杖八十,究黨與。詞連編修黃文煥、吏部主事陳天定、工部司務董養河、中書舍人文震亨,並繫獄。戶部主事葉廷秀、監生塗仲吉救之,亦繫獄。尚書李覺斯讞輕,嚴旨切責,再擬謫戍煙瘴,帝猶以為失出,除覺斯名,移獄鎮撫司掠治,乃還刑部獄。逾年,尚書劉澤深等言:「二人罪至永戍止矣,過此惟論死。論死非封疆則貪酷,未有以建言者。道周無封疆貪酷之罪,而有建言蒙戮之名,於道周得矣,非我聖主覆載之量也。陛下所疑者黨耳,黨者,見諸行事。道周抗疏,只託空言,一二知交相從罷斥,烏睹所謂黨,而煩朝廷大法乎?且陛下豈有積恨道周,萬一聖意轉圜,而臣已論定,悔之何及。」仍以原擬請,乃永戍廣西。

十五年八月,道周戍已經年。一日,帝召五輔臣入文華後殿,手一編從容問曰:「張溥、張采何如人也?」皆對曰:「讀書好學人也。」帝曰:「張溥已死,張采小臣,科道官何亟稱之?」對曰:「其胸中自有書,科道官以其用未竟而惜之。」帝曰:「亦不免偏。」時延儒自以嗣昌既已前死矣,而己方再入相,欲參用公議,為道周地也,即對曰:「張溥、黃道周皆未免偏,徒以其善學,故人人惜之。」帝默然。德璟曰:「道周前日蒙戍,上恩寬大,獨其家貧子幼,其實可憫。」帝微笑,演曰:「其事親亦極孝。」行甡曰:「道周學無不通,且極清苦。」帝不答,但微笑而已。明日傳旨復故官。道周在途疏謝,稱學龍、廷秀賢。既還,帝召見道周,道周見帝而泣:「臣不自意今復得見陛下,臣故有犬馬之疾。」請假,許之。

居久之,福王監國,用道周吏部左侍郎。道周不欲出,馬士英諷之曰:「人望在公,公不起,欲從史可法擁立潞王耶?」乃不得已趨朝。陳進取九策,拜禮部尚書,協理詹事府事。而朝政日非,大臣相繼去國,識者知其將亡矣。明年三月,遣祭告禹陵。瀕行,陳進取策,時不能用。甫竣事,南都亡,見唐王聿鍵於衢州,奉表勸進。王以道周為武英殿大學士。道周學行高,王敬禮之特甚,賜宴。鄭芝龍爵通侯,位道周上,眾議抑芝龍,文武由是不和。一諸生上書詆道周迂,不可居相位,王知出芝龍意,下督學御史撻之。

當是時,國勢衰,政歸鄭氏,大帥恃恩觀望,不肯一出關募兵。道周請自往江西圖恢復。以七月啟行,所至遠近響應,得義旅九千餘人,由廣信出衢州。十二月進至婺源,遇大清兵。戰敗,被執至江寧,幽別室中,囚服著書。臨刑,過東華門,坐不起,曰:「此與高皇帝陵寢近,可死矣。」監刑者從之。幕下士中書賴雍、蔡紹謹,兵部主事趙士超等皆死。

道周學貫古今,所至學者雲集。銅山在孤島中,有石室,道周自幼坐臥其中,故學者稱為石齋先生。精天文曆數皇極諸書,所著《易象正》、《三易洞璣》及《太函經》,學者窮年不能通其說,而道周用以推驗治亂。歿後,家人得其小冊,自謂終於丙戌,年六十二,始信其能知來也。

葉廷秀,濮州人。天啟五年進士。歷知南樂、衡水、獲鹿三縣,入為順天府推官。英國公張惟賢與民爭田,廷秀斷歸之民。惟賢屬御史袁弘勛駁勘,執如初。惟賢訴諸朝,帝卒用廷秀奏,還田於民。

崇禎中,遷南京戶部主事,遭內外艱。服闋,入都,未補官,疏陳吏治之弊,言:「催科一事,正供外有雜派,新增外有暗加,額辦外有貼助,小民破產傾家,安得不為盜賊。夫欲救州縣之弊,當自監司郡守始。不澄其源,流安能潔。乃保舉之令行已數年,而稱職者希覯,是連坐法不可不嚴也。」帝納之,授戶部主事。帝以傅永淳為吏部尚書。廷秀言永淳庸才,不當任統均。甫四月,永淳果敗。道周逮下獄,廷秀抗疏救之。帝怒,杖百,繫詔獄。明年冬,遣戍福建。

廷秀受業劉宗周門,造詣淵邃,宗周門人以廷秀為首。與道周未相識,冒死論救,獲重罪,處之恬然。及道周釋還,給事中左懋第、御史李悅心復相繼論薦,執政亦稱其賢,道周在途又為請。帝令所司核議,已而執政復薦。十六年冬,特旨起故官。會都城陷,未赴。福王時,兵部侍郎解學龍薦道周,並及廷秀,命以僉都御史用。及還朝,馬士英惡之,抑授光祿少卿。南都覆,唐王召拜左僉都御史,進兵部右侍郎。事敗,為僧以終。

贊曰:劉宗周、黃道周所指陳,深中時弊。其論才守,別忠佞,足為萬世龜鑒。而聽者迂而遠之,則救時濟變之說惑之也。《傳》曰:「雖危起居,竟信其志,猶將不忘百姓之病也」,二臣有焉。殺身成仁,不違其素,所守豈不卓哉!


\end{pinyinscope}