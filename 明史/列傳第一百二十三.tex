\article{列傳第一百二十三}

\begin{pinyinscope}
王汝訓餘懋學張養蒙孟一脈何士晉陸大受張庭李俸王德完蔣允儀鄒維璉吳羽文

王汝訓,字古師,聊城人。隆慶五年進士。除元城知縣。萬曆初,入為刑部主事。改兵部,累遷光祿少卿。吏科都給事中海寧陳與郊者,大學士王錫爵門生,又附申時行,恣甚。汝訓抗疏數其罪,言:「與郊今日薦巡撫,明日薦監司。每疏一出,受賄狼籍。部曹吳正志一發其奸,身投荒徼。吏部尚書楊巍亦嘗語侍郎趙煥,謂為小人。乞速罷譴。且科道以言為職,乃默默者顯,諤諤者絀。直犯乘輿,屢荷優容。稍涉當途,旋遭擯斥。言官不難於批鱗,而難於借劍,此何為也?天下惟公足以服人。今言者不論是非,被言者不論邪正,模棱兩可,曲事調停,而曰務存大體。是懲議論之紛紜,而反致政體之決裂也。乞特敕吏部,自後遷轉科道,毋惡異喜同,毋好諛醜正。」是時,巍以政府故,方厚與郊。聞汝訓言引己且刺之,大恚,言:「臣未嘗詆與郊。汝訓以寺臣攻言路,正決裂政體之大者。」乃調汝訓南京。頃之,御史王明復劾與郊,并及巍,詔奪明俸,擢與郊太常少卿。都人為之語曰:「欲京堂,須彈章。」與郊尋以憂去。後御史張應揚追劾其交通文選郎劉希孟,考選納賄,並免官。未幾,其子殺人論死,與郊悒悒卒。

汝訓入為太常少卿。孟秋饗廟,帝不親行。汝訓極諫。帝慍甚,以其言直,不罪也。尋進太僕卿,調光祿。汝訓先為少卿,寺中歲費二十萬,至是濫增四萬有奇。汝訓據《會典》,請盡裁內府冗食,不許。

二十二年,改左僉都御史。旋進右副都御史,巡撫浙江。汝訓性清介,方嚴疾惡。巡按御史南昌彭應參亦雅以強直名,相與力鋤豪右。烏程故尚書董份、祭酒范應期里居不法,汝訓將繩之。適應參行部至,應期怨家千人遮道陳牒。應參持之急,檄烏程知縣張應望按之。應期自縊死,其妻吳氏詣闕愬冤。帝命逮應參、應望詔獄,革汝訓職,詰吏部都察院任用非人。尚書孫丕揚、都御史衷貞吉等引罪,且論救。帝意未釋,謫救應參者給事中喬胤等於外。言官訟汝訓、應參,亦及胤,帝愈怒。疏入,輒重胤譴,至除名,而謫應望戍煙瘴,應參為民。

汝訓家居十五年,起南京刑部右侍郎。召改工部,署部事。初,礦稅興,以助大工為名。後悉輸內帑,不以供營繕。而四方採木之需多至千萬,費益不訾。汝訓屢請發帑佐工,皆不報。在部歲餘,力清夙弊。中官請乞,輒執奏不予,節冗費數萬。卒,贈工部尚書,謚恭介。

餘懋學,字行之,婺源人。隆慶二年進士。授撫州推官,擢南京戶科給事中。萬曆初,張居正當國,進《白燕白蓮頌》。懋學以帝方憂旱,下詔罪己,與百官圖修禳。而居正顧獻瑞,非大臣誼,抗疏論之。已,論南京守備太監申信不法,帝為罷信。久之,陳崇惇大、親謇諤、慎名器、戒紛更、防佞諛五事。時居正方務綜核,而懋學疏與之忤,斥為民,永不敘錄。居正死,起懋學故官,奏奪成國公朱希忠王爵,請召還光祿少卿岳相、給事中魏時亮等十八人。帝俱報可。尋擢南京尚寶卿。

十三年,御史李植、江東之等以言事忤執政。同官蔡系周、孫愈賢希執政指,紛然攻訐,懋學上言:

諸臣之不能容植等,一則以科場不能無私,而惡植等之訐發;一則以往者常保留居正,而忌吳中行、沈思孝等之召用。二疑交於中,故百妒發於外也。夫威福自上,則主勢尊。植等三臣,陛下所親擢者也,乃舉朝臣工百計排之;假令政府欲用一人,諸臣敢力挫之乎?臣謹以臣工之十蠹為陛下言之。

今執政大臣,一政之善,輒矜贊導之功,一事之失,輒諉挽回之難,是為誣上。其蠹一。進用一人,執政則曰我所注意也,冢宰則曰我所推轂也,選郎則曰我所登用也。受爵公朝,拜恩私室,是為招權。其蠹二。陛下天縱聖明,猶虛懷納諫。乃二三大僚,稍有規正,輒奮袂而起,惡聲相加,是為諱疾。其蠹三。中外臣工,率探政府意向,而不恤公論。論人則毀譽視其愛憎,行政則舉置徇其喜怒,是為承望。其蠹四。君子立身,和而不同。今當路意有所主,則群相附和,敢於抗天子,而難於違大臣,是為雷同。其蠹五。我國家諫無專官,今他曹稍有建白,不曰出位,則曰沽名,沮忠直之心,長壅蔽之漸,是為阻抑。其蠹六。自張居正蒙蔽主聰,道路以目,今餘風未殄,欺罔日滋。如潘季馴之斥,大快人心,而猶累牘連章為之申雪,是為欺罔。其蠹七。近中外臣僚或大臣交攻,或言官相訐,始以自用之私,終之好勝之習。好勝不已,必致忿爭,忿爭不已,必致黨比。唐之牛、李,宋之洛、蜀,其初豈不由一言之相失哉?是為競勝。其蠹八。佞諛成風,日以浸甚。言及大臣,則等之伊、傅;言及邊帥,則擬以方、召;言及中官,則誇呂、張復出;言及外吏,則頌卓、魯重生。非藉結歡,即因邀賂,是為佞諛。其蠹九。國家設官,各有常職。近兩京大臣,務建白以為名高,侵職掌而聽民訟。長告訐之風,失具瞻之體,是為乖戾。其蠹十也。

懋學夙以直節著稱,其摘季馴不無過當。然所言好勝之弊,必成朋黨,後果如其言。累遷南京戶部右侍郎,總理漕儲。疏白程任卿、江時之冤,二人遂得釋。二十二年,以拾遺論罷。卒,贈工部尚書。天啟初,追謚恭穆。

張養蒙,字泰亨,澤州人。萬曆五年進士。選庶吉士,歷吏科左給事中。少負才名,明習天下事。居言職,慷慨好建白。以南北多水旱,條上治奸民、恤流民、愛富民三事,帝嘉納之。錦衣都指揮羅秀營僉書,兵部尚書王遴格不行,失歡權要而去,秀竟夤緣得之。養蒙疏發其狀,事具遴傳。御史高維崧等言事被謫,養蒙偕同官論救,復特疏訟之。忤旨,奪俸。

尋遷工科都給事中。都御史潘季馴奏報河工,養蒙上言曰:「二十年來,河幾告患矣。當其決,隨議塞,當其淤,隨議濬,事竣輒論功。夫淤決則委之天災而不任其咎,濬塞則歸之人事而共蒙其賞。及報成未久,懼有後虞,急求謝事,而繼者復告患矣。其故皆由不久任也。夫官不久任,其弊有三:後先異時也,人己異見也,功罪難執也。請仿邊臣例,增秩久任,斯職守專而可責成功。」帝深然之。

有詔潞安進綢二千四百匹。未幾,復命增五千。養蒙率同官力爭,且曰:「從來傳奉職造,具題者內臣,擬旨者閣臣,抄發者科臣。今徑下部,非祖制。」不從。出為河南右參政。尋召為太僕少卿,四遷左副都御史。二十四年,極諫時政闕失,言:

邇來殿廷希御,上下不交。或疑外臣不可盡信,或疑外事未可盡從。君臣相猜,政事積廢。致市猾得以揣意旨,左右得以播威權。惟利是聞,禍將胡底。謹以三輕二重之弊為陛下陳之。

一、部院之體漸輕。或虛其位而不補,或用其人而不任。如冬官一曹,亞卿專署,已為異事,乃冢宰何官,數月虛位。法司議劉世延罪,竟爾留中,主事劉冠南疏入即發。何小臣聽而大臣不聽,單疏下而公疏不下哉!以至戶曹三疏諫開礦,臣院九疏催行取,皆置不報。議大事則十疏而九不行,遇廷推則十人而九不用。失大臣師表百僚,奈何輕之至此!

一、科道之職漸輕。五科都給事中久虛不補,御史曹學程一繫不釋,考選臺諫,屢請屢格,乃至服闋補任,亦皆廢閣。是不慾言路之充也。夫政無缺失,何憚人言。徒使唯諾風成,謇諤意絕,國是將何定乎?

一、撫按之任漸輕。如開礦一事,撫按有言,咸蒙切責。於是鄭一麟以千戶而妄劾李盛春。夫閽人、武弁得以制巡撫之命,紀綱不倒置乎?一璫得志,諸璫效尤,撫按斂手,何有於監司?從此陛下之赤子將無人拊循矣。

一、進獻之途漸重。下僚捐俸,儒士獻資,名為助工,實懷覬幸。甚者百戶王守仁以謀復世爵,妄構楚府,而使陛下恩薄於懿親;主簿張以述以求復舊秩,妄獻白鹿,而使陛下德損於玩物。部臣糾之不聽,言官糾之不聽,業已明示好惡,大開受獻之門。將見媚子宵人,投袂競起,今日獻靈瑞,明日貢珍奇,究使敗節文官、僨軍武帥,憑藉錢神,邀求故物,不至如嘉靖末年之濁亂不止也。

一、內差之勢漸重。中使紛然四出,乞請之章無日不上,批答之旨無言不溫。左右藉武弁以營差,武弁藉左右以網利,共構狂言,誑惑天聽。陛下方厭外臣沮撓,謂欲辦家事,必賴家奴,於是有言無不立聽。豈武弁皆急君,而朝紳盡誤國乎?今奸宄實繁有徒。採礦不已,必及採珠;皇店不止,漸及皇莊。繼而營市舶,繼而復鎮守,內可以謀坐營,外可以謀監軍。正德敝風,其鑒不遠。

凡此三輕二重,勢每相因。德與財不並立,中與外不兩勝,惟陛下早見而速圖之。

不報。

又明年六月,兩宮三殿繼災。養蒙復上疏曰:

近日之災,前古未有。自非君臣交儆,痛革敝風,恐虛文相謾,大禍必至。臣請陛下躬謁郊廟以謝嚴譴,立御便殿以通物情,早建國本以繫人心。停銀礦、皇店之役,杜四海亂階;減宦官宮妾之刑,弭蕭牆隱禍。然此皆應天實事,猶非應天實心也。罪己不如正己,格事不如格心。陛下平日成心有四:一曰好逸。朝享倦於躬臨,章奏倦於省覽。古帝王乾健不息,似不如此。一曰好疑。疑及近侍,則左右莫必其生;疑及外庭,則僚採不安於位。究且謀以疑敗,奸以疑容。古帝王至誠馭物,似不如此。一曰好勝。奮厲威嚴以震群工,喜諂諛而惡鯁直,厭封駮而樂順從。古帝王予違汝弼,似不如此。一曰好貨。以聚斂為奉公,以投獻為盡節。古帝王四海為家,似不如此。願陛下戒此四者,亟圖更張,庶天意可回,國祚可保。

帝亦不省。

尋遷戶部右侍郎。時再用師朝鮮,命養蒙督餉。事寧,予一子官。三十年,尚書陳蕖稱疾乞罷。詔養蒙署事。會養蒙亦有疾在告,固辭。給事中夏子陽劾其托疾,遂罷歸。卒於家。天啟初,賜謚毅敏。

孟一脈,字淑孔,東阿人。隆慶五年進士。為平遙知縣。以廉能擢南京御史。萬曆六年五月,上言:「近上兩宮徽號,覃恩內外,獨御史傅應禎、進士鄒元標、部郎艾穆、沈思孝,投荒萬里,遠絕親闈,非所以廣錫類溥仁施也。」疏入,忤張居正,黜為民。居正死,起故官,疏陳五事,言:

近再選宮女至九十七人,急徵一時,輦下甚擾。一也。

中外章奏,宜下部臣議覆,閣臣擬旨,脫有不當,臺諫得糾駁之。今乃不任臣工,顓取宸斷,明旨一出,臣下莫敢犯顏。二也。

士習邪正,繫世道污隆。今廉恥日喪,營求茍且。亟宜更化救弊,先實行而後才華。三也。

東南財賦之區,靡於淫巧,民力竭矣,非陛下有以倡之乎?數年以來,御用不給。今日取之光祿,明日取之太僕,浮梁之磁,南海之珠,玩好之奇,器用之巧,日新月異。遇聖節則有壽服,元宵則有燈服,端陽則有五毒吉服,年例則有歲進龍服。以至覃恩錫賚,小大畢沾;謁陵犒賜,耗費巨萬。錙銖取之,泥沙用之。於是民間習為麗侈,窮耳目之好,竭工藝之能,不知紀極。夫中人得十金,即足供終歲之用。今一物而常兼中人數家之產。或刻沉檀,鏤犀象,以珠寶金玉飾之。周鼎、商彞、秦鉈、漢鑑,皆搜求於海內。窮歲月之力,專一器之工;罄生平之資,取一盼之適。殊不知財賄易盡,嗜欲無窮。陛下誠能恭儉節約以先天下,禁彼浮淫,還之貞樸,則財用自裕,而風俗亦淳。四也。

邊疆之臣,日弛戎備,上下蒙蔽,莫以實聞。由邊臣相繼為本兵,題覆處分,盡在其口。言出而中傷隨之,誰肯為無益之談,自取禍敗哉?漁夫舍餌以得魚,未聞以餌養魚者也。今以中國之文帛綺繡為蕃戎常服,雖曰貢市,實則媚之。邊臣假貢市以賂戎,戎人肆剽竊而要我。彼此相欺,以誑君父。幸其不來,來則莫禦。所謂以餌養魚者也。請明詔樞臣,洗心易慮。戰守之備,一一請求,付之邊臣。使將識敵情,兵識將意,庶乎臂指如意,國可無虞。五也。

疏入,忤旨,謫建昌推官。屢遷南京右通政。移疾歸。四十一年,起右僉都御史,巡撫南、贛。居三年,廷推左副都御史。未得命,給事中官應震論其縱子驕恣。疏雖留中,一脈竟引疾去。年八十一卒。

一脈初以直諫著聲。晚膺節鉞,年力已衰,不克有所表樹云。

何士晉,字武莪,宜興人。父其孝,得士晉晚。族子利其資,結黨致之死。繼母吳氏匿士晉外家。讀書稍懈,母輒示以父血衣。士晉感厲,與人言,未嘗有笑容。萬曆二十六年舉進士。持血衣愬之官,罪人皆抵法。初授寧波推官,擢工科給事中。首疏請通章奏、緩聚斂。俄言:「袞職有闕,廷臣言雖逆耳,每荷優容。獨論及輔臣,必欲借主威以洩憤。是陛下負拒諫之名,輔臣收固寵之實,天下所以積憤輔臣而不能平也。如孫幰、郭子章、戴耀、沈子木,宜舍不舍,公論乖違,輔臣賡安得不任其咎?」無何,劾左都督王之楨久掌錦衣,為內閣爪牙,中樞心腹。又劾大學士王錫爵逢君賊善,召命宜停;戶部尚書趙世卿誤國,無大臣體。已,復言:「朝端大政,宜及今早行者,在放輔臣以清政地,罷大臣被論者以伸公議。斥王之楨以絕禍源,釋卞孔時、王邦才等以蘇冤獄。」

初,皇長孫生,有詔起廢,列上二百餘人。閱三年,止用顧憲成等四人。士晉請大起廢籍。瑞王將婚,詔典禮視福王,費當十九萬。初,帝弟潞王婚費不及其半,士晉請視潞王。帝將崇奉太后,詔建靈應宮,士晉以非禮力爭,且曰:「聖母所注念者東宮出講,諸王早婚,與遺賢之登進也,乃諸臣屢請不應。而不時內降者,非中貴之營求,即鬼神之香火,何也?」帝皆不省。

未幾,有張差梃擊之事。王之寀鉤得差供,帝遷延不決,士晉三上疏趣之。當是時,變起非常,中外咸疑謀出鄭國泰,然無敢直犯其鋒者。郎中陸大受稍及之,國泰大懼,急出揭自明,人言益籍籍。士晉乃抗疏曰:

陛下與東宮,情親父子,勢共安危,豈有禍逼蕭牆,不少動念者?候命踰期,旁疑轉棘。竊詳大受之疏,未嘗實指國泰主謀,何張皇自疑乃爾?因其自疑,人益不能無疑,然人之疑國泰,不自今日始也。陛下試問國泰,三王之議何由起?《閨範》之序何由進?妖書之毒何由構?此基禍之疑也。孟養浩等何由杖?戴士衡等何由戍?王德完等何由錮?此挑激之疑也。南宗順,刑餘也,而陰募死士千人,謂何?順義王,外寇也,而各宮門守以重兵,謂何?王曰乾,逆徒也,而疏中先有龐保、劉成名姓,謂何?此不軌之疑也。三者積疑至今日,忽有張差一事,正與往者舉措相符,安得令人不疑!且今日之疑國泰,又非張差一事已也。恐騎虎難下,駭鹿走險,一擊不效,別有陰謀。陛下不急護東宮,則東宮為孤注。萬一東宮失護,而陛下又轉為孤注矣。

國泰欲釋人疑,惟明告貴妃,力求陛下速執保、成下吏。如果國泰主謀,是乾坤之大逆,九廟之罪人,非但貴妃不能庇,即陛下亦不能庇也。借劍尚方,請自臣始。或別有主謀,無與國泰事,請令國泰自任,凡皇太子、皇長孫起居悉屬國泰保護,稍有疏虞,罪即坐之,則臣與在廷諸臣亦願陛下保全國泰身,無替恩禮。若國泰畏有連引,預熒惑聖聰,久稽廷訊,或潛散黨與,俾之遠逃,或陰斃張差,以冀滅口,則罪愈不容誅矣。惟聖明裁察。

疏入,帝大怒,欲罪之,念事已有跡,恐益致人言。而吏部先以士晉為東林黨,擬出為浙江僉事,候命三年未下。至是,帝急簡部疏,命如前擬。吏部言闕官已補,請改命。帝不許,命調前補者。吏部又以士晉積資已深,秩當參議。帝怒,切責尚書,奪郎中以下俸。士晉之官四年,移廣西參議。光宗立,擢尚寶少卿,遷太僕。

天啟二年,以右僉都御史巡撫廣西。安南入犯,督將吏屢擊卻之。四年,擢兵部右侍郎,總督兩廣軍務,兼巡撫廣東。明年四月,魏忠賢大熾,爭梃擊者率獲罪。御史田景新希旨,誣叛臣安邦彥賄士晉十萬金,阻援兵。遂除士晉名,徵賄助餉。士晉憤鬱而卒。有司徵贓急,家人但輸數百金,產已罄。會莊烈帝立,獲免,復官賜恤。

陸大受,字凝遠,武進人。萬歷三十五年進士。授行人,屢遷戶部郎中。福王將之國,詔賜莊田四萬頃。大受請大減田額,因劾鄭國泰驕恣亂法狀,疏留中。王之寀發張差事,大受抗疏言:「青宮何地,張差何人,敢白晝持梃,直犯儲蹕,此乾坤何等時耶!業承一內官,何以不知其名?業承一大第,何以不知其所?彼三老、三太互相表裏,而霸州武舉高順寧者,今皆匿於何地?奈何不嚴竟而速斷耶?」戶部主事蒲州張庭者,大受同年生也,亦上言:「奸人突入大內,狙擊青宮,陛下宜何如震怒,立窮主謀。乃廷臣交章,一無批答,何也?君側藏奸,上下蒙蔽,皆由陛下精神偏注,皇太子召見甚稀,而前此冊立、選婚及近時東宮出講、郭妃卜葬諸事,陛下皆弗勝遲回,強而後可。彼宦寺者安得不妄生測度,陰蓄不逞,以僥倖於萬一哉!」皆不報。大受尋出為撫州知府,以清潔著聞。居二年,徐紹吉、韓浚以京察奪其官。庭再遷郎中,被齮齕。引退,抑鬱以死。

又有聞喜李俸者,為刑部郎中。當諸司會鞫時,張差語涉逆謀,郎中胡士相等相顧不敢錄。俸力爭,乃得入獄詞,遂為鄭氏黨所惡。及遷鳳翔知府,諸黨人以言懾之,竟不敢之任。後復中以京察,卒於家。

天啟初,御史張慎言、方震孺、魏光緒、楊新期交章訟三人冤。乃贈庭、俸光祿寺少卿,大受起補韶州。已,都御史高攀龍請加庭、俸廕謚,不果。大受未幾卒。

王德完,字子醇,廣安人。萬曆十四年進士。選庶吉士,改兵科給事中。西陲失事,德完言:「諸邊歲糜餉數百萬,而士氣日衰,戎備日廢者,以三蠹未除,二策未審也。何為三蠹?一曰欺,邊吏罔上也。二曰徇,市賞增額也。三曰虛,邊防鮮實也。何謂二策?有目前之策,有經久之策。謹守誓盟,茍免搏噬,此計在目前。大修戰具,令賊不敢窺邊,則百年可保無事,此計在經久。今經略鄭洛主款,巡撫葉夢熊又言戰,邊臣不協,安望成功。」帝為飭二臣。石星為本兵,德完上十議以規時,帝納之。已,請裁李成梁父子權,劾褫黔國公沐昌祚冠服,罷巡撫朱孟震、賈待問、郭四維、少卿楊四知、趙卿。又發廣東總督劉繼文、總兵官李棟等冒功罪。半歲章數十上,率軍國大計。

累遷戶科都給事中。上籌畫邊餉議,言:「諸邊歲例,弘、正間止四十三萬,至嘉靖則二百七十餘萬,而今則三百八十餘萬。惟力行節儉,足以補救。蓋耗蠹之弊,外易剔而內難除。宜嚴劾內府諸庫,汰其不急。又加意屯田、鹽法,外開其源,而內節其流,庶幾國用可足。」時弗能用。倭寇久躪朝鮮,再議封貢。德完言:「封則必貢,貢則必市,是沈惟敬誤經略,經略誤總督,總督誤本兵,本兵誤朝廷也。」後封果不成。德完尋以疾歸。

二十八年,起任工科。極陳四川採木、榷稅及播州用兵之患。又言三殿未營,不宜復興玄殿、龍舟之役。皆不報。已,劾湖廣稅使陳奉四大罪。再疏極論,謂奉必激變。奉果為楚人所攻,僅以身免。尋因禱雨言:「今出虎兕以噬群黎,縱盜賊而吞赤子,幽憤沉結,叩訴無從,故雨澤緣天怒而屯,螟丱因人妖而出。願盡撤礦稅之使,釋逮系之臣,省愆贖過,用弭災變。」不報。四川妖人韓應龍奏請榷鹽、採木。尋甸知府蔡如川、趙州知州甘學書以忤稅使被逮。德完皆力爭。復劾山東稅使陳增、畿輔稅使王虎罪。不報。已極陳國計匱乏,言:「近歲寧夏用兵,費百八十餘萬;朝鮮之役,七百八十餘萬;播州之役,二百餘萬。今皇長子及諸王子冊封、冠婚至九百三十四萬,而袍服之費復二百七十餘萬,冗費如此,國何以支?」因請減織造,止營建,亟完殿工,停買珠寶,慎重採辦,大發內帑,語極切至。帝亦不省。

時帝寵鄭貴妃,疏皇后及皇長子。皇長子生母王恭妃幾殆,而皇后亦多疾。左右多竊意后崩,貴妃即正中宮位,其子為太子。中允黃輝,皇長子講官也,從內侍微探得其狀,謂德完曰:「此國家大事,旦夕不測,書之史冊,謂朝廷無人。」德完乃屬輝具草。十月,上疏言:「道路喧傳,謂中宮役使僅數人,伊鬱致疾,阽危弗自保,臣不勝驚疑。宮禁嚴秘,虛實未審。臣即愚昧,決知其不然。第臺諫之官得風聞言事。果中宮不得於陛下以致疾與?則子於父母之怒,當號泣幾諫。果陛下眷遇中宮有加無替歟?則子於父母之謗,當昭雪辨明。衡是兩端,皆難緘默。敢效漢朝袁盎卻坐之議,陳其愚誠。」疏入,帝震怒,立下詔獄拷訊。尚書李戴、御史周盤等連疏論救。忤旨,切責,御史奪俸有差。大學士沈一貫力疾草奏為德完解,帝亦不釋。旋廷杖百,除其名。復傳諭廷臣:「諸臣為皇長子耶?抑為德完耶?如為皇長子,慎無擾瀆。必欲為德完,則再遲冊立一歲。」廷臣乃不復言。然帝自是懼外廷議論,眷禮中宮,始終無間矣。

光宗立,召為太常少卿。俄擢左僉都御史。天啟元年,京師獲間諜,詞連司禮中官盧受。德完請出受南京。

初,德完直聲震天下。及居大僚,持論每與鄒元標等異。楊鎬、李如楨喪師論死,廷臣急欲誅之。德完乃上疏請酌公論,或遣戍立功,或即時正辟,蓋設兩途以俟帝寬之。且因薦順天府丞邵輔忠、通政參議吳殿邦,以兩人嘗力攻李三才也。疏出,果寬鎬等。於是給事中魏大中再疏論之,德完亦力辨。帝為詰責大中,事乃已。德完尋進戶部右侍郎。給事中朱欽相、倪思輝言事獲罪,疏救之。明年,遷左。亡何卒官。其後輔忠、殿邦以黨逆敗,僉為德完惜之。

蔣允儀,字聞韶,宜興人。萬曆四十四年進士。授桐鄉知縣,移嘉興。天啟二年,擢御史。時廣寧已失,熊廷弼、王化貞俱論死,而兵部尚書張鶴鳴如故,糾之者反獲譴。允儀不平,疏詆其同罪佚罰。因言:「近言官稍進苦口,輒見齟齬,遷謫未已,申之戒諭。使諸臣不遵明諭,而引裾折檻以甘斥逐,天下事猶可為也;使諸臣果遵明諭,而箝口結舌以保祿位,天下事尚忍言哉!頃者恒暘不雨,二麥無秋,皇上於宮中祈禱,反得冰雹之災。變不虛生,各以類應。夫以坤維之厚重而震撼於妖孽,以胡眉之丈夫而交關於婦寺,以籍叢煬灶之奸而托之奉公潔己,是皆陰脅陽之徵也。」報聞。鶴鳴既屢被劾,因詆劾者為群奸朋謀,而反與前尚書黃嘉善、崔景榮並以邊功晉宮保。允儀益憤,言:「鶴鳴既以斬級微功邀三次之賞,即當以失地大罪伏不赦之辜。且以七百里之榆關,兼旬而後至,畏縮無丈夫氣,偃蹇無人臣禮。猶且靦顏哆口評經、撫功罪,若身在功罪外者。陛下試問鶴鳴,為本兵,功罪殺於邊臣,今日經、撫俱論辟,鶴鳴應得何罪?又問鶴鳴,舊日經、撫俱論辟,嘉善、景榮應得何罪?赫然震怒,論究如法,庶封疆不致破壞。」帝不用。

會議紅丸事,力詆方從哲,請盡奪官階、祿廕。其黨惡之。徐州舊設參將,山東盜熾,以允儀請,改設總兵。尋疏論四川監司周著、林宰、徐如珂等功,請優敘。而劾總督張我續退縮,請罷斥。不從。踰月,請杜傳宣、慎爵賞、免立枷、除苛政。且言:「向者丁巳之察,凡抗論國本繫籍正人者,莫不巧加羅織。陰邪盛而陽氣傷,致有今日之禍。今計期已迫,願當事者早伐邪謀,亟培善類。」疏入,魏忠賢、劉朝輩皆不悅。以丁巳主察之人不指名直奏,責令置對。允儀言:「丁巳主察者鄭繼之、李志也,考功科道則趙士諤、徐紹吉、韓浚也。當日八法之處分,臺省之例轉,大僚之拾遺,黑白顛倒,私意橫行。凡抗論建籓,催請之國,保護先帝,有功國本者,靡不痛加摧抑;必欲敗其名,錮其身,盡其倫類而後快。於是方從哲獨居政府,亓詩教、趙興邦等分部要津。凡疆圉重臣,皆賄賂請托而得,如李維翰、楊鎬、熊廷弼、李如柏、如楨,何一不出其保舉?迨封疆破壞,囹圄充塞,而此輩宴然無恙。臣所以痛心遼事,追恨前此當軸之人也。」中旨將重譴允儀,以大學士葉向高言,停俸半歲。

已,復因災祲上言:「內降當停,內操當罷。陵工束手,非所以展孝思;直臣久廢,非所以光聖德。東南杼柚已空,重以屢次之加派;金吾冒濫已極,加以非分之襲封。聖心一轉移,天下無不順應。區區修禳虛文,安能格上穹哉!」帝不能用。

巡按陜西,條上籌邊八事。太常少卿王紹徽家居,與里人馮從吾不協。允儀重從吾,薄紹徽。魏忠賢擢紹徽佐都察院用事。五年,允儀還朝,即出為湖廣副使。其冬又使給事中蘇兆先劾其為門戶渠魁,遂削籍。崇禎元年,薦起御史,言:「奸黨王紹徽創《點將錄》,獻之逆奄。其後效之者有《同志》、《天監》、《盜柄》諸錄,清流遂芟刈無遺。乞加削奪,為傾陷忠良之戒。」從之。其冬,掌河南道事,陳計吏八則。明年,佐都御史曹於汴,大計京官,貶黜者二百餘人,坐不謹者百人,仕路為清。尋擢太僕少卿。

四年六月,以右僉都御史撫治鄖陽。諸府標兵止五百,餉六千,不及一大郡監司。且承平久,人不知兵,而屬城率庳薄,無守具。六年,流賊將窺湖廣。兵部令移鎮襄陽,鄖陽益虛。其冬,賊大至,陷鄖西上津。明年,陷房縣、保康。允儀兵少,不能禦,上章乞援,且請罪。會賊入川,鄖得少緩。中官陳大金與左良玉來援,副使徐景麟見其多攜婦女,疑為賊,用炮擊之,士馬多死。大金怒,訴諸朝,命逮景麟,責允儀陳狀。已而并逮允儀下獄,戍邊,而以盧象昇代。十五年,御史楊爾銘、給事中倪仁禎相繼論薦,未及用而卒。

鄒維璉,字德輝,江西新昌人。萬歷三十五年進士。授延平推官。耿介有大節。巡撫袁一驥以私憾摭布政竇子偁罪,維璉以去就爭。監司欲為一驥建生祠,維璉抗詞力阻。行取,授南京兵部主事,進員外郎。遼左用兵,疏陳數事。尋以憂去。

天啟三年,起官職方,進郎中。刑部主事譚謙益薦妖人宋明時能役神兵復遼左地,魏忠賢陰主之。維璉極言其妖妄。忠賢怒,矯旨譙責。海內方用師,將帥悉賄進,職方尤冗穢。維璉素清嚴,請寄皆絕,因極論債帥之弊,譏切中官、大臣。

吏部尚書趙南星知其賢,調為稽勳郎中。時言路橫恣,凡用吏部郎,必咨其同鄉居言路者。給事中傅櫆、陳良訓、章允儒以南星不先咨己,大怒,共詬誶維璉。及維璉調考功,櫆等益怒,交章力攻。又以江西有吳羽文,例不當用,兩人迫羽文去,以窘辱維璉。維璉憤,拜疏求罷,即日出城。疏中以章惇攻蘇軾、蔡京逐司馬光為言,櫆等愈怒。櫆遂顯攻魏大中、左光斗以及維璉。自是朝端水火,諸賢益不安其位矣。維璉欲去不得,詔留視事。乃嚴核官評,無少假借。

楊漣劾魏忠賢,被旨切責。維璉抗疏曰:「忠賢大姦大惡,罄竹難書。陛下憐其小信小忠,不忍割棄。豈知罪惡既盈,即不忍不可得。漢張讓、趙忠,靈帝以父母稱之;唐田令孜,僖宗亦以阿父稱之;我朝王振、曹吉祥、劉瑾,亦嘗寵之群臣之上。有一人老死牖下,獲保富貴哉?今陛下以太阿授忠賢,非所以為宗社計,亦非所以為忠賢計也。若夫黃扉元老,九列巨卿,安可自處於商輅、劉健、韓文下?」疏入,責其瀆奏。崔呈秀坐贓被劾,維璉論戍邊。諸媚璫者力別其是非,請託,拒不聽,諸逆黨交憾。及趙南星去國,維璉願與俱去,忠賢即放歸。無何,張訥劾南星,追論維璉調部非法,詔削籍。復構入汪文言獄,下吏,戍貴州。

崇禎初,起南京通政參議,就遷太僕少卿,疏陳卜相、久任、納言、議謚、籌兵五事。五年二月,擢右僉都御史,代熊文燦巡撫福建。海寇劉香亂,遣遊擊鄭芝龍擊破之。海外紅夷據彭湖,挾互市,後徙臺灣,漸泊廈門。維璉屢檄芝龍防遏之,不聽。明年夏,芝龍剿賊福寧,紅夷乘間襲陷廈門城,大掠。維璉急發兵水陸進,芝龍亦馳援,焚其三舟,官軍傷亦眾。寇乃泛舟大洋,轉掠青港、荊嶼、石灣。諸將禦之銅山,連戰數日,始敗去。維璉在事二年,勞績甚著。會當國者溫體仁輩雅忌維璉,而閩人宦京師者騰謗於朝,竟坐是罷官。八年春,敘卻賊功,詔許起用。旋召拜兵部右侍郎,遘疾不赴,卒於家。

吳羽文既謝病歸,至崇禎六年始復出。歷考功文選郎中。帝以積疑吏部有私,選郎十一人譴黜大半,遷者三人而已。羽文痛絕諸弊,數與溫體仁牴牾。賊毀皇陵,有詔肆赦。體仁令刑部尚書馮英以逆案入詔內。羽文執止之,而議起錢龍錫、李邦華等。偵事者誣羽文納二人賕,下獄。羽文用高鳳翔為大名知府。鳳翔故嘗坐小罰,言者復謂其徇私,坐謫戍。侍郎吳甡等交薦,復官,未赴卒。羽文,字長卿,南昌人。萬曆四十一年進士。

贊曰:王汝訓諸人建言,挺謇諤之節,洊歷卿貳,不隕厥問。餘懋學之言十蠹,有以哉!鄒維璉抗魏奄,拒逆黨,僅坐謫戍,幸矣。


\end{pinyinscope}