\article{神宗本紀}

\begin{pinyinscope}
神宗範天合道哲肅敦簡光文章武安仁止孝顯皇帝,諱翊鈞,穆宗第三子也。母貴妃李氏。隆慶二年,立為皇太子,時方六歲。性岐嶷,穆宗嘗馳馬宮中,諫曰:「陛下天下主,獨騎而騁,寧無銜橛憂。」穆宗喜,下馬勞之。陳皇后病居別宮,每晨隨貴妃候起居。后聞履聲輒喜,為強起。取經書問之,無不響答,貴妃亦喜。由是兩宮益和。

六年五月,穆宗崩。六月乙卯朔,日有食之。甲子,即皇帝位。以明年為萬曆元年,詔赦天下。祀建文朝盡節諸臣於鄉學,有苗裔者恤錄。庚午,罷高拱。丁丑,高儀卒。壬午,禮部尚書呂調陽兼文淵閣大學士,預機務。秋七月丁亥,初通漕運於密雲。庚寅,察京官。己亥,戒諭廷臣,詔曰:「近歲以來,士習澆漓,官方刓缺,詆老成為無用,矜便佞為有才。遂使朝廷威福之柄,徒為人臣報復之資。用是薄示懲戒,餘皆曲貸。諸臣宜祓除前愆,共維新政。若溺於故習,背公徇私,獲罪祖宗,朕不敢赦。」庚子,尊皇后曰仁聖皇太后,貴妃曰慈聖皇太后。八月戊午,祀大社大稷。九月甲午,葬莊皇帝於昭陵。冬十月己未,侍郎王遴、吳百朋、汪道昆分閱邊防。辛酉,停刑。十一月乙未,河工成。十二月辛酉,振榆林、延綏饑。甲戌,以大行未期,罷明年元夕燈火及宮中宴。

萬曆元年春二月癸丑,御經筵。三月丙申,詔內外官舉將材。夏四月乙丑,潮、惠賊平。庚午,旱,諭百官修省。五月甲申,詔內外官慎刑獄。六月壬申,振淮安水災。秋七月,河決徐州。九月癸未,振荊州、承天及濟南災。丙戌,四川都掌蠻平。癸卯,停刑。冬十一月庚辰,命諸司立程限文簿,以防稽緩。十二月己未,振遼東饑。是年,暹羅、琉球入貢。

二年春正月甲午,召見朝覲廉能官於皇極門。二月甲寅,振四川被寇諸縣。三月癸巳,賜孫繼皋等進士及第、出身有差。夏四月丙寅,詔內外官行久任之法。五月辛丑,穆宗神主祔太廟。八月己巳,振山西災。庚午,振淮、揚、徐水災。冬十月甲寅,決囚。丁卯,視朝閱銓選。閏十二月庚寅,詔罷明年元夕燈火。是年,琉球入貢。

三年春正月丁未,享太廟。二月戊寅,祀大社大稷。辛巳,詔南京職務清簡,官不必備。丙申,始命日講官分直記注起居,纂緝章奏,臨朝侍班。夏四月己巳朔,日有食之,既。壬申,書謹天戒、任賢能、親賢臣、遠嬖佞、明賞罰、謹出入、慎起居、節飲食、收放心、存敬畏、納忠言、節財用十二事於座右,以自警。五月庚子,淮、揚大水,詔察二府有司,貪酷老疾者罷之。六月戊辰,浙江海溢。戊寅,命撫、按官,有司賢否一體薦劾,不得偏重甲科。是夏,蘇、松、常、鎮大水。秋八月丙子,禮部侍郎張四維為禮部尚書兼東閣大學士,預機務。丁丑,河決高郵、碭山。戊子,免淮、揚、鳳、徐被水田租。九月戊午,京師地震。冬十月丁卯,地再震,敕群臣修省。戊辰,停刑。十一月乙巳,祀天於南郊。十二月辛未,詔罷明年元夕燈火。是年,安南、琉球、暹羅、土魯番入貢。

四年春正月丁巳,遼東巡按御史劉臺以論張居正逮下獄,削籍。夏五月戊申,祀地於北郊。六月庚辰,復遣內臣督蘇、杭織造。秋七月丁酉,諭吏、戶二部清吏治,蠲逋賦有差,明年漕糧折收十之三。壬寅,遣御史督修江、浙水利。甲辰,修泗州祖陵。辛亥,草灣河工成。八月壬戌,釋奠於先師孔子。是秋,河決崔鎮。冬十月乙亥,振徐州及豐、沛、睢寧、金鄉、魚臺、單、曹七縣水災,蠲租有差。是年,安南、琉球、烏斯藏、土魯番、天方、撒馬兒罕、魯迷、哈密入貢。

五年春正月己酉,詔鳳陽、淮安力舉營田。二月乙丑,振廣西饑。三月乙巳,賜沈懋學等進士及第、出身有差。夏五月癸巳,廣東羅旁瑤平。秋八月癸亥,河復決崔鎮。閏月乙酉朔,日食,陰雲不見。九月己卯,起復張居正。冬十月乙巳,以論張居正奪情,杖編修吳中行、檢討趙用賢、員外郎艾穆、主事沈思孝,罷黜謫戍有差。丁未,杖進士鄒元標,戍邊。十一月癸丑,以星變考察百官。是年,琉球入貢。

六年春正月,築決河堤。二月戊戌,免兗、青、登、萊所屬逋賦。庚子,立皇后王氏。三月甲寅,禮部尚書馬自強兼文淵閣大學士,吏部侍郎申時行兼東閣大學士,預機務。甲子,張居正葬父歸。夏四月乙未,免湖廣、四川逋賦。丙午,詔戶部歲增金花銀二十萬兩。六月乙未,張居正還京師。秋七月乙卯,呂調陽致仕。丙子,詔江北諸府民,年十五以上無田者,官給牛一頭、田五十畝開墾,三年後起科。九月庚午,詔蘇州諸府開墾荒田,六年後起科。辛未,停刑。冬十月辛卯,馬自強卒。十一月辛酉,禮天於南郊。是年,烏斯藏入貢。

七年春正月戊辰,詔毀天下書院。二月己丑,遣使分閱邊防。三月甲子,免淮、揚逋賦。夏五月癸亥,祀地於北郊。六月辛卯,核兩畿、山東、陜西勳戚田賦。秋七月壬子,振蘇、松水災,蠲稅糧。戊午,京師地震。是年,烏斯藏入貢。

八年春二月辛未朔,日有食之。戊子,耕耤田。戊戌,河工成。三月辛亥,奉兩宮皇太后如天壽山謁陵,免所過田租。甲寅,還宮。丁卯,賜張懋修等進士及第、出身有差。夏閏四月庚申,廣西八寨賊平。冬十月辛丑,汰內外冗官。乙巳,振蘇、松、常、鎮饑。十一月丙子,詔度民田。是年,琉球入貢。

九年春正月庚午,敕邊臣備警。辛未,裁諸司冗官。癸酉,土蠻犯錦州,遊擊周之望敗沒。己卯,命翰林官日四人入直。辛巳,裁南京冗官。甲申,遼東總兵官李成梁襲敗土蠻於襖郎兔。三月丙寅,大閱。是月,土蠻犯遼陽,副總兵曹簠禦之,敗績。夏四月丁酉,振山西被災州縣。乙卯,振蘇、松、淮、鳳、徐、宿災。戶部進《萬曆會計錄》。秋八月丁未,揚州大水。九月丁亥,停刑。冬十月己亥,土蠻犯廣寧、義州,李成梁禦卻之。十一月丙戌,振真定、順德、廣平災,免稅糧。是年,裁各省冗官,核徭賦,汰諸司冒濫冗費。琉球、安南、土魯番、天方、撒馬兒罕、魯迷、哈密、烏斯藏入貢。

十年春二月癸巳,順義王俺答卒。丁酉,免天下積年逋賦。三月庚申,杭州兵變,執巡撫吳善言。丁卯,兵部侍郎張佳胤巡撫浙江,討定之。丙子,泰寧衛部長速把亥犯義州,李成梁擊斬之。己卯,倭寇溫州。夏四月戊子朔,諭禮部,令民及時農桑,勿事游惰。甲午,寧夏土軍馬景殺參將許汝繼,巡撫都御史晉應槐討誅之。庚子,以久旱敕修省。五月庚申,免先師孔子及宋儒朱熹、李侗、羅從彥、蔡沈、胡安國、游酢、真德秀、劉子翬,故大學士楊榮後裔賦役有差。庚辰,振畿內饑。六月丁亥朔,日有食之。壬寅,振太原、平陽、潞安饑。乙巳,前禮部尚書潘晟兼武英殿大學士,吏部侍郎餘有丁為禮部尚書兼文淵閣大學士,預機務。晟尋罷。丙午,張居正卒。秋七月庚午,振平、慶、延、臨、鞏饑。九月丙辰,以皇長子生,詔赦天下。甲子,上兩宮皇太后徽號。冬十月丙申,蘇、松大水,蠲振有差。十二月壬辰,太監馮保謫奉御,籍其家。壬寅,復建言諸臣職。是年,免畿內、山西被災稅糧。哈密、烏斯藏入貢。

十一年春正月壬戌,敕嚴邊備。閏二月甲子,俺答子乞慶哈襲封順義王。緬甸寇永昌。乙丑,如天壽山謁九陵,免所過田租。庚午,如西山謁恭讓章皇后、景皇帝陵。辛未,還宮。乙酉,振臨、鞏、平、延、慶五府旱災,免田租。三月甲申,追奪張居正官階。庚子,賜朱國祚等進士及第、出身有差。夏四月丁巳,張四維以憂去。己未,吏部侍郎許國為禮部尚書兼東閣大學士,預機務。甲戌,承天大雨,江溢。是月,廣東羅定兵變。五月,我大清太祖高皇帝起兵征尼堪外蘭,克圖倫城。六月乙丑,振承天、漢陽、鄖陽、襄陽災。秋八月丙辰,免山西被災稅糧。九月甲申,如天壽山謁陵。己丑,還宮。冬十月癸亥,停刑。辛未,河南水災,蠲振有差。十一月己卯朔,日有食之。十二月庚午,慈寧宮災,敕修省。是年,琉球入貢。

十二年春二月丁卯,京師地震。己巳,釋建文諸臣外親謫戍者後裔。三月己亥,減江西燒造瓷器。夏四月乙卯,籍張居正家。丁巳,遊擊將軍劉綎討平隴川賊。五月甲午,京師地震。六月辛亥,以雲南用兵,免稅糧及逋賦。秋八月丙辰,榜張居正罪於天下,家屬戍邊。九月丙戌,奉兩宮皇太后如天壽山謁陵。己丑,作壽宮。辛卯,還宮。冬十月丁巳,停刑。丙寅,免湖廣、山東被災稅糧。十一月己丑,餘有丁卒。十二月甲辰,前禮部侍郎王錫爵為禮部尚書兼文淵閣大學士,吏部侍郎王家屏兼東閣大學士,預機務。癸亥,罷開銀礦。是年,安南、烏斯藏入貢。

十三年春正月辛卯,四川建武所兵變,擊傷總兵沈思學。二月丁未,南京地震。京師自去年八月不雨,至於是月。庚午,大雩。三月甲申,大雩。己丑,李成梁出塞襲把兔兒炒花,大破之。壬辰,減杭州織造及尚衣監料銀。尚寶司少卿徐貞明督治京畿水田。夏四月丙午,大雩。戊申,以旱詔中外理冤抑,釋鳳陽輕犯及禁錮年久罪宗。戊午,步禱於南郊,面諭大學士等曰:「天旱雖由朕不德,亦天下有司貪婪,剝害小民,以致上干天和,今後宜慎選有司。」蠲天下被災田租一年。五月丙戌,雨。六月辛丑,慈寧宮成。壬寅,建武所亂卒伏誅。是月,四川松、茂番作亂。秋八月己酉,京師地震。閏九月戊戌,振淮、鳳災。癸卯,如天壽山閱壽宮。戊申,還宮。庚申,停刑。冬十二月丁卯,汰惜薪司內官冗員。是月,順義王乞慶哈卒。是年,土魯番、烏斯藏入貢。

十四年春二月癸未,嚴外官餽遺。三月戊戌,以旱霾,諭廷臣陳時政。癸卯,禁部曹言事,罷治京畿水田。癸丑,賜唐文獻等進士及第、出身有差。戊午,久旱,敕修省。夏四月癸酉,京師地震。六月癸未,松茂番平。是夏,振直隸、河南、陜西及廣西潯、柳、平樂,廣東瓊山等十二縣饑。山西盜起。秋七月癸卯,振江西災。戊申,敕戶、兵二部撫安災民,嚴保甲。是月,淇縣賊王安聚眾流劫,尋剿平之。九月壬辰,王家屏以憂去。乙卯,停刑。己未,發帑遣使振河南、山東、直隸、陜西、遼東、淮、鳳災。冬十月丙寅,禮部主事盧洪春以疏請謹疾,杖闕下,削籍。十一月癸卯,祀天於南郊。是年,土魯番入貢。

十五年春正月壬辰,發帑振山西、陜西、河南、山東諸宗室。三月乙卯,乞慶哈子撦力克襲封順義王。夏四月,京師旱,大疫。六月戊辰,禁廷臣奢僭。是月,京師大雨。振恤貧民。秋七月,江北蝗,江南大水,山西、陜西、河南、山東旱,河決開封,蠲振有差。八月庚申,以災沴頻仍,敕撫、按官懲貪吏,理冤獄,蠲租、振恤。九月丁亥朔,日當食,陰雲不見。己丑,停刑。冬十月庚申,大學士申時行請發留中章奏。十一月戊子,鄖陽兵噪,巡撫都御史李材罷。是年,哈密、琉球、烏斯藏入貢。

十六年春三月壬辰,詔改《景皇帝實錄》,去郕戾王號,不果行。山西、陜西、河南及南畿、浙江並大饑疫。夏四月,振江北、大名、開封諸府饑。五月,四川建昌番作亂,討平之。乙巳,以軍儲倉火及各省災傷,敕內外官修省。六月庚申,京師地震。甲子,以災傷停減蘇、杭織造。秋七月乙卯,免山東被災夏稅。庚午,定邊臣考績法。八月乙未,詔取太倉銀二十萬充閱陵賞費。九月己未,停刑。庚申,如天壽山閱壽宮。甲子,次石景山觀渾河。乙丑,還宮。庚午,甘肅兵變,巡撫都御史曹子登罷。是月,青海部長他不囊犯西寧,殺副將李魁。冬十一月辛酉,禁章奏浮冗。是年,烏斯藏入貢。

十七年春正月己酉朔,日有食之。丁巳,太湖、宿、松賊劉汝國等作亂,安慶指揮陳越討之,敗死。二月丙申,吳淞指揮陳懋功討平之。三月丙辰,免升授官面謝。自是臨御遂簡。癸亥,雲南永昌兵變。乙丑,賜焦竑等進士及第、出身有差。夏四月己亥,王家屏復入閣。始興妖僧李圓朗作亂,犯南雄,有司討誅之。六月甲申,浙江大風,海溢。己丑,永昌亂卒平。乙巳,南畿、浙江大旱,太湖水涸,發帑金四十萬振之。秋八月壬寅,嚴匿名揭之禁。冬十月癸未,停刑。癸卯,黃河決口工成。十二月己丑,諭諸臣遇事勿得忿爭求勝。是年,安南、烏斯藏入貢。

十八年春正月甲辰朔,召見大學士申時行等於毓德宮,出皇長子見之。夏四月甲申,振湖廣饑。六月己卯,免畿內被災夏稅。甲申,青海部長火落赤犯舊洮州,副總兵李聯芳敗沒。乙酉,更定宗籓事例,始聽無爵者得自便。秋七月庚子朔,日有食之。乙丑,召見閣臣議邊事,命廷臣舉將材。己巳,兵部尚書鄭雒經略陜西四鎮及山西、宣、大邊務。是月,火落赤再犯河州、臨洮,總兵官劉承嗣敗績。八月癸酉,停撦力克市賞。冬十月戊寅,振臨洮被兵軍民。十二月甲申,遣廷臣九人閱邊。是年,安南入貢。

十九年春正月,頃甸寇永昌、騰越。二月乙酉,總兵官尤繼先敗火落赤餘眾於莽剌川。閏三月丁丑,以彗星見,敕修省。己卯以定天下之吉兇,責給事中、御史風聞訕上,各奪俸一年。夏四月丙申,享太廟。是後廟祀皆遣代。五月壬午,四川四哨番作亂,巡撫都御史李尚思討平之。六月壬子,王錫爵歸省。秋七月癸未,諭廷臣,國是紛紜,致大臣爭欲乞身,此後有肆行誣蔑者重治。八月丁酉,免河南被災田賦。九月壬申,許國致仕。甲戌,申時行致仕。丁丑,吏部侍郎趙志皋為禮部尚書,前禮部侍郎張位為吏部侍郎,並兼東閣大學士,預機務。冬十月癸巳,京營軍官嘩於長安門。十二月甲午,詔定戚臣莊田。癸丑,河套部敵犯榆林、延綏,總兵官杜桐敗之。是年,畿內蝗,南畿、浙江大水,蠲振有差。琉球入貢。

二十年春正月丙戌,給事中孟養浩以言建儲杖闕下,削籍。三月戊辰,寧夏致仕副總兵哱拜殺巡撫都御史黨馨、副使石繼芳,據城反。辛未漢儒散佚之作,以供考據之用。章學誠喻之為吃桑葉而不吐,王家屏致仕。壬申,總督軍務兵部尚書魏學曾討寧夏賊。戊寅,賜翁正春等進士及第、出身有差。夏四月甲辰,總兵官李如松提督陜西討賊軍務。甲寅,甘肅巡撫都御史葉夢熊帥師會魏學曾討賊。撦力克擒賊,叩關獻俘,復還二年市賞。五月,倭犯朝鮮,陷王京,朝鮮王李公奔義州求救。六月丁未,諸軍進次寧夏,賊誘河套部入犯,官軍擊卻之。秋七月癸酉,免陜西逋賦。甲戌,副總兵祖承訓帥師援朝鮮,與倭戰於平壤,敗績。甲申,罷三邊總督魏學曾,以葉夢熊代之,尋逮學曾下獄。八月乙巳,兵部右侍郎宋應昌經略備倭軍務。己酉,詔天下督撫舉將材。九月壬申,寧夏賊平。冬十月壬寅,李如松提督薊、遼、保定、山東軍務,充防海禦倭總兵官,救朝鮮。是月,振畿內、浙江、河南被災諸府蠲租有差。十一月戊辰,御午門,受寧夏俘。十二月甲午,以寧夏賊平,告天下。是年,暹羅、土魯番入貢。

二十一年春正月甲戌,李如松攻倭於平壤,克之。辛未,王錫爵還朝。辛巳,詔並封三皇子為王,廷臣力爭,尋報罷。壬午,李如松進攻王京,遇倭於碧蹄館,敗績。二月甲寅,敕勞東征將士。夏四月癸卯,倭棄王京遁。六月丁酉,詔天下每歲夏月錄囚,減釋輕繫,如兩京例。癸卯,倭使小西飛請款。秋七月癸丑,召援朝鮮諸邊鎮兵還。乙卯,慧星見,敕修省。八月丙戌,以災異敕戒內外諸臣修舉實政。冬十月丙申,停刑。十二月丙辰,薊遼總督顧養謙兼理朝鮮事,召宋應昌、李如松還。是年,振江北、湖廣、河南、浙江、山東饑。河南礦賊大起。烏斯藏入貢。

二十二年春正月己亥,詔以各省災傷,山東、河南、徐、淮尤甚,盜賊四起,有司玩愒,朝廷詔令不行。自今以安民弭盜為撫按有司黜陟。二月癸丑,皇長子常洛出閣講學。甲子,遣使振河南,免田租。三月癸卯,詔修國史。夏四月己酉朔,日有食之。五月辛卯,禮部尚書陳于陛、南京禮部尚書沈一貫並兼東閣大學士,預機務。庚子,王錫爵致仕。六月己酉,雷雨,西華門災。敕修省。秋七月丙申,河套部長失兔犯延綏。是月,延綏總兵官麻貴敗河套部敵於下馬關。冬十月己未,南京兵部右侍郎邢玠總督川、貴軍務,討播州宣慰使楊應龍。丁卯,詔倭使入朝。是月,炒花犯遼東,總兵官董一元敗之。是年,琉球、烏斯藏入貢。

二十三年春正月癸卯,遣都督僉事李宗城、指揮楊方亨封平秀吉為日本國王。三月乙未,賜朱之蕃等進士及第、出身有差。夏五月丁酉,京師地震,敕修省。秋九月戊寅,青海部長永邵卜犯甘肅,參將達雲敗之。乙酉,詔復建文年號。冬十一月辛未,湖廣災,蠲振有差。十二月辛丑,大學士趙志皋等請發留中章奏,不報。是年,江北大水,淮溢,浸泗州祖陵。

二十四年春二月戊申,麻貴襲河套部,敗之。三月乙亥,乾清、坤寧兩宮災,敕修省。壬辰,下詔自責。是月,火落赤犯洮河,總兵官劉綎破走之。夏四月己亥,李宗城自倭營奔還王京。五月戊辰,河套部敵犯甘肅,總兵官楊浚擊破之。庚午,復議封倭,命都督僉事楊方亨、遊擊沈惟敬往。六月,振福建饑。秋七月丁卯,吏部尚書孫丕揚請發推補官員章疏,不報。戊寅,仁聖皇太后崩。乙酉,始遣中官開礦於畿內。未幾,河南、山東、山西、浙江、陜西悉令開採,以中官領之。群臣屢諫不聽。閏八月乙丑朔,日有食之。丁卯,大學士趙志皋請視朝,發章奏,罷採礦,不報。九月乙未,楊方亨至日本,平秀吉不受封,復侵朝鮮。乙卯,葬孝安莊皇后。是月,河套部犯寧夏。總兵官李如柏擊敗之。是秋,河決黃堌口。冬十月丙子,停刑。乙酉,始命中官榷稅通州。是後,各省皆設稅使。群臣屢諫不聽。十二月乙亥,陳于陛卒。

二十五年春正月丙辰,朝鮮使來請援。二月丙寅,復議征倭。丙子,前都督同知麻貴為備倭總兵官,統南北諸軍。三月乙巳,山東右參政楊鎬為僉都御史,經略朝鮮軍務。己未,兵部侍郎邢玠為尚書,總督薊、遼、保定軍務,經略禦倭。夏六月戊寅,皇極、中極、建極三殿災。癸未,罷修國史。秋七月癸巳,誡諭群臣。丁酉,詔赦天下。是月,楊應龍叛,掠合江、綦江。八月丁丑,倭破朝鮮閑山,遂薄南原,副總兵楊元棄城走,倭逼王京。甲申,京師地震。九月壬辰,逮前兵部尚書石星下獄,論死。冬十月甲戌,安南黎惟潭篡立,款關請罪,詔授安南都統使。是年,琉球入貢。

二十六年春正月,官軍攻倭於蔚山,不克,楊鎬、麻貴奔王京。三月癸卯,賜趙秉忠等進士及第、出身有差。壬子,群臣詣文華門疏請皇長子冠婚,不允。夏四月丁卯,遼東總兵官李如松出塞,遇伏戰死。壬申,京師旱,敕修省。六月丁巳,楊鎬罷。戊午,中官李敬採珠廣東。丙寅,張位罷。丙子,巡撫天津僉都御史萬世德經略朝鮮。秋七月丙戌,中官魯保鬻兩淮餘鹽。八月丁丑,京師地震。九月壬辰,免浙江被災田租。冬十月乙卯,總兵官劉綎、麻貴分道擊倭,敗之。董一元攻倭新寨,敗績。十一月戊戌,倭棄蔚山遁,官軍分道進擊。十二月,總兵官陳璘破倭於乙山,朝鮮平。是年,烏斯藏入貢。

二十七年春二月壬子,分遣中官領浙江、福建、廣東市舶司。是月,貴州巡撫江東之遣兵討楊應龍,敗績。三月己亥,前兵部侍郎李化龍總督川、湖、貴州軍務,討楊應龍。夏四月甲戌,御午門,受倭俘。是月,臨清民變,焚稅使馬堂暑,殺其參隨三十四人。閏月丙戌,以倭平,詔天下,除東征加派田賦。己丑,久旱,敕修省。丙申,以諸皇子婚,詔取太倉銀二千四百萬兩。戶部告匱,命嚴核天下積儲。六月己亥,楊應龍陷綦江,參將房嘉寵、遊擊張良賢戰死。秋八月甲午,陜西狄道縣山崩。九月,土蠻犯錦州。

冬十月壬午,振京城饑民。丙戌,以播州用兵,加四川、湖廣田賦。戊子,貴州宣慰使安疆臣有罪,詔討賊自贖。十一月己酉,免河南被災田租。癸酉,振畿輔及鳳陽等處饑。十二月丁丑,武昌、漢陽民變,擊傷稅使陳奉。戊子,振京師就食流民。是年,琉球入貢。

二十八年春二月戊寅,京師地震。丙戌,李化龍帥師分八路進討播州。夏六月丁丑,克海龍囤,楊應龍自縊死,播州平。秋七月辛亥,旱,敕修省。八月辛未,慈慶宮成。丙子,罷朝鮮戍兵。九月甲寅,停刑。是秋,炒花犯遼東,副總兵解生等敗沒。冬十月辛未,貴州皮林苗叛,總兵官陳璘討之。丙子,雲南稅監楊榮開採阿瓦、孟密寶井。十二月乙未,御午門,受播州俘。是年,兩畿各省災傷,民饑盜起,內外群臣交章請罷礦稅諸監,皆不聽。大西洋利瑪竇進方物。

二十九年春正月壬子,以播州平,詔天下,蠲四川、貴州、湖廣、雲南加派田租逋賦,除官民詿誤罪。是月,皮林苗賊平。二月甲戌,振大同、宣府饑。三月乙卯,賜張以誠等進士及第、出身有差。是月,武昌民變,殺稅監陳奉參隨六人,焚巡撫公署。夏四月乙酉,征陳奉還,以守備承天中官杜茂代之。五月,蘇州民變,殺織造中官孫隆參隨數人。六月,京師自去年六月不雨,至是月乙亥始雨。山東、山西、河南皆大旱。丁亥,法司請熱審,不報。是夏,振畿內饑。秋九月壬寅,河決開封、歸德。丁未,趙志皋卒。癸丑,振貴州饑。戊午,前禮部尚書沈鯉、朱賡並兼東閣大學士,預機務。冬十月己卯,立皇長子常洛為皇太子,封諸子常洵福王,常浩瑞王,常潤惠王,常瀛桂王。詔赦天下。壬辰,加上慈聖皇太后尊號。十二月辛未,詔復朵顏馬市。是年,琉球入貢。

三十年春正月己未,以四方災異敕修省。二月己卯,不豫,召大學士沈一貫於啟祥宮,命罷礦稅,停織造,釋逮擊,復建言諸臣職。翼日,疾瘳,寢前詔。甲申,重建乾清、坤寧宮。閏月丙申,復河套諸部貢市。戊午,河州黃河竭。三月甲申,騰越民變,殺稅監委官。夏四月辛丑,振順天、永平饑。五月乙亥,法司請熱審,不報。秋七月辛巳,邊餉缺,命嚴催積逋。是月,緬賊陷蠻莫宣撫司,宣撫思正奔騰越,賊追至,有司殺正以謝賊,始解。冬十月戊戌,振江北災。丙辰,停刑。是年,琉球、哈密入貢。

三十一年春三月戊午,吏部奏天下郡守闕員,不報。是月,播州餘賊吳洪等作亂,有司討平之。夏四月丁亥朔,日有食之。五月丙辰,閣臣請熱審,不報。戊寅,京師地震。鳳陽大雨雹,毀皇陵殿脊。是夏,河決蘇家莊,北浸豐、沛、魚臺、單縣。秋九月甲子,江北盜起。冬十月甲申,停刑。丙申,睢州賊楊思敬作亂,有司討擒之。十一月甲子,獲妖書,言帝欲易太子,詔五城大索。十二月丙戌,召見皇太子於啟祥宮,賜手敕慰諭。

三十二年春二月壬寅,閣臣請補司道郡守及遣巡方御史,不報。三月甲子,乾清宮成。乙丑,賜楊守勤等進士及第、出身有差。夏四月辛巳朔,日有食之。是月,濬泇河工成。五月癸酉,雷火焚長陵明樓。六月丙戌,以陵災,命補闕官恤刑獄。丁酉,昌平大水,壞長、泰、康、昭四陵石梁。秋七月庚戌,京師大雨,壞城垣。辛酉,振被水居民。八月辛丑,群臣伏文華門,疏請修舉實政,降旨切責。丙午,分水河工成。九月戊申,振畿南六府饑。閏月辛丑,武昌宗人蘊鉁等作亂,殺巡撫都御史趙可懷。冬十月甲寅,始敘平播州功。

是年,琉球、烏斯藏入貢。

三十三年春正月,重修京師外城。庚辰,銀定、歹成犯鎮番,總兵官達雲擊敗之。夏四月辛亥,蘊鉁等伏誅。五月丙申,鳳陽大風雨,毀陵殿神座。庚子,雷擊圜丘望燈高桿。六月乙巳,以雷警,敕修省。秋八月己巳,停刑。九月甲午,昭和殿災。丙申,京師地震。

冬十一月辛巳,免淮陽被災田租。十二月壬寅,詔罷天下開礦。以稅務歸有司,歲輸所入之半於內府,半戶、工二部。丙午,免河南被災田租。乙卯,以皇長孫生,詔赦天下。開宗室科舉入仕例。罷採廣東珠池、雲南寶井。

三十四年春二月庚戌,加上皇太后徽號。辛亥,大學士沈鯉、朱賡請補六部大僚,不報。三月己卯,雲南人殺稅監楊榮,焚其屍。丁酉,真定、順德、廣平、大名災,蠲振有差。夏四月癸亥,浚朱旺口河工成。五月癸酉,河套部犯延綏,官軍擊走之。六月癸卯,緬甸陷木邦。是月,畿內大蝗。秋七月癸未,沈一貫、沈鯉致仕。九月甲午,詔陜西嚴敕邊備。冬十月丙申,停刑。十一月己巳,朵顏入犯,總兵官姜顯謨禦卻之。十二月壬子,南京妖賊劉天緒謀反,事覺伏誅。是年,安南、琉球入貢。蒙古喀爾喀諸部悉歸我大清。

三十五年春正月辛未,給事中翁憲祥言,撫、按官解任宜候命,不宜聽其自去,不報。二月戊戌,安南賊武德成犯雲南,總兵官沐睿禦卻之。三月辛巳,賜黃士俊等進士及第、出身有差。夏四月戊戌,銀定、歹成犯涼州,副總兵柴國柱擊走之。壬子,順義王撦力克卒。五月戊子,前禮部尚書于慎行及禮部侍郎李廷機、南京吏部侍郎葉向高並禮部尚書兼東閣大學士,預機務。六月,湖廣及徽、寧、太平、嚴州大水。閏月辛巳,復河套諸部貢市。秋七月庚子,京師久雨。刑部請發熱審疏,不報。八月丙寅,振畿內饑。九月甲午,停刑。冬十月癸酉,山東旱饑,蠲振有差。十一月壬子,于慎行卒。十二月,金沙江蠻阿克叛,陷武定,攻圍雲南,別陷嵩明、祿豐。安南賊犯欽州。是年,琉球入貢。

三十六年春正月,河南、江北饑。二月戊辰,京師地震。夏六月己卯,南畿大水。秋七月丁酉,京師地震。郴州礦賊起。八月癸亥,治雲南失事諸臣罪,巡撫都御史陳用賓、總兵官沐睿下獄,論死。庚辰,振南畿及嘉興、湖州饑。九月甲午,四川巡撫都御史喬璧星奏擒阿克於東川,賊平。冬十一月壬子,硃賡卒。十二月戊午,再振南畿,免稅糧。是年,琉球入貢。

三十七年春三月辛卯,拱兔陷大勝堡,遊擊于守志戰於小凌河,敗績。己酉,大學士葉向高請發群臣相攻諸疏,公論是非,以肅人心,不報。夏四月,倭寇溫州。秋九月癸卯,左都御史詹沂封印自去。丁未,停刑。是秋,福建、浙江、江西大水。湖廣、四川、河南、陜西、山西旱。畿內、山東、徐州蝗。冬十二月己巳,留畿內、山東諸省稅銀三分之一振饑民。徐州賊殺如皋知縣張籓。是年,日本入琉球,執其國王尚寧。哈密入貢。

三十八年春三月癸巳,賜韓敬等進士及第、出身有差。夏四月丁丑,正陽門樓災。辛卯,以旱災異常,諭群臣各修職業,勿彼此攻訐。辛丑,振畿內、山東、山西、河南、陜西、福建、四川饑。五月,河南賊陳自管等作亂,有司討擒之。冬十月辛丑,停刑。十一月壬寅朔,日有食之。丁卯,以軍乏餉,諭廷臣陳足國長策,不得請發內帑。是年,烏斯藏入貢。

三十九年春二月庚子,河套部敵犯甘州之紅崖、青湖,官軍禦卻之。夏四月,京師旱。戊子,怡神殿災。丙申,設邊鎮常平倉。五月壬寅,御史徐兆魁疏劾東林講學諸人陰持計典,自是諸臣益相攻擊。廣西、廣東大水。六月,自徐州北至京師大水。是夏,停熱審。冬十月丁卯,戶部尚書趙世卿拜疏自去。甲申,停刑。閣臣請釋輕犯,不報。是年,暹羅入貢。

四十年春二月癸未,吏部尚書孫丕揚拜疏自去。三月丙午,振京師流民。夏四月丙寅,南京各道御史言:「臺省空虛,諸務廢墮,上深居二十餘年,未嘗一接見大臣,天下將有陸沈之憂。」不報。五月甲午朔,日有食之。秋八月,河決徐州。九月庚戌,李廷機拜疏自去。冬十月甲申,停刑。是年,琉球中山王尚寧遣使報歸國。

四十一年春正月庚申,諭朝鮮練兵防倭。三月癸酉,賜周延儒等進士及第、出身有差。夏五月己巳,諭吏部都察院:「年來議論混淆,朝廷優容不問,遂益妄言排陷,致大臣疑畏,皆欲求去,甚傷國體。自今仍有結黨亂政者,罪不宥。」六月乙未,卜失兔襲封順義王。秋七月甲子,兵部尚書掌都察院事孫瑋拜疏自去。九月壬申,吏部左侍郎方從哲、前吏部左侍郎吳道南並禮部尚書兼東閣大學士,預機務。庚辰,吏部尚書趙煥拜疏自去。是年,兩畿、山東、江西、河南、廣西、湖廣、遼東大水。烏斯藏入貢。

四十二年春正月乙丑,總兵官劉綎討建昌叛蠻,平之。二月辛卯,慈聖皇太后崩。己酉,振畿內饑。三月丙子,福王之國。夏四月丙戌,以皇太后遺命赦天下。六月甲午,葬孝定皇后。秋八月甲午,禮部右侍郎孫慎行拜疏自去。癸卯,葉向高致仕。是年,安南、土魯番入貢。

四十三年春正月乙丑,徐州決河工成。三月丁未朔,日有食之。夏五月己酉,薊州男子張差持梃入慈慶宮,擊傷守門內侍,下獄。丁巳,刑部提牢主事王之寀揭言張差獄情,梃擊之案自是起。己巳,嚴皇城門禁。癸酉,召見廷臣於慈寧宮。御史劉光復下獄。甲戌,張差伏誅。六月戊寅,久旱,敕修省。秋七月己酉,振畿內饑。甲戌,停刑。閏八月庚戌,重建三殿。丁巳,山東大旱,詔留稅銀振之。丁卯,河套諸部犯延綏,官軍禦之,敗績,副將孫弘謨被執。冬十月辛酉,京師地震。十一月戊寅,振京師饑民。

四十四年春三月辛未朔,日有食之。乙酉,賜錢上升等進士及第、出身有差。是春,畿內、山東、河南、淮、徐大饑,蠲振有差。夏四月戊午,河南盜起,諭有司撫剿。六月壬寅,河套諸部犯延綏,總兵官杜文煥禦卻之。丁卯,河決祥符朱家口,浸陳、杞、睢、柘諸州縣。秋七月乙未,河套部長吉能犯高家堡,參將王國興敗沒。是月,陜西旱,江西、廣東水,河南、淮、揚、常、鎮蝗,山東盜賊大起。冬十月丁未,停刑。十一月己巳,隆德殿災。

四十五年春二月戊午,以去冬無雪,入春不雨,敕修省。三月辛未,鎮撫司缺官,獄囚久繫多死,大學士方從哲等以請,不報。乙亥,振江西饑。夏五月丙子,久旱,再諭修省。六月丙申,畿南大饑,有司請振,不報。是月,閣臣法司請熱審,不報。秋七月癸亥朔,日有食之。丁卯,吳道南以憂去。是年,兩畿、河南、山東、山西、陜西、江西、湖廣、福建、廣東災。暹羅、烏斯藏入貢。

四十六年春二月乙巳,振廣東饑。夏四月甲辰,大清兵克撫順城,千總王命印死之。庚戌,總兵官張承胤帥師援撫順,敗沒。閏月庚申,楊鎬為兵部左侍郎兼右僉都御史,經略遼東。六月壬午,京師地震。是夏,有司請熱審,不報。秋七月丙午,大清兵克清河堡,守將鄒儲賢、張旆死之。八月壬申,海運餉遼東。庚辰,乃蠻等七部款塞。辛巳,停刑。九月壬辰,遼師乏餉,有司請發各省稅銀,不報。辛亥,加天下田賦。乙卯,京師地震。冬十一月甲午,以災異敕修省。十二月丁巳,河套部長猛克什力來降。是年,土魯番、天方、撒馬兒罕、魯迷、哈密、烏斯藏入貢。

四十七年春二月乙丑,經略楊鎬誓師於遼陽,總兵官李如柏、杜松、劉綎、馬林分道出塞。三月甲早,杜松遇大清兵於吉林崖,戰死。乙酉,馬林兵敗於飛芬山,兵備僉事潘宗顏戰死。庚寅,劉綎兵深入阿布達里岡,戰死。辛丑,賜莊際昌等進士及第、出身有差。夏四月癸酉,盔甲廠災。六月丁卯,大清兵克開原,馬林敗沒。癸酉,大理寺丞熊廷弼為兵部右侍郎兼右僉都御史,經略遼東。甲戌,廷臣伏文華門,請發章奏及增兵發餉,不報。秋八月乙卯,山東蝗。癸亥,逮楊鎬。九月庚辰,停刑。戊子,百官伏闕,請視朝行政,不報。冬十月丁巳,振京師饑民。十二月,再加天下田賦。辛未,鎮江、寬奠、靉陽新募援兵潰。是年,暹羅入貢。

四十八年春正月庚子,朝鮮乞援。三月庚寅,復加天下田賦。夏四月癸丑,皇后王氏崩。戊午,帝不豫,召見方從哲於弘德殿。秋七月壬辰,大漸,召英國公張惟賢,大學士方從哲,尚書周嘉謨、李汝華、黃嘉善、張問達、黃克纘,侍郎孫如游於弘德殿,勉諸臣勤職。丙申,崩,年五十有八。遺詔罷一切榷稅併新增織造諸項。九月甲申,上尊謚,廟號神宗,葬定陵。

◎光宗

光宗崇天契道英睿恭純憲文景武淵仁懿孝貞皇帝,諱常洛,神宗長子也。母恭妃王氏。萬歷十年八月生。神宗御殿受賀,告祭郊廟社稷,頒詔天下,上兩宮徽號。未幾,鄭貴妃生子常洵,有寵。儲位久不定,廷臣交章固請,皆不聽。二十九年十月,乃立為皇太子。

三十一年,獲妖書,言神宗欲易太子,指斥鄭貴妃。神宗怒。捕逮株連者甚眾,最後得皦生光者,磔之。獄乃解。四十一年六月,奸人王曰乾上變,告孔學等為巫蠱,將謀不利於東宮,語連鄭貴妃、福王,事具《葉向高傳》。四十三年夏五月己酉,薊州男子張差持梃入慈慶宮,事復連貴妃內璫。太子請以屬吏。獄具,戮差於市,斃內璫二人於禁中。自是遂有「梃擊」之案。

四十八年七月,神宗崩。丁酉,太子遵遺詔發帑金百萬犒邊。盡罷天下礦稅,起建言得罪諸臣。己亥,再發帑金百萬充邊賞。八月丙午朔,即皇帝位。大赦天下,以明年為泰昌元年。蠲直省被災租賦。己酉,吏部侍郎史繼偕、南京禮部侍郎沈飀為禮部尚書兼東閣大學士,預機務。遼東大旱。庚申,蘭州黃河清,凡三日。甲子,禮部侍郎何宗彥、劉一燝、韓爌為禮部尚書兼東閣大學士,預機務。乙丑,南京禮部尚書朱國祚為禮部尚書兼東閣大學士,預機務。召葉向高。遣使恤刑。丙寅,帝不豫。戊辰,召對英國公張惟賢、大學士方從哲等十有三人於乾清宮,命皇長子出見。甲戌,大漸,復召從哲等受顧命。是日,鴻臚寺官李可灼進紅丸。九月乙亥朔,崩於乾清宮,在位一月,年三十有九。熹宗即位,從廷臣議,改萬歷四十八年八月後為泰昌元年。冬十月,上尊謚,廟號光宗,葬慶陵。

贊曰:神宗沖齡踐阼,江陵秉政,綜核名實,國勢幾於富強。繼乃因循牽制,晏處深宮,綱紀廢弛,君臣否隔。於是小人好權趨利者馳騖追逐,與名節之士為仇讎,門戶紛然角立。馴至悊、愍,邪黨滋蔓。在廷正類無深識遠慮以折其機牙,而不勝忿激,交相攻訐。以致人主蓄疑,賢奸雜用,潰敗決裂,不可振救。故論者謂明之亡,實亡於神宗,豈不諒歟。光宗潛德久彰,海內屬望,而嗣服一月,天不假年,措施未展,三案構爭,黨禍益熾,可哀也夫!


\end{pinyinscope}