\article{儀衛志}

\begin{pinyinscope}
《周官》,王之儀衛分掌於天官、春官、夏官之屬,而蹕事則專屬於秋官。漢朝會,則衛官陳車騎,張旗幟。唐沿隋制,置衛尉卿,掌儀仗帳幕之事。宋衛尉領左右金吾衛司、左右金吾仗司、六軍儀仗司,主清道、徼巡、排列、奉引儀仗。元置拱衛司,領控鶴戶以供其事。歷代制度雖有沿革異同,總以謹出入之防,嚴尊卑之分。慎重則尊嚴,尊嚴則整肅,是故文謂之儀,武謂之衛。天子出,車駕次第謂之鹵簿。而唐制四品以上皆給鹵簿,則君臣並得通稱也。明初詔禮官,鹵簿彌文,務從省節,以示尚質去奢之意。凡正、至、聖節、朝會,及冊拜、接見蕃臣,儀鸞司陳設儀仗。而中宮、東宮、親王皆有儀仗之制。後或隨時增飾,要以洪武創制為準則焉。茲撮《集禮》所載大凡,以備考核。其郡王及皇妃、東宮妃以下儀仗,載在《會典》者,並著於篇云。

皇帝儀仗:吳元年十二月辛酉,中書左相國李善長率禮官以即位禮儀進。是日清晨,拱衛司陳設鹵簿,列甲士於午門外之東西,列旗仗於奉天門外之東西。龍旗十二,分左右,用甲士十二人。北斗旗一、纛一居前,豹尾一居後,俱用甲士三人。虎豹各二,馴象六,分左右。布旗六十四:門旗、日旗、月旗,青龍、白虎、風、雲、雷、雨、江、河、淮、濟旗,天馬、天祿、白澤、朱雀、玄武等旗,木、火、土、金、水五星旗,五嶽旗,熊旗,鸞旗及二十八宿旗,各六行;每旗用甲士五人,一人執旗,四人執弓弩。設五輅於奉天門外:玉輅居中,左金輅,次革輅,右象輅,次木輅,俱並列。丹墀左右布黃麾仗、黃蓋、華蓋、曲蓋、紫方傘、紅方傘、雉扇、朱團扇、羽葆幢、豹尾、龍頭竿、信幡、傳教幡、告止幡、絳引幡、戟氅、戈氅、儀鍠氅等,各三行。丹陛左右陳幢節、響節、金節、燭籠、青龍白虎幢、班劍、吾杖、立瓜、臥瓜、儀刀、鐙杖、戟、骨朵、朱雀玄武幢等,各三行。殿門左右設圓蓋一、金交椅、金腳踏、水盆、水罐、團黃扇、紅扇。皆校尉擎執。

洪武元年十月,定元旦朝賀儀:金吾衛於奉天門外分設旗幟。宿衛於午門外分設兵仗。衛尉寺於奉天殿門及丹陛、丹墀設黃麾仗。內使監擎執於殿上。凡遇冬至、聖節、冊拜、親王及蕃使來朝,儀俱同。其宣詔赦、降香,則惟設奉天殿門及丹陛儀仗、殿上擎執云。其陳布次第,午門外,刀、盾、殳、叉各置於東西,甲士用赤。奉天門外中道,金吾、宿衛二衛設龍旗十二,分左右,用青甲士十二人。北斗旗一、纛一居前,豹尾一居後,俱用黑甲士三人。虎豹各二,馴象六,分左右。左右布旗六十四:左前第一行,門旗二,每旗用紅甲士五人,內一人執旗,旗下四人執弓箭。第二行,月旗一,用白甲士五人,內一人執旗,旗下四人執弩;青龍旗一,用青甲士五人,內一人執旗,旗下四人執弩。第三行,風、雲、雷、雨旗各一,每旗用黑甲士五人,內一人執旗,旗下四人執弓箭;天馬、白澤、朱雀旗各一,每旗用紅甲士五人,內一人執旗,旗下四人執弓箭。第四行,木、火、土、金、水五星旗各一,隨其方色,每旗用甲士五人,內一人執旗,旗下四人執弩,其甲木青、火紅、土黃、金白、水黑、熊旗、鸞旗各一,每旗用紅甲士五人,內一人執旗,旗下四人執弩。第五行角、亢、氐、房、心、尾、箕旗各一,每旗用青甲士五人,內一人執旗,旗下四人執弓箭。第六行斗、牛、女、虛、危、室、壁旗各一,每旗用青甲士五人,內一人執旗,旗下四人執弩。右前第一行,門旗二,每旗用紅甲士五人,內一人執旗,旗下四人執弓箭。第二行,日旗一,用紅甲士五人,內一人執旗,旗下四人執弩;白虎旗一,用白甲士五人,內一人執旗,旗下四人執弩。第三行,江、河、淮、濟旗各一,隨其方色,每旗用甲士五人,內一人執旗,旗下四人執弓箭,其甲江紅、河白、淮青、濟黑;天祿、白澤、玄武旗各一,每旗用甲士五人,內一人執旗,旗下四人執弓箭,天祿、白澤紅甲,玄武黑甲。第四行,東、南、中、西、北五嶽旗各一,隨其方色,每旗用甲士五人,內一人執旗,旗下四人執弩,其甲東嶽青、南嶽紅、中嶽黃、西嶽白、北岳黑;熊旗、麟旗各一,每旗用紅甲士五人,內一人執旗,旗下四人執弩。第五行,奎、婁、胃、昴、畢、觜、參旗各一,每旗用青甲士五人,內一人執旗,旗下四人執弓箭。第六行,井、鬼、柳、星、張、翼、軫旗各一,每旗用青甲士五人,內一人執旗,旗下四人執弩。奉天門外,拱衛司設五輅。玉輅居中;左金輅,次革輅;右象輅,次木輅。俱並列。典牧所設乘馬於文武樓之南,各三,東西相向。丹墀左右布黃麾仗凡九十,分左右,各三行。左前第一行,十五:黃蓋一,紅大傘二,華蓋一,曲蓋一,紫方傘一,紅方傘一,雉扇四,朱團扇四。第二行,十五:羽葆幢二,豹尾二,龍頭竿二,信幡二,傳教幡二,告止幡二,絳引幡二,黃麾一。第三行,十五:戟氅五,戈氅五,儀鍠氅五。右前第一行,十五:黃蓋一,紅大傘二,華蓋一,曲蓋一,紫方傘一,紅方傘一,雉扇四,硃團扇四。第二行,十五:羽葆幢二,豹尾二,龍頭竿二,信幡二,傳教幡二,告止幡二,絳引幡二,黃麾一。第三行,十五:戟氅五,戈氅五,儀鍠氅五。皆校尉擎執。丹陛左右,拱衛司陳幢節等仗九十,分左右,為四行。左前第一行,響節十二,金節三,燭籠三。第二行,青龍幢一,班劍三,吾杖三,立瓜三,臥瓜三,儀刀三,鐙杖三,戟三,骨朵三,朱雀幢一。右前第一行,響節十二,金節三,燭籠三。第二行,白虎幢一,班劍三,吾杖三,立瓜三,臥瓜三,儀刀三,鐙杖三,戟三,骨朵三,玄武幢一。皆校尉擎執。奉天殿門左右,拱衛司陳設:左行,圓蓋一,金腳踏一,金水盆一,團黃扇三,紅扇三;右行,圓蓋一,金交椅一,金水罐一,團黃扇三,紅扇三。皆校尉擎執。殿上左右內使監陳設:左,拂子二,金唾壺一,金香合一;右,拂子二,金唾盂一,金香爐一。皆內使擎執。和聲郎陳樂於丹墀文武官拜位之南,其器數詳見《樂志》內。

三年,命製郊丘祭祀拜褥。郊丘用席表蒲裏為褥,宗廟、社稷、先農、山川用紅文綺表紅木棉布裏為褥。十二年,命禮部增設丹墀儀仗,黃傘、華蓋、曲蓋、紫方傘、紅方傘各二,雉扇、紅團扇各四,羽葆幢、龍頭竿、絳引、傳教、告止、信幡各六,戟氅、戈氅、儀鍠氅各十。

永樂元年,禮部言鹵簿中宜有九龍車一乘,請增置。帝曰:「禮貴得中,過為奢,不及為儉,先朝審之精矣。當遵用舊章,豈可輒有增益,以啟後世之奢哉?九龍車既先朝所無,其仍舊便。」宣德元年,更造鹵簿儀仗,有具服幄殿一座,金交椅一,金腳踏一,金盆一,金罐一,金馬杌一,鞍籠一,金香爐一,金香合一,金唾盂一,金唾壺一,御杖二,擺錫明甲一百副,盔一百,弓一百,箭三千,刀一百。其執事校尉,每人鵝帽,只孫衣,銅帶靴履鞋一副。常朝,各色羅掌扇四十,各色羅絹傘十,萬壽傘一,黃雙龍扇二。筵宴,銷金羅傘四,銷金雨傘四,金龍響節二十四。

皇后儀仗,洪武元年定:丹陛儀仗三十六人:黃麾二,戟五色繡幡六,戈五色繡幡六,鍠五色錦幡六,小雉扇四,紅雜花團扇四,錦曲蓋二,紫方傘二,紅大傘四。丹墀儀仗五十八人:班劍四,金吾杖四,立瓜四,臥瓜四,儀刀四,鐙杖四,骨朵四,斧四,響節十二,錦花蓋二,金交椅一,金腳踏一,金水盆一,金水罐一,方扇八。宮中常用儀衛二十人:內使八人,色繡幡二,金斧二,金骨朵二,金交椅一,金腳踏一;宮女十二人,金水盆一,金水罐一,金香爐一,金香合一,金唾壺一,金唾盂一,拂子二,方扇四。永樂元年增製紅杖一對。太皇太后、皇太后儀仗與皇后同。

皇太子儀仗,洪武元年定:門外中道設龍旗六,其執龍旗者並戎服。黃旗一居中,左前青旗一,右前赤旗一,左後黑旗一,右後白旗一,每旗執弓弩軍士六人,服各隨旗色。殿下設三十六人:絳引幡二,戟氅六,戈氅六,儀鍠氅六,羽葆幢六,青方傘二,青小方扇四,青雜花團扇四,皆校尉擎執。殿前設四十八人:班劍四,吾杖四,立瓜四,臥瓜四,儀刀四,鐙杖四,骨朵四,斧四,響節十二,金節四,皆校尉擎執。殿門設十二人:金交椅一,金腳踏一,金水罐一,金水盆一,青羅團扇六,紅圓蓋二,皆校尉擎執。殿上設六人:金香爐一,香合一,唾盂一,唾壺一,拂子二,皆內使擎執。永樂二年,禮部言,東宮儀仗,有司失紀載,視親王差少,宜增製金香爐、金香合各一,殳二,叉二,傳教、告止、信幡各二,節二,幢二,夾槊二,槊、刀、盾各二十,戟八,紅紙油燈籠六,紅羅銷金邊圓傘、紅羅繡圓傘各一,紅羅曲蓋繡傘、紅羅素圓傘、紅羅素方傘、青羅素方傘各二,紅羅繡孔雀方扇、紅羅繡四季花團扇各四,拂子二,唾盂、唾壺各一,鞍籠一,誕馬八,紅令旗二,清道旗四,幰弩一,白澤旗二,弓箭二十副。從之。

親王儀仗,洪武六年定:宮門外設方色旗二,青色白澤旗二,執人服隨旗色,並戎服。殿下,絳引幡二,戟氅二,戈氅二,儀鍠氅二,皆校尉執。殿前,班劍二,吾杖二,立瓜二,臥瓜二,儀刀二,鐙杖二,骨朵二,斧二,響節八,皆校尉執。殿門,交椅一,腳踏一,水罐一,水盆一,團扇四,蓋二,皆校尉執。殿上,拂子二,香爐一,香合一,唾壺一,唾盂一。十六年詔,親王儀仗內交椅、盆、罐用銀者,悉改用金。建文四年,禮部言,親王儀仗合增紅油絹銷金雨傘一,紅紗燈籠、紅油紙燈籠各四,敔燈二,大小銅角四。從之。永樂三年命工部,親王儀仗內紅銷金傘,仍用寶珠龍文。凡世子儀仗同。

郡王儀仗:令旗二,清道旗二,幰弩一,刀盾十六,弓箭十八副,絳引、傳教、告止、信幡各二,吾杖、儀刀、立瓜、臥瓜、骨朵、斧各二,戟十六,槊十六,麾一,幢一,節一,響節六,紅銷金圓傘一,紅圓傘一,紅曲柄傘二,紅方傘二,青圓扇四,紅圓扇四,誕馬四,鞍籠一,馬杌一,拂子二,交椅一,腳踏一,水盆一,水罐一,香爐一,紅紗燈籠二,敔燈二,帳房一座。

皇妃儀仗:紅杖二,清道旗二,絳引幡二,戈氅、戟氅、儀鍠氅、吾杖、儀刀、班劍、立瓜、臥瓜、鐙杖、骨朵、金鉞各二,響節四,青方傘四,紅繡圓傘一,繡方扇四,紅花圓扇四,青繡圓扇四,交椅一,腳踏一,拂子二,水盆一,水罐一,香爐一,香合一,唾盂一,唾壺一,紅紗燈籠四。

東宮妃儀仗:紅杖二,清道旗二,絳引幡二,儀鍠氅、戈氅、戟氅、吾杖、儀刀、班劍、立瓜、臥瓜、鐙杖、骨朵、金鉞各二,響節四,青方傘二,紅素圓傘二,紅繡圓傘一,紅繡方扇四,紅繡花圓扇四,青繡圓扇四,交椅一,腳踏一,拂子二,水盆一,水罐一,香爐一,香合一,紅紗燈籠四。永樂二年,禮部言,東宮妃儀仗如親王妃,惟香爐、香合如中宮,但亦不用金,其水盆、水罐皆用銀,從之。

親王妃儀仗:紅杖二,清道旗二,絳引幡二,戟氅、吾杖、儀刀、班劍、立瓜、臥瓜、骨朵、鐙杖各二,響節四,青方傘二,紅彩畫雲鳳傘一,青孔雀圓扇四,紅花扇四,交椅一,腳踏一,水盆一,水罐一,紅紗燈籠四,拂子二。公主、世子妃儀仗俱同。

郡王妃儀仗:紅杖二,清道旗二,絳引幡二,戟氅、吾杖、班劍、立瓜、骨朵各二,響節二,青方傘二,紅圓傘一,青圓扇二,紅圓扇二,交椅一,腳踏一,拂子二,紅紗燈籠二,水盆一,水罐一。

郡主儀仗:紅杖二,清道旗二,班劍、吾杖、立瓜、骨朵各二,響節二,青方傘一,紅圓傘一,青圓扇二,紅圓扇二,交椅一,腳踏一,水盆一,水罐一,紅紗燈籠二,拂子二。

舊例,郡王儀仗有交椅、馬杌,皆木質銀裹;水盆、水罐及香爐、香合,皆銀質抹金;量折銀三百二十兩。郡王妃儀仗,有交椅等大器,量折銀一百六十兩。餘皆自備充用。嘉靖四十四年定,除親王及親王妃初封儀仗照例頒給外,其初封郡王及郡王妃折銀等項,並停止。萬歷十年定,郡王初封系帝孫者,儀仗照例全給,系王孫者免。蓋宗室分封漸多,勢難遍給也。


\end{pinyinscope}