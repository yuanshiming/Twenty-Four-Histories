\article{劉基、宋濂列傳}

\begin{pinyinscope}
劉基,字伯溫,青田人。曾祖濠,仕宋為翰林掌書。宋亡,邑子林融倡義旅。事敗,元遣使簿錄其黨,多連染。使道宿濠家,濠醉使者而焚其廬,籍悉毀。使者計無所出,乃為更其籍,連染者皆得免。基幼穎異,其師鄭復初謂其父爚曰:「君祖德厚,此子必大君之門矣。」元至順間,舉進士,除高安丞,有廉直聲。行省闢之,謝去。起為江浙儒學副提舉,論御史失職,為臺臣所阻,再投劾歸。基博通經史,於書無不窺,尤精象緯之學。西蜀趙天澤論江左人物,首稱基,以為諸葛孔明儔也。

方國珍起海上,掠郡縣,有司不能制。行省復辟基為元帥府都事。基議築慶元諸城以逼賊,國珍氣沮。及左丞帖里帖木兒招諭國珍,基言方氏兄弟首亂,不誅無以懲後。國珍懼,厚賂基。基不受。國珍乃使人浮海至京,賄用事者。遂詔撫國珍,授以官,而責基擅威福,羈管紹興,方氏遂愈橫。亡何,山寇蜂起,行省復辟基剿捕,與行院判石抹宜孫守處州。經畧使李國鳳上其功,執政以方氏故抑之,授總管府判,不與兵事。基遂棄官還青田,著《郁離子》以見志。時避方氏者爭依基,基稍為部署,寇不敢犯。

及太祖下金華,定括蒼,聞基及宋濂等名,以幣聘。基未應,總制孫炎再致書固邀之,基始出。既至,陳時務十八策。太祖大喜,築禮賢館以處基等,寵禮甚至。初,太祖以韓林兒稱宋後,遙奉之。歲首,中書省設御座行禮,基獨不拜,曰:「牧豎耳,奉之何為!」因見太祖,陳天命所在。太祖問征取計,基曰:「士誠自守虜,不足慮。友諒劫主脅下,名號不正,地據上流,其心無日忘我,宜先圖之。陳氏滅,張氏勢孤,一舉可定。然後北向中原,王業可成也。」太祖大悅曰:「先生有至計,勿惜盡言。」會陳友諒陷太平,謀東下,勢張甚,諸將或議降,或議奔據鐘山,基張目不言。太祖召入內,基奮曰:「主降及奔者,可斬也。」太祖曰:「先生計安出?」基曰:「賊驕矣,待其深入,伏兵邀取之易耳。天道後舉者勝,取威制敵以成王業,在此舉矣。」太祖用其策,誘友諒至,大破之,以克敵賞賞基。基辭。友諒兵復陷安慶,太祖欲自將討之,以問基。基力贊,遂出師攻安慶。自旦及暮不下,基請逕趨江州,搗友諒巢穴,遂悉軍西上。友諒出不意,帥妻子奔武昌,江州降。其龍興守將胡美遣子通款,請勿散其部曲。太祖有難色。基從後蹋胡牀。太祖悟,許之。美降,江西諸郡皆下。

基喪母,值兵事未敢言,至是請還葬。會苗軍反,殺金、處守將胡大海、耿再成等,浙東搖動。基至衢子弟子曾子撰著。宋代開始由《禮記》中抽出,朱熹以之為,為守將夏毅諭安諸屬邑,復與平章邵榮等謀復處州,亂遂定。國珍素畏基,致書唁。基答書,宣示太祖威德,國珍遂入貢。太祖數以書即家訪軍國事,基條答悉中機宜。尋赴京,太祖方親援安豐。基曰:「漢、吳伺隙,未可動也。」不聽。友諒聞之,乘間圍洪都。太祖曰:「不聽君言,幾失計。」遂自將救洪都,與友諒大戰鄱陽蝴,一日數十接。太祖坐胡床督戰,基侍側,忽躍起大呼,趣太祖更舟。太祖倉卒徙別舸,坐未定,飛礮擊舊所御舟立碎。友諒乘高見之,大喜。而太祖舟更進,漢軍皆失色。時湖中相持,三日未決,基請移軍湖口扼之,以金木相犯日決勝,友諒走死。其後太祖取士誠,北伐中原,遂成帝業,略如基謀。

吳元年以基為太史令,上《戊申大統曆》。熒惑守心,請下詔罪己。大旱,請決滯獄。即命基平反,雨隨注。因請立法定制,以止濫殺。太祖方欲刑人,基請其故,太祖語之以夢。基曰:「此得土得眾之象,宜停刑以待。」後三日,海寧降。太祖喜,悉以囚付基縱之。尋拜御史中丞兼太史令。

太祖即皇帝位,基奏立軍衛法,初定處州稅糧,視宋制畝加五合,惟青田命毋加,曰:「令伯溫鄉里世世為美談也。」帝幸汴梁,基與左丞相善長居守。基謂宋、元寬縱失天下,今宜肅紀綱。令御史糾劾無所避,宿衛宦侍有過者,皆啟皇太子置之法,人憚其嚴。中書省都事李彬坐貪縱抵罪,善長素匿之,請緩其獄。基不聽,馳奏。報可。方祈雨,即斬之。由是與善長忤。帝歸,愬基僇人壇壝下,不敬。諸怨基者亦交譖之。會以旱求言,基奏:「士卒物故者,其妻悉處別營,凡數萬人,陰氣鬱結。工匠死,胔骸暴露,吳將吏降者皆編軍戶,足乾和氣。」帝納其言,旬日仍不雨,帝怒。會基有妻喪,遂請告歸。時帝方營中都,又銳意滅擴廓。基瀕行,奏曰:「鳳陽雖帝鄉,非建都地。王保保未可輕也。」已而定西失利,擴廓竟走沙漠,迄為邊患。其冬,帝手詔敘基勳伐,召赴京,賜賚甚厚,追贈基祖、父皆永嘉郡公。累欲進基爵,基固辭不受。

初,太祖以事責丞相李善長,基言:「善長勳舊,能調和諸將。」太祖曰:「是數欲害君,君乃為之地耶?吾行相君矣。」基頓首曰:「是如易柱會動亂的原因。參見「法學」中的「韓非」。,須得大木。若束小木為之,且立覆。」及善長罷,帝欲相楊憲。憲素善基,基力言不可,曰:「憲有相才無相器。夫宰相者,持心如水,以義理為權衡,而己無與者也,憲則不然。」帝問汪廣洋,曰:「此褊淺殆甚於憲。」又問胡惟庸,曰:「譬之駕,懼其僨轅也。」帝曰:「吾之相,誠無逾先生。」基曰:「臣疾惡太甚,又不耐繁劇,為之且孤上恩。天下何患無才,惟明主悉心求之,目前諸人誠未見其可也。」後憲、廣洋、惟庸皆敗。三年授弘文館學士。十一月大封功臣,授基開國翊運守正文臣、資善大夫、上護軍,封誠意伯,祿二百四十石。明年賜歸老於鄉。

帝嘗手書問天象。基條答甚悉而焚其草。大要言霜雪之後,必有陽春,今國威已立,宜少濟以寬大。基佐定天下,料事如神。性剛嫉惡,與物多忤。至是還隱山中,惟飲酒弈棋,口不言功。邑令求見不得,微服為野人謁基。基方濯足,令從子引入茆舍,炊黍飯令。令告曰:「某青田知縣也。」基驚起稱民,謝去,終不復見。其韜跡如此,然究為惟庸所中。

初,基言甌、括間有隙地曰談洋,南抵閩界,為鹽盜藪,方氏所由亂,請設巡檢司守之。奸民弗便也。會茗洋逃軍反,吏匿不以聞。基令長子璉奏其事,不先白中書省。胡惟庸方以左丞掌省事,挾前憾,使吏訐基,謂談洋地有王氣,基圖為墓,民弗與,則請立巡檢逐民。帝雖不罪基,然頗為所動,遂奪基祿。基懼入謝,乃留京,不敢歸。未幾,惟庸相,基大慼曰:「使吾言不驗,蒼生福也。」憂憤疾作。八年三月,帝親製文賜之,遣使護歸。抵家,疾篤,以《天文書》授子璉曰:「亟上之,毋令後人習也。」又謂次子璟曰:「夫為政,寬猛如循環。當今之務在修德省刑,祈天永命。諸形勝要害之地,宜與京師聲勢連絡。我欲為遺表,惟庸在,無益也。惟庸敗後,上必思我,有所問,以是密奏之。」居一月而卒,年六十五。基在京病時,惟庸以醫來,飲其藥,有物積腹中如拳石。其後中丞塗節首惟庸逆謀,并謂其毒基致死云。

基虯髯,貌修偉,慷慨有大節,論天下安危,義形於色。帝察其至誠,任以心膂。每召基,輒屏人密語移時。基亦自謂不世遇,知無不言。遇急難,勇氣奮發,計畫立定,人莫能測。暇則敷陳王道。帝每恭己以聽,常呼為老先生而不名,曰:「吾子房也。」又曰:「數以孔子之言導予。」顧帷幄語秘莫能詳,而世所傳為神奇,多陰陽風角之說,非其至也。所為文章,氣昌而奇,與宋濂並為一代之宗。所著有《覆瓿集》,《犁眉公集》傳於世。子璉、璟。

璉,字孟藻,有文行。洪武十年授考功監丞,試監察御史,出為江西參政。太祖常欲大用之,為惟庸黨所脅,墮井死。璉子畾,字士端,洪武二十四年三月嗣伯,食祿五百石。初,基爵止及身,至是帝追念基功,又憫基父子皆為惟庸所厄,命增其祿,予世襲。明年坐事貶秩歸里。洪武末,坐事戍甘肅,尋赦還。建文帝及成祖皆欲用之,以奉親守墓力辭。永樂間卒,子法停襲。景泰三年命錄基後,授法曾孫祿世襲《五經》博士。弘治十三年以給事中吳士偉言,乃命祿孫瑜為處州衛指揮使。

正德八年加贈基太師,謚文成。嘉靖十年,刑部郎中李瑜言,基宜侑享高廟,封世爵「此書但可自治,不可示人」,「吾姑書之而姑藏之,以俟夫千,如中山王達。下廷臣議,僉言:「高帝收攬賢豪,一時佐命功臣並軌宣猷。而帷幄奇謀,中原大計,往往屬基,故在軍有子房之稱,剖符發諸葛之喻。基亡之後,孫廌實嗣,太祖召諭再三,鐵券丹書,誓言世祿。畾嗣未幾,旋即隕世,褫圭裳於末裔,委帶礪於空言。或謂後嗣孤貧,弗克負荷;或謂長陵紹統,遂至猜嫌。雖一辱泥塗,傳聞多謬,而載書盟府,績效具存。昔武王興滅,天下歸心,成季無後,君子所歎。基宜侑享太廟,其九世孫瑜宜嗣伯爵,與世襲。」制曰:「可。」瑜卒,孫世延嗣。嘉靖末,南京振武營兵變,世延掌右軍都督府事,撫定之。數上封事,不報,忿而恣橫。萬曆三十四年,坐罪論死,卒。適孫萊臣年幼,庶兄蓋臣借襲。蓋臣卒,萊臣當襲,蓋臣子孔昭復據之。崇禎時,出督南京操江,福王之立,與馬士英、阮大鋮比,後航海不知所終。

璟,字仲璟,基次子,弱冠通諸經。太祖念基,每歲召璟同章溢子允載、葉琛子永道、胡深子伯機究的基矗參見「物理」中的「靜止」。,入見便殿,燕語如家人。洪武二十三年命襲父爵。璟言有長兄子廌在。帝大喜,命廌襲封,以璟為閤門使,且諭之曰:「考宋制,閤門使即儀禮司。朕欲汝日夕左右,以宣達為職,不特禮儀也。」帝臨朝,出侍班,百官奏事有闕遺者,隨時糾正。都御史袁泰奏車牛事失實,帝宥之,泰忘引謝。璟糾之,服罪。帝因諭璟:「凡似此者,即面糾,朕雖不之罪,要令知朝廷綱紀。」己,復令同法司錄獄囚冤滯。谷王就封,擢為左長史。

璟論說英侃,喜談兵。初,溫州賊葉丁香叛,延安侯唐勝宗討之,決策於璟。破賊還一種。但手稿流傳極少,康、雍、乾、嘉時刻本更少。清道,稱璟才略。帝喜曰:「璟真伯溫兒矣。」嘗與成祖弈,成祖曰:「卿不少讓耶?」璟正色曰:「可讓處則讓,不可讓者不敢讓也。」成祖默然。靖難兵起,璟隨谷王歸京師,獻十六策,不聽。令參李景隆軍事。景隆敗,璟夜渡盧溝河,冰裂馬陷,冒雪行三十里。子貊自大同赴難,遇之良鄉,與俱歸。上《聞見錄》,不省,遂歸里。成祖即位,召璟,稱疾不至。逮入京,猶稱殿下。且云:「殿下百世後,逃不得一『篡』字。」下獄,自經死。法官希旨,緣坐其家。成祖以基故,不許。宣德二年授貊刑部照磨。

宋濂,字景濂,其先金華之潛溪人,至濂乃遷浦江。幼英敏強記,就學於聞人夢吉義的中國」的政治路線。詳盡地闡述了在中國現階段黨的一,通《五經》,復往從吳萊學。已,遊柳貫、黃溍之門,兩人皆亟遜濂,自謂弗如。元至正中,薦授翰林編修,以親老辭不行,入龍門山著書。

踰十餘年,太祖取婺州,召見濂。時已改寧越府,命知府王顯宗開郡學,因以濂及葉儀為《五經》師。明年三月,以李善長薦,與劉基、章、溢、葉琛並徵至應天,除江南儒學提舉,命授太子經,尋改起居注。濂長基一歲,皆起東南,負重名。基雄邁有奇氣,而濂自命儒者。基佐軍中謀議,濂亦首用文學受知,恒侍左右,備顧問。嘗召講《春秋左氏傳》,濂進曰:「《春秋》乃孔子褒善貶惡之書,茍能遵行,則賞罰適中,天下可定也。」太祖御端門,口釋黃石公《三略》。濂曰:「《尚書》二《典》、三《謨》,帝王大經大法畢具,願留意講明之。」已,論賞賚,復曰:「得天下以人心為本。人心不固,雖金帛充牣,將焉用之。」太祖悉稱善。乙巳三月,乞歸省。太祖與太子並加勞賜。濂上箋謝,并奉書太子,勉以孝友敬恭、進德修業。太祖覽書大悅,召太子,為語書意,賜札褒答,并令太子致書報焉。尋丁父憂。服除,召還。

洪武二年詔修元史,命充總裁官。是年八月史成,除翰林院學士。明年二月,儒士歐陽佑等採故元元統以後事跡還朝,仍命濂等續修,六越月再成,賜金帛。是月,以失朝參,降編修。四年遷國子司業,坐考祀孔子禮不以時奏,謫安遠知縣,旋召為禮部主事。明年遷贊善大夫。是時,帝留意文治,徵召四方儒士張唯等數十人,擇其年少俊異者,皆擢編修,令入禁中文華堂肄業,命濂為之師。濂傅太子先後十餘年,凡一言動,皆以禮法諷勸,使歸於道,至有關政教及前代興亡事,必拱手曰:「當如是,不當如彼。」皇太子每斂容嘉納,言必稱師父云。

帝剖符封功臣,召濂議五等封爵。宿大本堂,討論達旦,歷據漢、唐故實,量其中而奏之。甘露屢降再沒有其他的東西。宗教中的上帝只是人的自我異化的產物,,帝問災祥之故。對曰:「受命不於天,於其人,休符不於祥,於其仁。《春秋》書異不書祥,為是故也。」皇從子文正得罪,濂曰:「文正固當死,陛下體親親之誼,置諸遠地則善矣。」車駕祀方丘,患心不寧,濂從容言曰:「養心莫善於寡欲,審能行之,則心清而身泰矣。」帝稱善者良久。嘗問以帝王之學,何書為要。濂舉《大學衍義》。乃命大書揭之殿兩廡壁。頃之御西廡,諸大臣皆在,帝指《衍義》中司馬遷論黃、老事,命濂講析。講畢,因曰:「漢武溺方技謬悠之學,改文、景恭儉之風,民力既敝,然後嚴刑督之。人主誠以禮義治心,則邪說不入,以學校治民,則禍亂不興,刑罰非所先也。」問三代曆數及封疆廣狹,既備陳之,復曰:「三代治天下以仁義,故多歷年所。」又問:「三代以上,所讀何書?」對曰:「上古載籍未立,人不專講誦。君人者兼治教之責,率以躬行,則眾自化。」嘗奉制詠鷹,令七舉足即成,有「自古戒禽荒」之言。帝忻然曰:「卿可謂善陳矣。」濂之隨事納忠,皆此類也。

六年七月遷侍講學士,知制誥,同修國史,兼贊善大夫。命與詹同、樂韶鳳修日歷,又與吳伯宗等修寶訓。九月定散官資階在康德倫理學基礎上的道德理想,其目的只在於改進人們的,給濂中順大夫,欲任以政事。辭曰:「臣無他長,待罪禁近足矣。」帝益重之。八年九月,從太子及秦、晉、楚、靖江四王講武中都。帝得輿圖《濠梁古蹟》一卷,遣使賜太子,題其外,令濂詢訪,隨處言之。太子以示濂,因歷歷舉陳,隨事進說,甚有規益。

濂性誠謹,官內庭久,未嘗訐人過。所居室,署「溫樹」。客問禁中語,即指示之。嘗與客飲基本特徵是用科學來反對意識形態,否認意識形態的作用。認,帝密使人偵視。翼日,問濂昨飲酒否,坐客為誰,饌何物。濂具以實對。笑曰:「誠然,卿不朕欺。」間召問群臣臧否,濂惟舉其善者曰:「善者與臣友,臣知之;其不善者,不能知也。」主事茹太素上書萬餘言。帝怒,問廷臣,或指其書曰:「此不敬,此誹謗非法。」問濂,對曰:「彼盡忠於陛下耳。陛下方開言路,惡可深罪。」既而帝覽其書,有足採者。悉召廷臣詰責,因呼濂字曰:「微景濂幾誤罪言者。」於是帝廷譽之曰:「朕聞太上為聖,其次為賢,其次為君子。宋景濂事朕十九年,未嘗有一言之偽,誚一人之短,始終無二,非止君子,抑可謂賢矣。」每燕見,必設坐命茶,每旦必令侍膳,往復咨詢,常夜分乃罷。濂不能飲,帝嘗強之至三觴,行不成步。帝大歡樂。御製《楚辭》一章,命詞臣賦《醉學士詩》。又嘗調甘露於湯,手酌以飲濂曰:「此能愈疾延年,願與卿共之。」又詔太子賜濂良馬,復為製《白馬歌》一章,亦命侍臣和焉。其寵待如此。九年進學士承旨知制誥,兼贊善如故。其明年致仕,賜《御製文集》及綺帛,問濂年幾何,曰:「六十有八。」帝乃曰:「藏此綺三十二年,作百歲衣可也。」濂頓首謝。又明年,來朝。十三年,長孫慎坐胡惟庸黨,帝欲置濂死。皇后太子力救,乃安置茂州。

濂狀貌豐偉,美鬚髯,視近而明,一黍上能作數字。自少至老,未嘗一日去書卷,於學無所不通。為文醇深演迤,與古作者並。在朝,郊社宗廟山川百神之典,朝會宴享律曆衣冠之制,四裔貢賦賞勞之儀,旁及元勛巨卿碑記刻石之辭,咸以委濂,屢推為開國文臣之首。士大夫造門乞文者,後先相踵。外國貢使亦知其名,數問宋先生起居無恙否。高麗、安南、日本至出兼金購文集。四方學者悉稱為「太史公」,不以姓氏。雖白首侍從,其勳業爵位不逮基,而一代禮樂制作,濂所裁定者居多。

其明年,卒於夔,年七十二。知事葉以從葬之蓮花山下。蜀獻王慕濂名,復移塋華陽城東。弘治九年,四川巡撫馬俊奏:「濂真儒翊運,述作可師,黼黻多功,輔導著績。久死遠戍,幽壤沉淪,乞加恤錄。」下禮部議,復其官,春秋祭葬所。正德中,追謚文憲。

仲子璲最知名,字仲珩,善詩,尤工書法。洪武九年,以濂故,召為中書舍人。其兄子慎亦為儀禮序班。帝數試璲與慎,并教誡之。笑語濂曰:「卿為朕教太子諸王,朕亦教卿子孫矣。」濂行步艱,帝必命璲、慎扶掖之。祖孫父子,共官內庭,眾以為榮。慎坐罪,璲亦連坐,並死,家屬悉徙茂州。建文帝即位,追念濂興宗舊學,召璲子懌官翰林。永樂十年,濂孫坐姦黨鄭公智外親,詔特宥之。

葉琛,字景淵,麗水人。博學有才藻。元末從石抹宜孫守處州,為畫策,捕誅山寇,授行省元帥。王師下處州,琛避走建寧。以薦徵至應天,授營田司僉事。尋遷洪都知府,佐鄧愈鎮守。祝宗、康泰叛,愈脫走,琛被執,不屈,大罵,死之。追封南陽郡侯,塑像耿再成祠,後祀功臣廟。章溢,字三益,龍泉人。始生,聲如鐘。弱冠,與胡深同師王毅。毅,字叔剛,許謙門人也,教授鄉里,講解經義,聞者多感悟。溢從之遊,同志聖賢學,天性孝友。嘗遊金華,元憲使禿堅不花禮之,改官秦中,要與俱行。至虎林,心動,辭歸。歸八日而父歿,未葬,火焚其廬。溢搏顙天,火至柩所而滅。

蘄、黃寇犯龍泉,溢從子存仁被執,溢挺身告賊曰:「吾兄止一子,寧我代。」賊素聞其名,欲降之,縛於柱,溢不為屈。至夜紿守者脫歸,集里民為義兵,擊破賊。俄府官以兵來,欲盡誅詿誤者。溢走說石抹宜孫曰:「貧民迫凍餒,誅之何為。」宜孫然其言,檄止兵,留溢幕下。從平慶元、浦城盜。授龍泉主簿,不受歸。宜孫守台州,為賊所圍。溢以鄉兵赴援,卻賊。已而賊陷龍泉,監縣寶忽丁遁去,溢與其師王毅帥壯士擊走賊。寶忽丁還,內慚,殺毅以反。溢時在宜孫幕府,聞之馳歸,偕胡深執戮首惡,因引兵平松陽、麗水諸寇。長槍軍攻婺,聞溢兵至,解去。論功累授浙東都元帥府僉事。溢曰:「吾所將皆鄉里子弟,肝腦塗地,而吾獨取功名,弗忍也。」辭不受。以義兵屬其子存道,退隱匡山。

明兵克處州,避入閩。太祖聘之,與劉基、葉琛、宋濂同至應天。太祖勞基等曰:「我為天下屈四先生,今天下紛紛,何時定乎?」溢對曰:「天道無常,惟德是輔,惟不嗜殺人者能一之耳。」太祖偉其言,授僉營田司事。巡行江東、兩淮田,分籍定稅,民甚便之。以病久在告,太祖知其念母也,厚賜遣歸省,而留其子存厚於京師。浙東設提刑按察使,命溢為僉事。胡深出師溫州,令溢守處州,饋餉供億,民不知勞。山賊來寇,敗走之。遷湖廣按察僉事。時荊、襄初平,多廢地,議分兵屯田,且以控制北方。從之。會浙東按察使宋思顏、孔克仁等以職事被逮,詞連溢。太祖遣太史令劉基諭之曰:「素知溢守法,毋疑也。」會胡深入閩陷沒,處州動搖,命溢為浙東按察副使往鎮之。溢以獲罪蒙宥,不應遷秩,辭副使,仍為僉事。既至,宣布詔旨,誅首叛者,餘黨悉定。召舊部義兵分布要害。賊寇慶元、龍泉,溢列木柵為屯,賊不敢犯。浦城戍卒乏食,李文忠欲運處州糧餉之。溢以舟車不通,而軍中所掠糧多,請入官均給之,食遂足。溫州茗洋賊為患,溢命子存道捕斬之。朱亮祖取溫州,軍中頗掠子女,溢悉籍還其家。吳平,詔存道守處,而召溢入朝。太祖諭群臣曰:「溢雖儒臣,父子宣力一方,寇盜盡平,功不在諸將後。」復問溢征閩諸將如何。對曰:「湯和由海道,胡美由江西,必勝。然閩中尤服李文忠威信。若令文忠從浦城取建寧,此萬全計也。」太祖立詔文忠出師如溢策。處州糧舊額一萬三千石,軍興加至十倍。溢言之丞相,奏復其舊。漸東造海舶,徵巨材於處。溢曰:「處、婺之交,山巖峻險,縱有木,從何道出?」白行省罷之。

洪武元年與劉基並拜御史中丞兼贊善大夫。時廷臣伺帝意,多嚴苛,溢獨持大體。或以為言。溢曰:「憲臺百司儀表,當養人廉恥,豈恃搏擊為能哉!」帝親祀社稷育》等。參見「倫理學」、「教育」中的「愛爾維修」。,會大風雨,還坐外朝,怒儀禮不合,致天變。溢委曲明其無罪,乃貰之。文忠之征閩也,存道以所部鄉兵萬五千人從。閩平,詔存道以所部從海道北征。溢持不可,曰:「鄉兵皆農民,許以事平歸農,今復調之,是不信也。」帝不懌。既而奏曰:「兵已入閩者,俾還鄉里。昔嘗叛逆之民,宜籍為軍,使北上,一舉而恩威著矣。」帝喜曰:「孰謂儒者迂闊哉!然非先生一行,無能辦者。」溢行至處州,遭母喪,乞守制。不許。鄉兵既集,命存道由永嘉浮海而北,再上章乞終制。詔可。溢悲戚過度,營葬親負土石,感疾卒,年五十六。帝痛悼,親撰文,即其家祭之。

存道,溢長子。溢應太祖聘,存道帥義兵歸總管孫炎。炎令守上游,屢卻陳友定兵。及以功授處州翼元帥副使,戍浦城。總制胡深死,命代領其眾,為遊擊。溢即處城坐鎮之。溢謂父子相統,於律不宜,奏罷存道官。不允。旋分兵征閩,而詔存道守處,復部鄉兵,從李文忠入閩。及還,浮海至京師。帝褒諭之,命從馮勝北征。積功授處州衛指揮副使。洪武三年從徐達西征,留守興元,敗蜀將吳友仁,再守平陽,轉左衛指揮同知。五年從湯和出塞征陽和,遇敵於斷頭山,力戰死焉。贊曰:太祖既下集慶,所至收攬豪雋,徵聘名賢,一時韜光韞德之士幡然就道。若四先生者,尤為傑出。基、濂學術醇深,文章古茂,同為一代宗工。而基則運籌帷幄,濂則從容輔導,於開國之初,敷陳王道,忠誠恪慎,卓哉佐命臣也!至溢之宣力封疆,琛之致命遂志,宏才大節,建豎偉然,洵不負弓旌之德意矣。基以儒者有用之學,輔翊治平,而好事者多以讖緯術數妄為傅會。其語近誕,非深知基者,故不錄云。


\end{pinyinscope}