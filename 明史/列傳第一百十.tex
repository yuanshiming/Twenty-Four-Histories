\article{列傳第一百十}

\begin{pinyinscope}
譚綸徐甫宰王化李佑王崇古子謙孫之楨之採李棠方逢時吳兌孫孟明孟明子邦輔鄭洛張學顏張佳胤殷正茂李遷凌雲翼

譚綸,字子理,宜黃人。嘉靖二十三年進士。除南京禮部主事。歷職方郎中,遷台州知府。綸沉毅知兵。時東南倭患已四年,朝議練鄉兵禦賊。參將戚繼光請期三年而後用之。綸亦練千人。立束伍法,自裨將以下節節相制。分數既明,進止齊一,未久即成精銳。倭犯柵浦,綸自將擊之,三戰三捷。倭復由松門、澶湖掠旁六縣,進圍台州,不克而去。轉寇仙居、臨海,綸擒斬殆盡。進海道副使,益募浙東良家子教之,而繼光練兵已及期,綸因收之以為用,客兵罷不復調。倭自象山突台州,綸連破之馬崗、何家覽,又與繼光共破之葛埠、南灣。加右參政,會憂去。以尚書楊博薦起,復將浙兵,討饒平賊林朝曦。朝曦者,大盜張璉餘黨也。璉既滅,朝曦據巢不下,出攻程鄉。知縣餘甫宰嚴兵待,而遣主簿梁維棟入賊中,諭散其黨。朝曦窮,棄巢走,綸及廣東兵追擒之。尋改官福建,乞終制去。

繼光數破賊,浙東略定。倭轉入福建。自福寧至漳、泉,千里盡賊窟,繼光漸擊定之。師甫旋,其眾復犯邵武,陷興化。四十二年春,再起綸。道擢右僉都御史,巡撫福建。倭屯崎頭城,都指揮歐陽深搏戰中伏死,倭遂據平海衛,陷政和、壽寧,各扼海道為歸計。綸環柵斷路,賊不得去,移營渚林。繼光至,綸自將中軍,總兵官劉顯、俞大猷將左、右軍。令繼光以中軍薄賊壘,左右軍繼之,大破賊,復一府二縣。詔加右副都御史。綸以延、建、汀、邵間殘破甚,請緩征蠲賦。又考舊制,建水砦五,扼海口,薦繼光為總兵官以鎮守之。倭復圍仙遊,綸、繼光大破賊城下。已而繼光破賊王倉坪、蔡丕嶺,餘賊走,廣東境內悉定。綸上疏請復行服,世宗許之。

四十四年冬,起故官,巡撫陜西。未上而大足民作亂,陷七城。詔改綸四川,至已破滅。雲南叛酋鳳繼祖遁入會理,綸會師討平之。進兵部右侍郎兼右僉都御史,總督兩廣軍務兼巡撫廣西。招降嶺崗賊江月照等。

綸練兵事,朝廷倚以辦賊,遇警輒調,居官無淹歲。迨南寇略平,而邊患方未已。隆慶元年,給事中吳時來請召綸、繼光練兵。詔徵綸還部,進左侍郎兼右僉都御史,總督薊、遼、保定軍務。綸上疏曰:

薊、昌卒不滿十萬,而老弱居半,分屬諸將,散二千里間。敵聚攻,我分守,眾寡強弱不侔,故言者亟請練兵。然四難不去,兵終不可練。

夫敵之長技在騎,非召募三萬人勤習車戰,不足以制敵。計三萬人月餉,歲五十四萬,此一難也。燕、趙之士銳氣盡於防邊,非募吳、越習戰卒萬二千人雜教之,事必無成。臣與繼光召之可立至,議者以為不可。信任之不專,此二難也。軍事尚嚴,而燕、趙士素驕,驟見軍法,必大震駭。且去京師近,流言易生,徒令忠智之士掣肘廢功,更釀他患,此三難也。我兵素未當敵,戰而勝之,彼不心服。能再破,乃終身創,而忌嫉易生;欲再舉,禍已先至。此四難也。

以今之計,請調薊鎮、真定、大名、井陘及督撫標兵三萬,分為三營,令總兵參遊分將之,而授繼光以總理練兵之職。春秋兩防,三營兵各移近邊。至則遏之邊外,入則決死邊內。二者不效,臣無所逃罪。又練兵非旦夕可期,今秋防已近,請速調浙兵三千,以濟緩急。三年後,邊軍既練,遣還。

詔悉如所請,仍令綸、繼光議分立三營事宜。綸因言:「薊鎮練兵踰十年,然竟不效者,任之未專,而行之未實也。今宜責臣綸、繼光,令得專斷,勿使巡按、巡關御史參與其間。」自兵事起,邊臣牽制議論,不能有為,故綸疏言之。而巡撫劉應節果異議,巡按御史劉翾、巡關御史孫代又劾綸自專。穆宗用張居正言,悉以兵事委綸,而諭應節等無撓。

綸相度邊隘衝緩,道里遠近,分薊鎮為十二路,路置一小將,總立三營:東駐建昌備燕河以東,中駐三屯備馬蘭、松、太,西駐石匣備曹牆、古石。諸將以時訓練,互為掎角,節制詳明。是歲秋,薊、昌無警。異時調陜西、河間、正定兵防秋,至是悉罷。綸初至,按行塞上,謂將佐曰:「秣馬厲兵,角勝負呼吸者,宜於南;堅壁清野,坐制侵軼者,宜於北。」遂與繼光圖上方略,築敵臺三千,起居庸至山海,控守要害。綸召入為右都御史兼兵部左侍郎,協理戎政。會臺工成,益募浙兵九千餘守之。邊備大飭,敵不敢入犯。以功進兵部尚書兼右都御史,協理如故。其冬,予告歸。

神宗即位,起兵部尚書。萬歷初,加太子少保。給事中雒遵劾綸不稱職。綸三疏乞罷,優詔留之。五年卒官。贈太子太保,謚襄敏。

綸終始兵事垂三十年,積首功二萬一千五百。嘗戰酣,刃血漬腕,累沃乃脫。與繼光共事齊名,稱「譚、戚」。

徐甫宰,字允平,浙江山陰人。嘉靖中舉順天鄉試,除武平知縣。武平當閩、粵交,多盜,甫宰築城立堡者三。上官以程鄉賊盜藪,調之往。既平朝曦,超擢潮州兵備僉事,添注剿寇,任一子千戶。已而程鄉溫鑑、梁輝等合上杭賊窺江西。平遠知縣王化遮擊之檀嶺,賊敗奔瑞金,副使李佑三戰皆捷。賊由間道歸程鄉,甫宰討擒之,餘黨悉平。賚銀幣。已,補潮州分巡僉事兼理兵備事。東莞水兵徐永太等亂,停俸討賊。甫宰已疾亟,乞歸。未幾卒。

王化,字汝贊,廣西馬平人。父尚學,職方郎中。化登鄉薦。嘉靖四十年,新置平遠縣,授化知縣。以擊賊檀嶺,有知兵名。田坑賊梁國相既降復叛,約三圖賊葛鼎榮等分寇江西、福建。化寄妻子會昌,而身率鄉兵往擊。賊連敗,乃縱反間會昌,言化已歿,化妻計氏慟哭自刎。化怒,追賊益急,獲國相於石子嶺。遷潮州府同知,仍署縣事。計被旌,官為立祠。化舉卓異,超擢廣東副使。南贛巡撫吳百朋以貪黷劾之,削籍。巡按御史趙淳薦其知兵,乃命以僉事飭惠、潮兵備。久之,考察罷。

李佑,字吉甫,貴州清平衛人。嘉靖二十六年進士。歷官江西副使,邀賊瑞金有功。尋敗廣東賊吳志高、江西下歷賊賴清規等,皆賚銀幣。進江西右參政。偕總兵官俞大猷,大破劇賊李亞元。擢僉都御史,巡撫廣東。屢敗海寇林道乾、山寇張韶南等。隆慶中,被劾罷歸。

王崇古,字學甫,蒲州人。嘉靖二十年進士。除刑部主事。由郎中歷知安慶、汝寧二府。遷常鎮兵備副使,擊倭夏港,追殲之靖江。從巡撫曹邦輔戰滸墅。已,偕俞大猷追倭出海。累進陜西按察使,河南右布政使。

四十三年,改右僉都御史,巡撫寧夏。崇古喜譚兵,具知諸邊阨塞,身歷行陣,修戰守,納降附,數出兵搗巢。寇屢殘他鎮,寧夏獨完。隆慶初,加右副都御史。

吉囊子吉能據河套為西陲諸部長,別部賓兔駐牧大、小松山,南擾河、湟番族,環四鎮皆寇。總督陳其學無威略,總兵官郭江、黃演等皆敗死,陜西巡撫戴才亦坐免。其冬,進崇古兵部右侍郎兼右僉都御史,總督陜西、延、寧、甘肅軍務。崇古奏給四鎮旗牌,撫臣得用軍法督戰,又指畫地圖,分授諸大將趙岢、雷龍等。數有功。著力兔行牧河東,龍潛出興武襲破其營,斬獲多,加崇古右都御史。吉能犯邊,為防秋兵所遏,移營白城子。龍等出花馬池、長城關與戰,大敗之。崇古在陜七年,先後獲首功甚多。

自河套以東宣府、大同邊外,吉囊弟俺答、昆都力駐牧地也。又東薊、昌以北,吉囊、俺答主土蠻居之,皆強盛。俺答又納叛人趙全等,據古豐州地,招亡命數萬,屋居佃作,號曰板升。全等尊俺答為帝,為治城郭宮殿;亦自治第,制度如王者,署其門曰開化府。又日夜教俺答為兵。東入薊、昌,西掠忻、代,遊騎薄平陽、靈石,至潞安以北。起嘉靖辛丑,擾邊者三十年,邊臣坐失事得罪者甚眾,患視陜西四鎮尤劇。朝廷募獲全者官都指揮使,賞千金,卒不能得。邊將士率賄寇求和,或反為用;諸陷寇自拔歸者,輒殺之以冒功賞;敵情不可得,而軍中動靜敵輒知。四年正月,詔崇古總督宣、大、山西軍務。崇古禁邊卒闌出,而縱其素通寇者深入為間。又檄勞番、漢陷寇軍民,率眾降及自拔者,悉存撫之。歸者接踵。西番、瓦剌、黃毛諸種一歲中降者踰二千人。

其冬,把漢那吉來降。把漢那吉者,俺答第三子鐵背台吉子也。幼失父,育於俺答妻一克哈屯。長娶大成比妓不相得。把漢自聘我兒都司女,號三娘子,即俺答外孫女也。俺答見其美,奪之。把漢恚,又聞崇古方納降,是年十月,率妻子十餘人來歸。巡撫方逢時以告。崇古念因此制俺答,則趙全等可除也,留之大同,慰藉甚至。偕逢時疏聞於朝曰:「俺答橫行塞外幾五十年,威制諸部,侵擾邊圉。今神厭凶德,骨肉離叛,千里來降,宜給宅舍,授官職,豐餼廩服用,以悅其心,嚴禁出入,以虞其詐。若俺答臨邊索取,則因與為市,責令縛送板升諸逆,還被掠人口,然後以禮遣歸,策之上也。若遂桀驁稱兵,不可理諭,則明示欲殺,以撓其志。彼望生還,必懼我制其死命。志奪氣沮,不敢大逞,然後徐行吾計,策之中也。若遂棄而不求,則當厚加資養,結以恩信。其部眾繼降者,處之塞下,即令把漢統領,略如漢置屬國居烏桓之制。他日俺答死,子辛愛必有其眾。因加把漢名號,令收集餘眾,自為一部。辛愛必忿爭。彼兩族相持,則兩利俱存,若互相仇殺,則按兵稱助。彼無暇侵陵,我遂得休息,又一策也。若循舊例安置海濱,使俺答日南望,侵擾不已;又或給配諸將,使之隨營立功,彼素驕貴不受驅策,駕馭茍乖,必滋怨望,頓生颺去之心,終貽反噬之禍,均為無策。」奏至,朝議紛然。御史饒仁侃、武尚賢、葉夢熊皆言敵情叵測。夢熊至引宋受郭藥師、張彀事為喻。兵部尚書郭乾不能決,大學士高拱、張居正力主崇古議。詔授把漢指揮使,賜緋衣一襲,而黜夢熊於外,以息異議。

俺答方掠西番,聞變急歸,調辛愛兵分道入犯,索把漢甚急。辛愛佯發兵,陰擇便利,以故俺答不得志。一克哈屯思其孫,朝夕哭,俺答患之。巡撫逢時遣百戶鮑崇德入其營,俺答盛氣待之曰:「自吾用兵,而鎮將多死。」崇德曰:「鎮將孰與而孫?今朝廷待而孫甚厚,稱兵是速其死也。」俺答疑把漢已死,及聞言,心動,使使詗之。崇古令把漢緋袍金帶見使者,俺答喜過望,崇德因說之曰:「趙全等旦至,把漢夕返。」俺答大喜,屏人語曰:「我不為亂,亂由全等。令吾孫降漢,是天遣之合也。天子幸封我為王,永長北方,諸部孰敢為患。即不幸死,我孫當襲封,彼受朝廷厚恩,豈敢負耶?」遂遣使與崇德俱來,又為辛愛求官,并請互市。崇古以聞,帝悉報可。俺答遂縛全等十餘人以獻,崇古亦遣使送把漢歸。帝以叛人既得,祭告郊廟,磔全等於市。加崇古太子少保、兵部尚書,總督如故。

把漢既歸,俺答與其妻撫之泣。遣使報謝,誓不犯大同。崇古令要土蠻、昆都力、吉能等皆入貢,俺答報如約,惟土蠻不至。崇古念土蠻勢孤,薊、昌可無患,命將士勿燒荒搗巢,議通貢市,休息邊民。朝議復嘩。尚書郭乾謂馬市先帝明禁,不宜許。給事中章端甫請敕崇古無邀近功,忽遠慮。崇古上疏曰:「先帝既誅仇鸞,制復言開市者斬,邊臣何敢故違禁旨,自陷重辟。但敵勢既異昔強,我兵亦非昔怯,不當援以為例。夫先帝禁開馬市,未禁北敵之納款。今敵求貢市,不過如遼東、開原、廣寧之規,商人自以有無貿易,非請復開馬市也。俺答父子兄弟橫行四五十年,震驚宸嚴,流毒畿輔,莫收遏劉功者,緣議論太多,文網牽制,使邊臣無所措手足耳。昨秋,俺答東行,京師戒嚴,至倡運磚聚灰塞門乘城之計。今納款求貢,又必責以久要,欲保百年無事,否則治首事之罪。豈惟臣等不能逆料,他時雖俺答亦恐能保其身,不能制諸部於身後也。夫拒敵甚易,執先帝禁旨,一言可決。但敵既不得請,懷憤而去,縱以把漢之故,不擾宣、大,而土蠻三衛歲窺薊、遼,吉能、賓兔侵擾西鄙,息警無時,財務殫絀,雖智者無以善其後矣。昔也先以剋減馬價而稱兵,忠順王以元裔而封哈密,小王子由大同二年三貢,此皆前代封貢故事。夫揆之時勢,既當俯從,考之典故,非今創始。堂堂天朝,容荒服之來王,昭聖圖之廣大,以示東西諸部,傳天下萬世,諸臣何疑憚而不為耶?」因條封貢八事以上。

詔下廷議。定國公徐文璧、侍郎張四維以下二十二人以為可許,英國公張溶、尚書張守直以下十七人以為不可許。尚書朱衡等五人言封貢便、互市不便,獨僉都御史李棠極言當許狀。郭乾悉上眾議。會帝御經筵,閣臣面請外示羈縻,內修守備。乃詔封俺答順義王,名所居城曰歸化;昆都力、辛愛等皆授官;封把漢昭勇將軍,指揮使如故。俺答率諸部受詔甚恭,使使貢馬,執趙全餘黨以獻。帝嘉其誠,賜金幣。又雜采崇古及廷臣議,賜王印,給食用,加撫賞,惟貢使不聽入京。

河套吉能亦如約請命。以事在陜西,下總督王之誥議。之誥欲令吉能一二年不犯,方許封貢。崇古復上疏曰:「俺答、吉能親為叔姪,首尾相應。今收其叔而縱其侄,錮其首而舒其臂,俺答必呼吉能之眾就市河東宣、大;商販不能給,而吉能糾俺答擾陜西,四鎮之憂方大矣。」帝然其言,亦授吉能都督同知。崇古乃廣召商販,聽令貿易。布帛、菽粟、皮革遠自江淮、湖廣輻輳塞下,因收其稅以充犒賞。其大小部長則官給金繒,歲市馬各有數。崇古仍歲詣弘賜堡宣諭威德。諸部羅拜,無敢譁者。自是邊境休息。東起延、永,西抵嘉峪七鎮,數千里軍民樂業,不用兵革,歲省費什七。詔進太子太保。

萬曆初,召理戎政。給事中劉鉉劾崇古行賄營遷,詔責鉉妄言。已,加少保,遷刑部尚書,改兵部。初,俺答諸部嘗越甘肅掠西番。既通款,其從孫切盡台吉連歲盜番,不得志,求俺答西援。崇古每作書止之,俺答亦報書謝。是年,俺答請與三鎮通事約誓,欲西迎佛。崇古上言:「西行非俺答意,且以迎佛為名,不可沮,宜飭邊鎮嚴守備,而陰泄其謀於番族以示恩。」於是鉉及同官彭應時、南京御史陳堂交章論崇古弛防徇敵。崇古疏辯乞休。帝優詔報之,令勿以人言介意。給事中尹瑾、御史高維崧再劾之,崇古力請致仕,帝乃允歸。

俺答既死,辛愛、撦力克相繼襲封。十五年,詔以崇古竭忠首事,三封告成,蔭一子世錦衣千戶,有司以禮存問。又二年卒。贈太保,謚襄毅。

崇古身歷七鎮,勳著邊陲。封貢之初,廷議紛呶,有為危言撼帝者。閣臣力持之,乃得成功。順義歸款二十年,崇古乃歿。總督梅友松撫馭失宜,西邊始擾,而禍已紓於嘉靖時,宣、大則歸款迄明季不變。

子謙,萬歷五年進士。官工部主事,榷稅杭州。羅木營兵變,脅執巡撫吳善言。謙馳諭之乃解。終太僕少卿。孫之楨,以蔭累官太子太保、左都督,掌錦衣衛事凡十有七年;之采,萬曆二十六年進士,官兵部右侍郎,陜西三邊總督。

李棠,長沙人。由吏部郎中累遷右副都御史,巡撫南、贛。督僉事諸察討平韶州山賊。終南京吏部右侍郎。仕宦三十年,以介潔稱。天啟初,追謚恭懿。

方逢時,字行之,嘉魚人。嘉靖二十年進士。授宜興知縣,再徙寧津、曲周。擢戶部主事,歷工部郎中,遷寧國知府。廣東、江西盜起,詔於興寧、程鄉、安遠、武平間築伸威鎮,擢逢時廣東兵備副使,與參將俞大猷鎮之。已而程鄉賊平,移巡惠州。

隆慶初,改宣府口北道,加右參政。旋擢右僉都御史,巡撫遼東。四年正月,移大同。俺答犯威遠堡,別部千餘騎攻靖鹵,伏兵卻之。其冬,俺答孫把漢那吉來降,逢時告總督王崇古曰:「機不可失也。」遣中軍康綸率騎五百往受之。與崇古定計,挾把漢以索叛人趙全等。遣百戶鮑崇德出雲石堡語俺答部下五奴柱曰:「欲還把漢則速納款,若以兵來,是趣之死矣。」五奴柱白俺答,邀入營,說以執趙全易把漢。俺答心動,遣火力赤致書逢時。而全方從臾用兵,俺答又惑之,令其子辛愛將二萬騎入弘賜堡,兄子永邵卜趨威遠堡,自率眾犯平虜城。逢時曰:「此必趙全謀也。」全嘗投書逢時,言悔禍思漢,欲復歸中國。逢時以示俺答,俺答大驚,有執全意。及戰,又不利,乃引退。辛愛猶未知,奄至大同。逢時使人持把漢箭示之曰:「吾已與而父約,以報汝。」辛愛執箭泣曰:「此吾弟鐵背台吉故物也,我來求把漢,把漢既授官,又有成約,當更計之。」乃遣部下啞都善入見。逢時曉以大義,犒而遣之。辛愛喜,因使求幣,逢時笑曰:「台吉,豪傑也,若納款,方重加爵賞,何愛此區區,損盛名。」辛愛大慚,復遣啞都善來謝曰:「邊人不知書,蒙太師教,幸甚。俺答使者至故將田世威所,世威亦讓之曰:「爾來求和,兵何為者?」使者還報俺答,召辛愛還。辛愛東行,宣府總兵官趙岢遏之,復由大同北去。於是巡按御史姚繼可劾逢時輒通寇使,屏人語,導之東行,嫁禍鄰鎮。大學士高拱曰:「撫臣臨機設策,何可洩也。但當觀後效,不宜先事輒易。」帝然之。俺答乃遣使定約,夜召全等計事,即帳中縛之送大同。逢時受之,崇古亦送把漢歸。逢時以功進兵部右侍郎兼右僉都御史。甫拜命,以憂歸。後崇古入理京營,神宗問誰可代者,大學士張居正以逢時對。

萬曆初,起故官,總督宣、大、山西軍務。始逢時與崇古共決大計,而貢市之議崇古獨成之。逢時復代崇古,乃申明約信。兩人首尾共濟,邊境遂安。逢時分巡口北,時親行塞外,自龍門盤道墩以東至靖湖堡山梁一百餘里,形勢聯絡,歎曰:「此山天險。若修鑿,北可達獨石,南可援南山,誠陵京一籓籬也。」及赴陽和,道居庸,出關見邊務修舉,欲并遂前計。上疏曰:「獨石在宣府北,三面鄰敵,勢極孤懸。懷、永與陵寢止限一山,所係尤重。其地本相屬,而經行之路尚在塞外,以故聲援不便。若設盤道之險,舍迂就徑,自龍門黑峪以達寧遠,經行三十里,南山、獨石皆可朝發夕至,不惟拓地百里,亦可漸資屯牧,於戰守皆利。」遂與巡撫吳兌經營修築,設兵戍守。累進兵部尚書兼右副都御史,總督如故,加太子少保。

五年,召理戎政。時議者爭言貢市利害,逢時臨赴闕,上疏曰:

陛下特恩起臣草土中,代崇古任,賴陛下神武,八年以來,九邊生齒日繁,守備日固,田野日闢,商賈日通,邊民始知有生之樂。北部輸誠效貢,莫敢渝約,歲時請求,隨宜與之,得一果餅,輒稽首歡笑。有掠人要賞,如打喇明安兔者,告俺答罰治,即俯首聽命。而異議者或曰「敵使充斥為害」,或曰「日益費耗,彼欲終不可足」,或曰「與寇益狎,隱憂叵測」。此言心則忠矣,事機或未睹也。

夫使者之入,多者八九人,少者二三人,朝至夕去,守貢之使,賞至即歸,何有充斥。財貨之費,有市本,有撫賞,計三鎮歲費二十七萬,較之鄉時戶部客餉七十餘萬,太僕馬價十數萬,十纔二三耳。而民間耕獲之入,市賈之利不與焉。所省甚多,何有耗費。乃若所憂則有之,然非隱也。方庚午以前,三軍暴骨,萬姓流離,城郭丘墟,芻糧耗竭,邊臣首領不保,朝廷為旰食。七八年來,幸無此事矣。若使臣等處置乖方,吝小費而虧大信,使一旦肆行侵掠,則前日之憂立見,何隱之有哉?

其所不可知者,俺答老矣,誠恐數年之後,此人既死,諸部無所統一,其中狡黠,互相爭構,假托異辭,遂行侵擾。此則時變之或然,而不可預料者。在我處之,亦惟罷貢絕市,閉關固壘以待。仍禁邊將毋得輕舉,使曲常在彼,而直常在我。因機處置,顧後人方略何如耳。夫封疆之事,無定形亦無定機,惟朝廷任用得人,處置適宜,何必拘拘焉貢市非而戰守是哉?臣又聞之,禦戎無上策。征戰禍也,和親辱也,賂遺恥心。今曰貢,則非和親矣;曰市,則非賂遺矣;既貢且市,則無征戰矣。臣幸藉威靈,制伏強梗,得免斧鉞之誅。今受命還朝,不復與聞閫外之事,誠恐議者謂貢市非計,輒有敷陳,國是搖惑。內則邊臣畏縮,外則部落攜貳,事機乖迕,後悔無及。臣雖得去,而犬馬之心實有不能一日忘者,謹列上五事。

至京,復奏上款貢圖。尋代崇古為尚書,署吏部事,加太子太保。以平兩廣功,進少保。累疏致仕歸,御書「盡忠」字賜之。二十四年卒。

逢時才略明練。處置邊事,皆協機宜。其功名與崇古相亞,稱「方、王」云。

吳兌,字君澤,紹興山陰人。嘉靖三十八年進士。授兵部主事。隆慶三年,由郎中遷湖廣參議。調河南,遷薊州兵備副使。五年秋,擢右僉都御史,巡撫宣府。兌舉鄉試出高拱門。拱之初罷相也,兌獨送至潞河。及拱再起兼吏部,遂超擢之。釋褐十三年得節鉞,前此未有也。

時俺答初封貢,而昆都力、辛愛陰持兩端,助其主土蠻為患。兌事智計,操縱馴伏之。嘗偵俺答離營獵,從五騎直趨其營。守者愕,控弦。從騎呵之曰:「太師來犒軍耳!」皆拜跪迎導,且獻酪。兌遍閱廬帳,抵暮還。市者或潛盜所鬻馬,兌使人棓擊之,曰:「後復盜,即閉關停市。」諸部追究所奪馬,并執其人以謝。辛愛復擾邊,俺答曰「宣、大,我市場也。」戒勿動。然辛愛猶傑驁,俺答常以己馬代入貢。既得賞賜,抵地不肯受,又遣兵掠車夷。車夷者,不知其所出,自嘉靖中徙至,與史夷雜居,皆宣鎮保塞屬也。辛愛掠之,以其長革固去,其二比妓來駐龍門教場。兌以史、車脣齒,車被掠,史益孤,奏築堡居之。使使詰責辛愛,令還革固而勒其比妓遠邊。辛愛誘比妓五蘭且沁、威兀慎,歲盜葛峪堡器甲、牛羊。兌皆付三娘子罰治。三娘子有盛寵於俺答,辛愛嫉妒,數詛詈之。三娘子入貢,宿兌軍中,愬其事。兌贈以八寶冠、百鳳雲衣、紅骨朵雲裙,三娘子以此為兌盡力。辛愛、撦力克相繼襲王,皆妻三娘子,三娘子主貢市者三世。昆都力嘗求封王,會病死,其子青把都擁兵至塞,多所要挾。兌諭以禍福,而耀武震之。青把都懼,貢如初,其女東桂嫁朵顏都督長昂,嘗隨父入貢,訴其貧。兌諭其昆弟,每一馬分一繒畀之。後東桂報土蠻別騎掠三岔河東,兌得為備,有功。

萬曆二年春,推款貢功,加兌右副都御史。貢市畢,加兵部右侍郎兼右僉都御史。五年夏,代方逢時總督宣、大、山西軍務。俺答西掠瓦剌,聲言迎佛,寄帑於兌,留旗箭為信。尚書王崇古奏上方略,使兌諭俺答繞賀蘭山後行,勿道甘肅;又陰洩其謀於瓦剌。俺答兵遂挫,留青海未歸。而青把都復附土蠻,其部下時入寇。大學士張居正令兌趣俺答東還約束之,青把都亦罰治其下,款貢乃益堅。七年秋,以左侍郎召還部,尋加右都御史,仍佐部事。

九年夏,復以本官總督薊、遼、保定軍務兼巡撫順天。泰寧速把亥與青把都交通,陰入市宣府,而歲犯遼東以要款。朝廷拒不許,兌修義州城備之。明年春,速把亥來寇,總兵官李成梁擊斬之。其弟炒花、姪老撒卜兒悉遁去。詔進兌兵部尚書仍兼右都御史。尋進太子少保,召拜兵部尚書。御史魏允貞劾兌歷附高拱、張居正,且饋馮保金千兩,封識具存。給事王繼光亦言兌受將吏饋遺,御史林休徵助之攻。帝乃允兌去,後數年卒。

孫孟明,襲錦衣千戶,佐許顯純理北司刑。天啟初,讞中書汪文言,頗為之左右。顯純怒,誣孟明藏匿亡命。下本司拷訊,削籍歸。崇禎初,起故官,累遷都督同知,掌衛事。孟明居官貪,以附東林,頗得時譽。子邦輔襲職,亦理北司刑。崇禎末,給事中姜採、行人司副熊開元以言事同日繫詔獄,帝欲置之死,邦輔故緩其獄。帝怒稍解,令嚴訊主使者。邦輔乃略訊即具獄上,詔予杖百,二人由是獲免。

鄭洛,字禹秀,安肅人。嘉靖三十五年進士。除登州推官,徵授御史。劾罷嚴嵩黨鄢懋卿、萬寀、萬虞龍。出為四川參議,遷山西參政。佐總督王崇古款俺答有功。萬曆二年,由浙江左布政使改右僉都御史,巡撫山西。移大同,加右副都御史,入為兵部右侍郎。七年,以左侍郎總督宣、大、山西軍務。昆都力子滿五大令銀定入犯,洛奏停貢市,遣使責俺答罰贖駝馬牛羊,乃復許款。三娘子佐俺答主貢市,諸部皆受其約束。及辛愛襲封,年老且病,欲妻三娘子。三娘子不從,率眾西走,辛愛自追之,貢市久不至。洛計三娘子別屬,則辛愛雖王無益,乃使人語之曰:「夫人能歸王,不失恩寵,否則塞上一婦人耳。」三娘子聽命。辛愛更名乞慶哈,貢市惟謹。洛以功加兵部尚書兼右副都御史。十四年,乞慶哈死,子撦力克當襲。三娘子以年長,自練兵萬人,築城別居。洛恐貢市無主,復諭撦力克曰:「夫人三世歸順,汝能與之匹,則王,不然封別有屬也。」手奢力克盡逐諸妾,復妻三娘子。遂以明年嗣封,并奏封三娘子忠順夫人。洛乃上疏請定市馬數,宣府不得踰三萬,大同萬四千,山西六千,而申飭將吏嚴備,以防盜竊,且無輕遏其部落馳獵者。帝嘉納之。御史許守恩劾洛。乞歸,不允。自太子少保累加至太子太保,召為戎政尚書。

十八年,洮河用兵,詔兼右都御史,經略陜西、延、寧、甘肅及宣、大、山西邊務。松套賓兔等屢越甘肅侵擾河、湟諸番。及俺答迎佛,又建寺於青海,奏賜名仰華,留永邵卜別部把爾戶及丙兔、火落赤守之,俱牧海上。他部往來者,率取道甘肅,甘肅鎮臣以通款弗禁也。丙兔死,其子真相進據莽剌川,火落赤據捏工川,益併吞番族。河套都督卜失兔亦遣使邀撦力克,撦力克遺洛書,以赴仰華為名。洛使從塞外行,又諭忠順夫人曰:「彼中撫賞不能多,且王家在東,恐有內顧憂也。」撦力克遂行。未至,把爾戶部卒闌入西寧。副總兵李奎方醉,單騎馳之。卒持鞚自白,為奎所斫,遂大噪,射奎死。火落赤、真相進圍舊洮州,副總兵李聯芳敗歿。入臨洮、河州、渭源,總兵官劉承嗣失利,遊擊李芳等皆死。當是時,撦力克已至仰華,火落赤、真相益挾為重,關中大震,惟把爾戶不助逆。事聞,詔洛經略七鎮,以僉事萬世德、兵部員外郎梁雲龍隨軍贊畫,而停手奢力克貢市。俄罷總督梅友松,命洛兼領其事。洛以洮河之禍,由縱敵入青海,乃馳至甘肅,令曰:「北部自青海歸巢者,聽假道;自巢入青海者,即勒兵拒之。」未幾,卜失兔至水泉,欲趨青海。總兵官張臣與相持月餘,洛設伏掩擊之,卜失兔僅以身免。莊禿賴後至,聞之亦退去。

明年,洛與雲龍入西寧,控扼青海。撦力克聞之,西徙二百里,還洮河所掠人口,與忠順夫人輸罪請歸。火落赤、真相亦夜去,兩川餘黨留莽刺南山。洛慮諸部約結,先遣使趣撦力克北歸,別遣雲龍、世德收番族以弱其勢,而具以狀奏聞。言:「自順義南牧,借途收番,子女牛羊皆有之,生死唯所制。洮河之役,遂為嚮導,番戎之勢不分,則心腹之患無已。臣鼓舞勞來,招回諸番八萬餘人,皆陛下威德所致。」且具陳收番有六利。是時,撦力克觀望不即歸,洛與相羈縻,先遣總兵官尤繼先擊走莽剌餘寇。督撫魏學曾、葉夢熊等請決戰,夢熊又騰書都下,洛疏持不可。夢熊乃調苗兵三千為選鋒,詆洛為秦檜、賈似道。會撦力克北歸謝罪,乞復貢市,洛乃進兵青海,走火落赤、真相,焚仰華,置戍西寧、歸德而還。尚書石星以宣、大事急,請速召洛究款戰之計。洛既至,與總督蕭大亨、巡撫王世揚、邢玠等上疏曰:「撦力克諉罪火落赤、真相,桀驁之狀已斂。且其部落數千里,部長十餘輩。在巢保疆者,宣鎮則青把都兄弟未嘗東窺薊、遼,而兀慎、擺腰五路之在新平,馴服猶故。在西行牧者,不他失未嘗窺莽、捏,而大成比妓則又歸巢獨先。今以一人之罪,概絕諸部,消往日之恩,開將來之隙,臣未見其可。今史二外叛,屢犯邊疆,若令順義王縛獻以著信,然後酌議市賞,在我固未為失策也。」議遂定。尋加少保,仍召理戎政。順義王果縛史二來獻,復款如故。

初,閱邊給事中張棟言,洮河之衄,殞將喪師,洛為其所輕,故東西移帳自便。太僕寺丞徐琰復詆洛,乞處分以除誤國之罪。棟再疏劾洛欺罔,給事中章尚學亦請令洛回宣、大。至是撦力克歸,棟又言:「火、真亂首,順義亂階,洛宜除兇雪恥,乃虛詞誘敵,而重利媚之。今火、真依海為窟,出沒如故,洛輒侈然敘文武勞。乞敕所司,毋徇洛請。」洛乃謝病歸。尚書星言洛無重利啖敵事,且有威望,不宜久棄。逾三年,官軍與番人夾擊把爾戶於西寧,大破之。星復奏洛收番之功,再詔起用。當時以洛有物議,卒不推也。卒,贈太保,謚襄敏。

張學顏,字子愚,肥鄉人。生九月失母,事繼母以孝聞。親喪廬墓,有白雀來巢。登嘉靖三十二年進士,由曲沃知縣入為工科給事中。遷山西參議,以總督江東劾去官。事白,遷永平兵備副使,再調薊州。俺答封順義王,察罕土門汗語其下曰:「俺答,奴也,而封王,吾顧弗如。」挾三衛窺遼,欲以求王。而海、建諸部日強,皆建國稱汗。大將王治道、郎得功戰死,遼人大恐。隆慶五年二月,遼撫李秋免,大學士高拱欲用學顏,或疑之,拱曰:「張生卓犖倜儻,人未之識也,置諸盤錯,利器當見。」侍郎魏學曾後至,拱迎問曰:「遼撫誰可者?」學曾思良久,曰:「張學顏可。」拱喜曰:「得之矣。」遂以其名上,進右僉都御史,巡撫遼東。

遼鎮邊長二千餘里,城砦一百二十所,三面鄰敵。官軍七萬二千,月給米一石,折銀二錢五分,馬則冬春給料,月折銀一錢八分,即歲稔不足支數日。自嘉靖戊午大饑,士馬逃故者三分之二。前撫王之誥、魏學曾相繼綏輯,未復全盛之半。繼以荒旱,餓莩枕籍。學顏首請振恤,實軍伍,招流移,治甲仗,市戰馬,信賞罰。黜懦將數人,創平陽堡以通兩河,移遊擊於正安堡以衛鎮城,戰守具悉就經畫。大將李成梁敢力戰深入,而學顏則以收保為完策,敵至無所亡失,敵退備如初,公私力完,漸復其舊。十一月,與成梁破土蠻卓山,進右副都御史。明年春,土蠻謀入寇,聞有備而止。

奸民闌出海上,踞三十六島。閱視侍郎汪道昆議緝捕,學顏謂緝捕非便。命李成梁按兵海上,示將加誅,別遣使招諭,許免差役。未半載,招還四千四百餘口,積患以消。秋,建州都督王杲以索降人不得,入掠撫順,守將賈汝翼詰責之。杲益憾,約諸部為寇,副總兵趙完責汝翼啟釁,學顏奏曰:「汝翼卻杲饋遺,懲其違抗,實伸國威,茍緣此罷斥,是進退邊將皆敵主之矣。臣謂宜諭王杲送還俘掠,否則調兵剿殺,毋事姑息以蓄禍。」趙完懼,餽金貂,學顏發之,詔逮完,而宣諭王杲如學顏策。諸部聞大兵且出,悉竄匿山谷。杲懼,十二月約海西王台送俘獲就款,學顏因而撫之。

遼陽鎮東二百餘里舊有孤山堡,巡按御史張鐸增置險山五堡,然與遼鎮聲援不接。都御史王之誥奏設險山參將,轄六堡一十二城,分守靉陽。又以其地不毛,欲移置寬佃,以時絀不果。萬曆初,李成梁議移孤山堡於張其哈佃,移險山五堡於寬佃、長佃、雙墩、長領散等。皆據膏腴,扼要害。而邊人苦遠役,出怨言。工甫興,王杲復犯邊,殺遊擊裴承祖。巡按御史亟請罷役,學顏不可,曰:「如此則示弱也。」即日巡塞上,撫定王兀堂諸部,聽於所在貿易。卒築寬佃,斥地二百餘里。於是撫順以北,清河以南,皆遵約束。明年冬,發兵誅王杲,大破之,追奔至紅力寨。張居正第學顏功在總督楊兆上,加兵部侍郎。

五年夏,土蠻大集諸部犯錦州,要求封王。學顏奏曰:「敵方憑陵,而與之通,是畏之也。制和者在彼,其和必不可久。且無功與有功同封,犯順與效順同賞,既取輕諸部,亦見笑俺答。臣等謹以正言卻之。」會大雨,敵亦引退。其冬,召為戎政侍郎,加右都御史。未受代,而土蠻約泰寧速把亥分犯遼、沈、開原。明年正月破敵劈山,殺其長阿丑台等五人,學顏遂還部。踰年,拜戶部尚書。

時張居正當國,以學顏精心計,深倚任之。學顏撰會計錄以勾稽出納。又奏列清丈條例,釐兩京、山東、陜西勛戚莊田,清溢額、脫漏、詭借諸弊。又通行天下,得官民屯牧湖陂八十餘萬頃。民困賠累者,以其賦抵之。自正、嘉虛耗之後,至萬曆十年間,最稱富庶,學顏有力焉。然是時宮闈用度汰侈,多所徵索。學顏隨事納諫,得停發太倉銀十萬兩,減雲南黃金課一千兩,餘多弗能執爭。而金花銀歲增二十萬兩,遂為定額。人亦以是少之。

十一年四月,改兵部尚書,時方興內操,選內豎二千人雜廝養訓練,發太僕寺馬三千給之。學顏執不與馬,又請停內操,皆不聽。其年秋,車駕自山陵還,學顏上疏曰:「皇上恭奉聖母,扶輦前驅,拜祀陵園,考卜壽域,六軍將士十餘萬,部伍齊肅。惟內操隨駕軍士,進止自恣。前至涼水河,喧爭無紀律,奔逸衝突,上動天顏。今車駕已還,猶未解散。謹稽舊制,營軍隨駕郊祀,始受甲於內庫,事畢即還。宮中惟長隨內侍許佩弓矢。又律:不係宿衛軍士,持寸刃入宮殿門者,絞;入皇城門者,戍邊衛。祖宗防微弭亂之意甚深且遠。今皇城內被甲乘馬持鋒刃,科道不得糾巡,臣部不得檢閱。又招集廝養僕隸,出入禁苑,萬一驟起邪心,朋謀倡亂,嘩於內則外臣不敢入,嘩於夜則外兵不及知,嘩於都城白晝則曰天子親兵也,驅之不肯散,捕之莫敢攖。正德中,西城練兵之事,良可鑒也。」疏上,宦豎皆切齒,為蜚語中傷。神宗察知之,詰責主使者。學顏得免,然亦不能用也。

考滿,加太子少保。雲南岳鳳、罕虔平,進太子太保。時張居正既歿,朝論大異。初,御史劉臺以劾居正得罪,學顏復論其贓私。御史馮景隆論李成梁飾功,學顏亟稱成梁十大捷非妄,景隆坐貶斥。學顏故為居正所厚,與李成梁共事久,物論皆以學顏黨於居正、成梁。御史孫繼先、曾乾亨、給事中黃道瞻交章論學顏。學顏疏辯求去,又請留道瞻,不聽。明年,順天府通判周弘禴又論學顏交通太監張鯨,神宗皆黜之於外。學顏八疏乞休,許致仕去。二十六年;卒於家。贈少保。

張佳胤,字肖甫,銅梁人。嘉靖二十九年進士。知滑縣。劇盜高章者,詐為緹騎,直入官署,劫佳胤索帑金。佳胤色不變,偽書券貸金,悉署游徼名,召入立擒賊,由此知名。擢戶部主事,改職方,遷禮部郎中。以風霾考察,謫陳州同知。歷遷按察使。

隆慶五年冬,擢右僉都御史,巡撫應天十府。安慶兵變,坐勘獄辭不合,調南京鴻臚卿,就遷光祿。進右副都御史,巡撫保定,道聞喪歸。

萬曆七年,起故官,巡撫陜西。未上,改宣府。時青把都已服,其弟滿五大猶桀驁,所部八賴掠塞外史、車二部,總兵官麻錦擒之。佳胤命錦縛八賴將斬,而身馳赦之,八賴叩頭誓不敢犯邊。後與總督鄭洛計服滿五大。入為兵部右侍郎。

十年春,浙江巡撫吳善言奉詔減月餉。東、西二營兵馬文英、劉廷用等構黨大噪,縛毆善言。張居正以佳胤才,令兼右僉都御史代善言。甫入境,而杭民以行保甲故,亦亂。佳胤問告者曰:「亂兵與亂民合乎?」曰:「未也。」佳胤喜曰:「速驅之,尚可離而二也。」既至,民剽益甚。佳胤從數卒佯問民所苦,下令除之。眾益張,夜掠巨室,火光燭天。佳胤召游擊徐景星諭二營兵,令討亂民自贖。擒百五十人,斬其三之一。乃佯召文英、廷用,予冠帶。而密屬景星捕七人,并文英、廷用斬之。二亂悉定。帝優詔褒美。尋以左侍郎還部,錄功,加右都御史。

未幾,拜戎政尚書,尋兼右副都御史,總督薊、遼、保定軍務。以李成梁擊斬逞加努功,加太子少保。成梁破土蠻沈陽,復進太子太保。召還理部事。敘勞,予一品誥。御史許守恩劾佳胤營獲本兵,御史徐元復劾之,遂三疏謝病歸。越二年卒。贈少保。天啟初,謚襄憲。

殷正茂,字養實,歙人。嘉靖二十六年進士。由行人選兵科給事中。劾罷南京刑部侍郎沈應龍。歷廣西、雲南、湖廣兵備副使,遷江西按察使。隆慶初,古田僮韋銀豹、黃朝猛反。銀豹父朝威自弘治中敗官兵於三厄,殺副總兵馬俊、參議馬鉉,正德中嘗陷洛容。嘉靖時,銀豹及朝猛劫殺參政黎民衷,提督侍郎吳桂芳遣典史廖元招降之。遷元主簿以守,而銀豹數反覆。隆慶三年冬,廷議大征。擢正茂右僉都御史巡撫廣西。正茂與提督李遷調土、漢兵十四萬,令總兵俞大猷將之。先奪牛河、三厄險,諸軍連克東山鳳凰寨,蹙之潮水。廖元誘僮人斬朝猛,銀豹窮,令其黨陰斬貌類己者以獻。捷聞,進兵部右侍郎,巡撫如故。改古田為永寧州,設副使參將鎮守。未幾,僉事金柱捕得銀豹,正茂因自劾。詔磔銀豹京師,置正茂不問。

尋代遷提督兩廣軍務。當是時,群盜惠州藍一清、賴元爵,潮州林道乾、林鳳、諸良寶,瓊州李茂,處處屯結。廣中日告警,倭又數為害。正茂議守巡官畫地分守,而徙瀕海謫戍之民於雲南、川、湖,絕倭嚮導。乃令總兵官張元勳、參政江一麟等先後殺倭千餘,以次盡平諸盜。廣西巡撫郭應聘亦奏平懷遠、洛容瑤,語詳元勳及李錫傳。正茂以功累加兵部尚書兼右副都御史。倭復陷銅鼓、雙魚,元勳大破之儒峒;犯電白,正茂剿殺千餘人。嶺表略定。

萬曆三年,召為南京戶部尚書,以凌雲翼代。明年,改北部。疏請節用,又諫止採買珠寶。而張居正以正茂所餽鵝罽轉奉慈寧太后為坐褥。李幼孜與爭寵,嗾言官詹沂等劾之。遂屢引疾。六年,致仕歸。久之,起南京刑部尚書。居正卒之明年,御史張應詔言,正茂以金盤二,植珊瑚其中,高三尺許,賂居正,復取金珠、翡翠、象牙餽馮保及居正家人游七。正茂疏辨,請告,許之。二十年卒。

正茂在廣時,任法嚴,道將以下奉行惟謹。然性貪,歲受屬吏金萬計。初征古田,大學士高拱曰:「吾捐百萬金予正茂,縱乾沒者半,然事可立辦。」時以拱為善用人。

李遷,字子安,新建人。嘉靖二十年進士。隆慶四年官南京兵部右侍郎,以左侍郎總督兩廣。給事中光懋言兩廣向設提督,事權畫一,今兩巡撫相牽掣,不便。乃改遷提督兼巡撫廣東,而特命正茂為廣西巡撫。後遂為定制。以平銀豹功加右都御史。尋討惠、潮山寇,俘斬千二百餘級。召為刑部尚書。引疾歸,卒。謚恭介。遷出入中外三十年,不妄取一錢。年近七十,母終,廬墓。

凌雲翼,字洋山,太倉州人。嘉靖二十六年進士。授南京工部主事。隆慶中,累官右僉都御史,撫治鄖陽。疏論衛所兵消耗之弊,凡六事,多議行。萬曆元年,進右副都御史,巡撫江西。三遷兵部左侍郎兼右僉都御史,提督兩廣軍務,代殷正茂。時寇盜略盡,惟林鳳遁去。鳳初屯錢澳求撫,正茂不許,遂自彭湖奔東番魍港,為福建總兵官胡守仁所敗。是年冬,犯柘林、靖海、碣石,已,復犯福建。守仁追擊至淡水洋,沉其舟二十。賊失利,復入潮州。參政金淛諭降其黨馬志善、李成等,鳳夜遁。明年秋,把總王望高以呂宋番兵討平之。尋進徵羅旁。羅旁在德慶州上下江界、東西兩山間,延袤七百里。成化中,韓雍經略西山,頗安輯,惟東山瑤阻深箐剽掠,有司歲發卒戍守。正茂方建議大征,會遷去。雲翼乃大集兵,令兩廣總兵張元勛、李錫將之。四閱月,克巢五百六十,俘斬、招降四萬二千八百餘人。岑溪六十三山、七山、那留、連城諸處鄰境瑤、僮皆懼。賊首潘積善求撫,雲翼奏設官戍之。論功,加右都御史兼兵部侍郎。賜飛魚服。乃改瀧水縣為羅定州,設監司、參將。積患頓息。六年夏,與巡撫吳文華討平河池、咘咳、北三諸瑤,又捕斬廣東大廟諸山賊。嶺表悉定。召為南京工部尚書,就改兵部,以兵部尚書兼右副都御史總督漕運,巡撫淮、揚。河臣潘季馴召入,遂兼督河道。加太子少保。召為戎政尚書,以病歸。家居驕縱,給事、御史連章劾之。詔奪官,後卒。

雲翼有幹濟才。羅旁之役,繼正茂成功。然喜事好殺戮,為當時所譏。

贊曰:譚綸、王崇古諸人,受任巖疆,練達兵備,可與餘子俊、秦紘先後比跡。考其時,蓋張居正當國,究心於軍謀邊瑣。書疏往復,洞矚機要,委任責成,使得展布,是以各盡其材,事克有濟。觀於此,而居正之功不可泯也。


\end{pinyinscope}