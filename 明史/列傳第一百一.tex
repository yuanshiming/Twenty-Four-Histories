\article{列傳第一百一}

\begin{pinyinscope}
徐階弟陟子璠等高拱郭樸張居正曾孫同敞

徐階,字子升,松江華亭人。生甫周歲,墮眢井,出三日而蘇。五歲從父道括蒼,墮高嶺,衣掛於樹不死。人咸異之。嘉靖二年進士第三人。授翰林院編修,予歸娶。丁父憂,服除,補故官。階為人短小白皙,善容止。性穎敏,有權略,而陰重不泄。讀書為古文辭,從王守仁門人游,有聲士大夫間。

帝用張孚敬議,欲去孔子王號,易像為木主,籩豆禮樂皆有所損抑。下儒臣議,階獨持不可。孚敬召階盛氣詰之,階抗辯不屈。孚敬怒曰:「若叛我。」階正色曰:「叛生於附。階未嘗附公,何得言叛?」長揖出。斥為延平府推官。連攝郡事。出繫囚三百,毀淫祠,創鄉社學,捕劇盜百二十人。遷黃州府同知,擢浙江按察僉事,進江西按察副使,俱視學政。

皇太子出閣,召拜司經局洗馬兼翰林院侍講。丁母憂歸。服除,擢國子祭酒,遷禮部右侍郎,尋改吏部。故事,吏部率鐍門,所接見庶官不數語。階折節下之。見必深坐,咨邊腹要害,吏治民瘼。皆自喜得階意,願為用。尚書熊浹、唐龍、周用皆重階。階數署部事,所引用宋景、張岳、王道、歐陽德、范皆長者。用卒,聞淵代,自處前輩,取立斷。階意不樂,求出避之。命兼翰林院學士,教習庶吉士。尋掌院事,進禮部尚書。

帝察階勤,又所撰青詞獨稱旨,召直無逸殿。與大學士張治、李本俱賜飛魚服及上方珍饌、上尊無虛日。廷推吏部尚書,不聽,不欲階去左右也。階遂請立皇太子,不報。復連請之,皆不報。後當冠婚,復請先裕王,後景王,帝不懌。尋以推恩加太子太保。

俺答犯京,階請釋周尚文及戴綸、歐陽安等自效,報可。已,請帝還大內,召群臣計兵事,從之。中官陷寇歸,以俺答求貢書進。帝以示嚴嵩及階,召對便殿。嵩曰:「饑賊耳,不足患。」階曰:「傅城而軍,殺人若刈菅,何謂饑賊?」帝然之,問求貢書安在。嵩出諸袖曰:「禮部事也。」帝復問階。階曰:「寇深矣,不許恐激之怒,許則彼厚要我。請遣譯者紿緩之,我得益為備。援兵集,寇且走。」帝稱善者再。嵩、階因請帝出視朝。寇尋飽去,乃下階疏,弗許貢。

嵩怙寵弄權,猜害同列。既仇夏言置之死,而言嘗薦階,嵩以是忌之。初,孝烈皇后崩,帝欲祔之廟,念壓於先孝潔皇后,又睿宗入廟非公議,恐後世議祧,遂欲當己世預祧仁宗,以孝烈先祔廟,自為一世,下禮部議。階抗言女后無先入廟者,請祀之奉先殿。禮科都給事中楊思忠亦以為然。疏上,帝大怒。階皇恐謝罪,不能守前議。帝又使階往邯鄲落成呂仙祠。階不欲行,乃以議祔廟解,得緩期。至寇逼城,帝益懈,乃使尚書顧可學行,而內銜階。摘思忠元旦賀表誤,廷杖之百,斥為民,以怵階。嵩因謂階可間也,中傷之百方。一日獨召對,語及階,嵩徐曰:「階所乏非才,但多二心耳。」蓋以其嘗請立太子也。階危甚,度未可與爭,乃謹事嵩,而益精治齋詞迎帝意,左右亦多為地者。帝怒漸解。未幾,加少保,尋進兼文淵閣大學士,參預機務。密疏發咸寧侯仇鸞罪狀。嵩以階與鸞嘗同直,欲因鸞以傾階。及聞鸞罪發自階,乃愕然止,而忌階益甚。

帝既誅鸞,益重階,數與謀邊事。時議減鸞所益衛卒,階言:「不可減。又京營積弱之故,卒不在乏而在冗,宜精汰之,取其廩以資賞費。」又請罷提督侍郎孫禬。帝始格於嵩,久而皆用之。一品滿三載,進勳,為柱國,再進兼太子太傅、武英殿大學士。滿六載,兼食大學士俸,再錄子為中書舍人,加少傅。九載,改兼吏部尚書。賜宴禮部,璽書褒諭有加。帝雖重階,稍示形迹。嘗以五色芝授嵩,使練藥,謂階政本所關,不以相及。階皇恐請,乃得之。帝亦漸委任階,亞於嵩。

楊繼盛諭嵩罪,以二王為徵,下錦衣獄。嵩屬陸炳究主使者。階戒炳曰:「即不慎,一及皇子,如宗社何!」又為危語動嵩曰:「上惟二子,必不忍以謝公,所罪左右耳。公奈何顯結宮邸怨也。」嵩心雙懼,乃寢。倭躪東南,帝數以問階,階力主發兵。階又念邊卒苦饑,請收畿內麥數十萬石,自居庸輸宣府,紫荊輸大同。帝悅,密傳諭行之。楊繼盛之劾嵩也,嵩固疑階。趙錦、王宗茂劾嵩,階又議薄其罰。及是給事中吳時來、主事董傳策、張翀劾嵩不勝,皆下獄。傳策,階里人;時來、翀,階門生也。嵩遂疏辨,顯謂階主使,帝不聽。有所密詢,皆舍嵩而之階。尋加太子太師。

帝所居永壽宮災,徙居玉熙殿,隘甚,欲有所營建,以問嵩。嵩請還大內,帝不懌。問階,階請以三殿所餘材,責尚書雷禮營之,可計月而就。帝悅,如階議。命階子尚寶丞璠兼工部主事,董其役,十旬而功成。帝即日徙居之,命曰萬壽宮。以階忠,進少師,兼支尚書俸,予一子中書舍人。子璠亦超擢太常少卿。嵩乃日屈。嵩子世蕃貪橫淫縱狀亦漸聞,階乃令御史鄒應龍劾之。帝勒嵩致仕,擢應龍通政司參議。階遂代嵩為首輔。已而帝念嵩供奉勞,憐之。又以調去,忽忽不樂,乃降諭,欲退而修真且傳嗣,復責階等奈何以官與邪物,謂應龍也。階言:「退而傳嗣,臣等不敢奉命。應龍之轉,乃二部奉旨行之。」帝乃已。

帝以嵩在直久,而世蕃顧為奸於外,因命階無久直。階窺帝意,言茍為奸,在外猶在內,固請入直。帝以嵩直廬賜階。階榜三語其中曰:「以威福還主上,以政務還諸司,以用舍刑賞還公論。」於是朝士侃侃,得行其意。袁煒數出直,階請召與共擬旨。因言:「事同眾則公,公則百美基;專則私,私則百弊生。」帝頷之。階以張孚敬及嵩導帝猜刻,力反之,務以寬大開帝意。帝惡給事御史抨擊過當,欲有所行遣。階委曲調劑,得輕論。會問階知人之難,階對曰:「大奸似忠,大詐似信。惟廣聽納,則窮兇極惡,人為我攖之;深情隱慝,人為我發之。故聖帝明王,有言必察。即不實,小者置之,大則薄責而容之,以鼓來者。」帝稱善。言路益發舒。

寇由牆子嶺入,直趨通州。帝方祠釐,兵部尚書楊博不敢奏,謀之階,檄宣府總兵官馬芳、宣大總督江東入援。芳兵先至,階請亟賞之,又請重東權,俾統諸道兵。寇從通掠香河,階請亟備順義,而以奇兵邀之古北口。寇趨順義,不得入,乃走古北口。其後軍遇參將郭琥伏而敗,頗得其所掠人畜輜重。始帝怒博不早聞與總督楊選之任寇入也,欲罪之未發。階言:「博雖以祠釐禁不敢聞,而二鎮兵皆其所先檄。若選則非尾寇,乃送之出境耳。」帝竟誅選,不罪博。進階建極殿大學士。

袁煒以疾歸,道卒,階獨當國。屢請增閣臣,且乞骸骨。乃命嚴訥、李春芳入閣,而待階益隆。以一品十五載考,恩禮特厚,復賜玉帶、繡蟒、珍藥。帝手書問階疾,諄懇如家人,階益恭謹。帝或有所委,通夕不假寐,應制之文,未嘗踰頃刻期。帝日益愛階。階採輿論利便者,白而行之。嘉靖中葉,南北用兵。邊鎮大臣小不當帝指,輒逮下獄誅竄,閣臣復竊顏色為威福。階當國後,緹騎省減,詔獄漸虛,任事者亦得以功名終。於是論者翕然推階為名相。

嚴訥請告歸,命郭朴、高拱入閣,與春芳同輔政,事仍決於階。階數請立太子,不報。已而景王之籓,病薨,階奏奪景府所占陂田數萬頃還之民,楚人大悅。帝欲建雩壇及興都宮殿,階力止之。鄢懋卿驟增鹽課四十萬金,階風御史請復故額。方士胡大順等勸帝餌金丹,階力陳其矯誣狀,大順等尋伏法。帝服餌病躁,戶部主事海瑞極陳帝失,帝恚甚,欲即殺之,階力救得繫。帝病甚,忽欲幸興都,階力爭乃止。未幾,帝崩。階草遺詔,凡齋醮、土木、珠寶、織作悉罷,「大禮」大獄、言事得罪諸臣悉牽復之。詔下,朝野號慟感激,比之楊廷和所擬登極詔書,為世宗始終盛事云。

同列高拱、郭朴以階不與共謀,不樂。朴曰:「徐公謗先帝,可斬也。」拱初侍穆宗裕邸,階引之輔政,然階獨柄國,拱心不平。世宗不豫時,給事中胡應嘉嘗劾拱,拱疑階嗾之。隆慶元年,應嘉以救考察被黜者削籍去,言者謂拱修舊郤,脅階,斥應嘉。階復請薄應嘉罰,言者又劾拱。拱欲階擬杖,階從容譬解,拱益不悅。令御史齊康劾階,言其二子多干請及家人橫里中狀。階疏辯,乞休。九卿以下交章劾拱譽階,拱遂引疾歸。康竟斥,朴亦以言者攻之,乞身去。

給事、御史多起廢籍,恃階而強,言多過激。帝不能堪,諭階等處之。同列欲擬譴,階曰:「上欲譴,我曹當力爭,乃可導之譴乎。」請傳諭令省改。帝亦勿之罪。是年,詔翰林撰中秋宴致語,階言:「先帝未撤几筵,不可宴樂。」帝為罷宴。帝命中官分督團營,階力陳不可而止。南京振武營兵屢嘩,階欲汰之。慮其據孝陵不可攻也,先令操江都御史唐繼錄督江防兵駐陵傍,而徐下兵部分散之。事遂定。群小璫毆御史於午門,都御史王廷將糾之,階曰:「不得主名,劾何益?且慮彼先誣我。」乃使人以好語誘大璫,先錄其主名。廷疏上,乃分別逮治有差。階之持正應變,多此類也。

階所持諍,多宮禁事,行者十八九,中官多側目。會帝幸南海子,階諫,不從。方乞休,而給事中張齊以私怨劾階,階因請歸。帝意亦漸移,許之。賜馳驛。以春芳請,給夫廩,璽書褒美,行人導行,如故事。陛辭,賜白金、寶鈔、彩幣、襲衣。舉朝皆疏留,報聞而已。王廷後刺得張齊納賄事,劾戍之邊。階既行,春芳為首輔,未幾亦歸。拱再出,扼階不遺餘力。郡邑有司希拱指,爭齮晷階,盡奪其田,戍其二子。會拱復為居正所傾而罷,事乃解。萬曆十年,階年八十,詔遣行人存問,賜璽書、金幣。明年卒。贈太師,謚文貞。階立朝有相度,保全善類。嘉、隆之政,多所匡救。間有委蛇,亦不失大節。

階弟陟,嘉靖二十六年進士。累官南京刑部侍郎。子璠,以廕官太常卿;琨、瑛,尚寶卿。孫元春,進士,亦官太常卿。元春孫本高,官錦衣千戶,天啟中拒魏忠賢建祠奪職。崇禎改元,以薦起,累官左都督。諸生念祖,國變城破,與妻張,二妾陸、李,皆自縊。

高拱,字肅卿,新鄭人。嘉靖二十年進士。選庶吉士。踰年,授編修。穆宗居裕邸,出閣請讀,拱與檢討陳以勤並為侍講。世宗諱言立太子,而景王未之國,中外危疑。拱侍裕邸九年,啟王益敦孝謹,敷陳剴切。王甚重之,手書「懷賢忠貞」字賜焉。累遷侍講學士。

嚴嵩、徐階遞當國,以拱他日當得重,薦之世宗。拜太常卿,掌國子監祭酒事。四十一年,擢禮部左侍郎。尋改吏部,兼學士,掌詹事府事。進禮部尚書,召入直廬。撰齋詞,賜飛魚服。四十五年,拜文淵閣大學士,與郭朴同入閣。拱與朴皆階所薦也。

世宗居西苑,閣臣直廬在苑中。拱未有子,移家近直廬,時竊出。一日,帝不豫,誤傳非常,拱遽移具出。始階甚親拱,引入直。拱驟貴,負氣頗忤階。給事中胡應嘉,階鄉人也,以劾拱姻親自危。且瞷階方與拱郤,遂劾拱不守直廬,移器用於外。世宗病,勿省也。拱疑應嘉受階指,大憾之。

穆宗即位,進少保兼太子太保。階雖為首輔,而拱自以帝舊臣,數與之抗,朴復助之,階漸不能堪。而是時以勤與張居正皆入閣,居正亦侍裕邸講。階草遺詔,獨與居正計,拱心彌不平。會議登極賞軍及請上裁去留大臣事,階悉不從拱議,嫌益深。應嘉掌吏科,佐部院考察,事將竣,忽有所論救。帝責其牴牾,下閣臣議罰。朴奮然曰:「應嘉無人臣禮,當編氓。」階旁睨拱,見拱方怒,勉從之。言路謂拱以私怨逐應嘉,交章劾之。給事中歐陽一敬劾拱尤力。階於拱辯疏,擬旨慰留,而不甚譴言者。拱益怒,相與忿詆閣中。御史齊康為拱劾階,康坐黜。於是言路論拱者無虛日,南京科道至拾遺及之。拱不自安,乞歸,遂以少傅兼太子太傅、尚書、大學士養病去。隆慶元年五月也。拱以舊學蒙眷注,性強直自遂,頗快恩怨,卒不安其位去。既而階亦乞歸。

三年冬,帝召拱以大學士兼掌吏部事。拱乃盡反階所為,凡先朝得罪諸臣以遺詔錄用贈恤者,一切報罷。且上疏極論之曰:「《明倫大典》頒示已久。今議事之臣假託詔旨,凡議禮得罪者悉從褒顯,將使獻皇在廟之靈何以為享?先帝在天之靈何以為心?而陛下歲時入廟,亦何以對越二聖?臣以為未可。」帝深然之。法司坐方士王金等子弒父律。拱復上疏曰:「人君隕於非命,不得正終,其名至不美。先帝臨御四十五載,得歲六十有餘。末年抱病,經歲上賓,壽考令終,曾無暴遽。今謂先帝為王金所害,誣以不得正終,天下後世視先帝為何如主?乞下法司改議。」帝復然拱言,命減戍。拱之再出,專與階修郤,所論皆欲以中階重其罪。賴帝仁柔,弗之竟也。階子弟頗橫鄉里。拱以前知府蔡國熙為監司,簿錄其諸子,皆編戍。所以扼階者,無不至。逮拱去位,乃得解。

拱練習政體,負經濟才,所建白皆可行。其在吏部,欲遍識人才,授諸司以籍,使署賢否,志里姓氏,月要而歲會之。倉卒舉用,皆得其人。又以時方憂邊事,請增置兵部侍郎,以儲總督之選。由侍郎而總督,由總督而本兵,中外更番,邊材自裕。又以兵者專門之學,非素習不可應卒。儲養本兵,當自兵部司屬始。宜慎選司屬,多得智謀才力曉暢軍旅者,久而任之,勿遷他曹。他日邊方兵備督撫之選,皆於是取之。更各取邊地之人以備司屬,如銓司分省故事,則題覆情形可無扞格,并重其賞罰以鼓勵之。凡邊地有司,其責頗重,不宜付雜流及遷謫者。皆報可,著為令。拱又奏請科貢與進士並用,勿循資格。其在部考察,多所參伍,不盡憑文書為黜陟,亦不拘人數多寡,黜者必告以故,使眾咸服。古田瑤賊亂,用殷正茂總督兩廣。曰:「是雖貪,可以集事。」貴州撫臣奏土司安國亨將叛,命阮文中代為巡撫。臨行語之曰:「國亨必不叛,若往,無激變也。」即而如其言。以廣東有司多貪黷,特請旌廉能知府侯必登,以歷其餘。又言馬政、鹽政之官,名為卿、為使,而實以閑局視之,失人廢事,漸不可訓。惟教官驛遞諸司,職卑錄薄,遠道為難,宜銓注近地,以恤其私。詔皆從之。拱所經畫,皆此類也。

俺答孫把漢那吉來降,總督王崇古受之,請於朝,乞授以官。朝議多以為不可,拱與居正力主之。遂排眾議請於上,而封貢以成。事具崇古傳。進拱少師兼太子太師、尚書、大學士,改建極殿。拱以邊境稍寧,恐將士惰玩,復請敕邊臣及時閒暇,嚴為整頓,仍時遣大臣閱視。帝皆從之。遼東奏捷,進柱國、中極殿大學士。

尋考察科道,拱請與都察院同事。時大學士趙貞吉掌都察院,持議稍異同。給事中韓楫劾貞吉有所私庇。貞吉疑拱嗾之,遂抗章劾拱,拱亦疏辨。帝不直貞吉,令致仕去。拱既逐貞吉,專橫益著。尚寶卿劉奮庸上疏陰斥之,給事中曹大埜疏劾其不忠十事,皆謫外任。拱初持清操,後其門生、親串頗以賄聞,致物議。帝終眷拱不衰也。

始拱為祭酒,居正為司業,相友善,拱亟稱居正才。及是李春芳、陳以勤皆去,拱為首輔,居正肩隨之。拱性直而傲,同官殷士儋輩不能堪,居正獨退然下之,拱不之察也。馮保者,中人,性黠,次當掌司禮監,拱薦陳洪及孟沖,帝從之,保以是怨拱。而居正與保深相結。六年春,帝得疾,大漸,召拱與居正、高儀受顧命而崩。初,帝意專屬閣臣,而中官矯遺詔命與馮保共事。

神宗即位,拱以主上幼沖,懲中官專政,條奏請詘司禮權,還之內閣。又命給事中雒遒、程文合疏攻保,而己從中擬旨逐之。拱使人報居正,居正陽諾之,而私以語保。保訴於太后,謂拱擅權,不可容。太后頷之。明日,召群臣入,宣兩宮及帝詔。拱意必逐保也,急趨入。比宣詔,則數拱罪而逐之。拱伏地不能起,居正掖之出,僦騾車出宣武門。居正乃與儀請留拱,弗許。請得乘傳,許之。拱既去,保憾未釋。復構王大臣獄,欲連及拱,已而得寢。居家數年,卒。居正請復其官,與祭葬如例。中旨給半葬,祭文仍寓貶詞云。久之,廷議論拱功,贈太師,謚文襄,廕嗣子務觀為尚寶丞。

郭朴,字質夫,安陽人。嘉靖十四年進士。選庶吉士。累官禮部右侍郎,入直西苑。歷吏部左、右侍郎兼太子賓客。南京禮部缺尚書,帝憐朴久次,特加太子少保擢任之。朴辭曰:「幸與撰述,不欲遠離闕下。」帝大喜,命即以太子少保、禮部尚書、詹事府侍直如故。頃之,吏部尚書歐陽必進罷,即以朴代之。越二年,以父喪去。及嚴訥由吏部入閣,帝謀代者。時董份以工部尚書行吏部左侍郎事,方受帝眷,而為人貪狡無行。徐階慮其代訥,急言於帝,起朴故官。朴固請終制,不許。尋以考績,加太子太保。

四十五年,兼武英殿大學士,入預機務,與高拱並命。階早貴,權重,春芳、訥事之謹,至不敢講鈞禮。而朴與拱鄉里相得,事階稍倨,拱尤負才自恣。及世宗崩,階草遺詔,盡反時政之不便者。拱與朴不得與聞,大恚,兩人遂與階有隙。言路劾拱者多及朴。拱謝病歸,朴不自安,亦求去。帝固留之。時朴已加至少傅、太子太傅矣。御史龐尚鵬、凌儒等攻不止,遂三疏乞歸。家居二十餘年卒。贈太傅,謚文簡。

朴為人長者,兩典銓衡,以廉著。輔政二年無過。特以拱故,不容於朝,時頗有惜之者。張居正,字叔大,江陵人。少穎敏絕倫。十五為諸生。巡撫顧璘奇其文,曰:「國器也。」未幾,居正舉於鄉,璘解犀帶以贈,且曰:「君異日當腰玉,犀不足溷子。」嘉靖二十六年,居正成進士,改庶吉士。日討求國家典故。徐階輩皆器重之。授編修,請急歸,亡何還職。

居正為人,頎面秀眉目,鬚長至腹。勇敢任事,豪傑自許。然沉深有城府,莫能測也。嚴嵩為首輔,忌階,善階者皆避匿。居正自如,嵩亦器居正。遷右中允,領國子司業事。與祭酒高拱善,相期以相業。尋還理坊事,遷侍裕邸講讀。王甚賢之,邸中中官亦無不善居正者。而李芳數從問書義,頗及天下事。尋遷右諭德兼侍讀,進侍講學士,領院事。

階代嵩首輔,傾心委居正。世宗崩,階草遺詔,引與共謀。尋遷禮部右侍郎兼翰林院學士。月餘,與裕邸故講官陳以勤俱入閤,而居正為吏部左侍郎兼東閣大學士。尋充《世宗實錄》總裁,進禮部尚書兼武英殿大學士,加少保兼太子太保,去學士五品僅歲餘。時徐階以宿老居首輔,與李春芳皆折節禮士。居正最後入,獨引相體,倨見九卿,無所延納。間出一語輒中肯,人以是嚴憚之,重於他相。

高拱以很躁被論去,徐階亦去,春芳為首輔。亡何,趙貞吉入,易視居正。居正與故所善掌司禮者李芳謀,召用拱,俾領吏部,以扼貞吉,而奪春芳政。拱至,益與居正善。春芳尋引去,以勤亦自引,而貞吉、殷士儋皆為所構罷,獨居正與拱在,兩人益相密。拱主封俺答,居正亦贊之,授王崇古等以方略。加柱國、太子太傅。六年滿,加少傅、吏部尚書、建極殿大學士。以遼東戰功,加太子太師。和市成,加少師,餘如故。

初,徐階既去,令三子事居正謹。而拱銜階甚,嗾言路追論不已,階諸子多坐罪。居正從容為拱言,拱稍心動。而拱客構居正納階子三萬金,拱以誚居正。居正色變,指天誓,辭甚苦。拱謝不審,兩人交遂離。拱又與居正所善中人馮保卻。穆宗不豫,居正與保密處分後事,引保為內助,而拱欲去保。神宗即位,保以兩宮詔旨逐拱,事具拱傳,居正遂代拱為首輔。帝御平臺,召居正獎諭之,賜金幣及繡蟒斗牛服。自是賜賚無虛日。

帝虛己委居正,居正亦慨然以天下為己任,中外想望豐采。居正勸帝遵守祖宗舊制,不必紛更,至講學、親賢、愛民、節用皆急務。帝稱善。大計廷臣,斥諸不職及附麗拱者。復具詔召群臣廷飭之,百僚皆惕息。帝當尊崇兩宮。故事,皇后與天子生母並稱皇太后,而徽號有別。保欲媚帝生母李貴妃,風居正以並尊。居正不敢違,議尊皇后曰仁聖皇太后,皇貴妃曰慈聖皇太后,兩宮遂無別。慈聖徙乾清宮,撫視帝,內任保,而大柄悉以委居正。

居正為政,以尊主權、課吏職、信賞罰、一號令為主。雖萬里外,朝下而夕奉行。黔國公沐朝弼數犯法,當逮,朝議難之。居正擢用其子,馳使縛之,不敢動。既至,請貸其死,錮之南京。水曹河通,居正以歲賦逾春,發水橫溢,非決則涸,乃采水曹臣議,督艘卒以孟冬月兌運,及歲初畢發,少罹水患。行之久,太倉粟充盈,可支十年。互市饒馬,乃減太僕種馬,而令民以價納,太僕金亦積四百餘萬。又為考成法以責吏治。初,部院覆奏行撫按勘者,嘗稽不報。居正令以大小緩急為限,誤者抵罪。自是,一切不敢飾非,政體為肅。南京小奄醉辱給事中,言者請究治。居正謫其尤激者趙參魯於外以悅保,而徐說保裁抑其黨,毋與六部事。其奉使者,時令緹騎陰詗之。其黨以是怨居正,而心不附保。

居正以御史在外,往往凌撫臣,痛欲折之。一事小不合,詬責隨下,又敕其長加考察。給事中餘懋學請行寬大之政,居正以為風己,削其職。御史傅應禎繼言之,尤切。下詔獄,杖戍。給事中徐貞明等群擁入獄,視具橐饘,亦逮謫外。御史劉臺按遼東,誤奏捷。居正方引故事繩督之,臺抗章論居正專恣不法,居正怒甚。帝為下臺詔獄,命杖百,遠戍。居正陽具疏救之,僅奪其職。已,卒戍臺。由是諸給事御史益畏居正,而心不平。

當是時,太后以帝沖年,尊禮居正甚至,同列呂調陽莫敢異同。及吏部左侍郎張四維入,恂恂若屬吏,不敢以僚自處。

居正喜建豎,能以智數馭下,人多樂為之盡。俺答款塞,久不為害。獨小王子部眾十餘萬,東北直遼左,以不獲通互市,數入寇。居正用李成梁鎮遼,戚繼光鎮薊門。成梁力戰卻敵,功多至封伯,而繼光守備甚設。居正皆右之,邊境晏然。兩廣督撫殷正茂、凌雲翼等亦數破賊有功。浙江兵民再作亂,用張佳胤往撫即定,故世稱居正知人。然持法嚴。核驛遞,省冗官,清庠序,多所澄汰。公卿群吏不得乘傳,與商旅無別。郎署以缺少,需次者輒不得補。大邑士子額隘,艱於進取。亦多怨之者。

時承平久,群盜蝟起,至入城市劫府庫,有司恒諱之,居正嚴其禁。匿弗舉者,雖循吏必黜。得盜即斬決,有司莫敢飾情。盜邊海錢米盈數,例皆斬,然往往長繫或瘐死。居正獨亟斬之,而追捕其家屬。盜賊為衰止。而奉行不便者,相率為怨言,居正不恤也。

慈聖太后將還慈寧宮,諭居正謂:「我不能視皇帝朝夕,恐不若前者之向學、勤政,有累先帝付託。先生有師保之責,與諸臣異。其為我朝夕納誨,以輔台德,用終先帝憑几之誼。」因賜坐蟒、白金、綵幣。未幾,丁父憂。帝遣司禮中官慰問,視粥藥,止哭,絡繹道路,三宮膊贈甚厚。

戶部侍郎李幼孜欲媚居正,倡奪情議,居正惑之。馮保亦固留居正。諸翰林王錫爵、張位、趙志皋、吳中行、趙用賢、習孔教、沈懋學輩皆以為不可,弗聽。吏部尚書張瀚以持慰留旨,被逐去。御史曾士楚、給事中陳三謨等遂交章請留。中行、用賢及員外郎艾穆、主事沈思孝、進士鄒元標相繼爭之。皆坐廷杖,謫斥有差。時彗星從東南方起,長亙天。人情洶洶,指目居正,至懸謗書通衢。帝詔諭群臣,再及者誅無赦,謗乃已。於是使居正子編修嗣修與司禮太監魏朝馳傳往代司喪。禮部主事曹誥治祭,工部主事徐應聘治喪。居正請無造朝,以青衣、素服、角帶入閣治政,侍經筵講讀,又請辭歲俸。帝許之。及帝舉大婚禮,居正吉服從事。給事中李淶言其非禮,居正怒,出為僉事。時帝顧居正益重,常賜居正札,稱「元輔張少師先生」,待以師禮。

居正乞歸葬父,帝使尚寶少卿鄭欽、錦衣指揮史繼書護歸,期三月,葬畢即上道。仍命撫按諸臣先期馳賜璽書敦諭。範「帝賚忠良」銀印以賜之,如楊士奇、張孚敬例,得密封言事。戒次輔呂調陽等「有大事毋得專決,馳驛之江陵,聽張先生處分。」居正請廣內閣員,詔即令居正推。居正因推禮部尚書馬自強、吏部右侍郎申時行入閣。自強素迕居正,不自意得之,頗德居正,而時行與四維皆自暱於居正,居正乃安意去。帝及兩宮賜賚慰諭有加禮,遣司禮太監張宏供張餞郊外,百僚班送。所過地,有司節廚傳,治道路。遼東奏大捷,帝復歸功居正。使使馳諭,俾定爵賞。居正為條列以聞。調陽益內慚,堅臥,累疏乞休不出。

居正言母老不能冒炎暑,請俟清涼上道。於是內閣、兩都部院寺卿、給事、御史俱上章,請趣居正亟還朝。帝遣錦衣指揮翟汝敬馳傳往迎,計日以俟;而令中官護太夫人以秋日由水道行。居正所過,守臣率長跪,撫按大吏越界迎送,身為前驅。道經襄陽,襄王出候,要居正宴。故事,雖公侯謁王執臣禮,居正具,賓主而出。過南陽,唐王亦如之。抵郊外,詔遣司禮太監何進宴勞,兩宮亦各遣大璫李琦、李用宣諭,賜八寶金釘川扇、御膳、餅果、醪醴,百僚復班迎。入朝,帝慰勞懇篤,予假十日而後入閣,仍賜白金、彩幣、寶鈔、羊酒,因引見兩宮。及秋,魏朝奉居正母行,儀從煊赫,觀者如堵。比至,帝與兩宮復賜賚加等,慰諭居正母子,幾用家人禮。

時帝漸備六宮,太倉銀錢多所宣進。居正乃因戶部進御覽數目陳之,謂每歲入額不敵所出,請帝置坐隅時省覽,量入為出,罷節浮費。疏上,留中。帝復令工部鑄錢給用,居正以利不勝費止之。言官請停蘇、松織造,不聽。居正為面請,得損大半。復請停修武英殿工,及裁外戚遷官恩數,帝多曲從之。帝御文華殿,居正侍講讀畢,以給事中所上災傷疏聞,因請振。復言:「上愛民如子,而在外諸司營私背公,剝民罔上,宜痛鉗以法。而皇上加意撙節,於宮中一切用度、服御、賞賚、布施,裁省禁止。」帝首肯之,有所蠲貸。居正以江南貴豪怙勢及諸奸猾吏民善逋賦,選大吏精悍者嚴行督責。賦以時輸,國藏日益充,而豪猾率怨居正。

居正服將除,帝召吏部問期日,敕賜白玉帶、大紅坐蟒、盤蟒。御平臺召對,慰諭久之。使中官張宏引見慈慶、慈寧兩宮,皆有恩賚,而慈聖皇太后加賜御膳九品,使宏侍宴。

帝初即位,馮保朝夕視起居,擁護提抱有力,小扞格,即以聞慈聖。慈聖訓帝嚴,每切責之,且曰:「使張先生聞,奈何!」於是帝甚憚居正。及帝漸長,心厭之。乾清小璫孫海、客用等導上遊戲,皆愛幸。慈聖使保捕海、用,杖而逐之。居正復條其黨罪惡,請斥逐,而令司禮及諸內侍自陳,上裁去留。因勸帝戒遊宴以重起居,專精神以廣聖嗣,節賞賚以省浮費,卻珍玩以端好尚,親萬幾以明庶政,勤講學以資治理。帝迫於太后,不得已,皆報可,而心頗嗛保、居正矣。

帝初政,居正嘗纂古治亂事百餘條,繪圖,以俗語解之,使帝易曉。至是,復屬儒臣紀太祖列聖《寶訓》、《寶錄》分類成書,凡四十:曰創業艱難,曰勵精圖治,曰勤學,曰敬天,曰法祖,曰保民,曰謹祭祀,曰崇孝敬,曰端好尚,曰慎起居,曰戒遊佚,曰正宮闈,曰教儲貳,曰睦宗籓,曰親賢臣,曰去奸邪,曰納諫,曰理財,曰守法,曰儆戒,曰務實,曰正紀綱,曰審官,曰久任,曰重守令,曰馭近習,曰待外戚,曰重農桑,曰興教化,曰明賞罰,曰信詔令,曰謹名分,曰裁貢獻,曰慎賞賚,曰敦節儉,曰慎刑獄,曰褒功德,曰屏異端,曰節武備,曰御戎狄。其辭多警切,請以經筵之暇進講。又請立起居注,紀帝言動與朝內外事,日用翰林官四員入直,應制詩文及備顧問。帝皆優詔報許。

居正自奪情後,益偏恣。其所黜陟,多由愛憎。左右用事之人多通賄賂。馮保客徐爵擢用至錦衣衛指揮同知,署南鎮撫。居正三子皆登上第。蒼頭游七入貲為官,勛戚文武之臣多與往還,通姻好。七具衣冠報謁,列於士大夫。世以此益惡之。

亡何,居正病。帝頻頒敕諭問疾,大出金帛為醫藥資。四閱月不愈,百官並齋醮為祈禱。南都、秦、晉、楚、豫諸大吏,亡不建醮。帝令四維等理閣中細務,大事即家令居正平章。居正始自力,後憊甚不能遍閱,然尚不使四維等參之。及病革,乞歸。上復優詔慰留,稱「太師張太岳先生」。居正度不起,薦前禮部尚書潘晟及尚書梁夢龍、侍郎餘有丁、許國、陳經邦,已,復薦尚書徐學謨、曾省吾、張學顏、侍郎王篆等可大用。帝為黏御屏。晟,馮保所受書者也,強居正薦之。時居正已昏甚,不能自主矣。及卒,帝為輟朝,諭祭九壇,視國公兼師傅者。居正先以六載滿,加特進中極殿大學士;以九載滿,加賜坐蟒衣,進左柱國,蔭一子尚寶丞;以大婚,加歲祿百石,錄子錦衣千戶為指揮僉事;以十二載滿,加太傅;以遼東大捷,進太師,益歲祿二百石,子由指揮僉事進同知。至是,贈上柱國,謚文忠,命四品京卿、錦衣堂上官、司禮太監護喪歸葬。於是四維始為政,而與居正所薦引王篆、曾省吾等交惡。

初,帝所幸中官張誠見惡馮保,斥於外,帝使密詗保及居正。至是,誠復入,悉以兩人交結恣橫狀聞,且謂其寶藏踰天府。帝心動。左右亦浸言保過惡,而四維門人御史李植極論徐爵與保挾詐通奸諸罪。帝執保禁中,逮爵詔獄。謫保奉御居南京,盡籍其家金銀珠寶巨萬計。帝疑居正多蓄,益心艷之。言官劾篆、省吾,并劾居正,篆、省吾俱得罪。新進者益務攻居正。詔奪上柱國、太師,再奪謚。居正諸所引用者,斥削殆盡。召還中行、用賢等,遷官有差。劉臺贈官,還其產。御史羊可立復追論居正罪,指居正構遼庶人憲節獄。庶人妃因上疏辯冤,且曰:「庶人金寶萬計,悉入居正。」帝命司禮張誠及侍郎丘橓偕錦衣指揮、給事中籍居正家。誠等將至,荊州守令先期錄人口,錮其門,子女多遁避空室中。比門啟,餓死者十餘輩。誠等盡發其諸子兄弟藏,得黃金萬兩,白金十餘萬兩。其長子禮部主事敬修不勝刑,自誣服寄三十萬金於省吾、篆及傅作舟等,尋自縊死。事聞,時行等與六卿大臣合疏,請少緩之;刑部尚書潘季馴疏尤激楚。詔留空宅一所、田十頃,贍其母。而御史丁此呂復追論科場事,謂高啟愚以舜、禹命題,為居正策禪受。尚書楊巍等與相駁。此呂出外,啟愚削籍。後言者復攻居正不已。詔盡削居正官秩,奪前所賜璽書、四代誥命,以罪狀示天下,謂當剖棺戮死而姑免之。其弟都指揮居易、子編修嗣修,俱發戍煙瘴地。

終萬曆世,無敢白居正者。熹宗時,廷臣稍稍追述之。而鄒元標為都御史,亦稱居正。詔復故官,予葬祭。崇禎三年,禮部侍郎羅喻義等訟居正冤。帝令部議,復二廕及誥命。十三年,敬修孫同敞請復武廕,併復敬修官。帝授同敞中書舍人,而下部議敬修事。尚書李日宣等言:「故輔居正,受遺輔政,事皇祖者十年,肩勞任怨,舉廢飭弛,弼成萬曆初年之治。其時中外乂安,海內殷阜,紀綱法度,莫不修明。功在社稷,日久論定,人益追思。」帝可其奏,復敬修官。

同敞負志節,感帝恩,益自奮。十五年,奉敕慰問湖廣諸王,因令調兵雲南。未復命,兩京相繼失,走詣福建。唐王亦念居正功,復其錦衣世廕,授同敞指揮僉事。尋奉使湖南。聞汀州破,依何騰蛟於武岡。永明王用廷臣薦,改授同敞侍讀學士。為總兵官劉承胤所惡,言翰林、吏部、督學必用甲科,乃改同敞尚寶卿。以大學士瞿式耜薦,擢兵部右侍郎兼翰林侍讀學士,總督諸路軍務。

同敞有文武材,意氣慷慨。每出師,輒躍馬為諸將先。或敗奔,同敞危坐不去,諸將復還戰,或取勝。軍中以是服同敞。大將王永祚等久圍永州,大兵赴救,胡一青率眾迎敵,戰敗。同敞馳至全州,檄楊國棟兵策應,乃解去。順治七年,大兵破嚴關,諸將盡棄桂林走。城中虛無人,獨式耜端坐府中。適同敞自靈川至,見式耜。式耜曰:「我為留守,當死此。子無城守責,盍去諸?」同敞正色曰:「昔人恥獨為君子,公顧不許同敞共死乎?」式耜喜,取酒與飲,明燭達旦。侵晨被執,諭之降,不從。令為僧,亦不從。乃幽之民舍。雖異室,聲息相聞,兩人日賦詩倡和。閱四十餘日,整衣冠就刃,顏色不變。既死,同敞尸植立,首墜躍而前者三,人皆闢易。

而居正第五子允修,字建初,廕尚寶丞。崇禎十七年正月,張獻忠掠荊州,允修題詩於壁,不食而死。

贊曰:徐階以恭勤結主知,器量深沉。雖任智數,要為不失其正。高拱才略自許,負氣凌人。及為馮保所逐,柴車即路。傾輒相尋,有自來已。張居正通識時變,勇於任事。神宗初政,起衰振隳,不可謂非幹濟才。而威柄之操,幾於震主,卒致禍發身後。《書》曰「臣罔以寵利居成功」,可弗戒哉!


\end{pinyinscope}