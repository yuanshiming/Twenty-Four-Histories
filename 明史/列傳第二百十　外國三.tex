\article{列傳第二百十 外國三}

\begin{pinyinscope}
○日本

日本,古倭奴國。唐咸亨初,改日本,以近東海日出而名也。地環海,惟東北限大山,有五畿、七道、三島,共一百十五州,統五百八十七郡。其小國數十,皆服屬焉。國小者百里,大不過五百里。戶小者千,多不過一二萬。國主世以王為姓,群臣亦世官。宋以前皆通中國,朝貢不絕,事具前史。惟元世祖數遣使趙良弼招之不至,乃命忻都、範文虎等帥舟師十萬征之,至五龍山遭暴風,軍盡沒。後屢招不至,終元世未相通也。

明興,高皇帝即位,方國珍、張士誠相繼誅服。諸豪亡命,往往糾島人入寇山東濱海州縣。洪武二年三月,帝遣行人楊載詔諭其國,且詰以入寇之故,謂:「宜朝則來廷,不則修兵自固。倘必為寇盜,即命將徂征耳,王其圖之。」日本王良懷不奉命,復寇山東,轉掠溫、台、明州旁海民,遂寇福建沿海郡。

三年三月又遣萊州府同知趙秩責讓之,泛海至析木崖,入其境,守關者拒弗納。秩以書抵良懷,良懷延秩入。諭以中國威德,而詔書有責其不臣語。良懷曰:「吾國雖處扶桑東,未嘗不慕中國。惟蒙古與我等夷,乃欲臣妾我。我先王不服,乃使其臣趙姓者訹我以好語,語未既,水軍十萬列海岸矣。以天之靈,雷霆波濤,一時軍盡覆。今新天子帝中夏,天使亦趙姓,豈蒙古裔耶?亦將訹我以好語而襲我也。」自左右將兵之。秩不為動,徐曰:「我大明天子神聖文武,非蒙古比,我亦非蒙古使者後。能兵,兵我。」良懷氣沮,下堂延秩,禮遇甚優。遣其僧祖來奉表稱臣,貢馬及方物,且送還明、臺二郡被掠人口七十餘,以四年十月至京。太祖嘉之,宴賚其使者,念其俗佞佛,可以西方教誘之也,乃命僧祖闡、克勤等八人送使者還國,賜良懷《大統曆》及文綺、紗羅。是年掠溫州。五年寇海鹽、水敢浦,又寇福建海上諸郡。六年以於顯為總兵官,出海巡倭,倭寇萊、登。祖闡等既至,為其國演教,其國人頗敬信。而王則傲慢無禮,拘之二年,以七年五月還京。倭寇膠州。

時良懷年少,有持明者,與之爭立,國內亂。是年七月,其大臣遣僧宣聞溪等齎書上中書省,貢馬及方物,而無表。帝命卻之,仍賜其使者遣還。未幾,其別島守臣氏久遣僧奉表來貢。帝以無國王之命,且不奉正朔,亦卻之,而賜其使者,命禮臣移牒,責以越分私貢之非。又以頻入寇掠,命中書移牒責之。乃以九年四月,遣僧圭廷用等來貢,且謝罪。帝惡其表詞不誠,降詔戒諭,宴賚使者如制。十二年來貢。十三年復貢,無表,但持其征夷將軍源義滿奉丞相書,書辭又倨。乃卻其貢,遣使齎詔譙讓。十四年復來貢,帝再卻之,命禮官移書責其王,並責其征夷將軍,示以欲征之意。良懷上言:

臣聞三皇立極,五帝禪宗,惟中華之有主,豈夷狄而無君。乾坤浩蕩,非一主之獨權,宇宙寬洪,作諸邦以分守。蓋天下者,乃天下之天下,非一人之天下也。臣居遠弱之倭,褊小之國,城池不滿六十,封疆不足三千,尚存知足之心。陛下作中華之主,為萬乘之君,城池數千餘,封疆百萬里,猶有不足之心,常起滅絕之意。夫天發殺機,移星換宿。地發殺機,龍蛇走陸。人發殺機,天地反覆。昔堯、舜有德,四海來賓。湯、武施仁,八方奉貢。

臣聞天朝有興戰之策,小邦亦有禦敵之圖。論文有孔、孟道德之文章,論武有孫、吳韜略之兵法。又聞陛下選股肱之將,起精銳之師,來侵臣境。水澤之地,山海之洲,自有其備,豈肯跪途而奉之乎?順之未必其生,逆之未必其死。相逢賀蘭山前,聊以博戲,臣何懼哉。倘君勝臣負,且滿上國之意。設臣勝君負,反作小邦之差。自古講和為上,罷戰為強,免生靈之塗炭,拯黎庶之艱辛。特遣使臣,敬叩丹陛,惟上國圖之。

帝得表慍甚,終鑑蒙古之轍,不加兵也。

十六年,倭寇金鄉、平陽。十九年遣使來貢,卻之。明年命江夏侯周德興往福建濱海四郡,相視形勢。衛所城不當要害者移置之,民戶三丁取一,以充戍卒,乃築城一十六,增巡檢司四十五,得卒萬五千餘人。又命信國公湯和行視浙東、西諸郡,整飭海防,乃築城五十九。民戶四丁以上者以一為戍卒,得五萬八千七百餘人,分戍諸衛,海防大飭。閏六月命福建備海舟百艘,廣東倍之,以九月會浙江捕倭,既而不行。

先是,胡惟庸謀逆,欲藉日本為助。乃厚結寧波衛指揮林賢,佯奏賢罪,謫居日本,令交通其君臣。尋奏復賢職,遣使召之,密致書其王,借兵助己。賢還,其王遣僧如瑤率兵卒四百餘人,詐稱入貢,且獻巨燭,藏火藥、刀劍其中。既至,而惟庸已敗,計不行。帝亦未知其狡謀也。越數年,其事始露,乃族賢,而怒日本特甚,決意絕之,專以防海為務。然其時王子滕祐壽者,來入國學,帝猶善待之。二十四年五月特授觀察使,留之京師。後著《祖訓》,列不征之國十五,日本與焉。自是,朝貢不至,而海上之警亦漸息。

成祖即位,遣使以登極詔諭其國。永樂元年又遣左通政趙居任、行人張洪偕僧道成往。將行,而其貢使已達寧波。禮官李至剛奏:「故事,番使入中國,不得私攜兵器鬻民。宜敕所司核其舶,諸犯禁者悉籍送京師。」帝曰:「外夷修貢,履險蹈危,來遠,所費實多。有所齎以助資斧,亦人情,豈可概拘以禁令。至其兵器,亦準時直市之,毋阻向化。」十月,使者至,上王源道義表及貢物。帝厚禮之,遣官偕其使還,賚道義冠服、龜鈕金章及錦綺、紗羅。

明年十一月來賀冊立皇太子。時對馬、臺岐諸島賊掠濱海居民,因諭其王捕之。王發兵盡殲其眾,縶其魁二十人,以三年十一月獻於朝,且修貢。帝益嘉之,遣鴻臚寺少卿潘賜偕中官王進賜其王九章冕服及錢鈔、錦綺加等,而還其所獻之人,令其國自治之。使者至寧波,盡置其人於甑,烝殺之。明年正月又遣侍郎俞士吉齎璽書褒嘉,賜賚優渥。封其國之山為壽安鎮國之山,御製碑文,立其上。六月,使來謝,賜冕服。五年、六年頻入貢,且獻所獲海寇。使還,請賜仁孝皇后所製《勸善》、《內訓》二書,即命各給百本。十一月再貢。十二月,其國世子源義持遣使來告父喪,命中官周全往祭,賜謚恭獻,且致賻。又遣官齎敕,封義持為日本國王。時海上復以倭警告,再遣官諭義持剿捕。

八年四月,義持遣使謝恩,尋獻所獲海寇,帝嘉之。明年二月復遣王進齎敕褒賚,收市物貨。其君臣謀阻進不使歸,進潛登舶,從他道遁還。自是,久不貢。是年,倭寇盤石。十五年,倭寇松門、金鄉、平陽。有捕倭寇數十人至京者。廷臣請正法。帝曰:「威之以刑,不若懷之以德,宜還之。」乃命刑部員外郎呂淵等齎敕責讓,令悔罪自新。中華人被掠者,亦令送還。明年四月,其王遣使隨淵等來貢,謂:「海寇旁午,故貢使不能上達。其無賴鼠竊者,實非臣所知。願貸罪,容其朝貢。」帝以其詞順,許之,禮使者如故,然海寇猶不絕。

十七年,倭船入王家山島,都督劉榮率精兵疾馳入望海堝。賊數千人分乘二十舟,直抵馬雄島,進圍望海堝。榮發伏出戰,奇兵斷其歸路。賊奔櫻桃園,榮合兵攻之,斬首七百四十二,生擒八百五十七。召榮至京,封廣寧伯。自是,倭不敢窺遼東。二十年,倭寇象山。

宣德七年正月,帝念四方蕃國皆來朝,獨日本久不貢,命中官柴山往琉球,令其王轉諭日本,賜之敕。明年夏,王源義教遣使來。帝報之,賚白金、彩幣。秋復至。十年十月以英宗嗣位,遣使來貢。

正統元年二月,使者還,賚王及妃銀幣。四月,工部言:「宣德間,日本諸國皆給信符勘合,今改元伊始,例當更給。」從之。四年五月,倭船四十艘連破台州桃渚、寧波大嵩二千戶所,又陷昌國衛,大肆殺掠。八年五月,寇海寧。先是,洪熙時,黃巖民周來保、龍巖民鐘普福困於徭役,叛入倭。倭每來寇,為之鄉導。至是,導倭犯樂清,先登岸偵伺。俄倭去,二人留村中丐食,被獲,置極刑,梟其首於海上。倭性黠,時載方物、戎器,出沒海濱,得間則張其戎器而肆侵掠,不得則陳其方物而稱朝貢,東南海濱患之。

景泰四年入貢,至臨清,掠居民貨。有指揮往詰,歐幾死。所司請執治,帝恐失遠人心,不許。先是,永樂初,詔日本十年一貢,人止二百,船止二艘,不得攜軍器,違者以寇論。乃賜以二舟,為入貢用,後悉不如制。宣德初,申定要約,人毋過三百,舟毋過三艘。而倭人貪利,貢物外所攜私物增十倍,例當給直。禮官言:「宣德間所貢硫黃、蘇木、刀扇、漆器之屬,估時直給錢鈔,或折支布帛,為數無多,然已大獲利。今若仍舊制,當給錢二十一萬七千,銀價如之。宜大減其直,給銀三萬四千七百有奇。」從之。使臣不悅,請如舊制。詔增錢萬,猶以為少,求增賜物。詔增布帛千五百,終怏怏去。

天順初,其王源義政以前使臣獲罪天朝,蒙恩宥,欲遣使謝罪而不敢自達,移書朝鮮王令轉請,朝鮮以聞。廷議敕朝鮮核實,令擇老成識大體者充使,不得仍前肆擾,既而貢使亦不至。

成化四年夏,乃遣使貢馬謝恩,禮之如制。其通事三人,自言本寧波村民,幼為賊掠,市與日本,今請便道省祭,許之。戒其勿同使臣至家,引中國人下海。十一月,使臣清啟復來貢,傷人於市。有司請治其罪,詔付清啟,奏言犯法者當用本國之刑,容還國如法論治。且自服不能鈐束之罪,帝俱赦之。自是,使者益無忌。十三年九月來貢,求《佛祖統紀》諸書,詔以《法苑珠林》賜之。使者述其王意,請於常例外增賜,命賜錢五萬貫。二十年十一月復貢。弘治九年三月,王源義高遣使來,還至濟寧,其下復持刀殺人。所司請罪之,詔自今止許五十人入都,餘留舟次,嚴防禁焉。十八年冬來貢,時武宗已即位,命如故事,鑄金牌勘合給之。

正德四年冬來貢。禮官言:「明年正月,大祀慶成宴。朝鮮陪臣在展東第七班,日本向無例,請殿西第七班。」從之。禮官又言:「日本貢物向用舟三,今止一,所賜銀幣,宜如其舟之數。且無表文,賜敕與否,請上裁。」命所司移文答之。五年春,其王源義澄遣使臣宋素卿來貢,時劉瑾竊柄,納其黃金千兩,賜飛魚服,前所未有也。素卿,鄞縣朱氏子,名縞,幼習歌唱。倭使見,悅之,而縞叔澄負其直,因以縞償。至是,充正使,至蘇州,澄與相見。後事覺,法當死,劉瑾庇之,謂澄已自首,並獲免。七年,義澄使復來貢,浙江守臣言:「今畿輔、山東盜充斥,恐使臣遇之為所掠,請以貢物貯浙江官庫,收其表文送京師。」禮官會兵部議,請令南京守備官即所在宴賚,遣歸,附進方物,皆予全直,毋阻遠人向化心。從之。

嘉靖二年五月,其貢使宗設抵寧波。未幾,素卿偕瑞佐復至,互爭真偽。素卿賄市舶大監賴恩,宴時坐素卿於宗設上,船後至又先為驗發。宗設怒,與之斗,殺瑞佐,焚其舟,追素卿至紹興城下,素卿竄匿他所免。凶黨還寧波,所過焚掠,執指揮袁璡,奪船出海。都指揮劉錦追至海上,戰沒。巡按御史歐珠以聞,且言:「據素卿狀,西海路多羅氏義興者,向屬日本統轄,無入貢例。因貢道必經西海,正德朝勘合為所奪。我不得已,以弘治朝勘合,由南海路起程,比至寧波,因詰其偽,致啟釁。」章下禮部,部議:「素卿言未可信,不宜聽入朝。但釁起宗設,素卿之黨被殺者多,其前雖有投番罪,已經先朝宥赦,毋容問。惟宣諭素卿還國,移咨其王,令察勘合有無,行究治。」帝已報可,御史熊蘭、給事張翀交章言:「素卿罪重不可貸,請並治賴恩及海道副使張芹、分守參政朱鳴陽、分巡副使許完、都指揮張浩。閉關絕貢,振中國之威,寢狡寇之計。」事方議行,會宗設黨中林、望古多羅逸出之舟,為暴風飄至朝鮮。朝鮮人擊斬三十級,生擒二賊以獻。給事中夏言因請逮赴浙江,會所司與素卿雜治,因遣給事中劉稍、御史王道往。至四年,獄成,素卿及中林、望古多羅並論死,繫獄。久之,皆瘐死。時有琉球使臣鄭繩歸國,命傳諭日本以擒獻宗設,還袁璡及海濱被掠之人,否則閉關絕貢,徐議征討。

九年,琉球使臣蔡瀚者,道經日本,其王源義晴附表言:「向因本國多事,干戈梗道。正德勘合不達東都,以故素卿捧弘治勘合行,乞貸遣。望並賜新勘合、金印,修貢如常。」禮官驗其文,無印篆,言:「倭譎詐難信,宜敕琉球王傳諭,仍遵前命。」十八年七月,義晴貢使至寧波,守臣以聞。時不通貢者已十七年,敕巡按御史督同三司官核,果誠心效順,如制遣送,否則卻回,且嚴居民交通之禁。明年二月,貢使碩鼎等至京申前請,乞賜嘉靖新勘合,還素卿及原留貢物。部議:「勘合不可遽給,務繳舊易新。貢期限十年,人不過百,舟不過三,餘不可許。」詔如議。二十三年七月復來貢,未及期,且無表文。部臣謂不當納,卻之。其人利互市,留海濱不去。巡按御史高節請治沿海文武將吏罪,嚴禁奸豪交通,得旨允行。而內地諸奸利其交易,多為之囊橐,終不能盡絕。

二十六年六月,巡按御史楊九澤言:「浙江寧、紹、台、溫皆濱海,界連福建福、興、漳、泉諸郡,有倭患,雖設衛所城池及巡海副使、備倭都指揮,但海寇出沒無常,兩地官弁不能通攝,制禦為難。請如往例,特遣巡視重臣,盡統海濱諸郡,庶事權歸一,威令易行。」廷議稱善,乃命副都御史朱紈巡撫浙江兼制福、興、漳、泉、建寧五府軍事。未幾,其王義晴遣使周良等先期來貢,用舟四,人六百,泊於海外,以待明年貢期。守臣沮之,則以風為解。十一月事聞,帝以先期非制,且人船越額,敕守臣勒回。十二月,倭賊犯寧、台二郡,大肆殺掠,二郡將吏並獲罪。明年六月,周良復求貢,紈以聞。禮部言:「日本貢期及舟與人數雖違制,第表辭恭順,去貢期亦不遠,若概加拒絕,則航海之勞可憫,若稍務含容,則宗設、素卿之事可鑒。宜敕紈循十八年例,起送五十人,餘留嘉賓館,量加犒賞,諭令歸國。若互市防守事,宜在紈善處之。」報可。紈力言五十人過少,乃令百人赴都。部議但賞百人,餘罷勿賞。良訴貢舟高大。勢須五百人。中國商舶入海,往往藏匿島中為寇,故增一舟防寇,非敢違制。部議量增其賞,且謂:「百人之制,彼國勢難遵行,宜相其貢舟大小,以施禁令。」從之。

日本故有孝、武兩朝勘合幾二百道,使臣前此入貢請易新者,而令繳其舊。至是良持弘治勘合十五道,言其餘為素卿子所竊,捕之不獲。正德勘合留十五道為信,而以四十道來還。部議令異時悉繳舊,乃許易新,亦報可。當是時,日本王雖入貢,其各島諸倭歲常侵掠,濱海奸民又往往勾之。紈乃嚴為申禁,獲交通者,不俟命輒以便宜斬之。由是,浙、閩大姓素為倭內主者,失利而怨。紈又數騰疏於朝,顯言大姓通倭狀,以故閩、浙人皆惡之,而閩尤甚。巡按御史周亮,閩產也,上疏詆紈,請改巡撫為巡視,以殺其權。其黨在朝者左右之,竟如其請。又奪紈官。羅織其擅殺罪,紈自殺。自是不置巡撫者四年,海禁復弛,亂益滋甚。

祖制,浙江設市舶提舉司,以中官主之,駐寧波。海舶至則平其直,制馭之權在上。及世宗,盡撤天下鎮守中官,並撤市舶,而濱海奸人遂操其利。初市猶商主之,及嚴通番之禁,遂移之貴官家,負其直者愈甚。索之急,則以危言嚇之,或又以好言紿之,謂我終不負若直。倭喪其貲不得返,已大恨,而大奸若汪直、徐海、陳東、麻葉輩素窟其中,以內地不得逞,悉逸海島為主謀。倭聽指揮,誘之入寇。海中巨盜,遂襲倭服飾、旂號,並分艘掠內地,無不大利,故倭患日劇,於是廷議復設巡撫。三十一年七月以僉都御史王忬任之,而勢已不可撲滅。

明初,沿海要地建衛所,設戰船,董以都司、巡視、副使等官,控制周密。迨承平久,船敝伍虛。及遇警,乃募漁船以資哨守。兵非素練,船非專業,見寇舶至,輒望風逃匿,而上又無統率御之。以故賊帆所指,無不殘破。三十二年三月,汪直勾諸倭大舉入寇,連艦數百,蔽海而至。浙東、西,江南、北,濱海數千里,同時告警。破昌國衛。四月犯太倉,破上海縣,掠江陰,攻乍浦。八月劫金山衛,犯崇明及常熟、嘉定。三十三年正月自太倉掠蘇州,攻松江,復趨江北,薄通、泰。四月陷嘉善,破崇明,復薄蘇州,入崇德縣。六月由吳江掠嘉興,還屯柘林。縱橫來往,若入無人之境,忬亦不能有所為。未幾,忬改撫大同,以李天寵代,又命兵部尚書張經總督軍務。乃大徵兵四方,協力進剿。是時,倭以川沙窪、柘林為巢,抄掠四出。明年正月,賊奪舟犯乍浦、海寧,陷崇德,轉掠塘棲、新市、橫塘、雙林等處,攻德清縣。五月復合新倭,突犯嘉興,至王江涇,乃為經擊斬千九百餘級,餘奔柘林。其他倭復掠蘇州境,延及江陰、無錫,出入太湖。大抵真倭十之三,從倭者十之七。倭戰則驅其所掠之人為軍鋒,法嚴,人皆致死,而官軍素懦怯,所至潰奔。帝乃遣工部侍郎趙文華督察軍情。文華顛倒功罪,諸軍益解體。經、天寵並被逮,代以周珫、胡宗憲。逾月,珫罷,代以楊宜。

時賊勢蔓延,江浙無不蹂躪。新倭來益眾,益肆毒。每自焚其舟,登岸劫掠。自杭州北新關西剽淳安,突徽州歙縣,至績谿、旌德,過涇縣,趨南陵,遂達蕪湖。燒南岸,奔太平府,犯江寧鎮,徑侵南京。倭紅衣黃蓋,率眾犯大安德門,及夾岡,乃趨秣陵關而去,由溧水流劫溧陽、宜興。聞官兵自太湖出,遂越武進,抵無錫,駐惠山。一晝夜奔百八十餘里,抵滸墅。為官軍所圍,追及於楊林橋,殲之。是役也,賊不過六七十人,而經行數千里,殺戮戰傷者幾四千人,歷八十餘日始滅,此三十四年九月事也。

應天巡撫曹邦輔以捷聞,文華忌其功。以倭之巢於陶宅也,乃大集浙、直兵,與宗憲親將之。又約邦輔合剿,分道並進,營於松江之甎橋。倭悉銳來衝,遂大敗,文華氣奪,賊益熾。十月,倭自樂清登岸,流劫黃巖、仙居、奉化、餘姚、上虞,被殺擄者無算。至乘縣乃殲之,亦不滿二百人,顧深入三府,歷五十日始平。其先一枝自山東日照流劫東安衛,至淮安、贛榆、沭陽、桃源,至清河阻雨,為徐、邳官兵所殲,亦不過數十人,流害千里,殺戮千餘,其悍如此。而文華自甎橋之敗,見倭寇勢甚,其自柘林移於周浦,與泊於川沙舊巢及嘉定高橋者自如,他侵犯者無虛日,文華乃以寇息請還朝。

明年二月,罷宜,代以宗憲,以阮鶚巡撫浙江。於是宗憲乃請遣使諭日本國王,禁戢島寇,招還通番奸商,許立功免罪。既得旨,遂遣寧波諸生蔣洲、陳可願往。及是,可願還,言至其國五島,遇汪直、毛海峰,謂日本內亂,王與其相俱死,諸島不相統攝,須遍諭乃可杜其入犯。又言,有薩摩洲者,雖已揚帆入寇,非其本心,乞通貢互市,願殺賊自效。乃留洲傳諭各島,而送可願還。宗憲以聞,兵部言:「直等本編民,既稱效順,即當釋兵。乃絕不言及,第求開市通貢,隱若屬國然,其奸叵測。宜令督臣振揚國威,嚴加備禦。移檄直等,俾剿除舟山諸賊巢以自明。果海疆廓清,自有恩賚。」從之。時兩浙皆被倭,而慈谿焚殺獨慘,餘姚次之。浙西柘林、乍浦、烏鎮、皂林間,皆為賊巢,前後至者二萬餘人,命宗憲亟圖方略。七月,宗憲言:「賊首毛海峰自陳可願還,一敗倭寇於舟山,再敗之瀝表,又遣其黨招諭各島,相率效順,乞加重賞。」部令宗憲以便宜行。當是時,徐海、陳東、麻葉,方連兵攻圍桐鄉,宗憲設計間之,海遂擒東、葉以降,盡殲其餘眾於乍浦。未幾,復蹴海於梁莊,海亦授首,餘黨盡滅。江南、浙西諸寇略平,而江北倭則犯丹陽及掠瓜洲,燒漕艘者明春復犯如皋、海門,攻通州,掠揚州、高耶,入寶應,遂侵淮安府,集於廟灣,逾年乃克。其浙東之倭則盤踞於舟山,亦先後為官軍所襲。

先是,蔣洲宣諭諸島,至豐後被留,令僧人往山口等島傳諭禁戢。於是山口都督源義長具咨送還被掠人口,而咨乃用國王印。豐後太守源義鎮遣僧德陽等具方物,奉表謝罪,請頒勘合修貢,送洲還。前楊宜所遣鄭舜功出海哨探者,行至豐後島,島主亦遣僧清授附舟來謝罪,言前後侵犯,皆中國奸商潛引諸島夷眾,義鎮等實不知。於是宗憲疏陳其事,言:「洲奉使二年,止歷豐後、山口二島,或有貢物而無印信勘合,或有印信而無國王名稱,皆違朝典。然彼既以貢來,又送還被掠人口,實有畏罪乞恩意。宜禮遣其使,令傳諭義鎮、義長,轉諭日本王,擒獻倡亂諸渠,及中國奸宄,方許通貢。」詔可。

汪直之踞海島也,與其黨王滶、葉宗滿、謝和、王清溪等,各挾倭寇為雄。朝廷至懸伯爵、萬金之賞以購之,迄不能致。及是,內地官軍頗有備,倭雖橫,亦多被剿戮,有全島無一人歸者,往往怨直,直漸不自安。宗憲與直同郡,館直母與其妻孥於杭州,遣蔣洲齎其家書招之。直知家屬固無恙,頗心動。義鎮等以中國許互市,亦喜。乃裝巨舟,遣其屬善妙等四十餘人隨直等來貢市,於三十六年十月初,抵舟山之岑港。將吏以為入寇也,陳兵備。直乃遣王滶入見宗憲,謂:「我以好來,何故陳兵待我?」滶即毛海峰,直養子也。宗憲慰勞甚至,指心誓無他。俄善妙等見副將盧鏜於舟山,鏜令擒直以獻。語洩,直益疑。宗憲開諭百方,直終不信,曰:「果爾,可遣滶出,吾當入見。」宗憲立遣之。直又邀一貴官為質,即命指揮夏正往。直以為信,遂與宗滿、清溪偕來。宗憲大喜,禮接之甚厚,令謁巡按御史王本固於杭州,本固以屬吏。水敖等聞,大恨,支解夏正,焚舟登山,據岑港堅守。

逾年,新倭大至,屢寇浙東三郡。其在岑港者,徐移之柯梅,造新舟出海,宗憲不之追。十一月,賊揚帆南去,泊泉州之浯嶼,掠同安、惠安、南安諸縣,攻福寧州,破福安、寧德。明年四月遂圍福州,經月不解。福清、永福諸城皆被攻毀,蔓延於興化,奔突於漳州。其患盡移於福建,而潮、廣間亦紛紛以倭警聞矣。至四十年,浙東、江北諸寇以次平。宗憲尋坐罪被逮。明年十一月陷興化府,大殺掠,移據平海衛不去。初,倭之犯浙江也,破州縣衛所城以百數,然未有破府城者。至是,遠近震動,亟征俞大猷、戚繼光、劉顯諸將合擊,破之。其侵犯他州縣者,亦為諸將所破,福建亦平。

其後,廣東巨寇曾一本、黃朝太等,無不引倭為助。隆慶時,破碣石、甲子諸衛所。已,犯化州石城縣,陷錦囊所、神電衛。吳川、陽江、茂名、海豐、新寧、惠來諸縣,悉遭焚掠。轉入雷、謙、瓊三郡境,亦被其患。萬曆二年犯浙東寧、紹、台、溫四郡,又陷廣東銅鼓石雙魚所。三年犯電白。四年犯定海。八年犯浙江韭山及福建彭湖、東湧。十年犯溫州,又犯廣東。十六年犯浙江。然時疆吏懲嘉靖之禍,海防頗飭,賊來輒失利。其犯廣東者,為蜒賊梁本豪勾引,勢尤猖獗。總督陳瑞集眾軍擊之,斬首千六百餘級,沈其船百餘艘,本豪亦授首。帝為告謝郊廟,宣捷受賀云。

日本故有王,其下稱關白者最尊,時以山城州渠信長為之。偶出獵,遇一人臥樹下,驚起衝突,執而詰之。自言為平秀吉,薩摩州人之奴,雄健蹺捷,有口辯。信長悅之,令牧馬,名曰木下人。後漸用事,為信長畫策,奪並二十餘州,遂為攝津鎮守大將。有參謀阿奇支者,得罪信長,命秀吉統兵討之。俄信長為其下明智所殺,秀吉方攻滅阿奇支,聞變,與部將行長等乘勝還兵誅之,威名益振。尋廢信長三子,僭稱關白,盡有其眾,時為萬曆十四年。於是益治兵,征服六十六州,又以威脅琉球、呂宋、暹羅、佛郎機諸國,皆使奉貢。乃改國王所居山城為大閣,廣築城郭,建宮殿,其樓閣有至九重者,實婦女珍寶其中。其用法嚴,軍行有進無退,違者雖子婿必誅,以故所向無敵。乃改元文祿,並欲侵中國,滅朝鮮而有之。召問故時汪直遺黨,知唐人畏倭如虎,氣益驕。益大治兵甲,繕舟艦,與其下謀,入中國北京者用朝鮮人為導,入浙、閩沿海郡縣者用唐人為導。慮琉球洩其情,使毋入貢。

同安人陳甲者,商於琉球。懼為中國害,與琉球長史鄭迥謀,因進貢請封之使,具以其情來告。甲又旋故鄉,陳其事於巡撫趙參魯。參魯以聞,下兵部,部移咨朝鮮王。王但深辨嚮導之誣,亦不知其謀己也。

初,秀吉廣徵諸鎮兵,諸三歲糧,欲自將以犯中國。會其子死,旁無兄弟。前奪豐後島主妻為妾,慮其為後患。而諸鎮怨秀吉暴虐,咸曰:「此舉非襲大唐,乃襲我耳。」各懷異志。由是,秀吉不敢親行。二十年四月遣其將清正、行長、義智,僧玄蘇、宗逸等,將舟師數百艘,由對馬島渡海陷朝鮮之釜山,乘勝長驅,以五月渡臨津,掠開城,分陷豐德諸郡。朝鮮望風潰,清正等遂人畐王京。朝鮮王李昖棄城奔平壤,又奔義州,遣使絡繹告急。倭遂入王京,執其王妃、王子,追奔至平壤,放兵淫掠。七月命副總兵祖承訓赴援,與倭戰於平壤城外,大敗,承訓僅以身免。八月,中朝乃以兵部侍郎宋應昌為經略,都督李如松為提督,統兵討之。

當是時,寧夏未平,朝鮮事起,兵部尚書石星計無所出,募能說倭者偵之,於是嘉興人沈惟敬應募。星即假游擊將軍銜,送之如松麾下。明年,如松師大捷於平壤,朝鮮所失四道並復。如松乘勝趨碧蹄館,敗而退師。於是封貢之議起,中朝彌縫惟敬以成款局,事詳《朝鮮傳》。久之,秀吉死,諸倭揚帆盡歸,朝鮮患亦平。然自關白侵東國,前後七載,喪師數十萬糜餉數百萬,中朝與朝鮮迄無勝算。至關白死,兵禍始休,諸倭亦皆退守島巢,東南稍有安枕之日矣。秀吉凡再傳而亡。

終明之世,通倭之禁甚嚴,閭巷小民,至指倭相詈罵,甚以噤其小兒女云。


\end{pinyinscope}