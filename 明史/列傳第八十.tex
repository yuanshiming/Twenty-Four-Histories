\article{列傳第八十}

\begin{pinyinscope}
楊慎王元正王思王相張翀劉濟安磐張漢卿張原毛玉裴紹宗王時柯餘翱鄭本公張曰韜胡瓊楊淮申良張澯仵瑜臧應奎胡璉餘禎李可登安璽殷承敘郭楠俞敬李繼王懋

楊慎,字用修,新都人,少師廷和子也。年二十四,舉正德六年殿試第一,授翰林修撰。丁繼母憂,服闋起故官。十二年八月,武宗微行,始出居庸關,慎抗疏切諫。尋移疾歸。世宗嗣位,起充經筵講官。常講《舜典》,言:「聖人設贖刑,乃施於小過,俾民自新。若元惡大奸,無可贖之理。」時大榼張銳、于經論死,或言進金銀獲宥,故及之。

嘉靖三年,帝納桂萼、張璁言,召為翰林學士。慎偕同列三十六人上言:「臣等與萼輩學術不同,議論亦異。臣等所執者,程頤、朱熹之說也。萼等所執者,冷褒、段猶之餘也。今陛下既超擢萼輩,不以臣等言為是,臣等不能與同列,願賜罷斥。」帝怒,切責,停俸有差。踰月,又偕學士豐熙等疏諫。不得命,偕廷臣伏左順門力諫。帝震怒,命執首事八人下詔獄。於是慎及檢討王元正等撼門大哭,聲徹殿庭。帝益怒,悉下詔獄,廷杖之。閱十日,有言前此朝罷,群臣已散,慎、元正及給事中劉濟、安磐、張漢卿、張原,御史王時柯實糾眾伏哭。乃再杖七人於廷。慎、元正、濟並謫戍,餘削籍。慎得雲南永昌衛。先是,廷和當國,盡斥錦衣冒濫官。及是伺諸途,將害慎。慎知而謹備之。至臨清始散去。扶病馳萬里,憊甚。抵戍所,幾不起。

五年聞廷和疾,馳至家。廷和喜,疾愈。還永昌,聞尋甸安銓、武定鳳朝文作亂,率僮奴及步卒百餘,馳赴木密所與守臣擊敗賊。八年聞廷和訃,奔告巡撫歐陽重請於朝,獲歸葬,葬訖復還。自是,或歸蜀,或居雲南會城,或留戍所,大吏咸善視之。及年七十,還蜀,巡撫遣四指揮逮之還。嘉靖三十八年七月卒,年七十有二。

慎幼警敏,十一歲能詩。十二擬作《古戰場文》、《過秦論》,長老驚異。入京,賦《黃葉詩》,李東陽見而嗟賞,令受業門下。在翰林時,武宗問欽天監及翰林:「星有注張,又作汪張,是何星也?」眾不能對。慎曰:「柳星也。」歷舉《周禮》、《史記》、《漢書》以復。預修《武宗實錄》,事必直書。總裁蔣冕、費宏盡付稿草,俾削定。嘗奉使過鎮江,謁楊一清,閱所藏書。叩以疑義,一清皆成誦。慎驚異,益肆力古學。既投荒多暇,書無所不覽。嘗語人曰:「資性不足恃。日新德業,當自學問中來。」故好學窮理,老而彌篤。

世宗以議禮故,惡其父子特甚。每問慎作何狀,閣臣以老病對,乃稍解。慎聞之,益縱酒自放。明世記誦之博,著作之富,推慎為第一。詩文外,雜著至一百餘種,並行於世。隆慶初,贈光祿少卿。天啟中,追謚文憲。

王元正,字舜卿,盩厔人。與慎同年進士。由庶吉士授檢討。武宗幸宣、大,元正述《五子之歌》以諷。竟以爭「大禮」,謫戍茂州卒。隆慶初,贈修撰。

王思,字宜學,太保直曾孫也。正德六年進士。改庶吉士,授編修。九年春,乾清宮災。思應詔上疏曰:「天下之治賴紀綱,紀綱之立係君身而已。私恩不偏於近習,政柄不移於左右,則紀綱立,而宰輔得行其志,六卿得專其職。今者內閣執奏方堅,而或撓於傳奉,六卿擬議已定,而或阻於內批,此紀綱所由廢也。惟陛下抑私恩,端政本,用舍不以讒移,刑賞不以私拒,則體統正而朝廷尊矣。祖宗故事,正朝之外,日奏事左順門,又不時召對便殿。今每月御朝不過三五日,每朝進奏不踰一二事。其養德之功,求治之實,宰輔不得而知也。聞見之非,嗜好之過,宰輔不得而知也。天下之大,四海之遠,生民愁苦之狀,盜賊縱橫之由,豈能一一上達?伏願陛下悉遵舊典,凡遇宴間,少賜召問。勿以遇災而懼,災過而弛,然後可以享天心,保天命。」其年九月,帝狎虎而傷,閱月不視朝。思復上封事曰:「孝宗皇帝之子惟陛下一人,當為天下萬世自重。近者道路傳言,虎逸於柙,驚及聖躬。臣聞之,且駭且懼。陛下即位以來,於茲九年。朝宁不勤政,太廟不親享。兩宮曠於問安,經筵倦於聽講。揆厥所自,蓋有二端:嗜酒而荒其志,好勇而輕其身。由是,戒懼之心日忘,縱恣之欲日進,好惡由乎喜怒,政令出於多門。紀綱積弛。國是不立。士氣摧折,人心危疑。上天示警,日食地震。宗社之憂,凜若朝夕。夫勇不可好,陛下已薄有所懲矣。至於荒志廢業,惟酒為甚。《書》曰:『甘酒嗜音,峻宇雕牆,有一於此,未或不亡。』陛下露處外宮,日湎於酒。廝養雜侍,禁衛不嚴。即不幸變起倉卒,何以備之?此臣所大憂也。」疏入,留中者數日,忽傳旨降遠方雜職,遂謫潮州三河驛丞。

思年少氣銳,每眾中指切人是非。已悔之,自斂為質訥。及被謫,怡然就道。夜過瀧水,舟飄巨石上,緣石坐浩歌。家人後至,聞歌聲乃艤舟以濟。王守仁講學贛州,思從之遊。及守仁討宸濠,檄思贊軍議。

世宗嗣位,召復故官,仍加俸一級。思疏辭,且言:「陛下欲作敢言之氣,以防壅蔽之奸,莫若省覽奏章,召見大臣,勿使邪僻阿徇之說蠱惑聖聽,則堯、舜之治可成。不然,縱加恩於先朝譴責之臣,抑末矣。」帝不允,因命近日遷俸者,皆不得辭。尋充經筵講官。嘉靖三年與同官屢爭「大禮」,不報。時張璁、桂萼、方獻夫為學士,思羞與同列,疏乞罷歸。不許。其年七月,偕廷臣伏左順門哭諫。帝大怒,系之詔獄,杖三十。踰旬,再杖之。思與同官王相,給事中張原、毛玉、裴紹宗,御史張曰韜、胡瓊,郎中楊淮、胡璉,員外郎申良、張澯,主事安璽、仵瑜、臧應奎、餘禎、殷承敘,司務李可登,凡十有七人,皆病創先後卒。隆慶初,各廕一子,贈官有差。思贈右諭德。

思志行邁流俗,與李中、鄒守益善。高陵呂柟亟稱之,嘗曰:「聞過而喜似季路,欲寡未能似伯玉,則改齋其人也。」改齋者,思別號也。

王相,字懋卿,鄞人。正德十六年進士。由庶吉士授編修。豪邁尚志節。事親篤孝。家貧屢空,晏如。仕僅四年而卒。

張翀,字習之,潼川人。正德六年進士。選庶吉士,改刑科給事中。引疾歸,起戶科。世宗即位,詔罷天下額外貢獻。其明年,中都鎮守內官張陽復貢新茶。禮部請遵詔禁,不許。翀言:「陛下詔墨未乾,旋即反汗,人將窺測朝廷,玩侮政令。且陽名貢茶,實雜致他物。四方效尤,何所抵極。願守前詔,無墮奸謀。」不聽。寧夏歲貢紅花,大為軍民害;內外鎮守官蒞任,率貢馬謝恩。翀皆請罷之。帝雖是其言,不能從。尋言:「中官出鎮,非太祖、太宗舊制。景帝遭國家多故,偶一行之。謂內臣是朝廷家人,但有急事,令其來奏。乃往歲宸濠謀叛,鎮守太監王宏反助為逆,內臣果足恃耶?時平則坐享尊榮,肆毒百姓,遇變則心懷顧望,不恤封疆。不可不亟罷。」後張孚敬為相,竟罷諸鎮守,其論實自翀發之。

屢遷禮科都給事中。又言:「頃聞紫禁之內,禱祠繁興。乾清宮內官十數輩,究習經典,講誦科儀,賞賚踰涯,寵幸日密。此由先朝罪人遺黨若太監崔文輩,挾邪術為嘗試計。陛下為其愚弄,而已得肆其奸欺。干撓政事,牽引群邪,傷太平之業,失四海之望。竊計陛下寧遠君子而不忍斥其徒,寧棄讜言而不欲違其教,亦謂可以延年已疾耳。側聞頃來嬪御女謁,充塞閨幃,一二黠慧柔曼者為惑尤甚。由是,怠日講,疏召對,政令多僻,起居愆度。小人窺見間隙,遂以左道蠱惑。夫以齋醮為足恃而恣欲宮壺之間,以荒淫為無傷而邀福邪妄之術,甚非古帝王求福不回之道也。」

嘉靖二年四月,以災異,偕六科諸臣上疏曰:「昔成湯以六事自責曰:『政不節歟?民失職歟?宮壺崇歟?女謁盛歟?苞苴行歟?讒夫昌歟?』今誠以近事較之。快船方減而輒允戴保奏添,鎮戍方裁而更聽趙榮分守。詔核馬房矣,隨格於閻洪之一言;詔汰軍匠矣,尋奪於監門之群咻。是政不可謂節也。末作競於奇巧,遊手半於閭閻。耕桑時廢,缺俯仰之資;教化未聞,成偷薄之習。是民不可謂不失職也。兩宮營建,採運艱辛。或一木而役夫萬千,或一椽而廢財十百。死亡枕藉之狀,呻吟號嘆之聲,陛下不得而見聞。是宮壺不可謂不崇也。奉聖、保聖之後,先女寵於冊后;莊奉、肅奉之名,聯殊稱於乳母。或承恩漸鄰於飛燕,或黠慧不下於婉兒。內以移主上之性情,外以開近習之負倚。是女謁不可謂不盛也。窮奸之銳、雄,公肆賂遺而逃籍沒之律;極惡之鵬、鎧,密行請託而逋三載之誅。錢神靈而王英改問於錦衣,關節通而於喜竟漏於禁網。是苞苴不可謂不行也。獻廟主祀,屈府部之議,而用王槐諛佞之謀;重臣批答,乏體貌之宜,而入群小惎間之論。或譖發於內,陰肆毒螫;或讒行於外,顯逞擠排。上以汨朝廷之是非,下以亂人物之邪正。是讒夫不可謂不昌也。凡此,皆成湯之所無,而今日之所有,是以不避斧鉞之誅,用附責難之義。望陛下採納。」

其年冬,命中官督蘇、杭織造,舉朝阻之不能得。翀復偕同官張原等力爭。時世宗初政,楊廷和等在內閣。群小雖已用事,正論猶伸,翀前後指斥無所避。帝雖不見用,然亦嘗報聞,不罪也。

及明年三月,帝以桂萼言,銳欲考獻帝,且欲立廟禁中,翀復偕同官力諫。帝於是責以朋言亂政,命奪俸。既又助尚書喬宇等再疏爭內殿建室之議,被詔切讓。呂柟、鄒守益下獄,翀等抗疏救。及張璁、桂萼召至,翀與給事三十餘人連章言:「兩人賦性奸邪,立心憸佞,變亂宗廟,離間宮闈,詆毀詔書,中傷善類。望亟出之,為人臣不忠之戒。」皆不納。帝愈欲考獻帝,改孝宗為伯考,翀等憂之。

會給事中張漢卿劾席書振荒不法,戶部尚書秦金請命官往勘,帝是之。翀等乃取廷臣劾萼等章疏,送刑部令上請,且私相語曰:「倘上亦云是者,即撲殺之。」璁等以其語聞。帝留疏不下,而責刑部尚書趙鑑等朋邪害正,翀等陷義罔忠,而進璁、萼學士。廷臣相顧駭歎。諸曹乃各具一疏,力言孝宗不可稱伯考,署名者凡二百二十餘人。帝皆留中不報。七月戊寅,諸臣相率伏左順門懇請。帝兩遣中官諭之不退,遂震怒。先逮諸曹為首者八人於詔獄,翀與焉。尋杖於廷,謫戍瞿塘衛,而璁、萼寵益盛。翀居戍所十餘年,以東宮冊立恩放還,卒。

劉濟,字汝楫,騰驤衛人。正德六年進士。由庶吉士授吏科給事中。山西巡撫李鉞劾左、右布政使倪天民、陳達。吏部請黜之,帝不許。濟疏爭,不省。帝幸宣府、榆林,濟皆疏請回鑾。詔封許泰、江彬伯爵,又與諸給事中力爭,皆不報。世宗即位,出核甘肅邊餉。奏革涼州分守中官及永昌新添遊兵。再遷工科左給事中。

嘉靖改元,進刑科都給事中。主事陳嘉言坐事下獄,濟疏救,不許。廖鵬父子及錢寧黨王欽等,皆以從逆論斬,鵬等夤緣中人冀脫死。濟上言:「自來死囚臨斬,鼓下猶受訴詞。奏上得報,已及日旰,再請而後行刑,則已薄暮。殊非與眾棄之之意。乞自三請後,鼓下不得受詞。鵬、欽等罪甚當,幸陛下勿疑。」詔自今以申酉行刑,鵬等竟緩決。欽後以中旨免死。濟力爭,不聽。故事,廠衛有所逮,必取原奏情事送刑科簽發駕帖。千戶白壽齎帖至,濟索原奏,壽不與,濟亦不肯簽發。兩人列詞上。帝先入壽言,竟詘濟議。中官崔文僕李陽鳳坐罪,已下刑部。帝受文愬,移之鎮撫。濟率六科爭之,不聽。都督劉暉以奸黨論戍,有詔復官。甘肅總兵官李隆嗾亂軍殺巡撫許銘,逮入都,營免赴鞫。濟皆力陳不可,帝從其言。暉奪職,隆受訊伏辜。

定國公徐光祚規占民田,嗾灤州民訐前永平知府郭九皋。太監芮景賢主之,緹騎逮訊。濟請并治光祚,章下所司。給事中劉最以劾中官崔文調外任,景賢復劾其違禁,與御史黃國用皆逮下詔獄,戍最而謫國用。法司爭不得,濟言:「國家置三法司,專理刑獄,或主質成,或主平反。權臣不得以恩怨為出入,天子不得以喜怒為重輕。自錦衣鎮撫之官專理詔獄,而法司幾成虛設。如最等小過耳,羅織於告密之門,鍛煉於詔獄之手。旨從內降,大臣初不與知,為聖政累非淺。且李洪、陳宣罪至殺人,降級而已。王欽兄弟黨奸亂政,謫戍而已。以最等視之,奚啻天淵,而罪顧一律,何以示天下?」帝怒,奪濟俸一月。后父陳萬言奴何璽毆人死,帝命釋之。濟執奏曰:「萬言縱奴殺人,得免為幸,乃并釋璽等,是法不行於戚畹奴也。」濟在諫垣久,言論侃侃,多與權幸相枝柱,直聲甚震,帝滋不能堪。

「大禮」議起,廷臣爭者多得罪。濟疏救修撰呂柟,編修鄒守益,給事中鄧繼曾,御史馬明衡、朱淛、陳逅、季本,郎中林應驄,不聽。既而遮諸朝臣於金水橋,伏哭左順門,受杖闕廷。越十二日再杖,謫戍遼東。十六年冊立皇太子,赦諸謫戍者,濟不與,卒於戍所。隆慶初復官,贈太常少卿。

安磐,字公石,嘉定州人。弘治十八年進士。改庶吉士。正德時,歷吏、兵二科給事中,乞假去。世宗踐阼,起故官。帝手詔欲加興獻帝皇號,磐言:「興,籓國也,不可加於帝號之上。獻,謚法也,不可加於生存之母。本生、所後,勢不俱尊。大義私恩,自有輕重。」會廷臣多力爭,事得且止。

嘉靖元年,主事霍韜言,科道官褻服受詔,大不敬。磐偕同官論韜先以議禮得罪名教,恐言官發其奸,故摭拾細事,意在傾排。帝置不問。尋因事言:「先朝內外巨奸,若張忠、劉養、韋霦、魏彬、王瓊、寧杲等,漏網得全要領。其貨賂可以通神,未嘗不夤緣覬復用。宜嚴察預防,天下事毋令若輩再壞。」帝納其言,命錦衣官密訪緝之。中官張欽家人李賢者,帝許任為錦衣指揮。磐極言不可,不聽。錦衣千戶張儀以附中官張銳黜革,御史楊百之忽為訟冤,言;「儀當宸濠逆謀時,首倡大義,勸銳卻其饋遺。今銳以是免死,儀功不錄,無以示報。」磐疏言:「百之憸邪,陽為儀遊說,而陰與銳交關,為銳再起地。」百之情得,乃誣磐因請屬不行,挾私行謗。吏部尚書喬宇等議黜百之,刑部謂情狀未明,宜俱逮治。帝兩宥之,奪百之俸三月,磐一月。

帝頻興齋醮,磐又抗言:「曩武宗為左右所蠱,命番僧鎖南綽吉出入豹房,內官劉允迎佛西域。十數年間糜費大官,流謗道路。自劉允放,而鎖南囚,供億減,小人伏。奈何甫及二年,遽襲舊轍。不齋則醮,月無虛日。此豈陛下本意?實太監崔文等為之。文鐘鼓廝役,夤緣冒遷,既經降革,乃營求還職。導陛下至此,使貽譏天下後世,文可斬也。文嘗試陛下,欲行香則從之,欲登壇則從之,欲拜疏則又從之。無已則導以遊幸、土木,導以征伐,方且連類以進,伺便以逞。臣故曰文可斬也。」疏入,報聞。戶部主事羅洪載以杖錦衣百戶張瑾下詔獄,磐與同官張漢卿、張逵、葛鴊等請付之法司。不聽。永福長公主下嫁,擇昏於七月下旬。磐言:「長公主於孝惠皇太后為在室孫女,期服未滿,宜更其期。舊儀,駙馬見公主行兩拜禮,公主坐受。乖夫婦之分,亦當革正。」帝以遺旨格之,相見禮如故。

錦衣革職旗校王邦奇屢乞復職,磐言:「邦奇等在正德世,貪饕搏噬,有若虎狼。其捕奸盜也,或以一人而牽十餘人,或以一家而連數十家,鍛煉獄詞,付之司寇,謂之『鑄銅板』。其緝妖言也,或用番役四出搜愚民詭異之書,或購奸僧潛行誘愚民彌勒之教,然後從而掩之,無有解脫,謂之『種妖言』。數十年內,死者填獄,生者冤號。今不追正其罪,使得保首領,亦已幸矣,尚敢肆然無忌,屢瀆天聽,何為者哉且陛下收已渙之人心,奠將危之國脈,實在登極一詔。若使此輩攘臂,一朝壞之,則奸人環立蜂起,隄防潰決,不知所紀極矣。宜嚴究治,絕禍源。」帝不能從。其後邦奇卒為大厲,如磐言。

帝驛召席書、桂萼等,磐請斥之以謝天下,且言:「今欲別立一廟於大內,是明知恭穆不可入太廟矣。夫孝宗既不得考,恭穆又不得入,是無考也。世豈有無考之太廟哉。此其說之自相矛盾者也。」不聽。歷兵科都給事中。以率眾伏闕再受杖,除名為民。卒於家。

張漢卿,字元傑,儀封人。正德六年進士。授魏縣知縣,徵拜刑科給事中。嘗陳杜僥倖、廣儲積、慎刑獄三事,深切時弊。不報。武宗將南巡,偕同官伏闕諫。

世宗嗣位,從巡撫李鐸言,發帑金二十萬優恤宣府軍民。以漢卿言,並發十三萬於大同。屢遷戶科都給事中。嘉靖元年冬,與同官上言:「陛下軫念畿輔莊田之害,遣官會勘。敕自正德以後投獻及額外侵占者,盡以給民。王言一布,天下孰不誦陛下之仁!乃者給事中夏言、御史樊繼祖、主事張希尹勘上涿州薰皮廠、安州鷹房草場,詔旨留用。所司執奏,迄不肯從,非所以全大信昭至公也。皮廠起於馬永成,鷹房創於谷大用,皆奪民業為之。今馬俊、趙霦恃籓邸舊恩,妄求免革,是復蹈永成、大用故轍也。乞盡還之民,而嚴罪俊、霦,為欺罔者戒。」后父陳萬言請營新第,既又乞莊田,內官吳勳等請督蘇州織造,漢卿皆極諫。不納。應天諸府大旱,帝將鬻淮、浙餘鹽及所沒產,易銀振之。漢卿言:「易銀緩,非發帑金不可。」帝為發銀十五萬。未幾,復偕同官言:「今天下一歲之供,不給一歲之用,加以水旱頻仍,物力殫屈。陛下方躬行節儉,而中官梁棟等奏營造缺珠寶,是欲括戶部之銀也。梁政等又以蠲免三分之數,欲行京倉撥補,是欲耗太倉之粟也。夫內庫不足,取之計部;計部不足,取之郡邑小民。郡邑小民將安取哉?今東南洊饑,民至骨肉相食,而搜括之令頻行,臣等竊以為不可。」報聞。已,又劾席書振濟乖方,乞遣官往勘,正其欺罔罪。帝方眷書甚,驛召為禮部尚書,不罪也。

初,興獻帝議加皇號,漢卿力爭,至是,又倡眾伏闕。兩受杖,斥為民。二十年,言官邢如默、賈準等會薦天下遺賢,及漢卿,終不召。

張原,字士元,三原人。正德九年進士。授吏科給事中。疏陳汰冗食、慎工作、禁貢獻、明賞罰、廣言路、進德學六事。中言:「天下幅員萬里,一舉事而計臣輒告匱,民貧故也。民何以貧?守令之裒斂,中臣之貢獻,為之也。比年軍需雜輸十倍前制,皆取辦守令。守令假以自殖,又十倍於上供。民既困矣,而貢獻者復巧立名目,爭新競異,號曰『孝順』。取於民者十百,進於上者一二,朝廷何樂於此而受之。人君馭下惟賞與罰。邇者庸才廝養莫不封侯腰玉。或足不出門而受賞,身不履陣而奏功。禦敵者竟未沾恩,覆軍者多至逃罪。此士卒所由解體也。」疏入,權倖惡之,傳旨謫新添驛丞。

嘉靖初,召復兵科,仍加俸一級。南寧伯毛良殺其子,錦衣掌印指揮朱震等多違縱,原先後論之,皆奪職閒住。帝進張鶴齡昌國公;封陳萬言泰和伯,世襲,授萬言子紹祖尚寶丞;又以外戚蔣泰等五人為錦衣千、百戶。原抗疏極言,請行裁節。未幾,劾建昌侯張延齡強占民地,定國公徐光祚子、外戚玉田伯蔣輪、昌化伯邵蕙家人擅作威福。事雖不盡行,權貴皆震懾。

進戶科右給事中。撼門哭,再被杖,創重卒。貧不能歸葬。久之,都御史陳洪謨備陳原與毛玉、裴紹宗、王思、王相、胡瓊等妻子流離狀,請恤於朝。不許。隆慶元年贈光祿少卿。

毛玉,字國珍,更字用成,雲南右衛軍家子也,其先良鄉人。弘治十八年進士。正德五年,由行人擢南京吏科給事中。劉瑾既敗,大盜蜂起。玉言大學士焦芳、劉宇實亂天下,請顯僇扁,以謝萬姓。群盜擾山東、河南,玉請備留都。已而盜果渡江,以備嚴,不敢犯。外艱去,起南京兵科。御史林有年諫迎佛烏思藏下獄,玉抗疏救之,有年得薄罰。又以繼母艱去。服闋,除吏科。世宗即位逾年,興邸諸內官怙帝寵,漸驕佚。又故太監谷大用、魏彬等相次謀復起,事有萌芽。玉即抗疏歷敘武宗時事,勸帝戒嗜欲,杜請託,以破僥倖之門,塞蠱惑之隙。帝嘉納焉。

御史曹嘉素輕險,仿宋范仲淹《百官圖》,分廷臣四等,加以品題。給事中安磐疏駁之,言唐王珪之論房玄齡等,本朝解縉之論黃福等,皆承君命而品藻之,未有漫然恣其口吻,如嘉者也。玉復言嘉背違成法,變亂國是,乞斥。帝從其言,貶嘉於外。御史許宗魯為嘉訟,請斥玉,其同官倫以謀亦助為言。給事中張原以庶僚聚訟,朝廷為之多事,重損國體,乞身先斥罷。玉亦上疏求去,言:「宗魯等知朋友私恩,不顧朝廷大體。臣一身所係絕微,公論所關甚大,乞罷臣以謝御史。」帝皆慰留之。時宸濠戚屬連逮者數百人,玉奉命往訊,多所全活。且言宸濠稱亂,由左右貪賂釀成之。因劾守臣不死事者,而禁天下有司與籓府交通。帝俱從之。再遷左給事中。尋伏闕爭「大禮」,下獲受杖,竟卒。後贈光祿少卿。

裴紹宗,字伯修,渭南人。正德十二年進士。除海門知縣。武宗南巡,受檄署江都事,權倖憚之,供億大省。世宗即位,召入為兵科給事中。即疏請法祖定制,言:「太祖貽謀盡善。如重大臣,勤視朝,親歷田野,服浣濯衣,種蔬宮中,毀鏤金床,碎水晶漏,造觀心亭,揭《大學衍義》之類,陛下所當釋思祖述。而二三大臣尤宜朝夕納誨,以輔養聖德。陛下日御便殿,親儒臣,使耳目不蔽於淫邪,左右不惑於險佞,則君志素定,治功可成。」帝嘉納之。帝欲加興獻帝皇號,紹宗力諫。嘉靖二年冬,帝以災異頻仍,欲罷明年郊祀慶成宴。紹宗言:「祭祀之禮莫重於郊丘,君臣之情必通於宴享。往以國戚廢大禮,今且從吉,宜即舉行,豈可以災傷復免。」修撰唐皋亦言之。竟得如禮。明年,以伏闕受杖卒。贈官如毛玉。

王時柯,字敷英,萬安人。正德十二年進士。授行人。嘉靖三年擢御史,疏言:「桂萼輩以議禮迎合,傳陞美官。薛蕙、陳相、段續、胡侍等,連章論劾,實出至公。今佞人超遷而群賢獲罪,恐海內聞之,謂陛下好諛惡直。願採忠讜之言,消朋比之禍,特寬蕙等而聽席書、方獻夫辭職,除張璁、桂萼別任,則是非不謬,人情悅服。」忤旨切責。未幾,有伏闕之事,再予杖,除名。

時御史疏爭「大禮」居首者餘翱,字大振,定遠人,正德中進士。嘉靖二年為御史,嘗劾司禮太監張佐蒙蔽罪。明年七月,與時柯等被杖戍邊。居戍所十四年。皇子生,赦還。穆宗即位,時柯、翱皆復官,贈時柯光祿少卿。

鄭本公,朔州衛人。正德九年進士。歷御史。武宗不豫,國本未建,本公請慎選宗室親賢者正位東宮,繫天下望。不報。世宗嗣位,及冬而乾清宮成,帝由文華殿入居之。本公上言:「事之可思者有六。是宮八年營構,一旦告成。陛下居安思危,當遠群小,節燕遊,以防一朝之患;重妃配,廣繼嗣,以為萬世之計。慎終如始,兢兢業業,常若天祖之臨;求言益切,訪政益勤,用防壅蔽之患。持聖心,遠貨色,毋溺于鴆毒;重興作,惜財力,永鑒于先朝。」帝嘉納之。踰月,帝欲加興獻帝皇號,本公力言不可。嘉靖改元,出按遼東。劾罷副總兵張銘、都指揮周輔。還朝,論救給事中劉最,忤旨切責。二年十月,時享太廟,帝不親行。本公與同官彭占祺極言遣代非宜,報聞。

明年三月,帝欲考興獻帝,立廟禁中。本公偕同官力爭,謂:「陛下潛邸之日,則為孝宗之姪,興獻王之子。臨御之日,則為孝宗之子,興獻帝之姪。可兩言決也。至立廟大內,實為不經。獻帝之靈既不得入太廟,又空去一國之祀而託享于大內焉。陛下享太廟,其文曰『嗣皇帝』,於獻帝之廟,又當何稱?愛敬精誠,兩無所屬,獻帝將蹙然不安。」帝怒,責其朋言亂政,奪俸三月。

其年六月,以席書為禮部尚書,召張璁、桂萼入京。本公偕同官四十四人連章言:「萼首為亂階,璁再肆欺罔,黃綰、黃宗明、方獻夫、席書連匯接踵。尚書之命,由中而下。行取之旨,已罷再頒。大臣因此被逐,言官由之得罪。雖往日瑾、彬之奸,流禍不若是酷也。」不納。已,偕廷臣伏闕哭諫。繫獄,廷杖還職。當是時,爭「大禮」者,諸御史中,本公言最切中。

尋遷通政參議。九年不調,以疾請改南京。乃授大理寺丞,稍遷南京太僕少卿。謝病歸。二十年,言官邢如默、賈準等會薦,詔用不赴,卒。

張曰韜,字席珍,莆田人。正德十二年進士。授常州推官。武宗南巡,江彬縱其黨橫行州縣。將抵常州,民爭欲亡匿。時知府暨武進知縣咸入覲,曰韜兼綰府縣印,召父老約曰:「彬黨至,若等力與格。」又釋囚徒,令與丐者各具瓦石待。已,彬黨果累騎來。父老直遮之境上,曰:「常州比歲災,物力大屈,無可啖若曹。府中惟一張推官,一錢不入,即欲具芻秣,亦無以辦。」言已,彬黨疑有他變,乃稍退,馳使告彬。曰韜即上書巡按御史言狀。御史東郊行部過常州,謂曰:「事迫矣,彬將以他事縛君。」命曰韜登己舟先發,自以小舟尾之。彬黨果大至,索曰韜,誤截御史舟。郊使嚴捕截舟者,而陰令緩之。其黨恐御史上聞,咸散去,曰韜遂免。彬亦戒其黨毋擾,由是常以南諸府得安。

世宗即位,召為御史。楊廷和等之爭織造也,曰韜亦上言:「陛下既稱閣臣所奏惟愛主惜民,是明知織造之害矣。既知之,而猶不已,實由信任大臣弗專,而群小為政也。自古未有群小蒙蔽於內,而大臣能盡忠於外者。崔文輩二三小人嘗濁亂先朝,今復蒙惑聖衷,竊弄威福。陛下奈何任其逞私,不早加斥逐哉?臣聞織造一官,行金數萬方得之。既營之以重資,而欲其不責償於下,此必無之事也。」帝不能用。

席書以中旨拜尚書,曰韜與同官胡瓊各抗疏力爭。既受杖,猶占疏劾奸人陳洸罪。未幾,竟死。隆慶初,追贈光祿少卿。

胡瓊,字國華,南平人。正德六年進士。由慈谿知縣入為御史。歷按貴州、浙江有聲。哭諫,受杖卒。後贈官如曰韜。

楊淮,字東川,無錫人。正德十二年進士。授戶部主事,再遷郎中。始監京倉,革胥徒積弊殆盡。繼監淮、通二倉,罷中官茶果之供,除囤基及額外席草費。最後監內庫,奄人例有供饋,淮悉絕之。公勤廉慎,為尚書孫交、秦金所重。伏闕受杖,月餘卒。囊無一物,家人賣屋以斂。金與淮同里,為經紀歸其喪。後贈太常少卿。

申良,字延賢,高平人。登鄉薦,授招遠知縣。山東盜起,良豫為戰守具。盜至,追擊至黃縣,俘斬數百人。已,復至,再破走之。歷知諸城、良鄉。權貴人往來要索,良悉拒之。進安吉知州。錦衣葉瓊倚錢寧勢奪民田,良讞還之民。瓊因嗾奸人誣奏良,事竟得白。稍遷常州同知,入為戶部員外郎。與淮俱杖死。贈太僕少卿。招遠民懷其政,繪像祀之。

張澯,字景川,廣東順德人。祖善昭,四川僉事,謫臨江通判。先是,練子寧親黨戍臨江者八十餘人,善昭上書曰:「子寧忠貫日月,太宗謂『若使子寧在,朕固當用之』。仁宗亦謂『方孝孺等忠臣』。夫既忠之矣,何外親末屬,尚以奸惡賜配,百年不宥哉?」疏雖不行,中外皆壯之。澯登正德九年進士,授建平知縣。忤巡江御史賀洪,改調廣昌。訟洪罪,洪坐削籍。澯自廣昌遷禮部主事,監督會同館。尚書王瓊與都御史彭澤有隙,以澤遣使土魯番許金幣贖哈密城印為澤罪,嗾番人在館者暴澤過惡,誘澯為署牒,且曰:「澤所為,南宋覆轍也。事成當顯擢。」澯力拒曰:「王公誤矣。澤與土魯番檄具在,豈宋和戎比。昔范仲淹亦嘗致書元昊,寧獨澤也。」不肯署。尋進員外郎,受杖死。

仵瑜,字忠父,蒲圻人。父紳,工部主事。瑜少有志操,正德十二年釋褐,即謝病去。起補禮部主事,復引疾歸。世宗踐阼,起故官。疏陳勤聖學、篤親親、開言路、敬大臣、選諍臣、去浮屠、拯困窮、重守令、修武備、儲人材十事。已,竟死杖下。

臧應奎,字賢徵,長興人。正德十二年進士。授南京車駕主事。進貢中官索舟踰額,力裁損之。中官遣卒嘩於部,叱左右執之,遁去。父所生母卒,法不得承重,執私喪三年。入為禮部主事,未幾杖死。應奎受業湛若水之門,以聖賢自期。嘗過文廟,慨然謂其友曰「吾輩歿,亦當俎豆其間」,其立志如此。

郎中胡璉,字重器,新喻人。正德六年進士,官刑部。嘗諫武宗南巡受杖。

主事餘禎,字興邦,奉新人。正德九年進士。

司務李可登,字思善,輝縣人。弘治末鄉薦。俱官兵部。可登素慷慨,以忠義自許,竟如其志。

戶部主事安璽,宛平人。正德十六年進士。

刑部主事殷承敘,江夏人。正德九年進士。

穆宗嗣位,贈璉太常少卿,澯太僕少卿,瑜、應奎、承敘、璽、禎光祿少卿,可登寺丞。

郭楠,字世重,晉江人。正德九年進士。授浦江知縣。課最,入為御史。世宗即位,請召還直臣舒芬、王思、黃鞏、張衍瑞等。從之。嘉靖元年,核餉兩廣。劾總兵官撫寧侯朱麒貪懦,詔為戒飭。尋上章,請退朝之暇延見大臣,如祖宗故事。且言,主事陳嘉言忤中官,不宜逮繫。帝怒,奪其俸。

諸臣伏闕爭「大禮」,皆得罪。楠方巡按雲南,馳疏言:「人臣事君,阿意者未必忠,犯顏者未必悖。今群臣伏闕呼號,或榜掠殞身,或間關謫戍,不意聖明之朝,而忠良獲罪若此。乞復生者之職,恤死者之家,庶以收納人心,全君臣之義。」帝大怒,遣緹騎逮治,言官論救皆不納。既至,下鎮撫獄掠治,復廷杖之,削其籍。

先是,諸人既死,廷臣莫敢上聞。後府經歷俞敬奏言:「學士豐熙等皆以冒觸宸嚴,繫獄拷訊。諸臣跡雖狂悖,心實忠誠。今聞給事裴紹宗、編修王相、主事餘禎等俱已死,熙等在獄者亦垂亡矣。其呻吟衽席,創重不能起者,又不知凡幾。竊惟獻皇帝神主已奉迎入廟,正宜赦過宥罪,章大孝於天下。望霽雷霆之威,施雨露之澤。已死者恤其後,垂亡者宥其身,使人臣無復以言為諱,宗社之幸也。」

通政司經歷李繼先亦上言:「陛下追崇尊號,乃人子至情,誠不容已。群臣一時冒觸天威,重得罪譴,死者遂十餘人。大臣紛紛去位,小臣茍默自容。今日大同告變,曾無一人進一疏、畫一策者,則小大之臣,志不奮而氣不揚,亦可見矣。乞錄恤已死,赦還謫戍,追復去國諸臣,而在位者委任寬假之,使各陳邊計。臣愚不勝心卷心卷。」帝皆不省。

明年三月,御史王懋言:「廷臣以議禮死杖下者十有七人,其父母妻子顛沛可憫,乞賜優恤,贈官錄廕。」帝大怒,謫懋四川高縣典史。逾數日,而楠疏至。帝益怒,遂逮治削籍。六年春,以災變修省,從吏部言量與楠一官,得吉水教諭。終南寧知府。

贊曰:「大禮」之爭,群臣至撼門慟哭,亦過激且戇矣。然再受廷杖,或死或斥,廢錮終身,抑何慘也。楊慎博物洽聞,於文學為優。王思、張翀諸人,或納諫武宗之朝,或抗論世宗初政,侃侃鑿鑿,死節官下,非徒意氣奮發立效一時已也。


\end{pinyinscope}