\article{列傳第一百三十六}

\begin{pinyinscope}
梅之煥劉策徐縉芳陳一元李若星耿如杞胡士容顏繼祖王應豸等李繼貞方震孺徐從治謝璉余大成等

梅之煥,字彬父,麻城人,侍郎國楨從子也。年十四為諸生。御史行部閱武,之煥騎馬突教場。御史怒,命與材官角射,九發九中,長揖上馬而去。

萬曆三十二年舉進士,改庶吉士。居七年,授吏科給事中。東廠太監李浚誣拷商人,之煥劾其罪。尋上言:「今天下民窮餉匱,寇橫兵疲。言官舍國事爭時局,部曹舍職掌建空言,天下盡為虛文所束縛。有意振刷者,不曰生事,則曰苛求。事未就而謗興,法未伸而怨集,豪傑灰心,庸人養拙,國事將不可為矣。請陛下嚴綜核以責實事,通言路以重紀綱,別臧否以惜人才,庶於國事有濟。」時朝臣部黨角立,之煥廉觚自勝,嘗言:「附小人者必小人,附君子者未必君子。蠅之附驥,即千里猶蠅耳。」時有追論故相張居正者,之煥曰:「使今日有綜名實、振紀綱如江陵者,言翕訿之徒敢若此耶?」其持平不欲傅會人如此。出為廣東副使,擒誅豪民沈殺烈女者,民服其神。海寇袁進掠潮州,之煥扼海道,招散其黨,卒降進。改視山東學政。天啟元年以通政參議召遷太常少卿,擢右僉都御史,巡撫南、贛。丁內外艱,家居。當此之時,魏、客亂政,應山楊漣首發忠賢之奸。忠賢恚甚,拷殺漣。由此悍然益誅鋤善類,心買慀楚人矣。謂漣被逮時,過麻城,漣罪人也,之煥與盤桓流涕,當削籍,其實漣未嘗過麻城也。無何,逆黨梁克順誣以贓私,詔徵贓。

莊烈帝即位,乃免征,起故官,巡撫甘肅。大破套寇,斬首七百餘級,生得部長三人,降六百餘人。明年春,寇復大入,患豌豆創,環大黃山而病。諸將請掩之,之煥不可,曰:「幸災不仁,乘危不武,不如舍之,因以為德焉。」遂不戰。踰月,群寇望邊城搏顙涕泣而去。冬,京師戒嚴,有詔入衛。且行,西部乘虛犯河西。之煥止留,遣兵伏賀蘭山後,邀其歸路,大兵出水泉峽口,再戰再敗之,斬首八百四十有奇,引軍東。俄悍卒王進才殺參將孫懷忠等以叛,走蘭州。之煥遂西定其變,復整軍東。明年五月抵京師,已後時矣,有詔之煥入朝。翌日又詔之煥落職候勘,溫體仁已柄政矣。初,體仁訐錢謙益,之煥移書中朝,右謙益。至是,體仁修隙,之煥遂得罪。

之煥雖文士,負材武,善射,既廢,無所見。所居縣,阻山多盜。之煥無事,輒率健兒助吏捕,無脫者。先是,甘肅兵變,其潰卒畏捕誅,往往亡命山谷間,為群盜,賊勢益張。至是,賊數萬來攻麻城,望見之煥部署,輒引去。帝追敘甘肅前後功,復之煥官,廕子,然終不召。明年病卒。

劉策,字範董,武定人。萬曆二十九年進士。由保定新城知縣入為御史,疏劾太僕少卿徐兆魁,復力爭熊廷弼行勘及湯賓尹科場事。賓尹雖家居,遙執朝柄,嗾其黨逐攻者孫振基、王時熙。

已而給事中劉文炳劾兩淮巡鹽御史徐縉芳,言策入葉向高幕,乾票擬;策同官陳一元,向高姻親,顧權利。時策按宣、大,疏言:「文炳為湯賓尹死友,代韓敬反噬。昔年發奸如振基、時熙輩,今皆安在?」向高亦以策無私交,為辨雪。文炳、策屢疏相詆,南京御史吳良輔言:「文炳一疏而彈御史縉芳、一元、策及李若星,再疏而彈詞臣蔡毅中、焦竑及監司李維楨,他波及尚多。人才摧殘甚易,清品如策,雅望如竑,不免詆斥,天下寧有完人?」策復詆文炳倚方從哲為冰山,茍一時富貴,不顧清議。一元論銓政,嘗譏切向高,時按江西,見文炳疏,憤甚,遂揭文炳陰事。且曰:「向高行矣。今秉政者從哲,文炳鄉人,奴顏婢膝,任好為之。」御史馬孟楨亦言:「敬關節實真,既斥兩侍郎、兩給諫謝之矣。乃伉直之劉策,攻擊不休,而同發奸之張篤敬復驅除將及,何太甚也!」疏入,帝皆不省。策憤,謝病去。時攻兆魁、廷弼、賓尹輩者,黨人率指目為東林,以年例出之外。至四十六年秋,在朝者已無可逐,乃即家徙策為河南副使,策辭疾不赴。

天啟元年春,起天津兵備。擢右僉都御史,巡撫山西。召拜兵部右侍郎,協理戎政。五年冬,黨人劾策為東林遺奸,遂削籍。崇禎二年夏,起故官,兼右僉都御史,總理薊、遼、保定軍務。大清兵由大安口入內地,策不能禦,被劾。祖大壽東潰,策偕孫承宗招使還。明年正月與總兵張士顯並逮,論死,棄市。

縉芳,晉江人。為御史,首為顧憲成請謚,劾天津稅監馬堂九大罪,有敢言名。巡兩淮,頗通賓客賂遺,被劾,坐贓。天啟中,遣戍。

一元,侯官人。在江西,振饑有法。移疾去。天啟初,起歷應天府丞。御史餘文縉劾向高,及一元,遂落職。崇禎初,復官。溫體仁柄國,惡其附東林,而以為己門生也,引嫌不召。卒於家。

李若星,字紫垣,息縣人。萬歷三十二年進士。歷知棗強、真定。擢御史,首劾南京兵部尚書黃克纘。巡視庫藏,陳蠹國病商四弊,請得稽十庫出納,以杜侵漁,不報。巡按山西,請撤稅使。因再劾克纘為沈一貫私人、湯賓尹死友,宜罷,不從。還朝,出為福建右參議,移疾歸。

天啟初,起官陜西,召為尚寶少卿,再遷大理右少卿。三年春,以右僉都御史巡撫甘肅。陛辭,發魏忠賢、客氏之奸。明年,遣將丁孟科、官維賢擊河套松山諸部鎮番,斬首二百四十餘級。捷聞,未敘,有傳若星將起義兵清君側之惡者。忠賢聞之,即令許顯純入之汪文言獄詞,誣其賄趙南星,得節鉞。五年三月遂除若星名,下河南撫按提問。明年,獄上,杖之百,戍廉州。

莊烈帝即位,赦還。崇禎元年,起工部右侍郎兼右僉都御史,總理河道。追論甘肅功,進秩二品。黃河大決,淹泗州,沒睢寧城。若星請修祖陵,移睢寧縣治他所,從之。都城戒嚴,遣兵入衛。病歸,遭父憂。久之,召為兵部右侍郎。十一年,以本官兼右僉都御史,代朱燮元總督川、湖、雲、貴軍務,兼巡撫貴州。討安位餘孽安隴璧及苗仲諸賊有功。

福王時,解職。以鄉邑殘破,寓居貴州。桂王遷武岡,召為吏部尚書。未赴,遭亂,死於兵。

耿如杞,字楚材,館陶人。萬曆四十四年進士。除戶部主事。

天啟初,以才歷職方郎中。軍書旁午,日應數十事。出為陜西參議,遷遵化兵備副使。當是時,逆奄竊柄,諂子無所不至,至建祠祝禧。巡撫劉詔懸忠賢畫像於喜峰行署,率文武將吏五拜三稽首,呼九千歲。如杞見其像,冕旒也,半揖而出。忠賢令詔劾之,逮下詔獄,坐贓六千三百,論死。

時又有胡士容者,薊州參議也,數忤其鄉官崔呈秀,呈秀銜之。將為忠賢建祠,士容又不奉命。及士容遷江西副使,道通州,遂誣以多乘驛馬,侵盜倉儲,捕下詔獄掠治,坐贓七千,論死。

至秋,將行刑,而莊烈帝即位,崔、魏相繼伏誅。帝曰:「廠衛深文,附會鍛煉,朕深痛焉。其赦耿如杞,予復原官。胡士容等改擬。」於是如杞上疏言:「臣自入鎮撫司,五毒並施,縛赴市曹者,日有聞矣。幸皇上赦臣以不死,驚魂粗定,乞放臣還家養疾。」帝不許,立擢如杞右僉都御史,巡撫山西。

插漢虎墩兔據順義王地,為邊患,戰款無定策。如杞言守邊為上,修塞垣,繕戰壘,鏟山塹谷,事有緒矣。二年,京師戒嚴,如杞率總兵官張鴻功以勍卒五千人赴援,先至京師。軍令,卒至之明日,汛地既定,而後乃給餉。如杞兵既至,兵部令守通州,明日調昌平,又明日調良鄉,汛地累更,軍三日不得餉,乃噪而大掠。帝聞之,大怒,詔逮如杞、鴻功,廷臣莫敢救者。四年竟斬西市。

方如杞之為職方郎也,與主事鹿善繼黨張鶴鳴,排熊廷弼而庇王化貞,疆事由是大壞,及是得罪。

士容既釋出獄,二年除陜西副使,進右參政,卒於官。士容初令長洲,捕豪惡,築婁江石塘,有政聲。

福王時,贈如杞右僉都御史。子章光,進士,尚寶卿。士容,字仁常,廣濟人。

顏繼祖,漳州人。萬曆四十七年進士。歷工科給事中。崇禎元年正月,論工部冗員及三殿敘功之濫,汰去加秩寄俸二百餘人。又極論魏黨李魯生、霍維華罪狀。又有御史袁弘勛者,劾大學士劉鴻訓,錦衣張道濬佐之。繼祖言二人朋邪亂政,非重創,禍無極。帝皆納其言。

遷工科右給事中。三年,巡視京城十六門濠塹,疏列八事,劾監督主事方應明曠職。帝杖斥應明。外城庳薄,議加高厚,繼祖言時絀難舉贏而止。再遷吏科都給事中,疏陳時事十大弊。憂歸。

八年起故官,上言:「六部之政筦於尚書,諸司之務握之正郎,而侍郎及副郎、主事止陪列畫題,政事安得不廢?督撫諸臣獲罪者接踵,初皆由會推。然會推但六科掌篆者為主,卿貳、臺臣罕至。且九卿、臺諫止選郎傳語,有唯諾,無翻異,何名會推?」帝稱善。

尋擢太常少卿,以右僉都御史巡撫山東。分兵扼境上,河南賊不敢窺青、濟。劾故撫李懋芳侵軍餉二萬有奇,被旨嘉獎。十一年,畿輔戒嚴,命繼祖移駐德州。時標下卒僅三千,而奉本兵楊嗣昌令,五旬三更調。後令專防德州,濟南由此空虛。繼祖屢請敕諸將劉澤清、倪寵等赴援,皆逗遛不進。明年正月,大清兵克濟南,執德王。繼祖一人不能兼顧,言官交章劾繼祖,繼祖咎嗣昌,且曰:「臣兵少力弱,不敢居守德之功,不敢不分失濟之罪。請以爵祿還朝廷,以骸骨還父母。」帝不從,逮下獄,棄市。

終崇禎世,巡撫被戮者十有一人:薊鎮王應豸,山西耿如杞,宣府李養沖,登萊孫元化,大同張翼明,順天陳祖苞,保定張其平,山東顏繼祖,四川邵捷春,永平馬成名,順天潘永圖,而河南李仙風被逮自縊,不與焉。

王應豸,掖縣人。為戶部主事,諂魏忠賢,甫三歲,驟至巡撫,加右都御史。崇禎二年春,薊卒索餉,噪而甲,參政徐從治諭散其眾。應豸置毒飯中,欲誘而盡殺之,諸軍復大亂。帝命巡按方大任廉得其剋餉狀,論死。

李養沖,永年人。歷兵部右侍郎,巡撫宣府,崇禎二年既謝事,御史吳玉劾其侵盜撫賞銀七萬,及冒功匿敗諸狀。論死,斃於獄。

張翼明,永城人。以兵部右侍郎巡撫大同。崇禎元年,插漢虎墩兔入犯,殺掠萬計。翼明及總兵官渠家楨不能禦,並坐死。

陳祖苞,海寧人。崇禎十年,以右副都御史巡撫順天,明年坐失事繫獄,飲鴆卒。帝怒祖苞漏刑,錮其子編修之遴,永不敘。

張其平,偃師人。歷右僉都御史,巡撫保定。十一年冬,坐屬邑失亡多,與繼祖駢死西市。

馬成名,溧陽人。潘永圖,金壇人,與成名為姻婭。崇禎十四年冬,成名以右僉都御史巡撫永平。永圖亦起昌平兵備僉事,未浹歲,至巡撫。畿輔被兵,成名、永圖並以失機,十六年斬西市。餘自有傳。

李繼貞,字征尹,太倉州人。萬曆四十一年進士。除大名推官,歷遷兵部職方主事。天啟四年秋,典試山東,坐試錄刺魏忠賢,降級,已而削籍。

崇禎元年,起武選員外郎,進職方郎中。時軍書旁午,職方特增設郎中,協理司事。繼貞幹用精敏,尚書熊明遇深倚信之,曰:「副將以下若推擇,我畫諾而已。」四年,孔有德反山東,明遇主撫,繼貞疏陳不可,且請調關外兵入剿。明遇不能從,後訖用其言滅賊。初,延綏盜起,繼貞請發帑金,用董摶霄人運法,糴米輸軍前。且令四方贖鍰及捐納事例者,輸粟於邊,以撫饑民。又言:「兵法撫、剿並用,非撫賊也,撫饑民之從賊者耳。今斗米四錢,已從賊者猶少,未從賊而勢必從賊者無窮。請如神廟特遣御史振濟故事,齎三十萬石以往,安輯饑民,使不為賊,以孤賊勢。」帝感其言,遣御史吳甡以十萬金往。繼貞少之,帝不聽,後賊果日熾。

繼貞為人強項,在事清執,請謁不得行。大學士周延儒,繼貞同年生,屬總兵官於繼貞。繼貞瞠目謝曰:「我不奉命,必獲罪。刑部獄甚寬,可容繼貞也。」延儒銜之。已,加尚寶寺卿。當遷,帝輒令久任。田貴妃父弘遇以坐門功求優敘不獲,屢疏詆繼貞,帝不聽。中官曹化淳欲用私人為把總,繼貞不可;乃囑戎政尚書陸完學言於尚書張鳳翼以命繼貞,繼貞亦不可,鳳翼排繼貞議而用之。化淳怒,與弘遇日伺其隙,讒之帝,坐小誤,貶三秩。會敘甘肅功,繼貞請起用故巡撫梅之煥,帝遂發怒,削繼貞籍。已,論四川桃紅壩功,復官,致仕。

十一年用薦起,歷兩京尚寶卿。明年春召對,陳水利屯田甚悉,遷順天府丞。尋超拜兵部右侍郎兼右僉都御史,巡撫天津,督薊、遼軍餉。乃大興屯田,列上經地、招佃、用水、任人、薄賦五議。白塘、葛沽數十里間,田大熟。

十四年冬,詔發水師援遼,坐戰艦不具,除名。明年夏,召為兵部添注右侍郎。得疾,卒於途。是夕,星隕中庭。贈右都御史,官一子。

方震孺,字孩未,桐城人,移家壽州。萬曆四十一年進士。由沙縣知縣入為御史。

熹宗嗣位,逆璫魏忠賢內結客氏。震孺疏陳三朝艱危,言:「宮妾近侍,嚬笑易假,窺瞷可慮。中旨頻宣,恐蹈斜封隱禍。」元年陳《拔本塞源論》曰:「曩者梃擊之案,王之寀、陸大受、張庭、李俸悉遭廢斥,而東林如趙南星、高攀龍、劉宗周諸賢,廢錮終身,亟宜召復。至楊漣之爭移宮,可幸無罪,不知何以有居功之說,又有交通之疑?將使天下後世謂堯、舜在上,而有交通矯旨之閹宦。」疏入,直聲震朝廷。其春巡視南城。中官張曄、劉朝被訟,忠賢為請,震孺不從,卒上聞,忠賢益恚怨。

遼陽既破,震孺一日十三疏,請增巡撫,通海運,調邊兵,易司馬。日五鼓撾公卿門,籌畫痛哭,而自請犒師。是時,三岔河以西四百里,人煙絕,軍民盡竄,文武將吏無一騎東者。帝壯其言,發帑金二十萬震孺犒師。六月,震孺出關,延見將士,弔死扶傷,軍民大悅。因上言:「河廣不七十步,一葦可航,非有驚濤怒浪之險,不足恃者一。兵來,斬木為排,浮以土,多人推之,如履平地,不足恃者二。河去代子河不遠,兵從代子徑渡,守河之卒不滿二萬,能望其半渡而遏之乎?不足恃者三。沿河百六十里,築城則不能,列柵則無用,不足恃者四。黃泥窪、張叉站沖淺之處,可修守,今地非我有,不足恃者五。轉眼冰合,遂成平地,間次置防,猶得五十萬人,兵從何來?不足恃者六。」又言:「我以退為守,則守不足;我以進為守,則守有餘。專倚三岔作家,萬一時事偶非,榆關一線遂足鎖薊門哉?」疏入,帝命震孺巡按遼東,監紀軍事。

震孺按遼,居不廬、食不火者七月。議者欲棄三岔河,退守廣寧,震孺請駐兵振武。軍法不嚴,震孺請敕寧前監軍,專斬逃軍逃將。並從其言。然是時,經撫不和,疆事益壞。震孺再疏言山海無外衛,宜亟駐兵中前,以為眼目,不省。

明年正月,任滿,候代前屯,而大清兵已再渡三岔河。先鋒孫得功不戰,而呼於振武曰「兵敗矣」,遂走。巡撫王化貞在廣寧,亦倉皇走。列城聞之皆走,惟震孺前屯無動。當是時,西平守將羅一貫已戰死,參將祖大壽擁殘兵駐覺華島上。於是震孺召水師帥張國卿相與謀曰:「今東師四外搜糧,聞祖將軍在島上有米豆二十餘萬,兵十餘萬,人民數萬,戰艦、器仗、馬牛無數,東師即媾得島兵,得島兵以攻榆關,豈有幸哉?」於是震孺、國卿航海見大壽,慷慨語曰:「將軍歸,相保以富貴;不歸,震孺請以頸血濺將軍。」大壽泣,震孺亦泣,遂相攜以歸,獲軍民輜重無算。

有主事徐大化者,忠賢黨也,劾震孺曰「攘差」。都御史鄒元標奮筆曰:「方御史保全山海,無過且有社稷功。」給事中郭興治遂借道學以逐元標。元標去,震孺亦即罷歸。明年,忠賢、廣微興大獄,再募劾方御史者,興治再論震孺河西贓私。逮問掠治,坐贓六千有奇,擬絞。而揚州守劉鐸咒詛之獄又起,遂誣震孺與交通,坐大辟,繫獄。有邏卒時時佐震孺飲啖,問之,則曰:「小人有妻,聞公精忠,手治以獻者也。」輒報璫曰:「某病革,某瀕死。」璫以是防益疏。

明年,莊烈帝嗣位,得釋還。八年春,流賊犯壽州,州長吏適遷秩去,震孺倡士民固守,賊自是不敢逼壽州。巡撫史可法上其功,用為廣西參議。尋擢右僉都御史,巡撫廣西。京師陷,福王立南京,即日拜疏勤王。馬士英、阮大鋮憚之,敕還鎮。震孺竟鬱鬱憂憤而卒。

徐從治,字仲華,海鹽人。母夢神人舞戈於庭,寤而生。從治舉萬曆三十五年進士,除桐城知縣。累官濟南知府,以卓異遷兗東副使,駐沂州。

天啟元年,妖賊徐鴻儒反鄆城,連陷鄒、滕、嶧縣。從治捕得其黨之伏沂者殺之,請就家起故總兵楊肇基主兵事,而獻搗賊中堅之策,遂滅鴻儒。事詳《趙彥傳》。

從治警敏通變,其禦賊類主剿不主撫,故往往滅賊。旋以右參政分守濟南。錄功,從治最,進右布政使,督漕江南。妖賊再起,巡撫王惟儉奏留從治,仍守沂。按臣主撫,從治議不合,遂告歸。

中外計議調,崇禎初,以故秩飭薊州兵備。薊軍久缺餉,圍巡撫王應豸於遵化。從治單騎馳入,陰部署夷丁、標兵,分營四門,按甲不動,登城而呼曰:「給三月糧,趣歸守汛地,否將擊汝!」眾應聲而散。其應變多類此。進秩左布政使,再請告歸。

四年,起飭武德兵備。孔有德反山東,巡撫余大成檄從治監軍。明年正月馳赴萊州,而登州已陷。大成削籍,遂擢從治右副都御史代之,與登萊巡撫謝璉並命。詔璉駐萊州,從治駐青州,調度兵食。從治曰:「吾駐青,不足鎮萊人心;駐萊,足係全齊命。」乃與璉同受事於萊。

有德者,遼人。與耿仲明、李九成、毛承祿輩皆毛文龍帳下卒也。文龍死,走入登州。登萊巡撫孫元化官遼久,素言遼人可用,乃用承祿為副將,有德、仲明為遊擊,九成為偏裨,且多收遼人為牙兵。是年,大凌河新城被圍,部檄元化發勁卒泛海,趨耀州鹽場,示牽制。有德詭言風逆,改從陸赴寧遠。十月晦,有德及九成子千總應元統千餘人以行,經月抵吳橋,縣人罷市,眾無所得食。一卒與諸生角,有德抶之,眾大嘩。九成先齎元化銀市馬塞上,用盡無以償,適至吳橋。聞眾怨,遂與應元謀,劫有德,相與為亂,陷陵縣、臨邑、商河,殘齊東,圍德平。既而舍去,陷青城、新城,整眾東。

余大成者,江寧人也。不知兵。初為職方,嘗奏發大學士劉一燝私書,齮之去。後又以事忤魏忠賢,削籍歸,有清執名。而巡撫山東,則白蓮妖賊方熾,又有逃兵之變,皆不能討。及聞有德叛,即託疾數日不能出,不得已遣中軍沈廷諭參將陶廷鑨往禦,則皆敗而走。大成恐,遂定議撫,而元化軍亦至。

元化者,故所號善西洋大炮者也,至是亦主撫,檄賊所過郡縣無邀擊。賊長驅,無敢一矢加者。賊佯許元化降。元化師次黃山館而返,賊遂抵登州。元化遣將張燾率遼兵駐城外,總兵張可大率南兵拒賊。元化猶招降賊,賊不應。五年正月戰城東,遼兵遽退,南兵遂敗。燾兵多降賊,賊遣之歸,士民爭請拒勿內,元化不從,賊遂入。日夕,城中火起,中軍耿仲明、都司陳光福等導賊入自東門,城遂陷。可大死之。元化自刎不殊,與參議宋光蘭、僉事王征及府縣官悉被執。大成馳入萊州。

初,登州被圍,朝廷鐫大成、元化三級,令辦賊。及登失守,革元化職,而以謝璉代。有德既破登州,推九成為主,己次之,仲明又次之。用巡撫印檄州縣餉,趣元化移書求撫於大成曰:「畀以登州一郡,則解。」大成聞於朝。帝怒,命革大成職,而以從治代。

先是,賊攻破黃縣,知縣吳世揚死之。至是,攻萊,從治、璉與總兵楊御蕃等分陴守。御蕃,肇基子。肇基,從治所共剿滅妖賊鄒、滕者也。御蕃積戰功至通州副總兵。會登州陷,兵部尚書熊明遇奏署總兵官,盡將山東兵,與保定總兵劉國柱、天津總兵王洪兼程進。遇賊新城,洪先走。御蕃拒之二日,不勝,突圍出,遂入萊城,從治、璉倚以剿賊。賊攻萊不下,分兵陷平度,知州陳所問自經。賊益攻萊,輦元化所製西洋大炮,日穴城,城多頹。從治等投火灌水,穴者死無算。使死士時出掩擊之,毀其炮臺,斬獲多。而明遇卒惑大成撫議也,命主事張國臣為贊畫往撫之,曰「安輯遼人之在山東者」,以國臣亦遼人也。國臣先遣廢將金一鯨入賊營,已而國臣亦入,為賊移書,遣一鯨還報曰:「毋出兵壞撫局。」從治等知其詐,叱退一鯨,遣間使三上疏,言賊不可撫。最後言:「萊城被圍五十日,危如纍卵。日夜望援兵,卒不至,知必為撫議誤矣。國臣致書臣,內抄詔旨並兵部諭帖,乃知部臣已據國臣報,達聖聽。夫國臣桑梓情重,忍欺聖明而陷封疆。其初遣一鯨入賊營,何嘗有止兵不攻之事?果止兵,或稍退舍,臣等何故不樂撫?特國臣以撫為賊解,而賊實借撫為緩兵計。一鯨受賊賄,對援師則誑言賊數萬,不可輕進;對諸將則誑言賊用西洋炮攻,城將陷矣,賴我招撫,賊即止攻。夫一鯨三入賊營,每入,賊攻益急。而國臣乃云賊嗔我縋城下擊,致彼之攻。是使賊任意攻擊,我不以一矢加遺,如元化斷送登城,然後可成國臣之撫耶?當賊過青州,大成擁兵三千,剿賊甚易。元化遺書謂『賊已就撫,爾兵毋東』,大成遂止勿追,致賊延蔓。今賊視臣等猶元化,乃為賊解,曰吳橋激變有因也,一路封刀不殺也,聞天子詔遂止攻掠也。將誰欺!盈庭中國臣妄報,必謂一紙書賢於十萬兵,援師不來,職是故矣。臣死當為厲鬼以殺賊,斷不敢以撫謾至尊,淆國是,誤封疆,而戕生命也。」疏入,未報。

當是時,外圍日急,國柱、洪及山東援軍俱頓昌邑不敢進,兩撫臣困圍城中。於是廷議更設總督一人,以兵部右侍郎劉宇烈任之。調薊門、四川兵,統以總兵鄧,調密雲兵,統以副將牟文綬,以右布政使楊作楫監之,往援萊。三月,宇烈、作楫、國柱、洪、及監視中官呂直,巡按御史王道純,義勇副將劉澤清,新兵參將劉永昌、朱廷祿,監紀推官汪惟效等並集昌邑。、國柱、洪、澤清等至萊州,馬步軍二萬五千,氣甚盛。而宇烈無籌略,諸師懦怯,抵沙河,日十輩往議撫,縱還所獲賊陳文才。於是賊盡得我虛實,益以撫愚我,而潛兵繞其後,盡焚我輜重。宇烈懼,遂走青州,撤三將兵就食。等夜半拔營散,賊乘之,大敗。洪、國柱走青、濰,走昌邑,澤清接戰於萊城,傷二指,亦敗走平度,惟作楫能軍。三將既敗,舉朝嘩然,而明遇見官軍不可用,撫議益堅。

先是,登州總兵可大死,以副將吳安邦代之,安邦尤怯鈍。奉令屯寧海,規取登州。仲明揚言以城降,安邦信之,離城二十五里而軍。中軍徐樹聲薄城被擒,安邦走還寧海。登既不能下,而賊困萊久,璉、從治、御蕃日堅守待救。至四月十六日,從治中炮死,萊人大臨,守陴者皆哭。

山東士官南京者,合疏攻宇烈,請益兵。於是調昌平兵三千,以總兵陳洪範統之,洪範亦遼人。明遇日跂望曰:「往哉,其可撫也。」天津舊將孫應龍者,大言於眾曰:「仲明兄弟與我善,我能令其縛有德、九成來。」巡撫鄭宗周予之兵二千,從海道往。仲明聞之,偽函他死人頭紿之曰:「此有德也。」應龍率舟師抵水城。延之入,猝縛斬之,無一人脫者。賊得巨艦,勢益張。島帥黃龍攻之不克而還。遂破招遠,圍萊陽。知縣梁衡固守,賊敗去。

宇烈復至昌邑,洪範、文綬等亦至。萊州推官屈宜陽請入賊營講撫,賊佯禮之。宜陽使言賊已受命,宇烈奏得請,乃手書諭賊令解圍。賊邀宇烈,宇烈懼不往。營將嚴正中舁龍亭及河,賊擁之去,而令宜陽還萊,文武官出城開讀,圍即解。御蕃不可,璉曰:「圍且六月,既已無可奈何,宜且從之。」遂偕監視中官徐得時、翟升,知府朱萬年出。有德等叩頭扶伏,涕泣交頤,璉慰諭久之而還。明日復令宜陽入,請璉、御蕃同出。御蕃曰:「我將家子,知殺賊,何知撫事?」璉等遂出。有德執之,猝攻城,卻令萬年呼降。萬年呼曰:「吾死矣,汝等宜固守。」罵不絕口而死。賊送璉及二中官至登囚之,正中、宜陽皆死。

初,撫議興,獨從治持不可。宇烈諸將信之,而尚書明遇主其議。從治死,璉遂被擒。於是舉朝恚憤,逮宇烈下獄,調關外勁卒剿之,罷總督及登萊巡撫不設,專任代從治者朱大典以行。明遇坐主撫誤國,罷歸,遂絕撫議。八月,大典合兵救萊。兵甫接,賊輒大敗,圍解。有德走登州,九成殺璉及二中官。大典圍登,九成戰死。城破,追剿,有德、仲明入海遁。生擒承祿等,斬應元,賊盡平。事詳《大典傳》。詔贈從治兵部尚書,賜祭葬,蔭錦衣百戶,建祠曰「忠烈」;贈璉兵部右侍郎,亦賜祭葬,建祠,廕子;以御蕃功多,加署都督同知,總兵,鎮登、萊。而宇烈以次年遣戍。璉,字君實,監利人。宇烈,綿竹人,大學士宇亮兄也。其戍也,人以為失刑。大成逮下獄,遣戍。赦還,卒於家。

元化,字初陽,嘉定人。天啟間舉於鄉。所善西洋炮法,蓋得之徐光啟云。廣寧覆沒,條備京、防邊二策。孫承宗請於朝,得贊畫經略軍前。主建炮臺教練法,因請據寧遠、前屯,以策干王在晉,在晉不能用。承宗行邊,還奏,授兵部司務。承宗代在晉,遂破重關之非,築臺製炮,一如元化言。還授元化職方主事,已,元化贊畫袁崇煥寧遠。還朝,尋罷。

崇禎初,起武選員外郎,進職方郎中。崇煥已為經略,乞元化自輔,遂改元化山東右參議,整飭寧前兵備。三年,皮島副將劉興治為亂,廷議復設登萊巡撫,遂擢元化右僉都御史任之,駐登州。明年,島眾殺興治,元化奏副將黃龍代,汰其兵六千人。及有德反,朝野由是怨元化之不能討也。賊縱元化還,詔逮之。首輔周延儒謀脫其死,不得也;則援其師光啟入閣圖之,卒不得,同張燾棄市。光蘭、徵充軍。

贊曰:疆圉多故,則思任事之臣。梅之煥諸人,風采機略尚大異於巽懦恇怯之徒,而牽於文法,或廢或死,悲夫!叛將衡行,縛而斬之,一偏裨力耳。中撓撫議,委堅城畀之,援師觀望不進,徒擾擾焉。設官命將,何益之有?撫議之誤國也,可勝言哉!


\end{pinyinscope}