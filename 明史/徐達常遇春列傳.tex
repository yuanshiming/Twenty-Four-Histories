\article{徐達、常遇春列傳}

\begin{pinyinscope}
徐達,字天德,濠人,世業農。達少有大志,長身高顴若干斷片。,剛毅武勇。太祖之為郭子興部帥也,達時年二十二,往從之,一見語合。及太祖南略定遠,帥二十四人往,達首與焉。尋從破元兵於滁州澗,從取和州,子興授達鎮撫。子興執孫德崖,德崖軍亦執太祖,達挺身詣德崖軍請代,太祖乃得歸,達亦獲免。從渡江,拔采石,取太平,與常遇春皆為軍鋒冠。從破擒元將陳野先,別將兵取溧陽、溧水,從下集慶。太祖身居守,而命達為大將,帥諸軍東攻鎮江,拔之。號令明肅,城中宴然。授淮興翼統軍元帥。

時張士誠已據常州,挾江東叛將陳保二以舟師攻鎮江。達敗之於龍潭,遂請益兵以圍常州。士誠遣將來援。達以敵狡而銳,未易力取,乃離城設二伏以待,別遣將王均用為奇兵,而自督軍戰。敵退走遇伏,大敗之,獲其張、湯二將,進圍常州。明年克之。進僉樞密院事。繼克寧國,徇宜興,使前鋒趙德勝下常熟,擒士誠弟士德。明年復攻宜興,克之。太祖自將攻婺州,命達留守應天,別遣兵襲破天完將趙普勝,復池州。遷奉國上將軍、同知樞密院事。進攻安慶,自無為陸行,夜掩浮山寨,破普勝部將於青山,遂克潛山。還鎮池州,與遇春設伏,敗陳友諒軍於九華山下,斬首萬人,生擒三千人。遇春曰:「此勁旅也,不殺為後患。」達不可,乃以狀聞。而遇春先以夜坑其人過半,太祖不懌,悉縱遣餘眾。於是始命達盡護諸將。陳友諒犯龍江,達軍南門外,與諸將力戰破之,追及之慈湖,焚其舟。

明年,從伐漢,取江州。友諒走武昌,達追之。友諒出戰艦沔陽,達營漢陽沌口以遏之。進中書右丞。明年,太祖定南昌,降將祝宗、康泰叛。達以沌口軍討平之。從援安豐,破吳將呂珍,遂圍廬州。會漢人寇南昌,太祖召達自廬州來會師,遇於鄱陽湖。友諒軍甚盛,達身先諸將力戰,敗其前鋒,殺千五百人,獲一巨舟。太祖知敵可破,而慮士誠內犯,即夜遣達還守應天,自帥諸將鏖戰,竟斃友諒。

明年,太祖稱吳王,以達為左相國。復引兵圍廬州,克其城。略下江陵、辰州、衡州、寶慶諸路,湖、湘平。召還,帥遇春等徇淮東,克泰州。吳人陷宜興,達還救復之。復引兵渡江,克高郵,俘吳將士千餘人。會遇春攻淮安,破吳軍於馬騾港,守將梅思祖以城降。進破安豐,獲元將忻都,走左君弼,盡得其運艘。元兵侵徐州,迎擊,大破之,俘斬萬計。淮南、北悉平。

師還,太祖議征吳。右相國李善長請緩之。達曰:「張氏汰而苛,大將李伯昇輩徒擁子女玉帛,易與耳。用事者,黃、蔡、葉三參軍,書生不知大計。臣奉主上威德,以大軍蹙之,三吳可計日定。」太祖大悅,拜達大將軍,平章遇春為副將軍,帥舟師二十萬人薄湖州。敵三道出戰,達亦分三軍應之,別遣兵扼其歸路。敵戰敗返走,不得入城。還戰,大破之,擒將吏二百人,圍其城。士誠遣呂珍等以兵六萬赴救,屯舊館,築五寨自固。達使遇春等為十壘以遮之。士誠自以精兵來援,大破之於皁林。士誠走,遂拔昇山水陸寨。五太子、朱暹、呂珍等皆降,以徇於城下,湖州降。遂下吳江州,從太湖進圍平江。達軍葑門,遇春軍虎丘,郭子興軍婁門,華雲龍軍胥門,湯和軍閶門,王弼軍盤門,張溫軍西門,康茂才軍北門,耿炳文軍城東北,仇成軍城西南,何文輝軍城西北,築長圍困之。架木塔與城中浮屠等。別築臺三成,瞰城中,置弓弩火筒。臺上又置巨礮,所擊輒糜碎。城中大震。達遣使請事,太祖敕勞之曰:「將軍謀勇絕倫,故能遏亂略,削群雄。今事必稟命,此將軍之忠,吾甚嘉之。然將在外,君不御。軍中緩急,將軍其便宜行之,吾不中制。」既而平江破,執士誠,傳送應天,得勝兵二十五萬人。城之將破也,達與遇春約曰:「師入,我營其左,公營其右。」又令荅士曰:「掠民財者死,毀民居者死,離營二十里者死。」既入,吳人安堵如故。師還,封信國公。

尋拜征虜大將軍,以遇春為副,帥步騎二十五萬人,北取中原,太祖親祃於龍江。是時稱名將法」。,必推達、遇春。兩人才勇相類,皆太祖所倚重。遇春剽疾敢深入,而達尤長於謀略。遇春下城邑不能無誅僇,達所至不擾,即獲壯士與諜,結以恩義,俾為己用。由此多樂附大將軍者。至是,太祖諭諸將御軍持重有紀律,戰勝攻取得為將之體者,莫如大將軍達。又謂達,進取方略,宜自山東始。師行,克沂州,降守將王宣。進克嶧州,王宣復叛,擊斬之。莒、密、海諸州悉下。乃使韓政分兵扼河,張興祖取東平、濟寧,而自帥大軍拔益都,徇下濰、膠諸州縣。濟南降,分兵取登、萊。齊地悉定。

洪武元年,太祖即帝位,以達為右丞相。冊立皇太子,以達兼太子少傅。副將軍遇春克東昌,會師濟南,擊斬樂安反者。還軍濟寧,引舟師溯河,趨汴梁,守將李克彞走,左君弼、竹貞等降。遂自虎牢關入洛陽,與元將脫因帖木兒大戰洛水北,破走之。梁王阿魯溫以河南隆,略定嵩、陜、陳、汝諸州,遂搗潼關。李思齊奔鳳翔,張思道奔鹿阜城,遂入關,西至華州。

捷聞,太祖幸汴梁,召達詣行在所,置酒勞之,且謀北伐。達曰:「大軍平齊魯,掃河洛,王保保逡巡觀望;潼關既克,思齊輩狼狽西奔。元聲援已絕,今乘勢直搗元都,可不戰有也。」帝曰:「善。」達復進曰:「元都克,而其主北走,將窮追之乎?」帝曰:「元運衰矣,行自澌滅,不煩窮兵。出塞之後,固守封疆,防其侵軼可也。」達頓首受命。遂與副將軍會師河陰,遣裨將分道徇河北地,連下衛輝、彰德、廣平。師次臨清,使傅友德開陸道通步騎,顧時浚河通舟師,遂引而北。遇春已克德州,合兵取長蘆,扼直沽,作浮橋以濟師。水陸並進,大敗元軍於河西務,進克通州。順帝帥后妃太子北去。踰日,達陳兵齊化門,填濠登城。監國淮王帖木兒不花,左丞相慶童,平章迭兒必失、樸賽因不花,右丞張康伯,御史中丞滿川等不降,斬之,其餘不戮一人。封府庫,籍圖書寶物,令指揮張勝以兵千人守宮殿門,使宦者護視諸宮人、妃、主,禁士卒毋所侵暴。吏民安居,市不易肆。

捷聞,詔以元都為北平府,置六衛,留孫興祖等守之,而命達與遇春進取山西。遇春先下保定、中山、真定,馮勝、湯和下懷慶,度太行,取澤、潞,達以大軍繼之。時擴廓帖木兒方引兵出雁門,將由居庸以攻北平。達聞之,與諸將謀曰:「擴廓遠出,太原必虛。北平有孫都督在,足以禦之。今乘敵不備,直搗太原,使進不得戰,退無所守,所謂批亢搗虛者也。彼若西還自救,此成擒耳。」諸將皆曰:「善。」乃引兵趨太原。擴廓至保安,果還救。達選精兵夜襲其營。擴廓以十八騎遁去。盡降其眾,遂克太原。乘勢收大同,分兵徇未下州縣。山西悉平。

二年引兵西渡河。至鹿臺,張思道遁,遂克奉元。時遇春下鳳翔,李思齊走臨洮,達會諸將議所向。皆曰:「張思道之才不如李思齊,而慶陽易於臨洮,請先慶陽。」達曰:「不然,慶陽城險而兵精,猝未易拔也。臨洮北界河、湟,西控羌、戎,得之,其人足備戰鬥,物產足佐軍儲。蹙以大兵,思齊不走,則束手縛矣。臨洮既克,於旁郡何有?」遂渡隴,克秦州,下伏羌、寧遠,入鞏昌,遣右副將軍馮勝逼臨洮,思齊果不戰降。分兵克蘭州,襲走豫王,盡收其部落輜重。還出蕭關,下平涼。思道走寧夏,為擴廓所執,其弟良臣以慶陽降。達遣薛顯受之。良臣復叛,夜出兵襲傷顯。達督軍圍之。擴廓遣將來援,逆擊敗去,遂拔慶陽。良臣父子投於井,引出斬之。盡定陜西地。詔達班師,賜白金文綺甚厚。

將論功大封,會擴廓攻蘭州,殺指揮使,副將軍遇春已卒,三年春帝復以達為大將軍,平章李文忠為副將軍,分道出兵。達自潼關出西道,搗定西,取擴廓。文忠自居庸出東道,絕大漠,追元嗣主。達至定西,擴廓退屯沈兒峪,進軍薄之。隔溝而壘,日數交。擴廓遣精兵從間道劫東南壘,左丞胡德濟倉卒失措,軍驚擾,達帥兵擊卻之。德濟,大海子也,達以其功臣子,械送之京師,而斬其下指揮等數人以徇。明日,整兵奪溝,殊死戰,大破擴廓兵。擒郯王、文濟王及國公、平章以下文武僚屬千八百六十餘人,將士八萬四千五百餘人,馬駝雜畜以巨萬計。擴廓僅挾妻子數人奔和林。德濟至京,帝釋之,而以書諭達:「將軍效衛青不斬蘇建耳,獨不見穰苴之待莊賈乎?將軍誅之,則已。今下廷議,吾且念其信州、諸暨功,不忍加誅。繼自今,將軍毋事姑息。」達既破擴廓,即帥師自徽州南一百八渡至略陽,克沔州,入連雲棧,攻興元,取之。而副將軍文忠亦克應昌,獲元嫡孫妃主將相。先後露布聞,詔振旅還京師。帝迎勞於龍江。乃下詔大封功臣,授達開國輔運推誠宣力武臣,特進光祿大夫、左柱國、太傅、中書右丞相參軍國事,改封魏國公,歲祿五千石,予世券。明年帥盛熙等赴北平練軍馬,修城池,徙山後軍民實諸衛府,置二百五十四屯,墾田一千三百餘頃。其冬,召還。

五年復大發兵征擴廓。達以征虜大將軍出中道,左副將軍李文忠出東道,征西將軍馮勝出西道,各將五萬騎出塞。達遣都督藍玉擊敗擴廓於土刺河。擴廓與賀宗哲合兵力拒,達戰不利,死者數萬人。帝以達功大,弗問也。時文忠軍亦不利,引還。獨勝至西涼獲全勝,坐匿駝馬,賞不行,事具《文忠》、《勝傳》。明年,達復帥諸將行邊,破敵於答剌海,還軍北平,留三年而歸。十四年,復帥湯和等討乃兒不花。已,復還鎮。

每歲春出,冬暮召還,以為常。還輒上將印,賜休沐,宴見歡飲,有布衣兄弟稱,而達愈恭慎。帝嘗從容言:「徐兄功大,未有寧居,可賜以舊邸。」舊邸者,太祖為吳王時所居也。達固辭。一日,帝與達之邸,強飲之醉,而蒙之被,舁臥正寢。達醒,驚趨下階,俯伏呼死罪。帝覘之,大悅。乃命有司即舊邸前治甲第,表其坊曰「大功」。胡惟庸為丞相,欲結好於達,達薄其人,不答,則賂達閽者福壽使圖達。福壽發之,達亦不問;惟時時為帝言惟庸不任相。後果敗,帝益重達。十七年,太陰犯上將,帝心惡之。達在北平病背疽,稍愈,帝遣達長子輝祖齎敕往勞,尋召還。明年二月,病篤,遂卒,年五十四。帝為輟朝,臨喪悲慟不已。追封中山王,謚武寧,贈三世皆王爵。賜葬鐘山之陰,御製神道碑文。配享太廟,肖像功臣廟,位皆第一。

達言簡慮精。在軍,令出不二。諸將奉持凜凜,而帝前恭謹如不能言。善拊循,與下同甘苦,士無不感恩效死,以故所向克捷。尤嚴戢部伍,所平大都二,省會三,郡邑百數,閭井宴然,民不苦兵。歸朝之日,單車就舍,延禮儒生,談議終日,雍雍如也。帝嘗稱之曰:「受命而出,成功而旋,不矜不伐,婦女無所愛,財寶無所取,中正無疵,昭明乎日月,大將軍一人而已。」子四:輝祖、添福、膺緒、增壽。長女為文皇帝后,次代王妃,次安王妃。

輝祖,初名允恭,長八尺五寸,有才氣,以勛衛署左軍都督府事。達薨,嗣爵。以避皇太孫諱,賜今名。數出練兵陜西、北平、山東、河南。元將阿魯帖木兒隸燕府,有異志,捕誅之。還領中軍都督府。建文初,加太子太傅。燕王子高煦,輝祖甥也。王將起兵,高煦方留京師,竊其善馬而逃。輝祖大驚,遣人追之,不及,乃以聞,遂見親信。久之,命帥師援山東,敗燕兵於齊眉山。燕人大懼。俄被詔還,諸將勢孤,遂相次敗績。及燕兵渡江,輝祖猶引兵力戰。成祖入京師,輝祖獨守父祠弗迎。於是下吏命供罪狀,惟書其父開國勳及券中免死語。成祖大怒,削爵幽之私第。永樂五年卒。萬曆中錄建文忠臣,廟祀南都,以輝祖居首。後追贈太師,謚忠貞。

輝祖死踰月,成祖詔群臣:「輝祖與齊、黃輩謀危社稷。朕念中山王有大功,曲赦之。今輝祖死,中山王不可無後。」遂命輝祖長子欽嗣。九年,欽與成國公勇、定國公景昌、永康侯忠等,俱以縱恣為言官所劾。帝宥勇等,而令欽歸就學。十九年來朝,遽辭歸。帝怒,罷為民。仁宗即位,復故爵,傳子顯宗、承宗。承宗,天順初,守備南京,兼領中軍府,公廉恤士有賢聲。卒,子俌嗣。俌字公輔,持重,善容止。南京守備體最隆,懷柔伯施金監以協同守備位俌上。人甫不平,言於朝,詔以爵為序,著為令。弘治十二年,給事中胡易、御史胡獻以災異陳言下獄,俌上章救之。正德中,上書諫遊畋,語切直。嘗與無錫民爭田,賄劉瑾,為時所譏。俌嗣五十二年而卒,贈太傅,謚莊靖。孫鵬舉嗣,嬖其妾,冒封夫人,欲立其子為嫡,坐奪祿。傳子邦瑞,孫維志,曾孫弘基。自承宗至弘基六世,皆守備南京,領軍府事。弘基累加太傅,卒,謚莊武,子文爵嗣。明亡,爵除。

增壽以父任仕至左都督。建文帝疑燕王反,嘗以問增壽。增壽頓首曰:「燕王先帝同氣,富貴已極,何故反!」及燕師起,數以京師虛實輸於燕。帝覺之,未及問。比燕兵渡江,帝召增壽詰之,不對,手劍斬之殿廡下。王入,撫屍哭。即位,追封武陽侯,謚忠愍。尋進封定國公,祿二千五百石。以其子景昌嗣。驕縱,數被劾,成祖輒宥之。成祖崩,景昌坐居喪不出宿,奪冠服歲祿,已而復之。三傳至玄孫光祚,累典軍府,加太師,嗣四十五年卒,謚榮僖。傳子至孫文璧,萬曆中,領後軍府。以小心謹畏見親於帝,數代郊天,加太師。累上書請建儲,罷礦稅,釋逮繫。嗣三十五年卒,謚康惠。再傳至曾孫允禎,崇禎末為流賊所殺。洪武諸功臣,惟達子孫有二公,分居兩京。魏國之後多賢,而累朝恩數,定國常倍之。嘉靖中詔裁恩澤世封,有言定國功弗稱者,竟弗奪也。

添福早卒。膺緒,授尚寶司卿,累遷中軍都督僉事,奉朝請,世襲指揮使。常遇春,字伯仁,懷遠人。貌奇偉,勇力絕人,猿臂善射。初從劉聚為盜,察聚終無成,歸太祖於和陽。未至,困臥田間,夢神人被甲擁盾呼曰:「起起,主君來。」驚寤,而太祖適至,即迎拜。時至正十五年四月也。無何,自請為前鋒。太祖曰:「汝特饑來就食耳,吾安得汝留也。」遇春固請。太祖曰:「俟渡江,事我未晚也。」及兵薄牛渚磯,元兵陳磯上,舟距岸且三丈餘,莫能登。遇春飛舸至,太祖麾之前。遇春應聲,奮戈直前。敵接其戈,乘勢躍而上,大呼跳盪,元軍披靡。諸將乘之,遂拔采石,進取太平。授總管府先鋒,進總管都督。

時將士妻子輜重皆在和州,元中丞蠻子海牙復以舟師襲據采石,道中梗。太祖自將攻之,遣遇春多張疑兵分敵勢。戰既合,遇春操輕舸,衝海牙舟為二。左右縱擊,大敗之,盡得其舟。江路復通。尋命守溧陽,從攻集慶,功最。從元帥徐達取鎮江,進取常州。吳兵圍達於牛塘,遇春往援,破解之,擒其將,進統軍大元帥。克常州,遷中翼大元帥。從達攻寧國,中流矢,裹創斗,克之。別取馬駝沙,以舟師攻池州,下之,進行省都督馬步水軍大元帥。從取婺州,轉同僉樞密院事,守婺。移兵圍衢州,以奇兵突入南門甕城,毀其戰具,急攻之,遂下,得甲士萬人,進僉樞密院事。攻杭州,失利,召還應天。從達拔趙普勝之水寨,從守池州,大破漢兵於九華山下,語具《達傳》。

友諒薄龍灣,遇春以五翼軍設伏,大破之,遂復太平,功最。太祖追友諒於江州,命遇春留守,用法嚴,軍民肅然無敢犯,進行省參知政事。從取安慶。漢軍出江游徼,遇春擊之,皆反走,乘勝取江州。還守龍灣,援長興,俘殺吳兵五千餘人,其將李伯昇解圍遁。命甓安慶城。

先是,太祖所任將帥最著者,平章邵榮、右丞徐達與遇春為三。而榮尤宿將善戰,至是驕蹇有異志,與參政趙繼祖謀伏兵為變。事覺,太祖欲宥榮死,遇春直前曰:「人臣以反名,尚何可宥,臣義不與共生。」太祖乃飲榮酒,流涕而戮之,以是益愛重遇春。

池州帥羅友賢據神山寨,通張士誠,遇春破斬之。從援安豐。比至,呂珍已陷其城,殺劉福通,聞大軍至,盛兵拒守。太祖左右軍皆敗,遇春橫擊其陣,三戰三破之,俘獲士馬無算。遂從達圍廬州。城將下,陳友諒圍洪都,召還。會師伐漢,遇於彭蠡之康郎山。漢軍舟大,乘上流,鋒銳甚。遇春偕諸將大戰,呼聲動天地,無不一當百。友諒驍將張定邊直犯太祖舟,舟膠於淺,幾殆。遇春射中定邊,太祖舟得脫,而遇春舟復膠於淺。有敗舟順流下,觸遇春舟乃脫。轉戰三日,縱火焚漢舟,湖水皆赤,友諒不敢復戰。諸將以漢軍尚強,欲縱之去,遇春獨無言。比出湖口,諸將欲放舟東下,太祖命扼上流。遇春乃溯江而上,諸將從之。友諒窮蹙,以百艘突圍。諸將邀擊之,漢軍遂大潰,友諒死。師還,第功最,賚金帛土田甚厚。從圍武昌,太祖還應天,留遇春督軍困之。

明年,太祖即吳王位,進遇春平章政事。太祖復視師武昌。漢丞相張必先自岳來援。遇春乘其未集,急擊擒之。城中由是氣奪,陳理遂降,盡取荊、湖地。從左相國達取廬州,別將兵略定臨江之沙坑、麻嶺、牛陂諸寨,擒偽知州鄧克明,遂下吉安。圍贛州,熊天瑞固守不下。太祖使使諭遇春:「克城無多殺。茍得地,無民何益?」於是遇春浚壕立柵以困之。頓兵六月,天瑞力盡乃降,遇春果不殺。太祖大喜,賜書褒勉。遇春遂因兵威諭降南雄、韶州,還定安陸、襄陽。復從徐達克泰州,敗士誠援兵,督水軍壁海安壩以遏之。

其秋拜副將軍,伐吳。敗吳軍於太湖,於毘山,於三里橋,遂薄湖州。士誠遣兵來援,屯於舊館,出大軍後。遇春將奇兵由大全港營東阡,更出其後。敵出精卒搏戰,奮擊破之。襲其右丞徐義於平望,盡燔其赤龍船,復敗之於烏鎮,逐北至升山,破其水陸寨,悉俘舊館兵,湖州遂下。進圍平江,軍虎丘。士誠潛師趨遇春,遇春與戰北濠,破之,幾獲士誠。久之,諸將破葑門,遇春亦破閶門以入,吳平。進中書平章軍國重事,封鄂國公。

復拜副將軍,與大將軍達帥兵北征。帝親諭曰:「當百萬眾,摧鋒陷堅,莫如副將軍。不慮不能戰,慮輕戰耳。身為大將,顧好與小校角,甚非所望也。」遇春拜謝。既行,以遇春兼太子少保,從下山東諸郡,取汴梁,進攻河南。元兵五萬陳洛水北。遇春單騎突其陣,敵二十餘騎攢朔刺之。遇春一矢殪其前鋒,大呼馳入,麾下壯士從之。敵大潰,追奔五十餘里。降梁王阿魯溫,河南郡邑以次下。謁帝於汴梁,遂與大將軍下河北諸郡。先驅取德州,將舟師並河而進,破元兵於河西務,克通州,遂入元都。別下保定、河間、真定。

與大將軍攻太原,擴廓帖木兒來援。遇春言於達曰:「我騎兵雖集,步卒未至,驟與戰必多殺傷,夜劫之可得志。」達曰:「善。」會擴廓部將豁鼻馬來約降,且請為內應,乃選精騎夜銜枚往襲。擴廓方燃燭治軍書,倉卒不知所出,跣一足,乘孱馬,以十八騎走大同。豁鼻馬降,得甲士四萬,遂克太原。遇春追擴廓至忻州而還。詔改遇春左副將軍,居右副將軍馮勝上。北取大同,轉徇河東,下奉元路,與勝軍合,西拔鳳翔。

會元將也速攻通州,詔遇春還備,以平章李文忠副之,帥步騎九萬,發北平,徑會州,敗敵將江文清於錦州,敗也速於全寧。進攻大興州,分千騎為八伏。守將夜遁,盡擒之,遂拔開平。元帝北走,追奔數百里。獲其宗王慶生及平章鼎住等將士萬人,車萬輛,馬三千匹,牛五萬頭,子女寶貨稱是。師還,次柳河川,暴疾卒,年僅四十。太祖聞之,大震悼。喪至龍江,親出奠,命禮官議天子為大臣發哀禮。議上,用宋太宗喪韓王趙普故事。制曰「可」。賜葬鐘山原,給明器九十事納墓中。贈翊運推誠宣德靖遠功臣、開府儀同三司、上柱國、太保、中書右丞相,追封開平王,謚忠武。配享太廟,肖像功臣廟,位皆第二。

遇春沉鷙果敢,善撫士卒,摧鋒陷陣,未嘗敗北。雖不習書史,用兵輒與古合。長於大將軍達二歲,數從征伐,聽約束惟謹,一時名將稱徐、常。遇春嘗自言能將十萬眾,橫行天下,軍中又稱「常十萬」云。

遇春從弟榮,積功為指揮同知,從李文忠出塞,戰死臚朐河。遇春二子,茂、昇。

茂以遇春功,封鄭國公,食祿二千石,予世券,驕稚不習事。洪武二十年命從大將軍馮勝徵納哈出於金山。勝,茂婦翁也。茂多不奉勝約束,勝數誚責之。茂應之慢,勝益怒,未有以發也。會納哈出請降,詣右副將軍藍玉營,酒次,與玉相失,納哈出取酒澆地,顧其下咄咄語。茂方在坐,麾下趙指揮者,解蒙古語,密告茂:「納哈出將遁矣。」茂因出不意,直前搏之。納哈出大驚,起欲就馬。茂拔刀,砍其臂傷。納哈出所部聞之,有驚潰者。勝故怒茂,增飾其狀,奏茂激變,遂械繫至京。茂亦言勝諸不法事。帝收勝總兵印,而安置茂於龍州,二十四年卒。初,龍州土官趙貼堅死,從子宗壽當襲。貼堅妻黃以愛女予茂為小妻,擅州事。茂既死,黃與宗壽爭州印,相告訐。或構蜚語,謂茂實不死,宗壽知狀。帝怒,責令獻茂自贖,命楊文、韓觀出師討龍州。已而知茂果死,宗壽亦輸款,乃罷兵。

茂無子,弟昇,改封開國公,數出練軍,加太子太保。昇之沒,《實錄》不載。其他書紀傳謂,建文末,昇及魏國公輝祖力戰浦子口,死於永樂初。或謂昇洪武中坐藍玉黨,有告其聚兵三山者,誅死。常氏為興宗外戚,建文時恩禮宜厚,事遭革除,無可考,其死亦遂傳聞異詞。升子繼祖,永樂元年遷雲南之臨安衛,時甫七歲。繼祖子寧,寧子復。弘治五年詔曰:「太廟配享諸功臣,其贈王者,皆佐皇祖平定天下,有大功。而子孫或不沾寸祿,淪於氓隸。朕不忍,所司可求其世嫡,量授一官,奉先祀。」乃自雲南召復,授南京錦衣衛世指揮使。嘉靖十一年紹封四王後,封復孫玄振為懷遠侯,傳至曾孫延齡,有賢行。崇禎十六年,全楚淪陷,延齡請統京兵赴九江協守。又言江都有地名常家沙,族丁數千皆其始祖遠裔,請鼓以忠義,練為親兵。帝嘉之,不果行。南都諸勛戚多恣睢自肆,獨延齡以守職稱。國亡,身自灌園,蕭然布衣終老。

贊曰:明太祖奮自滁陽,戡定四方,雖曰天授,蓋二王之力多焉。中山持重有謀,功高不伐,自古名世之佐無以過之。開平摧鋒陷陣,所向必克,智勇不在中山下;而公忠謙遜,善持其功名,允為元勛之冠。身依日月,剖符錫土,若二王者,可謂極盛矣。顧中山賞延後裔,世叨榮寵;而開平天不假年,子孫亦復衰替。貴匹勛齊,而食報或爽,其故何也?太祖嘗語諸將曰:「為將不妄殺人,豈惟國家之利,爾子孫實受其福。」信哉,可為為將帥者鑒矣。


\end{pinyinscope}