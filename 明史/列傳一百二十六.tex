\article{列傳一百二十六}

\begin{pinyinscope}
李成梁子如松如伯如楨如樟如梅麻貴兄錦

李成梁,字汝契。高祖英自朝鮮內附,授世鐵嶺衛指揮僉事,遂家焉。成梁英毅驍健,有大將才。家貧,不能襲職,年四十猶為諸生。巡按御史器之,資入京,乃得襲。積功為遼東險山參將。隆慶元年,士蠻大入永平。成梁赴援有功,進副總兵,仍守險山。尋協守遼陽。三年四月,張擺失等屯塞下,成梁迎擊斬之,殲其卒百六十有奇。餘眾遠徙,遂空其地。錄功,進秩一等。四年九月,辛愛大入遼東。總兵官王治道戰死,擢成梁署都督僉事代之。當是時,俺答雖款塞,而插漢部長土蠻與從父黑石炭,弟委正、大委正,從弟煖兔、拱兔,子卜言台周,從子黃台吉勢方強。泰寧部長速把亥、炒花,朵顏部長董狐狸、長昂佐之。東則王杲、王兀堂、清佳砮、楊吉砮之屬,亦時窺塞下。十年之間,殷尚質、楊照、王治道三大將皆戰死。成梁乃大修戎備,甄拔將校,收召四方健兒,給以厚餼,用為選鋒。軍聲始振。

明年五月,敵犯盤山驛,指揮蘇成勛擊走之。無何,土蠻大入。成梁遇於卓山,麾副將趙完等夾擊,斷其首尾。乘勝抵巢,馘部長二人,斬首五百八十餘級。進署都督同知,世廕千戶。又明年十月,土蠻六百騎營舊遼陽北河,去邊二百餘里,俟眾集大舉,成梁擊走之。萬曆元年,又擊走之前屯。已,又破走之鐵嶺鎮西諸堡。增秩二等。朵顏兀魯思罕以四千騎毀牆入,成梁禦卻之。

建州都指揮王杲故與撫順通馬市。及是,誘殺備禦裴承祖,成梁謀討之。明年十月,杲復大舉入。成梁檄副將楊騰、游擊王惟屏分屯要害,而令參將曹簠挑戰。諸軍四面起,敵大奔,盡聚杲寨。寨地高,杲深溝堅壘以自固。成梁用火器攻之,破數柵,矢石雨下。把總於志文、秦得倚先登,諸將繼之。杲走高臺,射殺志文。會大風起,縱火焚之,先後斬馘千一百餘級,毀其營壘而還。進左都督,世廕都指揮同知。杲大創,不能軍,走匿阿哈納寨。曹簠勒精騎往,杲走南關。都督王台執以獻,斬之。

三年春,土蠻犯長勇堡,擊敗之。其冬,炒花大會黑石炭、黃台吉、卜言台周、以兒鄧、煖兔、拱兔、堵剌兒等二萬餘騎,從平虜堡南掠。副將曹簠馳擊,遂轉掠沈陽。見城外列營,乃據西北高墩。成梁邀戰,發火器。敵大潰,棄輜重走。追至河溝,乘勝渡河,擊斬以千計。加太子太保,世廕錦衣千戶。明年,黑石炭、大委正營大清堡邊外,謀錦、義。成梁率選鋒馳二百里,逼其營,攻破之。殺部長四人,獲級六十有奇。五年五月,士蠻復入,聯營河東,而遣零騎西掠。成梁掩其巢,得利而還。明年正月,速把亥糾土蠻大入,營劈山。成梁馳至丁字泊,敵方分騎繞牆入。成梁夜出塞二百里,搗破劈山營,獲級四百三十,馘其長五人。加太保,世廕本衛指揮使。三月,游擊陶承嚳擊敵長定堡,獻馘四百七十有奇。帝已告謝郊廟,大行賞賚,廕成梁世指揮僉事。有言所殺乃土蠻部曲,因盜牛羊事覺,懼罪來歸,承嚳掩殺之。給事中光懋因請治承嚳殺降罪,御史勘如懋言。兵部尚書方逢時,督撫梁夢龍、周詠先與承嚳同敘功,力為解。卒如御史奏,盡奪諸臣恩命。六月,敵犯鎮靜堡,復擊退之。十二月,速把亥、炒花、煖兔、拱兔會土蠻黃台吉,大、小委正,卜兒亥,慌忽太等三萬餘騎壁遼河,攻東昌堡,深入至耀州。成梁遣諸將分屯要害以遏之,而親提銳卒,出塞二百餘里,直搗圜山。斬首八百四十,及其長九人,獲馬千二百匹。敵聞之,皆倉皇走出塞。論功,封寧遠伯,歲祿八百石。是時,土蠻數求貢市,關吏不許,大恨。七年十月,復以四萬騎自前屯錦川營深入。成梁命諸將堅壁,自督參將楊粟等遏其沖。會戚繼光亦來援,敵遂退。俄又與速把亥合壁紅土城,聲言入海州,而分兵入錦、義。成梁踰塞二百餘里,直抵紅土城,擊敗之,獲首功四百七十有奇。

迤東都督王兀堂故通市寬奠,後參將徐國輔弟國臣強抑市價,兀堂乃與趙鎖羅骨數遣零騎侵邊。明年三月,以六百騎犯靉陽及黃岡嶺,指揮王宗義戰死。復以千餘騎從永奠入,成梁擊走之。追出塞二百里。敵以騎卒拒,而步卒登山鼓噪。成梁大敗之,斬首七百五十,盡毀其營壘。捷聞,並錄紅土城功,予成梁世襲。其秋,兀堂復犯寬奠,副將姚大節擊破之。兀堂由是不振。

土蠻數侵邊不得志,忿甚,益徵諸部兵分犯錦、義及右屯、大凌河。以城堡堅,不可克,而成梁及薊鎮兵亦集,乃引去。無何,復以二萬餘騎從大鎮堡入攻錦州。參將熊朝臣固守,而遣部將周之望、王應榮出戰,頗有斬獲。矢盡,皆戰死。敵乃分掠小凌河、松山、杏山。成梁馳援,始出境。九年正月,土蠻復與黑石炭,大、小委正,卜言臺周,腦毛大,黃臺吉,以兒鄧,煖兔,拱兔,炒戶兒聚兵塞下,謀入廣寧。成梁帥輕騎從大寧堡出。去塞四百餘里,至襖郎兔大戰。自辰迄未,敵不支,敗走。官軍將還,敵來追。成梁逆擊,且戰且行。先後斬首三百四十,及其長八人。錄功,增歲祿百石,世廕一等。四月,黑石炭、以兒鄧、小歹青、卜言兔入遼陽。副將曹簠追至長安堡,遇伏,失千總陳鵬以下三百十七人,馬死者四百六十匹,遂大掠人畜而去。簠等下吏,成梁不問。十月,土蠻復連速把亥等十餘萬騎攻圍廣寧,不克,轉掠團山堡、盤山驛及十三山驛,攻義州。成梁禦卻之。十年三月,速把亥率弟炒花、子卜言兔入犯義州。成梁禦之鎮夷堡,設伏待之。速把亥入,參將李平胡射中其脅,墜馬,蒼頭李有名前斬之。寇大奔,追馘百餘級。炒花等慟哭去。速把亥為遼左患二十年,至是死。帝大喜,詔賜甲第京師,世蔭錦衣指揮使。

初,王杲死,其子阿台走依王台長子虎兒罕。以王台獻其父,嘗欲報之。王台死,虎兒罕勢衰,阿台遂附北關合攻虎兒罕。又數犯孤山、汛河。成梁出塞,遇於曹子谷,斬首一千有奇,獲馬五百。阿台復糾阿海連兵入,抵沈陽城南渾河,大掠去。成梁從撫順出塞百餘里,火攻古勒塞,射死阿台。連破阿海寨,擊殺之,獻馘二千三百。杲部遂滅。錄功,增歲祿百石,世蔭指揮僉事。

北關清佳砮、楊吉砮素仇南關。王台沒,屢侵台季子猛骨孛羅,且藉土蠻、煖兔、慌忽太兵侵邊境。其年十二月,巡撫李松使備禦霍九皋許之貢市。清佳砮、楊吉砮率二千餘騎詣鎮北關謁。松、九皋見其兵盛,譙讓之,則以三百騎入。松先伏甲於旁,約二人不受撫則炮舉甲起。頃之,二人抵關,據鞍不遜,松叱之,九皋麾使下,其徒遽拔刀擊九皋,並殺侍卒十餘人。於是軍中砲鳴,伏盡起,擊斬二人并其從騎,與清佳砮子兀孫孛羅、楊吉砮子哈兒哈麻盡殲焉。成梁聞砲,急出塞,擊其留騎,斬首千五百有奇。餘眾刑白馬,攢刀,誓永受約束,乃旋師。錄功,增歲祿二百石,改前蔭指揮僉事為錦衣衛指揮使。方成梁之出塞也,炒花等以數萬騎入蒲河及大寧堡。將士防禦六日,始出塞。

十三年二月,把兔兒欲報父速把亥之怨,偕從父炒花、姑婿花大糾西部以兒鄧等以數萬騎入掠沈陽。既退,駐牧遼河,聲犯開原、鐵嶺。成梁與巡撫李松潛為浮橋濟師,踰塞百五十里,疾掩其帳。寇已先覺,整眾逆戰。成梁為疊陣,親督前陣擊,而松以後陣繼之,斬首八百有奇。捷聞,增歲祿百石,改廕錦衣指揮使為都指揮使。其年五月,敵犯沈陽,伏精騎塞下,誘官軍。游擊韓元功追襲之,敗死。閏九月,諸部長復犯蒲河,殺裨將數人,大剽掠,而西部銀燈亦窺遼、沈。成梁令部將李平胡出塞三百五十里,搗破銀燈營,斬首一百八級。諸部長聞之,始引去。十四年二月,士蠻部長一克灰正糾把兔兒、炒花、花大等三萬騎,約土蠻諸子共馳遼陽挾賞。成梁偵得之,率副將楊燮,參將李寧、李興、孫守廉以輕騎出鎮邊堡。晝伏夜行二百餘里,至可可毋林。大風雷,敵不覺。既至,風日晴朗,敵大驚,發矢如雨。將士冒死陷陣,獲首功九百,斬其長二十四人。其年十月,敵七八萬騎犯鎮夷諸堡,閱五日始去。十五年春,東西部連營入犯。其秋八月,復以七八萬騎犯鎮夷堡。十月,把漢大成糾土蠻十萬騎由鎮夷、大清二堡入,數日始出。

北關既被創,後清佳砮子卜寨與楊吉砮子那林孛羅漸強盛,數與南關虎兒罕子歹商構兵。成梁以南關勢弱,謀討北關以輔翊之。明年五月,率師直搗其巢。卜寨走,與那林孛羅合,憑城守。城四重,攻之不下。用巨炮擊之,碎其外郛,遂拔二城,斬馘五百餘級。卜寨等請降,設誓不復叛,乃班師。

十七年三月,敵犯義州,復入太平堡,把總朱永壽等一軍盡沒。九月,腦毛大合白洪大、長昂三萬騎復犯平虜堡,備禦李有年、把總馮文昇皆戰死,成梁選鋒沒者數百人。敵大掠沈陽蒲河、榆林,八日始去。明年二月,卜言台周,黃台吉,大、小委正結西部叉漢塔塔兒五萬餘騎復深入遼、沈、海、蓋。成梁潛遣兵出塞襲之,遇伏,死者千人。成梁乃報首功二百八十,得增祿廕。土蠻族弟士墨台豬借西部青把都、恰不慎及長昂、滾兔十萬騎深入海州。成梁不敢擊,縱掠數日而去。十九年閏三月,成梁乘給事侯先春閱視,謀邀搗巢功,使副將李寧等出鎮夷堡潛襲板升,殺二百八十人。師還遇敵,死者數千人。成梁及總督蹇達不以聞。巡按御史胡克儉盡發其先後欺罔狀,語多侵政府。疏雖不行,成梁由是不安於位。及先春還朝,詆尤力,帝意頗動。成梁再疏辭疾,言者亦踵至。其年十一月,帝竟從御史張鶴鳴言,解成梁任,以寧遠伯奉朝請。明年,哱拜反寧夏,御史梅國楨請用成梁,給事中王德完持不可,乃寢。

成梁鎮遼二十二年,先後奏大捷者十,帝輒祭告郊廟,受廷臣賀,蟒衣金繒歲賜稠疊。邊帥武功之盛,二百年來未有也。其始銳意封拜,師出必捷,威振絕域。已而位望益隆,子弟盡列崇階,僕隸無不榮顯。貴極而驕,奢侈無度。軍貲、馬價、鹽課、市賞,歲乾沒不貲,全遼商民之利盡籠入己。以是灌輸權門,結納朝士,中外要人,無不飽其重賕,為之左右。每一奏捷,內自閣部,外自督撫而下,大者進官蔭子,小亦增俸賚金。恩施優渥,震耀當世。而其戰功率在塞外,易為緣飾。若敵入內地,則以堅壁清野為詞,擁兵觀望;甚或掩敗為功,殺良民冒級。閣部共為蒙蔽,督撫、監司稍忤意,輒排去之,不得舉其法。先後巡按陳登雲、許守恩廉得其殺降冒功狀,擬論奏之,為巡撫李松、顧養謙所沮止。既而物議沸騰,御史朱應轂、給事中任應徵、僉事李琯交章抨擊。事頗有跡,卒賴奧援,反詰責言者。及申時行、許國、王錫爵相繼謝政,成梁失內主,遂以去位。

成梁諸戰功率藉健兒。其後健兒李平胡、李寧、李興、秦得倚、孫守廉輩皆富貴,擁專城。暮氣難振,又轉相掊克,士馬蕭耗。迨成梁去遼,十年之間更易八帥,邊備益弛。

二十九年八月,馬林獲罪。大學士沈一貫言成梁雖老,尚堪將兵。乃命再鎮遼東,年已七十有六矣。是時,土蠻、長昂及把兔兒已死,寇鈔漸稀。而開原、廣寧之前復開馬、木二市。諸部耽市賞利,爭就款。以故成梁復鎮八年,遼左少事。以閱視敘勞,加至太傅。

當萬曆初元時,兵部侍郎汪道昆閱邊,成梁獻議移建孤山堡於張其哈剌佃,險山堡於寬佃,沿江新安四堡於長佃、長嶺諸處,仍以孤山、險山二參將戍之,可拓地七八百里,益收耕牧之利。道昆上於朝,報可。自是生聚日繁,至六萬四千餘戶。及三十四年,成梁以地孤懸難守,與督、撫蹇達、趙楫建議棄之,盡徙居民於內地。居民戀家室,則以大軍驅迫之,死者狼籍。成梁等反以招復逃人功,增秩受賞。兵科給事中宋一韓力言棄地非策。巡按御史熊廷弼勘奏如一韓言,一韓復連章極論。帝素眷成梁,悉留中不下。久之卒,年九十。

弟成材,參將。子如松、如柏、如楨、如樟、如梅皆總兵官;如梓、如梧、如桂、如楠亦皆至參將。

如松,字子茂,成梁長子。以父蔭為都指揮同知,充寧遠伯勛衛。驍果敢戰,少從父諳兵機。再遷署都督僉事,為神機營右副將。萬曆十一年,出為山西總兵官。給事中黃道瞻等數言如松父子不當並居重鎮,大學士申時行請保全之,乃召僉書右府。尋提督京城巡捕。給事中邵庶嘗劾如松及其弟副總兵如柏不法,且請稍抑,以全終始,不納。十五年,復以總兵官鎮宣府。巡撫許守謙閱操,如松引坐與並。參政王學書卻之,語不相下,幾攘臂。巡按御史王之棟因劾如松驕橫,並詆學書,帝為兩奪其俸。已復被論,給事中葉初春請改調之,乃命與山西李迎恩更鎮。其後,軍政拾遺,給事中閱視,數遭論劾。帝終眷之,不為動,召僉書中府。

二十年,哱拜反寧夏,御史梅國楨薦如松大將才,其弟如梅、如樟並年少英傑,宜令討賊。乃命如松為提督陜西討逆軍務總兵官,即以國楨監之。武臣有提督,自如松始也。已命盡統遼東、宣府、大同、山西諸道援軍。六月抵寧夏。如松以權任既重,不欲受總督制,事輒專行。兵科許弘綱等以為非制,尚書石星亦言如松敕書受督臣節度,不得自專,帝乃下詔申飭。先是,諸將董一奎、麻貴等數攻城不下。如松至,攻益力。用布囊三萬,實以土,踐之登,為砲石所卻。如樟夜攀雲梯上,不克。游擊龔子敬提苗兵攻南關,如松乘勢將登,亦不克,乃決策水攻。拜窘,遣養子克力蓋往勾套寇,如松令部將李寧追斬之。已,套寇以萬餘騎至張亮堡。如松力戰,手斬士卒畏縮者,寇竟敗去。水侵北關,城崩。如松及蕭如薰等佯擊北關誘賊,而潛以銳師襲南關,攀雲梯而上。拜及子承恩自斬叛黨劉東暘、許朝乞貸死。於是如松先登,如薰及麻貴、劉承嗣等繼之,盡滅拜族。錄功,進都督,世廕錦衣指揮同知。

會朝鮮倭患棘,詔如松提督薊、遼、保定、山東諸軍,剋期東征。弟如柏、如梅並率師援剿。如松新立功,氣益驕,與經略宋應昌不相下。故事,大帥初見督師,甲胄庭謁,出易冠帶,始加禮貌。如松用監司謁督撫儀,素服側坐而已。十二月,如松至軍,沈惟敬自倭歸,言倭酋行長願封,請退平壤迄西,以大同江為界。如松叱惟敬憸邪,欲斬之。參謀李應試曰:「藉惟敬紿倭封,而陰襲之,奇計也。」如松以為然,乃置惟敬於營,誓師渡江。

二十一年正月四日,師次肅寧館。行長以為封使將至,遣牙將二十人來迎,如松檄游擊李寧生縛之。倭猝起格鬥,僅獲三人,餘走還。行長大駭,復遣所親信小西飛來謁,如松慰遣之。六日,次平壤。行長猶以為封使也,踔風月樓以待,群倭花衣夾道迎。如松分布諸軍,抵平壤城,諸將逡巡未入,形大露,倭悉登陴拒守。是夜,襲如柏營,擊卻之。明旦,如松下令諸軍無割首級,攻圍缺東面。以倭素易朝鮮軍,令副將祖承訓詭為其裝,潛伏西南。令游擊吳惟忠攻迄北牡丹峰。而如松親提大軍直抵城下,攻其東南。倭砲矢如雨,軍少卻。如松斬先退者以徇。募死士,援鉤梯直上。倭方輕南面朝鮮軍,承訓等乃卸裝露明甲。倭大驚,急分兵捍拒,如松已督副將楊元等軍自小西門先登,如柏等亦從大西門入。火器並發,煙焰蔽空。惟忠中砲傷胸,猶奮呼督戰。如松馬斃於炮,易馬馳,墮塹,躍而上,麾兵益進。將士無不一當百,遂克之。獲首功千二百有奇。倭退保風月樓。夜半,行長渡大同江,遁還龍山。寧及參將查大受率精卒三千潛伏東江間道,復斬級三百六十。乘勝逐北。十九日,如柏遂復開城。所失黃海、平安、京畿、江源四道並復。酋清正據咸鏡,亦遁還王京。

官軍既連勝,有輕敵心。二十七日再進師。朝鮮人以賊棄王京告。如松信之,將輕騎趨碧蹄館。距王京三十里,猝遇倭,圍數重。如松督部下鏖戰。一金甲倭搏如松急,指揮李有聲殊死救,被殺。如柏、寧等奮前夾擊,如梅射金甲倭墜馬,楊元兵亦至,斫重圍入,倭乃退,官軍喪失甚多。會天久雨,騎入稻畦中不得逞。倭背岳山,面漢水,聯營城中,廣樹飛樓,箭砲不絕,官軍乃退駐開城。二月既望,諜報倭以二十萬眾入寇。如松令元軍平壤,扼大同江,接餉道;如柏等軍寶山諸處為聲援;大受軍臨津;留寧、承訓軍開城;而身自東西調度。聞倭將平秀嘉據龍山倉,積粟數十萬,密令大受率死士從間焚之。倭遂乏食。

初,官軍捷平壤,鋒銳甚,不復問封貢事。及碧蹄敗衄,如松氣大索,應昌、如松急欲休息,而倭亦芻糧並絕,且懲平壤之敗,有歸志,於是惟敬款議復行。四月十八日,倭棄王京遁,如松與應昌入城,遣兵渡漢江尾倭後,將擊其惰歸。倭步步為營,分番迭休,官軍不敢擊。倭乃結營釜山,為久留計。時兵部尚書石星力主封貢,議撤兵,獨留劉綎拒守。如松乃以十二月班師。論功,加太子太保,增歲祿百石。言者詆其和親辱國,屢攻擊之。帝不問。

二十五年冬,遼東總兵董一元罷,廷推者三,中旨特用如松。言路復交章力爭,帝置不報。如松感帝知,氣益奮。明年四月,土蠻寇犯遼東。如松率輕騎遠出搗巢,中伏力戰死。帝痛悼,令具衣冠歸葬,贈少保、寧遠伯,立祠,謚忠烈。以其弟如梅代為總兵官,授長子世忠錦衣衛指揮使,掌南鎮撫司,仍充寧遠伯勳衛,復蔭一子本衛指揮使,世襲。恤典優渥,皆出特恩云。世忠未久卒,無子。弟顯忠由蔭歷遼東副總兵,當嗣爵,朝臣方惡李氏,無為言者。至崇禎中,如松妻武氏愬於朝。章下部議,竟寢。後莊烈帝念成梁功,顯忠子尊祖得嗣寧遠伯。闖賊陷京師,遇難。

如柏,字子貞,成梁第二子。由父蔭為錦衣千戶。嘗與客會飲,砲聲徹大內,下吏免官。再以廕為指揮僉事。數從父出塞有功,歷密雲游擊,黃花嶺參將,薊鎮副總兵。萬曆十六年,御史任養心言:「李氏兵權太盛。姻親廝養分操兵柄,環神京數千里,縱橫蟠據,不可動搖。如柏貪淫,跋扈尤甚。不早為計,恐生他變。」帝乃解如伯任。於是成梁上書乞罷,並請盡罷子弟官,帝慰留不許。久之,起故官,署宣府參將。引疾歸。

如松之禦倭朝鮮也,詔如柏署都督僉事,先率師赴援。既拔平壤,如柏疾趨開城,攻克之,斬首百六十有奇。師旋,進都督同知,為五軍營副將。尋出為貴州總兵官。二十三年,改鎮寧夏。著力兔犯平虜、橫城,如伯邀之,大獲,斬首二百七十有奇。進右都督。再以疾歸,家居二十餘年。會遼東總兵官張承廕戰歿,文武大臣英國公張惟賢等合疏薦如柏,詔以故官鎮遼東。蒙古炒花入犯,督諸將擊卻之。

始成梁、如松為將,厚畜健兒,故所向克捷。至是,父兄故部曲已無復存,而如柏暨諸弟放情酒色,亦無復少年英銳。特以李氏世將,起自廢籍中。顧如柏中情怯,惟左次避敵而已。我大清師臨河,如柏故引軍防懿路。及楊鎬四路出師,令如柏以一軍出鴉鶻關。甫抵虎攔路,鎬聞杜松、馬林兩軍已覆,急檄如柏還。大清哨兵二十人見之,登山鳴螺,作大軍追擊狀,如柏軍大驚,奔走相蹴死者千餘人。御史給事中交章論劾,給事中李奇珍連疏爭尤力。帝終念李氏,詔還候勘。既入都,言者不已。如柏懼,遂自裁。

如楨,成梁第三子。由父廕為指揮使。屢加至右都督,並在錦衣。嘗掌南、北鎮撫司,提督西司房,列環衛者四十年。最後,軍政拾遺,部議罷職,章久留不下。如楨雖將家子,然未歷行陣,不知兵。及兄如柏革任,遼人謂李氏世鎮遼東,邊人憚服,非再用李氏不可,巡撫周永春以為言。而是時如柏兄弟獨如楨在,兵部尚書黃嘉善遂徇其請,以如楨名上,帝即可之。時萬歷四十七年四月也。

如楨藉父兄勢,又自以錦衣近臣,不肯居人下。未出關,即遣使與總督汪可受講鈞禮,朝議嘩然,嘉善亦特疏言之。如楨始怏怏去。既抵遼,經略楊鎬使守鐵嶺。鐵嶺故李氏宗族墳墓所在。當如柏還京,其族黨部曲高貲者悉隨之而西,城中為空。後鎬以孤城難守,令如楨還屯沈陽,僅以參將丁碧等防守,力益弱。大清兵臨城,如楨擁兵不救,城遂失。言官交章論列,經略熊廷弼亦論如楨十不堪,乃罷任。天啟初,言者復力攻,下獄論死。崇禎四年,帝念成梁勛,特免死充軍。

如樟,亦由父廕,歷都指揮僉事。從兄如松徵寧夏,先登有功,累進都督僉事。歷廣西、延綏總兵官。

如梅,字子清。亦由父廕,歷都指揮僉事。從兄如松征日本,卻敵先登。屢遷遼東副總兵。二十四年,炒花、卜言兔將入犯,如梅謀先襲之。督部將方時新等出塞三百里,直搗其廬帳,斬首百餘級而還。明年,如梅與參政楊鎬謀復從鎮西堡出塞,潛襲敵營,失利,損部將十人,士卒百六十人。如梅以血戰重創,免罪。

日本封事敗,其年八月,進署都督僉事,充禦倭副總兵,赴朝鮮援剿,時麻貴三路進師,令如梅將左軍,與右軍共攻蔚山。如梅偕參將楊登山騎兵先進,設伏海濱,而令遊擊擺賽以輕騎誘賊,斬首四百有奇,餘賊遁歸島山。副將陳寅冒矢石奮呼上,破柵兩重。至第三柵,垂拔,楊鎬為總理,宿與如梅暱,不欲寅功出其上,遽鳴金收軍。翊日,如梅至,攻之,不能拔。已而賊援至,如梅軍先奔,諸軍亦相繼潰。贊畫主事丁應泰劾鎬,並劾如梅當斬者二,當罪者十,帝不納。旋用為禦倭總兵官。會其兄如松戰歿,即命如梅馳代之。踰年,坐擁兵畏敵,劾罷。久之,起僉書左府。四十年,鎬巡撫遼東,力薦如梅為帥。不得,至以死爭。給事中麻僖、御史楊州鶴力持不可,乃止。

成梁諸子,如松最果敢,有父風,其次稱如梅。然躁動,非大將才,獨楊鎬深信。後復倚任其兄如柏,卒以致敗。

麻貴,大同右衛人。父祿,嘉靖中為大同參將,從鎮帥劉漢襲板升,大獲。俺答圍右衛,祿與副將尚表固守,乘間擊斬其部長,寇乃引退。辛愛犯京東,祿以宣府副總兵入衛,與子遊擊錦並有卻敵功。

貴由舍人從軍,積功至都指揮僉事,充宣府遊擊將軍。隆慶中,遷大同新平堡參將。寇大入,掠山陰、懷仁、應州。將吏並獲罪,獨貴與兄副將錦拒戰有功,受賞。萬歷初,再遷大同副總兵。十年冬,以都督僉事充寧夏總兵官。無何,徙鎮大同。時諸部納款久,撦力克襲封順義王,奉中國益虔。貴頻以安邊勞蒙賜賚。

十九年,為閱視少卿曾乾亨所劾,謫戍邊。明年,寧夏哱拜反。廷議貴健將知兵,且多畜家丁,乃起戍中為副將,總兵討賊。屢攻城不克。其五月,哱拜以套寇五百騎圍平虜堡,貴選精卒三百間道馳卻之。俄以總督魏學曾命撫著力兔、銀定、宰僧於橫城,啖以重利,皆不應,貴乃還攻城。寧夏總兵董一奎攻其南,固原總兵李昫攻其西,故總兵劉承嗣攻其北,牛秉忠攻其東,貴以游兵主策應。哱拜自北門出戰,將往勾套部,貴逐之入城,別遣將馬孔英、麻承詔等擊套寇援兵,俘斬百二十人。拜初與套部深相結,諸部長稱之為王。日坐著力兔帳中,主籌畫,至是不敢復出。俄朝命蕭如薰代董一奎,盡將諸道援兵,以貴為副。而李如松軍亦至,攻益急。賊奉黃金、繡蟒于卜失兔等,請急徇靈州,先據下馬關,沮餉道。卜失兔與莊禿賴果合兵犯定邊,而宰僧從花馬池西沙湃入。貴迎擊,挫宰僧於石溝。會董一元搗土昧巢,諸部長俱解去。賊復乞援於著力兔,擁眾大入。如松率勁騎迎戰張亮堡,自卯迄巳,敵銳甚。會貴及李如樟等兵至,夾擊之,寇乃卻。逐北至賀蘭山,獲首級百二十餘。持示賊,賊益洶懼。無何城破,賊盡平。貴以功增秩,子予。尋擢總兵官,鎮守延綏。

二十二年七月,卜失兔糾諸部深入定邊,營張春井。貴乘虛搗其帳於套中,斬首二百五十有奇。還自寧塞,復邀其零騎。會寇留內地久,轉掠至下馬關。寧夏總兵蕭如薰不能禦,總督葉夢熊急檄貴赴援。督副將蕭如蘭等連戰曬馬臺、薛家窪,斬首二百三十有奇,獲畜產萬五千。帝為告廟宣捷,進署都督同知,予世蔭。明年,卜失兔復入塞,掠八日而還。順義王撦力克約之納款,不從,復擬大入。貴勒兵萬五千人:遊擊閻逢時等出紅山為中軍,參將師以律等出高家堡、神木、孤山為左軍,參將孫朝梁等出定邊、安邊、平山為右軍,而自以大軍當一面。銜枚疾趨,踰塞六十里。寇莫知所防,大潰。俘斬四百有奇,獲馬駝牛羊千五百。再進秩,予廕。尋以病歸。

二十五年,日本封事敗,起貴備倭總兵官,赴朝鮮。已,加提督,盡統南北諸軍。貴馳至王京,倭已入慶州,據閑山島,圍南原。守將楊元遁,全州守將陳愚衷亦遁,倭乘勢逼王京。貴別遣副將解生守稷山,朝鮮亦令都體察使李元翼出忠清道遮賊鋒。生頗有斬獲功,參將彭友德亦破賊青山。倭將行長退屯井邑,清正還慶州。經略邢玠、經理楊鎬先後至,分兵三協:左李如梅,右李芳春、解生,中高策。貴與鎬督左右協兵專攻清正。策駐宜寧,東援兩協,西扼行長。諸軍至慶州,倭悉退屯蔚山,如梅誘敗之。清正退保島山,築三砦自固。遊擊茅國器率死士拔其砦,斬馘六百五十,諸軍遂進圍其城。城新築以石,堅甚,將士仰攻多死。圍十日,倭襲敗生兵。明年正月二日,行長來援,九將兵俱潰。賊張旗幟江上,鎬大懼,倉皇撤師,以捷奏。既而敗狀聞,帝罷鎬,責貴以功贖。與劉綎、陳璘、董一元分四路。貴居東,當清正,數戰有功。會平秀吉死,官軍益力攻。十一月,清正先遁,貴遂入島山、西浦,諸路共俘斬二千二百有奇。明年三月,旋師。進右都督,予世廕。

三十八年,命貴鎮遼東。泰寧炒花素桀驁,九子各將兵,他部宰賽、煖兔又助之。邊將畏戰,但以增歲賞為事,寇益無所忌。明年,臨邊要賞,將士出不意擊之,拔營遁,徙額力素居焉。其地忽天鳴地震,炒花驚懼,再徙渡老河,去邊幾四百里,其第三子色特哂之,南移可可毋林,伺隙入犯。貴伏兵敗之,追北至白雲山,斬馘三百四十有奇。色特憤,謀復仇。糾宰賽、以兒鄧,皆不應。乃東糾卜言顧、伯要兒,西糾哈剌漢乃蠻,合犯清河,皆潰。以兒鄧等懼,代炒花求款,邊境乃寧。明年,插漢虎墩兔以三萬騎入掠穆家堡。禦之,敗去。其夏,貴引病乞罷,詔乘傳歸。

貴果毅驍捷,善用兵,東西並著功伐。先後承特賜者七,錫世廕者六。及歿,予祭葬。稱一時良將焉。

兄錦,少從父行陣,有戰功。累官千總,協守大同右衛。千戶魏昂者,坐罪亡入沙漠,引寇至城下,挾取妻子,錦伏甲擒之。俺答圍城,數突圍,城卒完。尋以殺人,並父奪官下吏。當事以塞上方用兵,而錦父子兄弟並敢戰,曲法貸之。屢遷宣府遊擊將軍。以勤王功,進秩一等,遷大同參將。隆慶初,進本鎮副總兵,從趙岢出塞敗寇兵,與弟貴並有保境功。俺答納款,錦招塞外叛人歸者甚眾。萬曆五年,擢山西總兵官。尋改鎮宣府,卒。

錦子承勛,遼東副總兵,都督僉事,南京後府僉書。從子承恩,都督同知,宣府、延綏、大同總兵官。更歷諸鎮,以勇力聞。後起援遼東,屢退避,下獄當死。詔納馬八百匹免罪,其家遂破。承詔,寧夏參將。從平哱拜有功。後為蒼頭所弒。承訓,薊鎮副總兵。承宣,洮、岷副總兵。承宗,遼東副總兵。天啟初,戰死沙嶺。

麻氏多將才。人以方鐵嶺李氏,曰「東李西麻」。

贊曰:自俺答款宣、大,薊門設守固,而遼獨被兵。成梁遂擅戰功,至剖符受封,震耀一時,倘亦有天幸歟!麻貴宣力東西,勛閥可稱。兩家子弟,多歷要鎮,是以時論以李、麻並列。然列戟擁麾,世傳將種,而恇怯退避,隳其家聲。語曰「將門有將」,諸人得無愧乎!


\end{pinyinscope}