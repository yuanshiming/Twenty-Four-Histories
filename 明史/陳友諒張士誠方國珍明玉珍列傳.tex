\article{陳友諒、張士誠、方國珍、明玉珍列傳}


陳友諒,沔陽漁家子也。本謝氏,祖贅於陳,因從其姓。少讀書,略通文義。有術者相其先世墓地,曰「法當貴」,友諒心竊喜。嘗為縣小吏,非其好也。徐壽輝兵起,友諒往從之,依其將倪文俊為簿掾。

壽輝,羅田人,又名真一,業販布。元末盜起,袁州僧彭瑩玉以妖術與麻城鄒普勝聚眾為亂,用紅巾為號,奇壽輝狀貌,遂推為主。至正十一年九月陷蘄水及黃州路,敗元威順王寬徹不花。遂即蘄水為都,稱皇帝,國號天完,建元治平,以普勝為太師。未幾,陷饒、信。明年分兵四出,連陷湖廣、江西諸郡縣。遂破昱嶺關,陷杭州。別將趙普勝等陷太平諸路。勢大振。然無遠志,所得不能守。明年為元師所破,壽輝走免。已而復熾,遷都漢陽,為其丞相倪文俊所制。

十七年九月,文俊謀弒壽輝,不克,奔黃州。時友諒隸文俊麾下,數有功,為領兵元帥。遂乘釁殺文俊,并其兵,自稱宣慰使,尋稱平章政事。

明年,陷安慶,又破龍興、瑞州,分兵取邵武、吉安,而自以兵入撫州。已或,又破建昌、贛、汀、信、衢。

當是時,江以南惟友諒兵最強。太祖之取太平也,與為鄰。友諒陷元池州,太祖遣常遇春擊取之,由是數相攻擊。趙普勝者,故驍將,號「雙刀趙」。初與俞通海等屯巢湖,同歸太祖,叛去歸壽輝。至是為友諒守安慶,數引兵爭池州、太平,往來掠境上。太祖患之,啖普勝客,使潛入友諒軍間普勝。普勝不之覺,見友諒使者輒訴功,悻悻有德色。友諒銜之,疑其貳於己,以會師為名,自江州猝至。普勝以燒羊逆於雁漢。甫登舟,友諒即殺普勝,并其軍。乃以輕兵襲池州,為徐達等擊敗,師盡覆。

始友諒破龍興,壽輝欲徙都之,友諒不可。未幾,壽輝遽發漢陽,次江州。江州,友諒治所也,伏兵郭外,迎壽輝入,即閉城門,悉殺其所部。即江州為都,奉壽輝以居,而自稱漢王,置王府官屬。遂挾壽輝東下,攻太平。太平城堅不可拔,乃引巨舟薄城西南。士卒緣舟尾攀堞而登,遂克之。志益驕。進駐采石磯,遣部將陽白事壽輝前,戒壯士挾鐵撾擊碎其首。壽輝既死,以采石五通廟為行殿,即皇帝位,國號漢,改元大義,太師鄒普勝以下皆仍故官。會大風雨,群臣班沙岸稱賀,不能成禮。

友諒性雄猜,好以權術馭下。既僭號,盡有江西、湖廣之地,恃其兵強,欲東取應天。太祖患友諒與張士誠合,乃設計令其故人康茂才為書誘之,令速來。友諒果引舟師東下,至江東橋,呼茂才不應,始知為所紿。戰於龍灣,大敗。潮落舟膠,死者無算,亡戰艦數百,乘輕舸走。張德勝追敗之慈湖,焚其舟。馮國勝以五翼軍蹙之,友諒出皂旗軍迎戰,又大敗。遂棄太平,走江州。太祖兵乘勝取安慶,其將于光、歐普祥皆降。明年,友諒遣兵復陷安慶。太祖自將伐之,復安慶,長驅至江州。友諒戰敗,夜挈妻子奔武昌。其將吳宏以饒降,王溥以建昌降,胡廷瑞以龍興降。

友諒忿疆土日蹙,乃大治樓船數百艘,皆高數丈,飾以丹漆,每船三重,置走馬棚,上下人語聲不相聞,艫箱皆裹以鐵。載家屬百官,盡銳攻南昌,飛梯衝車,百道並進。太祖從子文正及鄧愈堅守,三月不能下,太祖自將救之。友諒聞太祖至,撤圍,東出鄱陽湖,遇於康郎山。友諒集巨艦,連鎖為陣,太祖兵不能仰攻,連戰三日,幾殆。已,東北風起,乃縱火焚友諒舟,其弟友仁等皆燒死。友仁號五王,眇一目,有勇略,既死,友諒氣沮。是戰也,太祖舟雖小,然輕駛,友諒軍俱艨艟巨艦,不利進退,以是敗。

太祖所乘舟檣白,友諒約軍士明日併力攻白檣舟。太祖知之,令舟檣盡白。翌日復戰,自辰至午,友諒軍大敗。友諒欲退保奚山,太祖已先扼湖口,邀其歸路。持數日,友諒謀於眾。右金吾將軍曰:「出湖難,宜焚舟登陸,直趨湖南圖再舉。」左金吾將軍曰:「此示弱也,彼以步騎躡我,進退失所據,大事去矣。」友諒不能決,既而曰:「右金吾言是也。」左金吾以言不用,舉所部來降。右金吾知之,亦降。友諒益困。太祖凡再移友諒書,其略曰:「吾欲與公約從,各安一方,以俟天命。公失計,肆毒於我。我輕師間出,奄有公龍興十一郡,猶不自悔禍,復構兵端。一困於洪都,再敗於康郎,骨肉將士重罹塗炭。公即倖生還,亦宜卻帝號,坐待真主,不則喪家滅姓,悔晚矣。」友諒得書忿恚,不報。久之乏食,突圍出湖口。諸將自上流邀擊之,大戰涇江口。漢軍且鬥且走,日暮猶不解。友諒從舟中引首出,有所指捴,驟中流矢,貫晴及顱死。軍大潰,太子善兒被執。太尉張定邊夜挾友諒次子理,載其屍遁還武昌。友諒豪侈,嘗造鏤金床甚工,宮中器物類是。既亡,江西行省以床進。太祖歎曰:「此與孟昶七寶溺器何異!」命有司毀之。友諒僭號凡四年。

子理既還武昌,嗣偽位,改元德壽。是冬,太祖親征武昌。明年二月再親征。其丞相張必先自岳州來援,次洪山。常遇春擊擒之,徇於城下。必先,驍將也,軍中號「潑張」,倚為重。及被擒,城中大懼,由是欲降者眾。太祖乃遣其故臣羅復仁入城招理。理遂降,入軍門,俯伏不敢視。太祖見理幼弱,掖之起,握其手曰:「吾不汝罪也。」府庫財物恣理取,旋應天,授爵歸德侯。

友諒之從徐壽輝也,其父普才止之。不聽。及貴,往迎之。普才曰:「汝違吾命,吾不知死所矣。」普才五子:長友富,次友直,又次友諒,又次友仁、友貴。友仁、友貴前死鄱陽。太祖平武昌,封普才承恩侯,友富歸仁伯,友直懷恩伯,贈友仁康山王,命所司立廟祀之,以友貴祔。理居京師,邑邑出怨望語。帝曰:「此童孺小過耳,恐細人蠱惑,不克全朕恩,宜處之遠方。」洪武五年,理及歸義侯明昇並徙高麗,遣元降臣樞密使延安答理護行。賜高麗王羅綺,俾善視之。亦徙普才等滁陽。

熊天瑞者,本荊州樂工,從徐壽輝抄略江、湘間。後受陳友諒命,攻陷臨江、吉安,又陷贛州。友諒俾以參知政事,守贛,兼統吉安、南安、南雄、韶州諸路。久之,陽言東下,署其幟曰「無敵」,自稱金紫光祿大夫、司徒、平章軍國重事。友諒不能制。陰圖取廣東,造戰艦於南雄,帥數萬眾趨廣州。元將何真以兵迎於胥江。會天大雷雨,震其艦檣折,天瑞懼而還。太祖兵克臨江,遣常遇春等攻贛,天瑞拒守五越月,至正二十五年正月,乃帥其養子元震肉袒詣軍門降。太祖宥之,授指揮使。明年從攻浙西,叛降於張士誠,教士誠飛礮擊外軍。城中木石俱盡,外軍多傷者。士誠滅,天瑞伏誅。

有周時中者,龍泉人,嘗為壽輝平章。後帥所部降,策天瑞必叛。後果如其言。時中累官吏部尚書,出為鎮江知府,歷福建鹽運副使。

元震本姓田氏,善戰有名。遇春之圍贛也,元震竊出覘兵,遇春亦引數騎出,猝與遇。元震不知為遇春也,過之。及遇春還,始覺,遂單騎前襲遇春。遇春遣從騎揮刀擊之,元震奮鐵撾且鬥且走。遇春曰:「壯男子也。」舍之。由是喜其才勇。既從天瑞降,薦以為指揮使。天瑞誅,復故姓云。

張士誠,小字九四,泰州白駒場亭人。有弟三人,並以操舟運鹽為業,緣私作姦利。頗輕財好施,得群輩心。常鬻鹽諸富家,富家多陵侮之,或負其直不酬。而弓手丘義尤窘辱士誠甚。士誠忿,即帥諸弟及壯士李伯昇等十八人殺義,并滅諸富家,縱火焚其居。入旁郡場,招少年起兵。鹽丁方苦重役,遂共推為主,陷泰州。高郵守李齊諭降之,復叛。殺行省參政趙璉,并陷興化,結砦德勝湖,有眾萬餘。元以萬戶告身招之。不受。紿殺李齊,襲據高郵,自稱誠王,僭號大周,建元天祐。是歲至正十三年也。

明年,元右丞相脫脫總大軍出討,數敗士誠,圍高郵,隳其外城。城且下,順帝信讒,解脫脫兵柄,削官爵,以他將代之。士誠乘間奮擊,元兵潰去,由是復振。踰年,淮東饑,士誠乃遣弟士德由通州渡江入常熟。

十六年二月陷平江,並陷湖州、松江及常州諸路。改平江為隆平府,士誠自高郵來都之。即承天寺為府第,踞坐大殿中,射三矢於棟以識。是歲,太祖亦下集慶,遣楊憲通好於士誠。其書曰:「昔隗囂稱雄於天水,今足下亦擅號於姑蘇,事勢相等,吾深為足下喜。睦鄰守境,古人所貴,竊甚慕焉。自今信使往來,毋惑讒言,以生邊釁。」士誠得書,留憲不報。已,遣舟師攻鎮江。徐達敗之於龍潭。太祖遣達及湯和攻常州。士誠兵來援,大敗,失張、湯二將,乃以書求和,請歲輸粟二十萬石,黃金五百兩,白金三百斤。太祖答書,責其歸楊憲,歲輸五十萬石。士誠復不報。

初,士誠既得平江,即以兵攻嘉興。元守將苗帥楊完者數敗其兵。乃遣士德間道破杭州。完者還救,復敗歸。明年,耿炳文取長興,徐達取常州,吳良等取江陰,士誠兵不得四出,勢漸蹙。亡何,徐達兵徇宜興,攻常熟。士德迎戰敗,為前鋒趙德勝所擒。士德,小字九六,善戰有謀,能得士心,浙西地皆其所略定。既被擒,士誠大沮。太祖欲留士德以招士誠。士德間道貽士誠書,俾降元。士誠遂決計請降。江浙右丞相達識帖睦邇為言於朝,授士誠太尉,官其將吏有差。士德在金陵竟不食死。士誠雖去偽號,擅甲兵土地如故。達識帖睦邇在杭與楊完者有隙,陰召士誠兵。士誠遣史文炳襲殺完者,遂有杭州。順帝遣使徵糧,賜之龍衣御酒。士誠自海道輸糧十一萬石於大都,歲以為常。既而益驕,令其下頌功德,邀王爵。不許。

二十三年九月,士誠復自立為吳王,尊其母曹氏為王太妃,置官屬,別治府第於城中,以士信為浙江行省左丞相,幽達識帖睦邇於嘉興。元徵糧不復與。參軍俞思齊者,字中孚,泰州人,諫士誠曰:「向為賊,可無貢;今為臣,不貢可乎?」士誠怒,抵案仆地,思齊即引疾去。當是時,士誠所據,南抵紹興,北踰徐州,達於濟寧之金溝,西距汝、潁、濠、泗,東薄海,二千餘里,帶甲數十萬。以士信及女夫潘元紹為腹心,左丞徐義、李伯昇、呂珍為爪牙,參軍黃敬夫、蔡彥文、葉德新主謀議,元學士陳基、右丞饒介典文章。又好招延賓客,所贈遺輿馬、居室、什器甚具。諸僑寓貧無籍者爭趨之。

士誠為人,外遲重寡言,似有器量,而實無遠圖。既據有吳中,吳承平久,戶口殷盛,士誠漸奢縱,怠於政事。士信、元紹尤好聚斂,金玉珍寶及古法書名畫,無不充牣。日夜歌舞自娛。將帥亦偃蹇不用命,每有攻戰,輒稱疾,邀官爵田宅然後起。甫至軍,所載婢妾樂器踵相接不絕,或大會遊談之士,樗蒲蹴踘,皆不以軍務為意。及喪師失地還,士誠概置不問。已,復用為將。上下嬉娛,以至於亡。

太祖與士誠接境。士誠數以兵攻常州、江陰、建德、長興、諸全,輒不利去。而太祖遣邵榮攻湖州,胡大海攻紹興,常遇春攻杭州,亦皆不能下。廖永安被執,謝再興叛降士誠,會太祖與陳友諒相持,未暇及也。友諒亦遣使約士誠夾攻太祖,而士誠欲守境觀變,許使者,卒不行。太祖既平武昌,師還,即命徐達等規取準東,克泰州、通州,圍高郵。士誠以舟師溯江來援,太祖自將擊走之。達等遂拔高郵,取淮安,悉定淮北地。於是移檄平江,數士誠八罪。徐達、常遇春帥兵自太湖趨湖州,吳人迎戰於毘山,又戰於七里橋,皆敗,遂圍湖州。士誠遣朱暹、五太子等以六萬眾來援,屯於舊館,築五砦自固。達、遇春築十壘以遮之,斷其糧道。士誠知事急,親督兵來戰,敗於皂林。其將徐志堅敗於東遷,潘元紹敗於烏鎮,升山水陸寨皆破,舊館援絕,五太子、朱暹、呂珍皆降。五太子者,士誠養子,短小精悍,能平地躍丈餘,又善沒水,珍、暹皆宿將善戰,至是降。達等以徇於湖州。守將李伯昇等以城降,嘉興、松江相繼降。潘原明亦以杭州降於李文忠。

二十六年十一月,大軍進攻平江,築長圍困之。士誠距守數月。太祖貽書招之曰:「古之豪傑,以畏天順民為賢,以全身保族為智,漢竇融、宋錢人叔是也。爾宜三思,勿自取夷滅,為天下笑。」士誠不報,數突圍決戰,不利。李伯昇知士誠困甚,遣所善客踰城說士誠曰:「初公所恃者,湖州、嘉興、杭州耳,今皆失矣。獨守此城,恐變從中起,公雖欲死,不可得也。莫若順天命,遣使金陵,稱公所以歸義救民之意,開城門,幅巾待命,當不失萬戶侯。且公之地,譬如博者,得人之物而復失之,於公何損?」士誠仰觀良久曰:「吾將思之。」乃謝客,竟不降。士誠故有勇勝軍號「十條龍」者,皆驍猛善鬥,每被銀鎧錦衣出入陣中,至是亦悉敗,溺萬里橋下死。最後丞相士信中礮死,城中洶洶無固志。二十七年九月,城破,士誠收餘眾戰於萬壽寺東街,眾散走。倉皇歸府第,拒戶自縊。故部將趙世雄解之。大將軍達數遣李伯升、潘元紹等諭意,士誠瞑目不答。舁出葑門,入舟,不復食。至金陵,竟自縊死,年四十七。命具棺葬之。

方士誠之被圍也,語其妻劉曰:「吾敗且死矣,若曹何為?」劉答曰:「君無憂,妾必不負君。」積薪齊雲樓下。城破,驅群妾登樓,令養子辰保縱火焚之,亦自縊。有二幼子匿民間,不知所終。先是,黃敬夫等三人用事,吳人知士誠必敗,有「黃菜葉」十七字之謠,其後卒驗云。

莫天祐者,元末聚眾保無錫州,士誠招之。不從。以兵攻之,亦不克。士誠既受元官,天祐乃降。士誠累表為同僉樞密院事。及平江既圍,他城皆下,惟天祐堅守。士誠破,胡廷瑞急攻之,乃降。太祖以其多傷我兵,誅之。

李伯昇仕士誠至司徒,既降,命仍故官,進中書平章同知詹事府事。嘗將兵討平湖廣慈利蠻,又為征南右副將軍,同吳良討靖州蠻。後坐胡黨死。潘元明以平章守杭州降,仍為行省平章,與伯昇俱歲食祿七百五十石,不治事。雲南平,以元明署布政司事,卒官。

士誠自起至亡,凡十四年。

方國珍,黃巖人。長身黑面,體白如瓠,力逐奔馬。世以販鹽浮海為業。元至正八年,有蔡亂頭者,行剽海上,有司發兵捕之。國珍怨家告其通寇。國珍殺怨家,遂與兄國璋、弟國瑛、國氏亡入海,聚眾數千人,劫運艘,梗海道。行省參政朵兒只班討之,兵敗,為所執,脅使請於朝,授定海尉。尋叛,寇溫州。元以孛羅帖木兒為行省左丞,督兵往討,復敗,被執。乃遣大司農達識帖睦邇招之降。已而汝、潁兵起,元募舟師守江。國珍疑懼,復叛。誘殺台州路達魯花赤泰不華,亡入海。使人潛至京師,賂諸權貴,仍許降,授徽州路治中。國珍不聽命,陷台州,焚蘇之太倉。元復以海道漕運萬戶招之,乃受官。尋進行省參政,俾以兵攻張士誠。士誠遣將禦之崑山。國珍七戰七捷。會士誠亦降,乃罷兵。

先是,天下承平,國珍兄弟始倡亂海上,有司憚於用兵,一意招撫。惟都事劉基以國珍首逆,數降數叛,不可赦。朝議不聽。國珍既授官,據有慶元、溫、台之地,益強不可制。國珍之初作亂也,元出空名宣敕數十道募人擊賊。海濱壯士多應募立功。所司邀重賄,不輒與,有一家數人死事卒不得官者。而國珍之徒,一再招諭,皆至大官。由是民慕為盜,從國珍者益眾。元既失江、淮,資國珍舟以通海運,重以官爵羈縻之,而無以難也。有張子善者,好縱橫術,說國珍以師溯江窺江東,北略青、徐、遼海。國珍曰:「吾始志不及此。」謝之去。

太祖已取婺州,使主簿蔡元剛使慶元。國珍謀於其下曰:「江左號令嚴明,恐不能與抗。況為我敵者,西有吳,南有閩。莫若姑示順從,藉為聲援以觀變。」眾以為然。於是遣使奉書進黃金五十斤,白金百斤,文綺百匹。太祖復遣鎮撫孫養浩報之。國珍請以溫、台、慶元三郡獻,且遣次子關為質。太祖卻其質,厚賜而遣之;復使博士夏煜往,拜國珍福建行省平章事,弟國瑛參知政事,國氏樞密分院僉事。國珍名獻三郡,實陰持兩端。煜既至,乃詐稱疾,自言老不任職,惟受平章印誥而已。太祖察其情,以書諭曰:「吾始以汝豪傑識時務,故命汝專制一方。汝顧中懷叵測,欲覘我虛實則遣侍子,欲卻我官爵則稱老病。夫智者轉敗為功,賢者因禍成福,汝審圖之。」是時國珍歲歲治海舟,為元水曹張士誠粟十餘萬石於京師,元累進國珍官至江浙行省左丞相衢國公,分省慶元。國珍受之如故,特以甘言謝太祖,絕無內附意。及得所諭書,竟不省。太祖復以書諭曰:「福基於至誠,禍生於反覆,隗囂、公孫述故轍可鑒。大軍一出,不可虛辭解也。」國珍詐窮,復陽為惶懼謝罪,以金寶飾鞍馬獻。太祖復卻之。

已而苗帥蔣英等叛,殺胡大海,持首奔國珍,國珍不受,自臺州奔福建。國璋守台,邀擊之,為所敗,被殺,太祖遣使吊祭。踰年,溫人周宗道以平陽來降。國珍從子明善守溫以兵爭。參軍胡深擊敗之,遂下瑞安,進兵溫州。國珍恐,請歲輸白金三萬兩給軍,俟杭州下,即納土來歸。太祖詔深班師。

吳元年克杭州。國珍據境自如,遣間諜假貢獻名覘勝負,又數通好於擴廓帖木兒及陳友定,圖為掎角。太祖聞之怒,貽書數其十二罪,復責軍糧二十萬石。國珍集眾議,郎中張本仁、左丞劉庸等皆言不可從。有丘楠者,獨爭曰:「彼所言均非公福也。惟智可以決事,惟信可以守國,惟直可以用兵。公經營浙東十餘年矣,遷延猶豫,計不早定,不可謂智。既許之降,抑又倍焉,不可謂信。彼之徵師,則有詞矣,我實負彼,不可謂直。幸而扶服請命,庶幾可視錢人叔乎?」國珍不聽,惟日夜運珍寶,治舟楫,為航海計。

九月,太祖已破平江,命參政朱亮祖攻台州,國瑛迎戰敗走。進克溫州。征南將軍湯和以大軍長驅抵慶元。國珍帥所部遁入海。追敗之盤嶼,其部將相次降。和數令人示以順逆,國珍乃遣子關奉表乞降曰:「臣聞天無所不覆,地無所不載。王者體天法地,於人無所不容。臣荷主上覆載之德舊矣,不敢自絕於天地,故一陳愚衷。臣本庸才,遭時多故,起身海島,非有父兄相藉之力,又非有帝制自為之心。方主上霆擊電掣,至於婺州,臣愚即遣子入侍,固已知主上有今日矣,將以依日月之末光,望雨露之餘潤。而主上推誠布公,俾守鄉郡,如故吳越事。臣遵奉條約,不敢妄生節目。子姓不戒,潛構釁端,猥勞問罪之師,私心戰兢,用是俾守者出迎。然而未免浮海,何也?孝子之於親,小杖則受,大杖則走,臣之情事適與此類。即欲面縛待罪闕廷,復恐嬰斧鉞之誅,使天下後世不知臣得罪之深,將謂主上不能容臣,豈不累天地大德哉。」蓋幕下士詹鼎詞也。

太祖覽而憐之,賜書曰:「汝違吾諭,不即斂手歸命,次且海外,負恩實多。今者窮蹙無聊,情詞哀懇,吾當以汝此誠為誠,不以前過為過,汝勿自疑。」遂促國珍入朝,面讓之曰:「若來得毋晚乎!」國珍頓首謝。授廣西行省左丞,食祿不之官。數歲,卒於京師。

子禮,官廣洋衛指揮僉事;關,虎賁衛千戶所鎮撫。關弟行,字明敏,善詩,承旨宋濂嘗稱之。

劉仁本,字德元,國珍同縣人。元末進士乙科,歷官浙江行省郎中,與張本仁俱入國珍幕。數從名士趙人叔、謝理、朱右等賦詩,有稱於時。國珍海運輸元,實仁本司其事。朱亮祖之下溫州也,獲仁本。太祖數其罪,鞭背潰爛死。餘官屬從國珍降者皆徙滁州,獨赦丘楠,以為韶州知府。

詹鼎者,寧海人,有才學。為國珍府都事,判上虞,有治聲。既至京,未見用,草封事萬言,候駕出獻之。帝為立馬受讀,命丞相官鼎。楊憲忌其才,沮之。憲敗,除留守經歷,遷刑部郎中,坐累死。明玉珍,隨州人。身長八尺餘,目重瞳子。徐壽輝起,玉珍與里中父老團結千餘人,屯青山。及壽輝稱帝,使人招玉珍曰:「來則共富貴,不來舉兵屠之。」玉珍引眾降,以元帥守沔陽。與元將哈麻禿戰湖中,飛矢中右目,遂眇。久之,玉珍帥斗船五十艘掠糧川、峽間,將引還。時元右丞完者都募兵重慶,義兵元帥楊漢應募至,欲殺之而并其軍,不克。漢走出峽,遇玉珍為言:「重慶無重兵,完者都與右丞哈麻禿不相能,若回船出不意襲之,可取而有也。」玉珍意未決,部將戴壽曰:「機不可失也。可分船為二,半貯糧歸沔陽,半因漢兵攻重慶,不濟則掠財物而還。」玉珍從其策,襲重慶,走完者都,執哈麻禿獻壽輝。壽輝授玉珍隴蜀行省右丞。至正十七年也。

已而完者都自果州來,會平章朗革歹、參政趙資,謀復重慶,屯嘉定之大佛寺,玉珍遣萬勝禦之。勝,黃陂人,有智勇,玉珍寵愛之,使從己姓,眾呼為明二,後乃復姓名。勝攻嘉定,半年不下。玉珍帥眾圍之,遣勝以輕兵襲陷成都,虜朗革歹及資妻子。朗革歹妻自沉於江。以資妻子徇嘉定,招資降。資引弓射殺妻。俄城破,執資及完者都、朗革歹歸於重慶,館諸治平寺,欲使為己用。三人者執不可,乃斬於市,以禮葬之,蜀人謂之「三忠」。於是諸郡縣相次來附。

二十年,陳友諒弒徐壽輝自立。玉珍曰:「與友諒俱臣徐氏,顧悖逆如此。」命以兵塞瞿塘,絕不與通。立壽輝廟於城南隅,歲時致祀。自立為隴蜀王,以劉楨為參謀。

楨,字維周,瀘州人。元進士。嘗為大名路經歷,棄官家居。玉珍之攻重慶也,道瀘,部將劉澤民薦之。玉珍往見,與語大悅,即日延至舟中,尊禮備至。次年,楨屏人說曰:「西蜀形勝地,大王撫而有之,休養傷殘,用賢治兵,可以立不世業。不於此時稱大號以係人心,一旦將士思鄉土,瓦解星散,大王孰與建國乎。」玉珍善之,乃謀於眾,以二十二年春僭即皇帝位於重慶,國號夏,建元天統。立妻彭氏為皇后,子昇為太子。傚周制,設六卿,以劉楨為宗伯。分蜀地為八道,更置府州縣官名。蜀兵視諸國為弱,勝兵不滿萬人。玉珍素無遠略,然性節儉,頗好學,折節下士。既即位,設國子監,教公卿子弟,設提舉司教授,建社稷宗廟,求雅樂,開進士科,定賦稅,以十分取一。蜀人悉便安之。皆劉楨為之謀也。

明年,遣萬勝由界首,鄒興由建昌,又指揮李某者由八番,分道攻雲南。兩路皆不至,惟勝兵深入,元梁王走營金馬山。踰年,王挾大理兵擊勝,勝以孤軍無繼引還。復遣興取巴州。久之,復更六卿為中書省樞密院,改冢宰戴壽、司馬萬勝為左、右丞相,司寇向大亨、司空張文炳知樞密院事,司徒鄒興鎮成都,吳友仁鎮保寧,司寇莫仁壽鎮夔關,皆平章事。

是歲,遣勝取興元,使參政江儼通好於太祖。太祖遣都事孫養浩報聘,遺玉珍書曰:「足下處西蜀,予處江左,蓋與漢季孫、劉相類。近者王保保以鐵騎勁兵,虎踞中原,其志殆不在曹操下,使有謀臣如攸、彧,猛將如遼、合阜,予兩人能高枕無憂乎。予與足下實脣齒邦,願以孫劉相吞噬為鑒。」自後信使往返不絕。

二十六年春,玉珍病革,召壽等諭曰:「西蜀險固,若協力同心,左右嗣子,則可以自守。不然,後事非所知也。」遂卒。凡立五年,年三十六。

子升嗣,改元開熙,葬玉珍於江水之北,號永昌陵,廟號太祖。尊母彭氏為皇太后,同聽政。昇甫十歲,諸大臣皆粗暴,不肯相下。而萬勝與張文炳有隙,勝密遣人殺之。文炳所善玉珍養子明昭,復矯彭氏旨縊殺勝。勝於明氏功最多,其死,蜀人多憐之。吳友仁自保寧移檄,以清君側為名。昇命戴壽討之。友仁遺壽書謂:「不誅昭,則國必不安,眾必不服。昭朝誅,吾當夕至。」壽乃奏誅昭,友仁入朝謝罪。於是諸大臣用事,而友仁尤專恣,國柄旁落,遂益不振。萬勝既死,劉楨為右丞相,後三年卒。是歲,升遣使告哀於太祖,已,又遣使入聘。太祖亦遣侍御史蔡哲報之。

洪武元年,太祖克元都,升奉書稱賀。明年,太祖遣使求大木。昇遂并獻方物。帝答以璽書。其冬,遣平章楊璟諭昇歸命。升不從。璟復遺昇書曰:

古之為國者,同力度德,同德度義,,故能身家兩全,流譽無窮,反是者輒敗。足下幼沖,席先人業,據有巴、蜀,不咨至計,而聽群下之議,以瞿塘、劍閣之險,一夫負戈,萬人無如之何。此皆不達時變以誤足下之言也。昔據蜀最盛者,莫如漢昭烈。且以諸葛武侯佐之,綜核官守,訓練士卒,財用不足,皆取之南詔。然猶朝不謀夕,僅能自保。今足下疆場,南不過播州,北不過漢中,以此準彼,相去萬萬,而欲藉一隅之地,延命頃刻,可謂智乎?我主上仁聖威武,神明響應,順附者無不加恩,負固者然後致討。以足下先人通好之故,不忍加師,數使使諭意。又以足下年幼,未歷事變,恐惑於狂瞽,失遠大計,故復遣璟面諭禍福。深仁厚德,所以待明氏者不淺,足下可不深念乎?且向者如陳、張之屬,竊據吳、楚,造舟塞江河,積糧過山岳,強將勁兵,自謂無敵。然鄱陽一戰,友諒授首,旋師東討,張氏面縛。此非人力,實天命也。足下視此何如?友諒子竄歸江夏,王師致伐,勢窮銜璧。主上宥其罪愆,剖符錫爵,恩榮之盛,天下所知。足下無彼之過,而能翻然覺悟,自求多福,則必享茅土之封,保先人之祀,世世不絕,豈不賢智矣哉?若必欲崛強一隅,假息頃刻,魚遊沸鼎,燕巢危幕,禍害將至,恬不自知。璟恐天兵一臨,凡今為足下謀者,他日或各自為身計,以取富貴。當此之時,老母弱子,將安所歸?禍福利害,尞然可睹,在足下審之而已。

升終不聽。

又明年,興元守將以城降。吳友仁數往攻之,不克。是歲,太祖遣使假道征雲南,昇不奉詔。四年正月命征西將軍湯和帥副將軍廖永忠等以舟師由瞿塘趨重慶,前將軍傅友德帥副將軍顧時等以步騎由秦、隴趨成都,伐蜀。初,壽言於升曰:「以王保保、李思齊之強,猶莫能與明抗,況吾蜀乎!一旦有警,計安出?」友仁曰:「不然,吾蜀襟山帶江,非中原比,莫若外交好而內修備。」升以為然,遣莫仁壽以鐵索橫斷瞿塘峽口。至是又遣壽、友仁、鄒興等益兵為助。北倚羊角山,南倚南城砦,鑿兩岸石壁,引鐵索為飛橋,用木板置礮以拒敵。和軍至,不能進。傅友德覘階、文無備,進破之,又破綿州。壽乃留興等守瞿塘,而自與友仁還,會向大亨之師以援漢州。數戰皆大敗,壽、大亨走成都,友仁走保寧。時永忠亦破瞿塘關。飛橋鐵索皆燒斷,興中矢死,夏兵皆潰。遂下夔州,師次銅羅峽。昇大懼,右丞劉仁勸奔成都。昇母彭泣曰:「成都可到,亦僅延旦夕命。大軍所過,勢如破竹,不如早降以活民命。」於是遣使齎表乞降。昇面縛銜璧輿櫬,與母彭及官屬降於軍門。和受璧,永忠解縛,承旨撫慰,下令諸將不得有所侵擾。而壽、大亨亦以成都降於友德。升等悉送京師,禮臣奏言:「皇帝御奉天殿,明昇等俯伏待罪午門外,有司宣制赦,如孟昶降宋故事。」帝曰:「昇幼弱,事由臣下,與孟昶異,宜免其伏地上表待罪之儀。」是日授昇爵歸義侯,賜第京師。

冬十月,和等悉定川、蜀諸郡縣,執友仁於保寧,遂班師。壽、大亨、仁壽皆鑿舟自沉死。丁世貞者,文州守將也,友德攻文州,據險力戰,汪興祖死焉。文州破,遁去。已復以兵破文州,殺朱顯忠,友德擊走之。夏亡,復集餘眾圍秦州五十日。兵敗,夜宿梓潼廟,為其下所殺。友仁至京師,帝以其寇漢中,首造兵端,令明氏失國,僇於市。戍他將校於徐州。明年徙昇於高麗。

贊曰:友諒、士誠起刀筆負販,因亂僭竊,恃其富強,而卒皆敗於其所恃。迹其始終成敗之故,太祖料之審矣。國珍首亂,反覆無信,然竟獲良死,玉珍乘勢,割據一隅,僭號二世,皆不可謂非幸也。國珍又名谷珍,蓋降後避明諱云。


