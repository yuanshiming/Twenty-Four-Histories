\article{列傳第五十}

\begin{pinyinscope}
○尹昌隆耿通陳諤戴綸林長懋陳祚郭循劉球子鉞釪陳鑑何觀鐘同孟楊集章綸子玄應廖莊倪敬盛灊等楊瑄子源盛顒等

尹昌隆,字彥謙,泰和人。洪武中進士及第。授修撰,改監察御史。

惠帝初即位,視朝晏。昌隆疏諫曰:「高皇帝雞鳴而起,昧爽而朝,未日出而臨百官,故能庶績咸熙,天下乂安。陛下嗣守大業,宜追繩祖武,兢兢業業,憂勤萬幾。今乃即於晏安,日上數刻,猶未臨朝。群臣宿衛,疲於伺候,曠職廢業,上下懈弛。播之天下,傳之四裔,非社稷福也。」帝曰:「昌隆言切直,禮部其宣示天下,使知朕過。」未幾,以地震上言,謫福寧知縣。燕兵既逼,昌隆以北來奏章動引周公輔成王為詞,勸帝罷兵,許王入朝。設有蹉跌,便舉位讓之。若沈吟不斷,進退失據,將求為丹徒布衣且不可得。成祖入京師,昌隆名在奸臣中。以前奏貸死,命傅世子於北平。

永樂二年冊世子為皇太子,擢昌隆左春坊左中允。隨事匡諫,太子甚重之。解縉之黜,同日改昌隆禮部主事。尚書呂震方用事,性刻忮。當其獨處精思,以手指刮眉尾,則必有密謀深計。官屬相戒,無敢白事者。昌隆前白事,震怒不應;移時又白之,震愈怒,拂衣起。昌隆退白太子,取令旨行之。震大怒,奏昌隆假托宮僚,陰欲樹結,潛蓄無君心。逮下獄。尋遇赦復官。父憂起復。謁震,震溫言接之。入理前奏,復下錦衣衛獄,籍其家。帝凡巡幸,下詔獄者率輿以從,謂之隨駕重囚,昌隆與焉。

後數年,谷王謀反事發。以王前奏昌隆為長史,坐以同謀,詔公卿雜問。昌隆辯不已,震折之。獄具,置極刑死,夷其族。後震病且死,號呼「尹相」,言見昌隆守欲殺之云。

耿通,齊東人。洪武中舉於鄉。授襄陽教授。永樂初,擢刑科給事中,歷左右給事。剛直敢言。嘗劾都御史陳瑛、御史袁綱、覃珩朋比為蒙蔽,構陷無辜,綱、珩已下獄,瑛長官,不宜獨宥。又言:驍騎諸衛倉壞,工部侍郎陳壽不預修,糧至無所受,多損耗病民;工部尚書宋禮不恤下,匠役滿,不即遣歸,多至失所。瑛等皆被鐫責。當是時,給事中敢言者,通與陳諤。舉朝憚其風采。久之,擢大理寺右丞。

帝北巡,太子監國。漢王高煦謀奪嫡,陰結帝左右為讒間,宮僚多得罪者。監國所行事,率多更置。通從容諫帝:「太子事無大過誤,可無更也。」數言之,帝不悅。十年秋,有言通受請托故出人罪者。帝震怒,命都察院會文武大臣鞫之午門,曰:「必殺通無赦。」群臣如旨,當通罪斬。帝曰:「失出,細故耳,通為東宮關說,壞祖法,離間我父子,不可恕,其置之極刑。」廷臣不敢爭,竟論姦黨,磔死。

陳諤,字克忠,番愚人。永樂中,以鄉舉入太學,授刑科給事中。遇事剛果,彈劾無所避。每奏事,大聲如鐘。帝令餓之數日,奏對如故。曰:「是天性也。」每見,呼為「大聲秀才」。嘗言事忤旨,命坎瘞奉天門,露其首。七日不死,赦出還職。已,復忤旨,罰修象房。貧不能雇役,躬自操作。適駕至,問為誰。諤匍匐前,具道所以。帝憐之,命復官。

歷任順天府尹,政尚嚴鷙。執政忌之,出為湖廣按察使。改山西,坐事落職。仁宗即位,遇赦當還故官。帝以諤前在湖廣頗摭楚王細故,謫海鹽知縣。遷荊王長史,為王府所厭苦。宣德三年遷鎮江同知。致仕歸,卒。

戴綸,高密人。永樂中,自昌邑訓導擢禮科給事中,與編修林長懋俱侍皇太孫說書。歷中允、諭德。仁宗即位,太孫為太子,遷洗馬,仍侍講讀。始成祖命太孫習武事,太孫亦雅好之,時出騎射。綸與長懋以太孫春秋方富,不宜荒學問而事游畋,時時進諫。綸又具疏為帝言之。他日,太孫侍,帝問:「宮臣相得者誰也?」太孫以綸對。因出綸奏付之,太孫由此怨綸。

長懋者,莆田人。以鄉薦歷青州教授,擢編修。仁宗初,進中允。為人剛嚴,累進直言,與綸善。

宣宗即位,加恩宮僚,擢綸兵部侍郎。頃之,復以諫獵忤旨,命參贊交阯軍務。而長懋自南京來,後至,亦出為鬱林知州。無何,坐怨望,並逮至京,下錦衣衛獄。帝臨鞫之,綸抗辯,觸帝怒,立箠死,籍其家。諸父河南知府賢、太僕寺卿希文皆被繫。

而長懋在獄十年,英宗立,乃得釋。復其官,還守鬱林,有惠政。其卒也,州人立廟祀之。

陳祚,字永錫,吳人。永樂中進士。擢河南參議。十五年與布政使周文褒、王文振合疏,言建都北京非便,並謫均州太和山佃戶。躬耕力作,處之晏然。仁宗立,詔選用遷謫諸臣,祚在選中。會帝崩,不果用。

宣德二年命憲臣即均州群試之,祚策第一。試吏部,復第一。遂擢御史,巡按福建。方面大吏多被彈擊,禁止和買,閩人德之。還奏白塔河上通邵伯湖,下注大江,蘇、松舟楫,多從往來,淺狹湮塞,請開濬。從之,轉漕果便。尋出按江西。

時天下承平,帝頗事遊獵玩好。祚馳疏勸勤聖學。其略曰:「帝王之學先明理,明理在讀書。陛下雖有聖德,而經筵未甚興舉,講學未有程度,聖賢精微,古今治亂,豈能周知洞晰?真德秀《大學衍義》一書,聖賢格言,無不畢載。願於聽政之暇,命儒臣講說,非有大故,無得間斷。使知古今若何而治,政事若何而得。必能開廣聰明,增光德業。而邪佞之以奇巧蕩聖心者自見疏遠,天下人民受福無窮矣。」帝見疏大怒曰:「豎儒謂朕未讀《大學》耶!薄朕至此,不可不誅。」學士陳循頓首曰:「俗士處遠,不知上無書不讀也。」帝意稍解。下祚獄,逮其家人十餘口,隔別禁繫者五年,其父竟瘐死。其時,刑部主事郭循諫拓西內皇城修離宮,逮入面詰之。循抗辯不屈,亦下獄。英宗立,祚與循皆得釋復官。

祚再按湖廣。以奏遼王貴烚罪有所隱,與巡撫侍郎吳政逮至京,下獄。尋赦出。時王振用事,法務嚴峻,祚上言:「乃者法司論獄,多違定律。如侍郎吳璽誤舉主事吳軏,宜坐貢舉非其人律,乃坐以奏事有規避律斬。及軏自經死,獄官獄卒罪應遞減,乃援不應為重罪,概杖之。一事如此,餘可推矣。天時不順,災沴數見,未必非此。」帝是之,以其章示法司。尋改南京,遷福建按察使僉事。有威惠,神祠不載祀典者悉撤去。久之,以疾歸,卒。

祚天資嚴毅,雖子弟罕接其言笑,獨重里人邢量。量博學士,隱於卜,敝屋數椽,或竟日不舉火。祚數挾冊就質疑,往往至暮。

郭循,字循初,廬陵人。居官有才譽。既復職,進郎中,以尚書魏源薦,擢廣東參政,有剿寇功。景泰初卒。

劉球,字廷振,安福人。永樂十九年進士。家居讀書十年,從學者甚眾。授禮部主事。胡濙薦侍經筵,與修《宣宗實錄》,改翰林侍講。從弟玭知莆田,遺一夏布。球封還,貽書戒之。正統六年,帝以王振言,大舉征麓川。球上疏曰:

帝王之馭四裔,必宥其小而防其大。所以適緩急之宜,為天下久安計也。周伐崇不克,退修德教以待其降。至於玁狁,則命南仲城朔方以備之。漢征南越不利,即罷兵賜書通好。至於匈奴,雖已和親,猶募民徙居塞下,入粟實邊,復命魏尚守雲中拒之。

今麓川殘寇思任發素本羈屬,以邊將失馭,致勤大兵。雖渠魁未殲,亦多戮群醜,為誅為舍,無繫輕重。璽書原其罪釁,使得自新,甚盛德也。邊將不達聖意,復議大舉。欲屯十二萬眾於雲南,以趣其降,不降則攻之。不慮王師不可輕出,蠻性不可驟馴,地險不可用眾,客兵不可久淹。況南方水旱相仍,軍民交困,若復動眾,紛擾為憂。臣竊謂宜緩天誅,如周、漢之於崇、越也。

至於瓦剌,終為邊患。及其未即騷動,正宜以時防禦。乃欲移甘肅守將以事南征,卒然有警,何以為禦?臣竊以為宜慎防遏,如周、漢之於玁狁、匈奴也。

伏望陛下罷大舉之議。推選智謀將帥,輔以才識大臣,量調官軍,分屯金齒諸要害。結木邦諸蠻以為援,乘間進攻,因便撫諭,寇自可服。至於西北障塞,當敕邊臣巡視。浚築溝垣,增繕城堡,勤訓練,嚴守望,以防不虞,有備無患之道也。

章下兵部。謂南征已有成命,不用球言。

八年五月雷震奉天殿。球應詔上言所宜先者十事。其略曰:

古聖王不作無益,故心正而天不違之。臣願皇上勤御經筵,數進儒臣,講求至道。務使學問功至,理欲判然,則聖心正而天心自順。夫政由已出,則權不下移。太祖、太宗日視三朝,時召大臣於便殿裁決庶政,權歸總於上。皇上臨御九年,事體日熟。願守二聖成規,復親決故事,使權歸於一。

古之擇大臣者,必詢諸左右、大夫、國人。及其有犯,雖至大辟亦不加刑,第賜之死。今用大臣未嘗皆出公論。及有小失,輒桎梏箠楚之;然未幾時,又復其職。甚非所以待大臣也。自今擇任大臣,宜允愜眾論。小犯則置之。果不可容,下法司定罪,使自為計。勿輒繫,庶不乖共天職之意。

今之太常,即古之秩宗,必得清慎習禮之臣,然後可交神明。今卿貳皆缺,宜選擇儒臣,使領其職。

古者省方巡狩,所以察吏得失,問民疾苦。兩漢、唐、宋盛時,數遣使巡行郡縣,洪、永間亦嘗行之。今久不舉,故吏多貪虐,民不聊生,而軍衛尤甚。宜擇公明廉幹之臣,分行天下。

古人君不親刑獄,必付理官,蓋恐徇喜怒而有所輕重也。邇法司所上獄,多奉敕增減輕重,法司不能執奏。及訊他囚,又觀望以為輕重,民用多冤。宜使各舉其職。至運磚輸米諸例,均非古法,尤宜罷之。

《春秋》營築悉書,戒勞民也。京師興作五六年矣,曰「不煩民而役軍」,軍獨非國家赤子乎?況營作多完,宜罷工以蘇其力。

各處水旱,有司既不振救,請減租稅,或亦徒事虛文。宜令戶部以時振濟,量加減免,使不致失業。

麓川連年用兵,死者十七八,軍貲爵賞不可勝計。今又遣蔣貴遠征緬甸,責獻思任發。果擒以歸,不過梟諸通衢而已。緬將挾以為功,必求與木邦共分其地。不與則致怒,與之則兩蠻坐大,是減一麓川生二麓川也。設有蹉跎,兵事無已。臣見皇上每錄重囚,多宥令從軍,仁心若此。今欲生得一失地之竄寇,而驅數萬無罪之眾以就死地,豈不有乖於好生之仁哉?況思機發已嘗遣人來貢,非無悔過乞免之意。若敕緬斬任發首來獻,仍敕思機發盡削四境之地,分於各寨新附之蠻,則一方可寧矣。

迤北貢使日增,包藏禍心,誠為難測。宜分遣給事、御史閱視京邊官軍,及時訓練,勿使借工各廠,服役私家。公武舉之選以求良將,定召募之法以來武勇。廣屯田,公鹽法,以厚儲蓄。庶武備無缺,而外患有防。

疏入,下廷議。言球所奏,惟擇太常官宜從,令吏部推舉。修撰董璘遂乞改官太常,奉享祀事。

初,球言麓川事,振固已銜之。欽天監正彭德清者,球鄉人也,素為振腹心。凡天文有變,皆匿不奏,倚振勢為姦,公卿多趨謁。球絕不與通。德清恨之,遂摘疏中攬權語,謂振曰:「此指公耳。」振益大怒。會璘疏上,振遂指球同謀,並逮下詔獄,屬指揮馬順殺球。順深夜攜一小校持刀至球所。球方臥,起立,大呼太祖、太宗。頸斷,體猶植。遂支解之,瘞獄戶下。璘從旁竊血裙遺球家。後其子鉞求得一臂,裹裙以殮。順有子病久,忽起捽順髮,拳且蹴之曰:「老賊,令爾他日禍逾我!我,劉球也。」順驚悸。俄而子死,小校亦死。璘,字德文,高郵人。有孝行。獄解,遂歸,不復出。

球死數年,瓦剌果入寇。英宗北狩,振被殺。朝士立擊順,斃之。而德清自土木遁還,下獄論斬,尋瘐死。詔戮其屍。景帝憐球忠,贈翰林學士,謚忠愍,立祠於鄉。

球二子,長鉞、次釪。皆篤學,躬耕養母。球既得恤,兄弟乃出應舉,先後成進士。鉞,廣東參政;釪,雲南按察使。

陳鑑,字貞明,高安人。宣德二年進士。授行人。正統中,擢御史。

出按順天。言京師風俗澆漓,其故有五:一,事佛過甚;二,營喪破家;三,服食靡麗;四,優倡為蠹;五,博塞成風。章下禮部,格不行。

改按貴州。時麓川酋思任發子思機發遁孟養,屢上書求宥罪通貢。不許。復大舉遠征,兵連不解。雲、貴軍民疲敝。苗乘機煽動,閩、浙間盜賊大起。舉朝皆知其不可,懲劉球禍,無敢諫者。十四年正月,鑑抗疏言賊酋遠遁,不為邊患,宜專責雲南守臣相機剿滅,無遠勞禁旅。王振怒,欲困之,改鑑雲南參議,使赴騰衝招賊。已,復摭鑑為巡按時嘗請改四川播州宣慰司隸貴州,為鑒罪,令兵部劾之,論死繫獄。景帝嗣位,乃得赦。尋授河南參議。致仕歸,卒。

自正統中,劉球以忤王振冤死,鑑繼下獄,中外莫敢言事者數年。至景帝時,言路始開,爭發憤上書。有何觀者,復以言得罪去。

觀以善書為中書舍人。景泰二年劾尚書王直輩正統時阿附權奸,不宜在左右。中貴見「權奸」語,以為侵已,激帝怒,下科道參議。吏科毛玉主奏稿,力詆觀,林聰、葉盛持之,乃刪削奏上。會御史疏亦上,中有「觀考滿不遷,私憾吏部」語。帝怒,下觀詔獄,杖之,謫九溪衛經歷。

鐘同,字世京,吉安永豐人。父復,宣德中進士及第。歷官修撰,與劉球善。球上封事,約與俱,復妻勸止之。球詣復邸,邀偕行。復已他往,妻從屏間詈曰:「汝自上疏,何累他人為!」球出歎曰:「彼乃謀及婦人。」遂獨上奏,竟死。居無何,復亦病死。妻深悔之,每哭輒曰:「早知爾,曷若與劉君偕死。」同幼聞母言,即感奮,思成父志。嘗入吉安忠節祠,見所祀歐陽修、楊邦乂諸人,歎曰:「死不入此,非夫也。」

景泰二年舉進士,明年授御史。懷獻太子既薨,中外望復沂王於東宮。同與郎中章綸早朝,語及沂王,皆泣下,因與約疏請復儲。五年五月,同因上疏論時政,遂及復儲事,其略曰:

近得賊諜,言也先使偵京師及臨清虛實,期初秋大舉深入,直下河南。臣聞之不勝寒心,而廟堂大臣皆恬不介意。昔秦伐趙,諸侯自若,孔子順獨憂之,人皆以為狂。臣今者之言,何以異此。臣草茅時,聞寺人構惡,戕戮直臣劉球,遂致廷臣箝口。假使當時犯顏有人,必能諫止上皇之行,何至有蒙塵之禍。

陛下赫然中興,鋤奸黨,旌忠直。命六師禦敵於郊,不戰而三軍之氣自倍。臣謂陛下方且鞭撻四裔,坐致太平,奈何邊氛甫息,瘡痍未復,而侈心遽生,失天下望。伏願取鑒前車,厚自奮厲。毋徇貨色,毋甘嬉遊。親庶政以總威權,敦倫理以厚風俗,辨邪正以專委任,嚴賞罰以彰善惡,崇風憲以正紀綱。去浮費,罷冗員。禁僧道之蠹民,擇賢將以訓士。然後親率群臣,謝過郊廟,如成湯之六事自責,唐太宗之十漸即改,庶幾天意可回,國勢可振。

又言:

父有天下,固當傳之於子。乃者太子薨逝,足知天命有在。臣竊以為上皇之子,即陛下之子。沂王天資厚重,足令宗社有託。伏望擴天地之量,敦友于之仁,蠲吉具儀,建復儲位,實祖宗無疆之休。

又言:

陛下命將帥各陳方略。經旬踰時,互相委責。及石亨、柳溥有言,又不過庸人孺子之計。平時尚爾,一旦有急,將何策制之?夫禦敵之方,莫先用賢。陛下求賢若渴,而大臣之排抑尤甚,所舉者率多親舊富厚之家。即長材屈抑,孰肯為言?朝臣欺謾若此,臣所以撫膺流涕,為今日妨賢病國者醜也。

疏入,帝不懌。下廷臣集議。寧陽侯陳懋、吏部尚書王直等請帝納其言,因引罪求罷。帝慰留之。越數日,章綸亦疏言復儲事,遂並下詔獄。明年八月,大理少卿廖莊亦以言沂王事予杖。左右言:事由同倡,帝乃封巨梃就獄中杖之,同竟死。時年三十二。

同之上疏也,策馬出,馬伏地不肯起。同叱曰:「吾不畏死,爾奚為者!」馬猶盤辟再四,乃行。同死,馬長號數聲亦死。

英宗復位,贈同大理左寺丞,錄其子啟為國子生,尋授咸寧知縣。啟請父遺骸歸葬,詔給舟車路費。成化中,授次子越通政知事,給同妻羅氏月廩。尋賜同謚恭愍,從祀忠節祠,與球聯位,竟如同初志。

方同下獄時,有禮部郎孟者,亦疏言復儲事。帝不罪。而進士楊集上書于謙曰:「奸人黃矰獻議易儲,不過為逃死計耳,公等遽成之。公國家柱石,獨不思所以善後乎?今同等又下獄矣,脫諸人死杖下,而公等坐享崇高,如清議何!」謙以書示王文。文曰:「書生不知忌諱,要為有膽,當進一官處之。」乃以集知安州。玘,閩人;集,常熟人也。

章綸,字大經,樂清人。正統四年進士。授南京禮部主事。

景泰初,召為儀制郎中。綸見國家多故,每慷慨論事。嘗上太平十六策,反復萬餘言。也先既議和,請力圖修攘以待其變。中官興安請帝建大隆福寺成,將臨幸。綸具疏諫,河東鹽運判官濟南楊浩除官未行,亦上章諫,帝即罷幸。浩後累官副都御史,巡撫延綏。綸又因災異請求致變之由,語頗切至。

五年五月,鐘同上奏請復儲。越二日,綸亦抗疏陳修德弭災十四事。其大者謂:「內官不可干外政,佞臣不可假事權,後宮不可盛聲色。凡陰盛之屬,請悉禁罷。」又言:「孝弟者,百行之本。願退朝後朝謁兩宮皇太后,修問安視膳之儀。上皇君臨天下十有四年,是天下之父也;陛下親受冊封,是上皇之臣也。陛下與上皇,雖殊形體,實同一人。伏讀奉迎還宮之詔曰:『禮惟加而無替,義以卑而奉尊。』望陛下允蹈斯言。或朔望,或節旦,率群臣朝見延和門,以展友于之情,實天下之至願也。更請復汪后於中宮,正天下之母儀;還沂王之儲位,定天下之大本。如此則和氣充溢,災沴自弭。」疏入,帝大怒。時日已暝,宮門閉。乃傳旨自門隙中出,立執綸及鐘同下詔獄。榜掠慘酷,逼引主使及交通南宮狀。瀕死,無一語。會大風揚沙,晝晦,獄得稍緩,令錮之。明年杖廖莊闕下。因封杖就獄中杖綸、同各百。同竟死,綸長繫如故。

英宗復位,郭登言綸與廖莊、林聰、左鼎、倪敬等皆直言忤時,宜加旌擢。帝乃立釋綸。命內侍檢前疏,不得。內侍從旁誦數語,帝嗟歎再三,擢禮部右侍郎。

綸既以大節為帝所重,而性亢直,不能諧俗。石亨貴倖招公卿飲,綸辭不往,又數與尚書楊善論事不合。亨、善共短綸。乃調南京禮部,就改吏部。

憲宗即位,有司以遺詔請大婚。綸言:「山陵尚新,元朔未改,百日從吉,心寧自安。陛下踐阼之初,當以孝治天下,三綱五常實原於此。乞俟來春舉行。」議雖不從,天下咸重其言。

成化元年,兩淮饑,奏救荒四事。皆報可。四年秋,子玄應以冒籍舉京闈。給事中朱清、御史楊智等因劾綸,命侍郎葉盛勘之。明年,綸及僉都御史高明考察庶官,兩人議不協。疏既上,綸復獨奏給事中王讓不赴考察,且言明剛愎自用,己言多不見從,乞與明俱罷。章並下盛等。於是讓及下考諸臣連章劾綸。綸亦屢疏求罷。帝不聽。既而盛等勘上玄應實冒籍。帝宥綸,而所奏他事,亦悉不問。未幾,復轉禮部。溫州知府范奎被論調官。綸言:「溫州臣鄉郡,奎大得民心。解官之日,士民三萬人哭泣攀轅,留十八日乃得去。請還之以慰民望。」章下所司,竟報寢。

綸性戇,好直言,不為當事者所喜。為侍郎二十年,不得遷,請老去。久之卒。居數年,其妻張氏上其奏稿,且乞恩。帝嘉歎,贈南京禮部尚書,謚恭毅,官一子鴻臚典簿。

玄應後舉進士,為南京給事中。偕同官論陳鉞罪,忤旨停俸。孝宗嗣位,上治本五事。仕終廣東布政使。

廖莊,字安止,吉水人。宣德五年進士。八年改庶吉士,與知縣孔友諒等七人歷事六科。

英宗初,授刑科給事中。正統二年,御史元亮請如詔書蠲邊軍侵沒糧餉,不允。按察使龔鐩亦請如詔書宥盜犯之未獲者,法司亦寢不行。莊以詔書當信,上章爭之。五年詔京官出修荒政,兼徵民逋。莊慮使者督趣困民,請寬災傷州縣,俟秋成。從之。振荒陜西,全活甚眾。還奏寬恤九事,多議行。楊士奇家人犯法,偕同官論列。或曰:「獨不為楊公地乎?」曰:「正所以為楊公也。」八年命與御史張驥同署大理寺事。踰月,授左寺丞。

十一年遷南京大理少卿。踰二年,奸人陳玞者,與所親賈福爭襲指揮職。南京刑部侍郎齊韶納玞賄,欲奪福官與之,為莊所駁。韶捶福至死,被逮。玞亦誣莊,俱徵下詔獄。會韶他罪並發,棄市,莊乃得釋。

景泰五年七月上疏曰:「臣曩在朝,見上皇遣使冊封陛下,每遇慶節,必令群臣朝謁東廡,恩禮隆洽,群臣皆感歎,謂上皇兄弟友愛如此。今陛下奉天下以事上皇,願時時朝見南宮,或講明家法,或商略治道,歲時令節,俾群臣朝見,以慰上皇之心,則祖宗在天之神安,天地之心亦安矣。太子者,天下之本。上皇之子,陛下之猶子也。宜令親儒臣,習書策,以待皇嗣之生,使天下臣民曉然知陛下有公天下之心,豈不美歟?蓋天下者,太祖、太宗之天下。仁宗、宣宗繼體守成者,此天下也。上皇北征,亦為此天下也。今陛下撫而有之,宜念祖宗創業之艱難,思所以係屬天下之人心,即弭災召祥之道莫過於此。」疏入,不報。明年,莊以母喪,赴京關給勘合,詣東角門朝見。帝憶莊前疏,命廷杖八十,謫定羌驛丞。

天順初,召還。時母喪未終,復遭父喪,特予祭葬,命起復,仍官南京。天順五年就擢禮部右侍郎,改刑部。成化初,召為刑部左侍郎。逾年卒。贈尚書,謚恭敏。

莊性剛,喜面折人過,而實坦懷無芥蒂。不屑細謹,好存謝賓客為歡狎。既官法司,或勸稍屏謝往來,遠嫌疑。莊笑曰:「昔人有言『臣門如市,臣心如水』,吾無媿吾心而已。」卒之日,無以為斂,眾裒錢助其喪。

初,景帝時,英宗在南宮,左右為離間。及懷憲太子薨,群小恐沂王復立,讒構愈甚。故鐘同、章綸與莊相繼力言,皆得罪。然帝頗感悟。六年七月辛巳,刑科給事中徐正請間言事。亟召入,乃言:「上皇臨御歲久,沂王嘗位儲副,天下臣民仰戴。宜遷置所封之地,以絕人望。別選親王子育之宮中。」帝驚愕,大怒,立叱出之。欲正其罪,慮駭眾,乃命謫遠任,而帝怒未解。己,復得其淫穢事,謫戍鐵嶺衛。蓋帝雖怒同等所言過激,而小人之言亦未遽聽也。迨英宗復辟,于謙、王文以謀立外籓,誅死,其事遂不白云。

倪敬,字汝敬,無錫人。正統十三年進士。擢御史。景泰初,畿輔饑,命出視。請蠲田租,戶部持不可。再疏爭,竟得請。巡按山西。時有入粟補官令,敬奏罷之。戍將侵餉者,悉按治,豪猾斂迹。再按福建。時議將復銀冶,敬未行,抗疏論,得寢。既至,奏罷諸司器物濫取於民者。鎮守內臣戴細保貪橫,敬列其罪以聞。帝召細保還,命敬捕治其黨,吏民相慶。代還,留家四月,逮治,尋復職。

六年七月,以時多災異,偕同官吳江盛昶、江陰杜宥、蕪湖黃讓、安福羅俊、固始汪清上言:「府庫之財,不宜無故而予;遊觀之事,不宜非時而行。曩以齋僧,屢出帑金易米,不知櫛風沐雨之邊卒,趨事急公之貧民,又何以濟之?近聞造龍舟,作燕室,營繕日增,嬉遊不少,非所以養聖躬也。章綸、鐘同直言見忤,幽錮踰年,非所以昭聖德也。願罷桑門之供,輟宴佚之娛,止興作之役,寬直臣之囚。」帝得疏不懌,下之禮部。部臣稱其忠愛。帝報聞,然意終不釋。未幾,詔都御史蕭維禎考察其屬,諭令去之。御史罷黜者十六人,而敬等預焉;皆謫為典史,敬得廣西宜山。英宗復闢,詔皆授知縣,乃以敬知祥符。安遠侯柳溥器敬,西征,請以自隨,改都督府都事。踰年師還,卒。士類惜之。

盛昶等五人,皆進士。昶雋爽負氣。嘗按廣東,劾巡撫侍郎揭稽不職,稽坐左遷。昶後為羅江知縣,擢敘州知府,並有禦寇功。杜宥為英德知縣。鄰境多寇,創立縣城。嘗被圍糧盡,宥死守不下。夜縋死士焚其營,賊始驚潰。移韶州通判,謝病歸。黃讓知安岳,遷中府都事。以撻錦衣衛隸,為門達所譖,戍廣西。赦還,復冠帶。貧甚,課耕自給。羅俊嘗巡按四川,有廉聲。仕終南雄知府。

楊瑄,字廷獻,豐城人。景泰五年進士。授御史。剛直尚氣節。景帝不豫,廷臣請立東宮,帝不允。瑄與同官錢璡、樊英等約疏爭,會「奪門」事起,乃已。

天順初,印馬畿內。至河間,民訴曹吉祥、石亨奪其田。瑄以聞,並列二人怙寵專權狀。帝語大學士李賢、徐有貞曰:「真御史也。」遂遣官按核,而命吏部識瑄名,將擢用。吉祥聞之懼,訴於帝,請罪之。不許。

未幾,亨西征還,適彗星見,十三道掌道御史張鵬、盛顒、周斌、費廣、張寬、王鑑、趙文博、彭烈、張奎、李人儀、邵銅、鄭冕、陶復及御史劉泰、魏翰、康驥將劾亨、吉祥諸違法事。先一日,給事中王鉉洩於亨。亨與吉祥泣訴帝,誣鵬等為已誅內官張永從子,結黨排陷,欲為永報仇。明日疏入,帝大怒,收鵬及瑄。御文華殿,悉召諸御史,擲彈章,俾自讀。斌且讀且對,神色自若。至冒功濫職,帝詰之曰:「彼帥將士迎駕,朝廷論功行賞,何雲冒濫?」斌曰:「當時迎駕止數百人,光祿賜酒饌,名數具在。今超遷至數千人,非冒濫而何?」帝默然,竟下瑄、鵬及諸御史於獄。榜掠備至,詰主使者,瑄等無所引,乃坐都御史耿九疇、羅綺主謀,亦下獄。論瑄、鵬死,餘遣戍。亨等復譖諸言官。帝諭吏部,給事、御史年踰三十者留之,餘悉調外。尚書翱列上給事中何等十三人,御史吳禎等二十三人。詔以等為州判官,禎等為知縣。會大風震雷,拔木發屋,須臾大雨雹。亨、吉祥家大木俱折,二人亦懼。掌欽天監禮部侍郎湯序本亨黨,亦言上天示警,宜恤刑獄。於是帝感悟,戍瑄、鵬鐵嶺衛,餘貶知縣,泰、翰、驥三人復職,而、禎等亦得無調。、鵬行半道,適承天門災,肆赦放還。或謂當詣亨、吉祥謝,二人卒不往,復謫戍南丹。

憲宗即位,並還故官。瑄尋遷浙江副使。按行海道,禁將校私縱戍卒。修捍海塘,築海鹽堤岸二千三百丈,民得奠居。為副使十餘年,政績卓然,進按察使。西湖水舊可溉諸縣田四十六萬頃,時堙塞過半,瑄請浚之。設防置閘,以利灌溉,功未就,卒。海鹽人祠祀之。

子源,字本清,幼習天文,授五官監候。正德元年,劉瑾等亂政,源上言:「自八月初,大角及心宿中星動搖不止。大角,天王之坐,心宿中星,天王正位也,俱宜安靜,今乃動搖。其占曰:『人主不安,國有憂。』意者陛下輕舉逸遊,弋獵無度,以致然也。又北斗第二第三第四星,明不如常。第二曰天璇,后妃之象。后妃不得其寵則不明,廣營宮室妄鑿山陵則不明。第三曰天機,不愛百姓,驟興征徭則不明。第四曰天權,號令不當則不明。伏願陛下祗畏天戒,安居深宮,絕嬉戲,禁遊畋,罷騎射,停工作,申嚴號令,毋輕出入,抑遠寵倖,裁節賜予,親元老大臣,日事講習,克修厥德,以弭災變。」疏下禮部,尚書張昇等稱源忠愛。報聞。

迨十月,霾霧時作,源言:「此眾邪之氣,陰冒於陽,臣欺其君,小人擅權,下將叛上。」引譬甚切。瑾怒,矯旨杖三十,釋之。又上言:「自正德二年來,占得火星入太微垣帝座前,或東或西,往來不一,乞收攬政柄,思患預防。」蓋專指瑾也。瑾大怒,召而叱之曰:「若何官,亦學為忠臣?」源厲聲曰:「官大小異,忠一也。」又矯旨杖六十,謫戍肅州。行至河陽驛,以創卒。其妻斬蘆荻覆之,葬驛後。

楊氏父子以忠諫名天下,為士論重。而源小臣抗節,尤人所難。天啟初,賜謚忠懷。

盛顒,字時望,無錫人。周斌,字國用,昌黎人。王鑒,太原人。趙文博,代州人。彭烈,峽江人。李人儀,隆昌人。邵銅,閩縣人。鄭冕,樂平人。皆進士,授御史。顒降束鹿知縣;斌,江陰;鑑,膚施;文博,淳化;烈,江浦;人儀,襄陽;銅,博羅;冕,衡山。並有善政。

束鹿徭役苦不均,顒為立九則法,繼者莫能易。母憂去。服除,民相率詣闕乞還。顒再任,益不用鞭撲。訟者,諭之,輒叩頭不復辯。鄰邑訟不決,亦皆赴訴,片言折之,各心厭去。郊外有隙地,爭來築室居之,遂成市,號為「清官店」。

斌在江陰,有惠政。民歌曰:「旱為災,周公禱之甘露來;水為患,周公禱之陰雨散。」天順七年,先以薦擢開封知府。而顒等至憲宗嗣位,所司以治行聞。帝曰:「諸臣直諫為權倖所排,又能稱職,其悉予郡。」於是擢顒知邵武;鑒,延安;文博,衛輝;烈,河南;人儀,荊州;銅,溫州;冕,衡州。顒復以任治劇,調延平。巡按御史上顒政績;陜西、湖廣守臣亦上鑒、人儀居縣時治行。皆特賜封誥。

顒累遷陜西左布政使。時三邊多警,歲復洊饑。顒經畫饋餉無缺,軍民悉安。成化十七年召為刑部右侍郎。居二年,山東旱饑,盜起,改顒左副都御史往巡撫。顒至露禱,大雨霑溉,稿禾復蘇。舉救荒之政,既振,餘粟尚百餘萬石。又推行九則法於諸府,黜暴除苛,民甚德之。居三年,以老致仕。弘治中卒。

斌,歷廣東右布政使。初去江陰,民立生祠。及自開封遷去,民亦涕泣追送焉。鑒,初為御史,嘗於左順門面斥中官非禮。中官怒甚,因考察屬都御史蕭維禎去之,維禎不可而止。文博,終巡撫河南右副都御史。烈,廣東左布政使。費廣等無考。

贊曰:直言敢諫之士,激於事變,奮不顧身,獲罪固其所甘心耳。然觀尹昌隆死於呂震;耿通陷於高煦;劉球之斃,陳鑒之系,由於王振;楊瑄之戍,厄於石亨、曹吉祥;乃至戴綸諫游獵,陳祚請勤學,鐘同、章綸、廖莊倡復儲,倪敬等直言時事,皆用賈禍。忠臣之志抑而不伸,亦可悲夫。


\end{pinyinscope}