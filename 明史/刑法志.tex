\article{刑法志}


自漢以來,刑法沿革不一。隋更五刑之條,設三奏之令。唐撰律令,一準乎禮,以為出入。宋採用之,而所重者敕。律所不載者,則聽之於敕。故時輕時重,無一是之歸。元制,取所行一時之例為條格而已。明初,丞相李善長等言:「歷代之律,皆以漢《九章》為宗,至唐始集其成。今制宜遵唐舊。」太祖從其言。

始,太祖懲元縱弛之後,刑用重典,然特取決一時,非以為則。後屢詔釐正,至三十年,始申畫一之制,所以斟酌損益之者,至纖至悉,令子孫守之。群臣有稍議更改,即坐以變亂祖制之罪。而後乃滋弊者,由於人不知律,妄意律舉大綱,不足以盡情偽之變,於是因律起例,因例生例,例愈紛而弊愈無窮。初詔內外風憲官,以講讀律令一條,考校有司。其不能曉晰者,罰有差。庶幾人知律意。因循日久,視為具文。由此奸吏骫法,任意輕重。至如律有取自上裁、臨時處治者,因罪在八議不得擅自勾問、與一切疑獄罪名難定、及律無正文者設,非謂朝廷可任情生殺之也。英、憲以後,欽恤之意微,偵伺之風熾。巨惡大憝,案如山積,而旨從中下,從之不問。或本無死理,而片紙付詔獄,為禍尤烈。故綜明代刑法大略,而以廠衛終之。廠豎姓名,傳不備載,列之於此,使有所考焉。

明太祖平武昌,即議律令。吳元年冬十月,命左丞相李善長為律令總裁官,參知政事楊憲、傅瓛,御史中丞劉基,翰林學士陶安等二十人為議律官,諭之曰:「法貴簡當,使人易曉。若條緒繁多,或一事兩端,可輕可重,吏得因緣為奸,非法意也。夫網密則水無大魚,法密則國無全民。卿等悉心參究,日具刑名條目以上,吾親酌議焉。」每御西樓,召諸臣賜坐,從容講論律義。十二月,書成,凡為令一百四十五條,律二百八十五條。又恐小民不能周知,命大理卿周楨等取所定律令,自禮樂、制度、錢糧、選法之外,凡民間所行事宜,類聚成編,訓釋其義,頒之郡縣,名曰《律令直解》。太祖覽其書而喜曰:「吾民可以寡過矣。」

洪武元年,又命儒臣四人,同刑官講《唐律》,日進二十條。五年,定宦官禁令及親屬相容隱律,六年夏,刊《律令憲綱》,頒之諸司。其冬,詔刑部尚書劉惟謙詳定《大明律》。每奏一篇,命揭兩廡,親加裁酌。及成,翰林學士宋濂為表以進,曰:「臣以洪武六年冬十一月受詔,明年二月書成。篇目一準於唐:曰衛禁,曰職制,曰戶婚,曰廄庫,曰擅興,曰賊盜,曰鬥訟,曰詐偽,曰雜律,曰捕亡,曰斷獄,曰名例。採用舊律二百八十八條,續律百二十八條,舊令改律三十六條,因事制律三十一條,掇《唐律》以補遺百二十三條,合六百有六條,分為三十卷。或損或益,或仍其舊,務合輕重之宜。」九年,太祖覽律條猶有未當者,命丞相胡惟庸、御史大夫汪廣洋等詳議,釐正十有三條。十六年,命尚書開濟定詐偽律條。二十二年,刑部言:「比年條例增損不一,以致斷獄失當。請編類頒行,俾中外知所遵守。」遂命翰林院同刑部官,取比年所增者,以類附入。改《名例律》冠於篇首。

為卷凡三十,為條四百有六十。《名例》一卷,四十七條。《吏律》二卷,曰職制十五條,曰公式十八條。《戶律》七卷山西陽曲人。曾講學於三立書院。明亡後,隱居山中,堅持,曰戶役十五條,曰田宅十一條,曰婚姻十八條,曰倉庫二十四條,曰課程十九條,曰錢債三條,曰市廛五條。《禮律》二卷,曰祭祀六條,曰儀制二十條。《兵律》五卷,曰宮衛十九條,曰軍政二十條,曰關津七條,曰廄牧十一條,曰郵驛十八條。《刑律》十一卷,曰盜賊二十八條,曰人命二十條,曰鬥毆二十二條,曰罵詈八條,曰訴訟十二條,曰受贓十一條,曰詐偽十二條,曰犯姦十條,曰雜犯十一條,曰捕亡八條,曰斷獄二十九條。《工律》二卷,曰營造九條,曰河防四條。

為五刑之圖凡二。首圖五:曰笞,曰杖,曰徒,曰流,曰死。笞刑五,自一十至五十,每十為一等加減。杖刑五,自六十至一百,每十為一等加減。徒刑五,徒一年杖六十,一年半杖七十,二年杖八十,二年半杖九十,三年杖一百,每杖十及徒半年為一等加減。流刑三,二千里,二千五百里,三千里,皆杖一百,每五百里為一等加減。死刑二,絞、斬。五刑之外,徒有總徒四年,遇例減一年者,有準徒五年,斬、絞、雜犯減等者。流有安置,有遷徙,去鄉一千里,杖一百,準徒二年。有口外為民,其重者曰充軍。充軍者,明初唯邊方屯種。後定制,分極邊、煙瘴、邊遠、邊衛、沿海、附近。軍有終身,有永遠。二死之外,有凌遲,以處大逆不道諸罪者。充軍、凌遲,非五刑之正,故圖不列。凡徒流再犯者,流者於原配處所,依工、樂戶留住法。三流並決杖一百,拘役三年。拘役者,流人初止安置,今加以居作,即唐、宋所謂加役流也。徒者於原役之所,依所犯杖數年限決訖,應役無得過四年。

次圖七:曰笞,曰杖,曰訊杖,曰枷,曰杻,曰索,曰鐐。笞,大頭徑二分七釐,小頭減一分。杖,大頭徑三分二釐,小頭減如笞之數。笞、杖皆以荊條為之,皆臀受。訊杖,大頭徑四分五釐,小頭減如笞杖之數,以荊條為之,臀腿受。笞、杖、訊,皆長三尺五寸,用官降式較勘,毋以筋膠諸物裝釘。枷,自十五斤至二十五斤止,刻其上為長短輕重之數。長五尺五寸,頭廣尺五寸,杻長尺六寸,厚一寸,男子死罪者用之。索,鐵為之,以繫輕罪者,其長一丈。鐐,鐵連環之,以縶足,徒者帶以輸作,重三斤。

又為喪服之圖凡八:族親有犯,視服等差定刑之輕重。其因禮以起義者,養母、繼母、慈母皆服三年。毆殺之,與毆殺嫡母同罪。兄弟妻皆服小功,互為容隱者,罪得遞減。舅姑之服皆斬衰三年,毆殺罵詈之者,與夫毆殺罵詈之律同。姨之子、舅之子、姑之子皆緦麻,是曰表兄弟,不得相為婚姻。

大惡有十:曰謀反,曰謀大逆,曰謀叛,曰惡逆,曰不道,曰大不敬,曰不孝,曰不睦,曰不義,曰內亂。雖常赦不原。貪墨之贓有六:曰監守盜,曰常人盜,曰竊盜,曰枉法,曰不枉法,曰坐贓。當議者有八:曰議親,曰議故,曰議功,曰議賢,曰議能,曰議勤,曰議貴,曰議賓。

太祖諭太孫曰:「此書首列二刑圖,次列八禮圖者,重禮也。顧愚民無知,若於本條下即註寬恤之令,必易而犯法,故以廣大好生之意,總列《名例律》中。善用法者,會其意可也。」太孫請更定五條以上,太祖覽而善之。太孫又請曰:「明刑所以弼教,凡與五倫相涉者,宜皆屈法以伸情。」乃命改定七十三條,復諭之曰:「吾治亂世,刑不得不重。汝治平世,刑自當輕,所謂刑罰世輕世重也。」二十五年,刑部言,律條與條例不同者宜更定。太祖以條例特一時權宜,定律不可改,不從。

三十年,作《大明律》誥成。御午門,諭群臣曰:「朕仿古為治,明禮以導民,定律以繩頑,刊著為令。行之既久,犯者猶眾,故作《大誥》以示民,使知趨吉避凶之道。古人謂刑為祥刑,豈非欲民並生於天地間哉!然法在有司,民不周知,故命刑官取《大誥》條目,撮其要略,附載於律。凡榜文禁例悉除之,除謀逆及《律誥》該載外,其雜犯大小之罪,悉依贖罪例論斷,編次成書,刊布中外,令天下知所遵守。」

《大誥》者,太祖患民狃元習,徇私滅公,戾日滋,十八年,采輯官民過犯,條為《大誥》。其目十條:曰攬納戶,曰安保過付,曰詭寄田糧,曰民人經該不解物,曰灑派拋荒田土,曰倚法為奸,曰空引偷軍,曰黥刺在逃,曰官吏長解賣囚,曰寰中士夫不為君用。其罪至抄劄。次年復為《續編》、《三編》,皆頒學宮以課士,里置塾師教之。囚有《大誥》者,罪減等。於時,天下有講讀《大誥》師生來朝者十九萬餘人,並賜鈔遣還。自《律誥》出,而《大誥》所載諸峻令未嘗輕用。其後罪人率援《大誥》以減等,亦不復論其有無矣。

蓋太祖之於律令也,草創於吳元年,更定於洪武六年,整齊於二十二年,至三十年始頒示天下。日久而慮精,一代法始定。中外決獄,一準三十年所頒。其洪武元年之令,有律不載而具於令者,法司得援以為證,請於上而後行焉。凡違令者罪笞,特旨臨時決罪,不著為律令者,不在此例。有司輒引比律,致罪有輕重者,以故入論。罪無正條,則引律比附,定擬罪名,達部議定奏聞。若輒斷決,致罪有出入者,以故失論。

大抵明律視唐簡核,而寬厚不如宋。至其惻隱之意,散見於各條,可舉一以推也。如罪應加者,必贓滿數乃坐。如監守自盜,贓至四十貫絞。若止三十九貫九十九文,欠一文不坐也。加極於流三千里,以次增重,終不得至死。而減至流者,自死而之生,無絞斬之別。即唐律稱加就重條。稱日者以百刻,稱年者以三百六十日。如人命辜限及各文書違限,雖稍不及一時刻,仍不得以所限之年月科罪,即唐例稱日以百刻條。未老疾犯罪,而事發於老疾,以老疾論;幼小犯罪,而事發於長大,以幼小論。即唐律老小廢疾條。犯死罪,非常赦所不原,而祖父母、父母老無養者,得奏聞取上裁。犯徒流者,餘罪得收贖,存留養親。即唐律罪非十惡條。功臣及五品以上官禁獄者,許令親人入侍,徒流者並聽隨行,違者罪杖。同居親屬有罪,得互相容隱。即唐律同居相容隱條。奴婢不得首主。凡告人者,告人祖父不得指其子孫為證,弟不證兄,妻不證夫,奴婢不證主。文職責在奉法,犯杖則不敘。軍官至徒流,以世功猶得擢用。凡若此類,或間採唐律,或更立新制,所謂原父子之親,立君臣之義以權之者也。

建文帝即位,諭刑官曰:「《大明律》,皇祖所親定,命朕細閱,較前代往往加重。蓋刑亂國之典,非百世通行之道也。朕前所改定,皇祖已命施行。然罪可矜疑者,尚不止此。夫律設大法,禮順人情,齊民以刑,不若以禮。其諭天下有司,務崇禮教,赦疑獄,稱朕嘉與萬方之意。」成祖詔法司問囚,一依《大明律》擬議,毋妄引榜文條例為深文。永樂元年,定誣告法。成化元年,又令讞囚者一依正律,盡革所有條例。十五年,南直隸巡撫王恕言:「《大明律》後,有《會定見行律》百有八條,不知所起。如《兵律》多支廩給,《刑律》罵制使及罵本管長官條,皆輕重失倫。流傳四方,有誤官守。乞追板焚毀。」命即焚之,有依此律出入人罪者,以故論。十八年,定挾詐得財罪例。

弘治中,去定律時已百年,用法者日弛。五年,刑部尚書彭韶等以鴻臚少卿李鐩請,刪定《問刑條例》。至十三年,刑官復上言:「洪武末,定《大明律》,後又申明《大誥》,有罪減等,累朝遵用。其法外遺姦,列聖因時推廣之而有例,例以輔律,非以破律也。乃中外巧法吏或借便己私,律浸格不用。」於是下尚書白昂等會九卿議,增歷年問刑條例經久可行者二百九十七條。帝摘其中六事,令再議以聞。九卿執奏,乃不果改。然自是以後,律例並行而網亦少密。王府禁例六條,諸王無故出城有罰,其法尤嚴。嘉靖七年,保定巡撫王應鵬言:「正德間,新增問刑條例四十四款,深中情法,皆宜編入。」不從。惟詔偽造印信及竊盜三犯者不得用可矜例。刑部尚書胡世寧又請編斷獄新例,亦命止依律文及弘治十三年所欽定者。至二十八年,刑部尚書喻茂堅言:「自弘治間定例,垂五十年。乞敕臣等會同三法司,申明《問刑條例》及嘉靖元年後欽定事例,永為遵守。弘治十三年以後、嘉靖元年以前事例,雖奉詔革除,顧有因事條陳,擬議精當可採者,亦宜詳檢。若官司妄引條例,故入人罪者,當議黜罰。」會茂堅去官,詔尚書顧應祥等定議,增至二百四十九條。三十四年,又因尚書何鰲言,增入九事。萬歷時,給事中烏昇請續增條例。至十三年,刑部尚書舒化等乃輯嘉靖三十四年以後詔令及宗籓軍政條例、捕盜條格、漕運議單與刑名相關者,律為正文,例為附註,共三百八十二條,刪世宗時苛令特多。崇禎十四年,刑部尚書劉澤深復請議定《問刑條例》。帝以律應恪遵,例有上下,事同而二三其例者,刪定畫一為是。然時方急法,百司救過不暇,議未及行。

太祖之定律文也,歷代相承,無敢輕改。其一時變通,或由詔令,或發於廷臣奏議,有關治體,言獲施行者,不可以無詳也。

洪武元年,諭省臣:「鞫獄當平恕,古者非大逆不道,罪止及身。民有犯者,毋得連坐。」尚書夏恕嘗引漢法,請著律,反者夷三族。太祖曰:「古者父子兄弟罪不相及,漢仍秦舊,法太重。」卻其奏不行。民父以誣逮,其子訴於刑部,法司坐以越訴。太祖曰:「子訴父枉,出於至情,不可罪。」有子犯法,父賄求免者,御史欲并論父。太祖曰:「子論死,父救之,情也,但論其子,赦其父。」十七年,左都御史詹徽奏民毆孕婦至死者,律當絞,其子乞代。大理卿鄒俊議曰:「子代父死,情可矜。然死婦係二人之命,犯人當二死之條,與其存犯法之人,孰若全無辜之子。」詔從後議。二十年,詹徽言:「軍人有犯當杖,其人嘗兩得罪而免,宜并論前罪,誅之。」太祖曰:「前罪既宥,復論之則不信矣。」杖而遣之。二十四年,嘉興通判龐安獲鬻私鹽者送京師,而以鹽賞獲者。戶部以其違例,罰償鹽入官,且責取罪狀。安言:「律者萬世之常法,例者一時之旨意。今欲依例而行,則於律內非應捕人給賞之言,自相違悖,失信於天下也。」太祖然其言,詔如律。

永樂二年,刑部言河間民訟其母,有司反擬母罪。詔執其子及有司罪之。三年,定文職官及中外旗校軍民人等,凡犯重條,依律科斷,輕者免決,記罪。其有不應侵損於人等項及情犯重者,臨時奏請。十六年,嚴犯贓官吏之禁。初,太祖重懲貪吏,詔犯贓者無貸。復敕刑部:「官吏受贓者,并罪通賄之人,徙其家於邊。著為令。」日久法弛,故復申飭之。二十九年,大理卿虞謙言:「誑騙之律,當杖而流,今梟首,非詔書意。」命如律擬斷。宣德二年,江西按察使黃翰言:「民間無籍之徒,好興詞論,輒令老幼殘疾男婦誣告平人,必更議涉虛加罰乃可。」遂定老幼殘疾男婦誣告人罰鈔贖罪例。其後孝宗時,南京有犯誣告十人以上,例發口外為民。而年逾七十,律應收贖者,更著令,凡年七十以上、十五以下及廢疾者,依律論斷。例應充軍尞哨、口外為民者,仍依律發遣。若年八十以上及篤疾有犯應永戍者,以子孫發遣,應充軍以下者免之。

初制,凡官吏人等犯枉法贓者,不分南北,俱發北方邊衛充軍。正統五年,行在三法司言:「洪武定律時,鈔貴物賤,所以枉法贓至百二十貫者,免絞充軍。今鈔賤物貴,若以物估鈔至百二十貫枉法贓俱發充軍,輕重失倫矣。今後文職官吏人等,受枉法贓比律該絞者,估鈔八百貫之上,俱發北方邊衛充軍。其受贓不及前數者,視見行例發落。」從之。八年,大理寺言:「律載竊盜初犯刺右臂,再犯刺左臂,三犯絞。今竊盜遇赦再犯者,咸坐以初犯,或仍刺右臂,或不刺。請定為例。」章下三法司議,刺右遇赦再犯者刺左,刺左遇赦又犯者不刺,立案。赦後三犯者絞。」帝曰:「竊盜已刺,遇赦再犯者依常例擬,不論赦,仍通具前後所犯以聞。」後憲宗時,都御史李秉援舊例奏革。既而南京盜王阿童五犯皆遇赦免。帝聞之,詔仍以赦前後三犯為令。至神宗時,復議奏請改遣云。十二年,以知縣陳敏政言,民以後妻所攜前夫之女為子婦,及以所攜前夫之子為婿者,並依同父異母姊妹律,減等科斷。成化元年,遼東巡撫滕照言:「《大明律》乃一代定法,而決斷武臣,獨舍律用例,武臣益縱蕩不檢。請一切用律。」詔從之。武臣被黜降者,騰口謗訕,有司畏事,復奏革其令。十九年定,竊盜三犯罪例。法司以「南京有三犯竊盜,計贓滿百貫者犯,當絞斬。罪雖雜犯,其情頗重。」三犯前罪,即累惡不悛之人,難準常例。其不滿貫犯,徒流以下罪者,雖至三犯,原情實輕,宜特依常例治之。」議上,報允。

弘治六年,太常少卿李東陽言:「五刑最輕者笞杖,然杖有分寸,數有多寡。今在外諸司,笞杖之罪往往致死。縱令事覺,不過以因公還職。以極輕之刑,置之不可復生之地,多者數十,甚者數百,積骸滿獄,流血塗地,可為傷心。律故勘平人者抵命,刑具非法者除名,偶不出此,便謂之公。一以公名,雖多無害。此則情重而律輕者,不可以不議也。請凡考訊輕罪即時致死,累二十或三十人以上,本律外,仍議行降調,或病死不實者,并治其醫。」乃下所司議處。嘉靖十五年,時有以手足毆人傷重,延至辜限外死者,部擬鬥毆殺人論絞。大理寺執嘉靖四年例,謂當以毆傷論笞。部臣言:「律定辜限,而《問刑條例》又謂鬥毆殺人、情實事實者,雖延至限外,仍擬死罪,奏請定奪。臣部擬上,每奉宸斷,多發充軍,蓋雖不執前科,亦僅末減之耳。毆傷情實至限外死,即以笞斷,是乃僥倖兇人也。且如以兇器傷人,雖平復,例亦充軍,豈有實毆人致死,偶死限外,遂不當一兇器傷人之罪乎?矧四年例已報罷,請諭中外仍如《條例》便。」詔如部議。自後有犯辜限外人命者,俱遵律例議擬,奏請定奪。

隆慶三年,大理少卿王諍言:「問刑官每違背律例,獨任意見。如律文所謂『凡奉制書,有所施行而違者杖一百』,本指制誥而言。今則操軍違限,守備官軍不入直,開場賭博,概用此例。律文犯姦條下,所謂『買休賣休、和娶人妻者』,本指用財買求其妻,又使之休賣其妻,而因以娶之者言也。故律應離異歸宗,財禮入官。至若夫婦不合者,律應離異;婦人犯姦者,律從嫁賣;則後夫憑媒用財娶以為妻者,原非姦情,律所不禁。今則概引買休、賣休、和娶之律矣。所謂『不應得為而為者,笞四十,重者杖八十』。蓋謂律文該載不盡者,方用此律也。若所犯明有正條,自當依本條科斷。今所犯毆人成傷,罪宜笞,而議罪者則曰『除毆人成傷,律輕不坐外,合依不應得為而為之事理,重者律杖八十』。夫既除毆人輕罪不坐,則無罪可坐矣。而又坐以『不應得為』,臣誠不知其所謂。」刑部尚書毛愷力爭之,廷臣皆是諍議。得旨:「買休、賣休,本屬姦條,今後有犯,非係姦情者,不得引用。他如故。」

萬曆中,左都御史吳時來申明律例六條:

一、律稱庶人之家不許存養奴婢。蓋謂功臣家方給賞奴婢,庶民當自服勤勞,故不得存養。有犯者皆稱雇工人,初未言及縉紳之家也。縉紳之家,存養奴婢,勢所不免。合令法司酌議,無論官民之家,立券用值、工作有年限者,以雇工人論;受值微少、工作計日月者,以凡人論。若財買十五以下、恩養日久、十六以上、配有室家者,視同子孫論。或恩養未久,不曾配合者,庶人之家,仍以雇工人論;縉紳之家,視奴婢律論。

一、律稱偽造諸衙門印信者斬。惟銅鐵私鑄者,故斬。若篆文雖印,形質非印者,不可謂之偽造,故例又立描摸充軍之條。以後偽造印信人犯,如係木石泥蠟之類,止引描摸之例,若再犯擬斬。偽造行使止一次、而贓不滿徒者,亦準竊盜論。如再犯引例,三犯引律。

一、律稱竊盜三犯者絞,以曾經刺字為坐。但贓有多寡,即擬有輕重。以後凡遇竊盜,三犯俱在赦前、俱在赦後者,依律論絞。或赦前後所犯并計三次者,皆得奏請定奪。錄官附入矜疑辨問疏內,并與改遣。

一、強盜肆行劫殺,按贓擬辟,決不待時。但其中豈無羅織讎扳,妄收抵罪者?以後務加參詳。或贓證未明,遽難懸斷者,俱擬秋後斬。

一、律稱同謀共毆人,以致命傷重,下手者論絞,原謀餘人各得其罪。其有兩三人共毆一人,各成重傷,難定下手及係造謀主令之人,遇有在監禁斃者,即以論抵。今恤刑官遇有在家病故,且在數年之後者,即將見監下手之人擬從矜宥。是以病亡之軀,而抵毆死之命,殊屬縱濫。以後毋得一概準抵。

一、在京惡逆與強盜真犯,雖停刑之年,亦不時處決。乃兇惡至於殺父,即時凌遲,猶有餘憾。而在外此類反得遷延歲月,以故事當類奏,無單奏例耳。夫單奏,急詞也;類奏,緩詞也。如此獄在外數年,使其瘐死,將何以快神人之憤哉!今後在外,凡有此者,御史單詳到院,院寺單奏,決單一到,即時處決。其死者下府州縣戮其屍。庶典刑得正。

旨下部寺酌議,俱從之。惟偽造印文者,不問何物成造,皆斬。報可。

贖刑本《虞書》,《呂刑》有大辟之贖,後世皆重言之。至宋時,尤慎贖罪,非八議者不得與。明律頗嚴,凡朝廷有所矜恤、限於律而不得伸者,一寓之於贖例,所以濟法之太重也。又國家得時藉其入,以佐緩急。而實邊、足儲、振荒、宮府頒給諸大費,往往取給於贓贖二者。故贖法比歷代特詳。凡贖法有二,有律得收贖者,有例得納贖者。律贖無敢損益,而納贖之例則因時權宜,先後互異,其端實開於太祖云。

律凡文武官以公事犯笞罪者,官照等收贖錢,吏每季類決之,各還職役,不附過。杖以上記所犯罪名,每歲類送吏、兵二部,候九年滿考,通記所犯次數黜陟之。吏典亦備銓選降敘。至於私罪,其文官及吏典犯笞四十以下者,附過還職而不贖,笞五十者調用。軍官杖以上皆的決。文官及吏杖罪,並罷職不敘,至嚴也。然自洪武中年已三下令,準贖及雜犯死罪以下矣。三十年,命部院議定贖罪事例,凡內外官吏,犯笞杖者記過,徒流遷徙者俸贖之,三犯罪之如律。自是律與例互有異同。及頒行《大明律》,御製序:「雜犯死罪、徒流、遷徙等刑,悉視今定贖罪條例科斷。」於是例遂輔律而行。

仁宗初即位,諭都察院言:「輸罰工作之令行,有財者悉倖免,宜一論如律。」久之,其法復弛。正統間,侍講劉球言:「輸罪非古,自公罪許贖外,宜悉依律。」時不能從。其後循太祖之例,益推廣之。凡官吏公私雜犯準徒以下,俱聽運炭納米等項贖罪。其軍官軍人照例免徒流者,例贖亦如之矣。

贖罪之法,明初嘗納銅,成化間嘗納馬,後皆不行,不具載。惟納鈔、納錢、納銀常並行焉,而以初制納鈔為本。故律贖者曰收贖律鈔,納贖者曰贖罪例鈔。永樂十一年,令除公罪依例紀錄收贖,及死罪情重者依律處治,其情輕者,斬罪八千貫,絞罪及榜例死罪六千貫,流徒杖笞納鈔有差。無力者發天壽山種樹。宣德二年定,笞杖罪囚,每十贖鈔二十貫。徒流罪名,每徒一等折杖二十,三流並折杖百四十。其所罰鈔,悉如笞杖所定。無力者發天壽山種樹;死罪終身;徒流各按年限;杖,五百株;笞,一百株。景泰元年,令問擬笞杖罪囚,有力者納鈔。笞十,二百貫,每十以二百貫遞加,至笞五十為千貫。杖六十,千八百貫,每十以三百貫遞加,至杖百為三千貫。其官吏贓物,亦視今例折鈔。天順五年,令罪囚納鈔,每笞十,鈔二百貫,餘四笞,遞加百五十貫;至杖六十,增為千四百五十貫,餘杖各遞加二百貫,成化二年,令婦人犯法贖罪。

弘治十四年,定折收銀錢之制。例難的決人犯,并婦人有力者,每杖百,應鈔二千二百五十貫,折銀一兩;每十以二百貫遞減,至杖六十為銀六錢;笞五十,應減為鈔八百貫,折銀五錢,每十以百五十貫遞減;至笞二十為銀二錢;笞十應鈔二百貫,折銀一錢。如收銅錢,每銀一兩折七百文。其依律贖鈔,除過失殺人外,亦視此數折收。

正德二年,定錢鈔兼收之制。如杖一百,應鈔二千二百五十貫者,收鈔千一百二十五貫,錢三百五十文。嘉靖七年,巡撫湖廣都御史朱廷聲言:「收贖與贖罪有異,在京與在外不同,鈔貫止聚於都下,錢法不行於南方。故事,審有力及命婦、軍職正妻,及例難的決者,有贖罪例鈔;老幼廢疾及婦人餘罪,有收贖律鈔。贖罪例鈔,錢鈔兼收,如笞一十,收鈔百貫,收錢三十五文,其鈔二百貫,折銀一錢。杖一百,收鈔千一百二十五貫,收錢三百五十文,其鈔二千二百五十貫,折銀一兩。今收贖律鈔,笞一十,止贖六百文,比例鈔折銀不及一釐;杖一百,贖鈔六貫,折銀不及一分,似為太輕。蓋律鈔與例鈔,貫既不同,則折銀亦當有異。請更定為則,凡收贖者,每鈔一貫,折銀一分二釐五毫。如笞一十,贖鈔六百文,則折銀七釐五毫,以罪重輕遞加折收贖。」帝從其奏,令中外問刑諸司,皆以此例從事。

是時重修條例,奏定贖例。在京則做工、每笞一十,做工一月,折銀三錢。至徒五年,折銀十八兩。運囚糧、每笞一十,米五斗,折銀二錢五分。至徒五年,五十石,折銀二十五兩。運灰、每笞一十,一千二百斤,折銀一兩二錢六分。至徒五年,六萬斤,折銀六十三兩。運磚、每笞一十,七十個,折銀九錢一分。至徒五年,三千個,折銀三十九兩。運水和炭五等。每笞一十,二百斤,折銀四錢。至徒五年,八千五百斤,折銀十七兩。運灰最重,運炭最輕。在外則有力、稍有力二等。初有頗有力、次有力等,因御史言而革。其有力,視在京運囚糧,每米五斗,納穀一石。初折銀上庫,後折穀上倉。稍有力,視在京做工年月為折贖。婦人審有力,與命婦、軍職正妻,及例難的決之人,贖罪應錢鈔兼收者,笞、杖每一十,折收銀一錢。其老幼廢疾婦人及天文生餘罪收贖者,每笞一十應鈔六百文,折收銀七釐五毫。於是輕重適均,天下便之。至萬曆十三年,復申明焉,遂為定制。

凡律贖,若天文生習業已成、能專其事、犯徒及流者,決杖一百,餘罪收贖。婦人犯徒流者,決杖一百,餘罪收贖。

如杖六十,徒一年,全贖鈔應十二貫,除決杖準訖六貫,餘鈔六貫,折銀七分五釐,餘仿此。

其決杖一百,審有力又納例鈔二千二百五十貫,應收錢三百五十文,鈔一千一百二十五貫。

凡年七十以上十五以下及廢疾犯流以下,收贖;八十以上十歲以下及篤疾、盜及傷人者,亦收贖。凡犯罪時未老疾,事發時老疾者,依老疾論,犯罪時幼小,事發時長大者,依幼小論,並得收贖。

如六十九以下犯罪,年七十事發,或無疾時犯罪,廢疾後事發,得依老疾收贖。他或七十九以下犯死罪,八十事發,或廢疾時犯罪,篤疾時事發,得入上請。八十九犯死罪,九十事發,得勿論,不在收贖之例。

若在徒年限內老疾,亦如之。

如犯杖六十,徒一年,一月之後老疾,合計全贖鈔十二貫。除已杖六十,準三貫六百文,剩徒一年,應八貫四百文計算。每徒一月,贖鈔七百文,已役一月,准贖七百文外,未贖十一月,應收贖七貫七百文。餘仿此。

老幼廢疾收贖,惟雜犯五年仍科之。蓋在明初,即真犯死罪,不可以徒論也

其誣告例,告二事以上,輕實重虛,或告一事,誣輕為重者,已論決全抵剩罪,未論決笞杖收贖,徒流杖一百,餘罪亦聽收贖。

如告人笞三十,內止一十實已決,全抵,剩二十之罪未決,收贖一貫二百文。

如告人杖六十,內止二十實已決,全抵,剩四十之罪未決,收贖二貫四百文。

如告人杖六十,徒一年,內止杖五十實已決,全抵,剩杖一十、徒一年之罪未決,徒一年,折杖六十,併杖共七十,收贖四貫二百文。

如告人杖一百,流二千里,內止杖六十、徒一年實已決,以總徒四年論,全抵,剩杖四十、徒三年之罪未決,以連徒折杖流加一等論,共計杖二百二十,除告實杖六十、徒一年,折杖六十,剩杖一百,贖鈔六貫。若計剩罪,過杖一百以上,須決杖一百訖,餘罪方聽收贖。

又過失傷人,淮鬥毆傷人罪,依律收贖。至死者,准雜犯斬絞收贖,鈔四十二貫。內鈔八分,應三十三貫六百文,銅錢二分,應八千四百文,給付其家。已徒五年,再犯徒收贖。鈔三十六貫。若犯徒流,存留養親者,止杖一百,餘罪收贖。其法實杖一百,不准折贖,然後計徒流年限,一視老幼例贖之。此律自英宗時詔有司行之,後為制。天文生、婦女犯徒流,決杖一百,餘罪收贖者,雖罪止杖六十,徒一年,亦決杖一百,律所謂應加杖者是也。皆先依本律議,其所犯徒流之罪,以《誥》減之。至臨決時,某係天文生,某係婦人,依律決杖一百,餘收贖。所決之杖並須一百者,包五徒之數也。然與誣告收贖剩杖不同。蓋收贖餘徒者決杖,而贖徒收贖剩杖者,折流歸徒,折徒歸杖,而照數收贖之,其法各別也。其婦人犯徒流,成化八年定例,除姦盜不孝與樂婦外,若審有力并決杖,亦得以納鈔贖罪。例每杖十,折銀一錢為率,至杖一百,折銀一兩止。凡律所謂收贖者,贖餘罪也。其例得贖罪者,贖決杖一百也。徒、杖兩項分科之,除婦人,餘囚徒流皆杖決不贖。惟弘治十三年,許樂戶徒杖笞罪,亦不的決,此律鈔之大凡也。

例鈔自嘉靖二十九年定例。凡軍民諸色人役及舍餘審有力者,與文武官吏、監生、生員、冠帶官、知印、承差、陰陽生、醫生、老人、舍人,不分笞、杖、徒、流、雜犯死罪,俱令運灰、運炭、運磚、納米、納料等項贖罪。此上係不虧行止者。若官吏人等,例應革去職役,此係行止有虧者。與軍民人等審無力者,笞、杖罪的決,徒、流、雜犯死罪各做工、擺站、哨尞、發充儀從,情重者煎鹽炒鐵。死罪五年,流罪四年,徒按年限。其在京軍丁人等,無差占者與例難的決之人,笞杖亦令做工。時新例,犯姦盜受贓,為行止有虧之人,概不許贖罪。唯軍官革職者,俱運炭納米等項發落,不用五刑條例的決實配之文,所以寬武夫,重責文吏也。於是在京惟行做工、運囚糧等五項,在外惟行有力、稍有力二項,法令益徑省矣。

要而論之,律鈔輕,例鈔重。然律鈔本非輕也。祖制每鈔一文,當銀一釐,所謂笞一十折鈔六百文定銀七釐五毫者,即當時之銀六錢也。所謂杖一百折鈔六貫銀七分五釐者,即當時之銀六兩也。以銀六錢,比例鈔折銀不及一釐,以銀一兩,比例鈔折銀不及一分,而欲以此懲犯罪者之心,宜其勢有所不行矣。特以祖宗律文不可改也,於是不得已定為七釐五毫、七分五釐之制。而其實所定之數,猶不足以當所贖者之罪,然後例之變通生焉。

考洪武朝,官吏軍民犯罪聽贖者,大抵罰役之令居多,如發鳳陽屯種、滁州種苜蓿、代農民力役、運米輸邊贖罪之類,俱不用鈔納也。律之所載,笞若干,鈔若干文,杖若干,鈔若干貫者,垂一代之法也。然按三十年詔令,罪囚運米贖罪,死罪百石,徒流遞減,其力不及者,死罪自備米三十石,徒流十五石,俱運納甘州、威虜,就彼充軍。計其米價、腳價之費,與鈔數差不相遠,其定為贖鈔之等第,固不輕於後來之例矣。然罪無一定,而鈔法之久,日變日輕,此定律時所不及料也。即以永樂十一年令「斬罪情輕者,贖鈔八千貫,絞及榜例死罪六千貫」之詔言之,八千貫者,律之八千兩也;六千貫者,律之六千兩也;下至杖罪千貫,笞罪五百貫,亦一千兩、五百兩也。雖革除之際,用法特苛,豈有死罪納至八千兩,笞杖罪納至一千兩、五百兩而尚可行者?則知鈔法之弊,在永樂初年,已不啻輕十倍於洪武時矣。

宣德時,申交易用銀之禁,冀通鈔法。至弘治而鈔竟不可用,遂開準鈔折銀之例。及嘉靖新定條例,俱以有力、稍有力二科贖罪:有力米五斗,準律之納鈔六百文也;稍有力工價三錢,準律之做工一月也。是則後之例鈔,纔足比於初之律鈔耳。而況老幼廢疾,諸在律贖者之銀七釐五毫,準鈔六百文,銀七分五釐,準鈔六貫。凡所謂律贖者,以比於初之律鈔,其輕重相去尤甚懸絕乎?唯運炭、運石諸罪例稍重,蓋此諸罪,初皆令親自赴役,事完寧家,原無納贖之例。其後法令益寬,聽其折納,而估算事力,亦略相當,實不為病也。

大抵贖例有二:一罰役,一納鈔,而例復三變。罰役者,後多折工值納鈔,鈔法既壞,變為納銀、納米。然運灰、運炭、運石、運磚、運碎磚之名尚存也。至萬曆中年,中外通行有力、稍有力二科,在京諸例,並不見施行,而法益歸一矣。所謂通變而無失於古之意者此也。初,令罪人得以力役贖罪:死罪拘役終身,徒流按年限,笞杖計日月。或修造,或屯種,或煎鹽炒鐵,滿日疏放。疏放者,引赴御橋,叩頭畢,送應天府,給引寧家。合充軍者,發付陜西司,按籍編發。後皆折納工價,惟赴橋如舊。宣德二年,御史鄭道寧言:「納米贖罪,朝廷寬典,乃軍儲倉拘係罪囚,無米輸納,自去年二月至今,死者九十六人。」刑部郎俞士吉嘗奏:「囚無米者,請追納於原籍,匠仍輸作,軍仍備操,若非軍匠,則遣還所隸州縣追之。」詔從其奏。

初制流罪三等,視地遠近,邊衛充軍有定所。蓋降死一等,唯流與充軍為重。然《名例律》稱二死三流各同為一減。如二死遇恩赦減一等,即流三千里,流三等以《大誥》減一等,皆徒五年。犯流罪者,無不減至徒罪矣。故三流常設而不用。而充軍之例為獨重。律充軍凡四十六條,《諸司職掌》內二十二條,則洪武間例,皆律所不載者。其嘉靖二十九年條例,充軍凡二百十三條,與萬曆十三年所定大略相同。洪武二十六年定,應充軍者,大理寺審訖,開付陜西司,本部置立文簿,註姓名、年籍、鄉貫,依南北籍編排甲為二冊,一進內府,一付該管百戶,領去充軍。如浙江,河南,山東,陜西,山西,北平,福建,直隸應天、廬州、鳳陽、淮安、揚州、蘇州、松江、常州、和州、滁州、徐州人,發雲南、四川屬衛;江西、湖廣,四川,廣東,廣西,直隸太平、寧國、池州、徽州、廣德、安慶人,發北平、大寧、遼東屬衛。有逃故,按籍勾補。其後條例有發煙瘴地面、極邊沿海諸處者,例各不同。而軍有終身,有永遠。永遠者,罰及子孫,皆以實犯死罪減等者充之。明初法嚴,縣以千數,數傳之後,以萬計矣。有丁盡戶絕,止存軍產者,或并無軍產,戶名未除者,朝廷歲遣御史清軍,有缺必補。每當勾丁,逮捕族屬、里長,延及他甲,雞犬為之不寧。論者謂既減死罪一等,而法反加於刀鋸之上,如革除所遣謫,至國亡,戍籍猶有存者,刑莫慘於此矣。嘉靖間,有請開贖軍例者。世宗曰:「律聽贖者,徒杖以下小罪耳。死罪矜疑,乃減從謫發,不可贖。」御史周時亮復請廣贖例。部議審有力者銀十兩,得贖三年以上徒一年,稍有力者半之。而贖軍之議卒罷。御史胡宗憲言:「南方之人不任兵革,其發充邊軍者,宜令納銀自贖。」部議以為然,因擬納例以上。帝曰:「豈可設此例以待犯罪之人?」復不允。

萬曆二年,罷歲遣清軍御史,并於巡按,民獲稍安。給事中徐桓言:「死罪雜犯準徒充軍者,當如其例。」給事中嚴用和請以大審可矜人犯,免其永戍。皆不許。而命法司定例:「奉特旨處發叛逆家屬子孫,止於本犯親枝內勾補,盡絕即與開豁。若未經發遣而病故,免其勾補。其實犯死罪免死充軍者,以著伍後所生子孫替役,不許勾原籍子孫。其他充軍及發口外者,俱止終身。」崇禎十一年,諭兵部:「編遣事宜,以千里為附近,二千五百里為邊衛,三千里外為邊遠,其極邊煙瘴以四千里外為率。止拘本妻,無妻則已,不許擅勾親鄰。如衰痼老疾,準發口外為民。」十五年,又諭:「欲令引例充軍者,準其贖罪。」時天下已亂,議卒不行。

明制充軍之律最嚴,犯者亦最苦。親族有科斂軍裝之費,裡遞有長途押解之擾。至所充之衛,衛官必索常例。然利其逃走,可乾沒口糧,每私縱之。其後律漸弛,發解者不能十一。其發極邊者,長解輒賄兵部,持勘合至衛,虛出收管,而軍犯顧在家偃息云。


三法司曰刑部、都察院、大理寺。刑部受天下刑名,都察院糾察,大理寺駁正。太祖嘗曰:「凡有大獄,當面訊,防構陷鍛煉之弊。」故其時重案多親鞫,不委法司。洪武十四年,命刑部聽兩造之詞,議定入奏。既奏,錄所下旨,送四輔官、諫院官、給事中覆核無異,然後覆奏行之。有疑獄,則四輔官封駁之。踰年,四輔官罷,乃命議獄者一歸於三法司。十六年,命刑部尚書開濟等,議定五六日旬時三審五覆之法。十七年,建三法司於太平門外鐘山之陰,命之曰貫城。下敕言:「貫索七星如貫珠,環而成象名天牢。中虛則刑平,官無邪私,故獄無囚人;貫內空中有星或數枚者即刑繁,刑官非其人;有星而明,為貴人無罪而獄。今法天道置法司,爾諸司其各慎乃事,法天道行之,令貫索中虛,庶不負朕肇建之意。」又諭法司官:「布政、按察司所擬刑名,其間人命重獄,具奏轉達刑部、都察院參考,大理寺詳擬。著為令。」

刑部有十三清吏司,治各布政司刑名,而陵衛、王府、公侯伯府、在京諸曹及兩京州郡,亦分隸之。按察名提刑,蓋在外之法司也,參以副使、僉事,分治各府縣事。京師自笞以上罪,悉由部議。洪武初決獄,笞五十者縣決之,杖八十者州決之,一百者府決之,徒以上具獄送行省,移駁繁而賄賂行。乃命中書省御史臺詳讞,改月報為季報,以季報之數,類為歲報。凡府州縣輕重獄囚,依律決斷。違枉者,御史、按察司糾劾。至二十六年定制,布政司及直隸府州縣,笞杖就決;徒流、遷徙、充軍、雜犯死罪解部,審錄行下,具死囚所坐罪名上部詳議如律者,大理寺擬覆平允,監收侯決。其決不待時重囚,報可,即奏遣官往決之。情詞不明或失出入者,大理寺駁回改正,再問駁至三,改擬不當,將當該官吏奏問,謂之照駁。若亭疑讞決,而囚有番異,則改調隔別衙門問擬。二次番異不服,則具奏,會九卿鞫之,謂之圓審。至三四訊不服,而後請旨決焉。

正統四年,稍更直省決遣之制,徒流就彼決遣,死罪以聞。成化五年,南大理評事張鈺言:「南京法司多用嚴刑,迫囚誣服,其被糾者亦止改正而無罪,甚非律意。」乃詔申大理寺參問刑部之制。弘治十七年,刑部主事朱瑬言:「部囚送大理,第當駁正,不當用刑。」大理卿楊守隨言:「刑具永樂間設,不可廢。」帝是其言。

會官審錄之例,定於洪武三十年。初制,有大獄必面訊。十四年,命法司論囚,擬律以奏刊登,從翰林院、給事中及春坊正字、司直郎會議平允,然後覆奏論決。至是置政平、訟理二幡,審諭罪囚。諭刑部曰:「自今論囚,惟武臣、死罪,朕親審之,餘俱以所犯奏。然後引至承天門外,命行人持訟理幡,傳旨諭之;其無罪應釋者,持政平幡,宣德意遣之。」繼令五軍都督府、六部、都察院、六科、通政司、詹事府,間及駙馬雜聽之,錄冤者以狀聞,無冤者實犯死罪以下悉論如律,諸雜犯準贖。永樂七年,令大理寺官引法司囚犯赴承天門外,行人持節傳旨,會同府、部、通政司、六科等官審錄,如洪武制。十七年,令在外死罪重囚,悉赴京師審錄。仁宗特命內閣學士會審重囚,可疑者再問。宣德三年奏重囚,帝令多官覆閱之,曰:「古者斷獄,必訊於三公九卿,所以合至公,重民命。卿等往同覆審,毋致枉死。」英國公張輔等還奏,訴枉者五十六人,重命法司勘實,因切戒焉。

天順三年,令每歲霜降後,三法司同公、侯、伯會審重囚,謂之朝審。歷朝遂遵行之。成化十七年,命司禮太監一員會同三法司堂上官,於大理寺審錄,謂之大審。南京則命內守備行之。自此定例,每五年輒大審。初,成祖定熱審之例,英宗特行朝審,至是復有大審,所矜疑放遣,嘗倍於熱審時。內閣之與審也,自憲宗罷,至隆慶元年,高拱復行之。故事,朝審吏部尚書秉筆,時拱適兼吏部故也。至萬曆二十六年朝審,吏部尚書缺,以戶部尚書楊俊民主之。三十二年復缺,以戶部尚書趙世卿主之。崇禎十五年,命首輔周延儒同三法司清理淹獄,蓋出於特旨云。大審,自萬曆二十九年曠不舉,四十四年乃行之。

熱審始永樂二年,止決遣輕罪,命出獄聽候而已。尋并寬及徒流以下。宣德二年五、六、七月,連論三法司錄上繫囚罪狀,凡決遣二千八百餘人。七年二月,親閱法司所進繫囚罪狀,決遣千餘人,減等輸納,春審自此始。六月,又以炎暑,命自實犯死罪外,悉早發遣,且馳諭中外刑獄悉如之。成化時,熱審始有重罪矜疑、輕罪減等、枷號疏放諸例。正德元年,掌大理寺工部尚書楊守隨言:「每歲熱審事例,行於北京而不行於南京。五年一審錄事例,行於在京,而略於在外。今宜通行南京,凡審囚,三法司皆會審,其在外審錄,亦依此例。」詔可。嘉靖十年,令每年熱審并五年審錄之期,雜犯死罪、準徒五年者,皆減一年。二十三年,刑科羅崇奎言:「五、六月間,笞罪應釋放、徒罪應減等者,亦宜如成化時欽恤枷號例,暫與蠲免,至六月終止。南法司亦如之。」報可。隆慶五年,令贓銀止十兩以上、監久產絕、或身故者,熱審免追,釋其家屬。萬歷三十九年,方大暑省刑,而熱審矜疑疏未下。刑部侍郎沈應文以獄囚久滯,乞暫豁矜疑者。未報。明日,法司盡按囚籍軍徒杖罪未結者五十三人,發大興、宛平二縣監候,乃以疏聞。神宗亦不罪也。舊例,每年熱審自小滿後十餘日,司禮監傳旨下刑部,即會同都察院、錦衣衛題請,通行南京法司,一體審擬具奏。京師自命下之日至六月終止。南京自部移至日為始,亦滿兩月而止。四十四年不舉行。明年,又逾兩月,命未下,會暑雨,獄中多疫。言官以熱審愆期、朝審不行、詔獄理刑無人三事交章上請。又請釋楚宗英嫶、蘊鈁等五十餘人,罣誤知縣滿朝薦,同知王邦才、卞孔時等。皆不報。崇禎十五年四月亢旱,下詔清獄。中允黃道周言:「中外齋宿為百姓請命,而五日之內繫兩尚書,不聞有抗疏爭者,尚足回天意乎?」兩尚書謂李日宣、陳新甲也。帝方重怒二人,不能從。

歷朝無寒審之制,崇禎十年,以代州知州郭正中疏及寒審,命所司求故事。尚書鄭三俊乃引數事以奏,言:「謹按洪武二十三年十二月癸未,太祖諭刑部尚書楊靖,『自今惟犯十惡并殺人者論死,餘死罪皆令輸粟北邊以自贖』。永樂四年十一月,法司進月繫囚數,凡數百人,大辟僅十之一。成祖諭呂震曰:『此等既非死罪,而久繫不決,天氣冱寒,必有聽其冤死者。』凡雜犯死罪下約二百,悉準贖發遣。九年十一月,刑科曹潤等言:『昔以天寒,審釋輕囚。今囚或淹一年以上,且一月間瘐死者九百三十餘人,獄吏之毒所不忍言。』成祖召法司切責,遂詔:『徒流以下三日內決放,重罪當繫者恤之,無令死於饑寒。』十二年十一月,復令以疑獄名上,親閱之。宣德四年十月,以皇太子千秋節,減雜犯死罪以下,宥笞杖及枷鐐者。嗣後,世宗、神宗或以災異修刑,或以覃恩布德。寒審雖無近例,而先朝寬大,皆所宜取法者。」奏上,帝納其言。然永樂十一年十月,遣副都御史李慶齎璽書,命皇太子錄南京囚,贖雜犯死罪以下。宣德四年冬,以天氣冱寒,敕南北刑官悉錄繫囚以聞,不分輕重。因謂夏原吉等曰:「堯、舜之世,民不犯法,成、康之時,刑措不用,皆君臣同德所致。朕德薄,卿等其勉力匡扶,庶無愧古人。」此寒審最著者,三俊亦不暇詳也。

在外恤刑會審之例,定於成化時。初,太祖患刑獄壅蔽,分遣御史林願、石恆等治各道囚,而敕諭之。宣宗夜讀《周官·立政》:「式敬爾由獄,以長我王國。」慨然興歎,以為立國基命在於此。乃敕三法司:「朕體上帝好生之心,惟刑是恤。令爾等詳覆天下重獄,而犯者遠在千萬里外,需次當決,豈能無冤?」因遣官審錄之。正統六年四月,以災異頻見,敕遣三法司官詳審天下疑獄。於是御史張驥、刑部郎林厚、大理寺正李從智等十三人同奉敕往,而復以刑部侍郎何文淵、大理卿王文、巡撫侍郎周忱、刑科給事中郭瑾審兩京刑獄,亦賜敕。後評事馬豫言:「臣奉敕審刑,竊見各處捉獲強盜,多因仇人指攀,拷掠成獄,不待詳報,死傷者甚多。今後宜勿聽妄指,果有贓證,御史、按察司會審,方許論決。若未審錄有傷死者,毋得準例陞賞。」是年,出死囚以下無數。九年,山東副使王裕言:「囚獄當會審,而御史及三司官或踰年一會,囚多瘐死。往者常遣御史會按察司詳審,釋遣甚眾。今莫若罷會審之例,而行詳審之法,敕遣按察司官一員,專審諸獄。」部持舊制不可廢。帝命審例仍舊,復如詳審例,選按察司官一員與巡按御史同審。失出者姑勿問,涉贓私者究如律。成化元年,南京戶部侍郎陳翼因災異復請如正統例。部議以諸方多事,不行。八年,乃分遣刑部郎中劉秩等十四人會巡按御史及三司官審錄,敕書鄭重遣之。十二年,大學士商輅言:「自八年遣官後,五年於茲,乞更如例行。」帝從其請。至十七年,定在京五年大審。即於是年遣部寺官分行天下,會同巡按御史行事。於是恤刑者至,則多所放遣。嘉靖四十三年,定坐贓不及百兩,產絕者免監追。萬曆四年,敕雜犯死罪準徒五年者,并兩犯徒律應總徒四年者,各減一年,其他徒流等罪俱減等。皆由恤刑者奏定。所生全者益多矣。初,正統十一年,遣刑部郎中郭恂、員外郎陸瑜審南、北直隸獄囚,文職五品以下有罪,許執問。嘉靖間制,審錄官一省事竣,總計前後所奏,依準改駁多寡,通行考核。改駁數多者聽劾。故恤刑之權重,而責亦匪輕。此中外法司審錄之大較也。

凡刑部問發罪囚,所司通將所問囚數,不問罪名輕重,分南北人各若干,送山東司,呈堂奏聞,謂之歲報。每月以見監罪囚奏聞,謂之月報。其做工、運炭等項,每五日開送工科,填寫精微冊,月終分六科輪報之。凡法官治囚,皆有成法,提人勘事,必齎精微批文。京外官五品以上有犯必奏聞請旨,不得擅勾問罪。在八議者,實封以聞。民間獄訟,非通政司轉達於部,刑部不得聽理。誣告者反坐,越訴者笞,擊登聞鼓不實者杖。訐告問官,必核實乃逮問。至罪囚打斷起發有定期,刑具有定器,停刑有定月日,檢驗屍傷有定法,恤囚有定規,籍沒亦有定物,惟復仇者無明文。

弘治元年,刑部尚書何喬新言:「舊制提人,所在官司必驗精微批文,與符號相合,然後發遣。此祖宗杜漸防微深意也。近者中外提人,止憑駕帖,既不用符,真偽莫辨,姦人矯命,何以拒之?請給批文如故。」帝曰:「此祖宗舊例,不可廢。」命復行之。然旗校提人,率齎駕帖。嘉靖元年,錦衣衛千戶白壽等齎駕帖詣科,給事中劉濟謂當以御批原本送科,使知其事。兩人相爭並列,上命檢成、弘事例以聞。濟復言,自天順時例即如此。帝入壽言,責濟以狀對,亦無以罪也。天啟時,魏忠賢用駕帖提周順昌諸人,竟激蘇州之變。兩畿決囚,亦必驗精微批。嘉靖二十一年,恤刑主事戴楩、吳元璧、呂顒等行急失與內號相驗,比至,與原給外號不合,為巡按御史所糾,納贖還職。

成化時,六品以下官有罪,巡按御史輒令府官提問。陜西巡撫項忠言:「祖制,京外五品以上官有犯奏聞,不得擅勾問。今巡按輒提問六品官,甚乖律意,當聞於朝,命御史、按察司提問為是。」乃下部議,從之。凡罪在八議者,實封奏聞請旨,惟十惡不用此例。所屬官為上司非理凌虐,亦聽實封徑奏。軍官犯罪,都督府請旨。諸司事涉軍官及呈告軍官不法者,俱密以實封奏,無得擅勾問。嘉靖中,順天巡按御史鄭存仁檄府縣,凡法司有所追取,不得輒發。尚書鄭曉考故事,民間詞訟非自通政司轉達,不得聽。而諸司有應問罪人,必送刑部,各不相侵。曉乃言:「刑部追取人,府縣不當卻。存仁違制,宜罪。」存仁亦執自下而上之律,論曉欺罔。乃命在外者屬有司,在京者屬刑部。然自曉去位,民間詞訟,五城御史輒受之,不復遵祖制矣。

洪武時,有告謀反者勘問不實,刑部言當抵罪。帝以問秦裕伯。對曰:「元時若此者罪止杖一百,蓋以開來告之路也。」帝曰:「姦徒不抵,善人被誣者多矣。自今告謀反不實者,抵罪。」學正孫詢訐稅使孫必貴為胡黨,又訐元參政黎銘常自稱老豪傑,謗訕朝廷。帝以告訐非儒者所為,置不問。永樂間定制,誣三四人杖徒,五六人流三千里,十人以上者凌遲,家屬徙化外。

洪武末年,小民多越訴京師,及按其事,往往不實,乃嚴越訴之禁。命老人理一鄉詞訟,會里胥決之,事重者始白於官,然卒不能止,越訴者日多。乃用重法,戍之邊。宣德時,越訴得實者免罪,不實仍戍邊。景泰中,不問虛實,皆發口外充軍,後不以為例也。

登聞鼓,洪武元年置於午門外,一御史日監之,非大冤及機密重情不得擊,擊即引奏。後移置長安右門外,六科、錦衣衛輪收以聞。旨下,校尉領駕帖,送所司問理,蒙蔽阻遏者罪。龍江衛吏有過,罰令書寫,值母喪,乞守制,吏部尚書詹徽不聽,擊鼓訴冤。太祖切責徽,使吏終喪。永樂元年,縣令以贓戍,擊鼓陳狀。帝為下法司,其人言實受贓,年老昏眊所致,惟上哀憫。帝以其歸誠,屈法宥之。宣德時,直登聞鼓給事林富言:「重囚二十七人,以姦盜當決,擊鼓訴冤,煩瀆不可宥。」帝曰:「登聞鼓之設,正以達下情,何謂煩惱?自後凡擊鼓訴冤,阻遏者罪。」

凡訐告原問官司者,成化間定議,核究得實,然後逮問。弘治時,南京御史王良臣按指揮周愷等怙勢黜貨,愷等遂訐良臣。詔下南京法司逮繫會鞫。侍郎楊守隨言:「此與舊章不合。請自今以後,官吏軍民奏訴,牽緣別事,摭拾原問官者,立案不行。所奏事仍令問結,虛詐者擬罪,原問官枉斷者亦罪。」乃下其議於三法司。法司覆奏如所請,從之。洪武二十六年以前,刑部令主事廳會御史、五軍斷事司、大理寺、五城兵馬指揮使官,打斷罪囚。二十九年,并差錦衣衛官。其後惟主事會御史,將笞杖罪於打斷廳決訖,附卷,奉旨者次日覆命。萬曆中,刑部尚書孫丕揚言:「折獄之不速,由文移牽制故耳。議斷既成,部、寺各立長單,刑部送審掛號,次日即送大理。大理審允,次日即還本部。參差者究處,庶事體可一。至於打斷相驗,令御史三、六、九日遵例會同,餘日止會寺官以速遣。徒流以上,部、寺詳鞫,笞杖小罪,聽堂部處分。」命如議行。

凡獄囚已審錄,應決斷者限三日,應起發者限十日,逾銀計日以笞。囚淹滯至死者罪徒,此舊例也。嘉靖六年,給事中周瑯言:「比者獄吏苛刻,犯無輕重,概加幽繫,案無新故,動引歲時。意喻色授之間,論奏未成,囚骨已糜。又況偏州下邑,督察不及,姦吏悍卒倚獄為市,或扼其飲食以困之,或徙之穢溷以苦之,備諸痛楚,十不一生。臣觀律令所載,凡逮繫囚犯,老疾必散收,輕重以類分,枷杻薦席必以時飭,涼漿暖匣必以時備,無家者給之衣服,有疾者予之醫藥,淹禁有科,疏決有詔。此祖宗良法美意,宜敕臣下同為奉行。凡逮繫日月并已竟、未竟、疾病、死亡者,各載文冊,申報長吏,較其結竟之遲速,病故之多寡,以為功罪而黜陟之。」帝深然其言,且命中外有用法深刻,致戕民命者,即斥為民,雖才守可觀,不得推薦。

凡內外問刑官,惟死罪並竊盜重犯,始用拷訊,餘止鞭撲常刑。酷吏輒用挺棍、夾棍、腦箍、烙鐵及一封書、鼠彈箏、攔馬棍、燕兒飛,或灌鼻、釘指,用徑寸懶桿、不去棱節竹片,或鞭脊背、兩踝致傷以上者,俱奏請,罪至充軍。

停刑之月,自立春以後,至春分以前。停刑之日,初一、初八、十四、十五、十八、二十三、二十四、二十八、二十九、三十,凡十日。檢驗屍傷,照磨司取部印屍圖一幅,委五城兵馬司如法檢驗,府則通判、推官,州縣則長官親檢,毋得委下僚。

獄囚貧不自給者,洪武十五年定制,人給米日一升。二十四年革去。正統二年,以侍郎何文淵言,詔如舊,且令有贓罰敝衣得分給。成化十二年,令有司買藥餌送部,又廣設惠民藥局,療治囚人。至正德十四年,囚犯煤、油、藥料,皆設額銀定數。嘉靖六年,以運炭等有力罪囚,折色糴米,上本部倉,每年約五百石,乃停收。歲冬給綿衣褲各一事,提牢主事驗給之。

犯罪籍沒者,洪武元年定制,自反叛外,其餘罪犯止沒田產孳畜。二十一年,詔謀逆姦黨及造偽鈔者,沒貲產丁口,以農器耕牛給還之。凡應合鈔劄者,曰姦黨,曰謀反大逆,曰姦黨惡,曰造偽鈔,曰殺一家三人,曰採生拆割人為首。其《大誥》所定十條,後未嘗用也。復仇,惟祖父被毆條見之,曰:「祖父母、父母為人所殺,而子孫擅殺行兇人者,杖六十。其即時殺死者勿論。其餘親屬人等被人殺而擅殺之者,杖一百。」按律罪人應死,已就拘執,其捕者擅殺之,罪亦止此。則所謂家屬人等,自包兄弟在內,其例可類推也。

凡決囚,每歲朝審畢,法司以死罪請旨,刑科三覆奏,得旨行刑。在外者奏決單於冬至前,會審決之。正統元年,令重囚三覆奏畢,仍請駕帖,付錦衣衛監刑官,領校尉詣法司,取囚赴市。又制,臨決囚有訴冤者,直登聞鼓給事中取狀封進,仍批校尉手,馳赴市曹,暫停刑。嘉靖元年,給事中劉濟等以囚廖鵬父子及王欽、陶傑等頗有內援,懼上意不決,乃言:「往歲三覆奏畢,待駕帖則已日午,鼓下仍受訴詞,得報且及未申時,及再請始刑,時已過酉,大非刑人於市與眾棄之之意。請自今決囚,在未前畢事。」從之。七年定議,重囚有冤,家屬於臨決前一日撾鼓,翼日午前下,過午行刑,不覆奏。南京決囚,無刑科覆奏例。弘治十八年,南刑部奏決不待時者三人,大理寺已審允,下法司議,謂:「在京重囚,間有決不待時者,審允奏請,至刑科三覆奏,或蒙恩仍監候會審。南京無覆奏例,乞俟秋後審竟,類奏定奪。如有巨憝難依常例者,更具奏處決,著為令。」詔可。各省決囚,永樂元年,定制,死囚百人以上者,差御史審決。弘治十三年,定歲差審決重囚官,俱以霜降後至,限期復命。

凡有大慶及災荒皆赦,然有常赦,有不赦,有特赦。十惡及故犯者不赦。律文曰:「赦出臨時定罪名,特免或降減從輕者,不在此限。」十惡中,不睦又在會赦原宥之例,此則不赦者亦得原。若傳旨肆赦,不別定罪名者,則仍依常赦不原之律。自仁宗立赦條三十五,皆楊士奇代草,盡除永樂年間敝政,歷代因之。凡先朝不便於民者,皆援遺詔或登極詔革除之。凡以赦前事告言人罪者,即坐以所告者罪。弘治元年,民呂梁山等四人坐竊盜殺人死,遇赦,都御史馬文升請宥死戍邊,帝特命依律斬之。世宗雖屢停刑,尤慎無赦。廷臣屢援赦令,欲宥大禮大獄暨建言諸臣,益持不允。及嘉靖十六年,同知姜輅酷殺平民,都御史王廷相奏當發口外,乃特命如詔書宥免,而以違詔責廷相等。四十一年,三殿成,群臣請頒赦。帝曰:「赦乃小人之幸。」不允。穆宗登極覃恩,雖徒流人犯已至配所者,皆許放還,蓋為遷謫諸臣地也。

有明一代刑法大概。太祖開國之初,懲元季貪冒,重繩贓吏,揭諸司犯法者於申明亭以示戒。又命刑部,凡官吏有犯,宥罪復職,書過榜其門,使自省。不悛,論如律。累頒犯諭、戒諭、榜諭,悉象以刑,誥示天下。及十八年《大誥》成,序之曰:「諸司敢不急公而務私者,必窮搜其原而罪之。」凡三《誥》所列凌遲、梟示、種誅者,無慮千百,棄市以下萬數。貴溪儒士夏伯啟叔姪斷指不仕,蘇州人才姚潤、王謨被徵不至,皆誅而籍其家。「寰中士夫不為君用」之科所由設也。其《三編》稍寬容,然所記進士監生罪名,自一犯至四犯者猶三百六十四人。幸不死還職,率戴斬罪治事。其推原中外貪墨所起,以六曹為罪魁,郭桓為誅首。郭桓者,戶部侍郎也。帝疑北平二司官吏李彧、趙全德等與桓為姦利,自六部左右侍郎下皆死,贓七百萬,詞連直省諸官吏,繫死者數萬人。核贓所寄借遍天下,民中人之家大抵皆破。時咸歸謗御史餘敏、丁廷舉。或以為言,帝乃手詔列桓等罪,而論右審刑吳庸等極刑,以厭天下心,言:「朕詔有司除姦,顧復生姦擾吾民,今後有如此者,遇赦不宥。」先是,十五年空印事發。每歲布政司、府州縣吏詣戶部核錢糧、軍需諸事,以道遠,預持空印文書,遇部駁即改,以為常。及是,帝疑有姦,大怒,論諸長吏死,佐貳榜百戍邊。寧海人鄭士利上書訟其冤,復杖戍之。二獄所誅殺已過當。而胡惟庸、藍玉兩獄,株連死者且四萬。

然時引大體,有所縱舍。沅陵知縣張傑當輸作,自陳母賀,當元季亂離守節,今年老失養。帝謂可勵俗,特赦之,秩傑,令終養。給事中彭與民坐繫,其父為上表訴哀。立釋之,且免同繫十七人。有死囚妻妾訴夫冤,法司請黥之。帝以婦為夫訴,職也,不罪。都察院當囚死者二十四人,命群臣鞫,有冤者,減數人死。真州民十八人謀不軌,戮之,而釋其母子當連坐者。所用深文吏開濟、詹徽、陳寧、陶凱輩,後率以罪誅之。亦數宣仁言,不欲純任刑罰。嘗行郊壇,皇太子從,指道旁荊楚曰:「古用此為撲刑,取能去風,雖寒不傷也。」尚書開濟議法密,諭之曰:「竭澤而漁,害及鯤鮞,焚林而田,禍及麛鷇。法太巧密,民何以自全?」濟慚謝。參政楊憲欲重法,帝曰:「求生於重典,猶索魚於釜,得活難矣。」御史中丞陳寧曰:「法重則人不輕犯,吏察則下無遁情。」太祖曰:「不然。古人制刑以防惡衛善,故唐、虞畫衣冠、異章服以為戮,而民不犯。秦有鑿顛抽脅之刑、參夷之誅,而囹圄成市,天下怨叛。未聞用商、韓之法,可致堯、舜之治也。」寧慚而退。又嘗謂尚書劉惟謙曰:「仁義者,養民之膏粱也;刑罰者,懲惡之藥石也。舍仁義而專用刑罰,是以藥石養人,豈得謂善治乎?」蓋太祖用重典以懲一時,而酌中制以垂後世,故猛烈之治,寬仁之詔,相輔而行,未嘗偏廢也。建文帝繼體守文,專欲以仁義化民。元年刑部報囚,減太祖時十三矣。

成祖起靖難之師,悉指忠臣為姦黨,甚者加族誅、掘塚,妻女發浣衣局、教坊司,親黨謫戍者至隆、萬間猶勾伍不絕也。抗違者既盡殺戮,懼人竊議之,疾誹謗特甚。山陽民丁鈺訐其鄉誹謗,罪數十人。法司迎上旨,言鈺才可用,立命為刑科給事中。永樂十七年,復申其禁。而陳瑛、呂震、紀綱輩先後用事,專以刻深固寵。於是蕭議、周新、解縉等多無罪死。然帝心知苛法之非,間示寬大。千戶某灌桐油皮鞭中以決人,刑部當以杖,命并罷其職。法司奏冒支官糧者,命即戮之,刑部為覆奏。帝曰:「此朕一時之怒,過矣,其依律。自今犯罪皆五覆奏。」

至仁宗性甚仁恕,甫即位,謂金純、劉觀曰:「卿等皆國大臣,如朕處法失中,須更執奏,朕不難從善也。」因召學士楊士奇、楊榮、金幼孜至榻前,諭曰:「比年法司之濫,朕豈不知。其所擬大逆不道,往往出於文致,先帝數切戒之。故死刑必四五覆奏,而法司略不加意,甘為酷吏而不愧。自今審重囚,卿三人必往同讞,有冤抑者,雖細故必以聞。」洪熙改元,二月諭都御史劉觀、大理卿虞謙曰:「往者法司以誣陷為功,人或片言及國事,輒論誹謗,身家破滅,莫復辨理。今數月間,此風又萌。夫治道所急者求言,所患者以言為諱,奈何禁誹謗哉?」因顧士奇等曰:「此事必以詔書行之。」於是士奇承旨,載帝言於己丑詔書云:「若朕一時過於嫉惡,律外用籍沒及凌遲之刑者,法司再三執奏,三奏不允至五,五奏不允,同三公及大臣執奏,必允乃已,永為定制。文武諸司亦毋得暴酷用鞭背等刑,及擅用宮刑絕人嗣續。有自宮者以不孝論。除謀反及大逆者,餘犯止坐本身,毋一切用連坐法。告誹謗者勿治。」在位未一年,仁恩該洽矣。

宣宗承之,益多惠政。宣德元年,大理寺駁正猗氏民妻王骨都殺夫之冤,帝切責刑官,尚書金純等謝罪,乃已。義勇軍士閻群兒等九人被誣為盜,當斬,家人擊登聞鼓訴冤。覆按實不為盜。命釋群兒等,而切責都御史劉觀。其後每遇奏囚,色慘然,御膳為廢。或以手撤其牘,謂左右曰:「說與刑官少緩之。」一日,御文華殿與群臣論古肉刑,侍臣對:「漢除肉刑,人遂輕犯法。」帝曰:「此自由教化,豈關肉刑之有無。舜法有流宥金贖,而四凶之罪止於竄殛。可見當時被肉刑者,必皆重罪,不濫及也。況漢承秦敝,挾書有律,若概用肉刑,受傷者必多矣。」明年,著《帝訓》五十五篇,其一恤刑也。武進伯朱冕言:「比遣舍人林寬等送囚百十七人戍邊,到者僅五十人,餘皆道死。」帝怒,命法司窮治之。帝寬詔歲下,閱囚屢決遣,有至三千人者。諭刑官曰:「吾慮其瘐死,故寬貸之,非常制也。」是時,官吏納米百石若五十石,得贖雜犯死罪,軍民減十之二。諸邊衛十二石,遼東二十石,於例為太輕,然獨嚴贓吏之罰。命文職犯贓者俱依律科斷。由是用法輕,而貪墨之風亦不甚恣,然明制重朋比之誅。都御史夏迪催糧常州,御史何楚英誣以受金。諸司懼罪,明知其冤,不敢白,迪竟充驛夫憤死。以帝之寬仁,而大臣有冤死者,此立法之弊也。

英宗以後,仁、宣之政衰。正統初,三楊當國,猶恪守祖法,禁內外諸司鍛煉刑獄。刑部尚書魏源以災旱上疑獄,請命各巡撫審錄。從之。無巡撫者命巡按。清軍御史、行在都察院亦以疑獄上,通審錄之。御史陳祚言:「法司論獄,多違定律,專務刻深。如戶部侍郎吳璽舉淫行主事吳軏,宜坐貢舉非其人罪,乃加以奏事有規避律斬。及軏自經死,獄官卒之罪,明有遞減科,乃援不應為事理重者,概杖之。夫原情以定律,祖宗防範至周,而法司乃抑輕從重至此,非所以廣聖朝之仁厚也。今後有妄援重律者,請以變亂成法罪之。」帝是其言,為申警戒。至六年,王振始亂政,數辱廷臣,刑章大紊。侍講劉球條上十事,中言:「天降災譴,多感於刑罰之不中。宜一任法司,視其徇私不當者而加以罪。雖有觸忤,如漢犯蹕盜環之事,猶當聽張釋之之執奏而從之。」帝不能用。而球即以是疏觸振怒,死於獄。然諸酷虐事,大率振為之,帝心頗寬平。十一年,大理卿俞士悅以毆鬥殺人之類百餘人聞,請宥,俱減死戍邊。景泰中,陽穀主簿馬彥斌當斬,其子震請代死。特宥彥斌,編震充邊衛軍。大理少卿薛瑄曰:「法司發擬罪囚,多加參語奏請,變亂律意。」詔法官問獄,一依律令,不許妄加參語。六年,以災異審錄中外刑獄,全活者甚眾。天順中,詔獄繁興,三法司、錦衣獄多繫囚未決,吏往往洩獄情為姦。都御史蕭維楨附會徐有貞,枉殺王文、于謙等。而刑部侍郎劉廣衡即以詐撰制文,坐有貞斬罪。其後緹騎四出,海內不安。然霜降後審錄重囚,實自天順間始。至成化初,刑部尚書陸瑜等以請,命舉行之。獄上,杖其情可矜疑者,免死發戍。列代奉行,人獲沾法外恩矣。

憲宗之即位也,敕三法司:「中外文武群臣除贓罪外,所犯罪名紀錄在官者,悉與湔滌。」其後歲以為常。十年,當決囚,冬至節近,特命過節行刑。既而給事中言,冬至後行刑非時,遂詔俟來年冬月。山西巡撫何喬新劾奏遲延獄詞僉事尚敬、劉源,因言:「凡二司不決斷詞訟者,半年之上,悉宜奏請執問。」帝曰:「刑獄重事,《周書》曰:『要囚,服念五六日至於旬時』,特為未得其情者言耳。茍得其情,即宜決斷。無罪拘幽,往往瘐死,是刑官殺之也。故律特著淹禁罪囚之條,其即以喬新所奏,通行天下。」又定制,凡盜賊贓仗未真、人命死傷未經勘驗、輒加重刑致死獄中者,審勘有無故失明白,不分軍民職官,俱視酷刑事例為民。侍郎楊宣妻悍妒,殺婢十餘人,部擬命婦合坐者律,特命決杖五十。時帝多裨政,而於刑獄尤慎之,所失惟一二事。嘗欲殺一囚,不許覆奏。御史方佑復以請,帝怒,杖謫佑。吉安知府許總有罪,中官黃高嗾法司論斬。給事中白昂以未經審錄為請,不聽,竟乘夜斬之。

孝宗初立,免應決死罪四十八人。元年,知州劉概坐妖言罪斬,以王恕爭,得長繫。末年,刑部尚書閔珪讞重獄,忤旨,久不下。帝與劉大夏語及之,對曰:「人臣執法效忠,珪所為無足異。」帝曰:「且道自古君臣曾有此事否?」對曰:「臣幼讀《孟子》,見瞽瞍殺人,皋陶執之語。珪所執,未可深責也。」帝頷之。明日疏下,遂如擬。前後所任司寇何喬新、彭韶、白昂、閔珪皆持法平者,海內翕然頌仁德焉。

正德五年會審重囚,減死者二人。時冤濫滿獄,李東陽等因風霾以為言,特許寬恤。而刑官懼觸劉瑾怒,所上止此。後磔流賊趙鐩等於市,剝為魁者六人皮。法司奏祖訓有禁,不聽。尋以皮製鞍鐙,帝每騎乘之。而廷杖直言之臣,亦武宗為甚。

世宗即位七月,因日精門災,疏理冤抑,命再問緩死者三十八人,而廖鵬、王瓛、齊佐等與焉。給事中李復禮等言:「鵬等皆江彬、錢寧之黨。王法所必誅。」乃令禁之如故。後皆次第伏法。自杖諸爭大禮者,遂痛折廷臣。六年,命張璁、桂萼、方獻夫攝三法司,變李福達之獄,欲坐馬錄以姦黨律。楊一清力爭,乃戍錄,而坐罪者四十餘人。璁等以為己功,遂請帝編《欽明大獄錄》頒示天下。是獄所坐,大抵璁三人夙嫌者。以祖宗之法,供權臣排陷,而帝不悟也。八年,京師民張福殺母,訴為張柱所殺,刑部郎中魏應召覆治得實。而帝以柱乃武宗后家僕,有意曲殺之,命侍郎許言讚盡反讞詞,而下都御史熊浹及應召於獄。其後,猜忌日甚,冤濫者多,雖間命寬恤,而意主苛刻。嘗諭輔臣:「近連歲因災異免刑,今復當刑科三覆請旨。朕思死刑重事,欲將盜陵殿等物及毆罵父母大傷倫理者取決,餘令法司再理,與卿共論,慎之慎之。」時以為得大體。越數年,大理寺奉詔讞奏獄囚應減死者。帝謂諸囚罪皆不赦,乃假借恩例縱姦壞法,黜降寺丞以下有差。自九年舉秋謝醮免決囚,自後或因祥瑞,或因郊祀大報,停刑之典每歲舉行。然屢譴怒執法官,以為不時請旨,至上迫冬至,廢義而市恩也。遂削刑部尚書吳山職,降調刑科給事中劉三畏等。中年益肆誅戮,自宰輔夏言不免。至三十七年,乃出手諭,言:「司牧者未盡得人,任情作威。湖廣幼民吳一魁二命枉刑,母又就捕,情迫無控,萬里叩閽。以此推之,冤抑者不知其幾。爾等宜亟體朕心,加意矜恤。仍通行天下,咸使喻之。」是詔也,恤恤乎有哀痛之思焉。末年,主事海瑞上書觸忤,刑部當以死。帝持其章不下,瑞得長繫。穆宗立,徐階緣帝意為遺詔,盡還諸逐臣,優恤死亡,縱釋幽繫。讀詔書者無不歎息。

萬曆初,冬月,詔停刑者三矣。五年九月,司禮太監孫得勝復傳旨:「奉聖母諭,大婚期近,命閣臣於三覆奏本,擬旨免刑。」張居正言:「祖宗舊制,凡犯死罪鞫問既明,依律棄市。嘉靖末年,世宗皇帝因齋醮,始有暫免不決之令,或間從御筆所勾,量行取決。此特近年姑息之弊,非舊制也。臣等詳閱諸囚罪狀,皆滅絕天理,敗傷彞倫,聖母獨見犯罪者身被誅戮之可憫,而不知彼所戕害者皆含冤蓄憤於幽冥之中,使不一雪其痛,怨恨之氣,上干天和,所傷必多。今不行刑,年復一年,充滿囹圄,既費關防,又乖國典,其於政體又大謬也。」給事中嚴用和等亦以為言。詔許之。十二年,御史屠叔明請釋革除忠臣外親。命自齊、黃外,方孝孺等連及者俱勘豁。帝性仁柔,而獨惡言者。自十二年至三十四年,內外官杖戍為民者至百四十人。後不復視朝,刑辟罕用,死囚屢停免去。天啟中,酷刑多,別見,不具論。

莊烈帝即位,誅魏忠賢。崇禎二年,欽定逆案凡六等,天下稱快。然是時承神宗廢弛、熹宗昏亂之後,銳意綜理,用刑頗急,大臣多下獄者矣。六年冬論囚,素服御建極殿,召閣臣商榷,而溫體仁無所平反。陜西華亭知縣徐兆麒抵任七日,城陷,坐死。帝心憫之,體仁不為救。十一年,南通政徐石麒疏救鄭三俊,因言:「皇上御極以來,諸臣麗丹書者幾千,圜扉為滿。使情法盡協,猶屬可憐,況怵惕於威嚴之下者。有將順而無挽回,有揣摩而無補救,株連蔓引,九死一生,豈聖人惟刑之恤之意哉!」帝不能納也。是年冬,以彗見,停刑。其事關封疆及錢糧剿寇者,詔刑部五日具獄。十二年,御史魏景琦論囚西市,御史高欽舜、工部郎中胡璉等十五人將斬,忽中官本清銜命馳免,因釋十一人。明日,景琦回奏,被責下錦衣獄。蓋帝以囚有聲冤者,停刑請旨,而景琦倉卒不辨,故獲罪。十四年,大學士范復粹疏請清獄,言:「獄中文武纍臣至百四十有奇,大可痛。」不報。是時國事日棘,惟用重法以繩群臣,救過不暇,而卒無救於亂亡也。


刑法有創之自明,不衷古制者,廷杖、東西廠、錦衣衛、鎮撫司獄是已。是數者,殺人至慘,而不麗於法。踵而行之內,至末造而極。舉朝野命,一聽之武夫、宦豎之手,良可歎也。

太祖常與侍臣論待大臣禮。太史令劉基曰:「古者公卿有罪,盤水加劍,詣請室自裁,未嘗輕折辱之,所以存大臣之體。」侍讀學士詹同因取《大戴禮》及賈誼疏以進,且曰:「古者刑不上大夫。以勵廉恥也。必如是,君臣恩禮始兩盡。」帝深然之。

洪武六年,工部尚書王肅坐法當笞,太祖曰:「六卿貴重,不宜以細故辱。」命以俸贖罪。後群臣罣誤,許以俸贖斯坦·沃爾弗。,始此。然永嘉侯朱亮祖父子皆鞭死,工部尚書薛祥斃杖下,故上書者以大臣當誅不宜加辱為言。廷杖之刑,亦自太祖始矣。宣德三年,怒御史嚴皚、方鼎、何傑等沈湎酒色,久不朝參,命枷以徇。自此言官有荷校者。至正統中,王振擅權,尚書劉中敷,侍郎吳璽、陳瑺,祭酒李時勉率受此辱,而殿陛行杖習為故事矣。成化十五年,汪直誣陷侍郎馬文昇、都御史牟俸等,詔責給事御史李俊、王濬輩五十六人容隱,廷杖人二十。正德十四年,以諫止南巡,廷杖舒芬、黃鞏等百四十六人,死者十一人。嘉靖三年,群臣爭大禮,廷仗豐熙等百三十四人,死者十六人。中年刑法益峻,雖大臣不免笞辱。宣大總督翟鵬、薊州巡撫朱方以撤防早,宣大總督郭宗皋、大同巡撫陳耀以寇入大同,刑部侍郎彭黯、左都御史屠僑、大理卿沈良才以議丁汝夔獄緩,戎政侍郎蔣應奎、左通政唐國相以子弟冒功,皆逮杖之。方、耀斃於杖下,而黯、僑、良才等杖畢,趣治事。公卿之辱,前此未有。又因正旦朝賀,怒六科給事中張思靜等,皆朝服予杖,天下莫不駭然。四十餘年間,杖殺朝士,倍蓰前代。萬曆五年,以爭張居正奪情,杖吳中行等五人。其後盧洪春、孟養浩、王德完輩咸被杖,多者至一百。後帝益厭言者,疏多留中,廷杖寢不用。天啟時,太監王體乾奉赦大審,重笞戚畹李承恩,以悅魏忠賢。於是萬燝、吳裕中斃於杖下,臺省力爭不得。閣臣葉向高言:「數十年不行之敝政,三見於旬日,萬萬不可再行。」忠賢乃罷廷仗,而以所欲殺者悉下鎮撫司,士大夫益無噍類矣。

南京行杖,始於成化十八年。南御史李珊等以歲祲請振。帝摘其疏中訛字,令錦衣衛詣南京午門前,人杖二十,守備太監監之。至正德間有天地,自古以固存;神鬼神帝,生天生地」,無為無形,可,南御史李熙劾貪吏觸怒劉瑾,矯旨杖三十。時南京禁衛久不行刑,選卒習數日,乃杖之,幾斃。

東廠之設,始於成祖。錦衣衛之獄,太祖嘗用之,後已禁止,其復用亦自永樂時。廠與衛相倚,故言者並稱廠衛。初,成祖起北平,刺探宮中事,多以建文帝左右為耳目。故即位後專倚宦官,立東廠於東安門北,令嬖暱者提督之,緝訪謀逆妖言大奸惡等,與錦衣衛均權勢,蓋遷都後事也。然衛指揮紀綱、門達等大幸,更迭用事,廠權不能如。至憲宗時,尚銘領東廠,又別設西廠刺事,以汪直督之,所領緹騎倍東廠。自京師及天下,旁午偵事,雖王府不免。直中廢復用,先後凡六年,冤死者相屬,勢遠出衛上。會直數出邊監軍,大學士萬安乃言:「太宗建北京,命錦衣官校緝訪,猶恐外官徇情,故設東廠,令內臣提督,行五六十年,事有定規。往者妖狐夜出,人心驚惶,感勞聖慮,添設西廠,特命直督緝,用戒不虞,所以權一時之宜,慰安人心也。向所紛擾,臣不贅言。今直鎮大同,京城眾口一辭,皆以革去西廠為便。伏望聖恩特旨革罷,官校悉回原衛,宗社幸甚。」帝從之。尚銘專用事,未幾亦黜。弘治元年,員外郎張倫請廢東廠。不報。然孝宗仁厚,廠衛無敢橫,司廠者羅祥、楊鵬,奉職而已。

正德元年,殺東廠太監王岳,命丘聚代之,又設西廠以命谷大用,皆劉瑾黨也。兩廠爭用事「安排」了世界的一切,「如果心靈是支配者,那末心靈將把,遣邏卒刺事四方。南康吳登顯等戲競渡龍舟,身死家籍。遠州僻壤,見鮮衣怒馬作京師語者,轉相避匿。有司聞風,密行賄賂。於是無賴子乘機為奸,天下皆重足立。而衛使石文義亦瑾私人,廠衛之勢合矣。瑾又改惜薪司外薪廠為辦事廠,榮府舊倉地為內辦事廠,自領之。京師謂之內行廠,雖東西廠皆在伺察中,加酷烈焉。且創例,罪無輕重皆決杖,永遠戍邊,或枷項發遣。枷重至百五十斤,不數日輒死。尚寶卿顧璿、副使姚祥、工部郎張瑋、御史王時中輩並不免,瀕死而後謫戍。御史柴文顯、汪澄以微罪至凌遲。官吏軍民非法死者數千。瑾誅,西廠、內行廠俱革,獨東廠如故。張銳領之,與衛使錢寧並以輯事恣羅織。廠衛之稱由此著也。

嘉靖二年,東廠芮景賢任千戶陶淳,多所誣陷。給事中劉最執奏,謫判廣德州。御史黃德用使乘傳往。會有顏如環者同行,以黃袱裹裝。景賢即奏,逮下獄,最等編戍有差。給事中劉濟言:「最罪不至戍。且緝執於宦寺之門,鍛煉於武夫之手,裁決於內降之旨,何以示天下?」不報。是時盡罷天下鎮守太監,而大臣狃故事,謂東廠祖宗所設,不可廢,不知非太祖制也。然世宗馭中官嚴,不敢恣,廠權不及衛使陸炳遠矣。

萬曆初,馮保以司禮兼廠事,建廠東上北門之北,曰內廠,而以初建者為外廠。保與張居正興王大臣獄出一理,理又同出一原,但由於事物所居位置不同,理的體,欲族高拱,衛使朱希孝力持之,拱得無罪,衛猶不大附廠也。中年,礦稅使數出為害,而東廠張誠、孫暹、陳矩皆恬靜。矩治妖書獄,無株濫,時頗稱之。會帝亦無意刻核,刑罰用稀,廠衛獄中至生青草。及天啟時,魏忠賢以秉筆領廠事,用衛使田爾耕、鎮撫許顯純之徒,專以酷虐鉗中外,而廠衛之毒極矣。

凡中官掌司禮監印者,其屬稱之曰宗主,而督東廠者曰督主。東廠之屬無專官,掌刑千戶一,理刑百戶一有精神上的進取的沖動,失去了革命的精神。,亦謂之貼刑,皆衛官。其隸役悉取給於衛,最輕黠獧巧者乃撥充之。役長曰檔頭,帽上銳,衣青素衣旋褶,繫小絳,白皮靴,專主伺察。其下番子數人為幹事。京師亡命,誆財挾仇,視幹事者為窟穴。得一陰事,由之以密白於檔頭,檔頭視其事大小,先予之金。事曰起數,金曰買起數。既得事,帥番子至所犯家,左右坐曰打樁。番子即突入執訊之。無有左證符牒,賄如數,徑去。少不如意,手旁治之,名曰乾醡酒,亦曰搬罾兒,痛楚十倍官刑。且授意使牽有力者,有力者予多金,即無事。或靳不予,予不足,立聞上,下鎮撫司獄,立死矣。每月旦,廠役數百人,掣簽庭中,分瞰官府。其視中府諸處會審大獄、北鎮撫司考訊重犯者曰聽記。他官府及各城門訪緝曰坐記。某官行某事,某城門得某奸,胥吏疏白坐記者上之廠曰打事件。至東華門,雖夤夜,投隙中以入,即屏人達至尊。以故事無大小,天子皆得聞之。家人米鹽猥事,宮中或傳為笑謔,上下惴惴無不畏打事件者。衛之法亦如廠。然須具疏,乃得上聞,以此其勢不及廠遠甚。有四人夜飲密室,一人酒酣,謾罵魏忠賢,其三人噤不敢出聲。罵未訖,番人攝四人至忠賢所,即磔罵者,而勞三人金。三人者魄喪不敢動。

莊烈帝即位,忠賢伏誅,而王體乾、王永祚、鄭之惠、李承芳、曹化淳、王德化、王之心、王化民、齊本正等相繼領廠事,告密之風未嘗息也。之心、化淳敘緝奸功,廕弟侄錦衣衛百戶,而德化及東廠理刑吳道正等偵閣臣薛國觀陰事,國觀由此死。時衛使慴廠威已久,大抵俯首為所用。崇禎十五年,御史楊仁愿言:「高皇帝設官,無所謂緝事衙門者。臣下不法,言官直糾之,無陰訐也。後以肅清輦轂,乃建東廠。臣待罪南城,所閱詞訟,多以假番故訴冤。夫假稱東廠,害猶如此,況其真乎?此由積重之勢然也。所謂積重之勢者,功令比較事件,番役每懸價以買事件,受買者至誘人為奸盜而賣之,番役不問其從來,誘者分利去矣。挾忿首告,誣以重法,挾者志無不逞矣。伏願寬東廠事件,而後東廠之比較可緩,東廠之比較緩,而後番役之買事件與賣事件者俱可息,積重之勢庶幾可稍輕。」後復切言緹騎不當遣。帝為諭東廠,言所緝止謀逆亂倫,其作奸犯科,自有司存,不宜緝,并戒錦衣校尉之橫索者。然帝倚廠衛益甚,至國亡乃已。

錦衣衛獄者,世所稱詔獄也。古者獄訟掌於司寇而已。漢武帝始置詔獄二十六所,歷代因革不常。五代唐明宗設侍衛親軍馬步軍都指揮使,乃天子自將之名。至漢有侍衛司獄,凡大事皆決焉。明錦衣衛獄近之,幽系慘酷,害無甚於此者。

太祖時,天下重罪逮至京者,收繫獄中,數更大獄,多使斷治,所誅殺為多。後悉焚衛刑具,以囚送刑部審理。二十六年,申明其禁,詔內外獄毋得上錦衣衛,大小咸經法司。成祖幸紀綱,令治錦衣親兵,復典詔獄。綱遂用其黨莊敬、袁江、王謙、李春等,緣借作奸數百千端。久之,族綱,而錦衣典詔獄如故,廢洪武詔不用矣。英宗初,理衛事者劉勉、徐恭皆謹飭。而王振用指揮馬順流毒天下,枷李時勉,殺劉球,皆順為之。景帝初,有言官校緝事之弊者,帝切責其長,令所緝送法司,有誣罔者重罪。英宗復辟,召李賢,屏左右,問時政得失。賢因極論官校提人之害。帝然其言,陰察皆實,乃召其長,戒之。已緝弋陽王敗倫事虛,復申戒之。而是時指揮門達、鎮撫逯杲怙寵,賢亦為羅織者數矣。達遣旗校四出,杲又立程督併,以獲多為主。千戶黃麟之廣西,執御史吳禎至,索獄具二百餘副,天下朝覲官陷罪者甚眾。杲死,達兼治鎮撫司。構指揮使袁彬,繫訊之,五毒更下,僅免。朝官楊璡、李蕃、韓祺、李觀、包瑛、張祚、程萬鐘輩皆鋃鐺就逮,冤號道路者不可勝記。蓋自紀綱誅,其徒稍戢。至正統時復張,天順之末禍益熾,朝野相顧不自保。李賢雖極言之,不能救也。

鎮撫司職理獄訟,初止立一司,與外衛等。洪武十五年添設北司,而以軍匠諸職掌屬之南鎮撫司,於是北司專理詔獄。然大獄經訊,即送法司擬罪,未嘗具獄詞。成化元年,始令覆奏用參語,法司益掣肘。十四年,增鑄北司印信,一切刑獄毋關白本衛,即衛所行下者,亦徑自上請可否,衛使毋得與聞。故鎮撫職卑而其權日重。初,衛獄附衛治,至門達掌問刑,又於城西設獄舍,拘繫狼籍。達敗,用御史呂洪言,毀之。成化十年,都御史李賓言:「錦衣鎮撫司累獲妖書圖本,皆誕妄不經之言。小民無知,輒被幻惑。乞備錄其舊名目,榜示天下,使知畏避,免陷刑辟。」報可。緝事者誣告猶不止。十三年,捕寧晉人王鳳等,誣與瞽者受妖書,署偽職,并誣其鄉官知縣薛方、通判曹鼎與通謀,發卒圍其家,手旁掠誣伏。方、鼎家人數聲冤,下法司驗得實,坐妄報妖言,當斬。帝戒以不得戕害無辜而已,不能罪也。是年,令錦衣衛副千戶吳綬於鎮撫司同問刑。綬性狡險,附汪直以進。後知公議不容,凡文臣非罪下獄者,不復加箠楚,忤直意,黜去。是時惟衛使朱驥持法平,治妖人獄無冤者。詔獄下所司,獨用小杖,嘗命中使詰責,不為改。世以是稱之。弘治十三年,詔法司:「凡廠衛所送囚犯,從公審究,有枉即與辨理,勿拘成案。」正德時,衛使石文義與張採表裏作威福,時稱為劉瑾左右翼。然文義常侍瑾,不治事,治事者高得林。瑾誅,文義伏誅,得林亦罷。其後錢寧管事,復大恣,以叛誅。

世宗立,革錦衣傳奉官十六,汰旗校十五,復諭緝事官校,惟察不軌、妖言、人命、強盜重事此,和包括馬克,他詞訟及在外州縣事,毋得與。未幾,事多下鎮撫,鎮撫結內侍,多巧中。會太監崔文奸利事發,下刑部,尋以中旨送鎮撫司。尚書林俊言:「祖宗朝以刑獄付法司,事無大小,皆聽平鞫。自劉瑾、錢寧用事,專任鎮撫司,文致冤獄,法紀大壞。更化善治在今日,不宜復以小事撓法。」不聽。俊復言:「此途一開,恐後有重情,即夤緣內降以圖免,實長亂階。」御史曹懷亦諫曰:「朝廷專任一鎮撫,法司可以空曹,刑官為冗員矣。」帝俱不聽。六年,侍郎張璁等言:「祖宗設三法司以糾官邪,平獄訟,設東廠、錦衣衛以緝盜賊,詰奸宄。自今貪官冤獄仍責法司,其有徇情曲法,乃聽廠衛覺察。盜賊奸宄,仍責廠衛,亦必送法司擬罪。」詔如議行。然官校提人恣如故。給事中蔡經等論其害,願罷勿遣。尚書胡世寧請從其議。詹事霍韜亦言:「刑獄付三法司足矣,錦衣衛復橫撓之。昔漢光武尚名節,宋太祖刑法不加衣冠,其後忠義之徒爭死效節。夫士大夫有罪下刑曹,辱矣。有重罪,廢之、誅之可也,乃使官校眾執之,脫冠裳,就桎梏。朝列清班,暮幽犴獄,剛心壯氣,銷折殆盡。及覆案非罪,即冠帶立朝班,武夫捍卒指目之曰:『某,吾辱之,某,吾繫執之。』小人無所忌憚,君子遂致易行。此豪傑所以興山林之思,而變故罕仗節之士也。願自今東廠勿與朝儀,錦衣衛勿典刑獄。士大夫罪謫廢誅,勿加笞杖鎖梏,以養廉恥,振人心,勵士節。」帝以韜出位妄言,不納。祖制,凡朝會,廠衛率屬及校尉五百名,列侍奉天門下糾儀。凡失儀者,即褫衣冠,執下鎮撫司獄,杖之乃免,故韜言及之。迨萬曆時,失儀者始不付獄,罰俸而已。世宗銜張鶴齡、延齡,奸人劉東山等乃誣二人毒魘咒詛。帝大怒,下詔獄,東山因株引素所不快者。衛使王佐探得其情,論以誣罔法反坐。佐乃枷東山等闕門外,不及旬悉死,人以佐比牟斌。牟斌者,弘治中指揮也。李夢陽論延齡兄弟不法事,下獄,斌傅輕比,得不死云。世宗中年,衛使陸炳為忮,與嚴嵩比,而傾夏言。然帝數興大獄,而炳多保全之,故士大夫不疾炳。

萬歷中,建言及忤礦稅榼者,輒下詔獄。刑科給事中楊應文言:「監司守令及齊民被逮者百五十餘人,雖已打問,未送法司《莊子·知北游》:「古之人,外化而內不化,今之人,內化而,獄禁森嚴,水火不入,疫癘之氣,充斥囹圄。」衛使駱思恭亦言:「熱審歲舉,俱在小滿前,今二年不行。鎮撫司監犯且二百,多拋瓦聲冤。」鎮撫司陸逵亦言:「獄囚怨恨,有持刀斷指者。」俱不報。然是時,告訐風衰,大臣被錄者寡。其末年,稍寬逮繫諸臣,而錦衣獄漸清矣。

田爾耕、許顯純在熹宗時為魏忠賢義子,其黨孫雲鶴、楊寰、崔應元佐之,拷楊漣、左光斗輩,坐贓比較,立限嚴督之。兩日為一限,輸金不中程者,受全刑。全刑者曰械,曰鐐,曰棍,曰拶,曰夾棍。五毒備具,呼BK聲沸然,血肉潰爛,宛轉求死不得。顯純叱吒自若,然必伺忠賢旨,忠賢所遣聽記者未至,不敢訊也。一夕,令諸囚分舍宿。於是獄卒曰:「今夕有當壁挺者。」壁挺,獄中言死也。明日,漣死,光斗等次第皆鎖頭拉死。每一人死,停數日,葦席裹尸出牢戶,蟲蛆腐體。獄中事秘,其家人或不知死日。莊烈帝擒戮逆黨,冤死家子弟望獄門稽顙哀號,為文以祭。帝聞之惻然。

自劉瑾創立枷,錦衣獄常用之。神宗時,御史朱應轂具言其慘,請除之。不聽。至忠賢,益為大枷對」之說。後收入《王文公集》。,又設斷脊、墜指、刺心之刑。莊烈帝問左右:「立枷何為?」王體乾對曰:「以罪巨奸大憝耳。」帝愀然曰:「雖如此,終可憫。」忠賢為頸縮。東廠之禍,至忠賢而極。然廠衛未有不相結者,獄情輕重,廠能得於內。而外廷有扞格者,衛則東西兩司房訪緝之,北司拷問之,鍛煉周內,始送法司。即東廠所獲,亦必移鎮撫再鞫,而後刑部得擬其罪。故廠勢強,則衛附之,廠勢稍弱,則衛反氣凌其上。陸炳緝司禮李彬、東廠馬廣陰事,皆至死,以炳得內閣嵩意。及後中官愈重,閣勢日輕。閣臣反比廠為之下,而衛使無不競趨廠門,甘為役隸矣。

錦衣衛升授勳衛、任子、科目、功升,凡四途。嘉靖以前,文臣子弟多不屑就。萬曆初,劉守有以名臣子掌衛,其後皆樂居之。士大夫與往還,獄急時,頗賴其力。守有子承禧及吳孟明其著者也。莊烈帝疑群下,王德化掌東廠,以慘刻輔之,孟明掌衛印,時有縱舍,然觀望廠意不敢違。而鎮撫梁清宏、喬可用朋比為惡。凡縉紳之門,必有數人往來蹤跡,故常晏起早闔,毋敢偶語。旗校過門,如被大盜,官為囊橐,均分其利。京城中奸細潛入,傭夫販子陰為流賊所遣,無一舉發,而高門富豪跼蹐無寧居。其徒黠者恣行請托,稍拂其意,飛誣立構,摘竿牘片字,株連至十數人。姜採、熊開元下獄,帝諭掌衛駱養性潛殺之。養性洩上語,且言:「二臣當死,宜付所司,書其罪,使天下明知。若陰使臣殺之,天下後世謂陛下何如主?」會大臣多為採等言,遂得長繫。此養性之可稱者,然他事肆虐亦多矣。

錦衣舊例有功賞,惟緝不軌者當之。其後冒濫無紀,所報百無一實。吏民重困,而廠衛題請輒從。隆慶初,給事中歐陽一敬極言其弊嵇康集又名《嵇中散集》。三國魏嵇康著。據《隋書·經,言:「緝事員役,其勢易逞,而又各類計所獲功次,以為升授。則憑可逞之勢,邀必獲之功,枉人利己,何所不至。有盜經出首倖免,故令多引平民以充數者;有括家囊為盜贓,挾市豪以為證者;有潛構圖書,懷挾偽批,用妖言假印之律相誣陷者;或姓名相類,朦朧見收;父訴子孝,坐以忤逆。所以被訪之家,諺稱為刬,毒害可知矣。乞自今定制,機密重情,事干憲典者,廠衛如故題請。其情罪不明,未經讞審,必待法司詳擬成獄之後,方與紀功。仍敕兵、刑二部勘問明白,請旨升賞。或經緝拿未成獄者,不得虛冒比擬,及他詞訟不得概涉,以侵有司之事。如獄未成,而官校及鎮撫司拷打傷重,或至死者,許法司參治。法司容隱扶同,則聽科臣并參。如此則功必覆實,訪必當事,而刑無冤濫。」時不能用也。

內官同法司錄囚,始於正統六年,命何文淵、王文審行在疑獄,敕同內官興安。周忱、郭瑾往南京,敕亦如之。時雖未定五年大審之制,而南北內官得與三法司刑獄矣。景泰六年,命太監王誠會三法司審錄在京刑獄,不及南京者,因災創舉也。成化八年,命司禮太監王高、少監宋文毅兩京會審,而各省恤刑之差,亦以是歲而定。十七年辛卯,命太監懷恩同法司錄囚。其後審錄必以丙辛之歲。弘治九年不遣內官。十三年,以給事中丘俊言,復命會審。凡大審錄,齎敕張黃蓋於大理寺,為三尺壇,中坐,三法司左右坐,御史、郎中以下捧牘立,唯諾趨走惟謹。三法司視成案,有所出入輕重,俱視中官意,不敢忤也。成化時,會審有弟助兄斗,因毆殺人者,太監黃賜欲從末減。尚書陸瑜等持不可,賜曰:「同室鬥者,尚被髮纓冠救之,況其兄乎?」瑜等不敢難,卒為屈法。萬曆三十四年大審,御史曹學程以建言久繫,群臣請宥,皆不聽。刑部侍郎沈應文署尚書事,合院寺之長,以書抵太監陳矩,請寬學程罪。然後會審,獄具,署名同奏。矩復密啟,言學程母老可念。帝意解,釋之。其事甚美,而監權之重如此。錦衣衛使亦得與法司午門外鞫囚,及秋後承天門外會審,而大審不與也。每歲決囚後,圖諸囚罪狀於衛之外垣,令人觀省。內臣曾奉命審錄者,死則於墓寢畫壁,南面坐,旁列法司堂上官,及御史、刑部郎引囚鞠躬聽命狀,示後世為榮觀焉。

成化二年,命內官臨斬強盜宋全。嘉靖中,內臣犯法,詔免逮問,唯下司禮監治。刑部尚書林俊言:「宮府一體欲」則為飲食、男女一類生活欲望。程頤要求「損人欲以復,內臣所犯,宜下法司,明正其罪,不當廢祖宗法。」不聽。按太祖之制,內官不得識字、預政,備掃除之役而已。末年焚錦衣刑具,蓋示永不復用。而成祖違之,卒貽子孫之患,君子惜焉。
