\article{列傳第九十二}

\begin{pinyinscope}
陳九疇翟鵬張漢孫繼魯曾銑丁汝夔楊守謙商大節王抒楊選

陳九疇,字禹學,曹州人。倜儻多權略。自為諸生,即習武事。弘治十五年進士。除刑部主事。有重囚越獄,人莫敢攖,九疇挺槊逐得之,遂以武健名。正德初,錄囚南畿,忤劉瑾,謫陽山知縣。瑾敗,復故官。歷郎中,遷肅州兵備副使。總督彭澤之賂土魯番也,遣哈密都督寫亦虎仙往。九疇奮曰:「彭公受天子命,制邊疆,不能身當利害,何但模棱為!」乃練卒伍,繕營壘,常若臨大敵。寫亦虎仙果通賊。番酋速檀滿速兒犯嘉峪關,遊擊芮寧敗死。尋復遣斬巴思等以駝馬乞和,而陰遺書虎仙及其姻黨阿剌思罕兒、失拜煙答等俾內應。九疇知賊計,執阿剌思罕兒及斬巴思付獄。通事毛監等守之。監等故與通,欲縱去,眾番皆伺隙為變。九疇覺之,佼監等。賊失內應,遂拔帳走。兵部尚書王瓊惡澤,並坐九疇失事罪,逮繫法司獄。以失拜煙答繫死為罪,除其名。

世宗即位,起故官。俄進陜西按察使。居數月,甘肅總兵官李隆嗾部卒毆殺巡撫許銘,焚其屍。乃擢九疇右僉都御史,巡撫甘肅,按驗銘事,誅隆及亂卒首事者。九疇抵鎮,言額軍七萬餘,存者不及半,且多老弱,請令召募。詔可。

嘉靖三年,速檀滿速兒復以二萬餘騎圍肅州。九疇自甘州晝夜馳入城,射賊,賊多死。已,又出兵擊走之。其分掠甘州者,亦為總兵官姜奭所敗。論功,進副都御史,賚金幣。九疇上言:「番賊敢入犯者,以我納其朝貢,縱商販,使得稔虛實也。寫亦虎仙逆謀已露,輸貨權門,轉蒙寵幸,以犯邊之寇,為來享之賓。邊臣怵利害,拱手聽命,致內屬番人勾連接引,以至於今。今即不能如漢武興大宛之師,亦當效光武絕西域之計。先後入貢未歸者二百人,宜安置兩粵,其謀逆有迹者加之刑僇,則賊內無所恃,必不復有侵軼。倘更包含隱忍,恐河西十五衛所,永無息肩之期也。」事下,總制楊一清頗採其議。四年春致仕歸。

初,土魯番敗遁,都指揮王輔言速檀滿速兒及牙木蘭俱死於砲,九疇以聞。後二人上表求通貢,帝怪且疑。而番人先在京師者為蜚語,言肅州之圍,由九疇激之,帝益信。會百戶王邦奇訐楊廷和、彭澤,詞連九疇。吏部尚書桂萼等欲緣九疇以傾澤,因請許通貢,而追治九疇激變狀。大學士一清言事已前決。帝不聽,逮下詔獄。刑部尚書胡世寧言於朝曰:「世寧司刑而殺忠臣,寧殺世寧。」乃上疏為訟冤曰:「番人變詐,妄騰謗讟,欲害我謀臣耳。夫其畜謀內寇,為日已久。一旦擁兵深入,諸番約內應,非九疇先幾奮僇,且近遣屬夷卻其營帳,遠交瓦刺擾其窟巢,使彼內顧而返,則肅州孤城豈復能保?臣以為文臣之有勇知兵忘身殉國者,無如九疇,宜番人深忌而欲殺也。惟聽部下卒妄報,以滿速兒等為已死,則其罪有不免耳。」已,法司具獄亦如世寧言。帝卒中萼等言,謫戍極邊。居十年,赦還。

翟鵬,字志南,撫寧衛人。正德三年進士。除戶部主事。歷員外郎中,出為衛輝知府,調開封。擢陜西副使,進按察使。性剛介,歷官以清操聞。嘉靖七年,擢右僉都御史,巡撫寧夏。時邊政久馳,壯卒率占工匠私役中官家,守邊者並羸老不任兵。又番休無期,甚者夫守墩,妻坐鋪。鵬至,盡清占役,使得迭更。野雞臺二十餘墩孤懸塞外,久棄不守,鵬盡復之。歲大侵,請於朝以振。坐寇入停俸。復坐劾總兵官趙瑛失事,為所訐,奪職歸。

二十年八月,俺答入山西內地。兵部請遣大臣督軍儲,因薦鵬。乃起故官,整飭畿輔、山西、河南軍務兼督餉。鵬馳至,俺答已飽去,而吉囊軍復寇汾、石諸州。鵬往來馳驅,不能有所挫。寇退,乃召還。明年三月,宣大總督樊繼祖罷,除鵬兵部右侍郎代之。上疏言:「將吏遇被掠人牧近塞,宜多方招徠。殺降邀功者,宜罪。寇入,官軍遏敵雖無功,竟賴以安者,當錄。若賊眾我寡,奮身戰,雖有傷折、未至殘生民者,罪當原。於法,俘馘論功,損挫論罪。乃有摧鋒陷陣不暇斬首,而在後掩取者反積級受功,有逡巡觀望幸茍全,而力戰當先者反以損軍治罪,非戎律之平。」帝皆從其議。會有降人言寇且大入,鵬連乞兵餉。帝怒,令革職閒住,因罷總督官不設。鵬受事僅百日而去。

其年七月,俺答復大入山西,縱掠太原、潞安。兵部請復設總督,乃起鵬故官,令兼督山東、河南軍務,巡撫以下並聽節制。鵬受命,寇已出塞。即馳赴朔州,請調陜西、薊、遼客兵八支,及宣、大三關主兵,兼募土著,選驍銳者十萬,統以良將,列四營,分布塞上,每營當一面。寇入境,游兵挑之,誘其追,諸營夾攻。脫不可禦,急趨關南依牆守,邀擊其困歸。帝從之。鵬乃浚壕築垣,修邊牆三百九十餘里,增新墩二百九十二,護墩堡一十四,建營舍一千五百間,得地萬四千九百餘頃,募軍千五百人,人給五十畝,省倉儲無算。疏請東自平刑,西至偏關,畫地分守。增游兵三支,分駐雁門、寧武、偏關。寇攻牆,戍兵拒,游兵出關夾攻,此守中有戰。東大同,西老營堡,因地設伏,伺寇所向。又於宣、大、三關間,各設勁兵,而別選戰士六千,分兩營,遇警令總督武臣張鳳隨機策應,此戰中有守。帝從其議,且命自今遇敵,逗遛者都指揮以下即斬,總兵官以下先取死罪狀奏請。

先是,鵬遣千戶火力赤率兵三百哨至豐州灘,不見寇。復選精銳百,遠至豐州西北,遇牧馬者百餘人,擊斬二十三級,奪其馬還。未入塞,寇大至,官軍饑憊,盡棄所獲奔。鵬具實陳狀。帝以將士敢深入,仍行遷賞。舊例,兵皆團操鎮城,聞警出戰。自邊患熾,每夏秋間分駐邊堡,謂之暗伏。鵬請入秋悉令赴塞,畫地分守,謂之擺邊,九月中還鎮。遂著為令。

二十三年正月,帝以去歲無寇為將帥力,降敕獎鵬,賜以襲衣。至三月,俺答寇宣府龍門所,總兵官郤永等卻之,斬五十一級。論功,進兵部尚書。帝倚鵬殄寇,錫命屢加,所請多從,而責效甚急。鵬亦竭智力,然不能呼吸應變。御史曹邦輔嘗劾鵬,鵬乞罷,弗允。是年九月,蘇州巡撫朱方請撤諸路防秋兵,兵部尚書毛伯溫因併撤宣、大、三關客兵。俺答遂以十月初寇膳房堡。為郤永所拒,乃於萬全右衛毀牆入。由順聖川至蔚州,犯浮屠峪,直抵完縣,京師戒嚴。帝大怒,屢下詔責鵬。鵬在朔州聞警。夜半至馬邑,調兵食,復趨渾源,遣諸將遏敵。御史楊本深劾鵬逗遛,致賊震畿輔。兵科戴夢桂繼之。遂遣官械鵬,而以兵部左侍郎張漢代。鵬至,下詔獄,坐永戍。行至河西務,為民家所窘,告鈔關主事杖之。廠衛以聞,復逮至京,卒於獄。人皆惜之。

初,鵬在衛輝,將入覲,行李蕭然,通判王江懷金遺之。鵬曰:「豈我素履未孚於人耶?」江慚而退,其介如此。隆慶初,復官。

張漢,鐘祥人。代鵬時,寇已出境,乃命翁萬達總督宣、大,而以漢專督畿輔、河南、山東諸軍。漢條上選將、練兵、信賞、必罰四事,請令大將得專殺偏裨,而總督亦得斬大將,人知退怯必死,自爭赴敵。帝不欲假臣下權,惡之。兵部言:漢老邊事,言皆可從。帝令再議。部臣乃言漢議皆當,而專殺大將,與《會典》未合。帝姑報可。會考察拾遺,言官劾漢剛愎。遂械繫詔獄,謫戍鎮西衛。後數年邊警,御史陳九德薦漢。帝怒,斥九德為民。漢居戍所二十年卒。隆慶初,贈兵部尚書。

孫繼魯,字道甫,雲南右衛人。嘉靖二年進士。授澧州知州。坐事,改國子助教。歷戶部郎中,監通州倉。歷知衛輝、淮安二府。織造中官過淮,繼魯與之忤。誣逮至京,大學士夏言救免。繼魯不謝,言不悅。改補黎平。擢湖廣提學副使,進山西參政。數繩宗籓。暨遷按察使,宗籓百餘人擁馬發其裝,敞衣外無長物,乃載酒謝過。遷陜西右布政使。二十六年擢右副都御史,代楊守謙巡撫山西。繼魯耿介,所至以清節聞,然好剛使氣。總督都御史翁萬達議撤山西內邊兵,並力守大同外邊,帝報可。繼魯抗章爭,言:「紫荊、居庸、山海諸關,東枕溟渤;雁門、寧武、偏頭諸關,西據黃河。天設重險,以籓衛國家,豈可聚師曠野,洞開重門以延敵?夫紫刑諸關之拱護京師,與雁門諸關之屏蔽全晉,一也。今議者不撤紫荊以並守宣府,豈可獨撤雁門以並守大同耶?況自偏頭、寧武、雁門東抵平刑關為山西長邊,自右衛雙溝墩至東陽河、鎮口臺為大同長邊,自丫角山至雙溝百四十里為大同緊邊,自丫角山至老牛灣百四十里為山西緊邊,論長邊則大同為急,山西差緩,論緊邊則均為最急。此皆密邇河套,譬之門闔。山西守左,大同守右。山西並力守左尚不能支,又安能分力以守大同之右?近年寇不敢犯山西內郡者,以三關備嚴故也。使三關將士遠離堡戍,欲其不侵犯難矣。全師在外,強寇內侵,即紫荊、倒馬諸關不將徒守哉!」萬達聞之不悅,上疏言:「增兵擺邊,始於近歲,與額設守邊者不同。繼魯乃以危言相恐,復遺臣書,言往歲建雲中議,宰執幾不免;近年撤各路兵,督撫業蒙罪。其詆排如此。今防秋已逼,乞別調繼魯,否則早罷臣,無誤邊事。」兵部是繼魯言。帝不從,下廷議。廷臣請如萬達言。帝方倚萬達,怒繼魯騰私書,引往事議君上。而夏言亦惡繼魯,不為地,遂逮下詔獄。疽發於項,瘐死。繼魯為巡撫僅四月。山西人習其前政,冀有所設施,遽以非罪死,咸為痛惜。宗籓有上書訟其冤者,即前奪視其裝者也。穆宗即位,贈兵部左侍郎,賜祭葬,蔭一子,謚清愍。

曾銑,字子重,江都人。自為諸生,以才自豪。嘉靖八年成進士,授長樂知縣。徵為御史,巡按遼東。遼陽兵變,執辱都御史呂經。銑時按金、復,急檄副總兵李監罷經苛急事,為亂軍乞赦。經罷,趨廣寧,悍卒于蠻兒等復執辱經。其月,撫順卒亦縛指揮劉雄父子。會朝廷遣侍郎林庭昂往勘,亂卒懼。遼陽倡首者趙劓兒潛詣廣寧與蠻兒合謀,欲俟鎮城官拜表,集眾亂,為總兵官劉淮所覺,計不行。復結死囚,欲俟庭昂至,閉城門為變。而銑已刺得二城及撫順為惡者姓名,密授諸將,劓兒等數十人同日捕獲。銑上言:「往者甘肅、大同軍變,處之過輕。群小謂辱命臣,殺主帥,罪不過此,遂相率為亂。今首惡宜急誅。」乃召還庭昂,命銑勘實,悉斬諸首惡,懸首邊城,全遼大定。擢銑大理寺丞,遷右僉都御史,巡撫山東。俺答數入內地,銑請築臨清外城。工畢,進副都御史。居三年,改撫山西。經歲寇不犯邊,朝廷以為功,進兵部侍郎,巡撫如故。

二十五年夏,以原官總督陜西三邊軍務。寇十萬餘騎由寧塞營入,大掠延安、慶陽境。銑率兵數千駐塞門,而遣前參將李珍搗寇巢於馬梁山陰,斬首百餘級。寇聞之,始遁。捷奏,賚銀幣。既而寇屢入,游擊高極死焉,副總兵蕭漢敗績。銑疏諸將罪,治如律。時套寇牧近塞,零騎往來,居民不敢樵採。銑方築塞,慮為所擾,乃選銳卒擊之。寇稍北,間以輕騎入掠,銑復率諸軍驅之遠徙。參將李珍及韓欽功為多,詔增銑俸一級,賜銀幣有加。

銑素喜功名,又感帝知遇,益圖所報稱。念寇居河套,久為中國患,上疏曰:「賊據河套,侵擾邊鄙將百年。孝宗欲復而不能,武宗欲征而不果,使吉囊據為巢穴。出套則寇宣、大、三關,以震畿輔;入套則寇延、寧、甘、固,以擾關中。深山大川,勢顧在敵而不在我。封疆之臣曾無有以收復為陛下言者,蓋軍興重務也;小有挫失,媒孽踵至,鼎鑊刀鋸,面背森然。臣非不知兵凶戰危,而枕戈汗馬,切齒痛心有日矣。竊嘗計之:秋高馬肥,弓矢勁利,彼聚而攻,我散而守,則彼勝;冬深水枯,馬無宿槁,春寒陰雨,壞無燥土,彼勢漸弱,我乘其弊,則中國勝。臣請以銳卒六萬,益以山東鎗手二千,每當春夏交,攜五十日餉,水陸交進,直搗其巢。材官騶發,炮火雷激,則寇不能支。此一勞永逸之策,萬世社稷所賴也。」遂條八議以進。是時,銑與延、寧撫臣欲西自定邊營,東至黃甫川一千五百里,築邊牆禦寇,請帑金數十萬,期三年畢功。疏並下兵部。部臣難之,請令諸鎮文武將吏協議。詔報曰:「賊據套為中國患久矣,朕宵旰念之,邊臣無分主憂者。今銑倡恢復議甚壯,其令銑與諸鎮臣悉心上方略,予修邊費二十萬。」銑乃益銳。而諸巡撫延綏張問行、陜西謝蘭、寧夏王邦瑞及巡按御史盛唐以為難,久不會奏。銑怒,疏請於帝,帝為責讓諸巡撫。會問行已罷,楊守謙代之,意與銑同。銑遂合諸臣條上方略十八事,已,又獻營陣八圖,並優旨下廷議。

廷臣見上意向銑,一如銑言。帝忽出手詔諭輔臣曰:「今逐套賊,師果有名否?兵食果有餘?成功可必否?一銑何足言,如先民荼毒何?」初,銑建議時,輔臣夏言欲倚以成大功,主之甚力。及是,大駭,請帝自裁斷。帝命刊手詔,遍給與議諸臣。時嚴嵩方與言有隙,欲因以傾言,乃極言套必不可復。陰詆言,故引罪乞罷,以激帝怒。旋復顯攻言,謂「向擬旨褒銑,臣皆不預聞。」兵部尚書王以旂會廷臣覆奏,遂盡反前說,言套不可復。帝乃遣官逮銑、出以旂代之;責科道官不言,悉杖於廷,停俸四月。帝雖怒銑,然無意殺之也。咸寧侯仇鸞鎮甘肅時,以阻撓為銑所劾,逮問。嵩故雅親鸞。知銑所善同邑蘇綱者,言繼妻父,綱與銑、言嘗交關傳語,乃代鸞獄中草疏,誣銑掩敗不奏,剋軍餉鉅萬,遣子淳屬所親蘇綱賂當途。其言絕無左驗,而帝深入其說,立下淳、綱詔獄。給事中齊譽等見帝怒銑甚,請早正刑章。帝責譽黨奸避事,鐫級調外任。及銑至,法司比擬邊帥失陷城砦者律。帝必欲依正條,當銑交結近侍律斬,妻子流二千里,即日行刑。銑既死,言亦坐斬,而鸞出獄。

銑有膽略,長於用兵。歲除夜,猝命諸將出。時塞上無警,諸將方置酒,不欲行,賂鈴卒求緩於銑妾。銑斬鈴卒以徇。諸將不得已,丙夜被甲行。果遇寇,擊敗之。翼日入賀畢,前請故。銑笑曰:「見烏鵲非時噪,故知之耳。」皆大服。銑廉,既歿,家無餘貲。

隆慶初,給事中辛自修、御史王好問訟銑志在立功,身罹重辟,識與不識,痛悼至今。詔贈兵部尚書,謚襄愍。萬曆中,從御史周磐請,建祠陜西。

李珍者,故坐事失官。銑從徒中錄用,復積戰功至參將。銑既被誣,詔遣給事中申價等往核,因並劾珍與指揮田世威、郭震為銑爪牙,下之詔獄。連及巡撫謝蘭、張問行,御史盛唐,副總兵李琦等,皆斥罰。勒淳、綱贓,恤陣亡軍及居民被難者。銑嘗檄府衛銀三萬兩製車仗,亦責償於淳。且酷刑拷珍,令其實剋餉行賂事,幾死,卒不承。淳用是免,珍竟論死,世威、震謫戍。其後,俺答歲入寇,帝卒不悟,輒曰:「此銑欲開邊,故行報復耳。」

丁汝夔,字大章,霑化人。正德十六年進士。改庶吉士。嘉靖初,授禮部主事。爭「大禮」被杖,調吏部。累官山西左布政使,擢右副都御史,巡撫甘肅。歷撫保定、應天。入為左副都御史。坐事調湖廣參政。復以故官撫河南。歷吏部左、右侍郎。二十八年十月拜兵部尚書兼督團營。條上邊務十事,皆報可。當是時,俺答歲寇邊,羽書疊至。天子方齋居西內,厭兵事,而大學士嚴嵩竊權,邊帥率以賄進,疆事大壞。其明年八月甲子,俺答犯宣府,諸將拒之不得入。汝夔即上言:「寇不得志於宣府,必東趨遼、薊。請敕諸將嚴為備。潮河川乃陵京門戶,宜調遼東一軍赴白馬關,保定一軍赴古北口。」從之。寇果引而東,駐大興州,去古北口百七十里。大同總兵官仇鸞知之,率所部馳至居庸南。順天巡撫王汝孝駐薊州,誤聽諜者謂寇向西北。汝夔信之,請令鸞還大同勿東,詔俟後報。及興州報至,命鸞壁居庸,汝孝守薊州。未幾,寇循潮河川南下至古北口,薄關城。總兵官羅希韓、盧鉞不能卻,汝孝師大潰。寇遂由石匣營達密雲,轉掠懷柔,圍順義城。聞保定兵駐城內,乃解而南,至通州。阻白河不得渡,駐河東孤山,分剽昌平、三河,犯諸帝陵,殺掠不可勝紀。

京師戒嚴,召各鎮勤王。分遣文武大臣各九人,守京城九門,定西侯蔣傳、吏部侍郎王邦瑞總督之,而以錦衣都督陸炳,禮部侍郎王用賓,給事御史各四人,巡視皇城四門。詔大小文臣知兵者,許汝夔委用。汝夔條上八事,請列正兵四營於城外四隅,奇兵九營於九門外近郊。正兵營各一萬,奇兵營各六千。急遣大臣二人經略通州、涿州,且釋罪廢諸將使立功贖罪。帝悉從之。然是時冊籍皆虛數。禁軍僅四五萬,老弱半之,又半役內外提督大臣家不歸伍,在伍者亦涕泣不敢前。從武庫索甲仗,主庫奄人勒常例,不時發。久之不能軍。乃發居民及四方應武舉諸生乘城,且大頒賞格。仇鸞與副將徐玨、游擊張騰等軍白河西,楊守謙與副將朱楫等軍東直門外,諸路援兵亦稍集。議者率謂城內虛,城外有邊兵足恃,宜移京軍備內釁,汝夔亦以為然。遂量掣禁軍入營十王府、厭壽寺前。掌營務者成國公朱希忠恐以兵少獲譴,乃東西抽掣為掩飾計。士疲不得息,出怨言,而莫曉孰為調者,則爭詈汝夔。鸞兵無紀律,掠民間。帝方眷鸞,令勿捕。汝夔亦戒勿治鸞兵。民益怨怒。

寇游騎四出,去都城三十里。及辛巳,遂自通州渡河而西,前鋒七百騎駐安定門外教場。明日,大營薄都城。分掠西山、黃村、沙河、大小榆河,畿甸大震。初,寇逼通州,部所遣偵卒出城不數里,道遇傷者,輒奔還妄言誑汝夔。既而言不讎,汝夔弗罪也。募他卒偵之復如前。以故寇眾寡遠近皆不能知。

宣府總兵官趙國忠,參將趙臣、孫時謙、袁正,游擊姚冕,山西游擊羅恭等,各以兵入援,營玉河諸處。詔兵部核諸鎮兵數,行賞賚。勤王兵先後五六萬人,皆聞變即赴,未齎糗糧。制下犒師,牛酒無所出。越二三日,援軍始得數餅餌,益饑疲不任戰。

帝久不視朝,軍事無由面白。廷臣多以為言,帝不許。禮部尚書徐階復固請,帝乃許。癸未,群臣昧爽入。至日晡,帝始御奉天殿,不發一詞,但命階奉敕諭至午門,集群臣切責之而已。帝怒文武臣不任事,尤怒汝夔。吏部因請起楊守禮、劉源清、史道、許論於家。汝夔不自安,請督諸將出城戰,而以侍郎謝蘭署部事。帝責其推委,命居中如故。寇縱橫內地八日,諸軍不敢發一矢。寇本無意攻城,且所掠過望,乃整輜重,從容趨白羊口而去。

方事棘,帝趣諸將戰甚急。汝夔以咨嵩。嵩曰:「塞上敗或可掩也,失利輦下,帝無不知,誰執其咎?寇飽自颺去耳。」汝夔因不敢主戰,諸將亦益閉營,寇以此肆掠無所忌。既退,汝夔、蘭及戶、工尚書李士翱、胡松,侍郎駱顒、孫禬皆引罪。命革士翱職,停松俸,俱戴罪辦事,侍郎各停俸五月,而下汝夔獄。帝欲大行誅以懲後。汝夔窘,求救於嵩。嵩曰:「我在,必不令公死。」及見帝怒甚,竟不敢言。給事御史劾汝夔禦寇無策。帝責其不早言,奪俸有差。趣具獄,怒法司奏當緩,杖都御史屠僑、刑部侍郎彭黯、大理卿沈良才各四十,降俸五等。刑科張侃等循故事覆奏,各杖五十,斥侃為民。坐汝夔守備不設,即日斬於市,梟其首,妻流三千里,子戍鐵嶺。汝夔臨刑,始悔為嵩所賣。

方廷訊時,職方郎王尚學當從坐。汝夔曰,「罪在尚書,郎中無預」,得減死論戍。比赴市,問左右:「王郎中免乎?」尚學子化適在旁,謝曰:「荷公恩,免矣。」汝夔歎曰:「汝父勸我速戰,我為政府誤。汝父免,我死無恨。」聞者為泣下。隆慶初,復官。

汝夔既下獄,并逮汝孝、希韓、鉞。寇未盡去,官校不敢前,託言汝孝等追寇白羊口,遠不可卒至。比逮至,論死。帝怒漸解,而汝孝復以首功聞,命俱減死戍邊。

楊守謙,字允亨,徐州人。父志學,字遜夫,弘治六年進士。巡撫大同、寧夏,邊人愛之。累官刑部尚書,卒,謚康惠。

守謙登嘉靖八年進士,授屯田主事。改職方,歷郎中,練習兵計。出為陜西副使,改督學政,有聲,就拜參政。未任,擢右僉都御史,巡撫山西。上言偏頭、老營堡二所,餘地千九百餘頃,請興舉營田。因薦副使張鎬為提調,牛種取給本土。帝稱為忠,即報可。俄移撫延綏。請久任鎬,終其事。其後二年,營田大興。計秋獲可當帑銀十萬,邊關穀價減十五。守謙薦鎬可大用,且言延綏、安定諸邊可如例。戶部請推行之九邊。帝悅,命亟行之,錄守謙、鎬功。守謙未去延綏,而鎬已巡撫寧夏矣。

守謙至延綏,言:「激勸軍士在重賞。令斬一首者升一級,不願者予白金三十兩。賞已薄,又文移察勘,動涉歲時,以故士心不勸。近宣、大事棘稍加賞格,請倍增其數,鎮巡官驗明即給。蓋增級、襲廕,有官者利之,窮卒覬賞而已。」兵部以為然,定斬首一級者與五十兩,著為令。以前山西修邊功,增俸一級,賜金幣有加。請給新設游兵月餉,發倉儲貸饑卒,皆報許。

二十九年進副都御史,巡撫保定兼督紫荊諸關。去鎮之日,傾城號泣,有追送數百里外者。未幾,俺答入寇,守謙率師倍道入援。帝聞其至,甚喜,令營崇文門外。會副總兵朱楫,參將祝福、馮登亦各以兵至,人心稍安。寇游騎散掠枯柳諸村,去京城二十里。守謙及楫等兵移營東直門外。詔同仇鸞調度京城及各路援兵,相機戰守。

寇薄都城,諸將高秉元、徐鏞等御之,不能卻。帝拜鸞大將軍,進守謙兵部右侍郎,協同提督內外諸軍事。鸞時自孤山還,至東直門觀望,斬死人首六級,報功。守謙孤軍薄俺答營,而陣無後繼,不敢戰。帝聞不悅。而尚書丁汝夔慮喪師,戒勿輕戰。諸將離城遠,見守謙不戰,亦堅壁,輒引汝夔及守謙為辭。流聞禁中,帝益怒。

初,寇抵安定門,詔守謙與楫等合擊,莫敢前。守謙亦委無部檄,第申儆備。寇遂毀城外廬舍。城西北隅火光燭天,內臣園宅在焉,環泣帝前,稱將帥為文臣制,故寇得至此。帝怒曰:「守謙擁眾自全,朕親降旨趣戰,何得以部檄為解。」寇退,遂執守謙與汝夔廷鞫之。坐失誤軍機,即日戮於市。守謙臨刑時,慨然曰:「臣以勤王反獲罪,讒賊之口實蔽聖聰。皇天后土知臣此心,死何恨。」邊陲吏士知守謙死,無不流涕者。

守謙坦易無城府,馭下多恩意。守官廉,位至開府,蕭然若寒士。然性遲重,客有勸之戰者,應曰:「周亞夫何人乎?」客曰:「公誤矣,今日何得比漢法?」守謙不納,竟得罪。隆慶初,贈兵部尚書,謚恪愍。

商大節,字孟堅,鐘祥人。嘉靖二年進士。授豐城知縣。始為築城,捕境內盜幾盡。擢兵科給事中。京察竣,復命科道互相劾,被謫鹽城縣丞。三遷刑部郎中,出為廣東僉事。搗海南叛黎巢,增秩,賜金幣。累官山東按察使。擢右僉都御史,巡撫保定兼提督紫刑諸關。慮俺答內侵,疏請重根本,護神京。居四年,召理院事。俺答果大舉薄都城。詔城中居民及四方入應武舉者悉登陴守,以大節率五城御史統之。發帑金五千兩,命便宜募壯士。屢條上軍民急務。比寇退,復命兼管民兵,經略京城內外。訓練鼓舞,軍容甚壯。擢右副都御史,經略如故。所募民兵已四千,請以三等授餉。上者月二石,其次遞減五斗。帝亟從之。

仇鸞為大將軍,盡統中外兵馬,惡大節獨為一軍,不受其節制,欲困之。乃請畫地分守,以京師四郊委大節。大節言:「臣雖經略京城,實非有重兵專戰守責者也。京城四郊利害,鸞欲專以臣當。臣節制者,止巡捕軍,鸞又頻調遣,奸宄猝發,誰為捍禦哉?」所爭甚晰,而帝方寵鸞,不欲人撓其事,責大節懷奸避難,立下詔獄。法司希旨,當大節斬。嚴嵩言:「大節誠有罪,但法司引律非是。幸赦其死,戍極邊。」亦不聽。時三十年四月也。

明年八月,鸞死,大節故部曲石鏜、孫九思等數百人伏闕訟冤,章再上。兵部侍郎張時徹等言:「大節為逆鸞制肘,以抵於法,乞順群情赦之。」帝怒,鐫時徹二秩。明年竟卒於獄。隆慶初,復故官,贈兵部尚書,謚端愍。

王忬,字民應,太倉人。父倬,南京兵部右侍郎,以謹厚稱。忬登嘉靖二十年進士,授行人,遷御史。皇太子出閣,疏以武宗居青宮為戒。又劾罷東廠太監宋興。出視河東鹽政,以疾歸。已,起按湖廣,復按順天。

二十年,俺答大舉犯古北口。忬奏言潮河川有徑道,一日夜可達通州。因疾馳至通為守禦計,盡徙舟楫之在東岸者。夜半,寇果大至。不得渡,遂壁於河東。帝密遣中使覘軍,見忬方厲士乘城。還奏,帝大喜。副都御史王儀守通州,御史姜廷頤劾其不職,忬亦言儀縱士卒虐大同軍。大同軍者,仇鸞兵也。帝立命逮儀,而超擢忬右僉都御史代之。寇退,忬請振難民,築京師外郭,修通州城,築張家灣大小二堡,置沿河敵臺。皆報可。尋罷通州、易州守禦大臣,召忬還。

三十一年出撫山東。甫三月,以浙江倭寇亟,命忬提督軍務,巡視浙江及福、興、漳、泉四府。先後上方略十二事,任參將俞大猷、湯克寬,又奏釋參將尹鳳、盧鏜繫。賊犯溫州,克寬破之。其據昌國衛者,為大猷擊退。而賊首汪直復糾島倭及漳、泉群盜連巨艦百餘蔽海至,濱海數千里同告警。上海及南匯、吳淞、乍浦、蓁嶼諸所皆陷,蘇、松、寧、紹諸衛所州縣被焚掠者二十餘。留內地三月,飽而去。忬乃言將士逐毀其船五十餘艘。於是先所奪文武將吏俸,皆得復。尋以給事王國禎言,改巡撫。忬方視師閩中,賊復大至,犯浙江,盧鏜等頻失利。御史趙炳然劾其罪,帝特宥,忬因請築嘉善、崇德、桐鄉、德清、慈谿、奉化、象山城,而恤被寇諸府。

時已遣尚書張經總督諸軍。大同適中寇,督撫蘇祐、侯鉞俱被逮,乃進忬右副都御史,巡撫大同。秋防事竣,就加兵部右侍郎。薊遼總督楊博還朝,即移忬代之。尋進右都御史。忬言:「騎兵利平地,步兵利險阻。今薊鎮畫地守,請去他郡防秋馬兵八千,易之以步,歲省銀五萬六千餘兩。」從之。打來孫十餘萬騎深入廣寧諸處,總兵官殷尚質等戰歿。忬停俸三月。未幾,打來孫復以十萬騎屯青城,分遣精騎犯一片石、三道關。總兵官歐陽安拒卻之。事聞,賚銀幣。把都兒等犯遷安,副總兵蔣承勛戰死。降昂兵部侍郎,留任。

初,帝器忬才,甚眷之。及所部屢失事,則以為不足辦寇,諭嚴嵩與兵部計防守之宜。嵩奏流河口邊牆有缺,故寇乘之入,宜大修邊牆。且令忬選補額兵,操練戰守,不得專恃他鎮援兵。部條六事,如嵩指。帝乃下詔責忬,赦其罪,實主兵,減客兵,如議。於是練兵之議起。時寇別部入沈陽,有鄉兵金仲良者擒其長討賴。忬賚銀幣,官仲良三級。防秋畢,復忬官。尋復用沈陽卻寇功,蔭一子。已而寇復入遼陽,副總兵王重祿敗績。御史周斯盛以聞。帝置忬不問,治他將吏如律。

初,帝從楊博言,命薊鎮入衛兵聽宣大調遣。忬言:「古北諸口無險可守,獨恃入衛卒護陵京,奈何聽調發?」帝怒曰:「曩令薊鎮練兵,今一卒不練,遇防秋輒調他鎮兵,兵部詳議以聞。」部臣言:「薊鎮額兵多缺,宜察補」。乃遣郎中唐順之往核。還奏額兵九萬有奇,今惟五萬七千,又皆羸老。忬與總兵官安、巡撫馬珮及諸將袁正等,俱宜按治。乃降忬俸二級。帝因問嵩:「邊兵入衛,舊制乎?」嵩曰:「祖宗時無調邊兵入內地者。正德中劉六猖獗,始調許泰、郤永領邊兵討賊。庚戌之變,仇鸞選邊兵十八支護陵京,未用以守薊鎮。至何棟始借二支防守,忬始盡調邊兵守要害,去歲又徵全遼士馬入關,致寇乘虛入犯,遼左一空。若年復一年,調發不已,豈惟糜餉,更有他憂。」帝由是惡忬甚。踰月,寇犯清河,總兵官楊照禦之,斬首八百餘級。越四日,土蠻十萬騎薄界嶺口,副將馬芳拒卻之。明日,敵騎二百奔還,芳及安俘斬四十級。忬猶被賚。

三十八年二月,把都兒、辛愛數部屯會州,挾朵顏為鄉導,將西入,聲言東。忬遽引兵東。寇乃以其間由潘家口入,渡灤河而西,大掠遵化、遷安、薊州、玉田,駐內地五日,京師大震。御史王漸、方輅遂劾忬、安及巡撫王輪罪。帝大怒,斥安,貶輪於外,切責忬,令停俸自效。至五月,輅復劾忬失策者三,可罪者四,遂命逮忬及中軍游擊張倫下詔獄。刑部論忬戍邊,帝手批曰:「諸將皆斬,主軍令者顧得附輕典耶?」改論斬。明年冬,竟死西市。

忬才本通敏。其驟拜都御史,及屢更督撫也,皆帝特簡,所建請無不從。為總督數以敗聞,由是漸失寵。既有言不練主兵者,益大恚,謂:「忬怠事,負我。」嵩雅不悅忬。而忬子世貞復用口語積失歡於嵩子世蕃。嚴氏客又數以世貞家瑣事構於嵩父子。楊繼盛之死,世貞又經紀其喪,嵩父子大恨。灤河變聞,遂得行其計。穆宗即位,世貞與弟世懋伏闕訟冤。復故官,予恤。

楊選,字以公,章丘人。嘉靖二十三年進士。授行人。擢御史,遷易州兵備副使。俺答圍大同右衛,巡撫朱笈被逮,超拜選右僉都御史代之。與侍郎江東、總兵官張承勛解其圍。憂歸,再起,仍故職。四十年擢總督薊遼副都御史。條上封疆極弊十五事,多從其請。以居庸岔道卻敵功,進兵部右侍郎。

明年五月,古北口守將遣哨卒山塞,朵顏衛掠其四人。部長通漢叩關索賞,副總兵胡鎮執之,并縛其黨十餘人。通漢子懼,擁所執哨卒至牆下,請易其父。通漢者,辛愛妻義父也,選欲以牽制辛愛,要其子入質,乃遣還父。自是諸子迭為質,半歲而代。選馳疏以聞,自詡方略。選及巡撫徐紳等俱受賞。

十月丁卯,辛愛與把都兒等大舉自牆子嶺、磨刀峪潰牆入犯,京師戒嚴。帝大驚,諭閣臣徐階曰:「朕東見火光,此賊去京不遠,其令兵部諭諸軍并力剿逐。」明日,選以寇東遁聞,為將士祈賞。帝疑,以問階。對曰:「寇營尚在平谷,選等往通州矣,謂追殺者,妄也。」帝銜之。寇稍東,大掠三河、順義,圍諸將傅津等於鄭官屯。選遣副將胡鎮偕總兵官孫臏、游擊趙溱擊之。臏、溱戰歿,鎮力戰得脫。寇留內地八日不退。給事中李瑜遂劾選、紳與副使盧鎰,參將馮詔、胡粲,游擊嚴瞻等,俱逮下詔獄。又二日,寇始北去,京師解嚴。

初,諜者言寇將窺牆子嶺,部檄嚴待之,而三衛為寇導者紿選赴潘家口。寇已入,選、紳懼得罪,徑趨都城,屯東直門外,旋還通州。及遣鎮等禦,又不勝。內侍家薊西者,嘩言通漢父子實召寇。帝入其言,益怒。法司坐選、紳、詔守備不設律斬,鎰等戍。帝諭錦衣硃希孝坐以縱通漢勾賊罪,復下選詔獄。選不承,止承質通漢父子事,且言事已上聞。希孝錄其語上,刑部如帝指論選死。即戳於市,梟其首示邊,妻子流二千里。紳論死系獄,詔及鎰等戍邊。帝雖怒選甚,但欲誅其身,法司乃並坐其妻子。隆慶初,始釋還。

贊曰:世宗威柄自操,用重典以繩臣下,而弄權者借以行其私。於是賜闒冗廢職之徒事敗伏辜,而出力任事之臣亦中危法受戮,邊臣不得自展布,而武備綍矣。陳九疇、翟鵬、孫繼魯、曾銑皆可用之才,或謫或死,不以其罪。銑復套之議甚偉。然權臣當軸,而敵勢方強,雖頗、牧烏能有為?丁汝夔之戮,於法誠不為過。然戎律之弛,有由來矣,而汝夔獨蒙其咎。王忬、楊選於邊備甚疏,宜不免雲。


\end{pinyinscope}