\article{列傳第三十六}

\begin{pinyinscope}
楊士奇楊榮曾孫旦楊溥馬愉

楊士奇,名寓,以字行,泰和人。早孤,隨母適羅氏,已而復宗。貧甚。力學,授徒自給。多游湖、湘間,館江夏最久。建文初,集諸儒修《太祖實錄》,士奇已用薦征授教授當行,王叔英復以史才薦。遂召入翰林,充編纂官。尋命吏部考第史館諸儒。尚書張紞得士奇策,曰:「此非經生言也。」奏第一。授吳王府審理副,仍供館職。成祖即位,改編修。已,簡入內閣,典機務。數月進侍講。

永樂二年選宮僚,以士奇為左中允。五年進左諭德。士奇奉職甚謹,私居不言公事,雖至親厚不得聞。在帝前,舉止恭慎,善應對,言事輒中。人有小過,嘗為揜覆之。廣東布政使徐奇載嶺南土物饋廷臣,或得其目籍以進。帝閱無士奇名,召問。對曰:「奇赴廣時,群臣作詩文贈行,臣適病弗預,以故獨不及。今受否未可知,且物微,當無他意。」帝遽命毀籍。

六年,帝北巡,命與蹇義、黃淮留輔太子。太子喜文辭,贊善王汝玉以詩法進。士奇曰:「殿下當留意《六經》,暇則觀兩漢詔令。詩小技,不足為也。」太子稱善。

初,帝起兵時,漢王數力戰有功。帝許以事成立為太子。既而不得立,怨望。帝又憐趙王年少,寵異之。由是兩王合而間太子,帝頗心動。九年還南京,召士奇問監國狀。士奇以孝敬對,且曰:「殿下天資高,即有過必知,知必改,存心愛人,決不負陛下託。」帝悅。十一年正旦,日食。禮部尚書呂震請勿罷朝賀。侍郎儀智持不可。士奇亦引宋仁宗事力言之。遂罷賀。明年,帝北征。士奇仍輔太子居守。漢王譖太子益急。帝還,以迎駕緩,盡征東宮官黃淮等下獄。士奇後至,宥之。召問太子事。士奇頓首言:「太子孝敬如初。凡所稽遲,皆臣等罪。」帝意解。行在諸臣交章劾士奇不當獨宥,遂下錦衣衛獄,尋釋之。

十四年,帝還京師,微聞漢王奪嫡謀及諸不軌狀,以問蹇義。義不對,乃問士奇。對曰:「臣與義俱侍東宮,外人無敢為臣兩人言漢王事者。然漢王兩遣就籓,皆不肯行。今知陛下將徙都,輒請留守南京。惟陛下熟察其意。」帝默然,起還宮。居數日,帝盡得漢王事,削兩護衛,處之樂安。明年進士奇翰林學士,兼故官。十九年改左春坊大學士,仍兼學士。明年復坐輔導有闕,下錦衣衛獄,旬日而釋。

仁宗即位,擢禮部侍郎兼華蓋殿大學士。帝御便殿,蹇義、夏原吉奏事未退。帝望見士奇,謂二人曰:「新華蓋學士來,必有讜言,試共聽之。」士奇入言:「恩詔減歲供甫下二日,惜薪司傳旨征棗八十萬斤,與前詔戾。」帝立命減其半。服制二十七日期滿,呂震請即吉。士奇不可。震厲聲叱之。蹇義兼取二說進。明日,帝素冠麻衣絰而視朝。廷臣惟士奇及英國公張輔服如之。朝罷,帝謂左右曰:「梓宮在殯,易服豈臣子所忍言,士奇執是也。」進少保,與同官楊榮、金幼孜並賜「繩愆糾繆」銀章,得密封言事。尋進少傅。

時籓司守令來朝,尚書李慶建議發軍伍余馬給有司,歲課其駒。士奇曰:「朝廷選賢授官,乃使牧馬,是貴畜而賤士也,何以示天下後世。」帝許中旨罷之,已而寂然。士奇復力言。又不報。有頃,帝御思善門,召士奇謂曰:「朕向者豈真忘之。聞呂震、李慶輩皆不喜卿,朕念卿孤立,恐為所傷,不欲因卿言罷耳,今有辭矣。」手出陜西按察使陳智言養馬不便疏,使草敕行之。士奇頓首謝。群臣習朝正旦儀,呂震請用樂,士奇與黃淮疏止。未報。士奇復奏,待庭中至夜漏十刻。報可。越日,帝召謂曰:「震每事誤朕,非卿等言,悔無及。」命兼兵部尚書,並食三祿。士奇辭尚書祿。

帝監國時,憾御史舒仲成,至是欲罪之。士奇曰:「陛下即位,詔向忤旨者皆得宥。若治仲成,則詔書不信,懼者眾矣。如漢景帝之待衛綰,不亦可乎。」帝即罷弗治。或有言大理卿虞謙言事不密。帝怒,降一官。士奇為白其罔,得復秩。又大理少卿弋謙以言事得罪。士奇曰:「謙應詔陳言。若加之罪,則群臣自此結舌矣。」帝立進謙副都御史,而下敕引過。

時有上書頌太平者,帝以示諸大臣,皆以為然。士奇獨曰:「陛下雖澤被天下,然流徙尚未歸,瘡痍尚未復,民尚艱食。更休息數年,庶幾太平可期。」帝曰:「然。」因顧蹇義等曰:「朕待御等以至誠,望匡弼。惟士奇曾五上章,卿等皆無一言。豈果朝無闕政,天下太平耶?」諸臣慚謝。是年四月,帝賜士奇璽書曰:「往者朕膺監國之命,卿侍左右,同心合德,徇國忘身,屢歷艱虞,曾不易志。及朕嗣位以來,嘉謨入告,期予於治,正固不二,簡在朕心。茲創制『楊貞,一印賜卿,尚克交修,以成明良之譽。」尋修《太宗實錄》,與黃淮、金幼孜、楊溥俱充總裁官。未幾,帝不豫,召士奇與蹇義、黃淮、楊榮至思善門,命士奇書敕召太子於南京。

宣宗即位,修《仁宗實錄》,仍充總裁。宣德元年,漢王高煦反。帝親征,平之。師還,次獻縣之單家橋,侍郎陳山迎謁,言漢、趙二王實同心,請乘勢襲彰德執趙王。榮力贊決。士奇曰:「事當有實,天地鬼神可欺乎?」榮厲聲曰:「汝欲撓大計耶!今逆黨言趙實與謀,何謂無辭?」士奇曰:「太宗皇帝三子,今上惟兩叔父。有罪者不可赦,其無罪者宜厚待之,疑則防之,使無虞而已。何遽加兵,傷皇祖在天意乎?」時惟楊溥與士奇合。將入諫,榮先入,士奇繼之,閽者不納。尋召義、原吉入。二人以士奇言白帝。帝初無罪趙意,移兵事得寢。比還京,帝思士奇言,謂曰:「今議者多言趙王事,奈何?」士奇曰:「趙最親,陛下當保全之,毋惑群言。」帝曰:「吾欲封群臣章示王,令自處何如?」士奇曰:「善,更得一璽書幸甚。」於是發使奉書至趙。趙王得書大喜。泣曰:「吾生矣。」即上表謝,且獻護衛,言者始息。帝待趙王日益親而薄陳山。謂士奇曰:「趙王所以全,卿力也。」賜金幣。

時交阯數叛。屢發大軍征討,皆敗沒。交阯黎利遣人偽請立陳氏後。帝亦厭兵,欲許之。英國公張輔、尚書蹇義以下,皆言與之無名,徒示弱天下。帝召士奇、榮謀。二人力言:「陛下恤民命以綏荒服,不為無名。漢棄珠厓,前史以為美談,不為示弱,許之便。」尋命擇使交阯者。蹇義薦伏伯安口辨。士奇曰:「言不忠信,雖蠻貊之邦不可行。伯安小人,往且辱國。」帝是之,別遣使。於是棄交阯,罷兵,歲省軍興巨萬。

五年春,帝奉皇太后謁陵,召英國公張輔、尚書蹇義及士奇、榮、幼孜、溥,朝太后於行殿。太后慰勞之。帝又語士奇曰:「太后為朕言,先帝在青宮,惟卿不憚觸忤,先帝能從,以不敗事。又誨朕當受直言。」士奇對曰:「此皇太后盛德之言,願陛下念之。」尋敕鴻臚寺。士奇老有疾,趨朝或後,毋論奏。帝嘗微行,夜幸士奇宅。士奇倉皇出迎,頓首曰:「陛下奈何以社稷宗廟之身自輕?」帝曰:「朕欲與卿一言,故來耳。」後數日,獲二盜,有異謀。帝召士奇,告之故。且曰:「今而後知卿之愛朕也。」帝以四方屢水旱,召士奇議下詔寬恤,免災傷租稅及官馬虧額者。士奇因請並蠲逋賦薪芻錢,減官田額,理冤滯,汰工役,以廣德意。民大悅。踰二年,帝謂士奇曰:「恤民詔下已久,今更有可恤者乎?」士奇曰:「前詔減官田租,戶部徵如故。」帝怫然曰:「今首行之,廢格者論如法。」士奇復請撫逃民,察墨吏,舉文學、武勇之士,令極刑家子孫皆得仕進。又請廷臣三品以上及二司官,各舉所知,備方面郡守選。皆報可。當是時,帝勵精圖治,士奇等同心輔佐,海內號為治平。帝乃仿古君臣豫遊事,每歲首,賜百官旬休。車駕亦時幸西苑萬歲山,諸學士皆從。賦詩賡和,從容問民間疾苦。有所論奏,帝皆虛懷聽納。

帝之初即位也,內閣臣七人。陳山、張瑛以東宮舊恩入,不稱,出為他官。黃淮以疾致仕。金幼孜卒。閣中惟士奇、榮、溥三人。榮疏闓果毅,遇事敢為。數從成祖北征,能知邊將賢否、阨塞險易遠近、敵情順逆。然頗通饋遺,邊將歲時致良馬。帝頗知之,以問士奇。士奇力言:「榮曉暢邊務,臣等不及,不宜以小眚介意。」帝笑曰:「榮嘗短卿及原吉,卿乃為之地耶?」士奇曰:「願陛下以曲容臣者容榮。」帝意乃解。其後,語稍稍聞,榮以此愧士奇,相得甚歡。帝亦益親厚之,先後所賜珍果、牢醴、金綺衣、幣、書器無算。

宣宗崩,英宗即位,方九齡。軍國大政關白太皇太后。太后推心任士奇、榮、溥三人,有事遣中使詣閣諮議,然後裁決。三人者亦自信,侃侃行意。士奇首請練士卒,嚴邊防,設南京參贊機務大臣,分遣文武鎮撫江西、湖廣、河南、山東,罷偵事校尉。又請以次蠲租稅,慎刑獄,嚴核百司。皆允行。正統之初,朝政清明,士奇等之力也。三年,《宣宗實錄》成,進少師。四年乞致仕。不允。敕歸省墓。未幾,還。

是時中官王振有寵於帝,漸預外庭事,導帝以嚴御下,大臣往往下獄。靖江王佐敬私饋榮金。榮先省墓,歸不之知。振欲借以傾榮,士奇力解之,得已。榮尋卒,士奇、溥益孤。其明年遂大興師征麓川,帑藏耗費,士馬物故者數萬。又明年,太皇太后崩,振勢益盛,大作威福,百官小有牴牾,輒執而繫之。廷臣人人惴恐,士奇亦弗能制也。

士奇既耄,子稷傲很,嘗侵暴殺人。言官交章劾稷。朝議不即加法,封其狀示士奇。復有人發稷橫虐數十事,遂下之理。士奇以老疾在告。天子恐傷士奇意,降詔慰勉。士奇感泣,憂不能起。九年三月卒,年八十。贈太師,謚文貞。有司乃論殺稷。

初,正統初,士奇言瓦剌漸強,將為邊患,而邊軍缺馬,恐不能禦。請於附近太僕寺關領,西番貢馬亦悉給之。士奇歿未幾,也先果入寇,有土木之難,識者思其言。又雅善知人,好推轂寒士,所薦達有初未識面者。而於謙、周忱、況鍾之屬,皆用士奇薦,居官至一二十年,廉能冠天下,為世名臣云。

次子道,以廕補尚寶丞。成化中,進太常少卿,掌司事。

楊榮,字勉仁,建安人,初名子榮。建文二年進士。授編修。成祖初入京,榮迎謁馬首曰:「殿下先謁陵乎,先即位乎?」成祖遽趣駕謁陵。自是遂受知。既即位,簡入文淵閣,為更名榮。同值七人,榮最少,警敏。一日晚,寧夏報被圍。召七人,皆已出,獨榮在,帝示以奏。榮曰:「寧夏城堅,人皆習戰,奏上已十餘日,圍解矣。」夜半,果奏圍解。帝謂榮曰:「何料之審也!」江西盜起,遣使撫諭,而令都督韓觀將兵繼其後。賊就撫奏至,帝欲賜敕勞觀。榮曰:「計發奏時,觀尚未至,不得論功。」帝益重之,再遷至侍講。太子立,進右諭德,仍兼前職,與在直諸臣同賜二品服。評議諸司事宜,稱旨,復賜衣幣。帝威嚴,與諸大臣議事未決,或至發怒。榮至,輒為霽顏,事亦遂決。

五年,命往甘肅經畫軍務,所過覽山川形勢,察軍民,閱城堡。還奏武英殿。帝大悅,值盛暑,親剖瓜啖之。尋進右庶子,兼職如故。明年以父喪給傳歸。既葬,起復視事。又明年,母喪乞歸。帝以北行期迫不許,命同胡廣、金幼孜扈從。甘肅總兵官何福言脫脫不花等請降,需命於亦集乃。命榮往甘肅偕福受降,持節即軍中封福寧遠侯。因至寧夏,與寧陽侯陳懋規畫邊務。還陳便宜十事。帝嘉納之。

八年從出塞,次臚朐河。選勇士三百人為衛,不以隸諸將,令榮領之。師旋,餉不繼。榮請盡以供御之餘給軍,而令軍中有餘者得相貸,入塞,官為倍償。軍賴以濟。明年乞奔喪,命中官護行。還詢閩中民情及歲豐歉,榮具以對。尋命侍諸皇孫讀書文華殿。

十年,甘肅守臣宋琥言,叛寇老的罕逃赤斤蒙古,且為邊患。乃復遣榮至陜西,會豐城侯李彬議進兵方略。榮還奏言:「隆冬非用兵時,且有罪不過數人,兵未可出。」帝從其言,叛者亦降。明年復與廣、幼孜從北巡。又明年征瓦剌,太孫侍行。帝命榮以間陳說經史,兼領尚寶事。凡宣詔出令,及旗志符驗,必得榮奏乃發。帝嘗晚坐行幄,召榮計兵食。榮對曰:「擇將屯田,訓練有方,耕耨有時,即兵食足矣。」十四年與金幼孜俱進翰林學士,仍兼庶子,從還京師。明年復從北征。

十六年,胡廣卒,命榮掌翰林院事,益見親任。諸大臣多忌榮,欲疏之,共舉為祭酒。帝曰:「吾固知其可,第求代榮者。」諸大臣乃不敢言。十八年進文淵閣大學士,兼學士如故。明年定都北京。會三殿災,榮麾衛士出圖籍制誥,舁東華門外。帝褒之。榮與幼孜陳便宜十事。報可。

二十年,復從出塞,軍事悉令參決,賚予優渥。師還,勞將士,分四等賜宴,榮、幼孜皆列前席,受上賞。已,復下詔征阿魯台。或請調建文時江西所集民兵。帝問榮。榮曰:「陛下許民復業且二十年,一旦復徵之,非示天下信。」從之。明年從出塞,軍務悉委榮,晝夜見無時。帝時稱「楊學士」,不名也。又明年復從北征。當是時,帝凡五出塞,士卒饑凍,饋運不繼,死亡十二三。大軍抵答蘭納木兒河,不見敵。帝問群臣當復進否,群臣唯唯,惟榮、幼孜從容言宜班師。帝許之。

還次榆木川,帝崩。中官馬雲等莫知所措,密與榮、幼孜入御幄議。二人議:六師在外,去京師尚遠,祕不發喪。以禮斂,熔錫為椑,載輿中。所至朝夕進膳如常儀,益嚴軍令,人莫測。或請因他事為敕,馳報皇太子。二人曰:「誰敢爾!先帝在則稱敕,賓天而稱敕,詐也,罪不小。」眾曰:「然。」乃具大行月日及遺命傳位意,啟太子。榮與少監海壽先馳訃。既至,太子命與蹇義、楊士奇議諸所宜行者。

仁宗即位,進太常卿,餘官如故。尋進太子少傅、謹身殿大學士。既而有言榮當大行時,所行喪禮及處分軍事狀。帝賜敕褒勞,賚予甚厚。進工部尚書,食三祿。時士奇、淮皆辭尚書祿,榮、幼孜亦固辭。不允。

宣德元年,漢王高煦反。帝召榮等定計。榮首請帝親征,曰:「彼謂陛下新立,必不自行。今出不意,以天威臨之,事無不濟。」帝從其計。至樂安,高煦出降。師還,以決策功,受上賞,賜銀章五,褒予甚至。

三年從帝巡邊,至遵化。聞兀良哈將寇邊,帝留扈行諸文臣於大營,獨命榮從。自將輕騎出喜峰口,破敵而還。五年進少傅,辭大學士祿。九年復從巡邊,至洗馬林而還。

英宗即位,委寄如故。正統三年,與士奇俱進少師。五年乞歸展墓,命中官護行。還至武林驛而卒,年七十。贈太師,謚文敏,授世襲都指揮使。

榮歷事四朝,謀而能斷。永樂末,浙、閩山賊起,議發兵。帝時在塞外,奏至,以示榮。榮曰:「愚民苦有司,不得已相聚自保。兵出,將益聚不可解。遣使招撫,當不煩兵。」從之,盜果息。安南之棄,諸大臣多謂不可,獨榮與士奇力言不宜以荒服疲中國。其老成持重類如此。論事激發,不能容人過。然遇人觸帝怒致不測,往往以微言導帝意,輒得解。夏原吉、李時勉之不死,都御史劉觀之免戍邊,皆賴其力。嘗語人曰:「事君有體,進諫有方,以悻直取禍,吾不為也。」故其恩遇亦始終無間。重修《太祖實錄》及太宗、仁、宣三朝《實錄》,皆為總裁官。先後賜賚,不可勝計。性喜賓客,雖貴盛無稍崖岸,士多歸心焉。或謂榮處國家大事,不愧唐姚崇,而不拘小節,亦頗類之。

家富,曾孫曄為建寧指揮,以貲敗。詳《宦官傳》。

曄從弟旦,字晉叔,弘治中進士。歷官太常卿。以忤劉瑾,左遷知溫州府,治最,稍遷浙江提學副使。瑾誅,累擢至戶部侍郎,督京、通倉,出理餉甘肅。還,進右都御史,總督兩廣軍務。討平番禺、清遠、河源諸瑤。嘉靖初,遷至南京吏部尚書。張璁、桂萼驟進,旦率九卿極言不可。會吏部尚書喬宇罷,召旦代之,未至,為給事中陳洸所劾,勒致仕。年七十餘卒。

楊溥,字弘濟,石首人。與楊榮同舉進士。授編修。永樂初,侍皇太子為洗馬。太子嘗讀《漢書》,稱張釋之賢。溥曰:「釋之誠賢,非文帝寬仁,未得行其志也。」採文帝事編類以獻。太子大悅。久之,以喪歸。時太子監國,命起視事。十二年,東宮遣使迎帝遲,帝怒。黃淮逮至北京繫獄。及金問至,帝益怒曰:「問何人,得侍太子!」下法司鞫,連溥,逮繫錦衣衛獄。家人供食數絕。而帝意不可測,旦夕且死。溥益奮,讀書不輟。繫十年,讀經史諸子數周。

仁宗即位,釋出獄,擢翰林學士。嘗密疏言事。帝褒答之,賜鈔幣。已,念溥由己故久困,尤憐之。明年建弘文閣於思善門左,選諸臣有學行者侍值。士奇薦侍講王進、儒士陳繼,蹇義薦學錄楊敬、訓導何澄。詔官繼博士,敬編修,澄給事中,日值閣中。命溥掌閣事,親授閣印,曰:「朕用卿左右,非止學問。欲廣知民事,為治道輔。有所建白,封識以進。」尋進太常卿,兼職如故。

宣宗即位,弘文閣罷,召溥入內閣,與楊士奇等共典機務。居四年,以母喪去,起復。九年遷禮部尚書,學士值內閣如故。

英宗初立,與士奇、榮請開經筵。豫擇講官,必得學識平正、言行端謹、老成達大體者數人供職。且請慎選宮中朝夕侍從內臣。太后大喜。一日,太后坐便殿,帝西向立,召英國公張輔及士奇、榮、溥、尚書胡濙入。諭曰:「卿等老臣,嗣君幼,幸同心共安社稷。」又召溥前曰:「仁宗皇帝念卿忠,屢加歎息,不意今尚見卿。」溥感泣,太后亦泣,左右皆悲愴。始仁宗為太子,被讒,宮僚多死詔獄,溥及黃淮一繫十年,瀕死者數矣。仁宗時時於宮中念諸臣,太后亦久憐之,故為溥言之如此。太后復顧帝曰:「此五臣,三朝簡任,俾輔後人。皇帝萬幾,宜與五臣共計。」正統三年,《宣宗實錄》成,進少保、武英殿大學士。溥後士奇、榮二十餘年入閣,至是乃與士奇、榮並。六年歸省墓,尋還。

是時,王振尚未橫,天下清平,朝無失政,中外臣民翕然稱「三楊」。以居第目士奇曰「西楊」,榮曰「東楊」,而溥嘗自署郡望曰南郡,因號為「南楊」。溥質直廉靜,無城府。性恭謹,每入朝,循牆而走。諸大臣論事爭可否,或至違言。溥平心處之,諸大臣皆歎服。時謂士奇有學行,榮有才識,溥有雅操,皆人所不及云。比榮、士奇相繼卒,在閣者馬愉、高穀、曹鼐皆後進望輕。溥孤立,王振益用事。十一年七月,溥卒,年七十五。贈太師,謚文定。官其孫壽尚寶司丞。後三年,振遂導英宗北征,陷土木,幾至大亂。時人追思此三人者在,當不至此。而後起者爭暴其短,以為依違中旨,釀成賊奄之禍,亦過刻之端也。

馬愉,字性和,臨朐人。宣德二年進士第一。授翰林修撰。九年秋,特簡史官及庶吉士三十七人進學文淵閣,以愉為首。正統元年充經筵講官,再遷至侍讀學士。時王振用事,一日,語楊士奇、榮曰:「朝廷事久勞公等,公等皆高年,倦矣。」士奇曰:「老臣盡瘁報國,死而後已。」榮曰:「吾輩衰殘,無以效力,當擇後生可任者,報聖恩耳。」振喜而退。士奇咎榮失言。榮曰:「彼厭吾輩矣,一旦內中出片紙令某人入閣,且奈何?及此時進一二賢者,同心協力,尚可為也。」士奇以為然。翼日,遂列侍讀學士苗衷、侍講曹鼐及愉名以進。由是愉被擢用。五年詔以本官入內閣,參預機務,尋進禮部右侍郎。十二年卒。贈尚書兼學士。贈官兼職,自愉始。

愉端重簡默,門無私謁。論事務寬厚。嘗奏天下獄久者多瘐死,宜簡使者分道決遣。帝納焉。邊警,方命將,而別部使至,眾議執之。愉言:「賞善罰惡,為治之本。波及於善,非法。乘人之來執之,不武。」帝然之,厚遣其使。

贊曰:成祖時,士奇、榮與解縉等同直內閣,溥亦同為仁宗宮僚,而三人逮事四朝,為時耆碩。溥入閣雖後,德望相亞,是以明稱賢相,必首三楊。均能原本儒術,通達事幾,協力相資,靖共匪懈。史稱房、杜持眾美效之君,輔贊彌縫而藏諸用。又稱姚崇善應變,以成天下之務;宋璟善守文,以持天下之正。三楊其庶幾乎。


\end{pinyinscope}