\article{列傳第三十八}

\begin{pinyinscope}
○郁新趙羾金忠李慶師逵古朴向寶陳壽馬京許思溫劉季箎劉辰楊砥虞謙呂升仰瞻嚴本湯宗

郁新,字敦本,臨淮人。洪武中,以人才征,授戶部度支主事。遷郎中。踰年,擢本部右侍郎。嘗問天下戶口田賦,地理險易,應答無遺,帝稱其才。尋進尚書。時親王歲祿米五萬石,新定議減五之四,並定郡王以下祿有差。又以邊餉不繼,定召商開中法,令商輸粟塞下,按引支鹽,邊儲以足。夏原吉為戶部主事,新重之,諸曹事悉委任焉。建文二年引疾歸。

成祖即位,召掌戶部事,以古朴為侍郎佐之。永樂元年,河南蝗,有司不以聞,新劾治之。初,轉漕北京,新言:「自淮抵河,多淺灘跌坡,運舟艱阻。請別用淺船載三百石者,自淮河、沙河運至陳州潁溪口跌坡下,復用淺船載二百石者運至跌坡上,別用大船運入黃河。至八柳樹諸處,令河南車夫陸運入衛河,轉輸北京。」從之。又言:「湖廣屯田所產不一,請皆得輸官。粟穀、穈黍、大麥、蕎二石,準米一石。稻穀、敔秫二石五斗,穇稗三石,各準米一石。豆、麥、芝麻與米等。」著為令。二年,議公、侯、伯、駙馬、儀賓祿,二百石以上者,請如文武官例,米鈔兼給。三年,以士卒勞困,議減屯田歲收不如額者十之四五,又議改納米北京贖罪者於南京倉。皆允行。是年八月卒於官。帝歎曰:「新理邦賦十三年,量計出入,今誰可代者?」輟朝一日,賜葬祭,而召夏原吉還理部事。

新長於綜理,密而不繁。其所規畫,後不能易。

趙羾,字雲翰,夏人,徙祥符。洪武中,由鄉舉入太學,授兵部職方司主事。圖天下要害阨塞,並屯戍所宜以進。帝以為才,遷員外郎。建文初,遷浙江參政,建策捕海寇,有功。

永樂二年,使交阯,還奏稱旨。擢刑部侍郎,改工部,再改禮部。五年進尚書,賜宴華蓋殿,撤膳羞遺其母。初,羾每以事為言者所劾,帝不問。九年秋,朝鮮使臣將歸,例有賜賚,羾不以奏。帝怒曰:「是且使朕失遠人心。」遂下之獄。尋得釋,使督建隆慶、保安、永寧諸州縣,撫綏新集,民安其業。十五年丁母艱。起復,改兵部尚書,專理塞外兵事。帝北征,轉餉有方。

仁宗嗣位,改南京刑部。宣德五年,御史張楷劾羾及侍郎俞士吉怠縱。召至,命致仕。

羾性精敏,歷事五朝,位列卿,自奉如寒素。正統元年卒,年七十三。

金忠,鄞人。少讀書,善《易》卜。兄戍通州亡,忠補戍。貧不能行,相者袁珙資之。既至,編卒伍。賣卜北平市,多中。市人傳以為神。僧道衍稱於成祖。成祖將起兵,託疾召忠卜,得鑄印乘軒之卦。曰:「此象貴不可言。」自是出入燕府中,常以所占勸舉大事。成祖深信之。燕兵起,自署官屬,授忠王府紀善,守通州。南兵數攻城不克。已,召置左右,有疑輒問,術益驗,且時進謀畫。遂拜右長史,贊戎務,為謀臣矣。

成祖稱帝,論佐命功,擢工部右侍郎,贊世子守北京。尋召還,進兵部尚書。帝起兵時,次子高煦從戰有功,許以為太子。至是,淇國公邱福等黨高煦,勸帝立之。獨忠以為不可,在帝前歷數古嫡孽事。帝不能奪,密以告解縉、黃淮、尹昌隆。縉等皆以忠言為是。於是立世子為皇太子,而忠為東宮輔導官,以兵部尚書兼詹事府詹事。六年命兼輔皇太孫。

帝北征,留忠與蹇義、黃淮、楊士奇輔太子監國。是時高煦奪嫡謀愈急,蜚語譖太子。十二年北征還,悉征東宮官屬下獄。以忠勳舊不問,而密令審察太子事。忠言無有。帝怒。忠免冠頓首流涕,願連坐以保之。以故太子得無廢,而宮僚黃淮、楊溥等亦以是獲全。

忠起卒伍至大位,甚見親倚,每承顧問,知無不言,然慎密不洩。處僚友不持兩端,退恒推讓之。明年四月卒。給驛歸葬,命有司治祠墓,復其家。洪熙元年,追贈榮祿大夫少師,謚忠襄。官子達翰林檢討。達剛直敢言,仕至長蘆都轉運使。

忠有兄華,負志節。忠守通州有功,欲推恩官之,辭不就。嘗召賜金綺,亦不受。成祖目為迂叟,放還。一日,讀《宋史》至王倫附秦檜事,放聲長歎而逝。里中稱為「白雲先生」。

李慶,字德孚,順義人。洪武中,以國子生署右僉都御史,後授刑部員外郎,遷紹興知府。永樂元年召為刑部侍郎。性剛果,有幹局,馭下甚嚴。帝以為才,數命治他事,不得時至部。然屬吏與罪人交通私饋餉,慶輒知之,繩以重法。五年,改左副都御史。兩遭親喪,並起復。時勳貴武臣多令子弟家人行商中鹽,為官民害。慶言:「舊制,四品以上官員家不得與民爭利。今都督蔡福等既行罰,公侯有犯,亦乞按問。」帝命嚴禁如制。忻成伯趙彞擅殺運夫,盜賣軍餉。都督譚青、朱崇貪縱。慶劾之,皆下吏。已,劾都督費瓛欺罔、梁銘貪暴、鎮守德州都督曹得黷貨。皆被責。中外凜其風采。十八年進工部尚書,尋兼領兵部事。

仁宗立,改兵部,加太子少保。弋謙以言事忤旨,呂震等交口詆之,惟慶與夏原吉無所言。帝尋悟,降敕自責,並責震等,震等甚愧此兩人。山陵事多,趣辦中官有求,執不與,人多嚴憚之,號為「生李」。奉命侍皇太子謁孝陵,在途約束將士,秋毫無所擾。太子欲獵,慶諫止。及太子還北京,遂留慶南京兵部。

宣德二年,安遠侯柳升討黎利,命慶參贊軍務,許擇部曹賢能者自隨。師至鎮夷關,升意輕賊,不為備。郎中史安、主事陳鏞言於慶。時慶已病甚,強起告升。升不聽,直前,中伏敗死。慶病遂篤,明日亦死,一軍盡沒。

師逵,字九達,東阿人。少孤,事母至孝。年十三,母疾,思藤花菜。逵出城南二十餘里求得之。及歸,夜二鼓,遇虎。逵驚呼天,虎舍之去。母疾尋愈。洪武中,以國子生從御史出按事,為御史所劾,逮至。帝偉其貌,釋之,謫御史臺書案牘。久之,擢御史,遷陜西按察使。獄囚淹繫千人,浹旬盡決遣,悉當其罪。母憂去官,廬墓側,不飲酒食肉者三年。成祖即位,召為兵部侍郎,改吏部。永樂四年建北京宮殿,分遣大臣出採木。逵往湖、湘,以十萬眾入山闢道路,召商賈,軍役得貿易,事以辦。然頗嚴刻,民不堪,多從李法良為亂。左中允周乾劾之。時仁宗監國,以帝所特遣,置不問。八年,帝北征,命總督饋餉,逵請量程置頓堡,更遞轉輸。從之。

逵佐蹇義在吏部二十年,人不敢干以私。仁宗嗣位,與趙羾、古朴皆改官南京,而逵進戶部尚書,兼掌吏部。宣德二年正月卒官,年六十二。

逵廉,不殖生產,祿賜皆分宗黨。有子八人,至無以自贍。成祖在北京嘗語左右曰:「六部扈從臣,不貪者惟逵而已。」古朴,字文質,陳州人。洪武中以太學生清理郡縣田賦圖籍,還隸五軍斷事理刑。自陳家貧,願得祿養母。帝嘉之,除工部主事。母歿,官給舟歸葬。服闋,改兵部,累遷郎中。建文三年擢兵部侍郎。

成祖即位,改戶部。永樂二年,朴奏:「先奉詔令江西、湖廣及蘇、松諸府輸糧北京,今聞並患水潦,轉運艱難,而北京諸郡歲幸豐。宜發鈔命有司增價收糴,減南方運。」從之。營建北京,命採木江西,以恤民見褒。七年,帝北巡,皇太子監國。召還,佐夏原吉理戶部。仁宗即位,改南京通政使。明年就拜戶部尚書,出督畿內田賦。師逵病,命朴代之。宣德三年二月卒於官。

初,戶部主事劉良不檢,乞中貴人求上考。朴不可。良遂誣奏朴罪,朴就逮。成祖察其誣,得釋。他日,吏部奏予良誥。仁宗曰:「此人素無行,且嘗誣大臣,不可與。」良後果以贓敗。朴在朝三十餘年,自郎署至尚書,確然有守,不通乾請,與右都御史向寶,俱以清介稱。

寶,字克忠,進賢人。洪武中,以進士授兵部員外郎。九年無過,擢通政使,以不善奏對力辭,改應天府尹。建文時,坐事謫廣西。成祖即位,召復職。已,復坐事下獄,降兩浙鹽運判官。仁宗在東宮,知其廉。及即位,召為右都御史兼詹事,並給兩俸。尋應詔陳八事,多可採者。宣德初,改南京。三年入覲,帝憫其老,命致仕。歸,卒於途。

寶有文學,寬厚愛民,而持身廉直,屢遭困阨不稍易,平居言不及利。歷仕四十餘年,卒之日,家具蕭然。

陳壽,隨人。洪武中,由國子生授戶部主事。永樂元年遷員外郎。出為山東參政,所至以愛民為務。用夏原吉薦,召為工部左侍郎。皇太子監國南京,壽日陳兵民困,又乘間言左右乾恩澤者多,恐累明德。太子深納之。嘗目送之出,顧侍臣曰:「侍郎中第一人也。」九年以漢王高煦譖,下獄,貧不能給朝夕。官屬有饋之者,拒不受,竟死獄中。踰年,啟殯如生。仁宗即位,贈工部尚書,謚敏肅,官其子瑺中書舍人,後亦至工部侍郎。

與壽同下獄死者,有馬京、許思溫。

京,武功人。洪武中,以進士授翰林編修,歷左通政、大理卿。永樂元年為行部左侍郎。皇太子守北京,命兼輔導,盡誠翊贊,太子甚重之。數為高煦所譖,謫戍廣西,仍坐前事,逮下獄。

思溫,字叔雍,吳人。以國子生署刑部主事,累官北平按察副使。燕師起,思溫佐城守有勞,擢刑部侍郎,改吏部,兼贊善。亦以讒下獄。皆瘐死。仁宗立,贈京少傅,謚文簡;思溫吏部尚書,官其子俊贊禮郎,進學翰林。

劉季箎,名韶,以字行,餘姚人。洪武中進士。除行人。使朝鮮,卻其饋贐。帝聞,賜衣鈔,擢陜西參政。陜有逋賦,有司峻刑督,民不能輸。季箎至,與其僚分行郡縣,悉縱械者,緩為期。民感其德,悉完納。陜不產岡砂,而歲有課。季箎言於朝,罷之。洪渠水溢,為治堰蓄洩,遂為永利。

建文中,召為刑部侍郎。民有為盜所引者,逮至,盜已死,乃召盜妻子使識之。聽其辭,誣也,釋之。吏虧官錢,誣千餘人,悉為辨免。河陽逆旅朱、趙二人異室寢。趙被殺,有司疑朱殺之,考掠誣服。季箎獨曰:「是非夙仇,且其裝無可利。」緩其獄,竟得殺趙者。揚州民家,盜夜入殺人,遺刀屍傍,刀有記識,其鄰家也。官捕鞫之。鄰曰:「失此刀久矣。」不勝掠,誣服。季箎使人懷刀就其里潛察之。一童子識曰:「此吾家物。」盜乃得。

永樂初,纂修《大典》,命姚廣孝、解縉及季箎總其事。八年坐失出下獄,謫外任。逡巡未行,復下獄。久之始釋。命以儒服隸翰林院編纂。尋授工部主事,卒於官。劉辰,字伯靜,金華人。國初,以署典簽使方國珍。國珍飾二姬以進,叱卻之。李文忠駐師嚴州,辟置幕下。元帥葛俊守廣信,盛冬發民浚城濠。文忠止之。不聽。文忠怒,欲臨以兵。辰請往諭之。俊悔謝,事遂已。以親老辭歸。

建文中,用薦擢監察御史,出知鎮江府,勤於職事。瀕江田八十餘頃,久淪於水,賦如故,以辰言得除。京口閘廢,轉漕者道新河出江,舟數敗。辰修故閘,公私皆便。漕河易涸,仰練湖益水,三斗門久廢。辰修築之,運舟既通,湖下田益稔。

永樂初,李景隆言辰知國初事,召至,預修《太祖實錄》。遷江西布政司參政,奏蠲九郡荒田糧。歲饑,勸富民貸饑者,蠲其徭役以為之息。官為立券,期年而償。辰居官廉勤尚氣,與都司、按察使不相得,數爭,坐免官。十四年起行部左侍郎,復留南京者三年。帝念其老,賜敕及鈔幣,今致仕。卒於途,年七十八。

楊砥,字大用,澤州人。洪武末,由進士授行人司右司副。上疏言:「揚雄為莽大夫,貽譏萬世。董仲舒《天人三策》及正誼明道之言,足以扶翼世教。今孔廟從祀有雄無仲舒,非是。」帝從之。歷官湖廣布政司參議。建文中,言:「帝堯之德始於親九族。今宜惇睦諸籓,無自剪枝葉。」不報。父喪歸。

成祖即位,起鴻臚寺卿,乞終制。服闋,擢禮部侍郎,坐視河渠失職,降工部主事,改禮部。永樂十年遷北京行太僕寺卿。時吳橋至天津大水決堤傷稼。砥請開德州東南黃河故道及土河,以殺水勢。帝命工部侍郎藺芳經理之。定牧馬法,請令民五丁養種馬一匹,十馬立群頭一人,五十馬立群長一人,養馬家歲蠲租糧之半。而薊州以東至山海諸衛,土地寬廣,水草豐美,其屯軍人養種馬一匹,租亦免半。帝命軍租盡蠲之,餘悉從其議。於是馬大蕃息。

砥剛介有守,尤篤孝行。十六年,母喪哀毀,未至家,卒。

虞謙,字伯益,金壇人。洪武中,由國子生擢刑部郎中,出知杭州府。

建文中請限僧道田,人無過十畝,餘以均給貧民。從之。永樂初召為大理寺少卿。時有詔。建文中,上言改舊制者悉面陳。謙乃言前事請罪。帝見謙怖,笑曰:「此秀才闢老、佛耳。」釋弗問。而僧道限田制竟罷。都察院論誆騙罪,準洪武榜例梟首以徇。謙奏:「比奉詔準律斷罪,誆騙當杖流,梟首非詔書意。」帝從之。天津衛倉災,焚糧數十萬石。御史言主者盜用多,縱火自蓋。逮幾八百人,應死者百。謙白其濫,得論減。

七年,帝北巡,皇太子奏謙為右副都御史。明年,偕給事中杜欽巡視淮、鳳抵陳州災傷,免田租,贖民所鬻子女。明年,謙請振,太子諭之曰:「軍民困極,而卿等從容請啟,彼汲黯何如人也。」尋命督兩浙、蘇、松諸府糧,輸南、北京及徐州、淮安。富民賂有司,率得近地,而貧民多遠運。謙建議分四等:丁多糧最少者運北京,次少者運徐州,丁糧等者運南京、淮安,丁少糧多者存留本土。民利賴之。又言:徐州、呂梁二洪,行舟多阻。請每洪增挽夫二百,月給廩;官牛一百,暇時聽民耕,大舟至,用以挽。人以為便。嘗督運木,役者大疫。謙令散處之,疫遂息。未幾,偕給事中許能巡撫浙江。

仁宗即位,召還,改大理寺卿。時呂升為少卿,仰瞻為丞,而謙又薦嚴本為寺正。帝方矜慎刑獄,謙等亦悉心奏當。凡法司及四方所上獄,謙等再四參復,必求其平。嘗語人曰:「彼無憾,斯我無憾矣。」嘗應詔上言七事,皆切中時務。有言其奏事不密,市恩於外者。帝怒,降少卿。一日,楊士奇奏事畢,不退。帝問:「欲何言,得非為虞謙乎?」士奇因具白其誣,且言謙歷事三朝,得大臣體。帝曰:「吾亦悔之。」遂命復職。宣宗立,謙言:「舊制,犯死罪者,罰役終身。今所犯不等,宜依輕重分年限。」報可。宣德二年三月卒於官。謙美儀觀,風采凝重。工詩畫,自負才望。工部侍郎蘇瓚以鄙猥班謙上,恒怏怏,人以是隘其量云。

呂升,山陰人。永樂初為溧陽教諭,歷官江西、福建按察僉事,所至有清慎聲。入為大理寺少卿。宣德八年致仕卒。

仰瞻,長洲人。永樂中由虎賁衛經歷遷大理寺丞。正統間,宦官王振用事,百官多奔走其門,惟瞻與大理卿薛瑄不往。會與瑄辨殺夫冤獄,益忤振,下獄,謫戍大同。景泰初,召為右寺丞,執法愈堅,在位者多不悅。移疾歸,加大理少卿。

嚴本,字志道,江陰人。少通群籍,習法律,以傅霖《刑統賦》辭約義博,注者非一,乃著《輯義》四卷。永樂十一年以薦征,試以疑律,敷析明暢。授刑部主事。侍郎張本掌部事,官吏少當意者,獨重本,疑獄輒俾訊之。奉命使徽州,時督辦後期,例罰工,本不忍迫民。或以為言,本曰:「吾辦矣。」蓋已寓書其子,鬻田為工作償也。

仁宗立,以刑部尚書金純及虞謙薦,改大理寺正。斷獄者多以「知情故縱」及「大不敬」論罪。本爭之曰:「律自叛逆數條外,無『故縱』之文。即『不敬』,情有重輕,豈可概入重比?」謙韙之,悉為駁正。良鄉民失馬,疑其鄰,告於丞,拷死。丞坐決罰不如法,當徒,而告者坐絞。本曰:「丞罪當。告者因疑而訴,律以誣告致死,是丞與告者各殺一人,可乎?」駁正之。莒縣屯卒奪民田,民訟於官,卒被笞。夜盜民驢,民搜得之。卒反以為誣,擒送千戶,民被禁死。法司坐千戶徒。本曰:「千戶生,則死者冤矣。」遂正其故勘罪。蘇州衛卒十餘人夜劫客舟於河西務,一卒死。懼事覺,誣鄰舟解囚人為盜,其侶往救見殺。皆誣服。本疑之曰:「解人與囚同舟。為盜,囚必知之。」按驗,果得實,遂抵卒罪。

本立身方嚴,非禮弗履。其使徽也,知府饋酒肴亦不受。年七十八卒。

湯宗,字正傳,浙江平陽人。洪武末,由太學生擢河南按察僉事,改北平。建文時上變,言按察使陳瑛受燕邸金錢,有異謀。詔逮瑛,安置廣西,而遷宗山東按察使。坐事,左遷刑部郎中,出知蘇州府。蘇連歲水,民流,逋租百餘萬石。宗諭富民出米代輸。富民知其愛民,不三月悉完納。

永樂元年,有言其坐視水患者,逮下獄,謫判祿州。以黃淮薦,召為大理寺丞。或言宗曾發潛邸事。帝曰:「帝王惟才是使,何論舊嫌。」時外國貢使病死,從人謂醫殺之。獄具,宗閱牘歎曰:「醫與使者何仇,而故殺之乎?」卒辨出之。尋命振饑河南,還署戶部事。解縉下獄,詞連宗,坐繫十餘年。仁宗立,復官,再遷南京大理卿。宣宗初,清軍山東。會天久不雨,極陳民間饑困狀。帝為蠲租免役,罷不急之務。宣德二年卒。

贊曰:永、宣之際,嚴飭吏治,職事修舉。若郁新之理賦,楊砥之馬政,劉季箎、虞謙之治獄,可謂能其官矣。李慶、師逵諸人,清介有執,皆列卿之良也。陳壽、馬京遭讒早廢,惜乎未竟其用。金忠奮身卒伍,進自藝術末流,而有士君子之行。當其侃侃持論於文皇父子間,忠直不撓,卒以誠信悟主,豈不偉哉。


\end{pinyinscope}