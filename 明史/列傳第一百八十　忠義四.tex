\article{列傳第一百八十 忠義四}

\begin{pinyinscope}
○張允登郭景嵩郭應響張光奎楊于楷等李中正馬足輕等方國儒王紹正常存畏劉定國何承光高日臨等龐瑜董三謨等尹夢鰲趙士寬等盧謙張有俊等龔元祥子炳衡姚允恭王信史記言李君賜等梁志仁單思仁等王國訓胡爾純等黎弘業馬如蛟等張紹登張國勛等王燾魏時光蔣佳徵吳暢春等徐尚卿王時化等阮之鈿郝景春子鳴鑾等張克儉鄺曰廣等徐世淳子肇梁餘塙等

張允登,漢州人。萬歷三十八年進士。歷知咸寧、咸陽,有善政。其成進士,出湯賓尹之門,賓尹弗善也,而東林以賓尹故,惡之。舉卓異,得刑部主事,累遷河西兵備副使。鄜、延歲飢,亟遭盜,允登拊循備至,士民德之。崇禎四年閏十一月督餉至甘泉,降卒潛與流賊通,殺知縣郭永固,劫餉。允登力禦,不敵死。鄜人素服迎其喪,哭聲震十里,罷市三日。

當是時,流賊日熾,總督洪承疇往來奔擊,日不暇給。逾月陷宜君,又陷葭州,僉事郭景嵩死之。明年二月陷鄜州,兵備副使郭應響死之。應響,福清人,萬曆丙午舉鄉試第一。寧塞餘賊來犯,應響禦之,斬賊常山虎等十五人。至是,混天猴率眾夜突至,應響登北關,集士卒拒守,手殺三賊,力不支遂死。事聞,贈光祿寺少卿,謚忠烈,予祭葬,蔭一子入監讀書。

張光奎,澤州人。仕至山東右參政。崇禎五年,流賊躪山西,監司王肇生以便宜署歙人吳開先為將,使擊賊,戰澤州城西。賊敗去,從沁水轉掠陽城。開先恃勇渡沁,戰北留墩下,擊斬數百人,炮盡無援,一軍盡沒。賊乃再犯澤州,光奎方里居,與兄守備光璽、千總劉自安等率眾固守八日,援兵不至,城陷,並死之。澤,大州也,遠近為震動。事聞,贈光祿卿,光璽等贈恤有差。

是歲,紫金梁等寇遼州,里居行人楊于楷與主事張友程,佐知州信陽李呈章拒守,力屈城陷,于楷被執,罵賊死。呈章、友程及舉人趙一亨、侯標並死之。明年六月,賊陷和順,里居昌平副使樂濟眾被傷,不屈,投井死。贈於楷光祿少卿,濟眾太僕少卿。有徐明揚者,浮梁人,由選貢生為平順知縣。六年四月,賊來犯,設策守禦,城破不屈死。

李中正,盧氏人。萬曆末,舉會試,以天啟二年赴廷對,授承天府推官,遷兵部主事。崇禎初,謝病歸。六年,群盜大亂河北。其冬,乘冰渡河,遂由澠池犯盧氏。中州承平久,不設備。驟聞賊至,吏民惶駴,知縣金會嘉棄城遁。十二月,賊入城,中正勒家眾及里中壯士奮擊,眾寡不敵,力戰死。賊縱掠城中,執舉人靳謙書,使跪,不屈,大罵而死。

賊以是冬始入河南,自是屢陷名城,殺將吏無算,鄉官舉貢多被難。其宜陽馬足輕,靈寶許煇,新安劉君培、馬山、李登英,偃師裴君合,陜州張我正、張我德,孟津孫挺生,嵩縣傅世濟、李佩玉,上蔡劉時寵輩,則先後以布衣抗節顯。

足輕,性孝友。弟惑婦言,迫分產,乃取田磽薄者自予。萬曆末,歲大凶,出粟六百石以振,焚券千餘。崇禎六年冬,流賊渡河而南,挈家避之石龍崖。三女皆殊色,慮賊污,悉投崖死。足輕被執,厲聲大罵。賊怒,并三子殺之。家眾皆遇害,惟存次子駿一人,後登鄉薦。煇為縣陰陽官,為賊所掠,大罵見殺。

君培有義行,攜子及從孫避難,道遇賊,欲殺其從孫。君培曰:「我尚有男,此子乃遺孤,幸舍之而殺我。」賊如其言,二子獲免。

山性剛直,土寇于大中陷新安,獲山,使負米。叱曰:「我天朝百性,肯為賊負米邪!」大罵而死。登英亦以罵賊死。

君合幼孤,母苦節,孝養惟謹。賊至,聚眾保沙岸寨。攻圍十晝夜不克,說之降,大罵不從。寨破,被磔。

我正素豪俠,集眾保鄉里,一方賴之。十四年勒眾禦賊,馘三人。俄賊大至,眾悉奔,奮臂獨戰。賊愛其男,欲生致之,詬罵自刎死。我德知賊至,恐妻子受辱,驅一家二十七人登樓自焚。

挺生精星術,預卜十五年有寇禍,編茅河渚以居。賊蹤跡得之,語其妻梁氏曰:「此匹夫徇義之秋也。」夫婦對泣,詬賊而死。世濟與兄世舟並為土寇于大中所執,將殺之。兄弟相抱泣,賊議釋其一,世濟即奪賊刀自殺,世舟獲免。

佩玉者,御史興元孫也。崇禎末,中州盡殘,佩玉結遺民捍鄉井,與鄰寨相掎角,往往尾賊後,奪其輜重。賊憚之,不敢出其境。後大舉圍別寨,佩玉往救,力戰而死,里人聚哭之。

時寵有孝行。賊陷城,其父宗祀以年老不能行,命之速避,遂自殺。時寵慟哭,刺殺一子、三女,夫婦並自剄。其妹適歸寧,亦從死,一家死者八人。

方國儒,字道醇,歙縣人。四歲失父,奉母以孝聞。天啟元年舉於鄉。崇禎間,授保康知縣。流賊大入湖廣,將吏率望風先奔。保康小邑素無兵,七年正月賊至,國儒急率鄉兵出御,力不支,城遂陷。亡何,賊退,國儒還入城。踰月復至,督吏民固拒。賊至益眾,復陷。國儒官服坐堂上,被執大罵,身中七刃死。

賊陷竹溪,訓導王紹正死之。穀城舉人常存畏會試赴京,道遇賊,欲劫為首領,罵不絕口死。他賊犯興山,知縣劉定國堅守。城將陷,遣吏懷印送上官,罵賊死。

何承光,貴州鎮遠人。萬曆四十年舉於鄉。崇禎中,歷夔州同知。七年二月,賊由荊州入夔門,犯夔州。副使周士登在涪州,城中倉猝無備,通判、推官、知縣悉遁。承光攝府事,率吏民固守,力竭城陷。承光整冠帶危坐,賊入殺之,投屍於江。事聞,贈承光夔州知府。

自賊起陜西,轉寇山西、畿輔、河南、北及湖廣、四川,陷州縣以數十許,未有破大郡者,至是天下為震動。

其他部自漢中犯大寧,知縣高日臨見勢弱不能守,齧指書牒乞援上官,率眾禦之北門。兵敗被執,大罵不屈,賊碎其體焚之。訓導高錫及妻女,巡檢陳國俊及妻,皆遇害。日臨,字儼若,鄱陽恩貢生。

賊陷夔州,他賊即以次日陷巫山,通江巡檢郭纘化陣沒,通江指揮王永年力戰死。至四月,守備郭震辰、指揮田實擊賊百丈關,兵敗被執,罵賊死。

龐瑜,字堅白,公安人。家貧,躬耕自給。夏轉水灌田,執書從牛後,朗誦不輟。由歲貢生授京山訓導。崇禎七年擢陜西崇信知縣。縣無城,兵荒,貧民止百餘戶。瑜知賊必至,言於監司陸夢龍,以無兵辭。瑜集士民築土垣以守,流涕誓死職。閏八月天大雨,土垣盡圮。賊掩至,瑜急解印遣家人齎送上官,端坐堂上以待。賊至,捽令跪。瑜罵曰:「賊奴敢辱官長!」拔刀脅之,罵益厲。賊掠城中無所有,執至野外,剖心裂屍而去。贈固原知州。

時賊盡趨秦中,長吏多殉城者。

山陽陷,知縣董三謨,黎平舉人也,及父嗣成、弟三元俱死之,妻李氏亦攜子女偕死。贈光祿丞,立祠,與嗣成、三元並祀,妻女建坊旌表。

吉永祚,輝縣人。為鳳縣主簿,謝事將歸。會賊至,知縣棄城遁,永祚倡義拒守。城陷,北面再拜曰:「臣雖小吏,嘗食祿於朝,不敢以謝事逃責。」大罵死之。子士樞、士模皆死。教諭李之蔚、鄉官魏炳亦不屈死。永祚贈漢中衛經歷,餘贈恤有差。

婁琇知涇州。閏八月,城陷死,贈太僕少卿。

蒲來舉知甘泉。賊來犯,守備孫守法等擁兵不救。城破,來舉手刃一賊,傷六賊而後死。贈光祿少卿。

呂鳴世,福建人。由恩貢生為麟游知縣。兵燹後,拊居民有恩。城陷,賊不忍加害,自絕食六日卒。

有宋緒湯者,耀州諸生,被獲,大罵死。

尹夢鰲,雲南太和人。萬歷時舉於鄉。崇禎中知潁州。八年正月方謁上官於鳳陽,聞流賊大至,立馳還。賊已抵城下,乃偕通判趙士寬率民固守。城北有高樓瞷城中,諸生劉廷傳請先據之,夢鰲以為然。而廷傳所統皆市人,不可用。賊遂據樓以攻,且鑿城,頹數丈,城上人皆走,止之不可。夢鰲持大刀,獨當城壞處,殺賊十餘人,身被數刃。賊眾畢登,遂投城下烏龍潭死,弟姪七人皆死之。

廷傳者,故布政使九光從子,任俠好義,亦罵賊死。九光子廷石分守西城,中賊刃未絕,口授友人方略,令繕牘上當事,旋卒。

士寬,字汝良,掖縣人。由門廕為鳳陽通判,駐潁州。以正旦詣郡城,聞警,一日夜馳三百里返州。城陷,率家眾巷戰,力竭,亦投烏龍潭死。妻李攜三女登樓自焚,僕王丹亦罵賊死。鄉官尚書張鶴鳴、弟副使鶴勝、子大同,中書舍人田之穎,知縣劉道遠,光祿署正李生白,訓導丁嘉遇,舉人郭三傑,諸生韓光祿等,皆死之。

光祖,進士獻策父也,被執,賊捽使跪。叱曰:「吾生平讀書,止知忠義。」遂大罵。賊殺之,碎其屍。妻武偕一妹、一女並獻策妻李赴井死。妾李方有娠,賊剖腹剔胎死。次子定策、孫日曦罵賊死,獨獻策獲存。時被難者共一百三人,城中婦人死節者三十七人,烈女八人。潁州忠烈,稱獨盛云。

潁州衛隸河南,流賊至,指揮李從師、王廷俊,千戶孫升、田三震,百戶羅元慶、田得民、王之麟俱乘城戰死。賊既陷潁州,屠其民。其別部即以是月由壽州犯鳳陽。

鳳陽故無城,中都留守朱國相率指揮袁瑞徵、呂承廕、郭希聖、張鵬翼、周時望、李郁、岳光祚,千戶陳弘祖、陳其忠、金龍化等,以兵三千逆賊上窯山,多斬獲。俄賊數萬至,矢集如蝟,遂敗,國相自刎死,餘皆陣沒。賊遂犯皇陵,大肆焚掠。

知府顏容暄囚服匿於獄,釋囚獲之,容暄大罵,賊杖殺之。血浸石階,宛如其像,滌之不滅。士民乃取石立塚,建祠奉祀。

推官萬文英臥病,賊索之。子元亨,年十六,泣語父曰:「兒不得復事親矣!」出門呼曰:「若索官,何為?我即官也。賊縶之。顧見其師萬師尹亦被縶,紿賊曰:「若欲得者,官爾。何縶此賤隸?」賊遂釋之。元亨乃極口大罵。賊怒,斷脛死,文英獲免。

容暄,漳浦人。文英,南昌人。皆進士。一時同死者,千戶陳永齡、百戶盛可學等四十一人,諸生六十六人。舉人蔣思宸聞變,投繯死。

後給事中林正亨錄上其狀,贈夢鰲光祿少卿,士寬光祿丞,餘贈恤有差。

盧謙,字吉甫,廬江人。萬曆三十二年進士。授永豐知縣。擢御史,出為江西右參政,引疾歸。崇禎八年二月,流賊犯廬江,士民具財帛求免,賊偽許之。俄襲陷其城,謙服命服,危坐中門。賊至,欲屈之,罵曰:「吾朝廷憲臣,肯為賊屈邪?鼠輩滅亡在即,安敢無禮!」賊怒殺之,投屍於池,池水盡赤。舉人張受、畢尹周亦不屈被殺。是年正月,賊陷霍丘,縣丞張有俊,教諭倪可大,訓導何炳,鄉官田既庭、戴廷對,舉人王毓貞死焉。賊陷巢縣,知縣嚴覺被執不屈,一門皆死。二月犯太湖,知縣金應元據城東大濠以守。奸人導賊渡濠,執應元,斫之未殊,自經死。訓導扈永寧亦死之。謙贈光祿卿,餘贈恤如制。覺,歸安人。應元,浙江山陰人。皆舉人。

龔元祥,字子禎,長洲人。舉於鄉。崇禎四年為霍山教諭,厲廉隅,以名教自任,與訓導姚允恭友善。八年,賊陷鳳陽,元祥偕縣令守禦。賊掩至,令逸去,元祥督士民固守。或勸之避,元祥曰:「食祿而避難,不忠。臨危而棄城,不義。吾平日講說者謂何?倘不測,死爾。」及賊陷城,元祥整衣冠危坐。賊至,侃侃諭以大義。賊欲屈之,厲聲曰:「死即死,賊輩何敢辱我!」賊怒,執之去,罵不絕口,遂遇害。子炳衡號呼罵賊,賊又殺之。閱五日,允恭斂其屍,即自縊,適令至,解免。越日,賊復入,允恭卒死之。事聞,贈元祥國子助教,建祠曰忠孝,以其子配。允恭亦被旌。

王信,陜西寧州人。父歿,廬墓三年。母歿,信年已六十,足不踰閾者三年。崇禎初,由歲貢生除靈璧訓導,遷真陽知縣。八年二月出撫土寇,會流賊猝至,被執,使諭降羅山、真陽。信大罵不從,斷頭剖腹而死。閱四日,其子來覓,猶舒指握子手。贈光祿丞,建祠奉祀。

史記言,字司直,當塗人。崇禎中舉人,由長沙知縣遷知陜州。陜當賊衝,記言出私財募士,聘少室僧訓練之。八年冬十一月,流賊犯陜,記言禦之,斬數十級,生擒二十餘人。老回回憤,率數萬人攻城,不克,乘雪夜來襲,而所練士方調他郡,城遂陷。記言縱火自焚,兩僧掖之出曰:「死此,何以自明?」乃越女牆下。賊追獲之,令降,叱曰:「有死知州,無降知州也!」遂被殺。指揮李君賜殺數賊而死。訓導王誠心,里居教諭張敏行、姚良弼,指揮楊道泰、阮我疆,鎮撫陳三元,亦不屈死。是月,賊陷盧氏,知縣白楹自剄。十年九月陷澠池,知縣李邁林死之。記言贈光祿少卿,餘贈恤有差。

梁志仁,南京人,保定侯銘之裔也。萬曆末年舉於鄉。崇禎六年授衡陽知縣,調羅田。賊大擾湖廣,志仁日夕儆備。羅汝才謂左右曰:「羅田城小易克,然梁君長者,吾不忍加兵。俟其去,當取之。」會邑豪江猶龍與賊通,志仁捕下獄。猶龍知必死,潛導汝才別校來攻。八年二月猝攻城。志仁急偕典史單思仁、教諭吳鳳來、訓導盧大受督民守御。城陷,志仁持長矛巷戰,殺六賊。力屈被縶,抑使跪。罵曰:「我天子命官,肯屈膝賊輩邪!」賊怒,碎其支體,焚之。妻唐被逼,大罵,奪賊刀不得,口齧賊手,遂遇害。思仁等亦不屈死。汝才在英山,聞之,馳至羅田,斬其別校,曰:「奈何擅害長者!」以錦繡斂其夫婦屍。鳳來,福建舉人。大受,寶慶貢生。詔贈志仁蘄州知州,思仁羅田主簿,鳳來國子助教,大受學錄,廕子,祭葬有差。

王國訓,字振之,解州人。天啟二年進士。歷知金鄉、壽張、滋陽、武清。坐大計,久之,補調扶風。國訓性剛嚴,恥干進,故官久不遷。崇禎八年秋,賊來犯,偕主簿夏建忠、典史陳紹南、教諭張弘綱、訓導陳繻嬰城固守。閱兩月,外援不至,城陷,罵賊死。建忠等亦不屈死。贈國訓光祿少卿,建忠等皆贈恤。

當是時,大帥曹文詔、艾萬年等並戰歿,賊勢益張,關中諸州縣悉殘破。八月,賊陷永壽,殺知縣薄匡宇。尋陷咸陽,殺知縣趙躋昌。

其時長吏以死聞者,隴州知州胡爾純,自經死。延長知縣萬代芳與教諭譚恩、驛丞羅文魁協力守城,城陷皆死之。代芳妻劉、妾梁從死。爾純,山東人,贈光祿少卿。代芳贈光祿丞,妻妾建坊旌表。恩等亦賜祭。

有孫仲嗣者,膚施人,由貢生為階州學正。當事知其才,委以城守。賊大至,盡瘁死守。城破,與妻子十餘人並死之。贈國子博士。又有楊呈秀,華陰人。由進士歷官順慶知府,大計罷歸。賊攻城,佐有司禦賊以死,贈恤如制。

黎弘業,字孟擴,順德人。由舉人知和州。崇禎八年,流賊犯和州,禦卻之。十二月復至,與鄉官馬如蛟募死士,登陴固守。城將陷,弘業繫印於肘,跪告其母曰:「兒不肖,貪微官以累母,奈何!」母李泣諭曰:「汝勿以我為意,事至此,有死而已。」遂自縊。妻楊、妾李及女四人繼之。弘業北面慟哭再拜,自刎未殊,濡頸血大書曰:「為臣盡忠,為子盡孝,何惜一死。」賊入,傷數刃而死。贈太僕少卿,任一子。判官錢大用偕妻妾子婦俱死。吏目景一高被創死。學正康正諫,祁門人,舉人。偕妻汪、子婦章赴水死,贈國子監丞。訓導趙世選不屈死,贈國子學錄。

馬如蛟,字騰仲,州人。天啟二年進士。授浙江山陰知縣,有清操。崇禎元年征授御史,劾罷魏忠賢黨徐紹吉、張訥。出按四川,蜀中奸民悉以他人田產投勢家,如蛟列上十事,永革其弊。還朝,監武會試。武舉董姓者,以技勇聞於帝,及入試,文不中程,被黜。帝怒,黜考官,如蛟亦落職。八年論平邦彥功,復故官,以父憂未赴。流賊至,如蛟傾貲募士,佐弘業固守。麾壯士出擊,兩戰皆捷。賊將奔,會風雪大作,不辨人色,守者皆潰,賊遂入城。如蛟急下令,能擊賊者,予百金,須臾得百餘人。巷戰,賊多傷,力屈,遂戰死。兄鹽運司判官如虯、諸生如虹及家屬十四人皆死。事聞,贈太僕少卿,官一子。

張紹登,字振夫,南城人。崇禎中舉人,知應城縣。九年,賊來犯,偕訓導張國勛、鄉官饒可久悉力禦之。國勛曰:「賊不一創,城不易守。」率壯士出擊,力戰一日夜,斬獲甚眾。賊去,邑侍郎王瑊之子權結怨於族黨,怨家潛導賊復來攻。國勛佐紹登力守,而乞援於上官。副將鄧祖禹來救,守西南,國勛守東北,紹登往來策應。會賊射書索權,權懼,斬北關以出,賊乘間登南城。紹登還署,端坐堂上,賊至,奮拳擊之。群賊大至,乃被殺。賊渠歎其忠,以冠帶覆屍,埋堂側。

國勛,黃陂歲貢生。賊既入,朝服北面拜,走捧先聖神主,拱立以待。賊遂焚文廟,投國勛於烈焰中。祖禹亦不屈死。

可久,幼孤,事母孝,舉於鄉。知大興縣。崇禎初,疏請更《三朝要典》,時奄宦擅權,謫光祿典簿。遷應天府推官、刑部主事,歷知府,丁艱歸。賊入,語妻程曰:「臣死忠,婦死節,分也。」于是妻女相對自經。可久被執,賊強之拜,曰:「頭可斷,膝不可屈也!」遂遇害。瑊為賊支解。

事聞,贈紹登尚寶少卿,國勛國子學正。

王燾,字濬仲,崑山人。少孤貧,九歲為人後。族人有謀其產者,燾舉以讓之,迎養嗣祖母及母惟謹。萬曆末,舉於鄉,由教諭歷隨州知州。州經群盜焚掠,戶不滿千。燾訓民兵,繕守具。土寇李良喬為亂,殲滅之。十年正月,大賊奄至。燾且守且戰,擊斬三百餘人。賊攻益力,相持二十餘日。天大風雪,守者多散。燾知必敗,入署,整冠帶自經。賊焚其署,火燭不及燾死所,屍直立不仆,賊望見駭走。已,覓州印,得之燾所立尺士下。事聞,贈太常少卿。福王時,賜謚烈愍,建雙忠祠,與同邑蔡懋德並祀。

有魏時光者,南昌人。善舞雙刀。崇禎九年夏,為廣濟典史。邑遭殘破,長吏設排兵三百人,委之教練。其冬,賊據蘄州河口,憚時光不敢渡。時光益募死士,夜襲其營,手殺數賊,賊不敢逼。俄賊大至,部卒皆散,時光單騎據高坡,又殺數人。賊環繞之,靷斷被執,不屈死。其兄陳於上官,卻不奏。兄憤發病死,友人收殮之,哭盡哀,曰:「弟為國死,兄為弟死,吾獨不能表暴之乎!」具牘力陳,乃奏聞。贈廣濟主簿,予恤典。

蔣佳徵,灌陽人。天啟四年舉於鄉。崇禎中,知盱眙縣,有聲。縣故無城,佳徵知賊必至,訓民為兵。十年秋,賊果來犯,設伏要害,親率兵往誘,賊殲甚眾。賊怒,環攻之,力戰死。母聞之,亦投繯死。兵部議贈奉訓大夫、尚寶少卿。未幾,巡按御史言佳徵子忠母義,宜賜謚廕,以植倫常。乃建表忠祠,并母奉祀。

同時江北死難者,有吳暢春。崇禎八年為潛山天堂寨巡檢,練鄉兵防賊。明年冬,賊至,夜設燎,大驚去之。踰年,賊再至,暢春死守,力屈,仰天歎曰:「吾得死所矣!」手刃數賊,被執不屈死。贈迪功郎、安慶府經歷,廕子所鎮撫。

又有王寅,錢塘人。膂力絕人,舉武鄉試,以父征播功為千戶。崇禎中,擢撫標守備。見步卒脆弱,詫曰:「曩戚將軍練浙兵,聞天下,今若爾邪!」督教之,卒始可用。十年遷龍江都司,調赴泗州護祖陵。賊來犯,寅曰:「賊眾我寡,及其未集,可破也。」捲甲疾趨,至盱眙,斬其先鋒一人。戰自午迄申,賊來益眾,與守備陳正亨陷陣死。贈鎮國將軍、都指揮僉事。正亨贈昭勇將軍、指揮使。並官一子。

徐尚卿,南平人。舉於鄉,知劍州。崇禎十年十月,李自成、惠登相等以數十萬眾入四川,大將侯良柱敗歿於廣元,遂攻陷昭化,知縣王時化死之。尚卿知賊必至,集士民泣曰:「城必不能守,若輩速去,吾死此。」眾泣,請偕去,尚卿不可。閱二日,城陷,投繯死,吏目李英俊從之。賊遂長驅陷江油、彰明、安縣、羅江、德陽、漢州,吏民皆先遁。尋掠郫縣,主簿張應奇死之。陷金堂,典史潘孟科死之。是月也,賊陷州縣三十六,以死事聞者四人。事定,贈尚卿右參議,時化光祿丞,應奇按察司知事,孟科將仕郎,並賜恤典。時化,湖廣人,舉鄉試第一。

阮之鈿,字實甫,桐城諸生。崇禎中,下詔保舉人才,同郡諭德劉若宰以之鈿應,授穀城知縣。十一年正月,之鈿未至,張獻忠襲陷其城,據以求撫。總理熊文燦許之,處其眾數萬於四郊,居民洶洶欲竄。之鈿至,盡心調劑,民稍安,乃上疏言:「獻忠虎踞邑城,其謀叵測。所要求之地,實兵餉取道咽喉,秦、蜀交會脈絡,今皆為所據。奸民甘心效用,善良悉為迫脅。臣守土牧民之官,至無土可守,無民可牧。庫藏殫虛,民產被奪,無賦可徵。名雖縣令,實贅員爾。乃廟堂之上專主撫議,臣愚妄謂撫剿二策可合言,未可分言,致損國威,而挫士氣。」時不能用。賊眾漸出野外行劫,之鈿執之以告其營將,稍置之法。及再告,皆不應,曰:「官司不給餉耳,得餉自止。」由是村民徙亡殆盡,遂掠及闤闠。稍拒,輒挺刃相向,日有死者,一城大囂。監軍僉事張大經奉文燦令來鎮撫,亦不能禁。

明年,獻忠反形漸露,之鈿往說之曰:「將軍始所為甚悖,今幸得為王臣,當從軍立功,垂名竹帛。且不見劉將軍國能乎?天子手詔進宮,厚齎金帛,此赤誠效也。將軍若疑天朝有異論,之鈿請以百口保。何嫌何疑,而復懷他志。」獻忠素銜之鈿,遂惡言極罵之。之鈿憂憤成病,題數語於壁,自誓以死,遂不視事。至五月,獻忠果反,劫庫縱囚,毀其城。之鈿仰藥未絕,獻忠遣使索印,堅不予,賊遂殺之。旋縱火焚公署,骸骨為燼。而大經為賊劫去,不能死。迨瑪瑙山戰敗,偕賊將曹威等出降,士論醜之。之鈿後贈尚寶少卿。

郝景春,字和滿,江都人。舉於鄉,署鹽城教諭,坐事罷歸。起陜西苑馬寺萬安監錄事,量移黃州照磨,攝黃安縣事。甫三日,群賊奄至,堅守八日夜,始解去。

崇禎十一年,擢知房縣。羅汝才率九營之眾請降於熊文燦,文燦受之。汝才猶豫,景春單騎入其營,偕汝才及其黨白貴、黑雲祥歃血盟。汝才詣軍門降,分諸營於竹谿、保康、上津,而自與貴、雲祥居房縣之野。當是時,鄖陽諸屬邑,城郭為墟,獨房賴景春拊循,粗可守。及大眾雜處,居民日惴惴。景春乃與主簿朱邦聞、守備楊道選修守具,輯諸營。

明年五月,張獻忠反穀城,約汝才同反。景春子鳴鑾,諸生也,力敵萬夫,謂父曰:「吾城當賊衝,而羸卒止二百,城何以守?」乃擐甲詣汝才曰:「若不念香火盟乎?慎毋從亂。」汝才佯諾。鳴鑾覺其偽,歸與道選授兵登陴,而獻忠所遣前鋒已至,擊斬其將上天龍。遣使縋城乞援於文燦,凡十四往,不報。

已而賊大至,獻忠兵張白幟,汝才兵張赤幟,俄二幟相雜,環城力攻。貴、雲祥策馬呼曰:「以城讓我,保無他也。」獻忠又以張大經檄諭降,景春大罵碎之。鳴鸞且守且戰,閱五日,賊多死。乃負板穴城,城將崩,鳴鑾熱油灌之。又擊傷獻忠左足,殺其所愛善馬。乃用間入賊壘,陰識獻忠所臥帳,將襲擒之。指揮張三錫啟北門揖汝才入,道選巷戰死。大經使汝才說景春降,怒不答。問庫藏儲蓄安在,叱曰:「庫藏若有物,城豈為汝陷!」賊怒,殺一典史、一守備恐之,卒不屈,與鳴鑾俱被殺。僕陳宜赤死之。邦聞及其家人並不屈死。事聞,贈景春尚寶少卿,建祠奉祀,道選等亦贈恤。已,帝召見輔臣賀逢聖,備述其死事狀,改贈太僕少卿。三錫後為官軍所獲,磔死。

張克儉,字禹型,屯留人。崇禎四年進士。授輝縣知縣。六年春,賊犯武安,守備曹鳴鶚戰死,遂犯輝縣。克儉乘城固守,賊不能下,屯百泉書院,三日而去。遷兵部主事,被薦召對,稱旨。十二年擢湖廣僉事,監鄖、襄諸軍。楊嗣昌鎮襄陽,深倚仗之。張獻忠、羅汝才之敗也,小秦王、渾世王、過天星等皆降,嗣昌處之房、竹山中,命克儉安輯。而諸賊得免死牌,莫肯散,自擇便地,連營數百里。時河南、北大饑,流民就食襄、漢者日數萬,降卒多闌入流民中。克儉深憂之,上書嗣昌曰:「襄陽自古要區,本朝筦鑰獻陵,視昔尤重。近兩河饑民雲集,新舊降丁逼處其間,一夫叫呼,即足致亂。況秦兵以長、武之變,西歸鄖、房。軍府粗立,降營棋置,奚啻放虎自衛。紫、漢、西、興,初無重門之備,何恃不恐。」嗣昌不以為意,報曰:「昔高仁厚六日降賊百萬,迄擒阡能,監軍何怯耶?」及嗣昌入蜀,委克儉以留務。錄破賊功,加右參議,監軍如故。未幾,以本官移守下川南道,鄖陽巡撫袁繼咸奏留之。

十四年二月擢右僉都御史,巡撫河南。未聞命,獻忠令人假督府軍符誑入襄陽城。克儉不能辨,夜分,賊從中起,焚襄王府。克儉倉皇奔救,為賊所執,大罵死。推官鄺曰廣、攝縣事李大覺、游擊黎民安死焉。

曰廣,番禺人。崇禎十年進士。居官有守。奉檄核軍儲於荊州,甫還任而難作,中刃死,妻子女俱遇害。大覺,字覺之,金谿人。由鄉舉知穀城,兼署襄陽縣。聞變,繫印於肘,縊死堂上。民安,大覺同縣人。城中火起,率所部千餘人搏戰,矢盡被縛,抗罵死。獨知府夏邑王承曾遁免。

初,獻忠敗於瑪瑙山,其妻敖氏、高氏被獲,他將搜山,又獲其軍師潘獨鰲,皆繫襄陽獄。承曾年少輕佻,每夕托問賊中情形,與獻忠二妻笑語。獄吏又多納賊金,禁防盡弛,獨鰲等脫桎梏恣飲。嗣昌移牒戒之,承曾笑曰:「是豈能飛至耶?」及是,獨鰲果從獄中起,承曾率眾奪門走。事聞,命逮治。時河南亦大亂,久逮不至,未知所終。

徐世淳,字中明,秀水人。父必達,字德夫,萬曆二十年進士。知溧水縣,築石臼湖隄,奏除齊泰姻戚子孫軍籍二十六家。累遷吏部考功郎中,與吏科給事中儲純臣同領察事。純臣受贓吏賕,當大計日,必達進狀請黜純臣,面揖之退,一座大驚。遷光祿丞,陳白糧利弊十一事,悉允行。進少卿,巡漕御史孫居相以船壞不治,請雇民船濟運,必達爭止之。天啟初,以右僉都御史督操江軍。白蓮賊將窺徐州,必達募銳卒會山東兵擊破之。遷兵部右侍郎,以拾遺罷歸,卒。

世淳,崇禎中舉人。十三年冬,歷隨州知州。州嘗被賊,居民蕭然。世淳知賊必復至,集士民誓以死守。會歲大荒,士多就食粥廠,嘆曰:「可使士以餒失禮乎?」分粟振之。潰兵過隨索餉,世淳授兵登陴,而單騎入見軍帥曰:「軍食不供,有司罪也。殺我足矣,請械我以見督師。」帥氣奪,斂眾去。

明年三月,張獻忠自襄陽來犯,世淳寢食南城譙樓,曉夜固守,告急於巡撫宋一鶴。一鶴遣兵來援,為監司守承天者邀去。守月餘,援絕力窮,賊急攻南城,而潛兵墮北城以入。世淳命子肇梁薶印廨後,勒馬巷戰,矢貫頤,耳鼻橫斷,墜馬,亂刃斫死。肇梁奔赴,且哭且罵,賊將殺之,呼州人告以薶印處,乃死。世淳妾趙、王及臧獲十八人皆死。後贈太僕少卿,建祠,以肇梁祔。

隨自十年正月陷,及是再陷,至七月復陷,判官餘塙死焉。三陷之後,城中幾無孑遺。


\end{pinyinscope}