\article{列傳第二百七 廣西土司三}

\begin{pinyinscope}
△泗城利州龍州歸順向武奉議江州思陵廣東瓊州府附

泗城州,宋置,隸橫山寨。元屬田州路。其界東抵東蘭,西抵上林長官司,南抵田州,北抵永寧州。

洪武五年,征南副將軍周德興克泗城州,土官岑善忠歸附,授世襲知州。十三年,善忠子振作亂,寇利州,廣西都司討平之。十四年,善忠來貢方物。二十六年,振遣人貢馬及方物,詔賜以鈔錠。

宣德元年,女土官盧氏遣族人岑臺貢馬及銀器等物,賜賚有差。八年,致仕女土官盧氏奏,襲職土官岑豹率土兵千五百餘人謀害己,又棄毀故土官岑瑄塑像,所為不孝,難俾襲職。豹叔利州知州顏亦奏豹興兵謀殺盧氏,州民被害。都督山雲奏:「豹實故土官瑄侄,人所信服,應襲職。盧氏,瑄妻,豹伯母,初借襲,今致仕,宜量撥田土以贍終身。仍請敕豹無肆侵擾。」兵部請從雲奏。帝命行人章聰、侯璡齎敕,諭雲會三司巡按究豹與盧氏是非,從公判決。

正統元年,豹遣人入貢。二年,豹攻利州,掠其叔顏妻子財物。朝廷官至撫諭,負固不服,增兵拒守。雲以聞,乞發兵剿之。帝敕雲曰:「蠻夷梗化,罪固難容,然興師動眾,事亦不易,其更遣人諭之。」五年,顏奏豹侵占及掠擄罪。頭目黃祖亦奏豹殺其弟,籍其家。瑄女亦奏豹占奪田地人民,囚其母盧氏。帝復遣行人朱昇、黃恕齋敕諭之,并敕廣西、貴州總兵官親詣其地,令速還所侵掠,如不服,相機擒捕。六年,總兵官柳溥奏:「行人恕、昇同廣西三司委官諭豹退還原占利州地,豹時面從,及回,占如故。今顏欲以利州、利甲等莊易泗城、古那等甲,開設利州衙門,宜從其請,發附近官軍送顏赴彼撫治蠻民。倘豹仍拒逆,則率兵剿捕。」從之。八年,豹遣人奉貢,賜彩幣。十年,豹復奏顏占據其地,帝令速予議處,不可因循,貽邊方害。

成化元年,豹聚眾四萬,攻劫上林長官司,殺土官岑志威,據其境土。兵部言:「豹強獷如此,宜調兵擒捕,明正典刑。」從之。未幾,豹死。

弘治三年,土官知州岑應復據上林長官司及貴州鎮寧等處一十八城。時恩城土官岑欽攻奪田州府,逐知府岑溥。應與欽黨,既復相仇,兩家父子交相仇殺。事聞,兵部奏:「欽連年構禍,而應黨之,復據上林長官司,流毒不少,今天厭禍,假手相殘,實地方之幸。應所占鄰壤及土官印信數多,亦宜勘斷,以除禍本,并令應弟接退還侵地及印信,乃許承襲。泗城地廣兵多,宜選頭目,量授職銜,分轄以殺其勢。」詔下總鎮官區處。接遣人朝正,賜彩緞鈔錠。

十年,總督鄧廷瓚奏:「接往年隨征都勻、府江等處有功,乞略其祖父罪,令承襲世職,以圖報效。」廷臣議:「劫印侵地,雖係接祖父罪,然再四撫諭,接不肯歸之於官,遽使襲職,則志益驕,非馭土官法。」

十二年,田州土目黃驥作亂,要接為聲援,殺掠男婦,劫燒倉庫民廬,又劫府學及橫山驛印記,遂據興仁。十四年,貴州賊婦米魯作亂,提督王軾請調接領土兵二萬營於砦布河,因敕接自備兩月餉,剋期赴調。

十八年,泗城土官族人岑九仙奏:「自始祖岑彭以來,世襲土官。至豹子應罹欽之禍,子孫滅亡殆盡,其弟接,眾推護印,累著勞勩,乞令襲職,俾掌轄蠻眾。」兵部尚書劉大夏等議:「豹乃叛臣餘孽,子應復自取滅亡。今接者,人皆傳稱為梁接,非應親枝,又不知岑九仙是何逋逃,冒為奏擾。臣大夏先在兩廣,見岑氏譜。岑之始祖木納罕於元至正年間,與田州知府之祖伯顏,一時受官。今九仙妄援漢岑彭世次,塵瀆聖聽,請治其罪。其岑接應襲與否,前已令鎮巡官勘奏,岑九仙雖蠻人難以深究,亦當摘發以破其奸。」從之。

正德十二年,泗城及程縣各遣官族來貢。後期,賞減半。泗城貢厚,仍全給之。

嘉靖二年,田州岑猛率兵攻泗城,拔六寨,進薄州城,克之。接告急軍門,言猛無故攻寨。猛言接非岑氏後,據其祖業,欲得所侵地。詔下勘處。

十六年,田州盧蘇作亂。泗城土舍岑施以兵納岑邦佐。兵敗,弗克納。二十七年詔土舍施襲替,免赴京,以嘗聽調有勞也。隆慶二年,泗城蠻黃豹、黃豸等據貴州程番府麻嚮、大華等司,時出擄掠,官軍剿之,豹等遁去。

萬曆二年,泗城土官岑承勳等貢馬及香爐等物。四十一年,土官岑雲漢貢方物。初,雲漢乃紹勳嫡嗣,紹勳寵庶孽雷漢,頭目黃瑪等從中煽禍,以至焚劫稱兵。雲漢紿母出印,扶弟以奔,撫按以聞。廷議請釋紹勳罪以存大倫,權雷漢、黃瑪等以息囂孽,雲漢從寬削銜,戴罪管事。詔可。天啟二年,巡撫何士晉請復雲漢知州職,量加都司職銜,令率土兵援黔。從之。

泗城延袤頗廣,兵力亦勁,與慶遠諸州互相雄長。其流惡自豹而應而接,且三世。領縣一,曰程縣;長官司二,曰安隆,曰上林。

程縣在泗城州之東北,舊號程丑莊。明初歸附,隸泗城州。洪武二十一年改為縣,編戶一里。後改屬慶遠府,尋復隸泗城州,設流官知縣。正統間,為岑豹所逼,棄官遁去,典史攝印,旋亦罹害。豹遂奪其印,據縣治。事聞,屢遣官諭之,歷岑應、岑接凡七十餘年不服。嘉靖二年,接為諸土官攻殺,督府遣官按問,得縣印,貯於官,後僅存荒土。泗城、南丹、那地俱欲得之,時治兵相攻云。

安隆長官司,東抵泗城,西抵雲南,南抵上林長官司,北抵貴州宣慰司,元泗城州地也。洪武元年,泗城州土官岑善忠以次子子得領安隆峒。三十年,子得來朝,貢馬。設治所。永樂元年設安隆長官事,以子得為長官,撫其眾。十二年貢馬,賜鈔幣,予世襲。

上林長官司,東北俱抵泗城界,西抵安隆長官司,南抵雲南。宋、元號上林峒,屬泗城州,明興因焉。永樂實置長官司,以泗城州土官岑善忠三子子成為長官,撫其民。永樂四年,子成遣子保貢方物,賜鈔幣,自是貢賜不絕。成化元年,泗城岑豹攻劫上林,殺長官志威,滅其族,劫印,占其境上。兵部移文議豹罪,仍以地與印給上林。弘治三年,上林長官司遣頭目入貢,禮部以過期至,給半賞。既而泗城岑應復奪據上林長官司,然正、嘉、隆、萬間朝貢猶時至。

利州,漢屬交阯,號阪麗莊。宋建利州,隸橫山寨,元因焉。土官亦岑姓,洪武初歸附。授知州,以流官吏目佐之,直隸布政司。宣德二年,利州知州岑顏遣頭目羅嚮貢馬。正統元年,泗城岑豹侵據利州地,并掠顏妻子財物。總兵官山雲以聞,帝敕鎮、巡官撫諭之。四年,顏遣族人岑忻貢銀器方物。五年,顏奏:「本州地二十五甲,被豹興兵攻占,母覃被囚,妻財被掠,累奉敕撫諭,猖獗不服。」帝遣行人黃恕、朱昇敕諭豹,事具前傳。七年,豹復與顏相仇殺,帝敕總兵官吳亮宣布恩威,令各罷兵,而豹終殺顏及其子得,奪州印去,遂以流官判州事。數十年間,屢經諸司勘奏,移檄督追,歷岑應、岑接二世如故。嘉靖二年歸併泗城。

龍州,古百粵地。漢屬交阯。宋置龍州,隸太平寨。元大德中,升州為萬戶府。洪武二年,龍州土官趙帖堅遣使奉表,貢方物。詔以帖堅為龍州知州,世襲。八年改隸廣西布政司。時帖堅言:「地臨交阯,所守關隘二十七處,有警須申報太平,達總司,比報下,已涉旬月,恐誤事機,乞依奉議、泗城二州,隸廣西便。」從之。十六年,帖堅以孝慈皇后喪,上慰表,貢馬及方物,賜綺帛鈔錠有差。

二十一年,帖堅病,無子,以其從子宗壽代署州事。帖堅卒,宗壽襲。鄭國公常茂以罪謫居龍州。帖堅妻黃氏有二女,一為太平州土官李圓泰妻,茂納其一為妾。時宗壽雖襲職,帖堅妻猶持土官印,與茂、圓泰專擅州事,數陵逼宗壽。會茂以病卒,其閽者趙觀海等亦肆侮宗壽。宗壽乃與把事等以計取土官印,上奏,言茂已死,并械觀海等至京。於是帖堅妻惶懼,使人告宗壽擄掠,又與圓泰謀劫茂妾并其奴婢往太平州,又盡掠趙氏祖父官誥諸物,又欲併取龍州之地。乃自至京,告宗壽實從子,不應襲,宗壽亦上章言狀。帝乃詔宗壽勿問,下吏議帖堅妻與圓泰罪,既而以遠蠻俱釋之。

久之,復有人告茂匿龍州未死,前宗壽所言皆妄。遂詔右軍都督府榜諭宗壽及龍州官民,言:「昔鄭國公常茂有罪,上以開平王之功,不忍遽置於法,安置龍州。土官趙帖堅故,其妻與茂結為婚姻,誘合諸蠻,肆為不道。帖堅姪宗壽襲職,與黃氏互相告訐,言茂已死。上以功臣子,猶加憐憫,釋二人告訐罪。今有人言茂實未死,宗壽等知狀。已遣散騎舍人諭宗壽捕茂,延玩使者久不復命,其意莫測。特命榜諭爾宗壽等知之,如茂果存,則送至京師以贖罪,如茂果死,宗壽亦宜親率大小頭目至京,具陳其由。」

廣西布政司言宗壽屢詔赴京,拒命不出,又言南丹、奉議等蠻梗化。帝復命致仕兵部尚書唐鐸往諭宗壽,訖不從命。詔發湖廣、江西所屬衛所馬步官軍六萬餘,各齎三月糧,期以秋初俱赴廣西。命都督楊文佩征南將軍印,為總兵官,都指揮韓觀為左將軍,都督僉事宋晟為右將軍,劉真為參將,率京衛馬步軍三萬人至廣西,會討龍州及奉議、南丹、向武等州叛蠻。師行,帝撰文遣使祭嶽鎮海瀆,復遣禮部尚書任亨泰、監察御史嚴震直安南,諭以討龍州趙宗壽之故,令陳日焜慎守邊境,毋助逆,勿納叛。遣人諭文調南寧衛兵千人,江陰侯吳高領之,柳州衛兵千人,安陸侯吳傑領之,皆令其建功自贖。又詔文等,如兵至龍州,宗壽親來見,具陳茂已死之由,則宥其罪。若詐遣人來,則進兵討之。既,鐸還京,言宗壽伏罪來朝,乞罷兵勿征。詔文移兵於奉議,仍命鐸至軍參軍事。宗壽偕耆民農里等六十九人來朝謝罪,貢方物。

宗壽死,子景升襲。景升死,無嗣,以叔仁政襲。仁政再傳為趙源,源死無子。思恩土官岑浚率兵攻田州回,劫龍州,奪其印,納故知府源妻岑氏。詔下鎮巡官剿賊,而議立為源後者。以源庶兄浦有二子,相居長當立。相弟楷不能無望,則謀於岑氏,以僕韋隊子璋詭云遺腹。岑氏恃兄子猛方兵雄,楷遂奏言,璋實源子,當立,為相所篡。事下督府勘,未決。璋賂鎮守太監傅倫舍人,詭稱有詔,檄猛調二萬兵,納璋入龍州。左江大震,相挈印奔況村。都御史楊旦討璋,猛殺之,相乃歸。相二子,長遂,次寶。相枝拇,寶亦枝拇,相絕愛之,曰:「肖我當立。」猛乃以寶去,髡為奴。

嘉靖元年,相死,州人立遂。楷弒之,州人立其族弟煖。時王守仁提督兩廣,幕客岑伯高用事,楷賂伯高,言煖非趙氏裔,當立者楷也。遣上思州知州黃熊兆核之。熊兆黨伯高,言楷當立,以州印畀楷。楷遂殺煖,龍州大亂。州目黃安等潛往田州購寶。寶時為奴楊布家十三年矣,安等行百金購得之。言之督府,都御史林富謂楷勢已張,毋持之急,乃令楷攝職,俟寶長讓之。楷復,時時謀殺寶。富諭楷,令以印還寶,寶謝以五千金,益以腴田三十一村。楷計寶弱易與,不如邀厚利而徐圖之,遂聽命。楷復求韋璋之子應育之,令往來寶所。寶妻黃氏,思明府土官黃朝女也,貳於寶而與應通。應乃厚結州目,又數遣人與向武州締好,乞兵為衛。寶日荒悍,刑狡男子王良為閽。楷知良恨寶,激使內應,良許之。楷以千人夜至寶寢門呼良,良開門納楷兵,執寶寢所,斬之,以他盜聞。應以兵千人據州,并結朝自援。

都御史蔡經屬副使翁萬達謀之。萬達謂楷狙詐,未可速圖。韋應巽懦寡慮,可旦夕擒,斷其中堅,然後可次第獲,督撫善之。萬達行部至太平,使人以他事召朝,諭之計,論應當死,言楷才勇,正須藉為龍州當一面耳。時諸言楷事者,故不為理,州人大嘩。萬達愈厚楷,楷信之,遂統精兵千人詣萬達言狀,并以三十一村地獻。萬達召楷及州目鄧瑀等入見,伏壯士劫之,曰:「汝罪大,宜自為計。誠死,尚可為爾子留一官。」楷自分無生理,乃手書諭其黨曰:「業已如此,亂無益也,可善輔我子以存趙。」萬達即杖楷,斃之,以楷書諭其州人。時楷子匡時,生四年矣,立之,一州悉定。乃以十三村還龍州,十八村隸崇善縣,於是龍州趙氏仍得襲。

歸順州,舊為峒,隸鎮安府。永樂間,鎮安知府岑志綱分其第二子岑永綱領峒事,傳子瑛,屢率兵報效。弘治九年,總督鄧廷瓚言:「鎮安府之歸順峒,舊為州治,洪武初裁革。今其峒主岑瑛每效勞於官,乞設州治,授以土官知州。凡出兵令備土兵五千,仍歲領土兵二千赴梧州聽調。」詔從之,增設流官吏目一員。瑛死,子璋襲。復從璋奏,以本州改隸布政司。

璋多智略。田州岑猛以不法獲譴,都御史姚鏌將舉兵討之。璋,猛婦翁也。鏌慮璋黨猛,召都指揮沈希儀謀。希儀雅知璋女失寵,恨猛,又知部下千戶趙臣雅善璋。希儀因使趙臣語璋圖猛,璋受命。時猛子邦彥守工堯隘,璋詐遣兵千人助邦彥,言:「天兵至,以姻黨故,且與爾同禍。今發精兵來,幸努力堅守。」邦彥欣納之。璋遣人報希儀曰:「謹以千人內應矣。」時田州兵殊死拒戰,諸將莫利當隘者,希儀獨引兵當之。約戰三合,歸順兵大呼曰:「敗矣!」田州兵驚潰,希儀麾兵乘之,斬首數千級,邦彥死焉。猛聞敗,欲自經。而璋先已築別館,使人請猛。時猛倉皇不知所出,遂挈印從璋,使走歸順。璋詭為猛草奏,促猛出印實封之。璋既知猛印所在,乃鴆殺猛,斬其首,并府印函之,間道馳軍門。為讒言所阻,竟不論功。

璋死,次子瓛襲。嘉靖四年,提督盛應期以瓛先助猛逆攻泗城,許自新,出兵討賊自贖。從之。十四年,四州盧蘇叛,糾瓛攻鎮安府。瓛破鎮安,并發岑真寶父母墳墓。事聞,革冠帶,許立功贖。瓛後從征交阯,率於軍。子代襲,萬曆間以貢馬違限,給半賞。

向武州,宋置,隸橫山寨。元隸田州路。其界東北抵田州,西抵鎮安,南抵鎮遠。洪武二年七月,土官黃世鐵遣使貢馬及方物。詔以世鐵為向武州知州,許世襲。二十一年,廣西布政司言向武州叛蠻梗化。時都督楊文佩征南將軍印,討龍州、奉議等處,復奉命移師向武。文調右副將軍韓觀分兵進討都康、向武、富勞諸州縣,斬世鐵。以兵部尚書唐鐸言,置向武州守禦千戶所。

永樂二年,土官知州黃彧遣頭目羅以得貢馬,賜鈔幣。宣德四年,故土官知州黃謙昌子宗蔭貢馬,賜鈔。嘉靖四年,田州岑猛叛,向武土官以兵助猛。提督盛應期議大征,檄向武出兵討賊,以功贖罪。十六年,田州盧蘇叛,鎮安土官岑真寶以兵納岑邦佐,蘇求助於向武。時土官黃仲金怨真寶,遂與合兵,破鎮安。事聞,革仲金冠帶。二十七年,以仲金聽調有勞,詔許承襲原職,免赴京。四十二年,又以剿平瑤寇功,加仲金四品服。

向武領縣一,曰富勞,元置。洪武間,為蠻僚所據。建文時復置,仍隸向武州。永樂初,省武林入焉。土官亦黃氏世襲。

奉議州,宋置。初屬靜江軍,後屬廣西經略安撫司。元屬廣西兩江道宣慰司。洪武初,土官黃志威舊為田州府總管,來歸附。二年詔授其子世鐵為向武州知州,世襲。三年,志威入朝貢。六年招撫奉議等州百十七處人民,皆款服。帝嘉志威功,命以安州、侯州、陽縣屬之。七年以志威為奉議州知州兼守禦,直隸廣西行省。二十六年,奉議州知州黃嗣隆遣人貢馬及方物,賜以鈔錠。

二十八年,廣西布政司言,奉議、南丹等處蠻人梗化。時都督楊文討龍州,伏罪,帝命移兵奉議剿賊,遣使諭文等:「近聞奉議、兩江溪峒等處,林木陰翳,蛇虺遺毒草莽中,雨過,流毒溪澗,飲之令人死。師入其地,行營駐札,勿飲山溪水泉,恐余毒傷人。宜鑿井以飲,爾等其慎察之。」文發廣西都司及護衛官軍二萬人,調田州、泗城等土兵三萬八千九百人從征。師至奉議州,蠻寇聞官軍至,悉竄入山林,據險自固。文督諸將分兵捕之,復調參將劉真等領兵分道攻南丹叛寇。初,文等駐師奉議州之東南,分兵追捕賊黨,且遣人招降其脅從者。賊皆焚廬舍,走山谷,憑險阻立柵自固。文督將士屢攻破之,賊眾潰散。左副將軍韓觀等遂分兵追討都康、向武、富勞、上林諸州縣,破其更吾、蓮花、大藤峽等寨,斬向武土官黃世鐵并其黨萬八千三百餘人,招降蠻民復業者六百四十八戶,徙置象州武山縣,蠻寇遂平。時兵部尚書致仕唐鐸參議軍事,以朝廷嘗命征剿畢日,置衛守之。乃會諸將相度形勢,置奉議等衛并向武、河池、懷集、武仙、賀縣等處守禦千戶所,設官軍鎮守。詔從其言。

宣德二年,署州事土官黃宗廕遣頭目貢馬。正統五年,宗廕科斂劫殺,甚且欲戕其母。母避之,殺母侍者以洩怒,為母所告。僉事鄧義奏其事,帝敕總兵官柳溥及三司按驗以聞。嘉靖四年,田州岑猛叛,奉議土官嘗助猛攻泗城州。至是提督盛應期言,許其自新,令出兵討賊,以功贖罪。後土官知州死,皆以土判官掌州事。論者以奉議彈丸地,三面交迫田州,獨南界鎮安,其勢甚蹙。明初置衛,銓官如宋、元故事,蓋欲中斷田、鎮,以伐其謀云。

江州界,東抵忠州,西抵龍州,南抵思明,北抵太平府。其州宋置,隸古萬寨。元屬思明路。明初,土官黃威慶歸附。授世襲知州,設流官吏目以佐之,直隸布政司。嘉靖四十二年,以平瑤、僮功,準江州土官子黃恩暫署本職。領縣一,曰羅白。洪武初,土官梁敬賓歸附,授世襲知縣。敬賓死,子復昌襲。永樂間,從征交阯被陷,子福里襲。

思陵州,宋置,屬永平寨。元屬思明路。洪武初,省入思明府。二十一年復置思陵州。二十七年,土官韋延壽貢馬及方物。宣德四年,護印土官韋昌來朝,貢馬,賜鈔幣。正統間,貢賜如制。其界東至忠州,西北至思明,南至交阯。

瓊州,居環海中。漢武帝平南粵,始置珠崖、儋耳二郡。歷晉、隋、唐、宋叛服不一,事具前史。元改置瓊州路,屬海北海南道宣慰司。天曆初,改乾寧軍民安撫司。洪武元年,征南將軍廖永忠平廣東,改乾寧安撫司為瓊州府,以崖州吉陽軍、儋州萬安軍俱為州,南建州為定安縣隸焉。

六年,儋州宜倫縣民陳昆六等作亂,攻陷州城。廣東指揮使司奏言:「近儋州山賊亂,已調兵剿。其儋、萬二州,山深地曠,宜設兵衛鎮之。」詔置儋、萬二州守禦千戶所。七年,儋州黎人符均勝等作亂,海南衛指揮張仁率兵討平之。又海南羅屯等洞黎人作亂,千戶周旺等討平之。澄邁縣賊王官舍亂,典史彭禎領民兵捕斬之。十五年,萬、崖二州民陳鼎叔等作亂,陷陵水縣,為海南衛官軍擊敗,追至藤橋,斬鼎叔等三百餘人,餘黨悉平。十七年,儋州宜倫縣黎民唐那虎等亂,海南衛指揮張信發兵討之。那虎及其黨鄭銀等敗遁,信追擒之,送京師。知州魏世吉受賄,縱銀去。帝謂兵部曰:「知州不能捕賊,及官軍捕至而反縱之乎?」命遣力士即其州杖世吉,責捕所縱者。

永樂三年,廣東都司言:「瓊州所屬七縣八洞生黎八千五百人,崖州抱有等十八村一千餘戶,俱已向化,惟羅活諸洞生黎尚未歸附。」帝命遣通判劉銘齎敕撫諭之。御史汪俊民言:「瓊州周圍皆海,中有大、小五指,黎母等山,皆生熟黎人所居。比歲軍民有逃入黎洞者,甚且引誘生黎,侵擾居民。朝廷屢使招諭,黎性頑狠,未見信從。又山水峻惡,風氣亦異,罹其瘴毒,鮮能全活。近訪宜倫縣熟黎峒首王賢祐、嘗奉命招諭黎民,歸化者多。請仍詔賢祐,量授以官,俾招諭未服,戒約諸峒,無納逋逃。其熟黎則令隨產納稅,悉免差徭;其生黎歸化者,免稅三年;峒首則量所招民數多寡授以職。如此庶幾黎人順服。」從之。遣知縣潘隆本齎敕撫諭。

四年,瓊州屬縣生黎峒首羅顯、許志廣、陳忠等三十三人來朝。初以生黎多未向化,遣銘招撫。至是向化者萬餘戶,顯等從銘來朝,且乞以銘撫其眾。帝遂授銘瓊州知府,專職撫黎,仍授顯等知縣、縣丞、巡檢等官,賜冠帶鈔幣,遣還。自是諸黎感悅,相繼來歸。瓊山、臨高諸縣生黎峒首王罰、鐘異、王琳等來朝,命為主簿、巡檢。六年,銘復率土黎峒首王賢祐、王惠、王存禮等來朝,貢馬。命賢祐為儋州同知,惠、存禮為萬寧縣主簿。八年,文昌縣斬腳寨黎首周振生等來歸,賜以鈔幣,俾仍往招諸峒。九年,臨高縣典史王寄扶奉命招至生黎二千餘戶,而以峒首王乃等來朝。命寄扶為縣主簿,并賜王乃等鈔。十一年,瓊山縣東洋都民周孔洙招諭包黎等村黎人王觀巧等二百三十戶,願附籍為民。從之。臨高民黃茂奉命招撫深峒、那呆等二十四峒生黎,率黎首王聚、符喜等來朝貢馬,黎民來歸者戶四百有奇。通計前後所撫諸黎共千六百七十處,戶三萬有奇,蓋皆本廟算云。

十四年,王賢祐率生黎峒首王撒、黎佛金等來朝貢,帝嘉納之。命禮部曰:「黎人遠處海南,慕義來歸,若朝貢頻繁,非存撫意。自今生黎土官峒首俱三年一貢,著為令。」十六年,感恩土知縣樓吉祿率峒首貢馬。十九年,寧遠土縣丞邢京率峒首羅淋朝貢。時崖州民以私忿相戰鬥,衛將利漁所欲,發兵剿之。瓊州知州王伯貞執不可,曰:「彼自相仇殺耳,非有寇城邑殺良民之惡,不足煩官軍。」衛將不從,伯貞乃遣寧遠縣丞黃童按視。果仇殺,逮治數人,黎人遂安。

宣德元年,樂會土主簿王存禮等遣黎首黎寧及萬州黎民張初等來貢,帝謂尚書胡濙曰:「黎人居海島,不識禮儀,叛服不常,昔專設官撫綏,今來朝,當加賚之。」九月,澄邁縣黎王觀珠、瓊山縣黎王觀政等聚眾殺瓊山土知縣許志廣,流劫鄉村,殺掠人畜,命廣東三司勘實討之。二年,指揮王瑀等追捕黎賊,兵至金雞嶺,賊率眾拒敵,敗之,生擒賊首王觀政及從賊二百六十二人,斬首二百六十七級,餘眾潰,奔走入山,招撫復業黎八百一十二戶,以捷聞,械送觀政等至京。帝謂尚書蹇義曰:「蠻性雖難馴,然至為變,必有激。宜嚴戒撫黎諸官,寬以馭之,若生事激變,國有常刑。」

正統九年,崖州守禦千戶陳政聞黎賊出沒,偕副千戶洪瑜領軍搜捕賊,乃圍熟黎村,黎首出見,政等輒殺之。又令軍旗孫得等十五人焚其廬舍,殺其妻孥數人,擄其財物。各黎激變,政及官軍百人,皆為所殺。巡按御史趙忠以聞,坐瑜激變律斬。

景泰三年敕萬州判官王琥曰:「以爾祖父能招撫黎人,特授土官。爾能繼承父志,亦既有年。茲特降敕付爾,撫諭該管村峒黎人,各安生業,不得仿傚別峒生黎所為。其官軍亦不得擅入村峒,擾害激變。」

天順五年敕兩廣巡撫葉盛,以海南賊五百餘占據城池,可馳至瓊,相機撫捕,勿使滋蔓。

弘治二年,崖州故土官陳迪孫、冠帶舍人陳崇祐朝貢。以其能撫黎人之逋逃復業者,厚賜之。十五年,黎賊符南蛇反,鎮兵討之,不下。戶部主事馮顒奏:「府治在大海南。有五指山峒,黎人雜居。外有三州、十縣、一衛、十一所。永樂間,置土官州縣以統之,黎民安堵如故。成化間,黎人作亂,三度征討。將領貪功,殺戮無辜。迨弘治間,知府張桓、餘濬貪殘苛斂,大失黎心,釀成今日南蛇之禍。臣本土人,頗知事勢,乞仍考原設應襲土官子舍,使各集土兵,可得數萬,聽鎮巡官節制。有能擒首惡符南蛇者,復其祖職。以蠻攻蠻,不數月可奏績矣。」詔從之。

嘉靖十九年,總督蔡經以崖、萬二州黎岐叛亂,攻逼城邑,請設參將一員,駐札瓊州分守。二十八年,崖州賊首那燕等聚眾四千人為亂,詔發兩廣官軍九千剿之。給事鄭廷鵠言:

瓊州諸黎盤居山峒,而州縣反環其外。其地彼高而我下,其土彼膏腴而我鹹鹵,其勢彼聚而我散。故自開郡來千六百餘年,無歲不遭黎害,然無如今日甚矣。今日黎患,非九千兵可辦,必添調狼土官兵,兼召募打手,集數萬眾,一鼓而四面攻之,然後可克。

嘗考剿除黎患,其大舉有二。元至元辛卯,曾空其穴,勒石五指山。其時雖建屯田府,立定安、會同二縣,惜其經略未盡,故所得旋失。嘉靖庚子,又嘗大渡師徒,攻毀巢岡,無處不至。於是議者謂德霞地勢平衍,擬建城立邑,招新民耕守。業已舉行,中道而廢,旋為賊資,以至復有今日。謹條三事:

一,崖黎三面郡縣,惟東面連郎溫、嶺腳二峒岐賊,實當萬州陵水之衝。崖賊被攻,必借二峒東訌以分我兵勢。計須先分奇兵攻二峒,而以大兵徑搗崖賊。彼此自救不暇,莫能相顧,則殲滅可期。傳聞賊首那燕已入凡陽構集岐賊。此必多方誤我,且訛言搖惑,以堅諸部助逆之心。宜開示慰安,以解狐疑之黨。

一,隋、唐郡縣,輿圖可考,今多陷入黎中。蕩平後悉宜恢復,并以德霞、千家、羅活等膏腴之地盡還州縣,設立屯田,且耕且守。仍由羅活、磨斬開路,以達定安,由德霞沿溪水以達昌化。道路四達,井邑相望,非徒懾奸銷萌,而王路益開拓矣。

一,軍威既振,宜建參將府於德霞,各州縣許以便宜行事,以鎮安人心。其新附之民中有異志者,或遷之海北地方屯田,或編入附近衛所戎籍,如漢徙潳山蠻故事。又擇仁明慈惠之長,久任而安輯之,則瓊人受萬世利矣。

疏下兵部議,詔悉允行。

二十九年,總兵官陳圭、總督歐陽必進等督兵進剿,斬賊五千三百八十級,俘一千四十九人,奪牛羊器械倍之,招撫三百七十六人。捷聞,帝嘉其功,賜圭、必進祿米蔭襲有差。

萬歷十四年,長田峒黎出掠,兵備道遣兵執戮之。草子坡諸黎召眾來報復,戰於長沙營,斬黎首百餘級,於是黃村、田尾諸峒黎皆出降。

瓊州黎人,居五指山中者為生黎,不與州人交。其外為熟黎,雜耕州地。原姓黎,後多姓王及符。熟黎之產,半為湖廣、福建奸民亡命,及南、恩、藤、梧、高、化之征夫,利其土,占居之,各稱酋首。成化間,副使塗棐設計犁掃,漸就編差。弘治間,符南蛇之亂,連郡震驚,其小醜侵突,無時而息云。


\end{pinyinscope}