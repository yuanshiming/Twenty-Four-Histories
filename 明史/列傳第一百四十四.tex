\article{列傳第一百四十四}

\begin{pinyinscope}
崔景榮黃克纘畢自嚴李長庚王志道劉之鳳

崔景榮,字自強,長垣人。萬歷十一年進士。授平陽府推官。擢御史,劾東廠太監張鯨罪。巡按甘肅、湖廣、河南,最後按四川,積臺資十八年。

播州亂,景榮監大帥遇劉綎、吳廣輩軍。綎馳金帛至景榮家,為其父壽,景榮上疏劾之。播州平,或請以播北畀安氏,景榮不可。會總督李化龍憂去,景榮為請蠲蜀一歲租,恤上東五路,罷礦使。化龍疏敘監軍功,弗及景榮。已,晉太僕少卿。

三年滿,擢右僉都御史,巡撫寧夏。銀定素驕,歲入掠。景榮親督戰破之,因議革導賊諸部賞,諸部懼,請與銀定絕。銀定既失導,亦叩關求市。寧夏歲市費不貲,景榮議省之。在任三年,僅一市而已。其後延鎮吉能等挾款求補市,卒勿許,歲省金錢十餘萬。

四十一年,入為兵部右侍郎,總京營戎政。改吏部,以疾辭去。踰年,起宣府大同總督。召還,晉兵部尚書。會遼、沈失,熊廷弼、王化貞議不協,命廷臣議經、撫去留,景榮數為言官所論。御史方震孺請罷景榮,以孫承宗代之。遂引疾歸。

天啟四年十一月,特起為吏部尚書。當是時,魏忠賢盜國柄,群小更相倚附,逐尚書趙南星。即家起景榮,欲倚為助。比至,忠賢飾大宅以待,景榮不赴。錦衣帥田爾耕來謁,又辭不見。帝幸太學,忠賢欲先一日聽祭酒講,議裁諸聽講大臣賜坐賜茶禮,又議減考選員額,汰京堂添注官。景榮皆力持不行,浸忤忠賢指。又移書魏廣微,勸其申救楊漣、左光斗。廣微不得已,為具揭。尋以景榮書為征,曰:「景榮教我也。」於是御史倪文煥、門克新先後劾景榮陰護東林,媚奸邪而邀後福。得旨,削奪為民。崇禎改元,復原職。四年卒,贈少保。

黃克纘,字紹夫,晉江人。萬歷八年進士。除壽州知州,入為刑部員外郎。累官山東左布政使,就遷右副都御史,巡撫其地。請停礦稅,論劾稅使陳增、馬堂,他惠政甚著。屢以平盜功,加至兵部尚書。四十年,詔以故官參贊南京機務,為御史李若星、魏雲中所劾,還家候命。居三年,始履任。四十四年冬,隆德殿災,上疏陳時政,語極痛切。不報。

召理京營戎政,改刑部尚書,預受兩朝顧命。李選侍將移宮,其內侍王永福、姚進忠等八人坐盜乾清宮珠寶下吏。克纘擬二人辟,餘俱末減。帝不從,命辟六人,餘遣戍。克纘言:「姜升、鄭穩山、劉尚理不持一物,劉遜拾地上珠,還之選侍,而與永福、進忠同戮,輕重失倫。況選侍篋中物,安知非先朝所賜?」當是時,諸璫罪重,謀脫無自,惟請帝厚待選侍,則獄情自緩。於是流言四布,謂帝薄待先朝妃嬪,而克纘首入其言。帝不悅,責克纘偏聽,命如前旨。

已,楊漣陳「移宮」始末。帝即宣諭廷臣,備述選侍凌虐聖母狀。且曰:「大小臣工,惟私李黨,責備朕躬。」克纘皇恐上言:「禮,父母並尊。事有出於念母之誠,跡或涉於彰父之過,必委曲周全,渾然無跡,斯為大孝。若謂黨庇李氏,責備聖躬,臣萬死不敢出。」御史焦源溥力駁其持論之謬,末言:「群豎持貲百萬,借安選侍為名,妄希脫罪,克纘墮其術而不覺。」克纘奏辨,因乞罷。略言:「源溥謂在神宗時為元子者為忠,為福籓者非忠。臣敢廣之曰:神宗既保護先帝,授以大位,則為神考而全其貴妃,富貴其愛子者,尤忠之大也。又謂在先帝時為二后者為忠,為選侍者非忠。臣亦廣之曰:聖母既正名定位,則光昭刑于之令德,勿虛傳宮幃之忿爭,尤忠之大也。若如源溥言,必先帝不得正其始,聖母不得正其終,方可議斯獄耳。」疏入,帝怒甚,責以輕肆無忌,不諳忠孝。克纘皇恐引罪,大學士劉一燝等亦代為言,乃已。無何,給事中董承業、孫傑、毛士龍,御史潘雲翼、楊新期,南京御史王允成並劾克纘是非舛謬。克纘不服,言曩不舉李三才,故為諸人所惡。源溥復劾克纘借三才以傾言官。克纘奏辨,再乞休,帝不問。

天啟元年冬,加太子太保。尋復以兵部尚書協理戎政。廷臣議「紅丸」,克纘述進藥始末,力為方從哲辨。給事中薛文周詆其滅倫常,暱私交,昧大義。克纘憤,援《春秋》不書隱公、閔公之弒,力詆文周,且白選侍無毆聖母事。給事中沈惟炳助文周復劾克纘。先是,帝宣諭百官,明言選侍毆崩聖母。及惟炳疏上,得旨:「選侍向有觸忤,朕一時傳諭,不無過激。追念皇考,豈能恝然?」於是外議紛紜,咸言前此上諭,悉出王安矯託,而諸請安選侍者,益得藉為詞。蓋是時王安已死,魏忠賢方竊柄,故前後諭旨牴牾如此。

克纘歷官中外,清彊有執。持議與爭「三案」者異,攻擊紛起。自是群小排東林,創《要典》,率推克纘為首功。時東林方盛,克纘移疾。詔加太子太傅,乘傳歸。四年十二月,魏忠賢盡逐東林,召克纘為工部尚書。視事數月,復移疾歸。三殿成,加太子太師。崇禎元年,起南京吏部尚書。有劾之者,不就,卒於家。

畢自嚴,字景曾,淄川人。萬曆二十年進士。除松江推官。年少有才幹,徵授刑部主事。歷工部員外郎中,遷淮徐道參議。內艱闋,分守冀寧。改河東副使,引疾去。起洮岷兵備參政。以按察使徙治榆林西路,進右布政使。泰昌時,召為太僕卿。

天啟元年四月,遼陽覆。廷議設天津巡撫,專飭海防,改自嚴右僉都御史以往。置水軍,繕戰艦,備戎器。及熊廷弼建三方布置策,天津居其一,增設鎮海諸營,用戚繼光遺法,水軍先習陸戰,軍由是可用。魏忠賢令錦衣千戶劉僑逮天津廢將,自嚴以無駕帖疏論之,報聞。四方所募兵日逃亡,用自嚴言,攝其親屬補伍。兵部主事來斯行有武略,自嚴請為監軍。山東白蓮妖賊起,令斯行率五千人往,功多。

初,萬曆四十六年,遼左用兵,議行登、萊海運。明年二月,特設戶部侍郎一人,兼右僉都御史,出督遼餉,語詳《李長庚傳》。及是,長庚遷,乃命自嚴代。敘前平賊功,進右都御史兼戶部左侍郎。時議省天津巡撫,令督餉侍郎兼領其事,即以委自嚴。又議討朝鮮,自嚴言不可遽討,當俟請貢輸誠,東征效力,徐許其封耳。京師數地震,因言內批宜慎,恩澤宜節,人才宜惜,內操宜罷,語甚切直。自嚴在事數年,綜核撙節,公私賴之。

五年,以右都御史掌南京都察院。明年正月,就改戶部尚書。忠賢議鬻南太僕牧馬草場,助殿工。自嚴持不可,遂引疾歸。

崇禎元年,召拜戶部尚書。自嚴以度支大絀,請核逋賦,督屯田,嚴考成,汰冗卒,停薊、密、昌、永四鎮新增鹽菜銀二十二萬,俱報可。二年三月,疏言:「諸邊年例,自遼餉外,為銀三百二十七萬八千有奇。今薊、密諸鎮節省三十三萬,尚應二百九十四萬八千。統計京邊歲入之數,田賦百六十九萬二千,鹽課百一十萬三千,關稅十六萬一千,雜稅十萬三千,事例約二十萬,凡三百二十六萬五千有奇。而逋負相沿,所入不滿二百萬,即盡充邊餉,尚無贏餘。乃京支雜項八十四萬,遼東提塘三十餘萬,薊、遼撫賞十四萬,遼東舊餉改新餉二十萬,出浮於入,已一百十三萬六千。況內供召買,宣、大撫賞,及一切不時之需,又有出常額外者。乞敕下廷臣,各陳所見。」於是廷臣爭效計畫。自嚴擇其可者,先列上十二事,曰增鹽引,議鼓鑄,括雜稅,核隱田,稅寺產,核牙行,停修倉廒,止葺公署,南馬協濟,崇文鋪稅,京運撥兌,板木折價。已,復列上十二事,曰增關稅,捐公費,鬻生祠,酌市稅,汰冗役,核虛冒,加抵贖,班軍折銀,吏胥納班,河濱灘蕩,京東水田,殿工冠帶。帝悉允行。

詔輯《賦役全書》。自嚴言:「《全書》之作,自行一條鞭始,距今已四十五年。有一事而此多彼少者,其弊為混派;有司聽奸吏暗灑瓜分,其弊為花派。當大為申飭。」因條八式以獻。帝即命頒之天下。

給事中汪始亨極論盜屯損餉之弊。自嚴言:「相沿已久,難於核實。請無論軍種民種,一照民田起科。」帝是其議。先是,忠賢亂政,邊餉多缺,自嚴給發如期。又疏言:「最耗財者無如客餉。諸鎮年例合三百二十七萬,而客餉居三之一,宜大裁省。其次則有撫賞、召買、修築諸費,皆不可不節。」帝褒納之。其冬,京師被兵,帝憂勞國事,旨中夜數發。自嚴奏答無滯,不敢安寢,頭目臃腫,事幸無乏。明年夏,以六罪自劾,乞罷,優旨慰留。先以考滿加太子少保,敘遵、永克復功,再進太子太保。

兵部尚書梁廷棟請增天下田賦,自嚴不能止。於是舊增五百二十萬之外,更增百六十五萬有奇,天下益耗矣。已,陳時務十事,意主利民,帝悉採納。又以兵餉日增,屢請清核,而兵部及督撫率為寢閣。復乞汰內地無用之兵,帝即令嚴飭,然不能盡行也。

御史餘應桂劾自嚴殿試讀卷,首薦陳於泰,乃輔臣周延儒姻婭。自嚴引疾乞休,疏四上,不允。時有詔,縣令將行取者,戶部先核其錢穀。華亭知縣鄭友元已入為御史,先任青浦,逋金花銀二千九百。帝以詰戶部,自嚴言友元已輸十之七貯太倉。帝令主庫者核實,無有,帝怒責自嚴。自嚴飾詞辨,帝益怒,遂下自嚴獄,遣使逮友元。御史李若讜疏救,不納。踰月,給事中吳甘來復抗疏論救,帝乃釋之。八年五月,敘四川平賊功,復官,致仕。又三年卒,賜恤如制。

李長庚,字酉卿,麻城人。萬曆二十三年進士。授戶部主事。歷江西左、右布政使,所在勵清操。入為順天府尹。改右副都御史,巡撫山東。盡心荒政,民賴以蘇。盜蔓武定諸州縣,討擒其渠魁。

四十六年,遼東用兵,議行登、萊海運。長庚初言不便,後言:「自登州望鐵山西北口,至羊頭凹,歷中島、長行島抵北信口,又歷兔兒島至深井,達蓋州,剝運一百二十里,抵娘娘宮,陸行至廣寧一百八十里,至遼陽一百六十里,每石費一金。」部議以為便,遂行之。

明年二月,特設戶部侍郎一人兼右僉都御史,出督遼餉,駐天津,即以長庚為之。奏行造淮船、通津路、議牛車、酌海道、截幫運、議錢法、設按臣、開事例、嚴海防九事。時議歲運米百八十萬石,豆九十萬石,草二千一百六十萬束,銀三百二十四萬兩,長庚請留金花,行改折,借稅課,言:「臣考會計錄,每歲本色、折色通計千四百六十一萬有奇。入內府者六百餘萬,入太倉者,自本色外,折色四百餘萬。內府六百萬,自金花籽粒外,皆絲綿布帛蠟茶顏料之類,歲久皆朽敗。若改折一年,無損於上,有益於下。他若陜西羊絨,江、浙織造,亦當稍停一年,濟軍國急。」帝不悅,言:「金花籽粒本祖宗舊制,內供正額及軍官月俸,所費不貲,安得借留?其以今年天津、通州、江西、四川、廣西上供稅銀,盡充軍費。」於是戶科給事中官應震上言:「考《會典》,於內庫則云:金花銀,國初解南京供武俸,諸邊或有急,亦取給其中。正統元年,始自南京改解內庫。嗣後除武官俸外,皆為御用。是金花銀國初常以濟邊,而正統後方供御用也。《會典》於太倉庫則云:嘉靖二十二年,題準諸處京運錢糧,不拘金花籽粒,應解內府者悉解貯太倉庫,備各邊應用。是世宗朝金花盡充兵餉,不知陛下初年何故斂之於內也。今不考各邊取給應用之例,而反云正供舊額,何相左若是?至武官月俸,歲不過十餘萬,乃云所費不貲哉。且原數一百萬,陛下始增二十萬,年深日久,顛末都忘。以臣計之,毋論今年當借,即嗣後年年借用可也;毋論未來者當濟邊,即見在內帑者盡還太倉可也。若夫物料改折,隆慶元年曾行之以解部濟邊,六年又行於南京監局,亦以濟邊。此則祖宗舊制,陛下獨不聞耶?」帝卒不聽。

時諸事創始,百務坌集,長庚悉辦治。天啟二年,遷南京刑部尚書,就移戶部。明年,召拜戶部尚書,未任,以憂歸。

崇禎元年,起工部尚書,復以憂去。久之,代閔洪學為吏部尚書。六年正月,修撰陳於泰疏陳時弊,宣府監視中官王坤力詆之,侵及首輔周延儒。長庚率同列上言:「陛下博覽古今,曾見有內臣參論輔臣者否?自今以後,廷臣拱手屏息,豈盛朝所宜有。臣等溺職,祈立賜譴黜,終不忍開內臣輕議朝政之端,流禍無窮,為萬世口實。」帝不懌。次日召對平臺。時副都御史王志道劾坤語尤切,帝責令回奏。奏上,帝益怒。及面對,詰責者久之,竟削其籍。

志道,漳浦人,天啟時為給事中。議「三案」為高攀龍所駁,謝病歸。其後附魏忠賢,歷擢左通政,論者薄之。及是,以忤中官罷。

長庚不植黨援,與溫體仁不甚合。推郎中王茂學為真定知府,帝不允。復推為順德知府,帝怒,責以欺蒙,並追咎冠帶監生授職事,責令回奏。奏上,斥為民。家居十年,國變,久之卒。

劉之鳳,字雍鳴,中牟人。萬曆四十四年進士。歷南京御史。天啟三年六月,上疏別白孫承宗、王象乾、閻鳴泰本末,請定去留,而撤毛文龍海外軍,令居關內。又請亟罷內操。忤魏忠賢,傳旨切責,復宣諭廷臣,再瀆奏者罪無赦。六年,之鳳方視江防,期滿奏報。忠賢奪其職。

崇禎二年,起故官。帝召周延儒燕見,宵分始出。之鳳偕同官上疏曰:「臣等待罪陪京,去延儒原籍三百里,其立身居鄉,不堪置齒頰。今乃特蒙眷注,必將曰舉朝盡欺,獨延儒一人捐軀為國,使陛下真若廷臣無可信,而延儒乃得翦所忌,樹所私,曰為馮銓、霍維華等報怨。此一召也,於國事無纖毫益,而於聖德有丘山之損。」忤旨,詰責。已,復列上五事,曰舉謀勇,止援兵,練土著,密偵探,選守令,俱見採納。

累遷刑部侍郎,遂代鄭三俊為本部尚書。之鳳以天下囚徒皆五年一審錄,高墻罪獨不與,上疏言之,報可。嘗與左侍郎王命璿召對平臺,論律例及獄情,帝申飭而退。時有火星之變,之鳳特請修刑,言:「自今獄情大者,一月奏斷,小者半月。贓重人犯,結案在數年前者,大抵本犯無髓可敲,戚屬亦無脂可吸。祈悉宥免,全好生之仁。」從之。然之鳳雖為此奏,其後每上獄詞,帝必嚴駁,之鳳懼甚,諸司呈稿,遲疑不敢遽發,屢疏謝病,帝不從。會尚書范景文劾南京給事中荊可棟貪墨,下部訊,之鳳予輕比。帝疑其受賄,下之吏,法司希旨坐絞。給事中李清言於律未合,同官葛樞復論救。帝怒,鐫樞級,調外。十三年四月,之鳳獄中上書自白無贓賄,情可矜原。亦置不省,竟瘐死。

計崇禎朝刑部易尚書十七人。薛貞以奄黨抵死。蘇茂相半歲而罷。王在晉未任,改兵部。喬允升坐逸囚遣戍。韓繼思坐議獄除名。胡應台獨得善去。馮英被劾遣戍。鄭三俊坐議獄逮繫。之鳳論絞,瘐死獄中。甄淑坐納賄下詔獄,改繫刑部,瘐死。李覺斯坐議獄削籍。劉澤深卒於位。鄭三俊再為尚書,改吏部。范景文未任,改工部。徐石麒坐議獄,落職閒住。胡應台再召不赴。繼其後者張忻,賊陷京師,與子庶吉士端並降。

贊曰:崔景榮、黃克纘皆不為東林所與,然特不附東林耳。方東林勢盛,羅天下清流,士有落然自異者,詬誶隨之矣。攻東林者,幸其近己也,而援以為重。於是中立者類不免蒙小人之玷。核人品者,乃專以與東林厚薄為輕重,豈篤論哉?畢自嚴、李長庚計臣中辦治才,而自嚴增賦之議,識者病焉。劉之鳳議獄不當,罪止謫罷,竟予重比,刑罰不中,欲求治得乎!


\end{pinyinscope}