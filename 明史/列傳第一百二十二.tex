\article{列傳第一百二十二}

\begin{pinyinscope}
盧洪春范俊董基王就學等李懋檜李沂周弘禴潘士藻雒于仁馬經綸林熙春林培劉綱戴士衡曹學程子正儒郭實翁憲祥徐大相

盧洪春,字思仁,東陽人。父仲佃,廣西布政使。洪春舉萬曆五年進士,授旌德知縣,擢禮部祠祭主事。十四年十月,帝久不視朝,洪春上疏曰:「陛下自九月望後,連日免朝,前日又詔頭眩體虛,暫罷朝講。時享太廟,遣官恭代,且云『非敢偷逸,恐弗成禮』。臣愚捧讀,驚惶欲涕。夫禮莫重於祭,而疾莫甚於虛。陛下春秋鼎盛,諸癥皆非所宜有。不宜有而有之,上傷聖母之心,下駭臣民之聽,而又因以廢祖宗大典,臣不知陛下何以自安也。抑臣所聞,更有異者。先二十六日傳旨免朝,即聞人言籍籍,謂陛下試馬傷額,故引疾自諱。果如人言,則以一時馳騁之樂,而昧周身之防,其為患猶淺。倘如聖諭,則以目前衽席之娛,而忘保身之術,其為患更深。若乃為聖德之累,則均焉而已。且陛下毋謂身居九重,外廷莫知。天子起居,豈有寂然無聞於人者?然莫敢直言以導陛下,則將順之意多,而愛敬之心薄也。陛下平日遇頌諛必多喜,遇諫諍必多怒,一涉宮闈,嚴譴立至,孰肯觸諱,以蹈不測之禍哉?群臣如是,非主上福也。願陛下以宗社為重,毋務矯託以滋疑。力制此心,慎加防檢。勿以深宮燕閒有所恣縱,勿以左右近習有所假借,飭躬踐行,明示天下,以章律度,則天下萬世,將慕義無窮。較夫挾數用術,文過飾非,幾以聾瞽天下之耳目者,相去何如哉!」疏入,帝震怒。傳諭內閣百餘言,極明謹疾遣官之故。以洪春悖妄,命擬旨治罪。閣臣擬奪官,仍論救。帝不從,廷杖六十,斥為民。諸給事中申救,忤旨,切讓。諸御史疏繼之,帝怒,奪俸有差。洪春遂廢於家,久之卒。光宗嗣位,贈太僕少卿。

御史范俊嘗陳時政。帝方疾,見俊疏中「防人欲」語,斥之。主事董基以諫內操謫官。其後員外郎王就學因諫帝託疾不送梓宮,尋罷去。皆與洪春疏相類。

范俊,字國士,高安人。萬曆五年進士。為義烏知縣,徵授御史。十二年正月,陳時政十事,語皆切至,而中言「人欲宜防,力以靡曼麴蘗為戒」。先是,慈寧宮災,給事中鄒元標疏陳六事,忤帝意。及帝遘微疾,大臣方問安,而俊疏適入。帝恚曰:「向未罪元標,致俊復爾,當重懲之。」申時行等擬鐫秩。帝猶怒,將各予杖。是夜大雷雨,明日朝門外水三尺餘。帝怒少霽,時行等亦力救,乃斥為民。明年,給事中張維新請推用譴謫諸臣,詔許量移,惟俊不敘。給事中孫世禎、御史方萬山等言俊不宜獨遺,坐奪俸。自是屢薦不起,里居數十年卒。天啟初,復官,贈光祿少卿。

董基,字巢雄,掖縣人。萬曆八年進士。授刑部主事。十二年,帝集內豎三千人,授以戈甲,操於內廷。尚書張學顏諫,不納。基抗疏曰:「內廷清嚴地,無故聚三千之眾,輕以凶器嘗試,竊為陛下危之。陛下以為行幸山陵,有此三千人可無恐乎?不知此皆無當實用。設遇健卒勁騎,立見披靡,車駕不可恃以輕出也。夫此三千人安居美食,筋力柔靡,一旦使執銳衣堅,蒙寒犯暑,臣聞頃者竟日演練,中曷瀕死者數人,若輩未有不怨者。聚三千蓄怨之人於肘腋,危無逾此者。且自內操以來,賞賚已二萬金。長此不已,安有殫竭?有用之財,糜之無用之地,誠可惜也。」疏入,忤旨,命貶二秩,調邊方。九卿、給事、御史交章論救,且請納基言,不聽。竟謫基萬全都司都事。明年,兵科給事中王致祥言:「祖宗法,非宿衛士不得持寸兵。今授群不逞利器,出入禁門,禍不細。」大學士申時行亦語司禮監曰:「此事繫禁廷,諸人擐甲執戈,未明而入。設奸人竄其中,一旦緩急,外廷不得聞,宿衛不及備,此公等剝膚患也。」中官悚然,乘間力言。帝乃留致祥疏,即日罷之。會謫降官皆量移,基亦遷南京禮部主事,終南京大理卿。致祥,忻州人。隆慶五年進士。歷官右僉都御史,巡撫順天。

王就學,字所敬,武進人。萬曆十四年進士。授戶部主事。三王並封議起,朝論大嘩。就學,王錫爵門人也,偕同年生錢允元往規之,為流涕。會庶吉士李騰芳投錫爵書,與就學語相類。錫爵悟,並封詔得寢。就學改禮部,進員外郎,尋調吏部。二十四年,孝安陳太后梓宮發引,帝嫡母也,當送門外,以有疾,遣官代行。吏部侍郎孫繼皋言之,帝怒,抵其疏於地。就學抗疏曰:「人子於親惟送死為大事。今乃靳一攀送,致聖孝不終。豈獨有乖古禮,即聖心豈能自安。於此而不用其情,烏乎用其情?於此而可忍,烏乎不可忍?恐難以宣諸詔諭,書諸簡冊,傳示天下萬世也。」疏奏,不省。踰二年,詔甄別吏部諸郎,斥就學為民。尋卒於家。

繼皋抗疏未幾,給事中劉道亨劾文選員外郎蔡夢麟紊銓政,并及繼皋。乞罷,不報。及三殿災,大臣自陳,皆慰留,獨繼皋致仕去。卒,贈禮部尚書。繼皋,字以德,無錫人。萬曆二年進士第一。

李懋檜,字克蒼,安溪人。萬曆八年進士。除六安知州,入為刑部員外郎。十四年三月,帝方憂旱,命所司條上便宜。懋檜及部郎劉復初等爭言皇貴妃及恭妃冊封事,章一日並上。帝怒,欲加重譴,言者猶不已。閣臣請帝詔諸曹建言止及所司職掌,且不得專達,以慰解帝意。居數日,帝亦霽威,諸疏皆留中。而懋檜疏又有保聖躬、節內供、御近習、開言路、議蠲振、慎刑罰、重舉刺、限田制七事,亦寢不行。

明年,給事中邵庶因論誠意伯劉世延,刺及建言諸臣。懋檜上言:「庶因世延條奏,波及言者,欲概絕之。『防人之口,甚於防川』,庶豈不聞斯語哉?今天下民窮財殫,所在饑饉,山、陜、河南,婦子仳離,僵仆滿道,疾苦危急之狀,蓋有鄭俠所不能圖者,陛下不得聞且見也。邇者雷擊日壇,星墜如斗,天變示儆於上;畿輦之間,子殺父,僕殺主,人情乖離於下。庶以為海內盡無可言已乎?夫在廷之臣,其為言官者十僅二三。言官不必皆智,不為言官者不必皆愚。無論往事,即如邇歲馮保、張居正交通亂政,其連章保留,頌功詡德,若陳三謨、曾士楚者,並出臺垣,而請劍引裾杖謫以去者,非庶僚則新進書生也。果若庶言,天下幸無事則可,脫有不虞之變,陛下何從而知?庶復以堂上官禁止司屬為得計,伏睹《大明律》,百工技藝之人,若有可言之事,直至御前奏聞,但有阻遏者斬。《大明會典》及皇祖《臥碑》亦屢言之。百工技藝之人,有言尚不敢阻,況諸司百執事乎?庶言一出,志士解體,善言日壅,主上不得聞其過,群下無所獻其忠,禍天下必自庶始。陛下必欲重百官越職之禁,不若嚴言官失職之罰。當言不言,坐以負君誤國之罪。輕則記過,重則褫官。科道當遷,一視其章奏多寡得失為殿最,則言官無不直言,庶官無事可言,出位之禁無庸,太平之效自致矣。」

帝責其沽名,命貶一秩。科道合救,不允。庶偕同列胡時麟、梅國樓、郭顯忠復交章論劾,乃再降一秩,為湖廣按察司經歷。歷禮部主事,以憂歸,屢薦不起。家居二十年,始起故官。進南京兵部郎中。天啟初,終太僕少卿。

李沂,字景魯,嘉魚人。萬曆十四年進士。改庶吉士。十六年冬,授吏科給事中。中官張鯨掌東廠,橫肆無憚。御史何出光劾鯨死罪八,并及其黨錦衣都督劉守有、序班邢尚智。尚智論死,守有除名,鯨被切讓,而任職如故。御史馬象乾復劾鯨,詆執政甚力,帝下象乾詔獄。大學士申時行等力救,且封還御批,不報。許國、王錫爵復各申救,乃寢前命,而鯨竟不罪。外議謂鯨以金寶獻帝獲免。沂拜官甫一月,上疏曰:「陛下往年罪馮保,近日逐宋坤,鯨惡百保而萬坤,奈何獨濡忍不去?若謂其侍奉多年,則壞法亦多年;謂痛加省改,猶足供事,則未聞可馴虎狼使守門戶也。流傳鯨廣獻金寶,多方請乞,陛下猶豫,未忍斷決。中外臣民,初未肯信,以為陛下富有四海,豈愛金寶;威如雷霆,豈徇請乞。及見明旨許鯨策勵供事,外議藉藉,遂謂為真。虧損聖德,夫豈淺甚!且鯨奸謀既遂,而國家之禍將從此始,臣所大懼也。」是日,給事中唐堯欽亦具疏諫。帝獨手沂疏,震怒,謂沂欲為馮保、張居正報仇,立下詔獄嚴鞫。時行等乞宥,不從。讞上,詔廷杖六十,斥為民。御批至閣,時行等欲留御批,中使不可,持去。帝特遣司禮張誠出監杖。時行等上疏,俱詣會極門候進止。帝言:「沂置貪吏不言,而獨謂朕貪,謗誣君父,罪不可宥。」竟杖之。太常卿李尚智、給事中薛三才等抗章論救,俱不報。國、錫爵以言不見用,引罪乞歸。錫爵言:「廷杖非正刑,祖宗雖間一行之,亦未有詔獄、廷杖并加於一人者。故事,惟資賊大逆則有打問之旨,今豈可加之言官。」帝優詔慰留錫爵,卒不聽其言。

初,馮保獲罪,實鯨為之,故帝云然。或謂鯨罪不至如保。張誠掌司禮,素德保,授意言者發之,事秘莫能明也。其時,周弘禴、潘士藻皆以忤鯨得罪,而沂禍為烈。家居十八年,未召而卒。光宗嗣位,贈光祿少卿。

弘禴,字元孚,麻城人。倜儻負奇,好射獵。舉萬歷二年進士,授戶部主事。降無為州同知,遷順天通判。十三年春,上疏指斥朝貴,言:「兵部尚書張學顏被論屢矣。陛下以學顏故,逐一給事中、三御史,此人心所共憤也。學顏結張鯨為兄弟,言官指論學顏而不敢及鯨,畏其勢耳。若李植之論馮保,似乎忠讜矣,實張宏門客樂新聲為謀主。其巡按順天,納娼為小妻,猖狂干紀,則恃宏為內援也。鯨、宏既竊陛下權,而植又竊司禮勢,此公論所不容。《祖訓》,大小官許至御前言事。今吏科都給事中齊世臣乃請禁部曹建言。曩居正竊權,臺省群頌功德,而首發其奸者,顧在艾穆、沈思孝,部曹言事果何負於國哉?居正惡員外郎管志道之建白也,御史龔懋賢因誣以老疾;惡主事趙世卿之條奏也,尚書王國光遂錮以王官。論者切齒,為其附權奸而棄直言,長壅蔽之禍也。今學顏、植交附鯨、宏,鯨敢竊柄,世臣豈不聞?已不敢言,奈何反欲人不言乎?前此長吏垣者周邦傑、秦耀。當居正時,耀則甘心獵犬,邦傑則比迹寒蟬。今耀官太常,邦傑官太僕矣,諫職無補,坐陟京卿,尚謂臺省足恃乎?而乃禁諸臣言事也。夫逐一人之言者其罪小,禁諸臣之言者其罪大。往者嚴嵩及居正猶不敢明立此禁,何世臣無忌憚一至此哉!乞放學顏、植歸里,出耀、邦傑於外,屏張鯨使閒居,而奪世臣諫職,嚴敕司禮張誠等止掌內府禮儀,毋干政事,天下幸甚。」帝怒,謫代州判官,再遷南京兵部主事。

十七年,帝始倦勤,章奏多留中不下。弘禴疏諫,且請早建皇儲,不報。尋召為尚寶丞。明年冬,命監察御史閱視寧夏邊務。巡撫僉都御史梁問孟、巡茶御史鐘化民,取官帑銀交際,弘禴疏發之。詔褫問孟職,調化民於外。河東有秦、漢二壩,弘禴請以石為之,浚渠北達鴛鴦諸湖,大興水利。還朝,以將材薦哱承恩、土文秀、哱雲。明年,承恩等反,坐謫澄海典史。投劾歸,卒於家。天啟初,以嘗請建儲,贈太僕少卿。

潘士藻,字去華,婺源人。萬曆十一年進士。授溫州推官。擢御史,巡視北城。慈寧宮近侍侯進忠、牛承忠私出禁城,狎婦女。邏者執之,為所毆,訴於士藻。私牒司禮監治之。帝恚曰:「東廠何事?乃自外庭發。」杖兩閹,斃其一。鯨方掌東廠,怒。會火災修省,士藻言:「今天下之患,莫大於君臣之意不通。宜仿祖制,及近時平臺暖閣召對故事,面議所當施罷。撤大工以俟豐歲,蠲織造、燒造以昭儉德,免金花額外征以佐軍食。且時召講讀諸臣,問以經史。對賢人君子之時多,自能以敬易肆,以義奪欲。修省之實,無過於此。」鯨乃激帝怒,謫廣東布政司照磨。科道交章論救,不聽。尋擢南京吏部主事。再遷尚寶卿,卒官。

雒于仁,字少涇,涇陽人。父遵,吏科都給事中。神宗初即位,馮保竊權。帝御殿,保輒侍側。遵言:「保一侍從之僕,乃敢立天子寶座,文武群工拜天子邪,抑拜中官邪?欺陛下幼沖,無禮至此!」遵乃大學士高拱門生。保疑遵受拱指,遂謀逐拱。遵疏留中。尋劾兵部尚書譚綸,因薦海瑞。吏部尚書楊博稱綸才,詆瑞迂滯,疏遂寢。頃之,綸陪祀日壇,咳不止。御史景嵩、韓必顯劾綸衰病。居正素善綸,而馮保欲緣是為遵罪,因傳旨詰嵩、必顯欲用何人代綸,令會遵推舉,遵等惶懼不敢承。俱貶三秩,調外。遵得浙江布政司照磨。保敗,屢遷光祿卿。改右僉都御史,巡撫四川。罷歸,卒。

于仁舉萬曆十一年進士。歷知肥鄉、清豐二縣,有惠政。十七年,入為大理寺評事。疏獻四箴以諫。其略曰:

臣備官歲餘,僅朝見陛下者三。此外惟聞聖體違和,一切傳免。郊祀廟享遣官代行,政事不親,講筵久輟。臣知陛下之疾,所以致之者有由也。臣聞嗜酒則腐腸,戀色則伐性,貪財則喪志,尚氣則戕生。陛下八珍在御,觴酌是耽,卜晝不足,繼以長夜。此其病在嗜酒也。寵「十俊」以啟倖門,溺鄭妃,靡言不聽。忠謀擯斥,儲位久虛。此其病在戀色也。傳索帑金,括取幣帛。甚且掠問宦官,有獻則已,無則譴怒。李沂之瘡痍未平,而張鯨之貲賄復入。此其病在貪財也。今日榜宮女,明日抶中官,罪狀未明,立斃杖下。又宿怨藏怒於直臣,如范俊、姜應麟、孫如法輩,皆一詘不申,賜環無日。此其病在尚氣也。四者之病,膠繞身心,豈藥石所可治?今陛下春秋鼎盛,猶經年不朝,過此以往,更當何如?

孟軻有取於法家拂士,今鄒元標其人也。陛下棄而置之,臣有以得其故矣。元標入朝,必首言聖躬,次及左右。是以明知其賢,忌而弗用。獨不思直臣不利於陛下,不便於左右,深有利於宗社哉!陛下之溺此四者,不曰操生殺之權,人畏之而不敢言,則曰居邃密之地,人莫知而不能言。不知鼓鐘於宮,聲聞於外,幽獨之中,指視所集。且保祿全軀之士可以威權懼之,若懷忠守義者,即鼎鋸何避焉!臣今敢以四箴獻。若陛下肯用臣言,即立誅臣身,臣雖死猶生也。惟陛下垂察。

酒箴曰:耽彼麴蘗,昕夕不輟。心志內懵,威儀外缺。神禹疏狄,夏治興隆。進藥陛下,醲醑勿崇。

色箴曰:艷彼妖姬,寢興在側。啟寵納侮,爭妍誤國。成湯不邇,享有遐壽。進藥陛下,內嬖勿厚。

財箴曰:「競彼鏐鐐,錙銖必盡。公帑稱盈,私家懸罄。武散鹿臺,八百歸心。隋煬剝利,天命難諶。進藥陛下,貨賄勿侵。

氣箴曰:逞彼忿怒,恣睢任情。法尚操切,政盩公平。虞舜溫恭,和以致祥。秦皇暴戾,群怨孔彰。進藥陛下,舊怨勿藏。

疏入,帝震怒。會歲暮,留其疏十日。所云「十俊」,蓋十小閹也。明年正旦,召見閣臣申時行等於毓德宮,手於仁疏授之。帝自辨甚悉,將置之重典。時行等委曲慰解,見帝意不可回,乃曰:「此疏不可發外,恐外人信以為真。願陛下曲賜優容,臣等即傳諭寺卿,令於仁去位可也。」帝乃頷之。居數日,於仁引疾,遂斥為民。久之卒。天啟初,贈光祿少卿。

馬經綸,字主一,順天通州人。萬曆十七年進士。除肥城知縣,入為御史。二十三年冬,兵部考選軍政。帝謂中有副千戶者,不宜擅署四品職。責部臣徇私,兵科不糾發。降武選郎韓范、都給事中吳文梓雜職。鐫員外郎曾偉芳、主事江中信、程僖、陳楚產、給事中劉仕瞻三秩,調極邊。以御史區大倫、俞價、強思、給事中張同德言事常忤旨,亦鐫三秩。而五城御史夏之臣、朱鳳翔、塗喬遷、時偕行、楊述中籍中官客用家,不稱旨,並謫邊遠典史。又以客用貲財匿崇信伯費甲金家,刑部拷訊無實,謫郎中徐維濂於外。一時嚴旨頻下,且不得千戶主名,舉朝震駭。時東廠太監張誠失帝意。誠家奴錦衣副千戶霍文炳當遷指揮僉事,部臣先已奏請,而帝欲尋端罪言官,遂用是為罪。旋移怒兩京科道,以為緘默,命掌印者盡鐫三秩。於是給事中耿隨龍、鄒廷彥、黎道昭、孫羽侯、黃運泰、毛一公,御史李宗延、顧際明、彭可立、綦才、吳禮嘉、王有功、李固本,南京給事中伍文煥、費必興、盧大中,御史柳佐、聶應科、李文熙等十九人俱調外,留者並停俸一年。又令吏部列上職名,再罷御史馮從吾、薛繼茂、王慎德、姚三讓四人。大學士趙志皋、陳于陛、沈一貫及九卿各疏爭,尚書石星請罷職以寬諸臣,皆不納。於陛又特疏申救。帝怒,命降諸人雜職,悉調邊方。尚書孫丕揚等以詔旨轉嚴,再疏乞宥。帝益怒,盡奪職為民。經綸憤甚,抗疏曰:

頃屢奉嚴旨,斥逐南北言官。臣幸蒙恩,罰俸供職,今日乃臣諫諍之日矣。陛下數年以來,深居靜攝,君臣道否,中外俱抱隱憂。所恃言路諸臣,明目張膽為國家裁辨邪正,指斥奸雄。雖廟堂處分,未必盡協輿論,而縉紳公議,頗足維持世風,此高廟神靈實鑒佑之。所資臺省耳目之用大矣,陛下何為一旦自塗其耳目邪?

夫以兵部考察之故,而罪兵科是已。乃因而蔓及於他給事,又波連於諸御史。去者不明署其應得之罪,留者不明署其姑恕之由。雖聖意淵微,未易窺測,而道路傳說,嘖有煩言。陛下年來厭苦言官,動輒罪以瀆擾,今忽變而以箝口罪之。夫以無言罪言官,言官何辭。臣竊觀陛下所為罪言官者,猶淺之乎罪言官也。乃言官今日之箝口不言者,有五大罪焉。陛下不郊天有年矣,曾不能援故典排闥以諍,是陷陛下之不敬天者。罪一。陛下不享祖有年矣,曾不能開至誠牽裾以諍,是陷陛下之不敬祖者。罪二。陛下輟朝不御,停講不舉,言官言之而不能卒復之,是陷陛下不能如祖宗之勤政。罪三。陛下去邪不決,任賢不篤,言官言之而不能強得之,是陷陛下不能如祖宗之用人。罪四。陛下好貨成癖,御不少恩,肘腋之間,叢怨蓄變,言官俱慮之,而卒不能批鱗諫止,是陷陛下甘棄初政,而弗猶克終。罪五。言官負此大罪,陛下肯奮然勵精而以五罪罪之,豈不當哉!奈何責之箝口不言者,不於此而於彼也!

日者廷臣交章論救,不惟不肯還職,而且落職為民。夫諸臣本出草莽,今還初服,亦復何憾。獨念朝廷之過舉不可遂,大臣之忠懇不可拂。陛下不聽閣疏之救,改降級而為雜職,則輔臣何顏?是自離其腹心也。不聽部疏之救,改雜職而為編氓,則九卿何顏?是自戕其股肱也。夫君臣一體,元首雖明,亦賴股肱腹心耳目之用。今乃自塞其耳目,自離其腹心,自戕其股肱,陛下將誰與共理天下事乎!

夫人君受命於天,與人臣受命於君一也。言官本無大罪,一旦震怒,罪以失職,無一敢抗命者。既大失人心,必上拂天意。萬一上天震怒,以陛下之不郊不帝、不朝不講、不惜才、不賤貨,咎失人君之職,而赫然降非常之災,不知陛下爾時能抗天命否乎?臣不能抗君,君不能抗天,此理明甚,陛下獨不思自為社稷計乎?

帝大怒,亦貶三秩,出之外。

經綸既獲譴,工科都給事中海陽林熙春等上疏曰:「陛下怒言官緘默,斥逐三十餘人,臣等不勝悚懼。今御史經綸慷慨陳言,竊意必溫旨褒嘉,顧亦從貶斥。是以建言罪邪,抑以不言罪邪?臣等不能解也。前所罪者,既以不言之故,今所罪者又以敢言之故,令臣等安所適從哉?陛下誠以不言為溺職,則臣等不難進憂危之苦詞;誠以直言為忤旨,則臣等不難效喑默之成習。但恐廟堂之上,率諂佞取容,非君上之福也。臣等富貴榮辱之念豈與人殊,然寧為此不為彼者,毋亦沐二百餘年養士之恩,不負君父,且不負此生耳。陛下奈何深怒痛疾,而折辱至是哉!」帝益怒,謫熙春茶鹽判官,加貶經綸為典史。熙春遂引疾去。是日,御史定興鹿久徵等亦上疏,請與諸臣同罪,貶澤州判官。二疏列名凡數十人,悉奪俸。

頃之,南京御史東莞林培疏陳時政。帝追怒經綸,竟斥為民。既歸,杜門卻掃凡十年。卒,門人私謚聞道先生。

培由鄉舉為新化知縣。縣僻陋,廣置社學教之。民有死於盜者,不得。禱於神,隨蝴蝶所至獲盜,時驚為神。征授南京御史,劾罪誠意伯劉世延,置其爪牙於法。已,上書言徐維濂不當謫;陜西織花絨、購回青擾民,宜罷;湖廣以魚鮓、江南以織造並奪撫按官俸,蘇州通判至以織造故褫官,皆不可訓;并論及沈思孝等。帝怒,謫福建鹽運知事。告歸,卒。

天啟初,復經綸官,贈太僕少卿。培贈光祿少卿,熙春亦還故職。屢遷大理卿,年老乞罷。時李宗延、柳佐輩咸官於朝,頌其先朝建言事。詔加戶部右侍郎,致仕。

劉綱,邛州人。祖文恂,孝子。父應辰,舉鄉試,不仕,亦以孝義聞。綱舉萬曆二十三年進士,改庶吉士。二十五年七月,上疏曰:

去歲兩宮災,詔示天下,略無禹、湯罪己之誠,文、景蠲租之惠,臣已知天心之未厭矣。比大工肇興,伐木榷稅,採石運瓷,遠者萬里,近者亦數百里。小民竭膏血不足供費,絕筋骨不足任勞,鬻妻子不能償貸。加以旱魃為災,野無青草,人情胥怨,所在如仇。而天下悔禍,三殿復災。《五行志》曰:「君不思道,厥災燒宮。」陛下試自省,晝之為、夜之息,思在道乎,不在道乎?

凡敬天法祖,親賢遠奸,寡欲保身,賤貨慎德,俱謂之道,反是非道矣。陛下比年以來,簡禋祀,罷朝講,棄股肱,閡耳目,斷地脈,忽天象,君臣有數載之隔,堂陛若萬里而遙。陛下深居靜攝,所為祈天永命者何狀,即外廷有不知,上天寧不見邪?今日之災,其應以類,天若曰:皇之不極,於誰會歸,何以門為?朝儀久曠,於誰稟仰,何以殿為?元宰素餐,有污政地,何以閣為?其所以示警戒,勸更新者,至深切矣。尚可因循玩愒,重怒上帝哉!

臣聞五行之性,忌積喜暢。積者,災之伏也,請冒死而言積之狀。皇長子冠婚、冊立久未舉行,是曰積典。大小臣僚以職事請,強半不報,是曰積牘。外之司府有官無人,是曰積缺。罪斥諸臣,概不錄敘,是曰積才。閫外有揚帆之醜,中原起揭竿之徒,是曰積寇。守邊治河,諸臣虛詞罔上,恬不為怪,是曰積玩。諸所為積,陛下不能以明斷決,元輔趙志皋不能以去就爭,天應隨之,毫髮不爽。陛下何不召九卿、臺諫面議得失,見兔顧犬,未為晚也。若必專任志皋,處堂相安,小之隳政事而羞士類,大之叢民怨而益大怒。天下大計奈何以此匪人當之!此不可令關白諸酋聞也。

帝得疏,恚甚,將罪之。以方遘殿災,留中不報。

已而授編修。居二年,京察。坐浮躁,調外任,遂歸。明年卒。故事,翰林與政府聲氣相屬。綱直攻志皋短,故嗛之不置,假察典中之。明世以庶吉士專疏建言者,前惟鄒智,後則劉之綸與綱,並四川人。

戴士衡,字章尹,莆田人。萬曆十七年進士。除新建知縣,擢吏科給事中。薊州總兵官王保濫殺南兵,士衡極論其罪。已,請亟補言官,劾石星誤國大罪五。山東稅使陳增請假便宜得舉刺將吏,淮、揚魯保亦請節制有司,士衡力爭。仁聖太后梓宮發引,帝不親送,士衡言:「母子至情,送死大事,奈何於內庭數武地,靳一舉足勞。今山陵竣事,願陛下扶杖出迎神主,庶少慰聖母之靈,答臣民之望。」錦衣千戶鄭一麟奏開昌平銀礦。士衡以地逼天壽山,抗疏爭。皆不報。

二十五年正月,極陳天下大計,言:「方今事勢不可知者三:天意也,人心也,氣運也。大可慮者五:紀綱廢弛也,戎狄侵陵也,根本動搖也,武備疏略也,府藏殫竭也。其切要而當亟正者一,則君心也。陛下高拱九重,目不睹師保之容,耳不聞丞弼之議,美麗當前,燕惰自佚,即欲殫聰明以計安社稷,其道無由。誠宜時御便殿,召執政大臣講求化理,則心清欲寡,政事自修。」亦不報。

日本封事敗,再劾星及沈惟敬、楊方亨,且列上防倭八事。多議行。俄劾南京工部尚書葉夢熊、刑部侍郎呂坤、薊遼總督孫幰及通政參議李宜春。時幰已罷,宜春自引歸,坤亦以直諫去。給事中劉道亨右坤,力詆士衡,謂其受大學士張位指。士衡亦劾道亨與星同鄉,為星報復。帝以言官互爭,皆報寢。尋劾罷文選郎中白所知。帝惡吏部郎,貶黜者二十二人,因詰責吏科朋比。都給事中劉為楫、楊廷蘭、張正學、林應元及士衡俱引罪。詔貶為楫一秩,與廷蘭等並調外。士衡得蘄州判官。無何,詔改遠方,乃授陜西鹽課副提舉。未赴,會《憂危竑議》起,竟坐遣戍。

先是,士衡再劾坤,謂潛進《閨範圖說》,結納宮闈,因請舉冊立、冠婚諸禮。帝不悅。至是有跋《閨範》後者,名曰《憂危竑議》,誣坤與貴妃從父鄭承恩、戶部侍郎張養蒙、山西巡撫魏允貞、吏科給事中程紹、吏部員外郎鄧光祚及道亨、所知等同盟結納,羽翼貴妃子。承恩大懼。以坤、道亨、所知故與士衡有隙,而全椒知縣樊玉衡方上疏言國本,指斥貴妃,遂妄指士衡實為之,玉衡與其謀。帝震怒,貴妃復泣訴不已,夜半傳旨逮下詔獄拷訊。比明,命永戍士衡廉州、玉衡雷州。御史趙之翰復言:「是書非出一人,主謀者張位,奉行者士衡,同謀者右都御史徐作、禮部侍郎劉楚先、國子祭酒劉應秋、故給事中楊廷蘭、禮部主事萬建崑也。諸臣皆位心腹爪牙,宜并斥。」帝入其言,下之部院。時位已落職閒住,署事侍郎裴應章、副都御史郭惟賢力為作等解,不聽。奪楚先、作官,出應秋於外,廷蘭、建崑謫邊方,應章等復論救。帝不悅,斥位為民。

士衡等再更赦,皆不原。四十五年,士衡卒於戍所。巡按御史田生金請脫其戍籍,釋玉衡生還,帝不許。天啟中,贈太僕少卿。

曹學程,字希明,全州人。萬曆十一年進士。歷知石首、海寧。治行最,擢御史。帝命將援朝鮮。已而兵部尚書石星聽沈惟敬言,力請封貢。乃以李宗城、楊方亨為正副使,往行冊封禮。未至日本,而惟敬言漸不售,宗城先逃歸。帝復惑星言,欲遣給事中一人充使,因察視情實。學程抗疏言:「邇者封事大壞,而方亨之揭,謂封事有緒。星、方亨表裏應和,不足倚信。為今日計,遣科臣往勘則可,往封則不可。石星很很自用,趙志皋碌碌依違,東事之潰裂,元輔、樞臣俱不得辭其責。」初,朝鮮甫陷,御史郭實論經略宋應昌不足任,并陳七不可。帝以實沮撓,謫懷仁典史。後已遷刑部主事。會封貢議既罷,而朝鮮復懇請之。帝乃追怒前主議者,以實倡首,斥為民。并敕石星盡錄異議者名,將大譴責。志皋等力解乃已。及遣使不得要領,因欲別遣,已而罷之,即以方亨為正使矣。而學程方督畿輔屯田,不知也。疏入,帝大怒,謂有暗囑關節,逮下錦衣衛嚴訊。榜掠無所得,移刑部定罪。尚書蕭大亨請宥,帝不許,命坐逆臣失節罪斬。刑科給事中侯廷佩等訟其冤。志皋及陳于陛、沈一貫言尤切,皆不納。自是救者不絕,多言其母年九十餘,哭子待斃。帝卒弗聽,數遇赦亦不原。

其子正儒,朝夕不離犴狴。見父憔悴骨立,嘔血仆地,久之乃蘇,因刺血書奏乞代父死,終不省。三十四年九月,始用硃賡言,謫戍湖廣寧遠衛。久之,放歸,卒。天啟初,贈太僕少卿。崇禎時,旌正儒為孝子。

郭實,字伯華,高邑人。萬曆十一年進士。授朝邑知縣,選授御史。御史王麟趾劾湖廣巡撫秦耀結政府狀,謫徐溝丞。實復劾耀,耀乃罷。比去任,侵贓贖銀巨萬,為衡州同知沈鈇所發,下吏戍邊。故事,撫按贓贖率貯州縣為公費,自耀及都御史李采菲、御史沈汝梁、祝大舟咸以自潤敗。自是率預滅其籍,無可稽矣。實以論朝鮮事黜。久之,封貢不成,星下吏。給事中侯廷佩請還實官,不許。家居十五年,起南京刑部主事,終大理右寺丞。

翁憲祥,字兆隆,常熟人。萬曆二十年進士。為鄞縣知縣。課最,入為禮科給事中。以憂去。補吏科,疏陳銓政五事。其一論掣簽法,言:「使盡付之無心,則天官之職一吏可代。茍為不然,則地本預擬,何必於大廷中為掩飾之術。請亟停罷。」時不能從。故事,正郎不奉使,撫按必俟代,至是多反之。而江西巡撫許弘綱以父憂徑歸,廣西巡撫楊芳亦以憂乞免代,憲祥極言非制。弘綱貶官,芳亦被責。言者詆朱賡、李廷機輒被譴,憲祥疏論。已,劾雲南巡撫陳用賓、兩廣總督戴耀,並不報。是時大僚多缺。而侍郎楊時喬、楊道賓旬日間相繼物故,吏、禮二部長貳遂無一人。兵部止一尚書,養痾不出。戶、刑、工三部暨都察院堂上官,俱以人言註籍。通政大理亦無見官。憲祥言九卿俱曠,甚傷國體。因陳補缺官、起遺佚數事,報聞。屢遷刑科都給事中。吏部尚書孫丕揚、副都御史許弘綱以考察為言路所攻,求去。憲祥言:「一時賢者,直道難容,相率引避。國是如此,可為寒心。」既而軍政拾遺,疏為錦衣都督王之楨所撓,久不下。罪人陳用賓等已論死,疏亦留中。憲祥皆抗章論駁。知縣滿朝薦、李嗣善,同知王邦才,以忤稅使繫獄,乃請釋之。會冬至停決囚,復請推緩刑德意,宥捴臣、矜楚獄。帝皆不報。尋調吏科。四十一年,命輔臣葉向高典會試,給事中曾六德以論救被察官坐貶,旨皆從內出。憲祥力諫。中官黃勛、趙祿、李朝用、胡濱等不法,亦連疏彈劾。久之,擢太常少卿。居數年卒。

徐大相,字覺斯,江西安義人。萬曆四十四年進士。授東昌推官。改武學教授,稍遷國子博士。四十七年九月朔,百僚將早朝,司禮中官盧受傳免。眾趨出,受從後姍侮。大相憤,歸草二疏。一論遼左事,一論受奸邪。時接疏者即受也。見遼事疏曰:「此小臣,亦敢言事。」及帝閱第二疏,顧受曰:「此即論汝罪者。」受錯愕,叩頭流血請罪,曰:「奴當死。」疏乃留中。是日,南京國子學錄喬拱璧亦疏劾受,不報。明年,遷兵部主事。天啟二年,調吏部稽勳主事,移考功。明年,進驗封員外郎。進士薛邦瑞為其祖蕙請謚,大相與尚書張問達議如其請。熹宗方惡恤典冗濫,鐫大相三秩,出之外。問達等引罪,不問。大學士葉向高、都御史趙南星等連疏救,乃改鐫二秩。大相方候命,群奄黨受者數十輩,持梃噪於門。比搜大相橐,止俸金七十兩,乃哄然散。家居,杜門讀書,里人罕見其面。

崇禎元年,起故官。俄改考功,遷驗封郎中。歷考功、文選。奏陳遵明旨、疏淹滯、破請託、肅官評、正選規、重掌篆、崇禮讓、勵氣節、抑僥倖、核吏弊十事,帝即命飭行。故尚書孫丕揚等二十六人為魏忠賢削奪,大相請復其官,帝不許。旋以起廢忤旨,貶秩視事。給事中杜三策言大相端廉,起廢協輿論,不當譴,不聽。父憂歸,卒於家。

贊曰:神宗中年,德荒政圮。懷忠發憤之士,宜其激昂抗詞以匡君失。然納諫有方,務將以誠意。絞訐摩上,君子弗為。謂其忠厚之意薄,而衒沽之情勝也。雒於仁、馬經綸詆譏譙讓,幾為儕偶所不能堪矣。聖人取諷諫,意者殆不如是乎!


\end{pinyinscope}